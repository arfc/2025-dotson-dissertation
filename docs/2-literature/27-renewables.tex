\subsection{\textcolor{red}{If renewable energy can solve climate change, why
build anything else?}}

\begin{enumerate}
    \item What do proponents of renewable energy say about solar and wind?
    \begin{itemize}
        \item By virtue of being renewable, the ``fuel'' source is practically
        infinite. Solar energy will never run out on the timescale of human
        civilization, and wind energy (as a product of wind energy plus the
        Earth's rotation) will similarly never run out.
        \item Since renewable energy, specifically solar and wind, lend
        themselves more naturally to distribution, they are more democratic
        sources of energy (cite).
    \end{itemize}
    \item What are the technical objections to fully renewable energy grids?
    \begin{itemize}
        \item Wind and solar are variable and intermittent energy sources.
        \item Renewable energy sources (including non-intermittent sources like
        hydroelectric dams and biomass) have a much lower \ac{eroi} than fossil
        fuel sources \cite{hall_eroi_2014,weisbach_energy_2013}. However, recent
        harmonization studies belie this assertion by demonstrating that
        renewable sources have similar, if not greater, \ac{eroi} than coal or
        natural gas \cite{murphy_energy_2022}.
        \item Although the ``fuel'' for solar and wind resources is infinite,
        the materials required to extract energy from these sources is finite.
        \item In contrast with the nuclear energy, the renewable energy industry
        does not have a strong notion of the lifecycle of their technology.
        There are no plans to safely store or recycle solar panels and wind
        turbines at end-of-life.
    \end{itemize}
    \item What are some non-technical objectives to renewable energy?
    \begin{itemize}
        \item Democracy and inclusivity are not required features of renewable
        energy sources \cite{bell_toward_2020,winner_artifacts_1980}. Although
        renewables have the potential to be highly decentralized, this is not a
        requirement of their use. Many deployment schemes for solar and wind
        farms involve creating highly centralized ``farms'' that are owned and
        operated by a single corporate entity, thereby recreating the same power
        structures that currently exist and thus perpetuating inequities.
    \end{itemize}
    \item What drives the opposition for renewable energy projects (specifically
    wind and solar)?
    \begin{itemize}
        \item Due to the sprawling nature of energy sources like solar and wind,
        these projects are frequently subject to public opposition in proposed
        areas.
        \item The \ac{nimby} movement, or \acs{nimbyism}, is frequently cited as
        the cause of opposition (cite). \ac{nimbyism} describes people that do
        not support the execution of a specific project (energy-related or
        otherwise) due to the project's proximity to a certain location
        (typically residences), but would otherwise support a project if it were
        further away. Despite the popular understanding of \ac{nimbyism}'s role
        in delaying energy projects, the most comprehensive empirical research
        on \ac{nimbyism} demonstrated that \ac{nimby} attitudes do not drive
        public opposition \cite{konisky_proximity_2021}. 
    \end{itemize}
\end{enumerate}

% What do proponents of renewable energy say about solar and wind?

Renewable energy sources, particularly solar and wind energy, are appealing
alternatives to fossil fuel sources. Solar and wind energy have fewer
geographical constraints, compared to hydroelectric and geothermal power
\cite{lopez_us_2012}. With respect to nuclear energy, solar and wind are
similarly safe on a per-unit-energy basis and produce comparable lifecycle
carbon emissions
\cite{united_nations_economic_commission_for_europe_carbon_2022,sovacool_balancing_2016,
intergovernmental_panel_on_climate_change_climate_2014}. Contrary to popular
belief, solar and wind resources can have \ac{eroi} comparable to coal-fired
power plants \cite{murphy_energy_2022}.