% \noindent\hrulefill \textcolor{red}{Subsubsection: Energy justice and the
% Boundaries of Energy Systems.}
\section{Energy Justice}
\label{section:energy-justice}
% \item What is energy justice?


Energy justice is a conceptual and analytical tool regarding the ethical or
normative dimensions of energy systems and addresses the systemic causes of
burdens, and inequities \cite{sovacool_energy_2015}.
% \begin{enumerate} \item What is justice?
    There are many conceptions of justice; however, the most popular framework
    for understanding justice is a three-faceted approach originating from David
    Schlosberg: distributive, recognition, and procedural justice
    \cite{schlosberg_2_2007}.
    % \item What is distributive justice? 
% \noindent\hrulefill
    Distributive justice relates to the fair distribution of resources,
    burdens, and responsibilities. Studies on distributive justice seek to
    address the normative question: how should a just society distribute the
    benefits it produces and \textit{the burdens required to maintain it}
    \cite{brighouse_justice_2004}. Additionally, distributive justice
    considers \textit{how} poor distributions are created
    \cite{schlosberg_2_2007}.
% \noindent\hrulefill
    % \item What is procedural justice?
    Procedural (in)justice is defined as the presence of (un)fair and
    (in)equitable institutional processes of the state \cite{schlosberg_2_2007}.
    In other words, how decisions of societal import are made and who is
    involved in those decisions. Sovacool and Dworkin (2015) outline four
    elements of procedural justice: transparency, meaningful participation,
    impartiality, and avenues for redress \cite{sovacool_energy_2015}.    
% \noindent\hrulefill
    % \item What is recognition justice?
Justice of recognition is the most vague of the three tenets of justice and is
    frequently reduced to a component of either distributive or procedural
    justice \cite{schlosberg_2_2007, van_uffelen_revisiting_2022}. A common
    argument for this consolidation is that recognition is a precondition for
    achieving distributive justice or that achieving procedural justice
    necessarily includes recognition \cite{schlosberg_2_2007}. However,
    recognition is unique from distributive and procedural justices because it
    is concerned with a different family of injustice, namely,
    \textit{misrecognition} \cite{van_uffelen_revisiting_2022}. van Uffelen
    (2022) suggests a nuanced definition of recognition justice as ``the
    adequate recognition of all actors through love, law, and the status order''
    \cite{van_uffelen_revisiting_2022}.
% \end{enumerate} \noindent\hrulefill
Sovacool and Dworkin (2015) offer a framework for assessing energy policies from
a justice perspective. Table \ref{tab:justice-frameworks} map the relationships
between justice-as-a-decision-making-tool from Sovacool \& Dworkin, Paterson's
hazard response characterization, and Schlosberg's triumvirate of justice. 

\begin{table}[h]
    \centering
    \caption{Different ways to operationalize justice concepts.}
    \begin{tabular}{ccc}
    \toprule
    Schlosberg \cite{schlosberg_2_2007} & Sovacool \& Dworkin 
    \cite{sovacool_energy_2015}& Paterson et al. \cite{paterson_community-based_2019}\\
    \midrule
    & Intragenerational Equity & Material Well-being \\
    Distribution & Intergenerational Equity & Infrastructure \\
    & Responsibility & \\
    & Due Process & Awareness \\
    Procedure & Good Governance & Governance \\
    & & \\
    & Availability$^1$ & Relational Well-being \\
    Recognition & Affordability$^1$ & \\
    & Sustainability$^1$ & \\
    \bottomrule\\
    % \rule{0pt}{2ex}
    \multicolumn{3}{l}{$^1$ van Uffelen \cite{van_uffelen_revisiting_2022} argues for this categorization.}
\end{tabular}
    \label{tab:justice-frameworks}
\end{table}

Although Sovacool \& Dworkin do not explicitly discuss recognition justice, it
is a unique aspect of justice that can still be useful for contextualizing their
recommendations. For example, due to the psychological pressures introduced by a
lack of access to energy, either due to infrastructure or cost, interrupts
relational well-being and is an injustice \cite{van_uffelen_revisiting_2022}.
Further, (un)sustainable policies may be considered a misrecognition of the
humanity of future generations. Next, I examine the specific ways the social
science literature understands how energy systems and their infrastructure
(artifacts) contribute to the distribution of burdens.

% \noindent\hrulefill \item What is an energy system?
\subsection{Boundaries of Energy Systems}
\label{section:energy-system-boundaries}
Previous work defined energy systems in purely technical terms as spatially,
temporally, and topologically complex machines that coordinate the supply and
demand of energy, especially electricity \cite{dotson_influence_2022}. However,
this definition neglects the ways energy systems may be used to construct and
maintain power relations that contribute to inequitable distributions of
burdens. Energy access is necessary to support complex modern economies and
therefore possesses political power \cite{jones_building_2013,
bridge_energy_2018}. The literature on the political economy of energy
infrastructure locates this political influence in five distinct ways
\cite{bridge_energy_2018}. First, energy infrastructure affects competition and
collaboration among nation-states in the geo-political sphere. The current
situation in Ukraine makes this especially salient
\cite{figueiredo_impacts_2022}. 

The second subset of the literature focuses on the process of energy
infrastructure development and how these processes create social inequities. For
example, energy policies that subsidize residential solar panels have not led to
more equitable adoption of solar energy, with greater adoption in areas with
higher income, among other social indicators \cite{reames_distributional_2020}.
Other popular arguments in favor of renewable energy assert that these energy
sources are necessarily more egalitarian because the Sun and the wind cannot be
(or have not yet been) privatized. Another is the urgency of climate change.
Although these arguments have merit, they ignore or minimize the potential
environmental and social consequences of energy planning that does not consider
energy justice \cite{jones_building_2013}. Large-scale energy projects in the
Global South have already led to the dispossession of nearby indigenous
communities and other key actors \cite{yenneti_spatial_2016,
barragan-contreras_procedural_2022}.

Third, the development of energy infrastructure is not simply conducted via
policy measures, but also in the manner governments activate the public
imagination in favor of these policies
\cite{bridge_energy_2018,jasanoff_containing_2009}. Jasanoff and Kim (2009)
articulate this concept as `socio-technical imaginaries,' which are
simultaneously descriptive and prescriptive of possible energy futures
established by governments in the national zeitgeist
\cite{jasanoff_containing_2009}. This concept is demonstrated by the discourse
surrounding nuclear energy in the United States and South Korea
\cite{jasanoff_containing_2009} as well as in Japan
\cite{valentine_energy_2019}. Governments can employ `grand narratives' related
to national security, climate change, or modernization to enhance public support
while minimizing genuine participation \cite{bridge_energy_2018}.

Fourth, the political power of energy infrastructure can be traced further to
the cultural values and policy choices embedded in the design and operation of
seemingly technical systems \cite{bridge_energy_2018}. In other words, the
design and implementation of energy infrastructure may be used as a vehicle for
apparently unrelated agendas, a form of ``policy-making by other means''
\cite{bridge_energy_2018, clausewitz_chapter_1918}. Edwards and Hecht (2010)
refer to the co-constitution of technological and political order as
`\textit{technopolitics},' demonstrating the tangible material and political
outcomes of technological systems \cite{edwards_history_2010}.

Finally, energy systems and their infrastructure possess a unifying quality
through which new political identities may evolve \cite{bridge_energy_2018}.

From these various perspectives, we can observe that confining an energy system
to its technical characteristics is woefully incomplete. I propose that an
energy system is a spatially, temporally, and topologically complex machine that
coordinates the supply and demand of energy and resources and acts as an
important mediator of burdens that influence risks (such as risks from climate
change). This thesis takes the important step of analyzing energy system
planning and policy with this expanded definition. The next section reviews
current attempts to model energy systems and identifies gaps in conventional
methods.

% \textcolor{red}{Climate change \textit{is} a complex issue with multiple
% interacting and interwoven layers. However, it can and must be understood
% holistically instead of stripping it of its complexity and exclusively
% relegating solutions to the realms of economics and engineering. This
% over-simplification is done out of a misplaced sense of pragmatism and either
% an inability or unwillingness to completely apprehend the problem.}

% \noindent\hrulefill \item What are the obstacles to building more clean energy
% infrastructure? What is preventing the transition to a ``qualitatively new
% type of [environmental] society'' \cite{bluhdorn_legitimation_2020}?

% \textcolor{red}{This is where you can introduce different schools of thought
% about transitions.}

% \textcolor{blue}{Interesting that you chose to frame public acceptance as an
% ``obstacle'' rather than a source of greater accountability and
% participation.} \begin{enumerate} \item Legitimation crisis of democracy
% \cite{bluhdorn_legitimation_2020}. \item Social acceptance literature
% \end{enumerate}

% \end{enumerate}

% \section{Calls for a Just-Transition}

% Att

