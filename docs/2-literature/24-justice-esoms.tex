\section{Modeling and Quantifying Energy Justice}

The dearth of studies that incorporate energy justice into \acp{esom} highlights
the challenge of combining these techniques. The literature on energy justice
and socio-technical transitions tend to derogate modeling efforts as cold and
calculating \cite{sovacool_energy_2015,sovacool_energy_2016}, and most models do
not account for energy justice in either equations or analysis. However, there
have been some notable attempts to bridge this gap. The following studies by
Patrizio et al. (2020) \cite{patrizio_socially_2020} and Neumann \& Brown (2021)
\cite{neumann_near-optimal_2021} explicitly use \acp{esom} in their analyses.
Although the works by Chapman et al. (2018) \cite{chapman_prioritizing_2018} and
Mayfield et al. (2019) \cite{mayfield_quantifying_2019} do not use \acp{esom} as
described in Section \ref{section:esoms}, these contributions quantify some
features of their respective energy systems and how they relate to notions of
energy justice and equity.

Patrizio et al. (2020) conducted a technology-agnostic `social equity' scenario
that maximized the \ac{gva} of several countries' energy systems rather than
minimizing the total cost \cite{patrizio_socially_2020}. \Ac{gva} is also
distinct from social welfare because it measures contributions to \ac{gdp} from
individual producers rather than maximizing surplus. This metric enables
sector-specific analysis of the impacts of energy infrastructure on employment
and sales. Equity, in this context, is identical to socioeconomic development as
measured by \ac{gdp}. Using this definition of equity, the researchers looked at
a socio-technical transition for three countries: Spain, the United Kingdom, and
Poland. They found that a 100\% renewable energy system would reduce labor
compensation by 50-60\% in the UK and Poland but could increase benefits in
Spain. They argue this is due to the outsourcing of manufacturing and mining
jobs in the former cases, while Spain has enough domestic resources to
accommodate the transition. The researchers did not analyze possible shifts in
power dynamics related to the energy systems, but they did identify that there
is no one-size-fits-all solution to achieving net-zero carbon emissions.

Neumann \& Brown (2021) performed a detailed analysis of the European energy
system considering the expansion of transmission networks and energy producers
for a 100\% renewable energy system under cost minimization
\cite{neumann_near-optimal_2021}. They also used a novel formulation of \ac{mga}
to identify the boundaries of the feasible space for each technology within
different levels of tolerance. This study uses Lorenz curves and Gini
coefficients to measure the uniformity of the distribution of energy production
and consumption. In other words, the most equitable distribution of energy
resources would accord with energy consumption \cite{neumann_near-optimal_2021}.
The researchers conclude that wind power and greater transmission capacity are
associated with less regional equity, while solar power and storage technologies
lead to a more even distribution of the power supply. This is useful for
measuring the distribution of energy benefits from the energy system but does
not consider the distribution of costs nor consider regional preferences. 

Chapman et al. (2018) looked at the energy justice implications of transitioning
coal plants to renewable energy projects for the nearby communities
\cite{chapman_prioritizing_2018}. They measure distributional justice with
``relative equity'' and ``policy burden.'' Relative equity accounts for factors
such as \ac{ghg} reduction, employment, electricity cost, and health impacts.
Policy burden is a weighted value according to the income level of each
community. These two quantities were plotted together to identify a retirement
schedule that maximizes equity outcomes and ensures that burdens are borne by
the most capable communities \cite{chapman_prioritizing_2018}. Additionally, the
researchers argue that by using equity measures to inform policy choices, those
policy decisions are more procedurally just. However, this neglects meaningful
participation and may or may not address decision-making transparency
\cite{sovacool_energy_2015}. Further, this study does not consider how replacing
dispatchable suppliers with \ac{vre} will affect the availability and
affordability of electricity \cite{sovacool_energy_2015}. This latter challenge
could be addressed by incorporating methods from the \ac{esom} literature. The
former issue of decision-making transparency is one of the motivations for this
thesis.

Mayfield et al. (2019) quantified the social equity implications for the
expansion of natural gas infrastructure in Appalachia using spatial and temporal
metrics such as job-years generated by greater gas development, premature deaths
caused by air pollution, changes in poverty and income, and the distribution of
these various benefits along regional, racial, and economic lines. Additionally,
they identified some of the intergenerational equity impacts of climate change
and expanded gas infrastructure. 

\subsection{Enabling Procedural Justice Through Energy Models}

Traditionally, \acp{esom} are used to inform policy-makers \cite{li_open_2020}
in order to infuse policy choices with an appearance of objectivity. Indeed,
some of the studies reviewed in the previous section argue that this infusion
will lead to greater procedural and recognitional justice outcomes as long as
the policies maximize some measure of energy justice
\cite{chapman_prioritizing_2018, heffron_resolving_2015}. However, these types
of detailed analyses may also be used to dismiss concerns or opposition from the
public due to insufficient `technical expertise' \cite{johnson_dakota_2021, susser_better_2022}.
Further, without meaningful participation from the affected public, this
approach further entrenches procedural injustices. To credit the energy modeling
community, there is significant awareness of the importance of transparency and
repeatability in the space \cite{decarolis_case_2012, pfenninger_energy_2014,
pfenninger_openmod_2022, forster_open_2022, hilpert_open_2018}. Yet these two
goals are challenged by the computational resources required to run the more
complex and detailed models, as well as the learning curve necessary to
understand and modify the model inputs themselves. There has been some effort to
reduce this learning curve and make modeling itself more accessible. Frameworks
such as METIS, EnergyRT, and \ac{pygen} all emphasize reproducibility,
user-friendliness, and a shallower learning curve \cite{sakellaris_metis_2018,
lugovoy_energyrt_2022, dotson_python_2021}. The creators of METIS state their
goal is to ``close the gap between modelers and policy-makers, enabling
policy-makers to become modelers'' \cite{sakellaris_metis_2018}. However, these
frameworks do not offer computational resources to run their models. The
\ac{temoa} project offers limited cloud computing capabilities, free of charge
\cite{temoa_project_temoa_2023}. However, the responsibility for creating an
input file still falls to the user, which can be overwhelming even for
experienced modelers. 
For this reason, many municipalities hire external organizations to run models
since cities and towns typically do not have the capacity to do it themselves
\cite{ben_amer_too_2020, johannsen_designing_2021}. Additionally, researchers
identified gaps between the needs of modelers and users, such as the inflexibility
of models to reflect multiple policy options. ``Many models include a carbon price,
often labelled as `climate policy,' as the only explicit policy instrument'' 
\cite{susser_better_2022}. Despite broad agreement about the need to incorporate
stakeholders the ``process'' of modeling, there have been surprisingly limited
efforts so far \cite{ben_amer_too_2020,johannsen_designing_2021,susser_better_2022}
Finally, it's not clear that perfectly accessible and transparent modeling tools
will translate to more procedurally just policy-making. The next sections
outline some methods used to address this challenge.

\subsection{\Acl{pve}}

Even if the public could use modeling tools, their testimony may still be
dismissed due to a `lack of expertise.' However, the public has preferences that
should be incorporated into decision-making. Additionally, community members are
frequently able to assess trade-offs when presented with them. \Acf{pve} is one
method for translating community preferences into just policy outcomes.
Researchers in the Netherlands developed this method to enhance democratic
participation and infuse policies with genuine feedback from constituents
\cite{mouter_introduction_2019}. They observed that a common method of assessing
social impacts is \ac{wtp}, which is the maximum price an individual is willing
to pay for a good or service, yet individual purchasing habits do not
necessarily reflect their views on public policy due to the relative salience of
moral considerations \cite{mouter_introduction_2019}. With \ac{pve},
participants can allocate a specific amount of the public budget for certain
policies, including levying or reducing taxes for greater or lesser government
spending \cite{mouter_introduction_2019}. Researchers applied \ac{pve} in three
different settings, mobility and transportation \cite{mouter_contrasting_2021},
flood risk projects (i.e., a climate hazard \textit{infrastructure} response)
\cite{dekker_economics_2019}, and with a phaseout of natural gas
\cite{mouter_including_2021}. Importantly, the studies also measured the impact
of these interventions and found that \ac{pve} enables participation from people
that do not typically participate (recognition), the results were useful for
decision-making and participation was meaningful for the majority of subjects
\cite{mouter_including_2021}. Although previous applications of \ac{pve} focused
on economic policy levers, this approach offers a promising pathway toward
identifying equitable and just energy mixes for the future.
The next section reviews some efforts to enhance outcomes via direct collaboration 
between energy modelers and communities themselves.

\subsection{Participatory modeling processes}

Although \ac{pve} offers one method to elicit stakeholder input, researchers are
beginning to explore transdisciplinary methods for engaging members of the public.
McGookin et al. 2021 \cite{mcgookin_participatory_2021} systematically reviewed (``the review'') 
existing literature on participatory energy modeling. The review found a major
difference in goals for studies in either national or local contexts. Studies involving
participants on a national scale

Benefits of participatory energy modeling:
\begin{enumerate}
    \item Legitimacy and robustness --- discussion and tradeoff analysis lead to
    solutions which are more socially and politically feasible.
    \item Capacity building --- non-academic participants benefit from learning new
    tools, and modelers gain new perspectives and a better understanding of a community's
    reality.
    \item Consensus building 
\end{enumerate}

Although the review doesn't explicitly identify these benefits with energy justice.
The benefits of particpatory energy modeling evinced by the literature are well 
aligned with both procedural and recognition justice. Further, only one of the apep


The authors found that \acp{esom}, coupled with \ac{mcda}, were the most popular 
quantitative tools used in the literature. \Ac{mcda} is a tool used to analyze 
tradeoffs among different scenarios. 

The authors proposed a conceptual framework for successful integration of 
participatory methods.
\begin{enumerate}
    \item Stakeholder engagement must occur before drawing conclusions from a
    quantitative analysis. \textcolor{red}{I'm extending this conceptual framework
    by describing precisely where in the modeling process stakeholder input can 
    and should take place, and in what forms.}
    \item There should be an iterative process where stakeholders can shape the 
    process in addition to evaluating the results.
    \item Lastly, the ideal process should include stakeholder input at every stage.
\end{enumerate}
Do the authors indicate why it's better to include this, beyond the ``usefulness''
of `social intelligence?'


Persistent gaps

\begin{enumerate}
    \item Lack of awareness about the role of normativity in energy modeling. Only one 
    paper in the review by McGookin et al. 2021 investigated the normative assertion 
    that renewable energy is the ``cornerstone'' of decarbonization \cite{zelt_long-term_2019}.
    \item Lack of understanding the normative premises that inform the rationale
    for participatory methods. For example, there is little to no discussion about 
    disproportionality in the relevant literature. Only one paper mentions ``procedural justice''
    indicating that it's not yet a salient term among the modeling community \cite{knudsen_local_2015}.
    \item There is a dearth of studies considering a participation in a U.S. context.
    \item Studies reviewing model development focus on skill-based barriers rather than
    structural barriers to energy modeling at the municipal level.
    \item Although some studies in the McGookin review included non-academic participants, the
    connection between the study objectives and actionable policies is unclear. How does modeling
    fit into the policymaking process?
\end{enumerate}

% Questions to answer:
% \begin{enumerate}
%     \item Where have these studies been done? What are some characteristics of
%     these municipalities (e.g., size, country, etc.)?
%     \begin{enumerate}
%         \item McKenna et al. 2018 \cite{mckenna_combining_2018} conducted their
%         study in Ebhausen, Germany. This rural municipality had a population of
%         4,700 in 2013 and covered 25 square kilometers. It has no industrial
%         loads and few commercial uses. This raises a question about engagement
%         and decision-making power. Smaller governments are naturally more
%         responsive to their constituents than larger, representative, ones.
%         \item Johannsen et al. 2023 \cite{johannsen_municipal_2023} used the
%         town of Oud-Heverlee, Belgium as a case study. Oud-Heverlee had a
%         population of 11,000 in 2023 with mostly residential energy demand
%         (88\%) and no industrial demands. But they considered multiple energy
%         sectors such as electricity, heat, and transportation.
%         \item Fleischhacker et al. 2019 \cite{fleischhacker_portfolio_2019}
%         investigated a single district in Linz, Austrian.
%         \item Zelt et al. 2019 \cite{zelt_long-term_2019} looked decarbonization
%         scenarios for three countries: Jordan, Morocco, Tunisia.
%     \end{enumerate}
%     \item Do the researchers mention energy justice? What kinds of justice are
%     mentioned?
%     \begin{enumerate}
%         \item McKenna et al. 2018 \cite{mckenna_combining_2018} do not mention
%         justice, equity, nor fairness in the text of the paper.
%         \item Johannsen et al. 2023 \cite{johannsen_municipal_2023} do not
%         mention justice, equity, nor fairness in the text of the paper.
%         \item Fleischhacker et al. 2019 \cite{fleischhacker_portfolio_2019} do
%         not mention justice, equity, nor fairness in the text of the paper.
%         \item While Zelt et al. 2019 \cite{zelt_long-term_2019} do not
%         explicitly mention justice, they conclude that ``[r]epresentatives of the
%         population need to be included in the discussion on the countries’
%         future electricity supply to increase public support for national
%         targets. There should be a strategy to assure that all societal groups
%         are empowered to participate in this process. Finally, action needs to
%         be taken to ensure that the local population has its fair share of
%         benefits from the increase in renewable energy technologies, such as new
%         green jobs or financial participation in decentralised projects.'' However,
%         the authors do not expand on their view of fairness or justice, indicating
%         a normative uncertainty. Also of note, the authors wanted to explicitly
%         investigate the ``normative'' assumption that renewable energy is the
%         ``cornerstone'' to decarbonization.
%     \end{enumerate}
%     \item What modeling methods have the studies used?
%     \begin{enumerate}
%         \item McKenna et al. 2018 \cite{mckenna_combining_2018} used the
%         \ac{reason} model. This model allows users to optimize cost, carbon
%         emissions, and imports. Social acceptance is modeled as constraint be
%         strictly disallowing certain technologies. This is interesting because
%         \ac{osier} enables bespoke objective functions and could account for
%         social acceptance as a competing objective. Thus this study cannot
%         elucidate the tradeoffs between technology preferences and performance
%         objectives. Additionally, they did not fully co-optimize the three
%         objectives idenfitied by the community. Each option presented held one
%         of the three objectives as a free variable.
%         \item Johannsen et al. 2023 \cite{johannsen_municipal_2023} applied
%         \ac{moo} with the EnergyPLAN model to study the energy system of
%         Oud-Haverlee. They draw a distinction between ``expert-based
%         simulation,'' ``near-optimal solutions,'' and \ac{moo}, suggesting that
%         optimization methods are too structured for deep engagement with a
%         community, outside of filtering results. This thesis directly challenges
%         that assertion by developing \ac{osier}, a modeling tool capable of
%         user-defined objective functions.
%         \item Fleischhacker et al. 2019 \cite{fleischhacker_portfolio_2019} used
%         \acf{ec} to optimize solutions from the URBS model.
%         \item Zelt et al. 2019 \cite{zelt_long-term_2019} used the ``renpass'' model.
%     \end{enumerate}
%     \item Why do the researchers believe or assert that community engagement is
%     important?
%     \begin{enumerate}
%         \item McKenna et al. 2018 \cite{mckenna_combining_2018} focused on a
%         smaller community because smaller municipalities ``typically have fewer
%         technical, administrative, and economic resources to devote to
%         sustainability projects.'' However, the authors do not indicate that
%         participation yields superior outcomes in general. Rather, they simply
%         that local context is important and that the study was done to help an
%         enthusiastic but otherwise ill-equipped community. Thus the role of
%         participation was more scale-oriented than process-oriented.
%         \item Johannsen et al. 2023 \cite{johannsen_municipal_2023} were light
%         on discussion about the significance of local participation, except
%         where it does and doesn't play a role in the modeling process, and that
%         local context is important for translating issues into an optimization
%         problem.
%         \item Fleischhacker et al. 2019 \cite{fleischhacker_portfolio_2019} do
%         not mention ``engagement'' or ``participation'' anywhere in the article,
%         but assert that the tools they present will allow ``stakeholders'' to
%         calculate capabilities and restrictions of the local energy system. They
%         perceive ``city planners'' as the ultimate end-users.
%     \end{enumerate}
%     \item How did the researchers engage with communities?
%     \begin{enumerate}
%         \item McKenna et al. 2018 \cite{mckenna_combining_2018} engaged with the
%         community by iterating between a focus group workshop and model
%         refinement. The first workshop identifies community priorities, followed
%         by an iteration where preference ``weights'' are elicited, and a final
%         workshop where consensus is hopefully reached (though the authors
%         acknowledge the possible need for additional iterations). The authors
%         describe methods for structuring problems, such as developing
%         ``cognitive maps.''
%         \item Johannsen et al. 2023 \cite{johannsen_municipal_2023} did not
%         engage with the community of Oud-Haverlee in a meaningful way.
%         \item Fleischhacker et al. 2019 \cite{fleischhacker_portfolio_2019} did
%         not describe any engagement with the studied community.
%     \end{enumerate}
%     \item What objectives did the communities have?
%     \begin{enumerate}
%         \item In the study by McKenna et al. 2018 \cite{mckenna_combining_2018},
%         the community had three core objectives: affordability, environmental
%         (lower carbon emissions), and autonomy (fewer energy imports).
%         \item Johannsen et al. 2023 \cite{johannsen_municipal_2023} restricted
%         themselves to total cost and carbon dioxide emissions based on the
%         limitations of the EnergyPLAN model. They did not consult the studied
%         community for their priorities.
%         \item Fleischhacker et al. 2019 \cite{fleischhacker_portfolio_2019}
%         optimized on cost and emissions, without consulting local stakeholders.
%     \end{enumerate}
%     \item How did the researchers and communities select certain energy
%     portfolios?
%     \begin{enumerate}
%         \item Participants and the authors of McKenna et al. 2018
%         \cite{mckenna_combining_2018} used \ac{mavt}, a method used for
%         \ac{mcda}.
%         \item No final selection was made in Johannsen et al. 2023
%         \cite{johannsen_municipal_2023} since they did not conduct a visioning
%         exercise. Results consisted of complete Pareto fronts.
%         \item Similar to Johannsen et al. 2023, Fleischhacker et al. 2019
%         \cite{fleischhacker_portfolio_2019} did not conduct a visioning
%         exercise, rather they demonstrated tradeoffs and policy impacts (e.g.,
%         carbon taxes) without choosing or recommending a particular solution.
%     \end{enumerate}
%     \item Did the authors have any calls for action?
%     \begin{enumerate}
%         \item McKenna et al. 2018 \cite{mckenna_combining_2018} suggested
%         developing tools to allow communities to conduct studies on their own
%         with limited external guidance. \textcolor{red}{This is something I
%         directly tried to address in my interviews.}
%         \item The authors of Johannsen et al. 2023
%         \cite{johannsen_municipal_2023} indicate that strict focus on carbon
%         emissions and costs disregard the potential to ``evaluate scenarios
%         based on an additional set of parameters.'' Again, this is precisely why
%         I developed \ac{osier}.
%         \item Fleischhacker et al. 2019 \cite{fleischhacker_portfolio_2019} did
%         not have a call to action. For other modelers, but identified future
%         technical work.
%     \end{enumerate}
%     \item How is my work extending the work of this literature? Especially as it
%     pertains to the interviews I conducted.
%     \begin{enumerate}
%         \item .
%     \end{enumerate}
% \end{enumerate}

In summary, climate change is a multi-dimensional existential threat to society.
Transitioning to a zero-carbon economy by decarbonizing our energy systems may
prevent the worst outcomes of climate change. However, energy systems do not
only transport electrons and gas but also mediate socio-political power.
Therefore this transition must be done equitably in order to avoid entrenching
further injustices. The existing energy system modeling tools and literature
routinely ignore the social dimensions of these systems and forego true
trade-off analysis. Additionally, it's unclear whether improving these modeling
practices will correspond to just energy policy outcomes. This thesis attempts
to bridge the gap between energy system modeling and energy justice by
developing a novel framework that allows multiple, and perhaps non-economic,
objectives and is designed for transparency and usability by non-modelers to
inform energy policy decisions. A framework such as the one developed in this
thesis may be used in conjunction with a policy process like \ac{pve} or
participatory modeling to fully enclose the triumvirate of energy justice
tenets: distribution, procedure, and recognition.
