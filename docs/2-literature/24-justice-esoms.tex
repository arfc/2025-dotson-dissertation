\section{Modeling and Quantifying Energy Justice}

The dearth of studies that incorporate energy justice into \acp{esom} highlights
the challenge of combining these techniques. The literature on energy justice
and socio-technical transitions tend to derogate modeling efforts as cold and
calculating \cite{sovacool_energy_2015,sovacool_energy_2016}, and most models do
not account for energy justice in either equations or analysis. However, there
have been some notable attempts to bridge this gap. The following studies by
Patrizio et al. (2020) \cite{patrizio_socially_2020} and Neumann \& Brown (2021)
\cite{neumann_near-optimal_2021} explicitly use \acp{esom} in their analyses.
Although the works by Chapman et al. (2018) \cite{chapman_prioritizing_2018} and
Mayfield et al. (2019) \cite{mayfield_quantifying_2019} do not use \acp{esom} as
described in Section \ref{section:esoms}, these contributions quantify some
features of their respective energy systems and how they relate to notions of
energy justice and equity.

Patrizio et al. (2020) conducted a technology-agnostic `social equity' scenario
that maximized the \ac{gva} of several countries' energy systems rather than
minimizing the total cost \cite{patrizio_socially_2020}. \Ac{gva} is also
distinct from social welfare because it measures contributions to \ac{gdp} from
individual producers rather than maximizing surplus. This metric enables
sector-specific analysis of the impacts of energy infrastructure on employment
and sales. Equity, in this context, is identical to socioeconomic development as
measured by \ac{gdp}. Using this definition of equity, the researchers looked at
a socio-technical transition for three countries: Spain, the United Kingdom, and
Poland. They found that a 100\% renewable energy system would reduce labor
compensation by 50-60\% in the UK and Poland but could increase benefits in
Spain. They argue this is due to the outsourcing of manufacturing and mining
jobs in the former cases, while Spain has enough domestic resources to
accommodate the transition. The researchers did not analyze possible shifts in
power dynamics related to the energy systems, but they did identify that there
is no one-size-fits-all solution to achieving net-zero carbon emissions.

Neumann \& Brown (2021) performed a detailed analysis of the European energy
system considering the expansion of transmission networks and energy producers
for a 100\% renewable energy system under cost minimization
\cite{neumann_near-optimal_2021}. They also used a novel formulation of \ac{mga}
to identify the boundaries of the feasible space for each technology within
different levels of tolerance. That study used Lorenz curves and Gini
coefficients to measure the uniformity of the distribution of energy production
and consumption. In other words, the most equitable distribution of energy
resources would accord with energy consumption \cite{neumann_near-optimal_2021}.
The researchers conclude that wind power and greater transmission capacity are
associated with less regional equity, while solar power and storage technologies
lead to a more even distribution of the power supply. This is useful for
measuring the distribution of energy benefits from the energy system but does
not consider the distribution of costs nor consider regional preferences. 

Chapman et al. (2018) looked at the energy justice implications of transitioning
coal plants to renewable energy projects for the nearby communities
\cite{chapman_prioritizing_2018}. They measure distributional justice with
``relative equity'' and ``policy burden.'' Relative equity accounts for factors
such as \ac{ghg} reduction, employment, electricity cost, and health impacts.
Policy burden is a weighted value according to the income level of each
community. These two quantities were plotted together to identify a retirement
schedule that maximizes equity outcomes and ensures that burdens are borne by
the most capable communities \cite{chapman_prioritizing_2018}. Additionally, the
researchers argue that by using equity measures to inform policy choices, those
policy decisions are more procedurally just. However, this neglects meaningful
participation and may or may not address decision-making transparency
\cite{sovacool_energy_2015}. Further, this study does not consider how replacing
dispatchable suppliers with \ac{vre} will affect the availability and
affordability of electricity \cite{sovacool_energy_2015}. This latter challenge
could be addressed by incorporating methods from the \ac{esom} literature. The
former issue of decision-making transparency is one of the motivations for this
thesis.

Mayfield et al. (2019) quantified the social equity implications for the
expansion of natural gas infrastructure in Appalachia using spatial and temporal
metrics such as job-years generated by greater gas development, premature deaths
caused by air pollution, changes in poverty and income, and the distribution of
these various benefits along regional, racial, and economic lines
\cite{mayfield_quantifying_2019}. Additionally, they identified some of the
intergenerational equity impacts of climate change and expanded gas
infrastructure. 

\subsection{Enabling Procedural Justice Through Energy Models}

Traditionally, \acp{esom} are used to inform policy-makers \cite{li_open_2020}
in order to infuse policy choices with an appearance of objectivity. Indeed,
some of the studies reviewed in the previous section argue that this infusion
will lead to greater procedural and recognitional justice outcomes as long as
the policies maximize some measure of energy justice
\cite{chapman_prioritizing_2018, heffron_resolving_2015}. However, these types
of detailed analyses may also be used to dismiss concerns or opposition from the
public due to insufficient `technical expertise' \cite{johnson_dakota_2021,
susser_better_2022}. Further, without meaningful participation from the affected
public, this approach further entrenches procedural injustices. To credit the
energy modeling community, there is significant awareness of the importance of
transparency and repeatability in the space \cite{decarolis_case_2012,
pfenninger_energy_2014, pfenninger_openmod_2022, forster_open_2022,
hilpert_open_2018}. Yet these two goals are challenged by the computational
resources required to run the more complex and detailed models, as well as the
learning curve necessary to understand and modify the model inputs themselves.
There has been some effort to reduce this learning curve and make modeling
itself more accessible. Frameworks such as METIS, EnergyRT, and \ac{pygen} all
emphasize reproducibility, user-friendliness, and a shallower learning curve
\cite{sakellaris_metis_2018, lugovoy_energyrt_2022, dotson_python_2021}. The
creators of METIS state their goal is to ``close the gap between modelers and
policy-makers, enabling policy-makers to become modelers''
\cite{sakellaris_metis_2018}. However, these frameworks do not offer
computational resources to run their models. The \ac{temoa} project offers
limited cloud computing capabilities, free of charge
\cite{temoa_project_temoa_2023}. However, the responsibility for creating an
input file still falls to the user, which can be overwhelming even for
experienced modelers. For this reason, many municipalities hire external
organizations to run models since cities and towns typically do not have the
capacity to do it themselves \cite{ben_amer_too_2020, johannsen_designing_2021}.
Additionally, researchers identified gaps between the needs of modelers and
users, such as the inflexibility of models to reflect multiple policy options.
``Many models include a carbon price, often labelled as `climate policy,' as the
only explicit policy instrument'' \cite{susser_better_2022}. Despite broad
agreement about the need to incorporate stakeholders the ``process'' of
modeling, there have been surprisingly limited efforts so far
\cite{ben_amer_too_2020,johannsen_designing_2021,susser_better_2022}. Finally,
it's not clear that perfectly accessible and transparent modeling tools will
translate to more procedurally just policy-making. The next sections outline
some methods used to address this challenge.

\subsection{\Acl{pve}}

Even if the public could use modeling tools, their testimony may still be
dismissed due to a `lack of expertise.' However, the public has preferences that
should be incorporated into decision-making. Additionally, community members are
frequently able to assess tradeoffs when presented with them. \Acf{pve} is one
method for translating community preferences into just policy outcomes.
Researchers in the Netherlands developed this method to enhance democratic
participation and infuse policies with genuine feedback from constituents
\cite{mouter_introduction_2019}. They observed that a common method of assessing
social impacts is \ac{wtp}, which is the maximum price an individual is willing
to pay for a good or service, yet individual purchasing habits do not
necessarily reflect their views on public policy due to the relative salience of
moral considerations \cite{mouter_introduction_2019}. With \ac{pve},
participants can allocate a specific amount of the public budget for certain
policies, including levying or reducing taxes for greater or lesser government
spending \cite{mouter_introduction_2019}. Researchers applied \ac{pve} in three
different settings, mobility and transportation \cite{mouter_contrasting_2021},
flood risk projects (i.e., a climate hazard \textit{infrastructure} response)
\cite{dekker_economics_2019}, and with a phaseout of natural gas
\cite{mouter_including_2021}. Importantly, the studies also measured the impact
of these interventions and found that \ac{pve} enables participation from people
that do not typically participate (recognition), the results were useful for
decision-making and participation was meaningful for the majority of subjects
\cite{mouter_including_2021}. Although previous applications of \ac{pve} focused
on economic policy levers, this approach offers a promising pathway toward
identifying equitable and just energy mixes for the future. The next section
reviews some efforts to enhance outcomes via direct collaboration between energy
modelers and communities themselves.

\subsection{Participatory modeling processes}

Although \ac{pve} offers one method to elicit stakeholder input, researchers are
beginning to explore transdisciplinary methods for engaging members of the
public. McGookin et al. 2021 \cite{mcgookin_participatory_2021} systematically
reviewed (``the review'') existing literature on participatory energy modeling.
The review found a major difference in goals and participants for studies in
either national or local contexts. Studies at the national scale sought
impactful policy outcomes and engaged with policy technical experts, while
localized projects aimed to materially improve the studied communities
\cite{mcgookin_participatory_2021}. However, even in the latter cases, only a
fraction of the reviewed studies included non-academic stakeholders. This
suggests that local knowledge is still an under-utilized resource and that
researchers need to do more to identify communities' actual priorities.

The review offered three overarching benefits of participatory modeling. First,
involving more stakeholders improves perceptions of legitimacy and robustness
\cite{mcgookin_participatory_2021}. Discussion and tradeoff analysis lead to
solutions which are more socially and politically feasible. This corroborates
literature from social movement theory which show that support for energy
policies and projects is conditioned on genuine public participation rather than
\ac{nimby} sentiments \cite{summers_influencing_2020,ottinger_procedural_2014,
walker_procedural_2017,barragan-contreras_procedural_2022,gonyo_resident_2021,konisky_proximity_2021}.
McGookin et al. also indicate that transparency in the model assumptions builds
trust. Second, participatory methods build capacity for individuals and
communities. Non-academic participants benefit from learning about new
solutions, while policymakers and modelers gain new perspectives and a better
understanding of a community's reality, improving decision quality. Communities
benefit through the formation of new social networks and stronger relationships
among stakeholder groups. The benefit to communities found in the review aligns
with the understanding of energy systems as a ``gathering force'' that can
produce new polities and social identities \cite{bridge_energy_2018}. Further,
the review notes that one of the keys for enhancing decision-making at both
local and national levels ``was the insight into systems thinking and tradeoffs
or cause-effect relationships that participants gained from the methods used in
studies.'' \Ac{mcda}, a class of methods for analyzing tradeoffs in different
scenarios, was the most popular tool found in the review
\cite{mcgookin_participatory_2021}. Lastly, the authors note that consensus
building through controversial discussions, collective learning, and jointly
owned solutions was effective for building trust. Both national and local
studies found increased commitments for stakeholders to implement co-created
decisions \cite{mcgookin_participatory_2021}. These findings support greater
participation in a ``deliberative democracy'' \cite{dryzek_deliberative_2013}.
While the review does not explicitly identify these benefits with energy
justice. The benefits of particpatory energy modeling evinced by the literature
are consistent with both procedural and recognition justice.



% \textcolor{red}{This enumeration will be good for the following chapter where
% I introduce my theoretic framework --- \textit{why} should we engage
% stakeholders in modeling?} The authors proposed a conceptual framework for
% successful integration of participatory methods. \begin{enumerate} \item
% Stakeholder engagement must occur before drawing conclusions from a
% quantitative analysis. \textcolor{red}{I'm extending this conceptual framework
% by describing precisely where in the modeling process stakeholder input can
% and should take place, and in what forms.} \item There should be an iterative
% process where stakeholders can shape the process in addition to evaluating the
% results. \item Lastly, the ideal process should include stakeholder input at
% every stage. \end{enumerate}

\subsubsection{Persistent gaps}
Although there have been attempts to integrate energy justice and energy
modeling there are still gaps in the literature. These gaps include:
\begin{enumerate}
    \item Lack of awareness about the role of normativity in energy modeling.
    Only one paper in the review by McGookin et al. 2021 investigated the
    normative assertion that renewable energy is the ``cornerstone'' of
    decarbonization \cite{zelt_long-term_2019}.
    \item Lack of understanding the normative premises that inform the rationale
    for participatory methods. For example, there is little to no discussion
    about disproportionality in the relevant literature. Only one paper mentions
    ``procedural justice'' indicating that it's not yet a salient term among the
    modeling community \cite{knudsen_local_2015}.
    \item There is a dearth of studies considering participatory modeling in a
    U.S. context.
    \item Studies reviewing model development focus on skill-based barriers
    rather than structural barriers to energy modeling at the municipal level.
    \item Although some studies in the McGookin review
    \cite{mcgookin_participatory_2021} included non-academic participants, the
    connection between the study objectives and actionable policies is unclear.
    How does modeling fit into the policymaking process?
\end{enumerate}

In summary, climate change is a multi-dimensional existential threat to society.
Transitioning to a zero-carbon economy by decarbonizing our energy systems may
prevent the worst outcomes of climate change. However, energy systems do not
only transport electrons and gas but also mediate socio-political power.
Therefore this transition must be done equitably in order to avoid entrenching
further injustices. The existing energy system modeling tools and literature
routinely ignore the social dimensions of these systems and forego true
tradeoff analysis. Additionally, it's unclear whether improving these modeling
practices will correspond to just energy policy outcomes. This thesis attempts
to bridge the gap between energy system modeling and energy justice by
developing a novel framework that allows multiple, and perhaps non-economic,
objectives and is designed for transparency and usability by non-modelers to
inform energy policy decisions. A framework such as the one developed in this
thesis may be used in conjunction with a policy process like \ac{pve} or
participatory modeling to fully enclose the triumvirate of energy justice
tenets: distribution, procedure, and recognition.
