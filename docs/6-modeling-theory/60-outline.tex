Energy modeling is necessary to design energy systems that serve people,
especially because our energy systems are growing more complex due the increase
in different types of generation and the increase in time-dependent generation,
such as \acfp{vre}. Further, in order to capture the ``human dimension,'' energy
models should be designed to address three equal components of justice:
distribution, procedure, and recognition. There is already some work being done
to incorporate distribution justice into \acp{esom}
\cite{neumann_near-optimal_2021,jafino_enabling_2021}. Particularly because the
distribution of benefits and burdens are easily quantified. However, to capture
procedural and recognition justice, \acp{esom} must be part of a deliberative
participatory process where members of the public and other relevant
stakeholders co-create a model and model results. McGookin et al. 2024 describe
an ideal iterative process \cite{mcgookin_advancing_2024} and offer some methods
for engaging the public in discussions around input data and assumptions.
However, the study does not discuss how these discussions address uncertainties
present in modeling. It also does not elaborate methods for discussing results
with participants. In this chapter, I argue, first, that a participatory process
is not simply one way to encapsulate procedural and recognition justice, but the
only way. This is because of Arrow's Theorem, which says that a ``universal
preference order'' cannot exist without violating one or more democratic norms.
That is, faithfully aligning several preference orders has no algorithmic
solution and can only be resolved through deliberation. Second, I explain how
existing \acp{esom} fail to address all three types of uncertainty and how
\ac{osier} addresses this gap and thereby supports the inclusion of energy
justice into modeling and planning practices. Lastly, I expand on the procedure
described by McGookin et al. \cite{mcgookin_advancing_2024} by outlining the
process of developing an energy model according to three types of related
uncertainties.

Section \ref{section:arrows-thm} introduces Arrow's theorem and explains its
significance for procedural justice and energy planning. Section
\ref{section:participation-modeling} outlines a participatory modeling process
described by McGookin et al. \cite{mcgookin_advancing_2024}. Section
\ref{section:triple-uncertainty} elaborates on three types of uncertainty,
introduced in Chapter \ref{chapter:lit-review}, that modeling must address.
Lastly, Section \ref{section:modeling-phases} describes the relationships among
each type of uncertainty and how the correspond to different phases of a
modeling exercise.


\section{Arrow's Impossibility Theorem}
\label{section:arrows-thm}

Modern social-choice theory was founded in 1950 with the publication of Arrow's
\textit{A Difficulty in the Concept of Social Welfare}
\cite{arrow_difficulty_1950}. The field of social-choice theory concerns itself
with how society makes rational decisions given the plurality of individual
preferences. Formally, given a set of possible choices $a_i (i: 1,..., N)$, and
a set of individuals $x_j (j:1, ...,M)$, each with a preference order, $o_j$, over the
options that allows for strict ranking ($a_i > a_j$) and equivalence ($a_i \sim
a_j$), how should the group decide the best option $a_i$
\cite{franssen_arrows_2005}? Ideally, there exists a utility function, $f$, that maps
the collection of individual preference orders, $P$ onto a collective preference order, $O$.
$f: P \mapsto O$. Of course, there are trivial solutions, such as a dictatorship where
only one preference order matters. To ensure a ``faithful translation'' of $P$ onto $O$, 
Arrow proposed a set of constraints on $f$ \cite{arrow_difficulty_1950,franssen_arrows_2005}.
Given a set of options $A$ the following must be satisfied:
\begin{enumerate}
    \item \textbf{Collective Rationality:} The collection of preference orders, $P \subset A\times A$,
    must be
    \begin{enumerate}
        \item complete or connected --- all options and their relationships are represented in at
        least one preference order, i.e., $(a_i > a_j | a_i < a_j)$ $\in$ $P$, and
        \item transitive --- if there are preference orders $a_i > a_j$ and $a_j > a_k$ in $P$, then
        $a_i > a_k$ must also exist in $P$.
    \end{enumerate}
    \item \textbf{Unrestricted Domain:}
    There are no preference orders that are \textit{a priori} excluded from the domain $P$.
    \item \textbf{Non-dictatorship:} There is no individual $x_i$ whose preferences always prevail. That is,
    for all possible $P \in (P_1, ... P_N) \in \Pi(P)^N$ , when $x_i$ prefers $a_i > a_j$ that the utility function $f(P_i)$ always
    results in $a_i > a_j$.
    \item \textbf{Pareto Efficiency:} If all individuals prefer $a_i > a_j$, then $a_i$ is strictly better
    for any $f(P_i)$.
    \item \textbf{\ac{iia}:}
\end{enumerate}

Essentially, Arrow's theorem says that a universal preference order cannot exist
without violating one or more democratic norms. Although originally developed
with voting systems in mind, Franssen argues
that \blockcquote[p. 42]{franssen_arrows_2005}{Arrow’s theorem applies fully to
multi-criteria decision problems as they occur in engineering design, making
solution methods to such problems subject to the theorem’s negative result.}

\section{Participatory Energy Modeling}
\label{section:participation-modeling}

\section{Three Types of Uncertainty}
\label{section:triple-uncertainty}

\subsection{Parametric uncertainty}

\subsection{Structural uncertainty}

\subsection{On normative uncertainty}
In \acp{esom}, non-physical constraints are often introduced to direct the model
towards a desired outcome. For example, carbon emissions are often constrained
to identify a least cost pathway to achieving a specified emissions target by a
certain date. If the model's results are subsequently used to support policies
that advance society towards that emissions target, then the modeled constraint
represents a normative position. That society \textit{should} pursue this goal.
In the case of carbon emissions, this is frequently stated outright (cite some
papers that do this?). There are other constraints that researchers sometimes
apply which might be construed as a normative position but go unstated and
unsupported. Thus, introducing a \textit{normative uncertainty}. Normative
uncertainty, that uncertainty which arises from the existence of a plurality of
equally good alternatives, is seldom discussed in the literature (cite?).
\textcolor{red}{Can I say precisely how many papers discuss normative
uncertainty? If it's a small number, perhaps I can just state it.}

In modeling practice, normative uncertainty shows up when there are weakly
justified inputs. For example, constraining a model to building only a certain
amount of capacity for a particular technology because the results would be
otherwise ``unbelievable.'' There are good reasons to include such build limits.
If you're modeling a region with an explicit policy ban on a technology, or if
there is simply not enough land area to support an unlimited capacity of
renewable energy resources. In these cases, it makes sense to constrain a
technology's capacity to its \textit{technical potential.} Citing believability,
however, represents a normative uncertainty because there are infinitely many
narrative possibilities to explain a particular capacity constraint. For
concreteness, consider an \ac{esom} with a nuclear reactor technology and an
accompanying capacity constraint. There are myriad reasons to limit the
technical potential for nuclear capacity, including geography, politics (e.g., a
nuclear moratorium), and resource availability. However, limiting the capacity
of a nuclear technology because the results would otherwise show an
``unbelievable'' amount of nuclear capacity introduces a confounding normative
uncertainty. This should be avoided for at least four reasons.
\begin{enumerate}
    \item If limiting nuclear energy is a policy objective then artificially
    constraining the model's investment in nuclear may inadvertently lead to
    more nuclear capacity in the real world because the \ac{esom} modeled a
    policy environment that was actually beneficial to nuclear energy but could
    not demonstrate this due to the modeled constraints.
    \item Conventional \acp{esom} minimze system cost, limiting the buildout of
    nuclear energy is, at best, a non-binding constraint, and at worst increases
    the value of the objective function, creating an illusion of a ``least
    cost'' solution.
    \item If my goal is to inform a policy or set of policies that lead to a
    particular outcome, such as reducing carbon emissions, a constraint that
    increases the objective value might make the desired outcome economically
    unappealing. 
    \item Limiting the capacity obfuscates the policy changes that might be
    needed to achieve a desired future.
    \item Existence of such a constraint invites dismissal of the entire
    modeling endeavor via Hitchens' Razor: What can be asserted without evidence
    can be dismissed without evidence.
    \item In a multi-objective model, an unjustified build limit reduces the
    efficacy of tradeoff analysis.
\end{enumerate}

It is best to state and defend a normative position (e.g., "Nuclear energy is
undesirable, here is why") than to introduce normative uncertainty.
\textcolor{red}{Using a tool like \ac{osier} reduces normative uncertainty
because it forces modelers and decisionmakers to confront their normative
positions in tradeoff analysis. Choosing a particular solution among a set of
co-optimal solutions requires articulating why one solution is preferable to
another and lays bare the question of personal values.}

Some normative positions of this thesis are:
\begin{enumerate}
    \item Reducing carbon emissions is paramount.
    \item Technology agnostic: Models will not artificially favor any particular
    technology.
    \item Policy outcomes are better when more diverse voices are included in
    the process.
\end{enumerate}

\section{Modeling phases}
\label{section:modeling-phases}

\subsection{Pre-descriptive}

Link between normative and parametric uncertainties. Normative decisions (i.e.,
assumptions) are made about different model parameters. Along with which
technologies are represented. At this stage, the uncertainties addressed range
between aleatory and epistemic. The former is a ``known unknown'' that could be
improved with more research. The latter corresponds to ``unknown unknowns'' or a
``deep uncertainty'' that cannot be improved (cite).

Engagement with the public is still important, here, since addressing uncertain
parameters, such as technology costs, depends on your impression of the
underlying data distribution.


\subsection{Descriptive}
Link between parametric and structural. This is where the ``modeling'' happens.
Constraints and objectives give a model its descriptive power.

For example, in this phase it might be important to introduce specific
retirement schedules for power plants based on community feedback. Illinois
legislators included such a schedule in the \acf{ceja} to retire coal and gas
plants based on input from frontline communities in Illinois.

\subsection{Prescriptive}
Link between structural and normative. Model results are interpreted and policy
outcomes are prescribed. Alternatively, researchers may start the cycle anew by
using the model results to adjust the normative assumptions. Perhaps because the
model results don't show a desirable future?