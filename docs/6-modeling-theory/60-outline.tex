
\textcolor{red}{This chapter is new in the dissertation! This is where I will handle 
Denia's comments about being clear on my normative stances.}

\textcolor{red}{The argument advanced in this chapter is that the practice of modeling
involves iteratively addressing three types of uncertainties. In the chapter I will
describe these types of uncertainties and the phases they represent.}

\section{Arrow's Impossibility Theorem}

This section will motivate the importance of contextualizing an analysis based on 
dialogue with communities (at any level) using participatory methods rather than
surveys or assumptions about items.

Essentially, Arrow's theorem says that a universal preference order cannot exist without
violating one or more democratic norms.

This is Arrow's original paper on social choice theory \cite{arrow_difficulty_1950}.

This paper talks about \ac{mcda} and Arrow's Theorem \cite{franssen_arrows_2005}.

\section{Three Types of Uncertainty}

\subsection{Parametric uncertainty}

\subsection{Structural uncertainty}

\subsection{On normative uncertainty}
In \acp{esom}, non-physical constraints are often introduced to direct the model
towards a desired outcome. For example, carbon emissions are often constrained
to identify a least cost pathway to achieving a specified emissions target by a
certain date. If the model's results are subsequently used to support policies
that advance society towards that emissions target, then the modeled constraint
represents a normative position. That society \textit{should} pursue this goal.
In the case of carbon emissions, this is frequently stated outright (cite some
papers that do this?). There are other constraints that researchers sometimes
apply which might be construed as a normative position but go unstated and
unsupported. Thus, introducing a \textit{normative uncertainty}. Normative
uncertainty, that uncertainty which arises from the existence of a plurality of
equally good alternatives, is seldom discussed in the literature (cite?).
\textcolor{red}{Can I say precisely how many papers discuss normative
uncertainty? If it's a small number, perhaps I can just state it.}

In modeling practice, normative uncertainty shows up when there are weakly
justified inputs. For example, constraining a model to building only a certain
amount of capacity for a particular technology because the results would be
otherwise ``unbelievable.'' There are good reasons to include such build limits.
If you're modeling a region with an explicit policy ban on a technology, or if
there is simply not enough land area to support an unlimited capacity of
renewable energy resources. In these cases, it makes sense to constrain a
technology's capacity to its \textit{technical potential.} Citing believability,
however, represents a normative uncertainty because there are infinitely many
narrative possibilities to explain a particular capacity constraint. For
concreteness, consider an \ac{esom} with a nuclear reactor technology and an
accompanying capacity constraint. There are myriad reasons to limit the
technical potential for nuclear capacity, including geography, politics (e.g., a
nuclear moratorium), and resource availability. However, limiting the capacity
of a nuclear technology because the results would otherwise show an
``unbelievable'' amount of nuclear capacity introduces a confounding normative
uncertainty. This should be avoided for at least four reasons.
\begin{enumerate}
    \item If limiting nuclear energy is a policy objective then artificially
    constraining the model's investment in nuclear may inadvertently lead to
    more nuclear capacity in the real world because the \ac{esom} modeled a
    policy environment that was actually beneficial to nuclear energy but could
    not demonstrate this due to the modeled constraints.
    \item Conventional \acp{esom} minimze system cost, limiting the buildout of
    nuclear energy is, at best, a non-binding constraint, and at worst increases
    the value of the objective function, creating an illusion of a ``least
    cost'' solution.
    \item If my goal is to inform a policy or set of policies that lead to a
    particular outcome, such as reducing carbon emissions, a constraint that
    increases the objective value might make the desired outcome economically
    unappealing. 
    \item Limiting the capacity obfuscates the policy changes that might be
    needed to achieve a desired future.
    \item Existence of such a constraint invites dismissal of the entire
    modeling endeavor via Hitchens' Razor: What can be asserted without evidence
    can be dismissed without evidence.
    \item In a multi-objective model, an unjustified build limit reduces the
    efficacy of tradeoff analysis.
\end{enumerate}

It is best to state and defend a normative position (e.g., "Nuclear energy is
undesirable, here is why") than to introduce normative uncertainty.
\textcolor{red}{Using a tool like \ac{osier} reduces normative uncertainty
because it forces modelers and decisionmakers to confront their normative
positions in tradeoff analysis. Choosing a particular solution among a set of
co-optimal solutions requires articulating why one solution is preferable to
another and lays bare the question of personal values.}

Some normative positions of this thesis are:
\begin{enumerate}
    \item Reducing carbon emissions is paramount.
    \item Technology agnostic: Models will not artificially favor any particular technology.
    \item Policy outcomes are better when more diverse voices are included in the process.
\end{enumerate}

\section{Modeling phases}

\subsection{Pre-descriptive}

Link between normative and parametric uncertainties. Normative decisions (i.e., assumptions)
are made about different model parameters. Along with which technologies are represented.
At this stage, the uncertainties addressed range between aleatory and epistemic. The former 
is a ``known unknown'' that could be improved with more research. The latter corresponds to 
``unknown unknowns'' or a ``deep uncertainty'' that cannot be improved (cite).

\subsection{Descriptive}
Link between parametric and structural. This is where the ``modeling'' happens. Constraints
and objectives give a model its descriptive power.

\subsection{Prescriptive}
Link between structural and normative. Model results are interpreted and policy outcomes
are prescribed. Alternatively, researchers may start the cycle anew by using the model
results to adjust the normative assumptions. Perhaps because the model results don't show
a desirable future?