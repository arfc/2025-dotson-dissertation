\section{Literature Review}
Chapter \ref{chapter:lit-review} provided a broad review of the existing
literature that contextualizes this thesis. This section offers a focused review
of the literature on municipal energy planning and modeling practices.

\subsection{Municipal use of energy models}

Energy modeling has focused on energy systems at national and regional scales
due to the traditionally centralized nature of energy infrastructure. However,
the spread of \acp{der} suggests that municipalities have an opportunity to play
a larger role in shaping the clean energy transition
\cite{shen_facilitating_2021}. Despite this, both Ben Amer et al. 2020, and
Johannsen et al. 2021 found that \acp{esom} are seldom used at the municipal
level \cite{ben_amer_too_2020,johannsen_designing_2021}. These studies agree
that municipal planners do not have time nor the requisite expertise to use
\acp{esom} themselves and generally hire consultants to perform the analyses,
when needed. They further argue that municipal planners do not employ \acp{esom}
because input data are difficult to find, the models do not have a scope
suitable for municipalities, and they lack desirable features. Indeed,
S\"{u}sser et al. investigated the needs of modelers and users and found that
modelers prioritize ``realism'' and complexity while users prefer
understandability and clear communication \cite{susser_better_2022}. All three
studies conclude that greater collaboration and earlier involvement between
modelers and planners would yield superior results. Only Ben Amer et al.
identify consulting and personnel costs as potential barriers to incorporating
modeling results into decision making processes.

% \textcolor{red}{Is the hypothesis I am testing with this study that
% ``municipalities would be better equipped to plan and design their energy
% systems if the available modeling tools were accessible and relevant.'' This
% hypothesis comes from the implicit suppositions in a few papers. }

\subsection{Energy justice and planning}

There is, to date, limited empirical work that investigates the inclusion of
justice into formal planning proceses at either state or municipal levels. One
study by Diezmart\'inez et al. looked at 58 climate action plans sampled from
the 100 largets U.S. cities \cite{diezmartinez_us_2022}. This study found that,
of the 58 climate action plans, roughly two-thirds of them attend to either
justice or equity, while the final third do not discuss either concept. The
researchers also found that recency, high levels of inequality within a city,
and public engagement were determining factors in whether a climate action plan
addressed justice. With greater public engagement leading to greater emphasis on
justice. Further, climate action plans developed by cities prefer ``equity''
over ``justice'' because discourses tend to focus on the superficial
distributions of benefits and burdens rather than accounting for structural
sources of injustice \cite{diezmartinez_us_2022}.

Another study reviewed 60 energy transition ``visioning documents'' prepared by
non-profits and advocacy organizations in the United States, rather than formal
planning documents developed by city officials \cite{elmallah_frontlining_2022}.
From these visioning documents, this study derived six principles of planning
for energy justice.
\begin{enumerate}
    \item Being place-based --- recognizing that each city's situation is unique
    and that there are no universal solutions.
    \item Addressing the root cause and legacies of inequality --- energy policy
    should simultaneously address other social crises and prioritize the needs
    of those most impacted by systems of inequality.
    \item Shifting the balance of power from existing forms of energy governance
    --- related to the procedural justice tenets of representation, information,
    and consent.
    \item Creating new, cooperative, and participatory systems of energy
    governance and ownership --- democratizing energy away from government or
    corporate control to community control.
    \item Adopting a rights-based approach --- the right to energy, labor
    rights, and the right to remain and build community.
    \item Rejecting false solutions --- solutions that focus on carbon reduction
    at the expense of political, economic, or social justice.
\end{enumerate}
The authors argue that these six principles should be used together when
crafting just energy policy.

% \subsection{Modeling for justice}

% Discuss the paper \cite{lonergan_energy_2023}

% \subsection{Novelty of this work}

% \textcolor{red}{What are some of the novelties of the present study}
% \begin{enumerate}
%     \item Considers structural barriers to energy modeling.
%     \item Investigates the role concepts of energy justice do (or do not) play
%     in energy policy for some places in Illinois.
%     \item Evaluates a specific energy modeling tool and features of that tool.
% \end{enumerate}

