\section{Introduction}

Earlier chapters introduced a novel modeling tool, \ac{osier}, which sought to
fill a gap in the literature and available modeling tools by developing a
flexible multi-objective optimization framework that allows users to incorporate
concepts of justice, the ``human dimension'' \cite{pfenninger_energy_2014}. This
is possible because \ac{osier} enables users to optimize any quantifiable
objective which encourages modelers to seek input from the communities they
model for their energy system priorities, as suggested by McGookin et al. 2024
\cite{mcgookin_advancing_2024}, thereby elevating the recognition and procedural
aspects of justice described in Chapter \ref{chapter:lit-review}. Additionally,
\ac{osier} transcends scale limitations due to it's flexible design, making it
useful for municipalities as well as larger governing units. This chapter
attempts to validate these assertions by examining how local and state energy
planners use \acp{esom} and the ways in which justice does or does not play a
role in decision-making. To that end, this chapter considers a range of contexts
in the State of Illinois, including three municipalities (Champaign, Urbana, and
Naperville), a university with its own microgrid (\acf{uiuc}), and Illinois
energy organizations (\acf{icc}, \acf{ipa}, \acf{icea}, and the Illinois
\acf{cub}). This chapter also considers the varied perspectives of different
energy planning participants within their respective contexts by interviewing
city planners, a city council member, and various advisory staff. This study
contributes to the existing literature on participatory and distributed energy
planning as well as illustrating a path towards a holistic integration of energy
justice and energy system engineering.

The next section reviews the existing literature on decentralized energy
planning. Section \ref{section:interview-methods} describes the method for
collecting and analyzing interview data using thematic analysis. Section
\ref{section:cases} introduces the different case regions considered in this
chapter. Section \ref{section:interview-results} presents the results from the
interviews. Lastly, Section \ref{section:interview-discussion} discusses the 
implications for policy and model development.