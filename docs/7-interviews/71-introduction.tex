\section{Introduction}

Recent literature on the challenges facing energy system modeling call for new
\acp{esom} to address a so-called ``human dimension'' of energy systems
\cite{pfenninger_energy_2014}. The human dimension of energy systems relates to
the sociopolitical environment that both shapes and is shaped by access to
energy. These calls from the literature presuppose that if \acp{esom} can
adequately capture the human dimension then they can be used to create more
equitable and just energy policies. However, this argument rests on three
assumptions. First, that decision-makers use \acp{esom} and their results to
inform policy choices. Second, that decision-makers consider equity and justice
when designing policies. Third, that \acp{esom} can be designed in a way that
properly incorporates the human dimension. To date there are no examples in the
literature of researchers investigating all three assumptions about energy
planning with \acp{esom} simultaneously. To address this gap, I investigate how
the state of Illinois, and some municipalities therein, use \acp{esom} to inform
energy policy and advance goals of energy justice. I chose Illinois for two
reasons. First, its emissions reductions and clean energy targets are among the
most ambitious in the United States, making it an interesting case when
discussing the role of \acp{esom} in supporting the clean energy transition.
Second, Illinois has a variety of mechanisms for energy choice due to its status
as a restructured state but also because there are a number of independent
municipal utilities and \ac{mca} which add complexity to consumers' energy
options. To that end, this chapter considers a range of contexts in the State of
Illinois, including three municipalities (Champaign, Urbana, and Naperville) and
a university with its own microgrid (\acf{uiuc}). Together, these localities
provide a representative sample of municipalities and the available energy
choices within Illinois. Further the results may be extended to other places
that operate an open energy market. 

In this chapter I advance two arguments. First, the assumption that \acp{esom}
alone can ever adequately capture the human dimension is false due to the null
result from Arrow's Impossibility theorem. However, modelers can address the
human dimension, and thereby address issues of justice, by using \acp{esom} in a
participatory process that includes diverse perspectives. Earlier in this
thesis, I argue that \ac{osier}'s design supports and encourages the dialogue
necessary for these processes. This chapter provides evidence for this
assertion. Second I argue that, while knowledge deficits contribute to the full
adoption of \acp{esom} in energy planning, there are structural barriers built
into energy-related decision-making processes that prevent some municipalities
from using \acp{esom} effectively. In absence of policies that address these
structural barriers, it is essential that decision-making bodies with access to
\acp{esom} develop deliberative processes, supported by \acp{esom}, that include
voices from those municipalities that are structurally barred from using
\acp{esom}.

The next section reviews the existing literature on decentralized energy
planning and the intersections between energy justice and energy planning.
Section \ref{section:markets} provides important context by summarizing the
complex energy choices that consumers are faced with in Illinois. Section
\ref{section:cases} introduces the different case regions considered in this
chapter as well as a review of the climate and energy plans adopted by each
municipally. Section \ref{section:interview-methods} describes the method for
collecting and analyzing interview data using thematic analysis. Section
\ref{section:interview-results} presents the results from the interviews.
Lastly, Section \ref{section:interview-discussion} discusses the implications
for policy and model development.

% Arrow's theorem means that ESOMs can never adequately capture the human
% dimension by themselves but must be used as part of a participatory
% decision-making process.

