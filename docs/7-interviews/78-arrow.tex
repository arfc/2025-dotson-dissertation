\section{Arrow's Impossibility Theorem}
\label{section:arrows-thm}

Modern social-choice theory was founded in 1950 with the publication of Arrow's
\textit{A Difficulty in the Concept of Social Welfare}
\cite{arrow_difficulty_1950}. The field of social-choice theory concerns itself
with how society makes rational decisions given the plurality of individual
preferences. Formally, given a set of possible choices $a_i (i: 1,..., N)$, and
a set of individuals $x_j (j:1, ...,M)$, each with a preference order, $o_j$,
over the options that allows for strict ranking ($a_i > a_j$) and equivalence
($a_i \sim a_j$), how should the group decide the best option $a_i$
\cite{franssen_arrows_2005}? Ideally, there exists a utility function, $f$, that
maps the collection of individual preference orders, $P$ onto a collective
preference order, $O$. $f: P \mapsto O$. Of course, there are trivial solutions,
such as a dictatorship where only one preference order matters. To ensure a
``faithful translation'' of $P$ onto $O$, Arrow proposed a set of constraints on
$f$ \cite{arrow_difficulty_1950,franssen_arrows_2005}. Given a set of options
$A$ the following must be satisfied:
\begin{enumerate}
    \item \textbf{Collective Rationality:} The collection of preference orders,
    $P \subset A\times A$, must be
    \begin{enumerate}
        \item complete or connected --- all options and their relationships are
        represented in at least one preference order, i.e., $(a_i > a_j | a_i <
        a_j)$ $\in$ $P$, and
        \item transitive --- if there are preference orders $a_i > a_j$ and $a_j
        > a_k$ in $P$, then $a_i > a_k$ must also exist in $P$.
    \end{enumerate}
    \item \textbf{Unrestricted Domain:} There are no preference orders that are
    \textit{a priori} excluded from the domain $P$.
    \item \textbf{Non-dictatorship:} There is no individual $x_i$ whose
    preferences always prevail. That is, for all possible $P \in (P_1, ... P_N)
    \in \Pi(P)^N$ , when $x_i$ prefers $a_i > a_j$ that the utility function
    $f(P_i)$ always results in $a_i > a_j$.
    \item \textbf{Pareto Efficiency:} If all individuals prefer $a_i > a_j$,
    then $a_i$ is strictly better for any $f(P_i)$.
    \item \textbf{\Ac{iia}:} Given $a_i > a_j$ the addition or removal of a
    third option, $a_k \neq a_i, a_j$, does not change the original preference
    order to $a_i < a_j$.
\end{enumerate}

Essentially, Arrow's Impossibility theorem says that a universal preference
order cannot exist without violating one or more of the above constraints.
Franssen argues that \blockcquote[p. 42]{franssen_arrows_2005}{Arrow’s theorem
applies fully to multi-criteria decision problems as they occur in engineering
design, making solution methods to such problems subject to the theorem’s
negative result.} Arrow's theorem has at least three consequences for
participatory modeling. First, there is no one-size fits all approach to public
engagement or decision-making. Participatory processes must be appropriately
contextualized for each new situation, or ``place-based''
\cite{elmallah_frontlining_2022}. Second, ideals of justice can never be
adequately captured by an aggregated metric since this would imply a utility
function that maps individual preferences onto a collective preference
\cite{jafino_enabling_2021}. Finally, since such a utility function cannot exist
participatory processes must go beyond survey ranking of modeled solutions and
include a deliberative component to evaluate tradeoffs among alternatives and
their significance.