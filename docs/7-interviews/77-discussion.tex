\section{Discussion}

Some ideas to consider ...

\subsection{Who can benefit from energy modeling?}

\begin{enumerate}
    \item There is a growing body of literature which recognizes that, despite
    the important role for municipalities in the energy transition, municipal
    governments may not be the best venue for energy modeling due to their lack
    of resources, expertise, and agency over their energy systems. However,
    nonprofits and grassroots organizations may benefit from energy modeling to
    aid the development of their energy visions which can then be used to lobby
    for policy changes (e.g., \ac{nest}'s interest in \ac{osier}).
    \item How do I think \acp{ngo} can support the development of energy visions?
\end{enumerate}

Maybe discuss some of the benefits and limitations of \ac{osier} ...


\subsection{How \ac{osier} addresses the ``human dimension''}
Benefits include
\begin{enumerate}
    \item \ac{osier} allows for any number of constraints, objectives, and even
    allows for user-defined objectives, a design philosophy consistent with the
    first principle from Elmallah et al. which called for ``place-based''
    solutions \cite{elmallah_frontlining_2022}. However, place-based solutions
    require authentic engagement and ``quantitative metrics cannot replace on
    the ground work that includes qualitative, community-based approaches in
    devising policy solutions'' \cite{elmallah_frontlining_2022}. While
    \ac{osier} can stand alone as an \ac{esom}, it is likely most effective as
    part of an iterative, participatory, approach as outlined by McGookin et al.
    2024 \cite{mcgookin_advancing_2024}.
    \item It seems like municipal interviewees would like to use the \textit{results} from a model
    such as \ac{osier}, but do not necessarily want to conduct the modeling themselves. With the
    notable exception of \ac{nest}, who seemed enthusiastic about using the model.
    \item Although Illinois uses modeling results, as described in \ac{pcap}, and its regulating agencies
    have experience with public engagement and the reports from these modeling efforts are subject to public
    comments before being finalized, these two activies are distinct. Illinois has an opportunity to 
    employ participatory modeling to develop robust visions for its energy future.
\end{enumerate}

\subsection{Policy implications}