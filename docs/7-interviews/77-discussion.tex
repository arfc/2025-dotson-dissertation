\section{Discussion}
\label{section:interview-discussion}
This section synthesizes the interview results into recommendations that can be
applied in other contexts.

\subsection{Who can benefit from energy modeling?}
Much of the literature that exists on modeling practices at the municipal scale
find that \acp{esom} are seldom used
\cite{ben_amer_too_2020,johannsen_municipal_2023}. The reasons cited for this
are \cite{ben_amer_too_2020,johannsen_municipal_2023,susser_better_2022}
\begin{enumerate}
    \item poorly scoped models that do not capture relevant phenomena,
    \item lack of resources or expertise to conduct modeling, 
    \item and lack of collaboration between modelers and planners.
\end{enumerate} 
Specifying challenges in this way provides a list of issues that can be
addressed. This list suggests that knowledge deficits are the primary obstacle
to the adoption of \acp{esom} at the municipal level. The interviews in this
study corroborate that knowledge defecits contribute to the disuse of
\acp{esom}. Unfortunately, the interviews in this study further suggest that
even if each of these issues were appropriately handled that municipalities
would still not benefit from the use of energy models. Instead, there are
structural barriers to the municipal use of energy models. The typical focus of
\acp{esom} on national and regional energy systems evinces an implicit
understanding of these barriers. Municipalities can still benefit from
mathematical optimization tools as long as a municipality has sufficient
autonomy over its energy choices to impact the decision variables in the model.
For example, a municipality that does not own its electric utility or any
generating resources would not benefit from a capacity expansion model that
optimizes investments in generating resources. Of the municipalities considered
in this study, \ac{uiuc} satisfies this condition the best and is the most
likely to benefit from \acp{esom} in planning processes. Indeed, meeting its
climate goals might necessitate modeling.

Although municipal governments may not be the strongest candidates for energy
modeling, nonprofits and grassroots organizations could benefit from energy
modeling to aid the development of their energy visions which can then be used
to lobby for policy changes. For example, with some support \ac{nest} could use
a model to create an alternative energy vision than the one provided by
\ac{imea}.

\subsection{How \ac{osier} addresses the ``human dimension''} 

The null result from Arrow's Theorem means there is no single objective or set
of objectives that can perfectly capture public preferences for energy systems.
An \ac{esom} must be used as part of a participatory process. However,
\acp{esom} can be more or less easily incorporated in such a process, depending
on the \ac{esom} design. Designing for greater flexibility and transparency are
two ways to make adoption easier. \ac{osier} allows for any number of
constraints, objectives, and even allows for user-defined objectives, a design
philosophy consistent with the first principle from Elmallah et al. which called
for ``place-based'' solutions \cite{elmallah_frontlining_2022}. However,
place-based solutions require authentic engagement and ``quantitative metrics
cannot replace on the ground work that includes qualitative, community-based
approaches in devising policy solutions'' \cite{elmallah_frontlining_2022}.
While \ac{osier} can stand alone as an \ac{esom}, it is likely most effective as
part of an iterative, participatory, approach as outlined by McGookin et al.
2024 \cite{mcgookin_advancing_2024}. This approach breaks down the modeling
process into five distinct parts
\begin{enumerate}
    \item Research design,
    \item Model assumptions,
    \item Modelling results,
    \item Outreach and communication,
    \item and Evaluation.
\end{enumerate}
Each part in this process can be iterated upon based on the outcomes from a
later step \cite{mcgookin_advancing_2024}. \ac{osier} impacts the model
assumptions, and modeling results stages the most since model objectives become
assumptions when using \ac{osier} and understanding the tradeoffs of different
model solutions becomes part of the discussion. Analyzing tradeoffs is important
for reducing the normative uncertainty present in all modeling exercises and
policy discussions. The interviews in this study support \ac{osier}'s ability to
facilitate dialogue and communicate with decision makers and constituents.
Tradeoff analysis was identified as a particularly important feature of the
tool.

\subsection{Policy Recommendations}
Although the state of Illinois uses modeling results and hires consultants to
perform modeling, the responses from interviewees suggest that even more
transparency in the model data and the models themselves are desired. A public
comment period does not constitute genuine involvement of a participatory
process. This is because public comment periods occur after the modeling has
been done and a properly participatory process would include stakeholder
perspectives throughout the modeling process. Creating such a process would
align Illinois more closely with the energy justice planning principles outlined
by Elmallah et al. \cite{elmallah_frontlining_2022}. Specifically ``shifting the
balance of power from existing forms of energy governance.'' Illinois has an
opportunity to employ participatory modeling to develop robust visions for its
energy future. To that end, I suggest that the state of Illinois create such a
process and use open source modeling tools whenever possible.

\subsection{Model and Data Development}
Several of the model features requested by the interviewees would be better
suited by a different modeling paradigm than the one currently provided by
\ac{osier}. For example, questions about technology adoption rates, testing
incentive structures, or time of use rates --- issues related to human behavior
--- are likely better addressed by \acp{abm}. Questions about energy efficiency
might be better served by 1) \acp{bem} but also 2) better data about the cost of
efficiency improvements. \ac{nrel} has a large amount of simulated
building-level energy use data along with the performance of various efficiency
upgrades but no information on cost \cite{wilson_end-use_2022}. \ac{osier} could
readily represent efficiency as a ``technology'' option, but this would require
capital cost data on a per kW basis. Some of this data exists but the data have
large uncertainties \cite{less_cost_2021}.


\subsection{Limitations of this study}
This study has a couple of limitations. First, despite the varied contexts, all
of the study participants have an Illinois perspective. Illinois is a
progressive state with a deregulated electricity market. Municipalities in less
progressive states or states with vertically integrated electricity markets may
have even less choice over their energy supply than in Illinois. Second,
\ac{osier} was already mostly complete at the time of this study. Ideally, this
study could have been a co-creative process that developed a tool with the
features requested by city planners. Interviewees also had limited exposure to
\ac{osier} before offering their insights. Lastly, this study did not actually
employ \ac{osier} in a planning or energy visioning process.