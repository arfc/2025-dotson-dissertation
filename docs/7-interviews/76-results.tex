\section{Results}
\label{section:interview-results}
This section explores the results from the analysis of thirteen interviews. It
is organized according to the five themes derived from thematic analysis. Table
\ref{tab:themes} summarizes the themes and their sub-categories presented in
this section.

\begin{table}[ht!]
    \centering
    \caption{Summary of themes and categories derived from thematic analysis.}
    \label{tab:themes}
    % \resizebox{0.6\columnwidth}{!}{\begin{tabular}{lll}
    \toprule
    Categories & Themes & Section \\
    \midrule
    Community Engagement&\multirow{5}{*}{Existing planning
    processes}&\multirow{5}{*}{\ref{section:planning-processes}}\\
    Decision-making process&&\\
    Planning process&&\\
    Objectives&&\\
    Limits to planning&&\\
    \midrule
    Distributive Justice&\multirow{5}{*}{Planning for
    justice}&\multirow{5}{*}{\ref{section:muni-justice}}\\
    Procedural Justice&&\\
    Recognition Justice&&\\
    Justice definitions&&\\
    Justice programs&&\\
    \midrule
    Energy modeling views&\multirow{4}{*}{Energy model
    usage}&\multirow{4}{*}{\ref{section:muni-model-usage}}\\
    Examples of energy modeling&&\\
    Building energy models&&\\
    Limitations to energy modeling&&\\
    \midrule
    Energy Market&\multirow{3}{*}{Structural
    barriers}&\multirow{3}{*}{\ref{section:structural-barriers}}\\
    \Acl{mca}&&\\
    Barriers to ownership&&\\
    \midrule
    Positive feedback&\multirow{3}{*}{Assessing
    \ac{osier}}&\multirow{3}{*}{\ref{section:assess-osier}}\\
    Feature request or change&&\\
    Usage barriers&&\\
    \bottomrule
\end{tabular}}
    \begin{tabular}{lll}
    \toprule
    Categories & Themes & Section \\
    \midrule
    Community Engagement&\multirow{5}{*}{Existing planning
    processes}&\multirow{5}{*}{\ref{section:planning-processes}}\\
    Decision-making process&&\\
    Planning process&&\\
    Objectives&&\\
    Limits to planning&&\\
    \midrule
    Distributive Justice&\multirow{5}{*}{Planning for
    justice}&\multirow{5}{*}{\ref{section:muni-justice}}\\
    Procedural Justice&&\\
    Recognition Justice&&\\
    Justice definitions&&\\
    Justice programs&&\\
    \midrule
    Energy modeling views&\multirow{4}{*}{Energy model
    usage}&\multirow{4}{*}{\ref{section:muni-model-usage}}\\
    Examples of energy modeling&&\\
    Building energy models&&\\
    Limitations to energy modeling&&\\
    \midrule
    Energy Market&\multirow{3}{*}{Structural
    barriers}&\multirow{3}{*}{\ref{section:structural-barriers}}\\
    \Acl{mca}&&\\
    Barriers to ownership&&\\
    \midrule
    Positive feedback&\multirow{3}{*}{Assessing
    \ac{osier}}&\multirow{3}{*}{\ref{section:assess-osier}}\\
    Feature request or change&&\\
    Usage barriers&&\\
    \bottomrule
\end{tabular}
\end{table}
\FloatBarrier

\subsection{Existing planning processes}
\label{section:planning-processes}
Most of the interviewees could not speak to the process of developing the
\acf{pcap} for the state of Illinois. However, the \ac{pcap} itself outlines
several aspects of the planning process, which coordinated several Illinois
governmental agencies and was ``rooted in stakeholder consultation.''

\begin{quote}
    \blockcquote[10]{kibbey_state_2024}{Development of the plan also included
    consultation on the CPRG with multiple stakeholders, including: community
    organizations, including organizations representing disadvantaged
    communities; non-governmental organizations with expertise in climate
    mitigation; private businesses; higher education institutions; and trade
    associations. To identify market barriers to the equitable GHG emission
    reductions in Illinois, the Illinois Climate Bank, in partnership with other
    state agencies, held a series of stakeholder meetings, small group meetings
    and workshops, and virtual presentations, throughout 2023 and early 2024,
    with more than 150 different entities to get a broad perspective on market
    gaps.}
\end{quote}

Interviewee IP1 confirmed the impact of this inclusive process during the
development of the \ac{ceja} legislation.
\begin{quote}
     I'll tell you that the way that we developed the [retirement] schedule for
     the power plants, the coal- and gas-fired power plants, was specifically
     something that came because we had the frontline communities at the table.
     Because their main concern was not necessarily the greenhouse gas
     reductions. It was, ``I live in a neighborhood that has a coal-fired power
     plant in it, and I've got soot on my car, and I've got all of the negative
     outcomes from living next to a coal-fired power plant or a gas peaker
     plant.'' And their desire was to have the plants gone. And so rather than
     say, ``yeah, [coal plants] can stay open forever, [they] just have to buy
     enough carbon credits to be able to stay open,'' we went a different way.
     And [\dots] that's specifically because people in frontline communities
     were involved (Interview IP1).
\end{quote}

By comparison, \ac{uiuc}'s \acf{icap} report provides much more detail on the
procedures followed to develop the plan
\cite{institute_for_sustainability_energy_and_environment_illinois_2020}.
\ac{icap} recommendations are generated by \acp{swa}, the \ac{icap} Working
Group, and/or the \ac{uiuc} Sustainability Council, depending on the scale of
the intervention. Recommendations are shared with \ac{uiuc} Departments for
feedback and responses. The \ac{icap} indicates some pathways for feedback on
recommendations from neighboring cities of Champaign, Urbana, and Savoy.
However, the extent and depth of these solicitations are unclear and may be
limited to participation on \acp{swa}, whose members commonly live in these
communities. \ac{uiuc} makes an effort to involve its neighbors through projects
and initiatives proposed in the \ac{icap}, but opportunities are limited due to
the separation of each governmental entity. 

Although Champaign and Urbana each produced their own climate action plans,
neither city has published a new one within the last ten years. Interviewee UP1
indicated that any given planning process is dictated by the city council and
how much budget it allocates for a project.
\begin{quote}
    You know, we will defend the process when it's all said and done, regardless
    of everything else. It really is a practical matter. What does leadership
    want to spend? Okay. \$10,000 and nine months. We're going to do the
    greatest public engagement we can do at \$10,000 in nine months. [\dots]
    Certain public engagement methods are going to reach certain kinds of people
    and others, other kinds of people. So we want to mix methods to try and get
    as much of the population and a cross-section of demographics (Interviewee
    UP1).
\end{quote}
However, these public engagement methods primarily relate to delivering
information to constituents rather than inviting significant public
participation in a planning process. In the words of interviewee CP1, ``the
public's voice can be heard by who they put on council, [and] who is in the
mayor's office.''

Similar to Champaign and Urbana, Naperville does not have an active a climate
action plan. Nor does Naperville have a formal process for developing a plan.
However, \ac{nest}, Naperville's sustainability advisory organization, inserts
itself in public meetings by ``making at least one public comment at every city
council meeting. We've come to refer to it as `three minutes with \ac{nest}'"
(Interviewee NA1). Unlike Champaign and Urbana, Naperville controls its own
municipal electric utility that distributes electricity to Naperville residents.
Despite having its own utility, Naperville ceded control over their energy
supply decisions to \ac{imea}. Interviewees from \ac{nest} and Illinois \ac{cub}
find \ac{imea}'s planning processes to be inadequate and lacking transparency.
According to two interviewees

\begin{quote}
    [W]e found out that in 2002 the IMEA did an IRP\footnote{\Acf{irp}}. And we
    had the document. It was about 220 pages long. And they had maintained the
    position that, ``yeah, we [still] do planning. We [still] do resource
    planning. It's just as good as an IRP.'' The plan that they pulled out to
    justify that was about 20 pages and had some charts and graphs. But then
    when we pulled out the 220-page document, which was their work in creating
    an IRP, it was pretty obvious to every single person in the room that their
    planning is not at all as comprehensive as an IRP (Interviewee NA1).
\end{quote}

\begin{quote}
    [The people of Naperville] have gone to their city council and said, listen,
    we don't have to extend our contract [with \ac{imea}]. They're not doing the
    things that we want them to do. [\dots] And so the people of Naperville,
    when they go to their city council and they talk about this sort of thing,
    the city council defers to the electric director. And their electric
    director is a board member of the IMEA. So it's hard to parse out who is he
    representing and when. He's an executive member of this agency, but he also
    works for the city. So he has a fiscal responsibility to the agency and a
    responsibility to the city. And it really confuses people because like we
    want to be heard on this (Interviewee IA2).
\end{quote}

The next section discusses how the municipalities define equity or justice and
deploy strategies that advance equity and justice in their communities.

\subsection{Planning for justice}
\label{section:muni-justice}

This section discusses the ways municipalities in Illinois, and the state
government, conceptualize justice or equity and incorporate it into their
decision-making processes.

% justice definitions
First, municipalities use the term ``equity'' rather than ``justice'' which is
consistent with the findings of Diezmart\'inez et al
\cite{diezmartinez_us_2022}.  Second, most municipalities do not have formal
definitions of equity but generally acknowledge disproportionality and
marginalization. ``The equity angle of it is we don't want historically
disenfranchised populations or really any segment of the population to be left
behind we need everyone involved in the clean energy transition (Interviewee
UP1).'' However, the text of each municipality's climate action plan focuses
primarily on equitable distribution of benefits, which also corroborates the
findings from Diezmart\'inez et al \cite{diezmartinez_us_2022}. Three
municipalities (Champaign, Urbana, and Naperville) have an office or department
dedicated to equity issues. But these equity-focused departments are distinct
from planning departments which suggests there could be more work integrating
justice initiatives into planning processes.

% procedural justice
Although interviewees identified disproportionality as an issue, mechanisms for
active participation from, and engagement with, marginalized communities were
understated. As discussed in the previous section, residents' voices may be
heard by voting for council members and the mayor. While the level of community
engagement depends on the willingness for city councils to spend money on
process. Since Naperville's energy supply decisions are made by \ac{imea},
\ac{imea} should be responsible for public engagement, however their processes
lack transparency and public involvement seems to be discouraged.
\begin{quote}
    [T]he planning is really done behind closed doors by some of the
    experts at the IMEA. And when they think they have something that they want
    to propose to the board of directors, they start first with the executive
    board. Those folks have already been kind of pre-sold about the next
    proposal, and then that gets approved at the board of directors meetings.
    [...]Any communications about the planning is done as a one-way street. 
    [...]They're kind of a captured vote. It's
    rare. I think in the eight years while one of our guys has participated
    remotely or in person on attending every single IMEA meeting, I think he can
    remember only one time where a single member voted no. The votes are always
    either yes entirely or yes with some abstentions. So the planning is done
    behind closed doors. The results of the planning is offered as a slide
    presentation to members and then they're expected to vote after seeing the
    slides. That vote commits the entire membership to whatever they decide that
    they're going to do. And then the individual members go back to their
    communities. And within a week or so, they let us know what they've done.
   [\dots]But we would say the transparency, the
    planning that they do is non-transparent. The IMEA's position is it's 100\%
    transparent because the board members get to see the charts and they get to
    vote. But there's a big difference between having members, rate payers being
    involved in the process or not. There's a big difference between being able
    to look at all the numbers or just the numbers and charts, actually just the
    charts that the IMEA is willing to share with us. And those board meetings,
    to the best of our knowledge, are not recorded. So unless you're able to
    dedicate the time to sit on the Zoom and listen to the board meeting, you
    have no way of availing yourself to any planning information. The IMEA has
    refused to make these board meetings. If they recorded them, we're not aware
    that the recordings exist (Interviewee NA1).
\end{quote}

% state level
At the state level, Illinois has a clearer focus on equity and justice issues
after the passage of \ac{ceja}. In the words of interviewee IA2:
\begin{quote}
    We have to incorporate into law, not just affordability, but environmental
    justice, utility justice, utility democracy, all of these other concepts.
    And you know, did we get a law that properly focused all those things?
    Definitely not. But it did introduce these concepts into Illinois in, I
    think, a pretty significant way.
\end{quote}
In particular, Illinois clearly defined ``equity-eligible persons'' based on a
person's residence in specific census tracts. Two of the interviewees (IP1, IP3)
commented on the importance of including frontline communities when developing
climate policy. The \ac{pcap} also describes continued efforts to engage these
communities. However, the energy modeling prescribed in the \ac{pcap} does not
include avenues for active participation from constituents outside of the normal
public comment period. Modeling efforts are typically outsourced to consulting
firms and these projects do not involve members of the public.


% \textcolor{red}{Note for later: One of the key patterns I found in the
% interviews was that, although many interviewees were empathetic to
% justice-related efforts and initiatives, those initiatives were delegated to a
% dedicated person or department within governing institutions. This suggests
% there is more work to be done integrating energy justice and energy
% modeling/policy-making}

\subsection{Energy model usage}
\label{section:muni-model-usage}

The views of modeling, in general, were mixed among interviewees. Many
recognized the utility of models but some interviewees found them to be
inaccurate or obfuscating. Urbana and Naperville assess their \ac{ghg} emissions
using the ClearPath emissions and \ac{ghg} inventory model from the \acf{iclei}.
Regarding ClearPath:
\begin{quote}
    [I]t's trying to take some data you put into it and tell you, actually doing
    [X] creates this level of an outcome, and doing [Y] creates this level of
    outcome. So that you can have a rational math-based planning. But it's a
    model and probably bringing in a model of multiple types of data. For
    instance you can put transportation data in there [\dots] like here's
    [\dots] our community's average vehicle miles traveled [\dots] that number
    is modeled and then you're going to stick that into another model and so
    you're multiplying the scale of abstraction. If you want to do that to make
    yourself feel better, that you're making a rational decision, you may, and
    cities do. But the reality is, it's all pretty imprecise (Interviewee UP1). 
\end{quote}

None of the municipalities considered in this study, Naperville, Champaign,
Urbana, or \ac{uiuc}, use \acp{esom} to guide energy planning and investment
decisions. Interviewees from Naperville believe that \ac{imea} conducts energy
modeling to inform their purchasing and investment decisions but the details of
this modeling are not shared with constituents.

Practitioners at the municipal level also understood modeling differently from
the definition of \acp{esom} or, at least, do not distinguish \acp{esom} from
other types of models. Two interviewees, UI1 and CP1, indicated that ``energy
models'' were used but described \acp{bem}, which are distinct from \acp{esom}
because they focus on the flow of \textit{heat} energy into and and throughout a
building. \acp{bem} are useful for identifying efficiency measures but are not
typically mathematical optimization tools and therefore better classified as
``simulation tools.''

At the state level, the \ac{ipa} uses modeling \textit{results} to inform policy
recommendations but the modeling itself is generally outsourced to an external
consultant. However, \ac{ipa} studies typically employ \acp{esom} such as Aurora
or GE's MARS model\footnote{Both Aurora and GE MARS were excluded from the
review in Section \ref{section:esoms} because they are proprietary models and
prohibitively expensive for non-enterprise users.}
\cite{bringolf_evaluation_2024, carlson_illinois_2024}. This is still true for
upcoming analysis from Illinois EPA as described in the \ac{pcap}
\cite{kibbey_state_2024}. Interviewee IP3 made one notable exception 
\begin{quote}
    In developing our long-term renewable resource procurement plan, we do have
    a model for \ac{rec} prices, for example, basically based off of NREL's
    CREST\footnote{The \ac{crest} is a spreadsheet tool that ``contains
    economic, cash-flow models designed to assess project economics, design
    cost-based incentives, and evaluate the impact of state and federal support
    structures on renewable energy'' \cite{gifford_renewable_2011}. } model to
    determine what level of incentives we think are necessary in order to spur
    the development of projects at certain sizes, given certain inputs.
\end{quote}

Along with acknowledging the overall limits of models, interviewees suggested
that municipalities do not use \acp{esom} due to a lack of staff capacity and
expertise, a lack of financial resources, and difficulty acquiring reliable
input data. In addition to these barriers, the interviews revealed some
structural barriers that not only make using \ac{esom} challenging, but severely
restrict municipalities from developing a vision for their own energy futures at
all. The next section discusses these barriers.
% \block

\subsection{Structural barriers}
\label{section:structural-barriers}

This section describes ``structural barriers'' identified during the interviews.
These barriers relate to the use of energy models --- going beyond the
observation in Section \ref{section:muni-model-usage} that municipalities do not
seem to use models --- and surfaces the underlying conditions that prevent the
use of \acp{esom}. These barriers also relate to the structures that produce
various injustices and create friction for communities that want to exercise
agency over their energy choices. I began this study by assuming that
municipalities needed better tools to develop energy visions for their
communities, and that municipal governments would be the ones employing
\acp{esom}. Now, it appears these assumptions do not fully hold.

As shown in Figure \ref{fig:illinois-flow-chart}, consumers and municipalities
cannot directly participate in electricity markets. Instead, there are myriad
entities that broker electricity purchases and thus these entities control most
energy supply decisions. For municipalities that do not have their own electric
utility, which is vast majority of municipalities in Illinois, these entities
include various \ac{ares}, \acp{iou}, the \ac{ipa} --- which implements policy
directives from the Illinois legislature --- and the \ac{icc} which regulates
Illinois' competitive electricity markets. Municipalities that do have their own
electric utility frequently give control of their energy supply decisions to
\ac{imea} because they do not have the staff capacity to understand and
participate in energy markets, do not have the resources to own their own
generation, or do not want to own their own generation because it is ``expensive
and complicated (Interviewee IA2).''

Let us consider the first scenario where a municipality is served by one of the
two major \acp{iou} in Illinois: Ameren and ComEd. In the present case study,
the cities of Champaign and Urbana fall under this category. There are three
structural barriers to making energy supply choices: Access to data about
residents' energy usage in a municipality, inadequate grid infrastructure to
support \ac{btm} resources, and \acf{mca}.

The first barrier is quite simple. Since municipalities do not control the
delivery of electricity to their residents, they do not have access to their
energy usage data. This barrier challenges both energy modeling and distributive
justice. One, energy models require electricity demand data to provide insights. Two, details
about residents' electricity bills are necessary to fully characterize energy burden in a
community.

The second barrier is related to aging infrastructure and the ways \acp{iou} can
generate profit. Interviewee IP2 put it simply
\begin{quote}
     [T]here are other ways to help make it easier for people to get into
     \acs{der} ownership, one of the things that I think is impeding people in
     some areas, particularly in low income areas, is that the grid is just not
     up to par. It hasn't been properly maintained in some places. And so you've
     got an antiquated grid, which because of its age and poor maintenance, it's
     just not up to the task of accommodating a lot of behind the meter stuff
     (Interviewee IP2).
\end{quote}
Since \acfp{iou} are regulated by the \ac{icc} they cannot profit from the
delivery of electricity, depsite having monopoly control over electricity
distribution. However, \acp{iou} are allowed to recover their expenses.
Therefore, \acp{iou} can generate profit when they build new infrastructure by
passing along those costs to ratepayers. This makes it difficult for
municipalities that would like to encourage residents to install \ac{btm} solar
to do so. ``[\acp{iou} are] reluctant to want to enable the use of [\ac{btm}]
stuff like that because it means less investment they have to make (Interviewee
IP2)'' and therefore less profit for the utility.

% challenges with MCA
The third barrier relates to energy supply choices in a deregulated electricity
market. Since \acp{iou} do not profit from the supply or delivery of
electricity, these utilities do not have any interest or control in energy
supply decisions. Indeed, since Illinois has a deregulated electricity market,
consumers are theoretically able to receive electricity from any supplier they
prefer if they are unsatisfied with the default supply procured by the \ac{ipa}.
Despite this freedom to choose, consumers that contracted with \ac{ares} during
the 2024-2025 Illinois congressional session paid on average 2.5 cents more per
kilowatt-hour in the ComED region and 1.6 cents more in the Ameren region
compared with consumers receiving the default supply
\cite{office_of_retail_market_development_2025_2025}. Figure \ref{fig:ares-ptc}
shows the difference in cost between \ac{ares} and the default supply \ac{ptc}
for each utility region in Illinois over the last six fiscal years. Values greater than
zero indicate fiscal years when the cost of electricity from \ac{ares} exceeded the
cost of the default supply.

\begin{figure}[ht!]
    \centering
    \resizebox{0.9\columnwidth}{!}{%% Creator: Matplotlib, PGF backend
%%
%% To include the figure in your LaTeX document, write
%%   \input{<filename>.pgf}
%%
%% Make sure the required packages are loaded in your preamble
%%   \usepackage{pgf}
%%
%% Also ensure that all the required font packages are loaded; for instance,
%% the lmodern package is sometimes necessary when using math font.
%%   \usepackage{lmodern}
%%
%% Figures using additional raster images can only be included by \input if
%% they are in the same directory as the main LaTeX file. For loading figures
%% from other directories you can use the `import` package
%%   \usepackage{import}
%%
%% and then include the figures with
%%   \import{<path to file>}{<filename>.pgf}
%%
%% Matplotlib used the following preamble
%%   \def\mathdefault#1{#1}
%%   \everymath=\expandafter{\the\everymath\displaystyle}
%%   \IfFileExists{scrextend.sty}{
%%     \usepackage[fontsize=10.000000pt]{scrextend}
%%   }{
%%     \renewcommand{\normalsize}{\fontsize{10.000000}{12.000000}\selectfont}
%%     \normalsize
%%   }
%%   
%%   \makeatletter\@ifpackageloaded{underscore}{}{\usepackage[strings]{underscore}}\makeatother
%%
\begingroup%
\makeatletter%
\begin{pgfpicture}%
\pgfpathrectangle{\pgfpointorigin}{\pgfqpoint{7.231578in}{5.178333in}}%
\pgfusepath{use as bounding box, clip}%
\begin{pgfscope}%
\pgfsetbuttcap%
\pgfsetmiterjoin%
\definecolor{currentfill}{rgb}{1.000000,1.000000,1.000000}%
\pgfsetfillcolor{currentfill}%
\pgfsetlinewidth{0.000000pt}%
\definecolor{currentstroke}{rgb}{0.000000,0.000000,0.000000}%
\pgfsetstrokecolor{currentstroke}%
\pgfsetdash{}{0pt}%
\pgfpathmoveto{\pgfqpoint{0.000000in}{0.000000in}}%
\pgfpathlineto{\pgfqpoint{7.231578in}{0.000000in}}%
\pgfpathlineto{\pgfqpoint{7.231578in}{5.178333in}}%
\pgfpathlineto{\pgfqpoint{0.000000in}{5.178333in}}%
\pgfpathlineto{\pgfqpoint{0.000000in}{0.000000in}}%
\pgfpathclose%
\pgfusepath{fill}%
\end{pgfscope}%
\begin{pgfscope}%
\pgfsetbuttcap%
\pgfsetmiterjoin%
\definecolor{currentfill}{rgb}{1.000000,1.000000,1.000000}%
\pgfsetfillcolor{currentfill}%
\pgfsetlinewidth{0.000000pt}%
\definecolor{currentstroke}{rgb}{0.000000,0.000000,0.000000}%
\pgfsetstrokecolor{currentstroke}%
\pgfsetstrokeopacity{0.000000}%
\pgfsetdash{}{0pt}%
\pgfpathmoveto{\pgfqpoint{0.789096in}{0.375000in}}%
\pgfpathlineto{\pgfqpoint{6.989096in}{0.375000in}}%
\pgfpathlineto{\pgfqpoint{6.989096in}{4.995000in}}%
\pgfpathlineto{\pgfqpoint{0.789096in}{4.995000in}}%
\pgfpathlineto{\pgfqpoint{0.789096in}{0.375000in}}%
\pgfpathclose%
\pgfusepath{fill}%
\end{pgfscope}%
\begin{pgfscope}%
\pgfpathrectangle{\pgfqpoint{0.789096in}{0.375000in}}{\pgfqpoint{6.200000in}{4.620000in}}%
\pgfusepath{clip}%
\pgfsetrectcap%
\pgfsetroundjoin%
\pgfsetlinewidth{0.803000pt}%
\definecolor{currentstroke}{rgb}{0.690196,0.690196,0.690196}%
\pgfsetstrokecolor{currentstroke}%
\pgfsetdash{}{0pt}%
\pgfpathmoveto{\pgfqpoint{1.070914in}{0.375000in}}%
\pgfpathlineto{\pgfqpoint{1.070914in}{4.995000in}}%
\pgfusepath{stroke}%
\end{pgfscope}%
\begin{pgfscope}%
\pgfsetbuttcap%
\pgfsetroundjoin%
\definecolor{currentfill}{rgb}{0.000000,0.000000,0.000000}%
\pgfsetfillcolor{currentfill}%
\pgfsetlinewidth{0.803000pt}%
\definecolor{currentstroke}{rgb}{0.000000,0.000000,0.000000}%
\pgfsetstrokecolor{currentstroke}%
\pgfsetdash{}{0pt}%
\pgfsys@defobject{currentmarker}{\pgfqpoint{0.000000in}{-0.048611in}}{\pgfqpoint{0.000000in}{0.000000in}}{%
\pgfpathmoveto{\pgfqpoint{0.000000in}{0.000000in}}%
\pgfpathlineto{\pgfqpoint{0.000000in}{-0.048611in}}%
\pgfusepath{stroke,fill}%
}%
\begin{pgfscope}%
\pgfsys@transformshift{1.070914in}{0.375000in}%
\pgfsys@useobject{currentmarker}{}%
\end{pgfscope}%
\end{pgfscope}%
\begin{pgfscope}%
\definecolor{textcolor}{rgb}{0.000000,0.000000,0.000000}%
\pgfsetstrokecolor{textcolor}%
\pgfsetfillcolor{textcolor}%
\pgftext[x=1.070914in,y=0.277777in,,top]{\color{textcolor}{\rmfamily\fontsize{14.000000}{16.800000}\selectfont\catcode`\^=\active\def^{\ifmmode\sp\else\^{}\fi}\catcode`\%=\active\def%{\%}2019-2020}}%
\end{pgfscope}%
\begin{pgfscope}%
\pgfpathrectangle{\pgfqpoint{0.789096in}{0.375000in}}{\pgfqpoint{6.200000in}{4.620000in}}%
\pgfusepath{clip}%
\pgfsetrectcap%
\pgfsetroundjoin%
\pgfsetlinewidth{0.803000pt}%
\definecolor{currentstroke}{rgb}{0.690196,0.690196,0.690196}%
\pgfsetstrokecolor{currentstroke}%
\pgfsetdash{}{0pt}%
\pgfpathmoveto{\pgfqpoint{2.198187in}{0.375000in}}%
\pgfpathlineto{\pgfqpoint{2.198187in}{4.995000in}}%
\pgfusepath{stroke}%
\end{pgfscope}%
\begin{pgfscope}%
\pgfsetbuttcap%
\pgfsetroundjoin%
\definecolor{currentfill}{rgb}{0.000000,0.000000,0.000000}%
\pgfsetfillcolor{currentfill}%
\pgfsetlinewidth{0.803000pt}%
\definecolor{currentstroke}{rgb}{0.000000,0.000000,0.000000}%
\pgfsetstrokecolor{currentstroke}%
\pgfsetdash{}{0pt}%
\pgfsys@defobject{currentmarker}{\pgfqpoint{0.000000in}{-0.048611in}}{\pgfqpoint{0.000000in}{0.000000in}}{%
\pgfpathmoveto{\pgfqpoint{0.000000in}{0.000000in}}%
\pgfpathlineto{\pgfqpoint{0.000000in}{-0.048611in}}%
\pgfusepath{stroke,fill}%
}%
\begin{pgfscope}%
\pgfsys@transformshift{2.198187in}{0.375000in}%
\pgfsys@useobject{currentmarker}{}%
\end{pgfscope}%
\end{pgfscope}%
\begin{pgfscope}%
\definecolor{textcolor}{rgb}{0.000000,0.000000,0.000000}%
\pgfsetstrokecolor{textcolor}%
\pgfsetfillcolor{textcolor}%
\pgftext[x=2.198187in,y=0.277777in,,top]{\color{textcolor}{\rmfamily\fontsize{14.000000}{16.800000}\selectfont\catcode`\^=\active\def^{\ifmmode\sp\else\^{}\fi}\catcode`\%=\active\def%{\%}2020-2021}}%
\end{pgfscope}%
\begin{pgfscope}%
\pgfpathrectangle{\pgfqpoint{0.789096in}{0.375000in}}{\pgfqpoint{6.200000in}{4.620000in}}%
\pgfusepath{clip}%
\pgfsetrectcap%
\pgfsetroundjoin%
\pgfsetlinewidth{0.803000pt}%
\definecolor{currentstroke}{rgb}{0.690196,0.690196,0.690196}%
\pgfsetstrokecolor{currentstroke}%
\pgfsetdash{}{0pt}%
\pgfpathmoveto{\pgfqpoint{3.325460in}{0.375000in}}%
\pgfpathlineto{\pgfqpoint{3.325460in}{4.995000in}}%
\pgfusepath{stroke}%
\end{pgfscope}%
\begin{pgfscope}%
\pgfsetbuttcap%
\pgfsetroundjoin%
\definecolor{currentfill}{rgb}{0.000000,0.000000,0.000000}%
\pgfsetfillcolor{currentfill}%
\pgfsetlinewidth{0.803000pt}%
\definecolor{currentstroke}{rgb}{0.000000,0.000000,0.000000}%
\pgfsetstrokecolor{currentstroke}%
\pgfsetdash{}{0pt}%
\pgfsys@defobject{currentmarker}{\pgfqpoint{0.000000in}{-0.048611in}}{\pgfqpoint{0.000000in}{0.000000in}}{%
\pgfpathmoveto{\pgfqpoint{0.000000in}{0.000000in}}%
\pgfpathlineto{\pgfqpoint{0.000000in}{-0.048611in}}%
\pgfusepath{stroke,fill}%
}%
\begin{pgfscope}%
\pgfsys@transformshift{3.325460in}{0.375000in}%
\pgfsys@useobject{currentmarker}{}%
\end{pgfscope}%
\end{pgfscope}%
\begin{pgfscope}%
\definecolor{textcolor}{rgb}{0.000000,0.000000,0.000000}%
\pgfsetstrokecolor{textcolor}%
\pgfsetfillcolor{textcolor}%
\pgftext[x=3.325460in,y=0.277777in,,top]{\color{textcolor}{\rmfamily\fontsize{14.000000}{16.800000}\selectfont\catcode`\^=\active\def^{\ifmmode\sp\else\^{}\fi}\catcode`\%=\active\def%{\%}2021-2022}}%
\end{pgfscope}%
\begin{pgfscope}%
\pgfpathrectangle{\pgfqpoint{0.789096in}{0.375000in}}{\pgfqpoint{6.200000in}{4.620000in}}%
\pgfusepath{clip}%
\pgfsetrectcap%
\pgfsetroundjoin%
\pgfsetlinewidth{0.803000pt}%
\definecolor{currentstroke}{rgb}{0.690196,0.690196,0.690196}%
\pgfsetstrokecolor{currentstroke}%
\pgfsetdash{}{0pt}%
\pgfpathmoveto{\pgfqpoint{4.452732in}{0.375000in}}%
\pgfpathlineto{\pgfqpoint{4.452732in}{4.995000in}}%
\pgfusepath{stroke}%
\end{pgfscope}%
\begin{pgfscope}%
\pgfsetbuttcap%
\pgfsetroundjoin%
\definecolor{currentfill}{rgb}{0.000000,0.000000,0.000000}%
\pgfsetfillcolor{currentfill}%
\pgfsetlinewidth{0.803000pt}%
\definecolor{currentstroke}{rgb}{0.000000,0.000000,0.000000}%
\pgfsetstrokecolor{currentstroke}%
\pgfsetdash{}{0pt}%
\pgfsys@defobject{currentmarker}{\pgfqpoint{0.000000in}{-0.048611in}}{\pgfqpoint{0.000000in}{0.000000in}}{%
\pgfpathmoveto{\pgfqpoint{0.000000in}{0.000000in}}%
\pgfpathlineto{\pgfqpoint{0.000000in}{-0.048611in}}%
\pgfusepath{stroke,fill}%
}%
\begin{pgfscope}%
\pgfsys@transformshift{4.452732in}{0.375000in}%
\pgfsys@useobject{currentmarker}{}%
\end{pgfscope}%
\end{pgfscope}%
\begin{pgfscope}%
\definecolor{textcolor}{rgb}{0.000000,0.000000,0.000000}%
\pgfsetstrokecolor{textcolor}%
\pgfsetfillcolor{textcolor}%
\pgftext[x=4.452732in,y=0.277777in,,top]{\color{textcolor}{\rmfamily\fontsize{14.000000}{16.800000}\selectfont\catcode`\^=\active\def^{\ifmmode\sp\else\^{}\fi}\catcode`\%=\active\def%{\%}2022-2023}}%
\end{pgfscope}%
\begin{pgfscope}%
\pgfpathrectangle{\pgfqpoint{0.789096in}{0.375000in}}{\pgfqpoint{6.200000in}{4.620000in}}%
\pgfusepath{clip}%
\pgfsetrectcap%
\pgfsetroundjoin%
\pgfsetlinewidth{0.803000pt}%
\definecolor{currentstroke}{rgb}{0.690196,0.690196,0.690196}%
\pgfsetstrokecolor{currentstroke}%
\pgfsetdash{}{0pt}%
\pgfpathmoveto{\pgfqpoint{5.580005in}{0.375000in}}%
\pgfpathlineto{\pgfqpoint{5.580005in}{4.995000in}}%
\pgfusepath{stroke}%
\end{pgfscope}%
\begin{pgfscope}%
\pgfsetbuttcap%
\pgfsetroundjoin%
\definecolor{currentfill}{rgb}{0.000000,0.000000,0.000000}%
\pgfsetfillcolor{currentfill}%
\pgfsetlinewidth{0.803000pt}%
\definecolor{currentstroke}{rgb}{0.000000,0.000000,0.000000}%
\pgfsetstrokecolor{currentstroke}%
\pgfsetdash{}{0pt}%
\pgfsys@defobject{currentmarker}{\pgfqpoint{0.000000in}{-0.048611in}}{\pgfqpoint{0.000000in}{0.000000in}}{%
\pgfpathmoveto{\pgfqpoint{0.000000in}{0.000000in}}%
\pgfpathlineto{\pgfqpoint{0.000000in}{-0.048611in}}%
\pgfusepath{stroke,fill}%
}%
\begin{pgfscope}%
\pgfsys@transformshift{5.580005in}{0.375000in}%
\pgfsys@useobject{currentmarker}{}%
\end{pgfscope}%
\end{pgfscope}%
\begin{pgfscope}%
\definecolor{textcolor}{rgb}{0.000000,0.000000,0.000000}%
\pgfsetstrokecolor{textcolor}%
\pgfsetfillcolor{textcolor}%
\pgftext[x=5.580005in,y=0.277777in,,top]{\color{textcolor}{\rmfamily\fontsize{14.000000}{16.800000}\selectfont\catcode`\^=\active\def^{\ifmmode\sp\else\^{}\fi}\catcode`\%=\active\def%{\%}2023-2024}}%
\end{pgfscope}%
\begin{pgfscope}%
\pgfpathrectangle{\pgfqpoint{0.789096in}{0.375000in}}{\pgfqpoint{6.200000in}{4.620000in}}%
\pgfusepath{clip}%
\pgfsetrectcap%
\pgfsetroundjoin%
\pgfsetlinewidth{0.803000pt}%
\definecolor{currentstroke}{rgb}{0.690196,0.690196,0.690196}%
\pgfsetstrokecolor{currentstroke}%
\pgfsetdash{}{0pt}%
\pgfpathmoveto{\pgfqpoint{6.707278in}{0.375000in}}%
\pgfpathlineto{\pgfqpoint{6.707278in}{4.995000in}}%
\pgfusepath{stroke}%
\end{pgfscope}%
\begin{pgfscope}%
\pgfsetbuttcap%
\pgfsetroundjoin%
\definecolor{currentfill}{rgb}{0.000000,0.000000,0.000000}%
\pgfsetfillcolor{currentfill}%
\pgfsetlinewidth{0.803000pt}%
\definecolor{currentstroke}{rgb}{0.000000,0.000000,0.000000}%
\pgfsetstrokecolor{currentstroke}%
\pgfsetdash{}{0pt}%
\pgfsys@defobject{currentmarker}{\pgfqpoint{0.000000in}{-0.048611in}}{\pgfqpoint{0.000000in}{0.000000in}}{%
\pgfpathmoveto{\pgfqpoint{0.000000in}{0.000000in}}%
\pgfpathlineto{\pgfqpoint{0.000000in}{-0.048611in}}%
\pgfusepath{stroke,fill}%
}%
\begin{pgfscope}%
\pgfsys@transformshift{6.707278in}{0.375000in}%
\pgfsys@useobject{currentmarker}{}%
\end{pgfscope}%
\end{pgfscope}%
\begin{pgfscope}%
\definecolor{textcolor}{rgb}{0.000000,0.000000,0.000000}%
\pgfsetstrokecolor{textcolor}%
\pgfsetfillcolor{textcolor}%
\pgftext[x=6.707278in,y=0.277777in,,top]{\color{textcolor}{\rmfamily\fontsize{14.000000}{16.800000}\selectfont\catcode`\^=\active\def^{\ifmmode\sp\else\^{}\fi}\catcode`\%=\active\def%{\%}2024-2025}}%
\end{pgfscope}%
\begin{pgfscope}%
\pgfpathrectangle{\pgfqpoint{0.789096in}{0.375000in}}{\pgfqpoint{6.200000in}{4.620000in}}%
\pgfusepath{clip}%
\pgfsetrectcap%
\pgfsetroundjoin%
\pgfsetlinewidth{0.803000pt}%
\definecolor{currentstroke}{rgb}{0.690196,0.690196,0.690196}%
\pgfsetstrokecolor{currentstroke}%
\pgfsetdash{}{0pt}%
\pgfpathmoveto{\pgfqpoint{0.789096in}{0.375000in}}%
\pgfpathlineto{\pgfqpoint{6.989096in}{0.375000in}}%
\pgfusepath{stroke}%
\end{pgfscope}%
\begin{pgfscope}%
\pgfsetbuttcap%
\pgfsetroundjoin%
\definecolor{currentfill}{rgb}{0.000000,0.000000,0.000000}%
\pgfsetfillcolor{currentfill}%
\pgfsetlinewidth{0.803000pt}%
\definecolor{currentstroke}{rgb}{0.000000,0.000000,0.000000}%
\pgfsetstrokecolor{currentstroke}%
\pgfsetdash{}{0pt}%
\pgfsys@defobject{currentmarker}{\pgfqpoint{-0.048611in}{0.000000in}}{\pgfqpoint{-0.000000in}{0.000000in}}{%
\pgfpathmoveto{\pgfqpoint{-0.000000in}{0.000000in}}%
\pgfpathlineto{\pgfqpoint{-0.048611in}{0.000000in}}%
\pgfusepath{stroke,fill}%
}%
\begin{pgfscope}%
\pgfsys@transformshift{0.789096in}{0.375000in}%
\pgfsys@useobject{currentmarker}{}%
\end{pgfscope}%
\end{pgfscope}%
\begin{pgfscope}%
\definecolor{textcolor}{rgb}{0.000000,0.000000,0.000000}%
\pgfsetstrokecolor{textcolor}%
\pgfsetfillcolor{textcolor}%
\pgftext[x=0.395138in, y=0.291666in, left, base]{\color{textcolor}{\rmfamily\fontsize{16.000000}{19.200000}\selectfont\catcode`\^=\active\def^{\ifmmode\sp\else\^{}\fi}\catcode`\%=\active\def%{\%}$\mathdefault{\ensuremath{-}3}$}}%
\end{pgfscope}%
\begin{pgfscope}%
\pgfpathrectangle{\pgfqpoint{0.789096in}{0.375000in}}{\pgfqpoint{6.200000in}{4.620000in}}%
\pgfusepath{clip}%
\pgfsetrectcap%
\pgfsetroundjoin%
\pgfsetlinewidth{0.803000pt}%
\definecolor{currentstroke}{rgb}{0.690196,0.690196,0.690196}%
\pgfsetstrokecolor{currentstroke}%
\pgfsetdash{}{0pt}%
\pgfpathmoveto{\pgfqpoint{0.789096in}{1.145000in}}%
\pgfpathlineto{\pgfqpoint{6.989096in}{1.145000in}}%
\pgfusepath{stroke}%
\end{pgfscope}%
\begin{pgfscope}%
\pgfsetbuttcap%
\pgfsetroundjoin%
\definecolor{currentfill}{rgb}{0.000000,0.000000,0.000000}%
\pgfsetfillcolor{currentfill}%
\pgfsetlinewidth{0.803000pt}%
\definecolor{currentstroke}{rgb}{0.000000,0.000000,0.000000}%
\pgfsetstrokecolor{currentstroke}%
\pgfsetdash{}{0pt}%
\pgfsys@defobject{currentmarker}{\pgfqpoint{-0.048611in}{0.000000in}}{\pgfqpoint{-0.000000in}{0.000000in}}{%
\pgfpathmoveto{\pgfqpoint{-0.000000in}{0.000000in}}%
\pgfpathlineto{\pgfqpoint{-0.048611in}{0.000000in}}%
\pgfusepath{stroke,fill}%
}%
\begin{pgfscope}%
\pgfsys@transformshift{0.789096in}{1.145000in}%
\pgfsys@useobject{currentmarker}{}%
\end{pgfscope}%
\end{pgfscope}%
\begin{pgfscope}%
\definecolor{textcolor}{rgb}{0.000000,0.000000,0.000000}%
\pgfsetstrokecolor{textcolor}%
\pgfsetfillcolor{textcolor}%
\pgftext[x=0.395138in, y=1.061666in, left, base]{\color{textcolor}{\rmfamily\fontsize{16.000000}{19.200000}\selectfont\catcode`\^=\active\def^{\ifmmode\sp\else\^{}\fi}\catcode`\%=\active\def%{\%}$\mathdefault{\ensuremath{-}2}$}}%
\end{pgfscope}%
\begin{pgfscope}%
\pgfpathrectangle{\pgfqpoint{0.789096in}{0.375000in}}{\pgfqpoint{6.200000in}{4.620000in}}%
\pgfusepath{clip}%
\pgfsetrectcap%
\pgfsetroundjoin%
\pgfsetlinewidth{0.803000pt}%
\definecolor{currentstroke}{rgb}{0.690196,0.690196,0.690196}%
\pgfsetstrokecolor{currentstroke}%
\pgfsetdash{}{0pt}%
\pgfpathmoveto{\pgfqpoint{0.789096in}{1.915000in}}%
\pgfpathlineto{\pgfqpoint{6.989096in}{1.915000in}}%
\pgfusepath{stroke}%
\end{pgfscope}%
\begin{pgfscope}%
\pgfsetbuttcap%
\pgfsetroundjoin%
\definecolor{currentfill}{rgb}{0.000000,0.000000,0.000000}%
\pgfsetfillcolor{currentfill}%
\pgfsetlinewidth{0.803000pt}%
\definecolor{currentstroke}{rgb}{0.000000,0.000000,0.000000}%
\pgfsetstrokecolor{currentstroke}%
\pgfsetdash{}{0pt}%
\pgfsys@defobject{currentmarker}{\pgfqpoint{-0.048611in}{0.000000in}}{\pgfqpoint{-0.000000in}{0.000000in}}{%
\pgfpathmoveto{\pgfqpoint{-0.000000in}{0.000000in}}%
\pgfpathlineto{\pgfqpoint{-0.048611in}{0.000000in}}%
\pgfusepath{stroke,fill}%
}%
\begin{pgfscope}%
\pgfsys@transformshift{0.789096in}{1.915000in}%
\pgfsys@useobject{currentmarker}{}%
\end{pgfscope}%
\end{pgfscope}%
\begin{pgfscope}%
\definecolor{textcolor}{rgb}{0.000000,0.000000,0.000000}%
\pgfsetstrokecolor{textcolor}%
\pgfsetfillcolor{textcolor}%
\pgftext[x=0.395138in, y=1.831666in, left, base]{\color{textcolor}{\rmfamily\fontsize{16.000000}{19.200000}\selectfont\catcode`\^=\active\def^{\ifmmode\sp\else\^{}\fi}\catcode`\%=\active\def%{\%}$\mathdefault{\ensuremath{-}1}$}}%
\end{pgfscope}%
\begin{pgfscope}%
\pgfpathrectangle{\pgfqpoint{0.789096in}{0.375000in}}{\pgfqpoint{6.200000in}{4.620000in}}%
\pgfusepath{clip}%
\pgfsetrectcap%
\pgfsetroundjoin%
\pgfsetlinewidth{0.803000pt}%
\definecolor{currentstroke}{rgb}{0.690196,0.690196,0.690196}%
\pgfsetstrokecolor{currentstroke}%
\pgfsetdash{}{0pt}%
\pgfpathmoveto{\pgfqpoint{0.789096in}{2.685000in}}%
\pgfpathlineto{\pgfqpoint{6.989096in}{2.685000in}}%
\pgfusepath{stroke}%
\end{pgfscope}%
\begin{pgfscope}%
\pgfsetbuttcap%
\pgfsetroundjoin%
\definecolor{currentfill}{rgb}{0.000000,0.000000,0.000000}%
\pgfsetfillcolor{currentfill}%
\pgfsetlinewidth{0.803000pt}%
\definecolor{currentstroke}{rgb}{0.000000,0.000000,0.000000}%
\pgfsetstrokecolor{currentstroke}%
\pgfsetdash{}{0pt}%
\pgfsys@defobject{currentmarker}{\pgfqpoint{-0.048611in}{0.000000in}}{\pgfqpoint{-0.000000in}{0.000000in}}{%
\pgfpathmoveto{\pgfqpoint{-0.000000in}{0.000000in}}%
\pgfpathlineto{\pgfqpoint{-0.048611in}{0.000000in}}%
\pgfusepath{stroke,fill}%
}%
\begin{pgfscope}%
\pgfsys@transformshift{0.789096in}{2.685000in}%
\pgfsys@useobject{currentmarker}{}%
\end{pgfscope}%
\end{pgfscope}%
\begin{pgfscope}%
\definecolor{textcolor}{rgb}{0.000000,0.000000,0.000000}%
\pgfsetstrokecolor{textcolor}%
\pgfsetfillcolor{textcolor}%
\pgftext[x=0.581806in, y=2.601666in, left, base]{\color{textcolor}{\rmfamily\fontsize{16.000000}{19.200000}\selectfont\catcode`\^=\active\def^{\ifmmode\sp\else\^{}\fi}\catcode`\%=\active\def%{\%}$\mathdefault{0}$}}%
\end{pgfscope}%
\begin{pgfscope}%
\pgfpathrectangle{\pgfqpoint{0.789096in}{0.375000in}}{\pgfqpoint{6.200000in}{4.620000in}}%
\pgfusepath{clip}%
\pgfsetrectcap%
\pgfsetroundjoin%
\pgfsetlinewidth{0.803000pt}%
\definecolor{currentstroke}{rgb}{0.690196,0.690196,0.690196}%
\pgfsetstrokecolor{currentstroke}%
\pgfsetdash{}{0pt}%
\pgfpathmoveto{\pgfqpoint{0.789096in}{3.455000in}}%
\pgfpathlineto{\pgfqpoint{6.989096in}{3.455000in}}%
\pgfusepath{stroke}%
\end{pgfscope}%
\begin{pgfscope}%
\pgfsetbuttcap%
\pgfsetroundjoin%
\definecolor{currentfill}{rgb}{0.000000,0.000000,0.000000}%
\pgfsetfillcolor{currentfill}%
\pgfsetlinewidth{0.803000pt}%
\definecolor{currentstroke}{rgb}{0.000000,0.000000,0.000000}%
\pgfsetstrokecolor{currentstroke}%
\pgfsetdash{}{0pt}%
\pgfsys@defobject{currentmarker}{\pgfqpoint{-0.048611in}{0.000000in}}{\pgfqpoint{-0.000000in}{0.000000in}}{%
\pgfpathmoveto{\pgfqpoint{-0.000000in}{0.000000in}}%
\pgfpathlineto{\pgfqpoint{-0.048611in}{0.000000in}}%
\pgfusepath{stroke,fill}%
}%
\begin{pgfscope}%
\pgfsys@transformshift{0.789096in}{3.455000in}%
\pgfsys@useobject{currentmarker}{}%
\end{pgfscope}%
\end{pgfscope}%
\begin{pgfscope}%
\definecolor{textcolor}{rgb}{0.000000,0.000000,0.000000}%
\pgfsetstrokecolor{textcolor}%
\pgfsetfillcolor{textcolor}%
\pgftext[x=0.581806in, y=3.371666in, left, base]{\color{textcolor}{\rmfamily\fontsize{16.000000}{19.200000}\selectfont\catcode`\^=\active\def^{\ifmmode\sp\else\^{}\fi}\catcode`\%=\active\def%{\%}$\mathdefault{1}$}}%
\end{pgfscope}%
\begin{pgfscope}%
\pgfpathrectangle{\pgfqpoint{0.789096in}{0.375000in}}{\pgfqpoint{6.200000in}{4.620000in}}%
\pgfusepath{clip}%
\pgfsetrectcap%
\pgfsetroundjoin%
\pgfsetlinewidth{0.803000pt}%
\definecolor{currentstroke}{rgb}{0.690196,0.690196,0.690196}%
\pgfsetstrokecolor{currentstroke}%
\pgfsetdash{}{0pt}%
\pgfpathmoveto{\pgfqpoint{0.789096in}{4.225000in}}%
\pgfpathlineto{\pgfqpoint{6.989096in}{4.225000in}}%
\pgfusepath{stroke}%
\end{pgfscope}%
\begin{pgfscope}%
\pgfsetbuttcap%
\pgfsetroundjoin%
\definecolor{currentfill}{rgb}{0.000000,0.000000,0.000000}%
\pgfsetfillcolor{currentfill}%
\pgfsetlinewidth{0.803000pt}%
\definecolor{currentstroke}{rgb}{0.000000,0.000000,0.000000}%
\pgfsetstrokecolor{currentstroke}%
\pgfsetdash{}{0pt}%
\pgfsys@defobject{currentmarker}{\pgfqpoint{-0.048611in}{0.000000in}}{\pgfqpoint{-0.000000in}{0.000000in}}{%
\pgfpathmoveto{\pgfqpoint{-0.000000in}{0.000000in}}%
\pgfpathlineto{\pgfqpoint{-0.048611in}{0.000000in}}%
\pgfusepath{stroke,fill}%
}%
\begin{pgfscope}%
\pgfsys@transformshift{0.789096in}{4.225000in}%
\pgfsys@useobject{currentmarker}{}%
\end{pgfscope}%
\end{pgfscope}%
\begin{pgfscope}%
\definecolor{textcolor}{rgb}{0.000000,0.000000,0.000000}%
\pgfsetstrokecolor{textcolor}%
\pgfsetfillcolor{textcolor}%
\pgftext[x=0.581806in, y=4.141666in, left, base]{\color{textcolor}{\rmfamily\fontsize{16.000000}{19.200000}\selectfont\catcode`\^=\active\def^{\ifmmode\sp\else\^{}\fi}\catcode`\%=\active\def%{\%}$\mathdefault{2}$}}%
\end{pgfscope}%
\begin{pgfscope}%
\pgfpathrectangle{\pgfqpoint{0.789096in}{0.375000in}}{\pgfqpoint{6.200000in}{4.620000in}}%
\pgfusepath{clip}%
\pgfsetrectcap%
\pgfsetroundjoin%
\pgfsetlinewidth{0.803000pt}%
\definecolor{currentstroke}{rgb}{0.690196,0.690196,0.690196}%
\pgfsetstrokecolor{currentstroke}%
\pgfsetdash{}{0pt}%
\pgfpathmoveto{\pgfqpoint{0.789096in}{4.995000in}}%
\pgfpathlineto{\pgfqpoint{6.989096in}{4.995000in}}%
\pgfusepath{stroke}%
\end{pgfscope}%
\begin{pgfscope}%
\pgfsetbuttcap%
\pgfsetroundjoin%
\definecolor{currentfill}{rgb}{0.000000,0.000000,0.000000}%
\pgfsetfillcolor{currentfill}%
\pgfsetlinewidth{0.803000pt}%
\definecolor{currentstroke}{rgb}{0.000000,0.000000,0.000000}%
\pgfsetstrokecolor{currentstroke}%
\pgfsetdash{}{0pt}%
\pgfsys@defobject{currentmarker}{\pgfqpoint{-0.048611in}{0.000000in}}{\pgfqpoint{-0.000000in}{0.000000in}}{%
\pgfpathmoveto{\pgfqpoint{-0.000000in}{0.000000in}}%
\pgfpathlineto{\pgfqpoint{-0.048611in}{0.000000in}}%
\pgfusepath{stroke,fill}%
}%
\begin{pgfscope}%
\pgfsys@transformshift{0.789096in}{4.995000in}%
\pgfsys@useobject{currentmarker}{}%
\end{pgfscope}%
\end{pgfscope}%
\begin{pgfscope}%
\definecolor{textcolor}{rgb}{0.000000,0.000000,0.000000}%
\pgfsetstrokecolor{textcolor}%
\pgfsetfillcolor{textcolor}%
\pgftext[x=0.581806in, y=4.911666in, left, base]{\color{textcolor}{\rmfamily\fontsize{16.000000}{19.200000}\selectfont\catcode`\^=\active\def^{\ifmmode\sp\else\^{}\fi}\catcode`\%=\active\def%{\%}$\mathdefault{3}$}}%
\end{pgfscope}%
\begin{pgfscope}%
\definecolor{textcolor}{rgb}{0.000000,0.000000,0.000000}%
\pgfsetstrokecolor{textcolor}%
\pgfsetfillcolor{textcolor}%
\pgftext[x=0.339583in,y=2.685000in,,bottom,rotate=90.000000]{\color{textcolor}{\rmfamily\fontsize{18.000000}{21.600000}\selectfont\catcode`\^=\active\def^{\ifmmode\sp\else\^{}\fi}\catcode`\%=\active\def%{\%}$\Delta$ (ARES - PTC) [cent/kWh]}}%
\end{pgfscope}%
\begin{pgfscope}%
\pgfpathrectangle{\pgfqpoint{0.789096in}{0.375000in}}{\pgfqpoint{6.200000in}{4.620000in}}%
\pgfusepath{clip}%
\pgfsetrectcap%
\pgfsetroundjoin%
\pgfsetlinewidth{1.505625pt}%
\definecolor{currentstroke}{rgb}{0.121569,0.466667,0.705882}%
\pgfsetstrokecolor{currentstroke}%
\pgfsetdash{}{0pt}%
\pgfpathmoveto{\pgfqpoint{1.070914in}{3.850010in}}%
\pgfpathlineto{\pgfqpoint{2.198187in}{3.511210in}}%
\pgfpathlineto{\pgfqpoint{3.325460in}{2.955270in}}%
\pgfpathlineto{\pgfqpoint{4.452732in}{1.747140in}}%
\pgfpathlineto{\pgfqpoint{5.580005in}{4.325100in}}%
\pgfpathlineto{\pgfqpoint{6.707278in}{3.924700in}}%
\pgfusepath{stroke}%
\end{pgfscope}%
\begin{pgfscope}%
\pgfpathrectangle{\pgfqpoint{0.789096in}{0.375000in}}{\pgfqpoint{6.200000in}{4.620000in}}%
\pgfusepath{clip}%
\pgfsetbuttcap%
\pgfsetroundjoin%
\definecolor{currentfill}{rgb}{0.121569,0.466667,0.705882}%
\pgfsetfillcolor{currentfill}%
\pgfsetlinewidth{1.003750pt}%
\definecolor{currentstroke}{rgb}{0.121569,0.466667,0.705882}%
\pgfsetstrokecolor{currentstroke}%
\pgfsetdash{}{0pt}%
\pgfsys@defobject{currentmarker}{\pgfqpoint{-0.041667in}{-0.041667in}}{\pgfqpoint{0.041667in}{0.041667in}}{%
\pgfpathmoveto{\pgfqpoint{0.000000in}{-0.041667in}}%
\pgfpathcurveto{\pgfqpoint{0.011050in}{-0.041667in}}{\pgfqpoint{0.021649in}{-0.037276in}}{\pgfqpoint{0.029463in}{-0.029463in}}%
\pgfpathcurveto{\pgfqpoint{0.037276in}{-0.021649in}}{\pgfqpoint{0.041667in}{-0.011050in}}{\pgfqpoint{0.041667in}{0.000000in}}%
\pgfpathcurveto{\pgfqpoint{0.041667in}{0.011050in}}{\pgfqpoint{0.037276in}{0.021649in}}{\pgfqpoint{0.029463in}{0.029463in}}%
\pgfpathcurveto{\pgfqpoint{0.021649in}{0.037276in}}{\pgfqpoint{0.011050in}{0.041667in}}{\pgfqpoint{0.000000in}{0.041667in}}%
\pgfpathcurveto{\pgfqpoint{-0.011050in}{0.041667in}}{\pgfqpoint{-0.021649in}{0.037276in}}{\pgfqpoint{-0.029463in}{0.029463in}}%
\pgfpathcurveto{\pgfqpoint{-0.037276in}{0.021649in}}{\pgfqpoint{-0.041667in}{0.011050in}}{\pgfqpoint{-0.041667in}{0.000000in}}%
\pgfpathcurveto{\pgfqpoint{-0.041667in}{-0.011050in}}{\pgfqpoint{-0.037276in}{-0.021649in}}{\pgfqpoint{-0.029463in}{-0.029463in}}%
\pgfpathcurveto{\pgfqpoint{-0.021649in}{-0.037276in}}{\pgfqpoint{-0.011050in}{-0.041667in}}{\pgfqpoint{0.000000in}{-0.041667in}}%
\pgfpathlineto{\pgfqpoint{0.000000in}{-0.041667in}}%
\pgfpathclose%
\pgfusepath{stroke,fill}%
}%
\begin{pgfscope}%
\pgfsys@transformshift{1.070914in}{3.850010in}%
\pgfsys@useobject{currentmarker}{}%
\end{pgfscope}%
\begin{pgfscope}%
\pgfsys@transformshift{2.198187in}{3.511210in}%
\pgfsys@useobject{currentmarker}{}%
\end{pgfscope}%
\begin{pgfscope}%
\pgfsys@transformshift{3.325460in}{2.955270in}%
\pgfsys@useobject{currentmarker}{}%
\end{pgfscope}%
\begin{pgfscope}%
\pgfsys@transformshift{4.452732in}{1.747140in}%
\pgfsys@useobject{currentmarker}{}%
\end{pgfscope}%
\begin{pgfscope}%
\pgfsys@transformshift{5.580005in}{4.325100in}%
\pgfsys@useobject{currentmarker}{}%
\end{pgfscope}%
\begin{pgfscope}%
\pgfsys@transformshift{6.707278in}{3.924700in}%
\pgfsys@useobject{currentmarker}{}%
\end{pgfscope}%
\end{pgfscope}%
\begin{pgfscope}%
\pgfpathrectangle{\pgfqpoint{0.789096in}{0.375000in}}{\pgfqpoint{6.200000in}{4.620000in}}%
\pgfusepath{clip}%
\pgfsetrectcap%
\pgfsetroundjoin%
\pgfsetlinewidth{1.505625pt}%
\definecolor{currentstroke}{rgb}{1.000000,0.498039,0.054902}%
\pgfsetstrokecolor{currentstroke}%
\pgfsetdash{}{0pt}%
\pgfpathmoveto{\pgfqpoint{1.070914in}{3.989380in}}%
\pgfpathlineto{\pgfqpoint{2.198187in}{3.677530in}}%
\pgfpathlineto{\pgfqpoint{3.325460in}{3.964740in}}%
\pgfpathlineto{\pgfqpoint{4.452732in}{4.094100in}}%
\pgfpathlineto{\pgfqpoint{5.580005in}{4.910300in}}%
\pgfpathlineto{\pgfqpoint{6.707278in}{4.579200in}}%
\pgfusepath{stroke}%
\end{pgfscope}%
\begin{pgfscope}%
\pgfpathrectangle{\pgfqpoint{0.789096in}{0.375000in}}{\pgfqpoint{6.200000in}{4.620000in}}%
\pgfusepath{clip}%
\pgfsetbuttcap%
\pgfsetroundjoin%
\definecolor{currentfill}{rgb}{1.000000,0.498039,0.054902}%
\pgfsetfillcolor{currentfill}%
\pgfsetlinewidth{1.003750pt}%
\definecolor{currentstroke}{rgb}{1.000000,0.498039,0.054902}%
\pgfsetstrokecolor{currentstroke}%
\pgfsetdash{}{0pt}%
\pgfsys@defobject{currentmarker}{\pgfqpoint{-0.041667in}{-0.041667in}}{\pgfqpoint{0.041667in}{0.041667in}}{%
\pgfpathmoveto{\pgfqpoint{0.000000in}{-0.041667in}}%
\pgfpathcurveto{\pgfqpoint{0.011050in}{-0.041667in}}{\pgfqpoint{0.021649in}{-0.037276in}}{\pgfqpoint{0.029463in}{-0.029463in}}%
\pgfpathcurveto{\pgfqpoint{0.037276in}{-0.021649in}}{\pgfqpoint{0.041667in}{-0.011050in}}{\pgfqpoint{0.041667in}{0.000000in}}%
\pgfpathcurveto{\pgfqpoint{0.041667in}{0.011050in}}{\pgfqpoint{0.037276in}{0.021649in}}{\pgfqpoint{0.029463in}{0.029463in}}%
\pgfpathcurveto{\pgfqpoint{0.021649in}{0.037276in}}{\pgfqpoint{0.011050in}{0.041667in}}{\pgfqpoint{0.000000in}{0.041667in}}%
\pgfpathcurveto{\pgfqpoint{-0.011050in}{0.041667in}}{\pgfqpoint{-0.021649in}{0.037276in}}{\pgfqpoint{-0.029463in}{0.029463in}}%
\pgfpathcurveto{\pgfqpoint{-0.037276in}{0.021649in}}{\pgfqpoint{-0.041667in}{0.011050in}}{\pgfqpoint{-0.041667in}{0.000000in}}%
\pgfpathcurveto{\pgfqpoint{-0.041667in}{-0.011050in}}{\pgfqpoint{-0.037276in}{-0.021649in}}{\pgfqpoint{-0.029463in}{-0.029463in}}%
\pgfpathcurveto{\pgfqpoint{-0.021649in}{-0.037276in}}{\pgfqpoint{-0.011050in}{-0.041667in}}{\pgfqpoint{0.000000in}{-0.041667in}}%
\pgfpathlineto{\pgfqpoint{0.000000in}{-0.041667in}}%
\pgfpathclose%
\pgfusepath{stroke,fill}%
}%
\begin{pgfscope}%
\pgfsys@transformshift{1.070914in}{3.989380in}%
\pgfsys@useobject{currentmarker}{}%
\end{pgfscope}%
\begin{pgfscope}%
\pgfsys@transformshift{2.198187in}{3.677530in}%
\pgfsys@useobject{currentmarker}{}%
\end{pgfscope}%
\begin{pgfscope}%
\pgfsys@transformshift{3.325460in}{3.964740in}%
\pgfsys@useobject{currentmarker}{}%
\end{pgfscope}%
\begin{pgfscope}%
\pgfsys@transformshift{4.452732in}{4.094100in}%
\pgfsys@useobject{currentmarker}{}%
\end{pgfscope}%
\begin{pgfscope}%
\pgfsys@transformshift{5.580005in}{4.910300in}%
\pgfsys@useobject{currentmarker}{}%
\end{pgfscope}%
\begin{pgfscope}%
\pgfsys@transformshift{6.707278in}{4.579200in}%
\pgfsys@useobject{currentmarker}{}%
\end{pgfscope}%
\end{pgfscope}%
\begin{pgfscope}%
\pgfpathrectangle{\pgfqpoint{0.789096in}{0.375000in}}{\pgfqpoint{6.200000in}{4.620000in}}%
\pgfusepath{clip}%
\pgfsetbuttcap%
\pgfsetroundjoin%
\pgfsetlinewidth{1.505625pt}%
\definecolor{currentstroke}{rgb}{0.839216,0.152941,0.156863}%
\pgfsetstrokecolor{currentstroke}%
\pgfsetdash{{5.550000pt}{2.400000pt}}{0.000000pt}%
\pgfpathmoveto{\pgfqpoint{0.789096in}{2.685000in}}%
\pgfpathlineto{\pgfqpoint{6.989096in}{2.685000in}}%
\pgfusepath{stroke}%
\end{pgfscope}%
\begin{pgfscope}%
\pgfsetrectcap%
\pgfsetmiterjoin%
\pgfsetlinewidth{0.803000pt}%
\definecolor{currentstroke}{rgb}{0.000000,0.000000,0.000000}%
\pgfsetstrokecolor{currentstroke}%
\pgfsetdash{}{0pt}%
\pgfpathmoveto{\pgfqpoint{0.789096in}{0.375000in}}%
\pgfpathlineto{\pgfqpoint{0.789096in}{4.995000in}}%
\pgfusepath{stroke}%
\end{pgfscope}%
\begin{pgfscope}%
\pgfsetrectcap%
\pgfsetmiterjoin%
\pgfsetlinewidth{0.803000pt}%
\definecolor{currentstroke}{rgb}{0.000000,0.000000,0.000000}%
\pgfsetstrokecolor{currentstroke}%
\pgfsetdash{}{0pt}%
\pgfpathmoveto{\pgfqpoint{6.989096in}{0.375000in}}%
\pgfpathlineto{\pgfqpoint{6.989096in}{4.995000in}}%
\pgfusepath{stroke}%
\end{pgfscope}%
\begin{pgfscope}%
\pgfsetrectcap%
\pgfsetmiterjoin%
\pgfsetlinewidth{0.803000pt}%
\definecolor{currentstroke}{rgb}{0.000000,0.000000,0.000000}%
\pgfsetstrokecolor{currentstroke}%
\pgfsetdash{}{0pt}%
\pgfpathmoveto{\pgfqpoint{0.789096in}{0.375000in}}%
\pgfpathlineto{\pgfqpoint{6.989096in}{0.375000in}}%
\pgfusepath{stroke}%
\end{pgfscope}%
\begin{pgfscope}%
\pgfsetrectcap%
\pgfsetmiterjoin%
\pgfsetlinewidth{0.803000pt}%
\definecolor{currentstroke}{rgb}{0.000000,0.000000,0.000000}%
\pgfsetstrokecolor{currentstroke}%
\pgfsetdash{}{0pt}%
\pgfpathmoveto{\pgfqpoint{0.789096in}{4.995000in}}%
\pgfpathlineto{\pgfqpoint{6.989096in}{4.995000in}}%
\pgfusepath{stroke}%
\end{pgfscope}%
\begin{pgfscope}%
\pgfsetbuttcap%
\pgfsetmiterjoin%
\definecolor{currentfill}{rgb}{1.000000,1.000000,1.000000}%
\pgfsetfillcolor{currentfill}%
\pgfsetfillopacity{0.800000}%
\pgfsetlinewidth{1.003750pt}%
\definecolor{currentstroke}{rgb}{0.800000,0.800000,0.800000}%
\pgfsetstrokecolor{currentstroke}%
\pgfsetstrokeopacity{0.800000}%
\pgfsetdash{}{0pt}%
\pgfpathmoveto{\pgfqpoint{0.944652in}{4.168302in}}%
\pgfpathlineto{\pgfqpoint{2.410115in}{4.168302in}}%
\pgfpathquadraticcurveto{\pgfqpoint{2.454560in}{4.168302in}}{\pgfqpoint{2.454560in}{4.212747in}}%
\pgfpathlineto{\pgfqpoint{2.454560in}{4.839444in}}%
\pgfpathquadraticcurveto{\pgfqpoint{2.454560in}{4.883888in}}{\pgfqpoint{2.410115in}{4.883888in}}%
\pgfpathlineto{\pgfqpoint{0.944652in}{4.883888in}}%
\pgfpathquadraticcurveto{\pgfqpoint{0.900207in}{4.883888in}}{\pgfqpoint{0.900207in}{4.839444in}}%
\pgfpathlineto{\pgfqpoint{0.900207in}{4.212747in}}%
\pgfpathquadraticcurveto{\pgfqpoint{0.900207in}{4.168302in}}{\pgfqpoint{0.944652in}{4.168302in}}%
\pgfpathlineto{\pgfqpoint{0.944652in}{4.168302in}}%
\pgfpathclose%
\pgfusepath{stroke,fill}%
\end{pgfscope}%
\begin{pgfscope}%
\pgfsetrectcap%
\pgfsetroundjoin%
\pgfsetlinewidth{1.505625pt}%
\definecolor{currentstroke}{rgb}{0.121569,0.466667,0.705882}%
\pgfsetstrokecolor{currentstroke}%
\pgfsetdash{}{0pt}%
\pgfpathmoveto{\pgfqpoint{0.989096in}{4.706111in}}%
\pgfpathlineto{\pgfqpoint{1.211318in}{4.706111in}}%
\pgfpathlineto{\pgfqpoint{1.433540in}{4.706111in}}%
\pgfusepath{stroke}%
\end{pgfscope}%
\begin{pgfscope}%
\pgfsetbuttcap%
\pgfsetroundjoin%
\definecolor{currentfill}{rgb}{0.121569,0.466667,0.705882}%
\pgfsetfillcolor{currentfill}%
\pgfsetlinewidth{1.003750pt}%
\definecolor{currentstroke}{rgb}{0.121569,0.466667,0.705882}%
\pgfsetstrokecolor{currentstroke}%
\pgfsetdash{}{0pt}%
\pgfsys@defobject{currentmarker}{\pgfqpoint{-0.041667in}{-0.041667in}}{\pgfqpoint{0.041667in}{0.041667in}}{%
\pgfpathmoveto{\pgfqpoint{0.000000in}{-0.041667in}}%
\pgfpathcurveto{\pgfqpoint{0.011050in}{-0.041667in}}{\pgfqpoint{0.021649in}{-0.037276in}}{\pgfqpoint{0.029463in}{-0.029463in}}%
\pgfpathcurveto{\pgfqpoint{0.037276in}{-0.021649in}}{\pgfqpoint{0.041667in}{-0.011050in}}{\pgfqpoint{0.041667in}{0.000000in}}%
\pgfpathcurveto{\pgfqpoint{0.041667in}{0.011050in}}{\pgfqpoint{0.037276in}{0.021649in}}{\pgfqpoint{0.029463in}{0.029463in}}%
\pgfpathcurveto{\pgfqpoint{0.021649in}{0.037276in}}{\pgfqpoint{0.011050in}{0.041667in}}{\pgfqpoint{0.000000in}{0.041667in}}%
\pgfpathcurveto{\pgfqpoint{-0.011050in}{0.041667in}}{\pgfqpoint{-0.021649in}{0.037276in}}{\pgfqpoint{-0.029463in}{0.029463in}}%
\pgfpathcurveto{\pgfqpoint{-0.037276in}{0.021649in}}{\pgfqpoint{-0.041667in}{0.011050in}}{\pgfqpoint{-0.041667in}{0.000000in}}%
\pgfpathcurveto{\pgfqpoint{-0.041667in}{-0.011050in}}{\pgfqpoint{-0.037276in}{-0.021649in}}{\pgfqpoint{-0.029463in}{-0.029463in}}%
\pgfpathcurveto{\pgfqpoint{-0.021649in}{-0.037276in}}{\pgfqpoint{-0.011050in}{-0.041667in}}{\pgfqpoint{0.000000in}{-0.041667in}}%
\pgfpathlineto{\pgfqpoint{0.000000in}{-0.041667in}}%
\pgfpathclose%
\pgfusepath{stroke,fill}%
}%
\begin{pgfscope}%
\pgfsys@transformshift{1.211318in}{4.706111in}%
\pgfsys@useobject{currentmarker}{}%
\end{pgfscope}%
\end{pgfscope}%
\begin{pgfscope}%
\definecolor{textcolor}{rgb}{0.000000,0.000000,0.000000}%
\pgfsetstrokecolor{textcolor}%
\pgfsetfillcolor{textcolor}%
\pgftext[x=1.611318in,y=4.628333in,left,base]{\color{textcolor}{\rmfamily\fontsize{16.000000}{19.200000}\selectfont\catcode`\^=\active\def^{\ifmmode\sp\else\^{}\fi}\catcode`\%=\active\def%{\%}Ameren}}%
\end{pgfscope}%
\begin{pgfscope}%
\pgfsetrectcap%
\pgfsetroundjoin%
\pgfsetlinewidth{1.505625pt}%
\definecolor{currentstroke}{rgb}{1.000000,0.498039,0.054902}%
\pgfsetstrokecolor{currentstroke}%
\pgfsetdash{}{0pt}%
\pgfpathmoveto{\pgfqpoint{0.989096in}{4.381651in}}%
\pgfpathlineto{\pgfqpoint{1.211318in}{4.381651in}}%
\pgfpathlineto{\pgfqpoint{1.433540in}{4.381651in}}%
\pgfusepath{stroke}%
\end{pgfscope}%
\begin{pgfscope}%
\pgfsetbuttcap%
\pgfsetroundjoin%
\definecolor{currentfill}{rgb}{1.000000,0.498039,0.054902}%
\pgfsetfillcolor{currentfill}%
\pgfsetlinewidth{1.003750pt}%
\definecolor{currentstroke}{rgb}{1.000000,0.498039,0.054902}%
\pgfsetstrokecolor{currentstroke}%
\pgfsetdash{}{0pt}%
\pgfsys@defobject{currentmarker}{\pgfqpoint{-0.041667in}{-0.041667in}}{\pgfqpoint{0.041667in}{0.041667in}}{%
\pgfpathmoveto{\pgfqpoint{0.000000in}{-0.041667in}}%
\pgfpathcurveto{\pgfqpoint{0.011050in}{-0.041667in}}{\pgfqpoint{0.021649in}{-0.037276in}}{\pgfqpoint{0.029463in}{-0.029463in}}%
\pgfpathcurveto{\pgfqpoint{0.037276in}{-0.021649in}}{\pgfqpoint{0.041667in}{-0.011050in}}{\pgfqpoint{0.041667in}{0.000000in}}%
\pgfpathcurveto{\pgfqpoint{0.041667in}{0.011050in}}{\pgfqpoint{0.037276in}{0.021649in}}{\pgfqpoint{0.029463in}{0.029463in}}%
\pgfpathcurveto{\pgfqpoint{0.021649in}{0.037276in}}{\pgfqpoint{0.011050in}{0.041667in}}{\pgfqpoint{0.000000in}{0.041667in}}%
\pgfpathcurveto{\pgfqpoint{-0.011050in}{0.041667in}}{\pgfqpoint{-0.021649in}{0.037276in}}{\pgfqpoint{-0.029463in}{0.029463in}}%
\pgfpathcurveto{\pgfqpoint{-0.037276in}{0.021649in}}{\pgfqpoint{-0.041667in}{0.011050in}}{\pgfqpoint{-0.041667in}{0.000000in}}%
\pgfpathcurveto{\pgfqpoint{-0.041667in}{-0.011050in}}{\pgfqpoint{-0.037276in}{-0.021649in}}{\pgfqpoint{-0.029463in}{-0.029463in}}%
\pgfpathcurveto{\pgfqpoint{-0.021649in}{-0.037276in}}{\pgfqpoint{-0.011050in}{-0.041667in}}{\pgfqpoint{0.000000in}{-0.041667in}}%
\pgfpathlineto{\pgfqpoint{0.000000in}{-0.041667in}}%
\pgfpathclose%
\pgfusepath{stroke,fill}%
}%
\begin{pgfscope}%
\pgfsys@transformshift{1.211318in}{4.381651in}%
\pgfsys@useobject{currentmarker}{}%
\end{pgfscope}%
\end{pgfscope}%
\begin{pgfscope}%
\definecolor{textcolor}{rgb}{0.000000,0.000000,0.000000}%
\pgfsetstrokecolor{textcolor}%
\pgfsetfillcolor{textcolor}%
\pgftext[x=1.611318in,y=4.303873in,left,base]{\color{textcolor}{\rmfamily\fontsize{16.000000}{19.200000}\selectfont\catcode`\^=\active\def^{\ifmmode\sp\else\^{}\fi}\catcode`\%=\active\def%{\%}ComEd}}%
\end{pgfscope}%
\end{pgfpicture}%
\makeatother%
\endgroup%
}
    \caption{The average difference in cost between the \ac{ares} and the
    \ac{ptc} from the default supply.}
    \label{fig:ares-ptc}
\end{figure}

Consumers and municipalities can improve their bargaining position by
aggregating their constituents' electricity demand and negotiating on behalf of
the entire community in a process called \acf{mca}. Although \ac{mca} allows
communities to (sometimes) negotiate a better electricity price and (sometimes)
benefit from cleaner electricity with \ac{rec} purchases, energy supply choices
are still limited to those \ac{ares} that submit a bid to supply electricity (or
taking the default supply). Further, \acp{rec} generated through \ac{mca} do not
have \textit{additionality} \cite{illinois_power_agency_municipal_2023}. That
is, even when paying a premium for \acp{rec}, these \acp{rec} do not lead to new
renewable energy capacity. Interviewee IP3 describes this issue
\begin{quote}
     The challenge is, would [\ac{mca}] facilitate the development of new
     generation such that you could point to it as being additive or is it just
     a means for picking up environmental attributes from projects that are
     already built [and] not really as consequential to the public policy aims
     of why you're doing it in the first place. [\dots] The reality is that none
     of those purchases were making a new project come into the market that
     wouldn't exist otherwise (Interviewee IP3).
\end{quote}

So, for municipalities that do not have access to energy demand data, do not
control the condition of electric distribution infrastructure, and for whom
electric supply choices are not guaranteed to drive the development of new clean
energy projects, it is understandable that city planners do not use \acp{esom},
nor do they find the results from these tools particularly salient. 

Now, let us turn to the second scenario, where a municipality does have its own
electric utility. \ac{uiuc} and Naperville fall under this category. These
municipalities could exercise much more agency over energy supply decisions.
Especially \ac{uiuc} since it owns the majority of its own generating resources.
In addition, municipally-owned utilities are exempt from the Public Utilities
Act and therefore not regulated by the \ac{icc}. In theory, such municipalities
control 100\% of the energy supply decisions for their residents. However, in
the case of Naperville, its city council contracted with \ac{imea} to plan and
procure its energy supply. This is appealing because it does not require
Naperville to hire staff to manage energy supply, it allows \ac{imea}'s members
to pool their resources, and \ac{imea} provides additional administrative
services. However, \ac{imea} members are contractually obligated to receive
100\% of their electricity which means these municipalities have zero control in
their energy supply. This obligation either delays or prohibits investment in
certain \ac{btm} resources. ``Now they're legally bound to agree to everything
that was in that contract. So if in that contract it says that, well, you can't
install batteries behind the meter (Interviewee IA2).'' Naturally,
municipalities that are contractually prohibited from making energy supply
decisions will not benefit from energy modeling.

\subsection{Assessing \ac{osier}}
\label{section:assess-osier} 

The final section in each interview involved an introduction to \ac{osier}
through a short presentation and demonstration of the tool, followed by a
discussion and solicitation of feedback from the interviewees. 

The majority of the interviewees expressed strong positive impressions of the
tool along three major axes. First, they appreciated that \ac{osier} is capable
of modeling objectives besides cost because policy makers and their constituents
have priorities that are not or cannot be neatly captured by a cost metric. In
the words of interviewee IP3
\begin{quote}
    [L]east costs only make sense if you assume that markets are properly
    internalizing and socializing all costs. And that's impossible to have
    happen because certain things are more acutely felt by some than others. And
    certain costs don't want themselves being internalized into a market. So
    you're just never going to get a properly balanced kind of snapshot of
    efficient outcomes if you're focused only on cost in a narrow sense. And I
    think there's a broader recognition [\dots] that we need to balance other
    public policy objectives [\dots] and so you need a tool like this in order
    to properly balance other public policy objectives. Otherwise, you have
    outcomes that are tilted too firmly in favor of those things where the costs
    are socialized or are not quantified. So I see the utility --- that seems
    very evident to me.
\end{quote}
Further, interviewees identified the benefits of illustrating tradeoffs among
policy options or energy visions.
\begin{quote}
    I mean this is exactly what we need, an energy model that shows us like
    something other than cost but I think too people are starting to ask
    questions about [\dots] \acp{smr} and you know crossing that with land use
    with other things that people are like `I think nuclear is good but what are
    the tradeoffs for nuclear?' [T]o have something you can input those
    objectives into and show you, I think that that is going to be infinitely
    useful. (Interviewee IA2)
\end{quote}

Second, interviewees found that \ac{osier} could be useful for communicating
with both decision-makers and constituents. As well as generating buy-in with
other staff members at an organization.
\begin{quote}
    [W]hat I see is that this tool could give us the ability to communicate with
    those decision-makers who are not in the trenches about the details, maybe
    don't even know what renewable energy certificates are, or renewable
    identification numbers for renewable natural gas, or they're afraid of
    nuclear because they don't understand it at all. And those folks I think
    this would be a useful tool for communicating with them and getting buy-in
    but I also think it could be useful just within the utilities team as
    they're identifying you know what do they want to advocate for and why.
    (Interviewee UI1)
\end{quote}

Finally, interviewees suggested that \ac{osier} could be used to enhance
participation by members of the public and also, because \ac{osier} is open
source and open data, encourage engagement from municipalities that otherwise do
not have the resources to do energy modeling. Two interviewees (UP1, UP2)
caveated this insight with the need for ``technical translation'' or
``simplified charts'' but agreed that \ac{osier} could enhance dialogue and
participation. ``If it's open source then everybody can have access to it.
Everybody can be involved in the process and communities that don't have the
resources to buy software can do this themselves in a sense or be involved in
the process'' (interviewee UP2). 

Although not unique to \ac{osier}, the interviewees universally identified the
challenge of finding reliable input data to use in a model. \ac{osier}'s
programming interface received mixed feedback. Several interviewees did not have
experience with progamming and determined that the lack of a \ac{gui} would be a
barrier to \ac{osier}'s adoption for in-house modeling work. 
\begin{quote}
    I'd never heard of Python before. For me, it would require an education or
    maybe a little bit more ability to tweak using known methods. So like slider
    to move one up and one down, like percentage importance of cost versus
    clean. Yeah, kind of like, I think, like Power BI can be used sometimes in
    ways where the user can interact with the charts and move stuff around a
    little bit. Yeah, I would need help (Interviewee NP1).
\end{quote}
However, one interviewee with some exposure to Python appreciated the examples
available with \ac{osier}'s documentation. ``This is what I was looking for, a
low cost entrance into the tool. This is the standard way where you've got
discussion and then in between lines of code. I love it'' (Interviewee NA2).

Finally, interviewees identified some features that would be valuable for an
\ac{esom} tool, such as \ac{osier}, to support. Representing energy efficiency
measures, such as improvements to building heating envelopes or switching to LED
lights, was the most frequently mentioned feature. Electric vehicle charging and
distributed energy resources were the second most requested feature.
Interviewees also expressed interest in represeting policy levers such as time
of use rates for electricity or net energy metering.