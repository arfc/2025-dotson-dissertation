
\section{Example 1: Comparing \ac{osier} and the \acs{set}}

Existing and advanced nuclear reactors will be important resources for
addressing the multi-dimensional challenge of climate change. Further, new
nuclear fuel cycle infrastructure will be necessary to support the next
generation of reactors and handling existing \ac{hlw} waste. Given the myriad
possible fuel cycle options, how should decision makers choose a fuel cycle to
pursue? The comprehensive tool developed for the U.S. federal program is the
\acf{set}, which was developed as part of the \ac{fco} in 2011 after the United
States \ac{doe} commissioned a report called ``Nuclear Fuel Cycle Evaluation and
Screening Study'' (``the Study'')\cite{wigeland_nuclear_2014-2}. The \ac{set}
and corresponding report presented recommendations based on a metric weighting
analysis facilitated by the \ac{set} where fuel cycles were ranked from ``most
promising'' to ``less promising'' and the least promising options were given no
designation. This example extends the work done by the Study by using \ac{osier}
to identify Pareto optimal solutions and using \ac{mcda} tools to further select
interesting fuel cycle(s). The next section gives a high-level overview of the
nuclear fuel cycle. Followed by sections describing the \ac{set} and its
methodology. Lastly, new results from \ac{osier} are shared and discussed.

\subsection{Overview of the Nuclear Fuel Cycle}

The nuclear fuel cycle describes the cradle-to-grave lifecycle of nuclear
materials. The nuclear fuel cycle begins with extraction via mining or in-situ
leaching, followed by processing into usable fuel forms, then irradiation in a
reactor, and finally either disposed of or recycled
\cite{tsoulfanidis_review_2013}. Figure \ref{fig:nuclear-fuel-cycle} presents a
stylized illustration of the fuel cycle. The light green boxes correspond to
stages in the frontend of the fuel cycle, while the beige boxes represent the
backend of the fuel cycle. Nuclear material generally flows in the direction of
the solid lines, except in ``closed'' fuel cycles where irradiated material
takes a detour through some reprocessing stage or stages denoted by the dashed
arrows. Several stages could be expanded into further sub-steps --- these were
omitted for simplicity. The Study and the \ac{set} represent holistic analyses
that include every stage of the fuel cycle.

\begin{figure}[htbp!]
  \centering
  \resizebox{\columnwidth}{!}{\begin{tikzpicture}[node distance=2cm]
    \definecolor{front-color}{HTML}{d1e0cc}
    \definecolor{reactor-color}{HTML}{ccd7e0}
    \definecolor{back-color}{HTML}{eddac5}
    \tikzstyle{every node}=[font=\small]
    \tikzstyle{front} = [rectangle, draw, fill=front-color, text width=6em, text centered, rounded corners, minimum height=3em]
    \tikzstyle{reactor} = [rectangle, draw, fill=reactor-color, text width=6em, text centered, rounded corners, minimum height=3em]
    \tikzstyle{back} = [rectangle, draw, fill=back-color, text width=6em, text centered, rounded corners, minimum height=3em]
        \node (front1) [front] {\textbf{Mining}}; 
        \node (front2) [front, right of=front1, xshift=1cm] {\textbf{Milling}}; 
        \node (front3) [front, right of=front2, xshift=1cm]{\textbf{Conversion}}; 
        \node (front4) [front, right of=front3, xshift=1cm]{\textbf{Enrichment}}; 
        \node (front5) [front, right of=front4, xshift=1cm]{\textbf{Fabrication}}; 
        \node (reactor1) [reactor, below of=front5]{\textbf{Reactor}}; 
        \node (back1) [back, below of=reactor1, yshift=-2cm]{\textbf{Interim Storage}}; 
        \node (back2) [back, left of=reactor1, xshift=-1cm, yshift=-2cm]{\textbf{Reprocessing}}; 
        \node (back3) [back, left of=back2, xshift=-1cm, yshift=-2cm]{\textbf{Final\\Disposal}}; 
        \draw [arrow] (front1) -- (front2); 
        \draw [arrow] (front2) -- (front3); 
        \draw [arrow] (front3) -- (front4); 
        \draw [arrow] (front4) -- (front5); 
        \draw [arrow] (front5) -- (reactor1); 
        \draw [arrow] (reactor1) -- (back1);
        \draw [arrow, dashed] (back1) -- (back2); 
        \draw [arrow, dashed] (back2) -- (back3); 
        \draw [arrow] (back1) -- (back3); 
        \draw [arrow, dashed] (back2) to[out=45, in=210] node[anchor=east] {Plutonium} (front5);
        \draw [arrow, dashed] (back2) to[out=100, in=210] node[anchor=east] {Uranium} (front4);
\end{tikzpicture}}
  \caption{Overview of the nuclear fuel cycle. Arrows indicate the flow of
  materials through the fuel cycle. Dashed lines correspond to a closed fuel
  cycle.}
  \label{fig:nuclear-fuel-cycle}
\end{figure}

\subsection{\ac{set} Metrics and Methodology}

The \ac{set} is an Excel-based application containing metric data from the
Nuclear Fuel Cycle Evaluation and Screening Study \cite{wigeland_nuclear_2014}
and an objective weighting calculator to help decision makers identify fuel
cycles of interest based on their priorities and preferences. The \ac{set}
evaluates representative fuel cycle options, \acp{eg}, based on a set of nine
aggregated metrics \cite{wigeland_nuclear_2014}. Table
\ref{tab:evaluation-metrics} lists these metrics and their sub-criteria. Some
criteria do not influence the \ac{set} results because all \acp{eg} performed
equally well. One such metric is ``material attractiveness,'' because all
\acp{eg} could be implemented with unattractive materials
\cite{wigeland_nuclear_2014-1}. Fuel cycles were assigned to \acp{eg} based on
shared physical principles such that differences in implementation could be
adequately captured \cite{wigeland_nuclear_2014}.  
Each of the criteria in Table \ref{tab:evaluation-metrics} were further
simplified through a binning process that grouped similarly performing \acp{eg}.
Figure \ref{fig:bin-plot} provides an example of how the Study's authors binned
the data.

\begin{table}[htbp!]
    \centering
    \caption{Evaluation metrics and evaluation criteria
    \cite{wigeland_nuclear_2014-2}.}
    \label{tab:evaluation-metrics}
    \resizebox{\columnwidth}{!}{\begin{tabular}{ll}
    \toprule
    Metric & Criteria \\
    \midrule
    % Waste management
     & Mass of \ac{snf} + \ac{hlw}\\
     & Activity of \ac{snf} + \ac{hlw} (at 100 years)\\
    Nuclear Waste Management & Activity of \ac{snf} + \ac{hlw} (at 100,000 years)\\
    (per unit energy) & Mass of \ac{du} + \ac{ru}\\
     & \hspace{12em}+ \ac{rth}\\
     & Volume \ac{llw}\\
     \midrule
    % Proliferation
     Proliferation Risk & Material attractiveness -- normal operating conditions\\
     \midrule
    % Security
     Nuclear Material Security & Material attractiveness -- normal operating conditions\\
    Risk & Activity of \ac{snf} + \ac{hlw} (at 10 years) per unit energy\\
    \midrule
    % Safety
    Safety & Challenges of addressing safety hazards\\
    & Safety of deployed system\\
    \midrule
    % Environment
    & Land use\\
    Environmental Impact  & Water use\\
    (per unit energy) & Radiological exposure -- total estimated worker dose\\
     & \Ac{co2} emissions\\
     \midrule
    % Resource efficiency
     Resource Utilization & Natural uranium required\\
    (per unit energy) & Natural thorium required\\
    \midrule
    % Deployment risk
    & Development time\\
    Development and & Developemnt cost\\
    Deployment Risk & Cost from prototype to \ac{foak}\\
    & Compatibility with existing infrastructure\\
    &Familiarity with licensing regulations\\
    & Existence of market dis/incenctives\\
    \midrule
    % institutional issues
      & Compatibility with existing infrastructure\\
    Institutional Issues & Familiarity with licensing regulations\\
     & Existence of market dis/incentives\\
     \midrule
    Financial Risks & \ac{lcoe} at equilibrium\\
    and Economics &\\
    \bottomrule
\end{tabular}}
\end{table}


\begin{figure}[htbp!]
    \centering
    \resizebox{\columnwidth}{!}{\input{figures/05_examples_chapter/set_bin_plot.pgf}}    
    \caption{Example of the binning procedure used in the Study for \ac{snf} +
    \ac{hlw} \cite{wigeland_nuclear_2014-1}.}
    \label{fig:bin-plot}
\end{figure}

\FloatBarrier

\subsection{Limitations of the \ac{set}} Despite its status as the most
comprehensive available tool for evaluating fuel cycles in the Study, the
\ac{set} has some limitations primarily with the methods the Study used for
drawing conclusions. First, the \ac{set} does not consider Pareto optimality.
Users are able to adjust the weights for different objectives to obtain
different results, but the full space of options is obfuscated from users.
Second, the Study's authors binned data on two occasions, first to generate the
evaluation groups and a second time to make the analysis more tractable. The
\ac{set} calculates \ac{eg} performance based on these secondary bins, as shown
in Figure \ref{fig:bin-plot}. This exacerbates differences among \acp{eg} in
different bins and eliminates differences among \acp{eg} in the same bin.
Further challenging the comparison of the \acp{eg}. Lastly, the \ac{set} relies
totally on expert input. This is appropriate for generating data for each
\ac{eg} or even each fuel cycle option. It is unreasonable to expect
non-technical stakeholders to opine on which isotopes should be separated, or
whether a reactor should have a fast or thermal spectrum. But such stakeholders
can certainly weigh-in on the tradeoffs and prioritization of different metrics
in the Study based on the values and priorities guiding their choice. This
example with \ac{osier} addresses these gaps by using raw data from the Study to
identify Pareto optimal solutions and present them in a manner that could be
used for deliberation with various stakeholders. 

\subsection{\ac{osier} Methodology and Data}

The data for this simulation were pulled directly from the Study and its
appendices
\cite{wigeland_nuclear_2014,wigeland_nuclear_2014-2,wigeland_nuclear_2014-1}.
Table \ref{tab:metric-data} describes the data for the simulation. These data
were used to identify Pareto optimal solutions.  Importantly, \ac{osier} only
minimizes objectives, so a few metrics were adjusted from the \ac{set}. For
example, the maximization of ``compatibility'' with existing infrastructure was
inverted to minimize \textit{incompatibility} with existing infrastructure.
Lastly, since decision makers are frequently interested in ``compromise''
solutions a ``high tradeoff point'' was identified from the resulting Pareto
front by calculating the \ac{eg} that minimizes a tradeoff metric given by
Equation \ref{eqn:tradeoff} \cite{rachmawati_multiobjective_2009}. The \ac{mcda}
literature refers sometimes refers to these solutions as ``knee'' solutions
\cite{rachmawati_multiobjective_2009}.

\begin{align}
    \label{eqn:tradeoff}
    T\left(X_i, X_j\right) &= \frac{\sum_{1}^{M}\text{max}\left[0, \left(f_m\left(X_j\right) - f_m\left(X_i\right)\right]\right)}
    {\sum_{1}^{M}\text{max}\left[0, -\left(f_m\left(X_j\right) - f_m\left(X_i\right)\right)\right]},
    \intertext{where}
    T &= \text{the tradeoff between the i-th and j-th solutions},\nonumber\\
    X_i &= \text{the i-th solution vector}, \nonumber\\
    f_m &= \text{the m-th objective function}, \nonumber\\
    M &= \text{the total number of objectives}. \nonumber
\end{align}
\noindent
To prevent any metric from dominating the solutions, all data are normalized
with the $L_\infty$-norm. Due to the implicit assumption that all fuel cycles
and evaluation groups are mutually exclusive, the complete set of solutions in
design space is given by the identity matrix of size $40 \times 40$. Since the
design space is known \textit{a priori}, the genetic algorithm component of
\ac{osier} was skipped and the results are simply the non-dominated set (i.e.,
the Pareto front) as calculated by \ac{osier}. 

\begin{sidewaystable}[htbp!]
    \centering
    \caption{Raw data from the \ac{set} extracted from tables in the Study
    document used in the \ac{osier} simulation \cite{wigeland_nuclear_2014-1}.}
    \label{tab:metric-data}
    \resizebox*{\textwidth}{!}{\input{tables/metric_data_manual.tex}}
\end{sidewaystable}

\FloatBarrier

\subsection{Results}
The primary result for this example is the identification of a Pareto front for
the different \acp{eg} considered in the \ac{set}. Figure
\ref{fig:full-set-space} shows the resultant Pareto front.

\begin{figure}[htbp!]
  \centering
  \resizebox{0.85\columnwidth}{!}{\input{figures/05_examples_chapter/full_set_plot.pgf}}
  \caption{The Pareto front for the \ac{set}.}
  \label{fig:full-set-space}
\end{figure}

The solutions are colored based on the level of recycling present in the
representative fuel cycle. Red, green, and blue solutions correspond to
once-through, limited-recycle, and continuous-recycle fuel cycles, respectively.
By inspection, the limited- and continuous-recycle \acp{eg} perform similarly on
mass of \ac{hlw} requiring disposal, activity of \ac{snf} and \ac{hlw} at
100,000 years, water use, safety challenges, and natural thorium requirements.
The \acp{eg} with recycling have a larger spread over the activity at 100 years,
land use, and disposed mass of \ac{du}. This is because these fuel cycles vary
by degree of recycling and isotope separation. The once-through fuel cycles vary
widely in performance across all objectives. There is an apparent ``lower''
density of results in the latter five objectives: development cost, development
time, \ac{foak} cost, incompatibility, and unfamiliarity. This is because these
objectives were based on categorical variables rather than continuous ones. The
low density of solutions is the result of overlapping solutions.

% Figure \ref{fig:once-through-set-space} highlights the evaluation groups
% representing once-through fuel cycles.

% \begin{figure}[htbp!] \centering
%   \resizebox{0.8\columnwidth}{!}{\input{figures/05_examples_chapter/once-through_set_plot.pgf}}
%   \caption{The \ac{set} Pareto front with once-through fuel cycles
%   highlighted.} \label{fig:once-through-set-space} \end{figure}

Next, Figure \ref{fig:single-eg-set-space} identifies a tradeoff minimizing
``knee'' solution. This solution represents a once-through fuel cycle using
natural uranium fuel in fast critical reactors to very high burnup
\cite{wigeland_nuclear_2014-2}. EG04 performs well on several quantitative
metrics and middlingly on the metrics associated with institutional challenges
and development costs.

\begin{figure}[htbp!]
  \centering
  \resizebox{0.85\columnwidth}{!}{\input{figures/05_examples_chapter/single-eg_set_plot.pgf}}
  \caption{A high tradeoff ``knee'' solution from the \acp{eg} in the \ac{set}.}
  \label{fig:single-eg-set-space}
\end{figure}

\FloatBarrier
% \begin{figure}[htbp!] \centering
%   \resizebox{0.8\columnwidth}{!}{\input{figures/05_examples_chapter/non_optimal_set_plot.pgf}}
%   \caption{The set of non-Pareto optimal solutions given by the \ac{set}.}
%   \label{fig:non-optimal-eg-set-space} \end{figure}

\noindent
Finally, Table \ref{tab:summary-data} summarizes the results from the \ac{set}
and \ac{osier} comparison. Highlighted rows indicate places where the \ac{set}
and \ac{osier} disagree while the color corresponds to the nature of the
disagreement. Orange rows are Pareto optimal solutions, as identified by
\ac{osier}, that the Study did not designate as promising --- reinterpreted here
as ``not promising.'' Yellow rows are those \acp{eg} designated as ``most
promising'' or ``potentially promising'' by the Study that \ac{osier} found to
be sub-optimal. The solitary green row highlights that the Study categorized
EG04 as ``less promising'' while \ac{osier} identified it as the only ``knee''
solution which suggests a higher qualitative ranking (e.g., ``most'' or
``potentially'' promising). 

\begin{table}[htbp!]
    \centering
    \caption{Summary of \ac{set} and \ac{osier} data. Highlighted rows indicate
    disagreement between \ac{osier} and \ac{set} results.}
    \label{tab:summary-data}
    \resizebox*{\textwidth}{!}{\input{tables/summary_data_highlighted.tex}}
\end{table}

\FloatBarrier

\subsection{Discussion}

\ac{osier} and the Study disagree on precisely half of the \acp{eg}'
categorization. Although the Study rankings are more granular than whether a
solution is Pareto optimal (a binary), its recommendations elevated several
\acp{eg} that \ac{osier} found to be sub-optimal, while dismissing several
Pareto optimal ones. Solutions on the Pareto front are all considered
co-optimal, by definition. Differences emerge when \ac{mcda} tools are applied
based on inputs and preferences from decision makers and/or stakeholders. The
most likely source of disagreement between \ac{osier} and the \ac{set} is the
secondary binning process employed by the Study and shown in Figure
\ref{fig:bin-plot}. While the \acp{eg} are ``bins'' of fuel cycles sharing
similar physical characteristics, further binning the data into smaller
sub-groups results in a loss of information. The second source of disagreement
is possibly from the dependence on expert judgement. Table
\ref{tab:summary-data} evinces a bias from the \ac{set} and the Study in favor
of closed fuel cycles.

Finally, the data from the Study and the \ac{set} are static. If future studies
wish to evaluate different metrics associated with these fuel cycles or update
the data based on new information, new fuel cycle simulations must be run. The
future work section, Section \ref{section:limitations-future-work}, describes
how \ac{osier} could be used in conjunction with other codes to address this
issue. 