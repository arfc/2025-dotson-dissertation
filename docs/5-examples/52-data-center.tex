\section{Example 2: Powering a Data Center}

\subsection{Why data centers?}

\subsection{What technology options exist?}
There are myriad technology options available to power a data center. For simplicity, these choices were restricted
to 
\begin{itemize}
    \item natural gas,
    \item ``advanced'' natural gas with \ac{ccs},
    \item ``large'' nuclear (e.g., an AP1000),
    \item advanced nuclear (e.g., an \ac{sfr}),
    \item solar panels,
    \item 4-hour battery storage.
\end{itemize}

Table \ref{tab:dc-tech-options} summarizes the data for the different technology options. The
table was generated automatically using \ac{osier}'s built-in \texttt{technology\_dataframe} method. 
\begin{table}[htpb!]
    \centering
    \caption{Summary of technology data used in the datacenter example.}
    \label{tab:dc-tech-options}
    \resizebox{\columnwidth}{!}{\begin{tabular}{lllllll}
\toprule
 & NaturalGas Conv & NaturalGas Adv & SolarPanel & Battery & Nuclear & Nuclear Adv \\
\midrule
capital cost (1/kW) & 0.00165 & 0.00292 & 0.00107 & 0.00141 & 0.00749 & 0.00894 \\
om cost fixed (1/kW) & 3.7e-05 & 6.61e-05 & 1.68e-05 & 3.7e-05 & 0.000175 & 0.000136 \\
om cost variable (1/MWh) & 2.39e-06 & 4.62e-06 & 0 & 0 & 2.8e-06 & 2.6e-06 \\
fuel cost (1/(MW*hr)) & 2.24e-05 & 2.75e-05 & 0 & 0 & 5.81e-06 & 9.16e-06 \\
co2 rate (megatonnes/(MW*hr)) & 4.35e-07 & 2.18e-08 & 0 & 0 & 0 & 0 \\
eroi & 89 & 12.4 & 2.5 & 10 & 96.2 & 96.2 \\
\bottomrule
\end{tabular}
}
\end{table}

Note that the ``cost'' variables all have units of ``millions of dollars'' (M\$) in the numerator. \ac{eroi}
is a unitless quantity.
 
\FloatBarrier

\subsection{Data for the simulation}
\subsection{Objectives}
This example optimizes three objectives: total system cost (M\$), total carbon emissions (megatonnes), and \ac{eroi}.
\ac{eroi} is defined by 

\subsubsection{Synthetic Solar Data}

The solar availability curve was generated by combining a sinusoidal curve with some random noise and then forcing negative
values to zero. The formula used for this is given by Equation \ref{eqn:synthetic-solar}.

\begin{align}
    S(t) &= \textbf{max}\left(0, - \sin\left(\frac{2\pi t}{N} + \phi\right) + \chi\right)
    \label{eqn:synthetic-solar}
    \intertext{where}
    t &= \text{time} \quad \left[\text{hours}\right],\nonumber\\
    N &= \text{the total number of hours,}\nonumber\\
    \phi &= \text{a phase shift of $\frac{\pi}{2}$,}\nonumber\\
    \chi &= \text{a normally distributed random variable.}\nonumber
\end{align}

Figure \ref{fig:synthetic-solar} illustrates the synthetic solar data.

\begin{figure}
    \centering
    \resizebox{0.65\columnwidth}{!}{\input{}}
    \caption{Synthetic solar data for a period of a week.}
\end{figure}

The data for this simulation are provided by this table from Walmsley et al. 2018 \cite{walmsley_energy_2018}. This method
accounts for the time value of energy using an energy value accounting method. I also selected the \acs{eroi}$_{std}$ for the simulation
which accounts for the self-use of inputs by the generating technology.

\textcolor{red}{I chose monocrystalline-silicone technology for solar panels since this technology was used \ac{nrel}'s \ac{atb}.}

\subsection{Results}

\subsection{Discussion}
