\section{Example 2: Powering a Data Center}

\subsection{Why data centers?}

\subsection{Technology Options}
There are myriad technology options available to power a data center. For
simplicity, these choices were restricted to 
\begin{itemize}
    \item natural gas,
    \item ``advanced'' natural gas with \ac{ccs},
    \item ``large'' nuclear (e.g., an AP1000),
    \item advanced nuclear (e.g., an \ac{sfr}),
    \item solar panels (monocrystalline-silicone for consistency with \ac{nrel}'s \ac{atb}),
    \item 4-hour battery storage (lithium-ion).
\end{itemize}

\noindent
These technologies were chosen for their potential to be co-located with a datacenter. There is
a miniscule potential for new hydropower in the United States \cite{lopez_us_2012}, geothermal
requires geological activity that is similarly hard to find \cite{lopez_us_2012}, and wind energy
requires a vast amount of land.
Table \ref{tab:dc-tech-options} summarizes the data for the different technology
options. The table was generated automatically using \ac{osier}'s built-in
\texttt{technology\_dataframe} method. 
\begin{table}[htpb!]
    \centering
    \caption{Summary of technology data used in the datacenter example.}
    \label{tab:dc-tech-options}
    \resizebox{\columnwidth}{!}{\begin{tabular}{lllllll}
\toprule
 & NaturalGas Conv & NaturalGas Adv & SolarPanel & Battery & Nuclear & Nuclear Adv \\
\midrule
capital cost (1/kW) \cite{national_renewable_energy_laboratory_2023_2023}& 0.00165 & 0.00292 & 0.00107 & 0.00141 & 0.00749 & 0.00894 \\
om cost fixed (1/kW) \cite{national_renewable_energy_laboratory_2023_2023}& 3.7e-05 & 6.61e-05 & 1.68e-05 & 3.7e-05 & 0.000175 & 0.000136 \\
om cost variable (1/MWh) \cite{national_renewable_energy_laboratory_2023_2023}& 2.39e-06 & 4.62e-06 & 0 & 0 & 2.8e-06 & 2.6e-06 \\
fuel cost (1/(MW*hr))  \cite{national_renewable_energy_laboratory_2023_2023,desai_nuclear_2020}& 2.24e-05 & 2.75e-05 & 0 & 0 & 5.81e-06 & 9.16e-06 \\
co2 rate (megatonnes/(MW*hr)) \cite{energy_information_administration_how_2024}& 4.35e-07 & 2.18e-08 & 0 & 0 & 0 & 0 \\
eroi \cite{walmsley_energy_2018} & 89 & 12.4 & 2.5 & 10 & 96.2 & 96.2 \\
\bottomrule
\end{tabular}
}
\end{table}
\noindent
Note that the ``cost'' variables all have units of ``millions of dollars'' (M\$)
in the numerator. \ac{eroi} is a unitless quantity. The \ac{eroi} data come from
Walmsley et al. 2018 \cite{walmsley_energy_2018}.
 
\FloatBarrier

\subsection{Objectives}
This example optimizes three objectives: total system cost (M\$), total carbon
emissions (megatonnes), and \ac{eroi}. \ac{eroi} is defined by Equation
\ref{eqn:eroi-std} \cite{walmsley_energy_2018}.

\begin{align}
    EROI_{std} &= \frac{\dot E_{net}}{\dot E_{\text{om}} + \dot E_{\text{con}} + \dot E_{\text{dec}} + \dot E_{\text{is}} + \dot E_{\text{os}}},
    \label{eqn:eroi-std}
    \intertext{where}
    \dot E &= \text{energy flow} \quad \left[\frac{\text{Gj}}{\text{y}}\right]\nonumber,\\
    \text{net} &= \text{net generation},\nonumber\\
    \text{om} &= \text{operation and management},\nonumber\\
    \text{con} &= \text{construction},\nonumber\\
    \text{dec} &= \text{deconstruction},\nonumber\\
    \text{is} &= \text{inflow self-use},\nonumber\\
    \text{os} &= \text{outflow self-use}.\nonumber
\end{align}
\noindent
Typically, \ac{eroi} should be maximized. However, \ac{osier} only minimizes
objectives, so this example minimizes the inverse capacity-weighted \ac{eroi} as
shown by Equation \ref{eqn:eroi-obj}.

\begin{align}
    \textbf{Minimize}\left(\frac{1}{\sum^{G}\textbf{CAP}_gEROI_{std}^g}\right)
    \label{eqn:eroi-obj}
\end{align}

\subsection{Demand Data}
The electricity demand for this example was modeled after a stylized
``high-availability'' gigawatt-scale data center. For simplicity, this example 
considers ``high-availability'' equivalent to a constant power draw. Further, 
gigawatt-scale data center is reasonable given recent announcements for a 
two gigawatt data center in Louisiana \cite{swinhoe_meta_2024} and multiple
\acp{ppa} with existing and previously retired nuclear reactors, each capable of
producing around a gigawatt of electric power \cite{shaw_microsoft_2024,constellation_energy_constellation_2025}.

\subsection{Synthetic Solar Data}

The solar availability curve was generated by combining a sinusoidal curve with
some random noise and then forcing negative values to zero. The formula used for
this is given by Equation \ref{eqn:synthetic-solar}.

\begin{align}
    S(t) &= \textbf{max}\left(0, - \sin\left(\frac{2\pi t}{N} + \phi\right) + \chi\right)
    \label{eqn:synthetic-solar}
    \intertext{where}
    t &= \text{time} \quad \left[\text{hours}\right],\nonumber\\
    N &= \text{the total number of hours,}\nonumber\\
    \phi &= \text{a phase shift of $\frac{\pi}{2}$,}\nonumber\\
    \chi &= \text{a normally distributed random variable.}\nonumber
\end{align}
\noindent
Figure \ref{fig:synthetic-solar} illustrates the synthetic solar data.

\begin{figure}[htbp!]
    \centering
    \resizebox{0.65\columnwidth}{!}{%% Creator: Matplotlib, PGF backend
%%
%% To include the figure in your LaTeX document, write
%%   \input{<filename>.pgf}
%%
%% Make sure the required packages are loaded in your preamble
%%   \usepackage{pgf}
%%
%% Also ensure that all the required font packages are loaded; for instance,
%% the lmodern package is sometimes necessary when using math font.
%%   \usepackage{lmodern}
%%
%% Figures using additional raster images can only be included by \input if
%% they are in the same directory as the main LaTeX file. For loading figures
%% from other directories you can use the `import` package
%%   \usepackage{import}
%%
%% and then include the figures with
%%   \import{<path to file>}{<filename>.pgf}
%%
%% Matplotlib used the following preamble
%%   \def\mathdefault#1{#1}
%%   \everymath=\expandafter{\the\everymath\displaystyle}
%%   \IfFileExists{scrextend.sty}{
%%     \usepackage[fontsize=10.000000pt]{scrextend}
%%   }{
%%     \renewcommand{\normalsize}{\fontsize{10.000000}{12.000000}\selectfont}
%%     \normalsize
%%   }
%%   
%%   \makeatletter\@ifpackageloaded{underscore}{}{\usepackage[strings]{underscore}}\makeatother
%%
\begingroup%
\makeatletter%
\begin{pgfpicture}%
\pgfpathrectangle{\pgfpointorigin}{\pgfqpoint{5.947071in}{4.466138in}}%
\pgfusepath{use as bounding box, clip}%
\begin{pgfscope}%
\pgfsetbuttcap%
\pgfsetmiterjoin%
\definecolor{currentfill}{rgb}{1.000000,1.000000,1.000000}%
\pgfsetfillcolor{currentfill}%
\pgfsetlinewidth{0.000000pt}%
\definecolor{currentstroke}{rgb}{0.000000,0.000000,0.000000}%
\pgfsetstrokecolor{currentstroke}%
\pgfsetdash{}{0pt}%
\pgfpathmoveto{\pgfqpoint{0.000000in}{0.000000in}}%
\pgfpathlineto{\pgfqpoint{5.947071in}{0.000000in}}%
\pgfpathlineto{\pgfqpoint{5.947071in}{4.466138in}}%
\pgfpathlineto{\pgfqpoint{0.000000in}{4.466138in}}%
\pgfpathlineto{\pgfqpoint{0.000000in}{0.000000in}}%
\pgfpathclose%
\pgfusepath{fill}%
\end{pgfscope}%
\begin{pgfscope}%
\pgfsetbuttcap%
\pgfsetmiterjoin%
\definecolor{currentfill}{rgb}{1.000000,1.000000,1.000000}%
\pgfsetfillcolor{currentfill}%
\pgfsetlinewidth{0.000000pt}%
\definecolor{currentstroke}{rgb}{0.000000,0.000000,0.000000}%
\pgfsetstrokecolor{currentstroke}%
\pgfsetstrokeopacity{0.000000}%
\pgfsetdash{}{0pt}%
\pgfpathmoveto{\pgfqpoint{0.777774in}{0.670138in}}%
\pgfpathlineto{\pgfqpoint{5.737774in}{0.670138in}}%
\pgfpathlineto{\pgfqpoint{5.737774in}{4.366138in}}%
\pgfpathlineto{\pgfqpoint{0.777774in}{4.366138in}}%
\pgfpathlineto{\pgfqpoint{0.777774in}{0.670138in}}%
\pgfpathclose%
\pgfusepath{fill}%
\end{pgfscope}%
\begin{pgfscope}%
\pgfpathrectangle{\pgfqpoint{0.777774in}{0.670138in}}{\pgfqpoint{4.960000in}{3.696000in}}%
\pgfusepath{clip}%
\pgfsetrectcap%
\pgfsetroundjoin%
\pgfsetlinewidth{0.803000pt}%
\definecolor{currentstroke}{rgb}{0.690196,0.690196,0.690196}%
\pgfsetstrokecolor{currentstroke}%
\pgfsetstrokeopacity{0.200000}%
\pgfsetdash{}{0pt}%
\pgfpathmoveto{\pgfqpoint{1.003229in}{0.670138in}}%
\pgfpathlineto{\pgfqpoint{1.003229in}{4.366138in}}%
\pgfusepath{stroke}%
\end{pgfscope}%
\begin{pgfscope}%
\pgfsetbuttcap%
\pgfsetroundjoin%
\definecolor{currentfill}{rgb}{0.000000,0.000000,0.000000}%
\pgfsetfillcolor{currentfill}%
\pgfsetlinewidth{0.803000pt}%
\definecolor{currentstroke}{rgb}{0.000000,0.000000,0.000000}%
\pgfsetstrokecolor{currentstroke}%
\pgfsetdash{}{0pt}%
\pgfsys@defobject{currentmarker}{\pgfqpoint{0.000000in}{-0.048611in}}{\pgfqpoint{0.000000in}{0.000000in}}{%
\pgfpathmoveto{\pgfqpoint{0.000000in}{0.000000in}}%
\pgfpathlineto{\pgfqpoint{0.000000in}{-0.048611in}}%
\pgfusepath{stroke,fill}%
}%
\begin{pgfscope}%
\pgfsys@transformshift{1.003229in}{0.670138in}%
\pgfsys@useobject{currentmarker}{}%
\end{pgfscope}%
\end{pgfscope}%
\begin{pgfscope}%
\definecolor{textcolor}{rgb}{0.000000,0.000000,0.000000}%
\pgfsetstrokecolor{textcolor}%
\pgfsetfillcolor{textcolor}%
\pgftext[x=1.003229in,y=0.572916in,,top]{\color{textcolor}{\rmfamily\fontsize{14.000000}{16.800000}\selectfont\catcode`\^=\active\def^{\ifmmode\sp\else\^{}\fi}\catcode`\%=\active\def%{\%}$\mathdefault{0}$}}%
\end{pgfscope}%
\begin{pgfscope}%
\pgfpathrectangle{\pgfqpoint{0.777774in}{0.670138in}}{\pgfqpoint{4.960000in}{3.696000in}}%
\pgfusepath{clip}%
\pgfsetrectcap%
\pgfsetroundjoin%
\pgfsetlinewidth{0.803000pt}%
\definecolor{currentstroke}{rgb}{0.690196,0.690196,0.690196}%
\pgfsetstrokecolor{currentstroke}%
\pgfsetstrokeopacity{0.200000}%
\pgfsetdash{}{0pt}%
\pgfpathmoveto{\pgfqpoint{1.674224in}{0.670138in}}%
\pgfpathlineto{\pgfqpoint{1.674224in}{4.366138in}}%
\pgfusepath{stroke}%
\end{pgfscope}%
\begin{pgfscope}%
\pgfsetbuttcap%
\pgfsetroundjoin%
\definecolor{currentfill}{rgb}{0.000000,0.000000,0.000000}%
\pgfsetfillcolor{currentfill}%
\pgfsetlinewidth{0.803000pt}%
\definecolor{currentstroke}{rgb}{0.000000,0.000000,0.000000}%
\pgfsetstrokecolor{currentstroke}%
\pgfsetdash{}{0pt}%
\pgfsys@defobject{currentmarker}{\pgfqpoint{0.000000in}{-0.048611in}}{\pgfqpoint{0.000000in}{0.000000in}}{%
\pgfpathmoveto{\pgfqpoint{0.000000in}{0.000000in}}%
\pgfpathlineto{\pgfqpoint{0.000000in}{-0.048611in}}%
\pgfusepath{stroke,fill}%
}%
\begin{pgfscope}%
\pgfsys@transformshift{1.674224in}{0.670138in}%
\pgfsys@useobject{currentmarker}{}%
\end{pgfscope}%
\end{pgfscope}%
\begin{pgfscope}%
\definecolor{textcolor}{rgb}{0.000000,0.000000,0.000000}%
\pgfsetstrokecolor{textcolor}%
\pgfsetfillcolor{textcolor}%
\pgftext[x=1.674224in,y=0.572916in,,top]{\color{textcolor}{\rmfamily\fontsize{14.000000}{16.800000}\selectfont\catcode`\^=\active\def^{\ifmmode\sp\else\^{}\fi}\catcode`\%=\active\def%{\%}$\mathdefault{25}$}}%
\end{pgfscope}%
\begin{pgfscope}%
\pgfpathrectangle{\pgfqpoint{0.777774in}{0.670138in}}{\pgfqpoint{4.960000in}{3.696000in}}%
\pgfusepath{clip}%
\pgfsetrectcap%
\pgfsetroundjoin%
\pgfsetlinewidth{0.803000pt}%
\definecolor{currentstroke}{rgb}{0.690196,0.690196,0.690196}%
\pgfsetstrokecolor{currentstroke}%
\pgfsetstrokeopacity{0.200000}%
\pgfsetdash{}{0pt}%
\pgfpathmoveto{\pgfqpoint{2.345220in}{0.670138in}}%
\pgfpathlineto{\pgfqpoint{2.345220in}{4.366138in}}%
\pgfusepath{stroke}%
\end{pgfscope}%
\begin{pgfscope}%
\pgfsetbuttcap%
\pgfsetroundjoin%
\definecolor{currentfill}{rgb}{0.000000,0.000000,0.000000}%
\pgfsetfillcolor{currentfill}%
\pgfsetlinewidth{0.803000pt}%
\definecolor{currentstroke}{rgb}{0.000000,0.000000,0.000000}%
\pgfsetstrokecolor{currentstroke}%
\pgfsetdash{}{0pt}%
\pgfsys@defobject{currentmarker}{\pgfqpoint{0.000000in}{-0.048611in}}{\pgfqpoint{0.000000in}{0.000000in}}{%
\pgfpathmoveto{\pgfqpoint{0.000000in}{0.000000in}}%
\pgfpathlineto{\pgfqpoint{0.000000in}{-0.048611in}}%
\pgfusepath{stroke,fill}%
}%
\begin{pgfscope}%
\pgfsys@transformshift{2.345220in}{0.670138in}%
\pgfsys@useobject{currentmarker}{}%
\end{pgfscope}%
\end{pgfscope}%
\begin{pgfscope}%
\definecolor{textcolor}{rgb}{0.000000,0.000000,0.000000}%
\pgfsetstrokecolor{textcolor}%
\pgfsetfillcolor{textcolor}%
\pgftext[x=2.345220in,y=0.572916in,,top]{\color{textcolor}{\rmfamily\fontsize{14.000000}{16.800000}\selectfont\catcode`\^=\active\def^{\ifmmode\sp\else\^{}\fi}\catcode`\%=\active\def%{\%}$\mathdefault{50}$}}%
\end{pgfscope}%
\begin{pgfscope}%
\pgfpathrectangle{\pgfqpoint{0.777774in}{0.670138in}}{\pgfqpoint{4.960000in}{3.696000in}}%
\pgfusepath{clip}%
\pgfsetrectcap%
\pgfsetroundjoin%
\pgfsetlinewidth{0.803000pt}%
\definecolor{currentstroke}{rgb}{0.690196,0.690196,0.690196}%
\pgfsetstrokecolor{currentstroke}%
\pgfsetstrokeopacity{0.200000}%
\pgfsetdash{}{0pt}%
\pgfpathmoveto{\pgfqpoint{3.016216in}{0.670138in}}%
\pgfpathlineto{\pgfqpoint{3.016216in}{4.366138in}}%
\pgfusepath{stroke}%
\end{pgfscope}%
\begin{pgfscope}%
\pgfsetbuttcap%
\pgfsetroundjoin%
\definecolor{currentfill}{rgb}{0.000000,0.000000,0.000000}%
\pgfsetfillcolor{currentfill}%
\pgfsetlinewidth{0.803000pt}%
\definecolor{currentstroke}{rgb}{0.000000,0.000000,0.000000}%
\pgfsetstrokecolor{currentstroke}%
\pgfsetdash{}{0pt}%
\pgfsys@defobject{currentmarker}{\pgfqpoint{0.000000in}{-0.048611in}}{\pgfqpoint{0.000000in}{0.000000in}}{%
\pgfpathmoveto{\pgfqpoint{0.000000in}{0.000000in}}%
\pgfpathlineto{\pgfqpoint{0.000000in}{-0.048611in}}%
\pgfusepath{stroke,fill}%
}%
\begin{pgfscope}%
\pgfsys@transformshift{3.016216in}{0.670138in}%
\pgfsys@useobject{currentmarker}{}%
\end{pgfscope}%
\end{pgfscope}%
\begin{pgfscope}%
\definecolor{textcolor}{rgb}{0.000000,0.000000,0.000000}%
\pgfsetstrokecolor{textcolor}%
\pgfsetfillcolor{textcolor}%
\pgftext[x=3.016216in,y=0.572916in,,top]{\color{textcolor}{\rmfamily\fontsize{14.000000}{16.800000}\selectfont\catcode`\^=\active\def^{\ifmmode\sp\else\^{}\fi}\catcode`\%=\active\def%{\%}$\mathdefault{75}$}}%
\end{pgfscope}%
\begin{pgfscope}%
\pgfpathrectangle{\pgfqpoint{0.777774in}{0.670138in}}{\pgfqpoint{4.960000in}{3.696000in}}%
\pgfusepath{clip}%
\pgfsetrectcap%
\pgfsetroundjoin%
\pgfsetlinewidth{0.803000pt}%
\definecolor{currentstroke}{rgb}{0.690196,0.690196,0.690196}%
\pgfsetstrokecolor{currentstroke}%
\pgfsetstrokeopacity{0.200000}%
\pgfsetdash{}{0pt}%
\pgfpathmoveto{\pgfqpoint{3.687211in}{0.670138in}}%
\pgfpathlineto{\pgfqpoint{3.687211in}{4.366138in}}%
\pgfusepath{stroke}%
\end{pgfscope}%
\begin{pgfscope}%
\pgfsetbuttcap%
\pgfsetroundjoin%
\definecolor{currentfill}{rgb}{0.000000,0.000000,0.000000}%
\pgfsetfillcolor{currentfill}%
\pgfsetlinewidth{0.803000pt}%
\definecolor{currentstroke}{rgb}{0.000000,0.000000,0.000000}%
\pgfsetstrokecolor{currentstroke}%
\pgfsetdash{}{0pt}%
\pgfsys@defobject{currentmarker}{\pgfqpoint{0.000000in}{-0.048611in}}{\pgfqpoint{0.000000in}{0.000000in}}{%
\pgfpathmoveto{\pgfqpoint{0.000000in}{0.000000in}}%
\pgfpathlineto{\pgfqpoint{0.000000in}{-0.048611in}}%
\pgfusepath{stroke,fill}%
}%
\begin{pgfscope}%
\pgfsys@transformshift{3.687211in}{0.670138in}%
\pgfsys@useobject{currentmarker}{}%
\end{pgfscope}%
\end{pgfscope}%
\begin{pgfscope}%
\definecolor{textcolor}{rgb}{0.000000,0.000000,0.000000}%
\pgfsetstrokecolor{textcolor}%
\pgfsetfillcolor{textcolor}%
\pgftext[x=3.687211in,y=0.572916in,,top]{\color{textcolor}{\rmfamily\fontsize{14.000000}{16.800000}\selectfont\catcode`\^=\active\def^{\ifmmode\sp\else\^{}\fi}\catcode`\%=\active\def%{\%}$\mathdefault{100}$}}%
\end{pgfscope}%
\begin{pgfscope}%
\pgfpathrectangle{\pgfqpoint{0.777774in}{0.670138in}}{\pgfqpoint{4.960000in}{3.696000in}}%
\pgfusepath{clip}%
\pgfsetrectcap%
\pgfsetroundjoin%
\pgfsetlinewidth{0.803000pt}%
\definecolor{currentstroke}{rgb}{0.690196,0.690196,0.690196}%
\pgfsetstrokecolor{currentstroke}%
\pgfsetstrokeopacity{0.200000}%
\pgfsetdash{}{0pt}%
\pgfpathmoveto{\pgfqpoint{4.358207in}{0.670138in}}%
\pgfpathlineto{\pgfqpoint{4.358207in}{4.366138in}}%
\pgfusepath{stroke}%
\end{pgfscope}%
\begin{pgfscope}%
\pgfsetbuttcap%
\pgfsetroundjoin%
\definecolor{currentfill}{rgb}{0.000000,0.000000,0.000000}%
\pgfsetfillcolor{currentfill}%
\pgfsetlinewidth{0.803000pt}%
\definecolor{currentstroke}{rgb}{0.000000,0.000000,0.000000}%
\pgfsetstrokecolor{currentstroke}%
\pgfsetdash{}{0pt}%
\pgfsys@defobject{currentmarker}{\pgfqpoint{0.000000in}{-0.048611in}}{\pgfqpoint{0.000000in}{0.000000in}}{%
\pgfpathmoveto{\pgfqpoint{0.000000in}{0.000000in}}%
\pgfpathlineto{\pgfqpoint{0.000000in}{-0.048611in}}%
\pgfusepath{stroke,fill}%
}%
\begin{pgfscope}%
\pgfsys@transformshift{4.358207in}{0.670138in}%
\pgfsys@useobject{currentmarker}{}%
\end{pgfscope}%
\end{pgfscope}%
\begin{pgfscope}%
\definecolor{textcolor}{rgb}{0.000000,0.000000,0.000000}%
\pgfsetstrokecolor{textcolor}%
\pgfsetfillcolor{textcolor}%
\pgftext[x=4.358207in,y=0.572916in,,top]{\color{textcolor}{\rmfamily\fontsize{14.000000}{16.800000}\selectfont\catcode`\^=\active\def^{\ifmmode\sp\else\^{}\fi}\catcode`\%=\active\def%{\%}$\mathdefault{125}$}}%
\end{pgfscope}%
\begin{pgfscope}%
\pgfpathrectangle{\pgfqpoint{0.777774in}{0.670138in}}{\pgfqpoint{4.960000in}{3.696000in}}%
\pgfusepath{clip}%
\pgfsetrectcap%
\pgfsetroundjoin%
\pgfsetlinewidth{0.803000pt}%
\definecolor{currentstroke}{rgb}{0.690196,0.690196,0.690196}%
\pgfsetstrokecolor{currentstroke}%
\pgfsetstrokeopacity{0.200000}%
\pgfsetdash{}{0pt}%
\pgfpathmoveto{\pgfqpoint{5.029203in}{0.670138in}}%
\pgfpathlineto{\pgfqpoint{5.029203in}{4.366138in}}%
\pgfusepath{stroke}%
\end{pgfscope}%
\begin{pgfscope}%
\pgfsetbuttcap%
\pgfsetroundjoin%
\definecolor{currentfill}{rgb}{0.000000,0.000000,0.000000}%
\pgfsetfillcolor{currentfill}%
\pgfsetlinewidth{0.803000pt}%
\definecolor{currentstroke}{rgb}{0.000000,0.000000,0.000000}%
\pgfsetstrokecolor{currentstroke}%
\pgfsetdash{}{0pt}%
\pgfsys@defobject{currentmarker}{\pgfqpoint{0.000000in}{-0.048611in}}{\pgfqpoint{0.000000in}{0.000000in}}{%
\pgfpathmoveto{\pgfqpoint{0.000000in}{0.000000in}}%
\pgfpathlineto{\pgfqpoint{0.000000in}{-0.048611in}}%
\pgfusepath{stroke,fill}%
}%
\begin{pgfscope}%
\pgfsys@transformshift{5.029203in}{0.670138in}%
\pgfsys@useobject{currentmarker}{}%
\end{pgfscope}%
\end{pgfscope}%
\begin{pgfscope}%
\definecolor{textcolor}{rgb}{0.000000,0.000000,0.000000}%
\pgfsetstrokecolor{textcolor}%
\pgfsetfillcolor{textcolor}%
\pgftext[x=5.029203in,y=0.572916in,,top]{\color{textcolor}{\rmfamily\fontsize{14.000000}{16.800000}\selectfont\catcode`\^=\active\def^{\ifmmode\sp\else\^{}\fi}\catcode`\%=\active\def%{\%}$\mathdefault{150}$}}%
\end{pgfscope}%
\begin{pgfscope}%
\pgfpathrectangle{\pgfqpoint{0.777774in}{0.670138in}}{\pgfqpoint{4.960000in}{3.696000in}}%
\pgfusepath{clip}%
\pgfsetrectcap%
\pgfsetroundjoin%
\pgfsetlinewidth{0.803000pt}%
\definecolor{currentstroke}{rgb}{0.690196,0.690196,0.690196}%
\pgfsetstrokecolor{currentstroke}%
\pgfsetstrokeopacity{0.200000}%
\pgfsetdash{}{0pt}%
\pgfpathmoveto{\pgfqpoint{5.700198in}{0.670138in}}%
\pgfpathlineto{\pgfqpoint{5.700198in}{4.366138in}}%
\pgfusepath{stroke}%
\end{pgfscope}%
\begin{pgfscope}%
\pgfsetbuttcap%
\pgfsetroundjoin%
\definecolor{currentfill}{rgb}{0.000000,0.000000,0.000000}%
\pgfsetfillcolor{currentfill}%
\pgfsetlinewidth{0.803000pt}%
\definecolor{currentstroke}{rgb}{0.000000,0.000000,0.000000}%
\pgfsetstrokecolor{currentstroke}%
\pgfsetdash{}{0pt}%
\pgfsys@defobject{currentmarker}{\pgfqpoint{0.000000in}{-0.048611in}}{\pgfqpoint{0.000000in}{0.000000in}}{%
\pgfpathmoveto{\pgfqpoint{0.000000in}{0.000000in}}%
\pgfpathlineto{\pgfqpoint{0.000000in}{-0.048611in}}%
\pgfusepath{stroke,fill}%
}%
\begin{pgfscope}%
\pgfsys@transformshift{5.700198in}{0.670138in}%
\pgfsys@useobject{currentmarker}{}%
\end{pgfscope}%
\end{pgfscope}%
\begin{pgfscope}%
\definecolor{textcolor}{rgb}{0.000000,0.000000,0.000000}%
\pgfsetstrokecolor{textcolor}%
\pgfsetfillcolor{textcolor}%
\pgftext[x=5.700198in,y=0.572916in,,top]{\color{textcolor}{\rmfamily\fontsize{14.000000}{16.800000}\selectfont\catcode`\^=\active\def^{\ifmmode\sp\else\^{}\fi}\catcode`\%=\active\def%{\%}$\mathdefault{175}$}}%
\end{pgfscope}%
\begin{pgfscope}%
\definecolor{textcolor}{rgb}{0.000000,0.000000,0.000000}%
\pgfsetstrokecolor{textcolor}%
\pgfsetfillcolor{textcolor}%
\pgftext[x=3.257774in,y=0.339583in,,top]{\color{textcolor}{\rmfamily\fontsize{18.000000}{21.600000}\selectfont\catcode`\^=\active\def^{\ifmmode\sp\else\^{}\fi}\catcode`\%=\active\def%{\%}Time [hr]}}%
\end{pgfscope}%
\begin{pgfscope}%
\pgfpathrectangle{\pgfqpoint{0.777774in}{0.670138in}}{\pgfqpoint{4.960000in}{3.696000in}}%
\pgfusepath{clip}%
\pgfsetrectcap%
\pgfsetroundjoin%
\pgfsetlinewidth{0.803000pt}%
\definecolor{currentstroke}{rgb}{0.690196,0.690196,0.690196}%
\pgfsetstrokecolor{currentstroke}%
\pgfsetstrokeopacity{0.200000}%
\pgfsetdash{}{0pt}%
\pgfpathmoveto{\pgfqpoint{0.777774in}{0.838138in}}%
\pgfpathlineto{\pgfqpoint{5.737774in}{0.838138in}}%
\pgfusepath{stroke}%
\end{pgfscope}%
\begin{pgfscope}%
\pgfsetbuttcap%
\pgfsetroundjoin%
\definecolor{currentfill}{rgb}{0.000000,0.000000,0.000000}%
\pgfsetfillcolor{currentfill}%
\pgfsetlinewidth{0.803000pt}%
\definecolor{currentstroke}{rgb}{0.000000,0.000000,0.000000}%
\pgfsetstrokecolor{currentstroke}%
\pgfsetdash{}{0pt}%
\pgfsys@defobject{currentmarker}{\pgfqpoint{-0.048611in}{0.000000in}}{\pgfqpoint{-0.000000in}{0.000000in}}{%
\pgfpathmoveto{\pgfqpoint{-0.000000in}{0.000000in}}%
\pgfpathlineto{\pgfqpoint{-0.048611in}{0.000000in}}%
\pgfusepath{stroke,fill}%
}%
\begin{pgfscope}%
\pgfsys@transformshift{0.777774in}{0.838138in}%
\pgfsys@useobject{currentmarker}{}%
\end{pgfscope}%
\end{pgfscope}%
\begin{pgfscope}%
\definecolor{textcolor}{rgb}{0.000000,0.000000,0.000000}%
\pgfsetstrokecolor{textcolor}%
\pgfsetfillcolor{textcolor}%
\pgftext[x=0.395138in, y=0.754805in, left, base]{\color{textcolor}{\rmfamily\fontsize{16.000000}{19.200000}\selectfont\catcode`\^=\active\def^{\ifmmode\sp\else\^{}\fi}\catcode`\%=\active\def%{\%}$\mathdefault{0.0}$}}%
\end{pgfscope}%
\begin{pgfscope}%
\pgfpathrectangle{\pgfqpoint{0.777774in}{0.670138in}}{\pgfqpoint{4.960000in}{3.696000in}}%
\pgfusepath{clip}%
\pgfsetrectcap%
\pgfsetroundjoin%
\pgfsetlinewidth{0.803000pt}%
\definecolor{currentstroke}{rgb}{0.690196,0.690196,0.690196}%
\pgfsetstrokecolor{currentstroke}%
\pgfsetstrokeopacity{0.200000}%
\pgfsetdash{}{0pt}%
\pgfpathmoveto{\pgfqpoint{0.777774in}{1.510138in}}%
\pgfpathlineto{\pgfqpoint{5.737774in}{1.510138in}}%
\pgfusepath{stroke}%
\end{pgfscope}%
\begin{pgfscope}%
\pgfsetbuttcap%
\pgfsetroundjoin%
\definecolor{currentfill}{rgb}{0.000000,0.000000,0.000000}%
\pgfsetfillcolor{currentfill}%
\pgfsetlinewidth{0.803000pt}%
\definecolor{currentstroke}{rgb}{0.000000,0.000000,0.000000}%
\pgfsetstrokecolor{currentstroke}%
\pgfsetdash{}{0pt}%
\pgfsys@defobject{currentmarker}{\pgfqpoint{-0.048611in}{0.000000in}}{\pgfqpoint{-0.000000in}{0.000000in}}{%
\pgfpathmoveto{\pgfqpoint{-0.000000in}{0.000000in}}%
\pgfpathlineto{\pgfqpoint{-0.048611in}{0.000000in}}%
\pgfusepath{stroke,fill}%
}%
\begin{pgfscope}%
\pgfsys@transformshift{0.777774in}{1.510138in}%
\pgfsys@useobject{currentmarker}{}%
\end{pgfscope}%
\end{pgfscope}%
\begin{pgfscope}%
\definecolor{textcolor}{rgb}{0.000000,0.000000,0.000000}%
\pgfsetstrokecolor{textcolor}%
\pgfsetfillcolor{textcolor}%
\pgftext[x=0.395138in, y=1.426805in, left, base]{\color{textcolor}{\rmfamily\fontsize{16.000000}{19.200000}\selectfont\catcode`\^=\active\def^{\ifmmode\sp\else\^{}\fi}\catcode`\%=\active\def%{\%}$\mathdefault{0.2}$}}%
\end{pgfscope}%
\begin{pgfscope}%
\pgfpathrectangle{\pgfqpoint{0.777774in}{0.670138in}}{\pgfqpoint{4.960000in}{3.696000in}}%
\pgfusepath{clip}%
\pgfsetrectcap%
\pgfsetroundjoin%
\pgfsetlinewidth{0.803000pt}%
\definecolor{currentstroke}{rgb}{0.690196,0.690196,0.690196}%
\pgfsetstrokecolor{currentstroke}%
\pgfsetstrokeopacity{0.200000}%
\pgfsetdash{}{0pt}%
\pgfpathmoveto{\pgfqpoint{0.777774in}{2.182138in}}%
\pgfpathlineto{\pgfqpoint{5.737774in}{2.182138in}}%
\pgfusepath{stroke}%
\end{pgfscope}%
\begin{pgfscope}%
\pgfsetbuttcap%
\pgfsetroundjoin%
\definecolor{currentfill}{rgb}{0.000000,0.000000,0.000000}%
\pgfsetfillcolor{currentfill}%
\pgfsetlinewidth{0.803000pt}%
\definecolor{currentstroke}{rgb}{0.000000,0.000000,0.000000}%
\pgfsetstrokecolor{currentstroke}%
\pgfsetdash{}{0pt}%
\pgfsys@defobject{currentmarker}{\pgfqpoint{-0.048611in}{0.000000in}}{\pgfqpoint{-0.000000in}{0.000000in}}{%
\pgfpathmoveto{\pgfqpoint{-0.000000in}{0.000000in}}%
\pgfpathlineto{\pgfqpoint{-0.048611in}{0.000000in}}%
\pgfusepath{stroke,fill}%
}%
\begin{pgfscope}%
\pgfsys@transformshift{0.777774in}{2.182138in}%
\pgfsys@useobject{currentmarker}{}%
\end{pgfscope}%
\end{pgfscope}%
\begin{pgfscope}%
\definecolor{textcolor}{rgb}{0.000000,0.000000,0.000000}%
\pgfsetstrokecolor{textcolor}%
\pgfsetfillcolor{textcolor}%
\pgftext[x=0.395138in, y=2.098805in, left, base]{\color{textcolor}{\rmfamily\fontsize{16.000000}{19.200000}\selectfont\catcode`\^=\active\def^{\ifmmode\sp\else\^{}\fi}\catcode`\%=\active\def%{\%}$\mathdefault{0.4}$}}%
\end{pgfscope}%
\begin{pgfscope}%
\pgfpathrectangle{\pgfqpoint{0.777774in}{0.670138in}}{\pgfqpoint{4.960000in}{3.696000in}}%
\pgfusepath{clip}%
\pgfsetrectcap%
\pgfsetroundjoin%
\pgfsetlinewidth{0.803000pt}%
\definecolor{currentstroke}{rgb}{0.690196,0.690196,0.690196}%
\pgfsetstrokecolor{currentstroke}%
\pgfsetstrokeopacity{0.200000}%
\pgfsetdash{}{0pt}%
\pgfpathmoveto{\pgfqpoint{0.777774in}{2.854138in}}%
\pgfpathlineto{\pgfqpoint{5.737774in}{2.854138in}}%
\pgfusepath{stroke}%
\end{pgfscope}%
\begin{pgfscope}%
\pgfsetbuttcap%
\pgfsetroundjoin%
\definecolor{currentfill}{rgb}{0.000000,0.000000,0.000000}%
\pgfsetfillcolor{currentfill}%
\pgfsetlinewidth{0.803000pt}%
\definecolor{currentstroke}{rgb}{0.000000,0.000000,0.000000}%
\pgfsetstrokecolor{currentstroke}%
\pgfsetdash{}{0pt}%
\pgfsys@defobject{currentmarker}{\pgfqpoint{-0.048611in}{0.000000in}}{\pgfqpoint{-0.000000in}{0.000000in}}{%
\pgfpathmoveto{\pgfqpoint{-0.000000in}{0.000000in}}%
\pgfpathlineto{\pgfqpoint{-0.048611in}{0.000000in}}%
\pgfusepath{stroke,fill}%
}%
\begin{pgfscope}%
\pgfsys@transformshift{0.777774in}{2.854138in}%
\pgfsys@useobject{currentmarker}{}%
\end{pgfscope}%
\end{pgfscope}%
\begin{pgfscope}%
\definecolor{textcolor}{rgb}{0.000000,0.000000,0.000000}%
\pgfsetstrokecolor{textcolor}%
\pgfsetfillcolor{textcolor}%
\pgftext[x=0.395138in, y=2.770805in, left, base]{\color{textcolor}{\rmfamily\fontsize{16.000000}{19.200000}\selectfont\catcode`\^=\active\def^{\ifmmode\sp\else\^{}\fi}\catcode`\%=\active\def%{\%}$\mathdefault{0.6}$}}%
\end{pgfscope}%
\begin{pgfscope}%
\pgfpathrectangle{\pgfqpoint{0.777774in}{0.670138in}}{\pgfqpoint{4.960000in}{3.696000in}}%
\pgfusepath{clip}%
\pgfsetrectcap%
\pgfsetroundjoin%
\pgfsetlinewidth{0.803000pt}%
\definecolor{currentstroke}{rgb}{0.690196,0.690196,0.690196}%
\pgfsetstrokecolor{currentstroke}%
\pgfsetstrokeopacity{0.200000}%
\pgfsetdash{}{0pt}%
\pgfpathmoveto{\pgfqpoint{0.777774in}{3.526138in}}%
\pgfpathlineto{\pgfqpoint{5.737774in}{3.526138in}}%
\pgfusepath{stroke}%
\end{pgfscope}%
\begin{pgfscope}%
\pgfsetbuttcap%
\pgfsetroundjoin%
\definecolor{currentfill}{rgb}{0.000000,0.000000,0.000000}%
\pgfsetfillcolor{currentfill}%
\pgfsetlinewidth{0.803000pt}%
\definecolor{currentstroke}{rgb}{0.000000,0.000000,0.000000}%
\pgfsetstrokecolor{currentstroke}%
\pgfsetdash{}{0pt}%
\pgfsys@defobject{currentmarker}{\pgfqpoint{-0.048611in}{0.000000in}}{\pgfqpoint{-0.000000in}{0.000000in}}{%
\pgfpathmoveto{\pgfqpoint{-0.000000in}{0.000000in}}%
\pgfpathlineto{\pgfqpoint{-0.048611in}{0.000000in}}%
\pgfusepath{stroke,fill}%
}%
\begin{pgfscope}%
\pgfsys@transformshift{0.777774in}{3.526138in}%
\pgfsys@useobject{currentmarker}{}%
\end{pgfscope}%
\end{pgfscope}%
\begin{pgfscope}%
\definecolor{textcolor}{rgb}{0.000000,0.000000,0.000000}%
\pgfsetstrokecolor{textcolor}%
\pgfsetfillcolor{textcolor}%
\pgftext[x=0.395138in, y=3.442805in, left, base]{\color{textcolor}{\rmfamily\fontsize{16.000000}{19.200000}\selectfont\catcode`\^=\active\def^{\ifmmode\sp\else\^{}\fi}\catcode`\%=\active\def%{\%}$\mathdefault{0.8}$}}%
\end{pgfscope}%
\begin{pgfscope}%
\pgfpathrectangle{\pgfqpoint{0.777774in}{0.670138in}}{\pgfqpoint{4.960000in}{3.696000in}}%
\pgfusepath{clip}%
\pgfsetrectcap%
\pgfsetroundjoin%
\pgfsetlinewidth{0.803000pt}%
\definecolor{currentstroke}{rgb}{0.690196,0.690196,0.690196}%
\pgfsetstrokecolor{currentstroke}%
\pgfsetstrokeopacity{0.200000}%
\pgfsetdash{}{0pt}%
\pgfpathmoveto{\pgfqpoint{0.777774in}{4.198138in}}%
\pgfpathlineto{\pgfqpoint{5.737774in}{4.198138in}}%
\pgfusepath{stroke}%
\end{pgfscope}%
\begin{pgfscope}%
\pgfsetbuttcap%
\pgfsetroundjoin%
\definecolor{currentfill}{rgb}{0.000000,0.000000,0.000000}%
\pgfsetfillcolor{currentfill}%
\pgfsetlinewidth{0.803000pt}%
\definecolor{currentstroke}{rgb}{0.000000,0.000000,0.000000}%
\pgfsetstrokecolor{currentstroke}%
\pgfsetdash{}{0pt}%
\pgfsys@defobject{currentmarker}{\pgfqpoint{-0.048611in}{0.000000in}}{\pgfqpoint{-0.000000in}{0.000000in}}{%
\pgfpathmoveto{\pgfqpoint{-0.000000in}{0.000000in}}%
\pgfpathlineto{\pgfqpoint{-0.048611in}{0.000000in}}%
\pgfusepath{stroke,fill}%
}%
\begin{pgfscope}%
\pgfsys@transformshift{0.777774in}{4.198138in}%
\pgfsys@useobject{currentmarker}{}%
\end{pgfscope}%
\end{pgfscope}%
\begin{pgfscope}%
\definecolor{textcolor}{rgb}{0.000000,0.000000,0.000000}%
\pgfsetstrokecolor{textcolor}%
\pgfsetfillcolor{textcolor}%
\pgftext[x=0.395138in, y=4.114805in, left, base]{\color{textcolor}{\rmfamily\fontsize{16.000000}{19.200000}\selectfont\catcode`\^=\active\def^{\ifmmode\sp\else\^{}\fi}\catcode`\%=\active\def%{\%}$\mathdefault{1.0}$}}%
\end{pgfscope}%
\begin{pgfscope}%
\definecolor{textcolor}{rgb}{0.000000,0.000000,0.000000}%
\pgfsetstrokecolor{textcolor}%
\pgfsetfillcolor{textcolor}%
\pgftext[x=0.339583in,y=2.518138in,,bottom,rotate=90.000000]{\color{textcolor}{\rmfamily\fontsize{18.000000}{21.600000}\selectfont\catcode`\^=\active\def^{\ifmmode\sp\else\^{}\fi}\catcode`\%=\active\def%{\%}Solar Availability [-]}}%
\end{pgfscope}%
\begin{pgfscope}%
\pgfpathrectangle{\pgfqpoint{0.777774in}{0.670138in}}{\pgfqpoint{4.960000in}{3.696000in}}%
\pgfusepath{clip}%
\pgfsetrectcap%
\pgfsetroundjoin%
\pgfsetlinewidth{1.505625pt}%
\definecolor{currentstroke}{rgb}{1.000000,0.843137,0.000000}%
\pgfsetstrokecolor{currentstroke}%
\pgfsetdash{}{0pt}%
\pgfpathmoveto{\pgfqpoint{1.003229in}{0.838138in}}%
\pgfpathlineto{\pgfqpoint{1.111231in}{0.838138in}}%
\pgfpathlineto{\pgfqpoint{1.138231in}{1.297961in}}%
\pgfpathlineto{\pgfqpoint{1.165232in}{0.838138in}}%
\pgfpathlineto{\pgfqpoint{1.192232in}{1.856257in}}%
\pgfpathlineto{\pgfqpoint{1.219233in}{1.484084in}}%
\pgfpathlineto{\pgfqpoint{1.246233in}{2.740994in}}%
\pgfpathlineto{\pgfqpoint{1.273234in}{2.811100in}}%
\pgfpathlineto{\pgfqpoint{1.300235in}{3.731905in}}%
\pgfpathlineto{\pgfqpoint{1.327235in}{2.994040in}}%
\pgfpathlineto{\pgfqpoint{1.354236in}{3.408973in}}%
\pgfpathlineto{\pgfqpoint{1.381236in}{2.484741in}}%
\pgfpathlineto{\pgfqpoint{1.408237in}{1.745316in}}%
\pgfpathlineto{\pgfqpoint{1.435237in}{2.183336in}}%
\pgfpathlineto{\pgfqpoint{1.462238in}{2.058674in}}%
\pgfpathlineto{\pgfqpoint{1.489238in}{0.961449in}}%
\pgfpathlineto{\pgfqpoint{1.516239in}{0.838138in}}%
\pgfpathlineto{\pgfqpoint{1.786244in}{0.838138in}}%
\pgfpathlineto{\pgfqpoint{1.813245in}{1.448911in}}%
\pgfpathlineto{\pgfqpoint{1.840245in}{1.690471in}}%
\pgfpathlineto{\pgfqpoint{1.867246in}{2.003198in}}%
\pgfpathlineto{\pgfqpoint{1.894247in}{1.751610in}}%
\pgfpathlineto{\pgfqpoint{1.921247in}{2.719638in}}%
\pgfpathlineto{\pgfqpoint{1.948248in}{3.073411in}}%
\pgfpathlineto{\pgfqpoint{1.975248in}{2.833501in}}%
\pgfpathlineto{\pgfqpoint{2.002249in}{2.694384in}}%
\pgfpathlineto{\pgfqpoint{2.029249in}{3.211514in}}%
\pgfpathlineto{\pgfqpoint{2.056250in}{2.386763in}}%
\pgfpathlineto{\pgfqpoint{2.083250in}{1.308029in}}%
\pgfpathlineto{\pgfqpoint{2.110251in}{0.838138in}}%
\pgfpathlineto{\pgfqpoint{2.434257in}{0.838138in}}%
\pgfpathlineto{\pgfqpoint{2.461258in}{1.665968in}}%
\pgfpathlineto{\pgfqpoint{2.488259in}{2.015160in}}%
\pgfpathlineto{\pgfqpoint{2.515259in}{1.812658in}}%
\pgfpathlineto{\pgfqpoint{2.542260in}{3.777761in}}%
\pgfpathlineto{\pgfqpoint{2.569260in}{2.706565in}}%
\pgfpathlineto{\pgfqpoint{2.596261in}{3.593528in}}%
\pgfpathlineto{\pgfqpoint{2.623261in}{3.487994in}}%
\pgfpathlineto{\pgfqpoint{2.650262in}{3.126132in}}%
\pgfpathlineto{\pgfqpoint{2.677262in}{2.927217in}}%
\pgfpathlineto{\pgfqpoint{2.704263in}{2.346826in}}%
\pgfpathlineto{\pgfqpoint{2.731263in}{2.165927in}}%
\pgfpathlineto{\pgfqpoint{2.758264in}{0.967760in}}%
\pgfpathlineto{\pgfqpoint{2.785265in}{0.838138in}}%
\pgfpathlineto{\pgfqpoint{3.109271in}{0.838138in}}%
\pgfpathlineto{\pgfqpoint{3.136272in}{1.108436in}}%
\pgfpathlineto{\pgfqpoint{3.163272in}{2.675757in}}%
\pgfpathlineto{\pgfqpoint{3.190273in}{2.741061in}}%
\pgfpathlineto{\pgfqpoint{3.217273in}{3.496735in}}%
\pgfpathlineto{\pgfqpoint{3.244274in}{3.341185in}}%
\pgfpathlineto{\pgfqpoint{3.271274in}{3.005439in}}%
\pgfpathlineto{\pgfqpoint{3.298275in}{4.198138in}}%
\pgfpathlineto{\pgfqpoint{3.325275in}{3.307831in}}%
\pgfpathlineto{\pgfqpoint{3.352276in}{2.398199in}}%
\pgfpathlineto{\pgfqpoint{3.379276in}{1.777696in}}%
\pgfpathlineto{\pgfqpoint{3.406277in}{1.753886in}}%
\pgfpathlineto{\pgfqpoint{3.433278in}{0.838138in}}%
\pgfpathlineto{\pgfqpoint{3.730284in}{0.838138in}}%
\pgfpathlineto{\pgfqpoint{3.757284in}{1.541985in}}%
\pgfpathlineto{\pgfqpoint{3.784285in}{1.268254in}}%
\pgfpathlineto{\pgfqpoint{3.811285in}{2.782965in}}%
\pgfpathlineto{\pgfqpoint{3.838286in}{3.120099in}}%
\pgfpathlineto{\pgfqpoint{3.865286in}{2.940746in}}%
\pgfpathlineto{\pgfqpoint{3.892287in}{3.556610in}}%
\pgfpathlineto{\pgfqpoint{3.919287in}{3.466243in}}%
\pgfpathlineto{\pgfqpoint{3.946288in}{3.457725in}}%
\pgfpathlineto{\pgfqpoint{3.973288in}{3.003880in}}%
\pgfpathlineto{\pgfqpoint{4.000289in}{2.053396in}}%
\pgfpathlineto{\pgfqpoint{4.027290in}{1.702869in}}%
\pgfpathlineto{\pgfqpoint{4.054290in}{0.838138in}}%
\pgfpathlineto{\pgfqpoint{4.378297in}{0.838138in}}%
\pgfpathlineto{\pgfqpoint{4.405297in}{1.635455in}}%
\pgfpathlineto{\pgfqpoint{4.432298in}{2.726480in}}%
\pgfpathlineto{\pgfqpoint{4.459298in}{2.460163in}}%
\pgfpathlineto{\pgfqpoint{4.486299in}{3.025718in}}%
\pgfpathlineto{\pgfqpoint{4.513299in}{2.984800in}}%
\pgfpathlineto{\pgfqpoint{4.540300in}{3.209632in}}%
\pgfpathlineto{\pgfqpoint{4.567300in}{2.639908in}}%
\pgfpathlineto{\pgfqpoint{4.594301in}{3.075932in}}%
\pgfpathlineto{\pgfqpoint{4.621302in}{2.747628in}}%
\pgfpathlineto{\pgfqpoint{4.648302in}{2.067549in}}%
\pgfpathlineto{\pgfqpoint{4.675303in}{1.634115in}}%
\pgfpathlineto{\pgfqpoint{4.702303in}{0.838138in}}%
\pgfpathlineto{\pgfqpoint{4.999309in}{0.838138in}}%
\pgfpathlineto{\pgfqpoint{5.026310in}{0.914272in}}%
\pgfpathlineto{\pgfqpoint{5.053310in}{2.113874in}}%
\pgfpathlineto{\pgfqpoint{5.080311in}{1.744003in}}%
\pgfpathlineto{\pgfqpoint{5.107311in}{2.247454in}}%
\pgfpathlineto{\pgfqpoint{5.134312in}{3.198715in}}%
\pgfpathlineto{\pgfqpoint{5.161312in}{3.508044in}}%
\pgfpathlineto{\pgfqpoint{5.188313in}{3.431354in}}%
\pgfpathlineto{\pgfqpoint{5.215313in}{3.350255in}}%
\pgfpathlineto{\pgfqpoint{5.242314in}{2.754203in}}%
\pgfpathlineto{\pgfqpoint{5.269315in}{2.997850in}}%
\pgfpathlineto{\pgfqpoint{5.296315in}{2.116362in}}%
\pgfpathlineto{\pgfqpoint{5.323316in}{1.690279in}}%
\pgfpathlineto{\pgfqpoint{5.350316in}{0.838138in}}%
\pgfpathlineto{\pgfqpoint{5.377317in}{0.977704in}}%
\pgfpathlineto{\pgfqpoint{5.404317in}{0.838138in}}%
\pgfpathlineto{\pgfqpoint{5.512319in}{0.838138in}}%
\pgfpathlineto{\pgfqpoint{5.512319in}{0.838138in}}%
\pgfusepath{stroke}%
\end{pgfscope}%
\begin{pgfscope}%
\pgfsetrectcap%
\pgfsetmiterjoin%
\pgfsetlinewidth{0.803000pt}%
\definecolor{currentstroke}{rgb}{0.000000,0.000000,0.000000}%
\pgfsetstrokecolor{currentstroke}%
\pgfsetdash{}{0pt}%
\pgfpathmoveto{\pgfqpoint{0.777774in}{0.670138in}}%
\pgfpathlineto{\pgfqpoint{0.777774in}{4.366138in}}%
\pgfusepath{stroke}%
\end{pgfscope}%
\begin{pgfscope}%
\pgfsetrectcap%
\pgfsetmiterjoin%
\pgfsetlinewidth{0.803000pt}%
\definecolor{currentstroke}{rgb}{0.000000,0.000000,0.000000}%
\pgfsetstrokecolor{currentstroke}%
\pgfsetdash{}{0pt}%
\pgfpathmoveto{\pgfqpoint{5.737774in}{0.670138in}}%
\pgfpathlineto{\pgfqpoint{5.737774in}{4.366138in}}%
\pgfusepath{stroke}%
\end{pgfscope}%
\begin{pgfscope}%
\pgfsetrectcap%
\pgfsetmiterjoin%
\pgfsetlinewidth{0.803000pt}%
\definecolor{currentstroke}{rgb}{0.000000,0.000000,0.000000}%
\pgfsetstrokecolor{currentstroke}%
\pgfsetdash{}{0pt}%
\pgfpathmoveto{\pgfqpoint{0.777774in}{0.670138in}}%
\pgfpathlineto{\pgfqpoint{5.737774in}{0.670138in}}%
\pgfusepath{stroke}%
\end{pgfscope}%
\begin{pgfscope}%
\pgfsetrectcap%
\pgfsetmiterjoin%
\pgfsetlinewidth{0.803000pt}%
\definecolor{currentstroke}{rgb}{0.000000,0.000000,0.000000}%
\pgfsetstrokecolor{currentstroke}%
\pgfsetdash{}{0pt}%
\pgfpathmoveto{\pgfqpoint{0.777774in}{4.366138in}}%
\pgfpathlineto{\pgfqpoint{5.737774in}{4.366138in}}%
\pgfusepath{stroke}%
\end{pgfscope}%
\end{pgfpicture}%
\makeatother%
\endgroup%
}
    \caption{Synthetic solar data for a period of a week.}
    \label{fig:synthetic-solar}
\end{figure}




\subsection{Results}

\subsection{Discussion}
