\section{\acs{mga} with \acl{moo}}
\label{section:mga-moo}

This thesis applies some ideas from \ac{mga} to the analysis of the sub-optimal
space from a \acl{moo} problem. Due to their iterative process, \acp{ga}
naturally generate many samples in a problem's feasible space. However, this
does not lead to a ``limited set'' of solutions but rather a potentially
infinite set. Some literature developed \acp{ga} that directly use \ac{mga} in
the iterative process
\cite{zechman_evolutionary_2004,zechman_evolutionary_2013}. However, existing
Python libraries such as \ac{pymoo} and \ac{deap} do not implement these
methods, and the challenge is not an inability to sample the sub-optimal space,
but rather to provide a comprehensible subset of solutions. The algorithm I
developed in this thesis to search the near-feasible space is the following:

\begin{enumerate}
    \item Obtain a set of Pareto-optimal solutions \textit{using any \ac{ga}}.
    \item Decide on a slack value (e.g., 10\% or 0.1), which represents an
    acceptable deviation from the Pareto front.
    \item Create a ``near-feasible front'' where the coordinates of each point
    are multiplied by unity plus the slack value. This is equivalent to relaxing
    the objective functions and converting them to a constraint. 
    \item Every individual is checked if all of its coordinates are
    \begin{itemize}
        \item below all of the coordinates for at least one point on the
        near-feasible front and
        \item above all of the coordinates for at least one point on the Pareto
        front.
    \end{itemize}  
    \item Lastly, the set of interior points may be randomly sampled to further
    restrict the number of analyzed solutions.
\end{enumerate}
\noindent
Figure \ref{fig:nd-mga} and Figure \ref{fig:3d-mga} demonstrate this algorithm
with 10 percent slack for a 2-D and 3-D Pareto front, respectively. Figure
\ref{fig:nd-mga} shows clearly that only points within the near-optimal space
(gray) are considered. Illustrating this behavior in three dimensions (and
above) is considerably more difficult. The 3-D interior points should be covered
by both surfaces, obstructing their view. Figure \ref{fig:3d-mga} shows that
this is the case in three panels. First, a top view of an opaque Pareto front
(green) where no interior points can be observed. Second, the same view with a
translucent Pareto front, revealing interior points and the near-optimal front
(blue). Finally, the view from underneath the near-optimal front once again
obscures the interior points, except for two near the edges of the sub-optimal
space. The tested points are omitted for clarity.
 

\begin{figure}[h]
  \centering
  \resizebox{0.6\columnwidth}{!}{%% Creator: Matplotlib, PGF backend
%%
%% To include the figure in your LaTeX document, write
%%   \input{<filename>.pgf}
%%
%% Make sure the required packages are loaded in your preamble
%%   \usepackage{pgf}
%%
%% Also ensure that all the required font packages are loaded; for instance,
%% the lmodern package is sometimes necessary when using math font.
%%   \usepackage{lmodern}
%%
%% Figures using additional raster images can only be included by \input if
%% they are in the same directory as the main LaTeX file. For loading figures
%% from other directories you can use the `import` package
%%   \usepackage{import}
%%
%% and then include the figures with
%%   \import{<path to file>}{<filename>.pgf}
%%
%% Matplotlib used the following preamble
%%   \def\mathdefault#1{#1}
%%   \everymath=\expandafter{\the\everymath\displaystyle}
%%   \IfFileExists{scrextend.sty}{
%%     \usepackage[fontsize=10.000000pt]{scrextend}
%%   }{
%%     \renewcommand{\normalsize}{\fontsize{10.000000}{12.000000}\selectfont}
%%     \normalsize
%%   }
%%   
%%   \makeatletter\@ifpackageloaded{underscore}{}{\usepackage[strings]{underscore}}\makeatother
%%
\begingroup%
\makeatletter%
\begin{pgfpicture}%
\pgfpathrectangle{\pgfpointorigin}{\pgfqpoint{6.951690in}{5.397777in}}%
\pgfusepath{use as bounding box, clip}%
\begin{pgfscope}%
\pgfsetbuttcap%
\pgfsetmiterjoin%
\definecolor{currentfill}{rgb}{1.000000,1.000000,1.000000}%
\pgfsetfillcolor{currentfill}%
\pgfsetlinewidth{0.000000pt}%
\definecolor{currentstroke}{rgb}{0.000000,0.000000,0.000000}%
\pgfsetstrokecolor{currentstroke}%
\pgfsetdash{}{0pt}%
\pgfpathmoveto{\pgfqpoint{0.000000in}{0.000000in}}%
\pgfpathlineto{\pgfqpoint{6.951690in}{0.000000in}}%
\pgfpathlineto{\pgfqpoint{6.951690in}{5.397777in}}%
\pgfpathlineto{\pgfqpoint{0.000000in}{5.397777in}}%
\pgfpathlineto{\pgfqpoint{0.000000in}{0.000000in}}%
\pgfpathclose%
\pgfusepath{fill}%
\end{pgfscope}%
\begin{pgfscope}%
\pgfsetbuttcap%
\pgfsetmiterjoin%
\definecolor{currentfill}{rgb}{1.000000,1.000000,1.000000}%
\pgfsetfillcolor{currentfill}%
\pgfsetlinewidth{0.000000pt}%
\definecolor{currentstroke}{rgb}{0.000000,0.000000,0.000000}%
\pgfsetstrokecolor{currentstroke}%
\pgfsetstrokeopacity{0.000000}%
\pgfsetdash{}{0pt}%
\pgfpathmoveto{\pgfqpoint{0.626386in}{0.608332in}}%
\pgfpathlineto{\pgfqpoint{6.826386in}{0.608332in}}%
\pgfpathlineto{\pgfqpoint{6.826386in}{5.228333in}}%
\pgfpathlineto{\pgfqpoint{0.626386in}{5.228333in}}%
\pgfpathlineto{\pgfqpoint{0.626386in}{0.608332in}}%
\pgfpathclose%
\pgfusepath{fill}%
\end{pgfscope}%
\begin{pgfscope}%
\pgfpathrectangle{\pgfqpoint{0.626386in}{0.608332in}}{\pgfqpoint{6.200000in}{4.620000in}}%
\pgfusepath{clip}%
\pgfsetbuttcap%
\pgfsetroundjoin%
\definecolor{currentfill}{rgb}{0.121569,0.466667,0.705882}%
\pgfsetfillcolor{currentfill}%
\pgfsetlinewidth{1.003750pt}%
\definecolor{currentstroke}{rgb}{0.121569,0.466667,0.705882}%
\pgfsetstrokecolor{currentstroke}%
\pgfsetdash{}{0pt}%
\pgfsys@defobject{currentmarker}{\pgfqpoint{-0.012028in}{-0.012028in}}{\pgfqpoint{0.012028in}{0.012028in}}{%
\pgfpathmoveto{\pgfqpoint{0.000000in}{-0.012028in}}%
\pgfpathcurveto{\pgfqpoint{0.003190in}{-0.012028in}}{\pgfqpoint{0.006250in}{-0.010761in}}{\pgfqpoint{0.008505in}{-0.008505in}}%
\pgfpathcurveto{\pgfqpoint{0.010761in}{-0.006250in}}{\pgfqpoint{0.012028in}{-0.003190in}}{\pgfqpoint{0.012028in}{0.000000in}}%
\pgfpathcurveto{\pgfqpoint{0.012028in}{0.003190in}}{\pgfqpoint{0.010761in}{0.006250in}}{\pgfqpoint{0.008505in}{0.008505in}}%
\pgfpathcurveto{\pgfqpoint{0.006250in}{0.010761in}}{\pgfqpoint{0.003190in}{0.012028in}}{\pgfqpoint{0.000000in}{0.012028in}}%
\pgfpathcurveto{\pgfqpoint{-0.003190in}{0.012028in}}{\pgfqpoint{-0.006250in}{0.010761in}}{\pgfqpoint{-0.008505in}{0.008505in}}%
\pgfpathcurveto{\pgfqpoint{-0.010761in}{0.006250in}}{\pgfqpoint{-0.012028in}{0.003190in}}{\pgfqpoint{-0.012028in}{0.000000in}}%
\pgfpathcurveto{\pgfqpoint{-0.012028in}{-0.003190in}}{\pgfqpoint{-0.010761in}{-0.006250in}}{\pgfqpoint{-0.008505in}{-0.008505in}}%
\pgfpathcurveto{\pgfqpoint{-0.006250in}{-0.010761in}}{\pgfqpoint{-0.003190in}{-0.012028in}}{\pgfqpoint{0.000000in}{-0.012028in}}%
\pgfpathlineto{\pgfqpoint{0.000000in}{-0.012028in}}%
\pgfpathclose%
\pgfusepath{stroke,fill}%
}%
\begin{pgfscope}%
\pgfsys@transformshift{6.748714in}{2.274941in}%
\pgfsys@useobject{currentmarker}{}%
\end{pgfscope}%
\begin{pgfscope}%
\pgfsys@transformshift{6.413646in}{1.630257in}%
\pgfsys@useobject{currentmarker}{}%
\end{pgfscope}%
\begin{pgfscope}%
\pgfsys@transformshift{2.626608in}{0.849035in}%
\pgfsys@useobject{currentmarker}{}%
\end{pgfscope}%
\begin{pgfscope}%
\pgfsys@transformshift{2.141870in}{1.939487in}%
\pgfsys@useobject{currentmarker}{}%
\end{pgfscope}%
\begin{pgfscope}%
\pgfsys@transformshift{6.669603in}{1.640906in}%
\pgfsys@useobject{currentmarker}{}%
\end{pgfscope}%
\begin{pgfscope}%
\pgfsys@transformshift{3.390781in}{3.524423in}%
\pgfsys@useobject{currentmarker}{}%
\end{pgfscope}%
\begin{pgfscope}%
\pgfsys@transformshift{6.039894in}{4.905623in}%
\pgfsys@useobject{currentmarker}{}%
\end{pgfscope}%
\begin{pgfscope}%
\pgfsys@transformshift{4.856800in}{3.796453in}%
\pgfsys@useobject{currentmarker}{}%
\end{pgfscope}%
\begin{pgfscope}%
\pgfsys@transformshift{5.238397in}{1.418422in}%
\pgfsys@useobject{currentmarker}{}%
\end{pgfscope}%
\begin{pgfscope}%
\pgfsys@transformshift{1.704963in}{4.941840in}%
\pgfsys@useobject{currentmarker}{}%
\end{pgfscope}%
\begin{pgfscope}%
\pgfsys@transformshift{1.003358in}{3.926010in}%
\pgfsys@useobject{currentmarker}{}%
\end{pgfscope}%
\begin{pgfscope}%
\pgfsys@transformshift{4.833963in}{3.530664in}%
\pgfsys@useobject{currentmarker}{}%
\end{pgfscope}%
\begin{pgfscope}%
\pgfsys@transformshift{1.003350in}{5.525871in}%
\pgfsys@useobject{currentmarker}{}%
\end{pgfscope}%
\begin{pgfscope}%
\pgfsys@transformshift{3.377903in}{3.104026in}%
\pgfsys@useobject{currentmarker}{}%
\end{pgfscope}%
\begin{pgfscope}%
\pgfsys@transformshift{0.646020in}{1.573541in}%
\pgfsys@useobject{currentmarker}{}%
\end{pgfscope}%
\begin{pgfscope}%
\pgfsys@transformshift{6.007732in}{2.520309in}%
\pgfsys@useobject{currentmarker}{}%
\end{pgfscope}%
\begin{pgfscope}%
\pgfsys@transformshift{5.238781in}{5.222709in}%
\pgfsys@useobject{currentmarker}{}%
\end{pgfscope}%
\begin{pgfscope}%
\pgfsys@transformshift{1.588421in}{5.604699in}%
\pgfsys@useobject{currentmarker}{}%
\end{pgfscope}%
\begin{pgfscope}%
\pgfsys@transformshift{1.769311in}{5.321012in}%
\pgfsys@useobject{currentmarker}{}%
\end{pgfscope}%
\begin{pgfscope}%
\pgfsys@transformshift{1.171002in}{2.753761in}%
\pgfsys@useobject{currentmarker}{}%
\end{pgfscope}%
\begin{pgfscope}%
\pgfsys@transformshift{5.822807in}{1.735579in}%
\pgfsys@useobject{currentmarker}{}%
\end{pgfscope}%
\begin{pgfscope}%
\pgfsys@transformshift{6.074184in}{5.518512in}%
\pgfsys@useobject{currentmarker}{}%
\end{pgfscope}%
\begin{pgfscope}%
\pgfsys@transformshift{5.902569in}{2.648563in}%
\pgfsys@useobject{currentmarker}{}%
\end{pgfscope}%
\begin{pgfscope}%
\pgfsys@transformshift{2.948874in}{2.832420in}%
\pgfsys@useobject{currentmarker}{}%
\end{pgfscope}%
\begin{pgfscope}%
\pgfsys@transformshift{1.204215in}{1.439799in}%
\pgfsys@useobject{currentmarker}{}%
\end{pgfscope}%
\begin{pgfscope}%
\pgfsys@transformshift{3.989788in}{4.195533in}%
\pgfsys@useobject{currentmarker}{}%
\end{pgfscope}%
\begin{pgfscope}%
\pgfsys@transformshift{3.440330in}{1.868584in}%
\pgfsys@useobject{currentmarker}{}%
\end{pgfscope}%
\begin{pgfscope}%
\pgfsys@transformshift{6.564798in}{4.292381in}%
\pgfsys@useobject{currentmarker}{}%
\end{pgfscope}%
\begin{pgfscope}%
\pgfsys@transformshift{3.585366in}{2.694274in}%
\pgfsys@useobject{currentmarker}{}%
\end{pgfscope}%
\begin{pgfscope}%
\pgfsys@transformshift{5.304717in}{4.860155in}%
\pgfsys@useobject{currentmarker}{}%
\end{pgfscope}%
\begin{pgfscope}%
\pgfsys@transformshift{2.466166in}{4.195984in}%
\pgfsys@useobject{currentmarker}{}%
\end{pgfscope}%
\begin{pgfscope}%
\pgfsys@transformshift{5.696446in}{5.032539in}%
\pgfsys@useobject{currentmarker}{}%
\end{pgfscope}%
\begin{pgfscope}%
\pgfsys@transformshift{0.782050in}{2.994974in}%
\pgfsys@useobject{currentmarker}{}%
\end{pgfscope}%
\begin{pgfscope}%
\pgfsys@transformshift{4.467561in}{1.731753in}%
\pgfsys@useobject{currentmarker}{}%
\end{pgfscope}%
\begin{pgfscope}%
\pgfsys@transformshift{3.460521in}{3.096552in}%
\pgfsys@useobject{currentmarker}{}%
\end{pgfscope}%
\begin{pgfscope}%
\pgfsys@transformshift{6.823525in}{0.239116in}%
\pgfsys@useobject{currentmarker}{}%
\end{pgfscope}%
\begin{pgfscope}%
\pgfsys@transformshift{1.888138in}{1.604054in}%
\pgfsys@useobject{currentmarker}{}%
\end{pgfscope}%
\begin{pgfscope}%
\pgfsys@transformshift{2.762305in}{0.494936in}%
\pgfsys@useobject{currentmarker}{}%
\end{pgfscope}%
\begin{pgfscope}%
\pgfsys@transformshift{2.672978in}{1.285705in}%
\pgfsys@useobject{currentmarker}{}%
\end{pgfscope}%
\begin{pgfscope}%
\pgfsys@transformshift{5.824222in}{3.672674in}%
\pgfsys@useobject{currentmarker}{}%
\end{pgfscope}%
\begin{pgfscope}%
\pgfsys@transformshift{1.161961in}{1.404314in}%
\pgfsys@useobject{currentmarker}{}%
\end{pgfscope}%
\begin{pgfscope}%
\pgfsys@transformshift{4.896258in}{0.937813in}%
\pgfsys@useobject{currentmarker}{}%
\end{pgfscope}%
\begin{pgfscope}%
\pgfsys@transformshift{6.622180in}{0.949313in}%
\pgfsys@useobject{currentmarker}{}%
\end{pgfscope}%
\begin{pgfscope}%
\pgfsys@transformshift{5.626688in}{2.917418in}%
\pgfsys@useobject{currentmarker}{}%
\end{pgfscope}%
\begin{pgfscope}%
\pgfsys@transformshift{2.536411in}{1.480725in}%
\pgfsys@useobject{currentmarker}{}%
\end{pgfscope}%
\begin{pgfscope}%
\pgfsys@transformshift{6.224103in}{2.260474in}%
\pgfsys@useobject{currentmarker}{}%
\end{pgfscope}%
\begin{pgfscope}%
\pgfsys@transformshift{1.104948in}{0.390510in}%
\pgfsys@useobject{currentmarker}{}%
\end{pgfscope}%
\begin{pgfscope}%
\pgfsys@transformshift{3.519955in}{4.960386in}%
\pgfsys@useobject{currentmarker}{}%
\end{pgfscope}%
\begin{pgfscope}%
\pgfsys@transformshift{2.824140in}{4.686724in}%
\pgfsys@useobject{currentmarker}{}%
\end{pgfscope}%
\begin{pgfscope}%
\pgfsys@transformshift{6.563792in}{0.571309in}%
\pgfsys@useobject{currentmarker}{}%
\end{pgfscope}%
\begin{pgfscope}%
\pgfsys@transformshift{2.493362in}{3.574964in}%
\pgfsys@useobject{currentmarker}{}%
\end{pgfscope}%
\begin{pgfscope}%
\pgfsys@transformshift{2.517917in}{1.157796in}%
\pgfsys@useobject{currentmarker}{}%
\end{pgfscope}%
\begin{pgfscope}%
\pgfsys@transformshift{3.732402in}{4.344262in}%
\pgfsys@useobject{currentmarker}{}%
\end{pgfscope}%
\begin{pgfscope}%
\pgfsys@transformshift{4.934404in}{1.556609in}%
\pgfsys@useobject{currentmarker}{}%
\end{pgfscope}%
\begin{pgfscope}%
\pgfsys@transformshift{4.525612in}{1.923829in}%
\pgfsys@useobject{currentmarker}{}%
\end{pgfscope}%
\begin{pgfscope}%
\pgfsys@transformshift{2.364084in}{5.265262in}%
\pgfsys@useobject{currentmarker}{}%
\end{pgfscope}%
\begin{pgfscope}%
\pgfsys@transformshift{2.378996in}{5.368462in}%
\pgfsys@useobject{currentmarker}{}%
\end{pgfscope}%
\begin{pgfscope}%
\pgfsys@transformshift{1.892708in}{3.282919in}%
\pgfsys@useobject{currentmarker}{}%
\end{pgfscope}%
\begin{pgfscope}%
\pgfsys@transformshift{3.010271in}{4.602870in}%
\pgfsys@useobject{currentmarker}{}%
\end{pgfscope}%
\begin{pgfscope}%
\pgfsys@transformshift{1.425941in}{5.382087in}%
\pgfsys@useobject{currentmarker}{}%
\end{pgfscope}%
\begin{pgfscope}%
\pgfsys@transformshift{5.194955in}{1.161596in}%
\pgfsys@useobject{currentmarker}{}%
\end{pgfscope}%
\begin{pgfscope}%
\pgfsys@transformshift{4.968865in}{0.711625in}%
\pgfsys@useobject{currentmarker}{}%
\end{pgfscope}%
\begin{pgfscope}%
\pgfsys@transformshift{4.114124in}{0.764184in}%
\pgfsys@useobject{currentmarker}{}%
\end{pgfscope}%
\begin{pgfscope}%
\pgfsys@transformshift{0.900272in}{2.342915in}%
\pgfsys@useobject{currentmarker}{}%
\end{pgfscope}%
\begin{pgfscope}%
\pgfsys@transformshift{4.958177in}{4.161509in}%
\pgfsys@useobject{currentmarker}{}%
\end{pgfscope}%
\begin{pgfscope}%
\pgfsys@transformshift{4.820132in}{3.424780in}%
\pgfsys@useobject{currentmarker}{}%
\end{pgfscope}%
\begin{pgfscope}%
\pgfsys@transformshift{6.136298in}{3.131568in}%
\pgfsys@useobject{currentmarker}{}%
\end{pgfscope}%
\begin{pgfscope}%
\pgfsys@transformshift{1.239574in}{0.975273in}%
\pgfsys@useobject{currentmarker}{}%
\end{pgfscope}%
\begin{pgfscope}%
\pgfsys@transformshift{6.658326in}{4.548452in}%
\pgfsys@useobject{currentmarker}{}%
\end{pgfscope}%
\begin{pgfscope}%
\pgfsys@transformshift{0.668721in}{5.352014in}%
\pgfsys@useobject{currentmarker}{}%
\end{pgfscope}%
\begin{pgfscope}%
\pgfsys@transformshift{4.734825in}{5.160915in}%
\pgfsys@useobject{currentmarker}{}%
\end{pgfscope}%
\begin{pgfscope}%
\pgfsys@transformshift{1.582845in}{3.408881in}%
\pgfsys@useobject{currentmarker}{}%
\end{pgfscope}%
\begin{pgfscope}%
\pgfsys@transformshift{3.784181in}{1.152152in}%
\pgfsys@useobject{currentmarker}{}%
\end{pgfscope}%
\begin{pgfscope}%
\pgfsys@transformshift{1.020356in}{3.821881in}%
\pgfsys@useobject{currentmarker}{}%
\end{pgfscope}%
\begin{pgfscope}%
\pgfsys@transformshift{0.826380in}{1.502212in}%
\pgfsys@useobject{currentmarker}{}%
\end{pgfscope}%
\begin{pgfscope}%
\pgfsys@transformshift{3.857296in}{3.483551in}%
\pgfsys@useobject{currentmarker}{}%
\end{pgfscope}%
\begin{pgfscope}%
\pgfsys@transformshift{2.400085in}{4.793404in}%
\pgfsys@useobject{currentmarker}{}%
\end{pgfscope}%
\begin{pgfscope}%
\pgfsys@transformshift{6.234743in}{4.140911in}%
\pgfsys@useobject{currentmarker}{}%
\end{pgfscope}%
\begin{pgfscope}%
\pgfsys@transformshift{6.434302in}{0.527551in}%
\pgfsys@useobject{currentmarker}{}%
\end{pgfscope}%
\begin{pgfscope}%
\pgfsys@transformshift{0.940729in}{0.880528in}%
\pgfsys@useobject{currentmarker}{}%
\end{pgfscope}%
\begin{pgfscope}%
\pgfsys@transformshift{1.391633in}{3.096166in}%
\pgfsys@useobject{currentmarker}{}%
\end{pgfscope}%
\begin{pgfscope}%
\pgfsys@transformshift{6.350284in}{1.199255in}%
\pgfsys@useobject{currentmarker}{}%
\end{pgfscope}%
\begin{pgfscope}%
\pgfsys@transformshift{6.841479in}{4.993467in}%
\pgfsys@useobject{currentmarker}{}%
\end{pgfscope}%
\begin{pgfscope}%
\pgfsys@transformshift{2.650795in}{1.247392in}%
\pgfsys@useobject{currentmarker}{}%
\end{pgfscope}%
\begin{pgfscope}%
\pgfsys@transformshift{3.740935in}{1.917027in}%
\pgfsys@useobject{currentmarker}{}%
\end{pgfscope}%
\begin{pgfscope}%
\pgfsys@transformshift{0.816347in}{0.710485in}%
\pgfsys@useobject{currentmarker}{}%
\end{pgfscope}%
\begin{pgfscope}%
\pgfsys@transformshift{4.318591in}{3.655900in}%
\pgfsys@useobject{currentmarker}{}%
\end{pgfscope}%
\begin{pgfscope}%
\pgfsys@transformshift{6.683803in}{0.287542in}%
\pgfsys@useobject{currentmarker}{}%
\end{pgfscope}%
\begin{pgfscope}%
\pgfsys@transformshift{5.222113in}{3.649505in}%
\pgfsys@useobject{currentmarker}{}%
\end{pgfscope}%
\begin{pgfscope}%
\pgfsys@transformshift{3.771139in}{0.864515in}%
\pgfsys@useobject{currentmarker}{}%
\end{pgfscope}%
\begin{pgfscope}%
\pgfsys@transformshift{6.004284in}{4.027861in}%
\pgfsys@useobject{currentmarker}{}%
\end{pgfscope}%
\begin{pgfscope}%
\pgfsys@transformshift{4.190676in}{5.543986in}%
\pgfsys@useobject{currentmarker}{}%
\end{pgfscope}%
\begin{pgfscope}%
\pgfsys@transformshift{1.192751in}{4.162065in}%
\pgfsys@useobject{currentmarker}{}%
\end{pgfscope}%
\begin{pgfscope}%
\pgfsys@transformshift{3.689329in}{1.732570in}%
\pgfsys@useobject{currentmarker}{}%
\end{pgfscope}%
\begin{pgfscope}%
\pgfsys@transformshift{6.175975in}{4.600810in}%
\pgfsys@useobject{currentmarker}{}%
\end{pgfscope}%
\begin{pgfscope}%
\pgfsys@transformshift{3.381021in}{3.697299in}%
\pgfsys@useobject{currentmarker}{}%
\end{pgfscope}%
\begin{pgfscope}%
\pgfsys@transformshift{4.581980in}{0.252491in}%
\pgfsys@useobject{currentmarker}{}%
\end{pgfscope}%
\begin{pgfscope}%
\pgfsys@transformshift{2.015825in}{1.386537in}%
\pgfsys@useobject{currentmarker}{}%
\end{pgfscope}%
\begin{pgfscope}%
\pgfsys@transformshift{2.007116in}{4.238541in}%
\pgfsys@useobject{currentmarker}{}%
\end{pgfscope}%
\begin{pgfscope}%
\pgfsys@transformshift{6.528407in}{1.656070in}%
\pgfsys@useobject{currentmarker}{}%
\end{pgfscope}%
\begin{pgfscope}%
\pgfsys@transformshift{5.310581in}{0.217860in}%
\pgfsys@useobject{currentmarker}{}%
\end{pgfscope}%
\begin{pgfscope}%
\pgfsys@transformshift{5.699741in}{4.595809in}%
\pgfsys@useobject{currentmarker}{}%
\end{pgfscope}%
\begin{pgfscope}%
\pgfsys@transformshift{0.968218in}{0.744828in}%
\pgfsys@useobject{currentmarker}{}%
\end{pgfscope}%
\begin{pgfscope}%
\pgfsys@transformshift{4.094222in}{4.356993in}%
\pgfsys@useobject{currentmarker}{}%
\end{pgfscope}%
\begin{pgfscope}%
\pgfsys@transformshift{5.208142in}{3.919384in}%
\pgfsys@useobject{currentmarker}{}%
\end{pgfscope}%
\begin{pgfscope}%
\pgfsys@transformshift{5.406600in}{3.467528in}%
\pgfsys@useobject{currentmarker}{}%
\end{pgfscope}%
\begin{pgfscope}%
\pgfsys@transformshift{2.102448in}{0.359253in}%
\pgfsys@useobject{currentmarker}{}%
\end{pgfscope}%
\begin{pgfscope}%
\pgfsys@transformshift{6.341119in}{2.459282in}%
\pgfsys@useobject{currentmarker}{}%
\end{pgfscope}%
\begin{pgfscope}%
\pgfsys@transformshift{5.161335in}{1.653424in}%
\pgfsys@useobject{currentmarker}{}%
\end{pgfscope}%
\begin{pgfscope}%
\pgfsys@transformshift{6.771717in}{1.175442in}%
\pgfsys@useobject{currentmarker}{}%
\end{pgfscope}%
\begin{pgfscope}%
\pgfsys@transformshift{0.920489in}{2.894691in}%
\pgfsys@useobject{currentmarker}{}%
\end{pgfscope}%
\begin{pgfscope}%
\pgfsys@transformshift{2.949004in}{3.773060in}%
\pgfsys@useobject{currentmarker}{}%
\end{pgfscope}%
\begin{pgfscope}%
\pgfsys@transformshift{0.626675in}{0.984625in}%
\pgfsys@useobject{currentmarker}{}%
\end{pgfscope}%
\begin{pgfscope}%
\pgfsys@transformshift{6.010775in}{3.921228in}%
\pgfsys@useobject{currentmarker}{}%
\end{pgfscope}%
\begin{pgfscope}%
\pgfsys@transformshift{2.707698in}{0.341599in}%
\pgfsys@useobject{currentmarker}{}%
\end{pgfscope}%
\begin{pgfscope}%
\pgfsys@transformshift{0.890930in}{2.326007in}%
\pgfsys@useobject{currentmarker}{}%
\end{pgfscope}%
\begin{pgfscope}%
\pgfsys@transformshift{6.537541in}{5.204043in}%
\pgfsys@useobject{currentmarker}{}%
\end{pgfscope}%
\begin{pgfscope}%
\pgfsys@transformshift{2.599215in}{2.231556in}%
\pgfsys@useobject{currentmarker}{}%
\end{pgfscope}%
\begin{pgfscope}%
\pgfsys@transformshift{2.298475in}{0.978419in}%
\pgfsys@useobject{currentmarker}{}%
\end{pgfscope}%
\begin{pgfscope}%
\pgfsys@transformshift{4.334250in}{2.725847in}%
\pgfsys@useobject{currentmarker}{}%
\end{pgfscope}%
\begin{pgfscope}%
\pgfsys@transformshift{5.862556in}{2.901862in}%
\pgfsys@useobject{currentmarker}{}%
\end{pgfscope}%
\begin{pgfscope}%
\pgfsys@transformshift{0.737427in}{2.588836in}%
\pgfsys@useobject{currentmarker}{}%
\end{pgfscope}%
\begin{pgfscope}%
\pgfsys@transformshift{1.466590in}{2.014927in}%
\pgfsys@useobject{currentmarker}{}%
\end{pgfscope}%
\begin{pgfscope}%
\pgfsys@transformshift{3.768743in}{3.162637in}%
\pgfsys@useobject{currentmarker}{}%
\end{pgfscope}%
\begin{pgfscope}%
\pgfsys@transformshift{4.981965in}{5.452028in}%
\pgfsys@useobject{currentmarker}{}%
\end{pgfscope}%
\begin{pgfscope}%
\pgfsys@transformshift{2.976413in}{4.315850in}%
\pgfsys@useobject{currentmarker}{}%
\end{pgfscope}%
\begin{pgfscope}%
\pgfsys@transformshift{3.939099in}{4.910376in}%
\pgfsys@useobject{currentmarker}{}%
\end{pgfscope}%
\begin{pgfscope}%
\pgfsys@transformshift{4.142002in}{2.285504in}%
\pgfsys@useobject{currentmarker}{}%
\end{pgfscope}%
\begin{pgfscope}%
\pgfsys@transformshift{3.760338in}{3.088483in}%
\pgfsys@useobject{currentmarker}{}%
\end{pgfscope}%
\begin{pgfscope}%
\pgfsys@transformshift{6.804641in}{0.539977in}%
\pgfsys@useobject{currentmarker}{}%
\end{pgfscope}%
\begin{pgfscope}%
\pgfsys@transformshift{0.907503in}{1.263762in}%
\pgfsys@useobject{currentmarker}{}%
\end{pgfscope}%
\begin{pgfscope}%
\pgfsys@transformshift{0.650888in}{0.782229in}%
\pgfsys@useobject{currentmarker}{}%
\end{pgfscope}%
\begin{pgfscope}%
\pgfsys@transformshift{3.098873in}{1.320918in}%
\pgfsys@useobject{currentmarker}{}%
\end{pgfscope}%
\begin{pgfscope}%
\pgfsys@transformshift{3.647529in}{4.805754in}%
\pgfsys@useobject{currentmarker}{}%
\end{pgfscope}%
\begin{pgfscope}%
\pgfsys@transformshift{4.628549in}{4.381591in}%
\pgfsys@useobject{currentmarker}{}%
\end{pgfscope}%
\begin{pgfscope}%
\pgfsys@transformshift{5.832577in}{0.897172in}%
\pgfsys@useobject{currentmarker}{}%
\end{pgfscope}%
\begin{pgfscope}%
\pgfsys@transformshift{0.903870in}{2.644883in}%
\pgfsys@useobject{currentmarker}{}%
\end{pgfscope}%
\begin{pgfscope}%
\pgfsys@transformshift{3.550935in}{5.430657in}%
\pgfsys@useobject{currentmarker}{}%
\end{pgfscope}%
\begin{pgfscope}%
\pgfsys@transformshift{2.242411in}{2.622652in}%
\pgfsys@useobject{currentmarker}{}%
\end{pgfscope}%
\begin{pgfscope}%
\pgfsys@transformshift{4.130464in}{4.830717in}%
\pgfsys@useobject{currentmarker}{}%
\end{pgfscope}%
\begin{pgfscope}%
\pgfsys@transformshift{1.411996in}{5.257843in}%
\pgfsys@useobject{currentmarker}{}%
\end{pgfscope}%
\begin{pgfscope}%
\pgfsys@transformshift{4.260224in}{2.722876in}%
\pgfsys@useobject{currentmarker}{}%
\end{pgfscope}%
\begin{pgfscope}%
\pgfsys@transformshift{3.405747in}{0.478012in}%
\pgfsys@useobject{currentmarker}{}%
\end{pgfscope}%
\begin{pgfscope}%
\pgfsys@transformshift{2.350355in}{3.579143in}%
\pgfsys@useobject{currentmarker}{}%
\end{pgfscope}%
\begin{pgfscope}%
\pgfsys@transformshift{6.387106in}{5.215306in}%
\pgfsys@useobject{currentmarker}{}%
\end{pgfscope}%
\begin{pgfscope}%
\pgfsys@transformshift{2.231490in}{3.123346in}%
\pgfsys@useobject{currentmarker}{}%
\end{pgfscope}%
\begin{pgfscope}%
\pgfsys@transformshift{2.544036in}{2.207787in}%
\pgfsys@useobject{currentmarker}{}%
\end{pgfscope}%
\begin{pgfscope}%
\pgfsys@transformshift{2.827197in}{2.561162in}%
\pgfsys@useobject{currentmarker}{}%
\end{pgfscope}%
\begin{pgfscope}%
\pgfsys@transformshift{0.646834in}{2.681178in}%
\pgfsys@useobject{currentmarker}{}%
\end{pgfscope}%
\begin{pgfscope}%
\pgfsys@transformshift{2.594401in}{1.640905in}%
\pgfsys@useobject{currentmarker}{}%
\end{pgfscope}%
\begin{pgfscope}%
\pgfsys@transformshift{4.214490in}{2.877842in}%
\pgfsys@useobject{currentmarker}{}%
\end{pgfscope}%
\begin{pgfscope}%
\pgfsys@transformshift{5.137671in}{2.729692in}%
\pgfsys@useobject{currentmarker}{}%
\end{pgfscope}%
\begin{pgfscope}%
\pgfsys@transformshift{5.961491in}{4.495294in}%
\pgfsys@useobject{currentmarker}{}%
\end{pgfscope}%
\begin{pgfscope}%
\pgfsys@transformshift{4.017877in}{5.354591in}%
\pgfsys@useobject{currentmarker}{}%
\end{pgfscope}%
\begin{pgfscope}%
\pgfsys@transformshift{4.678181in}{2.499123in}%
\pgfsys@useobject{currentmarker}{}%
\end{pgfscope}%
\begin{pgfscope}%
\pgfsys@transformshift{2.045123in}{1.090396in}%
\pgfsys@useobject{currentmarker}{}%
\end{pgfscope}%
\begin{pgfscope}%
\pgfsys@transformshift{2.850108in}{1.542976in}%
\pgfsys@useobject{currentmarker}{}%
\end{pgfscope}%
\begin{pgfscope}%
\pgfsys@transformshift{3.559126in}{4.430788in}%
\pgfsys@useobject{currentmarker}{}%
\end{pgfscope}%
\begin{pgfscope}%
\pgfsys@transformshift{0.868814in}{4.461062in}%
\pgfsys@useobject{currentmarker}{}%
\end{pgfscope}%
\begin{pgfscope}%
\pgfsys@transformshift{1.680316in}{3.410044in}%
\pgfsys@useobject{currentmarker}{}%
\end{pgfscope}%
\begin{pgfscope}%
\pgfsys@transformshift{0.654024in}{4.739106in}%
\pgfsys@useobject{currentmarker}{}%
\end{pgfscope}%
\begin{pgfscope}%
\pgfsys@transformshift{5.124198in}{1.177305in}%
\pgfsys@useobject{currentmarker}{}%
\end{pgfscope}%
\begin{pgfscope}%
\pgfsys@transformshift{6.386361in}{2.124473in}%
\pgfsys@useobject{currentmarker}{}%
\end{pgfscope}%
\begin{pgfscope}%
\pgfsys@transformshift{4.465022in}{1.382993in}%
\pgfsys@useobject{currentmarker}{}%
\end{pgfscope}%
\begin{pgfscope}%
\pgfsys@transformshift{3.751574in}{3.923941in}%
\pgfsys@useobject{currentmarker}{}%
\end{pgfscope}%
\begin{pgfscope}%
\pgfsys@transformshift{1.865799in}{1.980264in}%
\pgfsys@useobject{currentmarker}{}%
\end{pgfscope}%
\begin{pgfscope}%
\pgfsys@transformshift{3.344360in}{3.226466in}%
\pgfsys@useobject{currentmarker}{}%
\end{pgfscope}%
\begin{pgfscope}%
\pgfsys@transformshift{2.451884in}{2.779128in}%
\pgfsys@useobject{currentmarker}{}%
\end{pgfscope}%
\begin{pgfscope}%
\pgfsys@transformshift{1.279225in}{2.588906in}%
\pgfsys@useobject{currentmarker}{}%
\end{pgfscope}%
\begin{pgfscope}%
\pgfsys@transformshift{0.759148in}{1.180563in}%
\pgfsys@useobject{currentmarker}{}%
\end{pgfscope}%
\begin{pgfscope}%
\pgfsys@transformshift{6.349859in}{1.331878in}%
\pgfsys@useobject{currentmarker}{}%
\end{pgfscope}%
\begin{pgfscope}%
\pgfsys@transformshift{2.908829in}{1.191898in}%
\pgfsys@useobject{currentmarker}{}%
\end{pgfscope}%
\begin{pgfscope}%
\pgfsys@transformshift{4.294240in}{3.406530in}%
\pgfsys@useobject{currentmarker}{}%
\end{pgfscope}%
\begin{pgfscope}%
\pgfsys@transformshift{6.223995in}{4.193718in}%
\pgfsys@useobject{currentmarker}{}%
\end{pgfscope}%
\begin{pgfscope}%
\pgfsys@transformshift{6.541431in}{1.266260in}%
\pgfsys@useobject{currentmarker}{}%
\end{pgfscope}%
\begin{pgfscope}%
\pgfsys@transformshift{2.137565in}{4.917569in}%
\pgfsys@useobject{currentmarker}{}%
\end{pgfscope}%
\begin{pgfscope}%
\pgfsys@transformshift{6.080457in}{0.934593in}%
\pgfsys@useobject{currentmarker}{}%
\end{pgfscope}%
\begin{pgfscope}%
\pgfsys@transformshift{2.901099in}{0.280599in}%
\pgfsys@useobject{currentmarker}{}%
\end{pgfscope}%
\begin{pgfscope}%
\pgfsys@transformshift{2.090748in}{2.087960in}%
\pgfsys@useobject{currentmarker}{}%
\end{pgfscope}%
\begin{pgfscope}%
\pgfsys@transformshift{2.560285in}{0.866113in}%
\pgfsys@useobject{currentmarker}{}%
\end{pgfscope}%
\begin{pgfscope}%
\pgfsys@transformshift{6.601782in}{2.153191in}%
\pgfsys@useobject{currentmarker}{}%
\end{pgfscope}%
\begin{pgfscope}%
\pgfsys@transformshift{3.588372in}{3.986237in}%
\pgfsys@useobject{currentmarker}{}%
\end{pgfscope}%
\begin{pgfscope}%
\pgfsys@transformshift{5.736898in}{5.045594in}%
\pgfsys@useobject{currentmarker}{}%
\end{pgfscope}%
\begin{pgfscope}%
\pgfsys@transformshift{4.309654in}{5.016243in}%
\pgfsys@useobject{currentmarker}{}%
\end{pgfscope}%
\begin{pgfscope}%
\pgfsys@transformshift{4.099087in}{4.534596in}%
\pgfsys@useobject{currentmarker}{}%
\end{pgfscope}%
\begin{pgfscope}%
\pgfsys@transformshift{2.145398in}{4.283886in}%
\pgfsys@useobject{currentmarker}{}%
\end{pgfscope}%
\begin{pgfscope}%
\pgfsys@transformshift{2.250270in}{3.345020in}%
\pgfsys@useobject{currentmarker}{}%
\end{pgfscope}%
\begin{pgfscope}%
\pgfsys@transformshift{6.491615in}{3.548322in}%
\pgfsys@useobject{currentmarker}{}%
\end{pgfscope}%
\begin{pgfscope}%
\pgfsys@transformshift{1.068840in}{5.064836in}%
\pgfsys@useobject{currentmarker}{}%
\end{pgfscope}%
\begin{pgfscope}%
\pgfsys@transformshift{3.883485in}{0.664182in}%
\pgfsys@useobject{currentmarker}{}%
\end{pgfscope}%
\begin{pgfscope}%
\pgfsys@transformshift{2.971665in}{4.192784in}%
\pgfsys@useobject{currentmarker}{}%
\end{pgfscope}%
\begin{pgfscope}%
\pgfsys@transformshift{1.209223in}{3.839503in}%
\pgfsys@useobject{currentmarker}{}%
\end{pgfscope}%
\begin{pgfscope}%
\pgfsys@transformshift{6.871656in}{1.235874in}%
\pgfsys@useobject{currentmarker}{}%
\end{pgfscope}%
\begin{pgfscope}%
\pgfsys@transformshift{4.623291in}{5.231267in}%
\pgfsys@useobject{currentmarker}{}%
\end{pgfscope}%
\begin{pgfscope}%
\pgfsys@transformshift{5.719362in}{1.486985in}%
\pgfsys@useobject{currentmarker}{}%
\end{pgfscope}%
\begin{pgfscope}%
\pgfsys@transformshift{4.762970in}{4.961946in}%
\pgfsys@useobject{currentmarker}{}%
\end{pgfscope}%
\begin{pgfscope}%
\pgfsys@transformshift{4.886365in}{1.147861in}%
\pgfsys@useobject{currentmarker}{}%
\end{pgfscope}%
\begin{pgfscope}%
\pgfsys@transformshift{5.786494in}{4.789347in}%
\pgfsys@useobject{currentmarker}{}%
\end{pgfscope}%
\begin{pgfscope}%
\pgfsys@transformshift{6.112708in}{0.869915in}%
\pgfsys@useobject{currentmarker}{}%
\end{pgfscope}%
\begin{pgfscope}%
\pgfsys@transformshift{3.058098in}{3.065259in}%
\pgfsys@useobject{currentmarker}{}%
\end{pgfscope}%
\begin{pgfscope}%
\pgfsys@transformshift{1.419679in}{2.355247in}%
\pgfsys@useobject{currentmarker}{}%
\end{pgfscope}%
\begin{pgfscope}%
\pgfsys@transformshift{6.365833in}{4.315201in}%
\pgfsys@useobject{currentmarker}{}%
\end{pgfscope}%
\begin{pgfscope}%
\pgfsys@transformshift{1.021013in}{4.198028in}%
\pgfsys@useobject{currentmarker}{}%
\end{pgfscope}%
\begin{pgfscope}%
\pgfsys@transformshift{3.973729in}{5.516341in}%
\pgfsys@useobject{currentmarker}{}%
\end{pgfscope}%
\begin{pgfscope}%
\pgfsys@transformshift{4.944733in}{5.586916in}%
\pgfsys@useobject{currentmarker}{}%
\end{pgfscope}%
\begin{pgfscope}%
\pgfsys@transformshift{3.107162in}{4.467529in}%
\pgfsys@useobject{currentmarker}{}%
\end{pgfscope}%
\begin{pgfscope}%
\pgfsys@transformshift{5.668124in}{0.924856in}%
\pgfsys@useobject{currentmarker}{}%
\end{pgfscope}%
\begin{pgfscope}%
\pgfsys@transformshift{1.849565in}{3.214984in}%
\pgfsys@useobject{currentmarker}{}%
\end{pgfscope}%
\begin{pgfscope}%
\pgfsys@transformshift{5.738722in}{1.604631in}%
\pgfsys@useobject{currentmarker}{}%
\end{pgfscope}%
\begin{pgfscope}%
\pgfsys@transformshift{0.750418in}{1.855157in}%
\pgfsys@useobject{currentmarker}{}%
\end{pgfscope}%
\begin{pgfscope}%
\pgfsys@transformshift{2.910595in}{3.519489in}%
\pgfsys@useobject{currentmarker}{}%
\end{pgfscope}%
\begin{pgfscope}%
\pgfsys@transformshift{1.803510in}{1.618952in}%
\pgfsys@useobject{currentmarker}{}%
\end{pgfscope}%
\begin{pgfscope}%
\pgfsys@transformshift{4.522746in}{0.923843in}%
\pgfsys@useobject{currentmarker}{}%
\end{pgfscope}%
\begin{pgfscope}%
\pgfsys@transformshift{3.232716in}{2.542740in}%
\pgfsys@useobject{currentmarker}{}%
\end{pgfscope}%
\begin{pgfscope}%
\pgfsys@transformshift{1.440932in}{1.007042in}%
\pgfsys@useobject{currentmarker}{}%
\end{pgfscope}%
\begin{pgfscope}%
\pgfsys@transformshift{3.097251in}{4.226800in}%
\pgfsys@useobject{currentmarker}{}%
\end{pgfscope}%
\begin{pgfscope}%
\pgfsys@transformshift{4.639182in}{4.513124in}%
\pgfsys@useobject{currentmarker}{}%
\end{pgfscope}%
\begin{pgfscope}%
\pgfsys@transformshift{4.729855in}{0.475843in}%
\pgfsys@useobject{currentmarker}{}%
\end{pgfscope}%
\begin{pgfscope}%
\pgfsys@transformshift{5.652688in}{1.824843in}%
\pgfsys@useobject{currentmarker}{}%
\end{pgfscope}%
\begin{pgfscope}%
\pgfsys@transformshift{3.226216in}{1.426385in}%
\pgfsys@useobject{currentmarker}{}%
\end{pgfscope}%
\begin{pgfscope}%
\pgfsys@transformshift{5.167749in}{4.568793in}%
\pgfsys@useobject{currentmarker}{}%
\end{pgfscope}%
\begin{pgfscope}%
\pgfsys@transformshift{6.010328in}{1.606239in}%
\pgfsys@useobject{currentmarker}{}%
\end{pgfscope}%
\begin{pgfscope}%
\pgfsys@transformshift{4.470339in}{1.975688in}%
\pgfsys@useobject{currentmarker}{}%
\end{pgfscope}%
\begin{pgfscope}%
\pgfsys@transformshift{5.772001in}{0.915793in}%
\pgfsys@useobject{currentmarker}{}%
\end{pgfscope}%
\begin{pgfscope}%
\pgfsys@transformshift{5.528211in}{3.267157in}%
\pgfsys@useobject{currentmarker}{}%
\end{pgfscope}%
\begin{pgfscope}%
\pgfsys@transformshift{1.920141in}{0.399767in}%
\pgfsys@useobject{currentmarker}{}%
\end{pgfscope}%
\begin{pgfscope}%
\pgfsys@transformshift{1.233618in}{4.969010in}%
\pgfsys@useobject{currentmarker}{}%
\end{pgfscope}%
\begin{pgfscope}%
\pgfsys@transformshift{5.169037in}{4.325928in}%
\pgfsys@useobject{currentmarker}{}%
\end{pgfscope}%
\begin{pgfscope}%
\pgfsys@transformshift{4.070491in}{2.299736in}%
\pgfsys@useobject{currentmarker}{}%
\end{pgfscope}%
\begin{pgfscope}%
\pgfsys@transformshift{3.110229in}{0.525900in}%
\pgfsys@useobject{currentmarker}{}%
\end{pgfscope}%
\begin{pgfscope}%
\pgfsys@transformshift{4.423411in}{3.893190in}%
\pgfsys@useobject{currentmarker}{}%
\end{pgfscope}%
\begin{pgfscope}%
\pgfsys@transformshift{1.296334in}{4.701201in}%
\pgfsys@useobject{currentmarker}{}%
\end{pgfscope}%
\begin{pgfscope}%
\pgfsys@transformshift{4.933271in}{5.359361in}%
\pgfsys@useobject{currentmarker}{}%
\end{pgfscope}%
\begin{pgfscope}%
\pgfsys@transformshift{6.616783in}{4.755727in}%
\pgfsys@useobject{currentmarker}{}%
\end{pgfscope}%
\begin{pgfscope}%
\pgfsys@transformshift{3.774014in}{2.483044in}%
\pgfsys@useobject{currentmarker}{}%
\end{pgfscope}%
\begin{pgfscope}%
\pgfsys@transformshift{0.714621in}{2.239258in}%
\pgfsys@useobject{currentmarker}{}%
\end{pgfscope}%
\begin{pgfscope}%
\pgfsys@transformshift{1.455364in}{0.361922in}%
\pgfsys@useobject{currentmarker}{}%
\end{pgfscope}%
\begin{pgfscope}%
\pgfsys@transformshift{4.502330in}{3.441172in}%
\pgfsys@useobject{currentmarker}{}%
\end{pgfscope}%
\begin{pgfscope}%
\pgfsys@transformshift{5.177961in}{3.455854in}%
\pgfsys@useobject{currentmarker}{}%
\end{pgfscope}%
\begin{pgfscope}%
\pgfsys@transformshift{4.047061in}{3.342995in}%
\pgfsys@useobject{currentmarker}{}%
\end{pgfscope}%
\begin{pgfscope}%
\pgfsys@transformshift{3.295777in}{4.783609in}%
\pgfsys@useobject{currentmarker}{}%
\end{pgfscope}%
\begin{pgfscope}%
\pgfsys@transformshift{1.063543in}{0.413376in}%
\pgfsys@useobject{currentmarker}{}%
\end{pgfscope}%
\begin{pgfscope}%
\pgfsys@transformshift{0.783832in}{2.739907in}%
\pgfsys@useobject{currentmarker}{}%
\end{pgfscope}%
\begin{pgfscope}%
\pgfsys@transformshift{5.078126in}{4.404576in}%
\pgfsys@useobject{currentmarker}{}%
\end{pgfscope}%
\begin{pgfscope}%
\pgfsys@transformshift{4.136988in}{4.658145in}%
\pgfsys@useobject{currentmarker}{}%
\end{pgfscope}%
\begin{pgfscope}%
\pgfsys@transformshift{5.736855in}{1.292936in}%
\pgfsys@useobject{currentmarker}{}%
\end{pgfscope}%
\begin{pgfscope}%
\pgfsys@transformshift{1.983580in}{4.886320in}%
\pgfsys@useobject{currentmarker}{}%
\end{pgfscope}%
\begin{pgfscope}%
\pgfsys@transformshift{6.873462in}{1.295461in}%
\pgfsys@useobject{currentmarker}{}%
\end{pgfscope}%
\begin{pgfscope}%
\pgfsys@transformshift{4.409221in}{1.675381in}%
\pgfsys@useobject{currentmarker}{}%
\end{pgfscope}%
\begin{pgfscope}%
\pgfsys@transformshift{4.047554in}{5.293867in}%
\pgfsys@useobject{currentmarker}{}%
\end{pgfscope}%
\begin{pgfscope}%
\pgfsys@transformshift{1.998868in}{0.481306in}%
\pgfsys@useobject{currentmarker}{}%
\end{pgfscope}%
\begin{pgfscope}%
\pgfsys@transformshift{4.261880in}{4.741855in}%
\pgfsys@useobject{currentmarker}{}%
\end{pgfscope}%
\begin{pgfscope}%
\pgfsys@transformshift{3.995381in}{3.332257in}%
\pgfsys@useobject{currentmarker}{}%
\end{pgfscope}%
\begin{pgfscope}%
\pgfsys@transformshift{4.523514in}{2.854564in}%
\pgfsys@useobject{currentmarker}{}%
\end{pgfscope}%
\begin{pgfscope}%
\pgfsys@transformshift{1.056228in}{2.867376in}%
\pgfsys@useobject{currentmarker}{}%
\end{pgfscope}%
\begin{pgfscope}%
\pgfsys@transformshift{4.359336in}{0.514376in}%
\pgfsys@useobject{currentmarker}{}%
\end{pgfscope}%
\begin{pgfscope}%
\pgfsys@transformshift{4.808662in}{4.480181in}%
\pgfsys@useobject{currentmarker}{}%
\end{pgfscope}%
\begin{pgfscope}%
\pgfsys@transformshift{6.488879in}{4.097878in}%
\pgfsys@useobject{currentmarker}{}%
\end{pgfscope}%
\begin{pgfscope}%
\pgfsys@transformshift{6.287794in}{4.801690in}%
\pgfsys@useobject{currentmarker}{}%
\end{pgfscope}%
\begin{pgfscope}%
\pgfsys@transformshift{4.998063in}{2.004002in}%
\pgfsys@useobject{currentmarker}{}%
\end{pgfscope}%
\begin{pgfscope}%
\pgfsys@transformshift{0.985787in}{5.511862in}%
\pgfsys@useobject{currentmarker}{}%
\end{pgfscope}%
\begin{pgfscope}%
\pgfsys@transformshift{4.566527in}{5.375120in}%
\pgfsys@useobject{currentmarker}{}%
\end{pgfscope}%
\begin{pgfscope}%
\pgfsys@transformshift{1.608226in}{1.416583in}%
\pgfsys@useobject{currentmarker}{}%
\end{pgfscope}%
\begin{pgfscope}%
\pgfsys@transformshift{4.195638in}{1.749215in}%
\pgfsys@useobject{currentmarker}{}%
\end{pgfscope}%
\begin{pgfscope}%
\pgfsys@transformshift{0.763440in}{3.773630in}%
\pgfsys@useobject{currentmarker}{}%
\end{pgfscope}%
\begin{pgfscope}%
\pgfsys@transformshift{1.843265in}{4.852228in}%
\pgfsys@useobject{currentmarker}{}%
\end{pgfscope}%
\begin{pgfscope}%
\pgfsys@transformshift{1.275953in}{2.071698in}%
\pgfsys@useobject{currentmarker}{}%
\end{pgfscope}%
\begin{pgfscope}%
\pgfsys@transformshift{2.621490in}{1.658213in}%
\pgfsys@useobject{currentmarker}{}%
\end{pgfscope}%
\begin{pgfscope}%
\pgfsys@transformshift{4.138927in}{3.125188in}%
\pgfsys@useobject{currentmarker}{}%
\end{pgfscope}%
\begin{pgfscope}%
\pgfsys@transformshift{1.196307in}{2.321273in}%
\pgfsys@useobject{currentmarker}{}%
\end{pgfscope}%
\begin{pgfscope}%
\pgfsys@transformshift{3.514381in}{4.916843in}%
\pgfsys@useobject{currentmarker}{}%
\end{pgfscope}%
\begin{pgfscope}%
\pgfsys@transformshift{5.521781in}{4.691254in}%
\pgfsys@useobject{currentmarker}{}%
\end{pgfscope}%
\begin{pgfscope}%
\pgfsys@transformshift{4.495839in}{2.987808in}%
\pgfsys@useobject{currentmarker}{}%
\end{pgfscope}%
\begin{pgfscope}%
\pgfsys@transformshift{6.548739in}{2.904807in}%
\pgfsys@useobject{currentmarker}{}%
\end{pgfscope}%
\begin{pgfscope}%
\pgfsys@transformshift{3.592530in}{3.343022in}%
\pgfsys@useobject{currentmarker}{}%
\end{pgfscope}%
\begin{pgfscope}%
\pgfsys@transformshift{2.527701in}{5.010536in}%
\pgfsys@useobject{currentmarker}{}%
\end{pgfscope}%
\begin{pgfscope}%
\pgfsys@transformshift{3.904015in}{3.203000in}%
\pgfsys@useobject{currentmarker}{}%
\end{pgfscope}%
\begin{pgfscope}%
\pgfsys@transformshift{2.004440in}{3.186267in}%
\pgfsys@useobject{currentmarker}{}%
\end{pgfscope}%
\begin{pgfscope}%
\pgfsys@transformshift{3.898349in}{1.531822in}%
\pgfsys@useobject{currentmarker}{}%
\end{pgfscope}%
\begin{pgfscope}%
\pgfsys@transformshift{3.307557in}{0.739362in}%
\pgfsys@useobject{currentmarker}{}%
\end{pgfscope}%
\begin{pgfscope}%
\pgfsys@transformshift{2.013122in}{5.165292in}%
\pgfsys@useobject{currentmarker}{}%
\end{pgfscope}%
\begin{pgfscope}%
\pgfsys@transformshift{6.131114in}{4.719844in}%
\pgfsys@useobject{currentmarker}{}%
\end{pgfscope}%
\begin{pgfscope}%
\pgfsys@transformshift{5.146627in}{1.968083in}%
\pgfsys@useobject{currentmarker}{}%
\end{pgfscope}%
\begin{pgfscope}%
\pgfsys@transformshift{1.105166in}{4.174348in}%
\pgfsys@useobject{currentmarker}{}%
\end{pgfscope}%
\begin{pgfscope}%
\pgfsys@transformshift{1.727846in}{3.988167in}%
\pgfsys@useobject{currentmarker}{}%
\end{pgfscope}%
\begin{pgfscope}%
\pgfsys@transformshift{5.477223in}{2.268647in}%
\pgfsys@useobject{currentmarker}{}%
\end{pgfscope}%
\begin{pgfscope}%
\pgfsys@transformshift{0.959381in}{3.622332in}%
\pgfsys@useobject{currentmarker}{}%
\end{pgfscope}%
\begin{pgfscope}%
\pgfsys@transformshift{3.248512in}{0.542400in}%
\pgfsys@useobject{currentmarker}{}%
\end{pgfscope}%
\begin{pgfscope}%
\pgfsys@transformshift{3.129105in}{5.246287in}%
\pgfsys@useobject{currentmarker}{}%
\end{pgfscope}%
\begin{pgfscope}%
\pgfsys@transformshift{1.943354in}{4.893697in}%
\pgfsys@useobject{currentmarker}{}%
\end{pgfscope}%
\begin{pgfscope}%
\pgfsys@transformshift{3.587215in}{3.580868in}%
\pgfsys@useobject{currentmarker}{}%
\end{pgfscope}%
\begin{pgfscope}%
\pgfsys@transformshift{2.342935in}{2.097338in}%
\pgfsys@useobject{currentmarker}{}%
\end{pgfscope}%
\begin{pgfscope}%
\pgfsys@transformshift{1.021497in}{0.690499in}%
\pgfsys@useobject{currentmarker}{}%
\end{pgfscope}%
\begin{pgfscope}%
\pgfsys@transformshift{3.817014in}{1.576762in}%
\pgfsys@useobject{currentmarker}{}%
\end{pgfscope}%
\begin{pgfscope}%
\pgfsys@transformshift{5.442389in}{2.100678in}%
\pgfsys@useobject{currentmarker}{}%
\end{pgfscope}%
\begin{pgfscope}%
\pgfsys@transformshift{6.058288in}{4.673716in}%
\pgfsys@useobject{currentmarker}{}%
\end{pgfscope}%
\begin{pgfscope}%
\pgfsys@transformshift{1.979398in}{0.840928in}%
\pgfsys@useobject{currentmarker}{}%
\end{pgfscope}%
\begin{pgfscope}%
\pgfsys@transformshift{2.648712in}{3.247655in}%
\pgfsys@useobject{currentmarker}{}%
\end{pgfscope}%
\begin{pgfscope}%
\pgfsys@transformshift{5.949050in}{5.533644in}%
\pgfsys@useobject{currentmarker}{}%
\end{pgfscope}%
\begin{pgfscope}%
\pgfsys@transformshift{5.550495in}{1.991920in}%
\pgfsys@useobject{currentmarker}{}%
\end{pgfscope}%
\begin{pgfscope}%
\pgfsys@transformshift{6.755225in}{5.103245in}%
\pgfsys@useobject{currentmarker}{}%
\end{pgfscope}%
\begin{pgfscope}%
\pgfsys@transformshift{3.865731in}{2.334149in}%
\pgfsys@useobject{currentmarker}{}%
\end{pgfscope}%
\begin{pgfscope}%
\pgfsys@transformshift{2.976125in}{5.561992in}%
\pgfsys@useobject{currentmarker}{}%
\end{pgfscope}%
\begin{pgfscope}%
\pgfsys@transformshift{3.081797in}{3.709330in}%
\pgfsys@useobject{currentmarker}{}%
\end{pgfscope}%
\begin{pgfscope}%
\pgfsys@transformshift{1.382073in}{0.821854in}%
\pgfsys@useobject{currentmarker}{}%
\end{pgfscope}%
\begin{pgfscope}%
\pgfsys@transformshift{5.184168in}{4.949794in}%
\pgfsys@useobject{currentmarker}{}%
\end{pgfscope}%
\begin{pgfscope}%
\pgfsys@transformshift{4.142106in}{4.831008in}%
\pgfsys@useobject{currentmarker}{}%
\end{pgfscope}%
\begin{pgfscope}%
\pgfsys@transformshift{0.655059in}{2.256314in}%
\pgfsys@useobject{currentmarker}{}%
\end{pgfscope}%
\begin{pgfscope}%
\pgfsys@transformshift{0.802744in}{2.327255in}%
\pgfsys@useobject{currentmarker}{}%
\end{pgfscope}%
\begin{pgfscope}%
\pgfsys@transformshift{2.579353in}{2.170771in}%
\pgfsys@useobject{currentmarker}{}%
\end{pgfscope}%
\begin{pgfscope}%
\pgfsys@transformshift{1.585746in}{4.271890in}%
\pgfsys@useobject{currentmarker}{}%
\end{pgfscope}%
\begin{pgfscope}%
\pgfsys@transformshift{6.865968in}{0.230952in}%
\pgfsys@useobject{currentmarker}{}%
\end{pgfscope}%
\begin{pgfscope}%
\pgfsys@transformshift{4.662028in}{2.002836in}%
\pgfsys@useobject{currentmarker}{}%
\end{pgfscope}%
\begin{pgfscope}%
\pgfsys@transformshift{4.027602in}{2.510529in}%
\pgfsys@useobject{currentmarker}{}%
\end{pgfscope}%
\begin{pgfscope}%
\pgfsys@transformshift{6.703161in}{1.286896in}%
\pgfsys@useobject{currentmarker}{}%
\end{pgfscope}%
\begin{pgfscope}%
\pgfsys@transformshift{4.412214in}{4.151845in}%
\pgfsys@useobject{currentmarker}{}%
\end{pgfscope}%
\begin{pgfscope}%
\pgfsys@transformshift{1.445357in}{3.910361in}%
\pgfsys@useobject{currentmarker}{}%
\end{pgfscope}%
\begin{pgfscope}%
\pgfsys@transformshift{5.808650in}{2.612827in}%
\pgfsys@useobject{currentmarker}{}%
\end{pgfscope}%
\begin{pgfscope}%
\pgfsys@transformshift{1.220703in}{0.896912in}%
\pgfsys@useobject{currentmarker}{}%
\end{pgfscope}%
\begin{pgfscope}%
\pgfsys@transformshift{3.603961in}{4.178180in}%
\pgfsys@useobject{currentmarker}{}%
\end{pgfscope}%
\begin{pgfscope}%
\pgfsys@transformshift{4.539430in}{0.774317in}%
\pgfsys@useobject{currentmarker}{}%
\end{pgfscope}%
\begin{pgfscope}%
\pgfsys@transformshift{2.716342in}{1.761126in}%
\pgfsys@useobject{currentmarker}{}%
\end{pgfscope}%
\begin{pgfscope}%
\pgfsys@transformshift{2.797144in}{0.838678in}%
\pgfsys@useobject{currentmarker}{}%
\end{pgfscope}%
\begin{pgfscope}%
\pgfsys@transformshift{3.762384in}{2.095525in}%
\pgfsys@useobject{currentmarker}{}%
\end{pgfscope}%
\begin{pgfscope}%
\pgfsys@transformshift{5.164025in}{0.672611in}%
\pgfsys@useobject{currentmarker}{}%
\end{pgfscope}%
\begin{pgfscope}%
\pgfsys@transformshift{1.870542in}{0.457542in}%
\pgfsys@useobject{currentmarker}{}%
\end{pgfscope}%
\begin{pgfscope}%
\pgfsys@transformshift{2.910121in}{4.505037in}%
\pgfsys@useobject{currentmarker}{}%
\end{pgfscope}%
\begin{pgfscope}%
\pgfsys@transformshift{1.411858in}{2.496224in}%
\pgfsys@useobject{currentmarker}{}%
\end{pgfscope}%
\begin{pgfscope}%
\pgfsys@transformshift{4.895000in}{0.989972in}%
\pgfsys@useobject{currentmarker}{}%
\end{pgfscope}%
\begin{pgfscope}%
\pgfsys@transformshift{0.738002in}{0.308557in}%
\pgfsys@useobject{currentmarker}{}%
\end{pgfscope}%
\begin{pgfscope}%
\pgfsys@transformshift{4.221218in}{1.777841in}%
\pgfsys@useobject{currentmarker}{}%
\end{pgfscope}%
\begin{pgfscope}%
\pgfsys@transformshift{4.841482in}{1.341076in}%
\pgfsys@useobject{currentmarker}{}%
\end{pgfscope}%
\begin{pgfscope}%
\pgfsys@transformshift{3.825316in}{5.064443in}%
\pgfsys@useobject{currentmarker}{}%
\end{pgfscope}%
\begin{pgfscope}%
\pgfsys@transformshift{4.349185in}{3.651545in}%
\pgfsys@useobject{currentmarker}{}%
\end{pgfscope}%
\begin{pgfscope}%
\pgfsys@transformshift{6.614449in}{2.078768in}%
\pgfsys@useobject{currentmarker}{}%
\end{pgfscope}%
\begin{pgfscope}%
\pgfsys@transformshift{3.440025in}{3.712999in}%
\pgfsys@useobject{currentmarker}{}%
\end{pgfscope}%
\begin{pgfscope}%
\pgfsys@transformshift{3.524073in}{3.416601in}%
\pgfsys@useobject{currentmarker}{}%
\end{pgfscope}%
\begin{pgfscope}%
\pgfsys@transformshift{0.656534in}{3.158901in}%
\pgfsys@useobject{currentmarker}{}%
\end{pgfscope}%
\begin{pgfscope}%
\pgfsys@transformshift{2.233883in}{2.221251in}%
\pgfsys@useobject{currentmarker}{}%
\end{pgfscope}%
\begin{pgfscope}%
\pgfsys@transformshift{5.568420in}{1.991570in}%
\pgfsys@useobject{currentmarker}{}%
\end{pgfscope}%
\begin{pgfscope}%
\pgfsys@transformshift{6.063286in}{3.907894in}%
\pgfsys@useobject{currentmarker}{}%
\end{pgfscope}%
\begin{pgfscope}%
\pgfsys@transformshift{5.342023in}{3.563392in}%
\pgfsys@useobject{currentmarker}{}%
\end{pgfscope}%
\begin{pgfscope}%
\pgfsys@transformshift{6.739944in}{5.157110in}%
\pgfsys@useobject{currentmarker}{}%
\end{pgfscope}%
\begin{pgfscope}%
\pgfsys@transformshift{6.204109in}{4.336997in}%
\pgfsys@useobject{currentmarker}{}%
\end{pgfscope}%
\begin{pgfscope}%
\pgfsys@transformshift{0.896520in}{0.404529in}%
\pgfsys@useobject{currentmarker}{}%
\end{pgfscope}%
\begin{pgfscope}%
\pgfsys@transformshift{5.692888in}{5.529288in}%
\pgfsys@useobject{currentmarker}{}%
\end{pgfscope}%
\begin{pgfscope}%
\pgfsys@transformshift{4.756665in}{2.321836in}%
\pgfsys@useobject{currentmarker}{}%
\end{pgfscope}%
\begin{pgfscope}%
\pgfsys@transformshift{4.620765in}{1.241046in}%
\pgfsys@useobject{currentmarker}{}%
\end{pgfscope}%
\begin{pgfscope}%
\pgfsys@transformshift{3.198662in}{2.226083in}%
\pgfsys@useobject{currentmarker}{}%
\end{pgfscope}%
\begin{pgfscope}%
\pgfsys@transformshift{2.414933in}{1.108528in}%
\pgfsys@useobject{currentmarker}{}%
\end{pgfscope}%
\begin{pgfscope}%
\pgfsys@transformshift{6.506654in}{5.601735in}%
\pgfsys@useobject{currentmarker}{}%
\end{pgfscope}%
\begin{pgfscope}%
\pgfsys@transformshift{4.142466in}{3.244470in}%
\pgfsys@useobject{currentmarker}{}%
\end{pgfscope}%
\begin{pgfscope}%
\pgfsys@transformshift{6.658868in}{3.905016in}%
\pgfsys@useobject{currentmarker}{}%
\end{pgfscope}%
\begin{pgfscope}%
\pgfsys@transformshift{0.934661in}{1.771176in}%
\pgfsys@useobject{currentmarker}{}%
\end{pgfscope}%
\begin{pgfscope}%
\pgfsys@transformshift{3.154612in}{2.342803in}%
\pgfsys@useobject{currentmarker}{}%
\end{pgfscope}%
\begin{pgfscope}%
\pgfsys@transformshift{2.002356in}{2.343359in}%
\pgfsys@useobject{currentmarker}{}%
\end{pgfscope}%
\begin{pgfscope}%
\pgfsys@transformshift{5.090432in}{4.517353in}%
\pgfsys@useobject{currentmarker}{}%
\end{pgfscope}%
\begin{pgfscope}%
\pgfsys@transformshift{6.080457in}{1.202822in}%
\pgfsys@useobject{currentmarker}{}%
\end{pgfscope}%
\begin{pgfscope}%
\pgfsys@transformshift{3.323354in}{1.022890in}%
\pgfsys@useobject{currentmarker}{}%
\end{pgfscope}%
\begin{pgfscope}%
\pgfsys@transformshift{2.354026in}{0.531485in}%
\pgfsys@useobject{currentmarker}{}%
\end{pgfscope}%
\begin{pgfscope}%
\pgfsys@transformshift{4.593579in}{1.800996in}%
\pgfsys@useobject{currentmarker}{}%
\end{pgfscope}%
\begin{pgfscope}%
\pgfsys@transformshift{6.272463in}{1.782078in}%
\pgfsys@useobject{currentmarker}{}%
\end{pgfscope}%
\begin{pgfscope}%
\pgfsys@transformshift{1.152253in}{2.650657in}%
\pgfsys@useobject{currentmarker}{}%
\end{pgfscope}%
\begin{pgfscope}%
\pgfsys@transformshift{1.944140in}{3.576537in}%
\pgfsys@useobject{currentmarker}{}%
\end{pgfscope}%
\begin{pgfscope}%
\pgfsys@transformshift{3.994197in}{5.242610in}%
\pgfsys@useobject{currentmarker}{}%
\end{pgfscope}%
\begin{pgfscope}%
\pgfsys@transformshift{2.653452in}{4.274385in}%
\pgfsys@useobject{currentmarker}{}%
\end{pgfscope}%
\begin{pgfscope}%
\pgfsys@transformshift{5.626037in}{0.483404in}%
\pgfsys@useobject{currentmarker}{}%
\end{pgfscope}%
\begin{pgfscope}%
\pgfsys@transformshift{2.255638in}{4.246827in}%
\pgfsys@useobject{currentmarker}{}%
\end{pgfscope}%
\begin{pgfscope}%
\pgfsys@transformshift{4.652513in}{3.859054in}%
\pgfsys@useobject{currentmarker}{}%
\end{pgfscope}%
\begin{pgfscope}%
\pgfsys@transformshift{3.277927in}{0.472024in}%
\pgfsys@useobject{currentmarker}{}%
\end{pgfscope}%
\begin{pgfscope}%
\pgfsys@transformshift{2.053643in}{2.791624in}%
\pgfsys@useobject{currentmarker}{}%
\end{pgfscope}%
\begin{pgfscope}%
\pgfsys@transformshift{3.166837in}{4.829729in}%
\pgfsys@useobject{currentmarker}{}%
\end{pgfscope}%
\begin{pgfscope}%
\pgfsys@transformshift{5.976142in}{5.391060in}%
\pgfsys@useobject{currentmarker}{}%
\end{pgfscope}%
\begin{pgfscope}%
\pgfsys@transformshift{6.825375in}{5.032743in}%
\pgfsys@useobject{currentmarker}{}%
\end{pgfscope}%
\begin{pgfscope}%
\pgfsys@transformshift{4.646199in}{2.772443in}%
\pgfsys@useobject{currentmarker}{}%
\end{pgfscope}%
\begin{pgfscope}%
\pgfsys@transformshift{1.824536in}{0.631393in}%
\pgfsys@useobject{currentmarker}{}%
\end{pgfscope}%
\begin{pgfscope}%
\pgfsys@transformshift{4.222637in}{3.845652in}%
\pgfsys@useobject{currentmarker}{}%
\end{pgfscope}%
\begin{pgfscope}%
\pgfsys@transformshift{4.493681in}{0.668277in}%
\pgfsys@useobject{currentmarker}{}%
\end{pgfscope}%
\begin{pgfscope}%
\pgfsys@transformshift{5.758562in}{0.396976in}%
\pgfsys@useobject{currentmarker}{}%
\end{pgfscope}%
\begin{pgfscope}%
\pgfsys@transformshift{4.139112in}{0.785824in}%
\pgfsys@useobject{currentmarker}{}%
\end{pgfscope}%
\begin{pgfscope}%
\pgfsys@transformshift{6.889347in}{3.604728in}%
\pgfsys@useobject{currentmarker}{}%
\end{pgfscope}%
\begin{pgfscope}%
\pgfsys@transformshift{2.493739in}{4.962428in}%
\pgfsys@useobject{currentmarker}{}%
\end{pgfscope}%
\begin{pgfscope}%
\pgfsys@transformshift{4.887080in}{1.597741in}%
\pgfsys@useobject{currentmarker}{}%
\end{pgfscope}%
\begin{pgfscope}%
\pgfsys@transformshift{6.166293in}{5.525876in}%
\pgfsys@useobject{currentmarker}{}%
\end{pgfscope}%
\begin{pgfscope}%
\pgfsys@transformshift{3.989516in}{3.609263in}%
\pgfsys@useobject{currentmarker}{}%
\end{pgfscope}%
\begin{pgfscope}%
\pgfsys@transformshift{3.344430in}{0.694691in}%
\pgfsys@useobject{currentmarker}{}%
\end{pgfscope}%
\begin{pgfscope}%
\pgfsys@transformshift{6.075360in}{0.974671in}%
\pgfsys@useobject{currentmarker}{}%
\end{pgfscope}%
\begin{pgfscope}%
\pgfsys@transformshift{4.786806in}{2.685359in}%
\pgfsys@useobject{currentmarker}{}%
\end{pgfscope}%
\begin{pgfscope}%
\pgfsys@transformshift{2.275116in}{4.214604in}%
\pgfsys@useobject{currentmarker}{}%
\end{pgfscope}%
\begin{pgfscope}%
\pgfsys@transformshift{5.820529in}{4.658387in}%
\pgfsys@useobject{currentmarker}{}%
\end{pgfscope}%
\begin{pgfscope}%
\pgfsys@transformshift{1.870196in}{3.240343in}%
\pgfsys@useobject{currentmarker}{}%
\end{pgfscope}%
\begin{pgfscope}%
\pgfsys@transformshift{1.672379in}{4.284370in}%
\pgfsys@useobject{currentmarker}{}%
\end{pgfscope}%
\begin{pgfscope}%
\pgfsys@transformshift{2.302752in}{4.962718in}%
\pgfsys@useobject{currentmarker}{}%
\end{pgfscope}%
\begin{pgfscope}%
\pgfsys@transformshift{0.673143in}{0.893077in}%
\pgfsys@useobject{currentmarker}{}%
\end{pgfscope}%
\begin{pgfscope}%
\pgfsys@transformshift{1.263833in}{1.611700in}%
\pgfsys@useobject{currentmarker}{}%
\end{pgfscope}%
\begin{pgfscope}%
\pgfsys@transformshift{4.282282in}{4.761581in}%
\pgfsys@useobject{currentmarker}{}%
\end{pgfscope}%
\begin{pgfscope}%
\pgfsys@transformshift{4.622965in}{3.963335in}%
\pgfsys@useobject{currentmarker}{}%
\end{pgfscope}%
\begin{pgfscope}%
\pgfsys@transformshift{1.301933in}{1.347133in}%
\pgfsys@useobject{currentmarker}{}%
\end{pgfscope}%
\begin{pgfscope}%
\pgfsys@transformshift{5.734062in}{3.964222in}%
\pgfsys@useobject{currentmarker}{}%
\end{pgfscope}%
\begin{pgfscope}%
\pgfsys@transformshift{6.543641in}{1.069784in}%
\pgfsys@useobject{currentmarker}{}%
\end{pgfscope}%
\begin{pgfscope}%
\pgfsys@transformshift{1.455448in}{1.633203in}%
\pgfsys@useobject{currentmarker}{}%
\end{pgfscope}%
\begin{pgfscope}%
\pgfsys@transformshift{5.604056in}{2.970101in}%
\pgfsys@useobject{currentmarker}{}%
\end{pgfscope}%
\begin{pgfscope}%
\pgfsys@transformshift{4.489020in}{1.154589in}%
\pgfsys@useobject{currentmarker}{}%
\end{pgfscope}%
\begin{pgfscope}%
\pgfsys@transformshift{4.421605in}{3.229968in}%
\pgfsys@useobject{currentmarker}{}%
\end{pgfscope}%
\begin{pgfscope}%
\pgfsys@transformshift{4.858586in}{2.609125in}%
\pgfsys@useobject{currentmarker}{}%
\end{pgfscope}%
\begin{pgfscope}%
\pgfsys@transformshift{4.648877in}{2.516551in}%
\pgfsys@useobject{currentmarker}{}%
\end{pgfscope}%
\begin{pgfscope}%
\pgfsys@transformshift{4.130094in}{1.571152in}%
\pgfsys@useobject{currentmarker}{}%
\end{pgfscope}%
\begin{pgfscope}%
\pgfsys@transformshift{0.896027in}{1.189427in}%
\pgfsys@useobject{currentmarker}{}%
\end{pgfscope}%
\begin{pgfscope}%
\pgfsys@transformshift{1.847530in}{0.612260in}%
\pgfsys@useobject{currentmarker}{}%
\end{pgfscope}%
\begin{pgfscope}%
\pgfsys@transformshift{6.351608in}{0.972429in}%
\pgfsys@useobject{currentmarker}{}%
\end{pgfscope}%
\begin{pgfscope}%
\pgfsys@transformshift{6.134735in}{5.352141in}%
\pgfsys@useobject{currentmarker}{}%
\end{pgfscope}%
\begin{pgfscope}%
\pgfsys@transformshift{6.182954in}{5.595977in}%
\pgfsys@useobject{currentmarker}{}%
\end{pgfscope}%
\begin{pgfscope}%
\pgfsys@transformshift{6.046383in}{4.049364in}%
\pgfsys@useobject{currentmarker}{}%
\end{pgfscope}%
\begin{pgfscope}%
\pgfsys@transformshift{5.741789in}{2.214999in}%
\pgfsys@useobject{currentmarker}{}%
\end{pgfscope}%
\begin{pgfscope}%
\pgfsys@transformshift{2.504074in}{4.901585in}%
\pgfsys@useobject{currentmarker}{}%
\end{pgfscope}%
\begin{pgfscope}%
\pgfsys@transformshift{4.351859in}{0.591350in}%
\pgfsys@useobject{currentmarker}{}%
\end{pgfscope}%
\begin{pgfscope}%
\pgfsys@transformshift{3.783058in}{2.728852in}%
\pgfsys@useobject{currentmarker}{}%
\end{pgfscope}%
\begin{pgfscope}%
\pgfsys@transformshift{6.770911in}{3.250488in}%
\pgfsys@useobject{currentmarker}{}%
\end{pgfscope}%
\begin{pgfscope}%
\pgfsys@transformshift{6.342129in}{0.486091in}%
\pgfsys@useobject{currentmarker}{}%
\end{pgfscope}%
\begin{pgfscope}%
\pgfsys@transformshift{2.417289in}{3.706067in}%
\pgfsys@useobject{currentmarker}{}%
\end{pgfscope}%
\begin{pgfscope}%
\pgfsys@transformshift{4.438185in}{3.896967in}%
\pgfsys@useobject{currentmarker}{}%
\end{pgfscope}%
\begin{pgfscope}%
\pgfsys@transformshift{1.602751in}{2.048253in}%
\pgfsys@useobject{currentmarker}{}%
\end{pgfscope}%
\begin{pgfscope}%
\pgfsys@transformshift{5.482158in}{0.576145in}%
\pgfsys@useobject{currentmarker}{}%
\end{pgfscope}%
\begin{pgfscope}%
\pgfsys@transformshift{2.180598in}{2.896309in}%
\pgfsys@useobject{currentmarker}{}%
\end{pgfscope}%
\begin{pgfscope}%
\pgfsys@transformshift{1.404368in}{0.948406in}%
\pgfsys@useobject{currentmarker}{}%
\end{pgfscope}%
\begin{pgfscope}%
\pgfsys@transformshift{4.519508in}{2.969509in}%
\pgfsys@useobject{currentmarker}{}%
\end{pgfscope}%
\begin{pgfscope}%
\pgfsys@transformshift{2.527479in}{4.051990in}%
\pgfsys@useobject{currentmarker}{}%
\end{pgfscope}%
\begin{pgfscope}%
\pgfsys@transformshift{3.248174in}{0.965477in}%
\pgfsys@useobject{currentmarker}{}%
\end{pgfscope}%
\begin{pgfscope}%
\pgfsys@transformshift{3.839753in}{0.806067in}%
\pgfsys@useobject{currentmarker}{}%
\end{pgfscope}%
\begin{pgfscope}%
\pgfsys@transformshift{6.845770in}{3.650208in}%
\pgfsys@useobject{currentmarker}{}%
\end{pgfscope}%
\begin{pgfscope}%
\pgfsys@transformshift{1.617632in}{3.341566in}%
\pgfsys@useobject{currentmarker}{}%
\end{pgfscope}%
\begin{pgfscope}%
\pgfsys@transformshift{2.896925in}{4.283457in}%
\pgfsys@useobject{currentmarker}{}%
\end{pgfscope}%
\begin{pgfscope}%
\pgfsys@transformshift{6.327943in}{3.286432in}%
\pgfsys@useobject{currentmarker}{}%
\end{pgfscope}%
\begin{pgfscope}%
\pgfsys@transformshift{3.189071in}{0.912943in}%
\pgfsys@useobject{currentmarker}{}%
\end{pgfscope}%
\begin{pgfscope}%
\pgfsys@transformshift{2.362344in}{2.960396in}%
\pgfsys@useobject{currentmarker}{}%
\end{pgfscope}%
\begin{pgfscope}%
\pgfsys@transformshift{2.429558in}{1.668869in}%
\pgfsys@useobject{currentmarker}{}%
\end{pgfscope}%
\begin{pgfscope}%
\pgfsys@transformshift{3.106341in}{1.308808in}%
\pgfsys@useobject{currentmarker}{}%
\end{pgfscope}%
\begin{pgfscope}%
\pgfsys@transformshift{3.037818in}{3.917910in}%
\pgfsys@useobject{currentmarker}{}%
\end{pgfscope}%
\begin{pgfscope}%
\pgfsys@transformshift{5.310423in}{0.359517in}%
\pgfsys@useobject{currentmarker}{}%
\end{pgfscope}%
\begin{pgfscope}%
\pgfsys@transformshift{5.540275in}{4.582562in}%
\pgfsys@useobject{currentmarker}{}%
\end{pgfscope}%
\begin{pgfscope}%
\pgfsys@transformshift{4.445540in}{2.355851in}%
\pgfsys@useobject{currentmarker}{}%
\end{pgfscope}%
\begin{pgfscope}%
\pgfsys@transformshift{6.144699in}{0.842884in}%
\pgfsys@useobject{currentmarker}{}%
\end{pgfscope}%
\begin{pgfscope}%
\pgfsys@transformshift{5.587596in}{4.316849in}%
\pgfsys@useobject{currentmarker}{}%
\end{pgfscope}%
\begin{pgfscope}%
\pgfsys@transformshift{5.860039in}{4.270332in}%
\pgfsys@useobject{currentmarker}{}%
\end{pgfscope}%
\begin{pgfscope}%
\pgfsys@transformshift{1.327747in}{0.342544in}%
\pgfsys@useobject{currentmarker}{}%
\end{pgfscope}%
\begin{pgfscope}%
\pgfsys@transformshift{5.284133in}{3.990551in}%
\pgfsys@useobject{currentmarker}{}%
\end{pgfscope}%
\begin{pgfscope}%
\pgfsys@transformshift{3.993120in}{0.899747in}%
\pgfsys@useobject{currentmarker}{}%
\end{pgfscope}%
\begin{pgfscope}%
\pgfsys@transformshift{1.820555in}{3.037024in}%
\pgfsys@useobject{currentmarker}{}%
\end{pgfscope}%
\begin{pgfscope}%
\pgfsys@transformshift{4.976421in}{5.467875in}%
\pgfsys@useobject{currentmarker}{}%
\end{pgfscope}%
\begin{pgfscope}%
\pgfsys@transformshift{4.518831in}{5.000925in}%
\pgfsys@useobject{currentmarker}{}%
\end{pgfscope}%
\begin{pgfscope}%
\pgfsys@transformshift{5.758990in}{0.597983in}%
\pgfsys@useobject{currentmarker}{}%
\end{pgfscope}%
\begin{pgfscope}%
\pgfsys@transformshift{5.322391in}{5.195192in}%
\pgfsys@useobject{currentmarker}{}%
\end{pgfscope}%
\begin{pgfscope}%
\pgfsys@transformshift{6.191893in}{3.808835in}%
\pgfsys@useobject{currentmarker}{}%
\end{pgfscope}%
\begin{pgfscope}%
\pgfsys@transformshift{4.112982in}{3.072140in}%
\pgfsys@useobject{currentmarker}{}%
\end{pgfscope}%
\begin{pgfscope}%
\pgfsys@transformshift{4.346249in}{2.073142in}%
\pgfsys@useobject{currentmarker}{}%
\end{pgfscope}%
\begin{pgfscope}%
\pgfsys@transformshift{5.031335in}{4.367788in}%
\pgfsys@useobject{currentmarker}{}%
\end{pgfscope}%
\begin{pgfscope}%
\pgfsys@transformshift{2.320682in}{1.898677in}%
\pgfsys@useobject{currentmarker}{}%
\end{pgfscope}%
\begin{pgfscope}%
\pgfsys@transformshift{2.639656in}{4.074777in}%
\pgfsys@useobject{currentmarker}{}%
\end{pgfscope}%
\begin{pgfscope}%
\pgfsys@transformshift{4.591237in}{3.148890in}%
\pgfsys@useobject{currentmarker}{}%
\end{pgfscope}%
\begin{pgfscope}%
\pgfsys@transformshift{5.707718in}{0.457513in}%
\pgfsys@useobject{currentmarker}{}%
\end{pgfscope}%
\begin{pgfscope}%
\pgfsys@transformshift{2.585349in}{5.586581in}%
\pgfsys@useobject{currentmarker}{}%
\end{pgfscope}%
\begin{pgfscope}%
\pgfsys@transformshift{1.537944in}{5.101799in}%
\pgfsys@useobject{currentmarker}{}%
\end{pgfscope}%
\begin{pgfscope}%
\pgfsys@transformshift{5.052617in}{2.372156in}%
\pgfsys@useobject{currentmarker}{}%
\end{pgfscope}%
\begin{pgfscope}%
\pgfsys@transformshift{3.531247in}{3.288660in}%
\pgfsys@useobject{currentmarker}{}%
\end{pgfscope}%
\begin{pgfscope}%
\pgfsys@transformshift{5.727482in}{2.427963in}%
\pgfsys@useobject{currentmarker}{}%
\end{pgfscope}%
\begin{pgfscope}%
\pgfsys@transformshift{1.090843in}{5.060215in}%
\pgfsys@useobject{currentmarker}{}%
\end{pgfscope}%
\begin{pgfscope}%
\pgfsys@transformshift{0.917944in}{4.411994in}%
\pgfsys@useobject{currentmarker}{}%
\end{pgfscope}%
\begin{pgfscope}%
\pgfsys@transformshift{5.944407in}{4.296100in}%
\pgfsys@useobject{currentmarker}{}%
\end{pgfscope}%
\begin{pgfscope}%
\pgfsys@transformshift{2.400886in}{0.241725in}%
\pgfsys@useobject{currentmarker}{}%
\end{pgfscope}%
\begin{pgfscope}%
\pgfsys@transformshift{3.933951in}{2.423825in}%
\pgfsys@useobject{currentmarker}{}%
\end{pgfscope}%
\begin{pgfscope}%
\pgfsys@transformshift{2.953373in}{0.403868in}%
\pgfsys@useobject{currentmarker}{}%
\end{pgfscope}%
\begin{pgfscope}%
\pgfsys@transformshift{3.684811in}{5.370307in}%
\pgfsys@useobject{currentmarker}{}%
\end{pgfscope}%
\begin{pgfscope}%
\pgfsys@transformshift{4.527017in}{2.440840in}%
\pgfsys@useobject{currentmarker}{}%
\end{pgfscope}%
\begin{pgfscope}%
\pgfsys@transformshift{1.597930in}{3.973651in}%
\pgfsys@useobject{currentmarker}{}%
\end{pgfscope}%
\begin{pgfscope}%
\pgfsys@transformshift{4.244140in}{5.201478in}%
\pgfsys@useobject{currentmarker}{}%
\end{pgfscope}%
\begin{pgfscope}%
\pgfsys@transformshift{2.731012in}{2.446474in}%
\pgfsys@useobject{currentmarker}{}%
\end{pgfscope}%
\begin{pgfscope}%
\pgfsys@transformshift{1.721705in}{3.275660in}%
\pgfsys@useobject{currentmarker}{}%
\end{pgfscope}%
\begin{pgfscope}%
\pgfsys@transformshift{6.883942in}{1.299839in}%
\pgfsys@useobject{currentmarker}{}%
\end{pgfscope}%
\begin{pgfscope}%
\pgfsys@transformshift{1.465171in}{0.970937in}%
\pgfsys@useobject{currentmarker}{}%
\end{pgfscope}%
\begin{pgfscope}%
\pgfsys@transformshift{2.667479in}{5.139195in}%
\pgfsys@useobject{currentmarker}{}%
\end{pgfscope}%
\begin{pgfscope}%
\pgfsys@transformshift{2.701238in}{2.707072in}%
\pgfsys@useobject{currentmarker}{}%
\end{pgfscope}%
\begin{pgfscope}%
\pgfsys@transformshift{6.547016in}{2.283006in}%
\pgfsys@useobject{currentmarker}{}%
\end{pgfscope}%
\begin{pgfscope}%
\pgfsys@transformshift{3.266648in}{5.347777in}%
\pgfsys@useobject{currentmarker}{}%
\end{pgfscope}%
\begin{pgfscope}%
\pgfsys@transformshift{2.378545in}{1.229894in}%
\pgfsys@useobject{currentmarker}{}%
\end{pgfscope}%
\begin{pgfscope}%
\pgfsys@transformshift{5.194616in}{2.453481in}%
\pgfsys@useobject{currentmarker}{}%
\end{pgfscope}%
\begin{pgfscope}%
\pgfsys@transformshift{4.251196in}{5.418875in}%
\pgfsys@useobject{currentmarker}{}%
\end{pgfscope}%
\begin{pgfscope}%
\pgfsys@transformshift{5.395600in}{2.679137in}%
\pgfsys@useobject{currentmarker}{}%
\end{pgfscope}%
\begin{pgfscope}%
\pgfsys@transformshift{1.518682in}{4.800976in}%
\pgfsys@useobject{currentmarker}{}%
\end{pgfscope}%
\begin{pgfscope}%
\pgfsys@transformshift{4.342434in}{0.678690in}%
\pgfsys@useobject{currentmarker}{}%
\end{pgfscope}%
\begin{pgfscope}%
\pgfsys@transformshift{6.703472in}{2.313088in}%
\pgfsys@useobject{currentmarker}{}%
\end{pgfscope}%
\begin{pgfscope}%
\pgfsys@transformshift{0.850509in}{5.345405in}%
\pgfsys@useobject{currentmarker}{}%
\end{pgfscope}%
\begin{pgfscope}%
\pgfsys@transformshift{4.473622in}{2.677575in}%
\pgfsys@useobject{currentmarker}{}%
\end{pgfscope}%
\begin{pgfscope}%
\pgfsys@transformshift{3.497820in}{5.397189in}%
\pgfsys@useobject{currentmarker}{}%
\end{pgfscope}%
\begin{pgfscope}%
\pgfsys@transformshift{2.507566in}{2.142928in}%
\pgfsys@useobject{currentmarker}{}%
\end{pgfscope}%
\begin{pgfscope}%
\pgfsys@transformshift{1.572349in}{1.477577in}%
\pgfsys@useobject{currentmarker}{}%
\end{pgfscope}%
\begin{pgfscope}%
\pgfsys@transformshift{0.918109in}{2.102665in}%
\pgfsys@useobject{currentmarker}{}%
\end{pgfscope}%
\begin{pgfscope}%
\pgfsys@transformshift{5.551334in}{2.724328in}%
\pgfsys@useobject{currentmarker}{}%
\end{pgfscope}%
\begin{pgfscope}%
\pgfsys@transformshift{3.804229in}{3.856476in}%
\pgfsys@useobject{currentmarker}{}%
\end{pgfscope}%
\begin{pgfscope}%
\pgfsys@transformshift{3.405157in}{1.312488in}%
\pgfsys@useobject{currentmarker}{}%
\end{pgfscope}%
\begin{pgfscope}%
\pgfsys@transformshift{6.210793in}{3.665697in}%
\pgfsys@useobject{currentmarker}{}%
\end{pgfscope}%
\begin{pgfscope}%
\pgfsys@transformshift{5.999147in}{3.448027in}%
\pgfsys@useobject{currentmarker}{}%
\end{pgfscope}%
\begin{pgfscope}%
\pgfsys@transformshift{3.648746in}{5.011550in}%
\pgfsys@useobject{currentmarker}{}%
\end{pgfscope}%
\begin{pgfscope}%
\pgfsys@transformshift{0.730912in}{5.317873in}%
\pgfsys@useobject{currentmarker}{}%
\end{pgfscope}%
\begin{pgfscope}%
\pgfsys@transformshift{1.721779in}{0.648734in}%
\pgfsys@useobject{currentmarker}{}%
\end{pgfscope}%
\begin{pgfscope}%
\pgfsys@transformshift{2.297439in}{3.943681in}%
\pgfsys@useobject{currentmarker}{}%
\end{pgfscope}%
\begin{pgfscope}%
\pgfsys@transformshift{4.635825in}{4.998243in}%
\pgfsys@useobject{currentmarker}{}%
\end{pgfscope}%
\begin{pgfscope}%
\pgfsys@transformshift{1.290406in}{1.104941in}%
\pgfsys@useobject{currentmarker}{}%
\end{pgfscope}%
\begin{pgfscope}%
\pgfsys@transformshift{0.954170in}{5.152561in}%
\pgfsys@useobject{currentmarker}{}%
\end{pgfscope}%
\begin{pgfscope}%
\pgfsys@transformshift{3.740532in}{1.895624in}%
\pgfsys@useobject{currentmarker}{}%
\end{pgfscope}%
\begin{pgfscope}%
\pgfsys@transformshift{1.671660in}{5.497159in}%
\pgfsys@useobject{currentmarker}{}%
\end{pgfscope}%
\begin{pgfscope}%
\pgfsys@transformshift{4.359756in}{3.536926in}%
\pgfsys@useobject{currentmarker}{}%
\end{pgfscope}%
\begin{pgfscope}%
\pgfsys@transformshift{4.913462in}{0.591345in}%
\pgfsys@useobject{currentmarker}{}%
\end{pgfscope}%
\begin{pgfscope}%
\pgfsys@transformshift{5.375068in}{1.908884in}%
\pgfsys@useobject{currentmarker}{}%
\end{pgfscope}%
\begin{pgfscope}%
\pgfsys@transformshift{4.581745in}{0.846596in}%
\pgfsys@useobject{currentmarker}{}%
\end{pgfscope}%
\begin{pgfscope}%
\pgfsys@transformshift{5.444748in}{5.453287in}%
\pgfsys@useobject{currentmarker}{}%
\end{pgfscope}%
\begin{pgfscope}%
\pgfsys@transformshift{3.442235in}{3.630797in}%
\pgfsys@useobject{currentmarker}{}%
\end{pgfscope}%
\begin{pgfscope}%
\pgfsys@transformshift{1.615806in}{4.731700in}%
\pgfsys@useobject{currentmarker}{}%
\end{pgfscope}%
\begin{pgfscope}%
\pgfsys@transformshift{3.854399in}{3.314784in}%
\pgfsys@useobject{currentmarker}{}%
\end{pgfscope}%
\begin{pgfscope}%
\pgfsys@transformshift{2.055857in}{1.570616in}%
\pgfsys@useobject{currentmarker}{}%
\end{pgfscope}%
\begin{pgfscope}%
\pgfsys@transformshift{1.436347in}{3.105986in}%
\pgfsys@useobject{currentmarker}{}%
\end{pgfscope}%
\begin{pgfscope}%
\pgfsys@transformshift{4.726454in}{3.149538in}%
\pgfsys@useobject{currentmarker}{}%
\end{pgfscope}%
\begin{pgfscope}%
\pgfsys@transformshift{4.935666in}{4.874458in}%
\pgfsys@useobject{currentmarker}{}%
\end{pgfscope}%
\begin{pgfscope}%
\pgfsys@transformshift{4.924599in}{2.981236in}%
\pgfsys@useobject{currentmarker}{}%
\end{pgfscope}%
\begin{pgfscope}%
\pgfsys@transformshift{5.646671in}{0.236834in}%
\pgfsys@useobject{currentmarker}{}%
\end{pgfscope}%
\begin{pgfscope}%
\pgfsys@transformshift{2.508063in}{5.385651in}%
\pgfsys@useobject{currentmarker}{}%
\end{pgfscope}%
\begin{pgfscope}%
\pgfsys@transformshift{4.318884in}{4.682130in}%
\pgfsys@useobject{currentmarker}{}%
\end{pgfscope}%
\begin{pgfscope}%
\pgfsys@transformshift{3.453664in}{5.476330in}%
\pgfsys@useobject{currentmarker}{}%
\end{pgfscope}%
\begin{pgfscope}%
\pgfsys@transformshift{6.008581in}{1.218234in}%
\pgfsys@useobject{currentmarker}{}%
\end{pgfscope}%
\begin{pgfscope}%
\pgfsys@transformshift{2.172534in}{1.988399in}%
\pgfsys@useobject{currentmarker}{}%
\end{pgfscope}%
\begin{pgfscope}%
\pgfsys@transformshift{4.574985in}{2.093693in}%
\pgfsys@useobject{currentmarker}{}%
\end{pgfscope}%
\begin{pgfscope}%
\pgfsys@transformshift{2.821949in}{0.915201in}%
\pgfsys@useobject{currentmarker}{}%
\end{pgfscope}%
\begin{pgfscope}%
\pgfsys@transformshift{2.291114in}{1.022960in}%
\pgfsys@useobject{currentmarker}{}%
\end{pgfscope}%
\begin{pgfscope}%
\pgfsys@transformshift{3.249111in}{3.451440in}%
\pgfsys@useobject{currentmarker}{}%
\end{pgfscope}%
\begin{pgfscope}%
\pgfsys@transformshift{3.418379in}{4.944464in}%
\pgfsys@useobject{currentmarker}{}%
\end{pgfscope}%
\begin{pgfscope}%
\pgfsys@transformshift{6.721855in}{4.742230in}%
\pgfsys@useobject{currentmarker}{}%
\end{pgfscope}%
\begin{pgfscope}%
\pgfsys@transformshift{3.195830in}{2.139203in}%
\pgfsys@useobject{currentmarker}{}%
\end{pgfscope}%
\begin{pgfscope}%
\pgfsys@transformshift{4.613722in}{3.598510in}%
\pgfsys@useobject{currentmarker}{}%
\end{pgfscope}%
\begin{pgfscope}%
\pgfsys@transformshift{4.571450in}{1.755305in}%
\pgfsys@useobject{currentmarker}{}%
\end{pgfscope}%
\begin{pgfscope}%
\pgfsys@transformshift{2.687764in}{4.679773in}%
\pgfsys@useobject{currentmarker}{}%
\end{pgfscope}%
\begin{pgfscope}%
\pgfsys@transformshift{4.919162in}{1.245477in}%
\pgfsys@useobject{currentmarker}{}%
\end{pgfscope}%
\begin{pgfscope}%
\pgfsys@transformshift{5.478505in}{3.023083in}%
\pgfsys@useobject{currentmarker}{}%
\end{pgfscope}%
\begin{pgfscope}%
\pgfsys@transformshift{5.286372in}{2.644489in}%
\pgfsys@useobject{currentmarker}{}%
\end{pgfscope}%
\begin{pgfscope}%
\pgfsys@transformshift{1.932215in}{3.910854in}%
\pgfsys@useobject{currentmarker}{}%
\end{pgfscope}%
\begin{pgfscope}%
\pgfsys@transformshift{5.826780in}{3.067734in}%
\pgfsys@useobject{currentmarker}{}%
\end{pgfscope}%
\begin{pgfscope}%
\pgfsys@transformshift{6.173009in}{5.541503in}%
\pgfsys@useobject{currentmarker}{}%
\end{pgfscope}%
\begin{pgfscope}%
\pgfsys@transformshift{4.563118in}{4.624819in}%
\pgfsys@useobject{currentmarker}{}%
\end{pgfscope}%
\begin{pgfscope}%
\pgfsys@transformshift{6.081851in}{5.283892in}%
\pgfsys@useobject{currentmarker}{}%
\end{pgfscope}%
\begin{pgfscope}%
\pgfsys@transformshift{3.816489in}{4.911076in}%
\pgfsys@useobject{currentmarker}{}%
\end{pgfscope}%
\begin{pgfscope}%
\pgfsys@transformshift{2.570919in}{0.951181in}%
\pgfsys@useobject{currentmarker}{}%
\end{pgfscope}%
\begin{pgfscope}%
\pgfsys@transformshift{6.077710in}{2.312357in}%
\pgfsys@useobject{currentmarker}{}%
\end{pgfscope}%
\begin{pgfscope}%
\pgfsys@transformshift{3.339419in}{1.969108in}%
\pgfsys@useobject{currentmarker}{}%
\end{pgfscope}%
\begin{pgfscope}%
\pgfsys@transformshift{6.293994in}{0.283070in}%
\pgfsys@useobject{currentmarker}{}%
\end{pgfscope}%
\begin{pgfscope}%
\pgfsys@transformshift{3.755276in}{2.885846in}%
\pgfsys@useobject{currentmarker}{}%
\end{pgfscope}%
\begin{pgfscope}%
\pgfsys@transformshift{6.775032in}{5.131919in}%
\pgfsys@useobject{currentmarker}{}%
\end{pgfscope}%
\begin{pgfscope}%
\pgfsys@transformshift{5.354899in}{1.732918in}%
\pgfsys@useobject{currentmarker}{}%
\end{pgfscope}%
\begin{pgfscope}%
\pgfsys@transformshift{5.345581in}{2.511129in}%
\pgfsys@useobject{currentmarker}{}%
\end{pgfscope}%
\begin{pgfscope}%
\pgfsys@transformshift{2.387974in}{3.463126in}%
\pgfsys@useobject{currentmarker}{}%
\end{pgfscope}%
\begin{pgfscope}%
\pgfsys@transformshift{6.222673in}{3.024721in}%
\pgfsys@useobject{currentmarker}{}%
\end{pgfscope}%
\begin{pgfscope}%
\pgfsys@transformshift{5.708974in}{3.028273in}%
\pgfsys@useobject{currentmarker}{}%
\end{pgfscope}%
\begin{pgfscope}%
\pgfsys@transformshift{2.121233in}{4.570489in}%
\pgfsys@useobject{currentmarker}{}%
\end{pgfscope}%
\begin{pgfscope}%
\pgfsys@transformshift{6.588359in}{2.987668in}%
\pgfsys@useobject{currentmarker}{}%
\end{pgfscope}%
\begin{pgfscope}%
\pgfsys@transformshift{0.660766in}{3.380144in}%
\pgfsys@useobject{currentmarker}{}%
\end{pgfscope}%
\begin{pgfscope}%
\pgfsys@transformshift{0.638422in}{0.377692in}%
\pgfsys@useobject{currentmarker}{}%
\end{pgfscope}%
\begin{pgfscope}%
\pgfsys@transformshift{4.883686in}{2.024349in}%
\pgfsys@useobject{currentmarker}{}%
\end{pgfscope}%
\begin{pgfscope}%
\pgfsys@transformshift{2.740257in}{3.626682in}%
\pgfsys@useobject{currentmarker}{}%
\end{pgfscope}%
\begin{pgfscope}%
\pgfsys@transformshift{0.959435in}{5.629592in}%
\pgfsys@useobject{currentmarker}{}%
\end{pgfscope}%
\begin{pgfscope}%
\pgfsys@transformshift{1.541526in}{4.402371in}%
\pgfsys@useobject{currentmarker}{}%
\end{pgfscope}%
\begin{pgfscope}%
\pgfsys@transformshift{6.276516in}{3.780366in}%
\pgfsys@useobject{currentmarker}{}%
\end{pgfscope}%
\begin{pgfscope}%
\pgfsys@transformshift{4.656231in}{5.099656in}%
\pgfsys@useobject{currentmarker}{}%
\end{pgfscope}%
\begin{pgfscope}%
\pgfsys@transformshift{1.403655in}{3.799710in}%
\pgfsys@useobject{currentmarker}{}%
\end{pgfscope}%
\begin{pgfscope}%
\pgfsys@transformshift{3.077661in}{4.281679in}%
\pgfsys@useobject{currentmarker}{}%
\end{pgfscope}%
\begin{pgfscope}%
\pgfsys@transformshift{3.978488in}{3.094930in}%
\pgfsys@useobject{currentmarker}{}%
\end{pgfscope}%
\begin{pgfscope}%
\pgfsys@transformshift{6.308239in}{1.314485in}%
\pgfsys@useobject{currentmarker}{}%
\end{pgfscope}%
\begin{pgfscope}%
\pgfsys@transformshift{5.665058in}{0.916599in}%
\pgfsys@useobject{currentmarker}{}%
\end{pgfscope}%
\begin{pgfscope}%
\pgfsys@transformshift{4.307675in}{1.537587in}%
\pgfsys@useobject{currentmarker}{}%
\end{pgfscope}%
\begin{pgfscope}%
\pgfsys@transformshift{1.783796in}{5.632632in}%
\pgfsys@useobject{currentmarker}{}%
\end{pgfscope}%
\begin{pgfscope}%
\pgfsys@transformshift{4.003026in}{4.943363in}%
\pgfsys@useobject{currentmarker}{}%
\end{pgfscope}%
\begin{pgfscope}%
\pgfsys@transformshift{5.169143in}{4.312399in}%
\pgfsys@useobject{currentmarker}{}%
\end{pgfscope}%
\begin{pgfscope}%
\pgfsys@transformshift{4.726654in}{2.312508in}%
\pgfsys@useobject{currentmarker}{}%
\end{pgfscope}%
\begin{pgfscope}%
\pgfsys@transformshift{2.266604in}{2.484720in}%
\pgfsys@useobject{currentmarker}{}%
\end{pgfscope}%
\begin{pgfscope}%
\pgfsys@transformshift{4.237641in}{4.877722in}%
\pgfsys@useobject{currentmarker}{}%
\end{pgfscope}%
\begin{pgfscope}%
\pgfsys@transformshift{6.127742in}{2.376750in}%
\pgfsys@useobject{currentmarker}{}%
\end{pgfscope}%
\begin{pgfscope}%
\pgfsys@transformshift{0.685222in}{2.435419in}%
\pgfsys@useobject{currentmarker}{}%
\end{pgfscope}%
\begin{pgfscope}%
\pgfsys@transformshift{3.995462in}{2.932399in}%
\pgfsys@useobject{currentmarker}{}%
\end{pgfscope}%
\begin{pgfscope}%
\pgfsys@transformshift{1.199929in}{4.919594in}%
\pgfsys@useobject{currentmarker}{}%
\end{pgfscope}%
\begin{pgfscope}%
\pgfsys@transformshift{4.786246in}{5.141406in}%
\pgfsys@useobject{currentmarker}{}%
\end{pgfscope}%
\begin{pgfscope}%
\pgfsys@transformshift{4.097274in}{4.381858in}%
\pgfsys@useobject{currentmarker}{}%
\end{pgfscope}%
\begin{pgfscope}%
\pgfsys@transformshift{1.456272in}{2.142124in}%
\pgfsys@useobject{currentmarker}{}%
\end{pgfscope}%
\begin{pgfscope}%
\pgfsys@transformshift{0.983666in}{3.334171in}%
\pgfsys@useobject{currentmarker}{}%
\end{pgfscope}%
\begin{pgfscope}%
\pgfsys@transformshift{4.540991in}{1.982214in}%
\pgfsys@useobject{currentmarker}{}%
\end{pgfscope}%
\begin{pgfscope}%
\pgfsys@transformshift{2.887081in}{2.480684in}%
\pgfsys@useobject{currentmarker}{}%
\end{pgfscope}%
\begin{pgfscope}%
\pgfsys@transformshift{1.974107in}{4.422461in}%
\pgfsys@useobject{currentmarker}{}%
\end{pgfscope}%
\begin{pgfscope}%
\pgfsys@transformshift{3.742329in}{1.808660in}%
\pgfsys@useobject{currentmarker}{}%
\end{pgfscope}%
\begin{pgfscope}%
\pgfsys@transformshift{5.625123in}{5.136882in}%
\pgfsys@useobject{currentmarker}{}%
\end{pgfscope}%
\begin{pgfscope}%
\pgfsys@transformshift{4.970478in}{2.101681in}%
\pgfsys@useobject{currentmarker}{}%
\end{pgfscope}%
\begin{pgfscope}%
\pgfsys@transformshift{4.449601in}{2.118021in}%
\pgfsys@useobject{currentmarker}{}%
\end{pgfscope}%
\begin{pgfscope}%
\pgfsys@transformshift{4.545370in}{1.625933in}%
\pgfsys@useobject{currentmarker}{}%
\end{pgfscope}%
\begin{pgfscope}%
\pgfsys@transformshift{1.701117in}{3.302885in}%
\pgfsys@useobject{currentmarker}{}%
\end{pgfscope}%
\begin{pgfscope}%
\pgfsys@transformshift{3.020819in}{0.698542in}%
\pgfsys@useobject{currentmarker}{}%
\end{pgfscope}%
\begin{pgfscope}%
\pgfsys@transformshift{2.882987in}{3.637747in}%
\pgfsys@useobject{currentmarker}{}%
\end{pgfscope}%
\begin{pgfscope}%
\pgfsys@transformshift{6.173915in}{5.168388in}%
\pgfsys@useobject{currentmarker}{}%
\end{pgfscope}%
\begin{pgfscope}%
\pgfsys@transformshift{6.016951in}{5.527557in}%
\pgfsys@useobject{currentmarker}{}%
\end{pgfscope}%
\begin{pgfscope}%
\pgfsys@transformshift{2.444148in}{5.354277in}%
\pgfsys@useobject{currentmarker}{}%
\end{pgfscope}%
\begin{pgfscope}%
\pgfsys@transformshift{3.541478in}{2.148269in}%
\pgfsys@useobject{currentmarker}{}%
\end{pgfscope}%
\begin{pgfscope}%
\pgfsys@transformshift{2.937770in}{5.471810in}%
\pgfsys@useobject{currentmarker}{}%
\end{pgfscope}%
\begin{pgfscope}%
\pgfsys@transformshift{6.844005in}{5.483905in}%
\pgfsys@useobject{currentmarker}{}%
\end{pgfscope}%
\begin{pgfscope}%
\pgfsys@transformshift{5.545303in}{0.229241in}%
\pgfsys@useobject{currentmarker}{}%
\end{pgfscope}%
\begin{pgfscope}%
\pgfsys@transformshift{2.322194in}{2.891862in}%
\pgfsys@useobject{currentmarker}{}%
\end{pgfscope}%
\begin{pgfscope}%
\pgfsys@transformshift{2.813031in}{1.667116in}%
\pgfsys@useobject{currentmarker}{}%
\end{pgfscope}%
\begin{pgfscope}%
\pgfsys@transformshift{6.512908in}{2.472316in}%
\pgfsys@useobject{currentmarker}{}%
\end{pgfscope}%
\begin{pgfscope}%
\pgfsys@transformshift{2.649575in}{3.713479in}%
\pgfsys@useobject{currentmarker}{}%
\end{pgfscope}%
\begin{pgfscope}%
\pgfsys@transformshift{2.685982in}{5.150939in}%
\pgfsys@useobject{currentmarker}{}%
\end{pgfscope}%
\begin{pgfscope}%
\pgfsys@transformshift{4.611924in}{2.043274in}%
\pgfsys@useobject{currentmarker}{}%
\end{pgfscope}%
\begin{pgfscope}%
\pgfsys@transformshift{1.008702in}{0.812180in}%
\pgfsys@useobject{currentmarker}{}%
\end{pgfscope}%
\begin{pgfscope}%
\pgfsys@transformshift{4.631452in}{4.610566in}%
\pgfsys@useobject{currentmarker}{}%
\end{pgfscope}%
\begin{pgfscope}%
\pgfsys@transformshift{3.566628in}{0.476345in}%
\pgfsys@useobject{currentmarker}{}%
\end{pgfscope}%
\begin{pgfscope}%
\pgfsys@transformshift{3.859542in}{1.606297in}%
\pgfsys@useobject{currentmarker}{}%
\end{pgfscope}%
\begin{pgfscope}%
\pgfsys@transformshift{3.126543in}{3.326074in}%
\pgfsys@useobject{currentmarker}{}%
\end{pgfscope}%
\begin{pgfscope}%
\pgfsys@transformshift{2.071983in}{0.819441in}%
\pgfsys@useobject{currentmarker}{}%
\end{pgfscope}%
\begin{pgfscope}%
\pgfsys@transformshift{2.557020in}{3.165861in}%
\pgfsys@useobject{currentmarker}{}%
\end{pgfscope}%
\begin{pgfscope}%
\pgfsys@transformshift{4.660763in}{3.961684in}%
\pgfsys@useobject{currentmarker}{}%
\end{pgfscope}%
\begin{pgfscope}%
\pgfsys@transformshift{6.095763in}{0.327096in}%
\pgfsys@useobject{currentmarker}{}%
\end{pgfscope}%
\begin{pgfscope}%
\pgfsys@transformshift{4.777143in}{5.321530in}%
\pgfsys@useobject{currentmarker}{}%
\end{pgfscope}%
\begin{pgfscope}%
\pgfsys@transformshift{5.019800in}{2.678550in}%
\pgfsys@useobject{currentmarker}{}%
\end{pgfscope}%
\begin{pgfscope}%
\pgfsys@transformshift{1.096184in}{4.373993in}%
\pgfsys@useobject{currentmarker}{}%
\end{pgfscope}%
\begin{pgfscope}%
\pgfsys@transformshift{4.817608in}{1.764063in}%
\pgfsys@useobject{currentmarker}{}%
\end{pgfscope}%
\begin{pgfscope}%
\pgfsys@transformshift{2.388446in}{2.056302in}%
\pgfsys@useobject{currentmarker}{}%
\end{pgfscope}%
\begin{pgfscope}%
\pgfsys@transformshift{6.638673in}{1.337319in}%
\pgfsys@useobject{currentmarker}{}%
\end{pgfscope}%
\begin{pgfscope}%
\pgfsys@transformshift{3.917515in}{4.306589in}%
\pgfsys@useobject{currentmarker}{}%
\end{pgfscope}%
\begin{pgfscope}%
\pgfsys@transformshift{5.262884in}{2.769688in}%
\pgfsys@useobject{currentmarker}{}%
\end{pgfscope}%
\begin{pgfscope}%
\pgfsys@transformshift{2.540800in}{2.195987in}%
\pgfsys@useobject{currentmarker}{}%
\end{pgfscope}%
\begin{pgfscope}%
\pgfsys@transformshift{2.278667in}{4.273157in}%
\pgfsys@useobject{currentmarker}{}%
\end{pgfscope}%
\begin{pgfscope}%
\pgfsys@transformshift{1.600388in}{5.157514in}%
\pgfsys@useobject{currentmarker}{}%
\end{pgfscope}%
\begin{pgfscope}%
\pgfsys@transformshift{1.256587in}{4.185291in}%
\pgfsys@useobject{currentmarker}{}%
\end{pgfscope}%
\begin{pgfscope}%
\pgfsys@transformshift{1.675103in}{2.367521in}%
\pgfsys@useobject{currentmarker}{}%
\end{pgfscope}%
\begin{pgfscope}%
\pgfsys@transformshift{1.140633in}{1.975087in}%
\pgfsys@useobject{currentmarker}{}%
\end{pgfscope}%
\begin{pgfscope}%
\pgfsys@transformshift{6.150431in}{1.526901in}%
\pgfsys@useobject{currentmarker}{}%
\end{pgfscope}%
\begin{pgfscope}%
\pgfsys@transformshift{2.754860in}{3.534710in}%
\pgfsys@useobject{currentmarker}{}%
\end{pgfscope}%
\begin{pgfscope}%
\pgfsys@transformshift{5.688736in}{3.249390in}%
\pgfsys@useobject{currentmarker}{}%
\end{pgfscope}%
\begin{pgfscope}%
\pgfsys@transformshift{3.462161in}{1.070390in}%
\pgfsys@useobject{currentmarker}{}%
\end{pgfscope}%
\begin{pgfscope}%
\pgfsys@transformshift{4.573148in}{2.042096in}%
\pgfsys@useobject{currentmarker}{}%
\end{pgfscope}%
\begin{pgfscope}%
\pgfsys@transformshift{1.477976in}{1.126700in}%
\pgfsys@useobject{currentmarker}{}%
\end{pgfscope}%
\begin{pgfscope}%
\pgfsys@transformshift{2.059752in}{5.603854in}%
\pgfsys@useobject{currentmarker}{}%
\end{pgfscope}%
\begin{pgfscope}%
\pgfsys@transformshift{3.761870in}{2.737511in}%
\pgfsys@useobject{currentmarker}{}%
\end{pgfscope}%
\begin{pgfscope}%
\pgfsys@transformshift{2.387502in}{2.618844in}%
\pgfsys@useobject{currentmarker}{}%
\end{pgfscope}%
\begin{pgfscope}%
\pgfsys@transformshift{1.177883in}{1.628458in}%
\pgfsys@useobject{currentmarker}{}%
\end{pgfscope}%
\begin{pgfscope}%
\pgfsys@transformshift{1.514227in}{1.497437in}%
\pgfsys@useobject{currentmarker}{}%
\end{pgfscope}%
\begin{pgfscope}%
\pgfsys@transformshift{1.995204in}{3.074388in}%
\pgfsys@useobject{currentmarker}{}%
\end{pgfscope}%
\begin{pgfscope}%
\pgfsys@transformshift{2.698425in}{5.229061in}%
\pgfsys@useobject{currentmarker}{}%
\end{pgfscope}%
\begin{pgfscope}%
\pgfsys@transformshift{2.848675in}{1.640935in}%
\pgfsys@useobject{currentmarker}{}%
\end{pgfscope}%
\begin{pgfscope}%
\pgfsys@transformshift{4.312361in}{4.661572in}%
\pgfsys@useobject{currentmarker}{}%
\end{pgfscope}%
\begin{pgfscope}%
\pgfsys@transformshift{6.711979in}{4.754123in}%
\pgfsys@useobject{currentmarker}{}%
\end{pgfscope}%
\begin{pgfscope}%
\pgfsys@transformshift{4.555861in}{1.916473in}%
\pgfsys@useobject{currentmarker}{}%
\end{pgfscope}%
\begin{pgfscope}%
\pgfsys@transformshift{4.456666in}{4.816556in}%
\pgfsys@useobject{currentmarker}{}%
\end{pgfscope}%
\begin{pgfscope}%
\pgfsys@transformshift{1.955766in}{4.913128in}%
\pgfsys@useobject{currentmarker}{}%
\end{pgfscope}%
\begin{pgfscope}%
\pgfsys@transformshift{4.702284in}{0.728509in}%
\pgfsys@useobject{currentmarker}{}%
\end{pgfscope}%
\begin{pgfscope}%
\pgfsys@transformshift{5.486862in}{1.687828in}%
\pgfsys@useobject{currentmarker}{}%
\end{pgfscope}%
\begin{pgfscope}%
\pgfsys@transformshift{6.662081in}{3.638517in}%
\pgfsys@useobject{currentmarker}{}%
\end{pgfscope}%
\begin{pgfscope}%
\pgfsys@transformshift{2.027746in}{5.261369in}%
\pgfsys@useobject{currentmarker}{}%
\end{pgfscope}%
\begin{pgfscope}%
\pgfsys@transformshift{3.231916in}{0.236334in}%
\pgfsys@useobject{currentmarker}{}%
\end{pgfscope}%
\begin{pgfscope}%
\pgfsys@transformshift{6.714576in}{3.312566in}%
\pgfsys@useobject{currentmarker}{}%
\end{pgfscope}%
\begin{pgfscope}%
\pgfsys@transformshift{2.461026in}{2.860383in}%
\pgfsys@useobject{currentmarker}{}%
\end{pgfscope}%
\begin{pgfscope}%
\pgfsys@transformshift{4.920990in}{3.284783in}%
\pgfsys@useobject{currentmarker}{}%
\end{pgfscope}%
\begin{pgfscope}%
\pgfsys@transformshift{0.633466in}{3.601195in}%
\pgfsys@useobject{currentmarker}{}%
\end{pgfscope}%
\begin{pgfscope}%
\pgfsys@transformshift{6.410720in}{4.771303in}%
\pgfsys@useobject{currentmarker}{}%
\end{pgfscope}%
\begin{pgfscope}%
\pgfsys@transformshift{0.737807in}{5.192324in}%
\pgfsys@useobject{currentmarker}{}%
\end{pgfscope}%
\begin{pgfscope}%
\pgfsys@transformshift{3.030773in}{1.004290in}%
\pgfsys@useobject{currentmarker}{}%
\end{pgfscope}%
\begin{pgfscope}%
\pgfsys@transformshift{5.721762in}{3.184300in}%
\pgfsys@useobject{currentmarker}{}%
\end{pgfscope}%
\begin{pgfscope}%
\pgfsys@transformshift{6.672696in}{1.345201in}%
\pgfsys@useobject{currentmarker}{}%
\end{pgfscope}%
\begin{pgfscope}%
\pgfsys@transformshift{4.262420in}{2.709797in}%
\pgfsys@useobject{currentmarker}{}%
\end{pgfscope}%
\begin{pgfscope}%
\pgfsys@transformshift{0.910133in}{1.575668in}%
\pgfsys@useobject{currentmarker}{}%
\end{pgfscope}%
\begin{pgfscope}%
\pgfsys@transformshift{3.702260in}{3.494981in}%
\pgfsys@useobject{currentmarker}{}%
\end{pgfscope}%
\begin{pgfscope}%
\pgfsys@transformshift{3.848048in}{4.539957in}%
\pgfsys@useobject{currentmarker}{}%
\end{pgfscope}%
\begin{pgfscope}%
\pgfsys@transformshift{0.823245in}{1.736091in}%
\pgfsys@useobject{currentmarker}{}%
\end{pgfscope}%
\begin{pgfscope}%
\pgfsys@transformshift{1.944397in}{5.536021in}%
\pgfsys@useobject{currentmarker}{}%
\end{pgfscope}%
\begin{pgfscope}%
\pgfsys@transformshift{5.748313in}{2.938554in}%
\pgfsys@useobject{currentmarker}{}%
\end{pgfscope}%
\begin{pgfscope}%
\pgfsys@transformshift{0.850308in}{2.263331in}%
\pgfsys@useobject{currentmarker}{}%
\end{pgfscope}%
\begin{pgfscope}%
\pgfsys@transformshift{5.533335in}{2.051328in}%
\pgfsys@useobject{currentmarker}{}%
\end{pgfscope}%
\begin{pgfscope}%
\pgfsys@transformshift{4.920908in}{1.178994in}%
\pgfsys@useobject{currentmarker}{}%
\end{pgfscope}%
\begin{pgfscope}%
\pgfsys@transformshift{3.078704in}{4.308053in}%
\pgfsys@useobject{currentmarker}{}%
\end{pgfscope}%
\begin{pgfscope}%
\pgfsys@transformshift{2.622395in}{4.775882in}%
\pgfsys@useobject{currentmarker}{}%
\end{pgfscope}%
\begin{pgfscope}%
\pgfsys@transformshift{3.070403in}{0.511672in}%
\pgfsys@useobject{currentmarker}{}%
\end{pgfscope}%
\begin{pgfscope}%
\pgfsys@transformshift{2.010159in}{4.587327in}%
\pgfsys@useobject{currentmarker}{}%
\end{pgfscope}%
\begin{pgfscope}%
\pgfsys@transformshift{4.615435in}{1.512341in}%
\pgfsys@useobject{currentmarker}{}%
\end{pgfscope}%
\begin{pgfscope}%
\pgfsys@transformshift{3.283848in}{1.923445in}%
\pgfsys@useobject{currentmarker}{}%
\end{pgfscope}%
\begin{pgfscope}%
\pgfsys@transformshift{3.516931in}{2.372846in}%
\pgfsys@useobject{currentmarker}{}%
\end{pgfscope}%
\begin{pgfscope}%
\pgfsys@transformshift{2.730136in}{4.537345in}%
\pgfsys@useobject{currentmarker}{}%
\end{pgfscope}%
\begin{pgfscope}%
\pgfsys@transformshift{4.197092in}{1.232488in}%
\pgfsys@useobject{currentmarker}{}%
\end{pgfscope}%
\begin{pgfscope}%
\pgfsys@transformshift{1.084735in}{5.441939in}%
\pgfsys@useobject{currentmarker}{}%
\end{pgfscope}%
\begin{pgfscope}%
\pgfsys@transformshift{5.055315in}{4.079259in}%
\pgfsys@useobject{currentmarker}{}%
\end{pgfscope}%
\begin{pgfscope}%
\pgfsys@transformshift{4.998613in}{2.542947in}%
\pgfsys@useobject{currentmarker}{}%
\end{pgfscope}%
\begin{pgfscope}%
\pgfsys@transformshift{3.753514in}{3.592069in}%
\pgfsys@useobject{currentmarker}{}%
\end{pgfscope}%
\begin{pgfscope}%
\pgfsys@transformshift{5.009085in}{1.180972in}%
\pgfsys@useobject{currentmarker}{}%
\end{pgfscope}%
\begin{pgfscope}%
\pgfsys@transformshift{3.916301in}{0.346932in}%
\pgfsys@useobject{currentmarker}{}%
\end{pgfscope}%
\begin{pgfscope}%
\pgfsys@transformshift{3.295930in}{0.419960in}%
\pgfsys@useobject{currentmarker}{}%
\end{pgfscope}%
\begin{pgfscope}%
\pgfsys@transformshift{2.941804in}{4.369959in}%
\pgfsys@useobject{currentmarker}{}%
\end{pgfscope}%
\begin{pgfscope}%
\pgfsys@transformshift{4.725180in}{3.528098in}%
\pgfsys@useobject{currentmarker}{}%
\end{pgfscope}%
\begin{pgfscope}%
\pgfsys@transformshift{2.788343in}{2.740159in}%
\pgfsys@useobject{currentmarker}{}%
\end{pgfscope}%
\begin{pgfscope}%
\pgfsys@transformshift{1.878187in}{1.349408in}%
\pgfsys@useobject{currentmarker}{}%
\end{pgfscope}%
\begin{pgfscope}%
\pgfsys@transformshift{1.664014in}{1.692202in}%
\pgfsys@useobject{currentmarker}{}%
\end{pgfscope}%
\begin{pgfscope}%
\pgfsys@transformshift{6.188220in}{4.797263in}%
\pgfsys@useobject{currentmarker}{}%
\end{pgfscope}%
\begin{pgfscope}%
\pgfsys@transformshift{6.450629in}{4.934804in}%
\pgfsys@useobject{currentmarker}{}%
\end{pgfscope}%
\begin{pgfscope}%
\pgfsys@transformshift{1.258554in}{4.447746in}%
\pgfsys@useobject{currentmarker}{}%
\end{pgfscope}%
\begin{pgfscope}%
\pgfsys@transformshift{3.528620in}{4.236590in}%
\pgfsys@useobject{currentmarker}{}%
\end{pgfscope}%
\begin{pgfscope}%
\pgfsys@transformshift{5.959390in}{4.101922in}%
\pgfsys@useobject{currentmarker}{}%
\end{pgfscope}%
\begin{pgfscope}%
\pgfsys@transformshift{5.097255in}{5.072753in}%
\pgfsys@useobject{currentmarker}{}%
\end{pgfscope}%
\begin{pgfscope}%
\pgfsys@transformshift{2.059371in}{1.976888in}%
\pgfsys@useobject{currentmarker}{}%
\end{pgfscope}%
\begin{pgfscope}%
\pgfsys@transformshift{6.460961in}{0.415587in}%
\pgfsys@useobject{currentmarker}{}%
\end{pgfscope}%
\begin{pgfscope}%
\pgfsys@transformshift{3.758362in}{5.370915in}%
\pgfsys@useobject{currentmarker}{}%
\end{pgfscope}%
\begin{pgfscope}%
\pgfsys@transformshift{3.653366in}{1.443312in}%
\pgfsys@useobject{currentmarker}{}%
\end{pgfscope}%
\begin{pgfscope}%
\pgfsys@transformshift{6.462242in}{4.811799in}%
\pgfsys@useobject{currentmarker}{}%
\end{pgfscope}%
\begin{pgfscope}%
\pgfsys@transformshift{1.189272in}{5.328609in}%
\pgfsys@useobject{currentmarker}{}%
\end{pgfscope}%
\begin{pgfscope}%
\pgfsys@transformshift{0.783360in}{3.137870in}%
\pgfsys@useobject{currentmarker}{}%
\end{pgfscope}%
\begin{pgfscope}%
\pgfsys@transformshift{3.754773in}{4.837084in}%
\pgfsys@useobject{currentmarker}{}%
\end{pgfscope}%
\begin{pgfscope}%
\pgfsys@transformshift{5.897790in}{3.739800in}%
\pgfsys@useobject{currentmarker}{}%
\end{pgfscope}%
\begin{pgfscope}%
\pgfsys@transformshift{3.670624in}{1.499253in}%
\pgfsys@useobject{currentmarker}{}%
\end{pgfscope}%
\begin{pgfscope}%
\pgfsys@transformshift{2.928574in}{1.683345in}%
\pgfsys@useobject{currentmarker}{}%
\end{pgfscope}%
\begin{pgfscope}%
\pgfsys@transformshift{4.163611in}{2.548539in}%
\pgfsys@useobject{currentmarker}{}%
\end{pgfscope}%
\begin{pgfscope}%
\pgfsys@transformshift{2.994857in}{2.965365in}%
\pgfsys@useobject{currentmarker}{}%
\end{pgfscope}%
\begin{pgfscope}%
\pgfsys@transformshift{1.687544in}{3.910411in}%
\pgfsys@useobject{currentmarker}{}%
\end{pgfscope}%
\begin{pgfscope}%
\pgfsys@transformshift{4.119993in}{3.466925in}%
\pgfsys@useobject{currentmarker}{}%
\end{pgfscope}%
\begin{pgfscope}%
\pgfsys@transformshift{2.882721in}{1.955034in}%
\pgfsys@useobject{currentmarker}{}%
\end{pgfscope}%
\begin{pgfscope}%
\pgfsys@transformshift{3.997017in}{3.791505in}%
\pgfsys@useobject{currentmarker}{}%
\end{pgfscope}%
\begin{pgfscope}%
\pgfsys@transformshift{6.666083in}{4.144635in}%
\pgfsys@useobject{currentmarker}{}%
\end{pgfscope}%
\begin{pgfscope}%
\pgfsys@transformshift{1.630066in}{2.230214in}%
\pgfsys@useobject{currentmarker}{}%
\end{pgfscope}%
\begin{pgfscope}%
\pgfsys@transformshift{3.940720in}{0.753932in}%
\pgfsys@useobject{currentmarker}{}%
\end{pgfscope}%
\begin{pgfscope}%
\pgfsys@transformshift{0.671154in}{1.163744in}%
\pgfsys@useobject{currentmarker}{}%
\end{pgfscope}%
\begin{pgfscope}%
\pgfsys@transformshift{3.659008in}{2.878941in}%
\pgfsys@useobject{currentmarker}{}%
\end{pgfscope}%
\begin{pgfscope}%
\pgfsys@transformshift{3.796370in}{0.920318in}%
\pgfsys@useobject{currentmarker}{}%
\end{pgfscope}%
\begin{pgfscope}%
\pgfsys@transformshift{5.193854in}{3.388259in}%
\pgfsys@useobject{currentmarker}{}%
\end{pgfscope}%
\begin{pgfscope}%
\pgfsys@transformshift{1.879139in}{4.064168in}%
\pgfsys@useobject{currentmarker}{}%
\end{pgfscope}%
\begin{pgfscope}%
\pgfsys@transformshift{5.683591in}{4.909548in}%
\pgfsys@useobject{currentmarker}{}%
\end{pgfscope}%
\begin{pgfscope}%
\pgfsys@transformshift{6.267351in}{4.062387in}%
\pgfsys@useobject{currentmarker}{}%
\end{pgfscope}%
\begin{pgfscope}%
\pgfsys@transformshift{5.605500in}{0.456673in}%
\pgfsys@useobject{currentmarker}{}%
\end{pgfscope}%
\begin{pgfscope}%
\pgfsys@transformshift{5.238620in}{2.651835in}%
\pgfsys@useobject{currentmarker}{}%
\end{pgfscope}%
\begin{pgfscope}%
\pgfsys@transformshift{1.665680in}{2.554861in}%
\pgfsys@useobject{currentmarker}{}%
\end{pgfscope}%
\begin{pgfscope}%
\pgfsys@transformshift{3.973932in}{3.794709in}%
\pgfsys@useobject{currentmarker}{}%
\end{pgfscope}%
\begin{pgfscope}%
\pgfsys@transformshift{1.749542in}{0.304289in}%
\pgfsys@useobject{currentmarker}{}%
\end{pgfscope}%
\begin{pgfscope}%
\pgfsys@transformshift{5.210576in}{5.541205in}%
\pgfsys@useobject{currentmarker}{}%
\end{pgfscope}%
\begin{pgfscope}%
\pgfsys@transformshift{3.657671in}{3.972159in}%
\pgfsys@useobject{currentmarker}{}%
\end{pgfscope}%
\begin{pgfscope}%
\pgfsys@transformshift{4.365775in}{1.854875in}%
\pgfsys@useobject{currentmarker}{}%
\end{pgfscope}%
\begin{pgfscope}%
\pgfsys@transformshift{4.581518in}{3.919286in}%
\pgfsys@useobject{currentmarker}{}%
\end{pgfscope}%
\begin{pgfscope}%
\pgfsys@transformshift{6.044030in}{3.977402in}%
\pgfsys@useobject{currentmarker}{}%
\end{pgfscope}%
\begin{pgfscope}%
\pgfsys@transformshift{3.257253in}{4.176009in}%
\pgfsys@useobject{currentmarker}{}%
\end{pgfscope}%
\begin{pgfscope}%
\pgfsys@transformshift{0.728425in}{2.306781in}%
\pgfsys@useobject{currentmarker}{}%
\end{pgfscope}%
\begin{pgfscope}%
\pgfsys@transformshift{0.699212in}{0.644844in}%
\pgfsys@useobject{currentmarker}{}%
\end{pgfscope}%
\begin{pgfscope}%
\pgfsys@transformshift{2.824735in}{3.322079in}%
\pgfsys@useobject{currentmarker}{}%
\end{pgfscope}%
\begin{pgfscope}%
\pgfsys@transformshift{5.575909in}{4.168814in}%
\pgfsys@useobject{currentmarker}{}%
\end{pgfscope}%
\begin{pgfscope}%
\pgfsys@transformshift{2.694993in}{0.745092in}%
\pgfsys@useobject{currentmarker}{}%
\end{pgfscope}%
\begin{pgfscope}%
\pgfsys@transformshift{2.827246in}{2.093276in}%
\pgfsys@useobject{currentmarker}{}%
\end{pgfscope}%
\begin{pgfscope}%
\pgfsys@transformshift{4.860671in}{1.087019in}%
\pgfsys@useobject{currentmarker}{}%
\end{pgfscope}%
\begin{pgfscope}%
\pgfsys@transformshift{2.873589in}{1.278336in}%
\pgfsys@useobject{currentmarker}{}%
\end{pgfscope}%
\begin{pgfscope}%
\pgfsys@transformshift{1.536825in}{0.259268in}%
\pgfsys@useobject{currentmarker}{}%
\end{pgfscope}%
\begin{pgfscope}%
\pgfsys@transformshift{2.764352in}{0.681245in}%
\pgfsys@useobject{currentmarker}{}%
\end{pgfscope}%
\begin{pgfscope}%
\pgfsys@transformshift{6.472715in}{3.118065in}%
\pgfsys@useobject{currentmarker}{}%
\end{pgfscope}%
\begin{pgfscope}%
\pgfsys@transformshift{5.451502in}{0.558177in}%
\pgfsys@useobject{currentmarker}{}%
\end{pgfscope}%
\begin{pgfscope}%
\pgfsys@transformshift{4.521637in}{4.721217in}%
\pgfsys@useobject{currentmarker}{}%
\end{pgfscope}%
\begin{pgfscope}%
\pgfsys@transformshift{5.541218in}{1.493272in}%
\pgfsys@useobject{currentmarker}{}%
\end{pgfscope}%
\begin{pgfscope}%
\pgfsys@transformshift{4.253324in}{3.800007in}%
\pgfsys@useobject{currentmarker}{}%
\end{pgfscope}%
\begin{pgfscope}%
\pgfsys@transformshift{5.895094in}{2.687622in}%
\pgfsys@useobject{currentmarker}{}%
\end{pgfscope}%
\begin{pgfscope}%
\pgfsys@transformshift{6.081673in}{1.079201in}%
\pgfsys@useobject{currentmarker}{}%
\end{pgfscope}%
\begin{pgfscope}%
\pgfsys@transformshift{2.484298in}{1.217911in}%
\pgfsys@useobject{currentmarker}{}%
\end{pgfscope}%
\begin{pgfscope}%
\pgfsys@transformshift{5.425424in}{5.125628in}%
\pgfsys@useobject{currentmarker}{}%
\end{pgfscope}%
\begin{pgfscope}%
\pgfsys@transformshift{2.759490in}{2.627771in}%
\pgfsys@useobject{currentmarker}{}%
\end{pgfscope}%
\begin{pgfscope}%
\pgfsys@transformshift{1.602078in}{5.464114in}%
\pgfsys@useobject{currentmarker}{}%
\end{pgfscope}%
\begin{pgfscope}%
\pgfsys@transformshift{2.916437in}{4.833342in}%
\pgfsys@useobject{currentmarker}{}%
\end{pgfscope}%
\begin{pgfscope}%
\pgfsys@transformshift{5.260685in}{3.002060in}%
\pgfsys@useobject{currentmarker}{}%
\end{pgfscope}%
\begin{pgfscope}%
\pgfsys@transformshift{1.180553in}{0.604540in}%
\pgfsys@useobject{currentmarker}{}%
\end{pgfscope}%
\begin{pgfscope}%
\pgfsys@transformshift{5.687892in}{4.590760in}%
\pgfsys@useobject{currentmarker}{}%
\end{pgfscope}%
\begin{pgfscope}%
\pgfsys@transformshift{0.761716in}{5.353630in}%
\pgfsys@useobject{currentmarker}{}%
\end{pgfscope}%
\begin{pgfscope}%
\pgfsys@transformshift{3.249295in}{1.538378in}%
\pgfsys@useobject{currentmarker}{}%
\end{pgfscope}%
\begin{pgfscope}%
\pgfsys@transformshift{5.306802in}{3.756128in}%
\pgfsys@useobject{currentmarker}{}%
\end{pgfscope}%
\begin{pgfscope}%
\pgfsys@transformshift{5.910930in}{2.450328in}%
\pgfsys@useobject{currentmarker}{}%
\end{pgfscope}%
\begin{pgfscope}%
\pgfsys@transformshift{4.532737in}{4.248805in}%
\pgfsys@useobject{currentmarker}{}%
\end{pgfscope}%
\begin{pgfscope}%
\pgfsys@transformshift{6.161062in}{4.110244in}%
\pgfsys@useobject{currentmarker}{}%
\end{pgfscope}%
\begin{pgfscope}%
\pgfsys@transformshift{0.787277in}{2.876043in}%
\pgfsys@useobject{currentmarker}{}%
\end{pgfscope}%
\begin{pgfscope}%
\pgfsys@transformshift{6.267104in}{3.324397in}%
\pgfsys@useobject{currentmarker}{}%
\end{pgfscope}%
\begin{pgfscope}%
\pgfsys@transformshift{2.836078in}{5.207461in}%
\pgfsys@useobject{currentmarker}{}%
\end{pgfscope}%
\begin{pgfscope}%
\pgfsys@transformshift{1.628938in}{3.940938in}%
\pgfsys@useobject{currentmarker}{}%
\end{pgfscope}%
\begin{pgfscope}%
\pgfsys@transformshift{1.324555in}{2.256327in}%
\pgfsys@useobject{currentmarker}{}%
\end{pgfscope}%
\begin{pgfscope}%
\pgfsys@transformshift{6.296936in}{0.536252in}%
\pgfsys@useobject{currentmarker}{}%
\end{pgfscope}%
\begin{pgfscope}%
\pgfsys@transformshift{1.336231in}{0.735668in}%
\pgfsys@useobject{currentmarker}{}%
\end{pgfscope}%
\begin{pgfscope}%
\pgfsys@transformshift{6.633723in}{4.558885in}%
\pgfsys@useobject{currentmarker}{}%
\end{pgfscope}%
\begin{pgfscope}%
\pgfsys@transformshift{0.911925in}{5.373046in}%
\pgfsys@useobject{currentmarker}{}%
\end{pgfscope}%
\begin{pgfscope}%
\pgfsys@transformshift{5.570592in}{2.806925in}%
\pgfsys@useobject{currentmarker}{}%
\end{pgfscope}%
\begin{pgfscope}%
\pgfsys@transformshift{5.204466in}{4.173171in}%
\pgfsys@useobject{currentmarker}{}%
\end{pgfscope}%
\begin{pgfscope}%
\pgfsys@transformshift{2.411060in}{1.472078in}%
\pgfsys@useobject{currentmarker}{}%
\end{pgfscope}%
\begin{pgfscope}%
\pgfsys@transformshift{4.028813in}{2.990967in}%
\pgfsys@useobject{currentmarker}{}%
\end{pgfscope}%
\begin{pgfscope}%
\pgfsys@transformshift{2.181372in}{1.436556in}%
\pgfsys@useobject{currentmarker}{}%
\end{pgfscope}%
\begin{pgfscope}%
\pgfsys@transformshift{2.042943in}{1.212113in}%
\pgfsys@useobject{currentmarker}{}%
\end{pgfscope}%
\begin{pgfscope}%
\pgfsys@transformshift{3.470396in}{1.039351in}%
\pgfsys@useobject{currentmarker}{}%
\end{pgfscope}%
\begin{pgfscope}%
\pgfsys@transformshift{4.279956in}{1.609999in}%
\pgfsys@useobject{currentmarker}{}%
\end{pgfscope}%
\begin{pgfscope}%
\pgfsys@transformshift{5.419420in}{5.133857in}%
\pgfsys@useobject{currentmarker}{}%
\end{pgfscope}%
\begin{pgfscope}%
\pgfsys@transformshift{1.923650in}{2.529221in}%
\pgfsys@useobject{currentmarker}{}%
\end{pgfscope}%
\begin{pgfscope}%
\pgfsys@transformshift{6.254481in}{2.411960in}%
\pgfsys@useobject{currentmarker}{}%
\end{pgfscope}%
\begin{pgfscope}%
\pgfsys@transformshift{1.386280in}{0.596923in}%
\pgfsys@useobject{currentmarker}{}%
\end{pgfscope}%
\begin{pgfscope}%
\pgfsys@transformshift{1.367933in}{5.261716in}%
\pgfsys@useobject{currentmarker}{}%
\end{pgfscope}%
\begin{pgfscope}%
\pgfsys@transformshift{5.902000in}{1.579038in}%
\pgfsys@useobject{currentmarker}{}%
\end{pgfscope}%
\begin{pgfscope}%
\pgfsys@transformshift{3.800574in}{4.788179in}%
\pgfsys@useobject{currentmarker}{}%
\end{pgfscope}%
\begin{pgfscope}%
\pgfsys@transformshift{3.371569in}{3.312704in}%
\pgfsys@useobject{currentmarker}{}%
\end{pgfscope}%
\begin{pgfscope}%
\pgfsys@transformshift{6.073693in}{4.359344in}%
\pgfsys@useobject{currentmarker}{}%
\end{pgfscope}%
\begin{pgfscope}%
\pgfsys@transformshift{4.193810in}{4.467048in}%
\pgfsys@useobject{currentmarker}{}%
\end{pgfscope}%
\begin{pgfscope}%
\pgfsys@transformshift{2.599471in}{4.124679in}%
\pgfsys@useobject{currentmarker}{}%
\end{pgfscope}%
\begin{pgfscope}%
\pgfsys@transformshift{4.514961in}{4.936874in}%
\pgfsys@useobject{currentmarker}{}%
\end{pgfscope}%
\begin{pgfscope}%
\pgfsys@transformshift{3.272950in}{3.195707in}%
\pgfsys@useobject{currentmarker}{}%
\end{pgfscope}%
\begin{pgfscope}%
\pgfsys@transformshift{1.510721in}{2.354600in}%
\pgfsys@useobject{currentmarker}{}%
\end{pgfscope}%
\begin{pgfscope}%
\pgfsys@transformshift{3.264662in}{0.622165in}%
\pgfsys@useobject{currentmarker}{}%
\end{pgfscope}%
\begin{pgfscope}%
\pgfsys@transformshift{3.625515in}{3.140835in}%
\pgfsys@useobject{currentmarker}{}%
\end{pgfscope}%
\begin{pgfscope}%
\pgfsys@transformshift{2.563533in}{1.864619in}%
\pgfsys@useobject{currentmarker}{}%
\end{pgfscope}%
\begin{pgfscope}%
\pgfsys@transformshift{4.278420in}{1.568148in}%
\pgfsys@useobject{currentmarker}{}%
\end{pgfscope}%
\begin{pgfscope}%
\pgfsys@transformshift{6.133139in}{1.593437in}%
\pgfsys@useobject{currentmarker}{}%
\end{pgfscope}%
\begin{pgfscope}%
\pgfsys@transformshift{4.848546in}{3.568652in}%
\pgfsys@useobject{currentmarker}{}%
\end{pgfscope}%
\begin{pgfscope}%
\pgfsys@transformshift{6.723860in}{3.848760in}%
\pgfsys@useobject{currentmarker}{}%
\end{pgfscope}%
\begin{pgfscope}%
\pgfsys@transformshift{1.457826in}{4.531374in}%
\pgfsys@useobject{currentmarker}{}%
\end{pgfscope}%
\begin{pgfscope}%
\pgfsys@transformshift{6.297737in}{1.887891in}%
\pgfsys@useobject{currentmarker}{}%
\end{pgfscope}%
\begin{pgfscope}%
\pgfsys@transformshift{2.346430in}{0.797244in}%
\pgfsys@useobject{currentmarker}{}%
\end{pgfscope}%
\begin{pgfscope}%
\pgfsys@transformshift{6.648502in}{3.096227in}%
\pgfsys@useobject{currentmarker}{}%
\end{pgfscope}%
\begin{pgfscope}%
\pgfsys@transformshift{4.173255in}{3.519895in}%
\pgfsys@useobject{currentmarker}{}%
\end{pgfscope}%
\begin{pgfscope}%
\pgfsys@transformshift{2.165725in}{1.381293in}%
\pgfsys@useobject{currentmarker}{}%
\end{pgfscope}%
\begin{pgfscope}%
\pgfsys@transformshift{2.721770in}{1.510947in}%
\pgfsys@useobject{currentmarker}{}%
\end{pgfscope}%
\begin{pgfscope}%
\pgfsys@transformshift{3.548642in}{1.282580in}%
\pgfsys@useobject{currentmarker}{}%
\end{pgfscope}%
\begin{pgfscope}%
\pgfsys@transformshift{2.475107in}{4.994021in}%
\pgfsys@useobject{currentmarker}{}%
\end{pgfscope}%
\begin{pgfscope}%
\pgfsys@transformshift{1.470327in}{2.482215in}%
\pgfsys@useobject{currentmarker}{}%
\end{pgfscope}%
\begin{pgfscope}%
\pgfsys@transformshift{6.669199in}{5.298142in}%
\pgfsys@useobject{currentmarker}{}%
\end{pgfscope}%
\begin{pgfscope}%
\pgfsys@transformshift{0.827830in}{1.636520in}%
\pgfsys@useobject{currentmarker}{}%
\end{pgfscope}%
\begin{pgfscope}%
\pgfsys@transformshift{5.032274in}{4.734021in}%
\pgfsys@useobject{currentmarker}{}%
\end{pgfscope}%
\begin{pgfscope}%
\pgfsys@transformshift{4.672835in}{2.191141in}%
\pgfsys@useobject{currentmarker}{}%
\end{pgfscope}%
\begin{pgfscope}%
\pgfsys@transformshift{3.169599in}{2.938631in}%
\pgfsys@useobject{currentmarker}{}%
\end{pgfscope}%
\begin{pgfscope}%
\pgfsys@transformshift{2.953476in}{1.578938in}%
\pgfsys@useobject{currentmarker}{}%
\end{pgfscope}%
\begin{pgfscope}%
\pgfsys@transformshift{2.521981in}{4.095214in}%
\pgfsys@useobject{currentmarker}{}%
\end{pgfscope}%
\begin{pgfscope}%
\pgfsys@transformshift{2.205016in}{0.421058in}%
\pgfsys@useobject{currentmarker}{}%
\end{pgfscope}%
\begin{pgfscope}%
\pgfsys@transformshift{5.352028in}{4.064159in}%
\pgfsys@useobject{currentmarker}{}%
\end{pgfscope}%
\begin{pgfscope}%
\pgfsys@transformshift{4.636411in}{5.380365in}%
\pgfsys@useobject{currentmarker}{}%
\end{pgfscope}%
\begin{pgfscope}%
\pgfsys@transformshift{2.770858in}{1.945693in}%
\pgfsys@useobject{currentmarker}{}%
\end{pgfscope}%
\begin{pgfscope}%
\pgfsys@transformshift{0.782278in}{5.103194in}%
\pgfsys@useobject{currentmarker}{}%
\end{pgfscope}%
\begin{pgfscope}%
\pgfsys@transformshift{1.775147in}{2.990035in}%
\pgfsys@useobject{currentmarker}{}%
\end{pgfscope}%
\begin{pgfscope}%
\pgfsys@transformshift{2.358027in}{3.214200in}%
\pgfsys@useobject{currentmarker}{}%
\end{pgfscope}%
\begin{pgfscope}%
\pgfsys@transformshift{3.922695in}{4.973715in}%
\pgfsys@useobject{currentmarker}{}%
\end{pgfscope}%
\begin{pgfscope}%
\pgfsys@transformshift{3.978582in}{4.103172in}%
\pgfsys@useobject{currentmarker}{}%
\end{pgfscope}%
\begin{pgfscope}%
\pgfsys@transformshift{4.368506in}{5.467180in}%
\pgfsys@useobject{currentmarker}{}%
\end{pgfscope}%
\begin{pgfscope}%
\pgfsys@transformshift{3.193502in}{1.561359in}%
\pgfsys@useobject{currentmarker}{}%
\end{pgfscope}%
\begin{pgfscope}%
\pgfsys@transformshift{6.288622in}{3.614347in}%
\pgfsys@useobject{currentmarker}{}%
\end{pgfscope}%
\begin{pgfscope}%
\pgfsys@transformshift{4.084287in}{3.312912in}%
\pgfsys@useobject{currentmarker}{}%
\end{pgfscope}%
\begin{pgfscope}%
\pgfsys@transformshift{1.347858in}{3.946126in}%
\pgfsys@useobject{currentmarker}{}%
\end{pgfscope}%
\begin{pgfscope}%
\pgfsys@transformshift{4.769809in}{3.591916in}%
\pgfsys@useobject{currentmarker}{}%
\end{pgfscope}%
\begin{pgfscope}%
\pgfsys@transformshift{3.773972in}{0.287757in}%
\pgfsys@useobject{currentmarker}{}%
\end{pgfscope}%
\begin{pgfscope}%
\pgfsys@transformshift{2.176257in}{4.675706in}%
\pgfsys@useobject{currentmarker}{}%
\end{pgfscope}%
\begin{pgfscope}%
\pgfsys@transformshift{4.512409in}{3.576521in}%
\pgfsys@useobject{currentmarker}{}%
\end{pgfscope}%
\begin{pgfscope}%
\pgfsys@transformshift{2.101242in}{4.547340in}%
\pgfsys@useobject{currentmarker}{}%
\end{pgfscope}%
\begin{pgfscope}%
\pgfsys@transformshift{1.753081in}{0.789429in}%
\pgfsys@useobject{currentmarker}{}%
\end{pgfscope}%
\begin{pgfscope}%
\pgfsys@transformshift{4.632241in}{0.212409in}%
\pgfsys@useobject{currentmarker}{}%
\end{pgfscope}%
\begin{pgfscope}%
\pgfsys@transformshift{0.632041in}{2.836243in}%
\pgfsys@useobject{currentmarker}{}%
\end{pgfscope}%
\begin{pgfscope}%
\pgfsys@transformshift{4.936656in}{4.301994in}%
\pgfsys@useobject{currentmarker}{}%
\end{pgfscope}%
\begin{pgfscope}%
\pgfsys@transformshift{6.689590in}{3.328316in}%
\pgfsys@useobject{currentmarker}{}%
\end{pgfscope}%
\begin{pgfscope}%
\pgfsys@transformshift{4.217622in}{4.612652in}%
\pgfsys@useobject{currentmarker}{}%
\end{pgfscope}%
\begin{pgfscope}%
\pgfsys@transformshift{2.619341in}{1.072985in}%
\pgfsys@useobject{currentmarker}{}%
\end{pgfscope}%
\begin{pgfscope}%
\pgfsys@transformshift{5.300125in}{5.145135in}%
\pgfsys@useobject{currentmarker}{}%
\end{pgfscope}%
\begin{pgfscope}%
\pgfsys@transformshift{3.358836in}{5.209199in}%
\pgfsys@useobject{currentmarker}{}%
\end{pgfscope}%
\begin{pgfscope}%
\pgfsys@transformshift{2.044740in}{2.143881in}%
\pgfsys@useobject{currentmarker}{}%
\end{pgfscope}%
\begin{pgfscope}%
\pgfsys@transformshift{5.852632in}{5.298387in}%
\pgfsys@useobject{currentmarker}{}%
\end{pgfscope}%
\begin{pgfscope}%
\pgfsys@transformshift{4.560976in}{5.361356in}%
\pgfsys@useobject{currentmarker}{}%
\end{pgfscope}%
\begin{pgfscope}%
\pgfsys@transformshift{2.888138in}{1.740370in}%
\pgfsys@useobject{currentmarker}{}%
\end{pgfscope}%
\begin{pgfscope}%
\pgfsys@transformshift{3.882673in}{5.078881in}%
\pgfsys@useobject{currentmarker}{}%
\end{pgfscope}%
\begin{pgfscope}%
\pgfsys@transformshift{1.316389in}{2.227657in}%
\pgfsys@useobject{currentmarker}{}%
\end{pgfscope}%
\begin{pgfscope}%
\pgfsys@transformshift{5.300434in}{3.557414in}%
\pgfsys@useobject{currentmarker}{}%
\end{pgfscope}%
\begin{pgfscope}%
\pgfsys@transformshift{1.980597in}{1.453799in}%
\pgfsys@useobject{currentmarker}{}%
\end{pgfscope}%
\begin{pgfscope}%
\pgfsys@transformshift{0.695441in}{3.262340in}%
\pgfsys@useobject{currentmarker}{}%
\end{pgfscope}%
\begin{pgfscope}%
\pgfsys@transformshift{6.706101in}{0.873051in}%
\pgfsys@useobject{currentmarker}{}%
\end{pgfscope}%
\begin{pgfscope}%
\pgfsys@transformshift{5.474031in}{1.664002in}%
\pgfsys@useobject{currentmarker}{}%
\end{pgfscope}%
\begin{pgfscope}%
\pgfsys@transformshift{4.718805in}{5.538386in}%
\pgfsys@useobject{currentmarker}{}%
\end{pgfscope}%
\begin{pgfscope}%
\pgfsys@transformshift{4.886382in}{2.207520in}%
\pgfsys@useobject{currentmarker}{}%
\end{pgfscope}%
\begin{pgfscope}%
\pgfsys@transformshift{2.099852in}{3.710927in}%
\pgfsys@useobject{currentmarker}{}%
\end{pgfscope}%
\begin{pgfscope}%
\pgfsys@transformshift{6.633788in}{4.370551in}%
\pgfsys@useobject{currentmarker}{}%
\end{pgfscope}%
\begin{pgfscope}%
\pgfsys@transformshift{6.665277in}{4.003804in}%
\pgfsys@useobject{currentmarker}{}%
\end{pgfscope}%
\begin{pgfscope}%
\pgfsys@transformshift{6.761160in}{5.179835in}%
\pgfsys@useobject{currentmarker}{}%
\end{pgfscope}%
\begin{pgfscope}%
\pgfsys@transformshift{0.733463in}{4.564859in}%
\pgfsys@useobject{currentmarker}{}%
\end{pgfscope}%
\begin{pgfscope}%
\pgfsys@transformshift{0.844439in}{3.911732in}%
\pgfsys@useobject{currentmarker}{}%
\end{pgfscope}%
\begin{pgfscope}%
\pgfsys@transformshift{5.381436in}{3.192049in}%
\pgfsys@useobject{currentmarker}{}%
\end{pgfscope}%
\begin{pgfscope}%
\pgfsys@transformshift{3.807021in}{0.858638in}%
\pgfsys@useobject{currentmarker}{}%
\end{pgfscope}%
\begin{pgfscope}%
\pgfsys@transformshift{2.391056in}{1.264711in}%
\pgfsys@useobject{currentmarker}{}%
\end{pgfscope}%
\begin{pgfscope}%
\pgfsys@transformshift{2.948632in}{0.706713in}%
\pgfsys@useobject{currentmarker}{}%
\end{pgfscope}%
\begin{pgfscope}%
\pgfsys@transformshift{3.242636in}{0.463250in}%
\pgfsys@useobject{currentmarker}{}%
\end{pgfscope}%
\begin{pgfscope}%
\pgfsys@transformshift{0.758804in}{5.598900in}%
\pgfsys@useobject{currentmarker}{}%
\end{pgfscope}%
\begin{pgfscope}%
\pgfsys@transformshift{2.299802in}{0.900808in}%
\pgfsys@useobject{currentmarker}{}%
\end{pgfscope}%
\begin{pgfscope}%
\pgfsys@transformshift{3.295763in}{1.279436in}%
\pgfsys@useobject{currentmarker}{}%
\end{pgfscope}%
\begin{pgfscope}%
\pgfsys@transformshift{6.553575in}{2.015242in}%
\pgfsys@useobject{currentmarker}{}%
\end{pgfscope}%
\begin{pgfscope}%
\pgfsys@transformshift{4.901427in}{2.011591in}%
\pgfsys@useobject{currentmarker}{}%
\end{pgfscope}%
\begin{pgfscope}%
\pgfsys@transformshift{3.869181in}{0.528128in}%
\pgfsys@useobject{currentmarker}{}%
\end{pgfscope}%
\begin{pgfscope}%
\pgfsys@transformshift{0.816007in}{3.698963in}%
\pgfsys@useobject{currentmarker}{}%
\end{pgfscope}%
\begin{pgfscope}%
\pgfsys@transformshift{3.384732in}{5.596170in}%
\pgfsys@useobject{currentmarker}{}%
\end{pgfscope}%
\begin{pgfscope}%
\pgfsys@transformshift{1.883288in}{5.003959in}%
\pgfsys@useobject{currentmarker}{}%
\end{pgfscope}%
\begin{pgfscope}%
\pgfsys@transformshift{1.174030in}{4.249498in}%
\pgfsys@useobject{currentmarker}{}%
\end{pgfscope}%
\begin{pgfscope}%
\pgfsys@transformshift{3.011708in}{4.125365in}%
\pgfsys@useobject{currentmarker}{}%
\end{pgfscope}%
\begin{pgfscope}%
\pgfsys@transformshift{2.049231in}{1.517104in}%
\pgfsys@useobject{currentmarker}{}%
\end{pgfscope}%
\begin{pgfscope}%
\pgfsys@transformshift{6.343334in}{1.037194in}%
\pgfsys@useobject{currentmarker}{}%
\end{pgfscope}%
\begin{pgfscope}%
\pgfsys@transformshift{6.000403in}{3.813646in}%
\pgfsys@useobject{currentmarker}{}%
\end{pgfscope}%
\begin{pgfscope}%
\pgfsys@transformshift{3.776016in}{5.515685in}%
\pgfsys@useobject{currentmarker}{}%
\end{pgfscope}%
\begin{pgfscope}%
\pgfsys@transformshift{4.647949in}{0.315952in}%
\pgfsys@useobject{currentmarker}{}%
\end{pgfscope}%
\begin{pgfscope}%
\pgfsys@transformshift{4.495272in}{5.387673in}%
\pgfsys@useobject{currentmarker}{}%
\end{pgfscope}%
\begin{pgfscope}%
\pgfsys@transformshift{4.731555in}{2.352484in}%
\pgfsys@useobject{currentmarker}{}%
\end{pgfscope}%
\begin{pgfscope}%
\pgfsys@transformshift{1.649823in}{1.237897in}%
\pgfsys@useobject{currentmarker}{}%
\end{pgfscope}%
\begin{pgfscope}%
\pgfsys@transformshift{1.616021in}{4.337931in}%
\pgfsys@useobject{currentmarker}{}%
\end{pgfscope}%
\begin{pgfscope}%
\pgfsys@transformshift{5.747019in}{4.866110in}%
\pgfsys@useobject{currentmarker}{}%
\end{pgfscope}%
\begin{pgfscope}%
\pgfsys@transformshift{0.883956in}{0.861120in}%
\pgfsys@useobject{currentmarker}{}%
\end{pgfscope}%
\begin{pgfscope}%
\pgfsys@transformshift{2.898364in}{4.087447in}%
\pgfsys@useobject{currentmarker}{}%
\end{pgfscope}%
\begin{pgfscope}%
\pgfsys@transformshift{3.156722in}{2.291898in}%
\pgfsys@useobject{currentmarker}{}%
\end{pgfscope}%
\begin{pgfscope}%
\pgfsys@transformshift{2.298231in}{4.039262in}%
\pgfsys@useobject{currentmarker}{}%
\end{pgfscope}%
\begin{pgfscope}%
\pgfsys@transformshift{1.889708in}{3.986759in}%
\pgfsys@useobject{currentmarker}{}%
\end{pgfscope}%
\begin{pgfscope}%
\pgfsys@transformshift{5.893202in}{3.444348in}%
\pgfsys@useobject{currentmarker}{}%
\end{pgfscope}%
\begin{pgfscope}%
\pgfsys@transformshift{5.171030in}{4.730581in}%
\pgfsys@useobject{currentmarker}{}%
\end{pgfscope}%
\begin{pgfscope}%
\pgfsys@transformshift{6.623684in}{4.946985in}%
\pgfsys@useobject{currentmarker}{}%
\end{pgfscope}%
\begin{pgfscope}%
\pgfsys@transformshift{2.202053in}{3.748833in}%
\pgfsys@useobject{currentmarker}{}%
\end{pgfscope}%
\begin{pgfscope}%
\pgfsys@transformshift{0.806785in}{2.518087in}%
\pgfsys@useobject{currentmarker}{}%
\end{pgfscope}%
\begin{pgfscope}%
\pgfsys@transformshift{6.119562in}{2.890613in}%
\pgfsys@useobject{currentmarker}{}%
\end{pgfscope}%
\begin{pgfscope}%
\pgfsys@transformshift{4.361458in}{2.021060in}%
\pgfsys@useobject{currentmarker}{}%
\end{pgfscope}%
\begin{pgfscope}%
\pgfsys@transformshift{4.079672in}{4.818521in}%
\pgfsys@useobject{currentmarker}{}%
\end{pgfscope}%
\begin{pgfscope}%
\pgfsys@transformshift{5.987951in}{1.287495in}%
\pgfsys@useobject{currentmarker}{}%
\end{pgfscope}%
\begin{pgfscope}%
\pgfsys@transformshift{4.502762in}{1.680264in}%
\pgfsys@useobject{currentmarker}{}%
\end{pgfscope}%
\begin{pgfscope}%
\pgfsys@transformshift{6.715823in}{0.617697in}%
\pgfsys@useobject{currentmarker}{}%
\end{pgfscope}%
\begin{pgfscope}%
\pgfsys@transformshift{4.070039in}{4.771017in}%
\pgfsys@useobject{currentmarker}{}%
\end{pgfscope}%
\begin{pgfscope}%
\pgfsys@transformshift{5.057400in}{2.521292in}%
\pgfsys@useobject{currentmarker}{}%
\end{pgfscope}%
\begin{pgfscope}%
\pgfsys@transformshift{4.514795in}{3.566155in}%
\pgfsys@useobject{currentmarker}{}%
\end{pgfscope}%
\begin{pgfscope}%
\pgfsys@transformshift{6.362345in}{1.868520in}%
\pgfsys@useobject{currentmarker}{}%
\end{pgfscope}%
\begin{pgfscope}%
\pgfsys@transformshift{1.841212in}{0.507564in}%
\pgfsys@useobject{currentmarker}{}%
\end{pgfscope}%
\begin{pgfscope}%
\pgfsys@transformshift{1.487338in}{4.673985in}%
\pgfsys@useobject{currentmarker}{}%
\end{pgfscope}%
\begin{pgfscope}%
\pgfsys@transformshift{6.597316in}{3.374792in}%
\pgfsys@useobject{currentmarker}{}%
\end{pgfscope}%
\begin{pgfscope}%
\pgfsys@transformshift{4.460198in}{3.902970in}%
\pgfsys@useobject{currentmarker}{}%
\end{pgfscope}%
\begin{pgfscope}%
\pgfsys@transformshift{1.318192in}{2.419967in}%
\pgfsys@useobject{currentmarker}{}%
\end{pgfscope}%
\begin{pgfscope}%
\pgfsys@transformshift{1.613930in}{0.413143in}%
\pgfsys@useobject{currentmarker}{}%
\end{pgfscope}%
\begin{pgfscope}%
\pgfsys@transformshift{5.257206in}{1.360614in}%
\pgfsys@useobject{currentmarker}{}%
\end{pgfscope}%
\begin{pgfscope}%
\pgfsys@transformshift{1.440141in}{1.772643in}%
\pgfsys@useobject{currentmarker}{}%
\end{pgfscope}%
\begin{pgfscope}%
\pgfsys@transformshift{2.953166in}{3.772788in}%
\pgfsys@useobject{currentmarker}{}%
\end{pgfscope}%
\begin{pgfscope}%
\pgfsys@transformshift{4.214887in}{4.188086in}%
\pgfsys@useobject{currentmarker}{}%
\end{pgfscope}%
\begin{pgfscope}%
\pgfsys@transformshift{4.789466in}{2.974967in}%
\pgfsys@useobject{currentmarker}{}%
\end{pgfscope}%
\begin{pgfscope}%
\pgfsys@transformshift{5.488053in}{3.383650in}%
\pgfsys@useobject{currentmarker}{}%
\end{pgfscope}%
\begin{pgfscope}%
\pgfsys@transformshift{4.578398in}{2.975727in}%
\pgfsys@useobject{currentmarker}{}%
\end{pgfscope}%
\begin{pgfscope}%
\pgfsys@transformshift{5.907316in}{1.755087in}%
\pgfsys@useobject{currentmarker}{}%
\end{pgfscope}%
\begin{pgfscope}%
\pgfsys@transformshift{0.794864in}{1.561967in}%
\pgfsys@useobject{currentmarker}{}%
\end{pgfscope}%
\begin{pgfscope}%
\pgfsys@transformshift{4.611415in}{0.866224in}%
\pgfsys@useobject{currentmarker}{}%
\end{pgfscope}%
\begin{pgfscope}%
\pgfsys@transformshift{1.024983in}{5.547819in}%
\pgfsys@useobject{currentmarker}{}%
\end{pgfscope}%
\begin{pgfscope}%
\pgfsys@transformshift{2.096234in}{0.799507in}%
\pgfsys@useobject{currentmarker}{}%
\end{pgfscope}%
\begin{pgfscope}%
\pgfsys@transformshift{1.569828in}{0.385262in}%
\pgfsys@useobject{currentmarker}{}%
\end{pgfscope}%
\begin{pgfscope}%
\pgfsys@transformshift{2.777036in}{5.405926in}%
\pgfsys@useobject{currentmarker}{}%
\end{pgfscope}%
\begin{pgfscope}%
\pgfsys@transformshift{2.785162in}{1.086089in}%
\pgfsys@useobject{currentmarker}{}%
\end{pgfscope}%
\begin{pgfscope}%
\pgfsys@transformshift{3.728300in}{1.947660in}%
\pgfsys@useobject{currentmarker}{}%
\end{pgfscope}%
\begin{pgfscope}%
\pgfsys@transformshift{5.521787in}{2.578131in}%
\pgfsys@useobject{currentmarker}{}%
\end{pgfscope}%
\begin{pgfscope}%
\pgfsys@transformshift{3.465046in}{3.551128in}%
\pgfsys@useobject{currentmarker}{}%
\end{pgfscope}%
\begin{pgfscope}%
\pgfsys@transformshift{6.422339in}{4.732718in}%
\pgfsys@useobject{currentmarker}{}%
\end{pgfscope}%
\begin{pgfscope}%
\pgfsys@transformshift{1.621743in}{4.623190in}%
\pgfsys@useobject{currentmarker}{}%
\end{pgfscope}%
\begin{pgfscope}%
\pgfsys@transformshift{0.890611in}{4.445590in}%
\pgfsys@useobject{currentmarker}{}%
\end{pgfscope}%
\begin{pgfscope}%
\pgfsys@transformshift{5.947438in}{4.060954in}%
\pgfsys@useobject{currentmarker}{}%
\end{pgfscope}%
\begin{pgfscope}%
\pgfsys@transformshift{2.512667in}{5.565949in}%
\pgfsys@useobject{currentmarker}{}%
\end{pgfscope}%
\begin{pgfscope}%
\pgfsys@transformshift{4.067232in}{5.013550in}%
\pgfsys@useobject{currentmarker}{}%
\end{pgfscope}%
\begin{pgfscope}%
\pgfsys@transformshift{2.448968in}{2.405912in}%
\pgfsys@useobject{currentmarker}{}%
\end{pgfscope}%
\begin{pgfscope}%
\pgfsys@transformshift{4.072426in}{5.401527in}%
\pgfsys@useobject{currentmarker}{}%
\end{pgfscope}%
\begin{pgfscope}%
\pgfsys@transformshift{0.857677in}{4.708843in}%
\pgfsys@useobject{currentmarker}{}%
\end{pgfscope}%
\begin{pgfscope}%
\pgfsys@transformshift{6.282338in}{2.736996in}%
\pgfsys@useobject{currentmarker}{}%
\end{pgfscope}%
\begin{pgfscope}%
\pgfsys@transformshift{5.093364in}{5.042830in}%
\pgfsys@useobject{currentmarker}{}%
\end{pgfscope}%
\begin{pgfscope}%
\pgfsys@transformshift{3.571786in}{1.453709in}%
\pgfsys@useobject{currentmarker}{}%
\end{pgfscope}%
\begin{pgfscope}%
\pgfsys@transformshift{1.562124in}{0.839372in}%
\pgfsys@useobject{currentmarker}{}%
\end{pgfscope}%
\begin{pgfscope}%
\pgfsys@transformshift{0.816111in}{3.179639in}%
\pgfsys@useobject{currentmarker}{}%
\end{pgfscope}%
\begin{pgfscope}%
\pgfsys@transformshift{2.160495in}{0.629547in}%
\pgfsys@useobject{currentmarker}{}%
\end{pgfscope}%
\begin{pgfscope}%
\pgfsys@transformshift{4.651263in}{3.714401in}%
\pgfsys@useobject{currentmarker}{}%
\end{pgfscope}%
\begin{pgfscope}%
\pgfsys@transformshift{4.577791in}{2.654746in}%
\pgfsys@useobject{currentmarker}{}%
\end{pgfscope}%
\begin{pgfscope}%
\pgfsys@transformshift{6.728982in}{2.689800in}%
\pgfsys@useobject{currentmarker}{}%
\end{pgfscope}%
\begin{pgfscope}%
\pgfsys@transformshift{4.891599in}{4.563249in}%
\pgfsys@useobject{currentmarker}{}%
\end{pgfscope}%
\begin{pgfscope}%
\pgfsys@transformshift{6.740999in}{3.849346in}%
\pgfsys@useobject{currentmarker}{}%
\end{pgfscope}%
\begin{pgfscope}%
\pgfsys@transformshift{4.099947in}{4.262015in}%
\pgfsys@useobject{currentmarker}{}%
\end{pgfscope}%
\begin{pgfscope}%
\pgfsys@transformshift{1.034388in}{0.344090in}%
\pgfsys@useobject{currentmarker}{}%
\end{pgfscope}%
\begin{pgfscope}%
\pgfsys@transformshift{4.420099in}{4.245184in}%
\pgfsys@useobject{currentmarker}{}%
\end{pgfscope}%
\begin{pgfscope}%
\pgfsys@transformshift{5.747745in}{4.768645in}%
\pgfsys@useobject{currentmarker}{}%
\end{pgfscope}%
\begin{pgfscope}%
\pgfsys@transformshift{3.239791in}{1.440911in}%
\pgfsys@useobject{currentmarker}{}%
\end{pgfscope}%
\begin{pgfscope}%
\pgfsys@transformshift{6.718639in}{1.029180in}%
\pgfsys@useobject{currentmarker}{}%
\end{pgfscope}%
\begin{pgfscope}%
\pgfsys@transformshift{6.744954in}{3.132132in}%
\pgfsys@useobject{currentmarker}{}%
\end{pgfscope}%
\begin{pgfscope}%
\pgfsys@transformshift{2.462530in}{4.768330in}%
\pgfsys@useobject{currentmarker}{}%
\end{pgfscope}%
\begin{pgfscope}%
\pgfsys@transformshift{4.109290in}{4.745653in}%
\pgfsys@useobject{currentmarker}{}%
\end{pgfscope}%
\begin{pgfscope}%
\pgfsys@transformshift{3.063840in}{4.490096in}%
\pgfsys@useobject{currentmarker}{}%
\end{pgfscope}%
\begin{pgfscope}%
\pgfsys@transformshift{3.621973in}{5.508516in}%
\pgfsys@useobject{currentmarker}{}%
\end{pgfscope}%
\begin{pgfscope}%
\pgfsys@transformshift{4.459954in}{4.106209in}%
\pgfsys@useobject{currentmarker}{}%
\end{pgfscope}%
\begin{pgfscope}%
\pgfsys@transformshift{6.170199in}{2.008509in}%
\pgfsys@useobject{currentmarker}{}%
\end{pgfscope}%
\begin{pgfscope}%
\pgfsys@transformshift{4.627067in}{0.828684in}%
\pgfsys@useobject{currentmarker}{}%
\end{pgfscope}%
\begin{pgfscope}%
\pgfsys@transformshift{4.432043in}{2.914979in}%
\pgfsys@useobject{currentmarker}{}%
\end{pgfscope}%
\begin{pgfscope}%
\pgfsys@transformshift{4.727994in}{4.194037in}%
\pgfsys@useobject{currentmarker}{}%
\end{pgfscope}%
\begin{pgfscope}%
\pgfsys@transformshift{6.576946in}{3.399158in}%
\pgfsys@useobject{currentmarker}{}%
\end{pgfscope}%
\begin{pgfscope}%
\pgfsys@transformshift{5.099498in}{0.575336in}%
\pgfsys@useobject{currentmarker}{}%
\end{pgfscope}%
\begin{pgfscope}%
\pgfsys@transformshift{3.780532in}{2.528300in}%
\pgfsys@useobject{currentmarker}{}%
\end{pgfscope}%
\begin{pgfscope}%
\pgfsys@transformshift{4.323725in}{2.496238in}%
\pgfsys@useobject{currentmarker}{}%
\end{pgfscope}%
\begin{pgfscope}%
\pgfsys@transformshift{3.260696in}{1.952726in}%
\pgfsys@useobject{currentmarker}{}%
\end{pgfscope}%
\begin{pgfscope}%
\pgfsys@transformshift{1.045688in}{1.410939in}%
\pgfsys@useobject{currentmarker}{}%
\end{pgfscope}%
\begin{pgfscope}%
\pgfsys@transformshift{5.821127in}{4.339093in}%
\pgfsys@useobject{currentmarker}{}%
\end{pgfscope}%
\begin{pgfscope}%
\pgfsys@transformshift{5.566394in}{4.342515in}%
\pgfsys@useobject{currentmarker}{}%
\end{pgfscope}%
\begin{pgfscope}%
\pgfsys@transformshift{3.639947in}{2.206803in}%
\pgfsys@useobject{currentmarker}{}%
\end{pgfscope}%
\begin{pgfscope}%
\pgfsys@transformshift{5.561833in}{4.281859in}%
\pgfsys@useobject{currentmarker}{}%
\end{pgfscope}%
\begin{pgfscope}%
\pgfsys@transformshift{1.847167in}{0.855194in}%
\pgfsys@useobject{currentmarker}{}%
\end{pgfscope}%
\begin{pgfscope}%
\pgfsys@transformshift{6.708460in}{3.253407in}%
\pgfsys@useobject{currentmarker}{}%
\end{pgfscope}%
\begin{pgfscope}%
\pgfsys@transformshift{3.909865in}{2.742576in}%
\pgfsys@useobject{currentmarker}{}%
\end{pgfscope}%
\begin{pgfscope}%
\pgfsys@transformshift{2.269464in}{3.395422in}%
\pgfsys@useobject{currentmarker}{}%
\end{pgfscope}%
\begin{pgfscope}%
\pgfsys@transformshift{6.875543in}{4.279453in}%
\pgfsys@useobject{currentmarker}{}%
\end{pgfscope}%
\begin{pgfscope}%
\pgfsys@transformshift{3.861631in}{2.252622in}%
\pgfsys@useobject{currentmarker}{}%
\end{pgfscope}%
\begin{pgfscope}%
\pgfsys@transformshift{2.729439in}{2.717699in}%
\pgfsys@useobject{currentmarker}{}%
\end{pgfscope}%
\end{pgfscope}%
\begin{pgfscope}%
\pgfpathrectangle{\pgfqpoint{0.626386in}{0.608332in}}{\pgfqpoint{6.200000in}{4.620000in}}%
\pgfusepath{clip}%
\pgfsetbuttcap%
\pgfsetroundjoin%
\definecolor{currentfill}{rgb}{0.839216,0.152941,0.156863}%
\pgfsetfillcolor{currentfill}%
\pgfsetlinewidth{1.003750pt}%
\definecolor{currentstroke}{rgb}{0.839216,0.152941,0.156863}%
\pgfsetstrokecolor{currentstroke}%
\pgfsetdash{}{0pt}%
\pgfsys@defobject{currentmarker}{\pgfqpoint{-0.031056in}{-0.031056in}}{\pgfqpoint{0.031056in}{0.031056in}}{%
\pgfpathmoveto{\pgfqpoint{0.000000in}{-0.031056in}}%
\pgfpathcurveto{\pgfqpoint{0.008236in}{-0.031056in}}{\pgfqpoint{0.016136in}{-0.027784in}}{\pgfqpoint{0.021960in}{-0.021960in}}%
\pgfpathcurveto{\pgfqpoint{0.027784in}{-0.016136in}}{\pgfqpoint{0.031056in}{-0.008236in}}{\pgfqpoint{0.031056in}{0.000000in}}%
\pgfpathcurveto{\pgfqpoint{0.031056in}{0.008236in}}{\pgfqpoint{0.027784in}{0.016136in}}{\pgfqpoint{0.021960in}{0.021960in}}%
\pgfpathcurveto{\pgfqpoint{0.016136in}{0.027784in}}{\pgfqpoint{0.008236in}{0.031056in}}{\pgfqpoint{0.000000in}{0.031056in}}%
\pgfpathcurveto{\pgfqpoint{-0.008236in}{0.031056in}}{\pgfqpoint{-0.016136in}{0.027784in}}{\pgfqpoint{-0.021960in}{0.021960in}}%
\pgfpathcurveto{\pgfqpoint{-0.027784in}{0.016136in}}{\pgfqpoint{-0.031056in}{0.008236in}}{\pgfqpoint{-0.031056in}{0.000000in}}%
\pgfpathcurveto{\pgfqpoint{-0.031056in}{-0.008236in}}{\pgfqpoint{-0.027784in}{-0.016136in}}{\pgfqpoint{-0.021960in}{-0.021960in}}%
\pgfpathcurveto{\pgfqpoint{-0.016136in}{-0.027784in}}{\pgfqpoint{-0.008236in}{-0.031056in}}{\pgfqpoint{0.000000in}{-0.031056in}}%
\pgfpathlineto{\pgfqpoint{0.000000in}{-0.031056in}}%
\pgfpathclose%
\pgfusepath{stroke,fill}%
}%
\begin{pgfscope}%
\pgfsys@transformshift{1.171002in}{2.753761in}%
\pgfsys@useobject{currentmarker}{}%
\end{pgfscope}%
\begin{pgfscope}%
\pgfsys@transformshift{2.672978in}{1.285705in}%
\pgfsys@useobject{currentmarker}{}%
\end{pgfscope}%
\begin{pgfscope}%
\pgfsys@transformshift{2.536411in}{1.480725in}%
\pgfsys@useobject{currentmarker}{}%
\end{pgfscope}%
\begin{pgfscope}%
\pgfsys@transformshift{6.563792in}{0.571309in}%
\pgfsys@useobject{currentmarker}{}%
\end{pgfscope}%
\begin{pgfscope}%
\pgfsys@transformshift{4.968865in}{0.711625in}%
\pgfsys@useobject{currentmarker}{}%
\end{pgfscope}%
\begin{pgfscope}%
\pgfsys@transformshift{4.114124in}{0.764184in}%
\pgfsys@useobject{currentmarker}{}%
\end{pgfscope}%
\begin{pgfscope}%
\pgfsys@transformshift{2.650795in}{1.247392in}%
\pgfsys@useobject{currentmarker}{}%
\end{pgfscope}%
\begin{pgfscope}%
\pgfsys@transformshift{3.771139in}{0.864515in}%
\pgfsys@useobject{currentmarker}{}%
\end{pgfscope}%
\begin{pgfscope}%
\pgfsys@transformshift{3.098873in}{1.320918in}%
\pgfsys@useobject{currentmarker}{}%
\end{pgfscope}%
\begin{pgfscope}%
\pgfsys@transformshift{2.594401in}{1.640905in}%
\pgfsys@useobject{currentmarker}{}%
\end{pgfscope}%
\begin{pgfscope}%
\pgfsys@transformshift{1.865799in}{1.980264in}%
\pgfsys@useobject{currentmarker}{}%
\end{pgfscope}%
\begin{pgfscope}%
\pgfsys@transformshift{1.279225in}{2.588906in}%
\pgfsys@useobject{currentmarker}{}%
\end{pgfscope}%
\begin{pgfscope}%
\pgfsys@transformshift{2.908829in}{1.191898in}%
\pgfsys@useobject{currentmarker}{}%
\end{pgfscope}%
\begin{pgfscope}%
\pgfsys@transformshift{1.419679in}{2.355247in}%
\pgfsys@useobject{currentmarker}{}%
\end{pgfscope}%
\begin{pgfscope}%
\pgfsys@transformshift{1.056228in}{2.867376in}%
\pgfsys@useobject{currentmarker}{}%
\end{pgfscope}%
\begin{pgfscope}%
\pgfsys@transformshift{0.763440in}{3.773630in}%
\pgfsys@useobject{currentmarker}{}%
\end{pgfscope}%
\begin{pgfscope}%
\pgfsys@transformshift{4.539430in}{0.774317in}%
\pgfsys@useobject{currentmarker}{}%
\end{pgfscope}%
\begin{pgfscope}%
\pgfsys@transformshift{5.164025in}{0.672611in}%
\pgfsys@useobject{currentmarker}{}%
\end{pgfscope}%
\begin{pgfscope}%
\pgfsys@transformshift{1.411858in}{2.496224in}%
\pgfsys@useobject{currentmarker}{}%
\end{pgfscope}%
\begin{pgfscope}%
\pgfsys@transformshift{3.323354in}{1.022890in}%
\pgfsys@useobject{currentmarker}{}%
\end{pgfscope}%
\begin{pgfscope}%
\pgfsys@transformshift{1.152253in}{2.650657in}%
\pgfsys@useobject{currentmarker}{}%
\end{pgfscope}%
\begin{pgfscope}%
\pgfsys@transformshift{4.493681in}{0.668277in}%
\pgfsys@useobject{currentmarker}{}%
\end{pgfscope}%
\begin{pgfscope}%
\pgfsys@transformshift{4.139112in}{0.785824in}%
\pgfsys@useobject{currentmarker}{}%
\end{pgfscope}%
\begin{pgfscope}%
\pgfsys@transformshift{1.602751in}{2.048253in}%
\pgfsys@useobject{currentmarker}{}%
\end{pgfscope}%
\begin{pgfscope}%
\pgfsys@transformshift{5.482158in}{0.576145in}%
\pgfsys@useobject{currentmarker}{}%
\end{pgfscope}%
\begin{pgfscope}%
\pgfsys@transformshift{3.248174in}{0.965477in}%
\pgfsys@useobject{currentmarker}{}%
\end{pgfscope}%
\begin{pgfscope}%
\pgfsys@transformshift{3.839753in}{0.806067in}%
\pgfsys@useobject{currentmarker}{}%
\end{pgfscope}%
\begin{pgfscope}%
\pgfsys@transformshift{2.429558in}{1.668869in}%
\pgfsys@useobject{currentmarker}{}%
\end{pgfscope}%
\begin{pgfscope}%
\pgfsys@transformshift{3.106341in}{1.308808in}%
\pgfsys@useobject{currentmarker}{}%
\end{pgfscope}%
\begin{pgfscope}%
\pgfsys@transformshift{3.993120in}{0.899747in}%
\pgfsys@useobject{currentmarker}{}%
\end{pgfscope}%
\begin{pgfscope}%
\pgfsys@transformshift{5.758990in}{0.597983in}%
\pgfsys@useobject{currentmarker}{}%
\end{pgfscope}%
\begin{pgfscope}%
\pgfsys@transformshift{4.342434in}{0.678690in}%
\pgfsys@useobject{currentmarker}{}%
\end{pgfscope}%
\begin{pgfscope}%
\pgfsys@transformshift{4.913462in}{0.591345in}%
\pgfsys@useobject{currentmarker}{}%
\end{pgfscope}%
\begin{pgfscope}%
\pgfsys@transformshift{1.456272in}{2.142124in}%
\pgfsys@useobject{currentmarker}{}%
\end{pgfscope}%
\begin{pgfscope}%
\pgfsys@transformshift{0.983666in}{3.334171in}%
\pgfsys@useobject{currentmarker}{}%
\end{pgfscope}%
\begin{pgfscope}%
\pgfsys@transformshift{1.675103in}{2.367521in}%
\pgfsys@useobject{currentmarker}{}%
\end{pgfscope}%
\begin{pgfscope}%
\pgfsys@transformshift{3.462161in}{1.070390in}%
\pgfsys@useobject{currentmarker}{}%
\end{pgfscope}%
\begin{pgfscope}%
\pgfsys@transformshift{4.702284in}{0.728509in}%
\pgfsys@useobject{currentmarker}{}%
\end{pgfscope}%
\begin{pgfscope}%
\pgfsys@transformshift{2.059371in}{1.976888in}%
\pgfsys@useobject{currentmarker}{}%
\end{pgfscope}%
\begin{pgfscope}%
\pgfsys@transformshift{1.630066in}{2.230214in}%
\pgfsys@useobject{currentmarker}{}%
\end{pgfscope}%
\begin{pgfscope}%
\pgfsys@transformshift{3.940720in}{0.753932in}%
\pgfsys@useobject{currentmarker}{}%
\end{pgfscope}%
\begin{pgfscope}%
\pgfsys@transformshift{3.796370in}{0.920318in}%
\pgfsys@useobject{currentmarker}{}%
\end{pgfscope}%
\begin{pgfscope}%
\pgfsys@transformshift{2.873589in}{1.278336in}%
\pgfsys@useobject{currentmarker}{}%
\end{pgfscope}%
\begin{pgfscope}%
\pgfsys@transformshift{5.451502in}{0.558177in}%
\pgfsys@useobject{currentmarker}{}%
\end{pgfscope}%
\begin{pgfscope}%
\pgfsys@transformshift{1.324555in}{2.256327in}%
\pgfsys@useobject{currentmarker}{}%
\end{pgfscope}%
\begin{pgfscope}%
\pgfsys@transformshift{2.411060in}{1.472078in}%
\pgfsys@useobject{currentmarker}{}%
\end{pgfscope}%
\begin{pgfscope}%
\pgfsys@transformshift{3.470396in}{1.039351in}%
\pgfsys@useobject{currentmarker}{}%
\end{pgfscope}%
\begin{pgfscope}%
\pgfsys@transformshift{1.510721in}{2.354600in}%
\pgfsys@useobject{currentmarker}{}%
\end{pgfscope}%
\begin{pgfscope}%
\pgfsys@transformshift{2.721770in}{1.510947in}%
\pgfsys@useobject{currentmarker}{}%
\end{pgfscope}%
\begin{pgfscope}%
\pgfsys@transformshift{1.470327in}{2.482215in}%
\pgfsys@useobject{currentmarker}{}%
\end{pgfscope}%
\begin{pgfscope}%
\pgfsys@transformshift{3.807021in}{0.858638in}%
\pgfsys@useobject{currentmarker}{}%
\end{pgfscope}%
\begin{pgfscope}%
\pgfsys@transformshift{0.816007in}{3.698963in}%
\pgfsys@useobject{currentmarker}{}%
\end{pgfscope}%
\begin{pgfscope}%
\pgfsys@transformshift{1.318192in}{2.419967in}%
\pgfsys@useobject{currentmarker}{}%
\end{pgfscope}%
\begin{pgfscope}%
\pgfsys@transformshift{5.099498in}{0.575336in}%
\pgfsys@useobject{currentmarker}{}%
\end{pgfscope}%
\end{pgfscope}%
\begin{pgfscope}%
\pgfpathrectangle{\pgfqpoint{0.626386in}{0.608332in}}{\pgfqpoint{6.200000in}{4.620000in}}%
\pgfusepath{clip}%
\pgfsetbuttcap%
\pgfsetmiterjoin%
\definecolor{currentfill}{rgb}{0.501961,0.501961,0.501961}%
\pgfsetfillcolor{currentfill}%
\pgfsetfillopacity{0.200000}%
\pgfsetlinewidth{1.003750pt}%
\definecolor{currentstroke}{rgb}{0.501961,0.501961,0.501961}%
\pgfsetstrokecolor{currentstroke}%
\pgfsetstrokeopacity{0.200000}%
\pgfsetdash{}{0pt}%
\pgfpathmoveto{\pgfqpoint{0.626386in}{4.391376in}}%
\pgfpathlineto{\pgfqpoint{0.627161in}{4.307262in}}%
\pgfpathlineto{\pgfqpoint{0.629487in}{4.224002in}}%
\pgfpathlineto{\pgfqpoint{0.633363in}{4.141596in}}%
\pgfpathlineto{\pgfqpoint{0.638790in}{4.060043in}}%
\pgfpathlineto{\pgfqpoint{0.645767in}{3.979345in}}%
\pgfpathlineto{\pgfqpoint{0.654294in}{3.899501in}}%
\pgfpathlineto{\pgfqpoint{0.664372in}{3.820510in}}%
\pgfpathlineto{\pgfqpoint{0.676001in}{3.742374in}}%
\pgfpathlineto{\pgfqpoint{0.689180in}{3.665091in}}%
\pgfpathlineto{\pgfqpoint{0.703909in}{3.588663in}}%
\pgfpathlineto{\pgfqpoint{0.720189in}{3.513088in}}%
\pgfpathlineto{\pgfqpoint{0.738019in}{3.438368in}}%
\pgfpathlineto{\pgfqpoint{0.757400in}{3.364501in}}%
\pgfpathlineto{\pgfqpoint{0.778331in}{3.291488in}}%
\pgfpathlineto{\pgfqpoint{0.800813in}{3.219329in}}%
\pgfpathlineto{\pgfqpoint{0.824845in}{3.148025in}}%
\pgfpathlineto{\pgfqpoint{0.850428in}{3.077574in}}%
\pgfpathlineto{\pgfqpoint{0.877561in}{3.007977in}}%
\pgfpathlineto{\pgfqpoint{0.906244in}{2.939234in}}%
\pgfpathlineto{\pgfqpoint{0.936478in}{2.871345in}}%
\pgfpathlineto{\pgfqpoint{0.968263in}{2.804310in}}%
\pgfpathlineto{\pgfqpoint{1.001598in}{2.738129in}}%
\pgfpathlineto{\pgfqpoint{1.036483in}{2.672801in}}%
\pgfpathlineto{\pgfqpoint{1.072919in}{2.608328in}}%
\pgfpathlineto{\pgfqpoint{1.110905in}{2.544709in}}%
\pgfpathlineto{\pgfqpoint{1.150442in}{2.481943in}}%
\pgfpathlineto{\pgfqpoint{1.191529in}{2.420032in}}%
\pgfpathlineto{\pgfqpoint{1.234167in}{2.358975in}}%
\pgfpathlineto{\pgfqpoint{1.278355in}{2.298771in}}%
\pgfpathlineto{\pgfqpoint{1.324094in}{2.239422in}}%
\pgfpathlineto{\pgfqpoint{1.371383in}{2.180926in}}%
\pgfpathlineto{\pgfqpoint{1.420223in}{2.123284in}}%
\pgfpathlineto{\pgfqpoint{1.470613in}{2.066497in}}%
\pgfpathlineto{\pgfqpoint{1.522553in}{2.010563in}}%
\pgfpathlineto{\pgfqpoint{1.576044in}{1.955483in}}%
\pgfpathlineto{\pgfqpoint{1.631085in}{1.901257in}}%
\pgfpathlineto{\pgfqpoint{1.687677in}{1.847886in}}%
\pgfpathlineto{\pgfqpoint{1.745820in}{1.795368in}}%
\pgfpathlineto{\pgfqpoint{1.805512in}{1.743704in}}%
\pgfpathlineto{\pgfqpoint{1.866756in}{1.692894in}}%
\pgfpathlineto{\pgfqpoint{1.929549in}{1.642937in}}%
\pgfpathlineto{\pgfqpoint{1.993893in}{1.593835in}}%
\pgfpathlineto{\pgfqpoint{2.059788in}{1.545587in}}%
\pgfpathlineto{\pgfqpoint{2.127233in}{1.498193in}}%
\pgfpathlineto{\pgfqpoint{2.196229in}{1.451653in}}%
\pgfpathlineto{\pgfqpoint{2.266775in}{1.405966in}}%
\pgfpathlineto{\pgfqpoint{2.338871in}{1.361134in}}%
\pgfpathlineto{\pgfqpoint{2.412518in}{1.317156in}}%
\pgfpathlineto{\pgfqpoint{2.487716in}{1.274031in}}%
\pgfpathlineto{\pgfqpoint{2.564464in}{1.231760in}}%
\pgfpathlineto{\pgfqpoint{2.642762in}{1.190344in}}%
\pgfpathlineto{\pgfqpoint{2.722611in}{1.149781in}}%
\pgfpathlineto{\pgfqpoint{2.804010in}{1.110073in}}%
\pgfpathlineto{\pgfqpoint{2.886960in}{1.071218in}}%
\pgfpathlineto{\pgfqpoint{2.971460in}{1.033217in}}%
\pgfpathlineto{\pgfqpoint{3.057510in}{0.996070in}}%
\pgfpathlineto{\pgfqpoint{3.145112in}{0.959777in}}%
\pgfpathlineto{\pgfqpoint{3.234263in}{0.924338in}}%
\pgfpathlineto{\pgfqpoint{3.324965in}{0.889753in}}%
\pgfpathlineto{\pgfqpoint{3.389449in}{0.866091in}}%
\pgfpathlineto{\pgfqpoint{3.436350in}{0.849652in}}%
\pgfpathlineto{\pgfqpoint{3.484027in}{0.833640in}}%
\pgfpathlineto{\pgfqpoint{3.532479in}{0.818056in}}%
\pgfpathlineto{\pgfqpoint{3.581706in}{0.802898in}}%
\pgfpathlineto{\pgfqpoint{3.631709in}{0.788168in}}%
\pgfpathlineto{\pgfqpoint{3.682486in}{0.773864in}}%
\pgfpathlineto{\pgfqpoint{3.734039in}{0.759987in}}%
\pgfpathlineto{\pgfqpoint{3.786367in}{0.746537in}}%
\pgfpathlineto{\pgfqpoint{3.839470in}{0.733515in}}%
\pgfpathlineto{\pgfqpoint{3.893349in}{0.720919in}}%
\pgfpathlineto{\pgfqpoint{3.948003in}{0.708750in}}%
\pgfpathlineto{\pgfqpoint{4.003432in}{0.697008in}}%
\pgfpathlineto{\pgfqpoint{4.059636in}{0.685694in}}%
\pgfpathlineto{\pgfqpoint{4.116616in}{0.674806in}}%
\pgfpathlineto{\pgfqpoint{4.174370in}{0.664345in}}%
\pgfpathlineto{\pgfqpoint{4.232900in}{0.654311in}}%
\pgfpathlineto{\pgfqpoint{4.292205in}{0.644704in}}%
\pgfpathlineto{\pgfqpoint{4.352286in}{0.635524in}}%
\pgfpathlineto{\pgfqpoint{4.413141in}{0.626771in}}%
\pgfpathlineto{\pgfqpoint{4.474772in}{0.618445in}}%
\pgfpathlineto{\pgfqpoint{4.537178in}{0.610546in}}%
\pgfpathlineto{\pgfqpoint{4.600360in}{0.603074in}}%
\pgfpathlineto{\pgfqpoint{4.664316in}{0.596029in}}%
\pgfpathlineto{\pgfqpoint{4.729048in}{0.589411in}}%
\pgfpathlineto{\pgfqpoint{4.794555in}{0.583220in}}%
\pgfpathlineto{\pgfqpoint{4.860837in}{0.577455in}}%
\pgfpathlineto{\pgfqpoint{4.927895in}{0.572118in}}%
\pgfpathlineto{\pgfqpoint{4.995728in}{0.567208in}}%
\pgfpathlineto{\pgfqpoint{5.064335in}{0.562725in}}%
\pgfpathlineto{\pgfqpoint{5.133719in}{0.558668in}}%
\pgfpathlineto{\pgfqpoint{5.203877in}{0.555039in}}%
\pgfpathlineto{\pgfqpoint{5.274811in}{0.551837in}}%
\pgfpathlineto{\pgfqpoint{5.346520in}{0.549062in}}%
\pgfpathlineto{\pgfqpoint{5.419004in}{0.546713in}}%
\pgfpathlineto{\pgfqpoint{5.492263in}{0.544792in}}%
\pgfpathlineto{\pgfqpoint{5.566298in}{0.543297in}}%
\pgfpathlineto{\pgfqpoint{5.641107in}{0.542230in}}%
\pgfpathlineto{\pgfqpoint{5.716692in}{0.541589in}}%
\pgfpathlineto{\pgfqpoint{5.793053in}{0.541376in}}%
\pgfpathlineto{\pgfqpoint{6.826386in}{0.608332in}}%
\pgfpathlineto{\pgfqpoint{6.734754in}{0.608589in}}%
\pgfpathlineto{\pgfqpoint{6.644052in}{0.609357in}}%
\pgfpathlineto{\pgfqpoint{6.554280in}{0.610638in}}%
\pgfpathlineto{\pgfqpoint{6.465438in}{0.612431in}}%
\pgfpathlineto{\pgfqpoint{6.377527in}{0.614737in}}%
\pgfpathlineto{\pgfqpoint{6.290546in}{0.617555in}}%
\pgfpathlineto{\pgfqpoint{6.204496in}{0.620886in}}%
\pgfpathlineto{\pgfqpoint{6.119375in}{0.624728in}}%
\pgfpathlineto{\pgfqpoint{6.035185in}{0.629083in}}%
\pgfpathlineto{\pgfqpoint{5.951925in}{0.633951in}}%
\pgfpathlineto{\pgfqpoint{5.869596in}{0.639331in}}%
\pgfpathlineto{\pgfqpoint{5.788197in}{0.645223in}}%
\pgfpathlineto{\pgfqpoint{5.707728in}{0.651628in}}%
\pgfpathlineto{\pgfqpoint{5.628189in}{0.658545in}}%
\pgfpathlineto{\pgfqpoint{5.549580in}{0.665974in}}%
\pgfpathlineto{\pgfqpoint{5.471902in}{0.673916in}}%
\pgfpathlineto{\pgfqpoint{5.395154in}{0.682370in}}%
\pgfpathlineto{\pgfqpoint{5.319337in}{0.691336in}}%
\pgfpathlineto{\pgfqpoint{5.244450in}{0.700815in}}%
\pgfpathlineto{\pgfqpoint{5.170493in}{0.710807in}}%
\pgfpathlineto{\pgfqpoint{5.097466in}{0.721310in}}%
\pgfpathlineto{\pgfqpoint{5.025369in}{0.732326in}}%
\pgfpathlineto{\pgfqpoint{4.954203in}{0.743854in}}%
\pgfpathlineto{\pgfqpoint{4.883967in}{0.755895in}}%
\pgfpathlineto{\pgfqpoint{4.814661in}{0.768448in}}%
\pgfpathlineto{\pgfqpoint{4.746286in}{0.781514in}}%
\pgfpathlineto{\pgfqpoint{4.678841in}{0.795091in}}%
\pgfpathlineto{\pgfqpoint{4.612326in}{0.809182in}}%
\pgfpathlineto{\pgfqpoint{4.546742in}{0.823784in}}%
\pgfpathlineto{\pgfqpoint{4.482087in}{0.838899in}}%
\pgfpathlineto{\pgfqpoint{4.418363in}{0.854526in}}%
\pgfpathlineto{\pgfqpoint{4.355570in}{0.870666in}}%
\pgfpathlineto{\pgfqpoint{4.293706in}{0.887318in}}%
\pgfpathlineto{\pgfqpoint{4.232773in}{0.904482in}}%
\pgfpathlineto{\pgfqpoint{4.172770in}{0.922159in}}%
\pgfpathlineto{\pgfqpoint{4.113698in}{0.940348in}}%
\pgfpathlineto{\pgfqpoint{4.055555in}{0.959050in}}%
\pgfpathlineto{\pgfqpoint{3.998343in}{0.978264in}}%
\pgfpathlineto{\pgfqpoint{3.942061in}{0.997990in}}%
\pgfpathlineto{\pgfqpoint{3.864681in}{1.026385in}}%
\pgfpathlineto{\pgfqpoint{3.755839in}{1.067887in}}%
\pgfpathlineto{\pgfqpoint{3.648857in}{1.110414in}}%
\pgfpathlineto{\pgfqpoint{3.543735in}{1.153966in}}%
\pgfpathlineto{\pgfqpoint{3.440475in}{1.198542in}}%
\pgfpathlineto{\pgfqpoint{3.339074in}{1.244143in}}%
\pgfpathlineto{\pgfqpoint{3.239535in}{1.290768in}}%
\pgfpathlineto{\pgfqpoint{3.141856in}{1.338419in}}%
\pgfpathlineto{\pgfqpoint{3.046037in}{1.387094in}}%
\pgfpathlineto{\pgfqpoint{2.952079in}{1.436794in}}%
\pgfpathlineto{\pgfqpoint{2.859982in}{1.487519in}}%
\pgfpathlineto{\pgfqpoint{2.769745in}{1.539268in}}%
\pgfpathlineto{\pgfqpoint{2.681368in}{1.592042in}}%
\pgfpathlineto{\pgfqpoint{2.594853in}{1.645841in}}%
\pgfpathlineto{\pgfqpoint{2.510197in}{1.700665in}}%
\pgfpathlineto{\pgfqpoint{2.427403in}{1.756513in}}%
\pgfpathlineto{\pgfqpoint{2.346469in}{1.813386in}}%
\pgfpathlineto{\pgfqpoint{2.267395in}{1.871284in}}%
\pgfpathlineto{\pgfqpoint{2.190182in}{1.930206in}}%
\pgfpathlineto{\pgfqpoint{2.114830in}{1.990154in}}%
\pgfpathlineto{\pgfqpoint{2.041338in}{2.051126in}}%
\pgfpathlineto{\pgfqpoint{1.969706in}{2.113122in}}%
\pgfpathlineto{\pgfqpoint{1.899935in}{2.176144in}}%
\pgfpathlineto{\pgfqpoint{1.832025in}{2.240190in}}%
\pgfpathlineto{\pgfqpoint{1.765976in}{2.305261in}}%
\pgfpathlineto{\pgfqpoint{1.701786in}{2.371357in}}%
\pgfpathlineto{\pgfqpoint{1.639458in}{2.438477in}}%
\pgfpathlineto{\pgfqpoint{1.578990in}{2.506623in}}%
\pgfpathlineto{\pgfqpoint{1.520382in}{2.575793in}}%
\pgfpathlineto{\pgfqpoint{1.463635in}{2.645987in}}%
\pgfpathlineto{\pgfqpoint{1.408749in}{2.717207in}}%
\pgfpathlineto{\pgfqpoint{1.355723in}{2.789451in}}%
\pgfpathlineto{\pgfqpoint{1.304558in}{2.862720in}}%
\pgfpathlineto{\pgfqpoint{1.255253in}{2.937014in}}%
\pgfpathlineto{\pgfqpoint{1.207809in}{3.012332in}}%
\pgfpathlineto{\pgfqpoint{1.162226in}{3.088675in}}%
\pgfpathlineto{\pgfqpoint{1.118503in}{3.166043in}}%
\pgfpathlineto{\pgfqpoint{1.076640in}{3.244436in}}%
\pgfpathlineto{\pgfqpoint{1.036638in}{3.323853in}}%
\pgfpathlineto{\pgfqpoint{0.998497in}{3.404295in}}%
\pgfpathlineto{\pgfqpoint{0.962216in}{3.485762in}}%
\pgfpathlineto{\pgfqpoint{0.927796in}{3.568253in}}%
\pgfpathlineto{\pgfqpoint{0.895236in}{3.651770in}}%
\pgfpathlineto{\pgfqpoint{0.864537in}{3.736311in}}%
\pgfpathlineto{\pgfqpoint{0.835698in}{3.821877in}}%
\pgfpathlineto{\pgfqpoint{0.808720in}{3.908467in}}%
\pgfpathlineto{\pgfqpoint{0.783603in}{3.996082in}}%
\pgfpathlineto{\pgfqpoint{0.760346in}{4.084722in}}%
\pgfpathlineto{\pgfqpoint{0.738950in}{4.174387in}}%
\pgfpathlineto{\pgfqpoint{0.719414in}{4.265077in}}%
\pgfpathlineto{\pgfqpoint{0.701738in}{4.356791in}}%
\pgfpathlineto{\pgfqpoint{0.685924in}{4.449530in}}%
\pgfpathlineto{\pgfqpoint{0.671970in}{4.543294in}}%
\pgfpathlineto{\pgfqpoint{0.659876in}{4.638082in}}%
\pgfpathlineto{\pgfqpoint{0.649643in}{4.733895in}}%
\pgfpathlineto{\pgfqpoint{0.641270in}{4.830733in}}%
\pgfpathlineto{\pgfqpoint{0.634758in}{4.928596in}}%
\pgfpathlineto{\pgfqpoint{0.630107in}{5.027483in}}%
\pgfpathlineto{\pgfqpoint{0.627316in}{5.127396in}}%
\pgfpathlineto{\pgfqpoint{0.626386in}{5.228333in}}%
\pgfpathlineto{\pgfqpoint{0.626386in}{4.391376in}}%
\pgfpathclose%
\pgfusepath{stroke,fill}%
\end{pgfscope}%
\begin{pgfscope}%
\pgfpathrectangle{\pgfqpoint{0.626386in}{0.608332in}}{\pgfqpoint{6.200000in}{4.620000in}}%
\pgfusepath{clip}%
\pgfsetbuttcap%
\pgfsetroundjoin%
\definecolor{currentfill}{rgb}{1.000000,1.000000,1.000000}%
\pgfsetfillcolor{currentfill}%
\pgfsetlinewidth{1.003750pt}%
\definecolor{currentstroke}{rgb}{1.000000,1.000000,1.000000}%
\pgfsetstrokecolor{currentstroke}%
\pgfsetdash{}{0pt}%
\pgfsys@defobject{currentmarker}{\pgfqpoint{0.626386in}{0.206593in}}{\pgfqpoint{5.689719in}{4.307680in}}{%
\pgfpathmoveto{\pgfqpoint{0.626386in}{0.206593in}}%
\pgfpathlineto{\pgfqpoint{0.626386in}{4.307680in}}%
\pgfpathlineto{\pgfqpoint{0.627146in}{4.225249in}}%
\pgfpathlineto{\pgfqpoint{0.629425in}{4.143654in}}%
\pgfpathlineto{\pgfqpoint{0.633224in}{4.062896in}}%
\pgfpathlineto{\pgfqpoint{0.638542in}{3.982974in}}%
\pgfpathlineto{\pgfqpoint{0.645379in}{3.903890in}}%
\pgfpathlineto{\pgfqpoint{0.653736in}{3.825643in}}%
\pgfpathlineto{\pgfqpoint{0.663613in}{3.748232in}}%
\pgfpathlineto{\pgfqpoint{0.675008in}{3.671658in}}%
\pgfpathlineto{\pgfqpoint{0.687924in}{3.595921in}}%
\pgfpathlineto{\pgfqpoint{0.702359in}{3.521021in}}%
\pgfpathlineto{\pgfqpoint{0.718313in}{3.446958in}}%
\pgfpathlineto{\pgfqpoint{0.735787in}{3.373732in}}%
\pgfpathlineto{\pgfqpoint{0.754780in}{3.301343in}}%
\pgfpathlineto{\pgfqpoint{0.775292in}{3.229790in}}%
\pgfpathlineto{\pgfqpoint{0.797324in}{3.159075in}}%
\pgfpathlineto{\pgfqpoint{0.820876in}{3.089196in}}%
\pgfpathlineto{\pgfqpoint{0.845947in}{3.020154in}}%
\pgfpathlineto{\pgfqpoint{0.872537in}{2.951949in}}%
\pgfpathlineto{\pgfqpoint{0.900647in}{2.884581in}}%
\pgfpathlineto{\pgfqpoint{0.930277in}{2.818050in}}%
\pgfpathlineto{\pgfqpoint{0.961425in}{2.752355in}}%
\pgfpathlineto{\pgfqpoint{0.994094in}{2.687498in}}%
\pgfpathlineto{\pgfqpoint{1.028281in}{2.623477in}}%
\pgfpathlineto{\pgfqpoint{1.063988in}{2.560293in}}%
\pgfpathlineto{\pgfqpoint{1.101215in}{2.497947in}}%
\pgfpathlineto{\pgfqpoint{1.139961in}{2.436436in}}%
\pgfpathlineto{\pgfqpoint{1.180227in}{2.375763in}}%
\pgfpathlineto{\pgfqpoint{1.222011in}{2.315927in}}%
\pgfpathlineto{\pgfqpoint{1.265316in}{2.256928in}}%
\pgfpathlineto{\pgfqpoint{1.310140in}{2.198765in}}%
\pgfpathlineto{\pgfqpoint{1.356483in}{2.141439in}}%
\pgfpathlineto{\pgfqpoint{1.404346in}{2.084951in}}%
\pgfpathlineto{\pgfqpoint{1.453728in}{2.029299in}}%
\pgfpathlineto{\pgfqpoint{1.504630in}{1.974484in}}%
\pgfpathlineto{\pgfqpoint{1.557051in}{1.920505in}}%
\pgfpathlineto{\pgfqpoint{1.610991in}{1.867364in}}%
\pgfpathlineto{\pgfqpoint{1.666451in}{1.815060in}}%
\pgfpathlineto{\pgfqpoint{1.723431in}{1.763592in}}%
\pgfpathlineto{\pgfqpoint{1.781930in}{1.712961in}}%
\pgfpathlineto{\pgfqpoint{1.841948in}{1.663168in}}%
\pgfpathlineto{\pgfqpoint{1.903486in}{1.614211in}}%
\pgfpathlineto{\pgfqpoint{1.966543in}{1.566091in}}%
\pgfpathlineto{\pgfqpoint{2.031120in}{1.518807in}}%
\pgfpathlineto{\pgfqpoint{2.097216in}{1.472361in}}%
\pgfpathlineto{\pgfqpoint{2.164832in}{1.426751in}}%
\pgfpathlineto{\pgfqpoint{2.233967in}{1.381979in}}%
\pgfpathlineto{\pgfqpoint{2.304622in}{1.338043in}}%
\pgfpathlineto{\pgfqpoint{2.376796in}{1.294944in}}%
\pgfpathlineto{\pgfqpoint{2.450489in}{1.252682in}}%
\pgfpathlineto{\pgfqpoint{2.525702in}{1.211257in}}%
\pgfpathlineto{\pgfqpoint{2.602434in}{1.170669in}}%
\pgfpathlineto{\pgfqpoint{2.680686in}{1.130918in}}%
\pgfpathlineto{\pgfqpoint{2.760457in}{1.092003in}}%
\pgfpathlineto{\pgfqpoint{2.841748in}{1.053925in}}%
\pgfpathlineto{\pgfqpoint{2.924558in}{1.016685in}}%
\pgfpathlineto{\pgfqpoint{3.008888in}{0.980281in}}%
\pgfpathlineto{\pgfqpoint{3.094737in}{0.944714in}}%
\pgfpathlineto{\pgfqpoint{3.182106in}{0.909984in}}%
\pgfpathlineto{\pgfqpoint{3.270994in}{0.876090in}}%
\pgfpathlineto{\pgfqpoint{3.334188in}{0.852901in}}%
\pgfpathlineto{\pgfqpoint{3.380151in}{0.836791in}}%
\pgfpathlineto{\pgfqpoint{3.426874in}{0.821099in}}%
\pgfpathlineto{\pgfqpoint{3.474357in}{0.805827in}}%
\pgfpathlineto{\pgfqpoint{3.522600in}{0.790972in}}%
\pgfpathlineto{\pgfqpoint{3.571602in}{0.776536in}}%
\pgfpathlineto{\pgfqpoint{3.621364in}{0.762518in}}%
\pgfpathlineto{\pgfqpoint{3.671886in}{0.748919in}}%
\pgfpathlineto{\pgfqpoint{3.723168in}{0.735739in}}%
\pgfpathlineto{\pgfqpoint{3.775209in}{0.722976in}}%
\pgfpathlineto{\pgfqpoint{3.828010in}{0.710632in}}%
\pgfpathlineto{\pgfqpoint{3.881570in}{0.698707in}}%
\pgfpathlineto{\pgfqpoint{3.935891in}{0.687200in}}%
\pgfpathlineto{\pgfqpoint{3.990971in}{0.676112in}}%
\pgfpathlineto{\pgfqpoint{4.046811in}{0.665441in}}%
\pgfpathlineto{\pgfqpoint{4.103411in}{0.655190in}}%
\pgfpathlineto{\pgfqpoint{4.160770in}{0.645357in}}%
\pgfpathlineto{\pgfqpoint{4.218889in}{0.635942in}}%
\pgfpathlineto{\pgfqpoint{4.277768in}{0.626945in}}%
\pgfpathlineto{\pgfqpoint{4.337406in}{0.618367in}}%
\pgfpathlineto{\pgfqpoint{4.397805in}{0.610208in}}%
\pgfpathlineto{\pgfqpoint{4.458963in}{0.602467in}}%
\pgfpathlineto{\pgfqpoint{4.520880in}{0.595144in}}%
\pgfpathlineto{\pgfqpoint{4.583558in}{0.588240in}}%
\pgfpathlineto{\pgfqpoint{4.646995in}{0.581754in}}%
\pgfpathlineto{\pgfqpoint{4.711192in}{0.575687in}}%
\pgfpathlineto{\pgfqpoint{4.776148in}{0.570038in}}%
\pgfpathlineto{\pgfqpoint{4.841865in}{0.564808in}}%
\pgfpathlineto{\pgfqpoint{4.908341in}{0.559996in}}%
\pgfpathlineto{\pgfqpoint{4.975576in}{0.555602in}}%
\pgfpathlineto{\pgfqpoint{5.043572in}{0.551627in}}%
\pgfpathlineto{\pgfqpoint{5.112327in}{0.548070in}}%
\pgfpathlineto{\pgfqpoint{5.181842in}{0.544932in}}%
\pgfpathlineto{\pgfqpoint{5.252117in}{0.542212in}}%
\pgfpathlineto{\pgfqpoint{5.323151in}{0.539911in}}%
\pgfpathlineto{\pgfqpoint{5.394945in}{0.538028in}}%
\pgfpathlineto{\pgfqpoint{5.467499in}{0.536563in}}%
\pgfpathlineto{\pgfqpoint{5.540813in}{0.535517in}}%
\pgfpathlineto{\pgfqpoint{5.614886in}{0.534890in}}%
\pgfpathlineto{\pgfqpoint{5.689719in}{0.534680in}}%
\pgfpathlineto{\pgfqpoint{5.689719in}{0.206593in}}%
\pgfpathlineto{\pgfqpoint{5.689719in}{0.206593in}}%
\pgfpathlineto{\pgfqpoint{5.614886in}{0.206593in}}%
\pgfpathlineto{\pgfqpoint{5.540813in}{0.206593in}}%
\pgfpathlineto{\pgfqpoint{5.467499in}{0.206593in}}%
\pgfpathlineto{\pgfqpoint{5.394945in}{0.206593in}}%
\pgfpathlineto{\pgfqpoint{5.323151in}{0.206593in}}%
\pgfpathlineto{\pgfqpoint{5.252117in}{0.206593in}}%
\pgfpathlineto{\pgfqpoint{5.181842in}{0.206593in}}%
\pgfpathlineto{\pgfqpoint{5.112327in}{0.206593in}}%
\pgfpathlineto{\pgfqpoint{5.043572in}{0.206593in}}%
\pgfpathlineto{\pgfqpoint{4.975576in}{0.206593in}}%
\pgfpathlineto{\pgfqpoint{4.908341in}{0.206593in}}%
\pgfpathlineto{\pgfqpoint{4.841865in}{0.206593in}}%
\pgfpathlineto{\pgfqpoint{4.776148in}{0.206593in}}%
\pgfpathlineto{\pgfqpoint{4.711192in}{0.206593in}}%
\pgfpathlineto{\pgfqpoint{4.646995in}{0.206593in}}%
\pgfpathlineto{\pgfqpoint{4.583558in}{0.206593in}}%
\pgfpathlineto{\pgfqpoint{4.520880in}{0.206593in}}%
\pgfpathlineto{\pgfqpoint{4.458963in}{0.206593in}}%
\pgfpathlineto{\pgfqpoint{4.397805in}{0.206593in}}%
\pgfpathlineto{\pgfqpoint{4.337406in}{0.206593in}}%
\pgfpathlineto{\pgfqpoint{4.277768in}{0.206593in}}%
\pgfpathlineto{\pgfqpoint{4.218889in}{0.206593in}}%
\pgfpathlineto{\pgfqpoint{4.160770in}{0.206593in}}%
\pgfpathlineto{\pgfqpoint{4.103411in}{0.206593in}}%
\pgfpathlineto{\pgfqpoint{4.046811in}{0.206593in}}%
\pgfpathlineto{\pgfqpoint{3.990971in}{0.206593in}}%
\pgfpathlineto{\pgfqpoint{3.935891in}{0.206593in}}%
\pgfpathlineto{\pgfqpoint{3.881570in}{0.206593in}}%
\pgfpathlineto{\pgfqpoint{3.828010in}{0.206593in}}%
\pgfpathlineto{\pgfqpoint{3.775209in}{0.206593in}}%
\pgfpathlineto{\pgfqpoint{3.723168in}{0.206593in}}%
\pgfpathlineto{\pgfqpoint{3.671886in}{0.206593in}}%
\pgfpathlineto{\pgfqpoint{3.621364in}{0.206593in}}%
\pgfpathlineto{\pgfqpoint{3.571602in}{0.206593in}}%
\pgfpathlineto{\pgfqpoint{3.522600in}{0.206593in}}%
\pgfpathlineto{\pgfqpoint{3.474357in}{0.206593in}}%
\pgfpathlineto{\pgfqpoint{3.426874in}{0.206593in}}%
\pgfpathlineto{\pgfqpoint{3.380151in}{0.206593in}}%
\pgfpathlineto{\pgfqpoint{3.334188in}{0.206593in}}%
\pgfpathlineto{\pgfqpoint{3.270994in}{0.206593in}}%
\pgfpathlineto{\pgfqpoint{3.182106in}{0.206593in}}%
\pgfpathlineto{\pgfqpoint{3.094737in}{0.206593in}}%
\pgfpathlineto{\pgfqpoint{3.008888in}{0.206593in}}%
\pgfpathlineto{\pgfqpoint{2.924558in}{0.206593in}}%
\pgfpathlineto{\pgfqpoint{2.841748in}{0.206593in}}%
\pgfpathlineto{\pgfqpoint{2.760457in}{0.206593in}}%
\pgfpathlineto{\pgfqpoint{2.680686in}{0.206593in}}%
\pgfpathlineto{\pgfqpoint{2.602434in}{0.206593in}}%
\pgfpathlineto{\pgfqpoint{2.525702in}{0.206593in}}%
\pgfpathlineto{\pgfqpoint{2.450489in}{0.206593in}}%
\pgfpathlineto{\pgfqpoint{2.376796in}{0.206593in}}%
\pgfpathlineto{\pgfqpoint{2.304622in}{0.206593in}}%
\pgfpathlineto{\pgfqpoint{2.233967in}{0.206593in}}%
\pgfpathlineto{\pgfqpoint{2.164832in}{0.206593in}}%
\pgfpathlineto{\pgfqpoint{2.097216in}{0.206593in}}%
\pgfpathlineto{\pgfqpoint{2.031120in}{0.206593in}}%
\pgfpathlineto{\pgfqpoint{1.966543in}{0.206593in}}%
\pgfpathlineto{\pgfqpoint{1.903486in}{0.206593in}}%
\pgfpathlineto{\pgfqpoint{1.841948in}{0.206593in}}%
\pgfpathlineto{\pgfqpoint{1.781930in}{0.206593in}}%
\pgfpathlineto{\pgfqpoint{1.723431in}{0.206593in}}%
\pgfpathlineto{\pgfqpoint{1.666451in}{0.206593in}}%
\pgfpathlineto{\pgfqpoint{1.610991in}{0.206593in}}%
\pgfpathlineto{\pgfqpoint{1.557051in}{0.206593in}}%
\pgfpathlineto{\pgfqpoint{1.504630in}{0.206593in}}%
\pgfpathlineto{\pgfqpoint{1.453728in}{0.206593in}}%
\pgfpathlineto{\pgfqpoint{1.404346in}{0.206593in}}%
\pgfpathlineto{\pgfqpoint{1.356483in}{0.206593in}}%
\pgfpathlineto{\pgfqpoint{1.310140in}{0.206593in}}%
\pgfpathlineto{\pgfqpoint{1.265316in}{0.206593in}}%
\pgfpathlineto{\pgfqpoint{1.222011in}{0.206593in}}%
\pgfpathlineto{\pgfqpoint{1.180227in}{0.206593in}}%
\pgfpathlineto{\pgfqpoint{1.139961in}{0.206593in}}%
\pgfpathlineto{\pgfqpoint{1.101215in}{0.206593in}}%
\pgfpathlineto{\pgfqpoint{1.063988in}{0.206593in}}%
\pgfpathlineto{\pgfqpoint{1.028281in}{0.206593in}}%
\pgfpathlineto{\pgfqpoint{0.994094in}{0.206593in}}%
\pgfpathlineto{\pgfqpoint{0.961425in}{0.206593in}}%
\pgfpathlineto{\pgfqpoint{0.930277in}{0.206593in}}%
\pgfpathlineto{\pgfqpoint{0.900647in}{0.206593in}}%
\pgfpathlineto{\pgfqpoint{0.872537in}{0.206593in}}%
\pgfpathlineto{\pgfqpoint{0.845947in}{0.206593in}}%
\pgfpathlineto{\pgfqpoint{0.820876in}{0.206593in}}%
\pgfpathlineto{\pgfqpoint{0.797324in}{0.206593in}}%
\pgfpathlineto{\pgfqpoint{0.775292in}{0.206593in}}%
\pgfpathlineto{\pgfqpoint{0.754780in}{0.206593in}}%
\pgfpathlineto{\pgfqpoint{0.735787in}{0.206593in}}%
\pgfpathlineto{\pgfqpoint{0.718313in}{0.206593in}}%
\pgfpathlineto{\pgfqpoint{0.702359in}{0.206593in}}%
\pgfpathlineto{\pgfqpoint{0.687924in}{0.206593in}}%
\pgfpathlineto{\pgfqpoint{0.675008in}{0.206593in}}%
\pgfpathlineto{\pgfqpoint{0.663613in}{0.206593in}}%
\pgfpathlineto{\pgfqpoint{0.653736in}{0.206593in}}%
\pgfpathlineto{\pgfqpoint{0.645379in}{0.206593in}}%
\pgfpathlineto{\pgfqpoint{0.638542in}{0.206593in}}%
\pgfpathlineto{\pgfqpoint{0.633224in}{0.206593in}}%
\pgfpathlineto{\pgfqpoint{0.629425in}{0.206593in}}%
\pgfpathlineto{\pgfqpoint{0.627146in}{0.206593in}}%
\pgfpathlineto{\pgfqpoint{0.626386in}{0.206593in}}%
\pgfpathlineto{\pgfqpoint{0.626386in}{0.206593in}}%
\pgfpathclose%
\pgfusepath{stroke,fill}%
}%
\begin{pgfscope}%
\pgfsys@transformshift{0.000000in}{0.000000in}%
\pgfsys@useobject{currentmarker}{}%
\end{pgfscope}%
\end{pgfscope}%
\begin{pgfscope}%
\pgfsetbuttcap%
\pgfsetroundjoin%
\definecolor{currentfill}{rgb}{0.000000,0.000000,0.000000}%
\pgfsetfillcolor{currentfill}%
\pgfsetlinewidth{0.803000pt}%
\definecolor{currentstroke}{rgb}{0.000000,0.000000,0.000000}%
\pgfsetstrokecolor{currentstroke}%
\pgfsetdash{}{0pt}%
\pgfsys@defobject{currentmarker}{\pgfqpoint{0.000000in}{-0.048611in}}{\pgfqpoint{0.000000in}{0.000000in}}{%
\pgfpathmoveto{\pgfqpoint{0.000000in}{0.000000in}}%
\pgfpathlineto{\pgfqpoint{0.000000in}{-0.048611in}}%
\pgfusepath{stroke,fill}%
}%
\begin{pgfscope}%
\pgfsys@transformshift{0.626386in}{0.608332in}%
\pgfsys@useobject{currentmarker}{}%
\end{pgfscope}%
\end{pgfscope}%
\begin{pgfscope}%
\definecolor{textcolor}{rgb}{0.000000,0.000000,0.000000}%
\pgfsetstrokecolor{textcolor}%
\pgfsetfillcolor{textcolor}%
\pgftext[x=0.626386in,y=0.511110in,,top]{\color{textcolor}{\rmfamily\fontsize{14.000000}{16.800000}\selectfont\catcode`\^=\active\def^{\ifmmode\sp\else\^{}\fi}\catcode`\%=\active\def%{\%}$\mathdefault{0}$}}%
\end{pgfscope}%
\begin{pgfscope}%
\pgfsetbuttcap%
\pgfsetroundjoin%
\definecolor{currentfill}{rgb}{0.000000,0.000000,0.000000}%
\pgfsetfillcolor{currentfill}%
\pgfsetlinewidth{0.803000pt}%
\definecolor{currentstroke}{rgb}{0.000000,0.000000,0.000000}%
\pgfsetstrokecolor{currentstroke}%
\pgfsetdash{}{0pt}%
\pgfsys@defobject{currentmarker}{\pgfqpoint{0.000000in}{-0.048611in}}{\pgfqpoint{0.000000in}{0.000000in}}{%
\pgfpathmoveto{\pgfqpoint{0.000000in}{0.000000in}}%
\pgfpathlineto{\pgfqpoint{0.000000in}{-0.048611in}}%
\pgfusepath{stroke,fill}%
}%
\begin{pgfscope}%
\pgfsys@transformshift{1.386190in}{0.608332in}%
\pgfsys@useobject{currentmarker}{}%
\end{pgfscope}%
\end{pgfscope}%
\begin{pgfscope}%
\definecolor{textcolor}{rgb}{0.000000,0.000000,0.000000}%
\pgfsetstrokecolor{textcolor}%
\pgfsetfillcolor{textcolor}%
\pgftext[x=1.386190in,y=0.511110in,,top]{\color{textcolor}{\rmfamily\fontsize{14.000000}{16.800000}\selectfont\catcode`\^=\active\def^{\ifmmode\sp\else\^{}\fi}\catcode`\%=\active\def%{\%}$\mathdefault{20}$}}%
\end{pgfscope}%
\begin{pgfscope}%
\pgfsetbuttcap%
\pgfsetroundjoin%
\definecolor{currentfill}{rgb}{0.000000,0.000000,0.000000}%
\pgfsetfillcolor{currentfill}%
\pgfsetlinewidth{0.803000pt}%
\definecolor{currentstroke}{rgb}{0.000000,0.000000,0.000000}%
\pgfsetstrokecolor{currentstroke}%
\pgfsetdash{}{0pt}%
\pgfsys@defobject{currentmarker}{\pgfqpoint{0.000000in}{-0.048611in}}{\pgfqpoint{0.000000in}{0.000000in}}{%
\pgfpathmoveto{\pgfqpoint{0.000000in}{0.000000in}}%
\pgfpathlineto{\pgfqpoint{0.000000in}{-0.048611in}}%
\pgfusepath{stroke,fill}%
}%
\begin{pgfscope}%
\pgfsys@transformshift{2.145994in}{0.608332in}%
\pgfsys@useobject{currentmarker}{}%
\end{pgfscope}%
\end{pgfscope}%
\begin{pgfscope}%
\definecolor{textcolor}{rgb}{0.000000,0.000000,0.000000}%
\pgfsetstrokecolor{textcolor}%
\pgfsetfillcolor{textcolor}%
\pgftext[x=2.145994in,y=0.511110in,,top]{\color{textcolor}{\rmfamily\fontsize{14.000000}{16.800000}\selectfont\catcode`\^=\active\def^{\ifmmode\sp\else\^{}\fi}\catcode`\%=\active\def%{\%}$\mathdefault{40}$}}%
\end{pgfscope}%
\begin{pgfscope}%
\pgfsetbuttcap%
\pgfsetroundjoin%
\definecolor{currentfill}{rgb}{0.000000,0.000000,0.000000}%
\pgfsetfillcolor{currentfill}%
\pgfsetlinewidth{0.803000pt}%
\definecolor{currentstroke}{rgb}{0.000000,0.000000,0.000000}%
\pgfsetstrokecolor{currentstroke}%
\pgfsetdash{}{0pt}%
\pgfsys@defobject{currentmarker}{\pgfqpoint{0.000000in}{-0.048611in}}{\pgfqpoint{0.000000in}{0.000000in}}{%
\pgfpathmoveto{\pgfqpoint{0.000000in}{0.000000in}}%
\pgfpathlineto{\pgfqpoint{0.000000in}{-0.048611in}}%
\pgfusepath{stroke,fill}%
}%
\begin{pgfscope}%
\pgfsys@transformshift{2.905798in}{0.608332in}%
\pgfsys@useobject{currentmarker}{}%
\end{pgfscope}%
\end{pgfscope}%
\begin{pgfscope}%
\definecolor{textcolor}{rgb}{0.000000,0.000000,0.000000}%
\pgfsetstrokecolor{textcolor}%
\pgfsetfillcolor{textcolor}%
\pgftext[x=2.905798in,y=0.511110in,,top]{\color{textcolor}{\rmfamily\fontsize{14.000000}{16.800000}\selectfont\catcode`\^=\active\def^{\ifmmode\sp\else\^{}\fi}\catcode`\%=\active\def%{\%}$\mathdefault{60}$}}%
\end{pgfscope}%
\begin{pgfscope}%
\pgfsetbuttcap%
\pgfsetroundjoin%
\definecolor{currentfill}{rgb}{0.000000,0.000000,0.000000}%
\pgfsetfillcolor{currentfill}%
\pgfsetlinewidth{0.803000pt}%
\definecolor{currentstroke}{rgb}{0.000000,0.000000,0.000000}%
\pgfsetstrokecolor{currentstroke}%
\pgfsetdash{}{0pt}%
\pgfsys@defobject{currentmarker}{\pgfqpoint{0.000000in}{-0.048611in}}{\pgfqpoint{0.000000in}{0.000000in}}{%
\pgfpathmoveto{\pgfqpoint{0.000000in}{0.000000in}}%
\pgfpathlineto{\pgfqpoint{0.000000in}{-0.048611in}}%
\pgfusepath{stroke,fill}%
}%
\begin{pgfscope}%
\pgfsys@transformshift{3.665602in}{0.608332in}%
\pgfsys@useobject{currentmarker}{}%
\end{pgfscope}%
\end{pgfscope}%
\begin{pgfscope}%
\definecolor{textcolor}{rgb}{0.000000,0.000000,0.000000}%
\pgfsetstrokecolor{textcolor}%
\pgfsetfillcolor{textcolor}%
\pgftext[x=3.665602in,y=0.511110in,,top]{\color{textcolor}{\rmfamily\fontsize{14.000000}{16.800000}\selectfont\catcode`\^=\active\def^{\ifmmode\sp\else\^{}\fi}\catcode`\%=\active\def%{\%}$\mathdefault{80}$}}%
\end{pgfscope}%
\begin{pgfscope}%
\pgfsetbuttcap%
\pgfsetroundjoin%
\definecolor{currentfill}{rgb}{0.000000,0.000000,0.000000}%
\pgfsetfillcolor{currentfill}%
\pgfsetlinewidth{0.803000pt}%
\definecolor{currentstroke}{rgb}{0.000000,0.000000,0.000000}%
\pgfsetstrokecolor{currentstroke}%
\pgfsetdash{}{0pt}%
\pgfsys@defobject{currentmarker}{\pgfqpoint{0.000000in}{-0.048611in}}{\pgfqpoint{0.000000in}{0.000000in}}{%
\pgfpathmoveto{\pgfqpoint{0.000000in}{0.000000in}}%
\pgfpathlineto{\pgfqpoint{0.000000in}{-0.048611in}}%
\pgfusepath{stroke,fill}%
}%
\begin{pgfscope}%
\pgfsys@transformshift{4.425406in}{0.608332in}%
\pgfsys@useobject{currentmarker}{}%
\end{pgfscope}%
\end{pgfscope}%
\begin{pgfscope}%
\definecolor{textcolor}{rgb}{0.000000,0.000000,0.000000}%
\pgfsetstrokecolor{textcolor}%
\pgfsetfillcolor{textcolor}%
\pgftext[x=4.425406in,y=0.511110in,,top]{\color{textcolor}{\rmfamily\fontsize{14.000000}{16.800000}\selectfont\catcode`\^=\active\def^{\ifmmode\sp\else\^{}\fi}\catcode`\%=\active\def%{\%}$\mathdefault{100}$}}%
\end{pgfscope}%
\begin{pgfscope}%
\pgfsetbuttcap%
\pgfsetroundjoin%
\definecolor{currentfill}{rgb}{0.000000,0.000000,0.000000}%
\pgfsetfillcolor{currentfill}%
\pgfsetlinewidth{0.803000pt}%
\definecolor{currentstroke}{rgb}{0.000000,0.000000,0.000000}%
\pgfsetstrokecolor{currentstroke}%
\pgfsetdash{}{0pt}%
\pgfsys@defobject{currentmarker}{\pgfqpoint{0.000000in}{-0.048611in}}{\pgfqpoint{0.000000in}{0.000000in}}{%
\pgfpathmoveto{\pgfqpoint{0.000000in}{0.000000in}}%
\pgfpathlineto{\pgfqpoint{0.000000in}{-0.048611in}}%
\pgfusepath{stroke,fill}%
}%
\begin{pgfscope}%
\pgfsys@transformshift{5.185210in}{0.608332in}%
\pgfsys@useobject{currentmarker}{}%
\end{pgfscope}%
\end{pgfscope}%
\begin{pgfscope}%
\definecolor{textcolor}{rgb}{0.000000,0.000000,0.000000}%
\pgfsetstrokecolor{textcolor}%
\pgfsetfillcolor{textcolor}%
\pgftext[x=5.185210in,y=0.511110in,,top]{\color{textcolor}{\rmfamily\fontsize{14.000000}{16.800000}\selectfont\catcode`\^=\active\def^{\ifmmode\sp\else\^{}\fi}\catcode`\%=\active\def%{\%}$\mathdefault{120}$}}%
\end{pgfscope}%
\begin{pgfscope}%
\pgfsetbuttcap%
\pgfsetroundjoin%
\definecolor{currentfill}{rgb}{0.000000,0.000000,0.000000}%
\pgfsetfillcolor{currentfill}%
\pgfsetlinewidth{0.803000pt}%
\definecolor{currentstroke}{rgb}{0.000000,0.000000,0.000000}%
\pgfsetstrokecolor{currentstroke}%
\pgfsetdash{}{0pt}%
\pgfsys@defobject{currentmarker}{\pgfqpoint{0.000000in}{-0.048611in}}{\pgfqpoint{0.000000in}{0.000000in}}{%
\pgfpathmoveto{\pgfqpoint{0.000000in}{0.000000in}}%
\pgfpathlineto{\pgfqpoint{0.000000in}{-0.048611in}}%
\pgfusepath{stroke,fill}%
}%
\begin{pgfscope}%
\pgfsys@transformshift{5.945013in}{0.608332in}%
\pgfsys@useobject{currentmarker}{}%
\end{pgfscope}%
\end{pgfscope}%
\begin{pgfscope}%
\definecolor{textcolor}{rgb}{0.000000,0.000000,0.000000}%
\pgfsetstrokecolor{textcolor}%
\pgfsetfillcolor{textcolor}%
\pgftext[x=5.945013in,y=0.511110in,,top]{\color{textcolor}{\rmfamily\fontsize{14.000000}{16.800000}\selectfont\catcode`\^=\active\def^{\ifmmode\sp\else\^{}\fi}\catcode`\%=\active\def%{\%}$\mathdefault{140}$}}%
\end{pgfscope}%
\begin{pgfscope}%
\pgfsetbuttcap%
\pgfsetroundjoin%
\definecolor{currentfill}{rgb}{0.000000,0.000000,0.000000}%
\pgfsetfillcolor{currentfill}%
\pgfsetlinewidth{0.803000pt}%
\definecolor{currentstroke}{rgb}{0.000000,0.000000,0.000000}%
\pgfsetstrokecolor{currentstroke}%
\pgfsetdash{}{0pt}%
\pgfsys@defobject{currentmarker}{\pgfqpoint{0.000000in}{-0.048611in}}{\pgfqpoint{0.000000in}{0.000000in}}{%
\pgfpathmoveto{\pgfqpoint{0.000000in}{0.000000in}}%
\pgfpathlineto{\pgfqpoint{0.000000in}{-0.048611in}}%
\pgfusepath{stroke,fill}%
}%
\begin{pgfscope}%
\pgfsys@transformshift{6.704817in}{0.608332in}%
\pgfsys@useobject{currentmarker}{}%
\end{pgfscope}%
\end{pgfscope}%
\begin{pgfscope}%
\definecolor{textcolor}{rgb}{0.000000,0.000000,0.000000}%
\pgfsetstrokecolor{textcolor}%
\pgfsetfillcolor{textcolor}%
\pgftext[x=6.704817in,y=0.511110in,,top]{\color{textcolor}{\rmfamily\fontsize{14.000000}{16.800000}\selectfont\catcode`\^=\active\def^{\ifmmode\sp\else\^{}\fi}\catcode`\%=\active\def%{\%}$\mathdefault{160}$}}%
\end{pgfscope}%
\begin{pgfscope}%
\definecolor{textcolor}{rgb}{0.000000,0.000000,0.000000}%
\pgfsetstrokecolor{textcolor}%
\pgfsetfillcolor{textcolor}%
\pgftext[x=3.726386in,y=0.277777in,,top]{\color{textcolor}{\rmfamily\fontsize{14.000000}{16.800000}\selectfont\catcode`\^=\active\def^{\ifmmode\sp\else\^{}\fi}\catcode`\%=\active\def%{\%}f1}}%
\end{pgfscope}%
\begin{pgfscope}%
\pgfsetbuttcap%
\pgfsetroundjoin%
\definecolor{currentfill}{rgb}{0.000000,0.000000,0.000000}%
\pgfsetfillcolor{currentfill}%
\pgfsetlinewidth{0.803000pt}%
\definecolor{currentstroke}{rgb}{0.000000,0.000000,0.000000}%
\pgfsetstrokecolor{currentstroke}%
\pgfsetdash{}{0pt}%
\pgfsys@defobject{currentmarker}{\pgfqpoint{-0.048611in}{0.000000in}}{\pgfqpoint{-0.000000in}{0.000000in}}{%
\pgfpathmoveto{\pgfqpoint{-0.000000in}{0.000000in}}%
\pgfpathlineto{\pgfqpoint{-0.048611in}{0.000000in}}%
\pgfusepath{stroke,fill}%
}%
\begin{pgfscope}%
\pgfsys@transformshift{0.626386in}{1.043550in}%
\pgfsys@useobject{currentmarker}{}%
\end{pgfscope}%
\end{pgfscope}%
\begin{pgfscope}%
\definecolor{textcolor}{rgb}{0.000000,0.000000,0.000000}%
\pgfsetstrokecolor{textcolor}%
\pgfsetfillcolor{textcolor}%
\pgftext[x=0.333333in, y=0.974106in, left, base]{\color{textcolor}{\rmfamily\fontsize{14.000000}{16.800000}\selectfont\catcode`\^=\active\def^{\ifmmode\sp\else\^{}\fi}\catcode`\%=\active\def%{\%}$\mathdefault{10}$}}%
\end{pgfscope}%
\begin{pgfscope}%
\pgfsetbuttcap%
\pgfsetroundjoin%
\definecolor{currentfill}{rgb}{0.000000,0.000000,0.000000}%
\pgfsetfillcolor{currentfill}%
\pgfsetlinewidth{0.803000pt}%
\definecolor{currentstroke}{rgb}{0.000000,0.000000,0.000000}%
\pgfsetstrokecolor{currentstroke}%
\pgfsetdash{}{0pt}%
\pgfsys@defobject{currentmarker}{\pgfqpoint{-0.048611in}{0.000000in}}{\pgfqpoint{-0.000000in}{0.000000in}}{%
\pgfpathmoveto{\pgfqpoint{-0.000000in}{0.000000in}}%
\pgfpathlineto{\pgfqpoint{-0.048611in}{0.000000in}}%
\pgfusepath{stroke,fill}%
}%
\begin{pgfscope}%
\pgfsys@transformshift{0.626386in}{1.880506in}%
\pgfsys@useobject{currentmarker}{}%
\end{pgfscope}%
\end{pgfscope}%
\begin{pgfscope}%
\definecolor{textcolor}{rgb}{0.000000,0.000000,0.000000}%
\pgfsetstrokecolor{textcolor}%
\pgfsetfillcolor{textcolor}%
\pgftext[x=0.333333in, y=1.811062in, left, base]{\color{textcolor}{\rmfamily\fontsize{14.000000}{16.800000}\selectfont\catcode`\^=\active\def^{\ifmmode\sp\else\^{}\fi}\catcode`\%=\active\def%{\%}$\mathdefault{20}$}}%
\end{pgfscope}%
\begin{pgfscope}%
\pgfsetbuttcap%
\pgfsetroundjoin%
\definecolor{currentfill}{rgb}{0.000000,0.000000,0.000000}%
\pgfsetfillcolor{currentfill}%
\pgfsetlinewidth{0.803000pt}%
\definecolor{currentstroke}{rgb}{0.000000,0.000000,0.000000}%
\pgfsetstrokecolor{currentstroke}%
\pgfsetdash{}{0pt}%
\pgfsys@defobject{currentmarker}{\pgfqpoint{-0.048611in}{0.000000in}}{\pgfqpoint{-0.000000in}{0.000000in}}{%
\pgfpathmoveto{\pgfqpoint{-0.000000in}{0.000000in}}%
\pgfpathlineto{\pgfqpoint{-0.048611in}{0.000000in}}%
\pgfusepath{stroke,fill}%
}%
\begin{pgfscope}%
\pgfsys@transformshift{0.626386in}{2.717463in}%
\pgfsys@useobject{currentmarker}{}%
\end{pgfscope}%
\end{pgfscope}%
\begin{pgfscope}%
\definecolor{textcolor}{rgb}{0.000000,0.000000,0.000000}%
\pgfsetstrokecolor{textcolor}%
\pgfsetfillcolor{textcolor}%
\pgftext[x=0.333333in, y=2.648019in, left, base]{\color{textcolor}{\rmfamily\fontsize{14.000000}{16.800000}\selectfont\catcode`\^=\active\def^{\ifmmode\sp\else\^{}\fi}\catcode`\%=\active\def%{\%}$\mathdefault{30}$}}%
\end{pgfscope}%
\begin{pgfscope}%
\pgfsetbuttcap%
\pgfsetroundjoin%
\definecolor{currentfill}{rgb}{0.000000,0.000000,0.000000}%
\pgfsetfillcolor{currentfill}%
\pgfsetlinewidth{0.803000pt}%
\definecolor{currentstroke}{rgb}{0.000000,0.000000,0.000000}%
\pgfsetstrokecolor{currentstroke}%
\pgfsetdash{}{0pt}%
\pgfsys@defobject{currentmarker}{\pgfqpoint{-0.048611in}{0.000000in}}{\pgfqpoint{-0.000000in}{0.000000in}}{%
\pgfpathmoveto{\pgfqpoint{-0.000000in}{0.000000in}}%
\pgfpathlineto{\pgfqpoint{-0.048611in}{0.000000in}}%
\pgfusepath{stroke,fill}%
}%
\begin{pgfscope}%
\pgfsys@transformshift{0.626386in}{3.554419in}%
\pgfsys@useobject{currentmarker}{}%
\end{pgfscope}%
\end{pgfscope}%
\begin{pgfscope}%
\definecolor{textcolor}{rgb}{0.000000,0.000000,0.000000}%
\pgfsetstrokecolor{textcolor}%
\pgfsetfillcolor{textcolor}%
\pgftext[x=0.333333in, y=3.484975in, left, base]{\color{textcolor}{\rmfamily\fontsize{14.000000}{16.800000}\selectfont\catcode`\^=\active\def^{\ifmmode\sp\else\^{}\fi}\catcode`\%=\active\def%{\%}$\mathdefault{40}$}}%
\end{pgfscope}%
\begin{pgfscope}%
\pgfsetbuttcap%
\pgfsetroundjoin%
\definecolor{currentfill}{rgb}{0.000000,0.000000,0.000000}%
\pgfsetfillcolor{currentfill}%
\pgfsetlinewidth{0.803000pt}%
\definecolor{currentstroke}{rgb}{0.000000,0.000000,0.000000}%
\pgfsetstrokecolor{currentstroke}%
\pgfsetdash{}{0pt}%
\pgfsys@defobject{currentmarker}{\pgfqpoint{-0.048611in}{0.000000in}}{\pgfqpoint{-0.000000in}{0.000000in}}{%
\pgfpathmoveto{\pgfqpoint{-0.000000in}{0.000000in}}%
\pgfpathlineto{\pgfqpoint{-0.048611in}{0.000000in}}%
\pgfusepath{stroke,fill}%
}%
\begin{pgfscope}%
\pgfsys@transformshift{0.626386in}{4.391376in}%
\pgfsys@useobject{currentmarker}{}%
\end{pgfscope}%
\end{pgfscope}%
\begin{pgfscope}%
\definecolor{textcolor}{rgb}{0.000000,0.000000,0.000000}%
\pgfsetstrokecolor{textcolor}%
\pgfsetfillcolor{textcolor}%
\pgftext[x=0.333333in, y=4.321932in, left, base]{\color{textcolor}{\rmfamily\fontsize{14.000000}{16.800000}\selectfont\catcode`\^=\active\def^{\ifmmode\sp\else\^{}\fi}\catcode`\%=\active\def%{\%}$\mathdefault{50}$}}%
\end{pgfscope}%
\begin{pgfscope}%
\pgfsetbuttcap%
\pgfsetroundjoin%
\definecolor{currentfill}{rgb}{0.000000,0.000000,0.000000}%
\pgfsetfillcolor{currentfill}%
\pgfsetlinewidth{0.803000pt}%
\definecolor{currentstroke}{rgb}{0.000000,0.000000,0.000000}%
\pgfsetstrokecolor{currentstroke}%
\pgfsetdash{}{0pt}%
\pgfsys@defobject{currentmarker}{\pgfqpoint{-0.048611in}{0.000000in}}{\pgfqpoint{-0.000000in}{0.000000in}}{%
\pgfpathmoveto{\pgfqpoint{-0.000000in}{0.000000in}}%
\pgfpathlineto{\pgfqpoint{-0.048611in}{0.000000in}}%
\pgfusepath{stroke,fill}%
}%
\begin{pgfscope}%
\pgfsys@transformshift{0.626386in}{5.228333in}%
\pgfsys@useobject{currentmarker}{}%
\end{pgfscope}%
\end{pgfscope}%
\begin{pgfscope}%
\definecolor{textcolor}{rgb}{0.000000,0.000000,0.000000}%
\pgfsetstrokecolor{textcolor}%
\pgfsetfillcolor{textcolor}%
\pgftext[x=0.333333in, y=5.158888in, left, base]{\color{textcolor}{\rmfamily\fontsize{14.000000}{16.800000}\selectfont\catcode`\^=\active\def^{\ifmmode\sp\else\^{}\fi}\catcode`\%=\active\def%{\%}$\mathdefault{60}$}}%
\end{pgfscope}%
\begin{pgfscope}%
\definecolor{textcolor}{rgb}{0.000000,0.000000,0.000000}%
\pgfsetstrokecolor{textcolor}%
\pgfsetfillcolor{textcolor}%
\pgftext[x=0.277777in,y=2.918332in,,bottom,rotate=90.000000]{\color{textcolor}{\rmfamily\fontsize{14.000000}{16.800000}\selectfont\catcode`\^=\active\def^{\ifmmode\sp\else\^{}\fi}\catcode`\%=\active\def%{\%}f2}}%
\end{pgfscope}%
\begin{pgfscope}%
\pgfpathrectangle{\pgfqpoint{0.626386in}{0.608332in}}{\pgfqpoint{6.200000in}{4.620000in}}%
\pgfusepath{clip}%
\pgfsetrectcap%
\pgfsetroundjoin%
\pgfsetlinewidth{3.011250pt}%
\definecolor{currentstroke}{rgb}{0.000000,0.000000,0.000000}%
\pgfsetstrokecolor{currentstroke}%
\pgfsetdash{}{0pt}%
\pgfpathmoveto{\pgfqpoint{0.626386in}{4.391376in}}%
\pgfpathlineto{\pgfqpoint{0.627161in}{4.307262in}}%
\pgfpathlineto{\pgfqpoint{0.629487in}{4.224002in}}%
\pgfpathlineto{\pgfqpoint{0.633363in}{4.141596in}}%
\pgfpathlineto{\pgfqpoint{0.638790in}{4.060043in}}%
\pgfpathlineto{\pgfqpoint{0.645767in}{3.979345in}}%
\pgfpathlineto{\pgfqpoint{0.654294in}{3.899501in}}%
\pgfpathlineto{\pgfqpoint{0.664372in}{3.820510in}}%
\pgfpathlineto{\pgfqpoint{0.676001in}{3.742374in}}%
\pgfpathlineto{\pgfqpoint{0.689180in}{3.665091in}}%
\pgfpathlineto{\pgfqpoint{0.703909in}{3.588663in}}%
\pgfpathlineto{\pgfqpoint{0.720189in}{3.513088in}}%
\pgfpathlineto{\pgfqpoint{0.738019in}{3.438368in}}%
\pgfpathlineto{\pgfqpoint{0.757400in}{3.364501in}}%
\pgfpathlineto{\pgfqpoint{0.778331in}{3.291488in}}%
\pgfpathlineto{\pgfqpoint{0.800813in}{3.219329in}}%
\pgfpathlineto{\pgfqpoint{0.824845in}{3.148025in}}%
\pgfpathlineto{\pgfqpoint{0.850428in}{3.077574in}}%
\pgfpathlineto{\pgfqpoint{0.877561in}{3.007977in}}%
\pgfpathlineto{\pgfqpoint{0.906244in}{2.939234in}}%
\pgfpathlineto{\pgfqpoint{0.936478in}{2.871345in}}%
\pgfpathlineto{\pgfqpoint{0.968263in}{2.804310in}}%
\pgfpathlineto{\pgfqpoint{1.001598in}{2.738129in}}%
\pgfpathlineto{\pgfqpoint{1.036483in}{2.672801in}}%
\pgfpathlineto{\pgfqpoint{1.072919in}{2.608328in}}%
\pgfpathlineto{\pgfqpoint{1.110905in}{2.544709in}}%
\pgfpathlineto{\pgfqpoint{1.150442in}{2.481943in}}%
\pgfpathlineto{\pgfqpoint{1.191529in}{2.420032in}}%
\pgfpathlineto{\pgfqpoint{1.234167in}{2.358975in}}%
\pgfpathlineto{\pgfqpoint{1.278355in}{2.298771in}}%
\pgfpathlineto{\pgfqpoint{1.324094in}{2.239422in}}%
\pgfpathlineto{\pgfqpoint{1.371383in}{2.180926in}}%
\pgfpathlineto{\pgfqpoint{1.420223in}{2.123284in}}%
\pgfpathlineto{\pgfqpoint{1.470613in}{2.066497in}}%
\pgfpathlineto{\pgfqpoint{1.522553in}{2.010563in}}%
\pgfpathlineto{\pgfqpoint{1.576044in}{1.955483in}}%
\pgfpathlineto{\pgfqpoint{1.631085in}{1.901257in}}%
\pgfpathlineto{\pgfqpoint{1.687677in}{1.847886in}}%
\pgfpathlineto{\pgfqpoint{1.745820in}{1.795368in}}%
\pgfpathlineto{\pgfqpoint{1.805512in}{1.743704in}}%
\pgfpathlineto{\pgfqpoint{1.866756in}{1.692894in}}%
\pgfpathlineto{\pgfqpoint{1.929549in}{1.642937in}}%
\pgfpathlineto{\pgfqpoint{1.993893in}{1.593835in}}%
\pgfpathlineto{\pgfqpoint{2.059788in}{1.545587in}}%
\pgfpathlineto{\pgfqpoint{2.127233in}{1.498193in}}%
\pgfpathlineto{\pgfqpoint{2.196229in}{1.451653in}}%
\pgfpathlineto{\pgfqpoint{2.266775in}{1.405966in}}%
\pgfpathlineto{\pgfqpoint{2.338871in}{1.361134in}}%
\pgfpathlineto{\pgfqpoint{2.412518in}{1.317156in}}%
\pgfpathlineto{\pgfqpoint{2.487716in}{1.274031in}}%
\pgfpathlineto{\pgfqpoint{2.564464in}{1.231760in}}%
\pgfpathlineto{\pgfqpoint{2.642762in}{1.190344in}}%
\pgfpathlineto{\pgfqpoint{2.722611in}{1.149781in}}%
\pgfpathlineto{\pgfqpoint{2.804010in}{1.110073in}}%
\pgfpathlineto{\pgfqpoint{2.886960in}{1.071218in}}%
\pgfpathlineto{\pgfqpoint{2.971460in}{1.033217in}}%
\pgfpathlineto{\pgfqpoint{3.057510in}{0.996070in}}%
\pgfpathlineto{\pgfqpoint{3.145112in}{0.959777in}}%
\pgfpathlineto{\pgfqpoint{3.234263in}{0.924338in}}%
\pgfpathlineto{\pgfqpoint{3.324965in}{0.889753in}}%
\pgfpathlineto{\pgfqpoint{3.389449in}{0.866091in}}%
\pgfpathlineto{\pgfqpoint{3.436350in}{0.849652in}}%
\pgfpathlineto{\pgfqpoint{3.484027in}{0.833640in}}%
\pgfpathlineto{\pgfqpoint{3.532479in}{0.818056in}}%
\pgfpathlineto{\pgfqpoint{3.581706in}{0.802898in}}%
\pgfpathlineto{\pgfqpoint{3.631709in}{0.788168in}}%
\pgfpathlineto{\pgfqpoint{3.682486in}{0.773864in}}%
\pgfpathlineto{\pgfqpoint{3.734039in}{0.759987in}}%
\pgfpathlineto{\pgfqpoint{3.786367in}{0.746537in}}%
\pgfpathlineto{\pgfqpoint{3.839470in}{0.733515in}}%
\pgfpathlineto{\pgfqpoint{3.893349in}{0.720919in}}%
\pgfpathlineto{\pgfqpoint{3.948003in}{0.708750in}}%
\pgfpathlineto{\pgfqpoint{4.003432in}{0.697008in}}%
\pgfpathlineto{\pgfqpoint{4.059636in}{0.685694in}}%
\pgfpathlineto{\pgfqpoint{4.116616in}{0.674806in}}%
\pgfpathlineto{\pgfqpoint{4.174370in}{0.664345in}}%
\pgfpathlineto{\pgfqpoint{4.232900in}{0.654311in}}%
\pgfpathlineto{\pgfqpoint{4.292205in}{0.644704in}}%
\pgfpathlineto{\pgfqpoint{4.352286in}{0.635524in}}%
\pgfpathlineto{\pgfqpoint{4.413141in}{0.626771in}}%
\pgfpathlineto{\pgfqpoint{4.474772in}{0.618445in}}%
\pgfpathlineto{\pgfqpoint{4.537178in}{0.610546in}}%
\pgfpathlineto{\pgfqpoint{4.569987in}{0.606666in}}%
\pgfusepath{stroke}%
\end{pgfscope}%
\begin{pgfscope}%
\pgfpathrectangle{\pgfqpoint{0.626386in}{0.608332in}}{\pgfqpoint{6.200000in}{4.620000in}}%
\pgfusepath{clip}%
\pgfsetrectcap%
\pgfsetroundjoin%
\pgfsetlinewidth{1.003750pt}%
\definecolor{currentstroke}{rgb}{0.000000,0.000000,0.000000}%
\pgfsetstrokecolor{currentstroke}%
\pgfsetstrokeopacity{0.200000}%
\pgfsetdash{}{0pt}%
\pgfpathmoveto{\pgfqpoint{0.626386in}{5.228333in}}%
\pgfpathlineto{\pgfqpoint{0.627316in}{5.127396in}}%
\pgfpathlineto{\pgfqpoint{0.630107in}{5.027483in}}%
\pgfpathlineto{\pgfqpoint{0.634758in}{4.928596in}}%
\pgfpathlineto{\pgfqpoint{0.641270in}{4.830733in}}%
\pgfpathlineto{\pgfqpoint{0.649643in}{4.733895in}}%
\pgfpathlineto{\pgfqpoint{0.659876in}{4.638082in}}%
\pgfpathlineto{\pgfqpoint{0.671970in}{4.543294in}}%
\pgfpathlineto{\pgfqpoint{0.685924in}{4.449530in}}%
\pgfpathlineto{\pgfqpoint{0.701738in}{4.356791in}}%
\pgfpathlineto{\pgfqpoint{0.719414in}{4.265077in}}%
\pgfpathlineto{\pgfqpoint{0.738950in}{4.174387in}}%
\pgfpathlineto{\pgfqpoint{0.760346in}{4.084722in}}%
\pgfpathlineto{\pgfqpoint{0.783603in}{3.996082in}}%
\pgfpathlineto{\pgfqpoint{0.808720in}{3.908467in}}%
\pgfpathlineto{\pgfqpoint{0.835698in}{3.821877in}}%
\pgfpathlineto{\pgfqpoint{0.864537in}{3.736311in}}%
\pgfpathlineto{\pgfqpoint{0.895236in}{3.651770in}}%
\pgfpathlineto{\pgfqpoint{0.927796in}{3.568253in}}%
\pgfpathlineto{\pgfqpoint{0.962216in}{3.485762in}}%
\pgfpathlineto{\pgfqpoint{0.998497in}{3.404295in}}%
\pgfpathlineto{\pgfqpoint{1.036638in}{3.323853in}}%
\pgfpathlineto{\pgfqpoint{1.076640in}{3.244436in}}%
\pgfpathlineto{\pgfqpoint{1.118503in}{3.166043in}}%
\pgfpathlineto{\pgfqpoint{1.162226in}{3.088675in}}%
\pgfpathlineto{\pgfqpoint{1.207809in}{3.012332in}}%
\pgfpathlineto{\pgfqpoint{1.255253in}{2.937014in}}%
\pgfpathlineto{\pgfqpoint{1.304558in}{2.862720in}}%
\pgfpathlineto{\pgfqpoint{1.355723in}{2.789451in}}%
\pgfpathlineto{\pgfqpoint{1.408749in}{2.717207in}}%
\pgfpathlineto{\pgfqpoint{1.463635in}{2.645987in}}%
\pgfpathlineto{\pgfqpoint{1.520382in}{2.575793in}}%
\pgfpathlineto{\pgfqpoint{1.578990in}{2.506623in}}%
\pgfpathlineto{\pgfqpoint{1.639458in}{2.438477in}}%
\pgfpathlineto{\pgfqpoint{1.701786in}{2.371357in}}%
\pgfpathlineto{\pgfqpoint{1.765976in}{2.305261in}}%
\pgfpathlineto{\pgfqpoint{1.832025in}{2.240190in}}%
\pgfpathlineto{\pgfqpoint{1.899935in}{2.176144in}}%
\pgfpathlineto{\pgfqpoint{1.969706in}{2.113122in}}%
\pgfpathlineto{\pgfqpoint{2.041338in}{2.051126in}}%
\pgfpathlineto{\pgfqpoint{2.114830in}{1.990154in}}%
\pgfpathlineto{\pgfqpoint{2.190182in}{1.930206in}}%
\pgfpathlineto{\pgfqpoint{2.267395in}{1.871284in}}%
\pgfpathlineto{\pgfqpoint{2.346469in}{1.813386in}}%
\pgfpathlineto{\pgfqpoint{2.427403in}{1.756513in}}%
\pgfpathlineto{\pgfqpoint{2.510197in}{1.700665in}}%
\pgfpathlineto{\pgfqpoint{2.594853in}{1.645841in}}%
\pgfpathlineto{\pgfqpoint{2.681368in}{1.592042in}}%
\pgfpathlineto{\pgfqpoint{2.769745in}{1.539268in}}%
\pgfpathlineto{\pgfqpoint{2.859982in}{1.487519in}}%
\pgfpathlineto{\pgfqpoint{2.952079in}{1.436794in}}%
\pgfpathlineto{\pgfqpoint{3.046037in}{1.387094in}}%
\pgfpathlineto{\pgfqpoint{3.141856in}{1.338419in}}%
\pgfpathlineto{\pgfqpoint{3.239535in}{1.290768in}}%
\pgfpathlineto{\pgfqpoint{3.339074in}{1.244143in}}%
\pgfpathlineto{\pgfqpoint{3.440475in}{1.198542in}}%
\pgfpathlineto{\pgfqpoint{3.543735in}{1.153966in}}%
\pgfpathlineto{\pgfqpoint{3.648857in}{1.110414in}}%
\pgfpathlineto{\pgfqpoint{3.755839in}{1.067887in}}%
\pgfpathlineto{\pgfqpoint{3.864681in}{1.026385in}}%
\pgfpathlineto{\pgfqpoint{3.942061in}{0.997990in}}%
\pgfpathlineto{\pgfqpoint{3.998343in}{0.978264in}}%
\pgfpathlineto{\pgfqpoint{4.055555in}{0.959050in}}%
\pgfpathlineto{\pgfqpoint{4.113698in}{0.940348in}}%
\pgfpathlineto{\pgfqpoint{4.172770in}{0.922159in}}%
\pgfpathlineto{\pgfqpoint{4.232773in}{0.904482in}}%
\pgfpathlineto{\pgfqpoint{4.293706in}{0.887318in}}%
\pgfpathlineto{\pgfqpoint{4.355570in}{0.870666in}}%
\pgfpathlineto{\pgfqpoint{4.418363in}{0.854526in}}%
\pgfpathlineto{\pgfqpoint{4.482087in}{0.838899in}}%
\pgfpathlineto{\pgfqpoint{4.546742in}{0.823784in}}%
\pgfpathlineto{\pgfqpoint{4.612326in}{0.809182in}}%
\pgfpathlineto{\pgfqpoint{4.678841in}{0.795091in}}%
\pgfpathlineto{\pgfqpoint{4.746286in}{0.781514in}}%
\pgfpathlineto{\pgfqpoint{4.814661in}{0.768448in}}%
\pgfpathlineto{\pgfqpoint{4.883967in}{0.755895in}}%
\pgfpathlineto{\pgfqpoint{4.954203in}{0.743854in}}%
\pgfpathlineto{\pgfqpoint{5.025369in}{0.732326in}}%
\pgfpathlineto{\pgfqpoint{5.097466in}{0.721310in}}%
\pgfpathlineto{\pgfqpoint{5.170493in}{0.710807in}}%
\pgfpathlineto{\pgfqpoint{5.244450in}{0.700815in}}%
\pgfpathlineto{\pgfqpoint{5.319337in}{0.691336in}}%
\pgfpathlineto{\pgfqpoint{5.395154in}{0.682370in}}%
\pgfpathlineto{\pgfqpoint{5.471902in}{0.673916in}}%
\pgfpathlineto{\pgfqpoint{5.549580in}{0.665974in}}%
\pgfpathlineto{\pgfqpoint{5.628189in}{0.658545in}}%
\pgfpathlineto{\pgfqpoint{5.707728in}{0.651628in}}%
\pgfpathlineto{\pgfqpoint{5.788197in}{0.645223in}}%
\pgfpathlineto{\pgfqpoint{5.869596in}{0.639331in}}%
\pgfpathlineto{\pgfqpoint{5.951925in}{0.633951in}}%
\pgfpathlineto{\pgfqpoint{6.035185in}{0.629083in}}%
\pgfpathlineto{\pgfqpoint{6.119375in}{0.624728in}}%
\pgfpathlineto{\pgfqpoint{6.204496in}{0.620886in}}%
\pgfpathlineto{\pgfqpoint{6.290546in}{0.617555in}}%
\pgfpathlineto{\pgfqpoint{6.377527in}{0.614737in}}%
\pgfpathlineto{\pgfqpoint{6.465438in}{0.612431in}}%
\pgfpathlineto{\pgfqpoint{6.554280in}{0.610638in}}%
\pgfpathlineto{\pgfqpoint{6.644052in}{0.609357in}}%
\pgfpathlineto{\pgfqpoint{6.734754in}{0.608589in}}%
\pgfpathlineto{\pgfqpoint{6.826386in}{0.608332in}}%
\pgfusepath{stroke}%
\end{pgfscope}%
\begin{pgfscope}%
\pgfsetrectcap%
\pgfsetmiterjoin%
\pgfsetlinewidth{0.803000pt}%
\definecolor{currentstroke}{rgb}{0.000000,0.000000,0.000000}%
\pgfsetstrokecolor{currentstroke}%
\pgfsetdash{}{0pt}%
\pgfpathmoveto{\pgfqpoint{0.626386in}{0.608332in}}%
\pgfpathlineto{\pgfqpoint{0.626386in}{5.228333in}}%
\pgfusepath{stroke}%
\end{pgfscope}%
\begin{pgfscope}%
\pgfsetrectcap%
\pgfsetmiterjoin%
\pgfsetlinewidth{0.803000pt}%
\definecolor{currentstroke}{rgb}{0.000000,0.000000,0.000000}%
\pgfsetstrokecolor{currentstroke}%
\pgfsetdash{}{0pt}%
\pgfpathmoveto{\pgfqpoint{6.826386in}{0.608332in}}%
\pgfpathlineto{\pgfqpoint{6.826386in}{5.228333in}}%
\pgfusepath{stroke}%
\end{pgfscope}%
\begin{pgfscope}%
\pgfsetrectcap%
\pgfsetmiterjoin%
\pgfsetlinewidth{0.803000pt}%
\definecolor{currentstroke}{rgb}{0.000000,0.000000,0.000000}%
\pgfsetstrokecolor{currentstroke}%
\pgfsetdash{}{0pt}%
\pgfpathmoveto{\pgfqpoint{0.626386in}{0.608332in}}%
\pgfpathlineto{\pgfqpoint{6.826386in}{0.608332in}}%
\pgfusepath{stroke}%
\end{pgfscope}%
\begin{pgfscope}%
\pgfsetrectcap%
\pgfsetmiterjoin%
\pgfsetlinewidth{0.803000pt}%
\definecolor{currentstroke}{rgb}{0.000000,0.000000,0.000000}%
\pgfsetstrokecolor{currentstroke}%
\pgfsetdash{}{0pt}%
\pgfpathmoveto{\pgfqpoint{0.626386in}{5.228333in}}%
\pgfpathlineto{\pgfqpoint{6.826386in}{5.228333in}}%
\pgfusepath{stroke}%
\end{pgfscope}%
\begin{pgfscope}%
\pgfsetbuttcap%
\pgfsetmiterjoin%
\definecolor{currentfill}{rgb}{0.300000,0.300000,0.300000}%
\pgfsetfillcolor{currentfill}%
\pgfsetfillopacity{0.500000}%
\pgfsetlinewidth{1.003750pt}%
\definecolor{currentstroke}{rgb}{0.300000,0.300000,0.300000}%
\pgfsetstrokecolor{currentstroke}%
\pgfsetstrokeopacity{0.500000}%
\pgfsetdash{}{0pt}%
\pgfpathmoveto{\pgfqpoint{4.314927in}{4.220000in}}%
\pgfpathlineto{\pgfqpoint{6.718053in}{4.220000in}}%
\pgfpathquadraticcurveto{\pgfqpoint{6.756942in}{4.220000in}}{\pgfqpoint{6.756942in}{4.258889in}}%
\pgfpathlineto{\pgfqpoint{6.756942in}{5.064444in}}%
\pgfpathquadraticcurveto{\pgfqpoint{6.756942in}{5.103333in}}{\pgfqpoint{6.718053in}{5.103333in}}%
\pgfpathlineto{\pgfqpoint{4.314927in}{5.103333in}}%
\pgfpathquadraticcurveto{\pgfqpoint{4.276038in}{5.103333in}}{\pgfqpoint{4.276038in}{5.064444in}}%
\pgfpathlineto{\pgfqpoint{4.276038in}{4.258889in}}%
\pgfpathquadraticcurveto{\pgfqpoint{4.276038in}{4.220000in}}{\pgfqpoint{4.314927in}{4.220000in}}%
\pgfpathlineto{\pgfqpoint{4.314927in}{4.220000in}}%
\pgfpathclose%
\pgfusepath{stroke,fill}%
\end{pgfscope}%
\begin{pgfscope}%
\pgfsetbuttcap%
\pgfsetmiterjoin%
\definecolor{currentfill}{rgb}{1.000000,1.000000,1.000000}%
\pgfsetfillcolor{currentfill}%
\pgfsetlinewidth{1.003750pt}%
\definecolor{currentstroke}{rgb}{0.800000,0.800000,0.800000}%
\pgfsetstrokecolor{currentstroke}%
\pgfsetdash{}{0pt}%
\pgfpathmoveto{\pgfqpoint{4.287149in}{4.247778in}}%
\pgfpathlineto{\pgfqpoint{6.690275in}{4.247778in}}%
\pgfpathquadraticcurveto{\pgfqpoint{6.729164in}{4.247778in}}{\pgfqpoint{6.729164in}{4.286667in}}%
\pgfpathlineto{\pgfqpoint{6.729164in}{5.092221in}}%
\pgfpathquadraticcurveto{\pgfqpoint{6.729164in}{5.131110in}}{\pgfqpoint{6.690275in}{5.131110in}}%
\pgfpathlineto{\pgfqpoint{4.287149in}{5.131110in}}%
\pgfpathquadraticcurveto{\pgfqpoint{4.248260in}{5.131110in}}{\pgfqpoint{4.248260in}{5.092221in}}%
\pgfpathlineto{\pgfqpoint{4.248260in}{4.286667in}}%
\pgfpathquadraticcurveto{\pgfqpoint{4.248260in}{4.247778in}}{\pgfqpoint{4.287149in}{4.247778in}}%
\pgfpathlineto{\pgfqpoint{4.287149in}{4.247778in}}%
\pgfpathclose%
\pgfusepath{stroke,fill}%
\end{pgfscope}%
\begin{pgfscope}%
\pgfsetrectcap%
\pgfsetroundjoin%
\pgfsetlinewidth{3.011250pt}%
\definecolor{currentstroke}{rgb}{0.000000,0.000000,0.000000}%
\pgfsetstrokecolor{currentstroke}%
\pgfsetdash{}{0pt}%
\pgfpathmoveto{\pgfqpoint{4.326038in}{4.982499in}}%
\pgfpathlineto{\pgfqpoint{4.520482in}{4.982499in}}%
\pgfpathlineto{\pgfqpoint{4.714927in}{4.982499in}}%
\pgfusepath{stroke}%
\end{pgfscope}%
\begin{pgfscope}%
\definecolor{textcolor}{rgb}{0.000000,0.000000,0.000000}%
\pgfsetstrokecolor{textcolor}%
\pgfsetfillcolor{textcolor}%
\pgftext[x=4.870482in,y=4.914444in,left,base]{\color{textcolor}{\rmfamily\fontsize{14.000000}{16.800000}\selectfont\catcode`\^=\active\def^{\ifmmode\sp\else\^{}\fi}\catcode`\%=\active\def%{\%}Pareto Front}}%
\end{pgfscope}%
\begin{pgfscope}%
\pgfsetbuttcap%
\pgfsetroundjoin%
\definecolor{currentfill}{rgb}{0.121569,0.466667,0.705882}%
\pgfsetfillcolor{currentfill}%
\pgfsetlinewidth{1.003750pt}%
\definecolor{currentstroke}{rgb}{0.121569,0.466667,0.705882}%
\pgfsetstrokecolor{currentstroke}%
\pgfsetdash{}{0pt}%
\pgfsys@defobject{currentmarker}{\pgfqpoint{-0.012028in}{-0.012028in}}{\pgfqpoint{0.012028in}{0.012028in}}{%
\pgfpathmoveto{\pgfqpoint{0.000000in}{-0.012028in}}%
\pgfpathcurveto{\pgfqpoint{0.003190in}{-0.012028in}}{\pgfqpoint{0.006250in}{-0.010761in}}{\pgfqpoint{0.008505in}{-0.008505in}}%
\pgfpathcurveto{\pgfqpoint{0.010761in}{-0.006250in}}{\pgfqpoint{0.012028in}{-0.003190in}}{\pgfqpoint{0.012028in}{0.000000in}}%
\pgfpathcurveto{\pgfqpoint{0.012028in}{0.003190in}}{\pgfqpoint{0.010761in}{0.006250in}}{\pgfqpoint{0.008505in}{0.008505in}}%
\pgfpathcurveto{\pgfqpoint{0.006250in}{0.010761in}}{\pgfqpoint{0.003190in}{0.012028in}}{\pgfqpoint{0.000000in}{0.012028in}}%
\pgfpathcurveto{\pgfqpoint{-0.003190in}{0.012028in}}{\pgfqpoint{-0.006250in}{0.010761in}}{\pgfqpoint{-0.008505in}{0.008505in}}%
\pgfpathcurveto{\pgfqpoint{-0.010761in}{0.006250in}}{\pgfqpoint{-0.012028in}{0.003190in}}{\pgfqpoint{-0.012028in}{0.000000in}}%
\pgfpathcurveto{\pgfqpoint{-0.012028in}{-0.003190in}}{\pgfqpoint{-0.010761in}{-0.006250in}}{\pgfqpoint{-0.008505in}{-0.008505in}}%
\pgfpathcurveto{\pgfqpoint{-0.006250in}{-0.010761in}}{\pgfqpoint{-0.003190in}{-0.012028in}}{\pgfqpoint{0.000000in}{-0.012028in}}%
\pgfpathlineto{\pgfqpoint{0.000000in}{-0.012028in}}%
\pgfpathclose%
\pgfusepath{stroke,fill}%
}%
\begin{pgfscope}%
\pgfsys@transformshift{4.520482in}{4.690486in}%
\pgfsys@useobject{currentmarker}{}%
\end{pgfscope}%
\end{pgfscope}%
\begin{pgfscope}%
\definecolor{textcolor}{rgb}{0.000000,0.000000,0.000000}%
\pgfsetstrokecolor{textcolor}%
\pgfsetfillcolor{textcolor}%
\pgftext[x=4.870482in,y=4.639444in,left,base]{\color{textcolor}{\rmfamily\fontsize{14.000000}{16.800000}\selectfont\catcode`\^=\active\def^{\ifmmode\sp\else\^{}\fi}\catcode`\%=\active\def%{\%}Tested points}}%
\end{pgfscope}%
\begin{pgfscope}%
\pgfsetbuttcap%
\pgfsetroundjoin%
\definecolor{currentfill}{rgb}{0.839216,0.152941,0.156863}%
\pgfsetfillcolor{currentfill}%
\pgfsetlinewidth{1.003750pt}%
\definecolor{currentstroke}{rgb}{0.839216,0.152941,0.156863}%
\pgfsetstrokecolor{currentstroke}%
\pgfsetdash{}{0pt}%
\pgfsys@defobject{currentmarker}{\pgfqpoint{-0.031056in}{-0.031056in}}{\pgfqpoint{0.031056in}{0.031056in}}{%
\pgfpathmoveto{\pgfqpoint{0.000000in}{-0.031056in}}%
\pgfpathcurveto{\pgfqpoint{0.008236in}{-0.031056in}}{\pgfqpoint{0.016136in}{-0.027784in}}{\pgfqpoint{0.021960in}{-0.021960in}}%
\pgfpathcurveto{\pgfqpoint{0.027784in}{-0.016136in}}{\pgfqpoint{0.031056in}{-0.008236in}}{\pgfqpoint{0.031056in}{0.000000in}}%
\pgfpathcurveto{\pgfqpoint{0.031056in}{0.008236in}}{\pgfqpoint{0.027784in}{0.016136in}}{\pgfqpoint{0.021960in}{0.021960in}}%
\pgfpathcurveto{\pgfqpoint{0.016136in}{0.027784in}}{\pgfqpoint{0.008236in}{0.031056in}}{\pgfqpoint{0.000000in}{0.031056in}}%
\pgfpathcurveto{\pgfqpoint{-0.008236in}{0.031056in}}{\pgfqpoint{-0.016136in}{0.027784in}}{\pgfqpoint{-0.021960in}{0.021960in}}%
\pgfpathcurveto{\pgfqpoint{-0.027784in}{0.016136in}}{\pgfqpoint{-0.031056in}{0.008236in}}{\pgfqpoint{-0.031056in}{0.000000in}}%
\pgfpathcurveto{\pgfqpoint{-0.031056in}{-0.008236in}}{\pgfqpoint{-0.027784in}{-0.016136in}}{\pgfqpoint{-0.021960in}{-0.021960in}}%
\pgfpathcurveto{\pgfqpoint{-0.016136in}{-0.027784in}}{\pgfqpoint{-0.008236in}{-0.031056in}}{\pgfqpoint{0.000000in}{-0.031056in}}%
\pgfpathlineto{\pgfqpoint{0.000000in}{-0.031056in}}%
\pgfpathclose%
\pgfusepath{stroke,fill}%
}%
\begin{pgfscope}%
\pgfsys@transformshift{4.520482in}{4.415486in}%
\pgfsys@useobject{currentmarker}{}%
\end{pgfscope}%
\end{pgfscope}%
\begin{pgfscope}%
\definecolor{textcolor}{rgb}{0.000000,0.000000,0.000000}%
\pgfsetstrokecolor{textcolor}%
\pgfsetfillcolor{textcolor}%
\pgftext[x=4.870482in,y=4.364445in,left,base]{\color{textcolor}{\rmfamily\fontsize{14.000000}{16.800000}\selectfont\catcode`\^=\active\def^{\ifmmode\sp\else\^{}\fi}\catcode`\%=\active\def%{\%}Alternative solutions}}%
\end{pgfscope}%
\end{pgfpicture}%
\makeatother%
\endgroup%
}
  \caption{All of the alternative points inside the near-feasible space selected
  using the algorithm described in Section \ref{section:mga-moo}.}
  \label{fig:nd-mga}
\end{figure}

\begin{figure}[H]
  \centering
  \resizebox{1\columnwidth}{!}{%% Creator: Matplotlib, PGF backend
%%
%% To include the figure in your LaTeX document, write
%%   \input{<filename>.pgf}
%%
%% Make sure the required packages are loaded in your preamble
%%   \usepackage{pgf}
%%
%% Also ensure that all the required font packages are loaded; for instance,
%% the lmodern package is sometimes necessary when using math font.
%%   \usepackage{lmodern}
%%
%% Figures using additional raster images can only be included by \input if
%% they are in the same directory as the main LaTeX file. For loading figures
%% from other directories you can use the `import` package
%%   \usepackage{import}
%%
%% and then include the figures with
%%   \import{<path to file>}{<filename>.pgf}
%%
%% Matplotlib used the following preamble
%%   \def\mathdefault#1{#1}
%%   \everymath=\expandafter{\the\everymath\displaystyle}
%%   \IfFileExists{scrextend.sty}{
%%     \usepackage[fontsize=10.000000pt]{scrextend}
%%   }{
%%     \renewcommand{\normalsize}{\fontsize{10.000000}{12.000000}\selectfont}
%%     \normalsize
%%   }
%%   
%%   \makeatletter\@ifpackageloaded{underscore}{}{\usepackage[strings]{underscore}}\makeatother
%%
\begingroup%
\makeatletter%
\begin{pgfpicture}%
\pgfpathrectangle{\pgfpointorigin}{\pgfqpoint{9.654231in}{3.182465in}}%
\pgfusepath{use as bounding box, clip}%
\begin{pgfscope}%
\pgfsetbuttcap%
\pgfsetmiterjoin%
\definecolor{currentfill}{rgb}{1.000000,1.000000,1.000000}%
\pgfsetfillcolor{currentfill}%
\pgfsetlinewidth{0.000000pt}%
\definecolor{currentstroke}{rgb}{0.000000,0.000000,0.000000}%
\pgfsetstrokecolor{currentstroke}%
\pgfsetdash{}{0pt}%
\pgfpathmoveto{\pgfqpoint{0.000000in}{0.000000in}}%
\pgfpathlineto{\pgfqpoint{9.654231in}{0.000000in}}%
\pgfpathlineto{\pgfqpoint{9.654231in}{3.182465in}}%
\pgfpathlineto{\pgfqpoint{0.000000in}{3.182465in}}%
\pgfpathlineto{\pgfqpoint{0.000000in}{0.000000in}}%
\pgfpathclose%
\pgfusepath{fill}%
\end{pgfscope}%
\begin{pgfscope}%
\pgfsetbuttcap%
\pgfsetmiterjoin%
\definecolor{currentfill}{rgb}{1.000000,1.000000,1.000000}%
\pgfsetfillcolor{currentfill}%
\pgfsetlinewidth{0.000000pt}%
\definecolor{currentstroke}{rgb}{0.000000,0.000000,0.000000}%
\pgfsetstrokecolor{currentstroke}%
\pgfsetstrokeopacity{0.000000}%
\pgfsetdash{}{0pt}%
\pgfpathmoveto{\pgfqpoint{3.536584in}{0.147348in}}%
\pgfpathlineto{\pgfqpoint{6.271879in}{0.147348in}}%
\pgfpathlineto{\pgfqpoint{6.271879in}{2.882642in}}%
\pgfpathlineto{\pgfqpoint{3.536584in}{2.882642in}}%
\pgfpathlineto{\pgfqpoint{3.536584in}{0.147348in}}%
\pgfpathclose%
\pgfusepath{fill}%
\end{pgfscope}%
\begin{pgfscope}%
\pgfsetbuttcap%
\pgfsetmiterjoin%
\definecolor{currentfill}{rgb}{0.950000,0.950000,0.950000}%
\pgfsetfillcolor{currentfill}%
\pgfsetfillopacity{0.500000}%
\pgfsetlinewidth{1.003750pt}%
\definecolor{currentstroke}{rgb}{0.950000,0.950000,0.950000}%
\pgfsetstrokecolor{currentstroke}%
\pgfsetstrokeopacity{0.500000}%
\pgfsetdash{}{0pt}%
\pgfpathmoveto{\pgfqpoint{4.941195in}{1.930798in}}%
\pgfpathlineto{\pgfqpoint{6.147731in}{1.099507in}}%
\pgfpathlineto{\pgfqpoint{6.226389in}{2.033906in}}%
\pgfpathlineto{\pgfqpoint{4.941195in}{2.862146in}}%
\pgfusepath{stroke,fill}%
\end{pgfscope}%
\begin{pgfscope}%
\pgfsetbuttcap%
\pgfsetmiterjoin%
\definecolor{currentfill}{rgb}{0.900000,0.900000,0.900000}%
\pgfsetfillcolor{currentfill}%
\pgfsetfillopacity{0.500000}%
\pgfsetlinewidth{1.003750pt}%
\definecolor{currentstroke}{rgb}{0.900000,0.900000,0.900000}%
\pgfsetstrokecolor{currentstroke}%
\pgfsetstrokeopacity{0.500000}%
\pgfsetdash{}{0pt}%
\pgfpathmoveto{\pgfqpoint{4.941195in}{1.930798in}}%
\pgfpathlineto{\pgfqpoint{3.734658in}{1.099507in}}%
\pgfpathlineto{\pgfqpoint{3.656001in}{2.033906in}}%
\pgfpathlineto{\pgfqpoint{4.941195in}{2.862146in}}%
\pgfusepath{stroke,fill}%
\end{pgfscope}%
\begin{pgfscope}%
\pgfsetbuttcap%
\pgfsetmiterjoin%
\definecolor{currentfill}{rgb}{0.925000,0.925000,0.925000}%
\pgfsetfillcolor{currentfill}%
\pgfsetfillopacity{0.500000}%
\pgfsetlinewidth{1.003750pt}%
\definecolor{currentstroke}{rgb}{0.925000,0.925000,0.925000}%
\pgfsetstrokecolor{currentstroke}%
\pgfsetstrokeopacity{0.500000}%
\pgfsetdash{}{0pt}%
\pgfpathmoveto{\pgfqpoint{4.941195in}{1.930798in}}%
\pgfpathlineto{\pgfqpoint{3.734658in}{1.099507in}}%
\pgfpathlineto{\pgfqpoint{4.941195in}{0.166408in}}%
\pgfpathlineto{\pgfqpoint{6.147731in}{1.099507in}}%
\pgfusepath{stroke,fill}%
\end{pgfscope}%
\begin{pgfscope}%
\pgfsetbuttcap%
\pgfsetroundjoin%
\pgfsetlinewidth{0.803000pt}%
\definecolor{currentstroke}{rgb}{0.690196,0.690196,0.690196}%
\pgfsetstrokecolor{currentstroke}%
\pgfsetdash{}{0pt}%
\pgfpathmoveto{\pgfqpoint{6.075116in}{1.043348in}}%
\pgfpathlineto{\pgfqpoint{4.868351in}{1.880609in}}%
\pgfpathlineto{\pgfqpoint{4.863861in}{2.812308in}}%
\pgfusepath{stroke}%
\end{pgfscope}%
\begin{pgfscope}%
\pgfsetbuttcap%
\pgfsetroundjoin%
\pgfsetlinewidth{0.803000pt}%
\definecolor{currentstroke}{rgb}{0.690196,0.690196,0.690196}%
\pgfsetstrokecolor{currentstroke}%
\pgfsetdash{}{0pt}%
\pgfpathmoveto{\pgfqpoint{5.845306in}{0.865621in}}%
\pgfpathlineto{\pgfqpoint{4.638013in}{1.721909in}}%
\pgfpathlineto{\pgfqpoint{4.619106in}{2.654576in}}%
\pgfusepath{stroke}%
\end{pgfscope}%
\begin{pgfscope}%
\pgfsetbuttcap%
\pgfsetroundjoin%
\pgfsetlinewidth{0.803000pt}%
\definecolor{currentstroke}{rgb}{0.690196,0.690196,0.690196}%
\pgfsetstrokecolor{currentstroke}%
\pgfsetdash{}{0pt}%
\pgfpathmoveto{\pgfqpoint{5.610093in}{0.683714in}}%
\pgfpathlineto{\pgfqpoint{4.402562in}{1.559686in}}%
\pgfpathlineto{\pgfqpoint{4.368575in}{2.493122in}}%
\pgfusepath{stroke}%
\end{pgfscope}%
\begin{pgfscope}%
\pgfsetbuttcap%
\pgfsetroundjoin%
\pgfsetlinewidth{0.803000pt}%
\definecolor{currentstroke}{rgb}{0.690196,0.690196,0.690196}%
\pgfsetstrokecolor{currentstroke}%
\pgfsetdash{}{0pt}%
\pgfpathmoveto{\pgfqpoint{5.369283in}{0.497478in}}%
\pgfpathlineto{\pgfqpoint{4.161826in}{1.393821in}}%
\pgfpathlineto{\pgfqpoint{4.112061in}{2.327813in}}%
\pgfusepath{stroke}%
\end{pgfscope}%
\begin{pgfscope}%
\pgfsetbuttcap%
\pgfsetroundjoin%
\pgfsetlinewidth{0.803000pt}%
\definecolor{currentstroke}{rgb}{0.690196,0.690196,0.690196}%
\pgfsetstrokecolor{currentstroke}%
\pgfsetdash{}{0pt}%
\pgfpathmoveto{\pgfqpoint{5.122674in}{0.306758in}}%
\pgfpathlineto{\pgfqpoint{3.915624in}{1.224191in}}%
\pgfpathlineto{\pgfqpoint{3.849347in}{2.158507in}}%
\pgfusepath{stroke}%
\end{pgfscope}%
\begin{pgfscope}%
\pgfsetbuttcap%
\pgfsetroundjoin%
\pgfsetlinewidth{0.803000pt}%
\definecolor{currentstroke}{rgb}{0.690196,0.690196,0.690196}%
\pgfsetstrokecolor{currentstroke}%
\pgfsetdash{}{0pt}%
\pgfpathmoveto{\pgfqpoint{5.018529in}{2.812308in}}%
\pgfpathlineto{\pgfqpoint{5.014039in}{1.880609in}}%
\pgfpathlineto{\pgfqpoint{3.807274in}{1.043348in}}%
\pgfusepath{stroke}%
\end{pgfscope}%
\begin{pgfscope}%
\pgfsetbuttcap%
\pgfsetroundjoin%
\pgfsetlinewidth{0.803000pt}%
\definecolor{currentstroke}{rgb}{0.690196,0.690196,0.690196}%
\pgfsetstrokecolor{currentstroke}%
\pgfsetdash{}{0pt}%
\pgfpathmoveto{\pgfqpoint{5.263284in}{2.654576in}}%
\pgfpathlineto{\pgfqpoint{5.244376in}{1.721909in}}%
\pgfpathlineto{\pgfqpoint{4.037084in}{0.865621in}}%
\pgfusepath{stroke}%
\end{pgfscope}%
\begin{pgfscope}%
\pgfsetbuttcap%
\pgfsetroundjoin%
\pgfsetlinewidth{0.803000pt}%
\definecolor{currentstroke}{rgb}{0.690196,0.690196,0.690196}%
\pgfsetstrokecolor{currentstroke}%
\pgfsetdash{}{0pt}%
\pgfpathmoveto{\pgfqpoint{5.513815in}{2.493122in}}%
\pgfpathlineto{\pgfqpoint{5.479827in}{1.559686in}}%
\pgfpathlineto{\pgfqpoint{4.272297in}{0.683714in}}%
\pgfusepath{stroke}%
\end{pgfscope}%
\begin{pgfscope}%
\pgfsetbuttcap%
\pgfsetroundjoin%
\pgfsetlinewidth{0.803000pt}%
\definecolor{currentstroke}{rgb}{0.690196,0.690196,0.690196}%
\pgfsetstrokecolor{currentstroke}%
\pgfsetdash{}{0pt}%
\pgfpathmoveto{\pgfqpoint{5.770329in}{2.327813in}}%
\pgfpathlineto{\pgfqpoint{5.720564in}{1.393821in}}%
\pgfpathlineto{\pgfqpoint{4.513107in}{0.497478in}}%
\pgfusepath{stroke}%
\end{pgfscope}%
\begin{pgfscope}%
\pgfsetbuttcap%
\pgfsetroundjoin%
\pgfsetlinewidth{0.803000pt}%
\definecolor{currentstroke}{rgb}{0.690196,0.690196,0.690196}%
\pgfsetstrokecolor{currentstroke}%
\pgfsetdash{}{0pt}%
\pgfpathmoveto{\pgfqpoint{6.033043in}{2.158507in}}%
\pgfpathlineto{\pgfqpoint{5.966766in}{1.224191in}}%
\pgfpathlineto{\pgfqpoint{4.759716in}{0.306758in}}%
\pgfusepath{stroke}%
\end{pgfscope}%
\begin{pgfscope}%
\pgfsetbuttcap%
\pgfsetroundjoin%
\pgfsetlinewidth{0.803000pt}%
\definecolor{currentstroke}{rgb}{0.690196,0.690196,0.690196}%
\pgfsetstrokecolor{currentstroke}%
\pgfsetdash{}{0pt}%
\pgfpathmoveto{\pgfqpoint{3.729941in}{1.155548in}}%
\pgfpathlineto{\pgfqpoint{4.941195in}{1.986842in}}%
\pgfpathlineto{\pgfqpoint{6.152449in}{1.155548in}}%
\pgfusepath{stroke}%
\end{pgfscope}%
\begin{pgfscope}%
\pgfsetbuttcap%
\pgfsetroundjoin%
\pgfsetlinewidth{0.803000pt}%
\definecolor{currentstroke}{rgb}{0.690196,0.690196,0.690196}%
\pgfsetstrokecolor{currentstroke}%
\pgfsetdash{}{0pt}%
\pgfpathmoveto{\pgfqpoint{3.714997in}{1.333067in}}%
\pgfpathlineto{\pgfqpoint{4.941195in}{2.164215in}}%
\pgfpathlineto{\pgfqpoint{6.167392in}{1.333067in}}%
\pgfusepath{stroke}%
\end{pgfscope}%
\begin{pgfscope}%
\pgfsetbuttcap%
\pgfsetroundjoin%
\pgfsetlinewidth{0.803000pt}%
\definecolor{currentstroke}{rgb}{0.690196,0.690196,0.690196}%
\pgfsetstrokecolor{currentstroke}%
\pgfsetdash{}{0pt}%
\pgfpathmoveto{\pgfqpoint{3.699681in}{1.515021in}}%
\pgfpathlineto{\pgfqpoint{4.941195in}{2.345771in}}%
\pgfpathlineto{\pgfqpoint{6.182709in}{1.515021in}}%
\pgfusepath{stroke}%
\end{pgfscope}%
\begin{pgfscope}%
\pgfsetbuttcap%
\pgfsetroundjoin%
\pgfsetlinewidth{0.803000pt}%
\definecolor{currentstroke}{rgb}{0.690196,0.690196,0.690196}%
\pgfsetstrokecolor{currentstroke}%
\pgfsetdash{}{0pt}%
\pgfpathmoveto{\pgfqpoint{3.683976in}{1.701578in}}%
\pgfpathlineto{\pgfqpoint{4.941195in}{2.531660in}}%
\pgfpathlineto{\pgfqpoint{6.198413in}{1.701578in}}%
\pgfusepath{stroke}%
\end{pgfscope}%
\begin{pgfscope}%
\pgfsetbuttcap%
\pgfsetroundjoin%
\pgfsetlinewidth{0.803000pt}%
\definecolor{currentstroke}{rgb}{0.690196,0.690196,0.690196}%
\pgfsetstrokecolor{currentstroke}%
\pgfsetdash{}{0pt}%
\pgfpathmoveto{\pgfqpoint{3.667870in}{1.892915in}}%
\pgfpathlineto{\pgfqpoint{4.941195in}{2.722038in}}%
\pgfpathlineto{\pgfqpoint{6.214520in}{1.892915in}}%
\pgfusepath{stroke}%
\end{pgfscope}%
\begin{pgfscope}%
\pgfsetrectcap%
\pgfsetroundjoin%
\pgfsetlinewidth{0.803000pt}%
\definecolor{currentstroke}{rgb}{0.000000,0.000000,0.000000}%
\pgfsetstrokecolor{currentstroke}%
\pgfsetdash{}{0pt}%
\pgfpathmoveto{\pgfqpoint{6.147731in}{1.099507in}}%
\pgfpathlineto{\pgfqpoint{4.941195in}{0.166408in}}%
\pgfusepath{stroke}%
\end{pgfscope}%
\begin{pgfscope}%
\pgfsetrectcap%
\pgfsetroundjoin%
\pgfsetlinewidth{0.803000pt}%
\definecolor{currentstroke}{rgb}{0.000000,0.000000,0.000000}%
\pgfsetstrokecolor{currentstroke}%
\pgfsetdash{}{0pt}%
\pgfpathmoveto{\pgfqpoint{6.064907in}{1.050431in}}%
\pgfpathlineto{\pgfqpoint{6.095561in}{1.029163in}}%
\pgfusepath{stroke}%
\end{pgfscope}%
\begin{pgfscope}%
\pgfsetrectcap%
\pgfsetroundjoin%
\pgfsetlinewidth{0.803000pt}%
\definecolor{currentstroke}{rgb}{0.000000,0.000000,0.000000}%
\pgfsetstrokecolor{currentstroke}%
\pgfsetdash{}{0pt}%
\pgfpathmoveto{\pgfqpoint{5.835087in}{0.872869in}}%
\pgfpathlineto{\pgfqpoint{5.865774in}{0.851104in}}%
\pgfusepath{stroke}%
\end{pgfscope}%
\begin{pgfscope}%
\pgfsetrectcap%
\pgfsetroundjoin%
\pgfsetlinewidth{0.803000pt}%
\definecolor{currentstroke}{rgb}{0.000000,0.000000,0.000000}%
\pgfsetstrokecolor{currentstroke}%
\pgfsetdash{}{0pt}%
\pgfpathmoveto{\pgfqpoint{5.599865in}{0.691133in}}%
\pgfpathlineto{\pgfqpoint{5.630578in}{0.668853in}}%
\pgfusepath{stroke}%
\end{pgfscope}%
\begin{pgfscope}%
\pgfsetrectcap%
\pgfsetroundjoin%
\pgfsetlinewidth{0.803000pt}%
\definecolor{currentstroke}{rgb}{0.000000,0.000000,0.000000}%
\pgfsetstrokecolor{currentstroke}%
\pgfsetdash{}{0pt}%
\pgfpathmoveto{\pgfqpoint{5.359049in}{0.505076in}}%
\pgfpathlineto{\pgfqpoint{5.389781in}{0.482262in}}%
\pgfusepath{stroke}%
\end{pgfscope}%
\begin{pgfscope}%
\pgfsetrectcap%
\pgfsetroundjoin%
\pgfsetlinewidth{0.803000pt}%
\definecolor{currentstroke}{rgb}{0.000000,0.000000,0.000000}%
\pgfsetstrokecolor{currentstroke}%
\pgfsetdash{}{0pt}%
\pgfpathmoveto{\pgfqpoint{5.112437in}{0.314540in}}%
\pgfpathlineto{\pgfqpoint{5.143179in}{0.291173in}}%
\pgfusepath{stroke}%
\end{pgfscope}%
\begin{pgfscope}%
\definecolor{textcolor}{rgb}{0.000000,0.000000,0.000000}%
\pgfsetstrokecolor{textcolor}%
\pgfsetfillcolor{textcolor}%
\pgftext[x=5.840237in,y=0.241958in,,]{\color{textcolor}{\rmfamily\fontsize{14.000000}{16.800000}\selectfont\catcode`\^=\active\def^{\ifmmode\sp\else\^{}\fi}\catcode`\%=\active\def%{\%}f1}}%
\end{pgfscope}%
\begin{pgfscope}%
\pgfsetrectcap%
\pgfsetroundjoin%
\pgfsetlinewidth{0.803000pt}%
\definecolor{currentstroke}{rgb}{0.000000,0.000000,0.000000}%
\pgfsetstrokecolor{currentstroke}%
\pgfsetdash{}{0pt}%
\pgfpathmoveto{\pgfqpoint{3.734658in}{1.099507in}}%
\pgfpathlineto{\pgfqpoint{4.941195in}{0.166408in}}%
\pgfusepath{stroke}%
\end{pgfscope}%
\begin{pgfscope}%
\pgfsetrectcap%
\pgfsetroundjoin%
\pgfsetlinewidth{0.803000pt}%
\definecolor{currentstroke}{rgb}{0.000000,0.000000,0.000000}%
\pgfsetstrokecolor{currentstroke}%
\pgfsetdash{}{0pt}%
\pgfpathmoveto{\pgfqpoint{3.817483in}{1.050431in}}%
\pgfpathlineto{\pgfqpoint{3.786829in}{1.029163in}}%
\pgfusepath{stroke}%
\end{pgfscope}%
\begin{pgfscope}%
\pgfsetrectcap%
\pgfsetroundjoin%
\pgfsetlinewidth{0.803000pt}%
\definecolor{currentstroke}{rgb}{0.000000,0.000000,0.000000}%
\pgfsetstrokecolor{currentstroke}%
\pgfsetdash{}{0pt}%
\pgfpathmoveto{\pgfqpoint{4.047303in}{0.872869in}}%
\pgfpathlineto{\pgfqpoint{4.016616in}{0.851104in}}%
\pgfusepath{stroke}%
\end{pgfscope}%
\begin{pgfscope}%
\pgfsetrectcap%
\pgfsetroundjoin%
\pgfsetlinewidth{0.803000pt}%
\definecolor{currentstroke}{rgb}{0.000000,0.000000,0.000000}%
\pgfsetstrokecolor{currentstroke}%
\pgfsetdash{}{0pt}%
\pgfpathmoveto{\pgfqpoint{4.282525in}{0.691133in}}%
\pgfpathlineto{\pgfqpoint{4.251812in}{0.668853in}}%
\pgfusepath{stroke}%
\end{pgfscope}%
\begin{pgfscope}%
\pgfsetrectcap%
\pgfsetroundjoin%
\pgfsetlinewidth{0.803000pt}%
\definecolor{currentstroke}{rgb}{0.000000,0.000000,0.000000}%
\pgfsetstrokecolor{currentstroke}%
\pgfsetdash{}{0pt}%
\pgfpathmoveto{\pgfqpoint{4.523341in}{0.505076in}}%
\pgfpathlineto{\pgfqpoint{4.492609in}{0.482262in}}%
\pgfusepath{stroke}%
\end{pgfscope}%
\begin{pgfscope}%
\pgfsetrectcap%
\pgfsetroundjoin%
\pgfsetlinewidth{0.803000pt}%
\definecolor{currentstroke}{rgb}{0.000000,0.000000,0.000000}%
\pgfsetstrokecolor{currentstroke}%
\pgfsetdash{}{0pt}%
\pgfpathmoveto{\pgfqpoint{4.769953in}{0.314540in}}%
\pgfpathlineto{\pgfqpoint{4.739210in}{0.291173in}}%
\pgfusepath{stroke}%
\end{pgfscope}%
\begin{pgfscope}%
\definecolor{textcolor}{rgb}{0.000000,0.000000,0.000000}%
\pgfsetstrokecolor{textcolor}%
\pgfsetfillcolor{textcolor}%
\pgftext[x=4.042153in,y=0.241958in,,]{\color{textcolor}{\rmfamily\fontsize{14.000000}{16.800000}\selectfont\catcode`\^=\active\def^{\ifmmode\sp\else\^{}\fi}\catcode`\%=\active\def%{\%}f2}}%
\end{pgfscope}%
\begin{pgfscope}%
\pgfsetrectcap%
\pgfsetroundjoin%
\pgfsetlinewidth{0.803000pt}%
\definecolor{currentstroke}{rgb}{0.000000,0.000000,0.000000}%
\pgfsetstrokecolor{currentstroke}%
\pgfsetdash{}{0pt}%
\pgfpathmoveto{\pgfqpoint{3.734658in}{1.099507in}}%
\pgfpathlineto{\pgfqpoint{3.656001in}{2.033906in}}%
\pgfusepath{stroke}%
\end{pgfscope}%
\begin{pgfscope}%
\pgfsetrectcap%
\pgfsetroundjoin%
\pgfsetlinewidth{0.803000pt}%
\definecolor{currentstroke}{rgb}{0.000000,0.000000,0.000000}%
\pgfsetstrokecolor{currentstroke}%
\pgfsetdash{}{0pt}%
\pgfpathmoveto{\pgfqpoint{3.740188in}{1.162580in}}%
\pgfpathlineto{\pgfqpoint{3.709419in}{1.141464in}}%
\pgfusepath{stroke}%
\end{pgfscope}%
\begin{pgfscope}%
\pgfsetrectcap%
\pgfsetroundjoin%
\pgfsetlinewidth{0.803000pt}%
\definecolor{currentstroke}{rgb}{0.000000,0.000000,0.000000}%
\pgfsetstrokecolor{currentstroke}%
\pgfsetdash{}{0pt}%
\pgfpathmoveto{\pgfqpoint{3.725377in}{1.340103in}}%
\pgfpathlineto{\pgfqpoint{3.694208in}{1.318976in}}%
\pgfusepath{stroke}%
\end{pgfscope}%
\begin{pgfscope}%
\pgfsetrectcap%
\pgfsetroundjoin%
\pgfsetlinewidth{0.803000pt}%
\definecolor{currentstroke}{rgb}{0.000000,0.000000,0.000000}%
\pgfsetstrokecolor{currentstroke}%
\pgfsetdash{}{0pt}%
\pgfpathmoveto{\pgfqpoint{3.710198in}{1.522058in}}%
\pgfpathlineto{\pgfqpoint{3.678617in}{1.500926in}}%
\pgfusepath{stroke}%
\end{pgfscope}%
\begin{pgfscope}%
\pgfsetrectcap%
\pgfsetroundjoin%
\pgfsetlinewidth{0.803000pt}%
\definecolor{currentstroke}{rgb}{0.000000,0.000000,0.000000}%
\pgfsetstrokecolor{currentstroke}%
\pgfsetdash{}{0pt}%
\pgfpathmoveto{\pgfqpoint{3.694634in}{1.708614in}}%
\pgfpathlineto{\pgfqpoint{3.662631in}{1.687484in}}%
\pgfusepath{stroke}%
\end{pgfscope}%
\begin{pgfscope}%
\pgfsetrectcap%
\pgfsetroundjoin%
\pgfsetlinewidth{0.803000pt}%
\definecolor{currentstroke}{rgb}{0.000000,0.000000,0.000000}%
\pgfsetstrokecolor{currentstroke}%
\pgfsetdash{}{0pt}%
\pgfpathmoveto{\pgfqpoint{3.678672in}{1.899948in}}%
\pgfpathlineto{\pgfqpoint{3.646235in}{1.878827in}}%
\pgfusepath{stroke}%
\end{pgfscope}%
\begin{pgfscope}%
\definecolor{textcolor}{rgb}{0.000000,0.000000,0.000000}%
\pgfsetstrokecolor{textcolor}%
\pgfsetfillcolor{textcolor}%
\pgftext[x=3.138409in,y=1.551958in,,]{\color{textcolor}{\rmfamily\fontsize{14.000000}{16.800000}\selectfont\catcode`\^=\active\def^{\ifmmode\sp\else\^{}\fi}\catcode`\%=\active\def%{\%}f3}}%
\end{pgfscope}%
\begin{pgfscope}%
\pgfpathrectangle{\pgfqpoint{3.536584in}{0.147348in}}{\pgfqpoint{2.735294in}{2.735294in}}%
\pgfusepath{clip}%
\pgfsetbuttcap%
\pgfsetroundjoin%
\definecolor{currentfill}{rgb}{0.050070,0.192203,0.290728}%
\pgfsetfillcolor{currentfill}%
\pgfsetlinewidth{0.000000pt}%
\definecolor{currentstroke}{rgb}{0.000000,0.000000,0.000000}%
\pgfsetstrokecolor{currentstroke}%
\pgfsetdash{}{0pt}%
\pgfpathmoveto{\pgfqpoint{5.897888in}{1.291641in}}%
\pgfpathlineto{\pgfqpoint{5.810696in}{1.165011in}}%
\pgfpathlineto{\pgfqpoint{5.897793in}{1.225003in}}%
\pgfpathlineto{\pgfqpoint{5.897888in}{1.291641in}}%
\pgfpathclose%
\pgfusepath{fill}%
\end{pgfscope}%
\begin{pgfscope}%
\pgfpathrectangle{\pgfqpoint{3.536584in}{0.147348in}}{\pgfqpoint{2.735294in}{2.735294in}}%
\pgfusepath{clip}%
\pgfsetbuttcap%
\pgfsetroundjoin%
\definecolor{currentfill}{rgb}{0.050070,0.192203,0.290728}%
\pgfsetfillcolor{currentfill}%
\pgfsetlinewidth{0.000000pt}%
\definecolor{currentstroke}{rgb}{0.000000,0.000000,0.000000}%
\pgfsetstrokecolor{currentstroke}%
\pgfsetdash{}{0pt}%
\pgfpathmoveto{\pgfqpoint{4.071693in}{1.165011in}}%
\pgfpathlineto{\pgfqpoint{3.984502in}{1.291641in}}%
\pgfpathlineto{\pgfqpoint{3.984596in}{1.225003in}}%
\pgfpathlineto{\pgfqpoint{4.071693in}{1.165011in}}%
\pgfpathclose%
\pgfusepath{fill}%
\end{pgfscope}%
\begin{pgfscope}%
\pgfpathrectangle{\pgfqpoint{3.536584in}{0.147348in}}{\pgfqpoint{2.735294in}{2.735294in}}%
\pgfusepath{clip}%
\pgfsetbuttcap%
\pgfsetroundjoin%
\definecolor{currentfill}{rgb}{0.090605,0.347808,0.526096}%
\pgfsetfillcolor{currentfill}%
\pgfsetlinewidth{0.000000pt}%
\definecolor{currentstroke}{rgb}{0.000000,0.000000,0.000000}%
\pgfsetstrokecolor{currentstroke}%
\pgfsetdash{}{0pt}%
\pgfpathmoveto{\pgfqpoint{4.854003in}{2.561465in}}%
\pgfpathlineto{\pgfqpoint{5.028387in}{2.561465in}}%
\pgfpathlineto{\pgfqpoint{4.941195in}{2.621838in}}%
\pgfpathlineto{\pgfqpoint{4.854003in}{2.561465in}}%
\pgfpathclose%
\pgfusepath{fill}%
\end{pgfscope}%
\begin{pgfscope}%
\pgfpathrectangle{\pgfqpoint{3.536584in}{0.147348in}}{\pgfqpoint{2.735294in}{2.735294in}}%
\pgfusepath{clip}%
\pgfsetbuttcap%
\pgfsetroundjoin%
\definecolor{currentfill}{rgb}{0.047548,0.182523,0.276086}%
\pgfsetfillcolor{currentfill}%
\pgfsetlinewidth{0.000000pt}%
\definecolor{currentstroke}{rgb}{0.000000,0.000000,0.000000}%
\pgfsetstrokecolor{currentstroke}%
\pgfsetdash{}{0pt}%
\pgfpathmoveto{\pgfqpoint{5.810696in}{1.165011in}}%
\pgfpathlineto{\pgfqpoint{5.801292in}{1.232896in}}%
\pgfpathlineto{\pgfqpoint{5.697728in}{1.101844in}}%
\pgfpathlineto{\pgfqpoint{5.810696in}{1.165011in}}%
\pgfpathclose%
\pgfusepath{fill}%
\end{pgfscope}%
\begin{pgfscope}%
\pgfpathrectangle{\pgfqpoint{3.536584in}{0.147348in}}{\pgfqpoint{2.735294in}{2.735294in}}%
\pgfusepath{clip}%
\pgfsetbuttcap%
\pgfsetroundjoin%
\definecolor{currentfill}{rgb}{0.047548,0.182523,0.276086}%
\pgfsetfillcolor{currentfill}%
\pgfsetlinewidth{0.000000pt}%
\definecolor{currentstroke}{rgb}{0.000000,0.000000,0.000000}%
\pgfsetstrokecolor{currentstroke}%
\pgfsetdash{}{0pt}%
\pgfpathmoveto{\pgfqpoint{4.184662in}{1.101844in}}%
\pgfpathlineto{\pgfqpoint{4.081098in}{1.232896in}}%
\pgfpathlineto{\pgfqpoint{4.071693in}{1.165011in}}%
\pgfpathlineto{\pgfqpoint{4.184662in}{1.101844in}}%
\pgfpathclose%
\pgfusepath{fill}%
\end{pgfscope}%
\begin{pgfscope}%
\pgfpathrectangle{\pgfqpoint{3.536584in}{0.147348in}}{\pgfqpoint{2.735294in}{2.735294in}}%
\pgfusepath{clip}%
\pgfsetbuttcap%
\pgfsetroundjoin%
\definecolor{currentfill}{rgb}{0.048960,0.187944,0.284285}%
\pgfsetfillcolor{currentfill}%
\pgfsetlinewidth{0.000000pt}%
\definecolor{currentstroke}{rgb}{0.000000,0.000000,0.000000}%
\pgfsetstrokecolor{currentstroke}%
\pgfsetdash{}{0pt}%
\pgfpathmoveto{\pgfqpoint{3.984502in}{1.291641in}}%
\pgfpathlineto{\pgfqpoint{4.071693in}{1.165011in}}%
\pgfpathlineto{\pgfqpoint{4.070905in}{1.618388in}}%
\pgfpathlineto{\pgfqpoint{3.984502in}{1.291641in}}%
\pgfpathclose%
\pgfusepath{fill}%
\end{pgfscope}%
\begin{pgfscope}%
\pgfpathrectangle{\pgfqpoint{3.536584in}{0.147348in}}{\pgfqpoint{2.735294in}{2.735294in}}%
\pgfusepath{clip}%
\pgfsetbuttcap%
\pgfsetroundjoin%
\definecolor{currentfill}{rgb}{0.048960,0.187944,0.284285}%
\pgfsetfillcolor{currentfill}%
\pgfsetlinewidth{0.000000pt}%
\definecolor{currentstroke}{rgb}{0.000000,0.000000,0.000000}%
\pgfsetstrokecolor{currentstroke}%
\pgfsetdash{}{0pt}%
\pgfpathmoveto{\pgfqpoint{5.897888in}{1.291641in}}%
\pgfpathlineto{\pgfqpoint{5.811485in}{1.618388in}}%
\pgfpathlineto{\pgfqpoint{5.810696in}{1.165011in}}%
\pgfpathlineto{\pgfqpoint{5.897888in}{1.291641in}}%
\pgfpathclose%
\pgfusepath{fill}%
\end{pgfscope}%
\begin{pgfscope}%
\pgfpathrectangle{\pgfqpoint{3.536584in}{0.147348in}}{\pgfqpoint{2.735294in}{2.735294in}}%
\pgfusepath{clip}%
\pgfsetbuttcap%
\pgfsetroundjoin%
\definecolor{currentfill}{rgb}{0.070254,0.269685,0.407928}%
\pgfsetfillcolor{currentfill}%
\pgfsetlinewidth{0.000000pt}%
\definecolor{currentstroke}{rgb}{0.000000,0.000000,0.000000}%
\pgfsetstrokecolor{currentstroke}%
\pgfsetdash{}{0pt}%
\pgfpathmoveto{\pgfqpoint{5.801292in}{1.232896in}}%
\pgfpathlineto{\pgfqpoint{5.810696in}{1.165011in}}%
\pgfpathlineto{\pgfqpoint{5.811485in}{1.618388in}}%
\pgfpathlineto{\pgfqpoint{5.801292in}{1.232896in}}%
\pgfpathclose%
\pgfusepath{fill}%
\end{pgfscope}%
\begin{pgfscope}%
\pgfpathrectangle{\pgfqpoint{3.536584in}{0.147348in}}{\pgfqpoint{2.735294in}{2.735294in}}%
\pgfusepath{clip}%
\pgfsetbuttcap%
\pgfsetroundjoin%
\definecolor{currentfill}{rgb}{0.070254,0.269685,0.407928}%
\pgfsetfillcolor{currentfill}%
\pgfsetlinewidth{0.000000pt}%
\definecolor{currentstroke}{rgb}{0.000000,0.000000,0.000000}%
\pgfsetstrokecolor{currentstroke}%
\pgfsetdash{}{0pt}%
\pgfpathmoveto{\pgfqpoint{4.070905in}{1.618388in}}%
\pgfpathlineto{\pgfqpoint{4.071693in}{1.165011in}}%
\pgfpathlineto{\pgfqpoint{4.081098in}{1.232896in}}%
\pgfpathlineto{\pgfqpoint{4.070905in}{1.618388in}}%
\pgfpathclose%
\pgfusepath{fill}%
\end{pgfscope}%
\begin{pgfscope}%
\pgfpathrectangle{\pgfqpoint{3.536584in}{0.147348in}}{\pgfqpoint{2.735294in}{2.735294in}}%
\pgfusepath{clip}%
\pgfsetbuttcap%
\pgfsetroundjoin%
\definecolor{currentfill}{rgb}{0.044978,0.172658,0.261163}%
\pgfsetfillcolor{currentfill}%
\pgfsetlinewidth{0.000000pt}%
\definecolor{currentstroke}{rgb}{0.000000,0.000000,0.000000}%
\pgfsetstrokecolor{currentstroke}%
\pgfsetdash{}{0pt}%
\pgfpathmoveto{\pgfqpoint{4.328767in}{1.039603in}}%
\pgfpathlineto{\pgfqpoint{4.209060in}{1.171211in}}%
\pgfpathlineto{\pgfqpoint{4.184662in}{1.101844in}}%
\pgfpathlineto{\pgfqpoint{4.328767in}{1.039603in}}%
\pgfpathclose%
\pgfusepath{fill}%
\end{pgfscope}%
\begin{pgfscope}%
\pgfpathrectangle{\pgfqpoint{3.536584in}{0.147348in}}{\pgfqpoint{2.735294in}{2.735294in}}%
\pgfusepath{clip}%
\pgfsetbuttcap%
\pgfsetroundjoin%
\definecolor{currentfill}{rgb}{0.044978,0.172658,0.261163}%
\pgfsetfillcolor{currentfill}%
\pgfsetlinewidth{0.000000pt}%
\definecolor{currentstroke}{rgb}{0.000000,0.000000,0.000000}%
\pgfsetstrokecolor{currentstroke}%
\pgfsetdash{}{0pt}%
\pgfpathmoveto{\pgfqpoint{5.697728in}{1.101844in}}%
\pgfpathlineto{\pgfqpoint{5.673330in}{1.171211in}}%
\pgfpathlineto{\pgfqpoint{5.553623in}{1.039603in}}%
\pgfpathlineto{\pgfqpoint{5.697728in}{1.101844in}}%
\pgfpathclose%
\pgfusepath{fill}%
\end{pgfscope}%
\begin{pgfscope}%
\pgfpathrectangle{\pgfqpoint{3.536584in}{0.147348in}}{\pgfqpoint{2.735294in}{2.735294in}}%
\pgfusepath{clip}%
\pgfsetbuttcap%
\pgfsetroundjoin%
\definecolor{currentfill}{rgb}{0.081954,0.314596,0.475860}%
\pgfsetfillcolor{currentfill}%
\pgfsetlinewidth{0.000000pt}%
\definecolor{currentstroke}{rgb}{0.000000,0.000000,0.000000}%
\pgfsetstrokecolor{currentstroke}%
\pgfsetdash{}{0pt}%
\pgfpathmoveto{\pgfqpoint{5.028387in}{2.561465in}}%
\pgfpathlineto{\pgfqpoint{4.854003in}{2.561465in}}%
\pgfpathlineto{\pgfqpoint{4.816407in}{2.124855in}}%
\pgfpathlineto{\pgfqpoint{5.028387in}{2.561465in}}%
\pgfpathclose%
\pgfusepath{fill}%
\end{pgfscope}%
\begin{pgfscope}%
\pgfpathrectangle{\pgfqpoint{3.536584in}{0.147348in}}{\pgfqpoint{2.735294in}{2.735294in}}%
\pgfusepath{clip}%
\pgfsetbuttcap%
\pgfsetroundjoin%
\definecolor{currentfill}{rgb}{0.047247,0.181368,0.274339}%
\pgfsetfillcolor{currentfill}%
\pgfsetlinewidth{0.000000pt}%
\definecolor{currentstroke}{rgb}{0.000000,0.000000,0.000000}%
\pgfsetstrokecolor{currentstroke}%
\pgfsetdash{}{0pt}%
\pgfpathmoveto{\pgfqpoint{5.664465in}{1.589448in}}%
\pgfpathlineto{\pgfqpoint{5.697728in}{1.101844in}}%
\pgfpathlineto{\pgfqpoint{5.801292in}{1.232896in}}%
\pgfpathlineto{\pgfqpoint{5.664465in}{1.589448in}}%
\pgfpathclose%
\pgfusepath{fill}%
\end{pgfscope}%
\begin{pgfscope}%
\pgfpathrectangle{\pgfqpoint{3.536584in}{0.147348in}}{\pgfqpoint{2.735294in}{2.735294in}}%
\pgfusepath{clip}%
\pgfsetbuttcap%
\pgfsetroundjoin%
\definecolor{currentfill}{rgb}{0.047247,0.181368,0.274339}%
\pgfsetfillcolor{currentfill}%
\pgfsetlinewidth{0.000000pt}%
\definecolor{currentstroke}{rgb}{0.000000,0.000000,0.000000}%
\pgfsetstrokecolor{currentstroke}%
\pgfsetdash{}{0pt}%
\pgfpathmoveto{\pgfqpoint{4.081098in}{1.232896in}}%
\pgfpathlineto{\pgfqpoint{4.184662in}{1.101844in}}%
\pgfpathlineto{\pgfqpoint{4.217925in}{1.589448in}}%
\pgfpathlineto{\pgfqpoint{4.081098in}{1.232896in}}%
\pgfpathclose%
\pgfusepath{fill}%
\end{pgfscope}%
\begin{pgfscope}%
\pgfpathrectangle{\pgfqpoint{3.536584in}{0.147348in}}{\pgfqpoint{2.735294in}{2.735294in}}%
\pgfusepath{clip}%
\pgfsetbuttcap%
\pgfsetroundjoin%
\definecolor{currentfill}{rgb}{0.067179,0.257880,0.390071}%
\pgfsetfillcolor{currentfill}%
\pgfsetlinewidth{0.000000pt}%
\definecolor{currentstroke}{rgb}{0.000000,0.000000,0.000000}%
\pgfsetstrokecolor{currentstroke}%
\pgfsetdash{}{0pt}%
\pgfpathmoveto{\pgfqpoint{4.217925in}{1.589448in}}%
\pgfpathlineto{\pgfqpoint{4.184662in}{1.101844in}}%
\pgfpathlineto{\pgfqpoint{4.209060in}{1.171211in}}%
\pgfpathlineto{\pgfqpoint{4.217925in}{1.589448in}}%
\pgfpathclose%
\pgfusepath{fill}%
\end{pgfscope}%
\begin{pgfscope}%
\pgfpathrectangle{\pgfqpoint{3.536584in}{0.147348in}}{\pgfqpoint{2.735294in}{2.735294in}}%
\pgfusepath{clip}%
\pgfsetbuttcap%
\pgfsetroundjoin%
\definecolor{currentfill}{rgb}{0.067179,0.257880,0.390071}%
\pgfsetfillcolor{currentfill}%
\pgfsetlinewidth{0.000000pt}%
\definecolor{currentstroke}{rgb}{0.000000,0.000000,0.000000}%
\pgfsetstrokecolor{currentstroke}%
\pgfsetdash{}{0pt}%
\pgfpathmoveto{\pgfqpoint{5.673330in}{1.171211in}}%
\pgfpathlineto{\pgfqpoint{5.697728in}{1.101844in}}%
\pgfpathlineto{\pgfqpoint{5.664465in}{1.589448in}}%
\pgfpathlineto{\pgfqpoint{5.673330in}{1.171211in}}%
\pgfpathclose%
\pgfusepath{fill}%
\end{pgfscope}%
\begin{pgfscope}%
\pgfpathrectangle{\pgfqpoint{3.536584in}{0.147348in}}{\pgfqpoint{2.735294in}{2.735294in}}%
\pgfusepath{clip}%
\pgfsetbuttcap%
\pgfsetroundjoin%
\definecolor{currentfill}{rgb}{0.042579,0.163449,0.247234}%
\pgfsetfillcolor{currentfill}%
\pgfsetlinewidth{0.000000pt}%
\definecolor{currentstroke}{rgb}{0.000000,0.000000,0.000000}%
\pgfsetstrokecolor{currentstroke}%
\pgfsetdash{}{0pt}%
\pgfpathmoveto{\pgfqpoint{4.328767in}{1.039603in}}%
\pgfpathlineto{\pgfqpoint{4.506598in}{0.985051in}}%
\pgfpathlineto{\pgfqpoint{4.374765in}{1.111775in}}%
\pgfpathlineto{\pgfqpoint{4.328767in}{1.039603in}}%
\pgfpathclose%
\pgfusepath{fill}%
\end{pgfscope}%
\begin{pgfscope}%
\pgfpathrectangle{\pgfqpoint{3.536584in}{0.147348in}}{\pgfqpoint{2.735294in}{2.735294in}}%
\pgfusepath{clip}%
\pgfsetbuttcap%
\pgfsetroundjoin%
\definecolor{currentfill}{rgb}{0.042579,0.163449,0.247234}%
\pgfsetfillcolor{currentfill}%
\pgfsetlinewidth{0.000000pt}%
\definecolor{currentstroke}{rgb}{0.000000,0.000000,0.000000}%
\pgfsetstrokecolor{currentstroke}%
\pgfsetdash{}{0pt}%
\pgfpathmoveto{\pgfqpoint{5.507624in}{1.111775in}}%
\pgfpathlineto{\pgfqpoint{5.375791in}{0.985051in}}%
\pgfpathlineto{\pgfqpoint{5.553623in}{1.039603in}}%
\pgfpathlineto{\pgfqpoint{5.507624in}{1.111775in}}%
\pgfpathclose%
\pgfusepath{fill}%
\end{pgfscope}%
\begin{pgfscope}%
\pgfpathrectangle{\pgfqpoint{3.536584in}{0.147348in}}{\pgfqpoint{2.735294in}{2.735294in}}%
\pgfusepath{clip}%
\pgfsetbuttcap%
\pgfsetroundjoin%
\definecolor{currentfill}{rgb}{0.052493,0.201505,0.304798}%
\pgfsetfillcolor{currentfill}%
\pgfsetlinewidth{0.000000pt}%
\definecolor{currentstroke}{rgb}{0.000000,0.000000,0.000000}%
\pgfsetstrokecolor{currentstroke}%
\pgfsetdash{}{0pt}%
\pgfpathmoveto{\pgfqpoint{5.664465in}{1.589448in}}%
\pgfpathlineto{\pgfqpoint{5.801292in}{1.232896in}}%
\pgfpathlineto{\pgfqpoint{5.811485in}{1.618388in}}%
\pgfpathlineto{\pgfqpoint{5.664465in}{1.589448in}}%
\pgfpathclose%
\pgfusepath{fill}%
\end{pgfscope}%
\begin{pgfscope}%
\pgfpathrectangle{\pgfqpoint{3.536584in}{0.147348in}}{\pgfqpoint{2.735294in}{2.735294in}}%
\pgfusepath{clip}%
\pgfsetbuttcap%
\pgfsetroundjoin%
\definecolor{currentfill}{rgb}{0.052493,0.201505,0.304798}%
\pgfsetfillcolor{currentfill}%
\pgfsetlinewidth{0.000000pt}%
\definecolor{currentstroke}{rgb}{0.000000,0.000000,0.000000}%
\pgfsetstrokecolor{currentstroke}%
\pgfsetdash{}{0pt}%
\pgfpathmoveto{\pgfqpoint{4.070905in}{1.618388in}}%
\pgfpathlineto{\pgfqpoint{4.081098in}{1.232896in}}%
\pgfpathlineto{\pgfqpoint{4.217925in}{1.589448in}}%
\pgfpathlineto{\pgfqpoint{4.070905in}{1.618388in}}%
\pgfpathclose%
\pgfusepath{fill}%
\end{pgfscope}%
\begin{pgfscope}%
\pgfpathrectangle{\pgfqpoint{3.536584in}{0.147348in}}{\pgfqpoint{2.735294in}{2.735294in}}%
\pgfusepath{clip}%
\pgfsetbuttcap%
\pgfsetroundjoin%
\definecolor{currentfill}{rgb}{0.082280,0.315849,0.477755}%
\pgfsetfillcolor{currentfill}%
\pgfsetlinewidth{0.000000pt}%
\definecolor{currentstroke}{rgb}{0.000000,0.000000,0.000000}%
\pgfsetstrokecolor{currentstroke}%
\pgfsetdash{}{0pt}%
\pgfpathmoveto{\pgfqpoint{5.376890in}{2.253326in}}%
\pgfpathlineto{\pgfqpoint{5.028387in}{2.561465in}}%
\pgfpathlineto{\pgfqpoint{5.065982in}{2.124855in}}%
\pgfpathlineto{\pgfqpoint{5.376890in}{2.253326in}}%
\pgfpathclose%
\pgfusepath{fill}%
\end{pgfscope}%
\begin{pgfscope}%
\pgfpathrectangle{\pgfqpoint{3.536584in}{0.147348in}}{\pgfqpoint{2.735294in}{2.735294in}}%
\pgfusepath{clip}%
\pgfsetbuttcap%
\pgfsetroundjoin%
\definecolor{currentfill}{rgb}{0.082280,0.315849,0.477755}%
\pgfsetfillcolor{currentfill}%
\pgfsetlinewidth{0.000000pt}%
\definecolor{currentstroke}{rgb}{0.000000,0.000000,0.000000}%
\pgfsetstrokecolor{currentstroke}%
\pgfsetdash{}{0pt}%
\pgfpathmoveto{\pgfqpoint{4.816407in}{2.124855in}}%
\pgfpathlineto{\pgfqpoint{4.854003in}{2.561465in}}%
\pgfpathlineto{\pgfqpoint{4.505500in}{2.253326in}}%
\pgfpathlineto{\pgfqpoint{4.816407in}{2.124855in}}%
\pgfpathclose%
\pgfusepath{fill}%
\end{pgfscope}%
\begin{pgfscope}%
\pgfpathrectangle{\pgfqpoint{3.536584in}{0.147348in}}{\pgfqpoint{2.735294in}{2.735294in}}%
\pgfusepath{clip}%
\pgfsetbuttcap%
\pgfsetroundjoin%
\definecolor{currentfill}{rgb}{0.045702,0.175435,0.265364}%
\pgfsetfillcolor{currentfill}%
\pgfsetlinewidth{0.000000pt}%
\definecolor{currentstroke}{rgb}{0.000000,0.000000,0.000000}%
\pgfsetstrokecolor{currentstroke}%
\pgfsetdash{}{0pt}%
\pgfpathmoveto{\pgfqpoint{4.209060in}{1.171211in}}%
\pgfpathlineto{\pgfqpoint{4.328767in}{1.039603in}}%
\pgfpathlineto{\pgfqpoint{4.414548in}{1.562087in}}%
\pgfpathlineto{\pgfqpoint{4.209060in}{1.171211in}}%
\pgfpathclose%
\pgfusepath{fill}%
\end{pgfscope}%
\begin{pgfscope}%
\pgfpathrectangle{\pgfqpoint{3.536584in}{0.147348in}}{\pgfqpoint{2.735294in}{2.735294in}}%
\pgfusepath{clip}%
\pgfsetbuttcap%
\pgfsetroundjoin%
\definecolor{currentfill}{rgb}{0.045702,0.175435,0.265364}%
\pgfsetfillcolor{currentfill}%
\pgfsetlinewidth{0.000000pt}%
\definecolor{currentstroke}{rgb}{0.000000,0.000000,0.000000}%
\pgfsetstrokecolor{currentstroke}%
\pgfsetdash{}{0pt}%
\pgfpathmoveto{\pgfqpoint{5.467842in}{1.562087in}}%
\pgfpathlineto{\pgfqpoint{5.553623in}{1.039603in}}%
\pgfpathlineto{\pgfqpoint{5.673330in}{1.171211in}}%
\pgfpathlineto{\pgfqpoint{5.467842in}{1.562087in}}%
\pgfpathclose%
\pgfusepath{fill}%
\end{pgfscope}%
\begin{pgfscope}%
\pgfpathrectangle{\pgfqpoint{3.536584in}{0.147348in}}{\pgfqpoint{2.735294in}{2.735294in}}%
\pgfusepath{clip}%
\pgfsetbuttcap%
\pgfsetroundjoin%
\definecolor{currentfill}{rgb}{0.040669,0.156116,0.236142}%
\pgfsetfillcolor{currentfill}%
\pgfsetlinewidth{0.000000pt}%
\definecolor{currentstroke}{rgb}{0.000000,0.000000,0.000000}%
\pgfsetstrokecolor{currentstroke}%
\pgfsetdash{}{0pt}%
\pgfpathmoveto{\pgfqpoint{4.579939in}{1.063020in}}%
\pgfpathlineto{\pgfqpoint{4.506598in}{0.985051in}}%
\pgfpathlineto{\pgfqpoint{4.714698in}{0.946905in}}%
\pgfpathlineto{\pgfqpoint{4.579939in}{1.063020in}}%
\pgfpathclose%
\pgfusepath{fill}%
\end{pgfscope}%
\begin{pgfscope}%
\pgfpathrectangle{\pgfqpoint{3.536584in}{0.147348in}}{\pgfqpoint{2.735294in}{2.735294in}}%
\pgfusepath{clip}%
\pgfsetbuttcap%
\pgfsetroundjoin%
\definecolor{currentfill}{rgb}{0.040669,0.156116,0.236142}%
\pgfsetfillcolor{currentfill}%
\pgfsetlinewidth{0.000000pt}%
\definecolor{currentstroke}{rgb}{0.000000,0.000000,0.000000}%
\pgfsetstrokecolor{currentstroke}%
\pgfsetdash{}{0pt}%
\pgfpathmoveto{\pgfqpoint{5.167692in}{0.946905in}}%
\pgfpathlineto{\pgfqpoint{5.375791in}{0.985051in}}%
\pgfpathlineto{\pgfqpoint{5.302451in}{1.063020in}}%
\pgfpathlineto{\pgfqpoint{5.167692in}{0.946905in}}%
\pgfpathclose%
\pgfusepath{fill}%
\end{pgfscope}%
\begin{pgfscope}%
\pgfpathrectangle{\pgfqpoint{3.536584in}{0.147348in}}{\pgfqpoint{2.735294in}{2.735294in}}%
\pgfusepath{clip}%
\pgfsetbuttcap%
\pgfsetroundjoin%
\definecolor{currentfill}{rgb}{0.063981,0.245604,0.371502}%
\pgfsetfillcolor{currentfill}%
\pgfsetlinewidth{0.000000pt}%
\definecolor{currentstroke}{rgb}{0.000000,0.000000,0.000000}%
\pgfsetstrokecolor{currentstroke}%
\pgfsetdash{}{0pt}%
\pgfpathmoveto{\pgfqpoint{4.414548in}{1.562087in}}%
\pgfpathlineto{\pgfqpoint{4.328767in}{1.039603in}}%
\pgfpathlineto{\pgfqpoint{4.374765in}{1.111775in}}%
\pgfpathlineto{\pgfqpoint{4.414548in}{1.562087in}}%
\pgfpathclose%
\pgfusepath{fill}%
\end{pgfscope}%
\begin{pgfscope}%
\pgfpathrectangle{\pgfqpoint{3.536584in}{0.147348in}}{\pgfqpoint{2.735294in}{2.735294in}}%
\pgfusepath{clip}%
\pgfsetbuttcap%
\pgfsetroundjoin%
\definecolor{currentfill}{rgb}{0.063981,0.245604,0.371502}%
\pgfsetfillcolor{currentfill}%
\pgfsetlinewidth{0.000000pt}%
\definecolor{currentstroke}{rgb}{0.000000,0.000000,0.000000}%
\pgfsetstrokecolor{currentstroke}%
\pgfsetdash{}{0pt}%
\pgfpathmoveto{\pgfqpoint{5.507624in}{1.111775in}}%
\pgfpathlineto{\pgfqpoint{5.553623in}{1.039603in}}%
\pgfpathlineto{\pgfqpoint{5.467842in}{1.562087in}}%
\pgfpathlineto{\pgfqpoint{5.507624in}{1.111775in}}%
\pgfpathclose%
\pgfusepath{fill}%
\end{pgfscope}%
\begin{pgfscope}%
\pgfpathrectangle{\pgfqpoint{3.536584in}{0.147348in}}{\pgfqpoint{2.735294in}{2.735294in}}%
\pgfusepath{clip}%
\pgfsetbuttcap%
\pgfsetroundjoin%
\definecolor{currentfill}{rgb}{0.060942,0.233938,0.353856}%
\pgfsetfillcolor{currentfill}%
\pgfsetlinewidth{0.000000pt}%
\definecolor{currentstroke}{rgb}{0.000000,0.000000,0.000000}%
\pgfsetstrokecolor{currentstroke}%
\pgfsetdash{}{0pt}%
\pgfpathmoveto{\pgfqpoint{4.217925in}{1.589448in}}%
\pgfpathlineto{\pgfqpoint{4.147150in}{1.772344in}}%
\pgfpathlineto{\pgfqpoint{4.070905in}{1.618388in}}%
\pgfpathlineto{\pgfqpoint{4.217925in}{1.589448in}}%
\pgfpathclose%
\pgfusepath{fill}%
\end{pgfscope}%
\begin{pgfscope}%
\pgfpathrectangle{\pgfqpoint{3.536584in}{0.147348in}}{\pgfqpoint{2.735294in}{2.735294in}}%
\pgfusepath{clip}%
\pgfsetbuttcap%
\pgfsetroundjoin%
\definecolor{currentfill}{rgb}{0.060942,0.233938,0.353856}%
\pgfsetfillcolor{currentfill}%
\pgfsetlinewidth{0.000000pt}%
\definecolor{currentstroke}{rgb}{0.000000,0.000000,0.000000}%
\pgfsetstrokecolor{currentstroke}%
\pgfsetdash{}{0pt}%
\pgfpathmoveto{\pgfqpoint{5.811485in}{1.618388in}}%
\pgfpathlineto{\pgfqpoint{5.735240in}{1.772344in}}%
\pgfpathlineto{\pgfqpoint{5.664465in}{1.589448in}}%
\pgfpathlineto{\pgfqpoint{5.811485in}{1.618388in}}%
\pgfpathclose%
\pgfusepath{fill}%
\end{pgfscope}%
\begin{pgfscope}%
\pgfpathrectangle{\pgfqpoint{3.536584in}{0.147348in}}{\pgfqpoint{2.735294in}{2.735294in}}%
\pgfusepath{clip}%
\pgfsetbuttcap%
\pgfsetroundjoin%
\definecolor{currentfill}{rgb}{0.081954,0.314596,0.475860}%
\pgfsetfillcolor{currentfill}%
\pgfsetlinewidth{0.000000pt}%
\definecolor{currentstroke}{rgb}{0.000000,0.000000,0.000000}%
\pgfsetstrokecolor{currentstroke}%
\pgfsetdash{}{0pt}%
\pgfpathmoveto{\pgfqpoint{4.816407in}{2.124855in}}%
\pgfpathlineto{\pgfqpoint{5.065982in}{2.124855in}}%
\pgfpathlineto{\pgfqpoint{5.028387in}{2.561465in}}%
\pgfpathlineto{\pgfqpoint{4.816407in}{2.124855in}}%
\pgfpathclose%
\pgfusepath{fill}%
\end{pgfscope}%
\begin{pgfscope}%
\pgfpathrectangle{\pgfqpoint{3.536584in}{0.147348in}}{\pgfqpoint{2.735294in}{2.735294in}}%
\pgfusepath{clip}%
\pgfsetbuttcap%
\pgfsetroundjoin%
\definecolor{currentfill}{rgb}{0.039595,0.151995,0.229908}%
\pgfsetfillcolor{currentfill}%
\pgfsetlinewidth{0.000000pt}%
\definecolor{currentstroke}{rgb}{0.000000,0.000000,0.000000}%
\pgfsetstrokecolor{currentstroke}%
\pgfsetdash{}{0pt}%
\pgfpathmoveto{\pgfqpoint{4.941195in}{0.933095in}}%
\pgfpathlineto{\pgfqpoint{4.816678in}{1.035006in}}%
\pgfpathlineto{\pgfqpoint{4.714698in}{0.946905in}}%
\pgfpathlineto{\pgfqpoint{4.941195in}{0.933095in}}%
\pgfpathclose%
\pgfusepath{fill}%
\end{pgfscope}%
\begin{pgfscope}%
\pgfpathrectangle{\pgfqpoint{3.536584in}{0.147348in}}{\pgfqpoint{2.735294in}{2.735294in}}%
\pgfusepath{clip}%
\pgfsetbuttcap%
\pgfsetroundjoin%
\definecolor{currentfill}{rgb}{0.039595,0.151995,0.229908}%
\pgfsetfillcolor{currentfill}%
\pgfsetlinewidth{0.000000pt}%
\definecolor{currentstroke}{rgb}{0.000000,0.000000,0.000000}%
\pgfsetstrokecolor{currentstroke}%
\pgfsetdash{}{0pt}%
\pgfpathmoveto{\pgfqpoint{5.167692in}{0.946905in}}%
\pgfpathlineto{\pgfqpoint{5.065712in}{1.035006in}}%
\pgfpathlineto{\pgfqpoint{4.941195in}{0.933095in}}%
\pgfpathlineto{\pgfqpoint{5.167692in}{0.946905in}}%
\pgfpathclose%
\pgfusepath{fill}%
\end{pgfscope}%
\begin{pgfscope}%
\pgfpathrectangle{\pgfqpoint{3.536584in}{0.147348in}}{\pgfqpoint{2.735294in}{2.735294in}}%
\pgfusepath{clip}%
\pgfsetbuttcap%
\pgfsetroundjoin%
\definecolor{currentfill}{rgb}{0.075436,0.289576,0.438014}%
\pgfsetfillcolor{currentfill}%
\pgfsetlinewidth{0.000000pt}%
\definecolor{currentstroke}{rgb}{0.000000,0.000000,0.000000}%
\pgfsetstrokecolor{currentstroke}%
\pgfsetdash{}{0pt}%
\pgfpathmoveto{\pgfqpoint{5.508743in}{2.103280in}}%
\pgfpathlineto{\pgfqpoint{5.376890in}{2.253326in}}%
\pgfpathlineto{\pgfqpoint{5.303210in}{2.116981in}}%
\pgfpathlineto{\pgfqpoint{5.508743in}{2.103280in}}%
\pgfpathclose%
\pgfusepath{fill}%
\end{pgfscope}%
\begin{pgfscope}%
\pgfpathrectangle{\pgfqpoint{3.536584in}{0.147348in}}{\pgfqpoint{2.735294in}{2.735294in}}%
\pgfusepath{clip}%
\pgfsetbuttcap%
\pgfsetroundjoin%
\definecolor{currentfill}{rgb}{0.075436,0.289576,0.438014}%
\pgfsetfillcolor{currentfill}%
\pgfsetlinewidth{0.000000pt}%
\definecolor{currentstroke}{rgb}{0.000000,0.000000,0.000000}%
\pgfsetstrokecolor{currentstroke}%
\pgfsetdash{}{0pt}%
\pgfpathmoveto{\pgfqpoint{4.579180in}{2.116981in}}%
\pgfpathlineto{\pgfqpoint{4.505500in}{2.253326in}}%
\pgfpathlineto{\pgfqpoint{4.373646in}{2.103280in}}%
\pgfpathlineto{\pgfqpoint{4.579180in}{2.116981in}}%
\pgfpathclose%
\pgfusepath{fill}%
\end{pgfscope}%
\begin{pgfscope}%
\pgfpathrectangle{\pgfqpoint{3.536584in}{0.147348in}}{\pgfqpoint{2.735294in}{2.735294in}}%
\pgfusepath{clip}%
\pgfsetbuttcap%
\pgfsetroundjoin%
\definecolor{currentfill}{rgb}{0.062760,0.240916,0.364410}%
\pgfsetfillcolor{currentfill}%
\pgfsetlinewidth{0.000000pt}%
\definecolor{currentstroke}{rgb}{0.000000,0.000000,0.000000}%
\pgfsetstrokecolor{currentstroke}%
\pgfsetdash{}{0pt}%
\pgfpathmoveto{\pgfqpoint{4.217925in}{1.589448in}}%
\pgfpathlineto{\pgfqpoint{4.373646in}{2.103280in}}%
\pgfpathlineto{\pgfqpoint{4.147150in}{1.772344in}}%
\pgfpathlineto{\pgfqpoint{4.217925in}{1.589448in}}%
\pgfpathclose%
\pgfusepath{fill}%
\end{pgfscope}%
\begin{pgfscope}%
\pgfpathrectangle{\pgfqpoint{3.536584in}{0.147348in}}{\pgfqpoint{2.735294in}{2.735294in}}%
\pgfusepath{clip}%
\pgfsetbuttcap%
\pgfsetroundjoin%
\definecolor{currentfill}{rgb}{0.062760,0.240916,0.364410}%
\pgfsetfillcolor{currentfill}%
\pgfsetlinewidth{0.000000pt}%
\definecolor{currentstroke}{rgb}{0.000000,0.000000,0.000000}%
\pgfsetstrokecolor{currentstroke}%
\pgfsetdash{}{0pt}%
\pgfpathmoveto{\pgfqpoint{5.735240in}{1.772344in}}%
\pgfpathlineto{\pgfqpoint{5.508743in}{2.103280in}}%
\pgfpathlineto{\pgfqpoint{5.664465in}{1.589448in}}%
\pgfpathlineto{\pgfqpoint{5.735240in}{1.772344in}}%
\pgfpathclose%
\pgfusepath{fill}%
\end{pgfscope}%
\begin{pgfscope}%
\pgfpathrectangle{\pgfqpoint{3.536584in}{0.147348in}}{\pgfqpoint{2.735294in}{2.735294in}}%
\pgfusepath{clip}%
\pgfsetbuttcap%
\pgfsetroundjoin%
\definecolor{currentfill}{rgb}{0.043508,0.167016,0.252629}%
\pgfsetfillcolor{currentfill}%
\pgfsetlinewidth{0.000000pt}%
\definecolor{currentstroke}{rgb}{0.000000,0.000000,0.000000}%
\pgfsetstrokecolor{currentstroke}%
\pgfsetdash{}{0pt}%
\pgfpathmoveto{\pgfqpoint{4.374765in}{1.111775in}}%
\pgfpathlineto{\pgfqpoint{4.506598in}{0.985051in}}%
\pgfpathlineto{\pgfqpoint{4.538549in}{1.360106in}}%
\pgfpathlineto{\pgfqpoint{4.374765in}{1.111775in}}%
\pgfpathclose%
\pgfusepath{fill}%
\end{pgfscope}%
\begin{pgfscope}%
\pgfpathrectangle{\pgfqpoint{3.536584in}{0.147348in}}{\pgfqpoint{2.735294in}{2.735294in}}%
\pgfusepath{clip}%
\pgfsetbuttcap%
\pgfsetroundjoin%
\definecolor{currentfill}{rgb}{0.043508,0.167016,0.252629}%
\pgfsetfillcolor{currentfill}%
\pgfsetlinewidth{0.000000pt}%
\definecolor{currentstroke}{rgb}{0.000000,0.000000,0.000000}%
\pgfsetstrokecolor{currentstroke}%
\pgfsetdash{}{0pt}%
\pgfpathmoveto{\pgfqpoint{5.343841in}{1.360106in}}%
\pgfpathlineto{\pgfqpoint{5.375791in}{0.985051in}}%
\pgfpathlineto{\pgfqpoint{5.507624in}{1.111775in}}%
\pgfpathlineto{\pgfqpoint{5.343841in}{1.360106in}}%
\pgfpathclose%
\pgfusepath{fill}%
\end{pgfscope}%
\begin{pgfscope}%
\pgfpathrectangle{\pgfqpoint{3.536584in}{0.147348in}}{\pgfqpoint{2.735294in}{2.735294in}}%
\pgfusepath{clip}%
\pgfsetbuttcap%
\pgfsetroundjoin%
\definecolor{currentfill}{rgb}{0.050011,0.191979,0.290388}%
\pgfsetfillcolor{currentfill}%
\pgfsetlinewidth{0.000000pt}%
\definecolor{currentstroke}{rgb}{0.000000,0.000000,0.000000}%
\pgfsetstrokecolor{currentstroke}%
\pgfsetdash{}{0pt}%
\pgfpathmoveto{\pgfqpoint{4.217925in}{1.589448in}}%
\pgfpathlineto{\pgfqpoint{4.209060in}{1.171211in}}%
\pgfpathlineto{\pgfqpoint{4.414548in}{1.562087in}}%
\pgfpathlineto{\pgfqpoint{4.217925in}{1.589448in}}%
\pgfpathclose%
\pgfusepath{fill}%
\end{pgfscope}%
\begin{pgfscope}%
\pgfpathrectangle{\pgfqpoint{3.536584in}{0.147348in}}{\pgfqpoint{2.735294in}{2.735294in}}%
\pgfusepath{clip}%
\pgfsetbuttcap%
\pgfsetroundjoin%
\definecolor{currentfill}{rgb}{0.050011,0.191979,0.290388}%
\pgfsetfillcolor{currentfill}%
\pgfsetlinewidth{0.000000pt}%
\definecolor{currentstroke}{rgb}{0.000000,0.000000,0.000000}%
\pgfsetstrokecolor{currentstroke}%
\pgfsetdash{}{0pt}%
\pgfpathmoveto{\pgfqpoint{5.467842in}{1.562087in}}%
\pgfpathlineto{\pgfqpoint{5.673330in}{1.171211in}}%
\pgfpathlineto{\pgfqpoint{5.664465in}{1.589448in}}%
\pgfpathlineto{\pgfqpoint{5.467842in}{1.562087in}}%
\pgfpathclose%
\pgfusepath{fill}%
\end{pgfscope}%
\begin{pgfscope}%
\pgfpathrectangle{\pgfqpoint{3.536584in}{0.147348in}}{\pgfqpoint{2.735294in}{2.735294in}}%
\pgfusepath{clip}%
\pgfsetbuttcap%
\pgfsetroundjoin%
\definecolor{currentfill}{rgb}{0.049941,0.191710,0.289982}%
\pgfsetfillcolor{currentfill}%
\pgfsetlinewidth{0.000000pt}%
\definecolor{currentstroke}{rgb}{0.000000,0.000000,0.000000}%
\pgfsetstrokecolor{currentstroke}%
\pgfsetdash{}{0pt}%
\pgfpathmoveto{\pgfqpoint{5.302451in}{1.063020in}}%
\pgfpathlineto{\pgfqpoint{5.375791in}{0.985051in}}%
\pgfpathlineto{\pgfqpoint{5.343841in}{1.360106in}}%
\pgfpathlineto{\pgfqpoint{5.302451in}{1.063020in}}%
\pgfpathclose%
\pgfusepath{fill}%
\end{pgfscope}%
\begin{pgfscope}%
\pgfpathrectangle{\pgfqpoint{3.536584in}{0.147348in}}{\pgfqpoint{2.735294in}{2.735294in}}%
\pgfusepath{clip}%
\pgfsetbuttcap%
\pgfsetroundjoin%
\definecolor{currentfill}{rgb}{0.049941,0.191710,0.289982}%
\pgfsetfillcolor{currentfill}%
\pgfsetlinewidth{0.000000pt}%
\definecolor{currentstroke}{rgb}{0.000000,0.000000,0.000000}%
\pgfsetstrokecolor{currentstroke}%
\pgfsetdash{}{0pt}%
\pgfpathmoveto{\pgfqpoint{4.538549in}{1.360106in}}%
\pgfpathlineto{\pgfqpoint{4.506598in}{0.985051in}}%
\pgfpathlineto{\pgfqpoint{4.579939in}{1.063020in}}%
\pgfpathlineto{\pgfqpoint{4.538549in}{1.360106in}}%
\pgfpathclose%
\pgfusepath{fill}%
\end{pgfscope}%
\begin{pgfscope}%
\pgfpathrectangle{\pgfqpoint{3.536584in}{0.147348in}}{\pgfqpoint{2.735294in}{2.735294in}}%
\pgfusepath{clip}%
\pgfsetbuttcap%
\pgfsetroundjoin%
\definecolor{currentfill}{rgb}{0.078663,0.301965,0.456754}%
\pgfsetfillcolor{currentfill}%
\pgfsetlinewidth{0.000000pt}%
\definecolor{currentstroke}{rgb}{0.000000,0.000000,0.000000}%
\pgfsetstrokecolor{currentstroke}%
\pgfsetdash{}{0pt}%
\pgfpathmoveto{\pgfqpoint{4.816407in}{2.124855in}}%
\pgfpathlineto{\pgfqpoint{4.505500in}{2.253326in}}%
\pgfpathlineto{\pgfqpoint{4.579180in}{2.116981in}}%
\pgfpathlineto{\pgfqpoint{4.816407in}{2.124855in}}%
\pgfpathclose%
\pgfusepath{fill}%
\end{pgfscope}%
\begin{pgfscope}%
\pgfpathrectangle{\pgfqpoint{3.536584in}{0.147348in}}{\pgfqpoint{2.735294in}{2.735294in}}%
\pgfusepath{clip}%
\pgfsetbuttcap%
\pgfsetroundjoin%
\definecolor{currentfill}{rgb}{0.078663,0.301965,0.456754}%
\pgfsetfillcolor{currentfill}%
\pgfsetlinewidth{0.000000pt}%
\definecolor{currentstroke}{rgb}{0.000000,0.000000,0.000000}%
\pgfsetstrokecolor{currentstroke}%
\pgfsetdash{}{0pt}%
\pgfpathmoveto{\pgfqpoint{5.303210in}{2.116981in}}%
\pgfpathlineto{\pgfqpoint{5.376890in}{2.253326in}}%
\pgfpathlineto{\pgfqpoint{5.065982in}{2.124855in}}%
\pgfpathlineto{\pgfqpoint{5.303210in}{2.116981in}}%
\pgfpathclose%
\pgfusepath{fill}%
\end{pgfscope}%
\begin{pgfscope}%
\pgfpathrectangle{\pgfqpoint{3.536584in}{0.147348in}}{\pgfqpoint{2.735294in}{2.735294in}}%
\pgfusepath{clip}%
\pgfsetbuttcap%
\pgfsetroundjoin%
\definecolor{currentfill}{rgb}{0.064954,0.249341,0.377155}%
\pgfsetfillcolor{currentfill}%
\pgfsetlinewidth{0.000000pt}%
\definecolor{currentstroke}{rgb}{0.000000,0.000000,0.000000}%
\pgfsetstrokecolor{currentstroke}%
\pgfsetdash{}{0pt}%
\pgfpathmoveto{\pgfqpoint{4.414548in}{1.562087in}}%
\pgfpathlineto{\pgfqpoint{4.373646in}{2.103280in}}%
\pgfpathlineto{\pgfqpoint{4.217925in}{1.589448in}}%
\pgfpathlineto{\pgfqpoint{4.414548in}{1.562087in}}%
\pgfpathclose%
\pgfusepath{fill}%
\end{pgfscope}%
\begin{pgfscope}%
\pgfpathrectangle{\pgfqpoint{3.536584in}{0.147348in}}{\pgfqpoint{2.735294in}{2.735294in}}%
\pgfusepath{clip}%
\pgfsetbuttcap%
\pgfsetroundjoin%
\definecolor{currentfill}{rgb}{0.064954,0.249341,0.377155}%
\pgfsetfillcolor{currentfill}%
\pgfsetlinewidth{0.000000pt}%
\definecolor{currentstroke}{rgb}{0.000000,0.000000,0.000000}%
\pgfsetstrokecolor{currentstroke}%
\pgfsetdash{}{0pt}%
\pgfpathmoveto{\pgfqpoint{5.664465in}{1.589448in}}%
\pgfpathlineto{\pgfqpoint{5.508743in}{2.103280in}}%
\pgfpathlineto{\pgfqpoint{5.467842in}{1.562087in}}%
\pgfpathlineto{\pgfqpoint{5.664465in}{1.589448in}}%
\pgfpathclose%
\pgfusepath{fill}%
\end{pgfscope}%
\begin{pgfscope}%
\pgfpathrectangle{\pgfqpoint{3.536584in}{0.147348in}}{\pgfqpoint{2.735294in}{2.735294in}}%
\pgfusepath{clip}%
\pgfsetbuttcap%
\pgfsetroundjoin%
\definecolor{currentfill}{rgb}{0.042669,0.163794,0.247755}%
\pgfsetfillcolor{currentfill}%
\pgfsetlinewidth{0.000000pt}%
\definecolor{currentstroke}{rgb}{0.000000,0.000000,0.000000}%
\pgfsetstrokecolor{currentstroke}%
\pgfsetdash{}{0pt}%
\pgfpathmoveto{\pgfqpoint{4.714698in}{0.946905in}}%
\pgfpathlineto{\pgfqpoint{4.801285in}{1.337788in}}%
\pgfpathlineto{\pgfqpoint{4.579939in}{1.063020in}}%
\pgfpathlineto{\pgfqpoint{4.714698in}{0.946905in}}%
\pgfpathclose%
\pgfusepath{fill}%
\end{pgfscope}%
\begin{pgfscope}%
\pgfpathrectangle{\pgfqpoint{3.536584in}{0.147348in}}{\pgfqpoint{2.735294in}{2.735294in}}%
\pgfusepath{clip}%
\pgfsetbuttcap%
\pgfsetroundjoin%
\definecolor{currentfill}{rgb}{0.042669,0.163794,0.247755}%
\pgfsetfillcolor{currentfill}%
\pgfsetlinewidth{0.000000pt}%
\definecolor{currentstroke}{rgb}{0.000000,0.000000,0.000000}%
\pgfsetstrokecolor{currentstroke}%
\pgfsetdash{}{0pt}%
\pgfpathmoveto{\pgfqpoint{5.302451in}{1.063020in}}%
\pgfpathlineto{\pgfqpoint{5.081105in}{1.337788in}}%
\pgfpathlineto{\pgfqpoint{5.167692in}{0.946905in}}%
\pgfpathlineto{\pgfqpoint{5.302451in}{1.063020in}}%
\pgfpathclose%
\pgfusepath{fill}%
\end{pgfscope}%
\begin{pgfscope}%
\pgfpathrectangle{\pgfqpoint{3.536584in}{0.147348in}}{\pgfqpoint{2.735294in}{2.735294in}}%
\pgfusepath{clip}%
\pgfsetbuttcap%
\pgfsetroundjoin%
\definecolor{currentfill}{rgb}{0.068541,0.263111,0.397982}%
\pgfsetfillcolor{currentfill}%
\pgfsetlinewidth{0.000000pt}%
\definecolor{currentstroke}{rgb}{0.000000,0.000000,0.000000}%
\pgfsetstrokecolor{currentstroke}%
\pgfsetdash{}{0pt}%
\pgfpathmoveto{\pgfqpoint{4.579180in}{2.116981in}}%
\pgfpathlineto{\pgfqpoint{4.373646in}{2.103280in}}%
\pgfpathlineto{\pgfqpoint{4.677670in}{1.946079in}}%
\pgfpathlineto{\pgfqpoint{4.579180in}{2.116981in}}%
\pgfpathclose%
\pgfusepath{fill}%
\end{pgfscope}%
\begin{pgfscope}%
\pgfpathrectangle{\pgfqpoint{3.536584in}{0.147348in}}{\pgfqpoint{2.735294in}{2.735294in}}%
\pgfusepath{clip}%
\pgfsetbuttcap%
\pgfsetroundjoin%
\definecolor{currentfill}{rgb}{0.068541,0.263111,0.397982}%
\pgfsetfillcolor{currentfill}%
\pgfsetlinewidth{0.000000pt}%
\definecolor{currentstroke}{rgb}{0.000000,0.000000,0.000000}%
\pgfsetstrokecolor{currentstroke}%
\pgfsetdash{}{0pt}%
\pgfpathmoveto{\pgfqpoint{5.204720in}{1.946079in}}%
\pgfpathlineto{\pgfqpoint{5.508743in}{2.103280in}}%
\pgfpathlineto{\pgfqpoint{5.303210in}{2.116981in}}%
\pgfpathlineto{\pgfqpoint{5.204720in}{1.946079in}}%
\pgfpathclose%
\pgfusepath{fill}%
\end{pgfscope}%
\begin{pgfscope}%
\pgfpathrectangle{\pgfqpoint{3.536584in}{0.147348in}}{\pgfqpoint{2.735294in}{2.735294in}}%
\pgfusepath{clip}%
\pgfsetbuttcap%
\pgfsetroundjoin%
\definecolor{currentfill}{rgb}{0.047555,0.182548,0.276123}%
\pgfsetfillcolor{currentfill}%
\pgfsetlinewidth{0.000000pt}%
\definecolor{currentstroke}{rgb}{0.000000,0.000000,0.000000}%
\pgfsetstrokecolor{currentstroke}%
\pgfsetdash{}{0pt}%
\pgfpathmoveto{\pgfqpoint{4.714698in}{0.946905in}}%
\pgfpathlineto{\pgfqpoint{4.816678in}{1.035006in}}%
\pgfpathlineto{\pgfqpoint{4.801285in}{1.337788in}}%
\pgfpathlineto{\pgfqpoint{4.714698in}{0.946905in}}%
\pgfpathclose%
\pgfusepath{fill}%
\end{pgfscope}%
\begin{pgfscope}%
\pgfpathrectangle{\pgfqpoint{3.536584in}{0.147348in}}{\pgfqpoint{2.735294in}{2.735294in}}%
\pgfusepath{clip}%
\pgfsetbuttcap%
\pgfsetroundjoin%
\definecolor{currentfill}{rgb}{0.047555,0.182548,0.276123}%
\pgfsetfillcolor{currentfill}%
\pgfsetlinewidth{0.000000pt}%
\definecolor{currentstroke}{rgb}{0.000000,0.000000,0.000000}%
\pgfsetstrokecolor{currentstroke}%
\pgfsetdash{}{0pt}%
\pgfpathmoveto{\pgfqpoint{5.081105in}{1.337788in}}%
\pgfpathlineto{\pgfqpoint{5.065712in}{1.035006in}}%
\pgfpathlineto{\pgfqpoint{5.167692in}{0.946905in}}%
\pgfpathlineto{\pgfqpoint{5.081105in}{1.337788in}}%
\pgfpathclose%
\pgfusepath{fill}%
\end{pgfscope}%
\begin{pgfscope}%
\pgfpathrectangle{\pgfqpoint{3.536584in}{0.147348in}}{\pgfqpoint{2.735294in}{2.735294in}}%
\pgfusepath{clip}%
\pgfsetbuttcap%
\pgfsetroundjoin%
\definecolor{currentfill}{rgb}{0.046101,0.176968,0.267683}%
\pgfsetfillcolor{currentfill}%
\pgfsetlinewidth{0.000000pt}%
\definecolor{currentstroke}{rgb}{0.000000,0.000000,0.000000}%
\pgfsetstrokecolor{currentstroke}%
\pgfsetdash{}{0pt}%
\pgfpathmoveto{\pgfqpoint{4.941195in}{0.933095in}}%
\pgfpathlineto{\pgfqpoint{4.801285in}{1.337788in}}%
\pgfpathlineto{\pgfqpoint{4.816678in}{1.035006in}}%
\pgfpathlineto{\pgfqpoint{4.941195in}{0.933095in}}%
\pgfpathclose%
\pgfusepath{fill}%
\end{pgfscope}%
\begin{pgfscope}%
\pgfpathrectangle{\pgfqpoint{3.536584in}{0.147348in}}{\pgfqpoint{2.735294in}{2.735294in}}%
\pgfusepath{clip}%
\pgfsetbuttcap%
\pgfsetroundjoin%
\definecolor{currentfill}{rgb}{0.046101,0.176968,0.267683}%
\pgfsetfillcolor{currentfill}%
\pgfsetlinewidth{0.000000pt}%
\definecolor{currentstroke}{rgb}{0.000000,0.000000,0.000000}%
\pgfsetstrokecolor{currentstroke}%
\pgfsetdash{}{0pt}%
\pgfpathmoveto{\pgfqpoint{5.065712in}{1.035006in}}%
\pgfpathlineto{\pgfqpoint{5.081105in}{1.337788in}}%
\pgfpathlineto{\pgfqpoint{4.941195in}{0.933095in}}%
\pgfpathlineto{\pgfqpoint{5.065712in}{1.035006in}}%
\pgfpathclose%
\pgfusepath{fill}%
\end{pgfscope}%
\begin{pgfscope}%
\pgfpathrectangle{\pgfqpoint{3.536584in}{0.147348in}}{\pgfqpoint{2.735294in}{2.735294in}}%
\pgfusepath{clip}%
\pgfsetbuttcap%
\pgfsetroundjoin%
\definecolor{currentfill}{rgb}{0.051850,0.199036,0.301063}%
\pgfsetfillcolor{currentfill}%
\pgfsetlinewidth{0.000000pt}%
\definecolor{currentstroke}{rgb}{0.000000,0.000000,0.000000}%
\pgfsetstrokecolor{currentstroke}%
\pgfsetdash{}{0pt}%
\pgfpathmoveto{\pgfqpoint{4.374765in}{1.111775in}}%
\pgfpathlineto{\pgfqpoint{4.538549in}{1.360106in}}%
\pgfpathlineto{\pgfqpoint{4.414548in}{1.562087in}}%
\pgfpathlineto{\pgfqpoint{4.374765in}{1.111775in}}%
\pgfpathclose%
\pgfusepath{fill}%
\end{pgfscope}%
\begin{pgfscope}%
\pgfpathrectangle{\pgfqpoint{3.536584in}{0.147348in}}{\pgfqpoint{2.735294in}{2.735294in}}%
\pgfusepath{clip}%
\pgfsetbuttcap%
\pgfsetroundjoin%
\definecolor{currentfill}{rgb}{0.051850,0.199036,0.301063}%
\pgfsetfillcolor{currentfill}%
\pgfsetlinewidth{0.000000pt}%
\definecolor{currentstroke}{rgb}{0.000000,0.000000,0.000000}%
\pgfsetstrokecolor{currentstroke}%
\pgfsetdash{}{0pt}%
\pgfpathmoveto{\pgfqpoint{5.467842in}{1.562087in}}%
\pgfpathlineto{\pgfqpoint{5.343841in}{1.360106in}}%
\pgfpathlineto{\pgfqpoint{5.507624in}{1.111775in}}%
\pgfpathlineto{\pgfqpoint{5.467842in}{1.562087in}}%
\pgfpathclose%
\pgfusepath{fill}%
\end{pgfscope}%
\begin{pgfscope}%
\pgfpathrectangle{\pgfqpoint{3.536584in}{0.147348in}}{\pgfqpoint{2.735294in}{2.735294in}}%
\pgfusepath{clip}%
\pgfsetbuttcap%
\pgfsetroundjoin%
\definecolor{currentfill}{rgb}{0.064759,0.248590,0.376018}%
\pgfsetfillcolor{currentfill}%
\pgfsetlinewidth{0.000000pt}%
\definecolor{currentstroke}{rgb}{0.000000,0.000000,0.000000}%
\pgfsetstrokecolor{currentstroke}%
\pgfsetdash{}{0pt}%
\pgfpathmoveto{\pgfqpoint{5.508743in}{2.103280in}}%
\pgfpathlineto{\pgfqpoint{5.204720in}{1.946079in}}%
\pgfpathlineto{\pgfqpoint{5.467842in}{1.562087in}}%
\pgfpathlineto{\pgfqpoint{5.508743in}{2.103280in}}%
\pgfpathclose%
\pgfusepath{fill}%
\end{pgfscope}%
\begin{pgfscope}%
\pgfpathrectangle{\pgfqpoint{3.536584in}{0.147348in}}{\pgfqpoint{2.735294in}{2.735294in}}%
\pgfusepath{clip}%
\pgfsetbuttcap%
\pgfsetroundjoin%
\definecolor{currentfill}{rgb}{0.064759,0.248590,0.376018}%
\pgfsetfillcolor{currentfill}%
\pgfsetlinewidth{0.000000pt}%
\definecolor{currentstroke}{rgb}{0.000000,0.000000,0.000000}%
\pgfsetstrokecolor{currentstroke}%
\pgfsetdash{}{0pt}%
\pgfpathmoveto{\pgfqpoint{4.414548in}{1.562087in}}%
\pgfpathlineto{\pgfqpoint{4.677670in}{1.946079in}}%
\pgfpathlineto{\pgfqpoint{4.373646in}{2.103280in}}%
\pgfpathlineto{\pgfqpoint{4.414548in}{1.562087in}}%
\pgfpathclose%
\pgfusepath{fill}%
\end{pgfscope}%
\begin{pgfscope}%
\pgfpathrectangle{\pgfqpoint{3.536584in}{0.147348in}}{\pgfqpoint{2.735294in}{2.735294in}}%
\pgfusepath{clip}%
\pgfsetbuttcap%
\pgfsetroundjoin%
\definecolor{currentfill}{rgb}{0.071694,0.275212,0.416288}%
\pgfsetfillcolor{currentfill}%
\pgfsetlinewidth{0.000000pt}%
\definecolor{currentstroke}{rgb}{0.000000,0.000000,0.000000}%
\pgfsetstrokecolor{currentstroke}%
\pgfsetdash{}{0pt}%
\pgfpathmoveto{\pgfqpoint{4.677670in}{1.946079in}}%
\pgfpathlineto{\pgfqpoint{4.816407in}{2.124855in}}%
\pgfpathlineto{\pgfqpoint{4.579180in}{2.116981in}}%
\pgfpathlineto{\pgfqpoint{4.677670in}{1.946079in}}%
\pgfpathclose%
\pgfusepath{fill}%
\end{pgfscope}%
\begin{pgfscope}%
\pgfpathrectangle{\pgfqpoint{3.536584in}{0.147348in}}{\pgfqpoint{2.735294in}{2.735294in}}%
\pgfusepath{clip}%
\pgfsetbuttcap%
\pgfsetroundjoin%
\definecolor{currentfill}{rgb}{0.071694,0.275212,0.416288}%
\pgfsetfillcolor{currentfill}%
\pgfsetlinewidth{0.000000pt}%
\definecolor{currentstroke}{rgb}{0.000000,0.000000,0.000000}%
\pgfsetstrokecolor{currentstroke}%
\pgfsetdash{}{0pt}%
\pgfpathmoveto{\pgfqpoint{5.303210in}{2.116981in}}%
\pgfpathlineto{\pgfqpoint{5.065982in}{2.124855in}}%
\pgfpathlineto{\pgfqpoint{5.204720in}{1.946079in}}%
\pgfpathlineto{\pgfqpoint{5.303210in}{2.116981in}}%
\pgfpathclose%
\pgfusepath{fill}%
\end{pgfscope}%
\begin{pgfscope}%
\pgfpathrectangle{\pgfqpoint{3.536584in}{0.147348in}}{\pgfqpoint{2.735294in}{2.735294in}}%
\pgfusepath{clip}%
\pgfsetbuttcap%
\pgfsetroundjoin%
\definecolor{currentfill}{rgb}{0.071636,0.274990,0.415951}%
\pgfsetfillcolor{currentfill}%
\pgfsetlinewidth{0.000000pt}%
\definecolor{currentstroke}{rgb}{0.000000,0.000000,0.000000}%
\pgfsetstrokecolor{currentstroke}%
\pgfsetdash{}{0pt}%
\pgfpathmoveto{\pgfqpoint{4.941195in}{1.947266in}}%
\pgfpathlineto{\pgfqpoint{5.065982in}{2.124855in}}%
\pgfpathlineto{\pgfqpoint{4.816407in}{2.124855in}}%
\pgfpathlineto{\pgfqpoint{4.941195in}{1.947266in}}%
\pgfpathclose%
\pgfusepath{fill}%
\end{pgfscope}%
\begin{pgfscope}%
\pgfpathrectangle{\pgfqpoint{3.536584in}{0.147348in}}{\pgfqpoint{2.735294in}{2.735294in}}%
\pgfusepath{clip}%
\pgfsetbuttcap%
\pgfsetroundjoin%
\definecolor{currentfill}{rgb}{0.045820,0.175891,0.266053}%
\pgfsetfillcolor{currentfill}%
\pgfsetlinewidth{0.000000pt}%
\definecolor{currentstroke}{rgb}{0.000000,0.000000,0.000000}%
\pgfsetstrokecolor{currentstroke}%
\pgfsetdash{}{0pt}%
\pgfpathmoveto{\pgfqpoint{4.579939in}{1.063020in}}%
\pgfpathlineto{\pgfqpoint{4.661262in}{1.541647in}}%
\pgfpathlineto{\pgfqpoint{4.538549in}{1.360106in}}%
\pgfpathlineto{\pgfqpoint{4.579939in}{1.063020in}}%
\pgfpathclose%
\pgfusepath{fill}%
\end{pgfscope}%
\begin{pgfscope}%
\pgfpathrectangle{\pgfqpoint{3.536584in}{0.147348in}}{\pgfqpoint{2.735294in}{2.735294in}}%
\pgfusepath{clip}%
\pgfsetbuttcap%
\pgfsetroundjoin%
\definecolor{currentfill}{rgb}{0.045820,0.175891,0.266053}%
\pgfsetfillcolor{currentfill}%
\pgfsetlinewidth{0.000000pt}%
\definecolor{currentstroke}{rgb}{0.000000,0.000000,0.000000}%
\pgfsetstrokecolor{currentstroke}%
\pgfsetdash{}{0pt}%
\pgfpathmoveto{\pgfqpoint{5.343841in}{1.360106in}}%
\pgfpathlineto{\pgfqpoint{5.221128in}{1.541647in}}%
\pgfpathlineto{\pgfqpoint{5.302451in}{1.063020in}}%
\pgfpathlineto{\pgfqpoint{5.343841in}{1.360106in}}%
\pgfpathclose%
\pgfusepath{fill}%
\end{pgfscope}%
\begin{pgfscope}%
\pgfpathrectangle{\pgfqpoint{3.536584in}{0.147348in}}{\pgfqpoint{2.735294in}{2.735294in}}%
\pgfusepath{clip}%
\pgfsetbuttcap%
\pgfsetroundjoin%
\definecolor{currentfill}{rgb}{0.046814,0.179706,0.271825}%
\pgfsetfillcolor{currentfill}%
\pgfsetlinewidth{0.000000pt}%
\definecolor{currentstroke}{rgb}{0.000000,0.000000,0.000000}%
\pgfsetstrokecolor{currentstroke}%
\pgfsetdash{}{0pt}%
\pgfpathmoveto{\pgfqpoint{4.941195in}{1.533948in}}%
\pgfpathlineto{\pgfqpoint{4.941195in}{0.933095in}}%
\pgfpathlineto{\pgfqpoint{5.081105in}{1.337788in}}%
\pgfpathlineto{\pgfqpoint{4.941195in}{1.533948in}}%
\pgfpathclose%
\pgfusepath{fill}%
\end{pgfscope}%
\begin{pgfscope}%
\pgfpathrectangle{\pgfqpoint{3.536584in}{0.147348in}}{\pgfqpoint{2.735294in}{2.735294in}}%
\pgfusepath{clip}%
\pgfsetbuttcap%
\pgfsetroundjoin%
\definecolor{currentfill}{rgb}{0.046814,0.179706,0.271825}%
\pgfsetfillcolor{currentfill}%
\pgfsetlinewidth{0.000000pt}%
\definecolor{currentstroke}{rgb}{0.000000,0.000000,0.000000}%
\pgfsetstrokecolor{currentstroke}%
\pgfsetdash{}{0pt}%
\pgfpathmoveto{\pgfqpoint{4.801285in}{1.337788in}}%
\pgfpathlineto{\pgfqpoint{4.941195in}{0.933095in}}%
\pgfpathlineto{\pgfqpoint{4.941195in}{1.533948in}}%
\pgfpathlineto{\pgfqpoint{4.801285in}{1.337788in}}%
\pgfpathclose%
\pgfusepath{fill}%
\end{pgfscope}%
\begin{pgfscope}%
\pgfpathrectangle{\pgfqpoint{3.536584in}{0.147348in}}{\pgfqpoint{2.735294in}{2.735294in}}%
\pgfusepath{clip}%
\pgfsetbuttcap%
\pgfsetroundjoin%
\definecolor{currentfill}{rgb}{0.069261,0.265872,0.402159}%
\pgfsetfillcolor{currentfill}%
\pgfsetlinewidth{0.000000pt}%
\definecolor{currentstroke}{rgb}{0.000000,0.000000,0.000000}%
\pgfsetstrokecolor{currentstroke}%
\pgfsetdash{}{0pt}%
\pgfpathmoveto{\pgfqpoint{5.204720in}{1.946079in}}%
\pgfpathlineto{\pgfqpoint{5.065982in}{2.124855in}}%
\pgfpathlineto{\pgfqpoint{4.941195in}{1.947266in}}%
\pgfpathlineto{\pgfqpoint{5.204720in}{1.946079in}}%
\pgfpathclose%
\pgfusepath{fill}%
\end{pgfscope}%
\begin{pgfscope}%
\pgfpathrectangle{\pgfqpoint{3.536584in}{0.147348in}}{\pgfqpoint{2.735294in}{2.735294in}}%
\pgfusepath{clip}%
\pgfsetbuttcap%
\pgfsetroundjoin%
\definecolor{currentfill}{rgb}{0.069261,0.265872,0.402159}%
\pgfsetfillcolor{currentfill}%
\pgfsetlinewidth{0.000000pt}%
\definecolor{currentstroke}{rgb}{0.000000,0.000000,0.000000}%
\pgfsetstrokecolor{currentstroke}%
\pgfsetdash{}{0pt}%
\pgfpathmoveto{\pgfqpoint{4.941195in}{1.947266in}}%
\pgfpathlineto{\pgfqpoint{4.816407in}{2.124855in}}%
\pgfpathlineto{\pgfqpoint{4.677670in}{1.946079in}}%
\pgfpathlineto{\pgfqpoint{4.941195in}{1.947266in}}%
\pgfpathclose%
\pgfusepath{fill}%
\end{pgfscope}%
\begin{pgfscope}%
\pgfpathrectangle{\pgfqpoint{3.536584in}{0.147348in}}{\pgfqpoint{2.735294in}{2.735294in}}%
\pgfusepath{clip}%
\pgfsetbuttcap%
\pgfsetroundjoin%
\definecolor{currentfill}{rgb}{0.049465,0.189883,0.287218}%
\pgfsetfillcolor{currentfill}%
\pgfsetlinewidth{0.000000pt}%
\definecolor{currentstroke}{rgb}{0.000000,0.000000,0.000000}%
\pgfsetstrokecolor{currentstroke}%
\pgfsetdash{}{0pt}%
\pgfpathmoveto{\pgfqpoint{4.579939in}{1.063020in}}%
\pgfpathlineto{\pgfqpoint{4.801285in}{1.337788in}}%
\pgfpathlineto{\pgfqpoint{4.661262in}{1.541647in}}%
\pgfpathlineto{\pgfqpoint{4.579939in}{1.063020in}}%
\pgfpathclose%
\pgfusepath{fill}%
\end{pgfscope}%
\begin{pgfscope}%
\pgfpathrectangle{\pgfqpoint{3.536584in}{0.147348in}}{\pgfqpoint{2.735294in}{2.735294in}}%
\pgfusepath{clip}%
\pgfsetbuttcap%
\pgfsetroundjoin%
\definecolor{currentfill}{rgb}{0.049465,0.189883,0.287218}%
\pgfsetfillcolor{currentfill}%
\pgfsetlinewidth{0.000000pt}%
\definecolor{currentstroke}{rgb}{0.000000,0.000000,0.000000}%
\pgfsetstrokecolor{currentstroke}%
\pgfsetdash{}{0pt}%
\pgfpathmoveto{\pgfqpoint{5.221128in}{1.541647in}}%
\pgfpathlineto{\pgfqpoint{5.081105in}{1.337788in}}%
\pgfpathlineto{\pgfqpoint{5.302451in}{1.063020in}}%
\pgfpathlineto{\pgfqpoint{5.221128in}{1.541647in}}%
\pgfpathclose%
\pgfusepath{fill}%
\end{pgfscope}%
\begin{pgfscope}%
\pgfpathrectangle{\pgfqpoint{3.536584in}{0.147348in}}{\pgfqpoint{2.735294in}{2.735294in}}%
\pgfusepath{clip}%
\pgfsetbuttcap%
\pgfsetroundjoin%
\definecolor{currentfill}{rgb}{0.061576,0.236373,0.357539}%
\pgfsetfillcolor{currentfill}%
\pgfsetlinewidth{0.000000pt}%
\definecolor{currentstroke}{rgb}{0.000000,0.000000,0.000000}%
\pgfsetstrokecolor{currentstroke}%
\pgfsetdash{}{0pt}%
\pgfpathmoveto{\pgfqpoint{4.661262in}{1.541647in}}%
\pgfpathlineto{\pgfqpoint{4.677670in}{1.946079in}}%
\pgfpathlineto{\pgfqpoint{4.414548in}{1.562087in}}%
\pgfpathlineto{\pgfqpoint{4.661262in}{1.541647in}}%
\pgfpathclose%
\pgfusepath{fill}%
\end{pgfscope}%
\begin{pgfscope}%
\pgfpathrectangle{\pgfqpoint{3.536584in}{0.147348in}}{\pgfqpoint{2.735294in}{2.735294in}}%
\pgfusepath{clip}%
\pgfsetbuttcap%
\pgfsetroundjoin%
\definecolor{currentfill}{rgb}{0.061576,0.236373,0.357539}%
\pgfsetfillcolor{currentfill}%
\pgfsetlinewidth{0.000000pt}%
\definecolor{currentstroke}{rgb}{0.000000,0.000000,0.000000}%
\pgfsetstrokecolor{currentstroke}%
\pgfsetdash{}{0pt}%
\pgfpathmoveto{\pgfqpoint{5.467842in}{1.562087in}}%
\pgfpathlineto{\pgfqpoint{5.204720in}{1.946079in}}%
\pgfpathlineto{\pgfqpoint{5.221128in}{1.541647in}}%
\pgfpathlineto{\pgfqpoint{5.467842in}{1.562087in}}%
\pgfpathclose%
\pgfusepath{fill}%
\end{pgfscope}%
\begin{pgfscope}%
\pgfpathrectangle{\pgfqpoint{3.536584in}{0.147348in}}{\pgfqpoint{2.735294in}{2.735294in}}%
\pgfusepath{clip}%
\pgfsetbuttcap%
\pgfsetroundjoin%
\definecolor{currentfill}{rgb}{0.053541,0.205528,0.310883}%
\pgfsetfillcolor{currentfill}%
\pgfsetlinewidth{0.000000pt}%
\definecolor{currentstroke}{rgb}{0.000000,0.000000,0.000000}%
\pgfsetstrokecolor{currentstroke}%
\pgfsetdash{}{0pt}%
\pgfpathmoveto{\pgfqpoint{4.414548in}{1.562087in}}%
\pgfpathlineto{\pgfqpoint{4.538549in}{1.360106in}}%
\pgfpathlineto{\pgfqpoint{4.661262in}{1.541647in}}%
\pgfpathlineto{\pgfqpoint{4.414548in}{1.562087in}}%
\pgfpathclose%
\pgfusepath{fill}%
\end{pgfscope}%
\begin{pgfscope}%
\pgfpathrectangle{\pgfqpoint{3.536584in}{0.147348in}}{\pgfqpoint{2.735294in}{2.735294in}}%
\pgfusepath{clip}%
\pgfsetbuttcap%
\pgfsetroundjoin%
\definecolor{currentfill}{rgb}{0.053541,0.205528,0.310883}%
\pgfsetfillcolor{currentfill}%
\pgfsetlinewidth{0.000000pt}%
\definecolor{currentstroke}{rgb}{0.000000,0.000000,0.000000}%
\pgfsetstrokecolor{currentstroke}%
\pgfsetdash{}{0pt}%
\pgfpathmoveto{\pgfqpoint{5.221128in}{1.541647in}}%
\pgfpathlineto{\pgfqpoint{5.343841in}{1.360106in}}%
\pgfpathlineto{\pgfqpoint{5.467842in}{1.562087in}}%
\pgfpathlineto{\pgfqpoint{5.221128in}{1.541647in}}%
\pgfpathclose%
\pgfusepath{fill}%
\end{pgfscope}%
\begin{pgfscope}%
\pgfpathrectangle{\pgfqpoint{3.536584in}{0.147348in}}{\pgfqpoint{2.735294in}{2.735294in}}%
\pgfusepath{clip}%
\pgfsetbuttcap%
\pgfsetroundjoin%
\definecolor{currentfill}{rgb}{0.060634,0.232757,0.352069}%
\pgfsetfillcolor{currentfill}%
\pgfsetlinewidth{0.000000pt}%
\definecolor{currentstroke}{rgb}{0.000000,0.000000,0.000000}%
\pgfsetstrokecolor{currentstroke}%
\pgfsetdash{}{0pt}%
\pgfpathmoveto{\pgfqpoint{4.677670in}{1.946079in}}%
\pgfpathlineto{\pgfqpoint{4.661262in}{1.541647in}}%
\pgfpathlineto{\pgfqpoint{4.941195in}{1.947266in}}%
\pgfpathlineto{\pgfqpoint{4.677670in}{1.946079in}}%
\pgfpathclose%
\pgfusepath{fill}%
\end{pgfscope}%
\begin{pgfscope}%
\pgfpathrectangle{\pgfqpoint{3.536584in}{0.147348in}}{\pgfqpoint{2.735294in}{2.735294in}}%
\pgfusepath{clip}%
\pgfsetbuttcap%
\pgfsetroundjoin%
\definecolor{currentfill}{rgb}{0.060634,0.232757,0.352069}%
\pgfsetfillcolor{currentfill}%
\pgfsetlinewidth{0.000000pt}%
\definecolor{currentstroke}{rgb}{0.000000,0.000000,0.000000}%
\pgfsetstrokecolor{currentstroke}%
\pgfsetdash{}{0pt}%
\pgfpathmoveto{\pgfqpoint{4.941195in}{1.947266in}}%
\pgfpathlineto{\pgfqpoint{5.221128in}{1.541647in}}%
\pgfpathlineto{\pgfqpoint{5.204720in}{1.946079in}}%
\pgfpathlineto{\pgfqpoint{4.941195in}{1.947266in}}%
\pgfpathclose%
\pgfusepath{fill}%
\end{pgfscope}%
\begin{pgfscope}%
\pgfpathrectangle{\pgfqpoint{3.536584in}{0.147348in}}{\pgfqpoint{2.735294in}{2.735294in}}%
\pgfusepath{clip}%
\pgfsetbuttcap%
\pgfsetroundjoin%
\definecolor{currentfill}{rgb}{0.060773,0.233289,0.352874}%
\pgfsetfillcolor{currentfill}%
\pgfsetlinewidth{0.000000pt}%
\definecolor{currentstroke}{rgb}{0.000000,0.000000,0.000000}%
\pgfsetstrokecolor{currentstroke}%
\pgfsetdash{}{0pt}%
\pgfpathmoveto{\pgfqpoint{4.941195in}{1.533948in}}%
\pgfpathlineto{\pgfqpoint{4.941195in}{1.947266in}}%
\pgfpathlineto{\pgfqpoint{4.661262in}{1.541647in}}%
\pgfpathlineto{\pgfqpoint{4.941195in}{1.533948in}}%
\pgfpathclose%
\pgfusepath{fill}%
\end{pgfscope}%
\begin{pgfscope}%
\pgfpathrectangle{\pgfqpoint{3.536584in}{0.147348in}}{\pgfqpoint{2.735294in}{2.735294in}}%
\pgfusepath{clip}%
\pgfsetbuttcap%
\pgfsetroundjoin%
\definecolor{currentfill}{rgb}{0.060773,0.233289,0.352874}%
\pgfsetfillcolor{currentfill}%
\pgfsetlinewidth{0.000000pt}%
\definecolor{currentstroke}{rgb}{0.000000,0.000000,0.000000}%
\pgfsetstrokecolor{currentstroke}%
\pgfsetdash{}{0pt}%
\pgfpathmoveto{\pgfqpoint{5.221128in}{1.541647in}}%
\pgfpathlineto{\pgfqpoint{4.941195in}{1.947266in}}%
\pgfpathlineto{\pgfqpoint{4.941195in}{1.533948in}}%
\pgfpathlineto{\pgfqpoint{5.221128in}{1.541647in}}%
\pgfpathclose%
\pgfusepath{fill}%
\end{pgfscope}%
\begin{pgfscope}%
\pgfpathrectangle{\pgfqpoint{3.536584in}{0.147348in}}{\pgfqpoint{2.735294in}{2.735294in}}%
\pgfusepath{clip}%
\pgfsetbuttcap%
\pgfsetroundjoin%
\definecolor{currentfill}{rgb}{0.052607,0.201942,0.305459}%
\pgfsetfillcolor{currentfill}%
\pgfsetlinewidth{0.000000pt}%
\definecolor{currentstroke}{rgb}{0.000000,0.000000,0.000000}%
\pgfsetstrokecolor{currentstroke}%
\pgfsetdash{}{0pt}%
\pgfpathmoveto{\pgfqpoint{4.661262in}{1.541647in}}%
\pgfpathlineto{\pgfqpoint{4.801285in}{1.337788in}}%
\pgfpathlineto{\pgfqpoint{4.941195in}{1.533948in}}%
\pgfpathlineto{\pgfqpoint{4.661262in}{1.541647in}}%
\pgfpathclose%
\pgfusepath{fill}%
\end{pgfscope}%
\begin{pgfscope}%
\pgfpathrectangle{\pgfqpoint{3.536584in}{0.147348in}}{\pgfqpoint{2.735294in}{2.735294in}}%
\pgfusepath{clip}%
\pgfsetbuttcap%
\pgfsetroundjoin%
\definecolor{currentfill}{rgb}{0.052607,0.201942,0.305459}%
\pgfsetfillcolor{currentfill}%
\pgfsetlinewidth{0.000000pt}%
\definecolor{currentstroke}{rgb}{0.000000,0.000000,0.000000}%
\pgfsetstrokecolor{currentstroke}%
\pgfsetdash{}{0pt}%
\pgfpathmoveto{\pgfqpoint{4.941195in}{1.533948in}}%
\pgfpathlineto{\pgfqpoint{5.081105in}{1.337788in}}%
\pgfpathlineto{\pgfqpoint{5.221128in}{1.541647in}}%
\pgfpathlineto{\pgfqpoint{4.941195in}{1.533948in}}%
\pgfpathclose%
\pgfusepath{fill}%
\end{pgfscope}%
\begin{pgfscope}%
\pgfpathrectangle{\pgfqpoint{3.536584in}{0.147348in}}{\pgfqpoint{2.735294in}{2.735294in}}%
\pgfusepath{clip}%
\pgfsetbuttcap%
\pgfsetroundjoin%
\definecolor{currentfill}{rgb}{0.839216,0.152941,0.156863}%
\pgfsetfillcolor{currentfill}%
\pgfsetfillopacity{0.300000}%
\pgfsetlinewidth{1.003750pt}%
\definecolor{currentstroke}{rgb}{0.839216,0.152941,0.156863}%
\pgfsetstrokecolor{currentstroke}%
\pgfsetstrokeopacity{0.300000}%
\pgfsetdash{}{0pt}%
\pgfpathmoveto{\pgfqpoint{4.347722in}{0.944929in}}%
\pgfpathcurveto{\pgfqpoint{4.357810in}{0.944929in}}{\pgfqpoint{4.367485in}{0.948937in}}{\pgfqpoint{4.374618in}{0.956070in}}%
\pgfpathcurveto{\pgfqpoint{4.381751in}{0.963203in}}{\pgfqpoint{4.385759in}{0.972878in}}{\pgfqpoint{4.385759in}{0.982966in}}%
\pgfpathcurveto{\pgfqpoint{4.385759in}{0.993053in}}{\pgfqpoint{4.381751in}{1.002729in}}{\pgfqpoint{4.374618in}{1.009861in}}%
\pgfpathcurveto{\pgfqpoint{4.367485in}{1.016994in}}{\pgfqpoint{4.357810in}{1.021002in}}{\pgfqpoint{4.347722in}{1.021002in}}%
\pgfpathcurveto{\pgfqpoint{4.337635in}{1.021002in}}{\pgfqpoint{4.327959in}{1.016994in}}{\pgfqpoint{4.320827in}{1.009861in}}%
\pgfpathcurveto{\pgfqpoint{4.313694in}{1.002729in}}{\pgfqpoint{4.309686in}{0.993053in}}{\pgfqpoint{4.309686in}{0.982966in}}%
\pgfpathcurveto{\pgfqpoint{4.309686in}{0.972878in}}{\pgfqpoint{4.313694in}{0.963203in}}{\pgfqpoint{4.320827in}{0.956070in}}%
\pgfpathcurveto{\pgfqpoint{4.327959in}{0.948937in}}{\pgfqpoint{4.337635in}{0.944929in}}{\pgfqpoint{4.347722in}{0.944929in}}%
\pgfpathlineto{\pgfqpoint{4.347722in}{0.944929in}}%
\pgfpathclose%
\pgfusepath{stroke,fill}%
\end{pgfscope}%
\begin{pgfscope}%
\pgfpathrectangle{\pgfqpoint{3.536584in}{0.147348in}}{\pgfqpoint{2.735294in}{2.735294in}}%
\pgfusepath{clip}%
\pgfsetbuttcap%
\pgfsetroundjoin%
\definecolor{currentfill}{rgb}{0.839216,0.152941,0.156863}%
\pgfsetfillcolor{currentfill}%
\pgfsetfillopacity{0.383610}%
\pgfsetlinewidth{1.003750pt}%
\definecolor{currentstroke}{rgb}{0.839216,0.152941,0.156863}%
\pgfsetstrokecolor{currentstroke}%
\pgfsetstrokeopacity{0.383610}%
\pgfsetdash{}{0pt}%
\pgfpathmoveto{\pgfqpoint{4.327859in}{1.009596in}}%
\pgfpathcurveto{\pgfqpoint{4.337946in}{1.009596in}}{\pgfqpoint{4.347622in}{1.013604in}}{\pgfqpoint{4.354755in}{1.020737in}}%
\pgfpathcurveto{\pgfqpoint{4.361888in}{1.027870in}}{\pgfqpoint{4.365895in}{1.037545in}}{\pgfqpoint{4.365895in}{1.047633in}}%
\pgfpathcurveto{\pgfqpoint{4.365895in}{1.057720in}}{\pgfqpoint{4.361888in}{1.067396in}}{\pgfqpoint{4.354755in}{1.074528in}}%
\pgfpathcurveto{\pgfqpoint{4.347622in}{1.081661in}}{\pgfqpoint{4.337946in}{1.085669in}}{\pgfqpoint{4.327859in}{1.085669in}}%
\pgfpathcurveto{\pgfqpoint{4.317772in}{1.085669in}}{\pgfqpoint{4.308096in}{1.081661in}}{\pgfqpoint{4.300963in}{1.074528in}}%
\pgfpathcurveto{\pgfqpoint{4.293831in}{1.067396in}}{\pgfqpoint{4.289823in}{1.057720in}}{\pgfqpoint{4.289823in}{1.047633in}}%
\pgfpathcurveto{\pgfqpoint{4.289823in}{1.037545in}}{\pgfqpoint{4.293831in}{1.027870in}}{\pgfqpoint{4.300963in}{1.020737in}}%
\pgfpathcurveto{\pgfqpoint{4.308096in}{1.013604in}}{\pgfqpoint{4.317772in}{1.009596in}}{\pgfqpoint{4.327859in}{1.009596in}}%
\pgfpathlineto{\pgfqpoint{4.327859in}{1.009596in}}%
\pgfpathclose%
\pgfusepath{stroke,fill}%
\end{pgfscope}%
\begin{pgfscope}%
\pgfpathrectangle{\pgfqpoint{3.536584in}{0.147348in}}{\pgfqpoint{2.735294in}{2.735294in}}%
\pgfusepath{clip}%
\pgfsetbuttcap%
\pgfsetroundjoin%
\definecolor{currentfill}{rgb}{0.839216,0.152941,0.156863}%
\pgfsetfillcolor{currentfill}%
\pgfsetfillopacity{0.457533}%
\pgfsetlinewidth{1.003750pt}%
\definecolor{currentstroke}{rgb}{0.839216,0.152941,0.156863}%
\pgfsetstrokecolor{currentstroke}%
\pgfsetstrokeopacity{0.457533}%
\pgfsetdash{}{0pt}%
\pgfpathmoveto{\pgfqpoint{4.483875in}{0.962534in}}%
\pgfpathcurveto{\pgfqpoint{4.493962in}{0.962534in}}{\pgfqpoint{4.503637in}{0.966542in}}{\pgfqpoint{4.510770in}{0.973675in}}%
\pgfpathcurveto{\pgfqpoint{4.517903in}{0.980808in}}{\pgfqpoint{4.521911in}{0.990483in}}{\pgfqpoint{4.521911in}{1.000570in}}%
\pgfpathcurveto{\pgfqpoint{4.521911in}{1.010658in}}{\pgfqpoint{4.517903in}{1.020333in}}{\pgfqpoint{4.510770in}{1.027466in}}%
\pgfpathcurveto{\pgfqpoint{4.503637in}{1.034599in}}{\pgfqpoint{4.493962in}{1.038607in}}{\pgfqpoint{4.483875in}{1.038607in}}%
\pgfpathcurveto{\pgfqpoint{4.473787in}{1.038607in}}{\pgfqpoint{4.464112in}{1.034599in}}{\pgfqpoint{4.456979in}{1.027466in}}%
\pgfpathcurveto{\pgfqpoint{4.449846in}{1.020333in}}{\pgfqpoint{4.445838in}{1.010658in}}{\pgfqpoint{4.445838in}{1.000570in}}%
\pgfpathcurveto{\pgfqpoint{4.445838in}{0.990483in}}{\pgfqpoint{4.449846in}{0.980808in}}{\pgfqpoint{4.456979in}{0.973675in}}%
\pgfpathcurveto{\pgfqpoint{4.464112in}{0.966542in}}{\pgfqpoint{4.473787in}{0.962534in}}{\pgfqpoint{4.483875in}{0.962534in}}%
\pgfpathlineto{\pgfqpoint{4.483875in}{0.962534in}}%
\pgfpathclose%
\pgfusepath{stroke,fill}%
\end{pgfscope}%
\begin{pgfscope}%
\pgfpathrectangle{\pgfqpoint{3.536584in}{0.147348in}}{\pgfqpoint{2.735294in}{2.735294in}}%
\pgfusepath{clip}%
\pgfsetbuttcap%
\pgfsetroundjoin%
\definecolor{currentfill}{rgb}{0.839216,0.152941,0.156863}%
\pgfsetfillcolor{currentfill}%
\pgfsetfillopacity{0.492303}%
\pgfsetlinewidth{1.003750pt}%
\definecolor{currentstroke}{rgb}{0.839216,0.152941,0.156863}%
\pgfsetstrokecolor{currentstroke}%
\pgfsetstrokeopacity{0.492303}%
\pgfsetdash{}{0pt}%
\pgfpathmoveto{\pgfqpoint{4.160550in}{1.511653in}}%
\pgfpathcurveto{\pgfqpoint{4.170638in}{1.511653in}}{\pgfqpoint{4.180313in}{1.515661in}}{\pgfqpoint{4.187446in}{1.522794in}}%
\pgfpathcurveto{\pgfqpoint{4.194579in}{1.529927in}}{\pgfqpoint{4.198587in}{1.539602in}}{\pgfqpoint{4.198587in}{1.549689in}}%
\pgfpathcurveto{\pgfqpoint{4.198587in}{1.559777in}}{\pgfqpoint{4.194579in}{1.569452in}}{\pgfqpoint{4.187446in}{1.576585in}}%
\pgfpathcurveto{\pgfqpoint{4.180313in}{1.583718in}}{\pgfqpoint{4.170638in}{1.587726in}}{\pgfqpoint{4.160550in}{1.587726in}}%
\pgfpathcurveto{\pgfqpoint{4.150463in}{1.587726in}}{\pgfqpoint{4.140788in}{1.583718in}}{\pgfqpoint{4.133655in}{1.576585in}}%
\pgfpathcurveto{\pgfqpoint{4.126522in}{1.569452in}}{\pgfqpoint{4.122514in}{1.559777in}}{\pgfqpoint{4.122514in}{1.549689in}}%
\pgfpathcurveto{\pgfqpoint{4.122514in}{1.539602in}}{\pgfqpoint{4.126522in}{1.529927in}}{\pgfqpoint{4.133655in}{1.522794in}}%
\pgfpathcurveto{\pgfqpoint{4.140788in}{1.515661in}}{\pgfqpoint{4.150463in}{1.511653in}}{\pgfqpoint{4.160550in}{1.511653in}}%
\pgfpathlineto{\pgfqpoint{4.160550in}{1.511653in}}%
\pgfpathclose%
\pgfusepath{stroke,fill}%
\end{pgfscope}%
\begin{pgfscope}%
\pgfpathrectangle{\pgfqpoint{3.536584in}{0.147348in}}{\pgfqpoint{2.735294in}{2.735294in}}%
\pgfusepath{clip}%
\pgfsetbuttcap%
\pgfsetroundjoin%
\definecolor{currentfill}{rgb}{0.839216,0.152941,0.156863}%
\pgfsetfillcolor{currentfill}%
\pgfsetfillopacity{0.498590}%
\pgfsetlinewidth{1.003750pt}%
\definecolor{currentstroke}{rgb}{0.839216,0.152941,0.156863}%
\pgfsetstrokecolor{currentstroke}%
\pgfsetstrokeopacity{0.498590}%
\pgfsetdash{}{0pt}%
\pgfpathmoveto{\pgfqpoint{5.807826in}{1.621636in}}%
\pgfpathcurveto{\pgfqpoint{5.817913in}{1.621636in}}{\pgfqpoint{5.827588in}{1.625643in}}{\pgfqpoint{5.834721in}{1.632776in}}%
\pgfpathcurveto{\pgfqpoint{5.841854in}{1.639909in}}{\pgfqpoint{5.845862in}{1.649585in}}{\pgfqpoint{5.845862in}{1.659672in}}%
\pgfpathcurveto{\pgfqpoint{5.845862in}{1.669759in}}{\pgfqpoint{5.841854in}{1.679435in}}{\pgfqpoint{5.834721in}{1.686568in}}%
\pgfpathcurveto{\pgfqpoint{5.827588in}{1.693701in}}{\pgfqpoint{5.817913in}{1.697708in}}{\pgfqpoint{5.807826in}{1.697708in}}%
\pgfpathcurveto{\pgfqpoint{5.797738in}{1.697708in}}{\pgfqpoint{5.788063in}{1.693701in}}{\pgfqpoint{5.780930in}{1.686568in}}%
\pgfpathcurveto{\pgfqpoint{5.773797in}{1.679435in}}{\pgfqpoint{5.769789in}{1.669759in}}{\pgfqpoint{5.769789in}{1.659672in}}%
\pgfpathcurveto{\pgfqpoint{5.769789in}{1.649585in}}{\pgfqpoint{5.773797in}{1.639909in}}{\pgfqpoint{5.780930in}{1.632776in}}%
\pgfpathcurveto{\pgfqpoint{5.788063in}{1.625643in}}{\pgfqpoint{5.797738in}{1.621636in}}{\pgfqpoint{5.807826in}{1.621636in}}%
\pgfpathlineto{\pgfqpoint{5.807826in}{1.621636in}}%
\pgfpathclose%
\pgfusepath{stroke,fill}%
\end{pgfscope}%
\begin{pgfscope}%
\pgfpathrectangle{\pgfqpoint{3.536584in}{0.147348in}}{\pgfqpoint{2.735294in}{2.735294in}}%
\pgfusepath{clip}%
\pgfsetbuttcap%
\pgfsetroundjoin%
\definecolor{currentfill}{rgb}{0.839216,0.152941,0.156863}%
\pgfsetfillcolor{currentfill}%
\pgfsetfillopacity{0.612876}%
\pgfsetlinewidth{1.003750pt}%
\definecolor{currentstroke}{rgb}{0.839216,0.152941,0.156863}%
\pgfsetstrokecolor{currentstroke}%
\pgfsetstrokeopacity{0.612876}%
\pgfsetdash{}{0pt}%
\pgfpathmoveto{\pgfqpoint{4.496123in}{0.979547in}}%
\pgfpathcurveto{\pgfqpoint{4.506210in}{0.979547in}}{\pgfqpoint{4.515886in}{0.983554in}}{\pgfqpoint{4.523019in}{0.990687in}}%
\pgfpathcurveto{\pgfqpoint{4.530151in}{0.997820in}}{\pgfqpoint{4.534159in}{1.007495in}}{\pgfqpoint{4.534159in}{1.017583in}}%
\pgfpathcurveto{\pgfqpoint{4.534159in}{1.027670in}}{\pgfqpoint{4.530151in}{1.037346in}}{\pgfqpoint{4.523019in}{1.044479in}}%
\pgfpathcurveto{\pgfqpoint{4.515886in}{1.051611in}}{\pgfqpoint{4.506210in}{1.055619in}}{\pgfqpoint{4.496123in}{1.055619in}}%
\pgfpathcurveto{\pgfqpoint{4.486036in}{1.055619in}}{\pgfqpoint{4.476360in}{1.051611in}}{\pgfqpoint{4.469227in}{1.044479in}}%
\pgfpathcurveto{\pgfqpoint{4.462094in}{1.037346in}}{\pgfqpoint{4.458087in}{1.027670in}}{\pgfqpoint{4.458087in}{1.017583in}}%
\pgfpathcurveto{\pgfqpoint{4.458087in}{1.007495in}}{\pgfqpoint{4.462094in}{0.997820in}}{\pgfqpoint{4.469227in}{0.990687in}}%
\pgfpathcurveto{\pgfqpoint{4.476360in}{0.983554in}}{\pgfqpoint{4.486036in}{0.979547in}}{\pgfqpoint{4.496123in}{0.979547in}}%
\pgfpathlineto{\pgfqpoint{4.496123in}{0.979547in}}%
\pgfpathclose%
\pgfusepath{stroke,fill}%
\end{pgfscope}%
\begin{pgfscope}%
\pgfpathrectangle{\pgfqpoint{3.536584in}{0.147348in}}{\pgfqpoint{2.735294in}{2.735294in}}%
\pgfusepath{clip}%
\pgfsetbuttcap%
\pgfsetroundjoin%
\definecolor{currentfill}{rgb}{0.839216,0.152941,0.156863}%
\pgfsetfillcolor{currentfill}%
\pgfsetfillopacity{0.625674}%
\pgfsetlinewidth{1.003750pt}%
\definecolor{currentstroke}{rgb}{0.839216,0.152941,0.156863}%
\pgfsetstrokecolor{currentstroke}%
\pgfsetstrokeopacity{0.625674}%
\pgfsetdash{}{0pt}%
\pgfpathmoveto{\pgfqpoint{5.375023in}{2.164239in}}%
\pgfpathcurveto{\pgfqpoint{5.385111in}{2.164239in}}{\pgfqpoint{5.394786in}{2.168247in}}{\pgfqpoint{5.401919in}{2.175380in}}%
\pgfpathcurveto{\pgfqpoint{5.409052in}{2.182513in}}{\pgfqpoint{5.413059in}{2.192188in}}{\pgfqpoint{5.413059in}{2.202275in}}%
\pgfpathcurveto{\pgfqpoint{5.413059in}{2.212363in}}{\pgfqpoint{5.409052in}{2.222038in}}{\pgfqpoint{5.401919in}{2.229171in}}%
\pgfpathcurveto{\pgfqpoint{5.394786in}{2.236304in}}{\pgfqpoint{5.385111in}{2.240312in}}{\pgfqpoint{5.375023in}{2.240312in}}%
\pgfpathcurveto{\pgfqpoint{5.364936in}{2.240312in}}{\pgfqpoint{5.355260in}{2.236304in}}{\pgfqpoint{5.348127in}{2.229171in}}%
\pgfpathcurveto{\pgfqpoint{5.340995in}{2.222038in}}{\pgfqpoint{5.336987in}{2.212363in}}{\pgfqpoint{5.336987in}{2.202275in}}%
\pgfpathcurveto{\pgfqpoint{5.336987in}{2.192188in}}{\pgfqpoint{5.340995in}{2.182513in}}{\pgfqpoint{5.348127in}{2.175380in}}%
\pgfpathcurveto{\pgfqpoint{5.355260in}{2.168247in}}{\pgfqpoint{5.364936in}{2.164239in}}{\pgfqpoint{5.375023in}{2.164239in}}%
\pgfpathlineto{\pgfqpoint{5.375023in}{2.164239in}}%
\pgfpathclose%
\pgfusepath{stroke,fill}%
\end{pgfscope}%
\begin{pgfscope}%
\pgfpathrectangle{\pgfqpoint{3.536584in}{0.147348in}}{\pgfqpoint{2.735294in}{2.735294in}}%
\pgfusepath{clip}%
\pgfsetbuttcap%
\pgfsetroundjoin%
\definecolor{currentfill}{rgb}{0.839216,0.152941,0.156863}%
\pgfsetfillcolor{currentfill}%
\pgfsetfillopacity{0.631635}%
\pgfsetlinewidth{1.003750pt}%
\definecolor{currentstroke}{rgb}{0.839216,0.152941,0.156863}%
\pgfsetstrokecolor{currentstroke}%
\pgfsetstrokeopacity{0.631635}%
\pgfsetdash{}{0pt}%
\pgfpathmoveto{\pgfqpoint{5.366125in}{2.104757in}}%
\pgfpathcurveto{\pgfqpoint{5.376212in}{2.104757in}}{\pgfqpoint{5.385888in}{2.108765in}}{\pgfqpoint{5.393021in}{2.115898in}}%
\pgfpathcurveto{\pgfqpoint{5.400154in}{2.123030in}}{\pgfqpoint{5.404161in}{2.132706in}}{\pgfqpoint{5.404161in}{2.142793in}}%
\pgfpathcurveto{\pgfqpoint{5.404161in}{2.152881in}}{\pgfqpoint{5.400154in}{2.162556in}}{\pgfqpoint{5.393021in}{2.169689in}}%
\pgfpathcurveto{\pgfqpoint{5.385888in}{2.176822in}}{\pgfqpoint{5.376212in}{2.180830in}}{\pgfqpoint{5.366125in}{2.180830in}}%
\pgfpathcurveto{\pgfqpoint{5.356038in}{2.180830in}}{\pgfqpoint{5.346362in}{2.176822in}}{\pgfqpoint{5.339229in}{2.169689in}}%
\pgfpathcurveto{\pgfqpoint{5.332097in}{2.162556in}}{\pgfqpoint{5.328089in}{2.152881in}}{\pgfqpoint{5.328089in}{2.142793in}}%
\pgfpathcurveto{\pgfqpoint{5.328089in}{2.132706in}}{\pgfqpoint{5.332097in}{2.123030in}}{\pgfqpoint{5.339229in}{2.115898in}}%
\pgfpathcurveto{\pgfqpoint{5.346362in}{2.108765in}}{\pgfqpoint{5.356038in}{2.104757in}}{\pgfqpoint{5.366125in}{2.104757in}}%
\pgfpathlineto{\pgfqpoint{5.366125in}{2.104757in}}%
\pgfpathclose%
\pgfusepath{stroke,fill}%
\end{pgfscope}%
\begin{pgfscope}%
\pgfpathrectangle{\pgfqpoint{3.536584in}{0.147348in}}{\pgfqpoint{2.735294in}{2.735294in}}%
\pgfusepath{clip}%
\pgfsetbuttcap%
\pgfsetroundjoin%
\definecolor{currentfill}{rgb}{0.839216,0.152941,0.156863}%
\pgfsetfillcolor{currentfill}%
\pgfsetfillopacity{0.634032}%
\pgfsetlinewidth{1.003750pt}%
\definecolor{currentstroke}{rgb}{0.839216,0.152941,0.156863}%
\pgfsetstrokecolor{currentstroke}%
\pgfsetstrokeopacity{0.634032}%
\pgfsetdash{}{0pt}%
\pgfpathmoveto{\pgfqpoint{5.637120in}{1.533362in}}%
\pgfpathcurveto{\pgfqpoint{5.647207in}{1.533362in}}{\pgfqpoint{5.656883in}{1.537370in}}{\pgfqpoint{5.664016in}{1.544503in}}%
\pgfpathcurveto{\pgfqpoint{5.671149in}{1.551635in}}{\pgfqpoint{5.675156in}{1.561311in}}{\pgfqpoint{5.675156in}{1.571398in}}%
\pgfpathcurveto{\pgfqpoint{5.675156in}{1.581486in}}{\pgfqpoint{5.671149in}{1.591161in}}{\pgfqpoint{5.664016in}{1.598294in}}%
\pgfpathcurveto{\pgfqpoint{5.656883in}{1.605427in}}{\pgfqpoint{5.647207in}{1.609435in}}{\pgfqpoint{5.637120in}{1.609435in}}%
\pgfpathcurveto{\pgfqpoint{5.627033in}{1.609435in}}{\pgfqpoint{5.617357in}{1.605427in}}{\pgfqpoint{5.610224in}{1.598294in}}%
\pgfpathcurveto{\pgfqpoint{5.603091in}{1.591161in}}{\pgfqpoint{5.599084in}{1.581486in}}{\pgfqpoint{5.599084in}{1.571398in}}%
\pgfpathcurveto{\pgfqpoint{5.599084in}{1.561311in}}{\pgfqpoint{5.603091in}{1.551635in}}{\pgfqpoint{5.610224in}{1.544503in}}%
\pgfpathcurveto{\pgfqpoint{5.617357in}{1.537370in}}{\pgfqpoint{5.627033in}{1.533362in}}{\pgfqpoint{5.637120in}{1.533362in}}%
\pgfpathlineto{\pgfqpoint{5.637120in}{1.533362in}}%
\pgfpathclose%
\pgfusepath{stroke,fill}%
\end{pgfscope}%
\begin{pgfscope}%
\pgfpathrectangle{\pgfqpoint{3.536584in}{0.147348in}}{\pgfqpoint{2.735294in}{2.735294in}}%
\pgfusepath{clip}%
\pgfsetbuttcap%
\pgfsetroundjoin%
\definecolor{currentfill}{rgb}{0.839216,0.152941,0.156863}%
\pgfsetfillcolor{currentfill}%
\pgfsetfillopacity{0.652064}%
\pgfsetlinewidth{1.003750pt}%
\definecolor{currentstroke}{rgb}{0.839216,0.152941,0.156863}%
\pgfsetstrokecolor{currentstroke}%
\pgfsetstrokeopacity{0.652064}%
\pgfsetdash{}{0pt}%
\pgfpathmoveto{\pgfqpoint{4.750408in}{0.952320in}}%
\pgfpathcurveto{\pgfqpoint{4.760495in}{0.952320in}}{\pgfqpoint{4.770171in}{0.956328in}}{\pgfqpoint{4.777304in}{0.963461in}}%
\pgfpathcurveto{\pgfqpoint{4.784437in}{0.970593in}}{\pgfqpoint{4.788444in}{0.980269in}}{\pgfqpoint{4.788444in}{0.990356in}}%
\pgfpathcurveto{\pgfqpoint{4.788444in}{1.000444in}}{\pgfqpoint{4.784437in}{1.010119in}}{\pgfqpoint{4.777304in}{1.017252in}}%
\pgfpathcurveto{\pgfqpoint{4.770171in}{1.024385in}}{\pgfqpoint{4.760495in}{1.028393in}}{\pgfqpoint{4.750408in}{1.028393in}}%
\pgfpathcurveto{\pgfqpoint{4.740321in}{1.028393in}}{\pgfqpoint{4.730645in}{1.024385in}}{\pgfqpoint{4.723512in}{1.017252in}}%
\pgfpathcurveto{\pgfqpoint{4.716379in}{1.010119in}}{\pgfqpoint{4.712372in}{1.000444in}}{\pgfqpoint{4.712372in}{0.990356in}}%
\pgfpathcurveto{\pgfqpoint{4.712372in}{0.980269in}}{\pgfqpoint{4.716379in}{0.970593in}}{\pgfqpoint{4.723512in}{0.963461in}}%
\pgfpathcurveto{\pgfqpoint{4.730645in}{0.956328in}}{\pgfqpoint{4.740321in}{0.952320in}}{\pgfqpoint{4.750408in}{0.952320in}}%
\pgfpathlineto{\pgfqpoint{4.750408in}{0.952320in}}%
\pgfpathclose%
\pgfusepath{stroke,fill}%
\end{pgfscope}%
\begin{pgfscope}%
\pgfpathrectangle{\pgfqpoint{3.536584in}{0.147348in}}{\pgfqpoint{2.735294in}{2.735294in}}%
\pgfusepath{clip}%
\pgfsetbuttcap%
\pgfsetroundjoin%
\definecolor{currentfill}{rgb}{0.839216,0.152941,0.156863}%
\pgfsetfillcolor{currentfill}%
\pgfsetfillopacity{0.738747}%
\pgfsetlinewidth{1.003750pt}%
\definecolor{currentstroke}{rgb}{0.839216,0.152941,0.156863}%
\pgfsetstrokecolor{currentstroke}%
\pgfsetstrokeopacity{0.738747}%
\pgfsetdash{}{0pt}%
\pgfpathmoveto{\pgfqpoint{4.140523in}{1.507716in}}%
\pgfpathcurveto{\pgfqpoint{4.150611in}{1.507716in}}{\pgfqpoint{4.160286in}{1.511723in}}{\pgfqpoint{4.167419in}{1.518856in}}%
\pgfpathcurveto{\pgfqpoint{4.174552in}{1.525989in}}{\pgfqpoint{4.178560in}{1.535665in}}{\pgfqpoint{4.178560in}{1.545752in}}%
\pgfpathcurveto{\pgfqpoint{4.178560in}{1.555839in}}{\pgfqpoint{4.174552in}{1.565515in}}{\pgfqpoint{4.167419in}{1.572648in}}%
\pgfpathcurveto{\pgfqpoint{4.160286in}{1.579780in}}{\pgfqpoint{4.150611in}{1.583788in}}{\pgfqpoint{4.140523in}{1.583788in}}%
\pgfpathcurveto{\pgfqpoint{4.130436in}{1.583788in}}{\pgfqpoint{4.120761in}{1.579780in}}{\pgfqpoint{4.113628in}{1.572648in}}%
\pgfpathcurveto{\pgfqpoint{4.106495in}{1.565515in}}{\pgfqpoint{4.102487in}{1.555839in}}{\pgfqpoint{4.102487in}{1.545752in}}%
\pgfpathcurveto{\pgfqpoint{4.102487in}{1.535665in}}{\pgfqpoint{4.106495in}{1.525989in}}{\pgfqpoint{4.113628in}{1.518856in}}%
\pgfpathcurveto{\pgfqpoint{4.120761in}{1.511723in}}{\pgfqpoint{4.130436in}{1.507716in}}{\pgfqpoint{4.140523in}{1.507716in}}%
\pgfpathlineto{\pgfqpoint{4.140523in}{1.507716in}}%
\pgfpathclose%
\pgfusepath{stroke,fill}%
\end{pgfscope}%
\begin{pgfscope}%
\pgfpathrectangle{\pgfqpoint{3.536584in}{0.147348in}}{\pgfqpoint{2.735294in}{2.735294in}}%
\pgfusepath{clip}%
\pgfsetbuttcap%
\pgfsetroundjoin%
\definecolor{currentfill}{rgb}{0.839216,0.152941,0.156863}%
\pgfsetfillcolor{currentfill}%
\pgfsetfillopacity{0.791813}%
\pgfsetlinewidth{1.003750pt}%
\definecolor{currentstroke}{rgb}{0.839216,0.152941,0.156863}%
\pgfsetstrokecolor{currentstroke}%
\pgfsetstrokeopacity{0.791813}%
\pgfsetdash{}{0pt}%
\pgfpathmoveto{\pgfqpoint{4.633661in}{2.014183in}}%
\pgfpathcurveto{\pgfqpoint{4.643748in}{2.014183in}}{\pgfqpoint{4.653424in}{2.018191in}}{\pgfqpoint{4.660557in}{2.025323in}}%
\pgfpathcurveto{\pgfqpoint{4.667690in}{2.032456in}}{\pgfqpoint{4.671697in}{2.042132in}}{\pgfqpoint{4.671697in}{2.052219in}}%
\pgfpathcurveto{\pgfqpoint{4.671697in}{2.062307in}}{\pgfqpoint{4.667690in}{2.071982in}}{\pgfqpoint{4.660557in}{2.079115in}}%
\pgfpathcurveto{\pgfqpoint{4.653424in}{2.086248in}}{\pgfqpoint{4.643748in}{2.090255in}}{\pgfqpoint{4.633661in}{2.090255in}}%
\pgfpathcurveto{\pgfqpoint{4.623574in}{2.090255in}}{\pgfqpoint{4.613898in}{2.086248in}}{\pgfqpoint{4.606765in}{2.079115in}}%
\pgfpathcurveto{\pgfqpoint{4.599632in}{2.071982in}}{\pgfqpoint{4.595625in}{2.062307in}}{\pgfqpoint{4.595625in}{2.052219in}}%
\pgfpathcurveto{\pgfqpoint{4.595625in}{2.042132in}}{\pgfqpoint{4.599632in}{2.032456in}}{\pgfqpoint{4.606765in}{2.025323in}}%
\pgfpathcurveto{\pgfqpoint{4.613898in}{2.018191in}}{\pgfqpoint{4.623574in}{2.014183in}}{\pgfqpoint{4.633661in}{2.014183in}}%
\pgfpathlineto{\pgfqpoint{4.633661in}{2.014183in}}%
\pgfpathclose%
\pgfusepath{stroke,fill}%
\end{pgfscope}%
\begin{pgfscope}%
\pgfpathrectangle{\pgfqpoint{3.536584in}{0.147348in}}{\pgfqpoint{2.735294in}{2.735294in}}%
\pgfusepath{clip}%
\pgfsetbuttcap%
\pgfsetroundjoin%
\definecolor{currentfill}{rgb}{0.839216,0.152941,0.156863}%
\pgfsetfillcolor{currentfill}%
\pgfsetfillopacity{0.864233}%
\pgfsetlinewidth{1.003750pt}%
\definecolor{currentstroke}{rgb}{0.839216,0.152941,0.156863}%
\pgfsetstrokecolor{currentstroke}%
\pgfsetstrokeopacity{0.864233}%
\pgfsetdash{}{0pt}%
\pgfpathmoveto{\pgfqpoint{5.299988in}{1.926870in}}%
\pgfpathcurveto{\pgfqpoint{5.310075in}{1.926870in}}{\pgfqpoint{5.319750in}{1.930878in}}{\pgfqpoint{5.326883in}{1.938011in}}%
\pgfpathcurveto{\pgfqpoint{5.334016in}{1.945143in}}{\pgfqpoint{5.338024in}{1.954819in}}{\pgfqpoint{5.338024in}{1.964906in}}%
\pgfpathcurveto{\pgfqpoint{5.338024in}{1.974994in}}{\pgfqpoint{5.334016in}{1.984669in}}{\pgfqpoint{5.326883in}{1.991802in}}%
\pgfpathcurveto{\pgfqpoint{5.319750in}{1.998935in}}{\pgfqpoint{5.310075in}{2.002943in}}{\pgfqpoint{5.299988in}{2.002943in}}%
\pgfpathcurveto{\pgfqpoint{5.289900in}{2.002943in}}{\pgfqpoint{5.280225in}{1.998935in}}{\pgfqpoint{5.273092in}{1.991802in}}%
\pgfpathcurveto{\pgfqpoint{5.265959in}{1.984669in}}{\pgfqpoint{5.261951in}{1.974994in}}{\pgfqpoint{5.261951in}{1.964906in}}%
\pgfpathcurveto{\pgfqpoint{5.261951in}{1.954819in}}{\pgfqpoint{5.265959in}{1.945143in}}{\pgfqpoint{5.273092in}{1.938011in}}%
\pgfpathcurveto{\pgfqpoint{5.280225in}{1.930878in}}{\pgfqpoint{5.289900in}{1.926870in}}{\pgfqpoint{5.299988in}{1.926870in}}%
\pgfpathlineto{\pgfqpoint{5.299988in}{1.926870in}}%
\pgfpathclose%
\pgfusepath{stroke,fill}%
\end{pgfscope}%
\begin{pgfscope}%
\pgfpathrectangle{\pgfqpoint{3.536584in}{0.147348in}}{\pgfqpoint{2.735294in}{2.735294in}}%
\pgfusepath{clip}%
\pgfsetbuttcap%
\pgfsetroundjoin%
\definecolor{currentfill}{rgb}{0.839216,0.152941,0.156863}%
\pgfsetfillcolor{currentfill}%
\pgfsetfillopacity{0.929084}%
\pgfsetlinewidth{1.003750pt}%
\definecolor{currentstroke}{rgb}{0.839216,0.152941,0.156863}%
\pgfsetstrokecolor{currentstroke}%
\pgfsetstrokeopacity{0.929084}%
\pgfsetdash{}{0pt}%
\pgfpathmoveto{\pgfqpoint{5.359400in}{1.288920in}}%
\pgfpathcurveto{\pgfqpoint{5.369488in}{1.288920in}}{\pgfqpoint{5.379163in}{1.292928in}}{\pgfqpoint{5.386296in}{1.300060in}}%
\pgfpathcurveto{\pgfqpoint{5.393429in}{1.307193in}}{\pgfqpoint{5.397437in}{1.316869in}}{\pgfqpoint{5.397437in}{1.326956in}}%
\pgfpathcurveto{\pgfqpoint{5.397437in}{1.337043in}}{\pgfqpoint{5.393429in}{1.346719in}}{\pgfqpoint{5.386296in}{1.353852in}}%
\pgfpathcurveto{\pgfqpoint{5.379163in}{1.360985in}}{\pgfqpoint{5.369488in}{1.364992in}}{\pgfqpoint{5.359400in}{1.364992in}}%
\pgfpathcurveto{\pgfqpoint{5.349313in}{1.364992in}}{\pgfqpoint{5.339637in}{1.360985in}}{\pgfqpoint{5.332505in}{1.353852in}}%
\pgfpathcurveto{\pgfqpoint{5.325372in}{1.346719in}}{\pgfqpoint{5.321364in}{1.337043in}}{\pgfqpoint{5.321364in}{1.326956in}}%
\pgfpathcurveto{\pgfqpoint{5.321364in}{1.316869in}}{\pgfqpoint{5.325372in}{1.307193in}}{\pgfqpoint{5.332505in}{1.300060in}}%
\pgfpathcurveto{\pgfqpoint{5.339637in}{1.292928in}}{\pgfqpoint{5.349313in}{1.288920in}}{\pgfqpoint{5.359400in}{1.288920in}}%
\pgfpathlineto{\pgfqpoint{5.359400in}{1.288920in}}%
\pgfpathclose%
\pgfusepath{stroke,fill}%
\end{pgfscope}%
\begin{pgfscope}%
\pgfpathrectangle{\pgfqpoint{3.536584in}{0.147348in}}{\pgfqpoint{2.735294in}{2.735294in}}%
\pgfusepath{clip}%
\pgfsetbuttcap%
\pgfsetroundjoin%
\definecolor{currentfill}{rgb}{0.839216,0.152941,0.156863}%
\pgfsetfillcolor{currentfill}%
\pgfsetlinewidth{1.003750pt}%
\definecolor{currentstroke}{rgb}{0.839216,0.152941,0.156863}%
\pgfsetstrokecolor{currentstroke}%
\pgfsetdash{}{0pt}%
\pgfpathmoveto{\pgfqpoint{5.239366in}{1.506712in}}%
\pgfpathcurveto{\pgfqpoint{5.249453in}{1.506712in}}{\pgfqpoint{5.259128in}{1.510720in}}{\pgfqpoint{5.266261in}{1.517853in}}%
\pgfpathcurveto{\pgfqpoint{5.273394in}{1.524986in}}{\pgfqpoint{5.277402in}{1.534661in}}{\pgfqpoint{5.277402in}{1.544748in}}%
\pgfpathcurveto{\pgfqpoint{5.277402in}{1.554836in}}{\pgfqpoint{5.273394in}{1.564511in}}{\pgfqpoint{5.266261in}{1.571644in}}%
\pgfpathcurveto{\pgfqpoint{5.259128in}{1.578777in}}{\pgfqpoint{5.249453in}{1.582785in}}{\pgfqpoint{5.239366in}{1.582785in}}%
\pgfpathcurveto{\pgfqpoint{5.229278in}{1.582785in}}{\pgfqpoint{5.219603in}{1.578777in}}{\pgfqpoint{5.212470in}{1.571644in}}%
\pgfpathcurveto{\pgfqpoint{5.205337in}{1.564511in}}{\pgfqpoint{5.201329in}{1.554836in}}{\pgfqpoint{5.201329in}{1.544748in}}%
\pgfpathcurveto{\pgfqpoint{5.201329in}{1.534661in}}{\pgfqpoint{5.205337in}{1.524986in}}{\pgfqpoint{5.212470in}{1.517853in}}%
\pgfpathcurveto{\pgfqpoint{5.219603in}{1.510720in}}{\pgfqpoint{5.229278in}{1.506712in}}{\pgfqpoint{5.239366in}{1.506712in}}%
\pgfpathlineto{\pgfqpoint{5.239366in}{1.506712in}}%
\pgfpathclose%
\pgfusepath{stroke,fill}%
\end{pgfscope}%
\begin{pgfscope}%
\pgfpathrectangle{\pgfqpoint{3.536584in}{0.147348in}}{\pgfqpoint{2.735294in}{2.735294in}}%
\pgfusepath{clip}%
\pgfsetbuttcap%
\pgfsetroundjoin%
\definecolor{currentfill}{rgb}{0.071067,0.258424,0.071067}%
\pgfsetfillcolor{currentfill}%
\pgfsetfillopacity{0.200000}%
\pgfsetlinewidth{0.000000pt}%
\definecolor{currentstroke}{rgb}{0.000000,0.000000,0.000000}%
\pgfsetstrokecolor{currentstroke}%
\pgfsetdash{}{0pt}%
\pgfpathmoveto{\pgfqpoint{3.979947in}{1.088921in}}%
\pgfpathlineto{\pgfqpoint{3.883527in}{1.228895in}}%
\pgfpathlineto{\pgfqpoint{3.884101in}{1.155548in}}%
\pgfpathlineto{\pgfqpoint{3.979947in}{1.088921in}}%
\pgfpathclose%
\pgfusepath{fill}%
\end{pgfscope}%
\begin{pgfscope}%
\pgfpathrectangle{\pgfqpoint{3.536584in}{0.147348in}}{\pgfqpoint{2.735294in}{2.735294in}}%
\pgfusepath{clip}%
\pgfsetbuttcap%
\pgfsetroundjoin%
\definecolor{currentfill}{rgb}{0.071067,0.258424,0.071067}%
\pgfsetfillcolor{currentfill}%
\pgfsetfillopacity{0.200000}%
\pgfsetlinewidth{0.000000pt}%
\definecolor{currentstroke}{rgb}{0.000000,0.000000,0.000000}%
\pgfsetstrokecolor{currentstroke}%
\pgfsetdash{}{0pt}%
\pgfpathmoveto{\pgfqpoint{5.998863in}{1.228895in}}%
\pgfpathlineto{\pgfqpoint{5.902443in}{1.088921in}}%
\pgfpathlineto{\pgfqpoint{5.998289in}{1.155548in}}%
\pgfpathlineto{\pgfqpoint{5.998863in}{1.228895in}}%
\pgfpathclose%
\pgfusepath{fill}%
\end{pgfscope}%
\begin{pgfscope}%
\pgfpathrectangle{\pgfqpoint{3.536584in}{0.147348in}}{\pgfqpoint{2.735294in}{2.735294in}}%
\pgfusepath{clip}%
\pgfsetbuttcap%
\pgfsetroundjoin%
\definecolor{currentfill}{rgb}{0.128601,0.467641,0.128601}%
\pgfsetfillcolor{currentfill}%
\pgfsetfillopacity{0.200000}%
\pgfsetlinewidth{0.000000pt}%
\definecolor{currentstroke}{rgb}{0.000000,0.000000,0.000000}%
\pgfsetstrokecolor{currentstroke}%
\pgfsetdash{}{0pt}%
\pgfpathmoveto{\pgfqpoint{4.844774in}{2.632943in}}%
\pgfpathlineto{\pgfqpoint{5.037616in}{2.632943in}}%
\pgfpathlineto{\pgfqpoint{4.941195in}{2.699366in}}%
\pgfpathlineto{\pgfqpoint{4.844774in}{2.632943in}}%
\pgfpathclose%
\pgfusepath{fill}%
\end{pgfscope}%
\begin{pgfscope}%
\pgfpathrectangle{\pgfqpoint{3.536584in}{0.147348in}}{\pgfqpoint{2.735294in}{2.735294in}}%
\pgfusepath{clip}%
\pgfsetbuttcap%
\pgfsetroundjoin%
\definecolor{currentfill}{rgb}{0.067488,0.245410,0.067488}%
\pgfsetfillcolor{currentfill}%
\pgfsetfillopacity{0.200000}%
\pgfsetlinewidth{0.000000pt}%
\definecolor{currentstroke}{rgb}{0.000000,0.000000,0.000000}%
\pgfsetstrokecolor{currentstroke}%
\pgfsetdash{}{0pt}%
\pgfpathmoveto{\pgfqpoint{4.104467in}{1.018707in}}%
\pgfpathlineto{\pgfqpoint{3.989844in}{1.163589in}}%
\pgfpathlineto{\pgfqpoint{3.979947in}{1.088921in}}%
\pgfpathlineto{\pgfqpoint{4.104467in}{1.018707in}}%
\pgfpathclose%
\pgfusepath{fill}%
\end{pgfscope}%
\begin{pgfscope}%
\pgfpathrectangle{\pgfqpoint{3.536584in}{0.147348in}}{\pgfqpoint{2.735294in}{2.735294in}}%
\pgfusepath{clip}%
\pgfsetbuttcap%
\pgfsetroundjoin%
\definecolor{currentfill}{rgb}{0.067488,0.245410,0.067488}%
\pgfsetfillcolor{currentfill}%
\pgfsetfillopacity{0.200000}%
\pgfsetlinewidth{0.000000pt}%
\definecolor{currentstroke}{rgb}{0.000000,0.000000,0.000000}%
\pgfsetstrokecolor{currentstroke}%
\pgfsetdash{}{0pt}%
\pgfpathmoveto{\pgfqpoint{5.902443in}{1.088921in}}%
\pgfpathlineto{\pgfqpoint{5.892546in}{1.163589in}}%
\pgfpathlineto{\pgfqpoint{5.777922in}{1.018707in}}%
\pgfpathlineto{\pgfqpoint{5.902443in}{1.088921in}}%
\pgfpathclose%
\pgfusepath{fill}%
\end{pgfscope}%
\begin{pgfscope}%
\pgfpathrectangle{\pgfqpoint{3.536584in}{0.147348in}}{\pgfqpoint{2.735294in}{2.735294in}}%
\pgfusepath{clip}%
\pgfsetbuttcap%
\pgfsetroundjoin%
\definecolor{currentfill}{rgb}{0.069492,0.252698,0.069492}%
\pgfsetfillcolor{currentfill}%
\pgfsetfillopacity{0.200000}%
\pgfsetlinewidth{0.000000pt}%
\definecolor{currentstroke}{rgb}{0.000000,0.000000,0.000000}%
\pgfsetstrokecolor{currentstroke}%
\pgfsetdash{}{0pt}%
\pgfpathmoveto{\pgfqpoint{3.883527in}{1.228895in}}%
\pgfpathlineto{\pgfqpoint{3.979947in}{1.088921in}}%
\pgfpathlineto{\pgfqpoint{3.977739in}{1.589726in}}%
\pgfpathlineto{\pgfqpoint{3.883527in}{1.228895in}}%
\pgfpathclose%
\pgfusepath{fill}%
\end{pgfscope}%
\begin{pgfscope}%
\pgfpathrectangle{\pgfqpoint{3.536584in}{0.147348in}}{\pgfqpoint{2.735294in}{2.735294in}}%
\pgfusepath{clip}%
\pgfsetbuttcap%
\pgfsetroundjoin%
\definecolor{currentfill}{rgb}{0.069492,0.252698,0.069492}%
\pgfsetfillcolor{currentfill}%
\pgfsetfillopacity{0.200000}%
\pgfsetlinewidth{0.000000pt}%
\definecolor{currentstroke}{rgb}{0.000000,0.000000,0.000000}%
\pgfsetstrokecolor{currentstroke}%
\pgfsetdash{}{0pt}%
\pgfpathmoveto{\pgfqpoint{5.998863in}{1.228895in}}%
\pgfpathlineto{\pgfqpoint{5.904651in}{1.589726in}}%
\pgfpathlineto{\pgfqpoint{5.902443in}{1.088921in}}%
\pgfpathlineto{\pgfqpoint{5.998863in}{1.228895in}}%
\pgfpathclose%
\pgfusepath{fill}%
\end{pgfscope}%
\begin{pgfscope}%
\pgfpathrectangle{\pgfqpoint{3.536584in}{0.147348in}}{\pgfqpoint{2.735294in}{2.735294in}}%
\pgfusepath{clip}%
\pgfsetbuttcap%
\pgfsetroundjoin%
\definecolor{currentfill}{rgb}{0.099716,0.362602,0.099716}%
\pgfsetfillcolor{currentfill}%
\pgfsetfillopacity{0.200000}%
\pgfsetlinewidth{0.000000pt}%
\definecolor{currentstroke}{rgb}{0.000000,0.000000,0.000000}%
\pgfsetstrokecolor{currentstroke}%
\pgfsetdash{}{0pt}%
\pgfpathmoveto{\pgfqpoint{3.977739in}{1.589726in}}%
\pgfpathlineto{\pgfqpoint{3.979947in}{1.088921in}}%
\pgfpathlineto{\pgfqpoint{3.989844in}{1.163589in}}%
\pgfpathlineto{\pgfqpoint{3.977739in}{1.589726in}}%
\pgfpathclose%
\pgfusepath{fill}%
\end{pgfscope}%
\begin{pgfscope}%
\pgfpathrectangle{\pgfqpoint{3.536584in}{0.147348in}}{\pgfqpoint{2.735294in}{2.735294in}}%
\pgfusepath{clip}%
\pgfsetbuttcap%
\pgfsetroundjoin%
\definecolor{currentfill}{rgb}{0.099716,0.362602,0.099716}%
\pgfsetfillcolor{currentfill}%
\pgfsetfillopacity{0.200000}%
\pgfsetlinewidth{0.000000pt}%
\definecolor{currentstroke}{rgb}{0.000000,0.000000,0.000000}%
\pgfsetstrokecolor{currentstroke}%
\pgfsetdash{}{0pt}%
\pgfpathmoveto{\pgfqpoint{5.892546in}{1.163589in}}%
\pgfpathlineto{\pgfqpoint{5.902443in}{1.088921in}}%
\pgfpathlineto{\pgfqpoint{5.904651in}{1.589726in}}%
\pgfpathlineto{\pgfqpoint{5.892546in}{1.163589in}}%
\pgfpathclose%
\pgfusepath{fill}%
\end{pgfscope}%
\begin{pgfscope}%
\pgfpathrectangle{\pgfqpoint{3.536584in}{0.147348in}}{\pgfqpoint{2.735294in}{2.735294in}}%
\pgfusepath{clip}%
\pgfsetbuttcap%
\pgfsetroundjoin%
\definecolor{currentfill}{rgb}{0.063840,0.232145,0.063840}%
\pgfsetfillcolor{currentfill}%
\pgfsetfillopacity{0.200000}%
\pgfsetlinewidth{0.000000pt}%
\definecolor{currentstroke}{rgb}{0.000000,0.000000,0.000000}%
\pgfsetstrokecolor{currentstroke}%
\pgfsetdash{}{0pt}%
\pgfpathmoveto{\pgfqpoint{4.263554in}{0.949464in}}%
\pgfpathlineto{\pgfqpoint{4.130959in}{1.094945in}}%
\pgfpathlineto{\pgfqpoint{4.104467in}{1.018707in}}%
\pgfpathlineto{\pgfqpoint{4.263554in}{0.949464in}}%
\pgfpathclose%
\pgfusepath{fill}%
\end{pgfscope}%
\begin{pgfscope}%
\pgfpathrectangle{\pgfqpoint{3.536584in}{0.147348in}}{\pgfqpoint{2.735294in}{2.735294in}}%
\pgfusepath{clip}%
\pgfsetbuttcap%
\pgfsetroundjoin%
\definecolor{currentfill}{rgb}{0.063840,0.232145,0.063840}%
\pgfsetfillcolor{currentfill}%
\pgfsetfillopacity{0.200000}%
\pgfsetlinewidth{0.000000pt}%
\definecolor{currentstroke}{rgb}{0.000000,0.000000,0.000000}%
\pgfsetstrokecolor{currentstroke}%
\pgfsetdash{}{0pt}%
\pgfpathmoveto{\pgfqpoint{5.777922in}{1.018707in}}%
\pgfpathlineto{\pgfqpoint{5.751431in}{1.094945in}}%
\pgfpathlineto{\pgfqpoint{5.618836in}{0.949464in}}%
\pgfpathlineto{\pgfqpoint{5.777922in}{1.018707in}}%
\pgfpathclose%
\pgfusepath{fill}%
\end{pgfscope}%
\begin{pgfscope}%
\pgfpathrectangle{\pgfqpoint{3.536584in}{0.147348in}}{\pgfqpoint{2.735294in}{2.735294in}}%
\pgfusepath{clip}%
\pgfsetbuttcap%
\pgfsetroundjoin%
\definecolor{currentfill}{rgb}{0.116321,0.422987,0.116321}%
\pgfsetfillcolor{currentfill}%
\pgfsetfillopacity{0.200000}%
\pgfsetlinewidth{0.000000pt}%
\definecolor{currentstroke}{rgb}{0.000000,0.000000,0.000000}%
\pgfsetstrokecolor{currentstroke}%
\pgfsetdash{}{0pt}%
\pgfpathmoveto{\pgfqpoint{5.037616in}{2.632943in}}%
\pgfpathlineto{\pgfqpoint{4.844774in}{2.632943in}}%
\pgfpathlineto{\pgfqpoint{4.802903in}{2.150689in}}%
\pgfpathlineto{\pgfqpoint{5.037616in}{2.632943in}}%
\pgfpathclose%
\pgfusepath{fill}%
\end{pgfscope}%
\begin{pgfscope}%
\pgfpathrectangle{\pgfqpoint{3.536584in}{0.147348in}}{\pgfqpoint{2.735294in}{2.735294in}}%
\pgfusepath{clip}%
\pgfsetbuttcap%
\pgfsetroundjoin%
\definecolor{currentfill}{rgb}{0.067061,0.243857,0.067061}%
\pgfsetfillcolor{currentfill}%
\pgfsetfillopacity{0.200000}%
\pgfsetlinewidth{0.000000pt}%
\definecolor{currentstroke}{rgb}{0.000000,0.000000,0.000000}%
\pgfsetstrokecolor{currentstroke}%
\pgfsetdash{}{0pt}%
\pgfpathmoveto{\pgfqpoint{3.989844in}{1.163589in}}%
\pgfpathlineto{\pgfqpoint{4.104467in}{1.018707in}}%
\pgfpathlineto{\pgfqpoint{4.139943in}{1.557460in}}%
\pgfpathlineto{\pgfqpoint{3.989844in}{1.163589in}}%
\pgfpathclose%
\pgfusepath{fill}%
\end{pgfscope}%
\begin{pgfscope}%
\pgfpathrectangle{\pgfqpoint{3.536584in}{0.147348in}}{\pgfqpoint{2.735294in}{2.735294in}}%
\pgfusepath{clip}%
\pgfsetbuttcap%
\pgfsetroundjoin%
\definecolor{currentfill}{rgb}{0.067061,0.243857,0.067061}%
\pgfsetfillcolor{currentfill}%
\pgfsetfillopacity{0.200000}%
\pgfsetlinewidth{0.000000pt}%
\definecolor{currentstroke}{rgb}{0.000000,0.000000,0.000000}%
\pgfsetstrokecolor{currentstroke}%
\pgfsetdash{}{0pt}%
\pgfpathmoveto{\pgfqpoint{5.742447in}{1.557460in}}%
\pgfpathlineto{\pgfqpoint{5.777922in}{1.018707in}}%
\pgfpathlineto{\pgfqpoint{5.892546in}{1.163589in}}%
\pgfpathlineto{\pgfqpoint{5.742447in}{1.557460in}}%
\pgfpathclose%
\pgfusepath{fill}%
\end{pgfscope}%
\begin{pgfscope}%
\pgfpathrectangle{\pgfqpoint{3.536584in}{0.147348in}}{\pgfqpoint{2.735294in}{2.735294in}}%
\pgfusepath{clip}%
\pgfsetbuttcap%
\pgfsetroundjoin%
\definecolor{currentfill}{rgb}{0.095351,0.346729,0.095351}%
\pgfsetfillcolor{currentfill}%
\pgfsetfillopacity{0.200000}%
\pgfsetlinewidth{0.000000pt}%
\definecolor{currentstroke}{rgb}{0.000000,0.000000,0.000000}%
\pgfsetstrokecolor{currentstroke}%
\pgfsetdash{}{0pt}%
\pgfpathmoveto{\pgfqpoint{4.139943in}{1.557460in}}%
\pgfpathlineto{\pgfqpoint{4.104467in}{1.018707in}}%
\pgfpathlineto{\pgfqpoint{4.130959in}{1.094945in}}%
\pgfpathlineto{\pgfqpoint{4.139943in}{1.557460in}}%
\pgfpathclose%
\pgfusepath{fill}%
\end{pgfscope}%
\begin{pgfscope}%
\pgfpathrectangle{\pgfqpoint{3.536584in}{0.147348in}}{\pgfqpoint{2.735294in}{2.735294in}}%
\pgfusepath{clip}%
\pgfsetbuttcap%
\pgfsetroundjoin%
\definecolor{currentfill}{rgb}{0.095351,0.346729,0.095351}%
\pgfsetfillcolor{currentfill}%
\pgfsetfillopacity{0.200000}%
\pgfsetlinewidth{0.000000pt}%
\definecolor{currentstroke}{rgb}{0.000000,0.000000,0.000000}%
\pgfsetstrokecolor{currentstroke}%
\pgfsetdash{}{0pt}%
\pgfpathmoveto{\pgfqpoint{5.751431in}{1.094945in}}%
\pgfpathlineto{\pgfqpoint{5.777922in}{1.018707in}}%
\pgfpathlineto{\pgfqpoint{5.742447in}{1.557460in}}%
\pgfpathlineto{\pgfqpoint{5.751431in}{1.094945in}}%
\pgfpathclose%
\pgfusepath{fill}%
\end{pgfscope}%
\begin{pgfscope}%
\pgfpathrectangle{\pgfqpoint{3.536584in}{0.147348in}}{\pgfqpoint{2.735294in}{2.735294in}}%
\pgfusepath{clip}%
\pgfsetbuttcap%
\pgfsetroundjoin%
\definecolor{currentfill}{rgb}{0.060435,0.219763,0.060435}%
\pgfsetfillcolor{currentfill}%
\pgfsetfillopacity{0.200000}%
\pgfsetlinewidth{0.000000pt}%
\definecolor{currentstroke}{rgb}{0.000000,0.000000,0.000000}%
\pgfsetstrokecolor{currentstroke}%
\pgfsetdash{}{0pt}%
\pgfpathmoveto{\pgfqpoint{5.568365in}{1.028737in}}%
\pgfpathlineto{\pgfqpoint{5.422251in}{0.888724in}}%
\pgfpathlineto{\pgfqpoint{5.618836in}{0.949464in}}%
\pgfpathlineto{\pgfqpoint{5.568365in}{1.028737in}}%
\pgfpathclose%
\pgfusepath{fill}%
\end{pgfscope}%
\begin{pgfscope}%
\pgfpathrectangle{\pgfqpoint{3.536584in}{0.147348in}}{\pgfqpoint{2.735294in}{2.735294in}}%
\pgfusepath{clip}%
\pgfsetbuttcap%
\pgfsetroundjoin%
\definecolor{currentfill}{rgb}{0.060435,0.219763,0.060435}%
\pgfsetfillcolor{currentfill}%
\pgfsetfillopacity{0.200000}%
\pgfsetlinewidth{0.000000pt}%
\definecolor{currentstroke}{rgb}{0.000000,0.000000,0.000000}%
\pgfsetstrokecolor{currentstroke}%
\pgfsetdash{}{0pt}%
\pgfpathmoveto{\pgfqpoint{4.263554in}{0.949464in}}%
\pgfpathlineto{\pgfqpoint{4.460139in}{0.888724in}}%
\pgfpathlineto{\pgfqpoint{4.314024in}{1.028737in}}%
\pgfpathlineto{\pgfqpoint{4.263554in}{0.949464in}}%
\pgfpathclose%
\pgfusepath{fill}%
\end{pgfscope}%
\begin{pgfscope}%
\pgfpathrectangle{\pgfqpoint{3.536584in}{0.147348in}}{\pgfqpoint{2.735294in}{2.735294in}}%
\pgfusepath{clip}%
\pgfsetbuttcap%
\pgfsetroundjoin%
\definecolor{currentfill}{rgb}{0.074506,0.270932,0.074506}%
\pgfsetfillcolor{currentfill}%
\pgfsetfillopacity{0.200000}%
\pgfsetlinewidth{0.000000pt}%
\definecolor{currentstroke}{rgb}{0.000000,0.000000,0.000000}%
\pgfsetstrokecolor{currentstroke}%
\pgfsetdash{}{0pt}%
\pgfpathmoveto{\pgfqpoint{3.977739in}{1.589726in}}%
\pgfpathlineto{\pgfqpoint{3.989844in}{1.163589in}}%
\pgfpathlineto{\pgfqpoint{4.139943in}{1.557460in}}%
\pgfpathlineto{\pgfqpoint{3.977739in}{1.589726in}}%
\pgfpathclose%
\pgfusepath{fill}%
\end{pgfscope}%
\begin{pgfscope}%
\pgfpathrectangle{\pgfqpoint{3.536584in}{0.147348in}}{\pgfqpoint{2.735294in}{2.735294in}}%
\pgfusepath{clip}%
\pgfsetbuttcap%
\pgfsetroundjoin%
\definecolor{currentfill}{rgb}{0.074506,0.270932,0.074506}%
\pgfsetfillcolor{currentfill}%
\pgfsetfillopacity{0.200000}%
\pgfsetlinewidth{0.000000pt}%
\definecolor{currentstroke}{rgb}{0.000000,0.000000,0.000000}%
\pgfsetstrokecolor{currentstroke}%
\pgfsetdash{}{0pt}%
\pgfpathmoveto{\pgfqpoint{5.742447in}{1.557460in}}%
\pgfpathlineto{\pgfqpoint{5.892546in}{1.163589in}}%
\pgfpathlineto{\pgfqpoint{5.904651in}{1.589726in}}%
\pgfpathlineto{\pgfqpoint{5.742447in}{1.557460in}}%
\pgfpathclose%
\pgfusepath{fill}%
\end{pgfscope}%
\begin{pgfscope}%
\pgfpathrectangle{\pgfqpoint{3.536584in}{0.147348in}}{\pgfqpoint{2.735294in}{2.735294in}}%
\pgfusepath{clip}%
\pgfsetbuttcap%
\pgfsetroundjoin%
\definecolor{currentfill}{rgb}{0.116785,0.424671,0.116785}%
\pgfsetfillcolor{currentfill}%
\pgfsetfillopacity{0.200000}%
\pgfsetlinewidth{0.000000pt}%
\definecolor{currentstroke}{rgb}{0.000000,0.000000,0.000000}%
\pgfsetstrokecolor{currentstroke}%
\pgfsetdash{}{0pt}%
\pgfpathmoveto{\pgfqpoint{5.423597in}{2.292690in}}%
\pgfpathlineto{\pgfqpoint{5.037616in}{2.632943in}}%
\pgfpathlineto{\pgfqpoint{5.079487in}{2.150689in}}%
\pgfpathlineto{\pgfqpoint{5.423597in}{2.292690in}}%
\pgfpathclose%
\pgfusepath{fill}%
\end{pgfscope}%
\begin{pgfscope}%
\pgfpathrectangle{\pgfqpoint{3.536584in}{0.147348in}}{\pgfqpoint{2.735294in}{2.735294in}}%
\pgfusepath{clip}%
\pgfsetbuttcap%
\pgfsetroundjoin%
\definecolor{currentfill}{rgb}{0.116785,0.424671,0.116785}%
\pgfsetfillcolor{currentfill}%
\pgfsetfillopacity{0.200000}%
\pgfsetlinewidth{0.000000pt}%
\definecolor{currentstroke}{rgb}{0.000000,0.000000,0.000000}%
\pgfsetstrokecolor{currentstroke}%
\pgfsetdash{}{0pt}%
\pgfpathmoveto{\pgfqpoint{4.802903in}{2.150689in}}%
\pgfpathlineto{\pgfqpoint{4.844774in}{2.632943in}}%
\pgfpathlineto{\pgfqpoint{4.458793in}{2.292690in}}%
\pgfpathlineto{\pgfqpoint{4.802903in}{2.150689in}}%
\pgfpathclose%
\pgfusepath{fill}%
\end{pgfscope}%
\begin{pgfscope}%
\pgfpathrectangle{\pgfqpoint{3.536584in}{0.147348in}}{\pgfqpoint{2.735294in}{2.735294in}}%
\pgfusepath{clip}%
\pgfsetbuttcap%
\pgfsetroundjoin%
\definecolor{currentfill}{rgb}{0.064867,0.235879,0.064867}%
\pgfsetfillcolor{currentfill}%
\pgfsetfillopacity{0.200000}%
\pgfsetlinewidth{0.000000pt}%
\definecolor{currentstroke}{rgb}{0.000000,0.000000,0.000000}%
\pgfsetstrokecolor{currentstroke}%
\pgfsetdash{}{0pt}%
\pgfpathmoveto{\pgfqpoint{4.130959in}{1.094945in}}%
\pgfpathlineto{\pgfqpoint{4.263554in}{0.949464in}}%
\pgfpathlineto{\pgfqpoint{4.357383in}{1.526914in}}%
\pgfpathlineto{\pgfqpoint{4.130959in}{1.094945in}}%
\pgfpathclose%
\pgfusepath{fill}%
\end{pgfscope}%
\begin{pgfscope}%
\pgfpathrectangle{\pgfqpoint{3.536584in}{0.147348in}}{\pgfqpoint{2.735294in}{2.735294in}}%
\pgfusepath{clip}%
\pgfsetbuttcap%
\pgfsetroundjoin%
\definecolor{currentfill}{rgb}{0.064867,0.235879,0.064867}%
\pgfsetfillcolor{currentfill}%
\pgfsetfillopacity{0.200000}%
\pgfsetlinewidth{0.000000pt}%
\definecolor{currentstroke}{rgb}{0.000000,0.000000,0.000000}%
\pgfsetstrokecolor{currentstroke}%
\pgfsetdash{}{0pt}%
\pgfpathmoveto{\pgfqpoint{5.525007in}{1.526914in}}%
\pgfpathlineto{\pgfqpoint{5.618836in}{0.949464in}}%
\pgfpathlineto{\pgfqpoint{5.751431in}{1.094945in}}%
\pgfpathlineto{\pgfqpoint{5.525007in}{1.526914in}}%
\pgfpathclose%
\pgfusepath{fill}%
\end{pgfscope}%
\begin{pgfscope}%
\pgfpathrectangle{\pgfqpoint{3.536584in}{0.147348in}}{\pgfqpoint{2.735294in}{2.735294in}}%
\pgfusepath{clip}%
\pgfsetbuttcap%
\pgfsetroundjoin%
\definecolor{currentfill}{rgb}{0.057724,0.209904,0.057724}%
\pgfsetfillcolor{currentfill}%
\pgfsetfillopacity{0.200000}%
\pgfsetlinewidth{0.000000pt}%
\definecolor{currentstroke}{rgb}{0.000000,0.000000,0.000000}%
\pgfsetstrokecolor{currentstroke}%
\pgfsetdash{}{0pt}%
\pgfpathmoveto{\pgfqpoint{4.541034in}{0.974376in}}%
\pgfpathlineto{\pgfqpoint{4.460139in}{0.888724in}}%
\pgfpathlineto{\pgfqpoint{4.690418in}{0.846223in}}%
\pgfpathlineto{\pgfqpoint{4.541034in}{0.974376in}}%
\pgfpathclose%
\pgfusepath{fill}%
\end{pgfscope}%
\begin{pgfscope}%
\pgfpathrectangle{\pgfqpoint{3.536584in}{0.147348in}}{\pgfqpoint{2.735294in}{2.735294in}}%
\pgfusepath{clip}%
\pgfsetbuttcap%
\pgfsetroundjoin%
\definecolor{currentfill}{rgb}{0.057724,0.209904,0.057724}%
\pgfsetfillcolor{currentfill}%
\pgfsetfillopacity{0.200000}%
\pgfsetlinewidth{0.000000pt}%
\definecolor{currentstroke}{rgb}{0.000000,0.000000,0.000000}%
\pgfsetstrokecolor{currentstroke}%
\pgfsetdash{}{0pt}%
\pgfpathmoveto{\pgfqpoint{5.191972in}{0.846223in}}%
\pgfpathlineto{\pgfqpoint{5.422251in}{0.888724in}}%
\pgfpathlineto{\pgfqpoint{5.341356in}{0.974376in}}%
\pgfpathlineto{\pgfqpoint{5.191972in}{0.846223in}}%
\pgfpathclose%
\pgfusepath{fill}%
\end{pgfscope}%
\begin{pgfscope}%
\pgfpathrectangle{\pgfqpoint{3.536584in}{0.147348in}}{\pgfqpoint{2.735294in}{2.735294in}}%
\pgfusepath{clip}%
\pgfsetbuttcap%
\pgfsetroundjoin%
\definecolor{currentfill}{rgb}{0.086498,0.314539,0.086498}%
\pgfsetfillcolor{currentfill}%
\pgfsetfillopacity{0.200000}%
\pgfsetlinewidth{0.000000pt}%
\definecolor{currentstroke}{rgb}{0.000000,0.000000,0.000000}%
\pgfsetstrokecolor{currentstroke}%
\pgfsetdash{}{0pt}%
\pgfpathmoveto{\pgfqpoint{4.139943in}{1.557460in}}%
\pgfpathlineto{\pgfqpoint{4.061876in}{1.760124in}}%
\pgfpathlineto{\pgfqpoint{3.977739in}{1.589726in}}%
\pgfpathlineto{\pgfqpoint{4.139943in}{1.557460in}}%
\pgfpathclose%
\pgfusepath{fill}%
\end{pgfscope}%
\begin{pgfscope}%
\pgfpathrectangle{\pgfqpoint{3.536584in}{0.147348in}}{\pgfqpoint{2.735294in}{2.735294in}}%
\pgfusepath{clip}%
\pgfsetbuttcap%
\pgfsetroundjoin%
\definecolor{currentfill}{rgb}{0.086498,0.314539,0.086498}%
\pgfsetfillcolor{currentfill}%
\pgfsetfillopacity{0.200000}%
\pgfsetlinewidth{0.000000pt}%
\definecolor{currentstroke}{rgb}{0.000000,0.000000,0.000000}%
\pgfsetstrokecolor{currentstroke}%
\pgfsetdash{}{0pt}%
\pgfpathmoveto{\pgfqpoint{5.904651in}{1.589726in}}%
\pgfpathlineto{\pgfqpoint{5.820514in}{1.760124in}}%
\pgfpathlineto{\pgfqpoint{5.742447in}{1.557460in}}%
\pgfpathlineto{\pgfqpoint{5.904651in}{1.589726in}}%
\pgfpathclose%
\pgfusepath{fill}%
\end{pgfscope}%
\begin{pgfscope}%
\pgfpathrectangle{\pgfqpoint{3.536584in}{0.147348in}}{\pgfqpoint{2.735294in}{2.735294in}}%
\pgfusepath{clip}%
\pgfsetbuttcap%
\pgfsetroundjoin%
\definecolor{currentfill}{rgb}{0.090812,0.330224,0.090812}%
\pgfsetfillcolor{currentfill}%
\pgfsetfillopacity{0.200000}%
\pgfsetlinewidth{0.000000pt}%
\definecolor{currentstroke}{rgb}{0.000000,0.000000,0.000000}%
\pgfsetstrokecolor{currentstroke}%
\pgfsetdash{}{0pt}%
\pgfpathmoveto{\pgfqpoint{4.357383in}{1.526914in}}%
\pgfpathlineto{\pgfqpoint{4.263554in}{0.949464in}}%
\pgfpathlineto{\pgfqpoint{4.314024in}{1.028737in}}%
\pgfpathlineto{\pgfqpoint{4.357383in}{1.526914in}}%
\pgfpathclose%
\pgfusepath{fill}%
\end{pgfscope}%
\begin{pgfscope}%
\pgfpathrectangle{\pgfqpoint{3.536584in}{0.147348in}}{\pgfqpoint{2.735294in}{2.735294in}}%
\pgfusepath{clip}%
\pgfsetbuttcap%
\pgfsetroundjoin%
\definecolor{currentfill}{rgb}{0.090812,0.330224,0.090812}%
\pgfsetfillcolor{currentfill}%
\pgfsetfillopacity{0.200000}%
\pgfsetlinewidth{0.000000pt}%
\definecolor{currentstroke}{rgb}{0.000000,0.000000,0.000000}%
\pgfsetstrokecolor{currentstroke}%
\pgfsetdash{}{0pt}%
\pgfpathmoveto{\pgfqpoint{5.568365in}{1.028737in}}%
\pgfpathlineto{\pgfqpoint{5.618836in}{0.949464in}}%
\pgfpathlineto{\pgfqpoint{5.525007in}{1.526914in}}%
\pgfpathlineto{\pgfqpoint{5.568365in}{1.028737in}}%
\pgfpathclose%
\pgfusepath{fill}%
\end{pgfscope}%
\begin{pgfscope}%
\pgfpathrectangle{\pgfqpoint{3.536584in}{0.147348in}}{\pgfqpoint{2.735294in}{2.735294in}}%
\pgfusepath{clip}%
\pgfsetbuttcap%
\pgfsetroundjoin%
\definecolor{currentfill}{rgb}{0.116321,0.422987,0.116321}%
\pgfsetfillcolor{currentfill}%
\pgfsetfillopacity{0.200000}%
\pgfsetlinewidth{0.000000pt}%
\definecolor{currentstroke}{rgb}{0.000000,0.000000,0.000000}%
\pgfsetstrokecolor{currentstroke}%
\pgfsetdash{}{0pt}%
\pgfpathmoveto{\pgfqpoint{4.802903in}{2.150689in}}%
\pgfpathlineto{\pgfqpoint{5.079487in}{2.150689in}}%
\pgfpathlineto{\pgfqpoint{5.037616in}{2.632943in}}%
\pgfpathlineto{\pgfqpoint{4.802903in}{2.150689in}}%
\pgfpathclose%
\pgfusepath{fill}%
\end{pgfscope}%
\begin{pgfscope}%
\pgfpathrectangle{\pgfqpoint{3.536584in}{0.147348in}}{\pgfqpoint{2.735294in}{2.735294in}}%
\pgfusepath{clip}%
\pgfsetbuttcap%
\pgfsetroundjoin%
\definecolor{currentfill}{rgb}{0.056200,0.204363,0.056200}%
\pgfsetfillcolor{currentfill}%
\pgfsetfillopacity{0.200000}%
\pgfsetlinewidth{0.000000pt}%
\definecolor{currentstroke}{rgb}{0.000000,0.000000,0.000000}%
\pgfsetstrokecolor{currentstroke}%
\pgfsetdash{}{0pt}%
\pgfpathmoveto{\pgfqpoint{4.941195in}{0.830831in}}%
\pgfpathlineto{\pgfqpoint{4.803235in}{0.943120in}}%
\pgfpathlineto{\pgfqpoint{4.690418in}{0.846223in}}%
\pgfpathlineto{\pgfqpoint{4.941195in}{0.830831in}}%
\pgfpathclose%
\pgfusepath{fill}%
\end{pgfscope}%
\begin{pgfscope}%
\pgfpathrectangle{\pgfqpoint{3.536584in}{0.147348in}}{\pgfqpoint{2.735294in}{2.735294in}}%
\pgfusepath{clip}%
\pgfsetbuttcap%
\pgfsetroundjoin%
\definecolor{currentfill}{rgb}{0.056200,0.204363,0.056200}%
\pgfsetfillcolor{currentfill}%
\pgfsetfillopacity{0.200000}%
\pgfsetlinewidth{0.000000pt}%
\definecolor{currentstroke}{rgb}{0.000000,0.000000,0.000000}%
\pgfsetstrokecolor{currentstroke}%
\pgfsetdash{}{0pt}%
\pgfpathmoveto{\pgfqpoint{5.191972in}{0.846223in}}%
\pgfpathlineto{\pgfqpoint{5.079154in}{0.943120in}}%
\pgfpathlineto{\pgfqpoint{4.941195in}{0.830831in}}%
\pgfpathlineto{\pgfqpoint{5.191972in}{0.846223in}}%
\pgfpathclose%
\pgfusepath{fill}%
\end{pgfscope}%
\begin{pgfscope}%
\pgfpathrectangle{\pgfqpoint{3.536584in}{0.147348in}}{\pgfqpoint{2.735294in}{2.735294in}}%
\pgfusepath{clip}%
\pgfsetbuttcap%
\pgfsetroundjoin%
\definecolor{currentfill}{rgb}{0.107070,0.389346,0.107070}%
\pgfsetfillcolor{currentfill}%
\pgfsetfillopacity{0.200000}%
\pgfsetlinewidth{0.000000pt}%
\definecolor{currentstroke}{rgb}{0.000000,0.000000,0.000000}%
\pgfsetstrokecolor{currentstroke}%
\pgfsetdash{}{0pt}%
\pgfpathmoveto{\pgfqpoint{4.540103in}{2.141900in}}%
\pgfpathlineto{\pgfqpoint{4.458793in}{2.292690in}}%
\pgfpathlineto{\pgfqpoint{4.312652in}{2.126617in}}%
\pgfpathlineto{\pgfqpoint{4.540103in}{2.141900in}}%
\pgfpathclose%
\pgfusepath{fill}%
\end{pgfscope}%
\begin{pgfscope}%
\pgfpathrectangle{\pgfqpoint{3.536584in}{0.147348in}}{\pgfqpoint{2.735294in}{2.735294in}}%
\pgfusepath{clip}%
\pgfsetbuttcap%
\pgfsetroundjoin%
\definecolor{currentfill}{rgb}{0.107070,0.389346,0.107070}%
\pgfsetfillcolor{currentfill}%
\pgfsetfillopacity{0.200000}%
\pgfsetlinewidth{0.000000pt}%
\definecolor{currentstroke}{rgb}{0.000000,0.000000,0.000000}%
\pgfsetstrokecolor{currentstroke}%
\pgfsetdash{}{0pt}%
\pgfpathmoveto{\pgfqpoint{5.569738in}{2.126617in}}%
\pgfpathlineto{\pgfqpoint{5.423597in}{2.292690in}}%
\pgfpathlineto{\pgfqpoint{5.342287in}{2.141900in}}%
\pgfpathlineto{\pgfqpoint{5.569738in}{2.126617in}}%
\pgfpathclose%
\pgfusepath{fill}%
\end{pgfscope}%
\begin{pgfscope}%
\pgfpathrectangle{\pgfqpoint{3.536584in}{0.147348in}}{\pgfqpoint{2.735294in}{2.735294in}}%
\pgfusepath{clip}%
\pgfsetbuttcap%
\pgfsetroundjoin%
\definecolor{currentfill}{rgb}{0.089078,0.323920,0.089078}%
\pgfsetfillcolor{currentfill}%
\pgfsetfillopacity{0.200000}%
\pgfsetlinewidth{0.000000pt}%
\definecolor{currentstroke}{rgb}{0.000000,0.000000,0.000000}%
\pgfsetstrokecolor{currentstroke}%
\pgfsetdash{}{0pt}%
\pgfpathmoveto{\pgfqpoint{4.139943in}{1.557460in}}%
\pgfpathlineto{\pgfqpoint{4.312652in}{2.126617in}}%
\pgfpathlineto{\pgfqpoint{4.061876in}{1.760124in}}%
\pgfpathlineto{\pgfqpoint{4.139943in}{1.557460in}}%
\pgfpathclose%
\pgfusepath{fill}%
\end{pgfscope}%
\begin{pgfscope}%
\pgfpathrectangle{\pgfqpoint{3.536584in}{0.147348in}}{\pgfqpoint{2.735294in}{2.735294in}}%
\pgfusepath{clip}%
\pgfsetbuttcap%
\pgfsetroundjoin%
\definecolor{currentfill}{rgb}{0.089078,0.323920,0.089078}%
\pgfsetfillcolor{currentfill}%
\pgfsetfillopacity{0.200000}%
\pgfsetlinewidth{0.000000pt}%
\definecolor{currentstroke}{rgb}{0.000000,0.000000,0.000000}%
\pgfsetstrokecolor{currentstroke}%
\pgfsetdash{}{0pt}%
\pgfpathmoveto{\pgfqpoint{5.820514in}{1.760124in}}%
\pgfpathlineto{\pgfqpoint{5.569738in}{2.126617in}}%
\pgfpathlineto{\pgfqpoint{5.742447in}{1.557460in}}%
\pgfpathlineto{\pgfqpoint{5.820514in}{1.760124in}}%
\pgfpathclose%
\pgfusepath{fill}%
\end{pgfscope}%
\begin{pgfscope}%
\pgfpathrectangle{\pgfqpoint{3.536584in}{0.147348in}}{\pgfqpoint{2.735294in}{2.735294in}}%
\pgfusepath{clip}%
\pgfsetbuttcap%
\pgfsetroundjoin%
\definecolor{currentfill}{rgb}{0.061754,0.224559,0.061754}%
\pgfsetfillcolor{currentfill}%
\pgfsetfillopacity{0.200000}%
\pgfsetlinewidth{0.000000pt}%
\definecolor{currentstroke}{rgb}{0.000000,0.000000,0.000000}%
\pgfsetstrokecolor{currentstroke}%
\pgfsetdash{}{0pt}%
\pgfpathmoveto{\pgfqpoint{4.314024in}{1.028737in}}%
\pgfpathlineto{\pgfqpoint{4.460139in}{0.888724in}}%
\pgfpathlineto{\pgfqpoint{4.494794in}{1.302943in}}%
\pgfpathlineto{\pgfqpoint{4.314024in}{1.028737in}}%
\pgfpathclose%
\pgfusepath{fill}%
\end{pgfscope}%
\begin{pgfscope}%
\pgfpathrectangle{\pgfqpoint{3.536584in}{0.147348in}}{\pgfqpoint{2.735294in}{2.735294in}}%
\pgfusepath{clip}%
\pgfsetbuttcap%
\pgfsetroundjoin%
\definecolor{currentfill}{rgb}{0.061754,0.224559,0.061754}%
\pgfsetfillcolor{currentfill}%
\pgfsetfillopacity{0.200000}%
\pgfsetlinewidth{0.000000pt}%
\definecolor{currentstroke}{rgb}{0.000000,0.000000,0.000000}%
\pgfsetstrokecolor{currentstroke}%
\pgfsetdash{}{0pt}%
\pgfpathmoveto{\pgfqpoint{5.387596in}{1.302943in}}%
\pgfpathlineto{\pgfqpoint{5.422251in}{0.888724in}}%
\pgfpathlineto{\pgfqpoint{5.568365in}{1.028737in}}%
\pgfpathlineto{\pgfqpoint{5.387596in}{1.302943in}}%
\pgfpathclose%
\pgfusepath{fill}%
\end{pgfscope}%
\begin{pgfscope}%
\pgfpathrectangle{\pgfqpoint{3.536584in}{0.147348in}}{\pgfqpoint{2.735294in}{2.735294in}}%
\pgfusepath{clip}%
\pgfsetbuttcap%
\pgfsetroundjoin%
\definecolor{currentfill}{rgb}{0.070984,0.258123,0.070984}%
\pgfsetfillcolor{currentfill}%
\pgfsetfillopacity{0.200000}%
\pgfsetlinewidth{0.000000pt}%
\definecolor{currentstroke}{rgb}{0.000000,0.000000,0.000000}%
\pgfsetstrokecolor{currentstroke}%
\pgfsetdash{}{0pt}%
\pgfpathmoveto{\pgfqpoint{4.139943in}{1.557460in}}%
\pgfpathlineto{\pgfqpoint{4.130959in}{1.094945in}}%
\pgfpathlineto{\pgfqpoint{4.357383in}{1.526914in}}%
\pgfpathlineto{\pgfqpoint{4.139943in}{1.557460in}}%
\pgfpathclose%
\pgfusepath{fill}%
\end{pgfscope}%
\begin{pgfscope}%
\pgfpathrectangle{\pgfqpoint{3.536584in}{0.147348in}}{\pgfqpoint{2.735294in}{2.735294in}}%
\pgfusepath{clip}%
\pgfsetbuttcap%
\pgfsetroundjoin%
\definecolor{currentfill}{rgb}{0.070984,0.258123,0.070984}%
\pgfsetfillcolor{currentfill}%
\pgfsetfillopacity{0.200000}%
\pgfsetlinewidth{0.000000pt}%
\definecolor{currentstroke}{rgb}{0.000000,0.000000,0.000000}%
\pgfsetstrokecolor{currentstroke}%
\pgfsetdash{}{0pt}%
\pgfpathmoveto{\pgfqpoint{5.525007in}{1.526914in}}%
\pgfpathlineto{\pgfqpoint{5.751431in}{1.094945in}}%
\pgfpathlineto{\pgfqpoint{5.742447in}{1.557460in}}%
\pgfpathlineto{\pgfqpoint{5.525007in}{1.526914in}}%
\pgfpathclose%
\pgfusepath{fill}%
\end{pgfscope}%
\begin{pgfscope}%
\pgfpathrectangle{\pgfqpoint{3.536584in}{0.147348in}}{\pgfqpoint{2.735294in}{2.735294in}}%
\pgfusepath{clip}%
\pgfsetbuttcap%
\pgfsetroundjoin%
\definecolor{currentfill}{rgb}{0.070885,0.257762,0.070885}%
\pgfsetfillcolor{currentfill}%
\pgfsetfillopacity{0.200000}%
\pgfsetlinewidth{0.000000pt}%
\definecolor{currentstroke}{rgb}{0.000000,0.000000,0.000000}%
\pgfsetstrokecolor{currentstroke}%
\pgfsetdash{}{0pt}%
\pgfpathmoveto{\pgfqpoint{5.341356in}{0.974376in}}%
\pgfpathlineto{\pgfqpoint{5.422251in}{0.888724in}}%
\pgfpathlineto{\pgfqpoint{5.387596in}{1.302943in}}%
\pgfpathlineto{\pgfqpoint{5.341356in}{0.974376in}}%
\pgfpathclose%
\pgfusepath{fill}%
\end{pgfscope}%
\begin{pgfscope}%
\pgfpathrectangle{\pgfqpoint{3.536584in}{0.147348in}}{\pgfqpoint{2.735294in}{2.735294in}}%
\pgfusepath{clip}%
\pgfsetbuttcap%
\pgfsetroundjoin%
\definecolor{currentfill}{rgb}{0.070885,0.257762,0.070885}%
\pgfsetfillcolor{currentfill}%
\pgfsetfillopacity{0.200000}%
\pgfsetlinewidth{0.000000pt}%
\definecolor{currentstroke}{rgb}{0.000000,0.000000,0.000000}%
\pgfsetstrokecolor{currentstroke}%
\pgfsetdash{}{0pt}%
\pgfpathmoveto{\pgfqpoint{4.494794in}{1.302943in}}%
\pgfpathlineto{\pgfqpoint{4.460139in}{0.888724in}}%
\pgfpathlineto{\pgfqpoint{4.541034in}{0.974376in}}%
\pgfpathlineto{\pgfqpoint{4.494794in}{1.302943in}}%
\pgfpathclose%
\pgfusepath{fill}%
\end{pgfscope}%
\begin{pgfscope}%
\pgfpathrectangle{\pgfqpoint{3.536584in}{0.147348in}}{\pgfqpoint{2.735294in}{2.735294in}}%
\pgfusepath{clip}%
\pgfsetbuttcap%
\pgfsetroundjoin%
\definecolor{currentfill}{rgb}{0.111651,0.406004,0.111651}%
\pgfsetfillcolor{currentfill}%
\pgfsetfillopacity{0.200000}%
\pgfsetlinewidth{0.000000pt}%
\definecolor{currentstroke}{rgb}{0.000000,0.000000,0.000000}%
\pgfsetstrokecolor{currentstroke}%
\pgfsetdash{}{0pt}%
\pgfpathmoveto{\pgfqpoint{5.342287in}{2.141900in}}%
\pgfpathlineto{\pgfqpoint{5.423597in}{2.292690in}}%
\pgfpathlineto{\pgfqpoint{5.079487in}{2.150689in}}%
\pgfpathlineto{\pgfqpoint{5.342287in}{2.141900in}}%
\pgfpathclose%
\pgfusepath{fill}%
\end{pgfscope}%
\begin{pgfscope}%
\pgfpathrectangle{\pgfqpoint{3.536584in}{0.147348in}}{\pgfqpoint{2.735294in}{2.735294in}}%
\pgfusepath{clip}%
\pgfsetbuttcap%
\pgfsetroundjoin%
\definecolor{currentfill}{rgb}{0.111651,0.406004,0.111651}%
\pgfsetfillcolor{currentfill}%
\pgfsetfillopacity{0.200000}%
\pgfsetlinewidth{0.000000pt}%
\definecolor{currentstroke}{rgb}{0.000000,0.000000,0.000000}%
\pgfsetstrokecolor{currentstroke}%
\pgfsetdash{}{0pt}%
\pgfpathmoveto{\pgfqpoint{4.802903in}{2.150689in}}%
\pgfpathlineto{\pgfqpoint{4.458793in}{2.292690in}}%
\pgfpathlineto{\pgfqpoint{4.540103in}{2.141900in}}%
\pgfpathlineto{\pgfqpoint{4.802903in}{2.150689in}}%
\pgfpathclose%
\pgfusepath{fill}%
\end{pgfscope}%
\begin{pgfscope}%
\pgfpathrectangle{\pgfqpoint{3.536584in}{0.147348in}}{\pgfqpoint{2.735294in}{2.735294in}}%
\pgfusepath{clip}%
\pgfsetbuttcap%
\pgfsetroundjoin%
\definecolor{currentfill}{rgb}{0.092193,0.335248,0.092193}%
\pgfsetfillcolor{currentfill}%
\pgfsetfillopacity{0.200000}%
\pgfsetlinewidth{0.000000pt}%
\definecolor{currentstroke}{rgb}{0.000000,0.000000,0.000000}%
\pgfsetstrokecolor{currentstroke}%
\pgfsetdash{}{0pt}%
\pgfpathmoveto{\pgfqpoint{4.357383in}{1.526914in}}%
\pgfpathlineto{\pgfqpoint{4.312652in}{2.126617in}}%
\pgfpathlineto{\pgfqpoint{4.139943in}{1.557460in}}%
\pgfpathlineto{\pgfqpoint{4.357383in}{1.526914in}}%
\pgfpathclose%
\pgfusepath{fill}%
\end{pgfscope}%
\begin{pgfscope}%
\pgfpathrectangle{\pgfqpoint{3.536584in}{0.147348in}}{\pgfqpoint{2.735294in}{2.735294in}}%
\pgfusepath{clip}%
\pgfsetbuttcap%
\pgfsetroundjoin%
\definecolor{currentfill}{rgb}{0.092193,0.335248,0.092193}%
\pgfsetfillcolor{currentfill}%
\pgfsetfillopacity{0.200000}%
\pgfsetlinewidth{0.000000pt}%
\definecolor{currentstroke}{rgb}{0.000000,0.000000,0.000000}%
\pgfsetstrokecolor{currentstroke}%
\pgfsetdash{}{0pt}%
\pgfpathmoveto{\pgfqpoint{5.742447in}{1.557460in}}%
\pgfpathlineto{\pgfqpoint{5.569738in}{2.126617in}}%
\pgfpathlineto{\pgfqpoint{5.525007in}{1.526914in}}%
\pgfpathlineto{\pgfqpoint{5.742447in}{1.557460in}}%
\pgfpathclose%
\pgfusepath{fill}%
\end{pgfscope}%
\begin{pgfscope}%
\pgfpathrectangle{\pgfqpoint{3.536584in}{0.147348in}}{\pgfqpoint{2.735294in}{2.735294in}}%
\pgfusepath{clip}%
\pgfsetbuttcap%
\pgfsetroundjoin%
\definecolor{currentfill}{rgb}{0.060562,0.220227,0.060562}%
\pgfsetfillcolor{currentfill}%
\pgfsetfillopacity{0.200000}%
\pgfsetlinewidth{0.000000pt}%
\definecolor{currentstroke}{rgb}{0.000000,0.000000,0.000000}%
\pgfsetstrokecolor{currentstroke}%
\pgfsetdash{}{0pt}%
\pgfpathmoveto{\pgfqpoint{5.341356in}{0.974376in}}%
\pgfpathlineto{\pgfqpoint{5.096361in}{1.277996in}}%
\pgfpathlineto{\pgfqpoint{5.191972in}{0.846223in}}%
\pgfpathlineto{\pgfqpoint{5.341356in}{0.974376in}}%
\pgfpathclose%
\pgfusepath{fill}%
\end{pgfscope}%
\begin{pgfscope}%
\pgfpathrectangle{\pgfqpoint{3.536584in}{0.147348in}}{\pgfqpoint{2.735294in}{2.735294in}}%
\pgfusepath{clip}%
\pgfsetbuttcap%
\pgfsetroundjoin%
\definecolor{currentfill}{rgb}{0.060562,0.220227,0.060562}%
\pgfsetfillcolor{currentfill}%
\pgfsetfillopacity{0.200000}%
\pgfsetlinewidth{0.000000pt}%
\definecolor{currentstroke}{rgb}{0.000000,0.000000,0.000000}%
\pgfsetstrokecolor{currentstroke}%
\pgfsetdash{}{0pt}%
\pgfpathmoveto{\pgfqpoint{4.690418in}{0.846223in}}%
\pgfpathlineto{\pgfqpoint{4.786029in}{1.277996in}}%
\pgfpathlineto{\pgfqpoint{4.541034in}{0.974376in}}%
\pgfpathlineto{\pgfqpoint{4.690418in}{0.846223in}}%
\pgfpathclose%
\pgfusepath{fill}%
\end{pgfscope}%
\begin{pgfscope}%
\pgfpathrectangle{\pgfqpoint{3.536584in}{0.147348in}}{\pgfqpoint{2.735294in}{2.735294in}}%
\pgfusepath{clip}%
\pgfsetbuttcap%
\pgfsetroundjoin%
\definecolor{currentfill}{rgb}{0.097285,0.353762,0.097285}%
\pgfsetfillcolor{currentfill}%
\pgfsetfillopacity{0.200000}%
\pgfsetlinewidth{0.000000pt}%
\definecolor{currentstroke}{rgb}{0.000000,0.000000,0.000000}%
\pgfsetstrokecolor{currentstroke}%
\pgfsetdash{}{0pt}%
\pgfpathmoveto{\pgfqpoint{4.540103in}{2.141900in}}%
\pgfpathlineto{\pgfqpoint{4.312652in}{2.126617in}}%
\pgfpathlineto{\pgfqpoint{4.649042in}{1.952591in}}%
\pgfpathlineto{\pgfqpoint{4.540103in}{2.141900in}}%
\pgfpathclose%
\pgfusepath{fill}%
\end{pgfscope}%
\begin{pgfscope}%
\pgfpathrectangle{\pgfqpoint{3.536584in}{0.147348in}}{\pgfqpoint{2.735294in}{2.735294in}}%
\pgfusepath{clip}%
\pgfsetbuttcap%
\pgfsetroundjoin%
\definecolor{currentfill}{rgb}{0.097285,0.353762,0.097285}%
\pgfsetfillcolor{currentfill}%
\pgfsetfillopacity{0.200000}%
\pgfsetlinewidth{0.000000pt}%
\definecolor{currentstroke}{rgb}{0.000000,0.000000,0.000000}%
\pgfsetstrokecolor{currentstroke}%
\pgfsetdash{}{0pt}%
\pgfpathmoveto{\pgfqpoint{5.233348in}{1.952591in}}%
\pgfpathlineto{\pgfqpoint{5.569738in}{2.126617in}}%
\pgfpathlineto{\pgfqpoint{5.342287in}{2.141900in}}%
\pgfpathlineto{\pgfqpoint{5.233348in}{1.952591in}}%
\pgfpathclose%
\pgfusepath{fill}%
\end{pgfscope}%
\begin{pgfscope}%
\pgfpathrectangle{\pgfqpoint{3.536584in}{0.147348in}}{\pgfqpoint{2.735294in}{2.735294in}}%
\pgfusepath{clip}%
\pgfsetbuttcap%
\pgfsetroundjoin%
\definecolor{currentfill}{rgb}{0.067497,0.245443,0.067497}%
\pgfsetfillcolor{currentfill}%
\pgfsetfillopacity{0.200000}%
\pgfsetlinewidth{0.000000pt}%
\definecolor{currentstroke}{rgb}{0.000000,0.000000,0.000000}%
\pgfsetstrokecolor{currentstroke}%
\pgfsetdash{}{0pt}%
\pgfpathmoveto{\pgfqpoint{4.690418in}{0.846223in}}%
\pgfpathlineto{\pgfqpoint{4.803235in}{0.943120in}}%
\pgfpathlineto{\pgfqpoint{4.786029in}{1.277996in}}%
\pgfpathlineto{\pgfqpoint{4.690418in}{0.846223in}}%
\pgfpathclose%
\pgfusepath{fill}%
\end{pgfscope}%
\begin{pgfscope}%
\pgfpathrectangle{\pgfqpoint{3.536584in}{0.147348in}}{\pgfqpoint{2.735294in}{2.735294in}}%
\pgfusepath{clip}%
\pgfsetbuttcap%
\pgfsetroundjoin%
\definecolor{currentfill}{rgb}{0.067497,0.245443,0.067497}%
\pgfsetfillcolor{currentfill}%
\pgfsetfillopacity{0.200000}%
\pgfsetlinewidth{0.000000pt}%
\definecolor{currentstroke}{rgb}{0.000000,0.000000,0.000000}%
\pgfsetstrokecolor{currentstroke}%
\pgfsetdash{}{0pt}%
\pgfpathmoveto{\pgfqpoint{5.096361in}{1.277996in}}%
\pgfpathlineto{\pgfqpoint{5.079154in}{0.943120in}}%
\pgfpathlineto{\pgfqpoint{5.191972in}{0.846223in}}%
\pgfpathlineto{\pgfqpoint{5.096361in}{1.277996in}}%
\pgfpathclose%
\pgfusepath{fill}%
\end{pgfscope}%
\begin{pgfscope}%
\pgfpathrectangle{\pgfqpoint{3.536584in}{0.147348in}}{\pgfqpoint{2.735294in}{2.735294in}}%
\pgfusepath{clip}%
\pgfsetbuttcap%
\pgfsetroundjoin%
\definecolor{currentfill}{rgb}{0.065434,0.237940,0.065434}%
\pgfsetfillcolor{currentfill}%
\pgfsetfillopacity{0.200000}%
\pgfsetlinewidth{0.000000pt}%
\definecolor{currentstroke}{rgb}{0.000000,0.000000,0.000000}%
\pgfsetstrokecolor{currentstroke}%
\pgfsetdash{}{0pt}%
\pgfpathmoveto{\pgfqpoint{4.941195in}{0.830831in}}%
\pgfpathlineto{\pgfqpoint{4.786029in}{1.277996in}}%
\pgfpathlineto{\pgfqpoint{4.803235in}{0.943120in}}%
\pgfpathlineto{\pgfqpoint{4.941195in}{0.830831in}}%
\pgfpathclose%
\pgfusepath{fill}%
\end{pgfscope}%
\begin{pgfscope}%
\pgfpathrectangle{\pgfqpoint{3.536584in}{0.147348in}}{\pgfqpoint{2.735294in}{2.735294in}}%
\pgfusepath{clip}%
\pgfsetbuttcap%
\pgfsetroundjoin%
\definecolor{currentfill}{rgb}{0.065434,0.237940,0.065434}%
\pgfsetfillcolor{currentfill}%
\pgfsetfillopacity{0.200000}%
\pgfsetlinewidth{0.000000pt}%
\definecolor{currentstroke}{rgb}{0.000000,0.000000,0.000000}%
\pgfsetstrokecolor{currentstroke}%
\pgfsetdash{}{0pt}%
\pgfpathmoveto{\pgfqpoint{5.079154in}{0.943120in}}%
\pgfpathlineto{\pgfqpoint{5.096361in}{1.277996in}}%
\pgfpathlineto{\pgfqpoint{4.941195in}{0.830831in}}%
\pgfpathlineto{\pgfqpoint{5.079154in}{0.943120in}}%
\pgfpathclose%
\pgfusepath{fill}%
\end{pgfscope}%
\begin{pgfscope}%
\pgfpathrectangle{\pgfqpoint{3.536584in}{0.147348in}}{\pgfqpoint{2.735294in}{2.735294in}}%
\pgfusepath{clip}%
\pgfsetbuttcap%
\pgfsetroundjoin%
\definecolor{currentfill}{rgb}{0.073593,0.267612,0.073593}%
\pgfsetfillcolor{currentfill}%
\pgfsetfillopacity{0.200000}%
\pgfsetlinewidth{0.000000pt}%
\definecolor{currentstroke}{rgb}{0.000000,0.000000,0.000000}%
\pgfsetstrokecolor{currentstroke}%
\pgfsetdash{}{0pt}%
\pgfpathmoveto{\pgfqpoint{4.314024in}{1.028737in}}%
\pgfpathlineto{\pgfqpoint{4.494794in}{1.302943in}}%
\pgfpathlineto{\pgfqpoint{4.357383in}{1.526914in}}%
\pgfpathlineto{\pgfqpoint{4.314024in}{1.028737in}}%
\pgfpathclose%
\pgfusepath{fill}%
\end{pgfscope}%
\begin{pgfscope}%
\pgfpathrectangle{\pgfqpoint{3.536584in}{0.147348in}}{\pgfqpoint{2.735294in}{2.735294in}}%
\pgfusepath{clip}%
\pgfsetbuttcap%
\pgfsetroundjoin%
\definecolor{currentfill}{rgb}{0.073593,0.267612,0.073593}%
\pgfsetfillcolor{currentfill}%
\pgfsetfillopacity{0.200000}%
\pgfsetlinewidth{0.000000pt}%
\definecolor{currentstroke}{rgb}{0.000000,0.000000,0.000000}%
\pgfsetstrokecolor{currentstroke}%
\pgfsetdash{}{0pt}%
\pgfpathmoveto{\pgfqpoint{5.525007in}{1.526914in}}%
\pgfpathlineto{\pgfqpoint{5.387596in}{1.302943in}}%
\pgfpathlineto{\pgfqpoint{5.568365in}{1.028737in}}%
\pgfpathlineto{\pgfqpoint{5.525007in}{1.526914in}}%
\pgfpathclose%
\pgfusepath{fill}%
\end{pgfscope}%
\begin{pgfscope}%
\pgfpathrectangle{\pgfqpoint{3.536584in}{0.147348in}}{\pgfqpoint{2.735294in}{2.735294in}}%
\pgfusepath{clip}%
\pgfsetbuttcap%
\pgfsetroundjoin%
\definecolor{currentfill}{rgb}{0.091915,0.334238,0.091915}%
\pgfsetfillcolor{currentfill}%
\pgfsetfillopacity{0.200000}%
\pgfsetlinewidth{0.000000pt}%
\definecolor{currentstroke}{rgb}{0.000000,0.000000,0.000000}%
\pgfsetstrokecolor{currentstroke}%
\pgfsetdash{}{0pt}%
\pgfpathmoveto{\pgfqpoint{4.357383in}{1.526914in}}%
\pgfpathlineto{\pgfqpoint{4.649042in}{1.952591in}}%
\pgfpathlineto{\pgfqpoint{4.312652in}{2.126617in}}%
\pgfpathlineto{\pgfqpoint{4.357383in}{1.526914in}}%
\pgfpathclose%
\pgfusepath{fill}%
\end{pgfscope}%
\begin{pgfscope}%
\pgfpathrectangle{\pgfqpoint{3.536584in}{0.147348in}}{\pgfqpoint{2.735294in}{2.735294in}}%
\pgfusepath{clip}%
\pgfsetbuttcap%
\pgfsetroundjoin%
\definecolor{currentfill}{rgb}{0.091915,0.334238,0.091915}%
\pgfsetfillcolor{currentfill}%
\pgfsetfillopacity{0.200000}%
\pgfsetlinewidth{0.000000pt}%
\definecolor{currentstroke}{rgb}{0.000000,0.000000,0.000000}%
\pgfsetstrokecolor{currentstroke}%
\pgfsetdash{}{0pt}%
\pgfpathmoveto{\pgfqpoint{5.569738in}{2.126617in}}%
\pgfpathlineto{\pgfqpoint{5.233348in}{1.952591in}}%
\pgfpathlineto{\pgfqpoint{5.525007in}{1.526914in}}%
\pgfpathlineto{\pgfqpoint{5.569738in}{2.126617in}}%
\pgfpathclose%
\pgfusepath{fill}%
\end{pgfscope}%
\begin{pgfscope}%
\pgfpathrectangle{\pgfqpoint{3.536584in}{0.147348in}}{\pgfqpoint{2.735294in}{2.735294in}}%
\pgfusepath{clip}%
\pgfsetbuttcap%
\pgfsetroundjoin%
\definecolor{currentfill}{rgb}{0.101759,0.370033,0.101759}%
\pgfsetfillcolor{currentfill}%
\pgfsetfillopacity{0.200000}%
\pgfsetlinewidth{0.000000pt}%
\definecolor{currentstroke}{rgb}{0.000000,0.000000,0.000000}%
\pgfsetstrokecolor{currentstroke}%
\pgfsetdash{}{0pt}%
\pgfpathmoveto{\pgfqpoint{4.649042in}{1.952591in}}%
\pgfpathlineto{\pgfqpoint{4.802903in}{2.150689in}}%
\pgfpathlineto{\pgfqpoint{4.540103in}{2.141900in}}%
\pgfpathlineto{\pgfqpoint{4.649042in}{1.952591in}}%
\pgfpathclose%
\pgfusepath{fill}%
\end{pgfscope}%
\begin{pgfscope}%
\pgfpathrectangle{\pgfqpoint{3.536584in}{0.147348in}}{\pgfqpoint{2.735294in}{2.735294in}}%
\pgfusepath{clip}%
\pgfsetbuttcap%
\pgfsetroundjoin%
\definecolor{currentfill}{rgb}{0.101759,0.370033,0.101759}%
\pgfsetfillcolor{currentfill}%
\pgfsetfillopacity{0.200000}%
\pgfsetlinewidth{0.000000pt}%
\definecolor{currentstroke}{rgb}{0.000000,0.000000,0.000000}%
\pgfsetstrokecolor{currentstroke}%
\pgfsetdash{}{0pt}%
\pgfpathmoveto{\pgfqpoint{5.342287in}{2.141900in}}%
\pgfpathlineto{\pgfqpoint{5.079487in}{2.150689in}}%
\pgfpathlineto{\pgfqpoint{5.233348in}{1.952591in}}%
\pgfpathlineto{\pgfqpoint{5.342287in}{2.141900in}}%
\pgfpathclose%
\pgfusepath{fill}%
\end{pgfscope}%
\begin{pgfscope}%
\pgfpathrectangle{\pgfqpoint{3.536584in}{0.147348in}}{\pgfqpoint{2.735294in}{2.735294in}}%
\pgfusepath{clip}%
\pgfsetbuttcap%
\pgfsetroundjoin%
\definecolor{currentfill}{rgb}{0.101677,0.369734,0.101677}%
\pgfsetfillcolor{currentfill}%
\pgfsetfillopacity{0.200000}%
\pgfsetlinewidth{0.000000pt}%
\definecolor{currentstroke}{rgb}{0.000000,0.000000,0.000000}%
\pgfsetstrokecolor{currentstroke}%
\pgfsetdash{}{0pt}%
\pgfpathmoveto{\pgfqpoint{4.941195in}{1.953918in}}%
\pgfpathlineto{\pgfqpoint{5.079487in}{2.150689in}}%
\pgfpathlineto{\pgfqpoint{4.802903in}{2.150689in}}%
\pgfpathlineto{\pgfqpoint{4.941195in}{1.953918in}}%
\pgfpathclose%
\pgfusepath{fill}%
\end{pgfscope}%
\begin{pgfscope}%
\pgfpathrectangle{\pgfqpoint{3.536584in}{0.147348in}}{\pgfqpoint{2.735294in}{2.735294in}}%
\pgfusepath{clip}%
\pgfsetbuttcap%
\pgfsetroundjoin%
\definecolor{currentfill}{rgb}{0.065035,0.236492,0.065035}%
\pgfsetfillcolor{currentfill}%
\pgfsetfillopacity{0.200000}%
\pgfsetlinewidth{0.000000pt}%
\definecolor{currentstroke}{rgb}{0.000000,0.000000,0.000000}%
\pgfsetstrokecolor{currentstroke}%
\pgfsetdash{}{0pt}%
\pgfpathmoveto{\pgfqpoint{4.541034in}{0.974376in}}%
\pgfpathlineto{\pgfqpoint{4.630724in}{1.504068in}}%
\pgfpathlineto{\pgfqpoint{4.494794in}{1.302943in}}%
\pgfpathlineto{\pgfqpoint{4.541034in}{0.974376in}}%
\pgfpathclose%
\pgfusepath{fill}%
\end{pgfscope}%
\begin{pgfscope}%
\pgfpathrectangle{\pgfqpoint{3.536584in}{0.147348in}}{\pgfqpoint{2.735294in}{2.735294in}}%
\pgfusepath{clip}%
\pgfsetbuttcap%
\pgfsetroundjoin%
\definecolor{currentfill}{rgb}{0.065035,0.236492,0.065035}%
\pgfsetfillcolor{currentfill}%
\pgfsetfillopacity{0.200000}%
\pgfsetlinewidth{0.000000pt}%
\definecolor{currentstroke}{rgb}{0.000000,0.000000,0.000000}%
\pgfsetstrokecolor{currentstroke}%
\pgfsetdash{}{0pt}%
\pgfpathmoveto{\pgfqpoint{5.387596in}{1.302943in}}%
\pgfpathlineto{\pgfqpoint{5.251666in}{1.504068in}}%
\pgfpathlineto{\pgfqpoint{5.341356in}{0.974376in}}%
\pgfpathlineto{\pgfqpoint{5.387596in}{1.302943in}}%
\pgfpathclose%
\pgfusepath{fill}%
\end{pgfscope}%
\begin{pgfscope}%
\pgfpathrectangle{\pgfqpoint{3.536584in}{0.147348in}}{\pgfqpoint{2.735294in}{2.735294in}}%
\pgfusepath{clip}%
\pgfsetbuttcap%
\pgfsetroundjoin%
\definecolor{currentfill}{rgb}{0.066446,0.241622,0.066446}%
\pgfsetfillcolor{currentfill}%
\pgfsetfillopacity{0.200000}%
\pgfsetlinewidth{0.000000pt}%
\definecolor{currentstroke}{rgb}{0.000000,0.000000,0.000000}%
\pgfsetstrokecolor{currentstroke}%
\pgfsetdash{}{0pt}%
\pgfpathmoveto{\pgfqpoint{4.786029in}{1.277996in}}%
\pgfpathlineto{\pgfqpoint{4.941195in}{0.830831in}}%
\pgfpathlineto{\pgfqpoint{4.941195in}{1.495457in}}%
\pgfpathlineto{\pgfqpoint{4.786029in}{1.277996in}}%
\pgfpathclose%
\pgfusepath{fill}%
\end{pgfscope}%
\begin{pgfscope}%
\pgfpathrectangle{\pgfqpoint{3.536584in}{0.147348in}}{\pgfqpoint{2.735294in}{2.735294in}}%
\pgfusepath{clip}%
\pgfsetbuttcap%
\pgfsetroundjoin%
\definecolor{currentfill}{rgb}{0.066446,0.241622,0.066446}%
\pgfsetfillcolor{currentfill}%
\pgfsetfillopacity{0.200000}%
\pgfsetlinewidth{0.000000pt}%
\definecolor{currentstroke}{rgb}{0.000000,0.000000,0.000000}%
\pgfsetstrokecolor{currentstroke}%
\pgfsetdash{}{0pt}%
\pgfpathmoveto{\pgfqpoint{4.941195in}{1.495457in}}%
\pgfpathlineto{\pgfqpoint{4.941195in}{0.830831in}}%
\pgfpathlineto{\pgfqpoint{5.096361in}{1.277996in}}%
\pgfpathlineto{\pgfqpoint{4.941195in}{1.495457in}}%
\pgfpathclose%
\pgfusepath{fill}%
\end{pgfscope}%
\begin{pgfscope}%
\pgfpathrectangle{\pgfqpoint{3.536584in}{0.147348in}}{\pgfqpoint{2.735294in}{2.735294in}}%
\pgfusepath{clip}%
\pgfsetbuttcap%
\pgfsetroundjoin%
\definecolor{currentfill}{rgb}{0.098306,0.357475,0.098306}%
\pgfsetfillcolor{currentfill}%
\pgfsetfillopacity{0.200000}%
\pgfsetlinewidth{0.000000pt}%
\definecolor{currentstroke}{rgb}{0.000000,0.000000,0.000000}%
\pgfsetstrokecolor{currentstroke}%
\pgfsetdash{}{0pt}%
\pgfpathmoveto{\pgfqpoint{4.941195in}{1.953918in}}%
\pgfpathlineto{\pgfqpoint{4.802903in}{2.150689in}}%
\pgfpathlineto{\pgfqpoint{4.649042in}{1.952591in}}%
\pgfpathlineto{\pgfqpoint{4.941195in}{1.953918in}}%
\pgfpathclose%
\pgfusepath{fill}%
\end{pgfscope}%
\begin{pgfscope}%
\pgfpathrectangle{\pgfqpoint{3.536584in}{0.147348in}}{\pgfqpoint{2.735294in}{2.735294in}}%
\pgfusepath{clip}%
\pgfsetbuttcap%
\pgfsetroundjoin%
\definecolor{currentfill}{rgb}{0.098306,0.357475,0.098306}%
\pgfsetfillcolor{currentfill}%
\pgfsetfillopacity{0.200000}%
\pgfsetlinewidth{0.000000pt}%
\definecolor{currentstroke}{rgb}{0.000000,0.000000,0.000000}%
\pgfsetstrokecolor{currentstroke}%
\pgfsetdash{}{0pt}%
\pgfpathmoveto{\pgfqpoint{5.233348in}{1.952591in}}%
\pgfpathlineto{\pgfqpoint{5.079487in}{2.150689in}}%
\pgfpathlineto{\pgfqpoint{4.941195in}{1.953918in}}%
\pgfpathlineto{\pgfqpoint{5.233348in}{1.952591in}}%
\pgfpathclose%
\pgfusepath{fill}%
\end{pgfscope}%
\begin{pgfscope}%
\pgfpathrectangle{\pgfqpoint{3.536584in}{0.147348in}}{\pgfqpoint{2.735294in}{2.735294in}}%
\pgfusepath{clip}%
\pgfsetbuttcap%
\pgfsetroundjoin%
\definecolor{currentfill}{rgb}{0.070209,0.255305,0.070209}%
\pgfsetfillcolor{currentfill}%
\pgfsetfillopacity{0.200000}%
\pgfsetlinewidth{0.000000pt}%
\definecolor{currentstroke}{rgb}{0.000000,0.000000,0.000000}%
\pgfsetstrokecolor{currentstroke}%
\pgfsetdash{}{0pt}%
\pgfpathmoveto{\pgfqpoint{4.541034in}{0.974376in}}%
\pgfpathlineto{\pgfqpoint{4.786029in}{1.277996in}}%
\pgfpathlineto{\pgfqpoint{4.630724in}{1.504068in}}%
\pgfpathlineto{\pgfqpoint{4.541034in}{0.974376in}}%
\pgfpathclose%
\pgfusepath{fill}%
\end{pgfscope}%
\begin{pgfscope}%
\pgfpathrectangle{\pgfqpoint{3.536584in}{0.147348in}}{\pgfqpoint{2.735294in}{2.735294in}}%
\pgfusepath{clip}%
\pgfsetbuttcap%
\pgfsetroundjoin%
\definecolor{currentfill}{rgb}{0.070209,0.255305,0.070209}%
\pgfsetfillcolor{currentfill}%
\pgfsetfillopacity{0.200000}%
\pgfsetlinewidth{0.000000pt}%
\definecolor{currentstroke}{rgb}{0.000000,0.000000,0.000000}%
\pgfsetstrokecolor{currentstroke}%
\pgfsetdash{}{0pt}%
\pgfpathmoveto{\pgfqpoint{5.251666in}{1.504068in}}%
\pgfpathlineto{\pgfqpoint{5.096361in}{1.277996in}}%
\pgfpathlineto{\pgfqpoint{5.341356in}{0.974376in}}%
\pgfpathlineto{\pgfqpoint{5.251666in}{1.504068in}}%
\pgfpathclose%
\pgfusepath{fill}%
\end{pgfscope}%
\begin{pgfscope}%
\pgfpathrectangle{\pgfqpoint{3.536584in}{0.147348in}}{\pgfqpoint{2.735294in}{2.735294in}}%
\pgfusepath{clip}%
\pgfsetbuttcap%
\pgfsetroundjoin%
\definecolor{currentfill}{rgb}{0.087398,0.317812,0.087398}%
\pgfsetfillcolor{currentfill}%
\pgfsetfillopacity{0.200000}%
\pgfsetlinewidth{0.000000pt}%
\definecolor{currentstroke}{rgb}{0.000000,0.000000,0.000000}%
\pgfsetstrokecolor{currentstroke}%
\pgfsetdash{}{0pt}%
\pgfpathmoveto{\pgfqpoint{5.525007in}{1.526914in}}%
\pgfpathlineto{\pgfqpoint{5.233348in}{1.952591in}}%
\pgfpathlineto{\pgfqpoint{5.251666in}{1.504068in}}%
\pgfpathlineto{\pgfqpoint{5.525007in}{1.526914in}}%
\pgfpathclose%
\pgfusepath{fill}%
\end{pgfscope}%
\begin{pgfscope}%
\pgfpathrectangle{\pgfqpoint{3.536584in}{0.147348in}}{\pgfqpoint{2.735294in}{2.735294in}}%
\pgfusepath{clip}%
\pgfsetbuttcap%
\pgfsetroundjoin%
\definecolor{currentfill}{rgb}{0.087398,0.317812,0.087398}%
\pgfsetfillcolor{currentfill}%
\pgfsetfillopacity{0.200000}%
\pgfsetlinewidth{0.000000pt}%
\definecolor{currentstroke}{rgb}{0.000000,0.000000,0.000000}%
\pgfsetstrokecolor{currentstroke}%
\pgfsetdash{}{0pt}%
\pgfpathmoveto{\pgfqpoint{4.630724in}{1.504068in}}%
\pgfpathlineto{\pgfqpoint{4.649042in}{1.952591in}}%
\pgfpathlineto{\pgfqpoint{4.357383in}{1.526914in}}%
\pgfpathlineto{\pgfqpoint{4.630724in}{1.504068in}}%
\pgfpathclose%
\pgfusepath{fill}%
\end{pgfscope}%
\begin{pgfscope}%
\pgfpathrectangle{\pgfqpoint{3.536584in}{0.147348in}}{\pgfqpoint{2.735294in}{2.735294in}}%
\pgfusepath{clip}%
\pgfsetbuttcap%
\pgfsetroundjoin%
\definecolor{currentfill}{rgb}{0.075994,0.276341,0.075994}%
\pgfsetfillcolor{currentfill}%
\pgfsetfillopacity{0.200000}%
\pgfsetlinewidth{0.000000pt}%
\definecolor{currentstroke}{rgb}{0.000000,0.000000,0.000000}%
\pgfsetstrokecolor{currentstroke}%
\pgfsetdash{}{0pt}%
\pgfpathmoveto{\pgfqpoint{4.357383in}{1.526914in}}%
\pgfpathlineto{\pgfqpoint{4.494794in}{1.302943in}}%
\pgfpathlineto{\pgfqpoint{4.630724in}{1.504068in}}%
\pgfpathlineto{\pgfqpoint{4.357383in}{1.526914in}}%
\pgfpathclose%
\pgfusepath{fill}%
\end{pgfscope}%
\begin{pgfscope}%
\pgfpathrectangle{\pgfqpoint{3.536584in}{0.147348in}}{\pgfqpoint{2.735294in}{2.735294in}}%
\pgfusepath{clip}%
\pgfsetbuttcap%
\pgfsetroundjoin%
\definecolor{currentfill}{rgb}{0.075994,0.276341,0.075994}%
\pgfsetfillcolor{currentfill}%
\pgfsetfillopacity{0.200000}%
\pgfsetlinewidth{0.000000pt}%
\definecolor{currentstroke}{rgb}{0.000000,0.000000,0.000000}%
\pgfsetstrokecolor{currentstroke}%
\pgfsetdash{}{0pt}%
\pgfpathmoveto{\pgfqpoint{5.251666in}{1.504068in}}%
\pgfpathlineto{\pgfqpoint{5.387596in}{1.302943in}}%
\pgfpathlineto{\pgfqpoint{5.525007in}{1.526914in}}%
\pgfpathlineto{\pgfqpoint{5.251666in}{1.504068in}}%
\pgfpathclose%
\pgfusepath{fill}%
\end{pgfscope}%
\begin{pgfscope}%
\pgfpathrectangle{\pgfqpoint{3.536584in}{0.147348in}}{\pgfqpoint{2.735294in}{2.735294in}}%
\pgfusepath{clip}%
\pgfsetbuttcap%
\pgfsetroundjoin%
\definecolor{currentfill}{rgb}{0.086061,0.312950,0.086061}%
\pgfsetfillcolor{currentfill}%
\pgfsetfillopacity{0.200000}%
\pgfsetlinewidth{0.000000pt}%
\definecolor{currentstroke}{rgb}{0.000000,0.000000,0.000000}%
\pgfsetstrokecolor{currentstroke}%
\pgfsetdash{}{0pt}%
\pgfpathmoveto{\pgfqpoint{4.649042in}{1.952591in}}%
\pgfpathlineto{\pgfqpoint{4.630724in}{1.504068in}}%
\pgfpathlineto{\pgfqpoint{4.941195in}{1.953918in}}%
\pgfpathlineto{\pgfqpoint{4.649042in}{1.952591in}}%
\pgfpathclose%
\pgfusepath{fill}%
\end{pgfscope}%
\begin{pgfscope}%
\pgfpathrectangle{\pgfqpoint{3.536584in}{0.147348in}}{\pgfqpoint{2.735294in}{2.735294in}}%
\pgfusepath{clip}%
\pgfsetbuttcap%
\pgfsetroundjoin%
\definecolor{currentfill}{rgb}{0.086061,0.312950,0.086061}%
\pgfsetfillcolor{currentfill}%
\pgfsetfillopacity{0.200000}%
\pgfsetlinewidth{0.000000pt}%
\definecolor{currentstroke}{rgb}{0.000000,0.000000,0.000000}%
\pgfsetstrokecolor{currentstroke}%
\pgfsetdash{}{0pt}%
\pgfpathmoveto{\pgfqpoint{4.941195in}{1.953918in}}%
\pgfpathlineto{\pgfqpoint{5.251666in}{1.504068in}}%
\pgfpathlineto{\pgfqpoint{5.233348in}{1.952591in}}%
\pgfpathlineto{\pgfqpoint{4.941195in}{1.953918in}}%
\pgfpathclose%
\pgfusepath{fill}%
\end{pgfscope}%
\begin{pgfscope}%
\pgfpathrectangle{\pgfqpoint{3.536584in}{0.147348in}}{\pgfqpoint{2.735294in}{2.735294in}}%
\pgfusepath{clip}%
\pgfsetbuttcap%
\pgfsetroundjoin%
\definecolor{currentfill}{rgb}{0.086258,0.313666,0.086258}%
\pgfsetfillcolor{currentfill}%
\pgfsetfillopacity{0.200000}%
\pgfsetlinewidth{0.000000pt}%
\definecolor{currentstroke}{rgb}{0.000000,0.000000,0.000000}%
\pgfsetstrokecolor{currentstroke}%
\pgfsetdash{}{0pt}%
\pgfpathmoveto{\pgfqpoint{4.941195in}{1.495457in}}%
\pgfpathlineto{\pgfqpoint{4.941195in}{1.953918in}}%
\pgfpathlineto{\pgfqpoint{4.630724in}{1.504068in}}%
\pgfpathlineto{\pgfqpoint{4.941195in}{1.495457in}}%
\pgfpathclose%
\pgfusepath{fill}%
\end{pgfscope}%
\begin{pgfscope}%
\pgfpathrectangle{\pgfqpoint{3.536584in}{0.147348in}}{\pgfqpoint{2.735294in}{2.735294in}}%
\pgfusepath{clip}%
\pgfsetbuttcap%
\pgfsetroundjoin%
\definecolor{currentfill}{rgb}{0.086258,0.313666,0.086258}%
\pgfsetfillcolor{currentfill}%
\pgfsetfillopacity{0.200000}%
\pgfsetlinewidth{0.000000pt}%
\definecolor{currentstroke}{rgb}{0.000000,0.000000,0.000000}%
\pgfsetstrokecolor{currentstroke}%
\pgfsetdash{}{0pt}%
\pgfpathmoveto{\pgfqpoint{5.251666in}{1.504068in}}%
\pgfpathlineto{\pgfqpoint{4.941195in}{1.953918in}}%
\pgfpathlineto{\pgfqpoint{4.941195in}{1.495457in}}%
\pgfpathlineto{\pgfqpoint{5.251666in}{1.504068in}}%
\pgfpathclose%
\pgfusepath{fill}%
\end{pgfscope}%
\begin{pgfscope}%
\pgfpathrectangle{\pgfqpoint{3.536584in}{0.147348in}}{\pgfqpoint{2.735294in}{2.735294in}}%
\pgfusepath{clip}%
\pgfsetbuttcap%
\pgfsetroundjoin%
\definecolor{currentfill}{rgb}{0.074668,0.271519,0.074668}%
\pgfsetfillcolor{currentfill}%
\pgfsetfillopacity{0.200000}%
\pgfsetlinewidth{0.000000pt}%
\definecolor{currentstroke}{rgb}{0.000000,0.000000,0.000000}%
\pgfsetstrokecolor{currentstroke}%
\pgfsetdash{}{0pt}%
\pgfpathmoveto{\pgfqpoint{4.630724in}{1.504068in}}%
\pgfpathlineto{\pgfqpoint{4.786029in}{1.277996in}}%
\pgfpathlineto{\pgfqpoint{4.941195in}{1.495457in}}%
\pgfpathlineto{\pgfqpoint{4.630724in}{1.504068in}}%
\pgfpathclose%
\pgfusepath{fill}%
\end{pgfscope}%
\begin{pgfscope}%
\pgfpathrectangle{\pgfqpoint{3.536584in}{0.147348in}}{\pgfqpoint{2.735294in}{2.735294in}}%
\pgfusepath{clip}%
\pgfsetbuttcap%
\pgfsetroundjoin%
\definecolor{currentfill}{rgb}{0.074668,0.271519,0.074668}%
\pgfsetfillcolor{currentfill}%
\pgfsetfillopacity{0.200000}%
\pgfsetlinewidth{0.000000pt}%
\definecolor{currentstroke}{rgb}{0.000000,0.000000,0.000000}%
\pgfsetstrokecolor{currentstroke}%
\pgfsetdash{}{0pt}%
\pgfpathmoveto{\pgfqpoint{4.941195in}{1.495457in}}%
\pgfpathlineto{\pgfqpoint{5.096361in}{1.277996in}}%
\pgfpathlineto{\pgfqpoint{5.251666in}{1.504068in}}%
\pgfpathlineto{\pgfqpoint{4.941195in}{1.495457in}}%
\pgfpathclose%
\pgfusepath{fill}%
\end{pgfscope}%
\begin{pgfscope}%
\pgfsetbuttcap%
\pgfsetmiterjoin%
\definecolor{currentfill}{rgb}{1.000000,1.000000,1.000000}%
\pgfsetfillcolor{currentfill}%
\pgfsetlinewidth{0.000000pt}%
\definecolor{currentstroke}{rgb}{0.000000,0.000000,0.000000}%
\pgfsetstrokecolor{currentstroke}%
\pgfsetstrokeopacity{0.000000}%
\pgfsetdash{}{0pt}%
\pgfpathmoveto{\pgfqpoint{6.818937in}{0.147348in}}%
\pgfpathlineto{\pgfqpoint{9.554231in}{0.147348in}}%
\pgfpathlineto{\pgfqpoint{9.554231in}{2.882642in}}%
\pgfpathlineto{\pgfqpoint{6.818937in}{2.882642in}}%
\pgfpathlineto{\pgfqpoint{6.818937in}{0.147348in}}%
\pgfpathclose%
\pgfusepath{fill}%
\end{pgfscope}%
\begin{pgfscope}%
\pgfsetbuttcap%
\pgfsetmiterjoin%
\definecolor{currentfill}{rgb}{0.950000,0.950000,0.950000}%
\pgfsetfillcolor{currentfill}%
\pgfsetfillopacity{0.500000}%
\pgfsetlinewidth{1.003750pt}%
\definecolor{currentstroke}{rgb}{0.950000,0.950000,0.950000}%
\pgfsetstrokecolor{currentstroke}%
\pgfsetstrokeopacity{0.500000}%
\pgfsetdash{}{0pt}%
\pgfpathmoveto{\pgfqpoint{9.508741in}{1.070011in}}%
\pgfpathlineto{\pgfqpoint{8.223548in}{0.241771in}}%
\pgfpathlineto{\pgfqpoint{8.223548in}{1.173119in}}%
\pgfpathlineto{\pgfqpoint{9.430084in}{2.004410in}}%
\pgfusepath{stroke,fill}%
\end{pgfscope}%
\begin{pgfscope}%
\pgfsetbuttcap%
\pgfsetmiterjoin%
\definecolor{currentfill}{rgb}{0.900000,0.900000,0.900000}%
\pgfsetfillcolor{currentfill}%
\pgfsetfillopacity{0.500000}%
\pgfsetlinewidth{1.003750pt}%
\definecolor{currentstroke}{rgb}{0.900000,0.900000,0.900000}%
\pgfsetstrokecolor{currentstroke}%
\pgfsetstrokeopacity{0.500000}%
\pgfsetdash{}{0pt}%
\pgfpathmoveto{\pgfqpoint{6.938354in}{1.070011in}}%
\pgfpathlineto{\pgfqpoint{8.223548in}{0.241771in}}%
\pgfpathlineto{\pgfqpoint{8.223548in}{1.173119in}}%
\pgfpathlineto{\pgfqpoint{7.017011in}{2.004410in}}%
\pgfusepath{stroke,fill}%
\end{pgfscope}%
\begin{pgfscope}%
\pgfsetbuttcap%
\pgfsetmiterjoin%
\definecolor{currentfill}{rgb}{0.925000,0.925000,0.925000}%
\pgfsetfillcolor{currentfill}%
\pgfsetfillopacity{0.500000}%
\pgfsetlinewidth{1.003750pt}%
\definecolor{currentstroke}{rgb}{0.925000,0.925000,0.925000}%
\pgfsetstrokecolor{currentstroke}%
\pgfsetstrokeopacity{0.500000}%
\pgfsetdash{}{0pt}%
\pgfpathmoveto{\pgfqpoint{8.223548in}{2.937509in}}%
\pgfpathlineto{\pgfqpoint{9.430084in}{2.004410in}}%
\pgfpathlineto{\pgfqpoint{8.223548in}{1.173119in}}%
\pgfpathlineto{\pgfqpoint{7.017011in}{2.004410in}}%
\pgfusepath{stroke,fill}%
\end{pgfscope}%
\begin{pgfscope}%
\pgfsetbuttcap%
\pgfsetroundjoin%
\pgfsetlinewidth{0.803000pt}%
\definecolor{currentstroke}{rgb}{0.690196,0.690196,0.690196}%
\pgfsetstrokecolor{currentstroke}%
\pgfsetdash{}{0pt}%
\pgfpathmoveto{\pgfqpoint{8.304713in}{2.874739in}}%
\pgfpathlineto{\pgfqpoint{7.097924in}{1.948662in}}%
\pgfpathlineto{\pgfqpoint{7.024828in}{1.014283in}}%
\pgfusepath{stroke}%
\end{pgfscope}%
\begin{pgfscope}%
\pgfsetbuttcap%
\pgfsetroundjoin%
\pgfsetlinewidth{0.803000pt}%
\definecolor{currentstroke}{rgb}{0.690196,0.690196,0.690196}%
\pgfsetstrokecolor{currentstroke}%
\pgfsetdash{}{0pt}%
\pgfpathmoveto{\pgfqpoint{8.553700in}{2.682179in}}%
\pgfpathlineto{\pgfqpoint{7.346364in}{1.777489in}}%
\pgfpathlineto{\pgfqpoint{7.290085in}{0.843339in}}%
\pgfusepath{stroke}%
\end{pgfscope}%
\begin{pgfscope}%
\pgfsetbuttcap%
\pgfsetroundjoin%
\pgfsetlinewidth{0.803000pt}%
\definecolor{currentstroke}{rgb}{0.690196,0.690196,0.690196}%
\pgfsetstrokecolor{currentstroke}%
\pgfsetdash{}{0pt}%
\pgfpathmoveto{\pgfqpoint{8.796805in}{2.494169in}}%
\pgfpathlineto{\pgfqpoint{7.589265in}{1.610133in}}%
\pgfpathlineto{\pgfqpoint{7.549052in}{0.676448in}}%
\pgfusepath{stroke}%
\end{pgfscope}%
\begin{pgfscope}%
\pgfsetbuttcap%
\pgfsetroundjoin%
\pgfsetlinewidth{0.803000pt}%
\definecolor{currentstroke}{rgb}{0.690196,0.690196,0.690196}%
\pgfsetstrokecolor{currentstroke}%
\pgfsetdash{}{0pt}%
\pgfpathmoveto{\pgfqpoint{9.034233in}{2.310549in}}%
\pgfpathlineto{\pgfqpoint{7.826809in}{1.446468in}}%
\pgfpathlineto{\pgfqpoint{7.801951in}{0.513468in}}%
\pgfusepath{stroke}%
\end{pgfscope}%
\begin{pgfscope}%
\pgfsetbuttcap%
\pgfsetroundjoin%
\pgfsetlinewidth{0.803000pt}%
\definecolor{currentstroke}{rgb}{0.690196,0.690196,0.690196}%
\pgfsetstrokecolor{currentstroke}%
\pgfsetdash{}{0pt}%
\pgfpathmoveto{\pgfqpoint{9.266181in}{2.131167in}}%
\pgfpathlineto{\pgfqpoint{8.059172in}{1.286373in}}%
\pgfpathlineto{\pgfqpoint{8.048992in}{0.354263in}}%
\pgfusepath{stroke}%
\end{pgfscope}%
\begin{pgfscope}%
\pgfsetbuttcap%
\pgfsetroundjoin%
\pgfsetlinewidth{0.803000pt}%
\definecolor{currentstroke}{rgb}{0.690196,0.690196,0.690196}%
\pgfsetstrokecolor{currentstroke}%
\pgfsetdash{}{0pt}%
\pgfpathmoveto{\pgfqpoint{9.422268in}{1.014283in}}%
\pgfpathlineto{\pgfqpoint{9.349172in}{1.948662in}}%
\pgfpathlineto{\pgfqpoint{8.142383in}{2.874739in}}%
\pgfusepath{stroke}%
\end{pgfscope}%
\begin{pgfscope}%
\pgfsetbuttcap%
\pgfsetroundjoin%
\pgfsetlinewidth{0.803000pt}%
\definecolor{currentstroke}{rgb}{0.690196,0.690196,0.690196}%
\pgfsetstrokecolor{currentstroke}%
\pgfsetdash{}{0pt}%
\pgfpathmoveto{\pgfqpoint{9.157011in}{0.843339in}}%
\pgfpathlineto{\pgfqpoint{9.100731in}{1.777489in}}%
\pgfpathlineto{\pgfqpoint{7.893396in}{2.682179in}}%
\pgfusepath{stroke}%
\end{pgfscope}%
\begin{pgfscope}%
\pgfsetbuttcap%
\pgfsetroundjoin%
\pgfsetlinewidth{0.803000pt}%
\definecolor{currentstroke}{rgb}{0.690196,0.690196,0.690196}%
\pgfsetstrokecolor{currentstroke}%
\pgfsetdash{}{0pt}%
\pgfpathmoveto{\pgfqpoint{8.898044in}{0.676448in}}%
\pgfpathlineto{\pgfqpoint{8.857831in}{1.610133in}}%
\pgfpathlineto{\pgfqpoint{7.650291in}{2.494169in}}%
\pgfusepath{stroke}%
\end{pgfscope}%
\begin{pgfscope}%
\pgfsetbuttcap%
\pgfsetroundjoin%
\pgfsetlinewidth{0.803000pt}%
\definecolor{currentstroke}{rgb}{0.690196,0.690196,0.690196}%
\pgfsetstrokecolor{currentstroke}%
\pgfsetdash{}{0pt}%
\pgfpathmoveto{\pgfqpoint{8.645145in}{0.513468in}}%
\pgfpathlineto{\pgfqpoint{8.620287in}{1.446468in}}%
\pgfpathlineto{\pgfqpoint{7.412863in}{2.310549in}}%
\pgfusepath{stroke}%
\end{pgfscope}%
\begin{pgfscope}%
\pgfsetbuttcap%
\pgfsetroundjoin%
\pgfsetlinewidth{0.803000pt}%
\definecolor{currentstroke}{rgb}{0.690196,0.690196,0.690196}%
\pgfsetstrokecolor{currentstroke}%
\pgfsetdash{}{0pt}%
\pgfpathmoveto{\pgfqpoint{8.398104in}{0.354263in}}%
\pgfpathlineto{\pgfqpoint{8.387924in}{1.286373in}}%
\pgfpathlineto{\pgfqpoint{7.180914in}{2.131167in}}%
\pgfusepath{stroke}%
\end{pgfscope}%
\begin{pgfscope}%
\pgfsetbuttcap%
\pgfsetroundjoin%
\pgfsetlinewidth{0.803000pt}%
\definecolor{currentstroke}{rgb}{0.690196,0.690196,0.690196}%
\pgfsetstrokecolor{currentstroke}%
\pgfsetdash{}{0pt}%
\pgfpathmoveto{\pgfqpoint{6.943664in}{1.133087in}}%
\pgfpathlineto{\pgfqpoint{8.223548in}{0.304434in}}%
\pgfpathlineto{\pgfqpoint{9.503432in}{1.133087in}}%
\pgfusepath{stroke}%
\end{pgfscope}%
\begin{pgfscope}%
\pgfsetbuttcap%
\pgfsetroundjoin%
\pgfsetlinewidth{0.803000pt}%
\definecolor{currentstroke}{rgb}{0.690196,0.690196,0.690196}%
\pgfsetstrokecolor{currentstroke}%
\pgfsetdash{}{0pt}%
\pgfpathmoveto{\pgfqpoint{6.959936in}{1.326388in}}%
\pgfpathlineto{\pgfqpoint{8.223548in}{0.496654in}}%
\pgfpathlineto{\pgfqpoint{9.487160in}{1.326388in}}%
\pgfusepath{stroke}%
\end{pgfscope}%
\begin{pgfscope}%
\pgfsetbuttcap%
\pgfsetroundjoin%
\pgfsetlinewidth{0.803000pt}%
\definecolor{currentstroke}{rgb}{0.690196,0.690196,0.690196}%
\pgfsetstrokecolor{currentstroke}%
\pgfsetdash{}{0pt}%
\pgfpathmoveto{\pgfqpoint{6.975799in}{1.514835in}}%
\pgfpathlineto{\pgfqpoint{8.223548in}{0.684319in}}%
\pgfpathlineto{\pgfqpoint{9.471296in}{1.514835in}}%
\pgfusepath{stroke}%
\end{pgfscope}%
\begin{pgfscope}%
\pgfsetbuttcap%
\pgfsetroundjoin%
\pgfsetlinewidth{0.803000pt}%
\definecolor{currentstroke}{rgb}{0.690196,0.690196,0.690196}%
\pgfsetstrokecolor{currentstroke}%
\pgfsetdash{}{0pt}%
\pgfpathmoveto{\pgfqpoint{6.991269in}{1.698610in}}%
\pgfpathlineto{\pgfqpoint{8.223548in}{0.867589in}}%
\pgfpathlineto{\pgfqpoint{9.455826in}{1.698610in}}%
\pgfusepath{stroke}%
\end{pgfscope}%
\begin{pgfscope}%
\pgfsetbuttcap%
\pgfsetroundjoin%
\pgfsetlinewidth{0.803000pt}%
\definecolor{currentstroke}{rgb}{0.690196,0.690196,0.690196}%
\pgfsetstrokecolor{currentstroke}%
\pgfsetdash{}{0pt}%
\pgfpathmoveto{\pgfqpoint{7.006360in}{1.877883in}}%
\pgfpathlineto{\pgfqpoint{8.223548in}{1.046618in}}%
\pgfpathlineto{\pgfqpoint{9.440735in}{1.877883in}}%
\pgfusepath{stroke}%
\end{pgfscope}%
\begin{pgfscope}%
\pgfsetrectcap%
\pgfsetroundjoin%
\pgfsetlinewidth{0.803000pt}%
\definecolor{currentstroke}{rgb}{0.000000,0.000000,0.000000}%
\pgfsetstrokecolor{currentstroke}%
\pgfsetdash{}{0pt}%
\pgfpathmoveto{\pgfqpoint{9.430084in}{2.004410in}}%
\pgfpathlineto{\pgfqpoint{8.223548in}{2.937509in}}%
\pgfusepath{stroke}%
\end{pgfscope}%
\begin{pgfscope}%
\pgfsetrectcap%
\pgfsetroundjoin%
\pgfsetlinewidth{0.803000pt}%
\definecolor{currentstroke}{rgb}{0.000000,0.000000,0.000000}%
\pgfsetstrokecolor{currentstroke}%
\pgfsetdash{}{0pt}%
\pgfpathmoveto{\pgfqpoint{8.294475in}{2.866882in}}%
\pgfpathlineto{\pgfqpoint{8.325219in}{2.890475in}}%
\pgfusepath{stroke}%
\end{pgfscope}%
\begin{pgfscope}%
\pgfsetrectcap%
\pgfsetroundjoin%
\pgfsetlinewidth{0.803000pt}%
\definecolor{currentstroke}{rgb}{0.000000,0.000000,0.000000}%
\pgfsetstrokecolor{currentstroke}%
\pgfsetdash{}{0pt}%
\pgfpathmoveto{\pgfqpoint{8.543464in}{2.674509in}}%
\pgfpathlineto{\pgfqpoint{8.574201in}{2.697542in}}%
\pgfusepath{stroke}%
\end{pgfscope}%
\begin{pgfscope}%
\pgfsetrectcap%
\pgfsetroundjoin%
\pgfsetlinewidth{0.803000pt}%
\definecolor{currentstroke}{rgb}{0.000000,0.000000,0.000000}%
\pgfsetstrokecolor{currentstroke}%
\pgfsetdash{}{0pt}%
\pgfpathmoveto{\pgfqpoint{8.786574in}{2.486679in}}%
\pgfpathlineto{\pgfqpoint{8.817296in}{2.509171in}}%
\pgfusepath{stroke}%
\end{pgfscope}%
\begin{pgfscope}%
\pgfsetrectcap%
\pgfsetroundjoin%
\pgfsetlinewidth{0.803000pt}%
\definecolor{currentstroke}{rgb}{0.000000,0.000000,0.000000}%
\pgfsetstrokecolor{currentstroke}%
\pgfsetdash{}{0pt}%
\pgfpathmoveto{\pgfqpoint{9.024010in}{2.303233in}}%
\pgfpathlineto{\pgfqpoint{9.054708in}{2.325202in}}%
\pgfusepath{stroke}%
\end{pgfscope}%
\begin{pgfscope}%
\pgfsetrectcap%
\pgfsetroundjoin%
\pgfsetlinewidth{0.803000pt}%
\definecolor{currentstroke}{rgb}{0.000000,0.000000,0.000000}%
\pgfsetstrokecolor{currentstroke}%
\pgfsetdash{}{0pt}%
\pgfpathmoveto{\pgfqpoint{9.255968in}{2.124019in}}%
\pgfpathlineto{\pgfqpoint{9.286636in}{2.145484in}}%
\pgfusepath{stroke}%
\end{pgfscope}%
\begin{pgfscope}%
\definecolor{textcolor}{rgb}{0.000000,0.000000,0.000000}%
\pgfsetstrokecolor{textcolor}%
\pgfsetfillcolor{textcolor}%
\pgftext[x=9.122590in,y=2.861959in,,]{\color{textcolor}{\rmfamily\fontsize{14.000000}{16.800000}\selectfont\catcode`\^=\active\def^{\ifmmode\sp\else\^{}\fi}\catcode`\%=\active\def%{\%}f1}}%
\end{pgfscope}%
\begin{pgfscope}%
\pgfsetrectcap%
\pgfsetroundjoin%
\pgfsetlinewidth{0.803000pt}%
\definecolor{currentstroke}{rgb}{0.000000,0.000000,0.000000}%
\pgfsetstrokecolor{currentstroke}%
\pgfsetdash{}{0pt}%
\pgfpathmoveto{\pgfqpoint{7.017011in}{2.004410in}}%
\pgfpathlineto{\pgfqpoint{8.223548in}{2.937509in}}%
\pgfusepath{stroke}%
\end{pgfscope}%
\begin{pgfscope}%
\pgfsetrectcap%
\pgfsetroundjoin%
\pgfsetlinewidth{0.803000pt}%
\definecolor{currentstroke}{rgb}{0.000000,0.000000,0.000000}%
\pgfsetstrokecolor{currentstroke}%
\pgfsetdash{}{0pt}%
\pgfpathmoveto{\pgfqpoint{8.152621in}{2.866882in}}%
\pgfpathlineto{\pgfqpoint{8.121876in}{2.890475in}}%
\pgfusepath{stroke}%
\end{pgfscope}%
\begin{pgfscope}%
\pgfsetrectcap%
\pgfsetroundjoin%
\pgfsetlinewidth{0.803000pt}%
\definecolor{currentstroke}{rgb}{0.000000,0.000000,0.000000}%
\pgfsetstrokecolor{currentstroke}%
\pgfsetdash{}{0pt}%
\pgfpathmoveto{\pgfqpoint{7.903631in}{2.674509in}}%
\pgfpathlineto{\pgfqpoint{7.872894in}{2.697542in}}%
\pgfusepath{stroke}%
\end{pgfscope}%
\begin{pgfscope}%
\pgfsetrectcap%
\pgfsetroundjoin%
\pgfsetlinewidth{0.803000pt}%
\definecolor{currentstroke}{rgb}{0.000000,0.000000,0.000000}%
\pgfsetstrokecolor{currentstroke}%
\pgfsetdash{}{0pt}%
\pgfpathmoveto{\pgfqpoint{7.660522in}{2.486679in}}%
\pgfpathlineto{\pgfqpoint{7.629800in}{2.509171in}}%
\pgfusepath{stroke}%
\end{pgfscope}%
\begin{pgfscope}%
\pgfsetrectcap%
\pgfsetroundjoin%
\pgfsetlinewidth{0.803000pt}%
\definecolor{currentstroke}{rgb}{0.000000,0.000000,0.000000}%
\pgfsetstrokecolor{currentstroke}%
\pgfsetdash{}{0pt}%
\pgfpathmoveto{\pgfqpoint{7.423086in}{2.303233in}}%
\pgfpathlineto{\pgfqpoint{7.392387in}{2.325202in}}%
\pgfusepath{stroke}%
\end{pgfscope}%
\begin{pgfscope}%
\pgfsetrectcap%
\pgfsetroundjoin%
\pgfsetlinewidth{0.803000pt}%
\definecolor{currentstroke}{rgb}{0.000000,0.000000,0.000000}%
\pgfsetstrokecolor{currentstroke}%
\pgfsetdash{}{0pt}%
\pgfpathmoveto{\pgfqpoint{7.191128in}{2.124019in}}%
\pgfpathlineto{\pgfqpoint{7.160460in}{2.145484in}}%
\pgfusepath{stroke}%
\end{pgfscope}%
\begin{pgfscope}%
\definecolor{textcolor}{rgb}{0.000000,0.000000,0.000000}%
\pgfsetstrokecolor{textcolor}%
\pgfsetfillcolor{textcolor}%
\pgftext[x=7.324506in,y=2.861959in,,]{\color{textcolor}{\rmfamily\fontsize{14.000000}{16.800000}\selectfont\catcode`\^=\active\def^{\ifmmode\sp\else\^{}\fi}\catcode`\%=\active\def%{\%}f2}}%
\end{pgfscope}%
\begin{pgfscope}%
\pgfsetrectcap%
\pgfsetroundjoin%
\pgfsetlinewidth{0.803000pt}%
\definecolor{currentstroke}{rgb}{0.000000,0.000000,0.000000}%
\pgfsetstrokecolor{currentstroke}%
\pgfsetdash{}{0pt}%
\pgfpathmoveto{\pgfqpoint{7.017011in}{2.004410in}}%
\pgfpathlineto{\pgfqpoint{6.938354in}{1.070011in}}%
\pgfusepath{stroke}%
\end{pgfscope}%
\begin{pgfscope}%
\pgfsetrectcap%
\pgfsetroundjoin%
\pgfsetlinewidth{0.803000pt}%
\definecolor{currentstroke}{rgb}{0.000000,0.000000,0.000000}%
\pgfsetstrokecolor{currentstroke}%
\pgfsetdash{}{0pt}%
\pgfpathmoveto{\pgfqpoint{6.954524in}{1.126056in}}%
\pgfpathlineto{\pgfqpoint{6.921911in}{1.147171in}}%
\pgfusepath{stroke}%
\end{pgfscope}%
\begin{pgfscope}%
\pgfsetrectcap%
\pgfsetroundjoin%
\pgfsetlinewidth{0.803000pt}%
\definecolor{currentstroke}{rgb}{0.000000,0.000000,0.000000}%
\pgfsetstrokecolor{currentstroke}%
\pgfsetdash{}{0pt}%
\pgfpathmoveto{\pgfqpoint{6.970651in}{1.319352in}}%
\pgfpathlineto{\pgfqpoint{6.938476in}{1.340480in}}%
\pgfusepath{stroke}%
\end{pgfscope}%
\begin{pgfscope}%
\pgfsetrectcap%
\pgfsetroundjoin%
\pgfsetlinewidth{0.803000pt}%
\definecolor{currentstroke}{rgb}{0.000000,0.000000,0.000000}%
\pgfsetstrokecolor{currentstroke}%
\pgfsetdash{}{0pt}%
\pgfpathmoveto{\pgfqpoint{6.986372in}{1.507798in}}%
\pgfpathlineto{\pgfqpoint{6.954624in}{1.528930in}}%
\pgfusepath{stroke}%
\end{pgfscope}%
\begin{pgfscope}%
\pgfsetrectcap%
\pgfsetroundjoin%
\pgfsetlinewidth{0.803000pt}%
\definecolor{currentstroke}{rgb}{0.000000,0.000000,0.000000}%
\pgfsetstrokecolor{currentstroke}%
\pgfsetdash{}{0pt}%
\pgfpathmoveto{\pgfqpoint{7.001704in}{1.691573in}}%
\pgfpathlineto{\pgfqpoint{6.970371in}{1.712703in}}%
\pgfusepath{stroke}%
\end{pgfscope}%
\begin{pgfscope}%
\pgfsetrectcap%
\pgfsetroundjoin%
\pgfsetlinewidth{0.803000pt}%
\definecolor{currentstroke}{rgb}{0.000000,0.000000,0.000000}%
\pgfsetstrokecolor{currentstroke}%
\pgfsetdash{}{0pt}%
\pgfpathmoveto{\pgfqpoint{7.016660in}{1.870849in}}%
\pgfpathlineto{\pgfqpoint{6.985733in}{1.891971in}}%
\pgfusepath{stroke}%
\end{pgfscope}%
\begin{pgfscope}%
\definecolor{textcolor}{rgb}{0.000000,0.000000,0.000000}%
\pgfsetstrokecolor{textcolor}%
\pgfsetfillcolor{textcolor}%
\pgftext[x=6.420762in,y=1.551958in,,]{\color{textcolor}{\rmfamily\fontsize{14.000000}{16.800000}\selectfont\catcode`\^=\active\def^{\ifmmode\sp\else\^{}\fi}\catcode`\%=\active\def%{\%}f3}}%
\end{pgfscope}%
\begin{pgfscope}%
\pgfpathrectangle{\pgfqpoint{6.818937in}{0.147348in}}{\pgfqpoint{2.735294in}{2.735294in}}%
\pgfusepath{clip}%
\pgfsetbuttcap%
\pgfsetroundjoin%
\definecolor{currentfill}{rgb}{0.839216,0.152941,0.156863}%
\pgfsetfillcolor{currentfill}%
\pgfsetfillopacity{0.300000}%
\pgfsetlinewidth{1.003750pt}%
\definecolor{currentstroke}{rgb}{0.839216,0.152941,0.156863}%
\pgfsetstrokecolor{currentstroke}%
\pgfsetstrokeopacity{0.300000}%
\pgfsetdash{}{0pt}%
\pgfpathmoveto{\pgfqpoint{7.927926in}{1.506774in}}%
\pgfpathcurveto{\pgfqpoint{7.938014in}{1.506774in}}{\pgfqpoint{7.947689in}{1.510782in}}{\pgfqpoint{7.954822in}{1.517914in}}%
\pgfpathcurveto{\pgfqpoint{7.961955in}{1.525047in}}{\pgfqpoint{7.965963in}{1.534723in}}{\pgfqpoint{7.965963in}{1.544810in}}%
\pgfpathcurveto{\pgfqpoint{7.965963in}{1.554897in}}{\pgfqpoint{7.961955in}{1.564573in}}{\pgfqpoint{7.954822in}{1.571706in}}%
\pgfpathcurveto{\pgfqpoint{7.947689in}{1.578839in}}{\pgfqpoint{7.938014in}{1.582846in}}{\pgfqpoint{7.927926in}{1.582846in}}%
\pgfpathcurveto{\pgfqpoint{7.917839in}{1.582846in}}{\pgfqpoint{7.908164in}{1.578839in}}{\pgfqpoint{7.901031in}{1.571706in}}%
\pgfpathcurveto{\pgfqpoint{7.893898in}{1.564573in}}{\pgfqpoint{7.889890in}{1.554897in}}{\pgfqpoint{7.889890in}{1.544810in}}%
\pgfpathcurveto{\pgfqpoint{7.889890in}{1.534723in}}{\pgfqpoint{7.893898in}{1.525047in}}{\pgfqpoint{7.901031in}{1.517914in}}%
\pgfpathcurveto{\pgfqpoint{7.908164in}{1.510782in}}{\pgfqpoint{7.917839in}{1.506774in}}{\pgfqpoint{7.927926in}{1.506774in}}%
\pgfpathlineto{\pgfqpoint{7.927926in}{1.506774in}}%
\pgfpathclose%
\pgfusepath{stroke,fill}%
\end{pgfscope}%
\begin{pgfscope}%
\pgfpathrectangle{\pgfqpoint{6.818937in}{0.147348in}}{\pgfqpoint{2.735294in}{2.735294in}}%
\pgfusepath{clip}%
\pgfsetbuttcap%
\pgfsetroundjoin%
\definecolor{currentfill}{rgb}{0.839216,0.152941,0.156863}%
\pgfsetfillcolor{currentfill}%
\pgfsetfillopacity{0.368519}%
\pgfsetlinewidth{1.003750pt}%
\definecolor{currentstroke}{rgb}{0.839216,0.152941,0.156863}%
\pgfsetstrokecolor{currentstroke}%
\pgfsetstrokeopacity{0.368519}%
\pgfsetdash{}{0pt}%
\pgfpathmoveto{\pgfqpoint{7.807338in}{1.289993in}}%
\pgfpathcurveto{\pgfqpoint{7.817425in}{1.289993in}}{\pgfqpoint{7.827101in}{1.294001in}}{\pgfqpoint{7.834233in}{1.301134in}}%
\pgfpathcurveto{\pgfqpoint{7.841366in}{1.308267in}}{\pgfqpoint{7.845374in}{1.317942in}}{\pgfqpoint{7.845374in}{1.328030in}}%
\pgfpathcurveto{\pgfqpoint{7.845374in}{1.338117in}}{\pgfqpoint{7.841366in}{1.347793in}}{\pgfqpoint{7.834233in}{1.354925in}}%
\pgfpathcurveto{\pgfqpoint{7.827101in}{1.362058in}}{\pgfqpoint{7.817425in}{1.366066in}}{\pgfqpoint{7.807338in}{1.366066in}}%
\pgfpathcurveto{\pgfqpoint{7.797250in}{1.366066in}}{\pgfqpoint{7.787575in}{1.362058in}}{\pgfqpoint{7.780442in}{1.354925in}}%
\pgfpathcurveto{\pgfqpoint{7.773309in}{1.347793in}}{\pgfqpoint{7.769301in}{1.338117in}}{\pgfqpoint{7.769301in}{1.328030in}}%
\pgfpathcurveto{\pgfqpoint{7.769301in}{1.317942in}}{\pgfqpoint{7.773309in}{1.308267in}}{\pgfqpoint{7.780442in}{1.301134in}}%
\pgfpathcurveto{\pgfqpoint{7.787575in}{1.294001in}}{\pgfqpoint{7.797250in}{1.289993in}}{\pgfqpoint{7.807338in}{1.289993in}}%
\pgfpathlineto{\pgfqpoint{7.807338in}{1.289993in}}%
\pgfpathclose%
\pgfusepath{stroke,fill}%
\end{pgfscope}%
\begin{pgfscope}%
\pgfpathrectangle{\pgfqpoint{6.818937in}{0.147348in}}{\pgfqpoint{2.735294in}{2.735294in}}%
\pgfusepath{clip}%
\pgfsetbuttcap%
\pgfsetroundjoin%
\definecolor{currentfill}{rgb}{0.839216,0.152941,0.156863}%
\pgfsetfillcolor{currentfill}%
\pgfsetfillopacity{0.431635}%
\pgfsetlinewidth{1.003750pt}%
\definecolor{currentstroke}{rgb}{0.839216,0.152941,0.156863}%
\pgfsetstrokecolor{currentstroke}%
\pgfsetstrokeopacity{0.431635}%
\pgfsetdash{}{0pt}%
\pgfpathmoveto{\pgfqpoint{7.865218in}{1.926337in}}%
\pgfpathcurveto{\pgfqpoint{7.875306in}{1.926337in}}{\pgfqpoint{7.884981in}{1.930345in}}{\pgfqpoint{7.892114in}{1.937478in}}%
\pgfpathcurveto{\pgfqpoint{7.899247in}{1.944610in}}{\pgfqpoint{7.903255in}{1.954286in}}{\pgfqpoint{7.903255in}{1.964373in}}%
\pgfpathcurveto{\pgfqpoint{7.903255in}{1.974461in}}{\pgfqpoint{7.899247in}{1.984136in}}{\pgfqpoint{7.892114in}{1.991269in}}%
\pgfpathcurveto{\pgfqpoint{7.884981in}{1.998402in}}{\pgfqpoint{7.875306in}{2.002410in}}{\pgfqpoint{7.865218in}{2.002410in}}%
\pgfpathcurveto{\pgfqpoint{7.855131in}{2.002410in}}{\pgfqpoint{7.845455in}{1.998402in}}{\pgfqpoint{7.838323in}{1.991269in}}%
\pgfpathcurveto{\pgfqpoint{7.831190in}{1.984136in}}{\pgfqpoint{7.827182in}{1.974461in}}{\pgfqpoint{7.827182in}{1.964373in}}%
\pgfpathcurveto{\pgfqpoint{7.827182in}{1.954286in}}{\pgfqpoint{7.831190in}{1.944610in}}{\pgfqpoint{7.838323in}{1.937478in}}%
\pgfpathcurveto{\pgfqpoint{7.845455in}{1.930345in}}{\pgfqpoint{7.855131in}{1.926337in}}{\pgfqpoint{7.865218in}{1.926337in}}%
\pgfpathlineto{\pgfqpoint{7.865218in}{1.926337in}}%
\pgfpathclose%
\pgfusepath{stroke,fill}%
\end{pgfscope}%
\begin{pgfscope}%
\pgfpathrectangle{\pgfqpoint{6.818937in}{0.147348in}}{\pgfqpoint{2.735294in}{2.735294in}}%
\pgfusepath{clip}%
\pgfsetbuttcap%
\pgfsetroundjoin%
\definecolor{currentfill}{rgb}{0.839216,0.152941,0.156863}%
\pgfsetfillcolor{currentfill}%
\pgfsetfillopacity{0.502644}%
\pgfsetlinewidth{1.003750pt}%
\definecolor{currentstroke}{rgb}{0.839216,0.152941,0.156863}%
\pgfsetstrokecolor{currentstroke}%
\pgfsetstrokeopacity{0.502644}%
\pgfsetdash{}{0pt}%
\pgfpathmoveto{\pgfqpoint{8.531889in}{2.015496in}}%
\pgfpathcurveto{\pgfqpoint{8.541976in}{2.015496in}}{\pgfqpoint{8.551652in}{2.019504in}}{\pgfqpoint{8.558785in}{2.026637in}}%
\pgfpathcurveto{\pgfqpoint{8.565918in}{2.033769in}}{\pgfqpoint{8.569925in}{2.043445in}}{\pgfqpoint{8.569925in}{2.053532in}}%
\pgfpathcurveto{\pgfqpoint{8.569925in}{2.063620in}}{\pgfqpoint{8.565918in}{2.073295in}}{\pgfqpoint{8.558785in}{2.080428in}}%
\pgfpathcurveto{\pgfqpoint{8.551652in}{2.087561in}}{\pgfqpoint{8.541976in}{2.091569in}}{\pgfqpoint{8.531889in}{2.091569in}}%
\pgfpathcurveto{\pgfqpoint{8.521802in}{2.091569in}}{\pgfqpoint{8.512126in}{2.087561in}}{\pgfqpoint{8.504993in}{2.080428in}}%
\pgfpathcurveto{\pgfqpoint{8.497860in}{2.073295in}}{\pgfqpoint{8.493853in}{2.063620in}}{\pgfqpoint{8.493853in}{2.053532in}}%
\pgfpathcurveto{\pgfqpoint{8.493853in}{2.043445in}}{\pgfqpoint{8.497860in}{2.033769in}}{\pgfqpoint{8.504993in}{2.026637in}}%
\pgfpathcurveto{\pgfqpoint{8.512126in}{2.019504in}}{\pgfqpoint{8.521802in}{2.015496in}}{\pgfqpoint{8.531889in}{2.015496in}}%
\pgfpathlineto{\pgfqpoint{8.531889in}{2.015496in}}%
\pgfpathclose%
\pgfusepath{stroke,fill}%
\end{pgfscope}%
\begin{pgfscope}%
\pgfpathrectangle{\pgfqpoint{6.818937in}{0.147348in}}{\pgfqpoint{2.735294in}{2.735294in}}%
\pgfusepath{clip}%
\pgfsetbuttcap%
\pgfsetroundjoin%
\definecolor{currentfill}{rgb}{0.839216,0.152941,0.156863}%
\pgfsetfillcolor{currentfill}%
\pgfsetfillopacity{0.555030}%
\pgfsetlinewidth{1.003750pt}%
\definecolor{currentstroke}{rgb}{0.839216,0.152941,0.156863}%
\pgfsetstrokecolor{currentstroke}%
\pgfsetstrokeopacity{0.555030}%
\pgfsetdash{}{0pt}%
\pgfpathmoveto{\pgfqpoint{9.028634in}{1.507681in}}%
\pgfpathcurveto{\pgfqpoint{9.038721in}{1.507681in}}{\pgfqpoint{9.048397in}{1.511689in}}{\pgfqpoint{9.055530in}{1.518822in}}%
\pgfpathcurveto{\pgfqpoint{9.062663in}{1.525955in}}{\pgfqpoint{9.066670in}{1.535630in}}{\pgfqpoint{9.066670in}{1.545718in}}%
\pgfpathcurveto{\pgfqpoint{9.066670in}{1.555805in}}{\pgfqpoint{9.062663in}{1.565481in}}{\pgfqpoint{9.055530in}{1.572613in}}%
\pgfpathcurveto{\pgfqpoint{9.048397in}{1.579746in}}{\pgfqpoint{9.038721in}{1.583754in}}{\pgfqpoint{9.028634in}{1.583754in}}%
\pgfpathcurveto{\pgfqpoint{9.018547in}{1.583754in}}{\pgfqpoint{9.008871in}{1.579746in}}{\pgfqpoint{9.001738in}{1.572613in}}%
\pgfpathcurveto{\pgfqpoint{8.994605in}{1.565481in}}{\pgfqpoint{8.990598in}{1.555805in}}{\pgfqpoint{8.990598in}{1.545718in}}%
\pgfpathcurveto{\pgfqpoint{8.990598in}{1.535630in}}{\pgfqpoint{8.994605in}{1.525955in}}{\pgfqpoint{9.001738in}{1.518822in}}%
\pgfpathcurveto{\pgfqpoint{9.008871in}{1.511689in}}{\pgfqpoint{9.018547in}{1.507681in}}{\pgfqpoint{9.028634in}{1.507681in}}%
\pgfpathlineto{\pgfqpoint{9.028634in}{1.507681in}}%
\pgfpathclose%
\pgfusepath{stroke,fill}%
\end{pgfscope}%
\begin{pgfscope}%
\pgfpathrectangle{\pgfqpoint{6.818937in}{0.147348in}}{\pgfqpoint{2.735294in}{2.735294in}}%
\pgfusepath{clip}%
\pgfsetbuttcap%
\pgfsetroundjoin%
\definecolor{currentfill}{rgb}{0.839216,0.152941,0.156863}%
\pgfsetfillcolor{currentfill}%
\pgfsetfillopacity{0.641254}%
\pgfsetlinewidth{1.003750pt}%
\definecolor{currentstroke}{rgb}{0.839216,0.152941,0.156863}%
\pgfsetstrokecolor{currentstroke}%
\pgfsetstrokeopacity{0.641254}%
\pgfsetdash{}{0pt}%
\pgfpathmoveto{\pgfqpoint{8.416294in}{0.946553in}}%
\pgfpathcurveto{\pgfqpoint{8.426381in}{0.946553in}}{\pgfqpoint{8.436057in}{0.950561in}}{\pgfqpoint{8.443190in}{0.957694in}}%
\pgfpathcurveto{\pgfqpoint{8.450322in}{0.964827in}}{\pgfqpoint{8.454330in}{0.974502in}}{\pgfqpoint{8.454330in}{0.984590in}}%
\pgfpathcurveto{\pgfqpoint{8.454330in}{0.994677in}}{\pgfqpoint{8.450322in}{1.004352in}}{\pgfqpoint{8.443190in}{1.011485in}}%
\pgfpathcurveto{\pgfqpoint{8.436057in}{1.018618in}}{\pgfqpoint{8.426381in}{1.022626in}}{\pgfqpoint{8.416294in}{1.022626in}}%
\pgfpathcurveto{\pgfqpoint{8.406207in}{1.022626in}}{\pgfqpoint{8.396531in}{1.018618in}}{\pgfqpoint{8.389398in}{1.011485in}}%
\pgfpathcurveto{\pgfqpoint{8.382265in}{1.004352in}}{\pgfqpoint{8.378258in}{0.994677in}}{\pgfqpoint{8.378258in}{0.984590in}}%
\pgfpathcurveto{\pgfqpoint{8.378258in}{0.974502in}}{\pgfqpoint{8.382265in}{0.964827in}}{\pgfqpoint{8.389398in}{0.957694in}}%
\pgfpathcurveto{\pgfqpoint{8.396531in}{0.950561in}}{\pgfqpoint{8.406207in}{0.946553in}}{\pgfqpoint{8.416294in}{0.946553in}}%
\pgfpathlineto{\pgfqpoint{8.416294in}{0.946553in}}%
\pgfpathclose%
\pgfusepath{stroke,fill}%
\end{pgfscope}%
\begin{pgfscope}%
\pgfpathrectangle{\pgfqpoint{6.818937in}{0.147348in}}{\pgfqpoint{2.735294in}{2.735294in}}%
\pgfusepath{clip}%
\pgfsetbuttcap%
\pgfsetroundjoin%
\definecolor{currentfill}{rgb}{0.839216,0.152941,0.156863}%
\pgfsetfillcolor{currentfill}%
\pgfsetfillopacity{0.659293}%
\pgfsetlinewidth{1.003750pt}%
\definecolor{currentstroke}{rgb}{0.839216,0.152941,0.156863}%
\pgfsetstrokecolor{currentstroke}%
\pgfsetstrokeopacity{0.659293}%
\pgfsetdash{}{0pt}%
\pgfpathmoveto{\pgfqpoint{7.519784in}{1.533581in}}%
\pgfpathcurveto{\pgfqpoint{7.529872in}{1.533581in}}{\pgfqpoint{7.539547in}{1.537589in}}{\pgfqpoint{7.546680in}{1.544722in}}%
\pgfpathcurveto{\pgfqpoint{7.553813in}{1.551854in}}{\pgfqpoint{7.557821in}{1.561530in}}{\pgfqpoint{7.557821in}{1.571617in}}%
\pgfpathcurveto{\pgfqpoint{7.557821in}{1.581705in}}{\pgfqpoint{7.553813in}{1.591380in}}{\pgfqpoint{7.546680in}{1.598513in}}%
\pgfpathcurveto{\pgfqpoint{7.539547in}{1.605646in}}{\pgfqpoint{7.529872in}{1.609654in}}{\pgfqpoint{7.519784in}{1.609654in}}%
\pgfpathcurveto{\pgfqpoint{7.509697in}{1.609654in}}{\pgfqpoint{7.500022in}{1.605646in}}{\pgfqpoint{7.492889in}{1.598513in}}%
\pgfpathcurveto{\pgfqpoint{7.485756in}{1.591380in}}{\pgfqpoint{7.481748in}{1.581705in}}{\pgfqpoint{7.481748in}{1.571617in}}%
\pgfpathcurveto{\pgfqpoint{7.481748in}{1.561530in}}{\pgfqpoint{7.485756in}{1.551854in}}{\pgfqpoint{7.492889in}{1.544722in}}%
\pgfpathcurveto{\pgfqpoint{7.500022in}{1.537589in}}{\pgfqpoint{7.509697in}{1.533581in}}{\pgfqpoint{7.519784in}{1.533581in}}%
\pgfpathlineto{\pgfqpoint{7.519784in}{1.533581in}}%
\pgfpathclose%
\pgfusepath{stroke,fill}%
\end{pgfscope}%
\begin{pgfscope}%
\pgfpathrectangle{\pgfqpoint{6.818937in}{0.147348in}}{\pgfqpoint{2.735294in}{2.735294in}}%
\pgfusepath{clip}%
\pgfsetbuttcap%
\pgfsetroundjoin%
\definecolor{currentfill}{rgb}{0.839216,0.152941,0.156863}%
\pgfsetfillcolor{currentfill}%
\pgfsetfillopacity{0.661693}%
\pgfsetlinewidth{1.003750pt}%
\definecolor{currentstroke}{rgb}{0.839216,0.152941,0.156863}%
\pgfsetstrokecolor{currentstroke}%
\pgfsetstrokeopacity{0.661693}%
\pgfsetdash{}{0pt}%
\pgfpathmoveto{\pgfqpoint{7.793775in}{2.111490in}}%
\pgfpathcurveto{\pgfqpoint{7.803863in}{2.111490in}}{\pgfqpoint{7.813538in}{2.115498in}}{\pgfqpoint{7.820671in}{2.122631in}}%
\pgfpathcurveto{\pgfqpoint{7.827804in}{2.129763in}}{\pgfqpoint{7.831812in}{2.139439in}}{\pgfqpoint{7.831812in}{2.149526in}}%
\pgfpathcurveto{\pgfqpoint{7.831812in}{2.159614in}}{\pgfqpoint{7.827804in}{2.169289in}}{\pgfqpoint{7.820671in}{2.176422in}}%
\pgfpathcurveto{\pgfqpoint{7.813538in}{2.183555in}}{\pgfqpoint{7.803863in}{2.187563in}}{\pgfqpoint{7.793775in}{2.187563in}}%
\pgfpathcurveto{\pgfqpoint{7.783688in}{2.187563in}}{\pgfqpoint{7.774013in}{2.183555in}}{\pgfqpoint{7.766880in}{2.176422in}}%
\pgfpathcurveto{\pgfqpoint{7.759747in}{2.169289in}}{\pgfqpoint{7.755739in}{2.159614in}}{\pgfqpoint{7.755739in}{2.149526in}}%
\pgfpathcurveto{\pgfqpoint{7.755739in}{2.139439in}}{\pgfqpoint{7.759747in}{2.129763in}}{\pgfqpoint{7.766880in}{2.122631in}}%
\pgfpathcurveto{\pgfqpoint{7.774013in}{2.115498in}}{\pgfqpoint{7.783688in}{2.111490in}}{\pgfqpoint{7.793775in}{2.111490in}}%
\pgfpathlineto{\pgfqpoint{7.793775in}{2.111490in}}%
\pgfpathclose%
\pgfusepath{stroke,fill}%
\end{pgfscope}%
\begin{pgfscope}%
\pgfpathrectangle{\pgfqpoint{6.818937in}{0.147348in}}{\pgfqpoint{2.735294in}{2.735294in}}%
\pgfusepath{clip}%
\pgfsetbuttcap%
\pgfsetroundjoin%
\definecolor{currentfill}{rgb}{0.839216,0.152941,0.156863}%
\pgfsetfillcolor{currentfill}%
\pgfsetfillopacity{0.667667}%
\pgfsetlinewidth{1.003750pt}%
\definecolor{currentstroke}{rgb}{0.839216,0.152941,0.156863}%
\pgfsetstrokecolor{currentstroke}%
\pgfsetstrokeopacity{0.667667}%
\pgfsetdash{}{0pt}%
\pgfpathmoveto{\pgfqpoint{7.784633in}{2.171864in}}%
\pgfpathcurveto{\pgfqpoint{7.794720in}{2.171864in}}{\pgfqpoint{7.804396in}{2.175872in}}{\pgfqpoint{7.811529in}{2.183005in}}%
\pgfpathcurveto{\pgfqpoint{7.818661in}{2.190137in}}{\pgfqpoint{7.822669in}{2.199813in}}{\pgfqpoint{7.822669in}{2.209900in}}%
\pgfpathcurveto{\pgfqpoint{7.822669in}{2.219988in}}{\pgfqpoint{7.818661in}{2.229663in}}{\pgfqpoint{7.811529in}{2.236796in}}%
\pgfpathcurveto{\pgfqpoint{7.804396in}{2.243929in}}{\pgfqpoint{7.794720in}{2.247937in}}{\pgfqpoint{7.784633in}{2.247937in}}%
\pgfpathcurveto{\pgfqpoint{7.774546in}{2.247937in}}{\pgfqpoint{7.764870in}{2.243929in}}{\pgfqpoint{7.757737in}{2.236796in}}%
\pgfpathcurveto{\pgfqpoint{7.750604in}{2.229663in}}{\pgfqpoint{7.746597in}{2.219988in}}{\pgfqpoint{7.746597in}{2.209900in}}%
\pgfpathcurveto{\pgfqpoint{7.746597in}{2.199813in}}{\pgfqpoint{7.750604in}{2.190137in}}{\pgfqpoint{7.757737in}{2.183005in}}%
\pgfpathcurveto{\pgfqpoint{7.764870in}{2.175872in}}{\pgfqpoint{7.774546in}{2.171864in}}{\pgfqpoint{7.784633in}{2.171864in}}%
\pgfpathlineto{\pgfqpoint{7.784633in}{2.171864in}}%
\pgfpathclose%
\pgfusepath{stroke,fill}%
\end{pgfscope}%
\begin{pgfscope}%
\pgfpathrectangle{\pgfqpoint{6.818937in}{0.147348in}}{\pgfqpoint{2.735294in}{2.735294in}}%
\pgfusepath{clip}%
\pgfsetbuttcap%
\pgfsetroundjoin%
\definecolor{currentfill}{rgb}{0.839216,0.152941,0.156863}%
\pgfsetfillcolor{currentfill}%
\pgfsetfillopacity{0.680503}%
\pgfsetlinewidth{1.003750pt}%
\definecolor{currentstroke}{rgb}{0.839216,0.152941,0.156863}%
\pgfsetstrokecolor{currentstroke}%
\pgfsetstrokeopacity{0.680503}%
\pgfsetdash{}{0pt}%
\pgfpathmoveto{\pgfqpoint{8.674153in}{0.972903in}}%
\pgfpathcurveto{\pgfqpoint{8.684241in}{0.972903in}}{\pgfqpoint{8.693916in}{0.976910in}}{\pgfqpoint{8.701049in}{0.984043in}}%
\pgfpathcurveto{\pgfqpoint{8.708182in}{0.991176in}}{\pgfqpoint{8.712190in}{1.000852in}}{\pgfqpoint{8.712190in}{1.010939in}}%
\pgfpathcurveto{\pgfqpoint{8.712190in}{1.021026in}}{\pgfqpoint{8.708182in}{1.030702in}}{\pgfqpoint{8.701049in}{1.037835in}}%
\pgfpathcurveto{\pgfqpoint{8.693916in}{1.044968in}}{\pgfqpoint{8.684241in}{1.048975in}}{\pgfqpoint{8.674153in}{1.048975in}}%
\pgfpathcurveto{\pgfqpoint{8.664066in}{1.048975in}}{\pgfqpoint{8.654390in}{1.044968in}}{\pgfqpoint{8.647258in}{1.037835in}}%
\pgfpathcurveto{\pgfqpoint{8.640125in}{1.030702in}}{\pgfqpoint{8.636117in}{1.021026in}}{\pgfqpoint{8.636117in}{1.010939in}}%
\pgfpathcurveto{\pgfqpoint{8.636117in}{1.000852in}}{\pgfqpoint{8.640125in}{0.991176in}}{\pgfqpoint{8.647258in}{0.984043in}}%
\pgfpathcurveto{\pgfqpoint{8.654390in}{0.976910in}}{\pgfqpoint{8.664066in}{0.972903in}}{\pgfqpoint{8.674153in}{0.972903in}}%
\pgfpathlineto{\pgfqpoint{8.674153in}{0.972903in}}%
\pgfpathclose%
\pgfusepath{stroke,fill}%
\end{pgfscope}%
\begin{pgfscope}%
\pgfpathrectangle{\pgfqpoint{6.818937in}{0.147348in}}{\pgfqpoint{2.735294in}{2.735294in}}%
\pgfusepath{clip}%
\pgfsetbuttcap%
\pgfsetroundjoin%
\definecolor{currentfill}{rgb}{0.839216,0.152941,0.156863}%
\pgfsetfillcolor{currentfill}%
\pgfsetfillopacity{0.795933}%
\pgfsetlinewidth{1.003750pt}%
\definecolor{currentstroke}{rgb}{0.839216,0.152941,0.156863}%
\pgfsetstrokecolor{currentstroke}%
\pgfsetstrokeopacity{0.795933}%
\pgfsetdash{}{0pt}%
\pgfpathmoveto{\pgfqpoint{7.340626in}{1.623661in}}%
\pgfpathcurveto{\pgfqpoint{7.350714in}{1.623661in}}{\pgfqpoint{7.360389in}{1.627668in}}{\pgfqpoint{7.367522in}{1.634801in}}%
\pgfpathcurveto{\pgfqpoint{7.374655in}{1.641934in}}{\pgfqpoint{7.378663in}{1.651609in}}{\pgfqpoint{7.378663in}{1.661697in}}%
\pgfpathcurveto{\pgfqpoint{7.378663in}{1.671784in}}{\pgfqpoint{7.374655in}{1.681460in}}{\pgfqpoint{7.367522in}{1.688593in}}%
\pgfpathcurveto{\pgfqpoint{7.360389in}{1.695725in}}{\pgfqpoint{7.350714in}{1.699733in}}{\pgfqpoint{7.340626in}{1.699733in}}%
\pgfpathcurveto{\pgfqpoint{7.330539in}{1.699733in}}{\pgfqpoint{7.320863in}{1.695725in}}{\pgfqpoint{7.313731in}{1.688593in}}%
\pgfpathcurveto{\pgfqpoint{7.306598in}{1.681460in}}{\pgfqpoint{7.302590in}{1.671784in}}{\pgfqpoint{7.302590in}{1.661697in}}%
\pgfpathcurveto{\pgfqpoint{7.302590in}{1.651609in}}{\pgfqpoint{7.306598in}{1.641934in}}{\pgfqpoint{7.313731in}{1.634801in}}%
\pgfpathcurveto{\pgfqpoint{7.320863in}{1.627668in}}{\pgfqpoint{7.330539in}{1.623661in}}{\pgfqpoint{7.340626in}{1.623661in}}%
\pgfpathlineto{\pgfqpoint{7.340626in}{1.623661in}}%
\pgfpathclose%
\pgfusepath{stroke,fill}%
\end{pgfscope}%
\begin{pgfscope}%
\pgfpathrectangle{\pgfqpoint{6.818937in}{0.147348in}}{\pgfqpoint{2.735294in}{2.735294in}}%
\pgfusepath{clip}%
\pgfsetbuttcap%
\pgfsetroundjoin%
\definecolor{currentfill}{rgb}{0.839216,0.152941,0.156863}%
\pgfsetfillcolor{currentfill}%
\pgfsetfillopacity{0.802324}%
\pgfsetlinewidth{1.003750pt}%
\definecolor{currentstroke}{rgb}{0.839216,0.152941,0.156863}%
\pgfsetstrokecolor{currentstroke}%
\pgfsetstrokeopacity{0.802324}%
\pgfsetdash{}{0pt}%
\pgfpathmoveto{\pgfqpoint{9.019142in}{1.511610in}}%
\pgfpathcurveto{\pgfqpoint{9.029229in}{1.511610in}}{\pgfqpoint{9.038905in}{1.515617in}}{\pgfqpoint{9.046038in}{1.522750in}}%
\pgfpathcurveto{\pgfqpoint{9.053171in}{1.529883in}}{\pgfqpoint{9.057178in}{1.539559in}}{\pgfqpoint{9.057178in}{1.549646in}}%
\pgfpathcurveto{\pgfqpoint{9.057178in}{1.559733in}}{\pgfqpoint{9.053171in}{1.569409in}}{\pgfqpoint{9.046038in}{1.576542in}}%
\pgfpathcurveto{\pgfqpoint{9.038905in}{1.583675in}}{\pgfqpoint{9.029229in}{1.587682in}}{\pgfqpoint{9.019142in}{1.587682in}}%
\pgfpathcurveto{\pgfqpoint{9.009055in}{1.587682in}}{\pgfqpoint{8.999379in}{1.583675in}}{\pgfqpoint{8.992246in}{1.576542in}}%
\pgfpathcurveto{\pgfqpoint{8.985114in}{1.569409in}}{\pgfqpoint{8.981106in}{1.559733in}}{\pgfqpoint{8.981106in}{1.549646in}}%
\pgfpathcurveto{\pgfqpoint{8.981106in}{1.539559in}}{\pgfqpoint{8.985114in}{1.529883in}}{\pgfqpoint{8.992246in}{1.522750in}}%
\pgfpathcurveto{\pgfqpoint{8.999379in}{1.515617in}}{\pgfqpoint{9.009055in}{1.511610in}}{\pgfqpoint{9.019142in}{1.511610in}}%
\pgfpathlineto{\pgfqpoint{9.019142in}{1.511610in}}%
\pgfpathclose%
\pgfusepath{stroke,fill}%
\end{pgfscope}%
\begin{pgfscope}%
\pgfpathrectangle{\pgfqpoint{6.818937in}{0.147348in}}{\pgfqpoint{2.735294in}{2.735294in}}%
\pgfusepath{clip}%
\pgfsetbuttcap%
\pgfsetroundjoin%
\definecolor{currentfill}{rgb}{0.839216,0.152941,0.156863}%
\pgfsetfillcolor{currentfill}%
\pgfsetfillopacity{0.837755}%
\pgfsetlinewidth{1.003750pt}%
\definecolor{currentstroke}{rgb}{0.839216,0.152941,0.156863}%
\pgfsetstrokecolor{currentstroke}%
\pgfsetstrokeopacity{0.837755}%
\pgfsetdash{}{0pt}%
\pgfpathmoveto{\pgfqpoint{8.690520in}{0.950898in}}%
\pgfpathcurveto{\pgfqpoint{8.700607in}{0.950898in}}{\pgfqpoint{8.710282in}{0.954905in}}{\pgfqpoint{8.717415in}{0.962038in}}%
\pgfpathcurveto{\pgfqpoint{8.724548in}{0.969171in}}{\pgfqpoint{8.728556in}{0.978846in}}{\pgfqpoint{8.728556in}{0.988934in}}%
\pgfpathcurveto{\pgfqpoint{8.728556in}{0.999021in}}{\pgfqpoint{8.724548in}{1.008697in}}{\pgfqpoint{8.717415in}{1.015830in}}%
\pgfpathcurveto{\pgfqpoint{8.710282in}{1.022962in}}{\pgfqpoint{8.700607in}{1.026970in}}{\pgfqpoint{8.690520in}{1.026970in}}%
\pgfpathcurveto{\pgfqpoint{8.680432in}{1.026970in}}{\pgfqpoint{8.670757in}{1.022962in}}{\pgfqpoint{8.663624in}{1.015830in}}%
\pgfpathcurveto{\pgfqpoint{8.656491in}{1.008697in}}{\pgfqpoint{8.652483in}{0.999021in}}{\pgfqpoint{8.652483in}{0.988934in}}%
\pgfpathcurveto{\pgfqpoint{8.652483in}{0.978846in}}{\pgfqpoint{8.656491in}{0.969171in}}{\pgfqpoint{8.663624in}{0.962038in}}%
\pgfpathcurveto{\pgfqpoint{8.670757in}{0.954905in}}{\pgfqpoint{8.680432in}{0.950898in}}{\pgfqpoint{8.690520in}{0.950898in}}%
\pgfpathlineto{\pgfqpoint{8.690520in}{0.950898in}}%
\pgfpathclose%
\pgfusepath{stroke,fill}%
\end{pgfscope}%
\begin{pgfscope}%
\pgfpathrectangle{\pgfqpoint{6.818937in}{0.147348in}}{\pgfqpoint{2.735294in}{2.735294in}}%
\pgfusepath{clip}%
\pgfsetbuttcap%
\pgfsetroundjoin%
\definecolor{currentfill}{rgb}{0.839216,0.152941,0.156863}%
\pgfsetfillcolor{currentfill}%
\pgfsetfillopacity{0.913537}%
\pgfsetlinewidth{1.003750pt}%
\definecolor{currentstroke}{rgb}{0.839216,0.152941,0.156863}%
\pgfsetstrokecolor{currentstroke}%
\pgfsetstrokeopacity{0.913537}%
\pgfsetdash{}{0pt}%
\pgfpathmoveto{\pgfqpoint{8.852391in}{0.996845in}}%
\pgfpathcurveto{\pgfqpoint{8.862478in}{0.996845in}}{\pgfqpoint{8.872154in}{1.000853in}}{\pgfqpoint{8.879286in}{1.007986in}}%
\pgfpathcurveto{\pgfqpoint{8.886419in}{1.015119in}}{\pgfqpoint{8.890427in}{1.024794in}}{\pgfqpoint{8.890427in}{1.034882in}}%
\pgfpathcurveto{\pgfqpoint{8.890427in}{1.044969in}}{\pgfqpoint{8.886419in}{1.054645in}}{\pgfqpoint{8.879286in}{1.061777in}}%
\pgfpathcurveto{\pgfqpoint{8.872154in}{1.068910in}}{\pgfqpoint{8.862478in}{1.072918in}}{\pgfqpoint{8.852391in}{1.072918in}}%
\pgfpathcurveto{\pgfqpoint{8.842303in}{1.072918in}}{\pgfqpoint{8.832628in}{1.068910in}}{\pgfqpoint{8.825495in}{1.061777in}}%
\pgfpathcurveto{\pgfqpoint{8.818362in}{1.054645in}}{\pgfqpoint{8.814354in}{1.044969in}}{\pgfqpoint{8.814354in}{1.034882in}}%
\pgfpathcurveto{\pgfqpoint{8.814354in}{1.024794in}}{\pgfqpoint{8.818362in}{1.015119in}}{\pgfqpoint{8.825495in}{1.007986in}}%
\pgfpathcurveto{\pgfqpoint{8.832628in}{1.000853in}}{\pgfqpoint{8.842303in}{0.996845in}}{\pgfqpoint{8.852391in}{0.996845in}}%
\pgfpathlineto{\pgfqpoint{8.852391in}{0.996845in}}%
\pgfpathclose%
\pgfusepath{stroke,fill}%
\end{pgfscope}%
\begin{pgfscope}%
\pgfpathrectangle{\pgfqpoint{6.818937in}{0.147348in}}{\pgfqpoint{2.735294in}{2.735294in}}%
\pgfusepath{clip}%
\pgfsetbuttcap%
\pgfsetroundjoin%
\definecolor{currentfill}{rgb}{0.839216,0.152941,0.156863}%
\pgfsetfillcolor{currentfill}%
\pgfsetlinewidth{1.003750pt}%
\definecolor{currentstroke}{rgb}{0.839216,0.152941,0.156863}%
\pgfsetstrokecolor{currentstroke}%
\pgfsetdash{}{0pt}%
\pgfpathmoveto{\pgfqpoint{8.834855in}{0.927831in}}%
\pgfpathcurveto{\pgfqpoint{8.844942in}{0.927831in}}{\pgfqpoint{8.854618in}{0.931838in}}{\pgfqpoint{8.861751in}{0.938971in}}%
\pgfpathcurveto{\pgfqpoint{8.868883in}{0.946104in}}{\pgfqpoint{8.872891in}{0.955779in}}{\pgfqpoint{8.872891in}{0.965867in}}%
\pgfpathcurveto{\pgfqpoint{8.872891in}{0.975954in}}{\pgfqpoint{8.868883in}{0.985630in}}{\pgfqpoint{8.861751in}{0.992763in}}%
\pgfpathcurveto{\pgfqpoint{8.854618in}{0.999895in}}{\pgfqpoint{8.844942in}{1.003903in}}{\pgfqpoint{8.834855in}{1.003903in}}%
\pgfpathcurveto{\pgfqpoint{8.824768in}{1.003903in}}{\pgfqpoint{8.815092in}{0.999895in}}{\pgfqpoint{8.807959in}{0.992763in}}%
\pgfpathcurveto{\pgfqpoint{8.800826in}{0.985630in}}{\pgfqpoint{8.796819in}{0.975954in}}{\pgfqpoint{8.796819in}{0.965867in}}%
\pgfpathcurveto{\pgfqpoint{8.796819in}{0.955779in}}{\pgfqpoint{8.800826in}{0.946104in}}{\pgfqpoint{8.807959in}{0.938971in}}%
\pgfpathcurveto{\pgfqpoint{8.815092in}{0.931838in}}{\pgfqpoint{8.824768in}{0.927831in}}{\pgfqpoint{8.834855in}{0.927831in}}%
\pgfpathlineto{\pgfqpoint{8.834855in}{0.927831in}}%
\pgfpathclose%
\pgfusepath{stroke,fill}%
\end{pgfscope}%
\begin{pgfscope}%
\pgfpathrectangle{\pgfqpoint{6.818937in}{0.147348in}}{\pgfqpoint{2.735294in}{2.735294in}}%
\pgfusepath{clip}%
\pgfsetbuttcap%
\pgfsetroundjoin%
\definecolor{currentfill}{rgb}{0.074668,0.271519,0.074668}%
\pgfsetfillcolor{currentfill}%
\pgfsetfillopacity{0.200000}%
\pgfsetlinewidth{0.000000pt}%
\definecolor{currentstroke}{rgb}{0.000000,0.000000,0.000000}%
\pgfsetstrokecolor{currentstroke}%
\pgfsetdash{}{0pt}%
\pgfpathmoveto{\pgfqpoint{8.527601in}{1.505058in}}%
\pgfpathlineto{\pgfqpoint{8.375641in}{1.283420in}}%
\pgfpathlineto{\pgfqpoint{8.223548in}{1.496830in}}%
\pgfpathlineto{\pgfqpoint{8.527601in}{1.505058in}}%
\pgfpathclose%
\pgfusepath{fill}%
\end{pgfscope}%
\begin{pgfscope}%
\pgfpathrectangle{\pgfqpoint{6.818937in}{0.147348in}}{\pgfqpoint{2.735294in}{2.735294in}}%
\pgfusepath{clip}%
\pgfsetbuttcap%
\pgfsetroundjoin%
\definecolor{currentfill}{rgb}{0.074668,0.271519,0.074668}%
\pgfsetfillcolor{currentfill}%
\pgfsetfillopacity{0.200000}%
\pgfsetlinewidth{0.000000pt}%
\definecolor{currentstroke}{rgb}{0.000000,0.000000,0.000000}%
\pgfsetstrokecolor{currentstroke}%
\pgfsetdash{}{0pt}%
\pgfpathmoveto{\pgfqpoint{8.223548in}{1.496830in}}%
\pgfpathlineto{\pgfqpoint{8.071454in}{1.283420in}}%
\pgfpathlineto{\pgfqpoint{7.919495in}{1.505058in}}%
\pgfpathlineto{\pgfqpoint{8.223548in}{1.496830in}}%
\pgfpathclose%
\pgfusepath{fill}%
\end{pgfscope}%
\begin{pgfscope}%
\pgfpathrectangle{\pgfqpoint{6.818937in}{0.147348in}}{\pgfqpoint{2.735294in}{2.735294in}}%
\pgfusepath{clip}%
\pgfsetbuttcap%
\pgfsetroundjoin%
\definecolor{currentfill}{rgb}{0.086258,0.313666,0.086258}%
\pgfsetfillcolor{currentfill}%
\pgfsetfillopacity{0.200000}%
\pgfsetlinewidth{0.000000pt}%
\definecolor{currentstroke}{rgb}{0.000000,0.000000,0.000000}%
\pgfsetstrokecolor{currentstroke}%
\pgfsetdash{}{0pt}%
\pgfpathmoveto{\pgfqpoint{7.919495in}{1.505058in}}%
\pgfpathlineto{\pgfqpoint{8.223548in}{1.947612in}}%
\pgfpathlineto{\pgfqpoint{8.223548in}{1.496830in}}%
\pgfpathlineto{\pgfqpoint{7.919495in}{1.505058in}}%
\pgfpathclose%
\pgfusepath{fill}%
\end{pgfscope}%
\begin{pgfscope}%
\pgfpathrectangle{\pgfqpoint{6.818937in}{0.147348in}}{\pgfqpoint{2.735294in}{2.735294in}}%
\pgfusepath{clip}%
\pgfsetbuttcap%
\pgfsetroundjoin%
\definecolor{currentfill}{rgb}{0.086258,0.313666,0.086258}%
\pgfsetfillcolor{currentfill}%
\pgfsetfillopacity{0.200000}%
\pgfsetlinewidth{0.000000pt}%
\definecolor{currentstroke}{rgb}{0.000000,0.000000,0.000000}%
\pgfsetstrokecolor{currentstroke}%
\pgfsetdash{}{0pt}%
\pgfpathmoveto{\pgfqpoint{8.223548in}{1.496830in}}%
\pgfpathlineto{\pgfqpoint{8.223548in}{1.947612in}}%
\pgfpathlineto{\pgfqpoint{8.527601in}{1.505058in}}%
\pgfpathlineto{\pgfqpoint{8.223548in}{1.496830in}}%
\pgfpathclose%
\pgfusepath{fill}%
\end{pgfscope}%
\begin{pgfscope}%
\pgfpathrectangle{\pgfqpoint{6.818937in}{0.147348in}}{\pgfqpoint{2.735294in}{2.735294in}}%
\pgfusepath{clip}%
\pgfsetbuttcap%
\pgfsetroundjoin%
\definecolor{currentfill}{rgb}{0.086061,0.312950,0.086061}%
\pgfsetfillcolor{currentfill}%
\pgfsetfillopacity{0.200000}%
\pgfsetlinewidth{0.000000pt}%
\definecolor{currentstroke}{rgb}{0.000000,0.000000,0.000000}%
\pgfsetstrokecolor{currentstroke}%
\pgfsetdash{}{0pt}%
\pgfpathmoveto{\pgfqpoint{8.223548in}{1.947612in}}%
\pgfpathlineto{\pgfqpoint{7.919495in}{1.505058in}}%
\pgfpathlineto{\pgfqpoint{7.935071in}{1.947550in}}%
\pgfpathlineto{\pgfqpoint{8.223548in}{1.947612in}}%
\pgfpathclose%
\pgfusepath{fill}%
\end{pgfscope}%
\begin{pgfscope}%
\pgfpathrectangle{\pgfqpoint{6.818937in}{0.147348in}}{\pgfqpoint{2.735294in}{2.735294in}}%
\pgfusepath{clip}%
\pgfsetbuttcap%
\pgfsetroundjoin%
\definecolor{currentfill}{rgb}{0.086061,0.312950,0.086061}%
\pgfsetfillcolor{currentfill}%
\pgfsetfillopacity{0.200000}%
\pgfsetlinewidth{0.000000pt}%
\definecolor{currentstroke}{rgb}{0.000000,0.000000,0.000000}%
\pgfsetstrokecolor{currentstroke}%
\pgfsetdash{}{0pt}%
\pgfpathmoveto{\pgfqpoint{8.512025in}{1.947550in}}%
\pgfpathlineto{\pgfqpoint{8.527601in}{1.505058in}}%
\pgfpathlineto{\pgfqpoint{8.223548in}{1.947612in}}%
\pgfpathlineto{\pgfqpoint{8.512025in}{1.947550in}}%
\pgfpathclose%
\pgfusepath{fill}%
\end{pgfscope}%
\begin{pgfscope}%
\pgfpathrectangle{\pgfqpoint{6.818937in}{0.147348in}}{\pgfqpoint{2.735294in}{2.735294in}}%
\pgfusepath{clip}%
\pgfsetbuttcap%
\pgfsetroundjoin%
\definecolor{currentfill}{rgb}{0.075994,0.276341,0.075994}%
\pgfsetfillcolor{currentfill}%
\pgfsetfillopacity{0.200000}%
\pgfsetlinewidth{0.000000pt}%
\definecolor{currentstroke}{rgb}{0.000000,0.000000,0.000000}%
\pgfsetstrokecolor{currentstroke}%
\pgfsetdash{}{0pt}%
\pgfpathmoveto{\pgfqpoint{8.800986in}{1.527187in}}%
\pgfpathlineto{\pgfqpoint{8.664078in}{1.306218in}}%
\pgfpathlineto{\pgfqpoint{8.527601in}{1.505058in}}%
\pgfpathlineto{\pgfqpoint{8.800986in}{1.527187in}}%
\pgfpathclose%
\pgfusepath{fill}%
\end{pgfscope}%
\begin{pgfscope}%
\pgfpathrectangle{\pgfqpoint{6.818937in}{0.147348in}}{\pgfqpoint{2.735294in}{2.735294in}}%
\pgfusepath{clip}%
\pgfsetbuttcap%
\pgfsetroundjoin%
\definecolor{currentfill}{rgb}{0.075994,0.276341,0.075994}%
\pgfsetfillcolor{currentfill}%
\pgfsetfillopacity{0.200000}%
\pgfsetlinewidth{0.000000pt}%
\definecolor{currentstroke}{rgb}{0.000000,0.000000,0.000000}%
\pgfsetstrokecolor{currentstroke}%
\pgfsetdash{}{0pt}%
\pgfpathmoveto{\pgfqpoint{7.919495in}{1.505058in}}%
\pgfpathlineto{\pgfqpoint{7.783018in}{1.306218in}}%
\pgfpathlineto{\pgfqpoint{7.646110in}{1.527187in}}%
\pgfpathlineto{\pgfqpoint{7.919495in}{1.505058in}}%
\pgfpathclose%
\pgfusepath{fill}%
\end{pgfscope}%
\begin{pgfscope}%
\pgfpathrectangle{\pgfqpoint{6.818937in}{0.147348in}}{\pgfqpoint{2.735294in}{2.735294in}}%
\pgfusepath{clip}%
\pgfsetbuttcap%
\pgfsetroundjoin%
\definecolor{currentfill}{rgb}{0.087398,0.317812,0.087398}%
\pgfsetfillcolor{currentfill}%
\pgfsetfillopacity{0.200000}%
\pgfsetlinewidth{0.000000pt}%
\definecolor{currentstroke}{rgb}{0.000000,0.000000,0.000000}%
\pgfsetstrokecolor{currentstroke}%
\pgfsetdash{}{0pt}%
\pgfpathmoveto{\pgfqpoint{8.527601in}{1.505058in}}%
\pgfpathlineto{\pgfqpoint{8.512025in}{1.947550in}}%
\pgfpathlineto{\pgfqpoint{8.800986in}{1.527187in}}%
\pgfpathlineto{\pgfqpoint{8.527601in}{1.505058in}}%
\pgfpathclose%
\pgfusepath{fill}%
\end{pgfscope}%
\begin{pgfscope}%
\pgfpathrectangle{\pgfqpoint{6.818937in}{0.147348in}}{\pgfqpoint{2.735294in}{2.735294in}}%
\pgfusepath{clip}%
\pgfsetbuttcap%
\pgfsetroundjoin%
\definecolor{currentfill}{rgb}{0.087398,0.317812,0.087398}%
\pgfsetfillcolor{currentfill}%
\pgfsetfillopacity{0.200000}%
\pgfsetlinewidth{0.000000pt}%
\definecolor{currentstroke}{rgb}{0.000000,0.000000,0.000000}%
\pgfsetstrokecolor{currentstroke}%
\pgfsetdash{}{0pt}%
\pgfpathmoveto{\pgfqpoint{7.646110in}{1.527187in}}%
\pgfpathlineto{\pgfqpoint{7.935071in}{1.947550in}}%
\pgfpathlineto{\pgfqpoint{7.919495in}{1.505058in}}%
\pgfpathlineto{\pgfqpoint{7.646110in}{1.527187in}}%
\pgfpathclose%
\pgfusepath{fill}%
\end{pgfscope}%
\begin{pgfscope}%
\pgfpathrectangle{\pgfqpoint{6.818937in}{0.147348in}}{\pgfqpoint{2.735294in}{2.735294in}}%
\pgfusepath{clip}%
\pgfsetbuttcap%
\pgfsetroundjoin%
\definecolor{currentfill}{rgb}{0.070209,0.255305,0.070209}%
\pgfsetfillcolor{currentfill}%
\pgfsetfillopacity{0.200000}%
\pgfsetlinewidth{0.000000pt}%
\definecolor{currentstroke}{rgb}{0.000000,0.000000,0.000000}%
\pgfsetstrokecolor{currentstroke}%
\pgfsetdash{}{0pt}%
\pgfpathmoveto{\pgfqpoint{8.625616in}{0.971623in}}%
\pgfpathlineto{\pgfqpoint{8.375641in}{1.283420in}}%
\pgfpathlineto{\pgfqpoint{8.527601in}{1.505058in}}%
\pgfpathlineto{\pgfqpoint{8.625616in}{0.971623in}}%
\pgfpathclose%
\pgfusepath{fill}%
\end{pgfscope}%
\begin{pgfscope}%
\pgfpathrectangle{\pgfqpoint{6.818937in}{0.147348in}}{\pgfqpoint{2.735294in}{2.735294in}}%
\pgfusepath{clip}%
\pgfsetbuttcap%
\pgfsetroundjoin%
\definecolor{currentfill}{rgb}{0.070209,0.255305,0.070209}%
\pgfsetfillcolor{currentfill}%
\pgfsetfillopacity{0.200000}%
\pgfsetlinewidth{0.000000pt}%
\definecolor{currentstroke}{rgb}{0.000000,0.000000,0.000000}%
\pgfsetstrokecolor{currentstroke}%
\pgfsetdash{}{0pt}%
\pgfpathmoveto{\pgfqpoint{7.919495in}{1.505058in}}%
\pgfpathlineto{\pgfqpoint{8.071454in}{1.283420in}}%
\pgfpathlineto{\pgfqpoint{7.821480in}{0.971623in}}%
\pgfpathlineto{\pgfqpoint{7.919495in}{1.505058in}}%
\pgfpathclose%
\pgfusepath{fill}%
\end{pgfscope}%
\begin{pgfscope}%
\pgfpathrectangle{\pgfqpoint{6.818937in}{0.147348in}}{\pgfqpoint{2.735294in}{2.735294in}}%
\pgfusepath{clip}%
\pgfsetbuttcap%
\pgfsetroundjoin%
\definecolor{currentfill}{rgb}{0.098306,0.357475,0.098306}%
\pgfsetfillcolor{currentfill}%
\pgfsetfillopacity{0.200000}%
\pgfsetlinewidth{0.000000pt}%
\definecolor{currentstroke}{rgb}{0.000000,0.000000,0.000000}%
\pgfsetstrokecolor{currentstroke}%
\pgfsetdash{}{0pt}%
\pgfpathmoveto{\pgfqpoint{8.223548in}{1.947612in}}%
\pgfpathlineto{\pgfqpoint{8.361157in}{2.147735in}}%
\pgfpathlineto{\pgfqpoint{8.512025in}{1.947550in}}%
\pgfpathlineto{\pgfqpoint{8.223548in}{1.947612in}}%
\pgfpathclose%
\pgfusepath{fill}%
\end{pgfscope}%
\begin{pgfscope}%
\pgfpathrectangle{\pgfqpoint{6.818937in}{0.147348in}}{\pgfqpoint{2.735294in}{2.735294in}}%
\pgfusepath{clip}%
\pgfsetbuttcap%
\pgfsetroundjoin%
\definecolor{currentfill}{rgb}{0.098306,0.357475,0.098306}%
\pgfsetfillcolor{currentfill}%
\pgfsetfillopacity{0.200000}%
\pgfsetlinewidth{0.000000pt}%
\definecolor{currentstroke}{rgb}{0.000000,0.000000,0.000000}%
\pgfsetstrokecolor{currentstroke}%
\pgfsetdash{}{0pt}%
\pgfpathmoveto{\pgfqpoint{7.935071in}{1.947550in}}%
\pgfpathlineto{\pgfqpoint{8.085938in}{2.147735in}}%
\pgfpathlineto{\pgfqpoint{8.223548in}{1.947612in}}%
\pgfpathlineto{\pgfqpoint{7.935071in}{1.947550in}}%
\pgfpathclose%
\pgfusepath{fill}%
\end{pgfscope}%
\begin{pgfscope}%
\pgfpathrectangle{\pgfqpoint{6.818937in}{0.147348in}}{\pgfqpoint{2.735294in}{2.735294in}}%
\pgfusepath{clip}%
\pgfsetbuttcap%
\pgfsetroundjoin%
\definecolor{currentfill}{rgb}{0.066446,0.241622,0.066446}%
\pgfsetfillcolor{currentfill}%
\pgfsetfillopacity{0.200000}%
\pgfsetlinewidth{0.000000pt}%
\definecolor{currentstroke}{rgb}{0.000000,0.000000,0.000000}%
\pgfsetstrokecolor{currentstroke}%
\pgfsetdash{}{0pt}%
\pgfpathmoveto{\pgfqpoint{8.375641in}{1.283420in}}%
\pgfpathlineto{\pgfqpoint{8.223548in}{0.822079in}}%
\pgfpathlineto{\pgfqpoint{8.223548in}{1.496830in}}%
\pgfpathlineto{\pgfqpoint{8.375641in}{1.283420in}}%
\pgfpathclose%
\pgfusepath{fill}%
\end{pgfscope}%
\begin{pgfscope}%
\pgfpathrectangle{\pgfqpoint{6.818937in}{0.147348in}}{\pgfqpoint{2.735294in}{2.735294in}}%
\pgfusepath{clip}%
\pgfsetbuttcap%
\pgfsetroundjoin%
\definecolor{currentfill}{rgb}{0.066446,0.241622,0.066446}%
\pgfsetfillcolor{currentfill}%
\pgfsetfillopacity{0.200000}%
\pgfsetlinewidth{0.000000pt}%
\definecolor{currentstroke}{rgb}{0.000000,0.000000,0.000000}%
\pgfsetstrokecolor{currentstroke}%
\pgfsetdash{}{0pt}%
\pgfpathmoveto{\pgfqpoint{8.223548in}{1.496830in}}%
\pgfpathlineto{\pgfqpoint{8.223548in}{0.822079in}}%
\pgfpathlineto{\pgfqpoint{8.071454in}{1.283420in}}%
\pgfpathlineto{\pgfqpoint{8.223548in}{1.496830in}}%
\pgfpathclose%
\pgfusepath{fill}%
\end{pgfscope}%
\begin{pgfscope}%
\pgfpathrectangle{\pgfqpoint{6.818937in}{0.147348in}}{\pgfqpoint{2.735294in}{2.735294in}}%
\pgfusepath{clip}%
\pgfsetbuttcap%
\pgfsetroundjoin%
\definecolor{currentfill}{rgb}{0.065035,0.236492,0.065035}%
\pgfsetfillcolor{currentfill}%
\pgfsetfillopacity{0.200000}%
\pgfsetlinewidth{0.000000pt}%
\definecolor{currentstroke}{rgb}{0.000000,0.000000,0.000000}%
\pgfsetstrokecolor{currentstroke}%
\pgfsetdash{}{0pt}%
\pgfpathmoveto{\pgfqpoint{8.625616in}{0.971623in}}%
\pgfpathlineto{\pgfqpoint{8.527601in}{1.505058in}}%
\pgfpathlineto{\pgfqpoint{8.664078in}{1.306218in}}%
\pgfpathlineto{\pgfqpoint{8.625616in}{0.971623in}}%
\pgfpathclose%
\pgfusepath{fill}%
\end{pgfscope}%
\begin{pgfscope}%
\pgfpathrectangle{\pgfqpoint{6.818937in}{0.147348in}}{\pgfqpoint{2.735294in}{2.735294in}}%
\pgfusepath{clip}%
\pgfsetbuttcap%
\pgfsetroundjoin%
\definecolor{currentfill}{rgb}{0.065035,0.236492,0.065035}%
\pgfsetfillcolor{currentfill}%
\pgfsetfillopacity{0.200000}%
\pgfsetlinewidth{0.000000pt}%
\definecolor{currentstroke}{rgb}{0.000000,0.000000,0.000000}%
\pgfsetstrokecolor{currentstroke}%
\pgfsetdash{}{0pt}%
\pgfpathmoveto{\pgfqpoint{7.783018in}{1.306218in}}%
\pgfpathlineto{\pgfqpoint{7.919495in}{1.505058in}}%
\pgfpathlineto{\pgfqpoint{7.821480in}{0.971623in}}%
\pgfpathlineto{\pgfqpoint{7.783018in}{1.306218in}}%
\pgfpathclose%
\pgfusepath{fill}%
\end{pgfscope}%
\begin{pgfscope}%
\pgfpathrectangle{\pgfqpoint{6.818937in}{0.147348in}}{\pgfqpoint{2.735294in}{2.735294in}}%
\pgfusepath{clip}%
\pgfsetbuttcap%
\pgfsetroundjoin%
\definecolor{currentfill}{rgb}{0.101677,0.369734,0.101677}%
\pgfsetfillcolor{currentfill}%
\pgfsetfillopacity{0.200000}%
\pgfsetlinewidth{0.000000pt}%
\definecolor{currentstroke}{rgb}{0.000000,0.000000,0.000000}%
\pgfsetstrokecolor{currentstroke}%
\pgfsetdash{}{0pt}%
\pgfpathmoveto{\pgfqpoint{8.223548in}{1.947612in}}%
\pgfpathlineto{\pgfqpoint{8.085938in}{2.147735in}}%
\pgfpathlineto{\pgfqpoint{8.361157in}{2.147735in}}%
\pgfpathlineto{\pgfqpoint{8.223548in}{1.947612in}}%
\pgfpathclose%
\pgfusepath{fill}%
\end{pgfscope}%
\begin{pgfscope}%
\pgfpathrectangle{\pgfqpoint{6.818937in}{0.147348in}}{\pgfqpoint{2.735294in}{2.735294in}}%
\pgfusepath{clip}%
\pgfsetbuttcap%
\pgfsetroundjoin%
\definecolor{currentfill}{rgb}{0.101759,0.370033,0.101759}%
\pgfsetfillcolor{currentfill}%
\pgfsetfillopacity{0.200000}%
\pgfsetlinewidth{0.000000pt}%
\definecolor{currentstroke}{rgb}{0.000000,0.000000,0.000000}%
\pgfsetstrokecolor{currentstroke}%
\pgfsetdash{}{0pt}%
\pgfpathmoveto{\pgfqpoint{8.512025in}{1.947550in}}%
\pgfpathlineto{\pgfqpoint{8.361157in}{2.147735in}}%
\pgfpathlineto{\pgfqpoint{8.624680in}{2.141959in}}%
\pgfpathlineto{\pgfqpoint{8.512025in}{1.947550in}}%
\pgfpathclose%
\pgfusepath{fill}%
\end{pgfscope}%
\begin{pgfscope}%
\pgfpathrectangle{\pgfqpoint{6.818937in}{0.147348in}}{\pgfqpoint{2.735294in}{2.735294in}}%
\pgfusepath{clip}%
\pgfsetbuttcap%
\pgfsetroundjoin%
\definecolor{currentfill}{rgb}{0.101759,0.370033,0.101759}%
\pgfsetfillcolor{currentfill}%
\pgfsetfillopacity{0.200000}%
\pgfsetlinewidth{0.000000pt}%
\definecolor{currentstroke}{rgb}{0.000000,0.000000,0.000000}%
\pgfsetstrokecolor{currentstroke}%
\pgfsetdash{}{0pt}%
\pgfpathmoveto{\pgfqpoint{7.822416in}{2.141959in}}%
\pgfpathlineto{\pgfqpoint{8.085938in}{2.147735in}}%
\pgfpathlineto{\pgfqpoint{7.935071in}{1.947550in}}%
\pgfpathlineto{\pgfqpoint{7.822416in}{2.141959in}}%
\pgfpathclose%
\pgfusepath{fill}%
\end{pgfscope}%
\begin{pgfscope}%
\pgfpathrectangle{\pgfqpoint{6.818937in}{0.147348in}}{\pgfqpoint{2.735294in}{2.735294in}}%
\pgfusepath{clip}%
\pgfsetbuttcap%
\pgfsetroundjoin%
\definecolor{currentfill}{rgb}{0.091915,0.334238,0.091915}%
\pgfsetfillcolor{currentfill}%
\pgfsetfillopacity{0.200000}%
\pgfsetlinewidth{0.000000pt}%
\definecolor{currentstroke}{rgb}{0.000000,0.000000,0.000000}%
\pgfsetstrokecolor{currentstroke}%
\pgfsetdash{}{0pt}%
\pgfpathmoveto{\pgfqpoint{8.800986in}{1.527187in}}%
\pgfpathlineto{\pgfqpoint{8.512025in}{1.947550in}}%
\pgfpathlineto{\pgfqpoint{8.857732in}{2.131775in}}%
\pgfpathlineto{\pgfqpoint{8.800986in}{1.527187in}}%
\pgfpathclose%
\pgfusepath{fill}%
\end{pgfscope}%
\begin{pgfscope}%
\pgfpathrectangle{\pgfqpoint{6.818937in}{0.147348in}}{\pgfqpoint{2.735294in}{2.735294in}}%
\pgfusepath{clip}%
\pgfsetbuttcap%
\pgfsetroundjoin%
\definecolor{currentfill}{rgb}{0.091915,0.334238,0.091915}%
\pgfsetfillcolor{currentfill}%
\pgfsetfillopacity{0.200000}%
\pgfsetlinewidth{0.000000pt}%
\definecolor{currentstroke}{rgb}{0.000000,0.000000,0.000000}%
\pgfsetstrokecolor{currentstroke}%
\pgfsetdash{}{0pt}%
\pgfpathmoveto{\pgfqpoint{7.589364in}{2.131775in}}%
\pgfpathlineto{\pgfqpoint{7.935071in}{1.947550in}}%
\pgfpathlineto{\pgfqpoint{7.646110in}{1.527187in}}%
\pgfpathlineto{\pgfqpoint{7.589364in}{2.131775in}}%
\pgfpathclose%
\pgfusepath{fill}%
\end{pgfscope}%
\begin{pgfscope}%
\pgfpathrectangle{\pgfqpoint{6.818937in}{0.147348in}}{\pgfqpoint{2.735294in}{2.735294in}}%
\pgfusepath{clip}%
\pgfsetbuttcap%
\pgfsetroundjoin%
\definecolor{currentfill}{rgb}{0.073593,0.267612,0.073593}%
\pgfsetfillcolor{currentfill}%
\pgfsetfillopacity{0.200000}%
\pgfsetlinewidth{0.000000pt}%
\definecolor{currentstroke}{rgb}{0.000000,0.000000,0.000000}%
\pgfsetstrokecolor{currentstroke}%
\pgfsetdash{}{0pt}%
\pgfpathmoveto{\pgfqpoint{8.859135in}{1.021716in}}%
\pgfpathlineto{\pgfqpoint{8.664078in}{1.306218in}}%
\pgfpathlineto{\pgfqpoint{8.800986in}{1.527187in}}%
\pgfpathlineto{\pgfqpoint{8.859135in}{1.021716in}}%
\pgfpathclose%
\pgfusepath{fill}%
\end{pgfscope}%
\begin{pgfscope}%
\pgfpathrectangle{\pgfqpoint{6.818937in}{0.147348in}}{\pgfqpoint{2.735294in}{2.735294in}}%
\pgfusepath{clip}%
\pgfsetbuttcap%
\pgfsetroundjoin%
\definecolor{currentfill}{rgb}{0.073593,0.267612,0.073593}%
\pgfsetfillcolor{currentfill}%
\pgfsetfillopacity{0.200000}%
\pgfsetlinewidth{0.000000pt}%
\definecolor{currentstroke}{rgb}{0.000000,0.000000,0.000000}%
\pgfsetstrokecolor{currentstroke}%
\pgfsetdash{}{0pt}%
\pgfpathmoveto{\pgfqpoint{7.646110in}{1.527187in}}%
\pgfpathlineto{\pgfqpoint{7.783018in}{1.306218in}}%
\pgfpathlineto{\pgfqpoint{7.587961in}{1.021716in}}%
\pgfpathlineto{\pgfqpoint{7.646110in}{1.527187in}}%
\pgfpathclose%
\pgfusepath{fill}%
\end{pgfscope}%
\begin{pgfscope}%
\pgfpathrectangle{\pgfqpoint{6.818937in}{0.147348in}}{\pgfqpoint{2.735294in}{2.735294in}}%
\pgfusepath{clip}%
\pgfsetbuttcap%
\pgfsetroundjoin%
\definecolor{currentfill}{rgb}{0.065434,0.237940,0.065434}%
\pgfsetfillcolor{currentfill}%
\pgfsetfillopacity{0.200000}%
\pgfsetlinewidth{0.000000pt}%
\definecolor{currentstroke}{rgb}{0.000000,0.000000,0.000000}%
\pgfsetstrokecolor{currentstroke}%
\pgfsetdash{}{0pt}%
\pgfpathmoveto{\pgfqpoint{8.223548in}{0.822079in}}%
\pgfpathlineto{\pgfqpoint{8.375641in}{1.283420in}}%
\pgfpathlineto{\pgfqpoint{8.361488in}{0.943207in}}%
\pgfpathlineto{\pgfqpoint{8.223548in}{0.822079in}}%
\pgfpathclose%
\pgfusepath{fill}%
\end{pgfscope}%
\begin{pgfscope}%
\pgfpathrectangle{\pgfqpoint{6.818937in}{0.147348in}}{\pgfqpoint{2.735294in}{2.735294in}}%
\pgfusepath{clip}%
\pgfsetbuttcap%
\pgfsetroundjoin%
\definecolor{currentfill}{rgb}{0.065434,0.237940,0.065434}%
\pgfsetfillcolor{currentfill}%
\pgfsetfillopacity{0.200000}%
\pgfsetlinewidth{0.000000pt}%
\definecolor{currentstroke}{rgb}{0.000000,0.000000,0.000000}%
\pgfsetstrokecolor{currentstroke}%
\pgfsetdash{}{0pt}%
\pgfpathmoveto{\pgfqpoint{8.085608in}{0.943207in}}%
\pgfpathlineto{\pgfqpoint{8.071454in}{1.283420in}}%
\pgfpathlineto{\pgfqpoint{8.223548in}{0.822079in}}%
\pgfpathlineto{\pgfqpoint{8.085608in}{0.943207in}}%
\pgfpathclose%
\pgfusepath{fill}%
\end{pgfscope}%
\begin{pgfscope}%
\pgfpathrectangle{\pgfqpoint{6.818937in}{0.147348in}}{\pgfqpoint{2.735294in}{2.735294in}}%
\pgfusepath{clip}%
\pgfsetbuttcap%
\pgfsetroundjoin%
\definecolor{currentfill}{rgb}{0.067497,0.245443,0.067497}%
\pgfsetfillcolor{currentfill}%
\pgfsetfillopacity{0.200000}%
\pgfsetlinewidth{0.000000pt}%
\definecolor{currentstroke}{rgb}{0.000000,0.000000,0.000000}%
\pgfsetstrokecolor{currentstroke}%
\pgfsetdash{}{0pt}%
\pgfpathmoveto{\pgfqpoint{8.477877in}{0.836228in}}%
\pgfpathlineto{\pgfqpoint{8.361488in}{0.943207in}}%
\pgfpathlineto{\pgfqpoint{8.375641in}{1.283420in}}%
\pgfpathlineto{\pgfqpoint{8.477877in}{0.836228in}}%
\pgfpathclose%
\pgfusepath{fill}%
\end{pgfscope}%
\begin{pgfscope}%
\pgfpathrectangle{\pgfqpoint{6.818937in}{0.147348in}}{\pgfqpoint{2.735294in}{2.735294in}}%
\pgfusepath{clip}%
\pgfsetbuttcap%
\pgfsetroundjoin%
\definecolor{currentfill}{rgb}{0.067497,0.245443,0.067497}%
\pgfsetfillcolor{currentfill}%
\pgfsetfillopacity{0.200000}%
\pgfsetlinewidth{0.000000pt}%
\definecolor{currentstroke}{rgb}{0.000000,0.000000,0.000000}%
\pgfsetstrokecolor{currentstroke}%
\pgfsetdash{}{0pt}%
\pgfpathmoveto{\pgfqpoint{8.071454in}{1.283420in}}%
\pgfpathlineto{\pgfqpoint{8.085608in}{0.943207in}}%
\pgfpathlineto{\pgfqpoint{7.969219in}{0.836228in}}%
\pgfpathlineto{\pgfqpoint{8.071454in}{1.283420in}}%
\pgfpathclose%
\pgfusepath{fill}%
\end{pgfscope}%
\begin{pgfscope}%
\pgfpathrectangle{\pgfqpoint{6.818937in}{0.147348in}}{\pgfqpoint{2.735294in}{2.735294in}}%
\pgfusepath{clip}%
\pgfsetbuttcap%
\pgfsetroundjoin%
\definecolor{currentfill}{rgb}{0.097285,0.353762,0.097285}%
\pgfsetfillcolor{currentfill}%
\pgfsetfillopacity{0.200000}%
\pgfsetlinewidth{0.000000pt}%
\definecolor{currentstroke}{rgb}{0.000000,0.000000,0.000000}%
\pgfsetstrokecolor{currentstroke}%
\pgfsetdash{}{0pt}%
\pgfpathmoveto{\pgfqpoint{8.624680in}{2.141959in}}%
\pgfpathlineto{\pgfqpoint{8.857732in}{2.131775in}}%
\pgfpathlineto{\pgfqpoint{8.512025in}{1.947550in}}%
\pgfpathlineto{\pgfqpoint{8.624680in}{2.141959in}}%
\pgfpathclose%
\pgfusepath{fill}%
\end{pgfscope}%
\begin{pgfscope}%
\pgfpathrectangle{\pgfqpoint{6.818937in}{0.147348in}}{\pgfqpoint{2.735294in}{2.735294in}}%
\pgfusepath{clip}%
\pgfsetbuttcap%
\pgfsetroundjoin%
\definecolor{currentfill}{rgb}{0.097285,0.353762,0.097285}%
\pgfsetfillcolor{currentfill}%
\pgfsetfillopacity{0.200000}%
\pgfsetlinewidth{0.000000pt}%
\definecolor{currentstroke}{rgb}{0.000000,0.000000,0.000000}%
\pgfsetstrokecolor{currentstroke}%
\pgfsetdash{}{0pt}%
\pgfpathmoveto{\pgfqpoint{7.935071in}{1.947550in}}%
\pgfpathlineto{\pgfqpoint{7.589364in}{2.131775in}}%
\pgfpathlineto{\pgfqpoint{7.822416in}{2.141959in}}%
\pgfpathlineto{\pgfqpoint{7.935071in}{1.947550in}}%
\pgfpathclose%
\pgfusepath{fill}%
\end{pgfscope}%
\begin{pgfscope}%
\pgfpathrectangle{\pgfqpoint{6.818937in}{0.147348in}}{\pgfqpoint{2.735294in}{2.735294in}}%
\pgfusepath{clip}%
\pgfsetbuttcap%
\pgfsetroundjoin%
\definecolor{currentfill}{rgb}{0.060562,0.220227,0.060562}%
\pgfsetfillcolor{currentfill}%
\pgfsetfillopacity{0.200000}%
\pgfsetlinewidth{0.000000pt}%
\definecolor{currentstroke}{rgb}{0.000000,0.000000,0.000000}%
\pgfsetstrokecolor{currentstroke}%
\pgfsetdash{}{0pt}%
\pgfpathmoveto{\pgfqpoint{8.477877in}{0.836228in}}%
\pgfpathlineto{\pgfqpoint{8.375641in}{1.283420in}}%
\pgfpathlineto{\pgfqpoint{8.625616in}{0.971623in}}%
\pgfpathlineto{\pgfqpoint{8.477877in}{0.836228in}}%
\pgfpathclose%
\pgfusepath{fill}%
\end{pgfscope}%
\begin{pgfscope}%
\pgfpathrectangle{\pgfqpoint{6.818937in}{0.147348in}}{\pgfqpoint{2.735294in}{2.735294in}}%
\pgfusepath{clip}%
\pgfsetbuttcap%
\pgfsetroundjoin%
\definecolor{currentfill}{rgb}{0.060562,0.220227,0.060562}%
\pgfsetfillcolor{currentfill}%
\pgfsetfillopacity{0.200000}%
\pgfsetlinewidth{0.000000pt}%
\definecolor{currentstroke}{rgb}{0.000000,0.000000,0.000000}%
\pgfsetstrokecolor{currentstroke}%
\pgfsetdash{}{0pt}%
\pgfpathmoveto{\pgfqpoint{7.821480in}{0.971623in}}%
\pgfpathlineto{\pgfqpoint{8.071454in}{1.283420in}}%
\pgfpathlineto{\pgfqpoint{7.969219in}{0.836228in}}%
\pgfpathlineto{\pgfqpoint{7.821480in}{0.971623in}}%
\pgfpathclose%
\pgfusepath{fill}%
\end{pgfscope}%
\begin{pgfscope}%
\pgfpathrectangle{\pgfqpoint{6.818937in}{0.147348in}}{\pgfqpoint{2.735294in}{2.735294in}}%
\pgfusepath{clip}%
\pgfsetbuttcap%
\pgfsetroundjoin%
\definecolor{currentfill}{rgb}{0.092193,0.335248,0.092193}%
\pgfsetfillcolor{currentfill}%
\pgfsetfillopacity{0.200000}%
\pgfsetlinewidth{0.000000pt}%
\definecolor{currentstroke}{rgb}{0.000000,0.000000,0.000000}%
\pgfsetstrokecolor{currentstroke}%
\pgfsetdash{}{0pt}%
\pgfpathmoveto{\pgfqpoint{8.800986in}{1.527187in}}%
\pgfpathlineto{\pgfqpoint{8.857732in}{2.131775in}}%
\pgfpathlineto{\pgfqpoint{9.026747in}{1.557473in}}%
\pgfpathlineto{\pgfqpoint{8.800986in}{1.527187in}}%
\pgfpathclose%
\pgfusepath{fill}%
\end{pgfscope}%
\begin{pgfscope}%
\pgfpathrectangle{\pgfqpoint{6.818937in}{0.147348in}}{\pgfqpoint{2.735294in}{2.735294in}}%
\pgfusepath{clip}%
\pgfsetbuttcap%
\pgfsetroundjoin%
\definecolor{currentfill}{rgb}{0.092193,0.335248,0.092193}%
\pgfsetfillcolor{currentfill}%
\pgfsetfillopacity{0.200000}%
\pgfsetlinewidth{0.000000pt}%
\definecolor{currentstroke}{rgb}{0.000000,0.000000,0.000000}%
\pgfsetstrokecolor{currentstroke}%
\pgfsetdash{}{0pt}%
\pgfpathmoveto{\pgfqpoint{7.420349in}{1.557473in}}%
\pgfpathlineto{\pgfqpoint{7.589364in}{2.131775in}}%
\pgfpathlineto{\pgfqpoint{7.646110in}{1.527187in}}%
\pgfpathlineto{\pgfqpoint{7.420349in}{1.557473in}}%
\pgfpathclose%
\pgfusepath{fill}%
\end{pgfscope}%
\begin{pgfscope}%
\pgfpathrectangle{\pgfqpoint{6.818937in}{0.147348in}}{\pgfqpoint{2.735294in}{2.735294in}}%
\pgfusepath{clip}%
\pgfsetbuttcap%
\pgfsetroundjoin%
\definecolor{currentfill}{rgb}{0.111651,0.406004,0.111651}%
\pgfsetfillcolor{currentfill}%
\pgfsetfillopacity{0.200000}%
\pgfsetlinewidth{0.000000pt}%
\definecolor{currentstroke}{rgb}{0.000000,0.000000,0.000000}%
\pgfsetstrokecolor{currentstroke}%
\pgfsetdash{}{0pt}%
\pgfpathmoveto{\pgfqpoint{7.822416in}{2.141959in}}%
\pgfpathlineto{\pgfqpoint{7.734361in}{2.303109in}}%
\pgfpathlineto{\pgfqpoint{8.085938in}{2.147735in}}%
\pgfpathlineto{\pgfqpoint{7.822416in}{2.141959in}}%
\pgfpathclose%
\pgfusepath{fill}%
\end{pgfscope}%
\begin{pgfscope}%
\pgfpathrectangle{\pgfqpoint{6.818937in}{0.147348in}}{\pgfqpoint{2.735294in}{2.735294in}}%
\pgfusepath{clip}%
\pgfsetbuttcap%
\pgfsetroundjoin%
\definecolor{currentfill}{rgb}{0.111651,0.406004,0.111651}%
\pgfsetfillcolor{currentfill}%
\pgfsetfillopacity{0.200000}%
\pgfsetlinewidth{0.000000pt}%
\definecolor{currentstroke}{rgb}{0.000000,0.000000,0.000000}%
\pgfsetstrokecolor{currentstroke}%
\pgfsetdash{}{0pt}%
\pgfpathmoveto{\pgfqpoint{8.361157in}{2.147735in}}%
\pgfpathlineto{\pgfqpoint{8.712735in}{2.303109in}}%
\pgfpathlineto{\pgfqpoint{8.624680in}{2.141959in}}%
\pgfpathlineto{\pgfqpoint{8.361157in}{2.147735in}}%
\pgfpathclose%
\pgfusepath{fill}%
\end{pgfscope}%
\begin{pgfscope}%
\pgfpathrectangle{\pgfqpoint{6.818937in}{0.147348in}}{\pgfqpoint{2.735294in}{2.735294in}}%
\pgfusepath{clip}%
\pgfsetbuttcap%
\pgfsetroundjoin%
\definecolor{currentfill}{rgb}{0.070885,0.257762,0.070885}%
\pgfsetfillcolor{currentfill}%
\pgfsetfillopacity{0.200000}%
\pgfsetlinewidth{0.000000pt}%
\definecolor{currentstroke}{rgb}{0.000000,0.000000,0.000000}%
\pgfsetstrokecolor{currentstroke}%
\pgfsetdash{}{0pt}%
\pgfpathmoveto{\pgfqpoint{8.664078in}{1.306218in}}%
\pgfpathlineto{\pgfqpoint{8.714128in}{0.875593in}}%
\pgfpathlineto{\pgfqpoint{8.625616in}{0.971623in}}%
\pgfpathlineto{\pgfqpoint{8.664078in}{1.306218in}}%
\pgfpathclose%
\pgfusepath{fill}%
\end{pgfscope}%
\begin{pgfscope}%
\pgfpathrectangle{\pgfqpoint{6.818937in}{0.147348in}}{\pgfqpoint{2.735294in}{2.735294in}}%
\pgfusepath{clip}%
\pgfsetbuttcap%
\pgfsetroundjoin%
\definecolor{currentfill}{rgb}{0.070885,0.257762,0.070885}%
\pgfsetfillcolor{currentfill}%
\pgfsetfillopacity{0.200000}%
\pgfsetlinewidth{0.000000pt}%
\definecolor{currentstroke}{rgb}{0.000000,0.000000,0.000000}%
\pgfsetstrokecolor{currentstroke}%
\pgfsetdash{}{0pt}%
\pgfpathmoveto{\pgfqpoint{7.821480in}{0.971623in}}%
\pgfpathlineto{\pgfqpoint{7.732968in}{0.875593in}}%
\pgfpathlineto{\pgfqpoint{7.783018in}{1.306218in}}%
\pgfpathlineto{\pgfqpoint{7.821480in}{0.971623in}}%
\pgfpathclose%
\pgfusepath{fill}%
\end{pgfscope}%
\begin{pgfscope}%
\pgfpathrectangle{\pgfqpoint{6.818937in}{0.147348in}}{\pgfqpoint{2.735294in}{2.735294in}}%
\pgfusepath{clip}%
\pgfsetbuttcap%
\pgfsetroundjoin%
\definecolor{currentfill}{rgb}{0.070984,0.258123,0.070984}%
\pgfsetfillcolor{currentfill}%
\pgfsetfillopacity{0.200000}%
\pgfsetlinewidth{0.000000pt}%
\definecolor{currentstroke}{rgb}{0.000000,0.000000,0.000000}%
\pgfsetstrokecolor{currentstroke}%
\pgfsetdash{}{0pt}%
\pgfpathmoveto{\pgfqpoint{9.026747in}{1.557473in}}%
\pgfpathlineto{\pgfqpoint{9.053362in}{1.083902in}}%
\pgfpathlineto{\pgfqpoint{8.800986in}{1.527187in}}%
\pgfpathlineto{\pgfqpoint{9.026747in}{1.557473in}}%
\pgfpathclose%
\pgfusepath{fill}%
\end{pgfscope}%
\begin{pgfscope}%
\pgfpathrectangle{\pgfqpoint{6.818937in}{0.147348in}}{\pgfqpoint{2.735294in}{2.735294in}}%
\pgfusepath{clip}%
\pgfsetbuttcap%
\pgfsetroundjoin%
\definecolor{currentfill}{rgb}{0.070984,0.258123,0.070984}%
\pgfsetfillcolor{currentfill}%
\pgfsetfillopacity{0.200000}%
\pgfsetlinewidth{0.000000pt}%
\definecolor{currentstroke}{rgb}{0.000000,0.000000,0.000000}%
\pgfsetstrokecolor{currentstroke}%
\pgfsetdash{}{0pt}%
\pgfpathmoveto{\pgfqpoint{7.646110in}{1.527187in}}%
\pgfpathlineto{\pgfqpoint{7.393734in}{1.083902in}}%
\pgfpathlineto{\pgfqpoint{7.420349in}{1.557473in}}%
\pgfpathlineto{\pgfqpoint{7.646110in}{1.527187in}}%
\pgfpathclose%
\pgfusepath{fill}%
\end{pgfscope}%
\begin{pgfscope}%
\pgfpathrectangle{\pgfqpoint{6.818937in}{0.147348in}}{\pgfqpoint{2.735294in}{2.735294in}}%
\pgfusepath{clip}%
\pgfsetbuttcap%
\pgfsetroundjoin%
\definecolor{currentfill}{rgb}{0.061754,0.224559,0.061754}%
\pgfsetfillcolor{currentfill}%
\pgfsetfillopacity{0.200000}%
\pgfsetlinewidth{0.000000pt}%
\definecolor{currentstroke}{rgb}{0.000000,0.000000,0.000000}%
\pgfsetstrokecolor{currentstroke}%
\pgfsetdash{}{0pt}%
\pgfpathmoveto{\pgfqpoint{8.859135in}{1.021716in}}%
\pgfpathlineto{\pgfqpoint{8.714128in}{0.875593in}}%
\pgfpathlineto{\pgfqpoint{8.664078in}{1.306218in}}%
\pgfpathlineto{\pgfqpoint{8.859135in}{1.021716in}}%
\pgfpathclose%
\pgfusepath{fill}%
\end{pgfscope}%
\begin{pgfscope}%
\pgfpathrectangle{\pgfqpoint{6.818937in}{0.147348in}}{\pgfqpoint{2.735294in}{2.735294in}}%
\pgfusepath{clip}%
\pgfsetbuttcap%
\pgfsetroundjoin%
\definecolor{currentfill}{rgb}{0.061754,0.224559,0.061754}%
\pgfsetfillcolor{currentfill}%
\pgfsetfillopacity{0.200000}%
\pgfsetlinewidth{0.000000pt}%
\definecolor{currentstroke}{rgb}{0.000000,0.000000,0.000000}%
\pgfsetstrokecolor{currentstroke}%
\pgfsetdash{}{0pt}%
\pgfpathmoveto{\pgfqpoint{7.783018in}{1.306218in}}%
\pgfpathlineto{\pgfqpoint{7.732968in}{0.875593in}}%
\pgfpathlineto{\pgfqpoint{7.587961in}{1.021716in}}%
\pgfpathlineto{\pgfqpoint{7.783018in}{1.306218in}}%
\pgfpathclose%
\pgfusepath{fill}%
\end{pgfscope}%
\begin{pgfscope}%
\pgfpathrectangle{\pgfqpoint{6.818937in}{0.147348in}}{\pgfqpoint{2.735294in}{2.735294in}}%
\pgfusepath{clip}%
\pgfsetbuttcap%
\pgfsetroundjoin%
\definecolor{currentfill}{rgb}{0.089078,0.323920,0.089078}%
\pgfsetfillcolor{currentfill}%
\pgfsetfillopacity{0.200000}%
\pgfsetlinewidth{0.000000pt}%
\definecolor{currentstroke}{rgb}{0.000000,0.000000,0.000000}%
\pgfsetstrokecolor{currentstroke}%
\pgfsetdash{}{0pt}%
\pgfpathmoveto{\pgfqpoint{9.026747in}{1.557473in}}%
\pgfpathlineto{\pgfqpoint{8.857732in}{2.131775in}}%
\pgfpathlineto{\pgfqpoint{9.112059in}{1.762300in}}%
\pgfpathlineto{\pgfqpoint{9.026747in}{1.557473in}}%
\pgfpathclose%
\pgfusepath{fill}%
\end{pgfscope}%
\begin{pgfscope}%
\pgfpathrectangle{\pgfqpoint{6.818937in}{0.147348in}}{\pgfqpoint{2.735294in}{2.735294in}}%
\pgfusepath{clip}%
\pgfsetbuttcap%
\pgfsetroundjoin%
\definecolor{currentfill}{rgb}{0.089078,0.323920,0.089078}%
\pgfsetfillcolor{currentfill}%
\pgfsetfillopacity{0.200000}%
\pgfsetlinewidth{0.000000pt}%
\definecolor{currentstroke}{rgb}{0.000000,0.000000,0.000000}%
\pgfsetstrokecolor{currentstroke}%
\pgfsetdash{}{0pt}%
\pgfpathmoveto{\pgfqpoint{7.335037in}{1.762300in}}%
\pgfpathlineto{\pgfqpoint{7.589364in}{2.131775in}}%
\pgfpathlineto{\pgfqpoint{7.420349in}{1.557473in}}%
\pgfpathlineto{\pgfqpoint{7.335037in}{1.762300in}}%
\pgfpathclose%
\pgfusepath{fill}%
\end{pgfscope}%
\begin{pgfscope}%
\pgfpathrectangle{\pgfqpoint{6.818937in}{0.147348in}}{\pgfqpoint{2.735294in}{2.735294in}}%
\pgfusepath{clip}%
\pgfsetbuttcap%
\pgfsetroundjoin%
\definecolor{currentfill}{rgb}{0.107070,0.389346,0.107070}%
\pgfsetfillcolor{currentfill}%
\pgfsetfillopacity{0.200000}%
\pgfsetlinewidth{0.000000pt}%
\definecolor{currentstroke}{rgb}{0.000000,0.000000,0.000000}%
\pgfsetstrokecolor{currentstroke}%
\pgfsetdash{}{0pt}%
\pgfpathmoveto{\pgfqpoint{8.624680in}{2.141959in}}%
\pgfpathlineto{\pgfqpoint{8.712735in}{2.303109in}}%
\pgfpathlineto{\pgfqpoint{8.857732in}{2.131775in}}%
\pgfpathlineto{\pgfqpoint{8.624680in}{2.141959in}}%
\pgfpathclose%
\pgfusepath{fill}%
\end{pgfscope}%
\begin{pgfscope}%
\pgfpathrectangle{\pgfqpoint{6.818937in}{0.147348in}}{\pgfqpoint{2.735294in}{2.735294in}}%
\pgfusepath{clip}%
\pgfsetbuttcap%
\pgfsetroundjoin%
\definecolor{currentfill}{rgb}{0.107070,0.389346,0.107070}%
\pgfsetfillcolor{currentfill}%
\pgfsetfillopacity{0.200000}%
\pgfsetlinewidth{0.000000pt}%
\definecolor{currentstroke}{rgb}{0.000000,0.000000,0.000000}%
\pgfsetstrokecolor{currentstroke}%
\pgfsetdash{}{0pt}%
\pgfpathmoveto{\pgfqpoint{7.589364in}{2.131775in}}%
\pgfpathlineto{\pgfqpoint{7.734361in}{2.303109in}}%
\pgfpathlineto{\pgfqpoint{7.822416in}{2.141959in}}%
\pgfpathlineto{\pgfqpoint{7.589364in}{2.131775in}}%
\pgfpathclose%
\pgfusepath{fill}%
\end{pgfscope}%
\begin{pgfscope}%
\pgfpathrectangle{\pgfqpoint{6.818937in}{0.147348in}}{\pgfqpoint{2.735294in}{2.735294in}}%
\pgfusepath{clip}%
\pgfsetbuttcap%
\pgfsetroundjoin%
\definecolor{currentfill}{rgb}{0.056200,0.204363,0.056200}%
\pgfsetfillcolor{currentfill}%
\pgfsetfillopacity{0.200000}%
\pgfsetlinewidth{0.000000pt}%
\definecolor{currentstroke}{rgb}{0.000000,0.000000,0.000000}%
\pgfsetstrokecolor{currentstroke}%
\pgfsetdash{}{0pt}%
\pgfpathmoveto{\pgfqpoint{7.969219in}{0.836228in}}%
\pgfpathlineto{\pgfqpoint{8.085608in}{0.943207in}}%
\pgfpathlineto{\pgfqpoint{8.223548in}{0.822079in}}%
\pgfpathlineto{\pgfqpoint{7.969219in}{0.836228in}}%
\pgfpathclose%
\pgfusepath{fill}%
\end{pgfscope}%
\begin{pgfscope}%
\pgfpathrectangle{\pgfqpoint{6.818937in}{0.147348in}}{\pgfqpoint{2.735294in}{2.735294in}}%
\pgfusepath{clip}%
\pgfsetbuttcap%
\pgfsetroundjoin%
\definecolor{currentfill}{rgb}{0.056200,0.204363,0.056200}%
\pgfsetfillcolor{currentfill}%
\pgfsetfillopacity{0.200000}%
\pgfsetlinewidth{0.000000pt}%
\definecolor{currentstroke}{rgb}{0.000000,0.000000,0.000000}%
\pgfsetstrokecolor{currentstroke}%
\pgfsetdash{}{0pt}%
\pgfpathmoveto{\pgfqpoint{8.223548in}{0.822079in}}%
\pgfpathlineto{\pgfqpoint{8.361488in}{0.943207in}}%
\pgfpathlineto{\pgfqpoint{8.477877in}{0.836228in}}%
\pgfpathlineto{\pgfqpoint{8.223548in}{0.822079in}}%
\pgfpathclose%
\pgfusepath{fill}%
\end{pgfscope}%
\begin{pgfscope}%
\pgfpathrectangle{\pgfqpoint{6.818937in}{0.147348in}}{\pgfqpoint{2.735294in}{2.735294in}}%
\pgfusepath{clip}%
\pgfsetbuttcap%
\pgfsetroundjoin%
\definecolor{currentfill}{rgb}{0.086498,0.314539,0.086498}%
\pgfsetfillcolor{currentfill}%
\pgfsetfillopacity{0.200000}%
\pgfsetlinewidth{0.000000pt}%
\definecolor{currentstroke}{rgb}{0.000000,0.000000,0.000000}%
\pgfsetstrokecolor{currentstroke}%
\pgfsetdash{}{0pt}%
\pgfpathmoveto{\pgfqpoint{9.026747in}{1.557473in}}%
\pgfpathlineto{\pgfqpoint{9.112059in}{1.762300in}}%
\pgfpathlineto{\pgfqpoint{9.203313in}{1.590365in}}%
\pgfpathlineto{\pgfqpoint{9.026747in}{1.557473in}}%
\pgfpathclose%
\pgfusepath{fill}%
\end{pgfscope}%
\begin{pgfscope}%
\pgfpathrectangle{\pgfqpoint{6.818937in}{0.147348in}}{\pgfqpoint{2.735294in}{2.735294in}}%
\pgfusepath{clip}%
\pgfsetbuttcap%
\pgfsetroundjoin%
\definecolor{currentfill}{rgb}{0.086498,0.314539,0.086498}%
\pgfsetfillcolor{currentfill}%
\pgfsetfillopacity{0.200000}%
\pgfsetlinewidth{0.000000pt}%
\definecolor{currentstroke}{rgb}{0.000000,0.000000,0.000000}%
\pgfsetstrokecolor{currentstroke}%
\pgfsetdash{}{0pt}%
\pgfpathmoveto{\pgfqpoint{7.243783in}{1.590365in}}%
\pgfpathlineto{\pgfqpoint{7.335037in}{1.762300in}}%
\pgfpathlineto{\pgfqpoint{7.420349in}{1.557473in}}%
\pgfpathlineto{\pgfqpoint{7.243783in}{1.590365in}}%
\pgfpathclose%
\pgfusepath{fill}%
\end{pgfscope}%
\begin{pgfscope}%
\pgfpathrectangle{\pgfqpoint{6.818937in}{0.147348in}}{\pgfqpoint{2.735294in}{2.735294in}}%
\pgfusepath{clip}%
\pgfsetbuttcap%
\pgfsetroundjoin%
\definecolor{currentfill}{rgb}{0.090812,0.330224,0.090812}%
\pgfsetfillcolor{currentfill}%
\pgfsetfillopacity{0.200000}%
\pgfsetlinewidth{0.000000pt}%
\definecolor{currentstroke}{rgb}{0.000000,0.000000,0.000000}%
\pgfsetstrokecolor{currentstroke}%
\pgfsetdash{}{0pt}%
\pgfpathmoveto{\pgfqpoint{8.800986in}{1.527187in}}%
\pgfpathlineto{\pgfqpoint{8.920136in}{0.932617in}}%
\pgfpathlineto{\pgfqpoint{8.859135in}{1.021716in}}%
\pgfpathlineto{\pgfqpoint{8.800986in}{1.527187in}}%
\pgfpathclose%
\pgfusepath{fill}%
\end{pgfscope}%
\begin{pgfscope}%
\pgfpathrectangle{\pgfqpoint{6.818937in}{0.147348in}}{\pgfqpoint{2.735294in}{2.735294in}}%
\pgfusepath{clip}%
\pgfsetbuttcap%
\pgfsetroundjoin%
\definecolor{currentfill}{rgb}{0.090812,0.330224,0.090812}%
\pgfsetfillcolor{currentfill}%
\pgfsetfillopacity{0.200000}%
\pgfsetlinewidth{0.000000pt}%
\definecolor{currentstroke}{rgb}{0.000000,0.000000,0.000000}%
\pgfsetstrokecolor{currentstroke}%
\pgfsetdash{}{0pt}%
\pgfpathmoveto{\pgfqpoint{7.587961in}{1.021716in}}%
\pgfpathlineto{\pgfqpoint{7.526959in}{0.932617in}}%
\pgfpathlineto{\pgfqpoint{7.646110in}{1.527187in}}%
\pgfpathlineto{\pgfqpoint{7.587961in}{1.021716in}}%
\pgfpathclose%
\pgfusepath{fill}%
\end{pgfscope}%
\begin{pgfscope}%
\pgfpathrectangle{\pgfqpoint{6.818937in}{0.147348in}}{\pgfqpoint{2.735294in}{2.735294in}}%
\pgfusepath{clip}%
\pgfsetbuttcap%
\pgfsetroundjoin%
\definecolor{currentfill}{rgb}{0.116321,0.422987,0.116321}%
\pgfsetfillcolor{currentfill}%
\pgfsetfillopacity{0.200000}%
\pgfsetlinewidth{0.000000pt}%
\definecolor{currentstroke}{rgb}{0.000000,0.000000,0.000000}%
\pgfsetstrokecolor{currentstroke}%
\pgfsetdash{}{0pt}%
\pgfpathmoveto{\pgfqpoint{8.361157in}{2.147735in}}%
\pgfpathlineto{\pgfqpoint{8.085938in}{2.147735in}}%
\pgfpathlineto{\pgfqpoint{8.123207in}{2.676888in}}%
\pgfpathlineto{\pgfqpoint{8.361157in}{2.147735in}}%
\pgfpathclose%
\pgfusepath{fill}%
\end{pgfscope}%
\begin{pgfscope}%
\pgfpathrectangle{\pgfqpoint{6.818937in}{0.147348in}}{\pgfqpoint{2.735294in}{2.735294in}}%
\pgfusepath{clip}%
\pgfsetbuttcap%
\pgfsetroundjoin%
\definecolor{currentfill}{rgb}{0.057724,0.209904,0.057724}%
\pgfsetfillcolor{currentfill}%
\pgfsetfillopacity{0.200000}%
\pgfsetlinewidth{0.000000pt}%
\definecolor{currentstroke}{rgb}{0.000000,0.000000,0.000000}%
\pgfsetstrokecolor{currentstroke}%
\pgfsetdash{}{0pt}%
\pgfpathmoveto{\pgfqpoint{7.969219in}{0.836228in}}%
\pgfpathlineto{\pgfqpoint{7.732968in}{0.875593in}}%
\pgfpathlineto{\pgfqpoint{7.821480in}{0.971623in}}%
\pgfpathlineto{\pgfqpoint{7.969219in}{0.836228in}}%
\pgfpathclose%
\pgfusepath{fill}%
\end{pgfscope}%
\begin{pgfscope}%
\pgfpathrectangle{\pgfqpoint{6.818937in}{0.147348in}}{\pgfqpoint{2.735294in}{2.735294in}}%
\pgfusepath{clip}%
\pgfsetbuttcap%
\pgfsetroundjoin%
\definecolor{currentfill}{rgb}{0.057724,0.209904,0.057724}%
\pgfsetfillcolor{currentfill}%
\pgfsetfillopacity{0.200000}%
\pgfsetlinewidth{0.000000pt}%
\definecolor{currentstroke}{rgb}{0.000000,0.000000,0.000000}%
\pgfsetstrokecolor{currentstroke}%
\pgfsetdash{}{0pt}%
\pgfpathmoveto{\pgfqpoint{8.625616in}{0.971623in}}%
\pgfpathlineto{\pgfqpoint{8.714128in}{0.875593in}}%
\pgfpathlineto{\pgfqpoint{8.477877in}{0.836228in}}%
\pgfpathlineto{\pgfqpoint{8.625616in}{0.971623in}}%
\pgfpathclose%
\pgfusepath{fill}%
\end{pgfscope}%
\begin{pgfscope}%
\pgfpathrectangle{\pgfqpoint{6.818937in}{0.147348in}}{\pgfqpoint{2.735294in}{2.735294in}}%
\pgfusepath{clip}%
\pgfsetbuttcap%
\pgfsetroundjoin%
\definecolor{currentfill}{rgb}{0.064867,0.235879,0.064867}%
\pgfsetfillcolor{currentfill}%
\pgfsetfillopacity{0.200000}%
\pgfsetlinewidth{0.000000pt}%
\definecolor{currentstroke}{rgb}{0.000000,0.000000,0.000000}%
\pgfsetstrokecolor{currentstroke}%
\pgfsetdash{}{0pt}%
\pgfpathmoveto{\pgfqpoint{9.053362in}{1.083902in}}%
\pgfpathlineto{\pgfqpoint{8.920136in}{0.932617in}}%
\pgfpathlineto{\pgfqpoint{8.800986in}{1.527187in}}%
\pgfpathlineto{\pgfqpoint{9.053362in}{1.083902in}}%
\pgfpathclose%
\pgfusepath{fill}%
\end{pgfscope}%
\begin{pgfscope}%
\pgfpathrectangle{\pgfqpoint{6.818937in}{0.147348in}}{\pgfqpoint{2.735294in}{2.735294in}}%
\pgfusepath{clip}%
\pgfsetbuttcap%
\pgfsetroundjoin%
\definecolor{currentfill}{rgb}{0.064867,0.235879,0.064867}%
\pgfsetfillcolor{currentfill}%
\pgfsetfillopacity{0.200000}%
\pgfsetlinewidth{0.000000pt}%
\definecolor{currentstroke}{rgb}{0.000000,0.000000,0.000000}%
\pgfsetstrokecolor{currentstroke}%
\pgfsetdash{}{0pt}%
\pgfpathmoveto{\pgfqpoint{7.646110in}{1.527187in}}%
\pgfpathlineto{\pgfqpoint{7.526959in}{0.932617in}}%
\pgfpathlineto{\pgfqpoint{7.393734in}{1.083902in}}%
\pgfpathlineto{\pgfqpoint{7.646110in}{1.527187in}}%
\pgfpathclose%
\pgfusepath{fill}%
\end{pgfscope}%
\begin{pgfscope}%
\pgfpathrectangle{\pgfqpoint{6.818937in}{0.147348in}}{\pgfqpoint{2.735294in}{2.735294in}}%
\pgfusepath{clip}%
\pgfsetbuttcap%
\pgfsetroundjoin%
\definecolor{currentfill}{rgb}{0.116785,0.424671,0.116785}%
\pgfsetfillcolor{currentfill}%
\pgfsetfillopacity{0.200000}%
\pgfsetlinewidth{0.000000pt}%
\definecolor{currentstroke}{rgb}{0.000000,0.000000,0.000000}%
\pgfsetstrokecolor{currentstroke}%
\pgfsetdash{}{0pt}%
\pgfpathmoveto{\pgfqpoint{8.361157in}{2.147735in}}%
\pgfpathlineto{\pgfqpoint{8.323888in}{2.676888in}}%
\pgfpathlineto{\pgfqpoint{8.712735in}{2.303109in}}%
\pgfpathlineto{\pgfqpoint{8.361157in}{2.147735in}}%
\pgfpathclose%
\pgfusepath{fill}%
\end{pgfscope}%
\begin{pgfscope}%
\pgfpathrectangle{\pgfqpoint{6.818937in}{0.147348in}}{\pgfqpoint{2.735294in}{2.735294in}}%
\pgfusepath{clip}%
\pgfsetbuttcap%
\pgfsetroundjoin%
\definecolor{currentfill}{rgb}{0.116785,0.424671,0.116785}%
\pgfsetfillcolor{currentfill}%
\pgfsetfillopacity{0.200000}%
\pgfsetlinewidth{0.000000pt}%
\definecolor{currentstroke}{rgb}{0.000000,0.000000,0.000000}%
\pgfsetstrokecolor{currentstroke}%
\pgfsetdash{}{0pt}%
\pgfpathmoveto{\pgfqpoint{7.734361in}{2.303109in}}%
\pgfpathlineto{\pgfqpoint{8.123207in}{2.676888in}}%
\pgfpathlineto{\pgfqpoint{8.085938in}{2.147735in}}%
\pgfpathlineto{\pgfqpoint{7.734361in}{2.303109in}}%
\pgfpathclose%
\pgfusepath{fill}%
\end{pgfscope}%
\begin{pgfscope}%
\pgfpathrectangle{\pgfqpoint{6.818937in}{0.147348in}}{\pgfqpoint{2.735294in}{2.735294in}}%
\pgfusepath{clip}%
\pgfsetbuttcap%
\pgfsetroundjoin%
\definecolor{currentfill}{rgb}{0.074506,0.270932,0.074506}%
\pgfsetfillcolor{currentfill}%
\pgfsetfillopacity{0.200000}%
\pgfsetlinewidth{0.000000pt}%
\definecolor{currentstroke}{rgb}{0.000000,0.000000,0.000000}%
\pgfsetstrokecolor{currentstroke}%
\pgfsetdash{}{0pt}%
\pgfpathmoveto{\pgfqpoint{9.203313in}{1.590365in}}%
\pgfpathlineto{\pgfqpoint{9.208715in}{1.149785in}}%
\pgfpathlineto{\pgfqpoint{9.026747in}{1.557473in}}%
\pgfpathlineto{\pgfqpoint{9.203313in}{1.590365in}}%
\pgfpathclose%
\pgfusepath{fill}%
\end{pgfscope}%
\begin{pgfscope}%
\pgfpathrectangle{\pgfqpoint{6.818937in}{0.147348in}}{\pgfqpoint{2.735294in}{2.735294in}}%
\pgfusepath{clip}%
\pgfsetbuttcap%
\pgfsetroundjoin%
\definecolor{currentfill}{rgb}{0.074506,0.270932,0.074506}%
\pgfsetfillcolor{currentfill}%
\pgfsetfillopacity{0.200000}%
\pgfsetlinewidth{0.000000pt}%
\definecolor{currentstroke}{rgb}{0.000000,0.000000,0.000000}%
\pgfsetstrokecolor{currentstroke}%
\pgfsetdash{}{0pt}%
\pgfpathmoveto{\pgfqpoint{7.420349in}{1.557473in}}%
\pgfpathlineto{\pgfqpoint{7.238381in}{1.149785in}}%
\pgfpathlineto{\pgfqpoint{7.243783in}{1.590365in}}%
\pgfpathlineto{\pgfqpoint{7.420349in}{1.557473in}}%
\pgfpathclose%
\pgfusepath{fill}%
\end{pgfscope}%
\begin{pgfscope}%
\pgfpathrectangle{\pgfqpoint{6.818937in}{0.147348in}}{\pgfqpoint{2.735294in}{2.735294in}}%
\pgfusepath{clip}%
\pgfsetbuttcap%
\pgfsetroundjoin%
\definecolor{currentfill}{rgb}{0.060435,0.219763,0.060435}%
\pgfsetfillcolor{currentfill}%
\pgfsetfillopacity{0.200000}%
\pgfsetlinewidth{0.000000pt}%
\definecolor{currentstroke}{rgb}{0.000000,0.000000,0.000000}%
\pgfsetstrokecolor{currentstroke}%
\pgfsetdash{}{0pt}%
\pgfpathmoveto{\pgfqpoint{8.920136in}{0.932617in}}%
\pgfpathlineto{\pgfqpoint{8.714128in}{0.875593in}}%
\pgfpathlineto{\pgfqpoint{8.859135in}{1.021716in}}%
\pgfpathlineto{\pgfqpoint{8.920136in}{0.932617in}}%
\pgfpathclose%
\pgfusepath{fill}%
\end{pgfscope}%
\begin{pgfscope}%
\pgfpathrectangle{\pgfqpoint{6.818937in}{0.147348in}}{\pgfqpoint{2.735294in}{2.735294in}}%
\pgfusepath{clip}%
\pgfsetbuttcap%
\pgfsetroundjoin%
\definecolor{currentfill}{rgb}{0.060435,0.219763,0.060435}%
\pgfsetfillcolor{currentfill}%
\pgfsetfillopacity{0.200000}%
\pgfsetlinewidth{0.000000pt}%
\definecolor{currentstroke}{rgb}{0.000000,0.000000,0.000000}%
\pgfsetstrokecolor{currentstroke}%
\pgfsetdash{}{0pt}%
\pgfpathmoveto{\pgfqpoint{7.587961in}{1.021716in}}%
\pgfpathlineto{\pgfqpoint{7.732968in}{0.875593in}}%
\pgfpathlineto{\pgfqpoint{7.526959in}{0.932617in}}%
\pgfpathlineto{\pgfqpoint{7.587961in}{1.021716in}}%
\pgfpathclose%
\pgfusepath{fill}%
\end{pgfscope}%
\begin{pgfscope}%
\pgfpathrectangle{\pgfqpoint{6.818937in}{0.147348in}}{\pgfqpoint{2.735294in}{2.735294in}}%
\pgfusepath{clip}%
\pgfsetbuttcap%
\pgfsetroundjoin%
\definecolor{currentfill}{rgb}{0.095351,0.346729,0.095351}%
\pgfsetfillcolor{currentfill}%
\pgfsetfillopacity{0.200000}%
\pgfsetlinewidth{0.000000pt}%
\definecolor{currentstroke}{rgb}{0.000000,0.000000,0.000000}%
\pgfsetstrokecolor{currentstroke}%
\pgfsetdash{}{0pt}%
\pgfpathmoveto{\pgfqpoint{9.026747in}{1.557473in}}%
\pgfpathlineto{\pgfqpoint{9.091593in}{0.998749in}}%
\pgfpathlineto{\pgfqpoint{9.053362in}{1.083902in}}%
\pgfpathlineto{\pgfqpoint{9.026747in}{1.557473in}}%
\pgfpathclose%
\pgfusepath{fill}%
\end{pgfscope}%
\begin{pgfscope}%
\pgfpathrectangle{\pgfqpoint{6.818937in}{0.147348in}}{\pgfqpoint{2.735294in}{2.735294in}}%
\pgfusepath{clip}%
\pgfsetbuttcap%
\pgfsetroundjoin%
\definecolor{currentfill}{rgb}{0.095351,0.346729,0.095351}%
\pgfsetfillcolor{currentfill}%
\pgfsetfillopacity{0.200000}%
\pgfsetlinewidth{0.000000pt}%
\definecolor{currentstroke}{rgb}{0.000000,0.000000,0.000000}%
\pgfsetstrokecolor{currentstroke}%
\pgfsetdash{}{0pt}%
\pgfpathmoveto{\pgfqpoint{7.393734in}{1.083902in}}%
\pgfpathlineto{\pgfqpoint{7.355503in}{0.998749in}}%
\pgfpathlineto{\pgfqpoint{7.420349in}{1.557473in}}%
\pgfpathlineto{\pgfqpoint{7.393734in}{1.083902in}}%
\pgfpathclose%
\pgfusepath{fill}%
\end{pgfscope}%
\begin{pgfscope}%
\pgfpathrectangle{\pgfqpoint{6.818937in}{0.147348in}}{\pgfqpoint{2.735294in}{2.735294in}}%
\pgfusepath{clip}%
\pgfsetbuttcap%
\pgfsetroundjoin%
\definecolor{currentfill}{rgb}{0.067061,0.243857,0.067061}%
\pgfsetfillcolor{currentfill}%
\pgfsetfillopacity{0.200000}%
\pgfsetlinewidth{0.000000pt}%
\definecolor{currentstroke}{rgb}{0.000000,0.000000,0.000000}%
\pgfsetstrokecolor{currentstroke}%
\pgfsetdash{}{0pt}%
\pgfpathmoveto{\pgfqpoint{9.208715in}{1.149785in}}%
\pgfpathlineto{\pgfqpoint{9.091593in}{0.998749in}}%
\pgfpathlineto{\pgfqpoint{9.026747in}{1.557473in}}%
\pgfpathlineto{\pgfqpoint{9.208715in}{1.149785in}}%
\pgfpathclose%
\pgfusepath{fill}%
\end{pgfscope}%
\begin{pgfscope}%
\pgfpathrectangle{\pgfqpoint{6.818937in}{0.147348in}}{\pgfqpoint{2.735294in}{2.735294in}}%
\pgfusepath{clip}%
\pgfsetbuttcap%
\pgfsetroundjoin%
\definecolor{currentfill}{rgb}{0.067061,0.243857,0.067061}%
\pgfsetfillcolor{currentfill}%
\pgfsetfillopacity{0.200000}%
\pgfsetlinewidth{0.000000pt}%
\definecolor{currentstroke}{rgb}{0.000000,0.000000,0.000000}%
\pgfsetstrokecolor{currentstroke}%
\pgfsetdash{}{0pt}%
\pgfpathmoveto{\pgfqpoint{7.420349in}{1.557473in}}%
\pgfpathlineto{\pgfqpoint{7.355503in}{0.998749in}}%
\pgfpathlineto{\pgfqpoint{7.238381in}{1.149785in}}%
\pgfpathlineto{\pgfqpoint{7.420349in}{1.557473in}}%
\pgfpathclose%
\pgfusepath{fill}%
\end{pgfscope}%
\begin{pgfscope}%
\pgfpathrectangle{\pgfqpoint{6.818937in}{0.147348in}}{\pgfqpoint{2.735294in}{2.735294in}}%
\pgfusepath{clip}%
\pgfsetbuttcap%
\pgfsetroundjoin%
\definecolor{currentfill}{rgb}{0.116321,0.422987,0.116321}%
\pgfsetfillcolor{currentfill}%
\pgfsetfillopacity{0.200000}%
\pgfsetlinewidth{0.000000pt}%
\definecolor{currentstroke}{rgb}{0.000000,0.000000,0.000000}%
\pgfsetstrokecolor{currentstroke}%
\pgfsetdash{}{0pt}%
\pgfpathmoveto{\pgfqpoint{8.123207in}{2.676888in}}%
\pgfpathlineto{\pgfqpoint{8.323888in}{2.676888in}}%
\pgfpathlineto{\pgfqpoint{8.361157in}{2.147735in}}%
\pgfpathlineto{\pgfqpoint{8.123207in}{2.676888in}}%
\pgfpathclose%
\pgfusepath{fill}%
\end{pgfscope}%
\begin{pgfscope}%
\pgfpathrectangle{\pgfqpoint{6.818937in}{0.147348in}}{\pgfqpoint{2.735294in}{2.735294in}}%
\pgfusepath{clip}%
\pgfsetbuttcap%
\pgfsetroundjoin%
\definecolor{currentfill}{rgb}{0.063840,0.232145,0.063840}%
\pgfsetfillcolor{currentfill}%
\pgfsetfillopacity{0.200000}%
\pgfsetlinewidth{0.000000pt}%
\definecolor{currentstroke}{rgb}{0.000000,0.000000,0.000000}%
\pgfsetstrokecolor{currentstroke}%
\pgfsetdash{}{0pt}%
\pgfpathmoveto{\pgfqpoint{8.920136in}{0.932617in}}%
\pgfpathlineto{\pgfqpoint{9.053362in}{1.083902in}}%
\pgfpathlineto{\pgfqpoint{9.091593in}{0.998749in}}%
\pgfpathlineto{\pgfqpoint{8.920136in}{0.932617in}}%
\pgfpathclose%
\pgfusepath{fill}%
\end{pgfscope}%
\begin{pgfscope}%
\pgfpathrectangle{\pgfqpoint{6.818937in}{0.147348in}}{\pgfqpoint{2.735294in}{2.735294in}}%
\pgfusepath{clip}%
\pgfsetbuttcap%
\pgfsetroundjoin%
\definecolor{currentfill}{rgb}{0.063840,0.232145,0.063840}%
\pgfsetfillcolor{currentfill}%
\pgfsetfillopacity{0.200000}%
\pgfsetlinewidth{0.000000pt}%
\definecolor{currentstroke}{rgb}{0.000000,0.000000,0.000000}%
\pgfsetstrokecolor{currentstroke}%
\pgfsetdash{}{0pt}%
\pgfpathmoveto{\pgfqpoint{7.355503in}{0.998749in}}%
\pgfpathlineto{\pgfqpoint{7.393734in}{1.083902in}}%
\pgfpathlineto{\pgfqpoint{7.526959in}{0.932617in}}%
\pgfpathlineto{\pgfqpoint{7.355503in}{0.998749in}}%
\pgfpathclose%
\pgfusepath{fill}%
\end{pgfscope}%
\begin{pgfscope}%
\pgfpathrectangle{\pgfqpoint{6.818937in}{0.147348in}}{\pgfqpoint{2.735294in}{2.735294in}}%
\pgfusepath{clip}%
\pgfsetbuttcap%
\pgfsetroundjoin%
\definecolor{currentfill}{rgb}{0.099716,0.362602,0.099716}%
\pgfsetfillcolor{currentfill}%
\pgfsetfillopacity{0.200000}%
\pgfsetlinewidth{0.000000pt}%
\definecolor{currentstroke}{rgb}{0.000000,0.000000,0.000000}%
\pgfsetstrokecolor{currentstroke}%
\pgfsetdash{}{0pt}%
\pgfpathmoveto{\pgfqpoint{9.203313in}{1.590365in}}%
\pgfpathlineto{\pgfqpoint{9.230175in}{1.067062in}}%
\pgfpathlineto{\pgfqpoint{9.208715in}{1.149785in}}%
\pgfpathlineto{\pgfqpoint{9.203313in}{1.590365in}}%
\pgfpathclose%
\pgfusepath{fill}%
\end{pgfscope}%
\begin{pgfscope}%
\pgfpathrectangle{\pgfqpoint{6.818937in}{0.147348in}}{\pgfqpoint{2.735294in}{2.735294in}}%
\pgfusepath{clip}%
\pgfsetbuttcap%
\pgfsetroundjoin%
\definecolor{currentfill}{rgb}{0.099716,0.362602,0.099716}%
\pgfsetfillcolor{currentfill}%
\pgfsetfillopacity{0.200000}%
\pgfsetlinewidth{0.000000pt}%
\definecolor{currentstroke}{rgb}{0.000000,0.000000,0.000000}%
\pgfsetstrokecolor{currentstroke}%
\pgfsetdash{}{0pt}%
\pgfpathmoveto{\pgfqpoint{7.238381in}{1.149785in}}%
\pgfpathlineto{\pgfqpoint{7.216921in}{1.067062in}}%
\pgfpathlineto{\pgfqpoint{7.243783in}{1.590365in}}%
\pgfpathlineto{\pgfqpoint{7.238381in}{1.149785in}}%
\pgfpathclose%
\pgfusepath{fill}%
\end{pgfscope}%
\begin{pgfscope}%
\pgfpathrectangle{\pgfqpoint{6.818937in}{0.147348in}}{\pgfqpoint{2.735294in}{2.735294in}}%
\pgfusepath{clip}%
\pgfsetbuttcap%
\pgfsetroundjoin%
\definecolor{currentfill}{rgb}{0.069492,0.252698,0.069492}%
\pgfsetfillcolor{currentfill}%
\pgfsetfillopacity{0.200000}%
\pgfsetlinewidth{0.000000pt}%
\definecolor{currentstroke}{rgb}{0.000000,0.000000,0.000000}%
\pgfsetstrokecolor{currentstroke}%
\pgfsetdash{}{0pt}%
\pgfpathmoveto{\pgfqpoint{9.330515in}{1.213837in}}%
\pgfpathlineto{\pgfqpoint{9.230175in}{1.067062in}}%
\pgfpathlineto{\pgfqpoint{9.203313in}{1.590365in}}%
\pgfpathlineto{\pgfqpoint{9.330515in}{1.213837in}}%
\pgfpathclose%
\pgfusepath{fill}%
\end{pgfscope}%
\begin{pgfscope}%
\pgfpathrectangle{\pgfqpoint{6.818937in}{0.147348in}}{\pgfqpoint{2.735294in}{2.735294in}}%
\pgfusepath{clip}%
\pgfsetbuttcap%
\pgfsetroundjoin%
\definecolor{currentfill}{rgb}{0.069492,0.252698,0.069492}%
\pgfsetfillcolor{currentfill}%
\pgfsetfillopacity{0.200000}%
\pgfsetlinewidth{0.000000pt}%
\definecolor{currentstroke}{rgb}{0.000000,0.000000,0.000000}%
\pgfsetstrokecolor{currentstroke}%
\pgfsetdash{}{0pt}%
\pgfpathmoveto{\pgfqpoint{7.116581in}{1.213837in}}%
\pgfpathlineto{\pgfqpoint{7.243783in}{1.590365in}}%
\pgfpathlineto{\pgfqpoint{7.216921in}{1.067062in}}%
\pgfpathlineto{\pgfqpoint{7.116581in}{1.213837in}}%
\pgfpathclose%
\pgfusepath{fill}%
\end{pgfscope}%
\begin{pgfscope}%
\pgfpathrectangle{\pgfqpoint{6.818937in}{0.147348in}}{\pgfqpoint{2.735294in}{2.735294in}}%
\pgfusepath{clip}%
\pgfsetbuttcap%
\pgfsetroundjoin%
\definecolor{currentfill}{rgb}{0.067488,0.245410,0.067488}%
\pgfsetfillcolor{currentfill}%
\pgfsetfillopacity{0.200000}%
\pgfsetlinewidth{0.000000pt}%
\definecolor{currentstroke}{rgb}{0.000000,0.000000,0.000000}%
\pgfsetstrokecolor{currentstroke}%
\pgfsetdash{}{0pt}%
\pgfpathmoveto{\pgfqpoint{9.091593in}{0.998749in}}%
\pgfpathlineto{\pgfqpoint{9.208715in}{1.149785in}}%
\pgfpathlineto{\pgfqpoint{9.230175in}{1.067062in}}%
\pgfpathlineto{\pgfqpoint{9.091593in}{0.998749in}}%
\pgfpathclose%
\pgfusepath{fill}%
\end{pgfscope}%
\begin{pgfscope}%
\pgfpathrectangle{\pgfqpoint{6.818937in}{0.147348in}}{\pgfqpoint{2.735294in}{2.735294in}}%
\pgfusepath{clip}%
\pgfsetbuttcap%
\pgfsetroundjoin%
\definecolor{currentfill}{rgb}{0.067488,0.245410,0.067488}%
\pgfsetfillcolor{currentfill}%
\pgfsetfillopacity{0.200000}%
\pgfsetlinewidth{0.000000pt}%
\definecolor{currentstroke}{rgb}{0.000000,0.000000,0.000000}%
\pgfsetstrokecolor{currentstroke}%
\pgfsetdash{}{0pt}%
\pgfpathmoveto{\pgfqpoint{7.216921in}{1.067062in}}%
\pgfpathlineto{\pgfqpoint{7.238381in}{1.149785in}}%
\pgfpathlineto{\pgfqpoint{7.355503in}{0.998749in}}%
\pgfpathlineto{\pgfqpoint{7.216921in}{1.067062in}}%
\pgfpathclose%
\pgfusepath{fill}%
\end{pgfscope}%
\begin{pgfscope}%
\pgfpathrectangle{\pgfqpoint{6.818937in}{0.147348in}}{\pgfqpoint{2.735294in}{2.735294in}}%
\pgfusepath{clip}%
\pgfsetbuttcap%
\pgfsetroundjoin%
\definecolor{currentfill}{rgb}{0.128601,0.467641,0.128601}%
\pgfsetfillcolor{currentfill}%
\pgfsetfillopacity{0.200000}%
\pgfsetlinewidth{0.000000pt}%
\definecolor{currentstroke}{rgb}{0.000000,0.000000,0.000000}%
\pgfsetstrokecolor{currentstroke}%
\pgfsetdash{}{0pt}%
\pgfpathmoveto{\pgfqpoint{8.323888in}{2.676888in}}%
\pgfpathlineto{\pgfqpoint{8.123207in}{2.676888in}}%
\pgfpathlineto{\pgfqpoint{8.223548in}{2.756756in}}%
\pgfpathlineto{\pgfqpoint{8.323888in}{2.676888in}}%
\pgfpathclose%
\pgfusepath{fill}%
\end{pgfscope}%
\begin{pgfscope}%
\pgfpathrectangle{\pgfqpoint{6.818937in}{0.147348in}}{\pgfqpoint{2.735294in}{2.735294in}}%
\pgfusepath{clip}%
\pgfsetbuttcap%
\pgfsetroundjoin%
\definecolor{currentfill}{rgb}{0.071067,0.258424,0.071067}%
\pgfsetfillcolor{currentfill}%
\pgfsetfillopacity{0.200000}%
\pgfsetlinewidth{0.000000pt}%
\definecolor{currentstroke}{rgb}{0.000000,0.000000,0.000000}%
\pgfsetstrokecolor{currentstroke}%
\pgfsetdash{}{0pt}%
\pgfpathmoveto{\pgfqpoint{9.230175in}{1.067062in}}%
\pgfpathlineto{\pgfqpoint{9.330515in}{1.213837in}}%
\pgfpathlineto{\pgfqpoint{9.340537in}{1.133087in}}%
\pgfpathlineto{\pgfqpoint{9.230175in}{1.067062in}}%
\pgfpathclose%
\pgfusepath{fill}%
\end{pgfscope}%
\begin{pgfscope}%
\pgfpathrectangle{\pgfqpoint{6.818937in}{0.147348in}}{\pgfqpoint{2.735294in}{2.735294in}}%
\pgfusepath{clip}%
\pgfsetbuttcap%
\pgfsetroundjoin%
\definecolor{currentfill}{rgb}{0.071067,0.258424,0.071067}%
\pgfsetfillcolor{currentfill}%
\pgfsetfillopacity{0.200000}%
\pgfsetlinewidth{0.000000pt}%
\definecolor{currentstroke}{rgb}{0.000000,0.000000,0.000000}%
\pgfsetstrokecolor{currentstroke}%
\pgfsetdash{}{0pt}%
\pgfpathmoveto{\pgfqpoint{7.116581in}{1.213837in}}%
\pgfpathlineto{\pgfqpoint{7.216921in}{1.067062in}}%
\pgfpathlineto{\pgfqpoint{7.106558in}{1.133087in}}%
\pgfpathlineto{\pgfqpoint{7.116581in}{1.213837in}}%
\pgfpathclose%
\pgfusepath{fill}%
\end{pgfscope}%
\begin{pgfscope}%
\pgfpathrectangle{\pgfqpoint{6.818937in}{0.147348in}}{\pgfqpoint{2.735294in}{2.735294in}}%
\pgfusepath{clip}%
\pgfsetbuttcap%
\pgfsetroundjoin%
\definecolor{currentfill}{rgb}{0.052607,0.201942,0.305459}%
\pgfsetfillcolor{currentfill}%
\pgfsetlinewidth{0.000000pt}%
\definecolor{currentstroke}{rgb}{0.000000,0.000000,0.000000}%
\pgfsetstrokecolor{currentstroke}%
\pgfsetdash{}{0pt}%
\pgfpathmoveto{\pgfqpoint{8.502215in}{1.541694in}}%
\pgfpathlineto{\pgfqpoint{8.362938in}{1.338584in}}%
\pgfpathlineto{\pgfqpoint{8.223548in}{1.534090in}}%
\pgfpathlineto{\pgfqpoint{8.502215in}{1.541694in}}%
\pgfpathclose%
\pgfusepath{fill}%
\end{pgfscope}%
\begin{pgfscope}%
\pgfpathrectangle{\pgfqpoint{6.818937in}{0.147348in}}{\pgfqpoint{2.735294in}{2.735294in}}%
\pgfusepath{clip}%
\pgfsetbuttcap%
\pgfsetroundjoin%
\definecolor{currentfill}{rgb}{0.052607,0.201942,0.305459}%
\pgfsetfillcolor{currentfill}%
\pgfsetlinewidth{0.000000pt}%
\definecolor{currentstroke}{rgb}{0.000000,0.000000,0.000000}%
\pgfsetstrokecolor{currentstroke}%
\pgfsetdash{}{0pt}%
\pgfpathmoveto{\pgfqpoint{8.223548in}{1.534090in}}%
\pgfpathlineto{\pgfqpoint{8.084158in}{1.338584in}}%
\pgfpathlineto{\pgfqpoint{7.944881in}{1.541694in}}%
\pgfpathlineto{\pgfqpoint{8.223548in}{1.534090in}}%
\pgfpathclose%
\pgfusepath{fill}%
\end{pgfscope}%
\begin{pgfscope}%
\pgfpathrectangle{\pgfqpoint{6.818937in}{0.147348in}}{\pgfqpoint{2.735294in}{2.735294in}}%
\pgfusepath{clip}%
\pgfsetbuttcap%
\pgfsetroundjoin%
\definecolor{currentfill}{rgb}{0.060773,0.233289,0.352874}%
\pgfsetfillcolor{currentfill}%
\pgfsetlinewidth{0.000000pt}%
\definecolor{currentstroke}{rgb}{0.000000,0.000000,0.000000}%
\pgfsetstrokecolor{currentstroke}%
\pgfsetdash{}{0pt}%
\pgfpathmoveto{\pgfqpoint{8.223548in}{1.534090in}}%
\pgfpathlineto{\pgfqpoint{8.223548in}{1.947297in}}%
\pgfpathlineto{\pgfqpoint{8.502215in}{1.541694in}}%
\pgfpathlineto{\pgfqpoint{8.223548in}{1.534090in}}%
\pgfpathclose%
\pgfusepath{fill}%
\end{pgfscope}%
\begin{pgfscope}%
\pgfpathrectangle{\pgfqpoint{6.818937in}{0.147348in}}{\pgfqpoint{2.735294in}{2.735294in}}%
\pgfusepath{clip}%
\pgfsetbuttcap%
\pgfsetroundjoin%
\definecolor{currentfill}{rgb}{0.060773,0.233289,0.352874}%
\pgfsetfillcolor{currentfill}%
\pgfsetlinewidth{0.000000pt}%
\definecolor{currentstroke}{rgb}{0.000000,0.000000,0.000000}%
\pgfsetstrokecolor{currentstroke}%
\pgfsetdash{}{0pt}%
\pgfpathmoveto{\pgfqpoint{7.944881in}{1.541694in}}%
\pgfpathlineto{\pgfqpoint{8.223548in}{1.947297in}}%
\pgfpathlineto{\pgfqpoint{8.223548in}{1.534090in}}%
\pgfpathlineto{\pgfqpoint{7.944881in}{1.541694in}}%
\pgfpathclose%
\pgfusepath{fill}%
\end{pgfscope}%
\begin{pgfscope}%
\pgfpathrectangle{\pgfqpoint{6.818937in}{0.147348in}}{\pgfqpoint{2.735294in}{2.735294in}}%
\pgfusepath{clip}%
\pgfsetbuttcap%
\pgfsetroundjoin%
\definecolor{currentfill}{rgb}{0.060634,0.232757,0.352069}%
\pgfsetfillcolor{currentfill}%
\pgfsetlinewidth{0.000000pt}%
\definecolor{currentstroke}{rgb}{0.000000,0.000000,0.000000}%
\pgfsetstrokecolor{currentstroke}%
\pgfsetdash{}{0pt}%
\pgfpathmoveto{\pgfqpoint{8.487849in}{1.947240in}}%
\pgfpathlineto{\pgfqpoint{8.502215in}{1.541694in}}%
\pgfpathlineto{\pgfqpoint{8.223548in}{1.947297in}}%
\pgfpathlineto{\pgfqpoint{8.487849in}{1.947240in}}%
\pgfpathclose%
\pgfusepath{fill}%
\end{pgfscope}%
\begin{pgfscope}%
\pgfpathrectangle{\pgfqpoint{6.818937in}{0.147348in}}{\pgfqpoint{2.735294in}{2.735294in}}%
\pgfusepath{clip}%
\pgfsetbuttcap%
\pgfsetroundjoin%
\definecolor{currentfill}{rgb}{0.060634,0.232757,0.352069}%
\pgfsetfillcolor{currentfill}%
\pgfsetlinewidth{0.000000pt}%
\definecolor{currentstroke}{rgb}{0.000000,0.000000,0.000000}%
\pgfsetstrokecolor{currentstroke}%
\pgfsetdash{}{0pt}%
\pgfpathmoveto{\pgfqpoint{8.223548in}{1.947297in}}%
\pgfpathlineto{\pgfqpoint{7.944881in}{1.541694in}}%
\pgfpathlineto{\pgfqpoint{7.959246in}{1.947240in}}%
\pgfpathlineto{\pgfqpoint{8.223548in}{1.947297in}}%
\pgfpathclose%
\pgfusepath{fill}%
\end{pgfscope}%
\begin{pgfscope}%
\pgfpathrectangle{\pgfqpoint{6.818937in}{0.147348in}}{\pgfqpoint{2.735294in}{2.735294in}}%
\pgfusepath{clip}%
\pgfsetbuttcap%
\pgfsetroundjoin%
\definecolor{currentfill}{rgb}{0.053541,0.205528,0.310883}%
\pgfsetfillcolor{currentfill}%
\pgfsetlinewidth{0.000000pt}%
\definecolor{currentstroke}{rgb}{0.000000,0.000000,0.000000}%
\pgfsetstrokecolor{currentstroke}%
\pgfsetdash{}{0pt}%
\pgfpathmoveto{\pgfqpoint{8.752556in}{1.562133in}}%
\pgfpathlineto{\pgfqpoint{8.627169in}{1.359642in}}%
\pgfpathlineto{\pgfqpoint{8.502215in}{1.541694in}}%
\pgfpathlineto{\pgfqpoint{8.752556in}{1.562133in}}%
\pgfpathclose%
\pgfusepath{fill}%
\end{pgfscope}%
\begin{pgfscope}%
\pgfpathrectangle{\pgfqpoint{6.818937in}{0.147348in}}{\pgfqpoint{2.735294in}{2.735294in}}%
\pgfusepath{clip}%
\pgfsetbuttcap%
\pgfsetroundjoin%
\definecolor{currentfill}{rgb}{0.053541,0.205528,0.310883}%
\pgfsetfillcolor{currentfill}%
\pgfsetlinewidth{0.000000pt}%
\definecolor{currentstroke}{rgb}{0.000000,0.000000,0.000000}%
\pgfsetstrokecolor{currentstroke}%
\pgfsetdash{}{0pt}%
\pgfpathmoveto{\pgfqpoint{7.944881in}{1.541694in}}%
\pgfpathlineto{\pgfqpoint{7.819927in}{1.359642in}}%
\pgfpathlineto{\pgfqpoint{7.694540in}{1.562133in}}%
\pgfpathlineto{\pgfqpoint{7.944881in}{1.541694in}}%
\pgfpathclose%
\pgfusepath{fill}%
\end{pgfscope}%
\begin{pgfscope}%
\pgfpathrectangle{\pgfqpoint{6.818937in}{0.147348in}}{\pgfqpoint{2.735294in}{2.735294in}}%
\pgfusepath{clip}%
\pgfsetbuttcap%
\pgfsetroundjoin%
\definecolor{currentfill}{rgb}{0.061576,0.236373,0.357539}%
\pgfsetfillcolor{currentfill}%
\pgfsetlinewidth{0.000000pt}%
\definecolor{currentstroke}{rgb}{0.000000,0.000000,0.000000}%
\pgfsetstrokecolor{currentstroke}%
\pgfsetdash{}{0pt}%
\pgfpathmoveto{\pgfqpoint{7.694540in}{1.562133in}}%
\pgfpathlineto{\pgfqpoint{7.959246in}{1.947240in}}%
\pgfpathlineto{\pgfqpoint{7.944881in}{1.541694in}}%
\pgfpathlineto{\pgfqpoint{7.694540in}{1.562133in}}%
\pgfpathclose%
\pgfusepath{fill}%
\end{pgfscope}%
\begin{pgfscope}%
\pgfpathrectangle{\pgfqpoint{6.818937in}{0.147348in}}{\pgfqpoint{2.735294in}{2.735294in}}%
\pgfusepath{clip}%
\pgfsetbuttcap%
\pgfsetroundjoin%
\definecolor{currentfill}{rgb}{0.061576,0.236373,0.357539}%
\pgfsetfillcolor{currentfill}%
\pgfsetlinewidth{0.000000pt}%
\definecolor{currentstroke}{rgb}{0.000000,0.000000,0.000000}%
\pgfsetstrokecolor{currentstroke}%
\pgfsetdash{}{0pt}%
\pgfpathmoveto{\pgfqpoint{8.502215in}{1.541694in}}%
\pgfpathlineto{\pgfqpoint{8.487849in}{1.947240in}}%
\pgfpathlineto{\pgfqpoint{8.752556in}{1.562133in}}%
\pgfpathlineto{\pgfqpoint{8.502215in}{1.541694in}}%
\pgfpathclose%
\pgfusepath{fill}%
\end{pgfscope}%
\begin{pgfscope}%
\pgfpathrectangle{\pgfqpoint{6.818937in}{0.147348in}}{\pgfqpoint{2.735294in}{2.735294in}}%
\pgfusepath{clip}%
\pgfsetbuttcap%
\pgfsetroundjoin%
\definecolor{currentfill}{rgb}{0.049465,0.189883,0.287218}%
\pgfsetfillcolor{currentfill}%
\pgfsetlinewidth{0.000000pt}%
\definecolor{currentstroke}{rgb}{0.000000,0.000000,0.000000}%
\pgfsetstrokecolor{currentstroke}%
\pgfsetdash{}{0pt}%
\pgfpathmoveto{\pgfqpoint{8.591651in}{1.053752in}}%
\pgfpathlineto{\pgfqpoint{8.362938in}{1.338584in}}%
\pgfpathlineto{\pgfqpoint{8.502215in}{1.541694in}}%
\pgfpathlineto{\pgfqpoint{8.591651in}{1.053752in}}%
\pgfpathclose%
\pgfusepath{fill}%
\end{pgfscope}%
\begin{pgfscope}%
\pgfpathrectangle{\pgfqpoint{6.818937in}{0.147348in}}{\pgfqpoint{2.735294in}{2.735294in}}%
\pgfusepath{clip}%
\pgfsetbuttcap%
\pgfsetroundjoin%
\definecolor{currentfill}{rgb}{0.049465,0.189883,0.287218}%
\pgfsetfillcolor{currentfill}%
\pgfsetlinewidth{0.000000pt}%
\definecolor{currentstroke}{rgb}{0.000000,0.000000,0.000000}%
\pgfsetstrokecolor{currentstroke}%
\pgfsetdash{}{0pt}%
\pgfpathmoveto{\pgfqpoint{7.944881in}{1.541694in}}%
\pgfpathlineto{\pgfqpoint{8.084158in}{1.338584in}}%
\pgfpathlineto{\pgfqpoint{7.855445in}{1.053752in}}%
\pgfpathlineto{\pgfqpoint{7.944881in}{1.541694in}}%
\pgfpathclose%
\pgfusepath{fill}%
\end{pgfscope}%
\begin{pgfscope}%
\pgfpathrectangle{\pgfqpoint{6.818937in}{0.147348in}}{\pgfqpoint{2.735294in}{2.735294in}}%
\pgfusepath{clip}%
\pgfsetbuttcap%
\pgfsetroundjoin%
\definecolor{currentfill}{rgb}{0.069261,0.265872,0.402159}%
\pgfsetfillcolor{currentfill}%
\pgfsetlinewidth{0.000000pt}%
\definecolor{currentstroke}{rgb}{0.000000,0.000000,0.000000}%
\pgfsetstrokecolor{currentstroke}%
\pgfsetdash{}{0pt}%
\pgfpathmoveto{\pgfqpoint{8.223548in}{1.947297in}}%
\pgfpathlineto{\pgfqpoint{8.349584in}{2.130589in}}%
\pgfpathlineto{\pgfqpoint{8.487849in}{1.947240in}}%
\pgfpathlineto{\pgfqpoint{8.223548in}{1.947297in}}%
\pgfpathclose%
\pgfusepath{fill}%
\end{pgfscope}%
\begin{pgfscope}%
\pgfpathrectangle{\pgfqpoint{6.818937in}{0.147348in}}{\pgfqpoint{2.735294in}{2.735294in}}%
\pgfusepath{clip}%
\pgfsetbuttcap%
\pgfsetroundjoin%
\definecolor{currentfill}{rgb}{0.069261,0.265872,0.402159}%
\pgfsetfillcolor{currentfill}%
\pgfsetlinewidth{0.000000pt}%
\definecolor{currentstroke}{rgb}{0.000000,0.000000,0.000000}%
\pgfsetstrokecolor{currentstroke}%
\pgfsetdash{}{0pt}%
\pgfpathmoveto{\pgfqpoint{7.959246in}{1.947240in}}%
\pgfpathlineto{\pgfqpoint{8.097511in}{2.130589in}}%
\pgfpathlineto{\pgfqpoint{8.223548in}{1.947297in}}%
\pgfpathlineto{\pgfqpoint{7.959246in}{1.947240in}}%
\pgfpathclose%
\pgfusepath{fill}%
\end{pgfscope}%
\begin{pgfscope}%
\pgfpathrectangle{\pgfqpoint{6.818937in}{0.147348in}}{\pgfqpoint{2.735294in}{2.735294in}}%
\pgfusepath{clip}%
\pgfsetbuttcap%
\pgfsetroundjoin%
\definecolor{currentfill}{rgb}{0.046814,0.179706,0.271825}%
\pgfsetfillcolor{currentfill}%
\pgfsetlinewidth{0.000000pt}%
\definecolor{currentstroke}{rgb}{0.000000,0.000000,0.000000}%
\pgfsetstrokecolor{currentstroke}%
\pgfsetdash{}{0pt}%
\pgfpathmoveto{\pgfqpoint{8.362938in}{1.338584in}}%
\pgfpathlineto{\pgfqpoint{8.223548in}{0.917160in}}%
\pgfpathlineto{\pgfqpoint{8.223548in}{1.534090in}}%
\pgfpathlineto{\pgfqpoint{8.362938in}{1.338584in}}%
\pgfpathclose%
\pgfusepath{fill}%
\end{pgfscope}%
\begin{pgfscope}%
\pgfpathrectangle{\pgfqpoint{6.818937in}{0.147348in}}{\pgfqpoint{2.735294in}{2.735294in}}%
\pgfusepath{clip}%
\pgfsetbuttcap%
\pgfsetroundjoin%
\definecolor{currentfill}{rgb}{0.046814,0.179706,0.271825}%
\pgfsetfillcolor{currentfill}%
\pgfsetlinewidth{0.000000pt}%
\definecolor{currentstroke}{rgb}{0.000000,0.000000,0.000000}%
\pgfsetstrokecolor{currentstroke}%
\pgfsetdash{}{0pt}%
\pgfpathmoveto{\pgfqpoint{8.223548in}{1.534090in}}%
\pgfpathlineto{\pgfqpoint{8.223548in}{0.917160in}}%
\pgfpathlineto{\pgfqpoint{8.084158in}{1.338584in}}%
\pgfpathlineto{\pgfqpoint{8.223548in}{1.534090in}}%
\pgfpathclose%
\pgfusepath{fill}%
\end{pgfscope}%
\begin{pgfscope}%
\pgfpathrectangle{\pgfqpoint{6.818937in}{0.147348in}}{\pgfqpoint{2.735294in}{2.735294in}}%
\pgfusepath{clip}%
\pgfsetbuttcap%
\pgfsetroundjoin%
\definecolor{currentfill}{rgb}{0.045820,0.175891,0.266053}%
\pgfsetfillcolor{currentfill}%
\pgfsetlinewidth{0.000000pt}%
\definecolor{currentstroke}{rgb}{0.000000,0.000000,0.000000}%
\pgfsetstrokecolor{currentstroke}%
\pgfsetdash{}{0pt}%
\pgfpathmoveto{\pgfqpoint{8.591651in}{1.053752in}}%
\pgfpathlineto{\pgfqpoint{8.502215in}{1.541694in}}%
\pgfpathlineto{\pgfqpoint{8.627169in}{1.359642in}}%
\pgfpathlineto{\pgfqpoint{8.591651in}{1.053752in}}%
\pgfpathclose%
\pgfusepath{fill}%
\end{pgfscope}%
\begin{pgfscope}%
\pgfpathrectangle{\pgfqpoint{6.818937in}{0.147348in}}{\pgfqpoint{2.735294in}{2.735294in}}%
\pgfusepath{clip}%
\pgfsetbuttcap%
\pgfsetroundjoin%
\definecolor{currentfill}{rgb}{0.045820,0.175891,0.266053}%
\pgfsetfillcolor{currentfill}%
\pgfsetlinewidth{0.000000pt}%
\definecolor{currentstroke}{rgb}{0.000000,0.000000,0.000000}%
\pgfsetstrokecolor{currentstroke}%
\pgfsetdash{}{0pt}%
\pgfpathmoveto{\pgfqpoint{7.819927in}{1.359642in}}%
\pgfpathlineto{\pgfqpoint{7.944881in}{1.541694in}}%
\pgfpathlineto{\pgfqpoint{7.855445in}{1.053752in}}%
\pgfpathlineto{\pgfqpoint{7.819927in}{1.359642in}}%
\pgfpathclose%
\pgfusepath{fill}%
\end{pgfscope}%
\begin{pgfscope}%
\pgfpathrectangle{\pgfqpoint{6.818937in}{0.147348in}}{\pgfqpoint{2.735294in}{2.735294in}}%
\pgfusepath{clip}%
\pgfsetbuttcap%
\pgfsetroundjoin%
\definecolor{currentfill}{rgb}{0.071636,0.274990,0.415951}%
\pgfsetfillcolor{currentfill}%
\pgfsetlinewidth{0.000000pt}%
\definecolor{currentstroke}{rgb}{0.000000,0.000000,0.000000}%
\pgfsetstrokecolor{currentstroke}%
\pgfsetdash{}{0pt}%
\pgfpathmoveto{\pgfqpoint{8.223548in}{1.947297in}}%
\pgfpathlineto{\pgfqpoint{8.097511in}{2.130589in}}%
\pgfpathlineto{\pgfqpoint{8.349584in}{2.130589in}}%
\pgfpathlineto{\pgfqpoint{8.223548in}{1.947297in}}%
\pgfpathclose%
\pgfusepath{fill}%
\end{pgfscope}%
\begin{pgfscope}%
\pgfpathrectangle{\pgfqpoint{6.818937in}{0.147348in}}{\pgfqpoint{2.735294in}{2.735294in}}%
\pgfusepath{clip}%
\pgfsetbuttcap%
\pgfsetroundjoin%
\definecolor{currentfill}{rgb}{0.071694,0.275212,0.416288}%
\pgfsetfillcolor{currentfill}%
\pgfsetlinewidth{0.000000pt}%
\definecolor{currentstroke}{rgb}{0.000000,0.000000,0.000000}%
\pgfsetstrokecolor{currentstroke}%
\pgfsetdash{}{0pt}%
\pgfpathmoveto{\pgfqpoint{8.487849in}{1.947240in}}%
\pgfpathlineto{\pgfqpoint{8.349584in}{2.130589in}}%
\pgfpathlineto{\pgfqpoint{8.590867in}{2.125260in}}%
\pgfpathlineto{\pgfqpoint{8.487849in}{1.947240in}}%
\pgfpathclose%
\pgfusepath{fill}%
\end{pgfscope}%
\begin{pgfscope}%
\pgfpathrectangle{\pgfqpoint{6.818937in}{0.147348in}}{\pgfqpoint{2.735294in}{2.735294in}}%
\pgfusepath{clip}%
\pgfsetbuttcap%
\pgfsetroundjoin%
\definecolor{currentfill}{rgb}{0.071694,0.275212,0.416288}%
\pgfsetfillcolor{currentfill}%
\pgfsetlinewidth{0.000000pt}%
\definecolor{currentstroke}{rgb}{0.000000,0.000000,0.000000}%
\pgfsetstrokecolor{currentstroke}%
\pgfsetdash{}{0pt}%
\pgfpathmoveto{\pgfqpoint{7.856229in}{2.125260in}}%
\pgfpathlineto{\pgfqpoint{8.097511in}{2.130589in}}%
\pgfpathlineto{\pgfqpoint{7.959246in}{1.947240in}}%
\pgfpathlineto{\pgfqpoint{7.856229in}{2.125260in}}%
\pgfpathclose%
\pgfusepath{fill}%
\end{pgfscope}%
\begin{pgfscope}%
\pgfpathrectangle{\pgfqpoint{6.818937in}{0.147348in}}{\pgfqpoint{2.735294in}{2.735294in}}%
\pgfusepath{clip}%
\pgfsetbuttcap%
\pgfsetroundjoin%
\definecolor{currentfill}{rgb}{0.064759,0.248590,0.376018}%
\pgfsetfillcolor{currentfill}%
\pgfsetlinewidth{0.000000pt}%
\definecolor{currentstroke}{rgb}{0.000000,0.000000,0.000000}%
\pgfsetstrokecolor{currentstroke}%
\pgfsetdash{}{0pt}%
\pgfpathmoveto{\pgfqpoint{8.752556in}{1.562133in}}%
\pgfpathlineto{\pgfqpoint{8.487849in}{1.947240in}}%
\pgfpathlineto{\pgfqpoint{8.804056in}{2.115869in}}%
\pgfpathlineto{\pgfqpoint{8.752556in}{1.562133in}}%
\pgfpathclose%
\pgfusepath{fill}%
\end{pgfscope}%
\begin{pgfscope}%
\pgfpathrectangle{\pgfqpoint{6.818937in}{0.147348in}}{\pgfqpoint{2.735294in}{2.735294in}}%
\pgfusepath{clip}%
\pgfsetbuttcap%
\pgfsetroundjoin%
\definecolor{currentfill}{rgb}{0.064759,0.248590,0.376018}%
\pgfsetfillcolor{currentfill}%
\pgfsetlinewidth{0.000000pt}%
\definecolor{currentstroke}{rgb}{0.000000,0.000000,0.000000}%
\pgfsetstrokecolor{currentstroke}%
\pgfsetdash{}{0pt}%
\pgfpathmoveto{\pgfqpoint{7.643040in}{2.115869in}}%
\pgfpathlineto{\pgfqpoint{7.959246in}{1.947240in}}%
\pgfpathlineto{\pgfqpoint{7.694540in}{1.562133in}}%
\pgfpathlineto{\pgfqpoint{7.643040in}{2.115869in}}%
\pgfpathclose%
\pgfusepath{fill}%
\end{pgfscope}%
\begin{pgfscope}%
\pgfpathrectangle{\pgfqpoint{6.818937in}{0.147348in}}{\pgfqpoint{2.735294in}{2.735294in}}%
\pgfusepath{clip}%
\pgfsetbuttcap%
\pgfsetroundjoin%
\definecolor{currentfill}{rgb}{0.051850,0.199036,0.301063}%
\pgfsetfillcolor{currentfill}%
\pgfsetlinewidth{0.000000pt}%
\definecolor{currentstroke}{rgb}{0.000000,0.000000,0.000000}%
\pgfsetstrokecolor{currentstroke}%
\pgfsetdash{}{0pt}%
\pgfpathmoveto{\pgfqpoint{8.805232in}{1.099921in}}%
\pgfpathlineto{\pgfqpoint{8.627169in}{1.359642in}}%
\pgfpathlineto{\pgfqpoint{8.752556in}{1.562133in}}%
\pgfpathlineto{\pgfqpoint{8.805232in}{1.099921in}}%
\pgfpathclose%
\pgfusepath{fill}%
\end{pgfscope}%
\begin{pgfscope}%
\pgfpathrectangle{\pgfqpoint{6.818937in}{0.147348in}}{\pgfqpoint{2.735294in}{2.735294in}}%
\pgfusepath{clip}%
\pgfsetbuttcap%
\pgfsetroundjoin%
\definecolor{currentfill}{rgb}{0.051850,0.199036,0.301063}%
\pgfsetfillcolor{currentfill}%
\pgfsetlinewidth{0.000000pt}%
\definecolor{currentstroke}{rgb}{0.000000,0.000000,0.000000}%
\pgfsetstrokecolor{currentstroke}%
\pgfsetdash{}{0pt}%
\pgfpathmoveto{\pgfqpoint{7.694540in}{1.562133in}}%
\pgfpathlineto{\pgfqpoint{7.819927in}{1.359642in}}%
\pgfpathlineto{\pgfqpoint{7.641864in}{1.099921in}}%
\pgfpathlineto{\pgfqpoint{7.694540in}{1.562133in}}%
\pgfpathclose%
\pgfusepath{fill}%
\end{pgfscope}%
\begin{pgfscope}%
\pgfpathrectangle{\pgfqpoint{6.818937in}{0.147348in}}{\pgfqpoint{2.735294in}{2.735294in}}%
\pgfusepath{clip}%
\pgfsetbuttcap%
\pgfsetroundjoin%
\definecolor{currentfill}{rgb}{0.046101,0.176968,0.267683}%
\pgfsetfillcolor{currentfill}%
\pgfsetlinewidth{0.000000pt}%
\definecolor{currentstroke}{rgb}{0.000000,0.000000,0.000000}%
\pgfsetstrokecolor{currentstroke}%
\pgfsetdash{}{0pt}%
\pgfpathmoveto{\pgfqpoint{8.223548in}{0.917160in}}%
\pgfpathlineto{\pgfqpoint{8.362938in}{1.338584in}}%
\pgfpathlineto{\pgfqpoint{8.349861in}{1.027548in}}%
\pgfpathlineto{\pgfqpoint{8.223548in}{0.917160in}}%
\pgfpathclose%
\pgfusepath{fill}%
\end{pgfscope}%
\begin{pgfscope}%
\pgfpathrectangle{\pgfqpoint{6.818937in}{0.147348in}}{\pgfqpoint{2.735294in}{2.735294in}}%
\pgfusepath{clip}%
\pgfsetbuttcap%
\pgfsetroundjoin%
\definecolor{currentfill}{rgb}{0.046101,0.176968,0.267683}%
\pgfsetfillcolor{currentfill}%
\pgfsetlinewidth{0.000000pt}%
\definecolor{currentstroke}{rgb}{0.000000,0.000000,0.000000}%
\pgfsetstrokecolor{currentstroke}%
\pgfsetdash{}{0pt}%
\pgfpathmoveto{\pgfqpoint{8.097234in}{1.027548in}}%
\pgfpathlineto{\pgfqpoint{8.084158in}{1.338584in}}%
\pgfpathlineto{\pgfqpoint{8.223548in}{0.917160in}}%
\pgfpathlineto{\pgfqpoint{8.097234in}{1.027548in}}%
\pgfpathclose%
\pgfusepath{fill}%
\end{pgfscope}%
\begin{pgfscope}%
\pgfpathrectangle{\pgfqpoint{6.818937in}{0.147348in}}{\pgfqpoint{2.735294in}{2.735294in}}%
\pgfusepath{clip}%
\pgfsetbuttcap%
\pgfsetroundjoin%
\definecolor{currentfill}{rgb}{0.047555,0.182548,0.276123}%
\pgfsetfillcolor{currentfill}%
\pgfsetlinewidth{0.000000pt}%
\definecolor{currentstroke}{rgb}{0.000000,0.000000,0.000000}%
\pgfsetstrokecolor{currentstroke}%
\pgfsetdash{}{0pt}%
\pgfpathmoveto{\pgfqpoint{8.456300in}{0.930196in}}%
\pgfpathlineto{\pgfqpoint{8.349861in}{1.027548in}}%
\pgfpathlineto{\pgfqpoint{8.362938in}{1.338584in}}%
\pgfpathlineto{\pgfqpoint{8.456300in}{0.930196in}}%
\pgfpathclose%
\pgfusepath{fill}%
\end{pgfscope}%
\begin{pgfscope}%
\pgfpathrectangle{\pgfqpoint{6.818937in}{0.147348in}}{\pgfqpoint{2.735294in}{2.735294in}}%
\pgfusepath{clip}%
\pgfsetbuttcap%
\pgfsetroundjoin%
\definecolor{currentfill}{rgb}{0.047555,0.182548,0.276123}%
\pgfsetfillcolor{currentfill}%
\pgfsetlinewidth{0.000000pt}%
\definecolor{currentstroke}{rgb}{0.000000,0.000000,0.000000}%
\pgfsetstrokecolor{currentstroke}%
\pgfsetdash{}{0pt}%
\pgfpathmoveto{\pgfqpoint{8.084158in}{1.338584in}}%
\pgfpathlineto{\pgfqpoint{8.097234in}{1.027548in}}%
\pgfpathlineto{\pgfqpoint{7.990796in}{0.930196in}}%
\pgfpathlineto{\pgfqpoint{8.084158in}{1.338584in}}%
\pgfpathclose%
\pgfusepath{fill}%
\end{pgfscope}%
\begin{pgfscope}%
\pgfpathrectangle{\pgfqpoint{6.818937in}{0.147348in}}{\pgfqpoint{2.735294in}{2.735294in}}%
\pgfusepath{clip}%
\pgfsetbuttcap%
\pgfsetroundjoin%
\definecolor{currentfill}{rgb}{0.068541,0.263111,0.397982}%
\pgfsetfillcolor{currentfill}%
\pgfsetlinewidth{0.000000pt}%
\definecolor{currentstroke}{rgb}{0.000000,0.000000,0.000000}%
\pgfsetstrokecolor{currentstroke}%
\pgfsetdash{}{0pt}%
\pgfpathmoveto{\pgfqpoint{8.590867in}{2.125260in}}%
\pgfpathlineto{\pgfqpoint{8.804056in}{2.115869in}}%
\pgfpathlineto{\pgfqpoint{8.487849in}{1.947240in}}%
\pgfpathlineto{\pgfqpoint{8.590867in}{2.125260in}}%
\pgfpathclose%
\pgfusepath{fill}%
\end{pgfscope}%
\begin{pgfscope}%
\pgfpathrectangle{\pgfqpoint{6.818937in}{0.147348in}}{\pgfqpoint{2.735294in}{2.735294in}}%
\pgfusepath{clip}%
\pgfsetbuttcap%
\pgfsetroundjoin%
\definecolor{currentfill}{rgb}{0.068541,0.263111,0.397982}%
\pgfsetfillcolor{currentfill}%
\pgfsetlinewidth{0.000000pt}%
\definecolor{currentstroke}{rgb}{0.000000,0.000000,0.000000}%
\pgfsetstrokecolor{currentstroke}%
\pgfsetdash{}{0pt}%
\pgfpathmoveto{\pgfqpoint{7.959246in}{1.947240in}}%
\pgfpathlineto{\pgfqpoint{7.643040in}{2.115869in}}%
\pgfpathlineto{\pgfqpoint{7.856229in}{2.125260in}}%
\pgfpathlineto{\pgfqpoint{7.959246in}{1.947240in}}%
\pgfpathclose%
\pgfusepath{fill}%
\end{pgfscope}%
\begin{pgfscope}%
\pgfpathrectangle{\pgfqpoint{6.818937in}{0.147348in}}{\pgfqpoint{2.735294in}{2.735294in}}%
\pgfusepath{clip}%
\pgfsetbuttcap%
\pgfsetroundjoin%
\definecolor{currentfill}{rgb}{0.042669,0.163794,0.247755}%
\pgfsetfillcolor{currentfill}%
\pgfsetlinewidth{0.000000pt}%
\definecolor{currentstroke}{rgb}{0.000000,0.000000,0.000000}%
\pgfsetstrokecolor{currentstroke}%
\pgfsetdash{}{0pt}%
\pgfpathmoveto{\pgfqpoint{8.456300in}{0.930196in}}%
\pgfpathlineto{\pgfqpoint{8.362938in}{1.338584in}}%
\pgfpathlineto{\pgfqpoint{8.591651in}{1.053752in}}%
\pgfpathlineto{\pgfqpoint{8.456300in}{0.930196in}}%
\pgfpathclose%
\pgfusepath{fill}%
\end{pgfscope}%
\begin{pgfscope}%
\pgfpathrectangle{\pgfqpoint{6.818937in}{0.147348in}}{\pgfqpoint{2.735294in}{2.735294in}}%
\pgfusepath{clip}%
\pgfsetbuttcap%
\pgfsetroundjoin%
\definecolor{currentfill}{rgb}{0.042669,0.163794,0.247755}%
\pgfsetfillcolor{currentfill}%
\pgfsetlinewidth{0.000000pt}%
\definecolor{currentstroke}{rgb}{0.000000,0.000000,0.000000}%
\pgfsetstrokecolor{currentstroke}%
\pgfsetdash{}{0pt}%
\pgfpathmoveto{\pgfqpoint{7.855445in}{1.053752in}}%
\pgfpathlineto{\pgfqpoint{8.084158in}{1.338584in}}%
\pgfpathlineto{\pgfqpoint{7.990796in}{0.930196in}}%
\pgfpathlineto{\pgfqpoint{7.855445in}{1.053752in}}%
\pgfpathclose%
\pgfusepath{fill}%
\end{pgfscope}%
\begin{pgfscope}%
\pgfpathrectangle{\pgfqpoint{6.818937in}{0.147348in}}{\pgfqpoint{2.735294in}{2.735294in}}%
\pgfusepath{clip}%
\pgfsetbuttcap%
\pgfsetroundjoin%
\definecolor{currentfill}{rgb}{0.064954,0.249341,0.377155}%
\pgfsetfillcolor{currentfill}%
\pgfsetlinewidth{0.000000pt}%
\definecolor{currentstroke}{rgb}{0.000000,0.000000,0.000000}%
\pgfsetstrokecolor{currentstroke}%
\pgfsetdash{}{0pt}%
\pgfpathmoveto{\pgfqpoint{8.752556in}{1.562133in}}%
\pgfpathlineto{\pgfqpoint{8.804056in}{2.115869in}}%
\pgfpathlineto{\pgfqpoint{8.958969in}{1.590078in}}%
\pgfpathlineto{\pgfqpoint{8.752556in}{1.562133in}}%
\pgfpathclose%
\pgfusepath{fill}%
\end{pgfscope}%
\begin{pgfscope}%
\pgfpathrectangle{\pgfqpoint{6.818937in}{0.147348in}}{\pgfqpoint{2.735294in}{2.735294in}}%
\pgfusepath{clip}%
\pgfsetbuttcap%
\pgfsetroundjoin%
\definecolor{currentfill}{rgb}{0.064954,0.249341,0.377155}%
\pgfsetfillcolor{currentfill}%
\pgfsetlinewidth{0.000000pt}%
\definecolor{currentstroke}{rgb}{0.000000,0.000000,0.000000}%
\pgfsetstrokecolor{currentstroke}%
\pgfsetdash{}{0pt}%
\pgfpathmoveto{\pgfqpoint{7.488127in}{1.590078in}}%
\pgfpathlineto{\pgfqpoint{7.643040in}{2.115869in}}%
\pgfpathlineto{\pgfqpoint{7.694540in}{1.562133in}}%
\pgfpathlineto{\pgfqpoint{7.488127in}{1.590078in}}%
\pgfpathclose%
\pgfusepath{fill}%
\end{pgfscope}%
\begin{pgfscope}%
\pgfpathrectangle{\pgfqpoint{6.818937in}{0.147348in}}{\pgfqpoint{2.735294in}{2.735294in}}%
\pgfusepath{clip}%
\pgfsetbuttcap%
\pgfsetroundjoin%
\definecolor{currentfill}{rgb}{0.078663,0.301965,0.456754}%
\pgfsetfillcolor{currentfill}%
\pgfsetlinewidth{0.000000pt}%
\definecolor{currentstroke}{rgb}{0.000000,0.000000,0.000000}%
\pgfsetstrokecolor{currentstroke}%
\pgfsetdash{}{0pt}%
\pgfpathmoveto{\pgfqpoint{7.856229in}{2.125260in}}%
\pgfpathlineto{\pgfqpoint{7.775860in}{2.272632in}}%
\pgfpathlineto{\pgfqpoint{8.097511in}{2.130589in}}%
\pgfpathlineto{\pgfqpoint{7.856229in}{2.125260in}}%
\pgfpathclose%
\pgfusepath{fill}%
\end{pgfscope}%
\begin{pgfscope}%
\pgfpathrectangle{\pgfqpoint{6.818937in}{0.147348in}}{\pgfqpoint{2.735294in}{2.735294in}}%
\pgfusepath{clip}%
\pgfsetbuttcap%
\pgfsetroundjoin%
\definecolor{currentfill}{rgb}{0.078663,0.301965,0.456754}%
\pgfsetfillcolor{currentfill}%
\pgfsetlinewidth{0.000000pt}%
\definecolor{currentstroke}{rgb}{0.000000,0.000000,0.000000}%
\pgfsetstrokecolor{currentstroke}%
\pgfsetdash{}{0pt}%
\pgfpathmoveto{\pgfqpoint{8.349584in}{2.130589in}}%
\pgfpathlineto{\pgfqpoint{8.671236in}{2.272632in}}%
\pgfpathlineto{\pgfqpoint{8.590867in}{2.125260in}}%
\pgfpathlineto{\pgfqpoint{8.349584in}{2.130589in}}%
\pgfpathclose%
\pgfusepath{fill}%
\end{pgfscope}%
\begin{pgfscope}%
\pgfpathrectangle{\pgfqpoint{6.818937in}{0.147348in}}{\pgfqpoint{2.735294in}{2.735294in}}%
\pgfusepath{clip}%
\pgfsetbuttcap%
\pgfsetroundjoin%
\definecolor{currentfill}{rgb}{0.049941,0.191710,0.289982}%
\pgfsetfillcolor{currentfill}%
\pgfsetlinewidth{0.000000pt}%
\definecolor{currentstroke}{rgb}{0.000000,0.000000,0.000000}%
\pgfsetstrokecolor{currentstroke}%
\pgfsetdash{}{0pt}%
\pgfpathmoveto{\pgfqpoint{7.855445in}{1.053752in}}%
\pgfpathlineto{\pgfqpoint{7.774694in}{0.966453in}}%
\pgfpathlineto{\pgfqpoint{7.819927in}{1.359642in}}%
\pgfpathlineto{\pgfqpoint{7.855445in}{1.053752in}}%
\pgfpathclose%
\pgfusepath{fill}%
\end{pgfscope}%
\begin{pgfscope}%
\pgfpathrectangle{\pgfqpoint{6.818937in}{0.147348in}}{\pgfqpoint{2.735294in}{2.735294in}}%
\pgfusepath{clip}%
\pgfsetbuttcap%
\pgfsetroundjoin%
\definecolor{currentfill}{rgb}{0.049941,0.191710,0.289982}%
\pgfsetfillcolor{currentfill}%
\pgfsetlinewidth{0.000000pt}%
\definecolor{currentstroke}{rgb}{0.000000,0.000000,0.000000}%
\pgfsetstrokecolor{currentstroke}%
\pgfsetdash{}{0pt}%
\pgfpathmoveto{\pgfqpoint{8.627169in}{1.359642in}}%
\pgfpathlineto{\pgfqpoint{8.672401in}{0.966453in}}%
\pgfpathlineto{\pgfqpoint{8.591651in}{1.053752in}}%
\pgfpathlineto{\pgfqpoint{8.627169in}{1.359642in}}%
\pgfpathclose%
\pgfusepath{fill}%
\end{pgfscope}%
\begin{pgfscope}%
\pgfpathrectangle{\pgfqpoint{6.818937in}{0.147348in}}{\pgfqpoint{2.735294in}{2.735294in}}%
\pgfusepath{clip}%
\pgfsetbuttcap%
\pgfsetroundjoin%
\definecolor{currentfill}{rgb}{0.050011,0.191979,0.290388}%
\pgfsetfillcolor{currentfill}%
\pgfsetlinewidth{0.000000pt}%
\definecolor{currentstroke}{rgb}{0.000000,0.000000,0.000000}%
\pgfsetstrokecolor{currentstroke}%
\pgfsetdash{}{0pt}%
\pgfpathmoveto{\pgfqpoint{8.958969in}{1.590078in}}%
\pgfpathlineto{\pgfqpoint{8.982643in}{1.157190in}}%
\pgfpathlineto{\pgfqpoint{8.752556in}{1.562133in}}%
\pgfpathlineto{\pgfqpoint{8.958969in}{1.590078in}}%
\pgfpathclose%
\pgfusepath{fill}%
\end{pgfscope}%
\begin{pgfscope}%
\pgfpathrectangle{\pgfqpoint{6.818937in}{0.147348in}}{\pgfqpoint{2.735294in}{2.735294in}}%
\pgfusepath{clip}%
\pgfsetbuttcap%
\pgfsetroundjoin%
\definecolor{currentfill}{rgb}{0.050011,0.191979,0.290388}%
\pgfsetfillcolor{currentfill}%
\pgfsetlinewidth{0.000000pt}%
\definecolor{currentstroke}{rgb}{0.000000,0.000000,0.000000}%
\pgfsetstrokecolor{currentstroke}%
\pgfsetdash{}{0pt}%
\pgfpathmoveto{\pgfqpoint{7.694540in}{1.562133in}}%
\pgfpathlineto{\pgfqpoint{7.464453in}{1.157190in}}%
\pgfpathlineto{\pgfqpoint{7.488127in}{1.590078in}}%
\pgfpathlineto{\pgfqpoint{7.694540in}{1.562133in}}%
\pgfpathclose%
\pgfusepath{fill}%
\end{pgfscope}%
\begin{pgfscope}%
\pgfpathrectangle{\pgfqpoint{6.818937in}{0.147348in}}{\pgfqpoint{2.735294in}{2.735294in}}%
\pgfusepath{clip}%
\pgfsetbuttcap%
\pgfsetroundjoin%
\definecolor{currentfill}{rgb}{0.043508,0.167016,0.252629}%
\pgfsetfillcolor{currentfill}%
\pgfsetlinewidth{0.000000pt}%
\definecolor{currentstroke}{rgb}{0.000000,0.000000,0.000000}%
\pgfsetstrokecolor{currentstroke}%
\pgfsetdash{}{0pt}%
\pgfpathmoveto{\pgfqpoint{8.805232in}{1.099921in}}%
\pgfpathlineto{\pgfqpoint{8.672401in}{0.966453in}}%
\pgfpathlineto{\pgfqpoint{8.627169in}{1.359642in}}%
\pgfpathlineto{\pgfqpoint{8.805232in}{1.099921in}}%
\pgfpathclose%
\pgfusepath{fill}%
\end{pgfscope}%
\begin{pgfscope}%
\pgfpathrectangle{\pgfqpoint{6.818937in}{0.147348in}}{\pgfqpoint{2.735294in}{2.735294in}}%
\pgfusepath{clip}%
\pgfsetbuttcap%
\pgfsetroundjoin%
\definecolor{currentfill}{rgb}{0.043508,0.167016,0.252629}%
\pgfsetfillcolor{currentfill}%
\pgfsetlinewidth{0.000000pt}%
\definecolor{currentstroke}{rgb}{0.000000,0.000000,0.000000}%
\pgfsetstrokecolor{currentstroke}%
\pgfsetdash{}{0pt}%
\pgfpathmoveto{\pgfqpoint{7.819927in}{1.359642in}}%
\pgfpathlineto{\pgfqpoint{7.774694in}{0.966453in}}%
\pgfpathlineto{\pgfqpoint{7.641864in}{1.099921in}}%
\pgfpathlineto{\pgfqpoint{7.819927in}{1.359642in}}%
\pgfpathclose%
\pgfusepath{fill}%
\end{pgfscope}%
\begin{pgfscope}%
\pgfpathrectangle{\pgfqpoint{6.818937in}{0.147348in}}{\pgfqpoint{2.735294in}{2.735294in}}%
\pgfusepath{clip}%
\pgfsetbuttcap%
\pgfsetroundjoin%
\definecolor{currentfill}{rgb}{0.062760,0.240916,0.364410}%
\pgfsetfillcolor{currentfill}%
\pgfsetlinewidth{0.000000pt}%
\definecolor{currentstroke}{rgb}{0.000000,0.000000,0.000000}%
\pgfsetstrokecolor{currentstroke}%
\pgfsetdash{}{0pt}%
\pgfpathmoveto{\pgfqpoint{8.958969in}{1.590078in}}%
\pgfpathlineto{\pgfqpoint{8.804056in}{2.115869in}}%
\pgfpathlineto{\pgfqpoint{9.036807in}{1.777676in}}%
\pgfpathlineto{\pgfqpoint{8.958969in}{1.590078in}}%
\pgfpathclose%
\pgfusepath{fill}%
\end{pgfscope}%
\begin{pgfscope}%
\pgfpathrectangle{\pgfqpoint{6.818937in}{0.147348in}}{\pgfqpoint{2.735294in}{2.735294in}}%
\pgfusepath{clip}%
\pgfsetbuttcap%
\pgfsetroundjoin%
\definecolor{currentfill}{rgb}{0.062760,0.240916,0.364410}%
\pgfsetfillcolor{currentfill}%
\pgfsetlinewidth{0.000000pt}%
\definecolor{currentstroke}{rgb}{0.000000,0.000000,0.000000}%
\pgfsetstrokecolor{currentstroke}%
\pgfsetdash{}{0pt}%
\pgfpathmoveto{\pgfqpoint{7.410289in}{1.777676in}}%
\pgfpathlineto{\pgfqpoint{7.643040in}{2.115869in}}%
\pgfpathlineto{\pgfqpoint{7.488127in}{1.590078in}}%
\pgfpathlineto{\pgfqpoint{7.410289in}{1.777676in}}%
\pgfpathclose%
\pgfusepath{fill}%
\end{pgfscope}%
\begin{pgfscope}%
\pgfpathrectangle{\pgfqpoint{6.818937in}{0.147348in}}{\pgfqpoint{2.735294in}{2.735294in}}%
\pgfusepath{clip}%
\pgfsetbuttcap%
\pgfsetroundjoin%
\definecolor{currentfill}{rgb}{0.075436,0.289576,0.438014}%
\pgfsetfillcolor{currentfill}%
\pgfsetlinewidth{0.000000pt}%
\definecolor{currentstroke}{rgb}{0.000000,0.000000,0.000000}%
\pgfsetstrokecolor{currentstroke}%
\pgfsetdash{}{0pt}%
\pgfpathmoveto{\pgfqpoint{8.590867in}{2.125260in}}%
\pgfpathlineto{\pgfqpoint{8.671236in}{2.272632in}}%
\pgfpathlineto{\pgfqpoint{8.804056in}{2.115869in}}%
\pgfpathlineto{\pgfqpoint{8.590867in}{2.125260in}}%
\pgfpathclose%
\pgfusepath{fill}%
\end{pgfscope}%
\begin{pgfscope}%
\pgfpathrectangle{\pgfqpoint{6.818937in}{0.147348in}}{\pgfqpoint{2.735294in}{2.735294in}}%
\pgfusepath{clip}%
\pgfsetbuttcap%
\pgfsetroundjoin%
\definecolor{currentfill}{rgb}{0.075436,0.289576,0.438014}%
\pgfsetfillcolor{currentfill}%
\pgfsetlinewidth{0.000000pt}%
\definecolor{currentstroke}{rgb}{0.000000,0.000000,0.000000}%
\pgfsetstrokecolor{currentstroke}%
\pgfsetdash{}{0pt}%
\pgfpathmoveto{\pgfqpoint{7.643040in}{2.115869in}}%
\pgfpathlineto{\pgfqpoint{7.775860in}{2.272632in}}%
\pgfpathlineto{\pgfqpoint{7.856229in}{2.125260in}}%
\pgfpathlineto{\pgfqpoint{7.643040in}{2.115869in}}%
\pgfpathclose%
\pgfusepath{fill}%
\end{pgfscope}%
\begin{pgfscope}%
\pgfpathrectangle{\pgfqpoint{6.818937in}{0.147348in}}{\pgfqpoint{2.735294in}{2.735294in}}%
\pgfusepath{clip}%
\pgfsetbuttcap%
\pgfsetroundjoin%
\definecolor{currentfill}{rgb}{0.039595,0.151995,0.229908}%
\pgfsetfillcolor{currentfill}%
\pgfsetlinewidth{0.000000pt}%
\definecolor{currentstroke}{rgb}{0.000000,0.000000,0.000000}%
\pgfsetstrokecolor{currentstroke}%
\pgfsetdash{}{0pt}%
\pgfpathmoveto{\pgfqpoint{8.223548in}{0.917160in}}%
\pgfpathlineto{\pgfqpoint{8.349861in}{1.027548in}}%
\pgfpathlineto{\pgfqpoint{8.456300in}{0.930196in}}%
\pgfpathlineto{\pgfqpoint{8.223548in}{0.917160in}}%
\pgfpathclose%
\pgfusepath{fill}%
\end{pgfscope}%
\begin{pgfscope}%
\pgfpathrectangle{\pgfqpoint{6.818937in}{0.147348in}}{\pgfqpoint{2.735294in}{2.735294in}}%
\pgfusepath{clip}%
\pgfsetbuttcap%
\pgfsetroundjoin%
\definecolor{currentfill}{rgb}{0.039595,0.151995,0.229908}%
\pgfsetfillcolor{currentfill}%
\pgfsetlinewidth{0.000000pt}%
\definecolor{currentstroke}{rgb}{0.000000,0.000000,0.000000}%
\pgfsetstrokecolor{currentstroke}%
\pgfsetdash{}{0pt}%
\pgfpathmoveto{\pgfqpoint{7.990796in}{0.930196in}}%
\pgfpathlineto{\pgfqpoint{8.097234in}{1.027548in}}%
\pgfpathlineto{\pgfqpoint{8.223548in}{0.917160in}}%
\pgfpathlineto{\pgfqpoint{7.990796in}{0.930196in}}%
\pgfpathclose%
\pgfusepath{fill}%
\end{pgfscope}%
\begin{pgfscope}%
\pgfpathrectangle{\pgfqpoint{6.818937in}{0.147348in}}{\pgfqpoint{2.735294in}{2.735294in}}%
\pgfusepath{clip}%
\pgfsetbuttcap%
\pgfsetroundjoin%
\definecolor{currentfill}{rgb}{0.060942,0.233938,0.353856}%
\pgfsetfillcolor{currentfill}%
\pgfsetlinewidth{0.000000pt}%
\definecolor{currentstroke}{rgb}{0.000000,0.000000,0.000000}%
\pgfsetstrokecolor{currentstroke}%
\pgfsetdash{}{0pt}%
\pgfpathmoveto{\pgfqpoint{7.327008in}{1.620392in}}%
\pgfpathlineto{\pgfqpoint{7.410289in}{1.777676in}}%
\pgfpathlineto{\pgfqpoint{7.488127in}{1.590078in}}%
\pgfpathlineto{\pgfqpoint{7.327008in}{1.620392in}}%
\pgfpathclose%
\pgfusepath{fill}%
\end{pgfscope}%
\begin{pgfscope}%
\pgfpathrectangle{\pgfqpoint{6.818937in}{0.147348in}}{\pgfqpoint{2.735294in}{2.735294in}}%
\pgfusepath{clip}%
\pgfsetbuttcap%
\pgfsetroundjoin%
\definecolor{currentfill}{rgb}{0.060942,0.233938,0.353856}%
\pgfsetfillcolor{currentfill}%
\pgfsetlinewidth{0.000000pt}%
\definecolor{currentstroke}{rgb}{0.000000,0.000000,0.000000}%
\pgfsetstrokecolor{currentstroke}%
\pgfsetdash{}{0pt}%
\pgfpathmoveto{\pgfqpoint{8.958969in}{1.590078in}}%
\pgfpathlineto{\pgfqpoint{9.036807in}{1.777676in}}%
\pgfpathlineto{\pgfqpoint{9.120088in}{1.620392in}}%
\pgfpathlineto{\pgfqpoint{8.958969in}{1.590078in}}%
\pgfpathclose%
\pgfusepath{fill}%
\end{pgfscope}%
\begin{pgfscope}%
\pgfpathrectangle{\pgfqpoint{6.818937in}{0.147348in}}{\pgfqpoint{2.735294in}{2.735294in}}%
\pgfusepath{clip}%
\pgfsetbuttcap%
\pgfsetroundjoin%
\definecolor{currentfill}{rgb}{0.063981,0.245604,0.371502}%
\pgfsetfillcolor{currentfill}%
\pgfsetlinewidth{0.000000pt}%
\definecolor{currentstroke}{rgb}{0.000000,0.000000,0.000000}%
\pgfsetstrokecolor{currentstroke}%
\pgfsetdash{}{0pt}%
\pgfpathmoveto{\pgfqpoint{7.641864in}{1.099921in}}%
\pgfpathlineto{\pgfqpoint{7.586426in}{1.018945in}}%
\pgfpathlineto{\pgfqpoint{7.694540in}{1.562133in}}%
\pgfpathlineto{\pgfqpoint{7.641864in}{1.099921in}}%
\pgfpathclose%
\pgfusepath{fill}%
\end{pgfscope}%
\begin{pgfscope}%
\pgfpathrectangle{\pgfqpoint{6.818937in}{0.147348in}}{\pgfqpoint{2.735294in}{2.735294in}}%
\pgfusepath{clip}%
\pgfsetbuttcap%
\pgfsetroundjoin%
\definecolor{currentfill}{rgb}{0.063981,0.245604,0.371502}%
\pgfsetfillcolor{currentfill}%
\pgfsetlinewidth{0.000000pt}%
\definecolor{currentstroke}{rgb}{0.000000,0.000000,0.000000}%
\pgfsetstrokecolor{currentstroke}%
\pgfsetdash{}{0pt}%
\pgfpathmoveto{\pgfqpoint{8.752556in}{1.562133in}}%
\pgfpathlineto{\pgfqpoint{8.860669in}{1.018945in}}%
\pgfpathlineto{\pgfqpoint{8.805232in}{1.099921in}}%
\pgfpathlineto{\pgfqpoint{8.752556in}{1.562133in}}%
\pgfpathclose%
\pgfusepath{fill}%
\end{pgfscope}%
\begin{pgfscope}%
\pgfpathrectangle{\pgfqpoint{6.818937in}{0.147348in}}{\pgfqpoint{2.735294in}{2.735294in}}%
\pgfusepath{clip}%
\pgfsetbuttcap%
\pgfsetroundjoin%
\definecolor{currentfill}{rgb}{0.081954,0.314596,0.475860}%
\pgfsetfillcolor{currentfill}%
\pgfsetlinewidth{0.000000pt}%
\definecolor{currentstroke}{rgb}{0.000000,0.000000,0.000000}%
\pgfsetstrokecolor{currentstroke}%
\pgfsetdash{}{0pt}%
\pgfpathmoveto{\pgfqpoint{8.349584in}{2.130589in}}%
\pgfpathlineto{\pgfqpoint{8.097511in}{2.130589in}}%
\pgfpathlineto{\pgfqpoint{8.131822in}{2.613951in}}%
\pgfpathlineto{\pgfqpoint{8.349584in}{2.130589in}}%
\pgfpathclose%
\pgfusepath{fill}%
\end{pgfscope}%
\begin{pgfscope}%
\pgfpathrectangle{\pgfqpoint{6.818937in}{0.147348in}}{\pgfqpoint{2.735294in}{2.735294in}}%
\pgfusepath{clip}%
\pgfsetbuttcap%
\pgfsetroundjoin%
\definecolor{currentfill}{rgb}{0.040669,0.156116,0.236142}%
\pgfsetfillcolor{currentfill}%
\pgfsetlinewidth{0.000000pt}%
\definecolor{currentstroke}{rgb}{0.000000,0.000000,0.000000}%
\pgfsetstrokecolor{currentstroke}%
\pgfsetdash{}{0pt}%
\pgfpathmoveto{\pgfqpoint{8.591651in}{1.053752in}}%
\pgfpathlineto{\pgfqpoint{8.672401in}{0.966453in}}%
\pgfpathlineto{\pgfqpoint{8.456300in}{0.930196in}}%
\pgfpathlineto{\pgfqpoint{8.591651in}{1.053752in}}%
\pgfpathclose%
\pgfusepath{fill}%
\end{pgfscope}%
\begin{pgfscope}%
\pgfpathrectangle{\pgfqpoint{6.818937in}{0.147348in}}{\pgfqpoint{2.735294in}{2.735294in}}%
\pgfusepath{clip}%
\pgfsetbuttcap%
\pgfsetroundjoin%
\definecolor{currentfill}{rgb}{0.040669,0.156116,0.236142}%
\pgfsetfillcolor{currentfill}%
\pgfsetlinewidth{0.000000pt}%
\definecolor{currentstroke}{rgb}{0.000000,0.000000,0.000000}%
\pgfsetstrokecolor{currentstroke}%
\pgfsetdash{}{0pt}%
\pgfpathmoveto{\pgfqpoint{7.990796in}{0.930196in}}%
\pgfpathlineto{\pgfqpoint{7.774694in}{0.966453in}}%
\pgfpathlineto{\pgfqpoint{7.855445in}{1.053752in}}%
\pgfpathlineto{\pgfqpoint{7.990796in}{0.930196in}}%
\pgfpathclose%
\pgfusepath{fill}%
\end{pgfscope}%
\begin{pgfscope}%
\pgfpathrectangle{\pgfqpoint{6.818937in}{0.147348in}}{\pgfqpoint{2.735294in}{2.735294in}}%
\pgfusepath{clip}%
\pgfsetbuttcap%
\pgfsetroundjoin%
\definecolor{currentfill}{rgb}{0.045702,0.175435,0.265364}%
\pgfsetfillcolor{currentfill}%
\pgfsetlinewidth{0.000000pt}%
\definecolor{currentstroke}{rgb}{0.000000,0.000000,0.000000}%
\pgfsetstrokecolor{currentstroke}%
\pgfsetdash{}{0pt}%
\pgfpathmoveto{\pgfqpoint{8.982643in}{1.157190in}}%
\pgfpathlineto{\pgfqpoint{8.860669in}{1.018945in}}%
\pgfpathlineto{\pgfqpoint{8.752556in}{1.562133in}}%
\pgfpathlineto{\pgfqpoint{8.982643in}{1.157190in}}%
\pgfpathclose%
\pgfusepath{fill}%
\end{pgfscope}%
\begin{pgfscope}%
\pgfpathrectangle{\pgfqpoint{6.818937in}{0.147348in}}{\pgfqpoint{2.735294in}{2.735294in}}%
\pgfusepath{clip}%
\pgfsetbuttcap%
\pgfsetroundjoin%
\definecolor{currentfill}{rgb}{0.045702,0.175435,0.265364}%
\pgfsetfillcolor{currentfill}%
\pgfsetlinewidth{0.000000pt}%
\definecolor{currentstroke}{rgb}{0.000000,0.000000,0.000000}%
\pgfsetstrokecolor{currentstroke}%
\pgfsetdash{}{0pt}%
\pgfpathmoveto{\pgfqpoint{7.694540in}{1.562133in}}%
\pgfpathlineto{\pgfqpoint{7.586426in}{1.018945in}}%
\pgfpathlineto{\pgfqpoint{7.464453in}{1.157190in}}%
\pgfpathlineto{\pgfqpoint{7.694540in}{1.562133in}}%
\pgfpathclose%
\pgfusepath{fill}%
\end{pgfscope}%
\begin{pgfscope}%
\pgfpathrectangle{\pgfqpoint{6.818937in}{0.147348in}}{\pgfqpoint{2.735294in}{2.735294in}}%
\pgfusepath{clip}%
\pgfsetbuttcap%
\pgfsetroundjoin%
\definecolor{currentfill}{rgb}{0.082280,0.315849,0.477755}%
\pgfsetfillcolor{currentfill}%
\pgfsetlinewidth{0.000000pt}%
\definecolor{currentstroke}{rgb}{0.000000,0.000000,0.000000}%
\pgfsetstrokecolor{currentstroke}%
\pgfsetdash{}{0pt}%
\pgfpathmoveto{\pgfqpoint{8.349584in}{2.130589in}}%
\pgfpathlineto{\pgfqpoint{8.315273in}{2.613951in}}%
\pgfpathlineto{\pgfqpoint{8.671236in}{2.272632in}}%
\pgfpathlineto{\pgfqpoint{8.349584in}{2.130589in}}%
\pgfpathclose%
\pgfusepath{fill}%
\end{pgfscope}%
\begin{pgfscope}%
\pgfpathrectangle{\pgfqpoint{6.818937in}{0.147348in}}{\pgfqpoint{2.735294in}{2.735294in}}%
\pgfusepath{clip}%
\pgfsetbuttcap%
\pgfsetroundjoin%
\definecolor{currentfill}{rgb}{0.082280,0.315849,0.477755}%
\pgfsetfillcolor{currentfill}%
\pgfsetlinewidth{0.000000pt}%
\definecolor{currentstroke}{rgb}{0.000000,0.000000,0.000000}%
\pgfsetstrokecolor{currentstroke}%
\pgfsetdash{}{0pt}%
\pgfpathmoveto{\pgfqpoint{7.775860in}{2.272632in}}%
\pgfpathlineto{\pgfqpoint{8.131822in}{2.613951in}}%
\pgfpathlineto{\pgfqpoint{8.097511in}{2.130589in}}%
\pgfpathlineto{\pgfqpoint{7.775860in}{2.272632in}}%
\pgfpathclose%
\pgfusepath{fill}%
\end{pgfscope}%
\begin{pgfscope}%
\pgfpathrectangle{\pgfqpoint{6.818937in}{0.147348in}}{\pgfqpoint{2.735294in}{2.735294in}}%
\pgfusepath{clip}%
\pgfsetbuttcap%
\pgfsetroundjoin%
\definecolor{currentfill}{rgb}{0.052493,0.201505,0.304798}%
\pgfsetfillcolor{currentfill}%
\pgfsetlinewidth{0.000000pt}%
\definecolor{currentstroke}{rgb}{0.000000,0.000000,0.000000}%
\pgfsetstrokecolor{currentstroke}%
\pgfsetdash{}{0pt}%
\pgfpathmoveto{\pgfqpoint{9.120088in}{1.620392in}}%
\pgfpathlineto{\pgfqpoint{9.124324in}{1.217806in}}%
\pgfpathlineto{\pgfqpoint{8.958969in}{1.590078in}}%
\pgfpathlineto{\pgfqpoint{9.120088in}{1.620392in}}%
\pgfpathclose%
\pgfusepath{fill}%
\end{pgfscope}%
\begin{pgfscope}%
\pgfpathrectangle{\pgfqpoint{6.818937in}{0.147348in}}{\pgfqpoint{2.735294in}{2.735294in}}%
\pgfusepath{clip}%
\pgfsetbuttcap%
\pgfsetroundjoin%
\definecolor{currentfill}{rgb}{0.052493,0.201505,0.304798}%
\pgfsetfillcolor{currentfill}%
\pgfsetlinewidth{0.000000pt}%
\definecolor{currentstroke}{rgb}{0.000000,0.000000,0.000000}%
\pgfsetstrokecolor{currentstroke}%
\pgfsetdash{}{0pt}%
\pgfpathmoveto{\pgfqpoint{7.488127in}{1.590078in}}%
\pgfpathlineto{\pgfqpoint{7.322771in}{1.217806in}}%
\pgfpathlineto{\pgfqpoint{7.327008in}{1.620392in}}%
\pgfpathlineto{\pgfqpoint{7.488127in}{1.590078in}}%
\pgfpathclose%
\pgfusepath{fill}%
\end{pgfscope}%
\begin{pgfscope}%
\pgfpathrectangle{\pgfqpoint{6.818937in}{0.147348in}}{\pgfqpoint{2.735294in}{2.735294in}}%
\pgfusepath{clip}%
\pgfsetbuttcap%
\pgfsetroundjoin%
\definecolor{currentfill}{rgb}{0.042579,0.163449,0.247234}%
\pgfsetfillcolor{currentfill}%
\pgfsetlinewidth{0.000000pt}%
\definecolor{currentstroke}{rgb}{0.000000,0.000000,0.000000}%
\pgfsetstrokecolor{currentstroke}%
\pgfsetdash{}{0pt}%
\pgfpathmoveto{\pgfqpoint{8.860669in}{1.018945in}}%
\pgfpathlineto{\pgfqpoint{8.672401in}{0.966453in}}%
\pgfpathlineto{\pgfqpoint{8.805232in}{1.099921in}}%
\pgfpathlineto{\pgfqpoint{8.860669in}{1.018945in}}%
\pgfpathclose%
\pgfusepath{fill}%
\end{pgfscope}%
\begin{pgfscope}%
\pgfpathrectangle{\pgfqpoint{6.818937in}{0.147348in}}{\pgfqpoint{2.735294in}{2.735294in}}%
\pgfusepath{clip}%
\pgfsetbuttcap%
\pgfsetroundjoin%
\definecolor{currentfill}{rgb}{0.042579,0.163449,0.247234}%
\pgfsetfillcolor{currentfill}%
\pgfsetlinewidth{0.000000pt}%
\definecolor{currentstroke}{rgb}{0.000000,0.000000,0.000000}%
\pgfsetstrokecolor{currentstroke}%
\pgfsetdash{}{0pt}%
\pgfpathmoveto{\pgfqpoint{7.641864in}{1.099921in}}%
\pgfpathlineto{\pgfqpoint{7.774694in}{0.966453in}}%
\pgfpathlineto{\pgfqpoint{7.586426in}{1.018945in}}%
\pgfpathlineto{\pgfqpoint{7.641864in}{1.099921in}}%
\pgfpathclose%
\pgfusepath{fill}%
\end{pgfscope}%
\begin{pgfscope}%
\pgfpathrectangle{\pgfqpoint{6.818937in}{0.147348in}}{\pgfqpoint{2.735294in}{2.735294in}}%
\pgfusepath{clip}%
\pgfsetbuttcap%
\pgfsetroundjoin%
\definecolor{currentfill}{rgb}{0.067179,0.257880,0.390071}%
\pgfsetfillcolor{currentfill}%
\pgfsetlinewidth{0.000000pt}%
\definecolor{currentstroke}{rgb}{0.000000,0.000000,0.000000}%
\pgfsetstrokecolor{currentstroke}%
\pgfsetdash{}{0pt}%
\pgfpathmoveto{\pgfqpoint{8.958969in}{1.590078in}}%
\pgfpathlineto{\pgfqpoint{9.017173in}{1.079776in}}%
\pgfpathlineto{\pgfqpoint{8.982643in}{1.157190in}}%
\pgfpathlineto{\pgfqpoint{8.958969in}{1.590078in}}%
\pgfpathclose%
\pgfusepath{fill}%
\end{pgfscope}%
\begin{pgfscope}%
\pgfpathrectangle{\pgfqpoint{6.818937in}{0.147348in}}{\pgfqpoint{2.735294in}{2.735294in}}%
\pgfusepath{clip}%
\pgfsetbuttcap%
\pgfsetroundjoin%
\definecolor{currentfill}{rgb}{0.067179,0.257880,0.390071}%
\pgfsetfillcolor{currentfill}%
\pgfsetlinewidth{0.000000pt}%
\definecolor{currentstroke}{rgb}{0.000000,0.000000,0.000000}%
\pgfsetstrokecolor{currentstroke}%
\pgfsetdash{}{0pt}%
\pgfpathmoveto{\pgfqpoint{7.464453in}{1.157190in}}%
\pgfpathlineto{\pgfqpoint{7.429923in}{1.079776in}}%
\pgfpathlineto{\pgfqpoint{7.488127in}{1.590078in}}%
\pgfpathlineto{\pgfqpoint{7.464453in}{1.157190in}}%
\pgfpathclose%
\pgfusepath{fill}%
\end{pgfscope}%
\begin{pgfscope}%
\pgfpathrectangle{\pgfqpoint{6.818937in}{0.147348in}}{\pgfqpoint{2.735294in}{2.735294in}}%
\pgfusepath{clip}%
\pgfsetbuttcap%
\pgfsetroundjoin%
\definecolor{currentfill}{rgb}{0.047247,0.181368,0.274339}%
\pgfsetfillcolor{currentfill}%
\pgfsetlinewidth{0.000000pt}%
\definecolor{currentstroke}{rgb}{0.000000,0.000000,0.000000}%
\pgfsetstrokecolor{currentstroke}%
\pgfsetdash{}{0pt}%
\pgfpathmoveto{\pgfqpoint{9.124324in}{1.217806in}}%
\pgfpathlineto{\pgfqpoint{9.017173in}{1.079776in}}%
\pgfpathlineto{\pgfqpoint{8.958969in}{1.590078in}}%
\pgfpathlineto{\pgfqpoint{9.124324in}{1.217806in}}%
\pgfpathclose%
\pgfusepath{fill}%
\end{pgfscope}%
\begin{pgfscope}%
\pgfpathrectangle{\pgfqpoint{6.818937in}{0.147348in}}{\pgfqpoint{2.735294in}{2.735294in}}%
\pgfusepath{clip}%
\pgfsetbuttcap%
\pgfsetroundjoin%
\definecolor{currentfill}{rgb}{0.047247,0.181368,0.274339}%
\pgfsetfillcolor{currentfill}%
\pgfsetlinewidth{0.000000pt}%
\definecolor{currentstroke}{rgb}{0.000000,0.000000,0.000000}%
\pgfsetstrokecolor{currentstroke}%
\pgfsetdash{}{0pt}%
\pgfpathmoveto{\pgfqpoint{7.488127in}{1.590078in}}%
\pgfpathlineto{\pgfqpoint{7.429923in}{1.079776in}}%
\pgfpathlineto{\pgfqpoint{7.322771in}{1.217806in}}%
\pgfpathlineto{\pgfqpoint{7.488127in}{1.590078in}}%
\pgfpathclose%
\pgfusepath{fill}%
\end{pgfscope}%
\begin{pgfscope}%
\pgfpathrectangle{\pgfqpoint{6.818937in}{0.147348in}}{\pgfqpoint{2.735294in}{2.735294in}}%
\pgfusepath{clip}%
\pgfsetbuttcap%
\pgfsetroundjoin%
\definecolor{currentfill}{rgb}{0.081954,0.314596,0.475860}%
\pgfsetfillcolor{currentfill}%
\pgfsetlinewidth{0.000000pt}%
\definecolor{currentstroke}{rgb}{0.000000,0.000000,0.000000}%
\pgfsetstrokecolor{currentstroke}%
\pgfsetdash{}{0pt}%
\pgfpathmoveto{\pgfqpoint{8.131822in}{2.613951in}}%
\pgfpathlineto{\pgfqpoint{8.315273in}{2.613951in}}%
\pgfpathlineto{\pgfqpoint{8.349584in}{2.130589in}}%
\pgfpathlineto{\pgfqpoint{8.131822in}{2.613951in}}%
\pgfpathclose%
\pgfusepath{fill}%
\end{pgfscope}%
\begin{pgfscope}%
\pgfpathrectangle{\pgfqpoint{6.818937in}{0.147348in}}{\pgfqpoint{2.735294in}{2.735294in}}%
\pgfusepath{clip}%
\pgfsetbuttcap%
\pgfsetroundjoin%
\definecolor{currentfill}{rgb}{0.044978,0.172658,0.261163}%
\pgfsetfillcolor{currentfill}%
\pgfsetlinewidth{0.000000pt}%
\definecolor{currentstroke}{rgb}{0.000000,0.000000,0.000000}%
\pgfsetstrokecolor{currentstroke}%
\pgfsetdash{}{0pt}%
\pgfpathmoveto{\pgfqpoint{8.860669in}{1.018945in}}%
\pgfpathlineto{\pgfqpoint{8.982643in}{1.157190in}}%
\pgfpathlineto{\pgfqpoint{9.017173in}{1.079776in}}%
\pgfpathlineto{\pgfqpoint{8.860669in}{1.018945in}}%
\pgfpathclose%
\pgfusepath{fill}%
\end{pgfscope}%
\begin{pgfscope}%
\pgfpathrectangle{\pgfqpoint{6.818937in}{0.147348in}}{\pgfqpoint{2.735294in}{2.735294in}}%
\pgfusepath{clip}%
\pgfsetbuttcap%
\pgfsetroundjoin%
\definecolor{currentfill}{rgb}{0.044978,0.172658,0.261163}%
\pgfsetfillcolor{currentfill}%
\pgfsetlinewidth{0.000000pt}%
\definecolor{currentstroke}{rgb}{0.000000,0.000000,0.000000}%
\pgfsetstrokecolor{currentstroke}%
\pgfsetdash{}{0pt}%
\pgfpathmoveto{\pgfqpoint{7.429923in}{1.079776in}}%
\pgfpathlineto{\pgfqpoint{7.464453in}{1.157190in}}%
\pgfpathlineto{\pgfqpoint{7.586426in}{1.018945in}}%
\pgfpathlineto{\pgfqpoint{7.429923in}{1.079776in}}%
\pgfpathclose%
\pgfusepath{fill}%
\end{pgfscope}%
\begin{pgfscope}%
\pgfpathrectangle{\pgfqpoint{6.818937in}{0.147348in}}{\pgfqpoint{2.735294in}{2.735294in}}%
\pgfusepath{clip}%
\pgfsetbuttcap%
\pgfsetroundjoin%
\definecolor{currentfill}{rgb}{0.070254,0.269685,0.407928}%
\pgfsetfillcolor{currentfill}%
\pgfsetlinewidth{0.000000pt}%
\definecolor{currentstroke}{rgb}{0.000000,0.000000,0.000000}%
\pgfsetstrokecolor{currentstroke}%
\pgfsetdash{}{0pt}%
\pgfpathmoveto{\pgfqpoint{9.120088in}{1.620392in}}%
\pgfpathlineto{\pgfqpoint{9.143495in}{1.142562in}}%
\pgfpathlineto{\pgfqpoint{9.124324in}{1.217806in}}%
\pgfpathlineto{\pgfqpoint{9.120088in}{1.620392in}}%
\pgfpathclose%
\pgfusepath{fill}%
\end{pgfscope}%
\begin{pgfscope}%
\pgfpathrectangle{\pgfqpoint{6.818937in}{0.147348in}}{\pgfqpoint{2.735294in}{2.735294in}}%
\pgfusepath{clip}%
\pgfsetbuttcap%
\pgfsetroundjoin%
\definecolor{currentfill}{rgb}{0.070254,0.269685,0.407928}%
\pgfsetfillcolor{currentfill}%
\pgfsetlinewidth{0.000000pt}%
\definecolor{currentstroke}{rgb}{0.000000,0.000000,0.000000}%
\pgfsetstrokecolor{currentstroke}%
\pgfsetdash{}{0pt}%
\pgfpathmoveto{\pgfqpoint{7.322771in}{1.217806in}}%
\pgfpathlineto{\pgfqpoint{7.303601in}{1.142562in}}%
\pgfpathlineto{\pgfqpoint{7.327008in}{1.620392in}}%
\pgfpathlineto{\pgfqpoint{7.322771in}{1.217806in}}%
\pgfpathclose%
\pgfusepath{fill}%
\end{pgfscope}%
\begin{pgfscope}%
\pgfpathrectangle{\pgfqpoint{6.818937in}{0.147348in}}{\pgfqpoint{2.735294in}{2.735294in}}%
\pgfusepath{clip}%
\pgfsetbuttcap%
\pgfsetroundjoin%
\definecolor{currentfill}{rgb}{0.048960,0.187944,0.284285}%
\pgfsetfillcolor{currentfill}%
\pgfsetlinewidth{0.000000pt}%
\definecolor{currentstroke}{rgb}{0.000000,0.000000,0.000000}%
\pgfsetstrokecolor{currentstroke}%
\pgfsetdash{}{0pt}%
\pgfpathmoveto{\pgfqpoint{9.235220in}{1.276682in}}%
\pgfpathlineto{\pgfqpoint{9.143495in}{1.142562in}}%
\pgfpathlineto{\pgfqpoint{9.120088in}{1.620392in}}%
\pgfpathlineto{\pgfqpoint{9.235220in}{1.276682in}}%
\pgfpathclose%
\pgfusepath{fill}%
\end{pgfscope}%
\begin{pgfscope}%
\pgfpathrectangle{\pgfqpoint{6.818937in}{0.147348in}}{\pgfqpoint{2.735294in}{2.735294in}}%
\pgfusepath{clip}%
\pgfsetbuttcap%
\pgfsetroundjoin%
\definecolor{currentfill}{rgb}{0.048960,0.187944,0.284285}%
\pgfsetfillcolor{currentfill}%
\pgfsetlinewidth{0.000000pt}%
\definecolor{currentstroke}{rgb}{0.000000,0.000000,0.000000}%
\pgfsetstrokecolor{currentstroke}%
\pgfsetdash{}{0pt}%
\pgfpathmoveto{\pgfqpoint{7.211876in}{1.276682in}}%
\pgfpathlineto{\pgfqpoint{7.327008in}{1.620392in}}%
\pgfpathlineto{\pgfqpoint{7.303601in}{1.142562in}}%
\pgfpathlineto{\pgfqpoint{7.211876in}{1.276682in}}%
\pgfpathclose%
\pgfusepath{fill}%
\end{pgfscope}%
\begin{pgfscope}%
\pgfpathrectangle{\pgfqpoint{6.818937in}{0.147348in}}{\pgfqpoint{2.735294in}{2.735294in}}%
\pgfusepath{clip}%
\pgfsetbuttcap%
\pgfsetroundjoin%
\definecolor{currentfill}{rgb}{0.047548,0.182523,0.276086}%
\pgfsetfillcolor{currentfill}%
\pgfsetlinewidth{0.000000pt}%
\definecolor{currentstroke}{rgb}{0.000000,0.000000,0.000000}%
\pgfsetstrokecolor{currentstroke}%
\pgfsetdash{}{0pt}%
\pgfpathmoveto{\pgfqpoint{9.017173in}{1.079776in}}%
\pgfpathlineto{\pgfqpoint{9.124324in}{1.217806in}}%
\pgfpathlineto{\pgfqpoint{9.143495in}{1.142562in}}%
\pgfpathlineto{\pgfqpoint{9.017173in}{1.079776in}}%
\pgfpathclose%
\pgfusepath{fill}%
\end{pgfscope}%
\begin{pgfscope}%
\pgfpathrectangle{\pgfqpoint{6.818937in}{0.147348in}}{\pgfqpoint{2.735294in}{2.735294in}}%
\pgfusepath{clip}%
\pgfsetbuttcap%
\pgfsetroundjoin%
\definecolor{currentfill}{rgb}{0.047548,0.182523,0.276086}%
\pgfsetfillcolor{currentfill}%
\pgfsetlinewidth{0.000000pt}%
\definecolor{currentstroke}{rgb}{0.000000,0.000000,0.000000}%
\pgfsetstrokecolor{currentstroke}%
\pgfsetdash{}{0pt}%
\pgfpathmoveto{\pgfqpoint{7.303601in}{1.142562in}}%
\pgfpathlineto{\pgfqpoint{7.322771in}{1.217806in}}%
\pgfpathlineto{\pgfqpoint{7.429923in}{1.079776in}}%
\pgfpathlineto{\pgfqpoint{7.303601in}{1.142562in}}%
\pgfpathclose%
\pgfusepath{fill}%
\end{pgfscope}%
\begin{pgfscope}%
\pgfpathrectangle{\pgfqpoint{6.818937in}{0.147348in}}{\pgfqpoint{2.735294in}{2.735294in}}%
\pgfusepath{clip}%
\pgfsetbuttcap%
\pgfsetroundjoin%
\definecolor{currentfill}{rgb}{0.090605,0.347808,0.526096}%
\pgfsetfillcolor{currentfill}%
\pgfsetlinewidth{0.000000pt}%
\definecolor{currentstroke}{rgb}{0.000000,0.000000,0.000000}%
\pgfsetstrokecolor{currentstroke}%
\pgfsetdash{}{0pt}%
\pgfpathmoveto{\pgfqpoint{8.315273in}{2.613951in}}%
\pgfpathlineto{\pgfqpoint{8.131822in}{2.613951in}}%
\pgfpathlineto{\pgfqpoint{8.223548in}{2.686670in}}%
\pgfpathlineto{\pgfqpoint{8.315273in}{2.613951in}}%
\pgfpathclose%
\pgfusepath{fill}%
\end{pgfscope}%
\begin{pgfscope}%
\pgfpathrectangle{\pgfqpoint{6.818937in}{0.147348in}}{\pgfqpoint{2.735294in}{2.735294in}}%
\pgfusepath{clip}%
\pgfsetbuttcap%
\pgfsetroundjoin%
\definecolor{currentfill}{rgb}{0.050070,0.192203,0.290728}%
\pgfsetfillcolor{currentfill}%
\pgfsetlinewidth{0.000000pt}%
\definecolor{currentstroke}{rgb}{0.000000,0.000000,0.000000}%
\pgfsetstrokecolor{currentstroke}%
\pgfsetdash{}{0pt}%
\pgfpathmoveto{\pgfqpoint{9.143495in}{1.142562in}}%
\pgfpathlineto{\pgfqpoint{9.235220in}{1.276682in}}%
\pgfpathlineto{\pgfqpoint{9.243948in}{1.203196in}}%
\pgfpathlineto{\pgfqpoint{9.143495in}{1.142562in}}%
\pgfpathclose%
\pgfusepath{fill}%
\end{pgfscope}%
\begin{pgfscope}%
\pgfpathrectangle{\pgfqpoint{6.818937in}{0.147348in}}{\pgfqpoint{2.735294in}{2.735294in}}%
\pgfusepath{clip}%
\pgfsetbuttcap%
\pgfsetroundjoin%
\definecolor{currentfill}{rgb}{0.050070,0.192203,0.290728}%
\pgfsetfillcolor{currentfill}%
\pgfsetlinewidth{0.000000pt}%
\definecolor{currentstroke}{rgb}{0.000000,0.000000,0.000000}%
\pgfsetstrokecolor{currentstroke}%
\pgfsetdash{}{0pt}%
\pgfpathmoveto{\pgfqpoint{7.211876in}{1.276682in}}%
\pgfpathlineto{\pgfqpoint{7.303601in}{1.142562in}}%
\pgfpathlineto{\pgfqpoint{7.203147in}{1.203196in}}%
\pgfpathlineto{\pgfqpoint{7.211876in}{1.276682in}}%
\pgfpathclose%
\pgfusepath{fill}%
\end{pgfscope}%
\begin{pgfscope}%
\pgfsetbuttcap%
\pgfsetmiterjoin%
\definecolor{currentfill}{rgb}{1.000000,1.000000,1.000000}%
\pgfsetfillcolor{currentfill}%
\pgfsetlinewidth{0.000000pt}%
\definecolor{currentstroke}{rgb}{0.000000,0.000000,0.000000}%
\pgfsetstrokecolor{currentstroke}%
\pgfsetstrokeopacity{0.000000}%
\pgfsetdash{}{0pt}%
\pgfpathmoveto{\pgfqpoint{0.254231in}{0.147348in}}%
\pgfpathlineto{\pgfqpoint{2.989526in}{0.147348in}}%
\pgfpathlineto{\pgfqpoint{2.989526in}{2.882642in}}%
\pgfpathlineto{\pgfqpoint{0.254231in}{2.882642in}}%
\pgfpathlineto{\pgfqpoint{0.254231in}{0.147348in}}%
\pgfpathclose%
\pgfusepath{fill}%
\end{pgfscope}%
\begin{pgfscope}%
\pgfsetbuttcap%
\pgfsetmiterjoin%
\definecolor{currentfill}{rgb}{0.950000,0.950000,0.950000}%
\pgfsetfillcolor{currentfill}%
\pgfsetfillopacity{0.500000}%
\pgfsetlinewidth{1.003750pt}%
\definecolor{currentstroke}{rgb}{0.950000,0.950000,0.950000}%
\pgfsetstrokecolor{currentstroke}%
\pgfsetstrokeopacity{0.500000}%
\pgfsetdash{}{0pt}%
\pgfpathmoveto{\pgfqpoint{1.658842in}{1.930798in}}%
\pgfpathlineto{\pgfqpoint{2.865378in}{1.099507in}}%
\pgfpathlineto{\pgfqpoint{2.944036in}{2.033906in}}%
\pgfpathlineto{\pgfqpoint{1.658842in}{2.862146in}}%
\pgfusepath{stroke,fill}%
\end{pgfscope}%
\begin{pgfscope}%
\pgfsetbuttcap%
\pgfsetmiterjoin%
\definecolor{currentfill}{rgb}{0.900000,0.900000,0.900000}%
\pgfsetfillcolor{currentfill}%
\pgfsetfillopacity{0.500000}%
\pgfsetlinewidth{1.003750pt}%
\definecolor{currentstroke}{rgb}{0.900000,0.900000,0.900000}%
\pgfsetstrokecolor{currentstroke}%
\pgfsetstrokeopacity{0.500000}%
\pgfsetdash{}{0pt}%
\pgfpathmoveto{\pgfqpoint{1.658842in}{1.930798in}}%
\pgfpathlineto{\pgfqpoint{0.452305in}{1.099507in}}%
\pgfpathlineto{\pgfqpoint{0.373648in}{2.033906in}}%
\pgfpathlineto{\pgfqpoint{1.658842in}{2.862146in}}%
\pgfusepath{stroke,fill}%
\end{pgfscope}%
\begin{pgfscope}%
\pgfsetbuttcap%
\pgfsetmiterjoin%
\definecolor{currentfill}{rgb}{0.925000,0.925000,0.925000}%
\pgfsetfillcolor{currentfill}%
\pgfsetfillopacity{0.500000}%
\pgfsetlinewidth{1.003750pt}%
\definecolor{currentstroke}{rgb}{0.925000,0.925000,0.925000}%
\pgfsetstrokecolor{currentstroke}%
\pgfsetstrokeopacity{0.500000}%
\pgfsetdash{}{0pt}%
\pgfpathmoveto{\pgfqpoint{1.658842in}{1.930798in}}%
\pgfpathlineto{\pgfqpoint{0.452305in}{1.099507in}}%
\pgfpathlineto{\pgfqpoint{1.658842in}{0.166408in}}%
\pgfpathlineto{\pgfqpoint{2.865378in}{1.099507in}}%
\pgfusepath{stroke,fill}%
\end{pgfscope}%
\begin{pgfscope}%
\pgfsetbuttcap%
\pgfsetroundjoin%
\pgfsetlinewidth{0.803000pt}%
\definecolor{currentstroke}{rgb}{0.690196,0.690196,0.690196}%
\pgfsetstrokecolor{currentstroke}%
\pgfsetdash{}{0pt}%
\pgfpathmoveto{\pgfqpoint{2.792763in}{1.043348in}}%
\pgfpathlineto{\pgfqpoint{1.585998in}{1.880609in}}%
\pgfpathlineto{\pgfqpoint{1.581508in}{2.812308in}}%
\pgfusepath{stroke}%
\end{pgfscope}%
\begin{pgfscope}%
\pgfsetbuttcap%
\pgfsetroundjoin%
\pgfsetlinewidth{0.803000pt}%
\definecolor{currentstroke}{rgb}{0.690196,0.690196,0.690196}%
\pgfsetstrokecolor{currentstroke}%
\pgfsetdash{}{0pt}%
\pgfpathmoveto{\pgfqpoint{2.562953in}{0.865621in}}%
\pgfpathlineto{\pgfqpoint{1.355660in}{1.721909in}}%
\pgfpathlineto{\pgfqpoint{1.336753in}{2.654576in}}%
\pgfusepath{stroke}%
\end{pgfscope}%
\begin{pgfscope}%
\pgfsetbuttcap%
\pgfsetroundjoin%
\pgfsetlinewidth{0.803000pt}%
\definecolor{currentstroke}{rgb}{0.690196,0.690196,0.690196}%
\pgfsetstrokecolor{currentstroke}%
\pgfsetdash{}{0pt}%
\pgfpathmoveto{\pgfqpoint{2.327740in}{0.683714in}}%
\pgfpathlineto{\pgfqpoint{1.120209in}{1.559686in}}%
\pgfpathlineto{\pgfqpoint{1.086222in}{2.493122in}}%
\pgfusepath{stroke}%
\end{pgfscope}%
\begin{pgfscope}%
\pgfsetbuttcap%
\pgfsetroundjoin%
\pgfsetlinewidth{0.803000pt}%
\definecolor{currentstroke}{rgb}{0.690196,0.690196,0.690196}%
\pgfsetstrokecolor{currentstroke}%
\pgfsetdash{}{0pt}%
\pgfpathmoveto{\pgfqpoint{2.086930in}{0.497478in}}%
\pgfpathlineto{\pgfqpoint{0.879473in}{1.393821in}}%
\pgfpathlineto{\pgfqpoint{0.829708in}{2.327813in}}%
\pgfusepath{stroke}%
\end{pgfscope}%
\begin{pgfscope}%
\pgfsetbuttcap%
\pgfsetroundjoin%
\pgfsetlinewidth{0.803000pt}%
\definecolor{currentstroke}{rgb}{0.690196,0.690196,0.690196}%
\pgfsetstrokecolor{currentstroke}%
\pgfsetdash{}{0pt}%
\pgfpathmoveto{\pgfqpoint{1.840321in}{0.306758in}}%
\pgfpathlineto{\pgfqpoint{0.633271in}{1.224191in}}%
\pgfpathlineto{\pgfqpoint{0.566994in}{2.158507in}}%
\pgfusepath{stroke}%
\end{pgfscope}%
\begin{pgfscope}%
\pgfsetbuttcap%
\pgfsetroundjoin%
\pgfsetlinewidth{0.803000pt}%
\definecolor{currentstroke}{rgb}{0.690196,0.690196,0.690196}%
\pgfsetstrokecolor{currentstroke}%
\pgfsetdash{}{0pt}%
\pgfpathmoveto{\pgfqpoint{1.736176in}{2.812308in}}%
\pgfpathlineto{\pgfqpoint{1.731686in}{1.880609in}}%
\pgfpathlineto{\pgfqpoint{0.524921in}{1.043348in}}%
\pgfusepath{stroke}%
\end{pgfscope}%
\begin{pgfscope}%
\pgfsetbuttcap%
\pgfsetroundjoin%
\pgfsetlinewidth{0.803000pt}%
\definecolor{currentstroke}{rgb}{0.690196,0.690196,0.690196}%
\pgfsetstrokecolor{currentstroke}%
\pgfsetdash{}{0pt}%
\pgfpathmoveto{\pgfqpoint{1.980931in}{2.654576in}}%
\pgfpathlineto{\pgfqpoint{1.962024in}{1.721909in}}%
\pgfpathlineto{\pgfqpoint{0.754731in}{0.865621in}}%
\pgfusepath{stroke}%
\end{pgfscope}%
\begin{pgfscope}%
\pgfsetbuttcap%
\pgfsetroundjoin%
\pgfsetlinewidth{0.803000pt}%
\definecolor{currentstroke}{rgb}{0.690196,0.690196,0.690196}%
\pgfsetstrokecolor{currentstroke}%
\pgfsetdash{}{0pt}%
\pgfpathmoveto{\pgfqpoint{2.231462in}{2.493122in}}%
\pgfpathlineto{\pgfqpoint{2.197475in}{1.559686in}}%
\pgfpathlineto{\pgfqpoint{0.989944in}{0.683714in}}%
\pgfusepath{stroke}%
\end{pgfscope}%
\begin{pgfscope}%
\pgfsetbuttcap%
\pgfsetroundjoin%
\pgfsetlinewidth{0.803000pt}%
\definecolor{currentstroke}{rgb}{0.690196,0.690196,0.690196}%
\pgfsetstrokecolor{currentstroke}%
\pgfsetdash{}{0pt}%
\pgfpathmoveto{\pgfqpoint{2.487976in}{2.327813in}}%
\pgfpathlineto{\pgfqpoint{2.438211in}{1.393821in}}%
\pgfpathlineto{\pgfqpoint{1.230754in}{0.497478in}}%
\pgfusepath{stroke}%
\end{pgfscope}%
\begin{pgfscope}%
\pgfsetbuttcap%
\pgfsetroundjoin%
\pgfsetlinewidth{0.803000pt}%
\definecolor{currentstroke}{rgb}{0.690196,0.690196,0.690196}%
\pgfsetstrokecolor{currentstroke}%
\pgfsetdash{}{0pt}%
\pgfpathmoveto{\pgfqpoint{2.750690in}{2.158507in}}%
\pgfpathlineto{\pgfqpoint{2.684413in}{1.224191in}}%
\pgfpathlineto{\pgfqpoint{1.477363in}{0.306758in}}%
\pgfusepath{stroke}%
\end{pgfscope}%
\begin{pgfscope}%
\pgfsetbuttcap%
\pgfsetroundjoin%
\pgfsetlinewidth{0.803000pt}%
\definecolor{currentstroke}{rgb}{0.690196,0.690196,0.690196}%
\pgfsetstrokecolor{currentstroke}%
\pgfsetdash{}{0pt}%
\pgfpathmoveto{\pgfqpoint{0.447588in}{1.155548in}}%
\pgfpathlineto{\pgfqpoint{1.658842in}{1.986842in}}%
\pgfpathlineto{\pgfqpoint{2.870096in}{1.155548in}}%
\pgfusepath{stroke}%
\end{pgfscope}%
\begin{pgfscope}%
\pgfsetbuttcap%
\pgfsetroundjoin%
\pgfsetlinewidth{0.803000pt}%
\definecolor{currentstroke}{rgb}{0.690196,0.690196,0.690196}%
\pgfsetstrokecolor{currentstroke}%
\pgfsetdash{}{0pt}%
\pgfpathmoveto{\pgfqpoint{0.432645in}{1.333067in}}%
\pgfpathlineto{\pgfqpoint{1.658842in}{2.164215in}}%
\pgfpathlineto{\pgfqpoint{2.885039in}{1.333067in}}%
\pgfusepath{stroke}%
\end{pgfscope}%
\begin{pgfscope}%
\pgfsetbuttcap%
\pgfsetroundjoin%
\pgfsetlinewidth{0.803000pt}%
\definecolor{currentstroke}{rgb}{0.690196,0.690196,0.690196}%
\pgfsetstrokecolor{currentstroke}%
\pgfsetdash{}{0pt}%
\pgfpathmoveto{\pgfqpoint{0.417328in}{1.515021in}}%
\pgfpathlineto{\pgfqpoint{1.658842in}{2.345771in}}%
\pgfpathlineto{\pgfqpoint{2.900356in}{1.515021in}}%
\pgfusepath{stroke}%
\end{pgfscope}%
\begin{pgfscope}%
\pgfsetbuttcap%
\pgfsetroundjoin%
\pgfsetlinewidth{0.803000pt}%
\definecolor{currentstroke}{rgb}{0.690196,0.690196,0.690196}%
\pgfsetstrokecolor{currentstroke}%
\pgfsetdash{}{0pt}%
\pgfpathmoveto{\pgfqpoint{0.401624in}{1.701578in}}%
\pgfpathlineto{\pgfqpoint{1.658842in}{2.531660in}}%
\pgfpathlineto{\pgfqpoint{2.916060in}{1.701578in}}%
\pgfusepath{stroke}%
\end{pgfscope}%
\begin{pgfscope}%
\pgfsetbuttcap%
\pgfsetroundjoin%
\pgfsetlinewidth{0.803000pt}%
\definecolor{currentstroke}{rgb}{0.690196,0.690196,0.690196}%
\pgfsetstrokecolor{currentstroke}%
\pgfsetdash{}{0pt}%
\pgfpathmoveto{\pgfqpoint{0.385517in}{1.892915in}}%
\pgfpathlineto{\pgfqpoint{1.658842in}{2.722038in}}%
\pgfpathlineto{\pgfqpoint{2.932167in}{1.892915in}}%
\pgfusepath{stroke}%
\end{pgfscope}%
\begin{pgfscope}%
\pgfsetrectcap%
\pgfsetroundjoin%
\pgfsetlinewidth{0.803000pt}%
\definecolor{currentstroke}{rgb}{0.000000,0.000000,0.000000}%
\pgfsetstrokecolor{currentstroke}%
\pgfsetdash{}{0pt}%
\pgfpathmoveto{\pgfqpoint{2.865378in}{1.099507in}}%
\pgfpathlineto{\pgfqpoint{1.658842in}{0.166408in}}%
\pgfusepath{stroke}%
\end{pgfscope}%
\begin{pgfscope}%
\pgfsetrectcap%
\pgfsetroundjoin%
\pgfsetlinewidth{0.803000pt}%
\definecolor{currentstroke}{rgb}{0.000000,0.000000,0.000000}%
\pgfsetstrokecolor{currentstroke}%
\pgfsetdash{}{0pt}%
\pgfpathmoveto{\pgfqpoint{2.782554in}{1.050431in}}%
\pgfpathlineto{\pgfqpoint{2.813208in}{1.029163in}}%
\pgfusepath{stroke}%
\end{pgfscope}%
\begin{pgfscope}%
\pgfsetrectcap%
\pgfsetroundjoin%
\pgfsetlinewidth{0.803000pt}%
\definecolor{currentstroke}{rgb}{0.000000,0.000000,0.000000}%
\pgfsetstrokecolor{currentstroke}%
\pgfsetdash{}{0pt}%
\pgfpathmoveto{\pgfqpoint{2.552734in}{0.872869in}}%
\pgfpathlineto{\pgfqpoint{2.583421in}{0.851104in}}%
\pgfusepath{stroke}%
\end{pgfscope}%
\begin{pgfscope}%
\pgfsetrectcap%
\pgfsetroundjoin%
\pgfsetlinewidth{0.803000pt}%
\definecolor{currentstroke}{rgb}{0.000000,0.000000,0.000000}%
\pgfsetstrokecolor{currentstroke}%
\pgfsetdash{}{0pt}%
\pgfpathmoveto{\pgfqpoint{2.317512in}{0.691133in}}%
\pgfpathlineto{\pgfqpoint{2.348225in}{0.668853in}}%
\pgfusepath{stroke}%
\end{pgfscope}%
\begin{pgfscope}%
\pgfsetrectcap%
\pgfsetroundjoin%
\pgfsetlinewidth{0.803000pt}%
\definecolor{currentstroke}{rgb}{0.000000,0.000000,0.000000}%
\pgfsetstrokecolor{currentstroke}%
\pgfsetdash{}{0pt}%
\pgfpathmoveto{\pgfqpoint{2.076696in}{0.505076in}}%
\pgfpathlineto{\pgfqpoint{2.107428in}{0.482262in}}%
\pgfusepath{stroke}%
\end{pgfscope}%
\begin{pgfscope}%
\pgfsetrectcap%
\pgfsetroundjoin%
\pgfsetlinewidth{0.803000pt}%
\definecolor{currentstroke}{rgb}{0.000000,0.000000,0.000000}%
\pgfsetstrokecolor{currentstroke}%
\pgfsetdash{}{0pt}%
\pgfpathmoveto{\pgfqpoint{1.830084in}{0.314540in}}%
\pgfpathlineto{\pgfqpoint{1.860826in}{0.291173in}}%
\pgfusepath{stroke}%
\end{pgfscope}%
\begin{pgfscope}%
\definecolor{textcolor}{rgb}{0.000000,0.000000,0.000000}%
\pgfsetstrokecolor{textcolor}%
\pgfsetfillcolor{textcolor}%
\pgftext[x=2.557884in,y=0.241958in,,]{\color{textcolor}{\rmfamily\fontsize{14.000000}{16.800000}\selectfont\catcode`\^=\active\def^{\ifmmode\sp\else\^{}\fi}\catcode`\%=\active\def%{\%}f1}}%
\end{pgfscope}%
\begin{pgfscope}%
\pgfsetrectcap%
\pgfsetroundjoin%
\pgfsetlinewidth{0.803000pt}%
\definecolor{currentstroke}{rgb}{0.000000,0.000000,0.000000}%
\pgfsetstrokecolor{currentstroke}%
\pgfsetdash{}{0pt}%
\pgfpathmoveto{\pgfqpoint{0.452305in}{1.099507in}}%
\pgfpathlineto{\pgfqpoint{1.658842in}{0.166408in}}%
\pgfusepath{stroke}%
\end{pgfscope}%
\begin{pgfscope}%
\pgfsetrectcap%
\pgfsetroundjoin%
\pgfsetlinewidth{0.803000pt}%
\definecolor{currentstroke}{rgb}{0.000000,0.000000,0.000000}%
\pgfsetstrokecolor{currentstroke}%
\pgfsetdash{}{0pt}%
\pgfpathmoveto{\pgfqpoint{0.535130in}{1.050431in}}%
\pgfpathlineto{\pgfqpoint{0.504476in}{1.029163in}}%
\pgfusepath{stroke}%
\end{pgfscope}%
\begin{pgfscope}%
\pgfsetrectcap%
\pgfsetroundjoin%
\pgfsetlinewidth{0.803000pt}%
\definecolor{currentstroke}{rgb}{0.000000,0.000000,0.000000}%
\pgfsetstrokecolor{currentstroke}%
\pgfsetdash{}{0pt}%
\pgfpathmoveto{\pgfqpoint{0.764950in}{0.872869in}}%
\pgfpathlineto{\pgfqpoint{0.734263in}{0.851104in}}%
\pgfusepath{stroke}%
\end{pgfscope}%
\begin{pgfscope}%
\pgfsetrectcap%
\pgfsetroundjoin%
\pgfsetlinewidth{0.803000pt}%
\definecolor{currentstroke}{rgb}{0.000000,0.000000,0.000000}%
\pgfsetstrokecolor{currentstroke}%
\pgfsetdash{}{0pt}%
\pgfpathmoveto{\pgfqpoint{1.000172in}{0.691133in}}%
\pgfpathlineto{\pgfqpoint{0.969459in}{0.668853in}}%
\pgfusepath{stroke}%
\end{pgfscope}%
\begin{pgfscope}%
\pgfsetrectcap%
\pgfsetroundjoin%
\pgfsetlinewidth{0.803000pt}%
\definecolor{currentstroke}{rgb}{0.000000,0.000000,0.000000}%
\pgfsetstrokecolor{currentstroke}%
\pgfsetdash{}{0pt}%
\pgfpathmoveto{\pgfqpoint{1.240988in}{0.505076in}}%
\pgfpathlineto{\pgfqpoint{1.210256in}{0.482262in}}%
\pgfusepath{stroke}%
\end{pgfscope}%
\begin{pgfscope}%
\pgfsetrectcap%
\pgfsetroundjoin%
\pgfsetlinewidth{0.803000pt}%
\definecolor{currentstroke}{rgb}{0.000000,0.000000,0.000000}%
\pgfsetstrokecolor{currentstroke}%
\pgfsetdash{}{0pt}%
\pgfpathmoveto{\pgfqpoint{1.487600in}{0.314540in}}%
\pgfpathlineto{\pgfqpoint{1.456858in}{0.291173in}}%
\pgfusepath{stroke}%
\end{pgfscope}%
\begin{pgfscope}%
\definecolor{textcolor}{rgb}{0.000000,0.000000,0.000000}%
\pgfsetstrokecolor{textcolor}%
\pgfsetfillcolor{textcolor}%
\pgftext[x=0.759800in,y=0.241958in,,]{\color{textcolor}{\rmfamily\fontsize{14.000000}{16.800000}\selectfont\catcode`\^=\active\def^{\ifmmode\sp\else\^{}\fi}\catcode`\%=\active\def%{\%}f2}}%
\end{pgfscope}%
\begin{pgfscope}%
\pgfsetrectcap%
\pgfsetroundjoin%
\pgfsetlinewidth{0.803000pt}%
\definecolor{currentstroke}{rgb}{0.000000,0.000000,0.000000}%
\pgfsetstrokecolor{currentstroke}%
\pgfsetdash{}{0pt}%
\pgfpathmoveto{\pgfqpoint{0.452305in}{1.099507in}}%
\pgfpathlineto{\pgfqpoint{0.373648in}{2.033906in}}%
\pgfusepath{stroke}%
\end{pgfscope}%
\begin{pgfscope}%
\pgfsetrectcap%
\pgfsetroundjoin%
\pgfsetlinewidth{0.803000pt}%
\definecolor{currentstroke}{rgb}{0.000000,0.000000,0.000000}%
\pgfsetstrokecolor{currentstroke}%
\pgfsetdash{}{0pt}%
\pgfpathmoveto{\pgfqpoint{0.457835in}{1.162580in}}%
\pgfpathlineto{\pgfqpoint{0.427066in}{1.141464in}}%
\pgfusepath{stroke}%
\end{pgfscope}%
\begin{pgfscope}%
\pgfsetrectcap%
\pgfsetroundjoin%
\pgfsetlinewidth{0.803000pt}%
\definecolor{currentstroke}{rgb}{0.000000,0.000000,0.000000}%
\pgfsetstrokecolor{currentstroke}%
\pgfsetdash{}{0pt}%
\pgfpathmoveto{\pgfqpoint{0.443025in}{1.340103in}}%
\pgfpathlineto{\pgfqpoint{0.411855in}{1.318976in}}%
\pgfusepath{stroke}%
\end{pgfscope}%
\begin{pgfscope}%
\pgfsetrectcap%
\pgfsetroundjoin%
\pgfsetlinewidth{0.803000pt}%
\definecolor{currentstroke}{rgb}{0.000000,0.000000,0.000000}%
\pgfsetstrokecolor{currentstroke}%
\pgfsetdash{}{0pt}%
\pgfpathmoveto{\pgfqpoint{0.427845in}{1.522058in}}%
\pgfpathlineto{\pgfqpoint{0.396264in}{1.500926in}}%
\pgfusepath{stroke}%
\end{pgfscope}%
\begin{pgfscope}%
\pgfsetrectcap%
\pgfsetroundjoin%
\pgfsetlinewidth{0.803000pt}%
\definecolor{currentstroke}{rgb}{0.000000,0.000000,0.000000}%
\pgfsetstrokecolor{currentstroke}%
\pgfsetdash{}{0pt}%
\pgfpathmoveto{\pgfqpoint{0.412281in}{1.708614in}}%
\pgfpathlineto{\pgfqpoint{0.380278in}{1.687484in}}%
\pgfusepath{stroke}%
\end{pgfscope}%
\begin{pgfscope}%
\pgfsetrectcap%
\pgfsetroundjoin%
\pgfsetlinewidth{0.803000pt}%
\definecolor{currentstroke}{rgb}{0.000000,0.000000,0.000000}%
\pgfsetstrokecolor{currentstroke}%
\pgfsetdash{}{0pt}%
\pgfpathmoveto{\pgfqpoint{0.396319in}{1.899948in}}%
\pgfpathlineto{\pgfqpoint{0.363882in}{1.878827in}}%
\pgfusepath{stroke}%
\end{pgfscope}%
\begin{pgfscope}%
\definecolor{textcolor}{rgb}{0.000000,0.000000,0.000000}%
\pgfsetstrokecolor{textcolor}%
\pgfsetfillcolor{textcolor}%
\pgftext[x=-0.143944in,y=1.551958in,,]{\color{textcolor}{\rmfamily\fontsize{14.000000}{16.800000}\selectfont\catcode`\^=\active\def^{\ifmmode\sp\else\^{}\fi}\catcode`\%=\active\def%{\%}f3}}%
\end{pgfscope}%
\begin{pgfscope}%
\pgfpathrectangle{\pgfqpoint{0.254231in}{0.147348in}}{\pgfqpoint{2.735294in}{2.735294in}}%
\pgfusepath{clip}%
\pgfsetbuttcap%
\pgfsetroundjoin%
\definecolor{currentfill}{rgb}{0.050070,0.192203,0.290728}%
\pgfsetfillcolor{currentfill}%
\pgfsetlinewidth{0.000000pt}%
\definecolor{currentstroke}{rgb}{0.000000,0.000000,0.000000}%
\pgfsetstrokecolor{currentstroke}%
\pgfsetdash{}{0pt}%
\pgfpathmoveto{\pgfqpoint{2.615535in}{1.291641in}}%
\pgfpathlineto{\pgfqpoint{2.528344in}{1.165011in}}%
\pgfpathlineto{\pgfqpoint{2.615440in}{1.225003in}}%
\pgfpathlineto{\pgfqpoint{2.615535in}{1.291641in}}%
\pgfpathclose%
\pgfusepath{fill}%
\end{pgfscope}%
\begin{pgfscope}%
\pgfpathrectangle{\pgfqpoint{0.254231in}{0.147348in}}{\pgfqpoint{2.735294in}{2.735294in}}%
\pgfusepath{clip}%
\pgfsetbuttcap%
\pgfsetroundjoin%
\definecolor{currentfill}{rgb}{0.050070,0.192203,0.290728}%
\pgfsetfillcolor{currentfill}%
\pgfsetlinewidth{0.000000pt}%
\definecolor{currentstroke}{rgb}{0.000000,0.000000,0.000000}%
\pgfsetstrokecolor{currentstroke}%
\pgfsetdash{}{0pt}%
\pgfpathmoveto{\pgfqpoint{0.789340in}{1.165011in}}%
\pgfpathlineto{\pgfqpoint{0.702149in}{1.291641in}}%
\pgfpathlineto{\pgfqpoint{0.702244in}{1.225003in}}%
\pgfpathlineto{\pgfqpoint{0.789340in}{1.165011in}}%
\pgfpathclose%
\pgfusepath{fill}%
\end{pgfscope}%
\begin{pgfscope}%
\pgfpathrectangle{\pgfqpoint{0.254231in}{0.147348in}}{\pgfqpoint{2.735294in}{2.735294in}}%
\pgfusepath{clip}%
\pgfsetbuttcap%
\pgfsetroundjoin%
\definecolor{currentfill}{rgb}{0.090605,0.347808,0.526096}%
\pgfsetfillcolor{currentfill}%
\pgfsetlinewidth{0.000000pt}%
\definecolor{currentstroke}{rgb}{0.000000,0.000000,0.000000}%
\pgfsetstrokecolor{currentstroke}%
\pgfsetdash{}{0pt}%
\pgfpathmoveto{\pgfqpoint{1.571650in}{2.561465in}}%
\pgfpathlineto{\pgfqpoint{1.746034in}{2.561465in}}%
\pgfpathlineto{\pgfqpoint{1.658842in}{2.621838in}}%
\pgfpathlineto{\pgfqpoint{1.571650in}{2.561465in}}%
\pgfpathclose%
\pgfusepath{fill}%
\end{pgfscope}%
\begin{pgfscope}%
\pgfpathrectangle{\pgfqpoint{0.254231in}{0.147348in}}{\pgfqpoint{2.735294in}{2.735294in}}%
\pgfusepath{clip}%
\pgfsetbuttcap%
\pgfsetroundjoin%
\definecolor{currentfill}{rgb}{0.047548,0.182523,0.276086}%
\pgfsetfillcolor{currentfill}%
\pgfsetlinewidth{0.000000pt}%
\definecolor{currentstroke}{rgb}{0.000000,0.000000,0.000000}%
\pgfsetstrokecolor{currentstroke}%
\pgfsetdash{}{0pt}%
\pgfpathmoveto{\pgfqpoint{2.528344in}{1.165011in}}%
\pgfpathlineto{\pgfqpoint{2.518939in}{1.232896in}}%
\pgfpathlineto{\pgfqpoint{2.415375in}{1.101844in}}%
\pgfpathlineto{\pgfqpoint{2.528344in}{1.165011in}}%
\pgfpathclose%
\pgfusepath{fill}%
\end{pgfscope}%
\begin{pgfscope}%
\pgfpathrectangle{\pgfqpoint{0.254231in}{0.147348in}}{\pgfqpoint{2.735294in}{2.735294in}}%
\pgfusepath{clip}%
\pgfsetbuttcap%
\pgfsetroundjoin%
\definecolor{currentfill}{rgb}{0.047548,0.182523,0.276086}%
\pgfsetfillcolor{currentfill}%
\pgfsetlinewidth{0.000000pt}%
\definecolor{currentstroke}{rgb}{0.000000,0.000000,0.000000}%
\pgfsetstrokecolor{currentstroke}%
\pgfsetdash{}{0pt}%
\pgfpathmoveto{\pgfqpoint{0.902309in}{1.101844in}}%
\pgfpathlineto{\pgfqpoint{0.798745in}{1.232896in}}%
\pgfpathlineto{\pgfqpoint{0.789340in}{1.165011in}}%
\pgfpathlineto{\pgfqpoint{0.902309in}{1.101844in}}%
\pgfpathclose%
\pgfusepath{fill}%
\end{pgfscope}%
\begin{pgfscope}%
\pgfpathrectangle{\pgfqpoint{0.254231in}{0.147348in}}{\pgfqpoint{2.735294in}{2.735294in}}%
\pgfusepath{clip}%
\pgfsetbuttcap%
\pgfsetroundjoin%
\definecolor{currentfill}{rgb}{0.048960,0.187944,0.284285}%
\pgfsetfillcolor{currentfill}%
\pgfsetlinewidth{0.000000pt}%
\definecolor{currentstroke}{rgb}{0.000000,0.000000,0.000000}%
\pgfsetstrokecolor{currentstroke}%
\pgfsetdash{}{0pt}%
\pgfpathmoveto{\pgfqpoint{0.702149in}{1.291641in}}%
\pgfpathlineto{\pgfqpoint{0.789340in}{1.165011in}}%
\pgfpathlineto{\pgfqpoint{0.788552in}{1.618388in}}%
\pgfpathlineto{\pgfqpoint{0.702149in}{1.291641in}}%
\pgfpathclose%
\pgfusepath{fill}%
\end{pgfscope}%
\begin{pgfscope}%
\pgfpathrectangle{\pgfqpoint{0.254231in}{0.147348in}}{\pgfqpoint{2.735294in}{2.735294in}}%
\pgfusepath{clip}%
\pgfsetbuttcap%
\pgfsetroundjoin%
\definecolor{currentfill}{rgb}{0.048960,0.187944,0.284285}%
\pgfsetfillcolor{currentfill}%
\pgfsetlinewidth{0.000000pt}%
\definecolor{currentstroke}{rgb}{0.000000,0.000000,0.000000}%
\pgfsetstrokecolor{currentstroke}%
\pgfsetdash{}{0pt}%
\pgfpathmoveto{\pgfqpoint{2.615535in}{1.291641in}}%
\pgfpathlineto{\pgfqpoint{2.529132in}{1.618388in}}%
\pgfpathlineto{\pgfqpoint{2.528344in}{1.165011in}}%
\pgfpathlineto{\pgfqpoint{2.615535in}{1.291641in}}%
\pgfpathclose%
\pgfusepath{fill}%
\end{pgfscope}%
\begin{pgfscope}%
\pgfpathrectangle{\pgfqpoint{0.254231in}{0.147348in}}{\pgfqpoint{2.735294in}{2.735294in}}%
\pgfusepath{clip}%
\pgfsetbuttcap%
\pgfsetroundjoin%
\definecolor{currentfill}{rgb}{0.070254,0.269685,0.407928}%
\pgfsetfillcolor{currentfill}%
\pgfsetlinewidth{0.000000pt}%
\definecolor{currentstroke}{rgb}{0.000000,0.000000,0.000000}%
\pgfsetstrokecolor{currentstroke}%
\pgfsetdash{}{0pt}%
\pgfpathmoveto{\pgfqpoint{2.518939in}{1.232896in}}%
\pgfpathlineto{\pgfqpoint{2.528344in}{1.165011in}}%
\pgfpathlineto{\pgfqpoint{2.529132in}{1.618388in}}%
\pgfpathlineto{\pgfqpoint{2.518939in}{1.232896in}}%
\pgfpathclose%
\pgfusepath{fill}%
\end{pgfscope}%
\begin{pgfscope}%
\pgfpathrectangle{\pgfqpoint{0.254231in}{0.147348in}}{\pgfqpoint{2.735294in}{2.735294in}}%
\pgfusepath{clip}%
\pgfsetbuttcap%
\pgfsetroundjoin%
\definecolor{currentfill}{rgb}{0.070254,0.269685,0.407928}%
\pgfsetfillcolor{currentfill}%
\pgfsetlinewidth{0.000000pt}%
\definecolor{currentstroke}{rgb}{0.000000,0.000000,0.000000}%
\pgfsetstrokecolor{currentstroke}%
\pgfsetdash{}{0pt}%
\pgfpathmoveto{\pgfqpoint{0.788552in}{1.618388in}}%
\pgfpathlineto{\pgfqpoint{0.789340in}{1.165011in}}%
\pgfpathlineto{\pgfqpoint{0.798745in}{1.232896in}}%
\pgfpathlineto{\pgfqpoint{0.788552in}{1.618388in}}%
\pgfpathclose%
\pgfusepath{fill}%
\end{pgfscope}%
\begin{pgfscope}%
\pgfpathrectangle{\pgfqpoint{0.254231in}{0.147348in}}{\pgfqpoint{2.735294in}{2.735294in}}%
\pgfusepath{clip}%
\pgfsetbuttcap%
\pgfsetroundjoin%
\definecolor{currentfill}{rgb}{0.044978,0.172658,0.261163}%
\pgfsetfillcolor{currentfill}%
\pgfsetlinewidth{0.000000pt}%
\definecolor{currentstroke}{rgb}{0.000000,0.000000,0.000000}%
\pgfsetstrokecolor{currentstroke}%
\pgfsetdash{}{0pt}%
\pgfpathmoveto{\pgfqpoint{1.046414in}{1.039603in}}%
\pgfpathlineto{\pgfqpoint{0.926707in}{1.171211in}}%
\pgfpathlineto{\pgfqpoint{0.902309in}{1.101844in}}%
\pgfpathlineto{\pgfqpoint{1.046414in}{1.039603in}}%
\pgfpathclose%
\pgfusepath{fill}%
\end{pgfscope}%
\begin{pgfscope}%
\pgfpathrectangle{\pgfqpoint{0.254231in}{0.147348in}}{\pgfqpoint{2.735294in}{2.735294in}}%
\pgfusepath{clip}%
\pgfsetbuttcap%
\pgfsetroundjoin%
\definecolor{currentfill}{rgb}{0.044978,0.172658,0.261163}%
\pgfsetfillcolor{currentfill}%
\pgfsetlinewidth{0.000000pt}%
\definecolor{currentstroke}{rgb}{0.000000,0.000000,0.000000}%
\pgfsetstrokecolor{currentstroke}%
\pgfsetdash{}{0pt}%
\pgfpathmoveto{\pgfqpoint{2.415375in}{1.101844in}}%
\pgfpathlineto{\pgfqpoint{2.390977in}{1.171211in}}%
\pgfpathlineto{\pgfqpoint{2.271270in}{1.039603in}}%
\pgfpathlineto{\pgfqpoint{2.415375in}{1.101844in}}%
\pgfpathclose%
\pgfusepath{fill}%
\end{pgfscope}%
\begin{pgfscope}%
\pgfpathrectangle{\pgfqpoint{0.254231in}{0.147348in}}{\pgfqpoint{2.735294in}{2.735294in}}%
\pgfusepath{clip}%
\pgfsetbuttcap%
\pgfsetroundjoin%
\definecolor{currentfill}{rgb}{0.081954,0.314596,0.475860}%
\pgfsetfillcolor{currentfill}%
\pgfsetlinewidth{0.000000pt}%
\definecolor{currentstroke}{rgb}{0.000000,0.000000,0.000000}%
\pgfsetstrokecolor{currentstroke}%
\pgfsetdash{}{0pt}%
\pgfpathmoveto{\pgfqpoint{1.746034in}{2.561465in}}%
\pgfpathlineto{\pgfqpoint{1.571650in}{2.561465in}}%
\pgfpathlineto{\pgfqpoint{1.534055in}{2.124855in}}%
\pgfpathlineto{\pgfqpoint{1.746034in}{2.561465in}}%
\pgfpathclose%
\pgfusepath{fill}%
\end{pgfscope}%
\begin{pgfscope}%
\pgfpathrectangle{\pgfqpoint{0.254231in}{0.147348in}}{\pgfqpoint{2.735294in}{2.735294in}}%
\pgfusepath{clip}%
\pgfsetbuttcap%
\pgfsetroundjoin%
\definecolor{currentfill}{rgb}{0.047247,0.181368,0.274339}%
\pgfsetfillcolor{currentfill}%
\pgfsetlinewidth{0.000000pt}%
\definecolor{currentstroke}{rgb}{0.000000,0.000000,0.000000}%
\pgfsetstrokecolor{currentstroke}%
\pgfsetdash{}{0pt}%
\pgfpathmoveto{\pgfqpoint{2.382112in}{1.589448in}}%
\pgfpathlineto{\pgfqpoint{2.415375in}{1.101844in}}%
\pgfpathlineto{\pgfqpoint{2.518939in}{1.232896in}}%
\pgfpathlineto{\pgfqpoint{2.382112in}{1.589448in}}%
\pgfpathclose%
\pgfusepath{fill}%
\end{pgfscope}%
\begin{pgfscope}%
\pgfpathrectangle{\pgfqpoint{0.254231in}{0.147348in}}{\pgfqpoint{2.735294in}{2.735294in}}%
\pgfusepath{clip}%
\pgfsetbuttcap%
\pgfsetroundjoin%
\definecolor{currentfill}{rgb}{0.047247,0.181368,0.274339}%
\pgfsetfillcolor{currentfill}%
\pgfsetlinewidth{0.000000pt}%
\definecolor{currentstroke}{rgb}{0.000000,0.000000,0.000000}%
\pgfsetstrokecolor{currentstroke}%
\pgfsetdash{}{0pt}%
\pgfpathmoveto{\pgfqpoint{0.798745in}{1.232896in}}%
\pgfpathlineto{\pgfqpoint{0.902309in}{1.101844in}}%
\pgfpathlineto{\pgfqpoint{0.935572in}{1.589448in}}%
\pgfpathlineto{\pgfqpoint{0.798745in}{1.232896in}}%
\pgfpathclose%
\pgfusepath{fill}%
\end{pgfscope}%
\begin{pgfscope}%
\pgfpathrectangle{\pgfqpoint{0.254231in}{0.147348in}}{\pgfqpoint{2.735294in}{2.735294in}}%
\pgfusepath{clip}%
\pgfsetbuttcap%
\pgfsetroundjoin%
\definecolor{currentfill}{rgb}{0.067179,0.257880,0.390071}%
\pgfsetfillcolor{currentfill}%
\pgfsetlinewidth{0.000000pt}%
\definecolor{currentstroke}{rgb}{0.000000,0.000000,0.000000}%
\pgfsetstrokecolor{currentstroke}%
\pgfsetdash{}{0pt}%
\pgfpathmoveto{\pgfqpoint{0.935572in}{1.589448in}}%
\pgfpathlineto{\pgfqpoint{0.902309in}{1.101844in}}%
\pgfpathlineto{\pgfqpoint{0.926707in}{1.171211in}}%
\pgfpathlineto{\pgfqpoint{0.935572in}{1.589448in}}%
\pgfpathclose%
\pgfusepath{fill}%
\end{pgfscope}%
\begin{pgfscope}%
\pgfpathrectangle{\pgfqpoint{0.254231in}{0.147348in}}{\pgfqpoint{2.735294in}{2.735294in}}%
\pgfusepath{clip}%
\pgfsetbuttcap%
\pgfsetroundjoin%
\definecolor{currentfill}{rgb}{0.067179,0.257880,0.390071}%
\pgfsetfillcolor{currentfill}%
\pgfsetlinewidth{0.000000pt}%
\definecolor{currentstroke}{rgb}{0.000000,0.000000,0.000000}%
\pgfsetstrokecolor{currentstroke}%
\pgfsetdash{}{0pt}%
\pgfpathmoveto{\pgfqpoint{2.390977in}{1.171211in}}%
\pgfpathlineto{\pgfqpoint{2.415375in}{1.101844in}}%
\pgfpathlineto{\pgfqpoint{2.382112in}{1.589448in}}%
\pgfpathlineto{\pgfqpoint{2.390977in}{1.171211in}}%
\pgfpathclose%
\pgfusepath{fill}%
\end{pgfscope}%
\begin{pgfscope}%
\pgfpathrectangle{\pgfqpoint{0.254231in}{0.147348in}}{\pgfqpoint{2.735294in}{2.735294in}}%
\pgfusepath{clip}%
\pgfsetbuttcap%
\pgfsetroundjoin%
\definecolor{currentfill}{rgb}{0.042579,0.163449,0.247234}%
\pgfsetfillcolor{currentfill}%
\pgfsetlinewidth{0.000000pt}%
\definecolor{currentstroke}{rgb}{0.000000,0.000000,0.000000}%
\pgfsetstrokecolor{currentstroke}%
\pgfsetdash{}{0pt}%
\pgfpathmoveto{\pgfqpoint{1.046414in}{1.039603in}}%
\pgfpathlineto{\pgfqpoint{1.224246in}{0.985051in}}%
\pgfpathlineto{\pgfqpoint{1.092413in}{1.111775in}}%
\pgfpathlineto{\pgfqpoint{1.046414in}{1.039603in}}%
\pgfpathclose%
\pgfusepath{fill}%
\end{pgfscope}%
\begin{pgfscope}%
\pgfpathrectangle{\pgfqpoint{0.254231in}{0.147348in}}{\pgfqpoint{2.735294in}{2.735294in}}%
\pgfusepath{clip}%
\pgfsetbuttcap%
\pgfsetroundjoin%
\definecolor{currentfill}{rgb}{0.042579,0.163449,0.247234}%
\pgfsetfillcolor{currentfill}%
\pgfsetlinewidth{0.000000pt}%
\definecolor{currentstroke}{rgb}{0.000000,0.000000,0.000000}%
\pgfsetstrokecolor{currentstroke}%
\pgfsetdash{}{0pt}%
\pgfpathmoveto{\pgfqpoint{2.225271in}{1.111775in}}%
\pgfpathlineto{\pgfqpoint{2.093438in}{0.985051in}}%
\pgfpathlineto{\pgfqpoint{2.271270in}{1.039603in}}%
\pgfpathlineto{\pgfqpoint{2.225271in}{1.111775in}}%
\pgfpathclose%
\pgfusepath{fill}%
\end{pgfscope}%
\begin{pgfscope}%
\pgfpathrectangle{\pgfqpoint{0.254231in}{0.147348in}}{\pgfqpoint{2.735294in}{2.735294in}}%
\pgfusepath{clip}%
\pgfsetbuttcap%
\pgfsetroundjoin%
\definecolor{currentfill}{rgb}{0.052493,0.201505,0.304798}%
\pgfsetfillcolor{currentfill}%
\pgfsetlinewidth{0.000000pt}%
\definecolor{currentstroke}{rgb}{0.000000,0.000000,0.000000}%
\pgfsetstrokecolor{currentstroke}%
\pgfsetdash{}{0pt}%
\pgfpathmoveto{\pgfqpoint{2.382112in}{1.589448in}}%
\pgfpathlineto{\pgfqpoint{2.518939in}{1.232896in}}%
\pgfpathlineto{\pgfqpoint{2.529132in}{1.618388in}}%
\pgfpathlineto{\pgfqpoint{2.382112in}{1.589448in}}%
\pgfpathclose%
\pgfusepath{fill}%
\end{pgfscope}%
\begin{pgfscope}%
\pgfpathrectangle{\pgfqpoint{0.254231in}{0.147348in}}{\pgfqpoint{2.735294in}{2.735294in}}%
\pgfusepath{clip}%
\pgfsetbuttcap%
\pgfsetroundjoin%
\definecolor{currentfill}{rgb}{0.052493,0.201505,0.304798}%
\pgfsetfillcolor{currentfill}%
\pgfsetlinewidth{0.000000pt}%
\definecolor{currentstroke}{rgb}{0.000000,0.000000,0.000000}%
\pgfsetstrokecolor{currentstroke}%
\pgfsetdash{}{0pt}%
\pgfpathmoveto{\pgfqpoint{0.788552in}{1.618388in}}%
\pgfpathlineto{\pgfqpoint{0.798745in}{1.232896in}}%
\pgfpathlineto{\pgfqpoint{0.935572in}{1.589448in}}%
\pgfpathlineto{\pgfqpoint{0.788552in}{1.618388in}}%
\pgfpathclose%
\pgfusepath{fill}%
\end{pgfscope}%
\begin{pgfscope}%
\pgfpathrectangle{\pgfqpoint{0.254231in}{0.147348in}}{\pgfqpoint{2.735294in}{2.735294in}}%
\pgfusepath{clip}%
\pgfsetbuttcap%
\pgfsetroundjoin%
\definecolor{currentfill}{rgb}{0.082280,0.315849,0.477755}%
\pgfsetfillcolor{currentfill}%
\pgfsetlinewidth{0.000000pt}%
\definecolor{currentstroke}{rgb}{0.000000,0.000000,0.000000}%
\pgfsetstrokecolor{currentstroke}%
\pgfsetdash{}{0pt}%
\pgfpathmoveto{\pgfqpoint{2.094537in}{2.253326in}}%
\pgfpathlineto{\pgfqpoint{1.746034in}{2.561465in}}%
\pgfpathlineto{\pgfqpoint{1.783629in}{2.124855in}}%
\pgfpathlineto{\pgfqpoint{2.094537in}{2.253326in}}%
\pgfpathclose%
\pgfusepath{fill}%
\end{pgfscope}%
\begin{pgfscope}%
\pgfpathrectangle{\pgfqpoint{0.254231in}{0.147348in}}{\pgfqpoint{2.735294in}{2.735294in}}%
\pgfusepath{clip}%
\pgfsetbuttcap%
\pgfsetroundjoin%
\definecolor{currentfill}{rgb}{0.082280,0.315849,0.477755}%
\pgfsetfillcolor{currentfill}%
\pgfsetlinewidth{0.000000pt}%
\definecolor{currentstroke}{rgb}{0.000000,0.000000,0.000000}%
\pgfsetstrokecolor{currentstroke}%
\pgfsetdash{}{0pt}%
\pgfpathmoveto{\pgfqpoint{1.534055in}{2.124855in}}%
\pgfpathlineto{\pgfqpoint{1.571650in}{2.561465in}}%
\pgfpathlineto{\pgfqpoint{1.223147in}{2.253326in}}%
\pgfpathlineto{\pgfqpoint{1.534055in}{2.124855in}}%
\pgfpathclose%
\pgfusepath{fill}%
\end{pgfscope}%
\begin{pgfscope}%
\pgfpathrectangle{\pgfqpoint{0.254231in}{0.147348in}}{\pgfqpoint{2.735294in}{2.735294in}}%
\pgfusepath{clip}%
\pgfsetbuttcap%
\pgfsetroundjoin%
\definecolor{currentfill}{rgb}{0.045702,0.175435,0.265364}%
\pgfsetfillcolor{currentfill}%
\pgfsetlinewidth{0.000000pt}%
\definecolor{currentstroke}{rgb}{0.000000,0.000000,0.000000}%
\pgfsetstrokecolor{currentstroke}%
\pgfsetdash{}{0pt}%
\pgfpathmoveto{\pgfqpoint{0.926707in}{1.171211in}}%
\pgfpathlineto{\pgfqpoint{1.046414in}{1.039603in}}%
\pgfpathlineto{\pgfqpoint{1.132195in}{1.562087in}}%
\pgfpathlineto{\pgfqpoint{0.926707in}{1.171211in}}%
\pgfpathclose%
\pgfusepath{fill}%
\end{pgfscope}%
\begin{pgfscope}%
\pgfpathrectangle{\pgfqpoint{0.254231in}{0.147348in}}{\pgfqpoint{2.735294in}{2.735294in}}%
\pgfusepath{clip}%
\pgfsetbuttcap%
\pgfsetroundjoin%
\definecolor{currentfill}{rgb}{0.045702,0.175435,0.265364}%
\pgfsetfillcolor{currentfill}%
\pgfsetlinewidth{0.000000pt}%
\definecolor{currentstroke}{rgb}{0.000000,0.000000,0.000000}%
\pgfsetstrokecolor{currentstroke}%
\pgfsetdash{}{0pt}%
\pgfpathmoveto{\pgfqpoint{2.185489in}{1.562087in}}%
\pgfpathlineto{\pgfqpoint{2.271270in}{1.039603in}}%
\pgfpathlineto{\pgfqpoint{2.390977in}{1.171211in}}%
\pgfpathlineto{\pgfqpoint{2.185489in}{1.562087in}}%
\pgfpathclose%
\pgfusepath{fill}%
\end{pgfscope}%
\begin{pgfscope}%
\pgfpathrectangle{\pgfqpoint{0.254231in}{0.147348in}}{\pgfqpoint{2.735294in}{2.735294in}}%
\pgfusepath{clip}%
\pgfsetbuttcap%
\pgfsetroundjoin%
\definecolor{currentfill}{rgb}{0.040669,0.156116,0.236142}%
\pgfsetfillcolor{currentfill}%
\pgfsetlinewidth{0.000000pt}%
\definecolor{currentstroke}{rgb}{0.000000,0.000000,0.000000}%
\pgfsetstrokecolor{currentstroke}%
\pgfsetdash{}{0pt}%
\pgfpathmoveto{\pgfqpoint{1.297586in}{1.063020in}}%
\pgfpathlineto{\pgfqpoint{1.224246in}{0.985051in}}%
\pgfpathlineto{\pgfqpoint{1.432345in}{0.946905in}}%
\pgfpathlineto{\pgfqpoint{1.297586in}{1.063020in}}%
\pgfpathclose%
\pgfusepath{fill}%
\end{pgfscope}%
\begin{pgfscope}%
\pgfpathrectangle{\pgfqpoint{0.254231in}{0.147348in}}{\pgfqpoint{2.735294in}{2.735294in}}%
\pgfusepath{clip}%
\pgfsetbuttcap%
\pgfsetroundjoin%
\definecolor{currentfill}{rgb}{0.040669,0.156116,0.236142}%
\pgfsetfillcolor{currentfill}%
\pgfsetlinewidth{0.000000pt}%
\definecolor{currentstroke}{rgb}{0.000000,0.000000,0.000000}%
\pgfsetstrokecolor{currentstroke}%
\pgfsetdash{}{0pt}%
\pgfpathmoveto{\pgfqpoint{1.885339in}{0.946905in}}%
\pgfpathlineto{\pgfqpoint{2.093438in}{0.985051in}}%
\pgfpathlineto{\pgfqpoint{2.020098in}{1.063020in}}%
\pgfpathlineto{\pgfqpoint{1.885339in}{0.946905in}}%
\pgfpathclose%
\pgfusepath{fill}%
\end{pgfscope}%
\begin{pgfscope}%
\pgfpathrectangle{\pgfqpoint{0.254231in}{0.147348in}}{\pgfqpoint{2.735294in}{2.735294in}}%
\pgfusepath{clip}%
\pgfsetbuttcap%
\pgfsetroundjoin%
\definecolor{currentfill}{rgb}{0.063981,0.245604,0.371502}%
\pgfsetfillcolor{currentfill}%
\pgfsetlinewidth{0.000000pt}%
\definecolor{currentstroke}{rgb}{0.000000,0.000000,0.000000}%
\pgfsetstrokecolor{currentstroke}%
\pgfsetdash{}{0pt}%
\pgfpathmoveto{\pgfqpoint{1.132195in}{1.562087in}}%
\pgfpathlineto{\pgfqpoint{1.046414in}{1.039603in}}%
\pgfpathlineto{\pgfqpoint{1.092413in}{1.111775in}}%
\pgfpathlineto{\pgfqpoint{1.132195in}{1.562087in}}%
\pgfpathclose%
\pgfusepath{fill}%
\end{pgfscope}%
\begin{pgfscope}%
\pgfpathrectangle{\pgfqpoint{0.254231in}{0.147348in}}{\pgfqpoint{2.735294in}{2.735294in}}%
\pgfusepath{clip}%
\pgfsetbuttcap%
\pgfsetroundjoin%
\definecolor{currentfill}{rgb}{0.063981,0.245604,0.371502}%
\pgfsetfillcolor{currentfill}%
\pgfsetlinewidth{0.000000pt}%
\definecolor{currentstroke}{rgb}{0.000000,0.000000,0.000000}%
\pgfsetstrokecolor{currentstroke}%
\pgfsetdash{}{0pt}%
\pgfpathmoveto{\pgfqpoint{2.225271in}{1.111775in}}%
\pgfpathlineto{\pgfqpoint{2.271270in}{1.039603in}}%
\pgfpathlineto{\pgfqpoint{2.185489in}{1.562087in}}%
\pgfpathlineto{\pgfqpoint{2.225271in}{1.111775in}}%
\pgfpathclose%
\pgfusepath{fill}%
\end{pgfscope}%
\begin{pgfscope}%
\pgfpathrectangle{\pgfqpoint{0.254231in}{0.147348in}}{\pgfqpoint{2.735294in}{2.735294in}}%
\pgfusepath{clip}%
\pgfsetbuttcap%
\pgfsetroundjoin%
\definecolor{currentfill}{rgb}{0.060942,0.233938,0.353856}%
\pgfsetfillcolor{currentfill}%
\pgfsetlinewidth{0.000000pt}%
\definecolor{currentstroke}{rgb}{0.000000,0.000000,0.000000}%
\pgfsetstrokecolor{currentstroke}%
\pgfsetdash{}{0pt}%
\pgfpathmoveto{\pgfqpoint{0.935572in}{1.589448in}}%
\pgfpathlineto{\pgfqpoint{0.864797in}{1.772344in}}%
\pgfpathlineto{\pgfqpoint{0.788552in}{1.618388in}}%
\pgfpathlineto{\pgfqpoint{0.935572in}{1.589448in}}%
\pgfpathclose%
\pgfusepath{fill}%
\end{pgfscope}%
\begin{pgfscope}%
\pgfpathrectangle{\pgfqpoint{0.254231in}{0.147348in}}{\pgfqpoint{2.735294in}{2.735294in}}%
\pgfusepath{clip}%
\pgfsetbuttcap%
\pgfsetroundjoin%
\definecolor{currentfill}{rgb}{0.060942,0.233938,0.353856}%
\pgfsetfillcolor{currentfill}%
\pgfsetlinewidth{0.000000pt}%
\definecolor{currentstroke}{rgb}{0.000000,0.000000,0.000000}%
\pgfsetstrokecolor{currentstroke}%
\pgfsetdash{}{0pt}%
\pgfpathmoveto{\pgfqpoint{2.529132in}{1.618388in}}%
\pgfpathlineto{\pgfqpoint{2.452887in}{1.772344in}}%
\pgfpathlineto{\pgfqpoint{2.382112in}{1.589448in}}%
\pgfpathlineto{\pgfqpoint{2.529132in}{1.618388in}}%
\pgfpathclose%
\pgfusepath{fill}%
\end{pgfscope}%
\begin{pgfscope}%
\pgfpathrectangle{\pgfqpoint{0.254231in}{0.147348in}}{\pgfqpoint{2.735294in}{2.735294in}}%
\pgfusepath{clip}%
\pgfsetbuttcap%
\pgfsetroundjoin%
\definecolor{currentfill}{rgb}{0.081954,0.314596,0.475860}%
\pgfsetfillcolor{currentfill}%
\pgfsetlinewidth{0.000000pt}%
\definecolor{currentstroke}{rgb}{0.000000,0.000000,0.000000}%
\pgfsetstrokecolor{currentstroke}%
\pgfsetdash{}{0pt}%
\pgfpathmoveto{\pgfqpoint{1.534055in}{2.124855in}}%
\pgfpathlineto{\pgfqpoint{1.783629in}{2.124855in}}%
\pgfpathlineto{\pgfqpoint{1.746034in}{2.561465in}}%
\pgfpathlineto{\pgfqpoint{1.534055in}{2.124855in}}%
\pgfpathclose%
\pgfusepath{fill}%
\end{pgfscope}%
\begin{pgfscope}%
\pgfpathrectangle{\pgfqpoint{0.254231in}{0.147348in}}{\pgfqpoint{2.735294in}{2.735294in}}%
\pgfusepath{clip}%
\pgfsetbuttcap%
\pgfsetroundjoin%
\definecolor{currentfill}{rgb}{0.039595,0.151995,0.229908}%
\pgfsetfillcolor{currentfill}%
\pgfsetlinewidth{0.000000pt}%
\definecolor{currentstroke}{rgb}{0.000000,0.000000,0.000000}%
\pgfsetstrokecolor{currentstroke}%
\pgfsetdash{}{0pt}%
\pgfpathmoveto{\pgfqpoint{1.658842in}{0.933095in}}%
\pgfpathlineto{\pgfqpoint{1.534325in}{1.035006in}}%
\pgfpathlineto{\pgfqpoint{1.432345in}{0.946905in}}%
\pgfpathlineto{\pgfqpoint{1.658842in}{0.933095in}}%
\pgfpathclose%
\pgfusepath{fill}%
\end{pgfscope}%
\begin{pgfscope}%
\pgfpathrectangle{\pgfqpoint{0.254231in}{0.147348in}}{\pgfqpoint{2.735294in}{2.735294in}}%
\pgfusepath{clip}%
\pgfsetbuttcap%
\pgfsetroundjoin%
\definecolor{currentfill}{rgb}{0.039595,0.151995,0.229908}%
\pgfsetfillcolor{currentfill}%
\pgfsetlinewidth{0.000000pt}%
\definecolor{currentstroke}{rgb}{0.000000,0.000000,0.000000}%
\pgfsetstrokecolor{currentstroke}%
\pgfsetdash{}{0pt}%
\pgfpathmoveto{\pgfqpoint{1.885339in}{0.946905in}}%
\pgfpathlineto{\pgfqpoint{1.783359in}{1.035006in}}%
\pgfpathlineto{\pgfqpoint{1.658842in}{0.933095in}}%
\pgfpathlineto{\pgfqpoint{1.885339in}{0.946905in}}%
\pgfpathclose%
\pgfusepath{fill}%
\end{pgfscope}%
\begin{pgfscope}%
\pgfpathrectangle{\pgfqpoint{0.254231in}{0.147348in}}{\pgfqpoint{2.735294in}{2.735294in}}%
\pgfusepath{clip}%
\pgfsetbuttcap%
\pgfsetroundjoin%
\definecolor{currentfill}{rgb}{0.075436,0.289576,0.438014}%
\pgfsetfillcolor{currentfill}%
\pgfsetlinewidth{0.000000pt}%
\definecolor{currentstroke}{rgb}{0.000000,0.000000,0.000000}%
\pgfsetstrokecolor{currentstroke}%
\pgfsetdash{}{0pt}%
\pgfpathmoveto{\pgfqpoint{2.226391in}{2.103280in}}%
\pgfpathlineto{\pgfqpoint{2.094537in}{2.253326in}}%
\pgfpathlineto{\pgfqpoint{2.020857in}{2.116981in}}%
\pgfpathlineto{\pgfqpoint{2.226391in}{2.103280in}}%
\pgfpathclose%
\pgfusepath{fill}%
\end{pgfscope}%
\begin{pgfscope}%
\pgfpathrectangle{\pgfqpoint{0.254231in}{0.147348in}}{\pgfqpoint{2.735294in}{2.735294in}}%
\pgfusepath{clip}%
\pgfsetbuttcap%
\pgfsetroundjoin%
\definecolor{currentfill}{rgb}{0.075436,0.289576,0.438014}%
\pgfsetfillcolor{currentfill}%
\pgfsetlinewidth{0.000000pt}%
\definecolor{currentstroke}{rgb}{0.000000,0.000000,0.000000}%
\pgfsetstrokecolor{currentstroke}%
\pgfsetdash{}{0pt}%
\pgfpathmoveto{\pgfqpoint{1.296827in}{2.116981in}}%
\pgfpathlineto{\pgfqpoint{1.223147in}{2.253326in}}%
\pgfpathlineto{\pgfqpoint{1.091293in}{2.103280in}}%
\pgfpathlineto{\pgfqpoint{1.296827in}{2.116981in}}%
\pgfpathclose%
\pgfusepath{fill}%
\end{pgfscope}%
\begin{pgfscope}%
\pgfpathrectangle{\pgfqpoint{0.254231in}{0.147348in}}{\pgfqpoint{2.735294in}{2.735294in}}%
\pgfusepath{clip}%
\pgfsetbuttcap%
\pgfsetroundjoin%
\definecolor{currentfill}{rgb}{0.062760,0.240916,0.364410}%
\pgfsetfillcolor{currentfill}%
\pgfsetlinewidth{0.000000pt}%
\definecolor{currentstroke}{rgb}{0.000000,0.000000,0.000000}%
\pgfsetstrokecolor{currentstroke}%
\pgfsetdash{}{0pt}%
\pgfpathmoveto{\pgfqpoint{0.935572in}{1.589448in}}%
\pgfpathlineto{\pgfqpoint{1.091293in}{2.103280in}}%
\pgfpathlineto{\pgfqpoint{0.864797in}{1.772344in}}%
\pgfpathlineto{\pgfqpoint{0.935572in}{1.589448in}}%
\pgfpathclose%
\pgfusepath{fill}%
\end{pgfscope}%
\begin{pgfscope}%
\pgfpathrectangle{\pgfqpoint{0.254231in}{0.147348in}}{\pgfqpoint{2.735294in}{2.735294in}}%
\pgfusepath{clip}%
\pgfsetbuttcap%
\pgfsetroundjoin%
\definecolor{currentfill}{rgb}{0.062760,0.240916,0.364410}%
\pgfsetfillcolor{currentfill}%
\pgfsetlinewidth{0.000000pt}%
\definecolor{currentstroke}{rgb}{0.000000,0.000000,0.000000}%
\pgfsetstrokecolor{currentstroke}%
\pgfsetdash{}{0pt}%
\pgfpathmoveto{\pgfqpoint{2.452887in}{1.772344in}}%
\pgfpathlineto{\pgfqpoint{2.226391in}{2.103280in}}%
\pgfpathlineto{\pgfqpoint{2.382112in}{1.589448in}}%
\pgfpathlineto{\pgfqpoint{2.452887in}{1.772344in}}%
\pgfpathclose%
\pgfusepath{fill}%
\end{pgfscope}%
\begin{pgfscope}%
\pgfpathrectangle{\pgfqpoint{0.254231in}{0.147348in}}{\pgfqpoint{2.735294in}{2.735294in}}%
\pgfusepath{clip}%
\pgfsetbuttcap%
\pgfsetroundjoin%
\definecolor{currentfill}{rgb}{0.043508,0.167016,0.252629}%
\pgfsetfillcolor{currentfill}%
\pgfsetlinewidth{0.000000pt}%
\definecolor{currentstroke}{rgb}{0.000000,0.000000,0.000000}%
\pgfsetstrokecolor{currentstroke}%
\pgfsetdash{}{0pt}%
\pgfpathmoveto{\pgfqpoint{1.092413in}{1.111775in}}%
\pgfpathlineto{\pgfqpoint{1.224246in}{0.985051in}}%
\pgfpathlineto{\pgfqpoint{1.256196in}{1.360106in}}%
\pgfpathlineto{\pgfqpoint{1.092413in}{1.111775in}}%
\pgfpathclose%
\pgfusepath{fill}%
\end{pgfscope}%
\begin{pgfscope}%
\pgfpathrectangle{\pgfqpoint{0.254231in}{0.147348in}}{\pgfqpoint{2.735294in}{2.735294in}}%
\pgfusepath{clip}%
\pgfsetbuttcap%
\pgfsetroundjoin%
\definecolor{currentfill}{rgb}{0.043508,0.167016,0.252629}%
\pgfsetfillcolor{currentfill}%
\pgfsetlinewidth{0.000000pt}%
\definecolor{currentstroke}{rgb}{0.000000,0.000000,0.000000}%
\pgfsetstrokecolor{currentstroke}%
\pgfsetdash{}{0pt}%
\pgfpathmoveto{\pgfqpoint{2.061488in}{1.360106in}}%
\pgfpathlineto{\pgfqpoint{2.093438in}{0.985051in}}%
\pgfpathlineto{\pgfqpoint{2.225271in}{1.111775in}}%
\pgfpathlineto{\pgfqpoint{2.061488in}{1.360106in}}%
\pgfpathclose%
\pgfusepath{fill}%
\end{pgfscope}%
\begin{pgfscope}%
\pgfpathrectangle{\pgfqpoint{0.254231in}{0.147348in}}{\pgfqpoint{2.735294in}{2.735294in}}%
\pgfusepath{clip}%
\pgfsetbuttcap%
\pgfsetroundjoin%
\definecolor{currentfill}{rgb}{0.050011,0.191979,0.290388}%
\pgfsetfillcolor{currentfill}%
\pgfsetlinewidth{0.000000pt}%
\definecolor{currentstroke}{rgb}{0.000000,0.000000,0.000000}%
\pgfsetstrokecolor{currentstroke}%
\pgfsetdash{}{0pt}%
\pgfpathmoveto{\pgfqpoint{0.935572in}{1.589448in}}%
\pgfpathlineto{\pgfqpoint{0.926707in}{1.171211in}}%
\pgfpathlineto{\pgfqpoint{1.132195in}{1.562087in}}%
\pgfpathlineto{\pgfqpoint{0.935572in}{1.589448in}}%
\pgfpathclose%
\pgfusepath{fill}%
\end{pgfscope}%
\begin{pgfscope}%
\pgfpathrectangle{\pgfqpoint{0.254231in}{0.147348in}}{\pgfqpoint{2.735294in}{2.735294in}}%
\pgfusepath{clip}%
\pgfsetbuttcap%
\pgfsetroundjoin%
\definecolor{currentfill}{rgb}{0.050011,0.191979,0.290388}%
\pgfsetfillcolor{currentfill}%
\pgfsetlinewidth{0.000000pt}%
\definecolor{currentstroke}{rgb}{0.000000,0.000000,0.000000}%
\pgfsetstrokecolor{currentstroke}%
\pgfsetdash{}{0pt}%
\pgfpathmoveto{\pgfqpoint{2.185489in}{1.562087in}}%
\pgfpathlineto{\pgfqpoint{2.390977in}{1.171211in}}%
\pgfpathlineto{\pgfqpoint{2.382112in}{1.589448in}}%
\pgfpathlineto{\pgfqpoint{2.185489in}{1.562087in}}%
\pgfpathclose%
\pgfusepath{fill}%
\end{pgfscope}%
\begin{pgfscope}%
\pgfpathrectangle{\pgfqpoint{0.254231in}{0.147348in}}{\pgfqpoint{2.735294in}{2.735294in}}%
\pgfusepath{clip}%
\pgfsetbuttcap%
\pgfsetroundjoin%
\definecolor{currentfill}{rgb}{0.049941,0.191710,0.289982}%
\pgfsetfillcolor{currentfill}%
\pgfsetlinewidth{0.000000pt}%
\definecolor{currentstroke}{rgb}{0.000000,0.000000,0.000000}%
\pgfsetstrokecolor{currentstroke}%
\pgfsetdash{}{0pt}%
\pgfpathmoveto{\pgfqpoint{2.020098in}{1.063020in}}%
\pgfpathlineto{\pgfqpoint{2.093438in}{0.985051in}}%
\pgfpathlineto{\pgfqpoint{2.061488in}{1.360106in}}%
\pgfpathlineto{\pgfqpoint{2.020098in}{1.063020in}}%
\pgfpathclose%
\pgfusepath{fill}%
\end{pgfscope}%
\begin{pgfscope}%
\pgfpathrectangle{\pgfqpoint{0.254231in}{0.147348in}}{\pgfqpoint{2.735294in}{2.735294in}}%
\pgfusepath{clip}%
\pgfsetbuttcap%
\pgfsetroundjoin%
\definecolor{currentfill}{rgb}{0.049941,0.191710,0.289982}%
\pgfsetfillcolor{currentfill}%
\pgfsetlinewidth{0.000000pt}%
\definecolor{currentstroke}{rgb}{0.000000,0.000000,0.000000}%
\pgfsetstrokecolor{currentstroke}%
\pgfsetdash{}{0pt}%
\pgfpathmoveto{\pgfqpoint{1.256196in}{1.360106in}}%
\pgfpathlineto{\pgfqpoint{1.224246in}{0.985051in}}%
\pgfpathlineto{\pgfqpoint{1.297586in}{1.063020in}}%
\pgfpathlineto{\pgfqpoint{1.256196in}{1.360106in}}%
\pgfpathclose%
\pgfusepath{fill}%
\end{pgfscope}%
\begin{pgfscope}%
\pgfpathrectangle{\pgfqpoint{0.254231in}{0.147348in}}{\pgfqpoint{2.735294in}{2.735294in}}%
\pgfusepath{clip}%
\pgfsetbuttcap%
\pgfsetroundjoin%
\definecolor{currentfill}{rgb}{0.078663,0.301965,0.456754}%
\pgfsetfillcolor{currentfill}%
\pgfsetlinewidth{0.000000pt}%
\definecolor{currentstroke}{rgb}{0.000000,0.000000,0.000000}%
\pgfsetstrokecolor{currentstroke}%
\pgfsetdash{}{0pt}%
\pgfpathmoveto{\pgfqpoint{1.534055in}{2.124855in}}%
\pgfpathlineto{\pgfqpoint{1.223147in}{2.253326in}}%
\pgfpathlineto{\pgfqpoint{1.296827in}{2.116981in}}%
\pgfpathlineto{\pgfqpoint{1.534055in}{2.124855in}}%
\pgfpathclose%
\pgfusepath{fill}%
\end{pgfscope}%
\begin{pgfscope}%
\pgfpathrectangle{\pgfqpoint{0.254231in}{0.147348in}}{\pgfqpoint{2.735294in}{2.735294in}}%
\pgfusepath{clip}%
\pgfsetbuttcap%
\pgfsetroundjoin%
\definecolor{currentfill}{rgb}{0.078663,0.301965,0.456754}%
\pgfsetfillcolor{currentfill}%
\pgfsetlinewidth{0.000000pt}%
\definecolor{currentstroke}{rgb}{0.000000,0.000000,0.000000}%
\pgfsetstrokecolor{currentstroke}%
\pgfsetdash{}{0pt}%
\pgfpathmoveto{\pgfqpoint{2.020857in}{2.116981in}}%
\pgfpathlineto{\pgfqpoint{2.094537in}{2.253326in}}%
\pgfpathlineto{\pgfqpoint{1.783629in}{2.124855in}}%
\pgfpathlineto{\pgfqpoint{2.020857in}{2.116981in}}%
\pgfpathclose%
\pgfusepath{fill}%
\end{pgfscope}%
\begin{pgfscope}%
\pgfpathrectangle{\pgfqpoint{0.254231in}{0.147348in}}{\pgfqpoint{2.735294in}{2.735294in}}%
\pgfusepath{clip}%
\pgfsetbuttcap%
\pgfsetroundjoin%
\definecolor{currentfill}{rgb}{0.064954,0.249341,0.377155}%
\pgfsetfillcolor{currentfill}%
\pgfsetlinewidth{0.000000pt}%
\definecolor{currentstroke}{rgb}{0.000000,0.000000,0.000000}%
\pgfsetstrokecolor{currentstroke}%
\pgfsetdash{}{0pt}%
\pgfpathmoveto{\pgfqpoint{1.132195in}{1.562087in}}%
\pgfpathlineto{\pgfqpoint{1.091293in}{2.103280in}}%
\pgfpathlineto{\pgfqpoint{0.935572in}{1.589448in}}%
\pgfpathlineto{\pgfqpoint{1.132195in}{1.562087in}}%
\pgfpathclose%
\pgfusepath{fill}%
\end{pgfscope}%
\begin{pgfscope}%
\pgfpathrectangle{\pgfqpoint{0.254231in}{0.147348in}}{\pgfqpoint{2.735294in}{2.735294in}}%
\pgfusepath{clip}%
\pgfsetbuttcap%
\pgfsetroundjoin%
\definecolor{currentfill}{rgb}{0.064954,0.249341,0.377155}%
\pgfsetfillcolor{currentfill}%
\pgfsetlinewidth{0.000000pt}%
\definecolor{currentstroke}{rgb}{0.000000,0.000000,0.000000}%
\pgfsetstrokecolor{currentstroke}%
\pgfsetdash{}{0pt}%
\pgfpathmoveto{\pgfqpoint{2.382112in}{1.589448in}}%
\pgfpathlineto{\pgfqpoint{2.226391in}{2.103280in}}%
\pgfpathlineto{\pgfqpoint{2.185489in}{1.562087in}}%
\pgfpathlineto{\pgfqpoint{2.382112in}{1.589448in}}%
\pgfpathclose%
\pgfusepath{fill}%
\end{pgfscope}%
\begin{pgfscope}%
\pgfpathrectangle{\pgfqpoint{0.254231in}{0.147348in}}{\pgfqpoint{2.735294in}{2.735294in}}%
\pgfusepath{clip}%
\pgfsetbuttcap%
\pgfsetroundjoin%
\definecolor{currentfill}{rgb}{0.042669,0.163794,0.247755}%
\pgfsetfillcolor{currentfill}%
\pgfsetlinewidth{0.000000pt}%
\definecolor{currentstroke}{rgb}{0.000000,0.000000,0.000000}%
\pgfsetstrokecolor{currentstroke}%
\pgfsetdash{}{0pt}%
\pgfpathmoveto{\pgfqpoint{1.432345in}{0.946905in}}%
\pgfpathlineto{\pgfqpoint{1.518932in}{1.337788in}}%
\pgfpathlineto{\pgfqpoint{1.297586in}{1.063020in}}%
\pgfpathlineto{\pgfqpoint{1.432345in}{0.946905in}}%
\pgfpathclose%
\pgfusepath{fill}%
\end{pgfscope}%
\begin{pgfscope}%
\pgfpathrectangle{\pgfqpoint{0.254231in}{0.147348in}}{\pgfqpoint{2.735294in}{2.735294in}}%
\pgfusepath{clip}%
\pgfsetbuttcap%
\pgfsetroundjoin%
\definecolor{currentfill}{rgb}{0.042669,0.163794,0.247755}%
\pgfsetfillcolor{currentfill}%
\pgfsetlinewidth{0.000000pt}%
\definecolor{currentstroke}{rgb}{0.000000,0.000000,0.000000}%
\pgfsetstrokecolor{currentstroke}%
\pgfsetdash{}{0pt}%
\pgfpathmoveto{\pgfqpoint{2.020098in}{1.063020in}}%
\pgfpathlineto{\pgfqpoint{1.798752in}{1.337788in}}%
\pgfpathlineto{\pgfqpoint{1.885339in}{0.946905in}}%
\pgfpathlineto{\pgfqpoint{2.020098in}{1.063020in}}%
\pgfpathclose%
\pgfusepath{fill}%
\end{pgfscope}%
\begin{pgfscope}%
\pgfpathrectangle{\pgfqpoint{0.254231in}{0.147348in}}{\pgfqpoint{2.735294in}{2.735294in}}%
\pgfusepath{clip}%
\pgfsetbuttcap%
\pgfsetroundjoin%
\definecolor{currentfill}{rgb}{0.068541,0.263111,0.397982}%
\pgfsetfillcolor{currentfill}%
\pgfsetlinewidth{0.000000pt}%
\definecolor{currentstroke}{rgb}{0.000000,0.000000,0.000000}%
\pgfsetstrokecolor{currentstroke}%
\pgfsetdash{}{0pt}%
\pgfpathmoveto{\pgfqpoint{1.296827in}{2.116981in}}%
\pgfpathlineto{\pgfqpoint{1.091293in}{2.103280in}}%
\pgfpathlineto{\pgfqpoint{1.395317in}{1.946079in}}%
\pgfpathlineto{\pgfqpoint{1.296827in}{2.116981in}}%
\pgfpathclose%
\pgfusepath{fill}%
\end{pgfscope}%
\begin{pgfscope}%
\pgfpathrectangle{\pgfqpoint{0.254231in}{0.147348in}}{\pgfqpoint{2.735294in}{2.735294in}}%
\pgfusepath{clip}%
\pgfsetbuttcap%
\pgfsetroundjoin%
\definecolor{currentfill}{rgb}{0.068541,0.263111,0.397982}%
\pgfsetfillcolor{currentfill}%
\pgfsetlinewidth{0.000000pt}%
\definecolor{currentstroke}{rgb}{0.000000,0.000000,0.000000}%
\pgfsetstrokecolor{currentstroke}%
\pgfsetdash{}{0pt}%
\pgfpathmoveto{\pgfqpoint{1.922367in}{1.946079in}}%
\pgfpathlineto{\pgfqpoint{2.226391in}{2.103280in}}%
\pgfpathlineto{\pgfqpoint{2.020857in}{2.116981in}}%
\pgfpathlineto{\pgfqpoint{1.922367in}{1.946079in}}%
\pgfpathclose%
\pgfusepath{fill}%
\end{pgfscope}%
\begin{pgfscope}%
\pgfpathrectangle{\pgfqpoint{0.254231in}{0.147348in}}{\pgfqpoint{2.735294in}{2.735294in}}%
\pgfusepath{clip}%
\pgfsetbuttcap%
\pgfsetroundjoin%
\definecolor{currentfill}{rgb}{0.047555,0.182548,0.276123}%
\pgfsetfillcolor{currentfill}%
\pgfsetlinewidth{0.000000pt}%
\definecolor{currentstroke}{rgb}{0.000000,0.000000,0.000000}%
\pgfsetstrokecolor{currentstroke}%
\pgfsetdash{}{0pt}%
\pgfpathmoveto{\pgfqpoint{1.432345in}{0.946905in}}%
\pgfpathlineto{\pgfqpoint{1.534325in}{1.035006in}}%
\pgfpathlineto{\pgfqpoint{1.518932in}{1.337788in}}%
\pgfpathlineto{\pgfqpoint{1.432345in}{0.946905in}}%
\pgfpathclose%
\pgfusepath{fill}%
\end{pgfscope}%
\begin{pgfscope}%
\pgfpathrectangle{\pgfqpoint{0.254231in}{0.147348in}}{\pgfqpoint{2.735294in}{2.735294in}}%
\pgfusepath{clip}%
\pgfsetbuttcap%
\pgfsetroundjoin%
\definecolor{currentfill}{rgb}{0.047555,0.182548,0.276123}%
\pgfsetfillcolor{currentfill}%
\pgfsetlinewidth{0.000000pt}%
\definecolor{currentstroke}{rgb}{0.000000,0.000000,0.000000}%
\pgfsetstrokecolor{currentstroke}%
\pgfsetdash{}{0pt}%
\pgfpathmoveto{\pgfqpoint{1.798752in}{1.337788in}}%
\pgfpathlineto{\pgfqpoint{1.783359in}{1.035006in}}%
\pgfpathlineto{\pgfqpoint{1.885339in}{0.946905in}}%
\pgfpathlineto{\pgfqpoint{1.798752in}{1.337788in}}%
\pgfpathclose%
\pgfusepath{fill}%
\end{pgfscope}%
\begin{pgfscope}%
\pgfpathrectangle{\pgfqpoint{0.254231in}{0.147348in}}{\pgfqpoint{2.735294in}{2.735294in}}%
\pgfusepath{clip}%
\pgfsetbuttcap%
\pgfsetroundjoin%
\definecolor{currentfill}{rgb}{0.046101,0.176968,0.267683}%
\pgfsetfillcolor{currentfill}%
\pgfsetlinewidth{0.000000pt}%
\definecolor{currentstroke}{rgb}{0.000000,0.000000,0.000000}%
\pgfsetstrokecolor{currentstroke}%
\pgfsetdash{}{0pt}%
\pgfpathmoveto{\pgfqpoint{1.658842in}{0.933095in}}%
\pgfpathlineto{\pgfqpoint{1.518932in}{1.337788in}}%
\pgfpathlineto{\pgfqpoint{1.534325in}{1.035006in}}%
\pgfpathlineto{\pgfqpoint{1.658842in}{0.933095in}}%
\pgfpathclose%
\pgfusepath{fill}%
\end{pgfscope}%
\begin{pgfscope}%
\pgfpathrectangle{\pgfqpoint{0.254231in}{0.147348in}}{\pgfqpoint{2.735294in}{2.735294in}}%
\pgfusepath{clip}%
\pgfsetbuttcap%
\pgfsetroundjoin%
\definecolor{currentfill}{rgb}{0.046101,0.176968,0.267683}%
\pgfsetfillcolor{currentfill}%
\pgfsetlinewidth{0.000000pt}%
\definecolor{currentstroke}{rgb}{0.000000,0.000000,0.000000}%
\pgfsetstrokecolor{currentstroke}%
\pgfsetdash{}{0pt}%
\pgfpathmoveto{\pgfqpoint{1.783359in}{1.035006in}}%
\pgfpathlineto{\pgfqpoint{1.798752in}{1.337788in}}%
\pgfpathlineto{\pgfqpoint{1.658842in}{0.933095in}}%
\pgfpathlineto{\pgfqpoint{1.783359in}{1.035006in}}%
\pgfpathclose%
\pgfusepath{fill}%
\end{pgfscope}%
\begin{pgfscope}%
\pgfpathrectangle{\pgfqpoint{0.254231in}{0.147348in}}{\pgfqpoint{2.735294in}{2.735294in}}%
\pgfusepath{clip}%
\pgfsetbuttcap%
\pgfsetroundjoin%
\definecolor{currentfill}{rgb}{0.051850,0.199036,0.301063}%
\pgfsetfillcolor{currentfill}%
\pgfsetlinewidth{0.000000pt}%
\definecolor{currentstroke}{rgb}{0.000000,0.000000,0.000000}%
\pgfsetstrokecolor{currentstroke}%
\pgfsetdash{}{0pt}%
\pgfpathmoveto{\pgfqpoint{1.092413in}{1.111775in}}%
\pgfpathlineto{\pgfqpoint{1.256196in}{1.360106in}}%
\pgfpathlineto{\pgfqpoint{1.132195in}{1.562087in}}%
\pgfpathlineto{\pgfqpoint{1.092413in}{1.111775in}}%
\pgfpathclose%
\pgfusepath{fill}%
\end{pgfscope}%
\begin{pgfscope}%
\pgfpathrectangle{\pgfqpoint{0.254231in}{0.147348in}}{\pgfqpoint{2.735294in}{2.735294in}}%
\pgfusepath{clip}%
\pgfsetbuttcap%
\pgfsetroundjoin%
\definecolor{currentfill}{rgb}{0.051850,0.199036,0.301063}%
\pgfsetfillcolor{currentfill}%
\pgfsetlinewidth{0.000000pt}%
\definecolor{currentstroke}{rgb}{0.000000,0.000000,0.000000}%
\pgfsetstrokecolor{currentstroke}%
\pgfsetdash{}{0pt}%
\pgfpathmoveto{\pgfqpoint{2.185489in}{1.562087in}}%
\pgfpathlineto{\pgfqpoint{2.061488in}{1.360106in}}%
\pgfpathlineto{\pgfqpoint{2.225271in}{1.111775in}}%
\pgfpathlineto{\pgfqpoint{2.185489in}{1.562087in}}%
\pgfpathclose%
\pgfusepath{fill}%
\end{pgfscope}%
\begin{pgfscope}%
\pgfpathrectangle{\pgfqpoint{0.254231in}{0.147348in}}{\pgfqpoint{2.735294in}{2.735294in}}%
\pgfusepath{clip}%
\pgfsetbuttcap%
\pgfsetroundjoin%
\definecolor{currentfill}{rgb}{0.064759,0.248590,0.376018}%
\pgfsetfillcolor{currentfill}%
\pgfsetlinewidth{0.000000pt}%
\definecolor{currentstroke}{rgb}{0.000000,0.000000,0.000000}%
\pgfsetstrokecolor{currentstroke}%
\pgfsetdash{}{0pt}%
\pgfpathmoveto{\pgfqpoint{2.226391in}{2.103280in}}%
\pgfpathlineto{\pgfqpoint{1.922367in}{1.946079in}}%
\pgfpathlineto{\pgfqpoint{2.185489in}{1.562087in}}%
\pgfpathlineto{\pgfqpoint{2.226391in}{2.103280in}}%
\pgfpathclose%
\pgfusepath{fill}%
\end{pgfscope}%
\begin{pgfscope}%
\pgfpathrectangle{\pgfqpoint{0.254231in}{0.147348in}}{\pgfqpoint{2.735294in}{2.735294in}}%
\pgfusepath{clip}%
\pgfsetbuttcap%
\pgfsetroundjoin%
\definecolor{currentfill}{rgb}{0.064759,0.248590,0.376018}%
\pgfsetfillcolor{currentfill}%
\pgfsetlinewidth{0.000000pt}%
\definecolor{currentstroke}{rgb}{0.000000,0.000000,0.000000}%
\pgfsetstrokecolor{currentstroke}%
\pgfsetdash{}{0pt}%
\pgfpathmoveto{\pgfqpoint{1.132195in}{1.562087in}}%
\pgfpathlineto{\pgfqpoint{1.395317in}{1.946079in}}%
\pgfpathlineto{\pgfqpoint{1.091293in}{2.103280in}}%
\pgfpathlineto{\pgfqpoint{1.132195in}{1.562087in}}%
\pgfpathclose%
\pgfusepath{fill}%
\end{pgfscope}%
\begin{pgfscope}%
\pgfpathrectangle{\pgfqpoint{0.254231in}{0.147348in}}{\pgfqpoint{2.735294in}{2.735294in}}%
\pgfusepath{clip}%
\pgfsetbuttcap%
\pgfsetroundjoin%
\definecolor{currentfill}{rgb}{0.071694,0.275212,0.416288}%
\pgfsetfillcolor{currentfill}%
\pgfsetlinewidth{0.000000pt}%
\definecolor{currentstroke}{rgb}{0.000000,0.000000,0.000000}%
\pgfsetstrokecolor{currentstroke}%
\pgfsetdash{}{0pt}%
\pgfpathmoveto{\pgfqpoint{1.395317in}{1.946079in}}%
\pgfpathlineto{\pgfqpoint{1.534055in}{2.124855in}}%
\pgfpathlineto{\pgfqpoint{1.296827in}{2.116981in}}%
\pgfpathlineto{\pgfqpoint{1.395317in}{1.946079in}}%
\pgfpathclose%
\pgfusepath{fill}%
\end{pgfscope}%
\begin{pgfscope}%
\pgfpathrectangle{\pgfqpoint{0.254231in}{0.147348in}}{\pgfqpoint{2.735294in}{2.735294in}}%
\pgfusepath{clip}%
\pgfsetbuttcap%
\pgfsetroundjoin%
\definecolor{currentfill}{rgb}{0.071694,0.275212,0.416288}%
\pgfsetfillcolor{currentfill}%
\pgfsetlinewidth{0.000000pt}%
\definecolor{currentstroke}{rgb}{0.000000,0.000000,0.000000}%
\pgfsetstrokecolor{currentstroke}%
\pgfsetdash{}{0pt}%
\pgfpathmoveto{\pgfqpoint{2.020857in}{2.116981in}}%
\pgfpathlineto{\pgfqpoint{1.783629in}{2.124855in}}%
\pgfpathlineto{\pgfqpoint{1.922367in}{1.946079in}}%
\pgfpathlineto{\pgfqpoint{2.020857in}{2.116981in}}%
\pgfpathclose%
\pgfusepath{fill}%
\end{pgfscope}%
\begin{pgfscope}%
\pgfpathrectangle{\pgfqpoint{0.254231in}{0.147348in}}{\pgfqpoint{2.735294in}{2.735294in}}%
\pgfusepath{clip}%
\pgfsetbuttcap%
\pgfsetroundjoin%
\definecolor{currentfill}{rgb}{0.071636,0.274990,0.415951}%
\pgfsetfillcolor{currentfill}%
\pgfsetlinewidth{0.000000pt}%
\definecolor{currentstroke}{rgb}{0.000000,0.000000,0.000000}%
\pgfsetstrokecolor{currentstroke}%
\pgfsetdash{}{0pt}%
\pgfpathmoveto{\pgfqpoint{1.658842in}{1.947266in}}%
\pgfpathlineto{\pgfqpoint{1.783629in}{2.124855in}}%
\pgfpathlineto{\pgfqpoint{1.534055in}{2.124855in}}%
\pgfpathlineto{\pgfqpoint{1.658842in}{1.947266in}}%
\pgfpathclose%
\pgfusepath{fill}%
\end{pgfscope}%
\begin{pgfscope}%
\pgfpathrectangle{\pgfqpoint{0.254231in}{0.147348in}}{\pgfqpoint{2.735294in}{2.735294in}}%
\pgfusepath{clip}%
\pgfsetbuttcap%
\pgfsetroundjoin%
\definecolor{currentfill}{rgb}{0.045820,0.175891,0.266053}%
\pgfsetfillcolor{currentfill}%
\pgfsetlinewidth{0.000000pt}%
\definecolor{currentstroke}{rgb}{0.000000,0.000000,0.000000}%
\pgfsetstrokecolor{currentstroke}%
\pgfsetdash{}{0pt}%
\pgfpathmoveto{\pgfqpoint{1.297586in}{1.063020in}}%
\pgfpathlineto{\pgfqpoint{1.378909in}{1.541647in}}%
\pgfpathlineto{\pgfqpoint{1.256196in}{1.360106in}}%
\pgfpathlineto{\pgfqpoint{1.297586in}{1.063020in}}%
\pgfpathclose%
\pgfusepath{fill}%
\end{pgfscope}%
\begin{pgfscope}%
\pgfpathrectangle{\pgfqpoint{0.254231in}{0.147348in}}{\pgfqpoint{2.735294in}{2.735294in}}%
\pgfusepath{clip}%
\pgfsetbuttcap%
\pgfsetroundjoin%
\definecolor{currentfill}{rgb}{0.045820,0.175891,0.266053}%
\pgfsetfillcolor{currentfill}%
\pgfsetlinewidth{0.000000pt}%
\definecolor{currentstroke}{rgb}{0.000000,0.000000,0.000000}%
\pgfsetstrokecolor{currentstroke}%
\pgfsetdash{}{0pt}%
\pgfpathmoveto{\pgfqpoint{2.061488in}{1.360106in}}%
\pgfpathlineto{\pgfqpoint{1.938775in}{1.541647in}}%
\pgfpathlineto{\pgfqpoint{2.020098in}{1.063020in}}%
\pgfpathlineto{\pgfqpoint{2.061488in}{1.360106in}}%
\pgfpathclose%
\pgfusepath{fill}%
\end{pgfscope}%
\begin{pgfscope}%
\pgfpathrectangle{\pgfqpoint{0.254231in}{0.147348in}}{\pgfqpoint{2.735294in}{2.735294in}}%
\pgfusepath{clip}%
\pgfsetbuttcap%
\pgfsetroundjoin%
\definecolor{currentfill}{rgb}{0.046814,0.179706,0.271825}%
\pgfsetfillcolor{currentfill}%
\pgfsetlinewidth{0.000000pt}%
\definecolor{currentstroke}{rgb}{0.000000,0.000000,0.000000}%
\pgfsetstrokecolor{currentstroke}%
\pgfsetdash{}{0pt}%
\pgfpathmoveto{\pgfqpoint{1.658842in}{1.533948in}}%
\pgfpathlineto{\pgfqpoint{1.658842in}{0.933095in}}%
\pgfpathlineto{\pgfqpoint{1.798752in}{1.337788in}}%
\pgfpathlineto{\pgfqpoint{1.658842in}{1.533948in}}%
\pgfpathclose%
\pgfusepath{fill}%
\end{pgfscope}%
\begin{pgfscope}%
\pgfpathrectangle{\pgfqpoint{0.254231in}{0.147348in}}{\pgfqpoint{2.735294in}{2.735294in}}%
\pgfusepath{clip}%
\pgfsetbuttcap%
\pgfsetroundjoin%
\definecolor{currentfill}{rgb}{0.046814,0.179706,0.271825}%
\pgfsetfillcolor{currentfill}%
\pgfsetlinewidth{0.000000pt}%
\definecolor{currentstroke}{rgb}{0.000000,0.000000,0.000000}%
\pgfsetstrokecolor{currentstroke}%
\pgfsetdash{}{0pt}%
\pgfpathmoveto{\pgfqpoint{1.518932in}{1.337788in}}%
\pgfpathlineto{\pgfqpoint{1.658842in}{0.933095in}}%
\pgfpathlineto{\pgfqpoint{1.658842in}{1.533948in}}%
\pgfpathlineto{\pgfqpoint{1.518932in}{1.337788in}}%
\pgfpathclose%
\pgfusepath{fill}%
\end{pgfscope}%
\begin{pgfscope}%
\pgfpathrectangle{\pgfqpoint{0.254231in}{0.147348in}}{\pgfqpoint{2.735294in}{2.735294in}}%
\pgfusepath{clip}%
\pgfsetbuttcap%
\pgfsetroundjoin%
\definecolor{currentfill}{rgb}{0.069261,0.265872,0.402159}%
\pgfsetfillcolor{currentfill}%
\pgfsetlinewidth{0.000000pt}%
\definecolor{currentstroke}{rgb}{0.000000,0.000000,0.000000}%
\pgfsetstrokecolor{currentstroke}%
\pgfsetdash{}{0pt}%
\pgfpathmoveto{\pgfqpoint{1.922367in}{1.946079in}}%
\pgfpathlineto{\pgfqpoint{1.783629in}{2.124855in}}%
\pgfpathlineto{\pgfqpoint{1.658842in}{1.947266in}}%
\pgfpathlineto{\pgfqpoint{1.922367in}{1.946079in}}%
\pgfpathclose%
\pgfusepath{fill}%
\end{pgfscope}%
\begin{pgfscope}%
\pgfpathrectangle{\pgfqpoint{0.254231in}{0.147348in}}{\pgfqpoint{2.735294in}{2.735294in}}%
\pgfusepath{clip}%
\pgfsetbuttcap%
\pgfsetroundjoin%
\definecolor{currentfill}{rgb}{0.069261,0.265872,0.402159}%
\pgfsetfillcolor{currentfill}%
\pgfsetlinewidth{0.000000pt}%
\definecolor{currentstroke}{rgb}{0.000000,0.000000,0.000000}%
\pgfsetstrokecolor{currentstroke}%
\pgfsetdash{}{0pt}%
\pgfpathmoveto{\pgfqpoint{1.658842in}{1.947266in}}%
\pgfpathlineto{\pgfqpoint{1.534055in}{2.124855in}}%
\pgfpathlineto{\pgfqpoint{1.395317in}{1.946079in}}%
\pgfpathlineto{\pgfqpoint{1.658842in}{1.947266in}}%
\pgfpathclose%
\pgfusepath{fill}%
\end{pgfscope}%
\begin{pgfscope}%
\pgfpathrectangle{\pgfqpoint{0.254231in}{0.147348in}}{\pgfqpoint{2.735294in}{2.735294in}}%
\pgfusepath{clip}%
\pgfsetbuttcap%
\pgfsetroundjoin%
\definecolor{currentfill}{rgb}{0.049465,0.189883,0.287218}%
\pgfsetfillcolor{currentfill}%
\pgfsetlinewidth{0.000000pt}%
\definecolor{currentstroke}{rgb}{0.000000,0.000000,0.000000}%
\pgfsetstrokecolor{currentstroke}%
\pgfsetdash{}{0pt}%
\pgfpathmoveto{\pgfqpoint{1.297586in}{1.063020in}}%
\pgfpathlineto{\pgfqpoint{1.518932in}{1.337788in}}%
\pgfpathlineto{\pgfqpoint{1.378909in}{1.541647in}}%
\pgfpathlineto{\pgfqpoint{1.297586in}{1.063020in}}%
\pgfpathclose%
\pgfusepath{fill}%
\end{pgfscope}%
\begin{pgfscope}%
\pgfpathrectangle{\pgfqpoint{0.254231in}{0.147348in}}{\pgfqpoint{2.735294in}{2.735294in}}%
\pgfusepath{clip}%
\pgfsetbuttcap%
\pgfsetroundjoin%
\definecolor{currentfill}{rgb}{0.049465,0.189883,0.287218}%
\pgfsetfillcolor{currentfill}%
\pgfsetlinewidth{0.000000pt}%
\definecolor{currentstroke}{rgb}{0.000000,0.000000,0.000000}%
\pgfsetstrokecolor{currentstroke}%
\pgfsetdash{}{0pt}%
\pgfpathmoveto{\pgfqpoint{1.938775in}{1.541647in}}%
\pgfpathlineto{\pgfqpoint{1.798752in}{1.337788in}}%
\pgfpathlineto{\pgfqpoint{2.020098in}{1.063020in}}%
\pgfpathlineto{\pgfqpoint{1.938775in}{1.541647in}}%
\pgfpathclose%
\pgfusepath{fill}%
\end{pgfscope}%
\begin{pgfscope}%
\pgfpathrectangle{\pgfqpoint{0.254231in}{0.147348in}}{\pgfqpoint{2.735294in}{2.735294in}}%
\pgfusepath{clip}%
\pgfsetbuttcap%
\pgfsetroundjoin%
\definecolor{currentfill}{rgb}{0.061576,0.236373,0.357539}%
\pgfsetfillcolor{currentfill}%
\pgfsetlinewidth{0.000000pt}%
\definecolor{currentstroke}{rgb}{0.000000,0.000000,0.000000}%
\pgfsetstrokecolor{currentstroke}%
\pgfsetdash{}{0pt}%
\pgfpathmoveto{\pgfqpoint{1.378909in}{1.541647in}}%
\pgfpathlineto{\pgfqpoint{1.395317in}{1.946079in}}%
\pgfpathlineto{\pgfqpoint{1.132195in}{1.562087in}}%
\pgfpathlineto{\pgfqpoint{1.378909in}{1.541647in}}%
\pgfpathclose%
\pgfusepath{fill}%
\end{pgfscope}%
\begin{pgfscope}%
\pgfpathrectangle{\pgfqpoint{0.254231in}{0.147348in}}{\pgfqpoint{2.735294in}{2.735294in}}%
\pgfusepath{clip}%
\pgfsetbuttcap%
\pgfsetroundjoin%
\definecolor{currentfill}{rgb}{0.061576,0.236373,0.357539}%
\pgfsetfillcolor{currentfill}%
\pgfsetlinewidth{0.000000pt}%
\definecolor{currentstroke}{rgb}{0.000000,0.000000,0.000000}%
\pgfsetstrokecolor{currentstroke}%
\pgfsetdash{}{0pt}%
\pgfpathmoveto{\pgfqpoint{2.185489in}{1.562087in}}%
\pgfpathlineto{\pgfqpoint{1.922367in}{1.946079in}}%
\pgfpathlineto{\pgfqpoint{1.938775in}{1.541647in}}%
\pgfpathlineto{\pgfqpoint{2.185489in}{1.562087in}}%
\pgfpathclose%
\pgfusepath{fill}%
\end{pgfscope}%
\begin{pgfscope}%
\pgfpathrectangle{\pgfqpoint{0.254231in}{0.147348in}}{\pgfqpoint{2.735294in}{2.735294in}}%
\pgfusepath{clip}%
\pgfsetbuttcap%
\pgfsetroundjoin%
\definecolor{currentfill}{rgb}{0.053541,0.205528,0.310883}%
\pgfsetfillcolor{currentfill}%
\pgfsetlinewidth{0.000000pt}%
\definecolor{currentstroke}{rgb}{0.000000,0.000000,0.000000}%
\pgfsetstrokecolor{currentstroke}%
\pgfsetdash{}{0pt}%
\pgfpathmoveto{\pgfqpoint{1.132195in}{1.562087in}}%
\pgfpathlineto{\pgfqpoint{1.256196in}{1.360106in}}%
\pgfpathlineto{\pgfqpoint{1.378909in}{1.541647in}}%
\pgfpathlineto{\pgfqpoint{1.132195in}{1.562087in}}%
\pgfpathclose%
\pgfusepath{fill}%
\end{pgfscope}%
\begin{pgfscope}%
\pgfpathrectangle{\pgfqpoint{0.254231in}{0.147348in}}{\pgfqpoint{2.735294in}{2.735294in}}%
\pgfusepath{clip}%
\pgfsetbuttcap%
\pgfsetroundjoin%
\definecolor{currentfill}{rgb}{0.053541,0.205528,0.310883}%
\pgfsetfillcolor{currentfill}%
\pgfsetlinewidth{0.000000pt}%
\definecolor{currentstroke}{rgb}{0.000000,0.000000,0.000000}%
\pgfsetstrokecolor{currentstroke}%
\pgfsetdash{}{0pt}%
\pgfpathmoveto{\pgfqpoint{1.938775in}{1.541647in}}%
\pgfpathlineto{\pgfqpoint{2.061488in}{1.360106in}}%
\pgfpathlineto{\pgfqpoint{2.185489in}{1.562087in}}%
\pgfpathlineto{\pgfqpoint{1.938775in}{1.541647in}}%
\pgfpathclose%
\pgfusepath{fill}%
\end{pgfscope}%
\begin{pgfscope}%
\pgfpathrectangle{\pgfqpoint{0.254231in}{0.147348in}}{\pgfqpoint{2.735294in}{2.735294in}}%
\pgfusepath{clip}%
\pgfsetbuttcap%
\pgfsetroundjoin%
\definecolor{currentfill}{rgb}{0.060634,0.232757,0.352069}%
\pgfsetfillcolor{currentfill}%
\pgfsetlinewidth{0.000000pt}%
\definecolor{currentstroke}{rgb}{0.000000,0.000000,0.000000}%
\pgfsetstrokecolor{currentstroke}%
\pgfsetdash{}{0pt}%
\pgfpathmoveto{\pgfqpoint{1.395317in}{1.946079in}}%
\pgfpathlineto{\pgfqpoint{1.378909in}{1.541647in}}%
\pgfpathlineto{\pgfqpoint{1.658842in}{1.947266in}}%
\pgfpathlineto{\pgfqpoint{1.395317in}{1.946079in}}%
\pgfpathclose%
\pgfusepath{fill}%
\end{pgfscope}%
\begin{pgfscope}%
\pgfpathrectangle{\pgfqpoint{0.254231in}{0.147348in}}{\pgfqpoint{2.735294in}{2.735294in}}%
\pgfusepath{clip}%
\pgfsetbuttcap%
\pgfsetroundjoin%
\definecolor{currentfill}{rgb}{0.060634,0.232757,0.352069}%
\pgfsetfillcolor{currentfill}%
\pgfsetlinewidth{0.000000pt}%
\definecolor{currentstroke}{rgb}{0.000000,0.000000,0.000000}%
\pgfsetstrokecolor{currentstroke}%
\pgfsetdash{}{0pt}%
\pgfpathmoveto{\pgfqpoint{1.658842in}{1.947266in}}%
\pgfpathlineto{\pgfqpoint{1.938775in}{1.541647in}}%
\pgfpathlineto{\pgfqpoint{1.922367in}{1.946079in}}%
\pgfpathlineto{\pgfqpoint{1.658842in}{1.947266in}}%
\pgfpathclose%
\pgfusepath{fill}%
\end{pgfscope}%
\begin{pgfscope}%
\pgfpathrectangle{\pgfqpoint{0.254231in}{0.147348in}}{\pgfqpoint{2.735294in}{2.735294in}}%
\pgfusepath{clip}%
\pgfsetbuttcap%
\pgfsetroundjoin%
\definecolor{currentfill}{rgb}{0.060773,0.233289,0.352874}%
\pgfsetfillcolor{currentfill}%
\pgfsetlinewidth{0.000000pt}%
\definecolor{currentstroke}{rgb}{0.000000,0.000000,0.000000}%
\pgfsetstrokecolor{currentstroke}%
\pgfsetdash{}{0pt}%
\pgfpathmoveto{\pgfqpoint{1.658842in}{1.533948in}}%
\pgfpathlineto{\pgfqpoint{1.658842in}{1.947266in}}%
\pgfpathlineto{\pgfqpoint{1.378909in}{1.541647in}}%
\pgfpathlineto{\pgfqpoint{1.658842in}{1.533948in}}%
\pgfpathclose%
\pgfusepath{fill}%
\end{pgfscope}%
\begin{pgfscope}%
\pgfpathrectangle{\pgfqpoint{0.254231in}{0.147348in}}{\pgfqpoint{2.735294in}{2.735294in}}%
\pgfusepath{clip}%
\pgfsetbuttcap%
\pgfsetroundjoin%
\definecolor{currentfill}{rgb}{0.060773,0.233289,0.352874}%
\pgfsetfillcolor{currentfill}%
\pgfsetlinewidth{0.000000pt}%
\definecolor{currentstroke}{rgb}{0.000000,0.000000,0.000000}%
\pgfsetstrokecolor{currentstroke}%
\pgfsetdash{}{0pt}%
\pgfpathmoveto{\pgfqpoint{1.938775in}{1.541647in}}%
\pgfpathlineto{\pgfqpoint{1.658842in}{1.947266in}}%
\pgfpathlineto{\pgfqpoint{1.658842in}{1.533948in}}%
\pgfpathlineto{\pgfqpoint{1.938775in}{1.541647in}}%
\pgfpathclose%
\pgfusepath{fill}%
\end{pgfscope}%
\begin{pgfscope}%
\pgfpathrectangle{\pgfqpoint{0.254231in}{0.147348in}}{\pgfqpoint{2.735294in}{2.735294in}}%
\pgfusepath{clip}%
\pgfsetbuttcap%
\pgfsetroundjoin%
\definecolor{currentfill}{rgb}{0.052607,0.201942,0.305459}%
\pgfsetfillcolor{currentfill}%
\pgfsetlinewidth{0.000000pt}%
\definecolor{currentstroke}{rgb}{0.000000,0.000000,0.000000}%
\pgfsetstrokecolor{currentstroke}%
\pgfsetdash{}{0pt}%
\pgfpathmoveto{\pgfqpoint{1.378909in}{1.541647in}}%
\pgfpathlineto{\pgfqpoint{1.518932in}{1.337788in}}%
\pgfpathlineto{\pgfqpoint{1.658842in}{1.533948in}}%
\pgfpathlineto{\pgfqpoint{1.378909in}{1.541647in}}%
\pgfpathclose%
\pgfusepath{fill}%
\end{pgfscope}%
\begin{pgfscope}%
\pgfpathrectangle{\pgfqpoint{0.254231in}{0.147348in}}{\pgfqpoint{2.735294in}{2.735294in}}%
\pgfusepath{clip}%
\pgfsetbuttcap%
\pgfsetroundjoin%
\definecolor{currentfill}{rgb}{0.052607,0.201942,0.305459}%
\pgfsetfillcolor{currentfill}%
\pgfsetlinewidth{0.000000pt}%
\definecolor{currentstroke}{rgb}{0.000000,0.000000,0.000000}%
\pgfsetstrokecolor{currentstroke}%
\pgfsetdash{}{0pt}%
\pgfpathmoveto{\pgfqpoint{1.658842in}{1.533948in}}%
\pgfpathlineto{\pgfqpoint{1.798752in}{1.337788in}}%
\pgfpathlineto{\pgfqpoint{1.938775in}{1.541647in}}%
\pgfpathlineto{\pgfqpoint{1.658842in}{1.533948in}}%
\pgfpathclose%
\pgfusepath{fill}%
\end{pgfscope}%
\begin{pgfscope}%
\pgfpathrectangle{\pgfqpoint{0.254231in}{0.147348in}}{\pgfqpoint{2.735294in}{2.735294in}}%
\pgfusepath{clip}%
\pgfsetbuttcap%
\pgfsetroundjoin%
\definecolor{currentfill}{rgb}{0.839216,0.152941,0.156863}%
\pgfsetfillcolor{currentfill}%
\pgfsetfillopacity{0.300000}%
\pgfsetlinewidth{1.003750pt}%
\definecolor{currentstroke}{rgb}{0.839216,0.152941,0.156863}%
\pgfsetstrokecolor{currentstroke}%
\pgfsetstrokeopacity{0.300000}%
\pgfsetdash{}{0pt}%
\pgfpathmoveto{\pgfqpoint{1.065369in}{0.944929in}}%
\pgfpathcurveto{\pgfqpoint{1.075457in}{0.944929in}}{\pgfqpoint{1.085132in}{0.948937in}}{\pgfqpoint{1.092265in}{0.956070in}}%
\pgfpathcurveto{\pgfqpoint{1.099398in}{0.963203in}}{\pgfqpoint{1.103406in}{0.972878in}}{\pgfqpoint{1.103406in}{0.982966in}}%
\pgfpathcurveto{\pgfqpoint{1.103406in}{0.993053in}}{\pgfqpoint{1.099398in}{1.002729in}}{\pgfqpoint{1.092265in}{1.009861in}}%
\pgfpathcurveto{\pgfqpoint{1.085132in}{1.016994in}}{\pgfqpoint{1.075457in}{1.021002in}}{\pgfqpoint{1.065369in}{1.021002in}}%
\pgfpathcurveto{\pgfqpoint{1.055282in}{1.021002in}}{\pgfqpoint{1.045607in}{1.016994in}}{\pgfqpoint{1.038474in}{1.009861in}}%
\pgfpathcurveto{\pgfqpoint{1.031341in}{1.002729in}}{\pgfqpoint{1.027333in}{0.993053in}}{\pgfqpoint{1.027333in}{0.982966in}}%
\pgfpathcurveto{\pgfqpoint{1.027333in}{0.972878in}}{\pgfqpoint{1.031341in}{0.963203in}}{\pgfqpoint{1.038474in}{0.956070in}}%
\pgfpathcurveto{\pgfqpoint{1.045607in}{0.948937in}}{\pgfqpoint{1.055282in}{0.944929in}}{\pgfqpoint{1.065369in}{0.944929in}}%
\pgfpathlineto{\pgfqpoint{1.065369in}{0.944929in}}%
\pgfpathclose%
\pgfusepath{stroke,fill}%
\end{pgfscope}%
\begin{pgfscope}%
\pgfpathrectangle{\pgfqpoint{0.254231in}{0.147348in}}{\pgfqpoint{2.735294in}{2.735294in}}%
\pgfusepath{clip}%
\pgfsetbuttcap%
\pgfsetroundjoin%
\definecolor{currentfill}{rgb}{0.839216,0.152941,0.156863}%
\pgfsetfillcolor{currentfill}%
\pgfsetfillopacity{0.383610}%
\pgfsetlinewidth{1.003750pt}%
\definecolor{currentstroke}{rgb}{0.839216,0.152941,0.156863}%
\pgfsetstrokecolor{currentstroke}%
\pgfsetstrokeopacity{0.383610}%
\pgfsetdash{}{0pt}%
\pgfpathmoveto{\pgfqpoint{1.045506in}{1.009596in}}%
\pgfpathcurveto{\pgfqpoint{1.055594in}{1.009596in}}{\pgfqpoint{1.065269in}{1.013604in}}{\pgfqpoint{1.072402in}{1.020737in}}%
\pgfpathcurveto{\pgfqpoint{1.079535in}{1.027870in}}{\pgfqpoint{1.083543in}{1.037545in}}{\pgfqpoint{1.083543in}{1.047633in}}%
\pgfpathcurveto{\pgfqpoint{1.083543in}{1.057720in}}{\pgfqpoint{1.079535in}{1.067396in}}{\pgfqpoint{1.072402in}{1.074528in}}%
\pgfpathcurveto{\pgfqpoint{1.065269in}{1.081661in}}{\pgfqpoint{1.055594in}{1.085669in}}{\pgfqpoint{1.045506in}{1.085669in}}%
\pgfpathcurveto{\pgfqpoint{1.035419in}{1.085669in}}{\pgfqpoint{1.025743in}{1.081661in}}{\pgfqpoint{1.018610in}{1.074528in}}%
\pgfpathcurveto{\pgfqpoint{1.011478in}{1.067396in}}{\pgfqpoint{1.007470in}{1.057720in}}{\pgfqpoint{1.007470in}{1.047633in}}%
\pgfpathcurveto{\pgfqpoint{1.007470in}{1.037545in}}{\pgfqpoint{1.011478in}{1.027870in}}{\pgfqpoint{1.018610in}{1.020737in}}%
\pgfpathcurveto{\pgfqpoint{1.025743in}{1.013604in}}{\pgfqpoint{1.035419in}{1.009596in}}{\pgfqpoint{1.045506in}{1.009596in}}%
\pgfpathlineto{\pgfqpoint{1.045506in}{1.009596in}}%
\pgfpathclose%
\pgfusepath{stroke,fill}%
\end{pgfscope}%
\begin{pgfscope}%
\pgfpathrectangle{\pgfqpoint{0.254231in}{0.147348in}}{\pgfqpoint{2.735294in}{2.735294in}}%
\pgfusepath{clip}%
\pgfsetbuttcap%
\pgfsetroundjoin%
\definecolor{currentfill}{rgb}{0.839216,0.152941,0.156863}%
\pgfsetfillcolor{currentfill}%
\pgfsetfillopacity{0.457533}%
\pgfsetlinewidth{1.003750pt}%
\definecolor{currentstroke}{rgb}{0.839216,0.152941,0.156863}%
\pgfsetstrokecolor{currentstroke}%
\pgfsetstrokeopacity{0.457533}%
\pgfsetdash{}{0pt}%
\pgfpathmoveto{\pgfqpoint{1.201522in}{0.962534in}}%
\pgfpathcurveto{\pgfqpoint{1.211609in}{0.962534in}}{\pgfqpoint{1.221285in}{0.966542in}}{\pgfqpoint{1.228417in}{0.973675in}}%
\pgfpathcurveto{\pgfqpoint{1.235550in}{0.980808in}}{\pgfqpoint{1.239558in}{0.990483in}}{\pgfqpoint{1.239558in}{1.000570in}}%
\pgfpathcurveto{\pgfqpoint{1.239558in}{1.010658in}}{\pgfqpoint{1.235550in}{1.020333in}}{\pgfqpoint{1.228417in}{1.027466in}}%
\pgfpathcurveto{\pgfqpoint{1.221285in}{1.034599in}}{\pgfqpoint{1.211609in}{1.038607in}}{\pgfqpoint{1.201522in}{1.038607in}}%
\pgfpathcurveto{\pgfqpoint{1.191434in}{1.038607in}}{\pgfqpoint{1.181759in}{1.034599in}}{\pgfqpoint{1.174626in}{1.027466in}}%
\pgfpathcurveto{\pgfqpoint{1.167493in}{1.020333in}}{\pgfqpoint{1.163485in}{1.010658in}}{\pgfqpoint{1.163485in}{1.000570in}}%
\pgfpathcurveto{\pgfqpoint{1.163485in}{0.990483in}}{\pgfqpoint{1.167493in}{0.980808in}}{\pgfqpoint{1.174626in}{0.973675in}}%
\pgfpathcurveto{\pgfqpoint{1.181759in}{0.966542in}}{\pgfqpoint{1.191434in}{0.962534in}}{\pgfqpoint{1.201522in}{0.962534in}}%
\pgfpathlineto{\pgfqpoint{1.201522in}{0.962534in}}%
\pgfpathclose%
\pgfusepath{stroke,fill}%
\end{pgfscope}%
\begin{pgfscope}%
\pgfpathrectangle{\pgfqpoint{0.254231in}{0.147348in}}{\pgfqpoint{2.735294in}{2.735294in}}%
\pgfusepath{clip}%
\pgfsetbuttcap%
\pgfsetroundjoin%
\definecolor{currentfill}{rgb}{0.839216,0.152941,0.156863}%
\pgfsetfillcolor{currentfill}%
\pgfsetfillopacity{0.492303}%
\pgfsetlinewidth{1.003750pt}%
\definecolor{currentstroke}{rgb}{0.839216,0.152941,0.156863}%
\pgfsetstrokecolor{currentstroke}%
\pgfsetstrokeopacity{0.492303}%
\pgfsetdash{}{0pt}%
\pgfpathmoveto{\pgfqpoint{0.878198in}{1.511653in}}%
\pgfpathcurveto{\pgfqpoint{0.888285in}{1.511653in}}{\pgfqpoint{0.897960in}{1.515661in}}{\pgfqpoint{0.905093in}{1.522794in}}%
\pgfpathcurveto{\pgfqpoint{0.912226in}{1.529927in}}{\pgfqpoint{0.916234in}{1.539602in}}{\pgfqpoint{0.916234in}{1.549689in}}%
\pgfpathcurveto{\pgfqpoint{0.916234in}{1.559777in}}{\pgfqpoint{0.912226in}{1.569452in}}{\pgfqpoint{0.905093in}{1.576585in}}%
\pgfpathcurveto{\pgfqpoint{0.897960in}{1.583718in}}{\pgfqpoint{0.888285in}{1.587726in}}{\pgfqpoint{0.878198in}{1.587726in}}%
\pgfpathcurveto{\pgfqpoint{0.868110in}{1.587726in}}{\pgfqpoint{0.858435in}{1.583718in}}{\pgfqpoint{0.851302in}{1.576585in}}%
\pgfpathcurveto{\pgfqpoint{0.844169in}{1.569452in}}{\pgfqpoint{0.840161in}{1.559777in}}{\pgfqpoint{0.840161in}{1.549689in}}%
\pgfpathcurveto{\pgfqpoint{0.840161in}{1.539602in}}{\pgfqpoint{0.844169in}{1.529927in}}{\pgfqpoint{0.851302in}{1.522794in}}%
\pgfpathcurveto{\pgfqpoint{0.858435in}{1.515661in}}{\pgfqpoint{0.868110in}{1.511653in}}{\pgfqpoint{0.878198in}{1.511653in}}%
\pgfpathlineto{\pgfqpoint{0.878198in}{1.511653in}}%
\pgfpathclose%
\pgfusepath{stroke,fill}%
\end{pgfscope}%
\begin{pgfscope}%
\pgfpathrectangle{\pgfqpoint{0.254231in}{0.147348in}}{\pgfqpoint{2.735294in}{2.735294in}}%
\pgfusepath{clip}%
\pgfsetbuttcap%
\pgfsetroundjoin%
\definecolor{currentfill}{rgb}{0.839216,0.152941,0.156863}%
\pgfsetfillcolor{currentfill}%
\pgfsetfillopacity{0.498590}%
\pgfsetlinewidth{1.003750pt}%
\definecolor{currentstroke}{rgb}{0.839216,0.152941,0.156863}%
\pgfsetstrokecolor{currentstroke}%
\pgfsetstrokeopacity{0.498590}%
\pgfsetdash{}{0pt}%
\pgfpathmoveto{\pgfqpoint{2.525473in}{1.621636in}}%
\pgfpathcurveto{\pgfqpoint{2.535560in}{1.621636in}}{\pgfqpoint{2.545236in}{1.625643in}}{\pgfqpoint{2.552368in}{1.632776in}}%
\pgfpathcurveto{\pgfqpoint{2.559501in}{1.639909in}}{\pgfqpoint{2.563509in}{1.649585in}}{\pgfqpoint{2.563509in}{1.659672in}}%
\pgfpathcurveto{\pgfqpoint{2.563509in}{1.669759in}}{\pgfqpoint{2.559501in}{1.679435in}}{\pgfqpoint{2.552368in}{1.686568in}}%
\pgfpathcurveto{\pgfqpoint{2.545236in}{1.693701in}}{\pgfqpoint{2.535560in}{1.697708in}}{\pgfqpoint{2.525473in}{1.697708in}}%
\pgfpathcurveto{\pgfqpoint{2.515385in}{1.697708in}}{\pgfqpoint{2.505710in}{1.693701in}}{\pgfqpoint{2.498577in}{1.686568in}}%
\pgfpathcurveto{\pgfqpoint{2.491444in}{1.679435in}}{\pgfqpoint{2.487436in}{1.669759in}}{\pgfqpoint{2.487436in}{1.659672in}}%
\pgfpathcurveto{\pgfqpoint{2.487436in}{1.649585in}}{\pgfqpoint{2.491444in}{1.639909in}}{\pgfqpoint{2.498577in}{1.632776in}}%
\pgfpathcurveto{\pgfqpoint{2.505710in}{1.625643in}}{\pgfqpoint{2.515385in}{1.621636in}}{\pgfqpoint{2.525473in}{1.621636in}}%
\pgfpathlineto{\pgfqpoint{2.525473in}{1.621636in}}%
\pgfpathclose%
\pgfusepath{stroke,fill}%
\end{pgfscope}%
\begin{pgfscope}%
\pgfpathrectangle{\pgfqpoint{0.254231in}{0.147348in}}{\pgfqpoint{2.735294in}{2.735294in}}%
\pgfusepath{clip}%
\pgfsetbuttcap%
\pgfsetroundjoin%
\definecolor{currentfill}{rgb}{0.839216,0.152941,0.156863}%
\pgfsetfillcolor{currentfill}%
\pgfsetfillopacity{0.612876}%
\pgfsetlinewidth{1.003750pt}%
\definecolor{currentstroke}{rgb}{0.839216,0.152941,0.156863}%
\pgfsetstrokecolor{currentstroke}%
\pgfsetstrokeopacity{0.612876}%
\pgfsetdash{}{0pt}%
\pgfpathmoveto{\pgfqpoint{1.213770in}{0.979547in}}%
\pgfpathcurveto{\pgfqpoint{1.223857in}{0.979547in}}{\pgfqpoint{1.233533in}{0.983554in}}{\pgfqpoint{1.240666in}{0.990687in}}%
\pgfpathcurveto{\pgfqpoint{1.247798in}{0.997820in}}{\pgfqpoint{1.251806in}{1.007495in}}{\pgfqpoint{1.251806in}{1.017583in}}%
\pgfpathcurveto{\pgfqpoint{1.251806in}{1.027670in}}{\pgfqpoint{1.247798in}{1.037346in}}{\pgfqpoint{1.240666in}{1.044479in}}%
\pgfpathcurveto{\pgfqpoint{1.233533in}{1.051611in}}{\pgfqpoint{1.223857in}{1.055619in}}{\pgfqpoint{1.213770in}{1.055619in}}%
\pgfpathcurveto{\pgfqpoint{1.203683in}{1.055619in}}{\pgfqpoint{1.194007in}{1.051611in}}{\pgfqpoint{1.186874in}{1.044479in}}%
\pgfpathcurveto{\pgfqpoint{1.179741in}{1.037346in}}{\pgfqpoint{1.175734in}{1.027670in}}{\pgfqpoint{1.175734in}{1.017583in}}%
\pgfpathcurveto{\pgfqpoint{1.175734in}{1.007495in}}{\pgfqpoint{1.179741in}{0.997820in}}{\pgfqpoint{1.186874in}{0.990687in}}%
\pgfpathcurveto{\pgfqpoint{1.194007in}{0.983554in}}{\pgfqpoint{1.203683in}{0.979547in}}{\pgfqpoint{1.213770in}{0.979547in}}%
\pgfpathlineto{\pgfqpoint{1.213770in}{0.979547in}}%
\pgfpathclose%
\pgfusepath{stroke,fill}%
\end{pgfscope}%
\begin{pgfscope}%
\pgfpathrectangle{\pgfqpoint{0.254231in}{0.147348in}}{\pgfqpoint{2.735294in}{2.735294in}}%
\pgfusepath{clip}%
\pgfsetbuttcap%
\pgfsetroundjoin%
\definecolor{currentfill}{rgb}{0.839216,0.152941,0.156863}%
\pgfsetfillcolor{currentfill}%
\pgfsetfillopacity{0.625674}%
\pgfsetlinewidth{1.003750pt}%
\definecolor{currentstroke}{rgb}{0.839216,0.152941,0.156863}%
\pgfsetstrokecolor{currentstroke}%
\pgfsetstrokeopacity{0.625674}%
\pgfsetdash{}{0pt}%
\pgfpathmoveto{\pgfqpoint{2.092670in}{2.164239in}}%
\pgfpathcurveto{\pgfqpoint{2.102758in}{2.164239in}}{\pgfqpoint{2.112433in}{2.168247in}}{\pgfqpoint{2.119566in}{2.175380in}}%
\pgfpathcurveto{\pgfqpoint{2.126699in}{2.182513in}}{\pgfqpoint{2.130707in}{2.192188in}}{\pgfqpoint{2.130707in}{2.202275in}}%
\pgfpathcurveto{\pgfqpoint{2.130707in}{2.212363in}}{\pgfqpoint{2.126699in}{2.222038in}}{\pgfqpoint{2.119566in}{2.229171in}}%
\pgfpathcurveto{\pgfqpoint{2.112433in}{2.236304in}}{\pgfqpoint{2.102758in}{2.240312in}}{\pgfqpoint{2.092670in}{2.240312in}}%
\pgfpathcurveto{\pgfqpoint{2.082583in}{2.240312in}}{\pgfqpoint{2.072907in}{2.236304in}}{\pgfqpoint{2.065775in}{2.229171in}}%
\pgfpathcurveto{\pgfqpoint{2.058642in}{2.222038in}}{\pgfqpoint{2.054634in}{2.212363in}}{\pgfqpoint{2.054634in}{2.202275in}}%
\pgfpathcurveto{\pgfqpoint{2.054634in}{2.192188in}}{\pgfqpoint{2.058642in}{2.182513in}}{\pgfqpoint{2.065775in}{2.175380in}}%
\pgfpathcurveto{\pgfqpoint{2.072907in}{2.168247in}}{\pgfqpoint{2.082583in}{2.164239in}}{\pgfqpoint{2.092670in}{2.164239in}}%
\pgfpathlineto{\pgfqpoint{2.092670in}{2.164239in}}%
\pgfpathclose%
\pgfusepath{stroke,fill}%
\end{pgfscope}%
\begin{pgfscope}%
\pgfpathrectangle{\pgfqpoint{0.254231in}{0.147348in}}{\pgfqpoint{2.735294in}{2.735294in}}%
\pgfusepath{clip}%
\pgfsetbuttcap%
\pgfsetroundjoin%
\definecolor{currentfill}{rgb}{0.839216,0.152941,0.156863}%
\pgfsetfillcolor{currentfill}%
\pgfsetfillopacity{0.631635}%
\pgfsetlinewidth{1.003750pt}%
\definecolor{currentstroke}{rgb}{0.839216,0.152941,0.156863}%
\pgfsetstrokecolor{currentstroke}%
\pgfsetstrokeopacity{0.631635}%
\pgfsetdash{}{0pt}%
\pgfpathmoveto{\pgfqpoint{2.083772in}{2.104757in}}%
\pgfpathcurveto{\pgfqpoint{2.093859in}{2.104757in}}{\pgfqpoint{2.103535in}{2.108765in}}{\pgfqpoint{2.110668in}{2.115898in}}%
\pgfpathcurveto{\pgfqpoint{2.117801in}{2.123030in}}{\pgfqpoint{2.121808in}{2.132706in}}{\pgfqpoint{2.121808in}{2.142793in}}%
\pgfpathcurveto{\pgfqpoint{2.121808in}{2.152881in}}{\pgfqpoint{2.117801in}{2.162556in}}{\pgfqpoint{2.110668in}{2.169689in}}%
\pgfpathcurveto{\pgfqpoint{2.103535in}{2.176822in}}{\pgfqpoint{2.093859in}{2.180830in}}{\pgfqpoint{2.083772in}{2.180830in}}%
\pgfpathcurveto{\pgfqpoint{2.073685in}{2.180830in}}{\pgfqpoint{2.064009in}{2.176822in}}{\pgfqpoint{2.056876in}{2.169689in}}%
\pgfpathcurveto{\pgfqpoint{2.049744in}{2.162556in}}{\pgfqpoint{2.045736in}{2.152881in}}{\pgfqpoint{2.045736in}{2.142793in}}%
\pgfpathcurveto{\pgfqpoint{2.045736in}{2.132706in}}{\pgfqpoint{2.049744in}{2.123030in}}{\pgfqpoint{2.056876in}{2.115898in}}%
\pgfpathcurveto{\pgfqpoint{2.064009in}{2.108765in}}{\pgfqpoint{2.073685in}{2.104757in}}{\pgfqpoint{2.083772in}{2.104757in}}%
\pgfpathlineto{\pgfqpoint{2.083772in}{2.104757in}}%
\pgfpathclose%
\pgfusepath{stroke,fill}%
\end{pgfscope}%
\begin{pgfscope}%
\pgfpathrectangle{\pgfqpoint{0.254231in}{0.147348in}}{\pgfqpoint{2.735294in}{2.735294in}}%
\pgfusepath{clip}%
\pgfsetbuttcap%
\pgfsetroundjoin%
\definecolor{currentfill}{rgb}{0.839216,0.152941,0.156863}%
\pgfsetfillcolor{currentfill}%
\pgfsetfillopacity{0.634032}%
\pgfsetlinewidth{1.003750pt}%
\definecolor{currentstroke}{rgb}{0.839216,0.152941,0.156863}%
\pgfsetstrokecolor{currentstroke}%
\pgfsetstrokeopacity{0.634032}%
\pgfsetdash{}{0pt}%
\pgfpathmoveto{\pgfqpoint{2.354767in}{1.533362in}}%
\pgfpathcurveto{\pgfqpoint{2.364854in}{1.533362in}}{\pgfqpoint{2.374530in}{1.537370in}}{\pgfqpoint{2.381663in}{1.544503in}}%
\pgfpathcurveto{\pgfqpoint{2.388796in}{1.551635in}}{\pgfqpoint{2.392803in}{1.561311in}}{\pgfqpoint{2.392803in}{1.571398in}}%
\pgfpathcurveto{\pgfqpoint{2.392803in}{1.581486in}}{\pgfqpoint{2.388796in}{1.591161in}}{\pgfqpoint{2.381663in}{1.598294in}}%
\pgfpathcurveto{\pgfqpoint{2.374530in}{1.605427in}}{\pgfqpoint{2.364854in}{1.609435in}}{\pgfqpoint{2.354767in}{1.609435in}}%
\pgfpathcurveto{\pgfqpoint{2.344680in}{1.609435in}}{\pgfqpoint{2.335004in}{1.605427in}}{\pgfqpoint{2.327871in}{1.598294in}}%
\pgfpathcurveto{\pgfqpoint{2.320739in}{1.591161in}}{\pgfqpoint{2.316731in}{1.581486in}}{\pgfqpoint{2.316731in}{1.571398in}}%
\pgfpathcurveto{\pgfqpoint{2.316731in}{1.561311in}}{\pgfqpoint{2.320739in}{1.551635in}}{\pgfqpoint{2.327871in}{1.544503in}}%
\pgfpathcurveto{\pgfqpoint{2.335004in}{1.537370in}}{\pgfqpoint{2.344680in}{1.533362in}}{\pgfqpoint{2.354767in}{1.533362in}}%
\pgfpathlineto{\pgfqpoint{2.354767in}{1.533362in}}%
\pgfpathclose%
\pgfusepath{stroke,fill}%
\end{pgfscope}%
\begin{pgfscope}%
\pgfpathrectangle{\pgfqpoint{0.254231in}{0.147348in}}{\pgfqpoint{2.735294in}{2.735294in}}%
\pgfusepath{clip}%
\pgfsetbuttcap%
\pgfsetroundjoin%
\definecolor{currentfill}{rgb}{0.839216,0.152941,0.156863}%
\pgfsetfillcolor{currentfill}%
\pgfsetfillopacity{0.652064}%
\pgfsetlinewidth{1.003750pt}%
\definecolor{currentstroke}{rgb}{0.839216,0.152941,0.156863}%
\pgfsetstrokecolor{currentstroke}%
\pgfsetstrokeopacity{0.652064}%
\pgfsetdash{}{0pt}%
\pgfpathmoveto{\pgfqpoint{1.468055in}{0.952320in}}%
\pgfpathcurveto{\pgfqpoint{1.478142in}{0.952320in}}{\pgfqpoint{1.487818in}{0.956328in}}{\pgfqpoint{1.494951in}{0.963461in}}%
\pgfpathcurveto{\pgfqpoint{1.502084in}{0.970593in}}{\pgfqpoint{1.506091in}{0.980269in}}{\pgfqpoint{1.506091in}{0.990356in}}%
\pgfpathcurveto{\pgfqpoint{1.506091in}{1.000444in}}{\pgfqpoint{1.502084in}{1.010119in}}{\pgfqpoint{1.494951in}{1.017252in}}%
\pgfpathcurveto{\pgfqpoint{1.487818in}{1.024385in}}{\pgfqpoint{1.478142in}{1.028393in}}{\pgfqpoint{1.468055in}{1.028393in}}%
\pgfpathcurveto{\pgfqpoint{1.457968in}{1.028393in}}{\pgfqpoint{1.448292in}{1.024385in}}{\pgfqpoint{1.441159in}{1.017252in}}%
\pgfpathcurveto{\pgfqpoint{1.434027in}{1.010119in}}{\pgfqpoint{1.430019in}{1.000444in}}{\pgfqpoint{1.430019in}{0.990356in}}%
\pgfpathcurveto{\pgfqpoint{1.430019in}{0.980269in}}{\pgfqpoint{1.434027in}{0.970593in}}{\pgfqpoint{1.441159in}{0.963461in}}%
\pgfpathcurveto{\pgfqpoint{1.448292in}{0.956328in}}{\pgfqpoint{1.457968in}{0.952320in}}{\pgfqpoint{1.468055in}{0.952320in}}%
\pgfpathlineto{\pgfqpoint{1.468055in}{0.952320in}}%
\pgfpathclose%
\pgfusepath{stroke,fill}%
\end{pgfscope}%
\begin{pgfscope}%
\pgfpathrectangle{\pgfqpoint{0.254231in}{0.147348in}}{\pgfqpoint{2.735294in}{2.735294in}}%
\pgfusepath{clip}%
\pgfsetbuttcap%
\pgfsetroundjoin%
\definecolor{currentfill}{rgb}{0.839216,0.152941,0.156863}%
\pgfsetfillcolor{currentfill}%
\pgfsetfillopacity{0.738747}%
\pgfsetlinewidth{1.003750pt}%
\definecolor{currentstroke}{rgb}{0.839216,0.152941,0.156863}%
\pgfsetstrokecolor{currentstroke}%
\pgfsetstrokeopacity{0.738747}%
\pgfsetdash{}{0pt}%
\pgfpathmoveto{\pgfqpoint{0.858170in}{1.507716in}}%
\pgfpathcurveto{\pgfqpoint{0.868258in}{1.507716in}}{\pgfqpoint{0.877933in}{1.511723in}}{\pgfqpoint{0.885066in}{1.518856in}}%
\pgfpathcurveto{\pgfqpoint{0.892199in}{1.525989in}}{\pgfqpoint{0.896207in}{1.535665in}}{\pgfqpoint{0.896207in}{1.545752in}}%
\pgfpathcurveto{\pgfqpoint{0.896207in}{1.555839in}}{\pgfqpoint{0.892199in}{1.565515in}}{\pgfqpoint{0.885066in}{1.572648in}}%
\pgfpathcurveto{\pgfqpoint{0.877933in}{1.579780in}}{\pgfqpoint{0.868258in}{1.583788in}}{\pgfqpoint{0.858170in}{1.583788in}}%
\pgfpathcurveto{\pgfqpoint{0.848083in}{1.583788in}}{\pgfqpoint{0.838408in}{1.579780in}}{\pgfqpoint{0.831275in}{1.572648in}}%
\pgfpathcurveto{\pgfqpoint{0.824142in}{1.565515in}}{\pgfqpoint{0.820134in}{1.555839in}}{\pgfqpoint{0.820134in}{1.545752in}}%
\pgfpathcurveto{\pgfqpoint{0.820134in}{1.535665in}}{\pgfqpoint{0.824142in}{1.525989in}}{\pgfqpoint{0.831275in}{1.518856in}}%
\pgfpathcurveto{\pgfqpoint{0.838408in}{1.511723in}}{\pgfqpoint{0.848083in}{1.507716in}}{\pgfqpoint{0.858170in}{1.507716in}}%
\pgfpathlineto{\pgfqpoint{0.858170in}{1.507716in}}%
\pgfpathclose%
\pgfusepath{stroke,fill}%
\end{pgfscope}%
\begin{pgfscope}%
\pgfpathrectangle{\pgfqpoint{0.254231in}{0.147348in}}{\pgfqpoint{2.735294in}{2.735294in}}%
\pgfusepath{clip}%
\pgfsetbuttcap%
\pgfsetroundjoin%
\definecolor{currentfill}{rgb}{0.839216,0.152941,0.156863}%
\pgfsetfillcolor{currentfill}%
\pgfsetfillopacity{0.791813}%
\pgfsetlinewidth{1.003750pt}%
\definecolor{currentstroke}{rgb}{0.839216,0.152941,0.156863}%
\pgfsetstrokecolor{currentstroke}%
\pgfsetstrokeopacity{0.791813}%
\pgfsetdash{}{0pt}%
\pgfpathmoveto{\pgfqpoint{1.351308in}{2.014183in}}%
\pgfpathcurveto{\pgfqpoint{1.361395in}{2.014183in}}{\pgfqpoint{1.371071in}{2.018191in}}{\pgfqpoint{1.378204in}{2.025323in}}%
\pgfpathcurveto{\pgfqpoint{1.385337in}{2.032456in}}{\pgfqpoint{1.389344in}{2.042132in}}{\pgfqpoint{1.389344in}{2.052219in}}%
\pgfpathcurveto{\pgfqpoint{1.389344in}{2.062307in}}{\pgfqpoint{1.385337in}{2.071982in}}{\pgfqpoint{1.378204in}{2.079115in}}%
\pgfpathcurveto{\pgfqpoint{1.371071in}{2.086248in}}{\pgfqpoint{1.361395in}{2.090255in}}{\pgfqpoint{1.351308in}{2.090255in}}%
\pgfpathcurveto{\pgfqpoint{1.341221in}{2.090255in}}{\pgfqpoint{1.331545in}{2.086248in}}{\pgfqpoint{1.324412in}{2.079115in}}%
\pgfpathcurveto{\pgfqpoint{1.317280in}{2.071982in}}{\pgfqpoint{1.313272in}{2.062307in}}{\pgfqpoint{1.313272in}{2.052219in}}%
\pgfpathcurveto{\pgfqpoint{1.313272in}{2.042132in}}{\pgfqpoint{1.317280in}{2.032456in}}{\pgfqpoint{1.324412in}{2.025323in}}%
\pgfpathcurveto{\pgfqpoint{1.331545in}{2.018191in}}{\pgfqpoint{1.341221in}{2.014183in}}{\pgfqpoint{1.351308in}{2.014183in}}%
\pgfpathlineto{\pgfqpoint{1.351308in}{2.014183in}}%
\pgfpathclose%
\pgfusepath{stroke,fill}%
\end{pgfscope}%
\begin{pgfscope}%
\pgfpathrectangle{\pgfqpoint{0.254231in}{0.147348in}}{\pgfqpoint{2.735294in}{2.735294in}}%
\pgfusepath{clip}%
\pgfsetbuttcap%
\pgfsetroundjoin%
\definecolor{currentfill}{rgb}{0.839216,0.152941,0.156863}%
\pgfsetfillcolor{currentfill}%
\pgfsetfillopacity{0.864233}%
\pgfsetlinewidth{1.003750pt}%
\definecolor{currentstroke}{rgb}{0.839216,0.152941,0.156863}%
\pgfsetstrokecolor{currentstroke}%
\pgfsetstrokeopacity{0.864233}%
\pgfsetdash{}{0pt}%
\pgfpathmoveto{\pgfqpoint{2.017635in}{1.926870in}}%
\pgfpathcurveto{\pgfqpoint{2.027722in}{1.926870in}}{\pgfqpoint{2.037397in}{1.930878in}}{\pgfqpoint{2.044530in}{1.938011in}}%
\pgfpathcurveto{\pgfqpoint{2.051663in}{1.945143in}}{\pgfqpoint{2.055671in}{1.954819in}}{\pgfqpoint{2.055671in}{1.964906in}}%
\pgfpathcurveto{\pgfqpoint{2.055671in}{1.974994in}}{\pgfqpoint{2.051663in}{1.984669in}}{\pgfqpoint{2.044530in}{1.991802in}}%
\pgfpathcurveto{\pgfqpoint{2.037397in}{1.998935in}}{\pgfqpoint{2.027722in}{2.002943in}}{\pgfqpoint{2.017635in}{2.002943in}}%
\pgfpathcurveto{\pgfqpoint{2.007547in}{2.002943in}}{\pgfqpoint{1.997872in}{1.998935in}}{\pgfqpoint{1.990739in}{1.991802in}}%
\pgfpathcurveto{\pgfqpoint{1.983606in}{1.984669in}}{\pgfqpoint{1.979598in}{1.974994in}}{\pgfqpoint{1.979598in}{1.964906in}}%
\pgfpathcurveto{\pgfqpoint{1.979598in}{1.954819in}}{\pgfqpoint{1.983606in}{1.945143in}}{\pgfqpoint{1.990739in}{1.938011in}}%
\pgfpathcurveto{\pgfqpoint{1.997872in}{1.930878in}}{\pgfqpoint{2.007547in}{1.926870in}}{\pgfqpoint{2.017635in}{1.926870in}}%
\pgfpathlineto{\pgfqpoint{2.017635in}{1.926870in}}%
\pgfpathclose%
\pgfusepath{stroke,fill}%
\end{pgfscope}%
\begin{pgfscope}%
\pgfpathrectangle{\pgfqpoint{0.254231in}{0.147348in}}{\pgfqpoint{2.735294in}{2.735294in}}%
\pgfusepath{clip}%
\pgfsetbuttcap%
\pgfsetroundjoin%
\definecolor{currentfill}{rgb}{0.839216,0.152941,0.156863}%
\pgfsetfillcolor{currentfill}%
\pgfsetfillopacity{0.929084}%
\pgfsetlinewidth{1.003750pt}%
\definecolor{currentstroke}{rgb}{0.839216,0.152941,0.156863}%
\pgfsetstrokecolor{currentstroke}%
\pgfsetstrokeopacity{0.929084}%
\pgfsetdash{}{0pt}%
\pgfpathmoveto{\pgfqpoint{2.077047in}{1.288920in}}%
\pgfpathcurveto{\pgfqpoint{2.087135in}{1.288920in}}{\pgfqpoint{2.096810in}{1.292928in}}{\pgfqpoint{2.103943in}{1.300060in}}%
\pgfpathcurveto{\pgfqpoint{2.111076in}{1.307193in}}{\pgfqpoint{2.115084in}{1.316869in}}{\pgfqpoint{2.115084in}{1.326956in}}%
\pgfpathcurveto{\pgfqpoint{2.115084in}{1.337043in}}{\pgfqpoint{2.111076in}{1.346719in}}{\pgfqpoint{2.103943in}{1.353852in}}%
\pgfpathcurveto{\pgfqpoint{2.096810in}{1.360985in}}{\pgfqpoint{2.087135in}{1.364992in}}{\pgfqpoint{2.077047in}{1.364992in}}%
\pgfpathcurveto{\pgfqpoint{2.066960in}{1.364992in}}{\pgfqpoint{2.057285in}{1.360985in}}{\pgfqpoint{2.050152in}{1.353852in}}%
\pgfpathcurveto{\pgfqpoint{2.043019in}{1.346719in}}{\pgfqpoint{2.039011in}{1.337043in}}{\pgfqpoint{2.039011in}{1.326956in}}%
\pgfpathcurveto{\pgfqpoint{2.039011in}{1.316869in}}{\pgfqpoint{2.043019in}{1.307193in}}{\pgfqpoint{2.050152in}{1.300060in}}%
\pgfpathcurveto{\pgfqpoint{2.057285in}{1.292928in}}{\pgfqpoint{2.066960in}{1.288920in}}{\pgfqpoint{2.077047in}{1.288920in}}%
\pgfpathlineto{\pgfqpoint{2.077047in}{1.288920in}}%
\pgfpathclose%
\pgfusepath{stroke,fill}%
\end{pgfscope}%
\begin{pgfscope}%
\pgfpathrectangle{\pgfqpoint{0.254231in}{0.147348in}}{\pgfqpoint{2.735294in}{2.735294in}}%
\pgfusepath{clip}%
\pgfsetbuttcap%
\pgfsetroundjoin%
\definecolor{currentfill}{rgb}{0.839216,0.152941,0.156863}%
\pgfsetfillcolor{currentfill}%
\pgfsetlinewidth{1.003750pt}%
\definecolor{currentstroke}{rgb}{0.839216,0.152941,0.156863}%
\pgfsetstrokecolor{currentstroke}%
\pgfsetdash{}{0pt}%
\pgfpathmoveto{\pgfqpoint{1.957013in}{1.506712in}}%
\pgfpathcurveto{\pgfqpoint{1.967100in}{1.506712in}}{\pgfqpoint{1.976776in}{1.510720in}}{\pgfqpoint{1.983908in}{1.517853in}}%
\pgfpathcurveto{\pgfqpoint{1.991041in}{1.524986in}}{\pgfqpoint{1.995049in}{1.534661in}}{\pgfqpoint{1.995049in}{1.544748in}}%
\pgfpathcurveto{\pgfqpoint{1.995049in}{1.554836in}}{\pgfqpoint{1.991041in}{1.564511in}}{\pgfqpoint{1.983908in}{1.571644in}}%
\pgfpathcurveto{\pgfqpoint{1.976776in}{1.578777in}}{\pgfqpoint{1.967100in}{1.582785in}}{\pgfqpoint{1.957013in}{1.582785in}}%
\pgfpathcurveto{\pgfqpoint{1.946925in}{1.582785in}}{\pgfqpoint{1.937250in}{1.578777in}}{\pgfqpoint{1.930117in}{1.571644in}}%
\pgfpathcurveto{\pgfqpoint{1.922984in}{1.564511in}}{\pgfqpoint{1.918976in}{1.554836in}}{\pgfqpoint{1.918976in}{1.544748in}}%
\pgfpathcurveto{\pgfqpoint{1.918976in}{1.534661in}}{\pgfqpoint{1.922984in}{1.524986in}}{\pgfqpoint{1.930117in}{1.517853in}}%
\pgfpathcurveto{\pgfqpoint{1.937250in}{1.510720in}}{\pgfqpoint{1.946925in}{1.506712in}}{\pgfqpoint{1.957013in}{1.506712in}}%
\pgfpathlineto{\pgfqpoint{1.957013in}{1.506712in}}%
\pgfpathclose%
\pgfusepath{stroke,fill}%
\end{pgfscope}%
\begin{pgfscope}%
\pgfpathrectangle{\pgfqpoint{0.254231in}{0.147348in}}{\pgfqpoint{2.735294in}{2.735294in}}%
\pgfusepath{clip}%
\pgfsetbuttcap%
\pgfsetroundjoin%
\definecolor{currentfill}{rgb}{0.071067,0.258424,0.071067}%
\pgfsetfillcolor{currentfill}%
\pgfsetlinewidth{0.000000pt}%
\definecolor{currentstroke}{rgb}{0.000000,0.000000,0.000000}%
\pgfsetstrokecolor{currentstroke}%
\pgfsetdash{}{0pt}%
\pgfpathmoveto{\pgfqpoint{0.697594in}{1.088921in}}%
\pgfpathlineto{\pgfqpoint{0.601174in}{1.228895in}}%
\pgfpathlineto{\pgfqpoint{0.601748in}{1.155548in}}%
\pgfpathlineto{\pgfqpoint{0.697594in}{1.088921in}}%
\pgfpathclose%
\pgfusepath{fill}%
\end{pgfscope}%
\begin{pgfscope}%
\pgfpathrectangle{\pgfqpoint{0.254231in}{0.147348in}}{\pgfqpoint{2.735294in}{2.735294in}}%
\pgfusepath{clip}%
\pgfsetbuttcap%
\pgfsetroundjoin%
\definecolor{currentfill}{rgb}{0.071067,0.258424,0.071067}%
\pgfsetfillcolor{currentfill}%
\pgfsetlinewidth{0.000000pt}%
\definecolor{currentstroke}{rgb}{0.000000,0.000000,0.000000}%
\pgfsetstrokecolor{currentstroke}%
\pgfsetdash{}{0pt}%
\pgfpathmoveto{\pgfqpoint{2.716510in}{1.228895in}}%
\pgfpathlineto{\pgfqpoint{2.620090in}{1.088921in}}%
\pgfpathlineto{\pgfqpoint{2.715936in}{1.155548in}}%
\pgfpathlineto{\pgfqpoint{2.716510in}{1.228895in}}%
\pgfpathclose%
\pgfusepath{fill}%
\end{pgfscope}%
\begin{pgfscope}%
\pgfpathrectangle{\pgfqpoint{0.254231in}{0.147348in}}{\pgfqpoint{2.735294in}{2.735294in}}%
\pgfusepath{clip}%
\pgfsetbuttcap%
\pgfsetroundjoin%
\definecolor{currentfill}{rgb}{0.128601,0.467641,0.128601}%
\pgfsetfillcolor{currentfill}%
\pgfsetlinewidth{0.000000pt}%
\definecolor{currentstroke}{rgb}{0.000000,0.000000,0.000000}%
\pgfsetstrokecolor{currentstroke}%
\pgfsetdash{}{0pt}%
\pgfpathmoveto{\pgfqpoint{1.562421in}{2.632943in}}%
\pgfpathlineto{\pgfqpoint{1.755263in}{2.632943in}}%
\pgfpathlineto{\pgfqpoint{1.658842in}{2.699366in}}%
\pgfpathlineto{\pgfqpoint{1.562421in}{2.632943in}}%
\pgfpathclose%
\pgfusepath{fill}%
\end{pgfscope}%
\begin{pgfscope}%
\pgfpathrectangle{\pgfqpoint{0.254231in}{0.147348in}}{\pgfqpoint{2.735294in}{2.735294in}}%
\pgfusepath{clip}%
\pgfsetbuttcap%
\pgfsetroundjoin%
\definecolor{currentfill}{rgb}{0.067488,0.245410,0.067488}%
\pgfsetfillcolor{currentfill}%
\pgfsetlinewidth{0.000000pt}%
\definecolor{currentstroke}{rgb}{0.000000,0.000000,0.000000}%
\pgfsetstrokecolor{currentstroke}%
\pgfsetdash{}{0pt}%
\pgfpathmoveto{\pgfqpoint{0.822114in}{1.018707in}}%
\pgfpathlineto{\pgfqpoint{0.707491in}{1.163589in}}%
\pgfpathlineto{\pgfqpoint{0.697594in}{1.088921in}}%
\pgfpathlineto{\pgfqpoint{0.822114in}{1.018707in}}%
\pgfpathclose%
\pgfusepath{fill}%
\end{pgfscope}%
\begin{pgfscope}%
\pgfpathrectangle{\pgfqpoint{0.254231in}{0.147348in}}{\pgfqpoint{2.735294in}{2.735294in}}%
\pgfusepath{clip}%
\pgfsetbuttcap%
\pgfsetroundjoin%
\definecolor{currentfill}{rgb}{0.067488,0.245410,0.067488}%
\pgfsetfillcolor{currentfill}%
\pgfsetlinewidth{0.000000pt}%
\definecolor{currentstroke}{rgb}{0.000000,0.000000,0.000000}%
\pgfsetstrokecolor{currentstroke}%
\pgfsetdash{}{0pt}%
\pgfpathmoveto{\pgfqpoint{2.620090in}{1.088921in}}%
\pgfpathlineto{\pgfqpoint{2.610193in}{1.163589in}}%
\pgfpathlineto{\pgfqpoint{2.495569in}{1.018707in}}%
\pgfpathlineto{\pgfqpoint{2.620090in}{1.088921in}}%
\pgfpathclose%
\pgfusepath{fill}%
\end{pgfscope}%
\begin{pgfscope}%
\pgfpathrectangle{\pgfqpoint{0.254231in}{0.147348in}}{\pgfqpoint{2.735294in}{2.735294in}}%
\pgfusepath{clip}%
\pgfsetbuttcap%
\pgfsetroundjoin%
\definecolor{currentfill}{rgb}{0.069492,0.252698,0.069492}%
\pgfsetfillcolor{currentfill}%
\pgfsetlinewidth{0.000000pt}%
\definecolor{currentstroke}{rgb}{0.000000,0.000000,0.000000}%
\pgfsetstrokecolor{currentstroke}%
\pgfsetdash{}{0pt}%
\pgfpathmoveto{\pgfqpoint{0.601174in}{1.228895in}}%
\pgfpathlineto{\pgfqpoint{0.697594in}{1.088921in}}%
\pgfpathlineto{\pgfqpoint{0.695386in}{1.589726in}}%
\pgfpathlineto{\pgfqpoint{0.601174in}{1.228895in}}%
\pgfpathclose%
\pgfusepath{fill}%
\end{pgfscope}%
\begin{pgfscope}%
\pgfpathrectangle{\pgfqpoint{0.254231in}{0.147348in}}{\pgfqpoint{2.735294in}{2.735294in}}%
\pgfusepath{clip}%
\pgfsetbuttcap%
\pgfsetroundjoin%
\definecolor{currentfill}{rgb}{0.069492,0.252698,0.069492}%
\pgfsetfillcolor{currentfill}%
\pgfsetlinewidth{0.000000pt}%
\definecolor{currentstroke}{rgb}{0.000000,0.000000,0.000000}%
\pgfsetstrokecolor{currentstroke}%
\pgfsetdash{}{0pt}%
\pgfpathmoveto{\pgfqpoint{2.716510in}{1.228895in}}%
\pgfpathlineto{\pgfqpoint{2.622298in}{1.589726in}}%
\pgfpathlineto{\pgfqpoint{2.620090in}{1.088921in}}%
\pgfpathlineto{\pgfqpoint{2.716510in}{1.228895in}}%
\pgfpathclose%
\pgfusepath{fill}%
\end{pgfscope}%
\begin{pgfscope}%
\pgfpathrectangle{\pgfqpoint{0.254231in}{0.147348in}}{\pgfqpoint{2.735294in}{2.735294in}}%
\pgfusepath{clip}%
\pgfsetbuttcap%
\pgfsetroundjoin%
\definecolor{currentfill}{rgb}{0.099716,0.362602,0.099716}%
\pgfsetfillcolor{currentfill}%
\pgfsetlinewidth{0.000000pt}%
\definecolor{currentstroke}{rgb}{0.000000,0.000000,0.000000}%
\pgfsetstrokecolor{currentstroke}%
\pgfsetdash{}{0pt}%
\pgfpathmoveto{\pgfqpoint{0.695386in}{1.589726in}}%
\pgfpathlineto{\pgfqpoint{0.697594in}{1.088921in}}%
\pgfpathlineto{\pgfqpoint{0.707491in}{1.163589in}}%
\pgfpathlineto{\pgfqpoint{0.695386in}{1.589726in}}%
\pgfpathclose%
\pgfusepath{fill}%
\end{pgfscope}%
\begin{pgfscope}%
\pgfpathrectangle{\pgfqpoint{0.254231in}{0.147348in}}{\pgfqpoint{2.735294in}{2.735294in}}%
\pgfusepath{clip}%
\pgfsetbuttcap%
\pgfsetroundjoin%
\definecolor{currentfill}{rgb}{0.099716,0.362602,0.099716}%
\pgfsetfillcolor{currentfill}%
\pgfsetlinewidth{0.000000pt}%
\definecolor{currentstroke}{rgb}{0.000000,0.000000,0.000000}%
\pgfsetstrokecolor{currentstroke}%
\pgfsetdash{}{0pt}%
\pgfpathmoveto{\pgfqpoint{2.610193in}{1.163589in}}%
\pgfpathlineto{\pgfqpoint{2.620090in}{1.088921in}}%
\pgfpathlineto{\pgfqpoint{2.622298in}{1.589726in}}%
\pgfpathlineto{\pgfqpoint{2.610193in}{1.163589in}}%
\pgfpathclose%
\pgfusepath{fill}%
\end{pgfscope}%
\begin{pgfscope}%
\pgfpathrectangle{\pgfqpoint{0.254231in}{0.147348in}}{\pgfqpoint{2.735294in}{2.735294in}}%
\pgfusepath{clip}%
\pgfsetbuttcap%
\pgfsetroundjoin%
\definecolor{currentfill}{rgb}{0.063840,0.232145,0.063840}%
\pgfsetfillcolor{currentfill}%
\pgfsetlinewidth{0.000000pt}%
\definecolor{currentstroke}{rgb}{0.000000,0.000000,0.000000}%
\pgfsetstrokecolor{currentstroke}%
\pgfsetdash{}{0pt}%
\pgfpathmoveto{\pgfqpoint{0.981201in}{0.949464in}}%
\pgfpathlineto{\pgfqpoint{0.848606in}{1.094945in}}%
\pgfpathlineto{\pgfqpoint{0.822114in}{1.018707in}}%
\pgfpathlineto{\pgfqpoint{0.981201in}{0.949464in}}%
\pgfpathclose%
\pgfusepath{fill}%
\end{pgfscope}%
\begin{pgfscope}%
\pgfpathrectangle{\pgfqpoint{0.254231in}{0.147348in}}{\pgfqpoint{2.735294in}{2.735294in}}%
\pgfusepath{clip}%
\pgfsetbuttcap%
\pgfsetroundjoin%
\definecolor{currentfill}{rgb}{0.063840,0.232145,0.063840}%
\pgfsetfillcolor{currentfill}%
\pgfsetlinewidth{0.000000pt}%
\definecolor{currentstroke}{rgb}{0.000000,0.000000,0.000000}%
\pgfsetstrokecolor{currentstroke}%
\pgfsetdash{}{0pt}%
\pgfpathmoveto{\pgfqpoint{2.495569in}{1.018707in}}%
\pgfpathlineto{\pgfqpoint{2.469078in}{1.094945in}}%
\pgfpathlineto{\pgfqpoint{2.336483in}{0.949464in}}%
\pgfpathlineto{\pgfqpoint{2.495569in}{1.018707in}}%
\pgfpathclose%
\pgfusepath{fill}%
\end{pgfscope}%
\begin{pgfscope}%
\pgfpathrectangle{\pgfqpoint{0.254231in}{0.147348in}}{\pgfqpoint{2.735294in}{2.735294in}}%
\pgfusepath{clip}%
\pgfsetbuttcap%
\pgfsetroundjoin%
\definecolor{currentfill}{rgb}{0.116321,0.422987,0.116321}%
\pgfsetfillcolor{currentfill}%
\pgfsetlinewidth{0.000000pt}%
\definecolor{currentstroke}{rgb}{0.000000,0.000000,0.000000}%
\pgfsetstrokecolor{currentstroke}%
\pgfsetdash{}{0pt}%
\pgfpathmoveto{\pgfqpoint{1.755263in}{2.632943in}}%
\pgfpathlineto{\pgfqpoint{1.562421in}{2.632943in}}%
\pgfpathlineto{\pgfqpoint{1.520550in}{2.150689in}}%
\pgfpathlineto{\pgfqpoint{1.755263in}{2.632943in}}%
\pgfpathclose%
\pgfusepath{fill}%
\end{pgfscope}%
\begin{pgfscope}%
\pgfpathrectangle{\pgfqpoint{0.254231in}{0.147348in}}{\pgfqpoint{2.735294in}{2.735294in}}%
\pgfusepath{clip}%
\pgfsetbuttcap%
\pgfsetroundjoin%
\definecolor{currentfill}{rgb}{0.067061,0.243857,0.067061}%
\pgfsetfillcolor{currentfill}%
\pgfsetlinewidth{0.000000pt}%
\definecolor{currentstroke}{rgb}{0.000000,0.000000,0.000000}%
\pgfsetstrokecolor{currentstroke}%
\pgfsetdash{}{0pt}%
\pgfpathmoveto{\pgfqpoint{0.707491in}{1.163589in}}%
\pgfpathlineto{\pgfqpoint{0.822114in}{1.018707in}}%
\pgfpathlineto{\pgfqpoint{0.857590in}{1.557460in}}%
\pgfpathlineto{\pgfqpoint{0.707491in}{1.163589in}}%
\pgfpathclose%
\pgfusepath{fill}%
\end{pgfscope}%
\begin{pgfscope}%
\pgfpathrectangle{\pgfqpoint{0.254231in}{0.147348in}}{\pgfqpoint{2.735294in}{2.735294in}}%
\pgfusepath{clip}%
\pgfsetbuttcap%
\pgfsetroundjoin%
\definecolor{currentfill}{rgb}{0.067061,0.243857,0.067061}%
\pgfsetfillcolor{currentfill}%
\pgfsetlinewidth{0.000000pt}%
\definecolor{currentstroke}{rgb}{0.000000,0.000000,0.000000}%
\pgfsetstrokecolor{currentstroke}%
\pgfsetdash{}{0pt}%
\pgfpathmoveto{\pgfqpoint{2.460094in}{1.557460in}}%
\pgfpathlineto{\pgfqpoint{2.495569in}{1.018707in}}%
\pgfpathlineto{\pgfqpoint{2.610193in}{1.163589in}}%
\pgfpathlineto{\pgfqpoint{2.460094in}{1.557460in}}%
\pgfpathclose%
\pgfusepath{fill}%
\end{pgfscope}%
\begin{pgfscope}%
\pgfpathrectangle{\pgfqpoint{0.254231in}{0.147348in}}{\pgfqpoint{2.735294in}{2.735294in}}%
\pgfusepath{clip}%
\pgfsetbuttcap%
\pgfsetroundjoin%
\definecolor{currentfill}{rgb}{0.095351,0.346729,0.095351}%
\pgfsetfillcolor{currentfill}%
\pgfsetlinewidth{0.000000pt}%
\definecolor{currentstroke}{rgb}{0.000000,0.000000,0.000000}%
\pgfsetstrokecolor{currentstroke}%
\pgfsetdash{}{0pt}%
\pgfpathmoveto{\pgfqpoint{0.857590in}{1.557460in}}%
\pgfpathlineto{\pgfqpoint{0.822114in}{1.018707in}}%
\pgfpathlineto{\pgfqpoint{0.848606in}{1.094945in}}%
\pgfpathlineto{\pgfqpoint{0.857590in}{1.557460in}}%
\pgfpathclose%
\pgfusepath{fill}%
\end{pgfscope}%
\begin{pgfscope}%
\pgfpathrectangle{\pgfqpoint{0.254231in}{0.147348in}}{\pgfqpoint{2.735294in}{2.735294in}}%
\pgfusepath{clip}%
\pgfsetbuttcap%
\pgfsetroundjoin%
\definecolor{currentfill}{rgb}{0.095351,0.346729,0.095351}%
\pgfsetfillcolor{currentfill}%
\pgfsetlinewidth{0.000000pt}%
\definecolor{currentstroke}{rgb}{0.000000,0.000000,0.000000}%
\pgfsetstrokecolor{currentstroke}%
\pgfsetdash{}{0pt}%
\pgfpathmoveto{\pgfqpoint{2.469078in}{1.094945in}}%
\pgfpathlineto{\pgfqpoint{2.495569in}{1.018707in}}%
\pgfpathlineto{\pgfqpoint{2.460094in}{1.557460in}}%
\pgfpathlineto{\pgfqpoint{2.469078in}{1.094945in}}%
\pgfpathclose%
\pgfusepath{fill}%
\end{pgfscope}%
\begin{pgfscope}%
\pgfpathrectangle{\pgfqpoint{0.254231in}{0.147348in}}{\pgfqpoint{2.735294in}{2.735294in}}%
\pgfusepath{clip}%
\pgfsetbuttcap%
\pgfsetroundjoin%
\definecolor{currentfill}{rgb}{0.060435,0.219763,0.060435}%
\pgfsetfillcolor{currentfill}%
\pgfsetlinewidth{0.000000pt}%
\definecolor{currentstroke}{rgb}{0.000000,0.000000,0.000000}%
\pgfsetstrokecolor{currentstroke}%
\pgfsetdash{}{0pt}%
\pgfpathmoveto{\pgfqpoint{2.286012in}{1.028737in}}%
\pgfpathlineto{\pgfqpoint{2.139898in}{0.888724in}}%
\pgfpathlineto{\pgfqpoint{2.336483in}{0.949464in}}%
\pgfpathlineto{\pgfqpoint{2.286012in}{1.028737in}}%
\pgfpathclose%
\pgfusepath{fill}%
\end{pgfscope}%
\begin{pgfscope}%
\pgfpathrectangle{\pgfqpoint{0.254231in}{0.147348in}}{\pgfqpoint{2.735294in}{2.735294in}}%
\pgfusepath{clip}%
\pgfsetbuttcap%
\pgfsetroundjoin%
\definecolor{currentfill}{rgb}{0.060435,0.219763,0.060435}%
\pgfsetfillcolor{currentfill}%
\pgfsetlinewidth{0.000000pt}%
\definecolor{currentstroke}{rgb}{0.000000,0.000000,0.000000}%
\pgfsetstrokecolor{currentstroke}%
\pgfsetdash{}{0pt}%
\pgfpathmoveto{\pgfqpoint{0.981201in}{0.949464in}}%
\pgfpathlineto{\pgfqpoint{1.177786in}{0.888724in}}%
\pgfpathlineto{\pgfqpoint{1.031671in}{1.028737in}}%
\pgfpathlineto{\pgfqpoint{0.981201in}{0.949464in}}%
\pgfpathclose%
\pgfusepath{fill}%
\end{pgfscope}%
\begin{pgfscope}%
\pgfpathrectangle{\pgfqpoint{0.254231in}{0.147348in}}{\pgfqpoint{2.735294in}{2.735294in}}%
\pgfusepath{clip}%
\pgfsetbuttcap%
\pgfsetroundjoin%
\definecolor{currentfill}{rgb}{0.074506,0.270932,0.074506}%
\pgfsetfillcolor{currentfill}%
\pgfsetlinewidth{0.000000pt}%
\definecolor{currentstroke}{rgb}{0.000000,0.000000,0.000000}%
\pgfsetstrokecolor{currentstroke}%
\pgfsetdash{}{0pt}%
\pgfpathmoveto{\pgfqpoint{0.695386in}{1.589726in}}%
\pgfpathlineto{\pgfqpoint{0.707491in}{1.163589in}}%
\pgfpathlineto{\pgfqpoint{0.857590in}{1.557460in}}%
\pgfpathlineto{\pgfqpoint{0.695386in}{1.589726in}}%
\pgfpathclose%
\pgfusepath{fill}%
\end{pgfscope}%
\begin{pgfscope}%
\pgfpathrectangle{\pgfqpoint{0.254231in}{0.147348in}}{\pgfqpoint{2.735294in}{2.735294in}}%
\pgfusepath{clip}%
\pgfsetbuttcap%
\pgfsetroundjoin%
\definecolor{currentfill}{rgb}{0.074506,0.270932,0.074506}%
\pgfsetfillcolor{currentfill}%
\pgfsetlinewidth{0.000000pt}%
\definecolor{currentstroke}{rgb}{0.000000,0.000000,0.000000}%
\pgfsetstrokecolor{currentstroke}%
\pgfsetdash{}{0pt}%
\pgfpathmoveto{\pgfqpoint{2.460094in}{1.557460in}}%
\pgfpathlineto{\pgfqpoint{2.610193in}{1.163589in}}%
\pgfpathlineto{\pgfqpoint{2.622298in}{1.589726in}}%
\pgfpathlineto{\pgfqpoint{2.460094in}{1.557460in}}%
\pgfpathclose%
\pgfusepath{fill}%
\end{pgfscope}%
\begin{pgfscope}%
\pgfpathrectangle{\pgfqpoint{0.254231in}{0.147348in}}{\pgfqpoint{2.735294in}{2.735294in}}%
\pgfusepath{clip}%
\pgfsetbuttcap%
\pgfsetroundjoin%
\definecolor{currentfill}{rgb}{0.116785,0.424671,0.116785}%
\pgfsetfillcolor{currentfill}%
\pgfsetlinewidth{0.000000pt}%
\definecolor{currentstroke}{rgb}{0.000000,0.000000,0.000000}%
\pgfsetstrokecolor{currentstroke}%
\pgfsetdash{}{0pt}%
\pgfpathmoveto{\pgfqpoint{2.141244in}{2.292690in}}%
\pgfpathlineto{\pgfqpoint{1.755263in}{2.632943in}}%
\pgfpathlineto{\pgfqpoint{1.797134in}{2.150689in}}%
\pgfpathlineto{\pgfqpoint{2.141244in}{2.292690in}}%
\pgfpathclose%
\pgfusepath{fill}%
\end{pgfscope}%
\begin{pgfscope}%
\pgfpathrectangle{\pgfqpoint{0.254231in}{0.147348in}}{\pgfqpoint{2.735294in}{2.735294in}}%
\pgfusepath{clip}%
\pgfsetbuttcap%
\pgfsetroundjoin%
\definecolor{currentfill}{rgb}{0.116785,0.424671,0.116785}%
\pgfsetfillcolor{currentfill}%
\pgfsetlinewidth{0.000000pt}%
\definecolor{currentstroke}{rgb}{0.000000,0.000000,0.000000}%
\pgfsetstrokecolor{currentstroke}%
\pgfsetdash{}{0pt}%
\pgfpathmoveto{\pgfqpoint{1.520550in}{2.150689in}}%
\pgfpathlineto{\pgfqpoint{1.562421in}{2.632943in}}%
\pgfpathlineto{\pgfqpoint{1.176440in}{2.292690in}}%
\pgfpathlineto{\pgfqpoint{1.520550in}{2.150689in}}%
\pgfpathclose%
\pgfusepath{fill}%
\end{pgfscope}%
\begin{pgfscope}%
\pgfpathrectangle{\pgfqpoint{0.254231in}{0.147348in}}{\pgfqpoint{2.735294in}{2.735294in}}%
\pgfusepath{clip}%
\pgfsetbuttcap%
\pgfsetroundjoin%
\definecolor{currentfill}{rgb}{0.064867,0.235879,0.064867}%
\pgfsetfillcolor{currentfill}%
\pgfsetlinewidth{0.000000pt}%
\definecolor{currentstroke}{rgb}{0.000000,0.000000,0.000000}%
\pgfsetstrokecolor{currentstroke}%
\pgfsetdash{}{0pt}%
\pgfpathmoveto{\pgfqpoint{0.848606in}{1.094945in}}%
\pgfpathlineto{\pgfqpoint{0.981201in}{0.949464in}}%
\pgfpathlineto{\pgfqpoint{1.075030in}{1.526914in}}%
\pgfpathlineto{\pgfqpoint{0.848606in}{1.094945in}}%
\pgfpathclose%
\pgfusepath{fill}%
\end{pgfscope}%
\begin{pgfscope}%
\pgfpathrectangle{\pgfqpoint{0.254231in}{0.147348in}}{\pgfqpoint{2.735294in}{2.735294in}}%
\pgfusepath{clip}%
\pgfsetbuttcap%
\pgfsetroundjoin%
\definecolor{currentfill}{rgb}{0.064867,0.235879,0.064867}%
\pgfsetfillcolor{currentfill}%
\pgfsetlinewidth{0.000000pt}%
\definecolor{currentstroke}{rgb}{0.000000,0.000000,0.000000}%
\pgfsetstrokecolor{currentstroke}%
\pgfsetdash{}{0pt}%
\pgfpathmoveto{\pgfqpoint{2.242654in}{1.526914in}}%
\pgfpathlineto{\pgfqpoint{2.336483in}{0.949464in}}%
\pgfpathlineto{\pgfqpoint{2.469078in}{1.094945in}}%
\pgfpathlineto{\pgfqpoint{2.242654in}{1.526914in}}%
\pgfpathclose%
\pgfusepath{fill}%
\end{pgfscope}%
\begin{pgfscope}%
\pgfpathrectangle{\pgfqpoint{0.254231in}{0.147348in}}{\pgfqpoint{2.735294in}{2.735294in}}%
\pgfusepath{clip}%
\pgfsetbuttcap%
\pgfsetroundjoin%
\definecolor{currentfill}{rgb}{0.057724,0.209904,0.057724}%
\pgfsetfillcolor{currentfill}%
\pgfsetlinewidth{0.000000pt}%
\definecolor{currentstroke}{rgb}{0.000000,0.000000,0.000000}%
\pgfsetstrokecolor{currentstroke}%
\pgfsetdash{}{0pt}%
\pgfpathmoveto{\pgfqpoint{1.258681in}{0.974376in}}%
\pgfpathlineto{\pgfqpoint{1.177786in}{0.888724in}}%
\pgfpathlineto{\pgfqpoint{1.408065in}{0.846223in}}%
\pgfpathlineto{\pgfqpoint{1.258681in}{0.974376in}}%
\pgfpathclose%
\pgfusepath{fill}%
\end{pgfscope}%
\begin{pgfscope}%
\pgfpathrectangle{\pgfqpoint{0.254231in}{0.147348in}}{\pgfqpoint{2.735294in}{2.735294in}}%
\pgfusepath{clip}%
\pgfsetbuttcap%
\pgfsetroundjoin%
\definecolor{currentfill}{rgb}{0.057724,0.209904,0.057724}%
\pgfsetfillcolor{currentfill}%
\pgfsetlinewidth{0.000000pt}%
\definecolor{currentstroke}{rgb}{0.000000,0.000000,0.000000}%
\pgfsetstrokecolor{currentstroke}%
\pgfsetdash{}{0pt}%
\pgfpathmoveto{\pgfqpoint{1.909619in}{0.846223in}}%
\pgfpathlineto{\pgfqpoint{2.139898in}{0.888724in}}%
\pgfpathlineto{\pgfqpoint{2.059003in}{0.974376in}}%
\pgfpathlineto{\pgfqpoint{1.909619in}{0.846223in}}%
\pgfpathclose%
\pgfusepath{fill}%
\end{pgfscope}%
\begin{pgfscope}%
\pgfpathrectangle{\pgfqpoint{0.254231in}{0.147348in}}{\pgfqpoint{2.735294in}{2.735294in}}%
\pgfusepath{clip}%
\pgfsetbuttcap%
\pgfsetroundjoin%
\definecolor{currentfill}{rgb}{0.086498,0.314539,0.086498}%
\pgfsetfillcolor{currentfill}%
\pgfsetlinewidth{0.000000pt}%
\definecolor{currentstroke}{rgb}{0.000000,0.000000,0.000000}%
\pgfsetstrokecolor{currentstroke}%
\pgfsetdash{}{0pt}%
\pgfpathmoveto{\pgfqpoint{0.857590in}{1.557460in}}%
\pgfpathlineto{\pgfqpoint{0.779523in}{1.760124in}}%
\pgfpathlineto{\pgfqpoint{0.695386in}{1.589726in}}%
\pgfpathlineto{\pgfqpoint{0.857590in}{1.557460in}}%
\pgfpathclose%
\pgfusepath{fill}%
\end{pgfscope}%
\begin{pgfscope}%
\pgfpathrectangle{\pgfqpoint{0.254231in}{0.147348in}}{\pgfqpoint{2.735294in}{2.735294in}}%
\pgfusepath{clip}%
\pgfsetbuttcap%
\pgfsetroundjoin%
\definecolor{currentfill}{rgb}{0.086498,0.314539,0.086498}%
\pgfsetfillcolor{currentfill}%
\pgfsetlinewidth{0.000000pt}%
\definecolor{currentstroke}{rgb}{0.000000,0.000000,0.000000}%
\pgfsetstrokecolor{currentstroke}%
\pgfsetdash{}{0pt}%
\pgfpathmoveto{\pgfqpoint{2.622298in}{1.589726in}}%
\pgfpathlineto{\pgfqpoint{2.538161in}{1.760124in}}%
\pgfpathlineto{\pgfqpoint{2.460094in}{1.557460in}}%
\pgfpathlineto{\pgfqpoint{2.622298in}{1.589726in}}%
\pgfpathclose%
\pgfusepath{fill}%
\end{pgfscope}%
\begin{pgfscope}%
\pgfpathrectangle{\pgfqpoint{0.254231in}{0.147348in}}{\pgfqpoint{2.735294in}{2.735294in}}%
\pgfusepath{clip}%
\pgfsetbuttcap%
\pgfsetroundjoin%
\definecolor{currentfill}{rgb}{0.090812,0.330224,0.090812}%
\pgfsetfillcolor{currentfill}%
\pgfsetlinewidth{0.000000pt}%
\definecolor{currentstroke}{rgb}{0.000000,0.000000,0.000000}%
\pgfsetstrokecolor{currentstroke}%
\pgfsetdash{}{0pt}%
\pgfpathmoveto{\pgfqpoint{1.075030in}{1.526914in}}%
\pgfpathlineto{\pgfqpoint{0.981201in}{0.949464in}}%
\pgfpathlineto{\pgfqpoint{1.031671in}{1.028737in}}%
\pgfpathlineto{\pgfqpoint{1.075030in}{1.526914in}}%
\pgfpathclose%
\pgfusepath{fill}%
\end{pgfscope}%
\begin{pgfscope}%
\pgfpathrectangle{\pgfqpoint{0.254231in}{0.147348in}}{\pgfqpoint{2.735294in}{2.735294in}}%
\pgfusepath{clip}%
\pgfsetbuttcap%
\pgfsetroundjoin%
\definecolor{currentfill}{rgb}{0.090812,0.330224,0.090812}%
\pgfsetfillcolor{currentfill}%
\pgfsetlinewidth{0.000000pt}%
\definecolor{currentstroke}{rgb}{0.000000,0.000000,0.000000}%
\pgfsetstrokecolor{currentstroke}%
\pgfsetdash{}{0pt}%
\pgfpathmoveto{\pgfqpoint{2.286012in}{1.028737in}}%
\pgfpathlineto{\pgfqpoint{2.336483in}{0.949464in}}%
\pgfpathlineto{\pgfqpoint{2.242654in}{1.526914in}}%
\pgfpathlineto{\pgfqpoint{2.286012in}{1.028737in}}%
\pgfpathclose%
\pgfusepath{fill}%
\end{pgfscope}%
\begin{pgfscope}%
\pgfpathrectangle{\pgfqpoint{0.254231in}{0.147348in}}{\pgfqpoint{2.735294in}{2.735294in}}%
\pgfusepath{clip}%
\pgfsetbuttcap%
\pgfsetroundjoin%
\definecolor{currentfill}{rgb}{0.116321,0.422987,0.116321}%
\pgfsetfillcolor{currentfill}%
\pgfsetlinewidth{0.000000pt}%
\definecolor{currentstroke}{rgb}{0.000000,0.000000,0.000000}%
\pgfsetstrokecolor{currentstroke}%
\pgfsetdash{}{0pt}%
\pgfpathmoveto{\pgfqpoint{1.520550in}{2.150689in}}%
\pgfpathlineto{\pgfqpoint{1.797134in}{2.150689in}}%
\pgfpathlineto{\pgfqpoint{1.755263in}{2.632943in}}%
\pgfpathlineto{\pgfqpoint{1.520550in}{2.150689in}}%
\pgfpathclose%
\pgfusepath{fill}%
\end{pgfscope}%
\begin{pgfscope}%
\pgfpathrectangle{\pgfqpoint{0.254231in}{0.147348in}}{\pgfqpoint{2.735294in}{2.735294in}}%
\pgfusepath{clip}%
\pgfsetbuttcap%
\pgfsetroundjoin%
\definecolor{currentfill}{rgb}{0.056200,0.204363,0.056200}%
\pgfsetfillcolor{currentfill}%
\pgfsetlinewidth{0.000000pt}%
\definecolor{currentstroke}{rgb}{0.000000,0.000000,0.000000}%
\pgfsetstrokecolor{currentstroke}%
\pgfsetdash{}{0pt}%
\pgfpathmoveto{\pgfqpoint{1.658842in}{0.830831in}}%
\pgfpathlineto{\pgfqpoint{1.520882in}{0.943120in}}%
\pgfpathlineto{\pgfqpoint{1.408065in}{0.846223in}}%
\pgfpathlineto{\pgfqpoint{1.658842in}{0.830831in}}%
\pgfpathclose%
\pgfusepath{fill}%
\end{pgfscope}%
\begin{pgfscope}%
\pgfpathrectangle{\pgfqpoint{0.254231in}{0.147348in}}{\pgfqpoint{2.735294in}{2.735294in}}%
\pgfusepath{clip}%
\pgfsetbuttcap%
\pgfsetroundjoin%
\definecolor{currentfill}{rgb}{0.056200,0.204363,0.056200}%
\pgfsetfillcolor{currentfill}%
\pgfsetlinewidth{0.000000pt}%
\definecolor{currentstroke}{rgb}{0.000000,0.000000,0.000000}%
\pgfsetstrokecolor{currentstroke}%
\pgfsetdash{}{0pt}%
\pgfpathmoveto{\pgfqpoint{1.909619in}{0.846223in}}%
\pgfpathlineto{\pgfqpoint{1.796802in}{0.943120in}}%
\pgfpathlineto{\pgfqpoint{1.658842in}{0.830831in}}%
\pgfpathlineto{\pgfqpoint{1.909619in}{0.846223in}}%
\pgfpathclose%
\pgfusepath{fill}%
\end{pgfscope}%
\begin{pgfscope}%
\pgfpathrectangle{\pgfqpoint{0.254231in}{0.147348in}}{\pgfqpoint{2.735294in}{2.735294in}}%
\pgfusepath{clip}%
\pgfsetbuttcap%
\pgfsetroundjoin%
\definecolor{currentfill}{rgb}{0.107070,0.389346,0.107070}%
\pgfsetfillcolor{currentfill}%
\pgfsetlinewidth{0.000000pt}%
\definecolor{currentstroke}{rgb}{0.000000,0.000000,0.000000}%
\pgfsetstrokecolor{currentstroke}%
\pgfsetdash{}{0pt}%
\pgfpathmoveto{\pgfqpoint{1.257750in}{2.141900in}}%
\pgfpathlineto{\pgfqpoint{1.176440in}{2.292690in}}%
\pgfpathlineto{\pgfqpoint{1.030299in}{2.126617in}}%
\pgfpathlineto{\pgfqpoint{1.257750in}{2.141900in}}%
\pgfpathclose%
\pgfusepath{fill}%
\end{pgfscope}%
\begin{pgfscope}%
\pgfpathrectangle{\pgfqpoint{0.254231in}{0.147348in}}{\pgfqpoint{2.735294in}{2.735294in}}%
\pgfusepath{clip}%
\pgfsetbuttcap%
\pgfsetroundjoin%
\definecolor{currentfill}{rgb}{0.107070,0.389346,0.107070}%
\pgfsetfillcolor{currentfill}%
\pgfsetlinewidth{0.000000pt}%
\definecolor{currentstroke}{rgb}{0.000000,0.000000,0.000000}%
\pgfsetstrokecolor{currentstroke}%
\pgfsetdash{}{0pt}%
\pgfpathmoveto{\pgfqpoint{2.287385in}{2.126617in}}%
\pgfpathlineto{\pgfqpoint{2.141244in}{2.292690in}}%
\pgfpathlineto{\pgfqpoint{2.059934in}{2.141900in}}%
\pgfpathlineto{\pgfqpoint{2.287385in}{2.126617in}}%
\pgfpathclose%
\pgfusepath{fill}%
\end{pgfscope}%
\begin{pgfscope}%
\pgfpathrectangle{\pgfqpoint{0.254231in}{0.147348in}}{\pgfqpoint{2.735294in}{2.735294in}}%
\pgfusepath{clip}%
\pgfsetbuttcap%
\pgfsetroundjoin%
\definecolor{currentfill}{rgb}{0.089078,0.323920,0.089078}%
\pgfsetfillcolor{currentfill}%
\pgfsetlinewidth{0.000000pt}%
\definecolor{currentstroke}{rgb}{0.000000,0.000000,0.000000}%
\pgfsetstrokecolor{currentstroke}%
\pgfsetdash{}{0pt}%
\pgfpathmoveto{\pgfqpoint{0.857590in}{1.557460in}}%
\pgfpathlineto{\pgfqpoint{1.030299in}{2.126617in}}%
\pgfpathlineto{\pgfqpoint{0.779523in}{1.760124in}}%
\pgfpathlineto{\pgfqpoint{0.857590in}{1.557460in}}%
\pgfpathclose%
\pgfusepath{fill}%
\end{pgfscope}%
\begin{pgfscope}%
\pgfpathrectangle{\pgfqpoint{0.254231in}{0.147348in}}{\pgfqpoint{2.735294in}{2.735294in}}%
\pgfusepath{clip}%
\pgfsetbuttcap%
\pgfsetroundjoin%
\definecolor{currentfill}{rgb}{0.089078,0.323920,0.089078}%
\pgfsetfillcolor{currentfill}%
\pgfsetlinewidth{0.000000pt}%
\definecolor{currentstroke}{rgb}{0.000000,0.000000,0.000000}%
\pgfsetstrokecolor{currentstroke}%
\pgfsetdash{}{0pt}%
\pgfpathmoveto{\pgfqpoint{2.538161in}{1.760124in}}%
\pgfpathlineto{\pgfqpoint{2.287385in}{2.126617in}}%
\pgfpathlineto{\pgfqpoint{2.460094in}{1.557460in}}%
\pgfpathlineto{\pgfqpoint{2.538161in}{1.760124in}}%
\pgfpathclose%
\pgfusepath{fill}%
\end{pgfscope}%
\begin{pgfscope}%
\pgfpathrectangle{\pgfqpoint{0.254231in}{0.147348in}}{\pgfqpoint{2.735294in}{2.735294in}}%
\pgfusepath{clip}%
\pgfsetbuttcap%
\pgfsetroundjoin%
\definecolor{currentfill}{rgb}{0.061754,0.224559,0.061754}%
\pgfsetfillcolor{currentfill}%
\pgfsetlinewidth{0.000000pt}%
\definecolor{currentstroke}{rgb}{0.000000,0.000000,0.000000}%
\pgfsetstrokecolor{currentstroke}%
\pgfsetdash{}{0pt}%
\pgfpathmoveto{\pgfqpoint{1.031671in}{1.028737in}}%
\pgfpathlineto{\pgfqpoint{1.177786in}{0.888724in}}%
\pgfpathlineto{\pgfqpoint{1.212441in}{1.302943in}}%
\pgfpathlineto{\pgfqpoint{1.031671in}{1.028737in}}%
\pgfpathclose%
\pgfusepath{fill}%
\end{pgfscope}%
\begin{pgfscope}%
\pgfpathrectangle{\pgfqpoint{0.254231in}{0.147348in}}{\pgfqpoint{2.735294in}{2.735294in}}%
\pgfusepath{clip}%
\pgfsetbuttcap%
\pgfsetroundjoin%
\definecolor{currentfill}{rgb}{0.061754,0.224559,0.061754}%
\pgfsetfillcolor{currentfill}%
\pgfsetlinewidth{0.000000pt}%
\definecolor{currentstroke}{rgb}{0.000000,0.000000,0.000000}%
\pgfsetstrokecolor{currentstroke}%
\pgfsetdash{}{0pt}%
\pgfpathmoveto{\pgfqpoint{2.105243in}{1.302943in}}%
\pgfpathlineto{\pgfqpoint{2.139898in}{0.888724in}}%
\pgfpathlineto{\pgfqpoint{2.286012in}{1.028737in}}%
\pgfpathlineto{\pgfqpoint{2.105243in}{1.302943in}}%
\pgfpathclose%
\pgfusepath{fill}%
\end{pgfscope}%
\begin{pgfscope}%
\pgfpathrectangle{\pgfqpoint{0.254231in}{0.147348in}}{\pgfqpoint{2.735294in}{2.735294in}}%
\pgfusepath{clip}%
\pgfsetbuttcap%
\pgfsetroundjoin%
\definecolor{currentfill}{rgb}{0.070984,0.258123,0.070984}%
\pgfsetfillcolor{currentfill}%
\pgfsetlinewidth{0.000000pt}%
\definecolor{currentstroke}{rgb}{0.000000,0.000000,0.000000}%
\pgfsetstrokecolor{currentstroke}%
\pgfsetdash{}{0pt}%
\pgfpathmoveto{\pgfqpoint{0.857590in}{1.557460in}}%
\pgfpathlineto{\pgfqpoint{0.848606in}{1.094945in}}%
\pgfpathlineto{\pgfqpoint{1.075030in}{1.526914in}}%
\pgfpathlineto{\pgfqpoint{0.857590in}{1.557460in}}%
\pgfpathclose%
\pgfusepath{fill}%
\end{pgfscope}%
\begin{pgfscope}%
\pgfpathrectangle{\pgfqpoint{0.254231in}{0.147348in}}{\pgfqpoint{2.735294in}{2.735294in}}%
\pgfusepath{clip}%
\pgfsetbuttcap%
\pgfsetroundjoin%
\definecolor{currentfill}{rgb}{0.070984,0.258123,0.070984}%
\pgfsetfillcolor{currentfill}%
\pgfsetlinewidth{0.000000pt}%
\definecolor{currentstroke}{rgb}{0.000000,0.000000,0.000000}%
\pgfsetstrokecolor{currentstroke}%
\pgfsetdash{}{0pt}%
\pgfpathmoveto{\pgfqpoint{2.242654in}{1.526914in}}%
\pgfpathlineto{\pgfqpoint{2.469078in}{1.094945in}}%
\pgfpathlineto{\pgfqpoint{2.460094in}{1.557460in}}%
\pgfpathlineto{\pgfqpoint{2.242654in}{1.526914in}}%
\pgfpathclose%
\pgfusepath{fill}%
\end{pgfscope}%
\begin{pgfscope}%
\pgfpathrectangle{\pgfqpoint{0.254231in}{0.147348in}}{\pgfqpoint{2.735294in}{2.735294in}}%
\pgfusepath{clip}%
\pgfsetbuttcap%
\pgfsetroundjoin%
\definecolor{currentfill}{rgb}{0.070885,0.257762,0.070885}%
\pgfsetfillcolor{currentfill}%
\pgfsetlinewidth{0.000000pt}%
\definecolor{currentstroke}{rgb}{0.000000,0.000000,0.000000}%
\pgfsetstrokecolor{currentstroke}%
\pgfsetdash{}{0pt}%
\pgfpathmoveto{\pgfqpoint{2.059003in}{0.974376in}}%
\pgfpathlineto{\pgfqpoint{2.139898in}{0.888724in}}%
\pgfpathlineto{\pgfqpoint{2.105243in}{1.302943in}}%
\pgfpathlineto{\pgfqpoint{2.059003in}{0.974376in}}%
\pgfpathclose%
\pgfusepath{fill}%
\end{pgfscope}%
\begin{pgfscope}%
\pgfpathrectangle{\pgfqpoint{0.254231in}{0.147348in}}{\pgfqpoint{2.735294in}{2.735294in}}%
\pgfusepath{clip}%
\pgfsetbuttcap%
\pgfsetroundjoin%
\definecolor{currentfill}{rgb}{0.070885,0.257762,0.070885}%
\pgfsetfillcolor{currentfill}%
\pgfsetlinewidth{0.000000pt}%
\definecolor{currentstroke}{rgb}{0.000000,0.000000,0.000000}%
\pgfsetstrokecolor{currentstroke}%
\pgfsetdash{}{0pt}%
\pgfpathmoveto{\pgfqpoint{1.212441in}{1.302943in}}%
\pgfpathlineto{\pgfqpoint{1.177786in}{0.888724in}}%
\pgfpathlineto{\pgfqpoint{1.258681in}{0.974376in}}%
\pgfpathlineto{\pgfqpoint{1.212441in}{1.302943in}}%
\pgfpathclose%
\pgfusepath{fill}%
\end{pgfscope}%
\begin{pgfscope}%
\pgfpathrectangle{\pgfqpoint{0.254231in}{0.147348in}}{\pgfqpoint{2.735294in}{2.735294in}}%
\pgfusepath{clip}%
\pgfsetbuttcap%
\pgfsetroundjoin%
\definecolor{currentfill}{rgb}{0.111651,0.406004,0.111651}%
\pgfsetfillcolor{currentfill}%
\pgfsetlinewidth{0.000000pt}%
\definecolor{currentstroke}{rgb}{0.000000,0.000000,0.000000}%
\pgfsetstrokecolor{currentstroke}%
\pgfsetdash{}{0pt}%
\pgfpathmoveto{\pgfqpoint{2.059934in}{2.141900in}}%
\pgfpathlineto{\pgfqpoint{2.141244in}{2.292690in}}%
\pgfpathlineto{\pgfqpoint{1.797134in}{2.150689in}}%
\pgfpathlineto{\pgfqpoint{2.059934in}{2.141900in}}%
\pgfpathclose%
\pgfusepath{fill}%
\end{pgfscope}%
\begin{pgfscope}%
\pgfpathrectangle{\pgfqpoint{0.254231in}{0.147348in}}{\pgfqpoint{2.735294in}{2.735294in}}%
\pgfusepath{clip}%
\pgfsetbuttcap%
\pgfsetroundjoin%
\definecolor{currentfill}{rgb}{0.111651,0.406004,0.111651}%
\pgfsetfillcolor{currentfill}%
\pgfsetlinewidth{0.000000pt}%
\definecolor{currentstroke}{rgb}{0.000000,0.000000,0.000000}%
\pgfsetstrokecolor{currentstroke}%
\pgfsetdash{}{0pt}%
\pgfpathmoveto{\pgfqpoint{1.520550in}{2.150689in}}%
\pgfpathlineto{\pgfqpoint{1.176440in}{2.292690in}}%
\pgfpathlineto{\pgfqpoint{1.257750in}{2.141900in}}%
\pgfpathlineto{\pgfqpoint{1.520550in}{2.150689in}}%
\pgfpathclose%
\pgfusepath{fill}%
\end{pgfscope}%
\begin{pgfscope}%
\pgfpathrectangle{\pgfqpoint{0.254231in}{0.147348in}}{\pgfqpoint{2.735294in}{2.735294in}}%
\pgfusepath{clip}%
\pgfsetbuttcap%
\pgfsetroundjoin%
\definecolor{currentfill}{rgb}{0.092193,0.335248,0.092193}%
\pgfsetfillcolor{currentfill}%
\pgfsetlinewidth{0.000000pt}%
\definecolor{currentstroke}{rgb}{0.000000,0.000000,0.000000}%
\pgfsetstrokecolor{currentstroke}%
\pgfsetdash{}{0pt}%
\pgfpathmoveto{\pgfqpoint{1.075030in}{1.526914in}}%
\pgfpathlineto{\pgfqpoint{1.030299in}{2.126617in}}%
\pgfpathlineto{\pgfqpoint{0.857590in}{1.557460in}}%
\pgfpathlineto{\pgfqpoint{1.075030in}{1.526914in}}%
\pgfpathclose%
\pgfusepath{fill}%
\end{pgfscope}%
\begin{pgfscope}%
\pgfpathrectangle{\pgfqpoint{0.254231in}{0.147348in}}{\pgfqpoint{2.735294in}{2.735294in}}%
\pgfusepath{clip}%
\pgfsetbuttcap%
\pgfsetroundjoin%
\definecolor{currentfill}{rgb}{0.092193,0.335248,0.092193}%
\pgfsetfillcolor{currentfill}%
\pgfsetlinewidth{0.000000pt}%
\definecolor{currentstroke}{rgb}{0.000000,0.000000,0.000000}%
\pgfsetstrokecolor{currentstroke}%
\pgfsetdash{}{0pt}%
\pgfpathmoveto{\pgfqpoint{2.460094in}{1.557460in}}%
\pgfpathlineto{\pgfqpoint{2.287385in}{2.126617in}}%
\pgfpathlineto{\pgfqpoint{2.242654in}{1.526914in}}%
\pgfpathlineto{\pgfqpoint{2.460094in}{1.557460in}}%
\pgfpathclose%
\pgfusepath{fill}%
\end{pgfscope}%
\begin{pgfscope}%
\pgfpathrectangle{\pgfqpoint{0.254231in}{0.147348in}}{\pgfqpoint{2.735294in}{2.735294in}}%
\pgfusepath{clip}%
\pgfsetbuttcap%
\pgfsetroundjoin%
\definecolor{currentfill}{rgb}{0.060562,0.220227,0.060562}%
\pgfsetfillcolor{currentfill}%
\pgfsetlinewidth{0.000000pt}%
\definecolor{currentstroke}{rgb}{0.000000,0.000000,0.000000}%
\pgfsetstrokecolor{currentstroke}%
\pgfsetdash{}{0pt}%
\pgfpathmoveto{\pgfqpoint{2.059003in}{0.974376in}}%
\pgfpathlineto{\pgfqpoint{1.814008in}{1.277996in}}%
\pgfpathlineto{\pgfqpoint{1.909619in}{0.846223in}}%
\pgfpathlineto{\pgfqpoint{2.059003in}{0.974376in}}%
\pgfpathclose%
\pgfusepath{fill}%
\end{pgfscope}%
\begin{pgfscope}%
\pgfpathrectangle{\pgfqpoint{0.254231in}{0.147348in}}{\pgfqpoint{2.735294in}{2.735294in}}%
\pgfusepath{clip}%
\pgfsetbuttcap%
\pgfsetroundjoin%
\definecolor{currentfill}{rgb}{0.060562,0.220227,0.060562}%
\pgfsetfillcolor{currentfill}%
\pgfsetlinewidth{0.000000pt}%
\definecolor{currentstroke}{rgb}{0.000000,0.000000,0.000000}%
\pgfsetstrokecolor{currentstroke}%
\pgfsetdash{}{0pt}%
\pgfpathmoveto{\pgfqpoint{1.408065in}{0.846223in}}%
\pgfpathlineto{\pgfqpoint{1.503676in}{1.277996in}}%
\pgfpathlineto{\pgfqpoint{1.258681in}{0.974376in}}%
\pgfpathlineto{\pgfqpoint{1.408065in}{0.846223in}}%
\pgfpathclose%
\pgfusepath{fill}%
\end{pgfscope}%
\begin{pgfscope}%
\pgfpathrectangle{\pgfqpoint{0.254231in}{0.147348in}}{\pgfqpoint{2.735294in}{2.735294in}}%
\pgfusepath{clip}%
\pgfsetbuttcap%
\pgfsetroundjoin%
\definecolor{currentfill}{rgb}{0.097285,0.353762,0.097285}%
\pgfsetfillcolor{currentfill}%
\pgfsetlinewidth{0.000000pt}%
\definecolor{currentstroke}{rgb}{0.000000,0.000000,0.000000}%
\pgfsetstrokecolor{currentstroke}%
\pgfsetdash{}{0pt}%
\pgfpathmoveto{\pgfqpoint{1.257750in}{2.141900in}}%
\pgfpathlineto{\pgfqpoint{1.030299in}{2.126617in}}%
\pgfpathlineto{\pgfqpoint{1.366689in}{1.952591in}}%
\pgfpathlineto{\pgfqpoint{1.257750in}{2.141900in}}%
\pgfpathclose%
\pgfusepath{fill}%
\end{pgfscope}%
\begin{pgfscope}%
\pgfpathrectangle{\pgfqpoint{0.254231in}{0.147348in}}{\pgfqpoint{2.735294in}{2.735294in}}%
\pgfusepath{clip}%
\pgfsetbuttcap%
\pgfsetroundjoin%
\definecolor{currentfill}{rgb}{0.097285,0.353762,0.097285}%
\pgfsetfillcolor{currentfill}%
\pgfsetlinewidth{0.000000pt}%
\definecolor{currentstroke}{rgb}{0.000000,0.000000,0.000000}%
\pgfsetstrokecolor{currentstroke}%
\pgfsetdash{}{0pt}%
\pgfpathmoveto{\pgfqpoint{1.950995in}{1.952591in}}%
\pgfpathlineto{\pgfqpoint{2.287385in}{2.126617in}}%
\pgfpathlineto{\pgfqpoint{2.059934in}{2.141900in}}%
\pgfpathlineto{\pgfqpoint{1.950995in}{1.952591in}}%
\pgfpathclose%
\pgfusepath{fill}%
\end{pgfscope}%
\begin{pgfscope}%
\pgfpathrectangle{\pgfqpoint{0.254231in}{0.147348in}}{\pgfqpoint{2.735294in}{2.735294in}}%
\pgfusepath{clip}%
\pgfsetbuttcap%
\pgfsetroundjoin%
\definecolor{currentfill}{rgb}{0.067497,0.245443,0.067497}%
\pgfsetfillcolor{currentfill}%
\pgfsetlinewidth{0.000000pt}%
\definecolor{currentstroke}{rgb}{0.000000,0.000000,0.000000}%
\pgfsetstrokecolor{currentstroke}%
\pgfsetdash{}{0pt}%
\pgfpathmoveto{\pgfqpoint{1.408065in}{0.846223in}}%
\pgfpathlineto{\pgfqpoint{1.520882in}{0.943120in}}%
\pgfpathlineto{\pgfqpoint{1.503676in}{1.277996in}}%
\pgfpathlineto{\pgfqpoint{1.408065in}{0.846223in}}%
\pgfpathclose%
\pgfusepath{fill}%
\end{pgfscope}%
\begin{pgfscope}%
\pgfpathrectangle{\pgfqpoint{0.254231in}{0.147348in}}{\pgfqpoint{2.735294in}{2.735294in}}%
\pgfusepath{clip}%
\pgfsetbuttcap%
\pgfsetroundjoin%
\definecolor{currentfill}{rgb}{0.067497,0.245443,0.067497}%
\pgfsetfillcolor{currentfill}%
\pgfsetlinewidth{0.000000pt}%
\definecolor{currentstroke}{rgb}{0.000000,0.000000,0.000000}%
\pgfsetstrokecolor{currentstroke}%
\pgfsetdash{}{0pt}%
\pgfpathmoveto{\pgfqpoint{1.814008in}{1.277996in}}%
\pgfpathlineto{\pgfqpoint{1.796802in}{0.943120in}}%
\pgfpathlineto{\pgfqpoint{1.909619in}{0.846223in}}%
\pgfpathlineto{\pgfqpoint{1.814008in}{1.277996in}}%
\pgfpathclose%
\pgfusepath{fill}%
\end{pgfscope}%
\begin{pgfscope}%
\pgfpathrectangle{\pgfqpoint{0.254231in}{0.147348in}}{\pgfqpoint{2.735294in}{2.735294in}}%
\pgfusepath{clip}%
\pgfsetbuttcap%
\pgfsetroundjoin%
\definecolor{currentfill}{rgb}{0.065434,0.237940,0.065434}%
\pgfsetfillcolor{currentfill}%
\pgfsetlinewidth{0.000000pt}%
\definecolor{currentstroke}{rgb}{0.000000,0.000000,0.000000}%
\pgfsetstrokecolor{currentstroke}%
\pgfsetdash{}{0pt}%
\pgfpathmoveto{\pgfqpoint{1.658842in}{0.830831in}}%
\pgfpathlineto{\pgfqpoint{1.503676in}{1.277996in}}%
\pgfpathlineto{\pgfqpoint{1.520882in}{0.943120in}}%
\pgfpathlineto{\pgfqpoint{1.658842in}{0.830831in}}%
\pgfpathclose%
\pgfusepath{fill}%
\end{pgfscope}%
\begin{pgfscope}%
\pgfpathrectangle{\pgfqpoint{0.254231in}{0.147348in}}{\pgfqpoint{2.735294in}{2.735294in}}%
\pgfusepath{clip}%
\pgfsetbuttcap%
\pgfsetroundjoin%
\definecolor{currentfill}{rgb}{0.065434,0.237940,0.065434}%
\pgfsetfillcolor{currentfill}%
\pgfsetlinewidth{0.000000pt}%
\definecolor{currentstroke}{rgb}{0.000000,0.000000,0.000000}%
\pgfsetstrokecolor{currentstroke}%
\pgfsetdash{}{0pt}%
\pgfpathmoveto{\pgfqpoint{1.796802in}{0.943120in}}%
\pgfpathlineto{\pgfqpoint{1.814008in}{1.277996in}}%
\pgfpathlineto{\pgfqpoint{1.658842in}{0.830831in}}%
\pgfpathlineto{\pgfqpoint{1.796802in}{0.943120in}}%
\pgfpathclose%
\pgfusepath{fill}%
\end{pgfscope}%
\begin{pgfscope}%
\pgfpathrectangle{\pgfqpoint{0.254231in}{0.147348in}}{\pgfqpoint{2.735294in}{2.735294in}}%
\pgfusepath{clip}%
\pgfsetbuttcap%
\pgfsetroundjoin%
\definecolor{currentfill}{rgb}{0.073593,0.267612,0.073593}%
\pgfsetfillcolor{currentfill}%
\pgfsetlinewidth{0.000000pt}%
\definecolor{currentstroke}{rgb}{0.000000,0.000000,0.000000}%
\pgfsetstrokecolor{currentstroke}%
\pgfsetdash{}{0pt}%
\pgfpathmoveto{\pgfqpoint{1.031671in}{1.028737in}}%
\pgfpathlineto{\pgfqpoint{1.212441in}{1.302943in}}%
\pgfpathlineto{\pgfqpoint{1.075030in}{1.526914in}}%
\pgfpathlineto{\pgfqpoint{1.031671in}{1.028737in}}%
\pgfpathclose%
\pgfusepath{fill}%
\end{pgfscope}%
\begin{pgfscope}%
\pgfpathrectangle{\pgfqpoint{0.254231in}{0.147348in}}{\pgfqpoint{2.735294in}{2.735294in}}%
\pgfusepath{clip}%
\pgfsetbuttcap%
\pgfsetroundjoin%
\definecolor{currentfill}{rgb}{0.073593,0.267612,0.073593}%
\pgfsetfillcolor{currentfill}%
\pgfsetlinewidth{0.000000pt}%
\definecolor{currentstroke}{rgb}{0.000000,0.000000,0.000000}%
\pgfsetstrokecolor{currentstroke}%
\pgfsetdash{}{0pt}%
\pgfpathmoveto{\pgfqpoint{2.242654in}{1.526914in}}%
\pgfpathlineto{\pgfqpoint{2.105243in}{1.302943in}}%
\pgfpathlineto{\pgfqpoint{2.286012in}{1.028737in}}%
\pgfpathlineto{\pgfqpoint{2.242654in}{1.526914in}}%
\pgfpathclose%
\pgfusepath{fill}%
\end{pgfscope}%
\begin{pgfscope}%
\pgfpathrectangle{\pgfqpoint{0.254231in}{0.147348in}}{\pgfqpoint{2.735294in}{2.735294in}}%
\pgfusepath{clip}%
\pgfsetbuttcap%
\pgfsetroundjoin%
\definecolor{currentfill}{rgb}{0.091915,0.334238,0.091915}%
\pgfsetfillcolor{currentfill}%
\pgfsetlinewidth{0.000000pt}%
\definecolor{currentstroke}{rgb}{0.000000,0.000000,0.000000}%
\pgfsetstrokecolor{currentstroke}%
\pgfsetdash{}{0pt}%
\pgfpathmoveto{\pgfqpoint{1.075030in}{1.526914in}}%
\pgfpathlineto{\pgfqpoint{1.366689in}{1.952591in}}%
\pgfpathlineto{\pgfqpoint{1.030299in}{2.126617in}}%
\pgfpathlineto{\pgfqpoint{1.075030in}{1.526914in}}%
\pgfpathclose%
\pgfusepath{fill}%
\end{pgfscope}%
\begin{pgfscope}%
\pgfpathrectangle{\pgfqpoint{0.254231in}{0.147348in}}{\pgfqpoint{2.735294in}{2.735294in}}%
\pgfusepath{clip}%
\pgfsetbuttcap%
\pgfsetroundjoin%
\definecolor{currentfill}{rgb}{0.091915,0.334238,0.091915}%
\pgfsetfillcolor{currentfill}%
\pgfsetlinewidth{0.000000pt}%
\definecolor{currentstroke}{rgb}{0.000000,0.000000,0.000000}%
\pgfsetstrokecolor{currentstroke}%
\pgfsetdash{}{0pt}%
\pgfpathmoveto{\pgfqpoint{2.287385in}{2.126617in}}%
\pgfpathlineto{\pgfqpoint{1.950995in}{1.952591in}}%
\pgfpathlineto{\pgfqpoint{2.242654in}{1.526914in}}%
\pgfpathlineto{\pgfqpoint{2.287385in}{2.126617in}}%
\pgfpathclose%
\pgfusepath{fill}%
\end{pgfscope}%
\begin{pgfscope}%
\pgfpathrectangle{\pgfqpoint{0.254231in}{0.147348in}}{\pgfqpoint{2.735294in}{2.735294in}}%
\pgfusepath{clip}%
\pgfsetbuttcap%
\pgfsetroundjoin%
\definecolor{currentfill}{rgb}{0.101759,0.370033,0.101759}%
\pgfsetfillcolor{currentfill}%
\pgfsetlinewidth{0.000000pt}%
\definecolor{currentstroke}{rgb}{0.000000,0.000000,0.000000}%
\pgfsetstrokecolor{currentstroke}%
\pgfsetdash{}{0pt}%
\pgfpathmoveto{\pgfqpoint{1.366689in}{1.952591in}}%
\pgfpathlineto{\pgfqpoint{1.520550in}{2.150689in}}%
\pgfpathlineto{\pgfqpoint{1.257750in}{2.141900in}}%
\pgfpathlineto{\pgfqpoint{1.366689in}{1.952591in}}%
\pgfpathclose%
\pgfusepath{fill}%
\end{pgfscope}%
\begin{pgfscope}%
\pgfpathrectangle{\pgfqpoint{0.254231in}{0.147348in}}{\pgfqpoint{2.735294in}{2.735294in}}%
\pgfusepath{clip}%
\pgfsetbuttcap%
\pgfsetroundjoin%
\definecolor{currentfill}{rgb}{0.101759,0.370033,0.101759}%
\pgfsetfillcolor{currentfill}%
\pgfsetlinewidth{0.000000pt}%
\definecolor{currentstroke}{rgb}{0.000000,0.000000,0.000000}%
\pgfsetstrokecolor{currentstroke}%
\pgfsetdash{}{0pt}%
\pgfpathmoveto{\pgfqpoint{2.059934in}{2.141900in}}%
\pgfpathlineto{\pgfqpoint{1.797134in}{2.150689in}}%
\pgfpathlineto{\pgfqpoint{1.950995in}{1.952591in}}%
\pgfpathlineto{\pgfqpoint{2.059934in}{2.141900in}}%
\pgfpathclose%
\pgfusepath{fill}%
\end{pgfscope}%
\begin{pgfscope}%
\pgfpathrectangle{\pgfqpoint{0.254231in}{0.147348in}}{\pgfqpoint{2.735294in}{2.735294in}}%
\pgfusepath{clip}%
\pgfsetbuttcap%
\pgfsetroundjoin%
\definecolor{currentfill}{rgb}{0.101677,0.369734,0.101677}%
\pgfsetfillcolor{currentfill}%
\pgfsetlinewidth{0.000000pt}%
\definecolor{currentstroke}{rgb}{0.000000,0.000000,0.000000}%
\pgfsetstrokecolor{currentstroke}%
\pgfsetdash{}{0pt}%
\pgfpathmoveto{\pgfqpoint{1.658842in}{1.953918in}}%
\pgfpathlineto{\pgfqpoint{1.797134in}{2.150689in}}%
\pgfpathlineto{\pgfqpoint{1.520550in}{2.150689in}}%
\pgfpathlineto{\pgfqpoint{1.658842in}{1.953918in}}%
\pgfpathclose%
\pgfusepath{fill}%
\end{pgfscope}%
\begin{pgfscope}%
\pgfpathrectangle{\pgfqpoint{0.254231in}{0.147348in}}{\pgfqpoint{2.735294in}{2.735294in}}%
\pgfusepath{clip}%
\pgfsetbuttcap%
\pgfsetroundjoin%
\definecolor{currentfill}{rgb}{0.065035,0.236492,0.065035}%
\pgfsetfillcolor{currentfill}%
\pgfsetlinewidth{0.000000pt}%
\definecolor{currentstroke}{rgb}{0.000000,0.000000,0.000000}%
\pgfsetstrokecolor{currentstroke}%
\pgfsetdash{}{0pt}%
\pgfpathmoveto{\pgfqpoint{1.258681in}{0.974376in}}%
\pgfpathlineto{\pgfqpoint{1.348371in}{1.504068in}}%
\pgfpathlineto{\pgfqpoint{1.212441in}{1.302943in}}%
\pgfpathlineto{\pgfqpoint{1.258681in}{0.974376in}}%
\pgfpathclose%
\pgfusepath{fill}%
\end{pgfscope}%
\begin{pgfscope}%
\pgfpathrectangle{\pgfqpoint{0.254231in}{0.147348in}}{\pgfqpoint{2.735294in}{2.735294in}}%
\pgfusepath{clip}%
\pgfsetbuttcap%
\pgfsetroundjoin%
\definecolor{currentfill}{rgb}{0.065035,0.236492,0.065035}%
\pgfsetfillcolor{currentfill}%
\pgfsetlinewidth{0.000000pt}%
\definecolor{currentstroke}{rgb}{0.000000,0.000000,0.000000}%
\pgfsetstrokecolor{currentstroke}%
\pgfsetdash{}{0pt}%
\pgfpathmoveto{\pgfqpoint{2.105243in}{1.302943in}}%
\pgfpathlineto{\pgfqpoint{1.969313in}{1.504068in}}%
\pgfpathlineto{\pgfqpoint{2.059003in}{0.974376in}}%
\pgfpathlineto{\pgfqpoint{2.105243in}{1.302943in}}%
\pgfpathclose%
\pgfusepath{fill}%
\end{pgfscope}%
\begin{pgfscope}%
\pgfpathrectangle{\pgfqpoint{0.254231in}{0.147348in}}{\pgfqpoint{2.735294in}{2.735294in}}%
\pgfusepath{clip}%
\pgfsetbuttcap%
\pgfsetroundjoin%
\definecolor{currentfill}{rgb}{0.066446,0.241622,0.066446}%
\pgfsetfillcolor{currentfill}%
\pgfsetlinewidth{0.000000pt}%
\definecolor{currentstroke}{rgb}{0.000000,0.000000,0.000000}%
\pgfsetstrokecolor{currentstroke}%
\pgfsetdash{}{0pt}%
\pgfpathmoveto{\pgfqpoint{1.503676in}{1.277996in}}%
\pgfpathlineto{\pgfqpoint{1.658842in}{0.830831in}}%
\pgfpathlineto{\pgfqpoint{1.658842in}{1.495457in}}%
\pgfpathlineto{\pgfqpoint{1.503676in}{1.277996in}}%
\pgfpathclose%
\pgfusepath{fill}%
\end{pgfscope}%
\begin{pgfscope}%
\pgfpathrectangle{\pgfqpoint{0.254231in}{0.147348in}}{\pgfqpoint{2.735294in}{2.735294in}}%
\pgfusepath{clip}%
\pgfsetbuttcap%
\pgfsetroundjoin%
\definecolor{currentfill}{rgb}{0.066446,0.241622,0.066446}%
\pgfsetfillcolor{currentfill}%
\pgfsetlinewidth{0.000000pt}%
\definecolor{currentstroke}{rgb}{0.000000,0.000000,0.000000}%
\pgfsetstrokecolor{currentstroke}%
\pgfsetdash{}{0pt}%
\pgfpathmoveto{\pgfqpoint{1.658842in}{1.495457in}}%
\pgfpathlineto{\pgfqpoint{1.658842in}{0.830831in}}%
\pgfpathlineto{\pgfqpoint{1.814008in}{1.277996in}}%
\pgfpathlineto{\pgfqpoint{1.658842in}{1.495457in}}%
\pgfpathclose%
\pgfusepath{fill}%
\end{pgfscope}%
\begin{pgfscope}%
\pgfpathrectangle{\pgfqpoint{0.254231in}{0.147348in}}{\pgfqpoint{2.735294in}{2.735294in}}%
\pgfusepath{clip}%
\pgfsetbuttcap%
\pgfsetroundjoin%
\definecolor{currentfill}{rgb}{0.098306,0.357475,0.098306}%
\pgfsetfillcolor{currentfill}%
\pgfsetlinewidth{0.000000pt}%
\definecolor{currentstroke}{rgb}{0.000000,0.000000,0.000000}%
\pgfsetstrokecolor{currentstroke}%
\pgfsetdash{}{0pt}%
\pgfpathmoveto{\pgfqpoint{1.658842in}{1.953918in}}%
\pgfpathlineto{\pgfqpoint{1.520550in}{2.150689in}}%
\pgfpathlineto{\pgfqpoint{1.366689in}{1.952591in}}%
\pgfpathlineto{\pgfqpoint{1.658842in}{1.953918in}}%
\pgfpathclose%
\pgfusepath{fill}%
\end{pgfscope}%
\begin{pgfscope}%
\pgfpathrectangle{\pgfqpoint{0.254231in}{0.147348in}}{\pgfqpoint{2.735294in}{2.735294in}}%
\pgfusepath{clip}%
\pgfsetbuttcap%
\pgfsetroundjoin%
\definecolor{currentfill}{rgb}{0.098306,0.357475,0.098306}%
\pgfsetfillcolor{currentfill}%
\pgfsetlinewidth{0.000000pt}%
\definecolor{currentstroke}{rgb}{0.000000,0.000000,0.000000}%
\pgfsetstrokecolor{currentstroke}%
\pgfsetdash{}{0pt}%
\pgfpathmoveto{\pgfqpoint{1.950995in}{1.952591in}}%
\pgfpathlineto{\pgfqpoint{1.797134in}{2.150689in}}%
\pgfpathlineto{\pgfqpoint{1.658842in}{1.953918in}}%
\pgfpathlineto{\pgfqpoint{1.950995in}{1.952591in}}%
\pgfpathclose%
\pgfusepath{fill}%
\end{pgfscope}%
\begin{pgfscope}%
\pgfpathrectangle{\pgfqpoint{0.254231in}{0.147348in}}{\pgfqpoint{2.735294in}{2.735294in}}%
\pgfusepath{clip}%
\pgfsetbuttcap%
\pgfsetroundjoin%
\definecolor{currentfill}{rgb}{0.070209,0.255305,0.070209}%
\pgfsetfillcolor{currentfill}%
\pgfsetlinewidth{0.000000pt}%
\definecolor{currentstroke}{rgb}{0.000000,0.000000,0.000000}%
\pgfsetstrokecolor{currentstroke}%
\pgfsetdash{}{0pt}%
\pgfpathmoveto{\pgfqpoint{1.258681in}{0.974376in}}%
\pgfpathlineto{\pgfqpoint{1.503676in}{1.277996in}}%
\pgfpathlineto{\pgfqpoint{1.348371in}{1.504068in}}%
\pgfpathlineto{\pgfqpoint{1.258681in}{0.974376in}}%
\pgfpathclose%
\pgfusepath{fill}%
\end{pgfscope}%
\begin{pgfscope}%
\pgfpathrectangle{\pgfqpoint{0.254231in}{0.147348in}}{\pgfqpoint{2.735294in}{2.735294in}}%
\pgfusepath{clip}%
\pgfsetbuttcap%
\pgfsetroundjoin%
\definecolor{currentfill}{rgb}{0.070209,0.255305,0.070209}%
\pgfsetfillcolor{currentfill}%
\pgfsetlinewidth{0.000000pt}%
\definecolor{currentstroke}{rgb}{0.000000,0.000000,0.000000}%
\pgfsetstrokecolor{currentstroke}%
\pgfsetdash{}{0pt}%
\pgfpathmoveto{\pgfqpoint{1.969313in}{1.504068in}}%
\pgfpathlineto{\pgfqpoint{1.814008in}{1.277996in}}%
\pgfpathlineto{\pgfqpoint{2.059003in}{0.974376in}}%
\pgfpathlineto{\pgfqpoint{1.969313in}{1.504068in}}%
\pgfpathclose%
\pgfusepath{fill}%
\end{pgfscope}%
\begin{pgfscope}%
\pgfpathrectangle{\pgfqpoint{0.254231in}{0.147348in}}{\pgfqpoint{2.735294in}{2.735294in}}%
\pgfusepath{clip}%
\pgfsetbuttcap%
\pgfsetroundjoin%
\definecolor{currentfill}{rgb}{0.087398,0.317812,0.087398}%
\pgfsetfillcolor{currentfill}%
\pgfsetlinewidth{0.000000pt}%
\definecolor{currentstroke}{rgb}{0.000000,0.000000,0.000000}%
\pgfsetstrokecolor{currentstroke}%
\pgfsetdash{}{0pt}%
\pgfpathmoveto{\pgfqpoint{2.242654in}{1.526914in}}%
\pgfpathlineto{\pgfqpoint{1.950995in}{1.952591in}}%
\pgfpathlineto{\pgfqpoint{1.969313in}{1.504068in}}%
\pgfpathlineto{\pgfqpoint{2.242654in}{1.526914in}}%
\pgfpathclose%
\pgfusepath{fill}%
\end{pgfscope}%
\begin{pgfscope}%
\pgfpathrectangle{\pgfqpoint{0.254231in}{0.147348in}}{\pgfqpoint{2.735294in}{2.735294in}}%
\pgfusepath{clip}%
\pgfsetbuttcap%
\pgfsetroundjoin%
\definecolor{currentfill}{rgb}{0.087398,0.317812,0.087398}%
\pgfsetfillcolor{currentfill}%
\pgfsetlinewidth{0.000000pt}%
\definecolor{currentstroke}{rgb}{0.000000,0.000000,0.000000}%
\pgfsetstrokecolor{currentstroke}%
\pgfsetdash{}{0pt}%
\pgfpathmoveto{\pgfqpoint{1.348371in}{1.504068in}}%
\pgfpathlineto{\pgfqpoint{1.366689in}{1.952591in}}%
\pgfpathlineto{\pgfqpoint{1.075030in}{1.526914in}}%
\pgfpathlineto{\pgfqpoint{1.348371in}{1.504068in}}%
\pgfpathclose%
\pgfusepath{fill}%
\end{pgfscope}%
\begin{pgfscope}%
\pgfpathrectangle{\pgfqpoint{0.254231in}{0.147348in}}{\pgfqpoint{2.735294in}{2.735294in}}%
\pgfusepath{clip}%
\pgfsetbuttcap%
\pgfsetroundjoin%
\definecolor{currentfill}{rgb}{0.075994,0.276341,0.075994}%
\pgfsetfillcolor{currentfill}%
\pgfsetlinewidth{0.000000pt}%
\definecolor{currentstroke}{rgb}{0.000000,0.000000,0.000000}%
\pgfsetstrokecolor{currentstroke}%
\pgfsetdash{}{0pt}%
\pgfpathmoveto{\pgfqpoint{1.075030in}{1.526914in}}%
\pgfpathlineto{\pgfqpoint{1.212441in}{1.302943in}}%
\pgfpathlineto{\pgfqpoint{1.348371in}{1.504068in}}%
\pgfpathlineto{\pgfqpoint{1.075030in}{1.526914in}}%
\pgfpathclose%
\pgfusepath{fill}%
\end{pgfscope}%
\begin{pgfscope}%
\pgfpathrectangle{\pgfqpoint{0.254231in}{0.147348in}}{\pgfqpoint{2.735294in}{2.735294in}}%
\pgfusepath{clip}%
\pgfsetbuttcap%
\pgfsetroundjoin%
\definecolor{currentfill}{rgb}{0.075994,0.276341,0.075994}%
\pgfsetfillcolor{currentfill}%
\pgfsetlinewidth{0.000000pt}%
\definecolor{currentstroke}{rgb}{0.000000,0.000000,0.000000}%
\pgfsetstrokecolor{currentstroke}%
\pgfsetdash{}{0pt}%
\pgfpathmoveto{\pgfqpoint{1.969313in}{1.504068in}}%
\pgfpathlineto{\pgfqpoint{2.105243in}{1.302943in}}%
\pgfpathlineto{\pgfqpoint{2.242654in}{1.526914in}}%
\pgfpathlineto{\pgfqpoint{1.969313in}{1.504068in}}%
\pgfpathclose%
\pgfusepath{fill}%
\end{pgfscope}%
\begin{pgfscope}%
\pgfpathrectangle{\pgfqpoint{0.254231in}{0.147348in}}{\pgfqpoint{2.735294in}{2.735294in}}%
\pgfusepath{clip}%
\pgfsetbuttcap%
\pgfsetroundjoin%
\definecolor{currentfill}{rgb}{0.086061,0.312950,0.086061}%
\pgfsetfillcolor{currentfill}%
\pgfsetlinewidth{0.000000pt}%
\definecolor{currentstroke}{rgb}{0.000000,0.000000,0.000000}%
\pgfsetstrokecolor{currentstroke}%
\pgfsetdash{}{0pt}%
\pgfpathmoveto{\pgfqpoint{1.366689in}{1.952591in}}%
\pgfpathlineto{\pgfqpoint{1.348371in}{1.504068in}}%
\pgfpathlineto{\pgfqpoint{1.658842in}{1.953918in}}%
\pgfpathlineto{\pgfqpoint{1.366689in}{1.952591in}}%
\pgfpathclose%
\pgfusepath{fill}%
\end{pgfscope}%
\begin{pgfscope}%
\pgfpathrectangle{\pgfqpoint{0.254231in}{0.147348in}}{\pgfqpoint{2.735294in}{2.735294in}}%
\pgfusepath{clip}%
\pgfsetbuttcap%
\pgfsetroundjoin%
\definecolor{currentfill}{rgb}{0.086061,0.312950,0.086061}%
\pgfsetfillcolor{currentfill}%
\pgfsetlinewidth{0.000000pt}%
\definecolor{currentstroke}{rgb}{0.000000,0.000000,0.000000}%
\pgfsetstrokecolor{currentstroke}%
\pgfsetdash{}{0pt}%
\pgfpathmoveto{\pgfqpoint{1.658842in}{1.953918in}}%
\pgfpathlineto{\pgfqpoint{1.969313in}{1.504068in}}%
\pgfpathlineto{\pgfqpoint{1.950995in}{1.952591in}}%
\pgfpathlineto{\pgfqpoint{1.658842in}{1.953918in}}%
\pgfpathclose%
\pgfusepath{fill}%
\end{pgfscope}%
\begin{pgfscope}%
\pgfpathrectangle{\pgfqpoint{0.254231in}{0.147348in}}{\pgfqpoint{2.735294in}{2.735294in}}%
\pgfusepath{clip}%
\pgfsetbuttcap%
\pgfsetroundjoin%
\definecolor{currentfill}{rgb}{0.086258,0.313666,0.086258}%
\pgfsetfillcolor{currentfill}%
\pgfsetlinewidth{0.000000pt}%
\definecolor{currentstroke}{rgb}{0.000000,0.000000,0.000000}%
\pgfsetstrokecolor{currentstroke}%
\pgfsetdash{}{0pt}%
\pgfpathmoveto{\pgfqpoint{1.658842in}{1.495457in}}%
\pgfpathlineto{\pgfqpoint{1.658842in}{1.953918in}}%
\pgfpathlineto{\pgfqpoint{1.348371in}{1.504068in}}%
\pgfpathlineto{\pgfqpoint{1.658842in}{1.495457in}}%
\pgfpathclose%
\pgfusepath{fill}%
\end{pgfscope}%
\begin{pgfscope}%
\pgfpathrectangle{\pgfqpoint{0.254231in}{0.147348in}}{\pgfqpoint{2.735294in}{2.735294in}}%
\pgfusepath{clip}%
\pgfsetbuttcap%
\pgfsetroundjoin%
\definecolor{currentfill}{rgb}{0.086258,0.313666,0.086258}%
\pgfsetfillcolor{currentfill}%
\pgfsetlinewidth{0.000000pt}%
\definecolor{currentstroke}{rgb}{0.000000,0.000000,0.000000}%
\pgfsetstrokecolor{currentstroke}%
\pgfsetdash{}{0pt}%
\pgfpathmoveto{\pgfqpoint{1.969313in}{1.504068in}}%
\pgfpathlineto{\pgfqpoint{1.658842in}{1.953918in}}%
\pgfpathlineto{\pgfqpoint{1.658842in}{1.495457in}}%
\pgfpathlineto{\pgfqpoint{1.969313in}{1.504068in}}%
\pgfpathclose%
\pgfusepath{fill}%
\end{pgfscope}%
\begin{pgfscope}%
\pgfpathrectangle{\pgfqpoint{0.254231in}{0.147348in}}{\pgfqpoint{2.735294in}{2.735294in}}%
\pgfusepath{clip}%
\pgfsetbuttcap%
\pgfsetroundjoin%
\definecolor{currentfill}{rgb}{0.074668,0.271519,0.074668}%
\pgfsetfillcolor{currentfill}%
\pgfsetlinewidth{0.000000pt}%
\definecolor{currentstroke}{rgb}{0.000000,0.000000,0.000000}%
\pgfsetstrokecolor{currentstroke}%
\pgfsetdash{}{0pt}%
\pgfpathmoveto{\pgfqpoint{1.348371in}{1.504068in}}%
\pgfpathlineto{\pgfqpoint{1.503676in}{1.277996in}}%
\pgfpathlineto{\pgfqpoint{1.658842in}{1.495457in}}%
\pgfpathlineto{\pgfqpoint{1.348371in}{1.504068in}}%
\pgfpathclose%
\pgfusepath{fill}%
\end{pgfscope}%
\begin{pgfscope}%
\pgfpathrectangle{\pgfqpoint{0.254231in}{0.147348in}}{\pgfqpoint{2.735294in}{2.735294in}}%
\pgfusepath{clip}%
\pgfsetbuttcap%
\pgfsetroundjoin%
\definecolor{currentfill}{rgb}{0.074668,0.271519,0.074668}%
\pgfsetfillcolor{currentfill}%
\pgfsetlinewidth{0.000000pt}%
\definecolor{currentstroke}{rgb}{0.000000,0.000000,0.000000}%
\pgfsetstrokecolor{currentstroke}%
\pgfsetdash{}{0pt}%
\pgfpathmoveto{\pgfqpoint{1.658842in}{1.495457in}}%
\pgfpathlineto{\pgfqpoint{1.814008in}{1.277996in}}%
\pgfpathlineto{\pgfqpoint{1.969313in}{1.504068in}}%
\pgfpathlineto{\pgfqpoint{1.658842in}{1.495457in}}%
\pgfpathclose%
\pgfusepath{fill}%
\end{pgfscope}%
\end{pgfpicture}%
\makeatother%
\endgroup%
}
  \caption{From left to right: An opaque Pareto front; a translucent Pareto front showing the interior points above a sub-optimal front; and the sub-optimal front hiding the interior points from a different angle.}
  \label{fig:3d-mga}
\end{figure}