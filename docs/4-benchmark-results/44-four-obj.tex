\subsection{Exercise 6: Four Simultaneous Objectives}
\label{section:four-obj-results}
Chapter \ref{chapter:lit-review} showed that conventional \acp{esom} virtually
always model a single objective and that objective is uniformly cost (or a
similar aggregated economic indicator). Further, Section
\ref{section:moo-in-energy} showed that the existing literature employing
\ac{moo} never model more than three objectives simultaneously. The purpose of
this final exercise is to demonstrate that \ac{osier} can optimize many
objectives, thereby providing more context and confidence for the tool. This
exercise minimized four objectives simultaneously: total system cost, lifecycle
carbon emissions, land-use change, and percentage of total energy from
non-renewable energy sources. Here, renewable energy sources include solar, wind, and
biomass. Although batteries are often used in conjunction with \acp{vre}, they
are not considered ``renewable'' (nor are they a true energy ``source'' since
they store energy from other sources rather than producing their own). For
clarity, the ``percent non-renewable'' objective refers to the penetration of
non-renewable sources as a percentage of the energy produced rather than as a
percentage of the system's total installed capacity. Figure
\ref{fig:4-obj-pareto} shows the objective-space Pareto front for this
4-dimensional problem.

\begin{noteBox}
\textbf{Reading \aclp{pcp}:} Visualizing the Pareto front for this problem
presents a challenge due to its high dimensionality. Therefore, I present the
results with a novel plot, called a \ac{pcp}. This plot is helpful for
highlighting differences among a small set of solutions with a potentially large
number of dimensions. Figure \ref{fig:4-obj-pareto} and Figure
\ref{fig:4-obj-design} are both \acp{pcp}. Although \acp{pcp} show continuous
lines, they do not show a ``trend.'' That is, for a given solution, each
objective takes on a single value that is plotted on its respective vertical
axis. The lines connecting these points simply emphasize that these points
belong to the same solution. Additionally, each objective axis has its own upper
and lower bound because each objective is scaled differently. The \ac{mga}
solutions presented in Figure \ref{fig:4-obj-design-mga} using a boxplot due to
the larger number of solutions included in \ac{mga}. 
\end{noteBox}


\begin{figure}[h]
  \centering
  \resizebox{\columnwidth}{!}{%% Creator: Matplotlib, PGF backend
%%
%% To include the figure in your LaTeX document, write
%%   \input{<filename>.pgf}
%%
%% Make sure the required packages are loaded in your preamble
%%   \usepackage{pgf}
%%
%% Also ensure that all the required font packages are loaded; for instance,
%% the lmodern package is sometimes necessary when using math font.
%%   \usepackage{lmodern}
%%
%% Figures using additional raster images can only be included by \input if
%% they are in the same directory as the main LaTeX file. For loading figures
%% from other directories you can use the `import` package
%%   \usepackage{import}
%%
%% and then include the figures with
%%   \import{<path to file>}{<filename>.pgf}
%%
%% Matplotlib used the following preamble
%%
\begingroup%
\makeatletter%
\begin{pgfpicture}%
\pgfpathrectangle{\pgfpointorigin}{\pgfqpoint{11.792596in}{5.900000in}}%
\pgfusepath{use as bounding box, clip}%
\begin{pgfscope}%
\pgfsetbuttcap%
\pgfsetmiterjoin%
\definecolor{currentfill}{rgb}{1.000000,1.000000,1.000000}%
\pgfsetfillcolor{currentfill}%
\pgfsetlinewidth{0.000000pt}%
\definecolor{currentstroke}{rgb}{0.000000,0.000000,0.000000}%
\pgfsetstrokecolor{currentstroke}%
\pgfsetdash{}{0pt}%
\pgfpathmoveto{\pgfqpoint{0.000000in}{0.000000in}}%
\pgfpathlineto{\pgfqpoint{11.792596in}{0.000000in}}%
\pgfpathlineto{\pgfqpoint{11.792596in}{5.900000in}}%
\pgfpathlineto{\pgfqpoint{0.000000in}{5.900000in}}%
\pgfpathlineto{\pgfqpoint{0.000000in}{0.000000in}}%
\pgfpathclose%
\pgfusepath{fill}%
\end{pgfscope}%
\begin{pgfscope}%
\pgfsetbuttcap%
\pgfsetmiterjoin%
\definecolor{currentfill}{rgb}{1.000000,1.000000,1.000000}%
\pgfsetfillcolor{currentfill}%
\pgfsetlinewidth{0.000000pt}%
\definecolor{currentstroke}{rgb}{0.000000,0.000000,0.000000}%
\pgfsetstrokecolor{currentstroke}%
\pgfsetstrokeopacity{0.000000}%
\pgfsetdash{}{0pt}%
\pgfpathmoveto{\pgfqpoint{0.100000in}{0.673611in}}%
\pgfpathlineto{\pgfqpoint{11.375527in}{0.673611in}}%
\pgfpathlineto{\pgfqpoint{11.375527in}{5.216667in}}%
\pgfpathlineto{\pgfqpoint{0.100000in}{5.216667in}}%
\pgfpathlineto{\pgfqpoint{0.100000in}{0.673611in}}%
\pgfpathclose%
\pgfusepath{fill}%
\end{pgfscope}%
\begin{pgfscope}%
\definecolor{textcolor}{rgb}{0.150000,0.150000,0.150000}%
\pgfsetstrokecolor{textcolor}%
\pgfsetfillcolor{textcolor}%
\pgftext[x=0.713019in,y=0.277777in,,top]{\color{textcolor}\rmfamily\fontsize{14.000000}{16.800000}\selectfont total\_cost}%
\end{pgfscope}%
\begin{pgfscope}%
\definecolor{textcolor}{rgb}{0.150000,0.150000,0.150000}%
\pgfsetstrokecolor{textcolor}%
\pgfsetfillcolor{textcolor}%
\pgftext[x=4.062849in,y=0.277777in,,top]{\color{textcolor}\rmfamily\fontsize{14.000000}{16.800000}\selectfont lifecycle\_co2}%
\end{pgfscope}%
\begin{pgfscope}%
\definecolor{textcolor}{rgb}{0.150000,0.150000,0.150000}%
\pgfsetstrokecolor{textcolor}%
\pgfsetfillcolor{textcolor}%
\pgftext[x=7.412679in,y=0.277777in,,top]{\color{textcolor}\rmfamily\fontsize{14.000000}{16.800000}\selectfont land\_use}%
\end{pgfscope}%
\begin{pgfscope}%
\definecolor{textcolor}{rgb}{0.150000,0.150000,0.150000}%
\pgfsetstrokecolor{textcolor}%
\pgfsetfillcolor{textcolor}%
\pgftext[x=10.762508in,y=0.277777in,,top]{\color{textcolor}\rmfamily\fontsize{14.000000}{16.800000}\selectfont percent\_nonrenewable}%
\end{pgfscope}%
\begin{pgfscope}%
\pgfpathrectangle{\pgfqpoint{0.100000in}{0.673611in}}{\pgfqpoint{11.275527in}{4.543056in}}%
\pgfusepath{clip}%
\pgfsetroundcap%
\pgfsetroundjoin%
\pgfsetlinewidth{1.505625pt}%
\definecolor{currentstroke}{rgb}{0.501961,0.501961,0.501961}%
\pgfsetstrokecolor{currentstroke}%
\pgfsetstrokeopacity{0.300000}%
\pgfsetdash{}{0pt}%
\pgfpathmoveto{\pgfqpoint{0.713019in}{2.302951in}}%
\pgfpathlineto{\pgfqpoint{4.062849in}{0.880113in}}%
\pgfpathlineto{\pgfqpoint{7.412679in}{0.908230in}}%
\pgfpathlineto{\pgfqpoint{10.762508in}{4.961362in}}%
\pgfusepath{stroke}%
\end{pgfscope}%
\begin{pgfscope}%
\pgfpathrectangle{\pgfqpoint{0.100000in}{0.673611in}}{\pgfqpoint{11.275527in}{4.543056in}}%
\pgfusepath{clip}%
\pgfsetroundcap%
\pgfsetroundjoin%
\pgfsetlinewidth{1.505625pt}%
\definecolor{currentstroke}{rgb}{0.501961,0.501961,0.501961}%
\pgfsetstrokecolor{currentstroke}%
\pgfsetstrokeopacity{0.300000}%
\pgfsetdash{}{0pt}%
\pgfpathmoveto{\pgfqpoint{0.713019in}{1.121782in}}%
\pgfpathlineto{\pgfqpoint{4.062849in}{2.016019in}}%
\pgfpathlineto{\pgfqpoint{7.412679in}{0.891269in}}%
\pgfpathlineto{\pgfqpoint{10.762508in}{4.852669in}}%
\pgfusepath{stroke}%
\end{pgfscope}%
\begin{pgfscope}%
\pgfpathrectangle{\pgfqpoint{0.100000in}{0.673611in}}{\pgfqpoint{11.275527in}{4.543056in}}%
\pgfusepath{clip}%
\pgfsetroundcap%
\pgfsetroundjoin%
\pgfsetlinewidth{1.505625pt}%
\definecolor{currentstroke}{rgb}{0.501961,0.501961,0.501961}%
\pgfsetstrokecolor{currentstroke}%
\pgfsetstrokeopacity{0.300000}%
\pgfsetdash{}{0pt}%
\pgfpathmoveto{\pgfqpoint{0.713019in}{1.071249in}}%
\pgfpathlineto{\pgfqpoint{4.062849in}{3.273656in}}%
\pgfpathlineto{\pgfqpoint{7.412679in}{1.193400in}}%
\pgfpathlineto{\pgfqpoint{10.762508in}{4.468086in}}%
\pgfusepath{stroke}%
\end{pgfscope}%
\begin{pgfscope}%
\pgfpathrectangle{\pgfqpoint{0.100000in}{0.673611in}}{\pgfqpoint{11.275527in}{4.543056in}}%
\pgfusepath{clip}%
\pgfsetroundcap%
\pgfsetroundjoin%
\pgfsetlinewidth{1.505625pt}%
\definecolor{currentstroke}{rgb}{0.501961,0.501961,0.501961}%
\pgfsetstrokecolor{currentstroke}%
\pgfsetstrokeopacity{0.300000}%
\pgfsetdash{}{0pt}%
\pgfpathmoveto{\pgfqpoint{0.713019in}{0.880113in}}%
\pgfpathlineto{\pgfqpoint{4.062849in}{3.634435in}}%
\pgfpathlineto{\pgfqpoint{7.412679in}{1.018092in}}%
\pgfpathlineto{\pgfqpoint{10.762508in}{4.745233in}}%
\pgfusepath{stroke}%
\end{pgfscope}%
\begin{pgfscope}%
\pgfpathrectangle{\pgfqpoint{0.100000in}{0.673611in}}{\pgfqpoint{11.275527in}{4.543056in}}%
\pgfusepath{clip}%
\pgfsetroundcap%
\pgfsetroundjoin%
\pgfsetlinewidth{1.505625pt}%
\definecolor{currentstroke}{rgb}{0.501961,0.501961,0.501961}%
\pgfsetstrokecolor{currentstroke}%
\pgfsetstrokeopacity{0.300000}%
\pgfsetdash{}{0pt}%
\pgfpathmoveto{\pgfqpoint{0.713019in}{5.010164in}}%
\pgfpathlineto{\pgfqpoint{4.062849in}{5.010164in}}%
\pgfpathlineto{\pgfqpoint{7.412679in}{0.895615in}}%
\pgfpathlineto{\pgfqpoint{10.762508in}{2.187877in}}%
\pgfusepath{stroke}%
\end{pgfscope}%
\begin{pgfscope}%
\pgfpathrectangle{\pgfqpoint{0.100000in}{0.673611in}}{\pgfqpoint{11.275527in}{4.543056in}}%
\pgfusepath{clip}%
\pgfsetroundcap%
\pgfsetroundjoin%
\pgfsetlinewidth{1.505625pt}%
\definecolor{currentstroke}{rgb}{0.501961,0.501961,0.501961}%
\pgfsetstrokecolor{currentstroke}%
\pgfsetstrokeopacity{0.300000}%
\pgfsetdash{}{0pt}%
\pgfpathmoveto{\pgfqpoint{0.713019in}{1.492299in}}%
\pgfpathlineto{\pgfqpoint{4.062849in}{2.195117in}}%
\pgfpathlineto{\pgfqpoint{7.412679in}{0.880113in}}%
\pgfpathlineto{\pgfqpoint{10.762508in}{5.010164in}}%
\pgfusepath{stroke}%
\end{pgfscope}%
\begin{pgfscope}%
\pgfpathrectangle{\pgfqpoint{0.100000in}{0.673611in}}{\pgfqpoint{11.275527in}{4.543056in}}%
\pgfusepath{clip}%
\pgfsetroundcap%
\pgfsetroundjoin%
\pgfsetlinewidth{1.505625pt}%
\definecolor{currentstroke}{rgb}{0.501961,0.501961,0.501961}%
\pgfsetstrokecolor{currentstroke}%
\pgfsetstrokeopacity{0.300000}%
\pgfsetdash{}{0pt}%
\pgfpathmoveto{\pgfqpoint{0.713019in}{1.170594in}}%
\pgfpathlineto{\pgfqpoint{4.062849in}{2.044689in}}%
\pgfpathlineto{\pgfqpoint{7.412679in}{0.951062in}}%
\pgfpathlineto{\pgfqpoint{10.762508in}{4.746450in}}%
\pgfusepath{stroke}%
\end{pgfscope}%
\begin{pgfscope}%
\pgfpathrectangle{\pgfqpoint{0.100000in}{0.673611in}}{\pgfqpoint{11.275527in}{4.543056in}}%
\pgfusepath{clip}%
\pgfsetroundcap%
\pgfsetroundjoin%
\pgfsetlinewidth{1.505625pt}%
\definecolor{currentstroke}{rgb}{0.501961,0.501961,0.501961}%
\pgfsetstrokecolor{currentstroke}%
\pgfsetstrokeopacity{0.300000}%
\pgfsetdash{}{0pt}%
\pgfpathmoveto{\pgfqpoint{0.713019in}{1.982322in}}%
\pgfpathlineto{\pgfqpoint{4.062849in}{1.006114in}}%
\pgfpathlineto{\pgfqpoint{7.412679in}{5.010164in}}%
\pgfpathlineto{\pgfqpoint{10.762508in}{0.880113in}}%
\pgfusepath{stroke}%
\end{pgfscope}%
\begin{pgfscope}%
\pgfpathrectangle{\pgfqpoint{0.100000in}{0.673611in}}{\pgfqpoint{11.275527in}{4.543056in}}%
\pgfusepath{clip}%
\pgfsetroundcap%
\pgfsetroundjoin%
\pgfsetlinewidth{5.018750pt}%
\definecolor{currentstroke}{rgb}{0.172549,0.627451,0.172549}%
\pgfsetstrokecolor{currentstroke}%
\pgfsetdash{}{0pt}%
\pgfpathmoveto{\pgfqpoint{0.713019in}{2.302951in}}%
\pgfpathlineto{\pgfqpoint{4.062849in}{0.880113in}}%
\pgfpathlineto{\pgfqpoint{7.412679in}{0.908230in}}%
\pgfpathlineto{\pgfqpoint{10.762508in}{4.961362in}}%
\pgfusepath{stroke}%
\end{pgfscope}%
\begin{pgfscope}%
\pgfpathrectangle{\pgfqpoint{0.100000in}{0.673611in}}{\pgfqpoint{11.275527in}{4.543056in}}%
\pgfusepath{clip}%
\pgfsetroundcap%
\pgfsetroundjoin%
\pgfsetlinewidth{5.018750pt}%
\definecolor{currentstroke}{rgb}{0.839216,0.152941,0.156863}%
\pgfsetstrokecolor{currentstroke}%
\pgfsetdash{}{0pt}%
\pgfpathmoveto{\pgfqpoint{0.713019in}{0.880113in}}%
\pgfpathlineto{\pgfqpoint{4.062849in}{3.634435in}}%
\pgfpathlineto{\pgfqpoint{7.412679in}{1.018092in}}%
\pgfpathlineto{\pgfqpoint{10.762508in}{4.745233in}}%
\pgfusepath{stroke}%
\end{pgfscope}%
\begin{pgfscope}%
\pgfpathrectangle{\pgfqpoint{0.100000in}{0.673611in}}{\pgfqpoint{11.275527in}{4.543056in}}%
\pgfusepath{clip}%
\pgfsetroundcap%
\pgfsetroundjoin%
\pgfsetlinewidth{5.018750pt}%
\definecolor{currentstroke}{rgb}{1.000000,0.498039,0.054902}%
\pgfsetstrokecolor{currentstroke}%
\pgfsetdash{}{0pt}%
\pgfpathmoveto{\pgfqpoint{0.713019in}{1.492299in}}%
\pgfpathlineto{\pgfqpoint{4.062849in}{2.195117in}}%
\pgfpathlineto{\pgfqpoint{7.412679in}{0.880113in}}%
\pgfpathlineto{\pgfqpoint{10.762508in}{5.010164in}}%
\pgfusepath{stroke}%
\end{pgfscope}%
\begin{pgfscope}%
\pgfpathrectangle{\pgfqpoint{0.100000in}{0.673611in}}{\pgfqpoint{11.275527in}{4.543056in}}%
\pgfusepath{clip}%
\pgfsetroundcap%
\pgfsetroundjoin%
\pgfsetlinewidth{5.018750pt}%
\definecolor{currentstroke}{rgb}{0.121569,0.466667,0.705882}%
\pgfsetstrokecolor{currentstroke}%
\pgfsetdash{}{0pt}%
\pgfpathmoveto{\pgfqpoint{0.713019in}{1.982322in}}%
\pgfpathlineto{\pgfqpoint{4.062849in}{1.006114in}}%
\pgfpathlineto{\pgfqpoint{7.412679in}{5.010164in}}%
\pgfpathlineto{\pgfqpoint{10.762508in}{0.880113in}}%
\pgfusepath{stroke}%
\end{pgfscope}%
\begin{pgfscope}%
\pgfpathrectangle{\pgfqpoint{0.100000in}{0.673611in}}{\pgfqpoint{11.275527in}{4.543056in}}%
\pgfusepath{clip}%
\pgfsetroundcap%
\pgfsetroundjoin%
\pgfsetlinewidth{2.007500pt}%
\definecolor{currentstroke}{rgb}{0.501961,0.501961,0.501961}%
\pgfsetstrokecolor{currentstroke}%
\pgfsetstrokeopacity{0.500000}%
\pgfsetdash{}{0pt}%
\pgfpathmoveto{\pgfqpoint{0.713019in}{0.673611in}}%
\pgfpathlineto{\pgfqpoint{0.713019in}{5.216667in}}%
\pgfusepath{stroke}%
\end{pgfscope}%
\begin{pgfscope}%
\pgfpathrectangle{\pgfqpoint{0.100000in}{0.673611in}}{\pgfqpoint{11.275527in}{4.543056in}}%
\pgfusepath{clip}%
\pgfsetbuttcap%
\pgfsetroundjoin%
\pgfsetlinewidth{2.007500pt}%
\definecolor{currentstroke}{rgb}{0.501961,0.501961,0.501961}%
\pgfsetstrokecolor{currentstroke}%
\pgfsetstrokeopacity{0.500000}%
\pgfsetdash{}{0pt}%
\pgfpathmoveto{\pgfqpoint{0.612524in}{0.880113in}}%
\pgfpathlineto{\pgfqpoint{0.813514in}{0.880113in}}%
\pgfusepath{stroke}%
\end{pgfscope}%
\begin{pgfscope}%
\pgfpathrectangle{\pgfqpoint{0.100000in}{0.673611in}}{\pgfqpoint{11.275527in}{4.543056in}}%
\pgfusepath{clip}%
\pgfsetbuttcap%
\pgfsetroundjoin%
\pgfsetlinewidth{2.007500pt}%
\definecolor{currentstroke}{rgb}{0.501961,0.501961,0.501961}%
\pgfsetstrokecolor{currentstroke}%
\pgfsetstrokeopacity{0.500000}%
\pgfsetdash{}{0pt}%
\pgfpathmoveto{\pgfqpoint{0.612524in}{1.339008in}}%
\pgfpathlineto{\pgfqpoint{0.813514in}{1.339008in}}%
\pgfusepath{stroke}%
\end{pgfscope}%
\begin{pgfscope}%
\pgfpathrectangle{\pgfqpoint{0.100000in}{0.673611in}}{\pgfqpoint{11.275527in}{4.543056in}}%
\pgfusepath{clip}%
\pgfsetbuttcap%
\pgfsetroundjoin%
\pgfsetlinewidth{2.007500pt}%
\definecolor{currentstroke}{rgb}{0.501961,0.501961,0.501961}%
\pgfsetstrokecolor{currentstroke}%
\pgfsetstrokeopacity{0.500000}%
\pgfsetdash{}{0pt}%
\pgfpathmoveto{\pgfqpoint{0.612524in}{1.797902in}}%
\pgfpathlineto{\pgfqpoint{0.813514in}{1.797902in}}%
\pgfusepath{stroke}%
\end{pgfscope}%
\begin{pgfscope}%
\pgfpathrectangle{\pgfqpoint{0.100000in}{0.673611in}}{\pgfqpoint{11.275527in}{4.543056in}}%
\pgfusepath{clip}%
\pgfsetbuttcap%
\pgfsetroundjoin%
\pgfsetlinewidth{2.007500pt}%
\definecolor{currentstroke}{rgb}{0.501961,0.501961,0.501961}%
\pgfsetstrokecolor{currentstroke}%
\pgfsetstrokeopacity{0.500000}%
\pgfsetdash{}{0pt}%
\pgfpathmoveto{\pgfqpoint{0.612524in}{2.256797in}}%
\pgfpathlineto{\pgfqpoint{0.813514in}{2.256797in}}%
\pgfusepath{stroke}%
\end{pgfscope}%
\begin{pgfscope}%
\pgfpathrectangle{\pgfqpoint{0.100000in}{0.673611in}}{\pgfqpoint{11.275527in}{4.543056in}}%
\pgfusepath{clip}%
\pgfsetbuttcap%
\pgfsetroundjoin%
\pgfsetlinewidth{2.007500pt}%
\definecolor{currentstroke}{rgb}{0.501961,0.501961,0.501961}%
\pgfsetstrokecolor{currentstroke}%
\pgfsetstrokeopacity{0.500000}%
\pgfsetdash{}{0pt}%
\pgfpathmoveto{\pgfqpoint{0.612524in}{2.715691in}}%
\pgfpathlineto{\pgfqpoint{0.813514in}{2.715691in}}%
\pgfusepath{stroke}%
\end{pgfscope}%
\begin{pgfscope}%
\pgfpathrectangle{\pgfqpoint{0.100000in}{0.673611in}}{\pgfqpoint{11.275527in}{4.543056in}}%
\pgfusepath{clip}%
\pgfsetbuttcap%
\pgfsetroundjoin%
\pgfsetlinewidth{2.007500pt}%
\definecolor{currentstroke}{rgb}{0.501961,0.501961,0.501961}%
\pgfsetstrokecolor{currentstroke}%
\pgfsetstrokeopacity{0.500000}%
\pgfsetdash{}{0pt}%
\pgfpathmoveto{\pgfqpoint{0.612524in}{3.174586in}}%
\pgfpathlineto{\pgfqpoint{0.813514in}{3.174586in}}%
\pgfusepath{stroke}%
\end{pgfscope}%
\begin{pgfscope}%
\pgfpathrectangle{\pgfqpoint{0.100000in}{0.673611in}}{\pgfqpoint{11.275527in}{4.543056in}}%
\pgfusepath{clip}%
\pgfsetbuttcap%
\pgfsetroundjoin%
\pgfsetlinewidth{2.007500pt}%
\definecolor{currentstroke}{rgb}{0.501961,0.501961,0.501961}%
\pgfsetstrokecolor{currentstroke}%
\pgfsetstrokeopacity{0.500000}%
\pgfsetdash{}{0pt}%
\pgfpathmoveto{\pgfqpoint{0.612524in}{3.633480in}}%
\pgfpathlineto{\pgfqpoint{0.813514in}{3.633480in}}%
\pgfusepath{stroke}%
\end{pgfscope}%
\begin{pgfscope}%
\pgfpathrectangle{\pgfqpoint{0.100000in}{0.673611in}}{\pgfqpoint{11.275527in}{4.543056in}}%
\pgfusepath{clip}%
\pgfsetbuttcap%
\pgfsetroundjoin%
\pgfsetlinewidth{2.007500pt}%
\definecolor{currentstroke}{rgb}{0.501961,0.501961,0.501961}%
\pgfsetstrokecolor{currentstroke}%
\pgfsetstrokeopacity{0.500000}%
\pgfsetdash{}{0pt}%
\pgfpathmoveto{\pgfqpoint{0.612524in}{4.092375in}}%
\pgfpathlineto{\pgfqpoint{0.813514in}{4.092375in}}%
\pgfusepath{stroke}%
\end{pgfscope}%
\begin{pgfscope}%
\pgfpathrectangle{\pgfqpoint{0.100000in}{0.673611in}}{\pgfqpoint{11.275527in}{4.543056in}}%
\pgfusepath{clip}%
\pgfsetbuttcap%
\pgfsetroundjoin%
\pgfsetlinewidth{2.007500pt}%
\definecolor{currentstroke}{rgb}{0.501961,0.501961,0.501961}%
\pgfsetstrokecolor{currentstroke}%
\pgfsetstrokeopacity{0.500000}%
\pgfsetdash{}{0pt}%
\pgfpathmoveto{\pgfqpoint{0.612524in}{4.551270in}}%
\pgfpathlineto{\pgfqpoint{0.813514in}{4.551270in}}%
\pgfusepath{stroke}%
\end{pgfscope}%
\begin{pgfscope}%
\pgfpathrectangle{\pgfqpoint{0.100000in}{0.673611in}}{\pgfqpoint{11.275527in}{4.543056in}}%
\pgfusepath{clip}%
\pgfsetbuttcap%
\pgfsetroundjoin%
\pgfsetlinewidth{2.007500pt}%
\definecolor{currentstroke}{rgb}{0.501961,0.501961,0.501961}%
\pgfsetstrokecolor{currentstroke}%
\pgfsetstrokeopacity{0.500000}%
\pgfsetdash{}{0pt}%
\pgfpathmoveto{\pgfqpoint{0.612524in}{5.010164in}}%
\pgfpathlineto{\pgfqpoint{0.813514in}{5.010164in}}%
\pgfusepath{stroke}%
\end{pgfscope}%
\begin{pgfscope}%
\pgfpathrectangle{\pgfqpoint{0.100000in}{0.673611in}}{\pgfqpoint{11.275527in}{4.543056in}}%
\pgfusepath{clip}%
\pgfsetroundcap%
\pgfsetroundjoin%
\pgfsetlinewidth{2.007500pt}%
\definecolor{currentstroke}{rgb}{0.501961,0.501961,0.501961}%
\pgfsetstrokecolor{currentstroke}%
\pgfsetstrokeopacity{0.500000}%
\pgfsetdash{}{0pt}%
\pgfpathmoveto{\pgfqpoint{4.062849in}{0.673611in}}%
\pgfpathlineto{\pgfqpoint{4.062849in}{5.216667in}}%
\pgfusepath{stroke}%
\end{pgfscope}%
\begin{pgfscope}%
\pgfpathrectangle{\pgfqpoint{0.100000in}{0.673611in}}{\pgfqpoint{11.275527in}{4.543056in}}%
\pgfusepath{clip}%
\pgfsetbuttcap%
\pgfsetroundjoin%
\pgfsetlinewidth{2.007500pt}%
\definecolor{currentstroke}{rgb}{0.501961,0.501961,0.501961}%
\pgfsetstrokecolor{currentstroke}%
\pgfsetstrokeopacity{0.500000}%
\pgfsetdash{}{0pt}%
\pgfpathmoveto{\pgfqpoint{3.962354in}{0.880113in}}%
\pgfpathlineto{\pgfqpoint{4.163344in}{0.880113in}}%
\pgfusepath{stroke}%
\end{pgfscope}%
\begin{pgfscope}%
\pgfpathrectangle{\pgfqpoint{0.100000in}{0.673611in}}{\pgfqpoint{11.275527in}{4.543056in}}%
\pgfusepath{clip}%
\pgfsetbuttcap%
\pgfsetroundjoin%
\pgfsetlinewidth{2.007500pt}%
\definecolor{currentstroke}{rgb}{0.501961,0.501961,0.501961}%
\pgfsetstrokecolor{currentstroke}%
\pgfsetstrokeopacity{0.500000}%
\pgfsetdash{}{0pt}%
\pgfpathmoveto{\pgfqpoint{3.962354in}{1.339008in}}%
\pgfpathlineto{\pgfqpoint{4.163344in}{1.339008in}}%
\pgfusepath{stroke}%
\end{pgfscope}%
\begin{pgfscope}%
\pgfpathrectangle{\pgfqpoint{0.100000in}{0.673611in}}{\pgfqpoint{11.275527in}{4.543056in}}%
\pgfusepath{clip}%
\pgfsetbuttcap%
\pgfsetroundjoin%
\pgfsetlinewidth{2.007500pt}%
\definecolor{currentstroke}{rgb}{0.501961,0.501961,0.501961}%
\pgfsetstrokecolor{currentstroke}%
\pgfsetstrokeopacity{0.500000}%
\pgfsetdash{}{0pt}%
\pgfpathmoveto{\pgfqpoint{3.962354in}{1.797902in}}%
\pgfpathlineto{\pgfqpoint{4.163344in}{1.797902in}}%
\pgfusepath{stroke}%
\end{pgfscope}%
\begin{pgfscope}%
\pgfpathrectangle{\pgfqpoint{0.100000in}{0.673611in}}{\pgfqpoint{11.275527in}{4.543056in}}%
\pgfusepath{clip}%
\pgfsetbuttcap%
\pgfsetroundjoin%
\pgfsetlinewidth{2.007500pt}%
\definecolor{currentstroke}{rgb}{0.501961,0.501961,0.501961}%
\pgfsetstrokecolor{currentstroke}%
\pgfsetstrokeopacity{0.500000}%
\pgfsetdash{}{0pt}%
\pgfpathmoveto{\pgfqpoint{3.962354in}{2.256797in}}%
\pgfpathlineto{\pgfqpoint{4.163344in}{2.256797in}}%
\pgfusepath{stroke}%
\end{pgfscope}%
\begin{pgfscope}%
\pgfpathrectangle{\pgfqpoint{0.100000in}{0.673611in}}{\pgfqpoint{11.275527in}{4.543056in}}%
\pgfusepath{clip}%
\pgfsetbuttcap%
\pgfsetroundjoin%
\pgfsetlinewidth{2.007500pt}%
\definecolor{currentstroke}{rgb}{0.501961,0.501961,0.501961}%
\pgfsetstrokecolor{currentstroke}%
\pgfsetstrokeopacity{0.500000}%
\pgfsetdash{}{0pt}%
\pgfpathmoveto{\pgfqpoint{3.962354in}{2.715691in}}%
\pgfpathlineto{\pgfqpoint{4.163344in}{2.715691in}}%
\pgfusepath{stroke}%
\end{pgfscope}%
\begin{pgfscope}%
\pgfpathrectangle{\pgfqpoint{0.100000in}{0.673611in}}{\pgfqpoint{11.275527in}{4.543056in}}%
\pgfusepath{clip}%
\pgfsetbuttcap%
\pgfsetroundjoin%
\pgfsetlinewidth{2.007500pt}%
\definecolor{currentstroke}{rgb}{0.501961,0.501961,0.501961}%
\pgfsetstrokecolor{currentstroke}%
\pgfsetstrokeopacity{0.500000}%
\pgfsetdash{}{0pt}%
\pgfpathmoveto{\pgfqpoint{3.962354in}{3.174586in}}%
\pgfpathlineto{\pgfqpoint{4.163344in}{3.174586in}}%
\pgfusepath{stroke}%
\end{pgfscope}%
\begin{pgfscope}%
\pgfpathrectangle{\pgfqpoint{0.100000in}{0.673611in}}{\pgfqpoint{11.275527in}{4.543056in}}%
\pgfusepath{clip}%
\pgfsetbuttcap%
\pgfsetroundjoin%
\pgfsetlinewidth{2.007500pt}%
\definecolor{currentstroke}{rgb}{0.501961,0.501961,0.501961}%
\pgfsetstrokecolor{currentstroke}%
\pgfsetstrokeopacity{0.500000}%
\pgfsetdash{}{0pt}%
\pgfpathmoveto{\pgfqpoint{3.962354in}{3.633480in}}%
\pgfpathlineto{\pgfqpoint{4.163344in}{3.633480in}}%
\pgfusepath{stroke}%
\end{pgfscope}%
\begin{pgfscope}%
\pgfpathrectangle{\pgfqpoint{0.100000in}{0.673611in}}{\pgfqpoint{11.275527in}{4.543056in}}%
\pgfusepath{clip}%
\pgfsetbuttcap%
\pgfsetroundjoin%
\pgfsetlinewidth{2.007500pt}%
\definecolor{currentstroke}{rgb}{0.501961,0.501961,0.501961}%
\pgfsetstrokecolor{currentstroke}%
\pgfsetstrokeopacity{0.500000}%
\pgfsetdash{}{0pt}%
\pgfpathmoveto{\pgfqpoint{3.962354in}{4.092375in}}%
\pgfpathlineto{\pgfqpoint{4.163344in}{4.092375in}}%
\pgfusepath{stroke}%
\end{pgfscope}%
\begin{pgfscope}%
\pgfpathrectangle{\pgfqpoint{0.100000in}{0.673611in}}{\pgfqpoint{11.275527in}{4.543056in}}%
\pgfusepath{clip}%
\pgfsetbuttcap%
\pgfsetroundjoin%
\pgfsetlinewidth{2.007500pt}%
\definecolor{currentstroke}{rgb}{0.501961,0.501961,0.501961}%
\pgfsetstrokecolor{currentstroke}%
\pgfsetstrokeopacity{0.500000}%
\pgfsetdash{}{0pt}%
\pgfpathmoveto{\pgfqpoint{3.962354in}{4.551270in}}%
\pgfpathlineto{\pgfqpoint{4.163344in}{4.551270in}}%
\pgfusepath{stroke}%
\end{pgfscope}%
\begin{pgfscope}%
\pgfpathrectangle{\pgfqpoint{0.100000in}{0.673611in}}{\pgfqpoint{11.275527in}{4.543056in}}%
\pgfusepath{clip}%
\pgfsetbuttcap%
\pgfsetroundjoin%
\pgfsetlinewidth{2.007500pt}%
\definecolor{currentstroke}{rgb}{0.501961,0.501961,0.501961}%
\pgfsetstrokecolor{currentstroke}%
\pgfsetstrokeopacity{0.500000}%
\pgfsetdash{}{0pt}%
\pgfpathmoveto{\pgfqpoint{3.962354in}{5.010164in}}%
\pgfpathlineto{\pgfqpoint{4.163344in}{5.010164in}}%
\pgfusepath{stroke}%
\end{pgfscope}%
\begin{pgfscope}%
\pgfpathrectangle{\pgfqpoint{0.100000in}{0.673611in}}{\pgfqpoint{11.275527in}{4.543056in}}%
\pgfusepath{clip}%
\pgfsetroundcap%
\pgfsetroundjoin%
\pgfsetlinewidth{2.007500pt}%
\definecolor{currentstroke}{rgb}{0.501961,0.501961,0.501961}%
\pgfsetstrokecolor{currentstroke}%
\pgfsetstrokeopacity{0.500000}%
\pgfsetdash{}{0pt}%
\pgfpathmoveto{\pgfqpoint{7.412679in}{0.673611in}}%
\pgfpathlineto{\pgfqpoint{7.412679in}{5.216667in}}%
\pgfusepath{stroke}%
\end{pgfscope}%
\begin{pgfscope}%
\pgfpathrectangle{\pgfqpoint{0.100000in}{0.673611in}}{\pgfqpoint{11.275527in}{4.543056in}}%
\pgfusepath{clip}%
\pgfsetbuttcap%
\pgfsetroundjoin%
\pgfsetlinewidth{2.007500pt}%
\definecolor{currentstroke}{rgb}{0.501961,0.501961,0.501961}%
\pgfsetstrokecolor{currentstroke}%
\pgfsetstrokeopacity{0.500000}%
\pgfsetdash{}{0pt}%
\pgfpathmoveto{\pgfqpoint{7.312184in}{0.880113in}}%
\pgfpathlineto{\pgfqpoint{7.513173in}{0.880113in}}%
\pgfusepath{stroke}%
\end{pgfscope}%
\begin{pgfscope}%
\pgfpathrectangle{\pgfqpoint{0.100000in}{0.673611in}}{\pgfqpoint{11.275527in}{4.543056in}}%
\pgfusepath{clip}%
\pgfsetbuttcap%
\pgfsetroundjoin%
\pgfsetlinewidth{2.007500pt}%
\definecolor{currentstroke}{rgb}{0.501961,0.501961,0.501961}%
\pgfsetstrokecolor{currentstroke}%
\pgfsetstrokeopacity{0.500000}%
\pgfsetdash{}{0pt}%
\pgfpathmoveto{\pgfqpoint{7.312184in}{1.339008in}}%
\pgfpathlineto{\pgfqpoint{7.513173in}{1.339008in}}%
\pgfusepath{stroke}%
\end{pgfscope}%
\begin{pgfscope}%
\pgfpathrectangle{\pgfqpoint{0.100000in}{0.673611in}}{\pgfqpoint{11.275527in}{4.543056in}}%
\pgfusepath{clip}%
\pgfsetbuttcap%
\pgfsetroundjoin%
\pgfsetlinewidth{2.007500pt}%
\definecolor{currentstroke}{rgb}{0.501961,0.501961,0.501961}%
\pgfsetstrokecolor{currentstroke}%
\pgfsetstrokeopacity{0.500000}%
\pgfsetdash{}{0pt}%
\pgfpathmoveto{\pgfqpoint{7.312184in}{1.797902in}}%
\pgfpathlineto{\pgfqpoint{7.513173in}{1.797902in}}%
\pgfusepath{stroke}%
\end{pgfscope}%
\begin{pgfscope}%
\pgfpathrectangle{\pgfqpoint{0.100000in}{0.673611in}}{\pgfqpoint{11.275527in}{4.543056in}}%
\pgfusepath{clip}%
\pgfsetbuttcap%
\pgfsetroundjoin%
\pgfsetlinewidth{2.007500pt}%
\definecolor{currentstroke}{rgb}{0.501961,0.501961,0.501961}%
\pgfsetstrokecolor{currentstroke}%
\pgfsetstrokeopacity{0.500000}%
\pgfsetdash{}{0pt}%
\pgfpathmoveto{\pgfqpoint{7.312184in}{2.256797in}}%
\pgfpathlineto{\pgfqpoint{7.513173in}{2.256797in}}%
\pgfusepath{stroke}%
\end{pgfscope}%
\begin{pgfscope}%
\pgfpathrectangle{\pgfqpoint{0.100000in}{0.673611in}}{\pgfqpoint{11.275527in}{4.543056in}}%
\pgfusepath{clip}%
\pgfsetbuttcap%
\pgfsetroundjoin%
\pgfsetlinewidth{2.007500pt}%
\definecolor{currentstroke}{rgb}{0.501961,0.501961,0.501961}%
\pgfsetstrokecolor{currentstroke}%
\pgfsetstrokeopacity{0.500000}%
\pgfsetdash{}{0pt}%
\pgfpathmoveto{\pgfqpoint{7.312184in}{2.715691in}}%
\pgfpathlineto{\pgfqpoint{7.513173in}{2.715691in}}%
\pgfusepath{stroke}%
\end{pgfscope}%
\begin{pgfscope}%
\pgfpathrectangle{\pgfqpoint{0.100000in}{0.673611in}}{\pgfqpoint{11.275527in}{4.543056in}}%
\pgfusepath{clip}%
\pgfsetbuttcap%
\pgfsetroundjoin%
\pgfsetlinewidth{2.007500pt}%
\definecolor{currentstroke}{rgb}{0.501961,0.501961,0.501961}%
\pgfsetstrokecolor{currentstroke}%
\pgfsetstrokeopacity{0.500000}%
\pgfsetdash{}{0pt}%
\pgfpathmoveto{\pgfqpoint{7.312184in}{3.174586in}}%
\pgfpathlineto{\pgfqpoint{7.513173in}{3.174586in}}%
\pgfusepath{stroke}%
\end{pgfscope}%
\begin{pgfscope}%
\pgfpathrectangle{\pgfqpoint{0.100000in}{0.673611in}}{\pgfqpoint{11.275527in}{4.543056in}}%
\pgfusepath{clip}%
\pgfsetbuttcap%
\pgfsetroundjoin%
\pgfsetlinewidth{2.007500pt}%
\definecolor{currentstroke}{rgb}{0.501961,0.501961,0.501961}%
\pgfsetstrokecolor{currentstroke}%
\pgfsetstrokeopacity{0.500000}%
\pgfsetdash{}{0pt}%
\pgfpathmoveto{\pgfqpoint{7.312184in}{3.633480in}}%
\pgfpathlineto{\pgfqpoint{7.513173in}{3.633480in}}%
\pgfusepath{stroke}%
\end{pgfscope}%
\begin{pgfscope}%
\pgfpathrectangle{\pgfqpoint{0.100000in}{0.673611in}}{\pgfqpoint{11.275527in}{4.543056in}}%
\pgfusepath{clip}%
\pgfsetbuttcap%
\pgfsetroundjoin%
\pgfsetlinewidth{2.007500pt}%
\definecolor{currentstroke}{rgb}{0.501961,0.501961,0.501961}%
\pgfsetstrokecolor{currentstroke}%
\pgfsetstrokeopacity{0.500000}%
\pgfsetdash{}{0pt}%
\pgfpathmoveto{\pgfqpoint{7.312184in}{4.092375in}}%
\pgfpathlineto{\pgfqpoint{7.513173in}{4.092375in}}%
\pgfusepath{stroke}%
\end{pgfscope}%
\begin{pgfscope}%
\pgfpathrectangle{\pgfqpoint{0.100000in}{0.673611in}}{\pgfqpoint{11.275527in}{4.543056in}}%
\pgfusepath{clip}%
\pgfsetbuttcap%
\pgfsetroundjoin%
\pgfsetlinewidth{2.007500pt}%
\definecolor{currentstroke}{rgb}{0.501961,0.501961,0.501961}%
\pgfsetstrokecolor{currentstroke}%
\pgfsetstrokeopacity{0.500000}%
\pgfsetdash{}{0pt}%
\pgfpathmoveto{\pgfqpoint{7.312184in}{4.551270in}}%
\pgfpathlineto{\pgfqpoint{7.513173in}{4.551270in}}%
\pgfusepath{stroke}%
\end{pgfscope}%
\begin{pgfscope}%
\pgfpathrectangle{\pgfqpoint{0.100000in}{0.673611in}}{\pgfqpoint{11.275527in}{4.543056in}}%
\pgfusepath{clip}%
\pgfsetbuttcap%
\pgfsetroundjoin%
\pgfsetlinewidth{2.007500pt}%
\definecolor{currentstroke}{rgb}{0.501961,0.501961,0.501961}%
\pgfsetstrokecolor{currentstroke}%
\pgfsetstrokeopacity{0.500000}%
\pgfsetdash{}{0pt}%
\pgfpathmoveto{\pgfqpoint{7.312184in}{5.010164in}}%
\pgfpathlineto{\pgfqpoint{7.513173in}{5.010164in}}%
\pgfusepath{stroke}%
\end{pgfscope}%
\begin{pgfscope}%
\pgfpathrectangle{\pgfqpoint{0.100000in}{0.673611in}}{\pgfqpoint{11.275527in}{4.543056in}}%
\pgfusepath{clip}%
\pgfsetroundcap%
\pgfsetroundjoin%
\pgfsetlinewidth{2.007500pt}%
\definecolor{currentstroke}{rgb}{0.501961,0.501961,0.501961}%
\pgfsetstrokecolor{currentstroke}%
\pgfsetstrokeopacity{0.500000}%
\pgfsetdash{}{0pt}%
\pgfpathmoveto{\pgfqpoint{10.762508in}{0.673611in}}%
\pgfpathlineto{\pgfqpoint{10.762508in}{5.216667in}}%
\pgfusepath{stroke}%
\end{pgfscope}%
\begin{pgfscope}%
\pgfpathrectangle{\pgfqpoint{0.100000in}{0.673611in}}{\pgfqpoint{11.275527in}{4.543056in}}%
\pgfusepath{clip}%
\pgfsetbuttcap%
\pgfsetroundjoin%
\pgfsetlinewidth{2.007500pt}%
\definecolor{currentstroke}{rgb}{0.501961,0.501961,0.501961}%
\pgfsetstrokecolor{currentstroke}%
\pgfsetstrokeopacity{0.500000}%
\pgfsetdash{}{0pt}%
\pgfpathmoveto{\pgfqpoint{10.662014in}{0.880113in}}%
\pgfpathlineto{\pgfqpoint{10.863003in}{0.880113in}}%
\pgfusepath{stroke}%
\end{pgfscope}%
\begin{pgfscope}%
\pgfpathrectangle{\pgfqpoint{0.100000in}{0.673611in}}{\pgfqpoint{11.275527in}{4.543056in}}%
\pgfusepath{clip}%
\pgfsetbuttcap%
\pgfsetroundjoin%
\pgfsetlinewidth{2.007500pt}%
\definecolor{currentstroke}{rgb}{0.501961,0.501961,0.501961}%
\pgfsetstrokecolor{currentstroke}%
\pgfsetstrokeopacity{0.500000}%
\pgfsetdash{}{0pt}%
\pgfpathmoveto{\pgfqpoint{10.662014in}{1.339008in}}%
\pgfpathlineto{\pgfqpoint{10.863003in}{1.339008in}}%
\pgfusepath{stroke}%
\end{pgfscope}%
\begin{pgfscope}%
\pgfpathrectangle{\pgfqpoint{0.100000in}{0.673611in}}{\pgfqpoint{11.275527in}{4.543056in}}%
\pgfusepath{clip}%
\pgfsetbuttcap%
\pgfsetroundjoin%
\pgfsetlinewidth{2.007500pt}%
\definecolor{currentstroke}{rgb}{0.501961,0.501961,0.501961}%
\pgfsetstrokecolor{currentstroke}%
\pgfsetstrokeopacity{0.500000}%
\pgfsetdash{}{0pt}%
\pgfpathmoveto{\pgfqpoint{10.662014in}{1.797902in}}%
\pgfpathlineto{\pgfqpoint{10.863003in}{1.797902in}}%
\pgfusepath{stroke}%
\end{pgfscope}%
\begin{pgfscope}%
\pgfpathrectangle{\pgfqpoint{0.100000in}{0.673611in}}{\pgfqpoint{11.275527in}{4.543056in}}%
\pgfusepath{clip}%
\pgfsetbuttcap%
\pgfsetroundjoin%
\pgfsetlinewidth{2.007500pt}%
\definecolor{currentstroke}{rgb}{0.501961,0.501961,0.501961}%
\pgfsetstrokecolor{currentstroke}%
\pgfsetstrokeopacity{0.500000}%
\pgfsetdash{}{0pt}%
\pgfpathmoveto{\pgfqpoint{10.662014in}{2.256797in}}%
\pgfpathlineto{\pgfqpoint{10.863003in}{2.256797in}}%
\pgfusepath{stroke}%
\end{pgfscope}%
\begin{pgfscope}%
\pgfpathrectangle{\pgfqpoint{0.100000in}{0.673611in}}{\pgfqpoint{11.275527in}{4.543056in}}%
\pgfusepath{clip}%
\pgfsetbuttcap%
\pgfsetroundjoin%
\pgfsetlinewidth{2.007500pt}%
\definecolor{currentstroke}{rgb}{0.501961,0.501961,0.501961}%
\pgfsetstrokecolor{currentstroke}%
\pgfsetstrokeopacity{0.500000}%
\pgfsetdash{}{0pt}%
\pgfpathmoveto{\pgfqpoint{10.662014in}{2.715691in}}%
\pgfpathlineto{\pgfqpoint{10.863003in}{2.715691in}}%
\pgfusepath{stroke}%
\end{pgfscope}%
\begin{pgfscope}%
\pgfpathrectangle{\pgfqpoint{0.100000in}{0.673611in}}{\pgfqpoint{11.275527in}{4.543056in}}%
\pgfusepath{clip}%
\pgfsetbuttcap%
\pgfsetroundjoin%
\pgfsetlinewidth{2.007500pt}%
\definecolor{currentstroke}{rgb}{0.501961,0.501961,0.501961}%
\pgfsetstrokecolor{currentstroke}%
\pgfsetstrokeopacity{0.500000}%
\pgfsetdash{}{0pt}%
\pgfpathmoveto{\pgfqpoint{10.662014in}{3.174586in}}%
\pgfpathlineto{\pgfqpoint{10.863003in}{3.174586in}}%
\pgfusepath{stroke}%
\end{pgfscope}%
\begin{pgfscope}%
\pgfpathrectangle{\pgfqpoint{0.100000in}{0.673611in}}{\pgfqpoint{11.275527in}{4.543056in}}%
\pgfusepath{clip}%
\pgfsetbuttcap%
\pgfsetroundjoin%
\pgfsetlinewidth{2.007500pt}%
\definecolor{currentstroke}{rgb}{0.501961,0.501961,0.501961}%
\pgfsetstrokecolor{currentstroke}%
\pgfsetstrokeopacity{0.500000}%
\pgfsetdash{}{0pt}%
\pgfpathmoveto{\pgfqpoint{10.662014in}{3.633480in}}%
\pgfpathlineto{\pgfqpoint{10.863003in}{3.633480in}}%
\pgfusepath{stroke}%
\end{pgfscope}%
\begin{pgfscope}%
\pgfpathrectangle{\pgfqpoint{0.100000in}{0.673611in}}{\pgfqpoint{11.275527in}{4.543056in}}%
\pgfusepath{clip}%
\pgfsetbuttcap%
\pgfsetroundjoin%
\pgfsetlinewidth{2.007500pt}%
\definecolor{currentstroke}{rgb}{0.501961,0.501961,0.501961}%
\pgfsetstrokecolor{currentstroke}%
\pgfsetstrokeopacity{0.500000}%
\pgfsetdash{}{0pt}%
\pgfpathmoveto{\pgfqpoint{10.662014in}{4.092375in}}%
\pgfpathlineto{\pgfqpoint{10.863003in}{4.092375in}}%
\pgfusepath{stroke}%
\end{pgfscope}%
\begin{pgfscope}%
\pgfpathrectangle{\pgfqpoint{0.100000in}{0.673611in}}{\pgfqpoint{11.275527in}{4.543056in}}%
\pgfusepath{clip}%
\pgfsetbuttcap%
\pgfsetroundjoin%
\pgfsetlinewidth{2.007500pt}%
\definecolor{currentstroke}{rgb}{0.501961,0.501961,0.501961}%
\pgfsetstrokecolor{currentstroke}%
\pgfsetstrokeopacity{0.500000}%
\pgfsetdash{}{0pt}%
\pgfpathmoveto{\pgfqpoint{10.662014in}{4.551270in}}%
\pgfpathlineto{\pgfqpoint{10.863003in}{4.551270in}}%
\pgfusepath{stroke}%
\end{pgfscope}%
\begin{pgfscope}%
\pgfpathrectangle{\pgfqpoint{0.100000in}{0.673611in}}{\pgfqpoint{11.275527in}{4.543056in}}%
\pgfusepath{clip}%
\pgfsetbuttcap%
\pgfsetroundjoin%
\pgfsetlinewidth{2.007500pt}%
\definecolor{currentstroke}{rgb}{0.501961,0.501961,0.501961}%
\pgfsetstrokecolor{currentstroke}%
\pgfsetstrokeopacity{0.500000}%
\pgfsetdash{}{0pt}%
\pgfpathmoveto{\pgfqpoint{10.662014in}{5.010164in}}%
\pgfpathlineto{\pgfqpoint{10.863003in}{5.010164in}}%
\pgfusepath{stroke}%
\end{pgfscope}%
\begin{pgfscope}%
\pgfsetrectcap%
\pgfsetmiterjoin%
\pgfsetlinewidth{0.803000pt}%
\definecolor{currentstroke}{rgb}{0.150000,0.150000,0.150000}%
\pgfsetstrokecolor{currentstroke}%
\pgfsetdash{}{0pt}%
\pgfpathmoveto{\pgfqpoint{0.100000in}{0.673611in}}%
\pgfpathlineto{\pgfqpoint{11.375527in}{0.673611in}}%
\pgfusepath{stroke}%
\end{pgfscope}%
\begin{pgfscope}%
\pgfsetrectcap%
\pgfsetmiterjoin%
\pgfsetlinewidth{0.803000pt}%
\definecolor{currentstroke}{rgb}{0.150000,0.150000,0.150000}%
\pgfsetstrokecolor{currentstroke}%
\pgfsetdash{}{0pt}%
\pgfpathmoveto{\pgfqpoint{0.100000in}{5.216667in}}%
\pgfpathlineto{\pgfqpoint{11.375527in}{5.216667in}}%
\pgfusepath{stroke}%
\end{pgfscope}%
\begin{pgfscope}%
\definecolor{textcolor}{rgb}{0.150000,0.150000,0.150000}%
\pgfsetstrokecolor{textcolor}%
\pgfsetfillcolor{textcolor}%
\pgftext[x=0.445032in,y=0.467108in,left,base]{\color{textcolor}\rmfamily\fontsize{10.000000}{12.000000}\selectfont 5.43e+03}%
\end{pgfscope}%
\begin{pgfscope}%
\definecolor{textcolor}{rgb}{0.150000,0.150000,0.150000}%
\pgfsetstrokecolor{textcolor}%
\pgfsetfillcolor{textcolor}%
\pgftext[x=0.445032in,y=5.319918in,left,base]{\color{textcolor}\rmfamily\fontsize{10.000000}{12.000000}\selectfont 1.37e+04}%
\end{pgfscope}%
\begin{pgfscope}%
\definecolor{textcolor}{rgb}{0.150000,0.150000,0.150000}%
\pgfsetstrokecolor{textcolor}%
\pgfsetfillcolor{textcolor}%
\pgftext[x=3.794862in,y=0.467108in,left,base]{\color{textcolor}\rmfamily\fontsize{10.000000}{12.000000}\selectfont 2.35}%
\end{pgfscope}%
\begin{pgfscope}%
\definecolor{textcolor}{rgb}{0.150000,0.150000,0.150000}%
\pgfsetstrokecolor{textcolor}%
\pgfsetfillcolor{textcolor}%
\pgftext[x=3.794862in,y=5.319918in,left,base]{\color{textcolor}\rmfamily\fontsize{10.000000}{12.000000}\selectfont 40.56}%
\end{pgfscope}%
\begin{pgfscope}%
\definecolor{textcolor}{rgb}{0.150000,0.150000,0.150000}%
\pgfsetstrokecolor{textcolor}%
\pgfsetfillcolor{textcolor}%
\pgftext[x=7.144692in,y=0.467108in,left,base]{\color{textcolor}\rmfamily\fontsize{10.000000}{12.000000}\selectfont 143.98}%
\end{pgfscope}%
\begin{pgfscope}%
\definecolor{textcolor}{rgb}{0.150000,0.150000,0.150000}%
\pgfsetstrokecolor{textcolor}%
\pgfsetfillcolor{textcolor}%
\pgftext[x=7.144692in,y=5.319918in,left,base]{\color{textcolor}\rmfamily\fontsize{10.000000}{12.000000}\selectfont 8.73e+05}%
\end{pgfscope}%
\begin{pgfscope}%
\definecolor{textcolor}{rgb}{0.150000,0.150000,0.150000}%
\pgfsetstrokecolor{textcolor}%
\pgfsetfillcolor{textcolor}%
\pgftext[x=10.494522in,y=0.467108in,left,base]{\color{textcolor}\rmfamily\fontsize{10.000000}{12.000000}\selectfont 0.00}%
\end{pgfscope}%
\begin{pgfscope}%
\definecolor{textcolor}{rgb}{0.150000,0.150000,0.150000}%
\pgfsetstrokecolor{textcolor}%
\pgfsetfillcolor{textcolor}%
\pgftext[x=10.494522in,y=5.319918in,left,base]{\color{textcolor}\rmfamily\fontsize{10.000000}{12.000000}\selectfont 1.00}%
\end{pgfscope}%
\begin{pgfscope}%
\definecolor{textcolor}{rgb}{0.150000,0.150000,0.150000}%
\pgfsetstrokecolor{textcolor}%
\pgfsetfillcolor{textcolor}%
\pgftext[x=5.737764in,y=5.633333in,,base]{\color{textcolor}\rmfamily\fontsize{16.000000}{19.200000}\selectfont Objective Space}%
\end{pgfscope}%
\begin{pgfscope}%
\pgfsetbuttcap%
\pgfsetmiterjoin%
\definecolor{currentfill}{rgb}{1.000000,1.000000,1.000000}%
\pgfsetfillcolor{currentfill}%
\pgfsetfillopacity{0.800000}%
\pgfsetlinewidth{1.003750pt}%
\definecolor{currentstroke}{rgb}{0.800000,0.800000,0.800000}%
\pgfsetstrokecolor{currentstroke}%
\pgfsetstrokeopacity{0.800000}%
\pgfsetdash{}{0pt}%
\pgfpathmoveto{\pgfqpoint{0.216667in}{4.153705in}}%
\pgfpathlineto{\pgfqpoint{2.103960in}{4.153705in}}%
\pgfpathquadraticcurveto{\pgfqpoint{2.137293in}{4.153705in}}{\pgfqpoint{2.137293in}{4.187038in}}%
\pgfpathlineto{\pgfqpoint{2.137293in}{5.100000in}}%
\pgfpathquadraticcurveto{\pgfqpoint{2.137293in}{5.133333in}}{\pgfqpoint{2.103960in}{5.133333in}}%
\pgfpathlineto{\pgfqpoint{0.216667in}{5.133333in}}%
\pgfpathquadraticcurveto{\pgfqpoint{0.183333in}{5.133333in}}{\pgfqpoint{0.183333in}{5.100000in}}%
\pgfpathlineto{\pgfqpoint{0.183333in}{4.187038in}}%
\pgfpathquadraticcurveto{\pgfqpoint{0.183333in}{4.153705in}}{\pgfqpoint{0.216667in}{4.153705in}}%
\pgfpathlineto{\pgfqpoint{0.216667in}{4.153705in}}%
\pgfpathclose%
\pgfusepath{stroke,fill}%
\end{pgfscope}%
\begin{pgfscope}%
\pgfsetroundcap%
\pgfsetroundjoin%
\pgfsetlinewidth{5.018750pt}%
\definecolor{currentstroke}{rgb}{0.172549,0.627451,0.172549}%
\pgfsetstrokecolor{currentstroke}%
\pgfsetdash{}{0pt}%
\pgfpathmoveto{\pgfqpoint{0.250000in}{5.008333in}}%
\pgfpathlineto{\pgfqpoint{0.416667in}{5.008333in}}%
\pgfpathlineto{\pgfqpoint{0.583333in}{5.008333in}}%
\pgfusepath{stroke}%
\end{pgfscope}%
\begin{pgfscope}%
\definecolor{textcolor}{rgb}{0.150000,0.150000,0.150000}%
\pgfsetstrokecolor{textcolor}%
\pgfsetfillcolor{textcolor}%
\pgftext[x=0.716667in,y=4.950000in,left,base]{\color{textcolor}\rmfamily\fontsize{12.000000}{14.400000}\selectfont Least CO\(\displaystyle _2\)}%
\end{pgfscope}%
\begin{pgfscope}%
\pgfsetroundcap%
\pgfsetroundjoin%
\pgfsetlinewidth{5.018750pt}%
\definecolor{currentstroke}{rgb}{0.839216,0.152941,0.156863}%
\pgfsetstrokecolor{currentstroke}%
\pgfsetdash{}{0pt}%
\pgfpathmoveto{\pgfqpoint{0.250000in}{4.775926in}}%
\pgfpathlineto{\pgfqpoint{0.416667in}{4.775926in}}%
\pgfpathlineto{\pgfqpoint{0.583333in}{4.775926in}}%
\pgfusepath{stroke}%
\end{pgfscope}%
\begin{pgfscope}%
\definecolor{textcolor}{rgb}{0.150000,0.150000,0.150000}%
\pgfsetstrokecolor{textcolor}%
\pgfsetfillcolor{textcolor}%
\pgftext[x=0.716667in,y=4.717593in,left,base]{\color{textcolor}\rmfamily\fontsize{12.000000}{14.400000}\selectfont Least Cost}%
\end{pgfscope}%
\begin{pgfscope}%
\pgfsetroundcap%
\pgfsetroundjoin%
\pgfsetlinewidth{5.018750pt}%
\definecolor{currentstroke}{rgb}{1.000000,0.498039,0.054902}%
\pgfsetstrokecolor{currentstroke}%
\pgfsetdash{}{0pt}%
\pgfpathmoveto{\pgfqpoint{0.250000in}{4.543519in}}%
\pgfpathlineto{\pgfqpoint{0.416667in}{4.543519in}}%
\pgfpathlineto{\pgfqpoint{0.583333in}{4.543519in}}%
\pgfusepath{stroke}%
\end{pgfscope}%
\begin{pgfscope}%
\definecolor{textcolor}{rgb}{0.150000,0.150000,0.150000}%
\pgfsetstrokecolor{textcolor}%
\pgfsetfillcolor{textcolor}%
\pgftext[x=0.716667in,y=4.485186in,left,base]{\color{textcolor}\rmfamily\fontsize{12.000000}{14.400000}\selectfont Least Land Use}%
\end{pgfscope}%
\begin{pgfscope}%
\pgfsetroundcap%
\pgfsetroundjoin%
\pgfsetlinewidth{5.018750pt}%
\definecolor{currentstroke}{rgb}{0.121569,0.466667,0.705882}%
\pgfsetstrokecolor{currentstroke}%
\pgfsetdash{}{0pt}%
\pgfpathmoveto{\pgfqpoint{0.250000in}{4.311112in}}%
\pgfpathlineto{\pgfqpoint{0.416667in}{4.311112in}}%
\pgfpathlineto{\pgfqpoint{0.583333in}{4.311112in}}%
\pgfusepath{stroke}%
\end{pgfscope}%
\begin{pgfscope}%
\definecolor{textcolor}{rgb}{0.150000,0.150000,0.150000}%
\pgfsetstrokecolor{textcolor}%
\pgfsetfillcolor{textcolor}%
\pgftext[x=0.716667in,y=4.252779in,left,base]{\color{textcolor}\rmfamily\fontsize{12.000000}{14.400000}\selectfont Highest Renewable}%
\end{pgfscope}%
\end{pgfpicture}%
\makeatother%
\endgroup%
}
  \caption{The Pareto front for a four objective problem. Extreme values for
  each objective are colored. The gray lines represent solutions on the Pareto
  front that are not extremum.}
  \label{fig:4-obj-pareto}
\end{figure}

Each of the colored lines in Figure \ref{fig:4-obj-pareto} belongs to a solution
with an `extreme' value on the Pareto-front. For instance, the blue line labeled
``Highest Renewable'' has the lowest percentage of non-renewable energy sources
of any solution. The gray lines are simply other points along the Pareto-front.
Figure \ref{fig:4-obj-pareto} shows that minimizing land-use change and
renewable energy maximization are strongly competing objectives, since the other
three extremum are grouped together on those two axes and diametrically opposed
to the ``highest renewable'' solution. Figure \ref{fig:4-obj-design} illustrates
the design space for each extreme solution. 


\begin{figure}[h]
  \centering
  \resizebox{\columnwidth}{!}{%% Creator: Matplotlib, PGF backend
%%
%% To include the figure in your LaTeX document, write
%%   \input{<filename>.pgf}
%%
%% Make sure the required packages are loaded in your preamble
%%   \usepackage{pgf}
%%
%% Also ensure that all the required font packages are loaded; for instance,
%% the lmodern package is sometimes necessary when using math font.
%%   \usepackage{lmodern}
%%
%% Figures using additional raster images can only be included by \input if
%% they are in the same directory as the main LaTeX file. For loading figures
%% from other directories you can use the `import` package
%%   \usepackage{import}
%%
%% and then include the figures with
%%   \import{<path to file>}{<filename>.pgf}
%%
%% Matplotlib used the following preamble
%%   \def\mathdefault#1{#1}
%%   \everymath=\expandafter{\the\everymath\displaystyle}
%%   \IfFileExists{scrextend.sty}{
%%     \usepackage[fontsize=10.000000pt]{scrextend}
%%   }{
%%     \renewcommand{\normalsize}{\fontsize{10.000000}{12.000000}\selectfont}
%%     \normalsize
%%   }
%%   
%%   \makeatletter\@ifpackageloaded{underscore}{}{\usepackage[strings]{underscore}}\makeatother
%%
\begingroup%
\makeatletter%
\begin{pgfpicture}%
\pgfpathrectangle{\pgfpointorigin}{\pgfqpoint{12.817144in}{5.900000in}}%
\pgfusepath{use as bounding box, clip}%
\begin{pgfscope}%
\pgfsetbuttcap%
\pgfsetmiterjoin%
\definecolor{currentfill}{rgb}{1.000000,1.000000,1.000000}%
\pgfsetfillcolor{currentfill}%
\pgfsetlinewidth{0.000000pt}%
\definecolor{currentstroke}{rgb}{0.000000,0.000000,0.000000}%
\pgfsetstrokecolor{currentstroke}%
\pgfsetdash{}{0pt}%
\pgfpathmoveto{\pgfqpoint{0.000000in}{0.000000in}}%
\pgfpathlineto{\pgfqpoint{12.817144in}{0.000000in}}%
\pgfpathlineto{\pgfqpoint{12.817144in}{5.900000in}}%
\pgfpathlineto{\pgfqpoint{0.000000in}{5.900000in}}%
\pgfpathlineto{\pgfqpoint{0.000000in}{0.000000in}}%
\pgfpathclose%
\pgfusepath{fill}%
\end{pgfscope}%
\begin{pgfscope}%
\pgfsetbuttcap%
\pgfsetmiterjoin%
\definecolor{currentfill}{rgb}{1.000000,1.000000,1.000000}%
\pgfsetfillcolor{currentfill}%
\pgfsetlinewidth{0.000000pt}%
\definecolor{currentstroke}{rgb}{0.000000,0.000000,0.000000}%
\pgfsetstrokecolor{currentstroke}%
\pgfsetstrokeopacity{0.000000}%
\pgfsetdash{}{0pt}%
\pgfpathmoveto{\pgfqpoint{0.100000in}{0.879166in}}%
\pgfpathlineto{\pgfqpoint{12.717144in}{0.879166in}}%
\pgfpathlineto{\pgfqpoint{12.717144in}{5.183295in}}%
\pgfpathlineto{\pgfqpoint{0.100000in}{5.183295in}}%
\pgfpathlineto{\pgfqpoint{0.100000in}{0.879166in}}%
\pgfpathclose%
\pgfusepath{fill}%
\end{pgfscope}%
\begin{pgfscope}%
\pgfsetbuttcap%
\pgfsetroundjoin%
\definecolor{currentfill}{rgb}{0.000000,0.000000,0.000000}%
\pgfsetfillcolor{currentfill}%
\pgfsetlinewidth{0.803000pt}%
\definecolor{currentstroke}{rgb}{0.000000,0.000000,0.000000}%
\pgfsetstrokecolor{currentstroke}%
\pgfsetdash{}{0pt}%
\pgfsys@defobject{currentmarker}{\pgfqpoint{0.000000in}{-0.048611in}}{\pgfqpoint{0.000000in}{0.000000in}}{%
\pgfpathmoveto{\pgfqpoint{0.000000in}{0.000000in}}%
\pgfpathlineto{\pgfqpoint{0.000000in}{-0.048611in}}%
\pgfusepath{stroke,fill}%
}%
\begin{pgfscope}%
\pgfsys@transformshift{0.711487in}{0.879166in}%
\pgfsys@useobject{currentmarker}{}%
\end{pgfscope}%
\end{pgfscope}%
\begin{pgfscope}%
\definecolor{textcolor}{rgb}{0.000000,0.000000,0.000000}%
\pgfsetstrokecolor{textcolor}%
\pgfsetfillcolor{textcolor}%
\pgftext[x=0.711487in,y=0.483332in,,top]{\color{textcolor}{\rmfamily\fontsize{14.000000}{16.800000}\selectfont\catcode`\^=\active\def^{\ifmmode\sp\else\^{}\fi}\catcode`\%=\active\def%{\%}Battery}}%
\end{pgfscope}%
\begin{pgfscope}%
\pgfsetbuttcap%
\pgfsetroundjoin%
\definecolor{currentfill}{rgb}{0.000000,0.000000,0.000000}%
\pgfsetfillcolor{currentfill}%
\pgfsetlinewidth{0.803000pt}%
\definecolor{currentstroke}{rgb}{0.000000,0.000000,0.000000}%
\pgfsetstrokecolor{currentstroke}%
\pgfsetdash{}{0pt}%
\pgfsys@defobject{currentmarker}{\pgfqpoint{0.000000in}{-0.048611in}}{\pgfqpoint{0.000000in}{0.000000in}}{%
\pgfpathmoveto{\pgfqpoint{0.000000in}{0.000000in}}%
\pgfpathlineto{\pgfqpoint{0.000000in}{-0.048611in}}%
\pgfusepath{stroke,fill}%
}%
\begin{pgfscope}%
\pgfsys@transformshift{1.977506in}{0.879166in}%
\pgfsys@useobject{currentmarker}{}%
\end{pgfscope}%
\end{pgfscope}%
\begin{pgfscope}%
\definecolor{textcolor}{rgb}{0.000000,0.000000,0.000000}%
\pgfsetstrokecolor{textcolor}%
\pgfsetfillcolor{textcolor}%
\pgftext[x=1.977506in,y=0.483332in,,top]{\color{textcolor}{\rmfamily\fontsize{14.000000}{16.800000}\selectfont\catcode`\^=\active\def^{\ifmmode\sp\else\^{}\fi}\catcode`\%=\active\def%{\%}Biomass}}%
\end{pgfscope}%
\begin{pgfscope}%
\pgfsetbuttcap%
\pgfsetroundjoin%
\definecolor{currentfill}{rgb}{0.000000,0.000000,0.000000}%
\pgfsetfillcolor{currentfill}%
\pgfsetlinewidth{0.803000pt}%
\definecolor{currentstroke}{rgb}{0.000000,0.000000,0.000000}%
\pgfsetstrokecolor{currentstroke}%
\pgfsetdash{}{0pt}%
\pgfsys@defobject{currentmarker}{\pgfqpoint{0.000000in}{-0.048611in}}{\pgfqpoint{0.000000in}{0.000000in}}{%
\pgfpathmoveto{\pgfqpoint{0.000000in}{0.000000in}}%
\pgfpathlineto{\pgfqpoint{0.000000in}{-0.048611in}}%
\pgfusepath{stroke,fill}%
}%
\begin{pgfscope}%
\pgfsys@transformshift{3.243525in}{0.879166in}%
\pgfsys@useobject{currentmarker}{}%
\end{pgfscope}%
\end{pgfscope}%
\begin{pgfscope}%
\definecolor{textcolor}{rgb}{0.000000,0.000000,0.000000}%
\pgfsetstrokecolor{textcolor}%
\pgfsetfillcolor{textcolor}%
\pgftext[x=3.047694in, y=0.344444in, left, base]{\color{textcolor}{\rmfamily\fontsize{14.000000}{16.800000}\selectfont\catcode`\^=\active\def^{\ifmmode\sp\else\^{}\fi}\catcode`\%=\active\def%{\%}Coal}}%
\end{pgfscope}%
\begin{pgfscope}%
\definecolor{textcolor}{rgb}{0.000000,0.000000,0.000000}%
\pgfsetstrokecolor{textcolor}%
\pgfsetfillcolor{textcolor}%
\pgftext[x=2.683231in, y=0.138889in, left, base]{\color{textcolor}{\rmfamily\fontsize{14.000000}{16.800000}\selectfont\catcode`\^=\active\def^{\ifmmode\sp\else\^{}\fi}\catcode`\%=\active\def%{\%}Conventional}}%
\end{pgfscope}%
\begin{pgfscope}%
\pgfsetbuttcap%
\pgfsetroundjoin%
\definecolor{currentfill}{rgb}{0.000000,0.000000,0.000000}%
\pgfsetfillcolor{currentfill}%
\pgfsetlinewidth{0.803000pt}%
\definecolor{currentstroke}{rgb}{0.000000,0.000000,0.000000}%
\pgfsetstrokecolor{currentstroke}%
\pgfsetdash{}{0pt}%
\pgfsys@defobject{currentmarker}{\pgfqpoint{0.000000in}{-0.048611in}}{\pgfqpoint{0.000000in}{0.000000in}}{%
\pgfpathmoveto{\pgfqpoint{0.000000in}{0.000000in}}%
\pgfpathlineto{\pgfqpoint{0.000000in}{-0.048611in}}%
\pgfusepath{stroke,fill}%
}%
\begin{pgfscope}%
\pgfsys@transformshift{4.509544in}{0.879166in}%
\pgfsys@useobject{currentmarker}{}%
\end{pgfscope}%
\end{pgfscope}%
\begin{pgfscope}%
\definecolor{textcolor}{rgb}{0.000000,0.000000,0.000000}%
\pgfsetstrokecolor{textcolor}%
\pgfsetfillcolor{textcolor}%
\pgftext[x=4.313713in, y=0.344444in, left, base]{\color{textcolor}{\rmfamily\fontsize{14.000000}{16.800000}\selectfont\catcode`\^=\active\def^{\ifmmode\sp\else\^{}\fi}\catcode`\%=\active\def%{\%}Coal}}%
\end{pgfscope}%
\begin{pgfscope}%
\definecolor{textcolor}{rgb}{0.000000,0.000000,0.000000}%
\pgfsetstrokecolor{textcolor}%
\pgfsetfillcolor{textcolor}%
\pgftext[x=4.090718in, y=0.138889in, left, base]{\color{textcolor}{\rmfamily\fontsize{14.000000}{16.800000}\selectfont\catcode`\^=\active\def^{\ifmmode\sp\else\^{}\fi}\catcode`\%=\active\def%{\%}Advanced}}%
\end{pgfscope}%
\begin{pgfscope}%
\pgfsetbuttcap%
\pgfsetroundjoin%
\definecolor{currentfill}{rgb}{0.000000,0.000000,0.000000}%
\pgfsetfillcolor{currentfill}%
\pgfsetlinewidth{0.803000pt}%
\definecolor{currentstroke}{rgb}{0.000000,0.000000,0.000000}%
\pgfsetstrokecolor{currentstroke}%
\pgfsetdash{}{0pt}%
\pgfsys@defobject{currentmarker}{\pgfqpoint{0.000000in}{-0.048611in}}{\pgfqpoint{0.000000in}{0.000000in}}{%
\pgfpathmoveto{\pgfqpoint{0.000000in}{0.000000in}}%
\pgfpathlineto{\pgfqpoint{0.000000in}{-0.048611in}}%
\pgfusepath{stroke,fill}%
}%
\begin{pgfscope}%
\pgfsys@transformshift{5.775562in}{0.879166in}%
\pgfsys@useobject{currentmarker}{}%
\end{pgfscope}%
\end{pgfscope}%
\begin{pgfscope}%
\definecolor{textcolor}{rgb}{0.000000,0.000000,0.000000}%
\pgfsetstrokecolor{textcolor}%
\pgfsetfillcolor{textcolor}%
\pgftext[x=5.216808in, y=0.344444in, left, base]{\color{textcolor}{\rmfamily\fontsize{14.000000}{16.800000}\selectfont\catcode`\^=\active\def^{\ifmmode\sp\else\^{}\fi}\catcode`\%=\active\def%{\%}Natural Gas }}%
\end{pgfscope}%
\begin{pgfscope}%
\definecolor{textcolor}{rgb}{0.000000,0.000000,0.000000}%
\pgfsetstrokecolor{textcolor}%
\pgfsetfillcolor{textcolor}%
\pgftext[x=5.215269in, y=0.138889in, left, base]{\color{textcolor}{\rmfamily\fontsize{14.000000}{16.800000}\selectfont\catcode`\^=\active\def^{\ifmmode\sp\else\^{}\fi}\catcode`\%=\active\def%{\%}Conventional}}%
\end{pgfscope}%
\begin{pgfscope}%
\pgfsetbuttcap%
\pgfsetroundjoin%
\definecolor{currentfill}{rgb}{0.000000,0.000000,0.000000}%
\pgfsetfillcolor{currentfill}%
\pgfsetlinewidth{0.803000pt}%
\definecolor{currentstroke}{rgb}{0.000000,0.000000,0.000000}%
\pgfsetstrokecolor{currentstroke}%
\pgfsetdash{}{0pt}%
\pgfsys@defobject{currentmarker}{\pgfqpoint{0.000000in}{-0.048611in}}{\pgfqpoint{0.000000in}{0.000000in}}{%
\pgfpathmoveto{\pgfqpoint{0.000000in}{0.000000in}}%
\pgfpathlineto{\pgfqpoint{0.000000in}{-0.048611in}}%
\pgfusepath{stroke,fill}%
}%
\begin{pgfscope}%
\pgfsys@transformshift{7.041581in}{0.879166in}%
\pgfsys@useobject{currentmarker}{}%
\end{pgfscope}%
\end{pgfscope}%
\begin{pgfscope}%
\definecolor{textcolor}{rgb}{0.000000,0.000000,0.000000}%
\pgfsetstrokecolor{textcolor}%
\pgfsetfillcolor{textcolor}%
\pgftext[x=6.482826in, y=0.344444in, left, base]{\color{textcolor}{\rmfamily\fontsize{14.000000}{16.800000}\selectfont\catcode`\^=\active\def^{\ifmmode\sp\else\^{}\fi}\catcode`\%=\active\def%{\%}Natural Gas }}%
\end{pgfscope}%
\begin{pgfscope}%
\definecolor{textcolor}{rgb}{0.000000,0.000000,0.000000}%
\pgfsetstrokecolor{textcolor}%
\pgfsetfillcolor{textcolor}%
\pgftext[x=6.622755in, y=0.138889in, left, base]{\color{textcolor}{\rmfamily\fontsize{14.000000}{16.800000}\selectfont\catcode`\^=\active\def^{\ifmmode\sp\else\^{}\fi}\catcode`\%=\active\def%{\%}Advanced}}%
\end{pgfscope}%
\begin{pgfscope}%
\pgfsetbuttcap%
\pgfsetroundjoin%
\definecolor{currentfill}{rgb}{0.000000,0.000000,0.000000}%
\pgfsetfillcolor{currentfill}%
\pgfsetlinewidth{0.803000pt}%
\definecolor{currentstroke}{rgb}{0.000000,0.000000,0.000000}%
\pgfsetstrokecolor{currentstroke}%
\pgfsetdash{}{0pt}%
\pgfsys@defobject{currentmarker}{\pgfqpoint{0.000000in}{-0.048611in}}{\pgfqpoint{0.000000in}{0.000000in}}{%
\pgfpathmoveto{\pgfqpoint{0.000000in}{0.000000in}}%
\pgfpathlineto{\pgfqpoint{0.000000in}{-0.048611in}}%
\pgfusepath{stroke,fill}%
}%
\begin{pgfscope}%
\pgfsys@transformshift{8.307600in}{0.879166in}%
\pgfsys@useobject{currentmarker}{}%
\end{pgfscope}%
\end{pgfscope}%
\begin{pgfscope}%
\definecolor{textcolor}{rgb}{0.000000,0.000000,0.000000}%
\pgfsetstrokecolor{textcolor}%
\pgfsetfillcolor{textcolor}%
\pgftext[x=8.307600in,y=0.483332in,,top]{\color{textcolor}{\rmfamily\fontsize{14.000000}{16.800000}\selectfont\catcode`\^=\active\def^{\ifmmode\sp\else\^{}\fi}\catcode`\%=\active\def%{\%}Nuclear}}%
\end{pgfscope}%
\begin{pgfscope}%
\pgfsetbuttcap%
\pgfsetroundjoin%
\definecolor{currentfill}{rgb}{0.000000,0.000000,0.000000}%
\pgfsetfillcolor{currentfill}%
\pgfsetlinewidth{0.803000pt}%
\definecolor{currentstroke}{rgb}{0.000000,0.000000,0.000000}%
\pgfsetstrokecolor{currentstroke}%
\pgfsetdash{}{0pt}%
\pgfsys@defobject{currentmarker}{\pgfqpoint{0.000000in}{-0.048611in}}{\pgfqpoint{0.000000in}{0.000000in}}{%
\pgfpathmoveto{\pgfqpoint{0.000000in}{0.000000in}}%
\pgfpathlineto{\pgfqpoint{0.000000in}{-0.048611in}}%
\pgfusepath{stroke,fill}%
}%
\begin{pgfscope}%
\pgfsys@transformshift{9.573619in}{0.879166in}%
\pgfsys@useobject{currentmarker}{}%
\end{pgfscope}%
\end{pgfscope}%
\begin{pgfscope}%
\definecolor{textcolor}{rgb}{0.000000,0.000000,0.000000}%
\pgfsetstrokecolor{textcolor}%
\pgfsetfillcolor{textcolor}%
\pgftext[x=9.244549in, y=0.344444in, left, base]{\color{textcolor}{\rmfamily\fontsize{14.000000}{16.800000}\selectfont\catcode`\^=\active\def^{\ifmmode\sp\else\^{}\fi}\catcode`\%=\active\def%{\%}Nuclear}}%
\end{pgfscope}%
\begin{pgfscope}%
\definecolor{textcolor}{rgb}{0.000000,0.000000,0.000000}%
\pgfsetstrokecolor{textcolor}%
\pgfsetfillcolor{textcolor}%
\pgftext[x=9.154793in, y=0.138889in, left, base]{\color{textcolor}{\rmfamily\fontsize{14.000000}{16.800000}\selectfont\catcode`\^=\active\def^{\ifmmode\sp\else\^{}\fi}\catcode`\%=\active\def%{\%}Advanced}}%
\end{pgfscope}%
\begin{pgfscope}%
\pgfsetbuttcap%
\pgfsetroundjoin%
\definecolor{currentfill}{rgb}{0.000000,0.000000,0.000000}%
\pgfsetfillcolor{currentfill}%
\pgfsetlinewidth{0.803000pt}%
\definecolor{currentstroke}{rgb}{0.000000,0.000000,0.000000}%
\pgfsetstrokecolor{currentstroke}%
\pgfsetdash{}{0pt}%
\pgfsys@defobject{currentmarker}{\pgfqpoint{0.000000in}{-0.048611in}}{\pgfqpoint{0.000000in}{0.000000in}}{%
\pgfpathmoveto{\pgfqpoint{0.000000in}{0.000000in}}%
\pgfpathlineto{\pgfqpoint{0.000000in}{-0.048611in}}%
\pgfusepath{stroke,fill}%
}%
\begin{pgfscope}%
\pgfsys@transformshift{10.839638in}{0.879166in}%
\pgfsys@useobject{currentmarker}{}%
\end{pgfscope}%
\end{pgfscope}%
\begin{pgfscope}%
\definecolor{textcolor}{rgb}{0.000000,0.000000,0.000000}%
\pgfsetstrokecolor{textcolor}%
\pgfsetfillcolor{textcolor}%
\pgftext[x=10.839638in,y=0.483332in,,top]{\color{textcolor}{\rmfamily\fontsize{14.000000}{16.800000}\selectfont\catcode`\^=\active\def^{\ifmmode\sp\else\^{}\fi}\catcode`\%=\active\def%{\%}SolarPanel}}%
\end{pgfscope}%
\begin{pgfscope}%
\pgfsetbuttcap%
\pgfsetroundjoin%
\definecolor{currentfill}{rgb}{0.000000,0.000000,0.000000}%
\pgfsetfillcolor{currentfill}%
\pgfsetlinewidth{0.803000pt}%
\definecolor{currentstroke}{rgb}{0.000000,0.000000,0.000000}%
\pgfsetstrokecolor{currentstroke}%
\pgfsetdash{}{0pt}%
\pgfsys@defobject{currentmarker}{\pgfqpoint{0.000000in}{-0.048611in}}{\pgfqpoint{0.000000in}{0.000000in}}{%
\pgfpathmoveto{\pgfqpoint{0.000000in}{0.000000in}}%
\pgfpathlineto{\pgfqpoint{0.000000in}{-0.048611in}}%
\pgfusepath{stroke,fill}%
}%
\begin{pgfscope}%
\pgfsys@transformshift{12.105657in}{0.879166in}%
\pgfsys@useobject{currentmarker}{}%
\end{pgfscope}%
\end{pgfscope}%
\begin{pgfscope}%
\definecolor{textcolor}{rgb}{0.000000,0.000000,0.000000}%
\pgfsetstrokecolor{textcolor}%
\pgfsetfillcolor{textcolor}%
\pgftext[x=12.105657in,y=0.483332in,,top]{\color{textcolor}{\rmfamily\fontsize{14.000000}{16.800000}\selectfont\catcode`\^=\active\def^{\ifmmode\sp\else\^{}\fi}\catcode`\%=\active\def%{\%}WindTurbine}}%
\end{pgfscope}%
\begin{pgfscope}%
\pgfpathrectangle{\pgfqpoint{0.100000in}{0.879166in}}{\pgfqpoint{12.617144in}{4.304129in}}%
\pgfusepath{clip}%
\pgfsetrectcap%
\pgfsetroundjoin%
\pgfsetlinewidth{1.505625pt}%
\definecolor{currentstroke}{rgb}{0.501961,0.501961,0.501961}%
\pgfsetstrokecolor{currentstroke}%
\pgfsetstrokeopacity{0.300000}%
\pgfsetdash{}{0pt}%
\pgfpathmoveto{\pgfqpoint{0.711487in}{1.244404in}}%
\pgfpathlineto{\pgfqpoint{1.977506in}{1.081758in}}%
\pgfpathlineto{\pgfqpoint{3.243525in}{2.407373in}}%
\pgfpathlineto{\pgfqpoint{4.509544in}{1.155948in}}%
\pgfpathlineto{\pgfqpoint{5.775562in}{1.879558in}}%
\pgfpathlineto{\pgfqpoint{7.041581in}{1.185271in}}%
\pgfpathlineto{\pgfqpoint{8.307600in}{4.203035in}}%
\pgfpathlineto{\pgfqpoint{9.573619in}{4.987653in}}%
\pgfpathlineto{\pgfqpoint{10.839638in}{1.431251in}}%
\pgfpathlineto{\pgfqpoint{12.105657in}{1.075156in}}%
\pgfusepath{stroke}%
\end{pgfscope}%
\begin{pgfscope}%
\pgfpathrectangle{\pgfqpoint{0.100000in}{0.879166in}}{\pgfqpoint{12.617144in}{4.304129in}}%
\pgfusepath{clip}%
\pgfsetrectcap%
\pgfsetroundjoin%
\pgfsetlinewidth{1.505625pt}%
\definecolor{currentstroke}{rgb}{0.501961,0.501961,0.501961}%
\pgfsetstrokecolor{currentstroke}%
\pgfsetstrokeopacity{0.300000}%
\pgfsetdash{}{0pt}%
\pgfpathmoveto{\pgfqpoint{0.711487in}{2.558452in}}%
\pgfpathlineto{\pgfqpoint{1.977506in}{1.190110in}}%
\pgfpathlineto{\pgfqpoint{3.243525in}{2.586080in}}%
\pgfpathlineto{\pgfqpoint{4.509544in}{1.382600in}}%
\pgfpathlineto{\pgfqpoint{5.775562in}{1.074808in}}%
\pgfpathlineto{\pgfqpoint{7.041581in}{1.074955in}}%
\pgfpathlineto{\pgfqpoint{8.307600in}{4.987653in}}%
\pgfpathlineto{\pgfqpoint{9.573619in}{1.960927in}}%
\pgfpathlineto{\pgfqpoint{10.839638in}{1.214639in}}%
\pgfpathlineto{\pgfqpoint{12.105657in}{1.074938in}}%
\pgfusepath{stroke}%
\end{pgfscope}%
\begin{pgfscope}%
\pgfpathrectangle{\pgfqpoint{0.100000in}{0.879166in}}{\pgfqpoint{12.617144in}{4.304129in}}%
\pgfusepath{clip}%
\pgfsetrectcap%
\pgfsetroundjoin%
\pgfsetlinewidth{1.505625pt}%
\definecolor{currentstroke}{rgb}{0.501961,0.501961,0.501961}%
\pgfsetstrokecolor{currentstroke}%
\pgfsetstrokeopacity{0.300000}%
\pgfsetdash{}{0pt}%
\pgfpathmoveto{\pgfqpoint{0.711487in}{2.092050in}}%
\pgfpathlineto{\pgfqpoint{1.977506in}{1.075561in}}%
\pgfpathlineto{\pgfqpoint{3.243525in}{3.338246in}}%
\pgfpathlineto{\pgfqpoint{4.509544in}{1.299757in}}%
\pgfpathlineto{\pgfqpoint{5.775562in}{2.893183in}}%
\pgfpathlineto{\pgfqpoint{7.041581in}{1.725315in}}%
\pgfpathlineto{\pgfqpoint{8.307600in}{4.674970in}}%
\pgfpathlineto{\pgfqpoint{9.573619in}{1.474680in}}%
\pgfpathlineto{\pgfqpoint{10.839638in}{4.987653in}}%
\pgfpathlineto{\pgfqpoint{12.105657in}{1.084058in}}%
\pgfusepath{stroke}%
\end{pgfscope}%
\begin{pgfscope}%
\pgfpathrectangle{\pgfqpoint{0.100000in}{0.879166in}}{\pgfqpoint{12.617144in}{4.304129in}}%
\pgfusepath{clip}%
\pgfsetrectcap%
\pgfsetroundjoin%
\pgfsetlinewidth{1.505625pt}%
\definecolor{currentstroke}{rgb}{0.501961,0.501961,0.501961}%
\pgfsetstrokecolor{currentstroke}%
\pgfsetstrokeopacity{0.300000}%
\pgfsetdash{}{0pt}%
\pgfpathmoveto{\pgfqpoint{0.711487in}{2.609826in}}%
\pgfpathlineto{\pgfqpoint{1.977506in}{1.075025in}}%
\pgfpathlineto{\pgfqpoint{3.243525in}{4.987653in}}%
\pgfpathlineto{\pgfqpoint{4.509544in}{1.167228in}}%
\pgfpathlineto{\pgfqpoint{5.775562in}{1.741186in}}%
\pgfpathlineto{\pgfqpoint{7.041581in}{1.731740in}}%
\pgfpathlineto{\pgfqpoint{8.307600in}{3.648885in}}%
\pgfpathlineto{\pgfqpoint{9.573619in}{1.480805in}}%
\pgfpathlineto{\pgfqpoint{10.839638in}{2.725090in}}%
\pgfpathlineto{\pgfqpoint{12.105657in}{1.084182in}}%
\pgfusepath{stroke}%
\end{pgfscope}%
\begin{pgfscope}%
\pgfpathrectangle{\pgfqpoint{0.100000in}{0.879166in}}{\pgfqpoint{12.617144in}{4.304129in}}%
\pgfusepath{clip}%
\pgfsetrectcap%
\pgfsetroundjoin%
\pgfsetlinewidth{1.505625pt}%
\definecolor{currentstroke}{rgb}{0.501961,0.501961,0.501961}%
\pgfsetstrokecolor{currentstroke}%
\pgfsetstrokeopacity{0.300000}%
\pgfsetdash{}{0pt}%
\pgfpathmoveto{\pgfqpoint{0.711487in}{1.074808in}}%
\pgfpathlineto{\pgfqpoint{1.977506in}{4.987653in}}%
\pgfpathlineto{\pgfqpoint{3.243525in}{2.514546in}}%
\pgfpathlineto{\pgfqpoint{4.509544in}{1.178682in}}%
\pgfpathlineto{\pgfqpoint{5.775562in}{1.248240in}}%
\pgfpathlineto{\pgfqpoint{7.041581in}{1.135751in}}%
\pgfpathlineto{\pgfqpoint{8.307600in}{1.784555in}}%
\pgfpathlineto{\pgfqpoint{9.573619in}{1.537888in}}%
\pgfpathlineto{\pgfqpoint{10.839638in}{1.173307in}}%
\pgfpathlineto{\pgfqpoint{12.105657in}{1.082028in}}%
\pgfusepath{stroke}%
\end{pgfscope}%
\begin{pgfscope}%
\pgfpathrectangle{\pgfqpoint{0.100000in}{0.879166in}}{\pgfqpoint{12.617144in}{4.304129in}}%
\pgfusepath{clip}%
\pgfsetrectcap%
\pgfsetroundjoin%
\pgfsetlinewidth{1.505625pt}%
\definecolor{currentstroke}{rgb}{0.501961,0.501961,0.501961}%
\pgfsetstrokecolor{currentstroke}%
\pgfsetstrokeopacity{0.300000}%
\pgfsetdash{}{0pt}%
\pgfpathmoveto{\pgfqpoint{0.711487in}{1.171624in}}%
\pgfpathlineto{\pgfqpoint{1.977506in}{1.094326in}}%
\pgfpathlineto{\pgfqpoint{3.243525in}{1.856248in}}%
\pgfpathlineto{\pgfqpoint{4.509544in}{1.074808in}}%
\pgfpathlineto{\pgfqpoint{5.775562in}{1.411527in}}%
\pgfpathlineto{\pgfqpoint{7.041581in}{4.987653in}}%
\pgfpathlineto{\pgfqpoint{8.307600in}{4.273440in}}%
\pgfpathlineto{\pgfqpoint{9.573619in}{1.074808in}}%
\pgfpathlineto{\pgfqpoint{10.839638in}{1.074808in}}%
\pgfpathlineto{\pgfqpoint{12.105657in}{1.074808in}}%
\pgfusepath{stroke}%
\end{pgfscope}%
\begin{pgfscope}%
\pgfpathrectangle{\pgfqpoint{0.100000in}{0.879166in}}{\pgfqpoint{12.617144in}{4.304129in}}%
\pgfusepath{clip}%
\pgfsetrectcap%
\pgfsetroundjoin%
\pgfsetlinewidth{1.505625pt}%
\definecolor{currentstroke}{rgb}{0.501961,0.501961,0.501961}%
\pgfsetstrokecolor{currentstroke}%
\pgfsetstrokeopacity{0.300000}%
\pgfsetdash{}{0pt}%
\pgfpathmoveto{\pgfqpoint{0.711487in}{2.558452in}}%
\pgfpathlineto{\pgfqpoint{1.977506in}{1.190110in}}%
\pgfpathlineto{\pgfqpoint{3.243525in}{2.586080in}}%
\pgfpathlineto{\pgfqpoint{4.509544in}{1.382600in}}%
\pgfpathlineto{\pgfqpoint{5.775562in}{1.074808in}}%
\pgfpathlineto{\pgfqpoint{7.041581in}{1.074955in}}%
\pgfpathlineto{\pgfqpoint{8.307600in}{4.987653in}}%
\pgfpathlineto{\pgfqpoint{9.573619in}{1.960927in}}%
\pgfpathlineto{\pgfqpoint{10.839638in}{1.985735in}}%
\pgfpathlineto{\pgfqpoint{12.105657in}{1.074938in}}%
\pgfusepath{stroke}%
\end{pgfscope}%
\begin{pgfscope}%
\pgfpathrectangle{\pgfqpoint{0.100000in}{0.879166in}}{\pgfqpoint{12.617144in}{4.304129in}}%
\pgfusepath{clip}%
\pgfsetrectcap%
\pgfsetroundjoin%
\pgfsetlinewidth{1.505625pt}%
\definecolor{currentstroke}{rgb}{0.501961,0.501961,0.501961}%
\pgfsetstrokecolor{currentstroke}%
\pgfsetstrokeopacity{0.300000}%
\pgfsetdash{}{0pt}%
\pgfpathmoveto{\pgfqpoint{0.711487in}{4.987653in}}%
\pgfpathlineto{\pgfqpoint{1.977506in}{1.074808in}}%
\pgfpathlineto{\pgfqpoint{3.243525in}{1.074808in}}%
\pgfpathlineto{\pgfqpoint{4.509544in}{4.987653in}}%
\pgfpathlineto{\pgfqpoint{5.775562in}{4.987653in}}%
\pgfpathlineto{\pgfqpoint{7.041581in}{1.074808in}}%
\pgfpathlineto{\pgfqpoint{8.307600in}{1.074808in}}%
\pgfpathlineto{\pgfqpoint{9.573619in}{1.074840in}}%
\pgfpathlineto{\pgfqpoint{10.839638in}{1.079258in}}%
\pgfpathlineto{\pgfqpoint{12.105657in}{4.987653in}}%
\pgfusepath{stroke}%
\end{pgfscope}%
\begin{pgfscope}%
\pgfpathrectangle{\pgfqpoint{0.100000in}{0.879166in}}{\pgfqpoint{12.617144in}{4.304129in}}%
\pgfusepath{clip}%
\pgfsetrectcap%
\pgfsetroundjoin%
\pgfsetlinewidth{5.018750pt}%
\definecolor{currentstroke}{rgb}{0.172549,0.627451,0.172549}%
\pgfsetstrokecolor{currentstroke}%
\pgfsetdash{}{0pt}%
\pgfpathmoveto{\pgfqpoint{0.711487in}{1.244404in}}%
\pgfpathlineto{\pgfqpoint{1.977506in}{1.081758in}}%
\pgfpathlineto{\pgfqpoint{3.243525in}{2.407373in}}%
\pgfpathlineto{\pgfqpoint{4.509544in}{1.155948in}}%
\pgfpathlineto{\pgfqpoint{5.775562in}{1.879558in}}%
\pgfpathlineto{\pgfqpoint{7.041581in}{1.185271in}}%
\pgfpathlineto{\pgfqpoint{8.307600in}{4.203035in}}%
\pgfpathlineto{\pgfqpoint{9.573619in}{4.987653in}}%
\pgfpathlineto{\pgfqpoint{10.839638in}{1.431251in}}%
\pgfpathlineto{\pgfqpoint{12.105657in}{1.075156in}}%
\pgfusepath{stroke}%
\end{pgfscope}%
\begin{pgfscope}%
\pgfpathrectangle{\pgfqpoint{0.100000in}{0.879166in}}{\pgfqpoint{12.617144in}{4.304129in}}%
\pgfusepath{clip}%
\pgfsetrectcap%
\pgfsetroundjoin%
\pgfsetlinewidth{5.018750pt}%
\definecolor{currentstroke}{rgb}{0.839216,0.152941,0.156863}%
\pgfsetstrokecolor{currentstroke}%
\pgfsetdash{}{0pt}%
\pgfpathmoveto{\pgfqpoint{0.711487in}{2.609826in}}%
\pgfpathlineto{\pgfqpoint{1.977506in}{1.075025in}}%
\pgfpathlineto{\pgfqpoint{3.243525in}{4.987653in}}%
\pgfpathlineto{\pgfqpoint{4.509544in}{1.167228in}}%
\pgfpathlineto{\pgfqpoint{5.775562in}{1.741186in}}%
\pgfpathlineto{\pgfqpoint{7.041581in}{1.731740in}}%
\pgfpathlineto{\pgfqpoint{8.307600in}{3.648885in}}%
\pgfpathlineto{\pgfqpoint{9.573619in}{1.480805in}}%
\pgfpathlineto{\pgfqpoint{10.839638in}{2.725090in}}%
\pgfpathlineto{\pgfqpoint{12.105657in}{1.084182in}}%
\pgfusepath{stroke}%
\end{pgfscope}%
\begin{pgfscope}%
\pgfpathrectangle{\pgfqpoint{0.100000in}{0.879166in}}{\pgfqpoint{12.617144in}{4.304129in}}%
\pgfusepath{clip}%
\pgfsetrectcap%
\pgfsetroundjoin%
\pgfsetlinewidth{5.018750pt}%
\definecolor{currentstroke}{rgb}{1.000000,0.498039,0.054902}%
\pgfsetstrokecolor{currentstroke}%
\pgfsetdash{}{0pt}%
\pgfpathmoveto{\pgfqpoint{0.711487in}{1.171624in}}%
\pgfpathlineto{\pgfqpoint{1.977506in}{1.094326in}}%
\pgfpathlineto{\pgfqpoint{3.243525in}{1.856248in}}%
\pgfpathlineto{\pgfqpoint{4.509544in}{1.074808in}}%
\pgfpathlineto{\pgfqpoint{5.775562in}{1.411527in}}%
\pgfpathlineto{\pgfqpoint{7.041581in}{4.987653in}}%
\pgfpathlineto{\pgfqpoint{8.307600in}{4.273440in}}%
\pgfpathlineto{\pgfqpoint{9.573619in}{1.074808in}}%
\pgfpathlineto{\pgfqpoint{10.839638in}{1.074808in}}%
\pgfpathlineto{\pgfqpoint{12.105657in}{1.074808in}}%
\pgfusepath{stroke}%
\end{pgfscope}%
\begin{pgfscope}%
\pgfpathrectangle{\pgfqpoint{0.100000in}{0.879166in}}{\pgfqpoint{12.617144in}{4.304129in}}%
\pgfusepath{clip}%
\pgfsetrectcap%
\pgfsetroundjoin%
\pgfsetlinewidth{5.018750pt}%
\definecolor{currentstroke}{rgb}{0.121569,0.466667,0.705882}%
\pgfsetstrokecolor{currentstroke}%
\pgfsetdash{}{0pt}%
\pgfpathmoveto{\pgfqpoint{0.711487in}{4.987653in}}%
\pgfpathlineto{\pgfqpoint{1.977506in}{1.074808in}}%
\pgfpathlineto{\pgfqpoint{3.243525in}{1.074808in}}%
\pgfpathlineto{\pgfqpoint{4.509544in}{4.987653in}}%
\pgfpathlineto{\pgfqpoint{5.775562in}{4.987653in}}%
\pgfpathlineto{\pgfqpoint{7.041581in}{1.074808in}}%
\pgfpathlineto{\pgfqpoint{8.307600in}{1.074808in}}%
\pgfpathlineto{\pgfqpoint{9.573619in}{1.074840in}}%
\pgfpathlineto{\pgfqpoint{10.839638in}{1.079258in}}%
\pgfpathlineto{\pgfqpoint{12.105657in}{4.987653in}}%
\pgfusepath{stroke}%
\end{pgfscope}%
\begin{pgfscope}%
\pgfpathrectangle{\pgfqpoint{0.100000in}{0.879166in}}{\pgfqpoint{12.617144in}{4.304129in}}%
\pgfusepath{clip}%
\pgfsetrectcap%
\pgfsetroundjoin%
\pgfsetlinewidth{2.007500pt}%
\definecolor{currentstroke}{rgb}{0.501961,0.501961,0.501961}%
\pgfsetstrokecolor{currentstroke}%
\pgfsetstrokeopacity{0.500000}%
\pgfsetdash{}{0pt}%
\pgfpathmoveto{\pgfqpoint{0.711487in}{0.879166in}}%
\pgfpathlineto{\pgfqpoint{0.711487in}{5.183295in}}%
\pgfusepath{stroke}%
\end{pgfscope}%
\begin{pgfscope}%
\pgfpathrectangle{\pgfqpoint{0.100000in}{0.879166in}}{\pgfqpoint{12.617144in}{4.304129in}}%
\pgfusepath{clip}%
\pgfsetbuttcap%
\pgfsetroundjoin%
\pgfsetlinewidth{2.007500pt}%
\definecolor{currentstroke}{rgb}{0.501961,0.501961,0.501961}%
\pgfsetstrokecolor{currentstroke}%
\pgfsetstrokeopacity{0.500000}%
\pgfsetdash{}{0pt}%
\pgfpathmoveto{\pgfqpoint{0.673507in}{1.074808in}}%
\pgfpathlineto{\pgfqpoint{0.749468in}{1.074808in}}%
\pgfusepath{stroke}%
\end{pgfscope}%
\begin{pgfscope}%
\pgfpathrectangle{\pgfqpoint{0.100000in}{0.879166in}}{\pgfqpoint{12.617144in}{4.304129in}}%
\pgfusepath{clip}%
\pgfsetbuttcap%
\pgfsetroundjoin%
\pgfsetlinewidth{2.007500pt}%
\definecolor{currentstroke}{rgb}{0.501961,0.501961,0.501961}%
\pgfsetstrokecolor{currentstroke}%
\pgfsetstrokeopacity{0.500000}%
\pgfsetdash{}{0pt}%
\pgfpathmoveto{\pgfqpoint{0.673507in}{1.509569in}}%
\pgfpathlineto{\pgfqpoint{0.749468in}{1.509569in}}%
\pgfusepath{stroke}%
\end{pgfscope}%
\begin{pgfscope}%
\pgfpathrectangle{\pgfqpoint{0.100000in}{0.879166in}}{\pgfqpoint{12.617144in}{4.304129in}}%
\pgfusepath{clip}%
\pgfsetbuttcap%
\pgfsetroundjoin%
\pgfsetlinewidth{2.007500pt}%
\definecolor{currentstroke}{rgb}{0.501961,0.501961,0.501961}%
\pgfsetstrokecolor{currentstroke}%
\pgfsetstrokeopacity{0.500000}%
\pgfsetdash{}{0pt}%
\pgfpathmoveto{\pgfqpoint{0.673507in}{1.944329in}}%
\pgfpathlineto{\pgfqpoint{0.749468in}{1.944329in}}%
\pgfusepath{stroke}%
\end{pgfscope}%
\begin{pgfscope}%
\pgfpathrectangle{\pgfqpoint{0.100000in}{0.879166in}}{\pgfqpoint{12.617144in}{4.304129in}}%
\pgfusepath{clip}%
\pgfsetbuttcap%
\pgfsetroundjoin%
\pgfsetlinewidth{2.007500pt}%
\definecolor{currentstroke}{rgb}{0.501961,0.501961,0.501961}%
\pgfsetstrokecolor{currentstroke}%
\pgfsetstrokeopacity{0.500000}%
\pgfsetdash{}{0pt}%
\pgfpathmoveto{\pgfqpoint{0.673507in}{2.379090in}}%
\pgfpathlineto{\pgfqpoint{0.749468in}{2.379090in}}%
\pgfusepath{stroke}%
\end{pgfscope}%
\begin{pgfscope}%
\pgfpathrectangle{\pgfqpoint{0.100000in}{0.879166in}}{\pgfqpoint{12.617144in}{4.304129in}}%
\pgfusepath{clip}%
\pgfsetbuttcap%
\pgfsetroundjoin%
\pgfsetlinewidth{2.007500pt}%
\definecolor{currentstroke}{rgb}{0.501961,0.501961,0.501961}%
\pgfsetstrokecolor{currentstroke}%
\pgfsetstrokeopacity{0.500000}%
\pgfsetdash{}{0pt}%
\pgfpathmoveto{\pgfqpoint{0.673507in}{2.813850in}}%
\pgfpathlineto{\pgfqpoint{0.749468in}{2.813850in}}%
\pgfusepath{stroke}%
\end{pgfscope}%
\begin{pgfscope}%
\pgfpathrectangle{\pgfqpoint{0.100000in}{0.879166in}}{\pgfqpoint{12.617144in}{4.304129in}}%
\pgfusepath{clip}%
\pgfsetbuttcap%
\pgfsetroundjoin%
\pgfsetlinewidth{2.007500pt}%
\definecolor{currentstroke}{rgb}{0.501961,0.501961,0.501961}%
\pgfsetstrokecolor{currentstroke}%
\pgfsetstrokeopacity{0.500000}%
\pgfsetdash{}{0pt}%
\pgfpathmoveto{\pgfqpoint{0.673507in}{3.248611in}}%
\pgfpathlineto{\pgfqpoint{0.749468in}{3.248611in}}%
\pgfusepath{stroke}%
\end{pgfscope}%
\begin{pgfscope}%
\pgfpathrectangle{\pgfqpoint{0.100000in}{0.879166in}}{\pgfqpoint{12.617144in}{4.304129in}}%
\pgfusepath{clip}%
\pgfsetbuttcap%
\pgfsetroundjoin%
\pgfsetlinewidth{2.007500pt}%
\definecolor{currentstroke}{rgb}{0.501961,0.501961,0.501961}%
\pgfsetstrokecolor{currentstroke}%
\pgfsetstrokeopacity{0.500000}%
\pgfsetdash{}{0pt}%
\pgfpathmoveto{\pgfqpoint{0.673507in}{3.683371in}}%
\pgfpathlineto{\pgfqpoint{0.749468in}{3.683371in}}%
\pgfusepath{stroke}%
\end{pgfscope}%
\begin{pgfscope}%
\pgfpathrectangle{\pgfqpoint{0.100000in}{0.879166in}}{\pgfqpoint{12.617144in}{4.304129in}}%
\pgfusepath{clip}%
\pgfsetbuttcap%
\pgfsetroundjoin%
\pgfsetlinewidth{2.007500pt}%
\definecolor{currentstroke}{rgb}{0.501961,0.501961,0.501961}%
\pgfsetstrokecolor{currentstroke}%
\pgfsetstrokeopacity{0.500000}%
\pgfsetdash{}{0pt}%
\pgfpathmoveto{\pgfqpoint{0.673507in}{4.118132in}}%
\pgfpathlineto{\pgfqpoint{0.749468in}{4.118132in}}%
\pgfusepath{stroke}%
\end{pgfscope}%
\begin{pgfscope}%
\pgfpathrectangle{\pgfqpoint{0.100000in}{0.879166in}}{\pgfqpoint{12.617144in}{4.304129in}}%
\pgfusepath{clip}%
\pgfsetbuttcap%
\pgfsetroundjoin%
\pgfsetlinewidth{2.007500pt}%
\definecolor{currentstroke}{rgb}{0.501961,0.501961,0.501961}%
\pgfsetstrokecolor{currentstroke}%
\pgfsetstrokeopacity{0.500000}%
\pgfsetdash{}{0pt}%
\pgfpathmoveto{\pgfqpoint{0.673507in}{4.552892in}}%
\pgfpathlineto{\pgfqpoint{0.749468in}{4.552892in}}%
\pgfusepath{stroke}%
\end{pgfscope}%
\begin{pgfscope}%
\pgfpathrectangle{\pgfqpoint{0.100000in}{0.879166in}}{\pgfqpoint{12.617144in}{4.304129in}}%
\pgfusepath{clip}%
\pgfsetbuttcap%
\pgfsetroundjoin%
\pgfsetlinewidth{2.007500pt}%
\definecolor{currentstroke}{rgb}{0.501961,0.501961,0.501961}%
\pgfsetstrokecolor{currentstroke}%
\pgfsetstrokeopacity{0.500000}%
\pgfsetdash{}{0pt}%
\pgfpathmoveto{\pgfqpoint{0.673507in}{4.987653in}}%
\pgfpathlineto{\pgfqpoint{0.749468in}{4.987653in}}%
\pgfusepath{stroke}%
\end{pgfscope}%
\begin{pgfscope}%
\pgfpathrectangle{\pgfqpoint{0.100000in}{0.879166in}}{\pgfqpoint{12.617144in}{4.304129in}}%
\pgfusepath{clip}%
\pgfsetrectcap%
\pgfsetroundjoin%
\pgfsetlinewidth{2.007500pt}%
\definecolor{currentstroke}{rgb}{0.501961,0.501961,0.501961}%
\pgfsetstrokecolor{currentstroke}%
\pgfsetstrokeopacity{0.500000}%
\pgfsetdash{}{0pt}%
\pgfpathmoveto{\pgfqpoint{1.977506in}{0.879166in}}%
\pgfpathlineto{\pgfqpoint{1.977506in}{5.183295in}}%
\pgfusepath{stroke}%
\end{pgfscope}%
\begin{pgfscope}%
\pgfpathrectangle{\pgfqpoint{0.100000in}{0.879166in}}{\pgfqpoint{12.617144in}{4.304129in}}%
\pgfusepath{clip}%
\pgfsetbuttcap%
\pgfsetroundjoin%
\pgfsetlinewidth{2.007500pt}%
\definecolor{currentstroke}{rgb}{0.501961,0.501961,0.501961}%
\pgfsetstrokecolor{currentstroke}%
\pgfsetstrokeopacity{0.500000}%
\pgfsetdash{}{0pt}%
\pgfpathmoveto{\pgfqpoint{1.939525in}{1.074808in}}%
\pgfpathlineto{\pgfqpoint{2.015487in}{1.074808in}}%
\pgfusepath{stroke}%
\end{pgfscope}%
\begin{pgfscope}%
\pgfpathrectangle{\pgfqpoint{0.100000in}{0.879166in}}{\pgfqpoint{12.617144in}{4.304129in}}%
\pgfusepath{clip}%
\pgfsetbuttcap%
\pgfsetroundjoin%
\pgfsetlinewidth{2.007500pt}%
\definecolor{currentstroke}{rgb}{0.501961,0.501961,0.501961}%
\pgfsetstrokecolor{currentstroke}%
\pgfsetstrokeopacity{0.500000}%
\pgfsetdash{}{0pt}%
\pgfpathmoveto{\pgfqpoint{1.939525in}{1.509569in}}%
\pgfpathlineto{\pgfqpoint{2.015487in}{1.509569in}}%
\pgfusepath{stroke}%
\end{pgfscope}%
\begin{pgfscope}%
\pgfpathrectangle{\pgfqpoint{0.100000in}{0.879166in}}{\pgfqpoint{12.617144in}{4.304129in}}%
\pgfusepath{clip}%
\pgfsetbuttcap%
\pgfsetroundjoin%
\pgfsetlinewidth{2.007500pt}%
\definecolor{currentstroke}{rgb}{0.501961,0.501961,0.501961}%
\pgfsetstrokecolor{currentstroke}%
\pgfsetstrokeopacity{0.500000}%
\pgfsetdash{}{0pt}%
\pgfpathmoveto{\pgfqpoint{1.939525in}{1.944329in}}%
\pgfpathlineto{\pgfqpoint{2.015487in}{1.944329in}}%
\pgfusepath{stroke}%
\end{pgfscope}%
\begin{pgfscope}%
\pgfpathrectangle{\pgfqpoint{0.100000in}{0.879166in}}{\pgfqpoint{12.617144in}{4.304129in}}%
\pgfusepath{clip}%
\pgfsetbuttcap%
\pgfsetroundjoin%
\pgfsetlinewidth{2.007500pt}%
\definecolor{currentstroke}{rgb}{0.501961,0.501961,0.501961}%
\pgfsetstrokecolor{currentstroke}%
\pgfsetstrokeopacity{0.500000}%
\pgfsetdash{}{0pt}%
\pgfpathmoveto{\pgfqpoint{1.939525in}{2.379090in}}%
\pgfpathlineto{\pgfqpoint{2.015487in}{2.379090in}}%
\pgfusepath{stroke}%
\end{pgfscope}%
\begin{pgfscope}%
\pgfpathrectangle{\pgfqpoint{0.100000in}{0.879166in}}{\pgfqpoint{12.617144in}{4.304129in}}%
\pgfusepath{clip}%
\pgfsetbuttcap%
\pgfsetroundjoin%
\pgfsetlinewidth{2.007500pt}%
\definecolor{currentstroke}{rgb}{0.501961,0.501961,0.501961}%
\pgfsetstrokecolor{currentstroke}%
\pgfsetstrokeopacity{0.500000}%
\pgfsetdash{}{0pt}%
\pgfpathmoveto{\pgfqpoint{1.939525in}{2.813850in}}%
\pgfpathlineto{\pgfqpoint{2.015487in}{2.813850in}}%
\pgfusepath{stroke}%
\end{pgfscope}%
\begin{pgfscope}%
\pgfpathrectangle{\pgfqpoint{0.100000in}{0.879166in}}{\pgfqpoint{12.617144in}{4.304129in}}%
\pgfusepath{clip}%
\pgfsetbuttcap%
\pgfsetroundjoin%
\pgfsetlinewidth{2.007500pt}%
\definecolor{currentstroke}{rgb}{0.501961,0.501961,0.501961}%
\pgfsetstrokecolor{currentstroke}%
\pgfsetstrokeopacity{0.500000}%
\pgfsetdash{}{0pt}%
\pgfpathmoveto{\pgfqpoint{1.939525in}{3.248611in}}%
\pgfpathlineto{\pgfqpoint{2.015487in}{3.248611in}}%
\pgfusepath{stroke}%
\end{pgfscope}%
\begin{pgfscope}%
\pgfpathrectangle{\pgfqpoint{0.100000in}{0.879166in}}{\pgfqpoint{12.617144in}{4.304129in}}%
\pgfusepath{clip}%
\pgfsetbuttcap%
\pgfsetroundjoin%
\pgfsetlinewidth{2.007500pt}%
\definecolor{currentstroke}{rgb}{0.501961,0.501961,0.501961}%
\pgfsetstrokecolor{currentstroke}%
\pgfsetstrokeopacity{0.500000}%
\pgfsetdash{}{0pt}%
\pgfpathmoveto{\pgfqpoint{1.939525in}{3.683371in}}%
\pgfpathlineto{\pgfqpoint{2.015487in}{3.683371in}}%
\pgfusepath{stroke}%
\end{pgfscope}%
\begin{pgfscope}%
\pgfpathrectangle{\pgfqpoint{0.100000in}{0.879166in}}{\pgfqpoint{12.617144in}{4.304129in}}%
\pgfusepath{clip}%
\pgfsetbuttcap%
\pgfsetroundjoin%
\pgfsetlinewidth{2.007500pt}%
\definecolor{currentstroke}{rgb}{0.501961,0.501961,0.501961}%
\pgfsetstrokecolor{currentstroke}%
\pgfsetstrokeopacity{0.500000}%
\pgfsetdash{}{0pt}%
\pgfpathmoveto{\pgfqpoint{1.939525in}{4.118132in}}%
\pgfpathlineto{\pgfqpoint{2.015487in}{4.118132in}}%
\pgfusepath{stroke}%
\end{pgfscope}%
\begin{pgfscope}%
\pgfpathrectangle{\pgfqpoint{0.100000in}{0.879166in}}{\pgfqpoint{12.617144in}{4.304129in}}%
\pgfusepath{clip}%
\pgfsetbuttcap%
\pgfsetroundjoin%
\pgfsetlinewidth{2.007500pt}%
\definecolor{currentstroke}{rgb}{0.501961,0.501961,0.501961}%
\pgfsetstrokecolor{currentstroke}%
\pgfsetstrokeopacity{0.500000}%
\pgfsetdash{}{0pt}%
\pgfpathmoveto{\pgfqpoint{1.939525in}{4.552892in}}%
\pgfpathlineto{\pgfqpoint{2.015487in}{4.552892in}}%
\pgfusepath{stroke}%
\end{pgfscope}%
\begin{pgfscope}%
\pgfpathrectangle{\pgfqpoint{0.100000in}{0.879166in}}{\pgfqpoint{12.617144in}{4.304129in}}%
\pgfusepath{clip}%
\pgfsetbuttcap%
\pgfsetroundjoin%
\pgfsetlinewidth{2.007500pt}%
\definecolor{currentstroke}{rgb}{0.501961,0.501961,0.501961}%
\pgfsetstrokecolor{currentstroke}%
\pgfsetstrokeopacity{0.500000}%
\pgfsetdash{}{0pt}%
\pgfpathmoveto{\pgfqpoint{1.939525in}{4.987653in}}%
\pgfpathlineto{\pgfqpoint{2.015487in}{4.987653in}}%
\pgfusepath{stroke}%
\end{pgfscope}%
\begin{pgfscope}%
\pgfpathrectangle{\pgfqpoint{0.100000in}{0.879166in}}{\pgfqpoint{12.617144in}{4.304129in}}%
\pgfusepath{clip}%
\pgfsetrectcap%
\pgfsetroundjoin%
\pgfsetlinewidth{2.007500pt}%
\definecolor{currentstroke}{rgb}{0.501961,0.501961,0.501961}%
\pgfsetstrokecolor{currentstroke}%
\pgfsetstrokeopacity{0.500000}%
\pgfsetdash{}{0pt}%
\pgfpathmoveto{\pgfqpoint{3.243525in}{0.879166in}}%
\pgfpathlineto{\pgfqpoint{3.243525in}{5.183295in}}%
\pgfusepath{stroke}%
\end{pgfscope}%
\begin{pgfscope}%
\pgfpathrectangle{\pgfqpoint{0.100000in}{0.879166in}}{\pgfqpoint{12.617144in}{4.304129in}}%
\pgfusepath{clip}%
\pgfsetbuttcap%
\pgfsetroundjoin%
\pgfsetlinewidth{2.007500pt}%
\definecolor{currentstroke}{rgb}{0.501961,0.501961,0.501961}%
\pgfsetstrokecolor{currentstroke}%
\pgfsetstrokeopacity{0.500000}%
\pgfsetdash{}{0pt}%
\pgfpathmoveto{\pgfqpoint{3.205544in}{1.074808in}}%
\pgfpathlineto{\pgfqpoint{3.281505in}{1.074808in}}%
\pgfusepath{stroke}%
\end{pgfscope}%
\begin{pgfscope}%
\pgfpathrectangle{\pgfqpoint{0.100000in}{0.879166in}}{\pgfqpoint{12.617144in}{4.304129in}}%
\pgfusepath{clip}%
\pgfsetbuttcap%
\pgfsetroundjoin%
\pgfsetlinewidth{2.007500pt}%
\definecolor{currentstroke}{rgb}{0.501961,0.501961,0.501961}%
\pgfsetstrokecolor{currentstroke}%
\pgfsetstrokeopacity{0.500000}%
\pgfsetdash{}{0pt}%
\pgfpathmoveto{\pgfqpoint{3.205544in}{1.509569in}}%
\pgfpathlineto{\pgfqpoint{3.281505in}{1.509569in}}%
\pgfusepath{stroke}%
\end{pgfscope}%
\begin{pgfscope}%
\pgfpathrectangle{\pgfqpoint{0.100000in}{0.879166in}}{\pgfqpoint{12.617144in}{4.304129in}}%
\pgfusepath{clip}%
\pgfsetbuttcap%
\pgfsetroundjoin%
\pgfsetlinewidth{2.007500pt}%
\definecolor{currentstroke}{rgb}{0.501961,0.501961,0.501961}%
\pgfsetstrokecolor{currentstroke}%
\pgfsetstrokeopacity{0.500000}%
\pgfsetdash{}{0pt}%
\pgfpathmoveto{\pgfqpoint{3.205544in}{1.944329in}}%
\pgfpathlineto{\pgfqpoint{3.281505in}{1.944329in}}%
\pgfusepath{stroke}%
\end{pgfscope}%
\begin{pgfscope}%
\pgfpathrectangle{\pgfqpoint{0.100000in}{0.879166in}}{\pgfqpoint{12.617144in}{4.304129in}}%
\pgfusepath{clip}%
\pgfsetbuttcap%
\pgfsetroundjoin%
\pgfsetlinewidth{2.007500pt}%
\definecolor{currentstroke}{rgb}{0.501961,0.501961,0.501961}%
\pgfsetstrokecolor{currentstroke}%
\pgfsetstrokeopacity{0.500000}%
\pgfsetdash{}{0pt}%
\pgfpathmoveto{\pgfqpoint{3.205544in}{2.379090in}}%
\pgfpathlineto{\pgfqpoint{3.281505in}{2.379090in}}%
\pgfusepath{stroke}%
\end{pgfscope}%
\begin{pgfscope}%
\pgfpathrectangle{\pgfqpoint{0.100000in}{0.879166in}}{\pgfqpoint{12.617144in}{4.304129in}}%
\pgfusepath{clip}%
\pgfsetbuttcap%
\pgfsetroundjoin%
\pgfsetlinewidth{2.007500pt}%
\definecolor{currentstroke}{rgb}{0.501961,0.501961,0.501961}%
\pgfsetstrokecolor{currentstroke}%
\pgfsetstrokeopacity{0.500000}%
\pgfsetdash{}{0pt}%
\pgfpathmoveto{\pgfqpoint{3.205544in}{2.813850in}}%
\pgfpathlineto{\pgfqpoint{3.281505in}{2.813850in}}%
\pgfusepath{stroke}%
\end{pgfscope}%
\begin{pgfscope}%
\pgfpathrectangle{\pgfqpoint{0.100000in}{0.879166in}}{\pgfqpoint{12.617144in}{4.304129in}}%
\pgfusepath{clip}%
\pgfsetbuttcap%
\pgfsetroundjoin%
\pgfsetlinewidth{2.007500pt}%
\definecolor{currentstroke}{rgb}{0.501961,0.501961,0.501961}%
\pgfsetstrokecolor{currentstroke}%
\pgfsetstrokeopacity{0.500000}%
\pgfsetdash{}{0pt}%
\pgfpathmoveto{\pgfqpoint{3.205544in}{3.248611in}}%
\pgfpathlineto{\pgfqpoint{3.281505in}{3.248611in}}%
\pgfusepath{stroke}%
\end{pgfscope}%
\begin{pgfscope}%
\pgfpathrectangle{\pgfqpoint{0.100000in}{0.879166in}}{\pgfqpoint{12.617144in}{4.304129in}}%
\pgfusepath{clip}%
\pgfsetbuttcap%
\pgfsetroundjoin%
\pgfsetlinewidth{2.007500pt}%
\definecolor{currentstroke}{rgb}{0.501961,0.501961,0.501961}%
\pgfsetstrokecolor{currentstroke}%
\pgfsetstrokeopacity{0.500000}%
\pgfsetdash{}{0pt}%
\pgfpathmoveto{\pgfqpoint{3.205544in}{3.683371in}}%
\pgfpathlineto{\pgfqpoint{3.281505in}{3.683371in}}%
\pgfusepath{stroke}%
\end{pgfscope}%
\begin{pgfscope}%
\pgfpathrectangle{\pgfqpoint{0.100000in}{0.879166in}}{\pgfqpoint{12.617144in}{4.304129in}}%
\pgfusepath{clip}%
\pgfsetbuttcap%
\pgfsetroundjoin%
\pgfsetlinewidth{2.007500pt}%
\definecolor{currentstroke}{rgb}{0.501961,0.501961,0.501961}%
\pgfsetstrokecolor{currentstroke}%
\pgfsetstrokeopacity{0.500000}%
\pgfsetdash{}{0pt}%
\pgfpathmoveto{\pgfqpoint{3.205544in}{4.118132in}}%
\pgfpathlineto{\pgfqpoint{3.281505in}{4.118132in}}%
\pgfusepath{stroke}%
\end{pgfscope}%
\begin{pgfscope}%
\pgfpathrectangle{\pgfqpoint{0.100000in}{0.879166in}}{\pgfqpoint{12.617144in}{4.304129in}}%
\pgfusepath{clip}%
\pgfsetbuttcap%
\pgfsetroundjoin%
\pgfsetlinewidth{2.007500pt}%
\definecolor{currentstroke}{rgb}{0.501961,0.501961,0.501961}%
\pgfsetstrokecolor{currentstroke}%
\pgfsetstrokeopacity{0.500000}%
\pgfsetdash{}{0pt}%
\pgfpathmoveto{\pgfqpoint{3.205544in}{4.552892in}}%
\pgfpathlineto{\pgfqpoint{3.281505in}{4.552892in}}%
\pgfusepath{stroke}%
\end{pgfscope}%
\begin{pgfscope}%
\pgfpathrectangle{\pgfqpoint{0.100000in}{0.879166in}}{\pgfqpoint{12.617144in}{4.304129in}}%
\pgfusepath{clip}%
\pgfsetbuttcap%
\pgfsetroundjoin%
\pgfsetlinewidth{2.007500pt}%
\definecolor{currentstroke}{rgb}{0.501961,0.501961,0.501961}%
\pgfsetstrokecolor{currentstroke}%
\pgfsetstrokeopacity{0.500000}%
\pgfsetdash{}{0pt}%
\pgfpathmoveto{\pgfqpoint{3.205544in}{4.987653in}}%
\pgfpathlineto{\pgfqpoint{3.281505in}{4.987653in}}%
\pgfusepath{stroke}%
\end{pgfscope}%
\begin{pgfscope}%
\pgfpathrectangle{\pgfqpoint{0.100000in}{0.879166in}}{\pgfqpoint{12.617144in}{4.304129in}}%
\pgfusepath{clip}%
\pgfsetrectcap%
\pgfsetroundjoin%
\pgfsetlinewidth{2.007500pt}%
\definecolor{currentstroke}{rgb}{0.501961,0.501961,0.501961}%
\pgfsetstrokecolor{currentstroke}%
\pgfsetstrokeopacity{0.500000}%
\pgfsetdash{}{0pt}%
\pgfpathmoveto{\pgfqpoint{4.509544in}{0.879166in}}%
\pgfpathlineto{\pgfqpoint{4.509544in}{5.183295in}}%
\pgfusepath{stroke}%
\end{pgfscope}%
\begin{pgfscope}%
\pgfpathrectangle{\pgfqpoint{0.100000in}{0.879166in}}{\pgfqpoint{12.617144in}{4.304129in}}%
\pgfusepath{clip}%
\pgfsetbuttcap%
\pgfsetroundjoin%
\pgfsetlinewidth{2.007500pt}%
\definecolor{currentstroke}{rgb}{0.501961,0.501961,0.501961}%
\pgfsetstrokecolor{currentstroke}%
\pgfsetstrokeopacity{0.500000}%
\pgfsetdash{}{0pt}%
\pgfpathmoveto{\pgfqpoint{4.471563in}{1.074808in}}%
\pgfpathlineto{\pgfqpoint{4.547524in}{1.074808in}}%
\pgfusepath{stroke}%
\end{pgfscope}%
\begin{pgfscope}%
\pgfpathrectangle{\pgfqpoint{0.100000in}{0.879166in}}{\pgfqpoint{12.617144in}{4.304129in}}%
\pgfusepath{clip}%
\pgfsetbuttcap%
\pgfsetroundjoin%
\pgfsetlinewidth{2.007500pt}%
\definecolor{currentstroke}{rgb}{0.501961,0.501961,0.501961}%
\pgfsetstrokecolor{currentstroke}%
\pgfsetstrokeopacity{0.500000}%
\pgfsetdash{}{0pt}%
\pgfpathmoveto{\pgfqpoint{4.471563in}{1.509569in}}%
\pgfpathlineto{\pgfqpoint{4.547524in}{1.509569in}}%
\pgfusepath{stroke}%
\end{pgfscope}%
\begin{pgfscope}%
\pgfpathrectangle{\pgfqpoint{0.100000in}{0.879166in}}{\pgfqpoint{12.617144in}{4.304129in}}%
\pgfusepath{clip}%
\pgfsetbuttcap%
\pgfsetroundjoin%
\pgfsetlinewidth{2.007500pt}%
\definecolor{currentstroke}{rgb}{0.501961,0.501961,0.501961}%
\pgfsetstrokecolor{currentstroke}%
\pgfsetstrokeopacity{0.500000}%
\pgfsetdash{}{0pt}%
\pgfpathmoveto{\pgfqpoint{4.471563in}{1.944329in}}%
\pgfpathlineto{\pgfqpoint{4.547524in}{1.944329in}}%
\pgfusepath{stroke}%
\end{pgfscope}%
\begin{pgfscope}%
\pgfpathrectangle{\pgfqpoint{0.100000in}{0.879166in}}{\pgfqpoint{12.617144in}{4.304129in}}%
\pgfusepath{clip}%
\pgfsetbuttcap%
\pgfsetroundjoin%
\pgfsetlinewidth{2.007500pt}%
\definecolor{currentstroke}{rgb}{0.501961,0.501961,0.501961}%
\pgfsetstrokecolor{currentstroke}%
\pgfsetstrokeopacity{0.500000}%
\pgfsetdash{}{0pt}%
\pgfpathmoveto{\pgfqpoint{4.471563in}{2.379090in}}%
\pgfpathlineto{\pgfqpoint{4.547524in}{2.379090in}}%
\pgfusepath{stroke}%
\end{pgfscope}%
\begin{pgfscope}%
\pgfpathrectangle{\pgfqpoint{0.100000in}{0.879166in}}{\pgfqpoint{12.617144in}{4.304129in}}%
\pgfusepath{clip}%
\pgfsetbuttcap%
\pgfsetroundjoin%
\pgfsetlinewidth{2.007500pt}%
\definecolor{currentstroke}{rgb}{0.501961,0.501961,0.501961}%
\pgfsetstrokecolor{currentstroke}%
\pgfsetstrokeopacity{0.500000}%
\pgfsetdash{}{0pt}%
\pgfpathmoveto{\pgfqpoint{4.471563in}{2.813850in}}%
\pgfpathlineto{\pgfqpoint{4.547524in}{2.813850in}}%
\pgfusepath{stroke}%
\end{pgfscope}%
\begin{pgfscope}%
\pgfpathrectangle{\pgfqpoint{0.100000in}{0.879166in}}{\pgfqpoint{12.617144in}{4.304129in}}%
\pgfusepath{clip}%
\pgfsetbuttcap%
\pgfsetroundjoin%
\pgfsetlinewidth{2.007500pt}%
\definecolor{currentstroke}{rgb}{0.501961,0.501961,0.501961}%
\pgfsetstrokecolor{currentstroke}%
\pgfsetstrokeopacity{0.500000}%
\pgfsetdash{}{0pt}%
\pgfpathmoveto{\pgfqpoint{4.471563in}{3.248611in}}%
\pgfpathlineto{\pgfqpoint{4.547524in}{3.248611in}}%
\pgfusepath{stroke}%
\end{pgfscope}%
\begin{pgfscope}%
\pgfpathrectangle{\pgfqpoint{0.100000in}{0.879166in}}{\pgfqpoint{12.617144in}{4.304129in}}%
\pgfusepath{clip}%
\pgfsetbuttcap%
\pgfsetroundjoin%
\pgfsetlinewidth{2.007500pt}%
\definecolor{currentstroke}{rgb}{0.501961,0.501961,0.501961}%
\pgfsetstrokecolor{currentstroke}%
\pgfsetstrokeopacity{0.500000}%
\pgfsetdash{}{0pt}%
\pgfpathmoveto{\pgfqpoint{4.471563in}{3.683371in}}%
\pgfpathlineto{\pgfqpoint{4.547524in}{3.683371in}}%
\pgfusepath{stroke}%
\end{pgfscope}%
\begin{pgfscope}%
\pgfpathrectangle{\pgfqpoint{0.100000in}{0.879166in}}{\pgfqpoint{12.617144in}{4.304129in}}%
\pgfusepath{clip}%
\pgfsetbuttcap%
\pgfsetroundjoin%
\pgfsetlinewidth{2.007500pt}%
\definecolor{currentstroke}{rgb}{0.501961,0.501961,0.501961}%
\pgfsetstrokecolor{currentstroke}%
\pgfsetstrokeopacity{0.500000}%
\pgfsetdash{}{0pt}%
\pgfpathmoveto{\pgfqpoint{4.471563in}{4.118132in}}%
\pgfpathlineto{\pgfqpoint{4.547524in}{4.118132in}}%
\pgfusepath{stroke}%
\end{pgfscope}%
\begin{pgfscope}%
\pgfpathrectangle{\pgfqpoint{0.100000in}{0.879166in}}{\pgfqpoint{12.617144in}{4.304129in}}%
\pgfusepath{clip}%
\pgfsetbuttcap%
\pgfsetroundjoin%
\pgfsetlinewidth{2.007500pt}%
\definecolor{currentstroke}{rgb}{0.501961,0.501961,0.501961}%
\pgfsetstrokecolor{currentstroke}%
\pgfsetstrokeopacity{0.500000}%
\pgfsetdash{}{0pt}%
\pgfpathmoveto{\pgfqpoint{4.471563in}{4.552892in}}%
\pgfpathlineto{\pgfqpoint{4.547524in}{4.552892in}}%
\pgfusepath{stroke}%
\end{pgfscope}%
\begin{pgfscope}%
\pgfpathrectangle{\pgfqpoint{0.100000in}{0.879166in}}{\pgfqpoint{12.617144in}{4.304129in}}%
\pgfusepath{clip}%
\pgfsetbuttcap%
\pgfsetroundjoin%
\pgfsetlinewidth{2.007500pt}%
\definecolor{currentstroke}{rgb}{0.501961,0.501961,0.501961}%
\pgfsetstrokecolor{currentstroke}%
\pgfsetstrokeopacity{0.500000}%
\pgfsetdash{}{0pt}%
\pgfpathmoveto{\pgfqpoint{4.471563in}{4.987653in}}%
\pgfpathlineto{\pgfqpoint{4.547524in}{4.987653in}}%
\pgfusepath{stroke}%
\end{pgfscope}%
\begin{pgfscope}%
\pgfpathrectangle{\pgfqpoint{0.100000in}{0.879166in}}{\pgfqpoint{12.617144in}{4.304129in}}%
\pgfusepath{clip}%
\pgfsetrectcap%
\pgfsetroundjoin%
\pgfsetlinewidth{2.007500pt}%
\definecolor{currentstroke}{rgb}{0.501961,0.501961,0.501961}%
\pgfsetstrokecolor{currentstroke}%
\pgfsetstrokeopacity{0.500000}%
\pgfsetdash{}{0pt}%
\pgfpathmoveto{\pgfqpoint{5.775562in}{0.879166in}}%
\pgfpathlineto{\pgfqpoint{5.775562in}{5.183295in}}%
\pgfusepath{stroke}%
\end{pgfscope}%
\begin{pgfscope}%
\pgfpathrectangle{\pgfqpoint{0.100000in}{0.879166in}}{\pgfqpoint{12.617144in}{4.304129in}}%
\pgfusepath{clip}%
\pgfsetbuttcap%
\pgfsetroundjoin%
\pgfsetlinewidth{2.007500pt}%
\definecolor{currentstroke}{rgb}{0.501961,0.501961,0.501961}%
\pgfsetstrokecolor{currentstroke}%
\pgfsetstrokeopacity{0.500000}%
\pgfsetdash{}{0pt}%
\pgfpathmoveto{\pgfqpoint{5.737582in}{1.074808in}}%
\pgfpathlineto{\pgfqpoint{5.813543in}{1.074808in}}%
\pgfusepath{stroke}%
\end{pgfscope}%
\begin{pgfscope}%
\pgfpathrectangle{\pgfqpoint{0.100000in}{0.879166in}}{\pgfqpoint{12.617144in}{4.304129in}}%
\pgfusepath{clip}%
\pgfsetbuttcap%
\pgfsetroundjoin%
\pgfsetlinewidth{2.007500pt}%
\definecolor{currentstroke}{rgb}{0.501961,0.501961,0.501961}%
\pgfsetstrokecolor{currentstroke}%
\pgfsetstrokeopacity{0.500000}%
\pgfsetdash{}{0pt}%
\pgfpathmoveto{\pgfqpoint{5.737582in}{1.509569in}}%
\pgfpathlineto{\pgfqpoint{5.813543in}{1.509569in}}%
\pgfusepath{stroke}%
\end{pgfscope}%
\begin{pgfscope}%
\pgfpathrectangle{\pgfqpoint{0.100000in}{0.879166in}}{\pgfqpoint{12.617144in}{4.304129in}}%
\pgfusepath{clip}%
\pgfsetbuttcap%
\pgfsetroundjoin%
\pgfsetlinewidth{2.007500pt}%
\definecolor{currentstroke}{rgb}{0.501961,0.501961,0.501961}%
\pgfsetstrokecolor{currentstroke}%
\pgfsetstrokeopacity{0.500000}%
\pgfsetdash{}{0pt}%
\pgfpathmoveto{\pgfqpoint{5.737582in}{1.944329in}}%
\pgfpathlineto{\pgfqpoint{5.813543in}{1.944329in}}%
\pgfusepath{stroke}%
\end{pgfscope}%
\begin{pgfscope}%
\pgfpathrectangle{\pgfqpoint{0.100000in}{0.879166in}}{\pgfqpoint{12.617144in}{4.304129in}}%
\pgfusepath{clip}%
\pgfsetbuttcap%
\pgfsetroundjoin%
\pgfsetlinewidth{2.007500pt}%
\definecolor{currentstroke}{rgb}{0.501961,0.501961,0.501961}%
\pgfsetstrokecolor{currentstroke}%
\pgfsetstrokeopacity{0.500000}%
\pgfsetdash{}{0pt}%
\pgfpathmoveto{\pgfqpoint{5.737582in}{2.379090in}}%
\pgfpathlineto{\pgfqpoint{5.813543in}{2.379090in}}%
\pgfusepath{stroke}%
\end{pgfscope}%
\begin{pgfscope}%
\pgfpathrectangle{\pgfqpoint{0.100000in}{0.879166in}}{\pgfqpoint{12.617144in}{4.304129in}}%
\pgfusepath{clip}%
\pgfsetbuttcap%
\pgfsetroundjoin%
\pgfsetlinewidth{2.007500pt}%
\definecolor{currentstroke}{rgb}{0.501961,0.501961,0.501961}%
\pgfsetstrokecolor{currentstroke}%
\pgfsetstrokeopacity{0.500000}%
\pgfsetdash{}{0pt}%
\pgfpathmoveto{\pgfqpoint{5.737582in}{2.813850in}}%
\pgfpathlineto{\pgfqpoint{5.813543in}{2.813850in}}%
\pgfusepath{stroke}%
\end{pgfscope}%
\begin{pgfscope}%
\pgfpathrectangle{\pgfqpoint{0.100000in}{0.879166in}}{\pgfqpoint{12.617144in}{4.304129in}}%
\pgfusepath{clip}%
\pgfsetbuttcap%
\pgfsetroundjoin%
\pgfsetlinewidth{2.007500pt}%
\definecolor{currentstroke}{rgb}{0.501961,0.501961,0.501961}%
\pgfsetstrokecolor{currentstroke}%
\pgfsetstrokeopacity{0.500000}%
\pgfsetdash{}{0pt}%
\pgfpathmoveto{\pgfqpoint{5.737582in}{3.248611in}}%
\pgfpathlineto{\pgfqpoint{5.813543in}{3.248611in}}%
\pgfusepath{stroke}%
\end{pgfscope}%
\begin{pgfscope}%
\pgfpathrectangle{\pgfqpoint{0.100000in}{0.879166in}}{\pgfqpoint{12.617144in}{4.304129in}}%
\pgfusepath{clip}%
\pgfsetbuttcap%
\pgfsetroundjoin%
\pgfsetlinewidth{2.007500pt}%
\definecolor{currentstroke}{rgb}{0.501961,0.501961,0.501961}%
\pgfsetstrokecolor{currentstroke}%
\pgfsetstrokeopacity{0.500000}%
\pgfsetdash{}{0pt}%
\pgfpathmoveto{\pgfqpoint{5.737582in}{3.683371in}}%
\pgfpathlineto{\pgfqpoint{5.813543in}{3.683371in}}%
\pgfusepath{stroke}%
\end{pgfscope}%
\begin{pgfscope}%
\pgfpathrectangle{\pgfqpoint{0.100000in}{0.879166in}}{\pgfqpoint{12.617144in}{4.304129in}}%
\pgfusepath{clip}%
\pgfsetbuttcap%
\pgfsetroundjoin%
\pgfsetlinewidth{2.007500pt}%
\definecolor{currentstroke}{rgb}{0.501961,0.501961,0.501961}%
\pgfsetstrokecolor{currentstroke}%
\pgfsetstrokeopacity{0.500000}%
\pgfsetdash{}{0pt}%
\pgfpathmoveto{\pgfqpoint{5.737582in}{4.118132in}}%
\pgfpathlineto{\pgfqpoint{5.813543in}{4.118132in}}%
\pgfusepath{stroke}%
\end{pgfscope}%
\begin{pgfscope}%
\pgfpathrectangle{\pgfqpoint{0.100000in}{0.879166in}}{\pgfqpoint{12.617144in}{4.304129in}}%
\pgfusepath{clip}%
\pgfsetbuttcap%
\pgfsetroundjoin%
\pgfsetlinewidth{2.007500pt}%
\definecolor{currentstroke}{rgb}{0.501961,0.501961,0.501961}%
\pgfsetstrokecolor{currentstroke}%
\pgfsetstrokeopacity{0.500000}%
\pgfsetdash{}{0pt}%
\pgfpathmoveto{\pgfqpoint{5.737582in}{4.552892in}}%
\pgfpathlineto{\pgfqpoint{5.813543in}{4.552892in}}%
\pgfusepath{stroke}%
\end{pgfscope}%
\begin{pgfscope}%
\pgfpathrectangle{\pgfqpoint{0.100000in}{0.879166in}}{\pgfqpoint{12.617144in}{4.304129in}}%
\pgfusepath{clip}%
\pgfsetbuttcap%
\pgfsetroundjoin%
\pgfsetlinewidth{2.007500pt}%
\definecolor{currentstroke}{rgb}{0.501961,0.501961,0.501961}%
\pgfsetstrokecolor{currentstroke}%
\pgfsetstrokeopacity{0.500000}%
\pgfsetdash{}{0pt}%
\pgfpathmoveto{\pgfqpoint{5.737582in}{4.987653in}}%
\pgfpathlineto{\pgfqpoint{5.813543in}{4.987653in}}%
\pgfusepath{stroke}%
\end{pgfscope}%
\begin{pgfscope}%
\pgfpathrectangle{\pgfqpoint{0.100000in}{0.879166in}}{\pgfqpoint{12.617144in}{4.304129in}}%
\pgfusepath{clip}%
\pgfsetrectcap%
\pgfsetroundjoin%
\pgfsetlinewidth{2.007500pt}%
\definecolor{currentstroke}{rgb}{0.501961,0.501961,0.501961}%
\pgfsetstrokecolor{currentstroke}%
\pgfsetstrokeopacity{0.500000}%
\pgfsetdash{}{0pt}%
\pgfpathmoveto{\pgfqpoint{7.041581in}{0.879166in}}%
\pgfpathlineto{\pgfqpoint{7.041581in}{5.183295in}}%
\pgfusepath{stroke}%
\end{pgfscope}%
\begin{pgfscope}%
\pgfpathrectangle{\pgfqpoint{0.100000in}{0.879166in}}{\pgfqpoint{12.617144in}{4.304129in}}%
\pgfusepath{clip}%
\pgfsetbuttcap%
\pgfsetroundjoin%
\pgfsetlinewidth{2.007500pt}%
\definecolor{currentstroke}{rgb}{0.501961,0.501961,0.501961}%
\pgfsetstrokecolor{currentstroke}%
\pgfsetstrokeopacity{0.500000}%
\pgfsetdash{}{0pt}%
\pgfpathmoveto{\pgfqpoint{7.003601in}{1.074808in}}%
\pgfpathlineto{\pgfqpoint{7.079562in}{1.074808in}}%
\pgfusepath{stroke}%
\end{pgfscope}%
\begin{pgfscope}%
\pgfpathrectangle{\pgfqpoint{0.100000in}{0.879166in}}{\pgfqpoint{12.617144in}{4.304129in}}%
\pgfusepath{clip}%
\pgfsetbuttcap%
\pgfsetroundjoin%
\pgfsetlinewidth{2.007500pt}%
\definecolor{currentstroke}{rgb}{0.501961,0.501961,0.501961}%
\pgfsetstrokecolor{currentstroke}%
\pgfsetstrokeopacity{0.500000}%
\pgfsetdash{}{0pt}%
\pgfpathmoveto{\pgfqpoint{7.003601in}{1.509569in}}%
\pgfpathlineto{\pgfqpoint{7.079562in}{1.509569in}}%
\pgfusepath{stroke}%
\end{pgfscope}%
\begin{pgfscope}%
\pgfpathrectangle{\pgfqpoint{0.100000in}{0.879166in}}{\pgfqpoint{12.617144in}{4.304129in}}%
\pgfusepath{clip}%
\pgfsetbuttcap%
\pgfsetroundjoin%
\pgfsetlinewidth{2.007500pt}%
\definecolor{currentstroke}{rgb}{0.501961,0.501961,0.501961}%
\pgfsetstrokecolor{currentstroke}%
\pgfsetstrokeopacity{0.500000}%
\pgfsetdash{}{0pt}%
\pgfpathmoveto{\pgfqpoint{7.003601in}{1.944329in}}%
\pgfpathlineto{\pgfqpoint{7.079562in}{1.944329in}}%
\pgfusepath{stroke}%
\end{pgfscope}%
\begin{pgfscope}%
\pgfpathrectangle{\pgfqpoint{0.100000in}{0.879166in}}{\pgfqpoint{12.617144in}{4.304129in}}%
\pgfusepath{clip}%
\pgfsetbuttcap%
\pgfsetroundjoin%
\pgfsetlinewidth{2.007500pt}%
\definecolor{currentstroke}{rgb}{0.501961,0.501961,0.501961}%
\pgfsetstrokecolor{currentstroke}%
\pgfsetstrokeopacity{0.500000}%
\pgfsetdash{}{0pt}%
\pgfpathmoveto{\pgfqpoint{7.003601in}{2.379090in}}%
\pgfpathlineto{\pgfqpoint{7.079562in}{2.379090in}}%
\pgfusepath{stroke}%
\end{pgfscope}%
\begin{pgfscope}%
\pgfpathrectangle{\pgfqpoint{0.100000in}{0.879166in}}{\pgfqpoint{12.617144in}{4.304129in}}%
\pgfusepath{clip}%
\pgfsetbuttcap%
\pgfsetroundjoin%
\pgfsetlinewidth{2.007500pt}%
\definecolor{currentstroke}{rgb}{0.501961,0.501961,0.501961}%
\pgfsetstrokecolor{currentstroke}%
\pgfsetstrokeopacity{0.500000}%
\pgfsetdash{}{0pt}%
\pgfpathmoveto{\pgfqpoint{7.003601in}{2.813850in}}%
\pgfpathlineto{\pgfqpoint{7.079562in}{2.813850in}}%
\pgfusepath{stroke}%
\end{pgfscope}%
\begin{pgfscope}%
\pgfpathrectangle{\pgfqpoint{0.100000in}{0.879166in}}{\pgfqpoint{12.617144in}{4.304129in}}%
\pgfusepath{clip}%
\pgfsetbuttcap%
\pgfsetroundjoin%
\pgfsetlinewidth{2.007500pt}%
\definecolor{currentstroke}{rgb}{0.501961,0.501961,0.501961}%
\pgfsetstrokecolor{currentstroke}%
\pgfsetstrokeopacity{0.500000}%
\pgfsetdash{}{0pt}%
\pgfpathmoveto{\pgfqpoint{7.003601in}{3.248611in}}%
\pgfpathlineto{\pgfqpoint{7.079562in}{3.248611in}}%
\pgfusepath{stroke}%
\end{pgfscope}%
\begin{pgfscope}%
\pgfpathrectangle{\pgfqpoint{0.100000in}{0.879166in}}{\pgfqpoint{12.617144in}{4.304129in}}%
\pgfusepath{clip}%
\pgfsetbuttcap%
\pgfsetroundjoin%
\pgfsetlinewidth{2.007500pt}%
\definecolor{currentstroke}{rgb}{0.501961,0.501961,0.501961}%
\pgfsetstrokecolor{currentstroke}%
\pgfsetstrokeopacity{0.500000}%
\pgfsetdash{}{0pt}%
\pgfpathmoveto{\pgfqpoint{7.003601in}{3.683371in}}%
\pgfpathlineto{\pgfqpoint{7.079562in}{3.683371in}}%
\pgfusepath{stroke}%
\end{pgfscope}%
\begin{pgfscope}%
\pgfpathrectangle{\pgfqpoint{0.100000in}{0.879166in}}{\pgfqpoint{12.617144in}{4.304129in}}%
\pgfusepath{clip}%
\pgfsetbuttcap%
\pgfsetroundjoin%
\pgfsetlinewidth{2.007500pt}%
\definecolor{currentstroke}{rgb}{0.501961,0.501961,0.501961}%
\pgfsetstrokecolor{currentstroke}%
\pgfsetstrokeopacity{0.500000}%
\pgfsetdash{}{0pt}%
\pgfpathmoveto{\pgfqpoint{7.003601in}{4.118132in}}%
\pgfpathlineto{\pgfqpoint{7.079562in}{4.118132in}}%
\pgfusepath{stroke}%
\end{pgfscope}%
\begin{pgfscope}%
\pgfpathrectangle{\pgfqpoint{0.100000in}{0.879166in}}{\pgfqpoint{12.617144in}{4.304129in}}%
\pgfusepath{clip}%
\pgfsetbuttcap%
\pgfsetroundjoin%
\pgfsetlinewidth{2.007500pt}%
\definecolor{currentstroke}{rgb}{0.501961,0.501961,0.501961}%
\pgfsetstrokecolor{currentstroke}%
\pgfsetstrokeopacity{0.500000}%
\pgfsetdash{}{0pt}%
\pgfpathmoveto{\pgfqpoint{7.003601in}{4.552892in}}%
\pgfpathlineto{\pgfqpoint{7.079562in}{4.552892in}}%
\pgfusepath{stroke}%
\end{pgfscope}%
\begin{pgfscope}%
\pgfpathrectangle{\pgfqpoint{0.100000in}{0.879166in}}{\pgfqpoint{12.617144in}{4.304129in}}%
\pgfusepath{clip}%
\pgfsetbuttcap%
\pgfsetroundjoin%
\pgfsetlinewidth{2.007500pt}%
\definecolor{currentstroke}{rgb}{0.501961,0.501961,0.501961}%
\pgfsetstrokecolor{currentstroke}%
\pgfsetstrokeopacity{0.500000}%
\pgfsetdash{}{0pt}%
\pgfpathmoveto{\pgfqpoint{7.003601in}{4.987653in}}%
\pgfpathlineto{\pgfqpoint{7.079562in}{4.987653in}}%
\pgfusepath{stroke}%
\end{pgfscope}%
\begin{pgfscope}%
\pgfpathrectangle{\pgfqpoint{0.100000in}{0.879166in}}{\pgfqpoint{12.617144in}{4.304129in}}%
\pgfusepath{clip}%
\pgfsetrectcap%
\pgfsetroundjoin%
\pgfsetlinewidth{2.007500pt}%
\definecolor{currentstroke}{rgb}{0.501961,0.501961,0.501961}%
\pgfsetstrokecolor{currentstroke}%
\pgfsetstrokeopacity{0.500000}%
\pgfsetdash{}{0pt}%
\pgfpathmoveto{\pgfqpoint{8.307600in}{0.879166in}}%
\pgfpathlineto{\pgfqpoint{8.307600in}{5.183295in}}%
\pgfusepath{stroke}%
\end{pgfscope}%
\begin{pgfscope}%
\pgfpathrectangle{\pgfqpoint{0.100000in}{0.879166in}}{\pgfqpoint{12.617144in}{4.304129in}}%
\pgfusepath{clip}%
\pgfsetbuttcap%
\pgfsetroundjoin%
\pgfsetlinewidth{2.007500pt}%
\definecolor{currentstroke}{rgb}{0.501961,0.501961,0.501961}%
\pgfsetstrokecolor{currentstroke}%
\pgfsetstrokeopacity{0.500000}%
\pgfsetdash{}{0pt}%
\pgfpathmoveto{\pgfqpoint{8.269620in}{1.074808in}}%
\pgfpathlineto{\pgfqpoint{8.345581in}{1.074808in}}%
\pgfusepath{stroke}%
\end{pgfscope}%
\begin{pgfscope}%
\pgfpathrectangle{\pgfqpoint{0.100000in}{0.879166in}}{\pgfqpoint{12.617144in}{4.304129in}}%
\pgfusepath{clip}%
\pgfsetbuttcap%
\pgfsetroundjoin%
\pgfsetlinewidth{2.007500pt}%
\definecolor{currentstroke}{rgb}{0.501961,0.501961,0.501961}%
\pgfsetstrokecolor{currentstroke}%
\pgfsetstrokeopacity{0.500000}%
\pgfsetdash{}{0pt}%
\pgfpathmoveto{\pgfqpoint{8.269620in}{1.509569in}}%
\pgfpathlineto{\pgfqpoint{8.345581in}{1.509569in}}%
\pgfusepath{stroke}%
\end{pgfscope}%
\begin{pgfscope}%
\pgfpathrectangle{\pgfqpoint{0.100000in}{0.879166in}}{\pgfqpoint{12.617144in}{4.304129in}}%
\pgfusepath{clip}%
\pgfsetbuttcap%
\pgfsetroundjoin%
\pgfsetlinewidth{2.007500pt}%
\definecolor{currentstroke}{rgb}{0.501961,0.501961,0.501961}%
\pgfsetstrokecolor{currentstroke}%
\pgfsetstrokeopacity{0.500000}%
\pgfsetdash{}{0pt}%
\pgfpathmoveto{\pgfqpoint{8.269620in}{1.944329in}}%
\pgfpathlineto{\pgfqpoint{8.345581in}{1.944329in}}%
\pgfusepath{stroke}%
\end{pgfscope}%
\begin{pgfscope}%
\pgfpathrectangle{\pgfqpoint{0.100000in}{0.879166in}}{\pgfqpoint{12.617144in}{4.304129in}}%
\pgfusepath{clip}%
\pgfsetbuttcap%
\pgfsetroundjoin%
\pgfsetlinewidth{2.007500pt}%
\definecolor{currentstroke}{rgb}{0.501961,0.501961,0.501961}%
\pgfsetstrokecolor{currentstroke}%
\pgfsetstrokeopacity{0.500000}%
\pgfsetdash{}{0pt}%
\pgfpathmoveto{\pgfqpoint{8.269620in}{2.379090in}}%
\pgfpathlineto{\pgfqpoint{8.345581in}{2.379090in}}%
\pgfusepath{stroke}%
\end{pgfscope}%
\begin{pgfscope}%
\pgfpathrectangle{\pgfqpoint{0.100000in}{0.879166in}}{\pgfqpoint{12.617144in}{4.304129in}}%
\pgfusepath{clip}%
\pgfsetbuttcap%
\pgfsetroundjoin%
\pgfsetlinewidth{2.007500pt}%
\definecolor{currentstroke}{rgb}{0.501961,0.501961,0.501961}%
\pgfsetstrokecolor{currentstroke}%
\pgfsetstrokeopacity{0.500000}%
\pgfsetdash{}{0pt}%
\pgfpathmoveto{\pgfqpoint{8.269620in}{2.813850in}}%
\pgfpathlineto{\pgfqpoint{8.345581in}{2.813850in}}%
\pgfusepath{stroke}%
\end{pgfscope}%
\begin{pgfscope}%
\pgfpathrectangle{\pgfqpoint{0.100000in}{0.879166in}}{\pgfqpoint{12.617144in}{4.304129in}}%
\pgfusepath{clip}%
\pgfsetbuttcap%
\pgfsetroundjoin%
\pgfsetlinewidth{2.007500pt}%
\definecolor{currentstroke}{rgb}{0.501961,0.501961,0.501961}%
\pgfsetstrokecolor{currentstroke}%
\pgfsetstrokeopacity{0.500000}%
\pgfsetdash{}{0pt}%
\pgfpathmoveto{\pgfqpoint{8.269620in}{3.248611in}}%
\pgfpathlineto{\pgfqpoint{8.345581in}{3.248611in}}%
\pgfusepath{stroke}%
\end{pgfscope}%
\begin{pgfscope}%
\pgfpathrectangle{\pgfqpoint{0.100000in}{0.879166in}}{\pgfqpoint{12.617144in}{4.304129in}}%
\pgfusepath{clip}%
\pgfsetbuttcap%
\pgfsetroundjoin%
\pgfsetlinewidth{2.007500pt}%
\definecolor{currentstroke}{rgb}{0.501961,0.501961,0.501961}%
\pgfsetstrokecolor{currentstroke}%
\pgfsetstrokeopacity{0.500000}%
\pgfsetdash{}{0pt}%
\pgfpathmoveto{\pgfqpoint{8.269620in}{3.683371in}}%
\pgfpathlineto{\pgfqpoint{8.345581in}{3.683371in}}%
\pgfusepath{stroke}%
\end{pgfscope}%
\begin{pgfscope}%
\pgfpathrectangle{\pgfqpoint{0.100000in}{0.879166in}}{\pgfqpoint{12.617144in}{4.304129in}}%
\pgfusepath{clip}%
\pgfsetbuttcap%
\pgfsetroundjoin%
\pgfsetlinewidth{2.007500pt}%
\definecolor{currentstroke}{rgb}{0.501961,0.501961,0.501961}%
\pgfsetstrokecolor{currentstroke}%
\pgfsetstrokeopacity{0.500000}%
\pgfsetdash{}{0pt}%
\pgfpathmoveto{\pgfqpoint{8.269620in}{4.118132in}}%
\pgfpathlineto{\pgfqpoint{8.345581in}{4.118132in}}%
\pgfusepath{stroke}%
\end{pgfscope}%
\begin{pgfscope}%
\pgfpathrectangle{\pgfqpoint{0.100000in}{0.879166in}}{\pgfqpoint{12.617144in}{4.304129in}}%
\pgfusepath{clip}%
\pgfsetbuttcap%
\pgfsetroundjoin%
\pgfsetlinewidth{2.007500pt}%
\definecolor{currentstroke}{rgb}{0.501961,0.501961,0.501961}%
\pgfsetstrokecolor{currentstroke}%
\pgfsetstrokeopacity{0.500000}%
\pgfsetdash{}{0pt}%
\pgfpathmoveto{\pgfqpoint{8.269620in}{4.552892in}}%
\pgfpathlineto{\pgfqpoint{8.345581in}{4.552892in}}%
\pgfusepath{stroke}%
\end{pgfscope}%
\begin{pgfscope}%
\pgfpathrectangle{\pgfqpoint{0.100000in}{0.879166in}}{\pgfqpoint{12.617144in}{4.304129in}}%
\pgfusepath{clip}%
\pgfsetbuttcap%
\pgfsetroundjoin%
\pgfsetlinewidth{2.007500pt}%
\definecolor{currentstroke}{rgb}{0.501961,0.501961,0.501961}%
\pgfsetstrokecolor{currentstroke}%
\pgfsetstrokeopacity{0.500000}%
\pgfsetdash{}{0pt}%
\pgfpathmoveto{\pgfqpoint{8.269620in}{4.987653in}}%
\pgfpathlineto{\pgfqpoint{8.345581in}{4.987653in}}%
\pgfusepath{stroke}%
\end{pgfscope}%
\begin{pgfscope}%
\pgfpathrectangle{\pgfqpoint{0.100000in}{0.879166in}}{\pgfqpoint{12.617144in}{4.304129in}}%
\pgfusepath{clip}%
\pgfsetrectcap%
\pgfsetroundjoin%
\pgfsetlinewidth{2.007500pt}%
\definecolor{currentstroke}{rgb}{0.501961,0.501961,0.501961}%
\pgfsetstrokecolor{currentstroke}%
\pgfsetstrokeopacity{0.500000}%
\pgfsetdash{}{0pt}%
\pgfpathmoveto{\pgfqpoint{9.573619in}{0.879166in}}%
\pgfpathlineto{\pgfqpoint{9.573619in}{5.183295in}}%
\pgfusepath{stroke}%
\end{pgfscope}%
\begin{pgfscope}%
\pgfpathrectangle{\pgfqpoint{0.100000in}{0.879166in}}{\pgfqpoint{12.617144in}{4.304129in}}%
\pgfusepath{clip}%
\pgfsetbuttcap%
\pgfsetroundjoin%
\pgfsetlinewidth{2.007500pt}%
\definecolor{currentstroke}{rgb}{0.501961,0.501961,0.501961}%
\pgfsetstrokecolor{currentstroke}%
\pgfsetstrokeopacity{0.500000}%
\pgfsetdash{}{0pt}%
\pgfpathmoveto{\pgfqpoint{9.535638in}{1.074808in}}%
\pgfpathlineto{\pgfqpoint{9.611600in}{1.074808in}}%
\pgfusepath{stroke}%
\end{pgfscope}%
\begin{pgfscope}%
\pgfpathrectangle{\pgfqpoint{0.100000in}{0.879166in}}{\pgfqpoint{12.617144in}{4.304129in}}%
\pgfusepath{clip}%
\pgfsetbuttcap%
\pgfsetroundjoin%
\pgfsetlinewidth{2.007500pt}%
\definecolor{currentstroke}{rgb}{0.501961,0.501961,0.501961}%
\pgfsetstrokecolor{currentstroke}%
\pgfsetstrokeopacity{0.500000}%
\pgfsetdash{}{0pt}%
\pgfpathmoveto{\pgfqpoint{9.535638in}{1.509569in}}%
\pgfpathlineto{\pgfqpoint{9.611600in}{1.509569in}}%
\pgfusepath{stroke}%
\end{pgfscope}%
\begin{pgfscope}%
\pgfpathrectangle{\pgfqpoint{0.100000in}{0.879166in}}{\pgfqpoint{12.617144in}{4.304129in}}%
\pgfusepath{clip}%
\pgfsetbuttcap%
\pgfsetroundjoin%
\pgfsetlinewidth{2.007500pt}%
\definecolor{currentstroke}{rgb}{0.501961,0.501961,0.501961}%
\pgfsetstrokecolor{currentstroke}%
\pgfsetstrokeopacity{0.500000}%
\pgfsetdash{}{0pt}%
\pgfpathmoveto{\pgfqpoint{9.535638in}{1.944329in}}%
\pgfpathlineto{\pgfqpoint{9.611600in}{1.944329in}}%
\pgfusepath{stroke}%
\end{pgfscope}%
\begin{pgfscope}%
\pgfpathrectangle{\pgfqpoint{0.100000in}{0.879166in}}{\pgfqpoint{12.617144in}{4.304129in}}%
\pgfusepath{clip}%
\pgfsetbuttcap%
\pgfsetroundjoin%
\pgfsetlinewidth{2.007500pt}%
\definecolor{currentstroke}{rgb}{0.501961,0.501961,0.501961}%
\pgfsetstrokecolor{currentstroke}%
\pgfsetstrokeopacity{0.500000}%
\pgfsetdash{}{0pt}%
\pgfpathmoveto{\pgfqpoint{9.535638in}{2.379090in}}%
\pgfpathlineto{\pgfqpoint{9.611600in}{2.379090in}}%
\pgfusepath{stroke}%
\end{pgfscope}%
\begin{pgfscope}%
\pgfpathrectangle{\pgfqpoint{0.100000in}{0.879166in}}{\pgfqpoint{12.617144in}{4.304129in}}%
\pgfusepath{clip}%
\pgfsetbuttcap%
\pgfsetroundjoin%
\pgfsetlinewidth{2.007500pt}%
\definecolor{currentstroke}{rgb}{0.501961,0.501961,0.501961}%
\pgfsetstrokecolor{currentstroke}%
\pgfsetstrokeopacity{0.500000}%
\pgfsetdash{}{0pt}%
\pgfpathmoveto{\pgfqpoint{9.535638in}{2.813850in}}%
\pgfpathlineto{\pgfqpoint{9.611600in}{2.813850in}}%
\pgfusepath{stroke}%
\end{pgfscope}%
\begin{pgfscope}%
\pgfpathrectangle{\pgfqpoint{0.100000in}{0.879166in}}{\pgfqpoint{12.617144in}{4.304129in}}%
\pgfusepath{clip}%
\pgfsetbuttcap%
\pgfsetroundjoin%
\pgfsetlinewidth{2.007500pt}%
\definecolor{currentstroke}{rgb}{0.501961,0.501961,0.501961}%
\pgfsetstrokecolor{currentstroke}%
\pgfsetstrokeopacity{0.500000}%
\pgfsetdash{}{0pt}%
\pgfpathmoveto{\pgfqpoint{9.535638in}{3.248611in}}%
\pgfpathlineto{\pgfqpoint{9.611600in}{3.248611in}}%
\pgfusepath{stroke}%
\end{pgfscope}%
\begin{pgfscope}%
\pgfpathrectangle{\pgfqpoint{0.100000in}{0.879166in}}{\pgfqpoint{12.617144in}{4.304129in}}%
\pgfusepath{clip}%
\pgfsetbuttcap%
\pgfsetroundjoin%
\pgfsetlinewidth{2.007500pt}%
\definecolor{currentstroke}{rgb}{0.501961,0.501961,0.501961}%
\pgfsetstrokecolor{currentstroke}%
\pgfsetstrokeopacity{0.500000}%
\pgfsetdash{}{0pt}%
\pgfpathmoveto{\pgfqpoint{9.535638in}{3.683371in}}%
\pgfpathlineto{\pgfqpoint{9.611600in}{3.683371in}}%
\pgfusepath{stroke}%
\end{pgfscope}%
\begin{pgfscope}%
\pgfpathrectangle{\pgfqpoint{0.100000in}{0.879166in}}{\pgfqpoint{12.617144in}{4.304129in}}%
\pgfusepath{clip}%
\pgfsetbuttcap%
\pgfsetroundjoin%
\pgfsetlinewidth{2.007500pt}%
\definecolor{currentstroke}{rgb}{0.501961,0.501961,0.501961}%
\pgfsetstrokecolor{currentstroke}%
\pgfsetstrokeopacity{0.500000}%
\pgfsetdash{}{0pt}%
\pgfpathmoveto{\pgfqpoint{9.535638in}{4.118132in}}%
\pgfpathlineto{\pgfqpoint{9.611600in}{4.118132in}}%
\pgfusepath{stroke}%
\end{pgfscope}%
\begin{pgfscope}%
\pgfpathrectangle{\pgfqpoint{0.100000in}{0.879166in}}{\pgfqpoint{12.617144in}{4.304129in}}%
\pgfusepath{clip}%
\pgfsetbuttcap%
\pgfsetroundjoin%
\pgfsetlinewidth{2.007500pt}%
\definecolor{currentstroke}{rgb}{0.501961,0.501961,0.501961}%
\pgfsetstrokecolor{currentstroke}%
\pgfsetstrokeopacity{0.500000}%
\pgfsetdash{}{0pt}%
\pgfpathmoveto{\pgfqpoint{9.535638in}{4.552892in}}%
\pgfpathlineto{\pgfqpoint{9.611600in}{4.552892in}}%
\pgfusepath{stroke}%
\end{pgfscope}%
\begin{pgfscope}%
\pgfpathrectangle{\pgfqpoint{0.100000in}{0.879166in}}{\pgfqpoint{12.617144in}{4.304129in}}%
\pgfusepath{clip}%
\pgfsetbuttcap%
\pgfsetroundjoin%
\pgfsetlinewidth{2.007500pt}%
\definecolor{currentstroke}{rgb}{0.501961,0.501961,0.501961}%
\pgfsetstrokecolor{currentstroke}%
\pgfsetstrokeopacity{0.500000}%
\pgfsetdash{}{0pt}%
\pgfpathmoveto{\pgfqpoint{9.535638in}{4.987653in}}%
\pgfpathlineto{\pgfqpoint{9.611600in}{4.987653in}}%
\pgfusepath{stroke}%
\end{pgfscope}%
\begin{pgfscope}%
\pgfpathrectangle{\pgfqpoint{0.100000in}{0.879166in}}{\pgfqpoint{12.617144in}{4.304129in}}%
\pgfusepath{clip}%
\pgfsetrectcap%
\pgfsetroundjoin%
\pgfsetlinewidth{2.007500pt}%
\definecolor{currentstroke}{rgb}{0.501961,0.501961,0.501961}%
\pgfsetstrokecolor{currentstroke}%
\pgfsetstrokeopacity{0.500000}%
\pgfsetdash{}{0pt}%
\pgfpathmoveto{\pgfqpoint{10.839638in}{0.879166in}}%
\pgfpathlineto{\pgfqpoint{10.839638in}{5.183295in}}%
\pgfusepath{stroke}%
\end{pgfscope}%
\begin{pgfscope}%
\pgfpathrectangle{\pgfqpoint{0.100000in}{0.879166in}}{\pgfqpoint{12.617144in}{4.304129in}}%
\pgfusepath{clip}%
\pgfsetbuttcap%
\pgfsetroundjoin%
\pgfsetlinewidth{2.007500pt}%
\definecolor{currentstroke}{rgb}{0.501961,0.501961,0.501961}%
\pgfsetstrokecolor{currentstroke}%
\pgfsetstrokeopacity{0.500000}%
\pgfsetdash{}{0pt}%
\pgfpathmoveto{\pgfqpoint{10.801657in}{1.074808in}}%
\pgfpathlineto{\pgfqpoint{10.877618in}{1.074808in}}%
\pgfusepath{stroke}%
\end{pgfscope}%
\begin{pgfscope}%
\pgfpathrectangle{\pgfqpoint{0.100000in}{0.879166in}}{\pgfqpoint{12.617144in}{4.304129in}}%
\pgfusepath{clip}%
\pgfsetbuttcap%
\pgfsetroundjoin%
\pgfsetlinewidth{2.007500pt}%
\definecolor{currentstroke}{rgb}{0.501961,0.501961,0.501961}%
\pgfsetstrokecolor{currentstroke}%
\pgfsetstrokeopacity{0.500000}%
\pgfsetdash{}{0pt}%
\pgfpathmoveto{\pgfqpoint{10.801657in}{1.509569in}}%
\pgfpathlineto{\pgfqpoint{10.877618in}{1.509569in}}%
\pgfusepath{stroke}%
\end{pgfscope}%
\begin{pgfscope}%
\pgfpathrectangle{\pgfqpoint{0.100000in}{0.879166in}}{\pgfqpoint{12.617144in}{4.304129in}}%
\pgfusepath{clip}%
\pgfsetbuttcap%
\pgfsetroundjoin%
\pgfsetlinewidth{2.007500pt}%
\definecolor{currentstroke}{rgb}{0.501961,0.501961,0.501961}%
\pgfsetstrokecolor{currentstroke}%
\pgfsetstrokeopacity{0.500000}%
\pgfsetdash{}{0pt}%
\pgfpathmoveto{\pgfqpoint{10.801657in}{1.944329in}}%
\pgfpathlineto{\pgfqpoint{10.877618in}{1.944329in}}%
\pgfusepath{stroke}%
\end{pgfscope}%
\begin{pgfscope}%
\pgfpathrectangle{\pgfqpoint{0.100000in}{0.879166in}}{\pgfqpoint{12.617144in}{4.304129in}}%
\pgfusepath{clip}%
\pgfsetbuttcap%
\pgfsetroundjoin%
\pgfsetlinewidth{2.007500pt}%
\definecolor{currentstroke}{rgb}{0.501961,0.501961,0.501961}%
\pgfsetstrokecolor{currentstroke}%
\pgfsetstrokeopacity{0.500000}%
\pgfsetdash{}{0pt}%
\pgfpathmoveto{\pgfqpoint{10.801657in}{2.379090in}}%
\pgfpathlineto{\pgfqpoint{10.877618in}{2.379090in}}%
\pgfusepath{stroke}%
\end{pgfscope}%
\begin{pgfscope}%
\pgfpathrectangle{\pgfqpoint{0.100000in}{0.879166in}}{\pgfqpoint{12.617144in}{4.304129in}}%
\pgfusepath{clip}%
\pgfsetbuttcap%
\pgfsetroundjoin%
\pgfsetlinewidth{2.007500pt}%
\definecolor{currentstroke}{rgb}{0.501961,0.501961,0.501961}%
\pgfsetstrokecolor{currentstroke}%
\pgfsetstrokeopacity{0.500000}%
\pgfsetdash{}{0pt}%
\pgfpathmoveto{\pgfqpoint{10.801657in}{2.813850in}}%
\pgfpathlineto{\pgfqpoint{10.877618in}{2.813850in}}%
\pgfusepath{stroke}%
\end{pgfscope}%
\begin{pgfscope}%
\pgfpathrectangle{\pgfqpoint{0.100000in}{0.879166in}}{\pgfqpoint{12.617144in}{4.304129in}}%
\pgfusepath{clip}%
\pgfsetbuttcap%
\pgfsetroundjoin%
\pgfsetlinewidth{2.007500pt}%
\definecolor{currentstroke}{rgb}{0.501961,0.501961,0.501961}%
\pgfsetstrokecolor{currentstroke}%
\pgfsetstrokeopacity{0.500000}%
\pgfsetdash{}{0pt}%
\pgfpathmoveto{\pgfqpoint{10.801657in}{3.248611in}}%
\pgfpathlineto{\pgfqpoint{10.877618in}{3.248611in}}%
\pgfusepath{stroke}%
\end{pgfscope}%
\begin{pgfscope}%
\pgfpathrectangle{\pgfqpoint{0.100000in}{0.879166in}}{\pgfqpoint{12.617144in}{4.304129in}}%
\pgfusepath{clip}%
\pgfsetbuttcap%
\pgfsetroundjoin%
\pgfsetlinewidth{2.007500pt}%
\definecolor{currentstroke}{rgb}{0.501961,0.501961,0.501961}%
\pgfsetstrokecolor{currentstroke}%
\pgfsetstrokeopacity{0.500000}%
\pgfsetdash{}{0pt}%
\pgfpathmoveto{\pgfqpoint{10.801657in}{3.683371in}}%
\pgfpathlineto{\pgfqpoint{10.877618in}{3.683371in}}%
\pgfusepath{stroke}%
\end{pgfscope}%
\begin{pgfscope}%
\pgfpathrectangle{\pgfqpoint{0.100000in}{0.879166in}}{\pgfqpoint{12.617144in}{4.304129in}}%
\pgfusepath{clip}%
\pgfsetbuttcap%
\pgfsetroundjoin%
\pgfsetlinewidth{2.007500pt}%
\definecolor{currentstroke}{rgb}{0.501961,0.501961,0.501961}%
\pgfsetstrokecolor{currentstroke}%
\pgfsetstrokeopacity{0.500000}%
\pgfsetdash{}{0pt}%
\pgfpathmoveto{\pgfqpoint{10.801657in}{4.118132in}}%
\pgfpathlineto{\pgfqpoint{10.877618in}{4.118132in}}%
\pgfusepath{stroke}%
\end{pgfscope}%
\begin{pgfscope}%
\pgfpathrectangle{\pgfqpoint{0.100000in}{0.879166in}}{\pgfqpoint{12.617144in}{4.304129in}}%
\pgfusepath{clip}%
\pgfsetbuttcap%
\pgfsetroundjoin%
\pgfsetlinewidth{2.007500pt}%
\definecolor{currentstroke}{rgb}{0.501961,0.501961,0.501961}%
\pgfsetstrokecolor{currentstroke}%
\pgfsetstrokeopacity{0.500000}%
\pgfsetdash{}{0pt}%
\pgfpathmoveto{\pgfqpoint{10.801657in}{4.552892in}}%
\pgfpathlineto{\pgfqpoint{10.877618in}{4.552892in}}%
\pgfusepath{stroke}%
\end{pgfscope}%
\begin{pgfscope}%
\pgfpathrectangle{\pgfqpoint{0.100000in}{0.879166in}}{\pgfqpoint{12.617144in}{4.304129in}}%
\pgfusepath{clip}%
\pgfsetbuttcap%
\pgfsetroundjoin%
\pgfsetlinewidth{2.007500pt}%
\definecolor{currentstroke}{rgb}{0.501961,0.501961,0.501961}%
\pgfsetstrokecolor{currentstroke}%
\pgfsetstrokeopacity{0.500000}%
\pgfsetdash{}{0pt}%
\pgfpathmoveto{\pgfqpoint{10.801657in}{4.987653in}}%
\pgfpathlineto{\pgfqpoint{10.877618in}{4.987653in}}%
\pgfusepath{stroke}%
\end{pgfscope}%
\begin{pgfscope}%
\pgfpathrectangle{\pgfqpoint{0.100000in}{0.879166in}}{\pgfqpoint{12.617144in}{4.304129in}}%
\pgfusepath{clip}%
\pgfsetrectcap%
\pgfsetroundjoin%
\pgfsetlinewidth{2.007500pt}%
\definecolor{currentstroke}{rgb}{0.501961,0.501961,0.501961}%
\pgfsetstrokecolor{currentstroke}%
\pgfsetstrokeopacity{0.500000}%
\pgfsetdash{}{0pt}%
\pgfpathmoveto{\pgfqpoint{12.105657in}{0.879166in}}%
\pgfpathlineto{\pgfqpoint{12.105657in}{5.183295in}}%
\pgfusepath{stroke}%
\end{pgfscope}%
\begin{pgfscope}%
\pgfpathrectangle{\pgfqpoint{0.100000in}{0.879166in}}{\pgfqpoint{12.617144in}{4.304129in}}%
\pgfusepath{clip}%
\pgfsetbuttcap%
\pgfsetroundjoin%
\pgfsetlinewidth{2.007500pt}%
\definecolor{currentstroke}{rgb}{0.501961,0.501961,0.501961}%
\pgfsetstrokecolor{currentstroke}%
\pgfsetstrokeopacity{0.500000}%
\pgfsetdash{}{0pt}%
\pgfpathmoveto{\pgfqpoint{12.067676in}{1.074808in}}%
\pgfpathlineto{\pgfqpoint{12.143637in}{1.074808in}}%
\pgfusepath{stroke}%
\end{pgfscope}%
\begin{pgfscope}%
\pgfpathrectangle{\pgfqpoint{0.100000in}{0.879166in}}{\pgfqpoint{12.617144in}{4.304129in}}%
\pgfusepath{clip}%
\pgfsetbuttcap%
\pgfsetroundjoin%
\pgfsetlinewidth{2.007500pt}%
\definecolor{currentstroke}{rgb}{0.501961,0.501961,0.501961}%
\pgfsetstrokecolor{currentstroke}%
\pgfsetstrokeopacity{0.500000}%
\pgfsetdash{}{0pt}%
\pgfpathmoveto{\pgfqpoint{12.067676in}{1.509569in}}%
\pgfpathlineto{\pgfqpoint{12.143637in}{1.509569in}}%
\pgfusepath{stroke}%
\end{pgfscope}%
\begin{pgfscope}%
\pgfpathrectangle{\pgfqpoint{0.100000in}{0.879166in}}{\pgfqpoint{12.617144in}{4.304129in}}%
\pgfusepath{clip}%
\pgfsetbuttcap%
\pgfsetroundjoin%
\pgfsetlinewidth{2.007500pt}%
\definecolor{currentstroke}{rgb}{0.501961,0.501961,0.501961}%
\pgfsetstrokecolor{currentstroke}%
\pgfsetstrokeopacity{0.500000}%
\pgfsetdash{}{0pt}%
\pgfpathmoveto{\pgfqpoint{12.067676in}{1.944329in}}%
\pgfpathlineto{\pgfqpoint{12.143637in}{1.944329in}}%
\pgfusepath{stroke}%
\end{pgfscope}%
\begin{pgfscope}%
\pgfpathrectangle{\pgfqpoint{0.100000in}{0.879166in}}{\pgfqpoint{12.617144in}{4.304129in}}%
\pgfusepath{clip}%
\pgfsetbuttcap%
\pgfsetroundjoin%
\pgfsetlinewidth{2.007500pt}%
\definecolor{currentstroke}{rgb}{0.501961,0.501961,0.501961}%
\pgfsetstrokecolor{currentstroke}%
\pgfsetstrokeopacity{0.500000}%
\pgfsetdash{}{0pt}%
\pgfpathmoveto{\pgfqpoint{12.067676in}{2.379090in}}%
\pgfpathlineto{\pgfqpoint{12.143637in}{2.379090in}}%
\pgfusepath{stroke}%
\end{pgfscope}%
\begin{pgfscope}%
\pgfpathrectangle{\pgfqpoint{0.100000in}{0.879166in}}{\pgfqpoint{12.617144in}{4.304129in}}%
\pgfusepath{clip}%
\pgfsetbuttcap%
\pgfsetroundjoin%
\pgfsetlinewidth{2.007500pt}%
\definecolor{currentstroke}{rgb}{0.501961,0.501961,0.501961}%
\pgfsetstrokecolor{currentstroke}%
\pgfsetstrokeopacity{0.500000}%
\pgfsetdash{}{0pt}%
\pgfpathmoveto{\pgfqpoint{12.067676in}{2.813850in}}%
\pgfpathlineto{\pgfqpoint{12.143637in}{2.813850in}}%
\pgfusepath{stroke}%
\end{pgfscope}%
\begin{pgfscope}%
\pgfpathrectangle{\pgfqpoint{0.100000in}{0.879166in}}{\pgfqpoint{12.617144in}{4.304129in}}%
\pgfusepath{clip}%
\pgfsetbuttcap%
\pgfsetroundjoin%
\pgfsetlinewidth{2.007500pt}%
\definecolor{currentstroke}{rgb}{0.501961,0.501961,0.501961}%
\pgfsetstrokecolor{currentstroke}%
\pgfsetstrokeopacity{0.500000}%
\pgfsetdash{}{0pt}%
\pgfpathmoveto{\pgfqpoint{12.067676in}{3.248611in}}%
\pgfpathlineto{\pgfqpoint{12.143637in}{3.248611in}}%
\pgfusepath{stroke}%
\end{pgfscope}%
\begin{pgfscope}%
\pgfpathrectangle{\pgfqpoint{0.100000in}{0.879166in}}{\pgfqpoint{12.617144in}{4.304129in}}%
\pgfusepath{clip}%
\pgfsetbuttcap%
\pgfsetroundjoin%
\pgfsetlinewidth{2.007500pt}%
\definecolor{currentstroke}{rgb}{0.501961,0.501961,0.501961}%
\pgfsetstrokecolor{currentstroke}%
\pgfsetstrokeopacity{0.500000}%
\pgfsetdash{}{0pt}%
\pgfpathmoveto{\pgfqpoint{12.067676in}{3.683371in}}%
\pgfpathlineto{\pgfqpoint{12.143637in}{3.683371in}}%
\pgfusepath{stroke}%
\end{pgfscope}%
\begin{pgfscope}%
\pgfpathrectangle{\pgfqpoint{0.100000in}{0.879166in}}{\pgfqpoint{12.617144in}{4.304129in}}%
\pgfusepath{clip}%
\pgfsetbuttcap%
\pgfsetroundjoin%
\pgfsetlinewidth{2.007500pt}%
\definecolor{currentstroke}{rgb}{0.501961,0.501961,0.501961}%
\pgfsetstrokecolor{currentstroke}%
\pgfsetstrokeopacity{0.500000}%
\pgfsetdash{}{0pt}%
\pgfpathmoveto{\pgfqpoint{12.067676in}{4.118132in}}%
\pgfpathlineto{\pgfqpoint{12.143637in}{4.118132in}}%
\pgfusepath{stroke}%
\end{pgfscope}%
\begin{pgfscope}%
\pgfpathrectangle{\pgfqpoint{0.100000in}{0.879166in}}{\pgfqpoint{12.617144in}{4.304129in}}%
\pgfusepath{clip}%
\pgfsetbuttcap%
\pgfsetroundjoin%
\pgfsetlinewidth{2.007500pt}%
\definecolor{currentstroke}{rgb}{0.501961,0.501961,0.501961}%
\pgfsetstrokecolor{currentstroke}%
\pgfsetstrokeopacity{0.500000}%
\pgfsetdash{}{0pt}%
\pgfpathmoveto{\pgfqpoint{12.067676in}{4.552892in}}%
\pgfpathlineto{\pgfqpoint{12.143637in}{4.552892in}}%
\pgfusepath{stroke}%
\end{pgfscope}%
\begin{pgfscope}%
\pgfpathrectangle{\pgfqpoint{0.100000in}{0.879166in}}{\pgfqpoint{12.617144in}{4.304129in}}%
\pgfusepath{clip}%
\pgfsetbuttcap%
\pgfsetroundjoin%
\pgfsetlinewidth{2.007500pt}%
\definecolor{currentstroke}{rgb}{0.501961,0.501961,0.501961}%
\pgfsetstrokecolor{currentstroke}%
\pgfsetstrokeopacity{0.500000}%
\pgfsetdash{}{0pt}%
\pgfpathmoveto{\pgfqpoint{12.067676in}{4.987653in}}%
\pgfpathlineto{\pgfqpoint{12.143637in}{4.987653in}}%
\pgfusepath{stroke}%
\end{pgfscope}%
\begin{pgfscope}%
\pgfsetrectcap%
\pgfsetmiterjoin%
\pgfsetlinewidth{0.803000pt}%
\definecolor{currentstroke}{rgb}{0.000000,0.000000,0.000000}%
\pgfsetstrokecolor{currentstroke}%
\pgfsetdash{}{0pt}%
\pgfpathmoveto{\pgfqpoint{0.100000in}{0.879166in}}%
\pgfpathlineto{\pgfqpoint{12.717144in}{0.879166in}}%
\pgfusepath{stroke}%
\end{pgfscope}%
\begin{pgfscope}%
\pgfsetrectcap%
\pgfsetmiterjoin%
\pgfsetlinewidth{0.803000pt}%
\definecolor{currentstroke}{rgb}{0.000000,0.000000,0.000000}%
\pgfsetstrokecolor{currentstroke}%
\pgfsetdash{}{0pt}%
\pgfpathmoveto{\pgfqpoint{0.100000in}{5.183295in}}%
\pgfpathlineto{\pgfqpoint{12.717144in}{5.183295in}}%
\pgfusepath{stroke}%
\end{pgfscope}%
\begin{pgfscope}%
\definecolor{textcolor}{rgb}{0.000000,0.000000,0.000000}%
\pgfsetstrokecolor{textcolor}%
\pgfsetfillcolor{textcolor}%
\pgftext[x=0.610206in,y=0.683524in,left,base]{\color{textcolor}{\rmfamily\fontsize{10.000000}{12.000000}\selectfont\catcode`\^=\active\def^{\ifmmode\sp\else\^{}\fi}\catcode`\%=\active\def%{\%}0.00}}%
\end{pgfscope}%
\begin{pgfscope}%
\definecolor{textcolor}{rgb}{0.000000,0.000000,0.000000}%
\pgfsetstrokecolor{textcolor}%
\pgfsetfillcolor{textcolor}%
\pgftext[x=0.610206in,y=5.281116in,left,base]{\color{textcolor}{\rmfamily\fontsize{10.000000}{12.000000}\selectfont\catcode`\^=\active\def^{\ifmmode\sp\else\^{}\fi}\catcode`\%=\active\def%{\%}26.41}}%
\end{pgfscope}%
\begin{pgfscope}%
\definecolor{textcolor}{rgb}{0.000000,0.000000,0.000000}%
\pgfsetstrokecolor{textcolor}%
\pgfsetfillcolor{textcolor}%
\pgftext[x=1.876224in,y=0.683524in,left,base]{\color{textcolor}{\rmfamily\fontsize{10.000000}{12.000000}\selectfont\catcode`\^=\active\def^{\ifmmode\sp\else\^{}\fi}\catcode`\%=\active\def%{\%}0.00}}%
\end{pgfscope}%
\begin{pgfscope}%
\definecolor{textcolor}{rgb}{0.000000,0.000000,0.000000}%
\pgfsetstrokecolor{textcolor}%
\pgfsetfillcolor{textcolor}%
\pgftext[x=1.876224in,y=5.281116in,left,base]{\color{textcolor}{\rmfamily\fontsize{10.000000}{12.000000}\selectfont\catcode`\^=\active\def^{\ifmmode\sp\else\^{}\fi}\catcode`\%=\active\def%{\%}22.23}}%
\end{pgfscope}%
\begin{pgfscope}%
\definecolor{textcolor}{rgb}{0.000000,0.000000,0.000000}%
\pgfsetstrokecolor{textcolor}%
\pgfsetfillcolor{textcolor}%
\pgftext[x=3.142243in,y=0.683524in,left,base]{\color{textcolor}{\rmfamily\fontsize{10.000000}{12.000000}\selectfont\catcode`\^=\active\def^{\ifmmode\sp\else\^{}\fi}\catcode`\%=\active\def%{\%}0.08}}%
\end{pgfscope}%
\begin{pgfscope}%
\definecolor{textcolor}{rgb}{0.000000,0.000000,0.000000}%
\pgfsetstrokecolor{textcolor}%
\pgfsetfillcolor{textcolor}%
\pgftext[x=3.142243in,y=5.281116in,left,base]{\color{textcolor}{\rmfamily\fontsize{10.000000}{12.000000}\selectfont\catcode`\^=\active\def^{\ifmmode\sp\else\^{}\fi}\catcode`\%=\active\def%{\%}2.04}}%
\end{pgfscope}%
\begin{pgfscope}%
\definecolor{textcolor}{rgb}{0.000000,0.000000,0.000000}%
\pgfsetstrokecolor{textcolor}%
\pgfsetfillcolor{textcolor}%
\pgftext[x=4.408262in,y=0.683524in,left,base]{\color{textcolor}{\rmfamily\fontsize{10.000000}{12.000000}\selectfont\catcode`\^=\active\def^{\ifmmode\sp\else\^{}\fi}\catcode`\%=\active\def%{\%}0.00}}%
\end{pgfscope}%
\begin{pgfscope}%
\definecolor{textcolor}{rgb}{0.000000,0.000000,0.000000}%
\pgfsetstrokecolor{textcolor}%
\pgfsetfillcolor{textcolor}%
\pgftext[x=4.408262in,y=5.281116in,left,base]{\color{textcolor}{\rmfamily\fontsize{10.000000}{12.000000}\selectfont\catcode`\^=\active\def^{\ifmmode\sp\else\^{}\fi}\catcode`\%=\active\def%{\%}1.31}}%
\end{pgfscope}%
\begin{pgfscope}%
\definecolor{textcolor}{rgb}{0.000000,0.000000,0.000000}%
\pgfsetstrokecolor{textcolor}%
\pgfsetfillcolor{textcolor}%
\pgftext[x=5.674281in,y=0.683524in,left,base]{\color{textcolor}{\rmfamily\fontsize{10.000000}{12.000000}\selectfont\catcode`\^=\active\def^{\ifmmode\sp\else\^{}\fi}\catcode`\%=\active\def%{\%}0.75}}%
\end{pgfscope}%
\begin{pgfscope}%
\definecolor{textcolor}{rgb}{0.000000,0.000000,0.000000}%
\pgfsetstrokecolor{textcolor}%
\pgfsetfillcolor{textcolor}%
\pgftext[x=5.674281in,y=5.281116in,left,base]{\color{textcolor}{\rmfamily\fontsize{10.000000}{12.000000}\selectfont\catcode`\^=\active\def^{\ifmmode\sp\else\^{}\fi}\catcode`\%=\active\def%{\%}9.66}}%
\end{pgfscope}%
\begin{pgfscope}%
\definecolor{textcolor}{rgb}{0.000000,0.000000,0.000000}%
\pgfsetstrokecolor{textcolor}%
\pgfsetfillcolor{textcolor}%
\pgftext[x=6.940300in,y=0.683524in,left,base]{\color{textcolor}{\rmfamily\fontsize{10.000000}{12.000000}\selectfont\catcode`\^=\active\def^{\ifmmode\sp\else\^{}\fi}\catcode`\%=\active\def%{\%}0.21}}%
\end{pgfscope}%
\begin{pgfscope}%
\definecolor{textcolor}{rgb}{0.000000,0.000000,0.000000}%
\pgfsetstrokecolor{textcolor}%
\pgfsetfillcolor{textcolor}%
\pgftext[x=6.940300in,y=5.281116in,left,base]{\color{textcolor}{\rmfamily\fontsize{10.000000}{12.000000}\selectfont\catcode`\^=\active\def^{\ifmmode\sp\else\^{}\fi}\catcode`\%=\active\def%{\%}15.08}}%
\end{pgfscope}%
\begin{pgfscope}%
\definecolor{textcolor}{rgb}{0.000000,0.000000,0.000000}%
\pgfsetstrokecolor{textcolor}%
\pgfsetfillcolor{textcolor}%
\pgftext[x=8.206319in,y=0.683524in,left,base]{\color{textcolor}{\rmfamily\fontsize{10.000000}{12.000000}\selectfont\catcode`\^=\active\def^{\ifmmode\sp\else\^{}\fi}\catcode`\%=\active\def%{\%}0.05}}%
\end{pgfscope}%
\begin{pgfscope}%
\definecolor{textcolor}{rgb}{0.000000,0.000000,0.000000}%
\pgfsetstrokecolor{textcolor}%
\pgfsetfillcolor{textcolor}%
\pgftext[x=8.206319in,y=5.281116in,left,base]{\color{textcolor}{\rmfamily\fontsize{10.000000}{12.000000}\selectfont\catcode`\^=\active\def^{\ifmmode\sp\else\^{}\fi}\catcode`\%=\active\def%{\%}13.86}}%
\end{pgfscope}%
\begin{pgfscope}%
\definecolor{textcolor}{rgb}{0.000000,0.000000,0.000000}%
\pgfsetstrokecolor{textcolor}%
\pgfsetfillcolor{textcolor}%
\pgftext[x=9.472338in,y=0.683524in,left,base]{\color{textcolor}{\rmfamily\fontsize{10.000000}{12.000000}\selectfont\catcode`\^=\active\def^{\ifmmode\sp\else\^{}\fi}\catcode`\%=\active\def%{\%}0.09}}%
\end{pgfscope}%
\begin{pgfscope}%
\definecolor{textcolor}{rgb}{0.000000,0.000000,0.000000}%
\pgfsetstrokecolor{textcolor}%
\pgfsetfillcolor{textcolor}%
\pgftext[x=9.472338in,y=5.281116in,left,base]{\color{textcolor}{\rmfamily\fontsize{10.000000}{12.000000}\selectfont\catcode`\^=\active\def^{\ifmmode\sp\else\^{}\fi}\catcode`\%=\active\def%{\%}14.47}}%
\end{pgfscope}%
\begin{pgfscope}%
\definecolor{textcolor}{rgb}{0.000000,0.000000,0.000000}%
\pgfsetstrokecolor{textcolor}%
\pgfsetfillcolor{textcolor}%
\pgftext[x=10.738356in,y=0.683524in,left,base]{\color{textcolor}{\rmfamily\fontsize{10.000000}{12.000000}\selectfont\catcode`\^=\active\def^{\ifmmode\sp\else\^{}\fi}\catcode`\%=\active\def%{\%}0.00}}%
\end{pgfscope}%
\begin{pgfscope}%
\definecolor{textcolor}{rgb}{0.000000,0.000000,0.000000}%
\pgfsetstrokecolor{textcolor}%
\pgfsetfillcolor{textcolor}%
\pgftext[x=10.738356in,y=5.281116in,left,base]{\color{textcolor}{\rmfamily\fontsize{10.000000}{12.000000}\selectfont\catcode`\^=\active\def^{\ifmmode\sp\else\^{}\fi}\catcode`\%=\active\def%{\%}14.57}}%
\end{pgfscope}%
\begin{pgfscope}%
\definecolor{textcolor}{rgb}{0.000000,0.000000,0.000000}%
\pgfsetstrokecolor{textcolor}%
\pgfsetfillcolor{textcolor}%
\pgftext[x=12.004375in,y=0.683524in,left,base]{\color{textcolor}{\rmfamily\fontsize{10.000000}{12.000000}\selectfont\catcode`\^=\active\def^{\ifmmode\sp\else\^{}\fi}\catcode`\%=\active\def%{\%}0.00}}%
\end{pgfscope}%
\begin{pgfscope}%
\definecolor{textcolor}{rgb}{0.000000,0.000000,0.000000}%
\pgfsetstrokecolor{textcolor}%
\pgfsetfillcolor{textcolor}%
\pgftext[x=12.004375in,y=5.281116in,left,base]{\color{textcolor}{\rmfamily\fontsize{10.000000}{12.000000}\selectfont\catcode`\^=\active\def^{\ifmmode\sp\else\^{}\fi}\catcode`\%=\active\def%{\%}70.93}}%
\end{pgfscope}%
\begin{pgfscope}%
\definecolor{textcolor}{rgb}{0.000000,0.000000,0.000000}%
\pgfsetstrokecolor{textcolor}%
\pgfsetfillcolor{textcolor}%
\pgftext[x=6.408572in,y=5.599962in,,base]{\color{textcolor}{\rmfamily\fontsize{20.000000}{24.000000}\selectfont\catcode`\^=\active\def^{\ifmmode\sp\else\^{}\fi}\catcode`\%=\active\def%{\%}Design Space}}%
\end{pgfscope}%
\begin{pgfscope}%
\pgfsetbuttcap%
\pgfsetmiterjoin%
\definecolor{currentfill}{rgb}{1.000000,1.000000,1.000000}%
\pgfsetfillcolor{currentfill}%
\pgfsetfillopacity{0.800000}%
\pgfsetlinewidth{1.003750pt}%
\definecolor{currentstroke}{rgb}{0.800000,0.800000,0.800000}%
\pgfsetstrokecolor{currentstroke}%
\pgfsetstrokeopacity{0.800000}%
\pgfsetdash{}{0pt}%
\pgfpathmoveto{\pgfqpoint{0.255556in}{3.707678in}}%
\pgfpathlineto{\pgfqpoint{2.791918in}{3.707678in}}%
\pgfpathquadraticcurveto{\pgfqpoint{2.836363in}{3.707678in}}{\pgfqpoint{2.836363in}{3.752123in}}%
\pgfpathlineto{\pgfqpoint{2.836363in}{5.027739in}}%
\pgfpathquadraticcurveto{\pgfqpoint{2.836363in}{5.072184in}}{\pgfqpoint{2.791918in}{5.072184in}}%
\pgfpathlineto{\pgfqpoint{0.255556in}{5.072184in}}%
\pgfpathquadraticcurveto{\pgfqpoint{0.211111in}{5.072184in}}{\pgfqpoint{0.211111in}{5.027739in}}%
\pgfpathlineto{\pgfqpoint{0.211111in}{3.752123in}}%
\pgfpathquadraticcurveto{\pgfqpoint{0.211111in}{3.707678in}}{\pgfqpoint{0.255556in}{3.707678in}}%
\pgfpathlineto{\pgfqpoint{0.255556in}{3.707678in}}%
\pgfpathclose%
\pgfusepath{stroke,fill}%
\end{pgfscope}%
\begin{pgfscope}%
\pgfsetrectcap%
\pgfsetroundjoin%
\pgfsetlinewidth{5.018750pt}%
\definecolor{currentstroke}{rgb}{0.172549,0.627451,0.172549}%
\pgfsetstrokecolor{currentstroke}%
\pgfsetdash{}{0pt}%
\pgfpathmoveto{\pgfqpoint{0.300000in}{4.894406in}}%
\pgfpathlineto{\pgfqpoint{0.522222in}{4.894406in}}%
\pgfpathlineto{\pgfqpoint{0.744444in}{4.894406in}}%
\pgfusepath{stroke}%
\end{pgfscope}%
\begin{pgfscope}%
\definecolor{textcolor}{rgb}{0.000000,0.000000,0.000000}%
\pgfsetstrokecolor{textcolor}%
\pgfsetfillcolor{textcolor}%
\pgftext[x=0.922222in,y=4.816628in,left,base]{\color{textcolor}{\rmfamily\fontsize{16.000000}{19.200000}\selectfont\catcode`\^=\active\def^{\ifmmode\sp\else\^{}\fi}\catcode`\%=\active\def%{\%}Least CO$_2$}}%
\end{pgfscope}%
\begin{pgfscope}%
\pgfsetrectcap%
\pgfsetroundjoin%
\pgfsetlinewidth{5.018750pt}%
\definecolor{currentstroke}{rgb}{0.839216,0.152941,0.156863}%
\pgfsetstrokecolor{currentstroke}%
\pgfsetdash{}{0pt}%
\pgfpathmoveto{\pgfqpoint{0.300000in}{4.569946in}}%
\pgfpathlineto{\pgfqpoint{0.522222in}{4.569946in}}%
\pgfpathlineto{\pgfqpoint{0.744444in}{4.569946in}}%
\pgfusepath{stroke}%
\end{pgfscope}%
\begin{pgfscope}%
\definecolor{textcolor}{rgb}{0.000000,0.000000,0.000000}%
\pgfsetstrokecolor{textcolor}%
\pgfsetfillcolor{textcolor}%
\pgftext[x=0.922222in,y=4.492168in,left,base]{\color{textcolor}{\rmfamily\fontsize{16.000000}{19.200000}\selectfont\catcode`\^=\active\def^{\ifmmode\sp\else\^{}\fi}\catcode`\%=\active\def%{\%}Least Cost}}%
\end{pgfscope}%
\begin{pgfscope}%
\pgfsetrectcap%
\pgfsetroundjoin%
\pgfsetlinewidth{5.018750pt}%
\definecolor{currentstroke}{rgb}{1.000000,0.498039,0.054902}%
\pgfsetstrokecolor{currentstroke}%
\pgfsetdash{}{0pt}%
\pgfpathmoveto{\pgfqpoint{0.300000in}{4.245487in}}%
\pgfpathlineto{\pgfqpoint{0.522222in}{4.245487in}}%
\pgfpathlineto{\pgfqpoint{0.744444in}{4.245487in}}%
\pgfusepath{stroke}%
\end{pgfscope}%
\begin{pgfscope}%
\definecolor{textcolor}{rgb}{0.000000,0.000000,0.000000}%
\pgfsetstrokecolor{textcolor}%
\pgfsetfillcolor{textcolor}%
\pgftext[x=0.922222in,y=4.167709in,left,base]{\color{textcolor}{\rmfamily\fontsize{16.000000}{19.200000}\selectfont\catcode`\^=\active\def^{\ifmmode\sp\else\^{}\fi}\catcode`\%=\active\def%{\%}Least Land Use}}%
\end{pgfscope}%
\begin{pgfscope}%
\pgfsetrectcap%
\pgfsetroundjoin%
\pgfsetlinewidth{5.018750pt}%
\definecolor{currentstroke}{rgb}{0.121569,0.466667,0.705882}%
\pgfsetstrokecolor{currentstroke}%
\pgfsetdash{}{0pt}%
\pgfpathmoveto{\pgfqpoint{0.300000in}{3.921027in}}%
\pgfpathlineto{\pgfqpoint{0.522222in}{3.921027in}}%
\pgfpathlineto{\pgfqpoint{0.744444in}{3.921027in}}%
\pgfusepath{stroke}%
\end{pgfscope}%
\begin{pgfscope}%
\definecolor{textcolor}{rgb}{0.000000,0.000000,0.000000}%
\pgfsetstrokecolor{textcolor}%
\pgfsetfillcolor{textcolor}%
\pgftext[x=0.922222in,y=3.843249in,left,base]{\color{textcolor}{\rmfamily\fontsize{16.000000}{19.200000}\selectfont\catcode`\^=\active\def^{\ifmmode\sp\else\^{}\fi}\catcode`\%=\active\def%{\%}Highest Renewable}}%
\end{pgfscope}%
\end{pgfpicture}%
\makeatother%
\endgroup%
}
  \caption{The design space for a four objective problem.}
  \label{fig:4-obj-design}
\end{figure}

Figure \ref{fig:4-obj-design} shows that conventional coal and advanced coal
technologies are largely uninteresting because they make up at most 7\% and 4\%
of a solution's peak demand, respectively. The ``highest renewable'' solution
achieves its goal of reaching approximately 100\% renewable energy (by
percentage of energy produced) with a significant overbuild of wind energy and
batteries, with natural gas and a small amount of coal for reliability.
Interestingly, this solution uses no solar energy, even though solar and wind
are frequently assumed to complement each other.

Figure \ref{fig:4-obj-design-mga} extends the design space results to include
the \ac{mga} solutions. This plot indicates the design preferences for a
middling solution, but hides the relationship among energy technologies. The
most prevalent technologies in Figure \ref{fig:4-obj-design-mga} are conventional
nuclear, battery storage, and solar panels. The least prevalent technologies are
wind turbines, biomass, and ``advanced'' coal plants. The ``highest renewable'' 
solution in Figure \ref{fig:4-obj-design} shows a wind turbine capacity of nearly 71
GW, this point is absent in Figure \ref{fig:4-obj-design-mga} because the vertical
axis is truncated to enhance the visibility of other technology spreads.

\begin{figure}[h]
  \centering
  \resizebox{\columnwidth}{!}{%% Creator: Matplotlib, PGF backend
%%
%% To include the figure in your LaTeX document, write
%%   \input{<filename>.pgf}
%%
%% Make sure the required packages are loaded in your preamble
%%   \usepackage{pgf}
%%
%% Also ensure that all the required font packages are loaded; for instance,
%% the lmodern package is sometimes necessary when using math font.
%%   \usepackage{lmodern}
%%
%% Figures using additional raster images can only be included by \input if
%% they are in the same directory as the main LaTeX file. For loading figures
%% from other directories you can use the `import` package
%%   \usepackage{import}
%%
%% and then include the figures with
%%   \import{<path to file>}{<filename>.pgf}
%%
%% Matplotlib used the following preamble
%%   \def\mathdefault#1{#1}
%%   \everymath=\expandafter{\the\everymath\displaystyle}
%%   \IfFileExists{scrextend.sty}{
%%     \usepackage[fontsize=10.000000pt]{scrextend}
%%   }{
%%     \renewcommand{\normalsize}{\fontsize{10.000000}{12.000000}\selectfont}
%%     \normalsize
%%   }
%%   
%%   \makeatletter\@ifpackageloaded{underscore}{}{\usepackage[strings]{underscore}}\makeatother
%%
\begingroup%
\makeatletter%
\begin{pgfpicture}%
\pgfpathrectangle{\pgfpointorigin}{\pgfqpoint{12.826910in}{5.900000in}}%
\pgfusepath{use as bounding box, clip}%
\begin{pgfscope}%
\pgfsetbuttcap%
\pgfsetmiterjoin%
\definecolor{currentfill}{rgb}{1.000000,1.000000,1.000000}%
\pgfsetfillcolor{currentfill}%
\pgfsetlinewidth{0.000000pt}%
\definecolor{currentstroke}{rgb}{0.000000,0.000000,0.000000}%
\pgfsetstrokecolor{currentstroke}%
\pgfsetdash{}{0pt}%
\pgfpathmoveto{\pgfqpoint{0.000000in}{0.000000in}}%
\pgfpathlineto{\pgfqpoint{12.826910in}{0.000000in}}%
\pgfpathlineto{\pgfqpoint{12.826910in}{5.900000in}}%
\pgfpathlineto{\pgfqpoint{0.000000in}{5.900000in}}%
\pgfpathlineto{\pgfqpoint{0.000000in}{0.000000in}}%
\pgfpathclose%
\pgfusepath{fill}%
\end{pgfscope}%
\begin{pgfscope}%
\pgfsetbuttcap%
\pgfsetmiterjoin%
\definecolor{currentfill}{rgb}{1.000000,1.000000,1.000000}%
\pgfsetfillcolor{currentfill}%
\pgfsetlinewidth{0.000000pt}%
\definecolor{currentstroke}{rgb}{0.000000,0.000000,0.000000}%
\pgfsetstrokecolor{currentstroke}%
\pgfsetstrokeopacity{0.000000}%
\pgfsetdash{}{0pt}%
\pgfpathmoveto{\pgfqpoint{0.417359in}{0.814008in}}%
\pgfpathlineto{\pgfqpoint{12.726910in}{0.814008in}}%
\pgfpathlineto{\pgfqpoint{12.726910in}{5.536883in}}%
\pgfpathlineto{\pgfqpoint{0.417359in}{5.536883in}}%
\pgfpathlineto{\pgfqpoint{0.417359in}{0.814008in}}%
\pgfpathclose%
\pgfusepath{fill}%
\end{pgfscope}%
\begin{pgfscope}%
\pgfpathrectangle{\pgfqpoint{0.417359in}{0.814008in}}{\pgfqpoint{12.309552in}{4.722875in}}%
\pgfusepath{clip}%
\pgfsetbuttcap%
\pgfsetroundjoin%
\definecolor{currentfill}{rgb}{0.194608,0.453431,0.632843}%
\pgfsetfillcolor{currentfill}%
\pgfsetlinewidth{0.752812pt}%
\definecolor{currentstroke}{rgb}{0.240000,0.240000,0.240000}%
\pgfsetstrokecolor{currentstroke}%
\pgfsetdash{}{0pt}%
\pgfsys@defobject{currentmarker}{\pgfqpoint{0.540454in}{2.018411in}}{\pgfqpoint{1.525218in}{2.508047in}}{%
\pgfpathmoveto{\pgfqpoint{0.540454in}{2.018411in}}%
\pgfpathlineto{\pgfqpoint{1.525218in}{2.018411in}}%
\pgfpathlineto{\pgfqpoint{1.525218in}{2.508047in}}%
\pgfpathlineto{\pgfqpoint{0.540454in}{2.508047in}}%
\pgfpathlineto{\pgfqpoint{0.540454in}{2.018411in}}%
\pgfpathclose%
\pgfusepath{stroke,fill}%
}%
\begin{pgfscope}%
\pgfsys@transformshift{0.000000in}{0.000000in}%
\pgfsys@useobject{currentmarker}{}%
\end{pgfscope}%
\end{pgfscope}%
\begin{pgfscope}%
\pgfpathrectangle{\pgfqpoint{0.417359in}{0.814008in}}{\pgfqpoint{12.309552in}{4.722875in}}%
\pgfusepath{clip}%
\pgfsetbuttcap%
\pgfsetroundjoin%
\pgfsetlinewidth{1.003750pt}%
\definecolor{currentstroke}{rgb}{0.450000,0.450000,0.450000}%
\pgfsetstrokecolor{currentstroke}%
\pgfsetdash{}{0pt}%
\pgfpathmoveto{\pgfqpoint{1.032836in}{4.194191in}}%
\pgfpathcurveto{\pgfqpoint{1.042045in}{4.194191in}}{\pgfqpoint{1.050877in}{4.197850in}}{\pgfqpoint{1.057389in}{4.204361in}}%
\pgfpathcurveto{\pgfqpoint{1.063900in}{4.210872in}}{\pgfqpoint{1.067558in}{4.219705in}}{\pgfqpoint{1.067558in}{4.228913in}}%
\pgfpathcurveto{\pgfqpoint{1.067558in}{4.238122in}}{\pgfqpoint{1.063900in}{4.246954in}}{\pgfqpoint{1.057389in}{4.253466in}}%
\pgfpathcurveto{\pgfqpoint{1.050877in}{4.259977in}}{\pgfqpoint{1.042045in}{4.263635in}}{\pgfqpoint{1.032836in}{4.263635in}}%
\pgfpathcurveto{\pgfqpoint{1.023628in}{4.263635in}}{\pgfqpoint{1.014795in}{4.259977in}}{\pgfqpoint{1.008284in}{4.253466in}}%
\pgfpathcurveto{\pgfqpoint{1.001773in}{4.246954in}}{\pgfqpoint{0.998114in}{4.238122in}}{\pgfqpoint{0.998114in}{4.228913in}}%
\pgfpathcurveto{\pgfqpoint{0.998114in}{4.219705in}}{\pgfqpoint{1.001773in}{4.210872in}}{\pgfqpoint{1.008284in}{4.204361in}}%
\pgfpathcurveto{\pgfqpoint{1.014795in}{4.197850in}}{\pgfqpoint{1.023628in}{4.194191in}}{\pgfqpoint{1.032836in}{4.194191in}}%
\pgfpathlineto{\pgfqpoint{1.032836in}{4.194191in}}%
\pgfpathclose%
\pgfusepath{stroke}%
\end{pgfscope}%
\begin{pgfscope}%
\pgfpathrectangle{\pgfqpoint{0.417359in}{0.814008in}}{\pgfqpoint{12.309552in}{4.722875in}}%
\pgfusepath{clip}%
\pgfsetbuttcap%
\pgfsetroundjoin%
\pgfsetlinewidth{1.003750pt}%
\definecolor{currentstroke}{rgb}{0.450000,0.450000,0.450000}%
\pgfsetstrokecolor{currentstroke}%
\pgfsetdash{}{0pt}%
\pgfpathmoveto{\pgfqpoint{1.032836in}{4.194191in}}%
\pgfpathcurveto{\pgfqpoint{1.042045in}{4.194191in}}{\pgfqpoint{1.050877in}{4.197850in}}{\pgfqpoint{1.057389in}{4.204361in}}%
\pgfpathcurveto{\pgfqpoint{1.063900in}{4.210872in}}{\pgfqpoint{1.067558in}{4.219705in}}{\pgfqpoint{1.067558in}{4.228913in}}%
\pgfpathcurveto{\pgfqpoint{1.067558in}{4.238122in}}{\pgfqpoint{1.063900in}{4.246954in}}{\pgfqpoint{1.057389in}{4.253466in}}%
\pgfpathcurveto{\pgfqpoint{1.050877in}{4.259977in}}{\pgfqpoint{1.042045in}{4.263635in}}{\pgfqpoint{1.032836in}{4.263635in}}%
\pgfpathcurveto{\pgfqpoint{1.023628in}{4.263635in}}{\pgfqpoint{1.014795in}{4.259977in}}{\pgfqpoint{1.008284in}{4.253466in}}%
\pgfpathcurveto{\pgfqpoint{1.001773in}{4.246954in}}{\pgfqpoint{0.998114in}{4.238122in}}{\pgfqpoint{0.998114in}{4.228913in}}%
\pgfpathcurveto{\pgfqpoint{0.998114in}{4.219705in}}{\pgfqpoint{1.001773in}{4.210872in}}{\pgfqpoint{1.008284in}{4.204361in}}%
\pgfpathcurveto{\pgfqpoint{1.014795in}{4.197850in}}{\pgfqpoint{1.023628in}{4.194191in}}{\pgfqpoint{1.032836in}{4.194191in}}%
\pgfpathlineto{\pgfqpoint{1.032836in}{4.194191in}}%
\pgfpathclose%
\pgfusepath{stroke}%
\end{pgfscope}%
\begin{pgfscope}%
\pgfpathrectangle{\pgfqpoint{0.417359in}{0.814008in}}{\pgfqpoint{12.309552in}{4.722875in}}%
\pgfusepath{clip}%
\pgfsetbuttcap%
\pgfsetroundjoin%
\pgfsetlinewidth{1.003750pt}%
\definecolor{currentstroke}{rgb}{0.450000,0.450000,0.450000}%
\pgfsetstrokecolor{currentstroke}%
\pgfsetdash{}{0pt}%
\pgfpathmoveto{\pgfqpoint{1.032836in}{0.842342in}}%
\pgfpathcurveto{\pgfqpoint{1.042045in}{0.842342in}}{\pgfqpoint{1.050877in}{0.846000in}}{\pgfqpoint{1.057389in}{0.852512in}}%
\pgfpathcurveto{\pgfqpoint{1.063900in}{0.859023in}}{\pgfqpoint{1.067558in}{0.867856in}}{\pgfqpoint{1.067558in}{0.877064in}}%
\pgfpathcurveto{\pgfqpoint{1.067558in}{0.886273in}}{\pgfqpoint{1.063900in}{0.895105in}}{\pgfqpoint{1.057389in}{0.901616in}}%
\pgfpathcurveto{\pgfqpoint{1.050877in}{0.908128in}}{\pgfqpoint{1.042045in}{0.911786in}}{\pgfqpoint{1.032836in}{0.911786in}}%
\pgfpathcurveto{\pgfqpoint{1.023628in}{0.911786in}}{\pgfqpoint{1.014795in}{0.908128in}}{\pgfqpoint{1.008284in}{0.901616in}}%
\pgfpathcurveto{\pgfqpoint{1.001773in}{0.895105in}}{\pgfqpoint{0.998114in}{0.886273in}}{\pgfqpoint{0.998114in}{0.877064in}}%
\pgfpathcurveto{\pgfqpoint{0.998114in}{0.867856in}}{\pgfqpoint{1.001773in}{0.859023in}}{\pgfqpoint{1.008284in}{0.852512in}}%
\pgfpathcurveto{\pgfqpoint{1.014795in}{0.846000in}}{\pgfqpoint{1.023628in}{0.842342in}}{\pgfqpoint{1.032836in}{0.842342in}}%
\pgfpathlineto{\pgfqpoint{1.032836in}{0.842342in}}%
\pgfpathclose%
\pgfusepath{stroke}%
\end{pgfscope}%
\begin{pgfscope}%
\pgfpathrectangle{\pgfqpoint{0.417359in}{0.814008in}}{\pgfqpoint{12.309552in}{4.722875in}}%
\pgfusepath{clip}%
\pgfsetbuttcap%
\pgfsetroundjoin%
\pgfsetlinewidth{1.003750pt}%
\definecolor{currentstroke}{rgb}{0.450000,0.450000,0.450000}%
\pgfsetstrokecolor{currentstroke}%
\pgfsetdash{}{0pt}%
\pgfpathmoveto{\pgfqpoint{1.032836in}{4.297936in}}%
\pgfpathcurveto{\pgfqpoint{1.042045in}{4.297936in}}{\pgfqpoint{1.050877in}{4.301594in}}{\pgfqpoint{1.057389in}{4.308106in}}%
\pgfpathcurveto{\pgfqpoint{1.063900in}{4.314617in}}{\pgfqpoint{1.067558in}{4.323449in}}{\pgfqpoint{1.067558in}{4.332658in}}%
\pgfpathcurveto{\pgfqpoint{1.067558in}{4.341866in}}{\pgfqpoint{1.063900in}{4.350699in}}{\pgfqpoint{1.057389in}{4.357210in}}%
\pgfpathcurveto{\pgfqpoint{1.050877in}{4.363722in}}{\pgfqpoint{1.042045in}{4.367380in}}{\pgfqpoint{1.032836in}{4.367380in}}%
\pgfpathcurveto{\pgfqpoint{1.023628in}{4.367380in}}{\pgfqpoint{1.014795in}{4.363722in}}{\pgfqpoint{1.008284in}{4.357210in}}%
\pgfpathcurveto{\pgfqpoint{1.001773in}{4.350699in}}{\pgfqpoint{0.998114in}{4.341866in}}{\pgfqpoint{0.998114in}{4.332658in}}%
\pgfpathcurveto{\pgfqpoint{0.998114in}{4.323449in}}{\pgfqpoint{1.001773in}{4.314617in}}{\pgfqpoint{1.008284in}{4.308106in}}%
\pgfpathcurveto{\pgfqpoint{1.014795in}{4.301594in}}{\pgfqpoint{1.023628in}{4.297936in}}{\pgfqpoint{1.032836in}{4.297936in}}%
\pgfpathlineto{\pgfqpoint{1.032836in}{4.297936in}}%
\pgfpathclose%
\pgfusepath{stroke}%
\end{pgfscope}%
\begin{pgfscope}%
\pgfpathrectangle{\pgfqpoint{0.417359in}{0.814008in}}{\pgfqpoint{12.309552in}{4.722875in}}%
\pgfusepath{clip}%
\pgfsetbuttcap%
\pgfsetroundjoin%
\pgfsetlinewidth{1.003750pt}%
\definecolor{currentstroke}{rgb}{0.450000,0.450000,0.450000}%
\pgfsetstrokecolor{currentstroke}%
\pgfsetdash{}{0pt}%
\pgfpathmoveto{\pgfqpoint{1.032836in}{1.955821in}}%
\pgfpathcurveto{\pgfqpoint{1.042045in}{1.955821in}}{\pgfqpoint{1.050877in}{1.959480in}}{\pgfqpoint{1.057389in}{1.965991in}}%
\pgfpathcurveto{\pgfqpoint{1.063900in}{1.972503in}}{\pgfqpoint{1.067558in}{1.981335in}}{\pgfqpoint{1.067558in}{1.990543in}}%
\pgfpathcurveto{\pgfqpoint{1.067558in}{1.999752in}}{\pgfqpoint{1.063900in}{2.008584in}}{\pgfqpoint{1.057389in}{2.015096in}}%
\pgfpathcurveto{\pgfqpoint{1.050877in}{2.021607in}}{\pgfqpoint{1.042045in}{2.025266in}}{\pgfqpoint{1.032836in}{2.025266in}}%
\pgfpathcurveto{\pgfqpoint{1.023628in}{2.025266in}}{\pgfqpoint{1.014795in}{2.021607in}}{\pgfqpoint{1.008284in}{2.015096in}}%
\pgfpathcurveto{\pgfqpoint{1.001773in}{2.008584in}}{\pgfqpoint{0.998114in}{1.999752in}}{\pgfqpoint{0.998114in}{1.990543in}}%
\pgfpathcurveto{\pgfqpoint{0.998114in}{1.981335in}}{\pgfqpoint{1.001773in}{1.972503in}}{\pgfqpoint{1.008284in}{1.965991in}}%
\pgfpathcurveto{\pgfqpoint{1.014795in}{1.959480in}}{\pgfqpoint{1.023628in}{1.955821in}}{\pgfqpoint{1.032836in}{1.955821in}}%
\pgfpathlineto{\pgfqpoint{1.032836in}{1.955821in}}%
\pgfpathclose%
\pgfusepath{stroke}%
\end{pgfscope}%
\begin{pgfscope}%
\pgfpathrectangle{\pgfqpoint{0.417359in}{0.814008in}}{\pgfqpoint{12.309552in}{4.722875in}}%
\pgfusepath{clip}%
\pgfsetbuttcap%
\pgfsetroundjoin%
\pgfsetlinewidth{1.003750pt}%
\definecolor{currentstroke}{rgb}{0.450000,0.450000,0.450000}%
\pgfsetstrokecolor{currentstroke}%
\pgfsetdash{}{0pt}%
\pgfpathmoveto{\pgfqpoint{1.032836in}{1.936024in}}%
\pgfpathcurveto{\pgfqpoint{1.042045in}{1.936024in}}{\pgfqpoint{1.050877in}{1.939683in}}{\pgfqpoint{1.057389in}{1.946194in}}%
\pgfpathcurveto{\pgfqpoint{1.063900in}{1.952705in}}{\pgfqpoint{1.067558in}{1.961538in}}{\pgfqpoint{1.067558in}{1.970746in}}%
\pgfpathcurveto{\pgfqpoint{1.067558in}{1.979955in}}{\pgfqpoint{1.063900in}{1.988787in}}{\pgfqpoint{1.057389in}{1.995299in}}%
\pgfpathcurveto{\pgfqpoint{1.050877in}{2.001810in}}{\pgfqpoint{1.042045in}{2.005468in}}{\pgfqpoint{1.032836in}{2.005468in}}%
\pgfpathcurveto{\pgfqpoint{1.023628in}{2.005468in}}{\pgfqpoint{1.014795in}{2.001810in}}{\pgfqpoint{1.008284in}{1.995299in}}%
\pgfpathcurveto{\pgfqpoint{1.001773in}{1.988787in}}{\pgfqpoint{0.998114in}{1.979955in}}{\pgfqpoint{0.998114in}{1.970746in}}%
\pgfpathcurveto{\pgfqpoint{0.998114in}{1.961538in}}{\pgfqpoint{1.001773in}{1.952705in}}{\pgfqpoint{1.008284in}{1.946194in}}%
\pgfpathcurveto{\pgfqpoint{1.014795in}{1.939683in}}{\pgfqpoint{1.023628in}{1.936024in}}{\pgfqpoint{1.032836in}{1.936024in}}%
\pgfpathlineto{\pgfqpoint{1.032836in}{1.936024in}}%
\pgfpathclose%
\pgfusepath{stroke}%
\end{pgfscope}%
\begin{pgfscope}%
\pgfpathrectangle{\pgfqpoint{0.417359in}{0.814008in}}{\pgfqpoint{12.309552in}{4.722875in}}%
\pgfusepath{clip}%
\pgfsetbuttcap%
\pgfsetroundjoin%
\pgfsetlinewidth{1.003750pt}%
\definecolor{currentstroke}{rgb}{0.450000,0.450000,0.450000}%
\pgfsetstrokecolor{currentstroke}%
\pgfsetdash{}{0pt}%
\pgfpathmoveto{\pgfqpoint{1.032836in}{2.488945in}}%
\pgfpathcurveto{\pgfqpoint{1.042045in}{2.488945in}}{\pgfqpoint{1.050877in}{2.492604in}}{\pgfqpoint{1.057389in}{2.499115in}}%
\pgfpathcurveto{\pgfqpoint{1.063900in}{2.505627in}}{\pgfqpoint{1.067558in}{2.514459in}}{\pgfqpoint{1.067558in}{2.523668in}}%
\pgfpathcurveto{\pgfqpoint{1.067558in}{2.532876in}}{\pgfqpoint{1.063900in}{2.541708in}}{\pgfqpoint{1.057389in}{2.548220in}}%
\pgfpathcurveto{\pgfqpoint{1.050877in}{2.554731in}}{\pgfqpoint{1.042045in}{2.558390in}}{\pgfqpoint{1.032836in}{2.558390in}}%
\pgfpathcurveto{\pgfqpoint{1.023628in}{2.558390in}}{\pgfqpoint{1.014795in}{2.554731in}}{\pgfqpoint{1.008284in}{2.548220in}}%
\pgfpathcurveto{\pgfqpoint{1.001773in}{2.541708in}}{\pgfqpoint{0.998114in}{2.532876in}}{\pgfqpoint{0.998114in}{2.523668in}}%
\pgfpathcurveto{\pgfqpoint{0.998114in}{2.514459in}}{\pgfqpoint{1.001773in}{2.505627in}}{\pgfqpoint{1.008284in}{2.499115in}}%
\pgfpathcurveto{\pgfqpoint{1.014795in}{2.492604in}}{\pgfqpoint{1.023628in}{2.488945in}}{\pgfqpoint{1.032836in}{2.488945in}}%
\pgfpathlineto{\pgfqpoint{1.032836in}{2.488945in}}%
\pgfpathclose%
\pgfusepath{stroke}%
\end{pgfscope}%
\begin{pgfscope}%
\pgfpathrectangle{\pgfqpoint{0.417359in}{0.814008in}}{\pgfqpoint{12.309552in}{4.722875in}}%
\pgfusepath{clip}%
\pgfsetbuttcap%
\pgfsetroundjoin%
\pgfsetlinewidth{1.003750pt}%
\definecolor{currentstroke}{rgb}{0.450000,0.450000,0.450000}%
\pgfsetstrokecolor{currentstroke}%
\pgfsetdash{}{0pt}%
\pgfpathmoveto{\pgfqpoint{1.032836in}{2.488945in}}%
\pgfpathcurveto{\pgfqpoint{1.042045in}{2.488945in}}{\pgfqpoint{1.050877in}{2.492604in}}{\pgfqpoint{1.057389in}{2.499115in}}%
\pgfpathcurveto{\pgfqpoint{1.063900in}{2.505627in}}{\pgfqpoint{1.067558in}{2.514459in}}{\pgfqpoint{1.067558in}{2.523668in}}%
\pgfpathcurveto{\pgfqpoint{1.067558in}{2.532876in}}{\pgfqpoint{1.063900in}{2.541708in}}{\pgfqpoint{1.057389in}{2.548220in}}%
\pgfpathcurveto{\pgfqpoint{1.050877in}{2.554731in}}{\pgfqpoint{1.042045in}{2.558390in}}{\pgfqpoint{1.032836in}{2.558390in}}%
\pgfpathcurveto{\pgfqpoint{1.023628in}{2.558390in}}{\pgfqpoint{1.014795in}{2.554731in}}{\pgfqpoint{1.008284in}{2.548220in}}%
\pgfpathcurveto{\pgfqpoint{1.001773in}{2.541708in}}{\pgfqpoint{0.998114in}{2.532876in}}{\pgfqpoint{0.998114in}{2.523668in}}%
\pgfpathcurveto{\pgfqpoint{0.998114in}{2.514459in}}{\pgfqpoint{1.001773in}{2.505627in}}{\pgfqpoint{1.008284in}{2.499115in}}%
\pgfpathcurveto{\pgfqpoint{1.014795in}{2.492604in}}{\pgfqpoint{1.023628in}{2.488945in}}{\pgfqpoint{1.032836in}{2.488945in}}%
\pgfpathlineto{\pgfqpoint{1.032836in}{2.488945in}}%
\pgfpathclose%
\pgfusepath{stroke}%
\end{pgfscope}%
\begin{pgfscope}%
\pgfpathrectangle{\pgfqpoint{0.417359in}{0.814008in}}{\pgfqpoint{12.309552in}{4.722875in}}%
\pgfusepath{clip}%
\pgfsetbuttcap%
\pgfsetroundjoin%
\pgfsetlinewidth{1.003750pt}%
\definecolor{currentstroke}{rgb}{0.450000,0.450000,0.450000}%
\pgfsetstrokecolor{currentstroke}%
\pgfsetdash{}{0pt}%
\pgfpathmoveto{\pgfqpoint{1.032836in}{3.821011in}}%
\pgfpathcurveto{\pgfqpoint{1.042045in}{3.821011in}}{\pgfqpoint{1.050877in}{3.824670in}}{\pgfqpoint{1.057389in}{3.831181in}}%
\pgfpathcurveto{\pgfqpoint{1.063900in}{3.837692in}}{\pgfqpoint{1.067558in}{3.846525in}}{\pgfqpoint{1.067558in}{3.855733in}}%
\pgfpathcurveto{\pgfqpoint{1.067558in}{3.864942in}}{\pgfqpoint{1.063900in}{3.873774in}}{\pgfqpoint{1.057389in}{3.880286in}}%
\pgfpathcurveto{\pgfqpoint{1.050877in}{3.886797in}}{\pgfqpoint{1.042045in}{3.890456in}}{\pgfqpoint{1.032836in}{3.890456in}}%
\pgfpathcurveto{\pgfqpoint{1.023628in}{3.890456in}}{\pgfqpoint{1.014795in}{3.886797in}}{\pgfqpoint{1.008284in}{3.880286in}}%
\pgfpathcurveto{\pgfqpoint{1.001773in}{3.873774in}}{\pgfqpoint{0.998114in}{3.864942in}}{\pgfqpoint{0.998114in}{3.855733in}}%
\pgfpathcurveto{\pgfqpoint{0.998114in}{3.846525in}}{\pgfqpoint{1.001773in}{3.837692in}}{\pgfqpoint{1.008284in}{3.831181in}}%
\pgfpathcurveto{\pgfqpoint{1.014795in}{3.824670in}}{\pgfqpoint{1.023628in}{3.821011in}}{\pgfqpoint{1.032836in}{3.821011in}}%
\pgfpathlineto{\pgfqpoint{1.032836in}{3.821011in}}%
\pgfpathclose%
\pgfusepath{stroke}%
\end{pgfscope}%
\begin{pgfscope}%
\pgfpathrectangle{\pgfqpoint{0.417359in}{0.814008in}}{\pgfqpoint{12.309552in}{4.722875in}}%
\pgfusepath{clip}%
\pgfsetbuttcap%
\pgfsetroundjoin%
\pgfsetlinewidth{1.003750pt}%
\definecolor{currentstroke}{rgb}{0.450000,0.450000,0.450000}%
\pgfsetstrokecolor{currentstroke}%
\pgfsetdash{}{0pt}%
\pgfpathmoveto{\pgfqpoint{1.032836in}{0.959924in}}%
\pgfpathcurveto{\pgfqpoint{1.042045in}{0.959924in}}{\pgfqpoint{1.050877in}{0.963583in}}{\pgfqpoint{1.057389in}{0.970094in}}%
\pgfpathcurveto{\pgfqpoint{1.063900in}{0.976605in}}{\pgfqpoint{1.067558in}{0.985438in}}{\pgfqpoint{1.067558in}{0.994646in}}%
\pgfpathcurveto{\pgfqpoint{1.067558in}{1.003855in}}{\pgfqpoint{1.063900in}{1.012687in}}{\pgfqpoint{1.057389in}{1.019199in}}%
\pgfpathcurveto{\pgfqpoint{1.050877in}{1.025710in}}{\pgfqpoint{1.042045in}{1.029369in}}{\pgfqpoint{1.032836in}{1.029369in}}%
\pgfpathcurveto{\pgfqpoint{1.023628in}{1.029369in}}{\pgfqpoint{1.014795in}{1.025710in}}{\pgfqpoint{1.008284in}{1.019199in}}%
\pgfpathcurveto{\pgfqpoint{1.001773in}{1.012687in}}{\pgfqpoint{0.998114in}{1.003855in}}{\pgfqpoint{0.998114in}{0.994646in}}%
\pgfpathcurveto{\pgfqpoint{0.998114in}{0.985438in}}{\pgfqpoint{1.001773in}{0.976605in}}{\pgfqpoint{1.008284in}{0.970094in}}%
\pgfpathcurveto{\pgfqpoint{1.014795in}{0.963583in}}{\pgfqpoint{1.023628in}{0.959924in}}{\pgfqpoint{1.032836in}{0.959924in}}%
\pgfpathlineto{\pgfqpoint{1.032836in}{0.959924in}}%
\pgfpathclose%
\pgfusepath{stroke}%
\end{pgfscope}%
\begin{pgfscope}%
\pgfpathrectangle{\pgfqpoint{0.417359in}{0.814008in}}{\pgfqpoint{12.309552in}{4.722875in}}%
\pgfusepath{clip}%
\pgfsetbuttcap%
\pgfsetroundjoin%
\pgfsetlinewidth{1.003750pt}%
\definecolor{currentstroke}{rgb}{0.450000,0.450000,0.450000}%
\pgfsetstrokecolor{currentstroke}%
\pgfsetdash{}{0pt}%
\pgfpathmoveto{\pgfqpoint{1.032836in}{1.860611in}}%
\pgfpathcurveto{\pgfqpoint{1.042045in}{1.860611in}}{\pgfqpoint{1.050877in}{1.864269in}}{\pgfqpoint{1.057389in}{1.870781in}}%
\pgfpathcurveto{\pgfqpoint{1.063900in}{1.877292in}}{\pgfqpoint{1.067558in}{1.886125in}}{\pgfqpoint{1.067558in}{1.895333in}}%
\pgfpathcurveto{\pgfqpoint{1.067558in}{1.904541in}}{\pgfqpoint{1.063900in}{1.913374in}}{\pgfqpoint{1.057389in}{1.919885in}}%
\pgfpathcurveto{\pgfqpoint{1.050877in}{1.926397in}}{\pgfqpoint{1.042045in}{1.930055in}}{\pgfqpoint{1.032836in}{1.930055in}}%
\pgfpathcurveto{\pgfqpoint{1.023628in}{1.930055in}}{\pgfqpoint{1.014795in}{1.926397in}}{\pgfqpoint{1.008284in}{1.919885in}}%
\pgfpathcurveto{\pgfqpoint{1.001773in}{1.913374in}}{\pgfqpoint{0.998114in}{1.904541in}}{\pgfqpoint{0.998114in}{1.895333in}}%
\pgfpathcurveto{\pgfqpoint{0.998114in}{1.886125in}}{\pgfqpoint{1.001773in}{1.877292in}}{\pgfqpoint{1.008284in}{1.870781in}}%
\pgfpathcurveto{\pgfqpoint{1.014795in}{1.864269in}}{\pgfqpoint{1.023628in}{1.860611in}}{\pgfqpoint{1.032836in}{1.860611in}}%
\pgfpathlineto{\pgfqpoint{1.032836in}{1.860611in}}%
\pgfpathclose%
\pgfusepath{stroke}%
\end{pgfscope}%
\begin{pgfscope}%
\pgfpathrectangle{\pgfqpoint{0.417359in}{0.814008in}}{\pgfqpoint{12.309552in}{4.722875in}}%
\pgfusepath{clip}%
\pgfsetbuttcap%
\pgfsetroundjoin%
\pgfsetlinewidth{1.003750pt}%
\definecolor{currentstroke}{rgb}{0.450000,0.450000,0.450000}%
\pgfsetstrokecolor{currentstroke}%
\pgfsetdash{}{0pt}%
\pgfpathmoveto{\pgfqpoint{1.032836in}{0.779715in}}%
\pgfpathcurveto{\pgfqpoint{1.042045in}{0.779715in}}{\pgfqpoint{1.050877in}{0.783374in}}{\pgfqpoint{1.057389in}{0.789885in}}%
\pgfpathcurveto{\pgfqpoint{1.063900in}{0.796397in}}{\pgfqpoint{1.067558in}{0.805229in}}{\pgfqpoint{1.067558in}{0.814438in}}%
\pgfpathcurveto{\pgfqpoint{1.067558in}{0.823646in}}{\pgfqpoint{1.063900in}{0.832479in}}{\pgfqpoint{1.057389in}{0.838990in}}%
\pgfpathcurveto{\pgfqpoint{1.050877in}{0.845501in}}{\pgfqpoint{1.042045in}{0.849160in}}{\pgfqpoint{1.032836in}{0.849160in}}%
\pgfpathcurveto{\pgfqpoint{1.023628in}{0.849160in}}{\pgfqpoint{1.014795in}{0.845501in}}{\pgfqpoint{1.008284in}{0.838990in}}%
\pgfpathcurveto{\pgfqpoint{1.001773in}{0.832479in}}{\pgfqpoint{0.998114in}{0.823646in}}{\pgfqpoint{0.998114in}{0.814438in}}%
\pgfpathcurveto{\pgfqpoint{0.998114in}{0.805229in}}{\pgfqpoint{1.001773in}{0.796397in}}{\pgfqpoint{1.008284in}{0.789885in}}%
\pgfpathcurveto{\pgfqpoint{1.014795in}{0.783374in}}{\pgfqpoint{1.023628in}{0.779715in}}{\pgfqpoint{1.032836in}{0.779715in}}%
\pgfusepath{stroke}%
\end{pgfscope}%
\begin{pgfscope}%
\pgfpathrectangle{\pgfqpoint{0.417359in}{0.814008in}}{\pgfqpoint{12.309552in}{4.722875in}}%
\pgfusepath{clip}%
\pgfsetbuttcap%
\pgfsetroundjoin%
\pgfsetlinewidth{1.003750pt}%
\definecolor{currentstroke}{rgb}{0.450000,0.450000,0.450000}%
\pgfsetstrokecolor{currentstroke}%
\pgfsetdash{}{0pt}%
\pgfpathmoveto{\pgfqpoint{1.032836in}{0.882590in}}%
\pgfpathcurveto{\pgfqpoint{1.042045in}{0.882590in}}{\pgfqpoint{1.050877in}{0.886248in}}{\pgfqpoint{1.057389in}{0.892759in}}%
\pgfpathcurveto{\pgfqpoint{1.063900in}{0.899271in}}{\pgfqpoint{1.067558in}{0.908103in}}{\pgfqpoint{1.067558in}{0.917312in}}%
\pgfpathcurveto{\pgfqpoint{1.067558in}{0.926520in}}{\pgfqpoint{1.063900in}{0.935353in}}{\pgfqpoint{1.057389in}{0.941864in}}%
\pgfpathcurveto{\pgfqpoint{1.050877in}{0.948375in}}{\pgfqpoint{1.042045in}{0.952034in}}{\pgfqpoint{1.032836in}{0.952034in}}%
\pgfpathcurveto{\pgfqpoint{1.023628in}{0.952034in}}{\pgfqpoint{1.014795in}{0.948375in}}{\pgfqpoint{1.008284in}{0.941864in}}%
\pgfpathcurveto{\pgfqpoint{1.001773in}{0.935353in}}{\pgfqpoint{0.998114in}{0.926520in}}{\pgfqpoint{0.998114in}{0.917312in}}%
\pgfpathcurveto{\pgfqpoint{0.998114in}{0.908103in}}{\pgfqpoint{1.001773in}{0.899271in}}{\pgfqpoint{1.008284in}{0.892759in}}%
\pgfpathcurveto{\pgfqpoint{1.014795in}{0.886248in}}{\pgfqpoint{1.023628in}{0.882590in}}{\pgfqpoint{1.032836in}{0.882590in}}%
\pgfpathlineto{\pgfqpoint{1.032836in}{0.882590in}}%
\pgfpathclose%
\pgfusepath{stroke}%
\end{pgfscope}%
\begin{pgfscope}%
\pgfpathrectangle{\pgfqpoint{0.417359in}{0.814008in}}{\pgfqpoint{12.309552in}{4.722875in}}%
\pgfusepath{clip}%
\pgfsetbuttcap%
\pgfsetroundjoin%
\pgfsetlinewidth{1.003750pt}%
\definecolor{currentstroke}{rgb}{0.450000,0.450000,0.450000}%
\pgfsetstrokecolor{currentstroke}%
\pgfsetdash{}{0pt}%
\pgfpathmoveto{\pgfqpoint{1.032836in}{4.937406in}}%
\pgfpathcurveto{\pgfqpoint{1.042045in}{4.937406in}}{\pgfqpoint{1.050877in}{4.941065in}}{\pgfqpoint{1.057389in}{4.947576in}}%
\pgfpathcurveto{\pgfqpoint{1.063900in}{4.954087in}}{\pgfqpoint{1.067558in}{4.962920in}}{\pgfqpoint{1.067558in}{4.972128in}}%
\pgfpathcurveto{\pgfqpoint{1.067558in}{4.981337in}}{\pgfqpoint{1.063900in}{4.990169in}}{\pgfqpoint{1.057389in}{4.996681in}}%
\pgfpathcurveto{\pgfqpoint{1.050877in}{5.003192in}}{\pgfqpoint{1.042045in}{5.006851in}}{\pgfqpoint{1.032836in}{5.006851in}}%
\pgfpathcurveto{\pgfqpoint{1.023628in}{5.006851in}}{\pgfqpoint{1.014795in}{5.003192in}}{\pgfqpoint{1.008284in}{4.996681in}}%
\pgfpathcurveto{\pgfqpoint{1.001773in}{4.990169in}}{\pgfqpoint{0.998114in}{4.981337in}}{\pgfqpoint{0.998114in}{4.972128in}}%
\pgfpathcurveto{\pgfqpoint{0.998114in}{4.962920in}}{\pgfqpoint{1.001773in}{4.954087in}}{\pgfqpoint{1.008284in}{4.947576in}}%
\pgfpathcurveto{\pgfqpoint{1.014795in}{4.941065in}}{\pgfqpoint{1.023628in}{4.937406in}}{\pgfqpoint{1.032836in}{4.937406in}}%
\pgfpathlineto{\pgfqpoint{1.032836in}{4.937406in}}%
\pgfpathclose%
\pgfusepath{stroke}%
\end{pgfscope}%
\begin{pgfscope}%
\pgfpathrectangle{\pgfqpoint{0.417359in}{0.814008in}}{\pgfqpoint{12.309552in}{4.722875in}}%
\pgfusepath{clip}%
\pgfsetbuttcap%
\pgfsetroundjoin%
\definecolor{currentfill}{rgb}{0.881863,0.505392,0.173039}%
\pgfsetfillcolor{currentfill}%
\pgfsetlinewidth{0.752812pt}%
\definecolor{currentstroke}{rgb}{0.240000,0.240000,0.240000}%
\pgfsetstrokecolor{currentstroke}%
\pgfsetdash{}{0pt}%
\pgfsys@defobject{currentmarker}{\pgfqpoint{1.771409in}{0.817609in}}{\pgfqpoint{2.756173in}{0.960462in}}{%
\pgfpathmoveto{\pgfqpoint{1.771409in}{0.817609in}}%
\pgfpathlineto{\pgfqpoint{2.756173in}{0.817609in}}%
\pgfpathlineto{\pgfqpoint{2.756173in}{0.960462in}}%
\pgfpathlineto{\pgfqpoint{1.771409in}{0.960462in}}%
\pgfpathlineto{\pgfqpoint{1.771409in}{0.817609in}}%
\pgfpathclose%
\pgfusepath{stroke,fill}%
}%
\begin{pgfscope}%
\pgfsys@transformshift{0.000000in}{0.000000in}%
\pgfsys@useobject{currentmarker}{}%
\end{pgfscope}%
\end{pgfscope}%
\begin{pgfscope}%
\pgfpathrectangle{\pgfqpoint{0.417359in}{0.814008in}}{\pgfqpoint{12.309552in}{4.722875in}}%
\pgfusepath{clip}%
\pgfsetbuttcap%
\pgfsetroundjoin%
\pgfsetlinewidth{1.003750pt}%
\definecolor{currentstroke}{rgb}{0.450000,0.450000,0.450000}%
\pgfsetstrokecolor{currentstroke}%
\pgfsetdash{}{0pt}%
\pgfpathmoveto{\pgfqpoint{2.263791in}{0.780530in}}%
\pgfpathcurveto{\pgfqpoint{2.273000in}{0.780530in}}{\pgfqpoint{2.281832in}{0.784188in}}{\pgfqpoint{2.288344in}{0.790699in}}%
\pgfpathcurveto{\pgfqpoint{2.294855in}{0.797211in}}{\pgfqpoint{2.298514in}{0.806043in}}{\pgfqpoint{2.298514in}{0.815252in}}%
\pgfpathcurveto{\pgfqpoint{2.298514in}{0.824460in}}{\pgfqpoint{2.294855in}{0.833293in}}{\pgfqpoint{2.288344in}{0.839804in}}%
\pgfpathcurveto{\pgfqpoint{2.281832in}{0.846315in}}{\pgfqpoint{2.273000in}{0.849974in}}{\pgfqpoint{2.263791in}{0.849974in}}%
\pgfpathcurveto{\pgfqpoint{2.254583in}{0.849974in}}{\pgfqpoint{2.245750in}{0.846315in}}{\pgfqpoint{2.239239in}{0.839804in}}%
\pgfpathcurveto{\pgfqpoint{2.232728in}{0.833293in}}{\pgfqpoint{2.229069in}{0.824460in}}{\pgfqpoint{2.229069in}{0.815252in}}%
\pgfpathcurveto{\pgfqpoint{2.229069in}{0.806043in}}{\pgfqpoint{2.232728in}{0.797211in}}{\pgfqpoint{2.239239in}{0.790699in}}%
\pgfpathcurveto{\pgfqpoint{2.245750in}{0.784188in}}{\pgfqpoint{2.254583in}{0.780530in}}{\pgfqpoint{2.263791in}{0.780530in}}%
\pgfusepath{stroke}%
\end{pgfscope}%
\begin{pgfscope}%
\pgfpathrectangle{\pgfqpoint{0.417359in}{0.814008in}}{\pgfqpoint{12.309552in}{4.722875in}}%
\pgfusepath{clip}%
\pgfsetbuttcap%
\pgfsetroundjoin%
\pgfsetlinewidth{1.003750pt}%
\definecolor{currentstroke}{rgb}{0.450000,0.450000,0.450000}%
\pgfsetstrokecolor{currentstroke}%
\pgfsetdash{}{0pt}%
\pgfpathmoveto{\pgfqpoint{2.263791in}{0.779586in}}%
\pgfpathcurveto{\pgfqpoint{2.273000in}{0.779586in}}{\pgfqpoint{2.281832in}{0.783244in}}{\pgfqpoint{2.288344in}{0.789756in}}%
\pgfpathcurveto{\pgfqpoint{2.294855in}{0.796267in}}{\pgfqpoint{2.298514in}{0.805099in}}{\pgfqpoint{2.298514in}{0.814308in}}%
\pgfpathcurveto{\pgfqpoint{2.298514in}{0.823516in}}{\pgfqpoint{2.294855in}{0.832349in}}{\pgfqpoint{2.288344in}{0.838860in}}%
\pgfpathcurveto{\pgfqpoint{2.281832in}{0.845372in}}{\pgfqpoint{2.273000in}{0.849030in}}{\pgfqpoint{2.263791in}{0.849030in}}%
\pgfpathcurveto{\pgfqpoint{2.254583in}{0.849030in}}{\pgfqpoint{2.245750in}{0.845372in}}{\pgfqpoint{2.239239in}{0.838860in}}%
\pgfpathcurveto{\pgfqpoint{2.232728in}{0.832349in}}{\pgfqpoint{2.229069in}{0.823516in}}{\pgfqpoint{2.229069in}{0.814308in}}%
\pgfpathcurveto{\pgfqpoint{2.229069in}{0.805099in}}{\pgfqpoint{2.232728in}{0.796267in}}{\pgfqpoint{2.239239in}{0.789756in}}%
\pgfpathcurveto{\pgfqpoint{2.245750in}{0.783244in}}{\pgfqpoint{2.254583in}{0.779586in}}{\pgfqpoint{2.263791in}{0.779586in}}%
\pgfusepath{stroke}%
\end{pgfscope}%
\begin{pgfscope}%
\pgfpathrectangle{\pgfqpoint{0.417359in}{0.814008in}}{\pgfqpoint{12.309552in}{4.722875in}}%
\pgfusepath{clip}%
\pgfsetbuttcap%
\pgfsetroundjoin%
\pgfsetlinewidth{1.003750pt}%
\definecolor{currentstroke}{rgb}{0.450000,0.450000,0.450000}%
\pgfsetstrokecolor{currentstroke}%
\pgfsetdash{}{0pt}%
\pgfpathmoveto{\pgfqpoint{2.263791in}{0.781023in}}%
\pgfpathcurveto{\pgfqpoint{2.273000in}{0.781023in}}{\pgfqpoint{2.281832in}{0.784682in}}{\pgfqpoint{2.288344in}{0.791193in}}%
\pgfpathcurveto{\pgfqpoint{2.294855in}{0.797704in}}{\pgfqpoint{2.298514in}{0.806537in}}{\pgfqpoint{2.298514in}{0.815745in}}%
\pgfpathcurveto{\pgfqpoint{2.298514in}{0.824954in}}{\pgfqpoint{2.294855in}{0.833786in}}{\pgfqpoint{2.288344in}{0.840298in}}%
\pgfpathcurveto{\pgfqpoint{2.281832in}{0.846809in}}{\pgfqpoint{2.273000in}{0.850467in}}{\pgfqpoint{2.263791in}{0.850467in}}%
\pgfpathcurveto{\pgfqpoint{2.254583in}{0.850467in}}{\pgfqpoint{2.245750in}{0.846809in}}{\pgfqpoint{2.239239in}{0.840298in}}%
\pgfpathcurveto{\pgfqpoint{2.232728in}{0.833786in}}{\pgfqpoint{2.229069in}{0.824954in}}{\pgfqpoint{2.229069in}{0.815745in}}%
\pgfpathcurveto{\pgfqpoint{2.229069in}{0.806537in}}{\pgfqpoint{2.232728in}{0.797704in}}{\pgfqpoint{2.239239in}{0.791193in}}%
\pgfpathcurveto{\pgfqpoint{2.245750in}{0.784682in}}{\pgfqpoint{2.254583in}{0.781023in}}{\pgfqpoint{2.263791in}{0.781023in}}%
\pgfusepath{stroke}%
\end{pgfscope}%
\begin{pgfscope}%
\pgfpathrectangle{\pgfqpoint{0.417359in}{0.814008in}}{\pgfqpoint{12.309552in}{4.722875in}}%
\pgfusepath{clip}%
\pgfsetbuttcap%
\pgfsetroundjoin%
\pgfsetlinewidth{1.003750pt}%
\definecolor{currentstroke}{rgb}{0.450000,0.450000,0.450000}%
\pgfsetstrokecolor{currentstroke}%
\pgfsetdash{}{0pt}%
\pgfpathmoveto{\pgfqpoint{2.263791in}{0.782014in}}%
\pgfpathcurveto{\pgfqpoint{2.273000in}{0.782014in}}{\pgfqpoint{2.281832in}{0.785673in}}{\pgfqpoint{2.288344in}{0.792184in}}%
\pgfpathcurveto{\pgfqpoint{2.294855in}{0.798695in}}{\pgfqpoint{2.298514in}{0.807528in}}{\pgfqpoint{2.298514in}{0.816736in}}%
\pgfpathcurveto{\pgfqpoint{2.298514in}{0.825945in}}{\pgfqpoint{2.294855in}{0.834777in}}{\pgfqpoint{2.288344in}{0.841289in}}%
\pgfpathcurveto{\pgfqpoint{2.281832in}{0.847800in}}{\pgfqpoint{2.273000in}{0.851458in}}{\pgfqpoint{2.263791in}{0.851458in}}%
\pgfpathcurveto{\pgfqpoint{2.254583in}{0.851458in}}{\pgfqpoint{2.245750in}{0.847800in}}{\pgfqpoint{2.239239in}{0.841289in}}%
\pgfpathcurveto{\pgfqpoint{2.232728in}{0.834777in}}{\pgfqpoint{2.229069in}{0.825945in}}{\pgfqpoint{2.229069in}{0.816736in}}%
\pgfpathcurveto{\pgfqpoint{2.229069in}{0.807528in}}{\pgfqpoint{2.232728in}{0.798695in}}{\pgfqpoint{2.239239in}{0.792184in}}%
\pgfpathcurveto{\pgfqpoint{2.245750in}{0.785673in}}{\pgfqpoint{2.254583in}{0.782014in}}{\pgfqpoint{2.263791in}{0.782014in}}%
\pgfusepath{stroke}%
\end{pgfscope}%
\begin{pgfscope}%
\pgfpathrectangle{\pgfqpoint{0.417359in}{0.814008in}}{\pgfqpoint{12.309552in}{4.722875in}}%
\pgfusepath{clip}%
\pgfsetbuttcap%
\pgfsetroundjoin%
\pgfsetlinewidth{1.003750pt}%
\definecolor{currentstroke}{rgb}{0.450000,0.450000,0.450000}%
\pgfsetstrokecolor{currentstroke}%
\pgfsetdash{}{0pt}%
\pgfpathmoveto{\pgfqpoint{2.263791in}{0.942525in}}%
\pgfpathcurveto{\pgfqpoint{2.273000in}{0.942525in}}{\pgfqpoint{2.281832in}{0.946184in}}{\pgfqpoint{2.288344in}{0.952695in}}%
\pgfpathcurveto{\pgfqpoint{2.294855in}{0.959207in}}{\pgfqpoint{2.298514in}{0.968039in}}{\pgfqpoint{2.298514in}{0.977248in}}%
\pgfpathcurveto{\pgfqpoint{2.298514in}{0.986456in}}{\pgfqpoint{2.294855in}{0.995289in}}{\pgfqpoint{2.288344in}{1.001800in}}%
\pgfpathcurveto{\pgfqpoint{2.281832in}{1.008311in}}{\pgfqpoint{2.273000in}{1.011970in}}{\pgfqpoint{2.263791in}{1.011970in}}%
\pgfpathcurveto{\pgfqpoint{2.254583in}{1.011970in}}{\pgfqpoint{2.245750in}{1.008311in}}{\pgfqpoint{2.239239in}{1.001800in}}%
\pgfpathcurveto{\pgfqpoint{2.232728in}{0.995289in}}{\pgfqpoint{2.229069in}{0.986456in}}{\pgfqpoint{2.229069in}{0.977248in}}%
\pgfpathcurveto{\pgfqpoint{2.229069in}{0.968039in}}{\pgfqpoint{2.232728in}{0.959207in}}{\pgfqpoint{2.239239in}{0.952695in}}%
\pgfpathcurveto{\pgfqpoint{2.245750in}{0.946184in}}{\pgfqpoint{2.254583in}{0.942525in}}{\pgfqpoint{2.263791in}{0.942525in}}%
\pgfpathlineto{\pgfqpoint{2.263791in}{0.942525in}}%
\pgfpathclose%
\pgfusepath{stroke}%
\end{pgfscope}%
\begin{pgfscope}%
\pgfpathrectangle{\pgfqpoint{0.417359in}{0.814008in}}{\pgfqpoint{12.309552in}{4.722875in}}%
\pgfusepath{clip}%
\pgfsetbuttcap%
\pgfsetroundjoin%
\pgfsetlinewidth{1.003750pt}%
\definecolor{currentstroke}{rgb}{0.450000,0.450000,0.450000}%
\pgfsetstrokecolor{currentstroke}%
\pgfsetdash{}{0pt}%
\pgfpathmoveto{\pgfqpoint{2.263791in}{0.942525in}}%
\pgfpathcurveto{\pgfqpoint{2.273000in}{0.942525in}}{\pgfqpoint{2.281832in}{0.946184in}}{\pgfqpoint{2.288344in}{0.952695in}}%
\pgfpathcurveto{\pgfqpoint{2.294855in}{0.959207in}}{\pgfqpoint{2.298514in}{0.968039in}}{\pgfqpoint{2.298514in}{0.977248in}}%
\pgfpathcurveto{\pgfqpoint{2.298514in}{0.986456in}}{\pgfqpoint{2.294855in}{0.995289in}}{\pgfqpoint{2.288344in}{1.001800in}}%
\pgfpathcurveto{\pgfqpoint{2.281832in}{1.008311in}}{\pgfqpoint{2.273000in}{1.011970in}}{\pgfqpoint{2.263791in}{1.011970in}}%
\pgfpathcurveto{\pgfqpoint{2.254583in}{1.011970in}}{\pgfqpoint{2.245750in}{1.008311in}}{\pgfqpoint{2.239239in}{1.001800in}}%
\pgfpathcurveto{\pgfqpoint{2.232728in}{0.995289in}}{\pgfqpoint{2.229069in}{0.986456in}}{\pgfqpoint{2.229069in}{0.977248in}}%
\pgfpathcurveto{\pgfqpoint{2.229069in}{0.968039in}}{\pgfqpoint{2.232728in}{0.959207in}}{\pgfqpoint{2.239239in}{0.952695in}}%
\pgfpathcurveto{\pgfqpoint{2.245750in}{0.946184in}}{\pgfqpoint{2.254583in}{0.942525in}}{\pgfqpoint{2.263791in}{0.942525in}}%
\pgfpathlineto{\pgfqpoint{2.263791in}{0.942525in}}%
\pgfpathclose%
\pgfusepath{stroke}%
\end{pgfscope}%
\begin{pgfscope}%
\pgfpathrectangle{\pgfqpoint{0.417359in}{0.814008in}}{\pgfqpoint{12.309552in}{4.722875in}}%
\pgfusepath{clip}%
\pgfsetbuttcap%
\pgfsetroundjoin%
\pgfsetlinewidth{1.003750pt}%
\definecolor{currentstroke}{rgb}{0.450000,0.450000,0.450000}%
\pgfsetstrokecolor{currentstroke}%
\pgfsetdash{}{0pt}%
\pgfpathmoveto{\pgfqpoint{2.263791in}{0.940177in}}%
\pgfpathcurveto{\pgfqpoint{2.273000in}{0.940177in}}{\pgfqpoint{2.281832in}{0.943836in}}{\pgfqpoint{2.288344in}{0.950347in}}%
\pgfpathcurveto{\pgfqpoint{2.294855in}{0.956858in}}{\pgfqpoint{2.298514in}{0.965691in}}{\pgfqpoint{2.298514in}{0.974899in}}%
\pgfpathcurveto{\pgfqpoint{2.298514in}{0.984108in}}{\pgfqpoint{2.294855in}{0.992940in}}{\pgfqpoint{2.288344in}{0.999452in}}%
\pgfpathcurveto{\pgfqpoint{2.281832in}{1.005963in}}{\pgfqpoint{2.273000in}{1.009622in}}{\pgfqpoint{2.263791in}{1.009622in}}%
\pgfpathcurveto{\pgfqpoint{2.254583in}{1.009622in}}{\pgfqpoint{2.245750in}{1.005963in}}{\pgfqpoint{2.239239in}{0.999452in}}%
\pgfpathcurveto{\pgfqpoint{2.232728in}{0.992940in}}{\pgfqpoint{2.229069in}{0.984108in}}{\pgfqpoint{2.229069in}{0.974899in}}%
\pgfpathcurveto{\pgfqpoint{2.229069in}{0.965691in}}{\pgfqpoint{2.232728in}{0.956858in}}{\pgfqpoint{2.239239in}{0.950347in}}%
\pgfpathcurveto{\pgfqpoint{2.245750in}{0.943836in}}{\pgfqpoint{2.254583in}{0.940177in}}{\pgfqpoint{2.263791in}{0.940177in}}%
\pgfpathlineto{\pgfqpoint{2.263791in}{0.940177in}}%
\pgfpathclose%
\pgfusepath{stroke}%
\end{pgfscope}%
\begin{pgfscope}%
\pgfpathrectangle{\pgfqpoint{0.417359in}{0.814008in}}{\pgfqpoint{12.309552in}{4.722875in}}%
\pgfusepath{clip}%
\pgfsetbuttcap%
\pgfsetroundjoin%
\pgfsetlinewidth{1.003750pt}%
\definecolor{currentstroke}{rgb}{0.450000,0.450000,0.450000}%
\pgfsetstrokecolor{currentstroke}%
\pgfsetdash{}{0pt}%
\pgfpathmoveto{\pgfqpoint{2.263791in}{0.940177in}}%
\pgfpathcurveto{\pgfqpoint{2.273000in}{0.940177in}}{\pgfqpoint{2.281832in}{0.943836in}}{\pgfqpoint{2.288344in}{0.950347in}}%
\pgfpathcurveto{\pgfqpoint{2.294855in}{0.956858in}}{\pgfqpoint{2.298514in}{0.965691in}}{\pgfqpoint{2.298514in}{0.974899in}}%
\pgfpathcurveto{\pgfqpoint{2.298514in}{0.984108in}}{\pgfqpoint{2.294855in}{0.992940in}}{\pgfqpoint{2.288344in}{0.999452in}}%
\pgfpathcurveto{\pgfqpoint{2.281832in}{1.005963in}}{\pgfqpoint{2.273000in}{1.009622in}}{\pgfqpoint{2.263791in}{1.009622in}}%
\pgfpathcurveto{\pgfqpoint{2.254583in}{1.009622in}}{\pgfqpoint{2.245750in}{1.005963in}}{\pgfqpoint{2.239239in}{0.999452in}}%
\pgfpathcurveto{\pgfqpoint{2.232728in}{0.992940in}}{\pgfqpoint{2.229069in}{0.984108in}}{\pgfqpoint{2.229069in}{0.974899in}}%
\pgfpathcurveto{\pgfqpoint{2.229069in}{0.965691in}}{\pgfqpoint{2.232728in}{0.956858in}}{\pgfqpoint{2.239239in}{0.950347in}}%
\pgfpathcurveto{\pgfqpoint{2.245750in}{0.943836in}}{\pgfqpoint{2.254583in}{0.940177in}}{\pgfqpoint{2.263791in}{0.940177in}}%
\pgfpathlineto{\pgfqpoint{2.263791in}{0.940177in}}%
\pgfpathclose%
\pgfusepath{stroke}%
\end{pgfscope}%
\begin{pgfscope}%
\pgfpathrectangle{\pgfqpoint{0.417359in}{0.814008in}}{\pgfqpoint{12.309552in}{4.722875in}}%
\pgfusepath{clip}%
\pgfsetbuttcap%
\pgfsetroundjoin%
\pgfsetlinewidth{1.003750pt}%
\definecolor{currentstroke}{rgb}{0.450000,0.450000,0.450000}%
\pgfsetstrokecolor{currentstroke}%
\pgfsetdash{}{0pt}%
\pgfpathmoveto{\pgfqpoint{2.263791in}{0.940177in}}%
\pgfpathcurveto{\pgfqpoint{2.273000in}{0.940177in}}{\pgfqpoint{2.281832in}{0.943836in}}{\pgfqpoint{2.288344in}{0.950347in}}%
\pgfpathcurveto{\pgfqpoint{2.294855in}{0.956858in}}{\pgfqpoint{2.298514in}{0.965691in}}{\pgfqpoint{2.298514in}{0.974899in}}%
\pgfpathcurveto{\pgfqpoint{2.298514in}{0.984108in}}{\pgfqpoint{2.294855in}{0.992940in}}{\pgfqpoint{2.288344in}{0.999452in}}%
\pgfpathcurveto{\pgfqpoint{2.281832in}{1.005963in}}{\pgfqpoint{2.273000in}{1.009622in}}{\pgfqpoint{2.263791in}{1.009622in}}%
\pgfpathcurveto{\pgfqpoint{2.254583in}{1.009622in}}{\pgfqpoint{2.245750in}{1.005963in}}{\pgfqpoint{2.239239in}{0.999452in}}%
\pgfpathcurveto{\pgfqpoint{2.232728in}{0.992940in}}{\pgfqpoint{2.229069in}{0.984108in}}{\pgfqpoint{2.229069in}{0.974899in}}%
\pgfpathcurveto{\pgfqpoint{2.229069in}{0.965691in}}{\pgfqpoint{2.232728in}{0.956858in}}{\pgfqpoint{2.239239in}{0.950347in}}%
\pgfpathcurveto{\pgfqpoint{2.245750in}{0.943836in}}{\pgfqpoint{2.254583in}{0.940177in}}{\pgfqpoint{2.263791in}{0.940177in}}%
\pgfpathlineto{\pgfqpoint{2.263791in}{0.940177in}}%
\pgfpathclose%
\pgfusepath{stroke}%
\end{pgfscope}%
\begin{pgfscope}%
\pgfpathrectangle{\pgfqpoint{0.417359in}{0.814008in}}{\pgfqpoint{12.309552in}{4.722875in}}%
\pgfusepath{clip}%
\pgfsetbuttcap%
\pgfsetroundjoin%
\pgfsetlinewidth{1.003750pt}%
\definecolor{currentstroke}{rgb}{0.450000,0.450000,0.450000}%
\pgfsetstrokecolor{currentstroke}%
\pgfsetdash{}{0pt}%
\pgfpathmoveto{\pgfqpoint{2.263791in}{0.940177in}}%
\pgfpathcurveto{\pgfqpoint{2.273000in}{0.940177in}}{\pgfqpoint{2.281832in}{0.943836in}}{\pgfqpoint{2.288344in}{0.950347in}}%
\pgfpathcurveto{\pgfqpoint{2.294855in}{0.956858in}}{\pgfqpoint{2.298514in}{0.965691in}}{\pgfqpoint{2.298514in}{0.974899in}}%
\pgfpathcurveto{\pgfqpoint{2.298514in}{0.984108in}}{\pgfqpoint{2.294855in}{0.992940in}}{\pgfqpoint{2.288344in}{0.999452in}}%
\pgfpathcurveto{\pgfqpoint{2.281832in}{1.005963in}}{\pgfqpoint{2.273000in}{1.009622in}}{\pgfqpoint{2.263791in}{1.009622in}}%
\pgfpathcurveto{\pgfqpoint{2.254583in}{1.009622in}}{\pgfqpoint{2.245750in}{1.005963in}}{\pgfqpoint{2.239239in}{0.999452in}}%
\pgfpathcurveto{\pgfqpoint{2.232728in}{0.992940in}}{\pgfqpoint{2.229069in}{0.984108in}}{\pgfqpoint{2.229069in}{0.974899in}}%
\pgfpathcurveto{\pgfqpoint{2.229069in}{0.965691in}}{\pgfqpoint{2.232728in}{0.956858in}}{\pgfqpoint{2.239239in}{0.950347in}}%
\pgfpathcurveto{\pgfqpoint{2.245750in}{0.943836in}}{\pgfqpoint{2.254583in}{0.940177in}}{\pgfqpoint{2.263791in}{0.940177in}}%
\pgfpathlineto{\pgfqpoint{2.263791in}{0.940177in}}%
\pgfpathclose%
\pgfusepath{stroke}%
\end{pgfscope}%
\begin{pgfscope}%
\pgfpathrectangle{\pgfqpoint{0.417359in}{0.814008in}}{\pgfqpoint{12.309552in}{4.722875in}}%
\pgfusepath{clip}%
\pgfsetbuttcap%
\pgfsetroundjoin%
\pgfsetlinewidth{1.003750pt}%
\definecolor{currentstroke}{rgb}{0.450000,0.450000,0.450000}%
\pgfsetstrokecolor{currentstroke}%
\pgfsetdash{}{0pt}%
\pgfpathmoveto{\pgfqpoint{2.263791in}{0.779959in}}%
\pgfpathcurveto{\pgfqpoint{2.273000in}{0.779959in}}{\pgfqpoint{2.281832in}{0.783618in}}{\pgfqpoint{2.288344in}{0.790129in}}%
\pgfpathcurveto{\pgfqpoint{2.294855in}{0.796640in}}{\pgfqpoint{2.298514in}{0.805473in}}{\pgfqpoint{2.298514in}{0.814681in}}%
\pgfpathcurveto{\pgfqpoint{2.298514in}{0.823890in}}{\pgfqpoint{2.294855in}{0.832722in}}{\pgfqpoint{2.288344in}{0.839234in}}%
\pgfpathcurveto{\pgfqpoint{2.281832in}{0.845745in}}{\pgfqpoint{2.273000in}{0.849404in}}{\pgfqpoint{2.263791in}{0.849404in}}%
\pgfpathcurveto{\pgfqpoint{2.254583in}{0.849404in}}{\pgfqpoint{2.245750in}{0.845745in}}{\pgfqpoint{2.239239in}{0.839234in}}%
\pgfpathcurveto{\pgfqpoint{2.232728in}{0.832722in}}{\pgfqpoint{2.229069in}{0.823890in}}{\pgfqpoint{2.229069in}{0.814681in}}%
\pgfpathcurveto{\pgfqpoint{2.229069in}{0.805473in}}{\pgfqpoint{2.232728in}{0.796640in}}{\pgfqpoint{2.239239in}{0.790129in}}%
\pgfpathcurveto{\pgfqpoint{2.245750in}{0.783618in}}{\pgfqpoint{2.254583in}{0.779959in}}{\pgfqpoint{2.263791in}{0.779959in}}%
\pgfusepath{stroke}%
\end{pgfscope}%
\begin{pgfscope}%
\pgfpathrectangle{\pgfqpoint{0.417359in}{0.814008in}}{\pgfqpoint{12.309552in}{4.722875in}}%
\pgfusepath{clip}%
\pgfsetbuttcap%
\pgfsetroundjoin%
\pgfsetlinewidth{1.003750pt}%
\definecolor{currentstroke}{rgb}{0.450000,0.450000,0.450000}%
\pgfsetstrokecolor{currentstroke}%
\pgfsetdash{}{0pt}%
\pgfpathmoveto{\pgfqpoint{2.263791in}{0.779480in}}%
\pgfpathcurveto{\pgfqpoint{2.273000in}{0.779480in}}{\pgfqpoint{2.281832in}{0.783139in}}{\pgfqpoint{2.288344in}{0.789650in}}%
\pgfpathcurveto{\pgfqpoint{2.294855in}{0.796161in}}{\pgfqpoint{2.298514in}{0.804994in}}{\pgfqpoint{2.298514in}{0.814202in}}%
\pgfpathcurveto{\pgfqpoint{2.298514in}{0.823411in}}{\pgfqpoint{2.294855in}{0.832243in}}{\pgfqpoint{2.288344in}{0.838755in}}%
\pgfpathcurveto{\pgfqpoint{2.281832in}{0.845266in}}{\pgfqpoint{2.273000in}{0.848925in}}{\pgfqpoint{2.263791in}{0.848925in}}%
\pgfpathcurveto{\pgfqpoint{2.254583in}{0.848925in}}{\pgfqpoint{2.245750in}{0.845266in}}{\pgfqpoint{2.239239in}{0.838755in}}%
\pgfpathcurveto{\pgfqpoint{2.232728in}{0.832243in}}{\pgfqpoint{2.229069in}{0.823411in}}{\pgfqpoint{2.229069in}{0.814202in}}%
\pgfpathcurveto{\pgfqpoint{2.229069in}{0.804994in}}{\pgfqpoint{2.232728in}{0.796161in}}{\pgfqpoint{2.239239in}{0.789650in}}%
\pgfpathcurveto{\pgfqpoint{2.245750in}{0.783139in}}{\pgfqpoint{2.254583in}{0.779480in}}{\pgfqpoint{2.263791in}{0.779480in}}%
\pgfusepath{stroke}%
\end{pgfscope}%
\begin{pgfscope}%
\pgfpathrectangle{\pgfqpoint{0.417359in}{0.814008in}}{\pgfqpoint{12.309552in}{4.722875in}}%
\pgfusepath{clip}%
\pgfsetbuttcap%
\pgfsetroundjoin%
\pgfsetlinewidth{1.003750pt}%
\definecolor{currentstroke}{rgb}{0.450000,0.450000,0.450000}%
\pgfsetstrokecolor{currentstroke}%
\pgfsetdash{}{0pt}%
\pgfpathmoveto{\pgfqpoint{2.263791in}{4.279490in}}%
\pgfpathcurveto{\pgfqpoint{2.273000in}{4.279490in}}{\pgfqpoint{2.281832in}{4.283148in}}{\pgfqpoint{2.288344in}{4.289659in}}%
\pgfpathcurveto{\pgfqpoint{2.294855in}{4.296171in}}{\pgfqpoint{2.298514in}{4.305003in}}{\pgfqpoint{2.298514in}{4.314212in}}%
\pgfpathcurveto{\pgfqpoint{2.298514in}{4.323420in}}{\pgfqpoint{2.294855in}{4.332253in}}{\pgfqpoint{2.288344in}{4.338764in}}%
\pgfpathcurveto{\pgfqpoint{2.281832in}{4.345275in}}{\pgfqpoint{2.273000in}{4.348934in}}{\pgfqpoint{2.263791in}{4.348934in}}%
\pgfpathcurveto{\pgfqpoint{2.254583in}{4.348934in}}{\pgfqpoint{2.245750in}{4.345275in}}{\pgfqpoint{2.239239in}{4.338764in}}%
\pgfpathcurveto{\pgfqpoint{2.232728in}{4.332253in}}{\pgfqpoint{2.229069in}{4.323420in}}{\pgfqpoint{2.229069in}{4.314212in}}%
\pgfpathcurveto{\pgfqpoint{2.229069in}{4.305003in}}{\pgfqpoint{2.232728in}{4.296171in}}{\pgfqpoint{2.239239in}{4.289659in}}%
\pgfpathcurveto{\pgfqpoint{2.245750in}{4.283148in}}{\pgfqpoint{2.254583in}{4.279490in}}{\pgfqpoint{2.263791in}{4.279490in}}%
\pgfpathlineto{\pgfqpoint{2.263791in}{4.279490in}}%
\pgfpathclose%
\pgfusepath{stroke}%
\end{pgfscope}%
\begin{pgfscope}%
\pgfpathrectangle{\pgfqpoint{0.417359in}{0.814008in}}{\pgfqpoint{12.309552in}{4.722875in}}%
\pgfusepath{clip}%
\pgfsetbuttcap%
\pgfsetroundjoin%
\pgfsetlinewidth{1.003750pt}%
\definecolor{currentstroke}{rgb}{0.450000,0.450000,0.450000}%
\pgfsetstrokecolor{currentstroke}%
\pgfsetdash{}{0pt}%
\pgfpathmoveto{\pgfqpoint{2.263791in}{0.779286in}}%
\pgfpathcurveto{\pgfqpoint{2.273000in}{0.779286in}}{\pgfqpoint{2.281832in}{0.782944in}}{\pgfqpoint{2.288344in}{0.789456in}}%
\pgfpathcurveto{\pgfqpoint{2.294855in}{0.795967in}}{\pgfqpoint{2.298514in}{0.804800in}}{\pgfqpoint{2.298514in}{0.814008in}}%
\pgfpathcurveto{\pgfqpoint{2.298514in}{0.823217in}}{\pgfqpoint{2.294855in}{0.832049in}}{\pgfqpoint{2.288344in}{0.838560in}}%
\pgfpathcurveto{\pgfqpoint{2.281832in}{0.845072in}}{\pgfqpoint{2.273000in}{0.848730in}}{\pgfqpoint{2.263791in}{0.848730in}}%
\pgfpathcurveto{\pgfqpoint{2.254583in}{0.848730in}}{\pgfqpoint{2.245750in}{0.845072in}}{\pgfqpoint{2.239239in}{0.838560in}}%
\pgfpathcurveto{\pgfqpoint{2.232728in}{0.832049in}}{\pgfqpoint{2.229069in}{0.823217in}}{\pgfqpoint{2.229069in}{0.814008in}}%
\pgfpathcurveto{\pgfqpoint{2.229069in}{0.804800in}}{\pgfqpoint{2.232728in}{0.795967in}}{\pgfqpoint{2.239239in}{0.789456in}}%
\pgfpathcurveto{\pgfqpoint{2.245750in}{0.782944in}}{\pgfqpoint{2.254583in}{0.779286in}}{\pgfqpoint{2.263791in}{0.779286in}}%
\pgfusepath{stroke}%
\end{pgfscope}%
\begin{pgfscope}%
\pgfpathrectangle{\pgfqpoint{0.417359in}{0.814008in}}{\pgfqpoint{12.309552in}{4.722875in}}%
\pgfusepath{clip}%
\pgfsetbuttcap%
\pgfsetroundjoin%
\definecolor{currentfill}{rgb}{0.229412,0.570588,0.229412}%
\pgfsetfillcolor{currentfill}%
\pgfsetlinewidth{0.752812pt}%
\definecolor{currentstroke}{rgb}{0.240000,0.240000,0.240000}%
\pgfsetstrokecolor{currentstroke}%
\pgfsetdash{}{0pt}%
\pgfsys@defobject{currentmarker}{\pgfqpoint{3.002364in}{0.926129in}}{\pgfqpoint{3.987129in}{1.133956in}}{%
\pgfpathmoveto{\pgfqpoint{3.002364in}{0.926129in}}%
\pgfpathlineto{\pgfqpoint{3.987129in}{0.926129in}}%
\pgfpathlineto{\pgfqpoint{3.987129in}{1.133956in}}%
\pgfpathlineto{\pgfqpoint{3.002364in}{1.133956in}}%
\pgfpathlineto{\pgfqpoint{3.002364in}{0.926129in}}%
\pgfpathclose%
\pgfusepath{stroke,fill}%
}%
\begin{pgfscope}%
\pgfsys@transformshift{0.000000in}{0.000000in}%
\pgfsys@useobject{currentmarker}{}%
\end{pgfscope}%
\end{pgfscope}%
\begin{pgfscope}%
\pgfpathrectangle{\pgfqpoint{0.417359in}{0.814008in}}{\pgfqpoint{12.309552in}{4.722875in}}%
\pgfusepath{clip}%
\pgfsetbuttcap%
\pgfsetroundjoin%
\pgfsetlinewidth{1.003750pt}%
\definecolor{currentstroke}{rgb}{0.450000,0.450000,0.450000}%
\pgfsetstrokecolor{currentstroke}%
\pgfsetdash{}{0pt}%
\pgfpathmoveto{\pgfqpoint{3.494747in}{0.828530in}}%
\pgfpathcurveto{\pgfqpoint{3.503955in}{0.828530in}}{\pgfqpoint{3.512788in}{0.832188in}}{\pgfqpoint{3.519299in}{0.838700in}}%
\pgfpathcurveto{\pgfqpoint{3.525810in}{0.845211in}}{\pgfqpoint{3.529469in}{0.854043in}}{\pgfqpoint{3.529469in}{0.863252in}}%
\pgfpathcurveto{\pgfqpoint{3.529469in}{0.872460in}}{\pgfqpoint{3.525810in}{0.881293in}}{\pgfqpoint{3.519299in}{0.887804in}}%
\pgfpathcurveto{\pgfqpoint{3.512788in}{0.894316in}}{\pgfqpoint{3.503955in}{0.897974in}}{\pgfqpoint{3.494747in}{0.897974in}}%
\pgfpathcurveto{\pgfqpoint{3.485538in}{0.897974in}}{\pgfqpoint{3.476706in}{0.894316in}}{\pgfqpoint{3.470194in}{0.887804in}}%
\pgfpathcurveto{\pgfqpoint{3.463683in}{0.881293in}}{\pgfqpoint{3.460024in}{0.872460in}}{\pgfqpoint{3.460024in}{0.863252in}}%
\pgfpathcurveto{\pgfqpoint{3.460024in}{0.854043in}}{\pgfqpoint{3.463683in}{0.845211in}}{\pgfqpoint{3.470194in}{0.838700in}}%
\pgfpathcurveto{\pgfqpoint{3.476706in}{0.832188in}}{\pgfqpoint{3.485538in}{0.828530in}}{\pgfqpoint{3.494747in}{0.828530in}}%
\pgfpathlineto{\pgfqpoint{3.494747in}{0.828530in}}%
\pgfpathclose%
\pgfusepath{stroke}%
\end{pgfscope}%
\begin{pgfscope}%
\pgfpathrectangle{\pgfqpoint{0.417359in}{0.814008in}}{\pgfqpoint{12.309552in}{4.722875in}}%
\pgfusepath{clip}%
\pgfsetbuttcap%
\pgfsetroundjoin%
\pgfsetlinewidth{1.003750pt}%
\definecolor{currentstroke}{rgb}{0.450000,0.450000,0.450000}%
\pgfsetstrokecolor{currentstroke}%
\pgfsetdash{}{0pt}%
\pgfpathmoveto{\pgfqpoint{3.494747in}{0.787788in}}%
\pgfpathcurveto{\pgfqpoint{3.503955in}{0.787788in}}{\pgfqpoint{3.512788in}{0.791446in}}{\pgfqpoint{3.519299in}{0.797958in}}%
\pgfpathcurveto{\pgfqpoint{3.525810in}{0.804469in}}{\pgfqpoint{3.529469in}{0.813302in}}{\pgfqpoint{3.529469in}{0.822510in}}%
\pgfpathcurveto{\pgfqpoint{3.529469in}{0.831719in}}{\pgfqpoint{3.525810in}{0.840551in}}{\pgfqpoint{3.519299in}{0.847062in}}%
\pgfpathcurveto{\pgfqpoint{3.512788in}{0.853574in}}{\pgfqpoint{3.503955in}{0.857232in}}{\pgfqpoint{3.494747in}{0.857232in}}%
\pgfpathcurveto{\pgfqpoint{3.485538in}{0.857232in}}{\pgfqpoint{3.476706in}{0.853574in}}{\pgfqpoint{3.470194in}{0.847062in}}%
\pgfpathcurveto{\pgfqpoint{3.463683in}{0.840551in}}{\pgfqpoint{3.460024in}{0.831719in}}{\pgfqpoint{3.460024in}{0.822510in}}%
\pgfpathcurveto{\pgfqpoint{3.460024in}{0.813302in}}{\pgfqpoint{3.463683in}{0.804469in}}{\pgfqpoint{3.470194in}{0.797958in}}%
\pgfpathcurveto{\pgfqpoint{3.476706in}{0.791446in}}{\pgfqpoint{3.485538in}{0.787788in}}{\pgfqpoint{3.494747in}{0.787788in}}%
\pgfusepath{stroke}%
\end{pgfscope}%
\begin{pgfscope}%
\pgfpathrectangle{\pgfqpoint{0.417359in}{0.814008in}}{\pgfqpoint{12.309552in}{4.722875in}}%
\pgfusepath{clip}%
\pgfsetbuttcap%
\pgfsetroundjoin%
\pgfsetlinewidth{1.003750pt}%
\definecolor{currentstroke}{rgb}{0.450000,0.450000,0.450000}%
\pgfsetstrokecolor{currentstroke}%
\pgfsetdash{}{0pt}%
\pgfpathmoveto{\pgfqpoint{3.494747in}{3.177813in}}%
\pgfpathcurveto{\pgfqpoint{3.503955in}{3.177813in}}{\pgfqpoint{3.512788in}{3.181472in}}{\pgfqpoint{3.519299in}{3.187983in}}%
\pgfpathcurveto{\pgfqpoint{3.525810in}{3.194495in}}{\pgfqpoint{3.529469in}{3.203327in}}{\pgfqpoint{3.529469in}{3.212536in}}%
\pgfpathcurveto{\pgfqpoint{3.529469in}{3.221744in}}{\pgfqpoint{3.525810in}{3.230577in}}{\pgfqpoint{3.519299in}{3.237088in}}%
\pgfpathcurveto{\pgfqpoint{3.512788in}{3.243599in}}{\pgfqpoint{3.503955in}{3.247258in}}{\pgfqpoint{3.494747in}{3.247258in}}%
\pgfpathcurveto{\pgfqpoint{3.485538in}{3.247258in}}{\pgfqpoint{3.476706in}{3.243599in}}{\pgfqpoint{3.470194in}{3.237088in}}%
\pgfpathcurveto{\pgfqpoint{3.463683in}{3.230577in}}{\pgfqpoint{3.460024in}{3.221744in}}{\pgfqpoint{3.460024in}{3.212536in}}%
\pgfpathcurveto{\pgfqpoint{3.460024in}{3.203327in}}{\pgfqpoint{3.463683in}{3.194495in}}{\pgfqpoint{3.470194in}{3.187983in}}%
\pgfpathcurveto{\pgfqpoint{3.476706in}{3.181472in}}{\pgfqpoint{3.485538in}{3.177813in}}{\pgfqpoint{3.494747in}{3.177813in}}%
\pgfpathlineto{\pgfqpoint{3.494747in}{3.177813in}}%
\pgfpathclose%
\pgfusepath{stroke}%
\end{pgfscope}%
\begin{pgfscope}%
\pgfpathrectangle{\pgfqpoint{0.417359in}{0.814008in}}{\pgfqpoint{12.309552in}{4.722875in}}%
\pgfusepath{clip}%
\pgfsetbuttcap%
\pgfsetroundjoin%
\pgfsetlinewidth{1.003750pt}%
\definecolor{currentstroke}{rgb}{0.450000,0.450000,0.450000}%
\pgfsetstrokecolor{currentstroke}%
\pgfsetdash{}{0pt}%
\pgfpathmoveto{\pgfqpoint{3.494747in}{3.048165in}}%
\pgfpathcurveto{\pgfqpoint{3.503955in}{3.048165in}}{\pgfqpoint{3.512788in}{3.051823in}}{\pgfqpoint{3.519299in}{3.058335in}}%
\pgfpathcurveto{\pgfqpoint{3.525810in}{3.064846in}}{\pgfqpoint{3.529469in}{3.073678in}}{\pgfqpoint{3.529469in}{3.082887in}}%
\pgfpathcurveto{\pgfqpoint{3.529469in}{3.092095in}}{\pgfqpoint{3.525810in}{3.100928in}}{\pgfqpoint{3.519299in}{3.107439in}}%
\pgfpathcurveto{\pgfqpoint{3.512788in}{3.113951in}}{\pgfqpoint{3.503955in}{3.117609in}}{\pgfqpoint{3.494747in}{3.117609in}}%
\pgfpathcurveto{\pgfqpoint{3.485538in}{3.117609in}}{\pgfqpoint{3.476706in}{3.113951in}}{\pgfqpoint{3.470194in}{3.107439in}}%
\pgfpathcurveto{\pgfqpoint{3.463683in}{3.100928in}}{\pgfqpoint{3.460024in}{3.092095in}}{\pgfqpoint{3.460024in}{3.082887in}}%
\pgfpathcurveto{\pgfqpoint{3.460024in}{3.073678in}}{\pgfqpoint{3.463683in}{3.064846in}}{\pgfqpoint{3.470194in}{3.058335in}}%
\pgfpathcurveto{\pgfqpoint{3.476706in}{3.051823in}}{\pgfqpoint{3.485538in}{3.048165in}}{\pgfqpoint{3.494747in}{3.048165in}}%
\pgfpathlineto{\pgfqpoint{3.494747in}{3.048165in}}%
\pgfpathclose%
\pgfusepath{stroke}%
\end{pgfscope}%
\begin{pgfscope}%
\pgfpathrectangle{\pgfqpoint{0.417359in}{0.814008in}}{\pgfqpoint{12.309552in}{4.722875in}}%
\pgfusepath{clip}%
\pgfsetbuttcap%
\pgfsetroundjoin%
\pgfsetlinewidth{1.003750pt}%
\definecolor{currentstroke}{rgb}{0.450000,0.450000,0.450000}%
\pgfsetstrokecolor{currentstroke}%
\pgfsetdash{}{0pt}%
\pgfpathmoveto{\pgfqpoint{3.494747in}{1.100814in}}%
\pgfpathcurveto{\pgfqpoint{3.503955in}{1.100814in}}{\pgfqpoint{3.512788in}{1.104472in}}{\pgfqpoint{3.519299in}{1.110984in}}%
\pgfpathcurveto{\pgfqpoint{3.525810in}{1.117495in}}{\pgfqpoint{3.529469in}{1.126328in}}{\pgfqpoint{3.529469in}{1.135536in}}%
\pgfpathcurveto{\pgfqpoint{3.529469in}{1.144744in}}{\pgfqpoint{3.525810in}{1.153577in}}{\pgfqpoint{3.519299in}{1.160088in}}%
\pgfpathcurveto{\pgfqpoint{3.512788in}{1.166600in}}{\pgfqpoint{3.503955in}{1.170258in}}{\pgfqpoint{3.494747in}{1.170258in}}%
\pgfpathcurveto{\pgfqpoint{3.485538in}{1.170258in}}{\pgfqpoint{3.476706in}{1.166600in}}{\pgfqpoint{3.470194in}{1.160088in}}%
\pgfpathcurveto{\pgfqpoint{3.463683in}{1.153577in}}{\pgfqpoint{3.460024in}{1.144744in}}{\pgfqpoint{3.460024in}{1.135536in}}%
\pgfpathcurveto{\pgfqpoint{3.460024in}{1.126328in}}{\pgfqpoint{3.463683in}{1.117495in}}{\pgfqpoint{3.470194in}{1.110984in}}%
\pgfpathcurveto{\pgfqpoint{3.476706in}{1.104472in}}{\pgfqpoint{3.485538in}{1.100814in}}{\pgfqpoint{3.494747in}{1.100814in}}%
\pgfpathlineto{\pgfqpoint{3.494747in}{1.100814in}}%
\pgfpathclose%
\pgfusepath{stroke}%
\end{pgfscope}%
\begin{pgfscope}%
\pgfpathrectangle{\pgfqpoint{0.417359in}{0.814008in}}{\pgfqpoint{12.309552in}{4.722875in}}%
\pgfusepath{clip}%
\pgfsetbuttcap%
\pgfsetroundjoin%
\pgfsetlinewidth{1.003750pt}%
\definecolor{currentstroke}{rgb}{0.450000,0.450000,0.450000}%
\pgfsetstrokecolor{currentstroke}%
\pgfsetdash{}{0pt}%
\pgfpathmoveto{\pgfqpoint{3.494747in}{1.112151in}}%
\pgfpathcurveto{\pgfqpoint{3.503955in}{1.112151in}}{\pgfqpoint{3.512788in}{1.115809in}}{\pgfqpoint{3.519299in}{1.122321in}}%
\pgfpathcurveto{\pgfqpoint{3.525810in}{1.128832in}}{\pgfqpoint{3.529469in}{1.137665in}}{\pgfqpoint{3.529469in}{1.146873in}}%
\pgfpathcurveto{\pgfqpoint{3.529469in}{1.156081in}}{\pgfqpoint{3.525810in}{1.164914in}}{\pgfqpoint{3.519299in}{1.171425in}}%
\pgfpathcurveto{\pgfqpoint{3.512788in}{1.177937in}}{\pgfqpoint{3.503955in}{1.181595in}}{\pgfqpoint{3.494747in}{1.181595in}}%
\pgfpathcurveto{\pgfqpoint{3.485538in}{1.181595in}}{\pgfqpoint{3.476706in}{1.177937in}}{\pgfqpoint{3.470194in}{1.171425in}}%
\pgfpathcurveto{\pgfqpoint{3.463683in}{1.164914in}}{\pgfqpoint{3.460024in}{1.156081in}}{\pgfqpoint{3.460024in}{1.146873in}}%
\pgfpathcurveto{\pgfqpoint{3.460024in}{1.137665in}}{\pgfqpoint{3.463683in}{1.128832in}}{\pgfqpoint{3.470194in}{1.122321in}}%
\pgfpathcurveto{\pgfqpoint{3.476706in}{1.115809in}}{\pgfqpoint{3.485538in}{1.112151in}}{\pgfqpoint{3.494747in}{1.112151in}}%
\pgfpathlineto{\pgfqpoint{3.494747in}{1.112151in}}%
\pgfpathclose%
\pgfusepath{stroke}%
\end{pgfscope}%
\begin{pgfscope}%
\pgfpathrectangle{\pgfqpoint{0.417359in}{0.814008in}}{\pgfqpoint{12.309552in}{4.722875in}}%
\pgfusepath{clip}%
\pgfsetbuttcap%
\pgfsetroundjoin%
\pgfsetlinewidth{1.003750pt}%
\definecolor{currentstroke}{rgb}{0.450000,0.450000,0.450000}%
\pgfsetstrokecolor{currentstroke}%
\pgfsetdash{}{0pt}%
\pgfpathmoveto{\pgfqpoint{3.494747in}{0.889375in}}%
\pgfpathcurveto{\pgfqpoint{3.503955in}{0.889375in}}{\pgfqpoint{3.512788in}{0.893034in}}{\pgfqpoint{3.519299in}{0.899545in}}%
\pgfpathcurveto{\pgfqpoint{3.525810in}{0.906056in}}{\pgfqpoint{3.529469in}{0.914889in}}{\pgfqpoint{3.529469in}{0.924097in}}%
\pgfpathcurveto{\pgfqpoint{3.529469in}{0.933306in}}{\pgfqpoint{3.525810in}{0.942138in}}{\pgfqpoint{3.519299in}{0.948650in}}%
\pgfpathcurveto{\pgfqpoint{3.512788in}{0.955161in}}{\pgfqpoint{3.503955in}{0.958820in}}{\pgfqpoint{3.494747in}{0.958820in}}%
\pgfpathcurveto{\pgfqpoint{3.485538in}{0.958820in}}{\pgfqpoint{3.476706in}{0.955161in}}{\pgfqpoint{3.470194in}{0.948650in}}%
\pgfpathcurveto{\pgfqpoint{3.463683in}{0.942138in}}{\pgfqpoint{3.460024in}{0.933306in}}{\pgfqpoint{3.460024in}{0.924097in}}%
\pgfpathcurveto{\pgfqpoint{3.460024in}{0.914889in}}{\pgfqpoint{3.463683in}{0.906056in}}{\pgfqpoint{3.470194in}{0.899545in}}%
\pgfpathcurveto{\pgfqpoint{3.476706in}{0.893034in}}{\pgfqpoint{3.485538in}{0.889375in}}{\pgfqpoint{3.494747in}{0.889375in}}%
\pgfpathlineto{\pgfqpoint{3.494747in}{0.889375in}}%
\pgfpathclose%
\pgfusepath{stroke}%
\end{pgfscope}%
\begin{pgfscope}%
\pgfpathrectangle{\pgfqpoint{0.417359in}{0.814008in}}{\pgfqpoint{12.309552in}{4.722875in}}%
\pgfusepath{clip}%
\pgfsetbuttcap%
\pgfsetroundjoin%
\pgfsetlinewidth{1.003750pt}%
\definecolor{currentstroke}{rgb}{0.450000,0.450000,0.450000}%
\pgfsetstrokecolor{currentstroke}%
\pgfsetdash{}{0pt}%
\pgfpathmoveto{\pgfqpoint{3.494747in}{0.825450in}}%
\pgfpathcurveto{\pgfqpoint{3.503955in}{0.825450in}}{\pgfqpoint{3.512788in}{0.829109in}}{\pgfqpoint{3.519299in}{0.835620in}}%
\pgfpathcurveto{\pgfqpoint{3.525810in}{0.842132in}}{\pgfqpoint{3.529469in}{0.850964in}}{\pgfqpoint{3.529469in}{0.860173in}}%
\pgfpathcurveto{\pgfqpoint{3.529469in}{0.869381in}}{\pgfqpoint{3.525810in}{0.878214in}}{\pgfqpoint{3.519299in}{0.884725in}}%
\pgfpathcurveto{\pgfqpoint{3.512788in}{0.891236in}}{\pgfqpoint{3.503955in}{0.894895in}}{\pgfqpoint{3.494747in}{0.894895in}}%
\pgfpathcurveto{\pgfqpoint{3.485538in}{0.894895in}}{\pgfqpoint{3.476706in}{0.891236in}}{\pgfqpoint{3.470194in}{0.884725in}}%
\pgfpathcurveto{\pgfqpoint{3.463683in}{0.878214in}}{\pgfqpoint{3.460024in}{0.869381in}}{\pgfqpoint{3.460024in}{0.860173in}}%
\pgfpathcurveto{\pgfqpoint{3.460024in}{0.850964in}}{\pgfqpoint{3.463683in}{0.842132in}}{\pgfqpoint{3.470194in}{0.835620in}}%
\pgfpathcurveto{\pgfqpoint{3.476706in}{0.829109in}}{\pgfqpoint{3.485538in}{0.825450in}}{\pgfqpoint{3.494747in}{0.825450in}}%
\pgfpathlineto{\pgfqpoint{3.494747in}{0.825450in}}%
\pgfpathclose%
\pgfusepath{stroke}%
\end{pgfscope}%
\begin{pgfscope}%
\pgfpathrectangle{\pgfqpoint{0.417359in}{0.814008in}}{\pgfqpoint{12.309552in}{4.722875in}}%
\pgfusepath{clip}%
\pgfsetbuttcap%
\pgfsetroundjoin%
\pgfsetlinewidth{1.003750pt}%
\definecolor{currentstroke}{rgb}{0.450000,0.450000,0.450000}%
\pgfsetstrokecolor{currentstroke}%
\pgfsetdash{}{0pt}%
\pgfpathmoveto{\pgfqpoint{3.494747in}{0.825188in}}%
\pgfpathcurveto{\pgfqpoint{3.503955in}{0.825188in}}{\pgfqpoint{3.512788in}{0.828846in}}{\pgfqpoint{3.519299in}{0.835357in}}%
\pgfpathcurveto{\pgfqpoint{3.525810in}{0.841869in}}{\pgfqpoint{3.529469in}{0.850701in}}{\pgfqpoint{3.529469in}{0.859910in}}%
\pgfpathcurveto{\pgfqpoint{3.529469in}{0.869118in}}{\pgfqpoint{3.525810in}{0.877951in}}{\pgfqpoint{3.519299in}{0.884462in}}%
\pgfpathcurveto{\pgfqpoint{3.512788in}{0.890973in}}{\pgfqpoint{3.503955in}{0.894632in}}{\pgfqpoint{3.494747in}{0.894632in}}%
\pgfpathcurveto{\pgfqpoint{3.485538in}{0.894632in}}{\pgfqpoint{3.476706in}{0.890973in}}{\pgfqpoint{3.470194in}{0.884462in}}%
\pgfpathcurveto{\pgfqpoint{3.463683in}{0.877951in}}{\pgfqpoint{3.460024in}{0.869118in}}{\pgfqpoint{3.460024in}{0.859910in}}%
\pgfpathcurveto{\pgfqpoint{3.460024in}{0.850701in}}{\pgfqpoint{3.463683in}{0.841869in}}{\pgfqpoint{3.470194in}{0.835357in}}%
\pgfpathcurveto{\pgfqpoint{3.476706in}{0.828846in}}{\pgfqpoint{3.485538in}{0.825188in}}{\pgfqpoint{3.494747in}{0.825188in}}%
\pgfpathlineto{\pgfqpoint{3.494747in}{0.825188in}}%
\pgfpathclose%
\pgfusepath{stroke}%
\end{pgfscope}%
\begin{pgfscope}%
\pgfpathrectangle{\pgfqpoint{0.417359in}{0.814008in}}{\pgfqpoint{12.309552in}{4.722875in}}%
\pgfusepath{clip}%
\pgfsetbuttcap%
\pgfsetroundjoin%
\pgfsetlinewidth{1.003750pt}%
\definecolor{currentstroke}{rgb}{0.450000,0.450000,0.450000}%
\pgfsetstrokecolor{currentstroke}%
\pgfsetdash{}{0pt}%
\pgfpathmoveto{\pgfqpoint{3.494747in}{1.100814in}}%
\pgfpathcurveto{\pgfqpoint{3.503955in}{1.100814in}}{\pgfqpoint{3.512788in}{1.104472in}}{\pgfqpoint{3.519299in}{1.110984in}}%
\pgfpathcurveto{\pgfqpoint{3.525810in}{1.117495in}}{\pgfqpoint{3.529469in}{1.126328in}}{\pgfqpoint{3.529469in}{1.135536in}}%
\pgfpathcurveto{\pgfqpoint{3.529469in}{1.144744in}}{\pgfqpoint{3.525810in}{1.153577in}}{\pgfqpoint{3.519299in}{1.160088in}}%
\pgfpathcurveto{\pgfqpoint{3.512788in}{1.166600in}}{\pgfqpoint{3.503955in}{1.170258in}}{\pgfqpoint{3.494747in}{1.170258in}}%
\pgfpathcurveto{\pgfqpoint{3.485538in}{1.170258in}}{\pgfqpoint{3.476706in}{1.166600in}}{\pgfqpoint{3.470194in}{1.160088in}}%
\pgfpathcurveto{\pgfqpoint{3.463683in}{1.153577in}}{\pgfqpoint{3.460024in}{1.144744in}}{\pgfqpoint{3.460024in}{1.135536in}}%
\pgfpathcurveto{\pgfqpoint{3.460024in}{1.126328in}}{\pgfqpoint{3.463683in}{1.117495in}}{\pgfqpoint{3.470194in}{1.110984in}}%
\pgfpathcurveto{\pgfqpoint{3.476706in}{1.104472in}}{\pgfqpoint{3.485538in}{1.100814in}}{\pgfqpoint{3.494747in}{1.100814in}}%
\pgfpathlineto{\pgfqpoint{3.494747in}{1.100814in}}%
\pgfpathclose%
\pgfusepath{stroke}%
\end{pgfscope}%
\begin{pgfscope}%
\pgfpathrectangle{\pgfqpoint{0.417359in}{0.814008in}}{\pgfqpoint{12.309552in}{4.722875in}}%
\pgfusepath{clip}%
\pgfsetbuttcap%
\pgfsetroundjoin%
\pgfsetlinewidth{1.003750pt}%
\definecolor{currentstroke}{rgb}{0.450000,0.450000,0.450000}%
\pgfsetstrokecolor{currentstroke}%
\pgfsetdash{}{0pt}%
\pgfpathmoveto{\pgfqpoint{3.494747in}{1.100814in}}%
\pgfpathcurveto{\pgfqpoint{3.503955in}{1.100814in}}{\pgfqpoint{3.512788in}{1.104472in}}{\pgfqpoint{3.519299in}{1.110984in}}%
\pgfpathcurveto{\pgfqpoint{3.525810in}{1.117495in}}{\pgfqpoint{3.529469in}{1.126328in}}{\pgfqpoint{3.529469in}{1.135536in}}%
\pgfpathcurveto{\pgfqpoint{3.529469in}{1.144744in}}{\pgfqpoint{3.525810in}{1.153577in}}{\pgfqpoint{3.519299in}{1.160088in}}%
\pgfpathcurveto{\pgfqpoint{3.512788in}{1.166600in}}{\pgfqpoint{3.503955in}{1.170258in}}{\pgfqpoint{3.494747in}{1.170258in}}%
\pgfpathcurveto{\pgfqpoint{3.485538in}{1.170258in}}{\pgfqpoint{3.476706in}{1.166600in}}{\pgfqpoint{3.470194in}{1.160088in}}%
\pgfpathcurveto{\pgfqpoint{3.463683in}{1.153577in}}{\pgfqpoint{3.460024in}{1.144744in}}{\pgfqpoint{3.460024in}{1.135536in}}%
\pgfpathcurveto{\pgfqpoint{3.460024in}{1.126328in}}{\pgfqpoint{3.463683in}{1.117495in}}{\pgfqpoint{3.470194in}{1.110984in}}%
\pgfpathcurveto{\pgfqpoint{3.476706in}{1.104472in}}{\pgfqpoint{3.485538in}{1.100814in}}{\pgfqpoint{3.494747in}{1.100814in}}%
\pgfpathlineto{\pgfqpoint{3.494747in}{1.100814in}}%
\pgfpathclose%
\pgfusepath{stroke}%
\end{pgfscope}%
\begin{pgfscope}%
\pgfpathrectangle{\pgfqpoint{0.417359in}{0.814008in}}{\pgfqpoint{12.309552in}{4.722875in}}%
\pgfusepath{clip}%
\pgfsetbuttcap%
\pgfsetroundjoin%
\pgfsetlinewidth{1.003750pt}%
\definecolor{currentstroke}{rgb}{0.450000,0.450000,0.450000}%
\pgfsetstrokecolor{currentstroke}%
\pgfsetdash{}{0pt}%
\pgfpathmoveto{\pgfqpoint{3.494747in}{1.100814in}}%
\pgfpathcurveto{\pgfqpoint{3.503955in}{1.100814in}}{\pgfqpoint{3.512788in}{1.104472in}}{\pgfqpoint{3.519299in}{1.110984in}}%
\pgfpathcurveto{\pgfqpoint{3.525810in}{1.117495in}}{\pgfqpoint{3.529469in}{1.126328in}}{\pgfqpoint{3.529469in}{1.135536in}}%
\pgfpathcurveto{\pgfqpoint{3.529469in}{1.144744in}}{\pgfqpoint{3.525810in}{1.153577in}}{\pgfqpoint{3.519299in}{1.160088in}}%
\pgfpathcurveto{\pgfqpoint{3.512788in}{1.166600in}}{\pgfqpoint{3.503955in}{1.170258in}}{\pgfqpoint{3.494747in}{1.170258in}}%
\pgfpathcurveto{\pgfqpoint{3.485538in}{1.170258in}}{\pgfqpoint{3.476706in}{1.166600in}}{\pgfqpoint{3.470194in}{1.160088in}}%
\pgfpathcurveto{\pgfqpoint{3.463683in}{1.153577in}}{\pgfqpoint{3.460024in}{1.144744in}}{\pgfqpoint{3.460024in}{1.135536in}}%
\pgfpathcurveto{\pgfqpoint{3.460024in}{1.126328in}}{\pgfqpoint{3.463683in}{1.117495in}}{\pgfqpoint{3.470194in}{1.110984in}}%
\pgfpathcurveto{\pgfqpoint{3.476706in}{1.104472in}}{\pgfqpoint{3.485538in}{1.100814in}}{\pgfqpoint{3.494747in}{1.100814in}}%
\pgfpathlineto{\pgfqpoint{3.494747in}{1.100814in}}%
\pgfpathclose%
\pgfusepath{stroke}%
\end{pgfscope}%
\begin{pgfscope}%
\pgfpathrectangle{\pgfqpoint{0.417359in}{0.814008in}}{\pgfqpoint{12.309552in}{4.722875in}}%
\pgfusepath{clip}%
\pgfsetbuttcap%
\pgfsetroundjoin%
\pgfsetlinewidth{1.003750pt}%
\definecolor{currentstroke}{rgb}{0.450000,0.450000,0.450000}%
\pgfsetstrokecolor{currentstroke}%
\pgfsetdash{}{0pt}%
\pgfpathmoveto{\pgfqpoint{3.494747in}{0.854078in}}%
\pgfpathcurveto{\pgfqpoint{3.503955in}{0.854078in}}{\pgfqpoint{3.512788in}{0.857736in}}{\pgfqpoint{3.519299in}{0.864248in}}%
\pgfpathcurveto{\pgfqpoint{3.525810in}{0.870759in}}{\pgfqpoint{3.529469in}{0.879592in}}{\pgfqpoint{3.529469in}{0.888800in}}%
\pgfpathcurveto{\pgfqpoint{3.529469in}{0.898008in}}{\pgfqpoint{3.525810in}{0.906841in}}{\pgfqpoint{3.519299in}{0.913352in}}%
\pgfpathcurveto{\pgfqpoint{3.512788in}{0.919864in}}{\pgfqpoint{3.503955in}{0.923522in}}{\pgfqpoint{3.494747in}{0.923522in}}%
\pgfpathcurveto{\pgfqpoint{3.485538in}{0.923522in}}{\pgfqpoint{3.476706in}{0.919864in}}{\pgfqpoint{3.470194in}{0.913352in}}%
\pgfpathcurveto{\pgfqpoint{3.463683in}{0.906841in}}{\pgfqpoint{3.460024in}{0.898008in}}{\pgfqpoint{3.460024in}{0.888800in}}%
\pgfpathcurveto{\pgfqpoint{3.460024in}{0.879592in}}{\pgfqpoint{3.463683in}{0.870759in}}{\pgfqpoint{3.470194in}{0.864248in}}%
\pgfpathcurveto{\pgfqpoint{3.476706in}{0.857736in}}{\pgfqpoint{3.485538in}{0.854078in}}{\pgfqpoint{3.494747in}{0.854078in}}%
\pgfpathlineto{\pgfqpoint{3.494747in}{0.854078in}}%
\pgfpathclose%
\pgfusepath{stroke}%
\end{pgfscope}%
\begin{pgfscope}%
\pgfpathrectangle{\pgfqpoint{0.417359in}{0.814008in}}{\pgfqpoint{12.309552in}{4.722875in}}%
\pgfusepath{clip}%
\pgfsetbuttcap%
\pgfsetroundjoin%
\pgfsetlinewidth{1.003750pt}%
\definecolor{currentstroke}{rgb}{0.450000,0.450000,0.450000}%
\pgfsetstrokecolor{currentstroke}%
\pgfsetdash{}{0pt}%
\pgfpathmoveto{\pgfqpoint{3.494747in}{0.792505in}}%
\pgfpathcurveto{\pgfqpoint{3.503955in}{0.792505in}}{\pgfqpoint{3.512788in}{0.796164in}}{\pgfqpoint{3.519299in}{0.802675in}}%
\pgfpathcurveto{\pgfqpoint{3.525810in}{0.809186in}}{\pgfqpoint{3.529469in}{0.818019in}}{\pgfqpoint{3.529469in}{0.827227in}}%
\pgfpathcurveto{\pgfqpoint{3.529469in}{0.836436in}}{\pgfqpoint{3.525810in}{0.845268in}}{\pgfqpoint{3.519299in}{0.851780in}}%
\pgfpathcurveto{\pgfqpoint{3.512788in}{0.858291in}}{\pgfqpoint{3.503955in}{0.861949in}}{\pgfqpoint{3.494747in}{0.861949in}}%
\pgfpathcurveto{\pgfqpoint{3.485538in}{0.861949in}}{\pgfqpoint{3.476706in}{0.858291in}}{\pgfqpoint{3.470194in}{0.851780in}}%
\pgfpathcurveto{\pgfqpoint{3.463683in}{0.845268in}}{\pgfqpoint{3.460024in}{0.836436in}}{\pgfqpoint{3.460024in}{0.827227in}}%
\pgfpathcurveto{\pgfqpoint{3.460024in}{0.818019in}}{\pgfqpoint{3.463683in}{0.809186in}}{\pgfqpoint{3.470194in}{0.802675in}}%
\pgfpathcurveto{\pgfqpoint{3.476706in}{0.796164in}}{\pgfqpoint{3.485538in}{0.792505in}}{\pgfqpoint{3.494747in}{0.792505in}}%
\pgfusepath{stroke}%
\end{pgfscope}%
\begin{pgfscope}%
\pgfpathrectangle{\pgfqpoint{0.417359in}{0.814008in}}{\pgfqpoint{12.309552in}{4.722875in}}%
\pgfusepath{clip}%
\pgfsetbuttcap%
\pgfsetroundjoin%
\definecolor{currentfill}{rgb}{0.753431,0.238725,0.241667}%
\pgfsetfillcolor{currentfill}%
\pgfsetlinewidth{0.752812pt}%
\definecolor{currentstroke}{rgb}{0.240000,0.240000,0.240000}%
\pgfsetstrokecolor{currentstroke}%
\pgfsetdash{}{0pt}%
\pgfsys@defobject{currentmarker}{\pgfqpoint{4.233320in}{0.819771in}}{\pgfqpoint{5.218084in}{0.862965in}}{%
\pgfpathmoveto{\pgfqpoint{4.233320in}{0.819771in}}%
\pgfpathlineto{\pgfqpoint{5.218084in}{0.819771in}}%
\pgfpathlineto{\pgfqpoint{5.218084in}{0.862965in}}%
\pgfpathlineto{\pgfqpoint{4.233320in}{0.862965in}}%
\pgfpathlineto{\pgfqpoint{4.233320in}{0.819771in}}%
\pgfpathclose%
\pgfusepath{stroke,fill}%
}%
\begin{pgfscope}%
\pgfsys@transformshift{0.000000in}{0.000000in}%
\pgfsys@useobject{currentmarker}{}%
\end{pgfscope}%
\end{pgfscope}%
\begin{pgfscope}%
\pgfpathrectangle{\pgfqpoint{0.417359in}{0.814008in}}{\pgfqpoint{12.309552in}{4.722875in}}%
\pgfusepath{clip}%
\pgfsetbuttcap%
\pgfsetroundjoin%
\pgfsetlinewidth{1.003750pt}%
\definecolor{currentstroke}{rgb}{0.450000,0.450000,0.450000}%
\pgfsetstrokecolor{currentstroke}%
\pgfsetdash{}{0pt}%
\pgfpathmoveto{\pgfqpoint{4.725702in}{0.965960in}}%
\pgfpathcurveto{\pgfqpoint{4.734910in}{0.965960in}}{\pgfqpoint{4.743743in}{0.969619in}}{\pgfqpoint{4.750254in}{0.976130in}}%
\pgfpathcurveto{\pgfqpoint{4.756765in}{0.982641in}}{\pgfqpoint{4.760424in}{0.991474in}}{\pgfqpoint{4.760424in}{1.000682in}}%
\pgfpathcurveto{\pgfqpoint{4.760424in}{1.009891in}}{\pgfqpoint{4.756765in}{1.018723in}}{\pgfqpoint{4.750254in}{1.025235in}}%
\pgfpathcurveto{\pgfqpoint{4.743743in}{1.031746in}}{\pgfqpoint{4.734910in}{1.035404in}}{\pgfqpoint{4.725702in}{1.035404in}}%
\pgfpathcurveto{\pgfqpoint{4.716493in}{1.035404in}}{\pgfqpoint{4.707661in}{1.031746in}}{\pgfqpoint{4.701149in}{1.025235in}}%
\pgfpathcurveto{\pgfqpoint{4.694638in}{1.018723in}}{\pgfqpoint{4.690980in}{1.009891in}}{\pgfqpoint{4.690980in}{1.000682in}}%
\pgfpathcurveto{\pgfqpoint{4.690980in}{0.991474in}}{\pgfqpoint{4.694638in}{0.982641in}}{\pgfqpoint{4.701149in}{0.976130in}}%
\pgfpathcurveto{\pgfqpoint{4.707661in}{0.969619in}}{\pgfqpoint{4.716493in}{0.965960in}}{\pgfqpoint{4.725702in}{0.965960in}}%
\pgfpathlineto{\pgfqpoint{4.725702in}{0.965960in}}%
\pgfpathclose%
\pgfusepath{stroke}%
\end{pgfscope}%
\begin{pgfscope}%
\pgfpathrectangle{\pgfqpoint{0.417359in}{0.814008in}}{\pgfqpoint{12.309552in}{4.722875in}}%
\pgfusepath{clip}%
\pgfsetbuttcap%
\pgfsetroundjoin%
\pgfsetlinewidth{1.003750pt}%
\definecolor{currentstroke}{rgb}{0.450000,0.450000,0.450000}%
\pgfsetstrokecolor{currentstroke}%
\pgfsetdash{}{0pt}%
\pgfpathmoveto{\pgfqpoint{4.725702in}{0.965960in}}%
\pgfpathcurveto{\pgfqpoint{4.734910in}{0.965960in}}{\pgfqpoint{4.743743in}{0.969619in}}{\pgfqpoint{4.750254in}{0.976130in}}%
\pgfpathcurveto{\pgfqpoint{4.756765in}{0.982641in}}{\pgfqpoint{4.760424in}{0.991474in}}{\pgfqpoint{4.760424in}{1.000682in}}%
\pgfpathcurveto{\pgfqpoint{4.760424in}{1.009891in}}{\pgfqpoint{4.756765in}{1.018723in}}{\pgfqpoint{4.750254in}{1.025235in}}%
\pgfpathcurveto{\pgfqpoint{4.743743in}{1.031746in}}{\pgfqpoint{4.734910in}{1.035404in}}{\pgfqpoint{4.725702in}{1.035404in}}%
\pgfpathcurveto{\pgfqpoint{4.716493in}{1.035404in}}{\pgfqpoint{4.707661in}{1.031746in}}{\pgfqpoint{4.701149in}{1.025235in}}%
\pgfpathcurveto{\pgfqpoint{4.694638in}{1.018723in}}{\pgfqpoint{4.690980in}{1.009891in}}{\pgfqpoint{4.690980in}{1.000682in}}%
\pgfpathcurveto{\pgfqpoint{4.690980in}{0.991474in}}{\pgfqpoint{4.694638in}{0.982641in}}{\pgfqpoint{4.701149in}{0.976130in}}%
\pgfpathcurveto{\pgfqpoint{4.707661in}{0.969619in}}{\pgfqpoint{4.716493in}{0.965960in}}{\pgfqpoint{4.725702in}{0.965960in}}%
\pgfpathlineto{\pgfqpoint{4.725702in}{0.965960in}}%
\pgfpathclose%
\pgfusepath{stroke}%
\end{pgfscope}%
\begin{pgfscope}%
\pgfpathrectangle{\pgfqpoint{0.417359in}{0.814008in}}{\pgfqpoint{12.309552in}{4.722875in}}%
\pgfusepath{clip}%
\pgfsetbuttcap%
\pgfsetroundjoin%
\pgfsetlinewidth{1.003750pt}%
\definecolor{currentstroke}{rgb}{0.450000,0.450000,0.450000}%
\pgfsetstrokecolor{currentstroke}%
\pgfsetdash{}{0pt}%
\pgfpathmoveto{\pgfqpoint{4.725702in}{0.954739in}}%
\pgfpathcurveto{\pgfqpoint{4.734910in}{0.954739in}}{\pgfqpoint{4.743743in}{0.958397in}}{\pgfqpoint{4.750254in}{0.964909in}}%
\pgfpathcurveto{\pgfqpoint{4.756765in}{0.971420in}}{\pgfqpoint{4.760424in}{0.980253in}}{\pgfqpoint{4.760424in}{0.989461in}}%
\pgfpathcurveto{\pgfqpoint{4.760424in}{0.998669in}}{\pgfqpoint{4.756765in}{1.007502in}}{\pgfqpoint{4.750254in}{1.014013in}}%
\pgfpathcurveto{\pgfqpoint{4.743743in}{1.020525in}}{\pgfqpoint{4.734910in}{1.024183in}}{\pgfqpoint{4.725702in}{1.024183in}}%
\pgfpathcurveto{\pgfqpoint{4.716493in}{1.024183in}}{\pgfqpoint{4.707661in}{1.020525in}}{\pgfqpoint{4.701149in}{1.014013in}}%
\pgfpathcurveto{\pgfqpoint{4.694638in}{1.007502in}}{\pgfqpoint{4.690980in}{0.998669in}}{\pgfqpoint{4.690980in}{0.989461in}}%
\pgfpathcurveto{\pgfqpoint{4.690980in}{0.980253in}}{\pgfqpoint{4.694638in}{0.971420in}}{\pgfqpoint{4.701149in}{0.964909in}}%
\pgfpathcurveto{\pgfqpoint{4.707661in}{0.958397in}}{\pgfqpoint{4.716493in}{0.954739in}}{\pgfqpoint{4.725702in}{0.954739in}}%
\pgfpathlineto{\pgfqpoint{4.725702in}{0.954739in}}%
\pgfpathclose%
\pgfusepath{stroke}%
\end{pgfscope}%
\begin{pgfscope}%
\pgfpathrectangle{\pgfqpoint{0.417359in}{0.814008in}}{\pgfqpoint{12.309552in}{4.722875in}}%
\pgfusepath{clip}%
\pgfsetbuttcap%
\pgfsetroundjoin%
\pgfsetlinewidth{1.003750pt}%
\definecolor{currentstroke}{rgb}{0.450000,0.450000,0.450000}%
\pgfsetstrokecolor{currentstroke}%
\pgfsetdash{}{0pt}%
\pgfpathmoveto{\pgfqpoint{4.725702in}{0.884154in}}%
\pgfpathcurveto{\pgfqpoint{4.734910in}{0.884154in}}{\pgfqpoint{4.743743in}{0.887813in}}{\pgfqpoint{4.750254in}{0.894324in}}%
\pgfpathcurveto{\pgfqpoint{4.756765in}{0.900836in}}{\pgfqpoint{4.760424in}{0.909668in}}{\pgfqpoint{4.760424in}{0.918877in}}%
\pgfpathcurveto{\pgfqpoint{4.760424in}{0.928085in}}{\pgfqpoint{4.756765in}{0.936918in}}{\pgfqpoint{4.750254in}{0.943429in}}%
\pgfpathcurveto{\pgfqpoint{4.743743in}{0.949940in}}{\pgfqpoint{4.734910in}{0.953599in}}{\pgfqpoint{4.725702in}{0.953599in}}%
\pgfpathcurveto{\pgfqpoint{4.716493in}{0.953599in}}{\pgfqpoint{4.707661in}{0.949940in}}{\pgfqpoint{4.701149in}{0.943429in}}%
\pgfpathcurveto{\pgfqpoint{4.694638in}{0.936918in}}{\pgfqpoint{4.690980in}{0.928085in}}{\pgfqpoint{4.690980in}{0.918877in}}%
\pgfpathcurveto{\pgfqpoint{4.690980in}{0.909668in}}{\pgfqpoint{4.694638in}{0.900836in}}{\pgfqpoint{4.701149in}{0.894324in}}%
\pgfpathcurveto{\pgfqpoint{4.707661in}{0.887813in}}{\pgfqpoint{4.716493in}{0.884154in}}{\pgfqpoint{4.725702in}{0.884154in}}%
\pgfpathlineto{\pgfqpoint{4.725702in}{0.884154in}}%
\pgfpathclose%
\pgfusepath{stroke}%
\end{pgfscope}%
\begin{pgfscope}%
\pgfpathrectangle{\pgfqpoint{0.417359in}{0.814008in}}{\pgfqpoint{12.309552in}{4.722875in}}%
\pgfusepath{clip}%
\pgfsetbuttcap%
\pgfsetroundjoin%
\pgfsetlinewidth{1.003750pt}%
\definecolor{currentstroke}{rgb}{0.450000,0.450000,0.450000}%
\pgfsetstrokecolor{currentstroke}%
\pgfsetdash{}{0pt}%
\pgfpathmoveto{\pgfqpoint{4.725702in}{0.784827in}}%
\pgfpathcurveto{\pgfqpoint{4.734910in}{0.784827in}}{\pgfqpoint{4.743743in}{0.788485in}}{\pgfqpoint{4.750254in}{0.794997in}}%
\pgfpathcurveto{\pgfqpoint{4.756765in}{0.801508in}}{\pgfqpoint{4.760424in}{0.810340in}}{\pgfqpoint{4.760424in}{0.819549in}}%
\pgfpathcurveto{\pgfqpoint{4.760424in}{0.828757in}}{\pgfqpoint{4.756765in}{0.837590in}}{\pgfqpoint{4.750254in}{0.844101in}}%
\pgfpathcurveto{\pgfqpoint{4.743743in}{0.850613in}}{\pgfqpoint{4.734910in}{0.854271in}}{\pgfqpoint{4.725702in}{0.854271in}}%
\pgfpathcurveto{\pgfqpoint{4.716493in}{0.854271in}}{\pgfqpoint{4.707661in}{0.850613in}}{\pgfqpoint{4.701149in}{0.844101in}}%
\pgfpathcurveto{\pgfqpoint{4.694638in}{0.837590in}}{\pgfqpoint{4.690980in}{0.828757in}}{\pgfqpoint{4.690980in}{0.819549in}}%
\pgfpathcurveto{\pgfqpoint{4.690980in}{0.810340in}}{\pgfqpoint{4.694638in}{0.801508in}}{\pgfqpoint{4.701149in}{0.794997in}}%
\pgfpathcurveto{\pgfqpoint{4.707661in}{0.788485in}}{\pgfqpoint{4.716493in}{0.784827in}}{\pgfqpoint{4.725702in}{0.784827in}}%
\pgfusepath{stroke}%
\end{pgfscope}%
\begin{pgfscope}%
\pgfpathrectangle{\pgfqpoint{0.417359in}{0.814008in}}{\pgfqpoint{12.309552in}{4.722875in}}%
\pgfusepath{clip}%
\pgfsetbuttcap%
\pgfsetroundjoin%
\pgfsetlinewidth{1.003750pt}%
\definecolor{currentstroke}{rgb}{0.450000,0.450000,0.450000}%
\pgfsetstrokecolor{currentstroke}%
\pgfsetdash{}{0pt}%
\pgfpathmoveto{\pgfqpoint{4.725702in}{0.830698in}}%
\pgfpathcurveto{\pgfqpoint{4.734910in}{0.830698in}}{\pgfqpoint{4.743743in}{0.834356in}}{\pgfqpoint{4.750254in}{0.840867in}}%
\pgfpathcurveto{\pgfqpoint{4.756765in}{0.847379in}}{\pgfqpoint{4.760424in}{0.856211in}}{\pgfqpoint{4.760424in}{0.865420in}}%
\pgfpathcurveto{\pgfqpoint{4.760424in}{0.874628in}}{\pgfqpoint{4.756765in}{0.883461in}}{\pgfqpoint{4.750254in}{0.889972in}}%
\pgfpathcurveto{\pgfqpoint{4.743743in}{0.896483in}}{\pgfqpoint{4.734910in}{0.900142in}}{\pgfqpoint{4.725702in}{0.900142in}}%
\pgfpathcurveto{\pgfqpoint{4.716493in}{0.900142in}}{\pgfqpoint{4.707661in}{0.896483in}}{\pgfqpoint{4.701149in}{0.889972in}}%
\pgfpathcurveto{\pgfqpoint{4.694638in}{0.883461in}}{\pgfqpoint{4.690980in}{0.874628in}}{\pgfqpoint{4.690980in}{0.865420in}}%
\pgfpathcurveto{\pgfqpoint{4.690980in}{0.856211in}}{\pgfqpoint{4.694638in}{0.847379in}}{\pgfqpoint{4.701149in}{0.840867in}}%
\pgfpathcurveto{\pgfqpoint{4.707661in}{0.834356in}}{\pgfqpoint{4.716493in}{0.830698in}}{\pgfqpoint{4.725702in}{0.830698in}}%
\pgfpathlineto{\pgfqpoint{4.725702in}{0.830698in}}%
\pgfpathclose%
\pgfusepath{stroke}%
\end{pgfscope}%
\begin{pgfscope}%
\pgfpathrectangle{\pgfqpoint{0.417359in}{0.814008in}}{\pgfqpoint{12.309552in}{4.722875in}}%
\pgfusepath{clip}%
\pgfsetbuttcap%
\pgfsetroundjoin%
\pgfsetlinewidth{1.003750pt}%
\definecolor{currentstroke}{rgb}{0.450000,0.450000,0.450000}%
\pgfsetstrokecolor{currentstroke}%
\pgfsetdash{}{0pt}%
\pgfpathmoveto{\pgfqpoint{4.725702in}{0.866354in}}%
\pgfpathcurveto{\pgfqpoint{4.734910in}{0.866354in}}{\pgfqpoint{4.743743in}{0.870012in}}{\pgfqpoint{4.750254in}{0.876524in}}%
\pgfpathcurveto{\pgfqpoint{4.756765in}{0.883035in}}{\pgfqpoint{4.760424in}{0.891868in}}{\pgfqpoint{4.760424in}{0.901076in}}%
\pgfpathcurveto{\pgfqpoint{4.760424in}{0.910285in}}{\pgfqpoint{4.756765in}{0.919117in}}{\pgfqpoint{4.750254in}{0.925628in}}%
\pgfpathcurveto{\pgfqpoint{4.743743in}{0.932140in}}{\pgfqpoint{4.734910in}{0.935798in}}{\pgfqpoint{4.725702in}{0.935798in}}%
\pgfpathcurveto{\pgfqpoint{4.716493in}{0.935798in}}{\pgfqpoint{4.707661in}{0.932140in}}{\pgfqpoint{4.701149in}{0.925628in}}%
\pgfpathcurveto{\pgfqpoint{4.694638in}{0.919117in}}{\pgfqpoint{4.690980in}{0.910285in}}{\pgfqpoint{4.690980in}{0.901076in}}%
\pgfpathcurveto{\pgfqpoint{4.690980in}{0.891868in}}{\pgfqpoint{4.694638in}{0.883035in}}{\pgfqpoint{4.701149in}{0.876524in}}%
\pgfpathcurveto{\pgfqpoint{4.707661in}{0.870012in}}{\pgfqpoint{4.716493in}{0.866354in}}{\pgfqpoint{4.725702in}{0.866354in}}%
\pgfpathlineto{\pgfqpoint{4.725702in}{0.866354in}}%
\pgfpathclose%
\pgfusepath{stroke}%
\end{pgfscope}%
\begin{pgfscope}%
\pgfpathrectangle{\pgfqpoint{0.417359in}{0.814008in}}{\pgfqpoint{12.309552in}{4.722875in}}%
\pgfusepath{clip}%
\pgfsetbuttcap%
\pgfsetroundjoin%
\pgfsetlinewidth{1.003750pt}%
\definecolor{currentstroke}{rgb}{0.450000,0.450000,0.450000}%
\pgfsetstrokecolor{currentstroke}%
\pgfsetdash{}{0pt}%
\pgfpathmoveto{\pgfqpoint{4.725702in}{0.783823in}}%
\pgfpathcurveto{\pgfqpoint{4.734910in}{0.783823in}}{\pgfqpoint{4.743743in}{0.787482in}}{\pgfqpoint{4.750254in}{0.793993in}}%
\pgfpathcurveto{\pgfqpoint{4.756765in}{0.800504in}}{\pgfqpoint{4.760424in}{0.809337in}}{\pgfqpoint{4.760424in}{0.818545in}}%
\pgfpathcurveto{\pgfqpoint{4.760424in}{0.827754in}}{\pgfqpoint{4.756765in}{0.836586in}}{\pgfqpoint{4.750254in}{0.843098in}}%
\pgfpathcurveto{\pgfqpoint{4.743743in}{0.849609in}}{\pgfqpoint{4.734910in}{0.853268in}}{\pgfqpoint{4.725702in}{0.853268in}}%
\pgfpathcurveto{\pgfqpoint{4.716493in}{0.853268in}}{\pgfqpoint{4.707661in}{0.849609in}}{\pgfqpoint{4.701149in}{0.843098in}}%
\pgfpathcurveto{\pgfqpoint{4.694638in}{0.836586in}}{\pgfqpoint{4.690980in}{0.827754in}}{\pgfqpoint{4.690980in}{0.818545in}}%
\pgfpathcurveto{\pgfqpoint{4.690980in}{0.809337in}}{\pgfqpoint{4.694638in}{0.800504in}}{\pgfqpoint{4.701149in}{0.793993in}}%
\pgfpathcurveto{\pgfqpoint{4.707661in}{0.787482in}}{\pgfqpoint{4.716493in}{0.783823in}}{\pgfqpoint{4.725702in}{0.783823in}}%
\pgfusepath{stroke}%
\end{pgfscope}%
\begin{pgfscope}%
\pgfpathrectangle{\pgfqpoint{0.417359in}{0.814008in}}{\pgfqpoint{12.309552in}{4.722875in}}%
\pgfusepath{clip}%
\pgfsetbuttcap%
\pgfsetroundjoin%
\pgfsetlinewidth{1.003750pt}%
\definecolor{currentstroke}{rgb}{0.450000,0.450000,0.450000}%
\pgfsetstrokecolor{currentstroke}%
\pgfsetdash{}{0pt}%
\pgfpathmoveto{\pgfqpoint{4.725702in}{0.784416in}}%
\pgfpathcurveto{\pgfqpoint{4.734910in}{0.784416in}}{\pgfqpoint{4.743743in}{0.788075in}}{\pgfqpoint{4.750254in}{0.794586in}}%
\pgfpathcurveto{\pgfqpoint{4.756765in}{0.801097in}}{\pgfqpoint{4.760424in}{0.809930in}}{\pgfqpoint{4.760424in}{0.819138in}}%
\pgfpathcurveto{\pgfqpoint{4.760424in}{0.828347in}}{\pgfqpoint{4.756765in}{0.837179in}}{\pgfqpoint{4.750254in}{0.843691in}}%
\pgfpathcurveto{\pgfqpoint{4.743743in}{0.850202in}}{\pgfqpoint{4.734910in}{0.853861in}}{\pgfqpoint{4.725702in}{0.853861in}}%
\pgfpathcurveto{\pgfqpoint{4.716493in}{0.853861in}}{\pgfqpoint{4.707661in}{0.850202in}}{\pgfqpoint{4.701149in}{0.843691in}}%
\pgfpathcurveto{\pgfqpoint{4.694638in}{0.837179in}}{\pgfqpoint{4.690980in}{0.828347in}}{\pgfqpoint{4.690980in}{0.819138in}}%
\pgfpathcurveto{\pgfqpoint{4.690980in}{0.809930in}}{\pgfqpoint{4.694638in}{0.801097in}}{\pgfqpoint{4.701149in}{0.794586in}}%
\pgfpathcurveto{\pgfqpoint{4.707661in}{0.788075in}}{\pgfqpoint{4.716493in}{0.784416in}}{\pgfqpoint{4.725702in}{0.784416in}}%
\pgfusepath{stroke}%
\end{pgfscope}%
\begin{pgfscope}%
\pgfpathrectangle{\pgfqpoint{0.417359in}{0.814008in}}{\pgfqpoint{12.309552in}{4.722875in}}%
\pgfusepath{clip}%
\pgfsetbuttcap%
\pgfsetroundjoin%
\pgfsetlinewidth{1.003750pt}%
\definecolor{currentstroke}{rgb}{0.450000,0.450000,0.450000}%
\pgfsetstrokecolor{currentstroke}%
\pgfsetdash{}{0pt}%
\pgfpathmoveto{\pgfqpoint{4.725702in}{0.785018in}}%
\pgfpathcurveto{\pgfqpoint{4.734910in}{0.785018in}}{\pgfqpoint{4.743743in}{0.788677in}}{\pgfqpoint{4.750254in}{0.795188in}}%
\pgfpathcurveto{\pgfqpoint{4.756765in}{0.801700in}}{\pgfqpoint{4.760424in}{0.810532in}}{\pgfqpoint{4.760424in}{0.819741in}}%
\pgfpathcurveto{\pgfqpoint{4.760424in}{0.828949in}}{\pgfqpoint{4.756765in}{0.837782in}}{\pgfqpoint{4.750254in}{0.844293in}}%
\pgfpathcurveto{\pgfqpoint{4.743743in}{0.850804in}}{\pgfqpoint{4.734910in}{0.854463in}}{\pgfqpoint{4.725702in}{0.854463in}}%
\pgfpathcurveto{\pgfqpoint{4.716493in}{0.854463in}}{\pgfqpoint{4.707661in}{0.850804in}}{\pgfqpoint{4.701149in}{0.844293in}}%
\pgfpathcurveto{\pgfqpoint{4.694638in}{0.837782in}}{\pgfqpoint{4.690980in}{0.828949in}}{\pgfqpoint{4.690980in}{0.819741in}}%
\pgfpathcurveto{\pgfqpoint{4.690980in}{0.810532in}}{\pgfqpoint{4.694638in}{0.801700in}}{\pgfqpoint{4.701149in}{0.795188in}}%
\pgfpathcurveto{\pgfqpoint{4.707661in}{0.788677in}}{\pgfqpoint{4.716493in}{0.785018in}}{\pgfqpoint{4.725702in}{0.785018in}}%
\pgfusepath{stroke}%
\end{pgfscope}%
\begin{pgfscope}%
\pgfpathrectangle{\pgfqpoint{0.417359in}{0.814008in}}{\pgfqpoint{12.309552in}{4.722875in}}%
\pgfusepath{clip}%
\pgfsetbuttcap%
\pgfsetroundjoin%
\pgfsetlinewidth{1.003750pt}%
\definecolor{currentstroke}{rgb}{0.450000,0.450000,0.450000}%
\pgfsetstrokecolor{currentstroke}%
\pgfsetdash{}{0pt}%
\pgfpathmoveto{\pgfqpoint{4.725702in}{0.779558in}}%
\pgfpathcurveto{\pgfqpoint{4.734910in}{0.779558in}}{\pgfqpoint{4.743743in}{0.783216in}}{\pgfqpoint{4.750254in}{0.789727in}}%
\pgfpathcurveto{\pgfqpoint{4.756765in}{0.796239in}}{\pgfqpoint{4.760424in}{0.805071in}}{\pgfqpoint{4.760424in}{0.814280in}}%
\pgfpathcurveto{\pgfqpoint{4.760424in}{0.823488in}}{\pgfqpoint{4.756765in}{0.832321in}}{\pgfqpoint{4.750254in}{0.838832in}}%
\pgfpathcurveto{\pgfqpoint{4.743743in}{0.845343in}}{\pgfqpoint{4.734910in}{0.849002in}}{\pgfqpoint{4.725702in}{0.849002in}}%
\pgfpathcurveto{\pgfqpoint{4.716493in}{0.849002in}}{\pgfqpoint{4.707661in}{0.845343in}}{\pgfqpoint{4.701149in}{0.838832in}}%
\pgfpathcurveto{\pgfqpoint{4.694638in}{0.832321in}}{\pgfqpoint{4.690980in}{0.823488in}}{\pgfqpoint{4.690980in}{0.814280in}}%
\pgfpathcurveto{\pgfqpoint{4.690980in}{0.805071in}}{\pgfqpoint{4.694638in}{0.796239in}}{\pgfqpoint{4.701149in}{0.789727in}}%
\pgfpathcurveto{\pgfqpoint{4.707661in}{0.783216in}}{\pgfqpoint{4.716493in}{0.779558in}}{\pgfqpoint{4.725702in}{0.779558in}}%
\pgfusepath{stroke}%
\end{pgfscope}%
\begin{pgfscope}%
\pgfpathrectangle{\pgfqpoint{0.417359in}{0.814008in}}{\pgfqpoint{12.309552in}{4.722875in}}%
\pgfusepath{clip}%
\pgfsetbuttcap%
\pgfsetroundjoin%
\pgfsetlinewidth{1.003750pt}%
\definecolor{currentstroke}{rgb}{0.450000,0.450000,0.450000}%
\pgfsetstrokecolor{currentstroke}%
\pgfsetdash{}{0pt}%
\pgfpathmoveto{\pgfqpoint{4.725702in}{0.985262in}}%
\pgfpathcurveto{\pgfqpoint{4.734910in}{0.985262in}}{\pgfqpoint{4.743743in}{0.988920in}}{\pgfqpoint{4.750254in}{0.995432in}}%
\pgfpathcurveto{\pgfqpoint{4.756765in}{1.001943in}}{\pgfqpoint{4.760424in}{1.010775in}}{\pgfqpoint{4.760424in}{1.019984in}}%
\pgfpathcurveto{\pgfqpoint{4.760424in}{1.029192in}}{\pgfqpoint{4.756765in}{1.038025in}}{\pgfqpoint{4.750254in}{1.044536in}}%
\pgfpathcurveto{\pgfqpoint{4.743743in}{1.051048in}}{\pgfqpoint{4.734910in}{1.054706in}}{\pgfqpoint{4.725702in}{1.054706in}}%
\pgfpathcurveto{\pgfqpoint{4.716493in}{1.054706in}}{\pgfqpoint{4.707661in}{1.051048in}}{\pgfqpoint{4.701149in}{1.044536in}}%
\pgfpathcurveto{\pgfqpoint{4.694638in}{1.038025in}}{\pgfqpoint{4.690980in}{1.029192in}}{\pgfqpoint{4.690980in}{1.019984in}}%
\pgfpathcurveto{\pgfqpoint{4.690980in}{1.010775in}}{\pgfqpoint{4.694638in}{1.001943in}}{\pgfqpoint{4.701149in}{0.995432in}}%
\pgfpathcurveto{\pgfqpoint{4.707661in}{0.988920in}}{\pgfqpoint{4.716493in}{0.985262in}}{\pgfqpoint{4.725702in}{0.985262in}}%
\pgfpathlineto{\pgfqpoint{4.725702in}{0.985262in}}%
\pgfpathclose%
\pgfusepath{stroke}%
\end{pgfscope}%
\begin{pgfscope}%
\pgfpathrectangle{\pgfqpoint{0.417359in}{0.814008in}}{\pgfqpoint{12.309552in}{4.722875in}}%
\pgfusepath{clip}%
\pgfsetbuttcap%
\pgfsetroundjoin%
\definecolor{currentfill}{rgb}{0.578431,0.446078,0.699020}%
\pgfsetfillcolor{currentfill}%
\pgfsetlinewidth{0.752812pt}%
\definecolor{currentstroke}{rgb}{0.240000,0.240000,0.240000}%
\pgfsetstrokecolor{currentstroke}%
\pgfsetdash{}{0pt}%
\pgfsys@defobject{currentmarker}{\pgfqpoint{5.464275in}{1.056899in}}{\pgfqpoint{6.449039in}{1.327444in}}{%
\pgfpathmoveto{\pgfqpoint{5.464275in}{1.056899in}}%
\pgfpathlineto{\pgfqpoint{6.449039in}{1.056899in}}%
\pgfpathlineto{\pgfqpoint{6.449039in}{1.327444in}}%
\pgfpathlineto{\pgfqpoint{5.464275in}{1.327444in}}%
\pgfpathlineto{\pgfqpoint{5.464275in}{1.056899in}}%
\pgfpathclose%
\pgfusepath{stroke,fill}%
}%
\begin{pgfscope}%
\pgfsys@transformshift{0.000000in}{0.000000in}%
\pgfsys@useobject{currentmarker}{}%
\end{pgfscope}%
\end{pgfscope}%
\begin{pgfscope}%
\pgfpathrectangle{\pgfqpoint{0.417359in}{0.814008in}}{\pgfqpoint{12.309552in}{4.722875in}}%
\pgfusepath{clip}%
\pgfsetbuttcap%
\pgfsetroundjoin%
\pgfsetlinewidth{1.003750pt}%
\definecolor{currentstroke}{rgb}{0.450000,0.450000,0.450000}%
\pgfsetstrokecolor{currentstroke}%
\pgfsetdash{}{0pt}%
\pgfpathmoveto{\pgfqpoint{5.956657in}{3.387405in}}%
\pgfpathcurveto{\pgfqpoint{5.965865in}{3.387405in}}{\pgfqpoint{5.974698in}{3.391064in}}{\pgfqpoint{5.981209in}{3.397575in}}%
\pgfpathcurveto{\pgfqpoint{5.987721in}{3.404086in}}{\pgfqpoint{5.991379in}{3.412919in}}{\pgfqpoint{5.991379in}{3.422127in}}%
\pgfpathcurveto{\pgfqpoint{5.991379in}{3.431336in}}{\pgfqpoint{5.987721in}{3.440168in}}{\pgfqpoint{5.981209in}{3.446680in}}%
\pgfpathcurveto{\pgfqpoint{5.974698in}{3.453191in}}{\pgfqpoint{5.965865in}{3.456850in}}{\pgfqpoint{5.956657in}{3.456850in}}%
\pgfpathcurveto{\pgfqpoint{5.947448in}{3.456850in}}{\pgfqpoint{5.938616in}{3.453191in}}{\pgfqpoint{5.932105in}{3.446680in}}%
\pgfpathcurveto{\pgfqpoint{5.925593in}{3.440168in}}{\pgfqpoint{5.921935in}{3.431336in}}{\pgfqpoint{5.921935in}{3.422127in}}%
\pgfpathcurveto{\pgfqpoint{5.921935in}{3.412919in}}{\pgfqpoint{5.925593in}{3.404086in}}{\pgfqpoint{5.932105in}{3.397575in}}%
\pgfpathcurveto{\pgfqpoint{5.938616in}{3.391064in}}{\pgfqpoint{5.947448in}{3.387405in}}{\pgfqpoint{5.956657in}{3.387405in}}%
\pgfpathlineto{\pgfqpoint{5.956657in}{3.387405in}}%
\pgfpathclose%
\pgfusepath{stroke}%
\end{pgfscope}%
\begin{pgfscope}%
\pgfpathrectangle{\pgfqpoint{0.417359in}{0.814008in}}{\pgfqpoint{12.309552in}{4.722875in}}%
\pgfusepath{clip}%
\pgfsetbuttcap%
\pgfsetroundjoin%
\pgfsetlinewidth{1.003750pt}%
\definecolor{currentstroke}{rgb}{0.450000,0.450000,0.450000}%
\pgfsetstrokecolor{currentstroke}%
\pgfsetdash{}{0pt}%
\pgfpathmoveto{\pgfqpoint{5.956657in}{2.548163in}}%
\pgfpathcurveto{\pgfqpoint{5.965865in}{2.548163in}}{\pgfqpoint{5.974698in}{2.551821in}}{\pgfqpoint{5.981209in}{2.558333in}}%
\pgfpathcurveto{\pgfqpoint{5.987721in}{2.564844in}}{\pgfqpoint{5.991379in}{2.573677in}}{\pgfqpoint{5.991379in}{2.582885in}}%
\pgfpathcurveto{\pgfqpoint{5.991379in}{2.592093in}}{\pgfqpoint{5.987721in}{2.600926in}}{\pgfqpoint{5.981209in}{2.607437in}}%
\pgfpathcurveto{\pgfqpoint{5.974698in}{2.613949in}}{\pgfqpoint{5.965865in}{2.617607in}}{\pgfqpoint{5.956657in}{2.617607in}}%
\pgfpathcurveto{\pgfqpoint{5.947448in}{2.617607in}}{\pgfqpoint{5.938616in}{2.613949in}}{\pgfqpoint{5.932105in}{2.607437in}}%
\pgfpathcurveto{\pgfqpoint{5.925593in}{2.600926in}}{\pgfqpoint{5.921935in}{2.592093in}}{\pgfqpoint{5.921935in}{2.582885in}}%
\pgfpathcurveto{\pgfqpoint{5.921935in}{2.573677in}}{\pgfqpoint{5.925593in}{2.564844in}}{\pgfqpoint{5.932105in}{2.558333in}}%
\pgfpathcurveto{\pgfqpoint{5.938616in}{2.551821in}}{\pgfqpoint{5.947448in}{2.548163in}}{\pgfqpoint{5.956657in}{2.548163in}}%
\pgfpathlineto{\pgfqpoint{5.956657in}{2.548163in}}%
\pgfpathclose%
\pgfusepath{stroke}%
\end{pgfscope}%
\begin{pgfscope}%
\pgfpathrectangle{\pgfqpoint{0.417359in}{0.814008in}}{\pgfqpoint{12.309552in}{4.722875in}}%
\pgfusepath{clip}%
\pgfsetbuttcap%
\pgfsetroundjoin%
\pgfsetlinewidth{1.003750pt}%
\definecolor{currentstroke}{rgb}{0.450000,0.450000,0.450000}%
\pgfsetstrokecolor{currentstroke}%
\pgfsetdash{}{0pt}%
\pgfpathmoveto{\pgfqpoint{5.956657in}{0.905410in}}%
\pgfpathcurveto{\pgfqpoint{5.965865in}{0.905410in}}{\pgfqpoint{5.974698in}{0.909068in}}{\pgfqpoint{5.981209in}{0.915580in}}%
\pgfpathcurveto{\pgfqpoint{5.987721in}{0.922091in}}{\pgfqpoint{5.991379in}{0.930924in}}{\pgfqpoint{5.991379in}{0.940132in}}%
\pgfpathcurveto{\pgfqpoint{5.991379in}{0.949341in}}{\pgfqpoint{5.987721in}{0.958173in}}{\pgfqpoint{5.981209in}{0.964684in}}%
\pgfpathcurveto{\pgfqpoint{5.974698in}{0.971196in}}{\pgfqpoint{5.965865in}{0.974854in}}{\pgfqpoint{5.956657in}{0.974854in}}%
\pgfpathcurveto{\pgfqpoint{5.947448in}{0.974854in}}{\pgfqpoint{5.938616in}{0.971196in}}{\pgfqpoint{5.932105in}{0.964684in}}%
\pgfpathcurveto{\pgfqpoint{5.925593in}{0.958173in}}{\pgfqpoint{5.921935in}{0.949341in}}{\pgfqpoint{5.921935in}{0.940132in}}%
\pgfpathcurveto{\pgfqpoint{5.921935in}{0.930924in}}{\pgfqpoint{5.925593in}{0.922091in}}{\pgfqpoint{5.932105in}{0.915580in}}%
\pgfpathcurveto{\pgfqpoint{5.938616in}{0.909068in}}{\pgfqpoint{5.947448in}{0.905410in}}{\pgfqpoint{5.956657in}{0.905410in}}%
\pgfpathlineto{\pgfqpoint{5.956657in}{0.905410in}}%
\pgfpathclose%
\pgfusepath{stroke}%
\end{pgfscope}%
\begin{pgfscope}%
\pgfpathrectangle{\pgfqpoint{0.417359in}{0.814008in}}{\pgfqpoint{12.309552in}{4.722875in}}%
\pgfusepath{clip}%
\pgfsetbuttcap%
\pgfsetroundjoin%
\pgfsetlinewidth{1.003750pt}%
\definecolor{currentstroke}{rgb}{0.450000,0.450000,0.450000}%
\pgfsetstrokecolor{currentstroke}%
\pgfsetdash{}{0pt}%
\pgfpathmoveto{\pgfqpoint{5.956657in}{1.296748in}}%
\pgfpathcurveto{\pgfqpoint{5.965865in}{1.296748in}}{\pgfqpoint{5.974698in}{1.300407in}}{\pgfqpoint{5.981209in}{1.306918in}}%
\pgfpathcurveto{\pgfqpoint{5.987721in}{1.313429in}}{\pgfqpoint{5.991379in}{1.322262in}}{\pgfqpoint{5.991379in}{1.331470in}}%
\pgfpathcurveto{\pgfqpoint{5.991379in}{1.340679in}}{\pgfqpoint{5.987721in}{1.349511in}}{\pgfqpoint{5.981209in}{1.356023in}}%
\pgfpathcurveto{\pgfqpoint{5.974698in}{1.362534in}}{\pgfqpoint{5.965865in}{1.366192in}}{\pgfqpoint{5.956657in}{1.366192in}}%
\pgfpathcurveto{\pgfqpoint{5.947448in}{1.366192in}}{\pgfqpoint{5.938616in}{1.362534in}}{\pgfqpoint{5.932105in}{1.356023in}}%
\pgfpathcurveto{\pgfqpoint{5.925593in}{1.349511in}}{\pgfqpoint{5.921935in}{1.340679in}}{\pgfqpoint{5.921935in}{1.331470in}}%
\pgfpathcurveto{\pgfqpoint{5.921935in}{1.322262in}}{\pgfqpoint{5.925593in}{1.313429in}}{\pgfqpoint{5.932105in}{1.306918in}}%
\pgfpathcurveto{\pgfqpoint{5.938616in}{1.300407in}}{\pgfqpoint{5.947448in}{1.296748in}}{\pgfqpoint{5.956657in}{1.296748in}}%
\pgfpathlineto{\pgfqpoint{5.956657in}{1.296748in}}%
\pgfpathclose%
\pgfusepath{stroke}%
\end{pgfscope}%
\begin{pgfscope}%
\pgfpathrectangle{\pgfqpoint{0.417359in}{0.814008in}}{\pgfqpoint{12.309552in}{4.722875in}}%
\pgfusepath{clip}%
\pgfsetbuttcap%
\pgfsetroundjoin%
\pgfsetlinewidth{1.003750pt}%
\definecolor{currentstroke}{rgb}{0.450000,0.450000,0.450000}%
\pgfsetstrokecolor{currentstroke}%
\pgfsetdash{}{0pt}%
\pgfpathmoveto{\pgfqpoint{5.956657in}{1.293695in}}%
\pgfpathcurveto{\pgfqpoint{5.965865in}{1.293695in}}{\pgfqpoint{5.974698in}{1.297354in}}{\pgfqpoint{5.981209in}{1.303865in}}%
\pgfpathcurveto{\pgfqpoint{5.987721in}{1.310377in}}{\pgfqpoint{5.991379in}{1.319209in}}{\pgfqpoint{5.991379in}{1.328417in}}%
\pgfpathcurveto{\pgfqpoint{5.991379in}{1.337626in}}{\pgfqpoint{5.987721in}{1.346458in}}{\pgfqpoint{5.981209in}{1.352970in}}%
\pgfpathcurveto{\pgfqpoint{5.974698in}{1.359481in}}{\pgfqpoint{5.965865in}{1.363140in}}{\pgfqpoint{5.956657in}{1.363140in}}%
\pgfpathcurveto{\pgfqpoint{5.947448in}{1.363140in}}{\pgfqpoint{5.938616in}{1.359481in}}{\pgfqpoint{5.932105in}{1.352970in}}%
\pgfpathcurveto{\pgfqpoint{5.925593in}{1.346458in}}{\pgfqpoint{5.921935in}{1.337626in}}{\pgfqpoint{5.921935in}{1.328417in}}%
\pgfpathcurveto{\pgfqpoint{5.921935in}{1.319209in}}{\pgfqpoint{5.925593in}{1.310377in}}{\pgfqpoint{5.932105in}{1.303865in}}%
\pgfpathcurveto{\pgfqpoint{5.938616in}{1.297354in}}{\pgfqpoint{5.947448in}{1.293695in}}{\pgfqpoint{5.956657in}{1.293695in}}%
\pgfpathlineto{\pgfqpoint{5.956657in}{1.293695in}}%
\pgfpathclose%
\pgfusepath{stroke}%
\end{pgfscope}%
\begin{pgfscope}%
\pgfpathrectangle{\pgfqpoint{0.417359in}{0.814008in}}{\pgfqpoint{12.309552in}{4.722875in}}%
\pgfusepath{clip}%
\pgfsetbuttcap%
\pgfsetroundjoin%
\pgfsetlinewidth{1.003750pt}%
\definecolor{currentstroke}{rgb}{0.450000,0.450000,0.450000}%
\pgfsetstrokecolor{currentstroke}%
\pgfsetdash{}{0pt}%
\pgfpathmoveto{\pgfqpoint{5.956657in}{1.293695in}}%
\pgfpathcurveto{\pgfqpoint{5.965865in}{1.293695in}}{\pgfqpoint{5.974698in}{1.297354in}}{\pgfqpoint{5.981209in}{1.303865in}}%
\pgfpathcurveto{\pgfqpoint{5.987721in}{1.310377in}}{\pgfqpoint{5.991379in}{1.319209in}}{\pgfqpoint{5.991379in}{1.328417in}}%
\pgfpathcurveto{\pgfqpoint{5.991379in}{1.337626in}}{\pgfqpoint{5.987721in}{1.346458in}}{\pgfqpoint{5.981209in}{1.352970in}}%
\pgfpathcurveto{\pgfqpoint{5.974698in}{1.359481in}}{\pgfqpoint{5.965865in}{1.363140in}}{\pgfqpoint{5.956657in}{1.363140in}}%
\pgfpathcurveto{\pgfqpoint{5.947448in}{1.363140in}}{\pgfqpoint{5.938616in}{1.359481in}}{\pgfqpoint{5.932105in}{1.352970in}}%
\pgfpathcurveto{\pgfqpoint{5.925593in}{1.346458in}}{\pgfqpoint{5.921935in}{1.337626in}}{\pgfqpoint{5.921935in}{1.328417in}}%
\pgfpathcurveto{\pgfqpoint{5.921935in}{1.319209in}}{\pgfqpoint{5.925593in}{1.310377in}}{\pgfqpoint{5.932105in}{1.303865in}}%
\pgfpathcurveto{\pgfqpoint{5.938616in}{1.297354in}}{\pgfqpoint{5.947448in}{1.293695in}}{\pgfqpoint{5.956657in}{1.293695in}}%
\pgfpathlineto{\pgfqpoint{5.956657in}{1.293695in}}%
\pgfpathclose%
\pgfusepath{stroke}%
\end{pgfscope}%
\begin{pgfscope}%
\pgfpathrectangle{\pgfqpoint{0.417359in}{0.814008in}}{\pgfqpoint{12.309552in}{4.722875in}}%
\pgfusepath{clip}%
\pgfsetbuttcap%
\pgfsetroundjoin%
\pgfsetlinewidth{1.003750pt}%
\definecolor{currentstroke}{rgb}{0.450000,0.450000,0.450000}%
\pgfsetstrokecolor{currentstroke}%
\pgfsetdash{}{0pt}%
\pgfpathmoveto{\pgfqpoint{5.956657in}{0.890200in}}%
\pgfpathcurveto{\pgfqpoint{5.965865in}{0.890200in}}{\pgfqpoint{5.974698in}{0.893859in}}{\pgfqpoint{5.981209in}{0.900370in}}%
\pgfpathcurveto{\pgfqpoint{5.987721in}{0.906881in}}{\pgfqpoint{5.991379in}{0.915714in}}{\pgfqpoint{5.991379in}{0.924922in}}%
\pgfpathcurveto{\pgfqpoint{5.991379in}{0.934131in}}{\pgfqpoint{5.987721in}{0.942963in}}{\pgfqpoint{5.981209in}{0.949475in}}%
\pgfpathcurveto{\pgfqpoint{5.974698in}{0.955986in}}{\pgfqpoint{5.965865in}{0.959645in}}{\pgfqpoint{5.956657in}{0.959645in}}%
\pgfpathcurveto{\pgfqpoint{5.947448in}{0.959645in}}{\pgfqpoint{5.938616in}{0.955986in}}{\pgfqpoint{5.932105in}{0.949475in}}%
\pgfpathcurveto{\pgfqpoint{5.925593in}{0.942963in}}{\pgfqpoint{5.921935in}{0.934131in}}{\pgfqpoint{5.921935in}{0.924922in}}%
\pgfpathcurveto{\pgfqpoint{5.921935in}{0.915714in}}{\pgfqpoint{5.925593in}{0.906881in}}{\pgfqpoint{5.932105in}{0.900370in}}%
\pgfpathcurveto{\pgfqpoint{5.938616in}{0.893859in}}{\pgfqpoint{5.947448in}{0.890200in}}{\pgfqpoint{5.956657in}{0.890200in}}%
\pgfpathlineto{\pgfqpoint{5.956657in}{0.890200in}}%
\pgfpathclose%
\pgfusepath{stroke}%
\end{pgfscope}%
\begin{pgfscope}%
\pgfpathrectangle{\pgfqpoint{0.417359in}{0.814008in}}{\pgfqpoint{12.309552in}{4.722875in}}%
\pgfusepath{clip}%
\pgfsetbuttcap%
\pgfsetroundjoin%
\pgfsetlinewidth{1.003750pt}%
\definecolor{currentstroke}{rgb}{0.450000,0.450000,0.450000}%
\pgfsetstrokecolor{currentstroke}%
\pgfsetdash{}{0pt}%
\pgfpathmoveto{\pgfqpoint{5.956657in}{0.896807in}}%
\pgfpathcurveto{\pgfqpoint{5.965865in}{0.896807in}}{\pgfqpoint{5.974698in}{0.900466in}}{\pgfqpoint{5.981209in}{0.906977in}}%
\pgfpathcurveto{\pgfqpoint{5.987721in}{0.913489in}}{\pgfqpoint{5.991379in}{0.922321in}}{\pgfqpoint{5.991379in}{0.931530in}}%
\pgfpathcurveto{\pgfqpoint{5.991379in}{0.940738in}}{\pgfqpoint{5.987721in}{0.949571in}}{\pgfqpoint{5.981209in}{0.956082in}}%
\pgfpathcurveto{\pgfqpoint{5.974698in}{0.962593in}}{\pgfqpoint{5.965865in}{0.966252in}}{\pgfqpoint{5.956657in}{0.966252in}}%
\pgfpathcurveto{\pgfqpoint{5.947448in}{0.966252in}}{\pgfqpoint{5.938616in}{0.962593in}}{\pgfqpoint{5.932105in}{0.956082in}}%
\pgfpathcurveto{\pgfqpoint{5.925593in}{0.949571in}}{\pgfqpoint{5.921935in}{0.940738in}}{\pgfqpoint{5.921935in}{0.931530in}}%
\pgfpathcurveto{\pgfqpoint{5.921935in}{0.922321in}}{\pgfqpoint{5.925593in}{0.913489in}}{\pgfqpoint{5.932105in}{0.906977in}}%
\pgfpathcurveto{\pgfqpoint{5.938616in}{0.900466in}}{\pgfqpoint{5.947448in}{0.896807in}}{\pgfqpoint{5.956657in}{0.896807in}}%
\pgfpathlineto{\pgfqpoint{5.956657in}{0.896807in}}%
\pgfpathclose%
\pgfusepath{stroke}%
\end{pgfscope}%
\begin{pgfscope}%
\pgfpathrectangle{\pgfqpoint{0.417359in}{0.814008in}}{\pgfqpoint{12.309552in}{4.722875in}}%
\pgfusepath{clip}%
\pgfsetbuttcap%
\pgfsetroundjoin%
\pgfsetlinewidth{1.003750pt}%
\definecolor{currentstroke}{rgb}{0.450000,0.450000,0.450000}%
\pgfsetstrokecolor{currentstroke}%
\pgfsetdash{}{0pt}%
\pgfpathmoveto{\pgfqpoint{5.956657in}{0.896807in}}%
\pgfpathcurveto{\pgfqpoint{5.965865in}{0.896807in}}{\pgfqpoint{5.974698in}{0.900466in}}{\pgfqpoint{5.981209in}{0.906977in}}%
\pgfpathcurveto{\pgfqpoint{5.987721in}{0.913489in}}{\pgfqpoint{5.991379in}{0.922321in}}{\pgfqpoint{5.991379in}{0.931530in}}%
\pgfpathcurveto{\pgfqpoint{5.991379in}{0.940738in}}{\pgfqpoint{5.987721in}{0.949571in}}{\pgfqpoint{5.981209in}{0.956082in}}%
\pgfpathcurveto{\pgfqpoint{5.974698in}{0.962593in}}{\pgfqpoint{5.965865in}{0.966252in}}{\pgfqpoint{5.956657in}{0.966252in}}%
\pgfpathcurveto{\pgfqpoint{5.947448in}{0.966252in}}{\pgfqpoint{5.938616in}{0.962593in}}{\pgfqpoint{5.932105in}{0.956082in}}%
\pgfpathcurveto{\pgfqpoint{5.925593in}{0.949571in}}{\pgfqpoint{5.921935in}{0.940738in}}{\pgfqpoint{5.921935in}{0.931530in}}%
\pgfpathcurveto{\pgfqpoint{5.921935in}{0.922321in}}{\pgfqpoint{5.925593in}{0.913489in}}{\pgfqpoint{5.932105in}{0.906977in}}%
\pgfpathcurveto{\pgfqpoint{5.938616in}{0.900466in}}{\pgfqpoint{5.947448in}{0.896807in}}{\pgfqpoint{5.956657in}{0.896807in}}%
\pgfpathlineto{\pgfqpoint{5.956657in}{0.896807in}}%
\pgfpathclose%
\pgfusepath{stroke}%
\end{pgfscope}%
\begin{pgfscope}%
\pgfpathrectangle{\pgfqpoint{0.417359in}{0.814008in}}{\pgfqpoint{12.309552in}{4.722875in}}%
\pgfusepath{clip}%
\pgfsetbuttcap%
\pgfsetroundjoin%
\pgfsetlinewidth{1.003750pt}%
\definecolor{currentstroke}{rgb}{0.450000,0.450000,0.450000}%
\pgfsetstrokecolor{currentstroke}%
\pgfsetdash{}{0pt}%
\pgfpathmoveto{\pgfqpoint{5.956657in}{1.549188in}}%
\pgfpathcurveto{\pgfqpoint{5.965865in}{1.549188in}}{\pgfqpoint{5.974698in}{1.552846in}}{\pgfqpoint{5.981209in}{1.559358in}}%
\pgfpathcurveto{\pgfqpoint{5.987721in}{1.565869in}}{\pgfqpoint{5.991379in}{1.574702in}}{\pgfqpoint{5.991379in}{1.583910in}}%
\pgfpathcurveto{\pgfqpoint{5.991379in}{1.593119in}}{\pgfqpoint{5.987721in}{1.601951in}}{\pgfqpoint{5.981209in}{1.608462in}}%
\pgfpathcurveto{\pgfqpoint{5.974698in}{1.614974in}}{\pgfqpoint{5.965865in}{1.618632in}}{\pgfqpoint{5.956657in}{1.618632in}}%
\pgfpathcurveto{\pgfqpoint{5.947448in}{1.618632in}}{\pgfqpoint{5.938616in}{1.614974in}}{\pgfqpoint{5.932105in}{1.608462in}}%
\pgfpathcurveto{\pgfqpoint{5.925593in}{1.601951in}}{\pgfqpoint{5.921935in}{1.593119in}}{\pgfqpoint{5.921935in}{1.583910in}}%
\pgfpathcurveto{\pgfqpoint{5.921935in}{1.574702in}}{\pgfqpoint{5.925593in}{1.565869in}}{\pgfqpoint{5.932105in}{1.559358in}}%
\pgfpathcurveto{\pgfqpoint{5.938616in}{1.552846in}}{\pgfqpoint{5.947448in}{1.549188in}}{\pgfqpoint{5.956657in}{1.549188in}}%
\pgfpathlineto{\pgfqpoint{5.956657in}{1.549188in}}%
\pgfpathclose%
\pgfusepath{stroke}%
\end{pgfscope}%
\begin{pgfscope}%
\pgfpathrectangle{\pgfqpoint{0.417359in}{0.814008in}}{\pgfqpoint{12.309552in}{4.722875in}}%
\pgfusepath{clip}%
\pgfsetbuttcap%
\pgfsetroundjoin%
\pgfsetlinewidth{1.003750pt}%
\definecolor{currentstroke}{rgb}{0.450000,0.450000,0.450000}%
\pgfsetstrokecolor{currentstroke}%
\pgfsetdash{}{0pt}%
\pgfpathmoveto{\pgfqpoint{5.956657in}{0.959030in}}%
\pgfpathcurveto{\pgfqpoint{5.965865in}{0.959030in}}{\pgfqpoint{5.974698in}{0.962688in}}{\pgfqpoint{5.981209in}{0.969200in}}%
\pgfpathcurveto{\pgfqpoint{5.987721in}{0.975711in}}{\pgfqpoint{5.991379in}{0.984544in}}{\pgfqpoint{5.991379in}{0.993752in}}%
\pgfpathcurveto{\pgfqpoint{5.991379in}{1.002960in}}{\pgfqpoint{5.987721in}{1.011793in}}{\pgfqpoint{5.981209in}{1.018304in}}%
\pgfpathcurveto{\pgfqpoint{5.974698in}{1.024816in}}{\pgfqpoint{5.965865in}{1.028474in}}{\pgfqpoint{5.956657in}{1.028474in}}%
\pgfpathcurveto{\pgfqpoint{5.947448in}{1.028474in}}{\pgfqpoint{5.938616in}{1.024816in}}{\pgfqpoint{5.932105in}{1.018304in}}%
\pgfpathcurveto{\pgfqpoint{5.925593in}{1.011793in}}{\pgfqpoint{5.921935in}{1.002960in}}{\pgfqpoint{5.921935in}{0.993752in}}%
\pgfpathcurveto{\pgfqpoint{5.921935in}{0.984544in}}{\pgfqpoint{5.925593in}{0.975711in}}{\pgfqpoint{5.932105in}{0.969200in}}%
\pgfpathcurveto{\pgfqpoint{5.938616in}{0.962688in}}{\pgfqpoint{5.947448in}{0.959030in}}{\pgfqpoint{5.956657in}{0.959030in}}%
\pgfpathlineto{\pgfqpoint{5.956657in}{0.959030in}}%
\pgfpathclose%
\pgfusepath{stroke}%
\end{pgfscope}%
\begin{pgfscope}%
\pgfpathrectangle{\pgfqpoint{0.417359in}{0.814008in}}{\pgfqpoint{12.309552in}{4.722875in}}%
\pgfusepath{clip}%
\pgfsetbuttcap%
\pgfsetroundjoin%
\pgfsetlinewidth{1.003750pt}%
\definecolor{currentstroke}{rgb}{0.450000,0.450000,0.450000}%
\pgfsetstrokecolor{currentstroke}%
\pgfsetdash{}{0pt}%
\pgfpathmoveto{\pgfqpoint{5.956657in}{1.017612in}}%
\pgfpathcurveto{\pgfqpoint{5.965865in}{1.017612in}}{\pgfqpoint{5.974698in}{1.021271in}}{\pgfqpoint{5.981209in}{1.027782in}}%
\pgfpathcurveto{\pgfqpoint{5.987721in}{1.034294in}}{\pgfqpoint{5.991379in}{1.043126in}}{\pgfqpoint{5.991379in}{1.052335in}}%
\pgfpathcurveto{\pgfqpoint{5.991379in}{1.061543in}}{\pgfqpoint{5.987721in}{1.070376in}}{\pgfqpoint{5.981209in}{1.076887in}}%
\pgfpathcurveto{\pgfqpoint{5.974698in}{1.083398in}}{\pgfqpoint{5.965865in}{1.087057in}}{\pgfqpoint{5.956657in}{1.087057in}}%
\pgfpathcurveto{\pgfqpoint{5.947448in}{1.087057in}}{\pgfqpoint{5.938616in}{1.083398in}}{\pgfqpoint{5.932105in}{1.076887in}}%
\pgfpathcurveto{\pgfqpoint{5.925593in}{1.070376in}}{\pgfqpoint{5.921935in}{1.061543in}}{\pgfqpoint{5.921935in}{1.052335in}}%
\pgfpathcurveto{\pgfqpoint{5.921935in}{1.043126in}}{\pgfqpoint{5.925593in}{1.034294in}}{\pgfqpoint{5.932105in}{1.027782in}}%
\pgfpathcurveto{\pgfqpoint{5.938616in}{1.021271in}}{\pgfqpoint{5.947448in}{1.017612in}}{\pgfqpoint{5.956657in}{1.017612in}}%
\pgfpathlineto{\pgfqpoint{5.956657in}{1.017612in}}%
\pgfpathclose%
\pgfusepath{stroke}%
\end{pgfscope}%
\begin{pgfscope}%
\pgfpathrectangle{\pgfqpoint{0.417359in}{0.814008in}}{\pgfqpoint{12.309552in}{4.722875in}}%
\pgfusepath{clip}%
\pgfsetbuttcap%
\pgfsetroundjoin%
\pgfsetlinewidth{1.003750pt}%
\definecolor{currentstroke}{rgb}{0.450000,0.450000,0.450000}%
\pgfsetstrokecolor{currentstroke}%
\pgfsetdash{}{0pt}%
\pgfpathmoveto{\pgfqpoint{5.956657in}{0.896807in}}%
\pgfpathcurveto{\pgfqpoint{5.965865in}{0.896807in}}{\pgfqpoint{5.974698in}{0.900466in}}{\pgfqpoint{5.981209in}{0.906977in}}%
\pgfpathcurveto{\pgfqpoint{5.987721in}{0.913489in}}{\pgfqpoint{5.991379in}{0.922321in}}{\pgfqpoint{5.991379in}{0.931530in}}%
\pgfpathcurveto{\pgfqpoint{5.991379in}{0.940738in}}{\pgfqpoint{5.987721in}{0.949571in}}{\pgfqpoint{5.981209in}{0.956082in}}%
\pgfpathcurveto{\pgfqpoint{5.974698in}{0.962593in}}{\pgfqpoint{5.965865in}{0.966252in}}{\pgfqpoint{5.956657in}{0.966252in}}%
\pgfpathcurveto{\pgfqpoint{5.947448in}{0.966252in}}{\pgfqpoint{5.938616in}{0.962593in}}{\pgfqpoint{5.932105in}{0.956082in}}%
\pgfpathcurveto{\pgfqpoint{5.925593in}{0.949571in}}{\pgfqpoint{5.921935in}{0.940738in}}{\pgfqpoint{5.921935in}{0.931530in}}%
\pgfpathcurveto{\pgfqpoint{5.921935in}{0.922321in}}{\pgfqpoint{5.925593in}{0.913489in}}{\pgfqpoint{5.932105in}{0.906977in}}%
\pgfpathcurveto{\pgfqpoint{5.938616in}{0.900466in}}{\pgfqpoint{5.947448in}{0.896807in}}{\pgfqpoint{5.956657in}{0.896807in}}%
\pgfpathlineto{\pgfqpoint{5.956657in}{0.896807in}}%
\pgfpathclose%
\pgfusepath{stroke}%
\end{pgfscope}%
\begin{pgfscope}%
\pgfpathrectangle{\pgfqpoint{0.417359in}{0.814008in}}{\pgfqpoint{12.309552in}{4.722875in}}%
\pgfusepath{clip}%
\pgfsetbuttcap%
\pgfsetroundjoin%
\pgfsetlinewidth{1.003750pt}%
\definecolor{currentstroke}{rgb}{0.450000,0.450000,0.450000}%
\pgfsetstrokecolor{currentstroke}%
\pgfsetdash{}{0pt}%
\pgfpathmoveto{\pgfqpoint{5.956657in}{2.300623in}}%
\pgfpathcurveto{\pgfqpoint{5.965865in}{2.300623in}}{\pgfqpoint{5.974698in}{2.304282in}}{\pgfqpoint{5.981209in}{2.310793in}}%
\pgfpathcurveto{\pgfqpoint{5.987721in}{2.317304in}}{\pgfqpoint{5.991379in}{2.326137in}}{\pgfqpoint{5.991379in}{2.335345in}}%
\pgfpathcurveto{\pgfqpoint{5.991379in}{2.344554in}}{\pgfqpoint{5.987721in}{2.353386in}}{\pgfqpoint{5.981209in}{2.359898in}}%
\pgfpathcurveto{\pgfqpoint{5.974698in}{2.366409in}}{\pgfqpoint{5.965865in}{2.370068in}}{\pgfqpoint{5.956657in}{2.370068in}}%
\pgfpathcurveto{\pgfqpoint{5.947448in}{2.370068in}}{\pgfqpoint{5.938616in}{2.366409in}}{\pgfqpoint{5.932105in}{2.359898in}}%
\pgfpathcurveto{\pgfqpoint{5.925593in}{2.353386in}}{\pgfqpoint{5.921935in}{2.344554in}}{\pgfqpoint{5.921935in}{2.335345in}}%
\pgfpathcurveto{\pgfqpoint{5.921935in}{2.326137in}}{\pgfqpoint{5.925593in}{2.317304in}}{\pgfqpoint{5.932105in}{2.310793in}}%
\pgfpathcurveto{\pgfqpoint{5.938616in}{2.304282in}}{\pgfqpoint{5.947448in}{2.300623in}}{\pgfqpoint{5.956657in}{2.300623in}}%
\pgfpathlineto{\pgfqpoint{5.956657in}{2.300623in}}%
\pgfpathclose%
\pgfusepath{stroke}%
\end{pgfscope}%
\begin{pgfscope}%
\pgfpathrectangle{\pgfqpoint{0.417359in}{0.814008in}}{\pgfqpoint{12.309552in}{4.722875in}}%
\pgfusepath{clip}%
\pgfsetbuttcap%
\pgfsetroundjoin%
\definecolor{currentfill}{rgb}{0.517157,0.358333,0.325980}%
\pgfsetfillcolor{currentfill}%
\pgfsetlinewidth{0.752812pt}%
\definecolor{currentstroke}{rgb}{0.240000,0.240000,0.240000}%
\pgfsetstrokecolor{currentstroke}%
\pgfsetdash{}{0pt}%
\pgfsys@defobject{currentmarker}{\pgfqpoint{6.695230in}{0.885028in}}{\pgfqpoint{7.679994in}{1.240975in}}{%
\pgfpathmoveto{\pgfqpoint{6.695230in}{0.885028in}}%
\pgfpathlineto{\pgfqpoint{7.679994in}{0.885028in}}%
\pgfpathlineto{\pgfqpoint{7.679994in}{1.240975in}}%
\pgfpathlineto{\pgfqpoint{6.695230in}{1.240975in}}%
\pgfpathlineto{\pgfqpoint{6.695230in}{0.885028in}}%
\pgfpathclose%
\pgfusepath{stroke,fill}%
}%
\begin{pgfscope}%
\pgfsys@transformshift{0.000000in}{0.000000in}%
\pgfsys@useobject{currentmarker}{}%
\end{pgfscope}%
\end{pgfscope}%
\begin{pgfscope}%
\pgfpathrectangle{\pgfqpoint{0.417359in}{0.814008in}}{\pgfqpoint{12.309552in}{4.722875in}}%
\pgfusepath{clip}%
\pgfsetbuttcap%
\pgfsetroundjoin%
\pgfsetlinewidth{1.003750pt}%
\definecolor{currentstroke}{rgb}{0.450000,0.450000,0.450000}%
\pgfsetstrokecolor{currentstroke}%
\pgfsetdash{}{0pt}%
\pgfpathmoveto{\pgfqpoint{7.187612in}{1.707208in}}%
\pgfpathcurveto{\pgfqpoint{7.196821in}{1.707208in}}{\pgfqpoint{7.205653in}{1.710867in}}{\pgfqpoint{7.212164in}{1.717378in}}%
\pgfpathcurveto{\pgfqpoint{7.218676in}{1.723890in}}{\pgfqpoint{7.222334in}{1.732722in}}{\pgfqpoint{7.222334in}{1.741931in}}%
\pgfpathcurveto{\pgfqpoint{7.222334in}{1.751139in}}{\pgfqpoint{7.218676in}{1.759972in}}{\pgfqpoint{7.212164in}{1.766483in}}%
\pgfpathcurveto{\pgfqpoint{7.205653in}{1.772994in}}{\pgfqpoint{7.196821in}{1.776653in}}{\pgfqpoint{7.187612in}{1.776653in}}%
\pgfpathcurveto{\pgfqpoint{7.178404in}{1.776653in}}{\pgfqpoint{7.169571in}{1.772994in}}{\pgfqpoint{7.163060in}{1.766483in}}%
\pgfpathcurveto{\pgfqpoint{7.156548in}{1.759972in}}{\pgfqpoint{7.152890in}{1.751139in}}{\pgfqpoint{7.152890in}{1.741931in}}%
\pgfpathcurveto{\pgfqpoint{7.152890in}{1.732722in}}{\pgfqpoint{7.156548in}{1.723890in}}{\pgfqpoint{7.163060in}{1.717378in}}%
\pgfpathcurveto{\pgfqpoint{7.169571in}{1.710867in}}{\pgfqpoint{7.178404in}{1.707208in}}{\pgfqpoint{7.187612in}{1.707208in}}%
\pgfpathlineto{\pgfqpoint{7.187612in}{1.707208in}}%
\pgfpathclose%
\pgfusepath{stroke}%
\end{pgfscope}%
\begin{pgfscope}%
\pgfpathrectangle{\pgfqpoint{0.417359in}{0.814008in}}{\pgfqpoint{12.309552in}{4.722875in}}%
\pgfusepath{clip}%
\pgfsetbuttcap%
\pgfsetroundjoin%
\pgfsetlinewidth{1.003750pt}%
\definecolor{currentstroke}{rgb}{0.450000,0.450000,0.450000}%
\pgfsetstrokecolor{currentstroke}%
\pgfsetdash{}{0pt}%
\pgfpathmoveto{\pgfqpoint{7.187612in}{1.707208in}}%
\pgfpathcurveto{\pgfqpoint{7.196821in}{1.707208in}}{\pgfqpoint{7.205653in}{1.710867in}}{\pgfqpoint{7.212164in}{1.717378in}}%
\pgfpathcurveto{\pgfqpoint{7.218676in}{1.723890in}}{\pgfqpoint{7.222334in}{1.732722in}}{\pgfqpoint{7.222334in}{1.741931in}}%
\pgfpathcurveto{\pgfqpoint{7.222334in}{1.751139in}}{\pgfqpoint{7.218676in}{1.759972in}}{\pgfqpoint{7.212164in}{1.766483in}}%
\pgfpathcurveto{\pgfqpoint{7.205653in}{1.772994in}}{\pgfqpoint{7.196821in}{1.776653in}}{\pgfqpoint{7.187612in}{1.776653in}}%
\pgfpathcurveto{\pgfqpoint{7.178404in}{1.776653in}}{\pgfqpoint{7.169571in}{1.772994in}}{\pgfqpoint{7.163060in}{1.766483in}}%
\pgfpathcurveto{\pgfqpoint{7.156548in}{1.759972in}}{\pgfqpoint{7.152890in}{1.751139in}}{\pgfqpoint{7.152890in}{1.741931in}}%
\pgfpathcurveto{\pgfqpoint{7.152890in}{1.732722in}}{\pgfqpoint{7.156548in}{1.723890in}}{\pgfqpoint{7.163060in}{1.717378in}}%
\pgfpathcurveto{\pgfqpoint{7.169571in}{1.710867in}}{\pgfqpoint{7.178404in}{1.707208in}}{\pgfqpoint{7.187612in}{1.707208in}}%
\pgfpathlineto{\pgfqpoint{7.187612in}{1.707208in}}%
\pgfpathclose%
\pgfusepath{stroke}%
\end{pgfscope}%
\begin{pgfscope}%
\pgfpathrectangle{\pgfqpoint{0.417359in}{0.814008in}}{\pgfqpoint{12.309552in}{4.722875in}}%
\pgfusepath{clip}%
\pgfsetbuttcap%
\pgfsetroundjoin%
\pgfsetlinewidth{1.003750pt}%
\definecolor{currentstroke}{rgb}{0.450000,0.450000,0.450000}%
\pgfsetstrokecolor{currentstroke}%
\pgfsetdash{}{0pt}%
\pgfpathmoveto{\pgfqpoint{7.187612in}{1.508672in}}%
\pgfpathcurveto{\pgfqpoint{7.196821in}{1.508672in}}{\pgfqpoint{7.205653in}{1.512331in}}{\pgfqpoint{7.212164in}{1.518842in}}%
\pgfpathcurveto{\pgfqpoint{7.218676in}{1.525353in}}{\pgfqpoint{7.222334in}{1.534186in}}{\pgfqpoint{7.222334in}{1.543394in}}%
\pgfpathcurveto{\pgfqpoint{7.222334in}{1.552603in}}{\pgfqpoint{7.218676in}{1.561435in}}{\pgfqpoint{7.212164in}{1.567947in}}%
\pgfpathcurveto{\pgfqpoint{7.205653in}{1.574458in}}{\pgfqpoint{7.196821in}{1.578116in}}{\pgfqpoint{7.187612in}{1.578116in}}%
\pgfpathcurveto{\pgfqpoint{7.178404in}{1.578116in}}{\pgfqpoint{7.169571in}{1.574458in}}{\pgfqpoint{7.163060in}{1.567947in}}%
\pgfpathcurveto{\pgfqpoint{7.156548in}{1.561435in}}{\pgfqpoint{7.152890in}{1.552603in}}{\pgfqpoint{7.152890in}{1.543394in}}%
\pgfpathcurveto{\pgfqpoint{7.152890in}{1.534186in}}{\pgfqpoint{7.156548in}{1.525353in}}{\pgfqpoint{7.163060in}{1.518842in}}%
\pgfpathcurveto{\pgfqpoint{7.169571in}{1.512331in}}{\pgfqpoint{7.178404in}{1.508672in}}{\pgfqpoint{7.187612in}{1.508672in}}%
\pgfpathlineto{\pgfqpoint{7.187612in}{1.508672in}}%
\pgfpathclose%
\pgfusepath{stroke}%
\end{pgfscope}%
\begin{pgfscope}%
\pgfpathrectangle{\pgfqpoint{0.417359in}{0.814008in}}{\pgfqpoint{12.309552in}{4.722875in}}%
\pgfusepath{clip}%
\pgfsetbuttcap%
\pgfsetroundjoin%
\pgfsetlinewidth{1.003750pt}%
\definecolor{currentstroke}{rgb}{0.450000,0.450000,0.450000}%
\pgfsetstrokecolor{currentstroke}%
\pgfsetdash{}{0pt}%
\pgfpathmoveto{\pgfqpoint{7.187612in}{1.232225in}}%
\pgfpathcurveto{\pgfqpoint{7.196821in}{1.232225in}}{\pgfqpoint{7.205653in}{1.235884in}}{\pgfqpoint{7.212164in}{1.242395in}}%
\pgfpathcurveto{\pgfqpoint{7.218676in}{1.248907in}}{\pgfqpoint{7.222334in}{1.257739in}}{\pgfqpoint{7.222334in}{1.266948in}}%
\pgfpathcurveto{\pgfqpoint{7.222334in}{1.276156in}}{\pgfqpoint{7.218676in}{1.284989in}}{\pgfqpoint{7.212164in}{1.291500in}}%
\pgfpathcurveto{\pgfqpoint{7.205653in}{1.298011in}}{\pgfqpoint{7.196821in}{1.301670in}}{\pgfqpoint{7.187612in}{1.301670in}}%
\pgfpathcurveto{\pgfqpoint{7.178404in}{1.301670in}}{\pgfqpoint{7.169571in}{1.298011in}}{\pgfqpoint{7.163060in}{1.291500in}}%
\pgfpathcurveto{\pgfqpoint{7.156548in}{1.284989in}}{\pgfqpoint{7.152890in}{1.276156in}}{\pgfqpoint{7.152890in}{1.266948in}}%
\pgfpathcurveto{\pgfqpoint{7.152890in}{1.257739in}}{\pgfqpoint{7.156548in}{1.248907in}}{\pgfqpoint{7.163060in}{1.242395in}}%
\pgfpathcurveto{\pgfqpoint{7.169571in}{1.235884in}}{\pgfqpoint{7.178404in}{1.232225in}}{\pgfqpoint{7.187612in}{1.232225in}}%
\pgfpathlineto{\pgfqpoint{7.187612in}{1.232225in}}%
\pgfpathclose%
\pgfusepath{stroke}%
\end{pgfscope}%
\begin{pgfscope}%
\pgfpathrectangle{\pgfqpoint{0.417359in}{0.814008in}}{\pgfqpoint{12.309552in}{4.722875in}}%
\pgfusepath{clip}%
\pgfsetbuttcap%
\pgfsetroundjoin%
\pgfsetlinewidth{1.003750pt}%
\definecolor{currentstroke}{rgb}{0.450000,0.450000,0.450000}%
\pgfsetstrokecolor{currentstroke}%
\pgfsetdash{}{0pt}%
\pgfpathmoveto{\pgfqpoint{7.187612in}{1.232225in}}%
\pgfpathcurveto{\pgfqpoint{7.196821in}{1.232225in}}{\pgfqpoint{7.205653in}{1.235884in}}{\pgfqpoint{7.212164in}{1.242395in}}%
\pgfpathcurveto{\pgfqpoint{7.218676in}{1.248907in}}{\pgfqpoint{7.222334in}{1.257739in}}{\pgfqpoint{7.222334in}{1.266948in}}%
\pgfpathcurveto{\pgfqpoint{7.222334in}{1.276156in}}{\pgfqpoint{7.218676in}{1.284989in}}{\pgfqpoint{7.212164in}{1.291500in}}%
\pgfpathcurveto{\pgfqpoint{7.205653in}{1.298011in}}{\pgfqpoint{7.196821in}{1.301670in}}{\pgfqpoint{7.187612in}{1.301670in}}%
\pgfpathcurveto{\pgfqpoint{7.178404in}{1.301670in}}{\pgfqpoint{7.169571in}{1.298011in}}{\pgfqpoint{7.163060in}{1.291500in}}%
\pgfpathcurveto{\pgfqpoint{7.156548in}{1.284989in}}{\pgfqpoint{7.152890in}{1.276156in}}{\pgfqpoint{7.152890in}{1.266948in}}%
\pgfpathcurveto{\pgfqpoint{7.152890in}{1.257739in}}{\pgfqpoint{7.156548in}{1.248907in}}{\pgfqpoint{7.163060in}{1.242395in}}%
\pgfpathcurveto{\pgfqpoint{7.169571in}{1.235884in}}{\pgfqpoint{7.178404in}{1.232225in}}{\pgfqpoint{7.187612in}{1.232225in}}%
\pgfpathlineto{\pgfqpoint{7.187612in}{1.232225in}}%
\pgfpathclose%
\pgfusepath{stroke}%
\end{pgfscope}%
\begin{pgfscope}%
\pgfpathrectangle{\pgfqpoint{0.417359in}{0.814008in}}{\pgfqpoint{12.309552in}{4.722875in}}%
\pgfusepath{clip}%
\pgfsetbuttcap%
\pgfsetroundjoin%
\pgfsetlinewidth{1.003750pt}%
\definecolor{currentstroke}{rgb}{0.450000,0.450000,0.450000}%
\pgfsetstrokecolor{currentstroke}%
\pgfsetdash{}{0pt}%
\pgfpathmoveto{\pgfqpoint{7.187612in}{0.849986in}}%
\pgfpathcurveto{\pgfqpoint{7.196821in}{0.849986in}}{\pgfqpoint{7.205653in}{0.853645in}}{\pgfqpoint{7.212164in}{0.860156in}}%
\pgfpathcurveto{\pgfqpoint{7.218676in}{0.866667in}}{\pgfqpoint{7.222334in}{0.875500in}}{\pgfqpoint{7.222334in}{0.884708in}}%
\pgfpathcurveto{\pgfqpoint{7.222334in}{0.893917in}}{\pgfqpoint{7.218676in}{0.902749in}}{\pgfqpoint{7.212164in}{0.909261in}}%
\pgfpathcurveto{\pgfqpoint{7.205653in}{0.915772in}}{\pgfqpoint{7.196821in}{0.919431in}}{\pgfqpoint{7.187612in}{0.919431in}}%
\pgfpathcurveto{\pgfqpoint{7.178404in}{0.919431in}}{\pgfqpoint{7.169571in}{0.915772in}}{\pgfqpoint{7.163060in}{0.909261in}}%
\pgfpathcurveto{\pgfqpoint{7.156548in}{0.902749in}}{\pgfqpoint{7.152890in}{0.893917in}}{\pgfqpoint{7.152890in}{0.884708in}}%
\pgfpathcurveto{\pgfqpoint{7.152890in}{0.875500in}}{\pgfqpoint{7.156548in}{0.866667in}}{\pgfqpoint{7.163060in}{0.860156in}}%
\pgfpathcurveto{\pgfqpoint{7.169571in}{0.853645in}}{\pgfqpoint{7.178404in}{0.849986in}}{\pgfqpoint{7.187612in}{0.849986in}}%
\pgfpathlineto{\pgfqpoint{7.187612in}{0.849986in}}%
\pgfpathclose%
\pgfusepath{stroke}%
\end{pgfscope}%
\begin{pgfscope}%
\pgfpathrectangle{\pgfqpoint{0.417359in}{0.814008in}}{\pgfqpoint{12.309552in}{4.722875in}}%
\pgfusepath{clip}%
\pgfsetbuttcap%
\pgfsetroundjoin%
\pgfsetlinewidth{1.003750pt}%
\definecolor{currentstroke}{rgb}{0.450000,0.450000,0.450000}%
\pgfsetstrokecolor{currentstroke}%
\pgfsetdash{}{0pt}%
\pgfpathmoveto{\pgfqpoint{7.187612in}{0.849986in}}%
\pgfpathcurveto{\pgfqpoint{7.196821in}{0.849986in}}{\pgfqpoint{7.205653in}{0.853645in}}{\pgfqpoint{7.212164in}{0.860156in}}%
\pgfpathcurveto{\pgfqpoint{7.218676in}{0.866667in}}{\pgfqpoint{7.222334in}{0.875500in}}{\pgfqpoint{7.222334in}{0.884708in}}%
\pgfpathcurveto{\pgfqpoint{7.222334in}{0.893917in}}{\pgfqpoint{7.218676in}{0.902749in}}{\pgfqpoint{7.212164in}{0.909261in}}%
\pgfpathcurveto{\pgfqpoint{7.205653in}{0.915772in}}{\pgfqpoint{7.196821in}{0.919431in}}{\pgfqpoint{7.187612in}{0.919431in}}%
\pgfpathcurveto{\pgfqpoint{7.178404in}{0.919431in}}{\pgfqpoint{7.169571in}{0.915772in}}{\pgfqpoint{7.163060in}{0.909261in}}%
\pgfpathcurveto{\pgfqpoint{7.156548in}{0.902749in}}{\pgfqpoint{7.152890in}{0.893917in}}{\pgfqpoint{7.152890in}{0.884708in}}%
\pgfpathcurveto{\pgfqpoint{7.152890in}{0.875500in}}{\pgfqpoint{7.156548in}{0.866667in}}{\pgfqpoint{7.163060in}{0.860156in}}%
\pgfpathcurveto{\pgfqpoint{7.169571in}{0.853645in}}{\pgfqpoint{7.178404in}{0.849986in}}{\pgfqpoint{7.187612in}{0.849986in}}%
\pgfpathlineto{\pgfqpoint{7.187612in}{0.849986in}}%
\pgfpathclose%
\pgfusepath{stroke}%
\end{pgfscope}%
\begin{pgfscope}%
\pgfpathrectangle{\pgfqpoint{0.417359in}{0.814008in}}{\pgfqpoint{12.309552in}{4.722875in}}%
\pgfusepath{clip}%
\pgfsetbuttcap%
\pgfsetroundjoin%
\pgfsetlinewidth{1.003750pt}%
\definecolor{currentstroke}{rgb}{0.450000,0.450000,0.450000}%
\pgfsetstrokecolor{currentstroke}%
\pgfsetdash{}{0pt}%
\pgfpathmoveto{\pgfqpoint{7.187612in}{1.206393in}}%
\pgfpathcurveto{\pgfqpoint{7.196821in}{1.206393in}}{\pgfqpoint{7.205653in}{1.210052in}}{\pgfqpoint{7.212164in}{1.216563in}}%
\pgfpathcurveto{\pgfqpoint{7.218676in}{1.223075in}}{\pgfqpoint{7.222334in}{1.231907in}}{\pgfqpoint{7.222334in}{1.241115in}}%
\pgfpathcurveto{\pgfqpoint{7.222334in}{1.250324in}}{\pgfqpoint{7.218676in}{1.259156in}}{\pgfqpoint{7.212164in}{1.265668in}}%
\pgfpathcurveto{\pgfqpoint{7.205653in}{1.272179in}}{\pgfqpoint{7.196821in}{1.275838in}}{\pgfqpoint{7.187612in}{1.275838in}}%
\pgfpathcurveto{\pgfqpoint{7.178404in}{1.275838in}}{\pgfqpoint{7.169571in}{1.272179in}}{\pgfqpoint{7.163060in}{1.265668in}}%
\pgfpathcurveto{\pgfqpoint{7.156548in}{1.259156in}}{\pgfqpoint{7.152890in}{1.250324in}}{\pgfqpoint{7.152890in}{1.241115in}}%
\pgfpathcurveto{\pgfqpoint{7.152890in}{1.231907in}}{\pgfqpoint{7.156548in}{1.223075in}}{\pgfqpoint{7.163060in}{1.216563in}}%
\pgfpathcurveto{\pgfqpoint{7.169571in}{1.210052in}}{\pgfqpoint{7.178404in}{1.206393in}}{\pgfqpoint{7.187612in}{1.206393in}}%
\pgfpathlineto{\pgfqpoint{7.187612in}{1.206393in}}%
\pgfpathclose%
\pgfusepath{stroke}%
\end{pgfscope}%
\begin{pgfscope}%
\pgfpathrectangle{\pgfqpoint{0.417359in}{0.814008in}}{\pgfqpoint{12.309552in}{4.722875in}}%
\pgfusepath{clip}%
\pgfsetbuttcap%
\pgfsetroundjoin%
\pgfsetlinewidth{1.003750pt}%
\definecolor{currentstroke}{rgb}{0.450000,0.450000,0.450000}%
\pgfsetstrokecolor{currentstroke}%
\pgfsetdash{}{0pt}%
\pgfpathmoveto{\pgfqpoint{7.187612in}{0.813074in}}%
\pgfpathcurveto{\pgfqpoint{7.196821in}{0.813074in}}{\pgfqpoint{7.205653in}{0.816733in}}{\pgfqpoint{7.212164in}{0.823244in}}%
\pgfpathcurveto{\pgfqpoint{7.218676in}{0.829755in}}{\pgfqpoint{7.222334in}{0.838588in}}{\pgfqpoint{7.222334in}{0.847796in}}%
\pgfpathcurveto{\pgfqpoint{7.222334in}{0.857005in}}{\pgfqpoint{7.218676in}{0.865837in}}{\pgfqpoint{7.212164in}{0.872349in}}%
\pgfpathcurveto{\pgfqpoint{7.205653in}{0.878860in}}{\pgfqpoint{7.196821in}{0.882519in}}{\pgfqpoint{7.187612in}{0.882519in}}%
\pgfpathcurveto{\pgfqpoint{7.178404in}{0.882519in}}{\pgfqpoint{7.169571in}{0.878860in}}{\pgfqpoint{7.163060in}{0.872349in}}%
\pgfpathcurveto{\pgfqpoint{7.156548in}{0.865837in}}{\pgfqpoint{7.152890in}{0.857005in}}{\pgfqpoint{7.152890in}{0.847796in}}%
\pgfpathcurveto{\pgfqpoint{7.152890in}{0.838588in}}{\pgfqpoint{7.156548in}{0.829755in}}{\pgfqpoint{7.163060in}{0.823244in}}%
\pgfpathcurveto{\pgfqpoint{7.169571in}{0.816733in}}{\pgfqpoint{7.178404in}{0.813074in}}{\pgfqpoint{7.187612in}{0.813074in}}%
\pgfpathlineto{\pgfqpoint{7.187612in}{0.813074in}}%
\pgfpathclose%
\pgfusepath{stroke}%
\end{pgfscope}%
\begin{pgfscope}%
\pgfpathrectangle{\pgfqpoint{0.417359in}{0.814008in}}{\pgfqpoint{12.309552in}{4.722875in}}%
\pgfusepath{clip}%
\pgfsetbuttcap%
\pgfsetroundjoin%
\pgfsetlinewidth{1.003750pt}%
\definecolor{currentstroke}{rgb}{0.450000,0.450000,0.450000}%
\pgfsetstrokecolor{currentstroke}%
\pgfsetdash{}{0pt}%
\pgfpathmoveto{\pgfqpoint{7.187612in}{0.849430in}}%
\pgfpathcurveto{\pgfqpoint{7.196821in}{0.849430in}}{\pgfqpoint{7.205653in}{0.853089in}}{\pgfqpoint{7.212164in}{0.859600in}}%
\pgfpathcurveto{\pgfqpoint{7.218676in}{0.866112in}}{\pgfqpoint{7.222334in}{0.874944in}}{\pgfqpoint{7.222334in}{0.884153in}}%
\pgfpathcurveto{\pgfqpoint{7.222334in}{0.893361in}}{\pgfqpoint{7.218676in}{0.902194in}}{\pgfqpoint{7.212164in}{0.908705in}}%
\pgfpathcurveto{\pgfqpoint{7.205653in}{0.915216in}}{\pgfqpoint{7.196821in}{0.918875in}}{\pgfqpoint{7.187612in}{0.918875in}}%
\pgfpathcurveto{\pgfqpoint{7.178404in}{0.918875in}}{\pgfqpoint{7.169571in}{0.915216in}}{\pgfqpoint{7.163060in}{0.908705in}}%
\pgfpathcurveto{\pgfqpoint{7.156548in}{0.902194in}}{\pgfqpoint{7.152890in}{0.893361in}}{\pgfqpoint{7.152890in}{0.884153in}}%
\pgfpathcurveto{\pgfqpoint{7.152890in}{0.874944in}}{\pgfqpoint{7.156548in}{0.866112in}}{\pgfqpoint{7.163060in}{0.859600in}}%
\pgfpathcurveto{\pgfqpoint{7.169571in}{0.853089in}}{\pgfqpoint{7.178404in}{0.849430in}}{\pgfqpoint{7.187612in}{0.849430in}}%
\pgfpathlineto{\pgfqpoint{7.187612in}{0.849430in}}%
\pgfpathclose%
\pgfusepath{stroke}%
\end{pgfscope}%
\begin{pgfscope}%
\pgfpathrectangle{\pgfqpoint{0.417359in}{0.814008in}}{\pgfqpoint{12.309552in}{4.722875in}}%
\pgfusepath{clip}%
\pgfsetbuttcap%
\pgfsetroundjoin%
\pgfsetlinewidth{1.003750pt}%
\definecolor{currentstroke}{rgb}{0.450000,0.450000,0.450000}%
\pgfsetstrokecolor{currentstroke}%
\pgfsetdash{}{0pt}%
\pgfpathmoveto{\pgfqpoint{7.187612in}{3.152878in}}%
\pgfpathcurveto{\pgfqpoint{7.196821in}{3.152878in}}{\pgfqpoint{7.205653in}{3.156536in}}{\pgfqpoint{7.212164in}{3.163048in}}%
\pgfpathcurveto{\pgfqpoint{7.218676in}{3.169559in}}{\pgfqpoint{7.222334in}{3.178391in}}{\pgfqpoint{7.222334in}{3.187600in}}%
\pgfpathcurveto{\pgfqpoint{7.222334in}{3.196808in}}{\pgfqpoint{7.218676in}{3.205641in}}{\pgfqpoint{7.212164in}{3.212152in}}%
\pgfpathcurveto{\pgfqpoint{7.205653in}{3.218664in}}{\pgfqpoint{7.196821in}{3.222322in}}{\pgfqpoint{7.187612in}{3.222322in}}%
\pgfpathcurveto{\pgfqpoint{7.178404in}{3.222322in}}{\pgfqpoint{7.169571in}{3.218664in}}{\pgfqpoint{7.163060in}{3.212152in}}%
\pgfpathcurveto{\pgfqpoint{7.156548in}{3.205641in}}{\pgfqpoint{7.152890in}{3.196808in}}{\pgfqpoint{7.152890in}{3.187600in}}%
\pgfpathcurveto{\pgfqpoint{7.152890in}{3.178391in}}{\pgfqpoint{7.156548in}{3.169559in}}{\pgfqpoint{7.163060in}{3.163048in}}%
\pgfpathcurveto{\pgfqpoint{7.169571in}{3.156536in}}{\pgfqpoint{7.178404in}{3.152878in}}{\pgfqpoint{7.187612in}{3.152878in}}%
\pgfpathlineto{\pgfqpoint{7.187612in}{3.152878in}}%
\pgfpathclose%
\pgfusepath{stroke}%
\end{pgfscope}%
\begin{pgfscope}%
\pgfpathrectangle{\pgfqpoint{0.417359in}{0.814008in}}{\pgfqpoint{12.309552in}{4.722875in}}%
\pgfusepath{clip}%
\pgfsetbuttcap%
\pgfsetroundjoin%
\pgfsetlinewidth{1.003750pt}%
\definecolor{currentstroke}{rgb}{0.450000,0.450000,0.450000}%
\pgfsetstrokecolor{currentstroke}%
\pgfsetdash{}{0pt}%
\pgfpathmoveto{\pgfqpoint{7.187612in}{0.813074in}}%
\pgfpathcurveto{\pgfqpoint{7.196821in}{0.813074in}}{\pgfqpoint{7.205653in}{0.816733in}}{\pgfqpoint{7.212164in}{0.823244in}}%
\pgfpathcurveto{\pgfqpoint{7.218676in}{0.829755in}}{\pgfqpoint{7.222334in}{0.838588in}}{\pgfqpoint{7.222334in}{0.847796in}}%
\pgfpathcurveto{\pgfqpoint{7.222334in}{0.857005in}}{\pgfqpoint{7.218676in}{0.865837in}}{\pgfqpoint{7.212164in}{0.872349in}}%
\pgfpathcurveto{\pgfqpoint{7.205653in}{0.878860in}}{\pgfqpoint{7.196821in}{0.882519in}}{\pgfqpoint{7.187612in}{0.882519in}}%
\pgfpathcurveto{\pgfqpoint{7.178404in}{0.882519in}}{\pgfqpoint{7.169571in}{0.878860in}}{\pgfqpoint{7.163060in}{0.872349in}}%
\pgfpathcurveto{\pgfqpoint{7.156548in}{0.865837in}}{\pgfqpoint{7.152890in}{0.857005in}}{\pgfqpoint{7.152890in}{0.847796in}}%
\pgfpathcurveto{\pgfqpoint{7.152890in}{0.838588in}}{\pgfqpoint{7.156548in}{0.829755in}}{\pgfqpoint{7.163060in}{0.823244in}}%
\pgfpathcurveto{\pgfqpoint{7.169571in}{0.816733in}}{\pgfqpoint{7.178404in}{0.813074in}}{\pgfqpoint{7.187612in}{0.813074in}}%
\pgfpathlineto{\pgfqpoint{7.187612in}{0.813074in}}%
\pgfpathclose%
\pgfusepath{stroke}%
\end{pgfscope}%
\begin{pgfscope}%
\pgfpathrectangle{\pgfqpoint{0.417359in}{0.814008in}}{\pgfqpoint{12.309552in}{4.722875in}}%
\pgfusepath{clip}%
\pgfsetbuttcap%
\pgfsetroundjoin%
\pgfsetlinewidth{1.003750pt}%
\definecolor{currentstroke}{rgb}{0.450000,0.450000,0.450000}%
\pgfsetstrokecolor{currentstroke}%
\pgfsetdash{}{0pt}%
\pgfpathmoveto{\pgfqpoint{7.187612in}{0.812986in}}%
\pgfpathcurveto{\pgfqpoint{7.196821in}{0.812986in}}{\pgfqpoint{7.205653in}{0.816645in}}{\pgfqpoint{7.212164in}{0.823156in}}%
\pgfpathcurveto{\pgfqpoint{7.218676in}{0.829668in}}{\pgfqpoint{7.222334in}{0.838500in}}{\pgfqpoint{7.222334in}{0.847709in}}%
\pgfpathcurveto{\pgfqpoint{7.222334in}{0.856917in}}{\pgfqpoint{7.218676in}{0.865750in}}{\pgfqpoint{7.212164in}{0.872261in}}%
\pgfpathcurveto{\pgfqpoint{7.205653in}{0.878772in}}{\pgfqpoint{7.196821in}{0.882431in}}{\pgfqpoint{7.187612in}{0.882431in}}%
\pgfpathcurveto{\pgfqpoint{7.178404in}{0.882431in}}{\pgfqpoint{7.169571in}{0.878772in}}{\pgfqpoint{7.163060in}{0.872261in}}%
\pgfpathcurveto{\pgfqpoint{7.156548in}{0.865750in}}{\pgfqpoint{7.152890in}{0.856917in}}{\pgfqpoint{7.152890in}{0.847709in}}%
\pgfpathcurveto{\pgfqpoint{7.152890in}{0.838500in}}{\pgfqpoint{7.156548in}{0.829668in}}{\pgfqpoint{7.163060in}{0.823156in}}%
\pgfpathcurveto{\pgfqpoint{7.169571in}{0.816645in}}{\pgfqpoint{7.178404in}{0.812986in}}{\pgfqpoint{7.187612in}{0.812986in}}%
\pgfpathlineto{\pgfqpoint{7.187612in}{0.812986in}}%
\pgfpathclose%
\pgfusepath{stroke}%
\end{pgfscope}%
\begin{pgfscope}%
\pgfpathrectangle{\pgfqpoint{0.417359in}{0.814008in}}{\pgfqpoint{12.309552in}{4.722875in}}%
\pgfusepath{clip}%
\pgfsetbuttcap%
\pgfsetroundjoin%
\definecolor{currentfill}{rgb}{0.837255,0.519608,0.740196}%
\pgfsetfillcolor{currentfill}%
\pgfsetlinewidth{0.752812pt}%
\definecolor{currentstroke}{rgb}{0.240000,0.240000,0.240000}%
\pgfsetstrokecolor{currentstroke}%
\pgfsetdash{}{0pt}%
\pgfsys@defobject{currentmarker}{\pgfqpoint{7.926185in}{2.328743in}}{\pgfqpoint{8.910949in}{3.059643in}}{%
\pgfpathmoveto{\pgfqpoint{7.926185in}{2.328743in}}%
\pgfpathlineto{\pgfqpoint{8.910949in}{2.328743in}}%
\pgfpathlineto{\pgfqpoint{8.910949in}{3.059643in}}%
\pgfpathlineto{\pgfqpoint{7.926185in}{3.059643in}}%
\pgfpathlineto{\pgfqpoint{7.926185in}{2.328743in}}%
\pgfpathclose%
\pgfusepath{stroke,fill}%
}%
\begin{pgfscope}%
\pgfsys@transformshift{0.000000in}{0.000000in}%
\pgfsys@useobject{currentmarker}{}%
\end{pgfscope}%
\end{pgfscope}%
\begin{pgfscope}%
\pgfpathrectangle{\pgfqpoint{0.417359in}{0.814008in}}{\pgfqpoint{12.309552in}{4.722875in}}%
\pgfusepath{clip}%
\pgfsetbuttcap%
\pgfsetroundjoin%
\pgfsetlinewidth{1.003750pt}%
\definecolor{currentstroke}{rgb}{0.450000,0.450000,0.450000}%
\pgfsetstrokecolor{currentstroke}%
\pgfsetdash{}{0pt}%
\pgfpathmoveto{\pgfqpoint{8.418567in}{1.779115in}}%
\pgfpathcurveto{\pgfqpoint{8.427776in}{1.779115in}}{\pgfqpoint{8.436608in}{1.782774in}}{\pgfqpoint{8.443120in}{1.789285in}}%
\pgfpathcurveto{\pgfqpoint{8.449631in}{1.795796in}}{\pgfqpoint{8.453290in}{1.804629in}}{\pgfqpoint{8.453290in}{1.813837in}}%
\pgfpathcurveto{\pgfqpoint{8.453290in}{1.823046in}}{\pgfqpoint{8.449631in}{1.831878in}}{\pgfqpoint{8.443120in}{1.838390in}}%
\pgfpathcurveto{\pgfqpoint{8.436608in}{1.844901in}}{\pgfqpoint{8.427776in}{1.848560in}}{\pgfqpoint{8.418567in}{1.848560in}}%
\pgfpathcurveto{\pgfqpoint{8.409359in}{1.848560in}}{\pgfqpoint{8.400526in}{1.844901in}}{\pgfqpoint{8.394015in}{1.838390in}}%
\pgfpathcurveto{\pgfqpoint{8.387504in}{1.831878in}}{\pgfqpoint{8.383845in}{1.823046in}}{\pgfqpoint{8.383845in}{1.813837in}}%
\pgfpathcurveto{\pgfqpoint{8.383845in}{1.804629in}}{\pgfqpoint{8.387504in}{1.795796in}}{\pgfqpoint{8.394015in}{1.789285in}}%
\pgfpathcurveto{\pgfqpoint{8.400526in}{1.782774in}}{\pgfqpoint{8.409359in}{1.779115in}}{\pgfqpoint{8.418567in}{1.779115in}}%
\pgfpathlineto{\pgfqpoint{8.418567in}{1.779115in}}%
\pgfpathclose%
\pgfusepath{stroke}%
\end{pgfscope}%
\begin{pgfscope}%
\pgfpathrectangle{\pgfqpoint{0.417359in}{0.814008in}}{\pgfqpoint{12.309552in}{4.722875in}}%
\pgfusepath{clip}%
\pgfsetbuttcap%
\pgfsetroundjoin%
\pgfsetlinewidth{1.003750pt}%
\definecolor{currentstroke}{rgb}{0.450000,0.450000,0.450000}%
\pgfsetstrokecolor{currentstroke}%
\pgfsetdash{}{0pt}%
\pgfpathmoveto{\pgfqpoint{8.418567in}{1.779115in}}%
\pgfpathcurveto{\pgfqpoint{8.427776in}{1.779115in}}{\pgfqpoint{8.436608in}{1.782774in}}{\pgfqpoint{8.443120in}{1.789285in}}%
\pgfpathcurveto{\pgfqpoint{8.449631in}{1.795796in}}{\pgfqpoint{8.453290in}{1.804629in}}{\pgfqpoint{8.453290in}{1.813837in}}%
\pgfpathcurveto{\pgfqpoint{8.453290in}{1.823046in}}{\pgfqpoint{8.449631in}{1.831878in}}{\pgfqpoint{8.443120in}{1.838390in}}%
\pgfpathcurveto{\pgfqpoint{8.436608in}{1.844901in}}{\pgfqpoint{8.427776in}{1.848560in}}{\pgfqpoint{8.418567in}{1.848560in}}%
\pgfpathcurveto{\pgfqpoint{8.409359in}{1.848560in}}{\pgfqpoint{8.400526in}{1.844901in}}{\pgfqpoint{8.394015in}{1.838390in}}%
\pgfpathcurveto{\pgfqpoint{8.387504in}{1.831878in}}{\pgfqpoint{8.383845in}{1.823046in}}{\pgfqpoint{8.383845in}{1.813837in}}%
\pgfpathcurveto{\pgfqpoint{8.383845in}{1.804629in}}{\pgfqpoint{8.387504in}{1.795796in}}{\pgfqpoint{8.394015in}{1.789285in}}%
\pgfpathcurveto{\pgfqpoint{8.400526in}{1.782774in}}{\pgfqpoint{8.409359in}{1.779115in}}{\pgfqpoint{8.418567in}{1.779115in}}%
\pgfpathlineto{\pgfqpoint{8.418567in}{1.779115in}}%
\pgfpathclose%
\pgfusepath{stroke}%
\end{pgfscope}%
\begin{pgfscope}%
\pgfpathrectangle{\pgfqpoint{0.417359in}{0.814008in}}{\pgfqpoint{12.309552in}{4.722875in}}%
\pgfusepath{clip}%
\pgfsetbuttcap%
\pgfsetroundjoin%
\pgfsetlinewidth{1.003750pt}%
\definecolor{currentstroke}{rgb}{0.450000,0.450000,0.450000}%
\pgfsetstrokecolor{currentstroke}%
\pgfsetdash{}{0pt}%
\pgfpathmoveto{\pgfqpoint{8.418567in}{3.423779in}}%
\pgfpathcurveto{\pgfqpoint{8.427776in}{3.423779in}}{\pgfqpoint{8.436608in}{3.427437in}}{\pgfqpoint{8.443120in}{3.433949in}}%
\pgfpathcurveto{\pgfqpoint{8.449631in}{3.440460in}}{\pgfqpoint{8.453290in}{3.449293in}}{\pgfqpoint{8.453290in}{3.458501in}}%
\pgfpathcurveto{\pgfqpoint{8.453290in}{3.467709in}}{\pgfqpoint{8.449631in}{3.476542in}}{\pgfqpoint{8.443120in}{3.483053in}}%
\pgfpathcurveto{\pgfqpoint{8.436608in}{3.489565in}}{\pgfqpoint{8.427776in}{3.493223in}}{\pgfqpoint{8.418567in}{3.493223in}}%
\pgfpathcurveto{\pgfqpoint{8.409359in}{3.493223in}}{\pgfqpoint{8.400526in}{3.489565in}}{\pgfqpoint{8.394015in}{3.483053in}}%
\pgfpathcurveto{\pgfqpoint{8.387504in}{3.476542in}}{\pgfqpoint{8.383845in}{3.467709in}}{\pgfqpoint{8.383845in}{3.458501in}}%
\pgfpathcurveto{\pgfqpoint{8.383845in}{3.449293in}}{\pgfqpoint{8.387504in}{3.440460in}}{\pgfqpoint{8.394015in}{3.433949in}}%
\pgfpathcurveto{\pgfqpoint{8.400526in}{3.427437in}}{\pgfqpoint{8.409359in}{3.423779in}}{\pgfqpoint{8.418567in}{3.423779in}}%
\pgfpathlineto{\pgfqpoint{8.418567in}{3.423779in}}%
\pgfpathclose%
\pgfusepath{stroke}%
\end{pgfscope}%
\begin{pgfscope}%
\pgfpathrectangle{\pgfqpoint{0.417359in}{0.814008in}}{\pgfqpoint{12.309552in}{4.722875in}}%
\pgfusepath{clip}%
\pgfsetbuttcap%
\pgfsetroundjoin%
\pgfsetlinewidth{1.003750pt}%
\definecolor{currentstroke}{rgb}{0.450000,0.450000,0.450000}%
\pgfsetstrokecolor{currentstroke}%
\pgfsetdash{}{0pt}%
\pgfpathmoveto{\pgfqpoint{8.418567in}{3.027628in}}%
\pgfpathcurveto{\pgfqpoint{8.427776in}{3.027628in}}{\pgfqpoint{8.436608in}{3.031287in}}{\pgfqpoint{8.443120in}{3.037798in}}%
\pgfpathcurveto{\pgfqpoint{8.449631in}{3.044309in}}{\pgfqpoint{8.453290in}{3.053142in}}{\pgfqpoint{8.453290in}{3.062350in}}%
\pgfpathcurveto{\pgfqpoint{8.453290in}{3.071559in}}{\pgfqpoint{8.449631in}{3.080391in}}{\pgfqpoint{8.443120in}{3.086903in}}%
\pgfpathcurveto{\pgfqpoint{8.436608in}{3.093414in}}{\pgfqpoint{8.427776in}{3.097073in}}{\pgfqpoint{8.418567in}{3.097073in}}%
\pgfpathcurveto{\pgfqpoint{8.409359in}{3.097073in}}{\pgfqpoint{8.400526in}{3.093414in}}{\pgfqpoint{8.394015in}{3.086903in}}%
\pgfpathcurveto{\pgfqpoint{8.387504in}{3.080391in}}{\pgfqpoint{8.383845in}{3.071559in}}{\pgfqpoint{8.383845in}{3.062350in}}%
\pgfpathcurveto{\pgfqpoint{8.383845in}{3.053142in}}{\pgfqpoint{8.387504in}{3.044309in}}{\pgfqpoint{8.394015in}{3.037798in}}%
\pgfpathcurveto{\pgfqpoint{8.400526in}{3.031287in}}{\pgfqpoint{8.409359in}{3.027628in}}{\pgfqpoint{8.418567in}{3.027628in}}%
\pgfpathlineto{\pgfqpoint{8.418567in}{3.027628in}}%
\pgfpathclose%
\pgfusepath{stroke}%
\end{pgfscope}%
\begin{pgfscope}%
\pgfpathrectangle{\pgfqpoint{0.417359in}{0.814008in}}{\pgfqpoint{12.309552in}{4.722875in}}%
\pgfusepath{clip}%
\pgfsetbuttcap%
\pgfsetroundjoin%
\pgfsetlinewidth{1.003750pt}%
\definecolor{currentstroke}{rgb}{0.450000,0.450000,0.450000}%
\pgfsetstrokecolor{currentstroke}%
\pgfsetdash{}{0pt}%
\pgfpathmoveto{\pgfqpoint{8.418567in}{4.502758in}}%
\pgfpathcurveto{\pgfqpoint{8.427776in}{4.502758in}}{\pgfqpoint{8.436608in}{4.506416in}}{\pgfqpoint{8.443120in}{4.512928in}}%
\pgfpathcurveto{\pgfqpoint{8.449631in}{4.519439in}}{\pgfqpoint{8.453290in}{4.528271in}}{\pgfqpoint{8.453290in}{4.537480in}}%
\pgfpathcurveto{\pgfqpoint{8.453290in}{4.546688in}}{\pgfqpoint{8.449631in}{4.555521in}}{\pgfqpoint{8.443120in}{4.562032in}}%
\pgfpathcurveto{\pgfqpoint{8.436608in}{4.568544in}}{\pgfqpoint{8.427776in}{4.572202in}}{\pgfqpoint{8.418567in}{4.572202in}}%
\pgfpathcurveto{\pgfqpoint{8.409359in}{4.572202in}}{\pgfqpoint{8.400526in}{4.568544in}}{\pgfqpoint{8.394015in}{4.562032in}}%
\pgfpathcurveto{\pgfqpoint{8.387504in}{4.555521in}}{\pgfqpoint{8.383845in}{4.546688in}}{\pgfqpoint{8.383845in}{4.537480in}}%
\pgfpathcurveto{\pgfqpoint{8.383845in}{4.528271in}}{\pgfqpoint{8.387504in}{4.519439in}}{\pgfqpoint{8.394015in}{4.512928in}}%
\pgfpathcurveto{\pgfqpoint{8.400526in}{4.506416in}}{\pgfqpoint{8.409359in}{4.502758in}}{\pgfqpoint{8.418567in}{4.502758in}}%
\pgfpathlineto{\pgfqpoint{8.418567in}{4.502758in}}%
\pgfpathclose%
\pgfusepath{stroke}%
\end{pgfscope}%
\begin{pgfscope}%
\pgfpathrectangle{\pgfqpoint{0.417359in}{0.814008in}}{\pgfqpoint{12.309552in}{4.722875in}}%
\pgfusepath{clip}%
\pgfsetbuttcap%
\pgfsetroundjoin%
\pgfsetlinewidth{1.003750pt}%
\definecolor{currentstroke}{rgb}{0.450000,0.450000,0.450000}%
\pgfsetstrokecolor{currentstroke}%
\pgfsetdash{}{0pt}%
\pgfpathmoveto{\pgfqpoint{8.418567in}{3.904225in}}%
\pgfpathcurveto{\pgfqpoint{8.427776in}{3.904225in}}{\pgfqpoint{8.436608in}{3.907884in}}{\pgfqpoint{8.443120in}{3.914395in}}%
\pgfpathcurveto{\pgfqpoint{8.449631in}{3.920906in}}{\pgfqpoint{8.453290in}{3.929739in}}{\pgfqpoint{8.453290in}{3.938947in}}%
\pgfpathcurveto{\pgfqpoint{8.453290in}{3.948156in}}{\pgfqpoint{8.449631in}{3.956988in}}{\pgfqpoint{8.443120in}{3.963500in}}%
\pgfpathcurveto{\pgfqpoint{8.436608in}{3.970011in}}{\pgfqpoint{8.427776in}{3.973670in}}{\pgfqpoint{8.418567in}{3.973670in}}%
\pgfpathcurveto{\pgfqpoint{8.409359in}{3.973670in}}{\pgfqpoint{8.400526in}{3.970011in}}{\pgfqpoint{8.394015in}{3.963500in}}%
\pgfpathcurveto{\pgfqpoint{8.387504in}{3.956988in}}{\pgfqpoint{8.383845in}{3.948156in}}{\pgfqpoint{8.383845in}{3.938947in}}%
\pgfpathcurveto{\pgfqpoint{8.383845in}{3.929739in}}{\pgfqpoint{8.387504in}{3.920906in}}{\pgfqpoint{8.394015in}{3.914395in}}%
\pgfpathcurveto{\pgfqpoint{8.400526in}{3.907884in}}{\pgfqpoint{8.409359in}{3.904225in}}{\pgfqpoint{8.418567in}{3.904225in}}%
\pgfpathlineto{\pgfqpoint{8.418567in}{3.904225in}}%
\pgfpathclose%
\pgfusepath{stroke}%
\end{pgfscope}%
\begin{pgfscope}%
\pgfpathrectangle{\pgfqpoint{0.417359in}{0.814008in}}{\pgfqpoint{12.309552in}{4.722875in}}%
\pgfusepath{clip}%
\pgfsetbuttcap%
\pgfsetroundjoin%
\pgfsetlinewidth{1.003750pt}%
\definecolor{currentstroke}{rgb}{0.450000,0.450000,0.450000}%
\pgfsetstrokecolor{currentstroke}%
\pgfsetdash{}{0pt}%
\pgfpathmoveto{\pgfqpoint{8.418567in}{2.217079in}}%
\pgfpathcurveto{\pgfqpoint{8.427776in}{2.217079in}}{\pgfqpoint{8.436608in}{2.220737in}}{\pgfqpoint{8.443120in}{2.227249in}}%
\pgfpathcurveto{\pgfqpoint{8.449631in}{2.233760in}}{\pgfqpoint{8.453290in}{2.242592in}}{\pgfqpoint{8.453290in}{2.251801in}}%
\pgfpathcurveto{\pgfqpoint{8.453290in}{2.261009in}}{\pgfqpoint{8.449631in}{2.269842in}}{\pgfqpoint{8.443120in}{2.276353in}}%
\pgfpathcurveto{\pgfqpoint{8.436608in}{2.282865in}}{\pgfqpoint{8.427776in}{2.286523in}}{\pgfqpoint{8.418567in}{2.286523in}}%
\pgfpathcurveto{\pgfqpoint{8.409359in}{2.286523in}}{\pgfqpoint{8.400526in}{2.282865in}}{\pgfqpoint{8.394015in}{2.276353in}}%
\pgfpathcurveto{\pgfqpoint{8.387504in}{2.269842in}}{\pgfqpoint{8.383845in}{2.261009in}}{\pgfqpoint{8.383845in}{2.251801in}}%
\pgfpathcurveto{\pgfqpoint{8.383845in}{2.242592in}}{\pgfqpoint{8.387504in}{2.233760in}}{\pgfqpoint{8.394015in}{2.227249in}}%
\pgfpathcurveto{\pgfqpoint{8.400526in}{2.220737in}}{\pgfqpoint{8.409359in}{2.217079in}}{\pgfqpoint{8.418567in}{2.217079in}}%
\pgfpathlineto{\pgfqpoint{8.418567in}{2.217079in}}%
\pgfpathclose%
\pgfusepath{stroke}%
\end{pgfscope}%
\begin{pgfscope}%
\pgfpathrectangle{\pgfqpoint{0.417359in}{0.814008in}}{\pgfqpoint{12.309552in}{4.722875in}}%
\pgfusepath{clip}%
\pgfsetbuttcap%
\pgfsetroundjoin%
\pgfsetlinewidth{1.003750pt}%
\definecolor{currentstroke}{rgb}{0.450000,0.450000,0.450000}%
\pgfsetstrokecolor{currentstroke}%
\pgfsetdash{}{0pt}%
\pgfpathmoveto{\pgfqpoint{8.418567in}{2.217079in}}%
\pgfpathcurveto{\pgfqpoint{8.427776in}{2.217079in}}{\pgfqpoint{8.436608in}{2.220737in}}{\pgfqpoint{8.443120in}{2.227249in}}%
\pgfpathcurveto{\pgfqpoint{8.449631in}{2.233760in}}{\pgfqpoint{8.453290in}{2.242592in}}{\pgfqpoint{8.453290in}{2.251801in}}%
\pgfpathcurveto{\pgfqpoint{8.453290in}{2.261009in}}{\pgfqpoint{8.449631in}{2.269842in}}{\pgfqpoint{8.443120in}{2.276353in}}%
\pgfpathcurveto{\pgfqpoint{8.436608in}{2.282865in}}{\pgfqpoint{8.427776in}{2.286523in}}{\pgfqpoint{8.418567in}{2.286523in}}%
\pgfpathcurveto{\pgfqpoint{8.409359in}{2.286523in}}{\pgfqpoint{8.400526in}{2.282865in}}{\pgfqpoint{8.394015in}{2.276353in}}%
\pgfpathcurveto{\pgfqpoint{8.387504in}{2.269842in}}{\pgfqpoint{8.383845in}{2.261009in}}{\pgfqpoint{8.383845in}{2.251801in}}%
\pgfpathcurveto{\pgfqpoint{8.383845in}{2.242592in}}{\pgfqpoint{8.387504in}{2.233760in}}{\pgfqpoint{8.394015in}{2.227249in}}%
\pgfpathcurveto{\pgfqpoint{8.400526in}{2.220737in}}{\pgfqpoint{8.409359in}{2.217079in}}{\pgfqpoint{8.418567in}{2.217079in}}%
\pgfpathlineto{\pgfqpoint{8.418567in}{2.217079in}}%
\pgfpathclose%
\pgfusepath{stroke}%
\end{pgfscope}%
\begin{pgfscope}%
\pgfpathrectangle{\pgfqpoint{0.417359in}{0.814008in}}{\pgfqpoint{12.309552in}{4.722875in}}%
\pgfusepath{clip}%
\pgfsetbuttcap%
\pgfsetroundjoin%
\pgfsetlinewidth{1.003750pt}%
\definecolor{currentstroke}{rgb}{0.450000,0.450000,0.450000}%
\pgfsetstrokecolor{currentstroke}%
\pgfsetdash{}{0pt}%
\pgfpathmoveto{\pgfqpoint{8.418567in}{3.036517in}}%
\pgfpathcurveto{\pgfqpoint{8.427776in}{3.036517in}}{\pgfqpoint{8.436608in}{3.040175in}}{\pgfqpoint{8.443120in}{3.046687in}}%
\pgfpathcurveto{\pgfqpoint{8.449631in}{3.053198in}}{\pgfqpoint{8.453290in}{3.062030in}}{\pgfqpoint{8.453290in}{3.071239in}}%
\pgfpathcurveto{\pgfqpoint{8.453290in}{3.080447in}}{\pgfqpoint{8.449631in}{3.089280in}}{\pgfqpoint{8.443120in}{3.095791in}}%
\pgfpathcurveto{\pgfqpoint{8.436608in}{3.102303in}}{\pgfqpoint{8.427776in}{3.105961in}}{\pgfqpoint{8.418567in}{3.105961in}}%
\pgfpathcurveto{\pgfqpoint{8.409359in}{3.105961in}}{\pgfqpoint{8.400526in}{3.102303in}}{\pgfqpoint{8.394015in}{3.095791in}}%
\pgfpathcurveto{\pgfqpoint{8.387504in}{3.089280in}}{\pgfqpoint{8.383845in}{3.080447in}}{\pgfqpoint{8.383845in}{3.071239in}}%
\pgfpathcurveto{\pgfqpoint{8.383845in}{3.062030in}}{\pgfqpoint{8.387504in}{3.053198in}}{\pgfqpoint{8.394015in}{3.046687in}}%
\pgfpathcurveto{\pgfqpoint{8.400526in}{3.040175in}}{\pgfqpoint{8.409359in}{3.036517in}}{\pgfqpoint{8.418567in}{3.036517in}}%
\pgfpathlineto{\pgfqpoint{8.418567in}{3.036517in}}%
\pgfpathclose%
\pgfusepath{stroke}%
\end{pgfscope}%
\begin{pgfscope}%
\pgfpathrectangle{\pgfqpoint{0.417359in}{0.814008in}}{\pgfqpoint{12.309552in}{4.722875in}}%
\pgfusepath{clip}%
\pgfsetbuttcap%
\pgfsetroundjoin%
\pgfsetlinewidth{1.003750pt}%
\definecolor{currentstroke}{rgb}{0.450000,0.450000,0.450000}%
\pgfsetstrokecolor{currentstroke}%
\pgfsetdash{}{0pt}%
\pgfpathmoveto{\pgfqpoint{8.418567in}{3.052105in}}%
\pgfpathcurveto{\pgfqpoint{8.427776in}{3.052105in}}{\pgfqpoint{8.436608in}{3.055764in}}{\pgfqpoint{8.443120in}{3.062275in}}%
\pgfpathcurveto{\pgfqpoint{8.449631in}{3.068787in}}{\pgfqpoint{8.453290in}{3.077619in}}{\pgfqpoint{8.453290in}{3.086828in}}%
\pgfpathcurveto{\pgfqpoint{8.453290in}{3.096036in}}{\pgfqpoint{8.449631in}{3.104869in}}{\pgfqpoint{8.443120in}{3.111380in}}%
\pgfpathcurveto{\pgfqpoint{8.436608in}{3.117891in}}{\pgfqpoint{8.427776in}{3.121550in}}{\pgfqpoint{8.418567in}{3.121550in}}%
\pgfpathcurveto{\pgfqpoint{8.409359in}{3.121550in}}{\pgfqpoint{8.400526in}{3.117891in}}{\pgfqpoint{8.394015in}{3.111380in}}%
\pgfpathcurveto{\pgfqpoint{8.387504in}{3.104869in}}{\pgfqpoint{8.383845in}{3.096036in}}{\pgfqpoint{8.383845in}{3.086828in}}%
\pgfpathcurveto{\pgfqpoint{8.383845in}{3.077619in}}{\pgfqpoint{8.387504in}{3.068787in}}{\pgfqpoint{8.394015in}{3.062275in}}%
\pgfpathcurveto{\pgfqpoint{8.400526in}{3.055764in}}{\pgfqpoint{8.409359in}{3.052105in}}{\pgfqpoint{8.418567in}{3.052105in}}%
\pgfpathlineto{\pgfqpoint{8.418567in}{3.052105in}}%
\pgfpathclose%
\pgfusepath{stroke}%
\end{pgfscope}%
\begin{pgfscope}%
\pgfpathrectangle{\pgfqpoint{0.417359in}{0.814008in}}{\pgfqpoint{12.309552in}{4.722875in}}%
\pgfusepath{clip}%
\pgfsetbuttcap%
\pgfsetroundjoin%
\pgfsetlinewidth{1.003750pt}%
\definecolor{currentstroke}{rgb}{0.450000,0.450000,0.450000}%
\pgfsetstrokecolor{currentstroke}%
\pgfsetdash{}{0pt}%
\pgfpathmoveto{\pgfqpoint{8.418567in}{3.052105in}}%
\pgfpathcurveto{\pgfqpoint{8.427776in}{3.052105in}}{\pgfqpoint{8.436608in}{3.055764in}}{\pgfqpoint{8.443120in}{3.062275in}}%
\pgfpathcurveto{\pgfqpoint{8.449631in}{3.068787in}}{\pgfqpoint{8.453290in}{3.077619in}}{\pgfqpoint{8.453290in}{3.086828in}}%
\pgfpathcurveto{\pgfqpoint{8.453290in}{3.096036in}}{\pgfqpoint{8.449631in}{3.104869in}}{\pgfqpoint{8.443120in}{3.111380in}}%
\pgfpathcurveto{\pgfqpoint{8.436608in}{3.117891in}}{\pgfqpoint{8.427776in}{3.121550in}}{\pgfqpoint{8.418567in}{3.121550in}}%
\pgfpathcurveto{\pgfqpoint{8.409359in}{3.121550in}}{\pgfqpoint{8.400526in}{3.117891in}}{\pgfqpoint{8.394015in}{3.111380in}}%
\pgfpathcurveto{\pgfqpoint{8.387504in}{3.104869in}}{\pgfqpoint{8.383845in}{3.096036in}}{\pgfqpoint{8.383845in}{3.086828in}}%
\pgfpathcurveto{\pgfqpoint{8.383845in}{3.077619in}}{\pgfqpoint{8.387504in}{3.068787in}}{\pgfqpoint{8.394015in}{3.062275in}}%
\pgfpathcurveto{\pgfqpoint{8.400526in}{3.055764in}}{\pgfqpoint{8.409359in}{3.052105in}}{\pgfqpoint{8.418567in}{3.052105in}}%
\pgfpathlineto{\pgfqpoint{8.418567in}{3.052105in}}%
\pgfpathclose%
\pgfusepath{stroke}%
\end{pgfscope}%
\begin{pgfscope}%
\pgfpathrectangle{\pgfqpoint{0.417359in}{0.814008in}}{\pgfqpoint{12.309552in}{4.722875in}}%
\pgfusepath{clip}%
\pgfsetbuttcap%
\pgfsetroundjoin%
\pgfsetlinewidth{1.003750pt}%
\definecolor{currentstroke}{rgb}{0.450000,0.450000,0.450000}%
\pgfsetstrokecolor{currentstroke}%
\pgfsetdash{}{0pt}%
\pgfpathmoveto{\pgfqpoint{8.418567in}{2.217079in}}%
\pgfpathcurveto{\pgfqpoint{8.427776in}{2.217079in}}{\pgfqpoint{8.436608in}{2.220737in}}{\pgfqpoint{8.443120in}{2.227249in}}%
\pgfpathcurveto{\pgfqpoint{8.449631in}{2.233760in}}{\pgfqpoint{8.453290in}{2.242592in}}{\pgfqpoint{8.453290in}{2.251801in}}%
\pgfpathcurveto{\pgfqpoint{8.453290in}{2.261009in}}{\pgfqpoint{8.449631in}{2.269842in}}{\pgfqpoint{8.443120in}{2.276353in}}%
\pgfpathcurveto{\pgfqpoint{8.436608in}{2.282865in}}{\pgfqpoint{8.427776in}{2.286523in}}{\pgfqpoint{8.418567in}{2.286523in}}%
\pgfpathcurveto{\pgfqpoint{8.409359in}{2.286523in}}{\pgfqpoint{8.400526in}{2.282865in}}{\pgfqpoint{8.394015in}{2.276353in}}%
\pgfpathcurveto{\pgfqpoint{8.387504in}{2.269842in}}{\pgfqpoint{8.383845in}{2.261009in}}{\pgfqpoint{8.383845in}{2.251801in}}%
\pgfpathcurveto{\pgfqpoint{8.383845in}{2.242592in}}{\pgfqpoint{8.387504in}{2.233760in}}{\pgfqpoint{8.394015in}{2.227249in}}%
\pgfpathcurveto{\pgfqpoint{8.400526in}{2.220737in}}{\pgfqpoint{8.409359in}{2.217079in}}{\pgfqpoint{8.418567in}{2.217079in}}%
\pgfpathlineto{\pgfqpoint{8.418567in}{2.217079in}}%
\pgfpathclose%
\pgfusepath{stroke}%
\end{pgfscope}%
\begin{pgfscope}%
\pgfpathrectangle{\pgfqpoint{0.417359in}{0.814008in}}{\pgfqpoint{12.309552in}{4.722875in}}%
\pgfusepath{clip}%
\pgfsetbuttcap%
\pgfsetroundjoin%
\pgfsetlinewidth{1.003750pt}%
\definecolor{currentstroke}{rgb}{0.450000,0.450000,0.450000}%
\pgfsetstrokecolor{currentstroke}%
\pgfsetdash{}{0pt}%
\pgfpathmoveto{\pgfqpoint{8.418567in}{1.181646in}}%
\pgfpathcurveto{\pgfqpoint{8.427776in}{1.181646in}}{\pgfqpoint{8.436608in}{1.185304in}}{\pgfqpoint{8.443120in}{1.191816in}}%
\pgfpathcurveto{\pgfqpoint{8.449631in}{1.198327in}}{\pgfqpoint{8.453290in}{1.207160in}}{\pgfqpoint{8.453290in}{1.216368in}}%
\pgfpathcurveto{\pgfqpoint{8.453290in}{1.225577in}}{\pgfqpoint{8.449631in}{1.234409in}}{\pgfqpoint{8.443120in}{1.240920in}}%
\pgfpathcurveto{\pgfqpoint{8.436608in}{1.247432in}}{\pgfqpoint{8.427776in}{1.251090in}}{\pgfqpoint{8.418567in}{1.251090in}}%
\pgfpathcurveto{\pgfqpoint{8.409359in}{1.251090in}}{\pgfqpoint{8.400526in}{1.247432in}}{\pgfqpoint{8.394015in}{1.240920in}}%
\pgfpathcurveto{\pgfqpoint{8.387504in}{1.234409in}}{\pgfqpoint{8.383845in}{1.225577in}}{\pgfqpoint{8.383845in}{1.216368in}}%
\pgfpathcurveto{\pgfqpoint{8.383845in}{1.207160in}}{\pgfqpoint{8.387504in}{1.198327in}}{\pgfqpoint{8.394015in}{1.191816in}}%
\pgfpathcurveto{\pgfqpoint{8.400526in}{1.185304in}}{\pgfqpoint{8.409359in}{1.181646in}}{\pgfqpoint{8.418567in}{1.181646in}}%
\pgfpathlineto{\pgfqpoint{8.418567in}{1.181646in}}%
\pgfpathclose%
\pgfusepath{stroke}%
\end{pgfscope}%
\begin{pgfscope}%
\pgfpathrectangle{\pgfqpoint{0.417359in}{0.814008in}}{\pgfqpoint{12.309552in}{4.722875in}}%
\pgfusepath{clip}%
\pgfsetbuttcap%
\pgfsetroundjoin%
\pgfsetlinewidth{1.003750pt}%
\definecolor{currentstroke}{rgb}{0.450000,0.450000,0.450000}%
\pgfsetstrokecolor{currentstroke}%
\pgfsetdash{}{0pt}%
\pgfpathmoveto{\pgfqpoint{8.418567in}{0.787458in}}%
\pgfpathcurveto{\pgfqpoint{8.427776in}{0.787458in}}{\pgfqpoint{8.436608in}{0.791117in}}{\pgfqpoint{8.443120in}{0.797628in}}%
\pgfpathcurveto{\pgfqpoint{8.449631in}{0.804140in}}{\pgfqpoint{8.453290in}{0.812972in}}{\pgfqpoint{8.453290in}{0.822181in}}%
\pgfpathcurveto{\pgfqpoint{8.453290in}{0.831389in}}{\pgfqpoint{8.449631in}{0.840222in}}{\pgfqpoint{8.443120in}{0.846733in}}%
\pgfpathcurveto{\pgfqpoint{8.436608in}{0.853244in}}{\pgfqpoint{8.427776in}{0.856903in}}{\pgfqpoint{8.418567in}{0.856903in}}%
\pgfpathcurveto{\pgfqpoint{8.409359in}{0.856903in}}{\pgfqpoint{8.400526in}{0.853244in}}{\pgfqpoint{8.394015in}{0.846733in}}%
\pgfpathcurveto{\pgfqpoint{8.387504in}{0.840222in}}{\pgfqpoint{8.383845in}{0.831389in}}{\pgfqpoint{8.383845in}{0.822181in}}%
\pgfpathcurveto{\pgfqpoint{8.383845in}{0.812972in}}{\pgfqpoint{8.387504in}{0.804140in}}{\pgfqpoint{8.394015in}{0.797628in}}%
\pgfpathcurveto{\pgfqpoint{8.400526in}{0.791117in}}{\pgfqpoint{8.409359in}{0.787458in}}{\pgfqpoint{8.418567in}{0.787458in}}%
\pgfusepath{stroke}%
\end{pgfscope}%
\begin{pgfscope}%
\pgfpathrectangle{\pgfqpoint{0.417359in}{0.814008in}}{\pgfqpoint{12.309552in}{4.722875in}}%
\pgfusepath{clip}%
\pgfsetbuttcap%
\pgfsetroundjoin%
\definecolor{currentfill}{rgb}{0.498039,0.498039,0.498039}%
\pgfsetfillcolor{currentfill}%
\pgfsetlinewidth{0.752812pt}%
\definecolor{currentstroke}{rgb}{0.240000,0.240000,0.240000}%
\pgfsetstrokecolor{currentstroke}%
\pgfsetdash{}{0pt}%
\pgfsys@defobject{currentmarker}{\pgfqpoint{9.157140in}{1.063487in}}{\pgfqpoint{10.141905in}{1.341194in}}{%
\pgfpathmoveto{\pgfqpoint{9.157140in}{1.063487in}}%
\pgfpathlineto{\pgfqpoint{10.141905in}{1.063487in}}%
\pgfpathlineto{\pgfqpoint{10.141905in}{1.341194in}}%
\pgfpathlineto{\pgfqpoint{9.157140in}{1.341194in}}%
\pgfpathlineto{\pgfqpoint{9.157140in}{1.063487in}}%
\pgfpathclose%
\pgfusepath{stroke,fill}%
}%
\begin{pgfscope}%
\pgfsys@transformshift{0.000000in}{0.000000in}%
\pgfsys@useobject{currentmarker}{}%
\end{pgfscope}%
\end{pgfscope}%
\begin{pgfscope}%
\pgfpathrectangle{\pgfqpoint{0.417359in}{0.814008in}}{\pgfqpoint{12.309552in}{4.722875in}}%
\pgfusepath{clip}%
\pgfsetbuttcap%
\pgfsetroundjoin%
\pgfsetlinewidth{1.003750pt}%
\definecolor{currentstroke}{rgb}{0.450000,0.450000,0.450000}%
\pgfsetstrokecolor{currentstroke}%
\pgfsetdash{}{0pt}%
\pgfpathmoveto{\pgfqpoint{9.649522in}{0.884412in}}%
\pgfpathcurveto{\pgfqpoint{9.658731in}{0.884412in}}{\pgfqpoint{9.667563in}{0.888071in}}{\pgfqpoint{9.674075in}{0.894582in}}%
\pgfpathcurveto{\pgfqpoint{9.680586in}{0.901093in}}{\pgfqpoint{9.684245in}{0.909926in}}{\pgfqpoint{9.684245in}{0.919134in}}%
\pgfpathcurveto{\pgfqpoint{9.684245in}{0.928343in}}{\pgfqpoint{9.680586in}{0.937175in}}{\pgfqpoint{9.674075in}{0.943687in}}%
\pgfpathcurveto{\pgfqpoint{9.667563in}{0.950198in}}{\pgfqpoint{9.658731in}{0.953857in}}{\pgfqpoint{9.649522in}{0.953857in}}%
\pgfpathcurveto{\pgfqpoint{9.640314in}{0.953857in}}{\pgfqpoint{9.631481in}{0.950198in}}{\pgfqpoint{9.624970in}{0.943687in}}%
\pgfpathcurveto{\pgfqpoint{9.618459in}{0.937175in}}{\pgfqpoint{9.614800in}{0.928343in}}{\pgfqpoint{9.614800in}{0.919134in}}%
\pgfpathcurveto{\pgfqpoint{9.614800in}{0.909926in}}{\pgfqpoint{9.618459in}{0.901093in}}{\pgfqpoint{9.624970in}{0.894582in}}%
\pgfpathcurveto{\pgfqpoint{9.631481in}{0.888071in}}{\pgfqpoint{9.640314in}{0.884412in}}{\pgfqpoint{9.649522in}{0.884412in}}%
\pgfpathlineto{\pgfqpoint{9.649522in}{0.884412in}}%
\pgfpathclose%
\pgfusepath{stroke}%
\end{pgfscope}%
\begin{pgfscope}%
\pgfpathrectangle{\pgfqpoint{0.417359in}{0.814008in}}{\pgfqpoint{12.309552in}{4.722875in}}%
\pgfusepath{clip}%
\pgfsetbuttcap%
\pgfsetroundjoin%
\pgfsetlinewidth{1.003750pt}%
\definecolor{currentstroke}{rgb}{0.450000,0.450000,0.450000}%
\pgfsetstrokecolor{currentstroke}%
\pgfsetdash{}{0pt}%
\pgfpathmoveto{\pgfqpoint{9.649522in}{0.884412in}}%
\pgfpathcurveto{\pgfqpoint{9.658731in}{0.884412in}}{\pgfqpoint{9.667563in}{0.888071in}}{\pgfqpoint{9.674075in}{0.894582in}}%
\pgfpathcurveto{\pgfqpoint{9.680586in}{0.901093in}}{\pgfqpoint{9.684245in}{0.909926in}}{\pgfqpoint{9.684245in}{0.919134in}}%
\pgfpathcurveto{\pgfqpoint{9.684245in}{0.928343in}}{\pgfqpoint{9.680586in}{0.937175in}}{\pgfqpoint{9.674075in}{0.943687in}}%
\pgfpathcurveto{\pgfqpoint{9.667563in}{0.950198in}}{\pgfqpoint{9.658731in}{0.953857in}}{\pgfqpoint{9.649522in}{0.953857in}}%
\pgfpathcurveto{\pgfqpoint{9.640314in}{0.953857in}}{\pgfqpoint{9.631481in}{0.950198in}}{\pgfqpoint{9.624970in}{0.943687in}}%
\pgfpathcurveto{\pgfqpoint{9.618459in}{0.937175in}}{\pgfqpoint{9.614800in}{0.928343in}}{\pgfqpoint{9.614800in}{0.919134in}}%
\pgfpathcurveto{\pgfqpoint{9.614800in}{0.909926in}}{\pgfqpoint{9.618459in}{0.901093in}}{\pgfqpoint{9.624970in}{0.894582in}}%
\pgfpathcurveto{\pgfqpoint{9.631481in}{0.888071in}}{\pgfqpoint{9.640314in}{0.884412in}}{\pgfqpoint{9.649522in}{0.884412in}}%
\pgfpathlineto{\pgfqpoint{9.649522in}{0.884412in}}%
\pgfpathclose%
\pgfusepath{stroke}%
\end{pgfscope}%
\begin{pgfscope}%
\pgfpathrectangle{\pgfqpoint{0.417359in}{0.814008in}}{\pgfqpoint{12.309552in}{4.722875in}}%
\pgfusepath{clip}%
\pgfsetbuttcap%
\pgfsetroundjoin%
\pgfsetlinewidth{1.003750pt}%
\definecolor{currentstroke}{rgb}{0.450000,0.450000,0.450000}%
\pgfsetstrokecolor{currentstroke}%
\pgfsetdash{}{0pt}%
\pgfpathmoveto{\pgfqpoint{9.649522in}{0.793761in}}%
\pgfpathcurveto{\pgfqpoint{9.658731in}{0.793761in}}{\pgfqpoint{9.667563in}{0.797419in}}{\pgfqpoint{9.674075in}{0.803930in}}%
\pgfpathcurveto{\pgfqpoint{9.680586in}{0.810442in}}{\pgfqpoint{9.684245in}{0.819274in}}{\pgfqpoint{9.684245in}{0.828483in}}%
\pgfpathcurveto{\pgfqpoint{9.684245in}{0.837691in}}{\pgfqpoint{9.680586in}{0.846524in}}{\pgfqpoint{9.674075in}{0.853035in}}%
\pgfpathcurveto{\pgfqpoint{9.667563in}{0.859546in}}{\pgfqpoint{9.658731in}{0.863205in}}{\pgfqpoint{9.649522in}{0.863205in}}%
\pgfpathcurveto{\pgfqpoint{9.640314in}{0.863205in}}{\pgfqpoint{9.631481in}{0.859546in}}{\pgfqpoint{9.624970in}{0.853035in}}%
\pgfpathcurveto{\pgfqpoint{9.618459in}{0.846524in}}{\pgfqpoint{9.614800in}{0.837691in}}{\pgfqpoint{9.614800in}{0.828483in}}%
\pgfpathcurveto{\pgfqpoint{9.614800in}{0.819274in}}{\pgfqpoint{9.618459in}{0.810442in}}{\pgfqpoint{9.624970in}{0.803930in}}%
\pgfpathcurveto{\pgfqpoint{9.631481in}{0.797419in}}{\pgfqpoint{9.640314in}{0.793761in}}{\pgfqpoint{9.649522in}{0.793761in}}%
\pgfusepath{stroke}%
\end{pgfscope}%
\begin{pgfscope}%
\pgfpathrectangle{\pgfqpoint{0.417359in}{0.814008in}}{\pgfqpoint{12.309552in}{4.722875in}}%
\pgfusepath{clip}%
\pgfsetbuttcap%
\pgfsetroundjoin%
\pgfsetlinewidth{1.003750pt}%
\definecolor{currentstroke}{rgb}{0.450000,0.450000,0.450000}%
\pgfsetstrokecolor{currentstroke}%
\pgfsetdash{}{0pt}%
\pgfpathmoveto{\pgfqpoint{9.649522in}{2.008331in}}%
\pgfpathcurveto{\pgfqpoint{9.658731in}{2.008331in}}{\pgfqpoint{9.667563in}{2.011990in}}{\pgfqpoint{9.674075in}{2.018501in}}%
\pgfpathcurveto{\pgfqpoint{9.680586in}{2.025012in}}{\pgfqpoint{9.684245in}{2.033845in}}{\pgfqpoint{9.684245in}{2.043053in}}%
\pgfpathcurveto{\pgfqpoint{9.684245in}{2.052262in}}{\pgfqpoint{9.680586in}{2.061094in}}{\pgfqpoint{9.674075in}{2.067606in}}%
\pgfpathcurveto{\pgfqpoint{9.667563in}{2.074117in}}{\pgfqpoint{9.658731in}{2.077776in}}{\pgfqpoint{9.649522in}{2.077776in}}%
\pgfpathcurveto{\pgfqpoint{9.640314in}{2.077776in}}{\pgfqpoint{9.631481in}{2.074117in}}{\pgfqpoint{9.624970in}{2.067606in}}%
\pgfpathcurveto{\pgfqpoint{9.618459in}{2.061094in}}{\pgfqpoint{9.614800in}{2.052262in}}{\pgfqpoint{9.614800in}{2.043053in}}%
\pgfpathcurveto{\pgfqpoint{9.614800in}{2.033845in}}{\pgfqpoint{9.618459in}{2.025012in}}{\pgfqpoint{9.624970in}{2.018501in}}%
\pgfpathcurveto{\pgfqpoint{9.631481in}{2.011990in}}{\pgfqpoint{9.640314in}{2.008331in}}{\pgfqpoint{9.649522in}{2.008331in}}%
\pgfpathlineto{\pgfqpoint{9.649522in}{2.008331in}}%
\pgfpathclose%
\pgfusepath{stroke}%
\end{pgfscope}%
\begin{pgfscope}%
\pgfpathrectangle{\pgfqpoint{0.417359in}{0.814008in}}{\pgfqpoint{12.309552in}{4.722875in}}%
\pgfusepath{clip}%
\pgfsetbuttcap%
\pgfsetroundjoin%
\pgfsetlinewidth{1.003750pt}%
\definecolor{currentstroke}{rgb}{0.450000,0.450000,0.450000}%
\pgfsetstrokecolor{currentstroke}%
\pgfsetdash{}{0pt}%
\pgfpathmoveto{\pgfqpoint{9.649522in}{1.433726in}}%
\pgfpathcurveto{\pgfqpoint{9.658731in}{1.433726in}}{\pgfqpoint{9.667563in}{1.437385in}}{\pgfqpoint{9.674075in}{1.443896in}}%
\pgfpathcurveto{\pgfqpoint{9.680586in}{1.450407in}}{\pgfqpoint{9.684245in}{1.459240in}}{\pgfqpoint{9.684245in}{1.468448in}}%
\pgfpathcurveto{\pgfqpoint{9.684245in}{1.477657in}}{\pgfqpoint{9.680586in}{1.486489in}}{\pgfqpoint{9.674075in}{1.493001in}}%
\pgfpathcurveto{\pgfqpoint{9.667563in}{1.499512in}}{\pgfqpoint{9.658731in}{1.503170in}}{\pgfqpoint{9.649522in}{1.503170in}}%
\pgfpathcurveto{\pgfqpoint{9.640314in}{1.503170in}}{\pgfqpoint{9.631481in}{1.499512in}}{\pgfqpoint{9.624970in}{1.493001in}}%
\pgfpathcurveto{\pgfqpoint{9.618459in}{1.486489in}}{\pgfqpoint{9.614800in}{1.477657in}}{\pgfqpoint{9.614800in}{1.468448in}}%
\pgfpathcurveto{\pgfqpoint{9.614800in}{1.459240in}}{\pgfqpoint{9.618459in}{1.450407in}}{\pgfqpoint{9.624970in}{1.443896in}}%
\pgfpathcurveto{\pgfqpoint{9.631481in}{1.437385in}}{\pgfqpoint{9.640314in}{1.433726in}}{\pgfqpoint{9.649522in}{1.433726in}}%
\pgfpathlineto{\pgfqpoint{9.649522in}{1.433726in}}%
\pgfpathclose%
\pgfusepath{stroke}%
\end{pgfscope}%
\begin{pgfscope}%
\pgfpathrectangle{\pgfqpoint{0.417359in}{0.814008in}}{\pgfqpoint{12.309552in}{4.722875in}}%
\pgfusepath{clip}%
\pgfsetbuttcap%
\pgfsetroundjoin%
\pgfsetlinewidth{1.003750pt}%
\definecolor{currentstroke}{rgb}{0.450000,0.450000,0.450000}%
\pgfsetstrokecolor{currentstroke}%
\pgfsetdash{}{0pt}%
\pgfpathmoveto{\pgfqpoint{9.649522in}{1.519328in}}%
\pgfpathcurveto{\pgfqpoint{9.658731in}{1.519328in}}{\pgfqpoint{9.667563in}{1.522987in}}{\pgfqpoint{9.674075in}{1.529498in}}%
\pgfpathcurveto{\pgfqpoint{9.680586in}{1.536009in}}{\pgfqpoint{9.684245in}{1.544842in}}{\pgfqpoint{9.684245in}{1.554050in}}%
\pgfpathcurveto{\pgfqpoint{9.684245in}{1.563259in}}{\pgfqpoint{9.680586in}{1.572091in}}{\pgfqpoint{9.674075in}{1.578603in}}%
\pgfpathcurveto{\pgfqpoint{9.667563in}{1.585114in}}{\pgfqpoint{9.658731in}{1.588773in}}{\pgfqpoint{9.649522in}{1.588773in}}%
\pgfpathcurveto{\pgfqpoint{9.640314in}{1.588773in}}{\pgfqpoint{9.631481in}{1.585114in}}{\pgfqpoint{9.624970in}{1.578603in}}%
\pgfpathcurveto{\pgfqpoint{9.618459in}{1.572091in}}{\pgfqpoint{9.614800in}{1.563259in}}{\pgfqpoint{9.614800in}{1.554050in}}%
\pgfpathcurveto{\pgfqpoint{9.614800in}{1.544842in}}{\pgfqpoint{9.618459in}{1.536009in}}{\pgfqpoint{9.624970in}{1.529498in}}%
\pgfpathcurveto{\pgfqpoint{9.631481in}{1.522987in}}{\pgfqpoint{9.640314in}{1.519328in}}{\pgfqpoint{9.649522in}{1.519328in}}%
\pgfpathlineto{\pgfqpoint{9.649522in}{1.519328in}}%
\pgfpathclose%
\pgfusepath{stroke}%
\end{pgfscope}%
\begin{pgfscope}%
\pgfpathrectangle{\pgfqpoint{0.417359in}{0.814008in}}{\pgfqpoint{12.309552in}{4.722875in}}%
\pgfusepath{clip}%
\pgfsetbuttcap%
\pgfsetroundjoin%
\pgfsetlinewidth{1.003750pt}%
\definecolor{currentstroke}{rgb}{0.450000,0.450000,0.450000}%
\pgfsetstrokecolor{currentstroke}%
\pgfsetdash{}{0pt}%
\pgfpathmoveto{\pgfqpoint{9.649522in}{1.413315in}}%
\pgfpathcurveto{\pgfqpoint{9.658731in}{1.413315in}}{\pgfqpoint{9.667563in}{1.416973in}}{\pgfqpoint{9.674075in}{1.423485in}}%
\pgfpathcurveto{\pgfqpoint{9.680586in}{1.429996in}}{\pgfqpoint{9.684245in}{1.438829in}}{\pgfqpoint{9.684245in}{1.448037in}}%
\pgfpathcurveto{\pgfqpoint{9.684245in}{1.457246in}}{\pgfqpoint{9.680586in}{1.466078in}}{\pgfqpoint{9.674075in}{1.472589in}}%
\pgfpathcurveto{\pgfqpoint{9.667563in}{1.479101in}}{\pgfqpoint{9.658731in}{1.482759in}}{\pgfqpoint{9.649522in}{1.482759in}}%
\pgfpathcurveto{\pgfqpoint{9.640314in}{1.482759in}}{\pgfqpoint{9.631481in}{1.479101in}}{\pgfqpoint{9.624970in}{1.472589in}}%
\pgfpathcurveto{\pgfqpoint{9.618459in}{1.466078in}}{\pgfqpoint{9.614800in}{1.457246in}}{\pgfqpoint{9.614800in}{1.448037in}}%
\pgfpathcurveto{\pgfqpoint{9.614800in}{1.438829in}}{\pgfqpoint{9.618459in}{1.429996in}}{\pgfqpoint{9.624970in}{1.423485in}}%
\pgfpathcurveto{\pgfqpoint{9.631481in}{1.416973in}}{\pgfqpoint{9.640314in}{1.413315in}}{\pgfqpoint{9.649522in}{1.413315in}}%
\pgfpathlineto{\pgfqpoint{9.649522in}{1.413315in}}%
\pgfpathclose%
\pgfusepath{stroke}%
\end{pgfscope}%
\begin{pgfscope}%
\pgfpathrectangle{\pgfqpoint{0.417359in}{0.814008in}}{\pgfqpoint{12.309552in}{4.722875in}}%
\pgfusepath{clip}%
\pgfsetbuttcap%
\pgfsetroundjoin%
\pgfsetlinewidth{1.003750pt}%
\definecolor{currentstroke}{rgb}{0.450000,0.450000,0.450000}%
\pgfsetstrokecolor{currentstroke}%
\pgfsetdash{}{0pt}%
\pgfpathmoveto{\pgfqpoint{9.649522in}{3.057164in}}%
\pgfpathcurveto{\pgfqpoint{9.658731in}{3.057164in}}{\pgfqpoint{9.667563in}{3.060822in}}{\pgfqpoint{9.674075in}{3.067333in}}%
\pgfpathcurveto{\pgfqpoint{9.680586in}{3.073845in}}{\pgfqpoint{9.684245in}{3.082677in}}{\pgfqpoint{9.684245in}{3.091886in}}%
\pgfpathcurveto{\pgfqpoint{9.684245in}{3.101094in}}{\pgfqpoint{9.680586in}{3.109927in}}{\pgfqpoint{9.674075in}{3.116438in}}%
\pgfpathcurveto{\pgfqpoint{9.667563in}{3.122949in}}{\pgfqpoint{9.658731in}{3.126608in}}{\pgfqpoint{9.649522in}{3.126608in}}%
\pgfpathcurveto{\pgfqpoint{9.640314in}{3.126608in}}{\pgfqpoint{9.631481in}{3.122949in}}{\pgfqpoint{9.624970in}{3.116438in}}%
\pgfpathcurveto{\pgfqpoint{9.618459in}{3.109927in}}{\pgfqpoint{9.614800in}{3.101094in}}{\pgfqpoint{9.614800in}{3.091886in}}%
\pgfpathcurveto{\pgfqpoint{9.614800in}{3.082677in}}{\pgfqpoint{9.618459in}{3.073845in}}{\pgfqpoint{9.624970in}{3.067333in}}%
\pgfpathcurveto{\pgfqpoint{9.631481in}{3.060822in}}{\pgfqpoint{9.640314in}{3.057164in}}{\pgfqpoint{9.649522in}{3.057164in}}%
\pgfpathlineto{\pgfqpoint{9.649522in}{3.057164in}}%
\pgfpathclose%
\pgfusepath{stroke}%
\end{pgfscope}%
\begin{pgfscope}%
\pgfpathrectangle{\pgfqpoint{0.417359in}{0.814008in}}{\pgfqpoint{12.309552in}{4.722875in}}%
\pgfusepath{clip}%
\pgfsetbuttcap%
\pgfsetroundjoin%
\pgfsetlinewidth{1.003750pt}%
\definecolor{currentstroke}{rgb}{0.450000,0.450000,0.450000}%
\pgfsetstrokecolor{currentstroke}%
\pgfsetdash{}{0pt}%
\pgfpathmoveto{\pgfqpoint{9.649522in}{1.025221in}}%
\pgfpathcurveto{\pgfqpoint{9.658731in}{1.025221in}}{\pgfqpoint{9.667563in}{1.028880in}}{\pgfqpoint{9.674075in}{1.035391in}}%
\pgfpathcurveto{\pgfqpoint{9.680586in}{1.041902in}}{\pgfqpoint{9.684245in}{1.050735in}}{\pgfqpoint{9.684245in}{1.059943in}}%
\pgfpathcurveto{\pgfqpoint{9.684245in}{1.069152in}}{\pgfqpoint{9.680586in}{1.077984in}}{\pgfqpoint{9.674075in}{1.084496in}}%
\pgfpathcurveto{\pgfqpoint{9.667563in}{1.091007in}}{\pgfqpoint{9.658731in}{1.094665in}}{\pgfqpoint{9.649522in}{1.094665in}}%
\pgfpathcurveto{\pgfqpoint{9.640314in}{1.094665in}}{\pgfqpoint{9.631481in}{1.091007in}}{\pgfqpoint{9.624970in}{1.084496in}}%
\pgfpathcurveto{\pgfqpoint{9.618459in}{1.077984in}}{\pgfqpoint{9.614800in}{1.069152in}}{\pgfqpoint{9.614800in}{1.059943in}}%
\pgfpathcurveto{\pgfqpoint{9.614800in}{1.050735in}}{\pgfqpoint{9.618459in}{1.041902in}}{\pgfqpoint{9.624970in}{1.035391in}}%
\pgfpathcurveto{\pgfqpoint{9.631481in}{1.028880in}}{\pgfqpoint{9.640314in}{1.025221in}}{\pgfqpoint{9.649522in}{1.025221in}}%
\pgfpathlineto{\pgfqpoint{9.649522in}{1.025221in}}%
\pgfpathclose%
\pgfusepath{stroke}%
\end{pgfscope}%
\begin{pgfscope}%
\pgfpathrectangle{\pgfqpoint{0.417359in}{0.814008in}}{\pgfqpoint{12.309552in}{4.722875in}}%
\pgfusepath{clip}%
\pgfsetbuttcap%
\pgfsetroundjoin%
\pgfsetlinewidth{1.003750pt}%
\definecolor{currentstroke}{rgb}{0.450000,0.450000,0.450000}%
\pgfsetstrokecolor{currentstroke}%
\pgfsetdash{}{0pt}%
\pgfpathmoveto{\pgfqpoint{9.649522in}{0.793931in}}%
\pgfpathcurveto{\pgfqpoint{9.658731in}{0.793931in}}{\pgfqpoint{9.667563in}{0.797589in}}{\pgfqpoint{9.674075in}{0.804101in}}%
\pgfpathcurveto{\pgfqpoint{9.680586in}{0.810612in}}{\pgfqpoint{9.684245in}{0.819445in}}{\pgfqpoint{9.684245in}{0.828653in}}%
\pgfpathcurveto{\pgfqpoint{9.684245in}{0.837862in}}{\pgfqpoint{9.680586in}{0.846694in}}{\pgfqpoint{9.674075in}{0.853205in}}%
\pgfpathcurveto{\pgfqpoint{9.667563in}{0.859717in}}{\pgfqpoint{9.658731in}{0.863375in}}{\pgfqpoint{9.649522in}{0.863375in}}%
\pgfpathcurveto{\pgfqpoint{9.640314in}{0.863375in}}{\pgfqpoint{9.631481in}{0.859717in}}{\pgfqpoint{9.624970in}{0.853205in}}%
\pgfpathcurveto{\pgfqpoint{9.618459in}{0.846694in}}{\pgfqpoint{9.614800in}{0.837862in}}{\pgfqpoint{9.614800in}{0.828653in}}%
\pgfpathcurveto{\pgfqpoint{9.614800in}{0.819445in}}{\pgfqpoint{9.618459in}{0.810612in}}{\pgfqpoint{9.624970in}{0.804101in}}%
\pgfpathcurveto{\pgfqpoint{9.631481in}{0.797589in}}{\pgfqpoint{9.640314in}{0.793931in}}{\pgfqpoint{9.649522in}{0.793931in}}%
\pgfusepath{stroke}%
\end{pgfscope}%
\begin{pgfscope}%
\pgfpathrectangle{\pgfqpoint{0.417359in}{0.814008in}}{\pgfqpoint{12.309552in}{4.722875in}}%
\pgfusepath{clip}%
\pgfsetbuttcap%
\pgfsetroundjoin%
\pgfsetlinewidth{1.003750pt}%
\definecolor{currentstroke}{rgb}{0.450000,0.450000,0.450000}%
\pgfsetstrokecolor{currentstroke}%
\pgfsetdash{}{0pt}%
\pgfpathmoveto{\pgfqpoint{9.649522in}{0.793950in}}%
\pgfpathcurveto{\pgfqpoint{9.658731in}{0.793950in}}{\pgfqpoint{9.667563in}{0.797608in}}{\pgfqpoint{9.674075in}{0.804119in}}%
\pgfpathcurveto{\pgfqpoint{9.680586in}{0.810631in}}{\pgfqpoint{9.684245in}{0.819463in}}{\pgfqpoint{9.684245in}{0.828672in}}%
\pgfpathcurveto{\pgfqpoint{9.684245in}{0.837880in}}{\pgfqpoint{9.680586in}{0.846713in}}{\pgfqpoint{9.674075in}{0.853224in}}%
\pgfpathcurveto{\pgfqpoint{9.667563in}{0.859735in}}{\pgfqpoint{9.658731in}{0.863394in}}{\pgfqpoint{9.649522in}{0.863394in}}%
\pgfpathcurveto{\pgfqpoint{9.640314in}{0.863394in}}{\pgfqpoint{9.631481in}{0.859735in}}{\pgfqpoint{9.624970in}{0.853224in}}%
\pgfpathcurveto{\pgfqpoint{9.618459in}{0.846713in}}{\pgfqpoint{9.614800in}{0.837880in}}{\pgfqpoint{9.614800in}{0.828672in}}%
\pgfpathcurveto{\pgfqpoint{9.614800in}{0.819463in}}{\pgfqpoint{9.618459in}{0.810631in}}{\pgfqpoint{9.624970in}{0.804119in}}%
\pgfpathcurveto{\pgfqpoint{9.631481in}{0.797608in}}{\pgfqpoint{9.640314in}{0.793950in}}{\pgfqpoint{9.649522in}{0.793950in}}%
\pgfusepath{stroke}%
\end{pgfscope}%
\begin{pgfscope}%
\pgfpathrectangle{\pgfqpoint{0.417359in}{0.814008in}}{\pgfqpoint{12.309552in}{4.722875in}}%
\pgfusepath{clip}%
\pgfsetbuttcap%
\pgfsetroundjoin%
\definecolor{currentfill}{rgb}{0.662255,0.665196,0.209314}%
\pgfsetfillcolor{currentfill}%
\pgfsetlinewidth{0.752812pt}%
\definecolor{currentstroke}{rgb}{0.240000,0.240000,0.240000}%
\pgfsetstrokecolor{currentstroke}%
\pgfsetdash{}{0pt}%
\pgfsys@defobject{currentmarker}{\pgfqpoint{10.388096in}{0.896537in}}{\pgfqpoint{11.372860in}{1.777273in}}{%
\pgfpathmoveto{\pgfqpoint{10.388096in}{0.896537in}}%
\pgfpathlineto{\pgfqpoint{11.372860in}{0.896537in}}%
\pgfpathlineto{\pgfqpoint{11.372860in}{1.777273in}}%
\pgfpathlineto{\pgfqpoint{10.388096in}{1.777273in}}%
\pgfpathlineto{\pgfqpoint{10.388096in}{0.896537in}}%
\pgfpathclose%
\pgfusepath{stroke,fill}%
}%
\begin{pgfscope}%
\pgfsys@transformshift{0.000000in}{0.000000in}%
\pgfsys@useobject{currentmarker}{}%
\end{pgfscope}%
\end{pgfscope}%
\begin{pgfscope}%
\pgfpathrectangle{\pgfqpoint{0.417359in}{0.814008in}}{\pgfqpoint{12.309552in}{4.722875in}}%
\pgfusepath{clip}%
\pgfsetbuttcap%
\pgfsetroundjoin%
\pgfsetlinewidth{1.003750pt}%
\definecolor{currentstroke}{rgb}{0.450000,0.450000,0.450000}%
\pgfsetstrokecolor{currentstroke}%
\pgfsetdash{}{0pt}%
\pgfpathmoveto{\pgfqpoint{10.880478in}{0.825500in}}%
\pgfpathcurveto{\pgfqpoint{10.889686in}{0.825500in}}{\pgfqpoint{10.898519in}{0.829158in}}{\pgfqpoint{10.905030in}{0.835669in}}%
\pgfpathcurveto{\pgfqpoint{10.911541in}{0.842181in}}{\pgfqpoint{10.915200in}{0.851013in}}{\pgfqpoint{10.915200in}{0.860222in}}%
\pgfpathcurveto{\pgfqpoint{10.915200in}{0.869430in}}{\pgfqpoint{10.911541in}{0.878263in}}{\pgfqpoint{10.905030in}{0.884774in}}%
\pgfpathcurveto{\pgfqpoint{10.898519in}{0.891285in}}{\pgfqpoint{10.889686in}{0.894944in}}{\pgfqpoint{10.880478in}{0.894944in}}%
\pgfpathcurveto{\pgfqpoint{10.871269in}{0.894944in}}{\pgfqpoint{10.862437in}{0.891285in}}{\pgfqpoint{10.855925in}{0.884774in}}%
\pgfpathcurveto{\pgfqpoint{10.849414in}{0.878263in}}{\pgfqpoint{10.845755in}{0.869430in}}{\pgfqpoint{10.845755in}{0.860222in}}%
\pgfpathcurveto{\pgfqpoint{10.845755in}{0.851013in}}{\pgfqpoint{10.849414in}{0.842181in}}{\pgfqpoint{10.855925in}{0.835669in}}%
\pgfpathcurveto{\pgfqpoint{10.862437in}{0.829158in}}{\pgfqpoint{10.871269in}{0.825500in}}{\pgfqpoint{10.880478in}{0.825500in}}%
\pgfpathlineto{\pgfqpoint{10.880478in}{0.825500in}}%
\pgfpathclose%
\pgfusepath{stroke}%
\end{pgfscope}%
\begin{pgfscope}%
\pgfpathrectangle{\pgfqpoint{0.417359in}{0.814008in}}{\pgfqpoint{12.309552in}{4.722875in}}%
\pgfusepath{clip}%
\pgfsetbuttcap%
\pgfsetroundjoin%
\pgfsetlinewidth{1.003750pt}%
\definecolor{currentstroke}{rgb}{0.450000,0.450000,0.450000}%
\pgfsetstrokecolor{currentstroke}%
\pgfsetdash{}{0pt}%
\pgfpathmoveto{\pgfqpoint{10.880478in}{0.792649in}}%
\pgfpathcurveto{\pgfqpoint{10.889686in}{0.792649in}}{\pgfqpoint{10.898519in}{0.796307in}}{\pgfqpoint{10.905030in}{0.802818in}}%
\pgfpathcurveto{\pgfqpoint{10.911541in}{0.809330in}}{\pgfqpoint{10.915200in}{0.818162in}}{\pgfqpoint{10.915200in}{0.827371in}}%
\pgfpathcurveto{\pgfqpoint{10.915200in}{0.836579in}}{\pgfqpoint{10.911541in}{0.845412in}}{\pgfqpoint{10.905030in}{0.851923in}}%
\pgfpathcurveto{\pgfqpoint{10.898519in}{0.858434in}}{\pgfqpoint{10.889686in}{0.862093in}}{\pgfqpoint{10.880478in}{0.862093in}}%
\pgfpathcurveto{\pgfqpoint{10.871269in}{0.862093in}}{\pgfqpoint{10.862437in}{0.858434in}}{\pgfqpoint{10.855925in}{0.851923in}}%
\pgfpathcurveto{\pgfqpoint{10.849414in}{0.845412in}}{\pgfqpoint{10.845755in}{0.836579in}}{\pgfqpoint{10.845755in}{0.827371in}}%
\pgfpathcurveto{\pgfqpoint{10.845755in}{0.818162in}}{\pgfqpoint{10.849414in}{0.809330in}}{\pgfqpoint{10.855925in}{0.802818in}}%
\pgfpathcurveto{\pgfqpoint{10.862437in}{0.796307in}}{\pgfqpoint{10.871269in}{0.792649in}}{\pgfqpoint{10.880478in}{0.792649in}}%
\pgfusepath{stroke}%
\end{pgfscope}%
\begin{pgfscope}%
\pgfpathrectangle{\pgfqpoint{0.417359in}{0.814008in}}{\pgfqpoint{12.309552in}{4.722875in}}%
\pgfusepath{clip}%
\pgfsetbuttcap%
\pgfsetroundjoin%
\pgfsetlinewidth{1.003750pt}%
\definecolor{currentstroke}{rgb}{0.450000,0.450000,0.450000}%
\pgfsetstrokecolor{currentstroke}%
\pgfsetdash{}{0pt}%
\pgfpathmoveto{\pgfqpoint{10.880478in}{1.747166in}}%
\pgfpathcurveto{\pgfqpoint{10.889686in}{1.747166in}}{\pgfqpoint{10.898519in}{1.750824in}}{\pgfqpoint{10.905030in}{1.757335in}}%
\pgfpathcurveto{\pgfqpoint{10.911541in}{1.763847in}}{\pgfqpoint{10.915200in}{1.772679in}}{\pgfqpoint{10.915200in}{1.781888in}}%
\pgfpathcurveto{\pgfqpoint{10.915200in}{1.791096in}}{\pgfqpoint{10.911541in}{1.799929in}}{\pgfqpoint{10.905030in}{1.806440in}}%
\pgfpathcurveto{\pgfqpoint{10.898519in}{1.812951in}}{\pgfqpoint{10.889686in}{1.816610in}}{\pgfqpoint{10.880478in}{1.816610in}}%
\pgfpathcurveto{\pgfqpoint{10.871269in}{1.816610in}}{\pgfqpoint{10.862437in}{1.812951in}}{\pgfqpoint{10.855925in}{1.806440in}}%
\pgfpathcurveto{\pgfqpoint{10.849414in}{1.799929in}}{\pgfqpoint{10.845755in}{1.791096in}}{\pgfqpoint{10.845755in}{1.781888in}}%
\pgfpathcurveto{\pgfqpoint{10.845755in}{1.772679in}}{\pgfqpoint{10.849414in}{1.763847in}}{\pgfqpoint{10.855925in}{1.757335in}}%
\pgfpathcurveto{\pgfqpoint{10.862437in}{1.750824in}}{\pgfqpoint{10.871269in}{1.747166in}}{\pgfqpoint{10.880478in}{1.747166in}}%
\pgfpathlineto{\pgfqpoint{10.880478in}{1.747166in}}%
\pgfpathclose%
\pgfusepath{stroke}%
\end{pgfscope}%
\begin{pgfscope}%
\pgfpathrectangle{\pgfqpoint{0.417359in}{0.814008in}}{\pgfqpoint{12.309552in}{4.722875in}}%
\pgfusepath{clip}%
\pgfsetbuttcap%
\pgfsetroundjoin%
\pgfsetlinewidth{1.003750pt}%
\definecolor{currentstroke}{rgb}{0.450000,0.450000,0.450000}%
\pgfsetstrokecolor{currentstroke}%
\pgfsetdash{}{0pt}%
\pgfpathmoveto{\pgfqpoint{10.880478in}{1.747166in}}%
\pgfpathcurveto{\pgfqpoint{10.889686in}{1.747166in}}{\pgfqpoint{10.898519in}{1.750824in}}{\pgfqpoint{10.905030in}{1.757335in}}%
\pgfpathcurveto{\pgfqpoint{10.911541in}{1.763847in}}{\pgfqpoint{10.915200in}{1.772679in}}{\pgfqpoint{10.915200in}{1.781888in}}%
\pgfpathcurveto{\pgfqpoint{10.915200in}{1.791096in}}{\pgfqpoint{10.911541in}{1.799929in}}{\pgfqpoint{10.905030in}{1.806440in}}%
\pgfpathcurveto{\pgfqpoint{10.898519in}{1.812951in}}{\pgfqpoint{10.889686in}{1.816610in}}{\pgfqpoint{10.880478in}{1.816610in}}%
\pgfpathcurveto{\pgfqpoint{10.871269in}{1.816610in}}{\pgfqpoint{10.862437in}{1.812951in}}{\pgfqpoint{10.855925in}{1.806440in}}%
\pgfpathcurveto{\pgfqpoint{10.849414in}{1.799929in}}{\pgfqpoint{10.845755in}{1.791096in}}{\pgfqpoint{10.845755in}{1.781888in}}%
\pgfpathcurveto{\pgfqpoint{10.845755in}{1.772679in}}{\pgfqpoint{10.849414in}{1.763847in}}{\pgfqpoint{10.855925in}{1.757335in}}%
\pgfpathcurveto{\pgfqpoint{10.862437in}{1.750824in}}{\pgfqpoint{10.871269in}{1.747166in}}{\pgfqpoint{10.880478in}{1.747166in}}%
\pgfpathlineto{\pgfqpoint{10.880478in}{1.747166in}}%
\pgfpathclose%
\pgfusepath{stroke}%
\end{pgfscope}%
\begin{pgfscope}%
\pgfpathrectangle{\pgfqpoint{0.417359in}{0.814008in}}{\pgfqpoint{12.309552in}{4.722875in}}%
\pgfusepath{clip}%
\pgfsetbuttcap%
\pgfsetroundjoin%
\pgfsetlinewidth{1.003750pt}%
\definecolor{currentstroke}{rgb}{0.450000,0.450000,0.450000}%
\pgfsetstrokecolor{currentstroke}%
\pgfsetdash{}{0pt}%
\pgfpathmoveto{\pgfqpoint{10.880478in}{0.832302in}}%
\pgfpathcurveto{\pgfqpoint{10.889686in}{0.832302in}}{\pgfqpoint{10.898519in}{0.835960in}}{\pgfqpoint{10.905030in}{0.842472in}}%
\pgfpathcurveto{\pgfqpoint{10.911541in}{0.848983in}}{\pgfqpoint{10.915200in}{0.857815in}}{\pgfqpoint{10.915200in}{0.867024in}}%
\pgfpathcurveto{\pgfqpoint{10.915200in}{0.876232in}}{\pgfqpoint{10.911541in}{0.885065in}}{\pgfqpoint{10.905030in}{0.891576in}}%
\pgfpathcurveto{\pgfqpoint{10.898519in}{0.898088in}}{\pgfqpoint{10.889686in}{0.901746in}}{\pgfqpoint{10.880478in}{0.901746in}}%
\pgfpathcurveto{\pgfqpoint{10.871269in}{0.901746in}}{\pgfqpoint{10.862437in}{0.898088in}}{\pgfqpoint{10.855925in}{0.891576in}}%
\pgfpathcurveto{\pgfqpoint{10.849414in}{0.885065in}}{\pgfqpoint{10.845755in}{0.876232in}}{\pgfqpoint{10.845755in}{0.867024in}}%
\pgfpathcurveto{\pgfqpoint{10.845755in}{0.857815in}}{\pgfqpoint{10.849414in}{0.848983in}}{\pgfqpoint{10.855925in}{0.842472in}}%
\pgfpathcurveto{\pgfqpoint{10.862437in}{0.835960in}}{\pgfqpoint{10.871269in}{0.832302in}}{\pgfqpoint{10.880478in}{0.832302in}}%
\pgfpathlineto{\pgfqpoint{10.880478in}{0.832302in}}%
\pgfpathclose%
\pgfusepath{stroke}%
\end{pgfscope}%
\begin{pgfscope}%
\pgfpathrectangle{\pgfqpoint{0.417359in}{0.814008in}}{\pgfqpoint{12.309552in}{4.722875in}}%
\pgfusepath{clip}%
\pgfsetbuttcap%
\pgfsetroundjoin%
\pgfsetlinewidth{1.003750pt}%
\definecolor{currentstroke}{rgb}{0.450000,0.450000,0.450000}%
\pgfsetstrokecolor{currentstroke}%
\pgfsetdash{}{0pt}%
\pgfpathmoveto{\pgfqpoint{10.880478in}{1.743835in}}%
\pgfpathcurveto{\pgfqpoint{10.889686in}{1.743835in}}{\pgfqpoint{10.898519in}{1.747494in}}{\pgfqpoint{10.905030in}{1.754005in}}%
\pgfpathcurveto{\pgfqpoint{10.911541in}{1.760517in}}{\pgfqpoint{10.915200in}{1.769349in}}{\pgfqpoint{10.915200in}{1.778558in}}%
\pgfpathcurveto{\pgfqpoint{10.915200in}{1.787766in}}{\pgfqpoint{10.911541in}{1.796599in}}{\pgfqpoint{10.905030in}{1.803110in}}%
\pgfpathcurveto{\pgfqpoint{10.898519in}{1.809621in}}{\pgfqpoint{10.889686in}{1.813280in}}{\pgfqpoint{10.880478in}{1.813280in}}%
\pgfpathcurveto{\pgfqpoint{10.871269in}{1.813280in}}{\pgfqpoint{10.862437in}{1.809621in}}{\pgfqpoint{10.855925in}{1.803110in}}%
\pgfpathcurveto{\pgfqpoint{10.849414in}{1.796599in}}{\pgfqpoint{10.845755in}{1.787766in}}{\pgfqpoint{10.845755in}{1.778558in}}%
\pgfpathcurveto{\pgfqpoint{10.845755in}{1.769349in}}{\pgfqpoint{10.849414in}{1.760517in}}{\pgfqpoint{10.855925in}{1.754005in}}%
\pgfpathcurveto{\pgfqpoint{10.862437in}{1.747494in}}{\pgfqpoint{10.871269in}{1.743835in}}{\pgfqpoint{10.880478in}{1.743835in}}%
\pgfpathlineto{\pgfqpoint{10.880478in}{1.743835in}}%
\pgfpathclose%
\pgfusepath{stroke}%
\end{pgfscope}%
\begin{pgfscope}%
\pgfpathrectangle{\pgfqpoint{0.417359in}{0.814008in}}{\pgfqpoint{12.309552in}{4.722875in}}%
\pgfusepath{clip}%
\pgfsetbuttcap%
\pgfsetroundjoin%
\pgfsetlinewidth{1.003750pt}%
\definecolor{currentstroke}{rgb}{0.450000,0.450000,0.450000}%
\pgfsetstrokecolor{currentstroke}%
\pgfsetdash{}{0pt}%
\pgfpathmoveto{\pgfqpoint{10.880478in}{1.794808in}}%
\pgfpathcurveto{\pgfqpoint{10.889686in}{1.794808in}}{\pgfqpoint{10.898519in}{1.798466in}}{\pgfqpoint{10.905030in}{1.804978in}}%
\pgfpathcurveto{\pgfqpoint{10.911541in}{1.811489in}}{\pgfqpoint{10.915200in}{1.820321in}}{\pgfqpoint{10.915200in}{1.829530in}}%
\pgfpathcurveto{\pgfqpoint{10.915200in}{1.838738in}}{\pgfqpoint{10.911541in}{1.847571in}}{\pgfqpoint{10.905030in}{1.854082in}}%
\pgfpathcurveto{\pgfqpoint{10.898519in}{1.860594in}}{\pgfqpoint{10.889686in}{1.864252in}}{\pgfqpoint{10.880478in}{1.864252in}}%
\pgfpathcurveto{\pgfqpoint{10.871269in}{1.864252in}}{\pgfqpoint{10.862437in}{1.860594in}}{\pgfqpoint{10.855925in}{1.854082in}}%
\pgfpathcurveto{\pgfqpoint{10.849414in}{1.847571in}}{\pgfqpoint{10.845755in}{1.838738in}}{\pgfqpoint{10.845755in}{1.829530in}}%
\pgfpathcurveto{\pgfqpoint{10.845755in}{1.820321in}}{\pgfqpoint{10.849414in}{1.811489in}}{\pgfqpoint{10.855925in}{1.804978in}}%
\pgfpathcurveto{\pgfqpoint{10.862437in}{1.798466in}}{\pgfqpoint{10.871269in}{1.794808in}}{\pgfqpoint{10.880478in}{1.794808in}}%
\pgfpathlineto{\pgfqpoint{10.880478in}{1.794808in}}%
\pgfpathclose%
\pgfusepath{stroke}%
\end{pgfscope}%
\begin{pgfscope}%
\pgfpathrectangle{\pgfqpoint{0.417359in}{0.814008in}}{\pgfqpoint{12.309552in}{4.722875in}}%
\pgfusepath{clip}%
\pgfsetbuttcap%
\pgfsetroundjoin%
\pgfsetlinewidth{1.003750pt}%
\definecolor{currentstroke}{rgb}{0.450000,0.450000,0.450000}%
\pgfsetstrokecolor{currentstroke}%
\pgfsetdash{}{0pt}%
\pgfpathmoveto{\pgfqpoint{10.880478in}{1.747166in}}%
\pgfpathcurveto{\pgfqpoint{10.889686in}{1.747166in}}{\pgfqpoint{10.898519in}{1.750824in}}{\pgfqpoint{10.905030in}{1.757335in}}%
\pgfpathcurveto{\pgfqpoint{10.911541in}{1.763847in}}{\pgfqpoint{10.915200in}{1.772679in}}{\pgfqpoint{10.915200in}{1.781888in}}%
\pgfpathcurveto{\pgfqpoint{10.915200in}{1.791096in}}{\pgfqpoint{10.911541in}{1.799929in}}{\pgfqpoint{10.905030in}{1.806440in}}%
\pgfpathcurveto{\pgfqpoint{10.898519in}{1.812951in}}{\pgfqpoint{10.889686in}{1.816610in}}{\pgfqpoint{10.880478in}{1.816610in}}%
\pgfpathcurveto{\pgfqpoint{10.871269in}{1.816610in}}{\pgfqpoint{10.862437in}{1.812951in}}{\pgfqpoint{10.855925in}{1.806440in}}%
\pgfpathcurveto{\pgfqpoint{10.849414in}{1.799929in}}{\pgfqpoint{10.845755in}{1.791096in}}{\pgfqpoint{10.845755in}{1.781888in}}%
\pgfpathcurveto{\pgfqpoint{10.845755in}{1.772679in}}{\pgfqpoint{10.849414in}{1.763847in}}{\pgfqpoint{10.855925in}{1.757335in}}%
\pgfpathcurveto{\pgfqpoint{10.862437in}{1.750824in}}{\pgfqpoint{10.871269in}{1.747166in}}{\pgfqpoint{10.880478in}{1.747166in}}%
\pgfpathlineto{\pgfqpoint{10.880478in}{1.747166in}}%
\pgfpathclose%
\pgfusepath{stroke}%
\end{pgfscope}%
\begin{pgfscope}%
\pgfpathrectangle{\pgfqpoint{0.417359in}{0.814008in}}{\pgfqpoint{12.309552in}{4.722875in}}%
\pgfusepath{clip}%
\pgfsetbuttcap%
\pgfsetroundjoin%
\pgfsetlinewidth{1.003750pt}%
\definecolor{currentstroke}{rgb}{0.450000,0.450000,0.450000}%
\pgfsetstrokecolor{currentstroke}%
\pgfsetdash{}{0pt}%
\pgfpathmoveto{\pgfqpoint{10.880478in}{3.073368in}}%
\pgfpathcurveto{\pgfqpoint{10.889686in}{3.073368in}}{\pgfqpoint{10.898519in}{3.077026in}}{\pgfqpoint{10.905030in}{3.083538in}}%
\pgfpathcurveto{\pgfqpoint{10.911541in}{3.090049in}}{\pgfqpoint{10.915200in}{3.098881in}}{\pgfqpoint{10.915200in}{3.108090in}}%
\pgfpathcurveto{\pgfqpoint{10.915200in}{3.117298in}}{\pgfqpoint{10.911541in}{3.126131in}}{\pgfqpoint{10.905030in}{3.132642in}}%
\pgfpathcurveto{\pgfqpoint{10.898519in}{3.139154in}}{\pgfqpoint{10.889686in}{3.142812in}}{\pgfqpoint{10.880478in}{3.142812in}}%
\pgfpathcurveto{\pgfqpoint{10.871269in}{3.142812in}}{\pgfqpoint{10.862437in}{3.139154in}}{\pgfqpoint{10.855925in}{3.132642in}}%
\pgfpathcurveto{\pgfqpoint{10.849414in}{3.126131in}}{\pgfqpoint{10.845755in}{3.117298in}}{\pgfqpoint{10.845755in}{3.108090in}}%
\pgfpathcurveto{\pgfqpoint{10.845755in}{3.098881in}}{\pgfqpoint{10.849414in}{3.090049in}}{\pgfqpoint{10.855925in}{3.083538in}}%
\pgfpathcurveto{\pgfqpoint{10.862437in}{3.077026in}}{\pgfqpoint{10.871269in}{3.073368in}}{\pgfqpoint{10.880478in}{3.073368in}}%
\pgfpathlineto{\pgfqpoint{10.880478in}{3.073368in}}%
\pgfpathclose%
\pgfusepath{stroke}%
\end{pgfscope}%
\begin{pgfscope}%
\pgfpathrectangle{\pgfqpoint{0.417359in}{0.814008in}}{\pgfqpoint{12.309552in}{4.722875in}}%
\pgfusepath{clip}%
\pgfsetbuttcap%
\pgfsetroundjoin%
\pgfsetlinewidth{1.003750pt}%
\definecolor{currentstroke}{rgb}{0.450000,0.450000,0.450000}%
\pgfsetstrokecolor{currentstroke}%
\pgfsetdash{}{0pt}%
\pgfpathmoveto{\pgfqpoint{10.880478in}{1.747166in}}%
\pgfpathcurveto{\pgfqpoint{10.889686in}{1.747166in}}{\pgfqpoint{10.898519in}{1.750824in}}{\pgfqpoint{10.905030in}{1.757335in}}%
\pgfpathcurveto{\pgfqpoint{10.911541in}{1.763847in}}{\pgfqpoint{10.915200in}{1.772679in}}{\pgfqpoint{10.915200in}{1.781888in}}%
\pgfpathcurveto{\pgfqpoint{10.915200in}{1.791096in}}{\pgfqpoint{10.911541in}{1.799929in}}{\pgfqpoint{10.905030in}{1.806440in}}%
\pgfpathcurveto{\pgfqpoint{10.898519in}{1.812951in}}{\pgfqpoint{10.889686in}{1.816610in}}{\pgfqpoint{10.880478in}{1.816610in}}%
\pgfpathcurveto{\pgfqpoint{10.871269in}{1.816610in}}{\pgfqpoint{10.862437in}{1.812951in}}{\pgfqpoint{10.855925in}{1.806440in}}%
\pgfpathcurveto{\pgfqpoint{10.849414in}{1.799929in}}{\pgfqpoint{10.845755in}{1.791096in}}{\pgfqpoint{10.845755in}{1.781888in}}%
\pgfpathcurveto{\pgfqpoint{10.845755in}{1.772679in}}{\pgfqpoint{10.849414in}{1.763847in}}{\pgfqpoint{10.855925in}{1.757335in}}%
\pgfpathcurveto{\pgfqpoint{10.862437in}{1.750824in}}{\pgfqpoint{10.871269in}{1.747166in}}{\pgfqpoint{10.880478in}{1.747166in}}%
\pgfpathlineto{\pgfqpoint{10.880478in}{1.747166in}}%
\pgfpathclose%
\pgfusepath{stroke}%
\end{pgfscope}%
\begin{pgfscope}%
\pgfpathrectangle{\pgfqpoint{0.417359in}{0.814008in}}{\pgfqpoint{12.309552in}{4.722875in}}%
\pgfusepath{clip}%
\pgfsetbuttcap%
\pgfsetroundjoin%
\pgfsetlinewidth{1.003750pt}%
\definecolor{currentstroke}{rgb}{0.450000,0.450000,0.450000}%
\pgfsetstrokecolor{currentstroke}%
\pgfsetdash{}{0pt}%
\pgfpathmoveto{\pgfqpoint{10.880478in}{0.837588in}}%
\pgfpathcurveto{\pgfqpoint{10.889686in}{0.837588in}}{\pgfqpoint{10.898519in}{0.841246in}}{\pgfqpoint{10.905030in}{0.847758in}}%
\pgfpathcurveto{\pgfqpoint{10.911541in}{0.854269in}}{\pgfqpoint{10.915200in}{0.863101in}}{\pgfqpoint{10.915200in}{0.872310in}}%
\pgfpathcurveto{\pgfqpoint{10.915200in}{0.881518in}}{\pgfqpoint{10.911541in}{0.890351in}}{\pgfqpoint{10.905030in}{0.896862in}}%
\pgfpathcurveto{\pgfqpoint{10.898519in}{0.903374in}}{\pgfqpoint{10.889686in}{0.907032in}}{\pgfqpoint{10.880478in}{0.907032in}}%
\pgfpathcurveto{\pgfqpoint{10.871269in}{0.907032in}}{\pgfqpoint{10.862437in}{0.903374in}}{\pgfqpoint{10.855925in}{0.896862in}}%
\pgfpathcurveto{\pgfqpoint{10.849414in}{0.890351in}}{\pgfqpoint{10.845755in}{0.881518in}}{\pgfqpoint{10.845755in}{0.872310in}}%
\pgfpathcurveto{\pgfqpoint{10.845755in}{0.863101in}}{\pgfqpoint{10.849414in}{0.854269in}}{\pgfqpoint{10.855925in}{0.847758in}}%
\pgfpathcurveto{\pgfqpoint{10.862437in}{0.841246in}}{\pgfqpoint{10.871269in}{0.837588in}}{\pgfqpoint{10.880478in}{0.837588in}}%
\pgfpathlineto{\pgfqpoint{10.880478in}{0.837588in}}%
\pgfpathclose%
\pgfusepath{stroke}%
\end{pgfscope}%
\begin{pgfscope}%
\pgfpathrectangle{\pgfqpoint{0.417359in}{0.814008in}}{\pgfqpoint{12.309552in}{4.722875in}}%
\pgfusepath{clip}%
\pgfsetbuttcap%
\pgfsetroundjoin%
\pgfsetlinewidth{1.003750pt}%
\definecolor{currentstroke}{rgb}{0.450000,0.450000,0.450000}%
\pgfsetstrokecolor{currentstroke}%
\pgfsetdash{}{0pt}%
\pgfpathmoveto{\pgfqpoint{10.880478in}{0.779852in}}%
\pgfpathcurveto{\pgfqpoint{10.889686in}{0.779852in}}{\pgfqpoint{10.898519in}{0.783511in}}{\pgfqpoint{10.905030in}{0.790022in}}%
\pgfpathcurveto{\pgfqpoint{10.911541in}{0.796534in}}{\pgfqpoint{10.915200in}{0.805366in}}{\pgfqpoint{10.915200in}{0.814575in}}%
\pgfpathcurveto{\pgfqpoint{10.915200in}{0.823783in}}{\pgfqpoint{10.911541in}{0.832615in}}{\pgfqpoint{10.905030in}{0.839127in}}%
\pgfpathcurveto{\pgfqpoint{10.898519in}{0.845638in}}{\pgfqpoint{10.889686in}{0.849297in}}{\pgfqpoint{10.880478in}{0.849297in}}%
\pgfpathcurveto{\pgfqpoint{10.871269in}{0.849297in}}{\pgfqpoint{10.862437in}{0.845638in}}{\pgfqpoint{10.855925in}{0.839127in}}%
\pgfpathcurveto{\pgfqpoint{10.849414in}{0.832615in}}{\pgfqpoint{10.845755in}{0.823783in}}{\pgfqpoint{10.845755in}{0.814575in}}%
\pgfpathcurveto{\pgfqpoint{10.845755in}{0.805366in}}{\pgfqpoint{10.849414in}{0.796534in}}{\pgfqpoint{10.855925in}{0.790022in}}%
\pgfpathcurveto{\pgfqpoint{10.862437in}{0.783511in}}{\pgfqpoint{10.871269in}{0.779852in}}{\pgfqpoint{10.880478in}{0.779852in}}%
\pgfusepath{stroke}%
\end{pgfscope}%
\begin{pgfscope}%
\pgfpathrectangle{\pgfqpoint{0.417359in}{0.814008in}}{\pgfqpoint{12.309552in}{4.722875in}}%
\pgfusepath{clip}%
\pgfsetbuttcap%
\pgfsetroundjoin%
\pgfsetlinewidth{1.003750pt}%
\definecolor{currentstroke}{rgb}{0.450000,0.450000,0.450000}%
\pgfsetstrokecolor{currentstroke}%
\pgfsetdash{}{0pt}%
\pgfpathmoveto{\pgfqpoint{10.880478in}{0.782461in}}%
\pgfpathcurveto{\pgfqpoint{10.889686in}{0.782461in}}{\pgfqpoint{10.898519in}{0.786119in}}{\pgfqpoint{10.905030in}{0.792631in}}%
\pgfpathcurveto{\pgfqpoint{10.911541in}{0.799142in}}{\pgfqpoint{10.915200in}{0.807975in}}{\pgfqpoint{10.915200in}{0.817183in}}%
\pgfpathcurveto{\pgfqpoint{10.915200in}{0.826392in}}{\pgfqpoint{10.911541in}{0.835224in}}{\pgfqpoint{10.905030in}{0.841735in}}%
\pgfpathcurveto{\pgfqpoint{10.898519in}{0.848247in}}{\pgfqpoint{10.889686in}{0.851905in}}{\pgfqpoint{10.880478in}{0.851905in}}%
\pgfpathcurveto{\pgfqpoint{10.871269in}{0.851905in}}{\pgfqpoint{10.862437in}{0.848247in}}{\pgfqpoint{10.855925in}{0.841735in}}%
\pgfpathcurveto{\pgfqpoint{10.849414in}{0.835224in}}{\pgfqpoint{10.845755in}{0.826392in}}{\pgfqpoint{10.845755in}{0.817183in}}%
\pgfpathcurveto{\pgfqpoint{10.845755in}{0.807975in}}{\pgfqpoint{10.849414in}{0.799142in}}{\pgfqpoint{10.855925in}{0.792631in}}%
\pgfpathcurveto{\pgfqpoint{10.862437in}{0.786119in}}{\pgfqpoint{10.871269in}{0.782461in}}{\pgfqpoint{10.880478in}{0.782461in}}%
\pgfusepath{stroke}%
\end{pgfscope}%
\begin{pgfscope}%
\pgfpathrectangle{\pgfqpoint{0.417359in}{0.814008in}}{\pgfqpoint{12.309552in}{4.722875in}}%
\pgfusepath{clip}%
\pgfsetbuttcap%
\pgfsetroundjoin%
\definecolor{currentfill}{rgb}{0.180392,0.671569,0.721569}%
\pgfsetfillcolor{currentfill}%
\pgfsetlinewidth{0.752812pt}%
\definecolor{currentstroke}{rgb}{0.240000,0.240000,0.240000}%
\pgfsetstrokecolor{currentstroke}%
\pgfsetdash{}{0pt}%
\pgfsys@defobject{currentmarker}{\pgfqpoint{11.619051in}{0.815503in}}{\pgfqpoint{12.603815in}{0.841701in}}{%
\pgfpathmoveto{\pgfqpoint{11.619051in}{0.815503in}}%
\pgfpathlineto{\pgfqpoint{12.603815in}{0.815503in}}%
\pgfpathlineto{\pgfqpoint{12.603815in}{0.841701in}}%
\pgfpathlineto{\pgfqpoint{11.619051in}{0.841701in}}%
\pgfpathlineto{\pgfqpoint{11.619051in}{0.815503in}}%
\pgfpathclose%
\pgfusepath{stroke,fill}%
}%
\begin{pgfscope}%
\pgfsys@transformshift{0.000000in}{0.000000in}%
\pgfsys@useobject{currentmarker}{}%
\end{pgfscope}%
\end{pgfscope}%
\begin{pgfscope}%
\pgfpathrectangle{\pgfqpoint{0.417359in}{0.814008in}}{\pgfqpoint{12.309552in}{4.722875in}}%
\pgfusepath{clip}%
\pgfsetbuttcap%
\pgfsetroundjoin%
\pgfsetlinewidth{1.003750pt}%
\definecolor{currentstroke}{rgb}{0.450000,0.450000,0.450000}%
\pgfsetstrokecolor{currentstroke}%
\pgfsetdash{}{0pt}%
\pgfpathmoveto{\pgfqpoint{12.111433in}{0.983599in}}%
\pgfpathcurveto{\pgfqpoint{12.120641in}{0.983599in}}{\pgfqpoint{12.129474in}{0.987257in}}{\pgfqpoint{12.135985in}{0.993769in}}%
\pgfpathcurveto{\pgfqpoint{12.142496in}{1.000280in}}{\pgfqpoint{12.146155in}{1.009113in}}{\pgfqpoint{12.146155in}{1.018321in}}%
\pgfpathcurveto{\pgfqpoint{12.146155in}{1.027530in}}{\pgfqpoint{12.142496in}{1.036362in}}{\pgfqpoint{12.135985in}{1.042873in}}%
\pgfpathcurveto{\pgfqpoint{12.129474in}{1.049385in}}{\pgfqpoint{12.120641in}{1.053043in}}{\pgfqpoint{12.111433in}{1.053043in}}%
\pgfpathcurveto{\pgfqpoint{12.102224in}{1.053043in}}{\pgfqpoint{12.093392in}{1.049385in}}{\pgfqpoint{12.086881in}{1.042873in}}%
\pgfpathcurveto{\pgfqpoint{12.080369in}{1.036362in}}{\pgfqpoint{12.076711in}{1.027530in}}{\pgfqpoint{12.076711in}{1.018321in}}%
\pgfpathcurveto{\pgfqpoint{12.076711in}{1.009113in}}{\pgfqpoint{12.080369in}{1.000280in}}{\pgfqpoint{12.086881in}{0.993769in}}%
\pgfpathcurveto{\pgfqpoint{12.093392in}{0.987257in}}{\pgfqpoint{12.102224in}{0.983599in}}{\pgfqpoint{12.111433in}{0.983599in}}%
\pgfpathlineto{\pgfqpoint{12.111433in}{0.983599in}}%
\pgfpathclose%
\pgfusepath{stroke}%
\end{pgfscope}%
\begin{pgfscope}%
\pgfpathrectangle{\pgfqpoint{0.417359in}{0.814008in}}{\pgfqpoint{12.309552in}{4.722875in}}%
\pgfusepath{clip}%
\pgfsetbuttcap%
\pgfsetroundjoin%
\pgfsetlinewidth{1.003750pt}%
\definecolor{currentstroke}{rgb}{0.450000,0.450000,0.450000}%
\pgfsetstrokecolor{currentstroke}%
\pgfsetdash{}{0pt}%
\pgfpathmoveto{\pgfqpoint{12.111433in}{0.983599in}}%
\pgfpathcurveto{\pgfqpoint{12.120641in}{0.983599in}}{\pgfqpoint{12.129474in}{0.987257in}}{\pgfqpoint{12.135985in}{0.993769in}}%
\pgfpathcurveto{\pgfqpoint{12.142496in}{1.000280in}}{\pgfqpoint{12.146155in}{1.009113in}}{\pgfqpoint{12.146155in}{1.018321in}}%
\pgfpathcurveto{\pgfqpoint{12.146155in}{1.027530in}}{\pgfqpoint{12.142496in}{1.036362in}}{\pgfqpoint{12.135985in}{1.042873in}}%
\pgfpathcurveto{\pgfqpoint{12.129474in}{1.049385in}}{\pgfqpoint{12.120641in}{1.053043in}}{\pgfqpoint{12.111433in}{1.053043in}}%
\pgfpathcurveto{\pgfqpoint{12.102224in}{1.053043in}}{\pgfqpoint{12.093392in}{1.049385in}}{\pgfqpoint{12.086881in}{1.042873in}}%
\pgfpathcurveto{\pgfqpoint{12.080369in}{1.036362in}}{\pgfqpoint{12.076711in}{1.027530in}}{\pgfqpoint{12.076711in}{1.018321in}}%
\pgfpathcurveto{\pgfqpoint{12.076711in}{1.009113in}}{\pgfqpoint{12.080369in}{1.000280in}}{\pgfqpoint{12.086881in}{0.993769in}}%
\pgfpathcurveto{\pgfqpoint{12.093392in}{0.987257in}}{\pgfqpoint{12.102224in}{0.983599in}}{\pgfqpoint{12.111433in}{0.983599in}}%
\pgfpathlineto{\pgfqpoint{12.111433in}{0.983599in}}%
\pgfpathclose%
\pgfusepath{stroke}%
\end{pgfscope}%
\begin{pgfscope}%
\pgfpathrectangle{\pgfqpoint{0.417359in}{0.814008in}}{\pgfqpoint{12.309552in}{4.722875in}}%
\pgfusepath{clip}%
\pgfsetbuttcap%
\pgfsetroundjoin%
\pgfsetlinewidth{1.003750pt}%
\definecolor{currentstroke}{rgb}{0.450000,0.450000,0.450000}%
\pgfsetstrokecolor{currentstroke}%
\pgfsetdash{}{0pt}%
\pgfpathmoveto{\pgfqpoint{12.111433in}{0.958299in}}%
\pgfpathcurveto{\pgfqpoint{12.120641in}{0.958299in}}{\pgfqpoint{12.129474in}{0.961958in}}{\pgfqpoint{12.135985in}{0.968469in}}%
\pgfpathcurveto{\pgfqpoint{12.142496in}{0.974981in}}{\pgfqpoint{12.146155in}{0.983813in}}{\pgfqpoint{12.146155in}{0.993022in}}%
\pgfpathcurveto{\pgfqpoint{12.146155in}{1.002230in}}{\pgfqpoint{12.142496in}{1.011063in}}{\pgfqpoint{12.135985in}{1.017574in}}%
\pgfpathcurveto{\pgfqpoint{12.129474in}{1.024085in}}{\pgfqpoint{12.120641in}{1.027744in}}{\pgfqpoint{12.111433in}{1.027744in}}%
\pgfpathcurveto{\pgfqpoint{12.102224in}{1.027744in}}{\pgfqpoint{12.093392in}{1.024085in}}{\pgfqpoint{12.086881in}{1.017574in}}%
\pgfpathcurveto{\pgfqpoint{12.080369in}{1.011063in}}{\pgfqpoint{12.076711in}{1.002230in}}{\pgfqpoint{12.076711in}{0.993022in}}%
\pgfpathcurveto{\pgfqpoint{12.076711in}{0.983813in}}{\pgfqpoint{12.080369in}{0.974981in}}{\pgfqpoint{12.086881in}{0.968469in}}%
\pgfpathcurveto{\pgfqpoint{12.093392in}{0.961958in}}{\pgfqpoint{12.102224in}{0.958299in}}{\pgfqpoint{12.111433in}{0.958299in}}%
\pgfpathlineto{\pgfqpoint{12.111433in}{0.958299in}}%
\pgfpathclose%
\pgfusepath{stroke}%
\end{pgfscope}%
\begin{pgfscope}%
\pgfpathrectangle{\pgfqpoint{0.417359in}{0.814008in}}{\pgfqpoint{12.309552in}{4.722875in}}%
\pgfusepath{clip}%
\pgfsetbuttcap%
\pgfsetroundjoin%
\pgfsetlinewidth{1.003750pt}%
\definecolor{currentstroke}{rgb}{0.450000,0.450000,0.450000}%
\pgfsetstrokecolor{currentstroke}%
\pgfsetdash{}{0pt}%
\pgfpathmoveto{\pgfqpoint{12.111433in}{0.851006in}}%
\pgfpathcurveto{\pgfqpoint{12.120641in}{0.851006in}}{\pgfqpoint{12.129474in}{0.854664in}}{\pgfqpoint{12.135985in}{0.861176in}}%
\pgfpathcurveto{\pgfqpoint{12.142496in}{0.867687in}}{\pgfqpoint{12.146155in}{0.876520in}}{\pgfqpoint{12.146155in}{0.885728in}}%
\pgfpathcurveto{\pgfqpoint{12.146155in}{0.894936in}}{\pgfqpoint{12.142496in}{0.903769in}}{\pgfqpoint{12.135985in}{0.910280in}}%
\pgfpathcurveto{\pgfqpoint{12.129474in}{0.916792in}}{\pgfqpoint{12.120641in}{0.920450in}}{\pgfqpoint{12.111433in}{0.920450in}}%
\pgfpathcurveto{\pgfqpoint{12.102224in}{0.920450in}}{\pgfqpoint{12.093392in}{0.916792in}}{\pgfqpoint{12.086881in}{0.910280in}}%
\pgfpathcurveto{\pgfqpoint{12.080369in}{0.903769in}}{\pgfqpoint{12.076711in}{0.894936in}}{\pgfqpoint{12.076711in}{0.885728in}}%
\pgfpathcurveto{\pgfqpoint{12.076711in}{0.876520in}}{\pgfqpoint{12.080369in}{0.867687in}}{\pgfqpoint{12.086881in}{0.861176in}}%
\pgfpathcurveto{\pgfqpoint{12.093392in}{0.854664in}}{\pgfqpoint{12.102224in}{0.851006in}}{\pgfqpoint{12.111433in}{0.851006in}}%
\pgfpathlineto{\pgfqpoint{12.111433in}{0.851006in}}%
\pgfpathclose%
\pgfusepath{stroke}%
\end{pgfscope}%
\begin{pgfscope}%
\pgfpathrectangle{\pgfqpoint{0.417359in}{0.814008in}}{\pgfqpoint{12.309552in}{4.722875in}}%
\pgfusepath{clip}%
\pgfsetbuttcap%
\pgfsetroundjoin%
\pgfsetlinewidth{1.003750pt}%
\definecolor{currentstroke}{rgb}{0.450000,0.450000,0.450000}%
\pgfsetstrokecolor{currentstroke}%
\pgfsetdash{}{0pt}%
\pgfpathmoveto{\pgfqpoint{12.111433in}{0.780680in}}%
\pgfpathcurveto{\pgfqpoint{12.120641in}{0.780680in}}{\pgfqpoint{12.129474in}{0.784338in}}{\pgfqpoint{12.135985in}{0.790850in}}%
\pgfpathcurveto{\pgfqpoint{12.142496in}{0.797361in}}{\pgfqpoint{12.146155in}{0.806194in}}{\pgfqpoint{12.146155in}{0.815402in}}%
\pgfpathcurveto{\pgfqpoint{12.146155in}{0.824611in}}{\pgfqpoint{12.142496in}{0.833443in}}{\pgfqpoint{12.135985in}{0.839954in}}%
\pgfpathcurveto{\pgfqpoint{12.129474in}{0.846466in}}{\pgfqpoint{12.120641in}{0.850124in}}{\pgfqpoint{12.111433in}{0.850124in}}%
\pgfpathcurveto{\pgfqpoint{12.102224in}{0.850124in}}{\pgfqpoint{12.093392in}{0.846466in}}{\pgfqpoint{12.086881in}{0.839954in}}%
\pgfpathcurveto{\pgfqpoint{12.080369in}{0.833443in}}{\pgfqpoint{12.076711in}{0.824611in}}{\pgfqpoint{12.076711in}{0.815402in}}%
\pgfpathcurveto{\pgfqpoint{12.076711in}{0.806194in}}{\pgfqpoint{12.080369in}{0.797361in}}{\pgfqpoint{12.086881in}{0.790850in}}%
\pgfpathcurveto{\pgfqpoint{12.093392in}{0.784338in}}{\pgfqpoint{12.102224in}{0.780680in}}{\pgfqpoint{12.111433in}{0.780680in}}%
\pgfusepath{stroke}%
\end{pgfscope}%
\begin{pgfscope}%
\pgfpathrectangle{\pgfqpoint{0.417359in}{0.814008in}}{\pgfqpoint{12.309552in}{4.722875in}}%
\pgfusepath{clip}%
\pgfsetbuttcap%
\pgfsetroundjoin%
\pgfsetlinewidth{1.003750pt}%
\definecolor{currentstroke}{rgb}{0.450000,0.450000,0.450000}%
\pgfsetstrokecolor{currentstroke}%
\pgfsetdash{}{0pt}%
\pgfpathmoveto{\pgfqpoint{12.111433in}{0.828901in}}%
\pgfpathcurveto{\pgfqpoint{12.120641in}{0.828901in}}{\pgfqpoint{12.129474in}{0.832560in}}{\pgfqpoint{12.135985in}{0.839071in}}%
\pgfpathcurveto{\pgfqpoint{12.142496in}{0.845583in}}{\pgfqpoint{12.146155in}{0.854415in}}{\pgfqpoint{12.146155in}{0.863623in}}%
\pgfpathcurveto{\pgfqpoint{12.146155in}{0.872832in}}{\pgfqpoint{12.142496in}{0.881664in}}{\pgfqpoint{12.135985in}{0.888176in}}%
\pgfpathcurveto{\pgfqpoint{12.129474in}{0.894687in}}{\pgfqpoint{12.120641in}{0.898346in}}{\pgfqpoint{12.111433in}{0.898346in}}%
\pgfpathcurveto{\pgfqpoint{12.102224in}{0.898346in}}{\pgfqpoint{12.093392in}{0.894687in}}{\pgfqpoint{12.086881in}{0.888176in}}%
\pgfpathcurveto{\pgfqpoint{12.080369in}{0.881664in}}{\pgfqpoint{12.076711in}{0.872832in}}{\pgfqpoint{12.076711in}{0.863623in}}%
\pgfpathcurveto{\pgfqpoint{12.076711in}{0.854415in}}{\pgfqpoint{12.080369in}{0.845583in}}{\pgfqpoint{12.086881in}{0.839071in}}%
\pgfpathcurveto{\pgfqpoint{12.093392in}{0.832560in}}{\pgfqpoint{12.102224in}{0.828901in}}{\pgfqpoint{12.111433in}{0.828901in}}%
\pgfpathlineto{\pgfqpoint{12.111433in}{0.828901in}}%
\pgfpathclose%
\pgfusepath{stroke}%
\end{pgfscope}%
\begin{pgfscope}%
\pgfpathrectangle{\pgfqpoint{0.417359in}{0.814008in}}{\pgfqpoint{12.309552in}{4.722875in}}%
\pgfusepath{clip}%
\pgfsetbuttcap%
\pgfsetroundjoin%
\pgfsetlinewidth{1.003750pt}%
\definecolor{currentstroke}{rgb}{0.450000,0.450000,0.450000}%
\pgfsetstrokecolor{currentstroke}%
\pgfsetdash{}{0pt}%
\pgfpathmoveto{\pgfqpoint{12.111433in}{0.779796in}}%
\pgfpathcurveto{\pgfqpoint{12.120641in}{0.779796in}}{\pgfqpoint{12.129474in}{0.783455in}}{\pgfqpoint{12.135985in}{0.789966in}}%
\pgfpathcurveto{\pgfqpoint{12.142496in}{0.796477in}}{\pgfqpoint{12.146155in}{0.805310in}}{\pgfqpoint{12.146155in}{0.814518in}}%
\pgfpathcurveto{\pgfqpoint{12.146155in}{0.823727in}}{\pgfqpoint{12.142496in}{0.832559in}}{\pgfqpoint{12.135985in}{0.839071in}}%
\pgfpathcurveto{\pgfqpoint{12.129474in}{0.845582in}}{\pgfqpoint{12.120641in}{0.849241in}}{\pgfqpoint{12.111433in}{0.849241in}}%
\pgfpathcurveto{\pgfqpoint{12.102224in}{0.849241in}}{\pgfqpoint{12.093392in}{0.845582in}}{\pgfqpoint{12.086881in}{0.839071in}}%
\pgfpathcurveto{\pgfqpoint{12.080369in}{0.832559in}}{\pgfqpoint{12.076711in}{0.823727in}}{\pgfqpoint{12.076711in}{0.814518in}}%
\pgfpathcurveto{\pgfqpoint{12.076711in}{0.805310in}}{\pgfqpoint{12.080369in}{0.796477in}}{\pgfqpoint{12.086881in}{0.789966in}}%
\pgfpathcurveto{\pgfqpoint{12.093392in}{0.783455in}}{\pgfqpoint{12.102224in}{0.779796in}}{\pgfqpoint{12.111433in}{0.779796in}}%
\pgfusepath{stroke}%
\end{pgfscope}%
\begin{pgfscope}%
\pgfpathrectangle{\pgfqpoint{0.417359in}{0.814008in}}{\pgfqpoint{12.309552in}{4.722875in}}%
\pgfusepath{clip}%
\pgfsetbuttcap%
\pgfsetroundjoin%
\pgfsetlinewidth{1.003750pt}%
\definecolor{currentstroke}{rgb}{0.450000,0.450000,0.450000}%
\pgfsetstrokecolor{currentstroke}%
\pgfsetdash{}{0pt}%
\pgfpathmoveto{\pgfqpoint{12.111433in}{0.779884in}}%
\pgfpathcurveto{\pgfqpoint{12.120641in}{0.779884in}}{\pgfqpoint{12.129474in}{0.783542in}}{\pgfqpoint{12.135985in}{0.790054in}}%
\pgfpathcurveto{\pgfqpoint{12.142496in}{0.796565in}}{\pgfqpoint{12.146155in}{0.805398in}}{\pgfqpoint{12.146155in}{0.814606in}}%
\pgfpathcurveto{\pgfqpoint{12.146155in}{0.823814in}}{\pgfqpoint{12.142496in}{0.832647in}}{\pgfqpoint{12.135985in}{0.839158in}}%
\pgfpathcurveto{\pgfqpoint{12.129474in}{0.845670in}}{\pgfqpoint{12.120641in}{0.849328in}}{\pgfqpoint{12.111433in}{0.849328in}}%
\pgfpathcurveto{\pgfqpoint{12.102224in}{0.849328in}}{\pgfqpoint{12.093392in}{0.845670in}}{\pgfqpoint{12.086881in}{0.839158in}}%
\pgfpathcurveto{\pgfqpoint{12.080369in}{0.832647in}}{\pgfqpoint{12.076711in}{0.823814in}}{\pgfqpoint{12.076711in}{0.814606in}}%
\pgfpathcurveto{\pgfqpoint{12.076711in}{0.805398in}}{\pgfqpoint{12.080369in}{0.796565in}}{\pgfqpoint{12.086881in}{0.790054in}}%
\pgfpathcurveto{\pgfqpoint{12.093392in}{0.783542in}}{\pgfqpoint{12.102224in}{0.779884in}}{\pgfqpoint{12.111433in}{0.779884in}}%
\pgfusepath{stroke}%
\end{pgfscope}%
\begin{pgfscope}%
\pgfpathrectangle{\pgfqpoint{0.417359in}{0.814008in}}{\pgfqpoint{12.309552in}{4.722875in}}%
\pgfusepath{clip}%
\pgfsetbuttcap%
\pgfsetroundjoin%
\pgfsetlinewidth{1.003750pt}%
\definecolor{currentstroke}{rgb}{0.450000,0.450000,0.450000}%
\pgfsetstrokecolor{currentstroke}%
\pgfsetdash{}{0pt}%
\pgfpathmoveto{\pgfqpoint{12.111433in}{0.780505in}}%
\pgfpathcurveto{\pgfqpoint{12.120641in}{0.780505in}}{\pgfqpoint{12.129474in}{0.784163in}}{\pgfqpoint{12.135985in}{0.790675in}}%
\pgfpathcurveto{\pgfqpoint{12.142496in}{0.797186in}}{\pgfqpoint{12.146155in}{0.806019in}}{\pgfqpoint{12.146155in}{0.815227in}}%
\pgfpathcurveto{\pgfqpoint{12.146155in}{0.824435in}}{\pgfqpoint{12.142496in}{0.833268in}}{\pgfqpoint{12.135985in}{0.839779in}}%
\pgfpathcurveto{\pgfqpoint{12.129474in}{0.846291in}}{\pgfqpoint{12.120641in}{0.849949in}}{\pgfqpoint{12.111433in}{0.849949in}}%
\pgfpathcurveto{\pgfqpoint{12.102224in}{0.849949in}}{\pgfqpoint{12.093392in}{0.846291in}}{\pgfqpoint{12.086881in}{0.839779in}}%
\pgfpathcurveto{\pgfqpoint{12.080369in}{0.833268in}}{\pgfqpoint{12.076711in}{0.824435in}}{\pgfqpoint{12.076711in}{0.815227in}}%
\pgfpathcurveto{\pgfqpoint{12.076711in}{0.806019in}}{\pgfqpoint{12.080369in}{0.797186in}}{\pgfqpoint{12.086881in}{0.790675in}}%
\pgfpathcurveto{\pgfqpoint{12.093392in}{0.784163in}}{\pgfqpoint{12.102224in}{0.780505in}}{\pgfqpoint{12.111433in}{0.780505in}}%
\pgfusepath{stroke}%
\end{pgfscope}%
\begin{pgfscope}%
\pgfpathrectangle{\pgfqpoint{0.417359in}{0.814008in}}{\pgfqpoint{12.309552in}{4.722875in}}%
\pgfusepath{clip}%
\pgfsetbuttcap%
\pgfsetroundjoin%
\pgfsetlinewidth{1.003750pt}%
\definecolor{currentstroke}{rgb}{0.450000,0.450000,0.450000}%
\pgfsetstrokecolor{currentstroke}%
\pgfsetdash{}{0pt}%
\pgfpathmoveto{\pgfqpoint{12.111433in}{0.779884in}}%
\pgfpathcurveto{\pgfqpoint{12.120641in}{0.779884in}}{\pgfqpoint{12.129474in}{0.783542in}}{\pgfqpoint{12.135985in}{0.790054in}}%
\pgfpathcurveto{\pgfqpoint{12.142496in}{0.796565in}}{\pgfqpoint{12.146155in}{0.805398in}}{\pgfqpoint{12.146155in}{0.814606in}}%
\pgfpathcurveto{\pgfqpoint{12.146155in}{0.823814in}}{\pgfqpoint{12.142496in}{0.832647in}}{\pgfqpoint{12.135985in}{0.839158in}}%
\pgfpathcurveto{\pgfqpoint{12.129474in}{0.845670in}}{\pgfqpoint{12.120641in}{0.849328in}}{\pgfqpoint{12.111433in}{0.849328in}}%
\pgfpathcurveto{\pgfqpoint{12.102224in}{0.849328in}}{\pgfqpoint{12.093392in}{0.845670in}}{\pgfqpoint{12.086881in}{0.839158in}}%
\pgfpathcurveto{\pgfqpoint{12.080369in}{0.832647in}}{\pgfqpoint{12.076711in}{0.823814in}}{\pgfqpoint{12.076711in}{0.814606in}}%
\pgfpathcurveto{\pgfqpoint{12.076711in}{0.805398in}}{\pgfqpoint{12.080369in}{0.796565in}}{\pgfqpoint{12.086881in}{0.790054in}}%
\pgfpathcurveto{\pgfqpoint{12.093392in}{0.783542in}}{\pgfqpoint{12.102224in}{0.779884in}}{\pgfqpoint{12.111433in}{0.779884in}}%
\pgfusepath{stroke}%
\end{pgfscope}%
\begin{pgfscope}%
\pgfpathrectangle{\pgfqpoint{0.417359in}{0.814008in}}{\pgfqpoint{12.309552in}{4.722875in}}%
\pgfusepath{clip}%
\pgfsetbuttcap%
\pgfsetroundjoin%
\pgfsetlinewidth{1.003750pt}%
\definecolor{currentstroke}{rgb}{0.450000,0.450000,0.450000}%
\pgfsetstrokecolor{currentstroke}%
\pgfsetdash{}{0pt}%
\pgfpathmoveto{\pgfqpoint{12.111433in}{0.779512in}}%
\pgfpathcurveto{\pgfqpoint{12.120641in}{0.779512in}}{\pgfqpoint{12.129474in}{0.783171in}}{\pgfqpoint{12.135985in}{0.789682in}}%
\pgfpathcurveto{\pgfqpoint{12.142496in}{0.796194in}}{\pgfqpoint{12.146155in}{0.805026in}}{\pgfqpoint{12.146155in}{0.814235in}}%
\pgfpathcurveto{\pgfqpoint{12.146155in}{0.823443in}}{\pgfqpoint{12.142496in}{0.832275in}}{\pgfqpoint{12.135985in}{0.838787in}}%
\pgfpathcurveto{\pgfqpoint{12.129474in}{0.845298in}}{\pgfqpoint{12.120641in}{0.848957in}}{\pgfqpoint{12.111433in}{0.848957in}}%
\pgfpathcurveto{\pgfqpoint{12.102224in}{0.848957in}}{\pgfqpoint{12.093392in}{0.845298in}}{\pgfqpoint{12.086881in}{0.838787in}}%
\pgfpathcurveto{\pgfqpoint{12.080369in}{0.832275in}}{\pgfqpoint{12.076711in}{0.823443in}}{\pgfqpoint{12.076711in}{0.814235in}}%
\pgfpathcurveto{\pgfqpoint{12.076711in}{0.805026in}}{\pgfqpoint{12.080369in}{0.796194in}}{\pgfqpoint{12.086881in}{0.789682in}}%
\pgfpathcurveto{\pgfqpoint{12.093392in}{0.783171in}}{\pgfqpoint{12.102224in}{0.779512in}}{\pgfqpoint{12.111433in}{0.779512in}}%
\pgfusepath{stroke}%
\end{pgfscope}%
\begin{pgfscope}%
\pgfpathrectangle{\pgfqpoint{0.417359in}{0.814008in}}{\pgfqpoint{12.309552in}{4.722875in}}%
\pgfusepath{clip}%
\pgfsetbuttcap%
\pgfsetroundjoin%
\pgfsetlinewidth{1.003750pt}%
\definecolor{currentstroke}{rgb}{0.450000,0.450000,0.450000}%
\pgfsetstrokecolor{currentstroke}%
\pgfsetdash{}{0pt}%
\pgfpathmoveto{\pgfqpoint{12.111433in}{0.779884in}}%
\pgfpathcurveto{\pgfqpoint{12.120641in}{0.779884in}}{\pgfqpoint{12.129474in}{0.783542in}}{\pgfqpoint{12.135985in}{0.790054in}}%
\pgfpathcurveto{\pgfqpoint{12.142496in}{0.796565in}}{\pgfqpoint{12.146155in}{0.805398in}}{\pgfqpoint{12.146155in}{0.814606in}}%
\pgfpathcurveto{\pgfqpoint{12.146155in}{0.823814in}}{\pgfqpoint{12.142496in}{0.832647in}}{\pgfqpoint{12.135985in}{0.839158in}}%
\pgfpathcurveto{\pgfqpoint{12.129474in}{0.845670in}}{\pgfqpoint{12.120641in}{0.849328in}}{\pgfqpoint{12.111433in}{0.849328in}}%
\pgfpathcurveto{\pgfqpoint{12.102224in}{0.849328in}}{\pgfqpoint{12.093392in}{0.845670in}}{\pgfqpoint{12.086881in}{0.839158in}}%
\pgfpathcurveto{\pgfqpoint{12.080369in}{0.832647in}}{\pgfqpoint{12.076711in}{0.823814in}}{\pgfqpoint{12.076711in}{0.814606in}}%
\pgfpathcurveto{\pgfqpoint{12.076711in}{0.805398in}}{\pgfqpoint{12.080369in}{0.796565in}}{\pgfqpoint{12.086881in}{0.790054in}}%
\pgfpathcurveto{\pgfqpoint{12.093392in}{0.783542in}}{\pgfqpoint{12.102224in}{0.779884in}}{\pgfqpoint{12.111433in}{0.779884in}}%
\pgfusepath{stroke}%
\end{pgfscope}%
\begin{pgfscope}%
\pgfpathrectangle{\pgfqpoint{0.417359in}{0.814008in}}{\pgfqpoint{12.309552in}{4.722875in}}%
\pgfusepath{clip}%
\pgfsetbuttcap%
\pgfsetroundjoin%
\pgfsetlinewidth{1.003750pt}%
\definecolor{currentstroke}{rgb}{0.450000,0.450000,0.450000}%
\pgfsetstrokecolor{currentstroke}%
\pgfsetdash{}{0pt}%
\pgfpathmoveto{\pgfqpoint{12.111433in}{11.945499in}}%
\pgfpathcurveto{\pgfqpoint{12.120641in}{11.945499in}}{\pgfqpoint{12.129474in}{11.949157in}}{\pgfqpoint{12.135985in}{11.955669in}}%
\pgfpathcurveto{\pgfqpoint{12.142496in}{11.962180in}}{\pgfqpoint{12.146155in}{11.971013in}}{\pgfqpoint{12.146155in}{11.980221in}}%
\pgfpathcurveto{\pgfqpoint{12.146155in}{11.989430in}}{\pgfqpoint{12.142496in}{11.998262in}}{\pgfqpoint{12.135985in}{12.004773in}}%
\pgfpathcurveto{\pgfqpoint{12.129474in}{12.011285in}}{\pgfqpoint{12.120641in}{12.014943in}}{\pgfqpoint{12.111433in}{12.014943in}}%
\pgfpathcurveto{\pgfqpoint{12.102224in}{12.014943in}}{\pgfqpoint{12.093392in}{12.011285in}}{\pgfqpoint{12.086881in}{12.004773in}}%
\pgfpathcurveto{\pgfqpoint{12.080369in}{11.998262in}}{\pgfqpoint{12.076711in}{11.989430in}}{\pgfqpoint{12.076711in}{11.980221in}}%
\pgfpathcurveto{\pgfqpoint{12.076711in}{11.971013in}}{\pgfqpoint{12.080369in}{11.962180in}}{\pgfqpoint{12.086881in}{11.955669in}}%
\pgfpathcurveto{\pgfqpoint{12.093392in}{11.949157in}}{\pgfqpoint{12.102224in}{11.945499in}}{\pgfqpoint{12.111433in}{11.945499in}}%
\pgfusepath{stroke}%
\end{pgfscope}%
\begin{pgfscope}%
\pgfpathrectangle{\pgfqpoint{0.417359in}{0.814008in}}{\pgfqpoint{12.309552in}{4.722875in}}%
\pgfusepath{clip}%
\pgfsetrectcap%
\pgfsetroundjoin%
\pgfsetlinewidth{0.803000pt}%
\definecolor{currentstroke}{rgb}{0.690196,0.690196,0.690196}%
\pgfsetstrokecolor{currentstroke}%
\pgfsetstrokeopacity{0.200000}%
\pgfsetdash{}{0pt}%
\pgfpathmoveto{\pgfqpoint{1.032836in}{0.814008in}}%
\pgfpathlineto{\pgfqpoint{1.032836in}{5.536883in}}%
\pgfusepath{stroke}%
\end{pgfscope}%
\begin{pgfscope}%
\pgfsetbuttcap%
\pgfsetroundjoin%
\definecolor{currentfill}{rgb}{0.000000,0.000000,0.000000}%
\pgfsetfillcolor{currentfill}%
\pgfsetlinewidth{0.803000pt}%
\definecolor{currentstroke}{rgb}{0.000000,0.000000,0.000000}%
\pgfsetstrokecolor{currentstroke}%
\pgfsetdash{}{0pt}%
\pgfsys@defobject{currentmarker}{\pgfqpoint{0.000000in}{-0.048611in}}{\pgfqpoint{0.000000in}{0.000000in}}{%
\pgfpathmoveto{\pgfqpoint{0.000000in}{0.000000in}}%
\pgfpathlineto{\pgfqpoint{0.000000in}{-0.048611in}}%
\pgfusepath{stroke,fill}%
}%
\begin{pgfscope}%
\pgfsys@transformshift{1.032836in}{0.814008in}%
\pgfsys@useobject{currentmarker}{}%
\end{pgfscope}%
\end{pgfscope}%
\begin{pgfscope}%
\definecolor{textcolor}{rgb}{0.000000,0.000000,0.000000}%
\pgfsetstrokecolor{textcolor}%
\pgfsetfillcolor{textcolor}%
\pgftext[x=0.723699in, y=0.439323in, left, base,rotate=12.500000]{\color{textcolor}{\rmfamily\fontsize{14.000000}{16.800000}\selectfont\catcode`\^=\active\def^{\ifmmode\sp\else\^{}\fi}\catcode`\%=\active\def%{\%}Battery}}%
\end{pgfscope}%
\begin{pgfscope}%
\pgfpathrectangle{\pgfqpoint{0.417359in}{0.814008in}}{\pgfqpoint{12.309552in}{4.722875in}}%
\pgfusepath{clip}%
\pgfsetrectcap%
\pgfsetroundjoin%
\pgfsetlinewidth{0.803000pt}%
\definecolor{currentstroke}{rgb}{0.690196,0.690196,0.690196}%
\pgfsetstrokecolor{currentstroke}%
\pgfsetstrokeopacity{0.200000}%
\pgfsetdash{}{0pt}%
\pgfpathmoveto{\pgfqpoint{2.263791in}{0.814008in}}%
\pgfpathlineto{\pgfqpoint{2.263791in}{5.536883in}}%
\pgfusepath{stroke}%
\end{pgfscope}%
\begin{pgfscope}%
\pgfsetbuttcap%
\pgfsetroundjoin%
\definecolor{currentfill}{rgb}{0.000000,0.000000,0.000000}%
\pgfsetfillcolor{currentfill}%
\pgfsetlinewidth{0.803000pt}%
\definecolor{currentstroke}{rgb}{0.000000,0.000000,0.000000}%
\pgfsetstrokecolor{currentstroke}%
\pgfsetdash{}{0pt}%
\pgfsys@defobject{currentmarker}{\pgfqpoint{0.000000in}{-0.048611in}}{\pgfqpoint{0.000000in}{0.000000in}}{%
\pgfpathmoveto{\pgfqpoint{0.000000in}{0.000000in}}%
\pgfpathlineto{\pgfqpoint{0.000000in}{-0.048611in}}%
\pgfusepath{stroke,fill}%
}%
\begin{pgfscope}%
\pgfsys@transformshift{2.263791in}{0.814008in}%
\pgfsys@useobject{currentmarker}{}%
\end{pgfscope}%
\end{pgfscope}%
\begin{pgfscope}%
\definecolor{textcolor}{rgb}{0.000000,0.000000,0.000000}%
\pgfsetstrokecolor{textcolor}%
\pgfsetfillcolor{textcolor}%
\pgftext[x=1.929693in, y=0.428255in, left, base,rotate=12.500000]{\color{textcolor}{\rmfamily\fontsize{14.000000}{16.800000}\selectfont\catcode`\^=\active\def^{\ifmmode\sp\else\^{}\fi}\catcode`\%=\active\def%{\%}Biomass}}%
\end{pgfscope}%
\begin{pgfscope}%
\pgfpathrectangle{\pgfqpoint{0.417359in}{0.814008in}}{\pgfqpoint{12.309552in}{4.722875in}}%
\pgfusepath{clip}%
\pgfsetrectcap%
\pgfsetroundjoin%
\pgfsetlinewidth{0.803000pt}%
\definecolor{currentstroke}{rgb}{0.690196,0.690196,0.690196}%
\pgfsetstrokecolor{currentstroke}%
\pgfsetstrokeopacity{0.200000}%
\pgfsetdash{}{0pt}%
\pgfpathmoveto{\pgfqpoint{3.494747in}{0.814008in}}%
\pgfpathlineto{\pgfqpoint{3.494747in}{5.536883in}}%
\pgfusepath{stroke}%
\end{pgfscope}%
\begin{pgfscope}%
\pgfsetbuttcap%
\pgfsetroundjoin%
\definecolor{currentfill}{rgb}{0.000000,0.000000,0.000000}%
\pgfsetfillcolor{currentfill}%
\pgfsetlinewidth{0.803000pt}%
\definecolor{currentstroke}{rgb}{0.000000,0.000000,0.000000}%
\pgfsetstrokecolor{currentstroke}%
\pgfsetdash{}{0pt}%
\pgfsys@defobject{currentmarker}{\pgfqpoint{0.000000in}{-0.048611in}}{\pgfqpoint{0.000000in}{0.000000in}}{%
\pgfpathmoveto{\pgfqpoint{0.000000in}{0.000000in}}%
\pgfpathlineto{\pgfqpoint{0.000000in}{-0.048611in}}%
\pgfusepath{stroke,fill}%
}%
\begin{pgfscope}%
\pgfsys@transformshift{3.494747in}{0.814008in}%
\pgfsys@useobject{currentmarker}{}%
\end{pgfscope}%
\end{pgfscope}%
\begin{pgfscope}%
\definecolor{textcolor}{rgb}{0.000000,0.000000,0.000000}%
\pgfsetstrokecolor{textcolor}%
\pgfsetfillcolor{textcolor}%
\pgftext[x=3.292134in, y=0.417534in, left, base,rotate=12.500000]{\color{textcolor}{\rmfamily\fontsize{14.000000}{16.800000}\selectfont\catcode`\^=\active\def^{\ifmmode\sp\else\^{}\fi}\catcode`\%=\active\def%{\%}Coal}}%
\end{pgfscope}%
\begin{pgfscope}%
\definecolor{textcolor}{rgb}{0.000000,0.000000,0.000000}%
\pgfsetstrokecolor{textcolor}%
\pgfsetfillcolor{textcolor}%
\pgftext[x=2.980801in, y=0.137967in, left, base,rotate=12.500000]{\color{textcolor}{\rmfamily\fontsize{14.000000}{16.800000}\selectfont\catcode`\^=\active\def^{\ifmmode\sp\else\^{}\fi}\catcode`\%=\active\def%{\%}Conventional}}%
\end{pgfscope}%
\begin{pgfscope}%
\pgfpathrectangle{\pgfqpoint{0.417359in}{0.814008in}}{\pgfqpoint{12.309552in}{4.722875in}}%
\pgfusepath{clip}%
\pgfsetrectcap%
\pgfsetroundjoin%
\pgfsetlinewidth{0.803000pt}%
\definecolor{currentstroke}{rgb}{0.690196,0.690196,0.690196}%
\pgfsetstrokecolor{currentstroke}%
\pgfsetstrokeopacity{0.200000}%
\pgfsetdash{}{0pt}%
\pgfpathmoveto{\pgfqpoint{4.725702in}{0.814008in}}%
\pgfpathlineto{\pgfqpoint{4.725702in}{5.536883in}}%
\pgfusepath{stroke}%
\end{pgfscope}%
\begin{pgfscope}%
\pgfsetbuttcap%
\pgfsetroundjoin%
\definecolor{currentfill}{rgb}{0.000000,0.000000,0.000000}%
\pgfsetfillcolor{currentfill}%
\pgfsetlinewidth{0.803000pt}%
\definecolor{currentstroke}{rgb}{0.000000,0.000000,0.000000}%
\pgfsetstrokecolor{currentstroke}%
\pgfsetdash{}{0pt}%
\pgfsys@defobject{currentmarker}{\pgfqpoint{0.000000in}{-0.048611in}}{\pgfqpoint{0.000000in}{0.000000in}}{%
\pgfpathmoveto{\pgfqpoint{0.000000in}{0.000000in}}%
\pgfpathlineto{\pgfqpoint{0.000000in}{-0.048611in}}%
\pgfusepath{stroke,fill}%
}%
\begin{pgfscope}%
\pgfsys@transformshift{4.725702in}{0.814008in}%
\pgfsys@useobject{currentmarker}{}%
\end{pgfscope}%
\end{pgfscope}%
\begin{pgfscope}%
\definecolor{textcolor}{rgb}{0.000000,0.000000,0.000000}%
\pgfsetstrokecolor{textcolor}%
\pgfsetfillcolor{textcolor}%
\pgftext[x=4.523090in, y=0.448153in, left, base,rotate=12.500000]{\color{textcolor}{\rmfamily\fontsize{14.000000}{16.800000}\selectfont\catcode`\^=\active\def^{\ifmmode\sp\else\^{}\fi}\catcode`\%=\active\def%{\%}Coal}}%
\end{pgfscope}%
\begin{pgfscope}%
\definecolor{textcolor}{rgb}{0.000000,0.000000,0.000000}%
\pgfsetstrokecolor{textcolor}%
\pgfsetfillcolor{textcolor}%
\pgftext[x=4.349871in, y=0.199205in, left, base,rotate=12.500000]{\color{textcolor}{\rmfamily\fontsize{14.000000}{16.800000}\selectfont\catcode`\^=\active\def^{\ifmmode\sp\else\^{}\fi}\catcode`\%=\active\def%{\%}Advanced}}%
\end{pgfscope}%
\begin{pgfscope}%
\pgfpathrectangle{\pgfqpoint{0.417359in}{0.814008in}}{\pgfqpoint{12.309552in}{4.722875in}}%
\pgfusepath{clip}%
\pgfsetrectcap%
\pgfsetroundjoin%
\pgfsetlinewidth{0.803000pt}%
\definecolor{currentstroke}{rgb}{0.690196,0.690196,0.690196}%
\pgfsetstrokecolor{currentstroke}%
\pgfsetstrokeopacity{0.200000}%
\pgfsetdash{}{0pt}%
\pgfpathmoveto{\pgfqpoint{5.956657in}{0.814008in}}%
\pgfpathlineto{\pgfqpoint{5.956657in}{5.536883in}}%
\pgfusepath{stroke}%
\end{pgfscope}%
\begin{pgfscope}%
\pgfsetbuttcap%
\pgfsetroundjoin%
\definecolor{currentfill}{rgb}{0.000000,0.000000,0.000000}%
\pgfsetfillcolor{currentfill}%
\pgfsetlinewidth{0.803000pt}%
\definecolor{currentstroke}{rgb}{0.000000,0.000000,0.000000}%
\pgfsetstrokecolor{currentstroke}%
\pgfsetdash{}{0pt}%
\pgfsys@defobject{currentmarker}{\pgfqpoint{0.000000in}{-0.048611in}}{\pgfqpoint{0.000000in}{0.000000in}}{%
\pgfpathmoveto{\pgfqpoint{0.000000in}{0.000000in}}%
\pgfpathlineto{\pgfqpoint{0.000000in}{-0.048611in}}%
\pgfusepath{stroke,fill}%
}%
\begin{pgfscope}%
\pgfsys@transformshift{5.956657in}{0.814008in}%
\pgfsys@useobject{currentmarker}{}%
\end{pgfscope}%
\end{pgfscope}%
\begin{pgfscope}%
\definecolor{textcolor}{rgb}{0.000000,0.000000,0.000000}%
\pgfsetstrokecolor{textcolor}%
\pgfsetfillcolor{textcolor}%
\pgftext[x=5.399724in, y=0.338983in, left, base,rotate=12.500000]{\color{textcolor}{\rmfamily\fontsize{14.000000}{16.800000}\selectfont\catcode`\^=\active\def^{\ifmmode\sp\else\^{}\fi}\catcode`\%=\active\def%{\%}Natural Gas }}%
\end{pgfscope}%
\begin{pgfscope}%
\definecolor{textcolor}{rgb}{0.000000,0.000000,0.000000}%
\pgfsetstrokecolor{textcolor}%
\pgfsetfillcolor{textcolor}%
\pgftext[x=5.442711in, y=0.137967in, left, base,rotate=12.500000]{\color{textcolor}{\rmfamily\fontsize{14.000000}{16.800000}\selectfont\catcode`\^=\active\def^{\ifmmode\sp\else\^{}\fi}\catcode`\%=\active\def%{\%}Conventional}}%
\end{pgfscope}%
\begin{pgfscope}%
\pgfpathrectangle{\pgfqpoint{0.417359in}{0.814008in}}{\pgfqpoint{12.309552in}{4.722875in}}%
\pgfusepath{clip}%
\pgfsetrectcap%
\pgfsetroundjoin%
\pgfsetlinewidth{0.803000pt}%
\definecolor{currentstroke}{rgb}{0.690196,0.690196,0.690196}%
\pgfsetstrokecolor{currentstroke}%
\pgfsetstrokeopacity{0.200000}%
\pgfsetdash{}{0pt}%
\pgfpathmoveto{\pgfqpoint{7.187612in}{0.814008in}}%
\pgfpathlineto{\pgfqpoint{7.187612in}{5.536883in}}%
\pgfusepath{stroke}%
\end{pgfscope}%
\begin{pgfscope}%
\pgfsetbuttcap%
\pgfsetroundjoin%
\definecolor{currentfill}{rgb}{0.000000,0.000000,0.000000}%
\pgfsetfillcolor{currentfill}%
\pgfsetlinewidth{0.803000pt}%
\definecolor{currentstroke}{rgb}{0.000000,0.000000,0.000000}%
\pgfsetstrokecolor{currentstroke}%
\pgfsetdash{}{0pt}%
\pgfsys@defobject{currentmarker}{\pgfqpoint{0.000000in}{-0.048611in}}{\pgfqpoint{0.000000in}{0.000000in}}{%
\pgfpathmoveto{\pgfqpoint{0.000000in}{0.000000in}}%
\pgfpathlineto{\pgfqpoint{0.000000in}{-0.048611in}}%
\pgfusepath{stroke,fill}%
}%
\begin{pgfscope}%
\pgfsys@transformshift{7.187612in}{0.814008in}%
\pgfsys@useobject{currentmarker}{}%
\end{pgfscope}%
\end{pgfscope}%
\begin{pgfscope}%
\definecolor{textcolor}{rgb}{0.000000,0.000000,0.000000}%
\pgfsetstrokecolor{textcolor}%
\pgfsetfillcolor{textcolor}%
\pgftext[x=6.630679in, y=0.339316in, left, base,rotate=12.500000]{\color{textcolor}{\rmfamily\fontsize{14.000000}{16.800000}\selectfont\catcode`\^=\active\def^{\ifmmode\sp\else\^{}\fi}\catcode`\%=\active\def%{\%}Natural Gas }}%
\end{pgfscope}%
\begin{pgfscope}%
\definecolor{textcolor}{rgb}{0.000000,0.000000,0.000000}%
\pgfsetstrokecolor{textcolor}%
\pgfsetfillcolor{textcolor}%
\pgftext[x=6.811781in, y=0.168919in, left, base,rotate=12.500000]{\color{textcolor}{\rmfamily\fontsize{14.000000}{16.800000}\selectfont\catcode`\^=\active\def^{\ifmmode\sp\else\^{}\fi}\catcode`\%=\active\def%{\%}Advanced}}%
\end{pgfscope}%
\begin{pgfscope}%
\pgfpathrectangle{\pgfqpoint{0.417359in}{0.814008in}}{\pgfqpoint{12.309552in}{4.722875in}}%
\pgfusepath{clip}%
\pgfsetrectcap%
\pgfsetroundjoin%
\pgfsetlinewidth{0.803000pt}%
\definecolor{currentstroke}{rgb}{0.690196,0.690196,0.690196}%
\pgfsetstrokecolor{currentstroke}%
\pgfsetstrokeopacity{0.200000}%
\pgfsetdash{}{0pt}%
\pgfpathmoveto{\pgfqpoint{8.418567in}{0.814008in}}%
\pgfpathlineto{\pgfqpoint{8.418567in}{5.536883in}}%
\pgfusepath{stroke}%
\end{pgfscope}%
\begin{pgfscope}%
\pgfsetbuttcap%
\pgfsetroundjoin%
\definecolor{currentfill}{rgb}{0.000000,0.000000,0.000000}%
\pgfsetfillcolor{currentfill}%
\pgfsetlinewidth{0.803000pt}%
\definecolor{currentstroke}{rgb}{0.000000,0.000000,0.000000}%
\pgfsetstrokecolor{currentstroke}%
\pgfsetdash{}{0pt}%
\pgfsys@defobject{currentmarker}{\pgfqpoint{0.000000in}{-0.048611in}}{\pgfqpoint{0.000000in}{0.000000in}}{%
\pgfpathmoveto{\pgfqpoint{0.000000in}{0.000000in}}%
\pgfpathlineto{\pgfqpoint{0.000000in}{-0.048611in}}%
\pgfusepath{stroke,fill}%
}%
\begin{pgfscope}%
\pgfsys@transformshift{8.418567in}{0.814008in}%
\pgfsys@useobject{currentmarker}{}%
\end{pgfscope}%
\end{pgfscope}%
\begin{pgfscope}%
\definecolor{textcolor}{rgb}{0.000000,0.000000,0.000000}%
\pgfsetstrokecolor{textcolor}%
\pgfsetfillcolor{textcolor}%
\pgftext[x=8.108119in, y=0.438741in, left, base,rotate=12.500000]{\color{textcolor}{\rmfamily\fontsize{14.000000}{16.800000}\selectfont\catcode`\^=\active\def^{\ifmmode\sp\else\^{}\fi}\catcode`\%=\active\def%{\%}Nuclear}}%
\end{pgfscope}%
\begin{pgfscope}%
\pgfpathrectangle{\pgfqpoint{0.417359in}{0.814008in}}{\pgfqpoint{12.309552in}{4.722875in}}%
\pgfusepath{clip}%
\pgfsetrectcap%
\pgfsetroundjoin%
\pgfsetlinewidth{0.803000pt}%
\definecolor{currentstroke}{rgb}{0.690196,0.690196,0.690196}%
\pgfsetstrokecolor{currentstroke}%
\pgfsetstrokeopacity{0.200000}%
\pgfsetdash{}{0pt}%
\pgfpathmoveto{\pgfqpoint{9.649522in}{0.814008in}}%
\pgfpathlineto{\pgfqpoint{9.649522in}{5.536883in}}%
\pgfusepath{stroke}%
\end{pgfscope}%
\begin{pgfscope}%
\pgfsetbuttcap%
\pgfsetroundjoin%
\definecolor{currentfill}{rgb}{0.000000,0.000000,0.000000}%
\pgfsetfillcolor{currentfill}%
\pgfsetlinewidth{0.803000pt}%
\definecolor{currentstroke}{rgb}{0.000000,0.000000,0.000000}%
\pgfsetstrokecolor{currentstroke}%
\pgfsetdash{}{0pt}%
\pgfsys@defobject{currentmarker}{\pgfqpoint{0.000000in}{-0.048611in}}{\pgfqpoint{0.000000in}{0.000000in}}{%
\pgfpathmoveto{\pgfqpoint{0.000000in}{0.000000in}}%
\pgfpathlineto{\pgfqpoint{0.000000in}{-0.048611in}}%
\pgfusepath{stroke,fill}%
}%
\begin{pgfscope}%
\pgfsys@transformshift{9.649522in}{0.814008in}%
\pgfsys@useobject{currentmarker}{}%
\end{pgfscope}%
\end{pgfscope}%
\begin{pgfscope}%
\definecolor{textcolor}{rgb}{0.000000,0.000000,0.000000}%
\pgfsetstrokecolor{textcolor}%
\pgfsetfillcolor{textcolor}%
\pgftext[x=9.316829in, y=0.419315in, left, base,rotate=12.500000]{\color{textcolor}{\rmfamily\fontsize{14.000000}{16.800000}\selectfont\catcode`\^=\active\def^{\ifmmode\sp\else\^{}\fi}\catcode`\%=\active\def%{\%}Nuclear}}%
\end{pgfscope}%
\begin{pgfscope}%
\definecolor{textcolor}{rgb}{0.000000,0.000000,0.000000}%
\pgfsetstrokecolor{textcolor}%
\pgfsetfillcolor{textcolor}%
\pgftext[x=9.273691in, y=0.199205in, left, base,rotate=12.500000]{\color{textcolor}{\rmfamily\fontsize{14.000000}{16.800000}\selectfont\catcode`\^=\active\def^{\ifmmode\sp\else\^{}\fi}\catcode`\%=\active\def%{\%}Advanced}}%
\end{pgfscope}%
\begin{pgfscope}%
\pgfpathrectangle{\pgfqpoint{0.417359in}{0.814008in}}{\pgfqpoint{12.309552in}{4.722875in}}%
\pgfusepath{clip}%
\pgfsetrectcap%
\pgfsetroundjoin%
\pgfsetlinewidth{0.803000pt}%
\definecolor{currentstroke}{rgb}{0.690196,0.690196,0.690196}%
\pgfsetstrokecolor{currentstroke}%
\pgfsetstrokeopacity{0.200000}%
\pgfsetdash{}{0pt}%
\pgfpathmoveto{\pgfqpoint{10.880478in}{0.814008in}}%
\pgfpathlineto{\pgfqpoint{10.880478in}{5.536883in}}%
\pgfusepath{stroke}%
\end{pgfscope}%
\begin{pgfscope}%
\pgfsetbuttcap%
\pgfsetroundjoin%
\definecolor{currentfill}{rgb}{0.000000,0.000000,0.000000}%
\pgfsetfillcolor{currentfill}%
\pgfsetlinewidth{0.803000pt}%
\definecolor{currentstroke}{rgb}{0.000000,0.000000,0.000000}%
\pgfsetstrokecolor{currentstroke}%
\pgfsetdash{}{0pt}%
\pgfsys@defobject{currentmarker}{\pgfqpoint{0.000000in}{-0.048611in}}{\pgfqpoint{0.000000in}{0.000000in}}{%
\pgfpathmoveto{\pgfqpoint{0.000000in}{0.000000in}}%
\pgfpathlineto{\pgfqpoint{0.000000in}{-0.048611in}}%
\pgfusepath{stroke,fill}%
}%
\begin{pgfscope}%
\pgfsys@transformshift{10.880478in}{0.814008in}%
\pgfsys@useobject{currentmarker}{}%
\end{pgfscope}%
\end{pgfscope}%
\begin{pgfscope}%
\definecolor{textcolor}{rgb}{0.000000,0.000000,0.000000}%
\pgfsetstrokecolor{textcolor}%
\pgfsetfillcolor{textcolor}%
\pgftext[x=10.446537in, y=0.383986in, left, base,rotate=12.500000]{\color{textcolor}{\rmfamily\fontsize{14.000000}{16.800000}\selectfont\catcode`\^=\active\def^{\ifmmode\sp\else\^{}\fi}\catcode`\%=\active\def%{\%}SolarPanel}}%
\end{pgfscope}%
\begin{pgfscope}%
\pgfpathrectangle{\pgfqpoint{0.417359in}{0.814008in}}{\pgfqpoint{12.309552in}{4.722875in}}%
\pgfusepath{clip}%
\pgfsetrectcap%
\pgfsetroundjoin%
\pgfsetlinewidth{0.803000pt}%
\definecolor{currentstroke}{rgb}{0.690196,0.690196,0.690196}%
\pgfsetstrokecolor{currentstroke}%
\pgfsetstrokeopacity{0.200000}%
\pgfsetdash{}{0pt}%
\pgfpathmoveto{\pgfqpoint{12.111433in}{0.814008in}}%
\pgfpathlineto{\pgfqpoint{12.111433in}{5.536883in}}%
\pgfusepath{stroke}%
\end{pgfscope}%
\begin{pgfscope}%
\pgfsetbuttcap%
\pgfsetroundjoin%
\definecolor{currentfill}{rgb}{0.000000,0.000000,0.000000}%
\pgfsetfillcolor{currentfill}%
\pgfsetlinewidth{0.803000pt}%
\definecolor{currentstroke}{rgb}{0.000000,0.000000,0.000000}%
\pgfsetstrokecolor{currentstroke}%
\pgfsetdash{}{0pt}%
\pgfsys@defobject{currentmarker}{\pgfqpoint{0.000000in}{-0.048611in}}{\pgfqpoint{0.000000in}{0.000000in}}{%
\pgfpathmoveto{\pgfqpoint{0.000000in}{0.000000in}}%
\pgfpathlineto{\pgfqpoint{0.000000in}{-0.048611in}}%
\pgfusepath{stroke,fill}%
}%
\begin{pgfscope}%
\pgfsys@transformshift{12.111433in}{0.814008in}%
\pgfsys@useobject{currentmarker}{}%
\end{pgfscope}%
\end{pgfscope}%
\begin{pgfscope}%
\definecolor{textcolor}{rgb}{0.000000,0.000000,0.000000}%
\pgfsetstrokecolor{textcolor}%
\pgfsetfillcolor{textcolor}%
\pgftext[x=11.564654in, y=0.333955in, left, base,rotate=12.500000]{\color{textcolor}{\rmfamily\fontsize{14.000000}{16.800000}\selectfont\catcode`\^=\active\def^{\ifmmode\sp\else\^{}\fi}\catcode`\%=\active\def%{\%}WindTurbine}}%
\end{pgfscope}%
\begin{pgfscope}%
\pgfpathrectangle{\pgfqpoint{0.417359in}{0.814008in}}{\pgfqpoint{12.309552in}{4.722875in}}%
\pgfusepath{clip}%
\pgfsetrectcap%
\pgfsetroundjoin%
\pgfsetlinewidth{0.803000pt}%
\definecolor{currentstroke}{rgb}{0.690196,0.690196,0.690196}%
\pgfsetstrokecolor{currentstroke}%
\pgfsetstrokeopacity{0.200000}%
\pgfsetdash{}{0pt}%
\pgfpathmoveto{\pgfqpoint{0.417359in}{0.814008in}}%
\pgfpathlineto{\pgfqpoint{12.726910in}{0.814008in}}%
\pgfusepath{stroke}%
\end{pgfscope}%
\begin{pgfscope}%
\pgfsetbuttcap%
\pgfsetroundjoin%
\definecolor{currentfill}{rgb}{0.000000,0.000000,0.000000}%
\pgfsetfillcolor{currentfill}%
\pgfsetlinewidth{0.803000pt}%
\definecolor{currentstroke}{rgb}{0.000000,0.000000,0.000000}%
\pgfsetstrokecolor{currentstroke}%
\pgfsetdash{}{0pt}%
\pgfsys@defobject{currentmarker}{\pgfqpoint{-0.048611in}{0.000000in}}{\pgfqpoint{-0.000000in}{0.000000in}}{%
\pgfpathmoveto{\pgfqpoint{-0.000000in}{0.000000in}}%
\pgfpathlineto{\pgfqpoint{-0.048611in}{0.000000in}}%
\pgfusepath{stroke,fill}%
}%
\begin{pgfscope}%
\pgfsys@transformshift{0.417359in}{0.814008in}%
\pgfsys@useobject{currentmarker}{}%
\end{pgfscope}%
\end{pgfscope}%
\begin{pgfscope}%
\definecolor{textcolor}{rgb}{0.000000,0.000000,0.000000}%
\pgfsetstrokecolor{textcolor}%
\pgfsetfillcolor{textcolor}%
\pgftext[x=0.210068in, y=0.730674in, left, base]{\color{textcolor}{\rmfamily\fontsize{16.000000}{19.200000}\selectfont\catcode`\^=\active\def^{\ifmmode\sp\else\^{}\fi}\catcode`\%=\active\def%{\%}$\mathdefault{0}$}}%
\end{pgfscope}%
\begin{pgfscope}%
\pgfpathrectangle{\pgfqpoint{0.417359in}{0.814008in}}{\pgfqpoint{12.309552in}{4.722875in}}%
\pgfusepath{clip}%
\pgfsetrectcap%
\pgfsetroundjoin%
\pgfsetlinewidth{0.803000pt}%
\definecolor{currentstroke}{rgb}{0.690196,0.690196,0.690196}%
\pgfsetstrokecolor{currentstroke}%
\pgfsetstrokeopacity{0.200000}%
\pgfsetdash{}{0pt}%
\pgfpathmoveto{\pgfqpoint{0.417359in}{1.601154in}}%
\pgfpathlineto{\pgfqpoint{12.726910in}{1.601154in}}%
\pgfusepath{stroke}%
\end{pgfscope}%
\begin{pgfscope}%
\pgfsetbuttcap%
\pgfsetroundjoin%
\definecolor{currentfill}{rgb}{0.000000,0.000000,0.000000}%
\pgfsetfillcolor{currentfill}%
\pgfsetlinewidth{0.803000pt}%
\definecolor{currentstroke}{rgb}{0.000000,0.000000,0.000000}%
\pgfsetstrokecolor{currentstroke}%
\pgfsetdash{}{0pt}%
\pgfsys@defobject{currentmarker}{\pgfqpoint{-0.048611in}{0.000000in}}{\pgfqpoint{-0.000000in}{0.000000in}}{%
\pgfpathmoveto{\pgfqpoint{-0.000000in}{0.000000in}}%
\pgfpathlineto{\pgfqpoint{-0.048611in}{0.000000in}}%
\pgfusepath{stroke,fill}%
}%
\begin{pgfscope}%
\pgfsys@transformshift{0.417359in}{1.601154in}%
\pgfsys@useobject{currentmarker}{}%
\end{pgfscope}%
\end{pgfscope}%
\begin{pgfscope}%
\definecolor{textcolor}{rgb}{0.000000,0.000000,0.000000}%
\pgfsetstrokecolor{textcolor}%
\pgfsetfillcolor{textcolor}%
\pgftext[x=0.210068in, y=1.517820in, left, base]{\color{textcolor}{\rmfamily\fontsize{16.000000}{19.200000}\selectfont\catcode`\^=\active\def^{\ifmmode\sp\else\^{}\fi}\catcode`\%=\active\def%{\%}$\mathdefault{5}$}}%
\end{pgfscope}%
\begin{pgfscope}%
\pgfpathrectangle{\pgfqpoint{0.417359in}{0.814008in}}{\pgfqpoint{12.309552in}{4.722875in}}%
\pgfusepath{clip}%
\pgfsetrectcap%
\pgfsetroundjoin%
\pgfsetlinewidth{0.803000pt}%
\definecolor{currentstroke}{rgb}{0.690196,0.690196,0.690196}%
\pgfsetstrokecolor{currentstroke}%
\pgfsetstrokeopacity{0.200000}%
\pgfsetdash{}{0pt}%
\pgfpathmoveto{\pgfqpoint{0.417359in}{2.388299in}}%
\pgfpathlineto{\pgfqpoint{12.726910in}{2.388299in}}%
\pgfusepath{stroke}%
\end{pgfscope}%
\begin{pgfscope}%
\pgfsetbuttcap%
\pgfsetroundjoin%
\definecolor{currentfill}{rgb}{0.000000,0.000000,0.000000}%
\pgfsetfillcolor{currentfill}%
\pgfsetlinewidth{0.803000pt}%
\definecolor{currentstroke}{rgb}{0.000000,0.000000,0.000000}%
\pgfsetstrokecolor{currentstroke}%
\pgfsetdash{}{0pt}%
\pgfsys@defobject{currentmarker}{\pgfqpoint{-0.048611in}{0.000000in}}{\pgfqpoint{-0.000000in}{0.000000in}}{%
\pgfpathmoveto{\pgfqpoint{-0.000000in}{0.000000in}}%
\pgfpathlineto{\pgfqpoint{-0.048611in}{0.000000in}}%
\pgfusepath{stroke,fill}%
}%
\begin{pgfscope}%
\pgfsys@transformshift{0.417359in}{2.388299in}%
\pgfsys@useobject{currentmarker}{}%
\end{pgfscope}%
\end{pgfscope}%
\begin{pgfscope}%
\definecolor{textcolor}{rgb}{0.000000,0.000000,0.000000}%
\pgfsetstrokecolor{textcolor}%
\pgfsetfillcolor{textcolor}%
\pgftext[x=0.100000in, y=2.304966in, left, base]{\color{textcolor}{\rmfamily\fontsize{16.000000}{19.200000}\selectfont\catcode`\^=\active\def^{\ifmmode\sp\else\^{}\fi}\catcode`\%=\active\def%{\%}$\mathdefault{10}$}}%
\end{pgfscope}%
\begin{pgfscope}%
\pgfpathrectangle{\pgfqpoint{0.417359in}{0.814008in}}{\pgfqpoint{12.309552in}{4.722875in}}%
\pgfusepath{clip}%
\pgfsetrectcap%
\pgfsetroundjoin%
\pgfsetlinewidth{0.803000pt}%
\definecolor{currentstroke}{rgb}{0.690196,0.690196,0.690196}%
\pgfsetstrokecolor{currentstroke}%
\pgfsetstrokeopacity{0.200000}%
\pgfsetdash{}{0pt}%
\pgfpathmoveto{\pgfqpoint{0.417359in}{3.175445in}}%
\pgfpathlineto{\pgfqpoint{12.726910in}{3.175445in}}%
\pgfusepath{stroke}%
\end{pgfscope}%
\begin{pgfscope}%
\pgfsetbuttcap%
\pgfsetroundjoin%
\definecolor{currentfill}{rgb}{0.000000,0.000000,0.000000}%
\pgfsetfillcolor{currentfill}%
\pgfsetlinewidth{0.803000pt}%
\definecolor{currentstroke}{rgb}{0.000000,0.000000,0.000000}%
\pgfsetstrokecolor{currentstroke}%
\pgfsetdash{}{0pt}%
\pgfsys@defobject{currentmarker}{\pgfqpoint{-0.048611in}{0.000000in}}{\pgfqpoint{-0.000000in}{0.000000in}}{%
\pgfpathmoveto{\pgfqpoint{-0.000000in}{0.000000in}}%
\pgfpathlineto{\pgfqpoint{-0.048611in}{0.000000in}}%
\pgfusepath{stroke,fill}%
}%
\begin{pgfscope}%
\pgfsys@transformshift{0.417359in}{3.175445in}%
\pgfsys@useobject{currentmarker}{}%
\end{pgfscope}%
\end{pgfscope}%
\begin{pgfscope}%
\definecolor{textcolor}{rgb}{0.000000,0.000000,0.000000}%
\pgfsetstrokecolor{textcolor}%
\pgfsetfillcolor{textcolor}%
\pgftext[x=0.100000in, y=3.092112in, left, base]{\color{textcolor}{\rmfamily\fontsize{16.000000}{19.200000}\selectfont\catcode`\^=\active\def^{\ifmmode\sp\else\^{}\fi}\catcode`\%=\active\def%{\%}$\mathdefault{15}$}}%
\end{pgfscope}%
\begin{pgfscope}%
\pgfpathrectangle{\pgfqpoint{0.417359in}{0.814008in}}{\pgfqpoint{12.309552in}{4.722875in}}%
\pgfusepath{clip}%
\pgfsetrectcap%
\pgfsetroundjoin%
\pgfsetlinewidth{0.803000pt}%
\definecolor{currentstroke}{rgb}{0.690196,0.690196,0.690196}%
\pgfsetstrokecolor{currentstroke}%
\pgfsetstrokeopacity{0.200000}%
\pgfsetdash{}{0pt}%
\pgfpathmoveto{\pgfqpoint{0.417359in}{3.962591in}}%
\pgfpathlineto{\pgfqpoint{12.726910in}{3.962591in}}%
\pgfusepath{stroke}%
\end{pgfscope}%
\begin{pgfscope}%
\pgfsetbuttcap%
\pgfsetroundjoin%
\definecolor{currentfill}{rgb}{0.000000,0.000000,0.000000}%
\pgfsetfillcolor{currentfill}%
\pgfsetlinewidth{0.803000pt}%
\definecolor{currentstroke}{rgb}{0.000000,0.000000,0.000000}%
\pgfsetstrokecolor{currentstroke}%
\pgfsetdash{}{0pt}%
\pgfsys@defobject{currentmarker}{\pgfqpoint{-0.048611in}{0.000000in}}{\pgfqpoint{-0.000000in}{0.000000in}}{%
\pgfpathmoveto{\pgfqpoint{-0.000000in}{0.000000in}}%
\pgfpathlineto{\pgfqpoint{-0.048611in}{0.000000in}}%
\pgfusepath{stroke,fill}%
}%
\begin{pgfscope}%
\pgfsys@transformshift{0.417359in}{3.962591in}%
\pgfsys@useobject{currentmarker}{}%
\end{pgfscope}%
\end{pgfscope}%
\begin{pgfscope}%
\definecolor{textcolor}{rgb}{0.000000,0.000000,0.000000}%
\pgfsetstrokecolor{textcolor}%
\pgfsetfillcolor{textcolor}%
\pgftext[x=0.100000in, y=3.879258in, left, base]{\color{textcolor}{\rmfamily\fontsize{16.000000}{19.200000}\selectfont\catcode`\^=\active\def^{\ifmmode\sp\else\^{}\fi}\catcode`\%=\active\def%{\%}$\mathdefault{20}$}}%
\end{pgfscope}%
\begin{pgfscope}%
\pgfpathrectangle{\pgfqpoint{0.417359in}{0.814008in}}{\pgfqpoint{12.309552in}{4.722875in}}%
\pgfusepath{clip}%
\pgfsetrectcap%
\pgfsetroundjoin%
\pgfsetlinewidth{0.803000pt}%
\definecolor{currentstroke}{rgb}{0.690196,0.690196,0.690196}%
\pgfsetstrokecolor{currentstroke}%
\pgfsetstrokeopacity{0.200000}%
\pgfsetdash{}{0pt}%
\pgfpathmoveto{\pgfqpoint{0.417359in}{4.749737in}}%
\pgfpathlineto{\pgfqpoint{12.726910in}{4.749737in}}%
\pgfusepath{stroke}%
\end{pgfscope}%
\begin{pgfscope}%
\pgfsetbuttcap%
\pgfsetroundjoin%
\definecolor{currentfill}{rgb}{0.000000,0.000000,0.000000}%
\pgfsetfillcolor{currentfill}%
\pgfsetlinewidth{0.803000pt}%
\definecolor{currentstroke}{rgb}{0.000000,0.000000,0.000000}%
\pgfsetstrokecolor{currentstroke}%
\pgfsetdash{}{0pt}%
\pgfsys@defobject{currentmarker}{\pgfqpoint{-0.048611in}{0.000000in}}{\pgfqpoint{-0.000000in}{0.000000in}}{%
\pgfpathmoveto{\pgfqpoint{-0.000000in}{0.000000in}}%
\pgfpathlineto{\pgfqpoint{-0.048611in}{0.000000in}}%
\pgfusepath{stroke,fill}%
}%
\begin{pgfscope}%
\pgfsys@transformshift{0.417359in}{4.749737in}%
\pgfsys@useobject{currentmarker}{}%
\end{pgfscope}%
\end{pgfscope}%
\begin{pgfscope}%
\definecolor{textcolor}{rgb}{0.000000,0.000000,0.000000}%
\pgfsetstrokecolor{textcolor}%
\pgfsetfillcolor{textcolor}%
\pgftext[x=0.100000in, y=4.666404in, left, base]{\color{textcolor}{\rmfamily\fontsize{16.000000}{19.200000}\selectfont\catcode`\^=\active\def^{\ifmmode\sp\else\^{}\fi}\catcode`\%=\active\def%{\%}$\mathdefault{25}$}}%
\end{pgfscope}%
\begin{pgfscope}%
\pgfpathrectangle{\pgfqpoint{0.417359in}{0.814008in}}{\pgfqpoint{12.309552in}{4.722875in}}%
\pgfusepath{clip}%
\pgfsetrectcap%
\pgfsetroundjoin%
\pgfsetlinewidth{0.803000pt}%
\definecolor{currentstroke}{rgb}{0.690196,0.690196,0.690196}%
\pgfsetstrokecolor{currentstroke}%
\pgfsetstrokeopacity{0.200000}%
\pgfsetdash{}{0pt}%
\pgfpathmoveto{\pgfqpoint{0.417359in}{5.536883in}}%
\pgfpathlineto{\pgfqpoint{12.726910in}{5.536883in}}%
\pgfusepath{stroke}%
\end{pgfscope}%
\begin{pgfscope}%
\pgfsetbuttcap%
\pgfsetroundjoin%
\definecolor{currentfill}{rgb}{0.000000,0.000000,0.000000}%
\pgfsetfillcolor{currentfill}%
\pgfsetlinewidth{0.803000pt}%
\definecolor{currentstroke}{rgb}{0.000000,0.000000,0.000000}%
\pgfsetstrokecolor{currentstroke}%
\pgfsetdash{}{0pt}%
\pgfsys@defobject{currentmarker}{\pgfqpoint{-0.048611in}{0.000000in}}{\pgfqpoint{-0.000000in}{0.000000in}}{%
\pgfpathmoveto{\pgfqpoint{-0.000000in}{0.000000in}}%
\pgfpathlineto{\pgfqpoint{-0.048611in}{0.000000in}}%
\pgfusepath{stroke,fill}%
}%
\begin{pgfscope}%
\pgfsys@transformshift{0.417359in}{5.536883in}%
\pgfsys@useobject{currentmarker}{}%
\end{pgfscope}%
\end{pgfscope}%
\begin{pgfscope}%
\definecolor{textcolor}{rgb}{0.000000,0.000000,0.000000}%
\pgfsetstrokecolor{textcolor}%
\pgfsetfillcolor{textcolor}%
\pgftext[x=0.100000in, y=5.453550in, left, base]{\color{textcolor}{\rmfamily\fontsize{16.000000}{19.200000}\selectfont\catcode`\^=\active\def^{\ifmmode\sp\else\^{}\fi}\catcode`\%=\active\def%{\%}$\mathdefault{30}$}}%
\end{pgfscope}%
\begin{pgfscope}%
\pgfpathrectangle{\pgfqpoint{0.417359in}{0.814008in}}{\pgfqpoint{12.309552in}{4.722875in}}%
\pgfusepath{clip}%
\pgfsetbuttcap%
\pgfsetroundjoin%
\pgfsetlinewidth{0.941016pt}%
\definecolor{currentstroke}{rgb}{0.240000,0.240000,0.240000}%
\pgfsetstrokecolor{currentstroke}%
\pgfsetdash{}{0pt}%
\pgfpathmoveto{\pgfqpoint{0.540454in}{2.418215in}}%
\pgfpathlineto{\pgfqpoint{1.525218in}{2.418215in}}%
\pgfusepath{stroke}%
\end{pgfscope}%
\begin{pgfscope}%
\pgfpathrectangle{\pgfqpoint{0.417359in}{0.814008in}}{\pgfqpoint{12.309552in}{4.722875in}}%
\pgfusepath{clip}%
\pgfsetbuttcap%
\pgfsetroundjoin%
\pgfsetlinewidth{0.941016pt}%
\definecolor{currentstroke}{rgb}{0.240000,0.240000,0.240000}%
\pgfsetstrokecolor{currentstroke}%
\pgfsetdash{}{0pt}%
\pgfpathmoveto{\pgfqpoint{1.771409in}{0.845451in}}%
\pgfpathlineto{\pgfqpoint{2.756173in}{0.845451in}}%
\pgfusepath{stroke}%
\end{pgfscope}%
\begin{pgfscope}%
\pgfpathrectangle{\pgfqpoint{0.417359in}{0.814008in}}{\pgfqpoint{12.309552in}{4.722875in}}%
\pgfusepath{clip}%
\pgfsetbuttcap%
\pgfsetroundjoin%
\pgfsetlinewidth{0.941016pt}%
\definecolor{currentstroke}{rgb}{0.240000,0.240000,0.240000}%
\pgfsetstrokecolor{currentstroke}%
\pgfsetdash{}{0pt}%
\pgfpathmoveto{\pgfqpoint{3.002364in}{0.946306in}}%
\pgfpathlineto{\pgfqpoint{3.987129in}{0.946306in}}%
\pgfusepath{stroke}%
\end{pgfscope}%
\begin{pgfscope}%
\pgfpathrectangle{\pgfqpoint{0.417359in}{0.814008in}}{\pgfqpoint{12.309552in}{4.722875in}}%
\pgfusepath{clip}%
\pgfsetbuttcap%
\pgfsetroundjoin%
\pgfsetlinewidth{0.941016pt}%
\definecolor{currentstroke}{rgb}{0.240000,0.240000,0.240000}%
\pgfsetstrokecolor{currentstroke}%
\pgfsetdash{}{0pt}%
\pgfpathmoveto{\pgfqpoint{4.233320in}{0.828283in}}%
\pgfpathlineto{\pgfqpoint{5.218084in}{0.828283in}}%
\pgfusepath{stroke}%
\end{pgfscope}%
\begin{pgfscope}%
\pgfpathrectangle{\pgfqpoint{0.417359in}{0.814008in}}{\pgfqpoint{12.309552in}{4.722875in}}%
\pgfusepath{clip}%
\pgfsetbuttcap%
\pgfsetroundjoin%
\pgfsetlinewidth{0.941016pt}%
\definecolor{currentstroke}{rgb}{0.240000,0.240000,0.240000}%
\pgfsetstrokecolor{currentstroke}%
\pgfsetdash{}{0pt}%
\pgfpathmoveto{\pgfqpoint{5.464275in}{1.163044in}}%
\pgfpathlineto{\pgfqpoint{6.449039in}{1.163044in}}%
\pgfusepath{stroke}%
\end{pgfscope}%
\begin{pgfscope}%
\pgfpathrectangle{\pgfqpoint{0.417359in}{0.814008in}}{\pgfqpoint{12.309552in}{4.722875in}}%
\pgfusepath{clip}%
\pgfsetbuttcap%
\pgfsetroundjoin%
\pgfsetlinewidth{0.941016pt}%
\definecolor{currentstroke}{rgb}{0.240000,0.240000,0.240000}%
\pgfsetstrokecolor{currentstroke}%
\pgfsetdash{}{0pt}%
\pgfpathmoveto{\pgfqpoint{6.695230in}{0.984350in}}%
\pgfpathlineto{\pgfqpoint{7.679994in}{0.984350in}}%
\pgfusepath{stroke}%
\end{pgfscope}%
\begin{pgfscope}%
\pgfpathrectangle{\pgfqpoint{0.417359in}{0.814008in}}{\pgfqpoint{12.309552in}{4.722875in}}%
\pgfusepath{clip}%
\pgfsetbuttcap%
\pgfsetroundjoin%
\pgfsetlinewidth{0.941016pt}%
\definecolor{currentstroke}{rgb}{0.240000,0.240000,0.240000}%
\pgfsetstrokecolor{currentstroke}%
\pgfsetdash{}{0pt}%
\pgfpathmoveto{\pgfqpoint{7.926185in}{2.995341in}}%
\pgfpathlineto{\pgfqpoint{8.910949in}{2.995341in}}%
\pgfusepath{stroke}%
\end{pgfscope}%
\begin{pgfscope}%
\pgfpathrectangle{\pgfqpoint{0.417359in}{0.814008in}}{\pgfqpoint{12.309552in}{4.722875in}}%
\pgfusepath{clip}%
\pgfsetbuttcap%
\pgfsetroundjoin%
\pgfsetlinewidth{0.941016pt}%
\definecolor{currentstroke}{rgb}{0.240000,0.240000,0.240000}%
\pgfsetstrokecolor{currentstroke}%
\pgfsetdash{}{0pt}%
\pgfpathmoveto{\pgfqpoint{9.157140in}{1.063487in}}%
\pgfpathlineto{\pgfqpoint{10.141905in}{1.063487in}}%
\pgfusepath{stroke}%
\end{pgfscope}%
\begin{pgfscope}%
\pgfpathrectangle{\pgfqpoint{0.417359in}{0.814008in}}{\pgfqpoint{12.309552in}{4.722875in}}%
\pgfusepath{clip}%
\pgfsetbuttcap%
\pgfsetroundjoin%
\pgfsetlinewidth{0.941016pt}%
\definecolor{currentstroke}{rgb}{0.240000,0.240000,0.240000}%
\pgfsetstrokecolor{currentstroke}%
\pgfsetdash{}{0pt}%
\pgfpathmoveto{\pgfqpoint{10.388096in}{1.005282in}}%
\pgfpathlineto{\pgfqpoint{11.372860in}{1.005282in}}%
\pgfusepath{stroke}%
\end{pgfscope}%
\begin{pgfscope}%
\pgfpathrectangle{\pgfqpoint{0.417359in}{0.814008in}}{\pgfqpoint{12.309552in}{4.722875in}}%
\pgfusepath{clip}%
\pgfsetbuttcap%
\pgfsetroundjoin%
\pgfsetlinewidth{0.941016pt}%
\definecolor{currentstroke}{rgb}{0.240000,0.240000,0.240000}%
\pgfsetstrokecolor{currentstroke}%
\pgfsetdash{}{0pt}%
\pgfpathmoveto{\pgfqpoint{11.619051in}{0.840986in}}%
\pgfpathlineto{\pgfqpoint{12.603815in}{0.840986in}}%
\pgfusepath{stroke}%
\end{pgfscope}%
\begin{pgfscope}%
\pgfsetrectcap%
\pgfsetmiterjoin%
\pgfsetlinewidth{0.803000pt}%
\definecolor{currentstroke}{rgb}{0.000000,0.000000,0.000000}%
\pgfsetstrokecolor{currentstroke}%
\pgfsetdash{}{0pt}%
\pgfpathmoveto{\pgfqpoint{0.417359in}{0.814008in}}%
\pgfpathlineto{\pgfqpoint{0.417359in}{5.536883in}}%
\pgfusepath{stroke}%
\end{pgfscope}%
\begin{pgfscope}%
\pgfsetrectcap%
\pgfsetmiterjoin%
\pgfsetlinewidth{0.803000pt}%
\definecolor{currentstroke}{rgb}{0.000000,0.000000,0.000000}%
\pgfsetstrokecolor{currentstroke}%
\pgfsetdash{}{0pt}%
\pgfpathmoveto{\pgfqpoint{12.726910in}{0.814008in}}%
\pgfpathlineto{\pgfqpoint{12.726910in}{5.536883in}}%
\pgfusepath{stroke}%
\end{pgfscope}%
\begin{pgfscope}%
\pgfsetrectcap%
\pgfsetmiterjoin%
\pgfsetlinewidth{0.803000pt}%
\definecolor{currentstroke}{rgb}{0.000000,0.000000,0.000000}%
\pgfsetstrokecolor{currentstroke}%
\pgfsetdash{}{0pt}%
\pgfpathmoveto{\pgfqpoint{0.417359in}{0.814008in}}%
\pgfpathlineto{\pgfqpoint{12.726910in}{0.814008in}}%
\pgfusepath{stroke}%
\end{pgfscope}%
\begin{pgfscope}%
\pgfsetrectcap%
\pgfsetmiterjoin%
\pgfsetlinewidth{0.803000pt}%
\definecolor{currentstroke}{rgb}{0.000000,0.000000,0.000000}%
\pgfsetstrokecolor{currentstroke}%
\pgfsetdash{}{0pt}%
\pgfpathmoveto{\pgfqpoint{0.417359in}{5.536883in}}%
\pgfpathlineto{\pgfqpoint{12.726910in}{5.536883in}}%
\pgfusepath{stroke}%
\end{pgfscope}%
\begin{pgfscope}%
\definecolor{textcolor}{rgb}{0.000000,0.000000,0.000000}%
\pgfsetstrokecolor{textcolor}%
\pgfsetfillcolor{textcolor}%
\pgftext[x=6.572135in,y=5.620216in,,base]{\color{textcolor}{\rmfamily\fontsize{16.000000}{19.200000}\selectfont\catcode`\^=\active\def^{\ifmmode\sp\else\^{}\fi}\catcode`\%=\active\def%{\%}Design Space (MGA)}}%
\end{pgfscope}%
\end{pgfpicture}%
\makeatother%
\endgroup%
}
  \caption{The design space for a four objective problem including alternative solutions suggested by MGA.}
  \label{fig:4-obj-design-mga}
\end{figure}