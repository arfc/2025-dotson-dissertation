
\subsection{Exercise 3: Four Simultaneous Objectives}
Chapter \ref{chapter:lit-review} showed that conventional \acp{esom} virtually
always model a single objective and that objective is uniformly cost (or a
similar aggregated economic indicator). Further, Section
\ref{section:moo-in-energy} showed that the existing literature employing
\ac{moo} never model more than three objectives simultaneously. The purpose of
this final exercise is to demonstrate that \ac{osier} can optimize many
objectives, thereby providing more context and confidence for the tool. This
exercise minimized four objectives simultaneously: total system cost, lifecycle
carbon emissions, land-use change, and percentage of total energy from
non-renewable energy sources. Renewable energy sources include solar, wind, and
biomass. Although batteries are often used in conjunction with \acp{vre}, they
are not considered ``renewable'' (nor are they a true energy ``source'' since
they store energy from other sources rather than producing their own). For
clarity, the ``percent non-renewable'' objective refers to the penetration of
non-renewable sources as a percentage of the energy produced rather than as a
percentage of the systems total installed capacity. Figure
\ref{fig:4-obj-pareto} shows the objective-space Pareto front for this
4-dimensional problem.

\begin{noteBox}
\textbf{Reading \Aclp{pcp}:} Visualizing the Pareto front for this problem
presents a challenge due to its high dimensionality. Therefore, I present the
results with a novel plot, called a \ac{pcp}. This plot is helpful for
highlighting differences among a small set of solutions with a potentially large
number of dimensions. Figure \ref{fig:4-obj-pareto} and Figure
\ref{fig:4-obj-design} are both \acp{pcp}. Although \acp{pcp} show continuous
lines, they do not show a ``trend''. That is, for a given solution, each
objective takes on a single value that is plotted on its respective vertical
axis. The lines connecting these points simply emphasize that these points
belong to the same solution. Additionally, each objective axis has its own upper
and lower bound because each objective is scaled differently. The \ac{mga}
solutions presented in Figure \ref{fig:4-obj-design-mga} using a boxplot due to
the larger number of solutions included in \ac{mga}. 
\end{noteBox}


\begin{figure}[h]
  \centering
  \resizebox{\columnwidth}{!}{\input{figures/04_benchmark_chapter/4_obj_objective_space.pgf}}
  \caption{The Pareto front for a four objective problem. Extreme values for
  each objective are colored. The gray lines represent solutions on the Pareto
  front that are not extremum.}
  \label{fig:4-obj-pareto}
\end{figure}

Each of the colored lines in Figure \ref{fig:4-obj-pareto} belongs to a solution
with an `extreme' value on the Pareto-front. For instance, the blue line labeled
``Highest Renewable'' has the lowest percentage of non-renewable energy sources
of any solution. The gray lines are simply other points along the Pareto-front.
Figure \ref{fig:4-obj-pareto} shows that minimizing land-use change and
renewable energy maximization are strongly competing objectives, since the other
three extremum are grouped together on those two axes and diametrically opposed
to the ``highest renewable'' solution. Figure \ref{fig:4-obj-design} illustrates
the design space for each extreme solution. 


\begin{figure}[h]
  \centering
  \resizebox{\columnwidth}{!}{\input{figures/04_benchmark_chapter/4_obj_design_space.pgf}}
  \caption{The design space for a four objective problem.}
  \label{fig:4-obj-design}
\end{figure}

Figure \ref{fig:4-obj-design} shows that conventional coal and advanced coal
technologies are largely uninteresting because they make up at most 7\% and 4\%
of a solution's peak demand, respectively. The ``highest renewable'' solution
achieves its goal of reaching approximately 100\% renewable energy (by
percentage of energy produced) with a significant overbuild of wind energy and
batteries, with natural gas and a small amount of coal for reliability.
Interestingly, this solution uses no solar energy, even though solar and wind
are frequently assumed to complement each other.

Figure \ref{fig:4-obj-design-mga} extends the design space results to include
the \ac{mga} solutions. This plot indicates the design preferences for a
middling solution, but hides the relationship among energy technologies. The
most popular technologies in Figure \ref{fig:4-obj-design-mga} are conventional
nuclear, battery storage, and solar panels. The least popular technologies are
wind turbines, biomass, and ``advanced'' coal plants.

\begin{figure}[h]
  \centering
  \resizebox{\columnwidth}{!}{\input{figures/04_benchmark_chapter/4-obj-mga-design-space.pgf}}
  \caption{The design space for a four objective problem including alternative solutions suggested by MGA.}
  \label{fig:4-obj-design-mga}
\end{figure}