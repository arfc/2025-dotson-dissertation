\subsection{Exercise 2: Exploring objective space}
Since structural uncertainty persists regardless of the number of objectives
used, it's important to check the near-optimal objective space for alternative
solutions. In the first benchmark exercise, I used \ac{temoa} to calculate the
least-cost solution. Then I generated 30 alternative solutions with \ac{mga} as
described in Section \ref{section:mga} with a 10\% slack variable added to
\ac{temoa}'s objective function. Figure \ref{fig:temoa-benchmark-01} shows the
points from \ac{temoa} in red and \ac{osier}'s Pareto-front for the same problem
in black. The red- and gray-shaded regions are the sub-optimal spaces (i.e.,
within 10\% of any objective) for \ac{temoa} and \ac{osier}, respectively.
The solid black points indicates points along the Pareto-front, while the open
black points are points tested in early generations of the \ac{osier} simulation.
\footnote{There are many more tested points than shown in Figure \ref{fig:temoa-benchmark-01}. 
For simplicity and clarity, Figure \ref{fig:temoa-benchmark-01} shows a random subset
of points.}

\begin{figure}[h]
  \centering
  % \resizebox{0.6\columnwidth}{!}{%% Creator: Matplotlib, PGF backend
%%
%% To include the figure in your LaTeX document, write
%%   \input{<filename>.pgf}
%%
%% Make sure the required packages are loaded in your preamble
%%   \usepackage{pgf}
%%
%% Also ensure that all the required font packages are loaded; for instance,
%% the lmodern package is sometimes necessary when using math font.
%%   \usepackage{lmodern}
%%
%% Figures using additional raster images can only be included by \input if
%% they are in the same directory as the main LaTeX file. For loading figures
%% from other directories you can use the `import` package
%%   \usepackage{import}
%%
%% and then include the figures with
%%   \import{<path to file>}{<filename>.pgf}
%%
%% Matplotlib used the following preamble
%%   \def\mathdefault#1{#1}
%%   \everymath=\expandafter{\the\everymath\displaystyle}
%%   \IfFileExists{scrextend.sty}{
%%     \usepackage[fontsize=10.000000pt]{scrextend}
%%   }{
%%     \renewcommand{\normalsize}{\fontsize{10.000000}{12.000000}\selectfont}
%%     \normalsize
%%   }
%%   
%%   \makeatletter\@ifpackageloaded{underscore}{}{\usepackage[strings]{underscore}}\makeatother
%%
\begingroup%
\makeatletter%
\begin{pgfpicture}%
\pgfpathrectangle{\pgfpointorigin}{\pgfqpoint{6.988192in}{5.458470in}}%
\pgfusepath{use as bounding box, clip}%
\begin{pgfscope}%
\pgfsetbuttcap%
\pgfsetmiterjoin%
\definecolor{currentfill}{rgb}{1.000000,1.000000,1.000000}%
\pgfsetfillcolor{currentfill}%
\pgfsetlinewidth{0.000000pt}%
\definecolor{currentstroke}{rgb}{0.000000,0.000000,0.000000}%
\pgfsetstrokecolor{currentstroke}%
\pgfsetdash{}{0pt}%
\pgfpathmoveto{\pgfqpoint{0.000000in}{0.000000in}}%
\pgfpathlineto{\pgfqpoint{6.988192in}{0.000000in}}%
\pgfpathlineto{\pgfqpoint{6.988192in}{5.458470in}}%
\pgfpathlineto{\pgfqpoint{0.000000in}{5.458470in}}%
\pgfpathlineto{\pgfqpoint{0.000000in}{0.000000in}}%
\pgfpathclose%
\pgfusepath{fill}%
\end{pgfscope}%
\begin{pgfscope}%
\pgfsetbuttcap%
\pgfsetmiterjoin%
\definecolor{currentfill}{rgb}{1.000000,1.000000,1.000000}%
\pgfsetfillcolor{currentfill}%
\pgfsetlinewidth{0.000000pt}%
\definecolor{currentstroke}{rgb}{0.000000,0.000000,0.000000}%
\pgfsetstrokecolor{currentstroke}%
\pgfsetstrokeopacity{0.000000}%
\pgfsetdash{}{0pt}%
\pgfpathmoveto{\pgfqpoint{0.688192in}{0.670138in}}%
\pgfpathlineto{\pgfqpoint{6.888192in}{0.670138in}}%
\pgfpathlineto{\pgfqpoint{6.888192in}{5.290138in}}%
\pgfpathlineto{\pgfqpoint{0.688192in}{5.290138in}}%
\pgfpathlineto{\pgfqpoint{0.688192in}{0.670138in}}%
\pgfpathclose%
\pgfusepath{fill}%
\end{pgfscope}%
\begin{pgfscope}%
\pgfpathrectangle{\pgfqpoint{0.688192in}{0.670138in}}{\pgfqpoint{6.200000in}{4.620000in}}%
\pgfusepath{clip}%
\pgfsetbuttcap%
\pgfsetmiterjoin%
\definecolor{currentfill}{rgb}{0.121569,0.466667,0.705882}%
\pgfsetfillcolor{currentfill}%
\pgfsetfillopacity{0.500000}%
\pgfsetlinewidth{1.003750pt}%
\definecolor{currentstroke}{rgb}{0.121569,0.466667,0.705882}%
\pgfsetstrokecolor{currentstroke}%
\pgfsetstrokeopacity{0.500000}%
\pgfsetdash{}{0pt}%
\pgfpathmoveto{\pgfqpoint{0.741425in}{1.377543in}}%
\pgfpathlineto{\pgfqpoint{0.758703in}{0.955032in}}%
\pgfpathlineto{\pgfqpoint{0.768198in}{0.875033in}}%
\pgfpathlineto{\pgfqpoint{0.774746in}{0.828781in}}%
\pgfpathlineto{\pgfqpoint{0.778243in}{0.822495in}}%
\pgfpathlineto{\pgfqpoint{0.782159in}{0.789611in}}%
\pgfpathlineto{\pgfqpoint{0.786516in}{0.779881in}}%
\pgfpathlineto{\pgfqpoint{0.792538in}{0.779145in}}%
\pgfpathlineto{\pgfqpoint{0.794668in}{0.758056in}}%
\pgfpathlineto{\pgfqpoint{0.799837in}{0.752930in}}%
\pgfpathlineto{\pgfqpoint{0.809370in}{0.751978in}}%
\pgfpathlineto{\pgfqpoint{0.812629in}{0.743975in}}%
\pgfpathlineto{\pgfqpoint{0.815972in}{0.742575in}}%
\pgfpathlineto{\pgfqpoint{0.822987in}{0.738477in}}%
\pgfpathlineto{\pgfqpoint{0.828825in}{0.734937in}}%
\pgfpathlineto{\pgfqpoint{0.829214in}{0.733319in}}%
\pgfpathlineto{\pgfqpoint{0.833044in}{0.730858in}}%
\pgfpathlineto{\pgfqpoint{0.848459in}{0.726329in}}%
\pgfpathlineto{\pgfqpoint{0.864854in}{0.720019in}}%
\pgfpathlineto{\pgfqpoint{0.887104in}{0.715517in}}%
\pgfpathlineto{\pgfqpoint{0.907479in}{0.714004in}}%
\pgfpathlineto{\pgfqpoint{0.908310in}{0.712008in}}%
\pgfpathlineto{\pgfqpoint{0.909513in}{0.708525in}}%
\pgfpathlineto{\pgfqpoint{0.912740in}{0.707284in}}%
\pgfpathlineto{\pgfqpoint{0.920440in}{0.706723in}}%
\pgfpathlineto{\pgfqpoint{0.925670in}{0.705238in}}%
\pgfpathlineto{\pgfqpoint{0.948903in}{0.702931in}}%
\pgfpathlineto{\pgfqpoint{0.951945in}{0.701707in}}%
\pgfpathlineto{\pgfqpoint{0.952035in}{0.700391in}}%
\pgfpathlineto{\pgfqpoint{0.957029in}{0.700173in}}%
\pgfpathlineto{\pgfqpoint{0.968828in}{0.697963in}}%
\pgfpathlineto{\pgfqpoint{0.974412in}{0.697738in}}%
\pgfpathlineto{\pgfqpoint{0.975275in}{0.696914in}}%
\pgfpathlineto{\pgfqpoint{1.021767in}{0.694795in}}%
\pgfpathlineto{\pgfqpoint{1.025407in}{0.690657in}}%
\pgfpathlineto{\pgfqpoint{1.027475in}{0.690338in}}%
\pgfpathlineto{\pgfqpoint{1.034837in}{0.689784in}}%
\pgfpathlineto{\pgfqpoint{1.049406in}{0.687676in}}%
\pgfpathlineto{\pgfqpoint{1.054714in}{0.687138in}}%
\pgfpathlineto{\pgfqpoint{1.059617in}{0.686467in}}%
\pgfpathlineto{\pgfqpoint{1.072141in}{0.685078in}}%
\pgfpathlineto{\pgfqpoint{1.092208in}{0.684413in}}%
\pgfpathlineto{\pgfqpoint{1.115209in}{0.684111in}}%
\pgfpathlineto{\pgfqpoint{1.131834in}{0.684071in}}%
\pgfpathlineto{\pgfqpoint{1.152628in}{0.684059in}}%
\pgfpathlineto{\pgfqpoint{1.251312in}{0.683263in}}%
\pgfpathlineto{\pgfqpoint{1.277476in}{0.683159in}}%
\pgfpathlineto{\pgfqpoint{1.314870in}{0.682855in}}%
\pgfpathlineto{\pgfqpoint{1.369253in}{0.682756in}}%
\pgfpathlineto{\pgfqpoint{1.398687in}{0.682288in}}%
\pgfpathlineto{\pgfqpoint{1.467852in}{0.682134in}}%
\pgfpathlineto{\pgfqpoint{1.557026in}{0.681680in}}%
\pgfpathlineto{\pgfqpoint{1.627242in}{0.680913in}}%
\pgfpathlineto{\pgfqpoint{1.737728in}{0.680478in}}%
\pgfpathlineto{\pgfqpoint{1.887036in}{0.679610in}}%
\pgfpathlineto{\pgfqpoint{2.037481in}{0.678826in}}%
\pgfpathlineto{\pgfqpoint{2.258348in}{0.677741in}}%
\pgfpathlineto{\pgfqpoint{2.626338in}{0.676361in}}%
\pgfpathlineto{\pgfqpoint{3.263784in}{0.674352in}}%
\pgfpathlineto{\pgfqpoint{5.322800in}{0.670138in}}%
\pgfpathlineto{\pgfqpoint{6.888192in}{0.683471in}}%
\pgfpathlineto{\pgfqpoint{4.623274in}{0.688107in}}%
\pgfpathlineto{\pgfqpoint{3.922083in}{0.690316in}}%
\pgfpathlineto{\pgfqpoint{3.517294in}{0.691835in}}%
\pgfpathlineto{\pgfqpoint{3.274341in}{0.693028in}}%
\pgfpathlineto{\pgfqpoint{3.108851in}{0.693891in}}%
\pgfpathlineto{\pgfqpoint{2.944612in}{0.694845in}}%
\pgfpathlineto{\pgfqpoint{2.823078in}{0.695324in}}%
\pgfpathlineto{\pgfqpoint{2.745840in}{0.696168in}}%
\pgfpathlineto{\pgfqpoint{2.647748in}{0.696667in}}%
\pgfpathlineto{\pgfqpoint{2.571667in}{0.696836in}}%
\pgfpathlineto{\pgfqpoint{2.539290in}{0.697351in}}%
\pgfpathlineto{\pgfqpoint{2.479468in}{0.697460in}}%
\pgfpathlineto{\pgfqpoint{2.438334in}{0.697795in}}%
\pgfpathlineto{\pgfqpoint{2.409554in}{0.697909in}}%
\pgfpathlineto{\pgfqpoint{2.301002in}{0.698784in}}%
\pgfpathlineto{\pgfqpoint{2.278129in}{0.698798in}}%
\pgfpathlineto{\pgfqpoint{2.259841in}{0.698842in}}%
\pgfpathlineto{\pgfqpoint{2.234540in}{0.699174in}}%
\pgfpathlineto{\pgfqpoint{2.212467in}{0.699905in}}%
\pgfpathlineto{\pgfqpoint{2.198690in}{0.701434in}}%
\pgfpathlineto{\pgfqpoint{2.193297in}{0.702172in}}%
\pgfpathlineto{\pgfqpoint{2.187458in}{0.702763in}}%
\pgfpathlineto{\pgfqpoint{2.171432in}{0.705082in}}%
\pgfpathlineto{\pgfqpoint{2.163334in}{0.705692in}}%
\pgfpathlineto{\pgfqpoint{2.161059in}{0.706043in}}%
\pgfpathlineto{\pgfqpoint{2.157055in}{0.710594in}}%
\pgfpathlineto{\pgfqpoint{2.105914in}{0.712924in}}%
\pgfpathlineto{\pgfqpoint{2.104964in}{0.713831in}}%
\pgfpathlineto{\pgfqpoint{2.098822in}{0.714079in}}%
\pgfpathlineto{\pgfqpoint{2.085843in}{0.716510in}}%
\pgfpathlineto{\pgfqpoint{2.080349in}{0.716749in}}%
\pgfpathlineto{\pgfqpoint{2.080251in}{0.718197in}}%
\pgfpathlineto{\pgfqpoint{2.076904in}{0.719544in}}%
\pgfpathlineto{\pgfqpoint{2.051349in}{0.722081in}}%
\pgfpathlineto{\pgfqpoint{2.045595in}{0.723715in}}%
\pgfpathlineto{\pgfqpoint{2.037125in}{0.724332in}}%
\pgfpathlineto{\pgfqpoint{2.033576in}{0.725697in}}%
\pgfpathlineto{\pgfqpoint{2.032252in}{0.729528in}}%
\pgfpathlineto{\pgfqpoint{2.031338in}{0.731724in}}%
\pgfpathlineto{\pgfqpoint{2.008926in}{0.733388in}}%
\pgfpathlineto{\pgfqpoint{1.984450in}{0.738340in}}%
\pgfpathlineto{\pgfqpoint{1.966417in}{0.745282in}}%
\pgfpathlineto{\pgfqpoint{1.949459in}{0.750264in}}%
\pgfpathlineto{\pgfqpoint{1.945246in}{0.752971in}}%
\pgfpathlineto{\pgfqpoint{1.944819in}{0.754750in}}%
\pgfpathlineto{\pgfqpoint{1.938397in}{0.758644in}}%
\pgfpathlineto{\pgfqpoint{1.930681in}{0.763152in}}%
\pgfpathlineto{\pgfqpoint{1.927003in}{0.764692in}}%
\pgfpathlineto{\pgfqpoint{1.923418in}{0.773495in}}%
\pgfpathlineto{\pgfqpoint{1.912932in}{0.774543in}}%
\pgfpathlineto{\pgfqpoint{1.907246in}{0.780181in}}%
\pgfpathlineto{\pgfqpoint{1.904902in}{0.803379in}}%
\pgfpathlineto{\pgfqpoint{1.898279in}{0.804188in}}%
\pgfpathlineto{\pgfqpoint{1.893486in}{0.814892in}}%
\pgfpathlineto{\pgfqpoint{1.889178in}{0.851064in}}%
\pgfpathlineto{\pgfqpoint{1.885331in}{0.857978in}}%
\pgfpathlineto{\pgfqpoint{1.878129in}{0.908856in}}%
\pgfpathlineto{\pgfqpoint{1.867684in}{0.996854in}}%
\pgfpathlineto{\pgfqpoint{1.848679in}{1.461617in}}%
\pgfpathlineto{\pgfqpoint{0.741425in}{1.377543in}}%
\pgfpathclose%
\pgfusepath{stroke,fill}%
\end{pgfscope}%
\begin{pgfscope}%
\pgfpathrectangle{\pgfqpoint{0.688192in}{0.670138in}}{\pgfqpoint{6.200000in}{4.620000in}}%
\pgfusepath{clip}%
\pgfsetbuttcap%
\pgfsetroundjoin%
\pgfsetlinewidth{1.003750pt}%
\definecolor{currentstroke}{rgb}{1.000000,0.000000,0.000000}%
\pgfsetstrokecolor{currentstroke}%
\pgfsetdash{}{0pt}%
\pgfpathmoveto{\pgfqpoint{1.380312in}{4.826960in}}%
\pgfpathcurveto{\pgfqpoint{1.388548in}{4.826960in}}{\pgfqpoint{1.396449in}{4.830233in}}{\pgfqpoint{1.402272in}{4.836057in}}%
\pgfpathcurveto{\pgfqpoint{1.408096in}{4.841881in}}{\pgfqpoint{1.411369in}{4.849781in}}{\pgfqpoint{1.411369in}{4.858017in}}%
\pgfpathcurveto{\pgfqpoint{1.411369in}{4.866253in}}{\pgfqpoint{1.408096in}{4.874153in}}{\pgfqpoint{1.402272in}{4.879977in}}%
\pgfpathcurveto{\pgfqpoint{1.396449in}{4.885801in}}{\pgfqpoint{1.388548in}{4.889073in}}{\pgfqpoint{1.380312in}{4.889073in}}%
\pgfpathcurveto{\pgfqpoint{1.372076in}{4.889073in}}{\pgfqpoint{1.364176in}{4.885801in}}{\pgfqpoint{1.358352in}{4.879977in}}%
\pgfpathcurveto{\pgfqpoint{1.352528in}{4.874153in}}{\pgfqpoint{1.349256in}{4.866253in}}{\pgfqpoint{1.349256in}{4.858017in}}%
\pgfpathcurveto{\pgfqpoint{1.349256in}{4.849781in}}{\pgfqpoint{1.352528in}{4.841881in}}{\pgfqpoint{1.358352in}{4.836057in}}%
\pgfpathcurveto{\pgfqpoint{1.364176in}{4.830233in}}{\pgfqpoint{1.372076in}{4.826960in}}{\pgfqpoint{1.380312in}{4.826960in}}%
\pgfpathlineto{\pgfqpoint{1.380312in}{4.826960in}}%
\pgfpathclose%
\pgfusepath{stroke}%
\end{pgfscope}%
\begin{pgfscope}%
\pgfpathrectangle{\pgfqpoint{0.688192in}{0.670138in}}{\pgfqpoint{6.200000in}{4.620000in}}%
\pgfusepath{clip}%
\pgfsetbuttcap%
\pgfsetroundjoin%
\pgfsetlinewidth{1.003750pt}%
\definecolor{currentstroke}{rgb}{1.000000,0.000000,0.000000}%
\pgfsetstrokecolor{currentstroke}%
\pgfsetdash{}{0pt}%
\pgfpathmoveto{\pgfqpoint{1.103776in}{2.239473in}}%
\pgfpathcurveto{\pgfqpoint{1.112013in}{2.239473in}}{\pgfqpoint{1.119913in}{2.242745in}}{\pgfqpoint{1.125737in}{2.248569in}}%
\pgfpathcurveto{\pgfqpoint{1.131560in}{2.254393in}}{\pgfqpoint{1.134833in}{2.262293in}}{\pgfqpoint{1.134833in}{2.270529in}}%
\pgfpathcurveto{\pgfqpoint{1.134833in}{2.278765in}}{\pgfqpoint{1.131560in}{2.286665in}}{\pgfqpoint{1.125737in}{2.292489in}}%
\pgfpathcurveto{\pgfqpoint{1.119913in}{2.298313in}}{\pgfqpoint{1.112013in}{2.301586in}}{\pgfqpoint{1.103776in}{2.301586in}}%
\pgfpathcurveto{\pgfqpoint{1.095540in}{2.301586in}}{\pgfqpoint{1.087640in}{2.298313in}}{\pgfqpoint{1.081816in}{2.292489in}}%
\pgfpathcurveto{\pgfqpoint{1.075992in}{2.286665in}}{\pgfqpoint{1.072720in}{2.278765in}}{\pgfqpoint{1.072720in}{2.270529in}}%
\pgfpathcurveto{\pgfqpoint{1.072720in}{2.262293in}}{\pgfqpoint{1.075992in}{2.254393in}}{\pgfqpoint{1.081816in}{2.248569in}}%
\pgfpathcurveto{\pgfqpoint{1.087640in}{2.242745in}}{\pgfqpoint{1.095540in}{2.239473in}}{\pgfqpoint{1.103776in}{2.239473in}}%
\pgfpathlineto{\pgfqpoint{1.103776in}{2.239473in}}%
\pgfpathclose%
\pgfusepath{stroke}%
\end{pgfscope}%
\begin{pgfscope}%
\pgfpathrectangle{\pgfqpoint{0.688192in}{0.670138in}}{\pgfqpoint{6.200000in}{4.620000in}}%
\pgfusepath{clip}%
\pgfsetbuttcap%
\pgfsetroundjoin%
\pgfsetlinewidth{1.003750pt}%
\definecolor{currentstroke}{rgb}{1.000000,0.000000,0.000000}%
\pgfsetstrokecolor{currentstroke}%
\pgfsetdash{}{0pt}%
\pgfpathmoveto{\pgfqpoint{1.146295in}{2.309659in}}%
\pgfpathcurveto{\pgfqpoint{1.154531in}{2.309659in}}{\pgfqpoint{1.162431in}{2.312932in}}{\pgfqpoint{1.168255in}{2.318756in}}%
\pgfpathcurveto{\pgfqpoint{1.174079in}{2.324579in}}{\pgfqpoint{1.177352in}{2.332480in}}{\pgfqpoint{1.177352in}{2.340716in}}%
\pgfpathcurveto{\pgfqpoint{1.177352in}{2.348952in}}{\pgfqpoint{1.174079in}{2.356852in}}{\pgfqpoint{1.168255in}{2.362676in}}%
\pgfpathcurveto{\pgfqpoint{1.162431in}{2.368500in}}{\pgfqpoint{1.154531in}{2.371772in}}{\pgfqpoint{1.146295in}{2.371772in}}%
\pgfpathcurveto{\pgfqpoint{1.138059in}{2.371772in}}{\pgfqpoint{1.130159in}{2.368500in}}{\pgfqpoint{1.124335in}{2.362676in}}%
\pgfpathcurveto{\pgfqpoint{1.118511in}{2.356852in}}{\pgfqpoint{1.115239in}{2.348952in}}{\pgfqpoint{1.115239in}{2.340716in}}%
\pgfpathcurveto{\pgfqpoint{1.115239in}{2.332480in}}{\pgfqpoint{1.118511in}{2.324579in}}{\pgfqpoint{1.124335in}{2.318756in}}%
\pgfpathcurveto{\pgfqpoint{1.130159in}{2.312932in}}{\pgfqpoint{1.138059in}{2.309659in}}{\pgfqpoint{1.146295in}{2.309659in}}%
\pgfpathlineto{\pgfqpoint{1.146295in}{2.309659in}}%
\pgfpathclose%
\pgfusepath{stroke}%
\end{pgfscope}%
\begin{pgfscope}%
\pgfpathrectangle{\pgfqpoint{0.688192in}{0.670138in}}{\pgfqpoint{6.200000in}{4.620000in}}%
\pgfusepath{clip}%
\pgfsetbuttcap%
\pgfsetroundjoin%
\pgfsetlinewidth{1.003750pt}%
\definecolor{currentstroke}{rgb}{1.000000,0.000000,0.000000}%
\pgfsetstrokecolor{currentstroke}%
\pgfsetdash{}{0pt}%
\pgfpathmoveto{\pgfqpoint{1.283972in}{2.337286in}}%
\pgfpathcurveto{\pgfqpoint{1.292208in}{2.337286in}}{\pgfqpoint{1.300108in}{2.340558in}}{\pgfqpoint{1.305932in}{2.346382in}}%
\pgfpathcurveto{\pgfqpoint{1.311756in}{2.352206in}}{\pgfqpoint{1.315028in}{2.360106in}}{\pgfqpoint{1.315028in}{2.368343in}}%
\pgfpathcurveto{\pgfqpoint{1.315028in}{2.376579in}}{\pgfqpoint{1.311756in}{2.384479in}}{\pgfqpoint{1.305932in}{2.390303in}}%
\pgfpathcurveto{\pgfqpoint{1.300108in}{2.396127in}}{\pgfqpoint{1.292208in}{2.399399in}}{\pgfqpoint{1.283972in}{2.399399in}}%
\pgfpathcurveto{\pgfqpoint{1.275735in}{2.399399in}}{\pgfqpoint{1.267835in}{2.396127in}}{\pgfqpoint{1.262011in}{2.390303in}}%
\pgfpathcurveto{\pgfqpoint{1.256187in}{2.384479in}}{\pgfqpoint{1.252915in}{2.376579in}}{\pgfqpoint{1.252915in}{2.368343in}}%
\pgfpathcurveto{\pgfqpoint{1.252915in}{2.360106in}}{\pgfqpoint{1.256187in}{2.352206in}}{\pgfqpoint{1.262011in}{2.346382in}}%
\pgfpathcurveto{\pgfqpoint{1.267835in}{2.340558in}}{\pgfqpoint{1.275735in}{2.337286in}}{\pgfqpoint{1.283972in}{2.337286in}}%
\pgfpathlineto{\pgfqpoint{1.283972in}{2.337286in}}%
\pgfpathclose%
\pgfusepath{stroke}%
\end{pgfscope}%
\begin{pgfscope}%
\pgfpathrectangle{\pgfqpoint{0.688192in}{0.670138in}}{\pgfqpoint{6.200000in}{4.620000in}}%
\pgfusepath{clip}%
\pgfsetbuttcap%
\pgfsetroundjoin%
\pgfsetlinewidth{1.003750pt}%
\definecolor{currentstroke}{rgb}{1.000000,0.000000,0.000000}%
\pgfsetstrokecolor{currentstroke}%
\pgfsetdash{}{0pt}%
\pgfpathmoveto{\pgfqpoint{1.194044in}{2.459126in}}%
\pgfpathcurveto{\pgfqpoint{1.202281in}{2.459126in}}{\pgfqpoint{1.210181in}{2.462398in}}{\pgfqpoint{1.216005in}{2.468222in}}%
\pgfpathcurveto{\pgfqpoint{1.221828in}{2.474046in}}{\pgfqpoint{1.225101in}{2.481946in}}{\pgfqpoint{1.225101in}{2.490182in}}%
\pgfpathcurveto{\pgfqpoint{1.225101in}{2.498419in}}{\pgfqpoint{1.221828in}{2.506319in}}{\pgfqpoint{1.216005in}{2.512143in}}%
\pgfpathcurveto{\pgfqpoint{1.210181in}{2.517967in}}{\pgfqpoint{1.202281in}{2.521239in}}{\pgfqpoint{1.194044in}{2.521239in}}%
\pgfpathcurveto{\pgfqpoint{1.185808in}{2.521239in}}{\pgfqpoint{1.177908in}{2.517967in}}{\pgfqpoint{1.172084in}{2.512143in}}%
\pgfpathcurveto{\pgfqpoint{1.166260in}{2.506319in}}{\pgfqpoint{1.162988in}{2.498419in}}{\pgfqpoint{1.162988in}{2.490182in}}%
\pgfpathcurveto{\pgfqpoint{1.162988in}{2.481946in}}{\pgfqpoint{1.166260in}{2.474046in}}{\pgfqpoint{1.172084in}{2.468222in}}%
\pgfpathcurveto{\pgfqpoint{1.177908in}{2.462398in}}{\pgfqpoint{1.185808in}{2.459126in}}{\pgfqpoint{1.194044in}{2.459126in}}%
\pgfpathlineto{\pgfqpoint{1.194044in}{2.459126in}}%
\pgfpathclose%
\pgfusepath{stroke}%
\end{pgfscope}%
\begin{pgfscope}%
\pgfpathrectangle{\pgfqpoint{0.688192in}{0.670138in}}{\pgfqpoint{6.200000in}{4.620000in}}%
\pgfusepath{clip}%
\pgfsetbuttcap%
\pgfsetroundjoin%
\pgfsetlinewidth{1.003750pt}%
\definecolor{currentstroke}{rgb}{1.000000,0.000000,0.000000}%
\pgfsetstrokecolor{currentstroke}%
\pgfsetdash{}{0pt}%
\pgfpathmoveto{\pgfqpoint{1.126156in}{2.039611in}}%
\pgfpathcurveto{\pgfqpoint{1.134392in}{2.039611in}}{\pgfqpoint{1.142292in}{2.042884in}}{\pgfqpoint{1.148116in}{2.048708in}}%
\pgfpathcurveto{\pgfqpoint{1.153940in}{2.054532in}}{\pgfqpoint{1.157212in}{2.062432in}}{\pgfqpoint{1.157212in}{2.070668in}}%
\pgfpathcurveto{\pgfqpoint{1.157212in}{2.078904in}}{\pgfqpoint{1.153940in}{2.086804in}}{\pgfqpoint{1.148116in}{2.092628in}}%
\pgfpathcurveto{\pgfqpoint{1.142292in}{2.098452in}}{\pgfqpoint{1.134392in}{2.101724in}}{\pgfqpoint{1.126156in}{2.101724in}}%
\pgfpathcurveto{\pgfqpoint{1.117920in}{2.101724in}}{\pgfqpoint{1.110020in}{2.098452in}}{\pgfqpoint{1.104196in}{2.092628in}}%
\pgfpathcurveto{\pgfqpoint{1.098372in}{2.086804in}}{\pgfqpoint{1.095099in}{2.078904in}}{\pgfqpoint{1.095099in}{2.070668in}}%
\pgfpathcurveto{\pgfqpoint{1.095099in}{2.062432in}}{\pgfqpoint{1.098372in}{2.054532in}}{\pgfqpoint{1.104196in}{2.048708in}}%
\pgfpathcurveto{\pgfqpoint{1.110020in}{2.042884in}}{\pgfqpoint{1.117920in}{2.039611in}}{\pgfqpoint{1.126156in}{2.039611in}}%
\pgfpathlineto{\pgfqpoint{1.126156in}{2.039611in}}%
\pgfpathclose%
\pgfusepath{stroke}%
\end{pgfscope}%
\begin{pgfscope}%
\pgfpathrectangle{\pgfqpoint{0.688192in}{0.670138in}}{\pgfqpoint{6.200000in}{4.620000in}}%
\pgfusepath{clip}%
\pgfsetbuttcap%
\pgfsetroundjoin%
\pgfsetlinewidth{1.003750pt}%
\definecolor{currentstroke}{rgb}{1.000000,0.000000,0.000000}%
\pgfsetstrokecolor{currentstroke}%
\pgfsetdash{}{0pt}%
\pgfpathmoveto{\pgfqpoint{1.073783in}{1.736453in}}%
\pgfpathcurveto{\pgfqpoint{1.082020in}{1.736453in}}{\pgfqpoint{1.089920in}{1.739725in}}{\pgfqpoint{1.095744in}{1.745549in}}%
\pgfpathcurveto{\pgfqpoint{1.101568in}{1.751373in}}{\pgfqpoint{1.104840in}{1.759273in}}{\pgfqpoint{1.104840in}{1.767510in}}%
\pgfpathcurveto{\pgfqpoint{1.104840in}{1.775746in}}{\pgfqpoint{1.101568in}{1.783646in}}{\pgfqpoint{1.095744in}{1.789470in}}%
\pgfpathcurveto{\pgfqpoint{1.089920in}{1.795294in}}{\pgfqpoint{1.082020in}{1.798566in}}{\pgfqpoint{1.073783in}{1.798566in}}%
\pgfpathcurveto{\pgfqpoint{1.065547in}{1.798566in}}{\pgfqpoint{1.057647in}{1.795294in}}{\pgfqpoint{1.051823in}{1.789470in}}%
\pgfpathcurveto{\pgfqpoint{1.045999in}{1.783646in}}{\pgfqpoint{1.042727in}{1.775746in}}{\pgfqpoint{1.042727in}{1.767510in}}%
\pgfpathcurveto{\pgfqpoint{1.042727in}{1.759273in}}{\pgfqpoint{1.045999in}{1.751373in}}{\pgfqpoint{1.051823in}{1.745549in}}%
\pgfpathcurveto{\pgfqpoint{1.057647in}{1.739725in}}{\pgfqpoint{1.065547in}{1.736453in}}{\pgfqpoint{1.073783in}{1.736453in}}%
\pgfpathlineto{\pgfqpoint{1.073783in}{1.736453in}}%
\pgfpathclose%
\pgfusepath{stroke}%
\end{pgfscope}%
\begin{pgfscope}%
\pgfpathrectangle{\pgfqpoint{0.688192in}{0.670138in}}{\pgfqpoint{6.200000in}{4.620000in}}%
\pgfusepath{clip}%
\pgfsetbuttcap%
\pgfsetroundjoin%
\pgfsetlinewidth{1.003750pt}%
\definecolor{currentstroke}{rgb}{1.000000,0.000000,0.000000}%
\pgfsetstrokecolor{currentstroke}%
\pgfsetdash{}{0pt}%
\pgfpathmoveto{\pgfqpoint{1.071350in}{1.723563in}}%
\pgfpathcurveto{\pgfqpoint{1.079586in}{1.723563in}}{\pgfqpoint{1.087486in}{1.726835in}}{\pgfqpoint{1.093310in}{1.732659in}}%
\pgfpathcurveto{\pgfqpoint{1.099134in}{1.738483in}}{\pgfqpoint{1.102407in}{1.746383in}}{\pgfqpoint{1.102407in}{1.754620in}}%
\pgfpathcurveto{\pgfqpoint{1.102407in}{1.762856in}}{\pgfqpoint{1.099134in}{1.770756in}}{\pgfqpoint{1.093310in}{1.776580in}}%
\pgfpathcurveto{\pgfqpoint{1.087486in}{1.782404in}}{\pgfqpoint{1.079586in}{1.785676in}}{\pgfqpoint{1.071350in}{1.785676in}}%
\pgfpathcurveto{\pgfqpoint{1.063114in}{1.785676in}}{\pgfqpoint{1.055214in}{1.782404in}}{\pgfqpoint{1.049390in}{1.776580in}}%
\pgfpathcurveto{\pgfqpoint{1.043566in}{1.770756in}}{\pgfqpoint{1.040294in}{1.762856in}}{\pgfqpoint{1.040294in}{1.754620in}}%
\pgfpathcurveto{\pgfqpoint{1.040294in}{1.746383in}}{\pgfqpoint{1.043566in}{1.738483in}}{\pgfqpoint{1.049390in}{1.732659in}}%
\pgfpathcurveto{\pgfqpoint{1.055214in}{1.726835in}}{\pgfqpoint{1.063114in}{1.723563in}}{\pgfqpoint{1.071350in}{1.723563in}}%
\pgfpathlineto{\pgfqpoint{1.071350in}{1.723563in}}%
\pgfpathclose%
\pgfusepath{stroke}%
\end{pgfscope}%
\begin{pgfscope}%
\pgfpathrectangle{\pgfqpoint{0.688192in}{0.670138in}}{\pgfqpoint{6.200000in}{4.620000in}}%
\pgfusepath{clip}%
\pgfsetbuttcap%
\pgfsetroundjoin%
\pgfsetlinewidth{1.003750pt}%
\definecolor{currentstroke}{rgb}{1.000000,0.000000,0.000000}%
\pgfsetstrokecolor{currentstroke}%
\pgfsetdash{}{0pt}%
\pgfpathmoveto{\pgfqpoint{1.176638in}{2.187985in}}%
\pgfpathcurveto{\pgfqpoint{1.184874in}{2.187985in}}{\pgfqpoint{1.192774in}{2.191257in}}{\pgfqpoint{1.198598in}{2.197081in}}%
\pgfpathcurveto{\pgfqpoint{1.204422in}{2.202905in}}{\pgfqpoint{1.207694in}{2.210805in}}{\pgfqpoint{1.207694in}{2.219041in}}%
\pgfpathcurveto{\pgfqpoint{1.207694in}{2.227277in}}{\pgfqpoint{1.204422in}{2.235178in}}{\pgfqpoint{1.198598in}{2.241001in}}%
\pgfpathcurveto{\pgfqpoint{1.192774in}{2.246825in}}{\pgfqpoint{1.184874in}{2.250098in}}{\pgfqpoint{1.176638in}{2.250098in}}%
\pgfpathcurveto{\pgfqpoint{1.168401in}{2.250098in}}{\pgfqpoint{1.160501in}{2.246825in}}{\pgfqpoint{1.154677in}{2.241001in}}%
\pgfpathcurveto{\pgfqpoint{1.148853in}{2.235178in}}{\pgfqpoint{1.145581in}{2.227277in}}{\pgfqpoint{1.145581in}{2.219041in}}%
\pgfpathcurveto{\pgfqpoint{1.145581in}{2.210805in}}{\pgfqpoint{1.148853in}{2.202905in}}{\pgfqpoint{1.154677in}{2.197081in}}%
\pgfpathcurveto{\pgfqpoint{1.160501in}{2.191257in}}{\pgfqpoint{1.168401in}{2.187985in}}{\pgfqpoint{1.176638in}{2.187985in}}%
\pgfpathlineto{\pgfqpoint{1.176638in}{2.187985in}}%
\pgfpathclose%
\pgfusepath{stroke}%
\end{pgfscope}%
\begin{pgfscope}%
\pgfpathrectangle{\pgfqpoint{0.688192in}{0.670138in}}{\pgfqpoint{6.200000in}{4.620000in}}%
\pgfusepath{clip}%
\pgfsetbuttcap%
\pgfsetroundjoin%
\pgfsetlinewidth{1.003750pt}%
\definecolor{currentstroke}{rgb}{1.000000,0.000000,0.000000}%
\pgfsetstrokecolor{currentstroke}%
\pgfsetdash{}{0pt}%
\pgfpathmoveto{\pgfqpoint{1.160265in}{1.939369in}}%
\pgfpathcurveto{\pgfqpoint{1.168501in}{1.939369in}}{\pgfqpoint{1.176401in}{1.942641in}}{\pgfqpoint{1.182225in}{1.948465in}}%
\pgfpathcurveto{\pgfqpoint{1.188049in}{1.954289in}}{\pgfqpoint{1.191321in}{1.962189in}}{\pgfqpoint{1.191321in}{1.970426in}}%
\pgfpathcurveto{\pgfqpoint{1.191321in}{1.978662in}}{\pgfqpoint{1.188049in}{1.986562in}}{\pgfqpoint{1.182225in}{1.992386in}}%
\pgfpathcurveto{\pgfqpoint{1.176401in}{1.998210in}}{\pgfqpoint{1.168501in}{2.001482in}}{\pgfqpoint{1.160265in}{2.001482in}}%
\pgfpathcurveto{\pgfqpoint{1.152029in}{2.001482in}}{\pgfqpoint{1.144129in}{1.998210in}}{\pgfqpoint{1.138305in}{1.992386in}}%
\pgfpathcurveto{\pgfqpoint{1.132481in}{1.986562in}}{\pgfqpoint{1.129208in}{1.978662in}}{\pgfqpoint{1.129208in}{1.970426in}}%
\pgfpathcurveto{\pgfqpoint{1.129208in}{1.962189in}}{\pgfqpoint{1.132481in}{1.954289in}}{\pgfqpoint{1.138305in}{1.948465in}}%
\pgfpathcurveto{\pgfqpoint{1.144129in}{1.942641in}}{\pgfqpoint{1.152029in}{1.939369in}}{\pgfqpoint{1.160265in}{1.939369in}}%
\pgfpathlineto{\pgfqpoint{1.160265in}{1.939369in}}%
\pgfpathclose%
\pgfusepath{stroke}%
\end{pgfscope}%
\begin{pgfscope}%
\pgfpathrectangle{\pgfqpoint{0.688192in}{0.670138in}}{\pgfqpoint{6.200000in}{4.620000in}}%
\pgfusepath{clip}%
\pgfsetbuttcap%
\pgfsetroundjoin%
\pgfsetlinewidth{1.003750pt}%
\definecolor{currentstroke}{rgb}{1.000000,0.000000,0.000000}%
\pgfsetstrokecolor{currentstroke}%
\pgfsetdash{}{0pt}%
\pgfpathmoveto{\pgfqpoint{1.160053in}{1.935902in}}%
\pgfpathcurveto{\pgfqpoint{1.168289in}{1.935902in}}{\pgfqpoint{1.176189in}{1.939174in}}{\pgfqpoint{1.182013in}{1.944998in}}%
\pgfpathcurveto{\pgfqpoint{1.187837in}{1.950822in}}{\pgfqpoint{1.191109in}{1.958722in}}{\pgfqpoint{1.191109in}{1.966959in}}%
\pgfpathcurveto{\pgfqpoint{1.191109in}{1.975195in}}{\pgfqpoint{1.187837in}{1.983095in}}{\pgfqpoint{1.182013in}{1.988919in}}%
\pgfpathcurveto{\pgfqpoint{1.176189in}{1.994743in}}{\pgfqpoint{1.168289in}{1.998015in}}{\pgfqpoint{1.160053in}{1.998015in}}%
\pgfpathcurveto{\pgfqpoint{1.151817in}{1.998015in}}{\pgfqpoint{1.143916in}{1.994743in}}{\pgfqpoint{1.138093in}{1.988919in}}%
\pgfpathcurveto{\pgfqpoint{1.132269in}{1.983095in}}{\pgfqpoint{1.128996in}{1.975195in}}{\pgfqpoint{1.128996in}{1.966959in}}%
\pgfpathcurveto{\pgfqpoint{1.128996in}{1.958722in}}{\pgfqpoint{1.132269in}{1.950822in}}{\pgfqpoint{1.138093in}{1.944998in}}%
\pgfpathcurveto{\pgfqpoint{1.143916in}{1.939174in}}{\pgfqpoint{1.151817in}{1.935902in}}{\pgfqpoint{1.160053in}{1.935902in}}%
\pgfpathlineto{\pgfqpoint{1.160053in}{1.935902in}}%
\pgfpathclose%
\pgfusepath{stroke}%
\end{pgfscope}%
\begin{pgfscope}%
\pgfpathrectangle{\pgfqpoint{0.688192in}{0.670138in}}{\pgfqpoint{6.200000in}{4.620000in}}%
\pgfusepath{clip}%
\pgfsetbuttcap%
\pgfsetroundjoin%
\pgfsetlinewidth{1.003750pt}%
\definecolor{currentstroke}{rgb}{1.000000,0.000000,0.000000}%
\pgfsetstrokecolor{currentstroke}%
\pgfsetdash{}{0pt}%
\pgfpathmoveto{\pgfqpoint{1.157986in}{1.610198in}}%
\pgfpathcurveto{\pgfqpoint{1.166223in}{1.610198in}}{\pgfqpoint{1.174123in}{1.613470in}}{\pgfqpoint{1.179947in}{1.619294in}}%
\pgfpathcurveto{\pgfqpoint{1.185771in}{1.625118in}}{\pgfqpoint{1.189043in}{1.633018in}}{\pgfqpoint{1.189043in}{1.641254in}}%
\pgfpathcurveto{\pgfqpoint{1.189043in}{1.649491in}}{\pgfqpoint{1.185771in}{1.657391in}}{\pgfqpoint{1.179947in}{1.663215in}}%
\pgfpathcurveto{\pgfqpoint{1.174123in}{1.669038in}}{\pgfqpoint{1.166223in}{1.672311in}}{\pgfqpoint{1.157986in}{1.672311in}}%
\pgfpathcurveto{\pgfqpoint{1.149750in}{1.672311in}}{\pgfqpoint{1.141850in}{1.669038in}}{\pgfqpoint{1.136026in}{1.663215in}}%
\pgfpathcurveto{\pgfqpoint{1.130202in}{1.657391in}}{\pgfqpoint{1.126930in}{1.649491in}}{\pgfqpoint{1.126930in}{1.641254in}}%
\pgfpathcurveto{\pgfqpoint{1.126930in}{1.633018in}}{\pgfqpoint{1.130202in}{1.625118in}}{\pgfqpoint{1.136026in}{1.619294in}}%
\pgfpathcurveto{\pgfqpoint{1.141850in}{1.613470in}}{\pgfqpoint{1.149750in}{1.610198in}}{\pgfqpoint{1.157986in}{1.610198in}}%
\pgfpathlineto{\pgfqpoint{1.157986in}{1.610198in}}%
\pgfpathclose%
\pgfusepath{stroke}%
\end{pgfscope}%
\begin{pgfscope}%
\pgfpathrectangle{\pgfqpoint{0.688192in}{0.670138in}}{\pgfqpoint{6.200000in}{4.620000in}}%
\pgfusepath{clip}%
\pgfsetbuttcap%
\pgfsetroundjoin%
\pgfsetlinewidth{1.003750pt}%
\definecolor{currentstroke}{rgb}{1.000000,0.000000,0.000000}%
\pgfsetstrokecolor{currentstroke}%
\pgfsetdash{}{0pt}%
\pgfpathmoveto{\pgfqpoint{1.155798in}{1.612752in}}%
\pgfpathcurveto{\pgfqpoint{1.164035in}{1.612752in}}{\pgfqpoint{1.171935in}{1.616024in}}{\pgfqpoint{1.177759in}{1.621848in}}%
\pgfpathcurveto{\pgfqpoint{1.183582in}{1.627672in}}{\pgfqpoint{1.186855in}{1.635572in}}{\pgfqpoint{1.186855in}{1.643808in}}%
\pgfpathcurveto{\pgfqpoint{1.186855in}{1.652044in}}{\pgfqpoint{1.183582in}{1.659945in}}{\pgfqpoint{1.177759in}{1.665768in}}%
\pgfpathcurveto{\pgfqpoint{1.171935in}{1.671592in}}{\pgfqpoint{1.164035in}{1.674865in}}{\pgfqpoint{1.155798in}{1.674865in}}%
\pgfpathcurveto{\pgfqpoint{1.147562in}{1.674865in}}{\pgfqpoint{1.139662in}{1.671592in}}{\pgfqpoint{1.133838in}{1.665768in}}%
\pgfpathcurveto{\pgfqpoint{1.128014in}{1.659945in}}{\pgfqpoint{1.124742in}{1.652044in}}{\pgfqpoint{1.124742in}{1.643808in}}%
\pgfpathcurveto{\pgfqpoint{1.124742in}{1.635572in}}{\pgfqpoint{1.128014in}{1.627672in}}{\pgfqpoint{1.133838in}{1.621848in}}%
\pgfpathcurveto{\pgfqpoint{1.139662in}{1.616024in}}{\pgfqpoint{1.147562in}{1.612752in}}{\pgfqpoint{1.155798in}{1.612752in}}%
\pgfpathlineto{\pgfqpoint{1.155798in}{1.612752in}}%
\pgfpathclose%
\pgfusepath{stroke}%
\end{pgfscope}%
\begin{pgfscope}%
\pgfpathrectangle{\pgfqpoint{0.688192in}{0.670138in}}{\pgfqpoint{6.200000in}{4.620000in}}%
\pgfusepath{clip}%
\pgfsetbuttcap%
\pgfsetroundjoin%
\pgfsetlinewidth{1.003750pt}%
\definecolor{currentstroke}{rgb}{1.000000,0.000000,0.000000}%
\pgfsetstrokecolor{currentstroke}%
\pgfsetdash{}{0pt}%
\pgfpathmoveto{\pgfqpoint{1.295347in}{1.895177in}}%
\pgfpathcurveto{\pgfqpoint{1.303583in}{1.895177in}}{\pgfqpoint{1.311483in}{1.898449in}}{\pgfqpoint{1.317307in}{1.904273in}}%
\pgfpathcurveto{\pgfqpoint{1.323131in}{1.910097in}}{\pgfqpoint{1.326403in}{1.917997in}}{\pgfqpoint{1.326403in}{1.926233in}}%
\pgfpathcurveto{\pgfqpoint{1.326403in}{1.934470in}}{\pgfqpoint{1.323131in}{1.942370in}}{\pgfqpoint{1.317307in}{1.948193in}}%
\pgfpathcurveto{\pgfqpoint{1.311483in}{1.954017in}}{\pgfqpoint{1.303583in}{1.957290in}}{\pgfqpoint{1.295347in}{1.957290in}}%
\pgfpathcurveto{\pgfqpoint{1.287111in}{1.957290in}}{\pgfqpoint{1.279211in}{1.954017in}}{\pgfqpoint{1.273387in}{1.948193in}}%
\pgfpathcurveto{\pgfqpoint{1.267563in}{1.942370in}}{\pgfqpoint{1.264290in}{1.934470in}}{\pgfqpoint{1.264290in}{1.926233in}}%
\pgfpathcurveto{\pgfqpoint{1.264290in}{1.917997in}}{\pgfqpoint{1.267563in}{1.910097in}}{\pgfqpoint{1.273387in}{1.904273in}}%
\pgfpathcurveto{\pgfqpoint{1.279211in}{1.898449in}}{\pgfqpoint{1.287111in}{1.895177in}}{\pgfqpoint{1.295347in}{1.895177in}}%
\pgfpathlineto{\pgfqpoint{1.295347in}{1.895177in}}%
\pgfpathclose%
\pgfusepath{stroke}%
\end{pgfscope}%
\begin{pgfscope}%
\pgfpathrectangle{\pgfqpoint{0.688192in}{0.670138in}}{\pgfqpoint{6.200000in}{4.620000in}}%
\pgfusepath{clip}%
\pgfsetbuttcap%
\pgfsetroundjoin%
\pgfsetlinewidth{1.003750pt}%
\definecolor{currentstroke}{rgb}{1.000000,0.000000,0.000000}%
\pgfsetstrokecolor{currentstroke}%
\pgfsetdash{}{0pt}%
\pgfpathmoveto{\pgfqpoint{1.160535in}{1.595409in}}%
\pgfpathcurveto{\pgfqpoint{1.168772in}{1.595409in}}{\pgfqpoint{1.176672in}{1.598682in}}{\pgfqpoint{1.182496in}{1.604506in}}%
\pgfpathcurveto{\pgfqpoint{1.188320in}{1.610330in}}{\pgfqpoint{1.191592in}{1.618230in}}{\pgfqpoint{1.191592in}{1.626466in}}%
\pgfpathcurveto{\pgfqpoint{1.191592in}{1.634702in}}{\pgfqpoint{1.188320in}{1.642602in}}{\pgfqpoint{1.182496in}{1.648426in}}%
\pgfpathcurveto{\pgfqpoint{1.176672in}{1.654250in}}{\pgfqpoint{1.168772in}{1.657522in}}{\pgfqpoint{1.160535in}{1.657522in}}%
\pgfpathcurveto{\pgfqpoint{1.152299in}{1.657522in}}{\pgfqpoint{1.144399in}{1.654250in}}{\pgfqpoint{1.138575in}{1.648426in}}%
\pgfpathcurveto{\pgfqpoint{1.132751in}{1.642602in}}{\pgfqpoint{1.129479in}{1.634702in}}{\pgfqpoint{1.129479in}{1.626466in}}%
\pgfpathcurveto{\pgfqpoint{1.129479in}{1.618230in}}{\pgfqpoint{1.132751in}{1.610330in}}{\pgfqpoint{1.138575in}{1.604506in}}%
\pgfpathcurveto{\pgfqpoint{1.144399in}{1.598682in}}{\pgfqpoint{1.152299in}{1.595409in}}{\pgfqpoint{1.160535in}{1.595409in}}%
\pgfpathlineto{\pgfqpoint{1.160535in}{1.595409in}}%
\pgfpathclose%
\pgfusepath{stroke}%
\end{pgfscope}%
\begin{pgfscope}%
\pgfpathrectangle{\pgfqpoint{0.688192in}{0.670138in}}{\pgfqpoint{6.200000in}{4.620000in}}%
\pgfusepath{clip}%
\pgfsetbuttcap%
\pgfsetroundjoin%
\pgfsetlinewidth{1.003750pt}%
\definecolor{currentstroke}{rgb}{1.000000,0.000000,0.000000}%
\pgfsetstrokecolor{currentstroke}%
\pgfsetdash{}{0pt}%
\pgfpathmoveto{\pgfqpoint{1.265571in}{1.691299in}}%
\pgfpathcurveto{\pgfqpoint{1.273807in}{1.691299in}}{\pgfqpoint{1.281707in}{1.694572in}}{\pgfqpoint{1.287531in}{1.700396in}}%
\pgfpathcurveto{\pgfqpoint{1.293355in}{1.706220in}}{\pgfqpoint{1.296627in}{1.714120in}}{\pgfqpoint{1.296627in}{1.722356in}}%
\pgfpathcurveto{\pgfqpoint{1.296627in}{1.730592in}}{\pgfqpoint{1.293355in}{1.738492in}}{\pgfqpoint{1.287531in}{1.744316in}}%
\pgfpathcurveto{\pgfqpoint{1.281707in}{1.750140in}}{\pgfqpoint{1.273807in}{1.753412in}}{\pgfqpoint{1.265571in}{1.753412in}}%
\pgfpathcurveto{\pgfqpoint{1.257335in}{1.753412in}}{\pgfqpoint{1.249435in}{1.750140in}}{\pgfqpoint{1.243611in}{1.744316in}}%
\pgfpathcurveto{\pgfqpoint{1.237787in}{1.738492in}}{\pgfqpoint{1.234514in}{1.730592in}}{\pgfqpoint{1.234514in}{1.722356in}}%
\pgfpathcurveto{\pgfqpoint{1.234514in}{1.714120in}}{\pgfqpoint{1.237787in}{1.706220in}}{\pgfqpoint{1.243611in}{1.700396in}}%
\pgfpathcurveto{\pgfqpoint{1.249435in}{1.694572in}}{\pgfqpoint{1.257335in}{1.691299in}}{\pgfqpoint{1.265571in}{1.691299in}}%
\pgfpathlineto{\pgfqpoint{1.265571in}{1.691299in}}%
\pgfpathclose%
\pgfusepath{stroke}%
\end{pgfscope}%
\begin{pgfscope}%
\pgfpathrectangle{\pgfqpoint{0.688192in}{0.670138in}}{\pgfqpoint{6.200000in}{4.620000in}}%
\pgfusepath{clip}%
\pgfsetbuttcap%
\pgfsetroundjoin%
\pgfsetlinewidth{1.003750pt}%
\definecolor{currentstroke}{rgb}{1.000000,0.000000,0.000000}%
\pgfsetstrokecolor{currentstroke}%
\pgfsetdash{}{0pt}%
\pgfpathmoveto{\pgfqpoint{1.297728in}{1.636949in}}%
\pgfpathcurveto{\pgfqpoint{1.305964in}{1.636949in}}{\pgfqpoint{1.313864in}{1.640222in}}{\pgfqpoint{1.319688in}{1.646045in}}%
\pgfpathcurveto{\pgfqpoint{1.325512in}{1.651869in}}{\pgfqpoint{1.328784in}{1.659769in}}{\pgfqpoint{1.328784in}{1.668006in}}%
\pgfpathcurveto{\pgfqpoint{1.328784in}{1.676242in}}{\pgfqpoint{1.325512in}{1.684142in}}{\pgfqpoint{1.319688in}{1.689966in}}%
\pgfpathcurveto{\pgfqpoint{1.313864in}{1.695790in}}{\pgfqpoint{1.305964in}{1.699062in}}{\pgfqpoint{1.297728in}{1.699062in}}%
\pgfpathcurveto{\pgfqpoint{1.289491in}{1.699062in}}{\pgfqpoint{1.281591in}{1.695790in}}{\pgfqpoint{1.275767in}{1.689966in}}%
\pgfpathcurveto{\pgfqpoint{1.269943in}{1.684142in}}{\pgfqpoint{1.266671in}{1.676242in}}{\pgfqpoint{1.266671in}{1.668006in}}%
\pgfpathcurveto{\pgfqpoint{1.266671in}{1.659769in}}{\pgfqpoint{1.269943in}{1.651869in}}{\pgfqpoint{1.275767in}{1.646045in}}%
\pgfpathcurveto{\pgfqpoint{1.281591in}{1.640222in}}{\pgfqpoint{1.289491in}{1.636949in}}{\pgfqpoint{1.297728in}{1.636949in}}%
\pgfpathlineto{\pgfqpoint{1.297728in}{1.636949in}}%
\pgfpathclose%
\pgfusepath{stroke}%
\end{pgfscope}%
\begin{pgfscope}%
\pgfpathrectangle{\pgfqpoint{0.688192in}{0.670138in}}{\pgfqpoint{6.200000in}{4.620000in}}%
\pgfusepath{clip}%
\pgfsetbuttcap%
\pgfsetroundjoin%
\pgfsetlinewidth{1.003750pt}%
\definecolor{currentstroke}{rgb}{1.000000,0.000000,0.000000}%
\pgfsetstrokecolor{currentstroke}%
\pgfsetdash{}{0pt}%
\pgfpathmoveto{\pgfqpoint{1.147178in}{1.734556in}}%
\pgfpathcurveto{\pgfqpoint{1.155415in}{1.734556in}}{\pgfqpoint{1.163315in}{1.737828in}}{\pgfqpoint{1.169139in}{1.743652in}}%
\pgfpathcurveto{\pgfqpoint{1.174962in}{1.749476in}}{\pgfqpoint{1.178235in}{1.757376in}}{\pgfqpoint{1.178235in}{1.765612in}}%
\pgfpathcurveto{\pgfqpoint{1.178235in}{1.773848in}}{\pgfqpoint{1.174962in}{1.781749in}}{\pgfqpoint{1.169139in}{1.787572in}}%
\pgfpathcurveto{\pgfqpoint{1.163315in}{1.793396in}}{\pgfqpoint{1.155415in}{1.796669in}}{\pgfqpoint{1.147178in}{1.796669in}}%
\pgfpathcurveto{\pgfqpoint{1.138942in}{1.796669in}}{\pgfqpoint{1.131042in}{1.793396in}}{\pgfqpoint{1.125218in}{1.787572in}}%
\pgfpathcurveto{\pgfqpoint{1.119394in}{1.781749in}}{\pgfqpoint{1.116122in}{1.773848in}}{\pgfqpoint{1.116122in}{1.765612in}}%
\pgfpathcurveto{\pgfqpoint{1.116122in}{1.757376in}}{\pgfqpoint{1.119394in}{1.749476in}}{\pgfqpoint{1.125218in}{1.743652in}}%
\pgfpathcurveto{\pgfqpoint{1.131042in}{1.737828in}}{\pgfqpoint{1.138942in}{1.734556in}}{\pgfqpoint{1.147178in}{1.734556in}}%
\pgfpathlineto{\pgfqpoint{1.147178in}{1.734556in}}%
\pgfpathclose%
\pgfusepath{stroke}%
\end{pgfscope}%
\begin{pgfscope}%
\pgfpathrectangle{\pgfqpoint{0.688192in}{0.670138in}}{\pgfqpoint{6.200000in}{4.620000in}}%
\pgfusepath{clip}%
\pgfsetbuttcap%
\pgfsetroundjoin%
\pgfsetlinewidth{1.003750pt}%
\definecolor{currentstroke}{rgb}{1.000000,0.000000,0.000000}%
\pgfsetstrokecolor{currentstroke}%
\pgfsetdash{}{0pt}%
\pgfpathmoveto{\pgfqpoint{1.240097in}{1.549585in}}%
\pgfpathcurveto{\pgfqpoint{1.248334in}{1.549585in}}{\pgfqpoint{1.256234in}{1.552857in}}{\pgfqpoint{1.262058in}{1.558681in}}%
\pgfpathcurveto{\pgfqpoint{1.267881in}{1.564505in}}{\pgfqpoint{1.271154in}{1.572405in}}{\pgfqpoint{1.271154in}{1.580642in}}%
\pgfpathcurveto{\pgfqpoint{1.271154in}{1.588878in}}{\pgfqpoint{1.267881in}{1.596778in}}{\pgfqpoint{1.262058in}{1.602602in}}%
\pgfpathcurveto{\pgfqpoint{1.256234in}{1.608426in}}{\pgfqpoint{1.248334in}{1.611698in}}{\pgfqpoint{1.240097in}{1.611698in}}%
\pgfpathcurveto{\pgfqpoint{1.231861in}{1.611698in}}{\pgfqpoint{1.223961in}{1.608426in}}{\pgfqpoint{1.218137in}{1.602602in}}%
\pgfpathcurveto{\pgfqpoint{1.212313in}{1.596778in}}{\pgfqpoint{1.209041in}{1.588878in}}{\pgfqpoint{1.209041in}{1.580642in}}%
\pgfpathcurveto{\pgfqpoint{1.209041in}{1.572405in}}{\pgfqpoint{1.212313in}{1.564505in}}{\pgfqpoint{1.218137in}{1.558681in}}%
\pgfpathcurveto{\pgfqpoint{1.223961in}{1.552857in}}{\pgfqpoint{1.231861in}{1.549585in}}{\pgfqpoint{1.240097in}{1.549585in}}%
\pgfpathlineto{\pgfqpoint{1.240097in}{1.549585in}}%
\pgfpathclose%
\pgfusepath{stroke}%
\end{pgfscope}%
\begin{pgfscope}%
\pgfpathrectangle{\pgfqpoint{0.688192in}{0.670138in}}{\pgfqpoint{6.200000in}{4.620000in}}%
\pgfusepath{clip}%
\pgfsetbuttcap%
\pgfsetroundjoin%
\pgfsetlinewidth{1.003750pt}%
\definecolor{currentstroke}{rgb}{1.000000,0.000000,0.000000}%
\pgfsetstrokecolor{currentstroke}%
\pgfsetdash{}{0pt}%
\pgfpathmoveto{\pgfqpoint{1.275017in}{1.474367in}}%
\pgfpathcurveto{\pgfqpoint{1.283253in}{1.474367in}}{\pgfqpoint{1.291153in}{1.477639in}}{\pgfqpoint{1.296977in}{1.483463in}}%
\pgfpathcurveto{\pgfqpoint{1.302801in}{1.489287in}}{\pgfqpoint{1.306073in}{1.497187in}}{\pgfqpoint{1.306073in}{1.505423in}}%
\pgfpathcurveto{\pgfqpoint{1.306073in}{1.513660in}}{\pgfqpoint{1.302801in}{1.521560in}}{\pgfqpoint{1.296977in}{1.527384in}}%
\pgfpathcurveto{\pgfqpoint{1.291153in}{1.533208in}}{\pgfqpoint{1.283253in}{1.536480in}}{\pgfqpoint{1.275017in}{1.536480in}}%
\pgfpathcurveto{\pgfqpoint{1.266780in}{1.536480in}}{\pgfqpoint{1.258880in}{1.533208in}}{\pgfqpoint{1.253056in}{1.527384in}}%
\pgfpathcurveto{\pgfqpoint{1.247232in}{1.521560in}}{\pgfqpoint{1.243960in}{1.513660in}}{\pgfqpoint{1.243960in}{1.505423in}}%
\pgfpathcurveto{\pgfqpoint{1.243960in}{1.497187in}}{\pgfqpoint{1.247232in}{1.489287in}}{\pgfqpoint{1.253056in}{1.483463in}}%
\pgfpathcurveto{\pgfqpoint{1.258880in}{1.477639in}}{\pgfqpoint{1.266780in}{1.474367in}}{\pgfqpoint{1.275017in}{1.474367in}}%
\pgfpathlineto{\pgfqpoint{1.275017in}{1.474367in}}%
\pgfpathclose%
\pgfusepath{stroke}%
\end{pgfscope}%
\begin{pgfscope}%
\pgfpathrectangle{\pgfqpoint{0.688192in}{0.670138in}}{\pgfqpoint{6.200000in}{4.620000in}}%
\pgfusepath{clip}%
\pgfsetbuttcap%
\pgfsetroundjoin%
\pgfsetlinewidth{1.003750pt}%
\definecolor{currentstroke}{rgb}{1.000000,0.000000,0.000000}%
\pgfsetstrokecolor{currentstroke}%
\pgfsetdash{}{0pt}%
\pgfpathmoveto{\pgfqpoint{1.274070in}{1.506532in}}%
\pgfpathcurveto{\pgfqpoint{1.282306in}{1.506532in}}{\pgfqpoint{1.290206in}{1.509804in}}{\pgfqpoint{1.296030in}{1.515628in}}%
\pgfpathcurveto{\pgfqpoint{1.301854in}{1.521452in}}{\pgfqpoint{1.305126in}{1.529352in}}{\pgfqpoint{1.305126in}{1.537589in}}%
\pgfpathcurveto{\pgfqpoint{1.305126in}{1.545825in}}{\pgfqpoint{1.301854in}{1.553725in}}{\pgfqpoint{1.296030in}{1.559549in}}%
\pgfpathcurveto{\pgfqpoint{1.290206in}{1.565373in}}{\pgfqpoint{1.282306in}{1.568645in}}{\pgfqpoint{1.274070in}{1.568645in}}%
\pgfpathcurveto{\pgfqpoint{1.265833in}{1.568645in}}{\pgfqpoint{1.257933in}{1.565373in}}{\pgfqpoint{1.252109in}{1.559549in}}%
\pgfpathcurveto{\pgfqpoint{1.246285in}{1.553725in}}{\pgfqpoint{1.243013in}{1.545825in}}{\pgfqpoint{1.243013in}{1.537589in}}%
\pgfpathcurveto{\pgfqpoint{1.243013in}{1.529352in}}{\pgfqpoint{1.246285in}{1.521452in}}{\pgfqpoint{1.252109in}{1.515628in}}%
\pgfpathcurveto{\pgfqpoint{1.257933in}{1.509804in}}{\pgfqpoint{1.265833in}{1.506532in}}{\pgfqpoint{1.274070in}{1.506532in}}%
\pgfpathlineto{\pgfqpoint{1.274070in}{1.506532in}}%
\pgfpathclose%
\pgfusepath{stroke}%
\end{pgfscope}%
\begin{pgfscope}%
\pgfpathrectangle{\pgfqpoint{0.688192in}{0.670138in}}{\pgfqpoint{6.200000in}{4.620000in}}%
\pgfusepath{clip}%
\pgfsetbuttcap%
\pgfsetroundjoin%
\pgfsetlinewidth{1.003750pt}%
\definecolor{currentstroke}{rgb}{1.000000,0.000000,0.000000}%
\pgfsetstrokecolor{currentstroke}%
\pgfsetdash{}{0pt}%
\pgfpathmoveto{\pgfqpoint{1.147538in}{1.722969in}}%
\pgfpathcurveto{\pgfqpoint{1.155775in}{1.722969in}}{\pgfqpoint{1.163675in}{1.726242in}}{\pgfqpoint{1.169499in}{1.732066in}}%
\pgfpathcurveto{\pgfqpoint{1.175323in}{1.737890in}}{\pgfqpoint{1.178595in}{1.745790in}}{\pgfqpoint{1.178595in}{1.754026in}}%
\pgfpathcurveto{\pgfqpoint{1.178595in}{1.762262in}}{\pgfqpoint{1.175323in}{1.770162in}}{\pgfqpoint{1.169499in}{1.775986in}}%
\pgfpathcurveto{\pgfqpoint{1.163675in}{1.781810in}}{\pgfqpoint{1.155775in}{1.785082in}}{\pgfqpoint{1.147538in}{1.785082in}}%
\pgfpathcurveto{\pgfqpoint{1.139302in}{1.785082in}}{\pgfqpoint{1.131402in}{1.781810in}}{\pgfqpoint{1.125578in}{1.775986in}}%
\pgfpathcurveto{\pgfqpoint{1.119754in}{1.770162in}}{\pgfqpoint{1.116482in}{1.762262in}}{\pgfqpoint{1.116482in}{1.754026in}}%
\pgfpathcurveto{\pgfqpoint{1.116482in}{1.745790in}}{\pgfqpoint{1.119754in}{1.737890in}}{\pgfqpoint{1.125578in}{1.732066in}}%
\pgfpathcurveto{\pgfqpoint{1.131402in}{1.726242in}}{\pgfqpoint{1.139302in}{1.722969in}}{\pgfqpoint{1.147538in}{1.722969in}}%
\pgfpathlineto{\pgfqpoint{1.147538in}{1.722969in}}%
\pgfpathclose%
\pgfusepath{stroke}%
\end{pgfscope}%
\begin{pgfscope}%
\pgfpathrectangle{\pgfqpoint{0.688192in}{0.670138in}}{\pgfqpoint{6.200000in}{4.620000in}}%
\pgfusepath{clip}%
\pgfsetbuttcap%
\pgfsetroundjoin%
\pgfsetlinewidth{1.003750pt}%
\definecolor{currentstroke}{rgb}{1.000000,0.000000,0.000000}%
\pgfsetstrokecolor{currentstroke}%
\pgfsetdash{}{0pt}%
\pgfpathmoveto{\pgfqpoint{1.194960in}{1.895058in}}%
\pgfpathcurveto{\pgfqpoint{1.203197in}{1.895058in}}{\pgfqpoint{1.211097in}{1.898330in}}{\pgfqpoint{1.216921in}{1.904154in}}%
\pgfpathcurveto{\pgfqpoint{1.222745in}{1.909978in}}{\pgfqpoint{1.226017in}{1.917878in}}{\pgfqpoint{1.226017in}{1.926114in}}%
\pgfpathcurveto{\pgfqpoint{1.226017in}{1.934351in}}{\pgfqpoint{1.222745in}{1.942251in}}{\pgfqpoint{1.216921in}{1.948075in}}%
\pgfpathcurveto{\pgfqpoint{1.211097in}{1.953898in}}{\pgfqpoint{1.203197in}{1.957171in}}{\pgfqpoint{1.194960in}{1.957171in}}%
\pgfpathcurveto{\pgfqpoint{1.186724in}{1.957171in}}{\pgfqpoint{1.178824in}{1.953898in}}{\pgfqpoint{1.173000in}{1.948075in}}%
\pgfpathcurveto{\pgfqpoint{1.167176in}{1.942251in}}{\pgfqpoint{1.163904in}{1.934351in}}{\pgfqpoint{1.163904in}{1.926114in}}%
\pgfpathcurveto{\pgfqpoint{1.163904in}{1.917878in}}{\pgfqpoint{1.167176in}{1.909978in}}{\pgfqpoint{1.173000in}{1.904154in}}%
\pgfpathcurveto{\pgfqpoint{1.178824in}{1.898330in}}{\pgfqpoint{1.186724in}{1.895058in}}{\pgfqpoint{1.194960in}{1.895058in}}%
\pgfpathlineto{\pgfqpoint{1.194960in}{1.895058in}}%
\pgfpathclose%
\pgfusepath{stroke}%
\end{pgfscope}%
\begin{pgfscope}%
\pgfpathrectangle{\pgfqpoint{0.688192in}{0.670138in}}{\pgfqpoint{6.200000in}{4.620000in}}%
\pgfusepath{clip}%
\pgfsetbuttcap%
\pgfsetroundjoin%
\pgfsetlinewidth{1.003750pt}%
\definecolor{currentstroke}{rgb}{1.000000,0.000000,0.000000}%
\pgfsetstrokecolor{currentstroke}%
\pgfsetdash{}{0pt}%
\pgfpathmoveto{\pgfqpoint{1.164778in}{1.905975in}}%
\pgfpathcurveto{\pgfqpoint{1.173015in}{1.905975in}}{\pgfqpoint{1.180915in}{1.909248in}}{\pgfqpoint{1.186738in}{1.915072in}}%
\pgfpathcurveto{\pgfqpoint{1.192562in}{1.920896in}}{\pgfqpoint{1.195835in}{1.928796in}}{\pgfqpoint{1.195835in}{1.937032in}}%
\pgfpathcurveto{\pgfqpoint{1.195835in}{1.945268in}}{\pgfqpoint{1.192562in}{1.953168in}}{\pgfqpoint{1.186738in}{1.958992in}}%
\pgfpathcurveto{\pgfqpoint{1.180915in}{1.964816in}}{\pgfqpoint{1.173015in}{1.968088in}}{\pgfqpoint{1.164778in}{1.968088in}}%
\pgfpathcurveto{\pgfqpoint{1.156542in}{1.968088in}}{\pgfqpoint{1.148642in}{1.964816in}}{\pgfqpoint{1.142818in}{1.958992in}}%
\pgfpathcurveto{\pgfqpoint{1.136994in}{1.953168in}}{\pgfqpoint{1.133722in}{1.945268in}}{\pgfqpoint{1.133722in}{1.937032in}}%
\pgfpathcurveto{\pgfqpoint{1.133722in}{1.928796in}}{\pgfqpoint{1.136994in}{1.920896in}}{\pgfqpoint{1.142818in}{1.915072in}}%
\pgfpathcurveto{\pgfqpoint{1.148642in}{1.909248in}}{\pgfqpoint{1.156542in}{1.905975in}}{\pgfqpoint{1.164778in}{1.905975in}}%
\pgfpathlineto{\pgfqpoint{1.164778in}{1.905975in}}%
\pgfpathclose%
\pgfusepath{stroke}%
\end{pgfscope}%
\begin{pgfscope}%
\pgfpathrectangle{\pgfqpoint{0.688192in}{0.670138in}}{\pgfqpoint{6.200000in}{4.620000in}}%
\pgfusepath{clip}%
\pgfsetbuttcap%
\pgfsetroundjoin%
\pgfsetlinewidth{1.003750pt}%
\definecolor{currentstroke}{rgb}{1.000000,0.000000,0.000000}%
\pgfsetstrokecolor{currentstroke}%
\pgfsetdash{}{0pt}%
\pgfpathmoveto{\pgfqpoint{1.143899in}{1.731794in}}%
\pgfpathcurveto{\pgfqpoint{1.152135in}{1.731794in}}{\pgfqpoint{1.160035in}{1.735067in}}{\pgfqpoint{1.165859in}{1.740890in}}%
\pgfpathcurveto{\pgfqpoint{1.171683in}{1.746714in}}{\pgfqpoint{1.174956in}{1.754614in}}{\pgfqpoint{1.174956in}{1.762851in}}%
\pgfpathcurveto{\pgfqpoint{1.174956in}{1.771087in}}{\pgfqpoint{1.171683in}{1.778987in}}{\pgfqpoint{1.165859in}{1.784811in}}%
\pgfpathcurveto{\pgfqpoint{1.160035in}{1.790635in}}{\pgfqpoint{1.152135in}{1.793907in}}{\pgfqpoint{1.143899in}{1.793907in}}%
\pgfpathcurveto{\pgfqpoint{1.135663in}{1.793907in}}{\pgfqpoint{1.127763in}{1.790635in}}{\pgfqpoint{1.121939in}{1.784811in}}%
\pgfpathcurveto{\pgfqpoint{1.116115in}{1.778987in}}{\pgfqpoint{1.112843in}{1.771087in}}{\pgfqpoint{1.112843in}{1.762851in}}%
\pgfpathcurveto{\pgfqpoint{1.112843in}{1.754614in}}{\pgfqpoint{1.116115in}{1.746714in}}{\pgfqpoint{1.121939in}{1.740890in}}%
\pgfpathcurveto{\pgfqpoint{1.127763in}{1.735067in}}{\pgfqpoint{1.135663in}{1.731794in}}{\pgfqpoint{1.143899in}{1.731794in}}%
\pgfpathlineto{\pgfqpoint{1.143899in}{1.731794in}}%
\pgfpathclose%
\pgfusepath{stroke}%
\end{pgfscope}%
\begin{pgfscope}%
\pgfpathrectangle{\pgfqpoint{0.688192in}{0.670138in}}{\pgfqpoint{6.200000in}{4.620000in}}%
\pgfusepath{clip}%
\pgfsetbuttcap%
\pgfsetroundjoin%
\pgfsetlinewidth{1.003750pt}%
\definecolor{currentstroke}{rgb}{1.000000,0.000000,0.000000}%
\pgfsetstrokecolor{currentstroke}%
\pgfsetdash{}{0pt}%
\pgfpathmoveto{\pgfqpoint{0.688192in}{1.225706in}}%
\pgfpathcurveto{\pgfqpoint{0.696428in}{1.225706in}}{\pgfqpoint{0.704328in}{1.228978in}}{\pgfqpoint{0.710152in}{1.234802in}}%
\pgfpathcurveto{\pgfqpoint{0.715976in}{1.240626in}}{\pgfqpoint{0.719248in}{1.248526in}}{\pgfqpoint{0.719248in}{1.256762in}}%
\pgfpathcurveto{\pgfqpoint{0.719248in}{1.264998in}}{\pgfqpoint{0.715976in}{1.272898in}}{\pgfqpoint{0.710152in}{1.278722in}}%
\pgfpathcurveto{\pgfqpoint{0.704328in}{1.284546in}}{\pgfqpoint{0.696428in}{1.287819in}}{\pgfqpoint{0.688192in}{1.287819in}}%
\pgfpathcurveto{\pgfqpoint{0.679955in}{1.287819in}}{\pgfqpoint{0.672055in}{1.284546in}}{\pgfqpoint{0.666231in}{1.278722in}}%
\pgfpathcurveto{\pgfqpoint{0.660407in}{1.272898in}}{\pgfqpoint{0.657135in}{1.264998in}}{\pgfqpoint{0.657135in}{1.256762in}}%
\pgfpathcurveto{\pgfqpoint{0.657135in}{1.248526in}}{\pgfqpoint{0.660407in}{1.240626in}}{\pgfqpoint{0.666231in}{1.234802in}}%
\pgfpathcurveto{\pgfqpoint{0.672055in}{1.228978in}}{\pgfqpoint{0.679955in}{1.225706in}}{\pgfqpoint{0.688192in}{1.225706in}}%
\pgfpathlineto{\pgfqpoint{0.688192in}{1.225706in}}%
\pgfpathclose%
\pgfusepath{stroke}%
\end{pgfscope}%
\begin{pgfscope}%
\pgfpathrectangle{\pgfqpoint{0.688192in}{0.670138in}}{\pgfqpoint{6.200000in}{4.620000in}}%
\pgfusepath{clip}%
\pgfsetbuttcap%
\pgfsetmiterjoin%
\definecolor{currentfill}{rgb}{0.839216,0.152941,0.156863}%
\pgfsetfillcolor{currentfill}%
\pgfsetfillopacity{0.200000}%
\pgfsetlinewidth{1.003750pt}%
\definecolor{currentstroke}{rgb}{0.839216,0.152941,0.156863}%
\pgfsetstrokecolor{currentstroke}%
\pgfsetstrokeopacity{0.200000}%
\pgfsetdash{}{0pt}%
\pgfpathmoveto{\pgfqpoint{0.688192in}{0.670138in}}%
\pgfpathlineto{\pgfqpoint{1.790122in}{0.670138in}}%
\pgfpathlineto{\pgfqpoint{1.790122in}{5.290138in}}%
\pgfpathlineto{\pgfqpoint{0.688192in}{5.290138in}}%
\pgfpathlineto{\pgfqpoint{0.688192in}{0.670138in}}%
\pgfpathclose%
\pgfusepath{stroke,fill}%
\end{pgfscope}%
\begin{pgfscope}%
\pgfsetbuttcap%
\pgfsetmiterjoin%
\definecolor{currentfill}{rgb}{0.839216,0.152941,0.156863}%
\pgfsetfillcolor{currentfill}%
\pgfsetfillopacity{0.200000}%
\pgfsetlinewidth{1.003750pt}%
\definecolor{currentstroke}{rgb}{0.839216,0.152941,0.156863}%
\pgfsetstrokecolor{currentstroke}%
\pgfsetstrokeopacity{0.200000}%
\pgfsetdash{}{0pt}%
\pgfpathrectangle{\pgfqpoint{0.688192in}{0.670138in}}{\pgfqpoint{6.200000in}{4.620000in}}%
\pgfusepath{clip}%
\pgfpathmoveto{\pgfqpoint{0.688192in}{0.670138in}}%
\pgfpathlineto{\pgfqpoint{1.790122in}{0.670138in}}%
\pgfpathlineto{\pgfqpoint{1.790122in}{5.290138in}}%
\pgfpathlineto{\pgfqpoint{0.688192in}{5.290138in}}%
\pgfpathlineto{\pgfqpoint{0.688192in}{0.670138in}}%
\pgfpathclose%
\pgfusepath{clip}%
\pgfsys@defobject{currentpattern}{\pgfqpoint{0in}{0in}}{\pgfqpoint{1in}{1in}}{%
\begin{pgfscope}%
\pgfpathrectangle{\pgfqpoint{0in}{0in}}{\pgfqpoint{1in}{1in}}%
\pgfusepath{clip}%
\pgfpathmoveto{\pgfqpoint{-0.500000in}{0.500000in}}%
\pgfpathlineto{\pgfqpoint{0.500000in}{1.500000in}}%
\pgfpathmoveto{\pgfqpoint{-0.333333in}{0.333333in}}%
\pgfpathlineto{\pgfqpoint{0.666667in}{1.333333in}}%
\pgfpathmoveto{\pgfqpoint{-0.166667in}{0.166667in}}%
\pgfpathlineto{\pgfqpoint{0.833333in}{1.166667in}}%
\pgfpathmoveto{\pgfqpoint{0.000000in}{0.000000in}}%
\pgfpathlineto{\pgfqpoint{1.000000in}{1.000000in}}%
\pgfpathmoveto{\pgfqpoint{0.166667in}{-0.166667in}}%
\pgfpathlineto{\pgfqpoint{1.166667in}{0.833333in}}%
\pgfpathmoveto{\pgfqpoint{0.333333in}{-0.333333in}}%
\pgfpathlineto{\pgfqpoint{1.333333in}{0.666667in}}%
\pgfpathmoveto{\pgfqpoint{0.500000in}{-0.500000in}}%
\pgfpathlineto{\pgfqpoint{1.500000in}{0.500000in}}%
\pgfusepath{stroke}%
\end{pgfscope}%
}%
\pgfsys@transformshift{0.688192in}{0.670138in}%
\pgfsys@useobject{currentpattern}{}%
\pgfsys@transformshift{1in}{0in}%
\pgfsys@useobject{currentpattern}{}%
\pgfsys@transformshift{1in}{0in}%
\pgfsys@transformshift{-2in}{0in}%
\pgfsys@transformshift{0in}{1in}%
\pgfsys@useobject{currentpattern}{}%
\pgfsys@transformshift{1in}{0in}%
\pgfsys@useobject{currentpattern}{}%
\pgfsys@transformshift{1in}{0in}%
\pgfsys@transformshift{-2in}{0in}%
\pgfsys@transformshift{0in}{1in}%
\pgfsys@useobject{currentpattern}{}%
\pgfsys@transformshift{1in}{0in}%
\pgfsys@useobject{currentpattern}{}%
\pgfsys@transformshift{1in}{0in}%
\pgfsys@transformshift{-2in}{0in}%
\pgfsys@transformshift{0in}{1in}%
\pgfsys@useobject{currentpattern}{}%
\pgfsys@transformshift{1in}{0in}%
\pgfsys@useobject{currentpattern}{}%
\pgfsys@transformshift{1in}{0in}%
\pgfsys@transformshift{-2in}{0in}%
\pgfsys@transformshift{0in}{1in}%
\pgfsys@useobject{currentpattern}{}%
\pgfsys@transformshift{1in}{0in}%
\pgfsys@useobject{currentpattern}{}%
\pgfsys@transformshift{1in}{0in}%
\pgfsys@transformshift{-2in}{0in}%
\pgfsys@transformshift{0in}{1in}%
\end{pgfscope}%
\begin{pgfscope}%
\pgfpathrectangle{\pgfqpoint{0.688192in}{0.670138in}}{\pgfqpoint{6.200000in}{4.620000in}}%
\pgfusepath{clip}%
\pgfsetrectcap%
\pgfsetroundjoin%
\pgfsetlinewidth{0.803000pt}%
\definecolor{currentstroke}{rgb}{0.690196,0.690196,0.690196}%
\pgfsetstrokecolor{currentstroke}%
\pgfsetdash{}{0pt}%
\pgfpathmoveto{\pgfqpoint{1.122474in}{0.670138in}}%
\pgfpathlineto{\pgfqpoint{1.122474in}{5.290138in}}%
\pgfusepath{stroke}%
\end{pgfscope}%
\begin{pgfscope}%
\pgfsetbuttcap%
\pgfsetroundjoin%
\definecolor{currentfill}{rgb}{0.000000,0.000000,0.000000}%
\pgfsetfillcolor{currentfill}%
\pgfsetlinewidth{0.803000pt}%
\definecolor{currentstroke}{rgb}{0.000000,0.000000,0.000000}%
\pgfsetstrokecolor{currentstroke}%
\pgfsetdash{}{0pt}%
\pgfsys@defobject{currentmarker}{\pgfqpoint{0.000000in}{-0.048611in}}{\pgfqpoint{0.000000in}{0.000000in}}{%
\pgfpathmoveto{\pgfqpoint{0.000000in}{0.000000in}}%
\pgfpathlineto{\pgfqpoint{0.000000in}{-0.048611in}}%
\pgfusepath{stroke,fill}%
}%
\begin{pgfscope}%
\pgfsys@transformshift{1.122474in}{0.670138in}%
\pgfsys@useobject{currentmarker}{}%
\end{pgfscope}%
\end{pgfscope}%
\begin{pgfscope}%
\definecolor{textcolor}{rgb}{0.000000,0.000000,0.000000}%
\pgfsetstrokecolor{textcolor}%
\pgfsetfillcolor{textcolor}%
\pgftext[x=1.122474in,y=0.572916in,,top]{\color{textcolor}{\rmfamily\fontsize{14.000000}{16.800000}\selectfont\catcode`\^=\active\def^{\ifmmode\sp\else\^{}\fi}\catcode`\%=\active\def%{\%}$\mathdefault{5500}$}}%
\end{pgfscope}%
\begin{pgfscope}%
\pgfpathrectangle{\pgfqpoint{0.688192in}{0.670138in}}{\pgfqpoint{6.200000in}{4.620000in}}%
\pgfusepath{clip}%
\pgfsetrectcap%
\pgfsetroundjoin%
\pgfsetlinewidth{0.803000pt}%
\definecolor{currentstroke}{rgb}{0.690196,0.690196,0.690196}%
\pgfsetstrokecolor{currentstroke}%
\pgfsetdash{}{0pt}%
\pgfpathmoveto{\pgfqpoint{2.163709in}{0.670138in}}%
\pgfpathlineto{\pgfqpoint{2.163709in}{5.290138in}}%
\pgfusepath{stroke}%
\end{pgfscope}%
\begin{pgfscope}%
\pgfsetbuttcap%
\pgfsetroundjoin%
\definecolor{currentfill}{rgb}{0.000000,0.000000,0.000000}%
\pgfsetfillcolor{currentfill}%
\pgfsetlinewidth{0.803000pt}%
\definecolor{currentstroke}{rgb}{0.000000,0.000000,0.000000}%
\pgfsetstrokecolor{currentstroke}%
\pgfsetdash{}{0pt}%
\pgfsys@defobject{currentmarker}{\pgfqpoint{0.000000in}{-0.048611in}}{\pgfqpoint{0.000000in}{0.000000in}}{%
\pgfpathmoveto{\pgfqpoint{0.000000in}{0.000000in}}%
\pgfpathlineto{\pgfqpoint{0.000000in}{-0.048611in}}%
\pgfusepath{stroke,fill}%
}%
\begin{pgfscope}%
\pgfsys@transformshift{2.163709in}{0.670138in}%
\pgfsys@useobject{currentmarker}{}%
\end{pgfscope}%
\end{pgfscope}%
\begin{pgfscope}%
\definecolor{textcolor}{rgb}{0.000000,0.000000,0.000000}%
\pgfsetstrokecolor{textcolor}%
\pgfsetfillcolor{textcolor}%
\pgftext[x=2.163709in,y=0.572916in,,top]{\color{textcolor}{\rmfamily\fontsize{14.000000}{16.800000}\selectfont\catcode`\^=\active\def^{\ifmmode\sp\else\^{}\fi}\catcode`\%=\active\def%{\%}$\mathdefault{6000}$}}%
\end{pgfscope}%
\begin{pgfscope}%
\pgfpathrectangle{\pgfqpoint{0.688192in}{0.670138in}}{\pgfqpoint{6.200000in}{4.620000in}}%
\pgfusepath{clip}%
\pgfsetrectcap%
\pgfsetroundjoin%
\pgfsetlinewidth{0.803000pt}%
\definecolor{currentstroke}{rgb}{0.690196,0.690196,0.690196}%
\pgfsetstrokecolor{currentstroke}%
\pgfsetdash{}{0pt}%
\pgfpathmoveto{\pgfqpoint{3.204944in}{0.670138in}}%
\pgfpathlineto{\pgfqpoint{3.204944in}{5.290138in}}%
\pgfusepath{stroke}%
\end{pgfscope}%
\begin{pgfscope}%
\pgfsetbuttcap%
\pgfsetroundjoin%
\definecolor{currentfill}{rgb}{0.000000,0.000000,0.000000}%
\pgfsetfillcolor{currentfill}%
\pgfsetlinewidth{0.803000pt}%
\definecolor{currentstroke}{rgb}{0.000000,0.000000,0.000000}%
\pgfsetstrokecolor{currentstroke}%
\pgfsetdash{}{0pt}%
\pgfsys@defobject{currentmarker}{\pgfqpoint{0.000000in}{-0.048611in}}{\pgfqpoint{0.000000in}{0.000000in}}{%
\pgfpathmoveto{\pgfqpoint{0.000000in}{0.000000in}}%
\pgfpathlineto{\pgfqpoint{0.000000in}{-0.048611in}}%
\pgfusepath{stroke,fill}%
}%
\begin{pgfscope}%
\pgfsys@transformshift{3.204944in}{0.670138in}%
\pgfsys@useobject{currentmarker}{}%
\end{pgfscope}%
\end{pgfscope}%
\begin{pgfscope}%
\definecolor{textcolor}{rgb}{0.000000,0.000000,0.000000}%
\pgfsetstrokecolor{textcolor}%
\pgfsetfillcolor{textcolor}%
\pgftext[x=3.204944in,y=0.572916in,,top]{\color{textcolor}{\rmfamily\fontsize{14.000000}{16.800000}\selectfont\catcode`\^=\active\def^{\ifmmode\sp\else\^{}\fi}\catcode`\%=\active\def%{\%}$\mathdefault{6500}$}}%
\end{pgfscope}%
\begin{pgfscope}%
\pgfpathrectangle{\pgfqpoint{0.688192in}{0.670138in}}{\pgfqpoint{6.200000in}{4.620000in}}%
\pgfusepath{clip}%
\pgfsetrectcap%
\pgfsetroundjoin%
\pgfsetlinewidth{0.803000pt}%
\definecolor{currentstroke}{rgb}{0.690196,0.690196,0.690196}%
\pgfsetstrokecolor{currentstroke}%
\pgfsetdash{}{0pt}%
\pgfpathmoveto{\pgfqpoint{4.246179in}{0.670138in}}%
\pgfpathlineto{\pgfqpoint{4.246179in}{5.290138in}}%
\pgfusepath{stroke}%
\end{pgfscope}%
\begin{pgfscope}%
\pgfsetbuttcap%
\pgfsetroundjoin%
\definecolor{currentfill}{rgb}{0.000000,0.000000,0.000000}%
\pgfsetfillcolor{currentfill}%
\pgfsetlinewidth{0.803000pt}%
\definecolor{currentstroke}{rgb}{0.000000,0.000000,0.000000}%
\pgfsetstrokecolor{currentstroke}%
\pgfsetdash{}{0pt}%
\pgfsys@defobject{currentmarker}{\pgfqpoint{0.000000in}{-0.048611in}}{\pgfqpoint{0.000000in}{0.000000in}}{%
\pgfpathmoveto{\pgfqpoint{0.000000in}{0.000000in}}%
\pgfpathlineto{\pgfqpoint{0.000000in}{-0.048611in}}%
\pgfusepath{stroke,fill}%
}%
\begin{pgfscope}%
\pgfsys@transformshift{4.246179in}{0.670138in}%
\pgfsys@useobject{currentmarker}{}%
\end{pgfscope}%
\end{pgfscope}%
\begin{pgfscope}%
\definecolor{textcolor}{rgb}{0.000000,0.000000,0.000000}%
\pgfsetstrokecolor{textcolor}%
\pgfsetfillcolor{textcolor}%
\pgftext[x=4.246179in,y=0.572916in,,top]{\color{textcolor}{\rmfamily\fontsize{14.000000}{16.800000}\selectfont\catcode`\^=\active\def^{\ifmmode\sp\else\^{}\fi}\catcode`\%=\active\def%{\%}$\mathdefault{7000}$}}%
\end{pgfscope}%
\begin{pgfscope}%
\pgfpathrectangle{\pgfqpoint{0.688192in}{0.670138in}}{\pgfqpoint{6.200000in}{4.620000in}}%
\pgfusepath{clip}%
\pgfsetrectcap%
\pgfsetroundjoin%
\pgfsetlinewidth{0.803000pt}%
\definecolor{currentstroke}{rgb}{0.690196,0.690196,0.690196}%
\pgfsetstrokecolor{currentstroke}%
\pgfsetdash{}{0pt}%
\pgfpathmoveto{\pgfqpoint{5.287414in}{0.670138in}}%
\pgfpathlineto{\pgfqpoint{5.287414in}{5.290138in}}%
\pgfusepath{stroke}%
\end{pgfscope}%
\begin{pgfscope}%
\pgfsetbuttcap%
\pgfsetroundjoin%
\definecolor{currentfill}{rgb}{0.000000,0.000000,0.000000}%
\pgfsetfillcolor{currentfill}%
\pgfsetlinewidth{0.803000pt}%
\definecolor{currentstroke}{rgb}{0.000000,0.000000,0.000000}%
\pgfsetstrokecolor{currentstroke}%
\pgfsetdash{}{0pt}%
\pgfsys@defobject{currentmarker}{\pgfqpoint{0.000000in}{-0.048611in}}{\pgfqpoint{0.000000in}{0.000000in}}{%
\pgfpathmoveto{\pgfqpoint{0.000000in}{0.000000in}}%
\pgfpathlineto{\pgfqpoint{0.000000in}{-0.048611in}}%
\pgfusepath{stroke,fill}%
}%
\begin{pgfscope}%
\pgfsys@transformshift{5.287414in}{0.670138in}%
\pgfsys@useobject{currentmarker}{}%
\end{pgfscope}%
\end{pgfscope}%
\begin{pgfscope}%
\definecolor{textcolor}{rgb}{0.000000,0.000000,0.000000}%
\pgfsetstrokecolor{textcolor}%
\pgfsetfillcolor{textcolor}%
\pgftext[x=5.287414in,y=0.572916in,,top]{\color{textcolor}{\rmfamily\fontsize{14.000000}{16.800000}\selectfont\catcode`\^=\active\def^{\ifmmode\sp\else\^{}\fi}\catcode`\%=\active\def%{\%}$\mathdefault{7500}$}}%
\end{pgfscope}%
\begin{pgfscope}%
\pgfpathrectangle{\pgfqpoint{0.688192in}{0.670138in}}{\pgfqpoint{6.200000in}{4.620000in}}%
\pgfusepath{clip}%
\pgfsetrectcap%
\pgfsetroundjoin%
\pgfsetlinewidth{0.803000pt}%
\definecolor{currentstroke}{rgb}{0.690196,0.690196,0.690196}%
\pgfsetstrokecolor{currentstroke}%
\pgfsetdash{}{0pt}%
\pgfpathmoveto{\pgfqpoint{6.328649in}{0.670138in}}%
\pgfpathlineto{\pgfqpoint{6.328649in}{5.290138in}}%
\pgfusepath{stroke}%
\end{pgfscope}%
\begin{pgfscope}%
\pgfsetbuttcap%
\pgfsetroundjoin%
\definecolor{currentfill}{rgb}{0.000000,0.000000,0.000000}%
\pgfsetfillcolor{currentfill}%
\pgfsetlinewidth{0.803000pt}%
\definecolor{currentstroke}{rgb}{0.000000,0.000000,0.000000}%
\pgfsetstrokecolor{currentstroke}%
\pgfsetdash{}{0pt}%
\pgfsys@defobject{currentmarker}{\pgfqpoint{0.000000in}{-0.048611in}}{\pgfqpoint{0.000000in}{0.000000in}}{%
\pgfpathmoveto{\pgfqpoint{0.000000in}{0.000000in}}%
\pgfpathlineto{\pgfqpoint{0.000000in}{-0.048611in}}%
\pgfusepath{stroke,fill}%
}%
\begin{pgfscope}%
\pgfsys@transformshift{6.328649in}{0.670138in}%
\pgfsys@useobject{currentmarker}{}%
\end{pgfscope}%
\end{pgfscope}%
\begin{pgfscope}%
\definecolor{textcolor}{rgb}{0.000000,0.000000,0.000000}%
\pgfsetstrokecolor{textcolor}%
\pgfsetfillcolor{textcolor}%
\pgftext[x=6.328649in,y=0.572916in,,top]{\color{textcolor}{\rmfamily\fontsize{14.000000}{16.800000}\selectfont\catcode`\^=\active\def^{\ifmmode\sp\else\^{}\fi}\catcode`\%=\active\def%{\%}$\mathdefault{8000}$}}%
\end{pgfscope}%
\begin{pgfscope}%
\definecolor{textcolor}{rgb}{0.000000,0.000000,0.000000}%
\pgfsetstrokecolor{textcolor}%
\pgfsetfillcolor{textcolor}%
\pgftext[x=3.788192in,y=0.339583in,,top]{\color{textcolor}{\rmfamily\fontsize{18.000000}{21.600000}\selectfont\catcode`\^=\active\def^{\ifmmode\sp\else\^{}\fi}\catcode`\%=\active\def%{\%}Total Cost (M\$)}}%
\end{pgfscope}%
\begin{pgfscope}%
\pgfpathrectangle{\pgfqpoint{0.688192in}{0.670138in}}{\pgfqpoint{6.200000in}{4.620000in}}%
\pgfusepath{clip}%
\pgfsetrectcap%
\pgfsetroundjoin%
\pgfsetlinewidth{0.803000pt}%
\definecolor{currentstroke}{rgb}{0.690196,0.690196,0.690196}%
\pgfsetstrokecolor{currentstroke}%
\pgfsetdash{}{0pt}%
\pgfpathmoveto{\pgfqpoint{0.688192in}{1.130833in}}%
\pgfpathlineto{\pgfqpoint{6.888192in}{1.130833in}}%
\pgfusepath{stroke}%
\end{pgfscope}%
\begin{pgfscope}%
\pgfsetbuttcap%
\pgfsetroundjoin%
\definecolor{currentfill}{rgb}{0.000000,0.000000,0.000000}%
\pgfsetfillcolor{currentfill}%
\pgfsetlinewidth{0.803000pt}%
\definecolor{currentstroke}{rgb}{0.000000,0.000000,0.000000}%
\pgfsetstrokecolor{currentstroke}%
\pgfsetdash{}{0pt}%
\pgfsys@defobject{currentmarker}{\pgfqpoint{-0.048611in}{0.000000in}}{\pgfqpoint{-0.000000in}{0.000000in}}{%
\pgfpathmoveto{\pgfqpoint{-0.000000in}{0.000000in}}%
\pgfpathlineto{\pgfqpoint{-0.048611in}{0.000000in}}%
\pgfusepath{stroke,fill}%
}%
\begin{pgfscope}%
\pgfsys@transformshift{0.688192in}{1.130833in}%
\pgfsys@useobject{currentmarker}{}%
\end{pgfscope}%
\end{pgfscope}%
\begin{pgfscope}%
\definecolor{textcolor}{rgb}{0.000000,0.000000,0.000000}%
\pgfsetstrokecolor{textcolor}%
\pgfsetfillcolor{textcolor}%
\pgftext[x=0.395138in, y=1.061389in, left, base]{\color{textcolor}{\rmfamily\fontsize{14.000000}{16.800000}\selectfont\catcode`\^=\active\def^{\ifmmode\sp\else\^{}\fi}\catcode`\%=\active\def%{\%}$\mathdefault{10}$}}%
\end{pgfscope}%
\begin{pgfscope}%
\pgfpathrectangle{\pgfqpoint{0.688192in}{0.670138in}}{\pgfqpoint{6.200000in}{4.620000in}}%
\pgfusepath{clip}%
\pgfsetrectcap%
\pgfsetroundjoin%
\pgfsetlinewidth{0.803000pt}%
\definecolor{currentstroke}{rgb}{0.690196,0.690196,0.690196}%
\pgfsetstrokecolor{currentstroke}%
\pgfsetdash{}{0pt}%
\pgfpathmoveto{\pgfqpoint{0.688192in}{1.724860in}}%
\pgfpathlineto{\pgfqpoint{6.888192in}{1.724860in}}%
\pgfusepath{stroke}%
\end{pgfscope}%
\begin{pgfscope}%
\pgfsetbuttcap%
\pgfsetroundjoin%
\definecolor{currentfill}{rgb}{0.000000,0.000000,0.000000}%
\pgfsetfillcolor{currentfill}%
\pgfsetlinewidth{0.803000pt}%
\definecolor{currentstroke}{rgb}{0.000000,0.000000,0.000000}%
\pgfsetstrokecolor{currentstroke}%
\pgfsetdash{}{0pt}%
\pgfsys@defobject{currentmarker}{\pgfqpoint{-0.048611in}{0.000000in}}{\pgfqpoint{-0.000000in}{0.000000in}}{%
\pgfpathmoveto{\pgfqpoint{-0.000000in}{0.000000in}}%
\pgfpathlineto{\pgfqpoint{-0.048611in}{0.000000in}}%
\pgfusepath{stroke,fill}%
}%
\begin{pgfscope}%
\pgfsys@transformshift{0.688192in}{1.724860in}%
\pgfsys@useobject{currentmarker}{}%
\end{pgfscope}%
\end{pgfscope}%
\begin{pgfscope}%
\definecolor{textcolor}{rgb}{0.000000,0.000000,0.000000}%
\pgfsetstrokecolor{textcolor}%
\pgfsetfillcolor{textcolor}%
\pgftext[x=0.395138in, y=1.655416in, left, base]{\color{textcolor}{\rmfamily\fontsize{14.000000}{16.800000}\selectfont\catcode`\^=\active\def^{\ifmmode\sp\else\^{}\fi}\catcode`\%=\active\def%{\%}$\mathdefault{20}$}}%
\end{pgfscope}%
\begin{pgfscope}%
\pgfpathrectangle{\pgfqpoint{0.688192in}{0.670138in}}{\pgfqpoint{6.200000in}{4.620000in}}%
\pgfusepath{clip}%
\pgfsetrectcap%
\pgfsetroundjoin%
\pgfsetlinewidth{0.803000pt}%
\definecolor{currentstroke}{rgb}{0.690196,0.690196,0.690196}%
\pgfsetstrokecolor{currentstroke}%
\pgfsetdash{}{0pt}%
\pgfpathmoveto{\pgfqpoint{0.688192in}{2.318888in}}%
\pgfpathlineto{\pgfqpoint{6.888192in}{2.318888in}}%
\pgfusepath{stroke}%
\end{pgfscope}%
\begin{pgfscope}%
\pgfsetbuttcap%
\pgfsetroundjoin%
\definecolor{currentfill}{rgb}{0.000000,0.000000,0.000000}%
\pgfsetfillcolor{currentfill}%
\pgfsetlinewidth{0.803000pt}%
\definecolor{currentstroke}{rgb}{0.000000,0.000000,0.000000}%
\pgfsetstrokecolor{currentstroke}%
\pgfsetdash{}{0pt}%
\pgfsys@defobject{currentmarker}{\pgfqpoint{-0.048611in}{0.000000in}}{\pgfqpoint{-0.000000in}{0.000000in}}{%
\pgfpathmoveto{\pgfqpoint{-0.000000in}{0.000000in}}%
\pgfpathlineto{\pgfqpoint{-0.048611in}{0.000000in}}%
\pgfusepath{stroke,fill}%
}%
\begin{pgfscope}%
\pgfsys@transformshift{0.688192in}{2.318888in}%
\pgfsys@useobject{currentmarker}{}%
\end{pgfscope}%
\end{pgfscope}%
\begin{pgfscope}%
\definecolor{textcolor}{rgb}{0.000000,0.000000,0.000000}%
\pgfsetstrokecolor{textcolor}%
\pgfsetfillcolor{textcolor}%
\pgftext[x=0.395138in, y=2.249444in, left, base]{\color{textcolor}{\rmfamily\fontsize{14.000000}{16.800000}\selectfont\catcode`\^=\active\def^{\ifmmode\sp\else\^{}\fi}\catcode`\%=\active\def%{\%}$\mathdefault{30}$}}%
\end{pgfscope}%
\begin{pgfscope}%
\pgfpathrectangle{\pgfqpoint{0.688192in}{0.670138in}}{\pgfqpoint{6.200000in}{4.620000in}}%
\pgfusepath{clip}%
\pgfsetrectcap%
\pgfsetroundjoin%
\pgfsetlinewidth{0.803000pt}%
\definecolor{currentstroke}{rgb}{0.690196,0.690196,0.690196}%
\pgfsetstrokecolor{currentstroke}%
\pgfsetdash{}{0pt}%
\pgfpathmoveto{\pgfqpoint{0.688192in}{2.912915in}}%
\pgfpathlineto{\pgfqpoint{6.888192in}{2.912915in}}%
\pgfusepath{stroke}%
\end{pgfscope}%
\begin{pgfscope}%
\pgfsetbuttcap%
\pgfsetroundjoin%
\definecolor{currentfill}{rgb}{0.000000,0.000000,0.000000}%
\pgfsetfillcolor{currentfill}%
\pgfsetlinewidth{0.803000pt}%
\definecolor{currentstroke}{rgb}{0.000000,0.000000,0.000000}%
\pgfsetstrokecolor{currentstroke}%
\pgfsetdash{}{0pt}%
\pgfsys@defobject{currentmarker}{\pgfqpoint{-0.048611in}{0.000000in}}{\pgfqpoint{-0.000000in}{0.000000in}}{%
\pgfpathmoveto{\pgfqpoint{-0.000000in}{0.000000in}}%
\pgfpathlineto{\pgfqpoint{-0.048611in}{0.000000in}}%
\pgfusepath{stroke,fill}%
}%
\begin{pgfscope}%
\pgfsys@transformshift{0.688192in}{2.912915in}%
\pgfsys@useobject{currentmarker}{}%
\end{pgfscope}%
\end{pgfscope}%
\begin{pgfscope}%
\definecolor{textcolor}{rgb}{0.000000,0.000000,0.000000}%
\pgfsetstrokecolor{textcolor}%
\pgfsetfillcolor{textcolor}%
\pgftext[x=0.395138in, y=2.843471in, left, base]{\color{textcolor}{\rmfamily\fontsize{14.000000}{16.800000}\selectfont\catcode`\^=\active\def^{\ifmmode\sp\else\^{}\fi}\catcode`\%=\active\def%{\%}$\mathdefault{40}$}}%
\end{pgfscope}%
\begin{pgfscope}%
\pgfpathrectangle{\pgfqpoint{0.688192in}{0.670138in}}{\pgfqpoint{6.200000in}{4.620000in}}%
\pgfusepath{clip}%
\pgfsetrectcap%
\pgfsetroundjoin%
\pgfsetlinewidth{0.803000pt}%
\definecolor{currentstroke}{rgb}{0.690196,0.690196,0.690196}%
\pgfsetstrokecolor{currentstroke}%
\pgfsetdash{}{0pt}%
\pgfpathmoveto{\pgfqpoint{0.688192in}{3.506943in}}%
\pgfpathlineto{\pgfqpoint{6.888192in}{3.506943in}}%
\pgfusepath{stroke}%
\end{pgfscope}%
\begin{pgfscope}%
\pgfsetbuttcap%
\pgfsetroundjoin%
\definecolor{currentfill}{rgb}{0.000000,0.000000,0.000000}%
\pgfsetfillcolor{currentfill}%
\pgfsetlinewidth{0.803000pt}%
\definecolor{currentstroke}{rgb}{0.000000,0.000000,0.000000}%
\pgfsetstrokecolor{currentstroke}%
\pgfsetdash{}{0pt}%
\pgfsys@defobject{currentmarker}{\pgfqpoint{-0.048611in}{0.000000in}}{\pgfqpoint{-0.000000in}{0.000000in}}{%
\pgfpathmoveto{\pgfqpoint{-0.000000in}{0.000000in}}%
\pgfpathlineto{\pgfqpoint{-0.048611in}{0.000000in}}%
\pgfusepath{stroke,fill}%
}%
\begin{pgfscope}%
\pgfsys@transformshift{0.688192in}{3.506943in}%
\pgfsys@useobject{currentmarker}{}%
\end{pgfscope}%
\end{pgfscope}%
\begin{pgfscope}%
\definecolor{textcolor}{rgb}{0.000000,0.000000,0.000000}%
\pgfsetstrokecolor{textcolor}%
\pgfsetfillcolor{textcolor}%
\pgftext[x=0.395138in, y=3.437499in, left, base]{\color{textcolor}{\rmfamily\fontsize{14.000000}{16.800000}\selectfont\catcode`\^=\active\def^{\ifmmode\sp\else\^{}\fi}\catcode`\%=\active\def%{\%}$\mathdefault{50}$}}%
\end{pgfscope}%
\begin{pgfscope}%
\pgfpathrectangle{\pgfqpoint{0.688192in}{0.670138in}}{\pgfqpoint{6.200000in}{4.620000in}}%
\pgfusepath{clip}%
\pgfsetrectcap%
\pgfsetroundjoin%
\pgfsetlinewidth{0.803000pt}%
\definecolor{currentstroke}{rgb}{0.690196,0.690196,0.690196}%
\pgfsetstrokecolor{currentstroke}%
\pgfsetdash{}{0pt}%
\pgfpathmoveto{\pgfqpoint{0.688192in}{4.100970in}}%
\pgfpathlineto{\pgfqpoint{6.888192in}{4.100970in}}%
\pgfusepath{stroke}%
\end{pgfscope}%
\begin{pgfscope}%
\pgfsetbuttcap%
\pgfsetroundjoin%
\definecolor{currentfill}{rgb}{0.000000,0.000000,0.000000}%
\pgfsetfillcolor{currentfill}%
\pgfsetlinewidth{0.803000pt}%
\definecolor{currentstroke}{rgb}{0.000000,0.000000,0.000000}%
\pgfsetstrokecolor{currentstroke}%
\pgfsetdash{}{0pt}%
\pgfsys@defobject{currentmarker}{\pgfqpoint{-0.048611in}{0.000000in}}{\pgfqpoint{-0.000000in}{0.000000in}}{%
\pgfpathmoveto{\pgfqpoint{-0.000000in}{0.000000in}}%
\pgfpathlineto{\pgfqpoint{-0.048611in}{0.000000in}}%
\pgfusepath{stroke,fill}%
}%
\begin{pgfscope}%
\pgfsys@transformshift{0.688192in}{4.100970in}%
\pgfsys@useobject{currentmarker}{}%
\end{pgfscope}%
\end{pgfscope}%
\begin{pgfscope}%
\definecolor{textcolor}{rgb}{0.000000,0.000000,0.000000}%
\pgfsetstrokecolor{textcolor}%
\pgfsetfillcolor{textcolor}%
\pgftext[x=0.395138in, y=4.031526in, left, base]{\color{textcolor}{\rmfamily\fontsize{14.000000}{16.800000}\selectfont\catcode`\^=\active\def^{\ifmmode\sp\else\^{}\fi}\catcode`\%=\active\def%{\%}$\mathdefault{60}$}}%
\end{pgfscope}%
\begin{pgfscope}%
\pgfpathrectangle{\pgfqpoint{0.688192in}{0.670138in}}{\pgfqpoint{6.200000in}{4.620000in}}%
\pgfusepath{clip}%
\pgfsetrectcap%
\pgfsetroundjoin%
\pgfsetlinewidth{0.803000pt}%
\definecolor{currentstroke}{rgb}{0.690196,0.690196,0.690196}%
\pgfsetstrokecolor{currentstroke}%
\pgfsetdash{}{0pt}%
\pgfpathmoveto{\pgfqpoint{0.688192in}{4.694998in}}%
\pgfpathlineto{\pgfqpoint{6.888192in}{4.694998in}}%
\pgfusepath{stroke}%
\end{pgfscope}%
\begin{pgfscope}%
\pgfsetbuttcap%
\pgfsetroundjoin%
\definecolor{currentfill}{rgb}{0.000000,0.000000,0.000000}%
\pgfsetfillcolor{currentfill}%
\pgfsetlinewidth{0.803000pt}%
\definecolor{currentstroke}{rgb}{0.000000,0.000000,0.000000}%
\pgfsetstrokecolor{currentstroke}%
\pgfsetdash{}{0pt}%
\pgfsys@defobject{currentmarker}{\pgfqpoint{-0.048611in}{0.000000in}}{\pgfqpoint{-0.000000in}{0.000000in}}{%
\pgfpathmoveto{\pgfqpoint{-0.000000in}{0.000000in}}%
\pgfpathlineto{\pgfqpoint{-0.048611in}{0.000000in}}%
\pgfusepath{stroke,fill}%
}%
\begin{pgfscope}%
\pgfsys@transformshift{0.688192in}{4.694998in}%
\pgfsys@useobject{currentmarker}{}%
\end{pgfscope}%
\end{pgfscope}%
\begin{pgfscope}%
\definecolor{textcolor}{rgb}{0.000000,0.000000,0.000000}%
\pgfsetstrokecolor{textcolor}%
\pgfsetfillcolor{textcolor}%
\pgftext[x=0.395138in, y=4.625553in, left, base]{\color{textcolor}{\rmfamily\fontsize{14.000000}{16.800000}\selectfont\catcode`\^=\active\def^{\ifmmode\sp\else\^{}\fi}\catcode`\%=\active\def%{\%}$\mathdefault{70}$}}%
\end{pgfscope}%
\begin{pgfscope}%
\pgfpathrectangle{\pgfqpoint{0.688192in}{0.670138in}}{\pgfqpoint{6.200000in}{4.620000in}}%
\pgfusepath{clip}%
\pgfsetrectcap%
\pgfsetroundjoin%
\pgfsetlinewidth{0.803000pt}%
\definecolor{currentstroke}{rgb}{0.690196,0.690196,0.690196}%
\pgfsetstrokecolor{currentstroke}%
\pgfsetdash{}{0pt}%
\pgfpathmoveto{\pgfqpoint{0.688192in}{5.289025in}}%
\pgfpathlineto{\pgfqpoint{6.888192in}{5.289025in}}%
\pgfusepath{stroke}%
\end{pgfscope}%
\begin{pgfscope}%
\pgfsetbuttcap%
\pgfsetroundjoin%
\definecolor{currentfill}{rgb}{0.000000,0.000000,0.000000}%
\pgfsetfillcolor{currentfill}%
\pgfsetlinewidth{0.803000pt}%
\definecolor{currentstroke}{rgb}{0.000000,0.000000,0.000000}%
\pgfsetstrokecolor{currentstroke}%
\pgfsetdash{}{0pt}%
\pgfsys@defobject{currentmarker}{\pgfqpoint{-0.048611in}{0.000000in}}{\pgfqpoint{-0.000000in}{0.000000in}}{%
\pgfpathmoveto{\pgfqpoint{-0.000000in}{0.000000in}}%
\pgfpathlineto{\pgfqpoint{-0.048611in}{0.000000in}}%
\pgfusepath{stroke,fill}%
}%
\begin{pgfscope}%
\pgfsys@transformshift{0.688192in}{5.289025in}%
\pgfsys@useobject{currentmarker}{}%
\end{pgfscope}%
\end{pgfscope}%
\begin{pgfscope}%
\definecolor{textcolor}{rgb}{0.000000,0.000000,0.000000}%
\pgfsetstrokecolor{textcolor}%
\pgfsetfillcolor{textcolor}%
\pgftext[x=0.395138in, y=5.219581in, left, base]{\color{textcolor}{\rmfamily\fontsize{14.000000}{16.800000}\selectfont\catcode`\^=\active\def^{\ifmmode\sp\else\^{}\fi}\catcode`\%=\active\def%{\%}$\mathdefault{80}$}}%
\end{pgfscope}%
\begin{pgfscope}%
\definecolor{textcolor}{rgb}{0.000000,0.000000,0.000000}%
\pgfsetstrokecolor{textcolor}%
\pgfsetfillcolor{textcolor}%
\pgftext[x=0.339583in,y=2.980138in,,bottom,rotate=90.000000]{\color{textcolor}{\rmfamily\fontsize{18.000000}{21.600000}\selectfont\catcode`\^=\active\def^{\ifmmode\sp\else\^{}\fi}\catcode`\%=\active\def%{\%}CO2 emissions (MT CO2)}}%
\end{pgfscope}%
\begin{pgfscope}%
\pgfpathrectangle{\pgfqpoint{0.688192in}{0.670138in}}{\pgfqpoint{6.200000in}{4.620000in}}%
\pgfusepath{clip}%
\pgfsetrectcap%
\pgfsetroundjoin%
\pgfsetlinewidth{1.505625pt}%
\definecolor{currentstroke}{rgb}{0.000000,0.000000,1.000000}%
\pgfsetstrokecolor{currentstroke}%
\pgfsetdash{}{0pt}%
\pgfpathmoveto{\pgfqpoint{0.741425in}{1.377543in}}%
\pgfpathlineto{\pgfqpoint{0.758703in}{0.955032in}}%
\pgfpathlineto{\pgfqpoint{0.768198in}{0.875033in}}%
\pgfpathlineto{\pgfqpoint{0.774746in}{0.828781in}}%
\pgfpathlineto{\pgfqpoint{0.778243in}{0.822495in}}%
\pgfpathlineto{\pgfqpoint{0.782159in}{0.789611in}}%
\pgfpathlineto{\pgfqpoint{0.786516in}{0.779881in}}%
\pgfpathlineto{\pgfqpoint{0.792538in}{0.779145in}}%
\pgfpathlineto{\pgfqpoint{0.794668in}{0.758056in}}%
\pgfpathlineto{\pgfqpoint{0.799837in}{0.752930in}}%
\pgfpathlineto{\pgfqpoint{0.809370in}{0.751978in}}%
\pgfpathlineto{\pgfqpoint{0.812629in}{0.743975in}}%
\pgfpathlineto{\pgfqpoint{0.815972in}{0.742575in}}%
\pgfpathlineto{\pgfqpoint{0.822987in}{0.738477in}}%
\pgfpathlineto{\pgfqpoint{0.828825in}{0.734937in}}%
\pgfpathlineto{\pgfqpoint{0.829214in}{0.733319in}}%
\pgfpathlineto{\pgfqpoint{0.833044in}{0.730858in}}%
\pgfpathlineto{\pgfqpoint{0.848459in}{0.726329in}}%
\pgfpathlineto{\pgfqpoint{0.864854in}{0.720019in}}%
\pgfpathlineto{\pgfqpoint{0.887104in}{0.715517in}}%
\pgfpathlineto{\pgfqpoint{0.907479in}{0.714004in}}%
\pgfpathlineto{\pgfqpoint{0.908310in}{0.712008in}}%
\pgfpathlineto{\pgfqpoint{0.909513in}{0.708525in}}%
\pgfpathlineto{\pgfqpoint{0.912740in}{0.707284in}}%
\pgfpathlineto{\pgfqpoint{0.920440in}{0.706723in}}%
\pgfpathlineto{\pgfqpoint{0.925670in}{0.705238in}}%
\pgfpathlineto{\pgfqpoint{0.948903in}{0.702931in}}%
\pgfpathlineto{\pgfqpoint{0.951945in}{0.701707in}}%
\pgfpathlineto{\pgfqpoint{0.952035in}{0.700391in}}%
\pgfpathlineto{\pgfqpoint{0.957029in}{0.700173in}}%
\pgfpathlineto{\pgfqpoint{0.968828in}{0.697963in}}%
\pgfpathlineto{\pgfqpoint{0.974412in}{0.697738in}}%
\pgfpathlineto{\pgfqpoint{0.975275in}{0.696914in}}%
\pgfpathlineto{\pgfqpoint{1.021767in}{0.694795in}}%
\pgfpathlineto{\pgfqpoint{1.025407in}{0.690657in}}%
\pgfpathlineto{\pgfqpoint{1.027475in}{0.690338in}}%
\pgfpathlineto{\pgfqpoint{1.034837in}{0.689784in}}%
\pgfpathlineto{\pgfqpoint{1.049406in}{0.687676in}}%
\pgfpathlineto{\pgfqpoint{1.054714in}{0.687138in}}%
\pgfpathlineto{\pgfqpoint{1.059617in}{0.686467in}}%
\pgfpathlineto{\pgfqpoint{1.072141in}{0.685078in}}%
\pgfpathlineto{\pgfqpoint{1.092208in}{0.684413in}}%
\pgfpathlineto{\pgfqpoint{1.115209in}{0.684111in}}%
\pgfpathlineto{\pgfqpoint{1.131834in}{0.684071in}}%
\pgfpathlineto{\pgfqpoint{1.152628in}{0.684059in}}%
\pgfpathlineto{\pgfqpoint{1.251312in}{0.683263in}}%
\pgfpathlineto{\pgfqpoint{1.277476in}{0.683159in}}%
\pgfpathlineto{\pgfqpoint{1.314870in}{0.682855in}}%
\pgfpathlineto{\pgfqpoint{1.369253in}{0.682756in}}%
\pgfpathlineto{\pgfqpoint{1.398687in}{0.682288in}}%
\pgfpathlineto{\pgfqpoint{1.467852in}{0.682134in}}%
\pgfpathlineto{\pgfqpoint{1.557026in}{0.681680in}}%
\pgfpathlineto{\pgfqpoint{1.627242in}{0.680913in}}%
\pgfpathlineto{\pgfqpoint{1.737728in}{0.680478in}}%
\pgfpathlineto{\pgfqpoint{1.887036in}{0.679610in}}%
\pgfpathlineto{\pgfqpoint{2.037481in}{0.678826in}}%
\pgfpathlineto{\pgfqpoint{2.258348in}{0.677741in}}%
\pgfpathlineto{\pgfqpoint{2.626338in}{0.676361in}}%
\pgfpathlineto{\pgfqpoint{3.263784in}{0.674352in}}%
\pgfpathlineto{\pgfqpoint{5.322800in}{0.670138in}}%
\pgfusepath{stroke}%
\end{pgfscope}%
\begin{pgfscope}%
\pgfpathrectangle{\pgfqpoint{0.688192in}{0.670138in}}{\pgfqpoint{6.200000in}{4.620000in}}%
\pgfusepath{clip}%
\pgfsetbuttcap%
\pgfsetroundjoin%
\definecolor{currentfill}{rgb}{0.000000,0.000000,1.000000}%
\pgfsetfillcolor{currentfill}%
\pgfsetlinewidth{1.003750pt}%
\definecolor{currentstroke}{rgb}{0.000000,0.000000,1.000000}%
\pgfsetstrokecolor{currentstroke}%
\pgfsetdash{}{0pt}%
\pgfsys@defobject{currentmarker}{\pgfqpoint{-0.006944in}{-0.006944in}}{\pgfqpoint{0.006944in}{0.006944in}}{%
\pgfpathmoveto{\pgfqpoint{0.000000in}{-0.006944in}}%
\pgfpathcurveto{\pgfqpoint{0.001842in}{-0.006944in}}{\pgfqpoint{0.003608in}{-0.006213in}}{\pgfqpoint{0.004910in}{-0.004910in}}%
\pgfpathcurveto{\pgfqpoint{0.006213in}{-0.003608in}}{\pgfqpoint{0.006944in}{-0.001842in}}{\pgfqpoint{0.006944in}{0.000000in}}%
\pgfpathcurveto{\pgfqpoint{0.006944in}{0.001842in}}{\pgfqpoint{0.006213in}{0.003608in}}{\pgfqpoint{0.004910in}{0.004910in}}%
\pgfpathcurveto{\pgfqpoint{0.003608in}{0.006213in}}{\pgfqpoint{0.001842in}{0.006944in}}{\pgfqpoint{0.000000in}{0.006944in}}%
\pgfpathcurveto{\pgfqpoint{-0.001842in}{0.006944in}}{\pgfqpoint{-0.003608in}{0.006213in}}{\pgfqpoint{-0.004910in}{0.004910in}}%
\pgfpathcurveto{\pgfqpoint{-0.006213in}{0.003608in}}{\pgfqpoint{-0.006944in}{0.001842in}}{\pgfqpoint{-0.006944in}{0.000000in}}%
\pgfpathcurveto{\pgfqpoint{-0.006944in}{-0.001842in}}{\pgfqpoint{-0.006213in}{-0.003608in}}{\pgfqpoint{-0.004910in}{-0.004910in}}%
\pgfpathcurveto{\pgfqpoint{-0.003608in}{-0.006213in}}{\pgfqpoint{-0.001842in}{-0.006944in}}{\pgfqpoint{0.000000in}{-0.006944in}}%
\pgfpathlineto{\pgfqpoint{0.000000in}{-0.006944in}}%
\pgfpathclose%
\pgfusepath{stroke,fill}%
}%
\begin{pgfscope}%
\pgfsys@transformshift{0.741425in}{1.377543in}%
\pgfsys@useobject{currentmarker}{}%
\end{pgfscope}%
\begin{pgfscope}%
\pgfsys@transformshift{0.758703in}{0.955032in}%
\pgfsys@useobject{currentmarker}{}%
\end{pgfscope}%
\begin{pgfscope}%
\pgfsys@transformshift{0.768198in}{0.875033in}%
\pgfsys@useobject{currentmarker}{}%
\end{pgfscope}%
\begin{pgfscope}%
\pgfsys@transformshift{0.774746in}{0.828781in}%
\pgfsys@useobject{currentmarker}{}%
\end{pgfscope}%
\begin{pgfscope}%
\pgfsys@transformshift{0.778243in}{0.822495in}%
\pgfsys@useobject{currentmarker}{}%
\end{pgfscope}%
\begin{pgfscope}%
\pgfsys@transformshift{0.782159in}{0.789611in}%
\pgfsys@useobject{currentmarker}{}%
\end{pgfscope}%
\begin{pgfscope}%
\pgfsys@transformshift{0.786516in}{0.779881in}%
\pgfsys@useobject{currentmarker}{}%
\end{pgfscope}%
\begin{pgfscope}%
\pgfsys@transformshift{0.792538in}{0.779145in}%
\pgfsys@useobject{currentmarker}{}%
\end{pgfscope}%
\begin{pgfscope}%
\pgfsys@transformshift{0.794668in}{0.758056in}%
\pgfsys@useobject{currentmarker}{}%
\end{pgfscope}%
\begin{pgfscope}%
\pgfsys@transformshift{0.799837in}{0.752930in}%
\pgfsys@useobject{currentmarker}{}%
\end{pgfscope}%
\begin{pgfscope}%
\pgfsys@transformshift{0.809370in}{0.751978in}%
\pgfsys@useobject{currentmarker}{}%
\end{pgfscope}%
\begin{pgfscope}%
\pgfsys@transformshift{0.812629in}{0.743975in}%
\pgfsys@useobject{currentmarker}{}%
\end{pgfscope}%
\begin{pgfscope}%
\pgfsys@transformshift{0.815972in}{0.742575in}%
\pgfsys@useobject{currentmarker}{}%
\end{pgfscope}%
\begin{pgfscope}%
\pgfsys@transformshift{0.822987in}{0.738477in}%
\pgfsys@useobject{currentmarker}{}%
\end{pgfscope}%
\begin{pgfscope}%
\pgfsys@transformshift{0.828825in}{0.734937in}%
\pgfsys@useobject{currentmarker}{}%
\end{pgfscope}%
\begin{pgfscope}%
\pgfsys@transformshift{0.829214in}{0.733319in}%
\pgfsys@useobject{currentmarker}{}%
\end{pgfscope}%
\begin{pgfscope}%
\pgfsys@transformshift{0.833044in}{0.730858in}%
\pgfsys@useobject{currentmarker}{}%
\end{pgfscope}%
\begin{pgfscope}%
\pgfsys@transformshift{0.848459in}{0.726329in}%
\pgfsys@useobject{currentmarker}{}%
\end{pgfscope}%
\begin{pgfscope}%
\pgfsys@transformshift{0.864854in}{0.720019in}%
\pgfsys@useobject{currentmarker}{}%
\end{pgfscope}%
\begin{pgfscope}%
\pgfsys@transformshift{0.887104in}{0.715517in}%
\pgfsys@useobject{currentmarker}{}%
\end{pgfscope}%
\begin{pgfscope}%
\pgfsys@transformshift{0.907479in}{0.714004in}%
\pgfsys@useobject{currentmarker}{}%
\end{pgfscope}%
\begin{pgfscope}%
\pgfsys@transformshift{0.908310in}{0.712008in}%
\pgfsys@useobject{currentmarker}{}%
\end{pgfscope}%
\begin{pgfscope}%
\pgfsys@transformshift{0.909513in}{0.708525in}%
\pgfsys@useobject{currentmarker}{}%
\end{pgfscope}%
\begin{pgfscope}%
\pgfsys@transformshift{0.912740in}{0.707284in}%
\pgfsys@useobject{currentmarker}{}%
\end{pgfscope}%
\begin{pgfscope}%
\pgfsys@transformshift{0.920440in}{0.706723in}%
\pgfsys@useobject{currentmarker}{}%
\end{pgfscope}%
\begin{pgfscope}%
\pgfsys@transformshift{0.925670in}{0.705238in}%
\pgfsys@useobject{currentmarker}{}%
\end{pgfscope}%
\begin{pgfscope}%
\pgfsys@transformshift{0.948903in}{0.702931in}%
\pgfsys@useobject{currentmarker}{}%
\end{pgfscope}%
\begin{pgfscope}%
\pgfsys@transformshift{0.951945in}{0.701707in}%
\pgfsys@useobject{currentmarker}{}%
\end{pgfscope}%
\begin{pgfscope}%
\pgfsys@transformshift{0.952035in}{0.700391in}%
\pgfsys@useobject{currentmarker}{}%
\end{pgfscope}%
\begin{pgfscope}%
\pgfsys@transformshift{0.957029in}{0.700173in}%
\pgfsys@useobject{currentmarker}{}%
\end{pgfscope}%
\begin{pgfscope}%
\pgfsys@transformshift{0.968828in}{0.697963in}%
\pgfsys@useobject{currentmarker}{}%
\end{pgfscope}%
\begin{pgfscope}%
\pgfsys@transformshift{0.974412in}{0.697738in}%
\pgfsys@useobject{currentmarker}{}%
\end{pgfscope}%
\begin{pgfscope}%
\pgfsys@transformshift{0.975275in}{0.696914in}%
\pgfsys@useobject{currentmarker}{}%
\end{pgfscope}%
\begin{pgfscope}%
\pgfsys@transformshift{1.021767in}{0.694795in}%
\pgfsys@useobject{currentmarker}{}%
\end{pgfscope}%
\begin{pgfscope}%
\pgfsys@transformshift{1.025407in}{0.690657in}%
\pgfsys@useobject{currentmarker}{}%
\end{pgfscope}%
\begin{pgfscope}%
\pgfsys@transformshift{1.027475in}{0.690338in}%
\pgfsys@useobject{currentmarker}{}%
\end{pgfscope}%
\begin{pgfscope}%
\pgfsys@transformshift{1.034837in}{0.689784in}%
\pgfsys@useobject{currentmarker}{}%
\end{pgfscope}%
\begin{pgfscope}%
\pgfsys@transformshift{1.049406in}{0.687676in}%
\pgfsys@useobject{currentmarker}{}%
\end{pgfscope}%
\begin{pgfscope}%
\pgfsys@transformshift{1.054714in}{0.687138in}%
\pgfsys@useobject{currentmarker}{}%
\end{pgfscope}%
\begin{pgfscope}%
\pgfsys@transformshift{1.059617in}{0.686467in}%
\pgfsys@useobject{currentmarker}{}%
\end{pgfscope}%
\begin{pgfscope}%
\pgfsys@transformshift{1.072141in}{0.685078in}%
\pgfsys@useobject{currentmarker}{}%
\end{pgfscope}%
\begin{pgfscope}%
\pgfsys@transformshift{1.092208in}{0.684413in}%
\pgfsys@useobject{currentmarker}{}%
\end{pgfscope}%
\begin{pgfscope}%
\pgfsys@transformshift{1.115209in}{0.684111in}%
\pgfsys@useobject{currentmarker}{}%
\end{pgfscope}%
\begin{pgfscope}%
\pgfsys@transformshift{1.131834in}{0.684071in}%
\pgfsys@useobject{currentmarker}{}%
\end{pgfscope}%
\begin{pgfscope}%
\pgfsys@transformshift{1.152628in}{0.684059in}%
\pgfsys@useobject{currentmarker}{}%
\end{pgfscope}%
\begin{pgfscope}%
\pgfsys@transformshift{1.251312in}{0.683263in}%
\pgfsys@useobject{currentmarker}{}%
\end{pgfscope}%
\begin{pgfscope}%
\pgfsys@transformshift{1.277476in}{0.683159in}%
\pgfsys@useobject{currentmarker}{}%
\end{pgfscope}%
\begin{pgfscope}%
\pgfsys@transformshift{1.314870in}{0.682855in}%
\pgfsys@useobject{currentmarker}{}%
\end{pgfscope}%
\begin{pgfscope}%
\pgfsys@transformshift{1.369253in}{0.682756in}%
\pgfsys@useobject{currentmarker}{}%
\end{pgfscope}%
\begin{pgfscope}%
\pgfsys@transformshift{1.398687in}{0.682288in}%
\pgfsys@useobject{currentmarker}{}%
\end{pgfscope}%
\begin{pgfscope}%
\pgfsys@transformshift{1.467852in}{0.682134in}%
\pgfsys@useobject{currentmarker}{}%
\end{pgfscope}%
\begin{pgfscope}%
\pgfsys@transformshift{1.557026in}{0.681680in}%
\pgfsys@useobject{currentmarker}{}%
\end{pgfscope}%
\begin{pgfscope}%
\pgfsys@transformshift{1.627242in}{0.680913in}%
\pgfsys@useobject{currentmarker}{}%
\end{pgfscope}%
\begin{pgfscope}%
\pgfsys@transformshift{1.737728in}{0.680478in}%
\pgfsys@useobject{currentmarker}{}%
\end{pgfscope}%
\begin{pgfscope}%
\pgfsys@transformshift{1.887036in}{0.679610in}%
\pgfsys@useobject{currentmarker}{}%
\end{pgfscope}%
\begin{pgfscope}%
\pgfsys@transformshift{2.037481in}{0.678826in}%
\pgfsys@useobject{currentmarker}{}%
\end{pgfscope}%
\begin{pgfscope}%
\pgfsys@transformshift{2.258348in}{0.677741in}%
\pgfsys@useobject{currentmarker}{}%
\end{pgfscope}%
\begin{pgfscope}%
\pgfsys@transformshift{2.626338in}{0.676361in}%
\pgfsys@useobject{currentmarker}{}%
\end{pgfscope}%
\begin{pgfscope}%
\pgfsys@transformshift{3.263784in}{0.674352in}%
\pgfsys@useobject{currentmarker}{}%
\end{pgfscope}%
\begin{pgfscope}%
\pgfsys@transformshift{5.322800in}{0.670138in}%
\pgfsys@useobject{currentmarker}{}%
\end{pgfscope}%
\end{pgfscope}%
\begin{pgfscope}%
\pgfpathrectangle{\pgfqpoint{0.688192in}{0.670138in}}{\pgfqpoint{6.200000in}{4.620000in}}%
\pgfusepath{clip}%
\pgfsetrectcap%
\pgfsetroundjoin%
\pgfsetlinewidth{1.505625pt}%
\definecolor{currentstroke}{rgb}{0.121569,0.466667,0.705882}%
\pgfsetstrokecolor{currentstroke}%
\pgfsetstrokeopacity{0.500000}%
\pgfsetdash{}{0pt}%
\pgfpathmoveto{\pgfqpoint{1.848679in}{1.461617in}}%
\pgfpathlineto{\pgfqpoint{1.867684in}{0.996854in}}%
\pgfpathlineto{\pgfqpoint{1.878129in}{0.908856in}}%
\pgfpathlineto{\pgfqpoint{1.885331in}{0.857978in}}%
\pgfpathlineto{\pgfqpoint{1.889178in}{0.851064in}}%
\pgfpathlineto{\pgfqpoint{1.893486in}{0.814892in}}%
\pgfpathlineto{\pgfqpoint{1.898279in}{0.804188in}}%
\pgfpathlineto{\pgfqpoint{1.904902in}{0.803379in}}%
\pgfpathlineto{\pgfqpoint{1.907246in}{0.780181in}}%
\pgfpathlineto{\pgfqpoint{1.912932in}{0.774543in}}%
\pgfpathlineto{\pgfqpoint{1.923418in}{0.773495in}}%
\pgfpathlineto{\pgfqpoint{1.927003in}{0.764692in}}%
\pgfpathlineto{\pgfqpoint{1.930681in}{0.763152in}}%
\pgfpathlineto{\pgfqpoint{1.938397in}{0.758644in}}%
\pgfpathlineto{\pgfqpoint{1.944819in}{0.754750in}}%
\pgfpathlineto{\pgfqpoint{1.945246in}{0.752971in}}%
\pgfpathlineto{\pgfqpoint{1.949459in}{0.750264in}}%
\pgfpathlineto{\pgfqpoint{1.966417in}{0.745282in}}%
\pgfpathlineto{\pgfqpoint{1.984450in}{0.738340in}}%
\pgfpathlineto{\pgfqpoint{2.008926in}{0.733388in}}%
\pgfpathlineto{\pgfqpoint{2.031338in}{0.731724in}}%
\pgfpathlineto{\pgfqpoint{2.032252in}{0.729528in}}%
\pgfpathlineto{\pgfqpoint{2.033576in}{0.725697in}}%
\pgfpathlineto{\pgfqpoint{2.037125in}{0.724332in}}%
\pgfpathlineto{\pgfqpoint{2.045595in}{0.723715in}}%
\pgfpathlineto{\pgfqpoint{2.051349in}{0.722081in}}%
\pgfpathlineto{\pgfqpoint{2.076904in}{0.719544in}}%
\pgfpathlineto{\pgfqpoint{2.080251in}{0.718197in}}%
\pgfpathlineto{\pgfqpoint{2.080349in}{0.716749in}}%
\pgfpathlineto{\pgfqpoint{2.085843in}{0.716510in}}%
\pgfpathlineto{\pgfqpoint{2.098822in}{0.714079in}}%
\pgfpathlineto{\pgfqpoint{2.104964in}{0.713831in}}%
\pgfpathlineto{\pgfqpoint{2.105914in}{0.712924in}}%
\pgfpathlineto{\pgfqpoint{2.157055in}{0.710594in}}%
\pgfpathlineto{\pgfqpoint{2.161059in}{0.706043in}}%
\pgfpathlineto{\pgfqpoint{2.163334in}{0.705692in}}%
\pgfpathlineto{\pgfqpoint{2.171432in}{0.705082in}}%
\pgfpathlineto{\pgfqpoint{2.187458in}{0.702763in}}%
\pgfpathlineto{\pgfqpoint{2.193297in}{0.702172in}}%
\pgfpathlineto{\pgfqpoint{2.198690in}{0.701434in}}%
\pgfpathlineto{\pgfqpoint{2.212467in}{0.699905in}}%
\pgfpathlineto{\pgfqpoint{2.234540in}{0.699174in}}%
\pgfpathlineto{\pgfqpoint{2.259841in}{0.698842in}}%
\pgfpathlineto{\pgfqpoint{2.278129in}{0.698798in}}%
\pgfpathlineto{\pgfqpoint{2.301002in}{0.698784in}}%
\pgfpathlineto{\pgfqpoint{2.409554in}{0.697909in}}%
\pgfpathlineto{\pgfqpoint{2.438334in}{0.697795in}}%
\pgfpathlineto{\pgfqpoint{2.479468in}{0.697460in}}%
\pgfpathlineto{\pgfqpoint{2.539290in}{0.697351in}}%
\pgfpathlineto{\pgfqpoint{2.571667in}{0.696836in}}%
\pgfpathlineto{\pgfqpoint{2.647748in}{0.696667in}}%
\pgfpathlineto{\pgfqpoint{2.745840in}{0.696168in}}%
\pgfpathlineto{\pgfqpoint{2.823078in}{0.695324in}}%
\pgfpathlineto{\pgfqpoint{2.944612in}{0.694845in}}%
\pgfpathlineto{\pgfqpoint{3.108851in}{0.693891in}}%
\pgfpathlineto{\pgfqpoint{3.274341in}{0.693028in}}%
\pgfpathlineto{\pgfqpoint{3.517294in}{0.691835in}}%
\pgfpathlineto{\pgfqpoint{3.922083in}{0.690316in}}%
\pgfpathlineto{\pgfqpoint{4.623274in}{0.688107in}}%
\pgfpathlineto{\pgfqpoint{6.888192in}{0.683471in}}%
\pgfusepath{stroke}%
\end{pgfscope}%
\begin{pgfscope}%
\pgfsetrectcap%
\pgfsetmiterjoin%
\pgfsetlinewidth{0.803000pt}%
\definecolor{currentstroke}{rgb}{0.000000,0.000000,0.000000}%
\pgfsetstrokecolor{currentstroke}%
\pgfsetdash{}{0pt}%
\pgfpathmoveto{\pgfqpoint{0.688192in}{0.670138in}}%
\pgfpathlineto{\pgfqpoint{0.688192in}{5.290138in}}%
\pgfusepath{stroke}%
\end{pgfscope}%
\begin{pgfscope}%
\pgfsetrectcap%
\pgfsetmiterjoin%
\pgfsetlinewidth{0.803000pt}%
\definecolor{currentstroke}{rgb}{0.000000,0.000000,0.000000}%
\pgfsetstrokecolor{currentstroke}%
\pgfsetdash{}{0pt}%
\pgfpathmoveto{\pgfqpoint{6.888192in}{0.670138in}}%
\pgfpathlineto{\pgfqpoint{6.888192in}{5.290138in}}%
\pgfusepath{stroke}%
\end{pgfscope}%
\begin{pgfscope}%
\pgfsetrectcap%
\pgfsetmiterjoin%
\pgfsetlinewidth{0.803000pt}%
\definecolor{currentstroke}{rgb}{0.000000,0.000000,0.000000}%
\pgfsetstrokecolor{currentstroke}%
\pgfsetdash{}{0pt}%
\pgfpathmoveto{\pgfqpoint{0.688192in}{0.670138in}}%
\pgfpathlineto{\pgfqpoint{6.888192in}{0.670138in}}%
\pgfusepath{stroke}%
\end{pgfscope}%
\begin{pgfscope}%
\pgfsetrectcap%
\pgfsetmiterjoin%
\pgfsetlinewidth{0.803000pt}%
\definecolor{currentstroke}{rgb}{0.000000,0.000000,0.000000}%
\pgfsetstrokecolor{currentstroke}%
\pgfsetdash{}{0pt}%
\pgfpathmoveto{\pgfqpoint{0.688192in}{5.290138in}}%
\pgfpathlineto{\pgfqpoint{6.888192in}{5.290138in}}%
\pgfusepath{stroke}%
\end{pgfscope}%
\begin{pgfscope}%
\pgfsetbuttcap%
\pgfsetmiterjoin%
\pgfsetlinewidth{1.003750pt}%
\definecolor{currentstroke}{rgb}{0.000000,0.000000,0.000000}%
\pgfsetstrokecolor{currentstroke}%
\pgfsetstrokeopacity{0.500000}%
\pgfsetdash{}{0pt}%
\pgfpathmoveto{\pgfqpoint{0.646542in}{1.071430in}}%
\pgfpathlineto{\pgfqpoint{0.810103in}{1.071430in}}%
\pgfpathlineto{\pgfqpoint{0.810103in}{1.434970in}}%
\pgfpathlineto{\pgfqpoint{0.646542in}{1.434970in}}%
\pgfpathlineto{\pgfqpoint{0.646542in}{1.071430in}}%
\pgfpathclose%
\pgfpathmoveto{\pgfqpoint{3.788192in}{5.151538in}}%
\pgfpathquadraticcurveto{\pgfqpoint{2.217367in}{3.293254in}}{\pgfqpoint{0.646542in}{1.434970in}}%
\pgfpathmoveto{\pgfqpoint{6.702192in}{2.980138in}}%
\pgfpathquadraticcurveto{\pgfqpoint{3.756147in}{2.025784in}}{\pgfqpoint{0.810103in}{1.071430in}}%
\pgfusepath{stroke}%
\end{pgfscope}%
\begin{pgfscope}%
\pgfsetbuttcap%
\pgfsetmiterjoin%
\definecolor{currentfill}{rgb}{1.000000,1.000000,1.000000}%
\pgfsetfillcolor{currentfill}%
\pgfsetlinewidth{0.000000pt}%
\definecolor{currentstroke}{rgb}{0.000000,0.000000,0.000000}%
\pgfsetstrokecolor{currentstroke}%
\pgfsetstrokeopacity{0.000000}%
\pgfsetdash{}{0pt}%
\pgfpathmoveto{\pgfqpoint{3.788192in}{2.980138in}}%
\pgfpathlineto{\pgfqpoint{6.702192in}{2.980138in}}%
\pgfpathlineto{\pgfqpoint{6.702192in}{5.151538in}}%
\pgfpathlineto{\pgfqpoint{3.788192in}{5.151538in}}%
\pgfpathlineto{\pgfqpoint{3.788192in}{2.980138in}}%
\pgfpathclose%
\pgfusepath{fill}%
\end{pgfscope}%
\begin{pgfscope}%
\pgfpathrectangle{\pgfqpoint{3.788192in}{2.980138in}}{\pgfqpoint{2.914000in}{2.171400in}}%
\pgfusepath{clip}%
\pgfsetbuttcap%
\pgfsetmiterjoin%
\definecolor{currentfill}{rgb}{0.121569,0.466667,0.705882}%
\pgfsetfillcolor{currentfill}%
\pgfsetfillopacity{0.500000}%
\pgfsetlinewidth{1.003750pt}%
\definecolor{currentstroke}{rgb}{0.121569,0.466667,0.705882}%
\pgfsetstrokecolor{currentstroke}%
\pgfsetstrokeopacity{0.500000}%
\pgfsetdash{}{0pt}%
\pgfpathmoveto{\pgfqpoint{5.478622in}{4.808529in}}%
\pgfpathlineto{\pgfqpoint{5.786444in}{2.284897in}}%
\pgfpathlineto{\pgfqpoint{5.955602in}{1.807070in}}%
\pgfpathlineto{\pgfqpoint{6.072257in}{1.530809in}}%
\pgfpathlineto{\pgfqpoint{6.134561in}{1.493263in}}%
\pgfpathlineto{\pgfqpoint{6.204334in}{1.296852in}}%
\pgfpathlineto{\pgfqpoint{6.281957in}{1.238734in}}%
\pgfpathlineto{\pgfqpoint{6.389240in}{1.234341in}}%
\pgfpathlineto{\pgfqpoint{6.427201in}{1.108377in}}%
\pgfpathlineto{\pgfqpoint{6.519285in}{1.077759in}}%
\pgfpathlineto{\pgfqpoint{6.689124in}{1.072073in}}%
\pgfpathlineto{\pgfqpoint{6.747190in}{1.024269in}}%
\pgfpathlineto{\pgfqpoint{6.806750in}{1.015907in}}%
\pgfpathlineto{\pgfqpoint{6.931730in}{0.991431in}}%
\pgfpathlineto{\pgfqpoint{7.035734in}{0.970288in}}%
\pgfpathlineto{\pgfqpoint{7.042663in}{0.960624in}}%
\pgfpathlineto{\pgfqpoint{7.110896in}{0.945926in}}%
\pgfpathlineto{\pgfqpoint{7.385541in}{0.918874in}}%
\pgfpathlineto{\pgfqpoint{7.677622in}{0.881182in}}%
\pgfpathlineto{\pgfqpoint{8.074033in}{0.854291in}}%
\pgfpathlineto{\pgfqpoint{8.437037in}{0.845259in}}%
\pgfpathlineto{\pgfqpoint{8.451833in}{0.833333in}}%
\pgfpathlineto{\pgfqpoint{8.473270in}{0.812531in}}%
\pgfpathlineto{\pgfqpoint{8.530759in}{0.805118in}}%
\pgfpathlineto{\pgfqpoint{8.667946in}{0.801766in}}%
\pgfpathlineto{\pgfqpoint{8.761129in}{0.792896in}}%
\pgfpathlineto{\pgfqpoint{9.175032in}{0.779118in}}%
\pgfpathlineto{\pgfqpoint{9.229242in}{0.771806in}}%
\pgfpathlineto{\pgfqpoint{9.230836in}{0.763945in}}%
\pgfpathlineto{\pgfqpoint{9.319803in}{0.762647in}}%
\pgfpathlineto{\pgfqpoint{9.530028in}{0.749444in}}%
\pgfpathlineto{\pgfqpoint{9.629502in}{0.748101in}}%
\pgfpathlineto{\pgfqpoint{9.644888in}{0.743176in}}%
\pgfpathlineto{\pgfqpoint{10.473184in}{0.730520in}}%
\pgfpathlineto{\pgfqpoint{10.538033in}{0.705808in}}%
\pgfpathlineto{\pgfqpoint{10.574876in}{0.703903in}}%
\pgfpathlineto{\pgfqpoint{10.706030in}{0.700593in}}%
\pgfpathlineto{\pgfqpoint{10.965602in}{0.687998in}}%
\pgfpathlineto{\pgfqpoint{11.060169in}{0.684789in}}%
\pgfpathlineto{\pgfqpoint{11.147519in}{0.680781in}}%
\pgfpathlineto{\pgfqpoint{11.370645in}{0.672484in}}%
\pgfpathlineto{\pgfqpoint{11.728159in}{0.668514in}}%
\pgfpathlineto{\pgfqpoint{12.137942in}{0.666709in}}%
\pgfpathlineto{\pgfqpoint{12.434130in}{0.666468in}}%
\pgfpathlineto{\pgfqpoint{12.804602in}{0.666397in}}%
\pgfpathlineto{\pgfqpoint{14.562739in}{0.661643in}}%
\pgfpathlineto{\pgfqpoint{15.028873in}{0.661022in}}%
\pgfpathlineto{\pgfqpoint{15.695083in}{0.659204in}}%
\pgfpathlineto{\pgfqpoint{16.663977in}{0.658612in}}%
\pgfpathlineto{\pgfqpoint{17.188373in}{0.655818in}}%
\pgfpathlineto{\pgfqpoint{18.420605in}{0.654898in}}%
\pgfpathlineto{\pgfqpoint{20.009331in}{0.652189in}}%
\pgfpathlineto{\pgfqpoint{21.260300in}{0.647607in}}%
\pgfpathlineto{\pgfqpoint{23.228701in}{0.645006in}}%
\pgfpathlineto{\pgfqpoint{25.888770in}{0.639824in}}%
\pgfpathlineto{\pgfqpoint{28.569097in}{0.635141in}}%
\pgfpathlineto{\pgfqpoint{32.504050in}{0.628662in}}%
\pgfpathlineto{\pgfqpoint{39.060135in}{0.620415in}}%
\pgfpathlineto{\pgfqpoint{50.416853in}{0.608419in}}%
\pgfpathlineto{\pgfqpoint{87.100182in}{0.583248in}}%
\pgfpathlineto{\pgfqpoint{114.989116in}{0.662886in}}%
\pgfpathlineto{\pgfqpoint{74.637454in}{0.690574in}}%
\pgfpathlineto{\pgfqpoint{62.145064in}{0.703770in}}%
\pgfpathlineto{\pgfqpoint{54.933371in}{0.712842in}}%
\pgfpathlineto{\pgfqpoint{50.604922in}{0.719969in}}%
\pgfpathlineto{\pgfqpoint{47.656562in}{0.725121in}}%
\pgfpathlineto{\pgfqpoint{44.730486in}{0.730820in}}%
\pgfpathlineto{\pgfqpoint{42.565245in}{0.733682in}}%
\pgfpathlineto{\pgfqpoint{41.189180in}{0.738722in}}%
\pgfpathlineto{\pgfqpoint{39.441581in}{0.741702in}}%
\pgfpathlineto{\pgfqpoint{38.086125in}{0.742714in}}%
\pgfpathlineto{\pgfqpoint{37.509290in}{0.745787in}}%
\pgfpathlineto{\pgfqpoint{36.443507in}{0.746439in}}%
\pgfpathlineto{\pgfqpoint{35.710675in}{0.748438in}}%
\pgfpathlineto{\pgfqpoint{35.197928in}{0.749121in}}%
\pgfpathlineto{\pgfqpoint{33.263977in}{0.754350in}}%
\pgfpathlineto{\pgfqpoint{32.856458in}{0.754429in}}%
\pgfpathlineto{\pgfqpoint{32.530652in}{0.754694in}}%
\pgfpathlineto{\pgfqpoint{32.079890in}{0.756679in}}%
\pgfpathlineto{\pgfqpoint{31.686625in}{0.761046in}}%
\pgfpathlineto{\pgfqpoint{31.441186in}{0.770173in}}%
\pgfpathlineto{\pgfqpoint{31.345101in}{0.774582in}}%
\pgfpathlineto{\pgfqpoint{31.241077in}{0.778112in}}%
\pgfpathlineto{\pgfqpoint{30.955549in}{0.791966in}}%
\pgfpathlineto{\pgfqpoint{30.811278in}{0.795608in}}%
\pgfpathlineto{\pgfqpoint{30.770752in}{0.797703in}}%
\pgfpathlineto{\pgfqpoint{30.699417in}{0.824886in}}%
\pgfpathlineto{\pgfqpoint{29.788292in}{0.838808in}}%
\pgfpathlineto{\pgfqpoint{29.771368in}{0.844225in}}%
\pgfpathlineto{\pgfqpoint{29.661946in}{0.845702in}}%
\pgfpathlineto{\pgfqpoint{29.430699in}{0.860225in}}%
\pgfpathlineto{\pgfqpoint{29.332834in}{0.861654in}}%
\pgfpathlineto{\pgfqpoint{29.331081in}{0.870301in}}%
\pgfpathlineto{\pgfqpoint{29.271451in}{0.878344in}}%
\pgfpathlineto{\pgfqpoint{28.816157in}{0.893499in}}%
\pgfpathlineto{\pgfqpoint{28.713656in}{0.903257in}}%
\pgfpathlineto{\pgfqpoint{28.562750in}{0.906943in}}%
\pgfpathlineto{\pgfqpoint{28.499512in}{0.915098in}}%
\pgfpathlineto{\pgfqpoint{28.475931in}{0.937980in}}%
\pgfpathlineto{\pgfqpoint{28.459655in}{0.951098in}}%
\pgfpathlineto{\pgfqpoint{28.060351in}{0.961034in}}%
\pgfpathlineto{\pgfqpoint{27.624299in}{0.990614in}}%
\pgfpathlineto{\pgfqpoint{27.303010in}{1.032076in}}%
\pgfpathlineto{\pgfqpoint{27.000901in}{1.061832in}}%
\pgfpathlineto{\pgfqpoint{26.925844in}{1.078000in}}%
\pgfpathlineto{\pgfqpoint{26.918222in}{1.088630in}}%
\pgfpathlineto{\pgfqpoint{26.803818in}{1.111888in}}%
\pgfpathlineto{\pgfqpoint{26.666341in}{1.138811in}}%
\pgfpathlineto{\pgfqpoint{26.600824in}{1.148010in}}%
\pgfpathlineto{\pgfqpoint{26.536952in}{1.200594in}}%
\pgfpathlineto{\pgfqpoint{26.350128in}{1.206849in}}%
\pgfpathlineto{\pgfqpoint{26.248836in}{1.240528in}}%
\pgfpathlineto{\pgfqpoint{26.207079in}{1.379089in}}%
\pgfpathlineto{\pgfqpoint{26.089068in}{1.383921in}}%
\pgfpathlineto{\pgfqpoint{26.003683in}{1.447851in}}%
\pgfpathlineto{\pgfqpoint{25.926932in}{1.663903in}}%
\pgfpathlineto{\pgfqpoint{25.858398in}{1.705204in}}%
\pgfpathlineto{\pgfqpoint{25.730077in}{2.009091in}}%
\pgfpathlineto{\pgfqpoint{25.544004in}{2.534701in}}%
\pgfpathlineto{\pgfqpoint{25.205400in}{5.310696in}}%
\pgfpathlineto{\pgfqpoint{5.478622in}{4.808529in}}%
\pgfpathclose%
\pgfusepath{stroke,fill}%
\end{pgfscope}%
\begin{pgfscope}%
\pgfpathrectangle{\pgfqpoint{3.788192in}{2.980138in}}{\pgfqpoint{2.914000in}{2.171400in}}%
\pgfusepath{clip}%
\pgfsetbuttcap%
\pgfsetroundjoin%
\pgfsetlinewidth{1.003750pt}%
\definecolor{currentstroke}{rgb}{1.000000,0.000000,0.000000}%
\pgfsetstrokecolor{currentstroke}%
\pgfsetdash{}{0pt}%
\pgfpathmoveto{\pgfqpoint{16.861003in}{25.566105in}}%
\pgfpathcurveto{\pgfqpoint{16.869240in}{25.566105in}}{\pgfqpoint{16.877140in}{25.569377in}}{\pgfqpoint{16.882964in}{25.575201in}}%
\pgfpathcurveto{\pgfqpoint{16.888788in}{25.581025in}}{\pgfqpoint{16.892060in}{25.588925in}}{\pgfqpoint{16.892060in}{25.597161in}}%
\pgfpathcurveto{\pgfqpoint{16.892060in}{25.605397in}}{\pgfqpoint{16.888788in}{25.613297in}}{\pgfqpoint{16.882964in}{25.619121in}}%
\pgfpathcurveto{\pgfqpoint{16.877140in}{25.624945in}}{\pgfqpoint{16.869240in}{25.628218in}}{\pgfqpoint{16.861003in}{25.628218in}}%
\pgfpathcurveto{\pgfqpoint{16.852767in}{25.628218in}}{\pgfqpoint{16.844867in}{25.624945in}}{\pgfqpoint{16.839043in}{25.619121in}}%
\pgfpathcurveto{\pgfqpoint{16.833219in}{25.613297in}}{\pgfqpoint{16.829947in}{25.605397in}}{\pgfqpoint{16.829947in}{25.597161in}}%
\pgfpathcurveto{\pgfqpoint{16.829947in}{25.588925in}}{\pgfqpoint{16.833219in}{25.581025in}}{\pgfqpoint{16.839043in}{25.575201in}}%
\pgfpathcurveto{\pgfqpoint{16.844867in}{25.569377in}}{\pgfqpoint{16.852767in}{25.566105in}}{\pgfqpoint{16.861003in}{25.566105in}}%
\pgfusepath{stroke}%
\end{pgfscope}%
\begin{pgfscope}%
\pgfpathrectangle{\pgfqpoint{3.788192in}{2.980138in}}{\pgfqpoint{2.914000in}{2.171400in}}%
\pgfusepath{clip}%
\pgfsetbuttcap%
\pgfsetroundjoin%
\pgfsetlinewidth{1.003750pt}%
\definecolor{currentstroke}{rgb}{1.000000,0.000000,0.000000}%
\pgfsetstrokecolor{currentstroke}%
\pgfsetdash{}{0pt}%
\pgfpathmoveto{\pgfqpoint{11.934253in}{10.111217in}}%
\pgfpathcurveto{\pgfqpoint{11.942489in}{10.111217in}}{\pgfqpoint{11.950389in}{10.114489in}}{\pgfqpoint{11.956213in}{10.120313in}}%
\pgfpathcurveto{\pgfqpoint{11.962037in}{10.126137in}}{\pgfqpoint{11.965309in}{10.134037in}}{\pgfqpoint{11.965309in}{10.142274in}}%
\pgfpathcurveto{\pgfqpoint{11.965309in}{10.150510in}}{\pgfqpoint{11.962037in}{10.158410in}}{\pgfqpoint{11.956213in}{10.164234in}}%
\pgfpathcurveto{\pgfqpoint{11.950389in}{10.170058in}}{\pgfqpoint{11.942489in}{10.173330in}}{\pgfqpoint{11.934253in}{10.173330in}}%
\pgfpathcurveto{\pgfqpoint{11.926016in}{10.173330in}}{\pgfqpoint{11.918116in}{10.170058in}}{\pgfqpoint{11.912292in}{10.164234in}}%
\pgfpathcurveto{\pgfqpoint{11.906469in}{10.158410in}}{\pgfqpoint{11.903196in}{10.150510in}}{\pgfqpoint{11.903196in}{10.142274in}}%
\pgfpathcurveto{\pgfqpoint{11.903196in}{10.134037in}}{\pgfqpoint{11.906469in}{10.126137in}}{\pgfqpoint{11.912292in}{10.120313in}}%
\pgfpathcurveto{\pgfqpoint{11.918116in}{10.114489in}}{\pgfqpoint{11.926016in}{10.111217in}}{\pgfqpoint{11.934253in}{10.111217in}}%
\pgfusepath{stroke}%
\end{pgfscope}%
\begin{pgfscope}%
\pgfpathrectangle{\pgfqpoint{3.788192in}{2.980138in}}{\pgfqpoint{2.914000in}{2.171400in}}%
\pgfusepath{clip}%
\pgfsetbuttcap%
\pgfsetroundjoin%
\pgfsetlinewidth{1.003750pt}%
\definecolor{currentstroke}{rgb}{1.000000,0.000000,0.000000}%
\pgfsetstrokecolor{currentstroke}%
\pgfsetdash{}{0pt}%
\pgfpathmoveto{\pgfqpoint{12.691767in}{10.530438in}}%
\pgfpathcurveto{\pgfqpoint{12.700003in}{10.530438in}}{\pgfqpoint{12.707903in}{10.533710in}}{\pgfqpoint{12.713727in}{10.539534in}}%
\pgfpathcurveto{\pgfqpoint{12.719551in}{10.545358in}}{\pgfqpoint{12.722823in}{10.553258in}}{\pgfqpoint{12.722823in}{10.561494in}}%
\pgfpathcurveto{\pgfqpoint{12.722823in}{10.569730in}}{\pgfqpoint{12.719551in}{10.577630in}}{\pgfqpoint{12.713727in}{10.583454in}}%
\pgfpathcurveto{\pgfqpoint{12.707903in}{10.589278in}}{\pgfqpoint{12.700003in}{10.592551in}}{\pgfqpoint{12.691767in}{10.592551in}}%
\pgfpathcurveto{\pgfqpoint{12.683530in}{10.592551in}}{\pgfqpoint{12.675630in}{10.589278in}}{\pgfqpoint{12.669806in}{10.583454in}}%
\pgfpathcurveto{\pgfqpoint{12.663983in}{10.577630in}}{\pgfqpoint{12.660710in}{10.569730in}}{\pgfqpoint{12.660710in}{10.561494in}}%
\pgfpathcurveto{\pgfqpoint{12.660710in}{10.553258in}}{\pgfqpoint{12.663983in}{10.545358in}}{\pgfqpoint{12.669806in}{10.539534in}}%
\pgfpathcurveto{\pgfqpoint{12.675630in}{10.533710in}}{\pgfqpoint{12.683530in}{10.530438in}}{\pgfqpoint{12.691767in}{10.530438in}}%
\pgfusepath{stroke}%
\end{pgfscope}%
\begin{pgfscope}%
\pgfpathrectangle{\pgfqpoint{3.788192in}{2.980138in}}{\pgfqpoint{2.914000in}{2.171400in}}%
\pgfusepath{clip}%
\pgfsetbuttcap%
\pgfsetroundjoin%
\pgfsetlinewidth{1.003750pt}%
\definecolor{currentstroke}{rgb}{1.000000,0.000000,0.000000}%
\pgfsetstrokecolor{currentstroke}%
\pgfsetdash{}{0pt}%
\pgfpathmoveto{\pgfqpoint{15.144604in}{10.695451in}}%
\pgfpathcurveto{\pgfqpoint{15.152841in}{10.695451in}}{\pgfqpoint{15.160741in}{10.698723in}}{\pgfqpoint{15.166565in}{10.704547in}}%
\pgfpathcurveto{\pgfqpoint{15.172389in}{10.710371in}}{\pgfqpoint{15.175661in}{10.718271in}}{\pgfqpoint{15.175661in}{10.726507in}}%
\pgfpathcurveto{\pgfqpoint{15.175661in}{10.734743in}}{\pgfqpoint{15.172389in}{10.742643in}}{\pgfqpoint{15.166565in}{10.748467in}}%
\pgfpathcurveto{\pgfqpoint{15.160741in}{10.754291in}}{\pgfqpoint{15.152841in}{10.757564in}}{\pgfqpoint{15.144604in}{10.757564in}}%
\pgfpathcurveto{\pgfqpoint{15.136368in}{10.757564in}}{\pgfqpoint{15.128468in}{10.754291in}}{\pgfqpoint{15.122644in}{10.748467in}}%
\pgfpathcurveto{\pgfqpoint{15.116820in}{10.742643in}}{\pgfqpoint{15.113548in}{10.734743in}}{\pgfqpoint{15.113548in}{10.726507in}}%
\pgfpathcurveto{\pgfqpoint{15.113548in}{10.718271in}}{\pgfqpoint{15.116820in}{10.710371in}}{\pgfqpoint{15.122644in}{10.704547in}}%
\pgfpathcurveto{\pgfqpoint{15.128468in}{10.698723in}}{\pgfqpoint{15.136368in}{10.695451in}}{\pgfqpoint{15.144604in}{10.695451in}}%
\pgfusepath{stroke}%
\end{pgfscope}%
\begin{pgfscope}%
\pgfpathrectangle{\pgfqpoint{3.788192in}{2.980138in}}{\pgfqpoint{2.914000in}{2.171400in}}%
\pgfusepath{clip}%
\pgfsetbuttcap%
\pgfsetroundjoin%
\pgfsetlinewidth{1.003750pt}%
\definecolor{currentstroke}{rgb}{1.000000,0.000000,0.000000}%
\pgfsetstrokecolor{currentstroke}%
\pgfsetdash{}{0pt}%
\pgfpathmoveto{\pgfqpoint{13.542463in}{11.423191in}}%
\pgfpathcurveto{\pgfqpoint{13.550699in}{11.423191in}}{\pgfqpoint{13.558599in}{11.426464in}}{\pgfqpoint{13.564423in}{11.432288in}}%
\pgfpathcurveto{\pgfqpoint{13.570247in}{11.438112in}}{\pgfqpoint{13.573519in}{11.446012in}}{\pgfqpoint{13.573519in}{11.454248in}}%
\pgfpathcurveto{\pgfqpoint{13.573519in}{11.462484in}}{\pgfqpoint{13.570247in}{11.470384in}}{\pgfqpoint{13.564423in}{11.476208in}}%
\pgfpathcurveto{\pgfqpoint{13.558599in}{11.482032in}}{\pgfqpoint{13.550699in}{11.485304in}}{\pgfqpoint{13.542463in}{11.485304in}}%
\pgfpathcurveto{\pgfqpoint{13.534227in}{11.485304in}}{\pgfqpoint{13.526327in}{11.482032in}}{\pgfqpoint{13.520503in}{11.476208in}}%
\pgfpathcurveto{\pgfqpoint{13.514679in}{11.470384in}}{\pgfqpoint{13.511406in}{11.462484in}}{\pgfqpoint{13.511406in}{11.454248in}}%
\pgfpathcurveto{\pgfqpoint{13.511406in}{11.446012in}}{\pgfqpoint{13.514679in}{11.438112in}}{\pgfqpoint{13.520503in}{11.432288in}}%
\pgfpathcurveto{\pgfqpoint{13.526327in}{11.426464in}}{\pgfqpoint{13.534227in}{11.423191in}}{\pgfqpoint{13.542463in}{11.423191in}}%
\pgfusepath{stroke}%
\end{pgfscope}%
\begin{pgfscope}%
\pgfpathrectangle{\pgfqpoint{3.788192in}{2.980138in}}{\pgfqpoint{2.914000in}{2.171400in}}%
\pgfusepath{clip}%
\pgfsetbuttcap%
\pgfsetroundjoin%
\pgfsetlinewidth{1.003750pt}%
\definecolor{currentstroke}{rgb}{1.000000,0.000000,0.000000}%
\pgfsetstrokecolor{currentstroke}%
\pgfsetdash{}{0pt}%
\pgfpathmoveto{\pgfqpoint{12.332968in}{8.917460in}}%
\pgfpathcurveto{\pgfqpoint{12.341204in}{8.917460in}}{\pgfqpoint{12.349104in}{8.920732in}}{\pgfqpoint{12.354928in}{8.926556in}}%
\pgfpathcurveto{\pgfqpoint{12.360752in}{8.932380in}}{\pgfqpoint{12.364025in}{8.940280in}}{\pgfqpoint{12.364025in}{8.948517in}}%
\pgfpathcurveto{\pgfqpoint{12.364025in}{8.956753in}}{\pgfqpoint{12.360752in}{8.964653in}}{\pgfqpoint{12.354928in}{8.970477in}}%
\pgfpathcurveto{\pgfqpoint{12.349104in}{8.976301in}}{\pgfqpoint{12.341204in}{8.979573in}}{\pgfqpoint{12.332968in}{8.979573in}}%
\pgfpathcurveto{\pgfqpoint{12.324732in}{8.979573in}}{\pgfqpoint{12.316832in}{8.976301in}}{\pgfqpoint{12.311008in}{8.970477in}}%
\pgfpathcurveto{\pgfqpoint{12.305184in}{8.964653in}}{\pgfqpoint{12.301912in}{8.956753in}}{\pgfqpoint{12.301912in}{8.948517in}}%
\pgfpathcurveto{\pgfqpoint{12.301912in}{8.940280in}}{\pgfqpoint{12.305184in}{8.932380in}}{\pgfqpoint{12.311008in}{8.926556in}}%
\pgfpathcurveto{\pgfqpoint{12.316832in}{8.920732in}}{\pgfqpoint{12.324732in}{8.917460in}}{\pgfqpoint{12.332968in}{8.917460in}}%
\pgfusepath{stroke}%
\end{pgfscope}%
\begin{pgfscope}%
\pgfpathrectangle{\pgfqpoint{3.788192in}{2.980138in}}{\pgfqpoint{2.914000in}{2.171400in}}%
\pgfusepath{clip}%
\pgfsetbuttcap%
\pgfsetroundjoin%
\pgfsetlinewidth{1.003750pt}%
\definecolor{currentstroke}{rgb}{1.000000,0.000000,0.000000}%
\pgfsetstrokecolor{currentstroke}%
\pgfsetdash{}{0pt}%
\pgfpathmoveto{\pgfqpoint{11.399902in}{7.106716in}}%
\pgfpathcurveto{\pgfqpoint{11.408139in}{7.106716in}}{\pgfqpoint{11.416039in}{7.109988in}}{\pgfqpoint{11.421863in}{7.115812in}}%
\pgfpathcurveto{\pgfqpoint{11.427686in}{7.121636in}}{\pgfqpoint{11.430959in}{7.129536in}}{\pgfqpoint{11.430959in}{7.137773in}}%
\pgfpathcurveto{\pgfqpoint{11.430959in}{7.146009in}}{\pgfqpoint{11.427686in}{7.153909in}}{\pgfqpoint{11.421863in}{7.159733in}}%
\pgfpathcurveto{\pgfqpoint{11.416039in}{7.165557in}}{\pgfqpoint{11.408139in}{7.168829in}}{\pgfqpoint{11.399902in}{7.168829in}}%
\pgfpathcurveto{\pgfqpoint{11.391666in}{7.168829in}}{\pgfqpoint{11.383766in}{7.165557in}}{\pgfqpoint{11.377942in}{7.159733in}}%
\pgfpathcurveto{\pgfqpoint{11.372118in}{7.153909in}}{\pgfqpoint{11.368846in}{7.146009in}}{\pgfqpoint{11.368846in}{7.137773in}}%
\pgfpathcurveto{\pgfqpoint{11.368846in}{7.129536in}}{\pgfqpoint{11.372118in}{7.121636in}}{\pgfqpoint{11.377942in}{7.115812in}}%
\pgfpathcurveto{\pgfqpoint{11.383766in}{7.109988in}}{\pgfqpoint{11.391666in}{7.106716in}}{\pgfqpoint{11.399902in}{7.106716in}}%
\pgfusepath{stroke}%
\end{pgfscope}%
\begin{pgfscope}%
\pgfpathrectangle{\pgfqpoint{3.788192in}{2.980138in}}{\pgfqpoint{2.914000in}{2.171400in}}%
\pgfusepath{clip}%
\pgfsetbuttcap%
\pgfsetroundjoin%
\pgfsetlinewidth{1.003750pt}%
\definecolor{currentstroke}{rgb}{1.000000,0.000000,0.000000}%
\pgfsetstrokecolor{currentstroke}%
\pgfsetdash{}{0pt}%
\pgfpathmoveto{\pgfqpoint{11.356549in}{7.029726in}}%
\pgfpathcurveto{\pgfqpoint{11.364785in}{7.029726in}}{\pgfqpoint{11.372685in}{7.032998in}}{\pgfqpoint{11.378509in}{7.038822in}}%
\pgfpathcurveto{\pgfqpoint{11.384333in}{7.044646in}}{\pgfqpoint{11.387605in}{7.052546in}}{\pgfqpoint{11.387605in}{7.060782in}}%
\pgfpathcurveto{\pgfqpoint{11.387605in}{7.069018in}}{\pgfqpoint{11.384333in}{7.076918in}}{\pgfqpoint{11.378509in}{7.082742in}}%
\pgfpathcurveto{\pgfqpoint{11.372685in}{7.088566in}}{\pgfqpoint{11.364785in}{7.091839in}}{\pgfqpoint{11.356549in}{7.091839in}}%
\pgfpathcurveto{\pgfqpoint{11.348313in}{7.091839in}}{\pgfqpoint{11.340412in}{7.088566in}}{\pgfqpoint{11.334589in}{7.082742in}}%
\pgfpathcurveto{\pgfqpoint{11.328765in}{7.076918in}}{\pgfqpoint{11.325492in}{7.069018in}}{\pgfqpoint{11.325492in}{7.060782in}}%
\pgfpathcurveto{\pgfqpoint{11.325492in}{7.052546in}}{\pgfqpoint{11.328765in}{7.044646in}}{\pgfqpoint{11.334589in}{7.038822in}}%
\pgfpathcurveto{\pgfqpoint{11.340412in}{7.032998in}}{\pgfqpoint{11.348313in}{7.029726in}}{\pgfqpoint{11.356549in}{7.029726in}}%
\pgfusepath{stroke}%
\end{pgfscope}%
\begin{pgfscope}%
\pgfpathrectangle{\pgfqpoint{3.788192in}{2.980138in}}{\pgfqpoint{2.914000in}{2.171400in}}%
\pgfusepath{clip}%
\pgfsetbuttcap%
\pgfsetroundjoin%
\pgfsetlinewidth{1.003750pt}%
\definecolor{currentstroke}{rgb}{1.000000,0.000000,0.000000}%
\pgfsetstrokecolor{currentstroke}%
\pgfsetdash{}{0pt}%
\pgfpathmoveto{\pgfqpoint{13.232346in}{9.803683in}}%
\pgfpathcurveto{\pgfqpoint{13.240582in}{9.803683in}}{\pgfqpoint{13.248482in}{9.806956in}}{\pgfqpoint{13.254306in}{9.812780in}}%
\pgfpathcurveto{\pgfqpoint{13.260130in}{9.818604in}}{\pgfqpoint{13.263402in}{9.826504in}}{\pgfqpoint{13.263402in}{9.834740in}}%
\pgfpathcurveto{\pgfqpoint{13.263402in}{9.842976in}}{\pgfqpoint{13.260130in}{9.850876in}}{\pgfqpoint{13.254306in}{9.856700in}}%
\pgfpathcurveto{\pgfqpoint{13.248482in}{9.862524in}}{\pgfqpoint{13.240582in}{9.865796in}}{\pgfqpoint{13.232346in}{9.865796in}}%
\pgfpathcurveto{\pgfqpoint{13.224110in}{9.865796in}}{\pgfqpoint{13.216210in}{9.862524in}}{\pgfqpoint{13.210386in}{9.856700in}}%
\pgfpathcurveto{\pgfqpoint{13.204562in}{9.850876in}}{\pgfqpoint{13.201289in}{9.842976in}}{\pgfqpoint{13.201289in}{9.834740in}}%
\pgfpathcurveto{\pgfqpoint{13.201289in}{9.826504in}}{\pgfqpoint{13.204562in}{9.818604in}}{\pgfqpoint{13.210386in}{9.812780in}}%
\pgfpathcurveto{\pgfqpoint{13.216210in}{9.806956in}}{\pgfqpoint{13.224110in}{9.803683in}}{\pgfqpoint{13.232346in}{9.803683in}}%
\pgfusepath{stroke}%
\end{pgfscope}%
\begin{pgfscope}%
\pgfpathrectangle{\pgfqpoint{3.788192in}{2.980138in}}{\pgfqpoint{2.914000in}{2.171400in}}%
\pgfusepath{clip}%
\pgfsetbuttcap%
\pgfsetroundjoin%
\pgfsetlinewidth{1.003750pt}%
\definecolor{currentstroke}{rgb}{1.000000,0.000000,0.000000}%
\pgfsetstrokecolor{currentstroke}%
\pgfsetdash{}{0pt}%
\pgfpathmoveto{\pgfqpoint{12.940652in}{8.318719in}}%
\pgfpathcurveto{\pgfqpoint{12.948888in}{8.318719in}}{\pgfqpoint{12.956788in}{8.321992in}}{\pgfqpoint{12.962612in}{8.327816in}}%
\pgfpathcurveto{\pgfqpoint{12.968436in}{8.333640in}}{\pgfqpoint{12.971708in}{8.341540in}}{\pgfqpoint{12.971708in}{8.349776in}}%
\pgfpathcurveto{\pgfqpoint{12.971708in}{8.358012in}}{\pgfqpoint{12.968436in}{8.365912in}}{\pgfqpoint{12.962612in}{8.371736in}}%
\pgfpathcurveto{\pgfqpoint{12.956788in}{8.377560in}}{\pgfqpoint{12.948888in}{8.380832in}}{\pgfqpoint{12.940652in}{8.380832in}}%
\pgfpathcurveto{\pgfqpoint{12.932415in}{8.380832in}}{\pgfqpoint{12.924515in}{8.377560in}}{\pgfqpoint{12.918691in}{8.371736in}}%
\pgfpathcurveto{\pgfqpoint{12.912867in}{8.365912in}}{\pgfqpoint{12.909595in}{8.358012in}}{\pgfqpoint{12.909595in}{8.349776in}}%
\pgfpathcurveto{\pgfqpoint{12.909595in}{8.341540in}}{\pgfqpoint{12.912867in}{8.333640in}}{\pgfqpoint{12.918691in}{8.327816in}}%
\pgfpathcurveto{\pgfqpoint{12.924515in}{8.321992in}}{\pgfqpoint{12.932415in}{8.318719in}}{\pgfqpoint{12.940652in}{8.318719in}}%
\pgfusepath{stroke}%
\end{pgfscope}%
\begin{pgfscope}%
\pgfpathrectangle{\pgfqpoint{3.788192in}{2.980138in}}{\pgfqpoint{2.914000in}{2.171400in}}%
\pgfusepath{clip}%
\pgfsetbuttcap%
\pgfsetroundjoin%
\pgfsetlinewidth{1.003750pt}%
\definecolor{currentstroke}{rgb}{1.000000,0.000000,0.000000}%
\pgfsetstrokecolor{currentstroke}%
\pgfsetdash{}{0pt}%
\pgfpathmoveto{\pgfqpoint{12.936873in}{8.298012in}}%
\pgfpathcurveto{\pgfqpoint{12.945109in}{8.298012in}}{\pgfqpoint{12.953009in}{8.301284in}}{\pgfqpoint{12.958833in}{8.307108in}}%
\pgfpathcurveto{\pgfqpoint{12.964657in}{8.312932in}}{\pgfqpoint{12.967929in}{8.320832in}}{\pgfqpoint{12.967929in}{8.329068in}}%
\pgfpathcurveto{\pgfqpoint{12.967929in}{8.337305in}}{\pgfqpoint{12.964657in}{8.345205in}}{\pgfqpoint{12.958833in}{8.351028in}}%
\pgfpathcurveto{\pgfqpoint{12.953009in}{8.356852in}}{\pgfqpoint{12.945109in}{8.360125in}}{\pgfqpoint{12.936873in}{8.360125in}}%
\pgfpathcurveto{\pgfqpoint{12.928637in}{8.360125in}}{\pgfqpoint{12.920737in}{8.356852in}}{\pgfqpoint{12.914913in}{8.351028in}}%
\pgfpathcurveto{\pgfqpoint{12.909089in}{8.345205in}}{\pgfqpoint{12.905816in}{8.337305in}}{\pgfqpoint{12.905816in}{8.329068in}}%
\pgfpathcurveto{\pgfqpoint{12.905816in}{8.320832in}}{\pgfqpoint{12.909089in}{8.312932in}}{\pgfqpoint{12.914913in}{8.307108in}}%
\pgfpathcurveto{\pgfqpoint{12.920737in}{8.301284in}}{\pgfqpoint{12.928637in}{8.298012in}}{\pgfqpoint{12.936873in}{8.298012in}}%
\pgfusepath{stroke}%
\end{pgfscope}%
\begin{pgfscope}%
\pgfpathrectangle{\pgfqpoint{3.788192in}{2.980138in}}{\pgfqpoint{2.914000in}{2.171400in}}%
\pgfusepath{clip}%
\pgfsetbuttcap%
\pgfsetroundjoin%
\pgfsetlinewidth{1.003750pt}%
\definecolor{currentstroke}{rgb}{1.000000,0.000000,0.000000}%
\pgfsetstrokecolor{currentstroke}%
\pgfsetdash{}{0pt}%
\pgfpathmoveto{\pgfqpoint{12.900060in}{6.352602in}}%
\pgfpathcurveto{\pgfqpoint{12.908296in}{6.352602in}}{\pgfqpoint{12.916196in}{6.355874in}}{\pgfqpoint{12.922020in}{6.361698in}}%
\pgfpathcurveto{\pgfqpoint{12.927844in}{6.367522in}}{\pgfqpoint{12.931116in}{6.375422in}}{\pgfqpoint{12.931116in}{6.383658in}}%
\pgfpathcurveto{\pgfqpoint{12.931116in}{6.391894in}}{\pgfqpoint{12.927844in}{6.399794in}}{\pgfqpoint{12.922020in}{6.405618in}}%
\pgfpathcurveto{\pgfqpoint{12.916196in}{6.411442in}}{\pgfqpoint{12.908296in}{6.414715in}}{\pgfqpoint{12.900060in}{6.414715in}}%
\pgfpathcurveto{\pgfqpoint{12.891823in}{6.414715in}}{\pgfqpoint{12.883923in}{6.411442in}}{\pgfqpoint{12.878099in}{6.405618in}}%
\pgfpathcurveto{\pgfqpoint{12.872275in}{6.399794in}}{\pgfqpoint{12.869003in}{6.391894in}}{\pgfqpoint{12.869003in}{6.383658in}}%
\pgfpathcurveto{\pgfqpoint{12.869003in}{6.375422in}}{\pgfqpoint{12.872275in}{6.367522in}}{\pgfqpoint{12.878099in}{6.361698in}}%
\pgfpathcurveto{\pgfqpoint{12.883923in}{6.355874in}}{\pgfqpoint{12.891823in}{6.352602in}}{\pgfqpoint{12.900060in}{6.352602in}}%
\pgfusepath{stroke}%
\end{pgfscope}%
\begin{pgfscope}%
\pgfpathrectangle{\pgfqpoint{3.788192in}{2.980138in}}{\pgfqpoint{2.914000in}{2.171400in}}%
\pgfusepath{clip}%
\pgfsetbuttcap%
\pgfsetroundjoin%
\pgfsetlinewidth{1.003750pt}%
\definecolor{currentstroke}{rgb}{1.000000,0.000000,0.000000}%
\pgfsetstrokecolor{currentstroke}%
\pgfsetdash{}{0pt}%
\pgfpathmoveto{\pgfqpoint{12.861075in}{6.367856in}}%
\pgfpathcurveto{\pgfqpoint{12.869311in}{6.367856in}}{\pgfqpoint{12.877211in}{6.371128in}}{\pgfqpoint{12.883035in}{6.376952in}}%
\pgfpathcurveto{\pgfqpoint{12.888859in}{6.382776in}}{\pgfqpoint{12.892131in}{6.390676in}}{\pgfqpoint{12.892131in}{6.398913in}}%
\pgfpathcurveto{\pgfqpoint{12.892131in}{6.407149in}}{\pgfqpoint{12.888859in}{6.415049in}}{\pgfqpoint{12.883035in}{6.420873in}}%
\pgfpathcurveto{\pgfqpoint{12.877211in}{6.426697in}}{\pgfqpoint{12.869311in}{6.429969in}}{\pgfqpoint{12.861075in}{6.429969in}}%
\pgfpathcurveto{\pgfqpoint{12.852838in}{6.429969in}}{\pgfqpoint{12.844938in}{6.426697in}}{\pgfqpoint{12.839114in}{6.420873in}}%
\pgfpathcurveto{\pgfqpoint{12.833291in}{6.415049in}}{\pgfqpoint{12.830018in}{6.407149in}}{\pgfqpoint{12.830018in}{6.398913in}}%
\pgfpathcurveto{\pgfqpoint{12.830018in}{6.390676in}}{\pgfqpoint{12.833291in}{6.382776in}}{\pgfqpoint{12.839114in}{6.376952in}}%
\pgfpathcurveto{\pgfqpoint{12.844938in}{6.371128in}}{\pgfqpoint{12.852838in}{6.367856in}}{\pgfqpoint{12.861075in}{6.367856in}}%
\pgfusepath{stroke}%
\end{pgfscope}%
\begin{pgfscope}%
\pgfpathrectangle{\pgfqpoint{3.788192in}{2.980138in}}{\pgfqpoint{2.914000in}{2.171400in}}%
\pgfusepath{clip}%
\pgfsetbuttcap%
\pgfsetroundjoin%
\pgfsetlinewidth{1.003750pt}%
\definecolor{currentstroke}{rgb}{1.000000,0.000000,0.000000}%
\pgfsetstrokecolor{currentstroke}%
\pgfsetdash{}{0pt}%
\pgfpathmoveto{\pgfqpoint{15.347266in}{8.054762in}}%
\pgfpathcurveto{\pgfqpoint{15.355502in}{8.054762in}}{\pgfqpoint{15.363402in}{8.058034in}}{\pgfqpoint{15.369226in}{8.063858in}}%
\pgfpathcurveto{\pgfqpoint{15.375050in}{8.069682in}}{\pgfqpoint{15.378323in}{8.077582in}}{\pgfqpoint{15.378323in}{8.085818in}}%
\pgfpathcurveto{\pgfqpoint{15.378323in}{8.094054in}}{\pgfqpoint{15.375050in}{8.101954in}}{\pgfqpoint{15.369226in}{8.107778in}}%
\pgfpathcurveto{\pgfqpoint{15.363402in}{8.113602in}}{\pgfqpoint{15.355502in}{8.116875in}}{\pgfqpoint{15.347266in}{8.116875in}}%
\pgfpathcurveto{\pgfqpoint{15.339030in}{8.116875in}}{\pgfqpoint{15.331130in}{8.113602in}}{\pgfqpoint{15.325306in}{8.107778in}}%
\pgfpathcurveto{\pgfqpoint{15.319482in}{8.101954in}}{\pgfqpoint{15.316210in}{8.094054in}}{\pgfqpoint{15.316210in}{8.085818in}}%
\pgfpathcurveto{\pgfqpoint{15.316210in}{8.077582in}}{\pgfqpoint{15.319482in}{8.069682in}}{\pgfqpoint{15.325306in}{8.063858in}}%
\pgfpathcurveto{\pgfqpoint{15.331130in}{8.058034in}}{\pgfqpoint{15.339030in}{8.054762in}}{\pgfqpoint{15.347266in}{8.054762in}}%
\pgfusepath{stroke}%
\end{pgfscope}%
\begin{pgfscope}%
\pgfpathrectangle{\pgfqpoint{3.788192in}{2.980138in}}{\pgfqpoint{2.914000in}{2.171400in}}%
\pgfusepath{clip}%
\pgfsetbuttcap%
\pgfsetroundjoin%
\pgfsetlinewidth{1.003750pt}%
\definecolor{currentstroke}{rgb}{1.000000,0.000000,0.000000}%
\pgfsetstrokecolor{currentstroke}%
\pgfsetdash{}{0pt}%
\pgfpathmoveto{\pgfqpoint{12.945472in}{6.264272in}}%
\pgfpathcurveto{\pgfqpoint{12.953708in}{6.264272in}}{\pgfqpoint{12.961608in}{6.267545in}}{\pgfqpoint{12.967432in}{6.273369in}}%
\pgfpathcurveto{\pgfqpoint{12.973256in}{6.279193in}}{\pgfqpoint{12.976528in}{6.287093in}}{\pgfqpoint{12.976528in}{6.295329in}}%
\pgfpathcurveto{\pgfqpoint{12.976528in}{6.303565in}}{\pgfqpoint{12.973256in}{6.311465in}}{\pgfqpoint{12.967432in}{6.317289in}}%
\pgfpathcurveto{\pgfqpoint{12.961608in}{6.323113in}}{\pgfqpoint{12.953708in}{6.326385in}}{\pgfqpoint{12.945472in}{6.326385in}}%
\pgfpathcurveto{\pgfqpoint{12.937236in}{6.326385in}}{\pgfqpoint{12.929336in}{6.323113in}}{\pgfqpoint{12.923512in}{6.317289in}}%
\pgfpathcurveto{\pgfqpoint{12.917688in}{6.311465in}}{\pgfqpoint{12.914415in}{6.303565in}}{\pgfqpoint{12.914415in}{6.295329in}}%
\pgfpathcurveto{\pgfqpoint{12.914415in}{6.287093in}}{\pgfqpoint{12.917688in}{6.279193in}}{\pgfqpoint{12.923512in}{6.273369in}}%
\pgfpathcurveto{\pgfqpoint{12.929336in}{6.267545in}}{\pgfqpoint{12.937236in}{6.264272in}}{\pgfqpoint{12.945472in}{6.264272in}}%
\pgfusepath{stroke}%
\end{pgfscope}%
\begin{pgfscope}%
\pgfpathrectangle{\pgfqpoint{3.788192in}{2.980138in}}{\pgfqpoint{2.914000in}{2.171400in}}%
\pgfusepath{clip}%
\pgfsetbuttcap%
\pgfsetroundjoin%
\pgfsetlinewidth{1.003750pt}%
\definecolor{currentstroke}{rgb}{1.000000,0.000000,0.000000}%
\pgfsetstrokecolor{currentstroke}%
\pgfsetdash{}{0pt}%
\pgfpathmoveto{\pgfqpoint{14.816780in}{6.837016in}}%
\pgfpathcurveto{\pgfqpoint{14.825016in}{6.837016in}}{\pgfqpoint{14.832916in}{6.840288in}}{\pgfqpoint{14.838740in}{6.846112in}}%
\pgfpathcurveto{\pgfqpoint{14.844564in}{6.851936in}}{\pgfqpoint{14.847836in}{6.859836in}}{\pgfqpoint{14.847836in}{6.868073in}}%
\pgfpathcurveto{\pgfqpoint{14.847836in}{6.876309in}}{\pgfqpoint{14.844564in}{6.884209in}}{\pgfqpoint{14.838740in}{6.890033in}}%
\pgfpathcurveto{\pgfqpoint{14.832916in}{6.895857in}}{\pgfqpoint{14.825016in}{6.899129in}}{\pgfqpoint{14.816780in}{6.899129in}}%
\pgfpathcurveto{\pgfqpoint{14.808543in}{6.899129in}}{\pgfqpoint{14.800643in}{6.895857in}}{\pgfqpoint{14.794819in}{6.890033in}}%
\pgfpathcurveto{\pgfqpoint{14.788995in}{6.884209in}}{\pgfqpoint{14.785723in}{6.876309in}}{\pgfqpoint{14.785723in}{6.868073in}}%
\pgfpathcurveto{\pgfqpoint{14.785723in}{6.859836in}}{\pgfqpoint{14.788995in}{6.851936in}}{\pgfqpoint{14.794819in}{6.846112in}}%
\pgfpathcurveto{\pgfqpoint{14.800643in}{6.840288in}}{\pgfqpoint{14.808543in}{6.837016in}}{\pgfqpoint{14.816780in}{6.837016in}}%
\pgfusepath{stroke}%
\end{pgfscope}%
\begin{pgfscope}%
\pgfpathrectangle{\pgfqpoint{3.788192in}{2.980138in}}{\pgfqpoint{2.914000in}{2.171400in}}%
\pgfusepath{clip}%
\pgfsetbuttcap%
\pgfsetroundjoin%
\pgfsetlinewidth{1.003750pt}%
\definecolor{currentstroke}{rgb}{1.000000,0.000000,0.000000}%
\pgfsetstrokecolor{currentstroke}%
\pgfsetdash{}{0pt}%
\pgfpathmoveto{\pgfqpoint{15.389681in}{6.512387in}}%
\pgfpathcurveto{\pgfqpoint{15.397917in}{6.512387in}}{\pgfqpoint{15.405817in}{6.515659in}}{\pgfqpoint{15.411641in}{6.521483in}}%
\pgfpathcurveto{\pgfqpoint{15.417465in}{6.527307in}}{\pgfqpoint{15.420738in}{6.535207in}}{\pgfqpoint{15.420738in}{6.543443in}}%
\pgfpathcurveto{\pgfqpoint{15.420738in}{6.551679in}}{\pgfqpoint{15.417465in}{6.559579in}}{\pgfqpoint{15.411641in}{6.565403in}}%
\pgfpathcurveto{\pgfqpoint{15.405817in}{6.571227in}}{\pgfqpoint{15.397917in}{6.574500in}}{\pgfqpoint{15.389681in}{6.574500in}}%
\pgfpathcurveto{\pgfqpoint{15.381445in}{6.574500in}}{\pgfqpoint{15.373545in}{6.571227in}}{\pgfqpoint{15.367721in}{6.565403in}}%
\pgfpathcurveto{\pgfqpoint{15.361897in}{6.559579in}}{\pgfqpoint{15.358625in}{6.551679in}}{\pgfqpoint{15.358625in}{6.543443in}}%
\pgfpathcurveto{\pgfqpoint{15.358625in}{6.535207in}}{\pgfqpoint{15.361897in}{6.527307in}}{\pgfqpoint{15.367721in}{6.521483in}}%
\pgfpathcurveto{\pgfqpoint{15.373545in}{6.515659in}}{\pgfqpoint{15.381445in}{6.512387in}}{\pgfqpoint{15.389681in}{6.512387in}}%
\pgfusepath{stroke}%
\end{pgfscope}%
\begin{pgfscope}%
\pgfpathrectangle{\pgfqpoint{3.788192in}{2.980138in}}{\pgfqpoint{2.914000in}{2.171400in}}%
\pgfusepath{clip}%
\pgfsetbuttcap%
\pgfsetroundjoin%
\pgfsetlinewidth{1.003750pt}%
\definecolor{currentstroke}{rgb}{1.000000,0.000000,0.000000}%
\pgfsetstrokecolor{currentstroke}%
\pgfsetdash{}{0pt}%
\pgfpathmoveto{\pgfqpoint{12.707501in}{7.095383in}}%
\pgfpathcurveto{\pgfqpoint{12.715737in}{7.095383in}}{\pgfqpoint{12.723637in}{7.098656in}}{\pgfqpoint{12.729461in}{7.104479in}}%
\pgfpathcurveto{\pgfqpoint{12.735285in}{7.110303in}}{\pgfqpoint{12.738557in}{7.118203in}}{\pgfqpoint{12.738557in}{7.126440in}}%
\pgfpathcurveto{\pgfqpoint{12.738557in}{7.134676in}}{\pgfqpoint{12.735285in}{7.142576in}}{\pgfqpoint{12.729461in}{7.148400in}}%
\pgfpathcurveto{\pgfqpoint{12.723637in}{7.154224in}}{\pgfqpoint{12.715737in}{7.157496in}}{\pgfqpoint{12.707501in}{7.157496in}}%
\pgfpathcurveto{\pgfqpoint{12.699265in}{7.157496in}}{\pgfqpoint{12.691365in}{7.154224in}}{\pgfqpoint{12.685541in}{7.148400in}}%
\pgfpathcurveto{\pgfqpoint{12.679717in}{7.142576in}}{\pgfqpoint{12.676444in}{7.134676in}}{\pgfqpoint{12.676444in}{7.126440in}}%
\pgfpathcurveto{\pgfqpoint{12.676444in}{7.118203in}}{\pgfqpoint{12.679717in}{7.110303in}}{\pgfqpoint{12.685541in}{7.104479in}}%
\pgfpathcurveto{\pgfqpoint{12.691365in}{7.098656in}}{\pgfqpoint{12.699265in}{7.095383in}}{\pgfqpoint{12.707501in}{7.095383in}}%
\pgfusepath{stroke}%
\end{pgfscope}%
\begin{pgfscope}%
\pgfpathrectangle{\pgfqpoint{3.788192in}{2.980138in}}{\pgfqpoint{2.914000in}{2.171400in}}%
\pgfusepath{clip}%
\pgfsetbuttcap%
\pgfsetroundjoin%
\pgfsetlinewidth{1.003750pt}%
\definecolor{currentstroke}{rgb}{1.000000,0.000000,0.000000}%
\pgfsetstrokecolor{currentstroke}%
\pgfsetdash{}{0pt}%
\pgfpathmoveto{\pgfqpoint{14.362942in}{5.990566in}}%
\pgfpathcurveto{\pgfqpoint{14.371178in}{5.990566in}}{\pgfqpoint{14.379078in}{5.993838in}}{\pgfqpoint{14.384902in}{5.999662in}}%
\pgfpathcurveto{\pgfqpoint{14.390726in}{6.005486in}}{\pgfqpoint{14.393998in}{6.013386in}}{\pgfqpoint{14.393998in}{6.021622in}}%
\pgfpathcurveto{\pgfqpoint{14.393998in}{6.029859in}}{\pgfqpoint{14.390726in}{6.037759in}}{\pgfqpoint{14.384902in}{6.043583in}}%
\pgfpathcurveto{\pgfqpoint{14.379078in}{6.049407in}}{\pgfqpoint{14.371178in}{6.052679in}}{\pgfqpoint{14.362942in}{6.052679in}}%
\pgfpathcurveto{\pgfqpoint{14.354705in}{6.052679in}}{\pgfqpoint{14.346805in}{6.049407in}}{\pgfqpoint{14.340982in}{6.043583in}}%
\pgfpathcurveto{\pgfqpoint{14.335158in}{6.037759in}}{\pgfqpoint{14.331885in}{6.029859in}}{\pgfqpoint{14.331885in}{6.021622in}}%
\pgfpathcurveto{\pgfqpoint{14.331885in}{6.013386in}}{\pgfqpoint{14.335158in}{6.005486in}}{\pgfqpoint{14.340982in}{5.999662in}}%
\pgfpathcurveto{\pgfqpoint{14.346805in}{5.993838in}}{\pgfqpoint{14.354705in}{5.990566in}}{\pgfqpoint{14.362942in}{5.990566in}}%
\pgfusepath{stroke}%
\end{pgfscope}%
\begin{pgfscope}%
\pgfpathrectangle{\pgfqpoint{3.788192in}{2.980138in}}{\pgfqpoint{2.914000in}{2.171400in}}%
\pgfusepath{clip}%
\pgfsetbuttcap%
\pgfsetroundjoin%
\pgfsetlinewidth{1.003750pt}%
\definecolor{currentstroke}{rgb}{1.000000,0.000000,0.000000}%
\pgfsetstrokecolor{currentstroke}%
\pgfsetdash{}{0pt}%
\pgfpathmoveto{\pgfqpoint{14.985061in}{5.541293in}}%
\pgfpathcurveto{\pgfqpoint{14.993297in}{5.541293in}}{\pgfqpoint{15.001197in}{5.544566in}}{\pgfqpoint{15.007021in}{5.550390in}}%
\pgfpathcurveto{\pgfqpoint{15.012845in}{5.556213in}}{\pgfqpoint{15.016118in}{5.564113in}}{\pgfqpoint{15.016118in}{5.572350in}}%
\pgfpathcurveto{\pgfqpoint{15.016118in}{5.580586in}}{\pgfqpoint{15.012845in}{5.588486in}}{\pgfqpoint{15.007021in}{5.594310in}}%
\pgfpathcurveto{\pgfqpoint{15.001197in}{5.600134in}}{\pgfqpoint{14.993297in}{5.603406in}}{\pgfqpoint{14.985061in}{5.603406in}}%
\pgfpathcurveto{\pgfqpoint{14.976825in}{5.603406in}}{\pgfqpoint{14.968925in}{5.600134in}}{\pgfqpoint{14.963101in}{5.594310in}}%
\pgfpathcurveto{\pgfqpoint{14.957277in}{5.588486in}}{\pgfqpoint{14.954005in}{5.580586in}}{\pgfqpoint{14.954005in}{5.572350in}}%
\pgfpathcurveto{\pgfqpoint{14.954005in}{5.564113in}}{\pgfqpoint{14.957277in}{5.556213in}}{\pgfqpoint{14.963101in}{5.550390in}}%
\pgfpathcurveto{\pgfqpoint{14.968925in}{5.544566in}}{\pgfqpoint{14.976825in}{5.541293in}}{\pgfqpoint{14.985061in}{5.541293in}}%
\pgfusepath{stroke}%
\end{pgfscope}%
\begin{pgfscope}%
\pgfpathrectangle{\pgfqpoint{3.788192in}{2.980138in}}{\pgfqpoint{2.914000in}{2.171400in}}%
\pgfusepath{clip}%
\pgfsetbuttcap%
\pgfsetroundjoin%
\pgfsetlinewidth{1.003750pt}%
\definecolor{currentstroke}{rgb}{1.000000,0.000000,0.000000}%
\pgfsetstrokecolor{currentstroke}%
\pgfsetdash{}{0pt}%
\pgfpathmoveto{\pgfqpoint{14.968190in}{5.733414in}}%
\pgfpathcurveto{\pgfqpoint{14.976426in}{5.733414in}}{\pgfqpoint{14.984326in}{5.736686in}}{\pgfqpoint{14.990150in}{5.742510in}}%
\pgfpathcurveto{\pgfqpoint{14.995974in}{5.748334in}}{\pgfqpoint{14.999246in}{5.756234in}}{\pgfqpoint{14.999246in}{5.764470in}}%
\pgfpathcurveto{\pgfqpoint{14.999246in}{5.772706in}}{\pgfqpoint{14.995974in}{5.780606in}}{\pgfqpoint{14.990150in}{5.786430in}}%
\pgfpathcurveto{\pgfqpoint{14.984326in}{5.792254in}}{\pgfqpoint{14.976426in}{5.795527in}}{\pgfqpoint{14.968190in}{5.795527in}}%
\pgfpathcurveto{\pgfqpoint{14.959954in}{5.795527in}}{\pgfqpoint{14.952053in}{5.792254in}}{\pgfqpoint{14.946230in}{5.786430in}}%
\pgfpathcurveto{\pgfqpoint{14.940406in}{5.780606in}}{\pgfqpoint{14.937133in}{5.772706in}}{\pgfqpoint{14.937133in}{5.764470in}}%
\pgfpathcurveto{\pgfqpoint{14.937133in}{5.756234in}}{\pgfqpoint{14.940406in}{5.748334in}}{\pgfqpoint{14.946230in}{5.742510in}}%
\pgfpathcurveto{\pgfqpoint{14.952053in}{5.736686in}}{\pgfqpoint{14.959954in}{5.733414in}}{\pgfqpoint{14.968190in}{5.733414in}}%
\pgfusepath{stroke}%
\end{pgfscope}%
\begin{pgfscope}%
\pgfpathrectangle{\pgfqpoint{3.788192in}{2.980138in}}{\pgfqpoint{2.914000in}{2.171400in}}%
\pgfusepath{clip}%
\pgfsetbuttcap%
\pgfsetroundjoin%
\pgfsetlinewidth{1.003750pt}%
\definecolor{currentstroke}{rgb}{1.000000,0.000000,0.000000}%
\pgfsetstrokecolor{currentstroke}%
\pgfsetdash{}{0pt}%
\pgfpathmoveto{\pgfqpoint{12.713917in}{7.026179in}}%
\pgfpathcurveto{\pgfqpoint{12.722154in}{7.026179in}}{\pgfqpoint{12.730054in}{7.029451in}}{\pgfqpoint{12.735878in}{7.035275in}}%
\pgfpathcurveto{\pgfqpoint{12.741701in}{7.041099in}}{\pgfqpoint{12.744974in}{7.048999in}}{\pgfqpoint{12.744974in}{7.057235in}}%
\pgfpathcurveto{\pgfqpoint{12.744974in}{7.065472in}}{\pgfqpoint{12.741701in}{7.073372in}}{\pgfqpoint{12.735878in}{7.079196in}}%
\pgfpathcurveto{\pgfqpoint{12.730054in}{7.085019in}}{\pgfqpoint{12.722154in}{7.088292in}}{\pgfqpoint{12.713917in}{7.088292in}}%
\pgfpathcurveto{\pgfqpoint{12.705681in}{7.088292in}}{\pgfqpoint{12.697781in}{7.085019in}}{\pgfqpoint{12.691957in}{7.079196in}}%
\pgfpathcurveto{\pgfqpoint{12.686133in}{7.073372in}}{\pgfqpoint{12.682861in}{7.065472in}}{\pgfqpoint{12.682861in}{7.057235in}}%
\pgfpathcurveto{\pgfqpoint{12.682861in}{7.048999in}}{\pgfqpoint{12.686133in}{7.041099in}}{\pgfqpoint{12.691957in}{7.035275in}}%
\pgfpathcurveto{\pgfqpoint{12.697781in}{7.029451in}}{\pgfqpoint{12.705681in}{7.026179in}}{\pgfqpoint{12.713917in}{7.026179in}}%
\pgfusepath{stroke}%
\end{pgfscope}%
\begin{pgfscope}%
\pgfpathrectangle{\pgfqpoint{3.788192in}{2.980138in}}{\pgfqpoint{2.914000in}{2.171400in}}%
\pgfusepath{clip}%
\pgfsetbuttcap%
\pgfsetroundjoin%
\pgfsetlinewidth{1.003750pt}%
\definecolor{currentstroke}{rgb}{1.000000,0.000000,0.000000}%
\pgfsetstrokecolor{currentstroke}%
\pgfsetdash{}{0pt}%
\pgfpathmoveto{\pgfqpoint{13.558784in}{8.054051in}}%
\pgfpathcurveto{\pgfqpoint{13.567020in}{8.054051in}}{\pgfqpoint{13.574920in}{8.057323in}}{\pgfqpoint{13.580744in}{8.063147in}}%
\pgfpathcurveto{\pgfqpoint{13.586568in}{8.068971in}}{\pgfqpoint{13.589840in}{8.076871in}}{\pgfqpoint{13.589840in}{8.085108in}}%
\pgfpathcurveto{\pgfqpoint{13.589840in}{8.093344in}}{\pgfqpoint{13.586568in}{8.101244in}}{\pgfqpoint{13.580744in}{8.107068in}}%
\pgfpathcurveto{\pgfqpoint{13.574920in}{8.112892in}}{\pgfqpoint{13.567020in}{8.116164in}}{\pgfqpoint{13.558784in}{8.116164in}}%
\pgfpathcurveto{\pgfqpoint{13.550548in}{8.116164in}}{\pgfqpoint{13.542648in}{8.112892in}}{\pgfqpoint{13.536824in}{8.107068in}}%
\pgfpathcurveto{\pgfqpoint{13.531000in}{8.101244in}}{\pgfqpoint{13.527727in}{8.093344in}}{\pgfqpoint{13.527727in}{8.085108in}}%
\pgfpathcurveto{\pgfqpoint{13.527727in}{8.076871in}}{\pgfqpoint{13.531000in}{8.068971in}}{\pgfqpoint{13.536824in}{8.063147in}}%
\pgfpathcurveto{\pgfqpoint{13.542648in}{8.057323in}}{\pgfqpoint{13.550548in}{8.054051in}}{\pgfqpoint{13.558784in}{8.054051in}}%
\pgfusepath{stroke}%
\end{pgfscope}%
\begin{pgfscope}%
\pgfpathrectangle{\pgfqpoint{3.788192in}{2.980138in}}{\pgfqpoint{2.914000in}{2.171400in}}%
\pgfusepath{clip}%
\pgfsetbuttcap%
\pgfsetroundjoin%
\pgfsetlinewidth{1.003750pt}%
\definecolor{currentstroke}{rgb}{1.000000,0.000000,0.000000}%
\pgfsetstrokecolor{currentstroke}%
\pgfsetdash{}{0pt}%
\pgfpathmoveto{\pgfqpoint{13.021061in}{8.119261in}}%
\pgfpathcurveto{\pgfqpoint{13.029297in}{8.119261in}}{\pgfqpoint{13.037197in}{8.122533in}}{\pgfqpoint{13.043021in}{8.128357in}}%
\pgfpathcurveto{\pgfqpoint{13.048845in}{8.134181in}}{\pgfqpoint{13.052117in}{8.142081in}}{\pgfqpoint{13.052117in}{8.150317in}}%
\pgfpathcurveto{\pgfqpoint{13.052117in}{8.158554in}}{\pgfqpoint{13.048845in}{8.166454in}}{\pgfqpoint{13.043021in}{8.172278in}}%
\pgfpathcurveto{\pgfqpoint{13.037197in}{8.178102in}}{\pgfqpoint{13.029297in}{8.181374in}}{\pgfqpoint{13.021061in}{8.181374in}}%
\pgfpathcurveto{\pgfqpoint{13.012825in}{8.181374in}}{\pgfqpoint{13.004925in}{8.178102in}}{\pgfqpoint{12.999101in}{8.172278in}}%
\pgfpathcurveto{\pgfqpoint{12.993277in}{8.166454in}}{\pgfqpoint{12.990004in}{8.158554in}}{\pgfqpoint{12.990004in}{8.150317in}}%
\pgfpathcurveto{\pgfqpoint{12.990004in}{8.142081in}}{\pgfqpoint{12.993277in}{8.134181in}}{\pgfqpoint{12.999101in}{8.128357in}}%
\pgfpathcurveto{\pgfqpoint{13.004925in}{8.122533in}}{\pgfqpoint{13.012825in}{8.119261in}}{\pgfqpoint{13.021061in}{8.119261in}}%
\pgfusepath{stroke}%
\end{pgfscope}%
\begin{pgfscope}%
\pgfpathrectangle{\pgfqpoint{3.788192in}{2.980138in}}{\pgfqpoint{2.914000in}{2.171400in}}%
\pgfusepath{clip}%
\pgfsetbuttcap%
\pgfsetroundjoin%
\pgfsetlinewidth{1.003750pt}%
\definecolor{currentstroke}{rgb}{1.000000,0.000000,0.000000}%
\pgfsetstrokecolor{currentstroke}%
\pgfsetdash{}{0pt}%
\pgfpathmoveto{\pgfqpoint{12.649080in}{7.078889in}}%
\pgfpathcurveto{\pgfqpoint{12.657316in}{7.078889in}}{\pgfqpoint{12.665216in}{7.082161in}}{\pgfqpoint{12.671040in}{7.087985in}}%
\pgfpathcurveto{\pgfqpoint{12.676864in}{7.093809in}}{\pgfqpoint{12.680136in}{7.101709in}}{\pgfqpoint{12.680136in}{7.109946in}}%
\pgfpathcurveto{\pgfqpoint{12.680136in}{7.118182in}}{\pgfqpoint{12.676864in}{7.126082in}}{\pgfqpoint{12.671040in}{7.131906in}}%
\pgfpathcurveto{\pgfqpoint{12.665216in}{7.137730in}}{\pgfqpoint{12.657316in}{7.141002in}}{\pgfqpoint{12.649080in}{7.141002in}}%
\pgfpathcurveto{\pgfqpoint{12.640843in}{7.141002in}}{\pgfqpoint{12.632943in}{7.137730in}}{\pgfqpoint{12.627119in}{7.131906in}}%
\pgfpathcurveto{\pgfqpoint{12.621295in}{7.126082in}}{\pgfqpoint{12.618023in}{7.118182in}}{\pgfqpoint{12.618023in}{7.109946in}}%
\pgfpathcurveto{\pgfqpoint{12.618023in}{7.101709in}}{\pgfqpoint{12.621295in}{7.093809in}}{\pgfqpoint{12.627119in}{7.087985in}}%
\pgfpathcurveto{\pgfqpoint{12.632943in}{7.082161in}}{\pgfqpoint{12.640843in}{7.078889in}}{\pgfqpoint{12.649080in}{7.078889in}}%
\pgfusepath{stroke}%
\end{pgfscope}%
\begin{pgfscope}%
\pgfpathrectangle{\pgfqpoint{3.788192in}{2.980138in}}{\pgfqpoint{2.914000in}{2.171400in}}%
\pgfusepath{clip}%
\pgfsetbuttcap%
\pgfsetroundjoin%
\pgfsetlinewidth{1.003750pt}%
\definecolor{currentstroke}{rgb}{1.000000,0.000000,0.000000}%
\pgfsetstrokecolor{currentstroke}%
\pgfsetdash{}{0pt}%
\pgfpathmoveto{\pgfqpoint{4.530215in}{4.056056in}}%
\pgfpathcurveto{\pgfqpoint{4.538452in}{4.056056in}}{\pgfqpoint{4.546352in}{4.059328in}}{\pgfqpoint{4.552176in}{4.065152in}}%
\pgfpathcurveto{\pgfqpoint{4.557999in}{4.070976in}}{\pgfqpoint{4.561272in}{4.078876in}}{\pgfqpoint{4.561272in}{4.087112in}}%
\pgfpathcurveto{\pgfqpoint{4.561272in}{4.095349in}}{\pgfqpoint{4.557999in}{4.103249in}}{\pgfqpoint{4.552176in}{4.109073in}}%
\pgfpathcurveto{\pgfqpoint{4.546352in}{4.114897in}}{\pgfqpoint{4.538452in}{4.118169in}}{\pgfqpoint{4.530215in}{4.118169in}}%
\pgfpathcurveto{\pgfqpoint{4.521979in}{4.118169in}}{\pgfqpoint{4.514079in}{4.114897in}}{\pgfqpoint{4.508255in}{4.109073in}}%
\pgfpathcurveto{\pgfqpoint{4.502431in}{4.103249in}}{\pgfqpoint{4.499159in}{4.095349in}}{\pgfqpoint{4.499159in}{4.087112in}}%
\pgfpathcurveto{\pgfqpoint{4.499159in}{4.078876in}}{\pgfqpoint{4.502431in}{4.070976in}}{\pgfqpoint{4.508255in}{4.065152in}}%
\pgfpathcurveto{\pgfqpoint{4.514079in}{4.059328in}}{\pgfqpoint{4.521979in}{4.056056in}}{\pgfqpoint{4.530215in}{4.056056in}}%
\pgfpathlineto{\pgfqpoint{4.530215in}{4.056056in}}%
\pgfpathclose%
\pgfusepath{stroke}%
\end{pgfscope}%
\begin{pgfscope}%
\pgfpathrectangle{\pgfqpoint{3.788192in}{2.980138in}}{\pgfqpoint{2.914000in}{2.171400in}}%
\pgfusepath{clip}%
\pgfsetbuttcap%
\pgfsetmiterjoin%
\definecolor{currentfill}{rgb}{0.839216,0.152941,0.156863}%
\pgfsetfillcolor{currentfill}%
\pgfsetfillopacity{0.200000}%
\pgfsetlinewidth{1.003750pt}%
\definecolor{currentstroke}{rgb}{0.839216,0.152941,0.156863}%
\pgfsetstrokecolor{currentstroke}%
\pgfsetstrokeopacity{0.200000}%
\pgfsetdash{}{0pt}%
\pgfpathmoveto{\pgfqpoint{4.530215in}{2.980138in}}%
\pgfpathlineto{\pgfqpoint{24.162152in}{2.980138in}}%
\pgfpathlineto{\pgfqpoint{24.162152in}{5.151538in}}%
\pgfpathlineto{\pgfqpoint{4.530215in}{5.151538in}}%
\pgfpathlineto{\pgfqpoint{4.530215in}{2.980138in}}%
\pgfpathclose%
\pgfusepath{stroke,fill}%
\end{pgfscope}%
\begin{pgfscope}%
\pgfsetbuttcap%
\pgfsetmiterjoin%
\definecolor{currentfill}{rgb}{0.839216,0.152941,0.156863}%
\pgfsetfillcolor{currentfill}%
\pgfsetfillopacity{0.200000}%
\pgfsetlinewidth{1.003750pt}%
\definecolor{currentstroke}{rgb}{0.839216,0.152941,0.156863}%
\pgfsetstrokecolor{currentstroke}%
\pgfsetstrokeopacity{0.200000}%
\pgfsetdash{}{0pt}%
\pgfpathrectangle{\pgfqpoint{3.788192in}{2.980138in}}{\pgfqpoint{2.914000in}{2.171400in}}%
\pgfusepath{clip}%
\pgfpathmoveto{\pgfqpoint{4.530215in}{2.980138in}}%
\pgfpathlineto{\pgfqpoint{24.162152in}{2.980138in}}%
\pgfpathlineto{\pgfqpoint{24.162152in}{5.151538in}}%
\pgfpathlineto{\pgfqpoint{4.530215in}{5.151538in}}%
\pgfpathlineto{\pgfqpoint{4.530215in}{2.980138in}}%
\pgfpathclose%
\pgfusepath{clip}%
\pgfsys@defobject{currentpattern}{\pgfqpoint{0in}{0in}}{\pgfqpoint{1in}{1in}}{%
\begin{pgfscope}%
\pgfpathrectangle{\pgfqpoint{0in}{0in}}{\pgfqpoint{1in}{1in}}%
\pgfusepath{clip}%
\pgfpathmoveto{\pgfqpoint{-0.500000in}{0.500000in}}%
\pgfpathlineto{\pgfqpoint{0.500000in}{1.500000in}}%
\pgfpathmoveto{\pgfqpoint{-0.333333in}{0.333333in}}%
\pgfpathlineto{\pgfqpoint{0.666667in}{1.333333in}}%
\pgfpathmoveto{\pgfqpoint{-0.166667in}{0.166667in}}%
\pgfpathlineto{\pgfqpoint{0.833333in}{1.166667in}}%
\pgfpathmoveto{\pgfqpoint{0.000000in}{0.000000in}}%
\pgfpathlineto{\pgfqpoint{1.000000in}{1.000000in}}%
\pgfpathmoveto{\pgfqpoint{0.166667in}{-0.166667in}}%
\pgfpathlineto{\pgfqpoint{1.166667in}{0.833333in}}%
\pgfpathmoveto{\pgfqpoint{0.333333in}{-0.333333in}}%
\pgfpathlineto{\pgfqpoint{1.333333in}{0.666667in}}%
\pgfpathmoveto{\pgfqpoint{0.500000in}{-0.500000in}}%
\pgfpathlineto{\pgfqpoint{1.500000in}{0.500000in}}%
\pgfusepath{stroke}%
\end{pgfscope}%
}%
\pgfsys@transformshift{4.530215in}{2.980138in}%
\pgfsys@useobject{currentpattern}{}%
\pgfsys@transformshift{1in}{0in}%
\pgfsys@useobject{currentpattern}{}%
\pgfsys@transformshift{1in}{0in}%
\pgfsys@useobject{currentpattern}{}%
\pgfsys@transformshift{1in}{0in}%
\pgfsys@useobject{currentpattern}{}%
\pgfsys@transformshift{1in}{0in}%
\pgfsys@useobject{currentpattern}{}%
\pgfsys@transformshift{1in}{0in}%
\pgfsys@useobject{currentpattern}{}%
\pgfsys@transformshift{1in}{0in}%
\pgfsys@useobject{currentpattern}{}%
\pgfsys@transformshift{1in}{0in}%
\pgfsys@useobject{currentpattern}{}%
\pgfsys@transformshift{1in}{0in}%
\pgfsys@useobject{currentpattern}{}%
\pgfsys@transformshift{1in}{0in}%
\pgfsys@useobject{currentpattern}{}%
\pgfsys@transformshift{1in}{0in}%
\pgfsys@useobject{currentpattern}{}%
\pgfsys@transformshift{1in}{0in}%
\pgfsys@useobject{currentpattern}{}%
\pgfsys@transformshift{1in}{0in}%
\pgfsys@useobject{currentpattern}{}%
\pgfsys@transformshift{1in}{0in}%
\pgfsys@useobject{currentpattern}{}%
\pgfsys@transformshift{1in}{0in}%
\pgfsys@useobject{currentpattern}{}%
\pgfsys@transformshift{1in}{0in}%
\pgfsys@useobject{currentpattern}{}%
\pgfsys@transformshift{1in}{0in}%
\pgfsys@useobject{currentpattern}{}%
\pgfsys@transformshift{1in}{0in}%
\pgfsys@useobject{currentpattern}{}%
\pgfsys@transformshift{1in}{0in}%
\pgfsys@useobject{currentpattern}{}%
\pgfsys@transformshift{1in}{0in}%
\pgfsys@useobject{currentpattern}{}%
\pgfsys@transformshift{1in}{0in}%
\pgfsys@transformshift{-20in}{0in}%
\pgfsys@transformshift{0in}{1in}%
\pgfsys@useobject{currentpattern}{}%
\pgfsys@transformshift{1in}{0in}%
\pgfsys@useobject{currentpattern}{}%
\pgfsys@transformshift{1in}{0in}%
\pgfsys@useobject{currentpattern}{}%
\pgfsys@transformshift{1in}{0in}%
\pgfsys@useobject{currentpattern}{}%
\pgfsys@transformshift{1in}{0in}%
\pgfsys@useobject{currentpattern}{}%
\pgfsys@transformshift{1in}{0in}%
\pgfsys@useobject{currentpattern}{}%
\pgfsys@transformshift{1in}{0in}%
\pgfsys@useobject{currentpattern}{}%
\pgfsys@transformshift{1in}{0in}%
\pgfsys@useobject{currentpattern}{}%
\pgfsys@transformshift{1in}{0in}%
\pgfsys@useobject{currentpattern}{}%
\pgfsys@transformshift{1in}{0in}%
\pgfsys@useobject{currentpattern}{}%
\pgfsys@transformshift{1in}{0in}%
\pgfsys@useobject{currentpattern}{}%
\pgfsys@transformshift{1in}{0in}%
\pgfsys@useobject{currentpattern}{}%
\pgfsys@transformshift{1in}{0in}%
\pgfsys@useobject{currentpattern}{}%
\pgfsys@transformshift{1in}{0in}%
\pgfsys@useobject{currentpattern}{}%
\pgfsys@transformshift{1in}{0in}%
\pgfsys@useobject{currentpattern}{}%
\pgfsys@transformshift{1in}{0in}%
\pgfsys@useobject{currentpattern}{}%
\pgfsys@transformshift{1in}{0in}%
\pgfsys@useobject{currentpattern}{}%
\pgfsys@transformshift{1in}{0in}%
\pgfsys@useobject{currentpattern}{}%
\pgfsys@transformshift{1in}{0in}%
\pgfsys@useobject{currentpattern}{}%
\pgfsys@transformshift{1in}{0in}%
\pgfsys@useobject{currentpattern}{}%
\pgfsys@transformshift{1in}{0in}%
\pgfsys@transformshift{-20in}{0in}%
\pgfsys@transformshift{0in}{1in}%
\pgfsys@useobject{currentpattern}{}%
\pgfsys@transformshift{1in}{0in}%
\pgfsys@useobject{currentpattern}{}%
\pgfsys@transformshift{1in}{0in}%
\pgfsys@useobject{currentpattern}{}%
\pgfsys@transformshift{1in}{0in}%
\pgfsys@useobject{currentpattern}{}%
\pgfsys@transformshift{1in}{0in}%
\pgfsys@useobject{currentpattern}{}%
\pgfsys@transformshift{1in}{0in}%
\pgfsys@useobject{currentpattern}{}%
\pgfsys@transformshift{1in}{0in}%
\pgfsys@useobject{currentpattern}{}%
\pgfsys@transformshift{1in}{0in}%
\pgfsys@useobject{currentpattern}{}%
\pgfsys@transformshift{1in}{0in}%
\pgfsys@useobject{currentpattern}{}%
\pgfsys@transformshift{1in}{0in}%
\pgfsys@useobject{currentpattern}{}%
\pgfsys@transformshift{1in}{0in}%
\pgfsys@useobject{currentpattern}{}%
\pgfsys@transformshift{1in}{0in}%
\pgfsys@useobject{currentpattern}{}%
\pgfsys@transformshift{1in}{0in}%
\pgfsys@useobject{currentpattern}{}%
\pgfsys@transformshift{1in}{0in}%
\pgfsys@useobject{currentpattern}{}%
\pgfsys@transformshift{1in}{0in}%
\pgfsys@useobject{currentpattern}{}%
\pgfsys@transformshift{1in}{0in}%
\pgfsys@useobject{currentpattern}{}%
\pgfsys@transformshift{1in}{0in}%
\pgfsys@useobject{currentpattern}{}%
\pgfsys@transformshift{1in}{0in}%
\pgfsys@useobject{currentpattern}{}%
\pgfsys@transformshift{1in}{0in}%
\pgfsys@useobject{currentpattern}{}%
\pgfsys@transformshift{1in}{0in}%
\pgfsys@useobject{currentpattern}{}%
\pgfsys@transformshift{1in}{0in}%
\pgfsys@transformshift{-20in}{0in}%
\pgfsys@transformshift{0in}{1in}%
\end{pgfscope}%
\begin{pgfscope}%
\pgfpathrectangle{\pgfqpoint{3.788192in}{2.980138in}}{\pgfqpoint{2.914000in}{2.171400in}}%
\pgfusepath{clip}%
\pgfsetrectcap%
\pgfsetroundjoin%
\pgfsetlinewidth{0.803000pt}%
\definecolor{currentstroke}{rgb}{0.690196,0.690196,0.690196}%
\pgfsetstrokecolor{currentstroke}%
\pgfsetdash{}{0pt}%
\pgfpathmoveto{\pgfqpoint{4.105109in}{2.980138in}}%
\pgfpathlineto{\pgfqpoint{4.105109in}{5.151538in}}%
\pgfusepath{stroke}%
\end{pgfscope}%
\begin{pgfscope}%
\pgfsetbuttcap%
\pgfsetroundjoin%
\definecolor{currentfill}{rgb}{0.000000,0.000000,0.000000}%
\pgfsetfillcolor{currentfill}%
\pgfsetlinewidth{0.803000pt}%
\definecolor{currentstroke}{rgb}{0.000000,0.000000,0.000000}%
\pgfsetstrokecolor{currentstroke}%
\pgfsetdash{}{0pt}%
\pgfsys@defobject{currentmarker}{\pgfqpoint{0.000000in}{-0.048611in}}{\pgfqpoint{0.000000in}{0.000000in}}{%
\pgfpathmoveto{\pgfqpoint{0.000000in}{0.000000in}}%
\pgfpathlineto{\pgfqpoint{0.000000in}{-0.048611in}}%
\pgfusepath{stroke,fill}%
}%
\begin{pgfscope}%
\pgfsys@transformshift{4.105109in}{2.980138in}%
\pgfsys@useobject{currentmarker}{}%
\end{pgfscope}%
\end{pgfscope}%
\begin{pgfscope}%
\definecolor{textcolor}{rgb}{0.000000,0.000000,0.000000}%
\pgfsetstrokecolor{textcolor}%
\pgfsetfillcolor{textcolor}%
\pgftext[x=4.105109in,y=2.882916in,,top]{\color{textcolor}{\rmfamily\fontsize{14.000000}{16.800000}\selectfont\catcode`\^=\active\def^{\ifmmode\sp\else\^{}\fi}\catcode`\%=\active\def%{\%}$\mathdefault{5280}$}}%
\end{pgfscope}%
\begin{pgfscope}%
\pgfpathrectangle{\pgfqpoint{3.788192in}{2.980138in}}{\pgfqpoint{2.914000in}{2.171400in}}%
\pgfusepath{clip}%
\pgfsetrectcap%
\pgfsetroundjoin%
\pgfsetlinewidth{0.803000pt}%
\definecolor{currentstroke}{rgb}{0.690196,0.690196,0.690196}%
\pgfsetstrokecolor{currentstroke}%
\pgfsetdash{}{0pt}%
\pgfpathmoveto{\pgfqpoint{4.847132in}{2.980138in}}%
\pgfpathlineto{\pgfqpoint{4.847132in}{5.151538in}}%
\pgfusepath{stroke}%
\end{pgfscope}%
\begin{pgfscope}%
\pgfsetbuttcap%
\pgfsetroundjoin%
\definecolor{currentfill}{rgb}{0.000000,0.000000,0.000000}%
\pgfsetfillcolor{currentfill}%
\pgfsetlinewidth{0.803000pt}%
\definecolor{currentstroke}{rgb}{0.000000,0.000000,0.000000}%
\pgfsetstrokecolor{currentstroke}%
\pgfsetdash{}{0pt}%
\pgfsys@defobject{currentmarker}{\pgfqpoint{0.000000in}{-0.048611in}}{\pgfqpoint{0.000000in}{0.000000in}}{%
\pgfpathmoveto{\pgfqpoint{0.000000in}{0.000000in}}%
\pgfpathlineto{\pgfqpoint{0.000000in}{-0.048611in}}%
\pgfusepath{stroke,fill}%
}%
\begin{pgfscope}%
\pgfsys@transformshift{4.847132in}{2.980138in}%
\pgfsys@useobject{currentmarker}{}%
\end{pgfscope}%
\end{pgfscope}%
\begin{pgfscope}%
\definecolor{textcolor}{rgb}{0.000000,0.000000,0.000000}%
\pgfsetstrokecolor{textcolor}%
\pgfsetfillcolor{textcolor}%
\pgftext[x=4.847132in,y=2.882916in,,top]{\color{textcolor}{\rmfamily\fontsize{14.000000}{16.800000}\selectfont\catcode`\^=\active\def^{\ifmmode\sp\else\^{}\fi}\catcode`\%=\active\def%{\%}$\mathdefault{5300}$}}%
\end{pgfscope}%
\begin{pgfscope}%
\pgfpathrectangle{\pgfqpoint{3.788192in}{2.980138in}}{\pgfqpoint{2.914000in}{2.171400in}}%
\pgfusepath{clip}%
\pgfsetrectcap%
\pgfsetroundjoin%
\pgfsetlinewidth{0.803000pt}%
\definecolor{currentstroke}{rgb}{0.690196,0.690196,0.690196}%
\pgfsetstrokecolor{currentstroke}%
\pgfsetdash{}{0pt}%
\pgfpathmoveto{\pgfqpoint{5.589156in}{2.980138in}}%
\pgfpathlineto{\pgfqpoint{5.589156in}{5.151538in}}%
\pgfusepath{stroke}%
\end{pgfscope}%
\begin{pgfscope}%
\pgfsetbuttcap%
\pgfsetroundjoin%
\definecolor{currentfill}{rgb}{0.000000,0.000000,0.000000}%
\pgfsetfillcolor{currentfill}%
\pgfsetlinewidth{0.803000pt}%
\definecolor{currentstroke}{rgb}{0.000000,0.000000,0.000000}%
\pgfsetstrokecolor{currentstroke}%
\pgfsetdash{}{0pt}%
\pgfsys@defobject{currentmarker}{\pgfqpoint{0.000000in}{-0.048611in}}{\pgfqpoint{0.000000in}{0.000000in}}{%
\pgfpathmoveto{\pgfqpoint{0.000000in}{0.000000in}}%
\pgfpathlineto{\pgfqpoint{0.000000in}{-0.048611in}}%
\pgfusepath{stroke,fill}%
}%
\begin{pgfscope}%
\pgfsys@transformshift{5.589156in}{2.980138in}%
\pgfsys@useobject{currentmarker}{}%
\end{pgfscope}%
\end{pgfscope}%
\begin{pgfscope}%
\definecolor{textcolor}{rgb}{0.000000,0.000000,0.000000}%
\pgfsetstrokecolor{textcolor}%
\pgfsetfillcolor{textcolor}%
\pgftext[x=5.589156in,y=2.882916in,,top]{\color{textcolor}{\rmfamily\fontsize{14.000000}{16.800000}\selectfont\catcode`\^=\active\def^{\ifmmode\sp\else\^{}\fi}\catcode`\%=\active\def%{\%}$\mathdefault{5320}$}}%
\end{pgfscope}%
\begin{pgfscope}%
\pgfpathrectangle{\pgfqpoint{3.788192in}{2.980138in}}{\pgfqpoint{2.914000in}{2.171400in}}%
\pgfusepath{clip}%
\pgfsetrectcap%
\pgfsetroundjoin%
\pgfsetlinewidth{0.803000pt}%
\definecolor{currentstroke}{rgb}{0.690196,0.690196,0.690196}%
\pgfsetstrokecolor{currentstroke}%
\pgfsetdash{}{0pt}%
\pgfpathmoveto{\pgfqpoint{6.331180in}{2.980138in}}%
\pgfpathlineto{\pgfqpoint{6.331180in}{5.151538in}}%
\pgfusepath{stroke}%
\end{pgfscope}%
\begin{pgfscope}%
\pgfsetbuttcap%
\pgfsetroundjoin%
\definecolor{currentfill}{rgb}{0.000000,0.000000,0.000000}%
\pgfsetfillcolor{currentfill}%
\pgfsetlinewidth{0.803000pt}%
\definecolor{currentstroke}{rgb}{0.000000,0.000000,0.000000}%
\pgfsetstrokecolor{currentstroke}%
\pgfsetdash{}{0pt}%
\pgfsys@defobject{currentmarker}{\pgfqpoint{0.000000in}{-0.048611in}}{\pgfqpoint{0.000000in}{0.000000in}}{%
\pgfpathmoveto{\pgfqpoint{0.000000in}{0.000000in}}%
\pgfpathlineto{\pgfqpoint{0.000000in}{-0.048611in}}%
\pgfusepath{stroke,fill}%
}%
\begin{pgfscope}%
\pgfsys@transformshift{6.331180in}{2.980138in}%
\pgfsys@useobject{currentmarker}{}%
\end{pgfscope}%
\end{pgfscope}%
\begin{pgfscope}%
\definecolor{textcolor}{rgb}{0.000000,0.000000,0.000000}%
\pgfsetstrokecolor{textcolor}%
\pgfsetfillcolor{textcolor}%
\pgftext[x=6.331180in,y=2.882916in,,top]{\color{textcolor}{\rmfamily\fontsize{14.000000}{16.800000}\selectfont\catcode`\^=\active\def^{\ifmmode\sp\else\^{}\fi}\catcode`\%=\active\def%{\%}$\mathdefault{5340}$}}%
\end{pgfscope}%
\begin{pgfscope}%
\pgfpathrectangle{\pgfqpoint{3.788192in}{2.980138in}}{\pgfqpoint{2.914000in}{2.171400in}}%
\pgfusepath{clip}%
\pgfsetrectcap%
\pgfsetroundjoin%
\pgfsetlinewidth{0.803000pt}%
\definecolor{currentstroke}{rgb}{0.690196,0.690196,0.690196}%
\pgfsetstrokecolor{currentstroke}%
\pgfsetdash{}{0pt}%
\pgfpathmoveto{\pgfqpoint{3.788192in}{3.334947in}}%
\pgfpathlineto{\pgfqpoint{6.702192in}{3.334947in}}%
\pgfusepath{stroke}%
\end{pgfscope}%
\begin{pgfscope}%
\pgfsetbuttcap%
\pgfsetroundjoin%
\definecolor{currentfill}{rgb}{0.000000,0.000000,0.000000}%
\pgfsetfillcolor{currentfill}%
\pgfsetlinewidth{0.803000pt}%
\definecolor{currentstroke}{rgb}{0.000000,0.000000,0.000000}%
\pgfsetstrokecolor{currentstroke}%
\pgfsetdash{}{0pt}%
\pgfsys@defobject{currentmarker}{\pgfqpoint{-0.048611in}{0.000000in}}{\pgfqpoint{-0.000000in}{0.000000in}}{%
\pgfpathmoveto{\pgfqpoint{-0.000000in}{0.000000in}}%
\pgfpathlineto{\pgfqpoint{-0.048611in}{0.000000in}}%
\pgfusepath{stroke,fill}%
}%
\begin{pgfscope}%
\pgfsys@transformshift{3.788192in}{3.334947in}%
\pgfsys@useobject{currentmarker}{}%
\end{pgfscope}%
\end{pgfscope}%
\begin{pgfscope}%
\definecolor{textcolor}{rgb}{0.000000,0.000000,0.000000}%
\pgfsetstrokecolor{textcolor}%
\pgfsetfillcolor{textcolor}%
\pgftext[x=3.495138in, y=3.265502in, left, base]{\color{textcolor}{\rmfamily\fontsize{14.000000}{16.800000}\selectfont\catcode`\^=\active\def^{\ifmmode\sp\else\^{}\fi}\catcode`\%=\active\def%{\%}$\mathdefault{10}$}}%
\end{pgfscope}%
\begin{pgfscope}%
\pgfpathrectangle{\pgfqpoint{3.788192in}{2.980138in}}{\pgfqpoint{2.914000in}{2.171400in}}%
\pgfusepath{clip}%
\pgfsetrectcap%
\pgfsetroundjoin%
\pgfsetlinewidth{0.803000pt}%
\definecolor{currentstroke}{rgb}{0.690196,0.690196,0.690196}%
\pgfsetstrokecolor{currentstroke}%
\pgfsetdash{}{0pt}%
\pgfpathmoveto{\pgfqpoint{3.788192in}{4.044564in}}%
\pgfpathlineto{\pgfqpoint{6.702192in}{4.044564in}}%
\pgfusepath{stroke}%
\end{pgfscope}%
\begin{pgfscope}%
\pgfsetbuttcap%
\pgfsetroundjoin%
\definecolor{currentfill}{rgb}{0.000000,0.000000,0.000000}%
\pgfsetfillcolor{currentfill}%
\pgfsetlinewidth{0.803000pt}%
\definecolor{currentstroke}{rgb}{0.000000,0.000000,0.000000}%
\pgfsetstrokecolor{currentstroke}%
\pgfsetdash{}{0pt}%
\pgfsys@defobject{currentmarker}{\pgfqpoint{-0.048611in}{0.000000in}}{\pgfqpoint{-0.000000in}{0.000000in}}{%
\pgfpathmoveto{\pgfqpoint{-0.000000in}{0.000000in}}%
\pgfpathlineto{\pgfqpoint{-0.048611in}{0.000000in}}%
\pgfusepath{stroke,fill}%
}%
\begin{pgfscope}%
\pgfsys@transformshift{3.788192in}{4.044564in}%
\pgfsys@useobject{currentmarker}{}%
\end{pgfscope}%
\end{pgfscope}%
\begin{pgfscope}%
\definecolor{textcolor}{rgb}{0.000000,0.000000,0.000000}%
\pgfsetstrokecolor{textcolor}%
\pgfsetfillcolor{textcolor}%
\pgftext[x=3.495138in, y=3.975119in, left, base]{\color{textcolor}{\rmfamily\fontsize{14.000000}{16.800000}\selectfont\catcode`\^=\active\def^{\ifmmode\sp\else\^{}\fi}\catcode`\%=\active\def%{\%}$\mathdefault{12}$}}%
\end{pgfscope}%
\begin{pgfscope}%
\pgfpathrectangle{\pgfqpoint{3.788192in}{2.980138in}}{\pgfqpoint{2.914000in}{2.171400in}}%
\pgfusepath{clip}%
\pgfsetrectcap%
\pgfsetroundjoin%
\pgfsetlinewidth{0.803000pt}%
\definecolor{currentstroke}{rgb}{0.690196,0.690196,0.690196}%
\pgfsetstrokecolor{currentstroke}%
\pgfsetdash{}{0pt}%
\pgfpathmoveto{\pgfqpoint{3.788192in}{4.754181in}}%
\pgfpathlineto{\pgfqpoint{6.702192in}{4.754181in}}%
\pgfusepath{stroke}%
\end{pgfscope}%
\begin{pgfscope}%
\pgfsetbuttcap%
\pgfsetroundjoin%
\definecolor{currentfill}{rgb}{0.000000,0.000000,0.000000}%
\pgfsetfillcolor{currentfill}%
\pgfsetlinewidth{0.803000pt}%
\definecolor{currentstroke}{rgb}{0.000000,0.000000,0.000000}%
\pgfsetstrokecolor{currentstroke}%
\pgfsetdash{}{0pt}%
\pgfsys@defobject{currentmarker}{\pgfqpoint{-0.048611in}{0.000000in}}{\pgfqpoint{-0.000000in}{0.000000in}}{%
\pgfpathmoveto{\pgfqpoint{-0.000000in}{0.000000in}}%
\pgfpathlineto{\pgfqpoint{-0.048611in}{0.000000in}}%
\pgfusepath{stroke,fill}%
}%
\begin{pgfscope}%
\pgfsys@transformshift{3.788192in}{4.754181in}%
\pgfsys@useobject{currentmarker}{}%
\end{pgfscope}%
\end{pgfscope}%
\begin{pgfscope}%
\definecolor{textcolor}{rgb}{0.000000,0.000000,0.000000}%
\pgfsetstrokecolor{textcolor}%
\pgfsetfillcolor{textcolor}%
\pgftext[x=3.495138in, y=4.684736in, left, base]{\color{textcolor}{\rmfamily\fontsize{14.000000}{16.800000}\selectfont\catcode`\^=\active\def^{\ifmmode\sp\else\^{}\fi}\catcode`\%=\active\def%{\%}$\mathdefault{14}$}}%
\end{pgfscope}%
\begin{pgfscope}%
\pgfpathrectangle{\pgfqpoint{3.788192in}{2.980138in}}{\pgfqpoint{2.914000in}{2.171400in}}%
\pgfusepath{clip}%
\pgfsetrectcap%
\pgfsetroundjoin%
\pgfsetlinewidth{1.505625pt}%
\definecolor{currentstroke}{rgb}{0.000000,0.000000,1.000000}%
\pgfsetstrokecolor{currentstroke}%
\pgfsetdash{}{0pt}%
\pgfpathmoveto{\pgfqpoint{5.478622in}{4.808529in}}%
\pgfpathlineto{\pgfqpoint{5.701947in}{2.977638in}}%
\pgfusepath{stroke}%
\end{pgfscope}%
\begin{pgfscope}%
\pgfpathrectangle{\pgfqpoint{3.788192in}{2.980138in}}{\pgfqpoint{2.914000in}{2.171400in}}%
\pgfusepath{clip}%
\pgfsetbuttcap%
\pgfsetroundjoin%
\definecolor{currentfill}{rgb}{0.000000,0.000000,1.000000}%
\pgfsetfillcolor{currentfill}%
\pgfsetlinewidth{1.003750pt}%
\definecolor{currentstroke}{rgb}{0.000000,0.000000,1.000000}%
\pgfsetstrokecolor{currentstroke}%
\pgfsetdash{}{0pt}%
\pgfsys@defobject{currentmarker}{\pgfqpoint{-0.041667in}{-0.041667in}}{\pgfqpoint{0.041667in}{0.041667in}}{%
\pgfpathmoveto{\pgfqpoint{0.000000in}{-0.041667in}}%
\pgfpathcurveto{\pgfqpoint{0.011050in}{-0.041667in}}{\pgfqpoint{0.021649in}{-0.037276in}}{\pgfqpoint{0.029463in}{-0.029463in}}%
\pgfpathcurveto{\pgfqpoint{0.037276in}{-0.021649in}}{\pgfqpoint{0.041667in}{-0.011050in}}{\pgfqpoint{0.041667in}{0.000000in}}%
\pgfpathcurveto{\pgfqpoint{0.041667in}{0.011050in}}{\pgfqpoint{0.037276in}{0.021649in}}{\pgfqpoint{0.029463in}{0.029463in}}%
\pgfpathcurveto{\pgfqpoint{0.021649in}{0.037276in}}{\pgfqpoint{0.011050in}{0.041667in}}{\pgfqpoint{0.000000in}{0.041667in}}%
\pgfpathcurveto{\pgfqpoint{-0.011050in}{0.041667in}}{\pgfqpoint{-0.021649in}{0.037276in}}{\pgfqpoint{-0.029463in}{0.029463in}}%
\pgfpathcurveto{\pgfqpoint{-0.037276in}{0.021649in}}{\pgfqpoint{-0.041667in}{0.011050in}}{\pgfqpoint{-0.041667in}{0.000000in}}%
\pgfpathcurveto{\pgfqpoint{-0.041667in}{-0.011050in}}{\pgfqpoint{-0.037276in}{-0.021649in}}{\pgfqpoint{-0.029463in}{-0.029463in}}%
\pgfpathcurveto{\pgfqpoint{-0.021649in}{-0.037276in}}{\pgfqpoint{-0.011050in}{-0.041667in}}{\pgfqpoint{0.000000in}{-0.041667in}}%
\pgfpathlineto{\pgfqpoint{0.000000in}{-0.041667in}}%
\pgfpathclose%
\pgfusepath{stroke,fill}%
}%
\begin{pgfscope}%
\pgfsys@transformshift{5.478622in}{4.808529in}%
\pgfsys@useobject{currentmarker}{}%
\end{pgfscope}%
\begin{pgfscope}%
\pgfsys@transformshift{5.786444in}{2.284897in}%
\pgfsys@useobject{currentmarker}{}%
\end{pgfscope}%
\begin{pgfscope}%
\pgfsys@transformshift{5.955602in}{1.807070in}%
\pgfsys@useobject{currentmarker}{}%
\end{pgfscope}%
\begin{pgfscope}%
\pgfsys@transformshift{6.072257in}{1.530809in}%
\pgfsys@useobject{currentmarker}{}%
\end{pgfscope}%
\begin{pgfscope}%
\pgfsys@transformshift{6.134561in}{1.493263in}%
\pgfsys@useobject{currentmarker}{}%
\end{pgfscope}%
\begin{pgfscope}%
\pgfsys@transformshift{6.204334in}{1.296852in}%
\pgfsys@useobject{currentmarker}{}%
\end{pgfscope}%
\begin{pgfscope}%
\pgfsys@transformshift{6.281957in}{1.238734in}%
\pgfsys@useobject{currentmarker}{}%
\end{pgfscope}%
\begin{pgfscope}%
\pgfsys@transformshift{6.389240in}{1.234341in}%
\pgfsys@useobject{currentmarker}{}%
\end{pgfscope}%
\begin{pgfscope}%
\pgfsys@transformshift{6.427201in}{1.108377in}%
\pgfsys@useobject{currentmarker}{}%
\end{pgfscope}%
\begin{pgfscope}%
\pgfsys@transformshift{6.519285in}{1.077759in}%
\pgfsys@useobject{currentmarker}{}%
\end{pgfscope}%
\begin{pgfscope}%
\pgfsys@transformshift{6.689124in}{1.072073in}%
\pgfsys@useobject{currentmarker}{}%
\end{pgfscope}%
\begin{pgfscope}%
\pgfsys@transformshift{6.747190in}{1.024269in}%
\pgfsys@useobject{currentmarker}{}%
\end{pgfscope}%
\begin{pgfscope}%
\pgfsys@transformshift{6.806750in}{1.015907in}%
\pgfsys@useobject{currentmarker}{}%
\end{pgfscope}%
\begin{pgfscope}%
\pgfsys@transformshift{6.931730in}{0.991431in}%
\pgfsys@useobject{currentmarker}{}%
\end{pgfscope}%
\begin{pgfscope}%
\pgfsys@transformshift{7.035734in}{0.970288in}%
\pgfsys@useobject{currentmarker}{}%
\end{pgfscope}%
\begin{pgfscope}%
\pgfsys@transformshift{7.042663in}{0.960624in}%
\pgfsys@useobject{currentmarker}{}%
\end{pgfscope}%
\begin{pgfscope}%
\pgfsys@transformshift{7.110896in}{0.945926in}%
\pgfsys@useobject{currentmarker}{}%
\end{pgfscope}%
\begin{pgfscope}%
\pgfsys@transformshift{7.385541in}{0.918874in}%
\pgfsys@useobject{currentmarker}{}%
\end{pgfscope}%
\begin{pgfscope}%
\pgfsys@transformshift{7.677622in}{0.881182in}%
\pgfsys@useobject{currentmarker}{}%
\end{pgfscope}%
\begin{pgfscope}%
\pgfsys@transformshift{8.074033in}{0.854291in}%
\pgfsys@useobject{currentmarker}{}%
\end{pgfscope}%
\begin{pgfscope}%
\pgfsys@transformshift{8.437037in}{0.845259in}%
\pgfsys@useobject{currentmarker}{}%
\end{pgfscope}%
\begin{pgfscope}%
\pgfsys@transformshift{8.451833in}{0.833333in}%
\pgfsys@useobject{currentmarker}{}%
\end{pgfscope}%
\begin{pgfscope}%
\pgfsys@transformshift{8.473270in}{0.812531in}%
\pgfsys@useobject{currentmarker}{}%
\end{pgfscope}%
\begin{pgfscope}%
\pgfsys@transformshift{8.530759in}{0.805118in}%
\pgfsys@useobject{currentmarker}{}%
\end{pgfscope}%
\begin{pgfscope}%
\pgfsys@transformshift{8.667946in}{0.801766in}%
\pgfsys@useobject{currentmarker}{}%
\end{pgfscope}%
\begin{pgfscope}%
\pgfsys@transformshift{8.761129in}{0.792896in}%
\pgfsys@useobject{currentmarker}{}%
\end{pgfscope}%
\begin{pgfscope}%
\pgfsys@transformshift{9.175032in}{0.779118in}%
\pgfsys@useobject{currentmarker}{}%
\end{pgfscope}%
\begin{pgfscope}%
\pgfsys@transformshift{9.229242in}{0.771806in}%
\pgfsys@useobject{currentmarker}{}%
\end{pgfscope}%
\begin{pgfscope}%
\pgfsys@transformshift{9.230836in}{0.763945in}%
\pgfsys@useobject{currentmarker}{}%
\end{pgfscope}%
\begin{pgfscope}%
\pgfsys@transformshift{9.319803in}{0.762647in}%
\pgfsys@useobject{currentmarker}{}%
\end{pgfscope}%
\begin{pgfscope}%
\pgfsys@transformshift{9.530028in}{0.749444in}%
\pgfsys@useobject{currentmarker}{}%
\end{pgfscope}%
\begin{pgfscope}%
\pgfsys@transformshift{9.629502in}{0.748101in}%
\pgfsys@useobject{currentmarker}{}%
\end{pgfscope}%
\begin{pgfscope}%
\pgfsys@transformshift{9.644888in}{0.743176in}%
\pgfsys@useobject{currentmarker}{}%
\end{pgfscope}%
\begin{pgfscope}%
\pgfsys@transformshift{10.473184in}{0.730520in}%
\pgfsys@useobject{currentmarker}{}%
\end{pgfscope}%
\begin{pgfscope}%
\pgfsys@transformshift{10.538033in}{0.705808in}%
\pgfsys@useobject{currentmarker}{}%
\end{pgfscope}%
\begin{pgfscope}%
\pgfsys@transformshift{10.574876in}{0.703903in}%
\pgfsys@useobject{currentmarker}{}%
\end{pgfscope}%
\begin{pgfscope}%
\pgfsys@transformshift{10.706030in}{0.700593in}%
\pgfsys@useobject{currentmarker}{}%
\end{pgfscope}%
\begin{pgfscope}%
\pgfsys@transformshift{10.965602in}{0.687998in}%
\pgfsys@useobject{currentmarker}{}%
\end{pgfscope}%
\begin{pgfscope}%
\pgfsys@transformshift{11.060169in}{0.684789in}%
\pgfsys@useobject{currentmarker}{}%
\end{pgfscope}%
\begin{pgfscope}%
\pgfsys@transformshift{11.147519in}{0.680781in}%
\pgfsys@useobject{currentmarker}{}%
\end{pgfscope}%
\begin{pgfscope}%
\pgfsys@transformshift{11.370645in}{0.672484in}%
\pgfsys@useobject{currentmarker}{}%
\end{pgfscope}%
\begin{pgfscope}%
\pgfsys@transformshift{11.728159in}{0.668514in}%
\pgfsys@useobject{currentmarker}{}%
\end{pgfscope}%
\begin{pgfscope}%
\pgfsys@transformshift{12.137942in}{0.666709in}%
\pgfsys@useobject{currentmarker}{}%
\end{pgfscope}%
\begin{pgfscope}%
\pgfsys@transformshift{12.434130in}{0.666468in}%
\pgfsys@useobject{currentmarker}{}%
\end{pgfscope}%
\begin{pgfscope}%
\pgfsys@transformshift{12.804602in}{0.666397in}%
\pgfsys@useobject{currentmarker}{}%
\end{pgfscope}%
\begin{pgfscope}%
\pgfsys@transformshift{14.562739in}{0.661643in}%
\pgfsys@useobject{currentmarker}{}%
\end{pgfscope}%
\begin{pgfscope}%
\pgfsys@transformshift{15.028873in}{0.661022in}%
\pgfsys@useobject{currentmarker}{}%
\end{pgfscope}%
\begin{pgfscope}%
\pgfsys@transformshift{15.695083in}{0.659204in}%
\pgfsys@useobject{currentmarker}{}%
\end{pgfscope}%
\begin{pgfscope}%
\pgfsys@transformshift{16.663977in}{0.658612in}%
\pgfsys@useobject{currentmarker}{}%
\end{pgfscope}%
\begin{pgfscope}%
\pgfsys@transformshift{17.188373in}{0.655818in}%
\pgfsys@useobject{currentmarker}{}%
\end{pgfscope}%
\begin{pgfscope}%
\pgfsys@transformshift{18.420605in}{0.654898in}%
\pgfsys@useobject{currentmarker}{}%
\end{pgfscope}%
\begin{pgfscope}%
\pgfsys@transformshift{20.009331in}{0.652189in}%
\pgfsys@useobject{currentmarker}{}%
\end{pgfscope}%
\begin{pgfscope}%
\pgfsys@transformshift{21.260300in}{0.647607in}%
\pgfsys@useobject{currentmarker}{}%
\end{pgfscope}%
\begin{pgfscope}%
\pgfsys@transformshift{23.228701in}{0.645006in}%
\pgfsys@useobject{currentmarker}{}%
\end{pgfscope}%
\begin{pgfscope}%
\pgfsys@transformshift{25.888770in}{0.639824in}%
\pgfsys@useobject{currentmarker}{}%
\end{pgfscope}%
\begin{pgfscope}%
\pgfsys@transformshift{28.569097in}{0.635141in}%
\pgfsys@useobject{currentmarker}{}%
\end{pgfscope}%
\begin{pgfscope}%
\pgfsys@transformshift{32.504050in}{0.628662in}%
\pgfsys@useobject{currentmarker}{}%
\end{pgfscope}%
\begin{pgfscope}%
\pgfsys@transformshift{39.060135in}{0.620415in}%
\pgfsys@useobject{currentmarker}{}%
\end{pgfscope}%
\begin{pgfscope}%
\pgfsys@transformshift{50.416853in}{0.608419in}%
\pgfsys@useobject{currentmarker}{}%
\end{pgfscope}%
\begin{pgfscope}%
\pgfsys@transformshift{87.100182in}{0.583248in}%
\pgfsys@useobject{currentmarker}{}%
\end{pgfscope}%
\end{pgfscope}%
\begin{pgfscope}%
\pgfpathrectangle{\pgfqpoint{3.788192in}{2.980138in}}{\pgfqpoint{2.914000in}{2.171400in}}%
\pgfusepath{clip}%
\pgfsetrectcap%
\pgfsetroundjoin%
\pgfsetlinewidth{1.505625pt}%
\definecolor{currentstroke}{rgb}{0.121569,0.466667,0.705882}%
\pgfsetstrokecolor{currentstroke}%
\pgfsetstrokeopacity{0.500000}%
\pgfsetdash{}{0pt}%
\pgfusepath{stroke}%
\end{pgfscope}%
\begin{pgfscope}%
\pgfsetrectcap%
\pgfsetmiterjoin%
\pgfsetlinewidth{0.803000pt}%
\definecolor{currentstroke}{rgb}{0.000000,0.000000,0.000000}%
\pgfsetstrokecolor{currentstroke}%
\pgfsetdash{}{0pt}%
\pgfpathmoveto{\pgfqpoint{3.788192in}{2.980138in}}%
\pgfpathlineto{\pgfqpoint{3.788192in}{5.151538in}}%
\pgfusepath{stroke}%
\end{pgfscope}%
\begin{pgfscope}%
\pgfsetrectcap%
\pgfsetmiterjoin%
\pgfsetlinewidth{0.803000pt}%
\definecolor{currentstroke}{rgb}{0.000000,0.000000,0.000000}%
\pgfsetstrokecolor{currentstroke}%
\pgfsetdash{}{0pt}%
\pgfpathmoveto{\pgfqpoint{6.702192in}{2.980138in}}%
\pgfpathlineto{\pgfqpoint{6.702192in}{5.151538in}}%
\pgfusepath{stroke}%
\end{pgfscope}%
\begin{pgfscope}%
\pgfsetrectcap%
\pgfsetmiterjoin%
\pgfsetlinewidth{0.803000pt}%
\definecolor{currentstroke}{rgb}{0.000000,0.000000,0.000000}%
\pgfsetstrokecolor{currentstroke}%
\pgfsetdash{}{0pt}%
\pgfpathmoveto{\pgfqpoint{3.788192in}{2.980138in}}%
\pgfpathlineto{\pgfqpoint{6.702192in}{2.980138in}}%
\pgfusepath{stroke}%
\end{pgfscope}%
\begin{pgfscope}%
\pgfsetrectcap%
\pgfsetmiterjoin%
\pgfsetlinewidth{0.803000pt}%
\definecolor{currentstroke}{rgb}{0.000000,0.000000,0.000000}%
\pgfsetstrokecolor{currentstroke}%
\pgfsetdash{}{0pt}%
\pgfpathmoveto{\pgfqpoint{3.788192in}{5.151538in}}%
\pgfpathlineto{\pgfqpoint{6.702192in}{5.151538in}}%
\pgfusepath{stroke}%
\end{pgfscope}%
\begin{pgfscope}%
\pgfsetbuttcap%
\pgfsetmiterjoin%
\definecolor{currentfill}{rgb}{1.000000,1.000000,1.000000}%
\pgfsetfillcolor{currentfill}%
\pgfsetfillopacity{0.800000}%
\pgfsetlinewidth{1.003750pt}%
\definecolor{currentstroke}{rgb}{0.800000,0.800000,0.800000}%
\pgfsetstrokecolor{currentstroke}%
\pgfsetstrokeopacity{0.800000}%
\pgfsetdash{}{0pt}%
\pgfpathmoveto{\pgfqpoint{3.327765in}{0.781249in}}%
\pgfpathlineto{\pgfqpoint{6.732636in}{0.781249in}}%
\pgfpathquadraticcurveto{\pgfqpoint{6.777080in}{0.781249in}}{\pgfqpoint{6.777080in}{0.825694in}}%
\pgfpathlineto{\pgfqpoint{6.777080in}{2.153779in}}%
\pgfpathquadraticcurveto{\pgfqpoint{6.777080in}{2.198223in}}{\pgfqpoint{6.732636in}{2.198223in}}%
\pgfpathlineto{\pgfqpoint{3.327765in}{2.198223in}}%
\pgfpathquadraticcurveto{\pgfqpoint{3.283320in}{2.198223in}}{\pgfqpoint{3.283320in}{2.153779in}}%
\pgfpathlineto{\pgfqpoint{3.283320in}{0.825694in}}%
\pgfpathquadraticcurveto{\pgfqpoint{3.283320in}{0.781249in}}{\pgfqpoint{3.327765in}{0.781249in}}%
\pgfpathlineto{\pgfqpoint{3.327765in}{0.781249in}}%
\pgfpathclose%
\pgfusepath{stroke,fill}%
\end{pgfscope}%
\begin{pgfscope}%
\pgfsetrectcap%
\pgfsetroundjoin%
\pgfsetlinewidth{1.505625pt}%
\definecolor{currentstroke}{rgb}{0.000000,0.000000,1.000000}%
\pgfsetstrokecolor{currentstroke}%
\pgfsetdash{}{0pt}%
\pgfpathmoveto{\pgfqpoint{3.372209in}{2.020446in}}%
\pgfpathlineto{\pgfqpoint{3.594431in}{2.020446in}}%
\pgfpathlineto{\pgfqpoint{3.816654in}{2.020446in}}%
\pgfusepath{stroke}%
\end{pgfscope}%
\begin{pgfscope}%
\pgfsetbuttcap%
\pgfsetroundjoin%
\definecolor{currentfill}{rgb}{0.000000,0.000000,1.000000}%
\pgfsetfillcolor{currentfill}%
\pgfsetlinewidth{1.003750pt}%
\definecolor{currentstroke}{rgb}{0.000000,0.000000,1.000000}%
\pgfsetstrokecolor{currentstroke}%
\pgfsetdash{}{0pt}%
\pgfsys@defobject{currentmarker}{\pgfqpoint{-0.006944in}{-0.006944in}}{\pgfqpoint{0.006944in}{0.006944in}}{%
\pgfpathmoveto{\pgfqpoint{0.000000in}{-0.006944in}}%
\pgfpathcurveto{\pgfqpoint{0.001842in}{-0.006944in}}{\pgfqpoint{0.003608in}{-0.006213in}}{\pgfqpoint{0.004910in}{-0.004910in}}%
\pgfpathcurveto{\pgfqpoint{0.006213in}{-0.003608in}}{\pgfqpoint{0.006944in}{-0.001842in}}{\pgfqpoint{0.006944in}{0.000000in}}%
\pgfpathcurveto{\pgfqpoint{0.006944in}{0.001842in}}{\pgfqpoint{0.006213in}{0.003608in}}{\pgfqpoint{0.004910in}{0.004910in}}%
\pgfpathcurveto{\pgfqpoint{0.003608in}{0.006213in}}{\pgfqpoint{0.001842in}{0.006944in}}{\pgfqpoint{0.000000in}{0.006944in}}%
\pgfpathcurveto{\pgfqpoint{-0.001842in}{0.006944in}}{\pgfqpoint{-0.003608in}{0.006213in}}{\pgfqpoint{-0.004910in}{0.004910in}}%
\pgfpathcurveto{\pgfqpoint{-0.006213in}{0.003608in}}{\pgfqpoint{-0.006944in}{0.001842in}}{\pgfqpoint{-0.006944in}{0.000000in}}%
\pgfpathcurveto{\pgfqpoint{-0.006944in}{-0.001842in}}{\pgfqpoint{-0.006213in}{-0.003608in}}{\pgfqpoint{-0.004910in}{-0.004910in}}%
\pgfpathcurveto{\pgfqpoint{-0.003608in}{-0.006213in}}{\pgfqpoint{-0.001842in}{-0.006944in}}{\pgfqpoint{0.000000in}{-0.006944in}}%
\pgfpathlineto{\pgfqpoint{0.000000in}{-0.006944in}}%
\pgfpathclose%
\pgfusepath{stroke,fill}%
}%
\begin{pgfscope}%
\pgfsys@transformshift{3.594431in}{2.020446in}%
\pgfsys@useobject{currentmarker}{}%
\end{pgfscope}%
\end{pgfscope}%
\begin{pgfscope}%
\definecolor{textcolor}{rgb}{0.000000,0.000000,0.000000}%
\pgfsetstrokecolor{textcolor}%
\pgfsetfillcolor{textcolor}%
\pgftext[x=3.994431in,y=1.942668in,left,base]{\color{textcolor}{\rmfamily\fontsize{16.000000}{19.200000}\selectfont\catcode`\^=\active\def^{\ifmmode\sp\else\^{}\fi}\catcode`\%=\active\def%{\%}osier}}%
\end{pgfscope}%
\begin{pgfscope}%
\pgfsetrectcap%
\pgfsetroundjoin%
\pgfsetlinewidth{1.505625pt}%
\definecolor{currentstroke}{rgb}{0.121569,0.466667,0.705882}%
\pgfsetstrokecolor{currentstroke}%
\pgfsetstrokeopacity{0.500000}%
\pgfsetdash{}{0pt}%
\pgfpathmoveto{\pgfqpoint{3.372209in}{1.682869in}}%
\pgfpathlineto{\pgfqpoint{3.594431in}{1.682869in}}%
\pgfpathlineto{\pgfqpoint{3.816654in}{1.682869in}}%
\pgfusepath{stroke}%
\end{pgfscope}%
\begin{pgfscope}%
\definecolor{textcolor}{rgb}{0.000000,0.000000,0.000000}%
\pgfsetstrokecolor{textcolor}%
\pgfsetfillcolor{textcolor}%
\pgftext[x=3.994431in,y=1.605091in,left,base]{\color{textcolor}{\rmfamily\fontsize{16.000000}{19.200000}\selectfont\catcode`\^=\active\def^{\ifmmode\sp\else\^{}\fi}\catcode`\%=\active\def%{\%}near-optimal space (osier)}}%
\end{pgfscope}%
\begin{pgfscope}%
\pgfsetbuttcap%
\pgfsetroundjoin%
\pgfsetlinewidth{1.003750pt}%
\definecolor{currentstroke}{rgb}{1.000000,0.000000,0.000000}%
\pgfsetstrokecolor{currentstroke}%
\pgfsetdash{}{0pt}%
\pgfpathmoveto{\pgfqpoint{3.594431in}{1.294791in}}%
\pgfpathcurveto{\pgfqpoint{3.602668in}{1.294791in}}{\pgfqpoint{3.610568in}{1.298063in}}{\pgfqpoint{3.616392in}{1.303887in}}%
\pgfpathcurveto{\pgfqpoint{3.622216in}{1.309711in}}{\pgfqpoint{3.625488in}{1.317611in}}{\pgfqpoint{3.625488in}{1.325847in}}%
\pgfpathcurveto{\pgfqpoint{3.625488in}{1.334084in}}{\pgfqpoint{3.622216in}{1.341984in}}{\pgfqpoint{3.616392in}{1.347808in}}%
\pgfpathcurveto{\pgfqpoint{3.610568in}{1.353632in}}{\pgfqpoint{3.602668in}{1.356904in}}{\pgfqpoint{3.594431in}{1.356904in}}%
\pgfpathcurveto{\pgfqpoint{3.586195in}{1.356904in}}{\pgfqpoint{3.578295in}{1.353632in}}{\pgfqpoint{3.572471in}{1.347808in}}%
\pgfpathcurveto{\pgfqpoint{3.566647in}{1.341984in}}{\pgfqpoint{3.563375in}{1.334084in}}{\pgfqpoint{3.563375in}{1.325847in}}%
\pgfpathcurveto{\pgfqpoint{3.563375in}{1.317611in}}{\pgfqpoint{3.566647in}{1.309711in}}{\pgfqpoint{3.572471in}{1.303887in}}%
\pgfpathcurveto{\pgfqpoint{3.578295in}{1.298063in}}{\pgfqpoint{3.586195in}{1.294791in}}{\pgfqpoint{3.594431in}{1.294791in}}%
\pgfpathlineto{\pgfqpoint{3.594431in}{1.294791in}}%
\pgfpathclose%
\pgfusepath{stroke}%
\end{pgfscope}%
\begin{pgfscope}%
\definecolor{textcolor}{rgb}{0.000000,0.000000,0.000000}%
\pgfsetstrokecolor{textcolor}%
\pgfsetfillcolor{textcolor}%
\pgftext[x=3.994431in,y=1.267514in,left,base]{\color{textcolor}{\rmfamily\fontsize{16.000000}{19.200000}\selectfont\catcode`\^=\active\def^{\ifmmode\sp\else\^{}\fi}\catcode`\%=\active\def%{\%}temoa+mga}}%
\end{pgfscope}%
\begin{pgfscope}%
\pgfsetbuttcap%
\pgfsetmiterjoin%
\definecolor{currentfill}{rgb}{0.839216,0.152941,0.156863}%
\pgfsetfillcolor{currentfill}%
\pgfsetfillopacity{0.200000}%
\pgfsetlinewidth{1.003750pt}%
\definecolor{currentstroke}{rgb}{0.839216,0.152941,0.156863}%
\pgfsetstrokecolor{currentstroke}%
\pgfsetstrokeopacity{0.200000}%
\pgfsetdash{}{0pt}%
\pgfpathmoveto{\pgfqpoint{3.372209in}{0.929937in}}%
\pgfpathlineto{\pgfqpoint{3.816654in}{0.929937in}}%
\pgfpathlineto{\pgfqpoint{3.816654in}{1.085493in}}%
\pgfpathlineto{\pgfqpoint{3.372209in}{1.085493in}}%
\pgfpathlineto{\pgfqpoint{3.372209in}{0.929937in}}%
\pgfpathclose%
\pgfusepath{stroke,fill}%
\end{pgfscope}%
\begin{pgfscope}%
\pgfsetbuttcap%
\pgfsetmiterjoin%
\definecolor{currentfill}{rgb}{0.839216,0.152941,0.156863}%
\pgfsetfillcolor{currentfill}%
\pgfsetfillopacity{0.200000}%
\pgfsetlinewidth{1.003750pt}%
\definecolor{currentstroke}{rgb}{0.839216,0.152941,0.156863}%
\pgfsetstrokecolor{currentstroke}%
\pgfsetstrokeopacity{0.200000}%
\pgfsetdash{}{0pt}%
\pgfpathmoveto{\pgfqpoint{3.372209in}{0.929937in}}%
\pgfpathlineto{\pgfqpoint{3.816654in}{0.929937in}}%
\pgfpathlineto{\pgfqpoint{3.816654in}{1.085493in}}%
\pgfpathlineto{\pgfqpoint{3.372209in}{1.085493in}}%
\pgfpathlineto{\pgfqpoint{3.372209in}{0.929937in}}%
\pgfpathclose%
\pgfusepath{clip}%
\pgfsys@defobject{currentpattern}{\pgfqpoint{0in}{0in}}{\pgfqpoint{1in}{1in}}{%
\begin{pgfscope}%
\pgfpathrectangle{\pgfqpoint{0in}{0in}}{\pgfqpoint{1in}{1in}}%
\pgfusepath{clip}%
\pgfpathmoveto{\pgfqpoint{-0.500000in}{0.500000in}}%
\pgfpathlineto{\pgfqpoint{0.500000in}{1.500000in}}%
\pgfpathmoveto{\pgfqpoint{-0.333333in}{0.333333in}}%
\pgfpathlineto{\pgfqpoint{0.666667in}{1.333333in}}%
\pgfpathmoveto{\pgfqpoint{-0.166667in}{0.166667in}}%
\pgfpathlineto{\pgfqpoint{0.833333in}{1.166667in}}%
\pgfpathmoveto{\pgfqpoint{0.000000in}{0.000000in}}%
\pgfpathlineto{\pgfqpoint{1.000000in}{1.000000in}}%
\pgfpathmoveto{\pgfqpoint{0.166667in}{-0.166667in}}%
\pgfpathlineto{\pgfqpoint{1.166667in}{0.833333in}}%
\pgfpathmoveto{\pgfqpoint{0.333333in}{-0.333333in}}%
\pgfpathlineto{\pgfqpoint{1.333333in}{0.666667in}}%
\pgfpathmoveto{\pgfqpoint{0.500000in}{-0.500000in}}%
\pgfpathlineto{\pgfqpoint{1.500000in}{0.500000in}}%
\pgfusepath{stroke}%
\end{pgfscope}%
}%
\pgfsys@transformshift{3.372209in}{0.929937in}%
\pgfsys@useobject{currentpattern}{}%
\pgfsys@transformshift{1in}{0in}%
\pgfsys@transformshift{-1in}{0in}%
\pgfsys@transformshift{0in}{1in}%
\end{pgfscope}%
\begin{pgfscope}%
\definecolor{textcolor}{rgb}{0.000000,0.000000,0.000000}%
\pgfsetstrokecolor{textcolor}%
\pgfsetfillcolor{textcolor}%
\pgftext[x=3.994431in,y=0.929937in,left,base]{\color{textcolor}{\rmfamily\fontsize{16.000000}{19.200000}\selectfont\catcode`\^=\active\def^{\ifmmode\sp\else\^{}\fi}\catcode`\%=\active\def%{\%}near-optimal space (Temoa)}}%
\end{pgfscope}%
\end{pgfpicture}%
\makeatother%
\endgroup%
}
  \resizebox{0.75\columnwidth}{!}{%% Creator: Matplotlib, PGF backend
%%
%% To include the figure in your LaTeX document, write
%%   \input{<filename>.pgf}
%%
%% Make sure the required packages are loaded in your preamble
%%   \usepackage{pgf}
%%
%% Also ensure that all the required font packages are loaded; for instance,
%% the lmodern package is sometimes necessary when using math font.
%%   \usepackage{lmodern}
%%
%% Figures using additional raster images can only be included by \input if
%% they are in the same directory as the main LaTeX file. For loading figures
%% from other directories you can use the `import` package
%%   \usepackage{import}
%%
%% and then include the figures with
%%   \import{<path to file>}{<filename>.pgf}
%%
%% Matplotlib used the following preamble
%%   \def\mathdefault#1{#1}
%%   \everymath=\expandafter{\the\everymath\displaystyle}
%%   \IfFileExists{scrextend.sty}{
%%     \usepackage[fontsize=10.000000pt]{scrextend}
%%   }{
%%     \renewcommand{\normalsize}{\fontsize{10.000000}{12.000000}\selectfont}
%%     \normalsize
%%   }
%%   
%%   \makeatletter\@ifpackageloaded{underscore}{}{\usepackage[strings]{underscore}}\makeatother
%%
\begingroup%
\makeatletter%
\begin{pgfpicture}%
\pgfpathrectangle{\pgfpointorigin}{\pgfqpoint{6.988192in}{5.458470in}}%
\pgfusepath{use as bounding box, clip}%
\begin{pgfscope}%
\pgfsetbuttcap%
\pgfsetmiterjoin%
\definecolor{currentfill}{rgb}{1.000000,1.000000,1.000000}%
\pgfsetfillcolor{currentfill}%
\pgfsetlinewidth{0.000000pt}%
\definecolor{currentstroke}{rgb}{0.000000,0.000000,0.000000}%
\pgfsetstrokecolor{currentstroke}%
\pgfsetdash{}{0pt}%
\pgfpathmoveto{\pgfqpoint{0.000000in}{0.000000in}}%
\pgfpathlineto{\pgfqpoint{6.988192in}{0.000000in}}%
\pgfpathlineto{\pgfqpoint{6.988192in}{5.458470in}}%
\pgfpathlineto{\pgfqpoint{0.000000in}{5.458470in}}%
\pgfpathlineto{\pgfqpoint{0.000000in}{0.000000in}}%
\pgfpathclose%
\pgfusepath{fill}%
\end{pgfscope}%
\begin{pgfscope}%
\pgfsetbuttcap%
\pgfsetmiterjoin%
\definecolor{currentfill}{rgb}{1.000000,1.000000,1.000000}%
\pgfsetfillcolor{currentfill}%
\pgfsetlinewidth{0.000000pt}%
\definecolor{currentstroke}{rgb}{0.000000,0.000000,0.000000}%
\pgfsetstrokecolor{currentstroke}%
\pgfsetstrokeopacity{0.000000}%
\pgfsetdash{}{0pt}%
\pgfpathmoveto{\pgfqpoint{0.688192in}{0.670138in}}%
\pgfpathlineto{\pgfqpoint{6.888192in}{0.670138in}}%
\pgfpathlineto{\pgfqpoint{6.888192in}{5.290138in}}%
\pgfpathlineto{\pgfqpoint{0.688192in}{5.290138in}}%
\pgfpathlineto{\pgfqpoint{0.688192in}{0.670138in}}%
\pgfpathclose%
\pgfusepath{fill}%
\end{pgfscope}%
\begin{pgfscope}%
\pgfpathrectangle{\pgfqpoint{0.688192in}{0.670138in}}{\pgfqpoint{6.200000in}{4.620000in}}%
\pgfusepath{clip}%
\pgfsetbuttcap%
\pgfsetmiterjoin%
\definecolor{currentfill}{rgb}{0.121569,0.466667,0.705882}%
\pgfsetfillcolor{currentfill}%
\pgfsetfillopacity{0.500000}%
\pgfsetlinewidth{1.003750pt}%
\definecolor{currentstroke}{rgb}{0.121569,0.466667,0.705882}%
\pgfsetstrokecolor{currentstroke}%
\pgfsetstrokeopacity{0.500000}%
\pgfsetdash{}{0pt}%
\pgfpathmoveto{\pgfqpoint{0.741425in}{1.377543in}}%
\pgfpathlineto{\pgfqpoint{0.758703in}{0.955032in}}%
\pgfpathlineto{\pgfqpoint{0.768198in}{0.875033in}}%
\pgfpathlineto{\pgfqpoint{0.774746in}{0.828781in}}%
\pgfpathlineto{\pgfqpoint{0.778243in}{0.822495in}}%
\pgfpathlineto{\pgfqpoint{0.782159in}{0.789611in}}%
\pgfpathlineto{\pgfqpoint{0.786516in}{0.779881in}}%
\pgfpathlineto{\pgfqpoint{0.792538in}{0.779145in}}%
\pgfpathlineto{\pgfqpoint{0.794668in}{0.758056in}}%
\pgfpathlineto{\pgfqpoint{0.799837in}{0.752930in}}%
\pgfpathlineto{\pgfqpoint{0.809370in}{0.751978in}}%
\pgfpathlineto{\pgfqpoint{0.812629in}{0.743975in}}%
\pgfpathlineto{\pgfqpoint{0.815972in}{0.742575in}}%
\pgfpathlineto{\pgfqpoint{0.822987in}{0.738477in}}%
\pgfpathlineto{\pgfqpoint{0.828825in}{0.734937in}}%
\pgfpathlineto{\pgfqpoint{0.829214in}{0.733319in}}%
\pgfpathlineto{\pgfqpoint{0.833044in}{0.730858in}}%
\pgfpathlineto{\pgfqpoint{0.848459in}{0.726329in}}%
\pgfpathlineto{\pgfqpoint{0.864854in}{0.720019in}}%
\pgfpathlineto{\pgfqpoint{0.887104in}{0.715517in}}%
\pgfpathlineto{\pgfqpoint{0.907479in}{0.714004in}}%
\pgfpathlineto{\pgfqpoint{0.908310in}{0.712008in}}%
\pgfpathlineto{\pgfqpoint{0.909513in}{0.708525in}}%
\pgfpathlineto{\pgfqpoint{0.912740in}{0.707284in}}%
\pgfpathlineto{\pgfqpoint{0.920440in}{0.706723in}}%
\pgfpathlineto{\pgfqpoint{0.925670in}{0.705238in}}%
\pgfpathlineto{\pgfqpoint{0.948903in}{0.702931in}}%
\pgfpathlineto{\pgfqpoint{0.951945in}{0.701707in}}%
\pgfpathlineto{\pgfqpoint{0.952035in}{0.700391in}}%
\pgfpathlineto{\pgfqpoint{0.957029in}{0.700173in}}%
\pgfpathlineto{\pgfqpoint{0.968828in}{0.697963in}}%
\pgfpathlineto{\pgfqpoint{0.974412in}{0.697738in}}%
\pgfpathlineto{\pgfqpoint{0.975275in}{0.696914in}}%
\pgfpathlineto{\pgfqpoint{1.021767in}{0.694795in}}%
\pgfpathlineto{\pgfqpoint{1.025407in}{0.690657in}}%
\pgfpathlineto{\pgfqpoint{1.027475in}{0.690338in}}%
\pgfpathlineto{\pgfqpoint{1.034837in}{0.689784in}}%
\pgfpathlineto{\pgfqpoint{1.049406in}{0.687676in}}%
\pgfpathlineto{\pgfqpoint{1.054714in}{0.687138in}}%
\pgfpathlineto{\pgfqpoint{1.059617in}{0.686467in}}%
\pgfpathlineto{\pgfqpoint{1.072141in}{0.685078in}}%
\pgfpathlineto{\pgfqpoint{1.092208in}{0.684413in}}%
\pgfpathlineto{\pgfqpoint{1.115209in}{0.684111in}}%
\pgfpathlineto{\pgfqpoint{1.131834in}{0.684071in}}%
\pgfpathlineto{\pgfqpoint{1.152628in}{0.684059in}}%
\pgfpathlineto{\pgfqpoint{1.251312in}{0.683263in}}%
\pgfpathlineto{\pgfqpoint{1.277476in}{0.683159in}}%
\pgfpathlineto{\pgfqpoint{1.314870in}{0.682855in}}%
\pgfpathlineto{\pgfqpoint{1.369253in}{0.682756in}}%
\pgfpathlineto{\pgfqpoint{1.398687in}{0.682288in}}%
\pgfpathlineto{\pgfqpoint{1.467852in}{0.682134in}}%
\pgfpathlineto{\pgfqpoint{1.557026in}{0.681680in}}%
\pgfpathlineto{\pgfqpoint{1.627242in}{0.680913in}}%
\pgfpathlineto{\pgfqpoint{1.737728in}{0.680478in}}%
\pgfpathlineto{\pgfqpoint{1.887036in}{0.679610in}}%
\pgfpathlineto{\pgfqpoint{2.037481in}{0.678826in}}%
\pgfpathlineto{\pgfqpoint{2.258348in}{0.677741in}}%
\pgfpathlineto{\pgfqpoint{2.626338in}{0.676361in}}%
\pgfpathlineto{\pgfqpoint{3.263784in}{0.674352in}}%
\pgfpathlineto{\pgfqpoint{5.322800in}{0.670138in}}%
\pgfpathlineto{\pgfqpoint{6.888192in}{0.683471in}}%
\pgfpathlineto{\pgfqpoint{4.623274in}{0.688107in}}%
\pgfpathlineto{\pgfqpoint{3.922083in}{0.690316in}}%
\pgfpathlineto{\pgfqpoint{3.517294in}{0.691835in}}%
\pgfpathlineto{\pgfqpoint{3.274341in}{0.693028in}}%
\pgfpathlineto{\pgfqpoint{3.108851in}{0.693891in}}%
\pgfpathlineto{\pgfqpoint{2.944612in}{0.694845in}}%
\pgfpathlineto{\pgfqpoint{2.823078in}{0.695324in}}%
\pgfpathlineto{\pgfqpoint{2.745840in}{0.696168in}}%
\pgfpathlineto{\pgfqpoint{2.647748in}{0.696667in}}%
\pgfpathlineto{\pgfqpoint{2.571667in}{0.696836in}}%
\pgfpathlineto{\pgfqpoint{2.539290in}{0.697351in}}%
\pgfpathlineto{\pgfqpoint{2.479468in}{0.697460in}}%
\pgfpathlineto{\pgfqpoint{2.438334in}{0.697795in}}%
\pgfpathlineto{\pgfqpoint{2.409554in}{0.697909in}}%
\pgfpathlineto{\pgfqpoint{2.301002in}{0.698784in}}%
\pgfpathlineto{\pgfqpoint{2.278129in}{0.698798in}}%
\pgfpathlineto{\pgfqpoint{2.259841in}{0.698842in}}%
\pgfpathlineto{\pgfqpoint{2.234540in}{0.699174in}}%
\pgfpathlineto{\pgfqpoint{2.212467in}{0.699905in}}%
\pgfpathlineto{\pgfqpoint{2.198690in}{0.701434in}}%
\pgfpathlineto{\pgfqpoint{2.193297in}{0.702172in}}%
\pgfpathlineto{\pgfqpoint{2.187458in}{0.702763in}}%
\pgfpathlineto{\pgfqpoint{2.171432in}{0.705082in}}%
\pgfpathlineto{\pgfqpoint{2.163334in}{0.705692in}}%
\pgfpathlineto{\pgfqpoint{2.161059in}{0.706043in}}%
\pgfpathlineto{\pgfqpoint{2.157055in}{0.710594in}}%
\pgfpathlineto{\pgfqpoint{2.105914in}{0.712924in}}%
\pgfpathlineto{\pgfqpoint{2.104964in}{0.713831in}}%
\pgfpathlineto{\pgfqpoint{2.098822in}{0.714079in}}%
\pgfpathlineto{\pgfqpoint{2.085843in}{0.716510in}}%
\pgfpathlineto{\pgfqpoint{2.080349in}{0.716749in}}%
\pgfpathlineto{\pgfqpoint{2.080251in}{0.718197in}}%
\pgfpathlineto{\pgfqpoint{2.076904in}{0.719544in}}%
\pgfpathlineto{\pgfqpoint{2.051349in}{0.722081in}}%
\pgfpathlineto{\pgfqpoint{2.045595in}{0.723715in}}%
\pgfpathlineto{\pgfqpoint{2.037125in}{0.724332in}}%
\pgfpathlineto{\pgfqpoint{2.033576in}{0.725697in}}%
\pgfpathlineto{\pgfqpoint{2.032252in}{0.729528in}}%
\pgfpathlineto{\pgfqpoint{2.031338in}{0.731724in}}%
\pgfpathlineto{\pgfqpoint{2.008926in}{0.733388in}}%
\pgfpathlineto{\pgfqpoint{1.984450in}{0.738340in}}%
\pgfpathlineto{\pgfqpoint{1.966417in}{0.745282in}}%
\pgfpathlineto{\pgfqpoint{1.949459in}{0.750264in}}%
\pgfpathlineto{\pgfqpoint{1.945246in}{0.752971in}}%
\pgfpathlineto{\pgfqpoint{1.944819in}{0.754750in}}%
\pgfpathlineto{\pgfqpoint{1.938397in}{0.758644in}}%
\pgfpathlineto{\pgfqpoint{1.930681in}{0.763152in}}%
\pgfpathlineto{\pgfqpoint{1.927003in}{0.764692in}}%
\pgfpathlineto{\pgfqpoint{1.923418in}{0.773495in}}%
\pgfpathlineto{\pgfqpoint{1.912932in}{0.774543in}}%
\pgfpathlineto{\pgfqpoint{1.907246in}{0.780181in}}%
\pgfpathlineto{\pgfqpoint{1.904902in}{0.803379in}}%
\pgfpathlineto{\pgfqpoint{1.898279in}{0.804188in}}%
\pgfpathlineto{\pgfqpoint{1.893486in}{0.814892in}}%
\pgfpathlineto{\pgfqpoint{1.889178in}{0.851064in}}%
\pgfpathlineto{\pgfqpoint{1.885331in}{0.857978in}}%
\pgfpathlineto{\pgfqpoint{1.878129in}{0.908856in}}%
\pgfpathlineto{\pgfqpoint{1.867684in}{0.996854in}}%
\pgfpathlineto{\pgfqpoint{1.848679in}{1.461617in}}%
\pgfpathlineto{\pgfqpoint{0.741425in}{1.377543in}}%
\pgfpathclose%
\pgfusepath{stroke,fill}%
\end{pgfscope}%
\begin{pgfscope}%
\pgfpathrectangle{\pgfqpoint{0.688192in}{0.670138in}}{\pgfqpoint{6.200000in}{4.620000in}}%
\pgfusepath{clip}%
\pgfsetbuttcap%
\pgfsetroundjoin%
\pgfsetlinewidth{1.003750pt}%
\definecolor{currentstroke}{rgb}{1.000000,0.000000,0.000000}%
\pgfsetstrokecolor{currentstroke}%
\pgfsetdash{}{0pt}%
\pgfpathmoveto{\pgfqpoint{1.380312in}{4.826960in}}%
\pgfpathcurveto{\pgfqpoint{1.388548in}{4.826960in}}{\pgfqpoint{1.396449in}{4.830233in}}{\pgfqpoint{1.402272in}{4.836057in}}%
\pgfpathcurveto{\pgfqpoint{1.408096in}{4.841881in}}{\pgfqpoint{1.411369in}{4.849781in}}{\pgfqpoint{1.411369in}{4.858017in}}%
\pgfpathcurveto{\pgfqpoint{1.411369in}{4.866253in}}{\pgfqpoint{1.408096in}{4.874153in}}{\pgfqpoint{1.402272in}{4.879977in}}%
\pgfpathcurveto{\pgfqpoint{1.396449in}{4.885801in}}{\pgfqpoint{1.388548in}{4.889073in}}{\pgfqpoint{1.380312in}{4.889073in}}%
\pgfpathcurveto{\pgfqpoint{1.372076in}{4.889073in}}{\pgfqpoint{1.364176in}{4.885801in}}{\pgfqpoint{1.358352in}{4.879977in}}%
\pgfpathcurveto{\pgfqpoint{1.352528in}{4.874153in}}{\pgfqpoint{1.349256in}{4.866253in}}{\pgfqpoint{1.349256in}{4.858017in}}%
\pgfpathcurveto{\pgfqpoint{1.349256in}{4.849781in}}{\pgfqpoint{1.352528in}{4.841881in}}{\pgfqpoint{1.358352in}{4.836057in}}%
\pgfpathcurveto{\pgfqpoint{1.364176in}{4.830233in}}{\pgfqpoint{1.372076in}{4.826960in}}{\pgfqpoint{1.380312in}{4.826960in}}%
\pgfpathlineto{\pgfqpoint{1.380312in}{4.826960in}}%
\pgfpathclose%
\pgfusepath{stroke}%
\end{pgfscope}%
\begin{pgfscope}%
\pgfpathrectangle{\pgfqpoint{0.688192in}{0.670138in}}{\pgfqpoint{6.200000in}{4.620000in}}%
\pgfusepath{clip}%
\pgfsetbuttcap%
\pgfsetroundjoin%
\pgfsetlinewidth{1.003750pt}%
\definecolor{currentstroke}{rgb}{1.000000,0.000000,0.000000}%
\pgfsetstrokecolor{currentstroke}%
\pgfsetdash{}{0pt}%
\pgfpathmoveto{\pgfqpoint{1.103776in}{2.239473in}}%
\pgfpathcurveto{\pgfqpoint{1.112013in}{2.239473in}}{\pgfqpoint{1.119913in}{2.242745in}}{\pgfqpoint{1.125737in}{2.248569in}}%
\pgfpathcurveto{\pgfqpoint{1.131560in}{2.254393in}}{\pgfqpoint{1.134833in}{2.262293in}}{\pgfqpoint{1.134833in}{2.270529in}}%
\pgfpathcurveto{\pgfqpoint{1.134833in}{2.278765in}}{\pgfqpoint{1.131560in}{2.286665in}}{\pgfqpoint{1.125737in}{2.292489in}}%
\pgfpathcurveto{\pgfqpoint{1.119913in}{2.298313in}}{\pgfqpoint{1.112013in}{2.301586in}}{\pgfqpoint{1.103776in}{2.301586in}}%
\pgfpathcurveto{\pgfqpoint{1.095540in}{2.301586in}}{\pgfqpoint{1.087640in}{2.298313in}}{\pgfqpoint{1.081816in}{2.292489in}}%
\pgfpathcurveto{\pgfqpoint{1.075992in}{2.286665in}}{\pgfqpoint{1.072720in}{2.278765in}}{\pgfqpoint{1.072720in}{2.270529in}}%
\pgfpathcurveto{\pgfqpoint{1.072720in}{2.262293in}}{\pgfqpoint{1.075992in}{2.254393in}}{\pgfqpoint{1.081816in}{2.248569in}}%
\pgfpathcurveto{\pgfqpoint{1.087640in}{2.242745in}}{\pgfqpoint{1.095540in}{2.239473in}}{\pgfqpoint{1.103776in}{2.239473in}}%
\pgfpathlineto{\pgfqpoint{1.103776in}{2.239473in}}%
\pgfpathclose%
\pgfusepath{stroke}%
\end{pgfscope}%
\begin{pgfscope}%
\pgfpathrectangle{\pgfqpoint{0.688192in}{0.670138in}}{\pgfqpoint{6.200000in}{4.620000in}}%
\pgfusepath{clip}%
\pgfsetbuttcap%
\pgfsetroundjoin%
\pgfsetlinewidth{1.003750pt}%
\definecolor{currentstroke}{rgb}{1.000000,0.000000,0.000000}%
\pgfsetstrokecolor{currentstroke}%
\pgfsetdash{}{0pt}%
\pgfpathmoveto{\pgfqpoint{1.146295in}{2.309659in}}%
\pgfpathcurveto{\pgfqpoint{1.154531in}{2.309659in}}{\pgfqpoint{1.162431in}{2.312932in}}{\pgfqpoint{1.168255in}{2.318756in}}%
\pgfpathcurveto{\pgfqpoint{1.174079in}{2.324579in}}{\pgfqpoint{1.177352in}{2.332480in}}{\pgfqpoint{1.177352in}{2.340716in}}%
\pgfpathcurveto{\pgfqpoint{1.177352in}{2.348952in}}{\pgfqpoint{1.174079in}{2.356852in}}{\pgfqpoint{1.168255in}{2.362676in}}%
\pgfpathcurveto{\pgfqpoint{1.162431in}{2.368500in}}{\pgfqpoint{1.154531in}{2.371772in}}{\pgfqpoint{1.146295in}{2.371772in}}%
\pgfpathcurveto{\pgfqpoint{1.138059in}{2.371772in}}{\pgfqpoint{1.130159in}{2.368500in}}{\pgfqpoint{1.124335in}{2.362676in}}%
\pgfpathcurveto{\pgfqpoint{1.118511in}{2.356852in}}{\pgfqpoint{1.115239in}{2.348952in}}{\pgfqpoint{1.115239in}{2.340716in}}%
\pgfpathcurveto{\pgfqpoint{1.115239in}{2.332480in}}{\pgfqpoint{1.118511in}{2.324579in}}{\pgfqpoint{1.124335in}{2.318756in}}%
\pgfpathcurveto{\pgfqpoint{1.130159in}{2.312932in}}{\pgfqpoint{1.138059in}{2.309659in}}{\pgfqpoint{1.146295in}{2.309659in}}%
\pgfpathlineto{\pgfqpoint{1.146295in}{2.309659in}}%
\pgfpathclose%
\pgfusepath{stroke}%
\end{pgfscope}%
\begin{pgfscope}%
\pgfpathrectangle{\pgfqpoint{0.688192in}{0.670138in}}{\pgfqpoint{6.200000in}{4.620000in}}%
\pgfusepath{clip}%
\pgfsetbuttcap%
\pgfsetroundjoin%
\pgfsetlinewidth{1.003750pt}%
\definecolor{currentstroke}{rgb}{1.000000,0.000000,0.000000}%
\pgfsetstrokecolor{currentstroke}%
\pgfsetdash{}{0pt}%
\pgfpathmoveto{\pgfqpoint{1.283972in}{2.337286in}}%
\pgfpathcurveto{\pgfqpoint{1.292208in}{2.337286in}}{\pgfqpoint{1.300108in}{2.340558in}}{\pgfqpoint{1.305932in}{2.346382in}}%
\pgfpathcurveto{\pgfqpoint{1.311756in}{2.352206in}}{\pgfqpoint{1.315028in}{2.360106in}}{\pgfqpoint{1.315028in}{2.368343in}}%
\pgfpathcurveto{\pgfqpoint{1.315028in}{2.376579in}}{\pgfqpoint{1.311756in}{2.384479in}}{\pgfqpoint{1.305932in}{2.390303in}}%
\pgfpathcurveto{\pgfqpoint{1.300108in}{2.396127in}}{\pgfqpoint{1.292208in}{2.399399in}}{\pgfqpoint{1.283972in}{2.399399in}}%
\pgfpathcurveto{\pgfqpoint{1.275735in}{2.399399in}}{\pgfqpoint{1.267835in}{2.396127in}}{\pgfqpoint{1.262011in}{2.390303in}}%
\pgfpathcurveto{\pgfqpoint{1.256187in}{2.384479in}}{\pgfqpoint{1.252915in}{2.376579in}}{\pgfqpoint{1.252915in}{2.368343in}}%
\pgfpathcurveto{\pgfqpoint{1.252915in}{2.360106in}}{\pgfqpoint{1.256187in}{2.352206in}}{\pgfqpoint{1.262011in}{2.346382in}}%
\pgfpathcurveto{\pgfqpoint{1.267835in}{2.340558in}}{\pgfqpoint{1.275735in}{2.337286in}}{\pgfqpoint{1.283972in}{2.337286in}}%
\pgfpathlineto{\pgfqpoint{1.283972in}{2.337286in}}%
\pgfpathclose%
\pgfusepath{stroke}%
\end{pgfscope}%
\begin{pgfscope}%
\pgfpathrectangle{\pgfqpoint{0.688192in}{0.670138in}}{\pgfqpoint{6.200000in}{4.620000in}}%
\pgfusepath{clip}%
\pgfsetbuttcap%
\pgfsetroundjoin%
\pgfsetlinewidth{1.003750pt}%
\definecolor{currentstroke}{rgb}{1.000000,0.000000,0.000000}%
\pgfsetstrokecolor{currentstroke}%
\pgfsetdash{}{0pt}%
\pgfpathmoveto{\pgfqpoint{1.194044in}{2.459126in}}%
\pgfpathcurveto{\pgfqpoint{1.202281in}{2.459126in}}{\pgfqpoint{1.210181in}{2.462398in}}{\pgfqpoint{1.216005in}{2.468222in}}%
\pgfpathcurveto{\pgfqpoint{1.221828in}{2.474046in}}{\pgfqpoint{1.225101in}{2.481946in}}{\pgfqpoint{1.225101in}{2.490182in}}%
\pgfpathcurveto{\pgfqpoint{1.225101in}{2.498419in}}{\pgfqpoint{1.221828in}{2.506319in}}{\pgfqpoint{1.216005in}{2.512143in}}%
\pgfpathcurveto{\pgfqpoint{1.210181in}{2.517967in}}{\pgfqpoint{1.202281in}{2.521239in}}{\pgfqpoint{1.194044in}{2.521239in}}%
\pgfpathcurveto{\pgfqpoint{1.185808in}{2.521239in}}{\pgfqpoint{1.177908in}{2.517967in}}{\pgfqpoint{1.172084in}{2.512143in}}%
\pgfpathcurveto{\pgfqpoint{1.166260in}{2.506319in}}{\pgfqpoint{1.162988in}{2.498419in}}{\pgfqpoint{1.162988in}{2.490182in}}%
\pgfpathcurveto{\pgfqpoint{1.162988in}{2.481946in}}{\pgfqpoint{1.166260in}{2.474046in}}{\pgfqpoint{1.172084in}{2.468222in}}%
\pgfpathcurveto{\pgfqpoint{1.177908in}{2.462398in}}{\pgfqpoint{1.185808in}{2.459126in}}{\pgfqpoint{1.194044in}{2.459126in}}%
\pgfpathlineto{\pgfqpoint{1.194044in}{2.459126in}}%
\pgfpathclose%
\pgfusepath{stroke}%
\end{pgfscope}%
\begin{pgfscope}%
\pgfpathrectangle{\pgfqpoint{0.688192in}{0.670138in}}{\pgfqpoint{6.200000in}{4.620000in}}%
\pgfusepath{clip}%
\pgfsetbuttcap%
\pgfsetroundjoin%
\pgfsetlinewidth{1.003750pt}%
\definecolor{currentstroke}{rgb}{1.000000,0.000000,0.000000}%
\pgfsetstrokecolor{currentstroke}%
\pgfsetdash{}{0pt}%
\pgfpathmoveto{\pgfqpoint{1.126156in}{2.039611in}}%
\pgfpathcurveto{\pgfqpoint{1.134392in}{2.039611in}}{\pgfqpoint{1.142292in}{2.042884in}}{\pgfqpoint{1.148116in}{2.048708in}}%
\pgfpathcurveto{\pgfqpoint{1.153940in}{2.054532in}}{\pgfqpoint{1.157212in}{2.062432in}}{\pgfqpoint{1.157212in}{2.070668in}}%
\pgfpathcurveto{\pgfqpoint{1.157212in}{2.078904in}}{\pgfqpoint{1.153940in}{2.086804in}}{\pgfqpoint{1.148116in}{2.092628in}}%
\pgfpathcurveto{\pgfqpoint{1.142292in}{2.098452in}}{\pgfqpoint{1.134392in}{2.101724in}}{\pgfqpoint{1.126156in}{2.101724in}}%
\pgfpathcurveto{\pgfqpoint{1.117920in}{2.101724in}}{\pgfqpoint{1.110020in}{2.098452in}}{\pgfqpoint{1.104196in}{2.092628in}}%
\pgfpathcurveto{\pgfqpoint{1.098372in}{2.086804in}}{\pgfqpoint{1.095099in}{2.078904in}}{\pgfqpoint{1.095099in}{2.070668in}}%
\pgfpathcurveto{\pgfqpoint{1.095099in}{2.062432in}}{\pgfqpoint{1.098372in}{2.054532in}}{\pgfqpoint{1.104196in}{2.048708in}}%
\pgfpathcurveto{\pgfqpoint{1.110020in}{2.042884in}}{\pgfqpoint{1.117920in}{2.039611in}}{\pgfqpoint{1.126156in}{2.039611in}}%
\pgfpathlineto{\pgfqpoint{1.126156in}{2.039611in}}%
\pgfpathclose%
\pgfusepath{stroke}%
\end{pgfscope}%
\begin{pgfscope}%
\pgfpathrectangle{\pgfqpoint{0.688192in}{0.670138in}}{\pgfqpoint{6.200000in}{4.620000in}}%
\pgfusepath{clip}%
\pgfsetbuttcap%
\pgfsetroundjoin%
\pgfsetlinewidth{1.003750pt}%
\definecolor{currentstroke}{rgb}{1.000000,0.000000,0.000000}%
\pgfsetstrokecolor{currentstroke}%
\pgfsetdash{}{0pt}%
\pgfpathmoveto{\pgfqpoint{1.073783in}{1.736453in}}%
\pgfpathcurveto{\pgfqpoint{1.082020in}{1.736453in}}{\pgfqpoint{1.089920in}{1.739725in}}{\pgfqpoint{1.095744in}{1.745549in}}%
\pgfpathcurveto{\pgfqpoint{1.101568in}{1.751373in}}{\pgfqpoint{1.104840in}{1.759273in}}{\pgfqpoint{1.104840in}{1.767510in}}%
\pgfpathcurveto{\pgfqpoint{1.104840in}{1.775746in}}{\pgfqpoint{1.101568in}{1.783646in}}{\pgfqpoint{1.095744in}{1.789470in}}%
\pgfpathcurveto{\pgfqpoint{1.089920in}{1.795294in}}{\pgfqpoint{1.082020in}{1.798566in}}{\pgfqpoint{1.073783in}{1.798566in}}%
\pgfpathcurveto{\pgfqpoint{1.065547in}{1.798566in}}{\pgfqpoint{1.057647in}{1.795294in}}{\pgfqpoint{1.051823in}{1.789470in}}%
\pgfpathcurveto{\pgfqpoint{1.045999in}{1.783646in}}{\pgfqpoint{1.042727in}{1.775746in}}{\pgfqpoint{1.042727in}{1.767510in}}%
\pgfpathcurveto{\pgfqpoint{1.042727in}{1.759273in}}{\pgfqpoint{1.045999in}{1.751373in}}{\pgfqpoint{1.051823in}{1.745549in}}%
\pgfpathcurveto{\pgfqpoint{1.057647in}{1.739725in}}{\pgfqpoint{1.065547in}{1.736453in}}{\pgfqpoint{1.073783in}{1.736453in}}%
\pgfpathlineto{\pgfqpoint{1.073783in}{1.736453in}}%
\pgfpathclose%
\pgfusepath{stroke}%
\end{pgfscope}%
\begin{pgfscope}%
\pgfpathrectangle{\pgfqpoint{0.688192in}{0.670138in}}{\pgfqpoint{6.200000in}{4.620000in}}%
\pgfusepath{clip}%
\pgfsetbuttcap%
\pgfsetroundjoin%
\pgfsetlinewidth{1.003750pt}%
\definecolor{currentstroke}{rgb}{1.000000,0.000000,0.000000}%
\pgfsetstrokecolor{currentstroke}%
\pgfsetdash{}{0pt}%
\pgfpathmoveto{\pgfqpoint{1.071350in}{1.723563in}}%
\pgfpathcurveto{\pgfqpoint{1.079586in}{1.723563in}}{\pgfqpoint{1.087486in}{1.726835in}}{\pgfqpoint{1.093310in}{1.732659in}}%
\pgfpathcurveto{\pgfqpoint{1.099134in}{1.738483in}}{\pgfqpoint{1.102407in}{1.746383in}}{\pgfqpoint{1.102407in}{1.754620in}}%
\pgfpathcurveto{\pgfqpoint{1.102407in}{1.762856in}}{\pgfqpoint{1.099134in}{1.770756in}}{\pgfqpoint{1.093310in}{1.776580in}}%
\pgfpathcurveto{\pgfqpoint{1.087486in}{1.782404in}}{\pgfqpoint{1.079586in}{1.785676in}}{\pgfqpoint{1.071350in}{1.785676in}}%
\pgfpathcurveto{\pgfqpoint{1.063114in}{1.785676in}}{\pgfqpoint{1.055214in}{1.782404in}}{\pgfqpoint{1.049390in}{1.776580in}}%
\pgfpathcurveto{\pgfqpoint{1.043566in}{1.770756in}}{\pgfqpoint{1.040294in}{1.762856in}}{\pgfqpoint{1.040294in}{1.754620in}}%
\pgfpathcurveto{\pgfqpoint{1.040294in}{1.746383in}}{\pgfqpoint{1.043566in}{1.738483in}}{\pgfqpoint{1.049390in}{1.732659in}}%
\pgfpathcurveto{\pgfqpoint{1.055214in}{1.726835in}}{\pgfqpoint{1.063114in}{1.723563in}}{\pgfqpoint{1.071350in}{1.723563in}}%
\pgfpathlineto{\pgfqpoint{1.071350in}{1.723563in}}%
\pgfpathclose%
\pgfusepath{stroke}%
\end{pgfscope}%
\begin{pgfscope}%
\pgfpathrectangle{\pgfqpoint{0.688192in}{0.670138in}}{\pgfqpoint{6.200000in}{4.620000in}}%
\pgfusepath{clip}%
\pgfsetbuttcap%
\pgfsetroundjoin%
\pgfsetlinewidth{1.003750pt}%
\definecolor{currentstroke}{rgb}{1.000000,0.000000,0.000000}%
\pgfsetstrokecolor{currentstroke}%
\pgfsetdash{}{0pt}%
\pgfpathmoveto{\pgfqpoint{1.176638in}{2.187985in}}%
\pgfpathcurveto{\pgfqpoint{1.184874in}{2.187985in}}{\pgfqpoint{1.192774in}{2.191257in}}{\pgfqpoint{1.198598in}{2.197081in}}%
\pgfpathcurveto{\pgfqpoint{1.204422in}{2.202905in}}{\pgfqpoint{1.207694in}{2.210805in}}{\pgfqpoint{1.207694in}{2.219041in}}%
\pgfpathcurveto{\pgfqpoint{1.207694in}{2.227277in}}{\pgfqpoint{1.204422in}{2.235178in}}{\pgfqpoint{1.198598in}{2.241001in}}%
\pgfpathcurveto{\pgfqpoint{1.192774in}{2.246825in}}{\pgfqpoint{1.184874in}{2.250098in}}{\pgfqpoint{1.176638in}{2.250098in}}%
\pgfpathcurveto{\pgfqpoint{1.168401in}{2.250098in}}{\pgfqpoint{1.160501in}{2.246825in}}{\pgfqpoint{1.154677in}{2.241001in}}%
\pgfpathcurveto{\pgfqpoint{1.148853in}{2.235178in}}{\pgfqpoint{1.145581in}{2.227277in}}{\pgfqpoint{1.145581in}{2.219041in}}%
\pgfpathcurveto{\pgfqpoint{1.145581in}{2.210805in}}{\pgfqpoint{1.148853in}{2.202905in}}{\pgfqpoint{1.154677in}{2.197081in}}%
\pgfpathcurveto{\pgfqpoint{1.160501in}{2.191257in}}{\pgfqpoint{1.168401in}{2.187985in}}{\pgfqpoint{1.176638in}{2.187985in}}%
\pgfpathlineto{\pgfqpoint{1.176638in}{2.187985in}}%
\pgfpathclose%
\pgfusepath{stroke}%
\end{pgfscope}%
\begin{pgfscope}%
\pgfpathrectangle{\pgfqpoint{0.688192in}{0.670138in}}{\pgfqpoint{6.200000in}{4.620000in}}%
\pgfusepath{clip}%
\pgfsetbuttcap%
\pgfsetroundjoin%
\pgfsetlinewidth{1.003750pt}%
\definecolor{currentstroke}{rgb}{1.000000,0.000000,0.000000}%
\pgfsetstrokecolor{currentstroke}%
\pgfsetdash{}{0pt}%
\pgfpathmoveto{\pgfqpoint{1.160265in}{1.939369in}}%
\pgfpathcurveto{\pgfqpoint{1.168501in}{1.939369in}}{\pgfqpoint{1.176401in}{1.942641in}}{\pgfqpoint{1.182225in}{1.948465in}}%
\pgfpathcurveto{\pgfqpoint{1.188049in}{1.954289in}}{\pgfqpoint{1.191321in}{1.962189in}}{\pgfqpoint{1.191321in}{1.970426in}}%
\pgfpathcurveto{\pgfqpoint{1.191321in}{1.978662in}}{\pgfqpoint{1.188049in}{1.986562in}}{\pgfqpoint{1.182225in}{1.992386in}}%
\pgfpathcurveto{\pgfqpoint{1.176401in}{1.998210in}}{\pgfqpoint{1.168501in}{2.001482in}}{\pgfqpoint{1.160265in}{2.001482in}}%
\pgfpathcurveto{\pgfqpoint{1.152029in}{2.001482in}}{\pgfqpoint{1.144129in}{1.998210in}}{\pgfqpoint{1.138305in}{1.992386in}}%
\pgfpathcurveto{\pgfqpoint{1.132481in}{1.986562in}}{\pgfqpoint{1.129208in}{1.978662in}}{\pgfqpoint{1.129208in}{1.970426in}}%
\pgfpathcurveto{\pgfqpoint{1.129208in}{1.962189in}}{\pgfqpoint{1.132481in}{1.954289in}}{\pgfqpoint{1.138305in}{1.948465in}}%
\pgfpathcurveto{\pgfqpoint{1.144129in}{1.942641in}}{\pgfqpoint{1.152029in}{1.939369in}}{\pgfqpoint{1.160265in}{1.939369in}}%
\pgfpathlineto{\pgfqpoint{1.160265in}{1.939369in}}%
\pgfpathclose%
\pgfusepath{stroke}%
\end{pgfscope}%
\begin{pgfscope}%
\pgfpathrectangle{\pgfqpoint{0.688192in}{0.670138in}}{\pgfqpoint{6.200000in}{4.620000in}}%
\pgfusepath{clip}%
\pgfsetbuttcap%
\pgfsetroundjoin%
\pgfsetlinewidth{1.003750pt}%
\definecolor{currentstroke}{rgb}{1.000000,0.000000,0.000000}%
\pgfsetstrokecolor{currentstroke}%
\pgfsetdash{}{0pt}%
\pgfpathmoveto{\pgfqpoint{1.160053in}{1.935902in}}%
\pgfpathcurveto{\pgfqpoint{1.168289in}{1.935902in}}{\pgfqpoint{1.176189in}{1.939174in}}{\pgfqpoint{1.182013in}{1.944998in}}%
\pgfpathcurveto{\pgfqpoint{1.187837in}{1.950822in}}{\pgfqpoint{1.191109in}{1.958722in}}{\pgfqpoint{1.191109in}{1.966959in}}%
\pgfpathcurveto{\pgfqpoint{1.191109in}{1.975195in}}{\pgfqpoint{1.187837in}{1.983095in}}{\pgfqpoint{1.182013in}{1.988919in}}%
\pgfpathcurveto{\pgfqpoint{1.176189in}{1.994743in}}{\pgfqpoint{1.168289in}{1.998015in}}{\pgfqpoint{1.160053in}{1.998015in}}%
\pgfpathcurveto{\pgfqpoint{1.151817in}{1.998015in}}{\pgfqpoint{1.143916in}{1.994743in}}{\pgfqpoint{1.138093in}{1.988919in}}%
\pgfpathcurveto{\pgfqpoint{1.132269in}{1.983095in}}{\pgfqpoint{1.128996in}{1.975195in}}{\pgfqpoint{1.128996in}{1.966959in}}%
\pgfpathcurveto{\pgfqpoint{1.128996in}{1.958722in}}{\pgfqpoint{1.132269in}{1.950822in}}{\pgfqpoint{1.138093in}{1.944998in}}%
\pgfpathcurveto{\pgfqpoint{1.143916in}{1.939174in}}{\pgfqpoint{1.151817in}{1.935902in}}{\pgfqpoint{1.160053in}{1.935902in}}%
\pgfpathlineto{\pgfqpoint{1.160053in}{1.935902in}}%
\pgfpathclose%
\pgfusepath{stroke}%
\end{pgfscope}%
\begin{pgfscope}%
\pgfpathrectangle{\pgfqpoint{0.688192in}{0.670138in}}{\pgfqpoint{6.200000in}{4.620000in}}%
\pgfusepath{clip}%
\pgfsetbuttcap%
\pgfsetroundjoin%
\pgfsetlinewidth{1.003750pt}%
\definecolor{currentstroke}{rgb}{1.000000,0.000000,0.000000}%
\pgfsetstrokecolor{currentstroke}%
\pgfsetdash{}{0pt}%
\pgfpathmoveto{\pgfqpoint{1.157986in}{1.610198in}}%
\pgfpathcurveto{\pgfqpoint{1.166223in}{1.610198in}}{\pgfqpoint{1.174123in}{1.613470in}}{\pgfqpoint{1.179947in}{1.619294in}}%
\pgfpathcurveto{\pgfqpoint{1.185771in}{1.625118in}}{\pgfqpoint{1.189043in}{1.633018in}}{\pgfqpoint{1.189043in}{1.641254in}}%
\pgfpathcurveto{\pgfqpoint{1.189043in}{1.649491in}}{\pgfqpoint{1.185771in}{1.657391in}}{\pgfqpoint{1.179947in}{1.663215in}}%
\pgfpathcurveto{\pgfqpoint{1.174123in}{1.669038in}}{\pgfqpoint{1.166223in}{1.672311in}}{\pgfqpoint{1.157986in}{1.672311in}}%
\pgfpathcurveto{\pgfqpoint{1.149750in}{1.672311in}}{\pgfqpoint{1.141850in}{1.669038in}}{\pgfqpoint{1.136026in}{1.663215in}}%
\pgfpathcurveto{\pgfqpoint{1.130202in}{1.657391in}}{\pgfqpoint{1.126930in}{1.649491in}}{\pgfqpoint{1.126930in}{1.641254in}}%
\pgfpathcurveto{\pgfqpoint{1.126930in}{1.633018in}}{\pgfqpoint{1.130202in}{1.625118in}}{\pgfqpoint{1.136026in}{1.619294in}}%
\pgfpathcurveto{\pgfqpoint{1.141850in}{1.613470in}}{\pgfqpoint{1.149750in}{1.610198in}}{\pgfqpoint{1.157986in}{1.610198in}}%
\pgfpathlineto{\pgfqpoint{1.157986in}{1.610198in}}%
\pgfpathclose%
\pgfusepath{stroke}%
\end{pgfscope}%
\begin{pgfscope}%
\pgfpathrectangle{\pgfqpoint{0.688192in}{0.670138in}}{\pgfqpoint{6.200000in}{4.620000in}}%
\pgfusepath{clip}%
\pgfsetbuttcap%
\pgfsetroundjoin%
\pgfsetlinewidth{1.003750pt}%
\definecolor{currentstroke}{rgb}{1.000000,0.000000,0.000000}%
\pgfsetstrokecolor{currentstroke}%
\pgfsetdash{}{0pt}%
\pgfpathmoveto{\pgfqpoint{1.155798in}{1.612752in}}%
\pgfpathcurveto{\pgfqpoint{1.164035in}{1.612752in}}{\pgfqpoint{1.171935in}{1.616024in}}{\pgfqpoint{1.177759in}{1.621848in}}%
\pgfpathcurveto{\pgfqpoint{1.183582in}{1.627672in}}{\pgfqpoint{1.186855in}{1.635572in}}{\pgfqpoint{1.186855in}{1.643808in}}%
\pgfpathcurveto{\pgfqpoint{1.186855in}{1.652044in}}{\pgfqpoint{1.183582in}{1.659945in}}{\pgfqpoint{1.177759in}{1.665768in}}%
\pgfpathcurveto{\pgfqpoint{1.171935in}{1.671592in}}{\pgfqpoint{1.164035in}{1.674865in}}{\pgfqpoint{1.155798in}{1.674865in}}%
\pgfpathcurveto{\pgfqpoint{1.147562in}{1.674865in}}{\pgfqpoint{1.139662in}{1.671592in}}{\pgfqpoint{1.133838in}{1.665768in}}%
\pgfpathcurveto{\pgfqpoint{1.128014in}{1.659945in}}{\pgfqpoint{1.124742in}{1.652044in}}{\pgfqpoint{1.124742in}{1.643808in}}%
\pgfpathcurveto{\pgfqpoint{1.124742in}{1.635572in}}{\pgfqpoint{1.128014in}{1.627672in}}{\pgfqpoint{1.133838in}{1.621848in}}%
\pgfpathcurveto{\pgfqpoint{1.139662in}{1.616024in}}{\pgfqpoint{1.147562in}{1.612752in}}{\pgfqpoint{1.155798in}{1.612752in}}%
\pgfpathlineto{\pgfqpoint{1.155798in}{1.612752in}}%
\pgfpathclose%
\pgfusepath{stroke}%
\end{pgfscope}%
\begin{pgfscope}%
\pgfpathrectangle{\pgfqpoint{0.688192in}{0.670138in}}{\pgfqpoint{6.200000in}{4.620000in}}%
\pgfusepath{clip}%
\pgfsetbuttcap%
\pgfsetroundjoin%
\pgfsetlinewidth{1.003750pt}%
\definecolor{currentstroke}{rgb}{1.000000,0.000000,0.000000}%
\pgfsetstrokecolor{currentstroke}%
\pgfsetdash{}{0pt}%
\pgfpathmoveto{\pgfqpoint{1.295347in}{1.895177in}}%
\pgfpathcurveto{\pgfqpoint{1.303583in}{1.895177in}}{\pgfqpoint{1.311483in}{1.898449in}}{\pgfqpoint{1.317307in}{1.904273in}}%
\pgfpathcurveto{\pgfqpoint{1.323131in}{1.910097in}}{\pgfqpoint{1.326403in}{1.917997in}}{\pgfqpoint{1.326403in}{1.926233in}}%
\pgfpathcurveto{\pgfqpoint{1.326403in}{1.934470in}}{\pgfqpoint{1.323131in}{1.942370in}}{\pgfqpoint{1.317307in}{1.948193in}}%
\pgfpathcurveto{\pgfqpoint{1.311483in}{1.954017in}}{\pgfqpoint{1.303583in}{1.957290in}}{\pgfqpoint{1.295347in}{1.957290in}}%
\pgfpathcurveto{\pgfqpoint{1.287111in}{1.957290in}}{\pgfqpoint{1.279211in}{1.954017in}}{\pgfqpoint{1.273387in}{1.948193in}}%
\pgfpathcurveto{\pgfqpoint{1.267563in}{1.942370in}}{\pgfqpoint{1.264290in}{1.934470in}}{\pgfqpoint{1.264290in}{1.926233in}}%
\pgfpathcurveto{\pgfqpoint{1.264290in}{1.917997in}}{\pgfqpoint{1.267563in}{1.910097in}}{\pgfqpoint{1.273387in}{1.904273in}}%
\pgfpathcurveto{\pgfqpoint{1.279211in}{1.898449in}}{\pgfqpoint{1.287111in}{1.895177in}}{\pgfqpoint{1.295347in}{1.895177in}}%
\pgfpathlineto{\pgfqpoint{1.295347in}{1.895177in}}%
\pgfpathclose%
\pgfusepath{stroke}%
\end{pgfscope}%
\begin{pgfscope}%
\pgfpathrectangle{\pgfqpoint{0.688192in}{0.670138in}}{\pgfqpoint{6.200000in}{4.620000in}}%
\pgfusepath{clip}%
\pgfsetbuttcap%
\pgfsetroundjoin%
\pgfsetlinewidth{1.003750pt}%
\definecolor{currentstroke}{rgb}{1.000000,0.000000,0.000000}%
\pgfsetstrokecolor{currentstroke}%
\pgfsetdash{}{0pt}%
\pgfpathmoveto{\pgfqpoint{1.160535in}{1.595409in}}%
\pgfpathcurveto{\pgfqpoint{1.168772in}{1.595409in}}{\pgfqpoint{1.176672in}{1.598682in}}{\pgfqpoint{1.182496in}{1.604506in}}%
\pgfpathcurveto{\pgfqpoint{1.188320in}{1.610330in}}{\pgfqpoint{1.191592in}{1.618230in}}{\pgfqpoint{1.191592in}{1.626466in}}%
\pgfpathcurveto{\pgfqpoint{1.191592in}{1.634702in}}{\pgfqpoint{1.188320in}{1.642602in}}{\pgfqpoint{1.182496in}{1.648426in}}%
\pgfpathcurveto{\pgfqpoint{1.176672in}{1.654250in}}{\pgfqpoint{1.168772in}{1.657522in}}{\pgfqpoint{1.160535in}{1.657522in}}%
\pgfpathcurveto{\pgfqpoint{1.152299in}{1.657522in}}{\pgfqpoint{1.144399in}{1.654250in}}{\pgfqpoint{1.138575in}{1.648426in}}%
\pgfpathcurveto{\pgfqpoint{1.132751in}{1.642602in}}{\pgfqpoint{1.129479in}{1.634702in}}{\pgfqpoint{1.129479in}{1.626466in}}%
\pgfpathcurveto{\pgfqpoint{1.129479in}{1.618230in}}{\pgfqpoint{1.132751in}{1.610330in}}{\pgfqpoint{1.138575in}{1.604506in}}%
\pgfpathcurveto{\pgfqpoint{1.144399in}{1.598682in}}{\pgfqpoint{1.152299in}{1.595409in}}{\pgfqpoint{1.160535in}{1.595409in}}%
\pgfpathlineto{\pgfqpoint{1.160535in}{1.595409in}}%
\pgfpathclose%
\pgfusepath{stroke}%
\end{pgfscope}%
\begin{pgfscope}%
\pgfpathrectangle{\pgfqpoint{0.688192in}{0.670138in}}{\pgfqpoint{6.200000in}{4.620000in}}%
\pgfusepath{clip}%
\pgfsetbuttcap%
\pgfsetroundjoin%
\pgfsetlinewidth{1.003750pt}%
\definecolor{currentstroke}{rgb}{1.000000,0.000000,0.000000}%
\pgfsetstrokecolor{currentstroke}%
\pgfsetdash{}{0pt}%
\pgfpathmoveto{\pgfqpoint{1.265571in}{1.691299in}}%
\pgfpathcurveto{\pgfqpoint{1.273807in}{1.691299in}}{\pgfqpoint{1.281707in}{1.694572in}}{\pgfqpoint{1.287531in}{1.700396in}}%
\pgfpathcurveto{\pgfqpoint{1.293355in}{1.706220in}}{\pgfqpoint{1.296627in}{1.714120in}}{\pgfqpoint{1.296627in}{1.722356in}}%
\pgfpathcurveto{\pgfqpoint{1.296627in}{1.730592in}}{\pgfqpoint{1.293355in}{1.738492in}}{\pgfqpoint{1.287531in}{1.744316in}}%
\pgfpathcurveto{\pgfqpoint{1.281707in}{1.750140in}}{\pgfqpoint{1.273807in}{1.753412in}}{\pgfqpoint{1.265571in}{1.753412in}}%
\pgfpathcurveto{\pgfqpoint{1.257335in}{1.753412in}}{\pgfqpoint{1.249435in}{1.750140in}}{\pgfqpoint{1.243611in}{1.744316in}}%
\pgfpathcurveto{\pgfqpoint{1.237787in}{1.738492in}}{\pgfqpoint{1.234514in}{1.730592in}}{\pgfqpoint{1.234514in}{1.722356in}}%
\pgfpathcurveto{\pgfqpoint{1.234514in}{1.714120in}}{\pgfqpoint{1.237787in}{1.706220in}}{\pgfqpoint{1.243611in}{1.700396in}}%
\pgfpathcurveto{\pgfqpoint{1.249435in}{1.694572in}}{\pgfqpoint{1.257335in}{1.691299in}}{\pgfqpoint{1.265571in}{1.691299in}}%
\pgfpathlineto{\pgfqpoint{1.265571in}{1.691299in}}%
\pgfpathclose%
\pgfusepath{stroke}%
\end{pgfscope}%
\begin{pgfscope}%
\pgfpathrectangle{\pgfqpoint{0.688192in}{0.670138in}}{\pgfqpoint{6.200000in}{4.620000in}}%
\pgfusepath{clip}%
\pgfsetbuttcap%
\pgfsetroundjoin%
\pgfsetlinewidth{1.003750pt}%
\definecolor{currentstroke}{rgb}{1.000000,0.000000,0.000000}%
\pgfsetstrokecolor{currentstroke}%
\pgfsetdash{}{0pt}%
\pgfpathmoveto{\pgfqpoint{1.297728in}{1.636949in}}%
\pgfpathcurveto{\pgfqpoint{1.305964in}{1.636949in}}{\pgfqpoint{1.313864in}{1.640222in}}{\pgfqpoint{1.319688in}{1.646045in}}%
\pgfpathcurveto{\pgfqpoint{1.325512in}{1.651869in}}{\pgfqpoint{1.328784in}{1.659769in}}{\pgfqpoint{1.328784in}{1.668006in}}%
\pgfpathcurveto{\pgfqpoint{1.328784in}{1.676242in}}{\pgfqpoint{1.325512in}{1.684142in}}{\pgfqpoint{1.319688in}{1.689966in}}%
\pgfpathcurveto{\pgfqpoint{1.313864in}{1.695790in}}{\pgfqpoint{1.305964in}{1.699062in}}{\pgfqpoint{1.297728in}{1.699062in}}%
\pgfpathcurveto{\pgfqpoint{1.289491in}{1.699062in}}{\pgfqpoint{1.281591in}{1.695790in}}{\pgfqpoint{1.275767in}{1.689966in}}%
\pgfpathcurveto{\pgfqpoint{1.269943in}{1.684142in}}{\pgfqpoint{1.266671in}{1.676242in}}{\pgfqpoint{1.266671in}{1.668006in}}%
\pgfpathcurveto{\pgfqpoint{1.266671in}{1.659769in}}{\pgfqpoint{1.269943in}{1.651869in}}{\pgfqpoint{1.275767in}{1.646045in}}%
\pgfpathcurveto{\pgfqpoint{1.281591in}{1.640222in}}{\pgfqpoint{1.289491in}{1.636949in}}{\pgfqpoint{1.297728in}{1.636949in}}%
\pgfpathlineto{\pgfqpoint{1.297728in}{1.636949in}}%
\pgfpathclose%
\pgfusepath{stroke}%
\end{pgfscope}%
\begin{pgfscope}%
\pgfpathrectangle{\pgfqpoint{0.688192in}{0.670138in}}{\pgfqpoint{6.200000in}{4.620000in}}%
\pgfusepath{clip}%
\pgfsetbuttcap%
\pgfsetroundjoin%
\pgfsetlinewidth{1.003750pt}%
\definecolor{currentstroke}{rgb}{1.000000,0.000000,0.000000}%
\pgfsetstrokecolor{currentstroke}%
\pgfsetdash{}{0pt}%
\pgfpathmoveto{\pgfqpoint{1.147178in}{1.734556in}}%
\pgfpathcurveto{\pgfqpoint{1.155415in}{1.734556in}}{\pgfqpoint{1.163315in}{1.737828in}}{\pgfqpoint{1.169139in}{1.743652in}}%
\pgfpathcurveto{\pgfqpoint{1.174962in}{1.749476in}}{\pgfqpoint{1.178235in}{1.757376in}}{\pgfqpoint{1.178235in}{1.765612in}}%
\pgfpathcurveto{\pgfqpoint{1.178235in}{1.773848in}}{\pgfqpoint{1.174962in}{1.781749in}}{\pgfqpoint{1.169139in}{1.787572in}}%
\pgfpathcurveto{\pgfqpoint{1.163315in}{1.793396in}}{\pgfqpoint{1.155415in}{1.796669in}}{\pgfqpoint{1.147178in}{1.796669in}}%
\pgfpathcurveto{\pgfqpoint{1.138942in}{1.796669in}}{\pgfqpoint{1.131042in}{1.793396in}}{\pgfqpoint{1.125218in}{1.787572in}}%
\pgfpathcurveto{\pgfqpoint{1.119394in}{1.781749in}}{\pgfqpoint{1.116122in}{1.773848in}}{\pgfqpoint{1.116122in}{1.765612in}}%
\pgfpathcurveto{\pgfqpoint{1.116122in}{1.757376in}}{\pgfqpoint{1.119394in}{1.749476in}}{\pgfqpoint{1.125218in}{1.743652in}}%
\pgfpathcurveto{\pgfqpoint{1.131042in}{1.737828in}}{\pgfqpoint{1.138942in}{1.734556in}}{\pgfqpoint{1.147178in}{1.734556in}}%
\pgfpathlineto{\pgfqpoint{1.147178in}{1.734556in}}%
\pgfpathclose%
\pgfusepath{stroke}%
\end{pgfscope}%
\begin{pgfscope}%
\pgfpathrectangle{\pgfqpoint{0.688192in}{0.670138in}}{\pgfqpoint{6.200000in}{4.620000in}}%
\pgfusepath{clip}%
\pgfsetbuttcap%
\pgfsetroundjoin%
\pgfsetlinewidth{1.003750pt}%
\definecolor{currentstroke}{rgb}{1.000000,0.000000,0.000000}%
\pgfsetstrokecolor{currentstroke}%
\pgfsetdash{}{0pt}%
\pgfpathmoveto{\pgfqpoint{1.240097in}{1.549585in}}%
\pgfpathcurveto{\pgfqpoint{1.248334in}{1.549585in}}{\pgfqpoint{1.256234in}{1.552857in}}{\pgfqpoint{1.262058in}{1.558681in}}%
\pgfpathcurveto{\pgfqpoint{1.267881in}{1.564505in}}{\pgfqpoint{1.271154in}{1.572405in}}{\pgfqpoint{1.271154in}{1.580642in}}%
\pgfpathcurveto{\pgfqpoint{1.271154in}{1.588878in}}{\pgfqpoint{1.267881in}{1.596778in}}{\pgfqpoint{1.262058in}{1.602602in}}%
\pgfpathcurveto{\pgfqpoint{1.256234in}{1.608426in}}{\pgfqpoint{1.248334in}{1.611698in}}{\pgfqpoint{1.240097in}{1.611698in}}%
\pgfpathcurveto{\pgfqpoint{1.231861in}{1.611698in}}{\pgfqpoint{1.223961in}{1.608426in}}{\pgfqpoint{1.218137in}{1.602602in}}%
\pgfpathcurveto{\pgfqpoint{1.212313in}{1.596778in}}{\pgfqpoint{1.209041in}{1.588878in}}{\pgfqpoint{1.209041in}{1.580642in}}%
\pgfpathcurveto{\pgfqpoint{1.209041in}{1.572405in}}{\pgfqpoint{1.212313in}{1.564505in}}{\pgfqpoint{1.218137in}{1.558681in}}%
\pgfpathcurveto{\pgfqpoint{1.223961in}{1.552857in}}{\pgfqpoint{1.231861in}{1.549585in}}{\pgfqpoint{1.240097in}{1.549585in}}%
\pgfpathlineto{\pgfqpoint{1.240097in}{1.549585in}}%
\pgfpathclose%
\pgfusepath{stroke}%
\end{pgfscope}%
\begin{pgfscope}%
\pgfpathrectangle{\pgfqpoint{0.688192in}{0.670138in}}{\pgfqpoint{6.200000in}{4.620000in}}%
\pgfusepath{clip}%
\pgfsetbuttcap%
\pgfsetroundjoin%
\pgfsetlinewidth{1.003750pt}%
\definecolor{currentstroke}{rgb}{1.000000,0.000000,0.000000}%
\pgfsetstrokecolor{currentstroke}%
\pgfsetdash{}{0pt}%
\pgfpathmoveto{\pgfqpoint{1.275017in}{1.474367in}}%
\pgfpathcurveto{\pgfqpoint{1.283253in}{1.474367in}}{\pgfqpoint{1.291153in}{1.477639in}}{\pgfqpoint{1.296977in}{1.483463in}}%
\pgfpathcurveto{\pgfqpoint{1.302801in}{1.489287in}}{\pgfqpoint{1.306073in}{1.497187in}}{\pgfqpoint{1.306073in}{1.505423in}}%
\pgfpathcurveto{\pgfqpoint{1.306073in}{1.513660in}}{\pgfqpoint{1.302801in}{1.521560in}}{\pgfqpoint{1.296977in}{1.527384in}}%
\pgfpathcurveto{\pgfqpoint{1.291153in}{1.533208in}}{\pgfqpoint{1.283253in}{1.536480in}}{\pgfqpoint{1.275017in}{1.536480in}}%
\pgfpathcurveto{\pgfqpoint{1.266780in}{1.536480in}}{\pgfqpoint{1.258880in}{1.533208in}}{\pgfqpoint{1.253056in}{1.527384in}}%
\pgfpathcurveto{\pgfqpoint{1.247232in}{1.521560in}}{\pgfqpoint{1.243960in}{1.513660in}}{\pgfqpoint{1.243960in}{1.505423in}}%
\pgfpathcurveto{\pgfqpoint{1.243960in}{1.497187in}}{\pgfqpoint{1.247232in}{1.489287in}}{\pgfqpoint{1.253056in}{1.483463in}}%
\pgfpathcurveto{\pgfqpoint{1.258880in}{1.477639in}}{\pgfqpoint{1.266780in}{1.474367in}}{\pgfqpoint{1.275017in}{1.474367in}}%
\pgfpathlineto{\pgfqpoint{1.275017in}{1.474367in}}%
\pgfpathclose%
\pgfusepath{stroke}%
\end{pgfscope}%
\begin{pgfscope}%
\pgfpathrectangle{\pgfqpoint{0.688192in}{0.670138in}}{\pgfqpoint{6.200000in}{4.620000in}}%
\pgfusepath{clip}%
\pgfsetbuttcap%
\pgfsetroundjoin%
\pgfsetlinewidth{1.003750pt}%
\definecolor{currentstroke}{rgb}{1.000000,0.000000,0.000000}%
\pgfsetstrokecolor{currentstroke}%
\pgfsetdash{}{0pt}%
\pgfpathmoveto{\pgfqpoint{1.274070in}{1.506532in}}%
\pgfpathcurveto{\pgfqpoint{1.282306in}{1.506532in}}{\pgfqpoint{1.290206in}{1.509804in}}{\pgfqpoint{1.296030in}{1.515628in}}%
\pgfpathcurveto{\pgfqpoint{1.301854in}{1.521452in}}{\pgfqpoint{1.305126in}{1.529352in}}{\pgfqpoint{1.305126in}{1.537589in}}%
\pgfpathcurveto{\pgfqpoint{1.305126in}{1.545825in}}{\pgfqpoint{1.301854in}{1.553725in}}{\pgfqpoint{1.296030in}{1.559549in}}%
\pgfpathcurveto{\pgfqpoint{1.290206in}{1.565373in}}{\pgfqpoint{1.282306in}{1.568645in}}{\pgfqpoint{1.274070in}{1.568645in}}%
\pgfpathcurveto{\pgfqpoint{1.265833in}{1.568645in}}{\pgfqpoint{1.257933in}{1.565373in}}{\pgfqpoint{1.252109in}{1.559549in}}%
\pgfpathcurveto{\pgfqpoint{1.246285in}{1.553725in}}{\pgfqpoint{1.243013in}{1.545825in}}{\pgfqpoint{1.243013in}{1.537589in}}%
\pgfpathcurveto{\pgfqpoint{1.243013in}{1.529352in}}{\pgfqpoint{1.246285in}{1.521452in}}{\pgfqpoint{1.252109in}{1.515628in}}%
\pgfpathcurveto{\pgfqpoint{1.257933in}{1.509804in}}{\pgfqpoint{1.265833in}{1.506532in}}{\pgfqpoint{1.274070in}{1.506532in}}%
\pgfpathlineto{\pgfqpoint{1.274070in}{1.506532in}}%
\pgfpathclose%
\pgfusepath{stroke}%
\end{pgfscope}%
\begin{pgfscope}%
\pgfpathrectangle{\pgfqpoint{0.688192in}{0.670138in}}{\pgfqpoint{6.200000in}{4.620000in}}%
\pgfusepath{clip}%
\pgfsetbuttcap%
\pgfsetroundjoin%
\pgfsetlinewidth{1.003750pt}%
\definecolor{currentstroke}{rgb}{1.000000,0.000000,0.000000}%
\pgfsetstrokecolor{currentstroke}%
\pgfsetdash{}{0pt}%
\pgfpathmoveto{\pgfqpoint{1.147538in}{1.722969in}}%
\pgfpathcurveto{\pgfqpoint{1.155775in}{1.722969in}}{\pgfqpoint{1.163675in}{1.726242in}}{\pgfqpoint{1.169499in}{1.732066in}}%
\pgfpathcurveto{\pgfqpoint{1.175323in}{1.737890in}}{\pgfqpoint{1.178595in}{1.745790in}}{\pgfqpoint{1.178595in}{1.754026in}}%
\pgfpathcurveto{\pgfqpoint{1.178595in}{1.762262in}}{\pgfqpoint{1.175323in}{1.770162in}}{\pgfqpoint{1.169499in}{1.775986in}}%
\pgfpathcurveto{\pgfqpoint{1.163675in}{1.781810in}}{\pgfqpoint{1.155775in}{1.785082in}}{\pgfqpoint{1.147538in}{1.785082in}}%
\pgfpathcurveto{\pgfqpoint{1.139302in}{1.785082in}}{\pgfqpoint{1.131402in}{1.781810in}}{\pgfqpoint{1.125578in}{1.775986in}}%
\pgfpathcurveto{\pgfqpoint{1.119754in}{1.770162in}}{\pgfqpoint{1.116482in}{1.762262in}}{\pgfqpoint{1.116482in}{1.754026in}}%
\pgfpathcurveto{\pgfqpoint{1.116482in}{1.745790in}}{\pgfqpoint{1.119754in}{1.737890in}}{\pgfqpoint{1.125578in}{1.732066in}}%
\pgfpathcurveto{\pgfqpoint{1.131402in}{1.726242in}}{\pgfqpoint{1.139302in}{1.722969in}}{\pgfqpoint{1.147538in}{1.722969in}}%
\pgfpathlineto{\pgfqpoint{1.147538in}{1.722969in}}%
\pgfpathclose%
\pgfusepath{stroke}%
\end{pgfscope}%
\begin{pgfscope}%
\pgfpathrectangle{\pgfqpoint{0.688192in}{0.670138in}}{\pgfqpoint{6.200000in}{4.620000in}}%
\pgfusepath{clip}%
\pgfsetbuttcap%
\pgfsetroundjoin%
\pgfsetlinewidth{1.003750pt}%
\definecolor{currentstroke}{rgb}{1.000000,0.000000,0.000000}%
\pgfsetstrokecolor{currentstroke}%
\pgfsetdash{}{0pt}%
\pgfpathmoveto{\pgfqpoint{1.194960in}{1.895058in}}%
\pgfpathcurveto{\pgfqpoint{1.203197in}{1.895058in}}{\pgfqpoint{1.211097in}{1.898330in}}{\pgfqpoint{1.216921in}{1.904154in}}%
\pgfpathcurveto{\pgfqpoint{1.222745in}{1.909978in}}{\pgfqpoint{1.226017in}{1.917878in}}{\pgfqpoint{1.226017in}{1.926114in}}%
\pgfpathcurveto{\pgfqpoint{1.226017in}{1.934351in}}{\pgfqpoint{1.222745in}{1.942251in}}{\pgfqpoint{1.216921in}{1.948075in}}%
\pgfpathcurveto{\pgfqpoint{1.211097in}{1.953898in}}{\pgfqpoint{1.203197in}{1.957171in}}{\pgfqpoint{1.194960in}{1.957171in}}%
\pgfpathcurveto{\pgfqpoint{1.186724in}{1.957171in}}{\pgfqpoint{1.178824in}{1.953898in}}{\pgfqpoint{1.173000in}{1.948075in}}%
\pgfpathcurveto{\pgfqpoint{1.167176in}{1.942251in}}{\pgfqpoint{1.163904in}{1.934351in}}{\pgfqpoint{1.163904in}{1.926114in}}%
\pgfpathcurveto{\pgfqpoint{1.163904in}{1.917878in}}{\pgfqpoint{1.167176in}{1.909978in}}{\pgfqpoint{1.173000in}{1.904154in}}%
\pgfpathcurveto{\pgfqpoint{1.178824in}{1.898330in}}{\pgfqpoint{1.186724in}{1.895058in}}{\pgfqpoint{1.194960in}{1.895058in}}%
\pgfpathlineto{\pgfqpoint{1.194960in}{1.895058in}}%
\pgfpathclose%
\pgfusepath{stroke}%
\end{pgfscope}%
\begin{pgfscope}%
\pgfpathrectangle{\pgfqpoint{0.688192in}{0.670138in}}{\pgfqpoint{6.200000in}{4.620000in}}%
\pgfusepath{clip}%
\pgfsetbuttcap%
\pgfsetroundjoin%
\pgfsetlinewidth{1.003750pt}%
\definecolor{currentstroke}{rgb}{1.000000,0.000000,0.000000}%
\pgfsetstrokecolor{currentstroke}%
\pgfsetdash{}{0pt}%
\pgfpathmoveto{\pgfqpoint{1.164778in}{1.905975in}}%
\pgfpathcurveto{\pgfqpoint{1.173015in}{1.905975in}}{\pgfqpoint{1.180915in}{1.909248in}}{\pgfqpoint{1.186738in}{1.915072in}}%
\pgfpathcurveto{\pgfqpoint{1.192562in}{1.920896in}}{\pgfqpoint{1.195835in}{1.928796in}}{\pgfqpoint{1.195835in}{1.937032in}}%
\pgfpathcurveto{\pgfqpoint{1.195835in}{1.945268in}}{\pgfqpoint{1.192562in}{1.953168in}}{\pgfqpoint{1.186738in}{1.958992in}}%
\pgfpathcurveto{\pgfqpoint{1.180915in}{1.964816in}}{\pgfqpoint{1.173015in}{1.968088in}}{\pgfqpoint{1.164778in}{1.968088in}}%
\pgfpathcurveto{\pgfqpoint{1.156542in}{1.968088in}}{\pgfqpoint{1.148642in}{1.964816in}}{\pgfqpoint{1.142818in}{1.958992in}}%
\pgfpathcurveto{\pgfqpoint{1.136994in}{1.953168in}}{\pgfqpoint{1.133722in}{1.945268in}}{\pgfqpoint{1.133722in}{1.937032in}}%
\pgfpathcurveto{\pgfqpoint{1.133722in}{1.928796in}}{\pgfqpoint{1.136994in}{1.920896in}}{\pgfqpoint{1.142818in}{1.915072in}}%
\pgfpathcurveto{\pgfqpoint{1.148642in}{1.909248in}}{\pgfqpoint{1.156542in}{1.905975in}}{\pgfqpoint{1.164778in}{1.905975in}}%
\pgfpathlineto{\pgfqpoint{1.164778in}{1.905975in}}%
\pgfpathclose%
\pgfusepath{stroke}%
\end{pgfscope}%
\begin{pgfscope}%
\pgfpathrectangle{\pgfqpoint{0.688192in}{0.670138in}}{\pgfqpoint{6.200000in}{4.620000in}}%
\pgfusepath{clip}%
\pgfsetbuttcap%
\pgfsetroundjoin%
\pgfsetlinewidth{1.003750pt}%
\definecolor{currentstroke}{rgb}{1.000000,0.000000,0.000000}%
\pgfsetstrokecolor{currentstroke}%
\pgfsetdash{}{0pt}%
\pgfpathmoveto{\pgfqpoint{1.143899in}{1.731794in}}%
\pgfpathcurveto{\pgfqpoint{1.152135in}{1.731794in}}{\pgfqpoint{1.160035in}{1.735067in}}{\pgfqpoint{1.165859in}{1.740890in}}%
\pgfpathcurveto{\pgfqpoint{1.171683in}{1.746714in}}{\pgfqpoint{1.174956in}{1.754614in}}{\pgfqpoint{1.174956in}{1.762851in}}%
\pgfpathcurveto{\pgfqpoint{1.174956in}{1.771087in}}{\pgfqpoint{1.171683in}{1.778987in}}{\pgfqpoint{1.165859in}{1.784811in}}%
\pgfpathcurveto{\pgfqpoint{1.160035in}{1.790635in}}{\pgfqpoint{1.152135in}{1.793907in}}{\pgfqpoint{1.143899in}{1.793907in}}%
\pgfpathcurveto{\pgfqpoint{1.135663in}{1.793907in}}{\pgfqpoint{1.127763in}{1.790635in}}{\pgfqpoint{1.121939in}{1.784811in}}%
\pgfpathcurveto{\pgfqpoint{1.116115in}{1.778987in}}{\pgfqpoint{1.112843in}{1.771087in}}{\pgfqpoint{1.112843in}{1.762851in}}%
\pgfpathcurveto{\pgfqpoint{1.112843in}{1.754614in}}{\pgfqpoint{1.116115in}{1.746714in}}{\pgfqpoint{1.121939in}{1.740890in}}%
\pgfpathcurveto{\pgfqpoint{1.127763in}{1.735067in}}{\pgfqpoint{1.135663in}{1.731794in}}{\pgfqpoint{1.143899in}{1.731794in}}%
\pgfpathlineto{\pgfqpoint{1.143899in}{1.731794in}}%
\pgfpathclose%
\pgfusepath{stroke}%
\end{pgfscope}%
\begin{pgfscope}%
\pgfpathrectangle{\pgfqpoint{0.688192in}{0.670138in}}{\pgfqpoint{6.200000in}{4.620000in}}%
\pgfusepath{clip}%
\pgfsetbuttcap%
\pgfsetroundjoin%
\pgfsetlinewidth{1.003750pt}%
\definecolor{currentstroke}{rgb}{1.000000,0.000000,0.000000}%
\pgfsetstrokecolor{currentstroke}%
\pgfsetdash{}{0pt}%
\pgfpathmoveto{\pgfqpoint{0.688192in}{1.225706in}}%
\pgfpathcurveto{\pgfqpoint{0.696428in}{1.225706in}}{\pgfqpoint{0.704328in}{1.228978in}}{\pgfqpoint{0.710152in}{1.234802in}}%
\pgfpathcurveto{\pgfqpoint{0.715976in}{1.240626in}}{\pgfqpoint{0.719248in}{1.248526in}}{\pgfqpoint{0.719248in}{1.256762in}}%
\pgfpathcurveto{\pgfqpoint{0.719248in}{1.264998in}}{\pgfqpoint{0.715976in}{1.272898in}}{\pgfqpoint{0.710152in}{1.278722in}}%
\pgfpathcurveto{\pgfqpoint{0.704328in}{1.284546in}}{\pgfqpoint{0.696428in}{1.287819in}}{\pgfqpoint{0.688192in}{1.287819in}}%
\pgfpathcurveto{\pgfqpoint{0.679955in}{1.287819in}}{\pgfqpoint{0.672055in}{1.284546in}}{\pgfqpoint{0.666231in}{1.278722in}}%
\pgfpathcurveto{\pgfqpoint{0.660407in}{1.272898in}}{\pgfqpoint{0.657135in}{1.264998in}}{\pgfqpoint{0.657135in}{1.256762in}}%
\pgfpathcurveto{\pgfqpoint{0.657135in}{1.248526in}}{\pgfqpoint{0.660407in}{1.240626in}}{\pgfqpoint{0.666231in}{1.234802in}}%
\pgfpathcurveto{\pgfqpoint{0.672055in}{1.228978in}}{\pgfqpoint{0.679955in}{1.225706in}}{\pgfqpoint{0.688192in}{1.225706in}}%
\pgfpathlineto{\pgfqpoint{0.688192in}{1.225706in}}%
\pgfpathclose%
\pgfusepath{stroke}%
\end{pgfscope}%
\begin{pgfscope}%
\pgfpathrectangle{\pgfqpoint{0.688192in}{0.670138in}}{\pgfqpoint{6.200000in}{4.620000in}}%
\pgfusepath{clip}%
\pgfsetbuttcap%
\pgfsetmiterjoin%
\definecolor{currentfill}{rgb}{0.839216,0.152941,0.156863}%
\pgfsetfillcolor{currentfill}%
\pgfsetfillopacity{0.200000}%
\pgfsetlinewidth{1.003750pt}%
\definecolor{currentstroke}{rgb}{0.839216,0.152941,0.156863}%
\pgfsetstrokecolor{currentstroke}%
\pgfsetstrokeopacity{0.200000}%
\pgfsetdash{}{0pt}%
\pgfpathmoveto{\pgfqpoint{0.688192in}{0.670138in}}%
\pgfpathlineto{\pgfqpoint{1.790122in}{0.670138in}}%
\pgfpathlineto{\pgfqpoint{1.790122in}{5.290138in}}%
\pgfpathlineto{\pgfqpoint{0.688192in}{5.290138in}}%
\pgfpathlineto{\pgfqpoint{0.688192in}{0.670138in}}%
\pgfpathclose%
\pgfusepath{stroke,fill}%
\end{pgfscope}%
\begin{pgfscope}%
\pgfsetbuttcap%
\pgfsetmiterjoin%
\definecolor{currentfill}{rgb}{0.839216,0.152941,0.156863}%
\pgfsetfillcolor{currentfill}%
\pgfsetfillopacity{0.200000}%
\pgfsetlinewidth{1.003750pt}%
\definecolor{currentstroke}{rgb}{0.839216,0.152941,0.156863}%
\pgfsetstrokecolor{currentstroke}%
\pgfsetstrokeopacity{0.200000}%
\pgfsetdash{}{0pt}%
\pgfpathrectangle{\pgfqpoint{0.688192in}{0.670138in}}{\pgfqpoint{6.200000in}{4.620000in}}%
\pgfusepath{clip}%
\pgfpathmoveto{\pgfqpoint{0.688192in}{0.670138in}}%
\pgfpathlineto{\pgfqpoint{1.790122in}{0.670138in}}%
\pgfpathlineto{\pgfqpoint{1.790122in}{5.290138in}}%
\pgfpathlineto{\pgfqpoint{0.688192in}{5.290138in}}%
\pgfpathlineto{\pgfqpoint{0.688192in}{0.670138in}}%
\pgfpathclose%
\pgfusepath{clip}%
\pgfsys@defobject{currentpattern}{\pgfqpoint{0in}{0in}}{\pgfqpoint{1in}{1in}}{%
\begin{pgfscope}%
\pgfpathrectangle{\pgfqpoint{0in}{0in}}{\pgfqpoint{1in}{1in}}%
\pgfusepath{clip}%
\pgfpathmoveto{\pgfqpoint{-0.500000in}{0.500000in}}%
\pgfpathlineto{\pgfqpoint{0.500000in}{1.500000in}}%
\pgfpathmoveto{\pgfqpoint{-0.333333in}{0.333333in}}%
\pgfpathlineto{\pgfqpoint{0.666667in}{1.333333in}}%
\pgfpathmoveto{\pgfqpoint{-0.166667in}{0.166667in}}%
\pgfpathlineto{\pgfqpoint{0.833333in}{1.166667in}}%
\pgfpathmoveto{\pgfqpoint{0.000000in}{0.000000in}}%
\pgfpathlineto{\pgfqpoint{1.000000in}{1.000000in}}%
\pgfpathmoveto{\pgfqpoint{0.166667in}{-0.166667in}}%
\pgfpathlineto{\pgfqpoint{1.166667in}{0.833333in}}%
\pgfpathmoveto{\pgfqpoint{0.333333in}{-0.333333in}}%
\pgfpathlineto{\pgfqpoint{1.333333in}{0.666667in}}%
\pgfpathmoveto{\pgfqpoint{0.500000in}{-0.500000in}}%
\pgfpathlineto{\pgfqpoint{1.500000in}{0.500000in}}%
\pgfusepath{stroke}%
\end{pgfscope}%
}%
\pgfsys@transformshift{0.688192in}{0.670138in}%
\pgfsys@useobject{currentpattern}{}%
\pgfsys@transformshift{1in}{0in}%
\pgfsys@useobject{currentpattern}{}%
\pgfsys@transformshift{1in}{0in}%
\pgfsys@transformshift{-2in}{0in}%
\pgfsys@transformshift{0in}{1in}%
\pgfsys@useobject{currentpattern}{}%
\pgfsys@transformshift{1in}{0in}%
\pgfsys@useobject{currentpattern}{}%
\pgfsys@transformshift{1in}{0in}%
\pgfsys@transformshift{-2in}{0in}%
\pgfsys@transformshift{0in}{1in}%
\pgfsys@useobject{currentpattern}{}%
\pgfsys@transformshift{1in}{0in}%
\pgfsys@useobject{currentpattern}{}%
\pgfsys@transformshift{1in}{0in}%
\pgfsys@transformshift{-2in}{0in}%
\pgfsys@transformshift{0in}{1in}%
\pgfsys@useobject{currentpattern}{}%
\pgfsys@transformshift{1in}{0in}%
\pgfsys@useobject{currentpattern}{}%
\pgfsys@transformshift{1in}{0in}%
\pgfsys@transformshift{-2in}{0in}%
\pgfsys@transformshift{0in}{1in}%
\pgfsys@useobject{currentpattern}{}%
\pgfsys@transformshift{1in}{0in}%
\pgfsys@useobject{currentpattern}{}%
\pgfsys@transformshift{1in}{0in}%
\pgfsys@transformshift{-2in}{0in}%
\pgfsys@transformshift{0in}{1in}%
\end{pgfscope}%
\begin{pgfscope}%
\pgfpathrectangle{\pgfqpoint{0.688192in}{0.670138in}}{\pgfqpoint{6.200000in}{4.620000in}}%
\pgfusepath{clip}%
\pgfsetrectcap%
\pgfsetroundjoin%
\pgfsetlinewidth{0.803000pt}%
\definecolor{currentstroke}{rgb}{0.690196,0.690196,0.690196}%
\pgfsetstrokecolor{currentstroke}%
\pgfsetdash{}{0pt}%
\pgfpathmoveto{\pgfqpoint{1.122474in}{0.670138in}}%
\pgfpathlineto{\pgfqpoint{1.122474in}{5.290138in}}%
\pgfusepath{stroke}%
\end{pgfscope}%
\begin{pgfscope}%
\pgfsetbuttcap%
\pgfsetroundjoin%
\definecolor{currentfill}{rgb}{0.000000,0.000000,0.000000}%
\pgfsetfillcolor{currentfill}%
\pgfsetlinewidth{0.803000pt}%
\definecolor{currentstroke}{rgb}{0.000000,0.000000,0.000000}%
\pgfsetstrokecolor{currentstroke}%
\pgfsetdash{}{0pt}%
\pgfsys@defobject{currentmarker}{\pgfqpoint{0.000000in}{-0.048611in}}{\pgfqpoint{0.000000in}{0.000000in}}{%
\pgfpathmoveto{\pgfqpoint{0.000000in}{0.000000in}}%
\pgfpathlineto{\pgfqpoint{0.000000in}{-0.048611in}}%
\pgfusepath{stroke,fill}%
}%
\begin{pgfscope}%
\pgfsys@transformshift{1.122474in}{0.670138in}%
\pgfsys@useobject{currentmarker}{}%
\end{pgfscope}%
\end{pgfscope}%
\begin{pgfscope}%
\definecolor{textcolor}{rgb}{0.000000,0.000000,0.000000}%
\pgfsetstrokecolor{textcolor}%
\pgfsetfillcolor{textcolor}%
\pgftext[x=1.122474in,y=0.572916in,,top]{\color{textcolor}{\rmfamily\fontsize{14.000000}{16.800000}\selectfont\catcode`\^=\active\def^{\ifmmode\sp\else\^{}\fi}\catcode`\%=\active\def%{\%}$\mathdefault{5500}$}}%
\end{pgfscope}%
\begin{pgfscope}%
\pgfpathrectangle{\pgfqpoint{0.688192in}{0.670138in}}{\pgfqpoint{6.200000in}{4.620000in}}%
\pgfusepath{clip}%
\pgfsetrectcap%
\pgfsetroundjoin%
\pgfsetlinewidth{0.803000pt}%
\definecolor{currentstroke}{rgb}{0.690196,0.690196,0.690196}%
\pgfsetstrokecolor{currentstroke}%
\pgfsetdash{}{0pt}%
\pgfpathmoveto{\pgfqpoint{2.163709in}{0.670138in}}%
\pgfpathlineto{\pgfqpoint{2.163709in}{5.290138in}}%
\pgfusepath{stroke}%
\end{pgfscope}%
\begin{pgfscope}%
\pgfsetbuttcap%
\pgfsetroundjoin%
\definecolor{currentfill}{rgb}{0.000000,0.000000,0.000000}%
\pgfsetfillcolor{currentfill}%
\pgfsetlinewidth{0.803000pt}%
\definecolor{currentstroke}{rgb}{0.000000,0.000000,0.000000}%
\pgfsetstrokecolor{currentstroke}%
\pgfsetdash{}{0pt}%
\pgfsys@defobject{currentmarker}{\pgfqpoint{0.000000in}{-0.048611in}}{\pgfqpoint{0.000000in}{0.000000in}}{%
\pgfpathmoveto{\pgfqpoint{0.000000in}{0.000000in}}%
\pgfpathlineto{\pgfqpoint{0.000000in}{-0.048611in}}%
\pgfusepath{stroke,fill}%
}%
\begin{pgfscope}%
\pgfsys@transformshift{2.163709in}{0.670138in}%
\pgfsys@useobject{currentmarker}{}%
\end{pgfscope}%
\end{pgfscope}%
\begin{pgfscope}%
\definecolor{textcolor}{rgb}{0.000000,0.000000,0.000000}%
\pgfsetstrokecolor{textcolor}%
\pgfsetfillcolor{textcolor}%
\pgftext[x=2.163709in,y=0.572916in,,top]{\color{textcolor}{\rmfamily\fontsize{14.000000}{16.800000}\selectfont\catcode`\^=\active\def^{\ifmmode\sp\else\^{}\fi}\catcode`\%=\active\def%{\%}$\mathdefault{6000}$}}%
\end{pgfscope}%
\begin{pgfscope}%
\pgfpathrectangle{\pgfqpoint{0.688192in}{0.670138in}}{\pgfqpoint{6.200000in}{4.620000in}}%
\pgfusepath{clip}%
\pgfsetrectcap%
\pgfsetroundjoin%
\pgfsetlinewidth{0.803000pt}%
\definecolor{currentstroke}{rgb}{0.690196,0.690196,0.690196}%
\pgfsetstrokecolor{currentstroke}%
\pgfsetdash{}{0pt}%
\pgfpathmoveto{\pgfqpoint{3.204944in}{0.670138in}}%
\pgfpathlineto{\pgfqpoint{3.204944in}{5.290138in}}%
\pgfusepath{stroke}%
\end{pgfscope}%
\begin{pgfscope}%
\pgfsetbuttcap%
\pgfsetroundjoin%
\definecolor{currentfill}{rgb}{0.000000,0.000000,0.000000}%
\pgfsetfillcolor{currentfill}%
\pgfsetlinewidth{0.803000pt}%
\definecolor{currentstroke}{rgb}{0.000000,0.000000,0.000000}%
\pgfsetstrokecolor{currentstroke}%
\pgfsetdash{}{0pt}%
\pgfsys@defobject{currentmarker}{\pgfqpoint{0.000000in}{-0.048611in}}{\pgfqpoint{0.000000in}{0.000000in}}{%
\pgfpathmoveto{\pgfqpoint{0.000000in}{0.000000in}}%
\pgfpathlineto{\pgfqpoint{0.000000in}{-0.048611in}}%
\pgfusepath{stroke,fill}%
}%
\begin{pgfscope}%
\pgfsys@transformshift{3.204944in}{0.670138in}%
\pgfsys@useobject{currentmarker}{}%
\end{pgfscope}%
\end{pgfscope}%
\begin{pgfscope}%
\definecolor{textcolor}{rgb}{0.000000,0.000000,0.000000}%
\pgfsetstrokecolor{textcolor}%
\pgfsetfillcolor{textcolor}%
\pgftext[x=3.204944in,y=0.572916in,,top]{\color{textcolor}{\rmfamily\fontsize{14.000000}{16.800000}\selectfont\catcode`\^=\active\def^{\ifmmode\sp\else\^{}\fi}\catcode`\%=\active\def%{\%}$\mathdefault{6500}$}}%
\end{pgfscope}%
\begin{pgfscope}%
\pgfpathrectangle{\pgfqpoint{0.688192in}{0.670138in}}{\pgfqpoint{6.200000in}{4.620000in}}%
\pgfusepath{clip}%
\pgfsetrectcap%
\pgfsetroundjoin%
\pgfsetlinewidth{0.803000pt}%
\definecolor{currentstroke}{rgb}{0.690196,0.690196,0.690196}%
\pgfsetstrokecolor{currentstroke}%
\pgfsetdash{}{0pt}%
\pgfpathmoveto{\pgfqpoint{4.246179in}{0.670138in}}%
\pgfpathlineto{\pgfqpoint{4.246179in}{5.290138in}}%
\pgfusepath{stroke}%
\end{pgfscope}%
\begin{pgfscope}%
\pgfsetbuttcap%
\pgfsetroundjoin%
\definecolor{currentfill}{rgb}{0.000000,0.000000,0.000000}%
\pgfsetfillcolor{currentfill}%
\pgfsetlinewidth{0.803000pt}%
\definecolor{currentstroke}{rgb}{0.000000,0.000000,0.000000}%
\pgfsetstrokecolor{currentstroke}%
\pgfsetdash{}{0pt}%
\pgfsys@defobject{currentmarker}{\pgfqpoint{0.000000in}{-0.048611in}}{\pgfqpoint{0.000000in}{0.000000in}}{%
\pgfpathmoveto{\pgfqpoint{0.000000in}{0.000000in}}%
\pgfpathlineto{\pgfqpoint{0.000000in}{-0.048611in}}%
\pgfusepath{stroke,fill}%
}%
\begin{pgfscope}%
\pgfsys@transformshift{4.246179in}{0.670138in}%
\pgfsys@useobject{currentmarker}{}%
\end{pgfscope}%
\end{pgfscope}%
\begin{pgfscope}%
\definecolor{textcolor}{rgb}{0.000000,0.000000,0.000000}%
\pgfsetstrokecolor{textcolor}%
\pgfsetfillcolor{textcolor}%
\pgftext[x=4.246179in,y=0.572916in,,top]{\color{textcolor}{\rmfamily\fontsize{14.000000}{16.800000}\selectfont\catcode`\^=\active\def^{\ifmmode\sp\else\^{}\fi}\catcode`\%=\active\def%{\%}$\mathdefault{7000}$}}%
\end{pgfscope}%
\begin{pgfscope}%
\pgfpathrectangle{\pgfqpoint{0.688192in}{0.670138in}}{\pgfqpoint{6.200000in}{4.620000in}}%
\pgfusepath{clip}%
\pgfsetrectcap%
\pgfsetroundjoin%
\pgfsetlinewidth{0.803000pt}%
\definecolor{currentstroke}{rgb}{0.690196,0.690196,0.690196}%
\pgfsetstrokecolor{currentstroke}%
\pgfsetdash{}{0pt}%
\pgfpathmoveto{\pgfqpoint{5.287414in}{0.670138in}}%
\pgfpathlineto{\pgfqpoint{5.287414in}{5.290138in}}%
\pgfusepath{stroke}%
\end{pgfscope}%
\begin{pgfscope}%
\pgfsetbuttcap%
\pgfsetroundjoin%
\definecolor{currentfill}{rgb}{0.000000,0.000000,0.000000}%
\pgfsetfillcolor{currentfill}%
\pgfsetlinewidth{0.803000pt}%
\definecolor{currentstroke}{rgb}{0.000000,0.000000,0.000000}%
\pgfsetstrokecolor{currentstroke}%
\pgfsetdash{}{0pt}%
\pgfsys@defobject{currentmarker}{\pgfqpoint{0.000000in}{-0.048611in}}{\pgfqpoint{0.000000in}{0.000000in}}{%
\pgfpathmoveto{\pgfqpoint{0.000000in}{0.000000in}}%
\pgfpathlineto{\pgfqpoint{0.000000in}{-0.048611in}}%
\pgfusepath{stroke,fill}%
}%
\begin{pgfscope}%
\pgfsys@transformshift{5.287414in}{0.670138in}%
\pgfsys@useobject{currentmarker}{}%
\end{pgfscope}%
\end{pgfscope}%
\begin{pgfscope}%
\definecolor{textcolor}{rgb}{0.000000,0.000000,0.000000}%
\pgfsetstrokecolor{textcolor}%
\pgfsetfillcolor{textcolor}%
\pgftext[x=5.287414in,y=0.572916in,,top]{\color{textcolor}{\rmfamily\fontsize{14.000000}{16.800000}\selectfont\catcode`\^=\active\def^{\ifmmode\sp\else\^{}\fi}\catcode`\%=\active\def%{\%}$\mathdefault{7500}$}}%
\end{pgfscope}%
\begin{pgfscope}%
\pgfpathrectangle{\pgfqpoint{0.688192in}{0.670138in}}{\pgfqpoint{6.200000in}{4.620000in}}%
\pgfusepath{clip}%
\pgfsetrectcap%
\pgfsetroundjoin%
\pgfsetlinewidth{0.803000pt}%
\definecolor{currentstroke}{rgb}{0.690196,0.690196,0.690196}%
\pgfsetstrokecolor{currentstroke}%
\pgfsetdash{}{0pt}%
\pgfpathmoveto{\pgfqpoint{6.328649in}{0.670138in}}%
\pgfpathlineto{\pgfqpoint{6.328649in}{5.290138in}}%
\pgfusepath{stroke}%
\end{pgfscope}%
\begin{pgfscope}%
\pgfsetbuttcap%
\pgfsetroundjoin%
\definecolor{currentfill}{rgb}{0.000000,0.000000,0.000000}%
\pgfsetfillcolor{currentfill}%
\pgfsetlinewidth{0.803000pt}%
\definecolor{currentstroke}{rgb}{0.000000,0.000000,0.000000}%
\pgfsetstrokecolor{currentstroke}%
\pgfsetdash{}{0pt}%
\pgfsys@defobject{currentmarker}{\pgfqpoint{0.000000in}{-0.048611in}}{\pgfqpoint{0.000000in}{0.000000in}}{%
\pgfpathmoveto{\pgfqpoint{0.000000in}{0.000000in}}%
\pgfpathlineto{\pgfqpoint{0.000000in}{-0.048611in}}%
\pgfusepath{stroke,fill}%
}%
\begin{pgfscope}%
\pgfsys@transformshift{6.328649in}{0.670138in}%
\pgfsys@useobject{currentmarker}{}%
\end{pgfscope}%
\end{pgfscope}%
\begin{pgfscope}%
\definecolor{textcolor}{rgb}{0.000000,0.000000,0.000000}%
\pgfsetstrokecolor{textcolor}%
\pgfsetfillcolor{textcolor}%
\pgftext[x=6.328649in,y=0.572916in,,top]{\color{textcolor}{\rmfamily\fontsize{14.000000}{16.800000}\selectfont\catcode`\^=\active\def^{\ifmmode\sp\else\^{}\fi}\catcode`\%=\active\def%{\%}$\mathdefault{8000}$}}%
\end{pgfscope}%
\begin{pgfscope}%
\definecolor{textcolor}{rgb}{0.000000,0.000000,0.000000}%
\pgfsetstrokecolor{textcolor}%
\pgfsetfillcolor{textcolor}%
\pgftext[x=3.788192in,y=0.339583in,,top]{\color{textcolor}{\rmfamily\fontsize{18.000000}{21.600000}\selectfont\catcode`\^=\active\def^{\ifmmode\sp\else\^{}\fi}\catcode`\%=\active\def%{\%}Total Cost (M\$)}}%
\end{pgfscope}%
\begin{pgfscope}%
\pgfpathrectangle{\pgfqpoint{0.688192in}{0.670138in}}{\pgfqpoint{6.200000in}{4.620000in}}%
\pgfusepath{clip}%
\pgfsetrectcap%
\pgfsetroundjoin%
\pgfsetlinewidth{0.803000pt}%
\definecolor{currentstroke}{rgb}{0.690196,0.690196,0.690196}%
\pgfsetstrokecolor{currentstroke}%
\pgfsetdash{}{0pt}%
\pgfpathmoveto{\pgfqpoint{0.688192in}{1.130833in}}%
\pgfpathlineto{\pgfqpoint{6.888192in}{1.130833in}}%
\pgfusepath{stroke}%
\end{pgfscope}%
\begin{pgfscope}%
\pgfsetbuttcap%
\pgfsetroundjoin%
\definecolor{currentfill}{rgb}{0.000000,0.000000,0.000000}%
\pgfsetfillcolor{currentfill}%
\pgfsetlinewidth{0.803000pt}%
\definecolor{currentstroke}{rgb}{0.000000,0.000000,0.000000}%
\pgfsetstrokecolor{currentstroke}%
\pgfsetdash{}{0pt}%
\pgfsys@defobject{currentmarker}{\pgfqpoint{-0.048611in}{0.000000in}}{\pgfqpoint{-0.000000in}{0.000000in}}{%
\pgfpathmoveto{\pgfqpoint{-0.000000in}{0.000000in}}%
\pgfpathlineto{\pgfqpoint{-0.048611in}{0.000000in}}%
\pgfusepath{stroke,fill}%
}%
\begin{pgfscope}%
\pgfsys@transformshift{0.688192in}{1.130833in}%
\pgfsys@useobject{currentmarker}{}%
\end{pgfscope}%
\end{pgfscope}%
\begin{pgfscope}%
\definecolor{textcolor}{rgb}{0.000000,0.000000,0.000000}%
\pgfsetstrokecolor{textcolor}%
\pgfsetfillcolor{textcolor}%
\pgftext[x=0.395138in, y=1.061389in, left, base]{\color{textcolor}{\rmfamily\fontsize{14.000000}{16.800000}\selectfont\catcode`\^=\active\def^{\ifmmode\sp\else\^{}\fi}\catcode`\%=\active\def%{\%}$\mathdefault{10}$}}%
\end{pgfscope}%
\begin{pgfscope}%
\pgfpathrectangle{\pgfqpoint{0.688192in}{0.670138in}}{\pgfqpoint{6.200000in}{4.620000in}}%
\pgfusepath{clip}%
\pgfsetrectcap%
\pgfsetroundjoin%
\pgfsetlinewidth{0.803000pt}%
\definecolor{currentstroke}{rgb}{0.690196,0.690196,0.690196}%
\pgfsetstrokecolor{currentstroke}%
\pgfsetdash{}{0pt}%
\pgfpathmoveto{\pgfqpoint{0.688192in}{1.724860in}}%
\pgfpathlineto{\pgfqpoint{6.888192in}{1.724860in}}%
\pgfusepath{stroke}%
\end{pgfscope}%
\begin{pgfscope}%
\pgfsetbuttcap%
\pgfsetroundjoin%
\definecolor{currentfill}{rgb}{0.000000,0.000000,0.000000}%
\pgfsetfillcolor{currentfill}%
\pgfsetlinewidth{0.803000pt}%
\definecolor{currentstroke}{rgb}{0.000000,0.000000,0.000000}%
\pgfsetstrokecolor{currentstroke}%
\pgfsetdash{}{0pt}%
\pgfsys@defobject{currentmarker}{\pgfqpoint{-0.048611in}{0.000000in}}{\pgfqpoint{-0.000000in}{0.000000in}}{%
\pgfpathmoveto{\pgfqpoint{-0.000000in}{0.000000in}}%
\pgfpathlineto{\pgfqpoint{-0.048611in}{0.000000in}}%
\pgfusepath{stroke,fill}%
}%
\begin{pgfscope}%
\pgfsys@transformshift{0.688192in}{1.724860in}%
\pgfsys@useobject{currentmarker}{}%
\end{pgfscope}%
\end{pgfscope}%
\begin{pgfscope}%
\definecolor{textcolor}{rgb}{0.000000,0.000000,0.000000}%
\pgfsetstrokecolor{textcolor}%
\pgfsetfillcolor{textcolor}%
\pgftext[x=0.395138in, y=1.655416in, left, base]{\color{textcolor}{\rmfamily\fontsize{14.000000}{16.800000}\selectfont\catcode`\^=\active\def^{\ifmmode\sp\else\^{}\fi}\catcode`\%=\active\def%{\%}$\mathdefault{20}$}}%
\end{pgfscope}%
\begin{pgfscope}%
\pgfpathrectangle{\pgfqpoint{0.688192in}{0.670138in}}{\pgfqpoint{6.200000in}{4.620000in}}%
\pgfusepath{clip}%
\pgfsetrectcap%
\pgfsetroundjoin%
\pgfsetlinewidth{0.803000pt}%
\definecolor{currentstroke}{rgb}{0.690196,0.690196,0.690196}%
\pgfsetstrokecolor{currentstroke}%
\pgfsetdash{}{0pt}%
\pgfpathmoveto{\pgfqpoint{0.688192in}{2.318888in}}%
\pgfpathlineto{\pgfqpoint{6.888192in}{2.318888in}}%
\pgfusepath{stroke}%
\end{pgfscope}%
\begin{pgfscope}%
\pgfsetbuttcap%
\pgfsetroundjoin%
\definecolor{currentfill}{rgb}{0.000000,0.000000,0.000000}%
\pgfsetfillcolor{currentfill}%
\pgfsetlinewidth{0.803000pt}%
\definecolor{currentstroke}{rgb}{0.000000,0.000000,0.000000}%
\pgfsetstrokecolor{currentstroke}%
\pgfsetdash{}{0pt}%
\pgfsys@defobject{currentmarker}{\pgfqpoint{-0.048611in}{0.000000in}}{\pgfqpoint{-0.000000in}{0.000000in}}{%
\pgfpathmoveto{\pgfqpoint{-0.000000in}{0.000000in}}%
\pgfpathlineto{\pgfqpoint{-0.048611in}{0.000000in}}%
\pgfusepath{stroke,fill}%
}%
\begin{pgfscope}%
\pgfsys@transformshift{0.688192in}{2.318888in}%
\pgfsys@useobject{currentmarker}{}%
\end{pgfscope}%
\end{pgfscope}%
\begin{pgfscope}%
\definecolor{textcolor}{rgb}{0.000000,0.000000,0.000000}%
\pgfsetstrokecolor{textcolor}%
\pgfsetfillcolor{textcolor}%
\pgftext[x=0.395138in, y=2.249444in, left, base]{\color{textcolor}{\rmfamily\fontsize{14.000000}{16.800000}\selectfont\catcode`\^=\active\def^{\ifmmode\sp\else\^{}\fi}\catcode`\%=\active\def%{\%}$\mathdefault{30}$}}%
\end{pgfscope}%
\begin{pgfscope}%
\pgfpathrectangle{\pgfqpoint{0.688192in}{0.670138in}}{\pgfqpoint{6.200000in}{4.620000in}}%
\pgfusepath{clip}%
\pgfsetrectcap%
\pgfsetroundjoin%
\pgfsetlinewidth{0.803000pt}%
\definecolor{currentstroke}{rgb}{0.690196,0.690196,0.690196}%
\pgfsetstrokecolor{currentstroke}%
\pgfsetdash{}{0pt}%
\pgfpathmoveto{\pgfqpoint{0.688192in}{2.912915in}}%
\pgfpathlineto{\pgfqpoint{6.888192in}{2.912915in}}%
\pgfusepath{stroke}%
\end{pgfscope}%
\begin{pgfscope}%
\pgfsetbuttcap%
\pgfsetroundjoin%
\definecolor{currentfill}{rgb}{0.000000,0.000000,0.000000}%
\pgfsetfillcolor{currentfill}%
\pgfsetlinewidth{0.803000pt}%
\definecolor{currentstroke}{rgb}{0.000000,0.000000,0.000000}%
\pgfsetstrokecolor{currentstroke}%
\pgfsetdash{}{0pt}%
\pgfsys@defobject{currentmarker}{\pgfqpoint{-0.048611in}{0.000000in}}{\pgfqpoint{-0.000000in}{0.000000in}}{%
\pgfpathmoveto{\pgfqpoint{-0.000000in}{0.000000in}}%
\pgfpathlineto{\pgfqpoint{-0.048611in}{0.000000in}}%
\pgfusepath{stroke,fill}%
}%
\begin{pgfscope}%
\pgfsys@transformshift{0.688192in}{2.912915in}%
\pgfsys@useobject{currentmarker}{}%
\end{pgfscope}%
\end{pgfscope}%
\begin{pgfscope}%
\definecolor{textcolor}{rgb}{0.000000,0.000000,0.000000}%
\pgfsetstrokecolor{textcolor}%
\pgfsetfillcolor{textcolor}%
\pgftext[x=0.395138in, y=2.843471in, left, base]{\color{textcolor}{\rmfamily\fontsize{14.000000}{16.800000}\selectfont\catcode`\^=\active\def^{\ifmmode\sp\else\^{}\fi}\catcode`\%=\active\def%{\%}$\mathdefault{40}$}}%
\end{pgfscope}%
\begin{pgfscope}%
\pgfpathrectangle{\pgfqpoint{0.688192in}{0.670138in}}{\pgfqpoint{6.200000in}{4.620000in}}%
\pgfusepath{clip}%
\pgfsetrectcap%
\pgfsetroundjoin%
\pgfsetlinewidth{0.803000pt}%
\definecolor{currentstroke}{rgb}{0.690196,0.690196,0.690196}%
\pgfsetstrokecolor{currentstroke}%
\pgfsetdash{}{0pt}%
\pgfpathmoveto{\pgfqpoint{0.688192in}{3.506943in}}%
\pgfpathlineto{\pgfqpoint{6.888192in}{3.506943in}}%
\pgfusepath{stroke}%
\end{pgfscope}%
\begin{pgfscope}%
\pgfsetbuttcap%
\pgfsetroundjoin%
\definecolor{currentfill}{rgb}{0.000000,0.000000,0.000000}%
\pgfsetfillcolor{currentfill}%
\pgfsetlinewidth{0.803000pt}%
\definecolor{currentstroke}{rgb}{0.000000,0.000000,0.000000}%
\pgfsetstrokecolor{currentstroke}%
\pgfsetdash{}{0pt}%
\pgfsys@defobject{currentmarker}{\pgfqpoint{-0.048611in}{0.000000in}}{\pgfqpoint{-0.000000in}{0.000000in}}{%
\pgfpathmoveto{\pgfqpoint{-0.000000in}{0.000000in}}%
\pgfpathlineto{\pgfqpoint{-0.048611in}{0.000000in}}%
\pgfusepath{stroke,fill}%
}%
\begin{pgfscope}%
\pgfsys@transformshift{0.688192in}{3.506943in}%
\pgfsys@useobject{currentmarker}{}%
\end{pgfscope}%
\end{pgfscope}%
\begin{pgfscope}%
\definecolor{textcolor}{rgb}{0.000000,0.000000,0.000000}%
\pgfsetstrokecolor{textcolor}%
\pgfsetfillcolor{textcolor}%
\pgftext[x=0.395138in, y=3.437499in, left, base]{\color{textcolor}{\rmfamily\fontsize{14.000000}{16.800000}\selectfont\catcode`\^=\active\def^{\ifmmode\sp\else\^{}\fi}\catcode`\%=\active\def%{\%}$\mathdefault{50}$}}%
\end{pgfscope}%
\begin{pgfscope}%
\pgfpathrectangle{\pgfqpoint{0.688192in}{0.670138in}}{\pgfqpoint{6.200000in}{4.620000in}}%
\pgfusepath{clip}%
\pgfsetrectcap%
\pgfsetroundjoin%
\pgfsetlinewidth{0.803000pt}%
\definecolor{currentstroke}{rgb}{0.690196,0.690196,0.690196}%
\pgfsetstrokecolor{currentstroke}%
\pgfsetdash{}{0pt}%
\pgfpathmoveto{\pgfqpoint{0.688192in}{4.100970in}}%
\pgfpathlineto{\pgfqpoint{6.888192in}{4.100970in}}%
\pgfusepath{stroke}%
\end{pgfscope}%
\begin{pgfscope}%
\pgfsetbuttcap%
\pgfsetroundjoin%
\definecolor{currentfill}{rgb}{0.000000,0.000000,0.000000}%
\pgfsetfillcolor{currentfill}%
\pgfsetlinewidth{0.803000pt}%
\definecolor{currentstroke}{rgb}{0.000000,0.000000,0.000000}%
\pgfsetstrokecolor{currentstroke}%
\pgfsetdash{}{0pt}%
\pgfsys@defobject{currentmarker}{\pgfqpoint{-0.048611in}{0.000000in}}{\pgfqpoint{-0.000000in}{0.000000in}}{%
\pgfpathmoveto{\pgfqpoint{-0.000000in}{0.000000in}}%
\pgfpathlineto{\pgfqpoint{-0.048611in}{0.000000in}}%
\pgfusepath{stroke,fill}%
}%
\begin{pgfscope}%
\pgfsys@transformshift{0.688192in}{4.100970in}%
\pgfsys@useobject{currentmarker}{}%
\end{pgfscope}%
\end{pgfscope}%
\begin{pgfscope}%
\definecolor{textcolor}{rgb}{0.000000,0.000000,0.000000}%
\pgfsetstrokecolor{textcolor}%
\pgfsetfillcolor{textcolor}%
\pgftext[x=0.395138in, y=4.031526in, left, base]{\color{textcolor}{\rmfamily\fontsize{14.000000}{16.800000}\selectfont\catcode`\^=\active\def^{\ifmmode\sp\else\^{}\fi}\catcode`\%=\active\def%{\%}$\mathdefault{60}$}}%
\end{pgfscope}%
\begin{pgfscope}%
\pgfpathrectangle{\pgfqpoint{0.688192in}{0.670138in}}{\pgfqpoint{6.200000in}{4.620000in}}%
\pgfusepath{clip}%
\pgfsetrectcap%
\pgfsetroundjoin%
\pgfsetlinewidth{0.803000pt}%
\definecolor{currentstroke}{rgb}{0.690196,0.690196,0.690196}%
\pgfsetstrokecolor{currentstroke}%
\pgfsetdash{}{0pt}%
\pgfpathmoveto{\pgfqpoint{0.688192in}{4.694998in}}%
\pgfpathlineto{\pgfqpoint{6.888192in}{4.694998in}}%
\pgfusepath{stroke}%
\end{pgfscope}%
\begin{pgfscope}%
\pgfsetbuttcap%
\pgfsetroundjoin%
\definecolor{currentfill}{rgb}{0.000000,0.000000,0.000000}%
\pgfsetfillcolor{currentfill}%
\pgfsetlinewidth{0.803000pt}%
\definecolor{currentstroke}{rgb}{0.000000,0.000000,0.000000}%
\pgfsetstrokecolor{currentstroke}%
\pgfsetdash{}{0pt}%
\pgfsys@defobject{currentmarker}{\pgfqpoint{-0.048611in}{0.000000in}}{\pgfqpoint{-0.000000in}{0.000000in}}{%
\pgfpathmoveto{\pgfqpoint{-0.000000in}{0.000000in}}%
\pgfpathlineto{\pgfqpoint{-0.048611in}{0.000000in}}%
\pgfusepath{stroke,fill}%
}%
\begin{pgfscope}%
\pgfsys@transformshift{0.688192in}{4.694998in}%
\pgfsys@useobject{currentmarker}{}%
\end{pgfscope}%
\end{pgfscope}%
\begin{pgfscope}%
\definecolor{textcolor}{rgb}{0.000000,0.000000,0.000000}%
\pgfsetstrokecolor{textcolor}%
\pgfsetfillcolor{textcolor}%
\pgftext[x=0.395138in, y=4.625553in, left, base]{\color{textcolor}{\rmfamily\fontsize{14.000000}{16.800000}\selectfont\catcode`\^=\active\def^{\ifmmode\sp\else\^{}\fi}\catcode`\%=\active\def%{\%}$\mathdefault{70}$}}%
\end{pgfscope}%
\begin{pgfscope}%
\pgfpathrectangle{\pgfqpoint{0.688192in}{0.670138in}}{\pgfqpoint{6.200000in}{4.620000in}}%
\pgfusepath{clip}%
\pgfsetrectcap%
\pgfsetroundjoin%
\pgfsetlinewidth{0.803000pt}%
\definecolor{currentstroke}{rgb}{0.690196,0.690196,0.690196}%
\pgfsetstrokecolor{currentstroke}%
\pgfsetdash{}{0pt}%
\pgfpathmoveto{\pgfqpoint{0.688192in}{5.289025in}}%
\pgfpathlineto{\pgfqpoint{6.888192in}{5.289025in}}%
\pgfusepath{stroke}%
\end{pgfscope}%
\begin{pgfscope}%
\pgfsetbuttcap%
\pgfsetroundjoin%
\definecolor{currentfill}{rgb}{0.000000,0.000000,0.000000}%
\pgfsetfillcolor{currentfill}%
\pgfsetlinewidth{0.803000pt}%
\definecolor{currentstroke}{rgb}{0.000000,0.000000,0.000000}%
\pgfsetstrokecolor{currentstroke}%
\pgfsetdash{}{0pt}%
\pgfsys@defobject{currentmarker}{\pgfqpoint{-0.048611in}{0.000000in}}{\pgfqpoint{-0.000000in}{0.000000in}}{%
\pgfpathmoveto{\pgfqpoint{-0.000000in}{0.000000in}}%
\pgfpathlineto{\pgfqpoint{-0.048611in}{0.000000in}}%
\pgfusepath{stroke,fill}%
}%
\begin{pgfscope}%
\pgfsys@transformshift{0.688192in}{5.289025in}%
\pgfsys@useobject{currentmarker}{}%
\end{pgfscope}%
\end{pgfscope}%
\begin{pgfscope}%
\definecolor{textcolor}{rgb}{0.000000,0.000000,0.000000}%
\pgfsetstrokecolor{textcolor}%
\pgfsetfillcolor{textcolor}%
\pgftext[x=0.395138in, y=5.219581in, left, base]{\color{textcolor}{\rmfamily\fontsize{14.000000}{16.800000}\selectfont\catcode`\^=\active\def^{\ifmmode\sp\else\^{}\fi}\catcode`\%=\active\def%{\%}$\mathdefault{80}$}}%
\end{pgfscope}%
\begin{pgfscope}%
\definecolor{textcolor}{rgb}{0.000000,0.000000,0.000000}%
\pgfsetstrokecolor{textcolor}%
\pgfsetfillcolor{textcolor}%
\pgftext[x=0.339583in,y=2.980138in,,bottom,rotate=90.000000]{\color{textcolor}{\rmfamily\fontsize{18.000000}{21.600000}\selectfont\catcode`\^=\active\def^{\ifmmode\sp\else\^{}\fi}\catcode`\%=\active\def%{\%}CO2 emissions (MT CO2)}}%
\end{pgfscope}%
\begin{pgfscope}%
\pgfpathrectangle{\pgfqpoint{0.688192in}{0.670138in}}{\pgfqpoint{6.200000in}{4.620000in}}%
\pgfusepath{clip}%
\pgfsetrectcap%
\pgfsetroundjoin%
\pgfsetlinewidth{1.505625pt}%
\definecolor{currentstroke}{rgb}{0.000000,0.000000,1.000000}%
\pgfsetstrokecolor{currentstroke}%
\pgfsetdash{}{0pt}%
\pgfpathmoveto{\pgfqpoint{0.741425in}{1.377543in}}%
\pgfpathlineto{\pgfqpoint{0.758703in}{0.955032in}}%
\pgfpathlineto{\pgfqpoint{0.768198in}{0.875033in}}%
\pgfpathlineto{\pgfqpoint{0.774746in}{0.828781in}}%
\pgfpathlineto{\pgfqpoint{0.778243in}{0.822495in}}%
\pgfpathlineto{\pgfqpoint{0.782159in}{0.789611in}}%
\pgfpathlineto{\pgfqpoint{0.786516in}{0.779881in}}%
\pgfpathlineto{\pgfqpoint{0.792538in}{0.779145in}}%
\pgfpathlineto{\pgfqpoint{0.794668in}{0.758056in}}%
\pgfpathlineto{\pgfqpoint{0.799837in}{0.752930in}}%
\pgfpathlineto{\pgfqpoint{0.809370in}{0.751978in}}%
\pgfpathlineto{\pgfqpoint{0.812629in}{0.743975in}}%
\pgfpathlineto{\pgfqpoint{0.815972in}{0.742575in}}%
\pgfpathlineto{\pgfqpoint{0.822987in}{0.738477in}}%
\pgfpathlineto{\pgfqpoint{0.828825in}{0.734937in}}%
\pgfpathlineto{\pgfqpoint{0.829214in}{0.733319in}}%
\pgfpathlineto{\pgfqpoint{0.833044in}{0.730858in}}%
\pgfpathlineto{\pgfqpoint{0.848459in}{0.726329in}}%
\pgfpathlineto{\pgfqpoint{0.864854in}{0.720019in}}%
\pgfpathlineto{\pgfqpoint{0.887104in}{0.715517in}}%
\pgfpathlineto{\pgfqpoint{0.907479in}{0.714004in}}%
\pgfpathlineto{\pgfqpoint{0.908310in}{0.712008in}}%
\pgfpathlineto{\pgfqpoint{0.909513in}{0.708525in}}%
\pgfpathlineto{\pgfqpoint{0.912740in}{0.707284in}}%
\pgfpathlineto{\pgfqpoint{0.920440in}{0.706723in}}%
\pgfpathlineto{\pgfqpoint{0.925670in}{0.705238in}}%
\pgfpathlineto{\pgfqpoint{0.948903in}{0.702931in}}%
\pgfpathlineto{\pgfqpoint{0.951945in}{0.701707in}}%
\pgfpathlineto{\pgfqpoint{0.952035in}{0.700391in}}%
\pgfpathlineto{\pgfqpoint{0.957029in}{0.700173in}}%
\pgfpathlineto{\pgfqpoint{0.968828in}{0.697963in}}%
\pgfpathlineto{\pgfqpoint{0.974412in}{0.697738in}}%
\pgfpathlineto{\pgfqpoint{0.975275in}{0.696914in}}%
\pgfpathlineto{\pgfqpoint{1.021767in}{0.694795in}}%
\pgfpathlineto{\pgfqpoint{1.025407in}{0.690657in}}%
\pgfpathlineto{\pgfqpoint{1.027475in}{0.690338in}}%
\pgfpathlineto{\pgfqpoint{1.034837in}{0.689784in}}%
\pgfpathlineto{\pgfqpoint{1.049406in}{0.687676in}}%
\pgfpathlineto{\pgfqpoint{1.054714in}{0.687138in}}%
\pgfpathlineto{\pgfqpoint{1.059617in}{0.686467in}}%
\pgfpathlineto{\pgfqpoint{1.072141in}{0.685078in}}%
\pgfpathlineto{\pgfqpoint{1.092208in}{0.684413in}}%
\pgfpathlineto{\pgfqpoint{1.115209in}{0.684111in}}%
\pgfpathlineto{\pgfqpoint{1.131834in}{0.684071in}}%
\pgfpathlineto{\pgfqpoint{1.152628in}{0.684059in}}%
\pgfpathlineto{\pgfqpoint{1.251312in}{0.683263in}}%
\pgfpathlineto{\pgfqpoint{1.277476in}{0.683159in}}%
\pgfpathlineto{\pgfqpoint{1.314870in}{0.682855in}}%
\pgfpathlineto{\pgfqpoint{1.369253in}{0.682756in}}%
\pgfpathlineto{\pgfqpoint{1.398687in}{0.682288in}}%
\pgfpathlineto{\pgfqpoint{1.467852in}{0.682134in}}%
\pgfpathlineto{\pgfqpoint{1.557026in}{0.681680in}}%
\pgfpathlineto{\pgfqpoint{1.627242in}{0.680913in}}%
\pgfpathlineto{\pgfqpoint{1.737728in}{0.680478in}}%
\pgfpathlineto{\pgfqpoint{1.887036in}{0.679610in}}%
\pgfpathlineto{\pgfqpoint{2.037481in}{0.678826in}}%
\pgfpathlineto{\pgfqpoint{2.258348in}{0.677741in}}%
\pgfpathlineto{\pgfqpoint{2.626338in}{0.676361in}}%
\pgfpathlineto{\pgfqpoint{3.263784in}{0.674352in}}%
\pgfpathlineto{\pgfqpoint{5.322800in}{0.670138in}}%
\pgfusepath{stroke}%
\end{pgfscope}%
\begin{pgfscope}%
\pgfpathrectangle{\pgfqpoint{0.688192in}{0.670138in}}{\pgfqpoint{6.200000in}{4.620000in}}%
\pgfusepath{clip}%
\pgfsetbuttcap%
\pgfsetroundjoin%
\definecolor{currentfill}{rgb}{0.000000,0.000000,1.000000}%
\pgfsetfillcolor{currentfill}%
\pgfsetlinewidth{1.003750pt}%
\definecolor{currentstroke}{rgb}{0.000000,0.000000,1.000000}%
\pgfsetstrokecolor{currentstroke}%
\pgfsetdash{}{0pt}%
\pgfsys@defobject{currentmarker}{\pgfqpoint{-0.006944in}{-0.006944in}}{\pgfqpoint{0.006944in}{0.006944in}}{%
\pgfpathmoveto{\pgfqpoint{0.000000in}{-0.006944in}}%
\pgfpathcurveto{\pgfqpoint{0.001842in}{-0.006944in}}{\pgfqpoint{0.003608in}{-0.006213in}}{\pgfqpoint{0.004910in}{-0.004910in}}%
\pgfpathcurveto{\pgfqpoint{0.006213in}{-0.003608in}}{\pgfqpoint{0.006944in}{-0.001842in}}{\pgfqpoint{0.006944in}{0.000000in}}%
\pgfpathcurveto{\pgfqpoint{0.006944in}{0.001842in}}{\pgfqpoint{0.006213in}{0.003608in}}{\pgfqpoint{0.004910in}{0.004910in}}%
\pgfpathcurveto{\pgfqpoint{0.003608in}{0.006213in}}{\pgfqpoint{0.001842in}{0.006944in}}{\pgfqpoint{0.000000in}{0.006944in}}%
\pgfpathcurveto{\pgfqpoint{-0.001842in}{0.006944in}}{\pgfqpoint{-0.003608in}{0.006213in}}{\pgfqpoint{-0.004910in}{0.004910in}}%
\pgfpathcurveto{\pgfqpoint{-0.006213in}{0.003608in}}{\pgfqpoint{-0.006944in}{0.001842in}}{\pgfqpoint{-0.006944in}{0.000000in}}%
\pgfpathcurveto{\pgfqpoint{-0.006944in}{-0.001842in}}{\pgfqpoint{-0.006213in}{-0.003608in}}{\pgfqpoint{-0.004910in}{-0.004910in}}%
\pgfpathcurveto{\pgfqpoint{-0.003608in}{-0.006213in}}{\pgfqpoint{-0.001842in}{-0.006944in}}{\pgfqpoint{0.000000in}{-0.006944in}}%
\pgfpathlineto{\pgfqpoint{0.000000in}{-0.006944in}}%
\pgfpathclose%
\pgfusepath{stroke,fill}%
}%
\begin{pgfscope}%
\pgfsys@transformshift{0.741425in}{1.377543in}%
\pgfsys@useobject{currentmarker}{}%
\end{pgfscope}%
\begin{pgfscope}%
\pgfsys@transformshift{0.758703in}{0.955032in}%
\pgfsys@useobject{currentmarker}{}%
\end{pgfscope}%
\begin{pgfscope}%
\pgfsys@transformshift{0.768198in}{0.875033in}%
\pgfsys@useobject{currentmarker}{}%
\end{pgfscope}%
\begin{pgfscope}%
\pgfsys@transformshift{0.774746in}{0.828781in}%
\pgfsys@useobject{currentmarker}{}%
\end{pgfscope}%
\begin{pgfscope}%
\pgfsys@transformshift{0.778243in}{0.822495in}%
\pgfsys@useobject{currentmarker}{}%
\end{pgfscope}%
\begin{pgfscope}%
\pgfsys@transformshift{0.782159in}{0.789611in}%
\pgfsys@useobject{currentmarker}{}%
\end{pgfscope}%
\begin{pgfscope}%
\pgfsys@transformshift{0.786516in}{0.779881in}%
\pgfsys@useobject{currentmarker}{}%
\end{pgfscope}%
\begin{pgfscope}%
\pgfsys@transformshift{0.792538in}{0.779145in}%
\pgfsys@useobject{currentmarker}{}%
\end{pgfscope}%
\begin{pgfscope}%
\pgfsys@transformshift{0.794668in}{0.758056in}%
\pgfsys@useobject{currentmarker}{}%
\end{pgfscope}%
\begin{pgfscope}%
\pgfsys@transformshift{0.799837in}{0.752930in}%
\pgfsys@useobject{currentmarker}{}%
\end{pgfscope}%
\begin{pgfscope}%
\pgfsys@transformshift{0.809370in}{0.751978in}%
\pgfsys@useobject{currentmarker}{}%
\end{pgfscope}%
\begin{pgfscope}%
\pgfsys@transformshift{0.812629in}{0.743975in}%
\pgfsys@useobject{currentmarker}{}%
\end{pgfscope}%
\begin{pgfscope}%
\pgfsys@transformshift{0.815972in}{0.742575in}%
\pgfsys@useobject{currentmarker}{}%
\end{pgfscope}%
\begin{pgfscope}%
\pgfsys@transformshift{0.822987in}{0.738477in}%
\pgfsys@useobject{currentmarker}{}%
\end{pgfscope}%
\begin{pgfscope}%
\pgfsys@transformshift{0.828825in}{0.734937in}%
\pgfsys@useobject{currentmarker}{}%
\end{pgfscope}%
\begin{pgfscope}%
\pgfsys@transformshift{0.829214in}{0.733319in}%
\pgfsys@useobject{currentmarker}{}%
\end{pgfscope}%
\begin{pgfscope}%
\pgfsys@transformshift{0.833044in}{0.730858in}%
\pgfsys@useobject{currentmarker}{}%
\end{pgfscope}%
\begin{pgfscope}%
\pgfsys@transformshift{0.848459in}{0.726329in}%
\pgfsys@useobject{currentmarker}{}%
\end{pgfscope}%
\begin{pgfscope}%
\pgfsys@transformshift{0.864854in}{0.720019in}%
\pgfsys@useobject{currentmarker}{}%
\end{pgfscope}%
\begin{pgfscope}%
\pgfsys@transformshift{0.887104in}{0.715517in}%
\pgfsys@useobject{currentmarker}{}%
\end{pgfscope}%
\begin{pgfscope}%
\pgfsys@transformshift{0.907479in}{0.714004in}%
\pgfsys@useobject{currentmarker}{}%
\end{pgfscope}%
\begin{pgfscope}%
\pgfsys@transformshift{0.908310in}{0.712008in}%
\pgfsys@useobject{currentmarker}{}%
\end{pgfscope}%
\begin{pgfscope}%
\pgfsys@transformshift{0.909513in}{0.708525in}%
\pgfsys@useobject{currentmarker}{}%
\end{pgfscope}%
\begin{pgfscope}%
\pgfsys@transformshift{0.912740in}{0.707284in}%
\pgfsys@useobject{currentmarker}{}%
\end{pgfscope}%
\begin{pgfscope}%
\pgfsys@transformshift{0.920440in}{0.706723in}%
\pgfsys@useobject{currentmarker}{}%
\end{pgfscope}%
\begin{pgfscope}%
\pgfsys@transformshift{0.925670in}{0.705238in}%
\pgfsys@useobject{currentmarker}{}%
\end{pgfscope}%
\begin{pgfscope}%
\pgfsys@transformshift{0.948903in}{0.702931in}%
\pgfsys@useobject{currentmarker}{}%
\end{pgfscope}%
\begin{pgfscope}%
\pgfsys@transformshift{0.951945in}{0.701707in}%
\pgfsys@useobject{currentmarker}{}%
\end{pgfscope}%
\begin{pgfscope}%
\pgfsys@transformshift{0.952035in}{0.700391in}%
\pgfsys@useobject{currentmarker}{}%
\end{pgfscope}%
\begin{pgfscope}%
\pgfsys@transformshift{0.957029in}{0.700173in}%
\pgfsys@useobject{currentmarker}{}%
\end{pgfscope}%
\begin{pgfscope}%
\pgfsys@transformshift{0.968828in}{0.697963in}%
\pgfsys@useobject{currentmarker}{}%
\end{pgfscope}%
\begin{pgfscope}%
\pgfsys@transformshift{0.974412in}{0.697738in}%
\pgfsys@useobject{currentmarker}{}%
\end{pgfscope}%
\begin{pgfscope}%
\pgfsys@transformshift{0.975275in}{0.696914in}%
\pgfsys@useobject{currentmarker}{}%
\end{pgfscope}%
\begin{pgfscope}%
\pgfsys@transformshift{1.021767in}{0.694795in}%
\pgfsys@useobject{currentmarker}{}%
\end{pgfscope}%
\begin{pgfscope}%
\pgfsys@transformshift{1.025407in}{0.690657in}%
\pgfsys@useobject{currentmarker}{}%
\end{pgfscope}%
\begin{pgfscope}%
\pgfsys@transformshift{1.027475in}{0.690338in}%
\pgfsys@useobject{currentmarker}{}%
\end{pgfscope}%
\begin{pgfscope}%
\pgfsys@transformshift{1.034837in}{0.689784in}%
\pgfsys@useobject{currentmarker}{}%
\end{pgfscope}%
\begin{pgfscope}%
\pgfsys@transformshift{1.049406in}{0.687676in}%
\pgfsys@useobject{currentmarker}{}%
\end{pgfscope}%
\begin{pgfscope}%
\pgfsys@transformshift{1.054714in}{0.687138in}%
\pgfsys@useobject{currentmarker}{}%
\end{pgfscope}%
\begin{pgfscope}%
\pgfsys@transformshift{1.059617in}{0.686467in}%
\pgfsys@useobject{currentmarker}{}%
\end{pgfscope}%
\begin{pgfscope}%
\pgfsys@transformshift{1.072141in}{0.685078in}%
\pgfsys@useobject{currentmarker}{}%
\end{pgfscope}%
\begin{pgfscope}%
\pgfsys@transformshift{1.092208in}{0.684413in}%
\pgfsys@useobject{currentmarker}{}%
\end{pgfscope}%
\begin{pgfscope}%
\pgfsys@transformshift{1.115209in}{0.684111in}%
\pgfsys@useobject{currentmarker}{}%
\end{pgfscope}%
\begin{pgfscope}%
\pgfsys@transformshift{1.131834in}{0.684071in}%
\pgfsys@useobject{currentmarker}{}%
\end{pgfscope}%
\begin{pgfscope}%
\pgfsys@transformshift{1.152628in}{0.684059in}%
\pgfsys@useobject{currentmarker}{}%
\end{pgfscope}%
\begin{pgfscope}%
\pgfsys@transformshift{1.251312in}{0.683263in}%
\pgfsys@useobject{currentmarker}{}%
\end{pgfscope}%
\begin{pgfscope}%
\pgfsys@transformshift{1.277476in}{0.683159in}%
\pgfsys@useobject{currentmarker}{}%
\end{pgfscope}%
\begin{pgfscope}%
\pgfsys@transformshift{1.314870in}{0.682855in}%
\pgfsys@useobject{currentmarker}{}%
\end{pgfscope}%
\begin{pgfscope}%
\pgfsys@transformshift{1.369253in}{0.682756in}%
\pgfsys@useobject{currentmarker}{}%
\end{pgfscope}%
\begin{pgfscope}%
\pgfsys@transformshift{1.398687in}{0.682288in}%
\pgfsys@useobject{currentmarker}{}%
\end{pgfscope}%
\begin{pgfscope}%
\pgfsys@transformshift{1.467852in}{0.682134in}%
\pgfsys@useobject{currentmarker}{}%
\end{pgfscope}%
\begin{pgfscope}%
\pgfsys@transformshift{1.557026in}{0.681680in}%
\pgfsys@useobject{currentmarker}{}%
\end{pgfscope}%
\begin{pgfscope}%
\pgfsys@transformshift{1.627242in}{0.680913in}%
\pgfsys@useobject{currentmarker}{}%
\end{pgfscope}%
\begin{pgfscope}%
\pgfsys@transformshift{1.737728in}{0.680478in}%
\pgfsys@useobject{currentmarker}{}%
\end{pgfscope}%
\begin{pgfscope}%
\pgfsys@transformshift{1.887036in}{0.679610in}%
\pgfsys@useobject{currentmarker}{}%
\end{pgfscope}%
\begin{pgfscope}%
\pgfsys@transformshift{2.037481in}{0.678826in}%
\pgfsys@useobject{currentmarker}{}%
\end{pgfscope}%
\begin{pgfscope}%
\pgfsys@transformshift{2.258348in}{0.677741in}%
\pgfsys@useobject{currentmarker}{}%
\end{pgfscope}%
\begin{pgfscope}%
\pgfsys@transformshift{2.626338in}{0.676361in}%
\pgfsys@useobject{currentmarker}{}%
\end{pgfscope}%
\begin{pgfscope}%
\pgfsys@transformshift{3.263784in}{0.674352in}%
\pgfsys@useobject{currentmarker}{}%
\end{pgfscope}%
\begin{pgfscope}%
\pgfsys@transformshift{5.322800in}{0.670138in}%
\pgfsys@useobject{currentmarker}{}%
\end{pgfscope}%
\end{pgfscope}%
\begin{pgfscope}%
\pgfpathrectangle{\pgfqpoint{0.688192in}{0.670138in}}{\pgfqpoint{6.200000in}{4.620000in}}%
\pgfusepath{clip}%
\pgfsetrectcap%
\pgfsetroundjoin%
\pgfsetlinewidth{1.505625pt}%
\definecolor{currentstroke}{rgb}{0.121569,0.466667,0.705882}%
\pgfsetstrokecolor{currentstroke}%
\pgfsetstrokeopacity{0.500000}%
\pgfsetdash{}{0pt}%
\pgfpathmoveto{\pgfqpoint{1.848679in}{1.461617in}}%
\pgfpathlineto{\pgfqpoint{1.867684in}{0.996854in}}%
\pgfpathlineto{\pgfqpoint{1.878129in}{0.908856in}}%
\pgfpathlineto{\pgfqpoint{1.885331in}{0.857978in}}%
\pgfpathlineto{\pgfqpoint{1.889178in}{0.851064in}}%
\pgfpathlineto{\pgfqpoint{1.893486in}{0.814892in}}%
\pgfpathlineto{\pgfqpoint{1.898279in}{0.804188in}}%
\pgfpathlineto{\pgfqpoint{1.904902in}{0.803379in}}%
\pgfpathlineto{\pgfqpoint{1.907246in}{0.780181in}}%
\pgfpathlineto{\pgfqpoint{1.912932in}{0.774543in}}%
\pgfpathlineto{\pgfqpoint{1.923418in}{0.773495in}}%
\pgfpathlineto{\pgfqpoint{1.927003in}{0.764692in}}%
\pgfpathlineto{\pgfqpoint{1.930681in}{0.763152in}}%
\pgfpathlineto{\pgfqpoint{1.938397in}{0.758644in}}%
\pgfpathlineto{\pgfqpoint{1.944819in}{0.754750in}}%
\pgfpathlineto{\pgfqpoint{1.945246in}{0.752971in}}%
\pgfpathlineto{\pgfqpoint{1.949459in}{0.750264in}}%
\pgfpathlineto{\pgfqpoint{1.966417in}{0.745282in}}%
\pgfpathlineto{\pgfqpoint{1.984450in}{0.738340in}}%
\pgfpathlineto{\pgfqpoint{2.008926in}{0.733388in}}%
\pgfpathlineto{\pgfqpoint{2.031338in}{0.731724in}}%
\pgfpathlineto{\pgfqpoint{2.032252in}{0.729528in}}%
\pgfpathlineto{\pgfqpoint{2.033576in}{0.725697in}}%
\pgfpathlineto{\pgfqpoint{2.037125in}{0.724332in}}%
\pgfpathlineto{\pgfqpoint{2.045595in}{0.723715in}}%
\pgfpathlineto{\pgfqpoint{2.051349in}{0.722081in}}%
\pgfpathlineto{\pgfqpoint{2.076904in}{0.719544in}}%
\pgfpathlineto{\pgfqpoint{2.080251in}{0.718197in}}%
\pgfpathlineto{\pgfqpoint{2.080349in}{0.716749in}}%
\pgfpathlineto{\pgfqpoint{2.085843in}{0.716510in}}%
\pgfpathlineto{\pgfqpoint{2.098822in}{0.714079in}}%
\pgfpathlineto{\pgfqpoint{2.104964in}{0.713831in}}%
\pgfpathlineto{\pgfqpoint{2.105914in}{0.712924in}}%
\pgfpathlineto{\pgfqpoint{2.157055in}{0.710594in}}%
\pgfpathlineto{\pgfqpoint{2.161059in}{0.706043in}}%
\pgfpathlineto{\pgfqpoint{2.163334in}{0.705692in}}%
\pgfpathlineto{\pgfqpoint{2.171432in}{0.705082in}}%
\pgfpathlineto{\pgfqpoint{2.187458in}{0.702763in}}%
\pgfpathlineto{\pgfqpoint{2.193297in}{0.702172in}}%
\pgfpathlineto{\pgfqpoint{2.198690in}{0.701434in}}%
\pgfpathlineto{\pgfqpoint{2.212467in}{0.699905in}}%
\pgfpathlineto{\pgfqpoint{2.234540in}{0.699174in}}%
\pgfpathlineto{\pgfqpoint{2.259841in}{0.698842in}}%
\pgfpathlineto{\pgfqpoint{2.278129in}{0.698798in}}%
\pgfpathlineto{\pgfqpoint{2.301002in}{0.698784in}}%
\pgfpathlineto{\pgfqpoint{2.409554in}{0.697909in}}%
\pgfpathlineto{\pgfqpoint{2.438334in}{0.697795in}}%
\pgfpathlineto{\pgfqpoint{2.479468in}{0.697460in}}%
\pgfpathlineto{\pgfqpoint{2.539290in}{0.697351in}}%
\pgfpathlineto{\pgfqpoint{2.571667in}{0.696836in}}%
\pgfpathlineto{\pgfqpoint{2.647748in}{0.696667in}}%
\pgfpathlineto{\pgfqpoint{2.745840in}{0.696168in}}%
\pgfpathlineto{\pgfqpoint{2.823078in}{0.695324in}}%
\pgfpathlineto{\pgfqpoint{2.944612in}{0.694845in}}%
\pgfpathlineto{\pgfqpoint{3.108851in}{0.693891in}}%
\pgfpathlineto{\pgfqpoint{3.274341in}{0.693028in}}%
\pgfpathlineto{\pgfqpoint{3.517294in}{0.691835in}}%
\pgfpathlineto{\pgfqpoint{3.922083in}{0.690316in}}%
\pgfpathlineto{\pgfqpoint{4.623274in}{0.688107in}}%
\pgfpathlineto{\pgfqpoint{6.888192in}{0.683471in}}%
\pgfusepath{stroke}%
\end{pgfscope}%
\begin{pgfscope}%
\pgfsetrectcap%
\pgfsetmiterjoin%
\pgfsetlinewidth{0.803000pt}%
\definecolor{currentstroke}{rgb}{0.000000,0.000000,0.000000}%
\pgfsetstrokecolor{currentstroke}%
\pgfsetdash{}{0pt}%
\pgfpathmoveto{\pgfqpoint{0.688192in}{0.670138in}}%
\pgfpathlineto{\pgfqpoint{0.688192in}{5.290138in}}%
\pgfusepath{stroke}%
\end{pgfscope}%
\begin{pgfscope}%
\pgfsetrectcap%
\pgfsetmiterjoin%
\pgfsetlinewidth{0.803000pt}%
\definecolor{currentstroke}{rgb}{0.000000,0.000000,0.000000}%
\pgfsetstrokecolor{currentstroke}%
\pgfsetdash{}{0pt}%
\pgfpathmoveto{\pgfqpoint{6.888192in}{0.670138in}}%
\pgfpathlineto{\pgfqpoint{6.888192in}{5.290138in}}%
\pgfusepath{stroke}%
\end{pgfscope}%
\begin{pgfscope}%
\pgfsetrectcap%
\pgfsetmiterjoin%
\pgfsetlinewidth{0.803000pt}%
\definecolor{currentstroke}{rgb}{0.000000,0.000000,0.000000}%
\pgfsetstrokecolor{currentstroke}%
\pgfsetdash{}{0pt}%
\pgfpathmoveto{\pgfqpoint{0.688192in}{0.670138in}}%
\pgfpathlineto{\pgfqpoint{6.888192in}{0.670138in}}%
\pgfusepath{stroke}%
\end{pgfscope}%
\begin{pgfscope}%
\pgfsetrectcap%
\pgfsetmiterjoin%
\pgfsetlinewidth{0.803000pt}%
\definecolor{currentstroke}{rgb}{0.000000,0.000000,0.000000}%
\pgfsetstrokecolor{currentstroke}%
\pgfsetdash{}{0pt}%
\pgfpathmoveto{\pgfqpoint{0.688192in}{5.290138in}}%
\pgfpathlineto{\pgfqpoint{6.888192in}{5.290138in}}%
\pgfusepath{stroke}%
\end{pgfscope}%
\begin{pgfscope}%
\pgfsetbuttcap%
\pgfsetmiterjoin%
\pgfsetlinewidth{1.003750pt}%
\definecolor{currentstroke}{rgb}{0.000000,0.000000,0.000000}%
\pgfsetstrokecolor{currentstroke}%
\pgfsetstrokeopacity{0.500000}%
\pgfsetdash{}{0pt}%
\pgfpathmoveto{\pgfqpoint{0.646542in}{1.071430in}}%
\pgfpathlineto{\pgfqpoint{0.810103in}{1.071430in}}%
\pgfpathlineto{\pgfqpoint{0.810103in}{1.434970in}}%
\pgfpathlineto{\pgfqpoint{0.646542in}{1.434970in}}%
\pgfpathlineto{\pgfqpoint{0.646542in}{1.071430in}}%
\pgfpathclose%
\pgfpathmoveto{\pgfqpoint{3.788192in}{5.151538in}}%
\pgfpathquadraticcurveto{\pgfqpoint{2.217367in}{3.293254in}}{\pgfqpoint{0.646542in}{1.434970in}}%
\pgfpathmoveto{\pgfqpoint{6.702192in}{2.980138in}}%
\pgfpathquadraticcurveto{\pgfqpoint{3.756147in}{2.025784in}}{\pgfqpoint{0.810103in}{1.071430in}}%
\pgfusepath{stroke}%
\end{pgfscope}%
\begin{pgfscope}%
\pgfsetbuttcap%
\pgfsetmiterjoin%
\definecolor{currentfill}{rgb}{1.000000,1.000000,1.000000}%
\pgfsetfillcolor{currentfill}%
\pgfsetlinewidth{0.000000pt}%
\definecolor{currentstroke}{rgb}{0.000000,0.000000,0.000000}%
\pgfsetstrokecolor{currentstroke}%
\pgfsetstrokeopacity{0.000000}%
\pgfsetdash{}{0pt}%
\pgfpathmoveto{\pgfqpoint{3.788192in}{2.980138in}}%
\pgfpathlineto{\pgfqpoint{6.702192in}{2.980138in}}%
\pgfpathlineto{\pgfqpoint{6.702192in}{5.151538in}}%
\pgfpathlineto{\pgfqpoint{3.788192in}{5.151538in}}%
\pgfpathlineto{\pgfqpoint{3.788192in}{2.980138in}}%
\pgfpathclose%
\pgfusepath{fill}%
\end{pgfscope}%
\begin{pgfscope}%
\pgfpathrectangle{\pgfqpoint{3.788192in}{2.980138in}}{\pgfqpoint{2.914000in}{2.171400in}}%
\pgfusepath{clip}%
\pgfsetbuttcap%
\pgfsetmiterjoin%
\definecolor{currentfill}{rgb}{0.121569,0.466667,0.705882}%
\pgfsetfillcolor{currentfill}%
\pgfsetfillopacity{0.500000}%
\pgfsetlinewidth{1.003750pt}%
\definecolor{currentstroke}{rgb}{0.121569,0.466667,0.705882}%
\pgfsetstrokecolor{currentstroke}%
\pgfsetstrokeopacity{0.500000}%
\pgfsetdash{}{0pt}%
\pgfpathmoveto{\pgfqpoint{5.478622in}{4.808529in}}%
\pgfpathlineto{\pgfqpoint{5.786444in}{2.284897in}}%
\pgfpathlineto{\pgfqpoint{5.955602in}{1.807070in}}%
\pgfpathlineto{\pgfqpoint{6.072257in}{1.530809in}}%
\pgfpathlineto{\pgfqpoint{6.134561in}{1.493263in}}%
\pgfpathlineto{\pgfqpoint{6.204334in}{1.296852in}}%
\pgfpathlineto{\pgfqpoint{6.281957in}{1.238734in}}%
\pgfpathlineto{\pgfqpoint{6.389240in}{1.234341in}}%
\pgfpathlineto{\pgfqpoint{6.427201in}{1.108377in}}%
\pgfpathlineto{\pgfqpoint{6.519285in}{1.077759in}}%
\pgfpathlineto{\pgfqpoint{6.689124in}{1.072073in}}%
\pgfpathlineto{\pgfqpoint{6.747190in}{1.024269in}}%
\pgfpathlineto{\pgfqpoint{6.806750in}{1.015907in}}%
\pgfpathlineto{\pgfqpoint{6.931730in}{0.991431in}}%
\pgfpathlineto{\pgfqpoint{7.035734in}{0.970288in}}%
\pgfpathlineto{\pgfqpoint{7.042663in}{0.960624in}}%
\pgfpathlineto{\pgfqpoint{7.110896in}{0.945926in}}%
\pgfpathlineto{\pgfqpoint{7.385541in}{0.918874in}}%
\pgfpathlineto{\pgfqpoint{7.677622in}{0.881182in}}%
\pgfpathlineto{\pgfqpoint{8.074033in}{0.854291in}}%
\pgfpathlineto{\pgfqpoint{8.437037in}{0.845259in}}%
\pgfpathlineto{\pgfqpoint{8.451833in}{0.833333in}}%
\pgfpathlineto{\pgfqpoint{8.473270in}{0.812531in}}%
\pgfpathlineto{\pgfqpoint{8.530759in}{0.805118in}}%
\pgfpathlineto{\pgfqpoint{8.667946in}{0.801766in}}%
\pgfpathlineto{\pgfqpoint{8.761129in}{0.792896in}}%
\pgfpathlineto{\pgfqpoint{9.175032in}{0.779118in}}%
\pgfpathlineto{\pgfqpoint{9.229242in}{0.771806in}}%
\pgfpathlineto{\pgfqpoint{9.230836in}{0.763945in}}%
\pgfpathlineto{\pgfqpoint{9.319803in}{0.762647in}}%
\pgfpathlineto{\pgfqpoint{9.530028in}{0.749444in}}%
\pgfpathlineto{\pgfqpoint{9.629502in}{0.748101in}}%
\pgfpathlineto{\pgfqpoint{9.644888in}{0.743176in}}%
\pgfpathlineto{\pgfqpoint{10.473184in}{0.730520in}}%
\pgfpathlineto{\pgfqpoint{10.538033in}{0.705808in}}%
\pgfpathlineto{\pgfqpoint{10.574876in}{0.703903in}}%
\pgfpathlineto{\pgfqpoint{10.706030in}{0.700593in}}%
\pgfpathlineto{\pgfqpoint{10.965602in}{0.687998in}}%
\pgfpathlineto{\pgfqpoint{11.060169in}{0.684789in}}%
\pgfpathlineto{\pgfqpoint{11.147519in}{0.680781in}}%
\pgfpathlineto{\pgfqpoint{11.370645in}{0.672484in}}%
\pgfpathlineto{\pgfqpoint{11.728159in}{0.668514in}}%
\pgfpathlineto{\pgfqpoint{12.137942in}{0.666709in}}%
\pgfpathlineto{\pgfqpoint{12.434130in}{0.666468in}}%
\pgfpathlineto{\pgfqpoint{12.804602in}{0.666397in}}%
\pgfpathlineto{\pgfqpoint{14.562739in}{0.661643in}}%
\pgfpathlineto{\pgfqpoint{15.028873in}{0.661022in}}%
\pgfpathlineto{\pgfqpoint{15.695083in}{0.659204in}}%
\pgfpathlineto{\pgfqpoint{16.663977in}{0.658612in}}%
\pgfpathlineto{\pgfqpoint{17.188373in}{0.655818in}}%
\pgfpathlineto{\pgfqpoint{18.420605in}{0.654898in}}%
\pgfpathlineto{\pgfqpoint{20.009331in}{0.652189in}}%
\pgfpathlineto{\pgfqpoint{21.260300in}{0.647607in}}%
\pgfpathlineto{\pgfqpoint{23.228701in}{0.645006in}}%
\pgfpathlineto{\pgfqpoint{25.888770in}{0.639824in}}%
\pgfpathlineto{\pgfqpoint{28.569097in}{0.635141in}}%
\pgfpathlineto{\pgfqpoint{32.504050in}{0.628662in}}%
\pgfpathlineto{\pgfqpoint{39.060135in}{0.620415in}}%
\pgfpathlineto{\pgfqpoint{50.416853in}{0.608419in}}%
\pgfpathlineto{\pgfqpoint{87.100182in}{0.583248in}}%
\pgfpathlineto{\pgfqpoint{114.989116in}{0.662886in}}%
\pgfpathlineto{\pgfqpoint{74.637454in}{0.690574in}}%
\pgfpathlineto{\pgfqpoint{62.145064in}{0.703770in}}%
\pgfpathlineto{\pgfqpoint{54.933371in}{0.712842in}}%
\pgfpathlineto{\pgfqpoint{50.604922in}{0.719969in}}%
\pgfpathlineto{\pgfqpoint{47.656562in}{0.725121in}}%
\pgfpathlineto{\pgfqpoint{44.730486in}{0.730820in}}%
\pgfpathlineto{\pgfqpoint{42.565245in}{0.733682in}}%
\pgfpathlineto{\pgfqpoint{41.189180in}{0.738722in}}%
\pgfpathlineto{\pgfqpoint{39.441581in}{0.741702in}}%
\pgfpathlineto{\pgfqpoint{38.086125in}{0.742714in}}%
\pgfpathlineto{\pgfqpoint{37.509290in}{0.745787in}}%
\pgfpathlineto{\pgfqpoint{36.443507in}{0.746439in}}%
\pgfpathlineto{\pgfqpoint{35.710675in}{0.748438in}}%
\pgfpathlineto{\pgfqpoint{35.197928in}{0.749121in}}%
\pgfpathlineto{\pgfqpoint{33.263977in}{0.754350in}}%
\pgfpathlineto{\pgfqpoint{32.856458in}{0.754429in}}%
\pgfpathlineto{\pgfqpoint{32.530652in}{0.754694in}}%
\pgfpathlineto{\pgfqpoint{32.079890in}{0.756679in}}%
\pgfpathlineto{\pgfqpoint{31.686625in}{0.761046in}}%
\pgfpathlineto{\pgfqpoint{31.441186in}{0.770173in}}%
\pgfpathlineto{\pgfqpoint{31.345101in}{0.774582in}}%
\pgfpathlineto{\pgfqpoint{31.241077in}{0.778112in}}%
\pgfpathlineto{\pgfqpoint{30.955549in}{0.791966in}}%
\pgfpathlineto{\pgfqpoint{30.811278in}{0.795608in}}%
\pgfpathlineto{\pgfqpoint{30.770752in}{0.797703in}}%
\pgfpathlineto{\pgfqpoint{30.699417in}{0.824886in}}%
\pgfpathlineto{\pgfqpoint{29.788292in}{0.838808in}}%
\pgfpathlineto{\pgfqpoint{29.771368in}{0.844225in}}%
\pgfpathlineto{\pgfqpoint{29.661946in}{0.845702in}}%
\pgfpathlineto{\pgfqpoint{29.430699in}{0.860225in}}%
\pgfpathlineto{\pgfqpoint{29.332834in}{0.861654in}}%
\pgfpathlineto{\pgfqpoint{29.331081in}{0.870301in}}%
\pgfpathlineto{\pgfqpoint{29.271451in}{0.878344in}}%
\pgfpathlineto{\pgfqpoint{28.816157in}{0.893499in}}%
\pgfpathlineto{\pgfqpoint{28.713656in}{0.903257in}}%
\pgfpathlineto{\pgfqpoint{28.562750in}{0.906943in}}%
\pgfpathlineto{\pgfqpoint{28.499512in}{0.915098in}}%
\pgfpathlineto{\pgfqpoint{28.475931in}{0.937980in}}%
\pgfpathlineto{\pgfqpoint{28.459655in}{0.951098in}}%
\pgfpathlineto{\pgfqpoint{28.060351in}{0.961034in}}%
\pgfpathlineto{\pgfqpoint{27.624299in}{0.990614in}}%
\pgfpathlineto{\pgfqpoint{27.303010in}{1.032076in}}%
\pgfpathlineto{\pgfqpoint{27.000901in}{1.061832in}}%
\pgfpathlineto{\pgfqpoint{26.925844in}{1.078000in}}%
\pgfpathlineto{\pgfqpoint{26.918222in}{1.088630in}}%
\pgfpathlineto{\pgfqpoint{26.803818in}{1.111888in}}%
\pgfpathlineto{\pgfqpoint{26.666341in}{1.138811in}}%
\pgfpathlineto{\pgfqpoint{26.600824in}{1.148010in}}%
\pgfpathlineto{\pgfqpoint{26.536952in}{1.200594in}}%
\pgfpathlineto{\pgfqpoint{26.350128in}{1.206849in}}%
\pgfpathlineto{\pgfqpoint{26.248836in}{1.240528in}}%
\pgfpathlineto{\pgfqpoint{26.207079in}{1.379089in}}%
\pgfpathlineto{\pgfqpoint{26.089068in}{1.383921in}}%
\pgfpathlineto{\pgfqpoint{26.003683in}{1.447851in}}%
\pgfpathlineto{\pgfqpoint{25.926932in}{1.663903in}}%
\pgfpathlineto{\pgfqpoint{25.858398in}{1.705204in}}%
\pgfpathlineto{\pgfqpoint{25.730077in}{2.009091in}}%
\pgfpathlineto{\pgfqpoint{25.544004in}{2.534701in}}%
\pgfpathlineto{\pgfqpoint{25.205400in}{5.310696in}}%
\pgfpathlineto{\pgfqpoint{5.478622in}{4.808529in}}%
\pgfpathclose%
\pgfusepath{stroke,fill}%
\end{pgfscope}%
\begin{pgfscope}%
\pgfpathrectangle{\pgfqpoint{3.788192in}{2.980138in}}{\pgfqpoint{2.914000in}{2.171400in}}%
\pgfusepath{clip}%
\pgfsetbuttcap%
\pgfsetroundjoin%
\pgfsetlinewidth{1.003750pt}%
\definecolor{currentstroke}{rgb}{1.000000,0.000000,0.000000}%
\pgfsetstrokecolor{currentstroke}%
\pgfsetdash{}{0pt}%
\pgfpathmoveto{\pgfqpoint{16.861003in}{25.566105in}}%
\pgfpathcurveto{\pgfqpoint{16.869240in}{25.566105in}}{\pgfqpoint{16.877140in}{25.569377in}}{\pgfqpoint{16.882964in}{25.575201in}}%
\pgfpathcurveto{\pgfqpoint{16.888788in}{25.581025in}}{\pgfqpoint{16.892060in}{25.588925in}}{\pgfqpoint{16.892060in}{25.597161in}}%
\pgfpathcurveto{\pgfqpoint{16.892060in}{25.605397in}}{\pgfqpoint{16.888788in}{25.613297in}}{\pgfqpoint{16.882964in}{25.619121in}}%
\pgfpathcurveto{\pgfqpoint{16.877140in}{25.624945in}}{\pgfqpoint{16.869240in}{25.628218in}}{\pgfqpoint{16.861003in}{25.628218in}}%
\pgfpathcurveto{\pgfqpoint{16.852767in}{25.628218in}}{\pgfqpoint{16.844867in}{25.624945in}}{\pgfqpoint{16.839043in}{25.619121in}}%
\pgfpathcurveto{\pgfqpoint{16.833219in}{25.613297in}}{\pgfqpoint{16.829947in}{25.605397in}}{\pgfqpoint{16.829947in}{25.597161in}}%
\pgfpathcurveto{\pgfqpoint{16.829947in}{25.588925in}}{\pgfqpoint{16.833219in}{25.581025in}}{\pgfqpoint{16.839043in}{25.575201in}}%
\pgfpathcurveto{\pgfqpoint{16.844867in}{25.569377in}}{\pgfqpoint{16.852767in}{25.566105in}}{\pgfqpoint{16.861003in}{25.566105in}}%
\pgfusepath{stroke}%
\end{pgfscope}%
\begin{pgfscope}%
\pgfpathrectangle{\pgfqpoint{3.788192in}{2.980138in}}{\pgfqpoint{2.914000in}{2.171400in}}%
\pgfusepath{clip}%
\pgfsetbuttcap%
\pgfsetroundjoin%
\pgfsetlinewidth{1.003750pt}%
\definecolor{currentstroke}{rgb}{1.000000,0.000000,0.000000}%
\pgfsetstrokecolor{currentstroke}%
\pgfsetdash{}{0pt}%
\pgfpathmoveto{\pgfqpoint{11.934253in}{10.111217in}}%
\pgfpathcurveto{\pgfqpoint{11.942489in}{10.111217in}}{\pgfqpoint{11.950389in}{10.114489in}}{\pgfqpoint{11.956213in}{10.120313in}}%
\pgfpathcurveto{\pgfqpoint{11.962037in}{10.126137in}}{\pgfqpoint{11.965309in}{10.134037in}}{\pgfqpoint{11.965309in}{10.142274in}}%
\pgfpathcurveto{\pgfqpoint{11.965309in}{10.150510in}}{\pgfqpoint{11.962037in}{10.158410in}}{\pgfqpoint{11.956213in}{10.164234in}}%
\pgfpathcurveto{\pgfqpoint{11.950389in}{10.170058in}}{\pgfqpoint{11.942489in}{10.173330in}}{\pgfqpoint{11.934253in}{10.173330in}}%
\pgfpathcurveto{\pgfqpoint{11.926016in}{10.173330in}}{\pgfqpoint{11.918116in}{10.170058in}}{\pgfqpoint{11.912292in}{10.164234in}}%
\pgfpathcurveto{\pgfqpoint{11.906469in}{10.158410in}}{\pgfqpoint{11.903196in}{10.150510in}}{\pgfqpoint{11.903196in}{10.142274in}}%
\pgfpathcurveto{\pgfqpoint{11.903196in}{10.134037in}}{\pgfqpoint{11.906469in}{10.126137in}}{\pgfqpoint{11.912292in}{10.120313in}}%
\pgfpathcurveto{\pgfqpoint{11.918116in}{10.114489in}}{\pgfqpoint{11.926016in}{10.111217in}}{\pgfqpoint{11.934253in}{10.111217in}}%
\pgfusepath{stroke}%
\end{pgfscope}%
\begin{pgfscope}%
\pgfpathrectangle{\pgfqpoint{3.788192in}{2.980138in}}{\pgfqpoint{2.914000in}{2.171400in}}%
\pgfusepath{clip}%
\pgfsetbuttcap%
\pgfsetroundjoin%
\pgfsetlinewidth{1.003750pt}%
\definecolor{currentstroke}{rgb}{1.000000,0.000000,0.000000}%
\pgfsetstrokecolor{currentstroke}%
\pgfsetdash{}{0pt}%
\pgfpathmoveto{\pgfqpoint{12.691767in}{10.530438in}}%
\pgfpathcurveto{\pgfqpoint{12.700003in}{10.530438in}}{\pgfqpoint{12.707903in}{10.533710in}}{\pgfqpoint{12.713727in}{10.539534in}}%
\pgfpathcurveto{\pgfqpoint{12.719551in}{10.545358in}}{\pgfqpoint{12.722823in}{10.553258in}}{\pgfqpoint{12.722823in}{10.561494in}}%
\pgfpathcurveto{\pgfqpoint{12.722823in}{10.569730in}}{\pgfqpoint{12.719551in}{10.577630in}}{\pgfqpoint{12.713727in}{10.583454in}}%
\pgfpathcurveto{\pgfqpoint{12.707903in}{10.589278in}}{\pgfqpoint{12.700003in}{10.592551in}}{\pgfqpoint{12.691767in}{10.592551in}}%
\pgfpathcurveto{\pgfqpoint{12.683530in}{10.592551in}}{\pgfqpoint{12.675630in}{10.589278in}}{\pgfqpoint{12.669806in}{10.583454in}}%
\pgfpathcurveto{\pgfqpoint{12.663983in}{10.577630in}}{\pgfqpoint{12.660710in}{10.569730in}}{\pgfqpoint{12.660710in}{10.561494in}}%
\pgfpathcurveto{\pgfqpoint{12.660710in}{10.553258in}}{\pgfqpoint{12.663983in}{10.545358in}}{\pgfqpoint{12.669806in}{10.539534in}}%
\pgfpathcurveto{\pgfqpoint{12.675630in}{10.533710in}}{\pgfqpoint{12.683530in}{10.530438in}}{\pgfqpoint{12.691767in}{10.530438in}}%
\pgfusepath{stroke}%
\end{pgfscope}%
\begin{pgfscope}%
\pgfpathrectangle{\pgfqpoint{3.788192in}{2.980138in}}{\pgfqpoint{2.914000in}{2.171400in}}%
\pgfusepath{clip}%
\pgfsetbuttcap%
\pgfsetroundjoin%
\pgfsetlinewidth{1.003750pt}%
\definecolor{currentstroke}{rgb}{1.000000,0.000000,0.000000}%
\pgfsetstrokecolor{currentstroke}%
\pgfsetdash{}{0pt}%
\pgfpathmoveto{\pgfqpoint{15.144604in}{10.695451in}}%
\pgfpathcurveto{\pgfqpoint{15.152841in}{10.695451in}}{\pgfqpoint{15.160741in}{10.698723in}}{\pgfqpoint{15.166565in}{10.704547in}}%
\pgfpathcurveto{\pgfqpoint{15.172389in}{10.710371in}}{\pgfqpoint{15.175661in}{10.718271in}}{\pgfqpoint{15.175661in}{10.726507in}}%
\pgfpathcurveto{\pgfqpoint{15.175661in}{10.734743in}}{\pgfqpoint{15.172389in}{10.742643in}}{\pgfqpoint{15.166565in}{10.748467in}}%
\pgfpathcurveto{\pgfqpoint{15.160741in}{10.754291in}}{\pgfqpoint{15.152841in}{10.757564in}}{\pgfqpoint{15.144604in}{10.757564in}}%
\pgfpathcurveto{\pgfqpoint{15.136368in}{10.757564in}}{\pgfqpoint{15.128468in}{10.754291in}}{\pgfqpoint{15.122644in}{10.748467in}}%
\pgfpathcurveto{\pgfqpoint{15.116820in}{10.742643in}}{\pgfqpoint{15.113548in}{10.734743in}}{\pgfqpoint{15.113548in}{10.726507in}}%
\pgfpathcurveto{\pgfqpoint{15.113548in}{10.718271in}}{\pgfqpoint{15.116820in}{10.710371in}}{\pgfqpoint{15.122644in}{10.704547in}}%
\pgfpathcurveto{\pgfqpoint{15.128468in}{10.698723in}}{\pgfqpoint{15.136368in}{10.695451in}}{\pgfqpoint{15.144604in}{10.695451in}}%
\pgfusepath{stroke}%
\end{pgfscope}%
\begin{pgfscope}%
\pgfpathrectangle{\pgfqpoint{3.788192in}{2.980138in}}{\pgfqpoint{2.914000in}{2.171400in}}%
\pgfusepath{clip}%
\pgfsetbuttcap%
\pgfsetroundjoin%
\pgfsetlinewidth{1.003750pt}%
\definecolor{currentstroke}{rgb}{1.000000,0.000000,0.000000}%
\pgfsetstrokecolor{currentstroke}%
\pgfsetdash{}{0pt}%
\pgfpathmoveto{\pgfqpoint{13.542463in}{11.423191in}}%
\pgfpathcurveto{\pgfqpoint{13.550699in}{11.423191in}}{\pgfqpoint{13.558599in}{11.426464in}}{\pgfqpoint{13.564423in}{11.432288in}}%
\pgfpathcurveto{\pgfqpoint{13.570247in}{11.438112in}}{\pgfqpoint{13.573519in}{11.446012in}}{\pgfqpoint{13.573519in}{11.454248in}}%
\pgfpathcurveto{\pgfqpoint{13.573519in}{11.462484in}}{\pgfqpoint{13.570247in}{11.470384in}}{\pgfqpoint{13.564423in}{11.476208in}}%
\pgfpathcurveto{\pgfqpoint{13.558599in}{11.482032in}}{\pgfqpoint{13.550699in}{11.485304in}}{\pgfqpoint{13.542463in}{11.485304in}}%
\pgfpathcurveto{\pgfqpoint{13.534227in}{11.485304in}}{\pgfqpoint{13.526327in}{11.482032in}}{\pgfqpoint{13.520503in}{11.476208in}}%
\pgfpathcurveto{\pgfqpoint{13.514679in}{11.470384in}}{\pgfqpoint{13.511406in}{11.462484in}}{\pgfqpoint{13.511406in}{11.454248in}}%
\pgfpathcurveto{\pgfqpoint{13.511406in}{11.446012in}}{\pgfqpoint{13.514679in}{11.438112in}}{\pgfqpoint{13.520503in}{11.432288in}}%
\pgfpathcurveto{\pgfqpoint{13.526327in}{11.426464in}}{\pgfqpoint{13.534227in}{11.423191in}}{\pgfqpoint{13.542463in}{11.423191in}}%
\pgfusepath{stroke}%
\end{pgfscope}%
\begin{pgfscope}%
\pgfpathrectangle{\pgfqpoint{3.788192in}{2.980138in}}{\pgfqpoint{2.914000in}{2.171400in}}%
\pgfusepath{clip}%
\pgfsetbuttcap%
\pgfsetroundjoin%
\pgfsetlinewidth{1.003750pt}%
\definecolor{currentstroke}{rgb}{1.000000,0.000000,0.000000}%
\pgfsetstrokecolor{currentstroke}%
\pgfsetdash{}{0pt}%
\pgfpathmoveto{\pgfqpoint{12.332968in}{8.917460in}}%
\pgfpathcurveto{\pgfqpoint{12.341204in}{8.917460in}}{\pgfqpoint{12.349104in}{8.920732in}}{\pgfqpoint{12.354928in}{8.926556in}}%
\pgfpathcurveto{\pgfqpoint{12.360752in}{8.932380in}}{\pgfqpoint{12.364025in}{8.940280in}}{\pgfqpoint{12.364025in}{8.948517in}}%
\pgfpathcurveto{\pgfqpoint{12.364025in}{8.956753in}}{\pgfqpoint{12.360752in}{8.964653in}}{\pgfqpoint{12.354928in}{8.970477in}}%
\pgfpathcurveto{\pgfqpoint{12.349104in}{8.976301in}}{\pgfqpoint{12.341204in}{8.979573in}}{\pgfqpoint{12.332968in}{8.979573in}}%
\pgfpathcurveto{\pgfqpoint{12.324732in}{8.979573in}}{\pgfqpoint{12.316832in}{8.976301in}}{\pgfqpoint{12.311008in}{8.970477in}}%
\pgfpathcurveto{\pgfqpoint{12.305184in}{8.964653in}}{\pgfqpoint{12.301912in}{8.956753in}}{\pgfqpoint{12.301912in}{8.948517in}}%
\pgfpathcurveto{\pgfqpoint{12.301912in}{8.940280in}}{\pgfqpoint{12.305184in}{8.932380in}}{\pgfqpoint{12.311008in}{8.926556in}}%
\pgfpathcurveto{\pgfqpoint{12.316832in}{8.920732in}}{\pgfqpoint{12.324732in}{8.917460in}}{\pgfqpoint{12.332968in}{8.917460in}}%
\pgfusepath{stroke}%
\end{pgfscope}%
\begin{pgfscope}%
\pgfpathrectangle{\pgfqpoint{3.788192in}{2.980138in}}{\pgfqpoint{2.914000in}{2.171400in}}%
\pgfusepath{clip}%
\pgfsetbuttcap%
\pgfsetroundjoin%
\pgfsetlinewidth{1.003750pt}%
\definecolor{currentstroke}{rgb}{1.000000,0.000000,0.000000}%
\pgfsetstrokecolor{currentstroke}%
\pgfsetdash{}{0pt}%
\pgfpathmoveto{\pgfqpoint{11.399902in}{7.106716in}}%
\pgfpathcurveto{\pgfqpoint{11.408139in}{7.106716in}}{\pgfqpoint{11.416039in}{7.109988in}}{\pgfqpoint{11.421863in}{7.115812in}}%
\pgfpathcurveto{\pgfqpoint{11.427686in}{7.121636in}}{\pgfqpoint{11.430959in}{7.129536in}}{\pgfqpoint{11.430959in}{7.137773in}}%
\pgfpathcurveto{\pgfqpoint{11.430959in}{7.146009in}}{\pgfqpoint{11.427686in}{7.153909in}}{\pgfqpoint{11.421863in}{7.159733in}}%
\pgfpathcurveto{\pgfqpoint{11.416039in}{7.165557in}}{\pgfqpoint{11.408139in}{7.168829in}}{\pgfqpoint{11.399902in}{7.168829in}}%
\pgfpathcurveto{\pgfqpoint{11.391666in}{7.168829in}}{\pgfqpoint{11.383766in}{7.165557in}}{\pgfqpoint{11.377942in}{7.159733in}}%
\pgfpathcurveto{\pgfqpoint{11.372118in}{7.153909in}}{\pgfqpoint{11.368846in}{7.146009in}}{\pgfqpoint{11.368846in}{7.137773in}}%
\pgfpathcurveto{\pgfqpoint{11.368846in}{7.129536in}}{\pgfqpoint{11.372118in}{7.121636in}}{\pgfqpoint{11.377942in}{7.115812in}}%
\pgfpathcurveto{\pgfqpoint{11.383766in}{7.109988in}}{\pgfqpoint{11.391666in}{7.106716in}}{\pgfqpoint{11.399902in}{7.106716in}}%
\pgfusepath{stroke}%
\end{pgfscope}%
\begin{pgfscope}%
\pgfpathrectangle{\pgfqpoint{3.788192in}{2.980138in}}{\pgfqpoint{2.914000in}{2.171400in}}%
\pgfusepath{clip}%
\pgfsetbuttcap%
\pgfsetroundjoin%
\pgfsetlinewidth{1.003750pt}%
\definecolor{currentstroke}{rgb}{1.000000,0.000000,0.000000}%
\pgfsetstrokecolor{currentstroke}%
\pgfsetdash{}{0pt}%
\pgfpathmoveto{\pgfqpoint{11.356549in}{7.029726in}}%
\pgfpathcurveto{\pgfqpoint{11.364785in}{7.029726in}}{\pgfqpoint{11.372685in}{7.032998in}}{\pgfqpoint{11.378509in}{7.038822in}}%
\pgfpathcurveto{\pgfqpoint{11.384333in}{7.044646in}}{\pgfqpoint{11.387605in}{7.052546in}}{\pgfqpoint{11.387605in}{7.060782in}}%
\pgfpathcurveto{\pgfqpoint{11.387605in}{7.069018in}}{\pgfqpoint{11.384333in}{7.076918in}}{\pgfqpoint{11.378509in}{7.082742in}}%
\pgfpathcurveto{\pgfqpoint{11.372685in}{7.088566in}}{\pgfqpoint{11.364785in}{7.091839in}}{\pgfqpoint{11.356549in}{7.091839in}}%
\pgfpathcurveto{\pgfqpoint{11.348313in}{7.091839in}}{\pgfqpoint{11.340412in}{7.088566in}}{\pgfqpoint{11.334589in}{7.082742in}}%
\pgfpathcurveto{\pgfqpoint{11.328765in}{7.076918in}}{\pgfqpoint{11.325492in}{7.069018in}}{\pgfqpoint{11.325492in}{7.060782in}}%
\pgfpathcurveto{\pgfqpoint{11.325492in}{7.052546in}}{\pgfqpoint{11.328765in}{7.044646in}}{\pgfqpoint{11.334589in}{7.038822in}}%
\pgfpathcurveto{\pgfqpoint{11.340412in}{7.032998in}}{\pgfqpoint{11.348313in}{7.029726in}}{\pgfqpoint{11.356549in}{7.029726in}}%
\pgfusepath{stroke}%
\end{pgfscope}%
\begin{pgfscope}%
\pgfpathrectangle{\pgfqpoint{3.788192in}{2.980138in}}{\pgfqpoint{2.914000in}{2.171400in}}%
\pgfusepath{clip}%
\pgfsetbuttcap%
\pgfsetroundjoin%
\pgfsetlinewidth{1.003750pt}%
\definecolor{currentstroke}{rgb}{1.000000,0.000000,0.000000}%
\pgfsetstrokecolor{currentstroke}%
\pgfsetdash{}{0pt}%
\pgfpathmoveto{\pgfqpoint{13.232346in}{9.803683in}}%
\pgfpathcurveto{\pgfqpoint{13.240582in}{9.803683in}}{\pgfqpoint{13.248482in}{9.806956in}}{\pgfqpoint{13.254306in}{9.812780in}}%
\pgfpathcurveto{\pgfqpoint{13.260130in}{9.818604in}}{\pgfqpoint{13.263402in}{9.826504in}}{\pgfqpoint{13.263402in}{9.834740in}}%
\pgfpathcurveto{\pgfqpoint{13.263402in}{9.842976in}}{\pgfqpoint{13.260130in}{9.850876in}}{\pgfqpoint{13.254306in}{9.856700in}}%
\pgfpathcurveto{\pgfqpoint{13.248482in}{9.862524in}}{\pgfqpoint{13.240582in}{9.865796in}}{\pgfqpoint{13.232346in}{9.865796in}}%
\pgfpathcurveto{\pgfqpoint{13.224110in}{9.865796in}}{\pgfqpoint{13.216210in}{9.862524in}}{\pgfqpoint{13.210386in}{9.856700in}}%
\pgfpathcurveto{\pgfqpoint{13.204562in}{9.850876in}}{\pgfqpoint{13.201289in}{9.842976in}}{\pgfqpoint{13.201289in}{9.834740in}}%
\pgfpathcurveto{\pgfqpoint{13.201289in}{9.826504in}}{\pgfqpoint{13.204562in}{9.818604in}}{\pgfqpoint{13.210386in}{9.812780in}}%
\pgfpathcurveto{\pgfqpoint{13.216210in}{9.806956in}}{\pgfqpoint{13.224110in}{9.803683in}}{\pgfqpoint{13.232346in}{9.803683in}}%
\pgfusepath{stroke}%
\end{pgfscope}%
\begin{pgfscope}%
\pgfpathrectangle{\pgfqpoint{3.788192in}{2.980138in}}{\pgfqpoint{2.914000in}{2.171400in}}%
\pgfusepath{clip}%
\pgfsetbuttcap%
\pgfsetroundjoin%
\pgfsetlinewidth{1.003750pt}%
\definecolor{currentstroke}{rgb}{1.000000,0.000000,0.000000}%
\pgfsetstrokecolor{currentstroke}%
\pgfsetdash{}{0pt}%
\pgfpathmoveto{\pgfqpoint{12.940652in}{8.318719in}}%
\pgfpathcurveto{\pgfqpoint{12.948888in}{8.318719in}}{\pgfqpoint{12.956788in}{8.321992in}}{\pgfqpoint{12.962612in}{8.327816in}}%
\pgfpathcurveto{\pgfqpoint{12.968436in}{8.333640in}}{\pgfqpoint{12.971708in}{8.341540in}}{\pgfqpoint{12.971708in}{8.349776in}}%
\pgfpathcurveto{\pgfqpoint{12.971708in}{8.358012in}}{\pgfqpoint{12.968436in}{8.365912in}}{\pgfqpoint{12.962612in}{8.371736in}}%
\pgfpathcurveto{\pgfqpoint{12.956788in}{8.377560in}}{\pgfqpoint{12.948888in}{8.380832in}}{\pgfqpoint{12.940652in}{8.380832in}}%
\pgfpathcurveto{\pgfqpoint{12.932415in}{8.380832in}}{\pgfqpoint{12.924515in}{8.377560in}}{\pgfqpoint{12.918691in}{8.371736in}}%
\pgfpathcurveto{\pgfqpoint{12.912867in}{8.365912in}}{\pgfqpoint{12.909595in}{8.358012in}}{\pgfqpoint{12.909595in}{8.349776in}}%
\pgfpathcurveto{\pgfqpoint{12.909595in}{8.341540in}}{\pgfqpoint{12.912867in}{8.333640in}}{\pgfqpoint{12.918691in}{8.327816in}}%
\pgfpathcurveto{\pgfqpoint{12.924515in}{8.321992in}}{\pgfqpoint{12.932415in}{8.318719in}}{\pgfqpoint{12.940652in}{8.318719in}}%
\pgfusepath{stroke}%
\end{pgfscope}%
\begin{pgfscope}%
\pgfpathrectangle{\pgfqpoint{3.788192in}{2.980138in}}{\pgfqpoint{2.914000in}{2.171400in}}%
\pgfusepath{clip}%
\pgfsetbuttcap%
\pgfsetroundjoin%
\pgfsetlinewidth{1.003750pt}%
\definecolor{currentstroke}{rgb}{1.000000,0.000000,0.000000}%
\pgfsetstrokecolor{currentstroke}%
\pgfsetdash{}{0pt}%
\pgfpathmoveto{\pgfqpoint{12.936873in}{8.298012in}}%
\pgfpathcurveto{\pgfqpoint{12.945109in}{8.298012in}}{\pgfqpoint{12.953009in}{8.301284in}}{\pgfqpoint{12.958833in}{8.307108in}}%
\pgfpathcurveto{\pgfqpoint{12.964657in}{8.312932in}}{\pgfqpoint{12.967929in}{8.320832in}}{\pgfqpoint{12.967929in}{8.329068in}}%
\pgfpathcurveto{\pgfqpoint{12.967929in}{8.337305in}}{\pgfqpoint{12.964657in}{8.345205in}}{\pgfqpoint{12.958833in}{8.351028in}}%
\pgfpathcurveto{\pgfqpoint{12.953009in}{8.356852in}}{\pgfqpoint{12.945109in}{8.360125in}}{\pgfqpoint{12.936873in}{8.360125in}}%
\pgfpathcurveto{\pgfqpoint{12.928637in}{8.360125in}}{\pgfqpoint{12.920737in}{8.356852in}}{\pgfqpoint{12.914913in}{8.351028in}}%
\pgfpathcurveto{\pgfqpoint{12.909089in}{8.345205in}}{\pgfqpoint{12.905816in}{8.337305in}}{\pgfqpoint{12.905816in}{8.329068in}}%
\pgfpathcurveto{\pgfqpoint{12.905816in}{8.320832in}}{\pgfqpoint{12.909089in}{8.312932in}}{\pgfqpoint{12.914913in}{8.307108in}}%
\pgfpathcurveto{\pgfqpoint{12.920737in}{8.301284in}}{\pgfqpoint{12.928637in}{8.298012in}}{\pgfqpoint{12.936873in}{8.298012in}}%
\pgfusepath{stroke}%
\end{pgfscope}%
\begin{pgfscope}%
\pgfpathrectangle{\pgfqpoint{3.788192in}{2.980138in}}{\pgfqpoint{2.914000in}{2.171400in}}%
\pgfusepath{clip}%
\pgfsetbuttcap%
\pgfsetroundjoin%
\pgfsetlinewidth{1.003750pt}%
\definecolor{currentstroke}{rgb}{1.000000,0.000000,0.000000}%
\pgfsetstrokecolor{currentstroke}%
\pgfsetdash{}{0pt}%
\pgfpathmoveto{\pgfqpoint{12.900060in}{6.352602in}}%
\pgfpathcurveto{\pgfqpoint{12.908296in}{6.352602in}}{\pgfqpoint{12.916196in}{6.355874in}}{\pgfqpoint{12.922020in}{6.361698in}}%
\pgfpathcurveto{\pgfqpoint{12.927844in}{6.367522in}}{\pgfqpoint{12.931116in}{6.375422in}}{\pgfqpoint{12.931116in}{6.383658in}}%
\pgfpathcurveto{\pgfqpoint{12.931116in}{6.391894in}}{\pgfqpoint{12.927844in}{6.399794in}}{\pgfqpoint{12.922020in}{6.405618in}}%
\pgfpathcurveto{\pgfqpoint{12.916196in}{6.411442in}}{\pgfqpoint{12.908296in}{6.414715in}}{\pgfqpoint{12.900060in}{6.414715in}}%
\pgfpathcurveto{\pgfqpoint{12.891823in}{6.414715in}}{\pgfqpoint{12.883923in}{6.411442in}}{\pgfqpoint{12.878099in}{6.405618in}}%
\pgfpathcurveto{\pgfqpoint{12.872275in}{6.399794in}}{\pgfqpoint{12.869003in}{6.391894in}}{\pgfqpoint{12.869003in}{6.383658in}}%
\pgfpathcurveto{\pgfqpoint{12.869003in}{6.375422in}}{\pgfqpoint{12.872275in}{6.367522in}}{\pgfqpoint{12.878099in}{6.361698in}}%
\pgfpathcurveto{\pgfqpoint{12.883923in}{6.355874in}}{\pgfqpoint{12.891823in}{6.352602in}}{\pgfqpoint{12.900060in}{6.352602in}}%
\pgfusepath{stroke}%
\end{pgfscope}%
\begin{pgfscope}%
\pgfpathrectangle{\pgfqpoint{3.788192in}{2.980138in}}{\pgfqpoint{2.914000in}{2.171400in}}%
\pgfusepath{clip}%
\pgfsetbuttcap%
\pgfsetroundjoin%
\pgfsetlinewidth{1.003750pt}%
\definecolor{currentstroke}{rgb}{1.000000,0.000000,0.000000}%
\pgfsetstrokecolor{currentstroke}%
\pgfsetdash{}{0pt}%
\pgfpathmoveto{\pgfqpoint{12.861075in}{6.367856in}}%
\pgfpathcurveto{\pgfqpoint{12.869311in}{6.367856in}}{\pgfqpoint{12.877211in}{6.371128in}}{\pgfqpoint{12.883035in}{6.376952in}}%
\pgfpathcurveto{\pgfqpoint{12.888859in}{6.382776in}}{\pgfqpoint{12.892131in}{6.390676in}}{\pgfqpoint{12.892131in}{6.398913in}}%
\pgfpathcurveto{\pgfqpoint{12.892131in}{6.407149in}}{\pgfqpoint{12.888859in}{6.415049in}}{\pgfqpoint{12.883035in}{6.420873in}}%
\pgfpathcurveto{\pgfqpoint{12.877211in}{6.426697in}}{\pgfqpoint{12.869311in}{6.429969in}}{\pgfqpoint{12.861075in}{6.429969in}}%
\pgfpathcurveto{\pgfqpoint{12.852838in}{6.429969in}}{\pgfqpoint{12.844938in}{6.426697in}}{\pgfqpoint{12.839114in}{6.420873in}}%
\pgfpathcurveto{\pgfqpoint{12.833291in}{6.415049in}}{\pgfqpoint{12.830018in}{6.407149in}}{\pgfqpoint{12.830018in}{6.398913in}}%
\pgfpathcurveto{\pgfqpoint{12.830018in}{6.390676in}}{\pgfqpoint{12.833291in}{6.382776in}}{\pgfqpoint{12.839114in}{6.376952in}}%
\pgfpathcurveto{\pgfqpoint{12.844938in}{6.371128in}}{\pgfqpoint{12.852838in}{6.367856in}}{\pgfqpoint{12.861075in}{6.367856in}}%
\pgfusepath{stroke}%
\end{pgfscope}%
\begin{pgfscope}%
\pgfpathrectangle{\pgfqpoint{3.788192in}{2.980138in}}{\pgfqpoint{2.914000in}{2.171400in}}%
\pgfusepath{clip}%
\pgfsetbuttcap%
\pgfsetroundjoin%
\pgfsetlinewidth{1.003750pt}%
\definecolor{currentstroke}{rgb}{1.000000,0.000000,0.000000}%
\pgfsetstrokecolor{currentstroke}%
\pgfsetdash{}{0pt}%
\pgfpathmoveto{\pgfqpoint{15.347266in}{8.054762in}}%
\pgfpathcurveto{\pgfqpoint{15.355502in}{8.054762in}}{\pgfqpoint{15.363402in}{8.058034in}}{\pgfqpoint{15.369226in}{8.063858in}}%
\pgfpathcurveto{\pgfqpoint{15.375050in}{8.069682in}}{\pgfqpoint{15.378323in}{8.077582in}}{\pgfqpoint{15.378323in}{8.085818in}}%
\pgfpathcurveto{\pgfqpoint{15.378323in}{8.094054in}}{\pgfqpoint{15.375050in}{8.101954in}}{\pgfqpoint{15.369226in}{8.107778in}}%
\pgfpathcurveto{\pgfqpoint{15.363402in}{8.113602in}}{\pgfqpoint{15.355502in}{8.116875in}}{\pgfqpoint{15.347266in}{8.116875in}}%
\pgfpathcurveto{\pgfqpoint{15.339030in}{8.116875in}}{\pgfqpoint{15.331130in}{8.113602in}}{\pgfqpoint{15.325306in}{8.107778in}}%
\pgfpathcurveto{\pgfqpoint{15.319482in}{8.101954in}}{\pgfqpoint{15.316210in}{8.094054in}}{\pgfqpoint{15.316210in}{8.085818in}}%
\pgfpathcurveto{\pgfqpoint{15.316210in}{8.077582in}}{\pgfqpoint{15.319482in}{8.069682in}}{\pgfqpoint{15.325306in}{8.063858in}}%
\pgfpathcurveto{\pgfqpoint{15.331130in}{8.058034in}}{\pgfqpoint{15.339030in}{8.054762in}}{\pgfqpoint{15.347266in}{8.054762in}}%
\pgfusepath{stroke}%
\end{pgfscope}%
\begin{pgfscope}%
\pgfpathrectangle{\pgfqpoint{3.788192in}{2.980138in}}{\pgfqpoint{2.914000in}{2.171400in}}%
\pgfusepath{clip}%
\pgfsetbuttcap%
\pgfsetroundjoin%
\pgfsetlinewidth{1.003750pt}%
\definecolor{currentstroke}{rgb}{1.000000,0.000000,0.000000}%
\pgfsetstrokecolor{currentstroke}%
\pgfsetdash{}{0pt}%
\pgfpathmoveto{\pgfqpoint{12.945472in}{6.264272in}}%
\pgfpathcurveto{\pgfqpoint{12.953708in}{6.264272in}}{\pgfqpoint{12.961608in}{6.267545in}}{\pgfqpoint{12.967432in}{6.273369in}}%
\pgfpathcurveto{\pgfqpoint{12.973256in}{6.279193in}}{\pgfqpoint{12.976528in}{6.287093in}}{\pgfqpoint{12.976528in}{6.295329in}}%
\pgfpathcurveto{\pgfqpoint{12.976528in}{6.303565in}}{\pgfqpoint{12.973256in}{6.311465in}}{\pgfqpoint{12.967432in}{6.317289in}}%
\pgfpathcurveto{\pgfqpoint{12.961608in}{6.323113in}}{\pgfqpoint{12.953708in}{6.326385in}}{\pgfqpoint{12.945472in}{6.326385in}}%
\pgfpathcurveto{\pgfqpoint{12.937236in}{6.326385in}}{\pgfqpoint{12.929336in}{6.323113in}}{\pgfqpoint{12.923512in}{6.317289in}}%
\pgfpathcurveto{\pgfqpoint{12.917688in}{6.311465in}}{\pgfqpoint{12.914415in}{6.303565in}}{\pgfqpoint{12.914415in}{6.295329in}}%
\pgfpathcurveto{\pgfqpoint{12.914415in}{6.287093in}}{\pgfqpoint{12.917688in}{6.279193in}}{\pgfqpoint{12.923512in}{6.273369in}}%
\pgfpathcurveto{\pgfqpoint{12.929336in}{6.267545in}}{\pgfqpoint{12.937236in}{6.264272in}}{\pgfqpoint{12.945472in}{6.264272in}}%
\pgfusepath{stroke}%
\end{pgfscope}%
\begin{pgfscope}%
\pgfpathrectangle{\pgfqpoint{3.788192in}{2.980138in}}{\pgfqpoint{2.914000in}{2.171400in}}%
\pgfusepath{clip}%
\pgfsetbuttcap%
\pgfsetroundjoin%
\pgfsetlinewidth{1.003750pt}%
\definecolor{currentstroke}{rgb}{1.000000,0.000000,0.000000}%
\pgfsetstrokecolor{currentstroke}%
\pgfsetdash{}{0pt}%
\pgfpathmoveto{\pgfqpoint{14.816780in}{6.837016in}}%
\pgfpathcurveto{\pgfqpoint{14.825016in}{6.837016in}}{\pgfqpoint{14.832916in}{6.840288in}}{\pgfqpoint{14.838740in}{6.846112in}}%
\pgfpathcurveto{\pgfqpoint{14.844564in}{6.851936in}}{\pgfqpoint{14.847836in}{6.859836in}}{\pgfqpoint{14.847836in}{6.868073in}}%
\pgfpathcurveto{\pgfqpoint{14.847836in}{6.876309in}}{\pgfqpoint{14.844564in}{6.884209in}}{\pgfqpoint{14.838740in}{6.890033in}}%
\pgfpathcurveto{\pgfqpoint{14.832916in}{6.895857in}}{\pgfqpoint{14.825016in}{6.899129in}}{\pgfqpoint{14.816780in}{6.899129in}}%
\pgfpathcurveto{\pgfqpoint{14.808543in}{6.899129in}}{\pgfqpoint{14.800643in}{6.895857in}}{\pgfqpoint{14.794819in}{6.890033in}}%
\pgfpathcurveto{\pgfqpoint{14.788995in}{6.884209in}}{\pgfqpoint{14.785723in}{6.876309in}}{\pgfqpoint{14.785723in}{6.868073in}}%
\pgfpathcurveto{\pgfqpoint{14.785723in}{6.859836in}}{\pgfqpoint{14.788995in}{6.851936in}}{\pgfqpoint{14.794819in}{6.846112in}}%
\pgfpathcurveto{\pgfqpoint{14.800643in}{6.840288in}}{\pgfqpoint{14.808543in}{6.837016in}}{\pgfqpoint{14.816780in}{6.837016in}}%
\pgfusepath{stroke}%
\end{pgfscope}%
\begin{pgfscope}%
\pgfpathrectangle{\pgfqpoint{3.788192in}{2.980138in}}{\pgfqpoint{2.914000in}{2.171400in}}%
\pgfusepath{clip}%
\pgfsetbuttcap%
\pgfsetroundjoin%
\pgfsetlinewidth{1.003750pt}%
\definecolor{currentstroke}{rgb}{1.000000,0.000000,0.000000}%
\pgfsetstrokecolor{currentstroke}%
\pgfsetdash{}{0pt}%
\pgfpathmoveto{\pgfqpoint{15.389681in}{6.512387in}}%
\pgfpathcurveto{\pgfqpoint{15.397917in}{6.512387in}}{\pgfqpoint{15.405817in}{6.515659in}}{\pgfqpoint{15.411641in}{6.521483in}}%
\pgfpathcurveto{\pgfqpoint{15.417465in}{6.527307in}}{\pgfqpoint{15.420738in}{6.535207in}}{\pgfqpoint{15.420738in}{6.543443in}}%
\pgfpathcurveto{\pgfqpoint{15.420738in}{6.551679in}}{\pgfqpoint{15.417465in}{6.559579in}}{\pgfqpoint{15.411641in}{6.565403in}}%
\pgfpathcurveto{\pgfqpoint{15.405817in}{6.571227in}}{\pgfqpoint{15.397917in}{6.574500in}}{\pgfqpoint{15.389681in}{6.574500in}}%
\pgfpathcurveto{\pgfqpoint{15.381445in}{6.574500in}}{\pgfqpoint{15.373545in}{6.571227in}}{\pgfqpoint{15.367721in}{6.565403in}}%
\pgfpathcurveto{\pgfqpoint{15.361897in}{6.559579in}}{\pgfqpoint{15.358625in}{6.551679in}}{\pgfqpoint{15.358625in}{6.543443in}}%
\pgfpathcurveto{\pgfqpoint{15.358625in}{6.535207in}}{\pgfqpoint{15.361897in}{6.527307in}}{\pgfqpoint{15.367721in}{6.521483in}}%
\pgfpathcurveto{\pgfqpoint{15.373545in}{6.515659in}}{\pgfqpoint{15.381445in}{6.512387in}}{\pgfqpoint{15.389681in}{6.512387in}}%
\pgfusepath{stroke}%
\end{pgfscope}%
\begin{pgfscope}%
\pgfpathrectangle{\pgfqpoint{3.788192in}{2.980138in}}{\pgfqpoint{2.914000in}{2.171400in}}%
\pgfusepath{clip}%
\pgfsetbuttcap%
\pgfsetroundjoin%
\pgfsetlinewidth{1.003750pt}%
\definecolor{currentstroke}{rgb}{1.000000,0.000000,0.000000}%
\pgfsetstrokecolor{currentstroke}%
\pgfsetdash{}{0pt}%
\pgfpathmoveto{\pgfqpoint{12.707501in}{7.095383in}}%
\pgfpathcurveto{\pgfqpoint{12.715737in}{7.095383in}}{\pgfqpoint{12.723637in}{7.098656in}}{\pgfqpoint{12.729461in}{7.104479in}}%
\pgfpathcurveto{\pgfqpoint{12.735285in}{7.110303in}}{\pgfqpoint{12.738557in}{7.118203in}}{\pgfqpoint{12.738557in}{7.126440in}}%
\pgfpathcurveto{\pgfqpoint{12.738557in}{7.134676in}}{\pgfqpoint{12.735285in}{7.142576in}}{\pgfqpoint{12.729461in}{7.148400in}}%
\pgfpathcurveto{\pgfqpoint{12.723637in}{7.154224in}}{\pgfqpoint{12.715737in}{7.157496in}}{\pgfqpoint{12.707501in}{7.157496in}}%
\pgfpathcurveto{\pgfqpoint{12.699265in}{7.157496in}}{\pgfqpoint{12.691365in}{7.154224in}}{\pgfqpoint{12.685541in}{7.148400in}}%
\pgfpathcurveto{\pgfqpoint{12.679717in}{7.142576in}}{\pgfqpoint{12.676444in}{7.134676in}}{\pgfqpoint{12.676444in}{7.126440in}}%
\pgfpathcurveto{\pgfqpoint{12.676444in}{7.118203in}}{\pgfqpoint{12.679717in}{7.110303in}}{\pgfqpoint{12.685541in}{7.104479in}}%
\pgfpathcurveto{\pgfqpoint{12.691365in}{7.098656in}}{\pgfqpoint{12.699265in}{7.095383in}}{\pgfqpoint{12.707501in}{7.095383in}}%
\pgfusepath{stroke}%
\end{pgfscope}%
\begin{pgfscope}%
\pgfpathrectangle{\pgfqpoint{3.788192in}{2.980138in}}{\pgfqpoint{2.914000in}{2.171400in}}%
\pgfusepath{clip}%
\pgfsetbuttcap%
\pgfsetroundjoin%
\pgfsetlinewidth{1.003750pt}%
\definecolor{currentstroke}{rgb}{1.000000,0.000000,0.000000}%
\pgfsetstrokecolor{currentstroke}%
\pgfsetdash{}{0pt}%
\pgfpathmoveto{\pgfqpoint{14.362942in}{5.990566in}}%
\pgfpathcurveto{\pgfqpoint{14.371178in}{5.990566in}}{\pgfqpoint{14.379078in}{5.993838in}}{\pgfqpoint{14.384902in}{5.999662in}}%
\pgfpathcurveto{\pgfqpoint{14.390726in}{6.005486in}}{\pgfqpoint{14.393998in}{6.013386in}}{\pgfqpoint{14.393998in}{6.021622in}}%
\pgfpathcurveto{\pgfqpoint{14.393998in}{6.029859in}}{\pgfqpoint{14.390726in}{6.037759in}}{\pgfqpoint{14.384902in}{6.043583in}}%
\pgfpathcurveto{\pgfqpoint{14.379078in}{6.049407in}}{\pgfqpoint{14.371178in}{6.052679in}}{\pgfqpoint{14.362942in}{6.052679in}}%
\pgfpathcurveto{\pgfqpoint{14.354705in}{6.052679in}}{\pgfqpoint{14.346805in}{6.049407in}}{\pgfqpoint{14.340982in}{6.043583in}}%
\pgfpathcurveto{\pgfqpoint{14.335158in}{6.037759in}}{\pgfqpoint{14.331885in}{6.029859in}}{\pgfqpoint{14.331885in}{6.021622in}}%
\pgfpathcurveto{\pgfqpoint{14.331885in}{6.013386in}}{\pgfqpoint{14.335158in}{6.005486in}}{\pgfqpoint{14.340982in}{5.999662in}}%
\pgfpathcurveto{\pgfqpoint{14.346805in}{5.993838in}}{\pgfqpoint{14.354705in}{5.990566in}}{\pgfqpoint{14.362942in}{5.990566in}}%
\pgfusepath{stroke}%
\end{pgfscope}%
\begin{pgfscope}%
\pgfpathrectangle{\pgfqpoint{3.788192in}{2.980138in}}{\pgfqpoint{2.914000in}{2.171400in}}%
\pgfusepath{clip}%
\pgfsetbuttcap%
\pgfsetroundjoin%
\pgfsetlinewidth{1.003750pt}%
\definecolor{currentstroke}{rgb}{1.000000,0.000000,0.000000}%
\pgfsetstrokecolor{currentstroke}%
\pgfsetdash{}{0pt}%
\pgfpathmoveto{\pgfqpoint{14.985061in}{5.541293in}}%
\pgfpathcurveto{\pgfqpoint{14.993297in}{5.541293in}}{\pgfqpoint{15.001197in}{5.544566in}}{\pgfqpoint{15.007021in}{5.550390in}}%
\pgfpathcurveto{\pgfqpoint{15.012845in}{5.556213in}}{\pgfqpoint{15.016118in}{5.564113in}}{\pgfqpoint{15.016118in}{5.572350in}}%
\pgfpathcurveto{\pgfqpoint{15.016118in}{5.580586in}}{\pgfqpoint{15.012845in}{5.588486in}}{\pgfqpoint{15.007021in}{5.594310in}}%
\pgfpathcurveto{\pgfqpoint{15.001197in}{5.600134in}}{\pgfqpoint{14.993297in}{5.603406in}}{\pgfqpoint{14.985061in}{5.603406in}}%
\pgfpathcurveto{\pgfqpoint{14.976825in}{5.603406in}}{\pgfqpoint{14.968925in}{5.600134in}}{\pgfqpoint{14.963101in}{5.594310in}}%
\pgfpathcurveto{\pgfqpoint{14.957277in}{5.588486in}}{\pgfqpoint{14.954005in}{5.580586in}}{\pgfqpoint{14.954005in}{5.572350in}}%
\pgfpathcurveto{\pgfqpoint{14.954005in}{5.564113in}}{\pgfqpoint{14.957277in}{5.556213in}}{\pgfqpoint{14.963101in}{5.550390in}}%
\pgfpathcurveto{\pgfqpoint{14.968925in}{5.544566in}}{\pgfqpoint{14.976825in}{5.541293in}}{\pgfqpoint{14.985061in}{5.541293in}}%
\pgfusepath{stroke}%
\end{pgfscope}%
\begin{pgfscope}%
\pgfpathrectangle{\pgfqpoint{3.788192in}{2.980138in}}{\pgfqpoint{2.914000in}{2.171400in}}%
\pgfusepath{clip}%
\pgfsetbuttcap%
\pgfsetroundjoin%
\pgfsetlinewidth{1.003750pt}%
\definecolor{currentstroke}{rgb}{1.000000,0.000000,0.000000}%
\pgfsetstrokecolor{currentstroke}%
\pgfsetdash{}{0pt}%
\pgfpathmoveto{\pgfqpoint{14.968190in}{5.733414in}}%
\pgfpathcurveto{\pgfqpoint{14.976426in}{5.733414in}}{\pgfqpoint{14.984326in}{5.736686in}}{\pgfqpoint{14.990150in}{5.742510in}}%
\pgfpathcurveto{\pgfqpoint{14.995974in}{5.748334in}}{\pgfqpoint{14.999246in}{5.756234in}}{\pgfqpoint{14.999246in}{5.764470in}}%
\pgfpathcurveto{\pgfqpoint{14.999246in}{5.772706in}}{\pgfqpoint{14.995974in}{5.780606in}}{\pgfqpoint{14.990150in}{5.786430in}}%
\pgfpathcurveto{\pgfqpoint{14.984326in}{5.792254in}}{\pgfqpoint{14.976426in}{5.795527in}}{\pgfqpoint{14.968190in}{5.795527in}}%
\pgfpathcurveto{\pgfqpoint{14.959954in}{5.795527in}}{\pgfqpoint{14.952053in}{5.792254in}}{\pgfqpoint{14.946230in}{5.786430in}}%
\pgfpathcurveto{\pgfqpoint{14.940406in}{5.780606in}}{\pgfqpoint{14.937133in}{5.772706in}}{\pgfqpoint{14.937133in}{5.764470in}}%
\pgfpathcurveto{\pgfqpoint{14.937133in}{5.756234in}}{\pgfqpoint{14.940406in}{5.748334in}}{\pgfqpoint{14.946230in}{5.742510in}}%
\pgfpathcurveto{\pgfqpoint{14.952053in}{5.736686in}}{\pgfqpoint{14.959954in}{5.733414in}}{\pgfqpoint{14.968190in}{5.733414in}}%
\pgfusepath{stroke}%
\end{pgfscope}%
\begin{pgfscope}%
\pgfpathrectangle{\pgfqpoint{3.788192in}{2.980138in}}{\pgfqpoint{2.914000in}{2.171400in}}%
\pgfusepath{clip}%
\pgfsetbuttcap%
\pgfsetroundjoin%
\pgfsetlinewidth{1.003750pt}%
\definecolor{currentstroke}{rgb}{1.000000,0.000000,0.000000}%
\pgfsetstrokecolor{currentstroke}%
\pgfsetdash{}{0pt}%
\pgfpathmoveto{\pgfqpoint{12.713917in}{7.026179in}}%
\pgfpathcurveto{\pgfqpoint{12.722154in}{7.026179in}}{\pgfqpoint{12.730054in}{7.029451in}}{\pgfqpoint{12.735878in}{7.035275in}}%
\pgfpathcurveto{\pgfqpoint{12.741701in}{7.041099in}}{\pgfqpoint{12.744974in}{7.048999in}}{\pgfqpoint{12.744974in}{7.057235in}}%
\pgfpathcurveto{\pgfqpoint{12.744974in}{7.065472in}}{\pgfqpoint{12.741701in}{7.073372in}}{\pgfqpoint{12.735878in}{7.079196in}}%
\pgfpathcurveto{\pgfqpoint{12.730054in}{7.085019in}}{\pgfqpoint{12.722154in}{7.088292in}}{\pgfqpoint{12.713917in}{7.088292in}}%
\pgfpathcurveto{\pgfqpoint{12.705681in}{7.088292in}}{\pgfqpoint{12.697781in}{7.085019in}}{\pgfqpoint{12.691957in}{7.079196in}}%
\pgfpathcurveto{\pgfqpoint{12.686133in}{7.073372in}}{\pgfqpoint{12.682861in}{7.065472in}}{\pgfqpoint{12.682861in}{7.057235in}}%
\pgfpathcurveto{\pgfqpoint{12.682861in}{7.048999in}}{\pgfqpoint{12.686133in}{7.041099in}}{\pgfqpoint{12.691957in}{7.035275in}}%
\pgfpathcurveto{\pgfqpoint{12.697781in}{7.029451in}}{\pgfqpoint{12.705681in}{7.026179in}}{\pgfqpoint{12.713917in}{7.026179in}}%
\pgfusepath{stroke}%
\end{pgfscope}%
\begin{pgfscope}%
\pgfpathrectangle{\pgfqpoint{3.788192in}{2.980138in}}{\pgfqpoint{2.914000in}{2.171400in}}%
\pgfusepath{clip}%
\pgfsetbuttcap%
\pgfsetroundjoin%
\pgfsetlinewidth{1.003750pt}%
\definecolor{currentstroke}{rgb}{1.000000,0.000000,0.000000}%
\pgfsetstrokecolor{currentstroke}%
\pgfsetdash{}{0pt}%
\pgfpathmoveto{\pgfqpoint{13.558784in}{8.054051in}}%
\pgfpathcurveto{\pgfqpoint{13.567020in}{8.054051in}}{\pgfqpoint{13.574920in}{8.057323in}}{\pgfqpoint{13.580744in}{8.063147in}}%
\pgfpathcurveto{\pgfqpoint{13.586568in}{8.068971in}}{\pgfqpoint{13.589840in}{8.076871in}}{\pgfqpoint{13.589840in}{8.085108in}}%
\pgfpathcurveto{\pgfqpoint{13.589840in}{8.093344in}}{\pgfqpoint{13.586568in}{8.101244in}}{\pgfqpoint{13.580744in}{8.107068in}}%
\pgfpathcurveto{\pgfqpoint{13.574920in}{8.112892in}}{\pgfqpoint{13.567020in}{8.116164in}}{\pgfqpoint{13.558784in}{8.116164in}}%
\pgfpathcurveto{\pgfqpoint{13.550548in}{8.116164in}}{\pgfqpoint{13.542648in}{8.112892in}}{\pgfqpoint{13.536824in}{8.107068in}}%
\pgfpathcurveto{\pgfqpoint{13.531000in}{8.101244in}}{\pgfqpoint{13.527727in}{8.093344in}}{\pgfqpoint{13.527727in}{8.085108in}}%
\pgfpathcurveto{\pgfqpoint{13.527727in}{8.076871in}}{\pgfqpoint{13.531000in}{8.068971in}}{\pgfqpoint{13.536824in}{8.063147in}}%
\pgfpathcurveto{\pgfqpoint{13.542648in}{8.057323in}}{\pgfqpoint{13.550548in}{8.054051in}}{\pgfqpoint{13.558784in}{8.054051in}}%
\pgfusepath{stroke}%
\end{pgfscope}%
\begin{pgfscope}%
\pgfpathrectangle{\pgfqpoint{3.788192in}{2.980138in}}{\pgfqpoint{2.914000in}{2.171400in}}%
\pgfusepath{clip}%
\pgfsetbuttcap%
\pgfsetroundjoin%
\pgfsetlinewidth{1.003750pt}%
\definecolor{currentstroke}{rgb}{1.000000,0.000000,0.000000}%
\pgfsetstrokecolor{currentstroke}%
\pgfsetdash{}{0pt}%
\pgfpathmoveto{\pgfqpoint{13.021061in}{8.119261in}}%
\pgfpathcurveto{\pgfqpoint{13.029297in}{8.119261in}}{\pgfqpoint{13.037197in}{8.122533in}}{\pgfqpoint{13.043021in}{8.128357in}}%
\pgfpathcurveto{\pgfqpoint{13.048845in}{8.134181in}}{\pgfqpoint{13.052117in}{8.142081in}}{\pgfqpoint{13.052117in}{8.150317in}}%
\pgfpathcurveto{\pgfqpoint{13.052117in}{8.158554in}}{\pgfqpoint{13.048845in}{8.166454in}}{\pgfqpoint{13.043021in}{8.172278in}}%
\pgfpathcurveto{\pgfqpoint{13.037197in}{8.178102in}}{\pgfqpoint{13.029297in}{8.181374in}}{\pgfqpoint{13.021061in}{8.181374in}}%
\pgfpathcurveto{\pgfqpoint{13.012825in}{8.181374in}}{\pgfqpoint{13.004925in}{8.178102in}}{\pgfqpoint{12.999101in}{8.172278in}}%
\pgfpathcurveto{\pgfqpoint{12.993277in}{8.166454in}}{\pgfqpoint{12.990004in}{8.158554in}}{\pgfqpoint{12.990004in}{8.150317in}}%
\pgfpathcurveto{\pgfqpoint{12.990004in}{8.142081in}}{\pgfqpoint{12.993277in}{8.134181in}}{\pgfqpoint{12.999101in}{8.128357in}}%
\pgfpathcurveto{\pgfqpoint{13.004925in}{8.122533in}}{\pgfqpoint{13.012825in}{8.119261in}}{\pgfqpoint{13.021061in}{8.119261in}}%
\pgfusepath{stroke}%
\end{pgfscope}%
\begin{pgfscope}%
\pgfpathrectangle{\pgfqpoint{3.788192in}{2.980138in}}{\pgfqpoint{2.914000in}{2.171400in}}%
\pgfusepath{clip}%
\pgfsetbuttcap%
\pgfsetroundjoin%
\pgfsetlinewidth{1.003750pt}%
\definecolor{currentstroke}{rgb}{1.000000,0.000000,0.000000}%
\pgfsetstrokecolor{currentstroke}%
\pgfsetdash{}{0pt}%
\pgfpathmoveto{\pgfqpoint{12.649080in}{7.078889in}}%
\pgfpathcurveto{\pgfqpoint{12.657316in}{7.078889in}}{\pgfqpoint{12.665216in}{7.082161in}}{\pgfqpoint{12.671040in}{7.087985in}}%
\pgfpathcurveto{\pgfqpoint{12.676864in}{7.093809in}}{\pgfqpoint{12.680136in}{7.101709in}}{\pgfqpoint{12.680136in}{7.109946in}}%
\pgfpathcurveto{\pgfqpoint{12.680136in}{7.118182in}}{\pgfqpoint{12.676864in}{7.126082in}}{\pgfqpoint{12.671040in}{7.131906in}}%
\pgfpathcurveto{\pgfqpoint{12.665216in}{7.137730in}}{\pgfqpoint{12.657316in}{7.141002in}}{\pgfqpoint{12.649080in}{7.141002in}}%
\pgfpathcurveto{\pgfqpoint{12.640843in}{7.141002in}}{\pgfqpoint{12.632943in}{7.137730in}}{\pgfqpoint{12.627119in}{7.131906in}}%
\pgfpathcurveto{\pgfqpoint{12.621295in}{7.126082in}}{\pgfqpoint{12.618023in}{7.118182in}}{\pgfqpoint{12.618023in}{7.109946in}}%
\pgfpathcurveto{\pgfqpoint{12.618023in}{7.101709in}}{\pgfqpoint{12.621295in}{7.093809in}}{\pgfqpoint{12.627119in}{7.087985in}}%
\pgfpathcurveto{\pgfqpoint{12.632943in}{7.082161in}}{\pgfqpoint{12.640843in}{7.078889in}}{\pgfqpoint{12.649080in}{7.078889in}}%
\pgfusepath{stroke}%
\end{pgfscope}%
\begin{pgfscope}%
\pgfpathrectangle{\pgfqpoint{3.788192in}{2.980138in}}{\pgfqpoint{2.914000in}{2.171400in}}%
\pgfusepath{clip}%
\pgfsetbuttcap%
\pgfsetroundjoin%
\pgfsetlinewidth{1.003750pt}%
\definecolor{currentstroke}{rgb}{1.000000,0.000000,0.000000}%
\pgfsetstrokecolor{currentstroke}%
\pgfsetdash{}{0pt}%
\pgfpathmoveto{\pgfqpoint{4.530215in}{4.056056in}}%
\pgfpathcurveto{\pgfqpoint{4.538452in}{4.056056in}}{\pgfqpoint{4.546352in}{4.059328in}}{\pgfqpoint{4.552176in}{4.065152in}}%
\pgfpathcurveto{\pgfqpoint{4.557999in}{4.070976in}}{\pgfqpoint{4.561272in}{4.078876in}}{\pgfqpoint{4.561272in}{4.087112in}}%
\pgfpathcurveto{\pgfqpoint{4.561272in}{4.095349in}}{\pgfqpoint{4.557999in}{4.103249in}}{\pgfqpoint{4.552176in}{4.109073in}}%
\pgfpathcurveto{\pgfqpoint{4.546352in}{4.114897in}}{\pgfqpoint{4.538452in}{4.118169in}}{\pgfqpoint{4.530215in}{4.118169in}}%
\pgfpathcurveto{\pgfqpoint{4.521979in}{4.118169in}}{\pgfqpoint{4.514079in}{4.114897in}}{\pgfqpoint{4.508255in}{4.109073in}}%
\pgfpathcurveto{\pgfqpoint{4.502431in}{4.103249in}}{\pgfqpoint{4.499159in}{4.095349in}}{\pgfqpoint{4.499159in}{4.087112in}}%
\pgfpathcurveto{\pgfqpoint{4.499159in}{4.078876in}}{\pgfqpoint{4.502431in}{4.070976in}}{\pgfqpoint{4.508255in}{4.065152in}}%
\pgfpathcurveto{\pgfqpoint{4.514079in}{4.059328in}}{\pgfqpoint{4.521979in}{4.056056in}}{\pgfqpoint{4.530215in}{4.056056in}}%
\pgfpathlineto{\pgfqpoint{4.530215in}{4.056056in}}%
\pgfpathclose%
\pgfusepath{stroke}%
\end{pgfscope}%
\begin{pgfscope}%
\pgfpathrectangle{\pgfqpoint{3.788192in}{2.980138in}}{\pgfqpoint{2.914000in}{2.171400in}}%
\pgfusepath{clip}%
\pgfsetbuttcap%
\pgfsetmiterjoin%
\definecolor{currentfill}{rgb}{0.839216,0.152941,0.156863}%
\pgfsetfillcolor{currentfill}%
\pgfsetfillopacity{0.200000}%
\pgfsetlinewidth{1.003750pt}%
\definecolor{currentstroke}{rgb}{0.839216,0.152941,0.156863}%
\pgfsetstrokecolor{currentstroke}%
\pgfsetstrokeopacity{0.200000}%
\pgfsetdash{}{0pt}%
\pgfpathmoveto{\pgfqpoint{4.530215in}{2.980138in}}%
\pgfpathlineto{\pgfqpoint{24.162152in}{2.980138in}}%
\pgfpathlineto{\pgfqpoint{24.162152in}{5.151538in}}%
\pgfpathlineto{\pgfqpoint{4.530215in}{5.151538in}}%
\pgfpathlineto{\pgfqpoint{4.530215in}{2.980138in}}%
\pgfpathclose%
\pgfusepath{stroke,fill}%
\end{pgfscope}%
\begin{pgfscope}%
\pgfsetbuttcap%
\pgfsetmiterjoin%
\definecolor{currentfill}{rgb}{0.839216,0.152941,0.156863}%
\pgfsetfillcolor{currentfill}%
\pgfsetfillopacity{0.200000}%
\pgfsetlinewidth{1.003750pt}%
\definecolor{currentstroke}{rgb}{0.839216,0.152941,0.156863}%
\pgfsetstrokecolor{currentstroke}%
\pgfsetstrokeopacity{0.200000}%
\pgfsetdash{}{0pt}%
\pgfpathrectangle{\pgfqpoint{3.788192in}{2.980138in}}{\pgfqpoint{2.914000in}{2.171400in}}%
\pgfusepath{clip}%
\pgfpathmoveto{\pgfqpoint{4.530215in}{2.980138in}}%
\pgfpathlineto{\pgfqpoint{24.162152in}{2.980138in}}%
\pgfpathlineto{\pgfqpoint{24.162152in}{5.151538in}}%
\pgfpathlineto{\pgfqpoint{4.530215in}{5.151538in}}%
\pgfpathlineto{\pgfqpoint{4.530215in}{2.980138in}}%
\pgfpathclose%
\pgfusepath{clip}%
\pgfsys@defobject{currentpattern}{\pgfqpoint{0in}{0in}}{\pgfqpoint{1in}{1in}}{%
\begin{pgfscope}%
\pgfpathrectangle{\pgfqpoint{0in}{0in}}{\pgfqpoint{1in}{1in}}%
\pgfusepath{clip}%
\pgfpathmoveto{\pgfqpoint{-0.500000in}{0.500000in}}%
\pgfpathlineto{\pgfqpoint{0.500000in}{1.500000in}}%
\pgfpathmoveto{\pgfqpoint{-0.333333in}{0.333333in}}%
\pgfpathlineto{\pgfqpoint{0.666667in}{1.333333in}}%
\pgfpathmoveto{\pgfqpoint{-0.166667in}{0.166667in}}%
\pgfpathlineto{\pgfqpoint{0.833333in}{1.166667in}}%
\pgfpathmoveto{\pgfqpoint{0.000000in}{0.000000in}}%
\pgfpathlineto{\pgfqpoint{1.000000in}{1.000000in}}%
\pgfpathmoveto{\pgfqpoint{0.166667in}{-0.166667in}}%
\pgfpathlineto{\pgfqpoint{1.166667in}{0.833333in}}%
\pgfpathmoveto{\pgfqpoint{0.333333in}{-0.333333in}}%
\pgfpathlineto{\pgfqpoint{1.333333in}{0.666667in}}%
\pgfpathmoveto{\pgfqpoint{0.500000in}{-0.500000in}}%
\pgfpathlineto{\pgfqpoint{1.500000in}{0.500000in}}%
\pgfusepath{stroke}%
\end{pgfscope}%
}%
\pgfsys@transformshift{4.530215in}{2.980138in}%
\pgfsys@useobject{currentpattern}{}%
\pgfsys@transformshift{1in}{0in}%
\pgfsys@useobject{currentpattern}{}%
\pgfsys@transformshift{1in}{0in}%
\pgfsys@useobject{currentpattern}{}%
\pgfsys@transformshift{1in}{0in}%
\pgfsys@useobject{currentpattern}{}%
\pgfsys@transformshift{1in}{0in}%
\pgfsys@useobject{currentpattern}{}%
\pgfsys@transformshift{1in}{0in}%
\pgfsys@useobject{currentpattern}{}%
\pgfsys@transformshift{1in}{0in}%
\pgfsys@useobject{currentpattern}{}%
\pgfsys@transformshift{1in}{0in}%
\pgfsys@useobject{currentpattern}{}%
\pgfsys@transformshift{1in}{0in}%
\pgfsys@useobject{currentpattern}{}%
\pgfsys@transformshift{1in}{0in}%
\pgfsys@useobject{currentpattern}{}%
\pgfsys@transformshift{1in}{0in}%
\pgfsys@useobject{currentpattern}{}%
\pgfsys@transformshift{1in}{0in}%
\pgfsys@useobject{currentpattern}{}%
\pgfsys@transformshift{1in}{0in}%
\pgfsys@useobject{currentpattern}{}%
\pgfsys@transformshift{1in}{0in}%
\pgfsys@useobject{currentpattern}{}%
\pgfsys@transformshift{1in}{0in}%
\pgfsys@useobject{currentpattern}{}%
\pgfsys@transformshift{1in}{0in}%
\pgfsys@useobject{currentpattern}{}%
\pgfsys@transformshift{1in}{0in}%
\pgfsys@useobject{currentpattern}{}%
\pgfsys@transformshift{1in}{0in}%
\pgfsys@useobject{currentpattern}{}%
\pgfsys@transformshift{1in}{0in}%
\pgfsys@useobject{currentpattern}{}%
\pgfsys@transformshift{1in}{0in}%
\pgfsys@useobject{currentpattern}{}%
\pgfsys@transformshift{1in}{0in}%
\pgfsys@transformshift{-20in}{0in}%
\pgfsys@transformshift{0in}{1in}%
\pgfsys@useobject{currentpattern}{}%
\pgfsys@transformshift{1in}{0in}%
\pgfsys@useobject{currentpattern}{}%
\pgfsys@transformshift{1in}{0in}%
\pgfsys@useobject{currentpattern}{}%
\pgfsys@transformshift{1in}{0in}%
\pgfsys@useobject{currentpattern}{}%
\pgfsys@transformshift{1in}{0in}%
\pgfsys@useobject{currentpattern}{}%
\pgfsys@transformshift{1in}{0in}%
\pgfsys@useobject{currentpattern}{}%
\pgfsys@transformshift{1in}{0in}%
\pgfsys@useobject{currentpattern}{}%
\pgfsys@transformshift{1in}{0in}%
\pgfsys@useobject{currentpattern}{}%
\pgfsys@transformshift{1in}{0in}%
\pgfsys@useobject{currentpattern}{}%
\pgfsys@transformshift{1in}{0in}%
\pgfsys@useobject{currentpattern}{}%
\pgfsys@transformshift{1in}{0in}%
\pgfsys@useobject{currentpattern}{}%
\pgfsys@transformshift{1in}{0in}%
\pgfsys@useobject{currentpattern}{}%
\pgfsys@transformshift{1in}{0in}%
\pgfsys@useobject{currentpattern}{}%
\pgfsys@transformshift{1in}{0in}%
\pgfsys@useobject{currentpattern}{}%
\pgfsys@transformshift{1in}{0in}%
\pgfsys@useobject{currentpattern}{}%
\pgfsys@transformshift{1in}{0in}%
\pgfsys@useobject{currentpattern}{}%
\pgfsys@transformshift{1in}{0in}%
\pgfsys@useobject{currentpattern}{}%
\pgfsys@transformshift{1in}{0in}%
\pgfsys@useobject{currentpattern}{}%
\pgfsys@transformshift{1in}{0in}%
\pgfsys@useobject{currentpattern}{}%
\pgfsys@transformshift{1in}{0in}%
\pgfsys@useobject{currentpattern}{}%
\pgfsys@transformshift{1in}{0in}%
\pgfsys@transformshift{-20in}{0in}%
\pgfsys@transformshift{0in}{1in}%
\pgfsys@useobject{currentpattern}{}%
\pgfsys@transformshift{1in}{0in}%
\pgfsys@useobject{currentpattern}{}%
\pgfsys@transformshift{1in}{0in}%
\pgfsys@useobject{currentpattern}{}%
\pgfsys@transformshift{1in}{0in}%
\pgfsys@useobject{currentpattern}{}%
\pgfsys@transformshift{1in}{0in}%
\pgfsys@useobject{currentpattern}{}%
\pgfsys@transformshift{1in}{0in}%
\pgfsys@useobject{currentpattern}{}%
\pgfsys@transformshift{1in}{0in}%
\pgfsys@useobject{currentpattern}{}%
\pgfsys@transformshift{1in}{0in}%
\pgfsys@useobject{currentpattern}{}%
\pgfsys@transformshift{1in}{0in}%
\pgfsys@useobject{currentpattern}{}%
\pgfsys@transformshift{1in}{0in}%
\pgfsys@useobject{currentpattern}{}%
\pgfsys@transformshift{1in}{0in}%
\pgfsys@useobject{currentpattern}{}%
\pgfsys@transformshift{1in}{0in}%
\pgfsys@useobject{currentpattern}{}%
\pgfsys@transformshift{1in}{0in}%
\pgfsys@useobject{currentpattern}{}%
\pgfsys@transformshift{1in}{0in}%
\pgfsys@useobject{currentpattern}{}%
\pgfsys@transformshift{1in}{0in}%
\pgfsys@useobject{currentpattern}{}%
\pgfsys@transformshift{1in}{0in}%
\pgfsys@useobject{currentpattern}{}%
\pgfsys@transformshift{1in}{0in}%
\pgfsys@useobject{currentpattern}{}%
\pgfsys@transformshift{1in}{0in}%
\pgfsys@useobject{currentpattern}{}%
\pgfsys@transformshift{1in}{0in}%
\pgfsys@useobject{currentpattern}{}%
\pgfsys@transformshift{1in}{0in}%
\pgfsys@useobject{currentpattern}{}%
\pgfsys@transformshift{1in}{0in}%
\pgfsys@transformshift{-20in}{0in}%
\pgfsys@transformshift{0in}{1in}%
\end{pgfscope}%
\begin{pgfscope}%
\pgfpathrectangle{\pgfqpoint{3.788192in}{2.980138in}}{\pgfqpoint{2.914000in}{2.171400in}}%
\pgfusepath{clip}%
\pgfsetrectcap%
\pgfsetroundjoin%
\pgfsetlinewidth{0.803000pt}%
\definecolor{currentstroke}{rgb}{0.690196,0.690196,0.690196}%
\pgfsetstrokecolor{currentstroke}%
\pgfsetdash{}{0pt}%
\pgfpathmoveto{\pgfqpoint{4.105109in}{2.980138in}}%
\pgfpathlineto{\pgfqpoint{4.105109in}{5.151538in}}%
\pgfusepath{stroke}%
\end{pgfscope}%
\begin{pgfscope}%
\pgfsetbuttcap%
\pgfsetroundjoin%
\definecolor{currentfill}{rgb}{0.000000,0.000000,0.000000}%
\pgfsetfillcolor{currentfill}%
\pgfsetlinewidth{0.803000pt}%
\definecolor{currentstroke}{rgb}{0.000000,0.000000,0.000000}%
\pgfsetstrokecolor{currentstroke}%
\pgfsetdash{}{0pt}%
\pgfsys@defobject{currentmarker}{\pgfqpoint{0.000000in}{-0.048611in}}{\pgfqpoint{0.000000in}{0.000000in}}{%
\pgfpathmoveto{\pgfqpoint{0.000000in}{0.000000in}}%
\pgfpathlineto{\pgfqpoint{0.000000in}{-0.048611in}}%
\pgfusepath{stroke,fill}%
}%
\begin{pgfscope}%
\pgfsys@transformshift{4.105109in}{2.980138in}%
\pgfsys@useobject{currentmarker}{}%
\end{pgfscope}%
\end{pgfscope}%
\begin{pgfscope}%
\definecolor{textcolor}{rgb}{0.000000,0.000000,0.000000}%
\pgfsetstrokecolor{textcolor}%
\pgfsetfillcolor{textcolor}%
\pgftext[x=4.105109in,y=2.882916in,,top]{\color{textcolor}{\rmfamily\fontsize{14.000000}{16.800000}\selectfont\catcode`\^=\active\def^{\ifmmode\sp\else\^{}\fi}\catcode`\%=\active\def%{\%}$\mathdefault{5280}$}}%
\end{pgfscope}%
\begin{pgfscope}%
\pgfpathrectangle{\pgfqpoint{3.788192in}{2.980138in}}{\pgfqpoint{2.914000in}{2.171400in}}%
\pgfusepath{clip}%
\pgfsetrectcap%
\pgfsetroundjoin%
\pgfsetlinewidth{0.803000pt}%
\definecolor{currentstroke}{rgb}{0.690196,0.690196,0.690196}%
\pgfsetstrokecolor{currentstroke}%
\pgfsetdash{}{0pt}%
\pgfpathmoveto{\pgfqpoint{4.847132in}{2.980138in}}%
\pgfpathlineto{\pgfqpoint{4.847132in}{5.151538in}}%
\pgfusepath{stroke}%
\end{pgfscope}%
\begin{pgfscope}%
\pgfsetbuttcap%
\pgfsetroundjoin%
\definecolor{currentfill}{rgb}{0.000000,0.000000,0.000000}%
\pgfsetfillcolor{currentfill}%
\pgfsetlinewidth{0.803000pt}%
\definecolor{currentstroke}{rgb}{0.000000,0.000000,0.000000}%
\pgfsetstrokecolor{currentstroke}%
\pgfsetdash{}{0pt}%
\pgfsys@defobject{currentmarker}{\pgfqpoint{0.000000in}{-0.048611in}}{\pgfqpoint{0.000000in}{0.000000in}}{%
\pgfpathmoveto{\pgfqpoint{0.000000in}{0.000000in}}%
\pgfpathlineto{\pgfqpoint{0.000000in}{-0.048611in}}%
\pgfusepath{stroke,fill}%
}%
\begin{pgfscope}%
\pgfsys@transformshift{4.847132in}{2.980138in}%
\pgfsys@useobject{currentmarker}{}%
\end{pgfscope}%
\end{pgfscope}%
\begin{pgfscope}%
\definecolor{textcolor}{rgb}{0.000000,0.000000,0.000000}%
\pgfsetstrokecolor{textcolor}%
\pgfsetfillcolor{textcolor}%
\pgftext[x=4.847132in,y=2.882916in,,top]{\color{textcolor}{\rmfamily\fontsize{14.000000}{16.800000}\selectfont\catcode`\^=\active\def^{\ifmmode\sp\else\^{}\fi}\catcode`\%=\active\def%{\%}$\mathdefault{5300}$}}%
\end{pgfscope}%
\begin{pgfscope}%
\pgfpathrectangle{\pgfqpoint{3.788192in}{2.980138in}}{\pgfqpoint{2.914000in}{2.171400in}}%
\pgfusepath{clip}%
\pgfsetrectcap%
\pgfsetroundjoin%
\pgfsetlinewidth{0.803000pt}%
\definecolor{currentstroke}{rgb}{0.690196,0.690196,0.690196}%
\pgfsetstrokecolor{currentstroke}%
\pgfsetdash{}{0pt}%
\pgfpathmoveto{\pgfqpoint{5.589156in}{2.980138in}}%
\pgfpathlineto{\pgfqpoint{5.589156in}{5.151538in}}%
\pgfusepath{stroke}%
\end{pgfscope}%
\begin{pgfscope}%
\pgfsetbuttcap%
\pgfsetroundjoin%
\definecolor{currentfill}{rgb}{0.000000,0.000000,0.000000}%
\pgfsetfillcolor{currentfill}%
\pgfsetlinewidth{0.803000pt}%
\definecolor{currentstroke}{rgb}{0.000000,0.000000,0.000000}%
\pgfsetstrokecolor{currentstroke}%
\pgfsetdash{}{0pt}%
\pgfsys@defobject{currentmarker}{\pgfqpoint{0.000000in}{-0.048611in}}{\pgfqpoint{0.000000in}{0.000000in}}{%
\pgfpathmoveto{\pgfqpoint{0.000000in}{0.000000in}}%
\pgfpathlineto{\pgfqpoint{0.000000in}{-0.048611in}}%
\pgfusepath{stroke,fill}%
}%
\begin{pgfscope}%
\pgfsys@transformshift{5.589156in}{2.980138in}%
\pgfsys@useobject{currentmarker}{}%
\end{pgfscope}%
\end{pgfscope}%
\begin{pgfscope}%
\definecolor{textcolor}{rgb}{0.000000,0.000000,0.000000}%
\pgfsetstrokecolor{textcolor}%
\pgfsetfillcolor{textcolor}%
\pgftext[x=5.589156in,y=2.882916in,,top]{\color{textcolor}{\rmfamily\fontsize{14.000000}{16.800000}\selectfont\catcode`\^=\active\def^{\ifmmode\sp\else\^{}\fi}\catcode`\%=\active\def%{\%}$\mathdefault{5320}$}}%
\end{pgfscope}%
\begin{pgfscope}%
\pgfpathrectangle{\pgfqpoint{3.788192in}{2.980138in}}{\pgfqpoint{2.914000in}{2.171400in}}%
\pgfusepath{clip}%
\pgfsetrectcap%
\pgfsetroundjoin%
\pgfsetlinewidth{0.803000pt}%
\definecolor{currentstroke}{rgb}{0.690196,0.690196,0.690196}%
\pgfsetstrokecolor{currentstroke}%
\pgfsetdash{}{0pt}%
\pgfpathmoveto{\pgfqpoint{6.331180in}{2.980138in}}%
\pgfpathlineto{\pgfqpoint{6.331180in}{5.151538in}}%
\pgfusepath{stroke}%
\end{pgfscope}%
\begin{pgfscope}%
\pgfsetbuttcap%
\pgfsetroundjoin%
\definecolor{currentfill}{rgb}{0.000000,0.000000,0.000000}%
\pgfsetfillcolor{currentfill}%
\pgfsetlinewidth{0.803000pt}%
\definecolor{currentstroke}{rgb}{0.000000,0.000000,0.000000}%
\pgfsetstrokecolor{currentstroke}%
\pgfsetdash{}{0pt}%
\pgfsys@defobject{currentmarker}{\pgfqpoint{0.000000in}{-0.048611in}}{\pgfqpoint{0.000000in}{0.000000in}}{%
\pgfpathmoveto{\pgfqpoint{0.000000in}{0.000000in}}%
\pgfpathlineto{\pgfqpoint{0.000000in}{-0.048611in}}%
\pgfusepath{stroke,fill}%
}%
\begin{pgfscope}%
\pgfsys@transformshift{6.331180in}{2.980138in}%
\pgfsys@useobject{currentmarker}{}%
\end{pgfscope}%
\end{pgfscope}%
\begin{pgfscope}%
\definecolor{textcolor}{rgb}{0.000000,0.000000,0.000000}%
\pgfsetstrokecolor{textcolor}%
\pgfsetfillcolor{textcolor}%
\pgftext[x=6.331180in,y=2.882916in,,top]{\color{textcolor}{\rmfamily\fontsize{14.000000}{16.800000}\selectfont\catcode`\^=\active\def^{\ifmmode\sp\else\^{}\fi}\catcode`\%=\active\def%{\%}$\mathdefault{5340}$}}%
\end{pgfscope}%
\begin{pgfscope}%
\pgfpathrectangle{\pgfqpoint{3.788192in}{2.980138in}}{\pgfqpoint{2.914000in}{2.171400in}}%
\pgfusepath{clip}%
\pgfsetrectcap%
\pgfsetroundjoin%
\pgfsetlinewidth{0.803000pt}%
\definecolor{currentstroke}{rgb}{0.690196,0.690196,0.690196}%
\pgfsetstrokecolor{currentstroke}%
\pgfsetdash{}{0pt}%
\pgfpathmoveto{\pgfqpoint{3.788192in}{3.334947in}}%
\pgfpathlineto{\pgfqpoint{6.702192in}{3.334947in}}%
\pgfusepath{stroke}%
\end{pgfscope}%
\begin{pgfscope}%
\pgfsetbuttcap%
\pgfsetroundjoin%
\definecolor{currentfill}{rgb}{0.000000,0.000000,0.000000}%
\pgfsetfillcolor{currentfill}%
\pgfsetlinewidth{0.803000pt}%
\definecolor{currentstroke}{rgb}{0.000000,0.000000,0.000000}%
\pgfsetstrokecolor{currentstroke}%
\pgfsetdash{}{0pt}%
\pgfsys@defobject{currentmarker}{\pgfqpoint{-0.048611in}{0.000000in}}{\pgfqpoint{-0.000000in}{0.000000in}}{%
\pgfpathmoveto{\pgfqpoint{-0.000000in}{0.000000in}}%
\pgfpathlineto{\pgfqpoint{-0.048611in}{0.000000in}}%
\pgfusepath{stroke,fill}%
}%
\begin{pgfscope}%
\pgfsys@transformshift{3.788192in}{3.334947in}%
\pgfsys@useobject{currentmarker}{}%
\end{pgfscope}%
\end{pgfscope}%
\begin{pgfscope}%
\definecolor{textcolor}{rgb}{0.000000,0.000000,0.000000}%
\pgfsetstrokecolor{textcolor}%
\pgfsetfillcolor{textcolor}%
\pgftext[x=3.495138in, y=3.265502in, left, base]{\color{textcolor}{\rmfamily\fontsize{14.000000}{16.800000}\selectfont\catcode`\^=\active\def^{\ifmmode\sp\else\^{}\fi}\catcode`\%=\active\def%{\%}$\mathdefault{10}$}}%
\end{pgfscope}%
\begin{pgfscope}%
\pgfpathrectangle{\pgfqpoint{3.788192in}{2.980138in}}{\pgfqpoint{2.914000in}{2.171400in}}%
\pgfusepath{clip}%
\pgfsetrectcap%
\pgfsetroundjoin%
\pgfsetlinewidth{0.803000pt}%
\definecolor{currentstroke}{rgb}{0.690196,0.690196,0.690196}%
\pgfsetstrokecolor{currentstroke}%
\pgfsetdash{}{0pt}%
\pgfpathmoveto{\pgfqpoint{3.788192in}{4.044564in}}%
\pgfpathlineto{\pgfqpoint{6.702192in}{4.044564in}}%
\pgfusepath{stroke}%
\end{pgfscope}%
\begin{pgfscope}%
\pgfsetbuttcap%
\pgfsetroundjoin%
\definecolor{currentfill}{rgb}{0.000000,0.000000,0.000000}%
\pgfsetfillcolor{currentfill}%
\pgfsetlinewidth{0.803000pt}%
\definecolor{currentstroke}{rgb}{0.000000,0.000000,0.000000}%
\pgfsetstrokecolor{currentstroke}%
\pgfsetdash{}{0pt}%
\pgfsys@defobject{currentmarker}{\pgfqpoint{-0.048611in}{0.000000in}}{\pgfqpoint{-0.000000in}{0.000000in}}{%
\pgfpathmoveto{\pgfqpoint{-0.000000in}{0.000000in}}%
\pgfpathlineto{\pgfqpoint{-0.048611in}{0.000000in}}%
\pgfusepath{stroke,fill}%
}%
\begin{pgfscope}%
\pgfsys@transformshift{3.788192in}{4.044564in}%
\pgfsys@useobject{currentmarker}{}%
\end{pgfscope}%
\end{pgfscope}%
\begin{pgfscope}%
\definecolor{textcolor}{rgb}{0.000000,0.000000,0.000000}%
\pgfsetstrokecolor{textcolor}%
\pgfsetfillcolor{textcolor}%
\pgftext[x=3.495138in, y=3.975119in, left, base]{\color{textcolor}{\rmfamily\fontsize{14.000000}{16.800000}\selectfont\catcode`\^=\active\def^{\ifmmode\sp\else\^{}\fi}\catcode`\%=\active\def%{\%}$\mathdefault{12}$}}%
\end{pgfscope}%
\begin{pgfscope}%
\pgfpathrectangle{\pgfqpoint{3.788192in}{2.980138in}}{\pgfqpoint{2.914000in}{2.171400in}}%
\pgfusepath{clip}%
\pgfsetrectcap%
\pgfsetroundjoin%
\pgfsetlinewidth{0.803000pt}%
\definecolor{currentstroke}{rgb}{0.690196,0.690196,0.690196}%
\pgfsetstrokecolor{currentstroke}%
\pgfsetdash{}{0pt}%
\pgfpathmoveto{\pgfqpoint{3.788192in}{4.754181in}}%
\pgfpathlineto{\pgfqpoint{6.702192in}{4.754181in}}%
\pgfusepath{stroke}%
\end{pgfscope}%
\begin{pgfscope}%
\pgfsetbuttcap%
\pgfsetroundjoin%
\definecolor{currentfill}{rgb}{0.000000,0.000000,0.000000}%
\pgfsetfillcolor{currentfill}%
\pgfsetlinewidth{0.803000pt}%
\definecolor{currentstroke}{rgb}{0.000000,0.000000,0.000000}%
\pgfsetstrokecolor{currentstroke}%
\pgfsetdash{}{0pt}%
\pgfsys@defobject{currentmarker}{\pgfqpoint{-0.048611in}{0.000000in}}{\pgfqpoint{-0.000000in}{0.000000in}}{%
\pgfpathmoveto{\pgfqpoint{-0.000000in}{0.000000in}}%
\pgfpathlineto{\pgfqpoint{-0.048611in}{0.000000in}}%
\pgfusepath{stroke,fill}%
}%
\begin{pgfscope}%
\pgfsys@transformshift{3.788192in}{4.754181in}%
\pgfsys@useobject{currentmarker}{}%
\end{pgfscope}%
\end{pgfscope}%
\begin{pgfscope}%
\definecolor{textcolor}{rgb}{0.000000,0.000000,0.000000}%
\pgfsetstrokecolor{textcolor}%
\pgfsetfillcolor{textcolor}%
\pgftext[x=3.495138in, y=4.684736in, left, base]{\color{textcolor}{\rmfamily\fontsize{14.000000}{16.800000}\selectfont\catcode`\^=\active\def^{\ifmmode\sp\else\^{}\fi}\catcode`\%=\active\def%{\%}$\mathdefault{14}$}}%
\end{pgfscope}%
\begin{pgfscope}%
\pgfpathrectangle{\pgfqpoint{3.788192in}{2.980138in}}{\pgfqpoint{2.914000in}{2.171400in}}%
\pgfusepath{clip}%
\pgfsetrectcap%
\pgfsetroundjoin%
\pgfsetlinewidth{1.505625pt}%
\definecolor{currentstroke}{rgb}{0.000000,0.000000,1.000000}%
\pgfsetstrokecolor{currentstroke}%
\pgfsetdash{}{0pt}%
\pgfpathmoveto{\pgfqpoint{5.478622in}{4.808529in}}%
\pgfpathlineto{\pgfqpoint{5.701947in}{2.977638in}}%
\pgfusepath{stroke}%
\end{pgfscope}%
\begin{pgfscope}%
\pgfpathrectangle{\pgfqpoint{3.788192in}{2.980138in}}{\pgfqpoint{2.914000in}{2.171400in}}%
\pgfusepath{clip}%
\pgfsetbuttcap%
\pgfsetroundjoin%
\definecolor{currentfill}{rgb}{0.000000,0.000000,1.000000}%
\pgfsetfillcolor{currentfill}%
\pgfsetlinewidth{1.003750pt}%
\definecolor{currentstroke}{rgb}{0.000000,0.000000,1.000000}%
\pgfsetstrokecolor{currentstroke}%
\pgfsetdash{}{0pt}%
\pgfsys@defobject{currentmarker}{\pgfqpoint{-0.041667in}{-0.041667in}}{\pgfqpoint{0.041667in}{0.041667in}}{%
\pgfpathmoveto{\pgfqpoint{0.000000in}{-0.041667in}}%
\pgfpathcurveto{\pgfqpoint{0.011050in}{-0.041667in}}{\pgfqpoint{0.021649in}{-0.037276in}}{\pgfqpoint{0.029463in}{-0.029463in}}%
\pgfpathcurveto{\pgfqpoint{0.037276in}{-0.021649in}}{\pgfqpoint{0.041667in}{-0.011050in}}{\pgfqpoint{0.041667in}{0.000000in}}%
\pgfpathcurveto{\pgfqpoint{0.041667in}{0.011050in}}{\pgfqpoint{0.037276in}{0.021649in}}{\pgfqpoint{0.029463in}{0.029463in}}%
\pgfpathcurveto{\pgfqpoint{0.021649in}{0.037276in}}{\pgfqpoint{0.011050in}{0.041667in}}{\pgfqpoint{0.000000in}{0.041667in}}%
\pgfpathcurveto{\pgfqpoint{-0.011050in}{0.041667in}}{\pgfqpoint{-0.021649in}{0.037276in}}{\pgfqpoint{-0.029463in}{0.029463in}}%
\pgfpathcurveto{\pgfqpoint{-0.037276in}{0.021649in}}{\pgfqpoint{-0.041667in}{0.011050in}}{\pgfqpoint{-0.041667in}{0.000000in}}%
\pgfpathcurveto{\pgfqpoint{-0.041667in}{-0.011050in}}{\pgfqpoint{-0.037276in}{-0.021649in}}{\pgfqpoint{-0.029463in}{-0.029463in}}%
\pgfpathcurveto{\pgfqpoint{-0.021649in}{-0.037276in}}{\pgfqpoint{-0.011050in}{-0.041667in}}{\pgfqpoint{0.000000in}{-0.041667in}}%
\pgfpathlineto{\pgfqpoint{0.000000in}{-0.041667in}}%
\pgfpathclose%
\pgfusepath{stroke,fill}%
}%
\begin{pgfscope}%
\pgfsys@transformshift{5.478622in}{4.808529in}%
\pgfsys@useobject{currentmarker}{}%
\end{pgfscope}%
\begin{pgfscope}%
\pgfsys@transformshift{5.786444in}{2.284897in}%
\pgfsys@useobject{currentmarker}{}%
\end{pgfscope}%
\begin{pgfscope}%
\pgfsys@transformshift{5.955602in}{1.807070in}%
\pgfsys@useobject{currentmarker}{}%
\end{pgfscope}%
\begin{pgfscope}%
\pgfsys@transformshift{6.072257in}{1.530809in}%
\pgfsys@useobject{currentmarker}{}%
\end{pgfscope}%
\begin{pgfscope}%
\pgfsys@transformshift{6.134561in}{1.493263in}%
\pgfsys@useobject{currentmarker}{}%
\end{pgfscope}%
\begin{pgfscope}%
\pgfsys@transformshift{6.204334in}{1.296852in}%
\pgfsys@useobject{currentmarker}{}%
\end{pgfscope}%
\begin{pgfscope}%
\pgfsys@transformshift{6.281957in}{1.238734in}%
\pgfsys@useobject{currentmarker}{}%
\end{pgfscope}%
\begin{pgfscope}%
\pgfsys@transformshift{6.389240in}{1.234341in}%
\pgfsys@useobject{currentmarker}{}%
\end{pgfscope}%
\begin{pgfscope}%
\pgfsys@transformshift{6.427201in}{1.108377in}%
\pgfsys@useobject{currentmarker}{}%
\end{pgfscope}%
\begin{pgfscope}%
\pgfsys@transformshift{6.519285in}{1.077759in}%
\pgfsys@useobject{currentmarker}{}%
\end{pgfscope}%
\begin{pgfscope}%
\pgfsys@transformshift{6.689124in}{1.072073in}%
\pgfsys@useobject{currentmarker}{}%
\end{pgfscope}%
\begin{pgfscope}%
\pgfsys@transformshift{6.747190in}{1.024269in}%
\pgfsys@useobject{currentmarker}{}%
\end{pgfscope}%
\begin{pgfscope}%
\pgfsys@transformshift{6.806750in}{1.015907in}%
\pgfsys@useobject{currentmarker}{}%
\end{pgfscope}%
\begin{pgfscope}%
\pgfsys@transformshift{6.931730in}{0.991431in}%
\pgfsys@useobject{currentmarker}{}%
\end{pgfscope}%
\begin{pgfscope}%
\pgfsys@transformshift{7.035734in}{0.970288in}%
\pgfsys@useobject{currentmarker}{}%
\end{pgfscope}%
\begin{pgfscope}%
\pgfsys@transformshift{7.042663in}{0.960624in}%
\pgfsys@useobject{currentmarker}{}%
\end{pgfscope}%
\begin{pgfscope}%
\pgfsys@transformshift{7.110896in}{0.945926in}%
\pgfsys@useobject{currentmarker}{}%
\end{pgfscope}%
\begin{pgfscope}%
\pgfsys@transformshift{7.385541in}{0.918874in}%
\pgfsys@useobject{currentmarker}{}%
\end{pgfscope}%
\begin{pgfscope}%
\pgfsys@transformshift{7.677622in}{0.881182in}%
\pgfsys@useobject{currentmarker}{}%
\end{pgfscope}%
\begin{pgfscope}%
\pgfsys@transformshift{8.074033in}{0.854291in}%
\pgfsys@useobject{currentmarker}{}%
\end{pgfscope}%
\begin{pgfscope}%
\pgfsys@transformshift{8.437037in}{0.845259in}%
\pgfsys@useobject{currentmarker}{}%
\end{pgfscope}%
\begin{pgfscope}%
\pgfsys@transformshift{8.451833in}{0.833333in}%
\pgfsys@useobject{currentmarker}{}%
\end{pgfscope}%
\begin{pgfscope}%
\pgfsys@transformshift{8.473270in}{0.812531in}%
\pgfsys@useobject{currentmarker}{}%
\end{pgfscope}%
\begin{pgfscope}%
\pgfsys@transformshift{8.530759in}{0.805118in}%
\pgfsys@useobject{currentmarker}{}%
\end{pgfscope}%
\begin{pgfscope}%
\pgfsys@transformshift{8.667946in}{0.801766in}%
\pgfsys@useobject{currentmarker}{}%
\end{pgfscope}%
\begin{pgfscope}%
\pgfsys@transformshift{8.761129in}{0.792896in}%
\pgfsys@useobject{currentmarker}{}%
\end{pgfscope}%
\begin{pgfscope}%
\pgfsys@transformshift{9.175032in}{0.779118in}%
\pgfsys@useobject{currentmarker}{}%
\end{pgfscope}%
\begin{pgfscope}%
\pgfsys@transformshift{9.229242in}{0.771806in}%
\pgfsys@useobject{currentmarker}{}%
\end{pgfscope}%
\begin{pgfscope}%
\pgfsys@transformshift{9.230836in}{0.763945in}%
\pgfsys@useobject{currentmarker}{}%
\end{pgfscope}%
\begin{pgfscope}%
\pgfsys@transformshift{9.319803in}{0.762647in}%
\pgfsys@useobject{currentmarker}{}%
\end{pgfscope}%
\begin{pgfscope}%
\pgfsys@transformshift{9.530028in}{0.749444in}%
\pgfsys@useobject{currentmarker}{}%
\end{pgfscope}%
\begin{pgfscope}%
\pgfsys@transformshift{9.629502in}{0.748101in}%
\pgfsys@useobject{currentmarker}{}%
\end{pgfscope}%
\begin{pgfscope}%
\pgfsys@transformshift{9.644888in}{0.743176in}%
\pgfsys@useobject{currentmarker}{}%
\end{pgfscope}%
\begin{pgfscope}%
\pgfsys@transformshift{10.473184in}{0.730520in}%
\pgfsys@useobject{currentmarker}{}%
\end{pgfscope}%
\begin{pgfscope}%
\pgfsys@transformshift{10.538033in}{0.705808in}%
\pgfsys@useobject{currentmarker}{}%
\end{pgfscope}%
\begin{pgfscope}%
\pgfsys@transformshift{10.574876in}{0.703903in}%
\pgfsys@useobject{currentmarker}{}%
\end{pgfscope}%
\begin{pgfscope}%
\pgfsys@transformshift{10.706030in}{0.700593in}%
\pgfsys@useobject{currentmarker}{}%
\end{pgfscope}%
\begin{pgfscope}%
\pgfsys@transformshift{10.965602in}{0.687998in}%
\pgfsys@useobject{currentmarker}{}%
\end{pgfscope}%
\begin{pgfscope}%
\pgfsys@transformshift{11.060169in}{0.684789in}%
\pgfsys@useobject{currentmarker}{}%
\end{pgfscope}%
\begin{pgfscope}%
\pgfsys@transformshift{11.147519in}{0.680781in}%
\pgfsys@useobject{currentmarker}{}%
\end{pgfscope}%
\begin{pgfscope}%
\pgfsys@transformshift{11.370645in}{0.672484in}%
\pgfsys@useobject{currentmarker}{}%
\end{pgfscope}%
\begin{pgfscope}%
\pgfsys@transformshift{11.728159in}{0.668514in}%
\pgfsys@useobject{currentmarker}{}%
\end{pgfscope}%
\begin{pgfscope}%
\pgfsys@transformshift{12.137942in}{0.666709in}%
\pgfsys@useobject{currentmarker}{}%
\end{pgfscope}%
\begin{pgfscope}%
\pgfsys@transformshift{12.434130in}{0.666468in}%
\pgfsys@useobject{currentmarker}{}%
\end{pgfscope}%
\begin{pgfscope}%
\pgfsys@transformshift{12.804602in}{0.666397in}%
\pgfsys@useobject{currentmarker}{}%
\end{pgfscope}%
\begin{pgfscope}%
\pgfsys@transformshift{14.562739in}{0.661643in}%
\pgfsys@useobject{currentmarker}{}%
\end{pgfscope}%
\begin{pgfscope}%
\pgfsys@transformshift{15.028873in}{0.661022in}%
\pgfsys@useobject{currentmarker}{}%
\end{pgfscope}%
\begin{pgfscope}%
\pgfsys@transformshift{15.695083in}{0.659204in}%
\pgfsys@useobject{currentmarker}{}%
\end{pgfscope}%
\begin{pgfscope}%
\pgfsys@transformshift{16.663977in}{0.658612in}%
\pgfsys@useobject{currentmarker}{}%
\end{pgfscope}%
\begin{pgfscope}%
\pgfsys@transformshift{17.188373in}{0.655818in}%
\pgfsys@useobject{currentmarker}{}%
\end{pgfscope}%
\begin{pgfscope}%
\pgfsys@transformshift{18.420605in}{0.654898in}%
\pgfsys@useobject{currentmarker}{}%
\end{pgfscope}%
\begin{pgfscope}%
\pgfsys@transformshift{20.009331in}{0.652189in}%
\pgfsys@useobject{currentmarker}{}%
\end{pgfscope}%
\begin{pgfscope}%
\pgfsys@transformshift{21.260300in}{0.647607in}%
\pgfsys@useobject{currentmarker}{}%
\end{pgfscope}%
\begin{pgfscope}%
\pgfsys@transformshift{23.228701in}{0.645006in}%
\pgfsys@useobject{currentmarker}{}%
\end{pgfscope}%
\begin{pgfscope}%
\pgfsys@transformshift{25.888770in}{0.639824in}%
\pgfsys@useobject{currentmarker}{}%
\end{pgfscope}%
\begin{pgfscope}%
\pgfsys@transformshift{28.569097in}{0.635141in}%
\pgfsys@useobject{currentmarker}{}%
\end{pgfscope}%
\begin{pgfscope}%
\pgfsys@transformshift{32.504050in}{0.628662in}%
\pgfsys@useobject{currentmarker}{}%
\end{pgfscope}%
\begin{pgfscope}%
\pgfsys@transformshift{39.060135in}{0.620415in}%
\pgfsys@useobject{currentmarker}{}%
\end{pgfscope}%
\begin{pgfscope}%
\pgfsys@transformshift{50.416853in}{0.608419in}%
\pgfsys@useobject{currentmarker}{}%
\end{pgfscope}%
\begin{pgfscope}%
\pgfsys@transformshift{87.100182in}{0.583248in}%
\pgfsys@useobject{currentmarker}{}%
\end{pgfscope}%
\end{pgfscope}%
\begin{pgfscope}%
\pgfpathrectangle{\pgfqpoint{3.788192in}{2.980138in}}{\pgfqpoint{2.914000in}{2.171400in}}%
\pgfusepath{clip}%
\pgfsetrectcap%
\pgfsetroundjoin%
\pgfsetlinewidth{1.505625pt}%
\definecolor{currentstroke}{rgb}{0.121569,0.466667,0.705882}%
\pgfsetstrokecolor{currentstroke}%
\pgfsetstrokeopacity{0.500000}%
\pgfsetdash{}{0pt}%
\pgfusepath{stroke}%
\end{pgfscope}%
\begin{pgfscope}%
\pgfsetrectcap%
\pgfsetmiterjoin%
\pgfsetlinewidth{0.803000pt}%
\definecolor{currentstroke}{rgb}{0.000000,0.000000,0.000000}%
\pgfsetstrokecolor{currentstroke}%
\pgfsetdash{}{0pt}%
\pgfpathmoveto{\pgfqpoint{3.788192in}{2.980138in}}%
\pgfpathlineto{\pgfqpoint{3.788192in}{5.151538in}}%
\pgfusepath{stroke}%
\end{pgfscope}%
\begin{pgfscope}%
\pgfsetrectcap%
\pgfsetmiterjoin%
\pgfsetlinewidth{0.803000pt}%
\definecolor{currentstroke}{rgb}{0.000000,0.000000,0.000000}%
\pgfsetstrokecolor{currentstroke}%
\pgfsetdash{}{0pt}%
\pgfpathmoveto{\pgfqpoint{6.702192in}{2.980138in}}%
\pgfpathlineto{\pgfqpoint{6.702192in}{5.151538in}}%
\pgfusepath{stroke}%
\end{pgfscope}%
\begin{pgfscope}%
\pgfsetrectcap%
\pgfsetmiterjoin%
\pgfsetlinewidth{0.803000pt}%
\definecolor{currentstroke}{rgb}{0.000000,0.000000,0.000000}%
\pgfsetstrokecolor{currentstroke}%
\pgfsetdash{}{0pt}%
\pgfpathmoveto{\pgfqpoint{3.788192in}{2.980138in}}%
\pgfpathlineto{\pgfqpoint{6.702192in}{2.980138in}}%
\pgfusepath{stroke}%
\end{pgfscope}%
\begin{pgfscope}%
\pgfsetrectcap%
\pgfsetmiterjoin%
\pgfsetlinewidth{0.803000pt}%
\definecolor{currentstroke}{rgb}{0.000000,0.000000,0.000000}%
\pgfsetstrokecolor{currentstroke}%
\pgfsetdash{}{0pt}%
\pgfpathmoveto{\pgfqpoint{3.788192in}{5.151538in}}%
\pgfpathlineto{\pgfqpoint{6.702192in}{5.151538in}}%
\pgfusepath{stroke}%
\end{pgfscope}%
\begin{pgfscope}%
\pgfsetbuttcap%
\pgfsetmiterjoin%
\definecolor{currentfill}{rgb}{1.000000,1.000000,1.000000}%
\pgfsetfillcolor{currentfill}%
\pgfsetfillopacity{0.800000}%
\pgfsetlinewidth{1.003750pt}%
\definecolor{currentstroke}{rgb}{0.800000,0.800000,0.800000}%
\pgfsetstrokecolor{currentstroke}%
\pgfsetstrokeopacity{0.800000}%
\pgfsetdash{}{0pt}%
\pgfpathmoveto{\pgfqpoint{3.327765in}{0.781249in}}%
\pgfpathlineto{\pgfqpoint{6.732636in}{0.781249in}}%
\pgfpathquadraticcurveto{\pgfqpoint{6.777080in}{0.781249in}}{\pgfqpoint{6.777080in}{0.825694in}}%
\pgfpathlineto{\pgfqpoint{6.777080in}{2.153779in}}%
\pgfpathquadraticcurveto{\pgfqpoint{6.777080in}{2.198223in}}{\pgfqpoint{6.732636in}{2.198223in}}%
\pgfpathlineto{\pgfqpoint{3.327765in}{2.198223in}}%
\pgfpathquadraticcurveto{\pgfqpoint{3.283320in}{2.198223in}}{\pgfqpoint{3.283320in}{2.153779in}}%
\pgfpathlineto{\pgfqpoint{3.283320in}{0.825694in}}%
\pgfpathquadraticcurveto{\pgfqpoint{3.283320in}{0.781249in}}{\pgfqpoint{3.327765in}{0.781249in}}%
\pgfpathlineto{\pgfqpoint{3.327765in}{0.781249in}}%
\pgfpathclose%
\pgfusepath{stroke,fill}%
\end{pgfscope}%
\begin{pgfscope}%
\pgfsetrectcap%
\pgfsetroundjoin%
\pgfsetlinewidth{1.505625pt}%
\definecolor{currentstroke}{rgb}{0.000000,0.000000,1.000000}%
\pgfsetstrokecolor{currentstroke}%
\pgfsetdash{}{0pt}%
\pgfpathmoveto{\pgfqpoint{3.372209in}{2.020446in}}%
\pgfpathlineto{\pgfqpoint{3.594431in}{2.020446in}}%
\pgfpathlineto{\pgfqpoint{3.816654in}{2.020446in}}%
\pgfusepath{stroke}%
\end{pgfscope}%
\begin{pgfscope}%
\pgfsetbuttcap%
\pgfsetroundjoin%
\definecolor{currentfill}{rgb}{0.000000,0.000000,1.000000}%
\pgfsetfillcolor{currentfill}%
\pgfsetlinewidth{1.003750pt}%
\definecolor{currentstroke}{rgb}{0.000000,0.000000,1.000000}%
\pgfsetstrokecolor{currentstroke}%
\pgfsetdash{}{0pt}%
\pgfsys@defobject{currentmarker}{\pgfqpoint{-0.006944in}{-0.006944in}}{\pgfqpoint{0.006944in}{0.006944in}}{%
\pgfpathmoveto{\pgfqpoint{0.000000in}{-0.006944in}}%
\pgfpathcurveto{\pgfqpoint{0.001842in}{-0.006944in}}{\pgfqpoint{0.003608in}{-0.006213in}}{\pgfqpoint{0.004910in}{-0.004910in}}%
\pgfpathcurveto{\pgfqpoint{0.006213in}{-0.003608in}}{\pgfqpoint{0.006944in}{-0.001842in}}{\pgfqpoint{0.006944in}{0.000000in}}%
\pgfpathcurveto{\pgfqpoint{0.006944in}{0.001842in}}{\pgfqpoint{0.006213in}{0.003608in}}{\pgfqpoint{0.004910in}{0.004910in}}%
\pgfpathcurveto{\pgfqpoint{0.003608in}{0.006213in}}{\pgfqpoint{0.001842in}{0.006944in}}{\pgfqpoint{0.000000in}{0.006944in}}%
\pgfpathcurveto{\pgfqpoint{-0.001842in}{0.006944in}}{\pgfqpoint{-0.003608in}{0.006213in}}{\pgfqpoint{-0.004910in}{0.004910in}}%
\pgfpathcurveto{\pgfqpoint{-0.006213in}{0.003608in}}{\pgfqpoint{-0.006944in}{0.001842in}}{\pgfqpoint{-0.006944in}{0.000000in}}%
\pgfpathcurveto{\pgfqpoint{-0.006944in}{-0.001842in}}{\pgfqpoint{-0.006213in}{-0.003608in}}{\pgfqpoint{-0.004910in}{-0.004910in}}%
\pgfpathcurveto{\pgfqpoint{-0.003608in}{-0.006213in}}{\pgfqpoint{-0.001842in}{-0.006944in}}{\pgfqpoint{0.000000in}{-0.006944in}}%
\pgfpathlineto{\pgfqpoint{0.000000in}{-0.006944in}}%
\pgfpathclose%
\pgfusepath{stroke,fill}%
}%
\begin{pgfscope}%
\pgfsys@transformshift{3.594431in}{2.020446in}%
\pgfsys@useobject{currentmarker}{}%
\end{pgfscope}%
\end{pgfscope}%
\begin{pgfscope}%
\definecolor{textcolor}{rgb}{0.000000,0.000000,0.000000}%
\pgfsetstrokecolor{textcolor}%
\pgfsetfillcolor{textcolor}%
\pgftext[x=3.994431in,y=1.942668in,left,base]{\color{textcolor}{\rmfamily\fontsize{16.000000}{19.200000}\selectfont\catcode`\^=\active\def^{\ifmmode\sp\else\^{}\fi}\catcode`\%=\active\def%{\%}osier}}%
\end{pgfscope}%
\begin{pgfscope}%
\pgfsetrectcap%
\pgfsetroundjoin%
\pgfsetlinewidth{1.505625pt}%
\definecolor{currentstroke}{rgb}{0.121569,0.466667,0.705882}%
\pgfsetstrokecolor{currentstroke}%
\pgfsetstrokeopacity{0.500000}%
\pgfsetdash{}{0pt}%
\pgfpathmoveto{\pgfqpoint{3.372209in}{1.682869in}}%
\pgfpathlineto{\pgfqpoint{3.594431in}{1.682869in}}%
\pgfpathlineto{\pgfqpoint{3.816654in}{1.682869in}}%
\pgfusepath{stroke}%
\end{pgfscope}%
\begin{pgfscope}%
\definecolor{textcolor}{rgb}{0.000000,0.000000,0.000000}%
\pgfsetstrokecolor{textcolor}%
\pgfsetfillcolor{textcolor}%
\pgftext[x=3.994431in,y=1.605091in,left,base]{\color{textcolor}{\rmfamily\fontsize{16.000000}{19.200000}\selectfont\catcode`\^=\active\def^{\ifmmode\sp\else\^{}\fi}\catcode`\%=\active\def%{\%}near-optimal space (osier)}}%
\end{pgfscope}%
\begin{pgfscope}%
\pgfsetbuttcap%
\pgfsetroundjoin%
\pgfsetlinewidth{1.003750pt}%
\definecolor{currentstroke}{rgb}{1.000000,0.000000,0.000000}%
\pgfsetstrokecolor{currentstroke}%
\pgfsetdash{}{0pt}%
\pgfpathmoveto{\pgfqpoint{3.594431in}{1.294791in}}%
\pgfpathcurveto{\pgfqpoint{3.602668in}{1.294791in}}{\pgfqpoint{3.610568in}{1.298063in}}{\pgfqpoint{3.616392in}{1.303887in}}%
\pgfpathcurveto{\pgfqpoint{3.622216in}{1.309711in}}{\pgfqpoint{3.625488in}{1.317611in}}{\pgfqpoint{3.625488in}{1.325847in}}%
\pgfpathcurveto{\pgfqpoint{3.625488in}{1.334084in}}{\pgfqpoint{3.622216in}{1.341984in}}{\pgfqpoint{3.616392in}{1.347808in}}%
\pgfpathcurveto{\pgfqpoint{3.610568in}{1.353632in}}{\pgfqpoint{3.602668in}{1.356904in}}{\pgfqpoint{3.594431in}{1.356904in}}%
\pgfpathcurveto{\pgfqpoint{3.586195in}{1.356904in}}{\pgfqpoint{3.578295in}{1.353632in}}{\pgfqpoint{3.572471in}{1.347808in}}%
\pgfpathcurveto{\pgfqpoint{3.566647in}{1.341984in}}{\pgfqpoint{3.563375in}{1.334084in}}{\pgfqpoint{3.563375in}{1.325847in}}%
\pgfpathcurveto{\pgfqpoint{3.563375in}{1.317611in}}{\pgfqpoint{3.566647in}{1.309711in}}{\pgfqpoint{3.572471in}{1.303887in}}%
\pgfpathcurveto{\pgfqpoint{3.578295in}{1.298063in}}{\pgfqpoint{3.586195in}{1.294791in}}{\pgfqpoint{3.594431in}{1.294791in}}%
\pgfpathlineto{\pgfqpoint{3.594431in}{1.294791in}}%
\pgfpathclose%
\pgfusepath{stroke}%
\end{pgfscope}%
\begin{pgfscope}%
\definecolor{textcolor}{rgb}{0.000000,0.000000,0.000000}%
\pgfsetstrokecolor{textcolor}%
\pgfsetfillcolor{textcolor}%
\pgftext[x=3.994431in,y=1.267514in,left,base]{\color{textcolor}{\rmfamily\fontsize{16.000000}{19.200000}\selectfont\catcode`\^=\active\def^{\ifmmode\sp\else\^{}\fi}\catcode`\%=\active\def%{\%}temoa+mga}}%
\end{pgfscope}%
\begin{pgfscope}%
\pgfsetbuttcap%
\pgfsetmiterjoin%
\definecolor{currentfill}{rgb}{0.839216,0.152941,0.156863}%
\pgfsetfillcolor{currentfill}%
\pgfsetfillopacity{0.200000}%
\pgfsetlinewidth{1.003750pt}%
\definecolor{currentstroke}{rgb}{0.839216,0.152941,0.156863}%
\pgfsetstrokecolor{currentstroke}%
\pgfsetstrokeopacity{0.200000}%
\pgfsetdash{}{0pt}%
\pgfpathmoveto{\pgfqpoint{3.372209in}{0.929937in}}%
\pgfpathlineto{\pgfqpoint{3.816654in}{0.929937in}}%
\pgfpathlineto{\pgfqpoint{3.816654in}{1.085493in}}%
\pgfpathlineto{\pgfqpoint{3.372209in}{1.085493in}}%
\pgfpathlineto{\pgfqpoint{3.372209in}{0.929937in}}%
\pgfpathclose%
\pgfusepath{stroke,fill}%
\end{pgfscope}%
\begin{pgfscope}%
\pgfsetbuttcap%
\pgfsetmiterjoin%
\definecolor{currentfill}{rgb}{0.839216,0.152941,0.156863}%
\pgfsetfillcolor{currentfill}%
\pgfsetfillopacity{0.200000}%
\pgfsetlinewidth{1.003750pt}%
\definecolor{currentstroke}{rgb}{0.839216,0.152941,0.156863}%
\pgfsetstrokecolor{currentstroke}%
\pgfsetstrokeopacity{0.200000}%
\pgfsetdash{}{0pt}%
\pgfpathmoveto{\pgfqpoint{3.372209in}{0.929937in}}%
\pgfpathlineto{\pgfqpoint{3.816654in}{0.929937in}}%
\pgfpathlineto{\pgfqpoint{3.816654in}{1.085493in}}%
\pgfpathlineto{\pgfqpoint{3.372209in}{1.085493in}}%
\pgfpathlineto{\pgfqpoint{3.372209in}{0.929937in}}%
\pgfpathclose%
\pgfusepath{clip}%
\pgfsys@defobject{currentpattern}{\pgfqpoint{0in}{0in}}{\pgfqpoint{1in}{1in}}{%
\begin{pgfscope}%
\pgfpathrectangle{\pgfqpoint{0in}{0in}}{\pgfqpoint{1in}{1in}}%
\pgfusepath{clip}%
\pgfpathmoveto{\pgfqpoint{-0.500000in}{0.500000in}}%
\pgfpathlineto{\pgfqpoint{0.500000in}{1.500000in}}%
\pgfpathmoveto{\pgfqpoint{-0.333333in}{0.333333in}}%
\pgfpathlineto{\pgfqpoint{0.666667in}{1.333333in}}%
\pgfpathmoveto{\pgfqpoint{-0.166667in}{0.166667in}}%
\pgfpathlineto{\pgfqpoint{0.833333in}{1.166667in}}%
\pgfpathmoveto{\pgfqpoint{0.000000in}{0.000000in}}%
\pgfpathlineto{\pgfqpoint{1.000000in}{1.000000in}}%
\pgfpathmoveto{\pgfqpoint{0.166667in}{-0.166667in}}%
\pgfpathlineto{\pgfqpoint{1.166667in}{0.833333in}}%
\pgfpathmoveto{\pgfqpoint{0.333333in}{-0.333333in}}%
\pgfpathlineto{\pgfqpoint{1.333333in}{0.666667in}}%
\pgfpathmoveto{\pgfqpoint{0.500000in}{-0.500000in}}%
\pgfpathlineto{\pgfqpoint{1.500000in}{0.500000in}}%
\pgfusepath{stroke}%
\end{pgfscope}%
}%
\pgfsys@transformshift{3.372209in}{0.929937in}%
\pgfsys@useobject{currentpattern}{}%
\pgfsys@transformshift{1in}{0in}%
\pgfsys@transformshift{-1in}{0in}%
\pgfsys@transformshift{0in}{1in}%
\end{pgfscope}%
\begin{pgfscope}%
\definecolor{textcolor}{rgb}{0.000000,0.000000,0.000000}%
\pgfsetstrokecolor{textcolor}%
\pgfsetfillcolor{textcolor}%
\pgftext[x=3.994431in,y=0.929937in,left,base]{\color{textcolor}{\rmfamily\fontsize{16.000000}{19.200000}\selectfont\catcode`\^=\active\def^{\ifmmode\sp\else\^{}\fi}\catcode`\%=\active\def%{\%}near-optimal space (Temoa)}}%
\end{pgfscope}%
\end{pgfpicture}%
\makeatother%
\endgroup%
}
  \caption{Compares the least-cost solutions between \acs{temoa}
  and \acs{osier} as well as their sub-optimal spaces. The least-cost solutions
  for \ac{osier} and \ac{temoa} are within 0.5\% of each other.}
  \label{fig:temoa-benchmark-01}
\end{figure}

First, \ac{temoa}'s least-cost solution is slightly better (within 0.5\%) than
\ac{osier}'s in terms of both cost and emissions. This happens because
\ac{temoa} optimizes energy dispatch slightly differently than \ac{osier}. In
particular, the initial storage value for energy storage technologies is a
decision variable in \ac{temoa} and not in \ac{osier}. A second reason for this
discrepancy has to do with convergence. \ac{osier}'s Pareto-front could likely
be improved with a lower convergence tolerance, but this would use additional
computational resources. Although, \ac{temoa} calculated an optimal solution
with slightly lower cost than \ac{osier}, modelers should not place too much
importance on this fact because \acp{esom} should be used to generate insight
rather than answers, due to the nature of the systems being modeled
\cite{decarolis_using_2011}.

Next, the sub-optimal spaces mostly overlap, indicating that \ac{temoa} could
find a solution with lower carbon emissions after sufficient iterations.
However, none of \ac{temoa}'s \ac{mga} solutions fall within \ac{osier}'s
sub-optimal space. This point highlights the necessity for \acl{moo}. The
objective of \ac{mga} is to produce a \textit{diverse subset} of points in the
sub-optimal region. \ac{mga} may capture appealing alternatives for some
unmodeled objective in the original problem, but it cannot guarantee that those
solutions will be an improvement along any other objective axis. This is
especially apparent here, where the least-cost solution happens also to be the
lowest carbon solution, for \ac{temoa}. The relatively small area where the two
\acp{esom} do not overlap is fully explained by the difference in their
least-cost solutions.

Even though \ac{moo} reduces structural uncertainty, it will always exist, as
discussed in Section \ref{section:uncertainty}. Therefore, identifying
alternative solutions by sampling points in the inferior region is still useful.
Figure \ref{fig:temoa-benchmark-02} focuses on the near-optimal space presented
in \ref{fig:temoa-benchmark-01} and shows both the complete set of near-optimal
solutions (green) and some randomly selected points, highlighted in red.

\begin{figure}[h]
  \centering
  \resizebox{0.75\columnwidth}{!}{\input{figures/04_benchmark_chapter/osier_mga_subset_01.pgf}}
  % \includegraphics[width=0.6\columnwidth]{figures/results/osier_mga_subset_01.png}
  % \resizebox{0.6\columnwidth}{!}{\input{figures/results/osier_mga_subset_01.png}}
  \caption{Points within \ac{osier}'s sub-optimal space.}
  \label{fig:temoa-benchmark-02}
\end{figure}

Both Figure \ref{fig:temoa-benchmark-01} and Figure \ref{fig:temoa-benchmark-02}
present solutions in the objective space. However, in order to be prescriptive,
the policy solutions must be formulated according to the decision space. In
other words, described according to the mix of technologies that produced a
solution. Figure \ref{fig:temoa-benchmark-03} presents the spread of results in
the decision space for each model. Figure \ref{fig:temoa-benchmark-03}a shows
the spread of each technology present in \ac{osier}'s Pareto front. Figure
\ref{fig:temoa-benchmark-03}b shows the same, but also includes the randomly
selected points from \ac{osier}'s near-optimal space. Lastly, Figure
\ref{fig:temoa-benchmark-03}c shows the same kind of distribution for
\ac{temoa}'s \ac{mga} solutions. Presented in this way, the design space results
indicate which technologies are always or usually present. Technologies that are
absent in all cases, including the near-optimal solutions, may be safely
ignored. For \ac{osier}, these technologies include both types of coal, biomass,
and largely ignores wind energy. In \ac{temoa}'s results, there are no
technologies that are totally absent. This result is due to the imperative built
into standard \ac{mga} to identify solutions that are maximally different in
design space, whereas \ac{osier} randomly selected points in its inferior
region. This suggests one avenue for improving \ac{osier}.

\newpage
\begin{figure}[ht!]
  \centering
  \resizebox{\columnwidth}{!}{\input{figures/04_benchmark_chapter/temoa_osier_mga_comparison1x3.pgf}}
  \caption{The design spaces for a) points on the Pareto-front in Figure
  \ref{fig:temoa-benchmark-01}, b) selected points in \ac{osier}'s sub-optimal
  space, identified in Figure \ref{fig:temoa-benchmark-02}, and c) points
  generated by \ac{temoa}'s \ac{mga} algorithm shown in Figure
  \ref{fig:temoa-benchmark-01}.}
  \label{fig:temoa-benchmark-03}
\end{figure}

% Natural gas with \ac{ccs} shows up in the randomly selected points in
% \ac{osier}'s sub-optimal region. A geo-political locus for energy
% infrastructure, described in Section \ref{section:energy-system-boundaries}
% offers one possible explanation for this technology since states with
% significant natural gas resources might seek to maintain their influence by
% developing low-carbon technology that still uses natural gas.
 
\FloatBarrier