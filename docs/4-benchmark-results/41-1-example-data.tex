
\subsection{Energy Demand}
The dispatch comparison studies were modeled with synthetic demand data. The
data for these exercises were generated with

\begin{align}
    E(t) &= -\sin\left(\frac{2\pi t}{N_{\text{day}}}\right) + \sigma\sin\left(\frac{\pi t}{N_{\text{year}}}\right) + \delta + \chi,
    \intertext{where}
    E &= \text{electricity demand} \quad \left[\text{MWh}\right] \\
    t &= \text{the indendent variable, time} \quad \left[\text{hours}\right],\nonumber\\
    N_{i} &= \text{the total number of hours in a given period (day/year)},\nonumber\\
    \sigma &= \text{a scaling factor} \quad \left[-\right],\nonumber\\
    \delta &= \text{a vertical shift, representing base load power} \quad \left[-\right],\nonumber\\
    \chi &= \text{a normally distributed random variable}\in \left[0,0.05\right]\nonumber.
\end{align}

\noindent The first term represents the diurnal fluctuation in energy demand and
the second term corresponds to an annual change in demand. The vertical shift,
$\delta$, represents the base load throughout the year. The random variable,
$\chi$, simulates fluctuations in demand caused by unpredictable human behavior,
thereby adding some ``realism'' to the data. The data were subsequently
normalized with the L$_{\infty}$-norm, and then multiplied by some peak demand
value. Figure \ref{fig:demand-plot} shows the normalized demand data for a seven
day period.
 
\begin{figure}[ht!]
    \centering
    \resizebox{0.75\columnwidth}{!}{%% Creator: Matplotlib, PGF backend
%%
%% To include the figure in your LaTeX document, write
%%   \input{<filename>.pgf}
%%
%% Make sure the required packages are loaded in your preamble
%%   \usepackage{pgf}
%%
%% Also ensure that all the required font packages are loaded; for instance,
%% the lmodern package is sometimes necessary when using math font.
%%   \usepackage{lmodern}
%%
%% Figures using additional raster images can only be included by \input if
%% they are in the same directory as the main LaTeX file. For loading figures
%% from other directories you can use the `import` package
%%   \usepackage{import}
%%
%% and then include the figures with
%%   \import{<path to file>}{<filename>.pgf}
%%
%% Matplotlib used the following preamble
%%   \def\mathdefault#1{#1}
%%   \everymath=\expandafter{\the\everymath\displaystyle}
%%   \IfFileExists{scrextend.sty}{
%%     \usepackage[fontsize=10.000000pt]{scrextend}
%%   }{
%%     \renewcommand{\normalsize}{\fontsize{10.000000}{12.000000}\selectfont}
%%     \normalsize
%%   }
%%   
%%   \makeatletter\@ifpackageloaded{underscore}{}{\usepackage[strings]{underscore}}\makeatother
%%
\begingroup%
\makeatletter%
\begin{pgfpicture}%
\pgfpathrectangle{\pgfpointorigin}{\pgfqpoint{5.947842in}{4.466138in}}%
\pgfusepath{use as bounding box, clip}%
\begin{pgfscope}%
\pgfsetbuttcap%
\pgfsetmiterjoin%
\definecolor{currentfill}{rgb}{1.000000,1.000000,1.000000}%
\pgfsetfillcolor{currentfill}%
\pgfsetlinewidth{0.000000pt}%
\definecolor{currentstroke}{rgb}{0.000000,0.000000,0.000000}%
\pgfsetstrokecolor{currentstroke}%
\pgfsetdash{}{0pt}%
\pgfpathmoveto{\pgfqpoint{0.000000in}{0.000000in}}%
\pgfpathlineto{\pgfqpoint{5.947842in}{0.000000in}}%
\pgfpathlineto{\pgfqpoint{5.947842in}{4.466138in}}%
\pgfpathlineto{\pgfqpoint{0.000000in}{4.466138in}}%
\pgfpathlineto{\pgfqpoint{0.000000in}{0.000000in}}%
\pgfpathclose%
\pgfusepath{fill}%
\end{pgfscope}%
\begin{pgfscope}%
\pgfsetbuttcap%
\pgfsetmiterjoin%
\definecolor{currentfill}{rgb}{1.000000,1.000000,1.000000}%
\pgfsetfillcolor{currentfill}%
\pgfsetlinewidth{0.000000pt}%
\definecolor{currentstroke}{rgb}{0.000000,0.000000,0.000000}%
\pgfsetstrokecolor{currentstroke}%
\pgfsetstrokeopacity{0.000000}%
\pgfsetdash{}{0pt}%
\pgfpathmoveto{\pgfqpoint{0.887842in}{0.670138in}}%
\pgfpathlineto{\pgfqpoint{5.847842in}{0.670138in}}%
\pgfpathlineto{\pgfqpoint{5.847842in}{4.366138in}}%
\pgfpathlineto{\pgfqpoint{0.887842in}{4.366138in}}%
\pgfpathlineto{\pgfqpoint{0.887842in}{0.670138in}}%
\pgfpathclose%
\pgfusepath{fill}%
\end{pgfscope}%
\begin{pgfscope}%
\pgfpathrectangle{\pgfqpoint{0.887842in}{0.670138in}}{\pgfqpoint{4.960000in}{3.696000in}}%
\pgfusepath{clip}%
\pgfsetrectcap%
\pgfsetroundjoin%
\pgfsetlinewidth{0.803000pt}%
\definecolor{currentstroke}{rgb}{0.690196,0.690196,0.690196}%
\pgfsetstrokecolor{currentstroke}%
\pgfsetdash{}{0pt}%
\pgfpathmoveto{\pgfqpoint{0.887842in}{0.670138in}}%
\pgfpathlineto{\pgfqpoint{0.887842in}{4.366138in}}%
\pgfusepath{stroke}%
\end{pgfscope}%
\begin{pgfscope}%
\pgfsetbuttcap%
\pgfsetroundjoin%
\definecolor{currentfill}{rgb}{0.000000,0.000000,0.000000}%
\pgfsetfillcolor{currentfill}%
\pgfsetlinewidth{0.803000pt}%
\definecolor{currentstroke}{rgb}{0.000000,0.000000,0.000000}%
\pgfsetstrokecolor{currentstroke}%
\pgfsetdash{}{0pt}%
\pgfsys@defobject{currentmarker}{\pgfqpoint{0.000000in}{-0.048611in}}{\pgfqpoint{0.000000in}{0.000000in}}{%
\pgfpathmoveto{\pgfqpoint{0.000000in}{0.000000in}}%
\pgfpathlineto{\pgfqpoint{0.000000in}{-0.048611in}}%
\pgfusepath{stroke,fill}%
}%
\begin{pgfscope}%
\pgfsys@transformshift{0.887842in}{0.670138in}%
\pgfsys@useobject{currentmarker}{}%
\end{pgfscope}%
\end{pgfscope}%
\begin{pgfscope}%
\definecolor{textcolor}{rgb}{0.000000,0.000000,0.000000}%
\pgfsetstrokecolor{textcolor}%
\pgfsetfillcolor{textcolor}%
\pgftext[x=0.887842in,y=0.572916in,,top]{\color{textcolor}{\rmfamily\fontsize{14.000000}{16.800000}\selectfont\catcode`\^=\active\def^{\ifmmode\sp\else\^{}\fi}\catcode`\%=\active\def%{\%}$\mathdefault{0}$}}%
\end{pgfscope}%
\begin{pgfscope}%
\pgfpathrectangle{\pgfqpoint{0.887842in}{0.670138in}}{\pgfqpoint{4.960000in}{3.696000in}}%
\pgfusepath{clip}%
\pgfsetrectcap%
\pgfsetroundjoin%
\pgfsetlinewidth{0.803000pt}%
\definecolor{currentstroke}{rgb}{0.690196,0.690196,0.690196}%
\pgfsetstrokecolor{currentstroke}%
\pgfsetdash{}{0pt}%
\pgfpathmoveto{\pgfqpoint{1.625937in}{0.670138in}}%
\pgfpathlineto{\pgfqpoint{1.625937in}{4.366138in}}%
\pgfusepath{stroke}%
\end{pgfscope}%
\begin{pgfscope}%
\pgfsetbuttcap%
\pgfsetroundjoin%
\definecolor{currentfill}{rgb}{0.000000,0.000000,0.000000}%
\pgfsetfillcolor{currentfill}%
\pgfsetlinewidth{0.803000pt}%
\definecolor{currentstroke}{rgb}{0.000000,0.000000,0.000000}%
\pgfsetstrokecolor{currentstroke}%
\pgfsetdash{}{0pt}%
\pgfsys@defobject{currentmarker}{\pgfqpoint{0.000000in}{-0.048611in}}{\pgfqpoint{0.000000in}{0.000000in}}{%
\pgfpathmoveto{\pgfqpoint{0.000000in}{0.000000in}}%
\pgfpathlineto{\pgfqpoint{0.000000in}{-0.048611in}}%
\pgfusepath{stroke,fill}%
}%
\begin{pgfscope}%
\pgfsys@transformshift{1.625937in}{0.670138in}%
\pgfsys@useobject{currentmarker}{}%
\end{pgfscope}%
\end{pgfscope}%
\begin{pgfscope}%
\definecolor{textcolor}{rgb}{0.000000,0.000000,0.000000}%
\pgfsetstrokecolor{textcolor}%
\pgfsetfillcolor{textcolor}%
\pgftext[x=1.625937in,y=0.572916in,,top]{\color{textcolor}{\rmfamily\fontsize{14.000000}{16.800000}\selectfont\catcode`\^=\active\def^{\ifmmode\sp\else\^{}\fi}\catcode`\%=\active\def%{\%}$\mathdefault{25}$}}%
\end{pgfscope}%
\begin{pgfscope}%
\pgfpathrectangle{\pgfqpoint{0.887842in}{0.670138in}}{\pgfqpoint{4.960000in}{3.696000in}}%
\pgfusepath{clip}%
\pgfsetrectcap%
\pgfsetroundjoin%
\pgfsetlinewidth{0.803000pt}%
\definecolor{currentstroke}{rgb}{0.690196,0.690196,0.690196}%
\pgfsetstrokecolor{currentstroke}%
\pgfsetdash{}{0pt}%
\pgfpathmoveto{\pgfqpoint{2.364033in}{0.670138in}}%
\pgfpathlineto{\pgfqpoint{2.364033in}{4.366138in}}%
\pgfusepath{stroke}%
\end{pgfscope}%
\begin{pgfscope}%
\pgfsetbuttcap%
\pgfsetroundjoin%
\definecolor{currentfill}{rgb}{0.000000,0.000000,0.000000}%
\pgfsetfillcolor{currentfill}%
\pgfsetlinewidth{0.803000pt}%
\definecolor{currentstroke}{rgb}{0.000000,0.000000,0.000000}%
\pgfsetstrokecolor{currentstroke}%
\pgfsetdash{}{0pt}%
\pgfsys@defobject{currentmarker}{\pgfqpoint{0.000000in}{-0.048611in}}{\pgfqpoint{0.000000in}{0.000000in}}{%
\pgfpathmoveto{\pgfqpoint{0.000000in}{0.000000in}}%
\pgfpathlineto{\pgfqpoint{0.000000in}{-0.048611in}}%
\pgfusepath{stroke,fill}%
}%
\begin{pgfscope}%
\pgfsys@transformshift{2.364033in}{0.670138in}%
\pgfsys@useobject{currentmarker}{}%
\end{pgfscope}%
\end{pgfscope}%
\begin{pgfscope}%
\definecolor{textcolor}{rgb}{0.000000,0.000000,0.000000}%
\pgfsetstrokecolor{textcolor}%
\pgfsetfillcolor{textcolor}%
\pgftext[x=2.364033in,y=0.572916in,,top]{\color{textcolor}{\rmfamily\fontsize{14.000000}{16.800000}\selectfont\catcode`\^=\active\def^{\ifmmode\sp\else\^{}\fi}\catcode`\%=\active\def%{\%}$\mathdefault{50}$}}%
\end{pgfscope}%
\begin{pgfscope}%
\pgfpathrectangle{\pgfqpoint{0.887842in}{0.670138in}}{\pgfqpoint{4.960000in}{3.696000in}}%
\pgfusepath{clip}%
\pgfsetrectcap%
\pgfsetroundjoin%
\pgfsetlinewidth{0.803000pt}%
\definecolor{currentstroke}{rgb}{0.690196,0.690196,0.690196}%
\pgfsetstrokecolor{currentstroke}%
\pgfsetdash{}{0pt}%
\pgfpathmoveto{\pgfqpoint{3.102128in}{0.670138in}}%
\pgfpathlineto{\pgfqpoint{3.102128in}{4.366138in}}%
\pgfusepath{stroke}%
\end{pgfscope}%
\begin{pgfscope}%
\pgfsetbuttcap%
\pgfsetroundjoin%
\definecolor{currentfill}{rgb}{0.000000,0.000000,0.000000}%
\pgfsetfillcolor{currentfill}%
\pgfsetlinewidth{0.803000pt}%
\definecolor{currentstroke}{rgb}{0.000000,0.000000,0.000000}%
\pgfsetstrokecolor{currentstroke}%
\pgfsetdash{}{0pt}%
\pgfsys@defobject{currentmarker}{\pgfqpoint{0.000000in}{-0.048611in}}{\pgfqpoint{0.000000in}{0.000000in}}{%
\pgfpathmoveto{\pgfqpoint{0.000000in}{0.000000in}}%
\pgfpathlineto{\pgfqpoint{0.000000in}{-0.048611in}}%
\pgfusepath{stroke,fill}%
}%
\begin{pgfscope}%
\pgfsys@transformshift{3.102128in}{0.670138in}%
\pgfsys@useobject{currentmarker}{}%
\end{pgfscope}%
\end{pgfscope}%
\begin{pgfscope}%
\definecolor{textcolor}{rgb}{0.000000,0.000000,0.000000}%
\pgfsetstrokecolor{textcolor}%
\pgfsetfillcolor{textcolor}%
\pgftext[x=3.102128in,y=0.572916in,,top]{\color{textcolor}{\rmfamily\fontsize{14.000000}{16.800000}\selectfont\catcode`\^=\active\def^{\ifmmode\sp\else\^{}\fi}\catcode`\%=\active\def%{\%}$\mathdefault{75}$}}%
\end{pgfscope}%
\begin{pgfscope}%
\pgfpathrectangle{\pgfqpoint{0.887842in}{0.670138in}}{\pgfqpoint{4.960000in}{3.696000in}}%
\pgfusepath{clip}%
\pgfsetrectcap%
\pgfsetroundjoin%
\pgfsetlinewidth{0.803000pt}%
\definecolor{currentstroke}{rgb}{0.690196,0.690196,0.690196}%
\pgfsetstrokecolor{currentstroke}%
\pgfsetdash{}{0pt}%
\pgfpathmoveto{\pgfqpoint{3.840223in}{0.670138in}}%
\pgfpathlineto{\pgfqpoint{3.840223in}{4.366138in}}%
\pgfusepath{stroke}%
\end{pgfscope}%
\begin{pgfscope}%
\pgfsetbuttcap%
\pgfsetroundjoin%
\definecolor{currentfill}{rgb}{0.000000,0.000000,0.000000}%
\pgfsetfillcolor{currentfill}%
\pgfsetlinewidth{0.803000pt}%
\definecolor{currentstroke}{rgb}{0.000000,0.000000,0.000000}%
\pgfsetstrokecolor{currentstroke}%
\pgfsetdash{}{0pt}%
\pgfsys@defobject{currentmarker}{\pgfqpoint{0.000000in}{-0.048611in}}{\pgfqpoint{0.000000in}{0.000000in}}{%
\pgfpathmoveto{\pgfqpoint{0.000000in}{0.000000in}}%
\pgfpathlineto{\pgfqpoint{0.000000in}{-0.048611in}}%
\pgfusepath{stroke,fill}%
}%
\begin{pgfscope}%
\pgfsys@transformshift{3.840223in}{0.670138in}%
\pgfsys@useobject{currentmarker}{}%
\end{pgfscope}%
\end{pgfscope}%
\begin{pgfscope}%
\definecolor{textcolor}{rgb}{0.000000,0.000000,0.000000}%
\pgfsetstrokecolor{textcolor}%
\pgfsetfillcolor{textcolor}%
\pgftext[x=3.840223in,y=0.572916in,,top]{\color{textcolor}{\rmfamily\fontsize{14.000000}{16.800000}\selectfont\catcode`\^=\active\def^{\ifmmode\sp\else\^{}\fi}\catcode`\%=\active\def%{\%}$\mathdefault{100}$}}%
\end{pgfscope}%
\begin{pgfscope}%
\pgfpathrectangle{\pgfqpoint{0.887842in}{0.670138in}}{\pgfqpoint{4.960000in}{3.696000in}}%
\pgfusepath{clip}%
\pgfsetrectcap%
\pgfsetroundjoin%
\pgfsetlinewidth{0.803000pt}%
\definecolor{currentstroke}{rgb}{0.690196,0.690196,0.690196}%
\pgfsetstrokecolor{currentstroke}%
\pgfsetdash{}{0pt}%
\pgfpathmoveto{\pgfqpoint{4.578318in}{0.670138in}}%
\pgfpathlineto{\pgfqpoint{4.578318in}{4.366138in}}%
\pgfusepath{stroke}%
\end{pgfscope}%
\begin{pgfscope}%
\pgfsetbuttcap%
\pgfsetroundjoin%
\definecolor{currentfill}{rgb}{0.000000,0.000000,0.000000}%
\pgfsetfillcolor{currentfill}%
\pgfsetlinewidth{0.803000pt}%
\definecolor{currentstroke}{rgb}{0.000000,0.000000,0.000000}%
\pgfsetstrokecolor{currentstroke}%
\pgfsetdash{}{0pt}%
\pgfsys@defobject{currentmarker}{\pgfqpoint{0.000000in}{-0.048611in}}{\pgfqpoint{0.000000in}{0.000000in}}{%
\pgfpathmoveto{\pgfqpoint{0.000000in}{0.000000in}}%
\pgfpathlineto{\pgfqpoint{0.000000in}{-0.048611in}}%
\pgfusepath{stroke,fill}%
}%
\begin{pgfscope}%
\pgfsys@transformshift{4.578318in}{0.670138in}%
\pgfsys@useobject{currentmarker}{}%
\end{pgfscope}%
\end{pgfscope}%
\begin{pgfscope}%
\definecolor{textcolor}{rgb}{0.000000,0.000000,0.000000}%
\pgfsetstrokecolor{textcolor}%
\pgfsetfillcolor{textcolor}%
\pgftext[x=4.578318in,y=0.572916in,,top]{\color{textcolor}{\rmfamily\fontsize{14.000000}{16.800000}\selectfont\catcode`\^=\active\def^{\ifmmode\sp\else\^{}\fi}\catcode`\%=\active\def%{\%}$\mathdefault{125}$}}%
\end{pgfscope}%
\begin{pgfscope}%
\pgfpathrectangle{\pgfqpoint{0.887842in}{0.670138in}}{\pgfqpoint{4.960000in}{3.696000in}}%
\pgfusepath{clip}%
\pgfsetrectcap%
\pgfsetroundjoin%
\pgfsetlinewidth{0.803000pt}%
\definecolor{currentstroke}{rgb}{0.690196,0.690196,0.690196}%
\pgfsetstrokecolor{currentstroke}%
\pgfsetdash{}{0pt}%
\pgfpathmoveto{\pgfqpoint{5.316414in}{0.670138in}}%
\pgfpathlineto{\pgfqpoint{5.316414in}{4.366138in}}%
\pgfusepath{stroke}%
\end{pgfscope}%
\begin{pgfscope}%
\pgfsetbuttcap%
\pgfsetroundjoin%
\definecolor{currentfill}{rgb}{0.000000,0.000000,0.000000}%
\pgfsetfillcolor{currentfill}%
\pgfsetlinewidth{0.803000pt}%
\definecolor{currentstroke}{rgb}{0.000000,0.000000,0.000000}%
\pgfsetstrokecolor{currentstroke}%
\pgfsetdash{}{0pt}%
\pgfsys@defobject{currentmarker}{\pgfqpoint{0.000000in}{-0.048611in}}{\pgfqpoint{0.000000in}{0.000000in}}{%
\pgfpathmoveto{\pgfqpoint{0.000000in}{0.000000in}}%
\pgfpathlineto{\pgfqpoint{0.000000in}{-0.048611in}}%
\pgfusepath{stroke,fill}%
}%
\begin{pgfscope}%
\pgfsys@transformshift{5.316414in}{0.670138in}%
\pgfsys@useobject{currentmarker}{}%
\end{pgfscope}%
\end{pgfscope}%
\begin{pgfscope}%
\definecolor{textcolor}{rgb}{0.000000,0.000000,0.000000}%
\pgfsetstrokecolor{textcolor}%
\pgfsetfillcolor{textcolor}%
\pgftext[x=5.316414in,y=0.572916in,,top]{\color{textcolor}{\rmfamily\fontsize{14.000000}{16.800000}\selectfont\catcode`\^=\active\def^{\ifmmode\sp\else\^{}\fi}\catcode`\%=\active\def%{\%}$\mathdefault{150}$}}%
\end{pgfscope}%
\begin{pgfscope}%
\definecolor{textcolor}{rgb}{0.000000,0.000000,0.000000}%
\pgfsetstrokecolor{textcolor}%
\pgfsetfillcolor{textcolor}%
\pgftext[x=3.367842in,y=0.339583in,,top]{\color{textcolor}{\rmfamily\fontsize{18.000000}{21.600000}\selectfont\catcode`\^=\active\def^{\ifmmode\sp\else\^{}\fi}\catcode`\%=\active\def%{\%}Time [hours]}}%
\end{pgfscope}%
\begin{pgfscope}%
\pgfpathrectangle{\pgfqpoint{0.887842in}{0.670138in}}{\pgfqpoint{4.960000in}{3.696000in}}%
\pgfusepath{clip}%
\pgfsetrectcap%
\pgfsetroundjoin%
\pgfsetlinewidth{0.803000pt}%
\definecolor{currentstroke}{rgb}{0.690196,0.690196,0.690196}%
\pgfsetstrokecolor{currentstroke}%
\pgfsetdash{}{0pt}%
\pgfpathmoveto{\pgfqpoint{0.887842in}{1.123110in}}%
\pgfpathlineto{\pgfqpoint{5.847842in}{1.123110in}}%
\pgfusepath{stroke}%
\end{pgfscope}%
\begin{pgfscope}%
\pgfsetbuttcap%
\pgfsetroundjoin%
\definecolor{currentfill}{rgb}{0.000000,0.000000,0.000000}%
\pgfsetfillcolor{currentfill}%
\pgfsetlinewidth{0.803000pt}%
\definecolor{currentstroke}{rgb}{0.000000,0.000000,0.000000}%
\pgfsetstrokecolor{currentstroke}%
\pgfsetdash{}{0pt}%
\pgfsys@defobject{currentmarker}{\pgfqpoint{-0.048611in}{0.000000in}}{\pgfqpoint{-0.000000in}{0.000000in}}{%
\pgfpathmoveto{\pgfqpoint{-0.000000in}{0.000000in}}%
\pgfpathlineto{\pgfqpoint{-0.048611in}{0.000000in}}%
\pgfusepath{stroke,fill}%
}%
\begin{pgfscope}%
\pgfsys@transformshift{0.887842in}{1.123110in}%
\pgfsys@useobject{currentmarker}{}%
\end{pgfscope}%
\end{pgfscope}%
\begin{pgfscope}%
\definecolor{textcolor}{rgb}{0.000000,0.000000,0.000000}%
\pgfsetstrokecolor{textcolor}%
\pgfsetfillcolor{textcolor}%
\pgftext[x=0.395138in, y=1.039776in, left, base]{\color{textcolor}{\rmfamily\fontsize{16.000000}{19.200000}\selectfont\catcode`\^=\active\def^{\ifmmode\sp\else\^{}\fi}\catcode`\%=\active\def%{\%}$\mathdefault{0.75}$}}%
\end{pgfscope}%
\begin{pgfscope}%
\pgfpathrectangle{\pgfqpoint{0.887842in}{0.670138in}}{\pgfqpoint{4.960000in}{3.696000in}}%
\pgfusepath{clip}%
\pgfsetrectcap%
\pgfsetroundjoin%
\pgfsetlinewidth{0.803000pt}%
\definecolor{currentstroke}{rgb}{0.690196,0.690196,0.690196}%
\pgfsetstrokecolor{currentstroke}%
\pgfsetdash{}{0pt}%
\pgfpathmoveto{\pgfqpoint{0.887842in}{1.738115in}}%
\pgfpathlineto{\pgfqpoint{5.847842in}{1.738115in}}%
\pgfusepath{stroke}%
\end{pgfscope}%
\begin{pgfscope}%
\pgfsetbuttcap%
\pgfsetroundjoin%
\definecolor{currentfill}{rgb}{0.000000,0.000000,0.000000}%
\pgfsetfillcolor{currentfill}%
\pgfsetlinewidth{0.803000pt}%
\definecolor{currentstroke}{rgb}{0.000000,0.000000,0.000000}%
\pgfsetstrokecolor{currentstroke}%
\pgfsetdash{}{0pt}%
\pgfsys@defobject{currentmarker}{\pgfqpoint{-0.048611in}{0.000000in}}{\pgfqpoint{-0.000000in}{0.000000in}}{%
\pgfpathmoveto{\pgfqpoint{-0.000000in}{0.000000in}}%
\pgfpathlineto{\pgfqpoint{-0.048611in}{0.000000in}}%
\pgfusepath{stroke,fill}%
}%
\begin{pgfscope}%
\pgfsys@transformshift{0.887842in}{1.738115in}%
\pgfsys@useobject{currentmarker}{}%
\end{pgfscope}%
\end{pgfscope}%
\begin{pgfscope}%
\definecolor{textcolor}{rgb}{0.000000,0.000000,0.000000}%
\pgfsetstrokecolor{textcolor}%
\pgfsetfillcolor{textcolor}%
\pgftext[x=0.395138in, y=1.654782in, left, base]{\color{textcolor}{\rmfamily\fontsize{16.000000}{19.200000}\selectfont\catcode`\^=\active\def^{\ifmmode\sp\else\^{}\fi}\catcode`\%=\active\def%{\%}$\mathdefault{0.80}$}}%
\end{pgfscope}%
\begin{pgfscope}%
\pgfpathrectangle{\pgfqpoint{0.887842in}{0.670138in}}{\pgfqpoint{4.960000in}{3.696000in}}%
\pgfusepath{clip}%
\pgfsetrectcap%
\pgfsetroundjoin%
\pgfsetlinewidth{0.803000pt}%
\definecolor{currentstroke}{rgb}{0.690196,0.690196,0.690196}%
\pgfsetstrokecolor{currentstroke}%
\pgfsetdash{}{0pt}%
\pgfpathmoveto{\pgfqpoint{0.887842in}{2.353121in}}%
\pgfpathlineto{\pgfqpoint{5.847842in}{2.353121in}}%
\pgfusepath{stroke}%
\end{pgfscope}%
\begin{pgfscope}%
\pgfsetbuttcap%
\pgfsetroundjoin%
\definecolor{currentfill}{rgb}{0.000000,0.000000,0.000000}%
\pgfsetfillcolor{currentfill}%
\pgfsetlinewidth{0.803000pt}%
\definecolor{currentstroke}{rgb}{0.000000,0.000000,0.000000}%
\pgfsetstrokecolor{currentstroke}%
\pgfsetdash{}{0pt}%
\pgfsys@defobject{currentmarker}{\pgfqpoint{-0.048611in}{0.000000in}}{\pgfqpoint{-0.000000in}{0.000000in}}{%
\pgfpathmoveto{\pgfqpoint{-0.000000in}{0.000000in}}%
\pgfpathlineto{\pgfqpoint{-0.048611in}{0.000000in}}%
\pgfusepath{stroke,fill}%
}%
\begin{pgfscope}%
\pgfsys@transformshift{0.887842in}{2.353121in}%
\pgfsys@useobject{currentmarker}{}%
\end{pgfscope}%
\end{pgfscope}%
\begin{pgfscope}%
\definecolor{textcolor}{rgb}{0.000000,0.000000,0.000000}%
\pgfsetstrokecolor{textcolor}%
\pgfsetfillcolor{textcolor}%
\pgftext[x=0.395138in, y=2.269788in, left, base]{\color{textcolor}{\rmfamily\fontsize{16.000000}{19.200000}\selectfont\catcode`\^=\active\def^{\ifmmode\sp\else\^{}\fi}\catcode`\%=\active\def%{\%}$\mathdefault{0.85}$}}%
\end{pgfscope}%
\begin{pgfscope}%
\pgfpathrectangle{\pgfqpoint{0.887842in}{0.670138in}}{\pgfqpoint{4.960000in}{3.696000in}}%
\pgfusepath{clip}%
\pgfsetrectcap%
\pgfsetroundjoin%
\pgfsetlinewidth{0.803000pt}%
\definecolor{currentstroke}{rgb}{0.690196,0.690196,0.690196}%
\pgfsetstrokecolor{currentstroke}%
\pgfsetdash{}{0pt}%
\pgfpathmoveto{\pgfqpoint{0.887842in}{2.968127in}}%
\pgfpathlineto{\pgfqpoint{5.847842in}{2.968127in}}%
\pgfusepath{stroke}%
\end{pgfscope}%
\begin{pgfscope}%
\pgfsetbuttcap%
\pgfsetroundjoin%
\definecolor{currentfill}{rgb}{0.000000,0.000000,0.000000}%
\pgfsetfillcolor{currentfill}%
\pgfsetlinewidth{0.803000pt}%
\definecolor{currentstroke}{rgb}{0.000000,0.000000,0.000000}%
\pgfsetstrokecolor{currentstroke}%
\pgfsetdash{}{0pt}%
\pgfsys@defobject{currentmarker}{\pgfqpoint{-0.048611in}{0.000000in}}{\pgfqpoint{-0.000000in}{0.000000in}}{%
\pgfpathmoveto{\pgfqpoint{-0.000000in}{0.000000in}}%
\pgfpathlineto{\pgfqpoint{-0.048611in}{0.000000in}}%
\pgfusepath{stroke,fill}%
}%
\begin{pgfscope}%
\pgfsys@transformshift{0.887842in}{2.968127in}%
\pgfsys@useobject{currentmarker}{}%
\end{pgfscope}%
\end{pgfscope}%
\begin{pgfscope}%
\definecolor{textcolor}{rgb}{0.000000,0.000000,0.000000}%
\pgfsetstrokecolor{textcolor}%
\pgfsetfillcolor{textcolor}%
\pgftext[x=0.395138in, y=2.884793in, left, base]{\color{textcolor}{\rmfamily\fontsize{16.000000}{19.200000}\selectfont\catcode`\^=\active\def^{\ifmmode\sp\else\^{}\fi}\catcode`\%=\active\def%{\%}$\mathdefault{0.90}$}}%
\end{pgfscope}%
\begin{pgfscope}%
\pgfpathrectangle{\pgfqpoint{0.887842in}{0.670138in}}{\pgfqpoint{4.960000in}{3.696000in}}%
\pgfusepath{clip}%
\pgfsetrectcap%
\pgfsetroundjoin%
\pgfsetlinewidth{0.803000pt}%
\definecolor{currentstroke}{rgb}{0.690196,0.690196,0.690196}%
\pgfsetstrokecolor{currentstroke}%
\pgfsetdash{}{0pt}%
\pgfpathmoveto{\pgfqpoint{0.887842in}{3.583132in}}%
\pgfpathlineto{\pgfqpoint{5.847842in}{3.583132in}}%
\pgfusepath{stroke}%
\end{pgfscope}%
\begin{pgfscope}%
\pgfsetbuttcap%
\pgfsetroundjoin%
\definecolor{currentfill}{rgb}{0.000000,0.000000,0.000000}%
\pgfsetfillcolor{currentfill}%
\pgfsetlinewidth{0.803000pt}%
\definecolor{currentstroke}{rgb}{0.000000,0.000000,0.000000}%
\pgfsetstrokecolor{currentstroke}%
\pgfsetdash{}{0pt}%
\pgfsys@defobject{currentmarker}{\pgfqpoint{-0.048611in}{0.000000in}}{\pgfqpoint{-0.000000in}{0.000000in}}{%
\pgfpathmoveto{\pgfqpoint{-0.000000in}{0.000000in}}%
\pgfpathlineto{\pgfqpoint{-0.048611in}{0.000000in}}%
\pgfusepath{stroke,fill}%
}%
\begin{pgfscope}%
\pgfsys@transformshift{0.887842in}{3.583132in}%
\pgfsys@useobject{currentmarker}{}%
\end{pgfscope}%
\end{pgfscope}%
\begin{pgfscope}%
\definecolor{textcolor}{rgb}{0.000000,0.000000,0.000000}%
\pgfsetstrokecolor{textcolor}%
\pgfsetfillcolor{textcolor}%
\pgftext[x=0.395138in, y=3.499799in, left, base]{\color{textcolor}{\rmfamily\fontsize{16.000000}{19.200000}\selectfont\catcode`\^=\active\def^{\ifmmode\sp\else\^{}\fi}\catcode`\%=\active\def%{\%}$\mathdefault{0.95}$}}%
\end{pgfscope}%
\begin{pgfscope}%
\pgfpathrectangle{\pgfqpoint{0.887842in}{0.670138in}}{\pgfqpoint{4.960000in}{3.696000in}}%
\pgfusepath{clip}%
\pgfsetrectcap%
\pgfsetroundjoin%
\pgfsetlinewidth{0.803000pt}%
\definecolor{currentstroke}{rgb}{0.690196,0.690196,0.690196}%
\pgfsetstrokecolor{currentstroke}%
\pgfsetdash{}{0pt}%
\pgfpathmoveto{\pgfqpoint{0.887842in}{4.198138in}}%
\pgfpathlineto{\pgfqpoint{5.847842in}{4.198138in}}%
\pgfusepath{stroke}%
\end{pgfscope}%
\begin{pgfscope}%
\pgfsetbuttcap%
\pgfsetroundjoin%
\definecolor{currentfill}{rgb}{0.000000,0.000000,0.000000}%
\pgfsetfillcolor{currentfill}%
\pgfsetlinewidth{0.803000pt}%
\definecolor{currentstroke}{rgb}{0.000000,0.000000,0.000000}%
\pgfsetstrokecolor{currentstroke}%
\pgfsetdash{}{0pt}%
\pgfsys@defobject{currentmarker}{\pgfqpoint{-0.048611in}{0.000000in}}{\pgfqpoint{-0.000000in}{0.000000in}}{%
\pgfpathmoveto{\pgfqpoint{-0.000000in}{0.000000in}}%
\pgfpathlineto{\pgfqpoint{-0.048611in}{0.000000in}}%
\pgfusepath{stroke,fill}%
}%
\begin{pgfscope}%
\pgfsys@transformshift{0.887842in}{4.198138in}%
\pgfsys@useobject{currentmarker}{}%
\end{pgfscope}%
\end{pgfscope}%
\begin{pgfscope}%
\definecolor{textcolor}{rgb}{0.000000,0.000000,0.000000}%
\pgfsetstrokecolor{textcolor}%
\pgfsetfillcolor{textcolor}%
\pgftext[x=0.395138in, y=4.114805in, left, base]{\color{textcolor}{\rmfamily\fontsize{16.000000}{19.200000}\selectfont\catcode`\^=\active\def^{\ifmmode\sp\else\^{}\fi}\catcode`\%=\active\def%{\%}$\mathdefault{1.00}$}}%
\end{pgfscope}%
\begin{pgfscope}%
\definecolor{textcolor}{rgb}{0.000000,0.000000,0.000000}%
\pgfsetstrokecolor{textcolor}%
\pgfsetfillcolor{textcolor}%
\pgftext[x=0.339583in,y=2.518138in,,bottom,rotate=90.000000]{\color{textcolor}{\rmfamily\fontsize{18.000000}{21.600000}\selectfont\catcode`\^=\active\def^{\ifmmode\sp\else\^{}\fi}\catcode`\%=\active\def%{\%}Normalized Demand [-]}}%
\end{pgfscope}%
\begin{pgfscope}%
\pgfpathrectangle{\pgfqpoint{0.887842in}{0.670138in}}{\pgfqpoint{4.960000in}{3.696000in}}%
\pgfusepath{clip}%
\pgfsetbuttcap%
\pgfsetroundjoin%
\pgfsetlinewidth{1.505625pt}%
\definecolor{currentstroke}{rgb}{0.000000,0.000000,0.000000}%
\pgfsetstrokecolor{currentstroke}%
\pgfsetdash{{5.550000pt}{2.400000pt}}{0.000000pt}%
\pgfpathmoveto{\pgfqpoint{0.887842in}{1.958176in}}%
\pgfpathlineto{\pgfqpoint{0.917543in}{2.044758in}}%
\pgfpathlineto{\pgfqpoint{0.947243in}{1.922382in}}%
\pgfpathlineto{\pgfqpoint{0.976944in}{1.546920in}}%
\pgfpathlineto{\pgfqpoint{1.006645in}{1.529475in}}%
\pgfpathlineto{\pgfqpoint{1.036345in}{1.891733in}}%
\pgfpathlineto{\pgfqpoint{1.066046in}{0.838138in}}%
\pgfpathlineto{\pgfqpoint{1.095746in}{1.444618in}}%
\pgfpathlineto{\pgfqpoint{1.125447in}{0.960647in}}%
\pgfpathlineto{\pgfqpoint{1.155148in}{1.616754in}}%
\pgfpathlineto{\pgfqpoint{1.184848in}{1.664165in}}%
\pgfpathlineto{\pgfqpoint{1.214549in}{2.374606in}}%
\pgfpathlineto{\pgfqpoint{1.244249in}{2.172424in}}%
\pgfpathlineto{\pgfqpoint{1.273950in}{2.794906in}}%
\pgfpathlineto{\pgfqpoint{1.303651in}{2.661553in}}%
\pgfpathlineto{\pgfqpoint{1.333351in}{2.693192in}}%
\pgfpathlineto{\pgfqpoint{1.363052in}{3.475558in}}%
\pgfpathlineto{\pgfqpoint{1.392752in}{3.887435in}}%
\pgfpathlineto{\pgfqpoint{1.422453in}{3.639311in}}%
\pgfpathlineto{\pgfqpoint{1.452154in}{3.239617in}}%
\pgfpathlineto{\pgfqpoint{1.481854in}{3.410796in}}%
\pgfpathlineto{\pgfqpoint{1.511555in}{3.335144in}}%
\pgfpathlineto{\pgfqpoint{1.541255in}{3.187536in}}%
\pgfpathlineto{\pgfqpoint{1.570956in}{2.359582in}}%
\pgfpathlineto{\pgfqpoint{1.600657in}{2.487936in}}%
\pgfpathlineto{\pgfqpoint{1.630357in}{2.109618in}}%
\pgfpathlineto{\pgfqpoint{1.660058in}{1.745980in}}%
\pgfpathlineto{\pgfqpoint{1.689758in}{1.524639in}}%
\pgfpathlineto{\pgfqpoint{1.719459in}{1.191070in}}%
\pgfpathlineto{\pgfqpoint{1.749160in}{1.102547in}}%
\pgfpathlineto{\pgfqpoint{1.778860in}{1.530939in}}%
\pgfpathlineto{\pgfqpoint{1.808561in}{1.338121in}}%
\pgfpathlineto{\pgfqpoint{1.838261in}{1.298395in}}%
\pgfpathlineto{\pgfqpoint{1.867962in}{1.031394in}}%
\pgfpathlineto{\pgfqpoint{1.897663in}{1.658393in}}%
\pgfpathlineto{\pgfqpoint{1.927363in}{2.033188in}}%
\pgfpathlineto{\pgfqpoint{1.957064in}{2.158119in}}%
\pgfpathlineto{\pgfqpoint{1.986764in}{2.447869in}}%
\pgfpathlineto{\pgfqpoint{2.016465in}{3.223311in}}%
\pgfpathlineto{\pgfqpoint{2.046166in}{3.206063in}}%
\pgfpathlineto{\pgfqpoint{2.075866in}{3.041018in}}%
\pgfpathlineto{\pgfqpoint{2.105567in}{3.225609in}}%
\pgfpathlineto{\pgfqpoint{2.135267in}{3.487362in}}%
\pgfpathlineto{\pgfqpoint{2.164968in}{3.757199in}}%
\pgfpathlineto{\pgfqpoint{2.194669in}{3.015842in}}%
\pgfpathlineto{\pgfqpoint{2.224369in}{3.533672in}}%
\pgfpathlineto{\pgfqpoint{2.254070in}{2.799670in}}%
\pgfpathlineto{\pgfqpoint{2.283770in}{2.228690in}}%
\pgfpathlineto{\pgfqpoint{2.313471in}{2.442073in}}%
\pgfpathlineto{\pgfqpoint{2.343172in}{2.360054in}}%
\pgfpathlineto{\pgfqpoint{2.372872in}{1.705816in}}%
\pgfpathlineto{\pgfqpoint{2.402573in}{1.683240in}}%
\pgfpathlineto{\pgfqpoint{2.432273in}{1.295588in}}%
\pgfpathlineto{\pgfqpoint{2.461974in}{1.275437in}}%
\pgfpathlineto{\pgfqpoint{2.491675in}{1.651424in}}%
\pgfpathlineto{\pgfqpoint{2.521375in}{1.540018in}}%
\pgfpathlineto{\pgfqpoint{2.551076in}{1.195720in}}%
\pgfpathlineto{\pgfqpoint{2.580776in}{2.331129in}}%
\pgfpathlineto{\pgfqpoint{2.610477in}{1.704241in}}%
\pgfpathlineto{\pgfqpoint{2.640178in}{2.429962in}}%
\pgfpathlineto{\pgfqpoint{2.669878in}{2.654665in}}%
\pgfpathlineto{\pgfqpoint{2.699579in}{2.817962in}}%
\pgfpathlineto{\pgfqpoint{2.729279in}{3.154851in}}%
\pgfpathlineto{\pgfqpoint{2.758980in}{3.293811in}}%
\pgfpathlineto{\pgfqpoint{2.788681in}{3.687888in}}%
\pgfpathlineto{\pgfqpoint{2.818381in}{3.417856in}}%
\pgfpathlineto{\pgfqpoint{2.848082in}{3.321083in}}%
\pgfpathlineto{\pgfqpoint{2.877782in}{3.634206in}}%
\pgfpathlineto{\pgfqpoint{2.907483in}{3.782931in}}%
\pgfpathlineto{\pgfqpoint{2.937184in}{3.140046in}}%
\pgfpathlineto{\pgfqpoint{2.966884in}{3.308055in}}%
\pgfpathlineto{\pgfqpoint{2.996585in}{2.602061in}}%
\pgfpathlineto{\pgfqpoint{3.026285in}{2.680850in}}%
\pgfpathlineto{\pgfqpoint{3.055986in}{2.249528in}}%
\pgfpathlineto{\pgfqpoint{3.085687in}{1.883078in}}%
\pgfpathlineto{\pgfqpoint{3.115387in}{1.762572in}}%
\pgfpathlineto{\pgfqpoint{3.145088in}{1.197491in}}%
\pgfpathlineto{\pgfqpoint{3.174788in}{1.333907in}}%
\pgfpathlineto{\pgfqpoint{3.204489in}{0.945895in}}%
\pgfpathlineto{\pgfqpoint{3.234190in}{0.975515in}}%
\pgfpathlineto{\pgfqpoint{3.263890in}{1.753345in}}%
\pgfpathlineto{\pgfqpoint{3.293591in}{1.721675in}}%
\pgfpathlineto{\pgfqpoint{3.323291in}{2.254162in}}%
\pgfpathlineto{\pgfqpoint{3.352992in}{2.346972in}}%
\pgfpathlineto{\pgfqpoint{3.382693in}{2.443451in}}%
\pgfpathlineto{\pgfqpoint{3.412393in}{3.589187in}}%
\pgfpathlineto{\pgfqpoint{3.442094in}{3.502263in}}%
\pgfpathlineto{\pgfqpoint{3.471794in}{3.438667in}}%
\pgfpathlineto{\pgfqpoint{3.501495in}{3.556286in}}%
\pgfpathlineto{\pgfqpoint{3.531196in}{4.012404in}}%
\pgfpathlineto{\pgfqpoint{3.560896in}{3.553468in}}%
\pgfpathlineto{\pgfqpoint{3.590597in}{3.901181in}}%
\pgfpathlineto{\pgfqpoint{3.620297in}{3.545689in}}%
\pgfpathlineto{\pgfqpoint{3.649998in}{2.870931in}}%
\pgfpathlineto{\pgfqpoint{3.679699in}{3.085369in}}%
\pgfpathlineto{\pgfqpoint{3.709399in}{2.878441in}}%
\pgfpathlineto{\pgfqpoint{3.739100in}{2.443426in}}%
\pgfpathlineto{\pgfqpoint{3.768800in}{2.249554in}}%
\pgfpathlineto{\pgfqpoint{3.798501in}{2.086236in}}%
\pgfpathlineto{\pgfqpoint{3.828202in}{1.445200in}}%
\pgfpathlineto{\pgfqpoint{3.857902in}{1.937734in}}%
\pgfpathlineto{\pgfqpoint{3.887603in}{1.383444in}}%
\pgfpathlineto{\pgfqpoint{3.917303in}{1.549526in}}%
\pgfpathlineto{\pgfqpoint{3.947004in}{1.079110in}}%
\pgfpathlineto{\pgfqpoint{3.976704in}{1.841743in}}%
\pgfpathlineto{\pgfqpoint{4.006405in}{1.998687in}}%
\pgfpathlineto{\pgfqpoint{4.036106in}{1.966415in}}%
\pgfpathlineto{\pgfqpoint{4.065806in}{2.558349in}}%
\pgfpathlineto{\pgfqpoint{4.095507in}{2.823050in}}%
\pgfpathlineto{\pgfqpoint{4.125207in}{3.230431in}}%
\pgfpathlineto{\pgfqpoint{4.154908in}{3.423068in}}%
\pgfpathlineto{\pgfqpoint{4.184609in}{3.336224in}}%
\pgfpathlineto{\pgfqpoint{4.214309in}{3.619612in}}%
\pgfpathlineto{\pgfqpoint{4.244010in}{3.280117in}}%
\pgfpathlineto{\pgfqpoint{4.273710in}{3.666236in}}%
\pgfpathlineto{\pgfqpoint{4.303411in}{3.450406in}}%
\pgfpathlineto{\pgfqpoint{4.333112in}{3.654189in}}%
\pgfpathlineto{\pgfqpoint{4.362812in}{3.493241in}}%
\pgfpathlineto{\pgfqpoint{4.392513in}{3.061002in}}%
\pgfpathlineto{\pgfqpoint{4.422213in}{2.186800in}}%
\pgfpathlineto{\pgfqpoint{4.451914in}{2.434646in}}%
\pgfpathlineto{\pgfqpoint{4.481615in}{1.770448in}}%
\pgfpathlineto{\pgfqpoint{4.511315in}{1.875688in}}%
\pgfpathlineto{\pgfqpoint{4.541016in}{1.610861in}}%
\pgfpathlineto{\pgfqpoint{4.570716in}{1.187651in}}%
\pgfpathlineto{\pgfqpoint{4.600417in}{1.076465in}}%
\pgfpathlineto{\pgfqpoint{4.630118in}{1.598518in}}%
\pgfpathlineto{\pgfqpoint{4.659818in}{1.995586in}}%
\pgfpathlineto{\pgfqpoint{4.689519in}{1.664359in}}%
\pgfpathlineto{\pgfqpoint{4.719219in}{1.982897in}}%
\pgfpathlineto{\pgfqpoint{4.748920in}{2.055840in}}%
\pgfpathlineto{\pgfqpoint{4.778621in}{2.420977in}}%
\pgfpathlineto{\pgfqpoint{4.808321in}{2.401049in}}%
\pgfpathlineto{\pgfqpoint{4.838022in}{3.096819in}}%
\pgfpathlineto{\pgfqpoint{4.867722in}{3.374260in}}%
\pgfpathlineto{\pgfqpoint{4.897423in}{3.457749in}}%
\pgfpathlineto{\pgfqpoint{4.927124in}{3.685918in}}%
\pgfpathlineto{\pgfqpoint{4.956824in}{3.634117in}}%
\pgfpathlineto{\pgfqpoint{4.986525in}{4.002873in}}%
\pgfpathlineto{\pgfqpoint{5.016225in}{3.573000in}}%
\pgfpathlineto{\pgfqpoint{5.045926in}{3.478674in}}%
\pgfpathlineto{\pgfqpoint{5.075627in}{3.408471in}}%
\pgfpathlineto{\pgfqpoint{5.105327in}{2.955125in}}%
\pgfpathlineto{\pgfqpoint{5.135028in}{2.706344in}}%
\pgfpathlineto{\pgfqpoint{5.164728in}{1.888140in}}%
\pgfpathlineto{\pgfqpoint{5.194429in}{1.970116in}}%
\pgfpathlineto{\pgfqpoint{5.224130in}{2.225714in}}%
\pgfpathlineto{\pgfqpoint{5.253830in}{1.681150in}}%
\pgfpathlineto{\pgfqpoint{5.283531in}{1.704615in}}%
\pgfpathlineto{\pgfqpoint{5.313231in}{1.511578in}}%
\pgfpathlineto{\pgfqpoint{5.342932in}{1.889828in}}%
\pgfpathlineto{\pgfqpoint{5.372633in}{1.390937in}}%
\pgfpathlineto{\pgfqpoint{5.402333in}{1.558361in}}%
\pgfpathlineto{\pgfqpoint{5.432034in}{2.136725in}}%
\pgfpathlineto{\pgfqpoint{5.461734in}{2.446965in}}%
\pgfpathlineto{\pgfqpoint{5.491435in}{2.640863in}}%
\pgfpathlineto{\pgfqpoint{5.521136in}{2.940073in}}%
\pgfpathlineto{\pgfqpoint{5.550836in}{3.001889in}}%
\pgfpathlineto{\pgfqpoint{5.580537in}{3.642115in}}%
\pgfpathlineto{\pgfqpoint{5.610237in}{3.600702in}}%
\pgfpathlineto{\pgfqpoint{5.639938in}{3.829289in}}%
\pgfpathlineto{\pgfqpoint{5.669639in}{3.647381in}}%
\pgfpathlineto{\pgfqpoint{5.699339in}{4.198138in}}%
\pgfpathlineto{\pgfqpoint{5.729040in}{3.505597in}}%
\pgfpathlineto{\pgfqpoint{5.758740in}{3.466256in}}%
\pgfpathlineto{\pgfqpoint{5.788441in}{2.932889in}}%
\pgfpathlineto{\pgfqpoint{5.818142in}{2.721003in}}%
\pgfpathlineto{\pgfqpoint{5.847842in}{2.598559in}}%
\pgfpathlineto{\pgfqpoint{5.847842in}{2.598559in}}%
\pgfusepath{stroke}%
\end{pgfscope}%
\begin{pgfscope}%
\pgfsetrectcap%
\pgfsetmiterjoin%
\pgfsetlinewidth{0.803000pt}%
\definecolor{currentstroke}{rgb}{0.000000,0.000000,0.000000}%
\pgfsetstrokecolor{currentstroke}%
\pgfsetdash{}{0pt}%
\pgfpathmoveto{\pgfqpoint{0.887842in}{0.670138in}}%
\pgfpathlineto{\pgfqpoint{0.887842in}{4.366138in}}%
\pgfusepath{stroke}%
\end{pgfscope}%
\begin{pgfscope}%
\pgfsetrectcap%
\pgfsetmiterjoin%
\pgfsetlinewidth{0.803000pt}%
\definecolor{currentstroke}{rgb}{0.000000,0.000000,0.000000}%
\pgfsetstrokecolor{currentstroke}%
\pgfsetdash{}{0pt}%
\pgfpathmoveto{\pgfqpoint{5.847842in}{0.670138in}}%
\pgfpathlineto{\pgfqpoint{5.847842in}{4.366138in}}%
\pgfusepath{stroke}%
\end{pgfscope}%
\begin{pgfscope}%
\pgfsetrectcap%
\pgfsetmiterjoin%
\pgfsetlinewidth{0.803000pt}%
\definecolor{currentstroke}{rgb}{0.000000,0.000000,0.000000}%
\pgfsetstrokecolor{currentstroke}%
\pgfsetdash{}{0pt}%
\pgfpathmoveto{\pgfqpoint{0.887842in}{0.670138in}}%
\pgfpathlineto{\pgfqpoint{5.847842in}{0.670138in}}%
\pgfusepath{stroke}%
\end{pgfscope}%
\begin{pgfscope}%
\pgfsetrectcap%
\pgfsetmiterjoin%
\pgfsetlinewidth{0.803000pt}%
\definecolor{currentstroke}{rgb}{0.000000,0.000000,0.000000}%
\pgfsetstrokecolor{currentstroke}%
\pgfsetdash{}{0pt}%
\pgfpathmoveto{\pgfqpoint{0.887842in}{4.366138in}}%
\pgfpathlineto{\pgfqpoint{5.847842in}{4.366138in}}%
\pgfusepath{stroke}%
\end{pgfscope}%
\end{pgfpicture}%
\makeatother%
\endgroup%
}
    \caption{A plot of the synthetic demand data for this example over a seven day period.}
    \label{fig:demand-plot}
\end{figure}
\FloatBarrier

\subsection{Wind Speed}

Similar to demand data, the dispatch examples use synthetically generated wind
power data. First, wind speeds are drawn from a Weibull distribution
\cite{manwell_wind_2009}, given by 
\begin{align}
    U &= \left(-\ln\left(X\right)\right)^{\frac{1}{\alpha}},
    \intertext{where}
    X &= \text{A uniformly distributed random variable} \in \text{(0,1]},\nonumber\\
    \alpha &= \text{a scale factor}\quad \left[-\right].\nonumber
\end{align}
\noindent Then the wind speed data are transformed into a turbine power with 

\begin{align}
  \label{eqn:windpower}
  P_{turbine} &= \begin{cases}
    0 & U \notin [U_{\text{in}}, U_{\text{out}}]\\
    \frac{1}{2}\eta\rho U^3 \left(\frac{\pi D^2}{4}\right) & U \in [U_{\text{in}}, U_{\text{rated}}]\\
    P_{\text{rated}} & U \in [U_{\text{rated}}, U_{\text{out}}]
\end{cases}
  \intertext{where}
  P_{turbine} &= \text{power generated by the wind turbine [MW]},\nonumber\\
  P_{\text{rated}} &= \text{the rated power of the turbine [MW]},\nonumber\\
  U_{\text{in}} &= \text{the turbine cut-in speed $\left[\frac{m}{s}\right]$},\nonumber\\
  U_{\text{out}} &= \text{the turbine cut-out speed $\left[\frac{m}{s}\right]$},\nonumber\\
  U_{\text{rated}} &= \text{the turbine rated speed $\left[\frac{m}{s}\right]$},\nonumber\\
  D &= \text{wind turbine diameter [m]}\nonumber\\
  \eta &= \text{wind turbine efficiency} \approx 0.35 \text{ } [-], \nonumber\\
  U &= \text{wind speed at the hub height of the turbine $\left[\frac{m}{s}\right]$}\nonumber\\
  \rho &= \text{air density $\left[\frac{kg}{m^3}\right]$}. \nonumber
  \label{eqn:airdensity}
\end{align}
\noindent Wind turbines have three operating regimes as shown in Equation
\ref{eqn:windpower}. Turbines capture no energy at wind speeds below the cut-in
speed, and, for safety reasons, brakes are applied at wind speeds above the
cut-out wind speed and capture no energy. A wind turbine generates its rated
power between the rated and cut-out speed. Table \ref{tab:turbine} summarizes
the wind turbine data and assumptions used for this analysis. Figure
\ref{fig:wind-plot} shows the normalized wind speed and turbine power over 48
hours. 

\begin{table}[H]
  \centering
  \caption{Summary of wind turbine data and assumptions \cite{bauer_ge_2010}.}
  \label{tab:turbine}

  \resizebox{\textwidth}{!}{  \begin{tabular}{lrrrrrr|r}
    \toprule
    Turbine Model & Rated Power & Cut-in Speed & Rated Speed & Cut-out Speed &
    Rotor Height & Diameter & Air Density\\
     & [MW] & [m/s] & [m/s] & [m/s] & [m] & [m] & [kg/m$^3$]\\
    \midrule
    GE 2.75 MW Series & 2.75 & 3.0 & 13 & 25 & 98.5 & 103 & 1.225\\
    \bottomrule
  \end{tabular}}
\end{table}


\begin{figure}[ht!]
    \centering
    \resizebox{0.75\columnwidth}{!}{%% Creator: Matplotlib, PGF backend
%%
%% To include the figure in your LaTeX document, write
%%   \input{<filename>.pgf}
%%
%% Make sure the required packages are loaded in your preamble
%%   \usepackage{pgf}
%%
%% Also ensure that all the required font packages are loaded; for instance,
%% the lmodern package is sometimes necessary when using math font.
%%   \usepackage{lmodern}
%%
%% Figures using additional raster images can only be included by \input if
%% they are in the same directory as the main LaTeX file. For loading figures
%% from other directories you can use the `import` package
%%   \usepackage{import}
%%
%% and then include the figures with
%%   \import{<path to file>}{<filename>.pgf}
%%
%% Matplotlib used the following preamble
%%   \def\mathdefault#1{#1}
%%   \everymath=\expandafter{\the\everymath\displaystyle}
%%   \IfFileExists{scrextend.sty}{
%%     \usepackage[fontsize=10.000000pt]{scrextend}
%%   }{
%%     \renewcommand{\normalsize}{\fontsize{10.000000}{12.000000}\selectfont}
%%     \normalsize
%%   }
%%   
%%   \makeatletter\@ifpackageloaded{underscore}{}{\usepackage[strings]{underscore}}\makeatother
%%
\begingroup%
\makeatletter%
\begin{pgfpicture}%
\pgfpathrectangle{\pgfpointorigin}{\pgfqpoint{5.501283in}{4.116679in}}%
\pgfusepath{use as bounding box, clip}%
\begin{pgfscope}%
\pgfsetbuttcap%
\pgfsetmiterjoin%
\definecolor{currentfill}{rgb}{1.000000,1.000000,1.000000}%
\pgfsetfillcolor{currentfill}%
\pgfsetlinewidth{0.000000pt}%
\definecolor{currentstroke}{rgb}{0.000000,0.000000,0.000000}%
\pgfsetstrokecolor{currentstroke}%
\pgfsetdash{}{0pt}%
\pgfpathmoveto{\pgfqpoint{0.000000in}{0.000000in}}%
\pgfpathlineto{\pgfqpoint{5.501283in}{0.000000in}}%
\pgfpathlineto{\pgfqpoint{5.501283in}{4.116679in}}%
\pgfpathlineto{\pgfqpoint{0.000000in}{4.116679in}}%
\pgfpathlineto{\pgfqpoint{0.000000in}{0.000000in}}%
\pgfpathclose%
\pgfusepath{fill}%
\end{pgfscope}%
\begin{pgfscope}%
\pgfsetbuttcap%
\pgfsetmiterjoin%
\definecolor{currentfill}{rgb}{1.000000,1.000000,1.000000}%
\pgfsetfillcolor{currentfill}%
\pgfsetlinewidth{0.000000pt}%
\definecolor{currentstroke}{rgb}{0.000000,0.000000,0.000000}%
\pgfsetstrokecolor{currentstroke}%
\pgfsetstrokeopacity{0.000000}%
\pgfsetdash{}{0pt}%
\pgfpathmoveto{\pgfqpoint{0.374692in}{0.320679in}}%
\pgfpathlineto{\pgfqpoint{5.334692in}{0.320679in}}%
\pgfpathlineto{\pgfqpoint{5.334692in}{4.016679in}}%
\pgfpathlineto{\pgfqpoint{0.374692in}{4.016679in}}%
\pgfpathlineto{\pgfqpoint{0.374692in}{0.320679in}}%
\pgfpathclose%
\pgfusepath{fill}%
\end{pgfscope}%
\begin{pgfscope}%
\pgfpathrectangle{\pgfqpoint{0.374692in}{0.320679in}}{\pgfqpoint{4.960000in}{3.696000in}}%
\pgfusepath{clip}%
\pgfsetrectcap%
\pgfsetroundjoin%
\pgfsetlinewidth{0.803000pt}%
\definecolor{currentstroke}{rgb}{0.690196,0.690196,0.690196}%
\pgfsetstrokecolor{currentstroke}%
\pgfsetdash{}{0pt}%
\pgfpathmoveto{\pgfqpoint{0.600146in}{0.320679in}}%
\pgfpathlineto{\pgfqpoint{0.600146in}{4.016679in}}%
\pgfusepath{stroke}%
\end{pgfscope}%
\begin{pgfscope}%
\pgfsetbuttcap%
\pgfsetroundjoin%
\definecolor{currentfill}{rgb}{0.000000,0.000000,0.000000}%
\pgfsetfillcolor{currentfill}%
\pgfsetlinewidth{0.803000pt}%
\definecolor{currentstroke}{rgb}{0.000000,0.000000,0.000000}%
\pgfsetstrokecolor{currentstroke}%
\pgfsetdash{}{0pt}%
\pgfsys@defobject{currentmarker}{\pgfqpoint{0.000000in}{-0.048611in}}{\pgfqpoint{0.000000in}{0.000000in}}{%
\pgfpathmoveto{\pgfqpoint{0.000000in}{0.000000in}}%
\pgfpathlineto{\pgfqpoint{0.000000in}{-0.048611in}}%
\pgfusepath{stroke,fill}%
}%
\begin{pgfscope}%
\pgfsys@transformshift{0.600146in}{0.320679in}%
\pgfsys@useobject{currentmarker}{}%
\end{pgfscope}%
\end{pgfscope}%
\begin{pgfscope}%
\definecolor{textcolor}{rgb}{0.000000,0.000000,0.000000}%
\pgfsetstrokecolor{textcolor}%
\pgfsetfillcolor{textcolor}%
\pgftext[x=0.600146in,y=0.223457in,,top]{\color{textcolor}{\rmfamily\fontsize{10.000000}{12.000000}\selectfont\catcode`\^=\active\def^{\ifmmode\sp\else\^{}\fi}\catcode`\%=\active\def%{\%}$\mathdefault{0}$}}%
\end{pgfscope}%
\begin{pgfscope}%
\pgfpathrectangle{\pgfqpoint{0.374692in}{0.320679in}}{\pgfqpoint{4.960000in}{3.696000in}}%
\pgfusepath{clip}%
\pgfsetrectcap%
\pgfsetroundjoin%
\pgfsetlinewidth{0.803000pt}%
\definecolor{currentstroke}{rgb}{0.690196,0.690196,0.690196}%
\pgfsetstrokecolor{currentstroke}%
\pgfsetdash{}{0pt}%
\pgfpathmoveto{\pgfqpoint{1.271142in}{0.320679in}}%
\pgfpathlineto{\pgfqpoint{1.271142in}{4.016679in}}%
\pgfusepath{stroke}%
\end{pgfscope}%
\begin{pgfscope}%
\pgfsetbuttcap%
\pgfsetroundjoin%
\definecolor{currentfill}{rgb}{0.000000,0.000000,0.000000}%
\pgfsetfillcolor{currentfill}%
\pgfsetlinewidth{0.803000pt}%
\definecolor{currentstroke}{rgb}{0.000000,0.000000,0.000000}%
\pgfsetstrokecolor{currentstroke}%
\pgfsetdash{}{0pt}%
\pgfsys@defobject{currentmarker}{\pgfqpoint{0.000000in}{-0.048611in}}{\pgfqpoint{0.000000in}{0.000000in}}{%
\pgfpathmoveto{\pgfqpoint{0.000000in}{0.000000in}}%
\pgfpathlineto{\pgfqpoint{0.000000in}{-0.048611in}}%
\pgfusepath{stroke,fill}%
}%
\begin{pgfscope}%
\pgfsys@transformshift{1.271142in}{0.320679in}%
\pgfsys@useobject{currentmarker}{}%
\end{pgfscope}%
\end{pgfscope}%
\begin{pgfscope}%
\definecolor{textcolor}{rgb}{0.000000,0.000000,0.000000}%
\pgfsetstrokecolor{textcolor}%
\pgfsetfillcolor{textcolor}%
\pgftext[x=1.271142in,y=0.223457in,,top]{\color{textcolor}{\rmfamily\fontsize{10.000000}{12.000000}\selectfont\catcode`\^=\active\def^{\ifmmode\sp\else\^{}\fi}\catcode`\%=\active\def%{\%}$\mathdefault{25}$}}%
\end{pgfscope}%
\begin{pgfscope}%
\pgfpathrectangle{\pgfqpoint{0.374692in}{0.320679in}}{\pgfqpoint{4.960000in}{3.696000in}}%
\pgfusepath{clip}%
\pgfsetrectcap%
\pgfsetroundjoin%
\pgfsetlinewidth{0.803000pt}%
\definecolor{currentstroke}{rgb}{0.690196,0.690196,0.690196}%
\pgfsetstrokecolor{currentstroke}%
\pgfsetdash{}{0pt}%
\pgfpathmoveto{\pgfqpoint{1.942138in}{0.320679in}}%
\pgfpathlineto{\pgfqpoint{1.942138in}{4.016679in}}%
\pgfusepath{stroke}%
\end{pgfscope}%
\begin{pgfscope}%
\pgfsetbuttcap%
\pgfsetroundjoin%
\definecolor{currentfill}{rgb}{0.000000,0.000000,0.000000}%
\pgfsetfillcolor{currentfill}%
\pgfsetlinewidth{0.803000pt}%
\definecolor{currentstroke}{rgb}{0.000000,0.000000,0.000000}%
\pgfsetstrokecolor{currentstroke}%
\pgfsetdash{}{0pt}%
\pgfsys@defobject{currentmarker}{\pgfqpoint{0.000000in}{-0.048611in}}{\pgfqpoint{0.000000in}{0.000000in}}{%
\pgfpathmoveto{\pgfqpoint{0.000000in}{0.000000in}}%
\pgfpathlineto{\pgfqpoint{0.000000in}{-0.048611in}}%
\pgfusepath{stroke,fill}%
}%
\begin{pgfscope}%
\pgfsys@transformshift{1.942138in}{0.320679in}%
\pgfsys@useobject{currentmarker}{}%
\end{pgfscope}%
\end{pgfscope}%
\begin{pgfscope}%
\definecolor{textcolor}{rgb}{0.000000,0.000000,0.000000}%
\pgfsetstrokecolor{textcolor}%
\pgfsetfillcolor{textcolor}%
\pgftext[x=1.942138in,y=0.223457in,,top]{\color{textcolor}{\rmfamily\fontsize{10.000000}{12.000000}\selectfont\catcode`\^=\active\def^{\ifmmode\sp\else\^{}\fi}\catcode`\%=\active\def%{\%}$\mathdefault{50}$}}%
\end{pgfscope}%
\begin{pgfscope}%
\pgfpathrectangle{\pgfqpoint{0.374692in}{0.320679in}}{\pgfqpoint{4.960000in}{3.696000in}}%
\pgfusepath{clip}%
\pgfsetrectcap%
\pgfsetroundjoin%
\pgfsetlinewidth{0.803000pt}%
\definecolor{currentstroke}{rgb}{0.690196,0.690196,0.690196}%
\pgfsetstrokecolor{currentstroke}%
\pgfsetdash{}{0pt}%
\pgfpathmoveto{\pgfqpoint{2.613134in}{0.320679in}}%
\pgfpathlineto{\pgfqpoint{2.613134in}{4.016679in}}%
\pgfusepath{stroke}%
\end{pgfscope}%
\begin{pgfscope}%
\pgfsetbuttcap%
\pgfsetroundjoin%
\definecolor{currentfill}{rgb}{0.000000,0.000000,0.000000}%
\pgfsetfillcolor{currentfill}%
\pgfsetlinewidth{0.803000pt}%
\definecolor{currentstroke}{rgb}{0.000000,0.000000,0.000000}%
\pgfsetstrokecolor{currentstroke}%
\pgfsetdash{}{0pt}%
\pgfsys@defobject{currentmarker}{\pgfqpoint{0.000000in}{-0.048611in}}{\pgfqpoint{0.000000in}{0.000000in}}{%
\pgfpathmoveto{\pgfqpoint{0.000000in}{0.000000in}}%
\pgfpathlineto{\pgfqpoint{0.000000in}{-0.048611in}}%
\pgfusepath{stroke,fill}%
}%
\begin{pgfscope}%
\pgfsys@transformshift{2.613134in}{0.320679in}%
\pgfsys@useobject{currentmarker}{}%
\end{pgfscope}%
\end{pgfscope}%
\begin{pgfscope}%
\definecolor{textcolor}{rgb}{0.000000,0.000000,0.000000}%
\pgfsetstrokecolor{textcolor}%
\pgfsetfillcolor{textcolor}%
\pgftext[x=2.613134in,y=0.223457in,,top]{\color{textcolor}{\rmfamily\fontsize{10.000000}{12.000000}\selectfont\catcode`\^=\active\def^{\ifmmode\sp\else\^{}\fi}\catcode`\%=\active\def%{\%}$\mathdefault{75}$}}%
\end{pgfscope}%
\begin{pgfscope}%
\pgfpathrectangle{\pgfqpoint{0.374692in}{0.320679in}}{\pgfqpoint{4.960000in}{3.696000in}}%
\pgfusepath{clip}%
\pgfsetrectcap%
\pgfsetroundjoin%
\pgfsetlinewidth{0.803000pt}%
\definecolor{currentstroke}{rgb}{0.690196,0.690196,0.690196}%
\pgfsetstrokecolor{currentstroke}%
\pgfsetdash{}{0pt}%
\pgfpathmoveto{\pgfqpoint{3.284129in}{0.320679in}}%
\pgfpathlineto{\pgfqpoint{3.284129in}{4.016679in}}%
\pgfusepath{stroke}%
\end{pgfscope}%
\begin{pgfscope}%
\pgfsetbuttcap%
\pgfsetroundjoin%
\definecolor{currentfill}{rgb}{0.000000,0.000000,0.000000}%
\pgfsetfillcolor{currentfill}%
\pgfsetlinewidth{0.803000pt}%
\definecolor{currentstroke}{rgb}{0.000000,0.000000,0.000000}%
\pgfsetstrokecolor{currentstroke}%
\pgfsetdash{}{0pt}%
\pgfsys@defobject{currentmarker}{\pgfqpoint{0.000000in}{-0.048611in}}{\pgfqpoint{0.000000in}{0.000000in}}{%
\pgfpathmoveto{\pgfqpoint{0.000000in}{0.000000in}}%
\pgfpathlineto{\pgfqpoint{0.000000in}{-0.048611in}}%
\pgfusepath{stroke,fill}%
}%
\begin{pgfscope}%
\pgfsys@transformshift{3.284129in}{0.320679in}%
\pgfsys@useobject{currentmarker}{}%
\end{pgfscope}%
\end{pgfscope}%
\begin{pgfscope}%
\definecolor{textcolor}{rgb}{0.000000,0.000000,0.000000}%
\pgfsetstrokecolor{textcolor}%
\pgfsetfillcolor{textcolor}%
\pgftext[x=3.284129in,y=0.223457in,,top]{\color{textcolor}{\rmfamily\fontsize{10.000000}{12.000000}\selectfont\catcode`\^=\active\def^{\ifmmode\sp\else\^{}\fi}\catcode`\%=\active\def%{\%}$\mathdefault{100}$}}%
\end{pgfscope}%
\begin{pgfscope}%
\pgfpathrectangle{\pgfqpoint{0.374692in}{0.320679in}}{\pgfqpoint{4.960000in}{3.696000in}}%
\pgfusepath{clip}%
\pgfsetrectcap%
\pgfsetroundjoin%
\pgfsetlinewidth{0.803000pt}%
\definecolor{currentstroke}{rgb}{0.690196,0.690196,0.690196}%
\pgfsetstrokecolor{currentstroke}%
\pgfsetdash{}{0pt}%
\pgfpathmoveto{\pgfqpoint{3.955125in}{0.320679in}}%
\pgfpathlineto{\pgfqpoint{3.955125in}{4.016679in}}%
\pgfusepath{stroke}%
\end{pgfscope}%
\begin{pgfscope}%
\pgfsetbuttcap%
\pgfsetroundjoin%
\definecolor{currentfill}{rgb}{0.000000,0.000000,0.000000}%
\pgfsetfillcolor{currentfill}%
\pgfsetlinewidth{0.803000pt}%
\definecolor{currentstroke}{rgb}{0.000000,0.000000,0.000000}%
\pgfsetstrokecolor{currentstroke}%
\pgfsetdash{}{0pt}%
\pgfsys@defobject{currentmarker}{\pgfqpoint{0.000000in}{-0.048611in}}{\pgfqpoint{0.000000in}{0.000000in}}{%
\pgfpathmoveto{\pgfqpoint{0.000000in}{0.000000in}}%
\pgfpathlineto{\pgfqpoint{0.000000in}{-0.048611in}}%
\pgfusepath{stroke,fill}%
}%
\begin{pgfscope}%
\pgfsys@transformshift{3.955125in}{0.320679in}%
\pgfsys@useobject{currentmarker}{}%
\end{pgfscope}%
\end{pgfscope}%
\begin{pgfscope}%
\definecolor{textcolor}{rgb}{0.000000,0.000000,0.000000}%
\pgfsetstrokecolor{textcolor}%
\pgfsetfillcolor{textcolor}%
\pgftext[x=3.955125in,y=0.223457in,,top]{\color{textcolor}{\rmfamily\fontsize{10.000000}{12.000000}\selectfont\catcode`\^=\active\def^{\ifmmode\sp\else\^{}\fi}\catcode`\%=\active\def%{\%}$\mathdefault{125}$}}%
\end{pgfscope}%
\begin{pgfscope}%
\pgfpathrectangle{\pgfqpoint{0.374692in}{0.320679in}}{\pgfqpoint{4.960000in}{3.696000in}}%
\pgfusepath{clip}%
\pgfsetrectcap%
\pgfsetroundjoin%
\pgfsetlinewidth{0.803000pt}%
\definecolor{currentstroke}{rgb}{0.690196,0.690196,0.690196}%
\pgfsetstrokecolor{currentstroke}%
\pgfsetdash{}{0pt}%
\pgfpathmoveto{\pgfqpoint{4.626121in}{0.320679in}}%
\pgfpathlineto{\pgfqpoint{4.626121in}{4.016679in}}%
\pgfusepath{stroke}%
\end{pgfscope}%
\begin{pgfscope}%
\pgfsetbuttcap%
\pgfsetroundjoin%
\definecolor{currentfill}{rgb}{0.000000,0.000000,0.000000}%
\pgfsetfillcolor{currentfill}%
\pgfsetlinewidth{0.803000pt}%
\definecolor{currentstroke}{rgb}{0.000000,0.000000,0.000000}%
\pgfsetstrokecolor{currentstroke}%
\pgfsetdash{}{0pt}%
\pgfsys@defobject{currentmarker}{\pgfqpoint{0.000000in}{-0.048611in}}{\pgfqpoint{0.000000in}{0.000000in}}{%
\pgfpathmoveto{\pgfqpoint{0.000000in}{0.000000in}}%
\pgfpathlineto{\pgfqpoint{0.000000in}{-0.048611in}}%
\pgfusepath{stroke,fill}%
}%
\begin{pgfscope}%
\pgfsys@transformshift{4.626121in}{0.320679in}%
\pgfsys@useobject{currentmarker}{}%
\end{pgfscope}%
\end{pgfscope}%
\begin{pgfscope}%
\definecolor{textcolor}{rgb}{0.000000,0.000000,0.000000}%
\pgfsetstrokecolor{textcolor}%
\pgfsetfillcolor{textcolor}%
\pgftext[x=4.626121in,y=0.223457in,,top]{\color{textcolor}{\rmfamily\fontsize{10.000000}{12.000000}\selectfont\catcode`\^=\active\def^{\ifmmode\sp\else\^{}\fi}\catcode`\%=\active\def%{\%}$\mathdefault{150}$}}%
\end{pgfscope}%
\begin{pgfscope}%
\pgfpathrectangle{\pgfqpoint{0.374692in}{0.320679in}}{\pgfqpoint{4.960000in}{3.696000in}}%
\pgfusepath{clip}%
\pgfsetrectcap%
\pgfsetroundjoin%
\pgfsetlinewidth{0.803000pt}%
\definecolor{currentstroke}{rgb}{0.690196,0.690196,0.690196}%
\pgfsetstrokecolor{currentstroke}%
\pgfsetdash{}{0pt}%
\pgfpathmoveto{\pgfqpoint{5.297116in}{0.320679in}}%
\pgfpathlineto{\pgfqpoint{5.297116in}{4.016679in}}%
\pgfusepath{stroke}%
\end{pgfscope}%
\begin{pgfscope}%
\pgfsetbuttcap%
\pgfsetroundjoin%
\definecolor{currentfill}{rgb}{0.000000,0.000000,0.000000}%
\pgfsetfillcolor{currentfill}%
\pgfsetlinewidth{0.803000pt}%
\definecolor{currentstroke}{rgb}{0.000000,0.000000,0.000000}%
\pgfsetstrokecolor{currentstroke}%
\pgfsetdash{}{0pt}%
\pgfsys@defobject{currentmarker}{\pgfqpoint{0.000000in}{-0.048611in}}{\pgfqpoint{0.000000in}{0.000000in}}{%
\pgfpathmoveto{\pgfqpoint{0.000000in}{0.000000in}}%
\pgfpathlineto{\pgfqpoint{0.000000in}{-0.048611in}}%
\pgfusepath{stroke,fill}%
}%
\begin{pgfscope}%
\pgfsys@transformshift{5.297116in}{0.320679in}%
\pgfsys@useobject{currentmarker}{}%
\end{pgfscope}%
\end{pgfscope}%
\begin{pgfscope}%
\definecolor{textcolor}{rgb}{0.000000,0.000000,0.000000}%
\pgfsetstrokecolor{textcolor}%
\pgfsetfillcolor{textcolor}%
\pgftext[x=5.297116in,y=0.223457in,,top]{\color{textcolor}{\rmfamily\fontsize{10.000000}{12.000000}\selectfont\catcode`\^=\active\def^{\ifmmode\sp\else\^{}\fi}\catcode`\%=\active\def%{\%}$\mathdefault{175}$}}%
\end{pgfscope}%
\begin{pgfscope}%
\pgfpathrectangle{\pgfqpoint{0.374692in}{0.320679in}}{\pgfqpoint{4.960000in}{3.696000in}}%
\pgfusepath{clip}%
\pgfsetrectcap%
\pgfsetroundjoin%
\pgfsetlinewidth{0.803000pt}%
\definecolor{currentstroke}{rgb}{0.690196,0.690196,0.690196}%
\pgfsetstrokecolor{currentstroke}%
\pgfsetdash{}{0pt}%
\pgfpathmoveto{\pgfqpoint{0.374692in}{0.474065in}}%
\pgfpathlineto{\pgfqpoint{5.334692in}{0.474065in}}%
\pgfusepath{stroke}%
\end{pgfscope}%
\begin{pgfscope}%
\pgfsetbuttcap%
\pgfsetroundjoin%
\definecolor{currentfill}{rgb}{0.000000,0.000000,0.000000}%
\pgfsetfillcolor{currentfill}%
\pgfsetlinewidth{0.803000pt}%
\definecolor{currentstroke}{rgb}{0.000000,0.000000,0.000000}%
\pgfsetstrokecolor{currentstroke}%
\pgfsetdash{}{0pt}%
\pgfsys@defobject{currentmarker}{\pgfqpoint{-0.048611in}{0.000000in}}{\pgfqpoint{-0.000000in}{0.000000in}}{%
\pgfpathmoveto{\pgfqpoint{-0.000000in}{0.000000in}}%
\pgfpathlineto{\pgfqpoint{-0.048611in}{0.000000in}}%
\pgfusepath{stroke,fill}%
}%
\begin{pgfscope}%
\pgfsys@transformshift{0.374692in}{0.474065in}%
\pgfsys@useobject{currentmarker}{}%
\end{pgfscope}%
\end{pgfscope}%
\begin{pgfscope}%
\definecolor{textcolor}{rgb}{0.000000,0.000000,0.000000}%
\pgfsetstrokecolor{textcolor}%
\pgfsetfillcolor{textcolor}%
\pgftext[x=0.100000in, y=0.425840in, left, base]{\color{textcolor}{\rmfamily\fontsize{10.000000}{12.000000}\selectfont\catcode`\^=\active\def^{\ifmmode\sp\else\^{}\fi}\catcode`\%=\active\def%{\%}$\mathdefault{0.0}$}}%
\end{pgfscope}%
\begin{pgfscope}%
\pgfpathrectangle{\pgfqpoint{0.374692in}{0.320679in}}{\pgfqpoint{4.960000in}{3.696000in}}%
\pgfusepath{clip}%
\pgfsetrectcap%
\pgfsetroundjoin%
\pgfsetlinewidth{0.803000pt}%
\definecolor{currentstroke}{rgb}{0.690196,0.690196,0.690196}%
\pgfsetstrokecolor{currentstroke}%
\pgfsetdash{}{0pt}%
\pgfpathmoveto{\pgfqpoint{0.374692in}{1.148988in}}%
\pgfpathlineto{\pgfqpoint{5.334692in}{1.148988in}}%
\pgfusepath{stroke}%
\end{pgfscope}%
\begin{pgfscope}%
\pgfsetbuttcap%
\pgfsetroundjoin%
\definecolor{currentfill}{rgb}{0.000000,0.000000,0.000000}%
\pgfsetfillcolor{currentfill}%
\pgfsetlinewidth{0.803000pt}%
\definecolor{currentstroke}{rgb}{0.000000,0.000000,0.000000}%
\pgfsetstrokecolor{currentstroke}%
\pgfsetdash{}{0pt}%
\pgfsys@defobject{currentmarker}{\pgfqpoint{-0.048611in}{0.000000in}}{\pgfqpoint{-0.000000in}{0.000000in}}{%
\pgfpathmoveto{\pgfqpoint{-0.000000in}{0.000000in}}%
\pgfpathlineto{\pgfqpoint{-0.048611in}{0.000000in}}%
\pgfusepath{stroke,fill}%
}%
\begin{pgfscope}%
\pgfsys@transformshift{0.374692in}{1.148988in}%
\pgfsys@useobject{currentmarker}{}%
\end{pgfscope}%
\end{pgfscope}%
\begin{pgfscope}%
\definecolor{textcolor}{rgb}{0.000000,0.000000,0.000000}%
\pgfsetstrokecolor{textcolor}%
\pgfsetfillcolor{textcolor}%
\pgftext[x=0.100000in, y=1.100762in, left, base]{\color{textcolor}{\rmfamily\fontsize{10.000000}{12.000000}\selectfont\catcode`\^=\active\def^{\ifmmode\sp\else\^{}\fi}\catcode`\%=\active\def%{\%}$\mathdefault{0.2}$}}%
\end{pgfscope}%
\begin{pgfscope}%
\pgfpathrectangle{\pgfqpoint{0.374692in}{0.320679in}}{\pgfqpoint{4.960000in}{3.696000in}}%
\pgfusepath{clip}%
\pgfsetrectcap%
\pgfsetroundjoin%
\pgfsetlinewidth{0.803000pt}%
\definecolor{currentstroke}{rgb}{0.690196,0.690196,0.690196}%
\pgfsetstrokecolor{currentstroke}%
\pgfsetdash{}{0pt}%
\pgfpathmoveto{\pgfqpoint{0.374692in}{1.823910in}}%
\pgfpathlineto{\pgfqpoint{5.334692in}{1.823910in}}%
\pgfusepath{stroke}%
\end{pgfscope}%
\begin{pgfscope}%
\pgfsetbuttcap%
\pgfsetroundjoin%
\definecolor{currentfill}{rgb}{0.000000,0.000000,0.000000}%
\pgfsetfillcolor{currentfill}%
\pgfsetlinewidth{0.803000pt}%
\definecolor{currentstroke}{rgb}{0.000000,0.000000,0.000000}%
\pgfsetstrokecolor{currentstroke}%
\pgfsetdash{}{0pt}%
\pgfsys@defobject{currentmarker}{\pgfqpoint{-0.048611in}{0.000000in}}{\pgfqpoint{-0.000000in}{0.000000in}}{%
\pgfpathmoveto{\pgfqpoint{-0.000000in}{0.000000in}}%
\pgfpathlineto{\pgfqpoint{-0.048611in}{0.000000in}}%
\pgfusepath{stroke,fill}%
}%
\begin{pgfscope}%
\pgfsys@transformshift{0.374692in}{1.823910in}%
\pgfsys@useobject{currentmarker}{}%
\end{pgfscope}%
\end{pgfscope}%
\begin{pgfscope}%
\definecolor{textcolor}{rgb}{0.000000,0.000000,0.000000}%
\pgfsetstrokecolor{textcolor}%
\pgfsetfillcolor{textcolor}%
\pgftext[x=0.100000in, y=1.775685in, left, base]{\color{textcolor}{\rmfamily\fontsize{10.000000}{12.000000}\selectfont\catcode`\^=\active\def^{\ifmmode\sp\else\^{}\fi}\catcode`\%=\active\def%{\%}$\mathdefault{0.4}$}}%
\end{pgfscope}%
\begin{pgfscope}%
\pgfpathrectangle{\pgfqpoint{0.374692in}{0.320679in}}{\pgfqpoint{4.960000in}{3.696000in}}%
\pgfusepath{clip}%
\pgfsetrectcap%
\pgfsetroundjoin%
\pgfsetlinewidth{0.803000pt}%
\definecolor{currentstroke}{rgb}{0.690196,0.690196,0.690196}%
\pgfsetstrokecolor{currentstroke}%
\pgfsetdash{}{0pt}%
\pgfpathmoveto{\pgfqpoint{0.374692in}{2.498833in}}%
\pgfpathlineto{\pgfqpoint{5.334692in}{2.498833in}}%
\pgfusepath{stroke}%
\end{pgfscope}%
\begin{pgfscope}%
\pgfsetbuttcap%
\pgfsetroundjoin%
\definecolor{currentfill}{rgb}{0.000000,0.000000,0.000000}%
\pgfsetfillcolor{currentfill}%
\pgfsetlinewidth{0.803000pt}%
\definecolor{currentstroke}{rgb}{0.000000,0.000000,0.000000}%
\pgfsetstrokecolor{currentstroke}%
\pgfsetdash{}{0pt}%
\pgfsys@defobject{currentmarker}{\pgfqpoint{-0.048611in}{0.000000in}}{\pgfqpoint{-0.000000in}{0.000000in}}{%
\pgfpathmoveto{\pgfqpoint{-0.000000in}{0.000000in}}%
\pgfpathlineto{\pgfqpoint{-0.048611in}{0.000000in}}%
\pgfusepath{stroke,fill}%
}%
\begin{pgfscope}%
\pgfsys@transformshift{0.374692in}{2.498833in}%
\pgfsys@useobject{currentmarker}{}%
\end{pgfscope}%
\end{pgfscope}%
\begin{pgfscope}%
\definecolor{textcolor}{rgb}{0.000000,0.000000,0.000000}%
\pgfsetstrokecolor{textcolor}%
\pgfsetfillcolor{textcolor}%
\pgftext[x=0.100000in, y=2.450608in, left, base]{\color{textcolor}{\rmfamily\fontsize{10.000000}{12.000000}\selectfont\catcode`\^=\active\def^{\ifmmode\sp\else\^{}\fi}\catcode`\%=\active\def%{\%}$\mathdefault{0.6}$}}%
\end{pgfscope}%
\begin{pgfscope}%
\pgfpathrectangle{\pgfqpoint{0.374692in}{0.320679in}}{\pgfqpoint{4.960000in}{3.696000in}}%
\pgfusepath{clip}%
\pgfsetrectcap%
\pgfsetroundjoin%
\pgfsetlinewidth{0.803000pt}%
\definecolor{currentstroke}{rgb}{0.690196,0.690196,0.690196}%
\pgfsetstrokecolor{currentstroke}%
\pgfsetdash{}{0pt}%
\pgfpathmoveto{\pgfqpoint{0.374692in}{3.173756in}}%
\pgfpathlineto{\pgfqpoint{5.334692in}{3.173756in}}%
\pgfusepath{stroke}%
\end{pgfscope}%
\begin{pgfscope}%
\pgfsetbuttcap%
\pgfsetroundjoin%
\definecolor{currentfill}{rgb}{0.000000,0.000000,0.000000}%
\pgfsetfillcolor{currentfill}%
\pgfsetlinewidth{0.803000pt}%
\definecolor{currentstroke}{rgb}{0.000000,0.000000,0.000000}%
\pgfsetstrokecolor{currentstroke}%
\pgfsetdash{}{0pt}%
\pgfsys@defobject{currentmarker}{\pgfqpoint{-0.048611in}{0.000000in}}{\pgfqpoint{-0.000000in}{0.000000in}}{%
\pgfpathmoveto{\pgfqpoint{-0.000000in}{0.000000in}}%
\pgfpathlineto{\pgfqpoint{-0.048611in}{0.000000in}}%
\pgfusepath{stroke,fill}%
}%
\begin{pgfscope}%
\pgfsys@transformshift{0.374692in}{3.173756in}%
\pgfsys@useobject{currentmarker}{}%
\end{pgfscope}%
\end{pgfscope}%
\begin{pgfscope}%
\definecolor{textcolor}{rgb}{0.000000,0.000000,0.000000}%
\pgfsetstrokecolor{textcolor}%
\pgfsetfillcolor{textcolor}%
\pgftext[x=0.100000in, y=3.125531in, left, base]{\color{textcolor}{\rmfamily\fontsize{10.000000}{12.000000}\selectfont\catcode`\^=\active\def^{\ifmmode\sp\else\^{}\fi}\catcode`\%=\active\def%{\%}$\mathdefault{0.8}$}}%
\end{pgfscope}%
\begin{pgfscope}%
\pgfpathrectangle{\pgfqpoint{0.374692in}{0.320679in}}{\pgfqpoint{4.960000in}{3.696000in}}%
\pgfusepath{clip}%
\pgfsetrectcap%
\pgfsetroundjoin%
\pgfsetlinewidth{0.803000pt}%
\definecolor{currentstroke}{rgb}{0.690196,0.690196,0.690196}%
\pgfsetstrokecolor{currentstroke}%
\pgfsetdash{}{0pt}%
\pgfpathmoveto{\pgfqpoint{0.374692in}{3.848679in}}%
\pgfpathlineto{\pgfqpoint{5.334692in}{3.848679in}}%
\pgfusepath{stroke}%
\end{pgfscope}%
\begin{pgfscope}%
\pgfsetbuttcap%
\pgfsetroundjoin%
\definecolor{currentfill}{rgb}{0.000000,0.000000,0.000000}%
\pgfsetfillcolor{currentfill}%
\pgfsetlinewidth{0.803000pt}%
\definecolor{currentstroke}{rgb}{0.000000,0.000000,0.000000}%
\pgfsetstrokecolor{currentstroke}%
\pgfsetdash{}{0pt}%
\pgfsys@defobject{currentmarker}{\pgfqpoint{-0.048611in}{0.000000in}}{\pgfqpoint{-0.000000in}{0.000000in}}{%
\pgfpathmoveto{\pgfqpoint{-0.000000in}{0.000000in}}%
\pgfpathlineto{\pgfqpoint{-0.048611in}{0.000000in}}%
\pgfusepath{stroke,fill}%
}%
\begin{pgfscope}%
\pgfsys@transformshift{0.374692in}{3.848679in}%
\pgfsys@useobject{currentmarker}{}%
\end{pgfscope}%
\end{pgfscope}%
\begin{pgfscope}%
\definecolor{textcolor}{rgb}{0.000000,0.000000,0.000000}%
\pgfsetstrokecolor{textcolor}%
\pgfsetfillcolor{textcolor}%
\pgftext[x=0.100000in, y=3.800454in, left, base]{\color{textcolor}{\rmfamily\fontsize{10.000000}{12.000000}\selectfont\catcode`\^=\active\def^{\ifmmode\sp\else\^{}\fi}\catcode`\%=\active\def%{\%}$\mathdefault{1.0}$}}%
\end{pgfscope}%
\begin{pgfscope}%
\pgfpathrectangle{\pgfqpoint{0.374692in}{0.320679in}}{\pgfqpoint{4.960000in}{3.696000in}}%
\pgfusepath{clip}%
\pgfsetrectcap%
\pgfsetroundjoin%
\pgfsetlinewidth{1.505625pt}%
\definecolor{currentstroke}{rgb}{0.090196,0.745098,0.811765}%
\pgfsetstrokecolor{currentstroke}%
\pgfsetdash{}{0pt}%
\pgfpathmoveto{\pgfqpoint{0.600146in}{3.461565in}}%
\pgfpathlineto{\pgfqpoint{0.627147in}{0.563974in}}%
\pgfpathlineto{\pgfqpoint{0.654148in}{1.538694in}}%
\pgfpathlineto{\pgfqpoint{0.681148in}{1.305655in}}%
\pgfpathlineto{\pgfqpoint{0.708149in}{1.998115in}}%
\pgfpathlineto{\pgfqpoint{0.735149in}{1.021774in}}%
\pgfpathlineto{\pgfqpoint{0.762150in}{1.574270in}}%
\pgfpathlineto{\pgfqpoint{0.789150in}{3.282706in}}%
\pgfpathlineto{\pgfqpoint{0.816151in}{1.472954in}}%
\pgfpathlineto{\pgfqpoint{0.843151in}{2.920698in}}%
\pgfpathlineto{\pgfqpoint{0.870152in}{0.918133in}}%
\pgfpathlineto{\pgfqpoint{0.897152in}{1.810272in}}%
\pgfpathlineto{\pgfqpoint{0.924153in}{1.994653in}}%
\pgfpathlineto{\pgfqpoint{0.951154in}{1.455339in}}%
\pgfpathlineto{\pgfqpoint{0.978154in}{2.904543in}}%
\pgfpathlineto{\pgfqpoint{1.005155in}{1.591524in}}%
\pgfpathlineto{\pgfqpoint{1.032155in}{1.569491in}}%
\pgfpathlineto{\pgfqpoint{1.059156in}{2.321666in}}%
\pgfpathlineto{\pgfqpoint{1.086156in}{2.436252in}}%
\pgfpathlineto{\pgfqpoint{1.113157in}{0.807365in}}%
\pgfpathlineto{\pgfqpoint{1.140157in}{2.129907in}}%
\pgfpathlineto{\pgfqpoint{1.167158in}{1.677054in}}%
\pgfpathlineto{\pgfqpoint{1.194158in}{1.581096in}}%
\pgfpathlineto{\pgfqpoint{1.221159in}{1.433432in}}%
\pgfpathlineto{\pgfqpoint{1.248160in}{2.600253in}}%
\pgfpathlineto{\pgfqpoint{1.275160in}{1.545001in}}%
\pgfpathlineto{\pgfqpoint{1.302161in}{2.617785in}}%
\pgfpathlineto{\pgfqpoint{1.329161in}{0.550988in}}%
\pgfpathlineto{\pgfqpoint{1.356162in}{1.926139in}}%
\pgfpathlineto{\pgfqpoint{1.383162in}{1.838263in}}%
\pgfpathlineto{\pgfqpoint{1.410163in}{1.020771in}}%
\pgfpathlineto{\pgfqpoint{1.437163in}{0.832588in}}%
\pgfpathlineto{\pgfqpoint{1.464164in}{2.490890in}}%
\pgfpathlineto{\pgfqpoint{1.491164in}{1.727084in}}%
\pgfpathlineto{\pgfqpoint{1.518165in}{2.347408in}}%
\pgfpathlineto{\pgfqpoint{1.545166in}{2.572833in}}%
\pgfpathlineto{\pgfqpoint{1.572166in}{1.301782in}}%
\pgfpathlineto{\pgfqpoint{1.599167in}{2.005106in}}%
\pgfpathlineto{\pgfqpoint{1.626167in}{1.477184in}}%
\pgfpathlineto{\pgfqpoint{1.653168in}{0.623115in}}%
\pgfpathlineto{\pgfqpoint{1.680168in}{1.916391in}}%
\pgfpathlineto{\pgfqpoint{1.707169in}{0.837545in}}%
\pgfpathlineto{\pgfqpoint{1.734169in}{1.432860in}}%
\pgfpathlineto{\pgfqpoint{1.761170in}{1.768354in}}%
\pgfpathlineto{\pgfqpoint{1.788170in}{1.814012in}}%
\pgfpathlineto{\pgfqpoint{1.842172in}{1.146096in}}%
\pgfpathlineto{\pgfqpoint{1.869172in}{1.013742in}}%
\pgfpathlineto{\pgfqpoint{1.896173in}{1.427946in}}%
\pgfpathlineto{\pgfqpoint{1.923173in}{3.004538in}}%
\pgfpathlineto{\pgfqpoint{1.950174in}{0.488679in}}%
\pgfpathlineto{\pgfqpoint{1.977174in}{1.196879in}}%
\pgfpathlineto{\pgfqpoint{2.004175in}{1.736980in}}%
\pgfpathlineto{\pgfqpoint{2.031175in}{1.666648in}}%
\pgfpathlineto{\pgfqpoint{2.058176in}{1.571198in}}%
\pgfpathlineto{\pgfqpoint{2.085176in}{0.799711in}}%
\pgfpathlineto{\pgfqpoint{2.112177in}{1.053262in}}%
\pgfpathlineto{\pgfqpoint{2.139178in}{1.603080in}}%
\pgfpathlineto{\pgfqpoint{2.166178in}{2.187747in}}%
\pgfpathlineto{\pgfqpoint{2.193179in}{3.122812in}}%
\pgfpathlineto{\pgfqpoint{2.220179in}{1.812428in}}%
\pgfpathlineto{\pgfqpoint{2.247180in}{1.951989in}}%
\pgfpathlineto{\pgfqpoint{2.274180in}{1.087540in}}%
\pgfpathlineto{\pgfqpoint{2.301181in}{1.116379in}}%
\pgfpathlineto{\pgfqpoint{2.328181in}{2.274633in}}%
\pgfpathlineto{\pgfqpoint{2.355182in}{2.122529in}}%
\pgfpathlineto{\pgfqpoint{2.382182in}{3.216088in}}%
\pgfpathlineto{\pgfqpoint{2.409183in}{2.115773in}}%
\pgfpathlineto{\pgfqpoint{2.436184in}{0.663837in}}%
\pgfpathlineto{\pgfqpoint{2.463184in}{2.401050in}}%
\pgfpathlineto{\pgfqpoint{2.490185in}{1.480969in}}%
\pgfpathlineto{\pgfqpoint{2.517185in}{1.661524in}}%
\pgfpathlineto{\pgfqpoint{2.544186in}{1.858541in}}%
\pgfpathlineto{\pgfqpoint{2.571186in}{1.185820in}}%
\pgfpathlineto{\pgfqpoint{2.598187in}{2.134832in}}%
\pgfpathlineto{\pgfqpoint{2.625187in}{1.882716in}}%
\pgfpathlineto{\pgfqpoint{2.652188in}{1.881120in}}%
\pgfpathlineto{\pgfqpoint{2.679188in}{1.975589in}}%
\pgfpathlineto{\pgfqpoint{2.706189in}{2.669000in}}%
\pgfpathlineto{\pgfqpoint{2.733189in}{1.659456in}}%
\pgfpathlineto{\pgfqpoint{2.760190in}{1.561354in}}%
\pgfpathlineto{\pgfqpoint{2.787191in}{2.057557in}}%
\pgfpathlineto{\pgfqpoint{2.814191in}{2.200229in}}%
\pgfpathlineto{\pgfqpoint{2.841192in}{1.940381in}}%
\pgfpathlineto{\pgfqpoint{2.868192in}{3.049380in}}%
\pgfpathlineto{\pgfqpoint{2.895193in}{0.683436in}}%
\pgfpathlineto{\pgfqpoint{2.922193in}{0.804237in}}%
\pgfpathlineto{\pgfqpoint{2.949194in}{0.976528in}}%
\pgfpathlineto{\pgfqpoint{2.976194in}{0.592189in}}%
\pgfpathlineto{\pgfqpoint{3.003195in}{0.867841in}}%
\pgfpathlineto{\pgfqpoint{3.030195in}{3.130188in}}%
\pgfpathlineto{\pgfqpoint{3.057196in}{1.815274in}}%
\pgfpathlineto{\pgfqpoint{3.084197in}{1.469995in}}%
\pgfpathlineto{\pgfqpoint{3.111197in}{1.345564in}}%
\pgfpathlineto{\pgfqpoint{3.138198in}{2.358811in}}%
\pgfpathlineto{\pgfqpoint{3.165198in}{2.443263in}}%
\pgfpathlineto{\pgfqpoint{3.192199in}{1.577295in}}%
\pgfpathlineto{\pgfqpoint{3.219199in}{1.218688in}}%
\pgfpathlineto{\pgfqpoint{3.246200in}{0.865425in}}%
\pgfpathlineto{\pgfqpoint{3.273200in}{2.264632in}}%
\pgfpathlineto{\pgfqpoint{3.300201in}{2.952631in}}%
\pgfpathlineto{\pgfqpoint{3.327201in}{2.984230in}}%
\pgfpathlineto{\pgfqpoint{3.354202in}{1.796797in}}%
\pgfpathlineto{\pgfqpoint{3.381203in}{1.590452in}}%
\pgfpathlineto{\pgfqpoint{3.408203in}{1.760217in}}%
\pgfpathlineto{\pgfqpoint{3.435204in}{2.422082in}}%
\pgfpathlineto{\pgfqpoint{3.462204in}{1.580476in}}%
\pgfpathlineto{\pgfqpoint{3.489205in}{2.315802in}}%
\pgfpathlineto{\pgfqpoint{3.516205in}{1.312535in}}%
\pgfpathlineto{\pgfqpoint{3.543206in}{2.650038in}}%
\pgfpathlineto{\pgfqpoint{3.570206in}{0.730559in}}%
\pgfpathlineto{\pgfqpoint{3.597207in}{1.522798in}}%
\pgfpathlineto{\pgfqpoint{3.624207in}{2.162079in}}%
\pgfpathlineto{\pgfqpoint{3.651208in}{2.507019in}}%
\pgfpathlineto{\pgfqpoint{3.678209in}{2.407155in}}%
\pgfpathlineto{\pgfqpoint{3.705209in}{1.624132in}}%
\pgfpathlineto{\pgfqpoint{3.732210in}{2.398095in}}%
\pgfpathlineto{\pgfqpoint{3.759210in}{1.987991in}}%
\pgfpathlineto{\pgfqpoint{3.786211in}{1.385246in}}%
\pgfpathlineto{\pgfqpoint{3.813211in}{1.223307in}}%
\pgfpathlineto{\pgfqpoint{3.840212in}{2.545713in}}%
\pgfpathlineto{\pgfqpoint{3.867212in}{0.628220in}}%
\pgfpathlineto{\pgfqpoint{3.894213in}{1.525584in}}%
\pgfpathlineto{\pgfqpoint{3.921213in}{1.594572in}}%
\pgfpathlineto{\pgfqpoint{3.948214in}{1.417895in}}%
\pgfpathlineto{\pgfqpoint{3.975215in}{1.591787in}}%
\pgfpathlineto{\pgfqpoint{4.002215in}{2.648101in}}%
\pgfpathlineto{\pgfqpoint{4.029216in}{2.286233in}}%
\pgfpathlineto{\pgfqpoint{4.056216in}{1.290792in}}%
\pgfpathlineto{\pgfqpoint{4.083217in}{2.941523in}}%
\pgfpathlineto{\pgfqpoint{4.110217in}{2.739038in}}%
\pgfpathlineto{\pgfqpoint{4.137218in}{1.858421in}}%
\pgfpathlineto{\pgfqpoint{4.164218in}{2.383392in}}%
\pgfpathlineto{\pgfqpoint{4.191219in}{2.392084in}}%
\pgfpathlineto{\pgfqpoint{4.218219in}{2.627419in}}%
\pgfpathlineto{\pgfqpoint{4.245220in}{1.583690in}}%
\pgfpathlineto{\pgfqpoint{4.272221in}{1.200178in}}%
\pgfpathlineto{\pgfqpoint{4.299221in}{2.074306in}}%
\pgfpathlineto{\pgfqpoint{4.326222in}{1.289288in}}%
\pgfpathlineto{\pgfqpoint{4.353222in}{3.127980in}}%
\pgfpathlineto{\pgfqpoint{4.380223in}{2.514084in}}%
\pgfpathlineto{\pgfqpoint{4.407223in}{0.924324in}}%
\pgfpathlineto{\pgfqpoint{4.434224in}{2.787753in}}%
\pgfpathlineto{\pgfqpoint{4.461224in}{1.167151in}}%
\pgfpathlineto{\pgfqpoint{4.488225in}{1.991552in}}%
\pgfpathlineto{\pgfqpoint{4.515225in}{0.650867in}}%
\pgfpathlineto{\pgfqpoint{4.542226in}{3.848679in}}%
\pgfpathlineto{\pgfqpoint{4.569227in}{3.083601in}}%
\pgfpathlineto{\pgfqpoint{4.596227in}{1.741364in}}%
\pgfpathlineto{\pgfqpoint{4.623228in}{2.760924in}}%
\pgfpathlineto{\pgfqpoint{4.650228in}{1.304843in}}%
\pgfpathlineto{\pgfqpoint{4.677229in}{2.938284in}}%
\pgfpathlineto{\pgfqpoint{4.704229in}{1.665239in}}%
\pgfpathlineto{\pgfqpoint{4.731230in}{1.366895in}}%
\pgfpathlineto{\pgfqpoint{4.758230in}{1.501863in}}%
\pgfpathlineto{\pgfqpoint{4.785231in}{1.271859in}}%
\pgfpathlineto{\pgfqpoint{4.812231in}{1.603217in}}%
\pgfpathlineto{\pgfqpoint{4.839232in}{2.488697in}}%
\pgfpathlineto{\pgfqpoint{4.866232in}{1.562140in}}%
\pgfpathlineto{\pgfqpoint{4.893233in}{1.485765in}}%
\pgfpathlineto{\pgfqpoint{4.920234in}{1.880025in}}%
\pgfpathlineto{\pgfqpoint{4.947234in}{1.004881in}}%
\pgfpathlineto{\pgfqpoint{4.974235in}{1.521507in}}%
\pgfpathlineto{\pgfqpoint{5.001235in}{0.870531in}}%
\pgfpathlineto{\pgfqpoint{5.028236in}{1.236318in}}%
\pgfpathlineto{\pgfqpoint{5.055236in}{2.060329in}}%
\pgfpathlineto{\pgfqpoint{5.082237in}{1.463399in}}%
\pgfpathlineto{\pgfqpoint{5.109237in}{1.122144in}}%
\pgfpathlineto{\pgfqpoint{5.109237in}{1.122144in}}%
\pgfusepath{stroke}%
\end{pgfscope}%
\begin{pgfscope}%
\pgfsetrectcap%
\pgfsetmiterjoin%
\pgfsetlinewidth{0.803000pt}%
\definecolor{currentstroke}{rgb}{0.000000,0.000000,0.000000}%
\pgfsetstrokecolor{currentstroke}%
\pgfsetdash{}{0pt}%
\pgfpathmoveto{\pgfqpoint{0.374692in}{0.320679in}}%
\pgfpathlineto{\pgfqpoint{0.374692in}{4.016679in}}%
\pgfusepath{stroke}%
\end{pgfscope}%
\begin{pgfscope}%
\pgfsetrectcap%
\pgfsetmiterjoin%
\pgfsetlinewidth{0.803000pt}%
\definecolor{currentstroke}{rgb}{0.000000,0.000000,0.000000}%
\pgfsetstrokecolor{currentstroke}%
\pgfsetdash{}{0pt}%
\pgfpathmoveto{\pgfqpoint{5.334692in}{0.320679in}}%
\pgfpathlineto{\pgfqpoint{5.334692in}{4.016679in}}%
\pgfusepath{stroke}%
\end{pgfscope}%
\begin{pgfscope}%
\pgfsetrectcap%
\pgfsetmiterjoin%
\pgfsetlinewidth{0.803000pt}%
\definecolor{currentstroke}{rgb}{0.000000,0.000000,0.000000}%
\pgfsetstrokecolor{currentstroke}%
\pgfsetdash{}{0pt}%
\pgfpathmoveto{\pgfqpoint{0.374692in}{0.320679in}}%
\pgfpathlineto{\pgfqpoint{5.334692in}{0.320679in}}%
\pgfusepath{stroke}%
\end{pgfscope}%
\begin{pgfscope}%
\pgfsetrectcap%
\pgfsetmiterjoin%
\pgfsetlinewidth{0.803000pt}%
\definecolor{currentstroke}{rgb}{0.000000,0.000000,0.000000}%
\pgfsetstrokecolor{currentstroke}%
\pgfsetdash{}{0pt}%
\pgfpathmoveto{\pgfqpoint{0.374692in}{4.016679in}}%
\pgfpathlineto{\pgfqpoint{5.334692in}{4.016679in}}%
\pgfusepath{stroke}%
\end{pgfscope}%
\end{pgfpicture}%
\makeatother%
\endgroup%
}
    \caption{A plot of the synthetic wind speed data used for this example over
    a two day period. $\alpha = 2$}
    \label{fig:wind-plot}
\end{figure}

\FloatBarrier