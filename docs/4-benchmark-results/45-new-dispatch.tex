\subsection{Exercise 1: Validating a Simplified Approach}

This exercise considers three different cases with different technology mixes.
The first case includes natural gas and nuclear resources, the second case adds
a wind resource, and the last case adds battery storage. Each case had seven
days with an hourly resolution (168 timesteps). Both algorithms were allowed to
curtail excess energy and both were required to meet demand at all time steps.
Table \ref{tab:dispatch-results} summarizes the technologies available, the
optimizer used, and the value of the objective function. 

\begin{table}[ht!]
    \centering
    \caption{Summary results for the three dispatch test cases.}
    \label{tab:dispatch-results}
    \begin{tabular}{llllllr}
\toprule
Case & Natural Gas & Nuclear & Wind & Storage & Optimizer & Value \\
\midrule
1&\checkmark&\checkmark&&& Logical & 1.84954 \\
1&\checkmark&\checkmark&&& Optimal & 1.84954 \\
2&\checkmark&\checkmark&\checkmark&& Logical & 0.85499 \\
2&\checkmark&\checkmark&\checkmark&& Optimal & 0.85499 \\
3&\checkmark&\checkmark&\checkmark&\checkmark& Logical & 0.73009 \\
3&\checkmark&\checkmark&\checkmark&\checkmark& Optimal & 0.58797 \\
\bottomrule
\end{tabular}

\end{table}

% \noindent
The first two cases in Table \ref{tab:dispatch-results} show perfect agreement between 
the two algorithms. However, they disagree on the final case
with a battery storage technology. Figure \ref{fig:dispatch-comparison} compares
the dispatch results for the two methods. Figure
\ref{fig:dispatch-comparison}a was calculated with the logical dispatch
algorithm and Figure \ref{fig:dispatch-comparison}b was calculated with the
optimal dispatch algorithm. These plots show that the two
algorithms dispatch the same amounts of wind and nuclear energy. However, the
two algorithms differ in their usage of battery storage which causes further
differences in the dispatch of natural gas and total curtailment.

\begin{figure}[ht!]
    \centering
    \resizebox{0.95\columnwidth}{!}{%% Creator: Matplotlib, PGF backend
%%
%% To include the figure in your LaTeX document, write
%%   \input{<filename>.pgf}
%%
%% Make sure the required packages are loaded in your preamble
%%   \usepackage{pgf}
%%
%% Also ensure that all the required font packages are loaded; for instance,
%% the lmodern package is sometimes necessary when using math font.
%%   \usepackage{lmodern}
%%
%% Figures using additional raster images can only be included by \input if
%% they are in the same directory as the main LaTeX file. For loading figures
%% from other directories you can use the `import` package
%%   \usepackage{import}
%%
%% and then include the figures with
%%   \import{<path to file>}{<filename>.pgf}
%%
%% Matplotlib used the following preamble
%%   \def\mathdefault#1{#1}
%%   \everymath=\expandafter{\the\everymath\displaystyle}
%%   \IfFileExists{scrextend.sty}{
%%     \usepackage[fontsize=10.000000pt]{scrextend}
%%   }{
%%     \renewcommand{\normalsize}{\fontsize{10.000000}{12.000000}\selectfont}
%%     \normalsize
%%   }
%%   
%%   \makeatletter\@ifpackageloaded{underscore}{}{\usepackage[strings]{underscore}}\makeatother
%%
\begingroup%
\makeatletter%
\begin{pgfpicture}%
\pgfpathrectangle{\pgfpointorigin}{\pgfqpoint{9.900000in}{7.900000in}}%
\pgfusepath{use as bounding box, clip}%
\begin{pgfscope}%
\pgfsetbuttcap%
\pgfsetmiterjoin%
\definecolor{currentfill}{rgb}{1.000000,1.000000,1.000000}%
\pgfsetfillcolor{currentfill}%
\pgfsetlinewidth{0.000000pt}%
\definecolor{currentstroke}{rgb}{0.000000,0.000000,0.000000}%
\pgfsetstrokecolor{currentstroke}%
\pgfsetdash{}{0pt}%
\pgfpathmoveto{\pgfqpoint{0.000000in}{0.000000in}}%
\pgfpathlineto{\pgfqpoint{9.900000in}{0.000000in}}%
\pgfpathlineto{\pgfqpoint{9.900000in}{7.900000in}}%
\pgfpathlineto{\pgfqpoint{0.000000in}{7.900000in}}%
\pgfpathlineto{\pgfqpoint{0.000000in}{0.000000in}}%
\pgfpathclose%
\pgfusepath{fill}%
\end{pgfscope}%
\begin{pgfscope}%
\pgfsetbuttcap%
\pgfsetmiterjoin%
\definecolor{currentfill}{rgb}{1.000000,1.000000,1.000000}%
\pgfsetfillcolor{currentfill}%
\pgfsetlinewidth{0.000000pt}%
\definecolor{currentstroke}{rgb}{0.000000,0.000000,0.000000}%
\pgfsetstrokecolor{currentstroke}%
\pgfsetstrokeopacity{0.000000}%
\pgfsetdash{}{0pt}%
\pgfpathmoveto{\pgfqpoint{0.941663in}{4.334375in}}%
\pgfpathlineto{\pgfqpoint{9.800000in}{4.334375in}}%
\pgfpathlineto{\pgfqpoint{9.800000in}{7.800000in}}%
\pgfpathlineto{\pgfqpoint{0.941663in}{7.800000in}}%
\pgfpathlineto{\pgfqpoint{0.941663in}{4.334375in}}%
\pgfpathclose%
\pgfusepath{fill}%
\end{pgfscope}%
\begin{pgfscope}%
\pgfpathrectangle{\pgfqpoint{0.941663in}{4.334375in}}{\pgfqpoint{8.858337in}{3.465625in}}%
\pgfusepath{clip}%
\pgfsetrectcap%
\pgfsetroundjoin%
\pgfsetlinewidth{0.803000pt}%
\definecolor{currentstroke}{rgb}{0.690196,0.690196,0.690196}%
\pgfsetstrokecolor{currentstroke}%
\pgfsetdash{}{0pt}%
\pgfpathmoveto{\pgfqpoint{0.941663in}{4.334375in}}%
\pgfpathlineto{\pgfqpoint{0.941663in}{7.800000in}}%
\pgfusepath{stroke}%
\end{pgfscope}%
\begin{pgfscope}%
\pgfsetbuttcap%
\pgfsetroundjoin%
\definecolor{currentfill}{rgb}{0.000000,0.000000,0.000000}%
\pgfsetfillcolor{currentfill}%
\pgfsetlinewidth{0.803000pt}%
\definecolor{currentstroke}{rgb}{0.000000,0.000000,0.000000}%
\pgfsetstrokecolor{currentstroke}%
\pgfsetdash{}{0pt}%
\pgfsys@defobject{currentmarker}{\pgfqpoint{0.000000in}{-0.048611in}}{\pgfqpoint{0.000000in}{0.000000in}}{%
\pgfpathmoveto{\pgfqpoint{0.000000in}{0.000000in}}%
\pgfpathlineto{\pgfqpoint{0.000000in}{-0.048611in}}%
\pgfusepath{stroke,fill}%
}%
\begin{pgfscope}%
\pgfsys@transformshift{0.941663in}{4.334375in}%
\pgfsys@useobject{currentmarker}{}%
\end{pgfscope}%
\end{pgfscope}%
\begin{pgfscope}%
\pgfpathrectangle{\pgfqpoint{0.941663in}{4.334375in}}{\pgfqpoint{8.858337in}{3.465625in}}%
\pgfusepath{clip}%
\pgfsetrectcap%
\pgfsetroundjoin%
\pgfsetlinewidth{0.803000pt}%
\definecolor{currentstroke}{rgb}{0.690196,0.690196,0.690196}%
\pgfsetstrokecolor{currentstroke}%
\pgfsetdash{}{0pt}%
\pgfpathmoveto{\pgfqpoint{2.002542in}{4.334375in}}%
\pgfpathlineto{\pgfqpoint{2.002542in}{7.800000in}}%
\pgfusepath{stroke}%
\end{pgfscope}%
\begin{pgfscope}%
\pgfsetbuttcap%
\pgfsetroundjoin%
\definecolor{currentfill}{rgb}{0.000000,0.000000,0.000000}%
\pgfsetfillcolor{currentfill}%
\pgfsetlinewidth{0.803000pt}%
\definecolor{currentstroke}{rgb}{0.000000,0.000000,0.000000}%
\pgfsetstrokecolor{currentstroke}%
\pgfsetdash{}{0pt}%
\pgfsys@defobject{currentmarker}{\pgfqpoint{0.000000in}{-0.048611in}}{\pgfqpoint{0.000000in}{0.000000in}}{%
\pgfpathmoveto{\pgfqpoint{0.000000in}{0.000000in}}%
\pgfpathlineto{\pgfqpoint{0.000000in}{-0.048611in}}%
\pgfusepath{stroke,fill}%
}%
\begin{pgfscope}%
\pgfsys@transformshift{2.002542in}{4.334375in}%
\pgfsys@useobject{currentmarker}{}%
\end{pgfscope}%
\end{pgfscope}%
\begin{pgfscope}%
\pgfpathrectangle{\pgfqpoint{0.941663in}{4.334375in}}{\pgfqpoint{8.858337in}{3.465625in}}%
\pgfusepath{clip}%
\pgfsetrectcap%
\pgfsetroundjoin%
\pgfsetlinewidth{0.803000pt}%
\definecolor{currentstroke}{rgb}{0.690196,0.690196,0.690196}%
\pgfsetstrokecolor{currentstroke}%
\pgfsetdash{}{0pt}%
\pgfpathmoveto{\pgfqpoint{3.063420in}{4.334375in}}%
\pgfpathlineto{\pgfqpoint{3.063420in}{7.800000in}}%
\pgfusepath{stroke}%
\end{pgfscope}%
\begin{pgfscope}%
\pgfsetbuttcap%
\pgfsetroundjoin%
\definecolor{currentfill}{rgb}{0.000000,0.000000,0.000000}%
\pgfsetfillcolor{currentfill}%
\pgfsetlinewidth{0.803000pt}%
\definecolor{currentstroke}{rgb}{0.000000,0.000000,0.000000}%
\pgfsetstrokecolor{currentstroke}%
\pgfsetdash{}{0pt}%
\pgfsys@defobject{currentmarker}{\pgfqpoint{0.000000in}{-0.048611in}}{\pgfqpoint{0.000000in}{0.000000in}}{%
\pgfpathmoveto{\pgfqpoint{0.000000in}{0.000000in}}%
\pgfpathlineto{\pgfqpoint{0.000000in}{-0.048611in}}%
\pgfusepath{stroke,fill}%
}%
\begin{pgfscope}%
\pgfsys@transformshift{3.063420in}{4.334375in}%
\pgfsys@useobject{currentmarker}{}%
\end{pgfscope}%
\end{pgfscope}%
\begin{pgfscope}%
\pgfpathrectangle{\pgfqpoint{0.941663in}{4.334375in}}{\pgfqpoint{8.858337in}{3.465625in}}%
\pgfusepath{clip}%
\pgfsetrectcap%
\pgfsetroundjoin%
\pgfsetlinewidth{0.803000pt}%
\definecolor{currentstroke}{rgb}{0.690196,0.690196,0.690196}%
\pgfsetstrokecolor{currentstroke}%
\pgfsetdash{}{0pt}%
\pgfpathmoveto{\pgfqpoint{4.124299in}{4.334375in}}%
\pgfpathlineto{\pgfqpoint{4.124299in}{7.800000in}}%
\pgfusepath{stroke}%
\end{pgfscope}%
\begin{pgfscope}%
\pgfsetbuttcap%
\pgfsetroundjoin%
\definecolor{currentfill}{rgb}{0.000000,0.000000,0.000000}%
\pgfsetfillcolor{currentfill}%
\pgfsetlinewidth{0.803000pt}%
\definecolor{currentstroke}{rgb}{0.000000,0.000000,0.000000}%
\pgfsetstrokecolor{currentstroke}%
\pgfsetdash{}{0pt}%
\pgfsys@defobject{currentmarker}{\pgfqpoint{0.000000in}{-0.048611in}}{\pgfqpoint{0.000000in}{0.000000in}}{%
\pgfpathmoveto{\pgfqpoint{0.000000in}{0.000000in}}%
\pgfpathlineto{\pgfqpoint{0.000000in}{-0.048611in}}%
\pgfusepath{stroke,fill}%
}%
\begin{pgfscope}%
\pgfsys@transformshift{4.124299in}{4.334375in}%
\pgfsys@useobject{currentmarker}{}%
\end{pgfscope}%
\end{pgfscope}%
\begin{pgfscope}%
\pgfpathrectangle{\pgfqpoint{0.941663in}{4.334375in}}{\pgfqpoint{8.858337in}{3.465625in}}%
\pgfusepath{clip}%
\pgfsetrectcap%
\pgfsetroundjoin%
\pgfsetlinewidth{0.803000pt}%
\definecolor{currentstroke}{rgb}{0.690196,0.690196,0.690196}%
\pgfsetstrokecolor{currentstroke}%
\pgfsetdash{}{0pt}%
\pgfpathmoveto{\pgfqpoint{5.185178in}{4.334375in}}%
\pgfpathlineto{\pgfqpoint{5.185178in}{7.800000in}}%
\pgfusepath{stroke}%
\end{pgfscope}%
\begin{pgfscope}%
\pgfsetbuttcap%
\pgfsetroundjoin%
\definecolor{currentfill}{rgb}{0.000000,0.000000,0.000000}%
\pgfsetfillcolor{currentfill}%
\pgfsetlinewidth{0.803000pt}%
\definecolor{currentstroke}{rgb}{0.000000,0.000000,0.000000}%
\pgfsetstrokecolor{currentstroke}%
\pgfsetdash{}{0pt}%
\pgfsys@defobject{currentmarker}{\pgfqpoint{0.000000in}{-0.048611in}}{\pgfqpoint{0.000000in}{0.000000in}}{%
\pgfpathmoveto{\pgfqpoint{0.000000in}{0.000000in}}%
\pgfpathlineto{\pgfqpoint{0.000000in}{-0.048611in}}%
\pgfusepath{stroke,fill}%
}%
\begin{pgfscope}%
\pgfsys@transformshift{5.185178in}{4.334375in}%
\pgfsys@useobject{currentmarker}{}%
\end{pgfscope}%
\end{pgfscope}%
\begin{pgfscope}%
\pgfpathrectangle{\pgfqpoint{0.941663in}{4.334375in}}{\pgfqpoint{8.858337in}{3.465625in}}%
\pgfusepath{clip}%
\pgfsetrectcap%
\pgfsetroundjoin%
\pgfsetlinewidth{0.803000pt}%
\definecolor{currentstroke}{rgb}{0.690196,0.690196,0.690196}%
\pgfsetstrokecolor{currentstroke}%
\pgfsetdash{}{0pt}%
\pgfpathmoveto{\pgfqpoint{6.246056in}{4.334375in}}%
\pgfpathlineto{\pgfqpoint{6.246056in}{7.800000in}}%
\pgfusepath{stroke}%
\end{pgfscope}%
\begin{pgfscope}%
\pgfsetbuttcap%
\pgfsetroundjoin%
\definecolor{currentfill}{rgb}{0.000000,0.000000,0.000000}%
\pgfsetfillcolor{currentfill}%
\pgfsetlinewidth{0.803000pt}%
\definecolor{currentstroke}{rgb}{0.000000,0.000000,0.000000}%
\pgfsetstrokecolor{currentstroke}%
\pgfsetdash{}{0pt}%
\pgfsys@defobject{currentmarker}{\pgfqpoint{0.000000in}{-0.048611in}}{\pgfqpoint{0.000000in}{0.000000in}}{%
\pgfpathmoveto{\pgfqpoint{0.000000in}{0.000000in}}%
\pgfpathlineto{\pgfqpoint{0.000000in}{-0.048611in}}%
\pgfusepath{stroke,fill}%
}%
\begin{pgfscope}%
\pgfsys@transformshift{6.246056in}{4.334375in}%
\pgfsys@useobject{currentmarker}{}%
\end{pgfscope}%
\end{pgfscope}%
\begin{pgfscope}%
\pgfpathrectangle{\pgfqpoint{0.941663in}{4.334375in}}{\pgfqpoint{8.858337in}{3.465625in}}%
\pgfusepath{clip}%
\pgfsetrectcap%
\pgfsetroundjoin%
\pgfsetlinewidth{0.803000pt}%
\definecolor{currentstroke}{rgb}{0.690196,0.690196,0.690196}%
\pgfsetstrokecolor{currentstroke}%
\pgfsetdash{}{0pt}%
\pgfpathmoveto{\pgfqpoint{7.306935in}{4.334375in}}%
\pgfpathlineto{\pgfqpoint{7.306935in}{7.800000in}}%
\pgfusepath{stroke}%
\end{pgfscope}%
\begin{pgfscope}%
\pgfsetbuttcap%
\pgfsetroundjoin%
\definecolor{currentfill}{rgb}{0.000000,0.000000,0.000000}%
\pgfsetfillcolor{currentfill}%
\pgfsetlinewidth{0.803000pt}%
\definecolor{currentstroke}{rgb}{0.000000,0.000000,0.000000}%
\pgfsetstrokecolor{currentstroke}%
\pgfsetdash{}{0pt}%
\pgfsys@defobject{currentmarker}{\pgfqpoint{0.000000in}{-0.048611in}}{\pgfqpoint{0.000000in}{0.000000in}}{%
\pgfpathmoveto{\pgfqpoint{0.000000in}{0.000000in}}%
\pgfpathlineto{\pgfqpoint{0.000000in}{-0.048611in}}%
\pgfusepath{stroke,fill}%
}%
\begin{pgfscope}%
\pgfsys@transformshift{7.306935in}{4.334375in}%
\pgfsys@useobject{currentmarker}{}%
\end{pgfscope}%
\end{pgfscope}%
\begin{pgfscope}%
\pgfpathrectangle{\pgfqpoint{0.941663in}{4.334375in}}{\pgfqpoint{8.858337in}{3.465625in}}%
\pgfusepath{clip}%
\pgfsetrectcap%
\pgfsetroundjoin%
\pgfsetlinewidth{0.803000pt}%
\definecolor{currentstroke}{rgb}{0.690196,0.690196,0.690196}%
\pgfsetstrokecolor{currentstroke}%
\pgfsetdash{}{0pt}%
\pgfpathmoveto{\pgfqpoint{8.367814in}{4.334375in}}%
\pgfpathlineto{\pgfqpoint{8.367814in}{7.800000in}}%
\pgfusepath{stroke}%
\end{pgfscope}%
\begin{pgfscope}%
\pgfsetbuttcap%
\pgfsetroundjoin%
\definecolor{currentfill}{rgb}{0.000000,0.000000,0.000000}%
\pgfsetfillcolor{currentfill}%
\pgfsetlinewidth{0.803000pt}%
\definecolor{currentstroke}{rgb}{0.000000,0.000000,0.000000}%
\pgfsetstrokecolor{currentstroke}%
\pgfsetdash{}{0pt}%
\pgfsys@defobject{currentmarker}{\pgfqpoint{0.000000in}{-0.048611in}}{\pgfqpoint{0.000000in}{0.000000in}}{%
\pgfpathmoveto{\pgfqpoint{0.000000in}{0.000000in}}%
\pgfpathlineto{\pgfqpoint{0.000000in}{-0.048611in}}%
\pgfusepath{stroke,fill}%
}%
\begin{pgfscope}%
\pgfsys@transformshift{8.367814in}{4.334375in}%
\pgfsys@useobject{currentmarker}{}%
\end{pgfscope}%
\end{pgfscope}%
\begin{pgfscope}%
\pgfpathrectangle{\pgfqpoint{0.941663in}{4.334375in}}{\pgfqpoint{8.858337in}{3.465625in}}%
\pgfusepath{clip}%
\pgfsetrectcap%
\pgfsetroundjoin%
\pgfsetlinewidth{0.803000pt}%
\definecolor{currentstroke}{rgb}{0.690196,0.690196,0.690196}%
\pgfsetstrokecolor{currentstroke}%
\pgfsetdash{}{0pt}%
\pgfpathmoveto{\pgfqpoint{9.428692in}{4.334375in}}%
\pgfpathlineto{\pgfqpoint{9.428692in}{7.800000in}}%
\pgfusepath{stroke}%
\end{pgfscope}%
\begin{pgfscope}%
\pgfsetbuttcap%
\pgfsetroundjoin%
\definecolor{currentfill}{rgb}{0.000000,0.000000,0.000000}%
\pgfsetfillcolor{currentfill}%
\pgfsetlinewidth{0.803000pt}%
\definecolor{currentstroke}{rgb}{0.000000,0.000000,0.000000}%
\pgfsetstrokecolor{currentstroke}%
\pgfsetdash{}{0pt}%
\pgfsys@defobject{currentmarker}{\pgfqpoint{0.000000in}{-0.048611in}}{\pgfqpoint{0.000000in}{0.000000in}}{%
\pgfpathmoveto{\pgfqpoint{0.000000in}{0.000000in}}%
\pgfpathlineto{\pgfqpoint{0.000000in}{-0.048611in}}%
\pgfusepath{stroke,fill}%
}%
\begin{pgfscope}%
\pgfsys@transformshift{9.428692in}{4.334375in}%
\pgfsys@useobject{currentmarker}{}%
\end{pgfscope}%
\end{pgfscope}%
\begin{pgfscope}%
\pgfpathrectangle{\pgfqpoint{0.941663in}{4.334375in}}{\pgfqpoint{8.858337in}{3.465625in}}%
\pgfusepath{clip}%
\pgfsetrectcap%
\pgfsetroundjoin%
\pgfsetlinewidth{0.803000pt}%
\definecolor{currentstroke}{rgb}{0.690196,0.690196,0.690196}%
\pgfsetstrokecolor{currentstroke}%
\pgfsetdash{}{0pt}%
\pgfpathmoveto{\pgfqpoint{0.941663in}{4.859785in}}%
\pgfpathlineto{\pgfqpoint{9.800000in}{4.859785in}}%
\pgfusepath{stroke}%
\end{pgfscope}%
\begin{pgfscope}%
\pgfsetbuttcap%
\pgfsetroundjoin%
\definecolor{currentfill}{rgb}{0.000000,0.000000,0.000000}%
\pgfsetfillcolor{currentfill}%
\pgfsetlinewidth{0.803000pt}%
\definecolor{currentstroke}{rgb}{0.000000,0.000000,0.000000}%
\pgfsetstrokecolor{currentstroke}%
\pgfsetdash{}{0pt}%
\pgfsys@defobject{currentmarker}{\pgfqpoint{-0.048611in}{0.000000in}}{\pgfqpoint{-0.000000in}{0.000000in}}{%
\pgfpathmoveto{\pgfqpoint{-0.000000in}{0.000000in}}%
\pgfpathlineto{\pgfqpoint{-0.048611in}{0.000000in}}%
\pgfusepath{stroke,fill}%
}%
\begin{pgfscope}%
\pgfsys@transformshift{0.941663in}{4.859785in}%
\pgfsys@useobject{currentmarker}{}%
\end{pgfscope}%
\end{pgfscope}%
\begin{pgfscope}%
\definecolor{textcolor}{rgb}{0.000000,0.000000,0.000000}%
\pgfsetstrokecolor{textcolor}%
\pgfsetfillcolor{textcolor}%
\pgftext[x=0.395138in, y=4.790340in, left, base]{\color{textcolor}{\rmfamily\fontsize{14.000000}{16.800000}\selectfont\catcode`\^=\active\def^{\ifmmode\sp\else\^{}\fi}\catcode`\%=\active\def%{\%}$\mathdefault{\ensuremath{-}500}$}}%
\end{pgfscope}%
\begin{pgfscope}%
\pgfpathrectangle{\pgfqpoint{0.941663in}{4.334375in}}{\pgfqpoint{8.858337in}{3.465625in}}%
\pgfusepath{clip}%
\pgfsetrectcap%
\pgfsetroundjoin%
\pgfsetlinewidth{0.803000pt}%
\definecolor{currentstroke}{rgb}{0.690196,0.690196,0.690196}%
\pgfsetstrokecolor{currentstroke}%
\pgfsetdash{}{0pt}%
\pgfpathmoveto{\pgfqpoint{0.941663in}{5.555456in}}%
\pgfpathlineto{\pgfqpoint{9.800000in}{5.555456in}}%
\pgfusepath{stroke}%
\end{pgfscope}%
\begin{pgfscope}%
\pgfsetbuttcap%
\pgfsetroundjoin%
\definecolor{currentfill}{rgb}{0.000000,0.000000,0.000000}%
\pgfsetfillcolor{currentfill}%
\pgfsetlinewidth{0.803000pt}%
\definecolor{currentstroke}{rgb}{0.000000,0.000000,0.000000}%
\pgfsetstrokecolor{currentstroke}%
\pgfsetdash{}{0pt}%
\pgfsys@defobject{currentmarker}{\pgfqpoint{-0.048611in}{0.000000in}}{\pgfqpoint{-0.000000in}{0.000000in}}{%
\pgfpathmoveto{\pgfqpoint{-0.000000in}{0.000000in}}%
\pgfpathlineto{\pgfqpoint{-0.048611in}{0.000000in}}%
\pgfusepath{stroke,fill}%
}%
\begin{pgfscope}%
\pgfsys@transformshift{0.941663in}{5.555456in}%
\pgfsys@useobject{currentmarker}{}%
\end{pgfscope}%
\end{pgfscope}%
\begin{pgfscope}%
\definecolor{textcolor}{rgb}{0.000000,0.000000,0.000000}%
\pgfsetstrokecolor{textcolor}%
\pgfsetfillcolor{textcolor}%
\pgftext[x=0.746525in, y=5.486012in, left, base]{\color{textcolor}{\rmfamily\fontsize{14.000000}{16.800000}\selectfont\catcode`\^=\active\def^{\ifmmode\sp\else\^{}\fi}\catcode`\%=\active\def%{\%}$\mathdefault{0}$}}%
\end{pgfscope}%
\begin{pgfscope}%
\pgfpathrectangle{\pgfqpoint{0.941663in}{4.334375in}}{\pgfqpoint{8.858337in}{3.465625in}}%
\pgfusepath{clip}%
\pgfsetrectcap%
\pgfsetroundjoin%
\pgfsetlinewidth{0.803000pt}%
\definecolor{currentstroke}{rgb}{0.690196,0.690196,0.690196}%
\pgfsetstrokecolor{currentstroke}%
\pgfsetdash{}{0pt}%
\pgfpathmoveto{\pgfqpoint{0.941663in}{6.251128in}}%
\pgfpathlineto{\pgfqpoint{9.800000in}{6.251128in}}%
\pgfusepath{stroke}%
\end{pgfscope}%
\begin{pgfscope}%
\pgfsetbuttcap%
\pgfsetroundjoin%
\definecolor{currentfill}{rgb}{0.000000,0.000000,0.000000}%
\pgfsetfillcolor{currentfill}%
\pgfsetlinewidth{0.803000pt}%
\definecolor{currentstroke}{rgb}{0.000000,0.000000,0.000000}%
\pgfsetstrokecolor{currentstroke}%
\pgfsetdash{}{0pt}%
\pgfsys@defobject{currentmarker}{\pgfqpoint{-0.048611in}{0.000000in}}{\pgfqpoint{-0.000000in}{0.000000in}}{%
\pgfpathmoveto{\pgfqpoint{-0.000000in}{0.000000in}}%
\pgfpathlineto{\pgfqpoint{-0.048611in}{0.000000in}}%
\pgfusepath{stroke,fill}%
}%
\begin{pgfscope}%
\pgfsys@transformshift{0.941663in}{6.251128in}%
\pgfsys@useobject{currentmarker}{}%
\end{pgfscope}%
\end{pgfscope}%
\begin{pgfscope}%
\definecolor{textcolor}{rgb}{0.000000,0.000000,0.000000}%
\pgfsetstrokecolor{textcolor}%
\pgfsetfillcolor{textcolor}%
\pgftext[x=0.550694in, y=6.181684in, left, base]{\color{textcolor}{\rmfamily\fontsize{14.000000}{16.800000}\selectfont\catcode`\^=\active\def^{\ifmmode\sp\else\^{}\fi}\catcode`\%=\active\def%{\%}$\mathdefault{500}$}}%
\end{pgfscope}%
\begin{pgfscope}%
\pgfpathrectangle{\pgfqpoint{0.941663in}{4.334375in}}{\pgfqpoint{8.858337in}{3.465625in}}%
\pgfusepath{clip}%
\pgfsetrectcap%
\pgfsetroundjoin%
\pgfsetlinewidth{0.803000pt}%
\definecolor{currentstroke}{rgb}{0.690196,0.690196,0.690196}%
\pgfsetstrokecolor{currentstroke}%
\pgfsetdash{}{0pt}%
\pgfpathmoveto{\pgfqpoint{0.941663in}{6.946800in}}%
\pgfpathlineto{\pgfqpoint{9.800000in}{6.946800in}}%
\pgfusepath{stroke}%
\end{pgfscope}%
\begin{pgfscope}%
\pgfsetbuttcap%
\pgfsetroundjoin%
\definecolor{currentfill}{rgb}{0.000000,0.000000,0.000000}%
\pgfsetfillcolor{currentfill}%
\pgfsetlinewidth{0.803000pt}%
\definecolor{currentstroke}{rgb}{0.000000,0.000000,0.000000}%
\pgfsetstrokecolor{currentstroke}%
\pgfsetdash{}{0pt}%
\pgfsys@defobject{currentmarker}{\pgfqpoint{-0.048611in}{0.000000in}}{\pgfqpoint{-0.000000in}{0.000000in}}{%
\pgfpathmoveto{\pgfqpoint{-0.000000in}{0.000000in}}%
\pgfpathlineto{\pgfqpoint{-0.048611in}{0.000000in}}%
\pgfusepath{stroke,fill}%
}%
\begin{pgfscope}%
\pgfsys@transformshift{0.941663in}{6.946800in}%
\pgfsys@useobject{currentmarker}{}%
\end{pgfscope}%
\end{pgfscope}%
\begin{pgfscope}%
\definecolor{textcolor}{rgb}{0.000000,0.000000,0.000000}%
\pgfsetstrokecolor{textcolor}%
\pgfsetfillcolor{textcolor}%
\pgftext[x=0.452779in, y=6.877356in, left, base]{\color{textcolor}{\rmfamily\fontsize{14.000000}{16.800000}\selectfont\catcode`\^=\active\def^{\ifmmode\sp\else\^{}\fi}\catcode`\%=\active\def%{\%}$\mathdefault{1000}$}}%
\end{pgfscope}%
\begin{pgfscope}%
\pgfpathrectangle{\pgfqpoint{0.941663in}{4.334375in}}{\pgfqpoint{8.858337in}{3.465625in}}%
\pgfusepath{clip}%
\pgfsetrectcap%
\pgfsetroundjoin%
\pgfsetlinewidth{0.803000pt}%
\definecolor{currentstroke}{rgb}{0.690196,0.690196,0.690196}%
\pgfsetstrokecolor{currentstroke}%
\pgfsetdash{}{0pt}%
\pgfpathmoveto{\pgfqpoint{0.941663in}{7.642472in}}%
\pgfpathlineto{\pgfqpoint{9.800000in}{7.642472in}}%
\pgfusepath{stroke}%
\end{pgfscope}%
\begin{pgfscope}%
\pgfsetbuttcap%
\pgfsetroundjoin%
\definecolor{currentfill}{rgb}{0.000000,0.000000,0.000000}%
\pgfsetfillcolor{currentfill}%
\pgfsetlinewidth{0.803000pt}%
\definecolor{currentstroke}{rgb}{0.000000,0.000000,0.000000}%
\pgfsetstrokecolor{currentstroke}%
\pgfsetdash{}{0pt}%
\pgfsys@defobject{currentmarker}{\pgfqpoint{-0.048611in}{0.000000in}}{\pgfqpoint{-0.000000in}{0.000000in}}{%
\pgfpathmoveto{\pgfqpoint{-0.000000in}{0.000000in}}%
\pgfpathlineto{\pgfqpoint{-0.048611in}{0.000000in}}%
\pgfusepath{stroke,fill}%
}%
\begin{pgfscope}%
\pgfsys@transformshift{0.941663in}{7.642472in}%
\pgfsys@useobject{currentmarker}{}%
\end{pgfscope}%
\end{pgfscope}%
\begin{pgfscope}%
\definecolor{textcolor}{rgb}{0.000000,0.000000,0.000000}%
\pgfsetstrokecolor{textcolor}%
\pgfsetfillcolor{textcolor}%
\pgftext[x=0.452779in, y=7.573027in, left, base]{\color{textcolor}{\rmfamily\fontsize{14.000000}{16.800000}\selectfont\catcode`\^=\active\def^{\ifmmode\sp\else\^{}\fi}\catcode`\%=\active\def%{\%}$\mathdefault{1500}$}}%
\end{pgfscope}%
\begin{pgfscope}%
\definecolor{textcolor}{rgb}{0.000000,0.000000,0.000000}%
\pgfsetstrokecolor{textcolor}%
\pgfsetfillcolor{textcolor}%
\pgftext[x=0.339583in,y=6.067187in,,bottom,rotate=90.000000]{\color{textcolor}{\rmfamily\fontsize{18.000000}{21.600000}\selectfont\catcode`\^=\active\def^{\ifmmode\sp\else\^{}\fi}\catcode`\%=\active\def%{\%}Energy [MWh]}}%
\end{pgfscope}%
\begin{pgfscope}%
\pgfpathrectangle{\pgfqpoint{0.941663in}{4.334375in}}{\pgfqpoint{8.858337in}{3.465625in}}%
\pgfusepath{clip}%
\pgfsetrectcap%
\pgfsetroundjoin%
\pgfsetlinewidth{1.505625pt}%
\definecolor{currentstroke}{rgb}{0.121569,0.466667,0.705882}%
\pgfsetstrokecolor{currentstroke}%
\pgfsetdash{}{0pt}%
\pgfpathmoveto{\pgfqpoint{0.941663in}{6.251128in}}%
\pgfpathlineto{\pgfqpoint{9.800000in}{6.251128in}}%
\pgfpathlineto{\pgfqpoint{9.800000in}{6.251128in}}%
\pgfusepath{stroke}%
\end{pgfscope}%
\begin{pgfscope}%
\pgfpathrectangle{\pgfqpoint{0.941663in}{4.334375in}}{\pgfqpoint{8.858337in}{3.465625in}}%
\pgfusepath{clip}%
\pgfsetbuttcap%
\pgfsetroundjoin%
\definecolor{currentfill}{rgb}{0.121569,0.466667,0.705882}%
\pgfsetfillcolor{currentfill}%
\pgfsetlinewidth{1.003750pt}%
\definecolor{currentstroke}{rgb}{0.121569,0.466667,0.705882}%
\pgfsetstrokecolor{currentstroke}%
\pgfsetdash{}{0pt}%
\pgfsys@defobject{currentmarker}{\pgfqpoint{0.941663in}{5.555456in}}{\pgfqpoint{9.800000in}{6.251128in}}{%
\pgfpathmoveto{\pgfqpoint{0.941663in}{6.251128in}}%
\pgfpathlineto{\pgfqpoint{0.941663in}{5.555456in}}%
\pgfpathlineto{\pgfqpoint{0.994707in}{5.555456in}}%
\pgfpathlineto{\pgfqpoint{1.047751in}{5.555456in}}%
\pgfpathlineto{\pgfqpoint{1.100795in}{5.555456in}}%
\pgfpathlineto{\pgfqpoint{1.153839in}{5.555456in}}%
\pgfpathlineto{\pgfqpoint{1.206883in}{5.555456in}}%
\pgfpathlineto{\pgfqpoint{1.259927in}{5.555456in}}%
\pgfpathlineto{\pgfqpoint{1.312970in}{5.555456in}}%
\pgfpathlineto{\pgfqpoint{1.366014in}{5.555456in}}%
\pgfpathlineto{\pgfqpoint{1.419058in}{5.555456in}}%
\pgfpathlineto{\pgfqpoint{1.472102in}{5.555456in}}%
\pgfpathlineto{\pgfqpoint{1.525146in}{5.555456in}}%
\pgfpathlineto{\pgfqpoint{1.578190in}{5.555456in}}%
\pgfpathlineto{\pgfqpoint{1.631234in}{5.555456in}}%
\pgfpathlineto{\pgfqpoint{1.684278in}{5.555456in}}%
\pgfpathlineto{\pgfqpoint{1.737322in}{5.555456in}}%
\pgfpathlineto{\pgfqpoint{1.790366in}{5.555456in}}%
\pgfpathlineto{\pgfqpoint{1.843410in}{5.555456in}}%
\pgfpathlineto{\pgfqpoint{1.896454in}{5.555456in}}%
\pgfpathlineto{\pgfqpoint{1.949498in}{5.555456in}}%
\pgfpathlineto{\pgfqpoint{2.002542in}{5.555456in}}%
\pgfpathlineto{\pgfqpoint{2.055586in}{5.555456in}}%
\pgfpathlineto{\pgfqpoint{2.108629in}{5.555456in}}%
\pgfpathlineto{\pgfqpoint{2.161673in}{5.555456in}}%
\pgfpathlineto{\pgfqpoint{2.214717in}{5.555456in}}%
\pgfpathlineto{\pgfqpoint{2.267761in}{5.555456in}}%
\pgfpathlineto{\pgfqpoint{2.320805in}{5.555456in}}%
\pgfpathlineto{\pgfqpoint{2.373849in}{5.555456in}}%
\pgfpathlineto{\pgfqpoint{2.426893in}{5.555456in}}%
\pgfpathlineto{\pgfqpoint{2.479937in}{5.555456in}}%
\pgfpathlineto{\pgfqpoint{2.532981in}{5.555456in}}%
\pgfpathlineto{\pgfqpoint{2.586025in}{5.555456in}}%
\pgfpathlineto{\pgfqpoint{2.639069in}{5.555456in}}%
\pgfpathlineto{\pgfqpoint{2.692113in}{5.555456in}}%
\pgfpathlineto{\pgfqpoint{2.745157in}{5.555456in}}%
\pgfpathlineto{\pgfqpoint{2.798201in}{5.555456in}}%
\pgfpathlineto{\pgfqpoint{2.851245in}{5.555456in}}%
\pgfpathlineto{\pgfqpoint{2.904288in}{5.555456in}}%
\pgfpathlineto{\pgfqpoint{2.957332in}{5.555456in}}%
\pgfpathlineto{\pgfqpoint{3.010376in}{5.555456in}}%
\pgfpathlineto{\pgfqpoint{3.063420in}{5.555456in}}%
\pgfpathlineto{\pgfqpoint{3.116464in}{5.555456in}}%
\pgfpathlineto{\pgfqpoint{3.169508in}{5.555456in}}%
\pgfpathlineto{\pgfqpoint{3.222552in}{5.555456in}}%
\pgfpathlineto{\pgfqpoint{3.275596in}{5.555456in}}%
\pgfpathlineto{\pgfqpoint{3.328640in}{5.555456in}}%
\pgfpathlineto{\pgfqpoint{3.381684in}{5.555456in}}%
\pgfpathlineto{\pgfqpoint{3.434728in}{5.555456in}}%
\pgfpathlineto{\pgfqpoint{3.487772in}{5.555456in}}%
\pgfpathlineto{\pgfqpoint{3.540816in}{5.555456in}}%
\pgfpathlineto{\pgfqpoint{3.593860in}{5.555456in}}%
\pgfpathlineto{\pgfqpoint{3.646904in}{5.555456in}}%
\pgfpathlineto{\pgfqpoint{3.699948in}{5.555456in}}%
\pgfpathlineto{\pgfqpoint{3.752991in}{5.555456in}}%
\pgfpathlineto{\pgfqpoint{3.806035in}{5.555456in}}%
\pgfpathlineto{\pgfqpoint{3.859079in}{5.555456in}}%
\pgfpathlineto{\pgfqpoint{3.912123in}{5.555456in}}%
\pgfpathlineto{\pgfqpoint{3.965167in}{5.555456in}}%
\pgfpathlineto{\pgfqpoint{4.018211in}{5.555456in}}%
\pgfpathlineto{\pgfqpoint{4.071255in}{5.555456in}}%
\pgfpathlineto{\pgfqpoint{4.124299in}{5.555456in}}%
\pgfpathlineto{\pgfqpoint{4.177343in}{5.555456in}}%
\pgfpathlineto{\pgfqpoint{4.230387in}{5.555456in}}%
\pgfpathlineto{\pgfqpoint{4.283431in}{5.555456in}}%
\pgfpathlineto{\pgfqpoint{4.336475in}{5.555456in}}%
\pgfpathlineto{\pgfqpoint{4.389519in}{5.555456in}}%
\pgfpathlineto{\pgfqpoint{4.442563in}{5.555456in}}%
\pgfpathlineto{\pgfqpoint{4.495607in}{5.555456in}}%
\pgfpathlineto{\pgfqpoint{4.548650in}{5.555456in}}%
\pgfpathlineto{\pgfqpoint{4.601694in}{5.555456in}}%
\pgfpathlineto{\pgfqpoint{4.654738in}{5.555456in}}%
\pgfpathlineto{\pgfqpoint{4.707782in}{5.555456in}}%
\pgfpathlineto{\pgfqpoint{4.760826in}{5.555456in}}%
\pgfpathlineto{\pgfqpoint{4.813870in}{5.555456in}}%
\pgfpathlineto{\pgfqpoint{4.866914in}{5.555456in}}%
\pgfpathlineto{\pgfqpoint{4.919958in}{5.555456in}}%
\pgfpathlineto{\pgfqpoint{4.973002in}{5.555456in}}%
\pgfpathlineto{\pgfqpoint{5.026046in}{5.555456in}}%
\pgfpathlineto{\pgfqpoint{5.079090in}{5.555456in}}%
\pgfpathlineto{\pgfqpoint{5.132134in}{5.555456in}}%
\pgfpathlineto{\pgfqpoint{5.185178in}{5.555456in}}%
\pgfpathlineto{\pgfqpoint{5.238222in}{5.555456in}}%
\pgfpathlineto{\pgfqpoint{5.291266in}{5.555456in}}%
\pgfpathlineto{\pgfqpoint{5.344309in}{5.555456in}}%
\pgfpathlineto{\pgfqpoint{5.397353in}{5.555456in}}%
\pgfpathlineto{\pgfqpoint{5.450397in}{5.555456in}}%
\pgfpathlineto{\pgfqpoint{5.503441in}{5.555456in}}%
\pgfpathlineto{\pgfqpoint{5.556485in}{5.555456in}}%
\pgfpathlineto{\pgfqpoint{5.609529in}{5.555456in}}%
\pgfpathlineto{\pgfqpoint{5.662573in}{5.555456in}}%
\pgfpathlineto{\pgfqpoint{5.715617in}{5.555456in}}%
\pgfpathlineto{\pgfqpoint{5.768661in}{5.555456in}}%
\pgfpathlineto{\pgfqpoint{5.821705in}{5.555456in}}%
\pgfpathlineto{\pgfqpoint{5.874749in}{5.555456in}}%
\pgfpathlineto{\pgfqpoint{5.927793in}{5.555456in}}%
\pgfpathlineto{\pgfqpoint{5.980837in}{5.555456in}}%
\pgfpathlineto{\pgfqpoint{6.033881in}{5.555456in}}%
\pgfpathlineto{\pgfqpoint{6.086925in}{5.555456in}}%
\pgfpathlineto{\pgfqpoint{6.139969in}{5.555456in}}%
\pgfpathlineto{\pgfqpoint{6.193012in}{5.555456in}}%
\pgfpathlineto{\pgfqpoint{6.246056in}{5.555456in}}%
\pgfpathlineto{\pgfqpoint{6.299100in}{5.555456in}}%
\pgfpathlineto{\pgfqpoint{6.352144in}{5.555456in}}%
\pgfpathlineto{\pgfqpoint{6.405188in}{5.555456in}}%
\pgfpathlineto{\pgfqpoint{6.458232in}{5.555456in}}%
\pgfpathlineto{\pgfqpoint{6.511276in}{5.555456in}}%
\pgfpathlineto{\pgfqpoint{6.564320in}{5.555456in}}%
\pgfpathlineto{\pgfqpoint{6.617364in}{5.555456in}}%
\pgfpathlineto{\pgfqpoint{6.670408in}{5.555456in}}%
\pgfpathlineto{\pgfqpoint{6.723452in}{5.555456in}}%
\pgfpathlineto{\pgfqpoint{6.776496in}{5.555456in}}%
\pgfpathlineto{\pgfqpoint{6.829540in}{5.555456in}}%
\pgfpathlineto{\pgfqpoint{6.882584in}{5.555456in}}%
\pgfpathlineto{\pgfqpoint{6.935628in}{5.555456in}}%
\pgfpathlineto{\pgfqpoint{6.988671in}{5.555456in}}%
\pgfpathlineto{\pgfqpoint{7.041715in}{5.555456in}}%
\pgfpathlineto{\pgfqpoint{7.094759in}{5.555456in}}%
\pgfpathlineto{\pgfqpoint{7.147803in}{5.555456in}}%
\pgfpathlineto{\pgfqpoint{7.200847in}{5.555456in}}%
\pgfpathlineto{\pgfqpoint{7.253891in}{5.555456in}}%
\pgfpathlineto{\pgfqpoint{7.306935in}{5.555456in}}%
\pgfpathlineto{\pgfqpoint{7.359979in}{5.555456in}}%
\pgfpathlineto{\pgfqpoint{7.413023in}{5.555456in}}%
\pgfpathlineto{\pgfqpoint{7.466067in}{5.555456in}}%
\pgfpathlineto{\pgfqpoint{7.519111in}{5.555456in}}%
\pgfpathlineto{\pgfqpoint{7.572155in}{5.555456in}}%
\pgfpathlineto{\pgfqpoint{7.625199in}{5.555456in}}%
\pgfpathlineto{\pgfqpoint{7.678243in}{5.555456in}}%
\pgfpathlineto{\pgfqpoint{7.731287in}{5.555456in}}%
\pgfpathlineto{\pgfqpoint{7.784330in}{5.555456in}}%
\pgfpathlineto{\pgfqpoint{7.837374in}{5.555456in}}%
\pgfpathlineto{\pgfqpoint{7.890418in}{5.555456in}}%
\pgfpathlineto{\pgfqpoint{7.943462in}{5.555456in}}%
\pgfpathlineto{\pgfqpoint{7.996506in}{5.555456in}}%
\pgfpathlineto{\pgfqpoint{8.049550in}{5.555456in}}%
\pgfpathlineto{\pgfqpoint{8.102594in}{5.555456in}}%
\pgfpathlineto{\pgfqpoint{8.155638in}{5.555456in}}%
\pgfpathlineto{\pgfqpoint{8.208682in}{5.555456in}}%
\pgfpathlineto{\pgfqpoint{8.261726in}{5.555456in}}%
\pgfpathlineto{\pgfqpoint{8.314770in}{5.555456in}}%
\pgfpathlineto{\pgfqpoint{8.367814in}{5.555456in}}%
\pgfpathlineto{\pgfqpoint{8.420858in}{5.555456in}}%
\pgfpathlineto{\pgfqpoint{8.473902in}{5.555456in}}%
\pgfpathlineto{\pgfqpoint{8.526946in}{5.555456in}}%
\pgfpathlineto{\pgfqpoint{8.579990in}{5.555456in}}%
\pgfpathlineto{\pgfqpoint{8.633033in}{5.555456in}}%
\pgfpathlineto{\pgfqpoint{8.686077in}{5.555456in}}%
\pgfpathlineto{\pgfqpoint{8.739121in}{5.555456in}}%
\pgfpathlineto{\pgfqpoint{8.792165in}{5.555456in}}%
\pgfpathlineto{\pgfqpoint{8.845209in}{5.555456in}}%
\pgfpathlineto{\pgfqpoint{8.898253in}{5.555456in}}%
\pgfpathlineto{\pgfqpoint{8.951297in}{5.555456in}}%
\pgfpathlineto{\pgfqpoint{9.004341in}{5.555456in}}%
\pgfpathlineto{\pgfqpoint{9.057385in}{5.555456in}}%
\pgfpathlineto{\pgfqpoint{9.110429in}{5.555456in}}%
\pgfpathlineto{\pgfqpoint{9.163473in}{5.555456in}}%
\pgfpathlineto{\pgfqpoint{9.216517in}{5.555456in}}%
\pgfpathlineto{\pgfqpoint{9.269561in}{5.555456in}}%
\pgfpathlineto{\pgfqpoint{9.322605in}{5.555456in}}%
\pgfpathlineto{\pgfqpoint{9.375649in}{5.555456in}}%
\pgfpathlineto{\pgfqpoint{9.428692in}{5.555456in}}%
\pgfpathlineto{\pgfqpoint{9.481736in}{5.555456in}}%
\pgfpathlineto{\pgfqpoint{9.534780in}{5.555456in}}%
\pgfpathlineto{\pgfqpoint{9.587824in}{5.555456in}}%
\pgfpathlineto{\pgfqpoint{9.640868in}{5.555456in}}%
\pgfpathlineto{\pgfqpoint{9.693912in}{5.555456in}}%
\pgfpathlineto{\pgfqpoint{9.746956in}{5.555456in}}%
\pgfpathlineto{\pgfqpoint{9.800000in}{5.555456in}}%
\pgfpathlineto{\pgfqpoint{9.800000in}{6.251128in}}%
\pgfpathlineto{\pgfqpoint{9.800000in}{6.251128in}}%
\pgfpathlineto{\pgfqpoint{9.746956in}{6.251128in}}%
\pgfpathlineto{\pgfqpoint{9.693912in}{6.251128in}}%
\pgfpathlineto{\pgfqpoint{9.640868in}{6.251128in}}%
\pgfpathlineto{\pgfqpoint{9.587824in}{6.251128in}}%
\pgfpathlineto{\pgfqpoint{9.534780in}{6.251128in}}%
\pgfpathlineto{\pgfqpoint{9.481736in}{6.251128in}}%
\pgfpathlineto{\pgfqpoint{9.428692in}{6.251128in}}%
\pgfpathlineto{\pgfqpoint{9.375649in}{6.251128in}}%
\pgfpathlineto{\pgfqpoint{9.322605in}{6.251128in}}%
\pgfpathlineto{\pgfqpoint{9.269561in}{6.251128in}}%
\pgfpathlineto{\pgfqpoint{9.216517in}{6.251128in}}%
\pgfpathlineto{\pgfqpoint{9.163473in}{6.251128in}}%
\pgfpathlineto{\pgfqpoint{9.110429in}{6.251128in}}%
\pgfpathlineto{\pgfqpoint{9.057385in}{6.251128in}}%
\pgfpathlineto{\pgfqpoint{9.004341in}{6.251128in}}%
\pgfpathlineto{\pgfqpoint{8.951297in}{6.251128in}}%
\pgfpathlineto{\pgfqpoint{8.898253in}{6.251128in}}%
\pgfpathlineto{\pgfqpoint{8.845209in}{6.251128in}}%
\pgfpathlineto{\pgfqpoint{8.792165in}{6.251128in}}%
\pgfpathlineto{\pgfqpoint{8.739121in}{6.251128in}}%
\pgfpathlineto{\pgfqpoint{8.686077in}{6.251128in}}%
\pgfpathlineto{\pgfqpoint{8.633033in}{6.251128in}}%
\pgfpathlineto{\pgfqpoint{8.579990in}{6.251128in}}%
\pgfpathlineto{\pgfqpoint{8.526946in}{6.251128in}}%
\pgfpathlineto{\pgfqpoint{8.473902in}{6.251128in}}%
\pgfpathlineto{\pgfqpoint{8.420858in}{6.251128in}}%
\pgfpathlineto{\pgfqpoint{8.367814in}{6.251128in}}%
\pgfpathlineto{\pgfqpoint{8.314770in}{6.251128in}}%
\pgfpathlineto{\pgfqpoint{8.261726in}{6.251128in}}%
\pgfpathlineto{\pgfqpoint{8.208682in}{6.251128in}}%
\pgfpathlineto{\pgfqpoint{8.155638in}{6.251128in}}%
\pgfpathlineto{\pgfqpoint{8.102594in}{6.251128in}}%
\pgfpathlineto{\pgfqpoint{8.049550in}{6.251128in}}%
\pgfpathlineto{\pgfqpoint{7.996506in}{6.251128in}}%
\pgfpathlineto{\pgfqpoint{7.943462in}{6.251128in}}%
\pgfpathlineto{\pgfqpoint{7.890418in}{6.251128in}}%
\pgfpathlineto{\pgfqpoint{7.837374in}{6.251128in}}%
\pgfpathlineto{\pgfqpoint{7.784330in}{6.251128in}}%
\pgfpathlineto{\pgfqpoint{7.731287in}{6.251128in}}%
\pgfpathlineto{\pgfqpoint{7.678243in}{6.251128in}}%
\pgfpathlineto{\pgfqpoint{7.625199in}{6.251128in}}%
\pgfpathlineto{\pgfqpoint{7.572155in}{6.251128in}}%
\pgfpathlineto{\pgfqpoint{7.519111in}{6.251128in}}%
\pgfpathlineto{\pgfqpoint{7.466067in}{6.251128in}}%
\pgfpathlineto{\pgfqpoint{7.413023in}{6.251128in}}%
\pgfpathlineto{\pgfqpoint{7.359979in}{6.251128in}}%
\pgfpathlineto{\pgfqpoint{7.306935in}{6.251128in}}%
\pgfpathlineto{\pgfqpoint{7.253891in}{6.251128in}}%
\pgfpathlineto{\pgfqpoint{7.200847in}{6.251128in}}%
\pgfpathlineto{\pgfqpoint{7.147803in}{6.251128in}}%
\pgfpathlineto{\pgfqpoint{7.094759in}{6.251128in}}%
\pgfpathlineto{\pgfqpoint{7.041715in}{6.251128in}}%
\pgfpathlineto{\pgfqpoint{6.988671in}{6.251128in}}%
\pgfpathlineto{\pgfqpoint{6.935628in}{6.251128in}}%
\pgfpathlineto{\pgfqpoint{6.882584in}{6.251128in}}%
\pgfpathlineto{\pgfqpoint{6.829540in}{6.251128in}}%
\pgfpathlineto{\pgfqpoint{6.776496in}{6.251128in}}%
\pgfpathlineto{\pgfqpoint{6.723452in}{6.251128in}}%
\pgfpathlineto{\pgfqpoint{6.670408in}{6.251128in}}%
\pgfpathlineto{\pgfqpoint{6.617364in}{6.251128in}}%
\pgfpathlineto{\pgfqpoint{6.564320in}{6.251128in}}%
\pgfpathlineto{\pgfqpoint{6.511276in}{6.251128in}}%
\pgfpathlineto{\pgfqpoint{6.458232in}{6.251128in}}%
\pgfpathlineto{\pgfqpoint{6.405188in}{6.251128in}}%
\pgfpathlineto{\pgfqpoint{6.352144in}{6.251128in}}%
\pgfpathlineto{\pgfqpoint{6.299100in}{6.251128in}}%
\pgfpathlineto{\pgfqpoint{6.246056in}{6.251128in}}%
\pgfpathlineto{\pgfqpoint{6.193012in}{6.251128in}}%
\pgfpathlineto{\pgfqpoint{6.139969in}{6.251128in}}%
\pgfpathlineto{\pgfqpoint{6.086925in}{6.251128in}}%
\pgfpathlineto{\pgfqpoint{6.033881in}{6.251128in}}%
\pgfpathlineto{\pgfqpoint{5.980837in}{6.251128in}}%
\pgfpathlineto{\pgfqpoint{5.927793in}{6.251128in}}%
\pgfpathlineto{\pgfqpoint{5.874749in}{6.251128in}}%
\pgfpathlineto{\pgfqpoint{5.821705in}{6.251128in}}%
\pgfpathlineto{\pgfqpoint{5.768661in}{6.251128in}}%
\pgfpathlineto{\pgfqpoint{5.715617in}{6.251128in}}%
\pgfpathlineto{\pgfqpoint{5.662573in}{6.251128in}}%
\pgfpathlineto{\pgfqpoint{5.609529in}{6.251128in}}%
\pgfpathlineto{\pgfqpoint{5.556485in}{6.251128in}}%
\pgfpathlineto{\pgfqpoint{5.503441in}{6.251128in}}%
\pgfpathlineto{\pgfqpoint{5.450397in}{6.251128in}}%
\pgfpathlineto{\pgfqpoint{5.397353in}{6.251128in}}%
\pgfpathlineto{\pgfqpoint{5.344309in}{6.251128in}}%
\pgfpathlineto{\pgfqpoint{5.291266in}{6.251128in}}%
\pgfpathlineto{\pgfqpoint{5.238222in}{6.251128in}}%
\pgfpathlineto{\pgfqpoint{5.185178in}{6.251128in}}%
\pgfpathlineto{\pgfqpoint{5.132134in}{6.251128in}}%
\pgfpathlineto{\pgfqpoint{5.079090in}{6.251128in}}%
\pgfpathlineto{\pgfqpoint{5.026046in}{6.251128in}}%
\pgfpathlineto{\pgfqpoint{4.973002in}{6.251128in}}%
\pgfpathlineto{\pgfqpoint{4.919958in}{6.251128in}}%
\pgfpathlineto{\pgfqpoint{4.866914in}{6.251128in}}%
\pgfpathlineto{\pgfqpoint{4.813870in}{6.251128in}}%
\pgfpathlineto{\pgfqpoint{4.760826in}{6.251128in}}%
\pgfpathlineto{\pgfqpoint{4.707782in}{6.251128in}}%
\pgfpathlineto{\pgfqpoint{4.654738in}{6.251128in}}%
\pgfpathlineto{\pgfqpoint{4.601694in}{6.251128in}}%
\pgfpathlineto{\pgfqpoint{4.548650in}{6.251128in}}%
\pgfpathlineto{\pgfqpoint{4.495607in}{6.251128in}}%
\pgfpathlineto{\pgfqpoint{4.442563in}{6.251128in}}%
\pgfpathlineto{\pgfqpoint{4.389519in}{6.251128in}}%
\pgfpathlineto{\pgfqpoint{4.336475in}{6.251128in}}%
\pgfpathlineto{\pgfqpoint{4.283431in}{6.251128in}}%
\pgfpathlineto{\pgfqpoint{4.230387in}{6.251128in}}%
\pgfpathlineto{\pgfqpoint{4.177343in}{6.251128in}}%
\pgfpathlineto{\pgfqpoint{4.124299in}{6.251128in}}%
\pgfpathlineto{\pgfqpoint{4.071255in}{6.251128in}}%
\pgfpathlineto{\pgfqpoint{4.018211in}{6.251128in}}%
\pgfpathlineto{\pgfqpoint{3.965167in}{6.251128in}}%
\pgfpathlineto{\pgfqpoint{3.912123in}{6.251128in}}%
\pgfpathlineto{\pgfqpoint{3.859079in}{6.251128in}}%
\pgfpathlineto{\pgfqpoint{3.806035in}{6.251128in}}%
\pgfpathlineto{\pgfqpoint{3.752991in}{6.251128in}}%
\pgfpathlineto{\pgfqpoint{3.699948in}{6.251128in}}%
\pgfpathlineto{\pgfqpoint{3.646904in}{6.251128in}}%
\pgfpathlineto{\pgfqpoint{3.593860in}{6.251128in}}%
\pgfpathlineto{\pgfqpoint{3.540816in}{6.251128in}}%
\pgfpathlineto{\pgfqpoint{3.487772in}{6.251128in}}%
\pgfpathlineto{\pgfqpoint{3.434728in}{6.251128in}}%
\pgfpathlineto{\pgfqpoint{3.381684in}{6.251128in}}%
\pgfpathlineto{\pgfqpoint{3.328640in}{6.251128in}}%
\pgfpathlineto{\pgfqpoint{3.275596in}{6.251128in}}%
\pgfpathlineto{\pgfqpoint{3.222552in}{6.251128in}}%
\pgfpathlineto{\pgfqpoint{3.169508in}{6.251128in}}%
\pgfpathlineto{\pgfqpoint{3.116464in}{6.251128in}}%
\pgfpathlineto{\pgfqpoint{3.063420in}{6.251128in}}%
\pgfpathlineto{\pgfqpoint{3.010376in}{6.251128in}}%
\pgfpathlineto{\pgfqpoint{2.957332in}{6.251128in}}%
\pgfpathlineto{\pgfqpoint{2.904288in}{6.251128in}}%
\pgfpathlineto{\pgfqpoint{2.851245in}{6.251128in}}%
\pgfpathlineto{\pgfqpoint{2.798201in}{6.251128in}}%
\pgfpathlineto{\pgfqpoint{2.745157in}{6.251128in}}%
\pgfpathlineto{\pgfqpoint{2.692113in}{6.251128in}}%
\pgfpathlineto{\pgfqpoint{2.639069in}{6.251128in}}%
\pgfpathlineto{\pgfqpoint{2.586025in}{6.251128in}}%
\pgfpathlineto{\pgfqpoint{2.532981in}{6.251128in}}%
\pgfpathlineto{\pgfqpoint{2.479937in}{6.251128in}}%
\pgfpathlineto{\pgfqpoint{2.426893in}{6.251128in}}%
\pgfpathlineto{\pgfqpoint{2.373849in}{6.251128in}}%
\pgfpathlineto{\pgfqpoint{2.320805in}{6.251128in}}%
\pgfpathlineto{\pgfqpoint{2.267761in}{6.251128in}}%
\pgfpathlineto{\pgfqpoint{2.214717in}{6.251128in}}%
\pgfpathlineto{\pgfqpoint{2.161673in}{6.251128in}}%
\pgfpathlineto{\pgfqpoint{2.108629in}{6.251128in}}%
\pgfpathlineto{\pgfqpoint{2.055586in}{6.251128in}}%
\pgfpathlineto{\pgfqpoint{2.002542in}{6.251128in}}%
\pgfpathlineto{\pgfqpoint{1.949498in}{6.251128in}}%
\pgfpathlineto{\pgfqpoint{1.896454in}{6.251128in}}%
\pgfpathlineto{\pgfqpoint{1.843410in}{6.251128in}}%
\pgfpathlineto{\pgfqpoint{1.790366in}{6.251128in}}%
\pgfpathlineto{\pgfqpoint{1.737322in}{6.251128in}}%
\pgfpathlineto{\pgfqpoint{1.684278in}{6.251128in}}%
\pgfpathlineto{\pgfqpoint{1.631234in}{6.251128in}}%
\pgfpathlineto{\pgfqpoint{1.578190in}{6.251128in}}%
\pgfpathlineto{\pgfqpoint{1.525146in}{6.251128in}}%
\pgfpathlineto{\pgfqpoint{1.472102in}{6.251128in}}%
\pgfpathlineto{\pgfqpoint{1.419058in}{6.251128in}}%
\pgfpathlineto{\pgfqpoint{1.366014in}{6.251128in}}%
\pgfpathlineto{\pgfqpoint{1.312970in}{6.251128in}}%
\pgfpathlineto{\pgfqpoint{1.259927in}{6.251128in}}%
\pgfpathlineto{\pgfqpoint{1.206883in}{6.251128in}}%
\pgfpathlineto{\pgfqpoint{1.153839in}{6.251128in}}%
\pgfpathlineto{\pgfqpoint{1.100795in}{6.251128in}}%
\pgfpathlineto{\pgfqpoint{1.047751in}{6.251128in}}%
\pgfpathlineto{\pgfqpoint{0.994707in}{6.251128in}}%
\pgfpathlineto{\pgfqpoint{0.941663in}{6.251128in}}%
\pgfpathlineto{\pgfqpoint{0.941663in}{6.251128in}}%
\pgfpathclose%
\pgfusepath{stroke,fill}%
}%
\begin{pgfscope}%
\pgfsys@transformshift{0.000000in}{0.000000in}%
\pgfsys@useobject{currentmarker}{}%
\end{pgfscope}%
\end{pgfscope}%
\begin{pgfscope}%
\pgfpathrectangle{\pgfqpoint{0.941663in}{4.334375in}}{\pgfqpoint{8.858337in}{3.465625in}}%
\pgfusepath{clip}%
\pgfsetrectcap%
\pgfsetroundjoin%
\pgfsetlinewidth{1.505625pt}%
\definecolor{currentstroke}{rgb}{0.501961,0.000000,0.501961}%
\pgfsetstrokecolor{currentstroke}%
\pgfsetdash{}{0pt}%
\pgfpathmoveto{\pgfqpoint{0.941663in}{6.251128in}}%
\pgfpathlineto{\pgfqpoint{0.994707in}{6.261680in}}%
\pgfpathlineto{\pgfqpoint{1.047751in}{6.251128in}}%
\pgfpathlineto{\pgfqpoint{1.153839in}{6.251128in}}%
\pgfpathlineto{\pgfqpoint{1.206883in}{6.598964in}}%
\pgfpathlineto{\pgfqpoint{1.312970in}{6.598964in}}%
\pgfpathlineto{\pgfqpoint{1.366014in}{6.251128in}}%
\pgfpathlineto{\pgfqpoint{1.578190in}{6.251128in}}%
\pgfpathlineto{\pgfqpoint{1.631234in}{6.598964in}}%
\pgfpathlineto{\pgfqpoint{1.684278in}{6.598964in}}%
\pgfpathlineto{\pgfqpoint{1.737322in}{6.251128in}}%
\pgfpathlineto{\pgfqpoint{1.790366in}{6.251128in}}%
\pgfpathlineto{\pgfqpoint{1.843410in}{6.598964in}}%
\pgfpathlineto{\pgfqpoint{1.896454in}{6.598964in}}%
\pgfpathlineto{\pgfqpoint{1.949498in}{6.251128in}}%
\pgfpathlineto{\pgfqpoint{2.002542in}{6.505445in}}%
\pgfpathlineto{\pgfqpoint{2.055586in}{6.251128in}}%
\pgfpathlineto{\pgfqpoint{2.108629in}{6.348747in}}%
\pgfpathlineto{\pgfqpoint{2.161673in}{6.251128in}}%
\pgfpathlineto{\pgfqpoint{2.214717in}{6.459099in}}%
\pgfpathlineto{\pgfqpoint{2.267761in}{6.251128in}}%
\pgfpathlineto{\pgfqpoint{2.320805in}{6.251128in}}%
\pgfpathlineto{\pgfqpoint{2.373849in}{6.541500in}}%
\pgfpathlineto{\pgfqpoint{2.426893in}{6.251128in}}%
\pgfpathlineto{\pgfqpoint{2.798201in}{6.251128in}}%
\pgfpathlineto{\pgfqpoint{2.851245in}{6.598964in}}%
\pgfpathlineto{\pgfqpoint{2.904288in}{6.251128in}}%
\pgfpathlineto{\pgfqpoint{2.957332in}{6.598964in}}%
\pgfpathlineto{\pgfqpoint{3.010376in}{6.251128in}}%
\pgfpathlineto{\pgfqpoint{3.063420in}{6.598964in}}%
\pgfpathlineto{\pgfqpoint{3.116464in}{6.598964in}}%
\pgfpathlineto{\pgfqpoint{3.169508in}{6.251128in}}%
\pgfpathlineto{\pgfqpoint{3.328640in}{6.251128in}}%
\pgfpathlineto{\pgfqpoint{3.381684in}{6.598964in}}%
\pgfpathlineto{\pgfqpoint{3.434728in}{6.251128in}}%
\pgfpathlineto{\pgfqpoint{3.487772in}{6.598964in}}%
\pgfpathlineto{\pgfqpoint{3.540816in}{6.427614in}}%
\pgfpathlineto{\pgfqpoint{3.593860in}{6.251128in}}%
\pgfpathlineto{\pgfqpoint{3.699948in}{6.251128in}}%
\pgfpathlineto{\pgfqpoint{3.752991in}{6.510498in}}%
\pgfpathlineto{\pgfqpoint{3.806035in}{6.251128in}}%
\pgfpathlineto{\pgfqpoint{3.859079in}{6.572998in}}%
\pgfpathlineto{\pgfqpoint{3.912123in}{6.481634in}}%
\pgfpathlineto{\pgfqpoint{3.965167in}{6.340259in}}%
\pgfpathlineto{\pgfqpoint{4.018211in}{6.251128in}}%
\pgfpathlineto{\pgfqpoint{4.283431in}{6.251128in}}%
\pgfpathlineto{\pgfqpoint{4.336475in}{6.598964in}}%
\pgfpathlineto{\pgfqpoint{4.389519in}{6.299258in}}%
\pgfpathlineto{\pgfqpoint{4.442563in}{6.598964in}}%
\pgfpathlineto{\pgfqpoint{4.495607in}{6.437705in}}%
\pgfpathlineto{\pgfqpoint{4.548650in}{6.251128in}}%
\pgfpathlineto{\pgfqpoint{4.654738in}{6.251128in}}%
\pgfpathlineto{\pgfqpoint{4.707782in}{6.351811in}}%
\pgfpathlineto{\pgfqpoint{4.760826in}{6.251128in}}%
\pgfpathlineto{\pgfqpoint{4.813870in}{6.251128in}}%
\pgfpathlineto{\pgfqpoint{4.866914in}{6.471548in}}%
\pgfpathlineto{\pgfqpoint{4.919958in}{6.251128in}}%
\pgfpathlineto{\pgfqpoint{4.973002in}{6.251128in}}%
\pgfpathlineto{\pgfqpoint{5.026046in}{6.590550in}}%
\pgfpathlineto{\pgfqpoint{5.079090in}{6.251128in}}%
\pgfpathlineto{\pgfqpoint{5.132134in}{6.598964in}}%
\pgfpathlineto{\pgfqpoint{5.185178in}{6.251128in}}%
\pgfpathlineto{\pgfqpoint{5.291266in}{6.251128in}}%
\pgfpathlineto{\pgfqpoint{5.344309in}{6.471024in}}%
\pgfpathlineto{\pgfqpoint{5.397353in}{6.251128in}}%
\pgfpathlineto{\pgfqpoint{5.503441in}{6.251128in}}%
\pgfpathlineto{\pgfqpoint{5.556485in}{6.329843in}}%
\pgfpathlineto{\pgfqpoint{5.609529in}{6.251128in}}%
\pgfpathlineto{\pgfqpoint{5.715617in}{6.251128in}}%
\pgfpathlineto{\pgfqpoint{5.768661in}{6.324051in}}%
\pgfpathlineto{\pgfqpoint{5.821705in}{6.251128in}}%
\pgfpathlineto{\pgfqpoint{5.980837in}{6.251128in}}%
\pgfpathlineto{\pgfqpoint{6.033881in}{6.400407in}}%
\pgfpathlineto{\pgfqpoint{6.086925in}{6.251128in}}%
\pgfpathlineto{\pgfqpoint{6.139969in}{6.251128in}}%
\pgfpathlineto{\pgfqpoint{6.193012in}{6.598964in}}%
\pgfpathlineto{\pgfqpoint{6.246056in}{6.362599in}}%
\pgfpathlineto{\pgfqpoint{6.299100in}{6.251128in}}%
\pgfpathlineto{\pgfqpoint{6.352144in}{6.569516in}}%
\pgfpathlineto{\pgfqpoint{6.405188in}{6.251128in}}%
\pgfpathlineto{\pgfqpoint{6.723452in}{6.251128in}}%
\pgfpathlineto{\pgfqpoint{6.776496in}{6.385890in}}%
\pgfpathlineto{\pgfqpoint{6.829540in}{6.251128in}}%
\pgfpathlineto{\pgfqpoint{6.882584in}{6.598964in}}%
\pgfpathlineto{\pgfqpoint{6.935628in}{6.251128in}}%
\pgfpathlineto{\pgfqpoint{6.988671in}{6.598964in}}%
\pgfpathlineto{\pgfqpoint{7.041715in}{6.251128in}}%
\pgfpathlineto{\pgfqpoint{7.147803in}{6.251128in}}%
\pgfpathlineto{\pgfqpoint{7.200847in}{6.598964in}}%
\pgfpathlineto{\pgfqpoint{7.253891in}{6.451777in}}%
\pgfpathlineto{\pgfqpoint{7.306935in}{6.251128in}}%
\pgfpathlineto{\pgfqpoint{7.359979in}{6.350974in}}%
\pgfpathlineto{\pgfqpoint{7.413023in}{6.251128in}}%
\pgfpathlineto{\pgfqpoint{7.996506in}{6.251128in}}%
\pgfpathlineto{\pgfqpoint{8.049550in}{6.598964in}}%
\pgfpathlineto{\pgfqpoint{8.102594in}{6.598964in}}%
\pgfpathlineto{\pgfqpoint{8.155638in}{6.251128in}}%
\pgfpathlineto{\pgfqpoint{8.208682in}{6.251128in}}%
\pgfpathlineto{\pgfqpoint{8.261726in}{6.598964in}}%
\pgfpathlineto{\pgfqpoint{8.314770in}{6.598964in}}%
\pgfpathlineto{\pgfqpoint{8.367814in}{6.251128in}}%
\pgfpathlineto{\pgfqpoint{8.420858in}{6.371312in}}%
\pgfpathlineto{\pgfqpoint{8.473902in}{6.251128in}}%
\pgfpathlineto{\pgfqpoint{8.526946in}{6.251128in}}%
\pgfpathlineto{\pgfqpoint{8.579990in}{6.560479in}}%
\pgfpathlineto{\pgfqpoint{8.633033in}{6.251128in}}%
\pgfpathlineto{\pgfqpoint{8.951297in}{6.251128in}}%
\pgfpathlineto{\pgfqpoint{9.004341in}{6.568668in}}%
\pgfpathlineto{\pgfqpoint{9.057385in}{6.251128in}}%
\pgfpathlineto{\pgfqpoint{9.110429in}{6.251128in}}%
\pgfpathlineto{\pgfqpoint{9.163473in}{6.598964in}}%
\pgfpathlineto{\pgfqpoint{9.216517in}{6.334558in}}%
\pgfpathlineto{\pgfqpoint{9.269561in}{6.251128in}}%
\pgfpathlineto{\pgfqpoint{9.428692in}{6.251128in}}%
\pgfpathlineto{\pgfqpoint{9.481736in}{6.256205in}}%
\pgfpathlineto{\pgfqpoint{9.534780in}{6.287774in}}%
\pgfpathlineto{\pgfqpoint{9.587824in}{6.251128in}}%
\pgfpathlineto{\pgfqpoint{9.800000in}{6.251128in}}%
\pgfpathlineto{\pgfqpoint{9.800000in}{6.251128in}}%
\pgfusepath{stroke}%
\end{pgfscope}%
\begin{pgfscope}%
\pgfpathrectangle{\pgfqpoint{0.941663in}{4.334375in}}{\pgfqpoint{8.858337in}{3.465625in}}%
\pgfusepath{clip}%
\pgfsetbuttcap%
\pgfsetroundjoin%
\definecolor{currentfill}{rgb}{0.501961,0.000000,0.501961}%
\pgfsetfillcolor{currentfill}%
\pgfsetlinewidth{1.003750pt}%
\definecolor{currentstroke}{rgb}{0.501961,0.000000,0.501961}%
\pgfsetstrokecolor{currentstroke}%
\pgfsetdash{}{0pt}%
\pgfsys@defobject{currentmarker}{\pgfqpoint{0.941663in}{6.251128in}}{\pgfqpoint{9.800000in}{6.598964in}}{%
\pgfpathmoveto{\pgfqpoint{0.941663in}{6.251128in}}%
\pgfpathlineto{\pgfqpoint{0.941663in}{6.251128in}}%
\pgfpathlineto{\pgfqpoint{0.994707in}{6.251128in}}%
\pgfpathlineto{\pgfqpoint{1.047751in}{6.251128in}}%
\pgfpathlineto{\pgfqpoint{1.100795in}{6.251128in}}%
\pgfpathlineto{\pgfqpoint{1.153839in}{6.251128in}}%
\pgfpathlineto{\pgfqpoint{1.206883in}{6.251128in}}%
\pgfpathlineto{\pgfqpoint{1.259927in}{6.251128in}}%
\pgfpathlineto{\pgfqpoint{1.312970in}{6.251128in}}%
\pgfpathlineto{\pgfqpoint{1.366014in}{6.251128in}}%
\pgfpathlineto{\pgfqpoint{1.419058in}{6.251128in}}%
\pgfpathlineto{\pgfqpoint{1.472102in}{6.251128in}}%
\pgfpathlineto{\pgfqpoint{1.525146in}{6.251128in}}%
\pgfpathlineto{\pgfqpoint{1.578190in}{6.251128in}}%
\pgfpathlineto{\pgfqpoint{1.631234in}{6.251128in}}%
\pgfpathlineto{\pgfqpoint{1.684278in}{6.251128in}}%
\pgfpathlineto{\pgfqpoint{1.737322in}{6.251128in}}%
\pgfpathlineto{\pgfqpoint{1.790366in}{6.251128in}}%
\pgfpathlineto{\pgfqpoint{1.843410in}{6.251128in}}%
\pgfpathlineto{\pgfqpoint{1.896454in}{6.251128in}}%
\pgfpathlineto{\pgfqpoint{1.949498in}{6.251128in}}%
\pgfpathlineto{\pgfqpoint{2.002542in}{6.251128in}}%
\pgfpathlineto{\pgfqpoint{2.055586in}{6.251128in}}%
\pgfpathlineto{\pgfqpoint{2.108629in}{6.251128in}}%
\pgfpathlineto{\pgfqpoint{2.161673in}{6.251128in}}%
\pgfpathlineto{\pgfqpoint{2.214717in}{6.251128in}}%
\pgfpathlineto{\pgfqpoint{2.267761in}{6.251128in}}%
\pgfpathlineto{\pgfqpoint{2.320805in}{6.251128in}}%
\pgfpathlineto{\pgfqpoint{2.373849in}{6.251128in}}%
\pgfpathlineto{\pgfqpoint{2.426893in}{6.251128in}}%
\pgfpathlineto{\pgfqpoint{2.479937in}{6.251128in}}%
\pgfpathlineto{\pgfqpoint{2.532981in}{6.251128in}}%
\pgfpathlineto{\pgfqpoint{2.586025in}{6.251128in}}%
\pgfpathlineto{\pgfqpoint{2.639069in}{6.251128in}}%
\pgfpathlineto{\pgfqpoint{2.692113in}{6.251128in}}%
\pgfpathlineto{\pgfqpoint{2.745157in}{6.251128in}}%
\pgfpathlineto{\pgfqpoint{2.798201in}{6.251128in}}%
\pgfpathlineto{\pgfqpoint{2.851245in}{6.251128in}}%
\pgfpathlineto{\pgfqpoint{2.904288in}{6.251128in}}%
\pgfpathlineto{\pgfqpoint{2.957332in}{6.251128in}}%
\pgfpathlineto{\pgfqpoint{3.010376in}{6.251128in}}%
\pgfpathlineto{\pgfqpoint{3.063420in}{6.251128in}}%
\pgfpathlineto{\pgfqpoint{3.116464in}{6.251128in}}%
\pgfpathlineto{\pgfqpoint{3.169508in}{6.251128in}}%
\pgfpathlineto{\pgfqpoint{3.222552in}{6.251128in}}%
\pgfpathlineto{\pgfqpoint{3.275596in}{6.251128in}}%
\pgfpathlineto{\pgfqpoint{3.328640in}{6.251128in}}%
\pgfpathlineto{\pgfqpoint{3.381684in}{6.251128in}}%
\pgfpathlineto{\pgfqpoint{3.434728in}{6.251128in}}%
\pgfpathlineto{\pgfqpoint{3.487772in}{6.251128in}}%
\pgfpathlineto{\pgfqpoint{3.540816in}{6.251128in}}%
\pgfpathlineto{\pgfqpoint{3.593860in}{6.251128in}}%
\pgfpathlineto{\pgfqpoint{3.646904in}{6.251128in}}%
\pgfpathlineto{\pgfqpoint{3.699948in}{6.251128in}}%
\pgfpathlineto{\pgfqpoint{3.752991in}{6.251128in}}%
\pgfpathlineto{\pgfqpoint{3.806035in}{6.251128in}}%
\pgfpathlineto{\pgfqpoint{3.859079in}{6.251128in}}%
\pgfpathlineto{\pgfqpoint{3.912123in}{6.251128in}}%
\pgfpathlineto{\pgfqpoint{3.965167in}{6.251128in}}%
\pgfpathlineto{\pgfqpoint{4.018211in}{6.251128in}}%
\pgfpathlineto{\pgfqpoint{4.071255in}{6.251128in}}%
\pgfpathlineto{\pgfqpoint{4.124299in}{6.251128in}}%
\pgfpathlineto{\pgfqpoint{4.177343in}{6.251128in}}%
\pgfpathlineto{\pgfqpoint{4.230387in}{6.251128in}}%
\pgfpathlineto{\pgfqpoint{4.283431in}{6.251128in}}%
\pgfpathlineto{\pgfqpoint{4.336475in}{6.251128in}}%
\pgfpathlineto{\pgfqpoint{4.389519in}{6.251128in}}%
\pgfpathlineto{\pgfqpoint{4.442563in}{6.251128in}}%
\pgfpathlineto{\pgfqpoint{4.495607in}{6.251128in}}%
\pgfpathlineto{\pgfqpoint{4.548650in}{6.251128in}}%
\pgfpathlineto{\pgfqpoint{4.601694in}{6.251128in}}%
\pgfpathlineto{\pgfqpoint{4.654738in}{6.251128in}}%
\pgfpathlineto{\pgfqpoint{4.707782in}{6.251128in}}%
\pgfpathlineto{\pgfqpoint{4.760826in}{6.251128in}}%
\pgfpathlineto{\pgfqpoint{4.813870in}{6.251128in}}%
\pgfpathlineto{\pgfqpoint{4.866914in}{6.251128in}}%
\pgfpathlineto{\pgfqpoint{4.919958in}{6.251128in}}%
\pgfpathlineto{\pgfqpoint{4.973002in}{6.251128in}}%
\pgfpathlineto{\pgfqpoint{5.026046in}{6.251128in}}%
\pgfpathlineto{\pgfqpoint{5.079090in}{6.251128in}}%
\pgfpathlineto{\pgfqpoint{5.132134in}{6.251128in}}%
\pgfpathlineto{\pgfqpoint{5.185178in}{6.251128in}}%
\pgfpathlineto{\pgfqpoint{5.238222in}{6.251128in}}%
\pgfpathlineto{\pgfqpoint{5.291266in}{6.251128in}}%
\pgfpathlineto{\pgfqpoint{5.344309in}{6.251128in}}%
\pgfpathlineto{\pgfqpoint{5.397353in}{6.251128in}}%
\pgfpathlineto{\pgfqpoint{5.450397in}{6.251128in}}%
\pgfpathlineto{\pgfqpoint{5.503441in}{6.251128in}}%
\pgfpathlineto{\pgfqpoint{5.556485in}{6.251128in}}%
\pgfpathlineto{\pgfqpoint{5.609529in}{6.251128in}}%
\pgfpathlineto{\pgfqpoint{5.662573in}{6.251128in}}%
\pgfpathlineto{\pgfqpoint{5.715617in}{6.251128in}}%
\pgfpathlineto{\pgfqpoint{5.768661in}{6.251128in}}%
\pgfpathlineto{\pgfqpoint{5.821705in}{6.251128in}}%
\pgfpathlineto{\pgfqpoint{5.874749in}{6.251128in}}%
\pgfpathlineto{\pgfqpoint{5.927793in}{6.251128in}}%
\pgfpathlineto{\pgfqpoint{5.980837in}{6.251128in}}%
\pgfpathlineto{\pgfqpoint{6.033881in}{6.251128in}}%
\pgfpathlineto{\pgfqpoint{6.086925in}{6.251128in}}%
\pgfpathlineto{\pgfqpoint{6.139969in}{6.251128in}}%
\pgfpathlineto{\pgfqpoint{6.193012in}{6.251128in}}%
\pgfpathlineto{\pgfqpoint{6.246056in}{6.251128in}}%
\pgfpathlineto{\pgfqpoint{6.299100in}{6.251128in}}%
\pgfpathlineto{\pgfqpoint{6.352144in}{6.251128in}}%
\pgfpathlineto{\pgfqpoint{6.405188in}{6.251128in}}%
\pgfpathlineto{\pgfqpoint{6.458232in}{6.251128in}}%
\pgfpathlineto{\pgfqpoint{6.511276in}{6.251128in}}%
\pgfpathlineto{\pgfqpoint{6.564320in}{6.251128in}}%
\pgfpathlineto{\pgfqpoint{6.617364in}{6.251128in}}%
\pgfpathlineto{\pgfqpoint{6.670408in}{6.251128in}}%
\pgfpathlineto{\pgfqpoint{6.723452in}{6.251128in}}%
\pgfpathlineto{\pgfqpoint{6.776496in}{6.251128in}}%
\pgfpathlineto{\pgfqpoint{6.829540in}{6.251128in}}%
\pgfpathlineto{\pgfqpoint{6.882584in}{6.251128in}}%
\pgfpathlineto{\pgfqpoint{6.935628in}{6.251128in}}%
\pgfpathlineto{\pgfqpoint{6.988671in}{6.251128in}}%
\pgfpathlineto{\pgfqpoint{7.041715in}{6.251128in}}%
\pgfpathlineto{\pgfqpoint{7.094759in}{6.251128in}}%
\pgfpathlineto{\pgfqpoint{7.147803in}{6.251128in}}%
\pgfpathlineto{\pgfqpoint{7.200847in}{6.251128in}}%
\pgfpathlineto{\pgfqpoint{7.253891in}{6.251128in}}%
\pgfpathlineto{\pgfqpoint{7.306935in}{6.251128in}}%
\pgfpathlineto{\pgfqpoint{7.359979in}{6.251128in}}%
\pgfpathlineto{\pgfqpoint{7.413023in}{6.251128in}}%
\pgfpathlineto{\pgfqpoint{7.466067in}{6.251128in}}%
\pgfpathlineto{\pgfqpoint{7.519111in}{6.251128in}}%
\pgfpathlineto{\pgfqpoint{7.572155in}{6.251128in}}%
\pgfpathlineto{\pgfqpoint{7.625199in}{6.251128in}}%
\pgfpathlineto{\pgfqpoint{7.678243in}{6.251128in}}%
\pgfpathlineto{\pgfqpoint{7.731287in}{6.251128in}}%
\pgfpathlineto{\pgfqpoint{7.784330in}{6.251128in}}%
\pgfpathlineto{\pgfqpoint{7.837374in}{6.251128in}}%
\pgfpathlineto{\pgfqpoint{7.890418in}{6.251128in}}%
\pgfpathlineto{\pgfqpoint{7.943462in}{6.251128in}}%
\pgfpathlineto{\pgfqpoint{7.996506in}{6.251128in}}%
\pgfpathlineto{\pgfqpoint{8.049550in}{6.251128in}}%
\pgfpathlineto{\pgfqpoint{8.102594in}{6.251128in}}%
\pgfpathlineto{\pgfqpoint{8.155638in}{6.251128in}}%
\pgfpathlineto{\pgfqpoint{8.208682in}{6.251128in}}%
\pgfpathlineto{\pgfqpoint{8.261726in}{6.251128in}}%
\pgfpathlineto{\pgfqpoint{8.314770in}{6.251128in}}%
\pgfpathlineto{\pgfqpoint{8.367814in}{6.251128in}}%
\pgfpathlineto{\pgfqpoint{8.420858in}{6.251128in}}%
\pgfpathlineto{\pgfqpoint{8.473902in}{6.251128in}}%
\pgfpathlineto{\pgfqpoint{8.526946in}{6.251128in}}%
\pgfpathlineto{\pgfqpoint{8.579990in}{6.251128in}}%
\pgfpathlineto{\pgfqpoint{8.633033in}{6.251128in}}%
\pgfpathlineto{\pgfqpoint{8.686077in}{6.251128in}}%
\pgfpathlineto{\pgfqpoint{8.739121in}{6.251128in}}%
\pgfpathlineto{\pgfqpoint{8.792165in}{6.251128in}}%
\pgfpathlineto{\pgfqpoint{8.845209in}{6.251128in}}%
\pgfpathlineto{\pgfqpoint{8.898253in}{6.251128in}}%
\pgfpathlineto{\pgfqpoint{8.951297in}{6.251128in}}%
\pgfpathlineto{\pgfqpoint{9.004341in}{6.251128in}}%
\pgfpathlineto{\pgfqpoint{9.057385in}{6.251128in}}%
\pgfpathlineto{\pgfqpoint{9.110429in}{6.251128in}}%
\pgfpathlineto{\pgfqpoint{9.163473in}{6.251128in}}%
\pgfpathlineto{\pgfqpoint{9.216517in}{6.251128in}}%
\pgfpathlineto{\pgfqpoint{9.269561in}{6.251128in}}%
\pgfpathlineto{\pgfqpoint{9.322605in}{6.251128in}}%
\pgfpathlineto{\pgfqpoint{9.375649in}{6.251128in}}%
\pgfpathlineto{\pgfqpoint{9.428692in}{6.251128in}}%
\pgfpathlineto{\pgfqpoint{9.481736in}{6.251128in}}%
\pgfpathlineto{\pgfqpoint{9.534780in}{6.251128in}}%
\pgfpathlineto{\pgfqpoint{9.587824in}{6.251128in}}%
\pgfpathlineto{\pgfqpoint{9.640868in}{6.251128in}}%
\pgfpathlineto{\pgfqpoint{9.693912in}{6.251128in}}%
\pgfpathlineto{\pgfqpoint{9.746956in}{6.251128in}}%
\pgfpathlineto{\pgfqpoint{9.800000in}{6.251128in}}%
\pgfpathlineto{\pgfqpoint{9.800000in}{6.251128in}}%
\pgfpathlineto{\pgfqpoint{9.800000in}{6.251128in}}%
\pgfpathlineto{\pgfqpoint{9.746956in}{6.251128in}}%
\pgfpathlineto{\pgfqpoint{9.693912in}{6.251128in}}%
\pgfpathlineto{\pgfqpoint{9.640868in}{6.251128in}}%
\pgfpathlineto{\pgfqpoint{9.587824in}{6.251128in}}%
\pgfpathlineto{\pgfqpoint{9.534780in}{6.287774in}}%
\pgfpathlineto{\pgfqpoint{9.481736in}{6.256205in}}%
\pgfpathlineto{\pgfqpoint{9.428692in}{6.251128in}}%
\pgfpathlineto{\pgfqpoint{9.375649in}{6.251128in}}%
\pgfpathlineto{\pgfqpoint{9.322605in}{6.251128in}}%
\pgfpathlineto{\pgfqpoint{9.269561in}{6.251128in}}%
\pgfpathlineto{\pgfqpoint{9.216517in}{6.334558in}}%
\pgfpathlineto{\pgfqpoint{9.163473in}{6.598964in}}%
\pgfpathlineto{\pgfqpoint{9.110429in}{6.251128in}}%
\pgfpathlineto{\pgfqpoint{9.057385in}{6.251128in}}%
\pgfpathlineto{\pgfqpoint{9.004341in}{6.568668in}}%
\pgfpathlineto{\pgfqpoint{8.951297in}{6.251128in}}%
\pgfpathlineto{\pgfqpoint{8.898253in}{6.251128in}}%
\pgfpathlineto{\pgfqpoint{8.845209in}{6.251128in}}%
\pgfpathlineto{\pgfqpoint{8.792165in}{6.251128in}}%
\pgfpathlineto{\pgfqpoint{8.739121in}{6.251128in}}%
\pgfpathlineto{\pgfqpoint{8.686077in}{6.251128in}}%
\pgfpathlineto{\pgfqpoint{8.633033in}{6.251128in}}%
\pgfpathlineto{\pgfqpoint{8.579990in}{6.560479in}}%
\pgfpathlineto{\pgfqpoint{8.526946in}{6.251128in}}%
\pgfpathlineto{\pgfqpoint{8.473902in}{6.251128in}}%
\pgfpathlineto{\pgfqpoint{8.420858in}{6.371312in}}%
\pgfpathlineto{\pgfqpoint{8.367814in}{6.251128in}}%
\pgfpathlineto{\pgfqpoint{8.314770in}{6.598964in}}%
\pgfpathlineto{\pgfqpoint{8.261726in}{6.598964in}}%
\pgfpathlineto{\pgfqpoint{8.208682in}{6.251128in}}%
\pgfpathlineto{\pgfqpoint{8.155638in}{6.251128in}}%
\pgfpathlineto{\pgfqpoint{8.102594in}{6.598964in}}%
\pgfpathlineto{\pgfqpoint{8.049550in}{6.598964in}}%
\pgfpathlineto{\pgfqpoint{7.996506in}{6.251128in}}%
\pgfpathlineto{\pgfqpoint{7.943462in}{6.251128in}}%
\pgfpathlineto{\pgfqpoint{7.890418in}{6.251128in}}%
\pgfpathlineto{\pgfqpoint{7.837374in}{6.251128in}}%
\pgfpathlineto{\pgfqpoint{7.784330in}{6.251128in}}%
\pgfpathlineto{\pgfqpoint{7.731287in}{6.251128in}}%
\pgfpathlineto{\pgfqpoint{7.678243in}{6.251128in}}%
\pgfpathlineto{\pgfqpoint{7.625199in}{6.251128in}}%
\pgfpathlineto{\pgfqpoint{7.572155in}{6.251128in}}%
\pgfpathlineto{\pgfqpoint{7.519111in}{6.251128in}}%
\pgfpathlineto{\pgfqpoint{7.466067in}{6.251128in}}%
\pgfpathlineto{\pgfqpoint{7.413023in}{6.251128in}}%
\pgfpathlineto{\pgfqpoint{7.359979in}{6.350974in}}%
\pgfpathlineto{\pgfqpoint{7.306935in}{6.251128in}}%
\pgfpathlineto{\pgfqpoint{7.253891in}{6.451777in}}%
\pgfpathlineto{\pgfqpoint{7.200847in}{6.598964in}}%
\pgfpathlineto{\pgfqpoint{7.147803in}{6.251128in}}%
\pgfpathlineto{\pgfqpoint{7.094759in}{6.251128in}}%
\pgfpathlineto{\pgfqpoint{7.041715in}{6.251128in}}%
\pgfpathlineto{\pgfqpoint{6.988671in}{6.598964in}}%
\pgfpathlineto{\pgfqpoint{6.935628in}{6.251128in}}%
\pgfpathlineto{\pgfqpoint{6.882584in}{6.598964in}}%
\pgfpathlineto{\pgfqpoint{6.829540in}{6.251128in}}%
\pgfpathlineto{\pgfqpoint{6.776496in}{6.385890in}}%
\pgfpathlineto{\pgfqpoint{6.723452in}{6.251128in}}%
\pgfpathlineto{\pgfqpoint{6.670408in}{6.251128in}}%
\pgfpathlineto{\pgfqpoint{6.617364in}{6.251128in}}%
\pgfpathlineto{\pgfqpoint{6.564320in}{6.251128in}}%
\pgfpathlineto{\pgfqpoint{6.511276in}{6.251128in}}%
\pgfpathlineto{\pgfqpoint{6.458232in}{6.251128in}}%
\pgfpathlineto{\pgfqpoint{6.405188in}{6.251128in}}%
\pgfpathlineto{\pgfqpoint{6.352144in}{6.569516in}}%
\pgfpathlineto{\pgfqpoint{6.299100in}{6.251128in}}%
\pgfpathlineto{\pgfqpoint{6.246056in}{6.362599in}}%
\pgfpathlineto{\pgfqpoint{6.193012in}{6.598964in}}%
\pgfpathlineto{\pgfqpoint{6.139969in}{6.251128in}}%
\pgfpathlineto{\pgfqpoint{6.086925in}{6.251128in}}%
\pgfpathlineto{\pgfqpoint{6.033881in}{6.400407in}}%
\pgfpathlineto{\pgfqpoint{5.980837in}{6.251128in}}%
\pgfpathlineto{\pgfqpoint{5.927793in}{6.251128in}}%
\pgfpathlineto{\pgfqpoint{5.874749in}{6.251128in}}%
\pgfpathlineto{\pgfqpoint{5.821705in}{6.251128in}}%
\pgfpathlineto{\pgfqpoint{5.768661in}{6.324051in}}%
\pgfpathlineto{\pgfqpoint{5.715617in}{6.251128in}}%
\pgfpathlineto{\pgfqpoint{5.662573in}{6.251128in}}%
\pgfpathlineto{\pgfqpoint{5.609529in}{6.251128in}}%
\pgfpathlineto{\pgfqpoint{5.556485in}{6.329843in}}%
\pgfpathlineto{\pgfqpoint{5.503441in}{6.251128in}}%
\pgfpathlineto{\pgfqpoint{5.450397in}{6.251128in}}%
\pgfpathlineto{\pgfqpoint{5.397353in}{6.251128in}}%
\pgfpathlineto{\pgfqpoint{5.344309in}{6.471024in}}%
\pgfpathlineto{\pgfqpoint{5.291266in}{6.251128in}}%
\pgfpathlineto{\pgfqpoint{5.238222in}{6.251128in}}%
\pgfpathlineto{\pgfqpoint{5.185178in}{6.251128in}}%
\pgfpathlineto{\pgfqpoint{5.132134in}{6.598964in}}%
\pgfpathlineto{\pgfqpoint{5.079090in}{6.251128in}}%
\pgfpathlineto{\pgfqpoint{5.026046in}{6.590550in}}%
\pgfpathlineto{\pgfqpoint{4.973002in}{6.251128in}}%
\pgfpathlineto{\pgfqpoint{4.919958in}{6.251128in}}%
\pgfpathlineto{\pgfqpoint{4.866914in}{6.471548in}}%
\pgfpathlineto{\pgfqpoint{4.813870in}{6.251128in}}%
\pgfpathlineto{\pgfqpoint{4.760826in}{6.251128in}}%
\pgfpathlineto{\pgfqpoint{4.707782in}{6.351811in}}%
\pgfpathlineto{\pgfqpoint{4.654738in}{6.251128in}}%
\pgfpathlineto{\pgfqpoint{4.601694in}{6.251128in}}%
\pgfpathlineto{\pgfqpoint{4.548650in}{6.251128in}}%
\pgfpathlineto{\pgfqpoint{4.495607in}{6.437705in}}%
\pgfpathlineto{\pgfqpoint{4.442563in}{6.598964in}}%
\pgfpathlineto{\pgfqpoint{4.389519in}{6.299258in}}%
\pgfpathlineto{\pgfqpoint{4.336475in}{6.598964in}}%
\pgfpathlineto{\pgfqpoint{4.283431in}{6.251128in}}%
\pgfpathlineto{\pgfqpoint{4.230387in}{6.251128in}}%
\pgfpathlineto{\pgfqpoint{4.177343in}{6.251128in}}%
\pgfpathlineto{\pgfqpoint{4.124299in}{6.251128in}}%
\pgfpathlineto{\pgfqpoint{4.071255in}{6.251128in}}%
\pgfpathlineto{\pgfqpoint{4.018211in}{6.251128in}}%
\pgfpathlineto{\pgfqpoint{3.965167in}{6.340259in}}%
\pgfpathlineto{\pgfqpoint{3.912123in}{6.481634in}}%
\pgfpathlineto{\pgfqpoint{3.859079in}{6.572998in}}%
\pgfpathlineto{\pgfqpoint{3.806035in}{6.251128in}}%
\pgfpathlineto{\pgfqpoint{3.752991in}{6.510498in}}%
\pgfpathlineto{\pgfqpoint{3.699948in}{6.251128in}}%
\pgfpathlineto{\pgfqpoint{3.646904in}{6.251128in}}%
\pgfpathlineto{\pgfqpoint{3.593860in}{6.251128in}}%
\pgfpathlineto{\pgfqpoint{3.540816in}{6.427614in}}%
\pgfpathlineto{\pgfqpoint{3.487772in}{6.598964in}}%
\pgfpathlineto{\pgfqpoint{3.434728in}{6.251128in}}%
\pgfpathlineto{\pgfqpoint{3.381684in}{6.598964in}}%
\pgfpathlineto{\pgfqpoint{3.328640in}{6.251128in}}%
\pgfpathlineto{\pgfqpoint{3.275596in}{6.251128in}}%
\pgfpathlineto{\pgfqpoint{3.222552in}{6.251128in}}%
\pgfpathlineto{\pgfqpoint{3.169508in}{6.251128in}}%
\pgfpathlineto{\pgfqpoint{3.116464in}{6.598964in}}%
\pgfpathlineto{\pgfqpoint{3.063420in}{6.598964in}}%
\pgfpathlineto{\pgfqpoint{3.010376in}{6.251128in}}%
\pgfpathlineto{\pgfqpoint{2.957332in}{6.598964in}}%
\pgfpathlineto{\pgfqpoint{2.904288in}{6.251128in}}%
\pgfpathlineto{\pgfqpoint{2.851245in}{6.598964in}}%
\pgfpathlineto{\pgfqpoint{2.798201in}{6.251128in}}%
\pgfpathlineto{\pgfqpoint{2.745157in}{6.251128in}}%
\pgfpathlineto{\pgfqpoint{2.692113in}{6.251128in}}%
\pgfpathlineto{\pgfqpoint{2.639069in}{6.251128in}}%
\pgfpathlineto{\pgfqpoint{2.586025in}{6.251128in}}%
\pgfpathlineto{\pgfqpoint{2.532981in}{6.251128in}}%
\pgfpathlineto{\pgfqpoint{2.479937in}{6.251128in}}%
\pgfpathlineto{\pgfqpoint{2.426893in}{6.251128in}}%
\pgfpathlineto{\pgfqpoint{2.373849in}{6.541500in}}%
\pgfpathlineto{\pgfqpoint{2.320805in}{6.251128in}}%
\pgfpathlineto{\pgfqpoint{2.267761in}{6.251128in}}%
\pgfpathlineto{\pgfqpoint{2.214717in}{6.459099in}}%
\pgfpathlineto{\pgfqpoint{2.161673in}{6.251128in}}%
\pgfpathlineto{\pgfqpoint{2.108629in}{6.348747in}}%
\pgfpathlineto{\pgfqpoint{2.055586in}{6.251128in}}%
\pgfpathlineto{\pgfqpoint{2.002542in}{6.505445in}}%
\pgfpathlineto{\pgfqpoint{1.949498in}{6.251128in}}%
\pgfpathlineto{\pgfqpoint{1.896454in}{6.598964in}}%
\pgfpathlineto{\pgfqpoint{1.843410in}{6.598964in}}%
\pgfpathlineto{\pgfqpoint{1.790366in}{6.251128in}}%
\pgfpathlineto{\pgfqpoint{1.737322in}{6.251128in}}%
\pgfpathlineto{\pgfqpoint{1.684278in}{6.598964in}}%
\pgfpathlineto{\pgfqpoint{1.631234in}{6.598964in}}%
\pgfpathlineto{\pgfqpoint{1.578190in}{6.251128in}}%
\pgfpathlineto{\pgfqpoint{1.525146in}{6.251128in}}%
\pgfpathlineto{\pgfqpoint{1.472102in}{6.251128in}}%
\pgfpathlineto{\pgfqpoint{1.419058in}{6.251128in}}%
\pgfpathlineto{\pgfqpoint{1.366014in}{6.251128in}}%
\pgfpathlineto{\pgfqpoint{1.312970in}{6.598964in}}%
\pgfpathlineto{\pgfqpoint{1.259927in}{6.598964in}}%
\pgfpathlineto{\pgfqpoint{1.206883in}{6.598964in}}%
\pgfpathlineto{\pgfqpoint{1.153839in}{6.251128in}}%
\pgfpathlineto{\pgfqpoint{1.100795in}{6.251128in}}%
\pgfpathlineto{\pgfqpoint{1.047751in}{6.251128in}}%
\pgfpathlineto{\pgfqpoint{0.994707in}{6.261680in}}%
\pgfpathlineto{\pgfqpoint{0.941663in}{6.251128in}}%
\pgfpathlineto{\pgfqpoint{0.941663in}{6.251128in}}%
\pgfpathclose%
\pgfusepath{stroke,fill}%
}%
\begin{pgfscope}%
\pgfsys@transformshift{0.000000in}{0.000000in}%
\pgfsys@useobject{currentmarker}{}%
\end{pgfscope}%
\end{pgfscope}%
\begin{pgfscope}%
\pgfpathrectangle{\pgfqpoint{0.941663in}{4.334375in}}{\pgfqpoint{8.858337in}{3.465625in}}%
\pgfusepath{clip}%
\pgfsetrectcap%
\pgfsetroundjoin%
\pgfsetlinewidth{1.505625pt}%
\definecolor{currentstroke}{rgb}{0.549020,0.337255,0.294118}%
\pgfsetstrokecolor{currentstroke}%
\pgfsetdash{}{0pt}%
\pgfpathmoveto{\pgfqpoint{0.941663in}{6.251128in}}%
\pgfpathlineto{\pgfqpoint{0.994707in}{6.261680in}}%
\pgfpathlineto{\pgfqpoint{1.047751in}{6.251128in}}%
\pgfpathlineto{\pgfqpoint{1.153839in}{6.251128in}}%
\pgfpathlineto{\pgfqpoint{1.206883in}{6.598964in}}%
\pgfpathlineto{\pgfqpoint{1.312970in}{6.598964in}}%
\pgfpathlineto{\pgfqpoint{1.366014in}{6.251128in}}%
\pgfpathlineto{\pgfqpoint{1.578190in}{6.251128in}}%
\pgfpathlineto{\pgfqpoint{1.631234in}{6.598964in}}%
\pgfpathlineto{\pgfqpoint{1.684278in}{6.598964in}}%
\pgfpathlineto{\pgfqpoint{1.737322in}{6.251128in}}%
\pgfpathlineto{\pgfqpoint{1.790366in}{6.251128in}}%
\pgfpathlineto{\pgfqpoint{1.843410in}{6.734592in}}%
\pgfpathlineto{\pgfqpoint{1.896454in}{6.883587in}}%
\pgfpathlineto{\pgfqpoint{1.949498in}{6.251128in}}%
\pgfpathlineto{\pgfqpoint{2.002542in}{6.724358in}}%
\pgfpathlineto{\pgfqpoint{2.055586in}{6.251128in}}%
\pgfpathlineto{\pgfqpoint{2.108629in}{6.348747in}}%
\pgfpathlineto{\pgfqpoint{2.161673in}{6.251128in}}%
\pgfpathlineto{\pgfqpoint{2.214717in}{6.574971in}}%
\pgfpathlineto{\pgfqpoint{2.267761in}{6.251128in}}%
\pgfpathlineto{\pgfqpoint{2.320805in}{6.251128in}}%
\pgfpathlineto{\pgfqpoint{2.373849in}{6.541500in}}%
\pgfpathlineto{\pgfqpoint{2.426893in}{6.606651in}}%
\pgfpathlineto{\pgfqpoint{2.479937in}{6.251128in}}%
\pgfpathlineto{\pgfqpoint{2.798201in}{6.251128in}}%
\pgfpathlineto{\pgfqpoint{2.851245in}{6.598964in}}%
\pgfpathlineto{\pgfqpoint{2.904288in}{6.251128in}}%
\pgfpathlineto{\pgfqpoint{2.957332in}{6.744748in}}%
\pgfpathlineto{\pgfqpoint{3.010376in}{6.251128in}}%
\pgfpathlineto{\pgfqpoint{3.063420in}{6.598964in}}%
\pgfpathlineto{\pgfqpoint{3.116464in}{6.598964in}}%
\pgfpathlineto{\pgfqpoint{3.169508in}{6.251128in}}%
\pgfpathlineto{\pgfqpoint{3.328640in}{6.251128in}}%
\pgfpathlineto{\pgfqpoint{3.381684in}{6.598964in}}%
\pgfpathlineto{\pgfqpoint{3.434728in}{6.251128in}}%
\pgfpathlineto{\pgfqpoint{3.487772in}{6.598964in}}%
\pgfpathlineto{\pgfqpoint{3.540816in}{6.572638in}}%
\pgfpathlineto{\pgfqpoint{3.593860in}{6.251128in}}%
\pgfpathlineto{\pgfqpoint{3.699948in}{6.251128in}}%
\pgfpathlineto{\pgfqpoint{3.752991in}{6.510498in}}%
\pgfpathlineto{\pgfqpoint{3.806035in}{6.251128in}}%
\pgfpathlineto{\pgfqpoint{3.859079in}{6.572998in}}%
\pgfpathlineto{\pgfqpoint{3.912123in}{6.481634in}}%
\pgfpathlineto{\pgfqpoint{3.965167in}{6.340259in}}%
\pgfpathlineto{\pgfqpoint{4.018211in}{6.251128in}}%
\pgfpathlineto{\pgfqpoint{4.283431in}{6.251128in}}%
\pgfpathlineto{\pgfqpoint{4.336475in}{6.889082in}}%
\pgfpathlineto{\pgfqpoint{4.389519in}{6.299258in}}%
\pgfpathlineto{\pgfqpoint{4.442563in}{6.598964in}}%
\pgfpathlineto{\pgfqpoint{4.495607in}{6.633012in}}%
\pgfpathlineto{\pgfqpoint{4.548650in}{6.251128in}}%
\pgfpathlineto{\pgfqpoint{4.601694in}{6.592100in}}%
\pgfpathlineto{\pgfqpoint{4.654738in}{6.251128in}}%
\pgfpathlineto{\pgfqpoint{4.707782in}{6.766257in}}%
\pgfpathlineto{\pgfqpoint{4.760826in}{6.251128in}}%
\pgfpathlineto{\pgfqpoint{4.813870in}{6.251128in}}%
\pgfpathlineto{\pgfqpoint{4.866914in}{6.548881in}}%
\pgfpathlineto{\pgfqpoint{4.919958in}{6.528186in}}%
\pgfpathlineto{\pgfqpoint{4.973002in}{6.251128in}}%
\pgfpathlineto{\pgfqpoint{5.026046in}{6.590550in}}%
\pgfpathlineto{\pgfqpoint{5.079090in}{6.251128in}}%
\pgfpathlineto{\pgfqpoint{5.132134in}{6.598964in}}%
\pgfpathlineto{\pgfqpoint{5.185178in}{6.574350in}}%
\pgfpathlineto{\pgfqpoint{5.238222in}{6.501650in}}%
\pgfpathlineto{\pgfqpoint{5.291266in}{6.251128in}}%
\pgfpathlineto{\pgfqpoint{5.344309in}{6.471024in}}%
\pgfpathlineto{\pgfqpoint{5.397353in}{6.325005in}}%
\pgfpathlineto{\pgfqpoint{5.450397in}{6.770026in}}%
\pgfpathlineto{\pgfqpoint{5.503441in}{6.251128in}}%
\pgfpathlineto{\pgfqpoint{5.556485in}{6.329843in}}%
\pgfpathlineto{\pgfqpoint{5.609529in}{6.514446in}}%
\pgfpathlineto{\pgfqpoint{5.662573in}{6.375963in}}%
\pgfpathlineto{\pgfqpoint{5.715617in}{6.251128in}}%
\pgfpathlineto{\pgfqpoint{5.768661in}{6.913209in}}%
\pgfpathlineto{\pgfqpoint{5.821705in}{6.872997in}}%
\pgfpathlineto{\pgfqpoint{5.874749in}{6.702075in}}%
\pgfpathlineto{\pgfqpoint{5.927793in}{6.598351in}}%
\pgfpathlineto{\pgfqpoint{5.980837in}{6.251128in}}%
\pgfpathlineto{\pgfqpoint{6.033881in}{6.400407in}}%
\pgfpathlineto{\pgfqpoint{6.086925in}{6.251128in}}%
\pgfpathlineto{\pgfqpoint{6.139969in}{6.251128in}}%
\pgfpathlineto{\pgfqpoint{6.193012in}{6.635398in}}%
\pgfpathlineto{\pgfqpoint{6.246056in}{6.691111in}}%
\pgfpathlineto{\pgfqpoint{6.299100in}{6.251128in}}%
\pgfpathlineto{\pgfqpoint{6.352144in}{6.569516in}}%
\pgfpathlineto{\pgfqpoint{6.405188in}{6.364054in}}%
\pgfpathlineto{\pgfqpoint{6.458232in}{6.251128in}}%
\pgfpathlineto{\pgfqpoint{6.723452in}{6.251128in}}%
\pgfpathlineto{\pgfqpoint{6.776496in}{6.385890in}}%
\pgfpathlineto{\pgfqpoint{6.829540in}{6.251128in}}%
\pgfpathlineto{\pgfqpoint{6.882584in}{6.780238in}}%
\pgfpathlineto{\pgfqpoint{6.935628in}{6.251128in}}%
\pgfpathlineto{\pgfqpoint{6.988671in}{6.598964in}}%
\pgfpathlineto{\pgfqpoint{7.041715in}{6.251128in}}%
\pgfpathlineto{\pgfqpoint{7.147803in}{6.251128in}}%
\pgfpathlineto{\pgfqpoint{7.200847in}{6.729225in}}%
\pgfpathlineto{\pgfqpoint{7.253891in}{6.451777in}}%
\pgfpathlineto{\pgfqpoint{7.306935in}{6.251128in}}%
\pgfpathlineto{\pgfqpoint{7.359979in}{6.470643in}}%
\pgfpathlineto{\pgfqpoint{7.413023in}{6.529690in}}%
\pgfpathlineto{\pgfqpoint{7.466067in}{6.251128in}}%
\pgfpathlineto{\pgfqpoint{7.519111in}{6.251128in}}%
\pgfpathlineto{\pgfqpoint{7.572155in}{6.487962in}}%
\pgfpathlineto{\pgfqpoint{7.625199in}{6.652740in}}%
\pgfpathlineto{\pgfqpoint{7.678243in}{6.251128in}}%
\pgfpathlineto{\pgfqpoint{7.996506in}{6.251128in}}%
\pgfpathlineto{\pgfqpoint{8.049550in}{6.766484in}}%
\pgfpathlineto{\pgfqpoint{8.102594in}{6.782745in}}%
\pgfpathlineto{\pgfqpoint{8.155638in}{6.251128in}}%
\pgfpathlineto{\pgfqpoint{8.208682in}{6.251128in}}%
\pgfpathlineto{\pgfqpoint{8.261726in}{6.924712in}}%
\pgfpathlineto{\pgfqpoint{8.314770in}{6.876087in}}%
\pgfpathlineto{\pgfqpoint{8.367814in}{6.251128in}}%
\pgfpathlineto{\pgfqpoint{8.420858in}{6.653636in}}%
\pgfpathlineto{\pgfqpoint{8.473902in}{6.251128in}}%
\pgfpathlineto{\pgfqpoint{8.526946in}{6.251128in}}%
\pgfpathlineto{\pgfqpoint{8.579990in}{6.685501in}}%
\pgfpathlineto{\pgfqpoint{8.633033in}{6.520595in}}%
\pgfpathlineto{\pgfqpoint{8.686077in}{6.723687in}}%
\pgfpathlineto{\pgfqpoint{8.739121in}{6.489640in}}%
\pgfpathlineto{\pgfqpoint{8.792165in}{6.251128in}}%
\pgfpathlineto{\pgfqpoint{8.845209in}{6.299108in}}%
\pgfpathlineto{\pgfqpoint{8.898253in}{6.354195in}}%
\pgfpathlineto{\pgfqpoint{8.951297in}{6.251128in}}%
\pgfpathlineto{\pgfqpoint{9.004341in}{6.568668in}}%
\pgfpathlineto{\pgfqpoint{9.057385in}{6.251128in}}%
\pgfpathlineto{\pgfqpoint{9.110429in}{6.251128in}}%
\pgfpathlineto{\pgfqpoint{9.163473in}{6.668425in}}%
\pgfpathlineto{\pgfqpoint{9.216517in}{6.804492in}}%
\pgfpathlineto{\pgfqpoint{9.269561in}{6.628250in}}%
\pgfpathlineto{\pgfqpoint{9.322605in}{6.727450in}}%
\pgfpathlineto{\pgfqpoint{9.375649in}{6.754769in}}%
\pgfpathlineto{\pgfqpoint{9.428692in}{6.251128in}}%
\pgfpathlineto{\pgfqpoint{9.481736in}{6.256205in}}%
\pgfpathlineto{\pgfqpoint{9.534780in}{6.946800in}}%
\pgfpathlineto{\pgfqpoint{9.587824in}{6.621011in}}%
\pgfpathlineto{\pgfqpoint{9.640868in}{6.251128in}}%
\pgfpathlineto{\pgfqpoint{9.693912in}{6.632876in}}%
\pgfpathlineto{\pgfqpoint{9.746956in}{6.779712in}}%
\pgfpathlineto{\pgfqpoint{9.800000in}{6.573653in}}%
\pgfpathlineto{\pgfqpoint{9.800000in}{6.573653in}}%
\pgfusepath{stroke}%
\end{pgfscope}%
\begin{pgfscope}%
\pgfpathrectangle{\pgfqpoint{0.941663in}{4.334375in}}{\pgfqpoint{8.858337in}{3.465625in}}%
\pgfusepath{clip}%
\pgfsetbuttcap%
\pgfsetroundjoin%
\definecolor{currentfill}{rgb}{0.549020,0.337255,0.294118}%
\pgfsetfillcolor{currentfill}%
\pgfsetlinewidth{1.003750pt}%
\definecolor{currentstroke}{rgb}{0.549020,0.337255,0.294118}%
\pgfsetstrokecolor{currentstroke}%
\pgfsetdash{}{0pt}%
\pgfsys@defobject{currentmarker}{\pgfqpoint{0.941663in}{6.251128in}}{\pgfqpoint{9.800000in}{6.946800in}}{%
\pgfpathmoveto{\pgfqpoint{0.941663in}{6.251128in}}%
\pgfpathlineto{\pgfqpoint{0.941663in}{6.251128in}}%
\pgfpathlineto{\pgfqpoint{0.994707in}{6.261680in}}%
\pgfpathlineto{\pgfqpoint{1.047751in}{6.251128in}}%
\pgfpathlineto{\pgfqpoint{1.100795in}{6.251128in}}%
\pgfpathlineto{\pgfqpoint{1.153839in}{6.251128in}}%
\pgfpathlineto{\pgfqpoint{1.206883in}{6.598964in}}%
\pgfpathlineto{\pgfqpoint{1.259927in}{6.598964in}}%
\pgfpathlineto{\pgfqpoint{1.312970in}{6.598964in}}%
\pgfpathlineto{\pgfqpoint{1.366014in}{6.251128in}}%
\pgfpathlineto{\pgfqpoint{1.419058in}{6.251128in}}%
\pgfpathlineto{\pgfqpoint{1.472102in}{6.251128in}}%
\pgfpathlineto{\pgfqpoint{1.525146in}{6.251128in}}%
\pgfpathlineto{\pgfqpoint{1.578190in}{6.251128in}}%
\pgfpathlineto{\pgfqpoint{1.631234in}{6.598964in}}%
\pgfpathlineto{\pgfqpoint{1.684278in}{6.598964in}}%
\pgfpathlineto{\pgfqpoint{1.737322in}{6.251128in}}%
\pgfpathlineto{\pgfqpoint{1.790366in}{6.251128in}}%
\pgfpathlineto{\pgfqpoint{1.843410in}{6.598964in}}%
\pgfpathlineto{\pgfqpoint{1.896454in}{6.598964in}}%
\pgfpathlineto{\pgfqpoint{1.949498in}{6.251128in}}%
\pgfpathlineto{\pgfqpoint{2.002542in}{6.505445in}}%
\pgfpathlineto{\pgfqpoint{2.055586in}{6.251128in}}%
\pgfpathlineto{\pgfqpoint{2.108629in}{6.348747in}}%
\pgfpathlineto{\pgfqpoint{2.161673in}{6.251128in}}%
\pgfpathlineto{\pgfqpoint{2.214717in}{6.459099in}}%
\pgfpathlineto{\pgfqpoint{2.267761in}{6.251128in}}%
\pgfpathlineto{\pgfqpoint{2.320805in}{6.251128in}}%
\pgfpathlineto{\pgfqpoint{2.373849in}{6.541500in}}%
\pgfpathlineto{\pgfqpoint{2.426893in}{6.251128in}}%
\pgfpathlineto{\pgfqpoint{2.479937in}{6.251128in}}%
\pgfpathlineto{\pgfqpoint{2.532981in}{6.251128in}}%
\pgfpathlineto{\pgfqpoint{2.586025in}{6.251128in}}%
\pgfpathlineto{\pgfqpoint{2.639069in}{6.251128in}}%
\pgfpathlineto{\pgfqpoint{2.692113in}{6.251128in}}%
\pgfpathlineto{\pgfqpoint{2.745157in}{6.251128in}}%
\pgfpathlineto{\pgfqpoint{2.798201in}{6.251128in}}%
\pgfpathlineto{\pgfqpoint{2.851245in}{6.598964in}}%
\pgfpathlineto{\pgfqpoint{2.904288in}{6.251128in}}%
\pgfpathlineto{\pgfqpoint{2.957332in}{6.598964in}}%
\pgfpathlineto{\pgfqpoint{3.010376in}{6.251128in}}%
\pgfpathlineto{\pgfqpoint{3.063420in}{6.598964in}}%
\pgfpathlineto{\pgfqpoint{3.116464in}{6.598964in}}%
\pgfpathlineto{\pgfqpoint{3.169508in}{6.251128in}}%
\pgfpathlineto{\pgfqpoint{3.222552in}{6.251128in}}%
\pgfpathlineto{\pgfqpoint{3.275596in}{6.251128in}}%
\pgfpathlineto{\pgfqpoint{3.328640in}{6.251128in}}%
\pgfpathlineto{\pgfqpoint{3.381684in}{6.598964in}}%
\pgfpathlineto{\pgfqpoint{3.434728in}{6.251128in}}%
\pgfpathlineto{\pgfqpoint{3.487772in}{6.598964in}}%
\pgfpathlineto{\pgfqpoint{3.540816in}{6.427614in}}%
\pgfpathlineto{\pgfqpoint{3.593860in}{6.251128in}}%
\pgfpathlineto{\pgfqpoint{3.646904in}{6.251128in}}%
\pgfpathlineto{\pgfqpoint{3.699948in}{6.251128in}}%
\pgfpathlineto{\pgfqpoint{3.752991in}{6.510498in}}%
\pgfpathlineto{\pgfqpoint{3.806035in}{6.251128in}}%
\pgfpathlineto{\pgfqpoint{3.859079in}{6.572998in}}%
\pgfpathlineto{\pgfqpoint{3.912123in}{6.481634in}}%
\pgfpathlineto{\pgfqpoint{3.965167in}{6.340259in}}%
\pgfpathlineto{\pgfqpoint{4.018211in}{6.251128in}}%
\pgfpathlineto{\pgfqpoint{4.071255in}{6.251128in}}%
\pgfpathlineto{\pgfqpoint{4.124299in}{6.251128in}}%
\pgfpathlineto{\pgfqpoint{4.177343in}{6.251128in}}%
\pgfpathlineto{\pgfqpoint{4.230387in}{6.251128in}}%
\pgfpathlineto{\pgfqpoint{4.283431in}{6.251128in}}%
\pgfpathlineto{\pgfqpoint{4.336475in}{6.598964in}}%
\pgfpathlineto{\pgfqpoint{4.389519in}{6.299258in}}%
\pgfpathlineto{\pgfqpoint{4.442563in}{6.598964in}}%
\pgfpathlineto{\pgfqpoint{4.495607in}{6.437705in}}%
\pgfpathlineto{\pgfqpoint{4.548650in}{6.251128in}}%
\pgfpathlineto{\pgfqpoint{4.601694in}{6.251128in}}%
\pgfpathlineto{\pgfqpoint{4.654738in}{6.251128in}}%
\pgfpathlineto{\pgfqpoint{4.707782in}{6.351811in}}%
\pgfpathlineto{\pgfqpoint{4.760826in}{6.251128in}}%
\pgfpathlineto{\pgfqpoint{4.813870in}{6.251128in}}%
\pgfpathlineto{\pgfqpoint{4.866914in}{6.471548in}}%
\pgfpathlineto{\pgfqpoint{4.919958in}{6.251128in}}%
\pgfpathlineto{\pgfqpoint{4.973002in}{6.251128in}}%
\pgfpathlineto{\pgfqpoint{5.026046in}{6.590550in}}%
\pgfpathlineto{\pgfqpoint{5.079090in}{6.251128in}}%
\pgfpathlineto{\pgfqpoint{5.132134in}{6.598964in}}%
\pgfpathlineto{\pgfqpoint{5.185178in}{6.251128in}}%
\pgfpathlineto{\pgfqpoint{5.238222in}{6.251128in}}%
\pgfpathlineto{\pgfqpoint{5.291266in}{6.251128in}}%
\pgfpathlineto{\pgfqpoint{5.344309in}{6.471024in}}%
\pgfpathlineto{\pgfqpoint{5.397353in}{6.251128in}}%
\pgfpathlineto{\pgfqpoint{5.450397in}{6.251128in}}%
\pgfpathlineto{\pgfqpoint{5.503441in}{6.251128in}}%
\pgfpathlineto{\pgfqpoint{5.556485in}{6.329843in}}%
\pgfpathlineto{\pgfqpoint{5.609529in}{6.251128in}}%
\pgfpathlineto{\pgfqpoint{5.662573in}{6.251128in}}%
\pgfpathlineto{\pgfqpoint{5.715617in}{6.251128in}}%
\pgfpathlineto{\pgfqpoint{5.768661in}{6.324051in}}%
\pgfpathlineto{\pgfqpoint{5.821705in}{6.251128in}}%
\pgfpathlineto{\pgfqpoint{5.874749in}{6.251128in}}%
\pgfpathlineto{\pgfqpoint{5.927793in}{6.251128in}}%
\pgfpathlineto{\pgfqpoint{5.980837in}{6.251128in}}%
\pgfpathlineto{\pgfqpoint{6.033881in}{6.400407in}}%
\pgfpathlineto{\pgfqpoint{6.086925in}{6.251128in}}%
\pgfpathlineto{\pgfqpoint{6.139969in}{6.251128in}}%
\pgfpathlineto{\pgfqpoint{6.193012in}{6.598964in}}%
\pgfpathlineto{\pgfqpoint{6.246056in}{6.362599in}}%
\pgfpathlineto{\pgfqpoint{6.299100in}{6.251128in}}%
\pgfpathlineto{\pgfqpoint{6.352144in}{6.569516in}}%
\pgfpathlineto{\pgfqpoint{6.405188in}{6.251128in}}%
\pgfpathlineto{\pgfqpoint{6.458232in}{6.251128in}}%
\pgfpathlineto{\pgfqpoint{6.511276in}{6.251128in}}%
\pgfpathlineto{\pgfqpoint{6.564320in}{6.251128in}}%
\pgfpathlineto{\pgfqpoint{6.617364in}{6.251128in}}%
\pgfpathlineto{\pgfqpoint{6.670408in}{6.251128in}}%
\pgfpathlineto{\pgfqpoint{6.723452in}{6.251128in}}%
\pgfpathlineto{\pgfqpoint{6.776496in}{6.385890in}}%
\pgfpathlineto{\pgfqpoint{6.829540in}{6.251128in}}%
\pgfpathlineto{\pgfqpoint{6.882584in}{6.598964in}}%
\pgfpathlineto{\pgfqpoint{6.935628in}{6.251128in}}%
\pgfpathlineto{\pgfqpoint{6.988671in}{6.598964in}}%
\pgfpathlineto{\pgfqpoint{7.041715in}{6.251128in}}%
\pgfpathlineto{\pgfqpoint{7.094759in}{6.251128in}}%
\pgfpathlineto{\pgfqpoint{7.147803in}{6.251128in}}%
\pgfpathlineto{\pgfqpoint{7.200847in}{6.598964in}}%
\pgfpathlineto{\pgfqpoint{7.253891in}{6.451777in}}%
\pgfpathlineto{\pgfqpoint{7.306935in}{6.251128in}}%
\pgfpathlineto{\pgfqpoint{7.359979in}{6.350974in}}%
\pgfpathlineto{\pgfqpoint{7.413023in}{6.251128in}}%
\pgfpathlineto{\pgfqpoint{7.466067in}{6.251128in}}%
\pgfpathlineto{\pgfqpoint{7.519111in}{6.251128in}}%
\pgfpathlineto{\pgfqpoint{7.572155in}{6.251128in}}%
\pgfpathlineto{\pgfqpoint{7.625199in}{6.251128in}}%
\pgfpathlineto{\pgfqpoint{7.678243in}{6.251128in}}%
\pgfpathlineto{\pgfqpoint{7.731287in}{6.251128in}}%
\pgfpathlineto{\pgfqpoint{7.784330in}{6.251128in}}%
\pgfpathlineto{\pgfqpoint{7.837374in}{6.251128in}}%
\pgfpathlineto{\pgfqpoint{7.890418in}{6.251128in}}%
\pgfpathlineto{\pgfqpoint{7.943462in}{6.251128in}}%
\pgfpathlineto{\pgfqpoint{7.996506in}{6.251128in}}%
\pgfpathlineto{\pgfqpoint{8.049550in}{6.598964in}}%
\pgfpathlineto{\pgfqpoint{8.102594in}{6.598964in}}%
\pgfpathlineto{\pgfqpoint{8.155638in}{6.251128in}}%
\pgfpathlineto{\pgfqpoint{8.208682in}{6.251128in}}%
\pgfpathlineto{\pgfqpoint{8.261726in}{6.598964in}}%
\pgfpathlineto{\pgfqpoint{8.314770in}{6.598964in}}%
\pgfpathlineto{\pgfqpoint{8.367814in}{6.251128in}}%
\pgfpathlineto{\pgfqpoint{8.420858in}{6.371312in}}%
\pgfpathlineto{\pgfqpoint{8.473902in}{6.251128in}}%
\pgfpathlineto{\pgfqpoint{8.526946in}{6.251128in}}%
\pgfpathlineto{\pgfqpoint{8.579990in}{6.560479in}}%
\pgfpathlineto{\pgfqpoint{8.633033in}{6.251128in}}%
\pgfpathlineto{\pgfqpoint{8.686077in}{6.251128in}}%
\pgfpathlineto{\pgfqpoint{8.739121in}{6.251128in}}%
\pgfpathlineto{\pgfqpoint{8.792165in}{6.251128in}}%
\pgfpathlineto{\pgfqpoint{8.845209in}{6.251128in}}%
\pgfpathlineto{\pgfqpoint{8.898253in}{6.251128in}}%
\pgfpathlineto{\pgfqpoint{8.951297in}{6.251128in}}%
\pgfpathlineto{\pgfqpoint{9.004341in}{6.568668in}}%
\pgfpathlineto{\pgfqpoint{9.057385in}{6.251128in}}%
\pgfpathlineto{\pgfqpoint{9.110429in}{6.251128in}}%
\pgfpathlineto{\pgfqpoint{9.163473in}{6.598964in}}%
\pgfpathlineto{\pgfqpoint{9.216517in}{6.334558in}}%
\pgfpathlineto{\pgfqpoint{9.269561in}{6.251128in}}%
\pgfpathlineto{\pgfqpoint{9.322605in}{6.251128in}}%
\pgfpathlineto{\pgfqpoint{9.375649in}{6.251128in}}%
\pgfpathlineto{\pgfqpoint{9.428692in}{6.251128in}}%
\pgfpathlineto{\pgfqpoint{9.481736in}{6.256205in}}%
\pgfpathlineto{\pgfqpoint{9.534780in}{6.287774in}}%
\pgfpathlineto{\pgfqpoint{9.587824in}{6.251128in}}%
\pgfpathlineto{\pgfqpoint{9.640868in}{6.251128in}}%
\pgfpathlineto{\pgfqpoint{9.693912in}{6.251128in}}%
\pgfpathlineto{\pgfqpoint{9.746956in}{6.251128in}}%
\pgfpathlineto{\pgfqpoint{9.800000in}{6.251128in}}%
\pgfpathlineto{\pgfqpoint{9.800000in}{6.573653in}}%
\pgfpathlineto{\pgfqpoint{9.800000in}{6.573653in}}%
\pgfpathlineto{\pgfqpoint{9.746956in}{6.779712in}}%
\pgfpathlineto{\pgfqpoint{9.693912in}{6.632876in}}%
\pgfpathlineto{\pgfqpoint{9.640868in}{6.251128in}}%
\pgfpathlineto{\pgfqpoint{9.587824in}{6.621011in}}%
\pgfpathlineto{\pgfqpoint{9.534780in}{6.946800in}}%
\pgfpathlineto{\pgfqpoint{9.481736in}{6.256205in}}%
\pgfpathlineto{\pgfqpoint{9.428692in}{6.251128in}}%
\pgfpathlineto{\pgfqpoint{9.375649in}{6.754769in}}%
\pgfpathlineto{\pgfqpoint{9.322605in}{6.727450in}}%
\pgfpathlineto{\pgfqpoint{9.269561in}{6.628250in}}%
\pgfpathlineto{\pgfqpoint{9.216517in}{6.804492in}}%
\pgfpathlineto{\pgfqpoint{9.163473in}{6.668425in}}%
\pgfpathlineto{\pgfqpoint{9.110429in}{6.251128in}}%
\pgfpathlineto{\pgfqpoint{9.057385in}{6.251128in}}%
\pgfpathlineto{\pgfqpoint{9.004341in}{6.568668in}}%
\pgfpathlineto{\pgfqpoint{8.951297in}{6.251128in}}%
\pgfpathlineto{\pgfqpoint{8.898253in}{6.354195in}}%
\pgfpathlineto{\pgfqpoint{8.845209in}{6.299108in}}%
\pgfpathlineto{\pgfqpoint{8.792165in}{6.251128in}}%
\pgfpathlineto{\pgfqpoint{8.739121in}{6.489640in}}%
\pgfpathlineto{\pgfqpoint{8.686077in}{6.723687in}}%
\pgfpathlineto{\pgfqpoint{8.633033in}{6.520595in}}%
\pgfpathlineto{\pgfqpoint{8.579990in}{6.685501in}}%
\pgfpathlineto{\pgfqpoint{8.526946in}{6.251128in}}%
\pgfpathlineto{\pgfqpoint{8.473902in}{6.251128in}}%
\pgfpathlineto{\pgfqpoint{8.420858in}{6.653636in}}%
\pgfpathlineto{\pgfqpoint{8.367814in}{6.251128in}}%
\pgfpathlineto{\pgfqpoint{8.314770in}{6.876087in}}%
\pgfpathlineto{\pgfqpoint{8.261726in}{6.924712in}}%
\pgfpathlineto{\pgfqpoint{8.208682in}{6.251128in}}%
\pgfpathlineto{\pgfqpoint{8.155638in}{6.251128in}}%
\pgfpathlineto{\pgfqpoint{8.102594in}{6.782745in}}%
\pgfpathlineto{\pgfqpoint{8.049550in}{6.766484in}}%
\pgfpathlineto{\pgfqpoint{7.996506in}{6.251128in}}%
\pgfpathlineto{\pgfqpoint{7.943462in}{6.251128in}}%
\pgfpathlineto{\pgfqpoint{7.890418in}{6.251128in}}%
\pgfpathlineto{\pgfqpoint{7.837374in}{6.251128in}}%
\pgfpathlineto{\pgfqpoint{7.784330in}{6.251128in}}%
\pgfpathlineto{\pgfqpoint{7.731287in}{6.251128in}}%
\pgfpathlineto{\pgfqpoint{7.678243in}{6.251128in}}%
\pgfpathlineto{\pgfqpoint{7.625199in}{6.652740in}}%
\pgfpathlineto{\pgfqpoint{7.572155in}{6.487962in}}%
\pgfpathlineto{\pgfqpoint{7.519111in}{6.251128in}}%
\pgfpathlineto{\pgfqpoint{7.466067in}{6.251128in}}%
\pgfpathlineto{\pgfqpoint{7.413023in}{6.529690in}}%
\pgfpathlineto{\pgfqpoint{7.359979in}{6.470643in}}%
\pgfpathlineto{\pgfqpoint{7.306935in}{6.251128in}}%
\pgfpathlineto{\pgfqpoint{7.253891in}{6.451777in}}%
\pgfpathlineto{\pgfqpoint{7.200847in}{6.729225in}}%
\pgfpathlineto{\pgfqpoint{7.147803in}{6.251128in}}%
\pgfpathlineto{\pgfqpoint{7.094759in}{6.251128in}}%
\pgfpathlineto{\pgfqpoint{7.041715in}{6.251128in}}%
\pgfpathlineto{\pgfqpoint{6.988671in}{6.598964in}}%
\pgfpathlineto{\pgfqpoint{6.935628in}{6.251128in}}%
\pgfpathlineto{\pgfqpoint{6.882584in}{6.780238in}}%
\pgfpathlineto{\pgfqpoint{6.829540in}{6.251128in}}%
\pgfpathlineto{\pgfqpoint{6.776496in}{6.385890in}}%
\pgfpathlineto{\pgfqpoint{6.723452in}{6.251128in}}%
\pgfpathlineto{\pgfqpoint{6.670408in}{6.251128in}}%
\pgfpathlineto{\pgfqpoint{6.617364in}{6.251128in}}%
\pgfpathlineto{\pgfqpoint{6.564320in}{6.251128in}}%
\pgfpathlineto{\pgfqpoint{6.511276in}{6.251128in}}%
\pgfpathlineto{\pgfqpoint{6.458232in}{6.251128in}}%
\pgfpathlineto{\pgfqpoint{6.405188in}{6.364054in}}%
\pgfpathlineto{\pgfqpoint{6.352144in}{6.569516in}}%
\pgfpathlineto{\pgfqpoint{6.299100in}{6.251128in}}%
\pgfpathlineto{\pgfqpoint{6.246056in}{6.691111in}}%
\pgfpathlineto{\pgfqpoint{6.193012in}{6.635398in}}%
\pgfpathlineto{\pgfqpoint{6.139969in}{6.251128in}}%
\pgfpathlineto{\pgfqpoint{6.086925in}{6.251128in}}%
\pgfpathlineto{\pgfqpoint{6.033881in}{6.400407in}}%
\pgfpathlineto{\pgfqpoint{5.980837in}{6.251128in}}%
\pgfpathlineto{\pgfqpoint{5.927793in}{6.598351in}}%
\pgfpathlineto{\pgfqpoint{5.874749in}{6.702075in}}%
\pgfpathlineto{\pgfqpoint{5.821705in}{6.872997in}}%
\pgfpathlineto{\pgfqpoint{5.768661in}{6.913209in}}%
\pgfpathlineto{\pgfqpoint{5.715617in}{6.251128in}}%
\pgfpathlineto{\pgfqpoint{5.662573in}{6.375963in}}%
\pgfpathlineto{\pgfqpoint{5.609529in}{6.514446in}}%
\pgfpathlineto{\pgfqpoint{5.556485in}{6.329843in}}%
\pgfpathlineto{\pgfqpoint{5.503441in}{6.251128in}}%
\pgfpathlineto{\pgfqpoint{5.450397in}{6.770026in}}%
\pgfpathlineto{\pgfqpoint{5.397353in}{6.325005in}}%
\pgfpathlineto{\pgfqpoint{5.344309in}{6.471024in}}%
\pgfpathlineto{\pgfqpoint{5.291266in}{6.251128in}}%
\pgfpathlineto{\pgfqpoint{5.238222in}{6.501650in}}%
\pgfpathlineto{\pgfqpoint{5.185178in}{6.574350in}}%
\pgfpathlineto{\pgfqpoint{5.132134in}{6.598964in}}%
\pgfpathlineto{\pgfqpoint{5.079090in}{6.251128in}}%
\pgfpathlineto{\pgfqpoint{5.026046in}{6.590550in}}%
\pgfpathlineto{\pgfqpoint{4.973002in}{6.251128in}}%
\pgfpathlineto{\pgfqpoint{4.919958in}{6.528186in}}%
\pgfpathlineto{\pgfqpoint{4.866914in}{6.548881in}}%
\pgfpathlineto{\pgfqpoint{4.813870in}{6.251128in}}%
\pgfpathlineto{\pgfqpoint{4.760826in}{6.251128in}}%
\pgfpathlineto{\pgfqpoint{4.707782in}{6.766257in}}%
\pgfpathlineto{\pgfqpoint{4.654738in}{6.251128in}}%
\pgfpathlineto{\pgfqpoint{4.601694in}{6.592100in}}%
\pgfpathlineto{\pgfqpoint{4.548650in}{6.251128in}}%
\pgfpathlineto{\pgfqpoint{4.495607in}{6.633012in}}%
\pgfpathlineto{\pgfqpoint{4.442563in}{6.598964in}}%
\pgfpathlineto{\pgfqpoint{4.389519in}{6.299258in}}%
\pgfpathlineto{\pgfqpoint{4.336475in}{6.889082in}}%
\pgfpathlineto{\pgfqpoint{4.283431in}{6.251128in}}%
\pgfpathlineto{\pgfqpoint{4.230387in}{6.251128in}}%
\pgfpathlineto{\pgfqpoint{4.177343in}{6.251128in}}%
\pgfpathlineto{\pgfqpoint{4.124299in}{6.251128in}}%
\pgfpathlineto{\pgfqpoint{4.071255in}{6.251128in}}%
\pgfpathlineto{\pgfqpoint{4.018211in}{6.251128in}}%
\pgfpathlineto{\pgfqpoint{3.965167in}{6.340259in}}%
\pgfpathlineto{\pgfqpoint{3.912123in}{6.481634in}}%
\pgfpathlineto{\pgfqpoint{3.859079in}{6.572998in}}%
\pgfpathlineto{\pgfqpoint{3.806035in}{6.251128in}}%
\pgfpathlineto{\pgfqpoint{3.752991in}{6.510498in}}%
\pgfpathlineto{\pgfqpoint{3.699948in}{6.251128in}}%
\pgfpathlineto{\pgfqpoint{3.646904in}{6.251128in}}%
\pgfpathlineto{\pgfqpoint{3.593860in}{6.251128in}}%
\pgfpathlineto{\pgfqpoint{3.540816in}{6.572638in}}%
\pgfpathlineto{\pgfqpoint{3.487772in}{6.598964in}}%
\pgfpathlineto{\pgfqpoint{3.434728in}{6.251128in}}%
\pgfpathlineto{\pgfqpoint{3.381684in}{6.598964in}}%
\pgfpathlineto{\pgfqpoint{3.328640in}{6.251128in}}%
\pgfpathlineto{\pgfqpoint{3.275596in}{6.251128in}}%
\pgfpathlineto{\pgfqpoint{3.222552in}{6.251128in}}%
\pgfpathlineto{\pgfqpoint{3.169508in}{6.251128in}}%
\pgfpathlineto{\pgfqpoint{3.116464in}{6.598964in}}%
\pgfpathlineto{\pgfqpoint{3.063420in}{6.598964in}}%
\pgfpathlineto{\pgfqpoint{3.010376in}{6.251128in}}%
\pgfpathlineto{\pgfqpoint{2.957332in}{6.744748in}}%
\pgfpathlineto{\pgfqpoint{2.904288in}{6.251128in}}%
\pgfpathlineto{\pgfqpoint{2.851245in}{6.598964in}}%
\pgfpathlineto{\pgfqpoint{2.798201in}{6.251128in}}%
\pgfpathlineto{\pgfqpoint{2.745157in}{6.251128in}}%
\pgfpathlineto{\pgfqpoint{2.692113in}{6.251128in}}%
\pgfpathlineto{\pgfqpoint{2.639069in}{6.251128in}}%
\pgfpathlineto{\pgfqpoint{2.586025in}{6.251128in}}%
\pgfpathlineto{\pgfqpoint{2.532981in}{6.251128in}}%
\pgfpathlineto{\pgfqpoint{2.479937in}{6.251128in}}%
\pgfpathlineto{\pgfqpoint{2.426893in}{6.606651in}}%
\pgfpathlineto{\pgfqpoint{2.373849in}{6.541500in}}%
\pgfpathlineto{\pgfqpoint{2.320805in}{6.251128in}}%
\pgfpathlineto{\pgfqpoint{2.267761in}{6.251128in}}%
\pgfpathlineto{\pgfqpoint{2.214717in}{6.574971in}}%
\pgfpathlineto{\pgfqpoint{2.161673in}{6.251128in}}%
\pgfpathlineto{\pgfqpoint{2.108629in}{6.348747in}}%
\pgfpathlineto{\pgfqpoint{2.055586in}{6.251128in}}%
\pgfpathlineto{\pgfqpoint{2.002542in}{6.724358in}}%
\pgfpathlineto{\pgfqpoint{1.949498in}{6.251128in}}%
\pgfpathlineto{\pgfqpoint{1.896454in}{6.883587in}}%
\pgfpathlineto{\pgfqpoint{1.843410in}{6.734592in}}%
\pgfpathlineto{\pgfqpoint{1.790366in}{6.251128in}}%
\pgfpathlineto{\pgfqpoint{1.737322in}{6.251128in}}%
\pgfpathlineto{\pgfqpoint{1.684278in}{6.598964in}}%
\pgfpathlineto{\pgfqpoint{1.631234in}{6.598964in}}%
\pgfpathlineto{\pgfqpoint{1.578190in}{6.251128in}}%
\pgfpathlineto{\pgfqpoint{1.525146in}{6.251128in}}%
\pgfpathlineto{\pgfqpoint{1.472102in}{6.251128in}}%
\pgfpathlineto{\pgfqpoint{1.419058in}{6.251128in}}%
\pgfpathlineto{\pgfqpoint{1.366014in}{6.251128in}}%
\pgfpathlineto{\pgfqpoint{1.312970in}{6.598964in}}%
\pgfpathlineto{\pgfqpoint{1.259927in}{6.598964in}}%
\pgfpathlineto{\pgfqpoint{1.206883in}{6.598964in}}%
\pgfpathlineto{\pgfqpoint{1.153839in}{6.251128in}}%
\pgfpathlineto{\pgfqpoint{1.100795in}{6.251128in}}%
\pgfpathlineto{\pgfqpoint{1.047751in}{6.251128in}}%
\pgfpathlineto{\pgfqpoint{0.994707in}{6.261680in}}%
\pgfpathlineto{\pgfqpoint{0.941663in}{6.251128in}}%
\pgfpathlineto{\pgfqpoint{0.941663in}{6.251128in}}%
\pgfpathclose%
\pgfusepath{stroke,fill}%
}%
\begin{pgfscope}%
\pgfsys@transformshift{0.000000in}{0.000000in}%
\pgfsys@useobject{currentmarker}{}%
\end{pgfscope}%
\end{pgfscope}%
\begin{pgfscope}%
\pgfpathrectangle{\pgfqpoint{0.941663in}{4.334375in}}{\pgfqpoint{8.858337in}{3.465625in}}%
\pgfusepath{clip}%
\pgfsetrectcap%
\pgfsetroundjoin%
\pgfsetlinewidth{1.505625pt}%
\definecolor{currentstroke}{rgb}{1.000000,0.647059,0.000000}%
\pgfsetstrokecolor{currentstroke}%
\pgfsetdash{}{0pt}%
\pgfpathmoveto{\pgfqpoint{0.941663in}{5.302080in}}%
\pgfpathlineto{\pgfqpoint{0.994707in}{5.555456in}}%
\pgfpathlineto{\pgfqpoint{1.047751in}{5.298031in}}%
\pgfpathlineto{\pgfqpoint{1.100795in}{5.267250in}}%
\pgfpathlineto{\pgfqpoint{1.153839in}{5.253587in}}%
\pgfpathlineto{\pgfqpoint{1.206883in}{5.555456in}}%
\pgfpathlineto{\pgfqpoint{1.312970in}{5.555456in}}%
\pgfpathlineto{\pgfqpoint{1.366014in}{5.207621in}}%
\pgfpathlineto{\pgfqpoint{1.419058in}{5.263460in}}%
\pgfpathlineto{\pgfqpoint{1.472102in}{5.268823in}}%
\pgfpathlineto{\pgfqpoint{1.525146in}{5.349185in}}%
\pgfpathlineto{\pgfqpoint{1.578190in}{5.343668in}}%
\pgfpathlineto{\pgfqpoint{1.631234in}{5.555456in}}%
\pgfpathlineto{\pgfqpoint{1.684278in}{5.555456in}}%
\pgfpathlineto{\pgfqpoint{1.737322in}{5.491300in}}%
\pgfpathlineto{\pgfqpoint{1.790366in}{5.473721in}}%
\pgfpathlineto{\pgfqpoint{1.843410in}{5.555456in}}%
\pgfpathlineto{\pgfqpoint{1.896454in}{5.555456in}}%
\pgfpathlineto{\pgfqpoint{1.949498in}{5.447032in}}%
\pgfpathlineto{\pgfqpoint{2.002542in}{5.555456in}}%
\pgfpathlineto{\pgfqpoint{2.055586in}{5.457838in}}%
\pgfpathlineto{\pgfqpoint{2.108629in}{5.555456in}}%
\pgfpathlineto{\pgfqpoint{2.161673in}{5.347486in}}%
\pgfpathlineto{\pgfqpoint{2.214717in}{5.555456in}}%
\pgfpathlineto{\pgfqpoint{2.267761in}{5.542464in}}%
\pgfpathlineto{\pgfqpoint{2.320805in}{5.278077in}}%
\pgfpathlineto{\pgfqpoint{2.373849in}{5.555456in}}%
\pgfpathlineto{\pgfqpoint{2.426893in}{5.555456in}}%
\pgfpathlineto{\pgfqpoint{2.479937in}{5.397997in}}%
\pgfpathlineto{\pgfqpoint{2.532981in}{5.253753in}}%
\pgfpathlineto{\pgfqpoint{2.586025in}{5.231942in}}%
\pgfpathlineto{\pgfqpoint{2.639069in}{5.227448in}}%
\pgfpathlineto{\pgfqpoint{2.692113in}{5.274799in}}%
\pgfpathlineto{\pgfqpoint{2.745157in}{5.555456in}}%
\pgfpathlineto{\pgfqpoint{2.851245in}{5.555456in}}%
\pgfpathlineto{\pgfqpoint{2.904288in}{5.357472in}}%
\pgfpathlineto{\pgfqpoint{2.957332in}{5.555456in}}%
\pgfpathlineto{\pgfqpoint{3.010376in}{5.443237in}}%
\pgfpathlineto{\pgfqpoint{3.063420in}{5.555456in}}%
\pgfpathlineto{\pgfqpoint{3.116464in}{5.555456in}}%
\pgfpathlineto{\pgfqpoint{3.169508in}{5.475056in}}%
\pgfpathlineto{\pgfqpoint{3.222552in}{5.505579in}}%
\pgfpathlineto{\pgfqpoint{3.275596in}{5.421720in}}%
\pgfpathlineto{\pgfqpoint{3.328640in}{5.480294in}}%
\pgfpathlineto{\pgfqpoint{3.381684in}{5.555456in}}%
\pgfpathlineto{\pgfqpoint{3.434728in}{5.332680in}}%
\pgfpathlineto{\pgfqpoint{3.487772in}{5.555456in}}%
\pgfpathlineto{\pgfqpoint{3.593860in}{5.555456in}}%
\pgfpathlineto{\pgfqpoint{3.646904in}{5.270980in}}%
\pgfpathlineto{\pgfqpoint{3.699948in}{5.227131in}}%
\pgfpathlineto{\pgfqpoint{3.752991in}{5.555456in}}%
\pgfpathlineto{\pgfqpoint{3.806035in}{5.267382in}}%
\pgfpathlineto{\pgfqpoint{3.859079in}{5.555456in}}%
\pgfpathlineto{\pgfqpoint{3.965167in}{5.555456in}}%
\pgfpathlineto{\pgfqpoint{4.018211in}{5.273356in}}%
\pgfpathlineto{\pgfqpoint{4.071255in}{5.355447in}}%
\pgfpathlineto{\pgfqpoint{4.124299in}{5.483615in}}%
\pgfpathlineto{\pgfqpoint{4.177343in}{5.399336in}}%
\pgfpathlineto{\pgfqpoint{4.230387in}{5.437444in}}%
\pgfpathlineto{\pgfqpoint{4.283431in}{5.453162in}}%
\pgfpathlineto{\pgfqpoint{4.336475in}{5.555456in}}%
\pgfpathlineto{\pgfqpoint{4.601694in}{5.555456in}}%
\pgfpathlineto{\pgfqpoint{4.654738in}{5.454774in}}%
\pgfpathlineto{\pgfqpoint{4.707782in}{5.555456in}}%
\pgfpathlineto{\pgfqpoint{4.760826in}{5.555456in}}%
\pgfpathlineto{\pgfqpoint{4.813870in}{5.335037in}}%
\pgfpathlineto{\pgfqpoint{4.866914in}{5.555456in}}%
\pgfpathlineto{\pgfqpoint{4.919958in}{5.555456in}}%
\pgfpathlineto{\pgfqpoint{4.973002in}{5.216034in}}%
\pgfpathlineto{\pgfqpoint{5.026046in}{5.555456in}}%
\pgfpathlineto{\pgfqpoint{5.079090in}{5.207621in}}%
\pgfpathlineto{\pgfqpoint{5.132134in}{5.555456in}}%
\pgfpathlineto{\pgfqpoint{5.238222in}{5.555456in}}%
\pgfpathlineto{\pgfqpoint{5.291266in}{5.335561in}}%
\pgfpathlineto{\pgfqpoint{5.344309in}{5.555456in}}%
\pgfpathlineto{\pgfqpoint{5.450397in}{5.555456in}}%
\pgfpathlineto{\pgfqpoint{5.503441in}{5.476742in}}%
\pgfpathlineto{\pgfqpoint{5.556485in}{5.555456in}}%
\pgfpathlineto{\pgfqpoint{5.662573in}{5.555456in}}%
\pgfpathlineto{\pgfqpoint{5.715617in}{5.482534in}}%
\pgfpathlineto{\pgfqpoint{5.768661in}{5.555456in}}%
\pgfpathlineto{\pgfqpoint{5.927793in}{5.555456in}}%
\pgfpathlineto{\pgfqpoint{5.980837in}{5.406177in}}%
\pgfpathlineto{\pgfqpoint{6.033881in}{5.555456in}}%
\pgfpathlineto{\pgfqpoint{6.086925in}{5.335040in}}%
\pgfpathlineto{\pgfqpoint{6.139969in}{5.316566in}}%
\pgfpathlineto{\pgfqpoint{6.193012in}{5.555456in}}%
\pgfpathlineto{\pgfqpoint{6.246056in}{5.555456in}}%
\pgfpathlineto{\pgfqpoint{6.299100in}{5.237069in}}%
\pgfpathlineto{\pgfqpoint{6.352144in}{5.555456in}}%
\pgfpathlineto{\pgfqpoint{6.458232in}{5.555456in}}%
\pgfpathlineto{\pgfqpoint{6.511276in}{5.306663in}}%
\pgfpathlineto{\pgfqpoint{6.564320in}{5.303012in}}%
\pgfpathlineto{\pgfqpoint{6.617364in}{5.369970in}}%
\pgfpathlineto{\pgfqpoint{6.670408in}{5.399912in}}%
\pgfpathlineto{\pgfqpoint{6.723452in}{5.445993in}}%
\pgfpathlineto{\pgfqpoint{6.776496in}{5.555456in}}%
\pgfpathlineto{\pgfqpoint{6.829540in}{5.457960in}}%
\pgfpathlineto{\pgfqpoint{6.882584in}{5.555456in}}%
\pgfpathlineto{\pgfqpoint{6.935628in}{5.451613in}}%
\pgfpathlineto{\pgfqpoint{6.988671in}{5.555456in}}%
\pgfpathlineto{\pgfqpoint{7.041715in}{5.470876in}}%
\pgfpathlineto{\pgfqpoint{7.094759in}{5.493927in}}%
\pgfpathlineto{\pgfqpoint{7.147803in}{5.475721in}}%
\pgfpathlineto{\pgfqpoint{7.200847in}{5.555456in}}%
\pgfpathlineto{\pgfqpoint{7.253891in}{5.555456in}}%
\pgfpathlineto{\pgfqpoint{7.306935in}{5.455611in}}%
\pgfpathlineto{\pgfqpoint{7.359979in}{5.555456in}}%
\pgfpathlineto{\pgfqpoint{7.678243in}{5.555456in}}%
\pgfpathlineto{\pgfqpoint{7.731287in}{5.268845in}}%
\pgfpathlineto{\pgfqpoint{7.784330in}{5.304877in}}%
\pgfpathlineto{\pgfqpoint{7.837374in}{5.313128in}}%
\pgfpathlineto{\pgfqpoint{7.890418in}{5.354431in}}%
\pgfpathlineto{\pgfqpoint{7.943462in}{5.352176in}}%
\pgfpathlineto{\pgfqpoint{7.996506in}{5.430879in}}%
\pgfpathlineto{\pgfqpoint{8.049550in}{5.555456in}}%
\pgfpathlineto{\pgfqpoint{8.102594in}{5.555456in}}%
\pgfpathlineto{\pgfqpoint{8.155638in}{5.497516in}}%
\pgfpathlineto{\pgfqpoint{8.208682in}{5.491656in}}%
\pgfpathlineto{\pgfqpoint{8.261726in}{5.555456in}}%
\pgfpathlineto{\pgfqpoint{8.314770in}{5.555456in}}%
\pgfpathlineto{\pgfqpoint{8.367814in}{5.474073in}}%
\pgfpathlineto{\pgfqpoint{8.420858in}{5.555456in}}%
\pgfpathlineto{\pgfqpoint{8.473902in}{5.414851in}}%
\pgfpathlineto{\pgfqpoint{8.526946in}{5.386710in}}%
\pgfpathlineto{\pgfqpoint{8.579990in}{5.555456in}}%
\pgfpathlineto{\pgfqpoint{8.898253in}{5.555456in}}%
\pgfpathlineto{\pgfqpoint{8.951297in}{5.237916in}}%
\pgfpathlineto{\pgfqpoint{9.004341in}{5.555456in}}%
\pgfpathlineto{\pgfqpoint{9.057385in}{5.322277in}}%
\pgfpathlineto{\pgfqpoint{9.110429in}{5.357370in}}%
\pgfpathlineto{\pgfqpoint{9.163473in}{5.555456in}}%
\pgfpathlineto{\pgfqpoint{9.375649in}{5.555456in}}%
\pgfpathlineto{\pgfqpoint{9.428692in}{5.513734in}}%
\pgfpathlineto{\pgfqpoint{9.481736in}{5.555456in}}%
\pgfpathlineto{\pgfqpoint{9.800000in}{5.555456in}}%
\pgfpathlineto{\pgfqpoint{9.800000in}{5.555456in}}%
\pgfusepath{stroke}%
\end{pgfscope}%
\begin{pgfscope}%
\pgfpathrectangle{\pgfqpoint{0.941663in}{4.334375in}}{\pgfqpoint{8.858337in}{3.465625in}}%
\pgfusepath{clip}%
\pgfsetbuttcap%
\pgfsetroundjoin%
\definecolor{currentfill}{rgb}{1.000000,0.647059,0.000000}%
\pgfsetfillcolor{currentfill}%
\pgfsetlinewidth{1.003750pt}%
\definecolor{currentstroke}{rgb}{1.000000,0.647059,0.000000}%
\pgfsetstrokecolor{currentstroke}%
\pgfsetdash{}{0pt}%
\pgfsys@defobject{currentmarker}{\pgfqpoint{0.941663in}{5.207621in}}{\pgfqpoint{9.800000in}{5.555456in}}{%
\pgfpathmoveto{\pgfqpoint{0.941663in}{5.302080in}}%
\pgfpathlineto{\pgfqpoint{0.941663in}{5.555456in}}%
\pgfpathlineto{\pgfqpoint{0.994707in}{5.555456in}}%
\pgfpathlineto{\pgfqpoint{1.047751in}{5.555456in}}%
\pgfpathlineto{\pgfqpoint{1.100795in}{5.555456in}}%
\pgfpathlineto{\pgfqpoint{1.153839in}{5.555456in}}%
\pgfpathlineto{\pgfqpoint{1.206883in}{5.555456in}}%
\pgfpathlineto{\pgfqpoint{1.259927in}{5.555456in}}%
\pgfpathlineto{\pgfqpoint{1.312970in}{5.555456in}}%
\pgfpathlineto{\pgfqpoint{1.366014in}{5.555456in}}%
\pgfpathlineto{\pgfqpoint{1.419058in}{5.555456in}}%
\pgfpathlineto{\pgfqpoint{1.472102in}{5.555456in}}%
\pgfpathlineto{\pgfqpoint{1.525146in}{5.555456in}}%
\pgfpathlineto{\pgfqpoint{1.578190in}{5.555456in}}%
\pgfpathlineto{\pgfqpoint{1.631234in}{5.555456in}}%
\pgfpathlineto{\pgfqpoint{1.684278in}{5.555456in}}%
\pgfpathlineto{\pgfqpoint{1.737322in}{5.555456in}}%
\pgfpathlineto{\pgfqpoint{1.790366in}{5.555456in}}%
\pgfpathlineto{\pgfqpoint{1.843410in}{5.555456in}}%
\pgfpathlineto{\pgfqpoint{1.896454in}{5.555456in}}%
\pgfpathlineto{\pgfqpoint{1.949498in}{5.555456in}}%
\pgfpathlineto{\pgfqpoint{2.002542in}{5.555456in}}%
\pgfpathlineto{\pgfqpoint{2.055586in}{5.555456in}}%
\pgfpathlineto{\pgfqpoint{2.108629in}{5.555456in}}%
\pgfpathlineto{\pgfqpoint{2.161673in}{5.555456in}}%
\pgfpathlineto{\pgfqpoint{2.214717in}{5.555456in}}%
\pgfpathlineto{\pgfqpoint{2.267761in}{5.555456in}}%
\pgfpathlineto{\pgfqpoint{2.320805in}{5.555456in}}%
\pgfpathlineto{\pgfqpoint{2.373849in}{5.555456in}}%
\pgfpathlineto{\pgfqpoint{2.426893in}{5.555456in}}%
\pgfpathlineto{\pgfqpoint{2.479937in}{5.555456in}}%
\pgfpathlineto{\pgfqpoint{2.532981in}{5.555456in}}%
\pgfpathlineto{\pgfqpoint{2.586025in}{5.555456in}}%
\pgfpathlineto{\pgfqpoint{2.639069in}{5.555456in}}%
\pgfpathlineto{\pgfqpoint{2.692113in}{5.555456in}}%
\pgfpathlineto{\pgfqpoint{2.745157in}{5.555456in}}%
\pgfpathlineto{\pgfqpoint{2.798201in}{5.555456in}}%
\pgfpathlineto{\pgfqpoint{2.851245in}{5.555456in}}%
\pgfpathlineto{\pgfqpoint{2.904288in}{5.555456in}}%
\pgfpathlineto{\pgfqpoint{2.957332in}{5.555456in}}%
\pgfpathlineto{\pgfqpoint{3.010376in}{5.555456in}}%
\pgfpathlineto{\pgfqpoint{3.063420in}{5.555456in}}%
\pgfpathlineto{\pgfqpoint{3.116464in}{5.555456in}}%
\pgfpathlineto{\pgfqpoint{3.169508in}{5.555456in}}%
\pgfpathlineto{\pgfqpoint{3.222552in}{5.555456in}}%
\pgfpathlineto{\pgfqpoint{3.275596in}{5.555456in}}%
\pgfpathlineto{\pgfqpoint{3.328640in}{5.555456in}}%
\pgfpathlineto{\pgfqpoint{3.381684in}{5.555456in}}%
\pgfpathlineto{\pgfqpoint{3.434728in}{5.555456in}}%
\pgfpathlineto{\pgfqpoint{3.487772in}{5.555456in}}%
\pgfpathlineto{\pgfqpoint{3.540816in}{5.555456in}}%
\pgfpathlineto{\pgfqpoint{3.593860in}{5.555456in}}%
\pgfpathlineto{\pgfqpoint{3.646904in}{5.555456in}}%
\pgfpathlineto{\pgfqpoint{3.699948in}{5.555456in}}%
\pgfpathlineto{\pgfqpoint{3.752991in}{5.555456in}}%
\pgfpathlineto{\pgfqpoint{3.806035in}{5.555456in}}%
\pgfpathlineto{\pgfqpoint{3.859079in}{5.555456in}}%
\pgfpathlineto{\pgfqpoint{3.912123in}{5.555456in}}%
\pgfpathlineto{\pgfqpoint{3.965167in}{5.555456in}}%
\pgfpathlineto{\pgfqpoint{4.018211in}{5.555456in}}%
\pgfpathlineto{\pgfqpoint{4.071255in}{5.555456in}}%
\pgfpathlineto{\pgfqpoint{4.124299in}{5.555456in}}%
\pgfpathlineto{\pgfqpoint{4.177343in}{5.555456in}}%
\pgfpathlineto{\pgfqpoint{4.230387in}{5.555456in}}%
\pgfpathlineto{\pgfqpoint{4.283431in}{5.555456in}}%
\pgfpathlineto{\pgfqpoint{4.336475in}{5.555456in}}%
\pgfpathlineto{\pgfqpoint{4.389519in}{5.555456in}}%
\pgfpathlineto{\pgfqpoint{4.442563in}{5.555456in}}%
\pgfpathlineto{\pgfqpoint{4.495607in}{5.555456in}}%
\pgfpathlineto{\pgfqpoint{4.548650in}{5.555456in}}%
\pgfpathlineto{\pgfqpoint{4.601694in}{5.555456in}}%
\pgfpathlineto{\pgfqpoint{4.654738in}{5.555456in}}%
\pgfpathlineto{\pgfqpoint{4.707782in}{5.555456in}}%
\pgfpathlineto{\pgfqpoint{4.760826in}{5.555456in}}%
\pgfpathlineto{\pgfqpoint{4.813870in}{5.555456in}}%
\pgfpathlineto{\pgfqpoint{4.866914in}{5.555456in}}%
\pgfpathlineto{\pgfqpoint{4.919958in}{5.555456in}}%
\pgfpathlineto{\pgfqpoint{4.973002in}{5.555456in}}%
\pgfpathlineto{\pgfqpoint{5.026046in}{5.555456in}}%
\pgfpathlineto{\pgfqpoint{5.079090in}{5.555456in}}%
\pgfpathlineto{\pgfqpoint{5.132134in}{5.555456in}}%
\pgfpathlineto{\pgfqpoint{5.185178in}{5.555456in}}%
\pgfpathlineto{\pgfqpoint{5.238222in}{5.555456in}}%
\pgfpathlineto{\pgfqpoint{5.291266in}{5.555456in}}%
\pgfpathlineto{\pgfqpoint{5.344309in}{5.555456in}}%
\pgfpathlineto{\pgfqpoint{5.397353in}{5.555456in}}%
\pgfpathlineto{\pgfqpoint{5.450397in}{5.555456in}}%
\pgfpathlineto{\pgfqpoint{5.503441in}{5.555456in}}%
\pgfpathlineto{\pgfqpoint{5.556485in}{5.555456in}}%
\pgfpathlineto{\pgfqpoint{5.609529in}{5.555456in}}%
\pgfpathlineto{\pgfqpoint{5.662573in}{5.555456in}}%
\pgfpathlineto{\pgfqpoint{5.715617in}{5.555456in}}%
\pgfpathlineto{\pgfqpoint{5.768661in}{5.555456in}}%
\pgfpathlineto{\pgfqpoint{5.821705in}{5.555456in}}%
\pgfpathlineto{\pgfqpoint{5.874749in}{5.555456in}}%
\pgfpathlineto{\pgfqpoint{5.927793in}{5.555456in}}%
\pgfpathlineto{\pgfqpoint{5.980837in}{5.555456in}}%
\pgfpathlineto{\pgfqpoint{6.033881in}{5.555456in}}%
\pgfpathlineto{\pgfqpoint{6.086925in}{5.555456in}}%
\pgfpathlineto{\pgfqpoint{6.139969in}{5.555456in}}%
\pgfpathlineto{\pgfqpoint{6.193012in}{5.555456in}}%
\pgfpathlineto{\pgfqpoint{6.246056in}{5.555456in}}%
\pgfpathlineto{\pgfqpoint{6.299100in}{5.555456in}}%
\pgfpathlineto{\pgfqpoint{6.352144in}{5.555456in}}%
\pgfpathlineto{\pgfqpoint{6.405188in}{5.555456in}}%
\pgfpathlineto{\pgfqpoint{6.458232in}{5.555456in}}%
\pgfpathlineto{\pgfqpoint{6.511276in}{5.555456in}}%
\pgfpathlineto{\pgfqpoint{6.564320in}{5.555456in}}%
\pgfpathlineto{\pgfqpoint{6.617364in}{5.555456in}}%
\pgfpathlineto{\pgfqpoint{6.670408in}{5.555456in}}%
\pgfpathlineto{\pgfqpoint{6.723452in}{5.555456in}}%
\pgfpathlineto{\pgfqpoint{6.776496in}{5.555456in}}%
\pgfpathlineto{\pgfqpoint{6.829540in}{5.555456in}}%
\pgfpathlineto{\pgfqpoint{6.882584in}{5.555456in}}%
\pgfpathlineto{\pgfqpoint{6.935628in}{5.555456in}}%
\pgfpathlineto{\pgfqpoint{6.988671in}{5.555456in}}%
\pgfpathlineto{\pgfqpoint{7.041715in}{5.555456in}}%
\pgfpathlineto{\pgfqpoint{7.094759in}{5.555456in}}%
\pgfpathlineto{\pgfqpoint{7.147803in}{5.555456in}}%
\pgfpathlineto{\pgfqpoint{7.200847in}{5.555456in}}%
\pgfpathlineto{\pgfqpoint{7.253891in}{5.555456in}}%
\pgfpathlineto{\pgfqpoint{7.306935in}{5.555456in}}%
\pgfpathlineto{\pgfqpoint{7.359979in}{5.555456in}}%
\pgfpathlineto{\pgfqpoint{7.413023in}{5.555456in}}%
\pgfpathlineto{\pgfqpoint{7.466067in}{5.555456in}}%
\pgfpathlineto{\pgfqpoint{7.519111in}{5.555456in}}%
\pgfpathlineto{\pgfqpoint{7.572155in}{5.555456in}}%
\pgfpathlineto{\pgfqpoint{7.625199in}{5.555456in}}%
\pgfpathlineto{\pgfqpoint{7.678243in}{5.555456in}}%
\pgfpathlineto{\pgfqpoint{7.731287in}{5.555456in}}%
\pgfpathlineto{\pgfqpoint{7.784330in}{5.555456in}}%
\pgfpathlineto{\pgfqpoint{7.837374in}{5.555456in}}%
\pgfpathlineto{\pgfqpoint{7.890418in}{5.555456in}}%
\pgfpathlineto{\pgfqpoint{7.943462in}{5.555456in}}%
\pgfpathlineto{\pgfqpoint{7.996506in}{5.555456in}}%
\pgfpathlineto{\pgfqpoint{8.049550in}{5.555456in}}%
\pgfpathlineto{\pgfqpoint{8.102594in}{5.555456in}}%
\pgfpathlineto{\pgfqpoint{8.155638in}{5.555456in}}%
\pgfpathlineto{\pgfqpoint{8.208682in}{5.555456in}}%
\pgfpathlineto{\pgfqpoint{8.261726in}{5.555456in}}%
\pgfpathlineto{\pgfqpoint{8.314770in}{5.555456in}}%
\pgfpathlineto{\pgfqpoint{8.367814in}{5.555456in}}%
\pgfpathlineto{\pgfqpoint{8.420858in}{5.555456in}}%
\pgfpathlineto{\pgfqpoint{8.473902in}{5.555456in}}%
\pgfpathlineto{\pgfqpoint{8.526946in}{5.555456in}}%
\pgfpathlineto{\pgfqpoint{8.579990in}{5.555456in}}%
\pgfpathlineto{\pgfqpoint{8.633033in}{5.555456in}}%
\pgfpathlineto{\pgfqpoint{8.686077in}{5.555456in}}%
\pgfpathlineto{\pgfqpoint{8.739121in}{5.555456in}}%
\pgfpathlineto{\pgfqpoint{8.792165in}{5.555456in}}%
\pgfpathlineto{\pgfqpoint{8.845209in}{5.555456in}}%
\pgfpathlineto{\pgfqpoint{8.898253in}{5.555456in}}%
\pgfpathlineto{\pgfqpoint{8.951297in}{5.555456in}}%
\pgfpathlineto{\pgfqpoint{9.004341in}{5.555456in}}%
\pgfpathlineto{\pgfqpoint{9.057385in}{5.555456in}}%
\pgfpathlineto{\pgfqpoint{9.110429in}{5.555456in}}%
\pgfpathlineto{\pgfqpoint{9.163473in}{5.555456in}}%
\pgfpathlineto{\pgfqpoint{9.216517in}{5.555456in}}%
\pgfpathlineto{\pgfqpoint{9.269561in}{5.555456in}}%
\pgfpathlineto{\pgfqpoint{9.322605in}{5.555456in}}%
\pgfpathlineto{\pgfqpoint{9.375649in}{5.555456in}}%
\pgfpathlineto{\pgfqpoint{9.428692in}{5.555456in}}%
\pgfpathlineto{\pgfqpoint{9.481736in}{5.555456in}}%
\pgfpathlineto{\pgfqpoint{9.534780in}{5.555456in}}%
\pgfpathlineto{\pgfqpoint{9.587824in}{5.555456in}}%
\pgfpathlineto{\pgfqpoint{9.640868in}{5.555456in}}%
\pgfpathlineto{\pgfqpoint{9.693912in}{5.555456in}}%
\pgfpathlineto{\pgfqpoint{9.746956in}{5.555456in}}%
\pgfpathlineto{\pgfqpoint{9.800000in}{5.555456in}}%
\pgfpathlineto{\pgfqpoint{9.800000in}{5.555456in}}%
\pgfpathlineto{\pgfqpoint{9.800000in}{5.555456in}}%
\pgfpathlineto{\pgfqpoint{9.746956in}{5.555456in}}%
\pgfpathlineto{\pgfqpoint{9.693912in}{5.555456in}}%
\pgfpathlineto{\pgfqpoint{9.640868in}{5.555456in}}%
\pgfpathlineto{\pgfqpoint{9.587824in}{5.555456in}}%
\pgfpathlineto{\pgfqpoint{9.534780in}{5.555456in}}%
\pgfpathlineto{\pgfqpoint{9.481736in}{5.555456in}}%
\pgfpathlineto{\pgfqpoint{9.428692in}{5.513734in}}%
\pgfpathlineto{\pgfqpoint{9.375649in}{5.555456in}}%
\pgfpathlineto{\pgfqpoint{9.322605in}{5.555456in}}%
\pgfpathlineto{\pgfqpoint{9.269561in}{5.555456in}}%
\pgfpathlineto{\pgfqpoint{9.216517in}{5.555456in}}%
\pgfpathlineto{\pgfqpoint{9.163473in}{5.555456in}}%
\pgfpathlineto{\pgfqpoint{9.110429in}{5.357370in}}%
\pgfpathlineto{\pgfqpoint{9.057385in}{5.322277in}}%
\pgfpathlineto{\pgfqpoint{9.004341in}{5.555456in}}%
\pgfpathlineto{\pgfqpoint{8.951297in}{5.237916in}}%
\pgfpathlineto{\pgfqpoint{8.898253in}{5.555456in}}%
\pgfpathlineto{\pgfqpoint{8.845209in}{5.555456in}}%
\pgfpathlineto{\pgfqpoint{8.792165in}{5.555456in}}%
\pgfpathlineto{\pgfqpoint{8.739121in}{5.555456in}}%
\pgfpathlineto{\pgfqpoint{8.686077in}{5.555456in}}%
\pgfpathlineto{\pgfqpoint{8.633033in}{5.555456in}}%
\pgfpathlineto{\pgfqpoint{8.579990in}{5.555456in}}%
\pgfpathlineto{\pgfqpoint{8.526946in}{5.386710in}}%
\pgfpathlineto{\pgfqpoint{8.473902in}{5.414851in}}%
\pgfpathlineto{\pgfqpoint{8.420858in}{5.555456in}}%
\pgfpathlineto{\pgfqpoint{8.367814in}{5.474073in}}%
\pgfpathlineto{\pgfqpoint{8.314770in}{5.555456in}}%
\pgfpathlineto{\pgfqpoint{8.261726in}{5.555456in}}%
\pgfpathlineto{\pgfqpoint{8.208682in}{5.491656in}}%
\pgfpathlineto{\pgfqpoint{8.155638in}{5.497516in}}%
\pgfpathlineto{\pgfqpoint{8.102594in}{5.555456in}}%
\pgfpathlineto{\pgfqpoint{8.049550in}{5.555456in}}%
\pgfpathlineto{\pgfqpoint{7.996506in}{5.430879in}}%
\pgfpathlineto{\pgfqpoint{7.943462in}{5.352176in}}%
\pgfpathlineto{\pgfqpoint{7.890418in}{5.354431in}}%
\pgfpathlineto{\pgfqpoint{7.837374in}{5.313128in}}%
\pgfpathlineto{\pgfqpoint{7.784330in}{5.304877in}}%
\pgfpathlineto{\pgfqpoint{7.731287in}{5.268845in}}%
\pgfpathlineto{\pgfqpoint{7.678243in}{5.555456in}}%
\pgfpathlineto{\pgfqpoint{7.625199in}{5.555456in}}%
\pgfpathlineto{\pgfqpoint{7.572155in}{5.555456in}}%
\pgfpathlineto{\pgfqpoint{7.519111in}{5.555456in}}%
\pgfpathlineto{\pgfqpoint{7.466067in}{5.555456in}}%
\pgfpathlineto{\pgfqpoint{7.413023in}{5.555456in}}%
\pgfpathlineto{\pgfqpoint{7.359979in}{5.555456in}}%
\pgfpathlineto{\pgfqpoint{7.306935in}{5.455611in}}%
\pgfpathlineto{\pgfqpoint{7.253891in}{5.555456in}}%
\pgfpathlineto{\pgfqpoint{7.200847in}{5.555456in}}%
\pgfpathlineto{\pgfqpoint{7.147803in}{5.475721in}}%
\pgfpathlineto{\pgfqpoint{7.094759in}{5.493927in}}%
\pgfpathlineto{\pgfqpoint{7.041715in}{5.470876in}}%
\pgfpathlineto{\pgfqpoint{6.988671in}{5.555456in}}%
\pgfpathlineto{\pgfqpoint{6.935628in}{5.451613in}}%
\pgfpathlineto{\pgfqpoint{6.882584in}{5.555456in}}%
\pgfpathlineto{\pgfqpoint{6.829540in}{5.457960in}}%
\pgfpathlineto{\pgfqpoint{6.776496in}{5.555456in}}%
\pgfpathlineto{\pgfqpoint{6.723452in}{5.445993in}}%
\pgfpathlineto{\pgfqpoint{6.670408in}{5.399912in}}%
\pgfpathlineto{\pgfqpoint{6.617364in}{5.369970in}}%
\pgfpathlineto{\pgfqpoint{6.564320in}{5.303012in}}%
\pgfpathlineto{\pgfqpoint{6.511276in}{5.306663in}}%
\pgfpathlineto{\pgfqpoint{6.458232in}{5.555456in}}%
\pgfpathlineto{\pgfqpoint{6.405188in}{5.555456in}}%
\pgfpathlineto{\pgfqpoint{6.352144in}{5.555456in}}%
\pgfpathlineto{\pgfqpoint{6.299100in}{5.237069in}}%
\pgfpathlineto{\pgfqpoint{6.246056in}{5.555456in}}%
\pgfpathlineto{\pgfqpoint{6.193012in}{5.555456in}}%
\pgfpathlineto{\pgfqpoint{6.139969in}{5.316566in}}%
\pgfpathlineto{\pgfqpoint{6.086925in}{5.335040in}}%
\pgfpathlineto{\pgfqpoint{6.033881in}{5.555456in}}%
\pgfpathlineto{\pgfqpoint{5.980837in}{5.406177in}}%
\pgfpathlineto{\pgfqpoint{5.927793in}{5.555456in}}%
\pgfpathlineto{\pgfqpoint{5.874749in}{5.555456in}}%
\pgfpathlineto{\pgfqpoint{5.821705in}{5.555456in}}%
\pgfpathlineto{\pgfqpoint{5.768661in}{5.555456in}}%
\pgfpathlineto{\pgfqpoint{5.715617in}{5.482534in}}%
\pgfpathlineto{\pgfqpoint{5.662573in}{5.555456in}}%
\pgfpathlineto{\pgfqpoint{5.609529in}{5.555456in}}%
\pgfpathlineto{\pgfqpoint{5.556485in}{5.555456in}}%
\pgfpathlineto{\pgfqpoint{5.503441in}{5.476742in}}%
\pgfpathlineto{\pgfqpoint{5.450397in}{5.555456in}}%
\pgfpathlineto{\pgfqpoint{5.397353in}{5.555456in}}%
\pgfpathlineto{\pgfqpoint{5.344309in}{5.555456in}}%
\pgfpathlineto{\pgfqpoint{5.291266in}{5.335561in}}%
\pgfpathlineto{\pgfqpoint{5.238222in}{5.555456in}}%
\pgfpathlineto{\pgfqpoint{5.185178in}{5.555456in}}%
\pgfpathlineto{\pgfqpoint{5.132134in}{5.555456in}}%
\pgfpathlineto{\pgfqpoint{5.079090in}{5.207621in}}%
\pgfpathlineto{\pgfqpoint{5.026046in}{5.555456in}}%
\pgfpathlineto{\pgfqpoint{4.973002in}{5.216034in}}%
\pgfpathlineto{\pgfqpoint{4.919958in}{5.555456in}}%
\pgfpathlineto{\pgfqpoint{4.866914in}{5.555456in}}%
\pgfpathlineto{\pgfqpoint{4.813870in}{5.335037in}}%
\pgfpathlineto{\pgfqpoint{4.760826in}{5.555456in}}%
\pgfpathlineto{\pgfqpoint{4.707782in}{5.555456in}}%
\pgfpathlineto{\pgfqpoint{4.654738in}{5.454774in}}%
\pgfpathlineto{\pgfqpoint{4.601694in}{5.555456in}}%
\pgfpathlineto{\pgfqpoint{4.548650in}{5.555456in}}%
\pgfpathlineto{\pgfqpoint{4.495607in}{5.555456in}}%
\pgfpathlineto{\pgfqpoint{4.442563in}{5.555456in}}%
\pgfpathlineto{\pgfqpoint{4.389519in}{5.555456in}}%
\pgfpathlineto{\pgfqpoint{4.336475in}{5.555456in}}%
\pgfpathlineto{\pgfqpoint{4.283431in}{5.453162in}}%
\pgfpathlineto{\pgfqpoint{4.230387in}{5.437444in}}%
\pgfpathlineto{\pgfqpoint{4.177343in}{5.399336in}}%
\pgfpathlineto{\pgfqpoint{4.124299in}{5.483615in}}%
\pgfpathlineto{\pgfqpoint{4.071255in}{5.355447in}}%
\pgfpathlineto{\pgfqpoint{4.018211in}{5.273356in}}%
\pgfpathlineto{\pgfqpoint{3.965167in}{5.555456in}}%
\pgfpathlineto{\pgfqpoint{3.912123in}{5.555456in}}%
\pgfpathlineto{\pgfqpoint{3.859079in}{5.555456in}}%
\pgfpathlineto{\pgfqpoint{3.806035in}{5.267382in}}%
\pgfpathlineto{\pgfqpoint{3.752991in}{5.555456in}}%
\pgfpathlineto{\pgfqpoint{3.699948in}{5.227131in}}%
\pgfpathlineto{\pgfqpoint{3.646904in}{5.270980in}}%
\pgfpathlineto{\pgfqpoint{3.593860in}{5.555456in}}%
\pgfpathlineto{\pgfqpoint{3.540816in}{5.555456in}}%
\pgfpathlineto{\pgfqpoint{3.487772in}{5.555456in}}%
\pgfpathlineto{\pgfqpoint{3.434728in}{5.332680in}}%
\pgfpathlineto{\pgfqpoint{3.381684in}{5.555456in}}%
\pgfpathlineto{\pgfqpoint{3.328640in}{5.480294in}}%
\pgfpathlineto{\pgfqpoint{3.275596in}{5.421720in}}%
\pgfpathlineto{\pgfqpoint{3.222552in}{5.505579in}}%
\pgfpathlineto{\pgfqpoint{3.169508in}{5.475056in}}%
\pgfpathlineto{\pgfqpoint{3.116464in}{5.555456in}}%
\pgfpathlineto{\pgfqpoint{3.063420in}{5.555456in}}%
\pgfpathlineto{\pgfqpoint{3.010376in}{5.443237in}}%
\pgfpathlineto{\pgfqpoint{2.957332in}{5.555456in}}%
\pgfpathlineto{\pgfqpoint{2.904288in}{5.357472in}}%
\pgfpathlineto{\pgfqpoint{2.851245in}{5.555456in}}%
\pgfpathlineto{\pgfqpoint{2.798201in}{5.555456in}}%
\pgfpathlineto{\pgfqpoint{2.745157in}{5.555456in}}%
\pgfpathlineto{\pgfqpoint{2.692113in}{5.274799in}}%
\pgfpathlineto{\pgfqpoint{2.639069in}{5.227448in}}%
\pgfpathlineto{\pgfqpoint{2.586025in}{5.231942in}}%
\pgfpathlineto{\pgfqpoint{2.532981in}{5.253753in}}%
\pgfpathlineto{\pgfqpoint{2.479937in}{5.397997in}}%
\pgfpathlineto{\pgfqpoint{2.426893in}{5.555456in}}%
\pgfpathlineto{\pgfqpoint{2.373849in}{5.555456in}}%
\pgfpathlineto{\pgfqpoint{2.320805in}{5.278077in}}%
\pgfpathlineto{\pgfqpoint{2.267761in}{5.542464in}}%
\pgfpathlineto{\pgfqpoint{2.214717in}{5.555456in}}%
\pgfpathlineto{\pgfqpoint{2.161673in}{5.347486in}}%
\pgfpathlineto{\pgfqpoint{2.108629in}{5.555456in}}%
\pgfpathlineto{\pgfqpoint{2.055586in}{5.457838in}}%
\pgfpathlineto{\pgfqpoint{2.002542in}{5.555456in}}%
\pgfpathlineto{\pgfqpoint{1.949498in}{5.447032in}}%
\pgfpathlineto{\pgfqpoint{1.896454in}{5.555456in}}%
\pgfpathlineto{\pgfqpoint{1.843410in}{5.555456in}}%
\pgfpathlineto{\pgfqpoint{1.790366in}{5.473721in}}%
\pgfpathlineto{\pgfqpoint{1.737322in}{5.491300in}}%
\pgfpathlineto{\pgfqpoint{1.684278in}{5.555456in}}%
\pgfpathlineto{\pgfqpoint{1.631234in}{5.555456in}}%
\pgfpathlineto{\pgfqpoint{1.578190in}{5.343668in}}%
\pgfpathlineto{\pgfqpoint{1.525146in}{5.349185in}}%
\pgfpathlineto{\pgfqpoint{1.472102in}{5.268823in}}%
\pgfpathlineto{\pgfqpoint{1.419058in}{5.263460in}}%
\pgfpathlineto{\pgfqpoint{1.366014in}{5.207621in}}%
\pgfpathlineto{\pgfqpoint{1.312970in}{5.555456in}}%
\pgfpathlineto{\pgfqpoint{1.259927in}{5.555456in}}%
\pgfpathlineto{\pgfqpoint{1.206883in}{5.555456in}}%
\pgfpathlineto{\pgfqpoint{1.153839in}{5.253587in}}%
\pgfpathlineto{\pgfqpoint{1.100795in}{5.267250in}}%
\pgfpathlineto{\pgfqpoint{1.047751in}{5.298031in}}%
\pgfpathlineto{\pgfqpoint{0.994707in}{5.555456in}}%
\pgfpathlineto{\pgfqpoint{0.941663in}{5.302080in}}%
\pgfpathlineto{\pgfqpoint{0.941663in}{5.302080in}}%
\pgfpathclose%
\pgfusepath{stroke,fill}%
}%
\begin{pgfscope}%
\pgfsys@transformshift{0.000000in}{0.000000in}%
\pgfsys@useobject{currentmarker}{}%
\end{pgfscope}%
\end{pgfscope}%
\begin{pgfscope}%
\pgfpathrectangle{\pgfqpoint{0.941663in}{4.334375in}}{\pgfqpoint{8.858337in}{3.465625in}}%
\pgfusepath{clip}%
\pgfsetrectcap%
\pgfsetroundjoin%
\pgfsetlinewidth{1.505625pt}%
\definecolor{currentstroke}{rgb}{0.501961,0.501961,0.501961}%
\pgfsetstrokecolor{currentstroke}%
\pgfsetdash{}{0pt}%
\pgfpathmoveto{\pgfqpoint{0.941663in}{4.606409in}}%
\pgfpathlineto{\pgfqpoint{0.994707in}{4.859785in}}%
\pgfpathlineto{\pgfqpoint{1.047751in}{4.602360in}}%
\pgfpathlineto{\pgfqpoint{1.100795in}{4.571579in}}%
\pgfpathlineto{\pgfqpoint{1.153839in}{4.557915in}}%
\pgfpathlineto{\pgfqpoint{1.206883in}{5.127392in}}%
\pgfpathlineto{\pgfqpoint{1.259927in}{5.369914in}}%
\pgfpathlineto{\pgfqpoint{1.312970in}{5.172617in}}%
\pgfpathlineto{\pgfqpoint{1.366014in}{4.493572in}}%
\pgfpathlineto{\pgfqpoint{1.419058in}{4.567788in}}%
\pgfpathlineto{\pgfqpoint{1.472102in}{4.573151in}}%
\pgfpathlineto{\pgfqpoint{1.525146in}{4.653514in}}%
\pgfpathlineto{\pgfqpoint{1.578190in}{4.630644in}}%
\pgfpathlineto{\pgfqpoint{1.631234in}{5.423055in}}%
\pgfpathlineto{\pgfqpoint{1.684278in}{4.875476in}}%
\pgfpathlineto{\pgfqpoint{1.737322in}{4.795628in}}%
\pgfpathlineto{\pgfqpoint{1.790366in}{4.778049in}}%
\pgfpathlineto{\pgfqpoint{1.843410in}{5.555456in}}%
\pgfpathlineto{\pgfqpoint{1.896454in}{5.555456in}}%
\pgfpathlineto{\pgfqpoint{1.949498in}{4.751360in}}%
\pgfpathlineto{\pgfqpoint{2.002542in}{5.555456in}}%
\pgfpathlineto{\pgfqpoint{2.055586in}{4.762166in}}%
\pgfpathlineto{\pgfqpoint{2.108629in}{5.048281in}}%
\pgfpathlineto{\pgfqpoint{2.161673in}{4.651814in}}%
\pgfpathlineto{\pgfqpoint{2.214717in}{5.555456in}}%
\pgfpathlineto{\pgfqpoint{2.267761in}{4.846792in}}%
\pgfpathlineto{\pgfqpoint{2.320805in}{4.582406in}}%
\pgfpathlineto{\pgfqpoint{2.373849in}{5.523078in}}%
\pgfpathlineto{\pgfqpoint{2.426893in}{5.555456in}}%
\pgfpathlineto{\pgfqpoint{2.479937in}{4.702326in}}%
\pgfpathlineto{\pgfqpoint{2.532981in}{4.558081in}}%
\pgfpathlineto{\pgfqpoint{2.586025in}{4.536270in}}%
\pgfpathlineto{\pgfqpoint{2.639069in}{4.531777in}}%
\pgfpathlineto{\pgfqpoint{2.692113in}{4.501574in}}%
\pgfpathlineto{\pgfqpoint{2.745157in}{4.572498in}}%
\pgfpathlineto{\pgfqpoint{2.798201in}{4.614894in}}%
\pgfpathlineto{\pgfqpoint{2.851245in}{5.097809in}}%
\pgfpathlineto{\pgfqpoint{2.904288in}{4.661801in}}%
\pgfpathlineto{\pgfqpoint{2.957332in}{5.555456in}}%
\pgfpathlineto{\pgfqpoint{3.010376in}{4.747565in}}%
\pgfpathlineto{\pgfqpoint{3.063420in}{5.529745in}}%
\pgfpathlineto{\pgfqpoint{3.116464in}{4.873985in}}%
\pgfpathlineto{\pgfqpoint{3.169508in}{4.779384in}}%
\pgfpathlineto{\pgfqpoint{3.222552in}{4.809907in}}%
\pgfpathlineto{\pgfqpoint{3.275596in}{4.726048in}}%
\pgfpathlineto{\pgfqpoint{3.328640in}{4.784623in}}%
\pgfpathlineto{\pgfqpoint{3.381684in}{4.982873in}}%
\pgfpathlineto{\pgfqpoint{3.434728in}{4.637008in}}%
\pgfpathlineto{\pgfqpoint{3.487772in}{5.331044in}}%
\pgfpathlineto{\pgfqpoint{3.540816in}{5.555456in}}%
\pgfpathlineto{\pgfqpoint{3.593860in}{5.052711in}}%
\pgfpathlineto{\pgfqpoint{3.646904in}{4.575309in}}%
\pgfpathlineto{\pgfqpoint{3.699948in}{4.531459in}}%
\pgfpathlineto{\pgfqpoint{3.752991in}{4.859785in}}%
\pgfpathlineto{\pgfqpoint{3.806035in}{4.571710in}}%
\pgfpathlineto{\pgfqpoint{3.859079in}{4.859785in}}%
\pgfpathlineto{\pgfqpoint{3.912123in}{4.859785in}}%
\pgfpathlineto{\pgfqpoint{3.965167in}{5.299182in}}%
\pgfpathlineto{\pgfqpoint{4.018211in}{4.577684in}}%
\pgfpathlineto{\pgfqpoint{4.071255in}{4.659775in}}%
\pgfpathlineto{\pgfqpoint{4.124299in}{4.787943in}}%
\pgfpathlineto{\pgfqpoint{4.177343in}{4.703664in}}%
\pgfpathlineto{\pgfqpoint{4.230387in}{4.741772in}}%
\pgfpathlineto{\pgfqpoint{4.283431in}{4.757491in}}%
\pgfpathlineto{\pgfqpoint{4.336475in}{5.555456in}}%
\pgfpathlineto{\pgfqpoint{4.389519in}{4.859785in}}%
\pgfpathlineto{\pgfqpoint{4.442563in}{4.896628in}}%
\pgfpathlineto{\pgfqpoint{4.495607in}{5.555456in}}%
\pgfpathlineto{\pgfqpoint{4.548650in}{5.101868in}}%
\pgfpathlineto{\pgfqpoint{4.601694in}{5.555456in}}%
\pgfpathlineto{\pgfqpoint{4.654738in}{4.759102in}}%
\pgfpathlineto{\pgfqpoint{4.707782in}{5.555456in}}%
\pgfpathlineto{\pgfqpoint{4.760826in}{4.951917in}}%
\pgfpathlineto{\pgfqpoint{4.813870in}{4.639365in}}%
\pgfpathlineto{\pgfqpoint{4.866914in}{5.555456in}}%
\pgfpathlineto{\pgfqpoint{4.919958in}{5.555456in}}%
\pgfpathlineto{\pgfqpoint{4.973002in}{4.520363in}}%
\pgfpathlineto{\pgfqpoint{5.026046in}{5.387465in}}%
\pgfpathlineto{\pgfqpoint{5.079090in}{4.491903in}}%
\pgfpathlineto{\pgfqpoint{5.132134in}{5.444876in}}%
\pgfpathlineto{\pgfqpoint{5.185178in}{5.555456in}}%
\pgfpathlineto{\pgfqpoint{5.238222in}{5.555456in}}%
\pgfpathlineto{\pgfqpoint{5.291266in}{4.639889in}}%
\pgfpathlineto{\pgfqpoint{5.344309in}{5.453871in}}%
\pgfpathlineto{\pgfqpoint{5.397353in}{5.555456in}}%
\pgfpathlineto{\pgfqpoint{5.450397in}{5.555456in}}%
\pgfpathlineto{\pgfqpoint{5.503441in}{4.781070in}}%
\pgfpathlineto{\pgfqpoint{5.556485in}{5.135041in}}%
\pgfpathlineto{\pgfqpoint{5.609529in}{5.555456in}}%
\pgfpathlineto{\pgfqpoint{5.662573in}{5.555456in}}%
\pgfpathlineto{\pgfqpoint{5.715617in}{4.786862in}}%
\pgfpathlineto{\pgfqpoint{5.768661in}{5.555456in}}%
\pgfpathlineto{\pgfqpoint{5.927793in}{5.555456in}}%
\pgfpathlineto{\pgfqpoint{5.980837in}{4.710506in}}%
\pgfpathlineto{\pgfqpoint{6.033881in}{5.193362in}}%
\pgfpathlineto{\pgfqpoint{6.086925in}{4.639368in}}%
\pgfpathlineto{\pgfqpoint{6.139969in}{4.620894in}}%
\pgfpathlineto{\pgfqpoint{6.193012in}{5.555456in}}%
\pgfpathlineto{\pgfqpoint{6.246056in}{5.555456in}}%
\pgfpathlineto{\pgfqpoint{6.299100in}{4.541397in}}%
\pgfpathlineto{\pgfqpoint{6.352144in}{5.178906in}}%
\pgfpathlineto{\pgfqpoint{6.405188in}{5.555456in}}%
\pgfpathlineto{\pgfqpoint{6.458232in}{5.077516in}}%
\pgfpathlineto{\pgfqpoint{6.511276in}{4.610991in}}%
\pgfpathlineto{\pgfqpoint{6.564320in}{4.607341in}}%
\pgfpathlineto{\pgfqpoint{6.617364in}{4.674298in}}%
\pgfpathlineto{\pgfqpoint{6.670408in}{4.704240in}}%
\pgfpathlineto{\pgfqpoint{6.723452in}{4.750321in}}%
\pgfpathlineto{\pgfqpoint{6.776496in}{4.859785in}}%
\pgfpathlineto{\pgfqpoint{6.829540in}{4.762288in}}%
\pgfpathlineto{\pgfqpoint{6.882584in}{5.555456in}}%
\pgfpathlineto{\pgfqpoint{6.935628in}{4.755942in}}%
\pgfpathlineto{\pgfqpoint{6.988671in}{5.000909in}}%
\pgfpathlineto{\pgfqpoint{7.041715in}{4.775204in}}%
\pgfpathlineto{\pgfqpoint{7.094759in}{4.798255in}}%
\pgfpathlineto{\pgfqpoint{7.147803in}{4.780049in}}%
\pgfpathlineto{\pgfqpoint{7.200847in}{5.555456in}}%
\pgfpathlineto{\pgfqpoint{7.253891in}{4.923953in}}%
\pgfpathlineto{\pgfqpoint{7.306935in}{4.759939in}}%
\pgfpathlineto{\pgfqpoint{7.359979in}{5.555456in}}%
\pgfpathlineto{\pgfqpoint{7.413023in}{5.555456in}}%
\pgfpathlineto{\pgfqpoint{7.466067in}{5.147602in}}%
\pgfpathlineto{\pgfqpoint{7.519111in}{5.074761in}}%
\pgfpathlineto{\pgfqpoint{7.572155in}{5.555456in}}%
\pgfpathlineto{\pgfqpoint{7.625199in}{5.555456in}}%
\pgfpathlineto{\pgfqpoint{7.678243in}{5.513983in}}%
\pgfpathlineto{\pgfqpoint{7.731287in}{4.573173in}}%
\pgfpathlineto{\pgfqpoint{7.784330in}{4.609205in}}%
\pgfpathlineto{\pgfqpoint{7.837374in}{4.617456in}}%
\pgfpathlineto{\pgfqpoint{7.890418in}{4.658759in}}%
\pgfpathlineto{\pgfqpoint{7.943462in}{4.656505in}}%
\pgfpathlineto{\pgfqpoint{7.996506in}{4.735208in}}%
\pgfpathlineto{\pgfqpoint{8.049550in}{5.555456in}}%
\pgfpathlineto{\pgfqpoint{8.102594in}{5.555456in}}%
\pgfpathlineto{\pgfqpoint{8.155638in}{4.801844in}}%
\pgfpathlineto{\pgfqpoint{8.208682in}{4.795985in}}%
\pgfpathlineto{\pgfqpoint{8.261726in}{5.555456in}}%
\pgfpathlineto{\pgfqpoint{8.314770in}{5.555456in}}%
\pgfpathlineto{\pgfqpoint{8.367814in}{4.778402in}}%
\pgfpathlineto{\pgfqpoint{8.420858in}{5.555456in}}%
\pgfpathlineto{\pgfqpoint{8.473902in}{4.719180in}}%
\pgfpathlineto{\pgfqpoint{8.526946in}{4.691039in}}%
\pgfpathlineto{\pgfqpoint{8.579990in}{5.555456in}}%
\pgfpathlineto{\pgfqpoint{8.739121in}{5.555456in}}%
\pgfpathlineto{\pgfqpoint{8.792165in}{5.420011in}}%
\pgfpathlineto{\pgfqpoint{8.845209in}{5.555456in}}%
\pgfpathlineto{\pgfqpoint{8.898253in}{5.555456in}}%
\pgfpathlineto{\pgfqpoint{8.951297in}{4.542245in}}%
\pgfpathlineto{\pgfqpoint{9.004341in}{4.990837in}}%
\pgfpathlineto{\pgfqpoint{9.057385in}{4.626605in}}%
\pgfpathlineto{\pgfqpoint{9.110429in}{4.661699in}}%
\pgfpathlineto{\pgfqpoint{9.163473in}{5.555456in}}%
\pgfpathlineto{\pgfqpoint{9.375649in}{5.555456in}}%
\pgfpathlineto{\pgfqpoint{9.428692in}{4.818062in}}%
\pgfpathlineto{\pgfqpoint{9.481736in}{4.859785in}}%
\pgfpathlineto{\pgfqpoint{9.534780in}{5.555456in}}%
\pgfpathlineto{\pgfqpoint{9.587824in}{5.555456in}}%
\pgfpathlineto{\pgfqpoint{9.640868in}{5.281518in}}%
\pgfpathlineto{\pgfqpoint{9.693912in}{5.555456in}}%
\pgfpathlineto{\pgfqpoint{9.800000in}{5.555456in}}%
\pgfpathlineto{\pgfqpoint{9.800000in}{5.555456in}}%
\pgfusepath{stroke}%
\end{pgfscope}%
\begin{pgfscope}%
\pgfpathrectangle{\pgfqpoint{0.941663in}{4.334375in}}{\pgfqpoint{8.858337in}{3.465625in}}%
\pgfusepath{clip}%
\pgfsetbuttcap%
\pgfsetroundjoin%
\definecolor{currentfill}{rgb}{0.501961,0.501961,0.501961}%
\pgfsetfillcolor{currentfill}%
\pgfsetlinewidth{1.003750pt}%
\definecolor{currentstroke}{rgb}{0.501961,0.501961,0.501961}%
\pgfsetstrokecolor{currentstroke}%
\pgfsetdash{}{0pt}%
\pgfsys@defobject{currentmarker}{\pgfqpoint{0.941663in}{4.491903in}}{\pgfqpoint{9.800000in}{5.555456in}}{%
\pgfpathmoveto{\pgfqpoint{0.941663in}{4.606409in}}%
\pgfpathlineto{\pgfqpoint{0.941663in}{5.302080in}}%
\pgfpathlineto{\pgfqpoint{0.994707in}{5.555456in}}%
\pgfpathlineto{\pgfqpoint{1.047751in}{5.298031in}}%
\pgfpathlineto{\pgfqpoint{1.100795in}{5.267250in}}%
\pgfpathlineto{\pgfqpoint{1.153839in}{5.253587in}}%
\pgfpathlineto{\pgfqpoint{1.206883in}{5.555456in}}%
\pgfpathlineto{\pgfqpoint{1.259927in}{5.555456in}}%
\pgfpathlineto{\pgfqpoint{1.312970in}{5.555456in}}%
\pgfpathlineto{\pgfqpoint{1.366014in}{5.207621in}}%
\pgfpathlineto{\pgfqpoint{1.419058in}{5.263460in}}%
\pgfpathlineto{\pgfqpoint{1.472102in}{5.268823in}}%
\pgfpathlineto{\pgfqpoint{1.525146in}{5.349185in}}%
\pgfpathlineto{\pgfqpoint{1.578190in}{5.343668in}}%
\pgfpathlineto{\pgfqpoint{1.631234in}{5.555456in}}%
\pgfpathlineto{\pgfqpoint{1.684278in}{5.555456in}}%
\pgfpathlineto{\pgfqpoint{1.737322in}{5.491300in}}%
\pgfpathlineto{\pgfqpoint{1.790366in}{5.473721in}}%
\pgfpathlineto{\pgfqpoint{1.843410in}{5.555456in}}%
\pgfpathlineto{\pgfqpoint{1.896454in}{5.555456in}}%
\pgfpathlineto{\pgfqpoint{1.949498in}{5.447032in}}%
\pgfpathlineto{\pgfqpoint{2.002542in}{5.555456in}}%
\pgfpathlineto{\pgfqpoint{2.055586in}{5.457838in}}%
\pgfpathlineto{\pgfqpoint{2.108629in}{5.555456in}}%
\pgfpathlineto{\pgfqpoint{2.161673in}{5.347486in}}%
\pgfpathlineto{\pgfqpoint{2.214717in}{5.555456in}}%
\pgfpathlineto{\pgfqpoint{2.267761in}{5.542464in}}%
\pgfpathlineto{\pgfqpoint{2.320805in}{5.278077in}}%
\pgfpathlineto{\pgfqpoint{2.373849in}{5.555456in}}%
\pgfpathlineto{\pgfqpoint{2.426893in}{5.555456in}}%
\pgfpathlineto{\pgfqpoint{2.479937in}{5.397997in}}%
\pgfpathlineto{\pgfqpoint{2.532981in}{5.253753in}}%
\pgfpathlineto{\pgfqpoint{2.586025in}{5.231942in}}%
\pgfpathlineto{\pgfqpoint{2.639069in}{5.227448in}}%
\pgfpathlineto{\pgfqpoint{2.692113in}{5.274799in}}%
\pgfpathlineto{\pgfqpoint{2.745157in}{5.555456in}}%
\pgfpathlineto{\pgfqpoint{2.798201in}{5.555456in}}%
\pgfpathlineto{\pgfqpoint{2.851245in}{5.555456in}}%
\pgfpathlineto{\pgfqpoint{2.904288in}{5.357472in}}%
\pgfpathlineto{\pgfqpoint{2.957332in}{5.555456in}}%
\pgfpathlineto{\pgfqpoint{3.010376in}{5.443237in}}%
\pgfpathlineto{\pgfqpoint{3.063420in}{5.555456in}}%
\pgfpathlineto{\pgfqpoint{3.116464in}{5.555456in}}%
\pgfpathlineto{\pgfqpoint{3.169508in}{5.475056in}}%
\pgfpathlineto{\pgfqpoint{3.222552in}{5.505579in}}%
\pgfpathlineto{\pgfqpoint{3.275596in}{5.421720in}}%
\pgfpathlineto{\pgfqpoint{3.328640in}{5.480294in}}%
\pgfpathlineto{\pgfqpoint{3.381684in}{5.555456in}}%
\pgfpathlineto{\pgfqpoint{3.434728in}{5.332680in}}%
\pgfpathlineto{\pgfqpoint{3.487772in}{5.555456in}}%
\pgfpathlineto{\pgfqpoint{3.540816in}{5.555456in}}%
\pgfpathlineto{\pgfqpoint{3.593860in}{5.555456in}}%
\pgfpathlineto{\pgfqpoint{3.646904in}{5.270980in}}%
\pgfpathlineto{\pgfqpoint{3.699948in}{5.227131in}}%
\pgfpathlineto{\pgfqpoint{3.752991in}{5.555456in}}%
\pgfpathlineto{\pgfqpoint{3.806035in}{5.267382in}}%
\pgfpathlineto{\pgfqpoint{3.859079in}{5.555456in}}%
\pgfpathlineto{\pgfqpoint{3.912123in}{5.555456in}}%
\pgfpathlineto{\pgfqpoint{3.965167in}{5.555456in}}%
\pgfpathlineto{\pgfqpoint{4.018211in}{5.273356in}}%
\pgfpathlineto{\pgfqpoint{4.071255in}{5.355447in}}%
\pgfpathlineto{\pgfqpoint{4.124299in}{5.483615in}}%
\pgfpathlineto{\pgfqpoint{4.177343in}{5.399336in}}%
\pgfpathlineto{\pgfqpoint{4.230387in}{5.437444in}}%
\pgfpathlineto{\pgfqpoint{4.283431in}{5.453162in}}%
\pgfpathlineto{\pgfqpoint{4.336475in}{5.555456in}}%
\pgfpathlineto{\pgfqpoint{4.389519in}{5.555456in}}%
\pgfpathlineto{\pgfqpoint{4.442563in}{5.555456in}}%
\pgfpathlineto{\pgfqpoint{4.495607in}{5.555456in}}%
\pgfpathlineto{\pgfqpoint{4.548650in}{5.555456in}}%
\pgfpathlineto{\pgfqpoint{4.601694in}{5.555456in}}%
\pgfpathlineto{\pgfqpoint{4.654738in}{5.454774in}}%
\pgfpathlineto{\pgfqpoint{4.707782in}{5.555456in}}%
\pgfpathlineto{\pgfqpoint{4.760826in}{5.555456in}}%
\pgfpathlineto{\pgfqpoint{4.813870in}{5.335037in}}%
\pgfpathlineto{\pgfqpoint{4.866914in}{5.555456in}}%
\pgfpathlineto{\pgfqpoint{4.919958in}{5.555456in}}%
\pgfpathlineto{\pgfqpoint{4.973002in}{5.216034in}}%
\pgfpathlineto{\pgfqpoint{5.026046in}{5.555456in}}%
\pgfpathlineto{\pgfqpoint{5.079090in}{5.207621in}}%
\pgfpathlineto{\pgfqpoint{5.132134in}{5.555456in}}%
\pgfpathlineto{\pgfqpoint{5.185178in}{5.555456in}}%
\pgfpathlineto{\pgfqpoint{5.238222in}{5.555456in}}%
\pgfpathlineto{\pgfqpoint{5.291266in}{5.335561in}}%
\pgfpathlineto{\pgfqpoint{5.344309in}{5.555456in}}%
\pgfpathlineto{\pgfqpoint{5.397353in}{5.555456in}}%
\pgfpathlineto{\pgfqpoint{5.450397in}{5.555456in}}%
\pgfpathlineto{\pgfqpoint{5.503441in}{5.476742in}}%
\pgfpathlineto{\pgfqpoint{5.556485in}{5.555456in}}%
\pgfpathlineto{\pgfqpoint{5.609529in}{5.555456in}}%
\pgfpathlineto{\pgfqpoint{5.662573in}{5.555456in}}%
\pgfpathlineto{\pgfqpoint{5.715617in}{5.482534in}}%
\pgfpathlineto{\pgfqpoint{5.768661in}{5.555456in}}%
\pgfpathlineto{\pgfqpoint{5.821705in}{5.555456in}}%
\pgfpathlineto{\pgfqpoint{5.874749in}{5.555456in}}%
\pgfpathlineto{\pgfqpoint{5.927793in}{5.555456in}}%
\pgfpathlineto{\pgfqpoint{5.980837in}{5.406177in}}%
\pgfpathlineto{\pgfqpoint{6.033881in}{5.555456in}}%
\pgfpathlineto{\pgfqpoint{6.086925in}{5.335040in}}%
\pgfpathlineto{\pgfqpoint{6.139969in}{5.316566in}}%
\pgfpathlineto{\pgfqpoint{6.193012in}{5.555456in}}%
\pgfpathlineto{\pgfqpoint{6.246056in}{5.555456in}}%
\pgfpathlineto{\pgfqpoint{6.299100in}{5.237069in}}%
\pgfpathlineto{\pgfqpoint{6.352144in}{5.555456in}}%
\pgfpathlineto{\pgfqpoint{6.405188in}{5.555456in}}%
\pgfpathlineto{\pgfqpoint{6.458232in}{5.555456in}}%
\pgfpathlineto{\pgfqpoint{6.511276in}{5.306663in}}%
\pgfpathlineto{\pgfqpoint{6.564320in}{5.303012in}}%
\pgfpathlineto{\pgfqpoint{6.617364in}{5.369970in}}%
\pgfpathlineto{\pgfqpoint{6.670408in}{5.399912in}}%
\pgfpathlineto{\pgfqpoint{6.723452in}{5.445993in}}%
\pgfpathlineto{\pgfqpoint{6.776496in}{5.555456in}}%
\pgfpathlineto{\pgfqpoint{6.829540in}{5.457960in}}%
\pgfpathlineto{\pgfqpoint{6.882584in}{5.555456in}}%
\pgfpathlineto{\pgfqpoint{6.935628in}{5.451613in}}%
\pgfpathlineto{\pgfqpoint{6.988671in}{5.555456in}}%
\pgfpathlineto{\pgfqpoint{7.041715in}{5.470876in}}%
\pgfpathlineto{\pgfqpoint{7.094759in}{5.493927in}}%
\pgfpathlineto{\pgfqpoint{7.147803in}{5.475721in}}%
\pgfpathlineto{\pgfqpoint{7.200847in}{5.555456in}}%
\pgfpathlineto{\pgfqpoint{7.253891in}{5.555456in}}%
\pgfpathlineto{\pgfqpoint{7.306935in}{5.455611in}}%
\pgfpathlineto{\pgfqpoint{7.359979in}{5.555456in}}%
\pgfpathlineto{\pgfqpoint{7.413023in}{5.555456in}}%
\pgfpathlineto{\pgfqpoint{7.466067in}{5.555456in}}%
\pgfpathlineto{\pgfqpoint{7.519111in}{5.555456in}}%
\pgfpathlineto{\pgfqpoint{7.572155in}{5.555456in}}%
\pgfpathlineto{\pgfqpoint{7.625199in}{5.555456in}}%
\pgfpathlineto{\pgfqpoint{7.678243in}{5.555456in}}%
\pgfpathlineto{\pgfqpoint{7.731287in}{5.268845in}}%
\pgfpathlineto{\pgfqpoint{7.784330in}{5.304877in}}%
\pgfpathlineto{\pgfqpoint{7.837374in}{5.313128in}}%
\pgfpathlineto{\pgfqpoint{7.890418in}{5.354431in}}%
\pgfpathlineto{\pgfqpoint{7.943462in}{5.352176in}}%
\pgfpathlineto{\pgfqpoint{7.996506in}{5.430879in}}%
\pgfpathlineto{\pgfqpoint{8.049550in}{5.555456in}}%
\pgfpathlineto{\pgfqpoint{8.102594in}{5.555456in}}%
\pgfpathlineto{\pgfqpoint{8.155638in}{5.497516in}}%
\pgfpathlineto{\pgfqpoint{8.208682in}{5.491656in}}%
\pgfpathlineto{\pgfqpoint{8.261726in}{5.555456in}}%
\pgfpathlineto{\pgfqpoint{8.314770in}{5.555456in}}%
\pgfpathlineto{\pgfqpoint{8.367814in}{5.474073in}}%
\pgfpathlineto{\pgfqpoint{8.420858in}{5.555456in}}%
\pgfpathlineto{\pgfqpoint{8.473902in}{5.414851in}}%
\pgfpathlineto{\pgfqpoint{8.526946in}{5.386710in}}%
\pgfpathlineto{\pgfqpoint{8.579990in}{5.555456in}}%
\pgfpathlineto{\pgfqpoint{8.633033in}{5.555456in}}%
\pgfpathlineto{\pgfqpoint{8.686077in}{5.555456in}}%
\pgfpathlineto{\pgfqpoint{8.739121in}{5.555456in}}%
\pgfpathlineto{\pgfqpoint{8.792165in}{5.555456in}}%
\pgfpathlineto{\pgfqpoint{8.845209in}{5.555456in}}%
\pgfpathlineto{\pgfqpoint{8.898253in}{5.555456in}}%
\pgfpathlineto{\pgfqpoint{8.951297in}{5.237916in}}%
\pgfpathlineto{\pgfqpoint{9.004341in}{5.555456in}}%
\pgfpathlineto{\pgfqpoint{9.057385in}{5.322277in}}%
\pgfpathlineto{\pgfqpoint{9.110429in}{5.357370in}}%
\pgfpathlineto{\pgfqpoint{9.163473in}{5.555456in}}%
\pgfpathlineto{\pgfqpoint{9.216517in}{5.555456in}}%
\pgfpathlineto{\pgfqpoint{9.269561in}{5.555456in}}%
\pgfpathlineto{\pgfqpoint{9.322605in}{5.555456in}}%
\pgfpathlineto{\pgfqpoint{9.375649in}{5.555456in}}%
\pgfpathlineto{\pgfqpoint{9.428692in}{5.513734in}}%
\pgfpathlineto{\pgfqpoint{9.481736in}{5.555456in}}%
\pgfpathlineto{\pgfqpoint{9.534780in}{5.555456in}}%
\pgfpathlineto{\pgfqpoint{9.587824in}{5.555456in}}%
\pgfpathlineto{\pgfqpoint{9.640868in}{5.555456in}}%
\pgfpathlineto{\pgfqpoint{9.693912in}{5.555456in}}%
\pgfpathlineto{\pgfqpoint{9.746956in}{5.555456in}}%
\pgfpathlineto{\pgfqpoint{9.800000in}{5.555456in}}%
\pgfpathlineto{\pgfqpoint{9.800000in}{5.555456in}}%
\pgfpathlineto{\pgfqpoint{9.800000in}{5.555456in}}%
\pgfpathlineto{\pgfqpoint{9.746956in}{5.555456in}}%
\pgfpathlineto{\pgfqpoint{9.693912in}{5.555456in}}%
\pgfpathlineto{\pgfqpoint{9.640868in}{5.281518in}}%
\pgfpathlineto{\pgfqpoint{9.587824in}{5.555456in}}%
\pgfpathlineto{\pgfqpoint{9.534780in}{5.555456in}}%
\pgfpathlineto{\pgfqpoint{9.481736in}{4.859785in}}%
\pgfpathlineto{\pgfqpoint{9.428692in}{4.818062in}}%
\pgfpathlineto{\pgfqpoint{9.375649in}{5.555456in}}%
\pgfpathlineto{\pgfqpoint{9.322605in}{5.555456in}}%
\pgfpathlineto{\pgfqpoint{9.269561in}{5.555456in}}%
\pgfpathlineto{\pgfqpoint{9.216517in}{5.555456in}}%
\pgfpathlineto{\pgfqpoint{9.163473in}{5.555456in}}%
\pgfpathlineto{\pgfqpoint{9.110429in}{4.661699in}}%
\pgfpathlineto{\pgfqpoint{9.057385in}{4.626605in}}%
\pgfpathlineto{\pgfqpoint{9.004341in}{4.990837in}}%
\pgfpathlineto{\pgfqpoint{8.951297in}{4.542245in}}%
\pgfpathlineto{\pgfqpoint{8.898253in}{5.555456in}}%
\pgfpathlineto{\pgfqpoint{8.845209in}{5.555456in}}%
\pgfpathlineto{\pgfqpoint{8.792165in}{5.420011in}}%
\pgfpathlineto{\pgfqpoint{8.739121in}{5.555456in}}%
\pgfpathlineto{\pgfqpoint{8.686077in}{5.555456in}}%
\pgfpathlineto{\pgfqpoint{8.633033in}{5.555456in}}%
\pgfpathlineto{\pgfqpoint{8.579990in}{5.555456in}}%
\pgfpathlineto{\pgfqpoint{8.526946in}{4.691039in}}%
\pgfpathlineto{\pgfqpoint{8.473902in}{4.719180in}}%
\pgfpathlineto{\pgfqpoint{8.420858in}{5.555456in}}%
\pgfpathlineto{\pgfqpoint{8.367814in}{4.778402in}}%
\pgfpathlineto{\pgfqpoint{8.314770in}{5.555456in}}%
\pgfpathlineto{\pgfqpoint{8.261726in}{5.555456in}}%
\pgfpathlineto{\pgfqpoint{8.208682in}{4.795985in}}%
\pgfpathlineto{\pgfqpoint{8.155638in}{4.801844in}}%
\pgfpathlineto{\pgfqpoint{8.102594in}{5.555456in}}%
\pgfpathlineto{\pgfqpoint{8.049550in}{5.555456in}}%
\pgfpathlineto{\pgfqpoint{7.996506in}{4.735208in}}%
\pgfpathlineto{\pgfqpoint{7.943462in}{4.656505in}}%
\pgfpathlineto{\pgfqpoint{7.890418in}{4.658759in}}%
\pgfpathlineto{\pgfqpoint{7.837374in}{4.617456in}}%
\pgfpathlineto{\pgfqpoint{7.784330in}{4.609205in}}%
\pgfpathlineto{\pgfqpoint{7.731287in}{4.573173in}}%
\pgfpathlineto{\pgfqpoint{7.678243in}{5.513983in}}%
\pgfpathlineto{\pgfqpoint{7.625199in}{5.555456in}}%
\pgfpathlineto{\pgfqpoint{7.572155in}{5.555456in}}%
\pgfpathlineto{\pgfqpoint{7.519111in}{5.074761in}}%
\pgfpathlineto{\pgfqpoint{7.466067in}{5.147602in}}%
\pgfpathlineto{\pgfqpoint{7.413023in}{5.555456in}}%
\pgfpathlineto{\pgfqpoint{7.359979in}{5.555456in}}%
\pgfpathlineto{\pgfqpoint{7.306935in}{4.759939in}}%
\pgfpathlineto{\pgfqpoint{7.253891in}{4.923953in}}%
\pgfpathlineto{\pgfqpoint{7.200847in}{5.555456in}}%
\pgfpathlineto{\pgfqpoint{7.147803in}{4.780049in}}%
\pgfpathlineto{\pgfqpoint{7.094759in}{4.798255in}}%
\pgfpathlineto{\pgfqpoint{7.041715in}{4.775204in}}%
\pgfpathlineto{\pgfqpoint{6.988671in}{5.000909in}}%
\pgfpathlineto{\pgfqpoint{6.935628in}{4.755942in}}%
\pgfpathlineto{\pgfqpoint{6.882584in}{5.555456in}}%
\pgfpathlineto{\pgfqpoint{6.829540in}{4.762288in}}%
\pgfpathlineto{\pgfqpoint{6.776496in}{4.859785in}}%
\pgfpathlineto{\pgfqpoint{6.723452in}{4.750321in}}%
\pgfpathlineto{\pgfqpoint{6.670408in}{4.704240in}}%
\pgfpathlineto{\pgfqpoint{6.617364in}{4.674298in}}%
\pgfpathlineto{\pgfqpoint{6.564320in}{4.607341in}}%
\pgfpathlineto{\pgfqpoint{6.511276in}{4.610991in}}%
\pgfpathlineto{\pgfqpoint{6.458232in}{5.077516in}}%
\pgfpathlineto{\pgfqpoint{6.405188in}{5.555456in}}%
\pgfpathlineto{\pgfqpoint{6.352144in}{5.178906in}}%
\pgfpathlineto{\pgfqpoint{6.299100in}{4.541397in}}%
\pgfpathlineto{\pgfqpoint{6.246056in}{5.555456in}}%
\pgfpathlineto{\pgfqpoint{6.193012in}{5.555456in}}%
\pgfpathlineto{\pgfqpoint{6.139969in}{4.620894in}}%
\pgfpathlineto{\pgfqpoint{6.086925in}{4.639368in}}%
\pgfpathlineto{\pgfqpoint{6.033881in}{5.193362in}}%
\pgfpathlineto{\pgfqpoint{5.980837in}{4.710506in}}%
\pgfpathlineto{\pgfqpoint{5.927793in}{5.555456in}}%
\pgfpathlineto{\pgfqpoint{5.874749in}{5.555456in}}%
\pgfpathlineto{\pgfqpoint{5.821705in}{5.555456in}}%
\pgfpathlineto{\pgfqpoint{5.768661in}{5.555456in}}%
\pgfpathlineto{\pgfqpoint{5.715617in}{4.786862in}}%
\pgfpathlineto{\pgfqpoint{5.662573in}{5.555456in}}%
\pgfpathlineto{\pgfqpoint{5.609529in}{5.555456in}}%
\pgfpathlineto{\pgfqpoint{5.556485in}{5.135041in}}%
\pgfpathlineto{\pgfqpoint{5.503441in}{4.781070in}}%
\pgfpathlineto{\pgfqpoint{5.450397in}{5.555456in}}%
\pgfpathlineto{\pgfqpoint{5.397353in}{5.555456in}}%
\pgfpathlineto{\pgfqpoint{5.344309in}{5.453871in}}%
\pgfpathlineto{\pgfqpoint{5.291266in}{4.639889in}}%
\pgfpathlineto{\pgfqpoint{5.238222in}{5.555456in}}%
\pgfpathlineto{\pgfqpoint{5.185178in}{5.555456in}}%
\pgfpathlineto{\pgfqpoint{5.132134in}{5.444876in}}%
\pgfpathlineto{\pgfqpoint{5.079090in}{4.491903in}}%
\pgfpathlineto{\pgfqpoint{5.026046in}{5.387465in}}%
\pgfpathlineto{\pgfqpoint{4.973002in}{4.520363in}}%
\pgfpathlineto{\pgfqpoint{4.919958in}{5.555456in}}%
\pgfpathlineto{\pgfqpoint{4.866914in}{5.555456in}}%
\pgfpathlineto{\pgfqpoint{4.813870in}{4.639365in}}%
\pgfpathlineto{\pgfqpoint{4.760826in}{4.951917in}}%
\pgfpathlineto{\pgfqpoint{4.707782in}{5.555456in}}%
\pgfpathlineto{\pgfqpoint{4.654738in}{4.759102in}}%
\pgfpathlineto{\pgfqpoint{4.601694in}{5.555456in}}%
\pgfpathlineto{\pgfqpoint{4.548650in}{5.101868in}}%
\pgfpathlineto{\pgfqpoint{4.495607in}{5.555456in}}%
\pgfpathlineto{\pgfqpoint{4.442563in}{4.896628in}}%
\pgfpathlineto{\pgfqpoint{4.389519in}{4.859785in}}%
\pgfpathlineto{\pgfqpoint{4.336475in}{5.555456in}}%
\pgfpathlineto{\pgfqpoint{4.283431in}{4.757491in}}%
\pgfpathlineto{\pgfqpoint{4.230387in}{4.741772in}}%
\pgfpathlineto{\pgfqpoint{4.177343in}{4.703664in}}%
\pgfpathlineto{\pgfqpoint{4.124299in}{4.787943in}}%
\pgfpathlineto{\pgfqpoint{4.071255in}{4.659775in}}%
\pgfpathlineto{\pgfqpoint{4.018211in}{4.577684in}}%
\pgfpathlineto{\pgfqpoint{3.965167in}{5.299182in}}%
\pgfpathlineto{\pgfqpoint{3.912123in}{4.859785in}}%
\pgfpathlineto{\pgfqpoint{3.859079in}{4.859785in}}%
\pgfpathlineto{\pgfqpoint{3.806035in}{4.571710in}}%
\pgfpathlineto{\pgfqpoint{3.752991in}{4.859785in}}%
\pgfpathlineto{\pgfqpoint{3.699948in}{4.531459in}}%
\pgfpathlineto{\pgfqpoint{3.646904in}{4.575309in}}%
\pgfpathlineto{\pgfqpoint{3.593860in}{5.052711in}}%
\pgfpathlineto{\pgfqpoint{3.540816in}{5.555456in}}%
\pgfpathlineto{\pgfqpoint{3.487772in}{5.331044in}}%
\pgfpathlineto{\pgfqpoint{3.434728in}{4.637008in}}%
\pgfpathlineto{\pgfqpoint{3.381684in}{4.982873in}}%
\pgfpathlineto{\pgfqpoint{3.328640in}{4.784623in}}%
\pgfpathlineto{\pgfqpoint{3.275596in}{4.726048in}}%
\pgfpathlineto{\pgfqpoint{3.222552in}{4.809907in}}%
\pgfpathlineto{\pgfqpoint{3.169508in}{4.779384in}}%
\pgfpathlineto{\pgfqpoint{3.116464in}{4.873985in}}%
\pgfpathlineto{\pgfqpoint{3.063420in}{5.529745in}}%
\pgfpathlineto{\pgfqpoint{3.010376in}{4.747565in}}%
\pgfpathlineto{\pgfqpoint{2.957332in}{5.555456in}}%
\pgfpathlineto{\pgfqpoint{2.904288in}{4.661801in}}%
\pgfpathlineto{\pgfqpoint{2.851245in}{5.097809in}}%
\pgfpathlineto{\pgfqpoint{2.798201in}{4.614894in}}%
\pgfpathlineto{\pgfqpoint{2.745157in}{4.572498in}}%
\pgfpathlineto{\pgfqpoint{2.692113in}{4.501574in}}%
\pgfpathlineto{\pgfqpoint{2.639069in}{4.531777in}}%
\pgfpathlineto{\pgfqpoint{2.586025in}{4.536270in}}%
\pgfpathlineto{\pgfqpoint{2.532981in}{4.558081in}}%
\pgfpathlineto{\pgfqpoint{2.479937in}{4.702326in}}%
\pgfpathlineto{\pgfqpoint{2.426893in}{5.555456in}}%
\pgfpathlineto{\pgfqpoint{2.373849in}{5.523078in}}%
\pgfpathlineto{\pgfqpoint{2.320805in}{4.582406in}}%
\pgfpathlineto{\pgfqpoint{2.267761in}{4.846792in}}%
\pgfpathlineto{\pgfqpoint{2.214717in}{5.555456in}}%
\pgfpathlineto{\pgfqpoint{2.161673in}{4.651814in}}%
\pgfpathlineto{\pgfqpoint{2.108629in}{5.048281in}}%
\pgfpathlineto{\pgfqpoint{2.055586in}{4.762166in}}%
\pgfpathlineto{\pgfqpoint{2.002542in}{5.555456in}}%
\pgfpathlineto{\pgfqpoint{1.949498in}{4.751360in}}%
\pgfpathlineto{\pgfqpoint{1.896454in}{5.555456in}}%
\pgfpathlineto{\pgfqpoint{1.843410in}{5.555456in}}%
\pgfpathlineto{\pgfqpoint{1.790366in}{4.778049in}}%
\pgfpathlineto{\pgfqpoint{1.737322in}{4.795628in}}%
\pgfpathlineto{\pgfqpoint{1.684278in}{4.875476in}}%
\pgfpathlineto{\pgfqpoint{1.631234in}{5.423055in}}%
\pgfpathlineto{\pgfqpoint{1.578190in}{4.630644in}}%
\pgfpathlineto{\pgfqpoint{1.525146in}{4.653514in}}%
\pgfpathlineto{\pgfqpoint{1.472102in}{4.573151in}}%
\pgfpathlineto{\pgfqpoint{1.419058in}{4.567788in}}%
\pgfpathlineto{\pgfqpoint{1.366014in}{4.493572in}}%
\pgfpathlineto{\pgfqpoint{1.312970in}{5.172617in}}%
\pgfpathlineto{\pgfqpoint{1.259927in}{5.369914in}}%
\pgfpathlineto{\pgfqpoint{1.206883in}{5.127392in}}%
\pgfpathlineto{\pgfqpoint{1.153839in}{4.557915in}}%
\pgfpathlineto{\pgfqpoint{1.100795in}{4.571579in}}%
\pgfpathlineto{\pgfqpoint{1.047751in}{4.602360in}}%
\pgfpathlineto{\pgfqpoint{0.994707in}{4.859785in}}%
\pgfpathlineto{\pgfqpoint{0.941663in}{4.606409in}}%
\pgfpathlineto{\pgfqpoint{0.941663in}{4.606409in}}%
\pgfpathclose%
\pgfusepath{stroke,fill}%
}%
\begin{pgfscope}%
\pgfsys@transformshift{0.000000in}{0.000000in}%
\pgfsys@useobject{currentmarker}{}%
\end{pgfscope}%
\end{pgfscope}%
\begin{pgfscope}%
\pgfpathrectangle{\pgfqpoint{0.941663in}{4.334375in}}{\pgfqpoint{8.858337in}{3.465625in}}%
\pgfusepath{clip}%
\pgfsetrectcap%
\pgfsetroundjoin%
\pgfsetlinewidth{1.505625pt}%
\definecolor{currentstroke}{rgb}{0.090196,0.745098,0.811765}%
\pgfsetstrokecolor{currentstroke}%
\pgfsetdash{}{0pt}%
\pgfpathmoveto{\pgfqpoint{0.941663in}{7.642472in}}%
\pgfpathlineto{\pgfqpoint{0.994707in}{7.398889in}}%
\pgfpathlineto{\pgfqpoint{1.047751in}{7.642472in}}%
\pgfpathlineto{\pgfqpoint{1.100795in}{7.630782in}}%
\pgfpathlineto{\pgfqpoint{1.153839in}{7.642472in}}%
\pgfpathlineto{\pgfqpoint{1.206883in}{7.113972in}}%
\pgfpathlineto{\pgfqpoint{1.259927in}{6.752271in}}%
\pgfpathlineto{\pgfqpoint{1.312970in}{7.018171in}}%
\pgfpathlineto{\pgfqpoint{1.366014in}{7.642472in}}%
\pgfpathlineto{\pgfqpoint{1.578190in}{7.642472in}}%
\pgfpathlineto{\pgfqpoint{1.631234in}{6.920473in}}%
\pgfpathlineto{\pgfqpoint{1.684278in}{7.452967in}}%
\pgfpathlineto{\pgfqpoint{1.737322in}{7.536394in}}%
\pgfpathlineto{\pgfqpoint{1.790366in}{7.642472in}}%
\pgfpathlineto{\pgfqpoint{1.843410in}{6.911654in}}%
\pgfpathlineto{\pgfqpoint{1.896454in}{6.883587in}}%
\pgfpathlineto{\pgfqpoint{1.949498in}{7.642472in}}%
\pgfpathlineto{\pgfqpoint{2.002542in}{6.857739in}}%
\pgfpathlineto{\pgfqpoint{2.055586in}{7.642472in}}%
\pgfpathlineto{\pgfqpoint{2.108629in}{7.339659in}}%
\pgfpathlineto{\pgfqpoint{2.161673in}{7.642472in}}%
\pgfpathlineto{\pgfqpoint{2.214717in}{6.753348in}}%
\pgfpathlineto{\pgfqpoint{2.267761in}{7.419218in}}%
\pgfpathlineto{\pgfqpoint{2.320805in}{7.642472in}}%
\pgfpathlineto{\pgfqpoint{2.373849in}{6.676762in}}%
\pgfpathlineto{\pgfqpoint{2.426893in}{6.606651in}}%
\pgfpathlineto{\pgfqpoint{2.479937in}{7.449769in}}%
\pgfpathlineto{\pgfqpoint{2.532981in}{7.642472in}}%
\pgfpathlineto{\pgfqpoint{2.798201in}{7.642472in}}%
\pgfpathlineto{\pgfqpoint{2.851245in}{7.173688in}}%
\pgfpathlineto{\pgfqpoint{2.904288in}{7.642472in}}%
\pgfpathlineto{\pgfqpoint{2.957332in}{6.836531in}}%
\pgfpathlineto{\pgfqpoint{3.010376in}{7.642472in}}%
\pgfpathlineto{\pgfqpoint{3.063420in}{6.841623in}}%
\pgfpathlineto{\pgfqpoint{3.116464in}{7.518262in}}%
\pgfpathlineto{\pgfqpoint{3.169508in}{7.642472in}}%
\pgfpathlineto{\pgfqpoint{3.328640in}{7.642472in}}%
\pgfpathlineto{\pgfqpoint{3.381684in}{7.361194in}}%
\pgfpathlineto{\pgfqpoint{3.434728in}{7.642472in}}%
\pgfpathlineto{\pgfqpoint{3.487772in}{6.972573in}}%
\pgfpathlineto{\pgfqpoint{3.540816in}{6.738883in}}%
\pgfpathlineto{\pgfqpoint{3.593860in}{7.167623in}}%
\pgfpathlineto{\pgfqpoint{3.646904in}{7.642472in}}%
\pgfpathlineto{\pgfqpoint{3.699948in}{7.642472in}}%
\pgfpathlineto{\pgfqpoint{3.752991in}{7.311866in}}%
\pgfpathlineto{\pgfqpoint{3.806035in}{7.642472in}}%
\pgfpathlineto{\pgfqpoint{3.859079in}{7.341795in}}%
\pgfpathlineto{\pgfqpoint{3.912123in}{7.302849in}}%
\pgfpathlineto{\pgfqpoint{3.965167in}{6.991885in}}%
\pgfpathlineto{\pgfqpoint{4.018211in}{7.642472in}}%
\pgfpathlineto{\pgfqpoint{4.071255in}{7.642472in}}%
\pgfpathlineto{\pgfqpoint{4.124299in}{7.539721in}}%
\pgfpathlineto{\pgfqpoint{4.177343in}{7.642472in}}%
\pgfpathlineto{\pgfqpoint{4.283431in}{7.642472in}}%
\pgfpathlineto{\pgfqpoint{4.336475in}{6.889082in}}%
\pgfpathlineto{\pgfqpoint{4.389519in}{7.554209in}}%
\pgfpathlineto{\pgfqpoint{4.442563in}{7.506419in}}%
\pgfpathlineto{\pgfqpoint{4.495607in}{6.883010in}}%
\pgfpathlineto{\pgfqpoint{4.548650in}{7.353421in}}%
\pgfpathlineto{\pgfqpoint{4.601694in}{6.827112in}}%
\pgfpathlineto{\pgfqpoint{4.654738in}{7.642472in}}%
\pgfpathlineto{\pgfqpoint{4.707782in}{6.766257in}}%
\pgfpathlineto{\pgfqpoint{4.760826in}{7.378709in}}%
\pgfpathlineto{\pgfqpoint{4.813870in}{7.642472in}}%
\pgfpathlineto{\pgfqpoint{4.866914in}{6.684929in}}%
\pgfpathlineto{\pgfqpoint{4.919958in}{6.671298in}}%
\pgfpathlineto{\pgfqpoint{4.973002in}{7.642472in}}%
\pgfpathlineto{\pgfqpoint{5.026046in}{6.790800in}}%
\pgfpathlineto{\pgfqpoint{5.079090in}{7.642472in}}%
\pgfpathlineto{\pgfqpoint{5.132134in}{6.692849in}}%
\pgfpathlineto{\pgfqpoint{5.185178in}{6.670254in}}%
\pgfpathlineto{\pgfqpoint{5.238222in}{6.666672in}}%
\pgfpathlineto{\pgfqpoint{5.291266in}{7.642472in}}%
\pgfpathlineto{\pgfqpoint{5.344309in}{6.838988in}}%
\pgfpathlineto{\pgfqpoint{5.397353in}{6.748316in}}%
\pgfpathlineto{\pgfqpoint{5.450397in}{6.877918in}}%
\pgfpathlineto{\pgfqpoint{5.503441in}{7.642472in}}%
\pgfpathlineto{\pgfqpoint{5.556485in}{7.281307in}}%
\pgfpathlineto{\pgfqpoint{5.609529in}{6.874196in}}%
\pgfpathlineto{\pgfqpoint{5.662573in}{6.925790in}}%
\pgfpathlineto{\pgfqpoint{5.715617in}{7.642472in}}%
\pgfpathlineto{\pgfqpoint{5.768661in}{6.913209in}}%
\pgfpathlineto{\pgfqpoint{5.821705in}{6.872997in}}%
\pgfpathlineto{\pgfqpoint{5.874749in}{6.796671in}}%
\pgfpathlineto{\pgfqpoint{5.927793in}{6.820928in}}%
\pgfpathlineto{\pgfqpoint{5.980837in}{7.642472in}}%
\pgfpathlineto{\pgfqpoint{6.033881in}{7.110408in}}%
\pgfpathlineto{\pgfqpoint{6.086925in}{7.642472in}}%
\pgfpathlineto{\pgfqpoint{6.139969in}{7.642472in}}%
\pgfpathlineto{\pgfqpoint{6.193012in}{6.635398in}}%
\pgfpathlineto{\pgfqpoint{6.246056in}{6.691111in}}%
\pgfpathlineto{\pgfqpoint{6.299100in}{7.642472in}}%
\pgfpathlineto{\pgfqpoint{6.352144in}{7.023749in}}%
\pgfpathlineto{\pgfqpoint{6.405188in}{6.593987in}}%
\pgfpathlineto{\pgfqpoint{6.458232in}{7.158194in}}%
\pgfpathlineto{\pgfqpoint{6.511276in}{7.642472in}}%
\pgfpathlineto{\pgfqpoint{6.723452in}{7.642472in}}%
\pgfpathlineto{\pgfqpoint{6.776496in}{7.554798in}}%
\pgfpathlineto{\pgfqpoint{6.829540in}{7.642472in}}%
\pgfpathlineto{\pgfqpoint{6.882584in}{6.881359in}}%
\pgfpathlineto{\pgfqpoint{6.935628in}{7.642472in}}%
\pgfpathlineto{\pgfqpoint{6.988671in}{7.441180in}}%
\pgfpathlineto{\pgfqpoint{7.041715in}{7.642472in}}%
\pgfpathlineto{\pgfqpoint{7.147803in}{7.642472in}}%
\pgfpathlineto{\pgfqpoint{7.200847in}{6.818171in}}%
\pgfpathlineto{\pgfqpoint{7.253891in}{7.350789in}}%
\pgfpathlineto{\pgfqpoint{7.306935in}{7.542837in}}%
\pgfpathlineto{\pgfqpoint{7.359979in}{6.672189in}}%
\pgfpathlineto{\pgfqpoint{7.413023in}{6.684093in}}%
\pgfpathlineto{\pgfqpoint{7.466067in}{7.061991in}}%
\pgfpathlineto{\pgfqpoint{7.519111in}{7.086960in}}%
\pgfpathlineto{\pgfqpoint{7.572155in}{6.593688in}}%
\pgfpathlineto{\pgfqpoint{7.625199in}{6.652740in}}%
\pgfpathlineto{\pgfqpoint{7.678243in}{6.739129in}}%
\pgfpathlineto{\pgfqpoint{7.731287in}{7.642472in}}%
\pgfpathlineto{\pgfqpoint{7.996506in}{7.642472in}}%
\pgfpathlineto{\pgfqpoint{8.049550in}{6.853606in}}%
\pgfpathlineto{\pgfqpoint{8.102594in}{6.863050in}}%
\pgfpathlineto{\pgfqpoint{8.155638in}{7.642472in}}%
\pgfpathlineto{\pgfqpoint{8.208682in}{7.642472in}}%
\pgfpathlineto{\pgfqpoint{8.261726in}{6.924712in}}%
\pgfpathlineto{\pgfqpoint{8.314770in}{6.876087in}}%
\pgfpathlineto{\pgfqpoint{8.367814in}{7.642472in}}%
\pgfpathlineto{\pgfqpoint{8.420858in}{6.857476in}}%
\pgfpathlineto{\pgfqpoint{8.473902in}{7.642472in}}%
\pgfpathlineto{\pgfqpoint{8.526946in}{7.642472in}}%
\pgfpathlineto{\pgfqpoint{8.579990in}{6.685501in}}%
\pgfpathlineto{\pgfqpoint{8.633033in}{6.694774in}}%
\pgfpathlineto{\pgfqpoint{8.686077in}{6.723687in}}%
\pgfpathlineto{\pgfqpoint{8.739121in}{6.662088in}}%
\pgfpathlineto{\pgfqpoint{8.792165in}{6.800188in}}%
\pgfpathlineto{\pgfqpoint{8.845209in}{6.642906in}}%
\pgfpathlineto{\pgfqpoint{8.898253in}{6.685692in}}%
\pgfpathlineto{\pgfqpoint{8.951297in}{7.642472in}}%
\pgfpathlineto{\pgfqpoint{9.004341in}{7.212818in}}%
\pgfpathlineto{\pgfqpoint{9.057385in}{7.642472in}}%
\pgfpathlineto{\pgfqpoint{9.110429in}{7.642472in}}%
\pgfpathlineto{\pgfqpoint{9.163473in}{6.770647in}}%
\pgfpathlineto{\pgfqpoint{9.216517in}{6.804492in}}%
\pgfpathlineto{\pgfqpoint{9.269561in}{6.811485in}}%
\pgfpathlineto{\pgfqpoint{9.322605in}{6.883905in}}%
\pgfpathlineto{\pgfqpoint{9.375649in}{6.879220in}}%
\pgfpathlineto{\pgfqpoint{9.428692in}{7.642472in}}%
\pgfpathlineto{\pgfqpoint{9.481736in}{7.580172in}}%
\pgfpathlineto{\pgfqpoint{9.534780in}{6.946800in}}%
\pgfpathlineto{\pgfqpoint{9.587824in}{6.868462in}}%
\pgfpathlineto{\pgfqpoint{9.640868in}{7.137950in}}%
\pgfpathlineto{\pgfqpoint{9.693912in}{6.803680in}}%
\pgfpathlineto{\pgfqpoint{9.746956in}{6.779712in}}%
\pgfpathlineto{\pgfqpoint{9.800000in}{6.765861in}}%
\pgfpathlineto{\pgfqpoint{9.800000in}{6.765861in}}%
\pgfusepath{stroke}%
\end{pgfscope}%
\begin{pgfscope}%
\pgfpathrectangle{\pgfqpoint{0.941663in}{4.334375in}}{\pgfqpoint{8.858337in}{3.465625in}}%
\pgfusepath{clip}%
\pgfsetbuttcap%
\pgfsetroundjoin%
\definecolor{currentfill}{rgb}{0.090196,0.745098,0.811765}%
\pgfsetfillcolor{currentfill}%
\pgfsetlinewidth{1.003750pt}%
\definecolor{currentstroke}{rgb}{0.090196,0.745098,0.811765}%
\pgfsetstrokecolor{currentstroke}%
\pgfsetdash{}{0pt}%
\pgfsys@defobject{currentmarker}{\pgfqpoint{0.941663in}{6.251128in}}{\pgfqpoint{9.800000in}{7.642472in}}{%
\pgfpathmoveto{\pgfqpoint{0.941663in}{7.642472in}}%
\pgfpathlineto{\pgfqpoint{0.941663in}{6.251128in}}%
\pgfpathlineto{\pgfqpoint{0.994707in}{6.261680in}}%
\pgfpathlineto{\pgfqpoint{1.047751in}{6.251128in}}%
\pgfpathlineto{\pgfqpoint{1.100795in}{6.251128in}}%
\pgfpathlineto{\pgfqpoint{1.153839in}{6.251128in}}%
\pgfpathlineto{\pgfqpoint{1.206883in}{6.598964in}}%
\pgfpathlineto{\pgfqpoint{1.259927in}{6.598964in}}%
\pgfpathlineto{\pgfqpoint{1.312970in}{6.598964in}}%
\pgfpathlineto{\pgfqpoint{1.366014in}{6.251128in}}%
\pgfpathlineto{\pgfqpoint{1.419058in}{6.251128in}}%
\pgfpathlineto{\pgfqpoint{1.472102in}{6.251128in}}%
\pgfpathlineto{\pgfqpoint{1.525146in}{6.251128in}}%
\pgfpathlineto{\pgfqpoint{1.578190in}{6.251128in}}%
\pgfpathlineto{\pgfqpoint{1.631234in}{6.598964in}}%
\pgfpathlineto{\pgfqpoint{1.684278in}{6.598964in}}%
\pgfpathlineto{\pgfqpoint{1.737322in}{6.251128in}}%
\pgfpathlineto{\pgfqpoint{1.790366in}{6.251128in}}%
\pgfpathlineto{\pgfqpoint{1.843410in}{6.734592in}}%
\pgfpathlineto{\pgfqpoint{1.896454in}{6.883587in}}%
\pgfpathlineto{\pgfqpoint{1.949498in}{6.251128in}}%
\pgfpathlineto{\pgfqpoint{2.002542in}{6.724358in}}%
\pgfpathlineto{\pgfqpoint{2.055586in}{6.251128in}}%
\pgfpathlineto{\pgfqpoint{2.108629in}{6.348747in}}%
\pgfpathlineto{\pgfqpoint{2.161673in}{6.251128in}}%
\pgfpathlineto{\pgfqpoint{2.214717in}{6.574971in}}%
\pgfpathlineto{\pgfqpoint{2.267761in}{6.251128in}}%
\pgfpathlineto{\pgfqpoint{2.320805in}{6.251128in}}%
\pgfpathlineto{\pgfqpoint{2.373849in}{6.541500in}}%
\pgfpathlineto{\pgfqpoint{2.426893in}{6.606651in}}%
\pgfpathlineto{\pgfqpoint{2.479937in}{6.251128in}}%
\pgfpathlineto{\pgfqpoint{2.532981in}{6.251128in}}%
\pgfpathlineto{\pgfqpoint{2.586025in}{6.251128in}}%
\pgfpathlineto{\pgfqpoint{2.639069in}{6.251128in}}%
\pgfpathlineto{\pgfqpoint{2.692113in}{6.251128in}}%
\pgfpathlineto{\pgfqpoint{2.745157in}{6.251128in}}%
\pgfpathlineto{\pgfqpoint{2.798201in}{6.251128in}}%
\pgfpathlineto{\pgfqpoint{2.851245in}{6.598964in}}%
\pgfpathlineto{\pgfqpoint{2.904288in}{6.251128in}}%
\pgfpathlineto{\pgfqpoint{2.957332in}{6.744748in}}%
\pgfpathlineto{\pgfqpoint{3.010376in}{6.251128in}}%
\pgfpathlineto{\pgfqpoint{3.063420in}{6.598964in}}%
\pgfpathlineto{\pgfqpoint{3.116464in}{6.598964in}}%
\pgfpathlineto{\pgfqpoint{3.169508in}{6.251128in}}%
\pgfpathlineto{\pgfqpoint{3.222552in}{6.251128in}}%
\pgfpathlineto{\pgfqpoint{3.275596in}{6.251128in}}%
\pgfpathlineto{\pgfqpoint{3.328640in}{6.251128in}}%
\pgfpathlineto{\pgfqpoint{3.381684in}{6.598964in}}%
\pgfpathlineto{\pgfqpoint{3.434728in}{6.251128in}}%
\pgfpathlineto{\pgfqpoint{3.487772in}{6.598964in}}%
\pgfpathlineto{\pgfqpoint{3.540816in}{6.572638in}}%
\pgfpathlineto{\pgfqpoint{3.593860in}{6.251128in}}%
\pgfpathlineto{\pgfqpoint{3.646904in}{6.251128in}}%
\pgfpathlineto{\pgfqpoint{3.699948in}{6.251128in}}%
\pgfpathlineto{\pgfqpoint{3.752991in}{6.510498in}}%
\pgfpathlineto{\pgfqpoint{3.806035in}{6.251128in}}%
\pgfpathlineto{\pgfqpoint{3.859079in}{6.572998in}}%
\pgfpathlineto{\pgfqpoint{3.912123in}{6.481634in}}%
\pgfpathlineto{\pgfqpoint{3.965167in}{6.340259in}}%
\pgfpathlineto{\pgfqpoint{4.018211in}{6.251128in}}%
\pgfpathlineto{\pgfqpoint{4.071255in}{6.251128in}}%
\pgfpathlineto{\pgfqpoint{4.124299in}{6.251128in}}%
\pgfpathlineto{\pgfqpoint{4.177343in}{6.251128in}}%
\pgfpathlineto{\pgfqpoint{4.230387in}{6.251128in}}%
\pgfpathlineto{\pgfqpoint{4.283431in}{6.251128in}}%
\pgfpathlineto{\pgfqpoint{4.336475in}{6.889082in}}%
\pgfpathlineto{\pgfqpoint{4.389519in}{6.299258in}}%
\pgfpathlineto{\pgfqpoint{4.442563in}{6.598964in}}%
\pgfpathlineto{\pgfqpoint{4.495607in}{6.633012in}}%
\pgfpathlineto{\pgfqpoint{4.548650in}{6.251128in}}%
\pgfpathlineto{\pgfqpoint{4.601694in}{6.592100in}}%
\pgfpathlineto{\pgfqpoint{4.654738in}{6.251128in}}%
\pgfpathlineto{\pgfqpoint{4.707782in}{6.766257in}}%
\pgfpathlineto{\pgfqpoint{4.760826in}{6.251128in}}%
\pgfpathlineto{\pgfqpoint{4.813870in}{6.251128in}}%
\pgfpathlineto{\pgfqpoint{4.866914in}{6.548881in}}%
\pgfpathlineto{\pgfqpoint{4.919958in}{6.528186in}}%
\pgfpathlineto{\pgfqpoint{4.973002in}{6.251128in}}%
\pgfpathlineto{\pgfqpoint{5.026046in}{6.590550in}}%
\pgfpathlineto{\pgfqpoint{5.079090in}{6.251128in}}%
\pgfpathlineto{\pgfqpoint{5.132134in}{6.598964in}}%
\pgfpathlineto{\pgfqpoint{5.185178in}{6.574350in}}%
\pgfpathlineto{\pgfqpoint{5.238222in}{6.501650in}}%
\pgfpathlineto{\pgfqpoint{5.291266in}{6.251128in}}%
\pgfpathlineto{\pgfqpoint{5.344309in}{6.471024in}}%
\pgfpathlineto{\pgfqpoint{5.397353in}{6.325005in}}%
\pgfpathlineto{\pgfqpoint{5.450397in}{6.770026in}}%
\pgfpathlineto{\pgfqpoint{5.503441in}{6.251128in}}%
\pgfpathlineto{\pgfqpoint{5.556485in}{6.329843in}}%
\pgfpathlineto{\pgfqpoint{5.609529in}{6.514446in}}%
\pgfpathlineto{\pgfqpoint{5.662573in}{6.375963in}}%
\pgfpathlineto{\pgfqpoint{5.715617in}{6.251128in}}%
\pgfpathlineto{\pgfqpoint{5.768661in}{6.913209in}}%
\pgfpathlineto{\pgfqpoint{5.821705in}{6.872997in}}%
\pgfpathlineto{\pgfqpoint{5.874749in}{6.702075in}}%
\pgfpathlineto{\pgfqpoint{5.927793in}{6.598351in}}%
\pgfpathlineto{\pgfqpoint{5.980837in}{6.251128in}}%
\pgfpathlineto{\pgfqpoint{6.033881in}{6.400407in}}%
\pgfpathlineto{\pgfqpoint{6.086925in}{6.251128in}}%
\pgfpathlineto{\pgfqpoint{6.139969in}{6.251128in}}%
\pgfpathlineto{\pgfqpoint{6.193012in}{6.635398in}}%
\pgfpathlineto{\pgfqpoint{6.246056in}{6.691111in}}%
\pgfpathlineto{\pgfqpoint{6.299100in}{6.251128in}}%
\pgfpathlineto{\pgfqpoint{6.352144in}{6.569516in}}%
\pgfpathlineto{\pgfqpoint{6.405188in}{6.364054in}}%
\pgfpathlineto{\pgfqpoint{6.458232in}{6.251128in}}%
\pgfpathlineto{\pgfqpoint{6.511276in}{6.251128in}}%
\pgfpathlineto{\pgfqpoint{6.564320in}{6.251128in}}%
\pgfpathlineto{\pgfqpoint{6.617364in}{6.251128in}}%
\pgfpathlineto{\pgfqpoint{6.670408in}{6.251128in}}%
\pgfpathlineto{\pgfqpoint{6.723452in}{6.251128in}}%
\pgfpathlineto{\pgfqpoint{6.776496in}{6.385890in}}%
\pgfpathlineto{\pgfqpoint{6.829540in}{6.251128in}}%
\pgfpathlineto{\pgfqpoint{6.882584in}{6.780238in}}%
\pgfpathlineto{\pgfqpoint{6.935628in}{6.251128in}}%
\pgfpathlineto{\pgfqpoint{6.988671in}{6.598964in}}%
\pgfpathlineto{\pgfqpoint{7.041715in}{6.251128in}}%
\pgfpathlineto{\pgfqpoint{7.094759in}{6.251128in}}%
\pgfpathlineto{\pgfqpoint{7.147803in}{6.251128in}}%
\pgfpathlineto{\pgfqpoint{7.200847in}{6.729225in}}%
\pgfpathlineto{\pgfqpoint{7.253891in}{6.451777in}}%
\pgfpathlineto{\pgfqpoint{7.306935in}{6.251128in}}%
\pgfpathlineto{\pgfqpoint{7.359979in}{6.470643in}}%
\pgfpathlineto{\pgfqpoint{7.413023in}{6.529690in}}%
\pgfpathlineto{\pgfqpoint{7.466067in}{6.251128in}}%
\pgfpathlineto{\pgfqpoint{7.519111in}{6.251128in}}%
\pgfpathlineto{\pgfqpoint{7.572155in}{6.487962in}}%
\pgfpathlineto{\pgfqpoint{7.625199in}{6.652740in}}%
\pgfpathlineto{\pgfqpoint{7.678243in}{6.251128in}}%
\pgfpathlineto{\pgfqpoint{7.731287in}{6.251128in}}%
\pgfpathlineto{\pgfqpoint{7.784330in}{6.251128in}}%
\pgfpathlineto{\pgfqpoint{7.837374in}{6.251128in}}%
\pgfpathlineto{\pgfqpoint{7.890418in}{6.251128in}}%
\pgfpathlineto{\pgfqpoint{7.943462in}{6.251128in}}%
\pgfpathlineto{\pgfqpoint{7.996506in}{6.251128in}}%
\pgfpathlineto{\pgfqpoint{8.049550in}{6.766484in}}%
\pgfpathlineto{\pgfqpoint{8.102594in}{6.782745in}}%
\pgfpathlineto{\pgfqpoint{8.155638in}{6.251128in}}%
\pgfpathlineto{\pgfqpoint{8.208682in}{6.251128in}}%
\pgfpathlineto{\pgfqpoint{8.261726in}{6.924712in}}%
\pgfpathlineto{\pgfqpoint{8.314770in}{6.876087in}}%
\pgfpathlineto{\pgfqpoint{8.367814in}{6.251128in}}%
\pgfpathlineto{\pgfqpoint{8.420858in}{6.653636in}}%
\pgfpathlineto{\pgfqpoint{8.473902in}{6.251128in}}%
\pgfpathlineto{\pgfqpoint{8.526946in}{6.251128in}}%
\pgfpathlineto{\pgfqpoint{8.579990in}{6.685501in}}%
\pgfpathlineto{\pgfqpoint{8.633033in}{6.520595in}}%
\pgfpathlineto{\pgfqpoint{8.686077in}{6.723687in}}%
\pgfpathlineto{\pgfqpoint{8.739121in}{6.489640in}}%
\pgfpathlineto{\pgfqpoint{8.792165in}{6.251128in}}%
\pgfpathlineto{\pgfqpoint{8.845209in}{6.299108in}}%
\pgfpathlineto{\pgfqpoint{8.898253in}{6.354195in}}%
\pgfpathlineto{\pgfqpoint{8.951297in}{6.251128in}}%
\pgfpathlineto{\pgfqpoint{9.004341in}{6.568668in}}%
\pgfpathlineto{\pgfqpoint{9.057385in}{6.251128in}}%
\pgfpathlineto{\pgfqpoint{9.110429in}{6.251128in}}%
\pgfpathlineto{\pgfqpoint{9.163473in}{6.668425in}}%
\pgfpathlineto{\pgfqpoint{9.216517in}{6.804492in}}%
\pgfpathlineto{\pgfqpoint{9.269561in}{6.628250in}}%
\pgfpathlineto{\pgfqpoint{9.322605in}{6.727450in}}%
\pgfpathlineto{\pgfqpoint{9.375649in}{6.754769in}}%
\pgfpathlineto{\pgfqpoint{9.428692in}{6.251128in}}%
\pgfpathlineto{\pgfqpoint{9.481736in}{6.256205in}}%
\pgfpathlineto{\pgfqpoint{9.534780in}{6.946800in}}%
\pgfpathlineto{\pgfqpoint{9.587824in}{6.621011in}}%
\pgfpathlineto{\pgfqpoint{9.640868in}{6.251128in}}%
\pgfpathlineto{\pgfqpoint{9.693912in}{6.632876in}}%
\pgfpathlineto{\pgfqpoint{9.746956in}{6.779712in}}%
\pgfpathlineto{\pgfqpoint{9.800000in}{6.573653in}}%
\pgfpathlineto{\pgfqpoint{9.800000in}{6.765861in}}%
\pgfpathlineto{\pgfqpoint{9.800000in}{6.765861in}}%
\pgfpathlineto{\pgfqpoint{9.746956in}{6.779712in}}%
\pgfpathlineto{\pgfqpoint{9.693912in}{6.803680in}}%
\pgfpathlineto{\pgfqpoint{9.640868in}{7.137950in}}%
\pgfpathlineto{\pgfqpoint{9.587824in}{6.868462in}}%
\pgfpathlineto{\pgfqpoint{9.534780in}{6.946800in}}%
\pgfpathlineto{\pgfqpoint{9.481736in}{7.580172in}}%
\pgfpathlineto{\pgfqpoint{9.428692in}{7.642472in}}%
\pgfpathlineto{\pgfqpoint{9.375649in}{6.879220in}}%
\pgfpathlineto{\pgfqpoint{9.322605in}{6.883905in}}%
\pgfpathlineto{\pgfqpoint{9.269561in}{6.811485in}}%
\pgfpathlineto{\pgfqpoint{9.216517in}{6.804492in}}%
\pgfpathlineto{\pgfqpoint{9.163473in}{6.770647in}}%
\pgfpathlineto{\pgfqpoint{9.110429in}{7.642472in}}%
\pgfpathlineto{\pgfqpoint{9.057385in}{7.642472in}}%
\pgfpathlineto{\pgfqpoint{9.004341in}{7.212818in}}%
\pgfpathlineto{\pgfqpoint{8.951297in}{7.642472in}}%
\pgfpathlineto{\pgfqpoint{8.898253in}{6.685692in}}%
\pgfpathlineto{\pgfqpoint{8.845209in}{6.642906in}}%
\pgfpathlineto{\pgfqpoint{8.792165in}{6.800188in}}%
\pgfpathlineto{\pgfqpoint{8.739121in}{6.662088in}}%
\pgfpathlineto{\pgfqpoint{8.686077in}{6.723687in}}%
\pgfpathlineto{\pgfqpoint{8.633033in}{6.694774in}}%
\pgfpathlineto{\pgfqpoint{8.579990in}{6.685501in}}%
\pgfpathlineto{\pgfqpoint{8.526946in}{7.642472in}}%
\pgfpathlineto{\pgfqpoint{8.473902in}{7.642472in}}%
\pgfpathlineto{\pgfqpoint{8.420858in}{6.857476in}}%
\pgfpathlineto{\pgfqpoint{8.367814in}{7.642472in}}%
\pgfpathlineto{\pgfqpoint{8.314770in}{6.876087in}}%
\pgfpathlineto{\pgfqpoint{8.261726in}{6.924712in}}%
\pgfpathlineto{\pgfqpoint{8.208682in}{7.642472in}}%
\pgfpathlineto{\pgfqpoint{8.155638in}{7.642472in}}%
\pgfpathlineto{\pgfqpoint{8.102594in}{6.863050in}}%
\pgfpathlineto{\pgfqpoint{8.049550in}{6.853606in}}%
\pgfpathlineto{\pgfqpoint{7.996506in}{7.642472in}}%
\pgfpathlineto{\pgfqpoint{7.943462in}{7.642472in}}%
\pgfpathlineto{\pgfqpoint{7.890418in}{7.642472in}}%
\pgfpathlineto{\pgfqpoint{7.837374in}{7.642472in}}%
\pgfpathlineto{\pgfqpoint{7.784330in}{7.642472in}}%
\pgfpathlineto{\pgfqpoint{7.731287in}{7.642472in}}%
\pgfpathlineto{\pgfqpoint{7.678243in}{6.739129in}}%
\pgfpathlineto{\pgfqpoint{7.625199in}{6.652740in}}%
\pgfpathlineto{\pgfqpoint{7.572155in}{6.593688in}}%
\pgfpathlineto{\pgfqpoint{7.519111in}{7.086960in}}%
\pgfpathlineto{\pgfqpoint{7.466067in}{7.061991in}}%
\pgfpathlineto{\pgfqpoint{7.413023in}{6.684093in}}%
\pgfpathlineto{\pgfqpoint{7.359979in}{6.672189in}}%
\pgfpathlineto{\pgfqpoint{7.306935in}{7.542837in}}%
\pgfpathlineto{\pgfqpoint{7.253891in}{7.350789in}}%
\pgfpathlineto{\pgfqpoint{7.200847in}{6.818171in}}%
\pgfpathlineto{\pgfqpoint{7.147803in}{7.642472in}}%
\pgfpathlineto{\pgfqpoint{7.094759in}{7.642472in}}%
\pgfpathlineto{\pgfqpoint{7.041715in}{7.642472in}}%
\pgfpathlineto{\pgfqpoint{6.988671in}{7.441180in}}%
\pgfpathlineto{\pgfqpoint{6.935628in}{7.642472in}}%
\pgfpathlineto{\pgfqpoint{6.882584in}{6.881359in}}%
\pgfpathlineto{\pgfqpoint{6.829540in}{7.642472in}}%
\pgfpathlineto{\pgfqpoint{6.776496in}{7.554798in}}%
\pgfpathlineto{\pgfqpoint{6.723452in}{7.642472in}}%
\pgfpathlineto{\pgfqpoint{6.670408in}{7.642472in}}%
\pgfpathlineto{\pgfqpoint{6.617364in}{7.642472in}}%
\pgfpathlineto{\pgfqpoint{6.564320in}{7.642472in}}%
\pgfpathlineto{\pgfqpoint{6.511276in}{7.642472in}}%
\pgfpathlineto{\pgfqpoint{6.458232in}{7.158194in}}%
\pgfpathlineto{\pgfqpoint{6.405188in}{6.593987in}}%
\pgfpathlineto{\pgfqpoint{6.352144in}{7.023749in}}%
\pgfpathlineto{\pgfqpoint{6.299100in}{7.642472in}}%
\pgfpathlineto{\pgfqpoint{6.246056in}{6.691111in}}%
\pgfpathlineto{\pgfqpoint{6.193012in}{6.635398in}}%
\pgfpathlineto{\pgfqpoint{6.139969in}{7.642472in}}%
\pgfpathlineto{\pgfqpoint{6.086925in}{7.642472in}}%
\pgfpathlineto{\pgfqpoint{6.033881in}{7.110408in}}%
\pgfpathlineto{\pgfqpoint{5.980837in}{7.642472in}}%
\pgfpathlineto{\pgfqpoint{5.927793in}{6.820928in}}%
\pgfpathlineto{\pgfqpoint{5.874749in}{6.796671in}}%
\pgfpathlineto{\pgfqpoint{5.821705in}{6.872997in}}%
\pgfpathlineto{\pgfqpoint{5.768661in}{6.913209in}}%
\pgfpathlineto{\pgfqpoint{5.715617in}{7.642472in}}%
\pgfpathlineto{\pgfqpoint{5.662573in}{6.925790in}}%
\pgfpathlineto{\pgfqpoint{5.609529in}{6.874196in}}%
\pgfpathlineto{\pgfqpoint{5.556485in}{7.281307in}}%
\pgfpathlineto{\pgfqpoint{5.503441in}{7.642472in}}%
\pgfpathlineto{\pgfqpoint{5.450397in}{6.877918in}}%
\pgfpathlineto{\pgfqpoint{5.397353in}{6.748316in}}%
\pgfpathlineto{\pgfqpoint{5.344309in}{6.838988in}}%
\pgfpathlineto{\pgfqpoint{5.291266in}{7.642472in}}%
\pgfpathlineto{\pgfqpoint{5.238222in}{6.666672in}}%
\pgfpathlineto{\pgfqpoint{5.185178in}{6.670254in}}%
\pgfpathlineto{\pgfqpoint{5.132134in}{6.692849in}}%
\pgfpathlineto{\pgfqpoint{5.079090in}{7.642472in}}%
\pgfpathlineto{\pgfqpoint{5.026046in}{6.790800in}}%
\pgfpathlineto{\pgfqpoint{4.973002in}{7.642472in}}%
\pgfpathlineto{\pgfqpoint{4.919958in}{6.671298in}}%
\pgfpathlineto{\pgfqpoint{4.866914in}{6.684929in}}%
\pgfpathlineto{\pgfqpoint{4.813870in}{7.642472in}}%
\pgfpathlineto{\pgfqpoint{4.760826in}{7.378709in}}%
\pgfpathlineto{\pgfqpoint{4.707782in}{6.766257in}}%
\pgfpathlineto{\pgfqpoint{4.654738in}{7.642472in}}%
\pgfpathlineto{\pgfqpoint{4.601694in}{6.827112in}}%
\pgfpathlineto{\pgfqpoint{4.548650in}{7.353421in}}%
\pgfpathlineto{\pgfqpoint{4.495607in}{6.883010in}}%
\pgfpathlineto{\pgfqpoint{4.442563in}{7.506419in}}%
\pgfpathlineto{\pgfqpoint{4.389519in}{7.554209in}}%
\pgfpathlineto{\pgfqpoint{4.336475in}{6.889082in}}%
\pgfpathlineto{\pgfqpoint{4.283431in}{7.642472in}}%
\pgfpathlineto{\pgfqpoint{4.230387in}{7.642472in}}%
\pgfpathlineto{\pgfqpoint{4.177343in}{7.642472in}}%
\pgfpathlineto{\pgfqpoint{4.124299in}{7.539721in}}%
\pgfpathlineto{\pgfqpoint{4.071255in}{7.642472in}}%
\pgfpathlineto{\pgfqpoint{4.018211in}{7.642472in}}%
\pgfpathlineto{\pgfqpoint{3.965167in}{6.991885in}}%
\pgfpathlineto{\pgfqpoint{3.912123in}{7.302849in}}%
\pgfpathlineto{\pgfqpoint{3.859079in}{7.341795in}}%
\pgfpathlineto{\pgfqpoint{3.806035in}{7.642472in}}%
\pgfpathlineto{\pgfqpoint{3.752991in}{7.311866in}}%
\pgfpathlineto{\pgfqpoint{3.699948in}{7.642472in}}%
\pgfpathlineto{\pgfqpoint{3.646904in}{7.642472in}}%
\pgfpathlineto{\pgfqpoint{3.593860in}{7.167623in}}%
\pgfpathlineto{\pgfqpoint{3.540816in}{6.738883in}}%
\pgfpathlineto{\pgfqpoint{3.487772in}{6.972573in}}%
\pgfpathlineto{\pgfqpoint{3.434728in}{7.642472in}}%
\pgfpathlineto{\pgfqpoint{3.381684in}{7.361194in}}%
\pgfpathlineto{\pgfqpoint{3.328640in}{7.642472in}}%
\pgfpathlineto{\pgfqpoint{3.275596in}{7.642472in}}%
\pgfpathlineto{\pgfqpoint{3.222552in}{7.642472in}}%
\pgfpathlineto{\pgfqpoint{3.169508in}{7.642472in}}%
\pgfpathlineto{\pgfqpoint{3.116464in}{7.518262in}}%
\pgfpathlineto{\pgfqpoint{3.063420in}{6.841623in}}%
\pgfpathlineto{\pgfqpoint{3.010376in}{7.642472in}}%
\pgfpathlineto{\pgfqpoint{2.957332in}{6.836531in}}%
\pgfpathlineto{\pgfqpoint{2.904288in}{7.642472in}}%
\pgfpathlineto{\pgfqpoint{2.851245in}{7.173688in}}%
\pgfpathlineto{\pgfqpoint{2.798201in}{7.642472in}}%
\pgfpathlineto{\pgfqpoint{2.745157in}{7.642472in}}%
\pgfpathlineto{\pgfqpoint{2.692113in}{7.642472in}}%
\pgfpathlineto{\pgfqpoint{2.639069in}{7.642472in}}%
\pgfpathlineto{\pgfqpoint{2.586025in}{7.642472in}}%
\pgfpathlineto{\pgfqpoint{2.532981in}{7.642472in}}%
\pgfpathlineto{\pgfqpoint{2.479937in}{7.449769in}}%
\pgfpathlineto{\pgfqpoint{2.426893in}{6.606651in}}%
\pgfpathlineto{\pgfqpoint{2.373849in}{6.676762in}}%
\pgfpathlineto{\pgfqpoint{2.320805in}{7.642472in}}%
\pgfpathlineto{\pgfqpoint{2.267761in}{7.419218in}}%
\pgfpathlineto{\pgfqpoint{2.214717in}{6.753348in}}%
\pgfpathlineto{\pgfqpoint{2.161673in}{7.642472in}}%
\pgfpathlineto{\pgfqpoint{2.108629in}{7.339659in}}%
\pgfpathlineto{\pgfqpoint{2.055586in}{7.642472in}}%
\pgfpathlineto{\pgfqpoint{2.002542in}{6.857739in}}%
\pgfpathlineto{\pgfqpoint{1.949498in}{7.642472in}}%
\pgfpathlineto{\pgfqpoint{1.896454in}{6.883587in}}%
\pgfpathlineto{\pgfqpoint{1.843410in}{6.911654in}}%
\pgfpathlineto{\pgfqpoint{1.790366in}{7.642472in}}%
\pgfpathlineto{\pgfqpoint{1.737322in}{7.536394in}}%
\pgfpathlineto{\pgfqpoint{1.684278in}{7.452967in}}%
\pgfpathlineto{\pgfqpoint{1.631234in}{6.920473in}}%
\pgfpathlineto{\pgfqpoint{1.578190in}{7.642472in}}%
\pgfpathlineto{\pgfqpoint{1.525146in}{7.642472in}}%
\pgfpathlineto{\pgfqpoint{1.472102in}{7.642472in}}%
\pgfpathlineto{\pgfqpoint{1.419058in}{7.642472in}}%
\pgfpathlineto{\pgfqpoint{1.366014in}{7.642472in}}%
\pgfpathlineto{\pgfqpoint{1.312970in}{7.018171in}}%
\pgfpathlineto{\pgfqpoint{1.259927in}{6.752271in}}%
\pgfpathlineto{\pgfqpoint{1.206883in}{7.113972in}}%
\pgfpathlineto{\pgfqpoint{1.153839in}{7.642472in}}%
\pgfpathlineto{\pgfqpoint{1.100795in}{7.630782in}}%
\pgfpathlineto{\pgfqpoint{1.047751in}{7.642472in}}%
\pgfpathlineto{\pgfqpoint{0.994707in}{7.398889in}}%
\pgfpathlineto{\pgfqpoint{0.941663in}{7.642472in}}%
\pgfpathlineto{\pgfqpoint{0.941663in}{7.642472in}}%
\pgfpathclose%
\pgfusepath{stroke,fill}%
}%
\begin{pgfscope}%
\pgfsys@transformshift{0.000000in}{0.000000in}%
\pgfsys@useobject{currentmarker}{}%
\end{pgfscope}%
\end{pgfscope}%
\begin{pgfscope}%
\pgfsetrectcap%
\pgfsetmiterjoin%
\pgfsetlinewidth{0.803000pt}%
\definecolor{currentstroke}{rgb}{0.000000,0.000000,0.000000}%
\pgfsetstrokecolor{currentstroke}%
\pgfsetdash{}{0pt}%
\pgfpathmoveto{\pgfqpoint{0.941663in}{4.334375in}}%
\pgfpathlineto{\pgfqpoint{0.941663in}{7.800000in}}%
\pgfusepath{stroke}%
\end{pgfscope}%
\begin{pgfscope}%
\pgfsetrectcap%
\pgfsetmiterjoin%
\pgfsetlinewidth{0.803000pt}%
\definecolor{currentstroke}{rgb}{0.000000,0.000000,0.000000}%
\pgfsetstrokecolor{currentstroke}%
\pgfsetdash{}{0pt}%
\pgfpathmoveto{\pgfqpoint{9.800000in}{4.334375in}}%
\pgfpathlineto{\pgfqpoint{9.800000in}{7.800000in}}%
\pgfusepath{stroke}%
\end{pgfscope}%
\begin{pgfscope}%
\pgfsetrectcap%
\pgfsetmiterjoin%
\pgfsetlinewidth{0.803000pt}%
\definecolor{currentstroke}{rgb}{0.000000,0.000000,0.000000}%
\pgfsetstrokecolor{currentstroke}%
\pgfsetdash{}{0pt}%
\pgfpathmoveto{\pgfqpoint{0.941663in}{4.334375in}}%
\pgfpathlineto{\pgfqpoint{9.800000in}{4.334375in}}%
\pgfusepath{stroke}%
\end{pgfscope}%
\begin{pgfscope}%
\pgfsetrectcap%
\pgfsetmiterjoin%
\pgfsetlinewidth{0.803000pt}%
\definecolor{currentstroke}{rgb}{0.000000,0.000000,0.000000}%
\pgfsetstrokecolor{currentstroke}%
\pgfsetdash{}{0pt}%
\pgfpathmoveto{\pgfqpoint{0.941663in}{7.800000in}}%
\pgfpathlineto{\pgfqpoint{9.800000in}{7.800000in}}%
\pgfusepath{stroke}%
\end{pgfscope}%
\begin{pgfscope}%
\pgfpathrectangle{\pgfqpoint{0.941663in}{4.334375in}}{\pgfqpoint{8.858337in}{3.465625in}}%
\pgfusepath{clip}%
\pgfsetbuttcap%
\pgfsetroundjoin%
\pgfsetlinewidth{1.505625pt}%
\definecolor{currentstroke}{rgb}{0.000000,0.000000,0.000000}%
\pgfsetstrokecolor{currentstroke}%
\pgfsetdash{{5.550000pt}{2.400000pt}}{0.000000pt}%
\pgfpathmoveto{\pgfqpoint{0.941663in}{6.693424in}}%
\pgfpathlineto{\pgfqpoint{0.994707in}{6.703217in}}%
\pgfpathlineto{\pgfqpoint{1.047751in}{6.689375in}}%
\pgfpathlineto{\pgfqpoint{1.100795in}{6.646904in}}%
\pgfpathlineto{\pgfqpoint{1.153839in}{6.644931in}}%
\pgfpathlineto{\pgfqpoint{1.206883in}{6.685908in}}%
\pgfpathlineto{\pgfqpoint{1.259927in}{6.566729in}}%
\pgfpathlineto{\pgfqpoint{1.312970in}{6.635332in}}%
\pgfpathlineto{\pgfqpoint{1.366014in}{6.580587in}}%
\pgfpathlineto{\pgfqpoint{1.419058in}{6.654803in}}%
\pgfpathlineto{\pgfqpoint{1.472102in}{6.660166in}}%
\pgfpathlineto{\pgfqpoint{1.525146in}{6.740529in}}%
\pgfpathlineto{\pgfqpoint{1.578190in}{6.717659in}}%
\pgfpathlineto{\pgfqpoint{1.631234in}{6.788071in}}%
\pgfpathlineto{\pgfqpoint{1.684278in}{6.772987in}}%
\pgfpathlineto{\pgfqpoint{1.737322in}{6.776566in}}%
\pgfpathlineto{\pgfqpoint{1.790366in}{6.865064in}}%
\pgfpathlineto{\pgfqpoint{1.843410in}{6.911654in}}%
\pgfpathlineto{\pgfqpoint{1.896454in}{6.883587in}}%
\pgfpathlineto{\pgfqpoint{1.949498in}{6.838375in}}%
\pgfpathlineto{\pgfqpoint{2.002542in}{6.857739in}}%
\pgfpathlineto{\pgfqpoint{2.055586in}{6.849181in}}%
\pgfpathlineto{\pgfqpoint{2.108629in}{6.832484in}}%
\pgfpathlineto{\pgfqpoint{2.161673in}{6.738829in}}%
\pgfpathlineto{\pgfqpoint{2.214717in}{6.753348in}}%
\pgfpathlineto{\pgfqpoint{2.267761in}{6.710554in}}%
\pgfpathlineto{\pgfqpoint{2.320805in}{6.669421in}}%
\pgfpathlineto{\pgfqpoint{2.373849in}{6.644384in}}%
\pgfpathlineto{\pgfqpoint{2.426893in}{6.606651in}}%
\pgfpathlineto{\pgfqpoint{2.479937in}{6.596638in}}%
\pgfpathlineto{\pgfqpoint{2.532981in}{6.645096in}}%
\pgfpathlineto{\pgfqpoint{2.586025in}{6.623285in}}%
\pgfpathlineto{\pgfqpoint{2.639069in}{6.618792in}}%
\pgfpathlineto{\pgfqpoint{2.692113in}{6.588590in}}%
\pgfpathlineto{\pgfqpoint{2.745157in}{6.659513in}}%
\pgfpathlineto{\pgfqpoint{2.798201in}{6.701909in}}%
\pgfpathlineto{\pgfqpoint{2.851245in}{6.716040in}}%
\pgfpathlineto{\pgfqpoint{2.904288in}{6.748816in}}%
\pgfpathlineto{\pgfqpoint{2.957332in}{6.836531in}}%
\pgfpathlineto{\pgfqpoint{3.010376in}{6.834580in}}%
\pgfpathlineto{\pgfqpoint{3.063420in}{6.815911in}}%
\pgfpathlineto{\pgfqpoint{3.116464in}{6.836791in}}%
\pgfpathlineto{\pgfqpoint{3.169508in}{6.866400in}}%
\pgfpathlineto{\pgfqpoint{3.222552in}{6.896923in}}%
\pgfpathlineto{\pgfqpoint{3.275596in}{6.813063in}}%
\pgfpathlineto{\pgfqpoint{3.328640in}{6.871638in}}%
\pgfpathlineto{\pgfqpoint{3.381684in}{6.788610in}}%
\pgfpathlineto{\pgfqpoint{3.434728in}{6.724023in}}%
\pgfpathlineto{\pgfqpoint{3.487772in}{6.748160in}}%
\pgfpathlineto{\pgfqpoint{3.540816in}{6.738883in}}%
\pgfpathlineto{\pgfqpoint{3.593860in}{6.664878in}}%
\pgfpathlineto{\pgfqpoint{3.646904in}{6.662324in}}%
\pgfpathlineto{\pgfqpoint{3.699948in}{6.618474in}}%
\pgfpathlineto{\pgfqpoint{3.752991in}{6.616195in}}%
\pgfpathlineto{\pgfqpoint{3.806035in}{6.658725in}}%
\pgfpathlineto{\pgfqpoint{3.859079in}{6.646123in}}%
\pgfpathlineto{\pgfqpoint{3.912123in}{6.607177in}}%
\pgfpathlineto{\pgfqpoint{3.965167in}{6.735611in}}%
\pgfpathlineto{\pgfqpoint{4.018211in}{6.664699in}}%
\pgfpathlineto{\pgfqpoint{4.071255in}{6.746790in}}%
\pgfpathlineto{\pgfqpoint{4.124299in}{6.772208in}}%
\pgfpathlineto{\pgfqpoint{4.177343in}{6.790679in}}%
\pgfpathlineto{\pgfqpoint{4.230387in}{6.828787in}}%
\pgfpathlineto{\pgfqpoint{4.283431in}{6.844506in}}%
\pgfpathlineto{\pgfqpoint{4.336475in}{6.889082in}}%
\pgfpathlineto{\pgfqpoint{4.389519in}{6.858537in}}%
\pgfpathlineto{\pgfqpoint{4.442563in}{6.847591in}}%
\pgfpathlineto{\pgfqpoint{4.495607in}{6.883010in}}%
\pgfpathlineto{\pgfqpoint{4.548650in}{6.899833in}}%
\pgfpathlineto{\pgfqpoint{4.601694in}{6.827112in}}%
\pgfpathlineto{\pgfqpoint{4.654738in}{6.846117in}}%
\pgfpathlineto{\pgfqpoint{4.707782in}{6.766257in}}%
\pgfpathlineto{\pgfqpoint{4.760826in}{6.775170in}}%
\pgfpathlineto{\pgfqpoint{4.813870in}{6.726380in}}%
\pgfpathlineto{\pgfqpoint{4.866914in}{6.684929in}}%
\pgfpathlineto{\pgfqpoint{4.919958in}{6.671298in}}%
\pgfpathlineto{\pgfqpoint{4.973002in}{6.607378in}}%
\pgfpathlineto{\pgfqpoint{5.026046in}{6.622809in}}%
\pgfpathlineto{\pgfqpoint{5.079090in}{6.578918in}}%
\pgfpathlineto{\pgfqpoint{5.132134in}{6.582269in}}%
\pgfpathlineto{\pgfqpoint{5.185178in}{6.670254in}}%
\pgfpathlineto{\pgfqpoint{5.238222in}{6.666672in}}%
\pgfpathlineto{\pgfqpoint{5.291266in}{6.726904in}}%
\pgfpathlineto{\pgfqpoint{5.344309in}{6.737403in}}%
\pgfpathlineto{\pgfqpoint{5.397353in}{6.748316in}}%
\pgfpathlineto{\pgfqpoint{5.450397in}{6.877918in}}%
\pgfpathlineto{\pgfqpoint{5.503441in}{6.868085in}}%
\pgfpathlineto{\pgfqpoint{5.556485in}{6.860891in}}%
\pgfpathlineto{\pgfqpoint{5.609529in}{6.874196in}}%
\pgfpathlineto{\pgfqpoint{5.662573in}{6.925790in}}%
\pgfpathlineto{\pgfqpoint{5.715617in}{6.873877in}}%
\pgfpathlineto{\pgfqpoint{5.768661in}{6.913209in}}%
\pgfpathlineto{\pgfqpoint{5.821705in}{6.872997in}}%
\pgfpathlineto{\pgfqpoint{5.874749in}{6.796671in}}%
\pgfpathlineto{\pgfqpoint{5.927793in}{6.820928in}}%
\pgfpathlineto{\pgfqpoint{5.980837in}{6.797521in}}%
\pgfpathlineto{\pgfqpoint{6.033881in}{6.748313in}}%
\pgfpathlineto{\pgfqpoint{6.086925in}{6.726383in}}%
\pgfpathlineto{\pgfqpoint{6.139969in}{6.707909in}}%
\pgfpathlineto{\pgfqpoint{6.193012in}{6.635398in}}%
\pgfpathlineto{\pgfqpoint{6.246056in}{6.691111in}}%
\pgfpathlineto{\pgfqpoint{6.299100in}{6.628412in}}%
\pgfpathlineto{\pgfqpoint{6.352144in}{6.647199in}}%
\pgfpathlineto{\pgfqpoint{6.405188in}{6.593987in}}%
\pgfpathlineto{\pgfqpoint{6.458232in}{6.680253in}}%
\pgfpathlineto{\pgfqpoint{6.511276in}{6.698006in}}%
\pgfpathlineto{\pgfqpoint{6.564320in}{6.694356in}}%
\pgfpathlineto{\pgfqpoint{6.617364in}{6.761313in}}%
\pgfpathlineto{\pgfqpoint{6.670408in}{6.791255in}}%
\pgfpathlineto{\pgfqpoint{6.723452in}{6.837336in}}%
\pgfpathlineto{\pgfqpoint{6.776496in}{6.859127in}}%
\pgfpathlineto{\pgfqpoint{6.829540in}{6.849303in}}%
\pgfpathlineto{\pgfqpoint{6.882584in}{6.881359in}}%
\pgfpathlineto{\pgfqpoint{6.935628in}{6.842957in}}%
\pgfpathlineto{\pgfqpoint{6.988671in}{6.886633in}}%
\pgfpathlineto{\pgfqpoint{7.041715in}{6.862219in}}%
\pgfpathlineto{\pgfqpoint{7.094759in}{6.885270in}}%
\pgfpathlineto{\pgfqpoint{7.147803in}{6.867065in}}%
\pgfpathlineto{\pgfqpoint{7.200847in}{6.818171in}}%
\pgfpathlineto{\pgfqpoint{7.253891in}{6.719285in}}%
\pgfpathlineto{\pgfqpoint{7.306935in}{6.747320in}}%
\pgfpathlineto{\pgfqpoint{7.359979in}{6.672189in}}%
\pgfpathlineto{\pgfqpoint{7.413023in}{6.684093in}}%
\pgfpathlineto{\pgfqpoint{7.466067in}{6.654137in}}%
\pgfpathlineto{\pgfqpoint{7.519111in}{6.606265in}}%
\pgfpathlineto{\pgfqpoint{7.572155in}{6.593688in}}%
\pgfpathlineto{\pgfqpoint{7.625199in}{6.652740in}}%
\pgfpathlineto{\pgfqpoint{7.678243in}{6.697655in}}%
\pgfpathlineto{\pgfqpoint{7.731287in}{6.660188in}}%
\pgfpathlineto{\pgfqpoint{7.784330in}{6.696220in}}%
\pgfpathlineto{\pgfqpoint{7.837374in}{6.704471in}}%
\pgfpathlineto{\pgfqpoint{7.890418in}{6.745774in}}%
\pgfpathlineto{\pgfqpoint{7.943462in}{6.743520in}}%
\pgfpathlineto{\pgfqpoint{7.996506in}{6.822223in}}%
\pgfpathlineto{\pgfqpoint{8.049550in}{6.853606in}}%
\pgfpathlineto{\pgfqpoint{8.102594in}{6.863050in}}%
\pgfpathlineto{\pgfqpoint{8.155638in}{6.888859in}}%
\pgfpathlineto{\pgfqpoint{8.208682in}{6.883000in}}%
\pgfpathlineto{\pgfqpoint{8.261726in}{6.924712in}}%
\pgfpathlineto{\pgfqpoint{8.314770in}{6.876087in}}%
\pgfpathlineto{\pgfqpoint{8.367814in}{6.865417in}}%
\pgfpathlineto{\pgfqpoint{8.420858in}{6.857476in}}%
\pgfpathlineto{\pgfqpoint{8.473902in}{6.806195in}}%
\pgfpathlineto{\pgfqpoint{8.526946in}{6.778054in}}%
\pgfpathlineto{\pgfqpoint{8.579990in}{6.685501in}}%
\pgfpathlineto{\pgfqpoint{8.633033in}{6.694774in}}%
\pgfpathlineto{\pgfqpoint{8.686077in}{6.723687in}}%
\pgfpathlineto{\pgfqpoint{8.739121in}{6.662088in}}%
\pgfpathlineto{\pgfqpoint{8.792165in}{6.664742in}}%
\pgfpathlineto{\pgfqpoint{8.845209in}{6.642906in}}%
\pgfpathlineto{\pgfqpoint{8.898253in}{6.685692in}}%
\pgfpathlineto{\pgfqpoint{8.951297in}{6.629260in}}%
\pgfpathlineto{\pgfqpoint{9.004341in}{6.648198in}}%
\pgfpathlineto{\pgfqpoint{9.057385in}{6.713621in}}%
\pgfpathlineto{\pgfqpoint{9.110429in}{6.748714in}}%
\pgfpathlineto{\pgfqpoint{9.163473in}{6.770647in}}%
\pgfpathlineto{\pgfqpoint{9.216517in}{6.804492in}}%
\pgfpathlineto{\pgfqpoint{9.269561in}{6.811485in}}%
\pgfpathlineto{\pgfqpoint{9.322605in}{6.883905in}}%
\pgfpathlineto{\pgfqpoint{9.375649in}{6.879220in}}%
\pgfpathlineto{\pgfqpoint{9.428692in}{6.905077in}}%
\pgfpathlineto{\pgfqpoint{9.481736in}{6.884500in}}%
\pgfpathlineto{\pgfqpoint{9.534780in}{6.946800in}}%
\pgfpathlineto{\pgfqpoint{9.587824in}{6.868462in}}%
\pgfpathlineto{\pgfqpoint{9.640868in}{6.864012in}}%
\pgfpathlineto{\pgfqpoint{9.693912in}{6.803680in}}%
\pgfpathlineto{\pgfqpoint{9.746956in}{6.779712in}}%
\pgfpathlineto{\pgfqpoint{9.800000in}{6.765861in}}%
\pgfpathlineto{\pgfqpoint{9.800000in}{6.765861in}}%
\pgfusepath{stroke}%
\end{pgfscope}%
\begin{pgfscope}%
\pgfsetbuttcap%
\pgfsetmiterjoin%
\definecolor{currentfill}{rgb}{1.000000,1.000000,1.000000}%
\pgfsetfillcolor{currentfill}%
\pgfsetlinewidth{1.003750pt}%
\definecolor{currentstroke}{rgb}{0.000000,0.000000,0.000000}%
\pgfsetstrokecolor{currentstroke}%
\pgfsetdash{}{0pt}%
\pgfpathmoveto{\pgfqpoint{1.017884in}{7.466516in}}%
\pgfpathlineto{\pgfqpoint{1.304733in}{7.466516in}}%
\pgfpathlineto{\pgfqpoint{1.304733in}{7.779293in}}%
\pgfpathlineto{\pgfqpoint{1.017884in}{7.779293in}}%
\pgfpathlineto{\pgfqpoint{1.017884in}{7.466516in}}%
\pgfpathclose%
\pgfusepath{stroke,fill}%
\end{pgfscope}%
\begin{pgfscope}%
\definecolor{textcolor}{rgb}{0.000000,0.000000,0.000000}%
\pgfsetstrokecolor{textcolor}%
\pgfsetfillcolor{textcolor}%
\pgftext[x=1.074273in,y=7.572904in,left,base]{\color{textcolor}{\rmfamily\fontsize{14.000000}{16.800000}\selectfont\catcode`\^=\active\def^{\ifmmode\sp\else\^{}\fi}\catcode`\%=\active\def%{\%}a)}}%
\end{pgfscope}%
\begin{pgfscope}%
\pgfsetbuttcap%
\pgfsetmiterjoin%
\definecolor{currentfill}{rgb}{1.000000,1.000000,1.000000}%
\pgfsetfillcolor{currentfill}%
\pgfsetlinewidth{0.000000pt}%
\definecolor{currentstroke}{rgb}{0.000000,0.000000,0.000000}%
\pgfsetstrokecolor{currentstroke}%
\pgfsetstrokeopacity{0.000000}%
\pgfsetdash{}{0pt}%
\pgfpathmoveto{\pgfqpoint{0.941663in}{0.670138in}}%
\pgfpathlineto{\pgfqpoint{9.800000in}{0.670138in}}%
\pgfpathlineto{\pgfqpoint{9.800000in}{4.135763in}}%
\pgfpathlineto{\pgfqpoint{0.941663in}{4.135763in}}%
\pgfpathlineto{\pgfqpoint{0.941663in}{0.670138in}}%
\pgfpathclose%
\pgfusepath{fill}%
\end{pgfscope}%
\begin{pgfscope}%
\pgfpathrectangle{\pgfqpoint{0.941663in}{0.670138in}}{\pgfqpoint{8.858337in}{3.465625in}}%
\pgfusepath{clip}%
\pgfsetrectcap%
\pgfsetroundjoin%
\pgfsetlinewidth{0.803000pt}%
\definecolor{currentstroke}{rgb}{0.690196,0.690196,0.690196}%
\pgfsetstrokecolor{currentstroke}%
\pgfsetdash{}{0pt}%
\pgfpathmoveto{\pgfqpoint{0.941663in}{0.670138in}}%
\pgfpathlineto{\pgfqpoint{0.941663in}{4.135763in}}%
\pgfusepath{stroke}%
\end{pgfscope}%
\begin{pgfscope}%
\pgfsetbuttcap%
\pgfsetroundjoin%
\definecolor{currentfill}{rgb}{0.000000,0.000000,0.000000}%
\pgfsetfillcolor{currentfill}%
\pgfsetlinewidth{0.803000pt}%
\definecolor{currentstroke}{rgb}{0.000000,0.000000,0.000000}%
\pgfsetstrokecolor{currentstroke}%
\pgfsetdash{}{0pt}%
\pgfsys@defobject{currentmarker}{\pgfqpoint{0.000000in}{-0.048611in}}{\pgfqpoint{0.000000in}{0.000000in}}{%
\pgfpathmoveto{\pgfqpoint{0.000000in}{0.000000in}}%
\pgfpathlineto{\pgfqpoint{0.000000in}{-0.048611in}}%
\pgfusepath{stroke,fill}%
}%
\begin{pgfscope}%
\pgfsys@transformshift{0.941663in}{0.670138in}%
\pgfsys@useobject{currentmarker}{}%
\end{pgfscope}%
\end{pgfscope}%
\begin{pgfscope}%
\definecolor{textcolor}{rgb}{0.000000,0.000000,0.000000}%
\pgfsetstrokecolor{textcolor}%
\pgfsetfillcolor{textcolor}%
\pgftext[x=0.941663in,y=0.572916in,,top]{\color{textcolor}{\rmfamily\fontsize{14.000000}{16.800000}\selectfont\catcode`\^=\active\def^{\ifmmode\sp\else\^{}\fi}\catcode`\%=\active\def%{\%}$\mathdefault{0}$}}%
\end{pgfscope}%
\begin{pgfscope}%
\pgfpathrectangle{\pgfqpoint{0.941663in}{0.670138in}}{\pgfqpoint{8.858337in}{3.465625in}}%
\pgfusepath{clip}%
\pgfsetrectcap%
\pgfsetroundjoin%
\pgfsetlinewidth{0.803000pt}%
\definecolor{currentstroke}{rgb}{0.690196,0.690196,0.690196}%
\pgfsetstrokecolor{currentstroke}%
\pgfsetdash{}{0pt}%
\pgfpathmoveto{\pgfqpoint{2.002542in}{0.670138in}}%
\pgfpathlineto{\pgfqpoint{2.002542in}{4.135763in}}%
\pgfusepath{stroke}%
\end{pgfscope}%
\begin{pgfscope}%
\pgfsetbuttcap%
\pgfsetroundjoin%
\definecolor{currentfill}{rgb}{0.000000,0.000000,0.000000}%
\pgfsetfillcolor{currentfill}%
\pgfsetlinewidth{0.803000pt}%
\definecolor{currentstroke}{rgb}{0.000000,0.000000,0.000000}%
\pgfsetstrokecolor{currentstroke}%
\pgfsetdash{}{0pt}%
\pgfsys@defobject{currentmarker}{\pgfqpoint{0.000000in}{-0.048611in}}{\pgfqpoint{0.000000in}{0.000000in}}{%
\pgfpathmoveto{\pgfqpoint{0.000000in}{0.000000in}}%
\pgfpathlineto{\pgfqpoint{0.000000in}{-0.048611in}}%
\pgfusepath{stroke,fill}%
}%
\begin{pgfscope}%
\pgfsys@transformshift{2.002542in}{0.670138in}%
\pgfsys@useobject{currentmarker}{}%
\end{pgfscope}%
\end{pgfscope}%
\begin{pgfscope}%
\definecolor{textcolor}{rgb}{0.000000,0.000000,0.000000}%
\pgfsetstrokecolor{textcolor}%
\pgfsetfillcolor{textcolor}%
\pgftext[x=2.002542in,y=0.572916in,,top]{\color{textcolor}{\rmfamily\fontsize{14.000000}{16.800000}\selectfont\catcode`\^=\active\def^{\ifmmode\sp\else\^{}\fi}\catcode`\%=\active\def%{\%}$\mathdefault{20}$}}%
\end{pgfscope}%
\begin{pgfscope}%
\pgfpathrectangle{\pgfqpoint{0.941663in}{0.670138in}}{\pgfqpoint{8.858337in}{3.465625in}}%
\pgfusepath{clip}%
\pgfsetrectcap%
\pgfsetroundjoin%
\pgfsetlinewidth{0.803000pt}%
\definecolor{currentstroke}{rgb}{0.690196,0.690196,0.690196}%
\pgfsetstrokecolor{currentstroke}%
\pgfsetdash{}{0pt}%
\pgfpathmoveto{\pgfqpoint{3.063420in}{0.670138in}}%
\pgfpathlineto{\pgfqpoint{3.063420in}{4.135763in}}%
\pgfusepath{stroke}%
\end{pgfscope}%
\begin{pgfscope}%
\pgfsetbuttcap%
\pgfsetroundjoin%
\definecolor{currentfill}{rgb}{0.000000,0.000000,0.000000}%
\pgfsetfillcolor{currentfill}%
\pgfsetlinewidth{0.803000pt}%
\definecolor{currentstroke}{rgb}{0.000000,0.000000,0.000000}%
\pgfsetstrokecolor{currentstroke}%
\pgfsetdash{}{0pt}%
\pgfsys@defobject{currentmarker}{\pgfqpoint{0.000000in}{-0.048611in}}{\pgfqpoint{0.000000in}{0.000000in}}{%
\pgfpathmoveto{\pgfqpoint{0.000000in}{0.000000in}}%
\pgfpathlineto{\pgfqpoint{0.000000in}{-0.048611in}}%
\pgfusepath{stroke,fill}%
}%
\begin{pgfscope}%
\pgfsys@transformshift{3.063420in}{0.670138in}%
\pgfsys@useobject{currentmarker}{}%
\end{pgfscope}%
\end{pgfscope}%
\begin{pgfscope}%
\definecolor{textcolor}{rgb}{0.000000,0.000000,0.000000}%
\pgfsetstrokecolor{textcolor}%
\pgfsetfillcolor{textcolor}%
\pgftext[x=3.063420in,y=0.572916in,,top]{\color{textcolor}{\rmfamily\fontsize{14.000000}{16.800000}\selectfont\catcode`\^=\active\def^{\ifmmode\sp\else\^{}\fi}\catcode`\%=\active\def%{\%}$\mathdefault{40}$}}%
\end{pgfscope}%
\begin{pgfscope}%
\pgfpathrectangle{\pgfqpoint{0.941663in}{0.670138in}}{\pgfqpoint{8.858337in}{3.465625in}}%
\pgfusepath{clip}%
\pgfsetrectcap%
\pgfsetroundjoin%
\pgfsetlinewidth{0.803000pt}%
\definecolor{currentstroke}{rgb}{0.690196,0.690196,0.690196}%
\pgfsetstrokecolor{currentstroke}%
\pgfsetdash{}{0pt}%
\pgfpathmoveto{\pgfqpoint{4.124299in}{0.670138in}}%
\pgfpathlineto{\pgfqpoint{4.124299in}{4.135763in}}%
\pgfusepath{stroke}%
\end{pgfscope}%
\begin{pgfscope}%
\pgfsetbuttcap%
\pgfsetroundjoin%
\definecolor{currentfill}{rgb}{0.000000,0.000000,0.000000}%
\pgfsetfillcolor{currentfill}%
\pgfsetlinewidth{0.803000pt}%
\definecolor{currentstroke}{rgb}{0.000000,0.000000,0.000000}%
\pgfsetstrokecolor{currentstroke}%
\pgfsetdash{}{0pt}%
\pgfsys@defobject{currentmarker}{\pgfqpoint{0.000000in}{-0.048611in}}{\pgfqpoint{0.000000in}{0.000000in}}{%
\pgfpathmoveto{\pgfqpoint{0.000000in}{0.000000in}}%
\pgfpathlineto{\pgfqpoint{0.000000in}{-0.048611in}}%
\pgfusepath{stroke,fill}%
}%
\begin{pgfscope}%
\pgfsys@transformshift{4.124299in}{0.670138in}%
\pgfsys@useobject{currentmarker}{}%
\end{pgfscope}%
\end{pgfscope}%
\begin{pgfscope}%
\definecolor{textcolor}{rgb}{0.000000,0.000000,0.000000}%
\pgfsetstrokecolor{textcolor}%
\pgfsetfillcolor{textcolor}%
\pgftext[x=4.124299in,y=0.572916in,,top]{\color{textcolor}{\rmfamily\fontsize{14.000000}{16.800000}\selectfont\catcode`\^=\active\def^{\ifmmode\sp\else\^{}\fi}\catcode`\%=\active\def%{\%}$\mathdefault{60}$}}%
\end{pgfscope}%
\begin{pgfscope}%
\pgfpathrectangle{\pgfqpoint{0.941663in}{0.670138in}}{\pgfqpoint{8.858337in}{3.465625in}}%
\pgfusepath{clip}%
\pgfsetrectcap%
\pgfsetroundjoin%
\pgfsetlinewidth{0.803000pt}%
\definecolor{currentstroke}{rgb}{0.690196,0.690196,0.690196}%
\pgfsetstrokecolor{currentstroke}%
\pgfsetdash{}{0pt}%
\pgfpathmoveto{\pgfqpoint{5.185178in}{0.670138in}}%
\pgfpathlineto{\pgfqpoint{5.185178in}{4.135763in}}%
\pgfusepath{stroke}%
\end{pgfscope}%
\begin{pgfscope}%
\pgfsetbuttcap%
\pgfsetroundjoin%
\definecolor{currentfill}{rgb}{0.000000,0.000000,0.000000}%
\pgfsetfillcolor{currentfill}%
\pgfsetlinewidth{0.803000pt}%
\definecolor{currentstroke}{rgb}{0.000000,0.000000,0.000000}%
\pgfsetstrokecolor{currentstroke}%
\pgfsetdash{}{0pt}%
\pgfsys@defobject{currentmarker}{\pgfqpoint{0.000000in}{-0.048611in}}{\pgfqpoint{0.000000in}{0.000000in}}{%
\pgfpathmoveto{\pgfqpoint{0.000000in}{0.000000in}}%
\pgfpathlineto{\pgfqpoint{0.000000in}{-0.048611in}}%
\pgfusepath{stroke,fill}%
}%
\begin{pgfscope}%
\pgfsys@transformshift{5.185178in}{0.670138in}%
\pgfsys@useobject{currentmarker}{}%
\end{pgfscope}%
\end{pgfscope}%
\begin{pgfscope}%
\definecolor{textcolor}{rgb}{0.000000,0.000000,0.000000}%
\pgfsetstrokecolor{textcolor}%
\pgfsetfillcolor{textcolor}%
\pgftext[x=5.185178in,y=0.572916in,,top]{\color{textcolor}{\rmfamily\fontsize{14.000000}{16.800000}\selectfont\catcode`\^=\active\def^{\ifmmode\sp\else\^{}\fi}\catcode`\%=\active\def%{\%}$\mathdefault{80}$}}%
\end{pgfscope}%
\begin{pgfscope}%
\pgfpathrectangle{\pgfqpoint{0.941663in}{0.670138in}}{\pgfqpoint{8.858337in}{3.465625in}}%
\pgfusepath{clip}%
\pgfsetrectcap%
\pgfsetroundjoin%
\pgfsetlinewidth{0.803000pt}%
\definecolor{currentstroke}{rgb}{0.690196,0.690196,0.690196}%
\pgfsetstrokecolor{currentstroke}%
\pgfsetdash{}{0pt}%
\pgfpathmoveto{\pgfqpoint{6.246056in}{0.670138in}}%
\pgfpathlineto{\pgfqpoint{6.246056in}{4.135763in}}%
\pgfusepath{stroke}%
\end{pgfscope}%
\begin{pgfscope}%
\pgfsetbuttcap%
\pgfsetroundjoin%
\definecolor{currentfill}{rgb}{0.000000,0.000000,0.000000}%
\pgfsetfillcolor{currentfill}%
\pgfsetlinewidth{0.803000pt}%
\definecolor{currentstroke}{rgb}{0.000000,0.000000,0.000000}%
\pgfsetstrokecolor{currentstroke}%
\pgfsetdash{}{0pt}%
\pgfsys@defobject{currentmarker}{\pgfqpoint{0.000000in}{-0.048611in}}{\pgfqpoint{0.000000in}{0.000000in}}{%
\pgfpathmoveto{\pgfqpoint{0.000000in}{0.000000in}}%
\pgfpathlineto{\pgfqpoint{0.000000in}{-0.048611in}}%
\pgfusepath{stroke,fill}%
}%
\begin{pgfscope}%
\pgfsys@transformshift{6.246056in}{0.670138in}%
\pgfsys@useobject{currentmarker}{}%
\end{pgfscope}%
\end{pgfscope}%
\begin{pgfscope}%
\definecolor{textcolor}{rgb}{0.000000,0.000000,0.000000}%
\pgfsetstrokecolor{textcolor}%
\pgfsetfillcolor{textcolor}%
\pgftext[x=6.246056in,y=0.572916in,,top]{\color{textcolor}{\rmfamily\fontsize{14.000000}{16.800000}\selectfont\catcode`\^=\active\def^{\ifmmode\sp\else\^{}\fi}\catcode`\%=\active\def%{\%}$\mathdefault{100}$}}%
\end{pgfscope}%
\begin{pgfscope}%
\pgfpathrectangle{\pgfqpoint{0.941663in}{0.670138in}}{\pgfqpoint{8.858337in}{3.465625in}}%
\pgfusepath{clip}%
\pgfsetrectcap%
\pgfsetroundjoin%
\pgfsetlinewidth{0.803000pt}%
\definecolor{currentstroke}{rgb}{0.690196,0.690196,0.690196}%
\pgfsetstrokecolor{currentstroke}%
\pgfsetdash{}{0pt}%
\pgfpathmoveto{\pgfqpoint{7.306935in}{0.670138in}}%
\pgfpathlineto{\pgfqpoint{7.306935in}{4.135763in}}%
\pgfusepath{stroke}%
\end{pgfscope}%
\begin{pgfscope}%
\pgfsetbuttcap%
\pgfsetroundjoin%
\definecolor{currentfill}{rgb}{0.000000,0.000000,0.000000}%
\pgfsetfillcolor{currentfill}%
\pgfsetlinewidth{0.803000pt}%
\definecolor{currentstroke}{rgb}{0.000000,0.000000,0.000000}%
\pgfsetstrokecolor{currentstroke}%
\pgfsetdash{}{0pt}%
\pgfsys@defobject{currentmarker}{\pgfqpoint{0.000000in}{-0.048611in}}{\pgfqpoint{0.000000in}{0.000000in}}{%
\pgfpathmoveto{\pgfqpoint{0.000000in}{0.000000in}}%
\pgfpathlineto{\pgfqpoint{0.000000in}{-0.048611in}}%
\pgfusepath{stroke,fill}%
}%
\begin{pgfscope}%
\pgfsys@transformshift{7.306935in}{0.670138in}%
\pgfsys@useobject{currentmarker}{}%
\end{pgfscope}%
\end{pgfscope}%
\begin{pgfscope}%
\definecolor{textcolor}{rgb}{0.000000,0.000000,0.000000}%
\pgfsetstrokecolor{textcolor}%
\pgfsetfillcolor{textcolor}%
\pgftext[x=7.306935in,y=0.572916in,,top]{\color{textcolor}{\rmfamily\fontsize{14.000000}{16.800000}\selectfont\catcode`\^=\active\def^{\ifmmode\sp\else\^{}\fi}\catcode`\%=\active\def%{\%}$\mathdefault{120}$}}%
\end{pgfscope}%
\begin{pgfscope}%
\pgfpathrectangle{\pgfqpoint{0.941663in}{0.670138in}}{\pgfqpoint{8.858337in}{3.465625in}}%
\pgfusepath{clip}%
\pgfsetrectcap%
\pgfsetroundjoin%
\pgfsetlinewidth{0.803000pt}%
\definecolor{currentstroke}{rgb}{0.690196,0.690196,0.690196}%
\pgfsetstrokecolor{currentstroke}%
\pgfsetdash{}{0pt}%
\pgfpathmoveto{\pgfqpoint{8.367814in}{0.670138in}}%
\pgfpathlineto{\pgfqpoint{8.367814in}{4.135763in}}%
\pgfusepath{stroke}%
\end{pgfscope}%
\begin{pgfscope}%
\pgfsetbuttcap%
\pgfsetroundjoin%
\definecolor{currentfill}{rgb}{0.000000,0.000000,0.000000}%
\pgfsetfillcolor{currentfill}%
\pgfsetlinewidth{0.803000pt}%
\definecolor{currentstroke}{rgb}{0.000000,0.000000,0.000000}%
\pgfsetstrokecolor{currentstroke}%
\pgfsetdash{}{0pt}%
\pgfsys@defobject{currentmarker}{\pgfqpoint{0.000000in}{-0.048611in}}{\pgfqpoint{0.000000in}{0.000000in}}{%
\pgfpathmoveto{\pgfqpoint{0.000000in}{0.000000in}}%
\pgfpathlineto{\pgfqpoint{0.000000in}{-0.048611in}}%
\pgfusepath{stroke,fill}%
}%
\begin{pgfscope}%
\pgfsys@transformshift{8.367814in}{0.670138in}%
\pgfsys@useobject{currentmarker}{}%
\end{pgfscope}%
\end{pgfscope}%
\begin{pgfscope}%
\definecolor{textcolor}{rgb}{0.000000,0.000000,0.000000}%
\pgfsetstrokecolor{textcolor}%
\pgfsetfillcolor{textcolor}%
\pgftext[x=8.367814in,y=0.572916in,,top]{\color{textcolor}{\rmfamily\fontsize{14.000000}{16.800000}\selectfont\catcode`\^=\active\def^{\ifmmode\sp\else\^{}\fi}\catcode`\%=\active\def%{\%}$\mathdefault{140}$}}%
\end{pgfscope}%
\begin{pgfscope}%
\pgfpathrectangle{\pgfqpoint{0.941663in}{0.670138in}}{\pgfqpoint{8.858337in}{3.465625in}}%
\pgfusepath{clip}%
\pgfsetrectcap%
\pgfsetroundjoin%
\pgfsetlinewidth{0.803000pt}%
\definecolor{currentstroke}{rgb}{0.690196,0.690196,0.690196}%
\pgfsetstrokecolor{currentstroke}%
\pgfsetdash{}{0pt}%
\pgfpathmoveto{\pgfqpoint{9.428692in}{0.670138in}}%
\pgfpathlineto{\pgfqpoint{9.428692in}{4.135763in}}%
\pgfusepath{stroke}%
\end{pgfscope}%
\begin{pgfscope}%
\pgfsetbuttcap%
\pgfsetroundjoin%
\definecolor{currentfill}{rgb}{0.000000,0.000000,0.000000}%
\pgfsetfillcolor{currentfill}%
\pgfsetlinewidth{0.803000pt}%
\definecolor{currentstroke}{rgb}{0.000000,0.000000,0.000000}%
\pgfsetstrokecolor{currentstroke}%
\pgfsetdash{}{0pt}%
\pgfsys@defobject{currentmarker}{\pgfqpoint{0.000000in}{-0.048611in}}{\pgfqpoint{0.000000in}{0.000000in}}{%
\pgfpathmoveto{\pgfqpoint{0.000000in}{0.000000in}}%
\pgfpathlineto{\pgfqpoint{0.000000in}{-0.048611in}}%
\pgfusepath{stroke,fill}%
}%
\begin{pgfscope}%
\pgfsys@transformshift{9.428692in}{0.670138in}%
\pgfsys@useobject{currentmarker}{}%
\end{pgfscope}%
\end{pgfscope}%
\begin{pgfscope}%
\definecolor{textcolor}{rgb}{0.000000,0.000000,0.000000}%
\pgfsetstrokecolor{textcolor}%
\pgfsetfillcolor{textcolor}%
\pgftext[x=9.428692in,y=0.572916in,,top]{\color{textcolor}{\rmfamily\fontsize{14.000000}{16.800000}\selectfont\catcode`\^=\active\def^{\ifmmode\sp\else\^{}\fi}\catcode`\%=\active\def%{\%}$\mathdefault{160}$}}%
\end{pgfscope}%
\begin{pgfscope}%
\definecolor{textcolor}{rgb}{0.000000,0.000000,0.000000}%
\pgfsetstrokecolor{textcolor}%
\pgfsetfillcolor{textcolor}%
\pgftext[x=5.370831in,y=0.339583in,,top]{\color{textcolor}{\rmfamily\fontsize{18.000000}{21.600000}\selectfont\catcode`\^=\active\def^{\ifmmode\sp\else\^{}\fi}\catcode`\%=\active\def%{\%}Time [hours]}}%
\end{pgfscope}%
\begin{pgfscope}%
\pgfpathrectangle{\pgfqpoint{0.941663in}{0.670138in}}{\pgfqpoint{8.858337in}{3.465625in}}%
\pgfusepath{clip}%
\pgfsetrectcap%
\pgfsetroundjoin%
\pgfsetlinewidth{0.803000pt}%
\definecolor{currentstroke}{rgb}{0.690196,0.690196,0.690196}%
\pgfsetstrokecolor{currentstroke}%
\pgfsetdash{}{0pt}%
\pgfpathmoveto{\pgfqpoint{0.941663in}{1.195548in}}%
\pgfpathlineto{\pgfqpoint{9.800000in}{1.195548in}}%
\pgfusepath{stroke}%
\end{pgfscope}%
\begin{pgfscope}%
\pgfsetbuttcap%
\pgfsetroundjoin%
\definecolor{currentfill}{rgb}{0.000000,0.000000,0.000000}%
\pgfsetfillcolor{currentfill}%
\pgfsetlinewidth{0.803000pt}%
\definecolor{currentstroke}{rgb}{0.000000,0.000000,0.000000}%
\pgfsetstrokecolor{currentstroke}%
\pgfsetdash{}{0pt}%
\pgfsys@defobject{currentmarker}{\pgfqpoint{-0.048611in}{0.000000in}}{\pgfqpoint{-0.000000in}{0.000000in}}{%
\pgfpathmoveto{\pgfqpoint{-0.000000in}{0.000000in}}%
\pgfpathlineto{\pgfqpoint{-0.048611in}{0.000000in}}%
\pgfusepath{stroke,fill}%
}%
\begin{pgfscope}%
\pgfsys@transformshift{0.941663in}{1.195548in}%
\pgfsys@useobject{currentmarker}{}%
\end{pgfscope}%
\end{pgfscope}%
\begin{pgfscope}%
\definecolor{textcolor}{rgb}{0.000000,0.000000,0.000000}%
\pgfsetstrokecolor{textcolor}%
\pgfsetfillcolor{textcolor}%
\pgftext[x=0.395138in, y=1.126104in, left, base]{\color{textcolor}{\rmfamily\fontsize{14.000000}{16.800000}\selectfont\catcode`\^=\active\def^{\ifmmode\sp\else\^{}\fi}\catcode`\%=\active\def%{\%}$\mathdefault{\ensuremath{-}500}$}}%
\end{pgfscope}%
\begin{pgfscope}%
\pgfpathrectangle{\pgfqpoint{0.941663in}{0.670138in}}{\pgfqpoint{8.858337in}{3.465625in}}%
\pgfusepath{clip}%
\pgfsetrectcap%
\pgfsetroundjoin%
\pgfsetlinewidth{0.803000pt}%
\definecolor{currentstroke}{rgb}{0.690196,0.690196,0.690196}%
\pgfsetstrokecolor{currentstroke}%
\pgfsetdash{}{0pt}%
\pgfpathmoveto{\pgfqpoint{0.941663in}{1.891220in}}%
\pgfpathlineto{\pgfqpoint{9.800000in}{1.891220in}}%
\pgfusepath{stroke}%
\end{pgfscope}%
\begin{pgfscope}%
\pgfsetbuttcap%
\pgfsetroundjoin%
\definecolor{currentfill}{rgb}{0.000000,0.000000,0.000000}%
\pgfsetfillcolor{currentfill}%
\pgfsetlinewidth{0.803000pt}%
\definecolor{currentstroke}{rgb}{0.000000,0.000000,0.000000}%
\pgfsetstrokecolor{currentstroke}%
\pgfsetdash{}{0pt}%
\pgfsys@defobject{currentmarker}{\pgfqpoint{-0.048611in}{0.000000in}}{\pgfqpoint{-0.000000in}{0.000000in}}{%
\pgfpathmoveto{\pgfqpoint{-0.000000in}{0.000000in}}%
\pgfpathlineto{\pgfqpoint{-0.048611in}{0.000000in}}%
\pgfusepath{stroke,fill}%
}%
\begin{pgfscope}%
\pgfsys@transformshift{0.941663in}{1.891220in}%
\pgfsys@useobject{currentmarker}{}%
\end{pgfscope}%
\end{pgfscope}%
\begin{pgfscope}%
\definecolor{textcolor}{rgb}{0.000000,0.000000,0.000000}%
\pgfsetstrokecolor{textcolor}%
\pgfsetfillcolor{textcolor}%
\pgftext[x=0.746525in, y=1.821776in, left, base]{\color{textcolor}{\rmfamily\fontsize{14.000000}{16.800000}\selectfont\catcode`\^=\active\def^{\ifmmode\sp\else\^{}\fi}\catcode`\%=\active\def%{\%}$\mathdefault{0}$}}%
\end{pgfscope}%
\begin{pgfscope}%
\pgfpathrectangle{\pgfqpoint{0.941663in}{0.670138in}}{\pgfqpoint{8.858337in}{3.465625in}}%
\pgfusepath{clip}%
\pgfsetrectcap%
\pgfsetroundjoin%
\pgfsetlinewidth{0.803000pt}%
\definecolor{currentstroke}{rgb}{0.690196,0.690196,0.690196}%
\pgfsetstrokecolor{currentstroke}%
\pgfsetdash{}{0pt}%
\pgfpathmoveto{\pgfqpoint{0.941663in}{2.586892in}}%
\pgfpathlineto{\pgfqpoint{9.800000in}{2.586892in}}%
\pgfusepath{stroke}%
\end{pgfscope}%
\begin{pgfscope}%
\pgfsetbuttcap%
\pgfsetroundjoin%
\definecolor{currentfill}{rgb}{0.000000,0.000000,0.000000}%
\pgfsetfillcolor{currentfill}%
\pgfsetlinewidth{0.803000pt}%
\definecolor{currentstroke}{rgb}{0.000000,0.000000,0.000000}%
\pgfsetstrokecolor{currentstroke}%
\pgfsetdash{}{0pt}%
\pgfsys@defobject{currentmarker}{\pgfqpoint{-0.048611in}{0.000000in}}{\pgfqpoint{-0.000000in}{0.000000in}}{%
\pgfpathmoveto{\pgfqpoint{-0.000000in}{0.000000in}}%
\pgfpathlineto{\pgfqpoint{-0.048611in}{0.000000in}}%
\pgfusepath{stroke,fill}%
}%
\begin{pgfscope}%
\pgfsys@transformshift{0.941663in}{2.586892in}%
\pgfsys@useobject{currentmarker}{}%
\end{pgfscope}%
\end{pgfscope}%
\begin{pgfscope}%
\definecolor{textcolor}{rgb}{0.000000,0.000000,0.000000}%
\pgfsetstrokecolor{textcolor}%
\pgfsetfillcolor{textcolor}%
\pgftext[x=0.550694in, y=2.517447in, left, base]{\color{textcolor}{\rmfamily\fontsize{14.000000}{16.800000}\selectfont\catcode`\^=\active\def^{\ifmmode\sp\else\^{}\fi}\catcode`\%=\active\def%{\%}$\mathdefault{500}$}}%
\end{pgfscope}%
\begin{pgfscope}%
\pgfpathrectangle{\pgfqpoint{0.941663in}{0.670138in}}{\pgfqpoint{8.858337in}{3.465625in}}%
\pgfusepath{clip}%
\pgfsetrectcap%
\pgfsetroundjoin%
\pgfsetlinewidth{0.803000pt}%
\definecolor{currentstroke}{rgb}{0.690196,0.690196,0.690196}%
\pgfsetstrokecolor{currentstroke}%
\pgfsetdash{}{0pt}%
\pgfpathmoveto{\pgfqpoint{0.941663in}{3.282563in}}%
\pgfpathlineto{\pgfqpoint{9.800000in}{3.282563in}}%
\pgfusepath{stroke}%
\end{pgfscope}%
\begin{pgfscope}%
\pgfsetbuttcap%
\pgfsetroundjoin%
\definecolor{currentfill}{rgb}{0.000000,0.000000,0.000000}%
\pgfsetfillcolor{currentfill}%
\pgfsetlinewidth{0.803000pt}%
\definecolor{currentstroke}{rgb}{0.000000,0.000000,0.000000}%
\pgfsetstrokecolor{currentstroke}%
\pgfsetdash{}{0pt}%
\pgfsys@defobject{currentmarker}{\pgfqpoint{-0.048611in}{0.000000in}}{\pgfqpoint{-0.000000in}{0.000000in}}{%
\pgfpathmoveto{\pgfqpoint{-0.000000in}{0.000000in}}%
\pgfpathlineto{\pgfqpoint{-0.048611in}{0.000000in}}%
\pgfusepath{stroke,fill}%
}%
\begin{pgfscope}%
\pgfsys@transformshift{0.941663in}{3.282563in}%
\pgfsys@useobject{currentmarker}{}%
\end{pgfscope}%
\end{pgfscope}%
\begin{pgfscope}%
\definecolor{textcolor}{rgb}{0.000000,0.000000,0.000000}%
\pgfsetstrokecolor{textcolor}%
\pgfsetfillcolor{textcolor}%
\pgftext[x=0.452779in, y=3.213119in, left, base]{\color{textcolor}{\rmfamily\fontsize{14.000000}{16.800000}\selectfont\catcode`\^=\active\def^{\ifmmode\sp\else\^{}\fi}\catcode`\%=\active\def%{\%}$\mathdefault{1000}$}}%
\end{pgfscope}%
\begin{pgfscope}%
\pgfpathrectangle{\pgfqpoint{0.941663in}{0.670138in}}{\pgfqpoint{8.858337in}{3.465625in}}%
\pgfusepath{clip}%
\pgfsetrectcap%
\pgfsetroundjoin%
\pgfsetlinewidth{0.803000pt}%
\definecolor{currentstroke}{rgb}{0.690196,0.690196,0.690196}%
\pgfsetstrokecolor{currentstroke}%
\pgfsetdash{}{0pt}%
\pgfpathmoveto{\pgfqpoint{0.941663in}{3.978235in}}%
\pgfpathlineto{\pgfqpoint{9.800000in}{3.978235in}}%
\pgfusepath{stroke}%
\end{pgfscope}%
\begin{pgfscope}%
\pgfsetbuttcap%
\pgfsetroundjoin%
\definecolor{currentfill}{rgb}{0.000000,0.000000,0.000000}%
\pgfsetfillcolor{currentfill}%
\pgfsetlinewidth{0.803000pt}%
\definecolor{currentstroke}{rgb}{0.000000,0.000000,0.000000}%
\pgfsetstrokecolor{currentstroke}%
\pgfsetdash{}{0pt}%
\pgfsys@defobject{currentmarker}{\pgfqpoint{-0.048611in}{0.000000in}}{\pgfqpoint{-0.000000in}{0.000000in}}{%
\pgfpathmoveto{\pgfqpoint{-0.000000in}{0.000000in}}%
\pgfpathlineto{\pgfqpoint{-0.048611in}{0.000000in}}%
\pgfusepath{stroke,fill}%
}%
\begin{pgfscope}%
\pgfsys@transformshift{0.941663in}{3.978235in}%
\pgfsys@useobject{currentmarker}{}%
\end{pgfscope}%
\end{pgfscope}%
\begin{pgfscope}%
\definecolor{textcolor}{rgb}{0.000000,0.000000,0.000000}%
\pgfsetstrokecolor{textcolor}%
\pgfsetfillcolor{textcolor}%
\pgftext[x=0.452779in, y=3.908791in, left, base]{\color{textcolor}{\rmfamily\fontsize{14.000000}{16.800000}\selectfont\catcode`\^=\active\def^{\ifmmode\sp\else\^{}\fi}\catcode`\%=\active\def%{\%}$\mathdefault{1500}$}}%
\end{pgfscope}%
\begin{pgfscope}%
\definecolor{textcolor}{rgb}{0.000000,0.000000,0.000000}%
\pgfsetstrokecolor{textcolor}%
\pgfsetfillcolor{textcolor}%
\pgftext[x=0.339583in,y=2.402951in,,bottom,rotate=90.000000]{\color{textcolor}{\rmfamily\fontsize{18.000000}{21.600000}\selectfont\catcode`\^=\active\def^{\ifmmode\sp\else\^{}\fi}\catcode`\%=\active\def%{\%}Energy [MWh]}}%
\end{pgfscope}%
\begin{pgfscope}%
\pgfpathrectangle{\pgfqpoint{0.941663in}{0.670138in}}{\pgfqpoint{8.858337in}{3.465625in}}%
\pgfusepath{clip}%
\pgfsetrectcap%
\pgfsetroundjoin%
\pgfsetlinewidth{1.505625pt}%
\definecolor{currentstroke}{rgb}{0.121569,0.466667,0.705882}%
\pgfsetstrokecolor{currentstroke}%
\pgfsetdash{}{0pt}%
\pgfpathmoveto{\pgfqpoint{0.941663in}{2.586892in}}%
\pgfpathlineto{\pgfqpoint{9.800000in}{2.586892in}}%
\pgfpathlineto{\pgfqpoint{9.800000in}{2.586892in}}%
\pgfusepath{stroke}%
\end{pgfscope}%
\begin{pgfscope}%
\pgfpathrectangle{\pgfqpoint{0.941663in}{0.670138in}}{\pgfqpoint{8.858337in}{3.465625in}}%
\pgfusepath{clip}%
\pgfsetbuttcap%
\pgfsetroundjoin%
\definecolor{currentfill}{rgb}{0.121569,0.466667,0.705882}%
\pgfsetfillcolor{currentfill}%
\pgfsetlinewidth{1.003750pt}%
\definecolor{currentstroke}{rgb}{0.121569,0.466667,0.705882}%
\pgfsetstrokecolor{currentstroke}%
\pgfsetdash{}{0pt}%
\pgfsys@defobject{currentmarker}{\pgfqpoint{0.941663in}{1.891220in}}{\pgfqpoint{9.800000in}{2.586892in}}{%
\pgfpathmoveto{\pgfqpoint{0.941663in}{2.586892in}}%
\pgfpathlineto{\pgfqpoint{0.941663in}{1.891220in}}%
\pgfpathlineto{\pgfqpoint{0.994707in}{1.891220in}}%
\pgfpathlineto{\pgfqpoint{1.047751in}{1.891220in}}%
\pgfpathlineto{\pgfqpoint{1.100795in}{1.891220in}}%
\pgfpathlineto{\pgfqpoint{1.153839in}{1.891220in}}%
\pgfpathlineto{\pgfqpoint{1.206883in}{1.891220in}}%
\pgfpathlineto{\pgfqpoint{1.259927in}{1.891220in}}%
\pgfpathlineto{\pgfqpoint{1.312970in}{1.891220in}}%
\pgfpathlineto{\pgfqpoint{1.366014in}{1.891220in}}%
\pgfpathlineto{\pgfqpoint{1.419058in}{1.891220in}}%
\pgfpathlineto{\pgfqpoint{1.472102in}{1.891220in}}%
\pgfpathlineto{\pgfqpoint{1.525146in}{1.891220in}}%
\pgfpathlineto{\pgfqpoint{1.578190in}{1.891220in}}%
\pgfpathlineto{\pgfqpoint{1.631234in}{1.891220in}}%
\pgfpathlineto{\pgfqpoint{1.684278in}{1.891220in}}%
\pgfpathlineto{\pgfqpoint{1.737322in}{1.891220in}}%
\pgfpathlineto{\pgfqpoint{1.790366in}{1.891220in}}%
\pgfpathlineto{\pgfqpoint{1.843410in}{1.891220in}}%
\pgfpathlineto{\pgfqpoint{1.896454in}{1.891220in}}%
\pgfpathlineto{\pgfqpoint{1.949498in}{1.891220in}}%
\pgfpathlineto{\pgfqpoint{2.002542in}{1.891220in}}%
\pgfpathlineto{\pgfqpoint{2.055586in}{1.891220in}}%
\pgfpathlineto{\pgfqpoint{2.108629in}{1.891220in}}%
\pgfpathlineto{\pgfqpoint{2.161673in}{1.891220in}}%
\pgfpathlineto{\pgfqpoint{2.214717in}{1.891220in}}%
\pgfpathlineto{\pgfqpoint{2.267761in}{1.891220in}}%
\pgfpathlineto{\pgfqpoint{2.320805in}{1.891220in}}%
\pgfpathlineto{\pgfqpoint{2.373849in}{1.891220in}}%
\pgfpathlineto{\pgfqpoint{2.426893in}{1.891220in}}%
\pgfpathlineto{\pgfqpoint{2.479937in}{1.891220in}}%
\pgfpathlineto{\pgfqpoint{2.532981in}{1.891220in}}%
\pgfpathlineto{\pgfqpoint{2.586025in}{1.891220in}}%
\pgfpathlineto{\pgfqpoint{2.639069in}{1.891220in}}%
\pgfpathlineto{\pgfqpoint{2.692113in}{1.891220in}}%
\pgfpathlineto{\pgfqpoint{2.745157in}{1.891220in}}%
\pgfpathlineto{\pgfqpoint{2.798201in}{1.891220in}}%
\pgfpathlineto{\pgfqpoint{2.851245in}{1.891220in}}%
\pgfpathlineto{\pgfqpoint{2.904288in}{1.891220in}}%
\pgfpathlineto{\pgfqpoint{2.957332in}{1.891220in}}%
\pgfpathlineto{\pgfqpoint{3.010376in}{1.891220in}}%
\pgfpathlineto{\pgfqpoint{3.063420in}{1.891220in}}%
\pgfpathlineto{\pgfqpoint{3.116464in}{1.891220in}}%
\pgfpathlineto{\pgfqpoint{3.169508in}{1.891220in}}%
\pgfpathlineto{\pgfqpoint{3.222552in}{1.891220in}}%
\pgfpathlineto{\pgfqpoint{3.275596in}{1.891220in}}%
\pgfpathlineto{\pgfqpoint{3.328640in}{1.891220in}}%
\pgfpathlineto{\pgfqpoint{3.381684in}{1.891220in}}%
\pgfpathlineto{\pgfqpoint{3.434728in}{1.891220in}}%
\pgfpathlineto{\pgfqpoint{3.487772in}{1.891220in}}%
\pgfpathlineto{\pgfqpoint{3.540816in}{1.891220in}}%
\pgfpathlineto{\pgfqpoint{3.593860in}{1.891220in}}%
\pgfpathlineto{\pgfqpoint{3.646904in}{1.891220in}}%
\pgfpathlineto{\pgfqpoint{3.699948in}{1.891220in}}%
\pgfpathlineto{\pgfqpoint{3.752991in}{1.891220in}}%
\pgfpathlineto{\pgfqpoint{3.806035in}{1.891220in}}%
\pgfpathlineto{\pgfqpoint{3.859079in}{1.891220in}}%
\pgfpathlineto{\pgfqpoint{3.912123in}{1.891220in}}%
\pgfpathlineto{\pgfqpoint{3.965167in}{1.891220in}}%
\pgfpathlineto{\pgfqpoint{4.018211in}{1.891220in}}%
\pgfpathlineto{\pgfqpoint{4.071255in}{1.891220in}}%
\pgfpathlineto{\pgfqpoint{4.124299in}{1.891220in}}%
\pgfpathlineto{\pgfqpoint{4.177343in}{1.891220in}}%
\pgfpathlineto{\pgfqpoint{4.230387in}{1.891220in}}%
\pgfpathlineto{\pgfqpoint{4.283431in}{1.891220in}}%
\pgfpathlineto{\pgfqpoint{4.336475in}{1.891220in}}%
\pgfpathlineto{\pgfqpoint{4.389519in}{1.891220in}}%
\pgfpathlineto{\pgfqpoint{4.442563in}{1.891220in}}%
\pgfpathlineto{\pgfqpoint{4.495607in}{1.891220in}}%
\pgfpathlineto{\pgfqpoint{4.548650in}{1.891220in}}%
\pgfpathlineto{\pgfqpoint{4.601694in}{1.891220in}}%
\pgfpathlineto{\pgfqpoint{4.654738in}{1.891220in}}%
\pgfpathlineto{\pgfqpoint{4.707782in}{1.891220in}}%
\pgfpathlineto{\pgfqpoint{4.760826in}{1.891220in}}%
\pgfpathlineto{\pgfqpoint{4.813870in}{1.891220in}}%
\pgfpathlineto{\pgfqpoint{4.866914in}{1.891220in}}%
\pgfpathlineto{\pgfqpoint{4.919958in}{1.891220in}}%
\pgfpathlineto{\pgfqpoint{4.973002in}{1.891220in}}%
\pgfpathlineto{\pgfqpoint{5.026046in}{1.891220in}}%
\pgfpathlineto{\pgfqpoint{5.079090in}{1.891220in}}%
\pgfpathlineto{\pgfqpoint{5.132134in}{1.891220in}}%
\pgfpathlineto{\pgfqpoint{5.185178in}{1.891220in}}%
\pgfpathlineto{\pgfqpoint{5.238222in}{1.891220in}}%
\pgfpathlineto{\pgfqpoint{5.291266in}{1.891220in}}%
\pgfpathlineto{\pgfqpoint{5.344309in}{1.891220in}}%
\pgfpathlineto{\pgfqpoint{5.397353in}{1.891220in}}%
\pgfpathlineto{\pgfqpoint{5.450397in}{1.891220in}}%
\pgfpathlineto{\pgfqpoint{5.503441in}{1.891220in}}%
\pgfpathlineto{\pgfqpoint{5.556485in}{1.891220in}}%
\pgfpathlineto{\pgfqpoint{5.609529in}{1.891220in}}%
\pgfpathlineto{\pgfqpoint{5.662573in}{1.891220in}}%
\pgfpathlineto{\pgfqpoint{5.715617in}{1.891220in}}%
\pgfpathlineto{\pgfqpoint{5.768661in}{1.891220in}}%
\pgfpathlineto{\pgfqpoint{5.821705in}{1.891220in}}%
\pgfpathlineto{\pgfqpoint{5.874749in}{1.891220in}}%
\pgfpathlineto{\pgfqpoint{5.927793in}{1.891220in}}%
\pgfpathlineto{\pgfqpoint{5.980837in}{1.891220in}}%
\pgfpathlineto{\pgfqpoint{6.033881in}{1.891220in}}%
\pgfpathlineto{\pgfqpoint{6.086925in}{1.891220in}}%
\pgfpathlineto{\pgfqpoint{6.139969in}{1.891220in}}%
\pgfpathlineto{\pgfqpoint{6.193012in}{1.891220in}}%
\pgfpathlineto{\pgfqpoint{6.246056in}{1.891220in}}%
\pgfpathlineto{\pgfqpoint{6.299100in}{1.891220in}}%
\pgfpathlineto{\pgfqpoint{6.352144in}{1.891220in}}%
\pgfpathlineto{\pgfqpoint{6.405188in}{1.891220in}}%
\pgfpathlineto{\pgfqpoint{6.458232in}{1.891220in}}%
\pgfpathlineto{\pgfqpoint{6.511276in}{1.891220in}}%
\pgfpathlineto{\pgfqpoint{6.564320in}{1.891220in}}%
\pgfpathlineto{\pgfqpoint{6.617364in}{1.891220in}}%
\pgfpathlineto{\pgfqpoint{6.670408in}{1.891220in}}%
\pgfpathlineto{\pgfqpoint{6.723452in}{1.891220in}}%
\pgfpathlineto{\pgfqpoint{6.776496in}{1.891220in}}%
\pgfpathlineto{\pgfqpoint{6.829540in}{1.891220in}}%
\pgfpathlineto{\pgfqpoint{6.882584in}{1.891220in}}%
\pgfpathlineto{\pgfqpoint{6.935628in}{1.891220in}}%
\pgfpathlineto{\pgfqpoint{6.988671in}{1.891220in}}%
\pgfpathlineto{\pgfqpoint{7.041715in}{1.891220in}}%
\pgfpathlineto{\pgfqpoint{7.094759in}{1.891220in}}%
\pgfpathlineto{\pgfqpoint{7.147803in}{1.891220in}}%
\pgfpathlineto{\pgfqpoint{7.200847in}{1.891220in}}%
\pgfpathlineto{\pgfqpoint{7.253891in}{1.891220in}}%
\pgfpathlineto{\pgfqpoint{7.306935in}{1.891220in}}%
\pgfpathlineto{\pgfqpoint{7.359979in}{1.891220in}}%
\pgfpathlineto{\pgfqpoint{7.413023in}{1.891220in}}%
\pgfpathlineto{\pgfqpoint{7.466067in}{1.891220in}}%
\pgfpathlineto{\pgfqpoint{7.519111in}{1.891220in}}%
\pgfpathlineto{\pgfqpoint{7.572155in}{1.891220in}}%
\pgfpathlineto{\pgfqpoint{7.625199in}{1.891220in}}%
\pgfpathlineto{\pgfqpoint{7.678243in}{1.891220in}}%
\pgfpathlineto{\pgfqpoint{7.731287in}{1.891220in}}%
\pgfpathlineto{\pgfqpoint{7.784330in}{1.891220in}}%
\pgfpathlineto{\pgfqpoint{7.837374in}{1.891220in}}%
\pgfpathlineto{\pgfqpoint{7.890418in}{1.891220in}}%
\pgfpathlineto{\pgfqpoint{7.943462in}{1.891220in}}%
\pgfpathlineto{\pgfqpoint{7.996506in}{1.891220in}}%
\pgfpathlineto{\pgfqpoint{8.049550in}{1.891220in}}%
\pgfpathlineto{\pgfqpoint{8.102594in}{1.891220in}}%
\pgfpathlineto{\pgfqpoint{8.155638in}{1.891220in}}%
\pgfpathlineto{\pgfqpoint{8.208682in}{1.891220in}}%
\pgfpathlineto{\pgfqpoint{8.261726in}{1.891220in}}%
\pgfpathlineto{\pgfqpoint{8.314770in}{1.891220in}}%
\pgfpathlineto{\pgfqpoint{8.367814in}{1.891220in}}%
\pgfpathlineto{\pgfqpoint{8.420858in}{1.891220in}}%
\pgfpathlineto{\pgfqpoint{8.473902in}{1.891220in}}%
\pgfpathlineto{\pgfqpoint{8.526946in}{1.891220in}}%
\pgfpathlineto{\pgfqpoint{8.579990in}{1.891220in}}%
\pgfpathlineto{\pgfqpoint{8.633033in}{1.891220in}}%
\pgfpathlineto{\pgfqpoint{8.686077in}{1.891220in}}%
\pgfpathlineto{\pgfqpoint{8.739121in}{1.891220in}}%
\pgfpathlineto{\pgfqpoint{8.792165in}{1.891220in}}%
\pgfpathlineto{\pgfqpoint{8.845209in}{1.891220in}}%
\pgfpathlineto{\pgfqpoint{8.898253in}{1.891220in}}%
\pgfpathlineto{\pgfqpoint{8.951297in}{1.891220in}}%
\pgfpathlineto{\pgfqpoint{9.004341in}{1.891220in}}%
\pgfpathlineto{\pgfqpoint{9.057385in}{1.891220in}}%
\pgfpathlineto{\pgfqpoint{9.110429in}{1.891220in}}%
\pgfpathlineto{\pgfqpoint{9.163473in}{1.891220in}}%
\pgfpathlineto{\pgfqpoint{9.216517in}{1.891220in}}%
\pgfpathlineto{\pgfqpoint{9.269561in}{1.891220in}}%
\pgfpathlineto{\pgfqpoint{9.322605in}{1.891220in}}%
\pgfpathlineto{\pgfqpoint{9.375649in}{1.891220in}}%
\pgfpathlineto{\pgfqpoint{9.428692in}{1.891220in}}%
\pgfpathlineto{\pgfqpoint{9.481736in}{1.891220in}}%
\pgfpathlineto{\pgfqpoint{9.534780in}{1.891220in}}%
\pgfpathlineto{\pgfqpoint{9.587824in}{1.891220in}}%
\pgfpathlineto{\pgfqpoint{9.640868in}{1.891220in}}%
\pgfpathlineto{\pgfqpoint{9.693912in}{1.891220in}}%
\pgfpathlineto{\pgfqpoint{9.746956in}{1.891220in}}%
\pgfpathlineto{\pgfqpoint{9.800000in}{1.891220in}}%
\pgfpathlineto{\pgfqpoint{9.800000in}{2.586892in}}%
\pgfpathlineto{\pgfqpoint{9.800000in}{2.586892in}}%
\pgfpathlineto{\pgfqpoint{9.746956in}{2.586892in}}%
\pgfpathlineto{\pgfqpoint{9.693912in}{2.586892in}}%
\pgfpathlineto{\pgfqpoint{9.640868in}{2.586892in}}%
\pgfpathlineto{\pgfqpoint{9.587824in}{2.586892in}}%
\pgfpathlineto{\pgfqpoint{9.534780in}{2.586892in}}%
\pgfpathlineto{\pgfqpoint{9.481736in}{2.586892in}}%
\pgfpathlineto{\pgfqpoint{9.428692in}{2.586892in}}%
\pgfpathlineto{\pgfqpoint{9.375649in}{2.586892in}}%
\pgfpathlineto{\pgfqpoint{9.322605in}{2.586892in}}%
\pgfpathlineto{\pgfqpoint{9.269561in}{2.586892in}}%
\pgfpathlineto{\pgfqpoint{9.216517in}{2.586892in}}%
\pgfpathlineto{\pgfqpoint{9.163473in}{2.586892in}}%
\pgfpathlineto{\pgfqpoint{9.110429in}{2.586892in}}%
\pgfpathlineto{\pgfqpoint{9.057385in}{2.586892in}}%
\pgfpathlineto{\pgfqpoint{9.004341in}{2.586892in}}%
\pgfpathlineto{\pgfqpoint{8.951297in}{2.586892in}}%
\pgfpathlineto{\pgfqpoint{8.898253in}{2.586892in}}%
\pgfpathlineto{\pgfqpoint{8.845209in}{2.586892in}}%
\pgfpathlineto{\pgfqpoint{8.792165in}{2.586892in}}%
\pgfpathlineto{\pgfqpoint{8.739121in}{2.586892in}}%
\pgfpathlineto{\pgfqpoint{8.686077in}{2.586892in}}%
\pgfpathlineto{\pgfqpoint{8.633033in}{2.586892in}}%
\pgfpathlineto{\pgfqpoint{8.579990in}{2.586892in}}%
\pgfpathlineto{\pgfqpoint{8.526946in}{2.586892in}}%
\pgfpathlineto{\pgfqpoint{8.473902in}{2.586892in}}%
\pgfpathlineto{\pgfqpoint{8.420858in}{2.586892in}}%
\pgfpathlineto{\pgfqpoint{8.367814in}{2.586892in}}%
\pgfpathlineto{\pgfqpoint{8.314770in}{2.586892in}}%
\pgfpathlineto{\pgfqpoint{8.261726in}{2.586892in}}%
\pgfpathlineto{\pgfqpoint{8.208682in}{2.586892in}}%
\pgfpathlineto{\pgfqpoint{8.155638in}{2.586892in}}%
\pgfpathlineto{\pgfqpoint{8.102594in}{2.586892in}}%
\pgfpathlineto{\pgfqpoint{8.049550in}{2.586892in}}%
\pgfpathlineto{\pgfqpoint{7.996506in}{2.586892in}}%
\pgfpathlineto{\pgfqpoint{7.943462in}{2.586892in}}%
\pgfpathlineto{\pgfqpoint{7.890418in}{2.586892in}}%
\pgfpathlineto{\pgfqpoint{7.837374in}{2.586892in}}%
\pgfpathlineto{\pgfqpoint{7.784330in}{2.586892in}}%
\pgfpathlineto{\pgfqpoint{7.731287in}{2.586892in}}%
\pgfpathlineto{\pgfqpoint{7.678243in}{2.586892in}}%
\pgfpathlineto{\pgfqpoint{7.625199in}{2.586892in}}%
\pgfpathlineto{\pgfqpoint{7.572155in}{2.586892in}}%
\pgfpathlineto{\pgfqpoint{7.519111in}{2.586892in}}%
\pgfpathlineto{\pgfqpoint{7.466067in}{2.586892in}}%
\pgfpathlineto{\pgfqpoint{7.413023in}{2.586892in}}%
\pgfpathlineto{\pgfqpoint{7.359979in}{2.586892in}}%
\pgfpathlineto{\pgfqpoint{7.306935in}{2.586892in}}%
\pgfpathlineto{\pgfqpoint{7.253891in}{2.586892in}}%
\pgfpathlineto{\pgfqpoint{7.200847in}{2.586892in}}%
\pgfpathlineto{\pgfqpoint{7.147803in}{2.586892in}}%
\pgfpathlineto{\pgfqpoint{7.094759in}{2.586892in}}%
\pgfpathlineto{\pgfqpoint{7.041715in}{2.586892in}}%
\pgfpathlineto{\pgfqpoint{6.988671in}{2.586892in}}%
\pgfpathlineto{\pgfqpoint{6.935628in}{2.586892in}}%
\pgfpathlineto{\pgfqpoint{6.882584in}{2.586892in}}%
\pgfpathlineto{\pgfqpoint{6.829540in}{2.586892in}}%
\pgfpathlineto{\pgfqpoint{6.776496in}{2.586892in}}%
\pgfpathlineto{\pgfqpoint{6.723452in}{2.586892in}}%
\pgfpathlineto{\pgfqpoint{6.670408in}{2.586892in}}%
\pgfpathlineto{\pgfqpoint{6.617364in}{2.586892in}}%
\pgfpathlineto{\pgfqpoint{6.564320in}{2.586892in}}%
\pgfpathlineto{\pgfqpoint{6.511276in}{2.586892in}}%
\pgfpathlineto{\pgfqpoint{6.458232in}{2.586892in}}%
\pgfpathlineto{\pgfqpoint{6.405188in}{2.586892in}}%
\pgfpathlineto{\pgfqpoint{6.352144in}{2.586892in}}%
\pgfpathlineto{\pgfqpoint{6.299100in}{2.586892in}}%
\pgfpathlineto{\pgfqpoint{6.246056in}{2.586892in}}%
\pgfpathlineto{\pgfqpoint{6.193012in}{2.586892in}}%
\pgfpathlineto{\pgfqpoint{6.139969in}{2.586892in}}%
\pgfpathlineto{\pgfqpoint{6.086925in}{2.586892in}}%
\pgfpathlineto{\pgfqpoint{6.033881in}{2.586892in}}%
\pgfpathlineto{\pgfqpoint{5.980837in}{2.586892in}}%
\pgfpathlineto{\pgfqpoint{5.927793in}{2.586892in}}%
\pgfpathlineto{\pgfqpoint{5.874749in}{2.586892in}}%
\pgfpathlineto{\pgfqpoint{5.821705in}{2.586892in}}%
\pgfpathlineto{\pgfqpoint{5.768661in}{2.586892in}}%
\pgfpathlineto{\pgfqpoint{5.715617in}{2.586892in}}%
\pgfpathlineto{\pgfqpoint{5.662573in}{2.586892in}}%
\pgfpathlineto{\pgfqpoint{5.609529in}{2.586892in}}%
\pgfpathlineto{\pgfqpoint{5.556485in}{2.586892in}}%
\pgfpathlineto{\pgfqpoint{5.503441in}{2.586892in}}%
\pgfpathlineto{\pgfqpoint{5.450397in}{2.586892in}}%
\pgfpathlineto{\pgfqpoint{5.397353in}{2.586892in}}%
\pgfpathlineto{\pgfqpoint{5.344309in}{2.586892in}}%
\pgfpathlineto{\pgfqpoint{5.291266in}{2.586892in}}%
\pgfpathlineto{\pgfqpoint{5.238222in}{2.586892in}}%
\pgfpathlineto{\pgfqpoint{5.185178in}{2.586892in}}%
\pgfpathlineto{\pgfqpoint{5.132134in}{2.586892in}}%
\pgfpathlineto{\pgfqpoint{5.079090in}{2.586892in}}%
\pgfpathlineto{\pgfqpoint{5.026046in}{2.586892in}}%
\pgfpathlineto{\pgfqpoint{4.973002in}{2.586892in}}%
\pgfpathlineto{\pgfqpoint{4.919958in}{2.586892in}}%
\pgfpathlineto{\pgfqpoint{4.866914in}{2.586892in}}%
\pgfpathlineto{\pgfqpoint{4.813870in}{2.586892in}}%
\pgfpathlineto{\pgfqpoint{4.760826in}{2.586892in}}%
\pgfpathlineto{\pgfqpoint{4.707782in}{2.586892in}}%
\pgfpathlineto{\pgfqpoint{4.654738in}{2.586892in}}%
\pgfpathlineto{\pgfqpoint{4.601694in}{2.586892in}}%
\pgfpathlineto{\pgfqpoint{4.548650in}{2.586892in}}%
\pgfpathlineto{\pgfqpoint{4.495607in}{2.586892in}}%
\pgfpathlineto{\pgfqpoint{4.442563in}{2.586892in}}%
\pgfpathlineto{\pgfqpoint{4.389519in}{2.586892in}}%
\pgfpathlineto{\pgfqpoint{4.336475in}{2.586892in}}%
\pgfpathlineto{\pgfqpoint{4.283431in}{2.586892in}}%
\pgfpathlineto{\pgfqpoint{4.230387in}{2.586892in}}%
\pgfpathlineto{\pgfqpoint{4.177343in}{2.586892in}}%
\pgfpathlineto{\pgfqpoint{4.124299in}{2.586892in}}%
\pgfpathlineto{\pgfqpoint{4.071255in}{2.586892in}}%
\pgfpathlineto{\pgfqpoint{4.018211in}{2.586892in}}%
\pgfpathlineto{\pgfqpoint{3.965167in}{2.586892in}}%
\pgfpathlineto{\pgfqpoint{3.912123in}{2.586892in}}%
\pgfpathlineto{\pgfqpoint{3.859079in}{2.586892in}}%
\pgfpathlineto{\pgfqpoint{3.806035in}{2.586892in}}%
\pgfpathlineto{\pgfqpoint{3.752991in}{2.586892in}}%
\pgfpathlineto{\pgfqpoint{3.699948in}{2.586892in}}%
\pgfpathlineto{\pgfqpoint{3.646904in}{2.586892in}}%
\pgfpathlineto{\pgfqpoint{3.593860in}{2.586892in}}%
\pgfpathlineto{\pgfqpoint{3.540816in}{2.586892in}}%
\pgfpathlineto{\pgfqpoint{3.487772in}{2.586892in}}%
\pgfpathlineto{\pgfqpoint{3.434728in}{2.586892in}}%
\pgfpathlineto{\pgfqpoint{3.381684in}{2.586892in}}%
\pgfpathlineto{\pgfqpoint{3.328640in}{2.586892in}}%
\pgfpathlineto{\pgfqpoint{3.275596in}{2.586892in}}%
\pgfpathlineto{\pgfqpoint{3.222552in}{2.586892in}}%
\pgfpathlineto{\pgfqpoint{3.169508in}{2.586892in}}%
\pgfpathlineto{\pgfqpoint{3.116464in}{2.586892in}}%
\pgfpathlineto{\pgfqpoint{3.063420in}{2.586892in}}%
\pgfpathlineto{\pgfqpoint{3.010376in}{2.586892in}}%
\pgfpathlineto{\pgfqpoint{2.957332in}{2.586892in}}%
\pgfpathlineto{\pgfqpoint{2.904288in}{2.586892in}}%
\pgfpathlineto{\pgfqpoint{2.851245in}{2.586892in}}%
\pgfpathlineto{\pgfqpoint{2.798201in}{2.586892in}}%
\pgfpathlineto{\pgfqpoint{2.745157in}{2.586892in}}%
\pgfpathlineto{\pgfqpoint{2.692113in}{2.586892in}}%
\pgfpathlineto{\pgfqpoint{2.639069in}{2.586892in}}%
\pgfpathlineto{\pgfqpoint{2.586025in}{2.586892in}}%
\pgfpathlineto{\pgfqpoint{2.532981in}{2.586892in}}%
\pgfpathlineto{\pgfqpoint{2.479937in}{2.586892in}}%
\pgfpathlineto{\pgfqpoint{2.426893in}{2.586892in}}%
\pgfpathlineto{\pgfqpoint{2.373849in}{2.586892in}}%
\pgfpathlineto{\pgfqpoint{2.320805in}{2.586892in}}%
\pgfpathlineto{\pgfqpoint{2.267761in}{2.586892in}}%
\pgfpathlineto{\pgfqpoint{2.214717in}{2.586892in}}%
\pgfpathlineto{\pgfqpoint{2.161673in}{2.586892in}}%
\pgfpathlineto{\pgfqpoint{2.108629in}{2.586892in}}%
\pgfpathlineto{\pgfqpoint{2.055586in}{2.586892in}}%
\pgfpathlineto{\pgfqpoint{2.002542in}{2.586892in}}%
\pgfpathlineto{\pgfqpoint{1.949498in}{2.586892in}}%
\pgfpathlineto{\pgfqpoint{1.896454in}{2.586892in}}%
\pgfpathlineto{\pgfqpoint{1.843410in}{2.586892in}}%
\pgfpathlineto{\pgfqpoint{1.790366in}{2.586892in}}%
\pgfpathlineto{\pgfqpoint{1.737322in}{2.586892in}}%
\pgfpathlineto{\pgfqpoint{1.684278in}{2.586892in}}%
\pgfpathlineto{\pgfqpoint{1.631234in}{2.586892in}}%
\pgfpathlineto{\pgfqpoint{1.578190in}{2.586892in}}%
\pgfpathlineto{\pgfqpoint{1.525146in}{2.586892in}}%
\pgfpathlineto{\pgfqpoint{1.472102in}{2.586892in}}%
\pgfpathlineto{\pgfqpoint{1.419058in}{2.586892in}}%
\pgfpathlineto{\pgfqpoint{1.366014in}{2.586892in}}%
\pgfpathlineto{\pgfqpoint{1.312970in}{2.586892in}}%
\pgfpathlineto{\pgfqpoint{1.259927in}{2.586892in}}%
\pgfpathlineto{\pgfqpoint{1.206883in}{2.586892in}}%
\pgfpathlineto{\pgfqpoint{1.153839in}{2.586892in}}%
\pgfpathlineto{\pgfqpoint{1.100795in}{2.586892in}}%
\pgfpathlineto{\pgfqpoint{1.047751in}{2.586892in}}%
\pgfpathlineto{\pgfqpoint{0.994707in}{2.586892in}}%
\pgfpathlineto{\pgfqpoint{0.941663in}{2.586892in}}%
\pgfpathlineto{\pgfqpoint{0.941663in}{2.586892in}}%
\pgfpathclose%
\pgfusepath{stroke,fill}%
}%
\begin{pgfscope}%
\pgfsys@transformshift{0.000000in}{0.000000in}%
\pgfsys@useobject{currentmarker}{}%
\end{pgfscope}%
\end{pgfscope}%
\begin{pgfscope}%
\pgfpathrectangle{\pgfqpoint{0.941663in}{0.670138in}}{\pgfqpoint{8.858337in}{3.465625in}}%
\pgfusepath{clip}%
\pgfsetrectcap%
\pgfsetroundjoin%
\pgfsetlinewidth{1.505625pt}%
\definecolor{currentstroke}{rgb}{0.501961,0.000000,0.501961}%
\pgfsetstrokecolor{currentstroke}%
\pgfsetdash{}{0pt}%
\pgfpathmoveto{\pgfqpoint{0.941663in}{2.586892in}}%
\pgfpathlineto{\pgfqpoint{1.206883in}{2.586892in}}%
\pgfpathlineto{\pgfqpoint{1.259927in}{2.749185in}}%
\pgfpathlineto{\pgfqpoint{1.312970in}{2.586892in}}%
\pgfpathlineto{\pgfqpoint{1.578190in}{2.586892in}}%
\pgfpathlineto{\pgfqpoint{1.631234in}{2.802326in}}%
\pgfpathlineto{\pgfqpoint{1.684278in}{2.586892in}}%
\pgfpathlineto{\pgfqpoint{1.790366in}{2.586892in}}%
\pgfpathlineto{\pgfqpoint{1.843410in}{2.934727in}}%
\pgfpathlineto{\pgfqpoint{1.896454in}{2.934727in}}%
\pgfpathlineto{\pgfqpoint{1.949498in}{2.586892in}}%
\pgfpathlineto{\pgfqpoint{2.002542in}{2.934727in}}%
\pgfpathlineto{\pgfqpoint{2.055586in}{2.586892in}}%
\pgfpathlineto{\pgfqpoint{2.161673in}{2.586892in}}%
\pgfpathlineto{\pgfqpoint{2.214717in}{2.910735in}}%
\pgfpathlineto{\pgfqpoint{2.267761in}{2.586892in}}%
\pgfpathlineto{\pgfqpoint{2.320805in}{2.586892in}}%
\pgfpathlineto{\pgfqpoint{2.373849in}{2.844885in}}%
\pgfpathlineto{\pgfqpoint{2.426893in}{2.934727in}}%
\pgfpathlineto{\pgfqpoint{2.479937in}{2.586892in}}%
\pgfpathlineto{\pgfqpoint{2.904288in}{2.586892in}}%
\pgfpathlineto{\pgfqpoint{2.957332in}{2.934727in}}%
\pgfpathlineto{\pgfqpoint{3.010376in}{2.586892in}}%
\pgfpathlineto{\pgfqpoint{3.063420in}{2.909016in}}%
\pgfpathlineto{\pgfqpoint{3.116464in}{2.586892in}}%
\pgfpathlineto{\pgfqpoint{3.434728in}{2.586892in}}%
\pgfpathlineto{\pgfqpoint{3.487772in}{2.710315in}}%
\pgfpathlineto{\pgfqpoint{3.540816in}{2.908402in}}%
\pgfpathlineto{\pgfqpoint{3.593860in}{2.586892in}}%
\pgfpathlineto{\pgfqpoint{4.283431in}{2.586892in}}%
\pgfpathlineto{\pgfqpoint{4.336475in}{2.934727in}}%
\pgfpathlineto{\pgfqpoint{4.389519in}{2.586892in}}%
\pgfpathlineto{\pgfqpoint{4.442563in}{2.586892in}}%
\pgfpathlineto{\pgfqpoint{4.495607in}{2.934727in}}%
\pgfpathlineto{\pgfqpoint{4.548650in}{2.586892in}}%
\pgfpathlineto{\pgfqpoint{4.601694in}{2.927863in}}%
\pgfpathlineto{\pgfqpoint{4.654738in}{2.586892in}}%
\pgfpathlineto{\pgfqpoint{4.707782in}{2.934727in}}%
\pgfpathlineto{\pgfqpoint{4.760826in}{2.586892in}}%
\pgfpathlineto{\pgfqpoint{4.813870in}{2.586892in}}%
\pgfpathlineto{\pgfqpoint{4.866914in}{2.884645in}}%
\pgfpathlineto{\pgfqpoint{4.919958in}{2.863950in}}%
\pgfpathlineto{\pgfqpoint{4.973002in}{2.586892in}}%
\pgfpathlineto{\pgfqpoint{5.026046in}{2.758322in}}%
\pgfpathlineto{\pgfqpoint{5.079090in}{2.586892in}}%
\pgfpathlineto{\pgfqpoint{5.132134in}{2.824147in}}%
\pgfpathlineto{\pgfqpoint{5.185178in}{2.910114in}}%
\pgfpathlineto{\pgfqpoint{5.238222in}{2.837414in}}%
\pgfpathlineto{\pgfqpoint{5.291266in}{2.586892in}}%
\pgfpathlineto{\pgfqpoint{5.344309in}{2.705202in}}%
\pgfpathlineto{\pgfqpoint{5.397353in}{2.660769in}}%
\pgfpathlineto{\pgfqpoint{5.450397in}{2.934727in}}%
\pgfpathlineto{\pgfqpoint{5.503441in}{2.586892in}}%
\pgfpathlineto{\pgfqpoint{5.556485in}{2.586892in}}%
\pgfpathlineto{\pgfqpoint{5.609529in}{2.850209in}}%
\pgfpathlineto{\pgfqpoint{5.662573in}{2.711726in}}%
\pgfpathlineto{\pgfqpoint{5.715617in}{2.586892in}}%
\pgfpathlineto{\pgfqpoint{5.768661in}{2.934727in}}%
\pgfpathlineto{\pgfqpoint{5.874749in}{2.934727in}}%
\pgfpathlineto{\pgfqpoint{5.927793in}{2.643205in}}%
\pgfpathlineto{\pgfqpoint{5.980837in}{2.586892in}}%
\pgfpathlineto{\pgfqpoint{6.139969in}{2.586892in}}%
\pgfpathlineto{\pgfqpoint{6.193012in}{2.934727in}}%
\pgfpathlineto{\pgfqpoint{6.246056in}{2.934727in}}%
\pgfpathlineto{\pgfqpoint{6.299100in}{2.586892in}}%
\pgfpathlineto{\pgfqpoint{6.352144in}{2.586892in}}%
\pgfpathlineto{\pgfqpoint{6.405188in}{2.699818in}}%
\pgfpathlineto{\pgfqpoint{6.458232in}{2.586892in}}%
\pgfpathlineto{\pgfqpoint{6.829540in}{2.586892in}}%
\pgfpathlineto{\pgfqpoint{6.882584in}{2.934727in}}%
\pgfpathlineto{\pgfqpoint{6.935628in}{2.586892in}}%
\pgfpathlineto{\pgfqpoint{7.147803in}{2.586892in}}%
\pgfpathlineto{\pgfqpoint{7.200847in}{2.934727in}}%
\pgfpathlineto{\pgfqpoint{7.253891in}{2.586892in}}%
\pgfpathlineto{\pgfqpoint{7.306935in}{2.586892in}}%
\pgfpathlineto{\pgfqpoint{7.359979in}{2.806406in}}%
\pgfpathlineto{\pgfqpoint{7.413023in}{2.865454in}}%
\pgfpathlineto{\pgfqpoint{7.466067in}{2.586892in}}%
\pgfpathlineto{\pgfqpoint{7.519111in}{2.586892in}}%
\pgfpathlineto{\pgfqpoint{7.572155in}{2.823725in}}%
\pgfpathlineto{\pgfqpoint{7.625199in}{2.934727in}}%
\pgfpathlineto{\pgfqpoint{7.678243in}{2.586892in}}%
\pgfpathlineto{\pgfqpoint{7.996506in}{2.586892in}}%
\pgfpathlineto{\pgfqpoint{8.049550in}{2.934727in}}%
\pgfpathlineto{\pgfqpoint{8.102594in}{2.934727in}}%
\pgfpathlineto{\pgfqpoint{8.155638in}{2.586892in}}%
\pgfpathlineto{\pgfqpoint{8.208682in}{2.586892in}}%
\pgfpathlineto{\pgfqpoint{8.261726in}{2.934727in}}%
\pgfpathlineto{\pgfqpoint{8.314770in}{2.934727in}}%
\pgfpathlineto{\pgfqpoint{8.367814in}{2.586892in}}%
\pgfpathlineto{\pgfqpoint{8.420858in}{2.934727in}}%
\pgfpathlineto{\pgfqpoint{8.473902in}{2.586892in}}%
\pgfpathlineto{\pgfqpoint{8.526946in}{2.586892in}}%
\pgfpathlineto{\pgfqpoint{8.579990in}{2.934727in}}%
\pgfpathlineto{\pgfqpoint{8.633033in}{2.856358in}}%
\pgfpathlineto{\pgfqpoint{8.686077in}{2.934727in}}%
\pgfpathlineto{\pgfqpoint{8.739121in}{2.825404in}}%
\pgfpathlineto{\pgfqpoint{8.792165in}{2.586892in}}%
\pgfpathlineto{\pgfqpoint{8.845209in}{2.634872in}}%
\pgfpathlineto{\pgfqpoint{8.898253in}{2.689958in}}%
\pgfpathlineto{\pgfqpoint{8.951297in}{2.586892in}}%
\pgfpathlineto{\pgfqpoint{9.110429in}{2.586892in}}%
\pgfpathlineto{\pgfqpoint{9.163473in}{2.934727in}}%
\pgfpathlineto{\pgfqpoint{9.322605in}{2.934727in}}%
\pgfpathlineto{\pgfqpoint{9.375649in}{2.695480in}}%
\pgfpathlineto{\pgfqpoint{9.428692in}{2.586892in}}%
\pgfpathlineto{\pgfqpoint{9.481736in}{2.586892in}}%
\pgfpathlineto{\pgfqpoint{9.534780in}{2.934727in}}%
\pgfpathlineto{\pgfqpoint{9.587824in}{2.916055in}}%
\pgfpathlineto{\pgfqpoint{9.640868in}{2.586892in}}%
\pgfpathlineto{\pgfqpoint{9.693912in}{2.865140in}}%
\pgfpathlineto{\pgfqpoint{9.746956in}{2.608608in}}%
\pgfpathlineto{\pgfqpoint{9.800000in}{2.588586in}}%
\pgfpathlineto{\pgfqpoint{9.800000in}{2.588586in}}%
\pgfusepath{stroke}%
\end{pgfscope}%
\begin{pgfscope}%
\pgfpathrectangle{\pgfqpoint{0.941663in}{0.670138in}}{\pgfqpoint{8.858337in}{3.465625in}}%
\pgfusepath{clip}%
\pgfsetbuttcap%
\pgfsetroundjoin%
\definecolor{currentfill}{rgb}{0.501961,0.000000,0.501961}%
\pgfsetfillcolor{currentfill}%
\pgfsetlinewidth{1.003750pt}%
\definecolor{currentstroke}{rgb}{0.501961,0.000000,0.501961}%
\pgfsetstrokecolor{currentstroke}%
\pgfsetdash{}{0pt}%
\pgfsys@defobject{currentmarker}{\pgfqpoint{0.941663in}{2.586892in}}{\pgfqpoint{9.800000in}{2.934727in}}{%
\pgfpathmoveto{\pgfqpoint{0.941663in}{2.586892in}}%
\pgfpathlineto{\pgfqpoint{0.941663in}{2.586892in}}%
\pgfpathlineto{\pgfqpoint{0.994707in}{2.586892in}}%
\pgfpathlineto{\pgfqpoint{1.047751in}{2.586892in}}%
\pgfpathlineto{\pgfqpoint{1.100795in}{2.586892in}}%
\pgfpathlineto{\pgfqpoint{1.153839in}{2.586892in}}%
\pgfpathlineto{\pgfqpoint{1.206883in}{2.586892in}}%
\pgfpathlineto{\pgfqpoint{1.259927in}{2.586892in}}%
\pgfpathlineto{\pgfqpoint{1.312970in}{2.586892in}}%
\pgfpathlineto{\pgfqpoint{1.366014in}{2.586892in}}%
\pgfpathlineto{\pgfqpoint{1.419058in}{2.586892in}}%
\pgfpathlineto{\pgfqpoint{1.472102in}{2.586892in}}%
\pgfpathlineto{\pgfqpoint{1.525146in}{2.586892in}}%
\pgfpathlineto{\pgfqpoint{1.578190in}{2.586892in}}%
\pgfpathlineto{\pgfqpoint{1.631234in}{2.586892in}}%
\pgfpathlineto{\pgfqpoint{1.684278in}{2.586892in}}%
\pgfpathlineto{\pgfqpoint{1.737322in}{2.586892in}}%
\pgfpathlineto{\pgfqpoint{1.790366in}{2.586892in}}%
\pgfpathlineto{\pgfqpoint{1.843410in}{2.586892in}}%
\pgfpathlineto{\pgfqpoint{1.896454in}{2.586892in}}%
\pgfpathlineto{\pgfqpoint{1.949498in}{2.586892in}}%
\pgfpathlineto{\pgfqpoint{2.002542in}{2.586892in}}%
\pgfpathlineto{\pgfqpoint{2.055586in}{2.586892in}}%
\pgfpathlineto{\pgfqpoint{2.108629in}{2.586892in}}%
\pgfpathlineto{\pgfqpoint{2.161673in}{2.586892in}}%
\pgfpathlineto{\pgfqpoint{2.214717in}{2.586892in}}%
\pgfpathlineto{\pgfqpoint{2.267761in}{2.586892in}}%
\pgfpathlineto{\pgfqpoint{2.320805in}{2.586892in}}%
\pgfpathlineto{\pgfqpoint{2.373849in}{2.586892in}}%
\pgfpathlineto{\pgfqpoint{2.426893in}{2.586892in}}%
\pgfpathlineto{\pgfqpoint{2.479937in}{2.586892in}}%
\pgfpathlineto{\pgfqpoint{2.532981in}{2.586892in}}%
\pgfpathlineto{\pgfqpoint{2.586025in}{2.586892in}}%
\pgfpathlineto{\pgfqpoint{2.639069in}{2.586892in}}%
\pgfpathlineto{\pgfqpoint{2.692113in}{2.586892in}}%
\pgfpathlineto{\pgfqpoint{2.745157in}{2.586892in}}%
\pgfpathlineto{\pgfqpoint{2.798201in}{2.586892in}}%
\pgfpathlineto{\pgfqpoint{2.851245in}{2.586892in}}%
\pgfpathlineto{\pgfqpoint{2.904288in}{2.586892in}}%
\pgfpathlineto{\pgfqpoint{2.957332in}{2.586892in}}%
\pgfpathlineto{\pgfqpoint{3.010376in}{2.586892in}}%
\pgfpathlineto{\pgfqpoint{3.063420in}{2.586892in}}%
\pgfpathlineto{\pgfqpoint{3.116464in}{2.586892in}}%
\pgfpathlineto{\pgfqpoint{3.169508in}{2.586892in}}%
\pgfpathlineto{\pgfqpoint{3.222552in}{2.586892in}}%
\pgfpathlineto{\pgfqpoint{3.275596in}{2.586892in}}%
\pgfpathlineto{\pgfqpoint{3.328640in}{2.586892in}}%
\pgfpathlineto{\pgfqpoint{3.381684in}{2.586892in}}%
\pgfpathlineto{\pgfqpoint{3.434728in}{2.586892in}}%
\pgfpathlineto{\pgfqpoint{3.487772in}{2.586892in}}%
\pgfpathlineto{\pgfqpoint{3.540816in}{2.586892in}}%
\pgfpathlineto{\pgfqpoint{3.593860in}{2.586892in}}%
\pgfpathlineto{\pgfqpoint{3.646904in}{2.586892in}}%
\pgfpathlineto{\pgfqpoint{3.699948in}{2.586892in}}%
\pgfpathlineto{\pgfqpoint{3.752991in}{2.586892in}}%
\pgfpathlineto{\pgfqpoint{3.806035in}{2.586892in}}%
\pgfpathlineto{\pgfqpoint{3.859079in}{2.586892in}}%
\pgfpathlineto{\pgfqpoint{3.912123in}{2.586892in}}%
\pgfpathlineto{\pgfqpoint{3.965167in}{2.586892in}}%
\pgfpathlineto{\pgfqpoint{4.018211in}{2.586892in}}%
\pgfpathlineto{\pgfqpoint{4.071255in}{2.586892in}}%
\pgfpathlineto{\pgfqpoint{4.124299in}{2.586892in}}%
\pgfpathlineto{\pgfqpoint{4.177343in}{2.586892in}}%
\pgfpathlineto{\pgfqpoint{4.230387in}{2.586892in}}%
\pgfpathlineto{\pgfqpoint{4.283431in}{2.586892in}}%
\pgfpathlineto{\pgfqpoint{4.336475in}{2.586892in}}%
\pgfpathlineto{\pgfqpoint{4.389519in}{2.586892in}}%
\pgfpathlineto{\pgfqpoint{4.442563in}{2.586892in}}%
\pgfpathlineto{\pgfqpoint{4.495607in}{2.586892in}}%
\pgfpathlineto{\pgfqpoint{4.548650in}{2.586892in}}%
\pgfpathlineto{\pgfqpoint{4.601694in}{2.586892in}}%
\pgfpathlineto{\pgfqpoint{4.654738in}{2.586892in}}%
\pgfpathlineto{\pgfqpoint{4.707782in}{2.586892in}}%
\pgfpathlineto{\pgfqpoint{4.760826in}{2.586892in}}%
\pgfpathlineto{\pgfqpoint{4.813870in}{2.586892in}}%
\pgfpathlineto{\pgfqpoint{4.866914in}{2.586892in}}%
\pgfpathlineto{\pgfqpoint{4.919958in}{2.586892in}}%
\pgfpathlineto{\pgfqpoint{4.973002in}{2.586892in}}%
\pgfpathlineto{\pgfqpoint{5.026046in}{2.586892in}}%
\pgfpathlineto{\pgfqpoint{5.079090in}{2.586892in}}%
\pgfpathlineto{\pgfqpoint{5.132134in}{2.586892in}}%
\pgfpathlineto{\pgfqpoint{5.185178in}{2.586892in}}%
\pgfpathlineto{\pgfqpoint{5.238222in}{2.586892in}}%
\pgfpathlineto{\pgfqpoint{5.291266in}{2.586892in}}%
\pgfpathlineto{\pgfqpoint{5.344309in}{2.586892in}}%
\pgfpathlineto{\pgfqpoint{5.397353in}{2.586892in}}%
\pgfpathlineto{\pgfqpoint{5.450397in}{2.586892in}}%
\pgfpathlineto{\pgfqpoint{5.503441in}{2.586892in}}%
\pgfpathlineto{\pgfqpoint{5.556485in}{2.586892in}}%
\pgfpathlineto{\pgfqpoint{5.609529in}{2.586892in}}%
\pgfpathlineto{\pgfqpoint{5.662573in}{2.586892in}}%
\pgfpathlineto{\pgfqpoint{5.715617in}{2.586892in}}%
\pgfpathlineto{\pgfqpoint{5.768661in}{2.586892in}}%
\pgfpathlineto{\pgfqpoint{5.821705in}{2.586892in}}%
\pgfpathlineto{\pgfqpoint{5.874749in}{2.586892in}}%
\pgfpathlineto{\pgfqpoint{5.927793in}{2.586892in}}%
\pgfpathlineto{\pgfqpoint{5.980837in}{2.586892in}}%
\pgfpathlineto{\pgfqpoint{6.033881in}{2.586892in}}%
\pgfpathlineto{\pgfqpoint{6.086925in}{2.586892in}}%
\pgfpathlineto{\pgfqpoint{6.139969in}{2.586892in}}%
\pgfpathlineto{\pgfqpoint{6.193012in}{2.586892in}}%
\pgfpathlineto{\pgfqpoint{6.246056in}{2.586892in}}%
\pgfpathlineto{\pgfqpoint{6.299100in}{2.586892in}}%
\pgfpathlineto{\pgfqpoint{6.352144in}{2.586892in}}%
\pgfpathlineto{\pgfqpoint{6.405188in}{2.586892in}}%
\pgfpathlineto{\pgfqpoint{6.458232in}{2.586892in}}%
\pgfpathlineto{\pgfqpoint{6.511276in}{2.586892in}}%
\pgfpathlineto{\pgfqpoint{6.564320in}{2.586892in}}%
\pgfpathlineto{\pgfqpoint{6.617364in}{2.586892in}}%
\pgfpathlineto{\pgfqpoint{6.670408in}{2.586892in}}%
\pgfpathlineto{\pgfqpoint{6.723452in}{2.586892in}}%
\pgfpathlineto{\pgfqpoint{6.776496in}{2.586892in}}%
\pgfpathlineto{\pgfqpoint{6.829540in}{2.586892in}}%
\pgfpathlineto{\pgfqpoint{6.882584in}{2.586892in}}%
\pgfpathlineto{\pgfqpoint{6.935628in}{2.586892in}}%
\pgfpathlineto{\pgfqpoint{6.988671in}{2.586892in}}%
\pgfpathlineto{\pgfqpoint{7.041715in}{2.586892in}}%
\pgfpathlineto{\pgfqpoint{7.094759in}{2.586892in}}%
\pgfpathlineto{\pgfqpoint{7.147803in}{2.586892in}}%
\pgfpathlineto{\pgfqpoint{7.200847in}{2.586892in}}%
\pgfpathlineto{\pgfqpoint{7.253891in}{2.586892in}}%
\pgfpathlineto{\pgfqpoint{7.306935in}{2.586892in}}%
\pgfpathlineto{\pgfqpoint{7.359979in}{2.586892in}}%
\pgfpathlineto{\pgfqpoint{7.413023in}{2.586892in}}%
\pgfpathlineto{\pgfqpoint{7.466067in}{2.586892in}}%
\pgfpathlineto{\pgfqpoint{7.519111in}{2.586892in}}%
\pgfpathlineto{\pgfqpoint{7.572155in}{2.586892in}}%
\pgfpathlineto{\pgfqpoint{7.625199in}{2.586892in}}%
\pgfpathlineto{\pgfqpoint{7.678243in}{2.586892in}}%
\pgfpathlineto{\pgfqpoint{7.731287in}{2.586892in}}%
\pgfpathlineto{\pgfqpoint{7.784330in}{2.586892in}}%
\pgfpathlineto{\pgfqpoint{7.837374in}{2.586892in}}%
\pgfpathlineto{\pgfqpoint{7.890418in}{2.586892in}}%
\pgfpathlineto{\pgfqpoint{7.943462in}{2.586892in}}%
\pgfpathlineto{\pgfqpoint{7.996506in}{2.586892in}}%
\pgfpathlineto{\pgfqpoint{8.049550in}{2.586892in}}%
\pgfpathlineto{\pgfqpoint{8.102594in}{2.586892in}}%
\pgfpathlineto{\pgfqpoint{8.155638in}{2.586892in}}%
\pgfpathlineto{\pgfqpoint{8.208682in}{2.586892in}}%
\pgfpathlineto{\pgfqpoint{8.261726in}{2.586892in}}%
\pgfpathlineto{\pgfqpoint{8.314770in}{2.586892in}}%
\pgfpathlineto{\pgfqpoint{8.367814in}{2.586892in}}%
\pgfpathlineto{\pgfqpoint{8.420858in}{2.586892in}}%
\pgfpathlineto{\pgfqpoint{8.473902in}{2.586892in}}%
\pgfpathlineto{\pgfqpoint{8.526946in}{2.586892in}}%
\pgfpathlineto{\pgfqpoint{8.579990in}{2.586892in}}%
\pgfpathlineto{\pgfqpoint{8.633033in}{2.586892in}}%
\pgfpathlineto{\pgfqpoint{8.686077in}{2.586892in}}%
\pgfpathlineto{\pgfqpoint{8.739121in}{2.586892in}}%
\pgfpathlineto{\pgfqpoint{8.792165in}{2.586892in}}%
\pgfpathlineto{\pgfqpoint{8.845209in}{2.586892in}}%
\pgfpathlineto{\pgfqpoint{8.898253in}{2.586892in}}%
\pgfpathlineto{\pgfqpoint{8.951297in}{2.586892in}}%
\pgfpathlineto{\pgfqpoint{9.004341in}{2.586892in}}%
\pgfpathlineto{\pgfqpoint{9.057385in}{2.586892in}}%
\pgfpathlineto{\pgfqpoint{9.110429in}{2.586892in}}%
\pgfpathlineto{\pgfqpoint{9.163473in}{2.586892in}}%
\pgfpathlineto{\pgfqpoint{9.216517in}{2.586892in}}%
\pgfpathlineto{\pgfqpoint{9.269561in}{2.586892in}}%
\pgfpathlineto{\pgfqpoint{9.322605in}{2.586892in}}%
\pgfpathlineto{\pgfqpoint{9.375649in}{2.586892in}}%
\pgfpathlineto{\pgfqpoint{9.428692in}{2.586892in}}%
\pgfpathlineto{\pgfqpoint{9.481736in}{2.586892in}}%
\pgfpathlineto{\pgfqpoint{9.534780in}{2.586892in}}%
\pgfpathlineto{\pgfqpoint{9.587824in}{2.586892in}}%
\pgfpathlineto{\pgfqpoint{9.640868in}{2.586892in}}%
\pgfpathlineto{\pgfqpoint{9.693912in}{2.586892in}}%
\pgfpathlineto{\pgfqpoint{9.746956in}{2.586892in}}%
\pgfpathlineto{\pgfqpoint{9.800000in}{2.586892in}}%
\pgfpathlineto{\pgfqpoint{9.800000in}{2.588586in}}%
\pgfpathlineto{\pgfqpoint{9.800000in}{2.588586in}}%
\pgfpathlineto{\pgfqpoint{9.746956in}{2.608608in}}%
\pgfpathlineto{\pgfqpoint{9.693912in}{2.865140in}}%
\pgfpathlineto{\pgfqpoint{9.640868in}{2.586892in}}%
\pgfpathlineto{\pgfqpoint{9.587824in}{2.916055in}}%
\pgfpathlineto{\pgfqpoint{9.534780in}{2.934727in}}%
\pgfpathlineto{\pgfqpoint{9.481736in}{2.586892in}}%
\pgfpathlineto{\pgfqpoint{9.428692in}{2.586892in}}%
\pgfpathlineto{\pgfqpoint{9.375649in}{2.695480in}}%
\pgfpathlineto{\pgfqpoint{9.322605in}{2.934727in}}%
\pgfpathlineto{\pgfqpoint{9.269561in}{2.934727in}}%
\pgfpathlineto{\pgfqpoint{9.216517in}{2.934727in}}%
\pgfpathlineto{\pgfqpoint{9.163473in}{2.934727in}}%
\pgfpathlineto{\pgfqpoint{9.110429in}{2.586892in}}%
\pgfpathlineto{\pgfqpoint{9.057385in}{2.586892in}}%
\pgfpathlineto{\pgfqpoint{9.004341in}{2.586892in}}%
\pgfpathlineto{\pgfqpoint{8.951297in}{2.586892in}}%
\pgfpathlineto{\pgfqpoint{8.898253in}{2.689958in}}%
\pgfpathlineto{\pgfqpoint{8.845209in}{2.634872in}}%
\pgfpathlineto{\pgfqpoint{8.792165in}{2.586892in}}%
\pgfpathlineto{\pgfqpoint{8.739121in}{2.825404in}}%
\pgfpathlineto{\pgfqpoint{8.686077in}{2.934727in}}%
\pgfpathlineto{\pgfqpoint{8.633033in}{2.856358in}}%
\pgfpathlineto{\pgfqpoint{8.579990in}{2.934727in}}%
\pgfpathlineto{\pgfqpoint{8.526946in}{2.586892in}}%
\pgfpathlineto{\pgfqpoint{8.473902in}{2.586892in}}%
\pgfpathlineto{\pgfqpoint{8.420858in}{2.934727in}}%
\pgfpathlineto{\pgfqpoint{8.367814in}{2.586892in}}%
\pgfpathlineto{\pgfqpoint{8.314770in}{2.934727in}}%
\pgfpathlineto{\pgfqpoint{8.261726in}{2.934727in}}%
\pgfpathlineto{\pgfqpoint{8.208682in}{2.586892in}}%
\pgfpathlineto{\pgfqpoint{8.155638in}{2.586892in}}%
\pgfpathlineto{\pgfqpoint{8.102594in}{2.934727in}}%
\pgfpathlineto{\pgfqpoint{8.049550in}{2.934727in}}%
\pgfpathlineto{\pgfqpoint{7.996506in}{2.586892in}}%
\pgfpathlineto{\pgfqpoint{7.943462in}{2.586892in}}%
\pgfpathlineto{\pgfqpoint{7.890418in}{2.586892in}}%
\pgfpathlineto{\pgfqpoint{7.837374in}{2.586892in}}%
\pgfpathlineto{\pgfqpoint{7.784330in}{2.586892in}}%
\pgfpathlineto{\pgfqpoint{7.731287in}{2.586892in}}%
\pgfpathlineto{\pgfqpoint{7.678243in}{2.586892in}}%
\pgfpathlineto{\pgfqpoint{7.625199in}{2.934727in}}%
\pgfpathlineto{\pgfqpoint{7.572155in}{2.823725in}}%
\pgfpathlineto{\pgfqpoint{7.519111in}{2.586892in}}%
\pgfpathlineto{\pgfqpoint{7.466067in}{2.586892in}}%
\pgfpathlineto{\pgfqpoint{7.413023in}{2.865454in}}%
\pgfpathlineto{\pgfqpoint{7.359979in}{2.806406in}}%
\pgfpathlineto{\pgfqpoint{7.306935in}{2.586892in}}%
\pgfpathlineto{\pgfqpoint{7.253891in}{2.586892in}}%
\pgfpathlineto{\pgfqpoint{7.200847in}{2.934727in}}%
\pgfpathlineto{\pgfqpoint{7.147803in}{2.586892in}}%
\pgfpathlineto{\pgfqpoint{7.094759in}{2.586892in}}%
\pgfpathlineto{\pgfqpoint{7.041715in}{2.586892in}}%
\pgfpathlineto{\pgfqpoint{6.988671in}{2.586892in}}%
\pgfpathlineto{\pgfqpoint{6.935628in}{2.586892in}}%
\pgfpathlineto{\pgfqpoint{6.882584in}{2.934727in}}%
\pgfpathlineto{\pgfqpoint{6.829540in}{2.586892in}}%
\pgfpathlineto{\pgfqpoint{6.776496in}{2.586892in}}%
\pgfpathlineto{\pgfqpoint{6.723452in}{2.586892in}}%
\pgfpathlineto{\pgfqpoint{6.670408in}{2.586892in}}%
\pgfpathlineto{\pgfqpoint{6.617364in}{2.586892in}}%
\pgfpathlineto{\pgfqpoint{6.564320in}{2.586892in}}%
\pgfpathlineto{\pgfqpoint{6.511276in}{2.586892in}}%
\pgfpathlineto{\pgfqpoint{6.458232in}{2.586892in}}%
\pgfpathlineto{\pgfqpoint{6.405188in}{2.699818in}}%
\pgfpathlineto{\pgfqpoint{6.352144in}{2.586892in}}%
\pgfpathlineto{\pgfqpoint{6.299100in}{2.586892in}}%
\pgfpathlineto{\pgfqpoint{6.246056in}{2.934727in}}%
\pgfpathlineto{\pgfqpoint{6.193012in}{2.934727in}}%
\pgfpathlineto{\pgfqpoint{6.139969in}{2.586892in}}%
\pgfpathlineto{\pgfqpoint{6.086925in}{2.586892in}}%
\pgfpathlineto{\pgfqpoint{6.033881in}{2.586892in}}%
\pgfpathlineto{\pgfqpoint{5.980837in}{2.586892in}}%
\pgfpathlineto{\pgfqpoint{5.927793in}{2.643205in}}%
\pgfpathlineto{\pgfqpoint{5.874749in}{2.934727in}}%
\pgfpathlineto{\pgfqpoint{5.821705in}{2.934727in}}%
\pgfpathlineto{\pgfqpoint{5.768661in}{2.934727in}}%
\pgfpathlineto{\pgfqpoint{5.715617in}{2.586892in}}%
\pgfpathlineto{\pgfqpoint{5.662573in}{2.711726in}}%
\pgfpathlineto{\pgfqpoint{5.609529in}{2.850209in}}%
\pgfpathlineto{\pgfqpoint{5.556485in}{2.586892in}}%
\pgfpathlineto{\pgfqpoint{5.503441in}{2.586892in}}%
\pgfpathlineto{\pgfqpoint{5.450397in}{2.934727in}}%
\pgfpathlineto{\pgfqpoint{5.397353in}{2.660769in}}%
\pgfpathlineto{\pgfqpoint{5.344309in}{2.705202in}}%
\pgfpathlineto{\pgfqpoint{5.291266in}{2.586892in}}%
\pgfpathlineto{\pgfqpoint{5.238222in}{2.837414in}}%
\pgfpathlineto{\pgfqpoint{5.185178in}{2.910114in}}%
\pgfpathlineto{\pgfqpoint{5.132134in}{2.824147in}}%
\pgfpathlineto{\pgfqpoint{5.079090in}{2.586892in}}%
\pgfpathlineto{\pgfqpoint{5.026046in}{2.758322in}}%
\pgfpathlineto{\pgfqpoint{4.973002in}{2.586892in}}%
\pgfpathlineto{\pgfqpoint{4.919958in}{2.863950in}}%
\pgfpathlineto{\pgfqpoint{4.866914in}{2.884645in}}%
\pgfpathlineto{\pgfqpoint{4.813870in}{2.586892in}}%
\pgfpathlineto{\pgfqpoint{4.760826in}{2.586892in}}%
\pgfpathlineto{\pgfqpoint{4.707782in}{2.934727in}}%
\pgfpathlineto{\pgfqpoint{4.654738in}{2.586892in}}%
\pgfpathlineto{\pgfqpoint{4.601694in}{2.927863in}}%
\pgfpathlineto{\pgfqpoint{4.548650in}{2.586892in}}%
\pgfpathlineto{\pgfqpoint{4.495607in}{2.934727in}}%
\pgfpathlineto{\pgfqpoint{4.442563in}{2.586892in}}%
\pgfpathlineto{\pgfqpoint{4.389519in}{2.586892in}}%
\pgfpathlineto{\pgfqpoint{4.336475in}{2.934727in}}%
\pgfpathlineto{\pgfqpoint{4.283431in}{2.586892in}}%
\pgfpathlineto{\pgfqpoint{4.230387in}{2.586892in}}%
\pgfpathlineto{\pgfqpoint{4.177343in}{2.586892in}}%
\pgfpathlineto{\pgfqpoint{4.124299in}{2.586892in}}%
\pgfpathlineto{\pgfqpoint{4.071255in}{2.586892in}}%
\pgfpathlineto{\pgfqpoint{4.018211in}{2.586892in}}%
\pgfpathlineto{\pgfqpoint{3.965167in}{2.586892in}}%
\pgfpathlineto{\pgfqpoint{3.912123in}{2.586892in}}%
\pgfpathlineto{\pgfqpoint{3.859079in}{2.586892in}}%
\pgfpathlineto{\pgfqpoint{3.806035in}{2.586892in}}%
\pgfpathlineto{\pgfqpoint{3.752991in}{2.586892in}}%
\pgfpathlineto{\pgfqpoint{3.699948in}{2.586892in}}%
\pgfpathlineto{\pgfqpoint{3.646904in}{2.586892in}}%
\pgfpathlineto{\pgfqpoint{3.593860in}{2.586892in}}%
\pgfpathlineto{\pgfqpoint{3.540816in}{2.908402in}}%
\pgfpathlineto{\pgfqpoint{3.487772in}{2.710315in}}%
\pgfpathlineto{\pgfqpoint{3.434728in}{2.586892in}}%
\pgfpathlineto{\pgfqpoint{3.381684in}{2.586892in}}%
\pgfpathlineto{\pgfqpoint{3.328640in}{2.586892in}}%
\pgfpathlineto{\pgfqpoint{3.275596in}{2.586892in}}%
\pgfpathlineto{\pgfqpoint{3.222552in}{2.586892in}}%
\pgfpathlineto{\pgfqpoint{3.169508in}{2.586892in}}%
\pgfpathlineto{\pgfqpoint{3.116464in}{2.586892in}}%
\pgfpathlineto{\pgfqpoint{3.063420in}{2.909016in}}%
\pgfpathlineto{\pgfqpoint{3.010376in}{2.586892in}}%
\pgfpathlineto{\pgfqpoint{2.957332in}{2.934727in}}%
\pgfpathlineto{\pgfqpoint{2.904288in}{2.586892in}}%
\pgfpathlineto{\pgfqpoint{2.851245in}{2.586892in}}%
\pgfpathlineto{\pgfqpoint{2.798201in}{2.586892in}}%
\pgfpathlineto{\pgfqpoint{2.745157in}{2.586892in}}%
\pgfpathlineto{\pgfqpoint{2.692113in}{2.586892in}}%
\pgfpathlineto{\pgfqpoint{2.639069in}{2.586892in}}%
\pgfpathlineto{\pgfqpoint{2.586025in}{2.586892in}}%
\pgfpathlineto{\pgfqpoint{2.532981in}{2.586892in}}%
\pgfpathlineto{\pgfqpoint{2.479937in}{2.586892in}}%
\pgfpathlineto{\pgfqpoint{2.426893in}{2.934727in}}%
\pgfpathlineto{\pgfqpoint{2.373849in}{2.844885in}}%
\pgfpathlineto{\pgfqpoint{2.320805in}{2.586892in}}%
\pgfpathlineto{\pgfqpoint{2.267761in}{2.586892in}}%
\pgfpathlineto{\pgfqpoint{2.214717in}{2.910735in}}%
\pgfpathlineto{\pgfqpoint{2.161673in}{2.586892in}}%
\pgfpathlineto{\pgfqpoint{2.108629in}{2.586892in}}%
\pgfpathlineto{\pgfqpoint{2.055586in}{2.586892in}}%
\pgfpathlineto{\pgfqpoint{2.002542in}{2.934727in}}%
\pgfpathlineto{\pgfqpoint{1.949498in}{2.586892in}}%
\pgfpathlineto{\pgfqpoint{1.896454in}{2.934727in}}%
\pgfpathlineto{\pgfqpoint{1.843410in}{2.934727in}}%
\pgfpathlineto{\pgfqpoint{1.790366in}{2.586892in}}%
\pgfpathlineto{\pgfqpoint{1.737322in}{2.586892in}}%
\pgfpathlineto{\pgfqpoint{1.684278in}{2.586892in}}%
\pgfpathlineto{\pgfqpoint{1.631234in}{2.802326in}}%
\pgfpathlineto{\pgfqpoint{1.578190in}{2.586892in}}%
\pgfpathlineto{\pgfqpoint{1.525146in}{2.586892in}}%
\pgfpathlineto{\pgfqpoint{1.472102in}{2.586892in}}%
\pgfpathlineto{\pgfqpoint{1.419058in}{2.586892in}}%
\pgfpathlineto{\pgfqpoint{1.366014in}{2.586892in}}%
\pgfpathlineto{\pgfqpoint{1.312970in}{2.586892in}}%
\pgfpathlineto{\pgfqpoint{1.259927in}{2.749185in}}%
\pgfpathlineto{\pgfqpoint{1.206883in}{2.586892in}}%
\pgfpathlineto{\pgfqpoint{1.153839in}{2.586892in}}%
\pgfpathlineto{\pgfqpoint{1.100795in}{2.586892in}}%
\pgfpathlineto{\pgfqpoint{1.047751in}{2.586892in}}%
\pgfpathlineto{\pgfqpoint{0.994707in}{2.586892in}}%
\pgfpathlineto{\pgfqpoint{0.941663in}{2.586892in}}%
\pgfpathlineto{\pgfqpoint{0.941663in}{2.586892in}}%
\pgfpathclose%
\pgfusepath{stroke,fill}%
}%
\begin{pgfscope}%
\pgfsys@transformshift{0.000000in}{0.000000in}%
\pgfsys@useobject{currentmarker}{}%
\end{pgfscope}%
\end{pgfscope}%
\begin{pgfscope}%
\pgfpathrectangle{\pgfqpoint{0.941663in}{0.670138in}}{\pgfqpoint{8.858337in}{3.465625in}}%
\pgfusepath{clip}%
\pgfsetrectcap%
\pgfsetroundjoin%
\pgfsetlinewidth{1.505625pt}%
\definecolor{currentstroke}{rgb}{0.549020,0.337255,0.294118}%
\pgfsetstrokecolor{currentstroke}%
\pgfsetdash{}{0pt}%
\pgfpathmoveto{\pgfqpoint{0.941663in}{2.586892in}}%
\pgfpathlineto{\pgfqpoint{1.206883in}{2.586892in}}%
\pgfpathlineto{\pgfqpoint{1.259927in}{2.749185in}}%
\pgfpathlineto{\pgfqpoint{1.312970in}{2.586892in}}%
\pgfpathlineto{\pgfqpoint{1.578190in}{2.586892in}}%
\pgfpathlineto{\pgfqpoint{1.631234in}{2.802326in}}%
\pgfpathlineto{\pgfqpoint{1.684278in}{2.586892in}}%
\pgfpathlineto{\pgfqpoint{1.790366in}{2.586892in}}%
\pgfpathlineto{\pgfqpoint{1.843410in}{3.070356in}}%
\pgfpathlineto{\pgfqpoint{1.896454in}{3.219351in}}%
\pgfpathlineto{\pgfqpoint{1.949498in}{2.586892in}}%
\pgfpathlineto{\pgfqpoint{2.002542in}{3.060121in}}%
\pgfpathlineto{\pgfqpoint{2.055586in}{2.586892in}}%
\pgfpathlineto{\pgfqpoint{2.161673in}{2.586892in}}%
\pgfpathlineto{\pgfqpoint{2.214717in}{2.910735in}}%
\pgfpathlineto{\pgfqpoint{2.267761in}{2.586892in}}%
\pgfpathlineto{\pgfqpoint{2.320805in}{2.586892in}}%
\pgfpathlineto{\pgfqpoint{2.373849in}{2.844885in}}%
\pgfpathlineto{\pgfqpoint{2.426893in}{2.942415in}}%
\pgfpathlineto{\pgfqpoint{2.479937in}{2.586892in}}%
\pgfpathlineto{\pgfqpoint{2.904288in}{2.586892in}}%
\pgfpathlineto{\pgfqpoint{2.957332in}{3.080511in}}%
\pgfpathlineto{\pgfqpoint{3.010376in}{2.586892in}}%
\pgfpathlineto{\pgfqpoint{3.063420in}{2.909016in}}%
\pgfpathlineto{\pgfqpoint{3.116464in}{2.586892in}}%
\pgfpathlineto{\pgfqpoint{3.434728in}{2.586892in}}%
\pgfpathlineto{\pgfqpoint{3.487772in}{2.710315in}}%
\pgfpathlineto{\pgfqpoint{3.540816in}{2.908402in}}%
\pgfpathlineto{\pgfqpoint{3.593860in}{2.586892in}}%
\pgfpathlineto{\pgfqpoint{4.283431in}{2.586892in}}%
\pgfpathlineto{\pgfqpoint{4.336475in}{3.224846in}}%
\pgfpathlineto{\pgfqpoint{4.389519in}{2.586892in}}%
\pgfpathlineto{\pgfqpoint{4.442563in}{2.586892in}}%
\pgfpathlineto{\pgfqpoint{4.495607in}{2.968775in}}%
\pgfpathlineto{\pgfqpoint{4.548650in}{2.586892in}}%
\pgfpathlineto{\pgfqpoint{4.601694in}{2.927863in}}%
\pgfpathlineto{\pgfqpoint{4.654738in}{2.586892in}}%
\pgfpathlineto{\pgfqpoint{4.707782in}{3.102021in}}%
\pgfpathlineto{\pgfqpoint{4.760826in}{2.586892in}}%
\pgfpathlineto{\pgfqpoint{4.813870in}{2.586892in}}%
\pgfpathlineto{\pgfqpoint{4.866914in}{2.884645in}}%
\pgfpathlineto{\pgfqpoint{4.919958in}{2.863950in}}%
\pgfpathlineto{\pgfqpoint{4.973002in}{2.586892in}}%
\pgfpathlineto{\pgfqpoint{5.026046in}{2.758322in}}%
\pgfpathlineto{\pgfqpoint{5.079090in}{2.586892in}}%
\pgfpathlineto{\pgfqpoint{5.132134in}{2.824147in}}%
\pgfpathlineto{\pgfqpoint{5.185178in}{2.910114in}}%
\pgfpathlineto{\pgfqpoint{5.238222in}{2.837414in}}%
\pgfpathlineto{\pgfqpoint{5.291266in}{2.586892in}}%
\pgfpathlineto{\pgfqpoint{5.344309in}{2.705202in}}%
\pgfpathlineto{\pgfqpoint{5.397353in}{2.660769in}}%
\pgfpathlineto{\pgfqpoint{5.450397in}{3.105789in}}%
\pgfpathlineto{\pgfqpoint{5.503441in}{2.586892in}}%
\pgfpathlineto{\pgfqpoint{5.556485in}{2.586892in}}%
\pgfpathlineto{\pgfqpoint{5.609529in}{2.850209in}}%
\pgfpathlineto{\pgfqpoint{5.662573in}{2.711726in}}%
\pgfpathlineto{\pgfqpoint{5.715617in}{2.586892in}}%
\pgfpathlineto{\pgfqpoint{5.768661in}{3.248973in}}%
\pgfpathlineto{\pgfqpoint{5.821705in}{3.208761in}}%
\pgfpathlineto{\pgfqpoint{5.874749in}{3.037838in}}%
\pgfpathlineto{\pgfqpoint{5.927793in}{2.934114in}}%
\pgfpathlineto{\pgfqpoint{5.980837in}{2.586892in}}%
\pgfpathlineto{\pgfqpoint{6.139969in}{2.586892in}}%
\pgfpathlineto{\pgfqpoint{6.193012in}{2.971161in}}%
\pgfpathlineto{\pgfqpoint{6.246056in}{3.026875in}}%
\pgfpathlineto{\pgfqpoint{6.299100in}{2.586892in}}%
\pgfpathlineto{\pgfqpoint{6.352144in}{2.586892in}}%
\pgfpathlineto{\pgfqpoint{6.405188in}{2.699818in}}%
\pgfpathlineto{\pgfqpoint{6.458232in}{2.586892in}}%
\pgfpathlineto{\pgfqpoint{6.829540in}{2.586892in}}%
\pgfpathlineto{\pgfqpoint{6.882584in}{3.116001in}}%
\pgfpathlineto{\pgfqpoint{6.935628in}{2.586892in}}%
\pgfpathlineto{\pgfqpoint{7.147803in}{2.586892in}}%
\pgfpathlineto{\pgfqpoint{7.200847in}{3.064989in}}%
\pgfpathlineto{\pgfqpoint{7.253891in}{2.586892in}}%
\pgfpathlineto{\pgfqpoint{7.306935in}{2.586892in}}%
\pgfpathlineto{\pgfqpoint{7.359979in}{2.806406in}}%
\pgfpathlineto{\pgfqpoint{7.413023in}{2.865454in}}%
\pgfpathlineto{\pgfqpoint{7.466067in}{2.586892in}}%
\pgfpathlineto{\pgfqpoint{7.519111in}{2.586892in}}%
\pgfpathlineto{\pgfqpoint{7.572155in}{2.823725in}}%
\pgfpathlineto{\pgfqpoint{7.625199in}{2.988504in}}%
\pgfpathlineto{\pgfqpoint{7.678243in}{2.586892in}}%
\pgfpathlineto{\pgfqpoint{7.996506in}{2.586892in}}%
\pgfpathlineto{\pgfqpoint{8.049550in}{3.102247in}}%
\pgfpathlineto{\pgfqpoint{8.102594in}{3.118508in}}%
\pgfpathlineto{\pgfqpoint{8.155638in}{2.586892in}}%
\pgfpathlineto{\pgfqpoint{8.208682in}{2.586892in}}%
\pgfpathlineto{\pgfqpoint{8.261726in}{3.260476in}}%
\pgfpathlineto{\pgfqpoint{8.314770in}{3.211850in}}%
\pgfpathlineto{\pgfqpoint{8.367814in}{2.586892in}}%
\pgfpathlineto{\pgfqpoint{8.420858in}{2.989399in}}%
\pgfpathlineto{\pgfqpoint{8.473902in}{2.586892in}}%
\pgfpathlineto{\pgfqpoint{8.526946in}{2.586892in}}%
\pgfpathlineto{\pgfqpoint{8.579990in}{3.021265in}}%
\pgfpathlineto{\pgfqpoint{8.633033in}{2.856358in}}%
\pgfpathlineto{\pgfqpoint{8.686077in}{3.059450in}}%
\pgfpathlineto{\pgfqpoint{8.739121in}{2.825404in}}%
\pgfpathlineto{\pgfqpoint{8.792165in}{2.586892in}}%
\pgfpathlineto{\pgfqpoint{8.845209in}{2.634872in}}%
\pgfpathlineto{\pgfqpoint{8.898253in}{2.689958in}}%
\pgfpathlineto{\pgfqpoint{8.951297in}{2.586892in}}%
\pgfpathlineto{\pgfqpoint{9.110429in}{2.586892in}}%
\pgfpathlineto{\pgfqpoint{9.163473in}{3.004188in}}%
\pgfpathlineto{\pgfqpoint{9.216517in}{3.140256in}}%
\pgfpathlineto{\pgfqpoint{9.269561in}{2.964013in}}%
\pgfpathlineto{\pgfqpoint{9.322605in}{3.063213in}}%
\pgfpathlineto{\pgfqpoint{9.375649in}{3.090533in}}%
\pgfpathlineto{\pgfqpoint{9.428692in}{2.586892in}}%
\pgfpathlineto{\pgfqpoint{9.481736in}{2.586892in}}%
\pgfpathlineto{\pgfqpoint{9.534780in}{3.282563in}}%
\pgfpathlineto{\pgfqpoint{9.587824in}{2.956774in}}%
\pgfpathlineto{\pgfqpoint{9.640868in}{2.586892in}}%
\pgfpathlineto{\pgfqpoint{9.693912in}{2.968640in}}%
\pgfpathlineto{\pgfqpoint{9.746956in}{3.115475in}}%
\pgfpathlineto{\pgfqpoint{9.800000in}{2.909416in}}%
\pgfpathlineto{\pgfqpoint{9.800000in}{2.909416in}}%
\pgfusepath{stroke}%
\end{pgfscope}%
\begin{pgfscope}%
\pgfpathrectangle{\pgfqpoint{0.941663in}{0.670138in}}{\pgfqpoint{8.858337in}{3.465625in}}%
\pgfusepath{clip}%
\pgfsetbuttcap%
\pgfsetroundjoin%
\definecolor{currentfill}{rgb}{0.549020,0.337255,0.294118}%
\pgfsetfillcolor{currentfill}%
\pgfsetlinewidth{1.003750pt}%
\definecolor{currentstroke}{rgb}{0.549020,0.337255,0.294118}%
\pgfsetstrokecolor{currentstroke}%
\pgfsetdash{}{0pt}%
\pgfsys@defobject{currentmarker}{\pgfqpoint{0.941663in}{2.586892in}}{\pgfqpoint{9.800000in}{3.282563in}}{%
\pgfpathmoveto{\pgfqpoint{0.941663in}{2.586892in}}%
\pgfpathlineto{\pgfqpoint{0.941663in}{2.586892in}}%
\pgfpathlineto{\pgfqpoint{0.994707in}{2.586892in}}%
\pgfpathlineto{\pgfqpoint{1.047751in}{2.586892in}}%
\pgfpathlineto{\pgfqpoint{1.100795in}{2.586892in}}%
\pgfpathlineto{\pgfqpoint{1.153839in}{2.586892in}}%
\pgfpathlineto{\pgfqpoint{1.206883in}{2.586892in}}%
\pgfpathlineto{\pgfqpoint{1.259927in}{2.749185in}}%
\pgfpathlineto{\pgfqpoint{1.312970in}{2.586892in}}%
\pgfpathlineto{\pgfqpoint{1.366014in}{2.586892in}}%
\pgfpathlineto{\pgfqpoint{1.419058in}{2.586892in}}%
\pgfpathlineto{\pgfqpoint{1.472102in}{2.586892in}}%
\pgfpathlineto{\pgfqpoint{1.525146in}{2.586892in}}%
\pgfpathlineto{\pgfqpoint{1.578190in}{2.586892in}}%
\pgfpathlineto{\pgfqpoint{1.631234in}{2.802326in}}%
\pgfpathlineto{\pgfqpoint{1.684278in}{2.586892in}}%
\pgfpathlineto{\pgfqpoint{1.737322in}{2.586892in}}%
\pgfpathlineto{\pgfqpoint{1.790366in}{2.586892in}}%
\pgfpathlineto{\pgfqpoint{1.843410in}{2.934727in}}%
\pgfpathlineto{\pgfqpoint{1.896454in}{2.934727in}}%
\pgfpathlineto{\pgfqpoint{1.949498in}{2.586892in}}%
\pgfpathlineto{\pgfqpoint{2.002542in}{2.934727in}}%
\pgfpathlineto{\pgfqpoint{2.055586in}{2.586892in}}%
\pgfpathlineto{\pgfqpoint{2.108629in}{2.586892in}}%
\pgfpathlineto{\pgfqpoint{2.161673in}{2.586892in}}%
\pgfpathlineto{\pgfqpoint{2.214717in}{2.910735in}}%
\pgfpathlineto{\pgfqpoint{2.267761in}{2.586892in}}%
\pgfpathlineto{\pgfqpoint{2.320805in}{2.586892in}}%
\pgfpathlineto{\pgfqpoint{2.373849in}{2.844885in}}%
\pgfpathlineto{\pgfqpoint{2.426893in}{2.934727in}}%
\pgfpathlineto{\pgfqpoint{2.479937in}{2.586892in}}%
\pgfpathlineto{\pgfqpoint{2.532981in}{2.586892in}}%
\pgfpathlineto{\pgfqpoint{2.586025in}{2.586892in}}%
\pgfpathlineto{\pgfqpoint{2.639069in}{2.586892in}}%
\pgfpathlineto{\pgfqpoint{2.692113in}{2.586892in}}%
\pgfpathlineto{\pgfqpoint{2.745157in}{2.586892in}}%
\pgfpathlineto{\pgfqpoint{2.798201in}{2.586892in}}%
\pgfpathlineto{\pgfqpoint{2.851245in}{2.586892in}}%
\pgfpathlineto{\pgfqpoint{2.904288in}{2.586892in}}%
\pgfpathlineto{\pgfqpoint{2.957332in}{2.934727in}}%
\pgfpathlineto{\pgfqpoint{3.010376in}{2.586892in}}%
\pgfpathlineto{\pgfqpoint{3.063420in}{2.909016in}}%
\pgfpathlineto{\pgfqpoint{3.116464in}{2.586892in}}%
\pgfpathlineto{\pgfqpoint{3.169508in}{2.586892in}}%
\pgfpathlineto{\pgfqpoint{3.222552in}{2.586892in}}%
\pgfpathlineto{\pgfqpoint{3.275596in}{2.586892in}}%
\pgfpathlineto{\pgfqpoint{3.328640in}{2.586892in}}%
\pgfpathlineto{\pgfqpoint{3.381684in}{2.586892in}}%
\pgfpathlineto{\pgfqpoint{3.434728in}{2.586892in}}%
\pgfpathlineto{\pgfqpoint{3.487772in}{2.710315in}}%
\pgfpathlineto{\pgfqpoint{3.540816in}{2.908402in}}%
\pgfpathlineto{\pgfqpoint{3.593860in}{2.586892in}}%
\pgfpathlineto{\pgfqpoint{3.646904in}{2.586892in}}%
\pgfpathlineto{\pgfqpoint{3.699948in}{2.586892in}}%
\pgfpathlineto{\pgfqpoint{3.752991in}{2.586892in}}%
\pgfpathlineto{\pgfqpoint{3.806035in}{2.586892in}}%
\pgfpathlineto{\pgfqpoint{3.859079in}{2.586892in}}%
\pgfpathlineto{\pgfqpoint{3.912123in}{2.586892in}}%
\pgfpathlineto{\pgfqpoint{3.965167in}{2.586892in}}%
\pgfpathlineto{\pgfqpoint{4.018211in}{2.586892in}}%
\pgfpathlineto{\pgfqpoint{4.071255in}{2.586892in}}%
\pgfpathlineto{\pgfqpoint{4.124299in}{2.586892in}}%
\pgfpathlineto{\pgfqpoint{4.177343in}{2.586892in}}%
\pgfpathlineto{\pgfqpoint{4.230387in}{2.586892in}}%
\pgfpathlineto{\pgfqpoint{4.283431in}{2.586892in}}%
\pgfpathlineto{\pgfqpoint{4.336475in}{2.934727in}}%
\pgfpathlineto{\pgfqpoint{4.389519in}{2.586892in}}%
\pgfpathlineto{\pgfqpoint{4.442563in}{2.586892in}}%
\pgfpathlineto{\pgfqpoint{4.495607in}{2.934727in}}%
\pgfpathlineto{\pgfqpoint{4.548650in}{2.586892in}}%
\pgfpathlineto{\pgfqpoint{4.601694in}{2.927863in}}%
\pgfpathlineto{\pgfqpoint{4.654738in}{2.586892in}}%
\pgfpathlineto{\pgfqpoint{4.707782in}{2.934727in}}%
\pgfpathlineto{\pgfqpoint{4.760826in}{2.586892in}}%
\pgfpathlineto{\pgfqpoint{4.813870in}{2.586892in}}%
\pgfpathlineto{\pgfqpoint{4.866914in}{2.884645in}}%
\pgfpathlineto{\pgfqpoint{4.919958in}{2.863950in}}%
\pgfpathlineto{\pgfqpoint{4.973002in}{2.586892in}}%
\pgfpathlineto{\pgfqpoint{5.026046in}{2.758322in}}%
\pgfpathlineto{\pgfqpoint{5.079090in}{2.586892in}}%
\pgfpathlineto{\pgfqpoint{5.132134in}{2.824147in}}%
\pgfpathlineto{\pgfqpoint{5.185178in}{2.910114in}}%
\pgfpathlineto{\pgfqpoint{5.238222in}{2.837414in}}%
\pgfpathlineto{\pgfqpoint{5.291266in}{2.586892in}}%
\pgfpathlineto{\pgfqpoint{5.344309in}{2.705202in}}%
\pgfpathlineto{\pgfqpoint{5.397353in}{2.660769in}}%
\pgfpathlineto{\pgfqpoint{5.450397in}{2.934727in}}%
\pgfpathlineto{\pgfqpoint{5.503441in}{2.586892in}}%
\pgfpathlineto{\pgfqpoint{5.556485in}{2.586892in}}%
\pgfpathlineto{\pgfqpoint{5.609529in}{2.850209in}}%
\pgfpathlineto{\pgfqpoint{5.662573in}{2.711726in}}%
\pgfpathlineto{\pgfqpoint{5.715617in}{2.586892in}}%
\pgfpathlineto{\pgfqpoint{5.768661in}{2.934727in}}%
\pgfpathlineto{\pgfqpoint{5.821705in}{2.934727in}}%
\pgfpathlineto{\pgfqpoint{5.874749in}{2.934727in}}%
\pgfpathlineto{\pgfqpoint{5.927793in}{2.643205in}}%
\pgfpathlineto{\pgfqpoint{5.980837in}{2.586892in}}%
\pgfpathlineto{\pgfqpoint{6.033881in}{2.586892in}}%
\pgfpathlineto{\pgfqpoint{6.086925in}{2.586892in}}%
\pgfpathlineto{\pgfqpoint{6.139969in}{2.586892in}}%
\pgfpathlineto{\pgfqpoint{6.193012in}{2.934727in}}%
\pgfpathlineto{\pgfqpoint{6.246056in}{2.934727in}}%
\pgfpathlineto{\pgfqpoint{6.299100in}{2.586892in}}%
\pgfpathlineto{\pgfqpoint{6.352144in}{2.586892in}}%
\pgfpathlineto{\pgfqpoint{6.405188in}{2.699818in}}%
\pgfpathlineto{\pgfqpoint{6.458232in}{2.586892in}}%
\pgfpathlineto{\pgfqpoint{6.511276in}{2.586892in}}%
\pgfpathlineto{\pgfqpoint{6.564320in}{2.586892in}}%
\pgfpathlineto{\pgfqpoint{6.617364in}{2.586892in}}%
\pgfpathlineto{\pgfqpoint{6.670408in}{2.586892in}}%
\pgfpathlineto{\pgfqpoint{6.723452in}{2.586892in}}%
\pgfpathlineto{\pgfqpoint{6.776496in}{2.586892in}}%
\pgfpathlineto{\pgfqpoint{6.829540in}{2.586892in}}%
\pgfpathlineto{\pgfqpoint{6.882584in}{2.934727in}}%
\pgfpathlineto{\pgfqpoint{6.935628in}{2.586892in}}%
\pgfpathlineto{\pgfqpoint{6.988671in}{2.586892in}}%
\pgfpathlineto{\pgfqpoint{7.041715in}{2.586892in}}%
\pgfpathlineto{\pgfqpoint{7.094759in}{2.586892in}}%
\pgfpathlineto{\pgfqpoint{7.147803in}{2.586892in}}%
\pgfpathlineto{\pgfqpoint{7.200847in}{2.934727in}}%
\pgfpathlineto{\pgfqpoint{7.253891in}{2.586892in}}%
\pgfpathlineto{\pgfqpoint{7.306935in}{2.586892in}}%
\pgfpathlineto{\pgfqpoint{7.359979in}{2.806406in}}%
\pgfpathlineto{\pgfqpoint{7.413023in}{2.865454in}}%
\pgfpathlineto{\pgfqpoint{7.466067in}{2.586892in}}%
\pgfpathlineto{\pgfqpoint{7.519111in}{2.586892in}}%
\pgfpathlineto{\pgfqpoint{7.572155in}{2.823725in}}%
\pgfpathlineto{\pgfqpoint{7.625199in}{2.934727in}}%
\pgfpathlineto{\pgfqpoint{7.678243in}{2.586892in}}%
\pgfpathlineto{\pgfqpoint{7.731287in}{2.586892in}}%
\pgfpathlineto{\pgfqpoint{7.784330in}{2.586892in}}%
\pgfpathlineto{\pgfqpoint{7.837374in}{2.586892in}}%
\pgfpathlineto{\pgfqpoint{7.890418in}{2.586892in}}%
\pgfpathlineto{\pgfqpoint{7.943462in}{2.586892in}}%
\pgfpathlineto{\pgfqpoint{7.996506in}{2.586892in}}%
\pgfpathlineto{\pgfqpoint{8.049550in}{2.934727in}}%
\pgfpathlineto{\pgfqpoint{8.102594in}{2.934727in}}%
\pgfpathlineto{\pgfqpoint{8.155638in}{2.586892in}}%
\pgfpathlineto{\pgfqpoint{8.208682in}{2.586892in}}%
\pgfpathlineto{\pgfqpoint{8.261726in}{2.934727in}}%
\pgfpathlineto{\pgfqpoint{8.314770in}{2.934727in}}%
\pgfpathlineto{\pgfqpoint{8.367814in}{2.586892in}}%
\pgfpathlineto{\pgfqpoint{8.420858in}{2.934727in}}%
\pgfpathlineto{\pgfqpoint{8.473902in}{2.586892in}}%
\pgfpathlineto{\pgfqpoint{8.526946in}{2.586892in}}%
\pgfpathlineto{\pgfqpoint{8.579990in}{2.934727in}}%
\pgfpathlineto{\pgfqpoint{8.633033in}{2.856358in}}%
\pgfpathlineto{\pgfqpoint{8.686077in}{2.934727in}}%
\pgfpathlineto{\pgfqpoint{8.739121in}{2.825404in}}%
\pgfpathlineto{\pgfqpoint{8.792165in}{2.586892in}}%
\pgfpathlineto{\pgfqpoint{8.845209in}{2.634872in}}%
\pgfpathlineto{\pgfqpoint{8.898253in}{2.689958in}}%
\pgfpathlineto{\pgfqpoint{8.951297in}{2.586892in}}%
\pgfpathlineto{\pgfqpoint{9.004341in}{2.586892in}}%
\pgfpathlineto{\pgfqpoint{9.057385in}{2.586892in}}%
\pgfpathlineto{\pgfqpoint{9.110429in}{2.586892in}}%
\pgfpathlineto{\pgfqpoint{9.163473in}{2.934727in}}%
\pgfpathlineto{\pgfqpoint{9.216517in}{2.934727in}}%
\pgfpathlineto{\pgfqpoint{9.269561in}{2.934727in}}%
\pgfpathlineto{\pgfqpoint{9.322605in}{2.934727in}}%
\pgfpathlineto{\pgfqpoint{9.375649in}{2.695480in}}%
\pgfpathlineto{\pgfqpoint{9.428692in}{2.586892in}}%
\pgfpathlineto{\pgfqpoint{9.481736in}{2.586892in}}%
\pgfpathlineto{\pgfqpoint{9.534780in}{2.934727in}}%
\pgfpathlineto{\pgfqpoint{9.587824in}{2.916055in}}%
\pgfpathlineto{\pgfqpoint{9.640868in}{2.586892in}}%
\pgfpathlineto{\pgfqpoint{9.693912in}{2.865140in}}%
\pgfpathlineto{\pgfqpoint{9.746956in}{2.608608in}}%
\pgfpathlineto{\pgfqpoint{9.800000in}{2.588586in}}%
\pgfpathlineto{\pgfqpoint{9.800000in}{2.909416in}}%
\pgfpathlineto{\pgfqpoint{9.800000in}{2.909416in}}%
\pgfpathlineto{\pgfqpoint{9.746956in}{3.115475in}}%
\pgfpathlineto{\pgfqpoint{9.693912in}{2.968640in}}%
\pgfpathlineto{\pgfqpoint{9.640868in}{2.586892in}}%
\pgfpathlineto{\pgfqpoint{9.587824in}{2.956774in}}%
\pgfpathlineto{\pgfqpoint{9.534780in}{3.282563in}}%
\pgfpathlineto{\pgfqpoint{9.481736in}{2.586892in}}%
\pgfpathlineto{\pgfqpoint{9.428692in}{2.586892in}}%
\pgfpathlineto{\pgfqpoint{9.375649in}{3.090533in}}%
\pgfpathlineto{\pgfqpoint{9.322605in}{3.063213in}}%
\pgfpathlineto{\pgfqpoint{9.269561in}{2.964013in}}%
\pgfpathlineto{\pgfqpoint{9.216517in}{3.140256in}}%
\pgfpathlineto{\pgfqpoint{9.163473in}{3.004188in}}%
\pgfpathlineto{\pgfqpoint{9.110429in}{2.586892in}}%
\pgfpathlineto{\pgfqpoint{9.057385in}{2.586892in}}%
\pgfpathlineto{\pgfqpoint{9.004341in}{2.586892in}}%
\pgfpathlineto{\pgfqpoint{8.951297in}{2.586892in}}%
\pgfpathlineto{\pgfqpoint{8.898253in}{2.689958in}}%
\pgfpathlineto{\pgfqpoint{8.845209in}{2.634872in}}%
\pgfpathlineto{\pgfqpoint{8.792165in}{2.586892in}}%
\pgfpathlineto{\pgfqpoint{8.739121in}{2.825404in}}%
\pgfpathlineto{\pgfqpoint{8.686077in}{3.059450in}}%
\pgfpathlineto{\pgfqpoint{8.633033in}{2.856358in}}%
\pgfpathlineto{\pgfqpoint{8.579990in}{3.021265in}}%
\pgfpathlineto{\pgfqpoint{8.526946in}{2.586892in}}%
\pgfpathlineto{\pgfqpoint{8.473902in}{2.586892in}}%
\pgfpathlineto{\pgfqpoint{8.420858in}{2.989399in}}%
\pgfpathlineto{\pgfqpoint{8.367814in}{2.586892in}}%
\pgfpathlineto{\pgfqpoint{8.314770in}{3.211850in}}%
\pgfpathlineto{\pgfqpoint{8.261726in}{3.260476in}}%
\pgfpathlineto{\pgfqpoint{8.208682in}{2.586892in}}%
\pgfpathlineto{\pgfqpoint{8.155638in}{2.586892in}}%
\pgfpathlineto{\pgfqpoint{8.102594in}{3.118508in}}%
\pgfpathlineto{\pgfqpoint{8.049550in}{3.102247in}}%
\pgfpathlineto{\pgfqpoint{7.996506in}{2.586892in}}%
\pgfpathlineto{\pgfqpoint{7.943462in}{2.586892in}}%
\pgfpathlineto{\pgfqpoint{7.890418in}{2.586892in}}%
\pgfpathlineto{\pgfqpoint{7.837374in}{2.586892in}}%
\pgfpathlineto{\pgfqpoint{7.784330in}{2.586892in}}%
\pgfpathlineto{\pgfqpoint{7.731287in}{2.586892in}}%
\pgfpathlineto{\pgfqpoint{7.678243in}{2.586892in}}%
\pgfpathlineto{\pgfqpoint{7.625199in}{2.988504in}}%
\pgfpathlineto{\pgfqpoint{7.572155in}{2.823725in}}%
\pgfpathlineto{\pgfqpoint{7.519111in}{2.586892in}}%
\pgfpathlineto{\pgfqpoint{7.466067in}{2.586892in}}%
\pgfpathlineto{\pgfqpoint{7.413023in}{2.865454in}}%
\pgfpathlineto{\pgfqpoint{7.359979in}{2.806406in}}%
\pgfpathlineto{\pgfqpoint{7.306935in}{2.586892in}}%
\pgfpathlineto{\pgfqpoint{7.253891in}{2.586892in}}%
\pgfpathlineto{\pgfqpoint{7.200847in}{3.064989in}}%
\pgfpathlineto{\pgfqpoint{7.147803in}{2.586892in}}%
\pgfpathlineto{\pgfqpoint{7.094759in}{2.586892in}}%
\pgfpathlineto{\pgfqpoint{7.041715in}{2.586892in}}%
\pgfpathlineto{\pgfqpoint{6.988671in}{2.586892in}}%
\pgfpathlineto{\pgfqpoint{6.935628in}{2.586892in}}%
\pgfpathlineto{\pgfqpoint{6.882584in}{3.116001in}}%
\pgfpathlineto{\pgfqpoint{6.829540in}{2.586892in}}%
\pgfpathlineto{\pgfqpoint{6.776496in}{2.586892in}}%
\pgfpathlineto{\pgfqpoint{6.723452in}{2.586892in}}%
\pgfpathlineto{\pgfqpoint{6.670408in}{2.586892in}}%
\pgfpathlineto{\pgfqpoint{6.617364in}{2.586892in}}%
\pgfpathlineto{\pgfqpoint{6.564320in}{2.586892in}}%
\pgfpathlineto{\pgfqpoint{6.511276in}{2.586892in}}%
\pgfpathlineto{\pgfqpoint{6.458232in}{2.586892in}}%
\pgfpathlineto{\pgfqpoint{6.405188in}{2.699818in}}%
\pgfpathlineto{\pgfqpoint{6.352144in}{2.586892in}}%
\pgfpathlineto{\pgfqpoint{6.299100in}{2.586892in}}%
\pgfpathlineto{\pgfqpoint{6.246056in}{3.026875in}}%
\pgfpathlineto{\pgfqpoint{6.193012in}{2.971161in}}%
\pgfpathlineto{\pgfqpoint{6.139969in}{2.586892in}}%
\pgfpathlineto{\pgfqpoint{6.086925in}{2.586892in}}%
\pgfpathlineto{\pgfqpoint{6.033881in}{2.586892in}}%
\pgfpathlineto{\pgfqpoint{5.980837in}{2.586892in}}%
\pgfpathlineto{\pgfqpoint{5.927793in}{2.934114in}}%
\pgfpathlineto{\pgfqpoint{5.874749in}{3.037838in}}%
\pgfpathlineto{\pgfqpoint{5.821705in}{3.208761in}}%
\pgfpathlineto{\pgfqpoint{5.768661in}{3.248973in}}%
\pgfpathlineto{\pgfqpoint{5.715617in}{2.586892in}}%
\pgfpathlineto{\pgfqpoint{5.662573in}{2.711726in}}%
\pgfpathlineto{\pgfqpoint{5.609529in}{2.850209in}}%
\pgfpathlineto{\pgfqpoint{5.556485in}{2.586892in}}%
\pgfpathlineto{\pgfqpoint{5.503441in}{2.586892in}}%
\pgfpathlineto{\pgfqpoint{5.450397in}{3.105789in}}%
\pgfpathlineto{\pgfqpoint{5.397353in}{2.660769in}}%
\pgfpathlineto{\pgfqpoint{5.344309in}{2.705202in}}%
\pgfpathlineto{\pgfqpoint{5.291266in}{2.586892in}}%
\pgfpathlineto{\pgfqpoint{5.238222in}{2.837414in}}%
\pgfpathlineto{\pgfqpoint{5.185178in}{2.910114in}}%
\pgfpathlineto{\pgfqpoint{5.132134in}{2.824147in}}%
\pgfpathlineto{\pgfqpoint{5.079090in}{2.586892in}}%
\pgfpathlineto{\pgfqpoint{5.026046in}{2.758322in}}%
\pgfpathlineto{\pgfqpoint{4.973002in}{2.586892in}}%
\pgfpathlineto{\pgfqpoint{4.919958in}{2.863950in}}%
\pgfpathlineto{\pgfqpoint{4.866914in}{2.884645in}}%
\pgfpathlineto{\pgfqpoint{4.813870in}{2.586892in}}%
\pgfpathlineto{\pgfqpoint{4.760826in}{2.586892in}}%
\pgfpathlineto{\pgfqpoint{4.707782in}{3.102021in}}%
\pgfpathlineto{\pgfqpoint{4.654738in}{2.586892in}}%
\pgfpathlineto{\pgfqpoint{4.601694in}{2.927863in}}%
\pgfpathlineto{\pgfqpoint{4.548650in}{2.586892in}}%
\pgfpathlineto{\pgfqpoint{4.495607in}{2.968775in}}%
\pgfpathlineto{\pgfqpoint{4.442563in}{2.586892in}}%
\pgfpathlineto{\pgfqpoint{4.389519in}{2.586892in}}%
\pgfpathlineto{\pgfqpoint{4.336475in}{3.224846in}}%
\pgfpathlineto{\pgfqpoint{4.283431in}{2.586892in}}%
\pgfpathlineto{\pgfqpoint{4.230387in}{2.586892in}}%
\pgfpathlineto{\pgfqpoint{4.177343in}{2.586892in}}%
\pgfpathlineto{\pgfqpoint{4.124299in}{2.586892in}}%
\pgfpathlineto{\pgfqpoint{4.071255in}{2.586892in}}%
\pgfpathlineto{\pgfqpoint{4.018211in}{2.586892in}}%
\pgfpathlineto{\pgfqpoint{3.965167in}{2.586892in}}%
\pgfpathlineto{\pgfqpoint{3.912123in}{2.586892in}}%
\pgfpathlineto{\pgfqpoint{3.859079in}{2.586892in}}%
\pgfpathlineto{\pgfqpoint{3.806035in}{2.586892in}}%
\pgfpathlineto{\pgfqpoint{3.752991in}{2.586892in}}%
\pgfpathlineto{\pgfqpoint{3.699948in}{2.586892in}}%
\pgfpathlineto{\pgfqpoint{3.646904in}{2.586892in}}%
\pgfpathlineto{\pgfqpoint{3.593860in}{2.586892in}}%
\pgfpathlineto{\pgfqpoint{3.540816in}{2.908402in}}%
\pgfpathlineto{\pgfqpoint{3.487772in}{2.710315in}}%
\pgfpathlineto{\pgfqpoint{3.434728in}{2.586892in}}%
\pgfpathlineto{\pgfqpoint{3.381684in}{2.586892in}}%
\pgfpathlineto{\pgfqpoint{3.328640in}{2.586892in}}%
\pgfpathlineto{\pgfqpoint{3.275596in}{2.586892in}}%
\pgfpathlineto{\pgfqpoint{3.222552in}{2.586892in}}%
\pgfpathlineto{\pgfqpoint{3.169508in}{2.586892in}}%
\pgfpathlineto{\pgfqpoint{3.116464in}{2.586892in}}%
\pgfpathlineto{\pgfqpoint{3.063420in}{2.909016in}}%
\pgfpathlineto{\pgfqpoint{3.010376in}{2.586892in}}%
\pgfpathlineto{\pgfqpoint{2.957332in}{3.080511in}}%
\pgfpathlineto{\pgfqpoint{2.904288in}{2.586892in}}%
\pgfpathlineto{\pgfqpoint{2.851245in}{2.586892in}}%
\pgfpathlineto{\pgfqpoint{2.798201in}{2.586892in}}%
\pgfpathlineto{\pgfqpoint{2.745157in}{2.586892in}}%
\pgfpathlineto{\pgfqpoint{2.692113in}{2.586892in}}%
\pgfpathlineto{\pgfqpoint{2.639069in}{2.586892in}}%
\pgfpathlineto{\pgfqpoint{2.586025in}{2.586892in}}%
\pgfpathlineto{\pgfqpoint{2.532981in}{2.586892in}}%
\pgfpathlineto{\pgfqpoint{2.479937in}{2.586892in}}%
\pgfpathlineto{\pgfqpoint{2.426893in}{2.942415in}}%
\pgfpathlineto{\pgfqpoint{2.373849in}{2.844885in}}%
\pgfpathlineto{\pgfqpoint{2.320805in}{2.586892in}}%
\pgfpathlineto{\pgfqpoint{2.267761in}{2.586892in}}%
\pgfpathlineto{\pgfqpoint{2.214717in}{2.910735in}}%
\pgfpathlineto{\pgfqpoint{2.161673in}{2.586892in}}%
\pgfpathlineto{\pgfqpoint{2.108629in}{2.586892in}}%
\pgfpathlineto{\pgfqpoint{2.055586in}{2.586892in}}%
\pgfpathlineto{\pgfqpoint{2.002542in}{3.060121in}}%
\pgfpathlineto{\pgfqpoint{1.949498in}{2.586892in}}%
\pgfpathlineto{\pgfqpoint{1.896454in}{3.219351in}}%
\pgfpathlineto{\pgfqpoint{1.843410in}{3.070356in}}%
\pgfpathlineto{\pgfqpoint{1.790366in}{2.586892in}}%
\pgfpathlineto{\pgfqpoint{1.737322in}{2.586892in}}%
\pgfpathlineto{\pgfqpoint{1.684278in}{2.586892in}}%
\pgfpathlineto{\pgfqpoint{1.631234in}{2.802326in}}%
\pgfpathlineto{\pgfqpoint{1.578190in}{2.586892in}}%
\pgfpathlineto{\pgfqpoint{1.525146in}{2.586892in}}%
\pgfpathlineto{\pgfqpoint{1.472102in}{2.586892in}}%
\pgfpathlineto{\pgfqpoint{1.419058in}{2.586892in}}%
\pgfpathlineto{\pgfqpoint{1.366014in}{2.586892in}}%
\pgfpathlineto{\pgfqpoint{1.312970in}{2.586892in}}%
\pgfpathlineto{\pgfqpoint{1.259927in}{2.749185in}}%
\pgfpathlineto{\pgfqpoint{1.206883in}{2.586892in}}%
\pgfpathlineto{\pgfqpoint{1.153839in}{2.586892in}}%
\pgfpathlineto{\pgfqpoint{1.100795in}{2.586892in}}%
\pgfpathlineto{\pgfqpoint{1.047751in}{2.586892in}}%
\pgfpathlineto{\pgfqpoint{0.994707in}{2.586892in}}%
\pgfpathlineto{\pgfqpoint{0.941663in}{2.586892in}}%
\pgfpathlineto{\pgfqpoint{0.941663in}{2.586892in}}%
\pgfpathclose%
\pgfusepath{stroke,fill}%
}%
\begin{pgfscope}%
\pgfsys@transformshift{0.000000in}{0.000000in}%
\pgfsys@useobject{currentmarker}{}%
\end{pgfscope}%
\end{pgfscope}%
\begin{pgfscope}%
\pgfpathrectangle{\pgfqpoint{0.941663in}{0.670138in}}{\pgfqpoint{8.858337in}{3.465625in}}%
\pgfusepath{clip}%
\pgfsetrectcap%
\pgfsetroundjoin%
\pgfsetlinewidth{1.505625pt}%
\definecolor{currentstroke}{rgb}{1.000000,0.647059,0.000000}%
\pgfsetstrokecolor{currentstroke}%
\pgfsetdash{}{0pt}%
\pgfpathmoveto{\pgfqpoint{0.941663in}{1.891220in}}%
\pgfpathlineto{\pgfqpoint{0.994707in}{1.543384in}}%
\pgfpathlineto{\pgfqpoint{1.047751in}{1.891220in}}%
\pgfpathlineto{\pgfqpoint{1.419058in}{1.891220in}}%
\pgfpathlineto{\pgfqpoint{1.472102in}{1.843091in}}%
\pgfpathlineto{\pgfqpoint{1.525146in}{1.891220in}}%
\pgfpathlineto{\pgfqpoint{1.631234in}{1.891220in}}%
\pgfpathlineto{\pgfqpoint{1.684278in}{1.880012in}}%
\pgfpathlineto{\pgfqpoint{1.737322in}{1.543384in}}%
\pgfpathlineto{\pgfqpoint{1.790366in}{1.543384in}}%
\pgfpathlineto{\pgfqpoint{1.843410in}{1.891220in}}%
\pgfpathlineto{\pgfqpoint{1.896454in}{1.891220in}}%
\pgfpathlineto{\pgfqpoint{1.949498in}{1.543384in}}%
\pgfpathlineto{\pgfqpoint{2.002542in}{1.891220in}}%
\pgfpathlineto{\pgfqpoint{2.055586in}{1.891220in}}%
\pgfpathlineto{\pgfqpoint{2.108629in}{1.569408in}}%
\pgfpathlineto{\pgfqpoint{2.161673in}{1.891220in}}%
\pgfpathlineto{\pgfqpoint{2.214717in}{1.891220in}}%
\pgfpathlineto{\pgfqpoint{2.267761in}{1.631196in}}%
\pgfpathlineto{\pgfqpoint{2.320805in}{1.543384in}}%
\pgfpathlineto{\pgfqpoint{2.373849in}{1.891220in}}%
\pgfpathlineto{\pgfqpoint{2.851245in}{1.891220in}}%
\pgfpathlineto{\pgfqpoint{2.904288in}{1.543384in}}%
\pgfpathlineto{\pgfqpoint{2.957332in}{1.891220in}}%
\pgfpathlineto{\pgfqpoint{3.010376in}{1.571272in}}%
\pgfpathlineto{\pgfqpoint{3.063420in}{1.891220in}}%
\pgfpathlineto{\pgfqpoint{3.328640in}{1.891220in}}%
\pgfpathlineto{\pgfqpoint{3.381684in}{1.794175in}}%
\pgfpathlineto{\pgfqpoint{3.434728in}{1.543384in}}%
\pgfpathlineto{\pgfqpoint{3.487772in}{1.891220in}}%
\pgfpathlineto{\pgfqpoint{4.071255in}{1.891220in}}%
\pgfpathlineto{\pgfqpoint{4.124299in}{1.770677in}}%
\pgfpathlineto{\pgfqpoint{4.177343in}{1.891220in}}%
\pgfpathlineto{\pgfqpoint{4.230387in}{1.543384in}}%
\pgfpathlineto{\pgfqpoint{4.283431in}{1.543384in}}%
\pgfpathlineto{\pgfqpoint{4.336475in}{1.891220in}}%
\pgfpathlineto{\pgfqpoint{4.389519in}{1.543384in}}%
\pgfpathlineto{\pgfqpoint{4.442563in}{1.580227in}}%
\pgfpathlineto{\pgfqpoint{4.495607in}{1.891220in}}%
\pgfpathlineto{\pgfqpoint{4.548650in}{1.543384in}}%
\pgfpathlineto{\pgfqpoint{4.601694in}{1.891220in}}%
\pgfpathlineto{\pgfqpoint{4.654738in}{1.543384in}}%
\pgfpathlineto{\pgfqpoint{4.707782in}{1.891220in}}%
\pgfpathlineto{\pgfqpoint{4.760826in}{1.543384in}}%
\pgfpathlineto{\pgfqpoint{4.813870in}{1.543384in}}%
\pgfpathlineto{\pgfqpoint{4.866914in}{1.891220in}}%
\pgfpathlineto{\pgfqpoint{4.919958in}{1.891220in}}%
\pgfpathlineto{\pgfqpoint{4.973002in}{1.543384in}}%
\pgfpathlineto{\pgfqpoint{5.026046in}{1.891220in}}%
\pgfpathlineto{\pgfqpoint{5.079090in}{1.543384in}}%
\pgfpathlineto{\pgfqpoint{5.132134in}{1.891220in}}%
\pgfpathlineto{\pgfqpoint{5.238222in}{1.891220in}}%
\pgfpathlineto{\pgfqpoint{5.291266in}{1.543384in}}%
\pgfpathlineto{\pgfqpoint{5.344309in}{1.891220in}}%
\pgfpathlineto{\pgfqpoint{5.450397in}{1.891220in}}%
\pgfpathlineto{\pgfqpoint{5.503441in}{1.543384in}}%
\pgfpathlineto{\pgfqpoint{5.556485in}{1.549519in}}%
\pgfpathlineto{\pgfqpoint{5.609529in}{1.891220in}}%
\pgfpathlineto{\pgfqpoint{5.662573in}{1.891220in}}%
\pgfpathlineto{\pgfqpoint{5.715617in}{1.543384in}}%
\pgfpathlineto{\pgfqpoint{5.768661in}{1.891220in}}%
\pgfpathlineto{\pgfqpoint{5.980837in}{1.891220in}}%
\pgfpathlineto{\pgfqpoint{6.033881in}{1.678405in}}%
\pgfpathlineto{\pgfqpoint{6.086925in}{1.543384in}}%
\pgfpathlineto{\pgfqpoint{6.139969in}{1.731521in}}%
\pgfpathlineto{\pgfqpoint{6.193012in}{1.891220in}}%
\pgfpathlineto{\pgfqpoint{6.246056in}{1.891220in}}%
\pgfpathlineto{\pgfqpoint{6.299100in}{1.798179in}}%
\pgfpathlineto{\pgfqpoint{6.352144in}{1.891220in}}%
\pgfpathlineto{\pgfqpoint{6.670408in}{1.891220in}}%
\pgfpathlineto{\pgfqpoint{6.723452in}{1.871334in}}%
\pgfpathlineto{\pgfqpoint{6.776496in}{1.543384in}}%
\pgfpathlineto{\pgfqpoint{6.829540in}{1.891220in}}%
\pgfpathlineto{\pgfqpoint{7.041715in}{1.891220in}}%
\pgfpathlineto{\pgfqpoint{7.094759in}{1.543384in}}%
\pgfpathlineto{\pgfqpoint{7.147803in}{1.891220in}}%
\pgfpathlineto{\pgfqpoint{7.200847in}{1.891220in}}%
\pgfpathlineto{\pgfqpoint{7.253891in}{1.543384in}}%
\pgfpathlineto{\pgfqpoint{7.306935in}{1.746843in}}%
\pgfpathlineto{\pgfqpoint{7.359979in}{1.891220in}}%
\pgfpathlineto{\pgfqpoint{7.413023in}{1.891220in}}%
\pgfpathlineto{\pgfqpoint{7.466067in}{1.648522in}}%
\pgfpathlineto{\pgfqpoint{7.519111in}{1.543384in}}%
\pgfpathlineto{\pgfqpoint{7.572155in}{1.891220in}}%
\pgfpathlineto{\pgfqpoint{7.625199in}{1.891220in}}%
\pgfpathlineto{\pgfqpoint{7.678243in}{1.849746in}}%
\pgfpathlineto{\pgfqpoint{7.731287in}{1.891220in}}%
\pgfpathlineto{\pgfqpoint{7.784330in}{1.567858in}}%
\pgfpathlineto{\pgfqpoint{7.837374in}{1.543384in}}%
\pgfpathlineto{\pgfqpoint{7.890418in}{1.891220in}}%
\pgfpathlineto{\pgfqpoint{7.943462in}{1.543384in}}%
\pgfpathlineto{\pgfqpoint{7.996506in}{1.543384in}}%
\pgfpathlineto{\pgfqpoint{8.049550in}{1.891220in}}%
\pgfpathlineto{\pgfqpoint{8.102594in}{1.891220in}}%
\pgfpathlineto{\pgfqpoint{8.155638in}{1.543384in}}%
\pgfpathlineto{\pgfqpoint{8.208682in}{1.543384in}}%
\pgfpathlineto{\pgfqpoint{8.261726in}{1.891220in}}%
\pgfpathlineto{\pgfqpoint{8.314770in}{1.891220in}}%
\pgfpathlineto{\pgfqpoint{8.367814in}{1.543384in}}%
\pgfpathlineto{\pgfqpoint{8.420858in}{1.891220in}}%
\pgfpathlineto{\pgfqpoint{8.473902in}{1.543384in}}%
\pgfpathlineto{\pgfqpoint{8.526946in}{1.543384in}}%
\pgfpathlineto{\pgfqpoint{8.579990in}{1.891220in}}%
\pgfpathlineto{\pgfqpoint{8.739121in}{1.891220in}}%
\pgfpathlineto{\pgfqpoint{8.792165in}{1.755774in}}%
\pgfpathlineto{\pgfqpoint{8.845209in}{1.891220in}}%
\pgfpathlineto{\pgfqpoint{8.898253in}{1.891220in}}%
\pgfpathlineto{\pgfqpoint{8.951297in}{1.543384in}}%
\pgfpathlineto{\pgfqpoint{9.004341in}{1.644141in}}%
\pgfpathlineto{\pgfqpoint{9.057385in}{1.543384in}}%
\pgfpathlineto{\pgfqpoint{9.110429in}{1.543384in}}%
\pgfpathlineto{\pgfqpoint{9.163473in}{1.891220in}}%
\pgfpathlineto{\pgfqpoint{9.375649in}{1.891220in}}%
\pgfpathlineto{\pgfqpoint{9.428692in}{1.543384in}}%
\pgfpathlineto{\pgfqpoint{9.481736in}{1.543384in}}%
\pgfpathlineto{\pgfqpoint{9.534780in}{1.891220in}}%
\pgfpathlineto{\pgfqpoint{9.587824in}{1.891220in}}%
\pgfpathlineto{\pgfqpoint{9.640868in}{1.617282in}}%
\pgfpathlineto{\pgfqpoint{9.693912in}{1.891220in}}%
\pgfpathlineto{\pgfqpoint{9.800000in}{1.891220in}}%
\pgfpathlineto{\pgfqpoint{9.800000in}{1.891220in}}%
\pgfusepath{stroke}%
\end{pgfscope}%
\begin{pgfscope}%
\pgfpathrectangle{\pgfqpoint{0.941663in}{0.670138in}}{\pgfqpoint{8.858337in}{3.465625in}}%
\pgfusepath{clip}%
\pgfsetbuttcap%
\pgfsetroundjoin%
\definecolor{currentfill}{rgb}{1.000000,0.647059,0.000000}%
\pgfsetfillcolor{currentfill}%
\pgfsetlinewidth{1.003750pt}%
\definecolor{currentstroke}{rgb}{1.000000,0.647059,0.000000}%
\pgfsetstrokecolor{currentstroke}%
\pgfsetdash{}{0pt}%
\pgfsys@defobject{currentmarker}{\pgfqpoint{0.941663in}{1.543384in}}{\pgfqpoint{9.800000in}{1.891220in}}{%
\pgfpathmoveto{\pgfqpoint{0.941663in}{1.891220in}}%
\pgfpathlineto{\pgfqpoint{0.941663in}{1.891220in}}%
\pgfpathlineto{\pgfqpoint{0.994707in}{1.891220in}}%
\pgfpathlineto{\pgfqpoint{1.047751in}{1.891220in}}%
\pgfpathlineto{\pgfqpoint{1.100795in}{1.891220in}}%
\pgfpathlineto{\pgfqpoint{1.153839in}{1.891220in}}%
\pgfpathlineto{\pgfqpoint{1.206883in}{1.891220in}}%
\pgfpathlineto{\pgfqpoint{1.259927in}{1.891220in}}%
\pgfpathlineto{\pgfqpoint{1.312970in}{1.891220in}}%
\pgfpathlineto{\pgfqpoint{1.366014in}{1.891220in}}%
\pgfpathlineto{\pgfqpoint{1.419058in}{1.891220in}}%
\pgfpathlineto{\pgfqpoint{1.472102in}{1.891220in}}%
\pgfpathlineto{\pgfqpoint{1.525146in}{1.891220in}}%
\pgfpathlineto{\pgfqpoint{1.578190in}{1.891220in}}%
\pgfpathlineto{\pgfqpoint{1.631234in}{1.891220in}}%
\pgfpathlineto{\pgfqpoint{1.684278in}{1.891220in}}%
\pgfpathlineto{\pgfqpoint{1.737322in}{1.891220in}}%
\pgfpathlineto{\pgfqpoint{1.790366in}{1.891220in}}%
\pgfpathlineto{\pgfqpoint{1.843410in}{1.891220in}}%
\pgfpathlineto{\pgfqpoint{1.896454in}{1.891220in}}%
\pgfpathlineto{\pgfqpoint{1.949498in}{1.891220in}}%
\pgfpathlineto{\pgfqpoint{2.002542in}{1.891220in}}%
\pgfpathlineto{\pgfqpoint{2.055586in}{1.891220in}}%
\pgfpathlineto{\pgfqpoint{2.108629in}{1.891220in}}%
\pgfpathlineto{\pgfqpoint{2.161673in}{1.891220in}}%
\pgfpathlineto{\pgfqpoint{2.214717in}{1.891220in}}%
\pgfpathlineto{\pgfqpoint{2.267761in}{1.891220in}}%
\pgfpathlineto{\pgfqpoint{2.320805in}{1.891220in}}%
\pgfpathlineto{\pgfqpoint{2.373849in}{1.891220in}}%
\pgfpathlineto{\pgfqpoint{2.426893in}{1.891220in}}%
\pgfpathlineto{\pgfqpoint{2.479937in}{1.891220in}}%
\pgfpathlineto{\pgfqpoint{2.532981in}{1.891220in}}%
\pgfpathlineto{\pgfqpoint{2.586025in}{1.891220in}}%
\pgfpathlineto{\pgfqpoint{2.639069in}{1.891220in}}%
\pgfpathlineto{\pgfqpoint{2.692113in}{1.891220in}}%
\pgfpathlineto{\pgfqpoint{2.745157in}{1.891220in}}%
\pgfpathlineto{\pgfqpoint{2.798201in}{1.891220in}}%
\pgfpathlineto{\pgfqpoint{2.851245in}{1.891220in}}%
\pgfpathlineto{\pgfqpoint{2.904288in}{1.891220in}}%
\pgfpathlineto{\pgfqpoint{2.957332in}{1.891220in}}%
\pgfpathlineto{\pgfqpoint{3.010376in}{1.891220in}}%
\pgfpathlineto{\pgfqpoint{3.063420in}{1.891220in}}%
\pgfpathlineto{\pgfqpoint{3.116464in}{1.891220in}}%
\pgfpathlineto{\pgfqpoint{3.169508in}{1.891220in}}%
\pgfpathlineto{\pgfqpoint{3.222552in}{1.891220in}}%
\pgfpathlineto{\pgfqpoint{3.275596in}{1.891220in}}%
\pgfpathlineto{\pgfqpoint{3.328640in}{1.891220in}}%
\pgfpathlineto{\pgfqpoint{3.381684in}{1.891220in}}%
\pgfpathlineto{\pgfqpoint{3.434728in}{1.891220in}}%
\pgfpathlineto{\pgfqpoint{3.487772in}{1.891220in}}%
\pgfpathlineto{\pgfqpoint{3.540816in}{1.891220in}}%
\pgfpathlineto{\pgfqpoint{3.593860in}{1.891220in}}%
\pgfpathlineto{\pgfqpoint{3.646904in}{1.891220in}}%
\pgfpathlineto{\pgfqpoint{3.699948in}{1.891220in}}%
\pgfpathlineto{\pgfqpoint{3.752991in}{1.891220in}}%
\pgfpathlineto{\pgfqpoint{3.806035in}{1.891220in}}%
\pgfpathlineto{\pgfqpoint{3.859079in}{1.891220in}}%
\pgfpathlineto{\pgfqpoint{3.912123in}{1.891220in}}%
\pgfpathlineto{\pgfqpoint{3.965167in}{1.891220in}}%
\pgfpathlineto{\pgfqpoint{4.018211in}{1.891220in}}%
\pgfpathlineto{\pgfqpoint{4.071255in}{1.891220in}}%
\pgfpathlineto{\pgfqpoint{4.124299in}{1.891220in}}%
\pgfpathlineto{\pgfqpoint{4.177343in}{1.891220in}}%
\pgfpathlineto{\pgfqpoint{4.230387in}{1.891220in}}%
\pgfpathlineto{\pgfqpoint{4.283431in}{1.891220in}}%
\pgfpathlineto{\pgfqpoint{4.336475in}{1.891220in}}%
\pgfpathlineto{\pgfqpoint{4.389519in}{1.891220in}}%
\pgfpathlineto{\pgfqpoint{4.442563in}{1.891220in}}%
\pgfpathlineto{\pgfqpoint{4.495607in}{1.891220in}}%
\pgfpathlineto{\pgfqpoint{4.548650in}{1.891220in}}%
\pgfpathlineto{\pgfqpoint{4.601694in}{1.891220in}}%
\pgfpathlineto{\pgfqpoint{4.654738in}{1.891220in}}%
\pgfpathlineto{\pgfqpoint{4.707782in}{1.891220in}}%
\pgfpathlineto{\pgfqpoint{4.760826in}{1.891220in}}%
\pgfpathlineto{\pgfqpoint{4.813870in}{1.891220in}}%
\pgfpathlineto{\pgfqpoint{4.866914in}{1.891220in}}%
\pgfpathlineto{\pgfqpoint{4.919958in}{1.891220in}}%
\pgfpathlineto{\pgfqpoint{4.973002in}{1.891220in}}%
\pgfpathlineto{\pgfqpoint{5.026046in}{1.891220in}}%
\pgfpathlineto{\pgfqpoint{5.079090in}{1.891220in}}%
\pgfpathlineto{\pgfqpoint{5.132134in}{1.891220in}}%
\pgfpathlineto{\pgfqpoint{5.185178in}{1.891220in}}%
\pgfpathlineto{\pgfqpoint{5.238222in}{1.891220in}}%
\pgfpathlineto{\pgfqpoint{5.291266in}{1.891220in}}%
\pgfpathlineto{\pgfqpoint{5.344309in}{1.891220in}}%
\pgfpathlineto{\pgfqpoint{5.397353in}{1.891220in}}%
\pgfpathlineto{\pgfqpoint{5.450397in}{1.891220in}}%
\pgfpathlineto{\pgfqpoint{5.503441in}{1.891220in}}%
\pgfpathlineto{\pgfqpoint{5.556485in}{1.891220in}}%
\pgfpathlineto{\pgfqpoint{5.609529in}{1.891220in}}%
\pgfpathlineto{\pgfqpoint{5.662573in}{1.891220in}}%
\pgfpathlineto{\pgfqpoint{5.715617in}{1.891220in}}%
\pgfpathlineto{\pgfqpoint{5.768661in}{1.891220in}}%
\pgfpathlineto{\pgfqpoint{5.821705in}{1.891220in}}%
\pgfpathlineto{\pgfqpoint{5.874749in}{1.891220in}}%
\pgfpathlineto{\pgfqpoint{5.927793in}{1.891220in}}%
\pgfpathlineto{\pgfqpoint{5.980837in}{1.891220in}}%
\pgfpathlineto{\pgfqpoint{6.033881in}{1.891220in}}%
\pgfpathlineto{\pgfqpoint{6.086925in}{1.891220in}}%
\pgfpathlineto{\pgfqpoint{6.139969in}{1.891220in}}%
\pgfpathlineto{\pgfqpoint{6.193012in}{1.891220in}}%
\pgfpathlineto{\pgfqpoint{6.246056in}{1.891220in}}%
\pgfpathlineto{\pgfqpoint{6.299100in}{1.891220in}}%
\pgfpathlineto{\pgfqpoint{6.352144in}{1.891220in}}%
\pgfpathlineto{\pgfqpoint{6.405188in}{1.891220in}}%
\pgfpathlineto{\pgfqpoint{6.458232in}{1.891220in}}%
\pgfpathlineto{\pgfqpoint{6.511276in}{1.891220in}}%
\pgfpathlineto{\pgfqpoint{6.564320in}{1.891220in}}%
\pgfpathlineto{\pgfqpoint{6.617364in}{1.891220in}}%
\pgfpathlineto{\pgfqpoint{6.670408in}{1.891220in}}%
\pgfpathlineto{\pgfqpoint{6.723452in}{1.891220in}}%
\pgfpathlineto{\pgfqpoint{6.776496in}{1.891220in}}%
\pgfpathlineto{\pgfqpoint{6.829540in}{1.891220in}}%
\pgfpathlineto{\pgfqpoint{6.882584in}{1.891220in}}%
\pgfpathlineto{\pgfqpoint{6.935628in}{1.891220in}}%
\pgfpathlineto{\pgfqpoint{6.988671in}{1.891220in}}%
\pgfpathlineto{\pgfqpoint{7.041715in}{1.891220in}}%
\pgfpathlineto{\pgfqpoint{7.094759in}{1.891220in}}%
\pgfpathlineto{\pgfqpoint{7.147803in}{1.891220in}}%
\pgfpathlineto{\pgfqpoint{7.200847in}{1.891220in}}%
\pgfpathlineto{\pgfqpoint{7.253891in}{1.891220in}}%
\pgfpathlineto{\pgfqpoint{7.306935in}{1.891220in}}%
\pgfpathlineto{\pgfqpoint{7.359979in}{1.891220in}}%
\pgfpathlineto{\pgfqpoint{7.413023in}{1.891220in}}%
\pgfpathlineto{\pgfqpoint{7.466067in}{1.891220in}}%
\pgfpathlineto{\pgfqpoint{7.519111in}{1.891220in}}%
\pgfpathlineto{\pgfqpoint{7.572155in}{1.891220in}}%
\pgfpathlineto{\pgfqpoint{7.625199in}{1.891220in}}%
\pgfpathlineto{\pgfqpoint{7.678243in}{1.891220in}}%
\pgfpathlineto{\pgfqpoint{7.731287in}{1.891220in}}%
\pgfpathlineto{\pgfqpoint{7.784330in}{1.891220in}}%
\pgfpathlineto{\pgfqpoint{7.837374in}{1.891220in}}%
\pgfpathlineto{\pgfqpoint{7.890418in}{1.891220in}}%
\pgfpathlineto{\pgfqpoint{7.943462in}{1.891220in}}%
\pgfpathlineto{\pgfqpoint{7.996506in}{1.891220in}}%
\pgfpathlineto{\pgfqpoint{8.049550in}{1.891220in}}%
\pgfpathlineto{\pgfqpoint{8.102594in}{1.891220in}}%
\pgfpathlineto{\pgfqpoint{8.155638in}{1.891220in}}%
\pgfpathlineto{\pgfqpoint{8.208682in}{1.891220in}}%
\pgfpathlineto{\pgfqpoint{8.261726in}{1.891220in}}%
\pgfpathlineto{\pgfqpoint{8.314770in}{1.891220in}}%
\pgfpathlineto{\pgfqpoint{8.367814in}{1.891220in}}%
\pgfpathlineto{\pgfqpoint{8.420858in}{1.891220in}}%
\pgfpathlineto{\pgfqpoint{8.473902in}{1.891220in}}%
\pgfpathlineto{\pgfqpoint{8.526946in}{1.891220in}}%
\pgfpathlineto{\pgfqpoint{8.579990in}{1.891220in}}%
\pgfpathlineto{\pgfqpoint{8.633033in}{1.891220in}}%
\pgfpathlineto{\pgfqpoint{8.686077in}{1.891220in}}%
\pgfpathlineto{\pgfqpoint{8.739121in}{1.891220in}}%
\pgfpathlineto{\pgfqpoint{8.792165in}{1.891220in}}%
\pgfpathlineto{\pgfqpoint{8.845209in}{1.891220in}}%
\pgfpathlineto{\pgfqpoint{8.898253in}{1.891220in}}%
\pgfpathlineto{\pgfqpoint{8.951297in}{1.891220in}}%
\pgfpathlineto{\pgfqpoint{9.004341in}{1.891220in}}%
\pgfpathlineto{\pgfqpoint{9.057385in}{1.891220in}}%
\pgfpathlineto{\pgfqpoint{9.110429in}{1.891220in}}%
\pgfpathlineto{\pgfqpoint{9.163473in}{1.891220in}}%
\pgfpathlineto{\pgfqpoint{9.216517in}{1.891220in}}%
\pgfpathlineto{\pgfqpoint{9.269561in}{1.891220in}}%
\pgfpathlineto{\pgfqpoint{9.322605in}{1.891220in}}%
\pgfpathlineto{\pgfqpoint{9.375649in}{1.891220in}}%
\pgfpathlineto{\pgfqpoint{9.428692in}{1.891220in}}%
\pgfpathlineto{\pgfqpoint{9.481736in}{1.891220in}}%
\pgfpathlineto{\pgfqpoint{9.534780in}{1.891220in}}%
\pgfpathlineto{\pgfqpoint{9.587824in}{1.891220in}}%
\pgfpathlineto{\pgfqpoint{9.640868in}{1.891220in}}%
\pgfpathlineto{\pgfqpoint{9.693912in}{1.891220in}}%
\pgfpathlineto{\pgfqpoint{9.746956in}{1.891220in}}%
\pgfpathlineto{\pgfqpoint{9.800000in}{1.891220in}}%
\pgfpathlineto{\pgfqpoint{9.800000in}{1.891220in}}%
\pgfpathlineto{\pgfqpoint{9.800000in}{1.891220in}}%
\pgfpathlineto{\pgfqpoint{9.746956in}{1.891220in}}%
\pgfpathlineto{\pgfqpoint{9.693912in}{1.891220in}}%
\pgfpathlineto{\pgfqpoint{9.640868in}{1.617282in}}%
\pgfpathlineto{\pgfqpoint{9.587824in}{1.891220in}}%
\pgfpathlineto{\pgfqpoint{9.534780in}{1.891220in}}%
\pgfpathlineto{\pgfqpoint{9.481736in}{1.543384in}}%
\pgfpathlineto{\pgfqpoint{9.428692in}{1.543384in}}%
\pgfpathlineto{\pgfqpoint{9.375649in}{1.891220in}}%
\pgfpathlineto{\pgfqpoint{9.322605in}{1.891220in}}%
\pgfpathlineto{\pgfqpoint{9.269561in}{1.891220in}}%
\pgfpathlineto{\pgfqpoint{9.216517in}{1.891220in}}%
\pgfpathlineto{\pgfqpoint{9.163473in}{1.891220in}}%
\pgfpathlineto{\pgfqpoint{9.110429in}{1.543384in}}%
\pgfpathlineto{\pgfqpoint{9.057385in}{1.543384in}}%
\pgfpathlineto{\pgfqpoint{9.004341in}{1.644141in}}%
\pgfpathlineto{\pgfqpoint{8.951297in}{1.543384in}}%
\pgfpathlineto{\pgfqpoint{8.898253in}{1.891220in}}%
\pgfpathlineto{\pgfqpoint{8.845209in}{1.891220in}}%
\pgfpathlineto{\pgfqpoint{8.792165in}{1.755774in}}%
\pgfpathlineto{\pgfqpoint{8.739121in}{1.891220in}}%
\pgfpathlineto{\pgfqpoint{8.686077in}{1.891220in}}%
\pgfpathlineto{\pgfqpoint{8.633033in}{1.891220in}}%
\pgfpathlineto{\pgfqpoint{8.579990in}{1.891220in}}%
\pgfpathlineto{\pgfqpoint{8.526946in}{1.543384in}}%
\pgfpathlineto{\pgfqpoint{8.473902in}{1.543384in}}%
\pgfpathlineto{\pgfqpoint{8.420858in}{1.891220in}}%
\pgfpathlineto{\pgfqpoint{8.367814in}{1.543384in}}%
\pgfpathlineto{\pgfqpoint{8.314770in}{1.891220in}}%
\pgfpathlineto{\pgfqpoint{8.261726in}{1.891220in}}%
\pgfpathlineto{\pgfqpoint{8.208682in}{1.543384in}}%
\pgfpathlineto{\pgfqpoint{8.155638in}{1.543384in}}%
\pgfpathlineto{\pgfqpoint{8.102594in}{1.891220in}}%
\pgfpathlineto{\pgfqpoint{8.049550in}{1.891220in}}%
\pgfpathlineto{\pgfqpoint{7.996506in}{1.543384in}}%
\pgfpathlineto{\pgfqpoint{7.943462in}{1.543384in}}%
\pgfpathlineto{\pgfqpoint{7.890418in}{1.891220in}}%
\pgfpathlineto{\pgfqpoint{7.837374in}{1.543384in}}%
\pgfpathlineto{\pgfqpoint{7.784330in}{1.567858in}}%
\pgfpathlineto{\pgfqpoint{7.731287in}{1.891220in}}%
\pgfpathlineto{\pgfqpoint{7.678243in}{1.849746in}}%
\pgfpathlineto{\pgfqpoint{7.625199in}{1.891220in}}%
\pgfpathlineto{\pgfqpoint{7.572155in}{1.891220in}}%
\pgfpathlineto{\pgfqpoint{7.519111in}{1.543384in}}%
\pgfpathlineto{\pgfqpoint{7.466067in}{1.648522in}}%
\pgfpathlineto{\pgfqpoint{7.413023in}{1.891220in}}%
\pgfpathlineto{\pgfqpoint{7.359979in}{1.891220in}}%
\pgfpathlineto{\pgfqpoint{7.306935in}{1.746843in}}%
\pgfpathlineto{\pgfqpoint{7.253891in}{1.543384in}}%
\pgfpathlineto{\pgfqpoint{7.200847in}{1.891220in}}%
\pgfpathlineto{\pgfqpoint{7.147803in}{1.891220in}}%
\pgfpathlineto{\pgfqpoint{7.094759in}{1.543384in}}%
\pgfpathlineto{\pgfqpoint{7.041715in}{1.891220in}}%
\pgfpathlineto{\pgfqpoint{6.988671in}{1.891220in}}%
\pgfpathlineto{\pgfqpoint{6.935628in}{1.891220in}}%
\pgfpathlineto{\pgfqpoint{6.882584in}{1.891220in}}%
\pgfpathlineto{\pgfqpoint{6.829540in}{1.891220in}}%
\pgfpathlineto{\pgfqpoint{6.776496in}{1.543384in}}%
\pgfpathlineto{\pgfqpoint{6.723452in}{1.871334in}}%
\pgfpathlineto{\pgfqpoint{6.670408in}{1.891220in}}%
\pgfpathlineto{\pgfqpoint{6.617364in}{1.891220in}}%
\pgfpathlineto{\pgfqpoint{6.564320in}{1.891220in}}%
\pgfpathlineto{\pgfqpoint{6.511276in}{1.891220in}}%
\pgfpathlineto{\pgfqpoint{6.458232in}{1.891220in}}%
\pgfpathlineto{\pgfqpoint{6.405188in}{1.891220in}}%
\pgfpathlineto{\pgfqpoint{6.352144in}{1.891220in}}%
\pgfpathlineto{\pgfqpoint{6.299100in}{1.798179in}}%
\pgfpathlineto{\pgfqpoint{6.246056in}{1.891220in}}%
\pgfpathlineto{\pgfqpoint{6.193012in}{1.891220in}}%
\pgfpathlineto{\pgfqpoint{6.139969in}{1.731521in}}%
\pgfpathlineto{\pgfqpoint{6.086925in}{1.543384in}}%
\pgfpathlineto{\pgfqpoint{6.033881in}{1.678405in}}%
\pgfpathlineto{\pgfqpoint{5.980837in}{1.891220in}}%
\pgfpathlineto{\pgfqpoint{5.927793in}{1.891220in}}%
\pgfpathlineto{\pgfqpoint{5.874749in}{1.891220in}}%
\pgfpathlineto{\pgfqpoint{5.821705in}{1.891220in}}%
\pgfpathlineto{\pgfqpoint{5.768661in}{1.891220in}}%
\pgfpathlineto{\pgfqpoint{5.715617in}{1.543384in}}%
\pgfpathlineto{\pgfqpoint{5.662573in}{1.891220in}}%
\pgfpathlineto{\pgfqpoint{5.609529in}{1.891220in}}%
\pgfpathlineto{\pgfqpoint{5.556485in}{1.549519in}}%
\pgfpathlineto{\pgfqpoint{5.503441in}{1.543384in}}%
\pgfpathlineto{\pgfqpoint{5.450397in}{1.891220in}}%
\pgfpathlineto{\pgfqpoint{5.397353in}{1.891220in}}%
\pgfpathlineto{\pgfqpoint{5.344309in}{1.891220in}}%
\pgfpathlineto{\pgfqpoint{5.291266in}{1.543384in}}%
\pgfpathlineto{\pgfqpoint{5.238222in}{1.891220in}}%
\pgfpathlineto{\pgfqpoint{5.185178in}{1.891220in}}%
\pgfpathlineto{\pgfqpoint{5.132134in}{1.891220in}}%
\pgfpathlineto{\pgfqpoint{5.079090in}{1.543384in}}%
\pgfpathlineto{\pgfqpoint{5.026046in}{1.891220in}}%
\pgfpathlineto{\pgfqpoint{4.973002in}{1.543384in}}%
\pgfpathlineto{\pgfqpoint{4.919958in}{1.891220in}}%
\pgfpathlineto{\pgfqpoint{4.866914in}{1.891220in}}%
\pgfpathlineto{\pgfqpoint{4.813870in}{1.543384in}}%
\pgfpathlineto{\pgfqpoint{4.760826in}{1.543384in}}%
\pgfpathlineto{\pgfqpoint{4.707782in}{1.891220in}}%
\pgfpathlineto{\pgfqpoint{4.654738in}{1.543384in}}%
\pgfpathlineto{\pgfqpoint{4.601694in}{1.891220in}}%
\pgfpathlineto{\pgfqpoint{4.548650in}{1.543384in}}%
\pgfpathlineto{\pgfqpoint{4.495607in}{1.891220in}}%
\pgfpathlineto{\pgfqpoint{4.442563in}{1.580227in}}%
\pgfpathlineto{\pgfqpoint{4.389519in}{1.543384in}}%
\pgfpathlineto{\pgfqpoint{4.336475in}{1.891220in}}%
\pgfpathlineto{\pgfqpoint{4.283431in}{1.543384in}}%
\pgfpathlineto{\pgfqpoint{4.230387in}{1.543384in}}%
\pgfpathlineto{\pgfqpoint{4.177343in}{1.891220in}}%
\pgfpathlineto{\pgfqpoint{4.124299in}{1.770677in}}%
\pgfpathlineto{\pgfqpoint{4.071255in}{1.891220in}}%
\pgfpathlineto{\pgfqpoint{4.018211in}{1.891220in}}%
\pgfpathlineto{\pgfqpoint{3.965167in}{1.891220in}}%
\pgfpathlineto{\pgfqpoint{3.912123in}{1.891220in}}%
\pgfpathlineto{\pgfqpoint{3.859079in}{1.891220in}}%
\pgfpathlineto{\pgfqpoint{3.806035in}{1.891220in}}%
\pgfpathlineto{\pgfqpoint{3.752991in}{1.891220in}}%
\pgfpathlineto{\pgfqpoint{3.699948in}{1.891220in}}%
\pgfpathlineto{\pgfqpoint{3.646904in}{1.891220in}}%
\pgfpathlineto{\pgfqpoint{3.593860in}{1.891220in}}%
\pgfpathlineto{\pgfqpoint{3.540816in}{1.891220in}}%
\pgfpathlineto{\pgfqpoint{3.487772in}{1.891220in}}%
\pgfpathlineto{\pgfqpoint{3.434728in}{1.543384in}}%
\pgfpathlineto{\pgfqpoint{3.381684in}{1.794175in}}%
\pgfpathlineto{\pgfqpoint{3.328640in}{1.891220in}}%
\pgfpathlineto{\pgfqpoint{3.275596in}{1.891220in}}%
\pgfpathlineto{\pgfqpoint{3.222552in}{1.891220in}}%
\pgfpathlineto{\pgfqpoint{3.169508in}{1.891220in}}%
\pgfpathlineto{\pgfqpoint{3.116464in}{1.891220in}}%
\pgfpathlineto{\pgfqpoint{3.063420in}{1.891220in}}%
\pgfpathlineto{\pgfqpoint{3.010376in}{1.571272in}}%
\pgfpathlineto{\pgfqpoint{2.957332in}{1.891220in}}%
\pgfpathlineto{\pgfqpoint{2.904288in}{1.543384in}}%
\pgfpathlineto{\pgfqpoint{2.851245in}{1.891220in}}%
\pgfpathlineto{\pgfqpoint{2.798201in}{1.891220in}}%
\pgfpathlineto{\pgfqpoint{2.745157in}{1.891220in}}%
\pgfpathlineto{\pgfqpoint{2.692113in}{1.891220in}}%
\pgfpathlineto{\pgfqpoint{2.639069in}{1.891220in}}%
\pgfpathlineto{\pgfqpoint{2.586025in}{1.891220in}}%
\pgfpathlineto{\pgfqpoint{2.532981in}{1.891220in}}%
\pgfpathlineto{\pgfqpoint{2.479937in}{1.891220in}}%
\pgfpathlineto{\pgfqpoint{2.426893in}{1.891220in}}%
\pgfpathlineto{\pgfqpoint{2.373849in}{1.891220in}}%
\pgfpathlineto{\pgfqpoint{2.320805in}{1.543384in}}%
\pgfpathlineto{\pgfqpoint{2.267761in}{1.631196in}}%
\pgfpathlineto{\pgfqpoint{2.214717in}{1.891220in}}%
\pgfpathlineto{\pgfqpoint{2.161673in}{1.891220in}}%
\pgfpathlineto{\pgfqpoint{2.108629in}{1.569408in}}%
\pgfpathlineto{\pgfqpoint{2.055586in}{1.891220in}}%
\pgfpathlineto{\pgfqpoint{2.002542in}{1.891220in}}%
\pgfpathlineto{\pgfqpoint{1.949498in}{1.543384in}}%
\pgfpathlineto{\pgfqpoint{1.896454in}{1.891220in}}%
\pgfpathlineto{\pgfqpoint{1.843410in}{1.891220in}}%
\pgfpathlineto{\pgfqpoint{1.790366in}{1.543384in}}%
\pgfpathlineto{\pgfqpoint{1.737322in}{1.543384in}}%
\pgfpathlineto{\pgfqpoint{1.684278in}{1.880012in}}%
\pgfpathlineto{\pgfqpoint{1.631234in}{1.891220in}}%
\pgfpathlineto{\pgfqpoint{1.578190in}{1.891220in}}%
\pgfpathlineto{\pgfqpoint{1.525146in}{1.891220in}}%
\pgfpathlineto{\pgfqpoint{1.472102in}{1.843091in}}%
\pgfpathlineto{\pgfqpoint{1.419058in}{1.891220in}}%
\pgfpathlineto{\pgfqpoint{1.366014in}{1.891220in}}%
\pgfpathlineto{\pgfqpoint{1.312970in}{1.891220in}}%
\pgfpathlineto{\pgfqpoint{1.259927in}{1.891220in}}%
\pgfpathlineto{\pgfqpoint{1.206883in}{1.891220in}}%
\pgfpathlineto{\pgfqpoint{1.153839in}{1.891220in}}%
\pgfpathlineto{\pgfqpoint{1.100795in}{1.891220in}}%
\pgfpathlineto{\pgfqpoint{1.047751in}{1.891220in}}%
\pgfpathlineto{\pgfqpoint{0.994707in}{1.543384in}}%
\pgfpathlineto{\pgfqpoint{0.941663in}{1.891220in}}%
\pgfpathlineto{\pgfqpoint{0.941663in}{1.891220in}}%
\pgfpathclose%
\pgfusepath{stroke,fill}%
}%
\begin{pgfscope}%
\pgfsys@transformshift{0.000000in}{0.000000in}%
\pgfsys@useobject{currentmarker}{}%
\end{pgfscope}%
\end{pgfscope}%
\begin{pgfscope}%
\pgfpathrectangle{\pgfqpoint{0.941663in}{0.670138in}}{\pgfqpoint{8.858337in}{3.465625in}}%
\pgfusepath{clip}%
\pgfsetrectcap%
\pgfsetroundjoin%
\pgfsetlinewidth{1.505625pt}%
\definecolor{currentstroke}{rgb}{0.501961,0.501961,0.501961}%
\pgfsetstrokecolor{currentstroke}%
\pgfsetdash{}{0pt}%
\pgfpathmoveto{\pgfqpoint{0.941663in}{0.942172in}}%
\pgfpathlineto{\pgfqpoint{0.994707in}{1.206100in}}%
\pgfpathlineto{\pgfqpoint{1.047751in}{0.938123in}}%
\pgfpathlineto{\pgfqpoint{1.100795in}{0.907342in}}%
\pgfpathlineto{\pgfqpoint{1.153839in}{0.893679in}}%
\pgfpathlineto{\pgfqpoint{1.206883in}{1.810992in}}%
\pgfpathlineto{\pgfqpoint{1.259927in}{1.891220in}}%
\pgfpathlineto{\pgfqpoint{1.312970in}{1.856216in}}%
\pgfpathlineto{\pgfqpoint{1.366014in}{0.829335in}}%
\pgfpathlineto{\pgfqpoint{1.419058in}{0.903552in}}%
\pgfpathlineto{\pgfqpoint{1.472102in}{0.908915in}}%
\pgfpathlineto{\pgfqpoint{1.525146in}{0.989277in}}%
\pgfpathlineto{\pgfqpoint{1.578190in}{0.966407in}}%
\pgfpathlineto{\pgfqpoint{1.631234in}{1.891220in}}%
\pgfpathlineto{\pgfqpoint{1.684278in}{1.559075in}}%
\pgfpathlineto{\pgfqpoint{1.737322in}{1.131392in}}%
\pgfpathlineto{\pgfqpoint{1.790366in}{1.113813in}}%
\pgfpathlineto{\pgfqpoint{1.843410in}{1.891220in}}%
\pgfpathlineto{\pgfqpoint{1.896454in}{1.891220in}}%
\pgfpathlineto{\pgfqpoint{1.949498in}{1.087124in}}%
\pgfpathlineto{\pgfqpoint{2.002542in}{1.891220in}}%
\pgfpathlineto{\pgfqpoint{2.055586in}{1.097929in}}%
\pgfpathlineto{\pgfqpoint{2.108629in}{1.481664in}}%
\pgfpathlineto{\pgfqpoint{2.161673in}{0.987578in}}%
\pgfpathlineto{\pgfqpoint{2.214717in}{1.891220in}}%
\pgfpathlineto{\pgfqpoint{2.267761in}{1.182556in}}%
\pgfpathlineto{\pgfqpoint{2.320805in}{0.918169in}}%
\pgfpathlineto{\pgfqpoint{2.373849in}{1.891220in}}%
\pgfpathlineto{\pgfqpoint{2.426893in}{1.891220in}}%
\pgfpathlineto{\pgfqpoint{2.479937in}{1.038089in}}%
\pgfpathlineto{\pgfqpoint{2.532981in}{0.893845in}}%
\pgfpathlineto{\pgfqpoint{2.586025in}{0.872034in}}%
\pgfpathlineto{\pgfqpoint{2.639069in}{0.867540in}}%
\pgfpathlineto{\pgfqpoint{2.692113in}{0.837338in}}%
\pgfpathlineto{\pgfqpoint{2.745157in}{0.908262in}}%
\pgfpathlineto{\pgfqpoint{2.798201in}{0.950657in}}%
\pgfpathlineto{\pgfqpoint{2.851245in}{1.781408in}}%
\pgfpathlineto{\pgfqpoint{2.904288in}{0.997564in}}%
\pgfpathlineto{\pgfqpoint{2.957332in}{1.891220in}}%
\pgfpathlineto{\pgfqpoint{3.010376in}{1.083328in}}%
\pgfpathlineto{\pgfqpoint{3.063420in}{1.891220in}}%
\pgfpathlineto{\pgfqpoint{3.116464in}{1.557584in}}%
\pgfpathlineto{\pgfqpoint{3.169508in}{1.115148in}}%
\pgfpathlineto{\pgfqpoint{3.222552in}{1.145671in}}%
\pgfpathlineto{\pgfqpoint{3.275596in}{1.061811in}}%
\pgfpathlineto{\pgfqpoint{3.328640in}{1.120386in}}%
\pgfpathlineto{\pgfqpoint{3.381684in}{1.666472in}}%
\pgfpathlineto{\pgfqpoint{3.434728in}{0.972772in}}%
\pgfpathlineto{\pgfqpoint{3.487772in}{1.891220in}}%
\pgfpathlineto{\pgfqpoint{3.540816in}{1.891220in}}%
\pgfpathlineto{\pgfqpoint{3.593860in}{1.388474in}}%
\pgfpathlineto{\pgfqpoint{3.646904in}{0.911072in}}%
\pgfpathlineto{\pgfqpoint{3.699948in}{0.867222in}}%
\pgfpathlineto{\pgfqpoint{3.752991in}{1.454919in}}%
\pgfpathlineto{\pgfqpoint{3.806035in}{0.907473in}}%
\pgfpathlineto{\pgfqpoint{3.859079in}{1.517418in}}%
\pgfpathlineto{\pgfqpoint{3.912123in}{1.426054in}}%
\pgfpathlineto{\pgfqpoint{3.965167in}{1.724076in}}%
\pgfpathlineto{\pgfqpoint{4.018211in}{0.913448in}}%
\pgfpathlineto{\pgfqpoint{4.071255in}{0.995539in}}%
\pgfpathlineto{\pgfqpoint{4.124299in}{1.123707in}}%
\pgfpathlineto{\pgfqpoint{4.177343in}{1.039428in}}%
\pgfpathlineto{\pgfqpoint{4.230387in}{1.077535in}}%
\pgfpathlineto{\pgfqpoint{4.283431in}{1.093254in}}%
\pgfpathlineto{\pgfqpoint{4.336475in}{1.891220in}}%
\pgfpathlineto{\pgfqpoint{4.389519in}{1.243678in}}%
\pgfpathlineto{\pgfqpoint{4.442563in}{1.580227in}}%
\pgfpathlineto{\pgfqpoint{4.495607in}{1.891220in}}%
\pgfpathlineto{\pgfqpoint{4.548650in}{1.437632in}}%
\pgfpathlineto{\pgfqpoint{4.601694in}{1.891220in}}%
\pgfpathlineto{\pgfqpoint{4.654738in}{1.094865in}}%
\pgfpathlineto{\pgfqpoint{4.707782in}{1.891220in}}%
\pgfpathlineto{\pgfqpoint{4.760826in}{1.287681in}}%
\pgfpathlineto{\pgfqpoint{4.813870in}{0.975129in}}%
\pgfpathlineto{\pgfqpoint{4.866914in}{1.891220in}}%
\pgfpathlineto{\pgfqpoint{4.919958in}{1.891220in}}%
\pgfpathlineto{\pgfqpoint{4.973002in}{0.856126in}}%
\pgfpathlineto{\pgfqpoint{5.026046in}{1.891220in}}%
\pgfpathlineto{\pgfqpoint{5.079090in}{0.827666in}}%
\pgfpathlineto{\pgfqpoint{5.132134in}{1.891220in}}%
\pgfpathlineto{\pgfqpoint{5.238222in}{1.891220in}}%
\pgfpathlineto{\pgfqpoint{5.291266in}{0.975653in}}%
\pgfpathlineto{\pgfqpoint{5.344309in}{1.891220in}}%
\pgfpathlineto{\pgfqpoint{5.450397in}{1.891220in}}%
\pgfpathlineto{\pgfqpoint{5.503441in}{1.116833in}}%
\pgfpathlineto{\pgfqpoint{5.556485in}{1.549519in}}%
\pgfpathlineto{\pgfqpoint{5.609529in}{1.891220in}}%
\pgfpathlineto{\pgfqpoint{5.662573in}{1.891220in}}%
\pgfpathlineto{\pgfqpoint{5.715617in}{1.122626in}}%
\pgfpathlineto{\pgfqpoint{5.768661in}{1.891220in}}%
\pgfpathlineto{\pgfqpoint{5.927793in}{1.891220in}}%
\pgfpathlineto{\pgfqpoint{5.980837in}{1.046269in}}%
\pgfpathlineto{\pgfqpoint{6.033881in}{1.678405in}}%
\pgfpathlineto{\pgfqpoint{6.086925in}{0.975132in}}%
\pgfpathlineto{\pgfqpoint{6.139969in}{0.956658in}}%
\pgfpathlineto{\pgfqpoint{6.193012in}{1.891220in}}%
\pgfpathlineto{\pgfqpoint{6.246056in}{1.891220in}}%
\pgfpathlineto{\pgfqpoint{6.299100in}{0.877160in}}%
\pgfpathlineto{\pgfqpoint{6.352144in}{1.833058in}}%
\pgfpathlineto{\pgfqpoint{6.405188in}{1.891220in}}%
\pgfpathlineto{\pgfqpoint{6.458232in}{1.413279in}}%
\pgfpathlineto{\pgfqpoint{6.511276in}{0.946754in}}%
\pgfpathlineto{\pgfqpoint{6.564320in}{0.943104in}}%
\pgfpathlineto{\pgfqpoint{6.617364in}{1.010061in}}%
\pgfpathlineto{\pgfqpoint{6.670408in}{1.040003in}}%
\pgfpathlineto{\pgfqpoint{6.723452in}{1.086085in}}%
\pgfpathlineto{\pgfqpoint{6.776496in}{1.330310in}}%
\pgfpathlineto{\pgfqpoint{6.829540in}{1.098052in}}%
\pgfpathlineto{\pgfqpoint{6.882584in}{1.891220in}}%
\pgfpathlineto{\pgfqpoint{6.935628in}{1.091705in}}%
\pgfpathlineto{\pgfqpoint{6.988671in}{1.684508in}}%
\pgfpathlineto{\pgfqpoint{7.041715in}{1.110968in}}%
\pgfpathlineto{\pgfqpoint{7.094759in}{1.134019in}}%
\pgfpathlineto{\pgfqpoint{7.147803in}{1.115813in}}%
\pgfpathlineto{\pgfqpoint{7.200847in}{1.891220in}}%
\pgfpathlineto{\pgfqpoint{7.253891in}{1.460365in}}%
\pgfpathlineto{\pgfqpoint{7.306935in}{1.095703in}}%
\pgfpathlineto{\pgfqpoint{7.359979in}{1.891220in}}%
\pgfpathlineto{\pgfqpoint{7.413023in}{1.891220in}}%
\pgfpathlineto{\pgfqpoint{7.466067in}{1.483365in}}%
\pgfpathlineto{\pgfqpoint{7.519111in}{1.410524in}}%
\pgfpathlineto{\pgfqpoint{7.572155in}{1.891220in}}%
\pgfpathlineto{\pgfqpoint{7.625199in}{1.891220in}}%
\pgfpathlineto{\pgfqpoint{7.678243in}{1.849746in}}%
\pgfpathlineto{\pgfqpoint{7.731287in}{0.908937in}}%
\pgfpathlineto{\pgfqpoint{7.784330in}{0.944968in}}%
\pgfpathlineto{\pgfqpoint{7.837374in}{0.953219in}}%
\pgfpathlineto{\pgfqpoint{7.890418in}{0.994522in}}%
\pgfpathlineto{\pgfqpoint{7.943462in}{0.992268in}}%
\pgfpathlineto{\pgfqpoint{7.996506in}{1.070971in}}%
\pgfpathlineto{\pgfqpoint{8.049550in}{1.891220in}}%
\pgfpathlineto{\pgfqpoint{8.102594in}{1.891220in}}%
\pgfpathlineto{\pgfqpoint{8.155638in}{1.137608in}}%
\pgfpathlineto{\pgfqpoint{8.208682in}{1.131748in}}%
\pgfpathlineto{\pgfqpoint{8.261726in}{1.891220in}}%
\pgfpathlineto{\pgfqpoint{8.314770in}{1.891220in}}%
\pgfpathlineto{\pgfqpoint{8.367814in}{1.114165in}}%
\pgfpathlineto{\pgfqpoint{8.420858in}{1.891220in}}%
\pgfpathlineto{\pgfqpoint{8.473902in}{1.054943in}}%
\pgfpathlineto{\pgfqpoint{8.526946in}{1.026802in}}%
\pgfpathlineto{\pgfqpoint{8.579990in}{1.891220in}}%
\pgfpathlineto{\pgfqpoint{8.739121in}{1.891220in}}%
\pgfpathlineto{\pgfqpoint{8.792165in}{1.755774in}}%
\pgfpathlineto{\pgfqpoint{8.845209in}{1.891220in}}%
\pgfpathlineto{\pgfqpoint{8.898253in}{1.891220in}}%
\pgfpathlineto{\pgfqpoint{8.951297in}{0.878008in}}%
\pgfpathlineto{\pgfqpoint{9.004341in}{1.644141in}}%
\pgfpathlineto{\pgfqpoint{9.057385in}{0.962369in}}%
\pgfpathlineto{\pgfqpoint{9.110429in}{0.997462in}}%
\pgfpathlineto{\pgfqpoint{9.163473in}{1.891220in}}%
\pgfpathlineto{\pgfqpoint{9.375649in}{1.891220in}}%
\pgfpathlineto{\pgfqpoint{9.428692in}{1.153825in}}%
\pgfpathlineto{\pgfqpoint{9.481736in}{1.200625in}}%
\pgfpathlineto{\pgfqpoint{9.534780in}{1.891220in}}%
\pgfpathlineto{\pgfqpoint{9.587824in}{1.891220in}}%
\pgfpathlineto{\pgfqpoint{9.640868in}{1.617282in}}%
\pgfpathlineto{\pgfqpoint{9.693912in}{1.891220in}}%
\pgfpathlineto{\pgfqpoint{9.800000in}{1.891220in}}%
\pgfpathlineto{\pgfqpoint{9.800000in}{1.891220in}}%
\pgfusepath{stroke}%
\end{pgfscope}%
\begin{pgfscope}%
\pgfpathrectangle{\pgfqpoint{0.941663in}{0.670138in}}{\pgfqpoint{8.858337in}{3.465625in}}%
\pgfusepath{clip}%
\pgfsetbuttcap%
\pgfsetroundjoin%
\definecolor{currentfill}{rgb}{0.501961,0.501961,0.501961}%
\pgfsetfillcolor{currentfill}%
\pgfsetlinewidth{1.003750pt}%
\definecolor{currentstroke}{rgb}{0.501961,0.501961,0.501961}%
\pgfsetstrokecolor{currentstroke}%
\pgfsetdash{}{0pt}%
\pgfsys@defobject{currentmarker}{\pgfqpoint{0.941663in}{0.827666in}}{\pgfqpoint{9.800000in}{1.891220in}}{%
\pgfpathmoveto{\pgfqpoint{0.941663in}{0.942172in}}%
\pgfpathlineto{\pgfqpoint{0.941663in}{1.891220in}}%
\pgfpathlineto{\pgfqpoint{0.994707in}{1.543384in}}%
\pgfpathlineto{\pgfqpoint{1.047751in}{1.891220in}}%
\pgfpathlineto{\pgfqpoint{1.100795in}{1.891220in}}%
\pgfpathlineto{\pgfqpoint{1.153839in}{1.891220in}}%
\pgfpathlineto{\pgfqpoint{1.206883in}{1.891220in}}%
\pgfpathlineto{\pgfqpoint{1.259927in}{1.891220in}}%
\pgfpathlineto{\pgfqpoint{1.312970in}{1.891220in}}%
\pgfpathlineto{\pgfqpoint{1.366014in}{1.891220in}}%
\pgfpathlineto{\pgfqpoint{1.419058in}{1.891220in}}%
\pgfpathlineto{\pgfqpoint{1.472102in}{1.843091in}}%
\pgfpathlineto{\pgfqpoint{1.525146in}{1.891220in}}%
\pgfpathlineto{\pgfqpoint{1.578190in}{1.891220in}}%
\pgfpathlineto{\pgfqpoint{1.631234in}{1.891220in}}%
\pgfpathlineto{\pgfqpoint{1.684278in}{1.880012in}}%
\pgfpathlineto{\pgfqpoint{1.737322in}{1.543384in}}%
\pgfpathlineto{\pgfqpoint{1.790366in}{1.543384in}}%
\pgfpathlineto{\pgfqpoint{1.843410in}{1.891220in}}%
\pgfpathlineto{\pgfqpoint{1.896454in}{1.891220in}}%
\pgfpathlineto{\pgfqpoint{1.949498in}{1.543384in}}%
\pgfpathlineto{\pgfqpoint{2.002542in}{1.891220in}}%
\pgfpathlineto{\pgfqpoint{2.055586in}{1.891220in}}%
\pgfpathlineto{\pgfqpoint{2.108629in}{1.569408in}}%
\pgfpathlineto{\pgfqpoint{2.161673in}{1.891220in}}%
\pgfpathlineto{\pgfqpoint{2.214717in}{1.891220in}}%
\pgfpathlineto{\pgfqpoint{2.267761in}{1.631196in}}%
\pgfpathlineto{\pgfqpoint{2.320805in}{1.543384in}}%
\pgfpathlineto{\pgfqpoint{2.373849in}{1.891220in}}%
\pgfpathlineto{\pgfqpoint{2.426893in}{1.891220in}}%
\pgfpathlineto{\pgfqpoint{2.479937in}{1.891220in}}%
\pgfpathlineto{\pgfqpoint{2.532981in}{1.891220in}}%
\pgfpathlineto{\pgfqpoint{2.586025in}{1.891220in}}%
\pgfpathlineto{\pgfqpoint{2.639069in}{1.891220in}}%
\pgfpathlineto{\pgfqpoint{2.692113in}{1.891220in}}%
\pgfpathlineto{\pgfqpoint{2.745157in}{1.891220in}}%
\pgfpathlineto{\pgfqpoint{2.798201in}{1.891220in}}%
\pgfpathlineto{\pgfqpoint{2.851245in}{1.891220in}}%
\pgfpathlineto{\pgfqpoint{2.904288in}{1.543384in}}%
\pgfpathlineto{\pgfqpoint{2.957332in}{1.891220in}}%
\pgfpathlineto{\pgfqpoint{3.010376in}{1.571272in}}%
\pgfpathlineto{\pgfqpoint{3.063420in}{1.891220in}}%
\pgfpathlineto{\pgfqpoint{3.116464in}{1.891220in}}%
\pgfpathlineto{\pgfqpoint{3.169508in}{1.891220in}}%
\pgfpathlineto{\pgfqpoint{3.222552in}{1.891220in}}%
\pgfpathlineto{\pgfqpoint{3.275596in}{1.891220in}}%
\pgfpathlineto{\pgfqpoint{3.328640in}{1.891220in}}%
\pgfpathlineto{\pgfqpoint{3.381684in}{1.794175in}}%
\pgfpathlineto{\pgfqpoint{3.434728in}{1.543384in}}%
\pgfpathlineto{\pgfqpoint{3.487772in}{1.891220in}}%
\pgfpathlineto{\pgfqpoint{3.540816in}{1.891220in}}%
\pgfpathlineto{\pgfqpoint{3.593860in}{1.891220in}}%
\pgfpathlineto{\pgfqpoint{3.646904in}{1.891220in}}%
\pgfpathlineto{\pgfqpoint{3.699948in}{1.891220in}}%
\pgfpathlineto{\pgfqpoint{3.752991in}{1.891220in}}%
\pgfpathlineto{\pgfqpoint{3.806035in}{1.891220in}}%
\pgfpathlineto{\pgfqpoint{3.859079in}{1.891220in}}%
\pgfpathlineto{\pgfqpoint{3.912123in}{1.891220in}}%
\pgfpathlineto{\pgfqpoint{3.965167in}{1.891220in}}%
\pgfpathlineto{\pgfqpoint{4.018211in}{1.891220in}}%
\pgfpathlineto{\pgfqpoint{4.071255in}{1.891220in}}%
\pgfpathlineto{\pgfqpoint{4.124299in}{1.770677in}}%
\pgfpathlineto{\pgfqpoint{4.177343in}{1.891220in}}%
\pgfpathlineto{\pgfqpoint{4.230387in}{1.543384in}}%
\pgfpathlineto{\pgfqpoint{4.283431in}{1.543384in}}%
\pgfpathlineto{\pgfqpoint{4.336475in}{1.891220in}}%
\pgfpathlineto{\pgfqpoint{4.389519in}{1.543384in}}%
\pgfpathlineto{\pgfqpoint{4.442563in}{1.580227in}}%
\pgfpathlineto{\pgfqpoint{4.495607in}{1.891220in}}%
\pgfpathlineto{\pgfqpoint{4.548650in}{1.543384in}}%
\pgfpathlineto{\pgfqpoint{4.601694in}{1.891220in}}%
\pgfpathlineto{\pgfqpoint{4.654738in}{1.543384in}}%
\pgfpathlineto{\pgfqpoint{4.707782in}{1.891220in}}%
\pgfpathlineto{\pgfqpoint{4.760826in}{1.543384in}}%
\pgfpathlineto{\pgfqpoint{4.813870in}{1.543384in}}%
\pgfpathlineto{\pgfqpoint{4.866914in}{1.891220in}}%
\pgfpathlineto{\pgfqpoint{4.919958in}{1.891220in}}%
\pgfpathlineto{\pgfqpoint{4.973002in}{1.543384in}}%
\pgfpathlineto{\pgfqpoint{5.026046in}{1.891220in}}%
\pgfpathlineto{\pgfqpoint{5.079090in}{1.543384in}}%
\pgfpathlineto{\pgfqpoint{5.132134in}{1.891220in}}%
\pgfpathlineto{\pgfqpoint{5.185178in}{1.891220in}}%
\pgfpathlineto{\pgfqpoint{5.238222in}{1.891220in}}%
\pgfpathlineto{\pgfqpoint{5.291266in}{1.543384in}}%
\pgfpathlineto{\pgfqpoint{5.344309in}{1.891220in}}%
\pgfpathlineto{\pgfqpoint{5.397353in}{1.891220in}}%
\pgfpathlineto{\pgfqpoint{5.450397in}{1.891220in}}%
\pgfpathlineto{\pgfqpoint{5.503441in}{1.543384in}}%
\pgfpathlineto{\pgfqpoint{5.556485in}{1.549519in}}%
\pgfpathlineto{\pgfqpoint{5.609529in}{1.891220in}}%
\pgfpathlineto{\pgfqpoint{5.662573in}{1.891220in}}%
\pgfpathlineto{\pgfqpoint{5.715617in}{1.543384in}}%
\pgfpathlineto{\pgfqpoint{5.768661in}{1.891220in}}%
\pgfpathlineto{\pgfqpoint{5.821705in}{1.891220in}}%
\pgfpathlineto{\pgfqpoint{5.874749in}{1.891220in}}%
\pgfpathlineto{\pgfqpoint{5.927793in}{1.891220in}}%
\pgfpathlineto{\pgfqpoint{5.980837in}{1.891220in}}%
\pgfpathlineto{\pgfqpoint{6.033881in}{1.678405in}}%
\pgfpathlineto{\pgfqpoint{6.086925in}{1.543384in}}%
\pgfpathlineto{\pgfqpoint{6.139969in}{1.731521in}}%
\pgfpathlineto{\pgfqpoint{6.193012in}{1.891220in}}%
\pgfpathlineto{\pgfqpoint{6.246056in}{1.891220in}}%
\pgfpathlineto{\pgfqpoint{6.299100in}{1.798179in}}%
\pgfpathlineto{\pgfqpoint{6.352144in}{1.891220in}}%
\pgfpathlineto{\pgfqpoint{6.405188in}{1.891220in}}%
\pgfpathlineto{\pgfqpoint{6.458232in}{1.891220in}}%
\pgfpathlineto{\pgfqpoint{6.511276in}{1.891220in}}%
\pgfpathlineto{\pgfqpoint{6.564320in}{1.891220in}}%
\pgfpathlineto{\pgfqpoint{6.617364in}{1.891220in}}%
\pgfpathlineto{\pgfqpoint{6.670408in}{1.891220in}}%
\pgfpathlineto{\pgfqpoint{6.723452in}{1.871334in}}%
\pgfpathlineto{\pgfqpoint{6.776496in}{1.543384in}}%
\pgfpathlineto{\pgfqpoint{6.829540in}{1.891220in}}%
\pgfpathlineto{\pgfqpoint{6.882584in}{1.891220in}}%
\pgfpathlineto{\pgfqpoint{6.935628in}{1.891220in}}%
\pgfpathlineto{\pgfqpoint{6.988671in}{1.891220in}}%
\pgfpathlineto{\pgfqpoint{7.041715in}{1.891220in}}%
\pgfpathlineto{\pgfqpoint{7.094759in}{1.543384in}}%
\pgfpathlineto{\pgfqpoint{7.147803in}{1.891220in}}%
\pgfpathlineto{\pgfqpoint{7.200847in}{1.891220in}}%
\pgfpathlineto{\pgfqpoint{7.253891in}{1.543384in}}%
\pgfpathlineto{\pgfqpoint{7.306935in}{1.746843in}}%
\pgfpathlineto{\pgfqpoint{7.359979in}{1.891220in}}%
\pgfpathlineto{\pgfqpoint{7.413023in}{1.891220in}}%
\pgfpathlineto{\pgfqpoint{7.466067in}{1.648522in}}%
\pgfpathlineto{\pgfqpoint{7.519111in}{1.543384in}}%
\pgfpathlineto{\pgfqpoint{7.572155in}{1.891220in}}%
\pgfpathlineto{\pgfqpoint{7.625199in}{1.891220in}}%
\pgfpathlineto{\pgfqpoint{7.678243in}{1.849746in}}%
\pgfpathlineto{\pgfqpoint{7.731287in}{1.891220in}}%
\pgfpathlineto{\pgfqpoint{7.784330in}{1.567858in}}%
\pgfpathlineto{\pgfqpoint{7.837374in}{1.543384in}}%
\pgfpathlineto{\pgfqpoint{7.890418in}{1.891220in}}%
\pgfpathlineto{\pgfqpoint{7.943462in}{1.543384in}}%
\pgfpathlineto{\pgfqpoint{7.996506in}{1.543384in}}%
\pgfpathlineto{\pgfqpoint{8.049550in}{1.891220in}}%
\pgfpathlineto{\pgfqpoint{8.102594in}{1.891220in}}%
\pgfpathlineto{\pgfqpoint{8.155638in}{1.543384in}}%
\pgfpathlineto{\pgfqpoint{8.208682in}{1.543384in}}%
\pgfpathlineto{\pgfqpoint{8.261726in}{1.891220in}}%
\pgfpathlineto{\pgfqpoint{8.314770in}{1.891220in}}%
\pgfpathlineto{\pgfqpoint{8.367814in}{1.543384in}}%
\pgfpathlineto{\pgfqpoint{8.420858in}{1.891220in}}%
\pgfpathlineto{\pgfqpoint{8.473902in}{1.543384in}}%
\pgfpathlineto{\pgfqpoint{8.526946in}{1.543384in}}%
\pgfpathlineto{\pgfqpoint{8.579990in}{1.891220in}}%
\pgfpathlineto{\pgfqpoint{8.633033in}{1.891220in}}%
\pgfpathlineto{\pgfqpoint{8.686077in}{1.891220in}}%
\pgfpathlineto{\pgfqpoint{8.739121in}{1.891220in}}%
\pgfpathlineto{\pgfqpoint{8.792165in}{1.755774in}}%
\pgfpathlineto{\pgfqpoint{8.845209in}{1.891220in}}%
\pgfpathlineto{\pgfqpoint{8.898253in}{1.891220in}}%
\pgfpathlineto{\pgfqpoint{8.951297in}{1.543384in}}%
\pgfpathlineto{\pgfqpoint{9.004341in}{1.644141in}}%
\pgfpathlineto{\pgfqpoint{9.057385in}{1.543384in}}%
\pgfpathlineto{\pgfqpoint{9.110429in}{1.543384in}}%
\pgfpathlineto{\pgfqpoint{9.163473in}{1.891220in}}%
\pgfpathlineto{\pgfqpoint{9.216517in}{1.891220in}}%
\pgfpathlineto{\pgfqpoint{9.269561in}{1.891220in}}%
\pgfpathlineto{\pgfqpoint{9.322605in}{1.891220in}}%
\pgfpathlineto{\pgfqpoint{9.375649in}{1.891220in}}%
\pgfpathlineto{\pgfqpoint{9.428692in}{1.543384in}}%
\pgfpathlineto{\pgfqpoint{9.481736in}{1.543384in}}%
\pgfpathlineto{\pgfqpoint{9.534780in}{1.891220in}}%
\pgfpathlineto{\pgfqpoint{9.587824in}{1.891220in}}%
\pgfpathlineto{\pgfqpoint{9.640868in}{1.617282in}}%
\pgfpathlineto{\pgfqpoint{9.693912in}{1.891220in}}%
\pgfpathlineto{\pgfqpoint{9.746956in}{1.891220in}}%
\pgfpathlineto{\pgfqpoint{9.800000in}{1.891220in}}%
\pgfpathlineto{\pgfqpoint{9.800000in}{1.891220in}}%
\pgfpathlineto{\pgfqpoint{9.800000in}{1.891220in}}%
\pgfpathlineto{\pgfqpoint{9.746956in}{1.891220in}}%
\pgfpathlineto{\pgfqpoint{9.693912in}{1.891220in}}%
\pgfpathlineto{\pgfqpoint{9.640868in}{1.617282in}}%
\pgfpathlineto{\pgfqpoint{9.587824in}{1.891220in}}%
\pgfpathlineto{\pgfqpoint{9.534780in}{1.891220in}}%
\pgfpathlineto{\pgfqpoint{9.481736in}{1.200625in}}%
\pgfpathlineto{\pgfqpoint{9.428692in}{1.153825in}}%
\pgfpathlineto{\pgfqpoint{9.375649in}{1.891220in}}%
\pgfpathlineto{\pgfqpoint{9.322605in}{1.891220in}}%
\pgfpathlineto{\pgfqpoint{9.269561in}{1.891220in}}%
\pgfpathlineto{\pgfqpoint{9.216517in}{1.891220in}}%
\pgfpathlineto{\pgfqpoint{9.163473in}{1.891220in}}%
\pgfpathlineto{\pgfqpoint{9.110429in}{0.997462in}}%
\pgfpathlineto{\pgfqpoint{9.057385in}{0.962369in}}%
\pgfpathlineto{\pgfqpoint{9.004341in}{1.644141in}}%
\pgfpathlineto{\pgfqpoint{8.951297in}{0.878008in}}%
\pgfpathlineto{\pgfqpoint{8.898253in}{1.891220in}}%
\pgfpathlineto{\pgfqpoint{8.845209in}{1.891220in}}%
\pgfpathlineto{\pgfqpoint{8.792165in}{1.755774in}}%
\pgfpathlineto{\pgfqpoint{8.739121in}{1.891220in}}%
\pgfpathlineto{\pgfqpoint{8.686077in}{1.891220in}}%
\pgfpathlineto{\pgfqpoint{8.633033in}{1.891220in}}%
\pgfpathlineto{\pgfqpoint{8.579990in}{1.891220in}}%
\pgfpathlineto{\pgfqpoint{8.526946in}{1.026802in}}%
\pgfpathlineto{\pgfqpoint{8.473902in}{1.054943in}}%
\pgfpathlineto{\pgfqpoint{8.420858in}{1.891220in}}%
\pgfpathlineto{\pgfqpoint{8.367814in}{1.114165in}}%
\pgfpathlineto{\pgfqpoint{8.314770in}{1.891220in}}%
\pgfpathlineto{\pgfqpoint{8.261726in}{1.891220in}}%
\pgfpathlineto{\pgfqpoint{8.208682in}{1.131748in}}%
\pgfpathlineto{\pgfqpoint{8.155638in}{1.137608in}}%
\pgfpathlineto{\pgfqpoint{8.102594in}{1.891220in}}%
\pgfpathlineto{\pgfqpoint{8.049550in}{1.891220in}}%
\pgfpathlineto{\pgfqpoint{7.996506in}{1.070971in}}%
\pgfpathlineto{\pgfqpoint{7.943462in}{0.992268in}}%
\pgfpathlineto{\pgfqpoint{7.890418in}{0.994522in}}%
\pgfpathlineto{\pgfqpoint{7.837374in}{0.953219in}}%
\pgfpathlineto{\pgfqpoint{7.784330in}{0.944968in}}%
\pgfpathlineto{\pgfqpoint{7.731287in}{0.908937in}}%
\pgfpathlineto{\pgfqpoint{7.678243in}{1.849746in}}%
\pgfpathlineto{\pgfqpoint{7.625199in}{1.891220in}}%
\pgfpathlineto{\pgfqpoint{7.572155in}{1.891220in}}%
\pgfpathlineto{\pgfqpoint{7.519111in}{1.410524in}}%
\pgfpathlineto{\pgfqpoint{7.466067in}{1.483365in}}%
\pgfpathlineto{\pgfqpoint{7.413023in}{1.891220in}}%
\pgfpathlineto{\pgfqpoint{7.359979in}{1.891220in}}%
\pgfpathlineto{\pgfqpoint{7.306935in}{1.095703in}}%
\pgfpathlineto{\pgfqpoint{7.253891in}{1.460365in}}%
\pgfpathlineto{\pgfqpoint{7.200847in}{1.891220in}}%
\pgfpathlineto{\pgfqpoint{7.147803in}{1.115813in}}%
\pgfpathlineto{\pgfqpoint{7.094759in}{1.134019in}}%
\pgfpathlineto{\pgfqpoint{7.041715in}{1.110968in}}%
\pgfpathlineto{\pgfqpoint{6.988671in}{1.684508in}}%
\pgfpathlineto{\pgfqpoint{6.935628in}{1.091705in}}%
\pgfpathlineto{\pgfqpoint{6.882584in}{1.891220in}}%
\pgfpathlineto{\pgfqpoint{6.829540in}{1.098052in}}%
\pgfpathlineto{\pgfqpoint{6.776496in}{1.330310in}}%
\pgfpathlineto{\pgfqpoint{6.723452in}{1.086085in}}%
\pgfpathlineto{\pgfqpoint{6.670408in}{1.040003in}}%
\pgfpathlineto{\pgfqpoint{6.617364in}{1.010061in}}%
\pgfpathlineto{\pgfqpoint{6.564320in}{0.943104in}}%
\pgfpathlineto{\pgfqpoint{6.511276in}{0.946754in}}%
\pgfpathlineto{\pgfqpoint{6.458232in}{1.413279in}}%
\pgfpathlineto{\pgfqpoint{6.405188in}{1.891220in}}%
\pgfpathlineto{\pgfqpoint{6.352144in}{1.833058in}}%
\pgfpathlineto{\pgfqpoint{6.299100in}{0.877160in}}%
\pgfpathlineto{\pgfqpoint{6.246056in}{1.891220in}}%
\pgfpathlineto{\pgfqpoint{6.193012in}{1.891220in}}%
\pgfpathlineto{\pgfqpoint{6.139969in}{0.956658in}}%
\pgfpathlineto{\pgfqpoint{6.086925in}{0.975132in}}%
\pgfpathlineto{\pgfqpoint{6.033881in}{1.678405in}}%
\pgfpathlineto{\pgfqpoint{5.980837in}{1.046269in}}%
\pgfpathlineto{\pgfqpoint{5.927793in}{1.891220in}}%
\pgfpathlineto{\pgfqpoint{5.874749in}{1.891220in}}%
\pgfpathlineto{\pgfqpoint{5.821705in}{1.891220in}}%
\pgfpathlineto{\pgfqpoint{5.768661in}{1.891220in}}%
\pgfpathlineto{\pgfqpoint{5.715617in}{1.122626in}}%
\pgfpathlineto{\pgfqpoint{5.662573in}{1.891220in}}%
\pgfpathlineto{\pgfqpoint{5.609529in}{1.891220in}}%
\pgfpathlineto{\pgfqpoint{5.556485in}{1.549519in}}%
\pgfpathlineto{\pgfqpoint{5.503441in}{1.116833in}}%
\pgfpathlineto{\pgfqpoint{5.450397in}{1.891220in}}%
\pgfpathlineto{\pgfqpoint{5.397353in}{1.891220in}}%
\pgfpathlineto{\pgfqpoint{5.344309in}{1.891220in}}%
\pgfpathlineto{\pgfqpoint{5.291266in}{0.975653in}}%
\pgfpathlineto{\pgfqpoint{5.238222in}{1.891220in}}%
\pgfpathlineto{\pgfqpoint{5.185178in}{1.891220in}}%
\pgfpathlineto{\pgfqpoint{5.132134in}{1.891220in}}%
\pgfpathlineto{\pgfqpoint{5.079090in}{0.827666in}}%
\pgfpathlineto{\pgfqpoint{5.026046in}{1.891220in}}%
\pgfpathlineto{\pgfqpoint{4.973002in}{0.856126in}}%
\pgfpathlineto{\pgfqpoint{4.919958in}{1.891220in}}%
\pgfpathlineto{\pgfqpoint{4.866914in}{1.891220in}}%
\pgfpathlineto{\pgfqpoint{4.813870in}{0.975129in}}%
\pgfpathlineto{\pgfqpoint{4.760826in}{1.287681in}}%
\pgfpathlineto{\pgfqpoint{4.707782in}{1.891220in}}%
\pgfpathlineto{\pgfqpoint{4.654738in}{1.094865in}}%
\pgfpathlineto{\pgfqpoint{4.601694in}{1.891220in}}%
\pgfpathlineto{\pgfqpoint{4.548650in}{1.437632in}}%
\pgfpathlineto{\pgfqpoint{4.495607in}{1.891220in}}%
\pgfpathlineto{\pgfqpoint{4.442563in}{1.580227in}}%
\pgfpathlineto{\pgfqpoint{4.389519in}{1.243678in}}%
\pgfpathlineto{\pgfqpoint{4.336475in}{1.891220in}}%
\pgfpathlineto{\pgfqpoint{4.283431in}{1.093254in}}%
\pgfpathlineto{\pgfqpoint{4.230387in}{1.077535in}}%
\pgfpathlineto{\pgfqpoint{4.177343in}{1.039428in}}%
\pgfpathlineto{\pgfqpoint{4.124299in}{1.123707in}}%
\pgfpathlineto{\pgfqpoint{4.071255in}{0.995539in}}%
\pgfpathlineto{\pgfqpoint{4.018211in}{0.913448in}}%
\pgfpathlineto{\pgfqpoint{3.965167in}{1.724076in}}%
\pgfpathlineto{\pgfqpoint{3.912123in}{1.426054in}}%
\pgfpathlineto{\pgfqpoint{3.859079in}{1.517418in}}%
\pgfpathlineto{\pgfqpoint{3.806035in}{0.907473in}}%
\pgfpathlineto{\pgfqpoint{3.752991in}{1.454919in}}%
\pgfpathlineto{\pgfqpoint{3.699948in}{0.867222in}}%
\pgfpathlineto{\pgfqpoint{3.646904in}{0.911072in}}%
\pgfpathlineto{\pgfqpoint{3.593860in}{1.388474in}}%
\pgfpathlineto{\pgfqpoint{3.540816in}{1.891220in}}%
\pgfpathlineto{\pgfqpoint{3.487772in}{1.891220in}}%
\pgfpathlineto{\pgfqpoint{3.434728in}{0.972772in}}%
\pgfpathlineto{\pgfqpoint{3.381684in}{1.666472in}}%
\pgfpathlineto{\pgfqpoint{3.328640in}{1.120386in}}%
\pgfpathlineto{\pgfqpoint{3.275596in}{1.061811in}}%
\pgfpathlineto{\pgfqpoint{3.222552in}{1.145671in}}%
\pgfpathlineto{\pgfqpoint{3.169508in}{1.115148in}}%
\pgfpathlineto{\pgfqpoint{3.116464in}{1.557584in}}%
\pgfpathlineto{\pgfqpoint{3.063420in}{1.891220in}}%
\pgfpathlineto{\pgfqpoint{3.010376in}{1.083328in}}%
\pgfpathlineto{\pgfqpoint{2.957332in}{1.891220in}}%
\pgfpathlineto{\pgfqpoint{2.904288in}{0.997564in}}%
\pgfpathlineto{\pgfqpoint{2.851245in}{1.781408in}}%
\pgfpathlineto{\pgfqpoint{2.798201in}{0.950657in}}%
\pgfpathlineto{\pgfqpoint{2.745157in}{0.908262in}}%
\pgfpathlineto{\pgfqpoint{2.692113in}{0.837338in}}%
\pgfpathlineto{\pgfqpoint{2.639069in}{0.867540in}}%
\pgfpathlineto{\pgfqpoint{2.586025in}{0.872034in}}%
\pgfpathlineto{\pgfqpoint{2.532981in}{0.893845in}}%
\pgfpathlineto{\pgfqpoint{2.479937in}{1.038089in}}%
\pgfpathlineto{\pgfqpoint{2.426893in}{1.891220in}}%
\pgfpathlineto{\pgfqpoint{2.373849in}{1.891220in}}%
\pgfpathlineto{\pgfqpoint{2.320805in}{0.918169in}}%
\pgfpathlineto{\pgfqpoint{2.267761in}{1.182556in}}%
\pgfpathlineto{\pgfqpoint{2.214717in}{1.891220in}}%
\pgfpathlineto{\pgfqpoint{2.161673in}{0.987578in}}%
\pgfpathlineto{\pgfqpoint{2.108629in}{1.481664in}}%
\pgfpathlineto{\pgfqpoint{2.055586in}{1.097929in}}%
\pgfpathlineto{\pgfqpoint{2.002542in}{1.891220in}}%
\pgfpathlineto{\pgfqpoint{1.949498in}{1.087124in}}%
\pgfpathlineto{\pgfqpoint{1.896454in}{1.891220in}}%
\pgfpathlineto{\pgfqpoint{1.843410in}{1.891220in}}%
\pgfpathlineto{\pgfqpoint{1.790366in}{1.113813in}}%
\pgfpathlineto{\pgfqpoint{1.737322in}{1.131392in}}%
\pgfpathlineto{\pgfqpoint{1.684278in}{1.559075in}}%
\pgfpathlineto{\pgfqpoint{1.631234in}{1.891220in}}%
\pgfpathlineto{\pgfqpoint{1.578190in}{0.966407in}}%
\pgfpathlineto{\pgfqpoint{1.525146in}{0.989277in}}%
\pgfpathlineto{\pgfqpoint{1.472102in}{0.908915in}}%
\pgfpathlineto{\pgfqpoint{1.419058in}{0.903552in}}%
\pgfpathlineto{\pgfqpoint{1.366014in}{0.829335in}}%
\pgfpathlineto{\pgfqpoint{1.312970in}{1.856216in}}%
\pgfpathlineto{\pgfqpoint{1.259927in}{1.891220in}}%
\pgfpathlineto{\pgfqpoint{1.206883in}{1.810992in}}%
\pgfpathlineto{\pgfqpoint{1.153839in}{0.893679in}}%
\pgfpathlineto{\pgfqpoint{1.100795in}{0.907342in}}%
\pgfpathlineto{\pgfqpoint{1.047751in}{0.938123in}}%
\pgfpathlineto{\pgfqpoint{0.994707in}{1.206100in}}%
\pgfpathlineto{\pgfqpoint{0.941663in}{0.942172in}}%
\pgfpathlineto{\pgfqpoint{0.941663in}{0.942172in}}%
\pgfpathclose%
\pgfusepath{stroke,fill}%
}%
\begin{pgfscope}%
\pgfsys@transformshift{0.000000in}{0.000000in}%
\pgfsys@useobject{currentmarker}{}%
\end{pgfscope}%
\end{pgfscope}%
\begin{pgfscope}%
\pgfpathrectangle{\pgfqpoint{0.941663in}{0.670138in}}{\pgfqpoint{8.858337in}{3.465625in}}%
\pgfusepath{clip}%
\pgfsetrectcap%
\pgfsetroundjoin%
\pgfsetlinewidth{1.505625pt}%
\definecolor{currentstroke}{rgb}{0.090196,0.745098,0.811765}%
\pgfsetstrokecolor{currentstroke}%
\pgfsetdash{}{0pt}%
\pgfpathmoveto{\pgfqpoint{0.941663in}{3.978235in}}%
\pgfpathlineto{\pgfqpoint{0.994707in}{3.724101in}}%
\pgfpathlineto{\pgfqpoint{1.047751in}{3.978235in}}%
\pgfpathlineto{\pgfqpoint{1.100795in}{3.966545in}}%
\pgfpathlineto{\pgfqpoint{1.153839in}{3.978235in}}%
\pgfpathlineto{\pgfqpoint{1.206883in}{3.101900in}}%
\pgfpathlineto{\pgfqpoint{1.259927in}{2.902493in}}%
\pgfpathlineto{\pgfqpoint{1.312970in}{3.006099in}}%
\pgfpathlineto{\pgfqpoint{1.366014in}{3.978235in}}%
\pgfpathlineto{\pgfqpoint{1.578190in}{3.978235in}}%
\pgfpathlineto{\pgfqpoint{1.631234in}{3.123835in}}%
\pgfpathlineto{\pgfqpoint{1.684278in}{3.440895in}}%
\pgfpathlineto{\pgfqpoint{1.737322in}{3.872158in}}%
\pgfpathlineto{\pgfqpoint{1.790366in}{3.978235in}}%
\pgfpathlineto{\pgfqpoint{1.843410in}{3.247418in}}%
\pgfpathlineto{\pgfqpoint{1.896454in}{3.219351in}}%
\pgfpathlineto{\pgfqpoint{1.949498in}{3.978235in}}%
\pgfpathlineto{\pgfqpoint{2.002542in}{3.193502in}}%
\pgfpathlineto{\pgfqpoint{2.055586in}{3.978235in}}%
\pgfpathlineto{\pgfqpoint{2.108629in}{3.577804in}}%
\pgfpathlineto{\pgfqpoint{2.161673in}{3.978235in}}%
\pgfpathlineto{\pgfqpoint{2.214717in}{3.089112in}}%
\pgfpathlineto{\pgfqpoint{2.267761in}{3.754982in}}%
\pgfpathlineto{\pgfqpoint{2.320805in}{3.978235in}}%
\pgfpathlineto{\pgfqpoint{2.373849in}{2.980147in}}%
\pgfpathlineto{\pgfqpoint{2.426893in}{2.942415in}}%
\pgfpathlineto{\pgfqpoint{2.479937in}{3.785532in}}%
\pgfpathlineto{\pgfqpoint{2.532981in}{3.978235in}}%
\pgfpathlineto{\pgfqpoint{2.798201in}{3.978235in}}%
\pgfpathlineto{\pgfqpoint{2.851245in}{3.161616in}}%
\pgfpathlineto{\pgfqpoint{2.904288in}{3.978235in}}%
\pgfpathlineto{\pgfqpoint{2.957332in}{3.172295in}}%
\pgfpathlineto{\pgfqpoint{3.010376in}{3.978235in}}%
\pgfpathlineto{\pgfqpoint{3.063420in}{3.151674in}}%
\pgfpathlineto{\pgfqpoint{3.116464in}{3.506190in}}%
\pgfpathlineto{\pgfqpoint{3.169508in}{3.978235in}}%
\pgfpathlineto{\pgfqpoint{3.328640in}{3.978235in}}%
\pgfpathlineto{\pgfqpoint{3.381684in}{3.349122in}}%
\pgfpathlineto{\pgfqpoint{3.434728in}{3.978235in}}%
\pgfpathlineto{\pgfqpoint{3.487772in}{3.083924in}}%
\pgfpathlineto{\pgfqpoint{3.540816in}{3.074646in}}%
\pgfpathlineto{\pgfqpoint{3.593860in}{3.503387in}}%
\pgfpathlineto{\pgfqpoint{3.646904in}{3.978235in}}%
\pgfpathlineto{\pgfqpoint{3.699948in}{3.978235in}}%
\pgfpathlineto{\pgfqpoint{3.752991in}{3.388260in}}%
\pgfpathlineto{\pgfqpoint{3.806035in}{3.978235in}}%
\pgfpathlineto{\pgfqpoint{3.859079in}{3.355689in}}%
\pgfpathlineto{\pgfqpoint{3.912123in}{3.408107in}}%
\pgfpathlineto{\pgfqpoint{3.965167in}{3.238518in}}%
\pgfpathlineto{\pgfqpoint{4.018211in}{3.978235in}}%
\pgfpathlineto{\pgfqpoint{4.071255in}{3.978235in}}%
\pgfpathlineto{\pgfqpoint{4.124299in}{3.875484in}}%
\pgfpathlineto{\pgfqpoint{4.177343in}{3.978235in}}%
\pgfpathlineto{\pgfqpoint{4.283431in}{3.978235in}}%
\pgfpathlineto{\pgfqpoint{4.336475in}{3.224846in}}%
\pgfpathlineto{\pgfqpoint{4.389519in}{3.841843in}}%
\pgfpathlineto{\pgfqpoint{4.442563in}{3.494347in}}%
\pgfpathlineto{\pgfqpoint{4.495607in}{3.218773in}}%
\pgfpathlineto{\pgfqpoint{4.548650in}{3.689185in}}%
\pgfpathlineto{\pgfqpoint{4.601694in}{3.162876in}}%
\pgfpathlineto{\pgfqpoint{4.654738in}{3.978235in}}%
\pgfpathlineto{\pgfqpoint{4.707782in}{3.102021in}}%
\pgfpathlineto{\pgfqpoint{4.760826in}{3.714472in}}%
\pgfpathlineto{\pgfqpoint{4.813870in}{3.978235in}}%
\pgfpathlineto{\pgfqpoint{4.866914in}{3.020692in}}%
\pgfpathlineto{\pgfqpoint{4.919958in}{3.007061in}}%
\pgfpathlineto{\pgfqpoint{4.973002in}{3.978235in}}%
\pgfpathlineto{\pgfqpoint{5.026046in}{2.958572in}}%
\pgfpathlineto{\pgfqpoint{5.079090in}{3.978235in}}%
\pgfpathlineto{\pgfqpoint{5.132134in}{2.918032in}}%
\pgfpathlineto{\pgfqpoint{5.185178in}{3.006017in}}%
\pgfpathlineto{\pgfqpoint{5.238222in}{3.002435in}}%
\pgfpathlineto{\pgfqpoint{5.291266in}{3.978235in}}%
\pgfpathlineto{\pgfqpoint{5.344309in}{3.073166in}}%
\pgfpathlineto{\pgfqpoint{5.397353in}{3.084080in}}%
\pgfpathlineto{\pgfqpoint{5.450397in}{3.213681in}}%
\pgfpathlineto{\pgfqpoint{5.503441in}{3.978235in}}%
\pgfpathlineto{\pgfqpoint{5.556485in}{3.538355in}}%
\pgfpathlineto{\pgfqpoint{5.609529in}{3.209959in}}%
\pgfpathlineto{\pgfqpoint{5.662573in}{3.261554in}}%
\pgfpathlineto{\pgfqpoint{5.715617in}{3.978235in}}%
\pgfpathlineto{\pgfqpoint{5.768661in}{3.248973in}}%
\pgfpathlineto{\pgfqpoint{5.821705in}{3.208761in}}%
\pgfpathlineto{\pgfqpoint{5.874749in}{3.132435in}}%
\pgfpathlineto{\pgfqpoint{5.927793in}{3.156691in}}%
\pgfpathlineto{\pgfqpoint{5.980837in}{3.978235in}}%
\pgfpathlineto{\pgfqpoint{6.033881in}{3.296892in}}%
\pgfpathlineto{\pgfqpoint{6.086925in}{3.978235in}}%
\pgfpathlineto{\pgfqpoint{6.139969in}{3.978235in}}%
\pgfpathlineto{\pgfqpoint{6.193012in}{2.971161in}}%
\pgfpathlineto{\pgfqpoint{6.246056in}{3.026875in}}%
\pgfpathlineto{\pgfqpoint{6.299100in}{3.978235in}}%
\pgfpathlineto{\pgfqpoint{6.352144in}{3.041125in}}%
\pgfpathlineto{\pgfqpoint{6.405188in}{2.929750in}}%
\pgfpathlineto{\pgfqpoint{6.458232in}{3.493957in}}%
\pgfpathlineto{\pgfqpoint{6.511276in}{3.978235in}}%
\pgfpathlineto{\pgfqpoint{6.723452in}{3.978235in}}%
\pgfpathlineto{\pgfqpoint{6.776496in}{3.755800in}}%
\pgfpathlineto{\pgfqpoint{6.829540in}{3.978235in}}%
\pgfpathlineto{\pgfqpoint{6.882584in}{3.217123in}}%
\pgfpathlineto{\pgfqpoint{6.935628in}{3.978235in}}%
\pgfpathlineto{\pgfqpoint{6.988671in}{3.429108in}}%
\pgfpathlineto{\pgfqpoint{7.041715in}{3.978235in}}%
\pgfpathlineto{\pgfqpoint{7.147803in}{3.978235in}}%
\pgfpathlineto{\pgfqpoint{7.200847in}{3.153935in}}%
\pgfpathlineto{\pgfqpoint{7.253891in}{3.485903in}}%
\pgfpathlineto{\pgfqpoint{7.306935in}{3.878601in}}%
\pgfpathlineto{\pgfqpoint{7.359979in}{3.007952in}}%
\pgfpathlineto{\pgfqpoint{7.413023in}{3.019856in}}%
\pgfpathlineto{\pgfqpoint{7.466067in}{3.397755in}}%
\pgfpathlineto{\pgfqpoint{7.519111in}{3.422724in}}%
\pgfpathlineto{\pgfqpoint{7.572155in}{2.929451in}}%
\pgfpathlineto{\pgfqpoint{7.625199in}{2.988504in}}%
\pgfpathlineto{\pgfqpoint{7.678243in}{3.074893in}}%
\pgfpathlineto{\pgfqpoint{7.731287in}{3.978235in}}%
\pgfpathlineto{\pgfqpoint{7.996506in}{3.978235in}}%
\pgfpathlineto{\pgfqpoint{8.049550in}{3.189369in}}%
\pgfpathlineto{\pgfqpoint{8.102594in}{3.198813in}}%
\pgfpathlineto{\pgfqpoint{8.155638in}{3.978235in}}%
\pgfpathlineto{\pgfqpoint{8.208682in}{3.978235in}}%
\pgfpathlineto{\pgfqpoint{8.261726in}{3.260476in}}%
\pgfpathlineto{\pgfqpoint{8.314770in}{3.211850in}}%
\pgfpathlineto{\pgfqpoint{8.367814in}{3.978235in}}%
\pgfpathlineto{\pgfqpoint{8.420858in}{3.193239in}}%
\pgfpathlineto{\pgfqpoint{8.473902in}{3.978235in}}%
\pgfpathlineto{\pgfqpoint{8.526946in}{3.978235in}}%
\pgfpathlineto{\pgfqpoint{8.579990in}{3.021265in}}%
\pgfpathlineto{\pgfqpoint{8.633033in}{3.030538in}}%
\pgfpathlineto{\pgfqpoint{8.686077in}{3.059450in}}%
\pgfpathlineto{\pgfqpoint{8.739121in}{2.997851in}}%
\pgfpathlineto{\pgfqpoint{8.792165in}{3.135951in}}%
\pgfpathlineto{\pgfqpoint{8.845209in}{2.978670in}}%
\pgfpathlineto{\pgfqpoint{8.898253in}{3.021456in}}%
\pgfpathlineto{\pgfqpoint{8.951297in}{3.978235in}}%
\pgfpathlineto{\pgfqpoint{9.004341in}{3.231041in}}%
\pgfpathlineto{\pgfqpoint{9.057385in}{3.978235in}}%
\pgfpathlineto{\pgfqpoint{9.110429in}{3.978235in}}%
\pgfpathlineto{\pgfqpoint{9.163473in}{3.106410in}}%
\pgfpathlineto{\pgfqpoint{9.216517in}{3.140256in}}%
\pgfpathlineto{\pgfqpoint{9.269561in}{3.147248in}}%
\pgfpathlineto{\pgfqpoint{9.322605in}{3.219668in}}%
\pgfpathlineto{\pgfqpoint{9.375649in}{3.214984in}}%
\pgfpathlineto{\pgfqpoint{9.428692in}{3.978235in}}%
\pgfpathlineto{\pgfqpoint{9.481736in}{3.910858in}}%
\pgfpathlineto{\pgfqpoint{9.534780in}{3.282563in}}%
\pgfpathlineto{\pgfqpoint{9.587824in}{3.204226in}}%
\pgfpathlineto{\pgfqpoint{9.640868in}{3.473714in}}%
\pgfpathlineto{\pgfqpoint{9.693912in}{3.139443in}}%
\pgfpathlineto{\pgfqpoint{9.746956in}{3.115475in}}%
\pgfpathlineto{\pgfqpoint{9.800000in}{3.101625in}}%
\pgfpathlineto{\pgfqpoint{9.800000in}{3.101625in}}%
\pgfusepath{stroke}%
\end{pgfscope}%
\begin{pgfscope}%
\pgfpathrectangle{\pgfqpoint{0.941663in}{0.670138in}}{\pgfqpoint{8.858337in}{3.465625in}}%
\pgfusepath{clip}%
\pgfsetbuttcap%
\pgfsetroundjoin%
\definecolor{currentfill}{rgb}{0.090196,0.745098,0.811765}%
\pgfsetfillcolor{currentfill}%
\pgfsetlinewidth{1.003750pt}%
\definecolor{currentstroke}{rgb}{0.090196,0.745098,0.811765}%
\pgfsetstrokecolor{currentstroke}%
\pgfsetdash{}{0pt}%
\pgfsys@defobject{currentmarker}{\pgfqpoint{0.941663in}{2.586892in}}{\pgfqpoint{9.800000in}{3.978235in}}{%
\pgfpathmoveto{\pgfqpoint{0.941663in}{3.978235in}}%
\pgfpathlineto{\pgfqpoint{0.941663in}{2.586892in}}%
\pgfpathlineto{\pgfqpoint{0.994707in}{2.586892in}}%
\pgfpathlineto{\pgfqpoint{1.047751in}{2.586892in}}%
\pgfpathlineto{\pgfqpoint{1.100795in}{2.586892in}}%
\pgfpathlineto{\pgfqpoint{1.153839in}{2.586892in}}%
\pgfpathlineto{\pgfqpoint{1.206883in}{2.586892in}}%
\pgfpathlineto{\pgfqpoint{1.259927in}{2.749185in}}%
\pgfpathlineto{\pgfqpoint{1.312970in}{2.586892in}}%
\pgfpathlineto{\pgfqpoint{1.366014in}{2.586892in}}%
\pgfpathlineto{\pgfqpoint{1.419058in}{2.586892in}}%
\pgfpathlineto{\pgfqpoint{1.472102in}{2.586892in}}%
\pgfpathlineto{\pgfqpoint{1.525146in}{2.586892in}}%
\pgfpathlineto{\pgfqpoint{1.578190in}{2.586892in}}%
\pgfpathlineto{\pgfqpoint{1.631234in}{2.802326in}}%
\pgfpathlineto{\pgfqpoint{1.684278in}{2.586892in}}%
\pgfpathlineto{\pgfqpoint{1.737322in}{2.586892in}}%
\pgfpathlineto{\pgfqpoint{1.790366in}{2.586892in}}%
\pgfpathlineto{\pgfqpoint{1.843410in}{3.070356in}}%
\pgfpathlineto{\pgfqpoint{1.896454in}{3.219351in}}%
\pgfpathlineto{\pgfqpoint{1.949498in}{2.586892in}}%
\pgfpathlineto{\pgfqpoint{2.002542in}{3.060121in}}%
\pgfpathlineto{\pgfqpoint{2.055586in}{2.586892in}}%
\pgfpathlineto{\pgfqpoint{2.108629in}{2.586892in}}%
\pgfpathlineto{\pgfqpoint{2.161673in}{2.586892in}}%
\pgfpathlineto{\pgfqpoint{2.214717in}{2.910735in}}%
\pgfpathlineto{\pgfqpoint{2.267761in}{2.586892in}}%
\pgfpathlineto{\pgfqpoint{2.320805in}{2.586892in}}%
\pgfpathlineto{\pgfqpoint{2.373849in}{2.844885in}}%
\pgfpathlineto{\pgfqpoint{2.426893in}{2.942415in}}%
\pgfpathlineto{\pgfqpoint{2.479937in}{2.586892in}}%
\pgfpathlineto{\pgfqpoint{2.532981in}{2.586892in}}%
\pgfpathlineto{\pgfqpoint{2.586025in}{2.586892in}}%
\pgfpathlineto{\pgfqpoint{2.639069in}{2.586892in}}%
\pgfpathlineto{\pgfqpoint{2.692113in}{2.586892in}}%
\pgfpathlineto{\pgfqpoint{2.745157in}{2.586892in}}%
\pgfpathlineto{\pgfqpoint{2.798201in}{2.586892in}}%
\pgfpathlineto{\pgfqpoint{2.851245in}{2.586892in}}%
\pgfpathlineto{\pgfqpoint{2.904288in}{2.586892in}}%
\pgfpathlineto{\pgfqpoint{2.957332in}{3.080511in}}%
\pgfpathlineto{\pgfqpoint{3.010376in}{2.586892in}}%
\pgfpathlineto{\pgfqpoint{3.063420in}{2.909016in}}%
\pgfpathlineto{\pgfqpoint{3.116464in}{2.586892in}}%
\pgfpathlineto{\pgfqpoint{3.169508in}{2.586892in}}%
\pgfpathlineto{\pgfqpoint{3.222552in}{2.586892in}}%
\pgfpathlineto{\pgfqpoint{3.275596in}{2.586892in}}%
\pgfpathlineto{\pgfqpoint{3.328640in}{2.586892in}}%
\pgfpathlineto{\pgfqpoint{3.381684in}{2.586892in}}%
\pgfpathlineto{\pgfqpoint{3.434728in}{2.586892in}}%
\pgfpathlineto{\pgfqpoint{3.487772in}{2.710315in}}%
\pgfpathlineto{\pgfqpoint{3.540816in}{2.908402in}}%
\pgfpathlineto{\pgfqpoint{3.593860in}{2.586892in}}%
\pgfpathlineto{\pgfqpoint{3.646904in}{2.586892in}}%
\pgfpathlineto{\pgfqpoint{3.699948in}{2.586892in}}%
\pgfpathlineto{\pgfqpoint{3.752991in}{2.586892in}}%
\pgfpathlineto{\pgfqpoint{3.806035in}{2.586892in}}%
\pgfpathlineto{\pgfqpoint{3.859079in}{2.586892in}}%
\pgfpathlineto{\pgfqpoint{3.912123in}{2.586892in}}%
\pgfpathlineto{\pgfqpoint{3.965167in}{2.586892in}}%
\pgfpathlineto{\pgfqpoint{4.018211in}{2.586892in}}%
\pgfpathlineto{\pgfqpoint{4.071255in}{2.586892in}}%
\pgfpathlineto{\pgfqpoint{4.124299in}{2.586892in}}%
\pgfpathlineto{\pgfqpoint{4.177343in}{2.586892in}}%
\pgfpathlineto{\pgfqpoint{4.230387in}{2.586892in}}%
\pgfpathlineto{\pgfqpoint{4.283431in}{2.586892in}}%
\pgfpathlineto{\pgfqpoint{4.336475in}{3.224846in}}%
\pgfpathlineto{\pgfqpoint{4.389519in}{2.586892in}}%
\pgfpathlineto{\pgfqpoint{4.442563in}{2.586892in}}%
\pgfpathlineto{\pgfqpoint{4.495607in}{2.968775in}}%
\pgfpathlineto{\pgfqpoint{4.548650in}{2.586892in}}%
\pgfpathlineto{\pgfqpoint{4.601694in}{2.927863in}}%
\pgfpathlineto{\pgfqpoint{4.654738in}{2.586892in}}%
\pgfpathlineto{\pgfqpoint{4.707782in}{3.102021in}}%
\pgfpathlineto{\pgfqpoint{4.760826in}{2.586892in}}%
\pgfpathlineto{\pgfqpoint{4.813870in}{2.586892in}}%
\pgfpathlineto{\pgfqpoint{4.866914in}{2.884645in}}%
\pgfpathlineto{\pgfqpoint{4.919958in}{2.863950in}}%
\pgfpathlineto{\pgfqpoint{4.973002in}{2.586892in}}%
\pgfpathlineto{\pgfqpoint{5.026046in}{2.758322in}}%
\pgfpathlineto{\pgfqpoint{5.079090in}{2.586892in}}%
\pgfpathlineto{\pgfqpoint{5.132134in}{2.824147in}}%
\pgfpathlineto{\pgfqpoint{5.185178in}{2.910114in}}%
\pgfpathlineto{\pgfqpoint{5.238222in}{2.837414in}}%
\pgfpathlineto{\pgfqpoint{5.291266in}{2.586892in}}%
\pgfpathlineto{\pgfqpoint{5.344309in}{2.705202in}}%
\pgfpathlineto{\pgfqpoint{5.397353in}{2.660769in}}%
\pgfpathlineto{\pgfqpoint{5.450397in}{3.105789in}}%
\pgfpathlineto{\pgfqpoint{5.503441in}{2.586892in}}%
\pgfpathlineto{\pgfqpoint{5.556485in}{2.586892in}}%
\pgfpathlineto{\pgfqpoint{5.609529in}{2.850209in}}%
\pgfpathlineto{\pgfqpoint{5.662573in}{2.711726in}}%
\pgfpathlineto{\pgfqpoint{5.715617in}{2.586892in}}%
\pgfpathlineto{\pgfqpoint{5.768661in}{3.248973in}}%
\pgfpathlineto{\pgfqpoint{5.821705in}{3.208761in}}%
\pgfpathlineto{\pgfqpoint{5.874749in}{3.037838in}}%
\pgfpathlineto{\pgfqpoint{5.927793in}{2.934114in}}%
\pgfpathlineto{\pgfqpoint{5.980837in}{2.586892in}}%
\pgfpathlineto{\pgfqpoint{6.033881in}{2.586892in}}%
\pgfpathlineto{\pgfqpoint{6.086925in}{2.586892in}}%
\pgfpathlineto{\pgfqpoint{6.139969in}{2.586892in}}%
\pgfpathlineto{\pgfqpoint{6.193012in}{2.971161in}}%
\pgfpathlineto{\pgfqpoint{6.246056in}{3.026875in}}%
\pgfpathlineto{\pgfqpoint{6.299100in}{2.586892in}}%
\pgfpathlineto{\pgfqpoint{6.352144in}{2.586892in}}%
\pgfpathlineto{\pgfqpoint{6.405188in}{2.699818in}}%
\pgfpathlineto{\pgfqpoint{6.458232in}{2.586892in}}%
\pgfpathlineto{\pgfqpoint{6.511276in}{2.586892in}}%
\pgfpathlineto{\pgfqpoint{6.564320in}{2.586892in}}%
\pgfpathlineto{\pgfqpoint{6.617364in}{2.586892in}}%
\pgfpathlineto{\pgfqpoint{6.670408in}{2.586892in}}%
\pgfpathlineto{\pgfqpoint{6.723452in}{2.586892in}}%
\pgfpathlineto{\pgfqpoint{6.776496in}{2.586892in}}%
\pgfpathlineto{\pgfqpoint{6.829540in}{2.586892in}}%
\pgfpathlineto{\pgfqpoint{6.882584in}{3.116001in}}%
\pgfpathlineto{\pgfqpoint{6.935628in}{2.586892in}}%
\pgfpathlineto{\pgfqpoint{6.988671in}{2.586892in}}%
\pgfpathlineto{\pgfqpoint{7.041715in}{2.586892in}}%
\pgfpathlineto{\pgfqpoint{7.094759in}{2.586892in}}%
\pgfpathlineto{\pgfqpoint{7.147803in}{2.586892in}}%
\pgfpathlineto{\pgfqpoint{7.200847in}{3.064989in}}%
\pgfpathlineto{\pgfqpoint{7.253891in}{2.586892in}}%
\pgfpathlineto{\pgfqpoint{7.306935in}{2.586892in}}%
\pgfpathlineto{\pgfqpoint{7.359979in}{2.806406in}}%
\pgfpathlineto{\pgfqpoint{7.413023in}{2.865454in}}%
\pgfpathlineto{\pgfqpoint{7.466067in}{2.586892in}}%
\pgfpathlineto{\pgfqpoint{7.519111in}{2.586892in}}%
\pgfpathlineto{\pgfqpoint{7.572155in}{2.823725in}}%
\pgfpathlineto{\pgfqpoint{7.625199in}{2.988504in}}%
\pgfpathlineto{\pgfqpoint{7.678243in}{2.586892in}}%
\pgfpathlineto{\pgfqpoint{7.731287in}{2.586892in}}%
\pgfpathlineto{\pgfqpoint{7.784330in}{2.586892in}}%
\pgfpathlineto{\pgfqpoint{7.837374in}{2.586892in}}%
\pgfpathlineto{\pgfqpoint{7.890418in}{2.586892in}}%
\pgfpathlineto{\pgfqpoint{7.943462in}{2.586892in}}%
\pgfpathlineto{\pgfqpoint{7.996506in}{2.586892in}}%
\pgfpathlineto{\pgfqpoint{8.049550in}{3.102247in}}%
\pgfpathlineto{\pgfqpoint{8.102594in}{3.118508in}}%
\pgfpathlineto{\pgfqpoint{8.155638in}{2.586892in}}%
\pgfpathlineto{\pgfqpoint{8.208682in}{2.586892in}}%
\pgfpathlineto{\pgfqpoint{8.261726in}{3.260476in}}%
\pgfpathlineto{\pgfqpoint{8.314770in}{3.211850in}}%
\pgfpathlineto{\pgfqpoint{8.367814in}{2.586892in}}%
\pgfpathlineto{\pgfqpoint{8.420858in}{2.989399in}}%
\pgfpathlineto{\pgfqpoint{8.473902in}{2.586892in}}%
\pgfpathlineto{\pgfqpoint{8.526946in}{2.586892in}}%
\pgfpathlineto{\pgfqpoint{8.579990in}{3.021265in}}%
\pgfpathlineto{\pgfqpoint{8.633033in}{2.856358in}}%
\pgfpathlineto{\pgfqpoint{8.686077in}{3.059450in}}%
\pgfpathlineto{\pgfqpoint{8.739121in}{2.825404in}}%
\pgfpathlineto{\pgfqpoint{8.792165in}{2.586892in}}%
\pgfpathlineto{\pgfqpoint{8.845209in}{2.634872in}}%
\pgfpathlineto{\pgfqpoint{8.898253in}{2.689958in}}%
\pgfpathlineto{\pgfqpoint{8.951297in}{2.586892in}}%
\pgfpathlineto{\pgfqpoint{9.004341in}{2.586892in}}%
\pgfpathlineto{\pgfqpoint{9.057385in}{2.586892in}}%
\pgfpathlineto{\pgfqpoint{9.110429in}{2.586892in}}%
\pgfpathlineto{\pgfqpoint{9.163473in}{3.004188in}}%
\pgfpathlineto{\pgfqpoint{9.216517in}{3.140256in}}%
\pgfpathlineto{\pgfqpoint{9.269561in}{2.964013in}}%
\pgfpathlineto{\pgfqpoint{9.322605in}{3.063213in}}%
\pgfpathlineto{\pgfqpoint{9.375649in}{3.090533in}}%
\pgfpathlineto{\pgfqpoint{9.428692in}{2.586892in}}%
\pgfpathlineto{\pgfqpoint{9.481736in}{2.586892in}}%
\pgfpathlineto{\pgfqpoint{9.534780in}{3.282563in}}%
\pgfpathlineto{\pgfqpoint{9.587824in}{2.956774in}}%
\pgfpathlineto{\pgfqpoint{9.640868in}{2.586892in}}%
\pgfpathlineto{\pgfqpoint{9.693912in}{2.968640in}}%
\pgfpathlineto{\pgfqpoint{9.746956in}{3.115475in}}%
\pgfpathlineto{\pgfqpoint{9.800000in}{2.909416in}}%
\pgfpathlineto{\pgfqpoint{9.800000in}{3.101625in}}%
\pgfpathlineto{\pgfqpoint{9.800000in}{3.101625in}}%
\pgfpathlineto{\pgfqpoint{9.746956in}{3.115475in}}%
\pgfpathlineto{\pgfqpoint{9.693912in}{3.139443in}}%
\pgfpathlineto{\pgfqpoint{9.640868in}{3.473714in}}%
\pgfpathlineto{\pgfqpoint{9.587824in}{3.204226in}}%
\pgfpathlineto{\pgfqpoint{9.534780in}{3.282563in}}%
\pgfpathlineto{\pgfqpoint{9.481736in}{3.910858in}}%
\pgfpathlineto{\pgfqpoint{9.428692in}{3.978235in}}%
\pgfpathlineto{\pgfqpoint{9.375649in}{3.214984in}}%
\pgfpathlineto{\pgfqpoint{9.322605in}{3.219668in}}%
\pgfpathlineto{\pgfqpoint{9.269561in}{3.147248in}}%
\pgfpathlineto{\pgfqpoint{9.216517in}{3.140256in}}%
\pgfpathlineto{\pgfqpoint{9.163473in}{3.106410in}}%
\pgfpathlineto{\pgfqpoint{9.110429in}{3.978235in}}%
\pgfpathlineto{\pgfqpoint{9.057385in}{3.978235in}}%
\pgfpathlineto{\pgfqpoint{9.004341in}{3.231041in}}%
\pgfpathlineto{\pgfqpoint{8.951297in}{3.978235in}}%
\pgfpathlineto{\pgfqpoint{8.898253in}{3.021456in}}%
\pgfpathlineto{\pgfqpoint{8.845209in}{2.978670in}}%
\pgfpathlineto{\pgfqpoint{8.792165in}{3.135951in}}%
\pgfpathlineto{\pgfqpoint{8.739121in}{2.997851in}}%
\pgfpathlineto{\pgfqpoint{8.686077in}{3.059450in}}%
\pgfpathlineto{\pgfqpoint{8.633033in}{3.030538in}}%
\pgfpathlineto{\pgfqpoint{8.579990in}{3.021265in}}%
\pgfpathlineto{\pgfqpoint{8.526946in}{3.978235in}}%
\pgfpathlineto{\pgfqpoint{8.473902in}{3.978235in}}%
\pgfpathlineto{\pgfqpoint{8.420858in}{3.193239in}}%
\pgfpathlineto{\pgfqpoint{8.367814in}{3.978235in}}%
\pgfpathlineto{\pgfqpoint{8.314770in}{3.211850in}}%
\pgfpathlineto{\pgfqpoint{8.261726in}{3.260476in}}%
\pgfpathlineto{\pgfqpoint{8.208682in}{3.978235in}}%
\pgfpathlineto{\pgfqpoint{8.155638in}{3.978235in}}%
\pgfpathlineto{\pgfqpoint{8.102594in}{3.198813in}}%
\pgfpathlineto{\pgfqpoint{8.049550in}{3.189369in}}%
\pgfpathlineto{\pgfqpoint{7.996506in}{3.978235in}}%
\pgfpathlineto{\pgfqpoint{7.943462in}{3.978235in}}%
\pgfpathlineto{\pgfqpoint{7.890418in}{3.978235in}}%
\pgfpathlineto{\pgfqpoint{7.837374in}{3.978235in}}%
\pgfpathlineto{\pgfqpoint{7.784330in}{3.978235in}}%
\pgfpathlineto{\pgfqpoint{7.731287in}{3.978235in}}%
\pgfpathlineto{\pgfqpoint{7.678243in}{3.074893in}}%
\pgfpathlineto{\pgfqpoint{7.625199in}{2.988504in}}%
\pgfpathlineto{\pgfqpoint{7.572155in}{2.929451in}}%
\pgfpathlineto{\pgfqpoint{7.519111in}{3.422724in}}%
\pgfpathlineto{\pgfqpoint{7.466067in}{3.397755in}}%
\pgfpathlineto{\pgfqpoint{7.413023in}{3.019856in}}%
\pgfpathlineto{\pgfqpoint{7.359979in}{3.007952in}}%
\pgfpathlineto{\pgfqpoint{7.306935in}{3.878601in}}%
\pgfpathlineto{\pgfqpoint{7.253891in}{3.485903in}}%
\pgfpathlineto{\pgfqpoint{7.200847in}{3.153935in}}%
\pgfpathlineto{\pgfqpoint{7.147803in}{3.978235in}}%
\pgfpathlineto{\pgfqpoint{7.094759in}{3.978235in}}%
\pgfpathlineto{\pgfqpoint{7.041715in}{3.978235in}}%
\pgfpathlineto{\pgfqpoint{6.988671in}{3.429108in}}%
\pgfpathlineto{\pgfqpoint{6.935628in}{3.978235in}}%
\pgfpathlineto{\pgfqpoint{6.882584in}{3.217123in}}%
\pgfpathlineto{\pgfqpoint{6.829540in}{3.978235in}}%
\pgfpathlineto{\pgfqpoint{6.776496in}{3.755800in}}%
\pgfpathlineto{\pgfqpoint{6.723452in}{3.978235in}}%
\pgfpathlineto{\pgfqpoint{6.670408in}{3.978235in}}%
\pgfpathlineto{\pgfqpoint{6.617364in}{3.978235in}}%
\pgfpathlineto{\pgfqpoint{6.564320in}{3.978235in}}%
\pgfpathlineto{\pgfqpoint{6.511276in}{3.978235in}}%
\pgfpathlineto{\pgfqpoint{6.458232in}{3.493957in}}%
\pgfpathlineto{\pgfqpoint{6.405188in}{2.929750in}}%
\pgfpathlineto{\pgfqpoint{6.352144in}{3.041125in}}%
\pgfpathlineto{\pgfqpoint{6.299100in}{3.978235in}}%
\pgfpathlineto{\pgfqpoint{6.246056in}{3.026875in}}%
\pgfpathlineto{\pgfqpoint{6.193012in}{2.971161in}}%
\pgfpathlineto{\pgfqpoint{6.139969in}{3.978235in}}%
\pgfpathlineto{\pgfqpoint{6.086925in}{3.978235in}}%
\pgfpathlineto{\pgfqpoint{6.033881in}{3.296892in}}%
\pgfpathlineto{\pgfqpoint{5.980837in}{3.978235in}}%
\pgfpathlineto{\pgfqpoint{5.927793in}{3.156691in}}%
\pgfpathlineto{\pgfqpoint{5.874749in}{3.132435in}}%
\pgfpathlineto{\pgfqpoint{5.821705in}{3.208761in}}%
\pgfpathlineto{\pgfqpoint{5.768661in}{3.248973in}}%
\pgfpathlineto{\pgfqpoint{5.715617in}{3.978235in}}%
\pgfpathlineto{\pgfqpoint{5.662573in}{3.261554in}}%
\pgfpathlineto{\pgfqpoint{5.609529in}{3.209959in}}%
\pgfpathlineto{\pgfqpoint{5.556485in}{3.538355in}}%
\pgfpathlineto{\pgfqpoint{5.503441in}{3.978235in}}%
\pgfpathlineto{\pgfqpoint{5.450397in}{3.213681in}}%
\pgfpathlineto{\pgfqpoint{5.397353in}{3.084080in}}%
\pgfpathlineto{\pgfqpoint{5.344309in}{3.073166in}}%
\pgfpathlineto{\pgfqpoint{5.291266in}{3.978235in}}%
\pgfpathlineto{\pgfqpoint{5.238222in}{3.002435in}}%
\pgfpathlineto{\pgfqpoint{5.185178in}{3.006017in}}%
\pgfpathlineto{\pgfqpoint{5.132134in}{2.918032in}}%
\pgfpathlineto{\pgfqpoint{5.079090in}{3.978235in}}%
\pgfpathlineto{\pgfqpoint{5.026046in}{2.958572in}}%
\pgfpathlineto{\pgfqpoint{4.973002in}{3.978235in}}%
\pgfpathlineto{\pgfqpoint{4.919958in}{3.007061in}}%
\pgfpathlineto{\pgfqpoint{4.866914in}{3.020692in}}%
\pgfpathlineto{\pgfqpoint{4.813870in}{3.978235in}}%
\pgfpathlineto{\pgfqpoint{4.760826in}{3.714472in}}%
\pgfpathlineto{\pgfqpoint{4.707782in}{3.102021in}}%
\pgfpathlineto{\pgfqpoint{4.654738in}{3.978235in}}%
\pgfpathlineto{\pgfqpoint{4.601694in}{3.162876in}}%
\pgfpathlineto{\pgfqpoint{4.548650in}{3.689185in}}%
\pgfpathlineto{\pgfqpoint{4.495607in}{3.218773in}}%
\pgfpathlineto{\pgfqpoint{4.442563in}{3.494347in}}%
\pgfpathlineto{\pgfqpoint{4.389519in}{3.841843in}}%
\pgfpathlineto{\pgfqpoint{4.336475in}{3.224846in}}%
\pgfpathlineto{\pgfqpoint{4.283431in}{3.978235in}}%
\pgfpathlineto{\pgfqpoint{4.230387in}{3.978235in}}%
\pgfpathlineto{\pgfqpoint{4.177343in}{3.978235in}}%
\pgfpathlineto{\pgfqpoint{4.124299in}{3.875484in}}%
\pgfpathlineto{\pgfqpoint{4.071255in}{3.978235in}}%
\pgfpathlineto{\pgfqpoint{4.018211in}{3.978235in}}%
\pgfpathlineto{\pgfqpoint{3.965167in}{3.238518in}}%
\pgfpathlineto{\pgfqpoint{3.912123in}{3.408107in}}%
\pgfpathlineto{\pgfqpoint{3.859079in}{3.355689in}}%
\pgfpathlineto{\pgfqpoint{3.806035in}{3.978235in}}%
\pgfpathlineto{\pgfqpoint{3.752991in}{3.388260in}}%
\pgfpathlineto{\pgfqpoint{3.699948in}{3.978235in}}%
\pgfpathlineto{\pgfqpoint{3.646904in}{3.978235in}}%
\pgfpathlineto{\pgfqpoint{3.593860in}{3.503387in}}%
\pgfpathlineto{\pgfqpoint{3.540816in}{3.074646in}}%
\pgfpathlineto{\pgfqpoint{3.487772in}{3.083924in}}%
\pgfpathlineto{\pgfqpoint{3.434728in}{3.978235in}}%
\pgfpathlineto{\pgfqpoint{3.381684in}{3.349122in}}%
\pgfpathlineto{\pgfqpoint{3.328640in}{3.978235in}}%
\pgfpathlineto{\pgfqpoint{3.275596in}{3.978235in}}%
\pgfpathlineto{\pgfqpoint{3.222552in}{3.978235in}}%
\pgfpathlineto{\pgfqpoint{3.169508in}{3.978235in}}%
\pgfpathlineto{\pgfqpoint{3.116464in}{3.506190in}}%
\pgfpathlineto{\pgfqpoint{3.063420in}{3.151674in}}%
\pgfpathlineto{\pgfqpoint{3.010376in}{3.978235in}}%
\pgfpathlineto{\pgfqpoint{2.957332in}{3.172295in}}%
\pgfpathlineto{\pgfqpoint{2.904288in}{3.978235in}}%
\pgfpathlineto{\pgfqpoint{2.851245in}{3.161616in}}%
\pgfpathlineto{\pgfqpoint{2.798201in}{3.978235in}}%
\pgfpathlineto{\pgfqpoint{2.745157in}{3.978235in}}%
\pgfpathlineto{\pgfqpoint{2.692113in}{3.978235in}}%
\pgfpathlineto{\pgfqpoint{2.639069in}{3.978235in}}%
\pgfpathlineto{\pgfqpoint{2.586025in}{3.978235in}}%
\pgfpathlineto{\pgfqpoint{2.532981in}{3.978235in}}%
\pgfpathlineto{\pgfqpoint{2.479937in}{3.785532in}}%
\pgfpathlineto{\pgfqpoint{2.426893in}{2.942415in}}%
\pgfpathlineto{\pgfqpoint{2.373849in}{2.980147in}}%
\pgfpathlineto{\pgfqpoint{2.320805in}{3.978235in}}%
\pgfpathlineto{\pgfqpoint{2.267761in}{3.754982in}}%
\pgfpathlineto{\pgfqpoint{2.214717in}{3.089112in}}%
\pgfpathlineto{\pgfqpoint{2.161673in}{3.978235in}}%
\pgfpathlineto{\pgfqpoint{2.108629in}{3.577804in}}%
\pgfpathlineto{\pgfqpoint{2.055586in}{3.978235in}}%
\pgfpathlineto{\pgfqpoint{2.002542in}{3.193502in}}%
\pgfpathlineto{\pgfqpoint{1.949498in}{3.978235in}}%
\pgfpathlineto{\pgfqpoint{1.896454in}{3.219351in}}%
\pgfpathlineto{\pgfqpoint{1.843410in}{3.247418in}}%
\pgfpathlineto{\pgfqpoint{1.790366in}{3.978235in}}%
\pgfpathlineto{\pgfqpoint{1.737322in}{3.872158in}}%
\pgfpathlineto{\pgfqpoint{1.684278in}{3.440895in}}%
\pgfpathlineto{\pgfqpoint{1.631234in}{3.123835in}}%
\pgfpathlineto{\pgfqpoint{1.578190in}{3.978235in}}%
\pgfpathlineto{\pgfqpoint{1.525146in}{3.978235in}}%
\pgfpathlineto{\pgfqpoint{1.472102in}{3.978235in}}%
\pgfpathlineto{\pgfqpoint{1.419058in}{3.978235in}}%
\pgfpathlineto{\pgfqpoint{1.366014in}{3.978235in}}%
\pgfpathlineto{\pgfqpoint{1.312970in}{3.006099in}}%
\pgfpathlineto{\pgfqpoint{1.259927in}{2.902493in}}%
\pgfpathlineto{\pgfqpoint{1.206883in}{3.101900in}}%
\pgfpathlineto{\pgfqpoint{1.153839in}{3.978235in}}%
\pgfpathlineto{\pgfqpoint{1.100795in}{3.966545in}}%
\pgfpathlineto{\pgfqpoint{1.047751in}{3.978235in}}%
\pgfpathlineto{\pgfqpoint{0.994707in}{3.724101in}}%
\pgfpathlineto{\pgfqpoint{0.941663in}{3.978235in}}%
\pgfpathlineto{\pgfqpoint{0.941663in}{3.978235in}}%
\pgfpathclose%
\pgfusepath{stroke,fill}%
}%
\begin{pgfscope}%
\pgfsys@transformshift{0.000000in}{0.000000in}%
\pgfsys@useobject{currentmarker}{}%
\end{pgfscope}%
\end{pgfscope}%
\begin{pgfscope}%
\pgfsetrectcap%
\pgfsetmiterjoin%
\pgfsetlinewidth{0.803000pt}%
\definecolor{currentstroke}{rgb}{0.000000,0.000000,0.000000}%
\pgfsetstrokecolor{currentstroke}%
\pgfsetdash{}{0pt}%
\pgfpathmoveto{\pgfqpoint{0.941663in}{0.670138in}}%
\pgfpathlineto{\pgfqpoint{0.941663in}{4.135763in}}%
\pgfusepath{stroke}%
\end{pgfscope}%
\begin{pgfscope}%
\pgfsetrectcap%
\pgfsetmiterjoin%
\pgfsetlinewidth{0.803000pt}%
\definecolor{currentstroke}{rgb}{0.000000,0.000000,0.000000}%
\pgfsetstrokecolor{currentstroke}%
\pgfsetdash{}{0pt}%
\pgfpathmoveto{\pgfqpoint{9.800000in}{0.670138in}}%
\pgfpathlineto{\pgfqpoint{9.800000in}{4.135763in}}%
\pgfusepath{stroke}%
\end{pgfscope}%
\begin{pgfscope}%
\pgfsetrectcap%
\pgfsetmiterjoin%
\pgfsetlinewidth{0.803000pt}%
\definecolor{currentstroke}{rgb}{0.000000,0.000000,0.000000}%
\pgfsetstrokecolor{currentstroke}%
\pgfsetdash{}{0pt}%
\pgfpathmoveto{\pgfqpoint{0.941663in}{0.670138in}}%
\pgfpathlineto{\pgfqpoint{9.800000in}{0.670138in}}%
\pgfusepath{stroke}%
\end{pgfscope}%
\begin{pgfscope}%
\pgfsetrectcap%
\pgfsetmiterjoin%
\pgfsetlinewidth{0.803000pt}%
\definecolor{currentstroke}{rgb}{0.000000,0.000000,0.000000}%
\pgfsetstrokecolor{currentstroke}%
\pgfsetdash{}{0pt}%
\pgfpathmoveto{\pgfqpoint{0.941663in}{4.135763in}}%
\pgfpathlineto{\pgfqpoint{9.800000in}{4.135763in}}%
\pgfusepath{stroke}%
\end{pgfscope}%
\begin{pgfscope}%
\pgfpathrectangle{\pgfqpoint{0.941663in}{0.670138in}}{\pgfqpoint{8.858337in}{3.465625in}}%
\pgfusepath{clip}%
\pgfsetbuttcap%
\pgfsetroundjoin%
\pgfsetlinewidth{1.505625pt}%
\definecolor{currentstroke}{rgb}{0.000000,0.000000,0.000000}%
\pgfsetstrokecolor{currentstroke}%
\pgfsetdash{{5.550000pt}{2.400000pt}}{0.000000pt}%
\pgfpathmoveto{\pgfqpoint{0.941663in}{3.029187in}}%
\pgfpathlineto{\pgfqpoint{0.994707in}{3.038981in}}%
\pgfpathlineto{\pgfqpoint{1.047751in}{3.025138in}}%
\pgfpathlineto{\pgfqpoint{1.100795in}{2.982667in}}%
\pgfpathlineto{\pgfqpoint{1.153839in}{2.980694in}}%
\pgfpathlineto{\pgfqpoint{1.206883in}{3.021671in}}%
\pgfpathlineto{\pgfqpoint{1.259927in}{2.902493in}}%
\pgfpathlineto{\pgfqpoint{1.312970in}{2.971095in}}%
\pgfpathlineto{\pgfqpoint{1.366014in}{2.916350in}}%
\pgfpathlineto{\pgfqpoint{1.419058in}{2.990567in}}%
\pgfpathlineto{\pgfqpoint{1.472102in}{2.995930in}}%
\pgfpathlineto{\pgfqpoint{1.525146in}{3.076292in}}%
\pgfpathlineto{\pgfqpoint{1.578190in}{3.053422in}}%
\pgfpathlineto{\pgfqpoint{1.631234in}{3.123835in}}%
\pgfpathlineto{\pgfqpoint{1.684278in}{3.108750in}}%
\pgfpathlineto{\pgfqpoint{1.737322in}{3.112329in}}%
\pgfpathlineto{\pgfqpoint{1.790366in}{3.200828in}}%
\pgfpathlineto{\pgfqpoint{1.843410in}{3.247418in}}%
\pgfpathlineto{\pgfqpoint{1.896454in}{3.219351in}}%
\pgfpathlineto{\pgfqpoint{1.949498in}{3.174139in}}%
\pgfpathlineto{\pgfqpoint{2.002542in}{3.193502in}}%
\pgfpathlineto{\pgfqpoint{2.055586in}{3.184945in}}%
\pgfpathlineto{\pgfqpoint{2.108629in}{3.168248in}}%
\pgfpathlineto{\pgfqpoint{2.161673in}{3.074593in}}%
\pgfpathlineto{\pgfqpoint{2.214717in}{3.089112in}}%
\pgfpathlineto{\pgfqpoint{2.267761in}{3.046318in}}%
\pgfpathlineto{\pgfqpoint{2.320805in}{3.005184in}}%
\pgfpathlineto{\pgfqpoint{2.373849in}{2.980147in}}%
\pgfpathlineto{\pgfqpoint{2.426893in}{2.942415in}}%
\pgfpathlineto{\pgfqpoint{2.479937in}{2.932402in}}%
\pgfpathlineto{\pgfqpoint{2.532981in}{2.980860in}}%
\pgfpathlineto{\pgfqpoint{2.586025in}{2.959049in}}%
\pgfpathlineto{\pgfqpoint{2.639069in}{2.954555in}}%
\pgfpathlineto{\pgfqpoint{2.692113in}{2.924353in}}%
\pgfpathlineto{\pgfqpoint{2.745157in}{2.995277in}}%
\pgfpathlineto{\pgfqpoint{2.798201in}{3.037672in}}%
\pgfpathlineto{\pgfqpoint{2.851245in}{3.051804in}}%
\pgfpathlineto{\pgfqpoint{2.904288in}{3.084579in}}%
\pgfpathlineto{\pgfqpoint{2.957332in}{3.172295in}}%
\pgfpathlineto{\pgfqpoint{3.010376in}{3.170343in}}%
\pgfpathlineto{\pgfqpoint{3.063420in}{3.151674in}}%
\pgfpathlineto{\pgfqpoint{3.116464in}{3.172554in}}%
\pgfpathlineto{\pgfqpoint{3.169508in}{3.202163in}}%
\pgfpathlineto{\pgfqpoint{3.222552in}{3.232686in}}%
\pgfpathlineto{\pgfqpoint{3.275596in}{3.148826in}}%
\pgfpathlineto{\pgfqpoint{3.328640in}{3.207401in}}%
\pgfpathlineto{\pgfqpoint{3.381684in}{3.124374in}}%
\pgfpathlineto{\pgfqpoint{3.434728in}{3.059787in}}%
\pgfpathlineto{\pgfqpoint{3.487772in}{3.083924in}}%
\pgfpathlineto{\pgfqpoint{3.540816in}{3.074646in}}%
\pgfpathlineto{\pgfqpoint{3.593860in}{3.000641in}}%
\pgfpathlineto{\pgfqpoint{3.646904in}{2.998087in}}%
\pgfpathlineto{\pgfqpoint{3.699948in}{2.954238in}}%
\pgfpathlineto{\pgfqpoint{3.752991in}{2.951958in}}%
\pgfpathlineto{\pgfqpoint{3.806035in}{2.994488in}}%
\pgfpathlineto{\pgfqpoint{3.859079in}{2.981887in}}%
\pgfpathlineto{\pgfqpoint{3.912123in}{2.942941in}}%
\pgfpathlineto{\pgfqpoint{3.965167in}{3.071374in}}%
\pgfpathlineto{\pgfqpoint{4.018211in}{3.000463in}}%
\pgfpathlineto{\pgfqpoint{4.071255in}{3.082554in}}%
\pgfpathlineto{\pgfqpoint{4.124299in}{3.107971in}}%
\pgfpathlineto{\pgfqpoint{4.177343in}{3.126443in}}%
\pgfpathlineto{\pgfqpoint{4.230387in}{3.164551in}}%
\pgfpathlineto{\pgfqpoint{4.283431in}{3.180269in}}%
\pgfpathlineto{\pgfqpoint{4.336475in}{3.224846in}}%
\pgfpathlineto{\pgfqpoint{4.389519in}{3.194301in}}%
\pgfpathlineto{\pgfqpoint{4.442563in}{3.183354in}}%
\pgfpathlineto{\pgfqpoint{4.495607in}{3.218773in}}%
\pgfpathlineto{\pgfqpoint{4.548650in}{3.235597in}}%
\pgfpathlineto{\pgfqpoint{4.601694in}{3.162876in}}%
\pgfpathlineto{\pgfqpoint{4.654738in}{3.181880in}}%
\pgfpathlineto{\pgfqpoint{4.707782in}{3.102021in}}%
\pgfpathlineto{\pgfqpoint{4.760826in}{3.110933in}}%
\pgfpathlineto{\pgfqpoint{4.813870in}{3.062144in}}%
\pgfpathlineto{\pgfqpoint{4.866914in}{3.020692in}}%
\pgfpathlineto{\pgfqpoint{4.919958in}{3.007061in}}%
\pgfpathlineto{\pgfqpoint{4.973002in}{2.943141in}}%
\pgfpathlineto{\pgfqpoint{5.026046in}{2.958572in}}%
\pgfpathlineto{\pgfqpoint{5.079090in}{2.914682in}}%
\pgfpathlineto{\pgfqpoint{5.132134in}{2.918032in}}%
\pgfpathlineto{\pgfqpoint{5.185178in}{3.006017in}}%
\pgfpathlineto{\pgfqpoint{5.238222in}{3.002435in}}%
\pgfpathlineto{\pgfqpoint{5.291266in}{3.062668in}}%
\pgfpathlineto{\pgfqpoint{5.344309in}{3.073166in}}%
\pgfpathlineto{\pgfqpoint{5.397353in}{3.084080in}}%
\pgfpathlineto{\pgfqpoint{5.450397in}{3.213681in}}%
\pgfpathlineto{\pgfqpoint{5.503441in}{3.203849in}}%
\pgfpathlineto{\pgfqpoint{5.556485in}{3.196655in}}%
\pgfpathlineto{\pgfqpoint{5.609529in}{3.209959in}}%
\pgfpathlineto{\pgfqpoint{5.662573in}{3.261554in}}%
\pgfpathlineto{\pgfqpoint{5.715617in}{3.209641in}}%
\pgfpathlineto{\pgfqpoint{5.768661in}{3.248973in}}%
\pgfpathlineto{\pgfqpoint{5.821705in}{3.208761in}}%
\pgfpathlineto{\pgfqpoint{5.874749in}{3.132435in}}%
\pgfpathlineto{\pgfqpoint{5.927793in}{3.156691in}}%
\pgfpathlineto{\pgfqpoint{5.980837in}{3.133284in}}%
\pgfpathlineto{\pgfqpoint{6.033881in}{3.084077in}}%
\pgfpathlineto{\pgfqpoint{6.086925in}{3.062147in}}%
\pgfpathlineto{\pgfqpoint{6.139969in}{3.043673in}}%
\pgfpathlineto{\pgfqpoint{6.193012in}{2.971161in}}%
\pgfpathlineto{\pgfqpoint{6.246056in}{3.026875in}}%
\pgfpathlineto{\pgfqpoint{6.299100in}{2.964176in}}%
\pgfpathlineto{\pgfqpoint{6.352144in}{2.982962in}}%
\pgfpathlineto{\pgfqpoint{6.405188in}{2.929750in}}%
\pgfpathlineto{\pgfqpoint{6.458232in}{3.016017in}}%
\pgfpathlineto{\pgfqpoint{6.511276in}{3.033770in}}%
\pgfpathlineto{\pgfqpoint{6.564320in}{3.030119in}}%
\pgfpathlineto{\pgfqpoint{6.617364in}{3.097076in}}%
\pgfpathlineto{\pgfqpoint{6.670408in}{3.127018in}}%
\pgfpathlineto{\pgfqpoint{6.723452in}{3.173100in}}%
\pgfpathlineto{\pgfqpoint{6.776496in}{3.194890in}}%
\pgfpathlineto{\pgfqpoint{6.829540in}{3.185067in}}%
\pgfpathlineto{\pgfqpoint{6.882584in}{3.217123in}}%
\pgfpathlineto{\pgfqpoint{6.935628in}{3.178720in}}%
\pgfpathlineto{\pgfqpoint{6.988671in}{3.222397in}}%
\pgfpathlineto{\pgfqpoint{7.041715in}{3.197983in}}%
\pgfpathlineto{\pgfqpoint{7.094759in}{3.221034in}}%
\pgfpathlineto{\pgfqpoint{7.147803in}{3.202828in}}%
\pgfpathlineto{\pgfqpoint{7.200847in}{3.153935in}}%
\pgfpathlineto{\pgfqpoint{7.253891in}{3.055048in}}%
\pgfpathlineto{\pgfqpoint{7.306935in}{3.083084in}}%
\pgfpathlineto{\pgfqpoint{7.359979in}{3.007952in}}%
\pgfpathlineto{\pgfqpoint{7.413023in}{3.019856in}}%
\pgfpathlineto{\pgfqpoint{7.466067in}{2.989900in}}%
\pgfpathlineto{\pgfqpoint{7.519111in}{2.942028in}}%
\pgfpathlineto{\pgfqpoint{7.572155in}{2.929451in}}%
\pgfpathlineto{\pgfqpoint{7.625199in}{2.988504in}}%
\pgfpathlineto{\pgfqpoint{7.678243in}{3.033419in}}%
\pgfpathlineto{\pgfqpoint{7.731287in}{2.995952in}}%
\pgfpathlineto{\pgfqpoint{7.784330in}{3.031983in}}%
\pgfpathlineto{\pgfqpoint{7.837374in}{3.040235in}}%
\pgfpathlineto{\pgfqpoint{7.890418in}{3.081537in}}%
\pgfpathlineto{\pgfqpoint{7.943462in}{3.079283in}}%
\pgfpathlineto{\pgfqpoint{7.996506in}{3.157986in}}%
\pgfpathlineto{\pgfqpoint{8.049550in}{3.189369in}}%
\pgfpathlineto{\pgfqpoint{8.102594in}{3.198813in}}%
\pgfpathlineto{\pgfqpoint{8.155638in}{3.224623in}}%
\pgfpathlineto{\pgfqpoint{8.208682in}{3.218763in}}%
\pgfpathlineto{\pgfqpoint{8.261726in}{3.260476in}}%
\pgfpathlineto{\pgfqpoint{8.314770in}{3.211850in}}%
\pgfpathlineto{\pgfqpoint{8.367814in}{3.201180in}}%
\pgfpathlineto{\pgfqpoint{8.420858in}{3.193239in}}%
\pgfpathlineto{\pgfqpoint{8.473902in}{3.141958in}}%
\pgfpathlineto{\pgfqpoint{8.526946in}{3.113817in}}%
\pgfpathlineto{\pgfqpoint{8.579990in}{3.021265in}}%
\pgfpathlineto{\pgfqpoint{8.633033in}{3.030538in}}%
\pgfpathlineto{\pgfqpoint{8.686077in}{3.059450in}}%
\pgfpathlineto{\pgfqpoint{8.739121in}{2.997851in}}%
\pgfpathlineto{\pgfqpoint{8.792165in}{3.000505in}}%
\pgfpathlineto{\pgfqpoint{8.845209in}{2.978670in}}%
\pgfpathlineto{\pgfqpoint{8.898253in}{3.021456in}}%
\pgfpathlineto{\pgfqpoint{8.951297in}{2.965023in}}%
\pgfpathlineto{\pgfqpoint{9.004341in}{2.983962in}}%
\pgfpathlineto{\pgfqpoint{9.057385in}{3.049384in}}%
\pgfpathlineto{\pgfqpoint{9.110429in}{3.084477in}}%
\pgfpathlineto{\pgfqpoint{9.163473in}{3.106410in}}%
\pgfpathlineto{\pgfqpoint{9.216517in}{3.140256in}}%
\pgfpathlineto{\pgfqpoint{9.269561in}{3.147248in}}%
\pgfpathlineto{\pgfqpoint{9.322605in}{3.219668in}}%
\pgfpathlineto{\pgfqpoint{9.375649in}{3.214984in}}%
\pgfpathlineto{\pgfqpoint{9.428692in}{3.240840in}}%
\pgfpathlineto{\pgfqpoint{9.481736in}{3.220264in}}%
\pgfpathlineto{\pgfqpoint{9.534780in}{3.282563in}}%
\pgfpathlineto{\pgfqpoint{9.587824in}{3.204226in}}%
\pgfpathlineto{\pgfqpoint{9.640868in}{3.199776in}}%
\pgfpathlineto{\pgfqpoint{9.693912in}{3.139443in}}%
\pgfpathlineto{\pgfqpoint{9.746956in}{3.115475in}}%
\pgfpathlineto{\pgfqpoint{9.800000in}{3.101625in}}%
\pgfpathlineto{\pgfqpoint{9.800000in}{3.101625in}}%
\pgfusepath{stroke}%
\end{pgfscope}%
\begin{pgfscope}%
\pgfsetbuttcap%
\pgfsetmiterjoin%
\definecolor{currentfill}{rgb}{1.000000,1.000000,1.000000}%
\pgfsetfillcolor{currentfill}%
\pgfsetlinewidth{1.003750pt}%
\definecolor{currentstroke}{rgb}{0.000000,0.000000,0.000000}%
\pgfsetstrokecolor{currentstroke}%
\pgfsetdash{}{0pt}%
\pgfpathmoveto{\pgfqpoint{1.017884in}{3.802279in}}%
\pgfpathlineto{\pgfqpoint{1.315613in}{3.802279in}}%
\pgfpathlineto{\pgfqpoint{1.315613in}{4.115057in}}%
\pgfpathlineto{\pgfqpoint{1.017884in}{4.115057in}}%
\pgfpathlineto{\pgfqpoint{1.017884in}{3.802279in}}%
\pgfpathclose%
\pgfusepath{stroke,fill}%
\end{pgfscope}%
\begin{pgfscope}%
\definecolor{textcolor}{rgb}{0.000000,0.000000,0.000000}%
\pgfsetstrokecolor{textcolor}%
\pgfsetfillcolor{textcolor}%
\pgftext[x=1.074273in,y=3.908668in,left,base]{\color{textcolor}{\rmfamily\fontsize{14.000000}{16.800000}\selectfont\catcode`\^=\active\def^{\ifmmode\sp\else\^{}\fi}\catcode`\%=\active\def%{\%}b)}}%
\end{pgfscope}%
\begin{pgfscope}%
\pgfsetbuttcap%
\pgfsetmiterjoin%
\definecolor{currentfill}{rgb}{1.000000,1.000000,1.000000}%
\pgfsetfillcolor{currentfill}%
\pgfsetfillopacity{0.800000}%
\pgfsetlinewidth{1.003750pt}%
\definecolor{currentstroke}{rgb}{0.800000,0.800000,0.800000}%
\pgfsetstrokecolor{currentstroke}%
\pgfsetstrokeopacity{0.800000}%
\pgfsetdash{}{0pt}%
\pgfpathmoveto{\pgfqpoint{1.058330in}{0.753471in}}%
\pgfpathlineto{\pgfqpoint{6.450599in}{0.753471in}}%
\pgfpathquadraticcurveto{\pgfqpoint{6.483933in}{0.753471in}}{\pgfqpoint{6.483933in}{0.786805in}}%
\pgfpathlineto{\pgfqpoint{6.483933in}{1.467359in}}%
\pgfpathquadraticcurveto{\pgfqpoint{6.483933in}{1.500693in}}{\pgfqpoint{6.450599in}{1.500693in}}%
\pgfpathlineto{\pgfqpoint{1.058330in}{1.500693in}}%
\pgfpathquadraticcurveto{\pgfqpoint{1.024996in}{1.500693in}}{\pgfqpoint{1.024996in}{1.467359in}}%
\pgfpathlineto{\pgfqpoint{1.024996in}{0.786805in}}%
\pgfpathquadraticcurveto{\pgfqpoint{1.024996in}{0.753471in}}{\pgfqpoint{1.058330in}{0.753471in}}%
\pgfpathlineto{\pgfqpoint{1.058330in}{0.753471in}}%
\pgfpathclose%
\pgfusepath{stroke,fill}%
\end{pgfscope}%
\begin{pgfscope}%
\pgfsetbuttcap%
\pgfsetmiterjoin%
\definecolor{currentfill}{rgb}{0.121569,0.466667,0.705882}%
\pgfsetfillcolor{currentfill}%
\pgfsetlinewidth{1.003750pt}%
\definecolor{currentstroke}{rgb}{0.121569,0.466667,0.705882}%
\pgfsetstrokecolor{currentstroke}%
\pgfsetdash{}{0pt}%
\pgfpathmoveto{\pgfqpoint{1.091663in}{1.317359in}}%
\pgfpathlineto{\pgfqpoint{1.424996in}{1.317359in}}%
\pgfpathlineto{\pgfqpoint{1.424996in}{1.434026in}}%
\pgfpathlineto{\pgfqpoint{1.091663in}{1.434026in}}%
\pgfpathlineto{\pgfqpoint{1.091663in}{1.317359in}}%
\pgfpathclose%
\pgfusepath{stroke,fill}%
\end{pgfscope}%
\begin{pgfscope}%
\definecolor{textcolor}{rgb}{0.000000,0.000000,0.000000}%
\pgfsetstrokecolor{textcolor}%
\pgfsetfillcolor{textcolor}%
\pgftext[x=1.558330in,y=1.317359in,left,base]{\color{textcolor}{\rmfamily\fontsize{12.000000}{14.400000}\selectfont\catcode`\^=\active\def^{\ifmmode\sp\else\^{}\fi}\catcode`\%=\active\def%{\%}Nuclear}}%
\end{pgfscope}%
\begin{pgfscope}%
\pgfsetbuttcap%
\pgfsetmiterjoin%
\definecolor{currentfill}{rgb}{0.501961,0.000000,0.501961}%
\pgfsetfillcolor{currentfill}%
\pgfsetlinewidth{1.003750pt}%
\definecolor{currentstroke}{rgb}{0.501961,0.000000,0.501961}%
\pgfsetstrokecolor{currentstroke}%
\pgfsetdash{}{0pt}%
\pgfpathmoveto{\pgfqpoint{1.091663in}{1.084952in}}%
\pgfpathlineto{\pgfqpoint{1.424996in}{1.084952in}}%
\pgfpathlineto{\pgfqpoint{1.424996in}{1.201619in}}%
\pgfpathlineto{\pgfqpoint{1.091663in}{1.201619in}}%
\pgfpathlineto{\pgfqpoint{1.091663in}{1.084952in}}%
\pgfpathclose%
\pgfusepath{stroke,fill}%
\end{pgfscope}%
\begin{pgfscope}%
\definecolor{textcolor}{rgb}{0.000000,0.000000,0.000000}%
\pgfsetstrokecolor{textcolor}%
\pgfsetfillcolor{textcolor}%
\pgftext[x=1.558330in,y=1.084952in,left,base]{\color{textcolor}{\rmfamily\fontsize{12.000000}{14.400000}\selectfont\catcode`\^=\active\def^{\ifmmode\sp\else\^{}\fi}\catcode`\%=\active\def%{\%}Battery}}%
\end{pgfscope}%
\begin{pgfscope}%
\pgfsetbuttcap%
\pgfsetmiterjoin%
\definecolor{currentfill}{rgb}{0.549020,0.337255,0.294118}%
\pgfsetfillcolor{currentfill}%
\pgfsetlinewidth{1.003750pt}%
\definecolor{currentstroke}{rgb}{0.549020,0.337255,0.294118}%
\pgfsetstrokecolor{currentstroke}%
\pgfsetdash{}{0pt}%
\pgfpathmoveto{\pgfqpoint{1.091663in}{0.852545in}}%
\pgfpathlineto{\pgfqpoint{1.424996in}{0.852545in}}%
\pgfpathlineto{\pgfqpoint{1.424996in}{0.969212in}}%
\pgfpathlineto{\pgfqpoint{1.091663in}{0.969212in}}%
\pgfpathlineto{\pgfqpoint{1.091663in}{0.852545in}}%
\pgfpathclose%
\pgfusepath{stroke,fill}%
\end{pgfscope}%
\begin{pgfscope}%
\definecolor{textcolor}{rgb}{0.000000,0.000000,0.000000}%
\pgfsetstrokecolor{textcolor}%
\pgfsetfillcolor{textcolor}%
\pgftext[x=1.558330in,y=0.852545in,left,base]{\color{textcolor}{\rmfamily\fontsize{12.000000}{14.400000}\selectfont\catcode`\^=\active\def^{\ifmmode\sp\else\^{}\fi}\catcode`\%=\active\def%{\%}NaturalGas Conv}}%
\end{pgfscope}%
\begin{pgfscope}%
\pgfsetbuttcap%
\pgfsetmiterjoin%
\definecolor{currentfill}{rgb}{1.000000,0.647059,0.000000}%
\pgfsetfillcolor{currentfill}%
\pgfsetlinewidth{1.003750pt}%
\definecolor{currentstroke}{rgb}{1.000000,0.647059,0.000000}%
\pgfsetstrokecolor{currentstroke}%
\pgfsetdash{}{0pt}%
\pgfpathmoveto{\pgfqpoint{3.140242in}{1.317359in}}%
\pgfpathlineto{\pgfqpoint{3.473575in}{1.317359in}}%
\pgfpathlineto{\pgfqpoint{3.473575in}{1.434026in}}%
\pgfpathlineto{\pgfqpoint{3.140242in}{1.434026in}}%
\pgfpathlineto{\pgfqpoint{3.140242in}{1.317359in}}%
\pgfpathclose%
\pgfusepath{stroke,fill}%
\end{pgfscope}%
\begin{pgfscope}%
\definecolor{textcolor}{rgb}{0.000000,0.000000,0.000000}%
\pgfsetstrokecolor{textcolor}%
\pgfsetfillcolor{textcolor}%
\pgftext[x=3.606909in,y=1.317359in,left,base]{\color{textcolor}{\rmfamily\fontsize{12.000000}{14.400000}\selectfont\catcode`\^=\active\def^{\ifmmode\sp\else\^{}\fi}\catcode`\%=\active\def%{\%}Battery charge}}%
\end{pgfscope}%
\begin{pgfscope}%
\pgfsetbuttcap%
\pgfsetmiterjoin%
\definecolor{currentfill}{rgb}{0.501961,0.501961,0.501961}%
\pgfsetfillcolor{currentfill}%
\pgfsetlinewidth{1.003750pt}%
\definecolor{currentstroke}{rgb}{0.501961,0.501961,0.501961}%
\pgfsetstrokecolor{currentstroke}%
\pgfsetdash{}{0pt}%
\pgfpathmoveto{\pgfqpoint{3.140242in}{1.084952in}}%
\pgfpathlineto{\pgfqpoint{3.473575in}{1.084952in}}%
\pgfpathlineto{\pgfqpoint{3.473575in}{1.201619in}}%
\pgfpathlineto{\pgfqpoint{3.140242in}{1.201619in}}%
\pgfpathlineto{\pgfqpoint{3.140242in}{1.084952in}}%
\pgfpathclose%
\pgfusepath{stroke,fill}%
\end{pgfscope}%
\begin{pgfscope}%
\definecolor{textcolor}{rgb}{0.000000,0.000000,0.000000}%
\pgfsetstrokecolor{textcolor}%
\pgfsetfillcolor{textcolor}%
\pgftext[x=3.606909in,y=1.084952in,left,base]{\color{textcolor}{\rmfamily\fontsize{12.000000}{14.400000}\selectfont\catcode`\^=\active\def^{\ifmmode\sp\else\^{}\fi}\catcode`\%=\active\def%{\%}Curtailment}}%
\end{pgfscope}%
\begin{pgfscope}%
\pgfsetbuttcap%
\pgfsetmiterjoin%
\definecolor{currentfill}{rgb}{0.090196,0.745098,0.811765}%
\pgfsetfillcolor{currentfill}%
\pgfsetlinewidth{1.003750pt}%
\definecolor{currentstroke}{rgb}{0.090196,0.745098,0.811765}%
\pgfsetstrokecolor{currentstroke}%
\pgfsetdash{}{0pt}%
\pgfpathmoveto{\pgfqpoint{4.998699in}{1.317359in}}%
\pgfpathlineto{\pgfqpoint{5.332033in}{1.317359in}}%
\pgfpathlineto{\pgfqpoint{5.332033in}{1.434026in}}%
\pgfpathlineto{\pgfqpoint{4.998699in}{1.434026in}}%
\pgfpathlineto{\pgfqpoint{4.998699in}{1.317359in}}%
\pgfpathclose%
\pgfusepath{stroke,fill}%
\end{pgfscope}%
\begin{pgfscope}%
\definecolor{textcolor}{rgb}{0.000000,0.000000,0.000000}%
\pgfsetstrokecolor{textcolor}%
\pgfsetfillcolor{textcolor}%
\pgftext[x=5.465366in,y=1.317359in,left,base]{\color{textcolor}{\rmfamily\fontsize{12.000000}{14.400000}\selectfont\catcode`\^=\active\def^{\ifmmode\sp\else\^{}\fi}\catcode`\%=\active\def%{\%}WindTurbine}}%
\end{pgfscope}%
\begin{pgfscope}%
\pgfsetbuttcap%
\pgfsetroundjoin%
\pgfsetlinewidth{1.505625pt}%
\definecolor{currentstroke}{rgb}{0.000000,0.000000,0.000000}%
\pgfsetstrokecolor{currentstroke}%
\pgfsetdash{{5.550000pt}{2.400000pt}}{0.000000pt}%
\pgfpathmoveto{\pgfqpoint{4.998699in}{1.143286in}}%
\pgfpathlineto{\pgfqpoint{5.165366in}{1.143286in}}%
\pgfpathlineto{\pgfqpoint{5.332033in}{1.143286in}}%
\pgfusepath{stroke}%
\end{pgfscope}%
\begin{pgfscope}%
\definecolor{textcolor}{rgb}{0.000000,0.000000,0.000000}%
\pgfsetstrokecolor{textcolor}%
\pgfsetfillcolor{textcolor}%
\pgftext[x=5.465366in,y=1.084952in,left,base]{\color{textcolor}{\rmfamily\fontsize{12.000000}{14.400000}\selectfont\catcode`\^=\active\def^{\ifmmode\sp\else\^{}\fi}\catcode`\%=\active\def%{\%}Demand}}%
\end{pgfscope}%
\end{pgfpicture}%
\makeatother%
\endgroup%
}
    \caption{Comparison between dispatch results for two algorithms. Plot a) was
    calculated with a logical dispatch algorithm and plot b) was calculated with
    optimal dispatch.}
    \label{fig:dispatch-comparison}
\end{figure}

The optimal dispatch algorithm uses a linear programming formulation to arrive
at an optimal solution with perfect foresight. The logical dispatch algorithm
uses a rule-based approach to dispatch energy according to merit order.
However, this algorithm is myopic since dispatch is calculated serially. These
differences totally account for the differences in their dispatch results. The
optimal dispatch algorithm uses battery storage more effectively than its
rule-based counterpart because it optimizes the entire time series at once.
Since the logical algorithm uses battery storage imperfectly, it fills the
energy gaps with natural gas and more energy is curtailed rather than used.
Although, the optimal dispatch solution performs better on a pure cost basis, the
myopia of the logical dispatch algorithm is possibly more realistic.
Further, since the logical dispatch algorithm does not have an energy balance
constraint for all time steps, users can more easily estimate reliability and
calculate costs from energy shortfalls.


\FloatBarrier

\subsection{Exercise 2: Time Scaling}

This exercise considers how the two dispatch algorithms scale with simulation duration.
For this exercise, the two algorithms were placed within a
\texttt{CapacityExpansion} problem with the parameters described in Table
\ref{tab:scaling-ga-params}.

\begin{table}[htbp!]
    \centering
    \caption{Capacity expansion parameters for the algorithm comparison exercise.}
    \label{tab:scaling-ga-params}
    \begin{tabular}{ll}
        \toprule
        Parameter & Value \\
        \midrule
        Algorithm & \acs{nsga2}\\
        Termination Criterion & Maximum generations\\
        Generations & 10 \\
        Population Size & 20 \\
        Objectives & 2 (cost, emissions)\\
        Threads & 1 \\
        \bottomrule
    \end{tabular}
\end{table}

\noindent The available technologies were the same four as in the previous
exercise. Rather than scaling the problem by number of objectives, technologies,
or population size, this exercise scales the problem by the length of the time
series. This is preferred because time series data typically increases the
problem size more dramatically than the number of objectives or number of
technologies. Further, scaling by population size would obfuscate the
differences between the two algorithms since neither are affected by population
size. Figure \ref{fig:alg-scaling} shows results of this scaling study. The
x-axis measures the number of modeled days at an hourly resolution.

\begin{figure}[htbp!]
    \centering
    \resizebox{0.75\columnwidth}{!}{%% Creator: Matplotlib, PGF backend
%%
%% To include the figure in your LaTeX document, write
%%   \input{<filename>.pgf}
%%
%% Make sure the required packages are loaded in your preamble
%%   \usepackage{pgf}
%%
%% Also ensure that all the required font packages are loaded; for instance,
%% the lmodern package is sometimes necessary when using math font.
%%   \usepackage{lmodern}
%%
%% Figures using additional raster images can only be included by \input if
%% they are in the same directory as the main LaTeX file. For loading figures
%% from other directories you can use the `import` package
%%   \usepackage{import}
%%
%% and then include the figures with
%%   \import{<path to file>}{<filename>.pgf}
%%
%% Matplotlib used the following preamble
%%   \def\mathdefault#1{#1}
%%   \everymath=\expandafter{\the\everymath\displaystyle}
%%   \IfFileExists{scrextend.sty}{
%%     \usepackage[fontsize=10.000000pt]{scrextend}
%%   }{
%%     \renewcommand{\normalsize}{\fontsize{10.000000}{12.000000}\selectfont}
%%     \normalsize
%%   }
%%   
%%   \makeatletter\@ifpackageloaded{underscore}{}{\usepackage[strings]{underscore}}\makeatother
%%
\begingroup%
\makeatletter%
\begin{pgfpicture}%
\pgfpathrectangle{\pgfpointorigin}{\pgfqpoint{7.064581in}{5.411797in}}%
\pgfusepath{use as bounding box, clip}%
\begin{pgfscope}%
\pgfsetbuttcap%
\pgfsetmiterjoin%
\definecolor{currentfill}{rgb}{1.000000,1.000000,1.000000}%
\pgfsetfillcolor{currentfill}%
\pgfsetlinewidth{0.000000pt}%
\definecolor{currentstroke}{rgb}{0.000000,0.000000,0.000000}%
\pgfsetstrokecolor{currentstroke}%
\pgfsetdash{}{0pt}%
\pgfpathmoveto{\pgfqpoint{0.000000in}{0.000000in}}%
\pgfpathlineto{\pgfqpoint{7.064581in}{0.000000in}}%
\pgfpathlineto{\pgfqpoint{7.064581in}{5.411797in}}%
\pgfpathlineto{\pgfqpoint{0.000000in}{5.411797in}}%
\pgfpathlineto{\pgfqpoint{0.000000in}{0.000000in}}%
\pgfpathclose%
\pgfusepath{fill}%
\end{pgfscope}%
\begin{pgfscope}%
\pgfsetbuttcap%
\pgfsetmiterjoin%
\definecolor{currentfill}{rgb}{1.000000,1.000000,1.000000}%
\pgfsetfillcolor{currentfill}%
\pgfsetlinewidth{0.000000pt}%
\definecolor{currentstroke}{rgb}{0.000000,0.000000,0.000000}%
\pgfsetstrokecolor{currentstroke}%
\pgfsetstrokeopacity{0.000000}%
\pgfsetdash{}{0pt}%
\pgfpathmoveto{\pgfqpoint{0.764581in}{0.643904in}}%
\pgfpathlineto{\pgfqpoint{6.964581in}{0.643904in}}%
\pgfpathlineto{\pgfqpoint{6.964581in}{5.263904in}}%
\pgfpathlineto{\pgfqpoint{0.764581in}{5.263904in}}%
\pgfpathlineto{\pgfqpoint{0.764581in}{0.643904in}}%
\pgfpathclose%
\pgfusepath{fill}%
\end{pgfscope}%
\begin{pgfscope}%
\pgfpathrectangle{\pgfqpoint{0.764581in}{0.643904in}}{\pgfqpoint{6.200000in}{4.620000in}}%
\pgfusepath{clip}%
\pgfsetrectcap%
\pgfsetroundjoin%
\pgfsetlinewidth{0.803000pt}%
\definecolor{currentstroke}{rgb}{0.690196,0.690196,0.690196}%
\pgfsetstrokecolor{currentstroke}%
\pgfsetdash{}{0pt}%
\pgfpathmoveto{\pgfqpoint{0.764581in}{0.643904in}}%
\pgfpathlineto{\pgfqpoint{0.764581in}{5.263904in}}%
\pgfusepath{stroke}%
\end{pgfscope}%
\begin{pgfscope}%
\pgfsetbuttcap%
\pgfsetroundjoin%
\definecolor{currentfill}{rgb}{0.000000,0.000000,0.000000}%
\pgfsetfillcolor{currentfill}%
\pgfsetlinewidth{0.803000pt}%
\definecolor{currentstroke}{rgb}{0.000000,0.000000,0.000000}%
\pgfsetstrokecolor{currentstroke}%
\pgfsetdash{}{0pt}%
\pgfsys@defobject{currentmarker}{\pgfqpoint{0.000000in}{-0.048611in}}{\pgfqpoint{0.000000in}{0.000000in}}{%
\pgfpathmoveto{\pgfqpoint{0.000000in}{0.000000in}}%
\pgfpathlineto{\pgfqpoint{0.000000in}{-0.048611in}}%
\pgfusepath{stroke,fill}%
}%
\begin{pgfscope}%
\pgfsys@transformshift{0.764581in}{0.643904in}%
\pgfsys@useobject{currentmarker}{}%
\end{pgfscope}%
\end{pgfscope}%
\begin{pgfscope}%
\definecolor{textcolor}{rgb}{0.000000,0.000000,0.000000}%
\pgfsetstrokecolor{textcolor}%
\pgfsetfillcolor{textcolor}%
\pgftext[x=0.764581in,y=0.546682in,,top]{\color{textcolor}{\rmfamily\fontsize{14.000000}{16.800000}\selectfont\catcode`\^=\active\def^{\ifmmode\sp\else\^{}\fi}\catcode`\%=\active\def%{\%}$\mathdefault{10^{0}}$}}%
\end{pgfscope}%
\begin{pgfscope}%
\pgfpathrectangle{\pgfqpoint{0.764581in}{0.643904in}}{\pgfqpoint{6.200000in}{4.620000in}}%
\pgfusepath{clip}%
\pgfsetrectcap%
\pgfsetroundjoin%
\pgfsetlinewidth{0.803000pt}%
\definecolor{currentstroke}{rgb}{0.690196,0.690196,0.690196}%
\pgfsetstrokecolor{currentstroke}%
\pgfsetdash{}{0pt}%
\pgfpathmoveto{\pgfqpoint{3.184288in}{0.643904in}}%
\pgfpathlineto{\pgfqpoint{3.184288in}{5.263904in}}%
\pgfusepath{stroke}%
\end{pgfscope}%
\begin{pgfscope}%
\pgfsetbuttcap%
\pgfsetroundjoin%
\definecolor{currentfill}{rgb}{0.000000,0.000000,0.000000}%
\pgfsetfillcolor{currentfill}%
\pgfsetlinewidth{0.803000pt}%
\definecolor{currentstroke}{rgb}{0.000000,0.000000,0.000000}%
\pgfsetstrokecolor{currentstroke}%
\pgfsetdash{}{0pt}%
\pgfsys@defobject{currentmarker}{\pgfqpoint{0.000000in}{-0.048611in}}{\pgfqpoint{0.000000in}{0.000000in}}{%
\pgfpathmoveto{\pgfqpoint{0.000000in}{0.000000in}}%
\pgfpathlineto{\pgfqpoint{0.000000in}{-0.048611in}}%
\pgfusepath{stroke,fill}%
}%
\begin{pgfscope}%
\pgfsys@transformshift{3.184288in}{0.643904in}%
\pgfsys@useobject{currentmarker}{}%
\end{pgfscope}%
\end{pgfscope}%
\begin{pgfscope}%
\definecolor{textcolor}{rgb}{0.000000,0.000000,0.000000}%
\pgfsetstrokecolor{textcolor}%
\pgfsetfillcolor{textcolor}%
\pgftext[x=3.184288in,y=0.546682in,,top]{\color{textcolor}{\rmfamily\fontsize{14.000000}{16.800000}\selectfont\catcode`\^=\active\def^{\ifmmode\sp\else\^{}\fi}\catcode`\%=\active\def%{\%}$\mathdefault{10^{1}}$}}%
\end{pgfscope}%
\begin{pgfscope}%
\pgfpathrectangle{\pgfqpoint{0.764581in}{0.643904in}}{\pgfqpoint{6.200000in}{4.620000in}}%
\pgfusepath{clip}%
\pgfsetrectcap%
\pgfsetroundjoin%
\pgfsetlinewidth{0.803000pt}%
\definecolor{currentstroke}{rgb}{0.690196,0.690196,0.690196}%
\pgfsetstrokecolor{currentstroke}%
\pgfsetdash{}{0pt}%
\pgfpathmoveto{\pgfqpoint{5.603996in}{0.643904in}}%
\pgfpathlineto{\pgfqpoint{5.603996in}{5.263904in}}%
\pgfusepath{stroke}%
\end{pgfscope}%
\begin{pgfscope}%
\pgfsetbuttcap%
\pgfsetroundjoin%
\definecolor{currentfill}{rgb}{0.000000,0.000000,0.000000}%
\pgfsetfillcolor{currentfill}%
\pgfsetlinewidth{0.803000pt}%
\definecolor{currentstroke}{rgb}{0.000000,0.000000,0.000000}%
\pgfsetstrokecolor{currentstroke}%
\pgfsetdash{}{0pt}%
\pgfsys@defobject{currentmarker}{\pgfqpoint{0.000000in}{-0.048611in}}{\pgfqpoint{0.000000in}{0.000000in}}{%
\pgfpathmoveto{\pgfqpoint{0.000000in}{0.000000in}}%
\pgfpathlineto{\pgfqpoint{0.000000in}{-0.048611in}}%
\pgfusepath{stroke,fill}%
}%
\begin{pgfscope}%
\pgfsys@transformshift{5.603996in}{0.643904in}%
\pgfsys@useobject{currentmarker}{}%
\end{pgfscope}%
\end{pgfscope}%
\begin{pgfscope}%
\definecolor{textcolor}{rgb}{0.000000,0.000000,0.000000}%
\pgfsetstrokecolor{textcolor}%
\pgfsetfillcolor{textcolor}%
\pgftext[x=5.603996in,y=0.546682in,,top]{\color{textcolor}{\rmfamily\fontsize{14.000000}{16.800000}\selectfont\catcode`\^=\active\def^{\ifmmode\sp\else\^{}\fi}\catcode`\%=\active\def%{\%}$\mathdefault{10^{2}}$}}%
\end{pgfscope}%
\begin{pgfscope}%
\pgfpathrectangle{\pgfqpoint{0.764581in}{0.643904in}}{\pgfqpoint{6.200000in}{4.620000in}}%
\pgfusepath{clip}%
\pgfsetbuttcap%
\pgfsetroundjoin%
\pgfsetlinewidth{0.803000pt}%
\definecolor{currentstroke}{rgb}{0.690196,0.690196,0.690196}%
\pgfsetstrokecolor{currentstroke}%
\pgfsetstrokeopacity{0.200000}%
\pgfsetdash{{2.960000pt}{1.280000pt}}{0.000000pt}%
\pgfpathmoveto{\pgfqpoint{1.492985in}{0.643904in}}%
\pgfpathlineto{\pgfqpoint{1.492985in}{5.263904in}}%
\pgfusepath{stroke}%
\end{pgfscope}%
\begin{pgfscope}%
\pgfsetbuttcap%
\pgfsetroundjoin%
\definecolor{currentfill}{rgb}{0.000000,0.000000,0.000000}%
\pgfsetfillcolor{currentfill}%
\pgfsetlinewidth{0.602250pt}%
\definecolor{currentstroke}{rgb}{0.000000,0.000000,0.000000}%
\pgfsetstrokecolor{currentstroke}%
\pgfsetdash{}{0pt}%
\pgfsys@defobject{currentmarker}{\pgfqpoint{0.000000in}{-0.027778in}}{\pgfqpoint{0.000000in}{0.000000in}}{%
\pgfpathmoveto{\pgfqpoint{0.000000in}{0.000000in}}%
\pgfpathlineto{\pgfqpoint{0.000000in}{-0.027778in}}%
\pgfusepath{stroke,fill}%
}%
\begin{pgfscope}%
\pgfsys@transformshift{1.492985in}{0.643904in}%
\pgfsys@useobject{currentmarker}{}%
\end{pgfscope}%
\end{pgfscope}%
\begin{pgfscope}%
\pgfpathrectangle{\pgfqpoint{0.764581in}{0.643904in}}{\pgfqpoint{6.200000in}{4.620000in}}%
\pgfusepath{clip}%
\pgfsetbuttcap%
\pgfsetroundjoin%
\pgfsetlinewidth{0.803000pt}%
\definecolor{currentstroke}{rgb}{0.690196,0.690196,0.690196}%
\pgfsetstrokecolor{currentstroke}%
\pgfsetstrokeopacity{0.200000}%
\pgfsetdash{{2.960000pt}{1.280000pt}}{0.000000pt}%
\pgfpathmoveto{\pgfqpoint{1.919075in}{0.643904in}}%
\pgfpathlineto{\pgfqpoint{1.919075in}{5.263904in}}%
\pgfusepath{stroke}%
\end{pgfscope}%
\begin{pgfscope}%
\pgfsetbuttcap%
\pgfsetroundjoin%
\definecolor{currentfill}{rgb}{0.000000,0.000000,0.000000}%
\pgfsetfillcolor{currentfill}%
\pgfsetlinewidth{0.602250pt}%
\definecolor{currentstroke}{rgb}{0.000000,0.000000,0.000000}%
\pgfsetstrokecolor{currentstroke}%
\pgfsetdash{}{0pt}%
\pgfsys@defobject{currentmarker}{\pgfqpoint{0.000000in}{-0.027778in}}{\pgfqpoint{0.000000in}{0.000000in}}{%
\pgfpathmoveto{\pgfqpoint{0.000000in}{0.000000in}}%
\pgfpathlineto{\pgfqpoint{0.000000in}{-0.027778in}}%
\pgfusepath{stroke,fill}%
}%
\begin{pgfscope}%
\pgfsys@transformshift{1.919075in}{0.643904in}%
\pgfsys@useobject{currentmarker}{}%
\end{pgfscope}%
\end{pgfscope}%
\begin{pgfscope}%
\pgfpathrectangle{\pgfqpoint{0.764581in}{0.643904in}}{\pgfqpoint{6.200000in}{4.620000in}}%
\pgfusepath{clip}%
\pgfsetbuttcap%
\pgfsetroundjoin%
\pgfsetlinewidth{0.803000pt}%
\definecolor{currentstroke}{rgb}{0.690196,0.690196,0.690196}%
\pgfsetstrokecolor{currentstroke}%
\pgfsetstrokeopacity{0.200000}%
\pgfsetdash{{2.960000pt}{1.280000pt}}{0.000000pt}%
\pgfpathmoveto{\pgfqpoint{2.221390in}{0.643904in}}%
\pgfpathlineto{\pgfqpoint{2.221390in}{5.263904in}}%
\pgfusepath{stroke}%
\end{pgfscope}%
\begin{pgfscope}%
\pgfsetbuttcap%
\pgfsetroundjoin%
\definecolor{currentfill}{rgb}{0.000000,0.000000,0.000000}%
\pgfsetfillcolor{currentfill}%
\pgfsetlinewidth{0.602250pt}%
\definecolor{currentstroke}{rgb}{0.000000,0.000000,0.000000}%
\pgfsetstrokecolor{currentstroke}%
\pgfsetdash{}{0pt}%
\pgfsys@defobject{currentmarker}{\pgfqpoint{0.000000in}{-0.027778in}}{\pgfqpoint{0.000000in}{0.000000in}}{%
\pgfpathmoveto{\pgfqpoint{0.000000in}{0.000000in}}%
\pgfpathlineto{\pgfqpoint{0.000000in}{-0.027778in}}%
\pgfusepath{stroke,fill}%
}%
\begin{pgfscope}%
\pgfsys@transformshift{2.221390in}{0.643904in}%
\pgfsys@useobject{currentmarker}{}%
\end{pgfscope}%
\end{pgfscope}%
\begin{pgfscope}%
\pgfpathrectangle{\pgfqpoint{0.764581in}{0.643904in}}{\pgfqpoint{6.200000in}{4.620000in}}%
\pgfusepath{clip}%
\pgfsetbuttcap%
\pgfsetroundjoin%
\pgfsetlinewidth{0.803000pt}%
\definecolor{currentstroke}{rgb}{0.690196,0.690196,0.690196}%
\pgfsetstrokecolor{currentstroke}%
\pgfsetstrokeopacity{0.200000}%
\pgfsetdash{{2.960000pt}{1.280000pt}}{0.000000pt}%
\pgfpathmoveto{\pgfqpoint{2.455884in}{0.643904in}}%
\pgfpathlineto{\pgfqpoint{2.455884in}{5.263904in}}%
\pgfusepath{stroke}%
\end{pgfscope}%
\begin{pgfscope}%
\pgfsetbuttcap%
\pgfsetroundjoin%
\definecolor{currentfill}{rgb}{0.000000,0.000000,0.000000}%
\pgfsetfillcolor{currentfill}%
\pgfsetlinewidth{0.602250pt}%
\definecolor{currentstroke}{rgb}{0.000000,0.000000,0.000000}%
\pgfsetstrokecolor{currentstroke}%
\pgfsetdash{}{0pt}%
\pgfsys@defobject{currentmarker}{\pgfqpoint{0.000000in}{-0.027778in}}{\pgfqpoint{0.000000in}{0.000000in}}{%
\pgfpathmoveto{\pgfqpoint{0.000000in}{0.000000in}}%
\pgfpathlineto{\pgfqpoint{0.000000in}{-0.027778in}}%
\pgfusepath{stroke,fill}%
}%
\begin{pgfscope}%
\pgfsys@transformshift{2.455884in}{0.643904in}%
\pgfsys@useobject{currentmarker}{}%
\end{pgfscope}%
\end{pgfscope}%
\begin{pgfscope}%
\pgfpathrectangle{\pgfqpoint{0.764581in}{0.643904in}}{\pgfqpoint{6.200000in}{4.620000in}}%
\pgfusepath{clip}%
\pgfsetbuttcap%
\pgfsetroundjoin%
\pgfsetlinewidth{0.803000pt}%
\definecolor{currentstroke}{rgb}{0.690196,0.690196,0.690196}%
\pgfsetstrokecolor{currentstroke}%
\pgfsetstrokeopacity{0.200000}%
\pgfsetdash{{2.960000pt}{1.280000pt}}{0.000000pt}%
\pgfpathmoveto{\pgfqpoint{2.647479in}{0.643904in}}%
\pgfpathlineto{\pgfqpoint{2.647479in}{5.263904in}}%
\pgfusepath{stroke}%
\end{pgfscope}%
\begin{pgfscope}%
\pgfsetbuttcap%
\pgfsetroundjoin%
\definecolor{currentfill}{rgb}{0.000000,0.000000,0.000000}%
\pgfsetfillcolor{currentfill}%
\pgfsetlinewidth{0.602250pt}%
\definecolor{currentstroke}{rgb}{0.000000,0.000000,0.000000}%
\pgfsetstrokecolor{currentstroke}%
\pgfsetdash{}{0pt}%
\pgfsys@defobject{currentmarker}{\pgfqpoint{0.000000in}{-0.027778in}}{\pgfqpoint{0.000000in}{0.000000in}}{%
\pgfpathmoveto{\pgfqpoint{0.000000in}{0.000000in}}%
\pgfpathlineto{\pgfqpoint{0.000000in}{-0.027778in}}%
\pgfusepath{stroke,fill}%
}%
\begin{pgfscope}%
\pgfsys@transformshift{2.647479in}{0.643904in}%
\pgfsys@useobject{currentmarker}{}%
\end{pgfscope}%
\end{pgfscope}%
\begin{pgfscope}%
\pgfpathrectangle{\pgfqpoint{0.764581in}{0.643904in}}{\pgfqpoint{6.200000in}{4.620000in}}%
\pgfusepath{clip}%
\pgfsetbuttcap%
\pgfsetroundjoin%
\pgfsetlinewidth{0.803000pt}%
\definecolor{currentstroke}{rgb}{0.690196,0.690196,0.690196}%
\pgfsetstrokecolor{currentstroke}%
\pgfsetstrokeopacity{0.200000}%
\pgfsetdash{{2.960000pt}{1.280000pt}}{0.000000pt}%
\pgfpathmoveto{\pgfqpoint{2.809471in}{0.643904in}}%
\pgfpathlineto{\pgfqpoint{2.809471in}{5.263904in}}%
\pgfusepath{stroke}%
\end{pgfscope}%
\begin{pgfscope}%
\pgfsetbuttcap%
\pgfsetroundjoin%
\definecolor{currentfill}{rgb}{0.000000,0.000000,0.000000}%
\pgfsetfillcolor{currentfill}%
\pgfsetlinewidth{0.602250pt}%
\definecolor{currentstroke}{rgb}{0.000000,0.000000,0.000000}%
\pgfsetstrokecolor{currentstroke}%
\pgfsetdash{}{0pt}%
\pgfsys@defobject{currentmarker}{\pgfqpoint{0.000000in}{-0.027778in}}{\pgfqpoint{0.000000in}{0.000000in}}{%
\pgfpathmoveto{\pgfqpoint{0.000000in}{0.000000in}}%
\pgfpathlineto{\pgfqpoint{0.000000in}{-0.027778in}}%
\pgfusepath{stroke,fill}%
}%
\begin{pgfscope}%
\pgfsys@transformshift{2.809471in}{0.643904in}%
\pgfsys@useobject{currentmarker}{}%
\end{pgfscope}%
\end{pgfscope}%
\begin{pgfscope}%
\pgfpathrectangle{\pgfqpoint{0.764581in}{0.643904in}}{\pgfqpoint{6.200000in}{4.620000in}}%
\pgfusepath{clip}%
\pgfsetbuttcap%
\pgfsetroundjoin%
\pgfsetlinewidth{0.803000pt}%
\definecolor{currentstroke}{rgb}{0.690196,0.690196,0.690196}%
\pgfsetstrokecolor{currentstroke}%
\pgfsetstrokeopacity{0.200000}%
\pgfsetdash{{2.960000pt}{1.280000pt}}{0.000000pt}%
\pgfpathmoveto{\pgfqpoint{2.949795in}{0.643904in}}%
\pgfpathlineto{\pgfqpoint{2.949795in}{5.263904in}}%
\pgfusepath{stroke}%
\end{pgfscope}%
\begin{pgfscope}%
\pgfsetbuttcap%
\pgfsetroundjoin%
\definecolor{currentfill}{rgb}{0.000000,0.000000,0.000000}%
\pgfsetfillcolor{currentfill}%
\pgfsetlinewidth{0.602250pt}%
\definecolor{currentstroke}{rgb}{0.000000,0.000000,0.000000}%
\pgfsetstrokecolor{currentstroke}%
\pgfsetdash{}{0pt}%
\pgfsys@defobject{currentmarker}{\pgfqpoint{0.000000in}{-0.027778in}}{\pgfqpoint{0.000000in}{0.000000in}}{%
\pgfpathmoveto{\pgfqpoint{0.000000in}{0.000000in}}%
\pgfpathlineto{\pgfqpoint{0.000000in}{-0.027778in}}%
\pgfusepath{stroke,fill}%
}%
\begin{pgfscope}%
\pgfsys@transformshift{2.949795in}{0.643904in}%
\pgfsys@useobject{currentmarker}{}%
\end{pgfscope}%
\end{pgfscope}%
\begin{pgfscope}%
\pgfpathrectangle{\pgfqpoint{0.764581in}{0.643904in}}{\pgfqpoint{6.200000in}{4.620000in}}%
\pgfusepath{clip}%
\pgfsetbuttcap%
\pgfsetroundjoin%
\pgfsetlinewidth{0.803000pt}%
\definecolor{currentstroke}{rgb}{0.690196,0.690196,0.690196}%
\pgfsetstrokecolor{currentstroke}%
\pgfsetstrokeopacity{0.200000}%
\pgfsetdash{{2.960000pt}{1.280000pt}}{0.000000pt}%
\pgfpathmoveto{\pgfqpoint{3.073569in}{0.643904in}}%
\pgfpathlineto{\pgfqpoint{3.073569in}{5.263904in}}%
\pgfusepath{stroke}%
\end{pgfscope}%
\begin{pgfscope}%
\pgfsetbuttcap%
\pgfsetroundjoin%
\definecolor{currentfill}{rgb}{0.000000,0.000000,0.000000}%
\pgfsetfillcolor{currentfill}%
\pgfsetlinewidth{0.602250pt}%
\definecolor{currentstroke}{rgb}{0.000000,0.000000,0.000000}%
\pgfsetstrokecolor{currentstroke}%
\pgfsetdash{}{0pt}%
\pgfsys@defobject{currentmarker}{\pgfqpoint{0.000000in}{-0.027778in}}{\pgfqpoint{0.000000in}{0.000000in}}{%
\pgfpathmoveto{\pgfqpoint{0.000000in}{0.000000in}}%
\pgfpathlineto{\pgfqpoint{0.000000in}{-0.027778in}}%
\pgfusepath{stroke,fill}%
}%
\begin{pgfscope}%
\pgfsys@transformshift{3.073569in}{0.643904in}%
\pgfsys@useobject{currentmarker}{}%
\end{pgfscope}%
\end{pgfscope}%
\begin{pgfscope}%
\pgfpathrectangle{\pgfqpoint{0.764581in}{0.643904in}}{\pgfqpoint{6.200000in}{4.620000in}}%
\pgfusepath{clip}%
\pgfsetbuttcap%
\pgfsetroundjoin%
\pgfsetlinewidth{0.803000pt}%
\definecolor{currentstroke}{rgb}{0.690196,0.690196,0.690196}%
\pgfsetstrokecolor{currentstroke}%
\pgfsetstrokeopacity{0.200000}%
\pgfsetdash{{2.960000pt}{1.280000pt}}{0.000000pt}%
\pgfpathmoveto{\pgfqpoint{3.912693in}{0.643904in}}%
\pgfpathlineto{\pgfqpoint{3.912693in}{5.263904in}}%
\pgfusepath{stroke}%
\end{pgfscope}%
\begin{pgfscope}%
\pgfsetbuttcap%
\pgfsetroundjoin%
\definecolor{currentfill}{rgb}{0.000000,0.000000,0.000000}%
\pgfsetfillcolor{currentfill}%
\pgfsetlinewidth{0.602250pt}%
\definecolor{currentstroke}{rgb}{0.000000,0.000000,0.000000}%
\pgfsetstrokecolor{currentstroke}%
\pgfsetdash{}{0pt}%
\pgfsys@defobject{currentmarker}{\pgfqpoint{0.000000in}{-0.027778in}}{\pgfqpoint{0.000000in}{0.000000in}}{%
\pgfpathmoveto{\pgfqpoint{0.000000in}{0.000000in}}%
\pgfpathlineto{\pgfqpoint{0.000000in}{-0.027778in}}%
\pgfusepath{stroke,fill}%
}%
\begin{pgfscope}%
\pgfsys@transformshift{3.912693in}{0.643904in}%
\pgfsys@useobject{currentmarker}{}%
\end{pgfscope}%
\end{pgfscope}%
\begin{pgfscope}%
\pgfpathrectangle{\pgfqpoint{0.764581in}{0.643904in}}{\pgfqpoint{6.200000in}{4.620000in}}%
\pgfusepath{clip}%
\pgfsetbuttcap%
\pgfsetroundjoin%
\pgfsetlinewidth{0.803000pt}%
\definecolor{currentstroke}{rgb}{0.690196,0.690196,0.690196}%
\pgfsetstrokecolor{currentstroke}%
\pgfsetstrokeopacity{0.200000}%
\pgfsetdash{{2.960000pt}{1.280000pt}}{0.000000pt}%
\pgfpathmoveto{\pgfqpoint{4.338782in}{0.643904in}}%
\pgfpathlineto{\pgfqpoint{4.338782in}{5.263904in}}%
\pgfusepath{stroke}%
\end{pgfscope}%
\begin{pgfscope}%
\pgfsetbuttcap%
\pgfsetroundjoin%
\definecolor{currentfill}{rgb}{0.000000,0.000000,0.000000}%
\pgfsetfillcolor{currentfill}%
\pgfsetlinewidth{0.602250pt}%
\definecolor{currentstroke}{rgb}{0.000000,0.000000,0.000000}%
\pgfsetstrokecolor{currentstroke}%
\pgfsetdash{}{0pt}%
\pgfsys@defobject{currentmarker}{\pgfqpoint{0.000000in}{-0.027778in}}{\pgfqpoint{0.000000in}{0.000000in}}{%
\pgfpathmoveto{\pgfqpoint{0.000000in}{0.000000in}}%
\pgfpathlineto{\pgfqpoint{0.000000in}{-0.027778in}}%
\pgfusepath{stroke,fill}%
}%
\begin{pgfscope}%
\pgfsys@transformshift{4.338782in}{0.643904in}%
\pgfsys@useobject{currentmarker}{}%
\end{pgfscope}%
\end{pgfscope}%
\begin{pgfscope}%
\pgfpathrectangle{\pgfqpoint{0.764581in}{0.643904in}}{\pgfqpoint{6.200000in}{4.620000in}}%
\pgfusepath{clip}%
\pgfsetbuttcap%
\pgfsetroundjoin%
\pgfsetlinewidth{0.803000pt}%
\definecolor{currentstroke}{rgb}{0.690196,0.690196,0.690196}%
\pgfsetstrokecolor{currentstroke}%
\pgfsetstrokeopacity{0.200000}%
\pgfsetdash{{2.960000pt}{1.280000pt}}{0.000000pt}%
\pgfpathmoveto{\pgfqpoint{4.641098in}{0.643904in}}%
\pgfpathlineto{\pgfqpoint{4.641098in}{5.263904in}}%
\pgfusepath{stroke}%
\end{pgfscope}%
\begin{pgfscope}%
\pgfsetbuttcap%
\pgfsetroundjoin%
\definecolor{currentfill}{rgb}{0.000000,0.000000,0.000000}%
\pgfsetfillcolor{currentfill}%
\pgfsetlinewidth{0.602250pt}%
\definecolor{currentstroke}{rgb}{0.000000,0.000000,0.000000}%
\pgfsetstrokecolor{currentstroke}%
\pgfsetdash{}{0pt}%
\pgfsys@defobject{currentmarker}{\pgfqpoint{0.000000in}{-0.027778in}}{\pgfqpoint{0.000000in}{0.000000in}}{%
\pgfpathmoveto{\pgfqpoint{0.000000in}{0.000000in}}%
\pgfpathlineto{\pgfqpoint{0.000000in}{-0.027778in}}%
\pgfusepath{stroke,fill}%
}%
\begin{pgfscope}%
\pgfsys@transformshift{4.641098in}{0.643904in}%
\pgfsys@useobject{currentmarker}{}%
\end{pgfscope}%
\end{pgfscope}%
\begin{pgfscope}%
\pgfpathrectangle{\pgfqpoint{0.764581in}{0.643904in}}{\pgfqpoint{6.200000in}{4.620000in}}%
\pgfusepath{clip}%
\pgfsetbuttcap%
\pgfsetroundjoin%
\pgfsetlinewidth{0.803000pt}%
\definecolor{currentstroke}{rgb}{0.690196,0.690196,0.690196}%
\pgfsetstrokecolor{currentstroke}%
\pgfsetstrokeopacity{0.200000}%
\pgfsetdash{{2.960000pt}{1.280000pt}}{0.000000pt}%
\pgfpathmoveto{\pgfqpoint{4.875592in}{0.643904in}}%
\pgfpathlineto{\pgfqpoint{4.875592in}{5.263904in}}%
\pgfusepath{stroke}%
\end{pgfscope}%
\begin{pgfscope}%
\pgfsetbuttcap%
\pgfsetroundjoin%
\definecolor{currentfill}{rgb}{0.000000,0.000000,0.000000}%
\pgfsetfillcolor{currentfill}%
\pgfsetlinewidth{0.602250pt}%
\definecolor{currentstroke}{rgb}{0.000000,0.000000,0.000000}%
\pgfsetstrokecolor{currentstroke}%
\pgfsetdash{}{0pt}%
\pgfsys@defobject{currentmarker}{\pgfqpoint{0.000000in}{-0.027778in}}{\pgfqpoint{0.000000in}{0.000000in}}{%
\pgfpathmoveto{\pgfqpoint{0.000000in}{0.000000in}}%
\pgfpathlineto{\pgfqpoint{0.000000in}{-0.027778in}}%
\pgfusepath{stroke,fill}%
}%
\begin{pgfscope}%
\pgfsys@transformshift{4.875592in}{0.643904in}%
\pgfsys@useobject{currentmarker}{}%
\end{pgfscope}%
\end{pgfscope}%
\begin{pgfscope}%
\pgfpathrectangle{\pgfqpoint{0.764581in}{0.643904in}}{\pgfqpoint{6.200000in}{4.620000in}}%
\pgfusepath{clip}%
\pgfsetbuttcap%
\pgfsetroundjoin%
\pgfsetlinewidth{0.803000pt}%
\definecolor{currentstroke}{rgb}{0.690196,0.690196,0.690196}%
\pgfsetstrokecolor{currentstroke}%
\pgfsetstrokeopacity{0.200000}%
\pgfsetdash{{2.960000pt}{1.280000pt}}{0.000000pt}%
\pgfpathmoveto{\pgfqpoint{5.067187in}{0.643904in}}%
\pgfpathlineto{\pgfqpoint{5.067187in}{5.263904in}}%
\pgfusepath{stroke}%
\end{pgfscope}%
\begin{pgfscope}%
\pgfsetbuttcap%
\pgfsetroundjoin%
\definecolor{currentfill}{rgb}{0.000000,0.000000,0.000000}%
\pgfsetfillcolor{currentfill}%
\pgfsetlinewidth{0.602250pt}%
\definecolor{currentstroke}{rgb}{0.000000,0.000000,0.000000}%
\pgfsetstrokecolor{currentstroke}%
\pgfsetdash{}{0pt}%
\pgfsys@defobject{currentmarker}{\pgfqpoint{0.000000in}{-0.027778in}}{\pgfqpoint{0.000000in}{0.000000in}}{%
\pgfpathmoveto{\pgfqpoint{0.000000in}{0.000000in}}%
\pgfpathlineto{\pgfqpoint{0.000000in}{-0.027778in}}%
\pgfusepath{stroke,fill}%
}%
\begin{pgfscope}%
\pgfsys@transformshift{5.067187in}{0.643904in}%
\pgfsys@useobject{currentmarker}{}%
\end{pgfscope}%
\end{pgfscope}%
\begin{pgfscope}%
\pgfpathrectangle{\pgfqpoint{0.764581in}{0.643904in}}{\pgfqpoint{6.200000in}{4.620000in}}%
\pgfusepath{clip}%
\pgfsetbuttcap%
\pgfsetroundjoin%
\pgfsetlinewidth{0.803000pt}%
\definecolor{currentstroke}{rgb}{0.690196,0.690196,0.690196}%
\pgfsetstrokecolor{currentstroke}%
\pgfsetstrokeopacity{0.200000}%
\pgfsetdash{{2.960000pt}{1.280000pt}}{0.000000pt}%
\pgfpathmoveto{\pgfqpoint{5.229179in}{0.643904in}}%
\pgfpathlineto{\pgfqpoint{5.229179in}{5.263904in}}%
\pgfusepath{stroke}%
\end{pgfscope}%
\begin{pgfscope}%
\pgfsetbuttcap%
\pgfsetroundjoin%
\definecolor{currentfill}{rgb}{0.000000,0.000000,0.000000}%
\pgfsetfillcolor{currentfill}%
\pgfsetlinewidth{0.602250pt}%
\definecolor{currentstroke}{rgb}{0.000000,0.000000,0.000000}%
\pgfsetstrokecolor{currentstroke}%
\pgfsetdash{}{0pt}%
\pgfsys@defobject{currentmarker}{\pgfqpoint{0.000000in}{-0.027778in}}{\pgfqpoint{0.000000in}{0.000000in}}{%
\pgfpathmoveto{\pgfqpoint{0.000000in}{0.000000in}}%
\pgfpathlineto{\pgfqpoint{0.000000in}{-0.027778in}}%
\pgfusepath{stroke,fill}%
}%
\begin{pgfscope}%
\pgfsys@transformshift{5.229179in}{0.643904in}%
\pgfsys@useobject{currentmarker}{}%
\end{pgfscope}%
\end{pgfscope}%
\begin{pgfscope}%
\pgfpathrectangle{\pgfqpoint{0.764581in}{0.643904in}}{\pgfqpoint{6.200000in}{4.620000in}}%
\pgfusepath{clip}%
\pgfsetbuttcap%
\pgfsetroundjoin%
\pgfsetlinewidth{0.803000pt}%
\definecolor{currentstroke}{rgb}{0.690196,0.690196,0.690196}%
\pgfsetstrokecolor{currentstroke}%
\pgfsetstrokeopacity{0.200000}%
\pgfsetdash{{2.960000pt}{1.280000pt}}{0.000000pt}%
\pgfpathmoveto{\pgfqpoint{5.369502in}{0.643904in}}%
\pgfpathlineto{\pgfqpoint{5.369502in}{5.263904in}}%
\pgfusepath{stroke}%
\end{pgfscope}%
\begin{pgfscope}%
\pgfsetbuttcap%
\pgfsetroundjoin%
\definecolor{currentfill}{rgb}{0.000000,0.000000,0.000000}%
\pgfsetfillcolor{currentfill}%
\pgfsetlinewidth{0.602250pt}%
\definecolor{currentstroke}{rgb}{0.000000,0.000000,0.000000}%
\pgfsetstrokecolor{currentstroke}%
\pgfsetdash{}{0pt}%
\pgfsys@defobject{currentmarker}{\pgfqpoint{0.000000in}{-0.027778in}}{\pgfqpoint{0.000000in}{0.000000in}}{%
\pgfpathmoveto{\pgfqpoint{0.000000in}{0.000000in}}%
\pgfpathlineto{\pgfqpoint{0.000000in}{-0.027778in}}%
\pgfusepath{stroke,fill}%
}%
\begin{pgfscope}%
\pgfsys@transformshift{5.369502in}{0.643904in}%
\pgfsys@useobject{currentmarker}{}%
\end{pgfscope}%
\end{pgfscope}%
\begin{pgfscope}%
\pgfpathrectangle{\pgfqpoint{0.764581in}{0.643904in}}{\pgfqpoint{6.200000in}{4.620000in}}%
\pgfusepath{clip}%
\pgfsetbuttcap%
\pgfsetroundjoin%
\pgfsetlinewidth{0.803000pt}%
\definecolor{currentstroke}{rgb}{0.690196,0.690196,0.690196}%
\pgfsetstrokecolor{currentstroke}%
\pgfsetstrokeopacity{0.200000}%
\pgfsetdash{{2.960000pt}{1.280000pt}}{0.000000pt}%
\pgfpathmoveto{\pgfqpoint{5.493277in}{0.643904in}}%
\pgfpathlineto{\pgfqpoint{5.493277in}{5.263904in}}%
\pgfusepath{stroke}%
\end{pgfscope}%
\begin{pgfscope}%
\pgfsetbuttcap%
\pgfsetroundjoin%
\definecolor{currentfill}{rgb}{0.000000,0.000000,0.000000}%
\pgfsetfillcolor{currentfill}%
\pgfsetlinewidth{0.602250pt}%
\definecolor{currentstroke}{rgb}{0.000000,0.000000,0.000000}%
\pgfsetstrokecolor{currentstroke}%
\pgfsetdash{}{0pt}%
\pgfsys@defobject{currentmarker}{\pgfqpoint{0.000000in}{-0.027778in}}{\pgfqpoint{0.000000in}{0.000000in}}{%
\pgfpathmoveto{\pgfqpoint{0.000000in}{0.000000in}}%
\pgfpathlineto{\pgfqpoint{0.000000in}{-0.027778in}}%
\pgfusepath{stroke,fill}%
}%
\begin{pgfscope}%
\pgfsys@transformshift{5.493277in}{0.643904in}%
\pgfsys@useobject{currentmarker}{}%
\end{pgfscope}%
\end{pgfscope}%
\begin{pgfscope}%
\pgfpathrectangle{\pgfqpoint{0.764581in}{0.643904in}}{\pgfqpoint{6.200000in}{4.620000in}}%
\pgfusepath{clip}%
\pgfsetbuttcap%
\pgfsetroundjoin%
\pgfsetlinewidth{0.803000pt}%
\definecolor{currentstroke}{rgb}{0.690196,0.690196,0.690196}%
\pgfsetstrokecolor{currentstroke}%
\pgfsetstrokeopacity{0.200000}%
\pgfsetdash{{2.960000pt}{1.280000pt}}{0.000000pt}%
\pgfpathmoveto{\pgfqpoint{6.332401in}{0.643904in}}%
\pgfpathlineto{\pgfqpoint{6.332401in}{5.263904in}}%
\pgfusepath{stroke}%
\end{pgfscope}%
\begin{pgfscope}%
\pgfsetbuttcap%
\pgfsetroundjoin%
\definecolor{currentfill}{rgb}{0.000000,0.000000,0.000000}%
\pgfsetfillcolor{currentfill}%
\pgfsetlinewidth{0.602250pt}%
\definecolor{currentstroke}{rgb}{0.000000,0.000000,0.000000}%
\pgfsetstrokecolor{currentstroke}%
\pgfsetdash{}{0pt}%
\pgfsys@defobject{currentmarker}{\pgfqpoint{0.000000in}{-0.027778in}}{\pgfqpoint{0.000000in}{0.000000in}}{%
\pgfpathmoveto{\pgfqpoint{0.000000in}{0.000000in}}%
\pgfpathlineto{\pgfqpoint{0.000000in}{-0.027778in}}%
\pgfusepath{stroke,fill}%
}%
\begin{pgfscope}%
\pgfsys@transformshift{6.332401in}{0.643904in}%
\pgfsys@useobject{currentmarker}{}%
\end{pgfscope}%
\end{pgfscope}%
\begin{pgfscope}%
\pgfpathrectangle{\pgfqpoint{0.764581in}{0.643904in}}{\pgfqpoint{6.200000in}{4.620000in}}%
\pgfusepath{clip}%
\pgfsetbuttcap%
\pgfsetroundjoin%
\pgfsetlinewidth{0.803000pt}%
\definecolor{currentstroke}{rgb}{0.690196,0.690196,0.690196}%
\pgfsetstrokecolor{currentstroke}%
\pgfsetstrokeopacity{0.200000}%
\pgfsetdash{{2.960000pt}{1.280000pt}}{0.000000pt}%
\pgfpathmoveto{\pgfqpoint{6.758490in}{0.643904in}}%
\pgfpathlineto{\pgfqpoint{6.758490in}{5.263904in}}%
\pgfusepath{stroke}%
\end{pgfscope}%
\begin{pgfscope}%
\pgfsetbuttcap%
\pgfsetroundjoin%
\definecolor{currentfill}{rgb}{0.000000,0.000000,0.000000}%
\pgfsetfillcolor{currentfill}%
\pgfsetlinewidth{0.602250pt}%
\definecolor{currentstroke}{rgb}{0.000000,0.000000,0.000000}%
\pgfsetstrokecolor{currentstroke}%
\pgfsetdash{}{0pt}%
\pgfsys@defobject{currentmarker}{\pgfqpoint{0.000000in}{-0.027778in}}{\pgfqpoint{0.000000in}{0.000000in}}{%
\pgfpathmoveto{\pgfqpoint{0.000000in}{0.000000in}}%
\pgfpathlineto{\pgfqpoint{0.000000in}{-0.027778in}}%
\pgfusepath{stroke,fill}%
}%
\begin{pgfscope}%
\pgfsys@transformshift{6.758490in}{0.643904in}%
\pgfsys@useobject{currentmarker}{}%
\end{pgfscope}%
\end{pgfscope}%
\begin{pgfscope}%
\definecolor{textcolor}{rgb}{0.000000,0.000000,0.000000}%
\pgfsetstrokecolor{textcolor}%
\pgfsetfillcolor{textcolor}%
\pgftext[x=3.864581in,y=0.313349in,,top]{\color{textcolor}{\rmfamily\fontsize{18.000000}{21.600000}\selectfont\catcode`\^=\active\def^{\ifmmode\sp\else\^{}\fi}\catcode`\%=\active\def%{\%}Number of Modeled Days}}%
\end{pgfscope}%
\begin{pgfscope}%
\pgfpathrectangle{\pgfqpoint{0.764581in}{0.643904in}}{\pgfqpoint{6.200000in}{4.620000in}}%
\pgfusepath{clip}%
\pgfsetrectcap%
\pgfsetroundjoin%
\pgfsetlinewidth{0.803000pt}%
\definecolor{currentstroke}{rgb}{0.690196,0.690196,0.690196}%
\pgfsetstrokecolor{currentstroke}%
\pgfsetdash{}{0pt}%
\pgfpathmoveto{\pgfqpoint{0.764581in}{2.067685in}}%
\pgfpathlineto{\pgfqpoint{6.964581in}{2.067685in}}%
\pgfusepath{stroke}%
\end{pgfscope}%
\begin{pgfscope}%
\pgfsetbuttcap%
\pgfsetroundjoin%
\definecolor{currentfill}{rgb}{0.000000,0.000000,0.000000}%
\pgfsetfillcolor{currentfill}%
\pgfsetlinewidth{0.803000pt}%
\definecolor{currentstroke}{rgb}{0.000000,0.000000,0.000000}%
\pgfsetstrokecolor{currentstroke}%
\pgfsetdash{}{0pt}%
\pgfsys@defobject{currentmarker}{\pgfqpoint{-0.048611in}{0.000000in}}{\pgfqpoint{-0.000000in}{0.000000in}}{%
\pgfpathmoveto{\pgfqpoint{-0.000000in}{0.000000in}}%
\pgfpathlineto{\pgfqpoint{-0.048611in}{0.000000in}}%
\pgfusepath{stroke,fill}%
}%
\begin{pgfscope}%
\pgfsys@transformshift{0.764581in}{2.067685in}%
\pgfsys@useobject{currentmarker}{}%
\end{pgfscope}%
\end{pgfscope}%
\begin{pgfscope}%
\definecolor{textcolor}{rgb}{0.000000,0.000000,0.000000}%
\pgfsetstrokecolor{textcolor}%
\pgfsetfillcolor{textcolor}%
\pgftext[x=0.395138in, y=1.998240in, left, base]{\color{textcolor}{\rmfamily\fontsize{14.000000}{16.800000}\selectfont\catcode`\^=\active\def^{\ifmmode\sp\else\^{}\fi}\catcode`\%=\active\def%{\%}$\mathdefault{10^{1}}$}}%
\end{pgfscope}%
\begin{pgfscope}%
\pgfpathrectangle{\pgfqpoint{0.764581in}{0.643904in}}{\pgfqpoint{6.200000in}{4.620000in}}%
\pgfusepath{clip}%
\pgfsetrectcap%
\pgfsetroundjoin%
\pgfsetlinewidth{0.803000pt}%
\definecolor{currentstroke}{rgb}{0.690196,0.690196,0.690196}%
\pgfsetstrokecolor{currentstroke}%
\pgfsetdash{}{0pt}%
\pgfpathmoveto{\pgfqpoint{0.764581in}{3.655018in}}%
\pgfpathlineto{\pgfqpoint{6.964581in}{3.655018in}}%
\pgfusepath{stroke}%
\end{pgfscope}%
\begin{pgfscope}%
\pgfsetbuttcap%
\pgfsetroundjoin%
\definecolor{currentfill}{rgb}{0.000000,0.000000,0.000000}%
\pgfsetfillcolor{currentfill}%
\pgfsetlinewidth{0.803000pt}%
\definecolor{currentstroke}{rgb}{0.000000,0.000000,0.000000}%
\pgfsetstrokecolor{currentstroke}%
\pgfsetdash{}{0pt}%
\pgfsys@defobject{currentmarker}{\pgfqpoint{-0.048611in}{0.000000in}}{\pgfqpoint{-0.000000in}{0.000000in}}{%
\pgfpathmoveto{\pgfqpoint{-0.000000in}{0.000000in}}%
\pgfpathlineto{\pgfqpoint{-0.048611in}{0.000000in}}%
\pgfusepath{stroke,fill}%
}%
\begin{pgfscope}%
\pgfsys@transformshift{0.764581in}{3.655018in}%
\pgfsys@useobject{currentmarker}{}%
\end{pgfscope}%
\end{pgfscope}%
\begin{pgfscope}%
\definecolor{textcolor}{rgb}{0.000000,0.000000,0.000000}%
\pgfsetstrokecolor{textcolor}%
\pgfsetfillcolor{textcolor}%
\pgftext[x=0.395138in, y=3.585574in, left, base]{\color{textcolor}{\rmfamily\fontsize{14.000000}{16.800000}\selectfont\catcode`\^=\active\def^{\ifmmode\sp\else\^{}\fi}\catcode`\%=\active\def%{\%}$\mathdefault{10^{2}}$}}%
\end{pgfscope}%
\begin{pgfscope}%
\pgfpathrectangle{\pgfqpoint{0.764581in}{0.643904in}}{\pgfqpoint{6.200000in}{4.620000in}}%
\pgfusepath{clip}%
\pgfsetrectcap%
\pgfsetroundjoin%
\pgfsetlinewidth{0.803000pt}%
\definecolor{currentstroke}{rgb}{0.690196,0.690196,0.690196}%
\pgfsetstrokecolor{currentstroke}%
\pgfsetdash{}{0pt}%
\pgfpathmoveto{\pgfqpoint{0.764581in}{5.242352in}}%
\pgfpathlineto{\pgfqpoint{6.964581in}{5.242352in}}%
\pgfusepath{stroke}%
\end{pgfscope}%
\begin{pgfscope}%
\pgfsetbuttcap%
\pgfsetroundjoin%
\definecolor{currentfill}{rgb}{0.000000,0.000000,0.000000}%
\pgfsetfillcolor{currentfill}%
\pgfsetlinewidth{0.803000pt}%
\definecolor{currentstroke}{rgb}{0.000000,0.000000,0.000000}%
\pgfsetstrokecolor{currentstroke}%
\pgfsetdash{}{0pt}%
\pgfsys@defobject{currentmarker}{\pgfqpoint{-0.048611in}{0.000000in}}{\pgfqpoint{-0.000000in}{0.000000in}}{%
\pgfpathmoveto{\pgfqpoint{-0.000000in}{0.000000in}}%
\pgfpathlineto{\pgfqpoint{-0.048611in}{0.000000in}}%
\pgfusepath{stroke,fill}%
}%
\begin{pgfscope}%
\pgfsys@transformshift{0.764581in}{5.242352in}%
\pgfsys@useobject{currentmarker}{}%
\end{pgfscope}%
\end{pgfscope}%
\begin{pgfscope}%
\definecolor{textcolor}{rgb}{0.000000,0.000000,0.000000}%
\pgfsetstrokecolor{textcolor}%
\pgfsetfillcolor{textcolor}%
\pgftext[x=0.395138in, y=5.172908in, left, base]{\color{textcolor}{\rmfamily\fontsize{14.000000}{16.800000}\selectfont\catcode`\^=\active\def^{\ifmmode\sp\else\^{}\fi}\catcode`\%=\active\def%{\%}$\mathdefault{10^{3}}$}}%
\end{pgfscope}%
\begin{pgfscope}%
\pgfpathrectangle{\pgfqpoint{0.764581in}{0.643904in}}{\pgfqpoint{6.200000in}{4.620000in}}%
\pgfusepath{clip}%
\pgfsetbuttcap%
\pgfsetroundjoin%
\pgfsetlinewidth{0.803000pt}%
\definecolor{currentstroke}{rgb}{0.690196,0.690196,0.690196}%
\pgfsetstrokecolor{currentstroke}%
\pgfsetstrokeopacity{0.200000}%
\pgfsetdash{{2.960000pt}{1.280000pt}}{0.000000pt}%
\pgfpathmoveto{\pgfqpoint{0.764581in}{0.958186in}}%
\pgfpathlineto{\pgfqpoint{6.964581in}{0.958186in}}%
\pgfusepath{stroke}%
\end{pgfscope}%
\begin{pgfscope}%
\pgfsetbuttcap%
\pgfsetroundjoin%
\definecolor{currentfill}{rgb}{0.000000,0.000000,0.000000}%
\pgfsetfillcolor{currentfill}%
\pgfsetlinewidth{0.602250pt}%
\definecolor{currentstroke}{rgb}{0.000000,0.000000,0.000000}%
\pgfsetstrokecolor{currentstroke}%
\pgfsetdash{}{0pt}%
\pgfsys@defobject{currentmarker}{\pgfqpoint{-0.027778in}{0.000000in}}{\pgfqpoint{-0.000000in}{0.000000in}}{%
\pgfpathmoveto{\pgfqpoint{-0.000000in}{0.000000in}}%
\pgfpathlineto{\pgfqpoint{-0.027778in}{0.000000in}}%
\pgfusepath{stroke,fill}%
}%
\begin{pgfscope}%
\pgfsys@transformshift{0.764581in}{0.958186in}%
\pgfsys@useobject{currentmarker}{}%
\end{pgfscope}%
\end{pgfscope}%
\begin{pgfscope}%
\pgfpathrectangle{\pgfqpoint{0.764581in}{0.643904in}}{\pgfqpoint{6.200000in}{4.620000in}}%
\pgfusepath{clip}%
\pgfsetbuttcap%
\pgfsetroundjoin%
\pgfsetlinewidth{0.803000pt}%
\definecolor{currentstroke}{rgb}{0.690196,0.690196,0.690196}%
\pgfsetstrokecolor{currentstroke}%
\pgfsetstrokeopacity{0.200000}%
\pgfsetdash{{2.960000pt}{1.280000pt}}{0.000000pt}%
\pgfpathmoveto{\pgfqpoint{0.764581in}{1.237701in}}%
\pgfpathlineto{\pgfqpoint{6.964581in}{1.237701in}}%
\pgfusepath{stroke}%
\end{pgfscope}%
\begin{pgfscope}%
\pgfsetbuttcap%
\pgfsetroundjoin%
\definecolor{currentfill}{rgb}{0.000000,0.000000,0.000000}%
\pgfsetfillcolor{currentfill}%
\pgfsetlinewidth{0.602250pt}%
\definecolor{currentstroke}{rgb}{0.000000,0.000000,0.000000}%
\pgfsetstrokecolor{currentstroke}%
\pgfsetdash{}{0pt}%
\pgfsys@defobject{currentmarker}{\pgfqpoint{-0.027778in}{0.000000in}}{\pgfqpoint{-0.000000in}{0.000000in}}{%
\pgfpathmoveto{\pgfqpoint{-0.000000in}{0.000000in}}%
\pgfpathlineto{\pgfqpoint{-0.027778in}{0.000000in}}%
\pgfusepath{stroke,fill}%
}%
\begin{pgfscope}%
\pgfsys@transformshift{0.764581in}{1.237701in}%
\pgfsys@useobject{currentmarker}{}%
\end{pgfscope}%
\end{pgfscope}%
\begin{pgfscope}%
\pgfpathrectangle{\pgfqpoint{0.764581in}{0.643904in}}{\pgfqpoint{6.200000in}{4.620000in}}%
\pgfusepath{clip}%
\pgfsetbuttcap%
\pgfsetroundjoin%
\pgfsetlinewidth{0.803000pt}%
\definecolor{currentstroke}{rgb}{0.690196,0.690196,0.690196}%
\pgfsetstrokecolor{currentstroke}%
\pgfsetstrokeopacity{0.200000}%
\pgfsetdash{{2.960000pt}{1.280000pt}}{0.000000pt}%
\pgfpathmoveto{\pgfqpoint{0.764581in}{1.436021in}}%
\pgfpathlineto{\pgfqpoint{6.964581in}{1.436021in}}%
\pgfusepath{stroke}%
\end{pgfscope}%
\begin{pgfscope}%
\pgfsetbuttcap%
\pgfsetroundjoin%
\definecolor{currentfill}{rgb}{0.000000,0.000000,0.000000}%
\pgfsetfillcolor{currentfill}%
\pgfsetlinewidth{0.602250pt}%
\definecolor{currentstroke}{rgb}{0.000000,0.000000,0.000000}%
\pgfsetstrokecolor{currentstroke}%
\pgfsetdash{}{0pt}%
\pgfsys@defobject{currentmarker}{\pgfqpoint{-0.027778in}{0.000000in}}{\pgfqpoint{-0.000000in}{0.000000in}}{%
\pgfpathmoveto{\pgfqpoint{-0.000000in}{0.000000in}}%
\pgfpathlineto{\pgfqpoint{-0.027778in}{0.000000in}}%
\pgfusepath{stroke,fill}%
}%
\begin{pgfscope}%
\pgfsys@transformshift{0.764581in}{1.436021in}%
\pgfsys@useobject{currentmarker}{}%
\end{pgfscope}%
\end{pgfscope}%
\begin{pgfscope}%
\pgfpathrectangle{\pgfqpoint{0.764581in}{0.643904in}}{\pgfqpoint{6.200000in}{4.620000in}}%
\pgfusepath{clip}%
\pgfsetbuttcap%
\pgfsetroundjoin%
\pgfsetlinewidth{0.803000pt}%
\definecolor{currentstroke}{rgb}{0.690196,0.690196,0.690196}%
\pgfsetstrokecolor{currentstroke}%
\pgfsetstrokeopacity{0.200000}%
\pgfsetdash{{2.960000pt}{1.280000pt}}{0.000000pt}%
\pgfpathmoveto{\pgfqpoint{0.764581in}{1.589849in}}%
\pgfpathlineto{\pgfqpoint{6.964581in}{1.589849in}}%
\pgfusepath{stroke}%
\end{pgfscope}%
\begin{pgfscope}%
\pgfsetbuttcap%
\pgfsetroundjoin%
\definecolor{currentfill}{rgb}{0.000000,0.000000,0.000000}%
\pgfsetfillcolor{currentfill}%
\pgfsetlinewidth{0.602250pt}%
\definecolor{currentstroke}{rgb}{0.000000,0.000000,0.000000}%
\pgfsetstrokecolor{currentstroke}%
\pgfsetdash{}{0pt}%
\pgfsys@defobject{currentmarker}{\pgfqpoint{-0.027778in}{0.000000in}}{\pgfqpoint{-0.000000in}{0.000000in}}{%
\pgfpathmoveto{\pgfqpoint{-0.000000in}{0.000000in}}%
\pgfpathlineto{\pgfqpoint{-0.027778in}{0.000000in}}%
\pgfusepath{stroke,fill}%
}%
\begin{pgfscope}%
\pgfsys@transformshift{0.764581in}{1.589849in}%
\pgfsys@useobject{currentmarker}{}%
\end{pgfscope}%
\end{pgfscope}%
\begin{pgfscope}%
\pgfpathrectangle{\pgfqpoint{0.764581in}{0.643904in}}{\pgfqpoint{6.200000in}{4.620000in}}%
\pgfusepath{clip}%
\pgfsetbuttcap%
\pgfsetroundjoin%
\pgfsetlinewidth{0.803000pt}%
\definecolor{currentstroke}{rgb}{0.690196,0.690196,0.690196}%
\pgfsetstrokecolor{currentstroke}%
\pgfsetstrokeopacity{0.200000}%
\pgfsetdash{{2.960000pt}{1.280000pt}}{0.000000pt}%
\pgfpathmoveto{\pgfqpoint{0.764581in}{1.715536in}}%
\pgfpathlineto{\pgfqpoint{6.964581in}{1.715536in}}%
\pgfusepath{stroke}%
\end{pgfscope}%
\begin{pgfscope}%
\pgfsetbuttcap%
\pgfsetroundjoin%
\definecolor{currentfill}{rgb}{0.000000,0.000000,0.000000}%
\pgfsetfillcolor{currentfill}%
\pgfsetlinewidth{0.602250pt}%
\definecolor{currentstroke}{rgb}{0.000000,0.000000,0.000000}%
\pgfsetstrokecolor{currentstroke}%
\pgfsetdash{}{0pt}%
\pgfsys@defobject{currentmarker}{\pgfqpoint{-0.027778in}{0.000000in}}{\pgfqpoint{-0.000000in}{0.000000in}}{%
\pgfpathmoveto{\pgfqpoint{-0.000000in}{0.000000in}}%
\pgfpathlineto{\pgfqpoint{-0.027778in}{0.000000in}}%
\pgfusepath{stroke,fill}%
}%
\begin{pgfscope}%
\pgfsys@transformshift{0.764581in}{1.715536in}%
\pgfsys@useobject{currentmarker}{}%
\end{pgfscope}%
\end{pgfscope}%
\begin{pgfscope}%
\pgfpathrectangle{\pgfqpoint{0.764581in}{0.643904in}}{\pgfqpoint{6.200000in}{4.620000in}}%
\pgfusepath{clip}%
\pgfsetbuttcap%
\pgfsetroundjoin%
\pgfsetlinewidth{0.803000pt}%
\definecolor{currentstroke}{rgb}{0.690196,0.690196,0.690196}%
\pgfsetstrokecolor{currentstroke}%
\pgfsetstrokeopacity{0.200000}%
\pgfsetdash{{2.960000pt}{1.280000pt}}{0.000000pt}%
\pgfpathmoveto{\pgfqpoint{0.764581in}{1.821803in}}%
\pgfpathlineto{\pgfqpoint{6.964581in}{1.821803in}}%
\pgfusepath{stroke}%
\end{pgfscope}%
\begin{pgfscope}%
\pgfsetbuttcap%
\pgfsetroundjoin%
\definecolor{currentfill}{rgb}{0.000000,0.000000,0.000000}%
\pgfsetfillcolor{currentfill}%
\pgfsetlinewidth{0.602250pt}%
\definecolor{currentstroke}{rgb}{0.000000,0.000000,0.000000}%
\pgfsetstrokecolor{currentstroke}%
\pgfsetdash{}{0pt}%
\pgfsys@defobject{currentmarker}{\pgfqpoint{-0.027778in}{0.000000in}}{\pgfqpoint{-0.000000in}{0.000000in}}{%
\pgfpathmoveto{\pgfqpoint{-0.000000in}{0.000000in}}%
\pgfpathlineto{\pgfqpoint{-0.027778in}{0.000000in}}%
\pgfusepath{stroke,fill}%
}%
\begin{pgfscope}%
\pgfsys@transformshift{0.764581in}{1.821803in}%
\pgfsys@useobject{currentmarker}{}%
\end{pgfscope}%
\end{pgfscope}%
\begin{pgfscope}%
\pgfpathrectangle{\pgfqpoint{0.764581in}{0.643904in}}{\pgfqpoint{6.200000in}{4.620000in}}%
\pgfusepath{clip}%
\pgfsetbuttcap%
\pgfsetroundjoin%
\pgfsetlinewidth{0.803000pt}%
\definecolor{currentstroke}{rgb}{0.690196,0.690196,0.690196}%
\pgfsetstrokecolor{currentstroke}%
\pgfsetstrokeopacity{0.200000}%
\pgfsetdash{{2.960000pt}{1.280000pt}}{0.000000pt}%
\pgfpathmoveto{\pgfqpoint{0.764581in}{1.913856in}}%
\pgfpathlineto{\pgfqpoint{6.964581in}{1.913856in}}%
\pgfusepath{stroke}%
\end{pgfscope}%
\begin{pgfscope}%
\pgfsetbuttcap%
\pgfsetroundjoin%
\definecolor{currentfill}{rgb}{0.000000,0.000000,0.000000}%
\pgfsetfillcolor{currentfill}%
\pgfsetlinewidth{0.602250pt}%
\definecolor{currentstroke}{rgb}{0.000000,0.000000,0.000000}%
\pgfsetstrokecolor{currentstroke}%
\pgfsetdash{}{0pt}%
\pgfsys@defobject{currentmarker}{\pgfqpoint{-0.027778in}{0.000000in}}{\pgfqpoint{-0.000000in}{0.000000in}}{%
\pgfpathmoveto{\pgfqpoint{-0.000000in}{0.000000in}}%
\pgfpathlineto{\pgfqpoint{-0.027778in}{0.000000in}}%
\pgfusepath{stroke,fill}%
}%
\begin{pgfscope}%
\pgfsys@transformshift{0.764581in}{1.913856in}%
\pgfsys@useobject{currentmarker}{}%
\end{pgfscope}%
\end{pgfscope}%
\begin{pgfscope}%
\pgfpathrectangle{\pgfqpoint{0.764581in}{0.643904in}}{\pgfqpoint{6.200000in}{4.620000in}}%
\pgfusepath{clip}%
\pgfsetbuttcap%
\pgfsetroundjoin%
\pgfsetlinewidth{0.803000pt}%
\definecolor{currentstroke}{rgb}{0.690196,0.690196,0.690196}%
\pgfsetstrokecolor{currentstroke}%
\pgfsetstrokeopacity{0.200000}%
\pgfsetdash{{2.960000pt}{1.280000pt}}{0.000000pt}%
\pgfpathmoveto{\pgfqpoint{0.764581in}{1.995052in}}%
\pgfpathlineto{\pgfqpoint{6.964581in}{1.995052in}}%
\pgfusepath{stroke}%
\end{pgfscope}%
\begin{pgfscope}%
\pgfsetbuttcap%
\pgfsetroundjoin%
\definecolor{currentfill}{rgb}{0.000000,0.000000,0.000000}%
\pgfsetfillcolor{currentfill}%
\pgfsetlinewidth{0.602250pt}%
\definecolor{currentstroke}{rgb}{0.000000,0.000000,0.000000}%
\pgfsetstrokecolor{currentstroke}%
\pgfsetdash{}{0pt}%
\pgfsys@defobject{currentmarker}{\pgfqpoint{-0.027778in}{0.000000in}}{\pgfqpoint{-0.000000in}{0.000000in}}{%
\pgfpathmoveto{\pgfqpoint{-0.000000in}{0.000000in}}%
\pgfpathlineto{\pgfqpoint{-0.027778in}{0.000000in}}%
\pgfusepath{stroke,fill}%
}%
\begin{pgfscope}%
\pgfsys@transformshift{0.764581in}{1.995052in}%
\pgfsys@useobject{currentmarker}{}%
\end{pgfscope}%
\end{pgfscope}%
\begin{pgfscope}%
\pgfpathrectangle{\pgfqpoint{0.764581in}{0.643904in}}{\pgfqpoint{6.200000in}{4.620000in}}%
\pgfusepath{clip}%
\pgfsetbuttcap%
\pgfsetroundjoin%
\pgfsetlinewidth{0.803000pt}%
\definecolor{currentstroke}{rgb}{0.690196,0.690196,0.690196}%
\pgfsetstrokecolor{currentstroke}%
\pgfsetstrokeopacity{0.200000}%
\pgfsetdash{{2.960000pt}{1.280000pt}}{0.000000pt}%
\pgfpathmoveto{\pgfqpoint{0.764581in}{2.545520in}}%
\pgfpathlineto{\pgfqpoint{6.964581in}{2.545520in}}%
\pgfusepath{stroke}%
\end{pgfscope}%
\begin{pgfscope}%
\pgfsetbuttcap%
\pgfsetroundjoin%
\definecolor{currentfill}{rgb}{0.000000,0.000000,0.000000}%
\pgfsetfillcolor{currentfill}%
\pgfsetlinewidth{0.602250pt}%
\definecolor{currentstroke}{rgb}{0.000000,0.000000,0.000000}%
\pgfsetstrokecolor{currentstroke}%
\pgfsetdash{}{0pt}%
\pgfsys@defobject{currentmarker}{\pgfqpoint{-0.027778in}{0.000000in}}{\pgfqpoint{-0.000000in}{0.000000in}}{%
\pgfpathmoveto{\pgfqpoint{-0.000000in}{0.000000in}}%
\pgfpathlineto{\pgfqpoint{-0.027778in}{0.000000in}}%
\pgfusepath{stroke,fill}%
}%
\begin{pgfscope}%
\pgfsys@transformshift{0.764581in}{2.545520in}%
\pgfsys@useobject{currentmarker}{}%
\end{pgfscope}%
\end{pgfscope}%
\begin{pgfscope}%
\pgfpathrectangle{\pgfqpoint{0.764581in}{0.643904in}}{\pgfqpoint{6.200000in}{4.620000in}}%
\pgfusepath{clip}%
\pgfsetbuttcap%
\pgfsetroundjoin%
\pgfsetlinewidth{0.803000pt}%
\definecolor{currentstroke}{rgb}{0.690196,0.690196,0.690196}%
\pgfsetstrokecolor{currentstroke}%
\pgfsetstrokeopacity{0.200000}%
\pgfsetdash{{2.960000pt}{1.280000pt}}{0.000000pt}%
\pgfpathmoveto{\pgfqpoint{0.764581in}{2.825035in}}%
\pgfpathlineto{\pgfqpoint{6.964581in}{2.825035in}}%
\pgfusepath{stroke}%
\end{pgfscope}%
\begin{pgfscope}%
\pgfsetbuttcap%
\pgfsetroundjoin%
\definecolor{currentfill}{rgb}{0.000000,0.000000,0.000000}%
\pgfsetfillcolor{currentfill}%
\pgfsetlinewidth{0.602250pt}%
\definecolor{currentstroke}{rgb}{0.000000,0.000000,0.000000}%
\pgfsetstrokecolor{currentstroke}%
\pgfsetdash{}{0pt}%
\pgfsys@defobject{currentmarker}{\pgfqpoint{-0.027778in}{0.000000in}}{\pgfqpoint{-0.000000in}{0.000000in}}{%
\pgfpathmoveto{\pgfqpoint{-0.000000in}{0.000000in}}%
\pgfpathlineto{\pgfqpoint{-0.027778in}{0.000000in}}%
\pgfusepath{stroke,fill}%
}%
\begin{pgfscope}%
\pgfsys@transformshift{0.764581in}{2.825035in}%
\pgfsys@useobject{currentmarker}{}%
\end{pgfscope}%
\end{pgfscope}%
\begin{pgfscope}%
\pgfpathrectangle{\pgfqpoint{0.764581in}{0.643904in}}{\pgfqpoint{6.200000in}{4.620000in}}%
\pgfusepath{clip}%
\pgfsetbuttcap%
\pgfsetroundjoin%
\pgfsetlinewidth{0.803000pt}%
\definecolor{currentstroke}{rgb}{0.690196,0.690196,0.690196}%
\pgfsetstrokecolor{currentstroke}%
\pgfsetstrokeopacity{0.200000}%
\pgfsetdash{{2.960000pt}{1.280000pt}}{0.000000pt}%
\pgfpathmoveto{\pgfqpoint{0.764581in}{3.023355in}}%
\pgfpathlineto{\pgfqpoint{6.964581in}{3.023355in}}%
\pgfusepath{stroke}%
\end{pgfscope}%
\begin{pgfscope}%
\pgfsetbuttcap%
\pgfsetroundjoin%
\definecolor{currentfill}{rgb}{0.000000,0.000000,0.000000}%
\pgfsetfillcolor{currentfill}%
\pgfsetlinewidth{0.602250pt}%
\definecolor{currentstroke}{rgb}{0.000000,0.000000,0.000000}%
\pgfsetstrokecolor{currentstroke}%
\pgfsetdash{}{0pt}%
\pgfsys@defobject{currentmarker}{\pgfqpoint{-0.027778in}{0.000000in}}{\pgfqpoint{-0.000000in}{0.000000in}}{%
\pgfpathmoveto{\pgfqpoint{-0.000000in}{0.000000in}}%
\pgfpathlineto{\pgfqpoint{-0.027778in}{0.000000in}}%
\pgfusepath{stroke,fill}%
}%
\begin{pgfscope}%
\pgfsys@transformshift{0.764581in}{3.023355in}%
\pgfsys@useobject{currentmarker}{}%
\end{pgfscope}%
\end{pgfscope}%
\begin{pgfscope}%
\pgfpathrectangle{\pgfqpoint{0.764581in}{0.643904in}}{\pgfqpoint{6.200000in}{4.620000in}}%
\pgfusepath{clip}%
\pgfsetbuttcap%
\pgfsetroundjoin%
\pgfsetlinewidth{0.803000pt}%
\definecolor{currentstroke}{rgb}{0.690196,0.690196,0.690196}%
\pgfsetstrokecolor{currentstroke}%
\pgfsetstrokeopacity{0.200000}%
\pgfsetdash{{2.960000pt}{1.280000pt}}{0.000000pt}%
\pgfpathmoveto{\pgfqpoint{0.764581in}{3.177183in}}%
\pgfpathlineto{\pgfqpoint{6.964581in}{3.177183in}}%
\pgfusepath{stroke}%
\end{pgfscope}%
\begin{pgfscope}%
\pgfsetbuttcap%
\pgfsetroundjoin%
\definecolor{currentfill}{rgb}{0.000000,0.000000,0.000000}%
\pgfsetfillcolor{currentfill}%
\pgfsetlinewidth{0.602250pt}%
\definecolor{currentstroke}{rgb}{0.000000,0.000000,0.000000}%
\pgfsetstrokecolor{currentstroke}%
\pgfsetdash{}{0pt}%
\pgfsys@defobject{currentmarker}{\pgfqpoint{-0.027778in}{0.000000in}}{\pgfqpoint{-0.000000in}{0.000000in}}{%
\pgfpathmoveto{\pgfqpoint{-0.000000in}{0.000000in}}%
\pgfpathlineto{\pgfqpoint{-0.027778in}{0.000000in}}%
\pgfusepath{stroke,fill}%
}%
\begin{pgfscope}%
\pgfsys@transformshift{0.764581in}{3.177183in}%
\pgfsys@useobject{currentmarker}{}%
\end{pgfscope}%
\end{pgfscope}%
\begin{pgfscope}%
\pgfpathrectangle{\pgfqpoint{0.764581in}{0.643904in}}{\pgfqpoint{6.200000in}{4.620000in}}%
\pgfusepath{clip}%
\pgfsetbuttcap%
\pgfsetroundjoin%
\pgfsetlinewidth{0.803000pt}%
\definecolor{currentstroke}{rgb}{0.690196,0.690196,0.690196}%
\pgfsetstrokecolor{currentstroke}%
\pgfsetstrokeopacity{0.200000}%
\pgfsetdash{{2.960000pt}{1.280000pt}}{0.000000pt}%
\pgfpathmoveto{\pgfqpoint{0.764581in}{3.302870in}}%
\pgfpathlineto{\pgfqpoint{6.964581in}{3.302870in}}%
\pgfusepath{stroke}%
\end{pgfscope}%
\begin{pgfscope}%
\pgfsetbuttcap%
\pgfsetroundjoin%
\definecolor{currentfill}{rgb}{0.000000,0.000000,0.000000}%
\pgfsetfillcolor{currentfill}%
\pgfsetlinewidth{0.602250pt}%
\definecolor{currentstroke}{rgb}{0.000000,0.000000,0.000000}%
\pgfsetstrokecolor{currentstroke}%
\pgfsetdash{}{0pt}%
\pgfsys@defobject{currentmarker}{\pgfqpoint{-0.027778in}{0.000000in}}{\pgfqpoint{-0.000000in}{0.000000in}}{%
\pgfpathmoveto{\pgfqpoint{-0.000000in}{0.000000in}}%
\pgfpathlineto{\pgfqpoint{-0.027778in}{0.000000in}}%
\pgfusepath{stroke,fill}%
}%
\begin{pgfscope}%
\pgfsys@transformshift{0.764581in}{3.302870in}%
\pgfsys@useobject{currentmarker}{}%
\end{pgfscope}%
\end{pgfscope}%
\begin{pgfscope}%
\pgfpathrectangle{\pgfqpoint{0.764581in}{0.643904in}}{\pgfqpoint{6.200000in}{4.620000in}}%
\pgfusepath{clip}%
\pgfsetbuttcap%
\pgfsetroundjoin%
\pgfsetlinewidth{0.803000pt}%
\definecolor{currentstroke}{rgb}{0.690196,0.690196,0.690196}%
\pgfsetstrokecolor{currentstroke}%
\pgfsetstrokeopacity{0.200000}%
\pgfsetdash{{2.960000pt}{1.280000pt}}{0.000000pt}%
\pgfpathmoveto{\pgfqpoint{0.764581in}{3.409137in}}%
\pgfpathlineto{\pgfqpoint{6.964581in}{3.409137in}}%
\pgfusepath{stroke}%
\end{pgfscope}%
\begin{pgfscope}%
\pgfsetbuttcap%
\pgfsetroundjoin%
\definecolor{currentfill}{rgb}{0.000000,0.000000,0.000000}%
\pgfsetfillcolor{currentfill}%
\pgfsetlinewidth{0.602250pt}%
\definecolor{currentstroke}{rgb}{0.000000,0.000000,0.000000}%
\pgfsetstrokecolor{currentstroke}%
\pgfsetdash{}{0pt}%
\pgfsys@defobject{currentmarker}{\pgfqpoint{-0.027778in}{0.000000in}}{\pgfqpoint{-0.000000in}{0.000000in}}{%
\pgfpathmoveto{\pgfqpoint{-0.000000in}{0.000000in}}%
\pgfpathlineto{\pgfqpoint{-0.027778in}{0.000000in}}%
\pgfusepath{stroke,fill}%
}%
\begin{pgfscope}%
\pgfsys@transformshift{0.764581in}{3.409137in}%
\pgfsys@useobject{currentmarker}{}%
\end{pgfscope}%
\end{pgfscope}%
\begin{pgfscope}%
\pgfpathrectangle{\pgfqpoint{0.764581in}{0.643904in}}{\pgfqpoint{6.200000in}{4.620000in}}%
\pgfusepath{clip}%
\pgfsetbuttcap%
\pgfsetroundjoin%
\pgfsetlinewidth{0.803000pt}%
\definecolor{currentstroke}{rgb}{0.690196,0.690196,0.690196}%
\pgfsetstrokecolor{currentstroke}%
\pgfsetstrokeopacity{0.200000}%
\pgfsetdash{{2.960000pt}{1.280000pt}}{0.000000pt}%
\pgfpathmoveto{\pgfqpoint{0.764581in}{3.501190in}}%
\pgfpathlineto{\pgfqpoint{6.964581in}{3.501190in}}%
\pgfusepath{stroke}%
\end{pgfscope}%
\begin{pgfscope}%
\pgfsetbuttcap%
\pgfsetroundjoin%
\definecolor{currentfill}{rgb}{0.000000,0.000000,0.000000}%
\pgfsetfillcolor{currentfill}%
\pgfsetlinewidth{0.602250pt}%
\definecolor{currentstroke}{rgb}{0.000000,0.000000,0.000000}%
\pgfsetstrokecolor{currentstroke}%
\pgfsetdash{}{0pt}%
\pgfsys@defobject{currentmarker}{\pgfqpoint{-0.027778in}{0.000000in}}{\pgfqpoint{-0.000000in}{0.000000in}}{%
\pgfpathmoveto{\pgfqpoint{-0.000000in}{0.000000in}}%
\pgfpathlineto{\pgfqpoint{-0.027778in}{0.000000in}}%
\pgfusepath{stroke,fill}%
}%
\begin{pgfscope}%
\pgfsys@transformshift{0.764581in}{3.501190in}%
\pgfsys@useobject{currentmarker}{}%
\end{pgfscope}%
\end{pgfscope}%
\begin{pgfscope}%
\pgfpathrectangle{\pgfqpoint{0.764581in}{0.643904in}}{\pgfqpoint{6.200000in}{4.620000in}}%
\pgfusepath{clip}%
\pgfsetbuttcap%
\pgfsetroundjoin%
\pgfsetlinewidth{0.803000pt}%
\definecolor{currentstroke}{rgb}{0.690196,0.690196,0.690196}%
\pgfsetstrokecolor{currentstroke}%
\pgfsetstrokeopacity{0.200000}%
\pgfsetdash{{2.960000pt}{1.280000pt}}{0.000000pt}%
\pgfpathmoveto{\pgfqpoint{0.764581in}{3.582386in}}%
\pgfpathlineto{\pgfqpoint{6.964581in}{3.582386in}}%
\pgfusepath{stroke}%
\end{pgfscope}%
\begin{pgfscope}%
\pgfsetbuttcap%
\pgfsetroundjoin%
\definecolor{currentfill}{rgb}{0.000000,0.000000,0.000000}%
\pgfsetfillcolor{currentfill}%
\pgfsetlinewidth{0.602250pt}%
\definecolor{currentstroke}{rgb}{0.000000,0.000000,0.000000}%
\pgfsetstrokecolor{currentstroke}%
\pgfsetdash{}{0pt}%
\pgfsys@defobject{currentmarker}{\pgfqpoint{-0.027778in}{0.000000in}}{\pgfqpoint{-0.000000in}{0.000000in}}{%
\pgfpathmoveto{\pgfqpoint{-0.000000in}{0.000000in}}%
\pgfpathlineto{\pgfqpoint{-0.027778in}{0.000000in}}%
\pgfusepath{stroke,fill}%
}%
\begin{pgfscope}%
\pgfsys@transformshift{0.764581in}{3.582386in}%
\pgfsys@useobject{currentmarker}{}%
\end{pgfscope}%
\end{pgfscope}%
\begin{pgfscope}%
\pgfpathrectangle{\pgfqpoint{0.764581in}{0.643904in}}{\pgfqpoint{6.200000in}{4.620000in}}%
\pgfusepath{clip}%
\pgfsetbuttcap%
\pgfsetroundjoin%
\pgfsetlinewidth{0.803000pt}%
\definecolor{currentstroke}{rgb}{0.690196,0.690196,0.690196}%
\pgfsetstrokecolor{currentstroke}%
\pgfsetstrokeopacity{0.200000}%
\pgfsetdash{{2.960000pt}{1.280000pt}}{0.000000pt}%
\pgfpathmoveto{\pgfqpoint{0.764581in}{4.132854in}}%
\pgfpathlineto{\pgfqpoint{6.964581in}{4.132854in}}%
\pgfusepath{stroke}%
\end{pgfscope}%
\begin{pgfscope}%
\pgfsetbuttcap%
\pgfsetroundjoin%
\definecolor{currentfill}{rgb}{0.000000,0.000000,0.000000}%
\pgfsetfillcolor{currentfill}%
\pgfsetlinewidth{0.602250pt}%
\definecolor{currentstroke}{rgb}{0.000000,0.000000,0.000000}%
\pgfsetstrokecolor{currentstroke}%
\pgfsetdash{}{0pt}%
\pgfsys@defobject{currentmarker}{\pgfqpoint{-0.027778in}{0.000000in}}{\pgfqpoint{-0.000000in}{0.000000in}}{%
\pgfpathmoveto{\pgfqpoint{-0.000000in}{0.000000in}}%
\pgfpathlineto{\pgfqpoint{-0.027778in}{0.000000in}}%
\pgfusepath{stroke,fill}%
}%
\begin{pgfscope}%
\pgfsys@transformshift{0.764581in}{4.132854in}%
\pgfsys@useobject{currentmarker}{}%
\end{pgfscope}%
\end{pgfscope}%
\begin{pgfscope}%
\pgfpathrectangle{\pgfqpoint{0.764581in}{0.643904in}}{\pgfqpoint{6.200000in}{4.620000in}}%
\pgfusepath{clip}%
\pgfsetbuttcap%
\pgfsetroundjoin%
\pgfsetlinewidth{0.803000pt}%
\definecolor{currentstroke}{rgb}{0.690196,0.690196,0.690196}%
\pgfsetstrokecolor{currentstroke}%
\pgfsetstrokeopacity{0.200000}%
\pgfsetdash{{2.960000pt}{1.280000pt}}{0.000000pt}%
\pgfpathmoveto{\pgfqpoint{0.764581in}{4.412369in}}%
\pgfpathlineto{\pgfqpoint{6.964581in}{4.412369in}}%
\pgfusepath{stroke}%
\end{pgfscope}%
\begin{pgfscope}%
\pgfsetbuttcap%
\pgfsetroundjoin%
\definecolor{currentfill}{rgb}{0.000000,0.000000,0.000000}%
\pgfsetfillcolor{currentfill}%
\pgfsetlinewidth{0.602250pt}%
\definecolor{currentstroke}{rgb}{0.000000,0.000000,0.000000}%
\pgfsetstrokecolor{currentstroke}%
\pgfsetdash{}{0pt}%
\pgfsys@defobject{currentmarker}{\pgfqpoint{-0.027778in}{0.000000in}}{\pgfqpoint{-0.000000in}{0.000000in}}{%
\pgfpathmoveto{\pgfqpoint{-0.000000in}{0.000000in}}%
\pgfpathlineto{\pgfqpoint{-0.027778in}{0.000000in}}%
\pgfusepath{stroke,fill}%
}%
\begin{pgfscope}%
\pgfsys@transformshift{0.764581in}{4.412369in}%
\pgfsys@useobject{currentmarker}{}%
\end{pgfscope}%
\end{pgfscope}%
\begin{pgfscope}%
\pgfpathrectangle{\pgfqpoint{0.764581in}{0.643904in}}{\pgfqpoint{6.200000in}{4.620000in}}%
\pgfusepath{clip}%
\pgfsetbuttcap%
\pgfsetroundjoin%
\pgfsetlinewidth{0.803000pt}%
\definecolor{currentstroke}{rgb}{0.690196,0.690196,0.690196}%
\pgfsetstrokecolor{currentstroke}%
\pgfsetstrokeopacity{0.200000}%
\pgfsetdash{{2.960000pt}{1.280000pt}}{0.000000pt}%
\pgfpathmoveto{\pgfqpoint{0.764581in}{4.610689in}}%
\pgfpathlineto{\pgfqpoint{6.964581in}{4.610689in}}%
\pgfusepath{stroke}%
\end{pgfscope}%
\begin{pgfscope}%
\pgfsetbuttcap%
\pgfsetroundjoin%
\definecolor{currentfill}{rgb}{0.000000,0.000000,0.000000}%
\pgfsetfillcolor{currentfill}%
\pgfsetlinewidth{0.602250pt}%
\definecolor{currentstroke}{rgb}{0.000000,0.000000,0.000000}%
\pgfsetstrokecolor{currentstroke}%
\pgfsetdash{}{0pt}%
\pgfsys@defobject{currentmarker}{\pgfqpoint{-0.027778in}{0.000000in}}{\pgfqpoint{-0.000000in}{0.000000in}}{%
\pgfpathmoveto{\pgfqpoint{-0.000000in}{0.000000in}}%
\pgfpathlineto{\pgfqpoint{-0.027778in}{0.000000in}}%
\pgfusepath{stroke,fill}%
}%
\begin{pgfscope}%
\pgfsys@transformshift{0.764581in}{4.610689in}%
\pgfsys@useobject{currentmarker}{}%
\end{pgfscope}%
\end{pgfscope}%
\begin{pgfscope}%
\pgfpathrectangle{\pgfqpoint{0.764581in}{0.643904in}}{\pgfqpoint{6.200000in}{4.620000in}}%
\pgfusepath{clip}%
\pgfsetbuttcap%
\pgfsetroundjoin%
\pgfsetlinewidth{0.803000pt}%
\definecolor{currentstroke}{rgb}{0.690196,0.690196,0.690196}%
\pgfsetstrokecolor{currentstroke}%
\pgfsetstrokeopacity{0.200000}%
\pgfsetdash{{2.960000pt}{1.280000pt}}{0.000000pt}%
\pgfpathmoveto{\pgfqpoint{0.764581in}{4.764517in}}%
\pgfpathlineto{\pgfqpoint{6.964581in}{4.764517in}}%
\pgfusepath{stroke}%
\end{pgfscope}%
\begin{pgfscope}%
\pgfsetbuttcap%
\pgfsetroundjoin%
\definecolor{currentfill}{rgb}{0.000000,0.000000,0.000000}%
\pgfsetfillcolor{currentfill}%
\pgfsetlinewidth{0.602250pt}%
\definecolor{currentstroke}{rgb}{0.000000,0.000000,0.000000}%
\pgfsetstrokecolor{currentstroke}%
\pgfsetdash{}{0pt}%
\pgfsys@defobject{currentmarker}{\pgfqpoint{-0.027778in}{0.000000in}}{\pgfqpoint{-0.000000in}{0.000000in}}{%
\pgfpathmoveto{\pgfqpoint{-0.000000in}{0.000000in}}%
\pgfpathlineto{\pgfqpoint{-0.027778in}{0.000000in}}%
\pgfusepath{stroke,fill}%
}%
\begin{pgfscope}%
\pgfsys@transformshift{0.764581in}{4.764517in}%
\pgfsys@useobject{currentmarker}{}%
\end{pgfscope}%
\end{pgfscope}%
\begin{pgfscope}%
\pgfpathrectangle{\pgfqpoint{0.764581in}{0.643904in}}{\pgfqpoint{6.200000in}{4.620000in}}%
\pgfusepath{clip}%
\pgfsetbuttcap%
\pgfsetroundjoin%
\pgfsetlinewidth{0.803000pt}%
\definecolor{currentstroke}{rgb}{0.690196,0.690196,0.690196}%
\pgfsetstrokecolor{currentstroke}%
\pgfsetstrokeopacity{0.200000}%
\pgfsetdash{{2.960000pt}{1.280000pt}}{0.000000pt}%
\pgfpathmoveto{\pgfqpoint{0.764581in}{4.890204in}}%
\pgfpathlineto{\pgfqpoint{6.964581in}{4.890204in}}%
\pgfusepath{stroke}%
\end{pgfscope}%
\begin{pgfscope}%
\pgfsetbuttcap%
\pgfsetroundjoin%
\definecolor{currentfill}{rgb}{0.000000,0.000000,0.000000}%
\pgfsetfillcolor{currentfill}%
\pgfsetlinewidth{0.602250pt}%
\definecolor{currentstroke}{rgb}{0.000000,0.000000,0.000000}%
\pgfsetstrokecolor{currentstroke}%
\pgfsetdash{}{0pt}%
\pgfsys@defobject{currentmarker}{\pgfqpoint{-0.027778in}{0.000000in}}{\pgfqpoint{-0.000000in}{0.000000in}}{%
\pgfpathmoveto{\pgfqpoint{-0.000000in}{0.000000in}}%
\pgfpathlineto{\pgfqpoint{-0.027778in}{0.000000in}}%
\pgfusepath{stroke,fill}%
}%
\begin{pgfscope}%
\pgfsys@transformshift{0.764581in}{4.890204in}%
\pgfsys@useobject{currentmarker}{}%
\end{pgfscope}%
\end{pgfscope}%
\begin{pgfscope}%
\pgfpathrectangle{\pgfqpoint{0.764581in}{0.643904in}}{\pgfqpoint{6.200000in}{4.620000in}}%
\pgfusepath{clip}%
\pgfsetbuttcap%
\pgfsetroundjoin%
\pgfsetlinewidth{0.803000pt}%
\definecolor{currentstroke}{rgb}{0.690196,0.690196,0.690196}%
\pgfsetstrokecolor{currentstroke}%
\pgfsetstrokeopacity{0.200000}%
\pgfsetdash{{2.960000pt}{1.280000pt}}{0.000000pt}%
\pgfpathmoveto{\pgfqpoint{0.764581in}{4.996471in}}%
\pgfpathlineto{\pgfqpoint{6.964581in}{4.996471in}}%
\pgfusepath{stroke}%
\end{pgfscope}%
\begin{pgfscope}%
\pgfsetbuttcap%
\pgfsetroundjoin%
\definecolor{currentfill}{rgb}{0.000000,0.000000,0.000000}%
\pgfsetfillcolor{currentfill}%
\pgfsetlinewidth{0.602250pt}%
\definecolor{currentstroke}{rgb}{0.000000,0.000000,0.000000}%
\pgfsetstrokecolor{currentstroke}%
\pgfsetdash{}{0pt}%
\pgfsys@defobject{currentmarker}{\pgfqpoint{-0.027778in}{0.000000in}}{\pgfqpoint{-0.000000in}{0.000000in}}{%
\pgfpathmoveto{\pgfqpoint{-0.000000in}{0.000000in}}%
\pgfpathlineto{\pgfqpoint{-0.027778in}{0.000000in}}%
\pgfusepath{stroke,fill}%
}%
\begin{pgfscope}%
\pgfsys@transformshift{0.764581in}{4.996471in}%
\pgfsys@useobject{currentmarker}{}%
\end{pgfscope}%
\end{pgfscope}%
\begin{pgfscope}%
\pgfpathrectangle{\pgfqpoint{0.764581in}{0.643904in}}{\pgfqpoint{6.200000in}{4.620000in}}%
\pgfusepath{clip}%
\pgfsetbuttcap%
\pgfsetroundjoin%
\pgfsetlinewidth{0.803000pt}%
\definecolor{currentstroke}{rgb}{0.690196,0.690196,0.690196}%
\pgfsetstrokecolor{currentstroke}%
\pgfsetstrokeopacity{0.200000}%
\pgfsetdash{{2.960000pt}{1.280000pt}}{0.000000pt}%
\pgfpathmoveto{\pgfqpoint{0.764581in}{5.088524in}}%
\pgfpathlineto{\pgfqpoint{6.964581in}{5.088524in}}%
\pgfusepath{stroke}%
\end{pgfscope}%
\begin{pgfscope}%
\pgfsetbuttcap%
\pgfsetroundjoin%
\definecolor{currentfill}{rgb}{0.000000,0.000000,0.000000}%
\pgfsetfillcolor{currentfill}%
\pgfsetlinewidth{0.602250pt}%
\definecolor{currentstroke}{rgb}{0.000000,0.000000,0.000000}%
\pgfsetstrokecolor{currentstroke}%
\pgfsetdash{}{0pt}%
\pgfsys@defobject{currentmarker}{\pgfqpoint{-0.027778in}{0.000000in}}{\pgfqpoint{-0.000000in}{0.000000in}}{%
\pgfpathmoveto{\pgfqpoint{-0.000000in}{0.000000in}}%
\pgfpathlineto{\pgfqpoint{-0.027778in}{0.000000in}}%
\pgfusepath{stroke,fill}%
}%
\begin{pgfscope}%
\pgfsys@transformshift{0.764581in}{5.088524in}%
\pgfsys@useobject{currentmarker}{}%
\end{pgfscope}%
\end{pgfscope}%
\begin{pgfscope}%
\pgfpathrectangle{\pgfqpoint{0.764581in}{0.643904in}}{\pgfqpoint{6.200000in}{4.620000in}}%
\pgfusepath{clip}%
\pgfsetbuttcap%
\pgfsetroundjoin%
\pgfsetlinewidth{0.803000pt}%
\definecolor{currentstroke}{rgb}{0.690196,0.690196,0.690196}%
\pgfsetstrokecolor{currentstroke}%
\pgfsetstrokeopacity{0.200000}%
\pgfsetdash{{2.960000pt}{1.280000pt}}{0.000000pt}%
\pgfpathmoveto{\pgfqpoint{0.764581in}{5.169720in}}%
\pgfpathlineto{\pgfqpoint{6.964581in}{5.169720in}}%
\pgfusepath{stroke}%
\end{pgfscope}%
\begin{pgfscope}%
\pgfsetbuttcap%
\pgfsetroundjoin%
\definecolor{currentfill}{rgb}{0.000000,0.000000,0.000000}%
\pgfsetfillcolor{currentfill}%
\pgfsetlinewidth{0.602250pt}%
\definecolor{currentstroke}{rgb}{0.000000,0.000000,0.000000}%
\pgfsetstrokecolor{currentstroke}%
\pgfsetdash{}{0pt}%
\pgfsys@defobject{currentmarker}{\pgfqpoint{-0.027778in}{0.000000in}}{\pgfqpoint{-0.000000in}{0.000000in}}{%
\pgfpathmoveto{\pgfqpoint{-0.000000in}{0.000000in}}%
\pgfpathlineto{\pgfqpoint{-0.027778in}{0.000000in}}%
\pgfusepath{stroke,fill}%
}%
\begin{pgfscope}%
\pgfsys@transformshift{0.764581in}{5.169720in}%
\pgfsys@useobject{currentmarker}{}%
\end{pgfscope}%
\end{pgfscope}%
\begin{pgfscope}%
\definecolor{textcolor}{rgb}{0.000000,0.000000,0.000000}%
\pgfsetstrokecolor{textcolor}%
\pgfsetfillcolor{textcolor}%
\pgftext[x=0.339583in,y=2.953904in,,bottom,rotate=90.000000]{\color{textcolor}{\rmfamily\fontsize{18.000000}{21.600000}\selectfont\catcode`\^=\active\def^{\ifmmode\sp\else\^{}\fi}\catcode`\%=\active\def%{\%}Time [seconds]}}%
\end{pgfscope}%
\begin{pgfscope}%
\pgfsetrectcap%
\pgfsetmiterjoin%
\pgfsetlinewidth{0.803000pt}%
\definecolor{currentstroke}{rgb}{0.000000,0.000000,0.000000}%
\pgfsetstrokecolor{currentstroke}%
\pgfsetdash{}{0pt}%
\pgfpathmoveto{\pgfqpoint{0.764581in}{0.643904in}}%
\pgfpathlineto{\pgfqpoint{0.764581in}{5.263904in}}%
\pgfusepath{stroke}%
\end{pgfscope}%
\begin{pgfscope}%
\pgfsetrectcap%
\pgfsetmiterjoin%
\pgfsetlinewidth{0.803000pt}%
\definecolor{currentstroke}{rgb}{0.000000,0.000000,0.000000}%
\pgfsetstrokecolor{currentstroke}%
\pgfsetdash{}{0pt}%
\pgfpathmoveto{\pgfqpoint{6.964581in}{0.643904in}}%
\pgfpathlineto{\pgfqpoint{6.964581in}{5.263904in}}%
\pgfusepath{stroke}%
\end{pgfscope}%
\begin{pgfscope}%
\pgfsetrectcap%
\pgfsetmiterjoin%
\pgfsetlinewidth{0.803000pt}%
\definecolor{currentstroke}{rgb}{0.000000,0.000000,0.000000}%
\pgfsetstrokecolor{currentstroke}%
\pgfsetdash{}{0pt}%
\pgfpathmoveto{\pgfqpoint{0.764581in}{0.643904in}}%
\pgfpathlineto{\pgfqpoint{6.964581in}{0.643904in}}%
\pgfusepath{stroke}%
\end{pgfscope}%
\begin{pgfscope}%
\pgfsetrectcap%
\pgfsetmiterjoin%
\pgfsetlinewidth{0.803000pt}%
\definecolor{currentstroke}{rgb}{0.000000,0.000000,0.000000}%
\pgfsetstrokecolor{currentstroke}%
\pgfsetdash{}{0pt}%
\pgfpathmoveto{\pgfqpoint{0.764581in}{5.263904in}}%
\pgfpathlineto{\pgfqpoint{6.964581in}{5.263904in}}%
\pgfusepath{stroke}%
\end{pgfscope}%
\begin{pgfscope}%
\pgfpathrectangle{\pgfqpoint{0.764581in}{0.643904in}}{\pgfqpoint{6.200000in}{4.620000in}}%
\pgfusepath{clip}%
\pgfsetrectcap%
\pgfsetroundjoin%
\pgfsetlinewidth{1.505625pt}%
\definecolor{currentstroke}{rgb}{0.000000,0.000000,1.000000}%
\pgfsetstrokecolor{currentstroke}%
\pgfsetdash{}{0pt}%
\pgfpathmoveto{\pgfqpoint{0.764581in}{0.853904in}}%
\pgfpathlineto{\pgfqpoint{2.455884in}{1.625606in}}%
\pgfpathlineto{\pgfqpoint{3.184288in}{2.056435in}}%
\pgfpathlineto{\pgfqpoint{4.147187in}{2.648049in}}%
\pgfpathlineto{\pgfqpoint{4.875592in}{3.110216in}}%
\pgfpathlineto{\pgfqpoint{5.603996in}{3.550323in}}%
\pgfpathlineto{\pgfqpoint{6.332401in}{4.052710in}}%
\pgfpathlineto{\pgfqpoint{6.758490in}{4.325160in}}%
\pgfpathlineto{\pgfqpoint{6.964581in}{4.448402in}}%
\pgfusepath{stroke}%
\end{pgfscope}%
\begin{pgfscope}%
\pgfpathrectangle{\pgfqpoint{0.764581in}{0.643904in}}{\pgfqpoint{6.200000in}{4.620000in}}%
\pgfusepath{clip}%
\pgfsetbuttcap%
\pgfsetroundjoin%
\definecolor{currentfill}{rgb}{0.000000,0.000000,1.000000}%
\pgfsetfillcolor{currentfill}%
\pgfsetlinewidth{1.003750pt}%
\definecolor{currentstroke}{rgb}{0.000000,0.000000,1.000000}%
\pgfsetstrokecolor{currentstroke}%
\pgfsetdash{}{0pt}%
\pgfsys@defobject{currentmarker}{\pgfqpoint{-0.041667in}{-0.041667in}}{\pgfqpoint{0.041667in}{0.041667in}}{%
\pgfpathmoveto{\pgfqpoint{0.000000in}{-0.041667in}}%
\pgfpathcurveto{\pgfqpoint{0.011050in}{-0.041667in}}{\pgfqpoint{0.021649in}{-0.037276in}}{\pgfqpoint{0.029463in}{-0.029463in}}%
\pgfpathcurveto{\pgfqpoint{0.037276in}{-0.021649in}}{\pgfqpoint{0.041667in}{-0.011050in}}{\pgfqpoint{0.041667in}{0.000000in}}%
\pgfpathcurveto{\pgfqpoint{0.041667in}{0.011050in}}{\pgfqpoint{0.037276in}{0.021649in}}{\pgfqpoint{0.029463in}{0.029463in}}%
\pgfpathcurveto{\pgfqpoint{0.021649in}{0.037276in}}{\pgfqpoint{0.011050in}{0.041667in}}{\pgfqpoint{0.000000in}{0.041667in}}%
\pgfpathcurveto{\pgfqpoint{-0.011050in}{0.041667in}}{\pgfqpoint{-0.021649in}{0.037276in}}{\pgfqpoint{-0.029463in}{0.029463in}}%
\pgfpathcurveto{\pgfqpoint{-0.037276in}{0.021649in}}{\pgfqpoint{-0.041667in}{0.011050in}}{\pgfqpoint{-0.041667in}{0.000000in}}%
\pgfpathcurveto{\pgfqpoint{-0.041667in}{-0.011050in}}{\pgfqpoint{-0.037276in}{-0.021649in}}{\pgfqpoint{-0.029463in}{-0.029463in}}%
\pgfpathcurveto{\pgfqpoint{-0.021649in}{-0.037276in}}{\pgfqpoint{-0.011050in}{-0.041667in}}{\pgfqpoint{0.000000in}{-0.041667in}}%
\pgfpathlineto{\pgfqpoint{0.000000in}{-0.041667in}}%
\pgfpathclose%
\pgfusepath{stroke,fill}%
}%
\begin{pgfscope}%
\pgfsys@transformshift{0.764581in}{0.853904in}%
\pgfsys@useobject{currentmarker}{}%
\end{pgfscope}%
\begin{pgfscope}%
\pgfsys@transformshift{2.455884in}{1.625606in}%
\pgfsys@useobject{currentmarker}{}%
\end{pgfscope}%
\begin{pgfscope}%
\pgfsys@transformshift{3.184288in}{2.056435in}%
\pgfsys@useobject{currentmarker}{}%
\end{pgfscope}%
\begin{pgfscope}%
\pgfsys@transformshift{4.147187in}{2.648049in}%
\pgfsys@useobject{currentmarker}{}%
\end{pgfscope}%
\begin{pgfscope}%
\pgfsys@transformshift{4.875592in}{3.110216in}%
\pgfsys@useobject{currentmarker}{}%
\end{pgfscope}%
\begin{pgfscope}%
\pgfsys@transformshift{5.603996in}{3.550323in}%
\pgfsys@useobject{currentmarker}{}%
\end{pgfscope}%
\begin{pgfscope}%
\pgfsys@transformshift{6.332401in}{4.052710in}%
\pgfsys@useobject{currentmarker}{}%
\end{pgfscope}%
\begin{pgfscope}%
\pgfsys@transformshift{6.758490in}{4.325160in}%
\pgfsys@useobject{currentmarker}{}%
\end{pgfscope}%
\begin{pgfscope}%
\pgfsys@transformshift{6.964581in}{4.448402in}%
\pgfsys@useobject{currentmarker}{}%
\end{pgfscope}%
\end{pgfscope}%
\begin{pgfscope}%
\pgfpathrectangle{\pgfqpoint{0.764581in}{0.643904in}}{\pgfqpoint{6.200000in}{4.620000in}}%
\pgfusepath{clip}%
\pgfsetbuttcap%
\pgfsetroundjoin%
\pgfsetlinewidth{1.505625pt}%
\definecolor{currentstroke}{rgb}{1.000000,0.000000,0.000000}%
\pgfsetstrokecolor{currentstroke}%
\pgfsetdash{{5.550000pt}{2.400000pt}}{0.000000pt}%
\pgfpathmoveto{\pgfqpoint{0.764581in}{3.393336in}}%
\pgfpathlineto{\pgfqpoint{2.455884in}{3.469107in}}%
\pgfpathlineto{\pgfqpoint{3.184288in}{3.545769in}}%
\pgfpathlineto{\pgfqpoint{4.147187in}{3.728731in}}%
\pgfpathlineto{\pgfqpoint{4.875592in}{3.946418in}}%
\pgfpathlineto{\pgfqpoint{5.603996in}{4.254357in}}%
\pgfpathlineto{\pgfqpoint{6.332401in}{4.649383in}}%
\pgfpathlineto{\pgfqpoint{6.758490in}{4.916608in}}%
\pgfpathlineto{\pgfqpoint{6.964581in}{5.053904in}}%
\pgfusepath{stroke}%
\end{pgfscope}%
\begin{pgfscope}%
\pgfpathrectangle{\pgfqpoint{0.764581in}{0.643904in}}{\pgfqpoint{6.200000in}{4.620000in}}%
\pgfusepath{clip}%
\pgfsetbuttcap%
\pgfsetroundjoin%
\definecolor{currentfill}{rgb}{1.000000,0.000000,0.000000}%
\pgfsetfillcolor{currentfill}%
\pgfsetlinewidth{1.003750pt}%
\definecolor{currentstroke}{rgb}{1.000000,0.000000,0.000000}%
\pgfsetstrokecolor{currentstroke}%
\pgfsetdash{}{0pt}%
\pgfsys@defobject{currentmarker}{\pgfqpoint{-0.041667in}{-0.041667in}}{\pgfqpoint{0.041667in}{0.041667in}}{%
\pgfpathmoveto{\pgfqpoint{0.000000in}{-0.041667in}}%
\pgfpathcurveto{\pgfqpoint{0.011050in}{-0.041667in}}{\pgfqpoint{0.021649in}{-0.037276in}}{\pgfqpoint{0.029463in}{-0.029463in}}%
\pgfpathcurveto{\pgfqpoint{0.037276in}{-0.021649in}}{\pgfqpoint{0.041667in}{-0.011050in}}{\pgfqpoint{0.041667in}{0.000000in}}%
\pgfpathcurveto{\pgfqpoint{0.041667in}{0.011050in}}{\pgfqpoint{0.037276in}{0.021649in}}{\pgfqpoint{0.029463in}{0.029463in}}%
\pgfpathcurveto{\pgfqpoint{0.021649in}{0.037276in}}{\pgfqpoint{0.011050in}{0.041667in}}{\pgfqpoint{0.000000in}{0.041667in}}%
\pgfpathcurveto{\pgfqpoint{-0.011050in}{0.041667in}}{\pgfqpoint{-0.021649in}{0.037276in}}{\pgfqpoint{-0.029463in}{0.029463in}}%
\pgfpathcurveto{\pgfqpoint{-0.037276in}{0.021649in}}{\pgfqpoint{-0.041667in}{0.011050in}}{\pgfqpoint{-0.041667in}{0.000000in}}%
\pgfpathcurveto{\pgfqpoint{-0.041667in}{-0.011050in}}{\pgfqpoint{-0.037276in}{-0.021649in}}{\pgfqpoint{-0.029463in}{-0.029463in}}%
\pgfpathcurveto{\pgfqpoint{-0.021649in}{-0.037276in}}{\pgfqpoint{-0.011050in}{-0.041667in}}{\pgfqpoint{0.000000in}{-0.041667in}}%
\pgfpathlineto{\pgfqpoint{0.000000in}{-0.041667in}}%
\pgfpathclose%
\pgfusepath{stroke,fill}%
}%
\begin{pgfscope}%
\pgfsys@transformshift{0.764581in}{3.393336in}%
\pgfsys@useobject{currentmarker}{}%
\end{pgfscope}%
\begin{pgfscope}%
\pgfsys@transformshift{2.455884in}{3.469107in}%
\pgfsys@useobject{currentmarker}{}%
\end{pgfscope}%
\begin{pgfscope}%
\pgfsys@transformshift{3.184288in}{3.545769in}%
\pgfsys@useobject{currentmarker}{}%
\end{pgfscope}%
\begin{pgfscope}%
\pgfsys@transformshift{4.147187in}{3.728731in}%
\pgfsys@useobject{currentmarker}{}%
\end{pgfscope}%
\begin{pgfscope}%
\pgfsys@transformshift{4.875592in}{3.946418in}%
\pgfsys@useobject{currentmarker}{}%
\end{pgfscope}%
\begin{pgfscope}%
\pgfsys@transformshift{5.603996in}{4.254357in}%
\pgfsys@useobject{currentmarker}{}%
\end{pgfscope}%
\begin{pgfscope}%
\pgfsys@transformshift{6.332401in}{4.649383in}%
\pgfsys@useobject{currentmarker}{}%
\end{pgfscope}%
\begin{pgfscope}%
\pgfsys@transformshift{6.758490in}{4.916608in}%
\pgfsys@useobject{currentmarker}{}%
\end{pgfscope}%
\begin{pgfscope}%
\pgfsys@transformshift{6.964581in}{5.053904in}%
\pgfsys@useobject{currentmarker}{}%
\end{pgfscope}%
\end{pgfscope}%
\begin{pgfscope}%
\pgfsetbuttcap%
\pgfsetmiterjoin%
\definecolor{currentfill}{rgb}{1.000000,1.000000,1.000000}%
\pgfsetfillcolor{currentfill}%
\pgfsetfillopacity{0.800000}%
\pgfsetlinewidth{1.003750pt}%
\definecolor{currentstroke}{rgb}{0.800000,0.800000,0.800000}%
\pgfsetstrokecolor{currentstroke}%
\pgfsetstrokeopacity{0.800000}%
\pgfsetdash{}{0pt}%
\pgfpathmoveto{\pgfqpoint{0.920136in}{4.437207in}}%
\pgfpathlineto{\pgfqpoint{3.236863in}{4.437207in}}%
\pgfpathquadraticcurveto{\pgfqpoint{3.281307in}{4.437207in}}{\pgfqpoint{3.281307in}{4.481651in}}%
\pgfpathlineto{\pgfqpoint{3.281307in}{5.108348in}}%
\pgfpathquadraticcurveto{\pgfqpoint{3.281307in}{5.152793in}}{\pgfqpoint{3.236863in}{5.152793in}}%
\pgfpathlineto{\pgfqpoint{0.920136in}{5.152793in}}%
\pgfpathquadraticcurveto{\pgfqpoint{0.875692in}{5.152793in}}{\pgfqpoint{0.875692in}{5.108348in}}%
\pgfpathlineto{\pgfqpoint{0.875692in}{4.481651in}}%
\pgfpathquadraticcurveto{\pgfqpoint{0.875692in}{4.437207in}}{\pgfqpoint{0.920136in}{4.437207in}}%
\pgfpathlineto{\pgfqpoint{0.920136in}{4.437207in}}%
\pgfpathclose%
\pgfusepath{stroke,fill}%
\end{pgfscope}%
\begin{pgfscope}%
\pgfsetrectcap%
\pgfsetroundjoin%
\pgfsetlinewidth{1.505625pt}%
\definecolor{currentstroke}{rgb}{0.000000,0.000000,1.000000}%
\pgfsetstrokecolor{currentstroke}%
\pgfsetdash{}{0pt}%
\pgfpathmoveto{\pgfqpoint{0.964581in}{4.975015in}}%
\pgfpathlineto{\pgfqpoint{1.186803in}{4.975015in}}%
\pgfpathlineto{\pgfqpoint{1.409025in}{4.975015in}}%
\pgfusepath{stroke}%
\end{pgfscope}%
\begin{pgfscope}%
\pgfsetbuttcap%
\pgfsetroundjoin%
\definecolor{currentfill}{rgb}{0.000000,0.000000,1.000000}%
\pgfsetfillcolor{currentfill}%
\pgfsetlinewidth{1.003750pt}%
\definecolor{currentstroke}{rgb}{0.000000,0.000000,1.000000}%
\pgfsetstrokecolor{currentstroke}%
\pgfsetdash{}{0pt}%
\pgfsys@defobject{currentmarker}{\pgfqpoint{-0.041667in}{-0.041667in}}{\pgfqpoint{0.041667in}{0.041667in}}{%
\pgfpathmoveto{\pgfqpoint{0.000000in}{-0.041667in}}%
\pgfpathcurveto{\pgfqpoint{0.011050in}{-0.041667in}}{\pgfqpoint{0.021649in}{-0.037276in}}{\pgfqpoint{0.029463in}{-0.029463in}}%
\pgfpathcurveto{\pgfqpoint{0.037276in}{-0.021649in}}{\pgfqpoint{0.041667in}{-0.011050in}}{\pgfqpoint{0.041667in}{0.000000in}}%
\pgfpathcurveto{\pgfqpoint{0.041667in}{0.011050in}}{\pgfqpoint{0.037276in}{0.021649in}}{\pgfqpoint{0.029463in}{0.029463in}}%
\pgfpathcurveto{\pgfqpoint{0.021649in}{0.037276in}}{\pgfqpoint{0.011050in}{0.041667in}}{\pgfqpoint{0.000000in}{0.041667in}}%
\pgfpathcurveto{\pgfqpoint{-0.011050in}{0.041667in}}{\pgfqpoint{-0.021649in}{0.037276in}}{\pgfqpoint{-0.029463in}{0.029463in}}%
\pgfpathcurveto{\pgfqpoint{-0.037276in}{0.021649in}}{\pgfqpoint{-0.041667in}{0.011050in}}{\pgfqpoint{-0.041667in}{0.000000in}}%
\pgfpathcurveto{\pgfqpoint{-0.041667in}{-0.011050in}}{\pgfqpoint{-0.037276in}{-0.021649in}}{\pgfqpoint{-0.029463in}{-0.029463in}}%
\pgfpathcurveto{\pgfqpoint{-0.021649in}{-0.037276in}}{\pgfqpoint{-0.011050in}{-0.041667in}}{\pgfqpoint{0.000000in}{-0.041667in}}%
\pgfpathlineto{\pgfqpoint{0.000000in}{-0.041667in}}%
\pgfpathclose%
\pgfusepath{stroke,fill}%
}%
\begin{pgfscope}%
\pgfsys@transformshift{1.186803in}{4.975015in}%
\pgfsys@useobject{currentmarker}{}%
\end{pgfscope}%
\end{pgfscope}%
\begin{pgfscope}%
\definecolor{textcolor}{rgb}{0.000000,0.000000,0.000000}%
\pgfsetstrokecolor{textcolor}%
\pgfsetfillcolor{textcolor}%
\pgftext[x=1.586803in,y=4.897237in,left,base]{\color{textcolor}{\rmfamily\fontsize{16.000000}{19.200000}\selectfont\catcode`\^=\active\def^{\ifmmode\sp\else\^{}\fi}\catcode`\%=\active\def%{\%}logical dispatch}}%
\end{pgfscope}%
\begin{pgfscope}%
\pgfsetbuttcap%
\pgfsetroundjoin%
\pgfsetlinewidth{1.505625pt}%
\definecolor{currentstroke}{rgb}{1.000000,0.000000,0.000000}%
\pgfsetstrokecolor{currentstroke}%
\pgfsetdash{{5.550000pt}{2.400000pt}}{0.000000pt}%
\pgfpathmoveto{\pgfqpoint{0.964581in}{4.650555in}}%
\pgfpathlineto{\pgfqpoint{1.186803in}{4.650555in}}%
\pgfpathlineto{\pgfqpoint{1.409025in}{4.650555in}}%
\pgfusepath{stroke}%
\end{pgfscope}%
\begin{pgfscope}%
\pgfsetbuttcap%
\pgfsetroundjoin%
\definecolor{currentfill}{rgb}{1.000000,0.000000,0.000000}%
\pgfsetfillcolor{currentfill}%
\pgfsetlinewidth{1.003750pt}%
\definecolor{currentstroke}{rgb}{1.000000,0.000000,0.000000}%
\pgfsetstrokecolor{currentstroke}%
\pgfsetdash{}{0pt}%
\pgfsys@defobject{currentmarker}{\pgfqpoint{-0.041667in}{-0.041667in}}{\pgfqpoint{0.041667in}{0.041667in}}{%
\pgfpathmoveto{\pgfqpoint{0.000000in}{-0.041667in}}%
\pgfpathcurveto{\pgfqpoint{0.011050in}{-0.041667in}}{\pgfqpoint{0.021649in}{-0.037276in}}{\pgfqpoint{0.029463in}{-0.029463in}}%
\pgfpathcurveto{\pgfqpoint{0.037276in}{-0.021649in}}{\pgfqpoint{0.041667in}{-0.011050in}}{\pgfqpoint{0.041667in}{0.000000in}}%
\pgfpathcurveto{\pgfqpoint{0.041667in}{0.011050in}}{\pgfqpoint{0.037276in}{0.021649in}}{\pgfqpoint{0.029463in}{0.029463in}}%
\pgfpathcurveto{\pgfqpoint{0.021649in}{0.037276in}}{\pgfqpoint{0.011050in}{0.041667in}}{\pgfqpoint{0.000000in}{0.041667in}}%
\pgfpathcurveto{\pgfqpoint{-0.011050in}{0.041667in}}{\pgfqpoint{-0.021649in}{0.037276in}}{\pgfqpoint{-0.029463in}{0.029463in}}%
\pgfpathcurveto{\pgfqpoint{-0.037276in}{0.021649in}}{\pgfqpoint{-0.041667in}{0.011050in}}{\pgfqpoint{-0.041667in}{0.000000in}}%
\pgfpathcurveto{\pgfqpoint{-0.041667in}{-0.011050in}}{\pgfqpoint{-0.037276in}{-0.021649in}}{\pgfqpoint{-0.029463in}{-0.029463in}}%
\pgfpathcurveto{\pgfqpoint{-0.021649in}{-0.037276in}}{\pgfqpoint{-0.011050in}{-0.041667in}}{\pgfqpoint{0.000000in}{-0.041667in}}%
\pgfpathlineto{\pgfqpoint{0.000000in}{-0.041667in}}%
\pgfpathclose%
\pgfusepath{stroke,fill}%
}%
\begin{pgfscope}%
\pgfsys@transformshift{1.186803in}{4.650555in}%
\pgfsys@useobject{currentmarker}{}%
\end{pgfscope}%
\end{pgfscope}%
\begin{pgfscope}%
\definecolor{textcolor}{rgb}{0.000000,0.000000,0.000000}%
\pgfsetstrokecolor{textcolor}%
\pgfsetfillcolor{textcolor}%
\pgftext[x=1.586803in,y=4.572777in,left,base]{\color{textcolor}{\rmfamily\fontsize{16.000000}{19.200000}\selectfont\catcode`\^=\active\def^{\ifmmode\sp\else\^{}\fi}\catcode`\%=\active\def%{\%}optimal dispatch}}%
\end{pgfscope}%
\end{pgfpicture}%
\makeatother%
\endgroup%
}
    \caption{Time scaling of a capacity expansion problem using either an optimal or logical dispatch algorithm.}
    \label{fig:alg-scaling}
\end{figure}

\noindent Initially, the logical dispatch algorithm outperforms the optimal
dispatch algorithm by nearly two orders of magnitude. This is because the linear
program has some overhead when writing and copying equations that the rule-based
calculation does not. The logical algorithm initially grows more quickly until
the models reach 100 modeled days after which the two algorithms scale
similarly and the logical dispatch algorithm remains approximately 2.5 times
faster than its optimal counterpart.

\subsection{Exercise 3: Parallelization}

\Acp{ga} are considered ``embarrassingly parallelizable'' since the performance
of each individual in a population is independent from the others. However, there
a some difficulties with solving multiple parallel instances of \ac{lp} solvers
since these solvers frequently have some parallel optimizations built-in. For
now, this restricts capacity expansion problems within \ac{osier} that use linear 
programming to serial calculations. This is not so for the logical dispatch algorithm 
since it does
not use an \ac{lp} solver. Therefore, this exercise looks exclusively at how the
logical dispatch algorithm scales with number of threads available. Once again,
the dispatch algorithm is driven by \ac{osier}'s \texttt{CapacityExpansion}
class whose parameters are described in Table \ref{tab:thread-scaling-params}. 

\begin{table}[htbp!]
    \centering
    \caption{Capacity expansion parameters for the parallelization exercise.}
    \label{tab:thread-scaling-params}
    \begin{tabular}{ll}
        \toprule
        Parameter & Value \\
        \midrule
        Algorithm & \acs{nsga2}\\
        Termination Criterion & Maximum Generations\\
        Generations & 10 \\
        Objectives & 2 (cost, emissions)\\
        Timesteps & 120 (5 days x 24 hours)\\
        \bottomrule
    \end{tabular}
\end{table}

\noindent In this exercise, the problem is scaled by the population size of each
generation. The study was performed on a 2024 MacBook Pro with an M4 Pro CPU, 
48 GB of RAM, and the macOS Sequoia 15.5 operating system. Figure
\ref{fig:thread-scaling} shows the results for this exercise.

\begin{figure}[htbp!]
    \centering
    \resizebox{0.75\columnwidth}{!}{%% Creator: Matplotlib, PGF backend
%%
%% To include the figure in your LaTeX document, write
%%   \input{<filename>.pgf}
%%
%% Make sure the required packages are loaded in your preamble
%%   \usepackage{pgf}
%%
%% Also ensure that all the required font packages are loaded; for instance,
%% the lmodern package is sometimes necessary when using math font.
%%   \usepackage{lmodern}
%%
%% Figures using additional raster images can only be included by \input if
%% they are in the same directory as the main LaTeX file. For loading figures
%% from other directories you can use the `import` package
%%   \usepackage{import}
%%
%% and then include the figures with
%%   \import{<path to file>}{<filename>.pgf}
%%
%% Matplotlib used the following preamble
%%   \def\mathdefault#1{#1}
%%   \everymath=\expandafter{\the\everymath\displaystyle}
%%   \IfFileExists{scrextend.sty}{
%%     \usepackage[fontsize=10.000000pt]{scrextend}
%%   }{
%%     \renewcommand{\normalsize}{\fontsize{10.000000}{12.000000}\selectfont}
%%     \normalsize
%%   }
%%   
%%   \makeatletter\@ifpackageloaded{underscore}{}{\usepackage[strings]{underscore}}\makeatother
%%
\begingroup%
\makeatletter%
\begin{pgfpicture}%
\pgfpathrectangle{\pgfpointorigin}{\pgfqpoint{7.135065in}{5.363904in}}%
\pgfusepath{use as bounding box, clip}%
\begin{pgfscope}%
\pgfsetbuttcap%
\pgfsetmiterjoin%
\definecolor{currentfill}{rgb}{1.000000,1.000000,1.000000}%
\pgfsetfillcolor{currentfill}%
\pgfsetlinewidth{0.000000pt}%
\definecolor{currentstroke}{rgb}{0.000000,0.000000,0.000000}%
\pgfsetstrokecolor{currentstroke}%
\pgfsetdash{}{0pt}%
\pgfpathmoveto{\pgfqpoint{0.000000in}{0.000000in}}%
\pgfpathlineto{\pgfqpoint{7.135065in}{0.000000in}}%
\pgfpathlineto{\pgfqpoint{7.135065in}{5.363904in}}%
\pgfpathlineto{\pgfqpoint{0.000000in}{5.363904in}}%
\pgfpathlineto{\pgfqpoint{0.000000in}{0.000000in}}%
\pgfpathclose%
\pgfusepath{fill}%
\end{pgfscope}%
\begin{pgfscope}%
\pgfsetbuttcap%
\pgfsetmiterjoin%
\definecolor{currentfill}{rgb}{1.000000,1.000000,1.000000}%
\pgfsetfillcolor{currentfill}%
\pgfsetlinewidth{0.000000pt}%
\definecolor{currentstroke}{rgb}{0.000000,0.000000,0.000000}%
\pgfsetstrokecolor{currentstroke}%
\pgfsetstrokeopacity{0.000000}%
\pgfsetdash{}{0pt}%
\pgfpathmoveto{\pgfqpoint{0.688192in}{0.643904in}}%
\pgfpathlineto{\pgfqpoint{6.888192in}{0.643904in}}%
\pgfpathlineto{\pgfqpoint{6.888192in}{5.263904in}}%
\pgfpathlineto{\pgfqpoint{0.688192in}{5.263904in}}%
\pgfpathlineto{\pgfqpoint{0.688192in}{0.643904in}}%
\pgfpathclose%
\pgfusepath{fill}%
\end{pgfscope}%
\begin{pgfscope}%
\pgfpathrectangle{\pgfqpoint{0.688192in}{0.643904in}}{\pgfqpoint{6.200000in}{4.620000in}}%
\pgfusepath{clip}%
\pgfsetrectcap%
\pgfsetroundjoin%
\pgfsetlinewidth{0.803000pt}%
\definecolor{currentstroke}{rgb}{0.690196,0.690196,0.690196}%
\pgfsetstrokecolor{currentstroke}%
\pgfsetdash{}{0pt}%
\pgfpathmoveto{\pgfqpoint{1.219620in}{0.643904in}}%
\pgfpathlineto{\pgfqpoint{1.219620in}{5.263904in}}%
\pgfusepath{stroke}%
\end{pgfscope}%
\begin{pgfscope}%
\pgfsetbuttcap%
\pgfsetroundjoin%
\definecolor{currentfill}{rgb}{0.000000,0.000000,0.000000}%
\pgfsetfillcolor{currentfill}%
\pgfsetlinewidth{0.803000pt}%
\definecolor{currentstroke}{rgb}{0.000000,0.000000,0.000000}%
\pgfsetstrokecolor{currentstroke}%
\pgfsetdash{}{0pt}%
\pgfsys@defobject{currentmarker}{\pgfqpoint{0.000000in}{-0.048611in}}{\pgfqpoint{0.000000in}{0.000000in}}{%
\pgfpathmoveto{\pgfqpoint{0.000000in}{0.000000in}}%
\pgfpathlineto{\pgfqpoint{0.000000in}{-0.048611in}}%
\pgfusepath{stroke,fill}%
}%
\begin{pgfscope}%
\pgfsys@transformshift{1.219620in}{0.643904in}%
\pgfsys@useobject{currentmarker}{}%
\end{pgfscope}%
\end{pgfscope}%
\begin{pgfscope}%
\definecolor{textcolor}{rgb}{0.000000,0.000000,0.000000}%
\pgfsetstrokecolor{textcolor}%
\pgfsetfillcolor{textcolor}%
\pgftext[x=1.219620in,y=0.546682in,,top]{\color{textcolor}{\rmfamily\fontsize{14.000000}{16.800000}\selectfont\catcode`\^=\active\def^{\ifmmode\sp\else\^{}\fi}\catcode`\%=\active\def%{\%}$\mathdefault{40}$}}%
\end{pgfscope}%
\begin{pgfscope}%
\pgfpathrectangle{\pgfqpoint{0.688192in}{0.643904in}}{\pgfqpoint{6.200000in}{4.620000in}}%
\pgfusepath{clip}%
\pgfsetrectcap%
\pgfsetroundjoin%
\pgfsetlinewidth{0.803000pt}%
\definecolor{currentstroke}{rgb}{0.690196,0.690196,0.690196}%
\pgfsetstrokecolor{currentstroke}%
\pgfsetdash{}{0pt}%
\pgfpathmoveto{\pgfqpoint{1.928192in}{0.643904in}}%
\pgfpathlineto{\pgfqpoint{1.928192in}{5.263904in}}%
\pgfusepath{stroke}%
\end{pgfscope}%
\begin{pgfscope}%
\pgfsetbuttcap%
\pgfsetroundjoin%
\definecolor{currentfill}{rgb}{0.000000,0.000000,0.000000}%
\pgfsetfillcolor{currentfill}%
\pgfsetlinewidth{0.803000pt}%
\definecolor{currentstroke}{rgb}{0.000000,0.000000,0.000000}%
\pgfsetstrokecolor{currentstroke}%
\pgfsetdash{}{0pt}%
\pgfsys@defobject{currentmarker}{\pgfqpoint{0.000000in}{-0.048611in}}{\pgfqpoint{0.000000in}{0.000000in}}{%
\pgfpathmoveto{\pgfqpoint{0.000000in}{0.000000in}}%
\pgfpathlineto{\pgfqpoint{0.000000in}{-0.048611in}}%
\pgfusepath{stroke,fill}%
}%
\begin{pgfscope}%
\pgfsys@transformshift{1.928192in}{0.643904in}%
\pgfsys@useobject{currentmarker}{}%
\end{pgfscope}%
\end{pgfscope}%
\begin{pgfscope}%
\definecolor{textcolor}{rgb}{0.000000,0.000000,0.000000}%
\pgfsetstrokecolor{textcolor}%
\pgfsetfillcolor{textcolor}%
\pgftext[x=1.928192in,y=0.546682in,,top]{\color{textcolor}{\rmfamily\fontsize{14.000000}{16.800000}\selectfont\catcode`\^=\active\def^{\ifmmode\sp\else\^{}\fi}\catcode`\%=\active\def%{\%}$\mathdefault{60}$}}%
\end{pgfscope}%
\begin{pgfscope}%
\pgfpathrectangle{\pgfqpoint{0.688192in}{0.643904in}}{\pgfqpoint{6.200000in}{4.620000in}}%
\pgfusepath{clip}%
\pgfsetrectcap%
\pgfsetroundjoin%
\pgfsetlinewidth{0.803000pt}%
\definecolor{currentstroke}{rgb}{0.690196,0.690196,0.690196}%
\pgfsetstrokecolor{currentstroke}%
\pgfsetdash{}{0pt}%
\pgfpathmoveto{\pgfqpoint{2.636763in}{0.643904in}}%
\pgfpathlineto{\pgfqpoint{2.636763in}{5.263904in}}%
\pgfusepath{stroke}%
\end{pgfscope}%
\begin{pgfscope}%
\pgfsetbuttcap%
\pgfsetroundjoin%
\definecolor{currentfill}{rgb}{0.000000,0.000000,0.000000}%
\pgfsetfillcolor{currentfill}%
\pgfsetlinewidth{0.803000pt}%
\definecolor{currentstroke}{rgb}{0.000000,0.000000,0.000000}%
\pgfsetstrokecolor{currentstroke}%
\pgfsetdash{}{0pt}%
\pgfsys@defobject{currentmarker}{\pgfqpoint{0.000000in}{-0.048611in}}{\pgfqpoint{0.000000in}{0.000000in}}{%
\pgfpathmoveto{\pgfqpoint{0.000000in}{0.000000in}}%
\pgfpathlineto{\pgfqpoint{0.000000in}{-0.048611in}}%
\pgfusepath{stroke,fill}%
}%
\begin{pgfscope}%
\pgfsys@transformshift{2.636763in}{0.643904in}%
\pgfsys@useobject{currentmarker}{}%
\end{pgfscope}%
\end{pgfscope}%
\begin{pgfscope}%
\definecolor{textcolor}{rgb}{0.000000,0.000000,0.000000}%
\pgfsetstrokecolor{textcolor}%
\pgfsetfillcolor{textcolor}%
\pgftext[x=2.636763in,y=0.546682in,,top]{\color{textcolor}{\rmfamily\fontsize{14.000000}{16.800000}\selectfont\catcode`\^=\active\def^{\ifmmode\sp\else\^{}\fi}\catcode`\%=\active\def%{\%}$\mathdefault{80}$}}%
\end{pgfscope}%
\begin{pgfscope}%
\pgfpathrectangle{\pgfqpoint{0.688192in}{0.643904in}}{\pgfqpoint{6.200000in}{4.620000in}}%
\pgfusepath{clip}%
\pgfsetrectcap%
\pgfsetroundjoin%
\pgfsetlinewidth{0.803000pt}%
\definecolor{currentstroke}{rgb}{0.690196,0.690196,0.690196}%
\pgfsetstrokecolor{currentstroke}%
\pgfsetdash{}{0pt}%
\pgfpathmoveto{\pgfqpoint{3.345334in}{0.643904in}}%
\pgfpathlineto{\pgfqpoint{3.345334in}{5.263904in}}%
\pgfusepath{stroke}%
\end{pgfscope}%
\begin{pgfscope}%
\pgfsetbuttcap%
\pgfsetroundjoin%
\definecolor{currentfill}{rgb}{0.000000,0.000000,0.000000}%
\pgfsetfillcolor{currentfill}%
\pgfsetlinewidth{0.803000pt}%
\definecolor{currentstroke}{rgb}{0.000000,0.000000,0.000000}%
\pgfsetstrokecolor{currentstroke}%
\pgfsetdash{}{0pt}%
\pgfsys@defobject{currentmarker}{\pgfqpoint{0.000000in}{-0.048611in}}{\pgfqpoint{0.000000in}{0.000000in}}{%
\pgfpathmoveto{\pgfqpoint{0.000000in}{0.000000in}}%
\pgfpathlineto{\pgfqpoint{0.000000in}{-0.048611in}}%
\pgfusepath{stroke,fill}%
}%
\begin{pgfscope}%
\pgfsys@transformshift{3.345334in}{0.643904in}%
\pgfsys@useobject{currentmarker}{}%
\end{pgfscope}%
\end{pgfscope}%
\begin{pgfscope}%
\definecolor{textcolor}{rgb}{0.000000,0.000000,0.000000}%
\pgfsetstrokecolor{textcolor}%
\pgfsetfillcolor{textcolor}%
\pgftext[x=3.345334in,y=0.546682in,,top]{\color{textcolor}{\rmfamily\fontsize{14.000000}{16.800000}\selectfont\catcode`\^=\active\def^{\ifmmode\sp\else\^{}\fi}\catcode`\%=\active\def%{\%}$\mathdefault{100}$}}%
\end{pgfscope}%
\begin{pgfscope}%
\pgfpathrectangle{\pgfqpoint{0.688192in}{0.643904in}}{\pgfqpoint{6.200000in}{4.620000in}}%
\pgfusepath{clip}%
\pgfsetrectcap%
\pgfsetroundjoin%
\pgfsetlinewidth{0.803000pt}%
\definecolor{currentstroke}{rgb}{0.690196,0.690196,0.690196}%
\pgfsetstrokecolor{currentstroke}%
\pgfsetdash{}{0pt}%
\pgfpathmoveto{\pgfqpoint{4.053906in}{0.643904in}}%
\pgfpathlineto{\pgfqpoint{4.053906in}{5.263904in}}%
\pgfusepath{stroke}%
\end{pgfscope}%
\begin{pgfscope}%
\pgfsetbuttcap%
\pgfsetroundjoin%
\definecolor{currentfill}{rgb}{0.000000,0.000000,0.000000}%
\pgfsetfillcolor{currentfill}%
\pgfsetlinewidth{0.803000pt}%
\definecolor{currentstroke}{rgb}{0.000000,0.000000,0.000000}%
\pgfsetstrokecolor{currentstroke}%
\pgfsetdash{}{0pt}%
\pgfsys@defobject{currentmarker}{\pgfqpoint{0.000000in}{-0.048611in}}{\pgfqpoint{0.000000in}{0.000000in}}{%
\pgfpathmoveto{\pgfqpoint{0.000000in}{0.000000in}}%
\pgfpathlineto{\pgfqpoint{0.000000in}{-0.048611in}}%
\pgfusepath{stroke,fill}%
}%
\begin{pgfscope}%
\pgfsys@transformshift{4.053906in}{0.643904in}%
\pgfsys@useobject{currentmarker}{}%
\end{pgfscope}%
\end{pgfscope}%
\begin{pgfscope}%
\definecolor{textcolor}{rgb}{0.000000,0.000000,0.000000}%
\pgfsetstrokecolor{textcolor}%
\pgfsetfillcolor{textcolor}%
\pgftext[x=4.053906in,y=0.546682in,,top]{\color{textcolor}{\rmfamily\fontsize{14.000000}{16.800000}\selectfont\catcode`\^=\active\def^{\ifmmode\sp\else\^{}\fi}\catcode`\%=\active\def%{\%}$\mathdefault{120}$}}%
\end{pgfscope}%
\begin{pgfscope}%
\pgfpathrectangle{\pgfqpoint{0.688192in}{0.643904in}}{\pgfqpoint{6.200000in}{4.620000in}}%
\pgfusepath{clip}%
\pgfsetrectcap%
\pgfsetroundjoin%
\pgfsetlinewidth{0.803000pt}%
\definecolor{currentstroke}{rgb}{0.690196,0.690196,0.690196}%
\pgfsetstrokecolor{currentstroke}%
\pgfsetdash{}{0pt}%
\pgfpathmoveto{\pgfqpoint{4.762477in}{0.643904in}}%
\pgfpathlineto{\pgfqpoint{4.762477in}{5.263904in}}%
\pgfusepath{stroke}%
\end{pgfscope}%
\begin{pgfscope}%
\pgfsetbuttcap%
\pgfsetroundjoin%
\definecolor{currentfill}{rgb}{0.000000,0.000000,0.000000}%
\pgfsetfillcolor{currentfill}%
\pgfsetlinewidth{0.803000pt}%
\definecolor{currentstroke}{rgb}{0.000000,0.000000,0.000000}%
\pgfsetstrokecolor{currentstroke}%
\pgfsetdash{}{0pt}%
\pgfsys@defobject{currentmarker}{\pgfqpoint{0.000000in}{-0.048611in}}{\pgfqpoint{0.000000in}{0.000000in}}{%
\pgfpathmoveto{\pgfqpoint{0.000000in}{0.000000in}}%
\pgfpathlineto{\pgfqpoint{0.000000in}{-0.048611in}}%
\pgfusepath{stroke,fill}%
}%
\begin{pgfscope}%
\pgfsys@transformshift{4.762477in}{0.643904in}%
\pgfsys@useobject{currentmarker}{}%
\end{pgfscope}%
\end{pgfscope}%
\begin{pgfscope}%
\definecolor{textcolor}{rgb}{0.000000,0.000000,0.000000}%
\pgfsetstrokecolor{textcolor}%
\pgfsetfillcolor{textcolor}%
\pgftext[x=4.762477in,y=0.546682in,,top]{\color{textcolor}{\rmfamily\fontsize{14.000000}{16.800000}\selectfont\catcode`\^=\active\def^{\ifmmode\sp\else\^{}\fi}\catcode`\%=\active\def%{\%}$\mathdefault{140}$}}%
\end{pgfscope}%
\begin{pgfscope}%
\pgfpathrectangle{\pgfqpoint{0.688192in}{0.643904in}}{\pgfqpoint{6.200000in}{4.620000in}}%
\pgfusepath{clip}%
\pgfsetrectcap%
\pgfsetroundjoin%
\pgfsetlinewidth{0.803000pt}%
\definecolor{currentstroke}{rgb}{0.690196,0.690196,0.690196}%
\pgfsetstrokecolor{currentstroke}%
\pgfsetdash{}{0pt}%
\pgfpathmoveto{\pgfqpoint{5.471049in}{0.643904in}}%
\pgfpathlineto{\pgfqpoint{5.471049in}{5.263904in}}%
\pgfusepath{stroke}%
\end{pgfscope}%
\begin{pgfscope}%
\pgfsetbuttcap%
\pgfsetroundjoin%
\definecolor{currentfill}{rgb}{0.000000,0.000000,0.000000}%
\pgfsetfillcolor{currentfill}%
\pgfsetlinewidth{0.803000pt}%
\definecolor{currentstroke}{rgb}{0.000000,0.000000,0.000000}%
\pgfsetstrokecolor{currentstroke}%
\pgfsetdash{}{0pt}%
\pgfsys@defobject{currentmarker}{\pgfqpoint{0.000000in}{-0.048611in}}{\pgfqpoint{0.000000in}{0.000000in}}{%
\pgfpathmoveto{\pgfqpoint{0.000000in}{0.000000in}}%
\pgfpathlineto{\pgfqpoint{0.000000in}{-0.048611in}}%
\pgfusepath{stroke,fill}%
}%
\begin{pgfscope}%
\pgfsys@transformshift{5.471049in}{0.643904in}%
\pgfsys@useobject{currentmarker}{}%
\end{pgfscope}%
\end{pgfscope}%
\begin{pgfscope}%
\definecolor{textcolor}{rgb}{0.000000,0.000000,0.000000}%
\pgfsetstrokecolor{textcolor}%
\pgfsetfillcolor{textcolor}%
\pgftext[x=5.471049in,y=0.546682in,,top]{\color{textcolor}{\rmfamily\fontsize{14.000000}{16.800000}\selectfont\catcode`\^=\active\def^{\ifmmode\sp\else\^{}\fi}\catcode`\%=\active\def%{\%}$\mathdefault{160}$}}%
\end{pgfscope}%
\begin{pgfscope}%
\pgfpathrectangle{\pgfqpoint{0.688192in}{0.643904in}}{\pgfqpoint{6.200000in}{4.620000in}}%
\pgfusepath{clip}%
\pgfsetrectcap%
\pgfsetroundjoin%
\pgfsetlinewidth{0.803000pt}%
\definecolor{currentstroke}{rgb}{0.690196,0.690196,0.690196}%
\pgfsetstrokecolor{currentstroke}%
\pgfsetdash{}{0pt}%
\pgfpathmoveto{\pgfqpoint{6.179620in}{0.643904in}}%
\pgfpathlineto{\pgfqpoint{6.179620in}{5.263904in}}%
\pgfusepath{stroke}%
\end{pgfscope}%
\begin{pgfscope}%
\pgfsetbuttcap%
\pgfsetroundjoin%
\definecolor{currentfill}{rgb}{0.000000,0.000000,0.000000}%
\pgfsetfillcolor{currentfill}%
\pgfsetlinewidth{0.803000pt}%
\definecolor{currentstroke}{rgb}{0.000000,0.000000,0.000000}%
\pgfsetstrokecolor{currentstroke}%
\pgfsetdash{}{0pt}%
\pgfsys@defobject{currentmarker}{\pgfqpoint{0.000000in}{-0.048611in}}{\pgfqpoint{0.000000in}{0.000000in}}{%
\pgfpathmoveto{\pgfqpoint{0.000000in}{0.000000in}}%
\pgfpathlineto{\pgfqpoint{0.000000in}{-0.048611in}}%
\pgfusepath{stroke,fill}%
}%
\begin{pgfscope}%
\pgfsys@transformshift{6.179620in}{0.643904in}%
\pgfsys@useobject{currentmarker}{}%
\end{pgfscope}%
\end{pgfscope}%
\begin{pgfscope}%
\definecolor{textcolor}{rgb}{0.000000,0.000000,0.000000}%
\pgfsetstrokecolor{textcolor}%
\pgfsetfillcolor{textcolor}%
\pgftext[x=6.179620in,y=0.546682in,,top]{\color{textcolor}{\rmfamily\fontsize{14.000000}{16.800000}\selectfont\catcode`\^=\active\def^{\ifmmode\sp\else\^{}\fi}\catcode`\%=\active\def%{\%}$\mathdefault{180}$}}%
\end{pgfscope}%
\begin{pgfscope}%
\pgfpathrectangle{\pgfqpoint{0.688192in}{0.643904in}}{\pgfqpoint{6.200000in}{4.620000in}}%
\pgfusepath{clip}%
\pgfsetrectcap%
\pgfsetroundjoin%
\pgfsetlinewidth{0.803000pt}%
\definecolor{currentstroke}{rgb}{0.690196,0.690196,0.690196}%
\pgfsetstrokecolor{currentstroke}%
\pgfsetdash{}{0pt}%
\pgfpathmoveto{\pgfqpoint{6.888192in}{0.643904in}}%
\pgfpathlineto{\pgfqpoint{6.888192in}{5.263904in}}%
\pgfusepath{stroke}%
\end{pgfscope}%
\begin{pgfscope}%
\pgfsetbuttcap%
\pgfsetroundjoin%
\definecolor{currentfill}{rgb}{0.000000,0.000000,0.000000}%
\pgfsetfillcolor{currentfill}%
\pgfsetlinewidth{0.803000pt}%
\definecolor{currentstroke}{rgb}{0.000000,0.000000,0.000000}%
\pgfsetstrokecolor{currentstroke}%
\pgfsetdash{}{0pt}%
\pgfsys@defobject{currentmarker}{\pgfqpoint{0.000000in}{-0.048611in}}{\pgfqpoint{0.000000in}{0.000000in}}{%
\pgfpathmoveto{\pgfqpoint{0.000000in}{0.000000in}}%
\pgfpathlineto{\pgfqpoint{0.000000in}{-0.048611in}}%
\pgfusepath{stroke,fill}%
}%
\begin{pgfscope}%
\pgfsys@transformshift{6.888192in}{0.643904in}%
\pgfsys@useobject{currentmarker}{}%
\end{pgfscope}%
\end{pgfscope}%
\begin{pgfscope}%
\definecolor{textcolor}{rgb}{0.000000,0.000000,0.000000}%
\pgfsetstrokecolor{textcolor}%
\pgfsetfillcolor{textcolor}%
\pgftext[x=6.888192in,y=0.546682in,,top]{\color{textcolor}{\rmfamily\fontsize{14.000000}{16.800000}\selectfont\catcode`\^=\active\def^{\ifmmode\sp\else\^{}\fi}\catcode`\%=\active\def%{\%}$\mathdefault{200}$}}%
\end{pgfscope}%
\begin{pgfscope}%
\pgfpathrectangle{\pgfqpoint{0.688192in}{0.643904in}}{\pgfqpoint{6.200000in}{4.620000in}}%
\pgfusepath{clip}%
\pgfsetbuttcap%
\pgfsetroundjoin%
\pgfsetlinewidth{0.803000pt}%
\definecolor{currentstroke}{rgb}{0.690196,0.690196,0.690196}%
\pgfsetstrokecolor{currentstroke}%
\pgfsetstrokeopacity{0.200000}%
\pgfsetdash{{2.960000pt}{1.280000pt}}{0.000000pt}%
\pgfpathmoveto{\pgfqpoint{0.688192in}{0.643904in}}%
\pgfpathlineto{\pgfqpoint{0.688192in}{5.263904in}}%
\pgfusepath{stroke}%
\end{pgfscope}%
\begin{pgfscope}%
\pgfsetbuttcap%
\pgfsetroundjoin%
\definecolor{currentfill}{rgb}{0.000000,0.000000,0.000000}%
\pgfsetfillcolor{currentfill}%
\pgfsetlinewidth{0.602250pt}%
\definecolor{currentstroke}{rgb}{0.000000,0.000000,0.000000}%
\pgfsetstrokecolor{currentstroke}%
\pgfsetdash{}{0pt}%
\pgfsys@defobject{currentmarker}{\pgfqpoint{0.000000in}{-0.027778in}}{\pgfqpoint{0.000000in}{0.000000in}}{%
\pgfpathmoveto{\pgfqpoint{0.000000in}{0.000000in}}%
\pgfpathlineto{\pgfqpoint{0.000000in}{-0.027778in}}%
\pgfusepath{stroke,fill}%
}%
\begin{pgfscope}%
\pgfsys@transformshift{0.688192in}{0.643904in}%
\pgfsys@useobject{currentmarker}{}%
\end{pgfscope}%
\end{pgfscope}%
\begin{pgfscope}%
\pgfpathrectangle{\pgfqpoint{0.688192in}{0.643904in}}{\pgfqpoint{6.200000in}{4.620000in}}%
\pgfusepath{clip}%
\pgfsetbuttcap%
\pgfsetroundjoin%
\pgfsetlinewidth{0.803000pt}%
\definecolor{currentstroke}{rgb}{0.690196,0.690196,0.690196}%
\pgfsetstrokecolor{currentstroke}%
\pgfsetstrokeopacity{0.200000}%
\pgfsetdash{{2.960000pt}{1.280000pt}}{0.000000pt}%
\pgfpathmoveto{\pgfqpoint{0.865334in}{0.643904in}}%
\pgfpathlineto{\pgfqpoint{0.865334in}{5.263904in}}%
\pgfusepath{stroke}%
\end{pgfscope}%
\begin{pgfscope}%
\pgfsetbuttcap%
\pgfsetroundjoin%
\definecolor{currentfill}{rgb}{0.000000,0.000000,0.000000}%
\pgfsetfillcolor{currentfill}%
\pgfsetlinewidth{0.602250pt}%
\definecolor{currentstroke}{rgb}{0.000000,0.000000,0.000000}%
\pgfsetstrokecolor{currentstroke}%
\pgfsetdash{}{0pt}%
\pgfsys@defobject{currentmarker}{\pgfqpoint{0.000000in}{-0.027778in}}{\pgfqpoint{0.000000in}{0.000000in}}{%
\pgfpathmoveto{\pgfqpoint{0.000000in}{0.000000in}}%
\pgfpathlineto{\pgfqpoint{0.000000in}{-0.027778in}}%
\pgfusepath{stroke,fill}%
}%
\begin{pgfscope}%
\pgfsys@transformshift{0.865334in}{0.643904in}%
\pgfsys@useobject{currentmarker}{}%
\end{pgfscope}%
\end{pgfscope}%
\begin{pgfscope}%
\pgfpathrectangle{\pgfqpoint{0.688192in}{0.643904in}}{\pgfqpoint{6.200000in}{4.620000in}}%
\pgfusepath{clip}%
\pgfsetbuttcap%
\pgfsetroundjoin%
\pgfsetlinewidth{0.803000pt}%
\definecolor{currentstroke}{rgb}{0.690196,0.690196,0.690196}%
\pgfsetstrokecolor{currentstroke}%
\pgfsetstrokeopacity{0.200000}%
\pgfsetdash{{2.960000pt}{1.280000pt}}{0.000000pt}%
\pgfpathmoveto{\pgfqpoint{1.042477in}{0.643904in}}%
\pgfpathlineto{\pgfqpoint{1.042477in}{5.263904in}}%
\pgfusepath{stroke}%
\end{pgfscope}%
\begin{pgfscope}%
\pgfsetbuttcap%
\pgfsetroundjoin%
\definecolor{currentfill}{rgb}{0.000000,0.000000,0.000000}%
\pgfsetfillcolor{currentfill}%
\pgfsetlinewidth{0.602250pt}%
\definecolor{currentstroke}{rgb}{0.000000,0.000000,0.000000}%
\pgfsetstrokecolor{currentstroke}%
\pgfsetdash{}{0pt}%
\pgfsys@defobject{currentmarker}{\pgfqpoint{0.000000in}{-0.027778in}}{\pgfqpoint{0.000000in}{0.000000in}}{%
\pgfpathmoveto{\pgfqpoint{0.000000in}{0.000000in}}%
\pgfpathlineto{\pgfqpoint{0.000000in}{-0.027778in}}%
\pgfusepath{stroke,fill}%
}%
\begin{pgfscope}%
\pgfsys@transformshift{1.042477in}{0.643904in}%
\pgfsys@useobject{currentmarker}{}%
\end{pgfscope}%
\end{pgfscope}%
\begin{pgfscope}%
\pgfpathrectangle{\pgfqpoint{0.688192in}{0.643904in}}{\pgfqpoint{6.200000in}{4.620000in}}%
\pgfusepath{clip}%
\pgfsetbuttcap%
\pgfsetroundjoin%
\pgfsetlinewidth{0.803000pt}%
\definecolor{currentstroke}{rgb}{0.690196,0.690196,0.690196}%
\pgfsetstrokecolor{currentstroke}%
\pgfsetstrokeopacity{0.200000}%
\pgfsetdash{{2.960000pt}{1.280000pt}}{0.000000pt}%
\pgfpathmoveto{\pgfqpoint{1.396763in}{0.643904in}}%
\pgfpathlineto{\pgfqpoint{1.396763in}{5.263904in}}%
\pgfusepath{stroke}%
\end{pgfscope}%
\begin{pgfscope}%
\pgfsetbuttcap%
\pgfsetroundjoin%
\definecolor{currentfill}{rgb}{0.000000,0.000000,0.000000}%
\pgfsetfillcolor{currentfill}%
\pgfsetlinewidth{0.602250pt}%
\definecolor{currentstroke}{rgb}{0.000000,0.000000,0.000000}%
\pgfsetstrokecolor{currentstroke}%
\pgfsetdash{}{0pt}%
\pgfsys@defobject{currentmarker}{\pgfqpoint{0.000000in}{-0.027778in}}{\pgfqpoint{0.000000in}{0.000000in}}{%
\pgfpathmoveto{\pgfqpoint{0.000000in}{0.000000in}}%
\pgfpathlineto{\pgfqpoint{0.000000in}{-0.027778in}}%
\pgfusepath{stroke,fill}%
}%
\begin{pgfscope}%
\pgfsys@transformshift{1.396763in}{0.643904in}%
\pgfsys@useobject{currentmarker}{}%
\end{pgfscope}%
\end{pgfscope}%
\begin{pgfscope}%
\pgfpathrectangle{\pgfqpoint{0.688192in}{0.643904in}}{\pgfqpoint{6.200000in}{4.620000in}}%
\pgfusepath{clip}%
\pgfsetbuttcap%
\pgfsetroundjoin%
\pgfsetlinewidth{0.803000pt}%
\definecolor{currentstroke}{rgb}{0.690196,0.690196,0.690196}%
\pgfsetstrokecolor{currentstroke}%
\pgfsetstrokeopacity{0.200000}%
\pgfsetdash{{2.960000pt}{1.280000pt}}{0.000000pt}%
\pgfpathmoveto{\pgfqpoint{1.573906in}{0.643904in}}%
\pgfpathlineto{\pgfqpoint{1.573906in}{5.263904in}}%
\pgfusepath{stroke}%
\end{pgfscope}%
\begin{pgfscope}%
\pgfsetbuttcap%
\pgfsetroundjoin%
\definecolor{currentfill}{rgb}{0.000000,0.000000,0.000000}%
\pgfsetfillcolor{currentfill}%
\pgfsetlinewidth{0.602250pt}%
\definecolor{currentstroke}{rgb}{0.000000,0.000000,0.000000}%
\pgfsetstrokecolor{currentstroke}%
\pgfsetdash{}{0pt}%
\pgfsys@defobject{currentmarker}{\pgfqpoint{0.000000in}{-0.027778in}}{\pgfqpoint{0.000000in}{0.000000in}}{%
\pgfpathmoveto{\pgfqpoint{0.000000in}{0.000000in}}%
\pgfpathlineto{\pgfqpoint{0.000000in}{-0.027778in}}%
\pgfusepath{stroke,fill}%
}%
\begin{pgfscope}%
\pgfsys@transformshift{1.573906in}{0.643904in}%
\pgfsys@useobject{currentmarker}{}%
\end{pgfscope}%
\end{pgfscope}%
\begin{pgfscope}%
\pgfpathrectangle{\pgfqpoint{0.688192in}{0.643904in}}{\pgfqpoint{6.200000in}{4.620000in}}%
\pgfusepath{clip}%
\pgfsetbuttcap%
\pgfsetroundjoin%
\pgfsetlinewidth{0.803000pt}%
\definecolor{currentstroke}{rgb}{0.690196,0.690196,0.690196}%
\pgfsetstrokecolor{currentstroke}%
\pgfsetstrokeopacity{0.200000}%
\pgfsetdash{{2.960000pt}{1.280000pt}}{0.000000pt}%
\pgfpathmoveto{\pgfqpoint{1.751049in}{0.643904in}}%
\pgfpathlineto{\pgfqpoint{1.751049in}{5.263904in}}%
\pgfusepath{stroke}%
\end{pgfscope}%
\begin{pgfscope}%
\pgfsetbuttcap%
\pgfsetroundjoin%
\definecolor{currentfill}{rgb}{0.000000,0.000000,0.000000}%
\pgfsetfillcolor{currentfill}%
\pgfsetlinewidth{0.602250pt}%
\definecolor{currentstroke}{rgb}{0.000000,0.000000,0.000000}%
\pgfsetstrokecolor{currentstroke}%
\pgfsetdash{}{0pt}%
\pgfsys@defobject{currentmarker}{\pgfqpoint{0.000000in}{-0.027778in}}{\pgfqpoint{0.000000in}{0.000000in}}{%
\pgfpathmoveto{\pgfqpoint{0.000000in}{0.000000in}}%
\pgfpathlineto{\pgfqpoint{0.000000in}{-0.027778in}}%
\pgfusepath{stroke,fill}%
}%
\begin{pgfscope}%
\pgfsys@transformshift{1.751049in}{0.643904in}%
\pgfsys@useobject{currentmarker}{}%
\end{pgfscope}%
\end{pgfscope}%
\begin{pgfscope}%
\pgfpathrectangle{\pgfqpoint{0.688192in}{0.643904in}}{\pgfqpoint{6.200000in}{4.620000in}}%
\pgfusepath{clip}%
\pgfsetbuttcap%
\pgfsetroundjoin%
\pgfsetlinewidth{0.803000pt}%
\definecolor{currentstroke}{rgb}{0.690196,0.690196,0.690196}%
\pgfsetstrokecolor{currentstroke}%
\pgfsetstrokeopacity{0.200000}%
\pgfsetdash{{2.960000pt}{1.280000pt}}{0.000000pt}%
\pgfpathmoveto{\pgfqpoint{2.105334in}{0.643904in}}%
\pgfpathlineto{\pgfqpoint{2.105334in}{5.263904in}}%
\pgfusepath{stroke}%
\end{pgfscope}%
\begin{pgfscope}%
\pgfsetbuttcap%
\pgfsetroundjoin%
\definecolor{currentfill}{rgb}{0.000000,0.000000,0.000000}%
\pgfsetfillcolor{currentfill}%
\pgfsetlinewidth{0.602250pt}%
\definecolor{currentstroke}{rgb}{0.000000,0.000000,0.000000}%
\pgfsetstrokecolor{currentstroke}%
\pgfsetdash{}{0pt}%
\pgfsys@defobject{currentmarker}{\pgfqpoint{0.000000in}{-0.027778in}}{\pgfqpoint{0.000000in}{0.000000in}}{%
\pgfpathmoveto{\pgfqpoint{0.000000in}{0.000000in}}%
\pgfpathlineto{\pgfqpoint{0.000000in}{-0.027778in}}%
\pgfusepath{stroke,fill}%
}%
\begin{pgfscope}%
\pgfsys@transformshift{2.105334in}{0.643904in}%
\pgfsys@useobject{currentmarker}{}%
\end{pgfscope}%
\end{pgfscope}%
\begin{pgfscope}%
\pgfpathrectangle{\pgfqpoint{0.688192in}{0.643904in}}{\pgfqpoint{6.200000in}{4.620000in}}%
\pgfusepath{clip}%
\pgfsetbuttcap%
\pgfsetroundjoin%
\pgfsetlinewidth{0.803000pt}%
\definecolor{currentstroke}{rgb}{0.690196,0.690196,0.690196}%
\pgfsetstrokecolor{currentstroke}%
\pgfsetstrokeopacity{0.200000}%
\pgfsetdash{{2.960000pt}{1.280000pt}}{0.000000pt}%
\pgfpathmoveto{\pgfqpoint{2.282477in}{0.643904in}}%
\pgfpathlineto{\pgfqpoint{2.282477in}{5.263904in}}%
\pgfusepath{stroke}%
\end{pgfscope}%
\begin{pgfscope}%
\pgfsetbuttcap%
\pgfsetroundjoin%
\definecolor{currentfill}{rgb}{0.000000,0.000000,0.000000}%
\pgfsetfillcolor{currentfill}%
\pgfsetlinewidth{0.602250pt}%
\definecolor{currentstroke}{rgb}{0.000000,0.000000,0.000000}%
\pgfsetstrokecolor{currentstroke}%
\pgfsetdash{}{0pt}%
\pgfsys@defobject{currentmarker}{\pgfqpoint{0.000000in}{-0.027778in}}{\pgfqpoint{0.000000in}{0.000000in}}{%
\pgfpathmoveto{\pgfqpoint{0.000000in}{0.000000in}}%
\pgfpathlineto{\pgfqpoint{0.000000in}{-0.027778in}}%
\pgfusepath{stroke,fill}%
}%
\begin{pgfscope}%
\pgfsys@transformshift{2.282477in}{0.643904in}%
\pgfsys@useobject{currentmarker}{}%
\end{pgfscope}%
\end{pgfscope}%
\begin{pgfscope}%
\pgfpathrectangle{\pgfqpoint{0.688192in}{0.643904in}}{\pgfqpoint{6.200000in}{4.620000in}}%
\pgfusepath{clip}%
\pgfsetbuttcap%
\pgfsetroundjoin%
\pgfsetlinewidth{0.803000pt}%
\definecolor{currentstroke}{rgb}{0.690196,0.690196,0.690196}%
\pgfsetstrokecolor{currentstroke}%
\pgfsetstrokeopacity{0.200000}%
\pgfsetdash{{2.960000pt}{1.280000pt}}{0.000000pt}%
\pgfpathmoveto{\pgfqpoint{2.459620in}{0.643904in}}%
\pgfpathlineto{\pgfqpoint{2.459620in}{5.263904in}}%
\pgfusepath{stroke}%
\end{pgfscope}%
\begin{pgfscope}%
\pgfsetbuttcap%
\pgfsetroundjoin%
\definecolor{currentfill}{rgb}{0.000000,0.000000,0.000000}%
\pgfsetfillcolor{currentfill}%
\pgfsetlinewidth{0.602250pt}%
\definecolor{currentstroke}{rgb}{0.000000,0.000000,0.000000}%
\pgfsetstrokecolor{currentstroke}%
\pgfsetdash{}{0pt}%
\pgfsys@defobject{currentmarker}{\pgfqpoint{0.000000in}{-0.027778in}}{\pgfqpoint{0.000000in}{0.000000in}}{%
\pgfpathmoveto{\pgfqpoint{0.000000in}{0.000000in}}%
\pgfpathlineto{\pgfqpoint{0.000000in}{-0.027778in}}%
\pgfusepath{stroke,fill}%
}%
\begin{pgfscope}%
\pgfsys@transformshift{2.459620in}{0.643904in}%
\pgfsys@useobject{currentmarker}{}%
\end{pgfscope}%
\end{pgfscope}%
\begin{pgfscope}%
\pgfpathrectangle{\pgfqpoint{0.688192in}{0.643904in}}{\pgfqpoint{6.200000in}{4.620000in}}%
\pgfusepath{clip}%
\pgfsetbuttcap%
\pgfsetroundjoin%
\pgfsetlinewidth{0.803000pt}%
\definecolor{currentstroke}{rgb}{0.690196,0.690196,0.690196}%
\pgfsetstrokecolor{currentstroke}%
\pgfsetstrokeopacity{0.200000}%
\pgfsetdash{{2.960000pt}{1.280000pt}}{0.000000pt}%
\pgfpathmoveto{\pgfqpoint{2.813906in}{0.643904in}}%
\pgfpathlineto{\pgfqpoint{2.813906in}{5.263904in}}%
\pgfusepath{stroke}%
\end{pgfscope}%
\begin{pgfscope}%
\pgfsetbuttcap%
\pgfsetroundjoin%
\definecolor{currentfill}{rgb}{0.000000,0.000000,0.000000}%
\pgfsetfillcolor{currentfill}%
\pgfsetlinewidth{0.602250pt}%
\definecolor{currentstroke}{rgb}{0.000000,0.000000,0.000000}%
\pgfsetstrokecolor{currentstroke}%
\pgfsetdash{}{0pt}%
\pgfsys@defobject{currentmarker}{\pgfqpoint{0.000000in}{-0.027778in}}{\pgfqpoint{0.000000in}{0.000000in}}{%
\pgfpathmoveto{\pgfqpoint{0.000000in}{0.000000in}}%
\pgfpathlineto{\pgfqpoint{0.000000in}{-0.027778in}}%
\pgfusepath{stroke,fill}%
}%
\begin{pgfscope}%
\pgfsys@transformshift{2.813906in}{0.643904in}%
\pgfsys@useobject{currentmarker}{}%
\end{pgfscope}%
\end{pgfscope}%
\begin{pgfscope}%
\pgfpathrectangle{\pgfqpoint{0.688192in}{0.643904in}}{\pgfqpoint{6.200000in}{4.620000in}}%
\pgfusepath{clip}%
\pgfsetbuttcap%
\pgfsetroundjoin%
\pgfsetlinewidth{0.803000pt}%
\definecolor{currentstroke}{rgb}{0.690196,0.690196,0.690196}%
\pgfsetstrokecolor{currentstroke}%
\pgfsetstrokeopacity{0.200000}%
\pgfsetdash{{2.960000pt}{1.280000pt}}{0.000000pt}%
\pgfpathmoveto{\pgfqpoint{2.991049in}{0.643904in}}%
\pgfpathlineto{\pgfqpoint{2.991049in}{5.263904in}}%
\pgfusepath{stroke}%
\end{pgfscope}%
\begin{pgfscope}%
\pgfsetbuttcap%
\pgfsetroundjoin%
\definecolor{currentfill}{rgb}{0.000000,0.000000,0.000000}%
\pgfsetfillcolor{currentfill}%
\pgfsetlinewidth{0.602250pt}%
\definecolor{currentstroke}{rgb}{0.000000,0.000000,0.000000}%
\pgfsetstrokecolor{currentstroke}%
\pgfsetdash{}{0pt}%
\pgfsys@defobject{currentmarker}{\pgfqpoint{0.000000in}{-0.027778in}}{\pgfqpoint{0.000000in}{0.000000in}}{%
\pgfpathmoveto{\pgfqpoint{0.000000in}{0.000000in}}%
\pgfpathlineto{\pgfqpoint{0.000000in}{-0.027778in}}%
\pgfusepath{stroke,fill}%
}%
\begin{pgfscope}%
\pgfsys@transformshift{2.991049in}{0.643904in}%
\pgfsys@useobject{currentmarker}{}%
\end{pgfscope}%
\end{pgfscope}%
\begin{pgfscope}%
\pgfpathrectangle{\pgfqpoint{0.688192in}{0.643904in}}{\pgfqpoint{6.200000in}{4.620000in}}%
\pgfusepath{clip}%
\pgfsetbuttcap%
\pgfsetroundjoin%
\pgfsetlinewidth{0.803000pt}%
\definecolor{currentstroke}{rgb}{0.690196,0.690196,0.690196}%
\pgfsetstrokecolor{currentstroke}%
\pgfsetstrokeopacity{0.200000}%
\pgfsetdash{{2.960000pt}{1.280000pt}}{0.000000pt}%
\pgfpathmoveto{\pgfqpoint{3.168192in}{0.643904in}}%
\pgfpathlineto{\pgfqpoint{3.168192in}{5.263904in}}%
\pgfusepath{stroke}%
\end{pgfscope}%
\begin{pgfscope}%
\pgfsetbuttcap%
\pgfsetroundjoin%
\definecolor{currentfill}{rgb}{0.000000,0.000000,0.000000}%
\pgfsetfillcolor{currentfill}%
\pgfsetlinewidth{0.602250pt}%
\definecolor{currentstroke}{rgb}{0.000000,0.000000,0.000000}%
\pgfsetstrokecolor{currentstroke}%
\pgfsetdash{}{0pt}%
\pgfsys@defobject{currentmarker}{\pgfqpoint{0.000000in}{-0.027778in}}{\pgfqpoint{0.000000in}{0.000000in}}{%
\pgfpathmoveto{\pgfqpoint{0.000000in}{0.000000in}}%
\pgfpathlineto{\pgfqpoint{0.000000in}{-0.027778in}}%
\pgfusepath{stroke,fill}%
}%
\begin{pgfscope}%
\pgfsys@transformshift{3.168192in}{0.643904in}%
\pgfsys@useobject{currentmarker}{}%
\end{pgfscope}%
\end{pgfscope}%
\begin{pgfscope}%
\pgfpathrectangle{\pgfqpoint{0.688192in}{0.643904in}}{\pgfqpoint{6.200000in}{4.620000in}}%
\pgfusepath{clip}%
\pgfsetbuttcap%
\pgfsetroundjoin%
\pgfsetlinewidth{0.803000pt}%
\definecolor{currentstroke}{rgb}{0.690196,0.690196,0.690196}%
\pgfsetstrokecolor{currentstroke}%
\pgfsetstrokeopacity{0.200000}%
\pgfsetdash{{2.960000pt}{1.280000pt}}{0.000000pt}%
\pgfpathmoveto{\pgfqpoint{3.522477in}{0.643904in}}%
\pgfpathlineto{\pgfqpoint{3.522477in}{5.263904in}}%
\pgfusepath{stroke}%
\end{pgfscope}%
\begin{pgfscope}%
\pgfsetbuttcap%
\pgfsetroundjoin%
\definecolor{currentfill}{rgb}{0.000000,0.000000,0.000000}%
\pgfsetfillcolor{currentfill}%
\pgfsetlinewidth{0.602250pt}%
\definecolor{currentstroke}{rgb}{0.000000,0.000000,0.000000}%
\pgfsetstrokecolor{currentstroke}%
\pgfsetdash{}{0pt}%
\pgfsys@defobject{currentmarker}{\pgfqpoint{0.000000in}{-0.027778in}}{\pgfqpoint{0.000000in}{0.000000in}}{%
\pgfpathmoveto{\pgfqpoint{0.000000in}{0.000000in}}%
\pgfpathlineto{\pgfqpoint{0.000000in}{-0.027778in}}%
\pgfusepath{stroke,fill}%
}%
\begin{pgfscope}%
\pgfsys@transformshift{3.522477in}{0.643904in}%
\pgfsys@useobject{currentmarker}{}%
\end{pgfscope}%
\end{pgfscope}%
\begin{pgfscope}%
\pgfpathrectangle{\pgfqpoint{0.688192in}{0.643904in}}{\pgfqpoint{6.200000in}{4.620000in}}%
\pgfusepath{clip}%
\pgfsetbuttcap%
\pgfsetroundjoin%
\pgfsetlinewidth{0.803000pt}%
\definecolor{currentstroke}{rgb}{0.690196,0.690196,0.690196}%
\pgfsetstrokecolor{currentstroke}%
\pgfsetstrokeopacity{0.200000}%
\pgfsetdash{{2.960000pt}{1.280000pt}}{0.000000pt}%
\pgfpathmoveto{\pgfqpoint{3.699620in}{0.643904in}}%
\pgfpathlineto{\pgfqpoint{3.699620in}{5.263904in}}%
\pgfusepath{stroke}%
\end{pgfscope}%
\begin{pgfscope}%
\pgfsetbuttcap%
\pgfsetroundjoin%
\definecolor{currentfill}{rgb}{0.000000,0.000000,0.000000}%
\pgfsetfillcolor{currentfill}%
\pgfsetlinewidth{0.602250pt}%
\definecolor{currentstroke}{rgb}{0.000000,0.000000,0.000000}%
\pgfsetstrokecolor{currentstroke}%
\pgfsetdash{}{0pt}%
\pgfsys@defobject{currentmarker}{\pgfqpoint{0.000000in}{-0.027778in}}{\pgfqpoint{0.000000in}{0.000000in}}{%
\pgfpathmoveto{\pgfqpoint{0.000000in}{0.000000in}}%
\pgfpathlineto{\pgfqpoint{0.000000in}{-0.027778in}}%
\pgfusepath{stroke,fill}%
}%
\begin{pgfscope}%
\pgfsys@transformshift{3.699620in}{0.643904in}%
\pgfsys@useobject{currentmarker}{}%
\end{pgfscope}%
\end{pgfscope}%
\begin{pgfscope}%
\pgfpathrectangle{\pgfqpoint{0.688192in}{0.643904in}}{\pgfqpoint{6.200000in}{4.620000in}}%
\pgfusepath{clip}%
\pgfsetbuttcap%
\pgfsetroundjoin%
\pgfsetlinewidth{0.803000pt}%
\definecolor{currentstroke}{rgb}{0.690196,0.690196,0.690196}%
\pgfsetstrokecolor{currentstroke}%
\pgfsetstrokeopacity{0.200000}%
\pgfsetdash{{2.960000pt}{1.280000pt}}{0.000000pt}%
\pgfpathmoveto{\pgfqpoint{3.876763in}{0.643904in}}%
\pgfpathlineto{\pgfqpoint{3.876763in}{5.263904in}}%
\pgfusepath{stroke}%
\end{pgfscope}%
\begin{pgfscope}%
\pgfsetbuttcap%
\pgfsetroundjoin%
\definecolor{currentfill}{rgb}{0.000000,0.000000,0.000000}%
\pgfsetfillcolor{currentfill}%
\pgfsetlinewidth{0.602250pt}%
\definecolor{currentstroke}{rgb}{0.000000,0.000000,0.000000}%
\pgfsetstrokecolor{currentstroke}%
\pgfsetdash{}{0pt}%
\pgfsys@defobject{currentmarker}{\pgfqpoint{0.000000in}{-0.027778in}}{\pgfqpoint{0.000000in}{0.000000in}}{%
\pgfpathmoveto{\pgfqpoint{0.000000in}{0.000000in}}%
\pgfpathlineto{\pgfqpoint{0.000000in}{-0.027778in}}%
\pgfusepath{stroke,fill}%
}%
\begin{pgfscope}%
\pgfsys@transformshift{3.876763in}{0.643904in}%
\pgfsys@useobject{currentmarker}{}%
\end{pgfscope}%
\end{pgfscope}%
\begin{pgfscope}%
\pgfpathrectangle{\pgfqpoint{0.688192in}{0.643904in}}{\pgfqpoint{6.200000in}{4.620000in}}%
\pgfusepath{clip}%
\pgfsetbuttcap%
\pgfsetroundjoin%
\pgfsetlinewidth{0.803000pt}%
\definecolor{currentstroke}{rgb}{0.690196,0.690196,0.690196}%
\pgfsetstrokecolor{currentstroke}%
\pgfsetstrokeopacity{0.200000}%
\pgfsetdash{{2.960000pt}{1.280000pt}}{0.000000pt}%
\pgfpathmoveto{\pgfqpoint{4.231049in}{0.643904in}}%
\pgfpathlineto{\pgfqpoint{4.231049in}{5.263904in}}%
\pgfusepath{stroke}%
\end{pgfscope}%
\begin{pgfscope}%
\pgfsetbuttcap%
\pgfsetroundjoin%
\definecolor{currentfill}{rgb}{0.000000,0.000000,0.000000}%
\pgfsetfillcolor{currentfill}%
\pgfsetlinewidth{0.602250pt}%
\definecolor{currentstroke}{rgb}{0.000000,0.000000,0.000000}%
\pgfsetstrokecolor{currentstroke}%
\pgfsetdash{}{0pt}%
\pgfsys@defobject{currentmarker}{\pgfqpoint{0.000000in}{-0.027778in}}{\pgfqpoint{0.000000in}{0.000000in}}{%
\pgfpathmoveto{\pgfqpoint{0.000000in}{0.000000in}}%
\pgfpathlineto{\pgfqpoint{0.000000in}{-0.027778in}}%
\pgfusepath{stroke,fill}%
}%
\begin{pgfscope}%
\pgfsys@transformshift{4.231049in}{0.643904in}%
\pgfsys@useobject{currentmarker}{}%
\end{pgfscope}%
\end{pgfscope}%
\begin{pgfscope}%
\pgfpathrectangle{\pgfqpoint{0.688192in}{0.643904in}}{\pgfqpoint{6.200000in}{4.620000in}}%
\pgfusepath{clip}%
\pgfsetbuttcap%
\pgfsetroundjoin%
\pgfsetlinewidth{0.803000pt}%
\definecolor{currentstroke}{rgb}{0.690196,0.690196,0.690196}%
\pgfsetstrokecolor{currentstroke}%
\pgfsetstrokeopacity{0.200000}%
\pgfsetdash{{2.960000pt}{1.280000pt}}{0.000000pt}%
\pgfpathmoveto{\pgfqpoint{4.408192in}{0.643904in}}%
\pgfpathlineto{\pgfqpoint{4.408192in}{5.263904in}}%
\pgfusepath{stroke}%
\end{pgfscope}%
\begin{pgfscope}%
\pgfsetbuttcap%
\pgfsetroundjoin%
\definecolor{currentfill}{rgb}{0.000000,0.000000,0.000000}%
\pgfsetfillcolor{currentfill}%
\pgfsetlinewidth{0.602250pt}%
\definecolor{currentstroke}{rgb}{0.000000,0.000000,0.000000}%
\pgfsetstrokecolor{currentstroke}%
\pgfsetdash{}{0pt}%
\pgfsys@defobject{currentmarker}{\pgfqpoint{0.000000in}{-0.027778in}}{\pgfqpoint{0.000000in}{0.000000in}}{%
\pgfpathmoveto{\pgfqpoint{0.000000in}{0.000000in}}%
\pgfpathlineto{\pgfqpoint{0.000000in}{-0.027778in}}%
\pgfusepath{stroke,fill}%
}%
\begin{pgfscope}%
\pgfsys@transformshift{4.408192in}{0.643904in}%
\pgfsys@useobject{currentmarker}{}%
\end{pgfscope}%
\end{pgfscope}%
\begin{pgfscope}%
\pgfpathrectangle{\pgfqpoint{0.688192in}{0.643904in}}{\pgfqpoint{6.200000in}{4.620000in}}%
\pgfusepath{clip}%
\pgfsetbuttcap%
\pgfsetroundjoin%
\pgfsetlinewidth{0.803000pt}%
\definecolor{currentstroke}{rgb}{0.690196,0.690196,0.690196}%
\pgfsetstrokecolor{currentstroke}%
\pgfsetstrokeopacity{0.200000}%
\pgfsetdash{{2.960000pt}{1.280000pt}}{0.000000pt}%
\pgfpathmoveto{\pgfqpoint{4.585334in}{0.643904in}}%
\pgfpathlineto{\pgfqpoint{4.585334in}{5.263904in}}%
\pgfusepath{stroke}%
\end{pgfscope}%
\begin{pgfscope}%
\pgfsetbuttcap%
\pgfsetroundjoin%
\definecolor{currentfill}{rgb}{0.000000,0.000000,0.000000}%
\pgfsetfillcolor{currentfill}%
\pgfsetlinewidth{0.602250pt}%
\definecolor{currentstroke}{rgb}{0.000000,0.000000,0.000000}%
\pgfsetstrokecolor{currentstroke}%
\pgfsetdash{}{0pt}%
\pgfsys@defobject{currentmarker}{\pgfqpoint{0.000000in}{-0.027778in}}{\pgfqpoint{0.000000in}{0.000000in}}{%
\pgfpathmoveto{\pgfqpoint{0.000000in}{0.000000in}}%
\pgfpathlineto{\pgfqpoint{0.000000in}{-0.027778in}}%
\pgfusepath{stroke,fill}%
}%
\begin{pgfscope}%
\pgfsys@transformshift{4.585334in}{0.643904in}%
\pgfsys@useobject{currentmarker}{}%
\end{pgfscope}%
\end{pgfscope}%
\begin{pgfscope}%
\pgfpathrectangle{\pgfqpoint{0.688192in}{0.643904in}}{\pgfqpoint{6.200000in}{4.620000in}}%
\pgfusepath{clip}%
\pgfsetbuttcap%
\pgfsetroundjoin%
\pgfsetlinewidth{0.803000pt}%
\definecolor{currentstroke}{rgb}{0.690196,0.690196,0.690196}%
\pgfsetstrokecolor{currentstroke}%
\pgfsetstrokeopacity{0.200000}%
\pgfsetdash{{2.960000pt}{1.280000pt}}{0.000000pt}%
\pgfpathmoveto{\pgfqpoint{4.939620in}{0.643904in}}%
\pgfpathlineto{\pgfqpoint{4.939620in}{5.263904in}}%
\pgfusepath{stroke}%
\end{pgfscope}%
\begin{pgfscope}%
\pgfsetbuttcap%
\pgfsetroundjoin%
\definecolor{currentfill}{rgb}{0.000000,0.000000,0.000000}%
\pgfsetfillcolor{currentfill}%
\pgfsetlinewidth{0.602250pt}%
\definecolor{currentstroke}{rgb}{0.000000,0.000000,0.000000}%
\pgfsetstrokecolor{currentstroke}%
\pgfsetdash{}{0pt}%
\pgfsys@defobject{currentmarker}{\pgfqpoint{0.000000in}{-0.027778in}}{\pgfqpoint{0.000000in}{0.000000in}}{%
\pgfpathmoveto{\pgfqpoint{0.000000in}{0.000000in}}%
\pgfpathlineto{\pgfqpoint{0.000000in}{-0.027778in}}%
\pgfusepath{stroke,fill}%
}%
\begin{pgfscope}%
\pgfsys@transformshift{4.939620in}{0.643904in}%
\pgfsys@useobject{currentmarker}{}%
\end{pgfscope}%
\end{pgfscope}%
\begin{pgfscope}%
\pgfpathrectangle{\pgfqpoint{0.688192in}{0.643904in}}{\pgfqpoint{6.200000in}{4.620000in}}%
\pgfusepath{clip}%
\pgfsetbuttcap%
\pgfsetroundjoin%
\pgfsetlinewidth{0.803000pt}%
\definecolor{currentstroke}{rgb}{0.690196,0.690196,0.690196}%
\pgfsetstrokecolor{currentstroke}%
\pgfsetstrokeopacity{0.200000}%
\pgfsetdash{{2.960000pt}{1.280000pt}}{0.000000pt}%
\pgfpathmoveto{\pgfqpoint{5.116763in}{0.643904in}}%
\pgfpathlineto{\pgfqpoint{5.116763in}{5.263904in}}%
\pgfusepath{stroke}%
\end{pgfscope}%
\begin{pgfscope}%
\pgfsetbuttcap%
\pgfsetroundjoin%
\definecolor{currentfill}{rgb}{0.000000,0.000000,0.000000}%
\pgfsetfillcolor{currentfill}%
\pgfsetlinewidth{0.602250pt}%
\definecolor{currentstroke}{rgb}{0.000000,0.000000,0.000000}%
\pgfsetstrokecolor{currentstroke}%
\pgfsetdash{}{0pt}%
\pgfsys@defobject{currentmarker}{\pgfqpoint{0.000000in}{-0.027778in}}{\pgfqpoint{0.000000in}{0.000000in}}{%
\pgfpathmoveto{\pgfqpoint{0.000000in}{0.000000in}}%
\pgfpathlineto{\pgfqpoint{0.000000in}{-0.027778in}}%
\pgfusepath{stroke,fill}%
}%
\begin{pgfscope}%
\pgfsys@transformshift{5.116763in}{0.643904in}%
\pgfsys@useobject{currentmarker}{}%
\end{pgfscope}%
\end{pgfscope}%
\begin{pgfscope}%
\pgfpathrectangle{\pgfqpoint{0.688192in}{0.643904in}}{\pgfqpoint{6.200000in}{4.620000in}}%
\pgfusepath{clip}%
\pgfsetbuttcap%
\pgfsetroundjoin%
\pgfsetlinewidth{0.803000pt}%
\definecolor{currentstroke}{rgb}{0.690196,0.690196,0.690196}%
\pgfsetstrokecolor{currentstroke}%
\pgfsetstrokeopacity{0.200000}%
\pgfsetdash{{2.960000pt}{1.280000pt}}{0.000000pt}%
\pgfpathmoveto{\pgfqpoint{5.293906in}{0.643904in}}%
\pgfpathlineto{\pgfqpoint{5.293906in}{5.263904in}}%
\pgfusepath{stroke}%
\end{pgfscope}%
\begin{pgfscope}%
\pgfsetbuttcap%
\pgfsetroundjoin%
\definecolor{currentfill}{rgb}{0.000000,0.000000,0.000000}%
\pgfsetfillcolor{currentfill}%
\pgfsetlinewidth{0.602250pt}%
\definecolor{currentstroke}{rgb}{0.000000,0.000000,0.000000}%
\pgfsetstrokecolor{currentstroke}%
\pgfsetdash{}{0pt}%
\pgfsys@defobject{currentmarker}{\pgfqpoint{0.000000in}{-0.027778in}}{\pgfqpoint{0.000000in}{0.000000in}}{%
\pgfpathmoveto{\pgfqpoint{0.000000in}{0.000000in}}%
\pgfpathlineto{\pgfqpoint{0.000000in}{-0.027778in}}%
\pgfusepath{stroke,fill}%
}%
\begin{pgfscope}%
\pgfsys@transformshift{5.293906in}{0.643904in}%
\pgfsys@useobject{currentmarker}{}%
\end{pgfscope}%
\end{pgfscope}%
\begin{pgfscope}%
\pgfpathrectangle{\pgfqpoint{0.688192in}{0.643904in}}{\pgfqpoint{6.200000in}{4.620000in}}%
\pgfusepath{clip}%
\pgfsetbuttcap%
\pgfsetroundjoin%
\pgfsetlinewidth{0.803000pt}%
\definecolor{currentstroke}{rgb}{0.690196,0.690196,0.690196}%
\pgfsetstrokecolor{currentstroke}%
\pgfsetstrokeopacity{0.200000}%
\pgfsetdash{{2.960000pt}{1.280000pt}}{0.000000pt}%
\pgfpathmoveto{\pgfqpoint{5.648192in}{0.643904in}}%
\pgfpathlineto{\pgfqpoint{5.648192in}{5.263904in}}%
\pgfusepath{stroke}%
\end{pgfscope}%
\begin{pgfscope}%
\pgfsetbuttcap%
\pgfsetroundjoin%
\definecolor{currentfill}{rgb}{0.000000,0.000000,0.000000}%
\pgfsetfillcolor{currentfill}%
\pgfsetlinewidth{0.602250pt}%
\definecolor{currentstroke}{rgb}{0.000000,0.000000,0.000000}%
\pgfsetstrokecolor{currentstroke}%
\pgfsetdash{}{0pt}%
\pgfsys@defobject{currentmarker}{\pgfqpoint{0.000000in}{-0.027778in}}{\pgfqpoint{0.000000in}{0.000000in}}{%
\pgfpathmoveto{\pgfqpoint{0.000000in}{0.000000in}}%
\pgfpathlineto{\pgfqpoint{0.000000in}{-0.027778in}}%
\pgfusepath{stroke,fill}%
}%
\begin{pgfscope}%
\pgfsys@transformshift{5.648192in}{0.643904in}%
\pgfsys@useobject{currentmarker}{}%
\end{pgfscope}%
\end{pgfscope}%
\begin{pgfscope}%
\pgfpathrectangle{\pgfqpoint{0.688192in}{0.643904in}}{\pgfqpoint{6.200000in}{4.620000in}}%
\pgfusepath{clip}%
\pgfsetbuttcap%
\pgfsetroundjoin%
\pgfsetlinewidth{0.803000pt}%
\definecolor{currentstroke}{rgb}{0.690196,0.690196,0.690196}%
\pgfsetstrokecolor{currentstroke}%
\pgfsetstrokeopacity{0.200000}%
\pgfsetdash{{2.960000pt}{1.280000pt}}{0.000000pt}%
\pgfpathmoveto{\pgfqpoint{5.825334in}{0.643904in}}%
\pgfpathlineto{\pgfqpoint{5.825334in}{5.263904in}}%
\pgfusepath{stroke}%
\end{pgfscope}%
\begin{pgfscope}%
\pgfsetbuttcap%
\pgfsetroundjoin%
\definecolor{currentfill}{rgb}{0.000000,0.000000,0.000000}%
\pgfsetfillcolor{currentfill}%
\pgfsetlinewidth{0.602250pt}%
\definecolor{currentstroke}{rgb}{0.000000,0.000000,0.000000}%
\pgfsetstrokecolor{currentstroke}%
\pgfsetdash{}{0pt}%
\pgfsys@defobject{currentmarker}{\pgfqpoint{0.000000in}{-0.027778in}}{\pgfqpoint{0.000000in}{0.000000in}}{%
\pgfpathmoveto{\pgfqpoint{0.000000in}{0.000000in}}%
\pgfpathlineto{\pgfqpoint{0.000000in}{-0.027778in}}%
\pgfusepath{stroke,fill}%
}%
\begin{pgfscope}%
\pgfsys@transformshift{5.825334in}{0.643904in}%
\pgfsys@useobject{currentmarker}{}%
\end{pgfscope}%
\end{pgfscope}%
\begin{pgfscope}%
\pgfpathrectangle{\pgfqpoint{0.688192in}{0.643904in}}{\pgfqpoint{6.200000in}{4.620000in}}%
\pgfusepath{clip}%
\pgfsetbuttcap%
\pgfsetroundjoin%
\pgfsetlinewidth{0.803000pt}%
\definecolor{currentstroke}{rgb}{0.690196,0.690196,0.690196}%
\pgfsetstrokecolor{currentstroke}%
\pgfsetstrokeopacity{0.200000}%
\pgfsetdash{{2.960000pt}{1.280000pt}}{0.000000pt}%
\pgfpathmoveto{\pgfqpoint{6.002477in}{0.643904in}}%
\pgfpathlineto{\pgfqpoint{6.002477in}{5.263904in}}%
\pgfusepath{stroke}%
\end{pgfscope}%
\begin{pgfscope}%
\pgfsetbuttcap%
\pgfsetroundjoin%
\definecolor{currentfill}{rgb}{0.000000,0.000000,0.000000}%
\pgfsetfillcolor{currentfill}%
\pgfsetlinewidth{0.602250pt}%
\definecolor{currentstroke}{rgb}{0.000000,0.000000,0.000000}%
\pgfsetstrokecolor{currentstroke}%
\pgfsetdash{}{0pt}%
\pgfsys@defobject{currentmarker}{\pgfqpoint{0.000000in}{-0.027778in}}{\pgfqpoint{0.000000in}{0.000000in}}{%
\pgfpathmoveto{\pgfqpoint{0.000000in}{0.000000in}}%
\pgfpathlineto{\pgfqpoint{0.000000in}{-0.027778in}}%
\pgfusepath{stroke,fill}%
}%
\begin{pgfscope}%
\pgfsys@transformshift{6.002477in}{0.643904in}%
\pgfsys@useobject{currentmarker}{}%
\end{pgfscope}%
\end{pgfscope}%
\begin{pgfscope}%
\pgfpathrectangle{\pgfqpoint{0.688192in}{0.643904in}}{\pgfqpoint{6.200000in}{4.620000in}}%
\pgfusepath{clip}%
\pgfsetbuttcap%
\pgfsetroundjoin%
\pgfsetlinewidth{0.803000pt}%
\definecolor{currentstroke}{rgb}{0.690196,0.690196,0.690196}%
\pgfsetstrokecolor{currentstroke}%
\pgfsetstrokeopacity{0.200000}%
\pgfsetdash{{2.960000pt}{1.280000pt}}{0.000000pt}%
\pgfpathmoveto{\pgfqpoint{6.356763in}{0.643904in}}%
\pgfpathlineto{\pgfqpoint{6.356763in}{5.263904in}}%
\pgfusepath{stroke}%
\end{pgfscope}%
\begin{pgfscope}%
\pgfsetbuttcap%
\pgfsetroundjoin%
\definecolor{currentfill}{rgb}{0.000000,0.000000,0.000000}%
\pgfsetfillcolor{currentfill}%
\pgfsetlinewidth{0.602250pt}%
\definecolor{currentstroke}{rgb}{0.000000,0.000000,0.000000}%
\pgfsetstrokecolor{currentstroke}%
\pgfsetdash{}{0pt}%
\pgfsys@defobject{currentmarker}{\pgfqpoint{0.000000in}{-0.027778in}}{\pgfqpoint{0.000000in}{0.000000in}}{%
\pgfpathmoveto{\pgfqpoint{0.000000in}{0.000000in}}%
\pgfpathlineto{\pgfqpoint{0.000000in}{-0.027778in}}%
\pgfusepath{stroke,fill}%
}%
\begin{pgfscope}%
\pgfsys@transformshift{6.356763in}{0.643904in}%
\pgfsys@useobject{currentmarker}{}%
\end{pgfscope}%
\end{pgfscope}%
\begin{pgfscope}%
\pgfpathrectangle{\pgfqpoint{0.688192in}{0.643904in}}{\pgfqpoint{6.200000in}{4.620000in}}%
\pgfusepath{clip}%
\pgfsetbuttcap%
\pgfsetroundjoin%
\pgfsetlinewidth{0.803000pt}%
\definecolor{currentstroke}{rgb}{0.690196,0.690196,0.690196}%
\pgfsetstrokecolor{currentstroke}%
\pgfsetstrokeopacity{0.200000}%
\pgfsetdash{{2.960000pt}{1.280000pt}}{0.000000pt}%
\pgfpathmoveto{\pgfqpoint{6.533906in}{0.643904in}}%
\pgfpathlineto{\pgfqpoint{6.533906in}{5.263904in}}%
\pgfusepath{stroke}%
\end{pgfscope}%
\begin{pgfscope}%
\pgfsetbuttcap%
\pgfsetroundjoin%
\definecolor{currentfill}{rgb}{0.000000,0.000000,0.000000}%
\pgfsetfillcolor{currentfill}%
\pgfsetlinewidth{0.602250pt}%
\definecolor{currentstroke}{rgb}{0.000000,0.000000,0.000000}%
\pgfsetstrokecolor{currentstroke}%
\pgfsetdash{}{0pt}%
\pgfsys@defobject{currentmarker}{\pgfqpoint{0.000000in}{-0.027778in}}{\pgfqpoint{0.000000in}{0.000000in}}{%
\pgfpathmoveto{\pgfqpoint{0.000000in}{0.000000in}}%
\pgfpathlineto{\pgfqpoint{0.000000in}{-0.027778in}}%
\pgfusepath{stroke,fill}%
}%
\begin{pgfscope}%
\pgfsys@transformshift{6.533906in}{0.643904in}%
\pgfsys@useobject{currentmarker}{}%
\end{pgfscope}%
\end{pgfscope}%
\begin{pgfscope}%
\pgfpathrectangle{\pgfqpoint{0.688192in}{0.643904in}}{\pgfqpoint{6.200000in}{4.620000in}}%
\pgfusepath{clip}%
\pgfsetbuttcap%
\pgfsetroundjoin%
\pgfsetlinewidth{0.803000pt}%
\definecolor{currentstroke}{rgb}{0.690196,0.690196,0.690196}%
\pgfsetstrokecolor{currentstroke}%
\pgfsetstrokeopacity{0.200000}%
\pgfsetdash{{2.960000pt}{1.280000pt}}{0.000000pt}%
\pgfpathmoveto{\pgfqpoint{6.711049in}{0.643904in}}%
\pgfpathlineto{\pgfqpoint{6.711049in}{5.263904in}}%
\pgfusepath{stroke}%
\end{pgfscope}%
\begin{pgfscope}%
\pgfsetbuttcap%
\pgfsetroundjoin%
\definecolor{currentfill}{rgb}{0.000000,0.000000,0.000000}%
\pgfsetfillcolor{currentfill}%
\pgfsetlinewidth{0.602250pt}%
\definecolor{currentstroke}{rgb}{0.000000,0.000000,0.000000}%
\pgfsetstrokecolor{currentstroke}%
\pgfsetdash{}{0pt}%
\pgfsys@defobject{currentmarker}{\pgfqpoint{0.000000in}{-0.027778in}}{\pgfqpoint{0.000000in}{0.000000in}}{%
\pgfpathmoveto{\pgfqpoint{0.000000in}{0.000000in}}%
\pgfpathlineto{\pgfqpoint{0.000000in}{-0.027778in}}%
\pgfusepath{stroke,fill}%
}%
\begin{pgfscope}%
\pgfsys@transformshift{6.711049in}{0.643904in}%
\pgfsys@useobject{currentmarker}{}%
\end{pgfscope}%
\end{pgfscope}%
\begin{pgfscope}%
\definecolor{textcolor}{rgb}{0.000000,0.000000,0.000000}%
\pgfsetstrokecolor{textcolor}%
\pgfsetfillcolor{textcolor}%
\pgftext[x=3.788192in,y=0.313349in,,top]{\color{textcolor}{\rmfamily\fontsize{18.000000}{21.600000}\selectfont\catcode`\^=\active\def^{\ifmmode\sp\else\^{}\fi}\catcode`\%=\active\def%{\%}Population per Generation}}%
\end{pgfscope}%
\begin{pgfscope}%
\pgfpathrectangle{\pgfqpoint{0.688192in}{0.643904in}}{\pgfqpoint{6.200000in}{4.620000in}}%
\pgfusepath{clip}%
\pgfsetrectcap%
\pgfsetroundjoin%
\pgfsetlinewidth{0.803000pt}%
\definecolor{currentstroke}{rgb}{0.690196,0.690196,0.690196}%
\pgfsetstrokecolor{currentstroke}%
\pgfsetdash{}{0pt}%
\pgfpathmoveto{\pgfqpoint{0.688192in}{1.118118in}}%
\pgfpathlineto{\pgfqpoint{6.888192in}{1.118118in}}%
\pgfusepath{stroke}%
\end{pgfscope}%
\begin{pgfscope}%
\pgfsetbuttcap%
\pgfsetroundjoin%
\definecolor{currentfill}{rgb}{0.000000,0.000000,0.000000}%
\pgfsetfillcolor{currentfill}%
\pgfsetlinewidth{0.803000pt}%
\definecolor{currentstroke}{rgb}{0.000000,0.000000,0.000000}%
\pgfsetstrokecolor{currentstroke}%
\pgfsetdash{}{0pt}%
\pgfsys@defobject{currentmarker}{\pgfqpoint{-0.048611in}{0.000000in}}{\pgfqpoint{-0.000000in}{0.000000in}}{%
\pgfpathmoveto{\pgfqpoint{-0.000000in}{0.000000in}}%
\pgfpathlineto{\pgfqpoint{-0.048611in}{0.000000in}}%
\pgfusepath{stroke,fill}%
}%
\begin{pgfscope}%
\pgfsys@transformshift{0.688192in}{1.118118in}%
\pgfsys@useobject{currentmarker}{}%
\end{pgfscope}%
\end{pgfscope}%
\begin{pgfscope}%
\definecolor{textcolor}{rgb}{0.000000,0.000000,0.000000}%
\pgfsetstrokecolor{textcolor}%
\pgfsetfillcolor{textcolor}%
\pgftext[x=0.493054in, y=1.048674in, left, base]{\color{textcolor}{\rmfamily\fontsize{14.000000}{16.800000}\selectfont\catcode`\^=\active\def^{\ifmmode\sp\else\^{}\fi}\catcode`\%=\active\def%{\%}$\mathdefault{5}$}}%
\end{pgfscope}%
\begin{pgfscope}%
\pgfpathrectangle{\pgfqpoint{0.688192in}{0.643904in}}{\pgfqpoint{6.200000in}{4.620000in}}%
\pgfusepath{clip}%
\pgfsetrectcap%
\pgfsetroundjoin%
\pgfsetlinewidth{0.803000pt}%
\definecolor{currentstroke}{rgb}{0.690196,0.690196,0.690196}%
\pgfsetstrokecolor{currentstroke}%
\pgfsetdash{}{0pt}%
\pgfpathmoveto{\pgfqpoint{0.688192in}{1.925862in}}%
\pgfpathlineto{\pgfqpoint{6.888192in}{1.925862in}}%
\pgfusepath{stroke}%
\end{pgfscope}%
\begin{pgfscope}%
\pgfsetbuttcap%
\pgfsetroundjoin%
\definecolor{currentfill}{rgb}{0.000000,0.000000,0.000000}%
\pgfsetfillcolor{currentfill}%
\pgfsetlinewidth{0.803000pt}%
\definecolor{currentstroke}{rgb}{0.000000,0.000000,0.000000}%
\pgfsetstrokecolor{currentstroke}%
\pgfsetdash{}{0pt}%
\pgfsys@defobject{currentmarker}{\pgfqpoint{-0.048611in}{0.000000in}}{\pgfqpoint{-0.000000in}{0.000000in}}{%
\pgfpathmoveto{\pgfqpoint{-0.000000in}{0.000000in}}%
\pgfpathlineto{\pgfqpoint{-0.048611in}{0.000000in}}%
\pgfusepath{stroke,fill}%
}%
\begin{pgfscope}%
\pgfsys@transformshift{0.688192in}{1.925862in}%
\pgfsys@useobject{currentmarker}{}%
\end{pgfscope}%
\end{pgfscope}%
\begin{pgfscope}%
\definecolor{textcolor}{rgb}{0.000000,0.000000,0.000000}%
\pgfsetstrokecolor{textcolor}%
\pgfsetfillcolor{textcolor}%
\pgftext[x=0.395138in, y=1.856418in, left, base]{\color{textcolor}{\rmfamily\fontsize{14.000000}{16.800000}\selectfont\catcode`\^=\active\def^{\ifmmode\sp\else\^{}\fi}\catcode`\%=\active\def%{\%}$\mathdefault{10}$}}%
\end{pgfscope}%
\begin{pgfscope}%
\pgfpathrectangle{\pgfqpoint{0.688192in}{0.643904in}}{\pgfqpoint{6.200000in}{4.620000in}}%
\pgfusepath{clip}%
\pgfsetrectcap%
\pgfsetroundjoin%
\pgfsetlinewidth{0.803000pt}%
\definecolor{currentstroke}{rgb}{0.690196,0.690196,0.690196}%
\pgfsetstrokecolor{currentstroke}%
\pgfsetdash{}{0pt}%
\pgfpathmoveto{\pgfqpoint{0.688192in}{2.733606in}}%
\pgfpathlineto{\pgfqpoint{6.888192in}{2.733606in}}%
\pgfusepath{stroke}%
\end{pgfscope}%
\begin{pgfscope}%
\pgfsetbuttcap%
\pgfsetroundjoin%
\definecolor{currentfill}{rgb}{0.000000,0.000000,0.000000}%
\pgfsetfillcolor{currentfill}%
\pgfsetlinewidth{0.803000pt}%
\definecolor{currentstroke}{rgb}{0.000000,0.000000,0.000000}%
\pgfsetstrokecolor{currentstroke}%
\pgfsetdash{}{0pt}%
\pgfsys@defobject{currentmarker}{\pgfqpoint{-0.048611in}{0.000000in}}{\pgfqpoint{-0.000000in}{0.000000in}}{%
\pgfpathmoveto{\pgfqpoint{-0.000000in}{0.000000in}}%
\pgfpathlineto{\pgfqpoint{-0.048611in}{0.000000in}}%
\pgfusepath{stroke,fill}%
}%
\begin{pgfscope}%
\pgfsys@transformshift{0.688192in}{2.733606in}%
\pgfsys@useobject{currentmarker}{}%
\end{pgfscope}%
\end{pgfscope}%
\begin{pgfscope}%
\definecolor{textcolor}{rgb}{0.000000,0.000000,0.000000}%
\pgfsetstrokecolor{textcolor}%
\pgfsetfillcolor{textcolor}%
\pgftext[x=0.395138in, y=2.664162in, left, base]{\color{textcolor}{\rmfamily\fontsize{14.000000}{16.800000}\selectfont\catcode`\^=\active\def^{\ifmmode\sp\else\^{}\fi}\catcode`\%=\active\def%{\%}$\mathdefault{15}$}}%
\end{pgfscope}%
\begin{pgfscope}%
\pgfpathrectangle{\pgfqpoint{0.688192in}{0.643904in}}{\pgfqpoint{6.200000in}{4.620000in}}%
\pgfusepath{clip}%
\pgfsetrectcap%
\pgfsetroundjoin%
\pgfsetlinewidth{0.803000pt}%
\definecolor{currentstroke}{rgb}{0.690196,0.690196,0.690196}%
\pgfsetstrokecolor{currentstroke}%
\pgfsetdash{}{0pt}%
\pgfpathmoveto{\pgfqpoint{0.688192in}{3.541350in}}%
\pgfpathlineto{\pgfqpoint{6.888192in}{3.541350in}}%
\pgfusepath{stroke}%
\end{pgfscope}%
\begin{pgfscope}%
\pgfsetbuttcap%
\pgfsetroundjoin%
\definecolor{currentfill}{rgb}{0.000000,0.000000,0.000000}%
\pgfsetfillcolor{currentfill}%
\pgfsetlinewidth{0.803000pt}%
\definecolor{currentstroke}{rgb}{0.000000,0.000000,0.000000}%
\pgfsetstrokecolor{currentstroke}%
\pgfsetdash{}{0pt}%
\pgfsys@defobject{currentmarker}{\pgfqpoint{-0.048611in}{0.000000in}}{\pgfqpoint{-0.000000in}{0.000000in}}{%
\pgfpathmoveto{\pgfqpoint{-0.000000in}{0.000000in}}%
\pgfpathlineto{\pgfqpoint{-0.048611in}{0.000000in}}%
\pgfusepath{stroke,fill}%
}%
\begin{pgfscope}%
\pgfsys@transformshift{0.688192in}{3.541350in}%
\pgfsys@useobject{currentmarker}{}%
\end{pgfscope}%
\end{pgfscope}%
\begin{pgfscope}%
\definecolor{textcolor}{rgb}{0.000000,0.000000,0.000000}%
\pgfsetstrokecolor{textcolor}%
\pgfsetfillcolor{textcolor}%
\pgftext[x=0.395138in, y=3.471906in, left, base]{\color{textcolor}{\rmfamily\fontsize{14.000000}{16.800000}\selectfont\catcode`\^=\active\def^{\ifmmode\sp\else\^{}\fi}\catcode`\%=\active\def%{\%}$\mathdefault{20}$}}%
\end{pgfscope}%
\begin{pgfscope}%
\pgfpathrectangle{\pgfqpoint{0.688192in}{0.643904in}}{\pgfqpoint{6.200000in}{4.620000in}}%
\pgfusepath{clip}%
\pgfsetrectcap%
\pgfsetroundjoin%
\pgfsetlinewidth{0.803000pt}%
\definecolor{currentstroke}{rgb}{0.690196,0.690196,0.690196}%
\pgfsetstrokecolor{currentstroke}%
\pgfsetdash{}{0pt}%
\pgfpathmoveto{\pgfqpoint{0.688192in}{4.349094in}}%
\pgfpathlineto{\pgfqpoint{6.888192in}{4.349094in}}%
\pgfusepath{stroke}%
\end{pgfscope}%
\begin{pgfscope}%
\pgfsetbuttcap%
\pgfsetroundjoin%
\definecolor{currentfill}{rgb}{0.000000,0.000000,0.000000}%
\pgfsetfillcolor{currentfill}%
\pgfsetlinewidth{0.803000pt}%
\definecolor{currentstroke}{rgb}{0.000000,0.000000,0.000000}%
\pgfsetstrokecolor{currentstroke}%
\pgfsetdash{}{0pt}%
\pgfsys@defobject{currentmarker}{\pgfqpoint{-0.048611in}{0.000000in}}{\pgfqpoint{-0.000000in}{0.000000in}}{%
\pgfpathmoveto{\pgfqpoint{-0.000000in}{0.000000in}}%
\pgfpathlineto{\pgfqpoint{-0.048611in}{0.000000in}}%
\pgfusepath{stroke,fill}%
}%
\begin{pgfscope}%
\pgfsys@transformshift{0.688192in}{4.349094in}%
\pgfsys@useobject{currentmarker}{}%
\end{pgfscope}%
\end{pgfscope}%
\begin{pgfscope}%
\definecolor{textcolor}{rgb}{0.000000,0.000000,0.000000}%
\pgfsetstrokecolor{textcolor}%
\pgfsetfillcolor{textcolor}%
\pgftext[x=0.395138in, y=4.279650in, left, base]{\color{textcolor}{\rmfamily\fontsize{14.000000}{16.800000}\selectfont\catcode`\^=\active\def^{\ifmmode\sp\else\^{}\fi}\catcode`\%=\active\def%{\%}$\mathdefault{25}$}}%
\end{pgfscope}%
\begin{pgfscope}%
\pgfpathrectangle{\pgfqpoint{0.688192in}{0.643904in}}{\pgfqpoint{6.200000in}{4.620000in}}%
\pgfusepath{clip}%
\pgfsetrectcap%
\pgfsetroundjoin%
\pgfsetlinewidth{0.803000pt}%
\definecolor{currentstroke}{rgb}{0.690196,0.690196,0.690196}%
\pgfsetstrokecolor{currentstroke}%
\pgfsetdash{}{0pt}%
\pgfpathmoveto{\pgfqpoint{0.688192in}{5.156838in}}%
\pgfpathlineto{\pgfqpoint{6.888192in}{5.156838in}}%
\pgfusepath{stroke}%
\end{pgfscope}%
\begin{pgfscope}%
\pgfsetbuttcap%
\pgfsetroundjoin%
\definecolor{currentfill}{rgb}{0.000000,0.000000,0.000000}%
\pgfsetfillcolor{currentfill}%
\pgfsetlinewidth{0.803000pt}%
\definecolor{currentstroke}{rgb}{0.000000,0.000000,0.000000}%
\pgfsetstrokecolor{currentstroke}%
\pgfsetdash{}{0pt}%
\pgfsys@defobject{currentmarker}{\pgfqpoint{-0.048611in}{0.000000in}}{\pgfqpoint{-0.000000in}{0.000000in}}{%
\pgfpathmoveto{\pgfqpoint{-0.000000in}{0.000000in}}%
\pgfpathlineto{\pgfqpoint{-0.048611in}{0.000000in}}%
\pgfusepath{stroke,fill}%
}%
\begin{pgfscope}%
\pgfsys@transformshift{0.688192in}{5.156838in}%
\pgfsys@useobject{currentmarker}{}%
\end{pgfscope}%
\end{pgfscope}%
\begin{pgfscope}%
\definecolor{textcolor}{rgb}{0.000000,0.000000,0.000000}%
\pgfsetstrokecolor{textcolor}%
\pgfsetfillcolor{textcolor}%
\pgftext[x=0.395138in, y=5.087394in, left, base]{\color{textcolor}{\rmfamily\fontsize{14.000000}{16.800000}\selectfont\catcode`\^=\active\def^{\ifmmode\sp\else\^{}\fi}\catcode`\%=\active\def%{\%}$\mathdefault{30}$}}%
\end{pgfscope}%
\begin{pgfscope}%
\pgfpathrectangle{\pgfqpoint{0.688192in}{0.643904in}}{\pgfqpoint{6.200000in}{4.620000in}}%
\pgfusepath{clip}%
\pgfsetbuttcap%
\pgfsetroundjoin%
\pgfsetlinewidth{0.803000pt}%
\definecolor{currentstroke}{rgb}{0.690196,0.690196,0.690196}%
\pgfsetstrokecolor{currentstroke}%
\pgfsetstrokeopacity{0.200000}%
\pgfsetdash{{2.960000pt}{1.280000pt}}{0.000000pt}%
\pgfpathmoveto{\pgfqpoint{0.688192in}{0.795020in}}%
\pgfpathlineto{\pgfqpoint{6.888192in}{0.795020in}}%
\pgfusepath{stroke}%
\end{pgfscope}%
\begin{pgfscope}%
\pgfsetbuttcap%
\pgfsetroundjoin%
\definecolor{currentfill}{rgb}{0.000000,0.000000,0.000000}%
\pgfsetfillcolor{currentfill}%
\pgfsetlinewidth{0.602250pt}%
\definecolor{currentstroke}{rgb}{0.000000,0.000000,0.000000}%
\pgfsetstrokecolor{currentstroke}%
\pgfsetdash{}{0pt}%
\pgfsys@defobject{currentmarker}{\pgfqpoint{-0.027778in}{0.000000in}}{\pgfqpoint{-0.000000in}{0.000000in}}{%
\pgfpathmoveto{\pgfqpoint{-0.000000in}{0.000000in}}%
\pgfpathlineto{\pgfqpoint{-0.027778in}{0.000000in}}%
\pgfusepath{stroke,fill}%
}%
\begin{pgfscope}%
\pgfsys@transformshift{0.688192in}{0.795020in}%
\pgfsys@useobject{currentmarker}{}%
\end{pgfscope}%
\end{pgfscope}%
\begin{pgfscope}%
\pgfpathrectangle{\pgfqpoint{0.688192in}{0.643904in}}{\pgfqpoint{6.200000in}{4.620000in}}%
\pgfusepath{clip}%
\pgfsetbuttcap%
\pgfsetroundjoin%
\pgfsetlinewidth{0.803000pt}%
\definecolor{currentstroke}{rgb}{0.690196,0.690196,0.690196}%
\pgfsetstrokecolor{currentstroke}%
\pgfsetstrokeopacity{0.200000}%
\pgfsetdash{{2.960000pt}{1.280000pt}}{0.000000pt}%
\pgfpathmoveto{\pgfqpoint{0.688192in}{0.956569in}}%
\pgfpathlineto{\pgfqpoint{6.888192in}{0.956569in}}%
\pgfusepath{stroke}%
\end{pgfscope}%
\begin{pgfscope}%
\pgfsetbuttcap%
\pgfsetroundjoin%
\definecolor{currentfill}{rgb}{0.000000,0.000000,0.000000}%
\pgfsetfillcolor{currentfill}%
\pgfsetlinewidth{0.602250pt}%
\definecolor{currentstroke}{rgb}{0.000000,0.000000,0.000000}%
\pgfsetstrokecolor{currentstroke}%
\pgfsetdash{}{0pt}%
\pgfsys@defobject{currentmarker}{\pgfqpoint{-0.027778in}{0.000000in}}{\pgfqpoint{-0.000000in}{0.000000in}}{%
\pgfpathmoveto{\pgfqpoint{-0.000000in}{0.000000in}}%
\pgfpathlineto{\pgfqpoint{-0.027778in}{0.000000in}}%
\pgfusepath{stroke,fill}%
}%
\begin{pgfscope}%
\pgfsys@transformshift{0.688192in}{0.956569in}%
\pgfsys@useobject{currentmarker}{}%
\end{pgfscope}%
\end{pgfscope}%
\begin{pgfscope}%
\pgfpathrectangle{\pgfqpoint{0.688192in}{0.643904in}}{\pgfqpoint{6.200000in}{4.620000in}}%
\pgfusepath{clip}%
\pgfsetbuttcap%
\pgfsetroundjoin%
\pgfsetlinewidth{0.803000pt}%
\definecolor{currentstroke}{rgb}{0.690196,0.690196,0.690196}%
\pgfsetstrokecolor{currentstroke}%
\pgfsetstrokeopacity{0.200000}%
\pgfsetdash{{2.960000pt}{1.280000pt}}{0.000000pt}%
\pgfpathmoveto{\pgfqpoint{0.688192in}{1.279667in}}%
\pgfpathlineto{\pgfqpoint{6.888192in}{1.279667in}}%
\pgfusepath{stroke}%
\end{pgfscope}%
\begin{pgfscope}%
\pgfsetbuttcap%
\pgfsetroundjoin%
\definecolor{currentfill}{rgb}{0.000000,0.000000,0.000000}%
\pgfsetfillcolor{currentfill}%
\pgfsetlinewidth{0.602250pt}%
\definecolor{currentstroke}{rgb}{0.000000,0.000000,0.000000}%
\pgfsetstrokecolor{currentstroke}%
\pgfsetdash{}{0pt}%
\pgfsys@defobject{currentmarker}{\pgfqpoint{-0.027778in}{0.000000in}}{\pgfqpoint{-0.000000in}{0.000000in}}{%
\pgfpathmoveto{\pgfqpoint{-0.000000in}{0.000000in}}%
\pgfpathlineto{\pgfqpoint{-0.027778in}{0.000000in}}%
\pgfusepath{stroke,fill}%
}%
\begin{pgfscope}%
\pgfsys@transformshift{0.688192in}{1.279667in}%
\pgfsys@useobject{currentmarker}{}%
\end{pgfscope}%
\end{pgfscope}%
\begin{pgfscope}%
\pgfpathrectangle{\pgfqpoint{0.688192in}{0.643904in}}{\pgfqpoint{6.200000in}{4.620000in}}%
\pgfusepath{clip}%
\pgfsetbuttcap%
\pgfsetroundjoin%
\pgfsetlinewidth{0.803000pt}%
\definecolor{currentstroke}{rgb}{0.690196,0.690196,0.690196}%
\pgfsetstrokecolor{currentstroke}%
\pgfsetstrokeopacity{0.200000}%
\pgfsetdash{{2.960000pt}{1.280000pt}}{0.000000pt}%
\pgfpathmoveto{\pgfqpoint{0.688192in}{1.441215in}}%
\pgfpathlineto{\pgfqpoint{6.888192in}{1.441215in}}%
\pgfusepath{stroke}%
\end{pgfscope}%
\begin{pgfscope}%
\pgfsetbuttcap%
\pgfsetroundjoin%
\definecolor{currentfill}{rgb}{0.000000,0.000000,0.000000}%
\pgfsetfillcolor{currentfill}%
\pgfsetlinewidth{0.602250pt}%
\definecolor{currentstroke}{rgb}{0.000000,0.000000,0.000000}%
\pgfsetstrokecolor{currentstroke}%
\pgfsetdash{}{0pt}%
\pgfsys@defobject{currentmarker}{\pgfqpoint{-0.027778in}{0.000000in}}{\pgfqpoint{-0.000000in}{0.000000in}}{%
\pgfpathmoveto{\pgfqpoint{-0.000000in}{0.000000in}}%
\pgfpathlineto{\pgfqpoint{-0.027778in}{0.000000in}}%
\pgfusepath{stroke,fill}%
}%
\begin{pgfscope}%
\pgfsys@transformshift{0.688192in}{1.441215in}%
\pgfsys@useobject{currentmarker}{}%
\end{pgfscope}%
\end{pgfscope}%
\begin{pgfscope}%
\pgfpathrectangle{\pgfqpoint{0.688192in}{0.643904in}}{\pgfqpoint{6.200000in}{4.620000in}}%
\pgfusepath{clip}%
\pgfsetbuttcap%
\pgfsetroundjoin%
\pgfsetlinewidth{0.803000pt}%
\definecolor{currentstroke}{rgb}{0.690196,0.690196,0.690196}%
\pgfsetstrokecolor{currentstroke}%
\pgfsetstrokeopacity{0.200000}%
\pgfsetdash{{2.960000pt}{1.280000pt}}{0.000000pt}%
\pgfpathmoveto{\pgfqpoint{0.688192in}{1.602764in}}%
\pgfpathlineto{\pgfqpoint{6.888192in}{1.602764in}}%
\pgfusepath{stroke}%
\end{pgfscope}%
\begin{pgfscope}%
\pgfsetbuttcap%
\pgfsetroundjoin%
\definecolor{currentfill}{rgb}{0.000000,0.000000,0.000000}%
\pgfsetfillcolor{currentfill}%
\pgfsetlinewidth{0.602250pt}%
\definecolor{currentstroke}{rgb}{0.000000,0.000000,0.000000}%
\pgfsetstrokecolor{currentstroke}%
\pgfsetdash{}{0pt}%
\pgfsys@defobject{currentmarker}{\pgfqpoint{-0.027778in}{0.000000in}}{\pgfqpoint{-0.000000in}{0.000000in}}{%
\pgfpathmoveto{\pgfqpoint{-0.000000in}{0.000000in}}%
\pgfpathlineto{\pgfqpoint{-0.027778in}{0.000000in}}%
\pgfusepath{stroke,fill}%
}%
\begin{pgfscope}%
\pgfsys@transformshift{0.688192in}{1.602764in}%
\pgfsys@useobject{currentmarker}{}%
\end{pgfscope}%
\end{pgfscope}%
\begin{pgfscope}%
\pgfpathrectangle{\pgfqpoint{0.688192in}{0.643904in}}{\pgfqpoint{6.200000in}{4.620000in}}%
\pgfusepath{clip}%
\pgfsetbuttcap%
\pgfsetroundjoin%
\pgfsetlinewidth{0.803000pt}%
\definecolor{currentstroke}{rgb}{0.690196,0.690196,0.690196}%
\pgfsetstrokecolor{currentstroke}%
\pgfsetstrokeopacity{0.200000}%
\pgfsetdash{{2.960000pt}{1.280000pt}}{0.000000pt}%
\pgfpathmoveto{\pgfqpoint{0.688192in}{1.764313in}}%
\pgfpathlineto{\pgfqpoint{6.888192in}{1.764313in}}%
\pgfusepath{stroke}%
\end{pgfscope}%
\begin{pgfscope}%
\pgfsetbuttcap%
\pgfsetroundjoin%
\definecolor{currentfill}{rgb}{0.000000,0.000000,0.000000}%
\pgfsetfillcolor{currentfill}%
\pgfsetlinewidth{0.602250pt}%
\definecolor{currentstroke}{rgb}{0.000000,0.000000,0.000000}%
\pgfsetstrokecolor{currentstroke}%
\pgfsetdash{}{0pt}%
\pgfsys@defobject{currentmarker}{\pgfqpoint{-0.027778in}{0.000000in}}{\pgfqpoint{-0.000000in}{0.000000in}}{%
\pgfpathmoveto{\pgfqpoint{-0.000000in}{0.000000in}}%
\pgfpathlineto{\pgfqpoint{-0.027778in}{0.000000in}}%
\pgfusepath{stroke,fill}%
}%
\begin{pgfscope}%
\pgfsys@transformshift{0.688192in}{1.764313in}%
\pgfsys@useobject{currentmarker}{}%
\end{pgfscope}%
\end{pgfscope}%
\begin{pgfscope}%
\pgfpathrectangle{\pgfqpoint{0.688192in}{0.643904in}}{\pgfqpoint{6.200000in}{4.620000in}}%
\pgfusepath{clip}%
\pgfsetbuttcap%
\pgfsetroundjoin%
\pgfsetlinewidth{0.803000pt}%
\definecolor{currentstroke}{rgb}{0.690196,0.690196,0.690196}%
\pgfsetstrokecolor{currentstroke}%
\pgfsetstrokeopacity{0.200000}%
\pgfsetdash{{2.960000pt}{1.280000pt}}{0.000000pt}%
\pgfpathmoveto{\pgfqpoint{0.688192in}{2.087411in}}%
\pgfpathlineto{\pgfqpoint{6.888192in}{2.087411in}}%
\pgfusepath{stroke}%
\end{pgfscope}%
\begin{pgfscope}%
\pgfsetbuttcap%
\pgfsetroundjoin%
\definecolor{currentfill}{rgb}{0.000000,0.000000,0.000000}%
\pgfsetfillcolor{currentfill}%
\pgfsetlinewidth{0.602250pt}%
\definecolor{currentstroke}{rgb}{0.000000,0.000000,0.000000}%
\pgfsetstrokecolor{currentstroke}%
\pgfsetdash{}{0pt}%
\pgfsys@defobject{currentmarker}{\pgfqpoint{-0.027778in}{0.000000in}}{\pgfqpoint{-0.000000in}{0.000000in}}{%
\pgfpathmoveto{\pgfqpoint{-0.000000in}{0.000000in}}%
\pgfpathlineto{\pgfqpoint{-0.027778in}{0.000000in}}%
\pgfusepath{stroke,fill}%
}%
\begin{pgfscope}%
\pgfsys@transformshift{0.688192in}{2.087411in}%
\pgfsys@useobject{currentmarker}{}%
\end{pgfscope}%
\end{pgfscope}%
\begin{pgfscope}%
\pgfpathrectangle{\pgfqpoint{0.688192in}{0.643904in}}{\pgfqpoint{6.200000in}{4.620000in}}%
\pgfusepath{clip}%
\pgfsetbuttcap%
\pgfsetroundjoin%
\pgfsetlinewidth{0.803000pt}%
\definecolor{currentstroke}{rgb}{0.690196,0.690196,0.690196}%
\pgfsetstrokecolor{currentstroke}%
\pgfsetstrokeopacity{0.200000}%
\pgfsetdash{{2.960000pt}{1.280000pt}}{0.000000pt}%
\pgfpathmoveto{\pgfqpoint{0.688192in}{2.248960in}}%
\pgfpathlineto{\pgfqpoint{6.888192in}{2.248960in}}%
\pgfusepath{stroke}%
\end{pgfscope}%
\begin{pgfscope}%
\pgfsetbuttcap%
\pgfsetroundjoin%
\definecolor{currentfill}{rgb}{0.000000,0.000000,0.000000}%
\pgfsetfillcolor{currentfill}%
\pgfsetlinewidth{0.602250pt}%
\definecolor{currentstroke}{rgb}{0.000000,0.000000,0.000000}%
\pgfsetstrokecolor{currentstroke}%
\pgfsetdash{}{0pt}%
\pgfsys@defobject{currentmarker}{\pgfqpoint{-0.027778in}{0.000000in}}{\pgfqpoint{-0.000000in}{0.000000in}}{%
\pgfpathmoveto{\pgfqpoint{-0.000000in}{0.000000in}}%
\pgfpathlineto{\pgfqpoint{-0.027778in}{0.000000in}}%
\pgfusepath{stroke,fill}%
}%
\begin{pgfscope}%
\pgfsys@transformshift{0.688192in}{2.248960in}%
\pgfsys@useobject{currentmarker}{}%
\end{pgfscope}%
\end{pgfscope}%
\begin{pgfscope}%
\pgfpathrectangle{\pgfqpoint{0.688192in}{0.643904in}}{\pgfqpoint{6.200000in}{4.620000in}}%
\pgfusepath{clip}%
\pgfsetbuttcap%
\pgfsetroundjoin%
\pgfsetlinewidth{0.803000pt}%
\definecolor{currentstroke}{rgb}{0.690196,0.690196,0.690196}%
\pgfsetstrokecolor{currentstroke}%
\pgfsetstrokeopacity{0.200000}%
\pgfsetdash{{2.960000pt}{1.280000pt}}{0.000000pt}%
\pgfpathmoveto{\pgfqpoint{0.688192in}{2.410508in}}%
\pgfpathlineto{\pgfqpoint{6.888192in}{2.410508in}}%
\pgfusepath{stroke}%
\end{pgfscope}%
\begin{pgfscope}%
\pgfsetbuttcap%
\pgfsetroundjoin%
\definecolor{currentfill}{rgb}{0.000000,0.000000,0.000000}%
\pgfsetfillcolor{currentfill}%
\pgfsetlinewidth{0.602250pt}%
\definecolor{currentstroke}{rgb}{0.000000,0.000000,0.000000}%
\pgfsetstrokecolor{currentstroke}%
\pgfsetdash{}{0pt}%
\pgfsys@defobject{currentmarker}{\pgfqpoint{-0.027778in}{0.000000in}}{\pgfqpoint{-0.000000in}{0.000000in}}{%
\pgfpathmoveto{\pgfqpoint{-0.000000in}{0.000000in}}%
\pgfpathlineto{\pgfqpoint{-0.027778in}{0.000000in}}%
\pgfusepath{stroke,fill}%
}%
\begin{pgfscope}%
\pgfsys@transformshift{0.688192in}{2.410508in}%
\pgfsys@useobject{currentmarker}{}%
\end{pgfscope}%
\end{pgfscope}%
\begin{pgfscope}%
\pgfpathrectangle{\pgfqpoint{0.688192in}{0.643904in}}{\pgfqpoint{6.200000in}{4.620000in}}%
\pgfusepath{clip}%
\pgfsetbuttcap%
\pgfsetroundjoin%
\pgfsetlinewidth{0.803000pt}%
\definecolor{currentstroke}{rgb}{0.690196,0.690196,0.690196}%
\pgfsetstrokecolor{currentstroke}%
\pgfsetstrokeopacity{0.200000}%
\pgfsetdash{{2.960000pt}{1.280000pt}}{0.000000pt}%
\pgfpathmoveto{\pgfqpoint{0.688192in}{2.572057in}}%
\pgfpathlineto{\pgfqpoint{6.888192in}{2.572057in}}%
\pgfusepath{stroke}%
\end{pgfscope}%
\begin{pgfscope}%
\pgfsetbuttcap%
\pgfsetroundjoin%
\definecolor{currentfill}{rgb}{0.000000,0.000000,0.000000}%
\pgfsetfillcolor{currentfill}%
\pgfsetlinewidth{0.602250pt}%
\definecolor{currentstroke}{rgb}{0.000000,0.000000,0.000000}%
\pgfsetstrokecolor{currentstroke}%
\pgfsetdash{}{0pt}%
\pgfsys@defobject{currentmarker}{\pgfqpoint{-0.027778in}{0.000000in}}{\pgfqpoint{-0.000000in}{0.000000in}}{%
\pgfpathmoveto{\pgfqpoint{-0.000000in}{0.000000in}}%
\pgfpathlineto{\pgfqpoint{-0.027778in}{0.000000in}}%
\pgfusepath{stroke,fill}%
}%
\begin{pgfscope}%
\pgfsys@transformshift{0.688192in}{2.572057in}%
\pgfsys@useobject{currentmarker}{}%
\end{pgfscope}%
\end{pgfscope}%
\begin{pgfscope}%
\pgfpathrectangle{\pgfqpoint{0.688192in}{0.643904in}}{\pgfqpoint{6.200000in}{4.620000in}}%
\pgfusepath{clip}%
\pgfsetbuttcap%
\pgfsetroundjoin%
\pgfsetlinewidth{0.803000pt}%
\definecolor{currentstroke}{rgb}{0.690196,0.690196,0.690196}%
\pgfsetstrokecolor{currentstroke}%
\pgfsetstrokeopacity{0.200000}%
\pgfsetdash{{2.960000pt}{1.280000pt}}{0.000000pt}%
\pgfpathmoveto{\pgfqpoint{0.688192in}{2.895155in}}%
\pgfpathlineto{\pgfqpoint{6.888192in}{2.895155in}}%
\pgfusepath{stroke}%
\end{pgfscope}%
\begin{pgfscope}%
\pgfsetbuttcap%
\pgfsetroundjoin%
\definecolor{currentfill}{rgb}{0.000000,0.000000,0.000000}%
\pgfsetfillcolor{currentfill}%
\pgfsetlinewidth{0.602250pt}%
\definecolor{currentstroke}{rgb}{0.000000,0.000000,0.000000}%
\pgfsetstrokecolor{currentstroke}%
\pgfsetdash{}{0pt}%
\pgfsys@defobject{currentmarker}{\pgfqpoint{-0.027778in}{0.000000in}}{\pgfqpoint{-0.000000in}{0.000000in}}{%
\pgfpathmoveto{\pgfqpoint{-0.000000in}{0.000000in}}%
\pgfpathlineto{\pgfqpoint{-0.027778in}{0.000000in}}%
\pgfusepath{stroke,fill}%
}%
\begin{pgfscope}%
\pgfsys@transformshift{0.688192in}{2.895155in}%
\pgfsys@useobject{currentmarker}{}%
\end{pgfscope}%
\end{pgfscope}%
\begin{pgfscope}%
\pgfpathrectangle{\pgfqpoint{0.688192in}{0.643904in}}{\pgfqpoint{6.200000in}{4.620000in}}%
\pgfusepath{clip}%
\pgfsetbuttcap%
\pgfsetroundjoin%
\pgfsetlinewidth{0.803000pt}%
\definecolor{currentstroke}{rgb}{0.690196,0.690196,0.690196}%
\pgfsetstrokecolor{currentstroke}%
\pgfsetstrokeopacity{0.200000}%
\pgfsetdash{{2.960000pt}{1.280000pt}}{0.000000pt}%
\pgfpathmoveto{\pgfqpoint{0.688192in}{3.056704in}}%
\pgfpathlineto{\pgfqpoint{6.888192in}{3.056704in}}%
\pgfusepath{stroke}%
\end{pgfscope}%
\begin{pgfscope}%
\pgfsetbuttcap%
\pgfsetroundjoin%
\definecolor{currentfill}{rgb}{0.000000,0.000000,0.000000}%
\pgfsetfillcolor{currentfill}%
\pgfsetlinewidth{0.602250pt}%
\definecolor{currentstroke}{rgb}{0.000000,0.000000,0.000000}%
\pgfsetstrokecolor{currentstroke}%
\pgfsetdash{}{0pt}%
\pgfsys@defobject{currentmarker}{\pgfqpoint{-0.027778in}{0.000000in}}{\pgfqpoint{-0.000000in}{0.000000in}}{%
\pgfpathmoveto{\pgfqpoint{-0.000000in}{0.000000in}}%
\pgfpathlineto{\pgfqpoint{-0.027778in}{0.000000in}}%
\pgfusepath{stroke,fill}%
}%
\begin{pgfscope}%
\pgfsys@transformshift{0.688192in}{3.056704in}%
\pgfsys@useobject{currentmarker}{}%
\end{pgfscope}%
\end{pgfscope}%
\begin{pgfscope}%
\pgfpathrectangle{\pgfqpoint{0.688192in}{0.643904in}}{\pgfqpoint{6.200000in}{4.620000in}}%
\pgfusepath{clip}%
\pgfsetbuttcap%
\pgfsetroundjoin%
\pgfsetlinewidth{0.803000pt}%
\definecolor{currentstroke}{rgb}{0.690196,0.690196,0.690196}%
\pgfsetstrokecolor{currentstroke}%
\pgfsetstrokeopacity{0.200000}%
\pgfsetdash{{2.960000pt}{1.280000pt}}{0.000000pt}%
\pgfpathmoveto{\pgfqpoint{0.688192in}{3.218253in}}%
\pgfpathlineto{\pgfqpoint{6.888192in}{3.218253in}}%
\pgfusepath{stroke}%
\end{pgfscope}%
\begin{pgfscope}%
\pgfsetbuttcap%
\pgfsetroundjoin%
\definecolor{currentfill}{rgb}{0.000000,0.000000,0.000000}%
\pgfsetfillcolor{currentfill}%
\pgfsetlinewidth{0.602250pt}%
\definecolor{currentstroke}{rgb}{0.000000,0.000000,0.000000}%
\pgfsetstrokecolor{currentstroke}%
\pgfsetdash{}{0pt}%
\pgfsys@defobject{currentmarker}{\pgfqpoint{-0.027778in}{0.000000in}}{\pgfqpoint{-0.000000in}{0.000000in}}{%
\pgfpathmoveto{\pgfqpoint{-0.000000in}{0.000000in}}%
\pgfpathlineto{\pgfqpoint{-0.027778in}{0.000000in}}%
\pgfusepath{stroke,fill}%
}%
\begin{pgfscope}%
\pgfsys@transformshift{0.688192in}{3.218253in}%
\pgfsys@useobject{currentmarker}{}%
\end{pgfscope}%
\end{pgfscope}%
\begin{pgfscope}%
\pgfpathrectangle{\pgfqpoint{0.688192in}{0.643904in}}{\pgfqpoint{6.200000in}{4.620000in}}%
\pgfusepath{clip}%
\pgfsetbuttcap%
\pgfsetroundjoin%
\pgfsetlinewidth{0.803000pt}%
\definecolor{currentstroke}{rgb}{0.690196,0.690196,0.690196}%
\pgfsetstrokecolor{currentstroke}%
\pgfsetstrokeopacity{0.200000}%
\pgfsetdash{{2.960000pt}{1.280000pt}}{0.000000pt}%
\pgfpathmoveto{\pgfqpoint{0.688192in}{3.379801in}}%
\pgfpathlineto{\pgfqpoint{6.888192in}{3.379801in}}%
\pgfusepath{stroke}%
\end{pgfscope}%
\begin{pgfscope}%
\pgfsetbuttcap%
\pgfsetroundjoin%
\definecolor{currentfill}{rgb}{0.000000,0.000000,0.000000}%
\pgfsetfillcolor{currentfill}%
\pgfsetlinewidth{0.602250pt}%
\definecolor{currentstroke}{rgb}{0.000000,0.000000,0.000000}%
\pgfsetstrokecolor{currentstroke}%
\pgfsetdash{}{0pt}%
\pgfsys@defobject{currentmarker}{\pgfqpoint{-0.027778in}{0.000000in}}{\pgfqpoint{-0.000000in}{0.000000in}}{%
\pgfpathmoveto{\pgfqpoint{-0.000000in}{0.000000in}}%
\pgfpathlineto{\pgfqpoint{-0.027778in}{0.000000in}}%
\pgfusepath{stroke,fill}%
}%
\begin{pgfscope}%
\pgfsys@transformshift{0.688192in}{3.379801in}%
\pgfsys@useobject{currentmarker}{}%
\end{pgfscope}%
\end{pgfscope}%
\begin{pgfscope}%
\pgfpathrectangle{\pgfqpoint{0.688192in}{0.643904in}}{\pgfqpoint{6.200000in}{4.620000in}}%
\pgfusepath{clip}%
\pgfsetbuttcap%
\pgfsetroundjoin%
\pgfsetlinewidth{0.803000pt}%
\definecolor{currentstroke}{rgb}{0.690196,0.690196,0.690196}%
\pgfsetstrokecolor{currentstroke}%
\pgfsetstrokeopacity{0.200000}%
\pgfsetdash{{2.960000pt}{1.280000pt}}{0.000000pt}%
\pgfpathmoveto{\pgfqpoint{0.688192in}{3.702899in}}%
\pgfpathlineto{\pgfqpoint{6.888192in}{3.702899in}}%
\pgfusepath{stroke}%
\end{pgfscope}%
\begin{pgfscope}%
\pgfsetbuttcap%
\pgfsetroundjoin%
\definecolor{currentfill}{rgb}{0.000000,0.000000,0.000000}%
\pgfsetfillcolor{currentfill}%
\pgfsetlinewidth{0.602250pt}%
\definecolor{currentstroke}{rgb}{0.000000,0.000000,0.000000}%
\pgfsetstrokecolor{currentstroke}%
\pgfsetdash{}{0pt}%
\pgfsys@defobject{currentmarker}{\pgfqpoint{-0.027778in}{0.000000in}}{\pgfqpoint{-0.000000in}{0.000000in}}{%
\pgfpathmoveto{\pgfqpoint{-0.000000in}{0.000000in}}%
\pgfpathlineto{\pgfqpoint{-0.027778in}{0.000000in}}%
\pgfusepath{stroke,fill}%
}%
\begin{pgfscope}%
\pgfsys@transformshift{0.688192in}{3.702899in}%
\pgfsys@useobject{currentmarker}{}%
\end{pgfscope}%
\end{pgfscope}%
\begin{pgfscope}%
\pgfpathrectangle{\pgfqpoint{0.688192in}{0.643904in}}{\pgfqpoint{6.200000in}{4.620000in}}%
\pgfusepath{clip}%
\pgfsetbuttcap%
\pgfsetroundjoin%
\pgfsetlinewidth{0.803000pt}%
\definecolor{currentstroke}{rgb}{0.690196,0.690196,0.690196}%
\pgfsetstrokecolor{currentstroke}%
\pgfsetstrokeopacity{0.200000}%
\pgfsetdash{{2.960000pt}{1.280000pt}}{0.000000pt}%
\pgfpathmoveto{\pgfqpoint{0.688192in}{3.864448in}}%
\pgfpathlineto{\pgfqpoint{6.888192in}{3.864448in}}%
\pgfusepath{stroke}%
\end{pgfscope}%
\begin{pgfscope}%
\pgfsetbuttcap%
\pgfsetroundjoin%
\definecolor{currentfill}{rgb}{0.000000,0.000000,0.000000}%
\pgfsetfillcolor{currentfill}%
\pgfsetlinewidth{0.602250pt}%
\definecolor{currentstroke}{rgb}{0.000000,0.000000,0.000000}%
\pgfsetstrokecolor{currentstroke}%
\pgfsetdash{}{0pt}%
\pgfsys@defobject{currentmarker}{\pgfqpoint{-0.027778in}{0.000000in}}{\pgfqpoint{-0.000000in}{0.000000in}}{%
\pgfpathmoveto{\pgfqpoint{-0.000000in}{0.000000in}}%
\pgfpathlineto{\pgfqpoint{-0.027778in}{0.000000in}}%
\pgfusepath{stroke,fill}%
}%
\begin{pgfscope}%
\pgfsys@transformshift{0.688192in}{3.864448in}%
\pgfsys@useobject{currentmarker}{}%
\end{pgfscope}%
\end{pgfscope}%
\begin{pgfscope}%
\pgfpathrectangle{\pgfqpoint{0.688192in}{0.643904in}}{\pgfqpoint{6.200000in}{4.620000in}}%
\pgfusepath{clip}%
\pgfsetbuttcap%
\pgfsetroundjoin%
\pgfsetlinewidth{0.803000pt}%
\definecolor{currentstroke}{rgb}{0.690196,0.690196,0.690196}%
\pgfsetstrokecolor{currentstroke}%
\pgfsetstrokeopacity{0.200000}%
\pgfsetdash{{2.960000pt}{1.280000pt}}{0.000000pt}%
\pgfpathmoveto{\pgfqpoint{0.688192in}{4.025997in}}%
\pgfpathlineto{\pgfqpoint{6.888192in}{4.025997in}}%
\pgfusepath{stroke}%
\end{pgfscope}%
\begin{pgfscope}%
\pgfsetbuttcap%
\pgfsetroundjoin%
\definecolor{currentfill}{rgb}{0.000000,0.000000,0.000000}%
\pgfsetfillcolor{currentfill}%
\pgfsetlinewidth{0.602250pt}%
\definecolor{currentstroke}{rgb}{0.000000,0.000000,0.000000}%
\pgfsetstrokecolor{currentstroke}%
\pgfsetdash{}{0pt}%
\pgfsys@defobject{currentmarker}{\pgfqpoint{-0.027778in}{0.000000in}}{\pgfqpoint{-0.000000in}{0.000000in}}{%
\pgfpathmoveto{\pgfqpoint{-0.000000in}{0.000000in}}%
\pgfpathlineto{\pgfqpoint{-0.027778in}{0.000000in}}%
\pgfusepath{stroke,fill}%
}%
\begin{pgfscope}%
\pgfsys@transformshift{0.688192in}{4.025997in}%
\pgfsys@useobject{currentmarker}{}%
\end{pgfscope}%
\end{pgfscope}%
\begin{pgfscope}%
\pgfpathrectangle{\pgfqpoint{0.688192in}{0.643904in}}{\pgfqpoint{6.200000in}{4.620000in}}%
\pgfusepath{clip}%
\pgfsetbuttcap%
\pgfsetroundjoin%
\pgfsetlinewidth{0.803000pt}%
\definecolor{currentstroke}{rgb}{0.690196,0.690196,0.690196}%
\pgfsetstrokecolor{currentstroke}%
\pgfsetstrokeopacity{0.200000}%
\pgfsetdash{{2.960000pt}{1.280000pt}}{0.000000pt}%
\pgfpathmoveto{\pgfqpoint{0.688192in}{4.187545in}}%
\pgfpathlineto{\pgfqpoint{6.888192in}{4.187545in}}%
\pgfusepath{stroke}%
\end{pgfscope}%
\begin{pgfscope}%
\pgfsetbuttcap%
\pgfsetroundjoin%
\definecolor{currentfill}{rgb}{0.000000,0.000000,0.000000}%
\pgfsetfillcolor{currentfill}%
\pgfsetlinewidth{0.602250pt}%
\definecolor{currentstroke}{rgb}{0.000000,0.000000,0.000000}%
\pgfsetstrokecolor{currentstroke}%
\pgfsetdash{}{0pt}%
\pgfsys@defobject{currentmarker}{\pgfqpoint{-0.027778in}{0.000000in}}{\pgfqpoint{-0.000000in}{0.000000in}}{%
\pgfpathmoveto{\pgfqpoint{-0.000000in}{0.000000in}}%
\pgfpathlineto{\pgfqpoint{-0.027778in}{0.000000in}}%
\pgfusepath{stroke,fill}%
}%
\begin{pgfscope}%
\pgfsys@transformshift{0.688192in}{4.187545in}%
\pgfsys@useobject{currentmarker}{}%
\end{pgfscope}%
\end{pgfscope}%
\begin{pgfscope}%
\pgfpathrectangle{\pgfqpoint{0.688192in}{0.643904in}}{\pgfqpoint{6.200000in}{4.620000in}}%
\pgfusepath{clip}%
\pgfsetbuttcap%
\pgfsetroundjoin%
\pgfsetlinewidth{0.803000pt}%
\definecolor{currentstroke}{rgb}{0.690196,0.690196,0.690196}%
\pgfsetstrokecolor{currentstroke}%
\pgfsetstrokeopacity{0.200000}%
\pgfsetdash{{2.960000pt}{1.280000pt}}{0.000000pt}%
\pgfpathmoveto{\pgfqpoint{0.688192in}{4.510643in}}%
\pgfpathlineto{\pgfqpoint{6.888192in}{4.510643in}}%
\pgfusepath{stroke}%
\end{pgfscope}%
\begin{pgfscope}%
\pgfsetbuttcap%
\pgfsetroundjoin%
\definecolor{currentfill}{rgb}{0.000000,0.000000,0.000000}%
\pgfsetfillcolor{currentfill}%
\pgfsetlinewidth{0.602250pt}%
\definecolor{currentstroke}{rgb}{0.000000,0.000000,0.000000}%
\pgfsetstrokecolor{currentstroke}%
\pgfsetdash{}{0pt}%
\pgfsys@defobject{currentmarker}{\pgfqpoint{-0.027778in}{0.000000in}}{\pgfqpoint{-0.000000in}{0.000000in}}{%
\pgfpathmoveto{\pgfqpoint{-0.000000in}{0.000000in}}%
\pgfpathlineto{\pgfqpoint{-0.027778in}{0.000000in}}%
\pgfusepath{stroke,fill}%
}%
\begin{pgfscope}%
\pgfsys@transformshift{0.688192in}{4.510643in}%
\pgfsys@useobject{currentmarker}{}%
\end{pgfscope}%
\end{pgfscope}%
\begin{pgfscope}%
\pgfpathrectangle{\pgfqpoint{0.688192in}{0.643904in}}{\pgfqpoint{6.200000in}{4.620000in}}%
\pgfusepath{clip}%
\pgfsetbuttcap%
\pgfsetroundjoin%
\pgfsetlinewidth{0.803000pt}%
\definecolor{currentstroke}{rgb}{0.690196,0.690196,0.690196}%
\pgfsetstrokecolor{currentstroke}%
\pgfsetstrokeopacity{0.200000}%
\pgfsetdash{{2.960000pt}{1.280000pt}}{0.000000pt}%
\pgfpathmoveto{\pgfqpoint{0.688192in}{4.672192in}}%
\pgfpathlineto{\pgfqpoint{6.888192in}{4.672192in}}%
\pgfusepath{stroke}%
\end{pgfscope}%
\begin{pgfscope}%
\pgfsetbuttcap%
\pgfsetroundjoin%
\definecolor{currentfill}{rgb}{0.000000,0.000000,0.000000}%
\pgfsetfillcolor{currentfill}%
\pgfsetlinewidth{0.602250pt}%
\definecolor{currentstroke}{rgb}{0.000000,0.000000,0.000000}%
\pgfsetstrokecolor{currentstroke}%
\pgfsetdash{}{0pt}%
\pgfsys@defobject{currentmarker}{\pgfqpoint{-0.027778in}{0.000000in}}{\pgfqpoint{-0.000000in}{0.000000in}}{%
\pgfpathmoveto{\pgfqpoint{-0.000000in}{0.000000in}}%
\pgfpathlineto{\pgfqpoint{-0.027778in}{0.000000in}}%
\pgfusepath{stroke,fill}%
}%
\begin{pgfscope}%
\pgfsys@transformshift{0.688192in}{4.672192in}%
\pgfsys@useobject{currentmarker}{}%
\end{pgfscope}%
\end{pgfscope}%
\begin{pgfscope}%
\pgfpathrectangle{\pgfqpoint{0.688192in}{0.643904in}}{\pgfqpoint{6.200000in}{4.620000in}}%
\pgfusepath{clip}%
\pgfsetbuttcap%
\pgfsetroundjoin%
\pgfsetlinewidth{0.803000pt}%
\definecolor{currentstroke}{rgb}{0.690196,0.690196,0.690196}%
\pgfsetstrokecolor{currentstroke}%
\pgfsetstrokeopacity{0.200000}%
\pgfsetdash{{2.960000pt}{1.280000pt}}{0.000000pt}%
\pgfpathmoveto{\pgfqpoint{0.688192in}{4.833741in}}%
\pgfpathlineto{\pgfqpoint{6.888192in}{4.833741in}}%
\pgfusepath{stroke}%
\end{pgfscope}%
\begin{pgfscope}%
\pgfsetbuttcap%
\pgfsetroundjoin%
\definecolor{currentfill}{rgb}{0.000000,0.000000,0.000000}%
\pgfsetfillcolor{currentfill}%
\pgfsetlinewidth{0.602250pt}%
\definecolor{currentstroke}{rgb}{0.000000,0.000000,0.000000}%
\pgfsetstrokecolor{currentstroke}%
\pgfsetdash{}{0pt}%
\pgfsys@defobject{currentmarker}{\pgfqpoint{-0.027778in}{0.000000in}}{\pgfqpoint{-0.000000in}{0.000000in}}{%
\pgfpathmoveto{\pgfqpoint{-0.000000in}{0.000000in}}%
\pgfpathlineto{\pgfqpoint{-0.027778in}{0.000000in}}%
\pgfusepath{stroke,fill}%
}%
\begin{pgfscope}%
\pgfsys@transformshift{0.688192in}{4.833741in}%
\pgfsys@useobject{currentmarker}{}%
\end{pgfscope}%
\end{pgfscope}%
\begin{pgfscope}%
\pgfpathrectangle{\pgfqpoint{0.688192in}{0.643904in}}{\pgfqpoint{6.200000in}{4.620000in}}%
\pgfusepath{clip}%
\pgfsetbuttcap%
\pgfsetroundjoin%
\pgfsetlinewidth{0.803000pt}%
\definecolor{currentstroke}{rgb}{0.690196,0.690196,0.690196}%
\pgfsetstrokecolor{currentstroke}%
\pgfsetstrokeopacity{0.200000}%
\pgfsetdash{{2.960000pt}{1.280000pt}}{0.000000pt}%
\pgfpathmoveto{\pgfqpoint{0.688192in}{4.995290in}}%
\pgfpathlineto{\pgfqpoint{6.888192in}{4.995290in}}%
\pgfusepath{stroke}%
\end{pgfscope}%
\begin{pgfscope}%
\pgfsetbuttcap%
\pgfsetroundjoin%
\definecolor{currentfill}{rgb}{0.000000,0.000000,0.000000}%
\pgfsetfillcolor{currentfill}%
\pgfsetlinewidth{0.602250pt}%
\definecolor{currentstroke}{rgb}{0.000000,0.000000,0.000000}%
\pgfsetstrokecolor{currentstroke}%
\pgfsetdash{}{0pt}%
\pgfsys@defobject{currentmarker}{\pgfqpoint{-0.027778in}{0.000000in}}{\pgfqpoint{-0.000000in}{0.000000in}}{%
\pgfpathmoveto{\pgfqpoint{-0.000000in}{0.000000in}}%
\pgfpathlineto{\pgfqpoint{-0.027778in}{0.000000in}}%
\pgfusepath{stroke,fill}%
}%
\begin{pgfscope}%
\pgfsys@transformshift{0.688192in}{4.995290in}%
\pgfsys@useobject{currentmarker}{}%
\end{pgfscope}%
\end{pgfscope}%
\begin{pgfscope}%
\definecolor{textcolor}{rgb}{0.000000,0.000000,0.000000}%
\pgfsetstrokecolor{textcolor}%
\pgfsetfillcolor{textcolor}%
\pgftext[x=0.339583in,y=2.953904in,,bottom,rotate=90.000000]{\color{textcolor}{\rmfamily\fontsize{18.000000}{21.600000}\selectfont\catcode`\^=\active\def^{\ifmmode\sp\else\^{}\fi}\catcode`\%=\active\def%{\%}Time [seconds]}}%
\end{pgfscope}%
\begin{pgfscope}%
\pgfpathrectangle{\pgfqpoint{0.688192in}{0.643904in}}{\pgfqpoint{6.200000in}{4.620000in}}%
\pgfusepath{clip}%
\pgfsetrectcap%
\pgfsetroundjoin%
\pgfsetlinewidth{1.505625pt}%
\definecolor{currentstroke}{rgb}{0.007843,0.243137,1.000000}%
\pgfsetstrokecolor{currentstroke}%
\pgfsetdash{}{0pt}%
\pgfpathmoveto{\pgfqpoint{0.688192in}{0.944593in}}%
\pgfpathlineto{\pgfqpoint{1.573906in}{1.613312in}}%
\pgfpathlineto{\pgfqpoint{3.345334in}{2.774081in}}%
\pgfpathlineto{\pgfqpoint{6.888192in}{5.053904in}}%
\pgfusepath{stroke}%
\end{pgfscope}%
\begin{pgfscope}%
\pgfpathrectangle{\pgfqpoint{0.688192in}{0.643904in}}{\pgfqpoint{6.200000in}{4.620000in}}%
\pgfusepath{clip}%
\pgfsetbuttcap%
\pgfsetroundjoin%
\definecolor{currentfill}{rgb}{0.007843,0.243137,1.000000}%
\pgfsetfillcolor{currentfill}%
\pgfsetlinewidth{0.752812pt}%
\definecolor{currentstroke}{rgb}{1.000000,1.000000,1.000000}%
\pgfsetstrokecolor{currentstroke}%
\pgfsetdash{}{0pt}%
\pgfsys@defobject{currentmarker}{\pgfqpoint{-0.041667in}{-0.041667in}}{\pgfqpoint{0.041667in}{0.041667in}}{%
\pgfpathmoveto{\pgfqpoint{0.000000in}{-0.041667in}}%
\pgfpathcurveto{\pgfqpoint{0.011050in}{-0.041667in}}{\pgfqpoint{0.021649in}{-0.037276in}}{\pgfqpoint{0.029463in}{-0.029463in}}%
\pgfpathcurveto{\pgfqpoint{0.037276in}{-0.021649in}}{\pgfqpoint{0.041667in}{-0.011050in}}{\pgfqpoint{0.041667in}{0.000000in}}%
\pgfpathcurveto{\pgfqpoint{0.041667in}{0.011050in}}{\pgfqpoint{0.037276in}{0.021649in}}{\pgfqpoint{0.029463in}{0.029463in}}%
\pgfpathcurveto{\pgfqpoint{0.021649in}{0.037276in}}{\pgfqpoint{0.011050in}{0.041667in}}{\pgfqpoint{0.000000in}{0.041667in}}%
\pgfpathcurveto{\pgfqpoint{-0.011050in}{0.041667in}}{\pgfqpoint{-0.021649in}{0.037276in}}{\pgfqpoint{-0.029463in}{0.029463in}}%
\pgfpathcurveto{\pgfqpoint{-0.037276in}{0.021649in}}{\pgfqpoint{-0.041667in}{0.011050in}}{\pgfqpoint{-0.041667in}{0.000000in}}%
\pgfpathcurveto{\pgfqpoint{-0.041667in}{-0.011050in}}{\pgfqpoint{-0.037276in}{-0.021649in}}{\pgfqpoint{-0.029463in}{-0.029463in}}%
\pgfpathcurveto{\pgfqpoint{-0.021649in}{-0.037276in}}{\pgfqpoint{-0.011050in}{-0.041667in}}{\pgfqpoint{0.000000in}{-0.041667in}}%
\pgfpathlineto{\pgfqpoint{0.000000in}{-0.041667in}}%
\pgfpathclose%
\pgfusepath{stroke,fill}%
}%
\begin{pgfscope}%
\pgfsys@transformshift{0.688192in}{0.944593in}%
\pgfsys@useobject{currentmarker}{}%
\end{pgfscope}%
\begin{pgfscope}%
\pgfsys@transformshift{1.573906in}{1.613312in}%
\pgfsys@useobject{currentmarker}{}%
\end{pgfscope}%
\begin{pgfscope}%
\pgfsys@transformshift{3.345334in}{2.774081in}%
\pgfsys@useobject{currentmarker}{}%
\end{pgfscope}%
\begin{pgfscope}%
\pgfsys@transformshift{6.888192in}{5.053904in}%
\pgfsys@useobject{currentmarker}{}%
\end{pgfscope}%
\end{pgfscope}%
\begin{pgfscope}%
\pgfpathrectangle{\pgfqpoint{0.688192in}{0.643904in}}{\pgfqpoint{6.200000in}{4.620000in}}%
\pgfusepath{clip}%
\pgfsetbuttcap%
\pgfsetroundjoin%
\pgfsetlinewidth{1.505625pt}%
\definecolor{currentstroke}{rgb}{1.000000,0.486275,0.000000}%
\pgfsetstrokecolor{currentstroke}%
\pgfsetdash{{6.000000pt}{2.250000pt}}{0.000000pt}%
\pgfpathmoveto{\pgfqpoint{0.688192in}{0.937993in}}%
\pgfpathlineto{\pgfqpoint{1.573906in}{1.570977in}}%
\pgfpathlineto{\pgfqpoint{3.345334in}{2.596766in}}%
\pgfpathlineto{\pgfqpoint{6.888192in}{4.725093in}}%
\pgfusepath{stroke}%
\end{pgfscope}%
\begin{pgfscope}%
\pgfpathrectangle{\pgfqpoint{0.688192in}{0.643904in}}{\pgfqpoint{6.200000in}{4.620000in}}%
\pgfusepath{clip}%
\pgfsetbuttcap%
\pgfsetroundjoin%
\definecolor{currentfill}{rgb}{1.000000,0.486275,0.000000}%
\pgfsetfillcolor{currentfill}%
\pgfsetlinewidth{0.752812pt}%
\definecolor{currentstroke}{rgb}{1.000000,1.000000,1.000000}%
\pgfsetstrokecolor{currentstroke}%
\pgfsetdash{}{0pt}%
\pgfsys@defobject{currentmarker}{\pgfqpoint{-0.041667in}{-0.041667in}}{\pgfqpoint{0.041667in}{0.041667in}}{%
\pgfpathmoveto{\pgfqpoint{0.000000in}{-0.041667in}}%
\pgfpathcurveto{\pgfqpoint{0.011050in}{-0.041667in}}{\pgfqpoint{0.021649in}{-0.037276in}}{\pgfqpoint{0.029463in}{-0.029463in}}%
\pgfpathcurveto{\pgfqpoint{0.037276in}{-0.021649in}}{\pgfqpoint{0.041667in}{-0.011050in}}{\pgfqpoint{0.041667in}{0.000000in}}%
\pgfpathcurveto{\pgfqpoint{0.041667in}{0.011050in}}{\pgfqpoint{0.037276in}{0.021649in}}{\pgfqpoint{0.029463in}{0.029463in}}%
\pgfpathcurveto{\pgfqpoint{0.021649in}{0.037276in}}{\pgfqpoint{0.011050in}{0.041667in}}{\pgfqpoint{0.000000in}{0.041667in}}%
\pgfpathcurveto{\pgfqpoint{-0.011050in}{0.041667in}}{\pgfqpoint{-0.021649in}{0.037276in}}{\pgfqpoint{-0.029463in}{0.029463in}}%
\pgfpathcurveto{\pgfqpoint{-0.037276in}{0.021649in}}{\pgfqpoint{-0.041667in}{0.011050in}}{\pgfqpoint{-0.041667in}{0.000000in}}%
\pgfpathcurveto{\pgfqpoint{-0.041667in}{-0.011050in}}{\pgfqpoint{-0.037276in}{-0.021649in}}{\pgfqpoint{-0.029463in}{-0.029463in}}%
\pgfpathcurveto{\pgfqpoint{-0.021649in}{-0.037276in}}{\pgfqpoint{-0.011050in}{-0.041667in}}{\pgfqpoint{0.000000in}{-0.041667in}}%
\pgfpathlineto{\pgfqpoint{0.000000in}{-0.041667in}}%
\pgfpathclose%
\pgfusepath{stroke,fill}%
}%
\begin{pgfscope}%
\pgfsys@transformshift{0.688192in}{0.937993in}%
\pgfsys@useobject{currentmarker}{}%
\end{pgfscope}%
\begin{pgfscope}%
\pgfsys@transformshift{1.573906in}{1.570977in}%
\pgfsys@useobject{currentmarker}{}%
\end{pgfscope}%
\begin{pgfscope}%
\pgfsys@transformshift{3.345334in}{2.596766in}%
\pgfsys@useobject{currentmarker}{}%
\end{pgfscope}%
\begin{pgfscope}%
\pgfsys@transformshift{6.888192in}{4.725093in}%
\pgfsys@useobject{currentmarker}{}%
\end{pgfscope}%
\end{pgfscope}%
\begin{pgfscope}%
\pgfpathrectangle{\pgfqpoint{0.688192in}{0.643904in}}{\pgfqpoint{6.200000in}{4.620000in}}%
\pgfusepath{clip}%
\pgfsetbuttcap%
\pgfsetroundjoin%
\pgfsetlinewidth{1.505625pt}%
\definecolor{currentstroke}{rgb}{0.101961,0.788235,0.219608}%
\pgfsetstrokecolor{currentstroke}%
\pgfsetdash{{1.500000pt}{1.500000pt}}{0.000000pt}%
\pgfpathmoveto{\pgfqpoint{0.688192in}{0.853904in}}%
\pgfpathlineto{\pgfqpoint{1.573906in}{1.519257in}}%
\pgfpathlineto{\pgfqpoint{3.345334in}{2.702531in}}%
\pgfpathlineto{\pgfqpoint{6.888192in}{4.463327in}}%
\pgfusepath{stroke}%
\end{pgfscope}%
\begin{pgfscope}%
\pgfpathrectangle{\pgfqpoint{0.688192in}{0.643904in}}{\pgfqpoint{6.200000in}{4.620000in}}%
\pgfusepath{clip}%
\pgfsetbuttcap%
\pgfsetroundjoin%
\definecolor{currentfill}{rgb}{0.101961,0.788235,0.219608}%
\pgfsetfillcolor{currentfill}%
\pgfsetlinewidth{0.752812pt}%
\definecolor{currentstroke}{rgb}{1.000000,1.000000,1.000000}%
\pgfsetstrokecolor{currentstroke}%
\pgfsetdash{}{0pt}%
\pgfsys@defobject{currentmarker}{\pgfqpoint{-0.041667in}{-0.041667in}}{\pgfqpoint{0.041667in}{0.041667in}}{%
\pgfpathmoveto{\pgfqpoint{0.000000in}{-0.041667in}}%
\pgfpathcurveto{\pgfqpoint{0.011050in}{-0.041667in}}{\pgfqpoint{0.021649in}{-0.037276in}}{\pgfqpoint{0.029463in}{-0.029463in}}%
\pgfpathcurveto{\pgfqpoint{0.037276in}{-0.021649in}}{\pgfqpoint{0.041667in}{-0.011050in}}{\pgfqpoint{0.041667in}{0.000000in}}%
\pgfpathcurveto{\pgfqpoint{0.041667in}{0.011050in}}{\pgfqpoint{0.037276in}{0.021649in}}{\pgfqpoint{0.029463in}{0.029463in}}%
\pgfpathcurveto{\pgfqpoint{0.021649in}{0.037276in}}{\pgfqpoint{0.011050in}{0.041667in}}{\pgfqpoint{0.000000in}{0.041667in}}%
\pgfpathcurveto{\pgfqpoint{-0.011050in}{0.041667in}}{\pgfqpoint{-0.021649in}{0.037276in}}{\pgfqpoint{-0.029463in}{0.029463in}}%
\pgfpathcurveto{\pgfqpoint{-0.037276in}{0.021649in}}{\pgfqpoint{-0.041667in}{0.011050in}}{\pgfqpoint{-0.041667in}{0.000000in}}%
\pgfpathcurveto{\pgfqpoint{-0.041667in}{-0.011050in}}{\pgfqpoint{-0.037276in}{-0.021649in}}{\pgfqpoint{-0.029463in}{-0.029463in}}%
\pgfpathcurveto{\pgfqpoint{-0.021649in}{-0.037276in}}{\pgfqpoint{-0.011050in}{-0.041667in}}{\pgfqpoint{0.000000in}{-0.041667in}}%
\pgfpathlineto{\pgfqpoint{0.000000in}{-0.041667in}}%
\pgfpathclose%
\pgfusepath{stroke,fill}%
}%
\begin{pgfscope}%
\pgfsys@transformshift{0.688192in}{0.853904in}%
\pgfsys@useobject{currentmarker}{}%
\end{pgfscope}%
\begin{pgfscope}%
\pgfsys@transformshift{1.573906in}{1.519257in}%
\pgfsys@useobject{currentmarker}{}%
\end{pgfscope}%
\begin{pgfscope}%
\pgfsys@transformshift{3.345334in}{2.702531in}%
\pgfsys@useobject{currentmarker}{}%
\end{pgfscope}%
\begin{pgfscope}%
\pgfsys@transformshift{6.888192in}{4.463327in}%
\pgfsys@useobject{currentmarker}{}%
\end{pgfscope}%
\end{pgfscope}%
\begin{pgfscope}%
\pgfpathrectangle{\pgfqpoint{0.688192in}{0.643904in}}{\pgfqpoint{6.200000in}{4.620000in}}%
\pgfusepath{clip}%
\pgfsetbuttcap%
\pgfsetroundjoin%
\pgfsetlinewidth{1.505625pt}%
\definecolor{currentstroke}{rgb}{0.909804,0.000000,0.043137}%
\pgfsetstrokecolor{currentstroke}%
\pgfsetdash{{4.500000pt}{1.875000pt}{2.250000pt}{1.875000pt}}{0.000000pt}%
\pgfpathmoveto{\pgfqpoint{0.688192in}{0.858156in}}%
\pgfpathlineto{\pgfqpoint{1.573906in}{1.407561in}}%
\pgfpathlineto{\pgfqpoint{3.345334in}{2.117402in}}%
\pgfpathlineto{\pgfqpoint{6.888192in}{4.376203in}}%
\pgfusepath{stroke}%
\end{pgfscope}%
\begin{pgfscope}%
\pgfpathrectangle{\pgfqpoint{0.688192in}{0.643904in}}{\pgfqpoint{6.200000in}{4.620000in}}%
\pgfusepath{clip}%
\pgfsetbuttcap%
\pgfsetroundjoin%
\definecolor{currentfill}{rgb}{0.909804,0.000000,0.043137}%
\pgfsetfillcolor{currentfill}%
\pgfsetlinewidth{0.752812pt}%
\definecolor{currentstroke}{rgb}{1.000000,1.000000,1.000000}%
\pgfsetstrokecolor{currentstroke}%
\pgfsetdash{}{0pt}%
\pgfsys@defobject{currentmarker}{\pgfqpoint{-0.041667in}{-0.041667in}}{\pgfqpoint{0.041667in}{0.041667in}}{%
\pgfpathmoveto{\pgfqpoint{0.000000in}{-0.041667in}}%
\pgfpathcurveto{\pgfqpoint{0.011050in}{-0.041667in}}{\pgfqpoint{0.021649in}{-0.037276in}}{\pgfqpoint{0.029463in}{-0.029463in}}%
\pgfpathcurveto{\pgfqpoint{0.037276in}{-0.021649in}}{\pgfqpoint{0.041667in}{-0.011050in}}{\pgfqpoint{0.041667in}{0.000000in}}%
\pgfpathcurveto{\pgfqpoint{0.041667in}{0.011050in}}{\pgfqpoint{0.037276in}{0.021649in}}{\pgfqpoint{0.029463in}{0.029463in}}%
\pgfpathcurveto{\pgfqpoint{0.021649in}{0.037276in}}{\pgfqpoint{0.011050in}{0.041667in}}{\pgfqpoint{0.000000in}{0.041667in}}%
\pgfpathcurveto{\pgfqpoint{-0.011050in}{0.041667in}}{\pgfqpoint{-0.021649in}{0.037276in}}{\pgfqpoint{-0.029463in}{0.029463in}}%
\pgfpathcurveto{\pgfqpoint{-0.037276in}{0.021649in}}{\pgfqpoint{-0.041667in}{0.011050in}}{\pgfqpoint{-0.041667in}{0.000000in}}%
\pgfpathcurveto{\pgfqpoint{-0.041667in}{-0.011050in}}{\pgfqpoint{-0.037276in}{-0.021649in}}{\pgfqpoint{-0.029463in}{-0.029463in}}%
\pgfpathcurveto{\pgfqpoint{-0.021649in}{-0.037276in}}{\pgfqpoint{-0.011050in}{-0.041667in}}{\pgfqpoint{0.000000in}{-0.041667in}}%
\pgfpathlineto{\pgfqpoint{0.000000in}{-0.041667in}}%
\pgfpathclose%
\pgfusepath{stroke,fill}%
}%
\begin{pgfscope}%
\pgfsys@transformshift{0.688192in}{0.858156in}%
\pgfsys@useobject{currentmarker}{}%
\end{pgfscope}%
\begin{pgfscope}%
\pgfsys@transformshift{1.573906in}{1.407561in}%
\pgfsys@useobject{currentmarker}{}%
\end{pgfscope}%
\begin{pgfscope}%
\pgfsys@transformshift{3.345334in}{2.117402in}%
\pgfsys@useobject{currentmarker}{}%
\end{pgfscope}%
\begin{pgfscope}%
\pgfsys@transformshift{6.888192in}{4.376203in}%
\pgfsys@useobject{currentmarker}{}%
\end{pgfscope}%
\end{pgfscope}%
\begin{pgfscope}%
\pgfpathrectangle{\pgfqpoint{0.688192in}{0.643904in}}{\pgfqpoint{6.200000in}{4.620000in}}%
\pgfusepath{clip}%
\pgfsetbuttcap%
\pgfsetroundjoin%
\pgfsetlinewidth{1.505625pt}%
\definecolor{currentstroke}{rgb}{0.545098,0.168627,0.886275}%
\pgfsetstrokecolor{currentstroke}%
\pgfsetdash{{7.500000pt}{1.500000pt}{1.500000pt}{1.500000pt}}{0.000000pt}%
\pgfpathmoveto{\pgfqpoint{0.688192in}{0.876810in}}%
\pgfpathlineto{\pgfqpoint{1.573906in}{1.419924in}}%
\pgfpathlineto{\pgfqpoint{3.345334in}{2.445519in}}%
\pgfpathlineto{\pgfqpoint{6.888192in}{4.301315in}}%
\pgfusepath{stroke}%
\end{pgfscope}%
\begin{pgfscope}%
\pgfpathrectangle{\pgfqpoint{0.688192in}{0.643904in}}{\pgfqpoint{6.200000in}{4.620000in}}%
\pgfusepath{clip}%
\pgfsetbuttcap%
\pgfsetroundjoin%
\definecolor{currentfill}{rgb}{0.545098,0.168627,0.886275}%
\pgfsetfillcolor{currentfill}%
\pgfsetlinewidth{0.752812pt}%
\definecolor{currentstroke}{rgb}{1.000000,1.000000,1.000000}%
\pgfsetstrokecolor{currentstroke}%
\pgfsetdash{}{0pt}%
\pgfsys@defobject{currentmarker}{\pgfqpoint{-0.041667in}{-0.041667in}}{\pgfqpoint{0.041667in}{0.041667in}}{%
\pgfpathmoveto{\pgfqpoint{0.000000in}{-0.041667in}}%
\pgfpathcurveto{\pgfqpoint{0.011050in}{-0.041667in}}{\pgfqpoint{0.021649in}{-0.037276in}}{\pgfqpoint{0.029463in}{-0.029463in}}%
\pgfpathcurveto{\pgfqpoint{0.037276in}{-0.021649in}}{\pgfqpoint{0.041667in}{-0.011050in}}{\pgfqpoint{0.041667in}{0.000000in}}%
\pgfpathcurveto{\pgfqpoint{0.041667in}{0.011050in}}{\pgfqpoint{0.037276in}{0.021649in}}{\pgfqpoint{0.029463in}{0.029463in}}%
\pgfpathcurveto{\pgfqpoint{0.021649in}{0.037276in}}{\pgfqpoint{0.011050in}{0.041667in}}{\pgfqpoint{0.000000in}{0.041667in}}%
\pgfpathcurveto{\pgfqpoint{-0.011050in}{0.041667in}}{\pgfqpoint{-0.021649in}{0.037276in}}{\pgfqpoint{-0.029463in}{0.029463in}}%
\pgfpathcurveto{\pgfqpoint{-0.037276in}{0.021649in}}{\pgfqpoint{-0.041667in}{0.011050in}}{\pgfqpoint{-0.041667in}{0.000000in}}%
\pgfpathcurveto{\pgfqpoint{-0.041667in}{-0.011050in}}{\pgfqpoint{-0.037276in}{-0.021649in}}{\pgfqpoint{-0.029463in}{-0.029463in}}%
\pgfpathcurveto{\pgfqpoint{-0.021649in}{-0.037276in}}{\pgfqpoint{-0.011050in}{-0.041667in}}{\pgfqpoint{0.000000in}{-0.041667in}}%
\pgfpathlineto{\pgfqpoint{0.000000in}{-0.041667in}}%
\pgfpathclose%
\pgfusepath{stroke,fill}%
}%
\begin{pgfscope}%
\pgfsys@transformshift{0.688192in}{0.876810in}%
\pgfsys@useobject{currentmarker}{}%
\end{pgfscope}%
\begin{pgfscope}%
\pgfsys@transformshift{1.573906in}{1.419924in}%
\pgfsys@useobject{currentmarker}{}%
\end{pgfscope}%
\begin{pgfscope}%
\pgfsys@transformshift{3.345334in}{2.445519in}%
\pgfsys@useobject{currentmarker}{}%
\end{pgfscope}%
\begin{pgfscope}%
\pgfsys@transformshift{6.888192in}{4.301315in}%
\pgfsys@useobject{currentmarker}{}%
\end{pgfscope}%
\end{pgfscope}%
\begin{pgfscope}%
\pgfsetrectcap%
\pgfsetmiterjoin%
\pgfsetlinewidth{0.803000pt}%
\definecolor{currentstroke}{rgb}{0.000000,0.000000,0.000000}%
\pgfsetstrokecolor{currentstroke}%
\pgfsetdash{}{0pt}%
\pgfpathmoveto{\pgfqpoint{0.688192in}{0.643904in}}%
\pgfpathlineto{\pgfqpoint{0.688192in}{5.263904in}}%
\pgfusepath{stroke}%
\end{pgfscope}%
\begin{pgfscope}%
\pgfsetrectcap%
\pgfsetmiterjoin%
\pgfsetlinewidth{0.803000pt}%
\definecolor{currentstroke}{rgb}{0.000000,0.000000,0.000000}%
\pgfsetstrokecolor{currentstroke}%
\pgfsetdash{}{0pt}%
\pgfpathmoveto{\pgfqpoint{6.888192in}{0.643904in}}%
\pgfpathlineto{\pgfqpoint{6.888192in}{5.263904in}}%
\pgfusepath{stroke}%
\end{pgfscope}%
\begin{pgfscope}%
\pgfsetrectcap%
\pgfsetmiterjoin%
\pgfsetlinewidth{0.803000pt}%
\definecolor{currentstroke}{rgb}{0.000000,0.000000,0.000000}%
\pgfsetstrokecolor{currentstroke}%
\pgfsetdash{}{0pt}%
\pgfpathmoveto{\pgfqpoint{0.688192in}{0.643904in}}%
\pgfpathlineto{\pgfqpoint{6.888192in}{0.643904in}}%
\pgfusepath{stroke}%
\end{pgfscope}%
\begin{pgfscope}%
\pgfsetrectcap%
\pgfsetmiterjoin%
\pgfsetlinewidth{0.803000pt}%
\definecolor{currentstroke}{rgb}{0.000000,0.000000,0.000000}%
\pgfsetstrokecolor{currentstroke}%
\pgfsetdash{}{0pt}%
\pgfpathmoveto{\pgfqpoint{0.688192in}{5.263904in}}%
\pgfpathlineto{\pgfqpoint{6.888192in}{5.263904in}}%
\pgfusepath{stroke}%
\end{pgfscope}%
\begin{pgfscope}%
\pgfsetbuttcap%
\pgfsetmiterjoin%
\definecolor{currentfill}{rgb}{1.000000,1.000000,1.000000}%
\pgfsetfillcolor{currentfill}%
\pgfsetfillopacity{0.800000}%
\pgfsetlinewidth{1.003750pt}%
\definecolor{currentstroke}{rgb}{0.800000,0.800000,0.800000}%
\pgfsetstrokecolor{currentstroke}%
\pgfsetstrokeopacity{0.800000}%
\pgfsetdash{}{0pt}%
\pgfpathmoveto{\pgfqpoint{0.824303in}{3.458351in}}%
\pgfpathlineto{\pgfqpoint{2.578541in}{3.458351in}}%
\pgfpathquadraticcurveto{\pgfqpoint{2.617430in}{3.458351in}}{\pgfqpoint{2.617430in}{3.497240in}}%
\pgfpathlineto{\pgfqpoint{2.617430in}{5.127793in}}%
\pgfpathquadraticcurveto{\pgfqpoint{2.617430in}{5.166682in}}{\pgfqpoint{2.578541in}{5.166682in}}%
\pgfpathlineto{\pgfqpoint{0.824303in}{5.166682in}}%
\pgfpathquadraticcurveto{\pgfqpoint{0.785414in}{5.166682in}}{\pgfqpoint{0.785414in}{5.127793in}}%
\pgfpathlineto{\pgfqpoint{0.785414in}{3.497240in}}%
\pgfpathquadraticcurveto{\pgfqpoint{0.785414in}{3.458351in}}{\pgfqpoint{0.824303in}{3.458351in}}%
\pgfpathlineto{\pgfqpoint{0.824303in}{3.458351in}}%
\pgfpathclose%
\pgfusepath{stroke,fill}%
\end{pgfscope}%
\begin{pgfscope}%
\definecolor{textcolor}{rgb}{0.000000,0.000000,0.000000}%
\pgfsetstrokecolor{textcolor}%
\pgfsetfillcolor{textcolor}%
\pgftext[x=0.863192in,y=4.950015in,left,base]{\color{textcolor}{\rmfamily\fontsize{14.000000}{16.800000}\selectfont\catcode`\^=\active\def^{\ifmmode\sp\else\^{}\fi}\catcode`\%=\active\def%{\%}Number of Threads}}%
\end{pgfscope}%
\begin{pgfscope}%
\pgfsetrectcap%
\pgfsetroundjoin%
\pgfsetlinewidth{1.505625pt}%
\definecolor{currentstroke}{rgb}{0.007843,0.243137,1.000000}%
\pgfsetstrokecolor{currentstroke}%
\pgfsetdash{}{0pt}%
\pgfpathmoveto{\pgfqpoint{1.331284in}{4.743071in}}%
\pgfpathlineto{\pgfqpoint{1.525729in}{4.743071in}}%
\pgfpathlineto{\pgfqpoint{1.720173in}{4.743071in}}%
\pgfusepath{stroke}%
\end{pgfscope}%
\begin{pgfscope}%
\pgfsetbuttcap%
\pgfsetroundjoin%
\definecolor{currentfill}{rgb}{0.007843,0.243137,1.000000}%
\pgfsetfillcolor{currentfill}%
\pgfsetlinewidth{0.752812pt}%
\definecolor{currentstroke}{rgb}{1.000000,1.000000,1.000000}%
\pgfsetstrokecolor{currentstroke}%
\pgfsetdash{}{0pt}%
\pgfsys@defobject{currentmarker}{\pgfqpoint{-0.041667in}{-0.041667in}}{\pgfqpoint{0.041667in}{0.041667in}}{%
\pgfpathmoveto{\pgfqpoint{0.000000in}{-0.041667in}}%
\pgfpathcurveto{\pgfqpoint{0.011050in}{-0.041667in}}{\pgfqpoint{0.021649in}{-0.037276in}}{\pgfqpoint{0.029463in}{-0.029463in}}%
\pgfpathcurveto{\pgfqpoint{0.037276in}{-0.021649in}}{\pgfqpoint{0.041667in}{-0.011050in}}{\pgfqpoint{0.041667in}{0.000000in}}%
\pgfpathcurveto{\pgfqpoint{0.041667in}{0.011050in}}{\pgfqpoint{0.037276in}{0.021649in}}{\pgfqpoint{0.029463in}{0.029463in}}%
\pgfpathcurveto{\pgfqpoint{0.021649in}{0.037276in}}{\pgfqpoint{0.011050in}{0.041667in}}{\pgfqpoint{0.000000in}{0.041667in}}%
\pgfpathcurveto{\pgfqpoint{-0.011050in}{0.041667in}}{\pgfqpoint{-0.021649in}{0.037276in}}{\pgfqpoint{-0.029463in}{0.029463in}}%
\pgfpathcurveto{\pgfqpoint{-0.037276in}{0.021649in}}{\pgfqpoint{-0.041667in}{0.011050in}}{\pgfqpoint{-0.041667in}{0.000000in}}%
\pgfpathcurveto{\pgfqpoint{-0.041667in}{-0.011050in}}{\pgfqpoint{-0.037276in}{-0.021649in}}{\pgfqpoint{-0.029463in}{-0.029463in}}%
\pgfpathcurveto{\pgfqpoint{-0.021649in}{-0.037276in}}{\pgfqpoint{-0.011050in}{-0.041667in}}{\pgfqpoint{0.000000in}{-0.041667in}}%
\pgfpathlineto{\pgfqpoint{0.000000in}{-0.041667in}}%
\pgfpathclose%
\pgfusepath{stroke,fill}%
}%
\begin{pgfscope}%
\pgfsys@transformshift{1.525729in}{4.743071in}%
\pgfsys@useobject{currentmarker}{}%
\end{pgfscope}%
\end{pgfscope}%
\begin{pgfscope}%
\definecolor{textcolor}{rgb}{0.000000,0.000000,0.000000}%
\pgfsetstrokecolor{textcolor}%
\pgfsetfillcolor{textcolor}%
\pgftext[x=1.875729in,y=4.675016in,left,base]{\color{textcolor}{\rmfamily\fontsize{14.000000}{16.800000}\selectfont\catcode`\^=\active\def^{\ifmmode\sp\else\^{}\fi}\catcode`\%=\active\def%{\%}1}}%
\end{pgfscope}%
\begin{pgfscope}%
\pgfsetbuttcap%
\pgfsetroundjoin%
\pgfsetlinewidth{1.505625pt}%
\definecolor{currentstroke}{rgb}{1.000000,0.486275,0.000000}%
\pgfsetstrokecolor{currentstroke}%
\pgfsetdash{{6.000000pt}{2.250000pt}}{0.000000pt}%
\pgfpathmoveto{\pgfqpoint{1.331284in}{4.468072in}}%
\pgfpathlineto{\pgfqpoint{1.525729in}{4.468072in}}%
\pgfpathlineto{\pgfqpoint{1.720173in}{4.468072in}}%
\pgfusepath{stroke}%
\end{pgfscope}%
\begin{pgfscope}%
\pgfsetbuttcap%
\pgfsetroundjoin%
\definecolor{currentfill}{rgb}{1.000000,0.486275,0.000000}%
\pgfsetfillcolor{currentfill}%
\pgfsetlinewidth{0.752812pt}%
\definecolor{currentstroke}{rgb}{1.000000,1.000000,1.000000}%
\pgfsetstrokecolor{currentstroke}%
\pgfsetdash{}{0pt}%
\pgfsys@defobject{currentmarker}{\pgfqpoint{-0.041667in}{-0.041667in}}{\pgfqpoint{0.041667in}{0.041667in}}{%
\pgfpathmoveto{\pgfqpoint{0.000000in}{-0.041667in}}%
\pgfpathcurveto{\pgfqpoint{0.011050in}{-0.041667in}}{\pgfqpoint{0.021649in}{-0.037276in}}{\pgfqpoint{0.029463in}{-0.029463in}}%
\pgfpathcurveto{\pgfqpoint{0.037276in}{-0.021649in}}{\pgfqpoint{0.041667in}{-0.011050in}}{\pgfqpoint{0.041667in}{0.000000in}}%
\pgfpathcurveto{\pgfqpoint{0.041667in}{0.011050in}}{\pgfqpoint{0.037276in}{0.021649in}}{\pgfqpoint{0.029463in}{0.029463in}}%
\pgfpathcurveto{\pgfqpoint{0.021649in}{0.037276in}}{\pgfqpoint{0.011050in}{0.041667in}}{\pgfqpoint{0.000000in}{0.041667in}}%
\pgfpathcurveto{\pgfqpoint{-0.011050in}{0.041667in}}{\pgfqpoint{-0.021649in}{0.037276in}}{\pgfqpoint{-0.029463in}{0.029463in}}%
\pgfpathcurveto{\pgfqpoint{-0.037276in}{0.021649in}}{\pgfqpoint{-0.041667in}{0.011050in}}{\pgfqpoint{-0.041667in}{0.000000in}}%
\pgfpathcurveto{\pgfqpoint{-0.041667in}{-0.011050in}}{\pgfqpoint{-0.037276in}{-0.021649in}}{\pgfqpoint{-0.029463in}{-0.029463in}}%
\pgfpathcurveto{\pgfqpoint{-0.021649in}{-0.037276in}}{\pgfqpoint{-0.011050in}{-0.041667in}}{\pgfqpoint{0.000000in}{-0.041667in}}%
\pgfpathlineto{\pgfqpoint{0.000000in}{-0.041667in}}%
\pgfpathclose%
\pgfusepath{stroke,fill}%
}%
\begin{pgfscope}%
\pgfsys@transformshift{1.525729in}{4.468072in}%
\pgfsys@useobject{currentmarker}{}%
\end{pgfscope}%
\end{pgfscope}%
\begin{pgfscope}%
\definecolor{textcolor}{rgb}{0.000000,0.000000,0.000000}%
\pgfsetstrokecolor{textcolor}%
\pgfsetfillcolor{textcolor}%
\pgftext[x=1.875729in,y=4.400016in,left,base]{\color{textcolor}{\rmfamily\fontsize{14.000000}{16.800000}\selectfont\catcode`\^=\active\def^{\ifmmode\sp\else\^{}\fi}\catcode`\%=\active\def%{\%}2}}%
\end{pgfscope}%
\begin{pgfscope}%
\pgfsetbuttcap%
\pgfsetroundjoin%
\pgfsetlinewidth{1.505625pt}%
\definecolor{currentstroke}{rgb}{0.101961,0.788235,0.219608}%
\pgfsetstrokecolor{currentstroke}%
\pgfsetdash{{1.500000pt}{1.500000pt}}{0.000000pt}%
\pgfpathmoveto{\pgfqpoint{1.331284in}{4.193072in}}%
\pgfpathlineto{\pgfqpoint{1.525729in}{4.193072in}}%
\pgfpathlineto{\pgfqpoint{1.720173in}{4.193072in}}%
\pgfusepath{stroke}%
\end{pgfscope}%
\begin{pgfscope}%
\pgfsetbuttcap%
\pgfsetroundjoin%
\definecolor{currentfill}{rgb}{0.101961,0.788235,0.219608}%
\pgfsetfillcolor{currentfill}%
\pgfsetlinewidth{0.752812pt}%
\definecolor{currentstroke}{rgb}{1.000000,1.000000,1.000000}%
\pgfsetstrokecolor{currentstroke}%
\pgfsetdash{}{0pt}%
\pgfsys@defobject{currentmarker}{\pgfqpoint{-0.041667in}{-0.041667in}}{\pgfqpoint{0.041667in}{0.041667in}}{%
\pgfpathmoveto{\pgfqpoint{0.000000in}{-0.041667in}}%
\pgfpathcurveto{\pgfqpoint{0.011050in}{-0.041667in}}{\pgfqpoint{0.021649in}{-0.037276in}}{\pgfqpoint{0.029463in}{-0.029463in}}%
\pgfpathcurveto{\pgfqpoint{0.037276in}{-0.021649in}}{\pgfqpoint{0.041667in}{-0.011050in}}{\pgfqpoint{0.041667in}{0.000000in}}%
\pgfpathcurveto{\pgfqpoint{0.041667in}{0.011050in}}{\pgfqpoint{0.037276in}{0.021649in}}{\pgfqpoint{0.029463in}{0.029463in}}%
\pgfpathcurveto{\pgfqpoint{0.021649in}{0.037276in}}{\pgfqpoint{0.011050in}{0.041667in}}{\pgfqpoint{0.000000in}{0.041667in}}%
\pgfpathcurveto{\pgfqpoint{-0.011050in}{0.041667in}}{\pgfqpoint{-0.021649in}{0.037276in}}{\pgfqpoint{-0.029463in}{0.029463in}}%
\pgfpathcurveto{\pgfqpoint{-0.037276in}{0.021649in}}{\pgfqpoint{-0.041667in}{0.011050in}}{\pgfqpoint{-0.041667in}{0.000000in}}%
\pgfpathcurveto{\pgfqpoint{-0.041667in}{-0.011050in}}{\pgfqpoint{-0.037276in}{-0.021649in}}{\pgfqpoint{-0.029463in}{-0.029463in}}%
\pgfpathcurveto{\pgfqpoint{-0.021649in}{-0.037276in}}{\pgfqpoint{-0.011050in}{-0.041667in}}{\pgfqpoint{0.000000in}{-0.041667in}}%
\pgfpathlineto{\pgfqpoint{0.000000in}{-0.041667in}}%
\pgfpathclose%
\pgfusepath{stroke,fill}%
}%
\begin{pgfscope}%
\pgfsys@transformshift{1.525729in}{4.193072in}%
\pgfsys@useobject{currentmarker}{}%
\end{pgfscope}%
\end{pgfscope}%
\begin{pgfscope}%
\definecolor{textcolor}{rgb}{0.000000,0.000000,0.000000}%
\pgfsetstrokecolor{textcolor}%
\pgfsetfillcolor{textcolor}%
\pgftext[x=1.875729in,y=4.125016in,left,base]{\color{textcolor}{\rmfamily\fontsize{14.000000}{16.800000}\selectfont\catcode`\^=\active\def^{\ifmmode\sp\else\^{}\fi}\catcode`\%=\active\def%{\%}4}}%
\end{pgfscope}%
\begin{pgfscope}%
\pgfsetbuttcap%
\pgfsetroundjoin%
\pgfsetlinewidth{1.505625pt}%
\definecolor{currentstroke}{rgb}{0.909804,0.000000,0.043137}%
\pgfsetstrokecolor{currentstroke}%
\pgfsetdash{{4.500000pt}{1.875000pt}{2.250000pt}{1.875000pt}}{0.000000pt}%
\pgfpathmoveto{\pgfqpoint{1.331284in}{3.918072in}}%
\pgfpathlineto{\pgfqpoint{1.525729in}{3.918072in}}%
\pgfpathlineto{\pgfqpoint{1.720173in}{3.918072in}}%
\pgfusepath{stroke}%
\end{pgfscope}%
\begin{pgfscope}%
\pgfsetbuttcap%
\pgfsetroundjoin%
\definecolor{currentfill}{rgb}{0.909804,0.000000,0.043137}%
\pgfsetfillcolor{currentfill}%
\pgfsetlinewidth{0.752812pt}%
\definecolor{currentstroke}{rgb}{1.000000,1.000000,1.000000}%
\pgfsetstrokecolor{currentstroke}%
\pgfsetdash{}{0pt}%
\pgfsys@defobject{currentmarker}{\pgfqpoint{-0.041667in}{-0.041667in}}{\pgfqpoint{0.041667in}{0.041667in}}{%
\pgfpathmoveto{\pgfqpoint{0.000000in}{-0.041667in}}%
\pgfpathcurveto{\pgfqpoint{0.011050in}{-0.041667in}}{\pgfqpoint{0.021649in}{-0.037276in}}{\pgfqpoint{0.029463in}{-0.029463in}}%
\pgfpathcurveto{\pgfqpoint{0.037276in}{-0.021649in}}{\pgfqpoint{0.041667in}{-0.011050in}}{\pgfqpoint{0.041667in}{0.000000in}}%
\pgfpathcurveto{\pgfqpoint{0.041667in}{0.011050in}}{\pgfqpoint{0.037276in}{0.021649in}}{\pgfqpoint{0.029463in}{0.029463in}}%
\pgfpathcurveto{\pgfqpoint{0.021649in}{0.037276in}}{\pgfqpoint{0.011050in}{0.041667in}}{\pgfqpoint{0.000000in}{0.041667in}}%
\pgfpathcurveto{\pgfqpoint{-0.011050in}{0.041667in}}{\pgfqpoint{-0.021649in}{0.037276in}}{\pgfqpoint{-0.029463in}{0.029463in}}%
\pgfpathcurveto{\pgfqpoint{-0.037276in}{0.021649in}}{\pgfqpoint{-0.041667in}{0.011050in}}{\pgfqpoint{-0.041667in}{0.000000in}}%
\pgfpathcurveto{\pgfqpoint{-0.041667in}{-0.011050in}}{\pgfqpoint{-0.037276in}{-0.021649in}}{\pgfqpoint{-0.029463in}{-0.029463in}}%
\pgfpathcurveto{\pgfqpoint{-0.021649in}{-0.037276in}}{\pgfqpoint{-0.011050in}{-0.041667in}}{\pgfqpoint{0.000000in}{-0.041667in}}%
\pgfpathlineto{\pgfqpoint{0.000000in}{-0.041667in}}%
\pgfpathclose%
\pgfusepath{stroke,fill}%
}%
\begin{pgfscope}%
\pgfsys@transformshift{1.525729in}{3.918072in}%
\pgfsys@useobject{currentmarker}{}%
\end{pgfscope}%
\end{pgfscope}%
\begin{pgfscope}%
\definecolor{textcolor}{rgb}{0.000000,0.000000,0.000000}%
\pgfsetstrokecolor{textcolor}%
\pgfsetfillcolor{textcolor}%
\pgftext[x=1.875729in,y=3.850017in,left,base]{\color{textcolor}{\rmfamily\fontsize{14.000000}{16.800000}\selectfont\catcode`\^=\active\def^{\ifmmode\sp\else\^{}\fi}\catcode`\%=\active\def%{\%}8}}%
\end{pgfscope}%
\begin{pgfscope}%
\pgfsetbuttcap%
\pgfsetroundjoin%
\pgfsetlinewidth{1.505625pt}%
\definecolor{currentstroke}{rgb}{0.545098,0.168627,0.886275}%
\pgfsetstrokecolor{currentstroke}%
\pgfsetdash{{7.500000pt}{1.500000pt}{1.500000pt}{1.500000pt}}{0.000000pt}%
\pgfpathmoveto{\pgfqpoint{1.331284in}{3.643073in}}%
\pgfpathlineto{\pgfqpoint{1.525729in}{3.643073in}}%
\pgfpathlineto{\pgfqpoint{1.720173in}{3.643073in}}%
\pgfusepath{stroke}%
\end{pgfscope}%
\begin{pgfscope}%
\pgfsetbuttcap%
\pgfsetroundjoin%
\definecolor{currentfill}{rgb}{0.545098,0.168627,0.886275}%
\pgfsetfillcolor{currentfill}%
\pgfsetlinewidth{0.752812pt}%
\definecolor{currentstroke}{rgb}{1.000000,1.000000,1.000000}%
\pgfsetstrokecolor{currentstroke}%
\pgfsetdash{}{0pt}%
\pgfsys@defobject{currentmarker}{\pgfqpoint{-0.041667in}{-0.041667in}}{\pgfqpoint{0.041667in}{0.041667in}}{%
\pgfpathmoveto{\pgfqpoint{0.000000in}{-0.041667in}}%
\pgfpathcurveto{\pgfqpoint{0.011050in}{-0.041667in}}{\pgfqpoint{0.021649in}{-0.037276in}}{\pgfqpoint{0.029463in}{-0.029463in}}%
\pgfpathcurveto{\pgfqpoint{0.037276in}{-0.021649in}}{\pgfqpoint{0.041667in}{-0.011050in}}{\pgfqpoint{0.041667in}{0.000000in}}%
\pgfpathcurveto{\pgfqpoint{0.041667in}{0.011050in}}{\pgfqpoint{0.037276in}{0.021649in}}{\pgfqpoint{0.029463in}{0.029463in}}%
\pgfpathcurveto{\pgfqpoint{0.021649in}{0.037276in}}{\pgfqpoint{0.011050in}{0.041667in}}{\pgfqpoint{0.000000in}{0.041667in}}%
\pgfpathcurveto{\pgfqpoint{-0.011050in}{0.041667in}}{\pgfqpoint{-0.021649in}{0.037276in}}{\pgfqpoint{-0.029463in}{0.029463in}}%
\pgfpathcurveto{\pgfqpoint{-0.037276in}{0.021649in}}{\pgfqpoint{-0.041667in}{0.011050in}}{\pgfqpoint{-0.041667in}{0.000000in}}%
\pgfpathcurveto{\pgfqpoint{-0.041667in}{-0.011050in}}{\pgfqpoint{-0.037276in}{-0.021649in}}{\pgfqpoint{-0.029463in}{-0.029463in}}%
\pgfpathcurveto{\pgfqpoint{-0.021649in}{-0.037276in}}{\pgfqpoint{-0.011050in}{-0.041667in}}{\pgfqpoint{0.000000in}{-0.041667in}}%
\pgfpathlineto{\pgfqpoint{0.000000in}{-0.041667in}}%
\pgfpathclose%
\pgfusepath{stroke,fill}%
}%
\begin{pgfscope}%
\pgfsys@transformshift{1.525729in}{3.643073in}%
\pgfsys@useobject{currentmarker}{}%
\end{pgfscope}%
\end{pgfscope}%
\begin{pgfscope}%
\definecolor{textcolor}{rgb}{0.000000,0.000000,0.000000}%
\pgfsetstrokecolor{textcolor}%
\pgfsetfillcolor{textcolor}%
\pgftext[x=1.875729in,y=3.575017in,left,base]{\color{textcolor}{\rmfamily\fontsize{14.000000}{16.800000}\selectfont\catcode`\^=\active\def^{\ifmmode\sp\else\^{}\fi}\catcode`\%=\active\def%{\%}12}}%
\end{pgfscope}%
\end{pgfpicture}%
\makeatother%
\endgroup%
}
    \caption{Time scaling of a capacity expansion problem using a range of
    threads for parallelization.}
    \label{fig:thread-scaling}
\end{figure}

All simulations perform similarly when the problem size is small
because multithreading has some overhead. Multiple threads outperformed the
single threaded simulation in every case. Eight threads outperformed twelve
threads in the middle of the population range, but otherwise performed
similarly. Simulations with two and four threads performed similarly until the
higher end of the population range where four threads proved faster. The overall
speed improvement from multithreading was modest with a roughly four second
improvement at best. This suggests that further code optimization could improve
performance and that better computer architecture might be needed to fully
realize the parallelization enhancement.

\FloatBarrier