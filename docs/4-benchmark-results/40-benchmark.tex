\ac{osier} is a key tool that will be leveraged in the analysis of subsequent
work. Accordingly, it must be shown that \ac{osier} generates reproducible and
reliable, results consistent with the results from an established framework.
This chapter has three objectives, the first is to illustrate some of the
differences between the two dispatch algorithms introduced in Section
\ref{section:dispatch_model}. The logical and optimal dispatch algorithms will
be tested on some simple cases. Then the time scaling of the two algorithms will
be compared. The second goal of this chapter, is to show that solutions
calculated by \ac{osier} agree with a more established \ac{esom}, \ac{temoa}.
Finally, this chapter demonstrates some of \ac{osier}'s advanced features, such
as many-objective objective problems and combining \ac{moo} with \ac{mga}.
The next section introduce some of the data and methods used to prepare the 
example problems that follow.

\section{Data for the dispatch algorithm comparison}
\subsection{Technology Data}
Another feature of \ac{osier} is automatically exporting technology data to a
\texttt{pandas} dataframe or a \LaTeX table. Table \ref{tab:tech-table}
summarizes the technology data used in this thesis and was generated by
\ac{osier}. The cost data comes from \ac{nrel}'s \ac{atb}
\cite{national_renewable_energy_laboratory_2023_2023}. Carbon intensity data
come from a life cycle analysis from the \ac{unece}
\cite{united_nations_economic_commission_for_europe_carbon_2022}.

\textcolor{red}{CITE OTHER SOURCES HERE!}

\begin{sidewaystable}[!ht]
  \centering
  \caption{Summary of Technologies and Parameters available in \ac{osier}. This
  table was generated by \ac{osier}.}
  \label{tab:tech-table}
  \resizebox{\textheight}{!}{
  \input{tables/technology_database2}
  } % end resize box
\end{sidewaystable}

\FloatBarrier
\subsubsection{Energy Demand}
The synthetic demand data for this exercise were generated with

\begin{align}
    E(t) = -\sin\left(\frac{\pi t}{12}\right) + \sin\left(\frac{\pi t}{8760}\right) + y 
\end{align}

\subsubsection{Wind Speed}
\section{Comparing dispatch algorithms}
The first exercise in this section directly compares the results of two dispatch algorithms
introduced in Section \ref{section:dispatch_model} to confirm that the two algorithms
give similar answers. The next exercise shows how the two algorithms scale. The final exercise
focuses on the logical dispatch algorithm and investigates how that algorithm scales with the
number of threads used for parallelization.
% \subsection{Exercise 0: Demonstrating \ac{osier}}
\textcolor{red}{This section is new in the dissertation!}

This section demonstrates some of the basic features of Osier. This section should be based 
on the examples in \ac{osier}'s documentation.

\subsection{Exercise 1: Validating a Simplified Approach}

This exercise considers three different cases with different technology mixes.
The first case includes natural gas and nuclear resources, the second case adds
a wind resource, and the last case adds battery storage. Each case had seven
days with an hourly resolution (168 timesteps). Both algorithms were allowed to
curtail excess energy and both were required to meet demand at all time steps.
Table \ref{tab:dispatch-results} summarizes the technologies available, the
optimizer used, and the value of the objective function. 

\begin{table}[ht!]
    \centering
    \caption{Summary results for the three dispatch test cases.}
    \label{tab:dispatch-results}
    \begin{tabular}{llllllr}
\toprule
Case & Natural Gas & Nuclear & Wind & Storage & Optimizer & Value \\
\midrule
1&\checkmark&\checkmark&&& Logical & 1.84954 \\
1&\checkmark&\checkmark&&& Optimal & 1.84954 \\
2&\checkmark&\checkmark&\checkmark&& Logical & 0.85499 \\
2&\checkmark&\checkmark&\checkmark&& Optimal & 0.85499 \\
3&\checkmark&\checkmark&\checkmark&\checkmark& Logical & 0.73009 \\
3&\checkmark&\checkmark&\checkmark&\checkmark& Optimal & 0.58797 \\
\bottomrule
\end{tabular}

\end{table}

% \noindent
The first two cases in Table \ref{tab:dispatch-results} show perfect agreement between 
the two algorithms. However, they disagree on the final case
with a battery storage technology. Figure \ref{fig:dispatch-comparison} compares
the dispatch results for the two methods. Figure
\ref{fig:dispatch-comparison}a was calculated with the logical dispatch
algorithm and Figure \ref{fig:dispatch-comparison}b was calculated with the
optimal dispatch algorithm. These plots show that the two
algorithms dispatch the same amounts of wind and nuclear energy. However, the
two algorithms differ in their usage of battery storage which causes further
differences in the dispatch of natural gas and total curtailment.

\begin{figure}[ht!]
    \centering
    \resizebox{0.95\columnwidth}{!}{%% Creator: Matplotlib, PGF backend
%%
%% To include the figure in your LaTeX document, write
%%   \input{<filename>.pgf}
%%
%% Make sure the required packages are loaded in your preamble
%%   \usepackage{pgf}
%%
%% Also ensure that all the required font packages are loaded; for instance,
%% the lmodern package is sometimes necessary when using math font.
%%   \usepackage{lmodern}
%%
%% Figures using additional raster images can only be included by \input if
%% they are in the same directory as the main LaTeX file. For loading figures
%% from other directories you can use the `import` package
%%   \usepackage{import}
%%
%% and then include the figures with
%%   \import{<path to file>}{<filename>.pgf}
%%
%% Matplotlib used the following preamble
%%   \def\mathdefault#1{#1}
%%   \everymath=\expandafter{\the\everymath\displaystyle}
%%   \IfFileExists{scrextend.sty}{
%%     \usepackage[fontsize=10.000000pt]{scrextend}
%%   }{
%%     \renewcommand{\normalsize}{\fontsize{10.000000}{12.000000}\selectfont}
%%     \normalsize
%%   }
%%   
%%   \makeatletter\@ifpackageloaded{underscore}{}{\usepackage[strings]{underscore}}\makeatother
%%
\begingroup%
\makeatletter%
\begin{pgfpicture}%
\pgfpathrectangle{\pgfpointorigin}{\pgfqpoint{9.900000in}{7.900000in}}%
\pgfusepath{use as bounding box, clip}%
\begin{pgfscope}%
\pgfsetbuttcap%
\pgfsetmiterjoin%
\definecolor{currentfill}{rgb}{1.000000,1.000000,1.000000}%
\pgfsetfillcolor{currentfill}%
\pgfsetlinewidth{0.000000pt}%
\definecolor{currentstroke}{rgb}{0.000000,0.000000,0.000000}%
\pgfsetstrokecolor{currentstroke}%
\pgfsetdash{}{0pt}%
\pgfpathmoveto{\pgfqpoint{0.000000in}{0.000000in}}%
\pgfpathlineto{\pgfqpoint{9.900000in}{0.000000in}}%
\pgfpathlineto{\pgfqpoint{9.900000in}{7.900000in}}%
\pgfpathlineto{\pgfqpoint{0.000000in}{7.900000in}}%
\pgfpathlineto{\pgfqpoint{0.000000in}{0.000000in}}%
\pgfpathclose%
\pgfusepath{fill}%
\end{pgfscope}%
\begin{pgfscope}%
\pgfsetbuttcap%
\pgfsetmiterjoin%
\definecolor{currentfill}{rgb}{1.000000,1.000000,1.000000}%
\pgfsetfillcolor{currentfill}%
\pgfsetlinewidth{0.000000pt}%
\definecolor{currentstroke}{rgb}{0.000000,0.000000,0.000000}%
\pgfsetstrokecolor{currentstroke}%
\pgfsetstrokeopacity{0.000000}%
\pgfsetdash{}{0pt}%
\pgfpathmoveto{\pgfqpoint{0.941663in}{4.334375in}}%
\pgfpathlineto{\pgfqpoint{9.800000in}{4.334375in}}%
\pgfpathlineto{\pgfqpoint{9.800000in}{7.800000in}}%
\pgfpathlineto{\pgfqpoint{0.941663in}{7.800000in}}%
\pgfpathlineto{\pgfqpoint{0.941663in}{4.334375in}}%
\pgfpathclose%
\pgfusepath{fill}%
\end{pgfscope}%
\begin{pgfscope}%
\pgfpathrectangle{\pgfqpoint{0.941663in}{4.334375in}}{\pgfqpoint{8.858337in}{3.465625in}}%
\pgfusepath{clip}%
\pgfsetrectcap%
\pgfsetroundjoin%
\pgfsetlinewidth{0.803000pt}%
\definecolor{currentstroke}{rgb}{0.690196,0.690196,0.690196}%
\pgfsetstrokecolor{currentstroke}%
\pgfsetdash{}{0pt}%
\pgfpathmoveto{\pgfqpoint{0.941663in}{4.334375in}}%
\pgfpathlineto{\pgfqpoint{0.941663in}{7.800000in}}%
\pgfusepath{stroke}%
\end{pgfscope}%
\begin{pgfscope}%
\pgfsetbuttcap%
\pgfsetroundjoin%
\definecolor{currentfill}{rgb}{0.000000,0.000000,0.000000}%
\pgfsetfillcolor{currentfill}%
\pgfsetlinewidth{0.803000pt}%
\definecolor{currentstroke}{rgb}{0.000000,0.000000,0.000000}%
\pgfsetstrokecolor{currentstroke}%
\pgfsetdash{}{0pt}%
\pgfsys@defobject{currentmarker}{\pgfqpoint{0.000000in}{-0.048611in}}{\pgfqpoint{0.000000in}{0.000000in}}{%
\pgfpathmoveto{\pgfqpoint{0.000000in}{0.000000in}}%
\pgfpathlineto{\pgfqpoint{0.000000in}{-0.048611in}}%
\pgfusepath{stroke,fill}%
}%
\begin{pgfscope}%
\pgfsys@transformshift{0.941663in}{4.334375in}%
\pgfsys@useobject{currentmarker}{}%
\end{pgfscope}%
\end{pgfscope}%
\begin{pgfscope}%
\pgfpathrectangle{\pgfqpoint{0.941663in}{4.334375in}}{\pgfqpoint{8.858337in}{3.465625in}}%
\pgfusepath{clip}%
\pgfsetrectcap%
\pgfsetroundjoin%
\pgfsetlinewidth{0.803000pt}%
\definecolor{currentstroke}{rgb}{0.690196,0.690196,0.690196}%
\pgfsetstrokecolor{currentstroke}%
\pgfsetdash{}{0pt}%
\pgfpathmoveto{\pgfqpoint{2.002542in}{4.334375in}}%
\pgfpathlineto{\pgfqpoint{2.002542in}{7.800000in}}%
\pgfusepath{stroke}%
\end{pgfscope}%
\begin{pgfscope}%
\pgfsetbuttcap%
\pgfsetroundjoin%
\definecolor{currentfill}{rgb}{0.000000,0.000000,0.000000}%
\pgfsetfillcolor{currentfill}%
\pgfsetlinewidth{0.803000pt}%
\definecolor{currentstroke}{rgb}{0.000000,0.000000,0.000000}%
\pgfsetstrokecolor{currentstroke}%
\pgfsetdash{}{0pt}%
\pgfsys@defobject{currentmarker}{\pgfqpoint{0.000000in}{-0.048611in}}{\pgfqpoint{0.000000in}{0.000000in}}{%
\pgfpathmoveto{\pgfqpoint{0.000000in}{0.000000in}}%
\pgfpathlineto{\pgfqpoint{0.000000in}{-0.048611in}}%
\pgfusepath{stroke,fill}%
}%
\begin{pgfscope}%
\pgfsys@transformshift{2.002542in}{4.334375in}%
\pgfsys@useobject{currentmarker}{}%
\end{pgfscope}%
\end{pgfscope}%
\begin{pgfscope}%
\pgfpathrectangle{\pgfqpoint{0.941663in}{4.334375in}}{\pgfqpoint{8.858337in}{3.465625in}}%
\pgfusepath{clip}%
\pgfsetrectcap%
\pgfsetroundjoin%
\pgfsetlinewidth{0.803000pt}%
\definecolor{currentstroke}{rgb}{0.690196,0.690196,0.690196}%
\pgfsetstrokecolor{currentstroke}%
\pgfsetdash{}{0pt}%
\pgfpathmoveto{\pgfqpoint{3.063420in}{4.334375in}}%
\pgfpathlineto{\pgfqpoint{3.063420in}{7.800000in}}%
\pgfusepath{stroke}%
\end{pgfscope}%
\begin{pgfscope}%
\pgfsetbuttcap%
\pgfsetroundjoin%
\definecolor{currentfill}{rgb}{0.000000,0.000000,0.000000}%
\pgfsetfillcolor{currentfill}%
\pgfsetlinewidth{0.803000pt}%
\definecolor{currentstroke}{rgb}{0.000000,0.000000,0.000000}%
\pgfsetstrokecolor{currentstroke}%
\pgfsetdash{}{0pt}%
\pgfsys@defobject{currentmarker}{\pgfqpoint{0.000000in}{-0.048611in}}{\pgfqpoint{0.000000in}{0.000000in}}{%
\pgfpathmoveto{\pgfqpoint{0.000000in}{0.000000in}}%
\pgfpathlineto{\pgfqpoint{0.000000in}{-0.048611in}}%
\pgfusepath{stroke,fill}%
}%
\begin{pgfscope}%
\pgfsys@transformshift{3.063420in}{4.334375in}%
\pgfsys@useobject{currentmarker}{}%
\end{pgfscope}%
\end{pgfscope}%
\begin{pgfscope}%
\pgfpathrectangle{\pgfqpoint{0.941663in}{4.334375in}}{\pgfqpoint{8.858337in}{3.465625in}}%
\pgfusepath{clip}%
\pgfsetrectcap%
\pgfsetroundjoin%
\pgfsetlinewidth{0.803000pt}%
\definecolor{currentstroke}{rgb}{0.690196,0.690196,0.690196}%
\pgfsetstrokecolor{currentstroke}%
\pgfsetdash{}{0pt}%
\pgfpathmoveto{\pgfqpoint{4.124299in}{4.334375in}}%
\pgfpathlineto{\pgfqpoint{4.124299in}{7.800000in}}%
\pgfusepath{stroke}%
\end{pgfscope}%
\begin{pgfscope}%
\pgfsetbuttcap%
\pgfsetroundjoin%
\definecolor{currentfill}{rgb}{0.000000,0.000000,0.000000}%
\pgfsetfillcolor{currentfill}%
\pgfsetlinewidth{0.803000pt}%
\definecolor{currentstroke}{rgb}{0.000000,0.000000,0.000000}%
\pgfsetstrokecolor{currentstroke}%
\pgfsetdash{}{0pt}%
\pgfsys@defobject{currentmarker}{\pgfqpoint{0.000000in}{-0.048611in}}{\pgfqpoint{0.000000in}{0.000000in}}{%
\pgfpathmoveto{\pgfqpoint{0.000000in}{0.000000in}}%
\pgfpathlineto{\pgfqpoint{0.000000in}{-0.048611in}}%
\pgfusepath{stroke,fill}%
}%
\begin{pgfscope}%
\pgfsys@transformshift{4.124299in}{4.334375in}%
\pgfsys@useobject{currentmarker}{}%
\end{pgfscope}%
\end{pgfscope}%
\begin{pgfscope}%
\pgfpathrectangle{\pgfqpoint{0.941663in}{4.334375in}}{\pgfqpoint{8.858337in}{3.465625in}}%
\pgfusepath{clip}%
\pgfsetrectcap%
\pgfsetroundjoin%
\pgfsetlinewidth{0.803000pt}%
\definecolor{currentstroke}{rgb}{0.690196,0.690196,0.690196}%
\pgfsetstrokecolor{currentstroke}%
\pgfsetdash{}{0pt}%
\pgfpathmoveto{\pgfqpoint{5.185178in}{4.334375in}}%
\pgfpathlineto{\pgfqpoint{5.185178in}{7.800000in}}%
\pgfusepath{stroke}%
\end{pgfscope}%
\begin{pgfscope}%
\pgfsetbuttcap%
\pgfsetroundjoin%
\definecolor{currentfill}{rgb}{0.000000,0.000000,0.000000}%
\pgfsetfillcolor{currentfill}%
\pgfsetlinewidth{0.803000pt}%
\definecolor{currentstroke}{rgb}{0.000000,0.000000,0.000000}%
\pgfsetstrokecolor{currentstroke}%
\pgfsetdash{}{0pt}%
\pgfsys@defobject{currentmarker}{\pgfqpoint{0.000000in}{-0.048611in}}{\pgfqpoint{0.000000in}{0.000000in}}{%
\pgfpathmoveto{\pgfqpoint{0.000000in}{0.000000in}}%
\pgfpathlineto{\pgfqpoint{0.000000in}{-0.048611in}}%
\pgfusepath{stroke,fill}%
}%
\begin{pgfscope}%
\pgfsys@transformshift{5.185178in}{4.334375in}%
\pgfsys@useobject{currentmarker}{}%
\end{pgfscope}%
\end{pgfscope}%
\begin{pgfscope}%
\pgfpathrectangle{\pgfqpoint{0.941663in}{4.334375in}}{\pgfqpoint{8.858337in}{3.465625in}}%
\pgfusepath{clip}%
\pgfsetrectcap%
\pgfsetroundjoin%
\pgfsetlinewidth{0.803000pt}%
\definecolor{currentstroke}{rgb}{0.690196,0.690196,0.690196}%
\pgfsetstrokecolor{currentstroke}%
\pgfsetdash{}{0pt}%
\pgfpathmoveto{\pgfqpoint{6.246056in}{4.334375in}}%
\pgfpathlineto{\pgfqpoint{6.246056in}{7.800000in}}%
\pgfusepath{stroke}%
\end{pgfscope}%
\begin{pgfscope}%
\pgfsetbuttcap%
\pgfsetroundjoin%
\definecolor{currentfill}{rgb}{0.000000,0.000000,0.000000}%
\pgfsetfillcolor{currentfill}%
\pgfsetlinewidth{0.803000pt}%
\definecolor{currentstroke}{rgb}{0.000000,0.000000,0.000000}%
\pgfsetstrokecolor{currentstroke}%
\pgfsetdash{}{0pt}%
\pgfsys@defobject{currentmarker}{\pgfqpoint{0.000000in}{-0.048611in}}{\pgfqpoint{0.000000in}{0.000000in}}{%
\pgfpathmoveto{\pgfqpoint{0.000000in}{0.000000in}}%
\pgfpathlineto{\pgfqpoint{0.000000in}{-0.048611in}}%
\pgfusepath{stroke,fill}%
}%
\begin{pgfscope}%
\pgfsys@transformshift{6.246056in}{4.334375in}%
\pgfsys@useobject{currentmarker}{}%
\end{pgfscope}%
\end{pgfscope}%
\begin{pgfscope}%
\pgfpathrectangle{\pgfqpoint{0.941663in}{4.334375in}}{\pgfqpoint{8.858337in}{3.465625in}}%
\pgfusepath{clip}%
\pgfsetrectcap%
\pgfsetroundjoin%
\pgfsetlinewidth{0.803000pt}%
\definecolor{currentstroke}{rgb}{0.690196,0.690196,0.690196}%
\pgfsetstrokecolor{currentstroke}%
\pgfsetdash{}{0pt}%
\pgfpathmoveto{\pgfqpoint{7.306935in}{4.334375in}}%
\pgfpathlineto{\pgfqpoint{7.306935in}{7.800000in}}%
\pgfusepath{stroke}%
\end{pgfscope}%
\begin{pgfscope}%
\pgfsetbuttcap%
\pgfsetroundjoin%
\definecolor{currentfill}{rgb}{0.000000,0.000000,0.000000}%
\pgfsetfillcolor{currentfill}%
\pgfsetlinewidth{0.803000pt}%
\definecolor{currentstroke}{rgb}{0.000000,0.000000,0.000000}%
\pgfsetstrokecolor{currentstroke}%
\pgfsetdash{}{0pt}%
\pgfsys@defobject{currentmarker}{\pgfqpoint{0.000000in}{-0.048611in}}{\pgfqpoint{0.000000in}{0.000000in}}{%
\pgfpathmoveto{\pgfqpoint{0.000000in}{0.000000in}}%
\pgfpathlineto{\pgfqpoint{0.000000in}{-0.048611in}}%
\pgfusepath{stroke,fill}%
}%
\begin{pgfscope}%
\pgfsys@transformshift{7.306935in}{4.334375in}%
\pgfsys@useobject{currentmarker}{}%
\end{pgfscope}%
\end{pgfscope}%
\begin{pgfscope}%
\pgfpathrectangle{\pgfqpoint{0.941663in}{4.334375in}}{\pgfqpoint{8.858337in}{3.465625in}}%
\pgfusepath{clip}%
\pgfsetrectcap%
\pgfsetroundjoin%
\pgfsetlinewidth{0.803000pt}%
\definecolor{currentstroke}{rgb}{0.690196,0.690196,0.690196}%
\pgfsetstrokecolor{currentstroke}%
\pgfsetdash{}{0pt}%
\pgfpathmoveto{\pgfqpoint{8.367814in}{4.334375in}}%
\pgfpathlineto{\pgfqpoint{8.367814in}{7.800000in}}%
\pgfusepath{stroke}%
\end{pgfscope}%
\begin{pgfscope}%
\pgfsetbuttcap%
\pgfsetroundjoin%
\definecolor{currentfill}{rgb}{0.000000,0.000000,0.000000}%
\pgfsetfillcolor{currentfill}%
\pgfsetlinewidth{0.803000pt}%
\definecolor{currentstroke}{rgb}{0.000000,0.000000,0.000000}%
\pgfsetstrokecolor{currentstroke}%
\pgfsetdash{}{0pt}%
\pgfsys@defobject{currentmarker}{\pgfqpoint{0.000000in}{-0.048611in}}{\pgfqpoint{0.000000in}{0.000000in}}{%
\pgfpathmoveto{\pgfqpoint{0.000000in}{0.000000in}}%
\pgfpathlineto{\pgfqpoint{0.000000in}{-0.048611in}}%
\pgfusepath{stroke,fill}%
}%
\begin{pgfscope}%
\pgfsys@transformshift{8.367814in}{4.334375in}%
\pgfsys@useobject{currentmarker}{}%
\end{pgfscope}%
\end{pgfscope}%
\begin{pgfscope}%
\pgfpathrectangle{\pgfqpoint{0.941663in}{4.334375in}}{\pgfqpoint{8.858337in}{3.465625in}}%
\pgfusepath{clip}%
\pgfsetrectcap%
\pgfsetroundjoin%
\pgfsetlinewidth{0.803000pt}%
\definecolor{currentstroke}{rgb}{0.690196,0.690196,0.690196}%
\pgfsetstrokecolor{currentstroke}%
\pgfsetdash{}{0pt}%
\pgfpathmoveto{\pgfqpoint{9.428692in}{4.334375in}}%
\pgfpathlineto{\pgfqpoint{9.428692in}{7.800000in}}%
\pgfusepath{stroke}%
\end{pgfscope}%
\begin{pgfscope}%
\pgfsetbuttcap%
\pgfsetroundjoin%
\definecolor{currentfill}{rgb}{0.000000,0.000000,0.000000}%
\pgfsetfillcolor{currentfill}%
\pgfsetlinewidth{0.803000pt}%
\definecolor{currentstroke}{rgb}{0.000000,0.000000,0.000000}%
\pgfsetstrokecolor{currentstroke}%
\pgfsetdash{}{0pt}%
\pgfsys@defobject{currentmarker}{\pgfqpoint{0.000000in}{-0.048611in}}{\pgfqpoint{0.000000in}{0.000000in}}{%
\pgfpathmoveto{\pgfqpoint{0.000000in}{0.000000in}}%
\pgfpathlineto{\pgfqpoint{0.000000in}{-0.048611in}}%
\pgfusepath{stroke,fill}%
}%
\begin{pgfscope}%
\pgfsys@transformshift{9.428692in}{4.334375in}%
\pgfsys@useobject{currentmarker}{}%
\end{pgfscope}%
\end{pgfscope}%
\begin{pgfscope}%
\pgfpathrectangle{\pgfqpoint{0.941663in}{4.334375in}}{\pgfqpoint{8.858337in}{3.465625in}}%
\pgfusepath{clip}%
\pgfsetrectcap%
\pgfsetroundjoin%
\pgfsetlinewidth{0.803000pt}%
\definecolor{currentstroke}{rgb}{0.690196,0.690196,0.690196}%
\pgfsetstrokecolor{currentstroke}%
\pgfsetdash{}{0pt}%
\pgfpathmoveto{\pgfqpoint{0.941663in}{4.859785in}}%
\pgfpathlineto{\pgfqpoint{9.800000in}{4.859785in}}%
\pgfusepath{stroke}%
\end{pgfscope}%
\begin{pgfscope}%
\pgfsetbuttcap%
\pgfsetroundjoin%
\definecolor{currentfill}{rgb}{0.000000,0.000000,0.000000}%
\pgfsetfillcolor{currentfill}%
\pgfsetlinewidth{0.803000pt}%
\definecolor{currentstroke}{rgb}{0.000000,0.000000,0.000000}%
\pgfsetstrokecolor{currentstroke}%
\pgfsetdash{}{0pt}%
\pgfsys@defobject{currentmarker}{\pgfqpoint{-0.048611in}{0.000000in}}{\pgfqpoint{-0.000000in}{0.000000in}}{%
\pgfpathmoveto{\pgfqpoint{-0.000000in}{0.000000in}}%
\pgfpathlineto{\pgfqpoint{-0.048611in}{0.000000in}}%
\pgfusepath{stroke,fill}%
}%
\begin{pgfscope}%
\pgfsys@transformshift{0.941663in}{4.859785in}%
\pgfsys@useobject{currentmarker}{}%
\end{pgfscope}%
\end{pgfscope}%
\begin{pgfscope}%
\definecolor{textcolor}{rgb}{0.000000,0.000000,0.000000}%
\pgfsetstrokecolor{textcolor}%
\pgfsetfillcolor{textcolor}%
\pgftext[x=0.395138in, y=4.790340in, left, base]{\color{textcolor}{\rmfamily\fontsize{14.000000}{16.800000}\selectfont\catcode`\^=\active\def^{\ifmmode\sp\else\^{}\fi}\catcode`\%=\active\def%{\%}$\mathdefault{\ensuremath{-}500}$}}%
\end{pgfscope}%
\begin{pgfscope}%
\pgfpathrectangle{\pgfqpoint{0.941663in}{4.334375in}}{\pgfqpoint{8.858337in}{3.465625in}}%
\pgfusepath{clip}%
\pgfsetrectcap%
\pgfsetroundjoin%
\pgfsetlinewidth{0.803000pt}%
\definecolor{currentstroke}{rgb}{0.690196,0.690196,0.690196}%
\pgfsetstrokecolor{currentstroke}%
\pgfsetdash{}{0pt}%
\pgfpathmoveto{\pgfqpoint{0.941663in}{5.555456in}}%
\pgfpathlineto{\pgfqpoint{9.800000in}{5.555456in}}%
\pgfusepath{stroke}%
\end{pgfscope}%
\begin{pgfscope}%
\pgfsetbuttcap%
\pgfsetroundjoin%
\definecolor{currentfill}{rgb}{0.000000,0.000000,0.000000}%
\pgfsetfillcolor{currentfill}%
\pgfsetlinewidth{0.803000pt}%
\definecolor{currentstroke}{rgb}{0.000000,0.000000,0.000000}%
\pgfsetstrokecolor{currentstroke}%
\pgfsetdash{}{0pt}%
\pgfsys@defobject{currentmarker}{\pgfqpoint{-0.048611in}{0.000000in}}{\pgfqpoint{-0.000000in}{0.000000in}}{%
\pgfpathmoveto{\pgfqpoint{-0.000000in}{0.000000in}}%
\pgfpathlineto{\pgfqpoint{-0.048611in}{0.000000in}}%
\pgfusepath{stroke,fill}%
}%
\begin{pgfscope}%
\pgfsys@transformshift{0.941663in}{5.555456in}%
\pgfsys@useobject{currentmarker}{}%
\end{pgfscope}%
\end{pgfscope}%
\begin{pgfscope}%
\definecolor{textcolor}{rgb}{0.000000,0.000000,0.000000}%
\pgfsetstrokecolor{textcolor}%
\pgfsetfillcolor{textcolor}%
\pgftext[x=0.746525in, y=5.486012in, left, base]{\color{textcolor}{\rmfamily\fontsize{14.000000}{16.800000}\selectfont\catcode`\^=\active\def^{\ifmmode\sp\else\^{}\fi}\catcode`\%=\active\def%{\%}$\mathdefault{0}$}}%
\end{pgfscope}%
\begin{pgfscope}%
\pgfpathrectangle{\pgfqpoint{0.941663in}{4.334375in}}{\pgfqpoint{8.858337in}{3.465625in}}%
\pgfusepath{clip}%
\pgfsetrectcap%
\pgfsetroundjoin%
\pgfsetlinewidth{0.803000pt}%
\definecolor{currentstroke}{rgb}{0.690196,0.690196,0.690196}%
\pgfsetstrokecolor{currentstroke}%
\pgfsetdash{}{0pt}%
\pgfpathmoveto{\pgfqpoint{0.941663in}{6.251128in}}%
\pgfpathlineto{\pgfqpoint{9.800000in}{6.251128in}}%
\pgfusepath{stroke}%
\end{pgfscope}%
\begin{pgfscope}%
\pgfsetbuttcap%
\pgfsetroundjoin%
\definecolor{currentfill}{rgb}{0.000000,0.000000,0.000000}%
\pgfsetfillcolor{currentfill}%
\pgfsetlinewidth{0.803000pt}%
\definecolor{currentstroke}{rgb}{0.000000,0.000000,0.000000}%
\pgfsetstrokecolor{currentstroke}%
\pgfsetdash{}{0pt}%
\pgfsys@defobject{currentmarker}{\pgfqpoint{-0.048611in}{0.000000in}}{\pgfqpoint{-0.000000in}{0.000000in}}{%
\pgfpathmoveto{\pgfqpoint{-0.000000in}{0.000000in}}%
\pgfpathlineto{\pgfqpoint{-0.048611in}{0.000000in}}%
\pgfusepath{stroke,fill}%
}%
\begin{pgfscope}%
\pgfsys@transformshift{0.941663in}{6.251128in}%
\pgfsys@useobject{currentmarker}{}%
\end{pgfscope}%
\end{pgfscope}%
\begin{pgfscope}%
\definecolor{textcolor}{rgb}{0.000000,0.000000,0.000000}%
\pgfsetstrokecolor{textcolor}%
\pgfsetfillcolor{textcolor}%
\pgftext[x=0.550694in, y=6.181684in, left, base]{\color{textcolor}{\rmfamily\fontsize{14.000000}{16.800000}\selectfont\catcode`\^=\active\def^{\ifmmode\sp\else\^{}\fi}\catcode`\%=\active\def%{\%}$\mathdefault{500}$}}%
\end{pgfscope}%
\begin{pgfscope}%
\pgfpathrectangle{\pgfqpoint{0.941663in}{4.334375in}}{\pgfqpoint{8.858337in}{3.465625in}}%
\pgfusepath{clip}%
\pgfsetrectcap%
\pgfsetroundjoin%
\pgfsetlinewidth{0.803000pt}%
\definecolor{currentstroke}{rgb}{0.690196,0.690196,0.690196}%
\pgfsetstrokecolor{currentstroke}%
\pgfsetdash{}{0pt}%
\pgfpathmoveto{\pgfqpoint{0.941663in}{6.946800in}}%
\pgfpathlineto{\pgfqpoint{9.800000in}{6.946800in}}%
\pgfusepath{stroke}%
\end{pgfscope}%
\begin{pgfscope}%
\pgfsetbuttcap%
\pgfsetroundjoin%
\definecolor{currentfill}{rgb}{0.000000,0.000000,0.000000}%
\pgfsetfillcolor{currentfill}%
\pgfsetlinewidth{0.803000pt}%
\definecolor{currentstroke}{rgb}{0.000000,0.000000,0.000000}%
\pgfsetstrokecolor{currentstroke}%
\pgfsetdash{}{0pt}%
\pgfsys@defobject{currentmarker}{\pgfqpoint{-0.048611in}{0.000000in}}{\pgfqpoint{-0.000000in}{0.000000in}}{%
\pgfpathmoveto{\pgfqpoint{-0.000000in}{0.000000in}}%
\pgfpathlineto{\pgfqpoint{-0.048611in}{0.000000in}}%
\pgfusepath{stroke,fill}%
}%
\begin{pgfscope}%
\pgfsys@transformshift{0.941663in}{6.946800in}%
\pgfsys@useobject{currentmarker}{}%
\end{pgfscope}%
\end{pgfscope}%
\begin{pgfscope}%
\definecolor{textcolor}{rgb}{0.000000,0.000000,0.000000}%
\pgfsetstrokecolor{textcolor}%
\pgfsetfillcolor{textcolor}%
\pgftext[x=0.452779in, y=6.877356in, left, base]{\color{textcolor}{\rmfamily\fontsize{14.000000}{16.800000}\selectfont\catcode`\^=\active\def^{\ifmmode\sp\else\^{}\fi}\catcode`\%=\active\def%{\%}$\mathdefault{1000}$}}%
\end{pgfscope}%
\begin{pgfscope}%
\pgfpathrectangle{\pgfqpoint{0.941663in}{4.334375in}}{\pgfqpoint{8.858337in}{3.465625in}}%
\pgfusepath{clip}%
\pgfsetrectcap%
\pgfsetroundjoin%
\pgfsetlinewidth{0.803000pt}%
\definecolor{currentstroke}{rgb}{0.690196,0.690196,0.690196}%
\pgfsetstrokecolor{currentstroke}%
\pgfsetdash{}{0pt}%
\pgfpathmoveto{\pgfqpoint{0.941663in}{7.642472in}}%
\pgfpathlineto{\pgfqpoint{9.800000in}{7.642472in}}%
\pgfusepath{stroke}%
\end{pgfscope}%
\begin{pgfscope}%
\pgfsetbuttcap%
\pgfsetroundjoin%
\definecolor{currentfill}{rgb}{0.000000,0.000000,0.000000}%
\pgfsetfillcolor{currentfill}%
\pgfsetlinewidth{0.803000pt}%
\definecolor{currentstroke}{rgb}{0.000000,0.000000,0.000000}%
\pgfsetstrokecolor{currentstroke}%
\pgfsetdash{}{0pt}%
\pgfsys@defobject{currentmarker}{\pgfqpoint{-0.048611in}{0.000000in}}{\pgfqpoint{-0.000000in}{0.000000in}}{%
\pgfpathmoveto{\pgfqpoint{-0.000000in}{0.000000in}}%
\pgfpathlineto{\pgfqpoint{-0.048611in}{0.000000in}}%
\pgfusepath{stroke,fill}%
}%
\begin{pgfscope}%
\pgfsys@transformshift{0.941663in}{7.642472in}%
\pgfsys@useobject{currentmarker}{}%
\end{pgfscope}%
\end{pgfscope}%
\begin{pgfscope}%
\definecolor{textcolor}{rgb}{0.000000,0.000000,0.000000}%
\pgfsetstrokecolor{textcolor}%
\pgfsetfillcolor{textcolor}%
\pgftext[x=0.452779in, y=7.573027in, left, base]{\color{textcolor}{\rmfamily\fontsize{14.000000}{16.800000}\selectfont\catcode`\^=\active\def^{\ifmmode\sp\else\^{}\fi}\catcode`\%=\active\def%{\%}$\mathdefault{1500}$}}%
\end{pgfscope}%
\begin{pgfscope}%
\definecolor{textcolor}{rgb}{0.000000,0.000000,0.000000}%
\pgfsetstrokecolor{textcolor}%
\pgfsetfillcolor{textcolor}%
\pgftext[x=0.339583in,y=6.067187in,,bottom,rotate=90.000000]{\color{textcolor}{\rmfamily\fontsize{18.000000}{21.600000}\selectfont\catcode`\^=\active\def^{\ifmmode\sp\else\^{}\fi}\catcode`\%=\active\def%{\%}Energy [MWh]}}%
\end{pgfscope}%
\begin{pgfscope}%
\pgfpathrectangle{\pgfqpoint{0.941663in}{4.334375in}}{\pgfqpoint{8.858337in}{3.465625in}}%
\pgfusepath{clip}%
\pgfsetrectcap%
\pgfsetroundjoin%
\pgfsetlinewidth{1.505625pt}%
\definecolor{currentstroke}{rgb}{0.121569,0.466667,0.705882}%
\pgfsetstrokecolor{currentstroke}%
\pgfsetdash{}{0pt}%
\pgfpathmoveto{\pgfqpoint{0.941663in}{6.251128in}}%
\pgfpathlineto{\pgfqpoint{9.800000in}{6.251128in}}%
\pgfpathlineto{\pgfqpoint{9.800000in}{6.251128in}}%
\pgfusepath{stroke}%
\end{pgfscope}%
\begin{pgfscope}%
\pgfpathrectangle{\pgfqpoint{0.941663in}{4.334375in}}{\pgfqpoint{8.858337in}{3.465625in}}%
\pgfusepath{clip}%
\pgfsetbuttcap%
\pgfsetroundjoin%
\definecolor{currentfill}{rgb}{0.121569,0.466667,0.705882}%
\pgfsetfillcolor{currentfill}%
\pgfsetlinewidth{1.003750pt}%
\definecolor{currentstroke}{rgb}{0.121569,0.466667,0.705882}%
\pgfsetstrokecolor{currentstroke}%
\pgfsetdash{}{0pt}%
\pgfsys@defobject{currentmarker}{\pgfqpoint{0.941663in}{5.555456in}}{\pgfqpoint{9.800000in}{6.251128in}}{%
\pgfpathmoveto{\pgfqpoint{0.941663in}{6.251128in}}%
\pgfpathlineto{\pgfqpoint{0.941663in}{5.555456in}}%
\pgfpathlineto{\pgfqpoint{0.994707in}{5.555456in}}%
\pgfpathlineto{\pgfqpoint{1.047751in}{5.555456in}}%
\pgfpathlineto{\pgfqpoint{1.100795in}{5.555456in}}%
\pgfpathlineto{\pgfqpoint{1.153839in}{5.555456in}}%
\pgfpathlineto{\pgfqpoint{1.206883in}{5.555456in}}%
\pgfpathlineto{\pgfqpoint{1.259927in}{5.555456in}}%
\pgfpathlineto{\pgfqpoint{1.312970in}{5.555456in}}%
\pgfpathlineto{\pgfqpoint{1.366014in}{5.555456in}}%
\pgfpathlineto{\pgfqpoint{1.419058in}{5.555456in}}%
\pgfpathlineto{\pgfqpoint{1.472102in}{5.555456in}}%
\pgfpathlineto{\pgfqpoint{1.525146in}{5.555456in}}%
\pgfpathlineto{\pgfqpoint{1.578190in}{5.555456in}}%
\pgfpathlineto{\pgfqpoint{1.631234in}{5.555456in}}%
\pgfpathlineto{\pgfqpoint{1.684278in}{5.555456in}}%
\pgfpathlineto{\pgfqpoint{1.737322in}{5.555456in}}%
\pgfpathlineto{\pgfqpoint{1.790366in}{5.555456in}}%
\pgfpathlineto{\pgfqpoint{1.843410in}{5.555456in}}%
\pgfpathlineto{\pgfqpoint{1.896454in}{5.555456in}}%
\pgfpathlineto{\pgfqpoint{1.949498in}{5.555456in}}%
\pgfpathlineto{\pgfqpoint{2.002542in}{5.555456in}}%
\pgfpathlineto{\pgfqpoint{2.055586in}{5.555456in}}%
\pgfpathlineto{\pgfqpoint{2.108629in}{5.555456in}}%
\pgfpathlineto{\pgfqpoint{2.161673in}{5.555456in}}%
\pgfpathlineto{\pgfqpoint{2.214717in}{5.555456in}}%
\pgfpathlineto{\pgfqpoint{2.267761in}{5.555456in}}%
\pgfpathlineto{\pgfqpoint{2.320805in}{5.555456in}}%
\pgfpathlineto{\pgfqpoint{2.373849in}{5.555456in}}%
\pgfpathlineto{\pgfqpoint{2.426893in}{5.555456in}}%
\pgfpathlineto{\pgfqpoint{2.479937in}{5.555456in}}%
\pgfpathlineto{\pgfqpoint{2.532981in}{5.555456in}}%
\pgfpathlineto{\pgfqpoint{2.586025in}{5.555456in}}%
\pgfpathlineto{\pgfqpoint{2.639069in}{5.555456in}}%
\pgfpathlineto{\pgfqpoint{2.692113in}{5.555456in}}%
\pgfpathlineto{\pgfqpoint{2.745157in}{5.555456in}}%
\pgfpathlineto{\pgfqpoint{2.798201in}{5.555456in}}%
\pgfpathlineto{\pgfqpoint{2.851245in}{5.555456in}}%
\pgfpathlineto{\pgfqpoint{2.904288in}{5.555456in}}%
\pgfpathlineto{\pgfqpoint{2.957332in}{5.555456in}}%
\pgfpathlineto{\pgfqpoint{3.010376in}{5.555456in}}%
\pgfpathlineto{\pgfqpoint{3.063420in}{5.555456in}}%
\pgfpathlineto{\pgfqpoint{3.116464in}{5.555456in}}%
\pgfpathlineto{\pgfqpoint{3.169508in}{5.555456in}}%
\pgfpathlineto{\pgfqpoint{3.222552in}{5.555456in}}%
\pgfpathlineto{\pgfqpoint{3.275596in}{5.555456in}}%
\pgfpathlineto{\pgfqpoint{3.328640in}{5.555456in}}%
\pgfpathlineto{\pgfqpoint{3.381684in}{5.555456in}}%
\pgfpathlineto{\pgfqpoint{3.434728in}{5.555456in}}%
\pgfpathlineto{\pgfqpoint{3.487772in}{5.555456in}}%
\pgfpathlineto{\pgfqpoint{3.540816in}{5.555456in}}%
\pgfpathlineto{\pgfqpoint{3.593860in}{5.555456in}}%
\pgfpathlineto{\pgfqpoint{3.646904in}{5.555456in}}%
\pgfpathlineto{\pgfqpoint{3.699948in}{5.555456in}}%
\pgfpathlineto{\pgfqpoint{3.752991in}{5.555456in}}%
\pgfpathlineto{\pgfqpoint{3.806035in}{5.555456in}}%
\pgfpathlineto{\pgfqpoint{3.859079in}{5.555456in}}%
\pgfpathlineto{\pgfqpoint{3.912123in}{5.555456in}}%
\pgfpathlineto{\pgfqpoint{3.965167in}{5.555456in}}%
\pgfpathlineto{\pgfqpoint{4.018211in}{5.555456in}}%
\pgfpathlineto{\pgfqpoint{4.071255in}{5.555456in}}%
\pgfpathlineto{\pgfqpoint{4.124299in}{5.555456in}}%
\pgfpathlineto{\pgfqpoint{4.177343in}{5.555456in}}%
\pgfpathlineto{\pgfqpoint{4.230387in}{5.555456in}}%
\pgfpathlineto{\pgfqpoint{4.283431in}{5.555456in}}%
\pgfpathlineto{\pgfqpoint{4.336475in}{5.555456in}}%
\pgfpathlineto{\pgfqpoint{4.389519in}{5.555456in}}%
\pgfpathlineto{\pgfqpoint{4.442563in}{5.555456in}}%
\pgfpathlineto{\pgfqpoint{4.495607in}{5.555456in}}%
\pgfpathlineto{\pgfqpoint{4.548650in}{5.555456in}}%
\pgfpathlineto{\pgfqpoint{4.601694in}{5.555456in}}%
\pgfpathlineto{\pgfqpoint{4.654738in}{5.555456in}}%
\pgfpathlineto{\pgfqpoint{4.707782in}{5.555456in}}%
\pgfpathlineto{\pgfqpoint{4.760826in}{5.555456in}}%
\pgfpathlineto{\pgfqpoint{4.813870in}{5.555456in}}%
\pgfpathlineto{\pgfqpoint{4.866914in}{5.555456in}}%
\pgfpathlineto{\pgfqpoint{4.919958in}{5.555456in}}%
\pgfpathlineto{\pgfqpoint{4.973002in}{5.555456in}}%
\pgfpathlineto{\pgfqpoint{5.026046in}{5.555456in}}%
\pgfpathlineto{\pgfqpoint{5.079090in}{5.555456in}}%
\pgfpathlineto{\pgfqpoint{5.132134in}{5.555456in}}%
\pgfpathlineto{\pgfqpoint{5.185178in}{5.555456in}}%
\pgfpathlineto{\pgfqpoint{5.238222in}{5.555456in}}%
\pgfpathlineto{\pgfqpoint{5.291266in}{5.555456in}}%
\pgfpathlineto{\pgfqpoint{5.344309in}{5.555456in}}%
\pgfpathlineto{\pgfqpoint{5.397353in}{5.555456in}}%
\pgfpathlineto{\pgfqpoint{5.450397in}{5.555456in}}%
\pgfpathlineto{\pgfqpoint{5.503441in}{5.555456in}}%
\pgfpathlineto{\pgfqpoint{5.556485in}{5.555456in}}%
\pgfpathlineto{\pgfqpoint{5.609529in}{5.555456in}}%
\pgfpathlineto{\pgfqpoint{5.662573in}{5.555456in}}%
\pgfpathlineto{\pgfqpoint{5.715617in}{5.555456in}}%
\pgfpathlineto{\pgfqpoint{5.768661in}{5.555456in}}%
\pgfpathlineto{\pgfqpoint{5.821705in}{5.555456in}}%
\pgfpathlineto{\pgfqpoint{5.874749in}{5.555456in}}%
\pgfpathlineto{\pgfqpoint{5.927793in}{5.555456in}}%
\pgfpathlineto{\pgfqpoint{5.980837in}{5.555456in}}%
\pgfpathlineto{\pgfqpoint{6.033881in}{5.555456in}}%
\pgfpathlineto{\pgfqpoint{6.086925in}{5.555456in}}%
\pgfpathlineto{\pgfqpoint{6.139969in}{5.555456in}}%
\pgfpathlineto{\pgfqpoint{6.193012in}{5.555456in}}%
\pgfpathlineto{\pgfqpoint{6.246056in}{5.555456in}}%
\pgfpathlineto{\pgfqpoint{6.299100in}{5.555456in}}%
\pgfpathlineto{\pgfqpoint{6.352144in}{5.555456in}}%
\pgfpathlineto{\pgfqpoint{6.405188in}{5.555456in}}%
\pgfpathlineto{\pgfqpoint{6.458232in}{5.555456in}}%
\pgfpathlineto{\pgfqpoint{6.511276in}{5.555456in}}%
\pgfpathlineto{\pgfqpoint{6.564320in}{5.555456in}}%
\pgfpathlineto{\pgfqpoint{6.617364in}{5.555456in}}%
\pgfpathlineto{\pgfqpoint{6.670408in}{5.555456in}}%
\pgfpathlineto{\pgfqpoint{6.723452in}{5.555456in}}%
\pgfpathlineto{\pgfqpoint{6.776496in}{5.555456in}}%
\pgfpathlineto{\pgfqpoint{6.829540in}{5.555456in}}%
\pgfpathlineto{\pgfqpoint{6.882584in}{5.555456in}}%
\pgfpathlineto{\pgfqpoint{6.935628in}{5.555456in}}%
\pgfpathlineto{\pgfqpoint{6.988671in}{5.555456in}}%
\pgfpathlineto{\pgfqpoint{7.041715in}{5.555456in}}%
\pgfpathlineto{\pgfqpoint{7.094759in}{5.555456in}}%
\pgfpathlineto{\pgfqpoint{7.147803in}{5.555456in}}%
\pgfpathlineto{\pgfqpoint{7.200847in}{5.555456in}}%
\pgfpathlineto{\pgfqpoint{7.253891in}{5.555456in}}%
\pgfpathlineto{\pgfqpoint{7.306935in}{5.555456in}}%
\pgfpathlineto{\pgfqpoint{7.359979in}{5.555456in}}%
\pgfpathlineto{\pgfqpoint{7.413023in}{5.555456in}}%
\pgfpathlineto{\pgfqpoint{7.466067in}{5.555456in}}%
\pgfpathlineto{\pgfqpoint{7.519111in}{5.555456in}}%
\pgfpathlineto{\pgfqpoint{7.572155in}{5.555456in}}%
\pgfpathlineto{\pgfqpoint{7.625199in}{5.555456in}}%
\pgfpathlineto{\pgfqpoint{7.678243in}{5.555456in}}%
\pgfpathlineto{\pgfqpoint{7.731287in}{5.555456in}}%
\pgfpathlineto{\pgfqpoint{7.784330in}{5.555456in}}%
\pgfpathlineto{\pgfqpoint{7.837374in}{5.555456in}}%
\pgfpathlineto{\pgfqpoint{7.890418in}{5.555456in}}%
\pgfpathlineto{\pgfqpoint{7.943462in}{5.555456in}}%
\pgfpathlineto{\pgfqpoint{7.996506in}{5.555456in}}%
\pgfpathlineto{\pgfqpoint{8.049550in}{5.555456in}}%
\pgfpathlineto{\pgfqpoint{8.102594in}{5.555456in}}%
\pgfpathlineto{\pgfqpoint{8.155638in}{5.555456in}}%
\pgfpathlineto{\pgfqpoint{8.208682in}{5.555456in}}%
\pgfpathlineto{\pgfqpoint{8.261726in}{5.555456in}}%
\pgfpathlineto{\pgfqpoint{8.314770in}{5.555456in}}%
\pgfpathlineto{\pgfqpoint{8.367814in}{5.555456in}}%
\pgfpathlineto{\pgfqpoint{8.420858in}{5.555456in}}%
\pgfpathlineto{\pgfqpoint{8.473902in}{5.555456in}}%
\pgfpathlineto{\pgfqpoint{8.526946in}{5.555456in}}%
\pgfpathlineto{\pgfqpoint{8.579990in}{5.555456in}}%
\pgfpathlineto{\pgfqpoint{8.633033in}{5.555456in}}%
\pgfpathlineto{\pgfqpoint{8.686077in}{5.555456in}}%
\pgfpathlineto{\pgfqpoint{8.739121in}{5.555456in}}%
\pgfpathlineto{\pgfqpoint{8.792165in}{5.555456in}}%
\pgfpathlineto{\pgfqpoint{8.845209in}{5.555456in}}%
\pgfpathlineto{\pgfqpoint{8.898253in}{5.555456in}}%
\pgfpathlineto{\pgfqpoint{8.951297in}{5.555456in}}%
\pgfpathlineto{\pgfqpoint{9.004341in}{5.555456in}}%
\pgfpathlineto{\pgfqpoint{9.057385in}{5.555456in}}%
\pgfpathlineto{\pgfqpoint{9.110429in}{5.555456in}}%
\pgfpathlineto{\pgfqpoint{9.163473in}{5.555456in}}%
\pgfpathlineto{\pgfqpoint{9.216517in}{5.555456in}}%
\pgfpathlineto{\pgfqpoint{9.269561in}{5.555456in}}%
\pgfpathlineto{\pgfqpoint{9.322605in}{5.555456in}}%
\pgfpathlineto{\pgfqpoint{9.375649in}{5.555456in}}%
\pgfpathlineto{\pgfqpoint{9.428692in}{5.555456in}}%
\pgfpathlineto{\pgfqpoint{9.481736in}{5.555456in}}%
\pgfpathlineto{\pgfqpoint{9.534780in}{5.555456in}}%
\pgfpathlineto{\pgfqpoint{9.587824in}{5.555456in}}%
\pgfpathlineto{\pgfqpoint{9.640868in}{5.555456in}}%
\pgfpathlineto{\pgfqpoint{9.693912in}{5.555456in}}%
\pgfpathlineto{\pgfqpoint{9.746956in}{5.555456in}}%
\pgfpathlineto{\pgfqpoint{9.800000in}{5.555456in}}%
\pgfpathlineto{\pgfqpoint{9.800000in}{6.251128in}}%
\pgfpathlineto{\pgfqpoint{9.800000in}{6.251128in}}%
\pgfpathlineto{\pgfqpoint{9.746956in}{6.251128in}}%
\pgfpathlineto{\pgfqpoint{9.693912in}{6.251128in}}%
\pgfpathlineto{\pgfqpoint{9.640868in}{6.251128in}}%
\pgfpathlineto{\pgfqpoint{9.587824in}{6.251128in}}%
\pgfpathlineto{\pgfqpoint{9.534780in}{6.251128in}}%
\pgfpathlineto{\pgfqpoint{9.481736in}{6.251128in}}%
\pgfpathlineto{\pgfqpoint{9.428692in}{6.251128in}}%
\pgfpathlineto{\pgfqpoint{9.375649in}{6.251128in}}%
\pgfpathlineto{\pgfqpoint{9.322605in}{6.251128in}}%
\pgfpathlineto{\pgfqpoint{9.269561in}{6.251128in}}%
\pgfpathlineto{\pgfqpoint{9.216517in}{6.251128in}}%
\pgfpathlineto{\pgfqpoint{9.163473in}{6.251128in}}%
\pgfpathlineto{\pgfqpoint{9.110429in}{6.251128in}}%
\pgfpathlineto{\pgfqpoint{9.057385in}{6.251128in}}%
\pgfpathlineto{\pgfqpoint{9.004341in}{6.251128in}}%
\pgfpathlineto{\pgfqpoint{8.951297in}{6.251128in}}%
\pgfpathlineto{\pgfqpoint{8.898253in}{6.251128in}}%
\pgfpathlineto{\pgfqpoint{8.845209in}{6.251128in}}%
\pgfpathlineto{\pgfqpoint{8.792165in}{6.251128in}}%
\pgfpathlineto{\pgfqpoint{8.739121in}{6.251128in}}%
\pgfpathlineto{\pgfqpoint{8.686077in}{6.251128in}}%
\pgfpathlineto{\pgfqpoint{8.633033in}{6.251128in}}%
\pgfpathlineto{\pgfqpoint{8.579990in}{6.251128in}}%
\pgfpathlineto{\pgfqpoint{8.526946in}{6.251128in}}%
\pgfpathlineto{\pgfqpoint{8.473902in}{6.251128in}}%
\pgfpathlineto{\pgfqpoint{8.420858in}{6.251128in}}%
\pgfpathlineto{\pgfqpoint{8.367814in}{6.251128in}}%
\pgfpathlineto{\pgfqpoint{8.314770in}{6.251128in}}%
\pgfpathlineto{\pgfqpoint{8.261726in}{6.251128in}}%
\pgfpathlineto{\pgfqpoint{8.208682in}{6.251128in}}%
\pgfpathlineto{\pgfqpoint{8.155638in}{6.251128in}}%
\pgfpathlineto{\pgfqpoint{8.102594in}{6.251128in}}%
\pgfpathlineto{\pgfqpoint{8.049550in}{6.251128in}}%
\pgfpathlineto{\pgfqpoint{7.996506in}{6.251128in}}%
\pgfpathlineto{\pgfqpoint{7.943462in}{6.251128in}}%
\pgfpathlineto{\pgfqpoint{7.890418in}{6.251128in}}%
\pgfpathlineto{\pgfqpoint{7.837374in}{6.251128in}}%
\pgfpathlineto{\pgfqpoint{7.784330in}{6.251128in}}%
\pgfpathlineto{\pgfqpoint{7.731287in}{6.251128in}}%
\pgfpathlineto{\pgfqpoint{7.678243in}{6.251128in}}%
\pgfpathlineto{\pgfqpoint{7.625199in}{6.251128in}}%
\pgfpathlineto{\pgfqpoint{7.572155in}{6.251128in}}%
\pgfpathlineto{\pgfqpoint{7.519111in}{6.251128in}}%
\pgfpathlineto{\pgfqpoint{7.466067in}{6.251128in}}%
\pgfpathlineto{\pgfqpoint{7.413023in}{6.251128in}}%
\pgfpathlineto{\pgfqpoint{7.359979in}{6.251128in}}%
\pgfpathlineto{\pgfqpoint{7.306935in}{6.251128in}}%
\pgfpathlineto{\pgfqpoint{7.253891in}{6.251128in}}%
\pgfpathlineto{\pgfqpoint{7.200847in}{6.251128in}}%
\pgfpathlineto{\pgfqpoint{7.147803in}{6.251128in}}%
\pgfpathlineto{\pgfqpoint{7.094759in}{6.251128in}}%
\pgfpathlineto{\pgfqpoint{7.041715in}{6.251128in}}%
\pgfpathlineto{\pgfqpoint{6.988671in}{6.251128in}}%
\pgfpathlineto{\pgfqpoint{6.935628in}{6.251128in}}%
\pgfpathlineto{\pgfqpoint{6.882584in}{6.251128in}}%
\pgfpathlineto{\pgfqpoint{6.829540in}{6.251128in}}%
\pgfpathlineto{\pgfqpoint{6.776496in}{6.251128in}}%
\pgfpathlineto{\pgfqpoint{6.723452in}{6.251128in}}%
\pgfpathlineto{\pgfqpoint{6.670408in}{6.251128in}}%
\pgfpathlineto{\pgfqpoint{6.617364in}{6.251128in}}%
\pgfpathlineto{\pgfqpoint{6.564320in}{6.251128in}}%
\pgfpathlineto{\pgfqpoint{6.511276in}{6.251128in}}%
\pgfpathlineto{\pgfqpoint{6.458232in}{6.251128in}}%
\pgfpathlineto{\pgfqpoint{6.405188in}{6.251128in}}%
\pgfpathlineto{\pgfqpoint{6.352144in}{6.251128in}}%
\pgfpathlineto{\pgfqpoint{6.299100in}{6.251128in}}%
\pgfpathlineto{\pgfqpoint{6.246056in}{6.251128in}}%
\pgfpathlineto{\pgfqpoint{6.193012in}{6.251128in}}%
\pgfpathlineto{\pgfqpoint{6.139969in}{6.251128in}}%
\pgfpathlineto{\pgfqpoint{6.086925in}{6.251128in}}%
\pgfpathlineto{\pgfqpoint{6.033881in}{6.251128in}}%
\pgfpathlineto{\pgfqpoint{5.980837in}{6.251128in}}%
\pgfpathlineto{\pgfqpoint{5.927793in}{6.251128in}}%
\pgfpathlineto{\pgfqpoint{5.874749in}{6.251128in}}%
\pgfpathlineto{\pgfqpoint{5.821705in}{6.251128in}}%
\pgfpathlineto{\pgfqpoint{5.768661in}{6.251128in}}%
\pgfpathlineto{\pgfqpoint{5.715617in}{6.251128in}}%
\pgfpathlineto{\pgfqpoint{5.662573in}{6.251128in}}%
\pgfpathlineto{\pgfqpoint{5.609529in}{6.251128in}}%
\pgfpathlineto{\pgfqpoint{5.556485in}{6.251128in}}%
\pgfpathlineto{\pgfqpoint{5.503441in}{6.251128in}}%
\pgfpathlineto{\pgfqpoint{5.450397in}{6.251128in}}%
\pgfpathlineto{\pgfqpoint{5.397353in}{6.251128in}}%
\pgfpathlineto{\pgfqpoint{5.344309in}{6.251128in}}%
\pgfpathlineto{\pgfqpoint{5.291266in}{6.251128in}}%
\pgfpathlineto{\pgfqpoint{5.238222in}{6.251128in}}%
\pgfpathlineto{\pgfqpoint{5.185178in}{6.251128in}}%
\pgfpathlineto{\pgfqpoint{5.132134in}{6.251128in}}%
\pgfpathlineto{\pgfqpoint{5.079090in}{6.251128in}}%
\pgfpathlineto{\pgfqpoint{5.026046in}{6.251128in}}%
\pgfpathlineto{\pgfqpoint{4.973002in}{6.251128in}}%
\pgfpathlineto{\pgfqpoint{4.919958in}{6.251128in}}%
\pgfpathlineto{\pgfqpoint{4.866914in}{6.251128in}}%
\pgfpathlineto{\pgfqpoint{4.813870in}{6.251128in}}%
\pgfpathlineto{\pgfqpoint{4.760826in}{6.251128in}}%
\pgfpathlineto{\pgfqpoint{4.707782in}{6.251128in}}%
\pgfpathlineto{\pgfqpoint{4.654738in}{6.251128in}}%
\pgfpathlineto{\pgfqpoint{4.601694in}{6.251128in}}%
\pgfpathlineto{\pgfqpoint{4.548650in}{6.251128in}}%
\pgfpathlineto{\pgfqpoint{4.495607in}{6.251128in}}%
\pgfpathlineto{\pgfqpoint{4.442563in}{6.251128in}}%
\pgfpathlineto{\pgfqpoint{4.389519in}{6.251128in}}%
\pgfpathlineto{\pgfqpoint{4.336475in}{6.251128in}}%
\pgfpathlineto{\pgfqpoint{4.283431in}{6.251128in}}%
\pgfpathlineto{\pgfqpoint{4.230387in}{6.251128in}}%
\pgfpathlineto{\pgfqpoint{4.177343in}{6.251128in}}%
\pgfpathlineto{\pgfqpoint{4.124299in}{6.251128in}}%
\pgfpathlineto{\pgfqpoint{4.071255in}{6.251128in}}%
\pgfpathlineto{\pgfqpoint{4.018211in}{6.251128in}}%
\pgfpathlineto{\pgfqpoint{3.965167in}{6.251128in}}%
\pgfpathlineto{\pgfqpoint{3.912123in}{6.251128in}}%
\pgfpathlineto{\pgfqpoint{3.859079in}{6.251128in}}%
\pgfpathlineto{\pgfqpoint{3.806035in}{6.251128in}}%
\pgfpathlineto{\pgfqpoint{3.752991in}{6.251128in}}%
\pgfpathlineto{\pgfqpoint{3.699948in}{6.251128in}}%
\pgfpathlineto{\pgfqpoint{3.646904in}{6.251128in}}%
\pgfpathlineto{\pgfqpoint{3.593860in}{6.251128in}}%
\pgfpathlineto{\pgfqpoint{3.540816in}{6.251128in}}%
\pgfpathlineto{\pgfqpoint{3.487772in}{6.251128in}}%
\pgfpathlineto{\pgfqpoint{3.434728in}{6.251128in}}%
\pgfpathlineto{\pgfqpoint{3.381684in}{6.251128in}}%
\pgfpathlineto{\pgfqpoint{3.328640in}{6.251128in}}%
\pgfpathlineto{\pgfqpoint{3.275596in}{6.251128in}}%
\pgfpathlineto{\pgfqpoint{3.222552in}{6.251128in}}%
\pgfpathlineto{\pgfqpoint{3.169508in}{6.251128in}}%
\pgfpathlineto{\pgfqpoint{3.116464in}{6.251128in}}%
\pgfpathlineto{\pgfqpoint{3.063420in}{6.251128in}}%
\pgfpathlineto{\pgfqpoint{3.010376in}{6.251128in}}%
\pgfpathlineto{\pgfqpoint{2.957332in}{6.251128in}}%
\pgfpathlineto{\pgfqpoint{2.904288in}{6.251128in}}%
\pgfpathlineto{\pgfqpoint{2.851245in}{6.251128in}}%
\pgfpathlineto{\pgfqpoint{2.798201in}{6.251128in}}%
\pgfpathlineto{\pgfqpoint{2.745157in}{6.251128in}}%
\pgfpathlineto{\pgfqpoint{2.692113in}{6.251128in}}%
\pgfpathlineto{\pgfqpoint{2.639069in}{6.251128in}}%
\pgfpathlineto{\pgfqpoint{2.586025in}{6.251128in}}%
\pgfpathlineto{\pgfqpoint{2.532981in}{6.251128in}}%
\pgfpathlineto{\pgfqpoint{2.479937in}{6.251128in}}%
\pgfpathlineto{\pgfqpoint{2.426893in}{6.251128in}}%
\pgfpathlineto{\pgfqpoint{2.373849in}{6.251128in}}%
\pgfpathlineto{\pgfqpoint{2.320805in}{6.251128in}}%
\pgfpathlineto{\pgfqpoint{2.267761in}{6.251128in}}%
\pgfpathlineto{\pgfqpoint{2.214717in}{6.251128in}}%
\pgfpathlineto{\pgfqpoint{2.161673in}{6.251128in}}%
\pgfpathlineto{\pgfqpoint{2.108629in}{6.251128in}}%
\pgfpathlineto{\pgfqpoint{2.055586in}{6.251128in}}%
\pgfpathlineto{\pgfqpoint{2.002542in}{6.251128in}}%
\pgfpathlineto{\pgfqpoint{1.949498in}{6.251128in}}%
\pgfpathlineto{\pgfqpoint{1.896454in}{6.251128in}}%
\pgfpathlineto{\pgfqpoint{1.843410in}{6.251128in}}%
\pgfpathlineto{\pgfqpoint{1.790366in}{6.251128in}}%
\pgfpathlineto{\pgfqpoint{1.737322in}{6.251128in}}%
\pgfpathlineto{\pgfqpoint{1.684278in}{6.251128in}}%
\pgfpathlineto{\pgfqpoint{1.631234in}{6.251128in}}%
\pgfpathlineto{\pgfqpoint{1.578190in}{6.251128in}}%
\pgfpathlineto{\pgfqpoint{1.525146in}{6.251128in}}%
\pgfpathlineto{\pgfqpoint{1.472102in}{6.251128in}}%
\pgfpathlineto{\pgfqpoint{1.419058in}{6.251128in}}%
\pgfpathlineto{\pgfqpoint{1.366014in}{6.251128in}}%
\pgfpathlineto{\pgfqpoint{1.312970in}{6.251128in}}%
\pgfpathlineto{\pgfqpoint{1.259927in}{6.251128in}}%
\pgfpathlineto{\pgfqpoint{1.206883in}{6.251128in}}%
\pgfpathlineto{\pgfqpoint{1.153839in}{6.251128in}}%
\pgfpathlineto{\pgfqpoint{1.100795in}{6.251128in}}%
\pgfpathlineto{\pgfqpoint{1.047751in}{6.251128in}}%
\pgfpathlineto{\pgfqpoint{0.994707in}{6.251128in}}%
\pgfpathlineto{\pgfqpoint{0.941663in}{6.251128in}}%
\pgfpathlineto{\pgfqpoint{0.941663in}{6.251128in}}%
\pgfpathclose%
\pgfusepath{stroke,fill}%
}%
\begin{pgfscope}%
\pgfsys@transformshift{0.000000in}{0.000000in}%
\pgfsys@useobject{currentmarker}{}%
\end{pgfscope}%
\end{pgfscope}%
\begin{pgfscope}%
\pgfpathrectangle{\pgfqpoint{0.941663in}{4.334375in}}{\pgfqpoint{8.858337in}{3.465625in}}%
\pgfusepath{clip}%
\pgfsetrectcap%
\pgfsetroundjoin%
\pgfsetlinewidth{1.505625pt}%
\definecolor{currentstroke}{rgb}{0.501961,0.000000,0.501961}%
\pgfsetstrokecolor{currentstroke}%
\pgfsetdash{}{0pt}%
\pgfpathmoveto{\pgfqpoint{0.941663in}{6.251128in}}%
\pgfpathlineto{\pgfqpoint{0.994707in}{6.261680in}}%
\pgfpathlineto{\pgfqpoint{1.047751in}{6.251128in}}%
\pgfpathlineto{\pgfqpoint{1.153839in}{6.251128in}}%
\pgfpathlineto{\pgfqpoint{1.206883in}{6.598964in}}%
\pgfpathlineto{\pgfqpoint{1.312970in}{6.598964in}}%
\pgfpathlineto{\pgfqpoint{1.366014in}{6.251128in}}%
\pgfpathlineto{\pgfqpoint{1.578190in}{6.251128in}}%
\pgfpathlineto{\pgfqpoint{1.631234in}{6.598964in}}%
\pgfpathlineto{\pgfqpoint{1.684278in}{6.598964in}}%
\pgfpathlineto{\pgfqpoint{1.737322in}{6.251128in}}%
\pgfpathlineto{\pgfqpoint{1.790366in}{6.251128in}}%
\pgfpathlineto{\pgfqpoint{1.843410in}{6.598964in}}%
\pgfpathlineto{\pgfqpoint{1.896454in}{6.598964in}}%
\pgfpathlineto{\pgfqpoint{1.949498in}{6.251128in}}%
\pgfpathlineto{\pgfqpoint{2.002542in}{6.505445in}}%
\pgfpathlineto{\pgfqpoint{2.055586in}{6.251128in}}%
\pgfpathlineto{\pgfqpoint{2.108629in}{6.348747in}}%
\pgfpathlineto{\pgfqpoint{2.161673in}{6.251128in}}%
\pgfpathlineto{\pgfqpoint{2.214717in}{6.459099in}}%
\pgfpathlineto{\pgfqpoint{2.267761in}{6.251128in}}%
\pgfpathlineto{\pgfqpoint{2.320805in}{6.251128in}}%
\pgfpathlineto{\pgfqpoint{2.373849in}{6.541500in}}%
\pgfpathlineto{\pgfqpoint{2.426893in}{6.251128in}}%
\pgfpathlineto{\pgfqpoint{2.798201in}{6.251128in}}%
\pgfpathlineto{\pgfqpoint{2.851245in}{6.598964in}}%
\pgfpathlineto{\pgfqpoint{2.904288in}{6.251128in}}%
\pgfpathlineto{\pgfqpoint{2.957332in}{6.598964in}}%
\pgfpathlineto{\pgfqpoint{3.010376in}{6.251128in}}%
\pgfpathlineto{\pgfqpoint{3.063420in}{6.598964in}}%
\pgfpathlineto{\pgfqpoint{3.116464in}{6.598964in}}%
\pgfpathlineto{\pgfqpoint{3.169508in}{6.251128in}}%
\pgfpathlineto{\pgfqpoint{3.328640in}{6.251128in}}%
\pgfpathlineto{\pgfqpoint{3.381684in}{6.598964in}}%
\pgfpathlineto{\pgfqpoint{3.434728in}{6.251128in}}%
\pgfpathlineto{\pgfqpoint{3.487772in}{6.598964in}}%
\pgfpathlineto{\pgfqpoint{3.540816in}{6.427614in}}%
\pgfpathlineto{\pgfqpoint{3.593860in}{6.251128in}}%
\pgfpathlineto{\pgfqpoint{3.699948in}{6.251128in}}%
\pgfpathlineto{\pgfqpoint{3.752991in}{6.510498in}}%
\pgfpathlineto{\pgfqpoint{3.806035in}{6.251128in}}%
\pgfpathlineto{\pgfqpoint{3.859079in}{6.572998in}}%
\pgfpathlineto{\pgfqpoint{3.912123in}{6.481634in}}%
\pgfpathlineto{\pgfqpoint{3.965167in}{6.340259in}}%
\pgfpathlineto{\pgfqpoint{4.018211in}{6.251128in}}%
\pgfpathlineto{\pgfqpoint{4.283431in}{6.251128in}}%
\pgfpathlineto{\pgfqpoint{4.336475in}{6.598964in}}%
\pgfpathlineto{\pgfqpoint{4.389519in}{6.299258in}}%
\pgfpathlineto{\pgfqpoint{4.442563in}{6.598964in}}%
\pgfpathlineto{\pgfqpoint{4.495607in}{6.437705in}}%
\pgfpathlineto{\pgfqpoint{4.548650in}{6.251128in}}%
\pgfpathlineto{\pgfqpoint{4.654738in}{6.251128in}}%
\pgfpathlineto{\pgfqpoint{4.707782in}{6.351811in}}%
\pgfpathlineto{\pgfqpoint{4.760826in}{6.251128in}}%
\pgfpathlineto{\pgfqpoint{4.813870in}{6.251128in}}%
\pgfpathlineto{\pgfqpoint{4.866914in}{6.471548in}}%
\pgfpathlineto{\pgfqpoint{4.919958in}{6.251128in}}%
\pgfpathlineto{\pgfqpoint{4.973002in}{6.251128in}}%
\pgfpathlineto{\pgfqpoint{5.026046in}{6.590550in}}%
\pgfpathlineto{\pgfqpoint{5.079090in}{6.251128in}}%
\pgfpathlineto{\pgfqpoint{5.132134in}{6.598964in}}%
\pgfpathlineto{\pgfqpoint{5.185178in}{6.251128in}}%
\pgfpathlineto{\pgfqpoint{5.291266in}{6.251128in}}%
\pgfpathlineto{\pgfqpoint{5.344309in}{6.471024in}}%
\pgfpathlineto{\pgfqpoint{5.397353in}{6.251128in}}%
\pgfpathlineto{\pgfqpoint{5.503441in}{6.251128in}}%
\pgfpathlineto{\pgfqpoint{5.556485in}{6.329843in}}%
\pgfpathlineto{\pgfqpoint{5.609529in}{6.251128in}}%
\pgfpathlineto{\pgfqpoint{5.715617in}{6.251128in}}%
\pgfpathlineto{\pgfqpoint{5.768661in}{6.324051in}}%
\pgfpathlineto{\pgfqpoint{5.821705in}{6.251128in}}%
\pgfpathlineto{\pgfqpoint{5.980837in}{6.251128in}}%
\pgfpathlineto{\pgfqpoint{6.033881in}{6.400407in}}%
\pgfpathlineto{\pgfqpoint{6.086925in}{6.251128in}}%
\pgfpathlineto{\pgfqpoint{6.139969in}{6.251128in}}%
\pgfpathlineto{\pgfqpoint{6.193012in}{6.598964in}}%
\pgfpathlineto{\pgfqpoint{6.246056in}{6.362599in}}%
\pgfpathlineto{\pgfqpoint{6.299100in}{6.251128in}}%
\pgfpathlineto{\pgfqpoint{6.352144in}{6.569516in}}%
\pgfpathlineto{\pgfqpoint{6.405188in}{6.251128in}}%
\pgfpathlineto{\pgfqpoint{6.723452in}{6.251128in}}%
\pgfpathlineto{\pgfqpoint{6.776496in}{6.385890in}}%
\pgfpathlineto{\pgfqpoint{6.829540in}{6.251128in}}%
\pgfpathlineto{\pgfqpoint{6.882584in}{6.598964in}}%
\pgfpathlineto{\pgfqpoint{6.935628in}{6.251128in}}%
\pgfpathlineto{\pgfqpoint{6.988671in}{6.598964in}}%
\pgfpathlineto{\pgfqpoint{7.041715in}{6.251128in}}%
\pgfpathlineto{\pgfqpoint{7.147803in}{6.251128in}}%
\pgfpathlineto{\pgfqpoint{7.200847in}{6.598964in}}%
\pgfpathlineto{\pgfqpoint{7.253891in}{6.451777in}}%
\pgfpathlineto{\pgfqpoint{7.306935in}{6.251128in}}%
\pgfpathlineto{\pgfqpoint{7.359979in}{6.350974in}}%
\pgfpathlineto{\pgfqpoint{7.413023in}{6.251128in}}%
\pgfpathlineto{\pgfqpoint{7.996506in}{6.251128in}}%
\pgfpathlineto{\pgfqpoint{8.049550in}{6.598964in}}%
\pgfpathlineto{\pgfqpoint{8.102594in}{6.598964in}}%
\pgfpathlineto{\pgfqpoint{8.155638in}{6.251128in}}%
\pgfpathlineto{\pgfqpoint{8.208682in}{6.251128in}}%
\pgfpathlineto{\pgfqpoint{8.261726in}{6.598964in}}%
\pgfpathlineto{\pgfqpoint{8.314770in}{6.598964in}}%
\pgfpathlineto{\pgfqpoint{8.367814in}{6.251128in}}%
\pgfpathlineto{\pgfqpoint{8.420858in}{6.371312in}}%
\pgfpathlineto{\pgfqpoint{8.473902in}{6.251128in}}%
\pgfpathlineto{\pgfqpoint{8.526946in}{6.251128in}}%
\pgfpathlineto{\pgfqpoint{8.579990in}{6.560479in}}%
\pgfpathlineto{\pgfqpoint{8.633033in}{6.251128in}}%
\pgfpathlineto{\pgfqpoint{8.951297in}{6.251128in}}%
\pgfpathlineto{\pgfqpoint{9.004341in}{6.568668in}}%
\pgfpathlineto{\pgfqpoint{9.057385in}{6.251128in}}%
\pgfpathlineto{\pgfqpoint{9.110429in}{6.251128in}}%
\pgfpathlineto{\pgfqpoint{9.163473in}{6.598964in}}%
\pgfpathlineto{\pgfqpoint{9.216517in}{6.334558in}}%
\pgfpathlineto{\pgfqpoint{9.269561in}{6.251128in}}%
\pgfpathlineto{\pgfqpoint{9.428692in}{6.251128in}}%
\pgfpathlineto{\pgfqpoint{9.481736in}{6.256205in}}%
\pgfpathlineto{\pgfqpoint{9.534780in}{6.287774in}}%
\pgfpathlineto{\pgfqpoint{9.587824in}{6.251128in}}%
\pgfpathlineto{\pgfqpoint{9.800000in}{6.251128in}}%
\pgfpathlineto{\pgfqpoint{9.800000in}{6.251128in}}%
\pgfusepath{stroke}%
\end{pgfscope}%
\begin{pgfscope}%
\pgfpathrectangle{\pgfqpoint{0.941663in}{4.334375in}}{\pgfqpoint{8.858337in}{3.465625in}}%
\pgfusepath{clip}%
\pgfsetbuttcap%
\pgfsetroundjoin%
\definecolor{currentfill}{rgb}{0.501961,0.000000,0.501961}%
\pgfsetfillcolor{currentfill}%
\pgfsetlinewidth{1.003750pt}%
\definecolor{currentstroke}{rgb}{0.501961,0.000000,0.501961}%
\pgfsetstrokecolor{currentstroke}%
\pgfsetdash{}{0pt}%
\pgfsys@defobject{currentmarker}{\pgfqpoint{0.941663in}{6.251128in}}{\pgfqpoint{9.800000in}{6.598964in}}{%
\pgfpathmoveto{\pgfqpoint{0.941663in}{6.251128in}}%
\pgfpathlineto{\pgfqpoint{0.941663in}{6.251128in}}%
\pgfpathlineto{\pgfqpoint{0.994707in}{6.251128in}}%
\pgfpathlineto{\pgfqpoint{1.047751in}{6.251128in}}%
\pgfpathlineto{\pgfqpoint{1.100795in}{6.251128in}}%
\pgfpathlineto{\pgfqpoint{1.153839in}{6.251128in}}%
\pgfpathlineto{\pgfqpoint{1.206883in}{6.251128in}}%
\pgfpathlineto{\pgfqpoint{1.259927in}{6.251128in}}%
\pgfpathlineto{\pgfqpoint{1.312970in}{6.251128in}}%
\pgfpathlineto{\pgfqpoint{1.366014in}{6.251128in}}%
\pgfpathlineto{\pgfqpoint{1.419058in}{6.251128in}}%
\pgfpathlineto{\pgfqpoint{1.472102in}{6.251128in}}%
\pgfpathlineto{\pgfqpoint{1.525146in}{6.251128in}}%
\pgfpathlineto{\pgfqpoint{1.578190in}{6.251128in}}%
\pgfpathlineto{\pgfqpoint{1.631234in}{6.251128in}}%
\pgfpathlineto{\pgfqpoint{1.684278in}{6.251128in}}%
\pgfpathlineto{\pgfqpoint{1.737322in}{6.251128in}}%
\pgfpathlineto{\pgfqpoint{1.790366in}{6.251128in}}%
\pgfpathlineto{\pgfqpoint{1.843410in}{6.251128in}}%
\pgfpathlineto{\pgfqpoint{1.896454in}{6.251128in}}%
\pgfpathlineto{\pgfqpoint{1.949498in}{6.251128in}}%
\pgfpathlineto{\pgfqpoint{2.002542in}{6.251128in}}%
\pgfpathlineto{\pgfqpoint{2.055586in}{6.251128in}}%
\pgfpathlineto{\pgfqpoint{2.108629in}{6.251128in}}%
\pgfpathlineto{\pgfqpoint{2.161673in}{6.251128in}}%
\pgfpathlineto{\pgfqpoint{2.214717in}{6.251128in}}%
\pgfpathlineto{\pgfqpoint{2.267761in}{6.251128in}}%
\pgfpathlineto{\pgfqpoint{2.320805in}{6.251128in}}%
\pgfpathlineto{\pgfqpoint{2.373849in}{6.251128in}}%
\pgfpathlineto{\pgfqpoint{2.426893in}{6.251128in}}%
\pgfpathlineto{\pgfqpoint{2.479937in}{6.251128in}}%
\pgfpathlineto{\pgfqpoint{2.532981in}{6.251128in}}%
\pgfpathlineto{\pgfqpoint{2.586025in}{6.251128in}}%
\pgfpathlineto{\pgfqpoint{2.639069in}{6.251128in}}%
\pgfpathlineto{\pgfqpoint{2.692113in}{6.251128in}}%
\pgfpathlineto{\pgfqpoint{2.745157in}{6.251128in}}%
\pgfpathlineto{\pgfqpoint{2.798201in}{6.251128in}}%
\pgfpathlineto{\pgfqpoint{2.851245in}{6.251128in}}%
\pgfpathlineto{\pgfqpoint{2.904288in}{6.251128in}}%
\pgfpathlineto{\pgfqpoint{2.957332in}{6.251128in}}%
\pgfpathlineto{\pgfqpoint{3.010376in}{6.251128in}}%
\pgfpathlineto{\pgfqpoint{3.063420in}{6.251128in}}%
\pgfpathlineto{\pgfqpoint{3.116464in}{6.251128in}}%
\pgfpathlineto{\pgfqpoint{3.169508in}{6.251128in}}%
\pgfpathlineto{\pgfqpoint{3.222552in}{6.251128in}}%
\pgfpathlineto{\pgfqpoint{3.275596in}{6.251128in}}%
\pgfpathlineto{\pgfqpoint{3.328640in}{6.251128in}}%
\pgfpathlineto{\pgfqpoint{3.381684in}{6.251128in}}%
\pgfpathlineto{\pgfqpoint{3.434728in}{6.251128in}}%
\pgfpathlineto{\pgfqpoint{3.487772in}{6.251128in}}%
\pgfpathlineto{\pgfqpoint{3.540816in}{6.251128in}}%
\pgfpathlineto{\pgfqpoint{3.593860in}{6.251128in}}%
\pgfpathlineto{\pgfqpoint{3.646904in}{6.251128in}}%
\pgfpathlineto{\pgfqpoint{3.699948in}{6.251128in}}%
\pgfpathlineto{\pgfqpoint{3.752991in}{6.251128in}}%
\pgfpathlineto{\pgfqpoint{3.806035in}{6.251128in}}%
\pgfpathlineto{\pgfqpoint{3.859079in}{6.251128in}}%
\pgfpathlineto{\pgfqpoint{3.912123in}{6.251128in}}%
\pgfpathlineto{\pgfqpoint{3.965167in}{6.251128in}}%
\pgfpathlineto{\pgfqpoint{4.018211in}{6.251128in}}%
\pgfpathlineto{\pgfqpoint{4.071255in}{6.251128in}}%
\pgfpathlineto{\pgfqpoint{4.124299in}{6.251128in}}%
\pgfpathlineto{\pgfqpoint{4.177343in}{6.251128in}}%
\pgfpathlineto{\pgfqpoint{4.230387in}{6.251128in}}%
\pgfpathlineto{\pgfqpoint{4.283431in}{6.251128in}}%
\pgfpathlineto{\pgfqpoint{4.336475in}{6.251128in}}%
\pgfpathlineto{\pgfqpoint{4.389519in}{6.251128in}}%
\pgfpathlineto{\pgfqpoint{4.442563in}{6.251128in}}%
\pgfpathlineto{\pgfqpoint{4.495607in}{6.251128in}}%
\pgfpathlineto{\pgfqpoint{4.548650in}{6.251128in}}%
\pgfpathlineto{\pgfqpoint{4.601694in}{6.251128in}}%
\pgfpathlineto{\pgfqpoint{4.654738in}{6.251128in}}%
\pgfpathlineto{\pgfqpoint{4.707782in}{6.251128in}}%
\pgfpathlineto{\pgfqpoint{4.760826in}{6.251128in}}%
\pgfpathlineto{\pgfqpoint{4.813870in}{6.251128in}}%
\pgfpathlineto{\pgfqpoint{4.866914in}{6.251128in}}%
\pgfpathlineto{\pgfqpoint{4.919958in}{6.251128in}}%
\pgfpathlineto{\pgfqpoint{4.973002in}{6.251128in}}%
\pgfpathlineto{\pgfqpoint{5.026046in}{6.251128in}}%
\pgfpathlineto{\pgfqpoint{5.079090in}{6.251128in}}%
\pgfpathlineto{\pgfqpoint{5.132134in}{6.251128in}}%
\pgfpathlineto{\pgfqpoint{5.185178in}{6.251128in}}%
\pgfpathlineto{\pgfqpoint{5.238222in}{6.251128in}}%
\pgfpathlineto{\pgfqpoint{5.291266in}{6.251128in}}%
\pgfpathlineto{\pgfqpoint{5.344309in}{6.251128in}}%
\pgfpathlineto{\pgfqpoint{5.397353in}{6.251128in}}%
\pgfpathlineto{\pgfqpoint{5.450397in}{6.251128in}}%
\pgfpathlineto{\pgfqpoint{5.503441in}{6.251128in}}%
\pgfpathlineto{\pgfqpoint{5.556485in}{6.251128in}}%
\pgfpathlineto{\pgfqpoint{5.609529in}{6.251128in}}%
\pgfpathlineto{\pgfqpoint{5.662573in}{6.251128in}}%
\pgfpathlineto{\pgfqpoint{5.715617in}{6.251128in}}%
\pgfpathlineto{\pgfqpoint{5.768661in}{6.251128in}}%
\pgfpathlineto{\pgfqpoint{5.821705in}{6.251128in}}%
\pgfpathlineto{\pgfqpoint{5.874749in}{6.251128in}}%
\pgfpathlineto{\pgfqpoint{5.927793in}{6.251128in}}%
\pgfpathlineto{\pgfqpoint{5.980837in}{6.251128in}}%
\pgfpathlineto{\pgfqpoint{6.033881in}{6.251128in}}%
\pgfpathlineto{\pgfqpoint{6.086925in}{6.251128in}}%
\pgfpathlineto{\pgfqpoint{6.139969in}{6.251128in}}%
\pgfpathlineto{\pgfqpoint{6.193012in}{6.251128in}}%
\pgfpathlineto{\pgfqpoint{6.246056in}{6.251128in}}%
\pgfpathlineto{\pgfqpoint{6.299100in}{6.251128in}}%
\pgfpathlineto{\pgfqpoint{6.352144in}{6.251128in}}%
\pgfpathlineto{\pgfqpoint{6.405188in}{6.251128in}}%
\pgfpathlineto{\pgfqpoint{6.458232in}{6.251128in}}%
\pgfpathlineto{\pgfqpoint{6.511276in}{6.251128in}}%
\pgfpathlineto{\pgfqpoint{6.564320in}{6.251128in}}%
\pgfpathlineto{\pgfqpoint{6.617364in}{6.251128in}}%
\pgfpathlineto{\pgfqpoint{6.670408in}{6.251128in}}%
\pgfpathlineto{\pgfqpoint{6.723452in}{6.251128in}}%
\pgfpathlineto{\pgfqpoint{6.776496in}{6.251128in}}%
\pgfpathlineto{\pgfqpoint{6.829540in}{6.251128in}}%
\pgfpathlineto{\pgfqpoint{6.882584in}{6.251128in}}%
\pgfpathlineto{\pgfqpoint{6.935628in}{6.251128in}}%
\pgfpathlineto{\pgfqpoint{6.988671in}{6.251128in}}%
\pgfpathlineto{\pgfqpoint{7.041715in}{6.251128in}}%
\pgfpathlineto{\pgfqpoint{7.094759in}{6.251128in}}%
\pgfpathlineto{\pgfqpoint{7.147803in}{6.251128in}}%
\pgfpathlineto{\pgfqpoint{7.200847in}{6.251128in}}%
\pgfpathlineto{\pgfqpoint{7.253891in}{6.251128in}}%
\pgfpathlineto{\pgfqpoint{7.306935in}{6.251128in}}%
\pgfpathlineto{\pgfqpoint{7.359979in}{6.251128in}}%
\pgfpathlineto{\pgfqpoint{7.413023in}{6.251128in}}%
\pgfpathlineto{\pgfqpoint{7.466067in}{6.251128in}}%
\pgfpathlineto{\pgfqpoint{7.519111in}{6.251128in}}%
\pgfpathlineto{\pgfqpoint{7.572155in}{6.251128in}}%
\pgfpathlineto{\pgfqpoint{7.625199in}{6.251128in}}%
\pgfpathlineto{\pgfqpoint{7.678243in}{6.251128in}}%
\pgfpathlineto{\pgfqpoint{7.731287in}{6.251128in}}%
\pgfpathlineto{\pgfqpoint{7.784330in}{6.251128in}}%
\pgfpathlineto{\pgfqpoint{7.837374in}{6.251128in}}%
\pgfpathlineto{\pgfqpoint{7.890418in}{6.251128in}}%
\pgfpathlineto{\pgfqpoint{7.943462in}{6.251128in}}%
\pgfpathlineto{\pgfqpoint{7.996506in}{6.251128in}}%
\pgfpathlineto{\pgfqpoint{8.049550in}{6.251128in}}%
\pgfpathlineto{\pgfqpoint{8.102594in}{6.251128in}}%
\pgfpathlineto{\pgfqpoint{8.155638in}{6.251128in}}%
\pgfpathlineto{\pgfqpoint{8.208682in}{6.251128in}}%
\pgfpathlineto{\pgfqpoint{8.261726in}{6.251128in}}%
\pgfpathlineto{\pgfqpoint{8.314770in}{6.251128in}}%
\pgfpathlineto{\pgfqpoint{8.367814in}{6.251128in}}%
\pgfpathlineto{\pgfqpoint{8.420858in}{6.251128in}}%
\pgfpathlineto{\pgfqpoint{8.473902in}{6.251128in}}%
\pgfpathlineto{\pgfqpoint{8.526946in}{6.251128in}}%
\pgfpathlineto{\pgfqpoint{8.579990in}{6.251128in}}%
\pgfpathlineto{\pgfqpoint{8.633033in}{6.251128in}}%
\pgfpathlineto{\pgfqpoint{8.686077in}{6.251128in}}%
\pgfpathlineto{\pgfqpoint{8.739121in}{6.251128in}}%
\pgfpathlineto{\pgfqpoint{8.792165in}{6.251128in}}%
\pgfpathlineto{\pgfqpoint{8.845209in}{6.251128in}}%
\pgfpathlineto{\pgfqpoint{8.898253in}{6.251128in}}%
\pgfpathlineto{\pgfqpoint{8.951297in}{6.251128in}}%
\pgfpathlineto{\pgfqpoint{9.004341in}{6.251128in}}%
\pgfpathlineto{\pgfqpoint{9.057385in}{6.251128in}}%
\pgfpathlineto{\pgfqpoint{9.110429in}{6.251128in}}%
\pgfpathlineto{\pgfqpoint{9.163473in}{6.251128in}}%
\pgfpathlineto{\pgfqpoint{9.216517in}{6.251128in}}%
\pgfpathlineto{\pgfqpoint{9.269561in}{6.251128in}}%
\pgfpathlineto{\pgfqpoint{9.322605in}{6.251128in}}%
\pgfpathlineto{\pgfqpoint{9.375649in}{6.251128in}}%
\pgfpathlineto{\pgfqpoint{9.428692in}{6.251128in}}%
\pgfpathlineto{\pgfqpoint{9.481736in}{6.251128in}}%
\pgfpathlineto{\pgfqpoint{9.534780in}{6.251128in}}%
\pgfpathlineto{\pgfqpoint{9.587824in}{6.251128in}}%
\pgfpathlineto{\pgfqpoint{9.640868in}{6.251128in}}%
\pgfpathlineto{\pgfqpoint{9.693912in}{6.251128in}}%
\pgfpathlineto{\pgfqpoint{9.746956in}{6.251128in}}%
\pgfpathlineto{\pgfqpoint{9.800000in}{6.251128in}}%
\pgfpathlineto{\pgfqpoint{9.800000in}{6.251128in}}%
\pgfpathlineto{\pgfqpoint{9.800000in}{6.251128in}}%
\pgfpathlineto{\pgfqpoint{9.746956in}{6.251128in}}%
\pgfpathlineto{\pgfqpoint{9.693912in}{6.251128in}}%
\pgfpathlineto{\pgfqpoint{9.640868in}{6.251128in}}%
\pgfpathlineto{\pgfqpoint{9.587824in}{6.251128in}}%
\pgfpathlineto{\pgfqpoint{9.534780in}{6.287774in}}%
\pgfpathlineto{\pgfqpoint{9.481736in}{6.256205in}}%
\pgfpathlineto{\pgfqpoint{9.428692in}{6.251128in}}%
\pgfpathlineto{\pgfqpoint{9.375649in}{6.251128in}}%
\pgfpathlineto{\pgfqpoint{9.322605in}{6.251128in}}%
\pgfpathlineto{\pgfqpoint{9.269561in}{6.251128in}}%
\pgfpathlineto{\pgfqpoint{9.216517in}{6.334558in}}%
\pgfpathlineto{\pgfqpoint{9.163473in}{6.598964in}}%
\pgfpathlineto{\pgfqpoint{9.110429in}{6.251128in}}%
\pgfpathlineto{\pgfqpoint{9.057385in}{6.251128in}}%
\pgfpathlineto{\pgfqpoint{9.004341in}{6.568668in}}%
\pgfpathlineto{\pgfqpoint{8.951297in}{6.251128in}}%
\pgfpathlineto{\pgfqpoint{8.898253in}{6.251128in}}%
\pgfpathlineto{\pgfqpoint{8.845209in}{6.251128in}}%
\pgfpathlineto{\pgfqpoint{8.792165in}{6.251128in}}%
\pgfpathlineto{\pgfqpoint{8.739121in}{6.251128in}}%
\pgfpathlineto{\pgfqpoint{8.686077in}{6.251128in}}%
\pgfpathlineto{\pgfqpoint{8.633033in}{6.251128in}}%
\pgfpathlineto{\pgfqpoint{8.579990in}{6.560479in}}%
\pgfpathlineto{\pgfqpoint{8.526946in}{6.251128in}}%
\pgfpathlineto{\pgfqpoint{8.473902in}{6.251128in}}%
\pgfpathlineto{\pgfqpoint{8.420858in}{6.371312in}}%
\pgfpathlineto{\pgfqpoint{8.367814in}{6.251128in}}%
\pgfpathlineto{\pgfqpoint{8.314770in}{6.598964in}}%
\pgfpathlineto{\pgfqpoint{8.261726in}{6.598964in}}%
\pgfpathlineto{\pgfqpoint{8.208682in}{6.251128in}}%
\pgfpathlineto{\pgfqpoint{8.155638in}{6.251128in}}%
\pgfpathlineto{\pgfqpoint{8.102594in}{6.598964in}}%
\pgfpathlineto{\pgfqpoint{8.049550in}{6.598964in}}%
\pgfpathlineto{\pgfqpoint{7.996506in}{6.251128in}}%
\pgfpathlineto{\pgfqpoint{7.943462in}{6.251128in}}%
\pgfpathlineto{\pgfqpoint{7.890418in}{6.251128in}}%
\pgfpathlineto{\pgfqpoint{7.837374in}{6.251128in}}%
\pgfpathlineto{\pgfqpoint{7.784330in}{6.251128in}}%
\pgfpathlineto{\pgfqpoint{7.731287in}{6.251128in}}%
\pgfpathlineto{\pgfqpoint{7.678243in}{6.251128in}}%
\pgfpathlineto{\pgfqpoint{7.625199in}{6.251128in}}%
\pgfpathlineto{\pgfqpoint{7.572155in}{6.251128in}}%
\pgfpathlineto{\pgfqpoint{7.519111in}{6.251128in}}%
\pgfpathlineto{\pgfqpoint{7.466067in}{6.251128in}}%
\pgfpathlineto{\pgfqpoint{7.413023in}{6.251128in}}%
\pgfpathlineto{\pgfqpoint{7.359979in}{6.350974in}}%
\pgfpathlineto{\pgfqpoint{7.306935in}{6.251128in}}%
\pgfpathlineto{\pgfqpoint{7.253891in}{6.451777in}}%
\pgfpathlineto{\pgfqpoint{7.200847in}{6.598964in}}%
\pgfpathlineto{\pgfqpoint{7.147803in}{6.251128in}}%
\pgfpathlineto{\pgfqpoint{7.094759in}{6.251128in}}%
\pgfpathlineto{\pgfqpoint{7.041715in}{6.251128in}}%
\pgfpathlineto{\pgfqpoint{6.988671in}{6.598964in}}%
\pgfpathlineto{\pgfqpoint{6.935628in}{6.251128in}}%
\pgfpathlineto{\pgfqpoint{6.882584in}{6.598964in}}%
\pgfpathlineto{\pgfqpoint{6.829540in}{6.251128in}}%
\pgfpathlineto{\pgfqpoint{6.776496in}{6.385890in}}%
\pgfpathlineto{\pgfqpoint{6.723452in}{6.251128in}}%
\pgfpathlineto{\pgfqpoint{6.670408in}{6.251128in}}%
\pgfpathlineto{\pgfqpoint{6.617364in}{6.251128in}}%
\pgfpathlineto{\pgfqpoint{6.564320in}{6.251128in}}%
\pgfpathlineto{\pgfqpoint{6.511276in}{6.251128in}}%
\pgfpathlineto{\pgfqpoint{6.458232in}{6.251128in}}%
\pgfpathlineto{\pgfqpoint{6.405188in}{6.251128in}}%
\pgfpathlineto{\pgfqpoint{6.352144in}{6.569516in}}%
\pgfpathlineto{\pgfqpoint{6.299100in}{6.251128in}}%
\pgfpathlineto{\pgfqpoint{6.246056in}{6.362599in}}%
\pgfpathlineto{\pgfqpoint{6.193012in}{6.598964in}}%
\pgfpathlineto{\pgfqpoint{6.139969in}{6.251128in}}%
\pgfpathlineto{\pgfqpoint{6.086925in}{6.251128in}}%
\pgfpathlineto{\pgfqpoint{6.033881in}{6.400407in}}%
\pgfpathlineto{\pgfqpoint{5.980837in}{6.251128in}}%
\pgfpathlineto{\pgfqpoint{5.927793in}{6.251128in}}%
\pgfpathlineto{\pgfqpoint{5.874749in}{6.251128in}}%
\pgfpathlineto{\pgfqpoint{5.821705in}{6.251128in}}%
\pgfpathlineto{\pgfqpoint{5.768661in}{6.324051in}}%
\pgfpathlineto{\pgfqpoint{5.715617in}{6.251128in}}%
\pgfpathlineto{\pgfqpoint{5.662573in}{6.251128in}}%
\pgfpathlineto{\pgfqpoint{5.609529in}{6.251128in}}%
\pgfpathlineto{\pgfqpoint{5.556485in}{6.329843in}}%
\pgfpathlineto{\pgfqpoint{5.503441in}{6.251128in}}%
\pgfpathlineto{\pgfqpoint{5.450397in}{6.251128in}}%
\pgfpathlineto{\pgfqpoint{5.397353in}{6.251128in}}%
\pgfpathlineto{\pgfqpoint{5.344309in}{6.471024in}}%
\pgfpathlineto{\pgfqpoint{5.291266in}{6.251128in}}%
\pgfpathlineto{\pgfqpoint{5.238222in}{6.251128in}}%
\pgfpathlineto{\pgfqpoint{5.185178in}{6.251128in}}%
\pgfpathlineto{\pgfqpoint{5.132134in}{6.598964in}}%
\pgfpathlineto{\pgfqpoint{5.079090in}{6.251128in}}%
\pgfpathlineto{\pgfqpoint{5.026046in}{6.590550in}}%
\pgfpathlineto{\pgfqpoint{4.973002in}{6.251128in}}%
\pgfpathlineto{\pgfqpoint{4.919958in}{6.251128in}}%
\pgfpathlineto{\pgfqpoint{4.866914in}{6.471548in}}%
\pgfpathlineto{\pgfqpoint{4.813870in}{6.251128in}}%
\pgfpathlineto{\pgfqpoint{4.760826in}{6.251128in}}%
\pgfpathlineto{\pgfqpoint{4.707782in}{6.351811in}}%
\pgfpathlineto{\pgfqpoint{4.654738in}{6.251128in}}%
\pgfpathlineto{\pgfqpoint{4.601694in}{6.251128in}}%
\pgfpathlineto{\pgfqpoint{4.548650in}{6.251128in}}%
\pgfpathlineto{\pgfqpoint{4.495607in}{6.437705in}}%
\pgfpathlineto{\pgfqpoint{4.442563in}{6.598964in}}%
\pgfpathlineto{\pgfqpoint{4.389519in}{6.299258in}}%
\pgfpathlineto{\pgfqpoint{4.336475in}{6.598964in}}%
\pgfpathlineto{\pgfqpoint{4.283431in}{6.251128in}}%
\pgfpathlineto{\pgfqpoint{4.230387in}{6.251128in}}%
\pgfpathlineto{\pgfqpoint{4.177343in}{6.251128in}}%
\pgfpathlineto{\pgfqpoint{4.124299in}{6.251128in}}%
\pgfpathlineto{\pgfqpoint{4.071255in}{6.251128in}}%
\pgfpathlineto{\pgfqpoint{4.018211in}{6.251128in}}%
\pgfpathlineto{\pgfqpoint{3.965167in}{6.340259in}}%
\pgfpathlineto{\pgfqpoint{3.912123in}{6.481634in}}%
\pgfpathlineto{\pgfqpoint{3.859079in}{6.572998in}}%
\pgfpathlineto{\pgfqpoint{3.806035in}{6.251128in}}%
\pgfpathlineto{\pgfqpoint{3.752991in}{6.510498in}}%
\pgfpathlineto{\pgfqpoint{3.699948in}{6.251128in}}%
\pgfpathlineto{\pgfqpoint{3.646904in}{6.251128in}}%
\pgfpathlineto{\pgfqpoint{3.593860in}{6.251128in}}%
\pgfpathlineto{\pgfqpoint{3.540816in}{6.427614in}}%
\pgfpathlineto{\pgfqpoint{3.487772in}{6.598964in}}%
\pgfpathlineto{\pgfqpoint{3.434728in}{6.251128in}}%
\pgfpathlineto{\pgfqpoint{3.381684in}{6.598964in}}%
\pgfpathlineto{\pgfqpoint{3.328640in}{6.251128in}}%
\pgfpathlineto{\pgfqpoint{3.275596in}{6.251128in}}%
\pgfpathlineto{\pgfqpoint{3.222552in}{6.251128in}}%
\pgfpathlineto{\pgfqpoint{3.169508in}{6.251128in}}%
\pgfpathlineto{\pgfqpoint{3.116464in}{6.598964in}}%
\pgfpathlineto{\pgfqpoint{3.063420in}{6.598964in}}%
\pgfpathlineto{\pgfqpoint{3.010376in}{6.251128in}}%
\pgfpathlineto{\pgfqpoint{2.957332in}{6.598964in}}%
\pgfpathlineto{\pgfqpoint{2.904288in}{6.251128in}}%
\pgfpathlineto{\pgfqpoint{2.851245in}{6.598964in}}%
\pgfpathlineto{\pgfqpoint{2.798201in}{6.251128in}}%
\pgfpathlineto{\pgfqpoint{2.745157in}{6.251128in}}%
\pgfpathlineto{\pgfqpoint{2.692113in}{6.251128in}}%
\pgfpathlineto{\pgfqpoint{2.639069in}{6.251128in}}%
\pgfpathlineto{\pgfqpoint{2.586025in}{6.251128in}}%
\pgfpathlineto{\pgfqpoint{2.532981in}{6.251128in}}%
\pgfpathlineto{\pgfqpoint{2.479937in}{6.251128in}}%
\pgfpathlineto{\pgfqpoint{2.426893in}{6.251128in}}%
\pgfpathlineto{\pgfqpoint{2.373849in}{6.541500in}}%
\pgfpathlineto{\pgfqpoint{2.320805in}{6.251128in}}%
\pgfpathlineto{\pgfqpoint{2.267761in}{6.251128in}}%
\pgfpathlineto{\pgfqpoint{2.214717in}{6.459099in}}%
\pgfpathlineto{\pgfqpoint{2.161673in}{6.251128in}}%
\pgfpathlineto{\pgfqpoint{2.108629in}{6.348747in}}%
\pgfpathlineto{\pgfqpoint{2.055586in}{6.251128in}}%
\pgfpathlineto{\pgfqpoint{2.002542in}{6.505445in}}%
\pgfpathlineto{\pgfqpoint{1.949498in}{6.251128in}}%
\pgfpathlineto{\pgfqpoint{1.896454in}{6.598964in}}%
\pgfpathlineto{\pgfqpoint{1.843410in}{6.598964in}}%
\pgfpathlineto{\pgfqpoint{1.790366in}{6.251128in}}%
\pgfpathlineto{\pgfqpoint{1.737322in}{6.251128in}}%
\pgfpathlineto{\pgfqpoint{1.684278in}{6.598964in}}%
\pgfpathlineto{\pgfqpoint{1.631234in}{6.598964in}}%
\pgfpathlineto{\pgfqpoint{1.578190in}{6.251128in}}%
\pgfpathlineto{\pgfqpoint{1.525146in}{6.251128in}}%
\pgfpathlineto{\pgfqpoint{1.472102in}{6.251128in}}%
\pgfpathlineto{\pgfqpoint{1.419058in}{6.251128in}}%
\pgfpathlineto{\pgfqpoint{1.366014in}{6.251128in}}%
\pgfpathlineto{\pgfqpoint{1.312970in}{6.598964in}}%
\pgfpathlineto{\pgfqpoint{1.259927in}{6.598964in}}%
\pgfpathlineto{\pgfqpoint{1.206883in}{6.598964in}}%
\pgfpathlineto{\pgfqpoint{1.153839in}{6.251128in}}%
\pgfpathlineto{\pgfqpoint{1.100795in}{6.251128in}}%
\pgfpathlineto{\pgfqpoint{1.047751in}{6.251128in}}%
\pgfpathlineto{\pgfqpoint{0.994707in}{6.261680in}}%
\pgfpathlineto{\pgfqpoint{0.941663in}{6.251128in}}%
\pgfpathlineto{\pgfqpoint{0.941663in}{6.251128in}}%
\pgfpathclose%
\pgfusepath{stroke,fill}%
}%
\begin{pgfscope}%
\pgfsys@transformshift{0.000000in}{0.000000in}%
\pgfsys@useobject{currentmarker}{}%
\end{pgfscope}%
\end{pgfscope}%
\begin{pgfscope}%
\pgfpathrectangle{\pgfqpoint{0.941663in}{4.334375in}}{\pgfqpoint{8.858337in}{3.465625in}}%
\pgfusepath{clip}%
\pgfsetrectcap%
\pgfsetroundjoin%
\pgfsetlinewidth{1.505625pt}%
\definecolor{currentstroke}{rgb}{0.549020,0.337255,0.294118}%
\pgfsetstrokecolor{currentstroke}%
\pgfsetdash{}{0pt}%
\pgfpathmoveto{\pgfqpoint{0.941663in}{6.251128in}}%
\pgfpathlineto{\pgfqpoint{0.994707in}{6.261680in}}%
\pgfpathlineto{\pgfqpoint{1.047751in}{6.251128in}}%
\pgfpathlineto{\pgfqpoint{1.153839in}{6.251128in}}%
\pgfpathlineto{\pgfqpoint{1.206883in}{6.598964in}}%
\pgfpathlineto{\pgfqpoint{1.312970in}{6.598964in}}%
\pgfpathlineto{\pgfqpoint{1.366014in}{6.251128in}}%
\pgfpathlineto{\pgfqpoint{1.578190in}{6.251128in}}%
\pgfpathlineto{\pgfqpoint{1.631234in}{6.598964in}}%
\pgfpathlineto{\pgfqpoint{1.684278in}{6.598964in}}%
\pgfpathlineto{\pgfqpoint{1.737322in}{6.251128in}}%
\pgfpathlineto{\pgfqpoint{1.790366in}{6.251128in}}%
\pgfpathlineto{\pgfqpoint{1.843410in}{6.734592in}}%
\pgfpathlineto{\pgfqpoint{1.896454in}{6.883587in}}%
\pgfpathlineto{\pgfqpoint{1.949498in}{6.251128in}}%
\pgfpathlineto{\pgfqpoint{2.002542in}{6.724358in}}%
\pgfpathlineto{\pgfqpoint{2.055586in}{6.251128in}}%
\pgfpathlineto{\pgfqpoint{2.108629in}{6.348747in}}%
\pgfpathlineto{\pgfqpoint{2.161673in}{6.251128in}}%
\pgfpathlineto{\pgfqpoint{2.214717in}{6.574971in}}%
\pgfpathlineto{\pgfqpoint{2.267761in}{6.251128in}}%
\pgfpathlineto{\pgfqpoint{2.320805in}{6.251128in}}%
\pgfpathlineto{\pgfqpoint{2.373849in}{6.541500in}}%
\pgfpathlineto{\pgfqpoint{2.426893in}{6.606651in}}%
\pgfpathlineto{\pgfqpoint{2.479937in}{6.251128in}}%
\pgfpathlineto{\pgfqpoint{2.798201in}{6.251128in}}%
\pgfpathlineto{\pgfqpoint{2.851245in}{6.598964in}}%
\pgfpathlineto{\pgfqpoint{2.904288in}{6.251128in}}%
\pgfpathlineto{\pgfqpoint{2.957332in}{6.744748in}}%
\pgfpathlineto{\pgfqpoint{3.010376in}{6.251128in}}%
\pgfpathlineto{\pgfqpoint{3.063420in}{6.598964in}}%
\pgfpathlineto{\pgfqpoint{3.116464in}{6.598964in}}%
\pgfpathlineto{\pgfqpoint{3.169508in}{6.251128in}}%
\pgfpathlineto{\pgfqpoint{3.328640in}{6.251128in}}%
\pgfpathlineto{\pgfqpoint{3.381684in}{6.598964in}}%
\pgfpathlineto{\pgfqpoint{3.434728in}{6.251128in}}%
\pgfpathlineto{\pgfqpoint{3.487772in}{6.598964in}}%
\pgfpathlineto{\pgfqpoint{3.540816in}{6.572638in}}%
\pgfpathlineto{\pgfqpoint{3.593860in}{6.251128in}}%
\pgfpathlineto{\pgfqpoint{3.699948in}{6.251128in}}%
\pgfpathlineto{\pgfqpoint{3.752991in}{6.510498in}}%
\pgfpathlineto{\pgfqpoint{3.806035in}{6.251128in}}%
\pgfpathlineto{\pgfqpoint{3.859079in}{6.572998in}}%
\pgfpathlineto{\pgfqpoint{3.912123in}{6.481634in}}%
\pgfpathlineto{\pgfqpoint{3.965167in}{6.340259in}}%
\pgfpathlineto{\pgfqpoint{4.018211in}{6.251128in}}%
\pgfpathlineto{\pgfqpoint{4.283431in}{6.251128in}}%
\pgfpathlineto{\pgfqpoint{4.336475in}{6.889082in}}%
\pgfpathlineto{\pgfqpoint{4.389519in}{6.299258in}}%
\pgfpathlineto{\pgfqpoint{4.442563in}{6.598964in}}%
\pgfpathlineto{\pgfqpoint{4.495607in}{6.633012in}}%
\pgfpathlineto{\pgfqpoint{4.548650in}{6.251128in}}%
\pgfpathlineto{\pgfqpoint{4.601694in}{6.592100in}}%
\pgfpathlineto{\pgfqpoint{4.654738in}{6.251128in}}%
\pgfpathlineto{\pgfqpoint{4.707782in}{6.766257in}}%
\pgfpathlineto{\pgfqpoint{4.760826in}{6.251128in}}%
\pgfpathlineto{\pgfqpoint{4.813870in}{6.251128in}}%
\pgfpathlineto{\pgfqpoint{4.866914in}{6.548881in}}%
\pgfpathlineto{\pgfqpoint{4.919958in}{6.528186in}}%
\pgfpathlineto{\pgfqpoint{4.973002in}{6.251128in}}%
\pgfpathlineto{\pgfqpoint{5.026046in}{6.590550in}}%
\pgfpathlineto{\pgfqpoint{5.079090in}{6.251128in}}%
\pgfpathlineto{\pgfqpoint{5.132134in}{6.598964in}}%
\pgfpathlineto{\pgfqpoint{5.185178in}{6.574350in}}%
\pgfpathlineto{\pgfqpoint{5.238222in}{6.501650in}}%
\pgfpathlineto{\pgfqpoint{5.291266in}{6.251128in}}%
\pgfpathlineto{\pgfqpoint{5.344309in}{6.471024in}}%
\pgfpathlineto{\pgfqpoint{5.397353in}{6.325005in}}%
\pgfpathlineto{\pgfqpoint{5.450397in}{6.770026in}}%
\pgfpathlineto{\pgfqpoint{5.503441in}{6.251128in}}%
\pgfpathlineto{\pgfqpoint{5.556485in}{6.329843in}}%
\pgfpathlineto{\pgfqpoint{5.609529in}{6.514446in}}%
\pgfpathlineto{\pgfqpoint{5.662573in}{6.375963in}}%
\pgfpathlineto{\pgfqpoint{5.715617in}{6.251128in}}%
\pgfpathlineto{\pgfqpoint{5.768661in}{6.913209in}}%
\pgfpathlineto{\pgfqpoint{5.821705in}{6.872997in}}%
\pgfpathlineto{\pgfqpoint{5.874749in}{6.702075in}}%
\pgfpathlineto{\pgfqpoint{5.927793in}{6.598351in}}%
\pgfpathlineto{\pgfqpoint{5.980837in}{6.251128in}}%
\pgfpathlineto{\pgfqpoint{6.033881in}{6.400407in}}%
\pgfpathlineto{\pgfqpoint{6.086925in}{6.251128in}}%
\pgfpathlineto{\pgfqpoint{6.139969in}{6.251128in}}%
\pgfpathlineto{\pgfqpoint{6.193012in}{6.635398in}}%
\pgfpathlineto{\pgfqpoint{6.246056in}{6.691111in}}%
\pgfpathlineto{\pgfqpoint{6.299100in}{6.251128in}}%
\pgfpathlineto{\pgfqpoint{6.352144in}{6.569516in}}%
\pgfpathlineto{\pgfqpoint{6.405188in}{6.364054in}}%
\pgfpathlineto{\pgfqpoint{6.458232in}{6.251128in}}%
\pgfpathlineto{\pgfqpoint{6.723452in}{6.251128in}}%
\pgfpathlineto{\pgfqpoint{6.776496in}{6.385890in}}%
\pgfpathlineto{\pgfqpoint{6.829540in}{6.251128in}}%
\pgfpathlineto{\pgfqpoint{6.882584in}{6.780238in}}%
\pgfpathlineto{\pgfqpoint{6.935628in}{6.251128in}}%
\pgfpathlineto{\pgfqpoint{6.988671in}{6.598964in}}%
\pgfpathlineto{\pgfqpoint{7.041715in}{6.251128in}}%
\pgfpathlineto{\pgfqpoint{7.147803in}{6.251128in}}%
\pgfpathlineto{\pgfqpoint{7.200847in}{6.729225in}}%
\pgfpathlineto{\pgfqpoint{7.253891in}{6.451777in}}%
\pgfpathlineto{\pgfqpoint{7.306935in}{6.251128in}}%
\pgfpathlineto{\pgfqpoint{7.359979in}{6.470643in}}%
\pgfpathlineto{\pgfqpoint{7.413023in}{6.529690in}}%
\pgfpathlineto{\pgfqpoint{7.466067in}{6.251128in}}%
\pgfpathlineto{\pgfqpoint{7.519111in}{6.251128in}}%
\pgfpathlineto{\pgfqpoint{7.572155in}{6.487962in}}%
\pgfpathlineto{\pgfqpoint{7.625199in}{6.652740in}}%
\pgfpathlineto{\pgfqpoint{7.678243in}{6.251128in}}%
\pgfpathlineto{\pgfqpoint{7.996506in}{6.251128in}}%
\pgfpathlineto{\pgfqpoint{8.049550in}{6.766484in}}%
\pgfpathlineto{\pgfqpoint{8.102594in}{6.782745in}}%
\pgfpathlineto{\pgfqpoint{8.155638in}{6.251128in}}%
\pgfpathlineto{\pgfqpoint{8.208682in}{6.251128in}}%
\pgfpathlineto{\pgfqpoint{8.261726in}{6.924712in}}%
\pgfpathlineto{\pgfqpoint{8.314770in}{6.876087in}}%
\pgfpathlineto{\pgfqpoint{8.367814in}{6.251128in}}%
\pgfpathlineto{\pgfqpoint{8.420858in}{6.653636in}}%
\pgfpathlineto{\pgfqpoint{8.473902in}{6.251128in}}%
\pgfpathlineto{\pgfqpoint{8.526946in}{6.251128in}}%
\pgfpathlineto{\pgfqpoint{8.579990in}{6.685501in}}%
\pgfpathlineto{\pgfqpoint{8.633033in}{6.520595in}}%
\pgfpathlineto{\pgfqpoint{8.686077in}{6.723687in}}%
\pgfpathlineto{\pgfqpoint{8.739121in}{6.489640in}}%
\pgfpathlineto{\pgfqpoint{8.792165in}{6.251128in}}%
\pgfpathlineto{\pgfqpoint{8.845209in}{6.299108in}}%
\pgfpathlineto{\pgfqpoint{8.898253in}{6.354195in}}%
\pgfpathlineto{\pgfqpoint{8.951297in}{6.251128in}}%
\pgfpathlineto{\pgfqpoint{9.004341in}{6.568668in}}%
\pgfpathlineto{\pgfqpoint{9.057385in}{6.251128in}}%
\pgfpathlineto{\pgfqpoint{9.110429in}{6.251128in}}%
\pgfpathlineto{\pgfqpoint{9.163473in}{6.668425in}}%
\pgfpathlineto{\pgfqpoint{9.216517in}{6.804492in}}%
\pgfpathlineto{\pgfqpoint{9.269561in}{6.628250in}}%
\pgfpathlineto{\pgfqpoint{9.322605in}{6.727450in}}%
\pgfpathlineto{\pgfqpoint{9.375649in}{6.754769in}}%
\pgfpathlineto{\pgfqpoint{9.428692in}{6.251128in}}%
\pgfpathlineto{\pgfqpoint{9.481736in}{6.256205in}}%
\pgfpathlineto{\pgfqpoint{9.534780in}{6.946800in}}%
\pgfpathlineto{\pgfqpoint{9.587824in}{6.621011in}}%
\pgfpathlineto{\pgfqpoint{9.640868in}{6.251128in}}%
\pgfpathlineto{\pgfqpoint{9.693912in}{6.632876in}}%
\pgfpathlineto{\pgfqpoint{9.746956in}{6.779712in}}%
\pgfpathlineto{\pgfqpoint{9.800000in}{6.573653in}}%
\pgfpathlineto{\pgfqpoint{9.800000in}{6.573653in}}%
\pgfusepath{stroke}%
\end{pgfscope}%
\begin{pgfscope}%
\pgfpathrectangle{\pgfqpoint{0.941663in}{4.334375in}}{\pgfqpoint{8.858337in}{3.465625in}}%
\pgfusepath{clip}%
\pgfsetbuttcap%
\pgfsetroundjoin%
\definecolor{currentfill}{rgb}{0.549020,0.337255,0.294118}%
\pgfsetfillcolor{currentfill}%
\pgfsetlinewidth{1.003750pt}%
\definecolor{currentstroke}{rgb}{0.549020,0.337255,0.294118}%
\pgfsetstrokecolor{currentstroke}%
\pgfsetdash{}{0pt}%
\pgfsys@defobject{currentmarker}{\pgfqpoint{0.941663in}{6.251128in}}{\pgfqpoint{9.800000in}{6.946800in}}{%
\pgfpathmoveto{\pgfqpoint{0.941663in}{6.251128in}}%
\pgfpathlineto{\pgfqpoint{0.941663in}{6.251128in}}%
\pgfpathlineto{\pgfqpoint{0.994707in}{6.261680in}}%
\pgfpathlineto{\pgfqpoint{1.047751in}{6.251128in}}%
\pgfpathlineto{\pgfqpoint{1.100795in}{6.251128in}}%
\pgfpathlineto{\pgfqpoint{1.153839in}{6.251128in}}%
\pgfpathlineto{\pgfqpoint{1.206883in}{6.598964in}}%
\pgfpathlineto{\pgfqpoint{1.259927in}{6.598964in}}%
\pgfpathlineto{\pgfqpoint{1.312970in}{6.598964in}}%
\pgfpathlineto{\pgfqpoint{1.366014in}{6.251128in}}%
\pgfpathlineto{\pgfqpoint{1.419058in}{6.251128in}}%
\pgfpathlineto{\pgfqpoint{1.472102in}{6.251128in}}%
\pgfpathlineto{\pgfqpoint{1.525146in}{6.251128in}}%
\pgfpathlineto{\pgfqpoint{1.578190in}{6.251128in}}%
\pgfpathlineto{\pgfqpoint{1.631234in}{6.598964in}}%
\pgfpathlineto{\pgfqpoint{1.684278in}{6.598964in}}%
\pgfpathlineto{\pgfqpoint{1.737322in}{6.251128in}}%
\pgfpathlineto{\pgfqpoint{1.790366in}{6.251128in}}%
\pgfpathlineto{\pgfqpoint{1.843410in}{6.598964in}}%
\pgfpathlineto{\pgfqpoint{1.896454in}{6.598964in}}%
\pgfpathlineto{\pgfqpoint{1.949498in}{6.251128in}}%
\pgfpathlineto{\pgfqpoint{2.002542in}{6.505445in}}%
\pgfpathlineto{\pgfqpoint{2.055586in}{6.251128in}}%
\pgfpathlineto{\pgfqpoint{2.108629in}{6.348747in}}%
\pgfpathlineto{\pgfqpoint{2.161673in}{6.251128in}}%
\pgfpathlineto{\pgfqpoint{2.214717in}{6.459099in}}%
\pgfpathlineto{\pgfqpoint{2.267761in}{6.251128in}}%
\pgfpathlineto{\pgfqpoint{2.320805in}{6.251128in}}%
\pgfpathlineto{\pgfqpoint{2.373849in}{6.541500in}}%
\pgfpathlineto{\pgfqpoint{2.426893in}{6.251128in}}%
\pgfpathlineto{\pgfqpoint{2.479937in}{6.251128in}}%
\pgfpathlineto{\pgfqpoint{2.532981in}{6.251128in}}%
\pgfpathlineto{\pgfqpoint{2.586025in}{6.251128in}}%
\pgfpathlineto{\pgfqpoint{2.639069in}{6.251128in}}%
\pgfpathlineto{\pgfqpoint{2.692113in}{6.251128in}}%
\pgfpathlineto{\pgfqpoint{2.745157in}{6.251128in}}%
\pgfpathlineto{\pgfqpoint{2.798201in}{6.251128in}}%
\pgfpathlineto{\pgfqpoint{2.851245in}{6.598964in}}%
\pgfpathlineto{\pgfqpoint{2.904288in}{6.251128in}}%
\pgfpathlineto{\pgfqpoint{2.957332in}{6.598964in}}%
\pgfpathlineto{\pgfqpoint{3.010376in}{6.251128in}}%
\pgfpathlineto{\pgfqpoint{3.063420in}{6.598964in}}%
\pgfpathlineto{\pgfqpoint{3.116464in}{6.598964in}}%
\pgfpathlineto{\pgfqpoint{3.169508in}{6.251128in}}%
\pgfpathlineto{\pgfqpoint{3.222552in}{6.251128in}}%
\pgfpathlineto{\pgfqpoint{3.275596in}{6.251128in}}%
\pgfpathlineto{\pgfqpoint{3.328640in}{6.251128in}}%
\pgfpathlineto{\pgfqpoint{3.381684in}{6.598964in}}%
\pgfpathlineto{\pgfqpoint{3.434728in}{6.251128in}}%
\pgfpathlineto{\pgfqpoint{3.487772in}{6.598964in}}%
\pgfpathlineto{\pgfqpoint{3.540816in}{6.427614in}}%
\pgfpathlineto{\pgfqpoint{3.593860in}{6.251128in}}%
\pgfpathlineto{\pgfqpoint{3.646904in}{6.251128in}}%
\pgfpathlineto{\pgfqpoint{3.699948in}{6.251128in}}%
\pgfpathlineto{\pgfqpoint{3.752991in}{6.510498in}}%
\pgfpathlineto{\pgfqpoint{3.806035in}{6.251128in}}%
\pgfpathlineto{\pgfqpoint{3.859079in}{6.572998in}}%
\pgfpathlineto{\pgfqpoint{3.912123in}{6.481634in}}%
\pgfpathlineto{\pgfqpoint{3.965167in}{6.340259in}}%
\pgfpathlineto{\pgfqpoint{4.018211in}{6.251128in}}%
\pgfpathlineto{\pgfqpoint{4.071255in}{6.251128in}}%
\pgfpathlineto{\pgfqpoint{4.124299in}{6.251128in}}%
\pgfpathlineto{\pgfqpoint{4.177343in}{6.251128in}}%
\pgfpathlineto{\pgfqpoint{4.230387in}{6.251128in}}%
\pgfpathlineto{\pgfqpoint{4.283431in}{6.251128in}}%
\pgfpathlineto{\pgfqpoint{4.336475in}{6.598964in}}%
\pgfpathlineto{\pgfqpoint{4.389519in}{6.299258in}}%
\pgfpathlineto{\pgfqpoint{4.442563in}{6.598964in}}%
\pgfpathlineto{\pgfqpoint{4.495607in}{6.437705in}}%
\pgfpathlineto{\pgfqpoint{4.548650in}{6.251128in}}%
\pgfpathlineto{\pgfqpoint{4.601694in}{6.251128in}}%
\pgfpathlineto{\pgfqpoint{4.654738in}{6.251128in}}%
\pgfpathlineto{\pgfqpoint{4.707782in}{6.351811in}}%
\pgfpathlineto{\pgfqpoint{4.760826in}{6.251128in}}%
\pgfpathlineto{\pgfqpoint{4.813870in}{6.251128in}}%
\pgfpathlineto{\pgfqpoint{4.866914in}{6.471548in}}%
\pgfpathlineto{\pgfqpoint{4.919958in}{6.251128in}}%
\pgfpathlineto{\pgfqpoint{4.973002in}{6.251128in}}%
\pgfpathlineto{\pgfqpoint{5.026046in}{6.590550in}}%
\pgfpathlineto{\pgfqpoint{5.079090in}{6.251128in}}%
\pgfpathlineto{\pgfqpoint{5.132134in}{6.598964in}}%
\pgfpathlineto{\pgfqpoint{5.185178in}{6.251128in}}%
\pgfpathlineto{\pgfqpoint{5.238222in}{6.251128in}}%
\pgfpathlineto{\pgfqpoint{5.291266in}{6.251128in}}%
\pgfpathlineto{\pgfqpoint{5.344309in}{6.471024in}}%
\pgfpathlineto{\pgfqpoint{5.397353in}{6.251128in}}%
\pgfpathlineto{\pgfqpoint{5.450397in}{6.251128in}}%
\pgfpathlineto{\pgfqpoint{5.503441in}{6.251128in}}%
\pgfpathlineto{\pgfqpoint{5.556485in}{6.329843in}}%
\pgfpathlineto{\pgfqpoint{5.609529in}{6.251128in}}%
\pgfpathlineto{\pgfqpoint{5.662573in}{6.251128in}}%
\pgfpathlineto{\pgfqpoint{5.715617in}{6.251128in}}%
\pgfpathlineto{\pgfqpoint{5.768661in}{6.324051in}}%
\pgfpathlineto{\pgfqpoint{5.821705in}{6.251128in}}%
\pgfpathlineto{\pgfqpoint{5.874749in}{6.251128in}}%
\pgfpathlineto{\pgfqpoint{5.927793in}{6.251128in}}%
\pgfpathlineto{\pgfqpoint{5.980837in}{6.251128in}}%
\pgfpathlineto{\pgfqpoint{6.033881in}{6.400407in}}%
\pgfpathlineto{\pgfqpoint{6.086925in}{6.251128in}}%
\pgfpathlineto{\pgfqpoint{6.139969in}{6.251128in}}%
\pgfpathlineto{\pgfqpoint{6.193012in}{6.598964in}}%
\pgfpathlineto{\pgfqpoint{6.246056in}{6.362599in}}%
\pgfpathlineto{\pgfqpoint{6.299100in}{6.251128in}}%
\pgfpathlineto{\pgfqpoint{6.352144in}{6.569516in}}%
\pgfpathlineto{\pgfqpoint{6.405188in}{6.251128in}}%
\pgfpathlineto{\pgfqpoint{6.458232in}{6.251128in}}%
\pgfpathlineto{\pgfqpoint{6.511276in}{6.251128in}}%
\pgfpathlineto{\pgfqpoint{6.564320in}{6.251128in}}%
\pgfpathlineto{\pgfqpoint{6.617364in}{6.251128in}}%
\pgfpathlineto{\pgfqpoint{6.670408in}{6.251128in}}%
\pgfpathlineto{\pgfqpoint{6.723452in}{6.251128in}}%
\pgfpathlineto{\pgfqpoint{6.776496in}{6.385890in}}%
\pgfpathlineto{\pgfqpoint{6.829540in}{6.251128in}}%
\pgfpathlineto{\pgfqpoint{6.882584in}{6.598964in}}%
\pgfpathlineto{\pgfqpoint{6.935628in}{6.251128in}}%
\pgfpathlineto{\pgfqpoint{6.988671in}{6.598964in}}%
\pgfpathlineto{\pgfqpoint{7.041715in}{6.251128in}}%
\pgfpathlineto{\pgfqpoint{7.094759in}{6.251128in}}%
\pgfpathlineto{\pgfqpoint{7.147803in}{6.251128in}}%
\pgfpathlineto{\pgfqpoint{7.200847in}{6.598964in}}%
\pgfpathlineto{\pgfqpoint{7.253891in}{6.451777in}}%
\pgfpathlineto{\pgfqpoint{7.306935in}{6.251128in}}%
\pgfpathlineto{\pgfqpoint{7.359979in}{6.350974in}}%
\pgfpathlineto{\pgfqpoint{7.413023in}{6.251128in}}%
\pgfpathlineto{\pgfqpoint{7.466067in}{6.251128in}}%
\pgfpathlineto{\pgfqpoint{7.519111in}{6.251128in}}%
\pgfpathlineto{\pgfqpoint{7.572155in}{6.251128in}}%
\pgfpathlineto{\pgfqpoint{7.625199in}{6.251128in}}%
\pgfpathlineto{\pgfqpoint{7.678243in}{6.251128in}}%
\pgfpathlineto{\pgfqpoint{7.731287in}{6.251128in}}%
\pgfpathlineto{\pgfqpoint{7.784330in}{6.251128in}}%
\pgfpathlineto{\pgfqpoint{7.837374in}{6.251128in}}%
\pgfpathlineto{\pgfqpoint{7.890418in}{6.251128in}}%
\pgfpathlineto{\pgfqpoint{7.943462in}{6.251128in}}%
\pgfpathlineto{\pgfqpoint{7.996506in}{6.251128in}}%
\pgfpathlineto{\pgfqpoint{8.049550in}{6.598964in}}%
\pgfpathlineto{\pgfqpoint{8.102594in}{6.598964in}}%
\pgfpathlineto{\pgfqpoint{8.155638in}{6.251128in}}%
\pgfpathlineto{\pgfqpoint{8.208682in}{6.251128in}}%
\pgfpathlineto{\pgfqpoint{8.261726in}{6.598964in}}%
\pgfpathlineto{\pgfqpoint{8.314770in}{6.598964in}}%
\pgfpathlineto{\pgfqpoint{8.367814in}{6.251128in}}%
\pgfpathlineto{\pgfqpoint{8.420858in}{6.371312in}}%
\pgfpathlineto{\pgfqpoint{8.473902in}{6.251128in}}%
\pgfpathlineto{\pgfqpoint{8.526946in}{6.251128in}}%
\pgfpathlineto{\pgfqpoint{8.579990in}{6.560479in}}%
\pgfpathlineto{\pgfqpoint{8.633033in}{6.251128in}}%
\pgfpathlineto{\pgfqpoint{8.686077in}{6.251128in}}%
\pgfpathlineto{\pgfqpoint{8.739121in}{6.251128in}}%
\pgfpathlineto{\pgfqpoint{8.792165in}{6.251128in}}%
\pgfpathlineto{\pgfqpoint{8.845209in}{6.251128in}}%
\pgfpathlineto{\pgfqpoint{8.898253in}{6.251128in}}%
\pgfpathlineto{\pgfqpoint{8.951297in}{6.251128in}}%
\pgfpathlineto{\pgfqpoint{9.004341in}{6.568668in}}%
\pgfpathlineto{\pgfqpoint{9.057385in}{6.251128in}}%
\pgfpathlineto{\pgfqpoint{9.110429in}{6.251128in}}%
\pgfpathlineto{\pgfqpoint{9.163473in}{6.598964in}}%
\pgfpathlineto{\pgfqpoint{9.216517in}{6.334558in}}%
\pgfpathlineto{\pgfqpoint{9.269561in}{6.251128in}}%
\pgfpathlineto{\pgfqpoint{9.322605in}{6.251128in}}%
\pgfpathlineto{\pgfqpoint{9.375649in}{6.251128in}}%
\pgfpathlineto{\pgfqpoint{9.428692in}{6.251128in}}%
\pgfpathlineto{\pgfqpoint{9.481736in}{6.256205in}}%
\pgfpathlineto{\pgfqpoint{9.534780in}{6.287774in}}%
\pgfpathlineto{\pgfqpoint{9.587824in}{6.251128in}}%
\pgfpathlineto{\pgfqpoint{9.640868in}{6.251128in}}%
\pgfpathlineto{\pgfqpoint{9.693912in}{6.251128in}}%
\pgfpathlineto{\pgfqpoint{9.746956in}{6.251128in}}%
\pgfpathlineto{\pgfqpoint{9.800000in}{6.251128in}}%
\pgfpathlineto{\pgfqpoint{9.800000in}{6.573653in}}%
\pgfpathlineto{\pgfqpoint{9.800000in}{6.573653in}}%
\pgfpathlineto{\pgfqpoint{9.746956in}{6.779712in}}%
\pgfpathlineto{\pgfqpoint{9.693912in}{6.632876in}}%
\pgfpathlineto{\pgfqpoint{9.640868in}{6.251128in}}%
\pgfpathlineto{\pgfqpoint{9.587824in}{6.621011in}}%
\pgfpathlineto{\pgfqpoint{9.534780in}{6.946800in}}%
\pgfpathlineto{\pgfqpoint{9.481736in}{6.256205in}}%
\pgfpathlineto{\pgfqpoint{9.428692in}{6.251128in}}%
\pgfpathlineto{\pgfqpoint{9.375649in}{6.754769in}}%
\pgfpathlineto{\pgfqpoint{9.322605in}{6.727450in}}%
\pgfpathlineto{\pgfqpoint{9.269561in}{6.628250in}}%
\pgfpathlineto{\pgfqpoint{9.216517in}{6.804492in}}%
\pgfpathlineto{\pgfqpoint{9.163473in}{6.668425in}}%
\pgfpathlineto{\pgfqpoint{9.110429in}{6.251128in}}%
\pgfpathlineto{\pgfqpoint{9.057385in}{6.251128in}}%
\pgfpathlineto{\pgfqpoint{9.004341in}{6.568668in}}%
\pgfpathlineto{\pgfqpoint{8.951297in}{6.251128in}}%
\pgfpathlineto{\pgfqpoint{8.898253in}{6.354195in}}%
\pgfpathlineto{\pgfqpoint{8.845209in}{6.299108in}}%
\pgfpathlineto{\pgfqpoint{8.792165in}{6.251128in}}%
\pgfpathlineto{\pgfqpoint{8.739121in}{6.489640in}}%
\pgfpathlineto{\pgfqpoint{8.686077in}{6.723687in}}%
\pgfpathlineto{\pgfqpoint{8.633033in}{6.520595in}}%
\pgfpathlineto{\pgfqpoint{8.579990in}{6.685501in}}%
\pgfpathlineto{\pgfqpoint{8.526946in}{6.251128in}}%
\pgfpathlineto{\pgfqpoint{8.473902in}{6.251128in}}%
\pgfpathlineto{\pgfqpoint{8.420858in}{6.653636in}}%
\pgfpathlineto{\pgfqpoint{8.367814in}{6.251128in}}%
\pgfpathlineto{\pgfqpoint{8.314770in}{6.876087in}}%
\pgfpathlineto{\pgfqpoint{8.261726in}{6.924712in}}%
\pgfpathlineto{\pgfqpoint{8.208682in}{6.251128in}}%
\pgfpathlineto{\pgfqpoint{8.155638in}{6.251128in}}%
\pgfpathlineto{\pgfqpoint{8.102594in}{6.782745in}}%
\pgfpathlineto{\pgfqpoint{8.049550in}{6.766484in}}%
\pgfpathlineto{\pgfqpoint{7.996506in}{6.251128in}}%
\pgfpathlineto{\pgfqpoint{7.943462in}{6.251128in}}%
\pgfpathlineto{\pgfqpoint{7.890418in}{6.251128in}}%
\pgfpathlineto{\pgfqpoint{7.837374in}{6.251128in}}%
\pgfpathlineto{\pgfqpoint{7.784330in}{6.251128in}}%
\pgfpathlineto{\pgfqpoint{7.731287in}{6.251128in}}%
\pgfpathlineto{\pgfqpoint{7.678243in}{6.251128in}}%
\pgfpathlineto{\pgfqpoint{7.625199in}{6.652740in}}%
\pgfpathlineto{\pgfqpoint{7.572155in}{6.487962in}}%
\pgfpathlineto{\pgfqpoint{7.519111in}{6.251128in}}%
\pgfpathlineto{\pgfqpoint{7.466067in}{6.251128in}}%
\pgfpathlineto{\pgfqpoint{7.413023in}{6.529690in}}%
\pgfpathlineto{\pgfqpoint{7.359979in}{6.470643in}}%
\pgfpathlineto{\pgfqpoint{7.306935in}{6.251128in}}%
\pgfpathlineto{\pgfqpoint{7.253891in}{6.451777in}}%
\pgfpathlineto{\pgfqpoint{7.200847in}{6.729225in}}%
\pgfpathlineto{\pgfqpoint{7.147803in}{6.251128in}}%
\pgfpathlineto{\pgfqpoint{7.094759in}{6.251128in}}%
\pgfpathlineto{\pgfqpoint{7.041715in}{6.251128in}}%
\pgfpathlineto{\pgfqpoint{6.988671in}{6.598964in}}%
\pgfpathlineto{\pgfqpoint{6.935628in}{6.251128in}}%
\pgfpathlineto{\pgfqpoint{6.882584in}{6.780238in}}%
\pgfpathlineto{\pgfqpoint{6.829540in}{6.251128in}}%
\pgfpathlineto{\pgfqpoint{6.776496in}{6.385890in}}%
\pgfpathlineto{\pgfqpoint{6.723452in}{6.251128in}}%
\pgfpathlineto{\pgfqpoint{6.670408in}{6.251128in}}%
\pgfpathlineto{\pgfqpoint{6.617364in}{6.251128in}}%
\pgfpathlineto{\pgfqpoint{6.564320in}{6.251128in}}%
\pgfpathlineto{\pgfqpoint{6.511276in}{6.251128in}}%
\pgfpathlineto{\pgfqpoint{6.458232in}{6.251128in}}%
\pgfpathlineto{\pgfqpoint{6.405188in}{6.364054in}}%
\pgfpathlineto{\pgfqpoint{6.352144in}{6.569516in}}%
\pgfpathlineto{\pgfqpoint{6.299100in}{6.251128in}}%
\pgfpathlineto{\pgfqpoint{6.246056in}{6.691111in}}%
\pgfpathlineto{\pgfqpoint{6.193012in}{6.635398in}}%
\pgfpathlineto{\pgfqpoint{6.139969in}{6.251128in}}%
\pgfpathlineto{\pgfqpoint{6.086925in}{6.251128in}}%
\pgfpathlineto{\pgfqpoint{6.033881in}{6.400407in}}%
\pgfpathlineto{\pgfqpoint{5.980837in}{6.251128in}}%
\pgfpathlineto{\pgfqpoint{5.927793in}{6.598351in}}%
\pgfpathlineto{\pgfqpoint{5.874749in}{6.702075in}}%
\pgfpathlineto{\pgfqpoint{5.821705in}{6.872997in}}%
\pgfpathlineto{\pgfqpoint{5.768661in}{6.913209in}}%
\pgfpathlineto{\pgfqpoint{5.715617in}{6.251128in}}%
\pgfpathlineto{\pgfqpoint{5.662573in}{6.375963in}}%
\pgfpathlineto{\pgfqpoint{5.609529in}{6.514446in}}%
\pgfpathlineto{\pgfqpoint{5.556485in}{6.329843in}}%
\pgfpathlineto{\pgfqpoint{5.503441in}{6.251128in}}%
\pgfpathlineto{\pgfqpoint{5.450397in}{6.770026in}}%
\pgfpathlineto{\pgfqpoint{5.397353in}{6.325005in}}%
\pgfpathlineto{\pgfqpoint{5.344309in}{6.471024in}}%
\pgfpathlineto{\pgfqpoint{5.291266in}{6.251128in}}%
\pgfpathlineto{\pgfqpoint{5.238222in}{6.501650in}}%
\pgfpathlineto{\pgfqpoint{5.185178in}{6.574350in}}%
\pgfpathlineto{\pgfqpoint{5.132134in}{6.598964in}}%
\pgfpathlineto{\pgfqpoint{5.079090in}{6.251128in}}%
\pgfpathlineto{\pgfqpoint{5.026046in}{6.590550in}}%
\pgfpathlineto{\pgfqpoint{4.973002in}{6.251128in}}%
\pgfpathlineto{\pgfqpoint{4.919958in}{6.528186in}}%
\pgfpathlineto{\pgfqpoint{4.866914in}{6.548881in}}%
\pgfpathlineto{\pgfqpoint{4.813870in}{6.251128in}}%
\pgfpathlineto{\pgfqpoint{4.760826in}{6.251128in}}%
\pgfpathlineto{\pgfqpoint{4.707782in}{6.766257in}}%
\pgfpathlineto{\pgfqpoint{4.654738in}{6.251128in}}%
\pgfpathlineto{\pgfqpoint{4.601694in}{6.592100in}}%
\pgfpathlineto{\pgfqpoint{4.548650in}{6.251128in}}%
\pgfpathlineto{\pgfqpoint{4.495607in}{6.633012in}}%
\pgfpathlineto{\pgfqpoint{4.442563in}{6.598964in}}%
\pgfpathlineto{\pgfqpoint{4.389519in}{6.299258in}}%
\pgfpathlineto{\pgfqpoint{4.336475in}{6.889082in}}%
\pgfpathlineto{\pgfqpoint{4.283431in}{6.251128in}}%
\pgfpathlineto{\pgfqpoint{4.230387in}{6.251128in}}%
\pgfpathlineto{\pgfqpoint{4.177343in}{6.251128in}}%
\pgfpathlineto{\pgfqpoint{4.124299in}{6.251128in}}%
\pgfpathlineto{\pgfqpoint{4.071255in}{6.251128in}}%
\pgfpathlineto{\pgfqpoint{4.018211in}{6.251128in}}%
\pgfpathlineto{\pgfqpoint{3.965167in}{6.340259in}}%
\pgfpathlineto{\pgfqpoint{3.912123in}{6.481634in}}%
\pgfpathlineto{\pgfqpoint{3.859079in}{6.572998in}}%
\pgfpathlineto{\pgfqpoint{3.806035in}{6.251128in}}%
\pgfpathlineto{\pgfqpoint{3.752991in}{6.510498in}}%
\pgfpathlineto{\pgfqpoint{3.699948in}{6.251128in}}%
\pgfpathlineto{\pgfqpoint{3.646904in}{6.251128in}}%
\pgfpathlineto{\pgfqpoint{3.593860in}{6.251128in}}%
\pgfpathlineto{\pgfqpoint{3.540816in}{6.572638in}}%
\pgfpathlineto{\pgfqpoint{3.487772in}{6.598964in}}%
\pgfpathlineto{\pgfqpoint{3.434728in}{6.251128in}}%
\pgfpathlineto{\pgfqpoint{3.381684in}{6.598964in}}%
\pgfpathlineto{\pgfqpoint{3.328640in}{6.251128in}}%
\pgfpathlineto{\pgfqpoint{3.275596in}{6.251128in}}%
\pgfpathlineto{\pgfqpoint{3.222552in}{6.251128in}}%
\pgfpathlineto{\pgfqpoint{3.169508in}{6.251128in}}%
\pgfpathlineto{\pgfqpoint{3.116464in}{6.598964in}}%
\pgfpathlineto{\pgfqpoint{3.063420in}{6.598964in}}%
\pgfpathlineto{\pgfqpoint{3.010376in}{6.251128in}}%
\pgfpathlineto{\pgfqpoint{2.957332in}{6.744748in}}%
\pgfpathlineto{\pgfqpoint{2.904288in}{6.251128in}}%
\pgfpathlineto{\pgfqpoint{2.851245in}{6.598964in}}%
\pgfpathlineto{\pgfqpoint{2.798201in}{6.251128in}}%
\pgfpathlineto{\pgfqpoint{2.745157in}{6.251128in}}%
\pgfpathlineto{\pgfqpoint{2.692113in}{6.251128in}}%
\pgfpathlineto{\pgfqpoint{2.639069in}{6.251128in}}%
\pgfpathlineto{\pgfqpoint{2.586025in}{6.251128in}}%
\pgfpathlineto{\pgfqpoint{2.532981in}{6.251128in}}%
\pgfpathlineto{\pgfqpoint{2.479937in}{6.251128in}}%
\pgfpathlineto{\pgfqpoint{2.426893in}{6.606651in}}%
\pgfpathlineto{\pgfqpoint{2.373849in}{6.541500in}}%
\pgfpathlineto{\pgfqpoint{2.320805in}{6.251128in}}%
\pgfpathlineto{\pgfqpoint{2.267761in}{6.251128in}}%
\pgfpathlineto{\pgfqpoint{2.214717in}{6.574971in}}%
\pgfpathlineto{\pgfqpoint{2.161673in}{6.251128in}}%
\pgfpathlineto{\pgfqpoint{2.108629in}{6.348747in}}%
\pgfpathlineto{\pgfqpoint{2.055586in}{6.251128in}}%
\pgfpathlineto{\pgfqpoint{2.002542in}{6.724358in}}%
\pgfpathlineto{\pgfqpoint{1.949498in}{6.251128in}}%
\pgfpathlineto{\pgfqpoint{1.896454in}{6.883587in}}%
\pgfpathlineto{\pgfqpoint{1.843410in}{6.734592in}}%
\pgfpathlineto{\pgfqpoint{1.790366in}{6.251128in}}%
\pgfpathlineto{\pgfqpoint{1.737322in}{6.251128in}}%
\pgfpathlineto{\pgfqpoint{1.684278in}{6.598964in}}%
\pgfpathlineto{\pgfqpoint{1.631234in}{6.598964in}}%
\pgfpathlineto{\pgfqpoint{1.578190in}{6.251128in}}%
\pgfpathlineto{\pgfqpoint{1.525146in}{6.251128in}}%
\pgfpathlineto{\pgfqpoint{1.472102in}{6.251128in}}%
\pgfpathlineto{\pgfqpoint{1.419058in}{6.251128in}}%
\pgfpathlineto{\pgfqpoint{1.366014in}{6.251128in}}%
\pgfpathlineto{\pgfqpoint{1.312970in}{6.598964in}}%
\pgfpathlineto{\pgfqpoint{1.259927in}{6.598964in}}%
\pgfpathlineto{\pgfqpoint{1.206883in}{6.598964in}}%
\pgfpathlineto{\pgfqpoint{1.153839in}{6.251128in}}%
\pgfpathlineto{\pgfqpoint{1.100795in}{6.251128in}}%
\pgfpathlineto{\pgfqpoint{1.047751in}{6.251128in}}%
\pgfpathlineto{\pgfqpoint{0.994707in}{6.261680in}}%
\pgfpathlineto{\pgfqpoint{0.941663in}{6.251128in}}%
\pgfpathlineto{\pgfqpoint{0.941663in}{6.251128in}}%
\pgfpathclose%
\pgfusepath{stroke,fill}%
}%
\begin{pgfscope}%
\pgfsys@transformshift{0.000000in}{0.000000in}%
\pgfsys@useobject{currentmarker}{}%
\end{pgfscope}%
\end{pgfscope}%
\begin{pgfscope}%
\pgfpathrectangle{\pgfqpoint{0.941663in}{4.334375in}}{\pgfqpoint{8.858337in}{3.465625in}}%
\pgfusepath{clip}%
\pgfsetrectcap%
\pgfsetroundjoin%
\pgfsetlinewidth{1.505625pt}%
\definecolor{currentstroke}{rgb}{1.000000,0.647059,0.000000}%
\pgfsetstrokecolor{currentstroke}%
\pgfsetdash{}{0pt}%
\pgfpathmoveto{\pgfqpoint{0.941663in}{5.302080in}}%
\pgfpathlineto{\pgfqpoint{0.994707in}{5.555456in}}%
\pgfpathlineto{\pgfqpoint{1.047751in}{5.298031in}}%
\pgfpathlineto{\pgfqpoint{1.100795in}{5.267250in}}%
\pgfpathlineto{\pgfqpoint{1.153839in}{5.253587in}}%
\pgfpathlineto{\pgfqpoint{1.206883in}{5.555456in}}%
\pgfpathlineto{\pgfqpoint{1.312970in}{5.555456in}}%
\pgfpathlineto{\pgfqpoint{1.366014in}{5.207621in}}%
\pgfpathlineto{\pgfqpoint{1.419058in}{5.263460in}}%
\pgfpathlineto{\pgfqpoint{1.472102in}{5.268823in}}%
\pgfpathlineto{\pgfqpoint{1.525146in}{5.349185in}}%
\pgfpathlineto{\pgfqpoint{1.578190in}{5.343668in}}%
\pgfpathlineto{\pgfqpoint{1.631234in}{5.555456in}}%
\pgfpathlineto{\pgfqpoint{1.684278in}{5.555456in}}%
\pgfpathlineto{\pgfqpoint{1.737322in}{5.491300in}}%
\pgfpathlineto{\pgfqpoint{1.790366in}{5.473721in}}%
\pgfpathlineto{\pgfqpoint{1.843410in}{5.555456in}}%
\pgfpathlineto{\pgfqpoint{1.896454in}{5.555456in}}%
\pgfpathlineto{\pgfqpoint{1.949498in}{5.447032in}}%
\pgfpathlineto{\pgfqpoint{2.002542in}{5.555456in}}%
\pgfpathlineto{\pgfqpoint{2.055586in}{5.457838in}}%
\pgfpathlineto{\pgfqpoint{2.108629in}{5.555456in}}%
\pgfpathlineto{\pgfqpoint{2.161673in}{5.347486in}}%
\pgfpathlineto{\pgfqpoint{2.214717in}{5.555456in}}%
\pgfpathlineto{\pgfqpoint{2.267761in}{5.542464in}}%
\pgfpathlineto{\pgfqpoint{2.320805in}{5.278077in}}%
\pgfpathlineto{\pgfqpoint{2.373849in}{5.555456in}}%
\pgfpathlineto{\pgfqpoint{2.426893in}{5.555456in}}%
\pgfpathlineto{\pgfqpoint{2.479937in}{5.397997in}}%
\pgfpathlineto{\pgfqpoint{2.532981in}{5.253753in}}%
\pgfpathlineto{\pgfqpoint{2.586025in}{5.231942in}}%
\pgfpathlineto{\pgfqpoint{2.639069in}{5.227448in}}%
\pgfpathlineto{\pgfqpoint{2.692113in}{5.274799in}}%
\pgfpathlineto{\pgfqpoint{2.745157in}{5.555456in}}%
\pgfpathlineto{\pgfqpoint{2.851245in}{5.555456in}}%
\pgfpathlineto{\pgfqpoint{2.904288in}{5.357472in}}%
\pgfpathlineto{\pgfqpoint{2.957332in}{5.555456in}}%
\pgfpathlineto{\pgfqpoint{3.010376in}{5.443237in}}%
\pgfpathlineto{\pgfqpoint{3.063420in}{5.555456in}}%
\pgfpathlineto{\pgfqpoint{3.116464in}{5.555456in}}%
\pgfpathlineto{\pgfqpoint{3.169508in}{5.475056in}}%
\pgfpathlineto{\pgfqpoint{3.222552in}{5.505579in}}%
\pgfpathlineto{\pgfqpoint{3.275596in}{5.421720in}}%
\pgfpathlineto{\pgfqpoint{3.328640in}{5.480294in}}%
\pgfpathlineto{\pgfqpoint{3.381684in}{5.555456in}}%
\pgfpathlineto{\pgfqpoint{3.434728in}{5.332680in}}%
\pgfpathlineto{\pgfqpoint{3.487772in}{5.555456in}}%
\pgfpathlineto{\pgfqpoint{3.593860in}{5.555456in}}%
\pgfpathlineto{\pgfqpoint{3.646904in}{5.270980in}}%
\pgfpathlineto{\pgfqpoint{3.699948in}{5.227131in}}%
\pgfpathlineto{\pgfqpoint{3.752991in}{5.555456in}}%
\pgfpathlineto{\pgfqpoint{3.806035in}{5.267382in}}%
\pgfpathlineto{\pgfqpoint{3.859079in}{5.555456in}}%
\pgfpathlineto{\pgfqpoint{3.965167in}{5.555456in}}%
\pgfpathlineto{\pgfqpoint{4.018211in}{5.273356in}}%
\pgfpathlineto{\pgfqpoint{4.071255in}{5.355447in}}%
\pgfpathlineto{\pgfqpoint{4.124299in}{5.483615in}}%
\pgfpathlineto{\pgfqpoint{4.177343in}{5.399336in}}%
\pgfpathlineto{\pgfqpoint{4.230387in}{5.437444in}}%
\pgfpathlineto{\pgfqpoint{4.283431in}{5.453162in}}%
\pgfpathlineto{\pgfqpoint{4.336475in}{5.555456in}}%
\pgfpathlineto{\pgfqpoint{4.601694in}{5.555456in}}%
\pgfpathlineto{\pgfqpoint{4.654738in}{5.454774in}}%
\pgfpathlineto{\pgfqpoint{4.707782in}{5.555456in}}%
\pgfpathlineto{\pgfqpoint{4.760826in}{5.555456in}}%
\pgfpathlineto{\pgfqpoint{4.813870in}{5.335037in}}%
\pgfpathlineto{\pgfqpoint{4.866914in}{5.555456in}}%
\pgfpathlineto{\pgfqpoint{4.919958in}{5.555456in}}%
\pgfpathlineto{\pgfqpoint{4.973002in}{5.216034in}}%
\pgfpathlineto{\pgfqpoint{5.026046in}{5.555456in}}%
\pgfpathlineto{\pgfqpoint{5.079090in}{5.207621in}}%
\pgfpathlineto{\pgfqpoint{5.132134in}{5.555456in}}%
\pgfpathlineto{\pgfqpoint{5.238222in}{5.555456in}}%
\pgfpathlineto{\pgfqpoint{5.291266in}{5.335561in}}%
\pgfpathlineto{\pgfqpoint{5.344309in}{5.555456in}}%
\pgfpathlineto{\pgfqpoint{5.450397in}{5.555456in}}%
\pgfpathlineto{\pgfqpoint{5.503441in}{5.476742in}}%
\pgfpathlineto{\pgfqpoint{5.556485in}{5.555456in}}%
\pgfpathlineto{\pgfqpoint{5.662573in}{5.555456in}}%
\pgfpathlineto{\pgfqpoint{5.715617in}{5.482534in}}%
\pgfpathlineto{\pgfqpoint{5.768661in}{5.555456in}}%
\pgfpathlineto{\pgfqpoint{5.927793in}{5.555456in}}%
\pgfpathlineto{\pgfqpoint{5.980837in}{5.406177in}}%
\pgfpathlineto{\pgfqpoint{6.033881in}{5.555456in}}%
\pgfpathlineto{\pgfqpoint{6.086925in}{5.335040in}}%
\pgfpathlineto{\pgfqpoint{6.139969in}{5.316566in}}%
\pgfpathlineto{\pgfqpoint{6.193012in}{5.555456in}}%
\pgfpathlineto{\pgfqpoint{6.246056in}{5.555456in}}%
\pgfpathlineto{\pgfqpoint{6.299100in}{5.237069in}}%
\pgfpathlineto{\pgfqpoint{6.352144in}{5.555456in}}%
\pgfpathlineto{\pgfqpoint{6.458232in}{5.555456in}}%
\pgfpathlineto{\pgfqpoint{6.511276in}{5.306663in}}%
\pgfpathlineto{\pgfqpoint{6.564320in}{5.303012in}}%
\pgfpathlineto{\pgfqpoint{6.617364in}{5.369970in}}%
\pgfpathlineto{\pgfqpoint{6.670408in}{5.399912in}}%
\pgfpathlineto{\pgfqpoint{6.723452in}{5.445993in}}%
\pgfpathlineto{\pgfqpoint{6.776496in}{5.555456in}}%
\pgfpathlineto{\pgfqpoint{6.829540in}{5.457960in}}%
\pgfpathlineto{\pgfqpoint{6.882584in}{5.555456in}}%
\pgfpathlineto{\pgfqpoint{6.935628in}{5.451613in}}%
\pgfpathlineto{\pgfqpoint{6.988671in}{5.555456in}}%
\pgfpathlineto{\pgfqpoint{7.041715in}{5.470876in}}%
\pgfpathlineto{\pgfqpoint{7.094759in}{5.493927in}}%
\pgfpathlineto{\pgfqpoint{7.147803in}{5.475721in}}%
\pgfpathlineto{\pgfqpoint{7.200847in}{5.555456in}}%
\pgfpathlineto{\pgfqpoint{7.253891in}{5.555456in}}%
\pgfpathlineto{\pgfqpoint{7.306935in}{5.455611in}}%
\pgfpathlineto{\pgfqpoint{7.359979in}{5.555456in}}%
\pgfpathlineto{\pgfqpoint{7.678243in}{5.555456in}}%
\pgfpathlineto{\pgfqpoint{7.731287in}{5.268845in}}%
\pgfpathlineto{\pgfqpoint{7.784330in}{5.304877in}}%
\pgfpathlineto{\pgfqpoint{7.837374in}{5.313128in}}%
\pgfpathlineto{\pgfqpoint{7.890418in}{5.354431in}}%
\pgfpathlineto{\pgfqpoint{7.943462in}{5.352176in}}%
\pgfpathlineto{\pgfqpoint{7.996506in}{5.430879in}}%
\pgfpathlineto{\pgfqpoint{8.049550in}{5.555456in}}%
\pgfpathlineto{\pgfqpoint{8.102594in}{5.555456in}}%
\pgfpathlineto{\pgfqpoint{8.155638in}{5.497516in}}%
\pgfpathlineto{\pgfqpoint{8.208682in}{5.491656in}}%
\pgfpathlineto{\pgfqpoint{8.261726in}{5.555456in}}%
\pgfpathlineto{\pgfqpoint{8.314770in}{5.555456in}}%
\pgfpathlineto{\pgfqpoint{8.367814in}{5.474073in}}%
\pgfpathlineto{\pgfqpoint{8.420858in}{5.555456in}}%
\pgfpathlineto{\pgfqpoint{8.473902in}{5.414851in}}%
\pgfpathlineto{\pgfqpoint{8.526946in}{5.386710in}}%
\pgfpathlineto{\pgfqpoint{8.579990in}{5.555456in}}%
\pgfpathlineto{\pgfqpoint{8.898253in}{5.555456in}}%
\pgfpathlineto{\pgfqpoint{8.951297in}{5.237916in}}%
\pgfpathlineto{\pgfqpoint{9.004341in}{5.555456in}}%
\pgfpathlineto{\pgfqpoint{9.057385in}{5.322277in}}%
\pgfpathlineto{\pgfqpoint{9.110429in}{5.357370in}}%
\pgfpathlineto{\pgfqpoint{9.163473in}{5.555456in}}%
\pgfpathlineto{\pgfqpoint{9.375649in}{5.555456in}}%
\pgfpathlineto{\pgfqpoint{9.428692in}{5.513734in}}%
\pgfpathlineto{\pgfqpoint{9.481736in}{5.555456in}}%
\pgfpathlineto{\pgfqpoint{9.800000in}{5.555456in}}%
\pgfpathlineto{\pgfqpoint{9.800000in}{5.555456in}}%
\pgfusepath{stroke}%
\end{pgfscope}%
\begin{pgfscope}%
\pgfpathrectangle{\pgfqpoint{0.941663in}{4.334375in}}{\pgfqpoint{8.858337in}{3.465625in}}%
\pgfusepath{clip}%
\pgfsetbuttcap%
\pgfsetroundjoin%
\definecolor{currentfill}{rgb}{1.000000,0.647059,0.000000}%
\pgfsetfillcolor{currentfill}%
\pgfsetlinewidth{1.003750pt}%
\definecolor{currentstroke}{rgb}{1.000000,0.647059,0.000000}%
\pgfsetstrokecolor{currentstroke}%
\pgfsetdash{}{0pt}%
\pgfsys@defobject{currentmarker}{\pgfqpoint{0.941663in}{5.207621in}}{\pgfqpoint{9.800000in}{5.555456in}}{%
\pgfpathmoveto{\pgfqpoint{0.941663in}{5.302080in}}%
\pgfpathlineto{\pgfqpoint{0.941663in}{5.555456in}}%
\pgfpathlineto{\pgfqpoint{0.994707in}{5.555456in}}%
\pgfpathlineto{\pgfqpoint{1.047751in}{5.555456in}}%
\pgfpathlineto{\pgfqpoint{1.100795in}{5.555456in}}%
\pgfpathlineto{\pgfqpoint{1.153839in}{5.555456in}}%
\pgfpathlineto{\pgfqpoint{1.206883in}{5.555456in}}%
\pgfpathlineto{\pgfqpoint{1.259927in}{5.555456in}}%
\pgfpathlineto{\pgfqpoint{1.312970in}{5.555456in}}%
\pgfpathlineto{\pgfqpoint{1.366014in}{5.555456in}}%
\pgfpathlineto{\pgfqpoint{1.419058in}{5.555456in}}%
\pgfpathlineto{\pgfqpoint{1.472102in}{5.555456in}}%
\pgfpathlineto{\pgfqpoint{1.525146in}{5.555456in}}%
\pgfpathlineto{\pgfqpoint{1.578190in}{5.555456in}}%
\pgfpathlineto{\pgfqpoint{1.631234in}{5.555456in}}%
\pgfpathlineto{\pgfqpoint{1.684278in}{5.555456in}}%
\pgfpathlineto{\pgfqpoint{1.737322in}{5.555456in}}%
\pgfpathlineto{\pgfqpoint{1.790366in}{5.555456in}}%
\pgfpathlineto{\pgfqpoint{1.843410in}{5.555456in}}%
\pgfpathlineto{\pgfqpoint{1.896454in}{5.555456in}}%
\pgfpathlineto{\pgfqpoint{1.949498in}{5.555456in}}%
\pgfpathlineto{\pgfqpoint{2.002542in}{5.555456in}}%
\pgfpathlineto{\pgfqpoint{2.055586in}{5.555456in}}%
\pgfpathlineto{\pgfqpoint{2.108629in}{5.555456in}}%
\pgfpathlineto{\pgfqpoint{2.161673in}{5.555456in}}%
\pgfpathlineto{\pgfqpoint{2.214717in}{5.555456in}}%
\pgfpathlineto{\pgfqpoint{2.267761in}{5.555456in}}%
\pgfpathlineto{\pgfqpoint{2.320805in}{5.555456in}}%
\pgfpathlineto{\pgfqpoint{2.373849in}{5.555456in}}%
\pgfpathlineto{\pgfqpoint{2.426893in}{5.555456in}}%
\pgfpathlineto{\pgfqpoint{2.479937in}{5.555456in}}%
\pgfpathlineto{\pgfqpoint{2.532981in}{5.555456in}}%
\pgfpathlineto{\pgfqpoint{2.586025in}{5.555456in}}%
\pgfpathlineto{\pgfqpoint{2.639069in}{5.555456in}}%
\pgfpathlineto{\pgfqpoint{2.692113in}{5.555456in}}%
\pgfpathlineto{\pgfqpoint{2.745157in}{5.555456in}}%
\pgfpathlineto{\pgfqpoint{2.798201in}{5.555456in}}%
\pgfpathlineto{\pgfqpoint{2.851245in}{5.555456in}}%
\pgfpathlineto{\pgfqpoint{2.904288in}{5.555456in}}%
\pgfpathlineto{\pgfqpoint{2.957332in}{5.555456in}}%
\pgfpathlineto{\pgfqpoint{3.010376in}{5.555456in}}%
\pgfpathlineto{\pgfqpoint{3.063420in}{5.555456in}}%
\pgfpathlineto{\pgfqpoint{3.116464in}{5.555456in}}%
\pgfpathlineto{\pgfqpoint{3.169508in}{5.555456in}}%
\pgfpathlineto{\pgfqpoint{3.222552in}{5.555456in}}%
\pgfpathlineto{\pgfqpoint{3.275596in}{5.555456in}}%
\pgfpathlineto{\pgfqpoint{3.328640in}{5.555456in}}%
\pgfpathlineto{\pgfqpoint{3.381684in}{5.555456in}}%
\pgfpathlineto{\pgfqpoint{3.434728in}{5.555456in}}%
\pgfpathlineto{\pgfqpoint{3.487772in}{5.555456in}}%
\pgfpathlineto{\pgfqpoint{3.540816in}{5.555456in}}%
\pgfpathlineto{\pgfqpoint{3.593860in}{5.555456in}}%
\pgfpathlineto{\pgfqpoint{3.646904in}{5.555456in}}%
\pgfpathlineto{\pgfqpoint{3.699948in}{5.555456in}}%
\pgfpathlineto{\pgfqpoint{3.752991in}{5.555456in}}%
\pgfpathlineto{\pgfqpoint{3.806035in}{5.555456in}}%
\pgfpathlineto{\pgfqpoint{3.859079in}{5.555456in}}%
\pgfpathlineto{\pgfqpoint{3.912123in}{5.555456in}}%
\pgfpathlineto{\pgfqpoint{3.965167in}{5.555456in}}%
\pgfpathlineto{\pgfqpoint{4.018211in}{5.555456in}}%
\pgfpathlineto{\pgfqpoint{4.071255in}{5.555456in}}%
\pgfpathlineto{\pgfqpoint{4.124299in}{5.555456in}}%
\pgfpathlineto{\pgfqpoint{4.177343in}{5.555456in}}%
\pgfpathlineto{\pgfqpoint{4.230387in}{5.555456in}}%
\pgfpathlineto{\pgfqpoint{4.283431in}{5.555456in}}%
\pgfpathlineto{\pgfqpoint{4.336475in}{5.555456in}}%
\pgfpathlineto{\pgfqpoint{4.389519in}{5.555456in}}%
\pgfpathlineto{\pgfqpoint{4.442563in}{5.555456in}}%
\pgfpathlineto{\pgfqpoint{4.495607in}{5.555456in}}%
\pgfpathlineto{\pgfqpoint{4.548650in}{5.555456in}}%
\pgfpathlineto{\pgfqpoint{4.601694in}{5.555456in}}%
\pgfpathlineto{\pgfqpoint{4.654738in}{5.555456in}}%
\pgfpathlineto{\pgfqpoint{4.707782in}{5.555456in}}%
\pgfpathlineto{\pgfqpoint{4.760826in}{5.555456in}}%
\pgfpathlineto{\pgfqpoint{4.813870in}{5.555456in}}%
\pgfpathlineto{\pgfqpoint{4.866914in}{5.555456in}}%
\pgfpathlineto{\pgfqpoint{4.919958in}{5.555456in}}%
\pgfpathlineto{\pgfqpoint{4.973002in}{5.555456in}}%
\pgfpathlineto{\pgfqpoint{5.026046in}{5.555456in}}%
\pgfpathlineto{\pgfqpoint{5.079090in}{5.555456in}}%
\pgfpathlineto{\pgfqpoint{5.132134in}{5.555456in}}%
\pgfpathlineto{\pgfqpoint{5.185178in}{5.555456in}}%
\pgfpathlineto{\pgfqpoint{5.238222in}{5.555456in}}%
\pgfpathlineto{\pgfqpoint{5.291266in}{5.555456in}}%
\pgfpathlineto{\pgfqpoint{5.344309in}{5.555456in}}%
\pgfpathlineto{\pgfqpoint{5.397353in}{5.555456in}}%
\pgfpathlineto{\pgfqpoint{5.450397in}{5.555456in}}%
\pgfpathlineto{\pgfqpoint{5.503441in}{5.555456in}}%
\pgfpathlineto{\pgfqpoint{5.556485in}{5.555456in}}%
\pgfpathlineto{\pgfqpoint{5.609529in}{5.555456in}}%
\pgfpathlineto{\pgfqpoint{5.662573in}{5.555456in}}%
\pgfpathlineto{\pgfqpoint{5.715617in}{5.555456in}}%
\pgfpathlineto{\pgfqpoint{5.768661in}{5.555456in}}%
\pgfpathlineto{\pgfqpoint{5.821705in}{5.555456in}}%
\pgfpathlineto{\pgfqpoint{5.874749in}{5.555456in}}%
\pgfpathlineto{\pgfqpoint{5.927793in}{5.555456in}}%
\pgfpathlineto{\pgfqpoint{5.980837in}{5.555456in}}%
\pgfpathlineto{\pgfqpoint{6.033881in}{5.555456in}}%
\pgfpathlineto{\pgfqpoint{6.086925in}{5.555456in}}%
\pgfpathlineto{\pgfqpoint{6.139969in}{5.555456in}}%
\pgfpathlineto{\pgfqpoint{6.193012in}{5.555456in}}%
\pgfpathlineto{\pgfqpoint{6.246056in}{5.555456in}}%
\pgfpathlineto{\pgfqpoint{6.299100in}{5.555456in}}%
\pgfpathlineto{\pgfqpoint{6.352144in}{5.555456in}}%
\pgfpathlineto{\pgfqpoint{6.405188in}{5.555456in}}%
\pgfpathlineto{\pgfqpoint{6.458232in}{5.555456in}}%
\pgfpathlineto{\pgfqpoint{6.511276in}{5.555456in}}%
\pgfpathlineto{\pgfqpoint{6.564320in}{5.555456in}}%
\pgfpathlineto{\pgfqpoint{6.617364in}{5.555456in}}%
\pgfpathlineto{\pgfqpoint{6.670408in}{5.555456in}}%
\pgfpathlineto{\pgfqpoint{6.723452in}{5.555456in}}%
\pgfpathlineto{\pgfqpoint{6.776496in}{5.555456in}}%
\pgfpathlineto{\pgfqpoint{6.829540in}{5.555456in}}%
\pgfpathlineto{\pgfqpoint{6.882584in}{5.555456in}}%
\pgfpathlineto{\pgfqpoint{6.935628in}{5.555456in}}%
\pgfpathlineto{\pgfqpoint{6.988671in}{5.555456in}}%
\pgfpathlineto{\pgfqpoint{7.041715in}{5.555456in}}%
\pgfpathlineto{\pgfqpoint{7.094759in}{5.555456in}}%
\pgfpathlineto{\pgfqpoint{7.147803in}{5.555456in}}%
\pgfpathlineto{\pgfqpoint{7.200847in}{5.555456in}}%
\pgfpathlineto{\pgfqpoint{7.253891in}{5.555456in}}%
\pgfpathlineto{\pgfqpoint{7.306935in}{5.555456in}}%
\pgfpathlineto{\pgfqpoint{7.359979in}{5.555456in}}%
\pgfpathlineto{\pgfqpoint{7.413023in}{5.555456in}}%
\pgfpathlineto{\pgfqpoint{7.466067in}{5.555456in}}%
\pgfpathlineto{\pgfqpoint{7.519111in}{5.555456in}}%
\pgfpathlineto{\pgfqpoint{7.572155in}{5.555456in}}%
\pgfpathlineto{\pgfqpoint{7.625199in}{5.555456in}}%
\pgfpathlineto{\pgfqpoint{7.678243in}{5.555456in}}%
\pgfpathlineto{\pgfqpoint{7.731287in}{5.555456in}}%
\pgfpathlineto{\pgfqpoint{7.784330in}{5.555456in}}%
\pgfpathlineto{\pgfqpoint{7.837374in}{5.555456in}}%
\pgfpathlineto{\pgfqpoint{7.890418in}{5.555456in}}%
\pgfpathlineto{\pgfqpoint{7.943462in}{5.555456in}}%
\pgfpathlineto{\pgfqpoint{7.996506in}{5.555456in}}%
\pgfpathlineto{\pgfqpoint{8.049550in}{5.555456in}}%
\pgfpathlineto{\pgfqpoint{8.102594in}{5.555456in}}%
\pgfpathlineto{\pgfqpoint{8.155638in}{5.555456in}}%
\pgfpathlineto{\pgfqpoint{8.208682in}{5.555456in}}%
\pgfpathlineto{\pgfqpoint{8.261726in}{5.555456in}}%
\pgfpathlineto{\pgfqpoint{8.314770in}{5.555456in}}%
\pgfpathlineto{\pgfqpoint{8.367814in}{5.555456in}}%
\pgfpathlineto{\pgfqpoint{8.420858in}{5.555456in}}%
\pgfpathlineto{\pgfqpoint{8.473902in}{5.555456in}}%
\pgfpathlineto{\pgfqpoint{8.526946in}{5.555456in}}%
\pgfpathlineto{\pgfqpoint{8.579990in}{5.555456in}}%
\pgfpathlineto{\pgfqpoint{8.633033in}{5.555456in}}%
\pgfpathlineto{\pgfqpoint{8.686077in}{5.555456in}}%
\pgfpathlineto{\pgfqpoint{8.739121in}{5.555456in}}%
\pgfpathlineto{\pgfqpoint{8.792165in}{5.555456in}}%
\pgfpathlineto{\pgfqpoint{8.845209in}{5.555456in}}%
\pgfpathlineto{\pgfqpoint{8.898253in}{5.555456in}}%
\pgfpathlineto{\pgfqpoint{8.951297in}{5.555456in}}%
\pgfpathlineto{\pgfqpoint{9.004341in}{5.555456in}}%
\pgfpathlineto{\pgfqpoint{9.057385in}{5.555456in}}%
\pgfpathlineto{\pgfqpoint{9.110429in}{5.555456in}}%
\pgfpathlineto{\pgfqpoint{9.163473in}{5.555456in}}%
\pgfpathlineto{\pgfqpoint{9.216517in}{5.555456in}}%
\pgfpathlineto{\pgfqpoint{9.269561in}{5.555456in}}%
\pgfpathlineto{\pgfqpoint{9.322605in}{5.555456in}}%
\pgfpathlineto{\pgfqpoint{9.375649in}{5.555456in}}%
\pgfpathlineto{\pgfqpoint{9.428692in}{5.555456in}}%
\pgfpathlineto{\pgfqpoint{9.481736in}{5.555456in}}%
\pgfpathlineto{\pgfqpoint{9.534780in}{5.555456in}}%
\pgfpathlineto{\pgfqpoint{9.587824in}{5.555456in}}%
\pgfpathlineto{\pgfqpoint{9.640868in}{5.555456in}}%
\pgfpathlineto{\pgfqpoint{9.693912in}{5.555456in}}%
\pgfpathlineto{\pgfqpoint{9.746956in}{5.555456in}}%
\pgfpathlineto{\pgfqpoint{9.800000in}{5.555456in}}%
\pgfpathlineto{\pgfqpoint{9.800000in}{5.555456in}}%
\pgfpathlineto{\pgfqpoint{9.800000in}{5.555456in}}%
\pgfpathlineto{\pgfqpoint{9.746956in}{5.555456in}}%
\pgfpathlineto{\pgfqpoint{9.693912in}{5.555456in}}%
\pgfpathlineto{\pgfqpoint{9.640868in}{5.555456in}}%
\pgfpathlineto{\pgfqpoint{9.587824in}{5.555456in}}%
\pgfpathlineto{\pgfqpoint{9.534780in}{5.555456in}}%
\pgfpathlineto{\pgfqpoint{9.481736in}{5.555456in}}%
\pgfpathlineto{\pgfqpoint{9.428692in}{5.513734in}}%
\pgfpathlineto{\pgfqpoint{9.375649in}{5.555456in}}%
\pgfpathlineto{\pgfqpoint{9.322605in}{5.555456in}}%
\pgfpathlineto{\pgfqpoint{9.269561in}{5.555456in}}%
\pgfpathlineto{\pgfqpoint{9.216517in}{5.555456in}}%
\pgfpathlineto{\pgfqpoint{9.163473in}{5.555456in}}%
\pgfpathlineto{\pgfqpoint{9.110429in}{5.357370in}}%
\pgfpathlineto{\pgfqpoint{9.057385in}{5.322277in}}%
\pgfpathlineto{\pgfqpoint{9.004341in}{5.555456in}}%
\pgfpathlineto{\pgfqpoint{8.951297in}{5.237916in}}%
\pgfpathlineto{\pgfqpoint{8.898253in}{5.555456in}}%
\pgfpathlineto{\pgfqpoint{8.845209in}{5.555456in}}%
\pgfpathlineto{\pgfqpoint{8.792165in}{5.555456in}}%
\pgfpathlineto{\pgfqpoint{8.739121in}{5.555456in}}%
\pgfpathlineto{\pgfqpoint{8.686077in}{5.555456in}}%
\pgfpathlineto{\pgfqpoint{8.633033in}{5.555456in}}%
\pgfpathlineto{\pgfqpoint{8.579990in}{5.555456in}}%
\pgfpathlineto{\pgfqpoint{8.526946in}{5.386710in}}%
\pgfpathlineto{\pgfqpoint{8.473902in}{5.414851in}}%
\pgfpathlineto{\pgfqpoint{8.420858in}{5.555456in}}%
\pgfpathlineto{\pgfqpoint{8.367814in}{5.474073in}}%
\pgfpathlineto{\pgfqpoint{8.314770in}{5.555456in}}%
\pgfpathlineto{\pgfqpoint{8.261726in}{5.555456in}}%
\pgfpathlineto{\pgfqpoint{8.208682in}{5.491656in}}%
\pgfpathlineto{\pgfqpoint{8.155638in}{5.497516in}}%
\pgfpathlineto{\pgfqpoint{8.102594in}{5.555456in}}%
\pgfpathlineto{\pgfqpoint{8.049550in}{5.555456in}}%
\pgfpathlineto{\pgfqpoint{7.996506in}{5.430879in}}%
\pgfpathlineto{\pgfqpoint{7.943462in}{5.352176in}}%
\pgfpathlineto{\pgfqpoint{7.890418in}{5.354431in}}%
\pgfpathlineto{\pgfqpoint{7.837374in}{5.313128in}}%
\pgfpathlineto{\pgfqpoint{7.784330in}{5.304877in}}%
\pgfpathlineto{\pgfqpoint{7.731287in}{5.268845in}}%
\pgfpathlineto{\pgfqpoint{7.678243in}{5.555456in}}%
\pgfpathlineto{\pgfqpoint{7.625199in}{5.555456in}}%
\pgfpathlineto{\pgfqpoint{7.572155in}{5.555456in}}%
\pgfpathlineto{\pgfqpoint{7.519111in}{5.555456in}}%
\pgfpathlineto{\pgfqpoint{7.466067in}{5.555456in}}%
\pgfpathlineto{\pgfqpoint{7.413023in}{5.555456in}}%
\pgfpathlineto{\pgfqpoint{7.359979in}{5.555456in}}%
\pgfpathlineto{\pgfqpoint{7.306935in}{5.455611in}}%
\pgfpathlineto{\pgfqpoint{7.253891in}{5.555456in}}%
\pgfpathlineto{\pgfqpoint{7.200847in}{5.555456in}}%
\pgfpathlineto{\pgfqpoint{7.147803in}{5.475721in}}%
\pgfpathlineto{\pgfqpoint{7.094759in}{5.493927in}}%
\pgfpathlineto{\pgfqpoint{7.041715in}{5.470876in}}%
\pgfpathlineto{\pgfqpoint{6.988671in}{5.555456in}}%
\pgfpathlineto{\pgfqpoint{6.935628in}{5.451613in}}%
\pgfpathlineto{\pgfqpoint{6.882584in}{5.555456in}}%
\pgfpathlineto{\pgfqpoint{6.829540in}{5.457960in}}%
\pgfpathlineto{\pgfqpoint{6.776496in}{5.555456in}}%
\pgfpathlineto{\pgfqpoint{6.723452in}{5.445993in}}%
\pgfpathlineto{\pgfqpoint{6.670408in}{5.399912in}}%
\pgfpathlineto{\pgfqpoint{6.617364in}{5.369970in}}%
\pgfpathlineto{\pgfqpoint{6.564320in}{5.303012in}}%
\pgfpathlineto{\pgfqpoint{6.511276in}{5.306663in}}%
\pgfpathlineto{\pgfqpoint{6.458232in}{5.555456in}}%
\pgfpathlineto{\pgfqpoint{6.405188in}{5.555456in}}%
\pgfpathlineto{\pgfqpoint{6.352144in}{5.555456in}}%
\pgfpathlineto{\pgfqpoint{6.299100in}{5.237069in}}%
\pgfpathlineto{\pgfqpoint{6.246056in}{5.555456in}}%
\pgfpathlineto{\pgfqpoint{6.193012in}{5.555456in}}%
\pgfpathlineto{\pgfqpoint{6.139969in}{5.316566in}}%
\pgfpathlineto{\pgfqpoint{6.086925in}{5.335040in}}%
\pgfpathlineto{\pgfqpoint{6.033881in}{5.555456in}}%
\pgfpathlineto{\pgfqpoint{5.980837in}{5.406177in}}%
\pgfpathlineto{\pgfqpoint{5.927793in}{5.555456in}}%
\pgfpathlineto{\pgfqpoint{5.874749in}{5.555456in}}%
\pgfpathlineto{\pgfqpoint{5.821705in}{5.555456in}}%
\pgfpathlineto{\pgfqpoint{5.768661in}{5.555456in}}%
\pgfpathlineto{\pgfqpoint{5.715617in}{5.482534in}}%
\pgfpathlineto{\pgfqpoint{5.662573in}{5.555456in}}%
\pgfpathlineto{\pgfqpoint{5.609529in}{5.555456in}}%
\pgfpathlineto{\pgfqpoint{5.556485in}{5.555456in}}%
\pgfpathlineto{\pgfqpoint{5.503441in}{5.476742in}}%
\pgfpathlineto{\pgfqpoint{5.450397in}{5.555456in}}%
\pgfpathlineto{\pgfqpoint{5.397353in}{5.555456in}}%
\pgfpathlineto{\pgfqpoint{5.344309in}{5.555456in}}%
\pgfpathlineto{\pgfqpoint{5.291266in}{5.335561in}}%
\pgfpathlineto{\pgfqpoint{5.238222in}{5.555456in}}%
\pgfpathlineto{\pgfqpoint{5.185178in}{5.555456in}}%
\pgfpathlineto{\pgfqpoint{5.132134in}{5.555456in}}%
\pgfpathlineto{\pgfqpoint{5.079090in}{5.207621in}}%
\pgfpathlineto{\pgfqpoint{5.026046in}{5.555456in}}%
\pgfpathlineto{\pgfqpoint{4.973002in}{5.216034in}}%
\pgfpathlineto{\pgfqpoint{4.919958in}{5.555456in}}%
\pgfpathlineto{\pgfqpoint{4.866914in}{5.555456in}}%
\pgfpathlineto{\pgfqpoint{4.813870in}{5.335037in}}%
\pgfpathlineto{\pgfqpoint{4.760826in}{5.555456in}}%
\pgfpathlineto{\pgfqpoint{4.707782in}{5.555456in}}%
\pgfpathlineto{\pgfqpoint{4.654738in}{5.454774in}}%
\pgfpathlineto{\pgfqpoint{4.601694in}{5.555456in}}%
\pgfpathlineto{\pgfqpoint{4.548650in}{5.555456in}}%
\pgfpathlineto{\pgfqpoint{4.495607in}{5.555456in}}%
\pgfpathlineto{\pgfqpoint{4.442563in}{5.555456in}}%
\pgfpathlineto{\pgfqpoint{4.389519in}{5.555456in}}%
\pgfpathlineto{\pgfqpoint{4.336475in}{5.555456in}}%
\pgfpathlineto{\pgfqpoint{4.283431in}{5.453162in}}%
\pgfpathlineto{\pgfqpoint{4.230387in}{5.437444in}}%
\pgfpathlineto{\pgfqpoint{4.177343in}{5.399336in}}%
\pgfpathlineto{\pgfqpoint{4.124299in}{5.483615in}}%
\pgfpathlineto{\pgfqpoint{4.071255in}{5.355447in}}%
\pgfpathlineto{\pgfqpoint{4.018211in}{5.273356in}}%
\pgfpathlineto{\pgfqpoint{3.965167in}{5.555456in}}%
\pgfpathlineto{\pgfqpoint{3.912123in}{5.555456in}}%
\pgfpathlineto{\pgfqpoint{3.859079in}{5.555456in}}%
\pgfpathlineto{\pgfqpoint{3.806035in}{5.267382in}}%
\pgfpathlineto{\pgfqpoint{3.752991in}{5.555456in}}%
\pgfpathlineto{\pgfqpoint{3.699948in}{5.227131in}}%
\pgfpathlineto{\pgfqpoint{3.646904in}{5.270980in}}%
\pgfpathlineto{\pgfqpoint{3.593860in}{5.555456in}}%
\pgfpathlineto{\pgfqpoint{3.540816in}{5.555456in}}%
\pgfpathlineto{\pgfqpoint{3.487772in}{5.555456in}}%
\pgfpathlineto{\pgfqpoint{3.434728in}{5.332680in}}%
\pgfpathlineto{\pgfqpoint{3.381684in}{5.555456in}}%
\pgfpathlineto{\pgfqpoint{3.328640in}{5.480294in}}%
\pgfpathlineto{\pgfqpoint{3.275596in}{5.421720in}}%
\pgfpathlineto{\pgfqpoint{3.222552in}{5.505579in}}%
\pgfpathlineto{\pgfqpoint{3.169508in}{5.475056in}}%
\pgfpathlineto{\pgfqpoint{3.116464in}{5.555456in}}%
\pgfpathlineto{\pgfqpoint{3.063420in}{5.555456in}}%
\pgfpathlineto{\pgfqpoint{3.010376in}{5.443237in}}%
\pgfpathlineto{\pgfqpoint{2.957332in}{5.555456in}}%
\pgfpathlineto{\pgfqpoint{2.904288in}{5.357472in}}%
\pgfpathlineto{\pgfqpoint{2.851245in}{5.555456in}}%
\pgfpathlineto{\pgfqpoint{2.798201in}{5.555456in}}%
\pgfpathlineto{\pgfqpoint{2.745157in}{5.555456in}}%
\pgfpathlineto{\pgfqpoint{2.692113in}{5.274799in}}%
\pgfpathlineto{\pgfqpoint{2.639069in}{5.227448in}}%
\pgfpathlineto{\pgfqpoint{2.586025in}{5.231942in}}%
\pgfpathlineto{\pgfqpoint{2.532981in}{5.253753in}}%
\pgfpathlineto{\pgfqpoint{2.479937in}{5.397997in}}%
\pgfpathlineto{\pgfqpoint{2.426893in}{5.555456in}}%
\pgfpathlineto{\pgfqpoint{2.373849in}{5.555456in}}%
\pgfpathlineto{\pgfqpoint{2.320805in}{5.278077in}}%
\pgfpathlineto{\pgfqpoint{2.267761in}{5.542464in}}%
\pgfpathlineto{\pgfqpoint{2.214717in}{5.555456in}}%
\pgfpathlineto{\pgfqpoint{2.161673in}{5.347486in}}%
\pgfpathlineto{\pgfqpoint{2.108629in}{5.555456in}}%
\pgfpathlineto{\pgfqpoint{2.055586in}{5.457838in}}%
\pgfpathlineto{\pgfqpoint{2.002542in}{5.555456in}}%
\pgfpathlineto{\pgfqpoint{1.949498in}{5.447032in}}%
\pgfpathlineto{\pgfqpoint{1.896454in}{5.555456in}}%
\pgfpathlineto{\pgfqpoint{1.843410in}{5.555456in}}%
\pgfpathlineto{\pgfqpoint{1.790366in}{5.473721in}}%
\pgfpathlineto{\pgfqpoint{1.737322in}{5.491300in}}%
\pgfpathlineto{\pgfqpoint{1.684278in}{5.555456in}}%
\pgfpathlineto{\pgfqpoint{1.631234in}{5.555456in}}%
\pgfpathlineto{\pgfqpoint{1.578190in}{5.343668in}}%
\pgfpathlineto{\pgfqpoint{1.525146in}{5.349185in}}%
\pgfpathlineto{\pgfqpoint{1.472102in}{5.268823in}}%
\pgfpathlineto{\pgfqpoint{1.419058in}{5.263460in}}%
\pgfpathlineto{\pgfqpoint{1.366014in}{5.207621in}}%
\pgfpathlineto{\pgfqpoint{1.312970in}{5.555456in}}%
\pgfpathlineto{\pgfqpoint{1.259927in}{5.555456in}}%
\pgfpathlineto{\pgfqpoint{1.206883in}{5.555456in}}%
\pgfpathlineto{\pgfqpoint{1.153839in}{5.253587in}}%
\pgfpathlineto{\pgfqpoint{1.100795in}{5.267250in}}%
\pgfpathlineto{\pgfqpoint{1.047751in}{5.298031in}}%
\pgfpathlineto{\pgfqpoint{0.994707in}{5.555456in}}%
\pgfpathlineto{\pgfqpoint{0.941663in}{5.302080in}}%
\pgfpathlineto{\pgfqpoint{0.941663in}{5.302080in}}%
\pgfpathclose%
\pgfusepath{stroke,fill}%
}%
\begin{pgfscope}%
\pgfsys@transformshift{0.000000in}{0.000000in}%
\pgfsys@useobject{currentmarker}{}%
\end{pgfscope}%
\end{pgfscope}%
\begin{pgfscope}%
\pgfpathrectangle{\pgfqpoint{0.941663in}{4.334375in}}{\pgfqpoint{8.858337in}{3.465625in}}%
\pgfusepath{clip}%
\pgfsetrectcap%
\pgfsetroundjoin%
\pgfsetlinewidth{1.505625pt}%
\definecolor{currentstroke}{rgb}{0.501961,0.501961,0.501961}%
\pgfsetstrokecolor{currentstroke}%
\pgfsetdash{}{0pt}%
\pgfpathmoveto{\pgfqpoint{0.941663in}{4.606409in}}%
\pgfpathlineto{\pgfqpoint{0.994707in}{4.859785in}}%
\pgfpathlineto{\pgfqpoint{1.047751in}{4.602360in}}%
\pgfpathlineto{\pgfqpoint{1.100795in}{4.571579in}}%
\pgfpathlineto{\pgfqpoint{1.153839in}{4.557915in}}%
\pgfpathlineto{\pgfqpoint{1.206883in}{5.127392in}}%
\pgfpathlineto{\pgfqpoint{1.259927in}{5.369914in}}%
\pgfpathlineto{\pgfqpoint{1.312970in}{5.172617in}}%
\pgfpathlineto{\pgfqpoint{1.366014in}{4.493572in}}%
\pgfpathlineto{\pgfqpoint{1.419058in}{4.567788in}}%
\pgfpathlineto{\pgfqpoint{1.472102in}{4.573151in}}%
\pgfpathlineto{\pgfqpoint{1.525146in}{4.653514in}}%
\pgfpathlineto{\pgfqpoint{1.578190in}{4.630644in}}%
\pgfpathlineto{\pgfqpoint{1.631234in}{5.423055in}}%
\pgfpathlineto{\pgfqpoint{1.684278in}{4.875476in}}%
\pgfpathlineto{\pgfqpoint{1.737322in}{4.795628in}}%
\pgfpathlineto{\pgfqpoint{1.790366in}{4.778049in}}%
\pgfpathlineto{\pgfqpoint{1.843410in}{5.555456in}}%
\pgfpathlineto{\pgfqpoint{1.896454in}{5.555456in}}%
\pgfpathlineto{\pgfqpoint{1.949498in}{4.751360in}}%
\pgfpathlineto{\pgfqpoint{2.002542in}{5.555456in}}%
\pgfpathlineto{\pgfqpoint{2.055586in}{4.762166in}}%
\pgfpathlineto{\pgfqpoint{2.108629in}{5.048281in}}%
\pgfpathlineto{\pgfqpoint{2.161673in}{4.651814in}}%
\pgfpathlineto{\pgfqpoint{2.214717in}{5.555456in}}%
\pgfpathlineto{\pgfqpoint{2.267761in}{4.846792in}}%
\pgfpathlineto{\pgfqpoint{2.320805in}{4.582406in}}%
\pgfpathlineto{\pgfqpoint{2.373849in}{5.523078in}}%
\pgfpathlineto{\pgfqpoint{2.426893in}{5.555456in}}%
\pgfpathlineto{\pgfqpoint{2.479937in}{4.702326in}}%
\pgfpathlineto{\pgfqpoint{2.532981in}{4.558081in}}%
\pgfpathlineto{\pgfqpoint{2.586025in}{4.536270in}}%
\pgfpathlineto{\pgfqpoint{2.639069in}{4.531777in}}%
\pgfpathlineto{\pgfqpoint{2.692113in}{4.501574in}}%
\pgfpathlineto{\pgfqpoint{2.745157in}{4.572498in}}%
\pgfpathlineto{\pgfqpoint{2.798201in}{4.614894in}}%
\pgfpathlineto{\pgfqpoint{2.851245in}{5.097809in}}%
\pgfpathlineto{\pgfqpoint{2.904288in}{4.661801in}}%
\pgfpathlineto{\pgfqpoint{2.957332in}{5.555456in}}%
\pgfpathlineto{\pgfqpoint{3.010376in}{4.747565in}}%
\pgfpathlineto{\pgfqpoint{3.063420in}{5.529745in}}%
\pgfpathlineto{\pgfqpoint{3.116464in}{4.873985in}}%
\pgfpathlineto{\pgfqpoint{3.169508in}{4.779384in}}%
\pgfpathlineto{\pgfqpoint{3.222552in}{4.809907in}}%
\pgfpathlineto{\pgfqpoint{3.275596in}{4.726048in}}%
\pgfpathlineto{\pgfqpoint{3.328640in}{4.784623in}}%
\pgfpathlineto{\pgfqpoint{3.381684in}{4.982873in}}%
\pgfpathlineto{\pgfqpoint{3.434728in}{4.637008in}}%
\pgfpathlineto{\pgfqpoint{3.487772in}{5.331044in}}%
\pgfpathlineto{\pgfqpoint{3.540816in}{5.555456in}}%
\pgfpathlineto{\pgfqpoint{3.593860in}{5.052711in}}%
\pgfpathlineto{\pgfqpoint{3.646904in}{4.575309in}}%
\pgfpathlineto{\pgfqpoint{3.699948in}{4.531459in}}%
\pgfpathlineto{\pgfqpoint{3.752991in}{4.859785in}}%
\pgfpathlineto{\pgfqpoint{3.806035in}{4.571710in}}%
\pgfpathlineto{\pgfqpoint{3.859079in}{4.859785in}}%
\pgfpathlineto{\pgfqpoint{3.912123in}{4.859785in}}%
\pgfpathlineto{\pgfqpoint{3.965167in}{5.299182in}}%
\pgfpathlineto{\pgfqpoint{4.018211in}{4.577684in}}%
\pgfpathlineto{\pgfqpoint{4.071255in}{4.659775in}}%
\pgfpathlineto{\pgfqpoint{4.124299in}{4.787943in}}%
\pgfpathlineto{\pgfqpoint{4.177343in}{4.703664in}}%
\pgfpathlineto{\pgfqpoint{4.230387in}{4.741772in}}%
\pgfpathlineto{\pgfqpoint{4.283431in}{4.757491in}}%
\pgfpathlineto{\pgfqpoint{4.336475in}{5.555456in}}%
\pgfpathlineto{\pgfqpoint{4.389519in}{4.859785in}}%
\pgfpathlineto{\pgfqpoint{4.442563in}{4.896628in}}%
\pgfpathlineto{\pgfqpoint{4.495607in}{5.555456in}}%
\pgfpathlineto{\pgfqpoint{4.548650in}{5.101868in}}%
\pgfpathlineto{\pgfqpoint{4.601694in}{5.555456in}}%
\pgfpathlineto{\pgfqpoint{4.654738in}{4.759102in}}%
\pgfpathlineto{\pgfqpoint{4.707782in}{5.555456in}}%
\pgfpathlineto{\pgfqpoint{4.760826in}{4.951917in}}%
\pgfpathlineto{\pgfqpoint{4.813870in}{4.639365in}}%
\pgfpathlineto{\pgfqpoint{4.866914in}{5.555456in}}%
\pgfpathlineto{\pgfqpoint{4.919958in}{5.555456in}}%
\pgfpathlineto{\pgfqpoint{4.973002in}{4.520363in}}%
\pgfpathlineto{\pgfqpoint{5.026046in}{5.387465in}}%
\pgfpathlineto{\pgfqpoint{5.079090in}{4.491903in}}%
\pgfpathlineto{\pgfqpoint{5.132134in}{5.444876in}}%
\pgfpathlineto{\pgfqpoint{5.185178in}{5.555456in}}%
\pgfpathlineto{\pgfqpoint{5.238222in}{5.555456in}}%
\pgfpathlineto{\pgfqpoint{5.291266in}{4.639889in}}%
\pgfpathlineto{\pgfqpoint{5.344309in}{5.453871in}}%
\pgfpathlineto{\pgfqpoint{5.397353in}{5.555456in}}%
\pgfpathlineto{\pgfqpoint{5.450397in}{5.555456in}}%
\pgfpathlineto{\pgfqpoint{5.503441in}{4.781070in}}%
\pgfpathlineto{\pgfqpoint{5.556485in}{5.135041in}}%
\pgfpathlineto{\pgfqpoint{5.609529in}{5.555456in}}%
\pgfpathlineto{\pgfqpoint{5.662573in}{5.555456in}}%
\pgfpathlineto{\pgfqpoint{5.715617in}{4.786862in}}%
\pgfpathlineto{\pgfqpoint{5.768661in}{5.555456in}}%
\pgfpathlineto{\pgfqpoint{5.927793in}{5.555456in}}%
\pgfpathlineto{\pgfqpoint{5.980837in}{4.710506in}}%
\pgfpathlineto{\pgfqpoint{6.033881in}{5.193362in}}%
\pgfpathlineto{\pgfqpoint{6.086925in}{4.639368in}}%
\pgfpathlineto{\pgfqpoint{6.139969in}{4.620894in}}%
\pgfpathlineto{\pgfqpoint{6.193012in}{5.555456in}}%
\pgfpathlineto{\pgfqpoint{6.246056in}{5.555456in}}%
\pgfpathlineto{\pgfqpoint{6.299100in}{4.541397in}}%
\pgfpathlineto{\pgfqpoint{6.352144in}{5.178906in}}%
\pgfpathlineto{\pgfqpoint{6.405188in}{5.555456in}}%
\pgfpathlineto{\pgfqpoint{6.458232in}{5.077516in}}%
\pgfpathlineto{\pgfqpoint{6.511276in}{4.610991in}}%
\pgfpathlineto{\pgfqpoint{6.564320in}{4.607341in}}%
\pgfpathlineto{\pgfqpoint{6.617364in}{4.674298in}}%
\pgfpathlineto{\pgfqpoint{6.670408in}{4.704240in}}%
\pgfpathlineto{\pgfqpoint{6.723452in}{4.750321in}}%
\pgfpathlineto{\pgfqpoint{6.776496in}{4.859785in}}%
\pgfpathlineto{\pgfqpoint{6.829540in}{4.762288in}}%
\pgfpathlineto{\pgfqpoint{6.882584in}{5.555456in}}%
\pgfpathlineto{\pgfqpoint{6.935628in}{4.755942in}}%
\pgfpathlineto{\pgfqpoint{6.988671in}{5.000909in}}%
\pgfpathlineto{\pgfqpoint{7.041715in}{4.775204in}}%
\pgfpathlineto{\pgfqpoint{7.094759in}{4.798255in}}%
\pgfpathlineto{\pgfqpoint{7.147803in}{4.780049in}}%
\pgfpathlineto{\pgfqpoint{7.200847in}{5.555456in}}%
\pgfpathlineto{\pgfqpoint{7.253891in}{4.923953in}}%
\pgfpathlineto{\pgfqpoint{7.306935in}{4.759939in}}%
\pgfpathlineto{\pgfqpoint{7.359979in}{5.555456in}}%
\pgfpathlineto{\pgfqpoint{7.413023in}{5.555456in}}%
\pgfpathlineto{\pgfqpoint{7.466067in}{5.147602in}}%
\pgfpathlineto{\pgfqpoint{7.519111in}{5.074761in}}%
\pgfpathlineto{\pgfqpoint{7.572155in}{5.555456in}}%
\pgfpathlineto{\pgfqpoint{7.625199in}{5.555456in}}%
\pgfpathlineto{\pgfqpoint{7.678243in}{5.513983in}}%
\pgfpathlineto{\pgfqpoint{7.731287in}{4.573173in}}%
\pgfpathlineto{\pgfqpoint{7.784330in}{4.609205in}}%
\pgfpathlineto{\pgfqpoint{7.837374in}{4.617456in}}%
\pgfpathlineto{\pgfqpoint{7.890418in}{4.658759in}}%
\pgfpathlineto{\pgfqpoint{7.943462in}{4.656505in}}%
\pgfpathlineto{\pgfqpoint{7.996506in}{4.735208in}}%
\pgfpathlineto{\pgfqpoint{8.049550in}{5.555456in}}%
\pgfpathlineto{\pgfqpoint{8.102594in}{5.555456in}}%
\pgfpathlineto{\pgfqpoint{8.155638in}{4.801844in}}%
\pgfpathlineto{\pgfqpoint{8.208682in}{4.795985in}}%
\pgfpathlineto{\pgfqpoint{8.261726in}{5.555456in}}%
\pgfpathlineto{\pgfqpoint{8.314770in}{5.555456in}}%
\pgfpathlineto{\pgfqpoint{8.367814in}{4.778402in}}%
\pgfpathlineto{\pgfqpoint{8.420858in}{5.555456in}}%
\pgfpathlineto{\pgfqpoint{8.473902in}{4.719180in}}%
\pgfpathlineto{\pgfqpoint{8.526946in}{4.691039in}}%
\pgfpathlineto{\pgfqpoint{8.579990in}{5.555456in}}%
\pgfpathlineto{\pgfqpoint{8.739121in}{5.555456in}}%
\pgfpathlineto{\pgfqpoint{8.792165in}{5.420011in}}%
\pgfpathlineto{\pgfqpoint{8.845209in}{5.555456in}}%
\pgfpathlineto{\pgfqpoint{8.898253in}{5.555456in}}%
\pgfpathlineto{\pgfqpoint{8.951297in}{4.542245in}}%
\pgfpathlineto{\pgfqpoint{9.004341in}{4.990837in}}%
\pgfpathlineto{\pgfqpoint{9.057385in}{4.626605in}}%
\pgfpathlineto{\pgfqpoint{9.110429in}{4.661699in}}%
\pgfpathlineto{\pgfqpoint{9.163473in}{5.555456in}}%
\pgfpathlineto{\pgfqpoint{9.375649in}{5.555456in}}%
\pgfpathlineto{\pgfqpoint{9.428692in}{4.818062in}}%
\pgfpathlineto{\pgfqpoint{9.481736in}{4.859785in}}%
\pgfpathlineto{\pgfqpoint{9.534780in}{5.555456in}}%
\pgfpathlineto{\pgfqpoint{9.587824in}{5.555456in}}%
\pgfpathlineto{\pgfqpoint{9.640868in}{5.281518in}}%
\pgfpathlineto{\pgfqpoint{9.693912in}{5.555456in}}%
\pgfpathlineto{\pgfqpoint{9.800000in}{5.555456in}}%
\pgfpathlineto{\pgfqpoint{9.800000in}{5.555456in}}%
\pgfusepath{stroke}%
\end{pgfscope}%
\begin{pgfscope}%
\pgfpathrectangle{\pgfqpoint{0.941663in}{4.334375in}}{\pgfqpoint{8.858337in}{3.465625in}}%
\pgfusepath{clip}%
\pgfsetbuttcap%
\pgfsetroundjoin%
\definecolor{currentfill}{rgb}{0.501961,0.501961,0.501961}%
\pgfsetfillcolor{currentfill}%
\pgfsetlinewidth{1.003750pt}%
\definecolor{currentstroke}{rgb}{0.501961,0.501961,0.501961}%
\pgfsetstrokecolor{currentstroke}%
\pgfsetdash{}{0pt}%
\pgfsys@defobject{currentmarker}{\pgfqpoint{0.941663in}{4.491903in}}{\pgfqpoint{9.800000in}{5.555456in}}{%
\pgfpathmoveto{\pgfqpoint{0.941663in}{4.606409in}}%
\pgfpathlineto{\pgfqpoint{0.941663in}{5.302080in}}%
\pgfpathlineto{\pgfqpoint{0.994707in}{5.555456in}}%
\pgfpathlineto{\pgfqpoint{1.047751in}{5.298031in}}%
\pgfpathlineto{\pgfqpoint{1.100795in}{5.267250in}}%
\pgfpathlineto{\pgfqpoint{1.153839in}{5.253587in}}%
\pgfpathlineto{\pgfqpoint{1.206883in}{5.555456in}}%
\pgfpathlineto{\pgfqpoint{1.259927in}{5.555456in}}%
\pgfpathlineto{\pgfqpoint{1.312970in}{5.555456in}}%
\pgfpathlineto{\pgfqpoint{1.366014in}{5.207621in}}%
\pgfpathlineto{\pgfqpoint{1.419058in}{5.263460in}}%
\pgfpathlineto{\pgfqpoint{1.472102in}{5.268823in}}%
\pgfpathlineto{\pgfqpoint{1.525146in}{5.349185in}}%
\pgfpathlineto{\pgfqpoint{1.578190in}{5.343668in}}%
\pgfpathlineto{\pgfqpoint{1.631234in}{5.555456in}}%
\pgfpathlineto{\pgfqpoint{1.684278in}{5.555456in}}%
\pgfpathlineto{\pgfqpoint{1.737322in}{5.491300in}}%
\pgfpathlineto{\pgfqpoint{1.790366in}{5.473721in}}%
\pgfpathlineto{\pgfqpoint{1.843410in}{5.555456in}}%
\pgfpathlineto{\pgfqpoint{1.896454in}{5.555456in}}%
\pgfpathlineto{\pgfqpoint{1.949498in}{5.447032in}}%
\pgfpathlineto{\pgfqpoint{2.002542in}{5.555456in}}%
\pgfpathlineto{\pgfqpoint{2.055586in}{5.457838in}}%
\pgfpathlineto{\pgfqpoint{2.108629in}{5.555456in}}%
\pgfpathlineto{\pgfqpoint{2.161673in}{5.347486in}}%
\pgfpathlineto{\pgfqpoint{2.214717in}{5.555456in}}%
\pgfpathlineto{\pgfqpoint{2.267761in}{5.542464in}}%
\pgfpathlineto{\pgfqpoint{2.320805in}{5.278077in}}%
\pgfpathlineto{\pgfqpoint{2.373849in}{5.555456in}}%
\pgfpathlineto{\pgfqpoint{2.426893in}{5.555456in}}%
\pgfpathlineto{\pgfqpoint{2.479937in}{5.397997in}}%
\pgfpathlineto{\pgfqpoint{2.532981in}{5.253753in}}%
\pgfpathlineto{\pgfqpoint{2.586025in}{5.231942in}}%
\pgfpathlineto{\pgfqpoint{2.639069in}{5.227448in}}%
\pgfpathlineto{\pgfqpoint{2.692113in}{5.274799in}}%
\pgfpathlineto{\pgfqpoint{2.745157in}{5.555456in}}%
\pgfpathlineto{\pgfqpoint{2.798201in}{5.555456in}}%
\pgfpathlineto{\pgfqpoint{2.851245in}{5.555456in}}%
\pgfpathlineto{\pgfqpoint{2.904288in}{5.357472in}}%
\pgfpathlineto{\pgfqpoint{2.957332in}{5.555456in}}%
\pgfpathlineto{\pgfqpoint{3.010376in}{5.443237in}}%
\pgfpathlineto{\pgfqpoint{3.063420in}{5.555456in}}%
\pgfpathlineto{\pgfqpoint{3.116464in}{5.555456in}}%
\pgfpathlineto{\pgfqpoint{3.169508in}{5.475056in}}%
\pgfpathlineto{\pgfqpoint{3.222552in}{5.505579in}}%
\pgfpathlineto{\pgfqpoint{3.275596in}{5.421720in}}%
\pgfpathlineto{\pgfqpoint{3.328640in}{5.480294in}}%
\pgfpathlineto{\pgfqpoint{3.381684in}{5.555456in}}%
\pgfpathlineto{\pgfqpoint{3.434728in}{5.332680in}}%
\pgfpathlineto{\pgfqpoint{3.487772in}{5.555456in}}%
\pgfpathlineto{\pgfqpoint{3.540816in}{5.555456in}}%
\pgfpathlineto{\pgfqpoint{3.593860in}{5.555456in}}%
\pgfpathlineto{\pgfqpoint{3.646904in}{5.270980in}}%
\pgfpathlineto{\pgfqpoint{3.699948in}{5.227131in}}%
\pgfpathlineto{\pgfqpoint{3.752991in}{5.555456in}}%
\pgfpathlineto{\pgfqpoint{3.806035in}{5.267382in}}%
\pgfpathlineto{\pgfqpoint{3.859079in}{5.555456in}}%
\pgfpathlineto{\pgfqpoint{3.912123in}{5.555456in}}%
\pgfpathlineto{\pgfqpoint{3.965167in}{5.555456in}}%
\pgfpathlineto{\pgfqpoint{4.018211in}{5.273356in}}%
\pgfpathlineto{\pgfqpoint{4.071255in}{5.355447in}}%
\pgfpathlineto{\pgfqpoint{4.124299in}{5.483615in}}%
\pgfpathlineto{\pgfqpoint{4.177343in}{5.399336in}}%
\pgfpathlineto{\pgfqpoint{4.230387in}{5.437444in}}%
\pgfpathlineto{\pgfqpoint{4.283431in}{5.453162in}}%
\pgfpathlineto{\pgfqpoint{4.336475in}{5.555456in}}%
\pgfpathlineto{\pgfqpoint{4.389519in}{5.555456in}}%
\pgfpathlineto{\pgfqpoint{4.442563in}{5.555456in}}%
\pgfpathlineto{\pgfqpoint{4.495607in}{5.555456in}}%
\pgfpathlineto{\pgfqpoint{4.548650in}{5.555456in}}%
\pgfpathlineto{\pgfqpoint{4.601694in}{5.555456in}}%
\pgfpathlineto{\pgfqpoint{4.654738in}{5.454774in}}%
\pgfpathlineto{\pgfqpoint{4.707782in}{5.555456in}}%
\pgfpathlineto{\pgfqpoint{4.760826in}{5.555456in}}%
\pgfpathlineto{\pgfqpoint{4.813870in}{5.335037in}}%
\pgfpathlineto{\pgfqpoint{4.866914in}{5.555456in}}%
\pgfpathlineto{\pgfqpoint{4.919958in}{5.555456in}}%
\pgfpathlineto{\pgfqpoint{4.973002in}{5.216034in}}%
\pgfpathlineto{\pgfqpoint{5.026046in}{5.555456in}}%
\pgfpathlineto{\pgfqpoint{5.079090in}{5.207621in}}%
\pgfpathlineto{\pgfqpoint{5.132134in}{5.555456in}}%
\pgfpathlineto{\pgfqpoint{5.185178in}{5.555456in}}%
\pgfpathlineto{\pgfqpoint{5.238222in}{5.555456in}}%
\pgfpathlineto{\pgfqpoint{5.291266in}{5.335561in}}%
\pgfpathlineto{\pgfqpoint{5.344309in}{5.555456in}}%
\pgfpathlineto{\pgfqpoint{5.397353in}{5.555456in}}%
\pgfpathlineto{\pgfqpoint{5.450397in}{5.555456in}}%
\pgfpathlineto{\pgfqpoint{5.503441in}{5.476742in}}%
\pgfpathlineto{\pgfqpoint{5.556485in}{5.555456in}}%
\pgfpathlineto{\pgfqpoint{5.609529in}{5.555456in}}%
\pgfpathlineto{\pgfqpoint{5.662573in}{5.555456in}}%
\pgfpathlineto{\pgfqpoint{5.715617in}{5.482534in}}%
\pgfpathlineto{\pgfqpoint{5.768661in}{5.555456in}}%
\pgfpathlineto{\pgfqpoint{5.821705in}{5.555456in}}%
\pgfpathlineto{\pgfqpoint{5.874749in}{5.555456in}}%
\pgfpathlineto{\pgfqpoint{5.927793in}{5.555456in}}%
\pgfpathlineto{\pgfqpoint{5.980837in}{5.406177in}}%
\pgfpathlineto{\pgfqpoint{6.033881in}{5.555456in}}%
\pgfpathlineto{\pgfqpoint{6.086925in}{5.335040in}}%
\pgfpathlineto{\pgfqpoint{6.139969in}{5.316566in}}%
\pgfpathlineto{\pgfqpoint{6.193012in}{5.555456in}}%
\pgfpathlineto{\pgfqpoint{6.246056in}{5.555456in}}%
\pgfpathlineto{\pgfqpoint{6.299100in}{5.237069in}}%
\pgfpathlineto{\pgfqpoint{6.352144in}{5.555456in}}%
\pgfpathlineto{\pgfqpoint{6.405188in}{5.555456in}}%
\pgfpathlineto{\pgfqpoint{6.458232in}{5.555456in}}%
\pgfpathlineto{\pgfqpoint{6.511276in}{5.306663in}}%
\pgfpathlineto{\pgfqpoint{6.564320in}{5.303012in}}%
\pgfpathlineto{\pgfqpoint{6.617364in}{5.369970in}}%
\pgfpathlineto{\pgfqpoint{6.670408in}{5.399912in}}%
\pgfpathlineto{\pgfqpoint{6.723452in}{5.445993in}}%
\pgfpathlineto{\pgfqpoint{6.776496in}{5.555456in}}%
\pgfpathlineto{\pgfqpoint{6.829540in}{5.457960in}}%
\pgfpathlineto{\pgfqpoint{6.882584in}{5.555456in}}%
\pgfpathlineto{\pgfqpoint{6.935628in}{5.451613in}}%
\pgfpathlineto{\pgfqpoint{6.988671in}{5.555456in}}%
\pgfpathlineto{\pgfqpoint{7.041715in}{5.470876in}}%
\pgfpathlineto{\pgfqpoint{7.094759in}{5.493927in}}%
\pgfpathlineto{\pgfqpoint{7.147803in}{5.475721in}}%
\pgfpathlineto{\pgfqpoint{7.200847in}{5.555456in}}%
\pgfpathlineto{\pgfqpoint{7.253891in}{5.555456in}}%
\pgfpathlineto{\pgfqpoint{7.306935in}{5.455611in}}%
\pgfpathlineto{\pgfqpoint{7.359979in}{5.555456in}}%
\pgfpathlineto{\pgfqpoint{7.413023in}{5.555456in}}%
\pgfpathlineto{\pgfqpoint{7.466067in}{5.555456in}}%
\pgfpathlineto{\pgfqpoint{7.519111in}{5.555456in}}%
\pgfpathlineto{\pgfqpoint{7.572155in}{5.555456in}}%
\pgfpathlineto{\pgfqpoint{7.625199in}{5.555456in}}%
\pgfpathlineto{\pgfqpoint{7.678243in}{5.555456in}}%
\pgfpathlineto{\pgfqpoint{7.731287in}{5.268845in}}%
\pgfpathlineto{\pgfqpoint{7.784330in}{5.304877in}}%
\pgfpathlineto{\pgfqpoint{7.837374in}{5.313128in}}%
\pgfpathlineto{\pgfqpoint{7.890418in}{5.354431in}}%
\pgfpathlineto{\pgfqpoint{7.943462in}{5.352176in}}%
\pgfpathlineto{\pgfqpoint{7.996506in}{5.430879in}}%
\pgfpathlineto{\pgfqpoint{8.049550in}{5.555456in}}%
\pgfpathlineto{\pgfqpoint{8.102594in}{5.555456in}}%
\pgfpathlineto{\pgfqpoint{8.155638in}{5.497516in}}%
\pgfpathlineto{\pgfqpoint{8.208682in}{5.491656in}}%
\pgfpathlineto{\pgfqpoint{8.261726in}{5.555456in}}%
\pgfpathlineto{\pgfqpoint{8.314770in}{5.555456in}}%
\pgfpathlineto{\pgfqpoint{8.367814in}{5.474073in}}%
\pgfpathlineto{\pgfqpoint{8.420858in}{5.555456in}}%
\pgfpathlineto{\pgfqpoint{8.473902in}{5.414851in}}%
\pgfpathlineto{\pgfqpoint{8.526946in}{5.386710in}}%
\pgfpathlineto{\pgfqpoint{8.579990in}{5.555456in}}%
\pgfpathlineto{\pgfqpoint{8.633033in}{5.555456in}}%
\pgfpathlineto{\pgfqpoint{8.686077in}{5.555456in}}%
\pgfpathlineto{\pgfqpoint{8.739121in}{5.555456in}}%
\pgfpathlineto{\pgfqpoint{8.792165in}{5.555456in}}%
\pgfpathlineto{\pgfqpoint{8.845209in}{5.555456in}}%
\pgfpathlineto{\pgfqpoint{8.898253in}{5.555456in}}%
\pgfpathlineto{\pgfqpoint{8.951297in}{5.237916in}}%
\pgfpathlineto{\pgfqpoint{9.004341in}{5.555456in}}%
\pgfpathlineto{\pgfqpoint{9.057385in}{5.322277in}}%
\pgfpathlineto{\pgfqpoint{9.110429in}{5.357370in}}%
\pgfpathlineto{\pgfqpoint{9.163473in}{5.555456in}}%
\pgfpathlineto{\pgfqpoint{9.216517in}{5.555456in}}%
\pgfpathlineto{\pgfqpoint{9.269561in}{5.555456in}}%
\pgfpathlineto{\pgfqpoint{9.322605in}{5.555456in}}%
\pgfpathlineto{\pgfqpoint{9.375649in}{5.555456in}}%
\pgfpathlineto{\pgfqpoint{9.428692in}{5.513734in}}%
\pgfpathlineto{\pgfqpoint{9.481736in}{5.555456in}}%
\pgfpathlineto{\pgfqpoint{9.534780in}{5.555456in}}%
\pgfpathlineto{\pgfqpoint{9.587824in}{5.555456in}}%
\pgfpathlineto{\pgfqpoint{9.640868in}{5.555456in}}%
\pgfpathlineto{\pgfqpoint{9.693912in}{5.555456in}}%
\pgfpathlineto{\pgfqpoint{9.746956in}{5.555456in}}%
\pgfpathlineto{\pgfqpoint{9.800000in}{5.555456in}}%
\pgfpathlineto{\pgfqpoint{9.800000in}{5.555456in}}%
\pgfpathlineto{\pgfqpoint{9.800000in}{5.555456in}}%
\pgfpathlineto{\pgfqpoint{9.746956in}{5.555456in}}%
\pgfpathlineto{\pgfqpoint{9.693912in}{5.555456in}}%
\pgfpathlineto{\pgfqpoint{9.640868in}{5.281518in}}%
\pgfpathlineto{\pgfqpoint{9.587824in}{5.555456in}}%
\pgfpathlineto{\pgfqpoint{9.534780in}{5.555456in}}%
\pgfpathlineto{\pgfqpoint{9.481736in}{4.859785in}}%
\pgfpathlineto{\pgfqpoint{9.428692in}{4.818062in}}%
\pgfpathlineto{\pgfqpoint{9.375649in}{5.555456in}}%
\pgfpathlineto{\pgfqpoint{9.322605in}{5.555456in}}%
\pgfpathlineto{\pgfqpoint{9.269561in}{5.555456in}}%
\pgfpathlineto{\pgfqpoint{9.216517in}{5.555456in}}%
\pgfpathlineto{\pgfqpoint{9.163473in}{5.555456in}}%
\pgfpathlineto{\pgfqpoint{9.110429in}{4.661699in}}%
\pgfpathlineto{\pgfqpoint{9.057385in}{4.626605in}}%
\pgfpathlineto{\pgfqpoint{9.004341in}{4.990837in}}%
\pgfpathlineto{\pgfqpoint{8.951297in}{4.542245in}}%
\pgfpathlineto{\pgfqpoint{8.898253in}{5.555456in}}%
\pgfpathlineto{\pgfqpoint{8.845209in}{5.555456in}}%
\pgfpathlineto{\pgfqpoint{8.792165in}{5.420011in}}%
\pgfpathlineto{\pgfqpoint{8.739121in}{5.555456in}}%
\pgfpathlineto{\pgfqpoint{8.686077in}{5.555456in}}%
\pgfpathlineto{\pgfqpoint{8.633033in}{5.555456in}}%
\pgfpathlineto{\pgfqpoint{8.579990in}{5.555456in}}%
\pgfpathlineto{\pgfqpoint{8.526946in}{4.691039in}}%
\pgfpathlineto{\pgfqpoint{8.473902in}{4.719180in}}%
\pgfpathlineto{\pgfqpoint{8.420858in}{5.555456in}}%
\pgfpathlineto{\pgfqpoint{8.367814in}{4.778402in}}%
\pgfpathlineto{\pgfqpoint{8.314770in}{5.555456in}}%
\pgfpathlineto{\pgfqpoint{8.261726in}{5.555456in}}%
\pgfpathlineto{\pgfqpoint{8.208682in}{4.795985in}}%
\pgfpathlineto{\pgfqpoint{8.155638in}{4.801844in}}%
\pgfpathlineto{\pgfqpoint{8.102594in}{5.555456in}}%
\pgfpathlineto{\pgfqpoint{8.049550in}{5.555456in}}%
\pgfpathlineto{\pgfqpoint{7.996506in}{4.735208in}}%
\pgfpathlineto{\pgfqpoint{7.943462in}{4.656505in}}%
\pgfpathlineto{\pgfqpoint{7.890418in}{4.658759in}}%
\pgfpathlineto{\pgfqpoint{7.837374in}{4.617456in}}%
\pgfpathlineto{\pgfqpoint{7.784330in}{4.609205in}}%
\pgfpathlineto{\pgfqpoint{7.731287in}{4.573173in}}%
\pgfpathlineto{\pgfqpoint{7.678243in}{5.513983in}}%
\pgfpathlineto{\pgfqpoint{7.625199in}{5.555456in}}%
\pgfpathlineto{\pgfqpoint{7.572155in}{5.555456in}}%
\pgfpathlineto{\pgfqpoint{7.519111in}{5.074761in}}%
\pgfpathlineto{\pgfqpoint{7.466067in}{5.147602in}}%
\pgfpathlineto{\pgfqpoint{7.413023in}{5.555456in}}%
\pgfpathlineto{\pgfqpoint{7.359979in}{5.555456in}}%
\pgfpathlineto{\pgfqpoint{7.306935in}{4.759939in}}%
\pgfpathlineto{\pgfqpoint{7.253891in}{4.923953in}}%
\pgfpathlineto{\pgfqpoint{7.200847in}{5.555456in}}%
\pgfpathlineto{\pgfqpoint{7.147803in}{4.780049in}}%
\pgfpathlineto{\pgfqpoint{7.094759in}{4.798255in}}%
\pgfpathlineto{\pgfqpoint{7.041715in}{4.775204in}}%
\pgfpathlineto{\pgfqpoint{6.988671in}{5.000909in}}%
\pgfpathlineto{\pgfqpoint{6.935628in}{4.755942in}}%
\pgfpathlineto{\pgfqpoint{6.882584in}{5.555456in}}%
\pgfpathlineto{\pgfqpoint{6.829540in}{4.762288in}}%
\pgfpathlineto{\pgfqpoint{6.776496in}{4.859785in}}%
\pgfpathlineto{\pgfqpoint{6.723452in}{4.750321in}}%
\pgfpathlineto{\pgfqpoint{6.670408in}{4.704240in}}%
\pgfpathlineto{\pgfqpoint{6.617364in}{4.674298in}}%
\pgfpathlineto{\pgfqpoint{6.564320in}{4.607341in}}%
\pgfpathlineto{\pgfqpoint{6.511276in}{4.610991in}}%
\pgfpathlineto{\pgfqpoint{6.458232in}{5.077516in}}%
\pgfpathlineto{\pgfqpoint{6.405188in}{5.555456in}}%
\pgfpathlineto{\pgfqpoint{6.352144in}{5.178906in}}%
\pgfpathlineto{\pgfqpoint{6.299100in}{4.541397in}}%
\pgfpathlineto{\pgfqpoint{6.246056in}{5.555456in}}%
\pgfpathlineto{\pgfqpoint{6.193012in}{5.555456in}}%
\pgfpathlineto{\pgfqpoint{6.139969in}{4.620894in}}%
\pgfpathlineto{\pgfqpoint{6.086925in}{4.639368in}}%
\pgfpathlineto{\pgfqpoint{6.033881in}{5.193362in}}%
\pgfpathlineto{\pgfqpoint{5.980837in}{4.710506in}}%
\pgfpathlineto{\pgfqpoint{5.927793in}{5.555456in}}%
\pgfpathlineto{\pgfqpoint{5.874749in}{5.555456in}}%
\pgfpathlineto{\pgfqpoint{5.821705in}{5.555456in}}%
\pgfpathlineto{\pgfqpoint{5.768661in}{5.555456in}}%
\pgfpathlineto{\pgfqpoint{5.715617in}{4.786862in}}%
\pgfpathlineto{\pgfqpoint{5.662573in}{5.555456in}}%
\pgfpathlineto{\pgfqpoint{5.609529in}{5.555456in}}%
\pgfpathlineto{\pgfqpoint{5.556485in}{5.135041in}}%
\pgfpathlineto{\pgfqpoint{5.503441in}{4.781070in}}%
\pgfpathlineto{\pgfqpoint{5.450397in}{5.555456in}}%
\pgfpathlineto{\pgfqpoint{5.397353in}{5.555456in}}%
\pgfpathlineto{\pgfqpoint{5.344309in}{5.453871in}}%
\pgfpathlineto{\pgfqpoint{5.291266in}{4.639889in}}%
\pgfpathlineto{\pgfqpoint{5.238222in}{5.555456in}}%
\pgfpathlineto{\pgfqpoint{5.185178in}{5.555456in}}%
\pgfpathlineto{\pgfqpoint{5.132134in}{5.444876in}}%
\pgfpathlineto{\pgfqpoint{5.079090in}{4.491903in}}%
\pgfpathlineto{\pgfqpoint{5.026046in}{5.387465in}}%
\pgfpathlineto{\pgfqpoint{4.973002in}{4.520363in}}%
\pgfpathlineto{\pgfqpoint{4.919958in}{5.555456in}}%
\pgfpathlineto{\pgfqpoint{4.866914in}{5.555456in}}%
\pgfpathlineto{\pgfqpoint{4.813870in}{4.639365in}}%
\pgfpathlineto{\pgfqpoint{4.760826in}{4.951917in}}%
\pgfpathlineto{\pgfqpoint{4.707782in}{5.555456in}}%
\pgfpathlineto{\pgfqpoint{4.654738in}{4.759102in}}%
\pgfpathlineto{\pgfqpoint{4.601694in}{5.555456in}}%
\pgfpathlineto{\pgfqpoint{4.548650in}{5.101868in}}%
\pgfpathlineto{\pgfqpoint{4.495607in}{5.555456in}}%
\pgfpathlineto{\pgfqpoint{4.442563in}{4.896628in}}%
\pgfpathlineto{\pgfqpoint{4.389519in}{4.859785in}}%
\pgfpathlineto{\pgfqpoint{4.336475in}{5.555456in}}%
\pgfpathlineto{\pgfqpoint{4.283431in}{4.757491in}}%
\pgfpathlineto{\pgfqpoint{4.230387in}{4.741772in}}%
\pgfpathlineto{\pgfqpoint{4.177343in}{4.703664in}}%
\pgfpathlineto{\pgfqpoint{4.124299in}{4.787943in}}%
\pgfpathlineto{\pgfqpoint{4.071255in}{4.659775in}}%
\pgfpathlineto{\pgfqpoint{4.018211in}{4.577684in}}%
\pgfpathlineto{\pgfqpoint{3.965167in}{5.299182in}}%
\pgfpathlineto{\pgfqpoint{3.912123in}{4.859785in}}%
\pgfpathlineto{\pgfqpoint{3.859079in}{4.859785in}}%
\pgfpathlineto{\pgfqpoint{3.806035in}{4.571710in}}%
\pgfpathlineto{\pgfqpoint{3.752991in}{4.859785in}}%
\pgfpathlineto{\pgfqpoint{3.699948in}{4.531459in}}%
\pgfpathlineto{\pgfqpoint{3.646904in}{4.575309in}}%
\pgfpathlineto{\pgfqpoint{3.593860in}{5.052711in}}%
\pgfpathlineto{\pgfqpoint{3.540816in}{5.555456in}}%
\pgfpathlineto{\pgfqpoint{3.487772in}{5.331044in}}%
\pgfpathlineto{\pgfqpoint{3.434728in}{4.637008in}}%
\pgfpathlineto{\pgfqpoint{3.381684in}{4.982873in}}%
\pgfpathlineto{\pgfqpoint{3.328640in}{4.784623in}}%
\pgfpathlineto{\pgfqpoint{3.275596in}{4.726048in}}%
\pgfpathlineto{\pgfqpoint{3.222552in}{4.809907in}}%
\pgfpathlineto{\pgfqpoint{3.169508in}{4.779384in}}%
\pgfpathlineto{\pgfqpoint{3.116464in}{4.873985in}}%
\pgfpathlineto{\pgfqpoint{3.063420in}{5.529745in}}%
\pgfpathlineto{\pgfqpoint{3.010376in}{4.747565in}}%
\pgfpathlineto{\pgfqpoint{2.957332in}{5.555456in}}%
\pgfpathlineto{\pgfqpoint{2.904288in}{4.661801in}}%
\pgfpathlineto{\pgfqpoint{2.851245in}{5.097809in}}%
\pgfpathlineto{\pgfqpoint{2.798201in}{4.614894in}}%
\pgfpathlineto{\pgfqpoint{2.745157in}{4.572498in}}%
\pgfpathlineto{\pgfqpoint{2.692113in}{4.501574in}}%
\pgfpathlineto{\pgfqpoint{2.639069in}{4.531777in}}%
\pgfpathlineto{\pgfqpoint{2.586025in}{4.536270in}}%
\pgfpathlineto{\pgfqpoint{2.532981in}{4.558081in}}%
\pgfpathlineto{\pgfqpoint{2.479937in}{4.702326in}}%
\pgfpathlineto{\pgfqpoint{2.426893in}{5.555456in}}%
\pgfpathlineto{\pgfqpoint{2.373849in}{5.523078in}}%
\pgfpathlineto{\pgfqpoint{2.320805in}{4.582406in}}%
\pgfpathlineto{\pgfqpoint{2.267761in}{4.846792in}}%
\pgfpathlineto{\pgfqpoint{2.214717in}{5.555456in}}%
\pgfpathlineto{\pgfqpoint{2.161673in}{4.651814in}}%
\pgfpathlineto{\pgfqpoint{2.108629in}{5.048281in}}%
\pgfpathlineto{\pgfqpoint{2.055586in}{4.762166in}}%
\pgfpathlineto{\pgfqpoint{2.002542in}{5.555456in}}%
\pgfpathlineto{\pgfqpoint{1.949498in}{4.751360in}}%
\pgfpathlineto{\pgfqpoint{1.896454in}{5.555456in}}%
\pgfpathlineto{\pgfqpoint{1.843410in}{5.555456in}}%
\pgfpathlineto{\pgfqpoint{1.790366in}{4.778049in}}%
\pgfpathlineto{\pgfqpoint{1.737322in}{4.795628in}}%
\pgfpathlineto{\pgfqpoint{1.684278in}{4.875476in}}%
\pgfpathlineto{\pgfqpoint{1.631234in}{5.423055in}}%
\pgfpathlineto{\pgfqpoint{1.578190in}{4.630644in}}%
\pgfpathlineto{\pgfqpoint{1.525146in}{4.653514in}}%
\pgfpathlineto{\pgfqpoint{1.472102in}{4.573151in}}%
\pgfpathlineto{\pgfqpoint{1.419058in}{4.567788in}}%
\pgfpathlineto{\pgfqpoint{1.366014in}{4.493572in}}%
\pgfpathlineto{\pgfqpoint{1.312970in}{5.172617in}}%
\pgfpathlineto{\pgfqpoint{1.259927in}{5.369914in}}%
\pgfpathlineto{\pgfqpoint{1.206883in}{5.127392in}}%
\pgfpathlineto{\pgfqpoint{1.153839in}{4.557915in}}%
\pgfpathlineto{\pgfqpoint{1.100795in}{4.571579in}}%
\pgfpathlineto{\pgfqpoint{1.047751in}{4.602360in}}%
\pgfpathlineto{\pgfqpoint{0.994707in}{4.859785in}}%
\pgfpathlineto{\pgfqpoint{0.941663in}{4.606409in}}%
\pgfpathlineto{\pgfqpoint{0.941663in}{4.606409in}}%
\pgfpathclose%
\pgfusepath{stroke,fill}%
}%
\begin{pgfscope}%
\pgfsys@transformshift{0.000000in}{0.000000in}%
\pgfsys@useobject{currentmarker}{}%
\end{pgfscope}%
\end{pgfscope}%
\begin{pgfscope}%
\pgfpathrectangle{\pgfqpoint{0.941663in}{4.334375in}}{\pgfqpoint{8.858337in}{3.465625in}}%
\pgfusepath{clip}%
\pgfsetrectcap%
\pgfsetroundjoin%
\pgfsetlinewidth{1.505625pt}%
\definecolor{currentstroke}{rgb}{0.090196,0.745098,0.811765}%
\pgfsetstrokecolor{currentstroke}%
\pgfsetdash{}{0pt}%
\pgfpathmoveto{\pgfqpoint{0.941663in}{7.642472in}}%
\pgfpathlineto{\pgfqpoint{0.994707in}{7.398889in}}%
\pgfpathlineto{\pgfqpoint{1.047751in}{7.642472in}}%
\pgfpathlineto{\pgfqpoint{1.100795in}{7.630782in}}%
\pgfpathlineto{\pgfqpoint{1.153839in}{7.642472in}}%
\pgfpathlineto{\pgfqpoint{1.206883in}{7.113972in}}%
\pgfpathlineto{\pgfqpoint{1.259927in}{6.752271in}}%
\pgfpathlineto{\pgfqpoint{1.312970in}{7.018171in}}%
\pgfpathlineto{\pgfqpoint{1.366014in}{7.642472in}}%
\pgfpathlineto{\pgfqpoint{1.578190in}{7.642472in}}%
\pgfpathlineto{\pgfqpoint{1.631234in}{6.920473in}}%
\pgfpathlineto{\pgfqpoint{1.684278in}{7.452967in}}%
\pgfpathlineto{\pgfqpoint{1.737322in}{7.536394in}}%
\pgfpathlineto{\pgfqpoint{1.790366in}{7.642472in}}%
\pgfpathlineto{\pgfqpoint{1.843410in}{6.911654in}}%
\pgfpathlineto{\pgfqpoint{1.896454in}{6.883587in}}%
\pgfpathlineto{\pgfqpoint{1.949498in}{7.642472in}}%
\pgfpathlineto{\pgfqpoint{2.002542in}{6.857739in}}%
\pgfpathlineto{\pgfqpoint{2.055586in}{7.642472in}}%
\pgfpathlineto{\pgfqpoint{2.108629in}{7.339659in}}%
\pgfpathlineto{\pgfqpoint{2.161673in}{7.642472in}}%
\pgfpathlineto{\pgfqpoint{2.214717in}{6.753348in}}%
\pgfpathlineto{\pgfqpoint{2.267761in}{7.419218in}}%
\pgfpathlineto{\pgfqpoint{2.320805in}{7.642472in}}%
\pgfpathlineto{\pgfqpoint{2.373849in}{6.676762in}}%
\pgfpathlineto{\pgfqpoint{2.426893in}{6.606651in}}%
\pgfpathlineto{\pgfqpoint{2.479937in}{7.449769in}}%
\pgfpathlineto{\pgfqpoint{2.532981in}{7.642472in}}%
\pgfpathlineto{\pgfqpoint{2.798201in}{7.642472in}}%
\pgfpathlineto{\pgfqpoint{2.851245in}{7.173688in}}%
\pgfpathlineto{\pgfqpoint{2.904288in}{7.642472in}}%
\pgfpathlineto{\pgfqpoint{2.957332in}{6.836531in}}%
\pgfpathlineto{\pgfqpoint{3.010376in}{7.642472in}}%
\pgfpathlineto{\pgfqpoint{3.063420in}{6.841623in}}%
\pgfpathlineto{\pgfqpoint{3.116464in}{7.518262in}}%
\pgfpathlineto{\pgfqpoint{3.169508in}{7.642472in}}%
\pgfpathlineto{\pgfqpoint{3.328640in}{7.642472in}}%
\pgfpathlineto{\pgfqpoint{3.381684in}{7.361194in}}%
\pgfpathlineto{\pgfqpoint{3.434728in}{7.642472in}}%
\pgfpathlineto{\pgfqpoint{3.487772in}{6.972573in}}%
\pgfpathlineto{\pgfqpoint{3.540816in}{6.738883in}}%
\pgfpathlineto{\pgfqpoint{3.593860in}{7.167623in}}%
\pgfpathlineto{\pgfqpoint{3.646904in}{7.642472in}}%
\pgfpathlineto{\pgfqpoint{3.699948in}{7.642472in}}%
\pgfpathlineto{\pgfqpoint{3.752991in}{7.311866in}}%
\pgfpathlineto{\pgfqpoint{3.806035in}{7.642472in}}%
\pgfpathlineto{\pgfqpoint{3.859079in}{7.341795in}}%
\pgfpathlineto{\pgfqpoint{3.912123in}{7.302849in}}%
\pgfpathlineto{\pgfqpoint{3.965167in}{6.991885in}}%
\pgfpathlineto{\pgfqpoint{4.018211in}{7.642472in}}%
\pgfpathlineto{\pgfqpoint{4.071255in}{7.642472in}}%
\pgfpathlineto{\pgfqpoint{4.124299in}{7.539721in}}%
\pgfpathlineto{\pgfqpoint{4.177343in}{7.642472in}}%
\pgfpathlineto{\pgfqpoint{4.283431in}{7.642472in}}%
\pgfpathlineto{\pgfqpoint{4.336475in}{6.889082in}}%
\pgfpathlineto{\pgfqpoint{4.389519in}{7.554209in}}%
\pgfpathlineto{\pgfqpoint{4.442563in}{7.506419in}}%
\pgfpathlineto{\pgfqpoint{4.495607in}{6.883010in}}%
\pgfpathlineto{\pgfqpoint{4.548650in}{7.353421in}}%
\pgfpathlineto{\pgfqpoint{4.601694in}{6.827112in}}%
\pgfpathlineto{\pgfqpoint{4.654738in}{7.642472in}}%
\pgfpathlineto{\pgfqpoint{4.707782in}{6.766257in}}%
\pgfpathlineto{\pgfqpoint{4.760826in}{7.378709in}}%
\pgfpathlineto{\pgfqpoint{4.813870in}{7.642472in}}%
\pgfpathlineto{\pgfqpoint{4.866914in}{6.684929in}}%
\pgfpathlineto{\pgfqpoint{4.919958in}{6.671298in}}%
\pgfpathlineto{\pgfqpoint{4.973002in}{7.642472in}}%
\pgfpathlineto{\pgfqpoint{5.026046in}{6.790800in}}%
\pgfpathlineto{\pgfqpoint{5.079090in}{7.642472in}}%
\pgfpathlineto{\pgfqpoint{5.132134in}{6.692849in}}%
\pgfpathlineto{\pgfqpoint{5.185178in}{6.670254in}}%
\pgfpathlineto{\pgfqpoint{5.238222in}{6.666672in}}%
\pgfpathlineto{\pgfqpoint{5.291266in}{7.642472in}}%
\pgfpathlineto{\pgfqpoint{5.344309in}{6.838988in}}%
\pgfpathlineto{\pgfqpoint{5.397353in}{6.748316in}}%
\pgfpathlineto{\pgfqpoint{5.450397in}{6.877918in}}%
\pgfpathlineto{\pgfqpoint{5.503441in}{7.642472in}}%
\pgfpathlineto{\pgfqpoint{5.556485in}{7.281307in}}%
\pgfpathlineto{\pgfqpoint{5.609529in}{6.874196in}}%
\pgfpathlineto{\pgfqpoint{5.662573in}{6.925790in}}%
\pgfpathlineto{\pgfqpoint{5.715617in}{7.642472in}}%
\pgfpathlineto{\pgfqpoint{5.768661in}{6.913209in}}%
\pgfpathlineto{\pgfqpoint{5.821705in}{6.872997in}}%
\pgfpathlineto{\pgfqpoint{5.874749in}{6.796671in}}%
\pgfpathlineto{\pgfqpoint{5.927793in}{6.820928in}}%
\pgfpathlineto{\pgfqpoint{5.980837in}{7.642472in}}%
\pgfpathlineto{\pgfqpoint{6.033881in}{7.110408in}}%
\pgfpathlineto{\pgfqpoint{6.086925in}{7.642472in}}%
\pgfpathlineto{\pgfqpoint{6.139969in}{7.642472in}}%
\pgfpathlineto{\pgfqpoint{6.193012in}{6.635398in}}%
\pgfpathlineto{\pgfqpoint{6.246056in}{6.691111in}}%
\pgfpathlineto{\pgfqpoint{6.299100in}{7.642472in}}%
\pgfpathlineto{\pgfqpoint{6.352144in}{7.023749in}}%
\pgfpathlineto{\pgfqpoint{6.405188in}{6.593987in}}%
\pgfpathlineto{\pgfqpoint{6.458232in}{7.158194in}}%
\pgfpathlineto{\pgfqpoint{6.511276in}{7.642472in}}%
\pgfpathlineto{\pgfqpoint{6.723452in}{7.642472in}}%
\pgfpathlineto{\pgfqpoint{6.776496in}{7.554798in}}%
\pgfpathlineto{\pgfqpoint{6.829540in}{7.642472in}}%
\pgfpathlineto{\pgfqpoint{6.882584in}{6.881359in}}%
\pgfpathlineto{\pgfqpoint{6.935628in}{7.642472in}}%
\pgfpathlineto{\pgfqpoint{6.988671in}{7.441180in}}%
\pgfpathlineto{\pgfqpoint{7.041715in}{7.642472in}}%
\pgfpathlineto{\pgfqpoint{7.147803in}{7.642472in}}%
\pgfpathlineto{\pgfqpoint{7.200847in}{6.818171in}}%
\pgfpathlineto{\pgfqpoint{7.253891in}{7.350789in}}%
\pgfpathlineto{\pgfqpoint{7.306935in}{7.542837in}}%
\pgfpathlineto{\pgfqpoint{7.359979in}{6.672189in}}%
\pgfpathlineto{\pgfqpoint{7.413023in}{6.684093in}}%
\pgfpathlineto{\pgfqpoint{7.466067in}{7.061991in}}%
\pgfpathlineto{\pgfqpoint{7.519111in}{7.086960in}}%
\pgfpathlineto{\pgfqpoint{7.572155in}{6.593688in}}%
\pgfpathlineto{\pgfqpoint{7.625199in}{6.652740in}}%
\pgfpathlineto{\pgfqpoint{7.678243in}{6.739129in}}%
\pgfpathlineto{\pgfqpoint{7.731287in}{7.642472in}}%
\pgfpathlineto{\pgfqpoint{7.996506in}{7.642472in}}%
\pgfpathlineto{\pgfqpoint{8.049550in}{6.853606in}}%
\pgfpathlineto{\pgfqpoint{8.102594in}{6.863050in}}%
\pgfpathlineto{\pgfqpoint{8.155638in}{7.642472in}}%
\pgfpathlineto{\pgfqpoint{8.208682in}{7.642472in}}%
\pgfpathlineto{\pgfqpoint{8.261726in}{6.924712in}}%
\pgfpathlineto{\pgfqpoint{8.314770in}{6.876087in}}%
\pgfpathlineto{\pgfqpoint{8.367814in}{7.642472in}}%
\pgfpathlineto{\pgfqpoint{8.420858in}{6.857476in}}%
\pgfpathlineto{\pgfqpoint{8.473902in}{7.642472in}}%
\pgfpathlineto{\pgfqpoint{8.526946in}{7.642472in}}%
\pgfpathlineto{\pgfqpoint{8.579990in}{6.685501in}}%
\pgfpathlineto{\pgfqpoint{8.633033in}{6.694774in}}%
\pgfpathlineto{\pgfqpoint{8.686077in}{6.723687in}}%
\pgfpathlineto{\pgfqpoint{8.739121in}{6.662088in}}%
\pgfpathlineto{\pgfqpoint{8.792165in}{6.800188in}}%
\pgfpathlineto{\pgfqpoint{8.845209in}{6.642906in}}%
\pgfpathlineto{\pgfqpoint{8.898253in}{6.685692in}}%
\pgfpathlineto{\pgfqpoint{8.951297in}{7.642472in}}%
\pgfpathlineto{\pgfqpoint{9.004341in}{7.212818in}}%
\pgfpathlineto{\pgfqpoint{9.057385in}{7.642472in}}%
\pgfpathlineto{\pgfqpoint{9.110429in}{7.642472in}}%
\pgfpathlineto{\pgfqpoint{9.163473in}{6.770647in}}%
\pgfpathlineto{\pgfqpoint{9.216517in}{6.804492in}}%
\pgfpathlineto{\pgfqpoint{9.269561in}{6.811485in}}%
\pgfpathlineto{\pgfqpoint{9.322605in}{6.883905in}}%
\pgfpathlineto{\pgfqpoint{9.375649in}{6.879220in}}%
\pgfpathlineto{\pgfqpoint{9.428692in}{7.642472in}}%
\pgfpathlineto{\pgfqpoint{9.481736in}{7.580172in}}%
\pgfpathlineto{\pgfqpoint{9.534780in}{6.946800in}}%
\pgfpathlineto{\pgfqpoint{9.587824in}{6.868462in}}%
\pgfpathlineto{\pgfqpoint{9.640868in}{7.137950in}}%
\pgfpathlineto{\pgfqpoint{9.693912in}{6.803680in}}%
\pgfpathlineto{\pgfqpoint{9.746956in}{6.779712in}}%
\pgfpathlineto{\pgfqpoint{9.800000in}{6.765861in}}%
\pgfpathlineto{\pgfqpoint{9.800000in}{6.765861in}}%
\pgfusepath{stroke}%
\end{pgfscope}%
\begin{pgfscope}%
\pgfpathrectangle{\pgfqpoint{0.941663in}{4.334375in}}{\pgfqpoint{8.858337in}{3.465625in}}%
\pgfusepath{clip}%
\pgfsetbuttcap%
\pgfsetroundjoin%
\definecolor{currentfill}{rgb}{0.090196,0.745098,0.811765}%
\pgfsetfillcolor{currentfill}%
\pgfsetlinewidth{1.003750pt}%
\definecolor{currentstroke}{rgb}{0.090196,0.745098,0.811765}%
\pgfsetstrokecolor{currentstroke}%
\pgfsetdash{}{0pt}%
\pgfsys@defobject{currentmarker}{\pgfqpoint{0.941663in}{6.251128in}}{\pgfqpoint{9.800000in}{7.642472in}}{%
\pgfpathmoveto{\pgfqpoint{0.941663in}{7.642472in}}%
\pgfpathlineto{\pgfqpoint{0.941663in}{6.251128in}}%
\pgfpathlineto{\pgfqpoint{0.994707in}{6.261680in}}%
\pgfpathlineto{\pgfqpoint{1.047751in}{6.251128in}}%
\pgfpathlineto{\pgfqpoint{1.100795in}{6.251128in}}%
\pgfpathlineto{\pgfqpoint{1.153839in}{6.251128in}}%
\pgfpathlineto{\pgfqpoint{1.206883in}{6.598964in}}%
\pgfpathlineto{\pgfqpoint{1.259927in}{6.598964in}}%
\pgfpathlineto{\pgfqpoint{1.312970in}{6.598964in}}%
\pgfpathlineto{\pgfqpoint{1.366014in}{6.251128in}}%
\pgfpathlineto{\pgfqpoint{1.419058in}{6.251128in}}%
\pgfpathlineto{\pgfqpoint{1.472102in}{6.251128in}}%
\pgfpathlineto{\pgfqpoint{1.525146in}{6.251128in}}%
\pgfpathlineto{\pgfqpoint{1.578190in}{6.251128in}}%
\pgfpathlineto{\pgfqpoint{1.631234in}{6.598964in}}%
\pgfpathlineto{\pgfqpoint{1.684278in}{6.598964in}}%
\pgfpathlineto{\pgfqpoint{1.737322in}{6.251128in}}%
\pgfpathlineto{\pgfqpoint{1.790366in}{6.251128in}}%
\pgfpathlineto{\pgfqpoint{1.843410in}{6.734592in}}%
\pgfpathlineto{\pgfqpoint{1.896454in}{6.883587in}}%
\pgfpathlineto{\pgfqpoint{1.949498in}{6.251128in}}%
\pgfpathlineto{\pgfqpoint{2.002542in}{6.724358in}}%
\pgfpathlineto{\pgfqpoint{2.055586in}{6.251128in}}%
\pgfpathlineto{\pgfqpoint{2.108629in}{6.348747in}}%
\pgfpathlineto{\pgfqpoint{2.161673in}{6.251128in}}%
\pgfpathlineto{\pgfqpoint{2.214717in}{6.574971in}}%
\pgfpathlineto{\pgfqpoint{2.267761in}{6.251128in}}%
\pgfpathlineto{\pgfqpoint{2.320805in}{6.251128in}}%
\pgfpathlineto{\pgfqpoint{2.373849in}{6.541500in}}%
\pgfpathlineto{\pgfqpoint{2.426893in}{6.606651in}}%
\pgfpathlineto{\pgfqpoint{2.479937in}{6.251128in}}%
\pgfpathlineto{\pgfqpoint{2.532981in}{6.251128in}}%
\pgfpathlineto{\pgfqpoint{2.586025in}{6.251128in}}%
\pgfpathlineto{\pgfqpoint{2.639069in}{6.251128in}}%
\pgfpathlineto{\pgfqpoint{2.692113in}{6.251128in}}%
\pgfpathlineto{\pgfqpoint{2.745157in}{6.251128in}}%
\pgfpathlineto{\pgfqpoint{2.798201in}{6.251128in}}%
\pgfpathlineto{\pgfqpoint{2.851245in}{6.598964in}}%
\pgfpathlineto{\pgfqpoint{2.904288in}{6.251128in}}%
\pgfpathlineto{\pgfqpoint{2.957332in}{6.744748in}}%
\pgfpathlineto{\pgfqpoint{3.010376in}{6.251128in}}%
\pgfpathlineto{\pgfqpoint{3.063420in}{6.598964in}}%
\pgfpathlineto{\pgfqpoint{3.116464in}{6.598964in}}%
\pgfpathlineto{\pgfqpoint{3.169508in}{6.251128in}}%
\pgfpathlineto{\pgfqpoint{3.222552in}{6.251128in}}%
\pgfpathlineto{\pgfqpoint{3.275596in}{6.251128in}}%
\pgfpathlineto{\pgfqpoint{3.328640in}{6.251128in}}%
\pgfpathlineto{\pgfqpoint{3.381684in}{6.598964in}}%
\pgfpathlineto{\pgfqpoint{3.434728in}{6.251128in}}%
\pgfpathlineto{\pgfqpoint{3.487772in}{6.598964in}}%
\pgfpathlineto{\pgfqpoint{3.540816in}{6.572638in}}%
\pgfpathlineto{\pgfqpoint{3.593860in}{6.251128in}}%
\pgfpathlineto{\pgfqpoint{3.646904in}{6.251128in}}%
\pgfpathlineto{\pgfqpoint{3.699948in}{6.251128in}}%
\pgfpathlineto{\pgfqpoint{3.752991in}{6.510498in}}%
\pgfpathlineto{\pgfqpoint{3.806035in}{6.251128in}}%
\pgfpathlineto{\pgfqpoint{3.859079in}{6.572998in}}%
\pgfpathlineto{\pgfqpoint{3.912123in}{6.481634in}}%
\pgfpathlineto{\pgfqpoint{3.965167in}{6.340259in}}%
\pgfpathlineto{\pgfqpoint{4.018211in}{6.251128in}}%
\pgfpathlineto{\pgfqpoint{4.071255in}{6.251128in}}%
\pgfpathlineto{\pgfqpoint{4.124299in}{6.251128in}}%
\pgfpathlineto{\pgfqpoint{4.177343in}{6.251128in}}%
\pgfpathlineto{\pgfqpoint{4.230387in}{6.251128in}}%
\pgfpathlineto{\pgfqpoint{4.283431in}{6.251128in}}%
\pgfpathlineto{\pgfqpoint{4.336475in}{6.889082in}}%
\pgfpathlineto{\pgfqpoint{4.389519in}{6.299258in}}%
\pgfpathlineto{\pgfqpoint{4.442563in}{6.598964in}}%
\pgfpathlineto{\pgfqpoint{4.495607in}{6.633012in}}%
\pgfpathlineto{\pgfqpoint{4.548650in}{6.251128in}}%
\pgfpathlineto{\pgfqpoint{4.601694in}{6.592100in}}%
\pgfpathlineto{\pgfqpoint{4.654738in}{6.251128in}}%
\pgfpathlineto{\pgfqpoint{4.707782in}{6.766257in}}%
\pgfpathlineto{\pgfqpoint{4.760826in}{6.251128in}}%
\pgfpathlineto{\pgfqpoint{4.813870in}{6.251128in}}%
\pgfpathlineto{\pgfqpoint{4.866914in}{6.548881in}}%
\pgfpathlineto{\pgfqpoint{4.919958in}{6.528186in}}%
\pgfpathlineto{\pgfqpoint{4.973002in}{6.251128in}}%
\pgfpathlineto{\pgfqpoint{5.026046in}{6.590550in}}%
\pgfpathlineto{\pgfqpoint{5.079090in}{6.251128in}}%
\pgfpathlineto{\pgfqpoint{5.132134in}{6.598964in}}%
\pgfpathlineto{\pgfqpoint{5.185178in}{6.574350in}}%
\pgfpathlineto{\pgfqpoint{5.238222in}{6.501650in}}%
\pgfpathlineto{\pgfqpoint{5.291266in}{6.251128in}}%
\pgfpathlineto{\pgfqpoint{5.344309in}{6.471024in}}%
\pgfpathlineto{\pgfqpoint{5.397353in}{6.325005in}}%
\pgfpathlineto{\pgfqpoint{5.450397in}{6.770026in}}%
\pgfpathlineto{\pgfqpoint{5.503441in}{6.251128in}}%
\pgfpathlineto{\pgfqpoint{5.556485in}{6.329843in}}%
\pgfpathlineto{\pgfqpoint{5.609529in}{6.514446in}}%
\pgfpathlineto{\pgfqpoint{5.662573in}{6.375963in}}%
\pgfpathlineto{\pgfqpoint{5.715617in}{6.251128in}}%
\pgfpathlineto{\pgfqpoint{5.768661in}{6.913209in}}%
\pgfpathlineto{\pgfqpoint{5.821705in}{6.872997in}}%
\pgfpathlineto{\pgfqpoint{5.874749in}{6.702075in}}%
\pgfpathlineto{\pgfqpoint{5.927793in}{6.598351in}}%
\pgfpathlineto{\pgfqpoint{5.980837in}{6.251128in}}%
\pgfpathlineto{\pgfqpoint{6.033881in}{6.400407in}}%
\pgfpathlineto{\pgfqpoint{6.086925in}{6.251128in}}%
\pgfpathlineto{\pgfqpoint{6.139969in}{6.251128in}}%
\pgfpathlineto{\pgfqpoint{6.193012in}{6.635398in}}%
\pgfpathlineto{\pgfqpoint{6.246056in}{6.691111in}}%
\pgfpathlineto{\pgfqpoint{6.299100in}{6.251128in}}%
\pgfpathlineto{\pgfqpoint{6.352144in}{6.569516in}}%
\pgfpathlineto{\pgfqpoint{6.405188in}{6.364054in}}%
\pgfpathlineto{\pgfqpoint{6.458232in}{6.251128in}}%
\pgfpathlineto{\pgfqpoint{6.511276in}{6.251128in}}%
\pgfpathlineto{\pgfqpoint{6.564320in}{6.251128in}}%
\pgfpathlineto{\pgfqpoint{6.617364in}{6.251128in}}%
\pgfpathlineto{\pgfqpoint{6.670408in}{6.251128in}}%
\pgfpathlineto{\pgfqpoint{6.723452in}{6.251128in}}%
\pgfpathlineto{\pgfqpoint{6.776496in}{6.385890in}}%
\pgfpathlineto{\pgfqpoint{6.829540in}{6.251128in}}%
\pgfpathlineto{\pgfqpoint{6.882584in}{6.780238in}}%
\pgfpathlineto{\pgfqpoint{6.935628in}{6.251128in}}%
\pgfpathlineto{\pgfqpoint{6.988671in}{6.598964in}}%
\pgfpathlineto{\pgfqpoint{7.041715in}{6.251128in}}%
\pgfpathlineto{\pgfqpoint{7.094759in}{6.251128in}}%
\pgfpathlineto{\pgfqpoint{7.147803in}{6.251128in}}%
\pgfpathlineto{\pgfqpoint{7.200847in}{6.729225in}}%
\pgfpathlineto{\pgfqpoint{7.253891in}{6.451777in}}%
\pgfpathlineto{\pgfqpoint{7.306935in}{6.251128in}}%
\pgfpathlineto{\pgfqpoint{7.359979in}{6.470643in}}%
\pgfpathlineto{\pgfqpoint{7.413023in}{6.529690in}}%
\pgfpathlineto{\pgfqpoint{7.466067in}{6.251128in}}%
\pgfpathlineto{\pgfqpoint{7.519111in}{6.251128in}}%
\pgfpathlineto{\pgfqpoint{7.572155in}{6.487962in}}%
\pgfpathlineto{\pgfqpoint{7.625199in}{6.652740in}}%
\pgfpathlineto{\pgfqpoint{7.678243in}{6.251128in}}%
\pgfpathlineto{\pgfqpoint{7.731287in}{6.251128in}}%
\pgfpathlineto{\pgfqpoint{7.784330in}{6.251128in}}%
\pgfpathlineto{\pgfqpoint{7.837374in}{6.251128in}}%
\pgfpathlineto{\pgfqpoint{7.890418in}{6.251128in}}%
\pgfpathlineto{\pgfqpoint{7.943462in}{6.251128in}}%
\pgfpathlineto{\pgfqpoint{7.996506in}{6.251128in}}%
\pgfpathlineto{\pgfqpoint{8.049550in}{6.766484in}}%
\pgfpathlineto{\pgfqpoint{8.102594in}{6.782745in}}%
\pgfpathlineto{\pgfqpoint{8.155638in}{6.251128in}}%
\pgfpathlineto{\pgfqpoint{8.208682in}{6.251128in}}%
\pgfpathlineto{\pgfqpoint{8.261726in}{6.924712in}}%
\pgfpathlineto{\pgfqpoint{8.314770in}{6.876087in}}%
\pgfpathlineto{\pgfqpoint{8.367814in}{6.251128in}}%
\pgfpathlineto{\pgfqpoint{8.420858in}{6.653636in}}%
\pgfpathlineto{\pgfqpoint{8.473902in}{6.251128in}}%
\pgfpathlineto{\pgfqpoint{8.526946in}{6.251128in}}%
\pgfpathlineto{\pgfqpoint{8.579990in}{6.685501in}}%
\pgfpathlineto{\pgfqpoint{8.633033in}{6.520595in}}%
\pgfpathlineto{\pgfqpoint{8.686077in}{6.723687in}}%
\pgfpathlineto{\pgfqpoint{8.739121in}{6.489640in}}%
\pgfpathlineto{\pgfqpoint{8.792165in}{6.251128in}}%
\pgfpathlineto{\pgfqpoint{8.845209in}{6.299108in}}%
\pgfpathlineto{\pgfqpoint{8.898253in}{6.354195in}}%
\pgfpathlineto{\pgfqpoint{8.951297in}{6.251128in}}%
\pgfpathlineto{\pgfqpoint{9.004341in}{6.568668in}}%
\pgfpathlineto{\pgfqpoint{9.057385in}{6.251128in}}%
\pgfpathlineto{\pgfqpoint{9.110429in}{6.251128in}}%
\pgfpathlineto{\pgfqpoint{9.163473in}{6.668425in}}%
\pgfpathlineto{\pgfqpoint{9.216517in}{6.804492in}}%
\pgfpathlineto{\pgfqpoint{9.269561in}{6.628250in}}%
\pgfpathlineto{\pgfqpoint{9.322605in}{6.727450in}}%
\pgfpathlineto{\pgfqpoint{9.375649in}{6.754769in}}%
\pgfpathlineto{\pgfqpoint{9.428692in}{6.251128in}}%
\pgfpathlineto{\pgfqpoint{9.481736in}{6.256205in}}%
\pgfpathlineto{\pgfqpoint{9.534780in}{6.946800in}}%
\pgfpathlineto{\pgfqpoint{9.587824in}{6.621011in}}%
\pgfpathlineto{\pgfqpoint{9.640868in}{6.251128in}}%
\pgfpathlineto{\pgfqpoint{9.693912in}{6.632876in}}%
\pgfpathlineto{\pgfqpoint{9.746956in}{6.779712in}}%
\pgfpathlineto{\pgfqpoint{9.800000in}{6.573653in}}%
\pgfpathlineto{\pgfqpoint{9.800000in}{6.765861in}}%
\pgfpathlineto{\pgfqpoint{9.800000in}{6.765861in}}%
\pgfpathlineto{\pgfqpoint{9.746956in}{6.779712in}}%
\pgfpathlineto{\pgfqpoint{9.693912in}{6.803680in}}%
\pgfpathlineto{\pgfqpoint{9.640868in}{7.137950in}}%
\pgfpathlineto{\pgfqpoint{9.587824in}{6.868462in}}%
\pgfpathlineto{\pgfqpoint{9.534780in}{6.946800in}}%
\pgfpathlineto{\pgfqpoint{9.481736in}{7.580172in}}%
\pgfpathlineto{\pgfqpoint{9.428692in}{7.642472in}}%
\pgfpathlineto{\pgfqpoint{9.375649in}{6.879220in}}%
\pgfpathlineto{\pgfqpoint{9.322605in}{6.883905in}}%
\pgfpathlineto{\pgfqpoint{9.269561in}{6.811485in}}%
\pgfpathlineto{\pgfqpoint{9.216517in}{6.804492in}}%
\pgfpathlineto{\pgfqpoint{9.163473in}{6.770647in}}%
\pgfpathlineto{\pgfqpoint{9.110429in}{7.642472in}}%
\pgfpathlineto{\pgfqpoint{9.057385in}{7.642472in}}%
\pgfpathlineto{\pgfqpoint{9.004341in}{7.212818in}}%
\pgfpathlineto{\pgfqpoint{8.951297in}{7.642472in}}%
\pgfpathlineto{\pgfqpoint{8.898253in}{6.685692in}}%
\pgfpathlineto{\pgfqpoint{8.845209in}{6.642906in}}%
\pgfpathlineto{\pgfqpoint{8.792165in}{6.800188in}}%
\pgfpathlineto{\pgfqpoint{8.739121in}{6.662088in}}%
\pgfpathlineto{\pgfqpoint{8.686077in}{6.723687in}}%
\pgfpathlineto{\pgfqpoint{8.633033in}{6.694774in}}%
\pgfpathlineto{\pgfqpoint{8.579990in}{6.685501in}}%
\pgfpathlineto{\pgfqpoint{8.526946in}{7.642472in}}%
\pgfpathlineto{\pgfqpoint{8.473902in}{7.642472in}}%
\pgfpathlineto{\pgfqpoint{8.420858in}{6.857476in}}%
\pgfpathlineto{\pgfqpoint{8.367814in}{7.642472in}}%
\pgfpathlineto{\pgfqpoint{8.314770in}{6.876087in}}%
\pgfpathlineto{\pgfqpoint{8.261726in}{6.924712in}}%
\pgfpathlineto{\pgfqpoint{8.208682in}{7.642472in}}%
\pgfpathlineto{\pgfqpoint{8.155638in}{7.642472in}}%
\pgfpathlineto{\pgfqpoint{8.102594in}{6.863050in}}%
\pgfpathlineto{\pgfqpoint{8.049550in}{6.853606in}}%
\pgfpathlineto{\pgfqpoint{7.996506in}{7.642472in}}%
\pgfpathlineto{\pgfqpoint{7.943462in}{7.642472in}}%
\pgfpathlineto{\pgfqpoint{7.890418in}{7.642472in}}%
\pgfpathlineto{\pgfqpoint{7.837374in}{7.642472in}}%
\pgfpathlineto{\pgfqpoint{7.784330in}{7.642472in}}%
\pgfpathlineto{\pgfqpoint{7.731287in}{7.642472in}}%
\pgfpathlineto{\pgfqpoint{7.678243in}{6.739129in}}%
\pgfpathlineto{\pgfqpoint{7.625199in}{6.652740in}}%
\pgfpathlineto{\pgfqpoint{7.572155in}{6.593688in}}%
\pgfpathlineto{\pgfqpoint{7.519111in}{7.086960in}}%
\pgfpathlineto{\pgfqpoint{7.466067in}{7.061991in}}%
\pgfpathlineto{\pgfqpoint{7.413023in}{6.684093in}}%
\pgfpathlineto{\pgfqpoint{7.359979in}{6.672189in}}%
\pgfpathlineto{\pgfqpoint{7.306935in}{7.542837in}}%
\pgfpathlineto{\pgfqpoint{7.253891in}{7.350789in}}%
\pgfpathlineto{\pgfqpoint{7.200847in}{6.818171in}}%
\pgfpathlineto{\pgfqpoint{7.147803in}{7.642472in}}%
\pgfpathlineto{\pgfqpoint{7.094759in}{7.642472in}}%
\pgfpathlineto{\pgfqpoint{7.041715in}{7.642472in}}%
\pgfpathlineto{\pgfqpoint{6.988671in}{7.441180in}}%
\pgfpathlineto{\pgfqpoint{6.935628in}{7.642472in}}%
\pgfpathlineto{\pgfqpoint{6.882584in}{6.881359in}}%
\pgfpathlineto{\pgfqpoint{6.829540in}{7.642472in}}%
\pgfpathlineto{\pgfqpoint{6.776496in}{7.554798in}}%
\pgfpathlineto{\pgfqpoint{6.723452in}{7.642472in}}%
\pgfpathlineto{\pgfqpoint{6.670408in}{7.642472in}}%
\pgfpathlineto{\pgfqpoint{6.617364in}{7.642472in}}%
\pgfpathlineto{\pgfqpoint{6.564320in}{7.642472in}}%
\pgfpathlineto{\pgfqpoint{6.511276in}{7.642472in}}%
\pgfpathlineto{\pgfqpoint{6.458232in}{7.158194in}}%
\pgfpathlineto{\pgfqpoint{6.405188in}{6.593987in}}%
\pgfpathlineto{\pgfqpoint{6.352144in}{7.023749in}}%
\pgfpathlineto{\pgfqpoint{6.299100in}{7.642472in}}%
\pgfpathlineto{\pgfqpoint{6.246056in}{6.691111in}}%
\pgfpathlineto{\pgfqpoint{6.193012in}{6.635398in}}%
\pgfpathlineto{\pgfqpoint{6.139969in}{7.642472in}}%
\pgfpathlineto{\pgfqpoint{6.086925in}{7.642472in}}%
\pgfpathlineto{\pgfqpoint{6.033881in}{7.110408in}}%
\pgfpathlineto{\pgfqpoint{5.980837in}{7.642472in}}%
\pgfpathlineto{\pgfqpoint{5.927793in}{6.820928in}}%
\pgfpathlineto{\pgfqpoint{5.874749in}{6.796671in}}%
\pgfpathlineto{\pgfqpoint{5.821705in}{6.872997in}}%
\pgfpathlineto{\pgfqpoint{5.768661in}{6.913209in}}%
\pgfpathlineto{\pgfqpoint{5.715617in}{7.642472in}}%
\pgfpathlineto{\pgfqpoint{5.662573in}{6.925790in}}%
\pgfpathlineto{\pgfqpoint{5.609529in}{6.874196in}}%
\pgfpathlineto{\pgfqpoint{5.556485in}{7.281307in}}%
\pgfpathlineto{\pgfqpoint{5.503441in}{7.642472in}}%
\pgfpathlineto{\pgfqpoint{5.450397in}{6.877918in}}%
\pgfpathlineto{\pgfqpoint{5.397353in}{6.748316in}}%
\pgfpathlineto{\pgfqpoint{5.344309in}{6.838988in}}%
\pgfpathlineto{\pgfqpoint{5.291266in}{7.642472in}}%
\pgfpathlineto{\pgfqpoint{5.238222in}{6.666672in}}%
\pgfpathlineto{\pgfqpoint{5.185178in}{6.670254in}}%
\pgfpathlineto{\pgfqpoint{5.132134in}{6.692849in}}%
\pgfpathlineto{\pgfqpoint{5.079090in}{7.642472in}}%
\pgfpathlineto{\pgfqpoint{5.026046in}{6.790800in}}%
\pgfpathlineto{\pgfqpoint{4.973002in}{7.642472in}}%
\pgfpathlineto{\pgfqpoint{4.919958in}{6.671298in}}%
\pgfpathlineto{\pgfqpoint{4.866914in}{6.684929in}}%
\pgfpathlineto{\pgfqpoint{4.813870in}{7.642472in}}%
\pgfpathlineto{\pgfqpoint{4.760826in}{7.378709in}}%
\pgfpathlineto{\pgfqpoint{4.707782in}{6.766257in}}%
\pgfpathlineto{\pgfqpoint{4.654738in}{7.642472in}}%
\pgfpathlineto{\pgfqpoint{4.601694in}{6.827112in}}%
\pgfpathlineto{\pgfqpoint{4.548650in}{7.353421in}}%
\pgfpathlineto{\pgfqpoint{4.495607in}{6.883010in}}%
\pgfpathlineto{\pgfqpoint{4.442563in}{7.506419in}}%
\pgfpathlineto{\pgfqpoint{4.389519in}{7.554209in}}%
\pgfpathlineto{\pgfqpoint{4.336475in}{6.889082in}}%
\pgfpathlineto{\pgfqpoint{4.283431in}{7.642472in}}%
\pgfpathlineto{\pgfqpoint{4.230387in}{7.642472in}}%
\pgfpathlineto{\pgfqpoint{4.177343in}{7.642472in}}%
\pgfpathlineto{\pgfqpoint{4.124299in}{7.539721in}}%
\pgfpathlineto{\pgfqpoint{4.071255in}{7.642472in}}%
\pgfpathlineto{\pgfqpoint{4.018211in}{7.642472in}}%
\pgfpathlineto{\pgfqpoint{3.965167in}{6.991885in}}%
\pgfpathlineto{\pgfqpoint{3.912123in}{7.302849in}}%
\pgfpathlineto{\pgfqpoint{3.859079in}{7.341795in}}%
\pgfpathlineto{\pgfqpoint{3.806035in}{7.642472in}}%
\pgfpathlineto{\pgfqpoint{3.752991in}{7.311866in}}%
\pgfpathlineto{\pgfqpoint{3.699948in}{7.642472in}}%
\pgfpathlineto{\pgfqpoint{3.646904in}{7.642472in}}%
\pgfpathlineto{\pgfqpoint{3.593860in}{7.167623in}}%
\pgfpathlineto{\pgfqpoint{3.540816in}{6.738883in}}%
\pgfpathlineto{\pgfqpoint{3.487772in}{6.972573in}}%
\pgfpathlineto{\pgfqpoint{3.434728in}{7.642472in}}%
\pgfpathlineto{\pgfqpoint{3.381684in}{7.361194in}}%
\pgfpathlineto{\pgfqpoint{3.328640in}{7.642472in}}%
\pgfpathlineto{\pgfqpoint{3.275596in}{7.642472in}}%
\pgfpathlineto{\pgfqpoint{3.222552in}{7.642472in}}%
\pgfpathlineto{\pgfqpoint{3.169508in}{7.642472in}}%
\pgfpathlineto{\pgfqpoint{3.116464in}{7.518262in}}%
\pgfpathlineto{\pgfqpoint{3.063420in}{6.841623in}}%
\pgfpathlineto{\pgfqpoint{3.010376in}{7.642472in}}%
\pgfpathlineto{\pgfqpoint{2.957332in}{6.836531in}}%
\pgfpathlineto{\pgfqpoint{2.904288in}{7.642472in}}%
\pgfpathlineto{\pgfqpoint{2.851245in}{7.173688in}}%
\pgfpathlineto{\pgfqpoint{2.798201in}{7.642472in}}%
\pgfpathlineto{\pgfqpoint{2.745157in}{7.642472in}}%
\pgfpathlineto{\pgfqpoint{2.692113in}{7.642472in}}%
\pgfpathlineto{\pgfqpoint{2.639069in}{7.642472in}}%
\pgfpathlineto{\pgfqpoint{2.586025in}{7.642472in}}%
\pgfpathlineto{\pgfqpoint{2.532981in}{7.642472in}}%
\pgfpathlineto{\pgfqpoint{2.479937in}{7.449769in}}%
\pgfpathlineto{\pgfqpoint{2.426893in}{6.606651in}}%
\pgfpathlineto{\pgfqpoint{2.373849in}{6.676762in}}%
\pgfpathlineto{\pgfqpoint{2.320805in}{7.642472in}}%
\pgfpathlineto{\pgfqpoint{2.267761in}{7.419218in}}%
\pgfpathlineto{\pgfqpoint{2.214717in}{6.753348in}}%
\pgfpathlineto{\pgfqpoint{2.161673in}{7.642472in}}%
\pgfpathlineto{\pgfqpoint{2.108629in}{7.339659in}}%
\pgfpathlineto{\pgfqpoint{2.055586in}{7.642472in}}%
\pgfpathlineto{\pgfqpoint{2.002542in}{6.857739in}}%
\pgfpathlineto{\pgfqpoint{1.949498in}{7.642472in}}%
\pgfpathlineto{\pgfqpoint{1.896454in}{6.883587in}}%
\pgfpathlineto{\pgfqpoint{1.843410in}{6.911654in}}%
\pgfpathlineto{\pgfqpoint{1.790366in}{7.642472in}}%
\pgfpathlineto{\pgfqpoint{1.737322in}{7.536394in}}%
\pgfpathlineto{\pgfqpoint{1.684278in}{7.452967in}}%
\pgfpathlineto{\pgfqpoint{1.631234in}{6.920473in}}%
\pgfpathlineto{\pgfqpoint{1.578190in}{7.642472in}}%
\pgfpathlineto{\pgfqpoint{1.525146in}{7.642472in}}%
\pgfpathlineto{\pgfqpoint{1.472102in}{7.642472in}}%
\pgfpathlineto{\pgfqpoint{1.419058in}{7.642472in}}%
\pgfpathlineto{\pgfqpoint{1.366014in}{7.642472in}}%
\pgfpathlineto{\pgfqpoint{1.312970in}{7.018171in}}%
\pgfpathlineto{\pgfqpoint{1.259927in}{6.752271in}}%
\pgfpathlineto{\pgfqpoint{1.206883in}{7.113972in}}%
\pgfpathlineto{\pgfqpoint{1.153839in}{7.642472in}}%
\pgfpathlineto{\pgfqpoint{1.100795in}{7.630782in}}%
\pgfpathlineto{\pgfqpoint{1.047751in}{7.642472in}}%
\pgfpathlineto{\pgfqpoint{0.994707in}{7.398889in}}%
\pgfpathlineto{\pgfqpoint{0.941663in}{7.642472in}}%
\pgfpathlineto{\pgfqpoint{0.941663in}{7.642472in}}%
\pgfpathclose%
\pgfusepath{stroke,fill}%
}%
\begin{pgfscope}%
\pgfsys@transformshift{0.000000in}{0.000000in}%
\pgfsys@useobject{currentmarker}{}%
\end{pgfscope}%
\end{pgfscope}%
\begin{pgfscope}%
\pgfsetrectcap%
\pgfsetmiterjoin%
\pgfsetlinewidth{0.803000pt}%
\definecolor{currentstroke}{rgb}{0.000000,0.000000,0.000000}%
\pgfsetstrokecolor{currentstroke}%
\pgfsetdash{}{0pt}%
\pgfpathmoveto{\pgfqpoint{0.941663in}{4.334375in}}%
\pgfpathlineto{\pgfqpoint{0.941663in}{7.800000in}}%
\pgfusepath{stroke}%
\end{pgfscope}%
\begin{pgfscope}%
\pgfsetrectcap%
\pgfsetmiterjoin%
\pgfsetlinewidth{0.803000pt}%
\definecolor{currentstroke}{rgb}{0.000000,0.000000,0.000000}%
\pgfsetstrokecolor{currentstroke}%
\pgfsetdash{}{0pt}%
\pgfpathmoveto{\pgfqpoint{9.800000in}{4.334375in}}%
\pgfpathlineto{\pgfqpoint{9.800000in}{7.800000in}}%
\pgfusepath{stroke}%
\end{pgfscope}%
\begin{pgfscope}%
\pgfsetrectcap%
\pgfsetmiterjoin%
\pgfsetlinewidth{0.803000pt}%
\definecolor{currentstroke}{rgb}{0.000000,0.000000,0.000000}%
\pgfsetstrokecolor{currentstroke}%
\pgfsetdash{}{0pt}%
\pgfpathmoveto{\pgfqpoint{0.941663in}{4.334375in}}%
\pgfpathlineto{\pgfqpoint{9.800000in}{4.334375in}}%
\pgfusepath{stroke}%
\end{pgfscope}%
\begin{pgfscope}%
\pgfsetrectcap%
\pgfsetmiterjoin%
\pgfsetlinewidth{0.803000pt}%
\definecolor{currentstroke}{rgb}{0.000000,0.000000,0.000000}%
\pgfsetstrokecolor{currentstroke}%
\pgfsetdash{}{0pt}%
\pgfpathmoveto{\pgfqpoint{0.941663in}{7.800000in}}%
\pgfpathlineto{\pgfqpoint{9.800000in}{7.800000in}}%
\pgfusepath{stroke}%
\end{pgfscope}%
\begin{pgfscope}%
\pgfpathrectangle{\pgfqpoint{0.941663in}{4.334375in}}{\pgfqpoint{8.858337in}{3.465625in}}%
\pgfusepath{clip}%
\pgfsetbuttcap%
\pgfsetroundjoin%
\pgfsetlinewidth{1.505625pt}%
\definecolor{currentstroke}{rgb}{0.000000,0.000000,0.000000}%
\pgfsetstrokecolor{currentstroke}%
\pgfsetdash{{5.550000pt}{2.400000pt}}{0.000000pt}%
\pgfpathmoveto{\pgfqpoint{0.941663in}{6.693424in}}%
\pgfpathlineto{\pgfqpoint{0.994707in}{6.703217in}}%
\pgfpathlineto{\pgfqpoint{1.047751in}{6.689375in}}%
\pgfpathlineto{\pgfqpoint{1.100795in}{6.646904in}}%
\pgfpathlineto{\pgfqpoint{1.153839in}{6.644931in}}%
\pgfpathlineto{\pgfqpoint{1.206883in}{6.685908in}}%
\pgfpathlineto{\pgfqpoint{1.259927in}{6.566729in}}%
\pgfpathlineto{\pgfqpoint{1.312970in}{6.635332in}}%
\pgfpathlineto{\pgfqpoint{1.366014in}{6.580587in}}%
\pgfpathlineto{\pgfqpoint{1.419058in}{6.654803in}}%
\pgfpathlineto{\pgfqpoint{1.472102in}{6.660166in}}%
\pgfpathlineto{\pgfqpoint{1.525146in}{6.740529in}}%
\pgfpathlineto{\pgfqpoint{1.578190in}{6.717659in}}%
\pgfpathlineto{\pgfqpoint{1.631234in}{6.788071in}}%
\pgfpathlineto{\pgfqpoint{1.684278in}{6.772987in}}%
\pgfpathlineto{\pgfqpoint{1.737322in}{6.776566in}}%
\pgfpathlineto{\pgfqpoint{1.790366in}{6.865064in}}%
\pgfpathlineto{\pgfqpoint{1.843410in}{6.911654in}}%
\pgfpathlineto{\pgfqpoint{1.896454in}{6.883587in}}%
\pgfpathlineto{\pgfqpoint{1.949498in}{6.838375in}}%
\pgfpathlineto{\pgfqpoint{2.002542in}{6.857739in}}%
\pgfpathlineto{\pgfqpoint{2.055586in}{6.849181in}}%
\pgfpathlineto{\pgfqpoint{2.108629in}{6.832484in}}%
\pgfpathlineto{\pgfqpoint{2.161673in}{6.738829in}}%
\pgfpathlineto{\pgfqpoint{2.214717in}{6.753348in}}%
\pgfpathlineto{\pgfqpoint{2.267761in}{6.710554in}}%
\pgfpathlineto{\pgfqpoint{2.320805in}{6.669421in}}%
\pgfpathlineto{\pgfqpoint{2.373849in}{6.644384in}}%
\pgfpathlineto{\pgfqpoint{2.426893in}{6.606651in}}%
\pgfpathlineto{\pgfqpoint{2.479937in}{6.596638in}}%
\pgfpathlineto{\pgfqpoint{2.532981in}{6.645096in}}%
\pgfpathlineto{\pgfqpoint{2.586025in}{6.623285in}}%
\pgfpathlineto{\pgfqpoint{2.639069in}{6.618792in}}%
\pgfpathlineto{\pgfqpoint{2.692113in}{6.588590in}}%
\pgfpathlineto{\pgfqpoint{2.745157in}{6.659513in}}%
\pgfpathlineto{\pgfqpoint{2.798201in}{6.701909in}}%
\pgfpathlineto{\pgfqpoint{2.851245in}{6.716040in}}%
\pgfpathlineto{\pgfqpoint{2.904288in}{6.748816in}}%
\pgfpathlineto{\pgfqpoint{2.957332in}{6.836531in}}%
\pgfpathlineto{\pgfqpoint{3.010376in}{6.834580in}}%
\pgfpathlineto{\pgfqpoint{3.063420in}{6.815911in}}%
\pgfpathlineto{\pgfqpoint{3.116464in}{6.836791in}}%
\pgfpathlineto{\pgfqpoint{3.169508in}{6.866400in}}%
\pgfpathlineto{\pgfqpoint{3.222552in}{6.896923in}}%
\pgfpathlineto{\pgfqpoint{3.275596in}{6.813063in}}%
\pgfpathlineto{\pgfqpoint{3.328640in}{6.871638in}}%
\pgfpathlineto{\pgfqpoint{3.381684in}{6.788610in}}%
\pgfpathlineto{\pgfqpoint{3.434728in}{6.724023in}}%
\pgfpathlineto{\pgfqpoint{3.487772in}{6.748160in}}%
\pgfpathlineto{\pgfqpoint{3.540816in}{6.738883in}}%
\pgfpathlineto{\pgfqpoint{3.593860in}{6.664878in}}%
\pgfpathlineto{\pgfqpoint{3.646904in}{6.662324in}}%
\pgfpathlineto{\pgfqpoint{3.699948in}{6.618474in}}%
\pgfpathlineto{\pgfqpoint{3.752991in}{6.616195in}}%
\pgfpathlineto{\pgfqpoint{3.806035in}{6.658725in}}%
\pgfpathlineto{\pgfqpoint{3.859079in}{6.646123in}}%
\pgfpathlineto{\pgfqpoint{3.912123in}{6.607177in}}%
\pgfpathlineto{\pgfqpoint{3.965167in}{6.735611in}}%
\pgfpathlineto{\pgfqpoint{4.018211in}{6.664699in}}%
\pgfpathlineto{\pgfqpoint{4.071255in}{6.746790in}}%
\pgfpathlineto{\pgfqpoint{4.124299in}{6.772208in}}%
\pgfpathlineto{\pgfqpoint{4.177343in}{6.790679in}}%
\pgfpathlineto{\pgfqpoint{4.230387in}{6.828787in}}%
\pgfpathlineto{\pgfqpoint{4.283431in}{6.844506in}}%
\pgfpathlineto{\pgfqpoint{4.336475in}{6.889082in}}%
\pgfpathlineto{\pgfqpoint{4.389519in}{6.858537in}}%
\pgfpathlineto{\pgfqpoint{4.442563in}{6.847591in}}%
\pgfpathlineto{\pgfqpoint{4.495607in}{6.883010in}}%
\pgfpathlineto{\pgfqpoint{4.548650in}{6.899833in}}%
\pgfpathlineto{\pgfqpoint{4.601694in}{6.827112in}}%
\pgfpathlineto{\pgfqpoint{4.654738in}{6.846117in}}%
\pgfpathlineto{\pgfqpoint{4.707782in}{6.766257in}}%
\pgfpathlineto{\pgfqpoint{4.760826in}{6.775170in}}%
\pgfpathlineto{\pgfqpoint{4.813870in}{6.726380in}}%
\pgfpathlineto{\pgfqpoint{4.866914in}{6.684929in}}%
\pgfpathlineto{\pgfqpoint{4.919958in}{6.671298in}}%
\pgfpathlineto{\pgfqpoint{4.973002in}{6.607378in}}%
\pgfpathlineto{\pgfqpoint{5.026046in}{6.622809in}}%
\pgfpathlineto{\pgfqpoint{5.079090in}{6.578918in}}%
\pgfpathlineto{\pgfqpoint{5.132134in}{6.582269in}}%
\pgfpathlineto{\pgfqpoint{5.185178in}{6.670254in}}%
\pgfpathlineto{\pgfqpoint{5.238222in}{6.666672in}}%
\pgfpathlineto{\pgfqpoint{5.291266in}{6.726904in}}%
\pgfpathlineto{\pgfqpoint{5.344309in}{6.737403in}}%
\pgfpathlineto{\pgfqpoint{5.397353in}{6.748316in}}%
\pgfpathlineto{\pgfqpoint{5.450397in}{6.877918in}}%
\pgfpathlineto{\pgfqpoint{5.503441in}{6.868085in}}%
\pgfpathlineto{\pgfqpoint{5.556485in}{6.860891in}}%
\pgfpathlineto{\pgfqpoint{5.609529in}{6.874196in}}%
\pgfpathlineto{\pgfqpoint{5.662573in}{6.925790in}}%
\pgfpathlineto{\pgfqpoint{5.715617in}{6.873877in}}%
\pgfpathlineto{\pgfqpoint{5.768661in}{6.913209in}}%
\pgfpathlineto{\pgfqpoint{5.821705in}{6.872997in}}%
\pgfpathlineto{\pgfqpoint{5.874749in}{6.796671in}}%
\pgfpathlineto{\pgfqpoint{5.927793in}{6.820928in}}%
\pgfpathlineto{\pgfqpoint{5.980837in}{6.797521in}}%
\pgfpathlineto{\pgfqpoint{6.033881in}{6.748313in}}%
\pgfpathlineto{\pgfqpoint{6.086925in}{6.726383in}}%
\pgfpathlineto{\pgfqpoint{6.139969in}{6.707909in}}%
\pgfpathlineto{\pgfqpoint{6.193012in}{6.635398in}}%
\pgfpathlineto{\pgfqpoint{6.246056in}{6.691111in}}%
\pgfpathlineto{\pgfqpoint{6.299100in}{6.628412in}}%
\pgfpathlineto{\pgfqpoint{6.352144in}{6.647199in}}%
\pgfpathlineto{\pgfqpoint{6.405188in}{6.593987in}}%
\pgfpathlineto{\pgfqpoint{6.458232in}{6.680253in}}%
\pgfpathlineto{\pgfqpoint{6.511276in}{6.698006in}}%
\pgfpathlineto{\pgfqpoint{6.564320in}{6.694356in}}%
\pgfpathlineto{\pgfqpoint{6.617364in}{6.761313in}}%
\pgfpathlineto{\pgfqpoint{6.670408in}{6.791255in}}%
\pgfpathlineto{\pgfqpoint{6.723452in}{6.837336in}}%
\pgfpathlineto{\pgfqpoint{6.776496in}{6.859127in}}%
\pgfpathlineto{\pgfqpoint{6.829540in}{6.849303in}}%
\pgfpathlineto{\pgfqpoint{6.882584in}{6.881359in}}%
\pgfpathlineto{\pgfqpoint{6.935628in}{6.842957in}}%
\pgfpathlineto{\pgfqpoint{6.988671in}{6.886633in}}%
\pgfpathlineto{\pgfqpoint{7.041715in}{6.862219in}}%
\pgfpathlineto{\pgfqpoint{7.094759in}{6.885270in}}%
\pgfpathlineto{\pgfqpoint{7.147803in}{6.867065in}}%
\pgfpathlineto{\pgfqpoint{7.200847in}{6.818171in}}%
\pgfpathlineto{\pgfqpoint{7.253891in}{6.719285in}}%
\pgfpathlineto{\pgfqpoint{7.306935in}{6.747320in}}%
\pgfpathlineto{\pgfqpoint{7.359979in}{6.672189in}}%
\pgfpathlineto{\pgfqpoint{7.413023in}{6.684093in}}%
\pgfpathlineto{\pgfqpoint{7.466067in}{6.654137in}}%
\pgfpathlineto{\pgfqpoint{7.519111in}{6.606265in}}%
\pgfpathlineto{\pgfqpoint{7.572155in}{6.593688in}}%
\pgfpathlineto{\pgfqpoint{7.625199in}{6.652740in}}%
\pgfpathlineto{\pgfqpoint{7.678243in}{6.697655in}}%
\pgfpathlineto{\pgfqpoint{7.731287in}{6.660188in}}%
\pgfpathlineto{\pgfqpoint{7.784330in}{6.696220in}}%
\pgfpathlineto{\pgfqpoint{7.837374in}{6.704471in}}%
\pgfpathlineto{\pgfqpoint{7.890418in}{6.745774in}}%
\pgfpathlineto{\pgfqpoint{7.943462in}{6.743520in}}%
\pgfpathlineto{\pgfqpoint{7.996506in}{6.822223in}}%
\pgfpathlineto{\pgfqpoint{8.049550in}{6.853606in}}%
\pgfpathlineto{\pgfqpoint{8.102594in}{6.863050in}}%
\pgfpathlineto{\pgfqpoint{8.155638in}{6.888859in}}%
\pgfpathlineto{\pgfqpoint{8.208682in}{6.883000in}}%
\pgfpathlineto{\pgfqpoint{8.261726in}{6.924712in}}%
\pgfpathlineto{\pgfqpoint{8.314770in}{6.876087in}}%
\pgfpathlineto{\pgfqpoint{8.367814in}{6.865417in}}%
\pgfpathlineto{\pgfqpoint{8.420858in}{6.857476in}}%
\pgfpathlineto{\pgfqpoint{8.473902in}{6.806195in}}%
\pgfpathlineto{\pgfqpoint{8.526946in}{6.778054in}}%
\pgfpathlineto{\pgfqpoint{8.579990in}{6.685501in}}%
\pgfpathlineto{\pgfqpoint{8.633033in}{6.694774in}}%
\pgfpathlineto{\pgfqpoint{8.686077in}{6.723687in}}%
\pgfpathlineto{\pgfqpoint{8.739121in}{6.662088in}}%
\pgfpathlineto{\pgfqpoint{8.792165in}{6.664742in}}%
\pgfpathlineto{\pgfqpoint{8.845209in}{6.642906in}}%
\pgfpathlineto{\pgfqpoint{8.898253in}{6.685692in}}%
\pgfpathlineto{\pgfqpoint{8.951297in}{6.629260in}}%
\pgfpathlineto{\pgfqpoint{9.004341in}{6.648198in}}%
\pgfpathlineto{\pgfqpoint{9.057385in}{6.713621in}}%
\pgfpathlineto{\pgfqpoint{9.110429in}{6.748714in}}%
\pgfpathlineto{\pgfqpoint{9.163473in}{6.770647in}}%
\pgfpathlineto{\pgfqpoint{9.216517in}{6.804492in}}%
\pgfpathlineto{\pgfqpoint{9.269561in}{6.811485in}}%
\pgfpathlineto{\pgfqpoint{9.322605in}{6.883905in}}%
\pgfpathlineto{\pgfqpoint{9.375649in}{6.879220in}}%
\pgfpathlineto{\pgfqpoint{9.428692in}{6.905077in}}%
\pgfpathlineto{\pgfqpoint{9.481736in}{6.884500in}}%
\pgfpathlineto{\pgfqpoint{9.534780in}{6.946800in}}%
\pgfpathlineto{\pgfqpoint{9.587824in}{6.868462in}}%
\pgfpathlineto{\pgfqpoint{9.640868in}{6.864012in}}%
\pgfpathlineto{\pgfqpoint{9.693912in}{6.803680in}}%
\pgfpathlineto{\pgfqpoint{9.746956in}{6.779712in}}%
\pgfpathlineto{\pgfqpoint{9.800000in}{6.765861in}}%
\pgfpathlineto{\pgfqpoint{9.800000in}{6.765861in}}%
\pgfusepath{stroke}%
\end{pgfscope}%
\begin{pgfscope}%
\pgfsetbuttcap%
\pgfsetmiterjoin%
\definecolor{currentfill}{rgb}{1.000000,1.000000,1.000000}%
\pgfsetfillcolor{currentfill}%
\pgfsetlinewidth{1.003750pt}%
\definecolor{currentstroke}{rgb}{0.000000,0.000000,0.000000}%
\pgfsetstrokecolor{currentstroke}%
\pgfsetdash{}{0pt}%
\pgfpathmoveto{\pgfqpoint{1.017884in}{7.466516in}}%
\pgfpathlineto{\pgfqpoint{1.304733in}{7.466516in}}%
\pgfpathlineto{\pgfqpoint{1.304733in}{7.779293in}}%
\pgfpathlineto{\pgfqpoint{1.017884in}{7.779293in}}%
\pgfpathlineto{\pgfqpoint{1.017884in}{7.466516in}}%
\pgfpathclose%
\pgfusepath{stroke,fill}%
\end{pgfscope}%
\begin{pgfscope}%
\definecolor{textcolor}{rgb}{0.000000,0.000000,0.000000}%
\pgfsetstrokecolor{textcolor}%
\pgfsetfillcolor{textcolor}%
\pgftext[x=1.074273in,y=7.572904in,left,base]{\color{textcolor}{\rmfamily\fontsize{14.000000}{16.800000}\selectfont\catcode`\^=\active\def^{\ifmmode\sp\else\^{}\fi}\catcode`\%=\active\def%{\%}a)}}%
\end{pgfscope}%
\begin{pgfscope}%
\pgfsetbuttcap%
\pgfsetmiterjoin%
\definecolor{currentfill}{rgb}{1.000000,1.000000,1.000000}%
\pgfsetfillcolor{currentfill}%
\pgfsetlinewidth{0.000000pt}%
\definecolor{currentstroke}{rgb}{0.000000,0.000000,0.000000}%
\pgfsetstrokecolor{currentstroke}%
\pgfsetstrokeopacity{0.000000}%
\pgfsetdash{}{0pt}%
\pgfpathmoveto{\pgfqpoint{0.941663in}{0.670138in}}%
\pgfpathlineto{\pgfqpoint{9.800000in}{0.670138in}}%
\pgfpathlineto{\pgfqpoint{9.800000in}{4.135763in}}%
\pgfpathlineto{\pgfqpoint{0.941663in}{4.135763in}}%
\pgfpathlineto{\pgfqpoint{0.941663in}{0.670138in}}%
\pgfpathclose%
\pgfusepath{fill}%
\end{pgfscope}%
\begin{pgfscope}%
\pgfpathrectangle{\pgfqpoint{0.941663in}{0.670138in}}{\pgfqpoint{8.858337in}{3.465625in}}%
\pgfusepath{clip}%
\pgfsetrectcap%
\pgfsetroundjoin%
\pgfsetlinewidth{0.803000pt}%
\definecolor{currentstroke}{rgb}{0.690196,0.690196,0.690196}%
\pgfsetstrokecolor{currentstroke}%
\pgfsetdash{}{0pt}%
\pgfpathmoveto{\pgfqpoint{0.941663in}{0.670138in}}%
\pgfpathlineto{\pgfqpoint{0.941663in}{4.135763in}}%
\pgfusepath{stroke}%
\end{pgfscope}%
\begin{pgfscope}%
\pgfsetbuttcap%
\pgfsetroundjoin%
\definecolor{currentfill}{rgb}{0.000000,0.000000,0.000000}%
\pgfsetfillcolor{currentfill}%
\pgfsetlinewidth{0.803000pt}%
\definecolor{currentstroke}{rgb}{0.000000,0.000000,0.000000}%
\pgfsetstrokecolor{currentstroke}%
\pgfsetdash{}{0pt}%
\pgfsys@defobject{currentmarker}{\pgfqpoint{0.000000in}{-0.048611in}}{\pgfqpoint{0.000000in}{0.000000in}}{%
\pgfpathmoveto{\pgfqpoint{0.000000in}{0.000000in}}%
\pgfpathlineto{\pgfqpoint{0.000000in}{-0.048611in}}%
\pgfusepath{stroke,fill}%
}%
\begin{pgfscope}%
\pgfsys@transformshift{0.941663in}{0.670138in}%
\pgfsys@useobject{currentmarker}{}%
\end{pgfscope}%
\end{pgfscope}%
\begin{pgfscope}%
\definecolor{textcolor}{rgb}{0.000000,0.000000,0.000000}%
\pgfsetstrokecolor{textcolor}%
\pgfsetfillcolor{textcolor}%
\pgftext[x=0.941663in,y=0.572916in,,top]{\color{textcolor}{\rmfamily\fontsize{14.000000}{16.800000}\selectfont\catcode`\^=\active\def^{\ifmmode\sp\else\^{}\fi}\catcode`\%=\active\def%{\%}$\mathdefault{0}$}}%
\end{pgfscope}%
\begin{pgfscope}%
\pgfpathrectangle{\pgfqpoint{0.941663in}{0.670138in}}{\pgfqpoint{8.858337in}{3.465625in}}%
\pgfusepath{clip}%
\pgfsetrectcap%
\pgfsetroundjoin%
\pgfsetlinewidth{0.803000pt}%
\definecolor{currentstroke}{rgb}{0.690196,0.690196,0.690196}%
\pgfsetstrokecolor{currentstroke}%
\pgfsetdash{}{0pt}%
\pgfpathmoveto{\pgfqpoint{2.002542in}{0.670138in}}%
\pgfpathlineto{\pgfqpoint{2.002542in}{4.135763in}}%
\pgfusepath{stroke}%
\end{pgfscope}%
\begin{pgfscope}%
\pgfsetbuttcap%
\pgfsetroundjoin%
\definecolor{currentfill}{rgb}{0.000000,0.000000,0.000000}%
\pgfsetfillcolor{currentfill}%
\pgfsetlinewidth{0.803000pt}%
\definecolor{currentstroke}{rgb}{0.000000,0.000000,0.000000}%
\pgfsetstrokecolor{currentstroke}%
\pgfsetdash{}{0pt}%
\pgfsys@defobject{currentmarker}{\pgfqpoint{0.000000in}{-0.048611in}}{\pgfqpoint{0.000000in}{0.000000in}}{%
\pgfpathmoveto{\pgfqpoint{0.000000in}{0.000000in}}%
\pgfpathlineto{\pgfqpoint{0.000000in}{-0.048611in}}%
\pgfusepath{stroke,fill}%
}%
\begin{pgfscope}%
\pgfsys@transformshift{2.002542in}{0.670138in}%
\pgfsys@useobject{currentmarker}{}%
\end{pgfscope}%
\end{pgfscope}%
\begin{pgfscope}%
\definecolor{textcolor}{rgb}{0.000000,0.000000,0.000000}%
\pgfsetstrokecolor{textcolor}%
\pgfsetfillcolor{textcolor}%
\pgftext[x=2.002542in,y=0.572916in,,top]{\color{textcolor}{\rmfamily\fontsize{14.000000}{16.800000}\selectfont\catcode`\^=\active\def^{\ifmmode\sp\else\^{}\fi}\catcode`\%=\active\def%{\%}$\mathdefault{20}$}}%
\end{pgfscope}%
\begin{pgfscope}%
\pgfpathrectangle{\pgfqpoint{0.941663in}{0.670138in}}{\pgfqpoint{8.858337in}{3.465625in}}%
\pgfusepath{clip}%
\pgfsetrectcap%
\pgfsetroundjoin%
\pgfsetlinewidth{0.803000pt}%
\definecolor{currentstroke}{rgb}{0.690196,0.690196,0.690196}%
\pgfsetstrokecolor{currentstroke}%
\pgfsetdash{}{0pt}%
\pgfpathmoveto{\pgfqpoint{3.063420in}{0.670138in}}%
\pgfpathlineto{\pgfqpoint{3.063420in}{4.135763in}}%
\pgfusepath{stroke}%
\end{pgfscope}%
\begin{pgfscope}%
\pgfsetbuttcap%
\pgfsetroundjoin%
\definecolor{currentfill}{rgb}{0.000000,0.000000,0.000000}%
\pgfsetfillcolor{currentfill}%
\pgfsetlinewidth{0.803000pt}%
\definecolor{currentstroke}{rgb}{0.000000,0.000000,0.000000}%
\pgfsetstrokecolor{currentstroke}%
\pgfsetdash{}{0pt}%
\pgfsys@defobject{currentmarker}{\pgfqpoint{0.000000in}{-0.048611in}}{\pgfqpoint{0.000000in}{0.000000in}}{%
\pgfpathmoveto{\pgfqpoint{0.000000in}{0.000000in}}%
\pgfpathlineto{\pgfqpoint{0.000000in}{-0.048611in}}%
\pgfusepath{stroke,fill}%
}%
\begin{pgfscope}%
\pgfsys@transformshift{3.063420in}{0.670138in}%
\pgfsys@useobject{currentmarker}{}%
\end{pgfscope}%
\end{pgfscope}%
\begin{pgfscope}%
\definecolor{textcolor}{rgb}{0.000000,0.000000,0.000000}%
\pgfsetstrokecolor{textcolor}%
\pgfsetfillcolor{textcolor}%
\pgftext[x=3.063420in,y=0.572916in,,top]{\color{textcolor}{\rmfamily\fontsize{14.000000}{16.800000}\selectfont\catcode`\^=\active\def^{\ifmmode\sp\else\^{}\fi}\catcode`\%=\active\def%{\%}$\mathdefault{40}$}}%
\end{pgfscope}%
\begin{pgfscope}%
\pgfpathrectangle{\pgfqpoint{0.941663in}{0.670138in}}{\pgfqpoint{8.858337in}{3.465625in}}%
\pgfusepath{clip}%
\pgfsetrectcap%
\pgfsetroundjoin%
\pgfsetlinewidth{0.803000pt}%
\definecolor{currentstroke}{rgb}{0.690196,0.690196,0.690196}%
\pgfsetstrokecolor{currentstroke}%
\pgfsetdash{}{0pt}%
\pgfpathmoveto{\pgfqpoint{4.124299in}{0.670138in}}%
\pgfpathlineto{\pgfqpoint{4.124299in}{4.135763in}}%
\pgfusepath{stroke}%
\end{pgfscope}%
\begin{pgfscope}%
\pgfsetbuttcap%
\pgfsetroundjoin%
\definecolor{currentfill}{rgb}{0.000000,0.000000,0.000000}%
\pgfsetfillcolor{currentfill}%
\pgfsetlinewidth{0.803000pt}%
\definecolor{currentstroke}{rgb}{0.000000,0.000000,0.000000}%
\pgfsetstrokecolor{currentstroke}%
\pgfsetdash{}{0pt}%
\pgfsys@defobject{currentmarker}{\pgfqpoint{0.000000in}{-0.048611in}}{\pgfqpoint{0.000000in}{0.000000in}}{%
\pgfpathmoveto{\pgfqpoint{0.000000in}{0.000000in}}%
\pgfpathlineto{\pgfqpoint{0.000000in}{-0.048611in}}%
\pgfusepath{stroke,fill}%
}%
\begin{pgfscope}%
\pgfsys@transformshift{4.124299in}{0.670138in}%
\pgfsys@useobject{currentmarker}{}%
\end{pgfscope}%
\end{pgfscope}%
\begin{pgfscope}%
\definecolor{textcolor}{rgb}{0.000000,0.000000,0.000000}%
\pgfsetstrokecolor{textcolor}%
\pgfsetfillcolor{textcolor}%
\pgftext[x=4.124299in,y=0.572916in,,top]{\color{textcolor}{\rmfamily\fontsize{14.000000}{16.800000}\selectfont\catcode`\^=\active\def^{\ifmmode\sp\else\^{}\fi}\catcode`\%=\active\def%{\%}$\mathdefault{60}$}}%
\end{pgfscope}%
\begin{pgfscope}%
\pgfpathrectangle{\pgfqpoint{0.941663in}{0.670138in}}{\pgfqpoint{8.858337in}{3.465625in}}%
\pgfusepath{clip}%
\pgfsetrectcap%
\pgfsetroundjoin%
\pgfsetlinewidth{0.803000pt}%
\definecolor{currentstroke}{rgb}{0.690196,0.690196,0.690196}%
\pgfsetstrokecolor{currentstroke}%
\pgfsetdash{}{0pt}%
\pgfpathmoveto{\pgfqpoint{5.185178in}{0.670138in}}%
\pgfpathlineto{\pgfqpoint{5.185178in}{4.135763in}}%
\pgfusepath{stroke}%
\end{pgfscope}%
\begin{pgfscope}%
\pgfsetbuttcap%
\pgfsetroundjoin%
\definecolor{currentfill}{rgb}{0.000000,0.000000,0.000000}%
\pgfsetfillcolor{currentfill}%
\pgfsetlinewidth{0.803000pt}%
\definecolor{currentstroke}{rgb}{0.000000,0.000000,0.000000}%
\pgfsetstrokecolor{currentstroke}%
\pgfsetdash{}{0pt}%
\pgfsys@defobject{currentmarker}{\pgfqpoint{0.000000in}{-0.048611in}}{\pgfqpoint{0.000000in}{0.000000in}}{%
\pgfpathmoveto{\pgfqpoint{0.000000in}{0.000000in}}%
\pgfpathlineto{\pgfqpoint{0.000000in}{-0.048611in}}%
\pgfusepath{stroke,fill}%
}%
\begin{pgfscope}%
\pgfsys@transformshift{5.185178in}{0.670138in}%
\pgfsys@useobject{currentmarker}{}%
\end{pgfscope}%
\end{pgfscope}%
\begin{pgfscope}%
\definecolor{textcolor}{rgb}{0.000000,0.000000,0.000000}%
\pgfsetstrokecolor{textcolor}%
\pgfsetfillcolor{textcolor}%
\pgftext[x=5.185178in,y=0.572916in,,top]{\color{textcolor}{\rmfamily\fontsize{14.000000}{16.800000}\selectfont\catcode`\^=\active\def^{\ifmmode\sp\else\^{}\fi}\catcode`\%=\active\def%{\%}$\mathdefault{80}$}}%
\end{pgfscope}%
\begin{pgfscope}%
\pgfpathrectangle{\pgfqpoint{0.941663in}{0.670138in}}{\pgfqpoint{8.858337in}{3.465625in}}%
\pgfusepath{clip}%
\pgfsetrectcap%
\pgfsetroundjoin%
\pgfsetlinewidth{0.803000pt}%
\definecolor{currentstroke}{rgb}{0.690196,0.690196,0.690196}%
\pgfsetstrokecolor{currentstroke}%
\pgfsetdash{}{0pt}%
\pgfpathmoveto{\pgfqpoint{6.246056in}{0.670138in}}%
\pgfpathlineto{\pgfqpoint{6.246056in}{4.135763in}}%
\pgfusepath{stroke}%
\end{pgfscope}%
\begin{pgfscope}%
\pgfsetbuttcap%
\pgfsetroundjoin%
\definecolor{currentfill}{rgb}{0.000000,0.000000,0.000000}%
\pgfsetfillcolor{currentfill}%
\pgfsetlinewidth{0.803000pt}%
\definecolor{currentstroke}{rgb}{0.000000,0.000000,0.000000}%
\pgfsetstrokecolor{currentstroke}%
\pgfsetdash{}{0pt}%
\pgfsys@defobject{currentmarker}{\pgfqpoint{0.000000in}{-0.048611in}}{\pgfqpoint{0.000000in}{0.000000in}}{%
\pgfpathmoveto{\pgfqpoint{0.000000in}{0.000000in}}%
\pgfpathlineto{\pgfqpoint{0.000000in}{-0.048611in}}%
\pgfusepath{stroke,fill}%
}%
\begin{pgfscope}%
\pgfsys@transformshift{6.246056in}{0.670138in}%
\pgfsys@useobject{currentmarker}{}%
\end{pgfscope}%
\end{pgfscope}%
\begin{pgfscope}%
\definecolor{textcolor}{rgb}{0.000000,0.000000,0.000000}%
\pgfsetstrokecolor{textcolor}%
\pgfsetfillcolor{textcolor}%
\pgftext[x=6.246056in,y=0.572916in,,top]{\color{textcolor}{\rmfamily\fontsize{14.000000}{16.800000}\selectfont\catcode`\^=\active\def^{\ifmmode\sp\else\^{}\fi}\catcode`\%=\active\def%{\%}$\mathdefault{100}$}}%
\end{pgfscope}%
\begin{pgfscope}%
\pgfpathrectangle{\pgfqpoint{0.941663in}{0.670138in}}{\pgfqpoint{8.858337in}{3.465625in}}%
\pgfusepath{clip}%
\pgfsetrectcap%
\pgfsetroundjoin%
\pgfsetlinewidth{0.803000pt}%
\definecolor{currentstroke}{rgb}{0.690196,0.690196,0.690196}%
\pgfsetstrokecolor{currentstroke}%
\pgfsetdash{}{0pt}%
\pgfpathmoveto{\pgfqpoint{7.306935in}{0.670138in}}%
\pgfpathlineto{\pgfqpoint{7.306935in}{4.135763in}}%
\pgfusepath{stroke}%
\end{pgfscope}%
\begin{pgfscope}%
\pgfsetbuttcap%
\pgfsetroundjoin%
\definecolor{currentfill}{rgb}{0.000000,0.000000,0.000000}%
\pgfsetfillcolor{currentfill}%
\pgfsetlinewidth{0.803000pt}%
\definecolor{currentstroke}{rgb}{0.000000,0.000000,0.000000}%
\pgfsetstrokecolor{currentstroke}%
\pgfsetdash{}{0pt}%
\pgfsys@defobject{currentmarker}{\pgfqpoint{0.000000in}{-0.048611in}}{\pgfqpoint{0.000000in}{0.000000in}}{%
\pgfpathmoveto{\pgfqpoint{0.000000in}{0.000000in}}%
\pgfpathlineto{\pgfqpoint{0.000000in}{-0.048611in}}%
\pgfusepath{stroke,fill}%
}%
\begin{pgfscope}%
\pgfsys@transformshift{7.306935in}{0.670138in}%
\pgfsys@useobject{currentmarker}{}%
\end{pgfscope}%
\end{pgfscope}%
\begin{pgfscope}%
\definecolor{textcolor}{rgb}{0.000000,0.000000,0.000000}%
\pgfsetstrokecolor{textcolor}%
\pgfsetfillcolor{textcolor}%
\pgftext[x=7.306935in,y=0.572916in,,top]{\color{textcolor}{\rmfamily\fontsize{14.000000}{16.800000}\selectfont\catcode`\^=\active\def^{\ifmmode\sp\else\^{}\fi}\catcode`\%=\active\def%{\%}$\mathdefault{120}$}}%
\end{pgfscope}%
\begin{pgfscope}%
\pgfpathrectangle{\pgfqpoint{0.941663in}{0.670138in}}{\pgfqpoint{8.858337in}{3.465625in}}%
\pgfusepath{clip}%
\pgfsetrectcap%
\pgfsetroundjoin%
\pgfsetlinewidth{0.803000pt}%
\definecolor{currentstroke}{rgb}{0.690196,0.690196,0.690196}%
\pgfsetstrokecolor{currentstroke}%
\pgfsetdash{}{0pt}%
\pgfpathmoveto{\pgfqpoint{8.367814in}{0.670138in}}%
\pgfpathlineto{\pgfqpoint{8.367814in}{4.135763in}}%
\pgfusepath{stroke}%
\end{pgfscope}%
\begin{pgfscope}%
\pgfsetbuttcap%
\pgfsetroundjoin%
\definecolor{currentfill}{rgb}{0.000000,0.000000,0.000000}%
\pgfsetfillcolor{currentfill}%
\pgfsetlinewidth{0.803000pt}%
\definecolor{currentstroke}{rgb}{0.000000,0.000000,0.000000}%
\pgfsetstrokecolor{currentstroke}%
\pgfsetdash{}{0pt}%
\pgfsys@defobject{currentmarker}{\pgfqpoint{0.000000in}{-0.048611in}}{\pgfqpoint{0.000000in}{0.000000in}}{%
\pgfpathmoveto{\pgfqpoint{0.000000in}{0.000000in}}%
\pgfpathlineto{\pgfqpoint{0.000000in}{-0.048611in}}%
\pgfusepath{stroke,fill}%
}%
\begin{pgfscope}%
\pgfsys@transformshift{8.367814in}{0.670138in}%
\pgfsys@useobject{currentmarker}{}%
\end{pgfscope}%
\end{pgfscope}%
\begin{pgfscope}%
\definecolor{textcolor}{rgb}{0.000000,0.000000,0.000000}%
\pgfsetstrokecolor{textcolor}%
\pgfsetfillcolor{textcolor}%
\pgftext[x=8.367814in,y=0.572916in,,top]{\color{textcolor}{\rmfamily\fontsize{14.000000}{16.800000}\selectfont\catcode`\^=\active\def^{\ifmmode\sp\else\^{}\fi}\catcode`\%=\active\def%{\%}$\mathdefault{140}$}}%
\end{pgfscope}%
\begin{pgfscope}%
\pgfpathrectangle{\pgfqpoint{0.941663in}{0.670138in}}{\pgfqpoint{8.858337in}{3.465625in}}%
\pgfusepath{clip}%
\pgfsetrectcap%
\pgfsetroundjoin%
\pgfsetlinewidth{0.803000pt}%
\definecolor{currentstroke}{rgb}{0.690196,0.690196,0.690196}%
\pgfsetstrokecolor{currentstroke}%
\pgfsetdash{}{0pt}%
\pgfpathmoveto{\pgfqpoint{9.428692in}{0.670138in}}%
\pgfpathlineto{\pgfqpoint{9.428692in}{4.135763in}}%
\pgfusepath{stroke}%
\end{pgfscope}%
\begin{pgfscope}%
\pgfsetbuttcap%
\pgfsetroundjoin%
\definecolor{currentfill}{rgb}{0.000000,0.000000,0.000000}%
\pgfsetfillcolor{currentfill}%
\pgfsetlinewidth{0.803000pt}%
\definecolor{currentstroke}{rgb}{0.000000,0.000000,0.000000}%
\pgfsetstrokecolor{currentstroke}%
\pgfsetdash{}{0pt}%
\pgfsys@defobject{currentmarker}{\pgfqpoint{0.000000in}{-0.048611in}}{\pgfqpoint{0.000000in}{0.000000in}}{%
\pgfpathmoveto{\pgfqpoint{0.000000in}{0.000000in}}%
\pgfpathlineto{\pgfqpoint{0.000000in}{-0.048611in}}%
\pgfusepath{stroke,fill}%
}%
\begin{pgfscope}%
\pgfsys@transformshift{9.428692in}{0.670138in}%
\pgfsys@useobject{currentmarker}{}%
\end{pgfscope}%
\end{pgfscope}%
\begin{pgfscope}%
\definecolor{textcolor}{rgb}{0.000000,0.000000,0.000000}%
\pgfsetstrokecolor{textcolor}%
\pgfsetfillcolor{textcolor}%
\pgftext[x=9.428692in,y=0.572916in,,top]{\color{textcolor}{\rmfamily\fontsize{14.000000}{16.800000}\selectfont\catcode`\^=\active\def^{\ifmmode\sp\else\^{}\fi}\catcode`\%=\active\def%{\%}$\mathdefault{160}$}}%
\end{pgfscope}%
\begin{pgfscope}%
\definecolor{textcolor}{rgb}{0.000000,0.000000,0.000000}%
\pgfsetstrokecolor{textcolor}%
\pgfsetfillcolor{textcolor}%
\pgftext[x=5.370831in,y=0.339583in,,top]{\color{textcolor}{\rmfamily\fontsize{18.000000}{21.600000}\selectfont\catcode`\^=\active\def^{\ifmmode\sp\else\^{}\fi}\catcode`\%=\active\def%{\%}Time [hours]}}%
\end{pgfscope}%
\begin{pgfscope}%
\pgfpathrectangle{\pgfqpoint{0.941663in}{0.670138in}}{\pgfqpoint{8.858337in}{3.465625in}}%
\pgfusepath{clip}%
\pgfsetrectcap%
\pgfsetroundjoin%
\pgfsetlinewidth{0.803000pt}%
\definecolor{currentstroke}{rgb}{0.690196,0.690196,0.690196}%
\pgfsetstrokecolor{currentstroke}%
\pgfsetdash{}{0pt}%
\pgfpathmoveto{\pgfqpoint{0.941663in}{1.195548in}}%
\pgfpathlineto{\pgfqpoint{9.800000in}{1.195548in}}%
\pgfusepath{stroke}%
\end{pgfscope}%
\begin{pgfscope}%
\pgfsetbuttcap%
\pgfsetroundjoin%
\definecolor{currentfill}{rgb}{0.000000,0.000000,0.000000}%
\pgfsetfillcolor{currentfill}%
\pgfsetlinewidth{0.803000pt}%
\definecolor{currentstroke}{rgb}{0.000000,0.000000,0.000000}%
\pgfsetstrokecolor{currentstroke}%
\pgfsetdash{}{0pt}%
\pgfsys@defobject{currentmarker}{\pgfqpoint{-0.048611in}{0.000000in}}{\pgfqpoint{-0.000000in}{0.000000in}}{%
\pgfpathmoveto{\pgfqpoint{-0.000000in}{0.000000in}}%
\pgfpathlineto{\pgfqpoint{-0.048611in}{0.000000in}}%
\pgfusepath{stroke,fill}%
}%
\begin{pgfscope}%
\pgfsys@transformshift{0.941663in}{1.195548in}%
\pgfsys@useobject{currentmarker}{}%
\end{pgfscope}%
\end{pgfscope}%
\begin{pgfscope}%
\definecolor{textcolor}{rgb}{0.000000,0.000000,0.000000}%
\pgfsetstrokecolor{textcolor}%
\pgfsetfillcolor{textcolor}%
\pgftext[x=0.395138in, y=1.126104in, left, base]{\color{textcolor}{\rmfamily\fontsize{14.000000}{16.800000}\selectfont\catcode`\^=\active\def^{\ifmmode\sp\else\^{}\fi}\catcode`\%=\active\def%{\%}$\mathdefault{\ensuremath{-}500}$}}%
\end{pgfscope}%
\begin{pgfscope}%
\pgfpathrectangle{\pgfqpoint{0.941663in}{0.670138in}}{\pgfqpoint{8.858337in}{3.465625in}}%
\pgfusepath{clip}%
\pgfsetrectcap%
\pgfsetroundjoin%
\pgfsetlinewidth{0.803000pt}%
\definecolor{currentstroke}{rgb}{0.690196,0.690196,0.690196}%
\pgfsetstrokecolor{currentstroke}%
\pgfsetdash{}{0pt}%
\pgfpathmoveto{\pgfqpoint{0.941663in}{1.891220in}}%
\pgfpathlineto{\pgfqpoint{9.800000in}{1.891220in}}%
\pgfusepath{stroke}%
\end{pgfscope}%
\begin{pgfscope}%
\pgfsetbuttcap%
\pgfsetroundjoin%
\definecolor{currentfill}{rgb}{0.000000,0.000000,0.000000}%
\pgfsetfillcolor{currentfill}%
\pgfsetlinewidth{0.803000pt}%
\definecolor{currentstroke}{rgb}{0.000000,0.000000,0.000000}%
\pgfsetstrokecolor{currentstroke}%
\pgfsetdash{}{0pt}%
\pgfsys@defobject{currentmarker}{\pgfqpoint{-0.048611in}{0.000000in}}{\pgfqpoint{-0.000000in}{0.000000in}}{%
\pgfpathmoveto{\pgfqpoint{-0.000000in}{0.000000in}}%
\pgfpathlineto{\pgfqpoint{-0.048611in}{0.000000in}}%
\pgfusepath{stroke,fill}%
}%
\begin{pgfscope}%
\pgfsys@transformshift{0.941663in}{1.891220in}%
\pgfsys@useobject{currentmarker}{}%
\end{pgfscope}%
\end{pgfscope}%
\begin{pgfscope}%
\definecolor{textcolor}{rgb}{0.000000,0.000000,0.000000}%
\pgfsetstrokecolor{textcolor}%
\pgfsetfillcolor{textcolor}%
\pgftext[x=0.746525in, y=1.821776in, left, base]{\color{textcolor}{\rmfamily\fontsize{14.000000}{16.800000}\selectfont\catcode`\^=\active\def^{\ifmmode\sp\else\^{}\fi}\catcode`\%=\active\def%{\%}$\mathdefault{0}$}}%
\end{pgfscope}%
\begin{pgfscope}%
\pgfpathrectangle{\pgfqpoint{0.941663in}{0.670138in}}{\pgfqpoint{8.858337in}{3.465625in}}%
\pgfusepath{clip}%
\pgfsetrectcap%
\pgfsetroundjoin%
\pgfsetlinewidth{0.803000pt}%
\definecolor{currentstroke}{rgb}{0.690196,0.690196,0.690196}%
\pgfsetstrokecolor{currentstroke}%
\pgfsetdash{}{0pt}%
\pgfpathmoveto{\pgfqpoint{0.941663in}{2.586892in}}%
\pgfpathlineto{\pgfqpoint{9.800000in}{2.586892in}}%
\pgfusepath{stroke}%
\end{pgfscope}%
\begin{pgfscope}%
\pgfsetbuttcap%
\pgfsetroundjoin%
\definecolor{currentfill}{rgb}{0.000000,0.000000,0.000000}%
\pgfsetfillcolor{currentfill}%
\pgfsetlinewidth{0.803000pt}%
\definecolor{currentstroke}{rgb}{0.000000,0.000000,0.000000}%
\pgfsetstrokecolor{currentstroke}%
\pgfsetdash{}{0pt}%
\pgfsys@defobject{currentmarker}{\pgfqpoint{-0.048611in}{0.000000in}}{\pgfqpoint{-0.000000in}{0.000000in}}{%
\pgfpathmoveto{\pgfqpoint{-0.000000in}{0.000000in}}%
\pgfpathlineto{\pgfqpoint{-0.048611in}{0.000000in}}%
\pgfusepath{stroke,fill}%
}%
\begin{pgfscope}%
\pgfsys@transformshift{0.941663in}{2.586892in}%
\pgfsys@useobject{currentmarker}{}%
\end{pgfscope}%
\end{pgfscope}%
\begin{pgfscope}%
\definecolor{textcolor}{rgb}{0.000000,0.000000,0.000000}%
\pgfsetstrokecolor{textcolor}%
\pgfsetfillcolor{textcolor}%
\pgftext[x=0.550694in, y=2.517447in, left, base]{\color{textcolor}{\rmfamily\fontsize{14.000000}{16.800000}\selectfont\catcode`\^=\active\def^{\ifmmode\sp\else\^{}\fi}\catcode`\%=\active\def%{\%}$\mathdefault{500}$}}%
\end{pgfscope}%
\begin{pgfscope}%
\pgfpathrectangle{\pgfqpoint{0.941663in}{0.670138in}}{\pgfqpoint{8.858337in}{3.465625in}}%
\pgfusepath{clip}%
\pgfsetrectcap%
\pgfsetroundjoin%
\pgfsetlinewidth{0.803000pt}%
\definecolor{currentstroke}{rgb}{0.690196,0.690196,0.690196}%
\pgfsetstrokecolor{currentstroke}%
\pgfsetdash{}{0pt}%
\pgfpathmoveto{\pgfqpoint{0.941663in}{3.282563in}}%
\pgfpathlineto{\pgfqpoint{9.800000in}{3.282563in}}%
\pgfusepath{stroke}%
\end{pgfscope}%
\begin{pgfscope}%
\pgfsetbuttcap%
\pgfsetroundjoin%
\definecolor{currentfill}{rgb}{0.000000,0.000000,0.000000}%
\pgfsetfillcolor{currentfill}%
\pgfsetlinewidth{0.803000pt}%
\definecolor{currentstroke}{rgb}{0.000000,0.000000,0.000000}%
\pgfsetstrokecolor{currentstroke}%
\pgfsetdash{}{0pt}%
\pgfsys@defobject{currentmarker}{\pgfqpoint{-0.048611in}{0.000000in}}{\pgfqpoint{-0.000000in}{0.000000in}}{%
\pgfpathmoveto{\pgfqpoint{-0.000000in}{0.000000in}}%
\pgfpathlineto{\pgfqpoint{-0.048611in}{0.000000in}}%
\pgfusepath{stroke,fill}%
}%
\begin{pgfscope}%
\pgfsys@transformshift{0.941663in}{3.282563in}%
\pgfsys@useobject{currentmarker}{}%
\end{pgfscope}%
\end{pgfscope}%
\begin{pgfscope}%
\definecolor{textcolor}{rgb}{0.000000,0.000000,0.000000}%
\pgfsetstrokecolor{textcolor}%
\pgfsetfillcolor{textcolor}%
\pgftext[x=0.452779in, y=3.213119in, left, base]{\color{textcolor}{\rmfamily\fontsize{14.000000}{16.800000}\selectfont\catcode`\^=\active\def^{\ifmmode\sp\else\^{}\fi}\catcode`\%=\active\def%{\%}$\mathdefault{1000}$}}%
\end{pgfscope}%
\begin{pgfscope}%
\pgfpathrectangle{\pgfqpoint{0.941663in}{0.670138in}}{\pgfqpoint{8.858337in}{3.465625in}}%
\pgfusepath{clip}%
\pgfsetrectcap%
\pgfsetroundjoin%
\pgfsetlinewidth{0.803000pt}%
\definecolor{currentstroke}{rgb}{0.690196,0.690196,0.690196}%
\pgfsetstrokecolor{currentstroke}%
\pgfsetdash{}{0pt}%
\pgfpathmoveto{\pgfqpoint{0.941663in}{3.978235in}}%
\pgfpathlineto{\pgfqpoint{9.800000in}{3.978235in}}%
\pgfusepath{stroke}%
\end{pgfscope}%
\begin{pgfscope}%
\pgfsetbuttcap%
\pgfsetroundjoin%
\definecolor{currentfill}{rgb}{0.000000,0.000000,0.000000}%
\pgfsetfillcolor{currentfill}%
\pgfsetlinewidth{0.803000pt}%
\definecolor{currentstroke}{rgb}{0.000000,0.000000,0.000000}%
\pgfsetstrokecolor{currentstroke}%
\pgfsetdash{}{0pt}%
\pgfsys@defobject{currentmarker}{\pgfqpoint{-0.048611in}{0.000000in}}{\pgfqpoint{-0.000000in}{0.000000in}}{%
\pgfpathmoveto{\pgfqpoint{-0.000000in}{0.000000in}}%
\pgfpathlineto{\pgfqpoint{-0.048611in}{0.000000in}}%
\pgfusepath{stroke,fill}%
}%
\begin{pgfscope}%
\pgfsys@transformshift{0.941663in}{3.978235in}%
\pgfsys@useobject{currentmarker}{}%
\end{pgfscope}%
\end{pgfscope}%
\begin{pgfscope}%
\definecolor{textcolor}{rgb}{0.000000,0.000000,0.000000}%
\pgfsetstrokecolor{textcolor}%
\pgfsetfillcolor{textcolor}%
\pgftext[x=0.452779in, y=3.908791in, left, base]{\color{textcolor}{\rmfamily\fontsize{14.000000}{16.800000}\selectfont\catcode`\^=\active\def^{\ifmmode\sp\else\^{}\fi}\catcode`\%=\active\def%{\%}$\mathdefault{1500}$}}%
\end{pgfscope}%
\begin{pgfscope}%
\definecolor{textcolor}{rgb}{0.000000,0.000000,0.000000}%
\pgfsetstrokecolor{textcolor}%
\pgfsetfillcolor{textcolor}%
\pgftext[x=0.339583in,y=2.402951in,,bottom,rotate=90.000000]{\color{textcolor}{\rmfamily\fontsize{18.000000}{21.600000}\selectfont\catcode`\^=\active\def^{\ifmmode\sp\else\^{}\fi}\catcode`\%=\active\def%{\%}Energy [MWh]}}%
\end{pgfscope}%
\begin{pgfscope}%
\pgfpathrectangle{\pgfqpoint{0.941663in}{0.670138in}}{\pgfqpoint{8.858337in}{3.465625in}}%
\pgfusepath{clip}%
\pgfsetrectcap%
\pgfsetroundjoin%
\pgfsetlinewidth{1.505625pt}%
\definecolor{currentstroke}{rgb}{0.121569,0.466667,0.705882}%
\pgfsetstrokecolor{currentstroke}%
\pgfsetdash{}{0pt}%
\pgfpathmoveto{\pgfqpoint{0.941663in}{2.586892in}}%
\pgfpathlineto{\pgfqpoint{9.800000in}{2.586892in}}%
\pgfpathlineto{\pgfqpoint{9.800000in}{2.586892in}}%
\pgfusepath{stroke}%
\end{pgfscope}%
\begin{pgfscope}%
\pgfpathrectangle{\pgfqpoint{0.941663in}{0.670138in}}{\pgfqpoint{8.858337in}{3.465625in}}%
\pgfusepath{clip}%
\pgfsetbuttcap%
\pgfsetroundjoin%
\definecolor{currentfill}{rgb}{0.121569,0.466667,0.705882}%
\pgfsetfillcolor{currentfill}%
\pgfsetlinewidth{1.003750pt}%
\definecolor{currentstroke}{rgb}{0.121569,0.466667,0.705882}%
\pgfsetstrokecolor{currentstroke}%
\pgfsetdash{}{0pt}%
\pgfsys@defobject{currentmarker}{\pgfqpoint{0.941663in}{1.891220in}}{\pgfqpoint{9.800000in}{2.586892in}}{%
\pgfpathmoveto{\pgfqpoint{0.941663in}{2.586892in}}%
\pgfpathlineto{\pgfqpoint{0.941663in}{1.891220in}}%
\pgfpathlineto{\pgfqpoint{0.994707in}{1.891220in}}%
\pgfpathlineto{\pgfqpoint{1.047751in}{1.891220in}}%
\pgfpathlineto{\pgfqpoint{1.100795in}{1.891220in}}%
\pgfpathlineto{\pgfqpoint{1.153839in}{1.891220in}}%
\pgfpathlineto{\pgfqpoint{1.206883in}{1.891220in}}%
\pgfpathlineto{\pgfqpoint{1.259927in}{1.891220in}}%
\pgfpathlineto{\pgfqpoint{1.312970in}{1.891220in}}%
\pgfpathlineto{\pgfqpoint{1.366014in}{1.891220in}}%
\pgfpathlineto{\pgfqpoint{1.419058in}{1.891220in}}%
\pgfpathlineto{\pgfqpoint{1.472102in}{1.891220in}}%
\pgfpathlineto{\pgfqpoint{1.525146in}{1.891220in}}%
\pgfpathlineto{\pgfqpoint{1.578190in}{1.891220in}}%
\pgfpathlineto{\pgfqpoint{1.631234in}{1.891220in}}%
\pgfpathlineto{\pgfqpoint{1.684278in}{1.891220in}}%
\pgfpathlineto{\pgfqpoint{1.737322in}{1.891220in}}%
\pgfpathlineto{\pgfqpoint{1.790366in}{1.891220in}}%
\pgfpathlineto{\pgfqpoint{1.843410in}{1.891220in}}%
\pgfpathlineto{\pgfqpoint{1.896454in}{1.891220in}}%
\pgfpathlineto{\pgfqpoint{1.949498in}{1.891220in}}%
\pgfpathlineto{\pgfqpoint{2.002542in}{1.891220in}}%
\pgfpathlineto{\pgfqpoint{2.055586in}{1.891220in}}%
\pgfpathlineto{\pgfqpoint{2.108629in}{1.891220in}}%
\pgfpathlineto{\pgfqpoint{2.161673in}{1.891220in}}%
\pgfpathlineto{\pgfqpoint{2.214717in}{1.891220in}}%
\pgfpathlineto{\pgfqpoint{2.267761in}{1.891220in}}%
\pgfpathlineto{\pgfqpoint{2.320805in}{1.891220in}}%
\pgfpathlineto{\pgfqpoint{2.373849in}{1.891220in}}%
\pgfpathlineto{\pgfqpoint{2.426893in}{1.891220in}}%
\pgfpathlineto{\pgfqpoint{2.479937in}{1.891220in}}%
\pgfpathlineto{\pgfqpoint{2.532981in}{1.891220in}}%
\pgfpathlineto{\pgfqpoint{2.586025in}{1.891220in}}%
\pgfpathlineto{\pgfqpoint{2.639069in}{1.891220in}}%
\pgfpathlineto{\pgfqpoint{2.692113in}{1.891220in}}%
\pgfpathlineto{\pgfqpoint{2.745157in}{1.891220in}}%
\pgfpathlineto{\pgfqpoint{2.798201in}{1.891220in}}%
\pgfpathlineto{\pgfqpoint{2.851245in}{1.891220in}}%
\pgfpathlineto{\pgfqpoint{2.904288in}{1.891220in}}%
\pgfpathlineto{\pgfqpoint{2.957332in}{1.891220in}}%
\pgfpathlineto{\pgfqpoint{3.010376in}{1.891220in}}%
\pgfpathlineto{\pgfqpoint{3.063420in}{1.891220in}}%
\pgfpathlineto{\pgfqpoint{3.116464in}{1.891220in}}%
\pgfpathlineto{\pgfqpoint{3.169508in}{1.891220in}}%
\pgfpathlineto{\pgfqpoint{3.222552in}{1.891220in}}%
\pgfpathlineto{\pgfqpoint{3.275596in}{1.891220in}}%
\pgfpathlineto{\pgfqpoint{3.328640in}{1.891220in}}%
\pgfpathlineto{\pgfqpoint{3.381684in}{1.891220in}}%
\pgfpathlineto{\pgfqpoint{3.434728in}{1.891220in}}%
\pgfpathlineto{\pgfqpoint{3.487772in}{1.891220in}}%
\pgfpathlineto{\pgfqpoint{3.540816in}{1.891220in}}%
\pgfpathlineto{\pgfqpoint{3.593860in}{1.891220in}}%
\pgfpathlineto{\pgfqpoint{3.646904in}{1.891220in}}%
\pgfpathlineto{\pgfqpoint{3.699948in}{1.891220in}}%
\pgfpathlineto{\pgfqpoint{3.752991in}{1.891220in}}%
\pgfpathlineto{\pgfqpoint{3.806035in}{1.891220in}}%
\pgfpathlineto{\pgfqpoint{3.859079in}{1.891220in}}%
\pgfpathlineto{\pgfqpoint{3.912123in}{1.891220in}}%
\pgfpathlineto{\pgfqpoint{3.965167in}{1.891220in}}%
\pgfpathlineto{\pgfqpoint{4.018211in}{1.891220in}}%
\pgfpathlineto{\pgfqpoint{4.071255in}{1.891220in}}%
\pgfpathlineto{\pgfqpoint{4.124299in}{1.891220in}}%
\pgfpathlineto{\pgfqpoint{4.177343in}{1.891220in}}%
\pgfpathlineto{\pgfqpoint{4.230387in}{1.891220in}}%
\pgfpathlineto{\pgfqpoint{4.283431in}{1.891220in}}%
\pgfpathlineto{\pgfqpoint{4.336475in}{1.891220in}}%
\pgfpathlineto{\pgfqpoint{4.389519in}{1.891220in}}%
\pgfpathlineto{\pgfqpoint{4.442563in}{1.891220in}}%
\pgfpathlineto{\pgfqpoint{4.495607in}{1.891220in}}%
\pgfpathlineto{\pgfqpoint{4.548650in}{1.891220in}}%
\pgfpathlineto{\pgfqpoint{4.601694in}{1.891220in}}%
\pgfpathlineto{\pgfqpoint{4.654738in}{1.891220in}}%
\pgfpathlineto{\pgfqpoint{4.707782in}{1.891220in}}%
\pgfpathlineto{\pgfqpoint{4.760826in}{1.891220in}}%
\pgfpathlineto{\pgfqpoint{4.813870in}{1.891220in}}%
\pgfpathlineto{\pgfqpoint{4.866914in}{1.891220in}}%
\pgfpathlineto{\pgfqpoint{4.919958in}{1.891220in}}%
\pgfpathlineto{\pgfqpoint{4.973002in}{1.891220in}}%
\pgfpathlineto{\pgfqpoint{5.026046in}{1.891220in}}%
\pgfpathlineto{\pgfqpoint{5.079090in}{1.891220in}}%
\pgfpathlineto{\pgfqpoint{5.132134in}{1.891220in}}%
\pgfpathlineto{\pgfqpoint{5.185178in}{1.891220in}}%
\pgfpathlineto{\pgfqpoint{5.238222in}{1.891220in}}%
\pgfpathlineto{\pgfqpoint{5.291266in}{1.891220in}}%
\pgfpathlineto{\pgfqpoint{5.344309in}{1.891220in}}%
\pgfpathlineto{\pgfqpoint{5.397353in}{1.891220in}}%
\pgfpathlineto{\pgfqpoint{5.450397in}{1.891220in}}%
\pgfpathlineto{\pgfqpoint{5.503441in}{1.891220in}}%
\pgfpathlineto{\pgfqpoint{5.556485in}{1.891220in}}%
\pgfpathlineto{\pgfqpoint{5.609529in}{1.891220in}}%
\pgfpathlineto{\pgfqpoint{5.662573in}{1.891220in}}%
\pgfpathlineto{\pgfqpoint{5.715617in}{1.891220in}}%
\pgfpathlineto{\pgfqpoint{5.768661in}{1.891220in}}%
\pgfpathlineto{\pgfqpoint{5.821705in}{1.891220in}}%
\pgfpathlineto{\pgfqpoint{5.874749in}{1.891220in}}%
\pgfpathlineto{\pgfqpoint{5.927793in}{1.891220in}}%
\pgfpathlineto{\pgfqpoint{5.980837in}{1.891220in}}%
\pgfpathlineto{\pgfqpoint{6.033881in}{1.891220in}}%
\pgfpathlineto{\pgfqpoint{6.086925in}{1.891220in}}%
\pgfpathlineto{\pgfqpoint{6.139969in}{1.891220in}}%
\pgfpathlineto{\pgfqpoint{6.193012in}{1.891220in}}%
\pgfpathlineto{\pgfqpoint{6.246056in}{1.891220in}}%
\pgfpathlineto{\pgfqpoint{6.299100in}{1.891220in}}%
\pgfpathlineto{\pgfqpoint{6.352144in}{1.891220in}}%
\pgfpathlineto{\pgfqpoint{6.405188in}{1.891220in}}%
\pgfpathlineto{\pgfqpoint{6.458232in}{1.891220in}}%
\pgfpathlineto{\pgfqpoint{6.511276in}{1.891220in}}%
\pgfpathlineto{\pgfqpoint{6.564320in}{1.891220in}}%
\pgfpathlineto{\pgfqpoint{6.617364in}{1.891220in}}%
\pgfpathlineto{\pgfqpoint{6.670408in}{1.891220in}}%
\pgfpathlineto{\pgfqpoint{6.723452in}{1.891220in}}%
\pgfpathlineto{\pgfqpoint{6.776496in}{1.891220in}}%
\pgfpathlineto{\pgfqpoint{6.829540in}{1.891220in}}%
\pgfpathlineto{\pgfqpoint{6.882584in}{1.891220in}}%
\pgfpathlineto{\pgfqpoint{6.935628in}{1.891220in}}%
\pgfpathlineto{\pgfqpoint{6.988671in}{1.891220in}}%
\pgfpathlineto{\pgfqpoint{7.041715in}{1.891220in}}%
\pgfpathlineto{\pgfqpoint{7.094759in}{1.891220in}}%
\pgfpathlineto{\pgfqpoint{7.147803in}{1.891220in}}%
\pgfpathlineto{\pgfqpoint{7.200847in}{1.891220in}}%
\pgfpathlineto{\pgfqpoint{7.253891in}{1.891220in}}%
\pgfpathlineto{\pgfqpoint{7.306935in}{1.891220in}}%
\pgfpathlineto{\pgfqpoint{7.359979in}{1.891220in}}%
\pgfpathlineto{\pgfqpoint{7.413023in}{1.891220in}}%
\pgfpathlineto{\pgfqpoint{7.466067in}{1.891220in}}%
\pgfpathlineto{\pgfqpoint{7.519111in}{1.891220in}}%
\pgfpathlineto{\pgfqpoint{7.572155in}{1.891220in}}%
\pgfpathlineto{\pgfqpoint{7.625199in}{1.891220in}}%
\pgfpathlineto{\pgfqpoint{7.678243in}{1.891220in}}%
\pgfpathlineto{\pgfqpoint{7.731287in}{1.891220in}}%
\pgfpathlineto{\pgfqpoint{7.784330in}{1.891220in}}%
\pgfpathlineto{\pgfqpoint{7.837374in}{1.891220in}}%
\pgfpathlineto{\pgfqpoint{7.890418in}{1.891220in}}%
\pgfpathlineto{\pgfqpoint{7.943462in}{1.891220in}}%
\pgfpathlineto{\pgfqpoint{7.996506in}{1.891220in}}%
\pgfpathlineto{\pgfqpoint{8.049550in}{1.891220in}}%
\pgfpathlineto{\pgfqpoint{8.102594in}{1.891220in}}%
\pgfpathlineto{\pgfqpoint{8.155638in}{1.891220in}}%
\pgfpathlineto{\pgfqpoint{8.208682in}{1.891220in}}%
\pgfpathlineto{\pgfqpoint{8.261726in}{1.891220in}}%
\pgfpathlineto{\pgfqpoint{8.314770in}{1.891220in}}%
\pgfpathlineto{\pgfqpoint{8.367814in}{1.891220in}}%
\pgfpathlineto{\pgfqpoint{8.420858in}{1.891220in}}%
\pgfpathlineto{\pgfqpoint{8.473902in}{1.891220in}}%
\pgfpathlineto{\pgfqpoint{8.526946in}{1.891220in}}%
\pgfpathlineto{\pgfqpoint{8.579990in}{1.891220in}}%
\pgfpathlineto{\pgfqpoint{8.633033in}{1.891220in}}%
\pgfpathlineto{\pgfqpoint{8.686077in}{1.891220in}}%
\pgfpathlineto{\pgfqpoint{8.739121in}{1.891220in}}%
\pgfpathlineto{\pgfqpoint{8.792165in}{1.891220in}}%
\pgfpathlineto{\pgfqpoint{8.845209in}{1.891220in}}%
\pgfpathlineto{\pgfqpoint{8.898253in}{1.891220in}}%
\pgfpathlineto{\pgfqpoint{8.951297in}{1.891220in}}%
\pgfpathlineto{\pgfqpoint{9.004341in}{1.891220in}}%
\pgfpathlineto{\pgfqpoint{9.057385in}{1.891220in}}%
\pgfpathlineto{\pgfqpoint{9.110429in}{1.891220in}}%
\pgfpathlineto{\pgfqpoint{9.163473in}{1.891220in}}%
\pgfpathlineto{\pgfqpoint{9.216517in}{1.891220in}}%
\pgfpathlineto{\pgfqpoint{9.269561in}{1.891220in}}%
\pgfpathlineto{\pgfqpoint{9.322605in}{1.891220in}}%
\pgfpathlineto{\pgfqpoint{9.375649in}{1.891220in}}%
\pgfpathlineto{\pgfqpoint{9.428692in}{1.891220in}}%
\pgfpathlineto{\pgfqpoint{9.481736in}{1.891220in}}%
\pgfpathlineto{\pgfqpoint{9.534780in}{1.891220in}}%
\pgfpathlineto{\pgfqpoint{9.587824in}{1.891220in}}%
\pgfpathlineto{\pgfqpoint{9.640868in}{1.891220in}}%
\pgfpathlineto{\pgfqpoint{9.693912in}{1.891220in}}%
\pgfpathlineto{\pgfqpoint{9.746956in}{1.891220in}}%
\pgfpathlineto{\pgfqpoint{9.800000in}{1.891220in}}%
\pgfpathlineto{\pgfqpoint{9.800000in}{2.586892in}}%
\pgfpathlineto{\pgfqpoint{9.800000in}{2.586892in}}%
\pgfpathlineto{\pgfqpoint{9.746956in}{2.586892in}}%
\pgfpathlineto{\pgfqpoint{9.693912in}{2.586892in}}%
\pgfpathlineto{\pgfqpoint{9.640868in}{2.586892in}}%
\pgfpathlineto{\pgfqpoint{9.587824in}{2.586892in}}%
\pgfpathlineto{\pgfqpoint{9.534780in}{2.586892in}}%
\pgfpathlineto{\pgfqpoint{9.481736in}{2.586892in}}%
\pgfpathlineto{\pgfqpoint{9.428692in}{2.586892in}}%
\pgfpathlineto{\pgfqpoint{9.375649in}{2.586892in}}%
\pgfpathlineto{\pgfqpoint{9.322605in}{2.586892in}}%
\pgfpathlineto{\pgfqpoint{9.269561in}{2.586892in}}%
\pgfpathlineto{\pgfqpoint{9.216517in}{2.586892in}}%
\pgfpathlineto{\pgfqpoint{9.163473in}{2.586892in}}%
\pgfpathlineto{\pgfqpoint{9.110429in}{2.586892in}}%
\pgfpathlineto{\pgfqpoint{9.057385in}{2.586892in}}%
\pgfpathlineto{\pgfqpoint{9.004341in}{2.586892in}}%
\pgfpathlineto{\pgfqpoint{8.951297in}{2.586892in}}%
\pgfpathlineto{\pgfqpoint{8.898253in}{2.586892in}}%
\pgfpathlineto{\pgfqpoint{8.845209in}{2.586892in}}%
\pgfpathlineto{\pgfqpoint{8.792165in}{2.586892in}}%
\pgfpathlineto{\pgfqpoint{8.739121in}{2.586892in}}%
\pgfpathlineto{\pgfqpoint{8.686077in}{2.586892in}}%
\pgfpathlineto{\pgfqpoint{8.633033in}{2.586892in}}%
\pgfpathlineto{\pgfqpoint{8.579990in}{2.586892in}}%
\pgfpathlineto{\pgfqpoint{8.526946in}{2.586892in}}%
\pgfpathlineto{\pgfqpoint{8.473902in}{2.586892in}}%
\pgfpathlineto{\pgfqpoint{8.420858in}{2.586892in}}%
\pgfpathlineto{\pgfqpoint{8.367814in}{2.586892in}}%
\pgfpathlineto{\pgfqpoint{8.314770in}{2.586892in}}%
\pgfpathlineto{\pgfqpoint{8.261726in}{2.586892in}}%
\pgfpathlineto{\pgfqpoint{8.208682in}{2.586892in}}%
\pgfpathlineto{\pgfqpoint{8.155638in}{2.586892in}}%
\pgfpathlineto{\pgfqpoint{8.102594in}{2.586892in}}%
\pgfpathlineto{\pgfqpoint{8.049550in}{2.586892in}}%
\pgfpathlineto{\pgfqpoint{7.996506in}{2.586892in}}%
\pgfpathlineto{\pgfqpoint{7.943462in}{2.586892in}}%
\pgfpathlineto{\pgfqpoint{7.890418in}{2.586892in}}%
\pgfpathlineto{\pgfqpoint{7.837374in}{2.586892in}}%
\pgfpathlineto{\pgfqpoint{7.784330in}{2.586892in}}%
\pgfpathlineto{\pgfqpoint{7.731287in}{2.586892in}}%
\pgfpathlineto{\pgfqpoint{7.678243in}{2.586892in}}%
\pgfpathlineto{\pgfqpoint{7.625199in}{2.586892in}}%
\pgfpathlineto{\pgfqpoint{7.572155in}{2.586892in}}%
\pgfpathlineto{\pgfqpoint{7.519111in}{2.586892in}}%
\pgfpathlineto{\pgfqpoint{7.466067in}{2.586892in}}%
\pgfpathlineto{\pgfqpoint{7.413023in}{2.586892in}}%
\pgfpathlineto{\pgfqpoint{7.359979in}{2.586892in}}%
\pgfpathlineto{\pgfqpoint{7.306935in}{2.586892in}}%
\pgfpathlineto{\pgfqpoint{7.253891in}{2.586892in}}%
\pgfpathlineto{\pgfqpoint{7.200847in}{2.586892in}}%
\pgfpathlineto{\pgfqpoint{7.147803in}{2.586892in}}%
\pgfpathlineto{\pgfqpoint{7.094759in}{2.586892in}}%
\pgfpathlineto{\pgfqpoint{7.041715in}{2.586892in}}%
\pgfpathlineto{\pgfqpoint{6.988671in}{2.586892in}}%
\pgfpathlineto{\pgfqpoint{6.935628in}{2.586892in}}%
\pgfpathlineto{\pgfqpoint{6.882584in}{2.586892in}}%
\pgfpathlineto{\pgfqpoint{6.829540in}{2.586892in}}%
\pgfpathlineto{\pgfqpoint{6.776496in}{2.586892in}}%
\pgfpathlineto{\pgfqpoint{6.723452in}{2.586892in}}%
\pgfpathlineto{\pgfqpoint{6.670408in}{2.586892in}}%
\pgfpathlineto{\pgfqpoint{6.617364in}{2.586892in}}%
\pgfpathlineto{\pgfqpoint{6.564320in}{2.586892in}}%
\pgfpathlineto{\pgfqpoint{6.511276in}{2.586892in}}%
\pgfpathlineto{\pgfqpoint{6.458232in}{2.586892in}}%
\pgfpathlineto{\pgfqpoint{6.405188in}{2.586892in}}%
\pgfpathlineto{\pgfqpoint{6.352144in}{2.586892in}}%
\pgfpathlineto{\pgfqpoint{6.299100in}{2.586892in}}%
\pgfpathlineto{\pgfqpoint{6.246056in}{2.586892in}}%
\pgfpathlineto{\pgfqpoint{6.193012in}{2.586892in}}%
\pgfpathlineto{\pgfqpoint{6.139969in}{2.586892in}}%
\pgfpathlineto{\pgfqpoint{6.086925in}{2.586892in}}%
\pgfpathlineto{\pgfqpoint{6.033881in}{2.586892in}}%
\pgfpathlineto{\pgfqpoint{5.980837in}{2.586892in}}%
\pgfpathlineto{\pgfqpoint{5.927793in}{2.586892in}}%
\pgfpathlineto{\pgfqpoint{5.874749in}{2.586892in}}%
\pgfpathlineto{\pgfqpoint{5.821705in}{2.586892in}}%
\pgfpathlineto{\pgfqpoint{5.768661in}{2.586892in}}%
\pgfpathlineto{\pgfqpoint{5.715617in}{2.586892in}}%
\pgfpathlineto{\pgfqpoint{5.662573in}{2.586892in}}%
\pgfpathlineto{\pgfqpoint{5.609529in}{2.586892in}}%
\pgfpathlineto{\pgfqpoint{5.556485in}{2.586892in}}%
\pgfpathlineto{\pgfqpoint{5.503441in}{2.586892in}}%
\pgfpathlineto{\pgfqpoint{5.450397in}{2.586892in}}%
\pgfpathlineto{\pgfqpoint{5.397353in}{2.586892in}}%
\pgfpathlineto{\pgfqpoint{5.344309in}{2.586892in}}%
\pgfpathlineto{\pgfqpoint{5.291266in}{2.586892in}}%
\pgfpathlineto{\pgfqpoint{5.238222in}{2.586892in}}%
\pgfpathlineto{\pgfqpoint{5.185178in}{2.586892in}}%
\pgfpathlineto{\pgfqpoint{5.132134in}{2.586892in}}%
\pgfpathlineto{\pgfqpoint{5.079090in}{2.586892in}}%
\pgfpathlineto{\pgfqpoint{5.026046in}{2.586892in}}%
\pgfpathlineto{\pgfqpoint{4.973002in}{2.586892in}}%
\pgfpathlineto{\pgfqpoint{4.919958in}{2.586892in}}%
\pgfpathlineto{\pgfqpoint{4.866914in}{2.586892in}}%
\pgfpathlineto{\pgfqpoint{4.813870in}{2.586892in}}%
\pgfpathlineto{\pgfqpoint{4.760826in}{2.586892in}}%
\pgfpathlineto{\pgfqpoint{4.707782in}{2.586892in}}%
\pgfpathlineto{\pgfqpoint{4.654738in}{2.586892in}}%
\pgfpathlineto{\pgfqpoint{4.601694in}{2.586892in}}%
\pgfpathlineto{\pgfqpoint{4.548650in}{2.586892in}}%
\pgfpathlineto{\pgfqpoint{4.495607in}{2.586892in}}%
\pgfpathlineto{\pgfqpoint{4.442563in}{2.586892in}}%
\pgfpathlineto{\pgfqpoint{4.389519in}{2.586892in}}%
\pgfpathlineto{\pgfqpoint{4.336475in}{2.586892in}}%
\pgfpathlineto{\pgfqpoint{4.283431in}{2.586892in}}%
\pgfpathlineto{\pgfqpoint{4.230387in}{2.586892in}}%
\pgfpathlineto{\pgfqpoint{4.177343in}{2.586892in}}%
\pgfpathlineto{\pgfqpoint{4.124299in}{2.586892in}}%
\pgfpathlineto{\pgfqpoint{4.071255in}{2.586892in}}%
\pgfpathlineto{\pgfqpoint{4.018211in}{2.586892in}}%
\pgfpathlineto{\pgfqpoint{3.965167in}{2.586892in}}%
\pgfpathlineto{\pgfqpoint{3.912123in}{2.586892in}}%
\pgfpathlineto{\pgfqpoint{3.859079in}{2.586892in}}%
\pgfpathlineto{\pgfqpoint{3.806035in}{2.586892in}}%
\pgfpathlineto{\pgfqpoint{3.752991in}{2.586892in}}%
\pgfpathlineto{\pgfqpoint{3.699948in}{2.586892in}}%
\pgfpathlineto{\pgfqpoint{3.646904in}{2.586892in}}%
\pgfpathlineto{\pgfqpoint{3.593860in}{2.586892in}}%
\pgfpathlineto{\pgfqpoint{3.540816in}{2.586892in}}%
\pgfpathlineto{\pgfqpoint{3.487772in}{2.586892in}}%
\pgfpathlineto{\pgfqpoint{3.434728in}{2.586892in}}%
\pgfpathlineto{\pgfqpoint{3.381684in}{2.586892in}}%
\pgfpathlineto{\pgfqpoint{3.328640in}{2.586892in}}%
\pgfpathlineto{\pgfqpoint{3.275596in}{2.586892in}}%
\pgfpathlineto{\pgfqpoint{3.222552in}{2.586892in}}%
\pgfpathlineto{\pgfqpoint{3.169508in}{2.586892in}}%
\pgfpathlineto{\pgfqpoint{3.116464in}{2.586892in}}%
\pgfpathlineto{\pgfqpoint{3.063420in}{2.586892in}}%
\pgfpathlineto{\pgfqpoint{3.010376in}{2.586892in}}%
\pgfpathlineto{\pgfqpoint{2.957332in}{2.586892in}}%
\pgfpathlineto{\pgfqpoint{2.904288in}{2.586892in}}%
\pgfpathlineto{\pgfqpoint{2.851245in}{2.586892in}}%
\pgfpathlineto{\pgfqpoint{2.798201in}{2.586892in}}%
\pgfpathlineto{\pgfqpoint{2.745157in}{2.586892in}}%
\pgfpathlineto{\pgfqpoint{2.692113in}{2.586892in}}%
\pgfpathlineto{\pgfqpoint{2.639069in}{2.586892in}}%
\pgfpathlineto{\pgfqpoint{2.586025in}{2.586892in}}%
\pgfpathlineto{\pgfqpoint{2.532981in}{2.586892in}}%
\pgfpathlineto{\pgfqpoint{2.479937in}{2.586892in}}%
\pgfpathlineto{\pgfqpoint{2.426893in}{2.586892in}}%
\pgfpathlineto{\pgfqpoint{2.373849in}{2.586892in}}%
\pgfpathlineto{\pgfqpoint{2.320805in}{2.586892in}}%
\pgfpathlineto{\pgfqpoint{2.267761in}{2.586892in}}%
\pgfpathlineto{\pgfqpoint{2.214717in}{2.586892in}}%
\pgfpathlineto{\pgfqpoint{2.161673in}{2.586892in}}%
\pgfpathlineto{\pgfqpoint{2.108629in}{2.586892in}}%
\pgfpathlineto{\pgfqpoint{2.055586in}{2.586892in}}%
\pgfpathlineto{\pgfqpoint{2.002542in}{2.586892in}}%
\pgfpathlineto{\pgfqpoint{1.949498in}{2.586892in}}%
\pgfpathlineto{\pgfqpoint{1.896454in}{2.586892in}}%
\pgfpathlineto{\pgfqpoint{1.843410in}{2.586892in}}%
\pgfpathlineto{\pgfqpoint{1.790366in}{2.586892in}}%
\pgfpathlineto{\pgfqpoint{1.737322in}{2.586892in}}%
\pgfpathlineto{\pgfqpoint{1.684278in}{2.586892in}}%
\pgfpathlineto{\pgfqpoint{1.631234in}{2.586892in}}%
\pgfpathlineto{\pgfqpoint{1.578190in}{2.586892in}}%
\pgfpathlineto{\pgfqpoint{1.525146in}{2.586892in}}%
\pgfpathlineto{\pgfqpoint{1.472102in}{2.586892in}}%
\pgfpathlineto{\pgfqpoint{1.419058in}{2.586892in}}%
\pgfpathlineto{\pgfqpoint{1.366014in}{2.586892in}}%
\pgfpathlineto{\pgfqpoint{1.312970in}{2.586892in}}%
\pgfpathlineto{\pgfqpoint{1.259927in}{2.586892in}}%
\pgfpathlineto{\pgfqpoint{1.206883in}{2.586892in}}%
\pgfpathlineto{\pgfqpoint{1.153839in}{2.586892in}}%
\pgfpathlineto{\pgfqpoint{1.100795in}{2.586892in}}%
\pgfpathlineto{\pgfqpoint{1.047751in}{2.586892in}}%
\pgfpathlineto{\pgfqpoint{0.994707in}{2.586892in}}%
\pgfpathlineto{\pgfqpoint{0.941663in}{2.586892in}}%
\pgfpathlineto{\pgfqpoint{0.941663in}{2.586892in}}%
\pgfpathclose%
\pgfusepath{stroke,fill}%
}%
\begin{pgfscope}%
\pgfsys@transformshift{0.000000in}{0.000000in}%
\pgfsys@useobject{currentmarker}{}%
\end{pgfscope}%
\end{pgfscope}%
\begin{pgfscope}%
\pgfpathrectangle{\pgfqpoint{0.941663in}{0.670138in}}{\pgfqpoint{8.858337in}{3.465625in}}%
\pgfusepath{clip}%
\pgfsetrectcap%
\pgfsetroundjoin%
\pgfsetlinewidth{1.505625pt}%
\definecolor{currentstroke}{rgb}{0.501961,0.000000,0.501961}%
\pgfsetstrokecolor{currentstroke}%
\pgfsetdash{}{0pt}%
\pgfpathmoveto{\pgfqpoint{0.941663in}{2.586892in}}%
\pgfpathlineto{\pgfqpoint{1.206883in}{2.586892in}}%
\pgfpathlineto{\pgfqpoint{1.259927in}{2.749185in}}%
\pgfpathlineto{\pgfqpoint{1.312970in}{2.586892in}}%
\pgfpathlineto{\pgfqpoint{1.578190in}{2.586892in}}%
\pgfpathlineto{\pgfqpoint{1.631234in}{2.802326in}}%
\pgfpathlineto{\pgfqpoint{1.684278in}{2.586892in}}%
\pgfpathlineto{\pgfqpoint{1.790366in}{2.586892in}}%
\pgfpathlineto{\pgfqpoint{1.843410in}{2.934727in}}%
\pgfpathlineto{\pgfqpoint{1.896454in}{2.934727in}}%
\pgfpathlineto{\pgfqpoint{1.949498in}{2.586892in}}%
\pgfpathlineto{\pgfqpoint{2.002542in}{2.934727in}}%
\pgfpathlineto{\pgfqpoint{2.055586in}{2.586892in}}%
\pgfpathlineto{\pgfqpoint{2.161673in}{2.586892in}}%
\pgfpathlineto{\pgfqpoint{2.214717in}{2.910735in}}%
\pgfpathlineto{\pgfqpoint{2.267761in}{2.586892in}}%
\pgfpathlineto{\pgfqpoint{2.320805in}{2.586892in}}%
\pgfpathlineto{\pgfqpoint{2.373849in}{2.844885in}}%
\pgfpathlineto{\pgfqpoint{2.426893in}{2.934727in}}%
\pgfpathlineto{\pgfqpoint{2.479937in}{2.586892in}}%
\pgfpathlineto{\pgfqpoint{2.904288in}{2.586892in}}%
\pgfpathlineto{\pgfqpoint{2.957332in}{2.934727in}}%
\pgfpathlineto{\pgfqpoint{3.010376in}{2.586892in}}%
\pgfpathlineto{\pgfqpoint{3.063420in}{2.909016in}}%
\pgfpathlineto{\pgfqpoint{3.116464in}{2.586892in}}%
\pgfpathlineto{\pgfqpoint{3.434728in}{2.586892in}}%
\pgfpathlineto{\pgfqpoint{3.487772in}{2.710315in}}%
\pgfpathlineto{\pgfqpoint{3.540816in}{2.908402in}}%
\pgfpathlineto{\pgfqpoint{3.593860in}{2.586892in}}%
\pgfpathlineto{\pgfqpoint{4.283431in}{2.586892in}}%
\pgfpathlineto{\pgfqpoint{4.336475in}{2.934727in}}%
\pgfpathlineto{\pgfqpoint{4.389519in}{2.586892in}}%
\pgfpathlineto{\pgfqpoint{4.442563in}{2.586892in}}%
\pgfpathlineto{\pgfqpoint{4.495607in}{2.934727in}}%
\pgfpathlineto{\pgfqpoint{4.548650in}{2.586892in}}%
\pgfpathlineto{\pgfqpoint{4.601694in}{2.927863in}}%
\pgfpathlineto{\pgfqpoint{4.654738in}{2.586892in}}%
\pgfpathlineto{\pgfqpoint{4.707782in}{2.934727in}}%
\pgfpathlineto{\pgfqpoint{4.760826in}{2.586892in}}%
\pgfpathlineto{\pgfqpoint{4.813870in}{2.586892in}}%
\pgfpathlineto{\pgfqpoint{4.866914in}{2.884645in}}%
\pgfpathlineto{\pgfqpoint{4.919958in}{2.863950in}}%
\pgfpathlineto{\pgfqpoint{4.973002in}{2.586892in}}%
\pgfpathlineto{\pgfqpoint{5.026046in}{2.758322in}}%
\pgfpathlineto{\pgfqpoint{5.079090in}{2.586892in}}%
\pgfpathlineto{\pgfqpoint{5.132134in}{2.824147in}}%
\pgfpathlineto{\pgfqpoint{5.185178in}{2.910114in}}%
\pgfpathlineto{\pgfqpoint{5.238222in}{2.837414in}}%
\pgfpathlineto{\pgfqpoint{5.291266in}{2.586892in}}%
\pgfpathlineto{\pgfqpoint{5.344309in}{2.705202in}}%
\pgfpathlineto{\pgfqpoint{5.397353in}{2.660769in}}%
\pgfpathlineto{\pgfqpoint{5.450397in}{2.934727in}}%
\pgfpathlineto{\pgfqpoint{5.503441in}{2.586892in}}%
\pgfpathlineto{\pgfqpoint{5.556485in}{2.586892in}}%
\pgfpathlineto{\pgfqpoint{5.609529in}{2.850209in}}%
\pgfpathlineto{\pgfqpoint{5.662573in}{2.711726in}}%
\pgfpathlineto{\pgfqpoint{5.715617in}{2.586892in}}%
\pgfpathlineto{\pgfqpoint{5.768661in}{2.934727in}}%
\pgfpathlineto{\pgfqpoint{5.874749in}{2.934727in}}%
\pgfpathlineto{\pgfqpoint{5.927793in}{2.643205in}}%
\pgfpathlineto{\pgfqpoint{5.980837in}{2.586892in}}%
\pgfpathlineto{\pgfqpoint{6.139969in}{2.586892in}}%
\pgfpathlineto{\pgfqpoint{6.193012in}{2.934727in}}%
\pgfpathlineto{\pgfqpoint{6.246056in}{2.934727in}}%
\pgfpathlineto{\pgfqpoint{6.299100in}{2.586892in}}%
\pgfpathlineto{\pgfqpoint{6.352144in}{2.586892in}}%
\pgfpathlineto{\pgfqpoint{6.405188in}{2.699818in}}%
\pgfpathlineto{\pgfqpoint{6.458232in}{2.586892in}}%
\pgfpathlineto{\pgfqpoint{6.829540in}{2.586892in}}%
\pgfpathlineto{\pgfqpoint{6.882584in}{2.934727in}}%
\pgfpathlineto{\pgfqpoint{6.935628in}{2.586892in}}%
\pgfpathlineto{\pgfqpoint{7.147803in}{2.586892in}}%
\pgfpathlineto{\pgfqpoint{7.200847in}{2.934727in}}%
\pgfpathlineto{\pgfqpoint{7.253891in}{2.586892in}}%
\pgfpathlineto{\pgfqpoint{7.306935in}{2.586892in}}%
\pgfpathlineto{\pgfqpoint{7.359979in}{2.806406in}}%
\pgfpathlineto{\pgfqpoint{7.413023in}{2.865454in}}%
\pgfpathlineto{\pgfqpoint{7.466067in}{2.586892in}}%
\pgfpathlineto{\pgfqpoint{7.519111in}{2.586892in}}%
\pgfpathlineto{\pgfqpoint{7.572155in}{2.823725in}}%
\pgfpathlineto{\pgfqpoint{7.625199in}{2.934727in}}%
\pgfpathlineto{\pgfqpoint{7.678243in}{2.586892in}}%
\pgfpathlineto{\pgfqpoint{7.996506in}{2.586892in}}%
\pgfpathlineto{\pgfqpoint{8.049550in}{2.934727in}}%
\pgfpathlineto{\pgfqpoint{8.102594in}{2.934727in}}%
\pgfpathlineto{\pgfqpoint{8.155638in}{2.586892in}}%
\pgfpathlineto{\pgfqpoint{8.208682in}{2.586892in}}%
\pgfpathlineto{\pgfqpoint{8.261726in}{2.934727in}}%
\pgfpathlineto{\pgfqpoint{8.314770in}{2.934727in}}%
\pgfpathlineto{\pgfqpoint{8.367814in}{2.586892in}}%
\pgfpathlineto{\pgfqpoint{8.420858in}{2.934727in}}%
\pgfpathlineto{\pgfqpoint{8.473902in}{2.586892in}}%
\pgfpathlineto{\pgfqpoint{8.526946in}{2.586892in}}%
\pgfpathlineto{\pgfqpoint{8.579990in}{2.934727in}}%
\pgfpathlineto{\pgfqpoint{8.633033in}{2.856358in}}%
\pgfpathlineto{\pgfqpoint{8.686077in}{2.934727in}}%
\pgfpathlineto{\pgfqpoint{8.739121in}{2.825404in}}%
\pgfpathlineto{\pgfqpoint{8.792165in}{2.586892in}}%
\pgfpathlineto{\pgfqpoint{8.845209in}{2.634872in}}%
\pgfpathlineto{\pgfqpoint{8.898253in}{2.689958in}}%
\pgfpathlineto{\pgfqpoint{8.951297in}{2.586892in}}%
\pgfpathlineto{\pgfqpoint{9.110429in}{2.586892in}}%
\pgfpathlineto{\pgfqpoint{9.163473in}{2.934727in}}%
\pgfpathlineto{\pgfqpoint{9.322605in}{2.934727in}}%
\pgfpathlineto{\pgfqpoint{9.375649in}{2.695480in}}%
\pgfpathlineto{\pgfqpoint{9.428692in}{2.586892in}}%
\pgfpathlineto{\pgfqpoint{9.481736in}{2.586892in}}%
\pgfpathlineto{\pgfqpoint{9.534780in}{2.934727in}}%
\pgfpathlineto{\pgfqpoint{9.587824in}{2.916055in}}%
\pgfpathlineto{\pgfqpoint{9.640868in}{2.586892in}}%
\pgfpathlineto{\pgfqpoint{9.693912in}{2.865140in}}%
\pgfpathlineto{\pgfqpoint{9.746956in}{2.608608in}}%
\pgfpathlineto{\pgfqpoint{9.800000in}{2.588586in}}%
\pgfpathlineto{\pgfqpoint{9.800000in}{2.588586in}}%
\pgfusepath{stroke}%
\end{pgfscope}%
\begin{pgfscope}%
\pgfpathrectangle{\pgfqpoint{0.941663in}{0.670138in}}{\pgfqpoint{8.858337in}{3.465625in}}%
\pgfusepath{clip}%
\pgfsetbuttcap%
\pgfsetroundjoin%
\definecolor{currentfill}{rgb}{0.501961,0.000000,0.501961}%
\pgfsetfillcolor{currentfill}%
\pgfsetlinewidth{1.003750pt}%
\definecolor{currentstroke}{rgb}{0.501961,0.000000,0.501961}%
\pgfsetstrokecolor{currentstroke}%
\pgfsetdash{}{0pt}%
\pgfsys@defobject{currentmarker}{\pgfqpoint{0.941663in}{2.586892in}}{\pgfqpoint{9.800000in}{2.934727in}}{%
\pgfpathmoveto{\pgfqpoint{0.941663in}{2.586892in}}%
\pgfpathlineto{\pgfqpoint{0.941663in}{2.586892in}}%
\pgfpathlineto{\pgfqpoint{0.994707in}{2.586892in}}%
\pgfpathlineto{\pgfqpoint{1.047751in}{2.586892in}}%
\pgfpathlineto{\pgfqpoint{1.100795in}{2.586892in}}%
\pgfpathlineto{\pgfqpoint{1.153839in}{2.586892in}}%
\pgfpathlineto{\pgfqpoint{1.206883in}{2.586892in}}%
\pgfpathlineto{\pgfqpoint{1.259927in}{2.586892in}}%
\pgfpathlineto{\pgfqpoint{1.312970in}{2.586892in}}%
\pgfpathlineto{\pgfqpoint{1.366014in}{2.586892in}}%
\pgfpathlineto{\pgfqpoint{1.419058in}{2.586892in}}%
\pgfpathlineto{\pgfqpoint{1.472102in}{2.586892in}}%
\pgfpathlineto{\pgfqpoint{1.525146in}{2.586892in}}%
\pgfpathlineto{\pgfqpoint{1.578190in}{2.586892in}}%
\pgfpathlineto{\pgfqpoint{1.631234in}{2.586892in}}%
\pgfpathlineto{\pgfqpoint{1.684278in}{2.586892in}}%
\pgfpathlineto{\pgfqpoint{1.737322in}{2.586892in}}%
\pgfpathlineto{\pgfqpoint{1.790366in}{2.586892in}}%
\pgfpathlineto{\pgfqpoint{1.843410in}{2.586892in}}%
\pgfpathlineto{\pgfqpoint{1.896454in}{2.586892in}}%
\pgfpathlineto{\pgfqpoint{1.949498in}{2.586892in}}%
\pgfpathlineto{\pgfqpoint{2.002542in}{2.586892in}}%
\pgfpathlineto{\pgfqpoint{2.055586in}{2.586892in}}%
\pgfpathlineto{\pgfqpoint{2.108629in}{2.586892in}}%
\pgfpathlineto{\pgfqpoint{2.161673in}{2.586892in}}%
\pgfpathlineto{\pgfqpoint{2.214717in}{2.586892in}}%
\pgfpathlineto{\pgfqpoint{2.267761in}{2.586892in}}%
\pgfpathlineto{\pgfqpoint{2.320805in}{2.586892in}}%
\pgfpathlineto{\pgfqpoint{2.373849in}{2.586892in}}%
\pgfpathlineto{\pgfqpoint{2.426893in}{2.586892in}}%
\pgfpathlineto{\pgfqpoint{2.479937in}{2.586892in}}%
\pgfpathlineto{\pgfqpoint{2.532981in}{2.586892in}}%
\pgfpathlineto{\pgfqpoint{2.586025in}{2.586892in}}%
\pgfpathlineto{\pgfqpoint{2.639069in}{2.586892in}}%
\pgfpathlineto{\pgfqpoint{2.692113in}{2.586892in}}%
\pgfpathlineto{\pgfqpoint{2.745157in}{2.586892in}}%
\pgfpathlineto{\pgfqpoint{2.798201in}{2.586892in}}%
\pgfpathlineto{\pgfqpoint{2.851245in}{2.586892in}}%
\pgfpathlineto{\pgfqpoint{2.904288in}{2.586892in}}%
\pgfpathlineto{\pgfqpoint{2.957332in}{2.586892in}}%
\pgfpathlineto{\pgfqpoint{3.010376in}{2.586892in}}%
\pgfpathlineto{\pgfqpoint{3.063420in}{2.586892in}}%
\pgfpathlineto{\pgfqpoint{3.116464in}{2.586892in}}%
\pgfpathlineto{\pgfqpoint{3.169508in}{2.586892in}}%
\pgfpathlineto{\pgfqpoint{3.222552in}{2.586892in}}%
\pgfpathlineto{\pgfqpoint{3.275596in}{2.586892in}}%
\pgfpathlineto{\pgfqpoint{3.328640in}{2.586892in}}%
\pgfpathlineto{\pgfqpoint{3.381684in}{2.586892in}}%
\pgfpathlineto{\pgfqpoint{3.434728in}{2.586892in}}%
\pgfpathlineto{\pgfqpoint{3.487772in}{2.586892in}}%
\pgfpathlineto{\pgfqpoint{3.540816in}{2.586892in}}%
\pgfpathlineto{\pgfqpoint{3.593860in}{2.586892in}}%
\pgfpathlineto{\pgfqpoint{3.646904in}{2.586892in}}%
\pgfpathlineto{\pgfqpoint{3.699948in}{2.586892in}}%
\pgfpathlineto{\pgfqpoint{3.752991in}{2.586892in}}%
\pgfpathlineto{\pgfqpoint{3.806035in}{2.586892in}}%
\pgfpathlineto{\pgfqpoint{3.859079in}{2.586892in}}%
\pgfpathlineto{\pgfqpoint{3.912123in}{2.586892in}}%
\pgfpathlineto{\pgfqpoint{3.965167in}{2.586892in}}%
\pgfpathlineto{\pgfqpoint{4.018211in}{2.586892in}}%
\pgfpathlineto{\pgfqpoint{4.071255in}{2.586892in}}%
\pgfpathlineto{\pgfqpoint{4.124299in}{2.586892in}}%
\pgfpathlineto{\pgfqpoint{4.177343in}{2.586892in}}%
\pgfpathlineto{\pgfqpoint{4.230387in}{2.586892in}}%
\pgfpathlineto{\pgfqpoint{4.283431in}{2.586892in}}%
\pgfpathlineto{\pgfqpoint{4.336475in}{2.586892in}}%
\pgfpathlineto{\pgfqpoint{4.389519in}{2.586892in}}%
\pgfpathlineto{\pgfqpoint{4.442563in}{2.586892in}}%
\pgfpathlineto{\pgfqpoint{4.495607in}{2.586892in}}%
\pgfpathlineto{\pgfqpoint{4.548650in}{2.586892in}}%
\pgfpathlineto{\pgfqpoint{4.601694in}{2.586892in}}%
\pgfpathlineto{\pgfqpoint{4.654738in}{2.586892in}}%
\pgfpathlineto{\pgfqpoint{4.707782in}{2.586892in}}%
\pgfpathlineto{\pgfqpoint{4.760826in}{2.586892in}}%
\pgfpathlineto{\pgfqpoint{4.813870in}{2.586892in}}%
\pgfpathlineto{\pgfqpoint{4.866914in}{2.586892in}}%
\pgfpathlineto{\pgfqpoint{4.919958in}{2.586892in}}%
\pgfpathlineto{\pgfqpoint{4.973002in}{2.586892in}}%
\pgfpathlineto{\pgfqpoint{5.026046in}{2.586892in}}%
\pgfpathlineto{\pgfqpoint{5.079090in}{2.586892in}}%
\pgfpathlineto{\pgfqpoint{5.132134in}{2.586892in}}%
\pgfpathlineto{\pgfqpoint{5.185178in}{2.586892in}}%
\pgfpathlineto{\pgfqpoint{5.238222in}{2.586892in}}%
\pgfpathlineto{\pgfqpoint{5.291266in}{2.586892in}}%
\pgfpathlineto{\pgfqpoint{5.344309in}{2.586892in}}%
\pgfpathlineto{\pgfqpoint{5.397353in}{2.586892in}}%
\pgfpathlineto{\pgfqpoint{5.450397in}{2.586892in}}%
\pgfpathlineto{\pgfqpoint{5.503441in}{2.586892in}}%
\pgfpathlineto{\pgfqpoint{5.556485in}{2.586892in}}%
\pgfpathlineto{\pgfqpoint{5.609529in}{2.586892in}}%
\pgfpathlineto{\pgfqpoint{5.662573in}{2.586892in}}%
\pgfpathlineto{\pgfqpoint{5.715617in}{2.586892in}}%
\pgfpathlineto{\pgfqpoint{5.768661in}{2.586892in}}%
\pgfpathlineto{\pgfqpoint{5.821705in}{2.586892in}}%
\pgfpathlineto{\pgfqpoint{5.874749in}{2.586892in}}%
\pgfpathlineto{\pgfqpoint{5.927793in}{2.586892in}}%
\pgfpathlineto{\pgfqpoint{5.980837in}{2.586892in}}%
\pgfpathlineto{\pgfqpoint{6.033881in}{2.586892in}}%
\pgfpathlineto{\pgfqpoint{6.086925in}{2.586892in}}%
\pgfpathlineto{\pgfqpoint{6.139969in}{2.586892in}}%
\pgfpathlineto{\pgfqpoint{6.193012in}{2.586892in}}%
\pgfpathlineto{\pgfqpoint{6.246056in}{2.586892in}}%
\pgfpathlineto{\pgfqpoint{6.299100in}{2.586892in}}%
\pgfpathlineto{\pgfqpoint{6.352144in}{2.586892in}}%
\pgfpathlineto{\pgfqpoint{6.405188in}{2.586892in}}%
\pgfpathlineto{\pgfqpoint{6.458232in}{2.586892in}}%
\pgfpathlineto{\pgfqpoint{6.511276in}{2.586892in}}%
\pgfpathlineto{\pgfqpoint{6.564320in}{2.586892in}}%
\pgfpathlineto{\pgfqpoint{6.617364in}{2.586892in}}%
\pgfpathlineto{\pgfqpoint{6.670408in}{2.586892in}}%
\pgfpathlineto{\pgfqpoint{6.723452in}{2.586892in}}%
\pgfpathlineto{\pgfqpoint{6.776496in}{2.586892in}}%
\pgfpathlineto{\pgfqpoint{6.829540in}{2.586892in}}%
\pgfpathlineto{\pgfqpoint{6.882584in}{2.586892in}}%
\pgfpathlineto{\pgfqpoint{6.935628in}{2.586892in}}%
\pgfpathlineto{\pgfqpoint{6.988671in}{2.586892in}}%
\pgfpathlineto{\pgfqpoint{7.041715in}{2.586892in}}%
\pgfpathlineto{\pgfqpoint{7.094759in}{2.586892in}}%
\pgfpathlineto{\pgfqpoint{7.147803in}{2.586892in}}%
\pgfpathlineto{\pgfqpoint{7.200847in}{2.586892in}}%
\pgfpathlineto{\pgfqpoint{7.253891in}{2.586892in}}%
\pgfpathlineto{\pgfqpoint{7.306935in}{2.586892in}}%
\pgfpathlineto{\pgfqpoint{7.359979in}{2.586892in}}%
\pgfpathlineto{\pgfqpoint{7.413023in}{2.586892in}}%
\pgfpathlineto{\pgfqpoint{7.466067in}{2.586892in}}%
\pgfpathlineto{\pgfqpoint{7.519111in}{2.586892in}}%
\pgfpathlineto{\pgfqpoint{7.572155in}{2.586892in}}%
\pgfpathlineto{\pgfqpoint{7.625199in}{2.586892in}}%
\pgfpathlineto{\pgfqpoint{7.678243in}{2.586892in}}%
\pgfpathlineto{\pgfqpoint{7.731287in}{2.586892in}}%
\pgfpathlineto{\pgfqpoint{7.784330in}{2.586892in}}%
\pgfpathlineto{\pgfqpoint{7.837374in}{2.586892in}}%
\pgfpathlineto{\pgfqpoint{7.890418in}{2.586892in}}%
\pgfpathlineto{\pgfqpoint{7.943462in}{2.586892in}}%
\pgfpathlineto{\pgfqpoint{7.996506in}{2.586892in}}%
\pgfpathlineto{\pgfqpoint{8.049550in}{2.586892in}}%
\pgfpathlineto{\pgfqpoint{8.102594in}{2.586892in}}%
\pgfpathlineto{\pgfqpoint{8.155638in}{2.586892in}}%
\pgfpathlineto{\pgfqpoint{8.208682in}{2.586892in}}%
\pgfpathlineto{\pgfqpoint{8.261726in}{2.586892in}}%
\pgfpathlineto{\pgfqpoint{8.314770in}{2.586892in}}%
\pgfpathlineto{\pgfqpoint{8.367814in}{2.586892in}}%
\pgfpathlineto{\pgfqpoint{8.420858in}{2.586892in}}%
\pgfpathlineto{\pgfqpoint{8.473902in}{2.586892in}}%
\pgfpathlineto{\pgfqpoint{8.526946in}{2.586892in}}%
\pgfpathlineto{\pgfqpoint{8.579990in}{2.586892in}}%
\pgfpathlineto{\pgfqpoint{8.633033in}{2.586892in}}%
\pgfpathlineto{\pgfqpoint{8.686077in}{2.586892in}}%
\pgfpathlineto{\pgfqpoint{8.739121in}{2.586892in}}%
\pgfpathlineto{\pgfqpoint{8.792165in}{2.586892in}}%
\pgfpathlineto{\pgfqpoint{8.845209in}{2.586892in}}%
\pgfpathlineto{\pgfqpoint{8.898253in}{2.586892in}}%
\pgfpathlineto{\pgfqpoint{8.951297in}{2.586892in}}%
\pgfpathlineto{\pgfqpoint{9.004341in}{2.586892in}}%
\pgfpathlineto{\pgfqpoint{9.057385in}{2.586892in}}%
\pgfpathlineto{\pgfqpoint{9.110429in}{2.586892in}}%
\pgfpathlineto{\pgfqpoint{9.163473in}{2.586892in}}%
\pgfpathlineto{\pgfqpoint{9.216517in}{2.586892in}}%
\pgfpathlineto{\pgfqpoint{9.269561in}{2.586892in}}%
\pgfpathlineto{\pgfqpoint{9.322605in}{2.586892in}}%
\pgfpathlineto{\pgfqpoint{9.375649in}{2.586892in}}%
\pgfpathlineto{\pgfqpoint{9.428692in}{2.586892in}}%
\pgfpathlineto{\pgfqpoint{9.481736in}{2.586892in}}%
\pgfpathlineto{\pgfqpoint{9.534780in}{2.586892in}}%
\pgfpathlineto{\pgfqpoint{9.587824in}{2.586892in}}%
\pgfpathlineto{\pgfqpoint{9.640868in}{2.586892in}}%
\pgfpathlineto{\pgfqpoint{9.693912in}{2.586892in}}%
\pgfpathlineto{\pgfqpoint{9.746956in}{2.586892in}}%
\pgfpathlineto{\pgfqpoint{9.800000in}{2.586892in}}%
\pgfpathlineto{\pgfqpoint{9.800000in}{2.588586in}}%
\pgfpathlineto{\pgfqpoint{9.800000in}{2.588586in}}%
\pgfpathlineto{\pgfqpoint{9.746956in}{2.608608in}}%
\pgfpathlineto{\pgfqpoint{9.693912in}{2.865140in}}%
\pgfpathlineto{\pgfqpoint{9.640868in}{2.586892in}}%
\pgfpathlineto{\pgfqpoint{9.587824in}{2.916055in}}%
\pgfpathlineto{\pgfqpoint{9.534780in}{2.934727in}}%
\pgfpathlineto{\pgfqpoint{9.481736in}{2.586892in}}%
\pgfpathlineto{\pgfqpoint{9.428692in}{2.586892in}}%
\pgfpathlineto{\pgfqpoint{9.375649in}{2.695480in}}%
\pgfpathlineto{\pgfqpoint{9.322605in}{2.934727in}}%
\pgfpathlineto{\pgfqpoint{9.269561in}{2.934727in}}%
\pgfpathlineto{\pgfqpoint{9.216517in}{2.934727in}}%
\pgfpathlineto{\pgfqpoint{9.163473in}{2.934727in}}%
\pgfpathlineto{\pgfqpoint{9.110429in}{2.586892in}}%
\pgfpathlineto{\pgfqpoint{9.057385in}{2.586892in}}%
\pgfpathlineto{\pgfqpoint{9.004341in}{2.586892in}}%
\pgfpathlineto{\pgfqpoint{8.951297in}{2.586892in}}%
\pgfpathlineto{\pgfqpoint{8.898253in}{2.689958in}}%
\pgfpathlineto{\pgfqpoint{8.845209in}{2.634872in}}%
\pgfpathlineto{\pgfqpoint{8.792165in}{2.586892in}}%
\pgfpathlineto{\pgfqpoint{8.739121in}{2.825404in}}%
\pgfpathlineto{\pgfqpoint{8.686077in}{2.934727in}}%
\pgfpathlineto{\pgfqpoint{8.633033in}{2.856358in}}%
\pgfpathlineto{\pgfqpoint{8.579990in}{2.934727in}}%
\pgfpathlineto{\pgfqpoint{8.526946in}{2.586892in}}%
\pgfpathlineto{\pgfqpoint{8.473902in}{2.586892in}}%
\pgfpathlineto{\pgfqpoint{8.420858in}{2.934727in}}%
\pgfpathlineto{\pgfqpoint{8.367814in}{2.586892in}}%
\pgfpathlineto{\pgfqpoint{8.314770in}{2.934727in}}%
\pgfpathlineto{\pgfqpoint{8.261726in}{2.934727in}}%
\pgfpathlineto{\pgfqpoint{8.208682in}{2.586892in}}%
\pgfpathlineto{\pgfqpoint{8.155638in}{2.586892in}}%
\pgfpathlineto{\pgfqpoint{8.102594in}{2.934727in}}%
\pgfpathlineto{\pgfqpoint{8.049550in}{2.934727in}}%
\pgfpathlineto{\pgfqpoint{7.996506in}{2.586892in}}%
\pgfpathlineto{\pgfqpoint{7.943462in}{2.586892in}}%
\pgfpathlineto{\pgfqpoint{7.890418in}{2.586892in}}%
\pgfpathlineto{\pgfqpoint{7.837374in}{2.586892in}}%
\pgfpathlineto{\pgfqpoint{7.784330in}{2.586892in}}%
\pgfpathlineto{\pgfqpoint{7.731287in}{2.586892in}}%
\pgfpathlineto{\pgfqpoint{7.678243in}{2.586892in}}%
\pgfpathlineto{\pgfqpoint{7.625199in}{2.934727in}}%
\pgfpathlineto{\pgfqpoint{7.572155in}{2.823725in}}%
\pgfpathlineto{\pgfqpoint{7.519111in}{2.586892in}}%
\pgfpathlineto{\pgfqpoint{7.466067in}{2.586892in}}%
\pgfpathlineto{\pgfqpoint{7.413023in}{2.865454in}}%
\pgfpathlineto{\pgfqpoint{7.359979in}{2.806406in}}%
\pgfpathlineto{\pgfqpoint{7.306935in}{2.586892in}}%
\pgfpathlineto{\pgfqpoint{7.253891in}{2.586892in}}%
\pgfpathlineto{\pgfqpoint{7.200847in}{2.934727in}}%
\pgfpathlineto{\pgfqpoint{7.147803in}{2.586892in}}%
\pgfpathlineto{\pgfqpoint{7.094759in}{2.586892in}}%
\pgfpathlineto{\pgfqpoint{7.041715in}{2.586892in}}%
\pgfpathlineto{\pgfqpoint{6.988671in}{2.586892in}}%
\pgfpathlineto{\pgfqpoint{6.935628in}{2.586892in}}%
\pgfpathlineto{\pgfqpoint{6.882584in}{2.934727in}}%
\pgfpathlineto{\pgfqpoint{6.829540in}{2.586892in}}%
\pgfpathlineto{\pgfqpoint{6.776496in}{2.586892in}}%
\pgfpathlineto{\pgfqpoint{6.723452in}{2.586892in}}%
\pgfpathlineto{\pgfqpoint{6.670408in}{2.586892in}}%
\pgfpathlineto{\pgfqpoint{6.617364in}{2.586892in}}%
\pgfpathlineto{\pgfqpoint{6.564320in}{2.586892in}}%
\pgfpathlineto{\pgfqpoint{6.511276in}{2.586892in}}%
\pgfpathlineto{\pgfqpoint{6.458232in}{2.586892in}}%
\pgfpathlineto{\pgfqpoint{6.405188in}{2.699818in}}%
\pgfpathlineto{\pgfqpoint{6.352144in}{2.586892in}}%
\pgfpathlineto{\pgfqpoint{6.299100in}{2.586892in}}%
\pgfpathlineto{\pgfqpoint{6.246056in}{2.934727in}}%
\pgfpathlineto{\pgfqpoint{6.193012in}{2.934727in}}%
\pgfpathlineto{\pgfqpoint{6.139969in}{2.586892in}}%
\pgfpathlineto{\pgfqpoint{6.086925in}{2.586892in}}%
\pgfpathlineto{\pgfqpoint{6.033881in}{2.586892in}}%
\pgfpathlineto{\pgfqpoint{5.980837in}{2.586892in}}%
\pgfpathlineto{\pgfqpoint{5.927793in}{2.643205in}}%
\pgfpathlineto{\pgfqpoint{5.874749in}{2.934727in}}%
\pgfpathlineto{\pgfqpoint{5.821705in}{2.934727in}}%
\pgfpathlineto{\pgfqpoint{5.768661in}{2.934727in}}%
\pgfpathlineto{\pgfqpoint{5.715617in}{2.586892in}}%
\pgfpathlineto{\pgfqpoint{5.662573in}{2.711726in}}%
\pgfpathlineto{\pgfqpoint{5.609529in}{2.850209in}}%
\pgfpathlineto{\pgfqpoint{5.556485in}{2.586892in}}%
\pgfpathlineto{\pgfqpoint{5.503441in}{2.586892in}}%
\pgfpathlineto{\pgfqpoint{5.450397in}{2.934727in}}%
\pgfpathlineto{\pgfqpoint{5.397353in}{2.660769in}}%
\pgfpathlineto{\pgfqpoint{5.344309in}{2.705202in}}%
\pgfpathlineto{\pgfqpoint{5.291266in}{2.586892in}}%
\pgfpathlineto{\pgfqpoint{5.238222in}{2.837414in}}%
\pgfpathlineto{\pgfqpoint{5.185178in}{2.910114in}}%
\pgfpathlineto{\pgfqpoint{5.132134in}{2.824147in}}%
\pgfpathlineto{\pgfqpoint{5.079090in}{2.586892in}}%
\pgfpathlineto{\pgfqpoint{5.026046in}{2.758322in}}%
\pgfpathlineto{\pgfqpoint{4.973002in}{2.586892in}}%
\pgfpathlineto{\pgfqpoint{4.919958in}{2.863950in}}%
\pgfpathlineto{\pgfqpoint{4.866914in}{2.884645in}}%
\pgfpathlineto{\pgfqpoint{4.813870in}{2.586892in}}%
\pgfpathlineto{\pgfqpoint{4.760826in}{2.586892in}}%
\pgfpathlineto{\pgfqpoint{4.707782in}{2.934727in}}%
\pgfpathlineto{\pgfqpoint{4.654738in}{2.586892in}}%
\pgfpathlineto{\pgfqpoint{4.601694in}{2.927863in}}%
\pgfpathlineto{\pgfqpoint{4.548650in}{2.586892in}}%
\pgfpathlineto{\pgfqpoint{4.495607in}{2.934727in}}%
\pgfpathlineto{\pgfqpoint{4.442563in}{2.586892in}}%
\pgfpathlineto{\pgfqpoint{4.389519in}{2.586892in}}%
\pgfpathlineto{\pgfqpoint{4.336475in}{2.934727in}}%
\pgfpathlineto{\pgfqpoint{4.283431in}{2.586892in}}%
\pgfpathlineto{\pgfqpoint{4.230387in}{2.586892in}}%
\pgfpathlineto{\pgfqpoint{4.177343in}{2.586892in}}%
\pgfpathlineto{\pgfqpoint{4.124299in}{2.586892in}}%
\pgfpathlineto{\pgfqpoint{4.071255in}{2.586892in}}%
\pgfpathlineto{\pgfqpoint{4.018211in}{2.586892in}}%
\pgfpathlineto{\pgfqpoint{3.965167in}{2.586892in}}%
\pgfpathlineto{\pgfqpoint{3.912123in}{2.586892in}}%
\pgfpathlineto{\pgfqpoint{3.859079in}{2.586892in}}%
\pgfpathlineto{\pgfqpoint{3.806035in}{2.586892in}}%
\pgfpathlineto{\pgfqpoint{3.752991in}{2.586892in}}%
\pgfpathlineto{\pgfqpoint{3.699948in}{2.586892in}}%
\pgfpathlineto{\pgfqpoint{3.646904in}{2.586892in}}%
\pgfpathlineto{\pgfqpoint{3.593860in}{2.586892in}}%
\pgfpathlineto{\pgfqpoint{3.540816in}{2.908402in}}%
\pgfpathlineto{\pgfqpoint{3.487772in}{2.710315in}}%
\pgfpathlineto{\pgfqpoint{3.434728in}{2.586892in}}%
\pgfpathlineto{\pgfqpoint{3.381684in}{2.586892in}}%
\pgfpathlineto{\pgfqpoint{3.328640in}{2.586892in}}%
\pgfpathlineto{\pgfqpoint{3.275596in}{2.586892in}}%
\pgfpathlineto{\pgfqpoint{3.222552in}{2.586892in}}%
\pgfpathlineto{\pgfqpoint{3.169508in}{2.586892in}}%
\pgfpathlineto{\pgfqpoint{3.116464in}{2.586892in}}%
\pgfpathlineto{\pgfqpoint{3.063420in}{2.909016in}}%
\pgfpathlineto{\pgfqpoint{3.010376in}{2.586892in}}%
\pgfpathlineto{\pgfqpoint{2.957332in}{2.934727in}}%
\pgfpathlineto{\pgfqpoint{2.904288in}{2.586892in}}%
\pgfpathlineto{\pgfqpoint{2.851245in}{2.586892in}}%
\pgfpathlineto{\pgfqpoint{2.798201in}{2.586892in}}%
\pgfpathlineto{\pgfqpoint{2.745157in}{2.586892in}}%
\pgfpathlineto{\pgfqpoint{2.692113in}{2.586892in}}%
\pgfpathlineto{\pgfqpoint{2.639069in}{2.586892in}}%
\pgfpathlineto{\pgfqpoint{2.586025in}{2.586892in}}%
\pgfpathlineto{\pgfqpoint{2.532981in}{2.586892in}}%
\pgfpathlineto{\pgfqpoint{2.479937in}{2.586892in}}%
\pgfpathlineto{\pgfqpoint{2.426893in}{2.934727in}}%
\pgfpathlineto{\pgfqpoint{2.373849in}{2.844885in}}%
\pgfpathlineto{\pgfqpoint{2.320805in}{2.586892in}}%
\pgfpathlineto{\pgfqpoint{2.267761in}{2.586892in}}%
\pgfpathlineto{\pgfqpoint{2.214717in}{2.910735in}}%
\pgfpathlineto{\pgfqpoint{2.161673in}{2.586892in}}%
\pgfpathlineto{\pgfqpoint{2.108629in}{2.586892in}}%
\pgfpathlineto{\pgfqpoint{2.055586in}{2.586892in}}%
\pgfpathlineto{\pgfqpoint{2.002542in}{2.934727in}}%
\pgfpathlineto{\pgfqpoint{1.949498in}{2.586892in}}%
\pgfpathlineto{\pgfqpoint{1.896454in}{2.934727in}}%
\pgfpathlineto{\pgfqpoint{1.843410in}{2.934727in}}%
\pgfpathlineto{\pgfqpoint{1.790366in}{2.586892in}}%
\pgfpathlineto{\pgfqpoint{1.737322in}{2.586892in}}%
\pgfpathlineto{\pgfqpoint{1.684278in}{2.586892in}}%
\pgfpathlineto{\pgfqpoint{1.631234in}{2.802326in}}%
\pgfpathlineto{\pgfqpoint{1.578190in}{2.586892in}}%
\pgfpathlineto{\pgfqpoint{1.525146in}{2.586892in}}%
\pgfpathlineto{\pgfqpoint{1.472102in}{2.586892in}}%
\pgfpathlineto{\pgfqpoint{1.419058in}{2.586892in}}%
\pgfpathlineto{\pgfqpoint{1.366014in}{2.586892in}}%
\pgfpathlineto{\pgfqpoint{1.312970in}{2.586892in}}%
\pgfpathlineto{\pgfqpoint{1.259927in}{2.749185in}}%
\pgfpathlineto{\pgfqpoint{1.206883in}{2.586892in}}%
\pgfpathlineto{\pgfqpoint{1.153839in}{2.586892in}}%
\pgfpathlineto{\pgfqpoint{1.100795in}{2.586892in}}%
\pgfpathlineto{\pgfqpoint{1.047751in}{2.586892in}}%
\pgfpathlineto{\pgfqpoint{0.994707in}{2.586892in}}%
\pgfpathlineto{\pgfqpoint{0.941663in}{2.586892in}}%
\pgfpathlineto{\pgfqpoint{0.941663in}{2.586892in}}%
\pgfpathclose%
\pgfusepath{stroke,fill}%
}%
\begin{pgfscope}%
\pgfsys@transformshift{0.000000in}{0.000000in}%
\pgfsys@useobject{currentmarker}{}%
\end{pgfscope}%
\end{pgfscope}%
\begin{pgfscope}%
\pgfpathrectangle{\pgfqpoint{0.941663in}{0.670138in}}{\pgfqpoint{8.858337in}{3.465625in}}%
\pgfusepath{clip}%
\pgfsetrectcap%
\pgfsetroundjoin%
\pgfsetlinewidth{1.505625pt}%
\definecolor{currentstroke}{rgb}{0.549020,0.337255,0.294118}%
\pgfsetstrokecolor{currentstroke}%
\pgfsetdash{}{0pt}%
\pgfpathmoveto{\pgfqpoint{0.941663in}{2.586892in}}%
\pgfpathlineto{\pgfqpoint{1.206883in}{2.586892in}}%
\pgfpathlineto{\pgfqpoint{1.259927in}{2.749185in}}%
\pgfpathlineto{\pgfqpoint{1.312970in}{2.586892in}}%
\pgfpathlineto{\pgfqpoint{1.578190in}{2.586892in}}%
\pgfpathlineto{\pgfqpoint{1.631234in}{2.802326in}}%
\pgfpathlineto{\pgfqpoint{1.684278in}{2.586892in}}%
\pgfpathlineto{\pgfqpoint{1.790366in}{2.586892in}}%
\pgfpathlineto{\pgfqpoint{1.843410in}{3.070356in}}%
\pgfpathlineto{\pgfqpoint{1.896454in}{3.219351in}}%
\pgfpathlineto{\pgfqpoint{1.949498in}{2.586892in}}%
\pgfpathlineto{\pgfqpoint{2.002542in}{3.060121in}}%
\pgfpathlineto{\pgfqpoint{2.055586in}{2.586892in}}%
\pgfpathlineto{\pgfqpoint{2.161673in}{2.586892in}}%
\pgfpathlineto{\pgfqpoint{2.214717in}{2.910735in}}%
\pgfpathlineto{\pgfqpoint{2.267761in}{2.586892in}}%
\pgfpathlineto{\pgfqpoint{2.320805in}{2.586892in}}%
\pgfpathlineto{\pgfqpoint{2.373849in}{2.844885in}}%
\pgfpathlineto{\pgfqpoint{2.426893in}{2.942415in}}%
\pgfpathlineto{\pgfqpoint{2.479937in}{2.586892in}}%
\pgfpathlineto{\pgfqpoint{2.904288in}{2.586892in}}%
\pgfpathlineto{\pgfqpoint{2.957332in}{3.080511in}}%
\pgfpathlineto{\pgfqpoint{3.010376in}{2.586892in}}%
\pgfpathlineto{\pgfqpoint{3.063420in}{2.909016in}}%
\pgfpathlineto{\pgfqpoint{3.116464in}{2.586892in}}%
\pgfpathlineto{\pgfqpoint{3.434728in}{2.586892in}}%
\pgfpathlineto{\pgfqpoint{3.487772in}{2.710315in}}%
\pgfpathlineto{\pgfqpoint{3.540816in}{2.908402in}}%
\pgfpathlineto{\pgfqpoint{3.593860in}{2.586892in}}%
\pgfpathlineto{\pgfqpoint{4.283431in}{2.586892in}}%
\pgfpathlineto{\pgfqpoint{4.336475in}{3.224846in}}%
\pgfpathlineto{\pgfqpoint{4.389519in}{2.586892in}}%
\pgfpathlineto{\pgfqpoint{4.442563in}{2.586892in}}%
\pgfpathlineto{\pgfqpoint{4.495607in}{2.968775in}}%
\pgfpathlineto{\pgfqpoint{4.548650in}{2.586892in}}%
\pgfpathlineto{\pgfqpoint{4.601694in}{2.927863in}}%
\pgfpathlineto{\pgfqpoint{4.654738in}{2.586892in}}%
\pgfpathlineto{\pgfqpoint{4.707782in}{3.102021in}}%
\pgfpathlineto{\pgfqpoint{4.760826in}{2.586892in}}%
\pgfpathlineto{\pgfqpoint{4.813870in}{2.586892in}}%
\pgfpathlineto{\pgfqpoint{4.866914in}{2.884645in}}%
\pgfpathlineto{\pgfqpoint{4.919958in}{2.863950in}}%
\pgfpathlineto{\pgfqpoint{4.973002in}{2.586892in}}%
\pgfpathlineto{\pgfqpoint{5.026046in}{2.758322in}}%
\pgfpathlineto{\pgfqpoint{5.079090in}{2.586892in}}%
\pgfpathlineto{\pgfqpoint{5.132134in}{2.824147in}}%
\pgfpathlineto{\pgfqpoint{5.185178in}{2.910114in}}%
\pgfpathlineto{\pgfqpoint{5.238222in}{2.837414in}}%
\pgfpathlineto{\pgfqpoint{5.291266in}{2.586892in}}%
\pgfpathlineto{\pgfqpoint{5.344309in}{2.705202in}}%
\pgfpathlineto{\pgfqpoint{5.397353in}{2.660769in}}%
\pgfpathlineto{\pgfqpoint{5.450397in}{3.105789in}}%
\pgfpathlineto{\pgfqpoint{5.503441in}{2.586892in}}%
\pgfpathlineto{\pgfqpoint{5.556485in}{2.586892in}}%
\pgfpathlineto{\pgfqpoint{5.609529in}{2.850209in}}%
\pgfpathlineto{\pgfqpoint{5.662573in}{2.711726in}}%
\pgfpathlineto{\pgfqpoint{5.715617in}{2.586892in}}%
\pgfpathlineto{\pgfqpoint{5.768661in}{3.248973in}}%
\pgfpathlineto{\pgfqpoint{5.821705in}{3.208761in}}%
\pgfpathlineto{\pgfqpoint{5.874749in}{3.037838in}}%
\pgfpathlineto{\pgfqpoint{5.927793in}{2.934114in}}%
\pgfpathlineto{\pgfqpoint{5.980837in}{2.586892in}}%
\pgfpathlineto{\pgfqpoint{6.139969in}{2.586892in}}%
\pgfpathlineto{\pgfqpoint{6.193012in}{2.971161in}}%
\pgfpathlineto{\pgfqpoint{6.246056in}{3.026875in}}%
\pgfpathlineto{\pgfqpoint{6.299100in}{2.586892in}}%
\pgfpathlineto{\pgfqpoint{6.352144in}{2.586892in}}%
\pgfpathlineto{\pgfqpoint{6.405188in}{2.699818in}}%
\pgfpathlineto{\pgfqpoint{6.458232in}{2.586892in}}%
\pgfpathlineto{\pgfqpoint{6.829540in}{2.586892in}}%
\pgfpathlineto{\pgfqpoint{6.882584in}{3.116001in}}%
\pgfpathlineto{\pgfqpoint{6.935628in}{2.586892in}}%
\pgfpathlineto{\pgfqpoint{7.147803in}{2.586892in}}%
\pgfpathlineto{\pgfqpoint{7.200847in}{3.064989in}}%
\pgfpathlineto{\pgfqpoint{7.253891in}{2.586892in}}%
\pgfpathlineto{\pgfqpoint{7.306935in}{2.586892in}}%
\pgfpathlineto{\pgfqpoint{7.359979in}{2.806406in}}%
\pgfpathlineto{\pgfqpoint{7.413023in}{2.865454in}}%
\pgfpathlineto{\pgfqpoint{7.466067in}{2.586892in}}%
\pgfpathlineto{\pgfqpoint{7.519111in}{2.586892in}}%
\pgfpathlineto{\pgfqpoint{7.572155in}{2.823725in}}%
\pgfpathlineto{\pgfqpoint{7.625199in}{2.988504in}}%
\pgfpathlineto{\pgfqpoint{7.678243in}{2.586892in}}%
\pgfpathlineto{\pgfqpoint{7.996506in}{2.586892in}}%
\pgfpathlineto{\pgfqpoint{8.049550in}{3.102247in}}%
\pgfpathlineto{\pgfqpoint{8.102594in}{3.118508in}}%
\pgfpathlineto{\pgfqpoint{8.155638in}{2.586892in}}%
\pgfpathlineto{\pgfqpoint{8.208682in}{2.586892in}}%
\pgfpathlineto{\pgfqpoint{8.261726in}{3.260476in}}%
\pgfpathlineto{\pgfqpoint{8.314770in}{3.211850in}}%
\pgfpathlineto{\pgfqpoint{8.367814in}{2.586892in}}%
\pgfpathlineto{\pgfqpoint{8.420858in}{2.989399in}}%
\pgfpathlineto{\pgfqpoint{8.473902in}{2.586892in}}%
\pgfpathlineto{\pgfqpoint{8.526946in}{2.586892in}}%
\pgfpathlineto{\pgfqpoint{8.579990in}{3.021265in}}%
\pgfpathlineto{\pgfqpoint{8.633033in}{2.856358in}}%
\pgfpathlineto{\pgfqpoint{8.686077in}{3.059450in}}%
\pgfpathlineto{\pgfqpoint{8.739121in}{2.825404in}}%
\pgfpathlineto{\pgfqpoint{8.792165in}{2.586892in}}%
\pgfpathlineto{\pgfqpoint{8.845209in}{2.634872in}}%
\pgfpathlineto{\pgfqpoint{8.898253in}{2.689958in}}%
\pgfpathlineto{\pgfqpoint{8.951297in}{2.586892in}}%
\pgfpathlineto{\pgfqpoint{9.110429in}{2.586892in}}%
\pgfpathlineto{\pgfqpoint{9.163473in}{3.004188in}}%
\pgfpathlineto{\pgfqpoint{9.216517in}{3.140256in}}%
\pgfpathlineto{\pgfqpoint{9.269561in}{2.964013in}}%
\pgfpathlineto{\pgfqpoint{9.322605in}{3.063213in}}%
\pgfpathlineto{\pgfqpoint{9.375649in}{3.090533in}}%
\pgfpathlineto{\pgfqpoint{9.428692in}{2.586892in}}%
\pgfpathlineto{\pgfqpoint{9.481736in}{2.586892in}}%
\pgfpathlineto{\pgfqpoint{9.534780in}{3.282563in}}%
\pgfpathlineto{\pgfqpoint{9.587824in}{2.956774in}}%
\pgfpathlineto{\pgfqpoint{9.640868in}{2.586892in}}%
\pgfpathlineto{\pgfqpoint{9.693912in}{2.968640in}}%
\pgfpathlineto{\pgfqpoint{9.746956in}{3.115475in}}%
\pgfpathlineto{\pgfqpoint{9.800000in}{2.909416in}}%
\pgfpathlineto{\pgfqpoint{9.800000in}{2.909416in}}%
\pgfusepath{stroke}%
\end{pgfscope}%
\begin{pgfscope}%
\pgfpathrectangle{\pgfqpoint{0.941663in}{0.670138in}}{\pgfqpoint{8.858337in}{3.465625in}}%
\pgfusepath{clip}%
\pgfsetbuttcap%
\pgfsetroundjoin%
\definecolor{currentfill}{rgb}{0.549020,0.337255,0.294118}%
\pgfsetfillcolor{currentfill}%
\pgfsetlinewidth{1.003750pt}%
\definecolor{currentstroke}{rgb}{0.549020,0.337255,0.294118}%
\pgfsetstrokecolor{currentstroke}%
\pgfsetdash{}{0pt}%
\pgfsys@defobject{currentmarker}{\pgfqpoint{0.941663in}{2.586892in}}{\pgfqpoint{9.800000in}{3.282563in}}{%
\pgfpathmoveto{\pgfqpoint{0.941663in}{2.586892in}}%
\pgfpathlineto{\pgfqpoint{0.941663in}{2.586892in}}%
\pgfpathlineto{\pgfqpoint{0.994707in}{2.586892in}}%
\pgfpathlineto{\pgfqpoint{1.047751in}{2.586892in}}%
\pgfpathlineto{\pgfqpoint{1.100795in}{2.586892in}}%
\pgfpathlineto{\pgfqpoint{1.153839in}{2.586892in}}%
\pgfpathlineto{\pgfqpoint{1.206883in}{2.586892in}}%
\pgfpathlineto{\pgfqpoint{1.259927in}{2.749185in}}%
\pgfpathlineto{\pgfqpoint{1.312970in}{2.586892in}}%
\pgfpathlineto{\pgfqpoint{1.366014in}{2.586892in}}%
\pgfpathlineto{\pgfqpoint{1.419058in}{2.586892in}}%
\pgfpathlineto{\pgfqpoint{1.472102in}{2.586892in}}%
\pgfpathlineto{\pgfqpoint{1.525146in}{2.586892in}}%
\pgfpathlineto{\pgfqpoint{1.578190in}{2.586892in}}%
\pgfpathlineto{\pgfqpoint{1.631234in}{2.802326in}}%
\pgfpathlineto{\pgfqpoint{1.684278in}{2.586892in}}%
\pgfpathlineto{\pgfqpoint{1.737322in}{2.586892in}}%
\pgfpathlineto{\pgfqpoint{1.790366in}{2.586892in}}%
\pgfpathlineto{\pgfqpoint{1.843410in}{2.934727in}}%
\pgfpathlineto{\pgfqpoint{1.896454in}{2.934727in}}%
\pgfpathlineto{\pgfqpoint{1.949498in}{2.586892in}}%
\pgfpathlineto{\pgfqpoint{2.002542in}{2.934727in}}%
\pgfpathlineto{\pgfqpoint{2.055586in}{2.586892in}}%
\pgfpathlineto{\pgfqpoint{2.108629in}{2.586892in}}%
\pgfpathlineto{\pgfqpoint{2.161673in}{2.586892in}}%
\pgfpathlineto{\pgfqpoint{2.214717in}{2.910735in}}%
\pgfpathlineto{\pgfqpoint{2.267761in}{2.586892in}}%
\pgfpathlineto{\pgfqpoint{2.320805in}{2.586892in}}%
\pgfpathlineto{\pgfqpoint{2.373849in}{2.844885in}}%
\pgfpathlineto{\pgfqpoint{2.426893in}{2.934727in}}%
\pgfpathlineto{\pgfqpoint{2.479937in}{2.586892in}}%
\pgfpathlineto{\pgfqpoint{2.532981in}{2.586892in}}%
\pgfpathlineto{\pgfqpoint{2.586025in}{2.586892in}}%
\pgfpathlineto{\pgfqpoint{2.639069in}{2.586892in}}%
\pgfpathlineto{\pgfqpoint{2.692113in}{2.586892in}}%
\pgfpathlineto{\pgfqpoint{2.745157in}{2.586892in}}%
\pgfpathlineto{\pgfqpoint{2.798201in}{2.586892in}}%
\pgfpathlineto{\pgfqpoint{2.851245in}{2.586892in}}%
\pgfpathlineto{\pgfqpoint{2.904288in}{2.586892in}}%
\pgfpathlineto{\pgfqpoint{2.957332in}{2.934727in}}%
\pgfpathlineto{\pgfqpoint{3.010376in}{2.586892in}}%
\pgfpathlineto{\pgfqpoint{3.063420in}{2.909016in}}%
\pgfpathlineto{\pgfqpoint{3.116464in}{2.586892in}}%
\pgfpathlineto{\pgfqpoint{3.169508in}{2.586892in}}%
\pgfpathlineto{\pgfqpoint{3.222552in}{2.586892in}}%
\pgfpathlineto{\pgfqpoint{3.275596in}{2.586892in}}%
\pgfpathlineto{\pgfqpoint{3.328640in}{2.586892in}}%
\pgfpathlineto{\pgfqpoint{3.381684in}{2.586892in}}%
\pgfpathlineto{\pgfqpoint{3.434728in}{2.586892in}}%
\pgfpathlineto{\pgfqpoint{3.487772in}{2.710315in}}%
\pgfpathlineto{\pgfqpoint{3.540816in}{2.908402in}}%
\pgfpathlineto{\pgfqpoint{3.593860in}{2.586892in}}%
\pgfpathlineto{\pgfqpoint{3.646904in}{2.586892in}}%
\pgfpathlineto{\pgfqpoint{3.699948in}{2.586892in}}%
\pgfpathlineto{\pgfqpoint{3.752991in}{2.586892in}}%
\pgfpathlineto{\pgfqpoint{3.806035in}{2.586892in}}%
\pgfpathlineto{\pgfqpoint{3.859079in}{2.586892in}}%
\pgfpathlineto{\pgfqpoint{3.912123in}{2.586892in}}%
\pgfpathlineto{\pgfqpoint{3.965167in}{2.586892in}}%
\pgfpathlineto{\pgfqpoint{4.018211in}{2.586892in}}%
\pgfpathlineto{\pgfqpoint{4.071255in}{2.586892in}}%
\pgfpathlineto{\pgfqpoint{4.124299in}{2.586892in}}%
\pgfpathlineto{\pgfqpoint{4.177343in}{2.586892in}}%
\pgfpathlineto{\pgfqpoint{4.230387in}{2.586892in}}%
\pgfpathlineto{\pgfqpoint{4.283431in}{2.586892in}}%
\pgfpathlineto{\pgfqpoint{4.336475in}{2.934727in}}%
\pgfpathlineto{\pgfqpoint{4.389519in}{2.586892in}}%
\pgfpathlineto{\pgfqpoint{4.442563in}{2.586892in}}%
\pgfpathlineto{\pgfqpoint{4.495607in}{2.934727in}}%
\pgfpathlineto{\pgfqpoint{4.548650in}{2.586892in}}%
\pgfpathlineto{\pgfqpoint{4.601694in}{2.927863in}}%
\pgfpathlineto{\pgfqpoint{4.654738in}{2.586892in}}%
\pgfpathlineto{\pgfqpoint{4.707782in}{2.934727in}}%
\pgfpathlineto{\pgfqpoint{4.760826in}{2.586892in}}%
\pgfpathlineto{\pgfqpoint{4.813870in}{2.586892in}}%
\pgfpathlineto{\pgfqpoint{4.866914in}{2.884645in}}%
\pgfpathlineto{\pgfqpoint{4.919958in}{2.863950in}}%
\pgfpathlineto{\pgfqpoint{4.973002in}{2.586892in}}%
\pgfpathlineto{\pgfqpoint{5.026046in}{2.758322in}}%
\pgfpathlineto{\pgfqpoint{5.079090in}{2.586892in}}%
\pgfpathlineto{\pgfqpoint{5.132134in}{2.824147in}}%
\pgfpathlineto{\pgfqpoint{5.185178in}{2.910114in}}%
\pgfpathlineto{\pgfqpoint{5.238222in}{2.837414in}}%
\pgfpathlineto{\pgfqpoint{5.291266in}{2.586892in}}%
\pgfpathlineto{\pgfqpoint{5.344309in}{2.705202in}}%
\pgfpathlineto{\pgfqpoint{5.397353in}{2.660769in}}%
\pgfpathlineto{\pgfqpoint{5.450397in}{2.934727in}}%
\pgfpathlineto{\pgfqpoint{5.503441in}{2.586892in}}%
\pgfpathlineto{\pgfqpoint{5.556485in}{2.586892in}}%
\pgfpathlineto{\pgfqpoint{5.609529in}{2.850209in}}%
\pgfpathlineto{\pgfqpoint{5.662573in}{2.711726in}}%
\pgfpathlineto{\pgfqpoint{5.715617in}{2.586892in}}%
\pgfpathlineto{\pgfqpoint{5.768661in}{2.934727in}}%
\pgfpathlineto{\pgfqpoint{5.821705in}{2.934727in}}%
\pgfpathlineto{\pgfqpoint{5.874749in}{2.934727in}}%
\pgfpathlineto{\pgfqpoint{5.927793in}{2.643205in}}%
\pgfpathlineto{\pgfqpoint{5.980837in}{2.586892in}}%
\pgfpathlineto{\pgfqpoint{6.033881in}{2.586892in}}%
\pgfpathlineto{\pgfqpoint{6.086925in}{2.586892in}}%
\pgfpathlineto{\pgfqpoint{6.139969in}{2.586892in}}%
\pgfpathlineto{\pgfqpoint{6.193012in}{2.934727in}}%
\pgfpathlineto{\pgfqpoint{6.246056in}{2.934727in}}%
\pgfpathlineto{\pgfqpoint{6.299100in}{2.586892in}}%
\pgfpathlineto{\pgfqpoint{6.352144in}{2.586892in}}%
\pgfpathlineto{\pgfqpoint{6.405188in}{2.699818in}}%
\pgfpathlineto{\pgfqpoint{6.458232in}{2.586892in}}%
\pgfpathlineto{\pgfqpoint{6.511276in}{2.586892in}}%
\pgfpathlineto{\pgfqpoint{6.564320in}{2.586892in}}%
\pgfpathlineto{\pgfqpoint{6.617364in}{2.586892in}}%
\pgfpathlineto{\pgfqpoint{6.670408in}{2.586892in}}%
\pgfpathlineto{\pgfqpoint{6.723452in}{2.586892in}}%
\pgfpathlineto{\pgfqpoint{6.776496in}{2.586892in}}%
\pgfpathlineto{\pgfqpoint{6.829540in}{2.586892in}}%
\pgfpathlineto{\pgfqpoint{6.882584in}{2.934727in}}%
\pgfpathlineto{\pgfqpoint{6.935628in}{2.586892in}}%
\pgfpathlineto{\pgfqpoint{6.988671in}{2.586892in}}%
\pgfpathlineto{\pgfqpoint{7.041715in}{2.586892in}}%
\pgfpathlineto{\pgfqpoint{7.094759in}{2.586892in}}%
\pgfpathlineto{\pgfqpoint{7.147803in}{2.586892in}}%
\pgfpathlineto{\pgfqpoint{7.200847in}{2.934727in}}%
\pgfpathlineto{\pgfqpoint{7.253891in}{2.586892in}}%
\pgfpathlineto{\pgfqpoint{7.306935in}{2.586892in}}%
\pgfpathlineto{\pgfqpoint{7.359979in}{2.806406in}}%
\pgfpathlineto{\pgfqpoint{7.413023in}{2.865454in}}%
\pgfpathlineto{\pgfqpoint{7.466067in}{2.586892in}}%
\pgfpathlineto{\pgfqpoint{7.519111in}{2.586892in}}%
\pgfpathlineto{\pgfqpoint{7.572155in}{2.823725in}}%
\pgfpathlineto{\pgfqpoint{7.625199in}{2.934727in}}%
\pgfpathlineto{\pgfqpoint{7.678243in}{2.586892in}}%
\pgfpathlineto{\pgfqpoint{7.731287in}{2.586892in}}%
\pgfpathlineto{\pgfqpoint{7.784330in}{2.586892in}}%
\pgfpathlineto{\pgfqpoint{7.837374in}{2.586892in}}%
\pgfpathlineto{\pgfqpoint{7.890418in}{2.586892in}}%
\pgfpathlineto{\pgfqpoint{7.943462in}{2.586892in}}%
\pgfpathlineto{\pgfqpoint{7.996506in}{2.586892in}}%
\pgfpathlineto{\pgfqpoint{8.049550in}{2.934727in}}%
\pgfpathlineto{\pgfqpoint{8.102594in}{2.934727in}}%
\pgfpathlineto{\pgfqpoint{8.155638in}{2.586892in}}%
\pgfpathlineto{\pgfqpoint{8.208682in}{2.586892in}}%
\pgfpathlineto{\pgfqpoint{8.261726in}{2.934727in}}%
\pgfpathlineto{\pgfqpoint{8.314770in}{2.934727in}}%
\pgfpathlineto{\pgfqpoint{8.367814in}{2.586892in}}%
\pgfpathlineto{\pgfqpoint{8.420858in}{2.934727in}}%
\pgfpathlineto{\pgfqpoint{8.473902in}{2.586892in}}%
\pgfpathlineto{\pgfqpoint{8.526946in}{2.586892in}}%
\pgfpathlineto{\pgfqpoint{8.579990in}{2.934727in}}%
\pgfpathlineto{\pgfqpoint{8.633033in}{2.856358in}}%
\pgfpathlineto{\pgfqpoint{8.686077in}{2.934727in}}%
\pgfpathlineto{\pgfqpoint{8.739121in}{2.825404in}}%
\pgfpathlineto{\pgfqpoint{8.792165in}{2.586892in}}%
\pgfpathlineto{\pgfqpoint{8.845209in}{2.634872in}}%
\pgfpathlineto{\pgfqpoint{8.898253in}{2.689958in}}%
\pgfpathlineto{\pgfqpoint{8.951297in}{2.586892in}}%
\pgfpathlineto{\pgfqpoint{9.004341in}{2.586892in}}%
\pgfpathlineto{\pgfqpoint{9.057385in}{2.586892in}}%
\pgfpathlineto{\pgfqpoint{9.110429in}{2.586892in}}%
\pgfpathlineto{\pgfqpoint{9.163473in}{2.934727in}}%
\pgfpathlineto{\pgfqpoint{9.216517in}{2.934727in}}%
\pgfpathlineto{\pgfqpoint{9.269561in}{2.934727in}}%
\pgfpathlineto{\pgfqpoint{9.322605in}{2.934727in}}%
\pgfpathlineto{\pgfqpoint{9.375649in}{2.695480in}}%
\pgfpathlineto{\pgfqpoint{9.428692in}{2.586892in}}%
\pgfpathlineto{\pgfqpoint{9.481736in}{2.586892in}}%
\pgfpathlineto{\pgfqpoint{9.534780in}{2.934727in}}%
\pgfpathlineto{\pgfqpoint{9.587824in}{2.916055in}}%
\pgfpathlineto{\pgfqpoint{9.640868in}{2.586892in}}%
\pgfpathlineto{\pgfqpoint{9.693912in}{2.865140in}}%
\pgfpathlineto{\pgfqpoint{9.746956in}{2.608608in}}%
\pgfpathlineto{\pgfqpoint{9.800000in}{2.588586in}}%
\pgfpathlineto{\pgfqpoint{9.800000in}{2.909416in}}%
\pgfpathlineto{\pgfqpoint{9.800000in}{2.909416in}}%
\pgfpathlineto{\pgfqpoint{9.746956in}{3.115475in}}%
\pgfpathlineto{\pgfqpoint{9.693912in}{2.968640in}}%
\pgfpathlineto{\pgfqpoint{9.640868in}{2.586892in}}%
\pgfpathlineto{\pgfqpoint{9.587824in}{2.956774in}}%
\pgfpathlineto{\pgfqpoint{9.534780in}{3.282563in}}%
\pgfpathlineto{\pgfqpoint{9.481736in}{2.586892in}}%
\pgfpathlineto{\pgfqpoint{9.428692in}{2.586892in}}%
\pgfpathlineto{\pgfqpoint{9.375649in}{3.090533in}}%
\pgfpathlineto{\pgfqpoint{9.322605in}{3.063213in}}%
\pgfpathlineto{\pgfqpoint{9.269561in}{2.964013in}}%
\pgfpathlineto{\pgfqpoint{9.216517in}{3.140256in}}%
\pgfpathlineto{\pgfqpoint{9.163473in}{3.004188in}}%
\pgfpathlineto{\pgfqpoint{9.110429in}{2.586892in}}%
\pgfpathlineto{\pgfqpoint{9.057385in}{2.586892in}}%
\pgfpathlineto{\pgfqpoint{9.004341in}{2.586892in}}%
\pgfpathlineto{\pgfqpoint{8.951297in}{2.586892in}}%
\pgfpathlineto{\pgfqpoint{8.898253in}{2.689958in}}%
\pgfpathlineto{\pgfqpoint{8.845209in}{2.634872in}}%
\pgfpathlineto{\pgfqpoint{8.792165in}{2.586892in}}%
\pgfpathlineto{\pgfqpoint{8.739121in}{2.825404in}}%
\pgfpathlineto{\pgfqpoint{8.686077in}{3.059450in}}%
\pgfpathlineto{\pgfqpoint{8.633033in}{2.856358in}}%
\pgfpathlineto{\pgfqpoint{8.579990in}{3.021265in}}%
\pgfpathlineto{\pgfqpoint{8.526946in}{2.586892in}}%
\pgfpathlineto{\pgfqpoint{8.473902in}{2.586892in}}%
\pgfpathlineto{\pgfqpoint{8.420858in}{2.989399in}}%
\pgfpathlineto{\pgfqpoint{8.367814in}{2.586892in}}%
\pgfpathlineto{\pgfqpoint{8.314770in}{3.211850in}}%
\pgfpathlineto{\pgfqpoint{8.261726in}{3.260476in}}%
\pgfpathlineto{\pgfqpoint{8.208682in}{2.586892in}}%
\pgfpathlineto{\pgfqpoint{8.155638in}{2.586892in}}%
\pgfpathlineto{\pgfqpoint{8.102594in}{3.118508in}}%
\pgfpathlineto{\pgfqpoint{8.049550in}{3.102247in}}%
\pgfpathlineto{\pgfqpoint{7.996506in}{2.586892in}}%
\pgfpathlineto{\pgfqpoint{7.943462in}{2.586892in}}%
\pgfpathlineto{\pgfqpoint{7.890418in}{2.586892in}}%
\pgfpathlineto{\pgfqpoint{7.837374in}{2.586892in}}%
\pgfpathlineto{\pgfqpoint{7.784330in}{2.586892in}}%
\pgfpathlineto{\pgfqpoint{7.731287in}{2.586892in}}%
\pgfpathlineto{\pgfqpoint{7.678243in}{2.586892in}}%
\pgfpathlineto{\pgfqpoint{7.625199in}{2.988504in}}%
\pgfpathlineto{\pgfqpoint{7.572155in}{2.823725in}}%
\pgfpathlineto{\pgfqpoint{7.519111in}{2.586892in}}%
\pgfpathlineto{\pgfqpoint{7.466067in}{2.586892in}}%
\pgfpathlineto{\pgfqpoint{7.413023in}{2.865454in}}%
\pgfpathlineto{\pgfqpoint{7.359979in}{2.806406in}}%
\pgfpathlineto{\pgfqpoint{7.306935in}{2.586892in}}%
\pgfpathlineto{\pgfqpoint{7.253891in}{2.586892in}}%
\pgfpathlineto{\pgfqpoint{7.200847in}{3.064989in}}%
\pgfpathlineto{\pgfqpoint{7.147803in}{2.586892in}}%
\pgfpathlineto{\pgfqpoint{7.094759in}{2.586892in}}%
\pgfpathlineto{\pgfqpoint{7.041715in}{2.586892in}}%
\pgfpathlineto{\pgfqpoint{6.988671in}{2.586892in}}%
\pgfpathlineto{\pgfqpoint{6.935628in}{2.586892in}}%
\pgfpathlineto{\pgfqpoint{6.882584in}{3.116001in}}%
\pgfpathlineto{\pgfqpoint{6.829540in}{2.586892in}}%
\pgfpathlineto{\pgfqpoint{6.776496in}{2.586892in}}%
\pgfpathlineto{\pgfqpoint{6.723452in}{2.586892in}}%
\pgfpathlineto{\pgfqpoint{6.670408in}{2.586892in}}%
\pgfpathlineto{\pgfqpoint{6.617364in}{2.586892in}}%
\pgfpathlineto{\pgfqpoint{6.564320in}{2.586892in}}%
\pgfpathlineto{\pgfqpoint{6.511276in}{2.586892in}}%
\pgfpathlineto{\pgfqpoint{6.458232in}{2.586892in}}%
\pgfpathlineto{\pgfqpoint{6.405188in}{2.699818in}}%
\pgfpathlineto{\pgfqpoint{6.352144in}{2.586892in}}%
\pgfpathlineto{\pgfqpoint{6.299100in}{2.586892in}}%
\pgfpathlineto{\pgfqpoint{6.246056in}{3.026875in}}%
\pgfpathlineto{\pgfqpoint{6.193012in}{2.971161in}}%
\pgfpathlineto{\pgfqpoint{6.139969in}{2.586892in}}%
\pgfpathlineto{\pgfqpoint{6.086925in}{2.586892in}}%
\pgfpathlineto{\pgfqpoint{6.033881in}{2.586892in}}%
\pgfpathlineto{\pgfqpoint{5.980837in}{2.586892in}}%
\pgfpathlineto{\pgfqpoint{5.927793in}{2.934114in}}%
\pgfpathlineto{\pgfqpoint{5.874749in}{3.037838in}}%
\pgfpathlineto{\pgfqpoint{5.821705in}{3.208761in}}%
\pgfpathlineto{\pgfqpoint{5.768661in}{3.248973in}}%
\pgfpathlineto{\pgfqpoint{5.715617in}{2.586892in}}%
\pgfpathlineto{\pgfqpoint{5.662573in}{2.711726in}}%
\pgfpathlineto{\pgfqpoint{5.609529in}{2.850209in}}%
\pgfpathlineto{\pgfqpoint{5.556485in}{2.586892in}}%
\pgfpathlineto{\pgfqpoint{5.503441in}{2.586892in}}%
\pgfpathlineto{\pgfqpoint{5.450397in}{3.105789in}}%
\pgfpathlineto{\pgfqpoint{5.397353in}{2.660769in}}%
\pgfpathlineto{\pgfqpoint{5.344309in}{2.705202in}}%
\pgfpathlineto{\pgfqpoint{5.291266in}{2.586892in}}%
\pgfpathlineto{\pgfqpoint{5.238222in}{2.837414in}}%
\pgfpathlineto{\pgfqpoint{5.185178in}{2.910114in}}%
\pgfpathlineto{\pgfqpoint{5.132134in}{2.824147in}}%
\pgfpathlineto{\pgfqpoint{5.079090in}{2.586892in}}%
\pgfpathlineto{\pgfqpoint{5.026046in}{2.758322in}}%
\pgfpathlineto{\pgfqpoint{4.973002in}{2.586892in}}%
\pgfpathlineto{\pgfqpoint{4.919958in}{2.863950in}}%
\pgfpathlineto{\pgfqpoint{4.866914in}{2.884645in}}%
\pgfpathlineto{\pgfqpoint{4.813870in}{2.586892in}}%
\pgfpathlineto{\pgfqpoint{4.760826in}{2.586892in}}%
\pgfpathlineto{\pgfqpoint{4.707782in}{3.102021in}}%
\pgfpathlineto{\pgfqpoint{4.654738in}{2.586892in}}%
\pgfpathlineto{\pgfqpoint{4.601694in}{2.927863in}}%
\pgfpathlineto{\pgfqpoint{4.548650in}{2.586892in}}%
\pgfpathlineto{\pgfqpoint{4.495607in}{2.968775in}}%
\pgfpathlineto{\pgfqpoint{4.442563in}{2.586892in}}%
\pgfpathlineto{\pgfqpoint{4.389519in}{2.586892in}}%
\pgfpathlineto{\pgfqpoint{4.336475in}{3.224846in}}%
\pgfpathlineto{\pgfqpoint{4.283431in}{2.586892in}}%
\pgfpathlineto{\pgfqpoint{4.230387in}{2.586892in}}%
\pgfpathlineto{\pgfqpoint{4.177343in}{2.586892in}}%
\pgfpathlineto{\pgfqpoint{4.124299in}{2.586892in}}%
\pgfpathlineto{\pgfqpoint{4.071255in}{2.586892in}}%
\pgfpathlineto{\pgfqpoint{4.018211in}{2.586892in}}%
\pgfpathlineto{\pgfqpoint{3.965167in}{2.586892in}}%
\pgfpathlineto{\pgfqpoint{3.912123in}{2.586892in}}%
\pgfpathlineto{\pgfqpoint{3.859079in}{2.586892in}}%
\pgfpathlineto{\pgfqpoint{3.806035in}{2.586892in}}%
\pgfpathlineto{\pgfqpoint{3.752991in}{2.586892in}}%
\pgfpathlineto{\pgfqpoint{3.699948in}{2.586892in}}%
\pgfpathlineto{\pgfqpoint{3.646904in}{2.586892in}}%
\pgfpathlineto{\pgfqpoint{3.593860in}{2.586892in}}%
\pgfpathlineto{\pgfqpoint{3.540816in}{2.908402in}}%
\pgfpathlineto{\pgfqpoint{3.487772in}{2.710315in}}%
\pgfpathlineto{\pgfqpoint{3.434728in}{2.586892in}}%
\pgfpathlineto{\pgfqpoint{3.381684in}{2.586892in}}%
\pgfpathlineto{\pgfqpoint{3.328640in}{2.586892in}}%
\pgfpathlineto{\pgfqpoint{3.275596in}{2.586892in}}%
\pgfpathlineto{\pgfqpoint{3.222552in}{2.586892in}}%
\pgfpathlineto{\pgfqpoint{3.169508in}{2.586892in}}%
\pgfpathlineto{\pgfqpoint{3.116464in}{2.586892in}}%
\pgfpathlineto{\pgfqpoint{3.063420in}{2.909016in}}%
\pgfpathlineto{\pgfqpoint{3.010376in}{2.586892in}}%
\pgfpathlineto{\pgfqpoint{2.957332in}{3.080511in}}%
\pgfpathlineto{\pgfqpoint{2.904288in}{2.586892in}}%
\pgfpathlineto{\pgfqpoint{2.851245in}{2.586892in}}%
\pgfpathlineto{\pgfqpoint{2.798201in}{2.586892in}}%
\pgfpathlineto{\pgfqpoint{2.745157in}{2.586892in}}%
\pgfpathlineto{\pgfqpoint{2.692113in}{2.586892in}}%
\pgfpathlineto{\pgfqpoint{2.639069in}{2.586892in}}%
\pgfpathlineto{\pgfqpoint{2.586025in}{2.586892in}}%
\pgfpathlineto{\pgfqpoint{2.532981in}{2.586892in}}%
\pgfpathlineto{\pgfqpoint{2.479937in}{2.586892in}}%
\pgfpathlineto{\pgfqpoint{2.426893in}{2.942415in}}%
\pgfpathlineto{\pgfqpoint{2.373849in}{2.844885in}}%
\pgfpathlineto{\pgfqpoint{2.320805in}{2.586892in}}%
\pgfpathlineto{\pgfqpoint{2.267761in}{2.586892in}}%
\pgfpathlineto{\pgfqpoint{2.214717in}{2.910735in}}%
\pgfpathlineto{\pgfqpoint{2.161673in}{2.586892in}}%
\pgfpathlineto{\pgfqpoint{2.108629in}{2.586892in}}%
\pgfpathlineto{\pgfqpoint{2.055586in}{2.586892in}}%
\pgfpathlineto{\pgfqpoint{2.002542in}{3.060121in}}%
\pgfpathlineto{\pgfqpoint{1.949498in}{2.586892in}}%
\pgfpathlineto{\pgfqpoint{1.896454in}{3.219351in}}%
\pgfpathlineto{\pgfqpoint{1.843410in}{3.070356in}}%
\pgfpathlineto{\pgfqpoint{1.790366in}{2.586892in}}%
\pgfpathlineto{\pgfqpoint{1.737322in}{2.586892in}}%
\pgfpathlineto{\pgfqpoint{1.684278in}{2.586892in}}%
\pgfpathlineto{\pgfqpoint{1.631234in}{2.802326in}}%
\pgfpathlineto{\pgfqpoint{1.578190in}{2.586892in}}%
\pgfpathlineto{\pgfqpoint{1.525146in}{2.586892in}}%
\pgfpathlineto{\pgfqpoint{1.472102in}{2.586892in}}%
\pgfpathlineto{\pgfqpoint{1.419058in}{2.586892in}}%
\pgfpathlineto{\pgfqpoint{1.366014in}{2.586892in}}%
\pgfpathlineto{\pgfqpoint{1.312970in}{2.586892in}}%
\pgfpathlineto{\pgfqpoint{1.259927in}{2.749185in}}%
\pgfpathlineto{\pgfqpoint{1.206883in}{2.586892in}}%
\pgfpathlineto{\pgfqpoint{1.153839in}{2.586892in}}%
\pgfpathlineto{\pgfqpoint{1.100795in}{2.586892in}}%
\pgfpathlineto{\pgfqpoint{1.047751in}{2.586892in}}%
\pgfpathlineto{\pgfqpoint{0.994707in}{2.586892in}}%
\pgfpathlineto{\pgfqpoint{0.941663in}{2.586892in}}%
\pgfpathlineto{\pgfqpoint{0.941663in}{2.586892in}}%
\pgfpathclose%
\pgfusepath{stroke,fill}%
}%
\begin{pgfscope}%
\pgfsys@transformshift{0.000000in}{0.000000in}%
\pgfsys@useobject{currentmarker}{}%
\end{pgfscope}%
\end{pgfscope}%
\begin{pgfscope}%
\pgfpathrectangle{\pgfqpoint{0.941663in}{0.670138in}}{\pgfqpoint{8.858337in}{3.465625in}}%
\pgfusepath{clip}%
\pgfsetrectcap%
\pgfsetroundjoin%
\pgfsetlinewidth{1.505625pt}%
\definecolor{currentstroke}{rgb}{1.000000,0.647059,0.000000}%
\pgfsetstrokecolor{currentstroke}%
\pgfsetdash{}{0pt}%
\pgfpathmoveto{\pgfqpoint{0.941663in}{1.891220in}}%
\pgfpathlineto{\pgfqpoint{0.994707in}{1.543384in}}%
\pgfpathlineto{\pgfqpoint{1.047751in}{1.891220in}}%
\pgfpathlineto{\pgfqpoint{1.419058in}{1.891220in}}%
\pgfpathlineto{\pgfqpoint{1.472102in}{1.843091in}}%
\pgfpathlineto{\pgfqpoint{1.525146in}{1.891220in}}%
\pgfpathlineto{\pgfqpoint{1.631234in}{1.891220in}}%
\pgfpathlineto{\pgfqpoint{1.684278in}{1.880012in}}%
\pgfpathlineto{\pgfqpoint{1.737322in}{1.543384in}}%
\pgfpathlineto{\pgfqpoint{1.790366in}{1.543384in}}%
\pgfpathlineto{\pgfqpoint{1.843410in}{1.891220in}}%
\pgfpathlineto{\pgfqpoint{1.896454in}{1.891220in}}%
\pgfpathlineto{\pgfqpoint{1.949498in}{1.543384in}}%
\pgfpathlineto{\pgfqpoint{2.002542in}{1.891220in}}%
\pgfpathlineto{\pgfqpoint{2.055586in}{1.891220in}}%
\pgfpathlineto{\pgfqpoint{2.108629in}{1.569408in}}%
\pgfpathlineto{\pgfqpoint{2.161673in}{1.891220in}}%
\pgfpathlineto{\pgfqpoint{2.214717in}{1.891220in}}%
\pgfpathlineto{\pgfqpoint{2.267761in}{1.631196in}}%
\pgfpathlineto{\pgfqpoint{2.320805in}{1.543384in}}%
\pgfpathlineto{\pgfqpoint{2.373849in}{1.891220in}}%
\pgfpathlineto{\pgfqpoint{2.851245in}{1.891220in}}%
\pgfpathlineto{\pgfqpoint{2.904288in}{1.543384in}}%
\pgfpathlineto{\pgfqpoint{2.957332in}{1.891220in}}%
\pgfpathlineto{\pgfqpoint{3.010376in}{1.571272in}}%
\pgfpathlineto{\pgfqpoint{3.063420in}{1.891220in}}%
\pgfpathlineto{\pgfqpoint{3.328640in}{1.891220in}}%
\pgfpathlineto{\pgfqpoint{3.381684in}{1.794175in}}%
\pgfpathlineto{\pgfqpoint{3.434728in}{1.543384in}}%
\pgfpathlineto{\pgfqpoint{3.487772in}{1.891220in}}%
\pgfpathlineto{\pgfqpoint{4.071255in}{1.891220in}}%
\pgfpathlineto{\pgfqpoint{4.124299in}{1.770677in}}%
\pgfpathlineto{\pgfqpoint{4.177343in}{1.891220in}}%
\pgfpathlineto{\pgfqpoint{4.230387in}{1.543384in}}%
\pgfpathlineto{\pgfqpoint{4.283431in}{1.543384in}}%
\pgfpathlineto{\pgfqpoint{4.336475in}{1.891220in}}%
\pgfpathlineto{\pgfqpoint{4.389519in}{1.543384in}}%
\pgfpathlineto{\pgfqpoint{4.442563in}{1.580227in}}%
\pgfpathlineto{\pgfqpoint{4.495607in}{1.891220in}}%
\pgfpathlineto{\pgfqpoint{4.548650in}{1.543384in}}%
\pgfpathlineto{\pgfqpoint{4.601694in}{1.891220in}}%
\pgfpathlineto{\pgfqpoint{4.654738in}{1.543384in}}%
\pgfpathlineto{\pgfqpoint{4.707782in}{1.891220in}}%
\pgfpathlineto{\pgfqpoint{4.760826in}{1.543384in}}%
\pgfpathlineto{\pgfqpoint{4.813870in}{1.543384in}}%
\pgfpathlineto{\pgfqpoint{4.866914in}{1.891220in}}%
\pgfpathlineto{\pgfqpoint{4.919958in}{1.891220in}}%
\pgfpathlineto{\pgfqpoint{4.973002in}{1.543384in}}%
\pgfpathlineto{\pgfqpoint{5.026046in}{1.891220in}}%
\pgfpathlineto{\pgfqpoint{5.079090in}{1.543384in}}%
\pgfpathlineto{\pgfqpoint{5.132134in}{1.891220in}}%
\pgfpathlineto{\pgfqpoint{5.238222in}{1.891220in}}%
\pgfpathlineto{\pgfqpoint{5.291266in}{1.543384in}}%
\pgfpathlineto{\pgfqpoint{5.344309in}{1.891220in}}%
\pgfpathlineto{\pgfqpoint{5.450397in}{1.891220in}}%
\pgfpathlineto{\pgfqpoint{5.503441in}{1.543384in}}%
\pgfpathlineto{\pgfqpoint{5.556485in}{1.549519in}}%
\pgfpathlineto{\pgfqpoint{5.609529in}{1.891220in}}%
\pgfpathlineto{\pgfqpoint{5.662573in}{1.891220in}}%
\pgfpathlineto{\pgfqpoint{5.715617in}{1.543384in}}%
\pgfpathlineto{\pgfqpoint{5.768661in}{1.891220in}}%
\pgfpathlineto{\pgfqpoint{5.980837in}{1.891220in}}%
\pgfpathlineto{\pgfqpoint{6.033881in}{1.678405in}}%
\pgfpathlineto{\pgfqpoint{6.086925in}{1.543384in}}%
\pgfpathlineto{\pgfqpoint{6.139969in}{1.731521in}}%
\pgfpathlineto{\pgfqpoint{6.193012in}{1.891220in}}%
\pgfpathlineto{\pgfqpoint{6.246056in}{1.891220in}}%
\pgfpathlineto{\pgfqpoint{6.299100in}{1.798179in}}%
\pgfpathlineto{\pgfqpoint{6.352144in}{1.891220in}}%
\pgfpathlineto{\pgfqpoint{6.670408in}{1.891220in}}%
\pgfpathlineto{\pgfqpoint{6.723452in}{1.871334in}}%
\pgfpathlineto{\pgfqpoint{6.776496in}{1.543384in}}%
\pgfpathlineto{\pgfqpoint{6.829540in}{1.891220in}}%
\pgfpathlineto{\pgfqpoint{7.041715in}{1.891220in}}%
\pgfpathlineto{\pgfqpoint{7.094759in}{1.543384in}}%
\pgfpathlineto{\pgfqpoint{7.147803in}{1.891220in}}%
\pgfpathlineto{\pgfqpoint{7.200847in}{1.891220in}}%
\pgfpathlineto{\pgfqpoint{7.253891in}{1.543384in}}%
\pgfpathlineto{\pgfqpoint{7.306935in}{1.746843in}}%
\pgfpathlineto{\pgfqpoint{7.359979in}{1.891220in}}%
\pgfpathlineto{\pgfqpoint{7.413023in}{1.891220in}}%
\pgfpathlineto{\pgfqpoint{7.466067in}{1.648522in}}%
\pgfpathlineto{\pgfqpoint{7.519111in}{1.543384in}}%
\pgfpathlineto{\pgfqpoint{7.572155in}{1.891220in}}%
\pgfpathlineto{\pgfqpoint{7.625199in}{1.891220in}}%
\pgfpathlineto{\pgfqpoint{7.678243in}{1.849746in}}%
\pgfpathlineto{\pgfqpoint{7.731287in}{1.891220in}}%
\pgfpathlineto{\pgfqpoint{7.784330in}{1.567858in}}%
\pgfpathlineto{\pgfqpoint{7.837374in}{1.543384in}}%
\pgfpathlineto{\pgfqpoint{7.890418in}{1.891220in}}%
\pgfpathlineto{\pgfqpoint{7.943462in}{1.543384in}}%
\pgfpathlineto{\pgfqpoint{7.996506in}{1.543384in}}%
\pgfpathlineto{\pgfqpoint{8.049550in}{1.891220in}}%
\pgfpathlineto{\pgfqpoint{8.102594in}{1.891220in}}%
\pgfpathlineto{\pgfqpoint{8.155638in}{1.543384in}}%
\pgfpathlineto{\pgfqpoint{8.208682in}{1.543384in}}%
\pgfpathlineto{\pgfqpoint{8.261726in}{1.891220in}}%
\pgfpathlineto{\pgfqpoint{8.314770in}{1.891220in}}%
\pgfpathlineto{\pgfqpoint{8.367814in}{1.543384in}}%
\pgfpathlineto{\pgfqpoint{8.420858in}{1.891220in}}%
\pgfpathlineto{\pgfqpoint{8.473902in}{1.543384in}}%
\pgfpathlineto{\pgfqpoint{8.526946in}{1.543384in}}%
\pgfpathlineto{\pgfqpoint{8.579990in}{1.891220in}}%
\pgfpathlineto{\pgfqpoint{8.739121in}{1.891220in}}%
\pgfpathlineto{\pgfqpoint{8.792165in}{1.755774in}}%
\pgfpathlineto{\pgfqpoint{8.845209in}{1.891220in}}%
\pgfpathlineto{\pgfqpoint{8.898253in}{1.891220in}}%
\pgfpathlineto{\pgfqpoint{8.951297in}{1.543384in}}%
\pgfpathlineto{\pgfqpoint{9.004341in}{1.644141in}}%
\pgfpathlineto{\pgfqpoint{9.057385in}{1.543384in}}%
\pgfpathlineto{\pgfqpoint{9.110429in}{1.543384in}}%
\pgfpathlineto{\pgfqpoint{9.163473in}{1.891220in}}%
\pgfpathlineto{\pgfqpoint{9.375649in}{1.891220in}}%
\pgfpathlineto{\pgfqpoint{9.428692in}{1.543384in}}%
\pgfpathlineto{\pgfqpoint{9.481736in}{1.543384in}}%
\pgfpathlineto{\pgfqpoint{9.534780in}{1.891220in}}%
\pgfpathlineto{\pgfqpoint{9.587824in}{1.891220in}}%
\pgfpathlineto{\pgfqpoint{9.640868in}{1.617282in}}%
\pgfpathlineto{\pgfqpoint{9.693912in}{1.891220in}}%
\pgfpathlineto{\pgfqpoint{9.800000in}{1.891220in}}%
\pgfpathlineto{\pgfqpoint{9.800000in}{1.891220in}}%
\pgfusepath{stroke}%
\end{pgfscope}%
\begin{pgfscope}%
\pgfpathrectangle{\pgfqpoint{0.941663in}{0.670138in}}{\pgfqpoint{8.858337in}{3.465625in}}%
\pgfusepath{clip}%
\pgfsetbuttcap%
\pgfsetroundjoin%
\definecolor{currentfill}{rgb}{1.000000,0.647059,0.000000}%
\pgfsetfillcolor{currentfill}%
\pgfsetlinewidth{1.003750pt}%
\definecolor{currentstroke}{rgb}{1.000000,0.647059,0.000000}%
\pgfsetstrokecolor{currentstroke}%
\pgfsetdash{}{0pt}%
\pgfsys@defobject{currentmarker}{\pgfqpoint{0.941663in}{1.543384in}}{\pgfqpoint{9.800000in}{1.891220in}}{%
\pgfpathmoveto{\pgfqpoint{0.941663in}{1.891220in}}%
\pgfpathlineto{\pgfqpoint{0.941663in}{1.891220in}}%
\pgfpathlineto{\pgfqpoint{0.994707in}{1.891220in}}%
\pgfpathlineto{\pgfqpoint{1.047751in}{1.891220in}}%
\pgfpathlineto{\pgfqpoint{1.100795in}{1.891220in}}%
\pgfpathlineto{\pgfqpoint{1.153839in}{1.891220in}}%
\pgfpathlineto{\pgfqpoint{1.206883in}{1.891220in}}%
\pgfpathlineto{\pgfqpoint{1.259927in}{1.891220in}}%
\pgfpathlineto{\pgfqpoint{1.312970in}{1.891220in}}%
\pgfpathlineto{\pgfqpoint{1.366014in}{1.891220in}}%
\pgfpathlineto{\pgfqpoint{1.419058in}{1.891220in}}%
\pgfpathlineto{\pgfqpoint{1.472102in}{1.891220in}}%
\pgfpathlineto{\pgfqpoint{1.525146in}{1.891220in}}%
\pgfpathlineto{\pgfqpoint{1.578190in}{1.891220in}}%
\pgfpathlineto{\pgfqpoint{1.631234in}{1.891220in}}%
\pgfpathlineto{\pgfqpoint{1.684278in}{1.891220in}}%
\pgfpathlineto{\pgfqpoint{1.737322in}{1.891220in}}%
\pgfpathlineto{\pgfqpoint{1.790366in}{1.891220in}}%
\pgfpathlineto{\pgfqpoint{1.843410in}{1.891220in}}%
\pgfpathlineto{\pgfqpoint{1.896454in}{1.891220in}}%
\pgfpathlineto{\pgfqpoint{1.949498in}{1.891220in}}%
\pgfpathlineto{\pgfqpoint{2.002542in}{1.891220in}}%
\pgfpathlineto{\pgfqpoint{2.055586in}{1.891220in}}%
\pgfpathlineto{\pgfqpoint{2.108629in}{1.891220in}}%
\pgfpathlineto{\pgfqpoint{2.161673in}{1.891220in}}%
\pgfpathlineto{\pgfqpoint{2.214717in}{1.891220in}}%
\pgfpathlineto{\pgfqpoint{2.267761in}{1.891220in}}%
\pgfpathlineto{\pgfqpoint{2.320805in}{1.891220in}}%
\pgfpathlineto{\pgfqpoint{2.373849in}{1.891220in}}%
\pgfpathlineto{\pgfqpoint{2.426893in}{1.891220in}}%
\pgfpathlineto{\pgfqpoint{2.479937in}{1.891220in}}%
\pgfpathlineto{\pgfqpoint{2.532981in}{1.891220in}}%
\pgfpathlineto{\pgfqpoint{2.586025in}{1.891220in}}%
\pgfpathlineto{\pgfqpoint{2.639069in}{1.891220in}}%
\pgfpathlineto{\pgfqpoint{2.692113in}{1.891220in}}%
\pgfpathlineto{\pgfqpoint{2.745157in}{1.891220in}}%
\pgfpathlineto{\pgfqpoint{2.798201in}{1.891220in}}%
\pgfpathlineto{\pgfqpoint{2.851245in}{1.891220in}}%
\pgfpathlineto{\pgfqpoint{2.904288in}{1.891220in}}%
\pgfpathlineto{\pgfqpoint{2.957332in}{1.891220in}}%
\pgfpathlineto{\pgfqpoint{3.010376in}{1.891220in}}%
\pgfpathlineto{\pgfqpoint{3.063420in}{1.891220in}}%
\pgfpathlineto{\pgfqpoint{3.116464in}{1.891220in}}%
\pgfpathlineto{\pgfqpoint{3.169508in}{1.891220in}}%
\pgfpathlineto{\pgfqpoint{3.222552in}{1.891220in}}%
\pgfpathlineto{\pgfqpoint{3.275596in}{1.891220in}}%
\pgfpathlineto{\pgfqpoint{3.328640in}{1.891220in}}%
\pgfpathlineto{\pgfqpoint{3.381684in}{1.891220in}}%
\pgfpathlineto{\pgfqpoint{3.434728in}{1.891220in}}%
\pgfpathlineto{\pgfqpoint{3.487772in}{1.891220in}}%
\pgfpathlineto{\pgfqpoint{3.540816in}{1.891220in}}%
\pgfpathlineto{\pgfqpoint{3.593860in}{1.891220in}}%
\pgfpathlineto{\pgfqpoint{3.646904in}{1.891220in}}%
\pgfpathlineto{\pgfqpoint{3.699948in}{1.891220in}}%
\pgfpathlineto{\pgfqpoint{3.752991in}{1.891220in}}%
\pgfpathlineto{\pgfqpoint{3.806035in}{1.891220in}}%
\pgfpathlineto{\pgfqpoint{3.859079in}{1.891220in}}%
\pgfpathlineto{\pgfqpoint{3.912123in}{1.891220in}}%
\pgfpathlineto{\pgfqpoint{3.965167in}{1.891220in}}%
\pgfpathlineto{\pgfqpoint{4.018211in}{1.891220in}}%
\pgfpathlineto{\pgfqpoint{4.071255in}{1.891220in}}%
\pgfpathlineto{\pgfqpoint{4.124299in}{1.891220in}}%
\pgfpathlineto{\pgfqpoint{4.177343in}{1.891220in}}%
\pgfpathlineto{\pgfqpoint{4.230387in}{1.891220in}}%
\pgfpathlineto{\pgfqpoint{4.283431in}{1.891220in}}%
\pgfpathlineto{\pgfqpoint{4.336475in}{1.891220in}}%
\pgfpathlineto{\pgfqpoint{4.389519in}{1.891220in}}%
\pgfpathlineto{\pgfqpoint{4.442563in}{1.891220in}}%
\pgfpathlineto{\pgfqpoint{4.495607in}{1.891220in}}%
\pgfpathlineto{\pgfqpoint{4.548650in}{1.891220in}}%
\pgfpathlineto{\pgfqpoint{4.601694in}{1.891220in}}%
\pgfpathlineto{\pgfqpoint{4.654738in}{1.891220in}}%
\pgfpathlineto{\pgfqpoint{4.707782in}{1.891220in}}%
\pgfpathlineto{\pgfqpoint{4.760826in}{1.891220in}}%
\pgfpathlineto{\pgfqpoint{4.813870in}{1.891220in}}%
\pgfpathlineto{\pgfqpoint{4.866914in}{1.891220in}}%
\pgfpathlineto{\pgfqpoint{4.919958in}{1.891220in}}%
\pgfpathlineto{\pgfqpoint{4.973002in}{1.891220in}}%
\pgfpathlineto{\pgfqpoint{5.026046in}{1.891220in}}%
\pgfpathlineto{\pgfqpoint{5.079090in}{1.891220in}}%
\pgfpathlineto{\pgfqpoint{5.132134in}{1.891220in}}%
\pgfpathlineto{\pgfqpoint{5.185178in}{1.891220in}}%
\pgfpathlineto{\pgfqpoint{5.238222in}{1.891220in}}%
\pgfpathlineto{\pgfqpoint{5.291266in}{1.891220in}}%
\pgfpathlineto{\pgfqpoint{5.344309in}{1.891220in}}%
\pgfpathlineto{\pgfqpoint{5.397353in}{1.891220in}}%
\pgfpathlineto{\pgfqpoint{5.450397in}{1.891220in}}%
\pgfpathlineto{\pgfqpoint{5.503441in}{1.891220in}}%
\pgfpathlineto{\pgfqpoint{5.556485in}{1.891220in}}%
\pgfpathlineto{\pgfqpoint{5.609529in}{1.891220in}}%
\pgfpathlineto{\pgfqpoint{5.662573in}{1.891220in}}%
\pgfpathlineto{\pgfqpoint{5.715617in}{1.891220in}}%
\pgfpathlineto{\pgfqpoint{5.768661in}{1.891220in}}%
\pgfpathlineto{\pgfqpoint{5.821705in}{1.891220in}}%
\pgfpathlineto{\pgfqpoint{5.874749in}{1.891220in}}%
\pgfpathlineto{\pgfqpoint{5.927793in}{1.891220in}}%
\pgfpathlineto{\pgfqpoint{5.980837in}{1.891220in}}%
\pgfpathlineto{\pgfqpoint{6.033881in}{1.891220in}}%
\pgfpathlineto{\pgfqpoint{6.086925in}{1.891220in}}%
\pgfpathlineto{\pgfqpoint{6.139969in}{1.891220in}}%
\pgfpathlineto{\pgfqpoint{6.193012in}{1.891220in}}%
\pgfpathlineto{\pgfqpoint{6.246056in}{1.891220in}}%
\pgfpathlineto{\pgfqpoint{6.299100in}{1.891220in}}%
\pgfpathlineto{\pgfqpoint{6.352144in}{1.891220in}}%
\pgfpathlineto{\pgfqpoint{6.405188in}{1.891220in}}%
\pgfpathlineto{\pgfqpoint{6.458232in}{1.891220in}}%
\pgfpathlineto{\pgfqpoint{6.511276in}{1.891220in}}%
\pgfpathlineto{\pgfqpoint{6.564320in}{1.891220in}}%
\pgfpathlineto{\pgfqpoint{6.617364in}{1.891220in}}%
\pgfpathlineto{\pgfqpoint{6.670408in}{1.891220in}}%
\pgfpathlineto{\pgfqpoint{6.723452in}{1.891220in}}%
\pgfpathlineto{\pgfqpoint{6.776496in}{1.891220in}}%
\pgfpathlineto{\pgfqpoint{6.829540in}{1.891220in}}%
\pgfpathlineto{\pgfqpoint{6.882584in}{1.891220in}}%
\pgfpathlineto{\pgfqpoint{6.935628in}{1.891220in}}%
\pgfpathlineto{\pgfqpoint{6.988671in}{1.891220in}}%
\pgfpathlineto{\pgfqpoint{7.041715in}{1.891220in}}%
\pgfpathlineto{\pgfqpoint{7.094759in}{1.891220in}}%
\pgfpathlineto{\pgfqpoint{7.147803in}{1.891220in}}%
\pgfpathlineto{\pgfqpoint{7.200847in}{1.891220in}}%
\pgfpathlineto{\pgfqpoint{7.253891in}{1.891220in}}%
\pgfpathlineto{\pgfqpoint{7.306935in}{1.891220in}}%
\pgfpathlineto{\pgfqpoint{7.359979in}{1.891220in}}%
\pgfpathlineto{\pgfqpoint{7.413023in}{1.891220in}}%
\pgfpathlineto{\pgfqpoint{7.466067in}{1.891220in}}%
\pgfpathlineto{\pgfqpoint{7.519111in}{1.891220in}}%
\pgfpathlineto{\pgfqpoint{7.572155in}{1.891220in}}%
\pgfpathlineto{\pgfqpoint{7.625199in}{1.891220in}}%
\pgfpathlineto{\pgfqpoint{7.678243in}{1.891220in}}%
\pgfpathlineto{\pgfqpoint{7.731287in}{1.891220in}}%
\pgfpathlineto{\pgfqpoint{7.784330in}{1.891220in}}%
\pgfpathlineto{\pgfqpoint{7.837374in}{1.891220in}}%
\pgfpathlineto{\pgfqpoint{7.890418in}{1.891220in}}%
\pgfpathlineto{\pgfqpoint{7.943462in}{1.891220in}}%
\pgfpathlineto{\pgfqpoint{7.996506in}{1.891220in}}%
\pgfpathlineto{\pgfqpoint{8.049550in}{1.891220in}}%
\pgfpathlineto{\pgfqpoint{8.102594in}{1.891220in}}%
\pgfpathlineto{\pgfqpoint{8.155638in}{1.891220in}}%
\pgfpathlineto{\pgfqpoint{8.208682in}{1.891220in}}%
\pgfpathlineto{\pgfqpoint{8.261726in}{1.891220in}}%
\pgfpathlineto{\pgfqpoint{8.314770in}{1.891220in}}%
\pgfpathlineto{\pgfqpoint{8.367814in}{1.891220in}}%
\pgfpathlineto{\pgfqpoint{8.420858in}{1.891220in}}%
\pgfpathlineto{\pgfqpoint{8.473902in}{1.891220in}}%
\pgfpathlineto{\pgfqpoint{8.526946in}{1.891220in}}%
\pgfpathlineto{\pgfqpoint{8.579990in}{1.891220in}}%
\pgfpathlineto{\pgfqpoint{8.633033in}{1.891220in}}%
\pgfpathlineto{\pgfqpoint{8.686077in}{1.891220in}}%
\pgfpathlineto{\pgfqpoint{8.739121in}{1.891220in}}%
\pgfpathlineto{\pgfqpoint{8.792165in}{1.891220in}}%
\pgfpathlineto{\pgfqpoint{8.845209in}{1.891220in}}%
\pgfpathlineto{\pgfqpoint{8.898253in}{1.891220in}}%
\pgfpathlineto{\pgfqpoint{8.951297in}{1.891220in}}%
\pgfpathlineto{\pgfqpoint{9.004341in}{1.891220in}}%
\pgfpathlineto{\pgfqpoint{9.057385in}{1.891220in}}%
\pgfpathlineto{\pgfqpoint{9.110429in}{1.891220in}}%
\pgfpathlineto{\pgfqpoint{9.163473in}{1.891220in}}%
\pgfpathlineto{\pgfqpoint{9.216517in}{1.891220in}}%
\pgfpathlineto{\pgfqpoint{9.269561in}{1.891220in}}%
\pgfpathlineto{\pgfqpoint{9.322605in}{1.891220in}}%
\pgfpathlineto{\pgfqpoint{9.375649in}{1.891220in}}%
\pgfpathlineto{\pgfqpoint{9.428692in}{1.891220in}}%
\pgfpathlineto{\pgfqpoint{9.481736in}{1.891220in}}%
\pgfpathlineto{\pgfqpoint{9.534780in}{1.891220in}}%
\pgfpathlineto{\pgfqpoint{9.587824in}{1.891220in}}%
\pgfpathlineto{\pgfqpoint{9.640868in}{1.891220in}}%
\pgfpathlineto{\pgfqpoint{9.693912in}{1.891220in}}%
\pgfpathlineto{\pgfqpoint{9.746956in}{1.891220in}}%
\pgfpathlineto{\pgfqpoint{9.800000in}{1.891220in}}%
\pgfpathlineto{\pgfqpoint{9.800000in}{1.891220in}}%
\pgfpathlineto{\pgfqpoint{9.800000in}{1.891220in}}%
\pgfpathlineto{\pgfqpoint{9.746956in}{1.891220in}}%
\pgfpathlineto{\pgfqpoint{9.693912in}{1.891220in}}%
\pgfpathlineto{\pgfqpoint{9.640868in}{1.617282in}}%
\pgfpathlineto{\pgfqpoint{9.587824in}{1.891220in}}%
\pgfpathlineto{\pgfqpoint{9.534780in}{1.891220in}}%
\pgfpathlineto{\pgfqpoint{9.481736in}{1.543384in}}%
\pgfpathlineto{\pgfqpoint{9.428692in}{1.543384in}}%
\pgfpathlineto{\pgfqpoint{9.375649in}{1.891220in}}%
\pgfpathlineto{\pgfqpoint{9.322605in}{1.891220in}}%
\pgfpathlineto{\pgfqpoint{9.269561in}{1.891220in}}%
\pgfpathlineto{\pgfqpoint{9.216517in}{1.891220in}}%
\pgfpathlineto{\pgfqpoint{9.163473in}{1.891220in}}%
\pgfpathlineto{\pgfqpoint{9.110429in}{1.543384in}}%
\pgfpathlineto{\pgfqpoint{9.057385in}{1.543384in}}%
\pgfpathlineto{\pgfqpoint{9.004341in}{1.644141in}}%
\pgfpathlineto{\pgfqpoint{8.951297in}{1.543384in}}%
\pgfpathlineto{\pgfqpoint{8.898253in}{1.891220in}}%
\pgfpathlineto{\pgfqpoint{8.845209in}{1.891220in}}%
\pgfpathlineto{\pgfqpoint{8.792165in}{1.755774in}}%
\pgfpathlineto{\pgfqpoint{8.739121in}{1.891220in}}%
\pgfpathlineto{\pgfqpoint{8.686077in}{1.891220in}}%
\pgfpathlineto{\pgfqpoint{8.633033in}{1.891220in}}%
\pgfpathlineto{\pgfqpoint{8.579990in}{1.891220in}}%
\pgfpathlineto{\pgfqpoint{8.526946in}{1.543384in}}%
\pgfpathlineto{\pgfqpoint{8.473902in}{1.543384in}}%
\pgfpathlineto{\pgfqpoint{8.420858in}{1.891220in}}%
\pgfpathlineto{\pgfqpoint{8.367814in}{1.543384in}}%
\pgfpathlineto{\pgfqpoint{8.314770in}{1.891220in}}%
\pgfpathlineto{\pgfqpoint{8.261726in}{1.891220in}}%
\pgfpathlineto{\pgfqpoint{8.208682in}{1.543384in}}%
\pgfpathlineto{\pgfqpoint{8.155638in}{1.543384in}}%
\pgfpathlineto{\pgfqpoint{8.102594in}{1.891220in}}%
\pgfpathlineto{\pgfqpoint{8.049550in}{1.891220in}}%
\pgfpathlineto{\pgfqpoint{7.996506in}{1.543384in}}%
\pgfpathlineto{\pgfqpoint{7.943462in}{1.543384in}}%
\pgfpathlineto{\pgfqpoint{7.890418in}{1.891220in}}%
\pgfpathlineto{\pgfqpoint{7.837374in}{1.543384in}}%
\pgfpathlineto{\pgfqpoint{7.784330in}{1.567858in}}%
\pgfpathlineto{\pgfqpoint{7.731287in}{1.891220in}}%
\pgfpathlineto{\pgfqpoint{7.678243in}{1.849746in}}%
\pgfpathlineto{\pgfqpoint{7.625199in}{1.891220in}}%
\pgfpathlineto{\pgfqpoint{7.572155in}{1.891220in}}%
\pgfpathlineto{\pgfqpoint{7.519111in}{1.543384in}}%
\pgfpathlineto{\pgfqpoint{7.466067in}{1.648522in}}%
\pgfpathlineto{\pgfqpoint{7.413023in}{1.891220in}}%
\pgfpathlineto{\pgfqpoint{7.359979in}{1.891220in}}%
\pgfpathlineto{\pgfqpoint{7.306935in}{1.746843in}}%
\pgfpathlineto{\pgfqpoint{7.253891in}{1.543384in}}%
\pgfpathlineto{\pgfqpoint{7.200847in}{1.891220in}}%
\pgfpathlineto{\pgfqpoint{7.147803in}{1.891220in}}%
\pgfpathlineto{\pgfqpoint{7.094759in}{1.543384in}}%
\pgfpathlineto{\pgfqpoint{7.041715in}{1.891220in}}%
\pgfpathlineto{\pgfqpoint{6.988671in}{1.891220in}}%
\pgfpathlineto{\pgfqpoint{6.935628in}{1.891220in}}%
\pgfpathlineto{\pgfqpoint{6.882584in}{1.891220in}}%
\pgfpathlineto{\pgfqpoint{6.829540in}{1.891220in}}%
\pgfpathlineto{\pgfqpoint{6.776496in}{1.543384in}}%
\pgfpathlineto{\pgfqpoint{6.723452in}{1.871334in}}%
\pgfpathlineto{\pgfqpoint{6.670408in}{1.891220in}}%
\pgfpathlineto{\pgfqpoint{6.617364in}{1.891220in}}%
\pgfpathlineto{\pgfqpoint{6.564320in}{1.891220in}}%
\pgfpathlineto{\pgfqpoint{6.511276in}{1.891220in}}%
\pgfpathlineto{\pgfqpoint{6.458232in}{1.891220in}}%
\pgfpathlineto{\pgfqpoint{6.405188in}{1.891220in}}%
\pgfpathlineto{\pgfqpoint{6.352144in}{1.891220in}}%
\pgfpathlineto{\pgfqpoint{6.299100in}{1.798179in}}%
\pgfpathlineto{\pgfqpoint{6.246056in}{1.891220in}}%
\pgfpathlineto{\pgfqpoint{6.193012in}{1.891220in}}%
\pgfpathlineto{\pgfqpoint{6.139969in}{1.731521in}}%
\pgfpathlineto{\pgfqpoint{6.086925in}{1.543384in}}%
\pgfpathlineto{\pgfqpoint{6.033881in}{1.678405in}}%
\pgfpathlineto{\pgfqpoint{5.980837in}{1.891220in}}%
\pgfpathlineto{\pgfqpoint{5.927793in}{1.891220in}}%
\pgfpathlineto{\pgfqpoint{5.874749in}{1.891220in}}%
\pgfpathlineto{\pgfqpoint{5.821705in}{1.891220in}}%
\pgfpathlineto{\pgfqpoint{5.768661in}{1.891220in}}%
\pgfpathlineto{\pgfqpoint{5.715617in}{1.543384in}}%
\pgfpathlineto{\pgfqpoint{5.662573in}{1.891220in}}%
\pgfpathlineto{\pgfqpoint{5.609529in}{1.891220in}}%
\pgfpathlineto{\pgfqpoint{5.556485in}{1.549519in}}%
\pgfpathlineto{\pgfqpoint{5.503441in}{1.543384in}}%
\pgfpathlineto{\pgfqpoint{5.450397in}{1.891220in}}%
\pgfpathlineto{\pgfqpoint{5.397353in}{1.891220in}}%
\pgfpathlineto{\pgfqpoint{5.344309in}{1.891220in}}%
\pgfpathlineto{\pgfqpoint{5.291266in}{1.543384in}}%
\pgfpathlineto{\pgfqpoint{5.238222in}{1.891220in}}%
\pgfpathlineto{\pgfqpoint{5.185178in}{1.891220in}}%
\pgfpathlineto{\pgfqpoint{5.132134in}{1.891220in}}%
\pgfpathlineto{\pgfqpoint{5.079090in}{1.543384in}}%
\pgfpathlineto{\pgfqpoint{5.026046in}{1.891220in}}%
\pgfpathlineto{\pgfqpoint{4.973002in}{1.543384in}}%
\pgfpathlineto{\pgfqpoint{4.919958in}{1.891220in}}%
\pgfpathlineto{\pgfqpoint{4.866914in}{1.891220in}}%
\pgfpathlineto{\pgfqpoint{4.813870in}{1.543384in}}%
\pgfpathlineto{\pgfqpoint{4.760826in}{1.543384in}}%
\pgfpathlineto{\pgfqpoint{4.707782in}{1.891220in}}%
\pgfpathlineto{\pgfqpoint{4.654738in}{1.543384in}}%
\pgfpathlineto{\pgfqpoint{4.601694in}{1.891220in}}%
\pgfpathlineto{\pgfqpoint{4.548650in}{1.543384in}}%
\pgfpathlineto{\pgfqpoint{4.495607in}{1.891220in}}%
\pgfpathlineto{\pgfqpoint{4.442563in}{1.580227in}}%
\pgfpathlineto{\pgfqpoint{4.389519in}{1.543384in}}%
\pgfpathlineto{\pgfqpoint{4.336475in}{1.891220in}}%
\pgfpathlineto{\pgfqpoint{4.283431in}{1.543384in}}%
\pgfpathlineto{\pgfqpoint{4.230387in}{1.543384in}}%
\pgfpathlineto{\pgfqpoint{4.177343in}{1.891220in}}%
\pgfpathlineto{\pgfqpoint{4.124299in}{1.770677in}}%
\pgfpathlineto{\pgfqpoint{4.071255in}{1.891220in}}%
\pgfpathlineto{\pgfqpoint{4.018211in}{1.891220in}}%
\pgfpathlineto{\pgfqpoint{3.965167in}{1.891220in}}%
\pgfpathlineto{\pgfqpoint{3.912123in}{1.891220in}}%
\pgfpathlineto{\pgfqpoint{3.859079in}{1.891220in}}%
\pgfpathlineto{\pgfqpoint{3.806035in}{1.891220in}}%
\pgfpathlineto{\pgfqpoint{3.752991in}{1.891220in}}%
\pgfpathlineto{\pgfqpoint{3.699948in}{1.891220in}}%
\pgfpathlineto{\pgfqpoint{3.646904in}{1.891220in}}%
\pgfpathlineto{\pgfqpoint{3.593860in}{1.891220in}}%
\pgfpathlineto{\pgfqpoint{3.540816in}{1.891220in}}%
\pgfpathlineto{\pgfqpoint{3.487772in}{1.891220in}}%
\pgfpathlineto{\pgfqpoint{3.434728in}{1.543384in}}%
\pgfpathlineto{\pgfqpoint{3.381684in}{1.794175in}}%
\pgfpathlineto{\pgfqpoint{3.328640in}{1.891220in}}%
\pgfpathlineto{\pgfqpoint{3.275596in}{1.891220in}}%
\pgfpathlineto{\pgfqpoint{3.222552in}{1.891220in}}%
\pgfpathlineto{\pgfqpoint{3.169508in}{1.891220in}}%
\pgfpathlineto{\pgfqpoint{3.116464in}{1.891220in}}%
\pgfpathlineto{\pgfqpoint{3.063420in}{1.891220in}}%
\pgfpathlineto{\pgfqpoint{3.010376in}{1.571272in}}%
\pgfpathlineto{\pgfqpoint{2.957332in}{1.891220in}}%
\pgfpathlineto{\pgfqpoint{2.904288in}{1.543384in}}%
\pgfpathlineto{\pgfqpoint{2.851245in}{1.891220in}}%
\pgfpathlineto{\pgfqpoint{2.798201in}{1.891220in}}%
\pgfpathlineto{\pgfqpoint{2.745157in}{1.891220in}}%
\pgfpathlineto{\pgfqpoint{2.692113in}{1.891220in}}%
\pgfpathlineto{\pgfqpoint{2.639069in}{1.891220in}}%
\pgfpathlineto{\pgfqpoint{2.586025in}{1.891220in}}%
\pgfpathlineto{\pgfqpoint{2.532981in}{1.891220in}}%
\pgfpathlineto{\pgfqpoint{2.479937in}{1.891220in}}%
\pgfpathlineto{\pgfqpoint{2.426893in}{1.891220in}}%
\pgfpathlineto{\pgfqpoint{2.373849in}{1.891220in}}%
\pgfpathlineto{\pgfqpoint{2.320805in}{1.543384in}}%
\pgfpathlineto{\pgfqpoint{2.267761in}{1.631196in}}%
\pgfpathlineto{\pgfqpoint{2.214717in}{1.891220in}}%
\pgfpathlineto{\pgfqpoint{2.161673in}{1.891220in}}%
\pgfpathlineto{\pgfqpoint{2.108629in}{1.569408in}}%
\pgfpathlineto{\pgfqpoint{2.055586in}{1.891220in}}%
\pgfpathlineto{\pgfqpoint{2.002542in}{1.891220in}}%
\pgfpathlineto{\pgfqpoint{1.949498in}{1.543384in}}%
\pgfpathlineto{\pgfqpoint{1.896454in}{1.891220in}}%
\pgfpathlineto{\pgfqpoint{1.843410in}{1.891220in}}%
\pgfpathlineto{\pgfqpoint{1.790366in}{1.543384in}}%
\pgfpathlineto{\pgfqpoint{1.737322in}{1.543384in}}%
\pgfpathlineto{\pgfqpoint{1.684278in}{1.880012in}}%
\pgfpathlineto{\pgfqpoint{1.631234in}{1.891220in}}%
\pgfpathlineto{\pgfqpoint{1.578190in}{1.891220in}}%
\pgfpathlineto{\pgfqpoint{1.525146in}{1.891220in}}%
\pgfpathlineto{\pgfqpoint{1.472102in}{1.843091in}}%
\pgfpathlineto{\pgfqpoint{1.419058in}{1.891220in}}%
\pgfpathlineto{\pgfqpoint{1.366014in}{1.891220in}}%
\pgfpathlineto{\pgfqpoint{1.312970in}{1.891220in}}%
\pgfpathlineto{\pgfqpoint{1.259927in}{1.891220in}}%
\pgfpathlineto{\pgfqpoint{1.206883in}{1.891220in}}%
\pgfpathlineto{\pgfqpoint{1.153839in}{1.891220in}}%
\pgfpathlineto{\pgfqpoint{1.100795in}{1.891220in}}%
\pgfpathlineto{\pgfqpoint{1.047751in}{1.891220in}}%
\pgfpathlineto{\pgfqpoint{0.994707in}{1.543384in}}%
\pgfpathlineto{\pgfqpoint{0.941663in}{1.891220in}}%
\pgfpathlineto{\pgfqpoint{0.941663in}{1.891220in}}%
\pgfpathclose%
\pgfusepath{stroke,fill}%
}%
\begin{pgfscope}%
\pgfsys@transformshift{0.000000in}{0.000000in}%
\pgfsys@useobject{currentmarker}{}%
\end{pgfscope}%
\end{pgfscope}%
\begin{pgfscope}%
\pgfpathrectangle{\pgfqpoint{0.941663in}{0.670138in}}{\pgfqpoint{8.858337in}{3.465625in}}%
\pgfusepath{clip}%
\pgfsetrectcap%
\pgfsetroundjoin%
\pgfsetlinewidth{1.505625pt}%
\definecolor{currentstroke}{rgb}{0.501961,0.501961,0.501961}%
\pgfsetstrokecolor{currentstroke}%
\pgfsetdash{}{0pt}%
\pgfpathmoveto{\pgfqpoint{0.941663in}{0.942172in}}%
\pgfpathlineto{\pgfqpoint{0.994707in}{1.206100in}}%
\pgfpathlineto{\pgfqpoint{1.047751in}{0.938123in}}%
\pgfpathlineto{\pgfqpoint{1.100795in}{0.907342in}}%
\pgfpathlineto{\pgfqpoint{1.153839in}{0.893679in}}%
\pgfpathlineto{\pgfqpoint{1.206883in}{1.810992in}}%
\pgfpathlineto{\pgfqpoint{1.259927in}{1.891220in}}%
\pgfpathlineto{\pgfqpoint{1.312970in}{1.856216in}}%
\pgfpathlineto{\pgfqpoint{1.366014in}{0.829335in}}%
\pgfpathlineto{\pgfqpoint{1.419058in}{0.903552in}}%
\pgfpathlineto{\pgfqpoint{1.472102in}{0.908915in}}%
\pgfpathlineto{\pgfqpoint{1.525146in}{0.989277in}}%
\pgfpathlineto{\pgfqpoint{1.578190in}{0.966407in}}%
\pgfpathlineto{\pgfqpoint{1.631234in}{1.891220in}}%
\pgfpathlineto{\pgfqpoint{1.684278in}{1.559075in}}%
\pgfpathlineto{\pgfqpoint{1.737322in}{1.131392in}}%
\pgfpathlineto{\pgfqpoint{1.790366in}{1.113813in}}%
\pgfpathlineto{\pgfqpoint{1.843410in}{1.891220in}}%
\pgfpathlineto{\pgfqpoint{1.896454in}{1.891220in}}%
\pgfpathlineto{\pgfqpoint{1.949498in}{1.087124in}}%
\pgfpathlineto{\pgfqpoint{2.002542in}{1.891220in}}%
\pgfpathlineto{\pgfqpoint{2.055586in}{1.097929in}}%
\pgfpathlineto{\pgfqpoint{2.108629in}{1.481664in}}%
\pgfpathlineto{\pgfqpoint{2.161673in}{0.987578in}}%
\pgfpathlineto{\pgfqpoint{2.214717in}{1.891220in}}%
\pgfpathlineto{\pgfqpoint{2.267761in}{1.182556in}}%
\pgfpathlineto{\pgfqpoint{2.320805in}{0.918169in}}%
\pgfpathlineto{\pgfqpoint{2.373849in}{1.891220in}}%
\pgfpathlineto{\pgfqpoint{2.426893in}{1.891220in}}%
\pgfpathlineto{\pgfqpoint{2.479937in}{1.038089in}}%
\pgfpathlineto{\pgfqpoint{2.532981in}{0.893845in}}%
\pgfpathlineto{\pgfqpoint{2.586025in}{0.872034in}}%
\pgfpathlineto{\pgfqpoint{2.639069in}{0.867540in}}%
\pgfpathlineto{\pgfqpoint{2.692113in}{0.837338in}}%
\pgfpathlineto{\pgfqpoint{2.745157in}{0.908262in}}%
\pgfpathlineto{\pgfqpoint{2.798201in}{0.950657in}}%
\pgfpathlineto{\pgfqpoint{2.851245in}{1.781408in}}%
\pgfpathlineto{\pgfqpoint{2.904288in}{0.997564in}}%
\pgfpathlineto{\pgfqpoint{2.957332in}{1.891220in}}%
\pgfpathlineto{\pgfqpoint{3.010376in}{1.083328in}}%
\pgfpathlineto{\pgfqpoint{3.063420in}{1.891220in}}%
\pgfpathlineto{\pgfqpoint{3.116464in}{1.557584in}}%
\pgfpathlineto{\pgfqpoint{3.169508in}{1.115148in}}%
\pgfpathlineto{\pgfqpoint{3.222552in}{1.145671in}}%
\pgfpathlineto{\pgfqpoint{3.275596in}{1.061811in}}%
\pgfpathlineto{\pgfqpoint{3.328640in}{1.120386in}}%
\pgfpathlineto{\pgfqpoint{3.381684in}{1.666472in}}%
\pgfpathlineto{\pgfqpoint{3.434728in}{0.972772in}}%
\pgfpathlineto{\pgfqpoint{3.487772in}{1.891220in}}%
\pgfpathlineto{\pgfqpoint{3.540816in}{1.891220in}}%
\pgfpathlineto{\pgfqpoint{3.593860in}{1.388474in}}%
\pgfpathlineto{\pgfqpoint{3.646904in}{0.911072in}}%
\pgfpathlineto{\pgfqpoint{3.699948in}{0.867222in}}%
\pgfpathlineto{\pgfqpoint{3.752991in}{1.454919in}}%
\pgfpathlineto{\pgfqpoint{3.806035in}{0.907473in}}%
\pgfpathlineto{\pgfqpoint{3.859079in}{1.517418in}}%
\pgfpathlineto{\pgfqpoint{3.912123in}{1.426054in}}%
\pgfpathlineto{\pgfqpoint{3.965167in}{1.724076in}}%
\pgfpathlineto{\pgfqpoint{4.018211in}{0.913448in}}%
\pgfpathlineto{\pgfqpoint{4.071255in}{0.995539in}}%
\pgfpathlineto{\pgfqpoint{4.124299in}{1.123707in}}%
\pgfpathlineto{\pgfqpoint{4.177343in}{1.039428in}}%
\pgfpathlineto{\pgfqpoint{4.230387in}{1.077535in}}%
\pgfpathlineto{\pgfqpoint{4.283431in}{1.093254in}}%
\pgfpathlineto{\pgfqpoint{4.336475in}{1.891220in}}%
\pgfpathlineto{\pgfqpoint{4.389519in}{1.243678in}}%
\pgfpathlineto{\pgfqpoint{4.442563in}{1.580227in}}%
\pgfpathlineto{\pgfqpoint{4.495607in}{1.891220in}}%
\pgfpathlineto{\pgfqpoint{4.548650in}{1.437632in}}%
\pgfpathlineto{\pgfqpoint{4.601694in}{1.891220in}}%
\pgfpathlineto{\pgfqpoint{4.654738in}{1.094865in}}%
\pgfpathlineto{\pgfqpoint{4.707782in}{1.891220in}}%
\pgfpathlineto{\pgfqpoint{4.760826in}{1.287681in}}%
\pgfpathlineto{\pgfqpoint{4.813870in}{0.975129in}}%
\pgfpathlineto{\pgfqpoint{4.866914in}{1.891220in}}%
\pgfpathlineto{\pgfqpoint{4.919958in}{1.891220in}}%
\pgfpathlineto{\pgfqpoint{4.973002in}{0.856126in}}%
\pgfpathlineto{\pgfqpoint{5.026046in}{1.891220in}}%
\pgfpathlineto{\pgfqpoint{5.079090in}{0.827666in}}%
\pgfpathlineto{\pgfqpoint{5.132134in}{1.891220in}}%
\pgfpathlineto{\pgfqpoint{5.238222in}{1.891220in}}%
\pgfpathlineto{\pgfqpoint{5.291266in}{0.975653in}}%
\pgfpathlineto{\pgfqpoint{5.344309in}{1.891220in}}%
\pgfpathlineto{\pgfqpoint{5.450397in}{1.891220in}}%
\pgfpathlineto{\pgfqpoint{5.503441in}{1.116833in}}%
\pgfpathlineto{\pgfqpoint{5.556485in}{1.549519in}}%
\pgfpathlineto{\pgfqpoint{5.609529in}{1.891220in}}%
\pgfpathlineto{\pgfqpoint{5.662573in}{1.891220in}}%
\pgfpathlineto{\pgfqpoint{5.715617in}{1.122626in}}%
\pgfpathlineto{\pgfqpoint{5.768661in}{1.891220in}}%
\pgfpathlineto{\pgfqpoint{5.927793in}{1.891220in}}%
\pgfpathlineto{\pgfqpoint{5.980837in}{1.046269in}}%
\pgfpathlineto{\pgfqpoint{6.033881in}{1.678405in}}%
\pgfpathlineto{\pgfqpoint{6.086925in}{0.975132in}}%
\pgfpathlineto{\pgfqpoint{6.139969in}{0.956658in}}%
\pgfpathlineto{\pgfqpoint{6.193012in}{1.891220in}}%
\pgfpathlineto{\pgfqpoint{6.246056in}{1.891220in}}%
\pgfpathlineto{\pgfqpoint{6.299100in}{0.877160in}}%
\pgfpathlineto{\pgfqpoint{6.352144in}{1.833058in}}%
\pgfpathlineto{\pgfqpoint{6.405188in}{1.891220in}}%
\pgfpathlineto{\pgfqpoint{6.458232in}{1.413279in}}%
\pgfpathlineto{\pgfqpoint{6.511276in}{0.946754in}}%
\pgfpathlineto{\pgfqpoint{6.564320in}{0.943104in}}%
\pgfpathlineto{\pgfqpoint{6.617364in}{1.010061in}}%
\pgfpathlineto{\pgfqpoint{6.670408in}{1.040003in}}%
\pgfpathlineto{\pgfqpoint{6.723452in}{1.086085in}}%
\pgfpathlineto{\pgfqpoint{6.776496in}{1.330310in}}%
\pgfpathlineto{\pgfqpoint{6.829540in}{1.098052in}}%
\pgfpathlineto{\pgfqpoint{6.882584in}{1.891220in}}%
\pgfpathlineto{\pgfqpoint{6.935628in}{1.091705in}}%
\pgfpathlineto{\pgfqpoint{6.988671in}{1.684508in}}%
\pgfpathlineto{\pgfqpoint{7.041715in}{1.110968in}}%
\pgfpathlineto{\pgfqpoint{7.094759in}{1.134019in}}%
\pgfpathlineto{\pgfqpoint{7.147803in}{1.115813in}}%
\pgfpathlineto{\pgfqpoint{7.200847in}{1.891220in}}%
\pgfpathlineto{\pgfqpoint{7.253891in}{1.460365in}}%
\pgfpathlineto{\pgfqpoint{7.306935in}{1.095703in}}%
\pgfpathlineto{\pgfqpoint{7.359979in}{1.891220in}}%
\pgfpathlineto{\pgfqpoint{7.413023in}{1.891220in}}%
\pgfpathlineto{\pgfqpoint{7.466067in}{1.483365in}}%
\pgfpathlineto{\pgfqpoint{7.519111in}{1.410524in}}%
\pgfpathlineto{\pgfqpoint{7.572155in}{1.891220in}}%
\pgfpathlineto{\pgfqpoint{7.625199in}{1.891220in}}%
\pgfpathlineto{\pgfqpoint{7.678243in}{1.849746in}}%
\pgfpathlineto{\pgfqpoint{7.731287in}{0.908937in}}%
\pgfpathlineto{\pgfqpoint{7.784330in}{0.944968in}}%
\pgfpathlineto{\pgfqpoint{7.837374in}{0.953219in}}%
\pgfpathlineto{\pgfqpoint{7.890418in}{0.994522in}}%
\pgfpathlineto{\pgfqpoint{7.943462in}{0.992268in}}%
\pgfpathlineto{\pgfqpoint{7.996506in}{1.070971in}}%
\pgfpathlineto{\pgfqpoint{8.049550in}{1.891220in}}%
\pgfpathlineto{\pgfqpoint{8.102594in}{1.891220in}}%
\pgfpathlineto{\pgfqpoint{8.155638in}{1.137608in}}%
\pgfpathlineto{\pgfqpoint{8.208682in}{1.131748in}}%
\pgfpathlineto{\pgfqpoint{8.261726in}{1.891220in}}%
\pgfpathlineto{\pgfqpoint{8.314770in}{1.891220in}}%
\pgfpathlineto{\pgfqpoint{8.367814in}{1.114165in}}%
\pgfpathlineto{\pgfqpoint{8.420858in}{1.891220in}}%
\pgfpathlineto{\pgfqpoint{8.473902in}{1.054943in}}%
\pgfpathlineto{\pgfqpoint{8.526946in}{1.026802in}}%
\pgfpathlineto{\pgfqpoint{8.579990in}{1.891220in}}%
\pgfpathlineto{\pgfqpoint{8.739121in}{1.891220in}}%
\pgfpathlineto{\pgfqpoint{8.792165in}{1.755774in}}%
\pgfpathlineto{\pgfqpoint{8.845209in}{1.891220in}}%
\pgfpathlineto{\pgfqpoint{8.898253in}{1.891220in}}%
\pgfpathlineto{\pgfqpoint{8.951297in}{0.878008in}}%
\pgfpathlineto{\pgfqpoint{9.004341in}{1.644141in}}%
\pgfpathlineto{\pgfqpoint{9.057385in}{0.962369in}}%
\pgfpathlineto{\pgfqpoint{9.110429in}{0.997462in}}%
\pgfpathlineto{\pgfqpoint{9.163473in}{1.891220in}}%
\pgfpathlineto{\pgfqpoint{9.375649in}{1.891220in}}%
\pgfpathlineto{\pgfqpoint{9.428692in}{1.153825in}}%
\pgfpathlineto{\pgfqpoint{9.481736in}{1.200625in}}%
\pgfpathlineto{\pgfqpoint{9.534780in}{1.891220in}}%
\pgfpathlineto{\pgfqpoint{9.587824in}{1.891220in}}%
\pgfpathlineto{\pgfqpoint{9.640868in}{1.617282in}}%
\pgfpathlineto{\pgfqpoint{9.693912in}{1.891220in}}%
\pgfpathlineto{\pgfqpoint{9.800000in}{1.891220in}}%
\pgfpathlineto{\pgfqpoint{9.800000in}{1.891220in}}%
\pgfusepath{stroke}%
\end{pgfscope}%
\begin{pgfscope}%
\pgfpathrectangle{\pgfqpoint{0.941663in}{0.670138in}}{\pgfqpoint{8.858337in}{3.465625in}}%
\pgfusepath{clip}%
\pgfsetbuttcap%
\pgfsetroundjoin%
\definecolor{currentfill}{rgb}{0.501961,0.501961,0.501961}%
\pgfsetfillcolor{currentfill}%
\pgfsetlinewidth{1.003750pt}%
\definecolor{currentstroke}{rgb}{0.501961,0.501961,0.501961}%
\pgfsetstrokecolor{currentstroke}%
\pgfsetdash{}{0pt}%
\pgfsys@defobject{currentmarker}{\pgfqpoint{0.941663in}{0.827666in}}{\pgfqpoint{9.800000in}{1.891220in}}{%
\pgfpathmoveto{\pgfqpoint{0.941663in}{0.942172in}}%
\pgfpathlineto{\pgfqpoint{0.941663in}{1.891220in}}%
\pgfpathlineto{\pgfqpoint{0.994707in}{1.543384in}}%
\pgfpathlineto{\pgfqpoint{1.047751in}{1.891220in}}%
\pgfpathlineto{\pgfqpoint{1.100795in}{1.891220in}}%
\pgfpathlineto{\pgfqpoint{1.153839in}{1.891220in}}%
\pgfpathlineto{\pgfqpoint{1.206883in}{1.891220in}}%
\pgfpathlineto{\pgfqpoint{1.259927in}{1.891220in}}%
\pgfpathlineto{\pgfqpoint{1.312970in}{1.891220in}}%
\pgfpathlineto{\pgfqpoint{1.366014in}{1.891220in}}%
\pgfpathlineto{\pgfqpoint{1.419058in}{1.891220in}}%
\pgfpathlineto{\pgfqpoint{1.472102in}{1.843091in}}%
\pgfpathlineto{\pgfqpoint{1.525146in}{1.891220in}}%
\pgfpathlineto{\pgfqpoint{1.578190in}{1.891220in}}%
\pgfpathlineto{\pgfqpoint{1.631234in}{1.891220in}}%
\pgfpathlineto{\pgfqpoint{1.684278in}{1.880012in}}%
\pgfpathlineto{\pgfqpoint{1.737322in}{1.543384in}}%
\pgfpathlineto{\pgfqpoint{1.790366in}{1.543384in}}%
\pgfpathlineto{\pgfqpoint{1.843410in}{1.891220in}}%
\pgfpathlineto{\pgfqpoint{1.896454in}{1.891220in}}%
\pgfpathlineto{\pgfqpoint{1.949498in}{1.543384in}}%
\pgfpathlineto{\pgfqpoint{2.002542in}{1.891220in}}%
\pgfpathlineto{\pgfqpoint{2.055586in}{1.891220in}}%
\pgfpathlineto{\pgfqpoint{2.108629in}{1.569408in}}%
\pgfpathlineto{\pgfqpoint{2.161673in}{1.891220in}}%
\pgfpathlineto{\pgfqpoint{2.214717in}{1.891220in}}%
\pgfpathlineto{\pgfqpoint{2.267761in}{1.631196in}}%
\pgfpathlineto{\pgfqpoint{2.320805in}{1.543384in}}%
\pgfpathlineto{\pgfqpoint{2.373849in}{1.891220in}}%
\pgfpathlineto{\pgfqpoint{2.426893in}{1.891220in}}%
\pgfpathlineto{\pgfqpoint{2.479937in}{1.891220in}}%
\pgfpathlineto{\pgfqpoint{2.532981in}{1.891220in}}%
\pgfpathlineto{\pgfqpoint{2.586025in}{1.891220in}}%
\pgfpathlineto{\pgfqpoint{2.639069in}{1.891220in}}%
\pgfpathlineto{\pgfqpoint{2.692113in}{1.891220in}}%
\pgfpathlineto{\pgfqpoint{2.745157in}{1.891220in}}%
\pgfpathlineto{\pgfqpoint{2.798201in}{1.891220in}}%
\pgfpathlineto{\pgfqpoint{2.851245in}{1.891220in}}%
\pgfpathlineto{\pgfqpoint{2.904288in}{1.543384in}}%
\pgfpathlineto{\pgfqpoint{2.957332in}{1.891220in}}%
\pgfpathlineto{\pgfqpoint{3.010376in}{1.571272in}}%
\pgfpathlineto{\pgfqpoint{3.063420in}{1.891220in}}%
\pgfpathlineto{\pgfqpoint{3.116464in}{1.891220in}}%
\pgfpathlineto{\pgfqpoint{3.169508in}{1.891220in}}%
\pgfpathlineto{\pgfqpoint{3.222552in}{1.891220in}}%
\pgfpathlineto{\pgfqpoint{3.275596in}{1.891220in}}%
\pgfpathlineto{\pgfqpoint{3.328640in}{1.891220in}}%
\pgfpathlineto{\pgfqpoint{3.381684in}{1.794175in}}%
\pgfpathlineto{\pgfqpoint{3.434728in}{1.543384in}}%
\pgfpathlineto{\pgfqpoint{3.487772in}{1.891220in}}%
\pgfpathlineto{\pgfqpoint{3.540816in}{1.891220in}}%
\pgfpathlineto{\pgfqpoint{3.593860in}{1.891220in}}%
\pgfpathlineto{\pgfqpoint{3.646904in}{1.891220in}}%
\pgfpathlineto{\pgfqpoint{3.699948in}{1.891220in}}%
\pgfpathlineto{\pgfqpoint{3.752991in}{1.891220in}}%
\pgfpathlineto{\pgfqpoint{3.806035in}{1.891220in}}%
\pgfpathlineto{\pgfqpoint{3.859079in}{1.891220in}}%
\pgfpathlineto{\pgfqpoint{3.912123in}{1.891220in}}%
\pgfpathlineto{\pgfqpoint{3.965167in}{1.891220in}}%
\pgfpathlineto{\pgfqpoint{4.018211in}{1.891220in}}%
\pgfpathlineto{\pgfqpoint{4.071255in}{1.891220in}}%
\pgfpathlineto{\pgfqpoint{4.124299in}{1.770677in}}%
\pgfpathlineto{\pgfqpoint{4.177343in}{1.891220in}}%
\pgfpathlineto{\pgfqpoint{4.230387in}{1.543384in}}%
\pgfpathlineto{\pgfqpoint{4.283431in}{1.543384in}}%
\pgfpathlineto{\pgfqpoint{4.336475in}{1.891220in}}%
\pgfpathlineto{\pgfqpoint{4.389519in}{1.543384in}}%
\pgfpathlineto{\pgfqpoint{4.442563in}{1.580227in}}%
\pgfpathlineto{\pgfqpoint{4.495607in}{1.891220in}}%
\pgfpathlineto{\pgfqpoint{4.548650in}{1.543384in}}%
\pgfpathlineto{\pgfqpoint{4.601694in}{1.891220in}}%
\pgfpathlineto{\pgfqpoint{4.654738in}{1.543384in}}%
\pgfpathlineto{\pgfqpoint{4.707782in}{1.891220in}}%
\pgfpathlineto{\pgfqpoint{4.760826in}{1.543384in}}%
\pgfpathlineto{\pgfqpoint{4.813870in}{1.543384in}}%
\pgfpathlineto{\pgfqpoint{4.866914in}{1.891220in}}%
\pgfpathlineto{\pgfqpoint{4.919958in}{1.891220in}}%
\pgfpathlineto{\pgfqpoint{4.973002in}{1.543384in}}%
\pgfpathlineto{\pgfqpoint{5.026046in}{1.891220in}}%
\pgfpathlineto{\pgfqpoint{5.079090in}{1.543384in}}%
\pgfpathlineto{\pgfqpoint{5.132134in}{1.891220in}}%
\pgfpathlineto{\pgfqpoint{5.185178in}{1.891220in}}%
\pgfpathlineto{\pgfqpoint{5.238222in}{1.891220in}}%
\pgfpathlineto{\pgfqpoint{5.291266in}{1.543384in}}%
\pgfpathlineto{\pgfqpoint{5.344309in}{1.891220in}}%
\pgfpathlineto{\pgfqpoint{5.397353in}{1.891220in}}%
\pgfpathlineto{\pgfqpoint{5.450397in}{1.891220in}}%
\pgfpathlineto{\pgfqpoint{5.503441in}{1.543384in}}%
\pgfpathlineto{\pgfqpoint{5.556485in}{1.549519in}}%
\pgfpathlineto{\pgfqpoint{5.609529in}{1.891220in}}%
\pgfpathlineto{\pgfqpoint{5.662573in}{1.891220in}}%
\pgfpathlineto{\pgfqpoint{5.715617in}{1.543384in}}%
\pgfpathlineto{\pgfqpoint{5.768661in}{1.891220in}}%
\pgfpathlineto{\pgfqpoint{5.821705in}{1.891220in}}%
\pgfpathlineto{\pgfqpoint{5.874749in}{1.891220in}}%
\pgfpathlineto{\pgfqpoint{5.927793in}{1.891220in}}%
\pgfpathlineto{\pgfqpoint{5.980837in}{1.891220in}}%
\pgfpathlineto{\pgfqpoint{6.033881in}{1.678405in}}%
\pgfpathlineto{\pgfqpoint{6.086925in}{1.543384in}}%
\pgfpathlineto{\pgfqpoint{6.139969in}{1.731521in}}%
\pgfpathlineto{\pgfqpoint{6.193012in}{1.891220in}}%
\pgfpathlineto{\pgfqpoint{6.246056in}{1.891220in}}%
\pgfpathlineto{\pgfqpoint{6.299100in}{1.798179in}}%
\pgfpathlineto{\pgfqpoint{6.352144in}{1.891220in}}%
\pgfpathlineto{\pgfqpoint{6.405188in}{1.891220in}}%
\pgfpathlineto{\pgfqpoint{6.458232in}{1.891220in}}%
\pgfpathlineto{\pgfqpoint{6.511276in}{1.891220in}}%
\pgfpathlineto{\pgfqpoint{6.564320in}{1.891220in}}%
\pgfpathlineto{\pgfqpoint{6.617364in}{1.891220in}}%
\pgfpathlineto{\pgfqpoint{6.670408in}{1.891220in}}%
\pgfpathlineto{\pgfqpoint{6.723452in}{1.871334in}}%
\pgfpathlineto{\pgfqpoint{6.776496in}{1.543384in}}%
\pgfpathlineto{\pgfqpoint{6.829540in}{1.891220in}}%
\pgfpathlineto{\pgfqpoint{6.882584in}{1.891220in}}%
\pgfpathlineto{\pgfqpoint{6.935628in}{1.891220in}}%
\pgfpathlineto{\pgfqpoint{6.988671in}{1.891220in}}%
\pgfpathlineto{\pgfqpoint{7.041715in}{1.891220in}}%
\pgfpathlineto{\pgfqpoint{7.094759in}{1.543384in}}%
\pgfpathlineto{\pgfqpoint{7.147803in}{1.891220in}}%
\pgfpathlineto{\pgfqpoint{7.200847in}{1.891220in}}%
\pgfpathlineto{\pgfqpoint{7.253891in}{1.543384in}}%
\pgfpathlineto{\pgfqpoint{7.306935in}{1.746843in}}%
\pgfpathlineto{\pgfqpoint{7.359979in}{1.891220in}}%
\pgfpathlineto{\pgfqpoint{7.413023in}{1.891220in}}%
\pgfpathlineto{\pgfqpoint{7.466067in}{1.648522in}}%
\pgfpathlineto{\pgfqpoint{7.519111in}{1.543384in}}%
\pgfpathlineto{\pgfqpoint{7.572155in}{1.891220in}}%
\pgfpathlineto{\pgfqpoint{7.625199in}{1.891220in}}%
\pgfpathlineto{\pgfqpoint{7.678243in}{1.849746in}}%
\pgfpathlineto{\pgfqpoint{7.731287in}{1.891220in}}%
\pgfpathlineto{\pgfqpoint{7.784330in}{1.567858in}}%
\pgfpathlineto{\pgfqpoint{7.837374in}{1.543384in}}%
\pgfpathlineto{\pgfqpoint{7.890418in}{1.891220in}}%
\pgfpathlineto{\pgfqpoint{7.943462in}{1.543384in}}%
\pgfpathlineto{\pgfqpoint{7.996506in}{1.543384in}}%
\pgfpathlineto{\pgfqpoint{8.049550in}{1.891220in}}%
\pgfpathlineto{\pgfqpoint{8.102594in}{1.891220in}}%
\pgfpathlineto{\pgfqpoint{8.155638in}{1.543384in}}%
\pgfpathlineto{\pgfqpoint{8.208682in}{1.543384in}}%
\pgfpathlineto{\pgfqpoint{8.261726in}{1.891220in}}%
\pgfpathlineto{\pgfqpoint{8.314770in}{1.891220in}}%
\pgfpathlineto{\pgfqpoint{8.367814in}{1.543384in}}%
\pgfpathlineto{\pgfqpoint{8.420858in}{1.891220in}}%
\pgfpathlineto{\pgfqpoint{8.473902in}{1.543384in}}%
\pgfpathlineto{\pgfqpoint{8.526946in}{1.543384in}}%
\pgfpathlineto{\pgfqpoint{8.579990in}{1.891220in}}%
\pgfpathlineto{\pgfqpoint{8.633033in}{1.891220in}}%
\pgfpathlineto{\pgfqpoint{8.686077in}{1.891220in}}%
\pgfpathlineto{\pgfqpoint{8.739121in}{1.891220in}}%
\pgfpathlineto{\pgfqpoint{8.792165in}{1.755774in}}%
\pgfpathlineto{\pgfqpoint{8.845209in}{1.891220in}}%
\pgfpathlineto{\pgfqpoint{8.898253in}{1.891220in}}%
\pgfpathlineto{\pgfqpoint{8.951297in}{1.543384in}}%
\pgfpathlineto{\pgfqpoint{9.004341in}{1.644141in}}%
\pgfpathlineto{\pgfqpoint{9.057385in}{1.543384in}}%
\pgfpathlineto{\pgfqpoint{9.110429in}{1.543384in}}%
\pgfpathlineto{\pgfqpoint{9.163473in}{1.891220in}}%
\pgfpathlineto{\pgfqpoint{9.216517in}{1.891220in}}%
\pgfpathlineto{\pgfqpoint{9.269561in}{1.891220in}}%
\pgfpathlineto{\pgfqpoint{9.322605in}{1.891220in}}%
\pgfpathlineto{\pgfqpoint{9.375649in}{1.891220in}}%
\pgfpathlineto{\pgfqpoint{9.428692in}{1.543384in}}%
\pgfpathlineto{\pgfqpoint{9.481736in}{1.543384in}}%
\pgfpathlineto{\pgfqpoint{9.534780in}{1.891220in}}%
\pgfpathlineto{\pgfqpoint{9.587824in}{1.891220in}}%
\pgfpathlineto{\pgfqpoint{9.640868in}{1.617282in}}%
\pgfpathlineto{\pgfqpoint{9.693912in}{1.891220in}}%
\pgfpathlineto{\pgfqpoint{9.746956in}{1.891220in}}%
\pgfpathlineto{\pgfqpoint{9.800000in}{1.891220in}}%
\pgfpathlineto{\pgfqpoint{9.800000in}{1.891220in}}%
\pgfpathlineto{\pgfqpoint{9.800000in}{1.891220in}}%
\pgfpathlineto{\pgfqpoint{9.746956in}{1.891220in}}%
\pgfpathlineto{\pgfqpoint{9.693912in}{1.891220in}}%
\pgfpathlineto{\pgfqpoint{9.640868in}{1.617282in}}%
\pgfpathlineto{\pgfqpoint{9.587824in}{1.891220in}}%
\pgfpathlineto{\pgfqpoint{9.534780in}{1.891220in}}%
\pgfpathlineto{\pgfqpoint{9.481736in}{1.200625in}}%
\pgfpathlineto{\pgfqpoint{9.428692in}{1.153825in}}%
\pgfpathlineto{\pgfqpoint{9.375649in}{1.891220in}}%
\pgfpathlineto{\pgfqpoint{9.322605in}{1.891220in}}%
\pgfpathlineto{\pgfqpoint{9.269561in}{1.891220in}}%
\pgfpathlineto{\pgfqpoint{9.216517in}{1.891220in}}%
\pgfpathlineto{\pgfqpoint{9.163473in}{1.891220in}}%
\pgfpathlineto{\pgfqpoint{9.110429in}{0.997462in}}%
\pgfpathlineto{\pgfqpoint{9.057385in}{0.962369in}}%
\pgfpathlineto{\pgfqpoint{9.004341in}{1.644141in}}%
\pgfpathlineto{\pgfqpoint{8.951297in}{0.878008in}}%
\pgfpathlineto{\pgfqpoint{8.898253in}{1.891220in}}%
\pgfpathlineto{\pgfqpoint{8.845209in}{1.891220in}}%
\pgfpathlineto{\pgfqpoint{8.792165in}{1.755774in}}%
\pgfpathlineto{\pgfqpoint{8.739121in}{1.891220in}}%
\pgfpathlineto{\pgfqpoint{8.686077in}{1.891220in}}%
\pgfpathlineto{\pgfqpoint{8.633033in}{1.891220in}}%
\pgfpathlineto{\pgfqpoint{8.579990in}{1.891220in}}%
\pgfpathlineto{\pgfqpoint{8.526946in}{1.026802in}}%
\pgfpathlineto{\pgfqpoint{8.473902in}{1.054943in}}%
\pgfpathlineto{\pgfqpoint{8.420858in}{1.891220in}}%
\pgfpathlineto{\pgfqpoint{8.367814in}{1.114165in}}%
\pgfpathlineto{\pgfqpoint{8.314770in}{1.891220in}}%
\pgfpathlineto{\pgfqpoint{8.261726in}{1.891220in}}%
\pgfpathlineto{\pgfqpoint{8.208682in}{1.131748in}}%
\pgfpathlineto{\pgfqpoint{8.155638in}{1.137608in}}%
\pgfpathlineto{\pgfqpoint{8.102594in}{1.891220in}}%
\pgfpathlineto{\pgfqpoint{8.049550in}{1.891220in}}%
\pgfpathlineto{\pgfqpoint{7.996506in}{1.070971in}}%
\pgfpathlineto{\pgfqpoint{7.943462in}{0.992268in}}%
\pgfpathlineto{\pgfqpoint{7.890418in}{0.994522in}}%
\pgfpathlineto{\pgfqpoint{7.837374in}{0.953219in}}%
\pgfpathlineto{\pgfqpoint{7.784330in}{0.944968in}}%
\pgfpathlineto{\pgfqpoint{7.731287in}{0.908937in}}%
\pgfpathlineto{\pgfqpoint{7.678243in}{1.849746in}}%
\pgfpathlineto{\pgfqpoint{7.625199in}{1.891220in}}%
\pgfpathlineto{\pgfqpoint{7.572155in}{1.891220in}}%
\pgfpathlineto{\pgfqpoint{7.519111in}{1.410524in}}%
\pgfpathlineto{\pgfqpoint{7.466067in}{1.483365in}}%
\pgfpathlineto{\pgfqpoint{7.413023in}{1.891220in}}%
\pgfpathlineto{\pgfqpoint{7.359979in}{1.891220in}}%
\pgfpathlineto{\pgfqpoint{7.306935in}{1.095703in}}%
\pgfpathlineto{\pgfqpoint{7.253891in}{1.460365in}}%
\pgfpathlineto{\pgfqpoint{7.200847in}{1.891220in}}%
\pgfpathlineto{\pgfqpoint{7.147803in}{1.115813in}}%
\pgfpathlineto{\pgfqpoint{7.094759in}{1.134019in}}%
\pgfpathlineto{\pgfqpoint{7.041715in}{1.110968in}}%
\pgfpathlineto{\pgfqpoint{6.988671in}{1.684508in}}%
\pgfpathlineto{\pgfqpoint{6.935628in}{1.091705in}}%
\pgfpathlineto{\pgfqpoint{6.882584in}{1.891220in}}%
\pgfpathlineto{\pgfqpoint{6.829540in}{1.098052in}}%
\pgfpathlineto{\pgfqpoint{6.776496in}{1.330310in}}%
\pgfpathlineto{\pgfqpoint{6.723452in}{1.086085in}}%
\pgfpathlineto{\pgfqpoint{6.670408in}{1.040003in}}%
\pgfpathlineto{\pgfqpoint{6.617364in}{1.010061in}}%
\pgfpathlineto{\pgfqpoint{6.564320in}{0.943104in}}%
\pgfpathlineto{\pgfqpoint{6.511276in}{0.946754in}}%
\pgfpathlineto{\pgfqpoint{6.458232in}{1.413279in}}%
\pgfpathlineto{\pgfqpoint{6.405188in}{1.891220in}}%
\pgfpathlineto{\pgfqpoint{6.352144in}{1.833058in}}%
\pgfpathlineto{\pgfqpoint{6.299100in}{0.877160in}}%
\pgfpathlineto{\pgfqpoint{6.246056in}{1.891220in}}%
\pgfpathlineto{\pgfqpoint{6.193012in}{1.891220in}}%
\pgfpathlineto{\pgfqpoint{6.139969in}{0.956658in}}%
\pgfpathlineto{\pgfqpoint{6.086925in}{0.975132in}}%
\pgfpathlineto{\pgfqpoint{6.033881in}{1.678405in}}%
\pgfpathlineto{\pgfqpoint{5.980837in}{1.046269in}}%
\pgfpathlineto{\pgfqpoint{5.927793in}{1.891220in}}%
\pgfpathlineto{\pgfqpoint{5.874749in}{1.891220in}}%
\pgfpathlineto{\pgfqpoint{5.821705in}{1.891220in}}%
\pgfpathlineto{\pgfqpoint{5.768661in}{1.891220in}}%
\pgfpathlineto{\pgfqpoint{5.715617in}{1.122626in}}%
\pgfpathlineto{\pgfqpoint{5.662573in}{1.891220in}}%
\pgfpathlineto{\pgfqpoint{5.609529in}{1.891220in}}%
\pgfpathlineto{\pgfqpoint{5.556485in}{1.549519in}}%
\pgfpathlineto{\pgfqpoint{5.503441in}{1.116833in}}%
\pgfpathlineto{\pgfqpoint{5.450397in}{1.891220in}}%
\pgfpathlineto{\pgfqpoint{5.397353in}{1.891220in}}%
\pgfpathlineto{\pgfqpoint{5.344309in}{1.891220in}}%
\pgfpathlineto{\pgfqpoint{5.291266in}{0.975653in}}%
\pgfpathlineto{\pgfqpoint{5.238222in}{1.891220in}}%
\pgfpathlineto{\pgfqpoint{5.185178in}{1.891220in}}%
\pgfpathlineto{\pgfqpoint{5.132134in}{1.891220in}}%
\pgfpathlineto{\pgfqpoint{5.079090in}{0.827666in}}%
\pgfpathlineto{\pgfqpoint{5.026046in}{1.891220in}}%
\pgfpathlineto{\pgfqpoint{4.973002in}{0.856126in}}%
\pgfpathlineto{\pgfqpoint{4.919958in}{1.891220in}}%
\pgfpathlineto{\pgfqpoint{4.866914in}{1.891220in}}%
\pgfpathlineto{\pgfqpoint{4.813870in}{0.975129in}}%
\pgfpathlineto{\pgfqpoint{4.760826in}{1.287681in}}%
\pgfpathlineto{\pgfqpoint{4.707782in}{1.891220in}}%
\pgfpathlineto{\pgfqpoint{4.654738in}{1.094865in}}%
\pgfpathlineto{\pgfqpoint{4.601694in}{1.891220in}}%
\pgfpathlineto{\pgfqpoint{4.548650in}{1.437632in}}%
\pgfpathlineto{\pgfqpoint{4.495607in}{1.891220in}}%
\pgfpathlineto{\pgfqpoint{4.442563in}{1.580227in}}%
\pgfpathlineto{\pgfqpoint{4.389519in}{1.243678in}}%
\pgfpathlineto{\pgfqpoint{4.336475in}{1.891220in}}%
\pgfpathlineto{\pgfqpoint{4.283431in}{1.093254in}}%
\pgfpathlineto{\pgfqpoint{4.230387in}{1.077535in}}%
\pgfpathlineto{\pgfqpoint{4.177343in}{1.039428in}}%
\pgfpathlineto{\pgfqpoint{4.124299in}{1.123707in}}%
\pgfpathlineto{\pgfqpoint{4.071255in}{0.995539in}}%
\pgfpathlineto{\pgfqpoint{4.018211in}{0.913448in}}%
\pgfpathlineto{\pgfqpoint{3.965167in}{1.724076in}}%
\pgfpathlineto{\pgfqpoint{3.912123in}{1.426054in}}%
\pgfpathlineto{\pgfqpoint{3.859079in}{1.517418in}}%
\pgfpathlineto{\pgfqpoint{3.806035in}{0.907473in}}%
\pgfpathlineto{\pgfqpoint{3.752991in}{1.454919in}}%
\pgfpathlineto{\pgfqpoint{3.699948in}{0.867222in}}%
\pgfpathlineto{\pgfqpoint{3.646904in}{0.911072in}}%
\pgfpathlineto{\pgfqpoint{3.593860in}{1.388474in}}%
\pgfpathlineto{\pgfqpoint{3.540816in}{1.891220in}}%
\pgfpathlineto{\pgfqpoint{3.487772in}{1.891220in}}%
\pgfpathlineto{\pgfqpoint{3.434728in}{0.972772in}}%
\pgfpathlineto{\pgfqpoint{3.381684in}{1.666472in}}%
\pgfpathlineto{\pgfqpoint{3.328640in}{1.120386in}}%
\pgfpathlineto{\pgfqpoint{3.275596in}{1.061811in}}%
\pgfpathlineto{\pgfqpoint{3.222552in}{1.145671in}}%
\pgfpathlineto{\pgfqpoint{3.169508in}{1.115148in}}%
\pgfpathlineto{\pgfqpoint{3.116464in}{1.557584in}}%
\pgfpathlineto{\pgfqpoint{3.063420in}{1.891220in}}%
\pgfpathlineto{\pgfqpoint{3.010376in}{1.083328in}}%
\pgfpathlineto{\pgfqpoint{2.957332in}{1.891220in}}%
\pgfpathlineto{\pgfqpoint{2.904288in}{0.997564in}}%
\pgfpathlineto{\pgfqpoint{2.851245in}{1.781408in}}%
\pgfpathlineto{\pgfqpoint{2.798201in}{0.950657in}}%
\pgfpathlineto{\pgfqpoint{2.745157in}{0.908262in}}%
\pgfpathlineto{\pgfqpoint{2.692113in}{0.837338in}}%
\pgfpathlineto{\pgfqpoint{2.639069in}{0.867540in}}%
\pgfpathlineto{\pgfqpoint{2.586025in}{0.872034in}}%
\pgfpathlineto{\pgfqpoint{2.532981in}{0.893845in}}%
\pgfpathlineto{\pgfqpoint{2.479937in}{1.038089in}}%
\pgfpathlineto{\pgfqpoint{2.426893in}{1.891220in}}%
\pgfpathlineto{\pgfqpoint{2.373849in}{1.891220in}}%
\pgfpathlineto{\pgfqpoint{2.320805in}{0.918169in}}%
\pgfpathlineto{\pgfqpoint{2.267761in}{1.182556in}}%
\pgfpathlineto{\pgfqpoint{2.214717in}{1.891220in}}%
\pgfpathlineto{\pgfqpoint{2.161673in}{0.987578in}}%
\pgfpathlineto{\pgfqpoint{2.108629in}{1.481664in}}%
\pgfpathlineto{\pgfqpoint{2.055586in}{1.097929in}}%
\pgfpathlineto{\pgfqpoint{2.002542in}{1.891220in}}%
\pgfpathlineto{\pgfqpoint{1.949498in}{1.087124in}}%
\pgfpathlineto{\pgfqpoint{1.896454in}{1.891220in}}%
\pgfpathlineto{\pgfqpoint{1.843410in}{1.891220in}}%
\pgfpathlineto{\pgfqpoint{1.790366in}{1.113813in}}%
\pgfpathlineto{\pgfqpoint{1.737322in}{1.131392in}}%
\pgfpathlineto{\pgfqpoint{1.684278in}{1.559075in}}%
\pgfpathlineto{\pgfqpoint{1.631234in}{1.891220in}}%
\pgfpathlineto{\pgfqpoint{1.578190in}{0.966407in}}%
\pgfpathlineto{\pgfqpoint{1.525146in}{0.989277in}}%
\pgfpathlineto{\pgfqpoint{1.472102in}{0.908915in}}%
\pgfpathlineto{\pgfqpoint{1.419058in}{0.903552in}}%
\pgfpathlineto{\pgfqpoint{1.366014in}{0.829335in}}%
\pgfpathlineto{\pgfqpoint{1.312970in}{1.856216in}}%
\pgfpathlineto{\pgfqpoint{1.259927in}{1.891220in}}%
\pgfpathlineto{\pgfqpoint{1.206883in}{1.810992in}}%
\pgfpathlineto{\pgfqpoint{1.153839in}{0.893679in}}%
\pgfpathlineto{\pgfqpoint{1.100795in}{0.907342in}}%
\pgfpathlineto{\pgfqpoint{1.047751in}{0.938123in}}%
\pgfpathlineto{\pgfqpoint{0.994707in}{1.206100in}}%
\pgfpathlineto{\pgfqpoint{0.941663in}{0.942172in}}%
\pgfpathlineto{\pgfqpoint{0.941663in}{0.942172in}}%
\pgfpathclose%
\pgfusepath{stroke,fill}%
}%
\begin{pgfscope}%
\pgfsys@transformshift{0.000000in}{0.000000in}%
\pgfsys@useobject{currentmarker}{}%
\end{pgfscope}%
\end{pgfscope}%
\begin{pgfscope}%
\pgfpathrectangle{\pgfqpoint{0.941663in}{0.670138in}}{\pgfqpoint{8.858337in}{3.465625in}}%
\pgfusepath{clip}%
\pgfsetrectcap%
\pgfsetroundjoin%
\pgfsetlinewidth{1.505625pt}%
\definecolor{currentstroke}{rgb}{0.090196,0.745098,0.811765}%
\pgfsetstrokecolor{currentstroke}%
\pgfsetdash{}{0pt}%
\pgfpathmoveto{\pgfqpoint{0.941663in}{3.978235in}}%
\pgfpathlineto{\pgfqpoint{0.994707in}{3.724101in}}%
\pgfpathlineto{\pgfqpoint{1.047751in}{3.978235in}}%
\pgfpathlineto{\pgfqpoint{1.100795in}{3.966545in}}%
\pgfpathlineto{\pgfqpoint{1.153839in}{3.978235in}}%
\pgfpathlineto{\pgfqpoint{1.206883in}{3.101900in}}%
\pgfpathlineto{\pgfqpoint{1.259927in}{2.902493in}}%
\pgfpathlineto{\pgfqpoint{1.312970in}{3.006099in}}%
\pgfpathlineto{\pgfqpoint{1.366014in}{3.978235in}}%
\pgfpathlineto{\pgfqpoint{1.578190in}{3.978235in}}%
\pgfpathlineto{\pgfqpoint{1.631234in}{3.123835in}}%
\pgfpathlineto{\pgfqpoint{1.684278in}{3.440895in}}%
\pgfpathlineto{\pgfqpoint{1.737322in}{3.872158in}}%
\pgfpathlineto{\pgfqpoint{1.790366in}{3.978235in}}%
\pgfpathlineto{\pgfqpoint{1.843410in}{3.247418in}}%
\pgfpathlineto{\pgfqpoint{1.896454in}{3.219351in}}%
\pgfpathlineto{\pgfqpoint{1.949498in}{3.978235in}}%
\pgfpathlineto{\pgfqpoint{2.002542in}{3.193502in}}%
\pgfpathlineto{\pgfqpoint{2.055586in}{3.978235in}}%
\pgfpathlineto{\pgfqpoint{2.108629in}{3.577804in}}%
\pgfpathlineto{\pgfqpoint{2.161673in}{3.978235in}}%
\pgfpathlineto{\pgfqpoint{2.214717in}{3.089112in}}%
\pgfpathlineto{\pgfqpoint{2.267761in}{3.754982in}}%
\pgfpathlineto{\pgfqpoint{2.320805in}{3.978235in}}%
\pgfpathlineto{\pgfqpoint{2.373849in}{2.980147in}}%
\pgfpathlineto{\pgfqpoint{2.426893in}{2.942415in}}%
\pgfpathlineto{\pgfqpoint{2.479937in}{3.785532in}}%
\pgfpathlineto{\pgfqpoint{2.532981in}{3.978235in}}%
\pgfpathlineto{\pgfqpoint{2.798201in}{3.978235in}}%
\pgfpathlineto{\pgfqpoint{2.851245in}{3.161616in}}%
\pgfpathlineto{\pgfqpoint{2.904288in}{3.978235in}}%
\pgfpathlineto{\pgfqpoint{2.957332in}{3.172295in}}%
\pgfpathlineto{\pgfqpoint{3.010376in}{3.978235in}}%
\pgfpathlineto{\pgfqpoint{3.063420in}{3.151674in}}%
\pgfpathlineto{\pgfqpoint{3.116464in}{3.506190in}}%
\pgfpathlineto{\pgfqpoint{3.169508in}{3.978235in}}%
\pgfpathlineto{\pgfqpoint{3.328640in}{3.978235in}}%
\pgfpathlineto{\pgfqpoint{3.381684in}{3.349122in}}%
\pgfpathlineto{\pgfqpoint{3.434728in}{3.978235in}}%
\pgfpathlineto{\pgfqpoint{3.487772in}{3.083924in}}%
\pgfpathlineto{\pgfqpoint{3.540816in}{3.074646in}}%
\pgfpathlineto{\pgfqpoint{3.593860in}{3.503387in}}%
\pgfpathlineto{\pgfqpoint{3.646904in}{3.978235in}}%
\pgfpathlineto{\pgfqpoint{3.699948in}{3.978235in}}%
\pgfpathlineto{\pgfqpoint{3.752991in}{3.388260in}}%
\pgfpathlineto{\pgfqpoint{3.806035in}{3.978235in}}%
\pgfpathlineto{\pgfqpoint{3.859079in}{3.355689in}}%
\pgfpathlineto{\pgfqpoint{3.912123in}{3.408107in}}%
\pgfpathlineto{\pgfqpoint{3.965167in}{3.238518in}}%
\pgfpathlineto{\pgfqpoint{4.018211in}{3.978235in}}%
\pgfpathlineto{\pgfqpoint{4.071255in}{3.978235in}}%
\pgfpathlineto{\pgfqpoint{4.124299in}{3.875484in}}%
\pgfpathlineto{\pgfqpoint{4.177343in}{3.978235in}}%
\pgfpathlineto{\pgfqpoint{4.283431in}{3.978235in}}%
\pgfpathlineto{\pgfqpoint{4.336475in}{3.224846in}}%
\pgfpathlineto{\pgfqpoint{4.389519in}{3.841843in}}%
\pgfpathlineto{\pgfqpoint{4.442563in}{3.494347in}}%
\pgfpathlineto{\pgfqpoint{4.495607in}{3.218773in}}%
\pgfpathlineto{\pgfqpoint{4.548650in}{3.689185in}}%
\pgfpathlineto{\pgfqpoint{4.601694in}{3.162876in}}%
\pgfpathlineto{\pgfqpoint{4.654738in}{3.978235in}}%
\pgfpathlineto{\pgfqpoint{4.707782in}{3.102021in}}%
\pgfpathlineto{\pgfqpoint{4.760826in}{3.714472in}}%
\pgfpathlineto{\pgfqpoint{4.813870in}{3.978235in}}%
\pgfpathlineto{\pgfqpoint{4.866914in}{3.020692in}}%
\pgfpathlineto{\pgfqpoint{4.919958in}{3.007061in}}%
\pgfpathlineto{\pgfqpoint{4.973002in}{3.978235in}}%
\pgfpathlineto{\pgfqpoint{5.026046in}{2.958572in}}%
\pgfpathlineto{\pgfqpoint{5.079090in}{3.978235in}}%
\pgfpathlineto{\pgfqpoint{5.132134in}{2.918032in}}%
\pgfpathlineto{\pgfqpoint{5.185178in}{3.006017in}}%
\pgfpathlineto{\pgfqpoint{5.238222in}{3.002435in}}%
\pgfpathlineto{\pgfqpoint{5.291266in}{3.978235in}}%
\pgfpathlineto{\pgfqpoint{5.344309in}{3.073166in}}%
\pgfpathlineto{\pgfqpoint{5.397353in}{3.084080in}}%
\pgfpathlineto{\pgfqpoint{5.450397in}{3.213681in}}%
\pgfpathlineto{\pgfqpoint{5.503441in}{3.978235in}}%
\pgfpathlineto{\pgfqpoint{5.556485in}{3.538355in}}%
\pgfpathlineto{\pgfqpoint{5.609529in}{3.209959in}}%
\pgfpathlineto{\pgfqpoint{5.662573in}{3.261554in}}%
\pgfpathlineto{\pgfqpoint{5.715617in}{3.978235in}}%
\pgfpathlineto{\pgfqpoint{5.768661in}{3.248973in}}%
\pgfpathlineto{\pgfqpoint{5.821705in}{3.208761in}}%
\pgfpathlineto{\pgfqpoint{5.874749in}{3.132435in}}%
\pgfpathlineto{\pgfqpoint{5.927793in}{3.156691in}}%
\pgfpathlineto{\pgfqpoint{5.980837in}{3.978235in}}%
\pgfpathlineto{\pgfqpoint{6.033881in}{3.296892in}}%
\pgfpathlineto{\pgfqpoint{6.086925in}{3.978235in}}%
\pgfpathlineto{\pgfqpoint{6.139969in}{3.978235in}}%
\pgfpathlineto{\pgfqpoint{6.193012in}{2.971161in}}%
\pgfpathlineto{\pgfqpoint{6.246056in}{3.026875in}}%
\pgfpathlineto{\pgfqpoint{6.299100in}{3.978235in}}%
\pgfpathlineto{\pgfqpoint{6.352144in}{3.041125in}}%
\pgfpathlineto{\pgfqpoint{6.405188in}{2.929750in}}%
\pgfpathlineto{\pgfqpoint{6.458232in}{3.493957in}}%
\pgfpathlineto{\pgfqpoint{6.511276in}{3.978235in}}%
\pgfpathlineto{\pgfqpoint{6.723452in}{3.978235in}}%
\pgfpathlineto{\pgfqpoint{6.776496in}{3.755800in}}%
\pgfpathlineto{\pgfqpoint{6.829540in}{3.978235in}}%
\pgfpathlineto{\pgfqpoint{6.882584in}{3.217123in}}%
\pgfpathlineto{\pgfqpoint{6.935628in}{3.978235in}}%
\pgfpathlineto{\pgfqpoint{6.988671in}{3.429108in}}%
\pgfpathlineto{\pgfqpoint{7.041715in}{3.978235in}}%
\pgfpathlineto{\pgfqpoint{7.147803in}{3.978235in}}%
\pgfpathlineto{\pgfqpoint{7.200847in}{3.153935in}}%
\pgfpathlineto{\pgfqpoint{7.253891in}{3.485903in}}%
\pgfpathlineto{\pgfqpoint{7.306935in}{3.878601in}}%
\pgfpathlineto{\pgfqpoint{7.359979in}{3.007952in}}%
\pgfpathlineto{\pgfqpoint{7.413023in}{3.019856in}}%
\pgfpathlineto{\pgfqpoint{7.466067in}{3.397755in}}%
\pgfpathlineto{\pgfqpoint{7.519111in}{3.422724in}}%
\pgfpathlineto{\pgfqpoint{7.572155in}{2.929451in}}%
\pgfpathlineto{\pgfqpoint{7.625199in}{2.988504in}}%
\pgfpathlineto{\pgfqpoint{7.678243in}{3.074893in}}%
\pgfpathlineto{\pgfqpoint{7.731287in}{3.978235in}}%
\pgfpathlineto{\pgfqpoint{7.996506in}{3.978235in}}%
\pgfpathlineto{\pgfqpoint{8.049550in}{3.189369in}}%
\pgfpathlineto{\pgfqpoint{8.102594in}{3.198813in}}%
\pgfpathlineto{\pgfqpoint{8.155638in}{3.978235in}}%
\pgfpathlineto{\pgfqpoint{8.208682in}{3.978235in}}%
\pgfpathlineto{\pgfqpoint{8.261726in}{3.260476in}}%
\pgfpathlineto{\pgfqpoint{8.314770in}{3.211850in}}%
\pgfpathlineto{\pgfqpoint{8.367814in}{3.978235in}}%
\pgfpathlineto{\pgfqpoint{8.420858in}{3.193239in}}%
\pgfpathlineto{\pgfqpoint{8.473902in}{3.978235in}}%
\pgfpathlineto{\pgfqpoint{8.526946in}{3.978235in}}%
\pgfpathlineto{\pgfqpoint{8.579990in}{3.021265in}}%
\pgfpathlineto{\pgfqpoint{8.633033in}{3.030538in}}%
\pgfpathlineto{\pgfqpoint{8.686077in}{3.059450in}}%
\pgfpathlineto{\pgfqpoint{8.739121in}{2.997851in}}%
\pgfpathlineto{\pgfqpoint{8.792165in}{3.135951in}}%
\pgfpathlineto{\pgfqpoint{8.845209in}{2.978670in}}%
\pgfpathlineto{\pgfqpoint{8.898253in}{3.021456in}}%
\pgfpathlineto{\pgfqpoint{8.951297in}{3.978235in}}%
\pgfpathlineto{\pgfqpoint{9.004341in}{3.231041in}}%
\pgfpathlineto{\pgfqpoint{9.057385in}{3.978235in}}%
\pgfpathlineto{\pgfqpoint{9.110429in}{3.978235in}}%
\pgfpathlineto{\pgfqpoint{9.163473in}{3.106410in}}%
\pgfpathlineto{\pgfqpoint{9.216517in}{3.140256in}}%
\pgfpathlineto{\pgfqpoint{9.269561in}{3.147248in}}%
\pgfpathlineto{\pgfqpoint{9.322605in}{3.219668in}}%
\pgfpathlineto{\pgfqpoint{9.375649in}{3.214984in}}%
\pgfpathlineto{\pgfqpoint{9.428692in}{3.978235in}}%
\pgfpathlineto{\pgfqpoint{9.481736in}{3.910858in}}%
\pgfpathlineto{\pgfqpoint{9.534780in}{3.282563in}}%
\pgfpathlineto{\pgfqpoint{9.587824in}{3.204226in}}%
\pgfpathlineto{\pgfqpoint{9.640868in}{3.473714in}}%
\pgfpathlineto{\pgfqpoint{9.693912in}{3.139443in}}%
\pgfpathlineto{\pgfqpoint{9.746956in}{3.115475in}}%
\pgfpathlineto{\pgfqpoint{9.800000in}{3.101625in}}%
\pgfpathlineto{\pgfqpoint{9.800000in}{3.101625in}}%
\pgfusepath{stroke}%
\end{pgfscope}%
\begin{pgfscope}%
\pgfpathrectangle{\pgfqpoint{0.941663in}{0.670138in}}{\pgfqpoint{8.858337in}{3.465625in}}%
\pgfusepath{clip}%
\pgfsetbuttcap%
\pgfsetroundjoin%
\definecolor{currentfill}{rgb}{0.090196,0.745098,0.811765}%
\pgfsetfillcolor{currentfill}%
\pgfsetlinewidth{1.003750pt}%
\definecolor{currentstroke}{rgb}{0.090196,0.745098,0.811765}%
\pgfsetstrokecolor{currentstroke}%
\pgfsetdash{}{0pt}%
\pgfsys@defobject{currentmarker}{\pgfqpoint{0.941663in}{2.586892in}}{\pgfqpoint{9.800000in}{3.978235in}}{%
\pgfpathmoveto{\pgfqpoint{0.941663in}{3.978235in}}%
\pgfpathlineto{\pgfqpoint{0.941663in}{2.586892in}}%
\pgfpathlineto{\pgfqpoint{0.994707in}{2.586892in}}%
\pgfpathlineto{\pgfqpoint{1.047751in}{2.586892in}}%
\pgfpathlineto{\pgfqpoint{1.100795in}{2.586892in}}%
\pgfpathlineto{\pgfqpoint{1.153839in}{2.586892in}}%
\pgfpathlineto{\pgfqpoint{1.206883in}{2.586892in}}%
\pgfpathlineto{\pgfqpoint{1.259927in}{2.749185in}}%
\pgfpathlineto{\pgfqpoint{1.312970in}{2.586892in}}%
\pgfpathlineto{\pgfqpoint{1.366014in}{2.586892in}}%
\pgfpathlineto{\pgfqpoint{1.419058in}{2.586892in}}%
\pgfpathlineto{\pgfqpoint{1.472102in}{2.586892in}}%
\pgfpathlineto{\pgfqpoint{1.525146in}{2.586892in}}%
\pgfpathlineto{\pgfqpoint{1.578190in}{2.586892in}}%
\pgfpathlineto{\pgfqpoint{1.631234in}{2.802326in}}%
\pgfpathlineto{\pgfqpoint{1.684278in}{2.586892in}}%
\pgfpathlineto{\pgfqpoint{1.737322in}{2.586892in}}%
\pgfpathlineto{\pgfqpoint{1.790366in}{2.586892in}}%
\pgfpathlineto{\pgfqpoint{1.843410in}{3.070356in}}%
\pgfpathlineto{\pgfqpoint{1.896454in}{3.219351in}}%
\pgfpathlineto{\pgfqpoint{1.949498in}{2.586892in}}%
\pgfpathlineto{\pgfqpoint{2.002542in}{3.060121in}}%
\pgfpathlineto{\pgfqpoint{2.055586in}{2.586892in}}%
\pgfpathlineto{\pgfqpoint{2.108629in}{2.586892in}}%
\pgfpathlineto{\pgfqpoint{2.161673in}{2.586892in}}%
\pgfpathlineto{\pgfqpoint{2.214717in}{2.910735in}}%
\pgfpathlineto{\pgfqpoint{2.267761in}{2.586892in}}%
\pgfpathlineto{\pgfqpoint{2.320805in}{2.586892in}}%
\pgfpathlineto{\pgfqpoint{2.373849in}{2.844885in}}%
\pgfpathlineto{\pgfqpoint{2.426893in}{2.942415in}}%
\pgfpathlineto{\pgfqpoint{2.479937in}{2.586892in}}%
\pgfpathlineto{\pgfqpoint{2.532981in}{2.586892in}}%
\pgfpathlineto{\pgfqpoint{2.586025in}{2.586892in}}%
\pgfpathlineto{\pgfqpoint{2.639069in}{2.586892in}}%
\pgfpathlineto{\pgfqpoint{2.692113in}{2.586892in}}%
\pgfpathlineto{\pgfqpoint{2.745157in}{2.586892in}}%
\pgfpathlineto{\pgfqpoint{2.798201in}{2.586892in}}%
\pgfpathlineto{\pgfqpoint{2.851245in}{2.586892in}}%
\pgfpathlineto{\pgfqpoint{2.904288in}{2.586892in}}%
\pgfpathlineto{\pgfqpoint{2.957332in}{3.080511in}}%
\pgfpathlineto{\pgfqpoint{3.010376in}{2.586892in}}%
\pgfpathlineto{\pgfqpoint{3.063420in}{2.909016in}}%
\pgfpathlineto{\pgfqpoint{3.116464in}{2.586892in}}%
\pgfpathlineto{\pgfqpoint{3.169508in}{2.586892in}}%
\pgfpathlineto{\pgfqpoint{3.222552in}{2.586892in}}%
\pgfpathlineto{\pgfqpoint{3.275596in}{2.586892in}}%
\pgfpathlineto{\pgfqpoint{3.328640in}{2.586892in}}%
\pgfpathlineto{\pgfqpoint{3.381684in}{2.586892in}}%
\pgfpathlineto{\pgfqpoint{3.434728in}{2.586892in}}%
\pgfpathlineto{\pgfqpoint{3.487772in}{2.710315in}}%
\pgfpathlineto{\pgfqpoint{3.540816in}{2.908402in}}%
\pgfpathlineto{\pgfqpoint{3.593860in}{2.586892in}}%
\pgfpathlineto{\pgfqpoint{3.646904in}{2.586892in}}%
\pgfpathlineto{\pgfqpoint{3.699948in}{2.586892in}}%
\pgfpathlineto{\pgfqpoint{3.752991in}{2.586892in}}%
\pgfpathlineto{\pgfqpoint{3.806035in}{2.586892in}}%
\pgfpathlineto{\pgfqpoint{3.859079in}{2.586892in}}%
\pgfpathlineto{\pgfqpoint{3.912123in}{2.586892in}}%
\pgfpathlineto{\pgfqpoint{3.965167in}{2.586892in}}%
\pgfpathlineto{\pgfqpoint{4.018211in}{2.586892in}}%
\pgfpathlineto{\pgfqpoint{4.071255in}{2.586892in}}%
\pgfpathlineto{\pgfqpoint{4.124299in}{2.586892in}}%
\pgfpathlineto{\pgfqpoint{4.177343in}{2.586892in}}%
\pgfpathlineto{\pgfqpoint{4.230387in}{2.586892in}}%
\pgfpathlineto{\pgfqpoint{4.283431in}{2.586892in}}%
\pgfpathlineto{\pgfqpoint{4.336475in}{3.224846in}}%
\pgfpathlineto{\pgfqpoint{4.389519in}{2.586892in}}%
\pgfpathlineto{\pgfqpoint{4.442563in}{2.586892in}}%
\pgfpathlineto{\pgfqpoint{4.495607in}{2.968775in}}%
\pgfpathlineto{\pgfqpoint{4.548650in}{2.586892in}}%
\pgfpathlineto{\pgfqpoint{4.601694in}{2.927863in}}%
\pgfpathlineto{\pgfqpoint{4.654738in}{2.586892in}}%
\pgfpathlineto{\pgfqpoint{4.707782in}{3.102021in}}%
\pgfpathlineto{\pgfqpoint{4.760826in}{2.586892in}}%
\pgfpathlineto{\pgfqpoint{4.813870in}{2.586892in}}%
\pgfpathlineto{\pgfqpoint{4.866914in}{2.884645in}}%
\pgfpathlineto{\pgfqpoint{4.919958in}{2.863950in}}%
\pgfpathlineto{\pgfqpoint{4.973002in}{2.586892in}}%
\pgfpathlineto{\pgfqpoint{5.026046in}{2.758322in}}%
\pgfpathlineto{\pgfqpoint{5.079090in}{2.586892in}}%
\pgfpathlineto{\pgfqpoint{5.132134in}{2.824147in}}%
\pgfpathlineto{\pgfqpoint{5.185178in}{2.910114in}}%
\pgfpathlineto{\pgfqpoint{5.238222in}{2.837414in}}%
\pgfpathlineto{\pgfqpoint{5.291266in}{2.586892in}}%
\pgfpathlineto{\pgfqpoint{5.344309in}{2.705202in}}%
\pgfpathlineto{\pgfqpoint{5.397353in}{2.660769in}}%
\pgfpathlineto{\pgfqpoint{5.450397in}{3.105789in}}%
\pgfpathlineto{\pgfqpoint{5.503441in}{2.586892in}}%
\pgfpathlineto{\pgfqpoint{5.556485in}{2.586892in}}%
\pgfpathlineto{\pgfqpoint{5.609529in}{2.850209in}}%
\pgfpathlineto{\pgfqpoint{5.662573in}{2.711726in}}%
\pgfpathlineto{\pgfqpoint{5.715617in}{2.586892in}}%
\pgfpathlineto{\pgfqpoint{5.768661in}{3.248973in}}%
\pgfpathlineto{\pgfqpoint{5.821705in}{3.208761in}}%
\pgfpathlineto{\pgfqpoint{5.874749in}{3.037838in}}%
\pgfpathlineto{\pgfqpoint{5.927793in}{2.934114in}}%
\pgfpathlineto{\pgfqpoint{5.980837in}{2.586892in}}%
\pgfpathlineto{\pgfqpoint{6.033881in}{2.586892in}}%
\pgfpathlineto{\pgfqpoint{6.086925in}{2.586892in}}%
\pgfpathlineto{\pgfqpoint{6.139969in}{2.586892in}}%
\pgfpathlineto{\pgfqpoint{6.193012in}{2.971161in}}%
\pgfpathlineto{\pgfqpoint{6.246056in}{3.026875in}}%
\pgfpathlineto{\pgfqpoint{6.299100in}{2.586892in}}%
\pgfpathlineto{\pgfqpoint{6.352144in}{2.586892in}}%
\pgfpathlineto{\pgfqpoint{6.405188in}{2.699818in}}%
\pgfpathlineto{\pgfqpoint{6.458232in}{2.586892in}}%
\pgfpathlineto{\pgfqpoint{6.511276in}{2.586892in}}%
\pgfpathlineto{\pgfqpoint{6.564320in}{2.586892in}}%
\pgfpathlineto{\pgfqpoint{6.617364in}{2.586892in}}%
\pgfpathlineto{\pgfqpoint{6.670408in}{2.586892in}}%
\pgfpathlineto{\pgfqpoint{6.723452in}{2.586892in}}%
\pgfpathlineto{\pgfqpoint{6.776496in}{2.586892in}}%
\pgfpathlineto{\pgfqpoint{6.829540in}{2.586892in}}%
\pgfpathlineto{\pgfqpoint{6.882584in}{3.116001in}}%
\pgfpathlineto{\pgfqpoint{6.935628in}{2.586892in}}%
\pgfpathlineto{\pgfqpoint{6.988671in}{2.586892in}}%
\pgfpathlineto{\pgfqpoint{7.041715in}{2.586892in}}%
\pgfpathlineto{\pgfqpoint{7.094759in}{2.586892in}}%
\pgfpathlineto{\pgfqpoint{7.147803in}{2.586892in}}%
\pgfpathlineto{\pgfqpoint{7.200847in}{3.064989in}}%
\pgfpathlineto{\pgfqpoint{7.253891in}{2.586892in}}%
\pgfpathlineto{\pgfqpoint{7.306935in}{2.586892in}}%
\pgfpathlineto{\pgfqpoint{7.359979in}{2.806406in}}%
\pgfpathlineto{\pgfqpoint{7.413023in}{2.865454in}}%
\pgfpathlineto{\pgfqpoint{7.466067in}{2.586892in}}%
\pgfpathlineto{\pgfqpoint{7.519111in}{2.586892in}}%
\pgfpathlineto{\pgfqpoint{7.572155in}{2.823725in}}%
\pgfpathlineto{\pgfqpoint{7.625199in}{2.988504in}}%
\pgfpathlineto{\pgfqpoint{7.678243in}{2.586892in}}%
\pgfpathlineto{\pgfqpoint{7.731287in}{2.586892in}}%
\pgfpathlineto{\pgfqpoint{7.784330in}{2.586892in}}%
\pgfpathlineto{\pgfqpoint{7.837374in}{2.586892in}}%
\pgfpathlineto{\pgfqpoint{7.890418in}{2.586892in}}%
\pgfpathlineto{\pgfqpoint{7.943462in}{2.586892in}}%
\pgfpathlineto{\pgfqpoint{7.996506in}{2.586892in}}%
\pgfpathlineto{\pgfqpoint{8.049550in}{3.102247in}}%
\pgfpathlineto{\pgfqpoint{8.102594in}{3.118508in}}%
\pgfpathlineto{\pgfqpoint{8.155638in}{2.586892in}}%
\pgfpathlineto{\pgfqpoint{8.208682in}{2.586892in}}%
\pgfpathlineto{\pgfqpoint{8.261726in}{3.260476in}}%
\pgfpathlineto{\pgfqpoint{8.314770in}{3.211850in}}%
\pgfpathlineto{\pgfqpoint{8.367814in}{2.586892in}}%
\pgfpathlineto{\pgfqpoint{8.420858in}{2.989399in}}%
\pgfpathlineto{\pgfqpoint{8.473902in}{2.586892in}}%
\pgfpathlineto{\pgfqpoint{8.526946in}{2.586892in}}%
\pgfpathlineto{\pgfqpoint{8.579990in}{3.021265in}}%
\pgfpathlineto{\pgfqpoint{8.633033in}{2.856358in}}%
\pgfpathlineto{\pgfqpoint{8.686077in}{3.059450in}}%
\pgfpathlineto{\pgfqpoint{8.739121in}{2.825404in}}%
\pgfpathlineto{\pgfqpoint{8.792165in}{2.586892in}}%
\pgfpathlineto{\pgfqpoint{8.845209in}{2.634872in}}%
\pgfpathlineto{\pgfqpoint{8.898253in}{2.689958in}}%
\pgfpathlineto{\pgfqpoint{8.951297in}{2.586892in}}%
\pgfpathlineto{\pgfqpoint{9.004341in}{2.586892in}}%
\pgfpathlineto{\pgfqpoint{9.057385in}{2.586892in}}%
\pgfpathlineto{\pgfqpoint{9.110429in}{2.586892in}}%
\pgfpathlineto{\pgfqpoint{9.163473in}{3.004188in}}%
\pgfpathlineto{\pgfqpoint{9.216517in}{3.140256in}}%
\pgfpathlineto{\pgfqpoint{9.269561in}{2.964013in}}%
\pgfpathlineto{\pgfqpoint{9.322605in}{3.063213in}}%
\pgfpathlineto{\pgfqpoint{9.375649in}{3.090533in}}%
\pgfpathlineto{\pgfqpoint{9.428692in}{2.586892in}}%
\pgfpathlineto{\pgfqpoint{9.481736in}{2.586892in}}%
\pgfpathlineto{\pgfqpoint{9.534780in}{3.282563in}}%
\pgfpathlineto{\pgfqpoint{9.587824in}{2.956774in}}%
\pgfpathlineto{\pgfqpoint{9.640868in}{2.586892in}}%
\pgfpathlineto{\pgfqpoint{9.693912in}{2.968640in}}%
\pgfpathlineto{\pgfqpoint{9.746956in}{3.115475in}}%
\pgfpathlineto{\pgfqpoint{9.800000in}{2.909416in}}%
\pgfpathlineto{\pgfqpoint{9.800000in}{3.101625in}}%
\pgfpathlineto{\pgfqpoint{9.800000in}{3.101625in}}%
\pgfpathlineto{\pgfqpoint{9.746956in}{3.115475in}}%
\pgfpathlineto{\pgfqpoint{9.693912in}{3.139443in}}%
\pgfpathlineto{\pgfqpoint{9.640868in}{3.473714in}}%
\pgfpathlineto{\pgfqpoint{9.587824in}{3.204226in}}%
\pgfpathlineto{\pgfqpoint{9.534780in}{3.282563in}}%
\pgfpathlineto{\pgfqpoint{9.481736in}{3.910858in}}%
\pgfpathlineto{\pgfqpoint{9.428692in}{3.978235in}}%
\pgfpathlineto{\pgfqpoint{9.375649in}{3.214984in}}%
\pgfpathlineto{\pgfqpoint{9.322605in}{3.219668in}}%
\pgfpathlineto{\pgfqpoint{9.269561in}{3.147248in}}%
\pgfpathlineto{\pgfqpoint{9.216517in}{3.140256in}}%
\pgfpathlineto{\pgfqpoint{9.163473in}{3.106410in}}%
\pgfpathlineto{\pgfqpoint{9.110429in}{3.978235in}}%
\pgfpathlineto{\pgfqpoint{9.057385in}{3.978235in}}%
\pgfpathlineto{\pgfqpoint{9.004341in}{3.231041in}}%
\pgfpathlineto{\pgfqpoint{8.951297in}{3.978235in}}%
\pgfpathlineto{\pgfqpoint{8.898253in}{3.021456in}}%
\pgfpathlineto{\pgfqpoint{8.845209in}{2.978670in}}%
\pgfpathlineto{\pgfqpoint{8.792165in}{3.135951in}}%
\pgfpathlineto{\pgfqpoint{8.739121in}{2.997851in}}%
\pgfpathlineto{\pgfqpoint{8.686077in}{3.059450in}}%
\pgfpathlineto{\pgfqpoint{8.633033in}{3.030538in}}%
\pgfpathlineto{\pgfqpoint{8.579990in}{3.021265in}}%
\pgfpathlineto{\pgfqpoint{8.526946in}{3.978235in}}%
\pgfpathlineto{\pgfqpoint{8.473902in}{3.978235in}}%
\pgfpathlineto{\pgfqpoint{8.420858in}{3.193239in}}%
\pgfpathlineto{\pgfqpoint{8.367814in}{3.978235in}}%
\pgfpathlineto{\pgfqpoint{8.314770in}{3.211850in}}%
\pgfpathlineto{\pgfqpoint{8.261726in}{3.260476in}}%
\pgfpathlineto{\pgfqpoint{8.208682in}{3.978235in}}%
\pgfpathlineto{\pgfqpoint{8.155638in}{3.978235in}}%
\pgfpathlineto{\pgfqpoint{8.102594in}{3.198813in}}%
\pgfpathlineto{\pgfqpoint{8.049550in}{3.189369in}}%
\pgfpathlineto{\pgfqpoint{7.996506in}{3.978235in}}%
\pgfpathlineto{\pgfqpoint{7.943462in}{3.978235in}}%
\pgfpathlineto{\pgfqpoint{7.890418in}{3.978235in}}%
\pgfpathlineto{\pgfqpoint{7.837374in}{3.978235in}}%
\pgfpathlineto{\pgfqpoint{7.784330in}{3.978235in}}%
\pgfpathlineto{\pgfqpoint{7.731287in}{3.978235in}}%
\pgfpathlineto{\pgfqpoint{7.678243in}{3.074893in}}%
\pgfpathlineto{\pgfqpoint{7.625199in}{2.988504in}}%
\pgfpathlineto{\pgfqpoint{7.572155in}{2.929451in}}%
\pgfpathlineto{\pgfqpoint{7.519111in}{3.422724in}}%
\pgfpathlineto{\pgfqpoint{7.466067in}{3.397755in}}%
\pgfpathlineto{\pgfqpoint{7.413023in}{3.019856in}}%
\pgfpathlineto{\pgfqpoint{7.359979in}{3.007952in}}%
\pgfpathlineto{\pgfqpoint{7.306935in}{3.878601in}}%
\pgfpathlineto{\pgfqpoint{7.253891in}{3.485903in}}%
\pgfpathlineto{\pgfqpoint{7.200847in}{3.153935in}}%
\pgfpathlineto{\pgfqpoint{7.147803in}{3.978235in}}%
\pgfpathlineto{\pgfqpoint{7.094759in}{3.978235in}}%
\pgfpathlineto{\pgfqpoint{7.041715in}{3.978235in}}%
\pgfpathlineto{\pgfqpoint{6.988671in}{3.429108in}}%
\pgfpathlineto{\pgfqpoint{6.935628in}{3.978235in}}%
\pgfpathlineto{\pgfqpoint{6.882584in}{3.217123in}}%
\pgfpathlineto{\pgfqpoint{6.829540in}{3.978235in}}%
\pgfpathlineto{\pgfqpoint{6.776496in}{3.755800in}}%
\pgfpathlineto{\pgfqpoint{6.723452in}{3.978235in}}%
\pgfpathlineto{\pgfqpoint{6.670408in}{3.978235in}}%
\pgfpathlineto{\pgfqpoint{6.617364in}{3.978235in}}%
\pgfpathlineto{\pgfqpoint{6.564320in}{3.978235in}}%
\pgfpathlineto{\pgfqpoint{6.511276in}{3.978235in}}%
\pgfpathlineto{\pgfqpoint{6.458232in}{3.493957in}}%
\pgfpathlineto{\pgfqpoint{6.405188in}{2.929750in}}%
\pgfpathlineto{\pgfqpoint{6.352144in}{3.041125in}}%
\pgfpathlineto{\pgfqpoint{6.299100in}{3.978235in}}%
\pgfpathlineto{\pgfqpoint{6.246056in}{3.026875in}}%
\pgfpathlineto{\pgfqpoint{6.193012in}{2.971161in}}%
\pgfpathlineto{\pgfqpoint{6.139969in}{3.978235in}}%
\pgfpathlineto{\pgfqpoint{6.086925in}{3.978235in}}%
\pgfpathlineto{\pgfqpoint{6.033881in}{3.296892in}}%
\pgfpathlineto{\pgfqpoint{5.980837in}{3.978235in}}%
\pgfpathlineto{\pgfqpoint{5.927793in}{3.156691in}}%
\pgfpathlineto{\pgfqpoint{5.874749in}{3.132435in}}%
\pgfpathlineto{\pgfqpoint{5.821705in}{3.208761in}}%
\pgfpathlineto{\pgfqpoint{5.768661in}{3.248973in}}%
\pgfpathlineto{\pgfqpoint{5.715617in}{3.978235in}}%
\pgfpathlineto{\pgfqpoint{5.662573in}{3.261554in}}%
\pgfpathlineto{\pgfqpoint{5.609529in}{3.209959in}}%
\pgfpathlineto{\pgfqpoint{5.556485in}{3.538355in}}%
\pgfpathlineto{\pgfqpoint{5.503441in}{3.978235in}}%
\pgfpathlineto{\pgfqpoint{5.450397in}{3.213681in}}%
\pgfpathlineto{\pgfqpoint{5.397353in}{3.084080in}}%
\pgfpathlineto{\pgfqpoint{5.344309in}{3.073166in}}%
\pgfpathlineto{\pgfqpoint{5.291266in}{3.978235in}}%
\pgfpathlineto{\pgfqpoint{5.238222in}{3.002435in}}%
\pgfpathlineto{\pgfqpoint{5.185178in}{3.006017in}}%
\pgfpathlineto{\pgfqpoint{5.132134in}{2.918032in}}%
\pgfpathlineto{\pgfqpoint{5.079090in}{3.978235in}}%
\pgfpathlineto{\pgfqpoint{5.026046in}{2.958572in}}%
\pgfpathlineto{\pgfqpoint{4.973002in}{3.978235in}}%
\pgfpathlineto{\pgfqpoint{4.919958in}{3.007061in}}%
\pgfpathlineto{\pgfqpoint{4.866914in}{3.020692in}}%
\pgfpathlineto{\pgfqpoint{4.813870in}{3.978235in}}%
\pgfpathlineto{\pgfqpoint{4.760826in}{3.714472in}}%
\pgfpathlineto{\pgfqpoint{4.707782in}{3.102021in}}%
\pgfpathlineto{\pgfqpoint{4.654738in}{3.978235in}}%
\pgfpathlineto{\pgfqpoint{4.601694in}{3.162876in}}%
\pgfpathlineto{\pgfqpoint{4.548650in}{3.689185in}}%
\pgfpathlineto{\pgfqpoint{4.495607in}{3.218773in}}%
\pgfpathlineto{\pgfqpoint{4.442563in}{3.494347in}}%
\pgfpathlineto{\pgfqpoint{4.389519in}{3.841843in}}%
\pgfpathlineto{\pgfqpoint{4.336475in}{3.224846in}}%
\pgfpathlineto{\pgfqpoint{4.283431in}{3.978235in}}%
\pgfpathlineto{\pgfqpoint{4.230387in}{3.978235in}}%
\pgfpathlineto{\pgfqpoint{4.177343in}{3.978235in}}%
\pgfpathlineto{\pgfqpoint{4.124299in}{3.875484in}}%
\pgfpathlineto{\pgfqpoint{4.071255in}{3.978235in}}%
\pgfpathlineto{\pgfqpoint{4.018211in}{3.978235in}}%
\pgfpathlineto{\pgfqpoint{3.965167in}{3.238518in}}%
\pgfpathlineto{\pgfqpoint{3.912123in}{3.408107in}}%
\pgfpathlineto{\pgfqpoint{3.859079in}{3.355689in}}%
\pgfpathlineto{\pgfqpoint{3.806035in}{3.978235in}}%
\pgfpathlineto{\pgfqpoint{3.752991in}{3.388260in}}%
\pgfpathlineto{\pgfqpoint{3.699948in}{3.978235in}}%
\pgfpathlineto{\pgfqpoint{3.646904in}{3.978235in}}%
\pgfpathlineto{\pgfqpoint{3.593860in}{3.503387in}}%
\pgfpathlineto{\pgfqpoint{3.540816in}{3.074646in}}%
\pgfpathlineto{\pgfqpoint{3.487772in}{3.083924in}}%
\pgfpathlineto{\pgfqpoint{3.434728in}{3.978235in}}%
\pgfpathlineto{\pgfqpoint{3.381684in}{3.349122in}}%
\pgfpathlineto{\pgfqpoint{3.328640in}{3.978235in}}%
\pgfpathlineto{\pgfqpoint{3.275596in}{3.978235in}}%
\pgfpathlineto{\pgfqpoint{3.222552in}{3.978235in}}%
\pgfpathlineto{\pgfqpoint{3.169508in}{3.978235in}}%
\pgfpathlineto{\pgfqpoint{3.116464in}{3.506190in}}%
\pgfpathlineto{\pgfqpoint{3.063420in}{3.151674in}}%
\pgfpathlineto{\pgfqpoint{3.010376in}{3.978235in}}%
\pgfpathlineto{\pgfqpoint{2.957332in}{3.172295in}}%
\pgfpathlineto{\pgfqpoint{2.904288in}{3.978235in}}%
\pgfpathlineto{\pgfqpoint{2.851245in}{3.161616in}}%
\pgfpathlineto{\pgfqpoint{2.798201in}{3.978235in}}%
\pgfpathlineto{\pgfqpoint{2.745157in}{3.978235in}}%
\pgfpathlineto{\pgfqpoint{2.692113in}{3.978235in}}%
\pgfpathlineto{\pgfqpoint{2.639069in}{3.978235in}}%
\pgfpathlineto{\pgfqpoint{2.586025in}{3.978235in}}%
\pgfpathlineto{\pgfqpoint{2.532981in}{3.978235in}}%
\pgfpathlineto{\pgfqpoint{2.479937in}{3.785532in}}%
\pgfpathlineto{\pgfqpoint{2.426893in}{2.942415in}}%
\pgfpathlineto{\pgfqpoint{2.373849in}{2.980147in}}%
\pgfpathlineto{\pgfqpoint{2.320805in}{3.978235in}}%
\pgfpathlineto{\pgfqpoint{2.267761in}{3.754982in}}%
\pgfpathlineto{\pgfqpoint{2.214717in}{3.089112in}}%
\pgfpathlineto{\pgfqpoint{2.161673in}{3.978235in}}%
\pgfpathlineto{\pgfqpoint{2.108629in}{3.577804in}}%
\pgfpathlineto{\pgfqpoint{2.055586in}{3.978235in}}%
\pgfpathlineto{\pgfqpoint{2.002542in}{3.193502in}}%
\pgfpathlineto{\pgfqpoint{1.949498in}{3.978235in}}%
\pgfpathlineto{\pgfqpoint{1.896454in}{3.219351in}}%
\pgfpathlineto{\pgfqpoint{1.843410in}{3.247418in}}%
\pgfpathlineto{\pgfqpoint{1.790366in}{3.978235in}}%
\pgfpathlineto{\pgfqpoint{1.737322in}{3.872158in}}%
\pgfpathlineto{\pgfqpoint{1.684278in}{3.440895in}}%
\pgfpathlineto{\pgfqpoint{1.631234in}{3.123835in}}%
\pgfpathlineto{\pgfqpoint{1.578190in}{3.978235in}}%
\pgfpathlineto{\pgfqpoint{1.525146in}{3.978235in}}%
\pgfpathlineto{\pgfqpoint{1.472102in}{3.978235in}}%
\pgfpathlineto{\pgfqpoint{1.419058in}{3.978235in}}%
\pgfpathlineto{\pgfqpoint{1.366014in}{3.978235in}}%
\pgfpathlineto{\pgfqpoint{1.312970in}{3.006099in}}%
\pgfpathlineto{\pgfqpoint{1.259927in}{2.902493in}}%
\pgfpathlineto{\pgfqpoint{1.206883in}{3.101900in}}%
\pgfpathlineto{\pgfqpoint{1.153839in}{3.978235in}}%
\pgfpathlineto{\pgfqpoint{1.100795in}{3.966545in}}%
\pgfpathlineto{\pgfqpoint{1.047751in}{3.978235in}}%
\pgfpathlineto{\pgfqpoint{0.994707in}{3.724101in}}%
\pgfpathlineto{\pgfqpoint{0.941663in}{3.978235in}}%
\pgfpathlineto{\pgfqpoint{0.941663in}{3.978235in}}%
\pgfpathclose%
\pgfusepath{stroke,fill}%
}%
\begin{pgfscope}%
\pgfsys@transformshift{0.000000in}{0.000000in}%
\pgfsys@useobject{currentmarker}{}%
\end{pgfscope}%
\end{pgfscope}%
\begin{pgfscope}%
\pgfsetrectcap%
\pgfsetmiterjoin%
\pgfsetlinewidth{0.803000pt}%
\definecolor{currentstroke}{rgb}{0.000000,0.000000,0.000000}%
\pgfsetstrokecolor{currentstroke}%
\pgfsetdash{}{0pt}%
\pgfpathmoveto{\pgfqpoint{0.941663in}{0.670138in}}%
\pgfpathlineto{\pgfqpoint{0.941663in}{4.135763in}}%
\pgfusepath{stroke}%
\end{pgfscope}%
\begin{pgfscope}%
\pgfsetrectcap%
\pgfsetmiterjoin%
\pgfsetlinewidth{0.803000pt}%
\definecolor{currentstroke}{rgb}{0.000000,0.000000,0.000000}%
\pgfsetstrokecolor{currentstroke}%
\pgfsetdash{}{0pt}%
\pgfpathmoveto{\pgfqpoint{9.800000in}{0.670138in}}%
\pgfpathlineto{\pgfqpoint{9.800000in}{4.135763in}}%
\pgfusepath{stroke}%
\end{pgfscope}%
\begin{pgfscope}%
\pgfsetrectcap%
\pgfsetmiterjoin%
\pgfsetlinewidth{0.803000pt}%
\definecolor{currentstroke}{rgb}{0.000000,0.000000,0.000000}%
\pgfsetstrokecolor{currentstroke}%
\pgfsetdash{}{0pt}%
\pgfpathmoveto{\pgfqpoint{0.941663in}{0.670138in}}%
\pgfpathlineto{\pgfqpoint{9.800000in}{0.670138in}}%
\pgfusepath{stroke}%
\end{pgfscope}%
\begin{pgfscope}%
\pgfsetrectcap%
\pgfsetmiterjoin%
\pgfsetlinewidth{0.803000pt}%
\definecolor{currentstroke}{rgb}{0.000000,0.000000,0.000000}%
\pgfsetstrokecolor{currentstroke}%
\pgfsetdash{}{0pt}%
\pgfpathmoveto{\pgfqpoint{0.941663in}{4.135763in}}%
\pgfpathlineto{\pgfqpoint{9.800000in}{4.135763in}}%
\pgfusepath{stroke}%
\end{pgfscope}%
\begin{pgfscope}%
\pgfpathrectangle{\pgfqpoint{0.941663in}{0.670138in}}{\pgfqpoint{8.858337in}{3.465625in}}%
\pgfusepath{clip}%
\pgfsetbuttcap%
\pgfsetroundjoin%
\pgfsetlinewidth{1.505625pt}%
\definecolor{currentstroke}{rgb}{0.000000,0.000000,0.000000}%
\pgfsetstrokecolor{currentstroke}%
\pgfsetdash{{5.550000pt}{2.400000pt}}{0.000000pt}%
\pgfpathmoveto{\pgfqpoint{0.941663in}{3.029187in}}%
\pgfpathlineto{\pgfqpoint{0.994707in}{3.038981in}}%
\pgfpathlineto{\pgfqpoint{1.047751in}{3.025138in}}%
\pgfpathlineto{\pgfqpoint{1.100795in}{2.982667in}}%
\pgfpathlineto{\pgfqpoint{1.153839in}{2.980694in}}%
\pgfpathlineto{\pgfqpoint{1.206883in}{3.021671in}}%
\pgfpathlineto{\pgfqpoint{1.259927in}{2.902493in}}%
\pgfpathlineto{\pgfqpoint{1.312970in}{2.971095in}}%
\pgfpathlineto{\pgfqpoint{1.366014in}{2.916350in}}%
\pgfpathlineto{\pgfqpoint{1.419058in}{2.990567in}}%
\pgfpathlineto{\pgfqpoint{1.472102in}{2.995930in}}%
\pgfpathlineto{\pgfqpoint{1.525146in}{3.076292in}}%
\pgfpathlineto{\pgfqpoint{1.578190in}{3.053422in}}%
\pgfpathlineto{\pgfqpoint{1.631234in}{3.123835in}}%
\pgfpathlineto{\pgfqpoint{1.684278in}{3.108750in}}%
\pgfpathlineto{\pgfqpoint{1.737322in}{3.112329in}}%
\pgfpathlineto{\pgfqpoint{1.790366in}{3.200828in}}%
\pgfpathlineto{\pgfqpoint{1.843410in}{3.247418in}}%
\pgfpathlineto{\pgfqpoint{1.896454in}{3.219351in}}%
\pgfpathlineto{\pgfqpoint{1.949498in}{3.174139in}}%
\pgfpathlineto{\pgfqpoint{2.002542in}{3.193502in}}%
\pgfpathlineto{\pgfqpoint{2.055586in}{3.184945in}}%
\pgfpathlineto{\pgfqpoint{2.108629in}{3.168248in}}%
\pgfpathlineto{\pgfqpoint{2.161673in}{3.074593in}}%
\pgfpathlineto{\pgfqpoint{2.214717in}{3.089112in}}%
\pgfpathlineto{\pgfqpoint{2.267761in}{3.046318in}}%
\pgfpathlineto{\pgfqpoint{2.320805in}{3.005184in}}%
\pgfpathlineto{\pgfqpoint{2.373849in}{2.980147in}}%
\pgfpathlineto{\pgfqpoint{2.426893in}{2.942415in}}%
\pgfpathlineto{\pgfqpoint{2.479937in}{2.932402in}}%
\pgfpathlineto{\pgfqpoint{2.532981in}{2.980860in}}%
\pgfpathlineto{\pgfqpoint{2.586025in}{2.959049in}}%
\pgfpathlineto{\pgfqpoint{2.639069in}{2.954555in}}%
\pgfpathlineto{\pgfqpoint{2.692113in}{2.924353in}}%
\pgfpathlineto{\pgfqpoint{2.745157in}{2.995277in}}%
\pgfpathlineto{\pgfqpoint{2.798201in}{3.037672in}}%
\pgfpathlineto{\pgfqpoint{2.851245in}{3.051804in}}%
\pgfpathlineto{\pgfqpoint{2.904288in}{3.084579in}}%
\pgfpathlineto{\pgfqpoint{2.957332in}{3.172295in}}%
\pgfpathlineto{\pgfqpoint{3.010376in}{3.170343in}}%
\pgfpathlineto{\pgfqpoint{3.063420in}{3.151674in}}%
\pgfpathlineto{\pgfqpoint{3.116464in}{3.172554in}}%
\pgfpathlineto{\pgfqpoint{3.169508in}{3.202163in}}%
\pgfpathlineto{\pgfqpoint{3.222552in}{3.232686in}}%
\pgfpathlineto{\pgfqpoint{3.275596in}{3.148826in}}%
\pgfpathlineto{\pgfqpoint{3.328640in}{3.207401in}}%
\pgfpathlineto{\pgfqpoint{3.381684in}{3.124374in}}%
\pgfpathlineto{\pgfqpoint{3.434728in}{3.059787in}}%
\pgfpathlineto{\pgfqpoint{3.487772in}{3.083924in}}%
\pgfpathlineto{\pgfqpoint{3.540816in}{3.074646in}}%
\pgfpathlineto{\pgfqpoint{3.593860in}{3.000641in}}%
\pgfpathlineto{\pgfqpoint{3.646904in}{2.998087in}}%
\pgfpathlineto{\pgfqpoint{3.699948in}{2.954238in}}%
\pgfpathlineto{\pgfqpoint{3.752991in}{2.951958in}}%
\pgfpathlineto{\pgfqpoint{3.806035in}{2.994488in}}%
\pgfpathlineto{\pgfqpoint{3.859079in}{2.981887in}}%
\pgfpathlineto{\pgfqpoint{3.912123in}{2.942941in}}%
\pgfpathlineto{\pgfqpoint{3.965167in}{3.071374in}}%
\pgfpathlineto{\pgfqpoint{4.018211in}{3.000463in}}%
\pgfpathlineto{\pgfqpoint{4.071255in}{3.082554in}}%
\pgfpathlineto{\pgfqpoint{4.124299in}{3.107971in}}%
\pgfpathlineto{\pgfqpoint{4.177343in}{3.126443in}}%
\pgfpathlineto{\pgfqpoint{4.230387in}{3.164551in}}%
\pgfpathlineto{\pgfqpoint{4.283431in}{3.180269in}}%
\pgfpathlineto{\pgfqpoint{4.336475in}{3.224846in}}%
\pgfpathlineto{\pgfqpoint{4.389519in}{3.194301in}}%
\pgfpathlineto{\pgfqpoint{4.442563in}{3.183354in}}%
\pgfpathlineto{\pgfqpoint{4.495607in}{3.218773in}}%
\pgfpathlineto{\pgfqpoint{4.548650in}{3.235597in}}%
\pgfpathlineto{\pgfqpoint{4.601694in}{3.162876in}}%
\pgfpathlineto{\pgfqpoint{4.654738in}{3.181880in}}%
\pgfpathlineto{\pgfqpoint{4.707782in}{3.102021in}}%
\pgfpathlineto{\pgfqpoint{4.760826in}{3.110933in}}%
\pgfpathlineto{\pgfqpoint{4.813870in}{3.062144in}}%
\pgfpathlineto{\pgfqpoint{4.866914in}{3.020692in}}%
\pgfpathlineto{\pgfqpoint{4.919958in}{3.007061in}}%
\pgfpathlineto{\pgfqpoint{4.973002in}{2.943141in}}%
\pgfpathlineto{\pgfqpoint{5.026046in}{2.958572in}}%
\pgfpathlineto{\pgfqpoint{5.079090in}{2.914682in}}%
\pgfpathlineto{\pgfqpoint{5.132134in}{2.918032in}}%
\pgfpathlineto{\pgfqpoint{5.185178in}{3.006017in}}%
\pgfpathlineto{\pgfqpoint{5.238222in}{3.002435in}}%
\pgfpathlineto{\pgfqpoint{5.291266in}{3.062668in}}%
\pgfpathlineto{\pgfqpoint{5.344309in}{3.073166in}}%
\pgfpathlineto{\pgfqpoint{5.397353in}{3.084080in}}%
\pgfpathlineto{\pgfqpoint{5.450397in}{3.213681in}}%
\pgfpathlineto{\pgfqpoint{5.503441in}{3.203849in}}%
\pgfpathlineto{\pgfqpoint{5.556485in}{3.196655in}}%
\pgfpathlineto{\pgfqpoint{5.609529in}{3.209959in}}%
\pgfpathlineto{\pgfqpoint{5.662573in}{3.261554in}}%
\pgfpathlineto{\pgfqpoint{5.715617in}{3.209641in}}%
\pgfpathlineto{\pgfqpoint{5.768661in}{3.248973in}}%
\pgfpathlineto{\pgfqpoint{5.821705in}{3.208761in}}%
\pgfpathlineto{\pgfqpoint{5.874749in}{3.132435in}}%
\pgfpathlineto{\pgfqpoint{5.927793in}{3.156691in}}%
\pgfpathlineto{\pgfqpoint{5.980837in}{3.133284in}}%
\pgfpathlineto{\pgfqpoint{6.033881in}{3.084077in}}%
\pgfpathlineto{\pgfqpoint{6.086925in}{3.062147in}}%
\pgfpathlineto{\pgfqpoint{6.139969in}{3.043673in}}%
\pgfpathlineto{\pgfqpoint{6.193012in}{2.971161in}}%
\pgfpathlineto{\pgfqpoint{6.246056in}{3.026875in}}%
\pgfpathlineto{\pgfqpoint{6.299100in}{2.964176in}}%
\pgfpathlineto{\pgfqpoint{6.352144in}{2.982962in}}%
\pgfpathlineto{\pgfqpoint{6.405188in}{2.929750in}}%
\pgfpathlineto{\pgfqpoint{6.458232in}{3.016017in}}%
\pgfpathlineto{\pgfqpoint{6.511276in}{3.033770in}}%
\pgfpathlineto{\pgfqpoint{6.564320in}{3.030119in}}%
\pgfpathlineto{\pgfqpoint{6.617364in}{3.097076in}}%
\pgfpathlineto{\pgfqpoint{6.670408in}{3.127018in}}%
\pgfpathlineto{\pgfqpoint{6.723452in}{3.173100in}}%
\pgfpathlineto{\pgfqpoint{6.776496in}{3.194890in}}%
\pgfpathlineto{\pgfqpoint{6.829540in}{3.185067in}}%
\pgfpathlineto{\pgfqpoint{6.882584in}{3.217123in}}%
\pgfpathlineto{\pgfqpoint{6.935628in}{3.178720in}}%
\pgfpathlineto{\pgfqpoint{6.988671in}{3.222397in}}%
\pgfpathlineto{\pgfqpoint{7.041715in}{3.197983in}}%
\pgfpathlineto{\pgfqpoint{7.094759in}{3.221034in}}%
\pgfpathlineto{\pgfqpoint{7.147803in}{3.202828in}}%
\pgfpathlineto{\pgfqpoint{7.200847in}{3.153935in}}%
\pgfpathlineto{\pgfqpoint{7.253891in}{3.055048in}}%
\pgfpathlineto{\pgfqpoint{7.306935in}{3.083084in}}%
\pgfpathlineto{\pgfqpoint{7.359979in}{3.007952in}}%
\pgfpathlineto{\pgfqpoint{7.413023in}{3.019856in}}%
\pgfpathlineto{\pgfqpoint{7.466067in}{2.989900in}}%
\pgfpathlineto{\pgfqpoint{7.519111in}{2.942028in}}%
\pgfpathlineto{\pgfqpoint{7.572155in}{2.929451in}}%
\pgfpathlineto{\pgfqpoint{7.625199in}{2.988504in}}%
\pgfpathlineto{\pgfqpoint{7.678243in}{3.033419in}}%
\pgfpathlineto{\pgfqpoint{7.731287in}{2.995952in}}%
\pgfpathlineto{\pgfqpoint{7.784330in}{3.031983in}}%
\pgfpathlineto{\pgfqpoint{7.837374in}{3.040235in}}%
\pgfpathlineto{\pgfqpoint{7.890418in}{3.081537in}}%
\pgfpathlineto{\pgfqpoint{7.943462in}{3.079283in}}%
\pgfpathlineto{\pgfqpoint{7.996506in}{3.157986in}}%
\pgfpathlineto{\pgfqpoint{8.049550in}{3.189369in}}%
\pgfpathlineto{\pgfqpoint{8.102594in}{3.198813in}}%
\pgfpathlineto{\pgfqpoint{8.155638in}{3.224623in}}%
\pgfpathlineto{\pgfqpoint{8.208682in}{3.218763in}}%
\pgfpathlineto{\pgfqpoint{8.261726in}{3.260476in}}%
\pgfpathlineto{\pgfqpoint{8.314770in}{3.211850in}}%
\pgfpathlineto{\pgfqpoint{8.367814in}{3.201180in}}%
\pgfpathlineto{\pgfqpoint{8.420858in}{3.193239in}}%
\pgfpathlineto{\pgfqpoint{8.473902in}{3.141958in}}%
\pgfpathlineto{\pgfqpoint{8.526946in}{3.113817in}}%
\pgfpathlineto{\pgfqpoint{8.579990in}{3.021265in}}%
\pgfpathlineto{\pgfqpoint{8.633033in}{3.030538in}}%
\pgfpathlineto{\pgfqpoint{8.686077in}{3.059450in}}%
\pgfpathlineto{\pgfqpoint{8.739121in}{2.997851in}}%
\pgfpathlineto{\pgfqpoint{8.792165in}{3.000505in}}%
\pgfpathlineto{\pgfqpoint{8.845209in}{2.978670in}}%
\pgfpathlineto{\pgfqpoint{8.898253in}{3.021456in}}%
\pgfpathlineto{\pgfqpoint{8.951297in}{2.965023in}}%
\pgfpathlineto{\pgfqpoint{9.004341in}{2.983962in}}%
\pgfpathlineto{\pgfqpoint{9.057385in}{3.049384in}}%
\pgfpathlineto{\pgfqpoint{9.110429in}{3.084477in}}%
\pgfpathlineto{\pgfqpoint{9.163473in}{3.106410in}}%
\pgfpathlineto{\pgfqpoint{9.216517in}{3.140256in}}%
\pgfpathlineto{\pgfqpoint{9.269561in}{3.147248in}}%
\pgfpathlineto{\pgfqpoint{9.322605in}{3.219668in}}%
\pgfpathlineto{\pgfqpoint{9.375649in}{3.214984in}}%
\pgfpathlineto{\pgfqpoint{9.428692in}{3.240840in}}%
\pgfpathlineto{\pgfqpoint{9.481736in}{3.220264in}}%
\pgfpathlineto{\pgfqpoint{9.534780in}{3.282563in}}%
\pgfpathlineto{\pgfqpoint{9.587824in}{3.204226in}}%
\pgfpathlineto{\pgfqpoint{9.640868in}{3.199776in}}%
\pgfpathlineto{\pgfqpoint{9.693912in}{3.139443in}}%
\pgfpathlineto{\pgfqpoint{9.746956in}{3.115475in}}%
\pgfpathlineto{\pgfqpoint{9.800000in}{3.101625in}}%
\pgfpathlineto{\pgfqpoint{9.800000in}{3.101625in}}%
\pgfusepath{stroke}%
\end{pgfscope}%
\begin{pgfscope}%
\pgfsetbuttcap%
\pgfsetmiterjoin%
\definecolor{currentfill}{rgb}{1.000000,1.000000,1.000000}%
\pgfsetfillcolor{currentfill}%
\pgfsetlinewidth{1.003750pt}%
\definecolor{currentstroke}{rgb}{0.000000,0.000000,0.000000}%
\pgfsetstrokecolor{currentstroke}%
\pgfsetdash{}{0pt}%
\pgfpathmoveto{\pgfqpoint{1.017884in}{3.802279in}}%
\pgfpathlineto{\pgfqpoint{1.315613in}{3.802279in}}%
\pgfpathlineto{\pgfqpoint{1.315613in}{4.115057in}}%
\pgfpathlineto{\pgfqpoint{1.017884in}{4.115057in}}%
\pgfpathlineto{\pgfqpoint{1.017884in}{3.802279in}}%
\pgfpathclose%
\pgfusepath{stroke,fill}%
\end{pgfscope}%
\begin{pgfscope}%
\definecolor{textcolor}{rgb}{0.000000,0.000000,0.000000}%
\pgfsetstrokecolor{textcolor}%
\pgfsetfillcolor{textcolor}%
\pgftext[x=1.074273in,y=3.908668in,left,base]{\color{textcolor}{\rmfamily\fontsize{14.000000}{16.800000}\selectfont\catcode`\^=\active\def^{\ifmmode\sp\else\^{}\fi}\catcode`\%=\active\def%{\%}b)}}%
\end{pgfscope}%
\begin{pgfscope}%
\pgfsetbuttcap%
\pgfsetmiterjoin%
\definecolor{currentfill}{rgb}{1.000000,1.000000,1.000000}%
\pgfsetfillcolor{currentfill}%
\pgfsetfillopacity{0.800000}%
\pgfsetlinewidth{1.003750pt}%
\definecolor{currentstroke}{rgb}{0.800000,0.800000,0.800000}%
\pgfsetstrokecolor{currentstroke}%
\pgfsetstrokeopacity{0.800000}%
\pgfsetdash{}{0pt}%
\pgfpathmoveto{\pgfqpoint{1.058330in}{0.753471in}}%
\pgfpathlineto{\pgfqpoint{6.450599in}{0.753471in}}%
\pgfpathquadraticcurveto{\pgfqpoint{6.483933in}{0.753471in}}{\pgfqpoint{6.483933in}{0.786805in}}%
\pgfpathlineto{\pgfqpoint{6.483933in}{1.467359in}}%
\pgfpathquadraticcurveto{\pgfqpoint{6.483933in}{1.500693in}}{\pgfqpoint{6.450599in}{1.500693in}}%
\pgfpathlineto{\pgfqpoint{1.058330in}{1.500693in}}%
\pgfpathquadraticcurveto{\pgfqpoint{1.024996in}{1.500693in}}{\pgfqpoint{1.024996in}{1.467359in}}%
\pgfpathlineto{\pgfqpoint{1.024996in}{0.786805in}}%
\pgfpathquadraticcurveto{\pgfqpoint{1.024996in}{0.753471in}}{\pgfqpoint{1.058330in}{0.753471in}}%
\pgfpathlineto{\pgfqpoint{1.058330in}{0.753471in}}%
\pgfpathclose%
\pgfusepath{stroke,fill}%
\end{pgfscope}%
\begin{pgfscope}%
\pgfsetbuttcap%
\pgfsetmiterjoin%
\definecolor{currentfill}{rgb}{0.121569,0.466667,0.705882}%
\pgfsetfillcolor{currentfill}%
\pgfsetlinewidth{1.003750pt}%
\definecolor{currentstroke}{rgb}{0.121569,0.466667,0.705882}%
\pgfsetstrokecolor{currentstroke}%
\pgfsetdash{}{0pt}%
\pgfpathmoveto{\pgfqpoint{1.091663in}{1.317359in}}%
\pgfpathlineto{\pgfqpoint{1.424996in}{1.317359in}}%
\pgfpathlineto{\pgfqpoint{1.424996in}{1.434026in}}%
\pgfpathlineto{\pgfqpoint{1.091663in}{1.434026in}}%
\pgfpathlineto{\pgfqpoint{1.091663in}{1.317359in}}%
\pgfpathclose%
\pgfusepath{stroke,fill}%
\end{pgfscope}%
\begin{pgfscope}%
\definecolor{textcolor}{rgb}{0.000000,0.000000,0.000000}%
\pgfsetstrokecolor{textcolor}%
\pgfsetfillcolor{textcolor}%
\pgftext[x=1.558330in,y=1.317359in,left,base]{\color{textcolor}{\rmfamily\fontsize{12.000000}{14.400000}\selectfont\catcode`\^=\active\def^{\ifmmode\sp\else\^{}\fi}\catcode`\%=\active\def%{\%}Nuclear}}%
\end{pgfscope}%
\begin{pgfscope}%
\pgfsetbuttcap%
\pgfsetmiterjoin%
\definecolor{currentfill}{rgb}{0.501961,0.000000,0.501961}%
\pgfsetfillcolor{currentfill}%
\pgfsetlinewidth{1.003750pt}%
\definecolor{currentstroke}{rgb}{0.501961,0.000000,0.501961}%
\pgfsetstrokecolor{currentstroke}%
\pgfsetdash{}{0pt}%
\pgfpathmoveto{\pgfqpoint{1.091663in}{1.084952in}}%
\pgfpathlineto{\pgfqpoint{1.424996in}{1.084952in}}%
\pgfpathlineto{\pgfqpoint{1.424996in}{1.201619in}}%
\pgfpathlineto{\pgfqpoint{1.091663in}{1.201619in}}%
\pgfpathlineto{\pgfqpoint{1.091663in}{1.084952in}}%
\pgfpathclose%
\pgfusepath{stroke,fill}%
\end{pgfscope}%
\begin{pgfscope}%
\definecolor{textcolor}{rgb}{0.000000,0.000000,0.000000}%
\pgfsetstrokecolor{textcolor}%
\pgfsetfillcolor{textcolor}%
\pgftext[x=1.558330in,y=1.084952in,left,base]{\color{textcolor}{\rmfamily\fontsize{12.000000}{14.400000}\selectfont\catcode`\^=\active\def^{\ifmmode\sp\else\^{}\fi}\catcode`\%=\active\def%{\%}Battery}}%
\end{pgfscope}%
\begin{pgfscope}%
\pgfsetbuttcap%
\pgfsetmiterjoin%
\definecolor{currentfill}{rgb}{0.549020,0.337255,0.294118}%
\pgfsetfillcolor{currentfill}%
\pgfsetlinewidth{1.003750pt}%
\definecolor{currentstroke}{rgb}{0.549020,0.337255,0.294118}%
\pgfsetstrokecolor{currentstroke}%
\pgfsetdash{}{0pt}%
\pgfpathmoveto{\pgfqpoint{1.091663in}{0.852545in}}%
\pgfpathlineto{\pgfqpoint{1.424996in}{0.852545in}}%
\pgfpathlineto{\pgfqpoint{1.424996in}{0.969212in}}%
\pgfpathlineto{\pgfqpoint{1.091663in}{0.969212in}}%
\pgfpathlineto{\pgfqpoint{1.091663in}{0.852545in}}%
\pgfpathclose%
\pgfusepath{stroke,fill}%
\end{pgfscope}%
\begin{pgfscope}%
\definecolor{textcolor}{rgb}{0.000000,0.000000,0.000000}%
\pgfsetstrokecolor{textcolor}%
\pgfsetfillcolor{textcolor}%
\pgftext[x=1.558330in,y=0.852545in,left,base]{\color{textcolor}{\rmfamily\fontsize{12.000000}{14.400000}\selectfont\catcode`\^=\active\def^{\ifmmode\sp\else\^{}\fi}\catcode`\%=\active\def%{\%}NaturalGas Conv}}%
\end{pgfscope}%
\begin{pgfscope}%
\pgfsetbuttcap%
\pgfsetmiterjoin%
\definecolor{currentfill}{rgb}{1.000000,0.647059,0.000000}%
\pgfsetfillcolor{currentfill}%
\pgfsetlinewidth{1.003750pt}%
\definecolor{currentstroke}{rgb}{1.000000,0.647059,0.000000}%
\pgfsetstrokecolor{currentstroke}%
\pgfsetdash{}{0pt}%
\pgfpathmoveto{\pgfqpoint{3.140242in}{1.317359in}}%
\pgfpathlineto{\pgfqpoint{3.473575in}{1.317359in}}%
\pgfpathlineto{\pgfqpoint{3.473575in}{1.434026in}}%
\pgfpathlineto{\pgfqpoint{3.140242in}{1.434026in}}%
\pgfpathlineto{\pgfqpoint{3.140242in}{1.317359in}}%
\pgfpathclose%
\pgfusepath{stroke,fill}%
\end{pgfscope}%
\begin{pgfscope}%
\definecolor{textcolor}{rgb}{0.000000,0.000000,0.000000}%
\pgfsetstrokecolor{textcolor}%
\pgfsetfillcolor{textcolor}%
\pgftext[x=3.606909in,y=1.317359in,left,base]{\color{textcolor}{\rmfamily\fontsize{12.000000}{14.400000}\selectfont\catcode`\^=\active\def^{\ifmmode\sp\else\^{}\fi}\catcode`\%=\active\def%{\%}Battery charge}}%
\end{pgfscope}%
\begin{pgfscope}%
\pgfsetbuttcap%
\pgfsetmiterjoin%
\definecolor{currentfill}{rgb}{0.501961,0.501961,0.501961}%
\pgfsetfillcolor{currentfill}%
\pgfsetlinewidth{1.003750pt}%
\definecolor{currentstroke}{rgb}{0.501961,0.501961,0.501961}%
\pgfsetstrokecolor{currentstroke}%
\pgfsetdash{}{0pt}%
\pgfpathmoveto{\pgfqpoint{3.140242in}{1.084952in}}%
\pgfpathlineto{\pgfqpoint{3.473575in}{1.084952in}}%
\pgfpathlineto{\pgfqpoint{3.473575in}{1.201619in}}%
\pgfpathlineto{\pgfqpoint{3.140242in}{1.201619in}}%
\pgfpathlineto{\pgfqpoint{3.140242in}{1.084952in}}%
\pgfpathclose%
\pgfusepath{stroke,fill}%
\end{pgfscope}%
\begin{pgfscope}%
\definecolor{textcolor}{rgb}{0.000000,0.000000,0.000000}%
\pgfsetstrokecolor{textcolor}%
\pgfsetfillcolor{textcolor}%
\pgftext[x=3.606909in,y=1.084952in,left,base]{\color{textcolor}{\rmfamily\fontsize{12.000000}{14.400000}\selectfont\catcode`\^=\active\def^{\ifmmode\sp\else\^{}\fi}\catcode`\%=\active\def%{\%}Curtailment}}%
\end{pgfscope}%
\begin{pgfscope}%
\pgfsetbuttcap%
\pgfsetmiterjoin%
\definecolor{currentfill}{rgb}{0.090196,0.745098,0.811765}%
\pgfsetfillcolor{currentfill}%
\pgfsetlinewidth{1.003750pt}%
\definecolor{currentstroke}{rgb}{0.090196,0.745098,0.811765}%
\pgfsetstrokecolor{currentstroke}%
\pgfsetdash{}{0pt}%
\pgfpathmoveto{\pgfqpoint{4.998699in}{1.317359in}}%
\pgfpathlineto{\pgfqpoint{5.332033in}{1.317359in}}%
\pgfpathlineto{\pgfqpoint{5.332033in}{1.434026in}}%
\pgfpathlineto{\pgfqpoint{4.998699in}{1.434026in}}%
\pgfpathlineto{\pgfqpoint{4.998699in}{1.317359in}}%
\pgfpathclose%
\pgfusepath{stroke,fill}%
\end{pgfscope}%
\begin{pgfscope}%
\definecolor{textcolor}{rgb}{0.000000,0.000000,0.000000}%
\pgfsetstrokecolor{textcolor}%
\pgfsetfillcolor{textcolor}%
\pgftext[x=5.465366in,y=1.317359in,left,base]{\color{textcolor}{\rmfamily\fontsize{12.000000}{14.400000}\selectfont\catcode`\^=\active\def^{\ifmmode\sp\else\^{}\fi}\catcode`\%=\active\def%{\%}WindTurbine}}%
\end{pgfscope}%
\begin{pgfscope}%
\pgfsetbuttcap%
\pgfsetroundjoin%
\pgfsetlinewidth{1.505625pt}%
\definecolor{currentstroke}{rgb}{0.000000,0.000000,0.000000}%
\pgfsetstrokecolor{currentstroke}%
\pgfsetdash{{5.550000pt}{2.400000pt}}{0.000000pt}%
\pgfpathmoveto{\pgfqpoint{4.998699in}{1.143286in}}%
\pgfpathlineto{\pgfqpoint{5.165366in}{1.143286in}}%
\pgfpathlineto{\pgfqpoint{5.332033in}{1.143286in}}%
\pgfusepath{stroke}%
\end{pgfscope}%
\begin{pgfscope}%
\definecolor{textcolor}{rgb}{0.000000,0.000000,0.000000}%
\pgfsetstrokecolor{textcolor}%
\pgfsetfillcolor{textcolor}%
\pgftext[x=5.465366in,y=1.084952in,left,base]{\color{textcolor}{\rmfamily\fontsize{12.000000}{14.400000}\selectfont\catcode`\^=\active\def^{\ifmmode\sp\else\^{}\fi}\catcode`\%=\active\def%{\%}Demand}}%
\end{pgfscope}%
\end{pgfpicture}%
\makeatother%
\endgroup%
}
    \caption{Comparison between dispatch results for two algorithms. Plot a) was
    calculated with a logical dispatch algorithm and plot b) was calculated with
    optimal dispatch.}
    \label{fig:dispatch-comparison}
\end{figure}

The optimal dispatch algorithm uses a linear programming formulation to arrive
at an optimal solution with perfect foresight. The logical dispatch algorithm
uses a rule-based approach to dispatch energy according to merit order.
However, this algorithm is myopic since dispatch is calculated serially. These
differences totally account for the differences in their dispatch results. The
optimal dispatch algorithm uses battery storage more effectively than its
rule-based counterpart because it optimizes the entire time series at once.
Since the logical algorithm uses battery storage imperfectly, it fills the
energy gaps with natural gas and more energy is curtailed rather than used.
Although, the optimal dispatch solution performs better on a pure cost basis, the
myopia of the logical dispatch algorithm is possibly more realistic.
Further, since the logical dispatch algorithm does not have an energy balance
constraint for all time steps, users can more easily estimate reliability and
calculate costs from energy shortfalls.


\FloatBarrier

\subsection{Exercise 2: Time Scaling}

This exercise considers how the two dispatch algorithms scale with simulation duration.
For this exercise, the two algorithms were placed within a
\texttt{CapacityExpansion} problem with the parameters described in Table
\ref{tab:scaling-ga-params}.

\begin{table}[htbp!]
    \centering
    \caption{Capacity expansion parameters for the algorithm comparison exercise.}
    \label{tab:scaling-ga-params}
    \begin{tabular}{ll}
        \toprule
        Parameter & Value \\
        \midrule
        Algorithm & \acs{nsga2}\\
        Termination Criterion & Maximum generations\\
        Generations & 10 \\
        Population Size & 20 \\
        Objectives & 2 (cost, emissions)\\
        Threads & 1 \\
        \bottomrule
    \end{tabular}
\end{table}

\noindent The available technologies were the same four as in the previous
exercise. Rather than scaling the problem by number of objectives, technologies,
or population size, this exercise scales the problem by the length of the time
series. This is preferred because time series data typically increases the
problem size more dramatically than the number of objectives or number of
technologies. Further, scaling by population size would obfuscate the
differences between the two algorithms since neither are affected by population
size. Figure \ref{fig:alg-scaling} shows results of this scaling study. The
x-axis measures the number of modeled days at an hourly resolution.

\begin{figure}[htbp!]
    \centering
    \resizebox{0.75\columnwidth}{!}{%% Creator: Matplotlib, PGF backend
%%
%% To include the figure in your LaTeX document, write
%%   \input{<filename>.pgf}
%%
%% Make sure the required packages are loaded in your preamble
%%   \usepackage{pgf}
%%
%% Also ensure that all the required font packages are loaded; for instance,
%% the lmodern package is sometimes necessary when using math font.
%%   \usepackage{lmodern}
%%
%% Figures using additional raster images can only be included by \input if
%% they are in the same directory as the main LaTeX file. For loading figures
%% from other directories you can use the `import` package
%%   \usepackage{import}
%%
%% and then include the figures with
%%   \import{<path to file>}{<filename>.pgf}
%%
%% Matplotlib used the following preamble
%%   \def\mathdefault#1{#1}
%%   \everymath=\expandafter{\the\everymath\displaystyle}
%%   \IfFileExists{scrextend.sty}{
%%     \usepackage[fontsize=10.000000pt]{scrextend}
%%   }{
%%     \renewcommand{\normalsize}{\fontsize{10.000000}{12.000000}\selectfont}
%%     \normalsize
%%   }
%%   
%%   \makeatletter\@ifpackageloaded{underscore}{}{\usepackage[strings]{underscore}}\makeatother
%%
\begingroup%
\makeatletter%
\begin{pgfpicture}%
\pgfpathrectangle{\pgfpointorigin}{\pgfqpoint{7.064581in}{5.411797in}}%
\pgfusepath{use as bounding box, clip}%
\begin{pgfscope}%
\pgfsetbuttcap%
\pgfsetmiterjoin%
\definecolor{currentfill}{rgb}{1.000000,1.000000,1.000000}%
\pgfsetfillcolor{currentfill}%
\pgfsetlinewidth{0.000000pt}%
\definecolor{currentstroke}{rgb}{0.000000,0.000000,0.000000}%
\pgfsetstrokecolor{currentstroke}%
\pgfsetdash{}{0pt}%
\pgfpathmoveto{\pgfqpoint{0.000000in}{0.000000in}}%
\pgfpathlineto{\pgfqpoint{7.064581in}{0.000000in}}%
\pgfpathlineto{\pgfqpoint{7.064581in}{5.411797in}}%
\pgfpathlineto{\pgfqpoint{0.000000in}{5.411797in}}%
\pgfpathlineto{\pgfqpoint{0.000000in}{0.000000in}}%
\pgfpathclose%
\pgfusepath{fill}%
\end{pgfscope}%
\begin{pgfscope}%
\pgfsetbuttcap%
\pgfsetmiterjoin%
\definecolor{currentfill}{rgb}{1.000000,1.000000,1.000000}%
\pgfsetfillcolor{currentfill}%
\pgfsetlinewidth{0.000000pt}%
\definecolor{currentstroke}{rgb}{0.000000,0.000000,0.000000}%
\pgfsetstrokecolor{currentstroke}%
\pgfsetstrokeopacity{0.000000}%
\pgfsetdash{}{0pt}%
\pgfpathmoveto{\pgfqpoint{0.764581in}{0.643904in}}%
\pgfpathlineto{\pgfqpoint{6.964581in}{0.643904in}}%
\pgfpathlineto{\pgfqpoint{6.964581in}{5.263904in}}%
\pgfpathlineto{\pgfqpoint{0.764581in}{5.263904in}}%
\pgfpathlineto{\pgfqpoint{0.764581in}{0.643904in}}%
\pgfpathclose%
\pgfusepath{fill}%
\end{pgfscope}%
\begin{pgfscope}%
\pgfpathrectangle{\pgfqpoint{0.764581in}{0.643904in}}{\pgfqpoint{6.200000in}{4.620000in}}%
\pgfusepath{clip}%
\pgfsetrectcap%
\pgfsetroundjoin%
\pgfsetlinewidth{0.803000pt}%
\definecolor{currentstroke}{rgb}{0.690196,0.690196,0.690196}%
\pgfsetstrokecolor{currentstroke}%
\pgfsetdash{}{0pt}%
\pgfpathmoveto{\pgfqpoint{0.764581in}{0.643904in}}%
\pgfpathlineto{\pgfqpoint{0.764581in}{5.263904in}}%
\pgfusepath{stroke}%
\end{pgfscope}%
\begin{pgfscope}%
\pgfsetbuttcap%
\pgfsetroundjoin%
\definecolor{currentfill}{rgb}{0.000000,0.000000,0.000000}%
\pgfsetfillcolor{currentfill}%
\pgfsetlinewidth{0.803000pt}%
\definecolor{currentstroke}{rgb}{0.000000,0.000000,0.000000}%
\pgfsetstrokecolor{currentstroke}%
\pgfsetdash{}{0pt}%
\pgfsys@defobject{currentmarker}{\pgfqpoint{0.000000in}{-0.048611in}}{\pgfqpoint{0.000000in}{0.000000in}}{%
\pgfpathmoveto{\pgfqpoint{0.000000in}{0.000000in}}%
\pgfpathlineto{\pgfqpoint{0.000000in}{-0.048611in}}%
\pgfusepath{stroke,fill}%
}%
\begin{pgfscope}%
\pgfsys@transformshift{0.764581in}{0.643904in}%
\pgfsys@useobject{currentmarker}{}%
\end{pgfscope}%
\end{pgfscope}%
\begin{pgfscope}%
\definecolor{textcolor}{rgb}{0.000000,0.000000,0.000000}%
\pgfsetstrokecolor{textcolor}%
\pgfsetfillcolor{textcolor}%
\pgftext[x=0.764581in,y=0.546682in,,top]{\color{textcolor}{\rmfamily\fontsize{14.000000}{16.800000}\selectfont\catcode`\^=\active\def^{\ifmmode\sp\else\^{}\fi}\catcode`\%=\active\def%{\%}$\mathdefault{10^{0}}$}}%
\end{pgfscope}%
\begin{pgfscope}%
\pgfpathrectangle{\pgfqpoint{0.764581in}{0.643904in}}{\pgfqpoint{6.200000in}{4.620000in}}%
\pgfusepath{clip}%
\pgfsetrectcap%
\pgfsetroundjoin%
\pgfsetlinewidth{0.803000pt}%
\definecolor{currentstroke}{rgb}{0.690196,0.690196,0.690196}%
\pgfsetstrokecolor{currentstroke}%
\pgfsetdash{}{0pt}%
\pgfpathmoveto{\pgfqpoint{3.184288in}{0.643904in}}%
\pgfpathlineto{\pgfqpoint{3.184288in}{5.263904in}}%
\pgfusepath{stroke}%
\end{pgfscope}%
\begin{pgfscope}%
\pgfsetbuttcap%
\pgfsetroundjoin%
\definecolor{currentfill}{rgb}{0.000000,0.000000,0.000000}%
\pgfsetfillcolor{currentfill}%
\pgfsetlinewidth{0.803000pt}%
\definecolor{currentstroke}{rgb}{0.000000,0.000000,0.000000}%
\pgfsetstrokecolor{currentstroke}%
\pgfsetdash{}{0pt}%
\pgfsys@defobject{currentmarker}{\pgfqpoint{0.000000in}{-0.048611in}}{\pgfqpoint{0.000000in}{0.000000in}}{%
\pgfpathmoveto{\pgfqpoint{0.000000in}{0.000000in}}%
\pgfpathlineto{\pgfqpoint{0.000000in}{-0.048611in}}%
\pgfusepath{stroke,fill}%
}%
\begin{pgfscope}%
\pgfsys@transformshift{3.184288in}{0.643904in}%
\pgfsys@useobject{currentmarker}{}%
\end{pgfscope}%
\end{pgfscope}%
\begin{pgfscope}%
\definecolor{textcolor}{rgb}{0.000000,0.000000,0.000000}%
\pgfsetstrokecolor{textcolor}%
\pgfsetfillcolor{textcolor}%
\pgftext[x=3.184288in,y=0.546682in,,top]{\color{textcolor}{\rmfamily\fontsize{14.000000}{16.800000}\selectfont\catcode`\^=\active\def^{\ifmmode\sp\else\^{}\fi}\catcode`\%=\active\def%{\%}$\mathdefault{10^{1}}$}}%
\end{pgfscope}%
\begin{pgfscope}%
\pgfpathrectangle{\pgfqpoint{0.764581in}{0.643904in}}{\pgfqpoint{6.200000in}{4.620000in}}%
\pgfusepath{clip}%
\pgfsetrectcap%
\pgfsetroundjoin%
\pgfsetlinewidth{0.803000pt}%
\definecolor{currentstroke}{rgb}{0.690196,0.690196,0.690196}%
\pgfsetstrokecolor{currentstroke}%
\pgfsetdash{}{0pt}%
\pgfpathmoveto{\pgfqpoint{5.603996in}{0.643904in}}%
\pgfpathlineto{\pgfqpoint{5.603996in}{5.263904in}}%
\pgfusepath{stroke}%
\end{pgfscope}%
\begin{pgfscope}%
\pgfsetbuttcap%
\pgfsetroundjoin%
\definecolor{currentfill}{rgb}{0.000000,0.000000,0.000000}%
\pgfsetfillcolor{currentfill}%
\pgfsetlinewidth{0.803000pt}%
\definecolor{currentstroke}{rgb}{0.000000,0.000000,0.000000}%
\pgfsetstrokecolor{currentstroke}%
\pgfsetdash{}{0pt}%
\pgfsys@defobject{currentmarker}{\pgfqpoint{0.000000in}{-0.048611in}}{\pgfqpoint{0.000000in}{0.000000in}}{%
\pgfpathmoveto{\pgfqpoint{0.000000in}{0.000000in}}%
\pgfpathlineto{\pgfqpoint{0.000000in}{-0.048611in}}%
\pgfusepath{stroke,fill}%
}%
\begin{pgfscope}%
\pgfsys@transformshift{5.603996in}{0.643904in}%
\pgfsys@useobject{currentmarker}{}%
\end{pgfscope}%
\end{pgfscope}%
\begin{pgfscope}%
\definecolor{textcolor}{rgb}{0.000000,0.000000,0.000000}%
\pgfsetstrokecolor{textcolor}%
\pgfsetfillcolor{textcolor}%
\pgftext[x=5.603996in,y=0.546682in,,top]{\color{textcolor}{\rmfamily\fontsize{14.000000}{16.800000}\selectfont\catcode`\^=\active\def^{\ifmmode\sp\else\^{}\fi}\catcode`\%=\active\def%{\%}$\mathdefault{10^{2}}$}}%
\end{pgfscope}%
\begin{pgfscope}%
\pgfpathrectangle{\pgfqpoint{0.764581in}{0.643904in}}{\pgfqpoint{6.200000in}{4.620000in}}%
\pgfusepath{clip}%
\pgfsetbuttcap%
\pgfsetroundjoin%
\pgfsetlinewidth{0.803000pt}%
\definecolor{currentstroke}{rgb}{0.690196,0.690196,0.690196}%
\pgfsetstrokecolor{currentstroke}%
\pgfsetstrokeopacity{0.200000}%
\pgfsetdash{{2.960000pt}{1.280000pt}}{0.000000pt}%
\pgfpathmoveto{\pgfqpoint{1.492985in}{0.643904in}}%
\pgfpathlineto{\pgfqpoint{1.492985in}{5.263904in}}%
\pgfusepath{stroke}%
\end{pgfscope}%
\begin{pgfscope}%
\pgfsetbuttcap%
\pgfsetroundjoin%
\definecolor{currentfill}{rgb}{0.000000,0.000000,0.000000}%
\pgfsetfillcolor{currentfill}%
\pgfsetlinewidth{0.602250pt}%
\definecolor{currentstroke}{rgb}{0.000000,0.000000,0.000000}%
\pgfsetstrokecolor{currentstroke}%
\pgfsetdash{}{0pt}%
\pgfsys@defobject{currentmarker}{\pgfqpoint{0.000000in}{-0.027778in}}{\pgfqpoint{0.000000in}{0.000000in}}{%
\pgfpathmoveto{\pgfqpoint{0.000000in}{0.000000in}}%
\pgfpathlineto{\pgfqpoint{0.000000in}{-0.027778in}}%
\pgfusepath{stroke,fill}%
}%
\begin{pgfscope}%
\pgfsys@transformshift{1.492985in}{0.643904in}%
\pgfsys@useobject{currentmarker}{}%
\end{pgfscope}%
\end{pgfscope}%
\begin{pgfscope}%
\pgfpathrectangle{\pgfqpoint{0.764581in}{0.643904in}}{\pgfqpoint{6.200000in}{4.620000in}}%
\pgfusepath{clip}%
\pgfsetbuttcap%
\pgfsetroundjoin%
\pgfsetlinewidth{0.803000pt}%
\definecolor{currentstroke}{rgb}{0.690196,0.690196,0.690196}%
\pgfsetstrokecolor{currentstroke}%
\pgfsetstrokeopacity{0.200000}%
\pgfsetdash{{2.960000pt}{1.280000pt}}{0.000000pt}%
\pgfpathmoveto{\pgfqpoint{1.919075in}{0.643904in}}%
\pgfpathlineto{\pgfqpoint{1.919075in}{5.263904in}}%
\pgfusepath{stroke}%
\end{pgfscope}%
\begin{pgfscope}%
\pgfsetbuttcap%
\pgfsetroundjoin%
\definecolor{currentfill}{rgb}{0.000000,0.000000,0.000000}%
\pgfsetfillcolor{currentfill}%
\pgfsetlinewidth{0.602250pt}%
\definecolor{currentstroke}{rgb}{0.000000,0.000000,0.000000}%
\pgfsetstrokecolor{currentstroke}%
\pgfsetdash{}{0pt}%
\pgfsys@defobject{currentmarker}{\pgfqpoint{0.000000in}{-0.027778in}}{\pgfqpoint{0.000000in}{0.000000in}}{%
\pgfpathmoveto{\pgfqpoint{0.000000in}{0.000000in}}%
\pgfpathlineto{\pgfqpoint{0.000000in}{-0.027778in}}%
\pgfusepath{stroke,fill}%
}%
\begin{pgfscope}%
\pgfsys@transformshift{1.919075in}{0.643904in}%
\pgfsys@useobject{currentmarker}{}%
\end{pgfscope}%
\end{pgfscope}%
\begin{pgfscope}%
\pgfpathrectangle{\pgfqpoint{0.764581in}{0.643904in}}{\pgfqpoint{6.200000in}{4.620000in}}%
\pgfusepath{clip}%
\pgfsetbuttcap%
\pgfsetroundjoin%
\pgfsetlinewidth{0.803000pt}%
\definecolor{currentstroke}{rgb}{0.690196,0.690196,0.690196}%
\pgfsetstrokecolor{currentstroke}%
\pgfsetstrokeopacity{0.200000}%
\pgfsetdash{{2.960000pt}{1.280000pt}}{0.000000pt}%
\pgfpathmoveto{\pgfqpoint{2.221390in}{0.643904in}}%
\pgfpathlineto{\pgfqpoint{2.221390in}{5.263904in}}%
\pgfusepath{stroke}%
\end{pgfscope}%
\begin{pgfscope}%
\pgfsetbuttcap%
\pgfsetroundjoin%
\definecolor{currentfill}{rgb}{0.000000,0.000000,0.000000}%
\pgfsetfillcolor{currentfill}%
\pgfsetlinewidth{0.602250pt}%
\definecolor{currentstroke}{rgb}{0.000000,0.000000,0.000000}%
\pgfsetstrokecolor{currentstroke}%
\pgfsetdash{}{0pt}%
\pgfsys@defobject{currentmarker}{\pgfqpoint{0.000000in}{-0.027778in}}{\pgfqpoint{0.000000in}{0.000000in}}{%
\pgfpathmoveto{\pgfqpoint{0.000000in}{0.000000in}}%
\pgfpathlineto{\pgfqpoint{0.000000in}{-0.027778in}}%
\pgfusepath{stroke,fill}%
}%
\begin{pgfscope}%
\pgfsys@transformshift{2.221390in}{0.643904in}%
\pgfsys@useobject{currentmarker}{}%
\end{pgfscope}%
\end{pgfscope}%
\begin{pgfscope}%
\pgfpathrectangle{\pgfqpoint{0.764581in}{0.643904in}}{\pgfqpoint{6.200000in}{4.620000in}}%
\pgfusepath{clip}%
\pgfsetbuttcap%
\pgfsetroundjoin%
\pgfsetlinewidth{0.803000pt}%
\definecolor{currentstroke}{rgb}{0.690196,0.690196,0.690196}%
\pgfsetstrokecolor{currentstroke}%
\pgfsetstrokeopacity{0.200000}%
\pgfsetdash{{2.960000pt}{1.280000pt}}{0.000000pt}%
\pgfpathmoveto{\pgfqpoint{2.455884in}{0.643904in}}%
\pgfpathlineto{\pgfqpoint{2.455884in}{5.263904in}}%
\pgfusepath{stroke}%
\end{pgfscope}%
\begin{pgfscope}%
\pgfsetbuttcap%
\pgfsetroundjoin%
\definecolor{currentfill}{rgb}{0.000000,0.000000,0.000000}%
\pgfsetfillcolor{currentfill}%
\pgfsetlinewidth{0.602250pt}%
\definecolor{currentstroke}{rgb}{0.000000,0.000000,0.000000}%
\pgfsetstrokecolor{currentstroke}%
\pgfsetdash{}{0pt}%
\pgfsys@defobject{currentmarker}{\pgfqpoint{0.000000in}{-0.027778in}}{\pgfqpoint{0.000000in}{0.000000in}}{%
\pgfpathmoveto{\pgfqpoint{0.000000in}{0.000000in}}%
\pgfpathlineto{\pgfqpoint{0.000000in}{-0.027778in}}%
\pgfusepath{stroke,fill}%
}%
\begin{pgfscope}%
\pgfsys@transformshift{2.455884in}{0.643904in}%
\pgfsys@useobject{currentmarker}{}%
\end{pgfscope}%
\end{pgfscope}%
\begin{pgfscope}%
\pgfpathrectangle{\pgfqpoint{0.764581in}{0.643904in}}{\pgfqpoint{6.200000in}{4.620000in}}%
\pgfusepath{clip}%
\pgfsetbuttcap%
\pgfsetroundjoin%
\pgfsetlinewidth{0.803000pt}%
\definecolor{currentstroke}{rgb}{0.690196,0.690196,0.690196}%
\pgfsetstrokecolor{currentstroke}%
\pgfsetstrokeopacity{0.200000}%
\pgfsetdash{{2.960000pt}{1.280000pt}}{0.000000pt}%
\pgfpathmoveto{\pgfqpoint{2.647479in}{0.643904in}}%
\pgfpathlineto{\pgfqpoint{2.647479in}{5.263904in}}%
\pgfusepath{stroke}%
\end{pgfscope}%
\begin{pgfscope}%
\pgfsetbuttcap%
\pgfsetroundjoin%
\definecolor{currentfill}{rgb}{0.000000,0.000000,0.000000}%
\pgfsetfillcolor{currentfill}%
\pgfsetlinewidth{0.602250pt}%
\definecolor{currentstroke}{rgb}{0.000000,0.000000,0.000000}%
\pgfsetstrokecolor{currentstroke}%
\pgfsetdash{}{0pt}%
\pgfsys@defobject{currentmarker}{\pgfqpoint{0.000000in}{-0.027778in}}{\pgfqpoint{0.000000in}{0.000000in}}{%
\pgfpathmoveto{\pgfqpoint{0.000000in}{0.000000in}}%
\pgfpathlineto{\pgfqpoint{0.000000in}{-0.027778in}}%
\pgfusepath{stroke,fill}%
}%
\begin{pgfscope}%
\pgfsys@transformshift{2.647479in}{0.643904in}%
\pgfsys@useobject{currentmarker}{}%
\end{pgfscope}%
\end{pgfscope}%
\begin{pgfscope}%
\pgfpathrectangle{\pgfqpoint{0.764581in}{0.643904in}}{\pgfqpoint{6.200000in}{4.620000in}}%
\pgfusepath{clip}%
\pgfsetbuttcap%
\pgfsetroundjoin%
\pgfsetlinewidth{0.803000pt}%
\definecolor{currentstroke}{rgb}{0.690196,0.690196,0.690196}%
\pgfsetstrokecolor{currentstroke}%
\pgfsetstrokeopacity{0.200000}%
\pgfsetdash{{2.960000pt}{1.280000pt}}{0.000000pt}%
\pgfpathmoveto{\pgfqpoint{2.809471in}{0.643904in}}%
\pgfpathlineto{\pgfqpoint{2.809471in}{5.263904in}}%
\pgfusepath{stroke}%
\end{pgfscope}%
\begin{pgfscope}%
\pgfsetbuttcap%
\pgfsetroundjoin%
\definecolor{currentfill}{rgb}{0.000000,0.000000,0.000000}%
\pgfsetfillcolor{currentfill}%
\pgfsetlinewidth{0.602250pt}%
\definecolor{currentstroke}{rgb}{0.000000,0.000000,0.000000}%
\pgfsetstrokecolor{currentstroke}%
\pgfsetdash{}{0pt}%
\pgfsys@defobject{currentmarker}{\pgfqpoint{0.000000in}{-0.027778in}}{\pgfqpoint{0.000000in}{0.000000in}}{%
\pgfpathmoveto{\pgfqpoint{0.000000in}{0.000000in}}%
\pgfpathlineto{\pgfqpoint{0.000000in}{-0.027778in}}%
\pgfusepath{stroke,fill}%
}%
\begin{pgfscope}%
\pgfsys@transformshift{2.809471in}{0.643904in}%
\pgfsys@useobject{currentmarker}{}%
\end{pgfscope}%
\end{pgfscope}%
\begin{pgfscope}%
\pgfpathrectangle{\pgfqpoint{0.764581in}{0.643904in}}{\pgfqpoint{6.200000in}{4.620000in}}%
\pgfusepath{clip}%
\pgfsetbuttcap%
\pgfsetroundjoin%
\pgfsetlinewidth{0.803000pt}%
\definecolor{currentstroke}{rgb}{0.690196,0.690196,0.690196}%
\pgfsetstrokecolor{currentstroke}%
\pgfsetstrokeopacity{0.200000}%
\pgfsetdash{{2.960000pt}{1.280000pt}}{0.000000pt}%
\pgfpathmoveto{\pgfqpoint{2.949795in}{0.643904in}}%
\pgfpathlineto{\pgfqpoint{2.949795in}{5.263904in}}%
\pgfusepath{stroke}%
\end{pgfscope}%
\begin{pgfscope}%
\pgfsetbuttcap%
\pgfsetroundjoin%
\definecolor{currentfill}{rgb}{0.000000,0.000000,0.000000}%
\pgfsetfillcolor{currentfill}%
\pgfsetlinewidth{0.602250pt}%
\definecolor{currentstroke}{rgb}{0.000000,0.000000,0.000000}%
\pgfsetstrokecolor{currentstroke}%
\pgfsetdash{}{0pt}%
\pgfsys@defobject{currentmarker}{\pgfqpoint{0.000000in}{-0.027778in}}{\pgfqpoint{0.000000in}{0.000000in}}{%
\pgfpathmoveto{\pgfqpoint{0.000000in}{0.000000in}}%
\pgfpathlineto{\pgfqpoint{0.000000in}{-0.027778in}}%
\pgfusepath{stroke,fill}%
}%
\begin{pgfscope}%
\pgfsys@transformshift{2.949795in}{0.643904in}%
\pgfsys@useobject{currentmarker}{}%
\end{pgfscope}%
\end{pgfscope}%
\begin{pgfscope}%
\pgfpathrectangle{\pgfqpoint{0.764581in}{0.643904in}}{\pgfqpoint{6.200000in}{4.620000in}}%
\pgfusepath{clip}%
\pgfsetbuttcap%
\pgfsetroundjoin%
\pgfsetlinewidth{0.803000pt}%
\definecolor{currentstroke}{rgb}{0.690196,0.690196,0.690196}%
\pgfsetstrokecolor{currentstroke}%
\pgfsetstrokeopacity{0.200000}%
\pgfsetdash{{2.960000pt}{1.280000pt}}{0.000000pt}%
\pgfpathmoveto{\pgfqpoint{3.073569in}{0.643904in}}%
\pgfpathlineto{\pgfqpoint{3.073569in}{5.263904in}}%
\pgfusepath{stroke}%
\end{pgfscope}%
\begin{pgfscope}%
\pgfsetbuttcap%
\pgfsetroundjoin%
\definecolor{currentfill}{rgb}{0.000000,0.000000,0.000000}%
\pgfsetfillcolor{currentfill}%
\pgfsetlinewidth{0.602250pt}%
\definecolor{currentstroke}{rgb}{0.000000,0.000000,0.000000}%
\pgfsetstrokecolor{currentstroke}%
\pgfsetdash{}{0pt}%
\pgfsys@defobject{currentmarker}{\pgfqpoint{0.000000in}{-0.027778in}}{\pgfqpoint{0.000000in}{0.000000in}}{%
\pgfpathmoveto{\pgfqpoint{0.000000in}{0.000000in}}%
\pgfpathlineto{\pgfqpoint{0.000000in}{-0.027778in}}%
\pgfusepath{stroke,fill}%
}%
\begin{pgfscope}%
\pgfsys@transformshift{3.073569in}{0.643904in}%
\pgfsys@useobject{currentmarker}{}%
\end{pgfscope}%
\end{pgfscope}%
\begin{pgfscope}%
\pgfpathrectangle{\pgfqpoint{0.764581in}{0.643904in}}{\pgfqpoint{6.200000in}{4.620000in}}%
\pgfusepath{clip}%
\pgfsetbuttcap%
\pgfsetroundjoin%
\pgfsetlinewidth{0.803000pt}%
\definecolor{currentstroke}{rgb}{0.690196,0.690196,0.690196}%
\pgfsetstrokecolor{currentstroke}%
\pgfsetstrokeopacity{0.200000}%
\pgfsetdash{{2.960000pt}{1.280000pt}}{0.000000pt}%
\pgfpathmoveto{\pgfqpoint{3.912693in}{0.643904in}}%
\pgfpathlineto{\pgfqpoint{3.912693in}{5.263904in}}%
\pgfusepath{stroke}%
\end{pgfscope}%
\begin{pgfscope}%
\pgfsetbuttcap%
\pgfsetroundjoin%
\definecolor{currentfill}{rgb}{0.000000,0.000000,0.000000}%
\pgfsetfillcolor{currentfill}%
\pgfsetlinewidth{0.602250pt}%
\definecolor{currentstroke}{rgb}{0.000000,0.000000,0.000000}%
\pgfsetstrokecolor{currentstroke}%
\pgfsetdash{}{0pt}%
\pgfsys@defobject{currentmarker}{\pgfqpoint{0.000000in}{-0.027778in}}{\pgfqpoint{0.000000in}{0.000000in}}{%
\pgfpathmoveto{\pgfqpoint{0.000000in}{0.000000in}}%
\pgfpathlineto{\pgfqpoint{0.000000in}{-0.027778in}}%
\pgfusepath{stroke,fill}%
}%
\begin{pgfscope}%
\pgfsys@transformshift{3.912693in}{0.643904in}%
\pgfsys@useobject{currentmarker}{}%
\end{pgfscope}%
\end{pgfscope}%
\begin{pgfscope}%
\pgfpathrectangle{\pgfqpoint{0.764581in}{0.643904in}}{\pgfqpoint{6.200000in}{4.620000in}}%
\pgfusepath{clip}%
\pgfsetbuttcap%
\pgfsetroundjoin%
\pgfsetlinewidth{0.803000pt}%
\definecolor{currentstroke}{rgb}{0.690196,0.690196,0.690196}%
\pgfsetstrokecolor{currentstroke}%
\pgfsetstrokeopacity{0.200000}%
\pgfsetdash{{2.960000pt}{1.280000pt}}{0.000000pt}%
\pgfpathmoveto{\pgfqpoint{4.338782in}{0.643904in}}%
\pgfpathlineto{\pgfqpoint{4.338782in}{5.263904in}}%
\pgfusepath{stroke}%
\end{pgfscope}%
\begin{pgfscope}%
\pgfsetbuttcap%
\pgfsetroundjoin%
\definecolor{currentfill}{rgb}{0.000000,0.000000,0.000000}%
\pgfsetfillcolor{currentfill}%
\pgfsetlinewidth{0.602250pt}%
\definecolor{currentstroke}{rgb}{0.000000,0.000000,0.000000}%
\pgfsetstrokecolor{currentstroke}%
\pgfsetdash{}{0pt}%
\pgfsys@defobject{currentmarker}{\pgfqpoint{0.000000in}{-0.027778in}}{\pgfqpoint{0.000000in}{0.000000in}}{%
\pgfpathmoveto{\pgfqpoint{0.000000in}{0.000000in}}%
\pgfpathlineto{\pgfqpoint{0.000000in}{-0.027778in}}%
\pgfusepath{stroke,fill}%
}%
\begin{pgfscope}%
\pgfsys@transformshift{4.338782in}{0.643904in}%
\pgfsys@useobject{currentmarker}{}%
\end{pgfscope}%
\end{pgfscope}%
\begin{pgfscope}%
\pgfpathrectangle{\pgfqpoint{0.764581in}{0.643904in}}{\pgfqpoint{6.200000in}{4.620000in}}%
\pgfusepath{clip}%
\pgfsetbuttcap%
\pgfsetroundjoin%
\pgfsetlinewidth{0.803000pt}%
\definecolor{currentstroke}{rgb}{0.690196,0.690196,0.690196}%
\pgfsetstrokecolor{currentstroke}%
\pgfsetstrokeopacity{0.200000}%
\pgfsetdash{{2.960000pt}{1.280000pt}}{0.000000pt}%
\pgfpathmoveto{\pgfqpoint{4.641098in}{0.643904in}}%
\pgfpathlineto{\pgfqpoint{4.641098in}{5.263904in}}%
\pgfusepath{stroke}%
\end{pgfscope}%
\begin{pgfscope}%
\pgfsetbuttcap%
\pgfsetroundjoin%
\definecolor{currentfill}{rgb}{0.000000,0.000000,0.000000}%
\pgfsetfillcolor{currentfill}%
\pgfsetlinewidth{0.602250pt}%
\definecolor{currentstroke}{rgb}{0.000000,0.000000,0.000000}%
\pgfsetstrokecolor{currentstroke}%
\pgfsetdash{}{0pt}%
\pgfsys@defobject{currentmarker}{\pgfqpoint{0.000000in}{-0.027778in}}{\pgfqpoint{0.000000in}{0.000000in}}{%
\pgfpathmoveto{\pgfqpoint{0.000000in}{0.000000in}}%
\pgfpathlineto{\pgfqpoint{0.000000in}{-0.027778in}}%
\pgfusepath{stroke,fill}%
}%
\begin{pgfscope}%
\pgfsys@transformshift{4.641098in}{0.643904in}%
\pgfsys@useobject{currentmarker}{}%
\end{pgfscope}%
\end{pgfscope}%
\begin{pgfscope}%
\pgfpathrectangle{\pgfqpoint{0.764581in}{0.643904in}}{\pgfqpoint{6.200000in}{4.620000in}}%
\pgfusepath{clip}%
\pgfsetbuttcap%
\pgfsetroundjoin%
\pgfsetlinewidth{0.803000pt}%
\definecolor{currentstroke}{rgb}{0.690196,0.690196,0.690196}%
\pgfsetstrokecolor{currentstroke}%
\pgfsetstrokeopacity{0.200000}%
\pgfsetdash{{2.960000pt}{1.280000pt}}{0.000000pt}%
\pgfpathmoveto{\pgfqpoint{4.875592in}{0.643904in}}%
\pgfpathlineto{\pgfqpoint{4.875592in}{5.263904in}}%
\pgfusepath{stroke}%
\end{pgfscope}%
\begin{pgfscope}%
\pgfsetbuttcap%
\pgfsetroundjoin%
\definecolor{currentfill}{rgb}{0.000000,0.000000,0.000000}%
\pgfsetfillcolor{currentfill}%
\pgfsetlinewidth{0.602250pt}%
\definecolor{currentstroke}{rgb}{0.000000,0.000000,0.000000}%
\pgfsetstrokecolor{currentstroke}%
\pgfsetdash{}{0pt}%
\pgfsys@defobject{currentmarker}{\pgfqpoint{0.000000in}{-0.027778in}}{\pgfqpoint{0.000000in}{0.000000in}}{%
\pgfpathmoveto{\pgfqpoint{0.000000in}{0.000000in}}%
\pgfpathlineto{\pgfqpoint{0.000000in}{-0.027778in}}%
\pgfusepath{stroke,fill}%
}%
\begin{pgfscope}%
\pgfsys@transformshift{4.875592in}{0.643904in}%
\pgfsys@useobject{currentmarker}{}%
\end{pgfscope}%
\end{pgfscope}%
\begin{pgfscope}%
\pgfpathrectangle{\pgfqpoint{0.764581in}{0.643904in}}{\pgfqpoint{6.200000in}{4.620000in}}%
\pgfusepath{clip}%
\pgfsetbuttcap%
\pgfsetroundjoin%
\pgfsetlinewidth{0.803000pt}%
\definecolor{currentstroke}{rgb}{0.690196,0.690196,0.690196}%
\pgfsetstrokecolor{currentstroke}%
\pgfsetstrokeopacity{0.200000}%
\pgfsetdash{{2.960000pt}{1.280000pt}}{0.000000pt}%
\pgfpathmoveto{\pgfqpoint{5.067187in}{0.643904in}}%
\pgfpathlineto{\pgfqpoint{5.067187in}{5.263904in}}%
\pgfusepath{stroke}%
\end{pgfscope}%
\begin{pgfscope}%
\pgfsetbuttcap%
\pgfsetroundjoin%
\definecolor{currentfill}{rgb}{0.000000,0.000000,0.000000}%
\pgfsetfillcolor{currentfill}%
\pgfsetlinewidth{0.602250pt}%
\definecolor{currentstroke}{rgb}{0.000000,0.000000,0.000000}%
\pgfsetstrokecolor{currentstroke}%
\pgfsetdash{}{0pt}%
\pgfsys@defobject{currentmarker}{\pgfqpoint{0.000000in}{-0.027778in}}{\pgfqpoint{0.000000in}{0.000000in}}{%
\pgfpathmoveto{\pgfqpoint{0.000000in}{0.000000in}}%
\pgfpathlineto{\pgfqpoint{0.000000in}{-0.027778in}}%
\pgfusepath{stroke,fill}%
}%
\begin{pgfscope}%
\pgfsys@transformshift{5.067187in}{0.643904in}%
\pgfsys@useobject{currentmarker}{}%
\end{pgfscope}%
\end{pgfscope}%
\begin{pgfscope}%
\pgfpathrectangle{\pgfqpoint{0.764581in}{0.643904in}}{\pgfqpoint{6.200000in}{4.620000in}}%
\pgfusepath{clip}%
\pgfsetbuttcap%
\pgfsetroundjoin%
\pgfsetlinewidth{0.803000pt}%
\definecolor{currentstroke}{rgb}{0.690196,0.690196,0.690196}%
\pgfsetstrokecolor{currentstroke}%
\pgfsetstrokeopacity{0.200000}%
\pgfsetdash{{2.960000pt}{1.280000pt}}{0.000000pt}%
\pgfpathmoveto{\pgfqpoint{5.229179in}{0.643904in}}%
\pgfpathlineto{\pgfqpoint{5.229179in}{5.263904in}}%
\pgfusepath{stroke}%
\end{pgfscope}%
\begin{pgfscope}%
\pgfsetbuttcap%
\pgfsetroundjoin%
\definecolor{currentfill}{rgb}{0.000000,0.000000,0.000000}%
\pgfsetfillcolor{currentfill}%
\pgfsetlinewidth{0.602250pt}%
\definecolor{currentstroke}{rgb}{0.000000,0.000000,0.000000}%
\pgfsetstrokecolor{currentstroke}%
\pgfsetdash{}{0pt}%
\pgfsys@defobject{currentmarker}{\pgfqpoint{0.000000in}{-0.027778in}}{\pgfqpoint{0.000000in}{0.000000in}}{%
\pgfpathmoveto{\pgfqpoint{0.000000in}{0.000000in}}%
\pgfpathlineto{\pgfqpoint{0.000000in}{-0.027778in}}%
\pgfusepath{stroke,fill}%
}%
\begin{pgfscope}%
\pgfsys@transformshift{5.229179in}{0.643904in}%
\pgfsys@useobject{currentmarker}{}%
\end{pgfscope}%
\end{pgfscope}%
\begin{pgfscope}%
\pgfpathrectangle{\pgfqpoint{0.764581in}{0.643904in}}{\pgfqpoint{6.200000in}{4.620000in}}%
\pgfusepath{clip}%
\pgfsetbuttcap%
\pgfsetroundjoin%
\pgfsetlinewidth{0.803000pt}%
\definecolor{currentstroke}{rgb}{0.690196,0.690196,0.690196}%
\pgfsetstrokecolor{currentstroke}%
\pgfsetstrokeopacity{0.200000}%
\pgfsetdash{{2.960000pt}{1.280000pt}}{0.000000pt}%
\pgfpathmoveto{\pgfqpoint{5.369502in}{0.643904in}}%
\pgfpathlineto{\pgfqpoint{5.369502in}{5.263904in}}%
\pgfusepath{stroke}%
\end{pgfscope}%
\begin{pgfscope}%
\pgfsetbuttcap%
\pgfsetroundjoin%
\definecolor{currentfill}{rgb}{0.000000,0.000000,0.000000}%
\pgfsetfillcolor{currentfill}%
\pgfsetlinewidth{0.602250pt}%
\definecolor{currentstroke}{rgb}{0.000000,0.000000,0.000000}%
\pgfsetstrokecolor{currentstroke}%
\pgfsetdash{}{0pt}%
\pgfsys@defobject{currentmarker}{\pgfqpoint{0.000000in}{-0.027778in}}{\pgfqpoint{0.000000in}{0.000000in}}{%
\pgfpathmoveto{\pgfqpoint{0.000000in}{0.000000in}}%
\pgfpathlineto{\pgfqpoint{0.000000in}{-0.027778in}}%
\pgfusepath{stroke,fill}%
}%
\begin{pgfscope}%
\pgfsys@transformshift{5.369502in}{0.643904in}%
\pgfsys@useobject{currentmarker}{}%
\end{pgfscope}%
\end{pgfscope}%
\begin{pgfscope}%
\pgfpathrectangle{\pgfqpoint{0.764581in}{0.643904in}}{\pgfqpoint{6.200000in}{4.620000in}}%
\pgfusepath{clip}%
\pgfsetbuttcap%
\pgfsetroundjoin%
\pgfsetlinewidth{0.803000pt}%
\definecolor{currentstroke}{rgb}{0.690196,0.690196,0.690196}%
\pgfsetstrokecolor{currentstroke}%
\pgfsetstrokeopacity{0.200000}%
\pgfsetdash{{2.960000pt}{1.280000pt}}{0.000000pt}%
\pgfpathmoveto{\pgfqpoint{5.493277in}{0.643904in}}%
\pgfpathlineto{\pgfqpoint{5.493277in}{5.263904in}}%
\pgfusepath{stroke}%
\end{pgfscope}%
\begin{pgfscope}%
\pgfsetbuttcap%
\pgfsetroundjoin%
\definecolor{currentfill}{rgb}{0.000000,0.000000,0.000000}%
\pgfsetfillcolor{currentfill}%
\pgfsetlinewidth{0.602250pt}%
\definecolor{currentstroke}{rgb}{0.000000,0.000000,0.000000}%
\pgfsetstrokecolor{currentstroke}%
\pgfsetdash{}{0pt}%
\pgfsys@defobject{currentmarker}{\pgfqpoint{0.000000in}{-0.027778in}}{\pgfqpoint{0.000000in}{0.000000in}}{%
\pgfpathmoveto{\pgfqpoint{0.000000in}{0.000000in}}%
\pgfpathlineto{\pgfqpoint{0.000000in}{-0.027778in}}%
\pgfusepath{stroke,fill}%
}%
\begin{pgfscope}%
\pgfsys@transformshift{5.493277in}{0.643904in}%
\pgfsys@useobject{currentmarker}{}%
\end{pgfscope}%
\end{pgfscope}%
\begin{pgfscope}%
\pgfpathrectangle{\pgfqpoint{0.764581in}{0.643904in}}{\pgfqpoint{6.200000in}{4.620000in}}%
\pgfusepath{clip}%
\pgfsetbuttcap%
\pgfsetroundjoin%
\pgfsetlinewidth{0.803000pt}%
\definecolor{currentstroke}{rgb}{0.690196,0.690196,0.690196}%
\pgfsetstrokecolor{currentstroke}%
\pgfsetstrokeopacity{0.200000}%
\pgfsetdash{{2.960000pt}{1.280000pt}}{0.000000pt}%
\pgfpathmoveto{\pgfqpoint{6.332401in}{0.643904in}}%
\pgfpathlineto{\pgfqpoint{6.332401in}{5.263904in}}%
\pgfusepath{stroke}%
\end{pgfscope}%
\begin{pgfscope}%
\pgfsetbuttcap%
\pgfsetroundjoin%
\definecolor{currentfill}{rgb}{0.000000,0.000000,0.000000}%
\pgfsetfillcolor{currentfill}%
\pgfsetlinewidth{0.602250pt}%
\definecolor{currentstroke}{rgb}{0.000000,0.000000,0.000000}%
\pgfsetstrokecolor{currentstroke}%
\pgfsetdash{}{0pt}%
\pgfsys@defobject{currentmarker}{\pgfqpoint{0.000000in}{-0.027778in}}{\pgfqpoint{0.000000in}{0.000000in}}{%
\pgfpathmoveto{\pgfqpoint{0.000000in}{0.000000in}}%
\pgfpathlineto{\pgfqpoint{0.000000in}{-0.027778in}}%
\pgfusepath{stroke,fill}%
}%
\begin{pgfscope}%
\pgfsys@transformshift{6.332401in}{0.643904in}%
\pgfsys@useobject{currentmarker}{}%
\end{pgfscope}%
\end{pgfscope}%
\begin{pgfscope}%
\pgfpathrectangle{\pgfqpoint{0.764581in}{0.643904in}}{\pgfqpoint{6.200000in}{4.620000in}}%
\pgfusepath{clip}%
\pgfsetbuttcap%
\pgfsetroundjoin%
\pgfsetlinewidth{0.803000pt}%
\definecolor{currentstroke}{rgb}{0.690196,0.690196,0.690196}%
\pgfsetstrokecolor{currentstroke}%
\pgfsetstrokeopacity{0.200000}%
\pgfsetdash{{2.960000pt}{1.280000pt}}{0.000000pt}%
\pgfpathmoveto{\pgfqpoint{6.758490in}{0.643904in}}%
\pgfpathlineto{\pgfqpoint{6.758490in}{5.263904in}}%
\pgfusepath{stroke}%
\end{pgfscope}%
\begin{pgfscope}%
\pgfsetbuttcap%
\pgfsetroundjoin%
\definecolor{currentfill}{rgb}{0.000000,0.000000,0.000000}%
\pgfsetfillcolor{currentfill}%
\pgfsetlinewidth{0.602250pt}%
\definecolor{currentstroke}{rgb}{0.000000,0.000000,0.000000}%
\pgfsetstrokecolor{currentstroke}%
\pgfsetdash{}{0pt}%
\pgfsys@defobject{currentmarker}{\pgfqpoint{0.000000in}{-0.027778in}}{\pgfqpoint{0.000000in}{0.000000in}}{%
\pgfpathmoveto{\pgfqpoint{0.000000in}{0.000000in}}%
\pgfpathlineto{\pgfqpoint{0.000000in}{-0.027778in}}%
\pgfusepath{stroke,fill}%
}%
\begin{pgfscope}%
\pgfsys@transformshift{6.758490in}{0.643904in}%
\pgfsys@useobject{currentmarker}{}%
\end{pgfscope}%
\end{pgfscope}%
\begin{pgfscope}%
\definecolor{textcolor}{rgb}{0.000000,0.000000,0.000000}%
\pgfsetstrokecolor{textcolor}%
\pgfsetfillcolor{textcolor}%
\pgftext[x=3.864581in,y=0.313349in,,top]{\color{textcolor}{\rmfamily\fontsize{18.000000}{21.600000}\selectfont\catcode`\^=\active\def^{\ifmmode\sp\else\^{}\fi}\catcode`\%=\active\def%{\%}Number of Modeled Days}}%
\end{pgfscope}%
\begin{pgfscope}%
\pgfpathrectangle{\pgfqpoint{0.764581in}{0.643904in}}{\pgfqpoint{6.200000in}{4.620000in}}%
\pgfusepath{clip}%
\pgfsetrectcap%
\pgfsetroundjoin%
\pgfsetlinewidth{0.803000pt}%
\definecolor{currentstroke}{rgb}{0.690196,0.690196,0.690196}%
\pgfsetstrokecolor{currentstroke}%
\pgfsetdash{}{0pt}%
\pgfpathmoveto{\pgfqpoint{0.764581in}{2.067685in}}%
\pgfpathlineto{\pgfqpoint{6.964581in}{2.067685in}}%
\pgfusepath{stroke}%
\end{pgfscope}%
\begin{pgfscope}%
\pgfsetbuttcap%
\pgfsetroundjoin%
\definecolor{currentfill}{rgb}{0.000000,0.000000,0.000000}%
\pgfsetfillcolor{currentfill}%
\pgfsetlinewidth{0.803000pt}%
\definecolor{currentstroke}{rgb}{0.000000,0.000000,0.000000}%
\pgfsetstrokecolor{currentstroke}%
\pgfsetdash{}{0pt}%
\pgfsys@defobject{currentmarker}{\pgfqpoint{-0.048611in}{0.000000in}}{\pgfqpoint{-0.000000in}{0.000000in}}{%
\pgfpathmoveto{\pgfqpoint{-0.000000in}{0.000000in}}%
\pgfpathlineto{\pgfqpoint{-0.048611in}{0.000000in}}%
\pgfusepath{stroke,fill}%
}%
\begin{pgfscope}%
\pgfsys@transformshift{0.764581in}{2.067685in}%
\pgfsys@useobject{currentmarker}{}%
\end{pgfscope}%
\end{pgfscope}%
\begin{pgfscope}%
\definecolor{textcolor}{rgb}{0.000000,0.000000,0.000000}%
\pgfsetstrokecolor{textcolor}%
\pgfsetfillcolor{textcolor}%
\pgftext[x=0.395138in, y=1.998240in, left, base]{\color{textcolor}{\rmfamily\fontsize{14.000000}{16.800000}\selectfont\catcode`\^=\active\def^{\ifmmode\sp\else\^{}\fi}\catcode`\%=\active\def%{\%}$\mathdefault{10^{1}}$}}%
\end{pgfscope}%
\begin{pgfscope}%
\pgfpathrectangle{\pgfqpoint{0.764581in}{0.643904in}}{\pgfqpoint{6.200000in}{4.620000in}}%
\pgfusepath{clip}%
\pgfsetrectcap%
\pgfsetroundjoin%
\pgfsetlinewidth{0.803000pt}%
\definecolor{currentstroke}{rgb}{0.690196,0.690196,0.690196}%
\pgfsetstrokecolor{currentstroke}%
\pgfsetdash{}{0pt}%
\pgfpathmoveto{\pgfqpoint{0.764581in}{3.655018in}}%
\pgfpathlineto{\pgfqpoint{6.964581in}{3.655018in}}%
\pgfusepath{stroke}%
\end{pgfscope}%
\begin{pgfscope}%
\pgfsetbuttcap%
\pgfsetroundjoin%
\definecolor{currentfill}{rgb}{0.000000,0.000000,0.000000}%
\pgfsetfillcolor{currentfill}%
\pgfsetlinewidth{0.803000pt}%
\definecolor{currentstroke}{rgb}{0.000000,0.000000,0.000000}%
\pgfsetstrokecolor{currentstroke}%
\pgfsetdash{}{0pt}%
\pgfsys@defobject{currentmarker}{\pgfqpoint{-0.048611in}{0.000000in}}{\pgfqpoint{-0.000000in}{0.000000in}}{%
\pgfpathmoveto{\pgfqpoint{-0.000000in}{0.000000in}}%
\pgfpathlineto{\pgfqpoint{-0.048611in}{0.000000in}}%
\pgfusepath{stroke,fill}%
}%
\begin{pgfscope}%
\pgfsys@transformshift{0.764581in}{3.655018in}%
\pgfsys@useobject{currentmarker}{}%
\end{pgfscope}%
\end{pgfscope}%
\begin{pgfscope}%
\definecolor{textcolor}{rgb}{0.000000,0.000000,0.000000}%
\pgfsetstrokecolor{textcolor}%
\pgfsetfillcolor{textcolor}%
\pgftext[x=0.395138in, y=3.585574in, left, base]{\color{textcolor}{\rmfamily\fontsize{14.000000}{16.800000}\selectfont\catcode`\^=\active\def^{\ifmmode\sp\else\^{}\fi}\catcode`\%=\active\def%{\%}$\mathdefault{10^{2}}$}}%
\end{pgfscope}%
\begin{pgfscope}%
\pgfpathrectangle{\pgfqpoint{0.764581in}{0.643904in}}{\pgfqpoint{6.200000in}{4.620000in}}%
\pgfusepath{clip}%
\pgfsetrectcap%
\pgfsetroundjoin%
\pgfsetlinewidth{0.803000pt}%
\definecolor{currentstroke}{rgb}{0.690196,0.690196,0.690196}%
\pgfsetstrokecolor{currentstroke}%
\pgfsetdash{}{0pt}%
\pgfpathmoveto{\pgfqpoint{0.764581in}{5.242352in}}%
\pgfpathlineto{\pgfqpoint{6.964581in}{5.242352in}}%
\pgfusepath{stroke}%
\end{pgfscope}%
\begin{pgfscope}%
\pgfsetbuttcap%
\pgfsetroundjoin%
\definecolor{currentfill}{rgb}{0.000000,0.000000,0.000000}%
\pgfsetfillcolor{currentfill}%
\pgfsetlinewidth{0.803000pt}%
\definecolor{currentstroke}{rgb}{0.000000,0.000000,0.000000}%
\pgfsetstrokecolor{currentstroke}%
\pgfsetdash{}{0pt}%
\pgfsys@defobject{currentmarker}{\pgfqpoint{-0.048611in}{0.000000in}}{\pgfqpoint{-0.000000in}{0.000000in}}{%
\pgfpathmoveto{\pgfqpoint{-0.000000in}{0.000000in}}%
\pgfpathlineto{\pgfqpoint{-0.048611in}{0.000000in}}%
\pgfusepath{stroke,fill}%
}%
\begin{pgfscope}%
\pgfsys@transformshift{0.764581in}{5.242352in}%
\pgfsys@useobject{currentmarker}{}%
\end{pgfscope}%
\end{pgfscope}%
\begin{pgfscope}%
\definecolor{textcolor}{rgb}{0.000000,0.000000,0.000000}%
\pgfsetstrokecolor{textcolor}%
\pgfsetfillcolor{textcolor}%
\pgftext[x=0.395138in, y=5.172908in, left, base]{\color{textcolor}{\rmfamily\fontsize{14.000000}{16.800000}\selectfont\catcode`\^=\active\def^{\ifmmode\sp\else\^{}\fi}\catcode`\%=\active\def%{\%}$\mathdefault{10^{3}}$}}%
\end{pgfscope}%
\begin{pgfscope}%
\pgfpathrectangle{\pgfqpoint{0.764581in}{0.643904in}}{\pgfqpoint{6.200000in}{4.620000in}}%
\pgfusepath{clip}%
\pgfsetbuttcap%
\pgfsetroundjoin%
\pgfsetlinewidth{0.803000pt}%
\definecolor{currentstroke}{rgb}{0.690196,0.690196,0.690196}%
\pgfsetstrokecolor{currentstroke}%
\pgfsetstrokeopacity{0.200000}%
\pgfsetdash{{2.960000pt}{1.280000pt}}{0.000000pt}%
\pgfpathmoveto{\pgfqpoint{0.764581in}{0.958186in}}%
\pgfpathlineto{\pgfqpoint{6.964581in}{0.958186in}}%
\pgfusepath{stroke}%
\end{pgfscope}%
\begin{pgfscope}%
\pgfsetbuttcap%
\pgfsetroundjoin%
\definecolor{currentfill}{rgb}{0.000000,0.000000,0.000000}%
\pgfsetfillcolor{currentfill}%
\pgfsetlinewidth{0.602250pt}%
\definecolor{currentstroke}{rgb}{0.000000,0.000000,0.000000}%
\pgfsetstrokecolor{currentstroke}%
\pgfsetdash{}{0pt}%
\pgfsys@defobject{currentmarker}{\pgfqpoint{-0.027778in}{0.000000in}}{\pgfqpoint{-0.000000in}{0.000000in}}{%
\pgfpathmoveto{\pgfqpoint{-0.000000in}{0.000000in}}%
\pgfpathlineto{\pgfqpoint{-0.027778in}{0.000000in}}%
\pgfusepath{stroke,fill}%
}%
\begin{pgfscope}%
\pgfsys@transformshift{0.764581in}{0.958186in}%
\pgfsys@useobject{currentmarker}{}%
\end{pgfscope}%
\end{pgfscope}%
\begin{pgfscope}%
\pgfpathrectangle{\pgfqpoint{0.764581in}{0.643904in}}{\pgfqpoint{6.200000in}{4.620000in}}%
\pgfusepath{clip}%
\pgfsetbuttcap%
\pgfsetroundjoin%
\pgfsetlinewidth{0.803000pt}%
\definecolor{currentstroke}{rgb}{0.690196,0.690196,0.690196}%
\pgfsetstrokecolor{currentstroke}%
\pgfsetstrokeopacity{0.200000}%
\pgfsetdash{{2.960000pt}{1.280000pt}}{0.000000pt}%
\pgfpathmoveto{\pgfqpoint{0.764581in}{1.237701in}}%
\pgfpathlineto{\pgfqpoint{6.964581in}{1.237701in}}%
\pgfusepath{stroke}%
\end{pgfscope}%
\begin{pgfscope}%
\pgfsetbuttcap%
\pgfsetroundjoin%
\definecolor{currentfill}{rgb}{0.000000,0.000000,0.000000}%
\pgfsetfillcolor{currentfill}%
\pgfsetlinewidth{0.602250pt}%
\definecolor{currentstroke}{rgb}{0.000000,0.000000,0.000000}%
\pgfsetstrokecolor{currentstroke}%
\pgfsetdash{}{0pt}%
\pgfsys@defobject{currentmarker}{\pgfqpoint{-0.027778in}{0.000000in}}{\pgfqpoint{-0.000000in}{0.000000in}}{%
\pgfpathmoveto{\pgfqpoint{-0.000000in}{0.000000in}}%
\pgfpathlineto{\pgfqpoint{-0.027778in}{0.000000in}}%
\pgfusepath{stroke,fill}%
}%
\begin{pgfscope}%
\pgfsys@transformshift{0.764581in}{1.237701in}%
\pgfsys@useobject{currentmarker}{}%
\end{pgfscope}%
\end{pgfscope}%
\begin{pgfscope}%
\pgfpathrectangle{\pgfqpoint{0.764581in}{0.643904in}}{\pgfqpoint{6.200000in}{4.620000in}}%
\pgfusepath{clip}%
\pgfsetbuttcap%
\pgfsetroundjoin%
\pgfsetlinewidth{0.803000pt}%
\definecolor{currentstroke}{rgb}{0.690196,0.690196,0.690196}%
\pgfsetstrokecolor{currentstroke}%
\pgfsetstrokeopacity{0.200000}%
\pgfsetdash{{2.960000pt}{1.280000pt}}{0.000000pt}%
\pgfpathmoveto{\pgfqpoint{0.764581in}{1.436021in}}%
\pgfpathlineto{\pgfqpoint{6.964581in}{1.436021in}}%
\pgfusepath{stroke}%
\end{pgfscope}%
\begin{pgfscope}%
\pgfsetbuttcap%
\pgfsetroundjoin%
\definecolor{currentfill}{rgb}{0.000000,0.000000,0.000000}%
\pgfsetfillcolor{currentfill}%
\pgfsetlinewidth{0.602250pt}%
\definecolor{currentstroke}{rgb}{0.000000,0.000000,0.000000}%
\pgfsetstrokecolor{currentstroke}%
\pgfsetdash{}{0pt}%
\pgfsys@defobject{currentmarker}{\pgfqpoint{-0.027778in}{0.000000in}}{\pgfqpoint{-0.000000in}{0.000000in}}{%
\pgfpathmoveto{\pgfqpoint{-0.000000in}{0.000000in}}%
\pgfpathlineto{\pgfqpoint{-0.027778in}{0.000000in}}%
\pgfusepath{stroke,fill}%
}%
\begin{pgfscope}%
\pgfsys@transformshift{0.764581in}{1.436021in}%
\pgfsys@useobject{currentmarker}{}%
\end{pgfscope}%
\end{pgfscope}%
\begin{pgfscope}%
\pgfpathrectangle{\pgfqpoint{0.764581in}{0.643904in}}{\pgfqpoint{6.200000in}{4.620000in}}%
\pgfusepath{clip}%
\pgfsetbuttcap%
\pgfsetroundjoin%
\pgfsetlinewidth{0.803000pt}%
\definecolor{currentstroke}{rgb}{0.690196,0.690196,0.690196}%
\pgfsetstrokecolor{currentstroke}%
\pgfsetstrokeopacity{0.200000}%
\pgfsetdash{{2.960000pt}{1.280000pt}}{0.000000pt}%
\pgfpathmoveto{\pgfqpoint{0.764581in}{1.589849in}}%
\pgfpathlineto{\pgfqpoint{6.964581in}{1.589849in}}%
\pgfusepath{stroke}%
\end{pgfscope}%
\begin{pgfscope}%
\pgfsetbuttcap%
\pgfsetroundjoin%
\definecolor{currentfill}{rgb}{0.000000,0.000000,0.000000}%
\pgfsetfillcolor{currentfill}%
\pgfsetlinewidth{0.602250pt}%
\definecolor{currentstroke}{rgb}{0.000000,0.000000,0.000000}%
\pgfsetstrokecolor{currentstroke}%
\pgfsetdash{}{0pt}%
\pgfsys@defobject{currentmarker}{\pgfqpoint{-0.027778in}{0.000000in}}{\pgfqpoint{-0.000000in}{0.000000in}}{%
\pgfpathmoveto{\pgfqpoint{-0.000000in}{0.000000in}}%
\pgfpathlineto{\pgfqpoint{-0.027778in}{0.000000in}}%
\pgfusepath{stroke,fill}%
}%
\begin{pgfscope}%
\pgfsys@transformshift{0.764581in}{1.589849in}%
\pgfsys@useobject{currentmarker}{}%
\end{pgfscope}%
\end{pgfscope}%
\begin{pgfscope}%
\pgfpathrectangle{\pgfqpoint{0.764581in}{0.643904in}}{\pgfqpoint{6.200000in}{4.620000in}}%
\pgfusepath{clip}%
\pgfsetbuttcap%
\pgfsetroundjoin%
\pgfsetlinewidth{0.803000pt}%
\definecolor{currentstroke}{rgb}{0.690196,0.690196,0.690196}%
\pgfsetstrokecolor{currentstroke}%
\pgfsetstrokeopacity{0.200000}%
\pgfsetdash{{2.960000pt}{1.280000pt}}{0.000000pt}%
\pgfpathmoveto{\pgfqpoint{0.764581in}{1.715536in}}%
\pgfpathlineto{\pgfqpoint{6.964581in}{1.715536in}}%
\pgfusepath{stroke}%
\end{pgfscope}%
\begin{pgfscope}%
\pgfsetbuttcap%
\pgfsetroundjoin%
\definecolor{currentfill}{rgb}{0.000000,0.000000,0.000000}%
\pgfsetfillcolor{currentfill}%
\pgfsetlinewidth{0.602250pt}%
\definecolor{currentstroke}{rgb}{0.000000,0.000000,0.000000}%
\pgfsetstrokecolor{currentstroke}%
\pgfsetdash{}{0pt}%
\pgfsys@defobject{currentmarker}{\pgfqpoint{-0.027778in}{0.000000in}}{\pgfqpoint{-0.000000in}{0.000000in}}{%
\pgfpathmoveto{\pgfqpoint{-0.000000in}{0.000000in}}%
\pgfpathlineto{\pgfqpoint{-0.027778in}{0.000000in}}%
\pgfusepath{stroke,fill}%
}%
\begin{pgfscope}%
\pgfsys@transformshift{0.764581in}{1.715536in}%
\pgfsys@useobject{currentmarker}{}%
\end{pgfscope}%
\end{pgfscope}%
\begin{pgfscope}%
\pgfpathrectangle{\pgfqpoint{0.764581in}{0.643904in}}{\pgfqpoint{6.200000in}{4.620000in}}%
\pgfusepath{clip}%
\pgfsetbuttcap%
\pgfsetroundjoin%
\pgfsetlinewidth{0.803000pt}%
\definecolor{currentstroke}{rgb}{0.690196,0.690196,0.690196}%
\pgfsetstrokecolor{currentstroke}%
\pgfsetstrokeopacity{0.200000}%
\pgfsetdash{{2.960000pt}{1.280000pt}}{0.000000pt}%
\pgfpathmoveto{\pgfqpoint{0.764581in}{1.821803in}}%
\pgfpathlineto{\pgfqpoint{6.964581in}{1.821803in}}%
\pgfusepath{stroke}%
\end{pgfscope}%
\begin{pgfscope}%
\pgfsetbuttcap%
\pgfsetroundjoin%
\definecolor{currentfill}{rgb}{0.000000,0.000000,0.000000}%
\pgfsetfillcolor{currentfill}%
\pgfsetlinewidth{0.602250pt}%
\definecolor{currentstroke}{rgb}{0.000000,0.000000,0.000000}%
\pgfsetstrokecolor{currentstroke}%
\pgfsetdash{}{0pt}%
\pgfsys@defobject{currentmarker}{\pgfqpoint{-0.027778in}{0.000000in}}{\pgfqpoint{-0.000000in}{0.000000in}}{%
\pgfpathmoveto{\pgfqpoint{-0.000000in}{0.000000in}}%
\pgfpathlineto{\pgfqpoint{-0.027778in}{0.000000in}}%
\pgfusepath{stroke,fill}%
}%
\begin{pgfscope}%
\pgfsys@transformshift{0.764581in}{1.821803in}%
\pgfsys@useobject{currentmarker}{}%
\end{pgfscope}%
\end{pgfscope}%
\begin{pgfscope}%
\pgfpathrectangle{\pgfqpoint{0.764581in}{0.643904in}}{\pgfqpoint{6.200000in}{4.620000in}}%
\pgfusepath{clip}%
\pgfsetbuttcap%
\pgfsetroundjoin%
\pgfsetlinewidth{0.803000pt}%
\definecolor{currentstroke}{rgb}{0.690196,0.690196,0.690196}%
\pgfsetstrokecolor{currentstroke}%
\pgfsetstrokeopacity{0.200000}%
\pgfsetdash{{2.960000pt}{1.280000pt}}{0.000000pt}%
\pgfpathmoveto{\pgfqpoint{0.764581in}{1.913856in}}%
\pgfpathlineto{\pgfqpoint{6.964581in}{1.913856in}}%
\pgfusepath{stroke}%
\end{pgfscope}%
\begin{pgfscope}%
\pgfsetbuttcap%
\pgfsetroundjoin%
\definecolor{currentfill}{rgb}{0.000000,0.000000,0.000000}%
\pgfsetfillcolor{currentfill}%
\pgfsetlinewidth{0.602250pt}%
\definecolor{currentstroke}{rgb}{0.000000,0.000000,0.000000}%
\pgfsetstrokecolor{currentstroke}%
\pgfsetdash{}{0pt}%
\pgfsys@defobject{currentmarker}{\pgfqpoint{-0.027778in}{0.000000in}}{\pgfqpoint{-0.000000in}{0.000000in}}{%
\pgfpathmoveto{\pgfqpoint{-0.000000in}{0.000000in}}%
\pgfpathlineto{\pgfqpoint{-0.027778in}{0.000000in}}%
\pgfusepath{stroke,fill}%
}%
\begin{pgfscope}%
\pgfsys@transformshift{0.764581in}{1.913856in}%
\pgfsys@useobject{currentmarker}{}%
\end{pgfscope}%
\end{pgfscope}%
\begin{pgfscope}%
\pgfpathrectangle{\pgfqpoint{0.764581in}{0.643904in}}{\pgfqpoint{6.200000in}{4.620000in}}%
\pgfusepath{clip}%
\pgfsetbuttcap%
\pgfsetroundjoin%
\pgfsetlinewidth{0.803000pt}%
\definecolor{currentstroke}{rgb}{0.690196,0.690196,0.690196}%
\pgfsetstrokecolor{currentstroke}%
\pgfsetstrokeopacity{0.200000}%
\pgfsetdash{{2.960000pt}{1.280000pt}}{0.000000pt}%
\pgfpathmoveto{\pgfqpoint{0.764581in}{1.995052in}}%
\pgfpathlineto{\pgfqpoint{6.964581in}{1.995052in}}%
\pgfusepath{stroke}%
\end{pgfscope}%
\begin{pgfscope}%
\pgfsetbuttcap%
\pgfsetroundjoin%
\definecolor{currentfill}{rgb}{0.000000,0.000000,0.000000}%
\pgfsetfillcolor{currentfill}%
\pgfsetlinewidth{0.602250pt}%
\definecolor{currentstroke}{rgb}{0.000000,0.000000,0.000000}%
\pgfsetstrokecolor{currentstroke}%
\pgfsetdash{}{0pt}%
\pgfsys@defobject{currentmarker}{\pgfqpoint{-0.027778in}{0.000000in}}{\pgfqpoint{-0.000000in}{0.000000in}}{%
\pgfpathmoveto{\pgfqpoint{-0.000000in}{0.000000in}}%
\pgfpathlineto{\pgfqpoint{-0.027778in}{0.000000in}}%
\pgfusepath{stroke,fill}%
}%
\begin{pgfscope}%
\pgfsys@transformshift{0.764581in}{1.995052in}%
\pgfsys@useobject{currentmarker}{}%
\end{pgfscope}%
\end{pgfscope}%
\begin{pgfscope}%
\pgfpathrectangle{\pgfqpoint{0.764581in}{0.643904in}}{\pgfqpoint{6.200000in}{4.620000in}}%
\pgfusepath{clip}%
\pgfsetbuttcap%
\pgfsetroundjoin%
\pgfsetlinewidth{0.803000pt}%
\definecolor{currentstroke}{rgb}{0.690196,0.690196,0.690196}%
\pgfsetstrokecolor{currentstroke}%
\pgfsetstrokeopacity{0.200000}%
\pgfsetdash{{2.960000pt}{1.280000pt}}{0.000000pt}%
\pgfpathmoveto{\pgfqpoint{0.764581in}{2.545520in}}%
\pgfpathlineto{\pgfqpoint{6.964581in}{2.545520in}}%
\pgfusepath{stroke}%
\end{pgfscope}%
\begin{pgfscope}%
\pgfsetbuttcap%
\pgfsetroundjoin%
\definecolor{currentfill}{rgb}{0.000000,0.000000,0.000000}%
\pgfsetfillcolor{currentfill}%
\pgfsetlinewidth{0.602250pt}%
\definecolor{currentstroke}{rgb}{0.000000,0.000000,0.000000}%
\pgfsetstrokecolor{currentstroke}%
\pgfsetdash{}{0pt}%
\pgfsys@defobject{currentmarker}{\pgfqpoint{-0.027778in}{0.000000in}}{\pgfqpoint{-0.000000in}{0.000000in}}{%
\pgfpathmoveto{\pgfqpoint{-0.000000in}{0.000000in}}%
\pgfpathlineto{\pgfqpoint{-0.027778in}{0.000000in}}%
\pgfusepath{stroke,fill}%
}%
\begin{pgfscope}%
\pgfsys@transformshift{0.764581in}{2.545520in}%
\pgfsys@useobject{currentmarker}{}%
\end{pgfscope}%
\end{pgfscope}%
\begin{pgfscope}%
\pgfpathrectangle{\pgfqpoint{0.764581in}{0.643904in}}{\pgfqpoint{6.200000in}{4.620000in}}%
\pgfusepath{clip}%
\pgfsetbuttcap%
\pgfsetroundjoin%
\pgfsetlinewidth{0.803000pt}%
\definecolor{currentstroke}{rgb}{0.690196,0.690196,0.690196}%
\pgfsetstrokecolor{currentstroke}%
\pgfsetstrokeopacity{0.200000}%
\pgfsetdash{{2.960000pt}{1.280000pt}}{0.000000pt}%
\pgfpathmoveto{\pgfqpoint{0.764581in}{2.825035in}}%
\pgfpathlineto{\pgfqpoint{6.964581in}{2.825035in}}%
\pgfusepath{stroke}%
\end{pgfscope}%
\begin{pgfscope}%
\pgfsetbuttcap%
\pgfsetroundjoin%
\definecolor{currentfill}{rgb}{0.000000,0.000000,0.000000}%
\pgfsetfillcolor{currentfill}%
\pgfsetlinewidth{0.602250pt}%
\definecolor{currentstroke}{rgb}{0.000000,0.000000,0.000000}%
\pgfsetstrokecolor{currentstroke}%
\pgfsetdash{}{0pt}%
\pgfsys@defobject{currentmarker}{\pgfqpoint{-0.027778in}{0.000000in}}{\pgfqpoint{-0.000000in}{0.000000in}}{%
\pgfpathmoveto{\pgfqpoint{-0.000000in}{0.000000in}}%
\pgfpathlineto{\pgfqpoint{-0.027778in}{0.000000in}}%
\pgfusepath{stroke,fill}%
}%
\begin{pgfscope}%
\pgfsys@transformshift{0.764581in}{2.825035in}%
\pgfsys@useobject{currentmarker}{}%
\end{pgfscope}%
\end{pgfscope}%
\begin{pgfscope}%
\pgfpathrectangle{\pgfqpoint{0.764581in}{0.643904in}}{\pgfqpoint{6.200000in}{4.620000in}}%
\pgfusepath{clip}%
\pgfsetbuttcap%
\pgfsetroundjoin%
\pgfsetlinewidth{0.803000pt}%
\definecolor{currentstroke}{rgb}{0.690196,0.690196,0.690196}%
\pgfsetstrokecolor{currentstroke}%
\pgfsetstrokeopacity{0.200000}%
\pgfsetdash{{2.960000pt}{1.280000pt}}{0.000000pt}%
\pgfpathmoveto{\pgfqpoint{0.764581in}{3.023355in}}%
\pgfpathlineto{\pgfqpoint{6.964581in}{3.023355in}}%
\pgfusepath{stroke}%
\end{pgfscope}%
\begin{pgfscope}%
\pgfsetbuttcap%
\pgfsetroundjoin%
\definecolor{currentfill}{rgb}{0.000000,0.000000,0.000000}%
\pgfsetfillcolor{currentfill}%
\pgfsetlinewidth{0.602250pt}%
\definecolor{currentstroke}{rgb}{0.000000,0.000000,0.000000}%
\pgfsetstrokecolor{currentstroke}%
\pgfsetdash{}{0pt}%
\pgfsys@defobject{currentmarker}{\pgfqpoint{-0.027778in}{0.000000in}}{\pgfqpoint{-0.000000in}{0.000000in}}{%
\pgfpathmoveto{\pgfqpoint{-0.000000in}{0.000000in}}%
\pgfpathlineto{\pgfqpoint{-0.027778in}{0.000000in}}%
\pgfusepath{stroke,fill}%
}%
\begin{pgfscope}%
\pgfsys@transformshift{0.764581in}{3.023355in}%
\pgfsys@useobject{currentmarker}{}%
\end{pgfscope}%
\end{pgfscope}%
\begin{pgfscope}%
\pgfpathrectangle{\pgfqpoint{0.764581in}{0.643904in}}{\pgfqpoint{6.200000in}{4.620000in}}%
\pgfusepath{clip}%
\pgfsetbuttcap%
\pgfsetroundjoin%
\pgfsetlinewidth{0.803000pt}%
\definecolor{currentstroke}{rgb}{0.690196,0.690196,0.690196}%
\pgfsetstrokecolor{currentstroke}%
\pgfsetstrokeopacity{0.200000}%
\pgfsetdash{{2.960000pt}{1.280000pt}}{0.000000pt}%
\pgfpathmoveto{\pgfqpoint{0.764581in}{3.177183in}}%
\pgfpathlineto{\pgfqpoint{6.964581in}{3.177183in}}%
\pgfusepath{stroke}%
\end{pgfscope}%
\begin{pgfscope}%
\pgfsetbuttcap%
\pgfsetroundjoin%
\definecolor{currentfill}{rgb}{0.000000,0.000000,0.000000}%
\pgfsetfillcolor{currentfill}%
\pgfsetlinewidth{0.602250pt}%
\definecolor{currentstroke}{rgb}{0.000000,0.000000,0.000000}%
\pgfsetstrokecolor{currentstroke}%
\pgfsetdash{}{0pt}%
\pgfsys@defobject{currentmarker}{\pgfqpoint{-0.027778in}{0.000000in}}{\pgfqpoint{-0.000000in}{0.000000in}}{%
\pgfpathmoveto{\pgfqpoint{-0.000000in}{0.000000in}}%
\pgfpathlineto{\pgfqpoint{-0.027778in}{0.000000in}}%
\pgfusepath{stroke,fill}%
}%
\begin{pgfscope}%
\pgfsys@transformshift{0.764581in}{3.177183in}%
\pgfsys@useobject{currentmarker}{}%
\end{pgfscope}%
\end{pgfscope}%
\begin{pgfscope}%
\pgfpathrectangle{\pgfqpoint{0.764581in}{0.643904in}}{\pgfqpoint{6.200000in}{4.620000in}}%
\pgfusepath{clip}%
\pgfsetbuttcap%
\pgfsetroundjoin%
\pgfsetlinewidth{0.803000pt}%
\definecolor{currentstroke}{rgb}{0.690196,0.690196,0.690196}%
\pgfsetstrokecolor{currentstroke}%
\pgfsetstrokeopacity{0.200000}%
\pgfsetdash{{2.960000pt}{1.280000pt}}{0.000000pt}%
\pgfpathmoveto{\pgfqpoint{0.764581in}{3.302870in}}%
\pgfpathlineto{\pgfqpoint{6.964581in}{3.302870in}}%
\pgfusepath{stroke}%
\end{pgfscope}%
\begin{pgfscope}%
\pgfsetbuttcap%
\pgfsetroundjoin%
\definecolor{currentfill}{rgb}{0.000000,0.000000,0.000000}%
\pgfsetfillcolor{currentfill}%
\pgfsetlinewidth{0.602250pt}%
\definecolor{currentstroke}{rgb}{0.000000,0.000000,0.000000}%
\pgfsetstrokecolor{currentstroke}%
\pgfsetdash{}{0pt}%
\pgfsys@defobject{currentmarker}{\pgfqpoint{-0.027778in}{0.000000in}}{\pgfqpoint{-0.000000in}{0.000000in}}{%
\pgfpathmoveto{\pgfqpoint{-0.000000in}{0.000000in}}%
\pgfpathlineto{\pgfqpoint{-0.027778in}{0.000000in}}%
\pgfusepath{stroke,fill}%
}%
\begin{pgfscope}%
\pgfsys@transformshift{0.764581in}{3.302870in}%
\pgfsys@useobject{currentmarker}{}%
\end{pgfscope}%
\end{pgfscope}%
\begin{pgfscope}%
\pgfpathrectangle{\pgfqpoint{0.764581in}{0.643904in}}{\pgfqpoint{6.200000in}{4.620000in}}%
\pgfusepath{clip}%
\pgfsetbuttcap%
\pgfsetroundjoin%
\pgfsetlinewidth{0.803000pt}%
\definecolor{currentstroke}{rgb}{0.690196,0.690196,0.690196}%
\pgfsetstrokecolor{currentstroke}%
\pgfsetstrokeopacity{0.200000}%
\pgfsetdash{{2.960000pt}{1.280000pt}}{0.000000pt}%
\pgfpathmoveto{\pgfqpoint{0.764581in}{3.409137in}}%
\pgfpathlineto{\pgfqpoint{6.964581in}{3.409137in}}%
\pgfusepath{stroke}%
\end{pgfscope}%
\begin{pgfscope}%
\pgfsetbuttcap%
\pgfsetroundjoin%
\definecolor{currentfill}{rgb}{0.000000,0.000000,0.000000}%
\pgfsetfillcolor{currentfill}%
\pgfsetlinewidth{0.602250pt}%
\definecolor{currentstroke}{rgb}{0.000000,0.000000,0.000000}%
\pgfsetstrokecolor{currentstroke}%
\pgfsetdash{}{0pt}%
\pgfsys@defobject{currentmarker}{\pgfqpoint{-0.027778in}{0.000000in}}{\pgfqpoint{-0.000000in}{0.000000in}}{%
\pgfpathmoveto{\pgfqpoint{-0.000000in}{0.000000in}}%
\pgfpathlineto{\pgfqpoint{-0.027778in}{0.000000in}}%
\pgfusepath{stroke,fill}%
}%
\begin{pgfscope}%
\pgfsys@transformshift{0.764581in}{3.409137in}%
\pgfsys@useobject{currentmarker}{}%
\end{pgfscope}%
\end{pgfscope}%
\begin{pgfscope}%
\pgfpathrectangle{\pgfqpoint{0.764581in}{0.643904in}}{\pgfqpoint{6.200000in}{4.620000in}}%
\pgfusepath{clip}%
\pgfsetbuttcap%
\pgfsetroundjoin%
\pgfsetlinewidth{0.803000pt}%
\definecolor{currentstroke}{rgb}{0.690196,0.690196,0.690196}%
\pgfsetstrokecolor{currentstroke}%
\pgfsetstrokeopacity{0.200000}%
\pgfsetdash{{2.960000pt}{1.280000pt}}{0.000000pt}%
\pgfpathmoveto{\pgfqpoint{0.764581in}{3.501190in}}%
\pgfpathlineto{\pgfqpoint{6.964581in}{3.501190in}}%
\pgfusepath{stroke}%
\end{pgfscope}%
\begin{pgfscope}%
\pgfsetbuttcap%
\pgfsetroundjoin%
\definecolor{currentfill}{rgb}{0.000000,0.000000,0.000000}%
\pgfsetfillcolor{currentfill}%
\pgfsetlinewidth{0.602250pt}%
\definecolor{currentstroke}{rgb}{0.000000,0.000000,0.000000}%
\pgfsetstrokecolor{currentstroke}%
\pgfsetdash{}{0pt}%
\pgfsys@defobject{currentmarker}{\pgfqpoint{-0.027778in}{0.000000in}}{\pgfqpoint{-0.000000in}{0.000000in}}{%
\pgfpathmoveto{\pgfqpoint{-0.000000in}{0.000000in}}%
\pgfpathlineto{\pgfqpoint{-0.027778in}{0.000000in}}%
\pgfusepath{stroke,fill}%
}%
\begin{pgfscope}%
\pgfsys@transformshift{0.764581in}{3.501190in}%
\pgfsys@useobject{currentmarker}{}%
\end{pgfscope}%
\end{pgfscope}%
\begin{pgfscope}%
\pgfpathrectangle{\pgfqpoint{0.764581in}{0.643904in}}{\pgfqpoint{6.200000in}{4.620000in}}%
\pgfusepath{clip}%
\pgfsetbuttcap%
\pgfsetroundjoin%
\pgfsetlinewidth{0.803000pt}%
\definecolor{currentstroke}{rgb}{0.690196,0.690196,0.690196}%
\pgfsetstrokecolor{currentstroke}%
\pgfsetstrokeopacity{0.200000}%
\pgfsetdash{{2.960000pt}{1.280000pt}}{0.000000pt}%
\pgfpathmoveto{\pgfqpoint{0.764581in}{3.582386in}}%
\pgfpathlineto{\pgfqpoint{6.964581in}{3.582386in}}%
\pgfusepath{stroke}%
\end{pgfscope}%
\begin{pgfscope}%
\pgfsetbuttcap%
\pgfsetroundjoin%
\definecolor{currentfill}{rgb}{0.000000,0.000000,0.000000}%
\pgfsetfillcolor{currentfill}%
\pgfsetlinewidth{0.602250pt}%
\definecolor{currentstroke}{rgb}{0.000000,0.000000,0.000000}%
\pgfsetstrokecolor{currentstroke}%
\pgfsetdash{}{0pt}%
\pgfsys@defobject{currentmarker}{\pgfqpoint{-0.027778in}{0.000000in}}{\pgfqpoint{-0.000000in}{0.000000in}}{%
\pgfpathmoveto{\pgfqpoint{-0.000000in}{0.000000in}}%
\pgfpathlineto{\pgfqpoint{-0.027778in}{0.000000in}}%
\pgfusepath{stroke,fill}%
}%
\begin{pgfscope}%
\pgfsys@transformshift{0.764581in}{3.582386in}%
\pgfsys@useobject{currentmarker}{}%
\end{pgfscope}%
\end{pgfscope}%
\begin{pgfscope}%
\pgfpathrectangle{\pgfqpoint{0.764581in}{0.643904in}}{\pgfqpoint{6.200000in}{4.620000in}}%
\pgfusepath{clip}%
\pgfsetbuttcap%
\pgfsetroundjoin%
\pgfsetlinewidth{0.803000pt}%
\definecolor{currentstroke}{rgb}{0.690196,0.690196,0.690196}%
\pgfsetstrokecolor{currentstroke}%
\pgfsetstrokeopacity{0.200000}%
\pgfsetdash{{2.960000pt}{1.280000pt}}{0.000000pt}%
\pgfpathmoveto{\pgfqpoint{0.764581in}{4.132854in}}%
\pgfpathlineto{\pgfqpoint{6.964581in}{4.132854in}}%
\pgfusepath{stroke}%
\end{pgfscope}%
\begin{pgfscope}%
\pgfsetbuttcap%
\pgfsetroundjoin%
\definecolor{currentfill}{rgb}{0.000000,0.000000,0.000000}%
\pgfsetfillcolor{currentfill}%
\pgfsetlinewidth{0.602250pt}%
\definecolor{currentstroke}{rgb}{0.000000,0.000000,0.000000}%
\pgfsetstrokecolor{currentstroke}%
\pgfsetdash{}{0pt}%
\pgfsys@defobject{currentmarker}{\pgfqpoint{-0.027778in}{0.000000in}}{\pgfqpoint{-0.000000in}{0.000000in}}{%
\pgfpathmoveto{\pgfqpoint{-0.000000in}{0.000000in}}%
\pgfpathlineto{\pgfqpoint{-0.027778in}{0.000000in}}%
\pgfusepath{stroke,fill}%
}%
\begin{pgfscope}%
\pgfsys@transformshift{0.764581in}{4.132854in}%
\pgfsys@useobject{currentmarker}{}%
\end{pgfscope}%
\end{pgfscope}%
\begin{pgfscope}%
\pgfpathrectangle{\pgfqpoint{0.764581in}{0.643904in}}{\pgfqpoint{6.200000in}{4.620000in}}%
\pgfusepath{clip}%
\pgfsetbuttcap%
\pgfsetroundjoin%
\pgfsetlinewidth{0.803000pt}%
\definecolor{currentstroke}{rgb}{0.690196,0.690196,0.690196}%
\pgfsetstrokecolor{currentstroke}%
\pgfsetstrokeopacity{0.200000}%
\pgfsetdash{{2.960000pt}{1.280000pt}}{0.000000pt}%
\pgfpathmoveto{\pgfqpoint{0.764581in}{4.412369in}}%
\pgfpathlineto{\pgfqpoint{6.964581in}{4.412369in}}%
\pgfusepath{stroke}%
\end{pgfscope}%
\begin{pgfscope}%
\pgfsetbuttcap%
\pgfsetroundjoin%
\definecolor{currentfill}{rgb}{0.000000,0.000000,0.000000}%
\pgfsetfillcolor{currentfill}%
\pgfsetlinewidth{0.602250pt}%
\definecolor{currentstroke}{rgb}{0.000000,0.000000,0.000000}%
\pgfsetstrokecolor{currentstroke}%
\pgfsetdash{}{0pt}%
\pgfsys@defobject{currentmarker}{\pgfqpoint{-0.027778in}{0.000000in}}{\pgfqpoint{-0.000000in}{0.000000in}}{%
\pgfpathmoveto{\pgfqpoint{-0.000000in}{0.000000in}}%
\pgfpathlineto{\pgfqpoint{-0.027778in}{0.000000in}}%
\pgfusepath{stroke,fill}%
}%
\begin{pgfscope}%
\pgfsys@transformshift{0.764581in}{4.412369in}%
\pgfsys@useobject{currentmarker}{}%
\end{pgfscope}%
\end{pgfscope}%
\begin{pgfscope}%
\pgfpathrectangle{\pgfqpoint{0.764581in}{0.643904in}}{\pgfqpoint{6.200000in}{4.620000in}}%
\pgfusepath{clip}%
\pgfsetbuttcap%
\pgfsetroundjoin%
\pgfsetlinewidth{0.803000pt}%
\definecolor{currentstroke}{rgb}{0.690196,0.690196,0.690196}%
\pgfsetstrokecolor{currentstroke}%
\pgfsetstrokeopacity{0.200000}%
\pgfsetdash{{2.960000pt}{1.280000pt}}{0.000000pt}%
\pgfpathmoveto{\pgfqpoint{0.764581in}{4.610689in}}%
\pgfpathlineto{\pgfqpoint{6.964581in}{4.610689in}}%
\pgfusepath{stroke}%
\end{pgfscope}%
\begin{pgfscope}%
\pgfsetbuttcap%
\pgfsetroundjoin%
\definecolor{currentfill}{rgb}{0.000000,0.000000,0.000000}%
\pgfsetfillcolor{currentfill}%
\pgfsetlinewidth{0.602250pt}%
\definecolor{currentstroke}{rgb}{0.000000,0.000000,0.000000}%
\pgfsetstrokecolor{currentstroke}%
\pgfsetdash{}{0pt}%
\pgfsys@defobject{currentmarker}{\pgfqpoint{-0.027778in}{0.000000in}}{\pgfqpoint{-0.000000in}{0.000000in}}{%
\pgfpathmoveto{\pgfqpoint{-0.000000in}{0.000000in}}%
\pgfpathlineto{\pgfqpoint{-0.027778in}{0.000000in}}%
\pgfusepath{stroke,fill}%
}%
\begin{pgfscope}%
\pgfsys@transformshift{0.764581in}{4.610689in}%
\pgfsys@useobject{currentmarker}{}%
\end{pgfscope}%
\end{pgfscope}%
\begin{pgfscope}%
\pgfpathrectangle{\pgfqpoint{0.764581in}{0.643904in}}{\pgfqpoint{6.200000in}{4.620000in}}%
\pgfusepath{clip}%
\pgfsetbuttcap%
\pgfsetroundjoin%
\pgfsetlinewidth{0.803000pt}%
\definecolor{currentstroke}{rgb}{0.690196,0.690196,0.690196}%
\pgfsetstrokecolor{currentstroke}%
\pgfsetstrokeopacity{0.200000}%
\pgfsetdash{{2.960000pt}{1.280000pt}}{0.000000pt}%
\pgfpathmoveto{\pgfqpoint{0.764581in}{4.764517in}}%
\pgfpathlineto{\pgfqpoint{6.964581in}{4.764517in}}%
\pgfusepath{stroke}%
\end{pgfscope}%
\begin{pgfscope}%
\pgfsetbuttcap%
\pgfsetroundjoin%
\definecolor{currentfill}{rgb}{0.000000,0.000000,0.000000}%
\pgfsetfillcolor{currentfill}%
\pgfsetlinewidth{0.602250pt}%
\definecolor{currentstroke}{rgb}{0.000000,0.000000,0.000000}%
\pgfsetstrokecolor{currentstroke}%
\pgfsetdash{}{0pt}%
\pgfsys@defobject{currentmarker}{\pgfqpoint{-0.027778in}{0.000000in}}{\pgfqpoint{-0.000000in}{0.000000in}}{%
\pgfpathmoveto{\pgfqpoint{-0.000000in}{0.000000in}}%
\pgfpathlineto{\pgfqpoint{-0.027778in}{0.000000in}}%
\pgfusepath{stroke,fill}%
}%
\begin{pgfscope}%
\pgfsys@transformshift{0.764581in}{4.764517in}%
\pgfsys@useobject{currentmarker}{}%
\end{pgfscope}%
\end{pgfscope}%
\begin{pgfscope}%
\pgfpathrectangle{\pgfqpoint{0.764581in}{0.643904in}}{\pgfqpoint{6.200000in}{4.620000in}}%
\pgfusepath{clip}%
\pgfsetbuttcap%
\pgfsetroundjoin%
\pgfsetlinewidth{0.803000pt}%
\definecolor{currentstroke}{rgb}{0.690196,0.690196,0.690196}%
\pgfsetstrokecolor{currentstroke}%
\pgfsetstrokeopacity{0.200000}%
\pgfsetdash{{2.960000pt}{1.280000pt}}{0.000000pt}%
\pgfpathmoveto{\pgfqpoint{0.764581in}{4.890204in}}%
\pgfpathlineto{\pgfqpoint{6.964581in}{4.890204in}}%
\pgfusepath{stroke}%
\end{pgfscope}%
\begin{pgfscope}%
\pgfsetbuttcap%
\pgfsetroundjoin%
\definecolor{currentfill}{rgb}{0.000000,0.000000,0.000000}%
\pgfsetfillcolor{currentfill}%
\pgfsetlinewidth{0.602250pt}%
\definecolor{currentstroke}{rgb}{0.000000,0.000000,0.000000}%
\pgfsetstrokecolor{currentstroke}%
\pgfsetdash{}{0pt}%
\pgfsys@defobject{currentmarker}{\pgfqpoint{-0.027778in}{0.000000in}}{\pgfqpoint{-0.000000in}{0.000000in}}{%
\pgfpathmoveto{\pgfqpoint{-0.000000in}{0.000000in}}%
\pgfpathlineto{\pgfqpoint{-0.027778in}{0.000000in}}%
\pgfusepath{stroke,fill}%
}%
\begin{pgfscope}%
\pgfsys@transformshift{0.764581in}{4.890204in}%
\pgfsys@useobject{currentmarker}{}%
\end{pgfscope}%
\end{pgfscope}%
\begin{pgfscope}%
\pgfpathrectangle{\pgfqpoint{0.764581in}{0.643904in}}{\pgfqpoint{6.200000in}{4.620000in}}%
\pgfusepath{clip}%
\pgfsetbuttcap%
\pgfsetroundjoin%
\pgfsetlinewidth{0.803000pt}%
\definecolor{currentstroke}{rgb}{0.690196,0.690196,0.690196}%
\pgfsetstrokecolor{currentstroke}%
\pgfsetstrokeopacity{0.200000}%
\pgfsetdash{{2.960000pt}{1.280000pt}}{0.000000pt}%
\pgfpathmoveto{\pgfqpoint{0.764581in}{4.996471in}}%
\pgfpathlineto{\pgfqpoint{6.964581in}{4.996471in}}%
\pgfusepath{stroke}%
\end{pgfscope}%
\begin{pgfscope}%
\pgfsetbuttcap%
\pgfsetroundjoin%
\definecolor{currentfill}{rgb}{0.000000,0.000000,0.000000}%
\pgfsetfillcolor{currentfill}%
\pgfsetlinewidth{0.602250pt}%
\definecolor{currentstroke}{rgb}{0.000000,0.000000,0.000000}%
\pgfsetstrokecolor{currentstroke}%
\pgfsetdash{}{0pt}%
\pgfsys@defobject{currentmarker}{\pgfqpoint{-0.027778in}{0.000000in}}{\pgfqpoint{-0.000000in}{0.000000in}}{%
\pgfpathmoveto{\pgfqpoint{-0.000000in}{0.000000in}}%
\pgfpathlineto{\pgfqpoint{-0.027778in}{0.000000in}}%
\pgfusepath{stroke,fill}%
}%
\begin{pgfscope}%
\pgfsys@transformshift{0.764581in}{4.996471in}%
\pgfsys@useobject{currentmarker}{}%
\end{pgfscope}%
\end{pgfscope}%
\begin{pgfscope}%
\pgfpathrectangle{\pgfqpoint{0.764581in}{0.643904in}}{\pgfqpoint{6.200000in}{4.620000in}}%
\pgfusepath{clip}%
\pgfsetbuttcap%
\pgfsetroundjoin%
\pgfsetlinewidth{0.803000pt}%
\definecolor{currentstroke}{rgb}{0.690196,0.690196,0.690196}%
\pgfsetstrokecolor{currentstroke}%
\pgfsetstrokeopacity{0.200000}%
\pgfsetdash{{2.960000pt}{1.280000pt}}{0.000000pt}%
\pgfpathmoveto{\pgfqpoint{0.764581in}{5.088524in}}%
\pgfpathlineto{\pgfqpoint{6.964581in}{5.088524in}}%
\pgfusepath{stroke}%
\end{pgfscope}%
\begin{pgfscope}%
\pgfsetbuttcap%
\pgfsetroundjoin%
\definecolor{currentfill}{rgb}{0.000000,0.000000,0.000000}%
\pgfsetfillcolor{currentfill}%
\pgfsetlinewidth{0.602250pt}%
\definecolor{currentstroke}{rgb}{0.000000,0.000000,0.000000}%
\pgfsetstrokecolor{currentstroke}%
\pgfsetdash{}{0pt}%
\pgfsys@defobject{currentmarker}{\pgfqpoint{-0.027778in}{0.000000in}}{\pgfqpoint{-0.000000in}{0.000000in}}{%
\pgfpathmoveto{\pgfqpoint{-0.000000in}{0.000000in}}%
\pgfpathlineto{\pgfqpoint{-0.027778in}{0.000000in}}%
\pgfusepath{stroke,fill}%
}%
\begin{pgfscope}%
\pgfsys@transformshift{0.764581in}{5.088524in}%
\pgfsys@useobject{currentmarker}{}%
\end{pgfscope}%
\end{pgfscope}%
\begin{pgfscope}%
\pgfpathrectangle{\pgfqpoint{0.764581in}{0.643904in}}{\pgfqpoint{6.200000in}{4.620000in}}%
\pgfusepath{clip}%
\pgfsetbuttcap%
\pgfsetroundjoin%
\pgfsetlinewidth{0.803000pt}%
\definecolor{currentstroke}{rgb}{0.690196,0.690196,0.690196}%
\pgfsetstrokecolor{currentstroke}%
\pgfsetstrokeopacity{0.200000}%
\pgfsetdash{{2.960000pt}{1.280000pt}}{0.000000pt}%
\pgfpathmoveto{\pgfqpoint{0.764581in}{5.169720in}}%
\pgfpathlineto{\pgfqpoint{6.964581in}{5.169720in}}%
\pgfusepath{stroke}%
\end{pgfscope}%
\begin{pgfscope}%
\pgfsetbuttcap%
\pgfsetroundjoin%
\definecolor{currentfill}{rgb}{0.000000,0.000000,0.000000}%
\pgfsetfillcolor{currentfill}%
\pgfsetlinewidth{0.602250pt}%
\definecolor{currentstroke}{rgb}{0.000000,0.000000,0.000000}%
\pgfsetstrokecolor{currentstroke}%
\pgfsetdash{}{0pt}%
\pgfsys@defobject{currentmarker}{\pgfqpoint{-0.027778in}{0.000000in}}{\pgfqpoint{-0.000000in}{0.000000in}}{%
\pgfpathmoveto{\pgfqpoint{-0.000000in}{0.000000in}}%
\pgfpathlineto{\pgfqpoint{-0.027778in}{0.000000in}}%
\pgfusepath{stroke,fill}%
}%
\begin{pgfscope}%
\pgfsys@transformshift{0.764581in}{5.169720in}%
\pgfsys@useobject{currentmarker}{}%
\end{pgfscope}%
\end{pgfscope}%
\begin{pgfscope}%
\definecolor{textcolor}{rgb}{0.000000,0.000000,0.000000}%
\pgfsetstrokecolor{textcolor}%
\pgfsetfillcolor{textcolor}%
\pgftext[x=0.339583in,y=2.953904in,,bottom,rotate=90.000000]{\color{textcolor}{\rmfamily\fontsize{18.000000}{21.600000}\selectfont\catcode`\^=\active\def^{\ifmmode\sp\else\^{}\fi}\catcode`\%=\active\def%{\%}Time [seconds]}}%
\end{pgfscope}%
\begin{pgfscope}%
\pgfsetrectcap%
\pgfsetmiterjoin%
\pgfsetlinewidth{0.803000pt}%
\definecolor{currentstroke}{rgb}{0.000000,0.000000,0.000000}%
\pgfsetstrokecolor{currentstroke}%
\pgfsetdash{}{0pt}%
\pgfpathmoveto{\pgfqpoint{0.764581in}{0.643904in}}%
\pgfpathlineto{\pgfqpoint{0.764581in}{5.263904in}}%
\pgfusepath{stroke}%
\end{pgfscope}%
\begin{pgfscope}%
\pgfsetrectcap%
\pgfsetmiterjoin%
\pgfsetlinewidth{0.803000pt}%
\definecolor{currentstroke}{rgb}{0.000000,0.000000,0.000000}%
\pgfsetstrokecolor{currentstroke}%
\pgfsetdash{}{0pt}%
\pgfpathmoveto{\pgfqpoint{6.964581in}{0.643904in}}%
\pgfpathlineto{\pgfqpoint{6.964581in}{5.263904in}}%
\pgfusepath{stroke}%
\end{pgfscope}%
\begin{pgfscope}%
\pgfsetrectcap%
\pgfsetmiterjoin%
\pgfsetlinewidth{0.803000pt}%
\definecolor{currentstroke}{rgb}{0.000000,0.000000,0.000000}%
\pgfsetstrokecolor{currentstroke}%
\pgfsetdash{}{0pt}%
\pgfpathmoveto{\pgfqpoint{0.764581in}{0.643904in}}%
\pgfpathlineto{\pgfqpoint{6.964581in}{0.643904in}}%
\pgfusepath{stroke}%
\end{pgfscope}%
\begin{pgfscope}%
\pgfsetrectcap%
\pgfsetmiterjoin%
\pgfsetlinewidth{0.803000pt}%
\definecolor{currentstroke}{rgb}{0.000000,0.000000,0.000000}%
\pgfsetstrokecolor{currentstroke}%
\pgfsetdash{}{0pt}%
\pgfpathmoveto{\pgfqpoint{0.764581in}{5.263904in}}%
\pgfpathlineto{\pgfqpoint{6.964581in}{5.263904in}}%
\pgfusepath{stroke}%
\end{pgfscope}%
\begin{pgfscope}%
\pgfpathrectangle{\pgfqpoint{0.764581in}{0.643904in}}{\pgfqpoint{6.200000in}{4.620000in}}%
\pgfusepath{clip}%
\pgfsetrectcap%
\pgfsetroundjoin%
\pgfsetlinewidth{1.505625pt}%
\definecolor{currentstroke}{rgb}{0.000000,0.000000,1.000000}%
\pgfsetstrokecolor{currentstroke}%
\pgfsetdash{}{0pt}%
\pgfpathmoveto{\pgfqpoint{0.764581in}{0.853904in}}%
\pgfpathlineto{\pgfqpoint{2.455884in}{1.625606in}}%
\pgfpathlineto{\pgfqpoint{3.184288in}{2.056435in}}%
\pgfpathlineto{\pgfqpoint{4.147187in}{2.648049in}}%
\pgfpathlineto{\pgfqpoint{4.875592in}{3.110216in}}%
\pgfpathlineto{\pgfqpoint{5.603996in}{3.550323in}}%
\pgfpathlineto{\pgfqpoint{6.332401in}{4.052710in}}%
\pgfpathlineto{\pgfqpoint{6.758490in}{4.325160in}}%
\pgfpathlineto{\pgfqpoint{6.964581in}{4.448402in}}%
\pgfusepath{stroke}%
\end{pgfscope}%
\begin{pgfscope}%
\pgfpathrectangle{\pgfqpoint{0.764581in}{0.643904in}}{\pgfqpoint{6.200000in}{4.620000in}}%
\pgfusepath{clip}%
\pgfsetbuttcap%
\pgfsetroundjoin%
\pgfsetlinewidth{1.505625pt}%
\definecolor{currentstroke}{rgb}{1.000000,0.000000,0.000000}%
\pgfsetstrokecolor{currentstroke}%
\pgfsetdash{{5.550000pt}{2.400000pt}}{0.000000pt}%
\pgfpathmoveto{\pgfqpoint{0.764581in}{3.393336in}}%
\pgfpathlineto{\pgfqpoint{2.455884in}{3.469107in}}%
\pgfpathlineto{\pgfqpoint{3.184288in}{3.545769in}}%
\pgfpathlineto{\pgfqpoint{4.147187in}{3.728731in}}%
\pgfpathlineto{\pgfqpoint{4.875592in}{3.946418in}}%
\pgfpathlineto{\pgfqpoint{5.603996in}{4.254357in}}%
\pgfpathlineto{\pgfqpoint{6.332401in}{4.649383in}}%
\pgfpathlineto{\pgfqpoint{6.758490in}{4.916608in}}%
\pgfpathlineto{\pgfqpoint{6.964581in}{5.053904in}}%
\pgfusepath{stroke}%
\end{pgfscope}%
\begin{pgfscope}%
\pgfsetbuttcap%
\pgfsetmiterjoin%
\definecolor{currentfill}{rgb}{1.000000,1.000000,1.000000}%
\pgfsetfillcolor{currentfill}%
\pgfsetfillopacity{0.800000}%
\pgfsetlinewidth{1.003750pt}%
\definecolor{currentstroke}{rgb}{0.800000,0.800000,0.800000}%
\pgfsetstrokecolor{currentstroke}%
\pgfsetstrokeopacity{0.800000}%
\pgfsetdash{}{0pt}%
\pgfpathmoveto{\pgfqpoint{0.920136in}{4.437207in}}%
\pgfpathlineto{\pgfqpoint{3.236863in}{4.437207in}}%
\pgfpathquadraticcurveto{\pgfqpoint{3.281307in}{4.437207in}}{\pgfqpoint{3.281307in}{4.481651in}}%
\pgfpathlineto{\pgfqpoint{3.281307in}{5.108348in}}%
\pgfpathquadraticcurveto{\pgfqpoint{3.281307in}{5.152793in}}{\pgfqpoint{3.236863in}{5.152793in}}%
\pgfpathlineto{\pgfqpoint{0.920136in}{5.152793in}}%
\pgfpathquadraticcurveto{\pgfqpoint{0.875692in}{5.152793in}}{\pgfqpoint{0.875692in}{5.108348in}}%
\pgfpathlineto{\pgfqpoint{0.875692in}{4.481651in}}%
\pgfpathquadraticcurveto{\pgfqpoint{0.875692in}{4.437207in}}{\pgfqpoint{0.920136in}{4.437207in}}%
\pgfpathlineto{\pgfqpoint{0.920136in}{4.437207in}}%
\pgfpathclose%
\pgfusepath{stroke,fill}%
\end{pgfscope}%
\begin{pgfscope}%
\pgfsetrectcap%
\pgfsetroundjoin%
\pgfsetlinewidth{1.505625pt}%
\definecolor{currentstroke}{rgb}{0.000000,0.000000,1.000000}%
\pgfsetstrokecolor{currentstroke}%
\pgfsetdash{}{0pt}%
\pgfpathmoveto{\pgfqpoint{0.964581in}{4.975015in}}%
\pgfpathlineto{\pgfqpoint{1.186803in}{4.975015in}}%
\pgfpathlineto{\pgfqpoint{1.409025in}{4.975015in}}%
\pgfusepath{stroke}%
\end{pgfscope}%
\begin{pgfscope}%
\definecolor{textcolor}{rgb}{0.000000,0.000000,0.000000}%
\pgfsetstrokecolor{textcolor}%
\pgfsetfillcolor{textcolor}%
\pgftext[x=1.586803in,y=4.897237in,left,base]{\color{textcolor}{\rmfamily\fontsize{16.000000}{19.200000}\selectfont\catcode`\^=\active\def^{\ifmmode\sp\else\^{}\fi}\catcode`\%=\active\def%{\%}logical dispatch}}%
\end{pgfscope}%
\begin{pgfscope}%
\pgfsetbuttcap%
\pgfsetroundjoin%
\pgfsetlinewidth{1.505625pt}%
\definecolor{currentstroke}{rgb}{1.000000,0.000000,0.000000}%
\pgfsetstrokecolor{currentstroke}%
\pgfsetdash{{5.550000pt}{2.400000pt}}{0.000000pt}%
\pgfpathmoveto{\pgfqpoint{0.964581in}{4.650555in}}%
\pgfpathlineto{\pgfqpoint{1.186803in}{4.650555in}}%
\pgfpathlineto{\pgfqpoint{1.409025in}{4.650555in}}%
\pgfusepath{stroke}%
\end{pgfscope}%
\begin{pgfscope}%
\definecolor{textcolor}{rgb}{0.000000,0.000000,0.000000}%
\pgfsetstrokecolor{textcolor}%
\pgfsetfillcolor{textcolor}%
\pgftext[x=1.586803in,y=4.572777in,left,base]{\color{textcolor}{\rmfamily\fontsize{16.000000}{19.200000}\selectfont\catcode`\^=\active\def^{\ifmmode\sp\else\^{}\fi}\catcode`\%=\active\def%{\%}optimal dispatch}}%
\end{pgfscope}%
\end{pgfpicture}%
\makeatother%
\endgroup%
}
    \caption{Time scaling of a capacity expansion problem using either an optimal or logical dispatch algorithm.}
    \label{fig:alg-scaling}
\end{figure}

\noindent Initially, the logical dispatch algorithm outperforms the optimal
dispatch algorithm by nearly two orders of magnitude. This is because the linear
program has some overhead when writing and copying equations that the rule-based
calculation does not. The logical algorithm initially grows more quickly until
the models reach 100 modeled days after which the two algorithms scale
similarly and the logical dispatch algorithm remains approximately 2.5 times
faster than its optimal counterpart.

\subsection{Exercise 3: Parallelization}

\Acp{ga} are considered ``embarrassingly parallelizable'' since the performance
of each individual in a population is independent from the others. However, there
a some difficulties with solving multiple parallel instances of \ac{lp} solvers
since these solvers frequently have some parallel optimizations built-in. For
now, this restricts capacity expansion problems within \ac{osier} that use linear 
programming to serial calculations. This is not so for the logical dispatch algorithm 
since it does
not use an \ac{lp} solver. Therefore, this exercise looks exclusively at how the
logical dispatch algorithm scales with number of threads available. Once again,
the dispatch algorithm is driven by \ac{osier}'s \texttt{CapacityExpansion}
class whose parameters are described in Table \ref{tab:thread-scaling-params}. 

\begin{table}[htbp!]
    \centering
    \caption{Capacity expansion parameters for the parallelization exercise.}
    \label{tab:thread-scaling-params}
    \begin{tabular}{ll}
        \toprule
        Parameter & Value \\
        \midrule
        Algorithm & \acs{nsga2}\\
        Termination Criterion & Maximum Generations\\
        Generations & 10 \\
        Objectives & 2 (cost, emissions)\\
        Timesteps & 120 (5 days x 24 hours)\\
        \bottomrule
    \end{tabular}
\end{table}

\noindent In this exercise, the problem is scaled by the population size of each
generation. The study was performed on a 2024 MacBook Pro with an M4 Pro CPU, 
48 GB of RAM, and the macOS Sequoia 15.5 operating system. Figure
\ref{fig:thread-scaling} shows the results for this exercise.

\begin{figure}[htbp!]
    \centering
    \resizebox{0.75\columnwidth}{!}{%% Creator: Matplotlib, PGF backend
%%
%% To include the figure in your LaTeX document, write
%%   \input{<filename>.pgf}
%%
%% Make sure the required packages are loaded in your preamble
%%   \usepackage{pgf}
%%
%% Also ensure that all the required font packages are loaded; for instance,
%% the lmodern package is sometimes necessary when using math font.
%%   \usepackage{lmodern}
%%
%% Figures using additional raster images can only be included by \input if
%% they are in the same directory as the main LaTeX file. For loading figures
%% from other directories you can use the `import` package
%%   \usepackage{import}
%%
%% and then include the figures with
%%   \import{<path to file>}{<filename>.pgf}
%%
%% Matplotlib used the following preamble
%%   \def\mathdefault#1{#1}
%%   \everymath=\expandafter{\the\everymath\displaystyle}
%%   \IfFileExists{scrextend.sty}{
%%     \usepackage[fontsize=10.000000pt]{scrextend}
%%   }{
%%     \renewcommand{\normalsize}{\fontsize{10.000000}{12.000000}\selectfont}
%%     \normalsize
%%   }
%%   
%%   \makeatletter\@ifpackageloaded{underscore}{}{\usepackage[strings]{underscore}}\makeatother
%%
\begingroup%
\makeatletter%
\begin{pgfpicture}%
\pgfpathrectangle{\pgfpointorigin}{\pgfqpoint{7.135065in}{5.363904in}}%
\pgfusepath{use as bounding box, clip}%
\begin{pgfscope}%
\pgfsetbuttcap%
\pgfsetmiterjoin%
\definecolor{currentfill}{rgb}{1.000000,1.000000,1.000000}%
\pgfsetfillcolor{currentfill}%
\pgfsetlinewidth{0.000000pt}%
\definecolor{currentstroke}{rgb}{0.000000,0.000000,0.000000}%
\pgfsetstrokecolor{currentstroke}%
\pgfsetdash{}{0pt}%
\pgfpathmoveto{\pgfqpoint{0.000000in}{0.000000in}}%
\pgfpathlineto{\pgfqpoint{7.135065in}{0.000000in}}%
\pgfpathlineto{\pgfqpoint{7.135065in}{5.363904in}}%
\pgfpathlineto{\pgfqpoint{0.000000in}{5.363904in}}%
\pgfpathlineto{\pgfqpoint{0.000000in}{0.000000in}}%
\pgfpathclose%
\pgfusepath{fill}%
\end{pgfscope}%
\begin{pgfscope}%
\pgfsetbuttcap%
\pgfsetmiterjoin%
\definecolor{currentfill}{rgb}{1.000000,1.000000,1.000000}%
\pgfsetfillcolor{currentfill}%
\pgfsetlinewidth{0.000000pt}%
\definecolor{currentstroke}{rgb}{0.000000,0.000000,0.000000}%
\pgfsetstrokecolor{currentstroke}%
\pgfsetstrokeopacity{0.000000}%
\pgfsetdash{}{0pt}%
\pgfpathmoveto{\pgfqpoint{0.688192in}{0.643904in}}%
\pgfpathlineto{\pgfqpoint{6.888192in}{0.643904in}}%
\pgfpathlineto{\pgfqpoint{6.888192in}{5.263904in}}%
\pgfpathlineto{\pgfqpoint{0.688192in}{5.263904in}}%
\pgfpathlineto{\pgfqpoint{0.688192in}{0.643904in}}%
\pgfpathclose%
\pgfusepath{fill}%
\end{pgfscope}%
\begin{pgfscope}%
\pgfpathrectangle{\pgfqpoint{0.688192in}{0.643904in}}{\pgfqpoint{6.200000in}{4.620000in}}%
\pgfusepath{clip}%
\pgfsetrectcap%
\pgfsetroundjoin%
\pgfsetlinewidth{0.803000pt}%
\definecolor{currentstroke}{rgb}{0.690196,0.690196,0.690196}%
\pgfsetstrokecolor{currentstroke}%
\pgfsetdash{}{0pt}%
\pgfpathmoveto{\pgfqpoint{1.219620in}{0.643904in}}%
\pgfpathlineto{\pgfqpoint{1.219620in}{5.263904in}}%
\pgfusepath{stroke}%
\end{pgfscope}%
\begin{pgfscope}%
\pgfsetbuttcap%
\pgfsetroundjoin%
\definecolor{currentfill}{rgb}{0.000000,0.000000,0.000000}%
\pgfsetfillcolor{currentfill}%
\pgfsetlinewidth{0.803000pt}%
\definecolor{currentstroke}{rgb}{0.000000,0.000000,0.000000}%
\pgfsetstrokecolor{currentstroke}%
\pgfsetdash{}{0pt}%
\pgfsys@defobject{currentmarker}{\pgfqpoint{0.000000in}{-0.048611in}}{\pgfqpoint{0.000000in}{0.000000in}}{%
\pgfpathmoveto{\pgfqpoint{0.000000in}{0.000000in}}%
\pgfpathlineto{\pgfqpoint{0.000000in}{-0.048611in}}%
\pgfusepath{stroke,fill}%
}%
\begin{pgfscope}%
\pgfsys@transformshift{1.219620in}{0.643904in}%
\pgfsys@useobject{currentmarker}{}%
\end{pgfscope}%
\end{pgfscope}%
\begin{pgfscope}%
\definecolor{textcolor}{rgb}{0.000000,0.000000,0.000000}%
\pgfsetstrokecolor{textcolor}%
\pgfsetfillcolor{textcolor}%
\pgftext[x=1.219620in,y=0.546682in,,top]{\color{textcolor}{\rmfamily\fontsize{14.000000}{16.800000}\selectfont\catcode`\^=\active\def^{\ifmmode\sp\else\^{}\fi}\catcode`\%=\active\def%{\%}$\mathdefault{40}$}}%
\end{pgfscope}%
\begin{pgfscope}%
\pgfpathrectangle{\pgfqpoint{0.688192in}{0.643904in}}{\pgfqpoint{6.200000in}{4.620000in}}%
\pgfusepath{clip}%
\pgfsetrectcap%
\pgfsetroundjoin%
\pgfsetlinewidth{0.803000pt}%
\definecolor{currentstroke}{rgb}{0.690196,0.690196,0.690196}%
\pgfsetstrokecolor{currentstroke}%
\pgfsetdash{}{0pt}%
\pgfpathmoveto{\pgfqpoint{1.928192in}{0.643904in}}%
\pgfpathlineto{\pgfqpoint{1.928192in}{5.263904in}}%
\pgfusepath{stroke}%
\end{pgfscope}%
\begin{pgfscope}%
\pgfsetbuttcap%
\pgfsetroundjoin%
\definecolor{currentfill}{rgb}{0.000000,0.000000,0.000000}%
\pgfsetfillcolor{currentfill}%
\pgfsetlinewidth{0.803000pt}%
\definecolor{currentstroke}{rgb}{0.000000,0.000000,0.000000}%
\pgfsetstrokecolor{currentstroke}%
\pgfsetdash{}{0pt}%
\pgfsys@defobject{currentmarker}{\pgfqpoint{0.000000in}{-0.048611in}}{\pgfqpoint{0.000000in}{0.000000in}}{%
\pgfpathmoveto{\pgfqpoint{0.000000in}{0.000000in}}%
\pgfpathlineto{\pgfqpoint{0.000000in}{-0.048611in}}%
\pgfusepath{stroke,fill}%
}%
\begin{pgfscope}%
\pgfsys@transformshift{1.928192in}{0.643904in}%
\pgfsys@useobject{currentmarker}{}%
\end{pgfscope}%
\end{pgfscope}%
\begin{pgfscope}%
\definecolor{textcolor}{rgb}{0.000000,0.000000,0.000000}%
\pgfsetstrokecolor{textcolor}%
\pgfsetfillcolor{textcolor}%
\pgftext[x=1.928192in,y=0.546682in,,top]{\color{textcolor}{\rmfamily\fontsize{14.000000}{16.800000}\selectfont\catcode`\^=\active\def^{\ifmmode\sp\else\^{}\fi}\catcode`\%=\active\def%{\%}$\mathdefault{60}$}}%
\end{pgfscope}%
\begin{pgfscope}%
\pgfpathrectangle{\pgfqpoint{0.688192in}{0.643904in}}{\pgfqpoint{6.200000in}{4.620000in}}%
\pgfusepath{clip}%
\pgfsetrectcap%
\pgfsetroundjoin%
\pgfsetlinewidth{0.803000pt}%
\definecolor{currentstroke}{rgb}{0.690196,0.690196,0.690196}%
\pgfsetstrokecolor{currentstroke}%
\pgfsetdash{}{0pt}%
\pgfpathmoveto{\pgfqpoint{2.636763in}{0.643904in}}%
\pgfpathlineto{\pgfqpoint{2.636763in}{5.263904in}}%
\pgfusepath{stroke}%
\end{pgfscope}%
\begin{pgfscope}%
\pgfsetbuttcap%
\pgfsetroundjoin%
\definecolor{currentfill}{rgb}{0.000000,0.000000,0.000000}%
\pgfsetfillcolor{currentfill}%
\pgfsetlinewidth{0.803000pt}%
\definecolor{currentstroke}{rgb}{0.000000,0.000000,0.000000}%
\pgfsetstrokecolor{currentstroke}%
\pgfsetdash{}{0pt}%
\pgfsys@defobject{currentmarker}{\pgfqpoint{0.000000in}{-0.048611in}}{\pgfqpoint{0.000000in}{0.000000in}}{%
\pgfpathmoveto{\pgfqpoint{0.000000in}{0.000000in}}%
\pgfpathlineto{\pgfqpoint{0.000000in}{-0.048611in}}%
\pgfusepath{stroke,fill}%
}%
\begin{pgfscope}%
\pgfsys@transformshift{2.636763in}{0.643904in}%
\pgfsys@useobject{currentmarker}{}%
\end{pgfscope}%
\end{pgfscope}%
\begin{pgfscope}%
\definecolor{textcolor}{rgb}{0.000000,0.000000,0.000000}%
\pgfsetstrokecolor{textcolor}%
\pgfsetfillcolor{textcolor}%
\pgftext[x=2.636763in,y=0.546682in,,top]{\color{textcolor}{\rmfamily\fontsize{14.000000}{16.800000}\selectfont\catcode`\^=\active\def^{\ifmmode\sp\else\^{}\fi}\catcode`\%=\active\def%{\%}$\mathdefault{80}$}}%
\end{pgfscope}%
\begin{pgfscope}%
\pgfpathrectangle{\pgfqpoint{0.688192in}{0.643904in}}{\pgfqpoint{6.200000in}{4.620000in}}%
\pgfusepath{clip}%
\pgfsetrectcap%
\pgfsetroundjoin%
\pgfsetlinewidth{0.803000pt}%
\definecolor{currentstroke}{rgb}{0.690196,0.690196,0.690196}%
\pgfsetstrokecolor{currentstroke}%
\pgfsetdash{}{0pt}%
\pgfpathmoveto{\pgfqpoint{3.345334in}{0.643904in}}%
\pgfpathlineto{\pgfqpoint{3.345334in}{5.263904in}}%
\pgfusepath{stroke}%
\end{pgfscope}%
\begin{pgfscope}%
\pgfsetbuttcap%
\pgfsetroundjoin%
\definecolor{currentfill}{rgb}{0.000000,0.000000,0.000000}%
\pgfsetfillcolor{currentfill}%
\pgfsetlinewidth{0.803000pt}%
\definecolor{currentstroke}{rgb}{0.000000,0.000000,0.000000}%
\pgfsetstrokecolor{currentstroke}%
\pgfsetdash{}{0pt}%
\pgfsys@defobject{currentmarker}{\pgfqpoint{0.000000in}{-0.048611in}}{\pgfqpoint{0.000000in}{0.000000in}}{%
\pgfpathmoveto{\pgfqpoint{0.000000in}{0.000000in}}%
\pgfpathlineto{\pgfqpoint{0.000000in}{-0.048611in}}%
\pgfusepath{stroke,fill}%
}%
\begin{pgfscope}%
\pgfsys@transformshift{3.345334in}{0.643904in}%
\pgfsys@useobject{currentmarker}{}%
\end{pgfscope}%
\end{pgfscope}%
\begin{pgfscope}%
\definecolor{textcolor}{rgb}{0.000000,0.000000,0.000000}%
\pgfsetstrokecolor{textcolor}%
\pgfsetfillcolor{textcolor}%
\pgftext[x=3.345334in,y=0.546682in,,top]{\color{textcolor}{\rmfamily\fontsize{14.000000}{16.800000}\selectfont\catcode`\^=\active\def^{\ifmmode\sp\else\^{}\fi}\catcode`\%=\active\def%{\%}$\mathdefault{100}$}}%
\end{pgfscope}%
\begin{pgfscope}%
\pgfpathrectangle{\pgfqpoint{0.688192in}{0.643904in}}{\pgfqpoint{6.200000in}{4.620000in}}%
\pgfusepath{clip}%
\pgfsetrectcap%
\pgfsetroundjoin%
\pgfsetlinewidth{0.803000pt}%
\definecolor{currentstroke}{rgb}{0.690196,0.690196,0.690196}%
\pgfsetstrokecolor{currentstroke}%
\pgfsetdash{}{0pt}%
\pgfpathmoveto{\pgfqpoint{4.053906in}{0.643904in}}%
\pgfpathlineto{\pgfqpoint{4.053906in}{5.263904in}}%
\pgfusepath{stroke}%
\end{pgfscope}%
\begin{pgfscope}%
\pgfsetbuttcap%
\pgfsetroundjoin%
\definecolor{currentfill}{rgb}{0.000000,0.000000,0.000000}%
\pgfsetfillcolor{currentfill}%
\pgfsetlinewidth{0.803000pt}%
\definecolor{currentstroke}{rgb}{0.000000,0.000000,0.000000}%
\pgfsetstrokecolor{currentstroke}%
\pgfsetdash{}{0pt}%
\pgfsys@defobject{currentmarker}{\pgfqpoint{0.000000in}{-0.048611in}}{\pgfqpoint{0.000000in}{0.000000in}}{%
\pgfpathmoveto{\pgfqpoint{0.000000in}{0.000000in}}%
\pgfpathlineto{\pgfqpoint{0.000000in}{-0.048611in}}%
\pgfusepath{stroke,fill}%
}%
\begin{pgfscope}%
\pgfsys@transformshift{4.053906in}{0.643904in}%
\pgfsys@useobject{currentmarker}{}%
\end{pgfscope}%
\end{pgfscope}%
\begin{pgfscope}%
\definecolor{textcolor}{rgb}{0.000000,0.000000,0.000000}%
\pgfsetstrokecolor{textcolor}%
\pgfsetfillcolor{textcolor}%
\pgftext[x=4.053906in,y=0.546682in,,top]{\color{textcolor}{\rmfamily\fontsize{14.000000}{16.800000}\selectfont\catcode`\^=\active\def^{\ifmmode\sp\else\^{}\fi}\catcode`\%=\active\def%{\%}$\mathdefault{120}$}}%
\end{pgfscope}%
\begin{pgfscope}%
\pgfpathrectangle{\pgfqpoint{0.688192in}{0.643904in}}{\pgfqpoint{6.200000in}{4.620000in}}%
\pgfusepath{clip}%
\pgfsetrectcap%
\pgfsetroundjoin%
\pgfsetlinewidth{0.803000pt}%
\definecolor{currentstroke}{rgb}{0.690196,0.690196,0.690196}%
\pgfsetstrokecolor{currentstroke}%
\pgfsetdash{}{0pt}%
\pgfpathmoveto{\pgfqpoint{4.762477in}{0.643904in}}%
\pgfpathlineto{\pgfqpoint{4.762477in}{5.263904in}}%
\pgfusepath{stroke}%
\end{pgfscope}%
\begin{pgfscope}%
\pgfsetbuttcap%
\pgfsetroundjoin%
\definecolor{currentfill}{rgb}{0.000000,0.000000,0.000000}%
\pgfsetfillcolor{currentfill}%
\pgfsetlinewidth{0.803000pt}%
\definecolor{currentstroke}{rgb}{0.000000,0.000000,0.000000}%
\pgfsetstrokecolor{currentstroke}%
\pgfsetdash{}{0pt}%
\pgfsys@defobject{currentmarker}{\pgfqpoint{0.000000in}{-0.048611in}}{\pgfqpoint{0.000000in}{0.000000in}}{%
\pgfpathmoveto{\pgfqpoint{0.000000in}{0.000000in}}%
\pgfpathlineto{\pgfqpoint{0.000000in}{-0.048611in}}%
\pgfusepath{stroke,fill}%
}%
\begin{pgfscope}%
\pgfsys@transformshift{4.762477in}{0.643904in}%
\pgfsys@useobject{currentmarker}{}%
\end{pgfscope}%
\end{pgfscope}%
\begin{pgfscope}%
\definecolor{textcolor}{rgb}{0.000000,0.000000,0.000000}%
\pgfsetstrokecolor{textcolor}%
\pgfsetfillcolor{textcolor}%
\pgftext[x=4.762477in,y=0.546682in,,top]{\color{textcolor}{\rmfamily\fontsize{14.000000}{16.800000}\selectfont\catcode`\^=\active\def^{\ifmmode\sp\else\^{}\fi}\catcode`\%=\active\def%{\%}$\mathdefault{140}$}}%
\end{pgfscope}%
\begin{pgfscope}%
\pgfpathrectangle{\pgfqpoint{0.688192in}{0.643904in}}{\pgfqpoint{6.200000in}{4.620000in}}%
\pgfusepath{clip}%
\pgfsetrectcap%
\pgfsetroundjoin%
\pgfsetlinewidth{0.803000pt}%
\definecolor{currentstroke}{rgb}{0.690196,0.690196,0.690196}%
\pgfsetstrokecolor{currentstroke}%
\pgfsetdash{}{0pt}%
\pgfpathmoveto{\pgfqpoint{5.471049in}{0.643904in}}%
\pgfpathlineto{\pgfqpoint{5.471049in}{5.263904in}}%
\pgfusepath{stroke}%
\end{pgfscope}%
\begin{pgfscope}%
\pgfsetbuttcap%
\pgfsetroundjoin%
\definecolor{currentfill}{rgb}{0.000000,0.000000,0.000000}%
\pgfsetfillcolor{currentfill}%
\pgfsetlinewidth{0.803000pt}%
\definecolor{currentstroke}{rgb}{0.000000,0.000000,0.000000}%
\pgfsetstrokecolor{currentstroke}%
\pgfsetdash{}{0pt}%
\pgfsys@defobject{currentmarker}{\pgfqpoint{0.000000in}{-0.048611in}}{\pgfqpoint{0.000000in}{0.000000in}}{%
\pgfpathmoveto{\pgfqpoint{0.000000in}{0.000000in}}%
\pgfpathlineto{\pgfqpoint{0.000000in}{-0.048611in}}%
\pgfusepath{stroke,fill}%
}%
\begin{pgfscope}%
\pgfsys@transformshift{5.471049in}{0.643904in}%
\pgfsys@useobject{currentmarker}{}%
\end{pgfscope}%
\end{pgfscope}%
\begin{pgfscope}%
\definecolor{textcolor}{rgb}{0.000000,0.000000,0.000000}%
\pgfsetstrokecolor{textcolor}%
\pgfsetfillcolor{textcolor}%
\pgftext[x=5.471049in,y=0.546682in,,top]{\color{textcolor}{\rmfamily\fontsize{14.000000}{16.800000}\selectfont\catcode`\^=\active\def^{\ifmmode\sp\else\^{}\fi}\catcode`\%=\active\def%{\%}$\mathdefault{160}$}}%
\end{pgfscope}%
\begin{pgfscope}%
\pgfpathrectangle{\pgfqpoint{0.688192in}{0.643904in}}{\pgfqpoint{6.200000in}{4.620000in}}%
\pgfusepath{clip}%
\pgfsetrectcap%
\pgfsetroundjoin%
\pgfsetlinewidth{0.803000pt}%
\definecolor{currentstroke}{rgb}{0.690196,0.690196,0.690196}%
\pgfsetstrokecolor{currentstroke}%
\pgfsetdash{}{0pt}%
\pgfpathmoveto{\pgfqpoint{6.179620in}{0.643904in}}%
\pgfpathlineto{\pgfqpoint{6.179620in}{5.263904in}}%
\pgfusepath{stroke}%
\end{pgfscope}%
\begin{pgfscope}%
\pgfsetbuttcap%
\pgfsetroundjoin%
\definecolor{currentfill}{rgb}{0.000000,0.000000,0.000000}%
\pgfsetfillcolor{currentfill}%
\pgfsetlinewidth{0.803000pt}%
\definecolor{currentstroke}{rgb}{0.000000,0.000000,0.000000}%
\pgfsetstrokecolor{currentstroke}%
\pgfsetdash{}{0pt}%
\pgfsys@defobject{currentmarker}{\pgfqpoint{0.000000in}{-0.048611in}}{\pgfqpoint{0.000000in}{0.000000in}}{%
\pgfpathmoveto{\pgfqpoint{0.000000in}{0.000000in}}%
\pgfpathlineto{\pgfqpoint{0.000000in}{-0.048611in}}%
\pgfusepath{stroke,fill}%
}%
\begin{pgfscope}%
\pgfsys@transformshift{6.179620in}{0.643904in}%
\pgfsys@useobject{currentmarker}{}%
\end{pgfscope}%
\end{pgfscope}%
\begin{pgfscope}%
\definecolor{textcolor}{rgb}{0.000000,0.000000,0.000000}%
\pgfsetstrokecolor{textcolor}%
\pgfsetfillcolor{textcolor}%
\pgftext[x=6.179620in,y=0.546682in,,top]{\color{textcolor}{\rmfamily\fontsize{14.000000}{16.800000}\selectfont\catcode`\^=\active\def^{\ifmmode\sp\else\^{}\fi}\catcode`\%=\active\def%{\%}$\mathdefault{180}$}}%
\end{pgfscope}%
\begin{pgfscope}%
\pgfpathrectangle{\pgfqpoint{0.688192in}{0.643904in}}{\pgfqpoint{6.200000in}{4.620000in}}%
\pgfusepath{clip}%
\pgfsetrectcap%
\pgfsetroundjoin%
\pgfsetlinewidth{0.803000pt}%
\definecolor{currentstroke}{rgb}{0.690196,0.690196,0.690196}%
\pgfsetstrokecolor{currentstroke}%
\pgfsetdash{}{0pt}%
\pgfpathmoveto{\pgfqpoint{6.888192in}{0.643904in}}%
\pgfpathlineto{\pgfqpoint{6.888192in}{5.263904in}}%
\pgfusepath{stroke}%
\end{pgfscope}%
\begin{pgfscope}%
\pgfsetbuttcap%
\pgfsetroundjoin%
\definecolor{currentfill}{rgb}{0.000000,0.000000,0.000000}%
\pgfsetfillcolor{currentfill}%
\pgfsetlinewidth{0.803000pt}%
\definecolor{currentstroke}{rgb}{0.000000,0.000000,0.000000}%
\pgfsetstrokecolor{currentstroke}%
\pgfsetdash{}{0pt}%
\pgfsys@defobject{currentmarker}{\pgfqpoint{0.000000in}{-0.048611in}}{\pgfqpoint{0.000000in}{0.000000in}}{%
\pgfpathmoveto{\pgfqpoint{0.000000in}{0.000000in}}%
\pgfpathlineto{\pgfqpoint{0.000000in}{-0.048611in}}%
\pgfusepath{stroke,fill}%
}%
\begin{pgfscope}%
\pgfsys@transformshift{6.888192in}{0.643904in}%
\pgfsys@useobject{currentmarker}{}%
\end{pgfscope}%
\end{pgfscope}%
\begin{pgfscope}%
\definecolor{textcolor}{rgb}{0.000000,0.000000,0.000000}%
\pgfsetstrokecolor{textcolor}%
\pgfsetfillcolor{textcolor}%
\pgftext[x=6.888192in,y=0.546682in,,top]{\color{textcolor}{\rmfamily\fontsize{14.000000}{16.800000}\selectfont\catcode`\^=\active\def^{\ifmmode\sp\else\^{}\fi}\catcode`\%=\active\def%{\%}$\mathdefault{200}$}}%
\end{pgfscope}%
\begin{pgfscope}%
\pgfpathrectangle{\pgfqpoint{0.688192in}{0.643904in}}{\pgfqpoint{6.200000in}{4.620000in}}%
\pgfusepath{clip}%
\pgfsetbuttcap%
\pgfsetroundjoin%
\pgfsetlinewidth{0.803000pt}%
\definecolor{currentstroke}{rgb}{0.690196,0.690196,0.690196}%
\pgfsetstrokecolor{currentstroke}%
\pgfsetstrokeopacity{0.200000}%
\pgfsetdash{{2.960000pt}{1.280000pt}}{0.000000pt}%
\pgfpathmoveto{\pgfqpoint{0.688192in}{0.643904in}}%
\pgfpathlineto{\pgfqpoint{0.688192in}{5.263904in}}%
\pgfusepath{stroke}%
\end{pgfscope}%
\begin{pgfscope}%
\pgfsetbuttcap%
\pgfsetroundjoin%
\definecolor{currentfill}{rgb}{0.000000,0.000000,0.000000}%
\pgfsetfillcolor{currentfill}%
\pgfsetlinewidth{0.602250pt}%
\definecolor{currentstroke}{rgb}{0.000000,0.000000,0.000000}%
\pgfsetstrokecolor{currentstroke}%
\pgfsetdash{}{0pt}%
\pgfsys@defobject{currentmarker}{\pgfqpoint{0.000000in}{-0.027778in}}{\pgfqpoint{0.000000in}{0.000000in}}{%
\pgfpathmoveto{\pgfqpoint{0.000000in}{0.000000in}}%
\pgfpathlineto{\pgfqpoint{0.000000in}{-0.027778in}}%
\pgfusepath{stroke,fill}%
}%
\begin{pgfscope}%
\pgfsys@transformshift{0.688192in}{0.643904in}%
\pgfsys@useobject{currentmarker}{}%
\end{pgfscope}%
\end{pgfscope}%
\begin{pgfscope}%
\pgfpathrectangle{\pgfqpoint{0.688192in}{0.643904in}}{\pgfqpoint{6.200000in}{4.620000in}}%
\pgfusepath{clip}%
\pgfsetbuttcap%
\pgfsetroundjoin%
\pgfsetlinewidth{0.803000pt}%
\definecolor{currentstroke}{rgb}{0.690196,0.690196,0.690196}%
\pgfsetstrokecolor{currentstroke}%
\pgfsetstrokeopacity{0.200000}%
\pgfsetdash{{2.960000pt}{1.280000pt}}{0.000000pt}%
\pgfpathmoveto{\pgfqpoint{0.865334in}{0.643904in}}%
\pgfpathlineto{\pgfqpoint{0.865334in}{5.263904in}}%
\pgfusepath{stroke}%
\end{pgfscope}%
\begin{pgfscope}%
\pgfsetbuttcap%
\pgfsetroundjoin%
\definecolor{currentfill}{rgb}{0.000000,0.000000,0.000000}%
\pgfsetfillcolor{currentfill}%
\pgfsetlinewidth{0.602250pt}%
\definecolor{currentstroke}{rgb}{0.000000,0.000000,0.000000}%
\pgfsetstrokecolor{currentstroke}%
\pgfsetdash{}{0pt}%
\pgfsys@defobject{currentmarker}{\pgfqpoint{0.000000in}{-0.027778in}}{\pgfqpoint{0.000000in}{0.000000in}}{%
\pgfpathmoveto{\pgfqpoint{0.000000in}{0.000000in}}%
\pgfpathlineto{\pgfqpoint{0.000000in}{-0.027778in}}%
\pgfusepath{stroke,fill}%
}%
\begin{pgfscope}%
\pgfsys@transformshift{0.865334in}{0.643904in}%
\pgfsys@useobject{currentmarker}{}%
\end{pgfscope}%
\end{pgfscope}%
\begin{pgfscope}%
\pgfpathrectangle{\pgfqpoint{0.688192in}{0.643904in}}{\pgfqpoint{6.200000in}{4.620000in}}%
\pgfusepath{clip}%
\pgfsetbuttcap%
\pgfsetroundjoin%
\pgfsetlinewidth{0.803000pt}%
\definecolor{currentstroke}{rgb}{0.690196,0.690196,0.690196}%
\pgfsetstrokecolor{currentstroke}%
\pgfsetstrokeopacity{0.200000}%
\pgfsetdash{{2.960000pt}{1.280000pt}}{0.000000pt}%
\pgfpathmoveto{\pgfqpoint{1.042477in}{0.643904in}}%
\pgfpathlineto{\pgfqpoint{1.042477in}{5.263904in}}%
\pgfusepath{stroke}%
\end{pgfscope}%
\begin{pgfscope}%
\pgfsetbuttcap%
\pgfsetroundjoin%
\definecolor{currentfill}{rgb}{0.000000,0.000000,0.000000}%
\pgfsetfillcolor{currentfill}%
\pgfsetlinewidth{0.602250pt}%
\definecolor{currentstroke}{rgb}{0.000000,0.000000,0.000000}%
\pgfsetstrokecolor{currentstroke}%
\pgfsetdash{}{0pt}%
\pgfsys@defobject{currentmarker}{\pgfqpoint{0.000000in}{-0.027778in}}{\pgfqpoint{0.000000in}{0.000000in}}{%
\pgfpathmoveto{\pgfqpoint{0.000000in}{0.000000in}}%
\pgfpathlineto{\pgfqpoint{0.000000in}{-0.027778in}}%
\pgfusepath{stroke,fill}%
}%
\begin{pgfscope}%
\pgfsys@transformshift{1.042477in}{0.643904in}%
\pgfsys@useobject{currentmarker}{}%
\end{pgfscope}%
\end{pgfscope}%
\begin{pgfscope}%
\pgfpathrectangle{\pgfqpoint{0.688192in}{0.643904in}}{\pgfqpoint{6.200000in}{4.620000in}}%
\pgfusepath{clip}%
\pgfsetbuttcap%
\pgfsetroundjoin%
\pgfsetlinewidth{0.803000pt}%
\definecolor{currentstroke}{rgb}{0.690196,0.690196,0.690196}%
\pgfsetstrokecolor{currentstroke}%
\pgfsetstrokeopacity{0.200000}%
\pgfsetdash{{2.960000pt}{1.280000pt}}{0.000000pt}%
\pgfpathmoveto{\pgfqpoint{1.396763in}{0.643904in}}%
\pgfpathlineto{\pgfqpoint{1.396763in}{5.263904in}}%
\pgfusepath{stroke}%
\end{pgfscope}%
\begin{pgfscope}%
\pgfsetbuttcap%
\pgfsetroundjoin%
\definecolor{currentfill}{rgb}{0.000000,0.000000,0.000000}%
\pgfsetfillcolor{currentfill}%
\pgfsetlinewidth{0.602250pt}%
\definecolor{currentstroke}{rgb}{0.000000,0.000000,0.000000}%
\pgfsetstrokecolor{currentstroke}%
\pgfsetdash{}{0pt}%
\pgfsys@defobject{currentmarker}{\pgfqpoint{0.000000in}{-0.027778in}}{\pgfqpoint{0.000000in}{0.000000in}}{%
\pgfpathmoveto{\pgfqpoint{0.000000in}{0.000000in}}%
\pgfpathlineto{\pgfqpoint{0.000000in}{-0.027778in}}%
\pgfusepath{stroke,fill}%
}%
\begin{pgfscope}%
\pgfsys@transformshift{1.396763in}{0.643904in}%
\pgfsys@useobject{currentmarker}{}%
\end{pgfscope}%
\end{pgfscope}%
\begin{pgfscope}%
\pgfpathrectangle{\pgfqpoint{0.688192in}{0.643904in}}{\pgfqpoint{6.200000in}{4.620000in}}%
\pgfusepath{clip}%
\pgfsetbuttcap%
\pgfsetroundjoin%
\pgfsetlinewidth{0.803000pt}%
\definecolor{currentstroke}{rgb}{0.690196,0.690196,0.690196}%
\pgfsetstrokecolor{currentstroke}%
\pgfsetstrokeopacity{0.200000}%
\pgfsetdash{{2.960000pt}{1.280000pt}}{0.000000pt}%
\pgfpathmoveto{\pgfqpoint{1.573906in}{0.643904in}}%
\pgfpathlineto{\pgfqpoint{1.573906in}{5.263904in}}%
\pgfusepath{stroke}%
\end{pgfscope}%
\begin{pgfscope}%
\pgfsetbuttcap%
\pgfsetroundjoin%
\definecolor{currentfill}{rgb}{0.000000,0.000000,0.000000}%
\pgfsetfillcolor{currentfill}%
\pgfsetlinewidth{0.602250pt}%
\definecolor{currentstroke}{rgb}{0.000000,0.000000,0.000000}%
\pgfsetstrokecolor{currentstroke}%
\pgfsetdash{}{0pt}%
\pgfsys@defobject{currentmarker}{\pgfqpoint{0.000000in}{-0.027778in}}{\pgfqpoint{0.000000in}{0.000000in}}{%
\pgfpathmoveto{\pgfqpoint{0.000000in}{0.000000in}}%
\pgfpathlineto{\pgfqpoint{0.000000in}{-0.027778in}}%
\pgfusepath{stroke,fill}%
}%
\begin{pgfscope}%
\pgfsys@transformshift{1.573906in}{0.643904in}%
\pgfsys@useobject{currentmarker}{}%
\end{pgfscope}%
\end{pgfscope}%
\begin{pgfscope}%
\pgfpathrectangle{\pgfqpoint{0.688192in}{0.643904in}}{\pgfqpoint{6.200000in}{4.620000in}}%
\pgfusepath{clip}%
\pgfsetbuttcap%
\pgfsetroundjoin%
\pgfsetlinewidth{0.803000pt}%
\definecolor{currentstroke}{rgb}{0.690196,0.690196,0.690196}%
\pgfsetstrokecolor{currentstroke}%
\pgfsetstrokeopacity{0.200000}%
\pgfsetdash{{2.960000pt}{1.280000pt}}{0.000000pt}%
\pgfpathmoveto{\pgfqpoint{1.751049in}{0.643904in}}%
\pgfpathlineto{\pgfqpoint{1.751049in}{5.263904in}}%
\pgfusepath{stroke}%
\end{pgfscope}%
\begin{pgfscope}%
\pgfsetbuttcap%
\pgfsetroundjoin%
\definecolor{currentfill}{rgb}{0.000000,0.000000,0.000000}%
\pgfsetfillcolor{currentfill}%
\pgfsetlinewidth{0.602250pt}%
\definecolor{currentstroke}{rgb}{0.000000,0.000000,0.000000}%
\pgfsetstrokecolor{currentstroke}%
\pgfsetdash{}{0pt}%
\pgfsys@defobject{currentmarker}{\pgfqpoint{0.000000in}{-0.027778in}}{\pgfqpoint{0.000000in}{0.000000in}}{%
\pgfpathmoveto{\pgfqpoint{0.000000in}{0.000000in}}%
\pgfpathlineto{\pgfqpoint{0.000000in}{-0.027778in}}%
\pgfusepath{stroke,fill}%
}%
\begin{pgfscope}%
\pgfsys@transformshift{1.751049in}{0.643904in}%
\pgfsys@useobject{currentmarker}{}%
\end{pgfscope}%
\end{pgfscope}%
\begin{pgfscope}%
\pgfpathrectangle{\pgfqpoint{0.688192in}{0.643904in}}{\pgfqpoint{6.200000in}{4.620000in}}%
\pgfusepath{clip}%
\pgfsetbuttcap%
\pgfsetroundjoin%
\pgfsetlinewidth{0.803000pt}%
\definecolor{currentstroke}{rgb}{0.690196,0.690196,0.690196}%
\pgfsetstrokecolor{currentstroke}%
\pgfsetstrokeopacity{0.200000}%
\pgfsetdash{{2.960000pt}{1.280000pt}}{0.000000pt}%
\pgfpathmoveto{\pgfqpoint{2.105334in}{0.643904in}}%
\pgfpathlineto{\pgfqpoint{2.105334in}{5.263904in}}%
\pgfusepath{stroke}%
\end{pgfscope}%
\begin{pgfscope}%
\pgfsetbuttcap%
\pgfsetroundjoin%
\definecolor{currentfill}{rgb}{0.000000,0.000000,0.000000}%
\pgfsetfillcolor{currentfill}%
\pgfsetlinewidth{0.602250pt}%
\definecolor{currentstroke}{rgb}{0.000000,0.000000,0.000000}%
\pgfsetstrokecolor{currentstroke}%
\pgfsetdash{}{0pt}%
\pgfsys@defobject{currentmarker}{\pgfqpoint{0.000000in}{-0.027778in}}{\pgfqpoint{0.000000in}{0.000000in}}{%
\pgfpathmoveto{\pgfqpoint{0.000000in}{0.000000in}}%
\pgfpathlineto{\pgfqpoint{0.000000in}{-0.027778in}}%
\pgfusepath{stroke,fill}%
}%
\begin{pgfscope}%
\pgfsys@transformshift{2.105334in}{0.643904in}%
\pgfsys@useobject{currentmarker}{}%
\end{pgfscope}%
\end{pgfscope}%
\begin{pgfscope}%
\pgfpathrectangle{\pgfqpoint{0.688192in}{0.643904in}}{\pgfqpoint{6.200000in}{4.620000in}}%
\pgfusepath{clip}%
\pgfsetbuttcap%
\pgfsetroundjoin%
\pgfsetlinewidth{0.803000pt}%
\definecolor{currentstroke}{rgb}{0.690196,0.690196,0.690196}%
\pgfsetstrokecolor{currentstroke}%
\pgfsetstrokeopacity{0.200000}%
\pgfsetdash{{2.960000pt}{1.280000pt}}{0.000000pt}%
\pgfpathmoveto{\pgfqpoint{2.282477in}{0.643904in}}%
\pgfpathlineto{\pgfqpoint{2.282477in}{5.263904in}}%
\pgfusepath{stroke}%
\end{pgfscope}%
\begin{pgfscope}%
\pgfsetbuttcap%
\pgfsetroundjoin%
\definecolor{currentfill}{rgb}{0.000000,0.000000,0.000000}%
\pgfsetfillcolor{currentfill}%
\pgfsetlinewidth{0.602250pt}%
\definecolor{currentstroke}{rgb}{0.000000,0.000000,0.000000}%
\pgfsetstrokecolor{currentstroke}%
\pgfsetdash{}{0pt}%
\pgfsys@defobject{currentmarker}{\pgfqpoint{0.000000in}{-0.027778in}}{\pgfqpoint{0.000000in}{0.000000in}}{%
\pgfpathmoveto{\pgfqpoint{0.000000in}{0.000000in}}%
\pgfpathlineto{\pgfqpoint{0.000000in}{-0.027778in}}%
\pgfusepath{stroke,fill}%
}%
\begin{pgfscope}%
\pgfsys@transformshift{2.282477in}{0.643904in}%
\pgfsys@useobject{currentmarker}{}%
\end{pgfscope}%
\end{pgfscope}%
\begin{pgfscope}%
\pgfpathrectangle{\pgfqpoint{0.688192in}{0.643904in}}{\pgfqpoint{6.200000in}{4.620000in}}%
\pgfusepath{clip}%
\pgfsetbuttcap%
\pgfsetroundjoin%
\pgfsetlinewidth{0.803000pt}%
\definecolor{currentstroke}{rgb}{0.690196,0.690196,0.690196}%
\pgfsetstrokecolor{currentstroke}%
\pgfsetstrokeopacity{0.200000}%
\pgfsetdash{{2.960000pt}{1.280000pt}}{0.000000pt}%
\pgfpathmoveto{\pgfqpoint{2.459620in}{0.643904in}}%
\pgfpathlineto{\pgfqpoint{2.459620in}{5.263904in}}%
\pgfusepath{stroke}%
\end{pgfscope}%
\begin{pgfscope}%
\pgfsetbuttcap%
\pgfsetroundjoin%
\definecolor{currentfill}{rgb}{0.000000,0.000000,0.000000}%
\pgfsetfillcolor{currentfill}%
\pgfsetlinewidth{0.602250pt}%
\definecolor{currentstroke}{rgb}{0.000000,0.000000,0.000000}%
\pgfsetstrokecolor{currentstroke}%
\pgfsetdash{}{0pt}%
\pgfsys@defobject{currentmarker}{\pgfqpoint{0.000000in}{-0.027778in}}{\pgfqpoint{0.000000in}{0.000000in}}{%
\pgfpathmoveto{\pgfqpoint{0.000000in}{0.000000in}}%
\pgfpathlineto{\pgfqpoint{0.000000in}{-0.027778in}}%
\pgfusepath{stroke,fill}%
}%
\begin{pgfscope}%
\pgfsys@transformshift{2.459620in}{0.643904in}%
\pgfsys@useobject{currentmarker}{}%
\end{pgfscope}%
\end{pgfscope}%
\begin{pgfscope}%
\pgfpathrectangle{\pgfqpoint{0.688192in}{0.643904in}}{\pgfqpoint{6.200000in}{4.620000in}}%
\pgfusepath{clip}%
\pgfsetbuttcap%
\pgfsetroundjoin%
\pgfsetlinewidth{0.803000pt}%
\definecolor{currentstroke}{rgb}{0.690196,0.690196,0.690196}%
\pgfsetstrokecolor{currentstroke}%
\pgfsetstrokeopacity{0.200000}%
\pgfsetdash{{2.960000pt}{1.280000pt}}{0.000000pt}%
\pgfpathmoveto{\pgfqpoint{2.813906in}{0.643904in}}%
\pgfpathlineto{\pgfqpoint{2.813906in}{5.263904in}}%
\pgfusepath{stroke}%
\end{pgfscope}%
\begin{pgfscope}%
\pgfsetbuttcap%
\pgfsetroundjoin%
\definecolor{currentfill}{rgb}{0.000000,0.000000,0.000000}%
\pgfsetfillcolor{currentfill}%
\pgfsetlinewidth{0.602250pt}%
\definecolor{currentstroke}{rgb}{0.000000,0.000000,0.000000}%
\pgfsetstrokecolor{currentstroke}%
\pgfsetdash{}{0pt}%
\pgfsys@defobject{currentmarker}{\pgfqpoint{0.000000in}{-0.027778in}}{\pgfqpoint{0.000000in}{0.000000in}}{%
\pgfpathmoveto{\pgfqpoint{0.000000in}{0.000000in}}%
\pgfpathlineto{\pgfqpoint{0.000000in}{-0.027778in}}%
\pgfusepath{stroke,fill}%
}%
\begin{pgfscope}%
\pgfsys@transformshift{2.813906in}{0.643904in}%
\pgfsys@useobject{currentmarker}{}%
\end{pgfscope}%
\end{pgfscope}%
\begin{pgfscope}%
\pgfpathrectangle{\pgfqpoint{0.688192in}{0.643904in}}{\pgfqpoint{6.200000in}{4.620000in}}%
\pgfusepath{clip}%
\pgfsetbuttcap%
\pgfsetroundjoin%
\pgfsetlinewidth{0.803000pt}%
\definecolor{currentstroke}{rgb}{0.690196,0.690196,0.690196}%
\pgfsetstrokecolor{currentstroke}%
\pgfsetstrokeopacity{0.200000}%
\pgfsetdash{{2.960000pt}{1.280000pt}}{0.000000pt}%
\pgfpathmoveto{\pgfqpoint{2.991049in}{0.643904in}}%
\pgfpathlineto{\pgfqpoint{2.991049in}{5.263904in}}%
\pgfusepath{stroke}%
\end{pgfscope}%
\begin{pgfscope}%
\pgfsetbuttcap%
\pgfsetroundjoin%
\definecolor{currentfill}{rgb}{0.000000,0.000000,0.000000}%
\pgfsetfillcolor{currentfill}%
\pgfsetlinewidth{0.602250pt}%
\definecolor{currentstroke}{rgb}{0.000000,0.000000,0.000000}%
\pgfsetstrokecolor{currentstroke}%
\pgfsetdash{}{0pt}%
\pgfsys@defobject{currentmarker}{\pgfqpoint{0.000000in}{-0.027778in}}{\pgfqpoint{0.000000in}{0.000000in}}{%
\pgfpathmoveto{\pgfqpoint{0.000000in}{0.000000in}}%
\pgfpathlineto{\pgfqpoint{0.000000in}{-0.027778in}}%
\pgfusepath{stroke,fill}%
}%
\begin{pgfscope}%
\pgfsys@transformshift{2.991049in}{0.643904in}%
\pgfsys@useobject{currentmarker}{}%
\end{pgfscope}%
\end{pgfscope}%
\begin{pgfscope}%
\pgfpathrectangle{\pgfqpoint{0.688192in}{0.643904in}}{\pgfqpoint{6.200000in}{4.620000in}}%
\pgfusepath{clip}%
\pgfsetbuttcap%
\pgfsetroundjoin%
\pgfsetlinewidth{0.803000pt}%
\definecolor{currentstroke}{rgb}{0.690196,0.690196,0.690196}%
\pgfsetstrokecolor{currentstroke}%
\pgfsetstrokeopacity{0.200000}%
\pgfsetdash{{2.960000pt}{1.280000pt}}{0.000000pt}%
\pgfpathmoveto{\pgfqpoint{3.168192in}{0.643904in}}%
\pgfpathlineto{\pgfqpoint{3.168192in}{5.263904in}}%
\pgfusepath{stroke}%
\end{pgfscope}%
\begin{pgfscope}%
\pgfsetbuttcap%
\pgfsetroundjoin%
\definecolor{currentfill}{rgb}{0.000000,0.000000,0.000000}%
\pgfsetfillcolor{currentfill}%
\pgfsetlinewidth{0.602250pt}%
\definecolor{currentstroke}{rgb}{0.000000,0.000000,0.000000}%
\pgfsetstrokecolor{currentstroke}%
\pgfsetdash{}{0pt}%
\pgfsys@defobject{currentmarker}{\pgfqpoint{0.000000in}{-0.027778in}}{\pgfqpoint{0.000000in}{0.000000in}}{%
\pgfpathmoveto{\pgfqpoint{0.000000in}{0.000000in}}%
\pgfpathlineto{\pgfqpoint{0.000000in}{-0.027778in}}%
\pgfusepath{stroke,fill}%
}%
\begin{pgfscope}%
\pgfsys@transformshift{3.168192in}{0.643904in}%
\pgfsys@useobject{currentmarker}{}%
\end{pgfscope}%
\end{pgfscope}%
\begin{pgfscope}%
\pgfpathrectangle{\pgfqpoint{0.688192in}{0.643904in}}{\pgfqpoint{6.200000in}{4.620000in}}%
\pgfusepath{clip}%
\pgfsetbuttcap%
\pgfsetroundjoin%
\pgfsetlinewidth{0.803000pt}%
\definecolor{currentstroke}{rgb}{0.690196,0.690196,0.690196}%
\pgfsetstrokecolor{currentstroke}%
\pgfsetstrokeopacity{0.200000}%
\pgfsetdash{{2.960000pt}{1.280000pt}}{0.000000pt}%
\pgfpathmoveto{\pgfqpoint{3.522477in}{0.643904in}}%
\pgfpathlineto{\pgfqpoint{3.522477in}{5.263904in}}%
\pgfusepath{stroke}%
\end{pgfscope}%
\begin{pgfscope}%
\pgfsetbuttcap%
\pgfsetroundjoin%
\definecolor{currentfill}{rgb}{0.000000,0.000000,0.000000}%
\pgfsetfillcolor{currentfill}%
\pgfsetlinewidth{0.602250pt}%
\definecolor{currentstroke}{rgb}{0.000000,0.000000,0.000000}%
\pgfsetstrokecolor{currentstroke}%
\pgfsetdash{}{0pt}%
\pgfsys@defobject{currentmarker}{\pgfqpoint{0.000000in}{-0.027778in}}{\pgfqpoint{0.000000in}{0.000000in}}{%
\pgfpathmoveto{\pgfqpoint{0.000000in}{0.000000in}}%
\pgfpathlineto{\pgfqpoint{0.000000in}{-0.027778in}}%
\pgfusepath{stroke,fill}%
}%
\begin{pgfscope}%
\pgfsys@transformshift{3.522477in}{0.643904in}%
\pgfsys@useobject{currentmarker}{}%
\end{pgfscope}%
\end{pgfscope}%
\begin{pgfscope}%
\pgfpathrectangle{\pgfqpoint{0.688192in}{0.643904in}}{\pgfqpoint{6.200000in}{4.620000in}}%
\pgfusepath{clip}%
\pgfsetbuttcap%
\pgfsetroundjoin%
\pgfsetlinewidth{0.803000pt}%
\definecolor{currentstroke}{rgb}{0.690196,0.690196,0.690196}%
\pgfsetstrokecolor{currentstroke}%
\pgfsetstrokeopacity{0.200000}%
\pgfsetdash{{2.960000pt}{1.280000pt}}{0.000000pt}%
\pgfpathmoveto{\pgfqpoint{3.699620in}{0.643904in}}%
\pgfpathlineto{\pgfqpoint{3.699620in}{5.263904in}}%
\pgfusepath{stroke}%
\end{pgfscope}%
\begin{pgfscope}%
\pgfsetbuttcap%
\pgfsetroundjoin%
\definecolor{currentfill}{rgb}{0.000000,0.000000,0.000000}%
\pgfsetfillcolor{currentfill}%
\pgfsetlinewidth{0.602250pt}%
\definecolor{currentstroke}{rgb}{0.000000,0.000000,0.000000}%
\pgfsetstrokecolor{currentstroke}%
\pgfsetdash{}{0pt}%
\pgfsys@defobject{currentmarker}{\pgfqpoint{0.000000in}{-0.027778in}}{\pgfqpoint{0.000000in}{0.000000in}}{%
\pgfpathmoveto{\pgfqpoint{0.000000in}{0.000000in}}%
\pgfpathlineto{\pgfqpoint{0.000000in}{-0.027778in}}%
\pgfusepath{stroke,fill}%
}%
\begin{pgfscope}%
\pgfsys@transformshift{3.699620in}{0.643904in}%
\pgfsys@useobject{currentmarker}{}%
\end{pgfscope}%
\end{pgfscope}%
\begin{pgfscope}%
\pgfpathrectangle{\pgfqpoint{0.688192in}{0.643904in}}{\pgfqpoint{6.200000in}{4.620000in}}%
\pgfusepath{clip}%
\pgfsetbuttcap%
\pgfsetroundjoin%
\pgfsetlinewidth{0.803000pt}%
\definecolor{currentstroke}{rgb}{0.690196,0.690196,0.690196}%
\pgfsetstrokecolor{currentstroke}%
\pgfsetstrokeopacity{0.200000}%
\pgfsetdash{{2.960000pt}{1.280000pt}}{0.000000pt}%
\pgfpathmoveto{\pgfqpoint{3.876763in}{0.643904in}}%
\pgfpathlineto{\pgfqpoint{3.876763in}{5.263904in}}%
\pgfusepath{stroke}%
\end{pgfscope}%
\begin{pgfscope}%
\pgfsetbuttcap%
\pgfsetroundjoin%
\definecolor{currentfill}{rgb}{0.000000,0.000000,0.000000}%
\pgfsetfillcolor{currentfill}%
\pgfsetlinewidth{0.602250pt}%
\definecolor{currentstroke}{rgb}{0.000000,0.000000,0.000000}%
\pgfsetstrokecolor{currentstroke}%
\pgfsetdash{}{0pt}%
\pgfsys@defobject{currentmarker}{\pgfqpoint{0.000000in}{-0.027778in}}{\pgfqpoint{0.000000in}{0.000000in}}{%
\pgfpathmoveto{\pgfqpoint{0.000000in}{0.000000in}}%
\pgfpathlineto{\pgfqpoint{0.000000in}{-0.027778in}}%
\pgfusepath{stroke,fill}%
}%
\begin{pgfscope}%
\pgfsys@transformshift{3.876763in}{0.643904in}%
\pgfsys@useobject{currentmarker}{}%
\end{pgfscope}%
\end{pgfscope}%
\begin{pgfscope}%
\pgfpathrectangle{\pgfqpoint{0.688192in}{0.643904in}}{\pgfqpoint{6.200000in}{4.620000in}}%
\pgfusepath{clip}%
\pgfsetbuttcap%
\pgfsetroundjoin%
\pgfsetlinewidth{0.803000pt}%
\definecolor{currentstroke}{rgb}{0.690196,0.690196,0.690196}%
\pgfsetstrokecolor{currentstroke}%
\pgfsetstrokeopacity{0.200000}%
\pgfsetdash{{2.960000pt}{1.280000pt}}{0.000000pt}%
\pgfpathmoveto{\pgfqpoint{4.231049in}{0.643904in}}%
\pgfpathlineto{\pgfqpoint{4.231049in}{5.263904in}}%
\pgfusepath{stroke}%
\end{pgfscope}%
\begin{pgfscope}%
\pgfsetbuttcap%
\pgfsetroundjoin%
\definecolor{currentfill}{rgb}{0.000000,0.000000,0.000000}%
\pgfsetfillcolor{currentfill}%
\pgfsetlinewidth{0.602250pt}%
\definecolor{currentstroke}{rgb}{0.000000,0.000000,0.000000}%
\pgfsetstrokecolor{currentstroke}%
\pgfsetdash{}{0pt}%
\pgfsys@defobject{currentmarker}{\pgfqpoint{0.000000in}{-0.027778in}}{\pgfqpoint{0.000000in}{0.000000in}}{%
\pgfpathmoveto{\pgfqpoint{0.000000in}{0.000000in}}%
\pgfpathlineto{\pgfqpoint{0.000000in}{-0.027778in}}%
\pgfusepath{stroke,fill}%
}%
\begin{pgfscope}%
\pgfsys@transformshift{4.231049in}{0.643904in}%
\pgfsys@useobject{currentmarker}{}%
\end{pgfscope}%
\end{pgfscope}%
\begin{pgfscope}%
\pgfpathrectangle{\pgfqpoint{0.688192in}{0.643904in}}{\pgfqpoint{6.200000in}{4.620000in}}%
\pgfusepath{clip}%
\pgfsetbuttcap%
\pgfsetroundjoin%
\pgfsetlinewidth{0.803000pt}%
\definecolor{currentstroke}{rgb}{0.690196,0.690196,0.690196}%
\pgfsetstrokecolor{currentstroke}%
\pgfsetstrokeopacity{0.200000}%
\pgfsetdash{{2.960000pt}{1.280000pt}}{0.000000pt}%
\pgfpathmoveto{\pgfqpoint{4.408192in}{0.643904in}}%
\pgfpathlineto{\pgfqpoint{4.408192in}{5.263904in}}%
\pgfusepath{stroke}%
\end{pgfscope}%
\begin{pgfscope}%
\pgfsetbuttcap%
\pgfsetroundjoin%
\definecolor{currentfill}{rgb}{0.000000,0.000000,0.000000}%
\pgfsetfillcolor{currentfill}%
\pgfsetlinewidth{0.602250pt}%
\definecolor{currentstroke}{rgb}{0.000000,0.000000,0.000000}%
\pgfsetstrokecolor{currentstroke}%
\pgfsetdash{}{0pt}%
\pgfsys@defobject{currentmarker}{\pgfqpoint{0.000000in}{-0.027778in}}{\pgfqpoint{0.000000in}{0.000000in}}{%
\pgfpathmoveto{\pgfqpoint{0.000000in}{0.000000in}}%
\pgfpathlineto{\pgfqpoint{0.000000in}{-0.027778in}}%
\pgfusepath{stroke,fill}%
}%
\begin{pgfscope}%
\pgfsys@transformshift{4.408192in}{0.643904in}%
\pgfsys@useobject{currentmarker}{}%
\end{pgfscope}%
\end{pgfscope}%
\begin{pgfscope}%
\pgfpathrectangle{\pgfqpoint{0.688192in}{0.643904in}}{\pgfqpoint{6.200000in}{4.620000in}}%
\pgfusepath{clip}%
\pgfsetbuttcap%
\pgfsetroundjoin%
\pgfsetlinewidth{0.803000pt}%
\definecolor{currentstroke}{rgb}{0.690196,0.690196,0.690196}%
\pgfsetstrokecolor{currentstroke}%
\pgfsetstrokeopacity{0.200000}%
\pgfsetdash{{2.960000pt}{1.280000pt}}{0.000000pt}%
\pgfpathmoveto{\pgfqpoint{4.585334in}{0.643904in}}%
\pgfpathlineto{\pgfqpoint{4.585334in}{5.263904in}}%
\pgfusepath{stroke}%
\end{pgfscope}%
\begin{pgfscope}%
\pgfsetbuttcap%
\pgfsetroundjoin%
\definecolor{currentfill}{rgb}{0.000000,0.000000,0.000000}%
\pgfsetfillcolor{currentfill}%
\pgfsetlinewidth{0.602250pt}%
\definecolor{currentstroke}{rgb}{0.000000,0.000000,0.000000}%
\pgfsetstrokecolor{currentstroke}%
\pgfsetdash{}{0pt}%
\pgfsys@defobject{currentmarker}{\pgfqpoint{0.000000in}{-0.027778in}}{\pgfqpoint{0.000000in}{0.000000in}}{%
\pgfpathmoveto{\pgfqpoint{0.000000in}{0.000000in}}%
\pgfpathlineto{\pgfqpoint{0.000000in}{-0.027778in}}%
\pgfusepath{stroke,fill}%
}%
\begin{pgfscope}%
\pgfsys@transformshift{4.585334in}{0.643904in}%
\pgfsys@useobject{currentmarker}{}%
\end{pgfscope}%
\end{pgfscope}%
\begin{pgfscope}%
\pgfpathrectangle{\pgfqpoint{0.688192in}{0.643904in}}{\pgfqpoint{6.200000in}{4.620000in}}%
\pgfusepath{clip}%
\pgfsetbuttcap%
\pgfsetroundjoin%
\pgfsetlinewidth{0.803000pt}%
\definecolor{currentstroke}{rgb}{0.690196,0.690196,0.690196}%
\pgfsetstrokecolor{currentstroke}%
\pgfsetstrokeopacity{0.200000}%
\pgfsetdash{{2.960000pt}{1.280000pt}}{0.000000pt}%
\pgfpathmoveto{\pgfqpoint{4.939620in}{0.643904in}}%
\pgfpathlineto{\pgfqpoint{4.939620in}{5.263904in}}%
\pgfusepath{stroke}%
\end{pgfscope}%
\begin{pgfscope}%
\pgfsetbuttcap%
\pgfsetroundjoin%
\definecolor{currentfill}{rgb}{0.000000,0.000000,0.000000}%
\pgfsetfillcolor{currentfill}%
\pgfsetlinewidth{0.602250pt}%
\definecolor{currentstroke}{rgb}{0.000000,0.000000,0.000000}%
\pgfsetstrokecolor{currentstroke}%
\pgfsetdash{}{0pt}%
\pgfsys@defobject{currentmarker}{\pgfqpoint{0.000000in}{-0.027778in}}{\pgfqpoint{0.000000in}{0.000000in}}{%
\pgfpathmoveto{\pgfqpoint{0.000000in}{0.000000in}}%
\pgfpathlineto{\pgfqpoint{0.000000in}{-0.027778in}}%
\pgfusepath{stroke,fill}%
}%
\begin{pgfscope}%
\pgfsys@transformshift{4.939620in}{0.643904in}%
\pgfsys@useobject{currentmarker}{}%
\end{pgfscope}%
\end{pgfscope}%
\begin{pgfscope}%
\pgfpathrectangle{\pgfqpoint{0.688192in}{0.643904in}}{\pgfqpoint{6.200000in}{4.620000in}}%
\pgfusepath{clip}%
\pgfsetbuttcap%
\pgfsetroundjoin%
\pgfsetlinewidth{0.803000pt}%
\definecolor{currentstroke}{rgb}{0.690196,0.690196,0.690196}%
\pgfsetstrokecolor{currentstroke}%
\pgfsetstrokeopacity{0.200000}%
\pgfsetdash{{2.960000pt}{1.280000pt}}{0.000000pt}%
\pgfpathmoveto{\pgfqpoint{5.116763in}{0.643904in}}%
\pgfpathlineto{\pgfqpoint{5.116763in}{5.263904in}}%
\pgfusepath{stroke}%
\end{pgfscope}%
\begin{pgfscope}%
\pgfsetbuttcap%
\pgfsetroundjoin%
\definecolor{currentfill}{rgb}{0.000000,0.000000,0.000000}%
\pgfsetfillcolor{currentfill}%
\pgfsetlinewidth{0.602250pt}%
\definecolor{currentstroke}{rgb}{0.000000,0.000000,0.000000}%
\pgfsetstrokecolor{currentstroke}%
\pgfsetdash{}{0pt}%
\pgfsys@defobject{currentmarker}{\pgfqpoint{0.000000in}{-0.027778in}}{\pgfqpoint{0.000000in}{0.000000in}}{%
\pgfpathmoveto{\pgfqpoint{0.000000in}{0.000000in}}%
\pgfpathlineto{\pgfqpoint{0.000000in}{-0.027778in}}%
\pgfusepath{stroke,fill}%
}%
\begin{pgfscope}%
\pgfsys@transformshift{5.116763in}{0.643904in}%
\pgfsys@useobject{currentmarker}{}%
\end{pgfscope}%
\end{pgfscope}%
\begin{pgfscope}%
\pgfpathrectangle{\pgfqpoint{0.688192in}{0.643904in}}{\pgfqpoint{6.200000in}{4.620000in}}%
\pgfusepath{clip}%
\pgfsetbuttcap%
\pgfsetroundjoin%
\pgfsetlinewidth{0.803000pt}%
\definecolor{currentstroke}{rgb}{0.690196,0.690196,0.690196}%
\pgfsetstrokecolor{currentstroke}%
\pgfsetstrokeopacity{0.200000}%
\pgfsetdash{{2.960000pt}{1.280000pt}}{0.000000pt}%
\pgfpathmoveto{\pgfqpoint{5.293906in}{0.643904in}}%
\pgfpathlineto{\pgfqpoint{5.293906in}{5.263904in}}%
\pgfusepath{stroke}%
\end{pgfscope}%
\begin{pgfscope}%
\pgfsetbuttcap%
\pgfsetroundjoin%
\definecolor{currentfill}{rgb}{0.000000,0.000000,0.000000}%
\pgfsetfillcolor{currentfill}%
\pgfsetlinewidth{0.602250pt}%
\definecolor{currentstroke}{rgb}{0.000000,0.000000,0.000000}%
\pgfsetstrokecolor{currentstroke}%
\pgfsetdash{}{0pt}%
\pgfsys@defobject{currentmarker}{\pgfqpoint{0.000000in}{-0.027778in}}{\pgfqpoint{0.000000in}{0.000000in}}{%
\pgfpathmoveto{\pgfqpoint{0.000000in}{0.000000in}}%
\pgfpathlineto{\pgfqpoint{0.000000in}{-0.027778in}}%
\pgfusepath{stroke,fill}%
}%
\begin{pgfscope}%
\pgfsys@transformshift{5.293906in}{0.643904in}%
\pgfsys@useobject{currentmarker}{}%
\end{pgfscope}%
\end{pgfscope}%
\begin{pgfscope}%
\pgfpathrectangle{\pgfqpoint{0.688192in}{0.643904in}}{\pgfqpoint{6.200000in}{4.620000in}}%
\pgfusepath{clip}%
\pgfsetbuttcap%
\pgfsetroundjoin%
\pgfsetlinewidth{0.803000pt}%
\definecolor{currentstroke}{rgb}{0.690196,0.690196,0.690196}%
\pgfsetstrokecolor{currentstroke}%
\pgfsetstrokeopacity{0.200000}%
\pgfsetdash{{2.960000pt}{1.280000pt}}{0.000000pt}%
\pgfpathmoveto{\pgfqpoint{5.648192in}{0.643904in}}%
\pgfpathlineto{\pgfqpoint{5.648192in}{5.263904in}}%
\pgfusepath{stroke}%
\end{pgfscope}%
\begin{pgfscope}%
\pgfsetbuttcap%
\pgfsetroundjoin%
\definecolor{currentfill}{rgb}{0.000000,0.000000,0.000000}%
\pgfsetfillcolor{currentfill}%
\pgfsetlinewidth{0.602250pt}%
\definecolor{currentstroke}{rgb}{0.000000,0.000000,0.000000}%
\pgfsetstrokecolor{currentstroke}%
\pgfsetdash{}{0pt}%
\pgfsys@defobject{currentmarker}{\pgfqpoint{0.000000in}{-0.027778in}}{\pgfqpoint{0.000000in}{0.000000in}}{%
\pgfpathmoveto{\pgfqpoint{0.000000in}{0.000000in}}%
\pgfpathlineto{\pgfqpoint{0.000000in}{-0.027778in}}%
\pgfusepath{stroke,fill}%
}%
\begin{pgfscope}%
\pgfsys@transformshift{5.648192in}{0.643904in}%
\pgfsys@useobject{currentmarker}{}%
\end{pgfscope}%
\end{pgfscope}%
\begin{pgfscope}%
\pgfpathrectangle{\pgfqpoint{0.688192in}{0.643904in}}{\pgfqpoint{6.200000in}{4.620000in}}%
\pgfusepath{clip}%
\pgfsetbuttcap%
\pgfsetroundjoin%
\pgfsetlinewidth{0.803000pt}%
\definecolor{currentstroke}{rgb}{0.690196,0.690196,0.690196}%
\pgfsetstrokecolor{currentstroke}%
\pgfsetstrokeopacity{0.200000}%
\pgfsetdash{{2.960000pt}{1.280000pt}}{0.000000pt}%
\pgfpathmoveto{\pgfqpoint{5.825334in}{0.643904in}}%
\pgfpathlineto{\pgfqpoint{5.825334in}{5.263904in}}%
\pgfusepath{stroke}%
\end{pgfscope}%
\begin{pgfscope}%
\pgfsetbuttcap%
\pgfsetroundjoin%
\definecolor{currentfill}{rgb}{0.000000,0.000000,0.000000}%
\pgfsetfillcolor{currentfill}%
\pgfsetlinewidth{0.602250pt}%
\definecolor{currentstroke}{rgb}{0.000000,0.000000,0.000000}%
\pgfsetstrokecolor{currentstroke}%
\pgfsetdash{}{0pt}%
\pgfsys@defobject{currentmarker}{\pgfqpoint{0.000000in}{-0.027778in}}{\pgfqpoint{0.000000in}{0.000000in}}{%
\pgfpathmoveto{\pgfqpoint{0.000000in}{0.000000in}}%
\pgfpathlineto{\pgfqpoint{0.000000in}{-0.027778in}}%
\pgfusepath{stroke,fill}%
}%
\begin{pgfscope}%
\pgfsys@transformshift{5.825334in}{0.643904in}%
\pgfsys@useobject{currentmarker}{}%
\end{pgfscope}%
\end{pgfscope}%
\begin{pgfscope}%
\pgfpathrectangle{\pgfqpoint{0.688192in}{0.643904in}}{\pgfqpoint{6.200000in}{4.620000in}}%
\pgfusepath{clip}%
\pgfsetbuttcap%
\pgfsetroundjoin%
\pgfsetlinewidth{0.803000pt}%
\definecolor{currentstroke}{rgb}{0.690196,0.690196,0.690196}%
\pgfsetstrokecolor{currentstroke}%
\pgfsetstrokeopacity{0.200000}%
\pgfsetdash{{2.960000pt}{1.280000pt}}{0.000000pt}%
\pgfpathmoveto{\pgfqpoint{6.002477in}{0.643904in}}%
\pgfpathlineto{\pgfqpoint{6.002477in}{5.263904in}}%
\pgfusepath{stroke}%
\end{pgfscope}%
\begin{pgfscope}%
\pgfsetbuttcap%
\pgfsetroundjoin%
\definecolor{currentfill}{rgb}{0.000000,0.000000,0.000000}%
\pgfsetfillcolor{currentfill}%
\pgfsetlinewidth{0.602250pt}%
\definecolor{currentstroke}{rgb}{0.000000,0.000000,0.000000}%
\pgfsetstrokecolor{currentstroke}%
\pgfsetdash{}{0pt}%
\pgfsys@defobject{currentmarker}{\pgfqpoint{0.000000in}{-0.027778in}}{\pgfqpoint{0.000000in}{0.000000in}}{%
\pgfpathmoveto{\pgfqpoint{0.000000in}{0.000000in}}%
\pgfpathlineto{\pgfqpoint{0.000000in}{-0.027778in}}%
\pgfusepath{stroke,fill}%
}%
\begin{pgfscope}%
\pgfsys@transformshift{6.002477in}{0.643904in}%
\pgfsys@useobject{currentmarker}{}%
\end{pgfscope}%
\end{pgfscope}%
\begin{pgfscope}%
\pgfpathrectangle{\pgfqpoint{0.688192in}{0.643904in}}{\pgfqpoint{6.200000in}{4.620000in}}%
\pgfusepath{clip}%
\pgfsetbuttcap%
\pgfsetroundjoin%
\pgfsetlinewidth{0.803000pt}%
\definecolor{currentstroke}{rgb}{0.690196,0.690196,0.690196}%
\pgfsetstrokecolor{currentstroke}%
\pgfsetstrokeopacity{0.200000}%
\pgfsetdash{{2.960000pt}{1.280000pt}}{0.000000pt}%
\pgfpathmoveto{\pgfqpoint{6.356763in}{0.643904in}}%
\pgfpathlineto{\pgfqpoint{6.356763in}{5.263904in}}%
\pgfusepath{stroke}%
\end{pgfscope}%
\begin{pgfscope}%
\pgfsetbuttcap%
\pgfsetroundjoin%
\definecolor{currentfill}{rgb}{0.000000,0.000000,0.000000}%
\pgfsetfillcolor{currentfill}%
\pgfsetlinewidth{0.602250pt}%
\definecolor{currentstroke}{rgb}{0.000000,0.000000,0.000000}%
\pgfsetstrokecolor{currentstroke}%
\pgfsetdash{}{0pt}%
\pgfsys@defobject{currentmarker}{\pgfqpoint{0.000000in}{-0.027778in}}{\pgfqpoint{0.000000in}{0.000000in}}{%
\pgfpathmoveto{\pgfqpoint{0.000000in}{0.000000in}}%
\pgfpathlineto{\pgfqpoint{0.000000in}{-0.027778in}}%
\pgfusepath{stroke,fill}%
}%
\begin{pgfscope}%
\pgfsys@transformshift{6.356763in}{0.643904in}%
\pgfsys@useobject{currentmarker}{}%
\end{pgfscope}%
\end{pgfscope}%
\begin{pgfscope}%
\pgfpathrectangle{\pgfqpoint{0.688192in}{0.643904in}}{\pgfqpoint{6.200000in}{4.620000in}}%
\pgfusepath{clip}%
\pgfsetbuttcap%
\pgfsetroundjoin%
\pgfsetlinewidth{0.803000pt}%
\definecolor{currentstroke}{rgb}{0.690196,0.690196,0.690196}%
\pgfsetstrokecolor{currentstroke}%
\pgfsetstrokeopacity{0.200000}%
\pgfsetdash{{2.960000pt}{1.280000pt}}{0.000000pt}%
\pgfpathmoveto{\pgfqpoint{6.533906in}{0.643904in}}%
\pgfpathlineto{\pgfqpoint{6.533906in}{5.263904in}}%
\pgfusepath{stroke}%
\end{pgfscope}%
\begin{pgfscope}%
\pgfsetbuttcap%
\pgfsetroundjoin%
\definecolor{currentfill}{rgb}{0.000000,0.000000,0.000000}%
\pgfsetfillcolor{currentfill}%
\pgfsetlinewidth{0.602250pt}%
\definecolor{currentstroke}{rgb}{0.000000,0.000000,0.000000}%
\pgfsetstrokecolor{currentstroke}%
\pgfsetdash{}{0pt}%
\pgfsys@defobject{currentmarker}{\pgfqpoint{0.000000in}{-0.027778in}}{\pgfqpoint{0.000000in}{0.000000in}}{%
\pgfpathmoveto{\pgfqpoint{0.000000in}{0.000000in}}%
\pgfpathlineto{\pgfqpoint{0.000000in}{-0.027778in}}%
\pgfusepath{stroke,fill}%
}%
\begin{pgfscope}%
\pgfsys@transformshift{6.533906in}{0.643904in}%
\pgfsys@useobject{currentmarker}{}%
\end{pgfscope}%
\end{pgfscope}%
\begin{pgfscope}%
\pgfpathrectangle{\pgfqpoint{0.688192in}{0.643904in}}{\pgfqpoint{6.200000in}{4.620000in}}%
\pgfusepath{clip}%
\pgfsetbuttcap%
\pgfsetroundjoin%
\pgfsetlinewidth{0.803000pt}%
\definecolor{currentstroke}{rgb}{0.690196,0.690196,0.690196}%
\pgfsetstrokecolor{currentstroke}%
\pgfsetstrokeopacity{0.200000}%
\pgfsetdash{{2.960000pt}{1.280000pt}}{0.000000pt}%
\pgfpathmoveto{\pgfqpoint{6.711049in}{0.643904in}}%
\pgfpathlineto{\pgfqpoint{6.711049in}{5.263904in}}%
\pgfusepath{stroke}%
\end{pgfscope}%
\begin{pgfscope}%
\pgfsetbuttcap%
\pgfsetroundjoin%
\definecolor{currentfill}{rgb}{0.000000,0.000000,0.000000}%
\pgfsetfillcolor{currentfill}%
\pgfsetlinewidth{0.602250pt}%
\definecolor{currentstroke}{rgb}{0.000000,0.000000,0.000000}%
\pgfsetstrokecolor{currentstroke}%
\pgfsetdash{}{0pt}%
\pgfsys@defobject{currentmarker}{\pgfqpoint{0.000000in}{-0.027778in}}{\pgfqpoint{0.000000in}{0.000000in}}{%
\pgfpathmoveto{\pgfqpoint{0.000000in}{0.000000in}}%
\pgfpathlineto{\pgfqpoint{0.000000in}{-0.027778in}}%
\pgfusepath{stroke,fill}%
}%
\begin{pgfscope}%
\pgfsys@transformshift{6.711049in}{0.643904in}%
\pgfsys@useobject{currentmarker}{}%
\end{pgfscope}%
\end{pgfscope}%
\begin{pgfscope}%
\definecolor{textcolor}{rgb}{0.000000,0.000000,0.000000}%
\pgfsetstrokecolor{textcolor}%
\pgfsetfillcolor{textcolor}%
\pgftext[x=3.788192in,y=0.313349in,,top]{\color{textcolor}{\rmfamily\fontsize{18.000000}{21.600000}\selectfont\catcode`\^=\active\def^{\ifmmode\sp\else\^{}\fi}\catcode`\%=\active\def%{\%}Population per Generation}}%
\end{pgfscope}%
\begin{pgfscope}%
\pgfpathrectangle{\pgfqpoint{0.688192in}{0.643904in}}{\pgfqpoint{6.200000in}{4.620000in}}%
\pgfusepath{clip}%
\pgfsetrectcap%
\pgfsetroundjoin%
\pgfsetlinewidth{0.803000pt}%
\definecolor{currentstroke}{rgb}{0.690196,0.690196,0.690196}%
\pgfsetstrokecolor{currentstroke}%
\pgfsetdash{}{0pt}%
\pgfpathmoveto{\pgfqpoint{0.688192in}{1.118118in}}%
\pgfpathlineto{\pgfqpoint{6.888192in}{1.118118in}}%
\pgfusepath{stroke}%
\end{pgfscope}%
\begin{pgfscope}%
\pgfsetbuttcap%
\pgfsetroundjoin%
\definecolor{currentfill}{rgb}{0.000000,0.000000,0.000000}%
\pgfsetfillcolor{currentfill}%
\pgfsetlinewidth{0.803000pt}%
\definecolor{currentstroke}{rgb}{0.000000,0.000000,0.000000}%
\pgfsetstrokecolor{currentstroke}%
\pgfsetdash{}{0pt}%
\pgfsys@defobject{currentmarker}{\pgfqpoint{-0.048611in}{0.000000in}}{\pgfqpoint{-0.000000in}{0.000000in}}{%
\pgfpathmoveto{\pgfqpoint{-0.000000in}{0.000000in}}%
\pgfpathlineto{\pgfqpoint{-0.048611in}{0.000000in}}%
\pgfusepath{stroke,fill}%
}%
\begin{pgfscope}%
\pgfsys@transformshift{0.688192in}{1.118118in}%
\pgfsys@useobject{currentmarker}{}%
\end{pgfscope}%
\end{pgfscope}%
\begin{pgfscope}%
\definecolor{textcolor}{rgb}{0.000000,0.000000,0.000000}%
\pgfsetstrokecolor{textcolor}%
\pgfsetfillcolor{textcolor}%
\pgftext[x=0.493054in, y=1.048674in, left, base]{\color{textcolor}{\rmfamily\fontsize{14.000000}{16.800000}\selectfont\catcode`\^=\active\def^{\ifmmode\sp\else\^{}\fi}\catcode`\%=\active\def%{\%}$\mathdefault{5}$}}%
\end{pgfscope}%
\begin{pgfscope}%
\pgfpathrectangle{\pgfqpoint{0.688192in}{0.643904in}}{\pgfqpoint{6.200000in}{4.620000in}}%
\pgfusepath{clip}%
\pgfsetrectcap%
\pgfsetroundjoin%
\pgfsetlinewidth{0.803000pt}%
\definecolor{currentstroke}{rgb}{0.690196,0.690196,0.690196}%
\pgfsetstrokecolor{currentstroke}%
\pgfsetdash{}{0pt}%
\pgfpathmoveto{\pgfqpoint{0.688192in}{1.925862in}}%
\pgfpathlineto{\pgfqpoint{6.888192in}{1.925862in}}%
\pgfusepath{stroke}%
\end{pgfscope}%
\begin{pgfscope}%
\pgfsetbuttcap%
\pgfsetroundjoin%
\definecolor{currentfill}{rgb}{0.000000,0.000000,0.000000}%
\pgfsetfillcolor{currentfill}%
\pgfsetlinewidth{0.803000pt}%
\definecolor{currentstroke}{rgb}{0.000000,0.000000,0.000000}%
\pgfsetstrokecolor{currentstroke}%
\pgfsetdash{}{0pt}%
\pgfsys@defobject{currentmarker}{\pgfqpoint{-0.048611in}{0.000000in}}{\pgfqpoint{-0.000000in}{0.000000in}}{%
\pgfpathmoveto{\pgfqpoint{-0.000000in}{0.000000in}}%
\pgfpathlineto{\pgfqpoint{-0.048611in}{0.000000in}}%
\pgfusepath{stroke,fill}%
}%
\begin{pgfscope}%
\pgfsys@transformshift{0.688192in}{1.925862in}%
\pgfsys@useobject{currentmarker}{}%
\end{pgfscope}%
\end{pgfscope}%
\begin{pgfscope}%
\definecolor{textcolor}{rgb}{0.000000,0.000000,0.000000}%
\pgfsetstrokecolor{textcolor}%
\pgfsetfillcolor{textcolor}%
\pgftext[x=0.395138in, y=1.856418in, left, base]{\color{textcolor}{\rmfamily\fontsize{14.000000}{16.800000}\selectfont\catcode`\^=\active\def^{\ifmmode\sp\else\^{}\fi}\catcode`\%=\active\def%{\%}$\mathdefault{10}$}}%
\end{pgfscope}%
\begin{pgfscope}%
\pgfpathrectangle{\pgfqpoint{0.688192in}{0.643904in}}{\pgfqpoint{6.200000in}{4.620000in}}%
\pgfusepath{clip}%
\pgfsetrectcap%
\pgfsetroundjoin%
\pgfsetlinewidth{0.803000pt}%
\definecolor{currentstroke}{rgb}{0.690196,0.690196,0.690196}%
\pgfsetstrokecolor{currentstroke}%
\pgfsetdash{}{0pt}%
\pgfpathmoveto{\pgfqpoint{0.688192in}{2.733606in}}%
\pgfpathlineto{\pgfqpoint{6.888192in}{2.733606in}}%
\pgfusepath{stroke}%
\end{pgfscope}%
\begin{pgfscope}%
\pgfsetbuttcap%
\pgfsetroundjoin%
\definecolor{currentfill}{rgb}{0.000000,0.000000,0.000000}%
\pgfsetfillcolor{currentfill}%
\pgfsetlinewidth{0.803000pt}%
\definecolor{currentstroke}{rgb}{0.000000,0.000000,0.000000}%
\pgfsetstrokecolor{currentstroke}%
\pgfsetdash{}{0pt}%
\pgfsys@defobject{currentmarker}{\pgfqpoint{-0.048611in}{0.000000in}}{\pgfqpoint{-0.000000in}{0.000000in}}{%
\pgfpathmoveto{\pgfqpoint{-0.000000in}{0.000000in}}%
\pgfpathlineto{\pgfqpoint{-0.048611in}{0.000000in}}%
\pgfusepath{stroke,fill}%
}%
\begin{pgfscope}%
\pgfsys@transformshift{0.688192in}{2.733606in}%
\pgfsys@useobject{currentmarker}{}%
\end{pgfscope}%
\end{pgfscope}%
\begin{pgfscope}%
\definecolor{textcolor}{rgb}{0.000000,0.000000,0.000000}%
\pgfsetstrokecolor{textcolor}%
\pgfsetfillcolor{textcolor}%
\pgftext[x=0.395138in, y=2.664162in, left, base]{\color{textcolor}{\rmfamily\fontsize{14.000000}{16.800000}\selectfont\catcode`\^=\active\def^{\ifmmode\sp\else\^{}\fi}\catcode`\%=\active\def%{\%}$\mathdefault{15}$}}%
\end{pgfscope}%
\begin{pgfscope}%
\pgfpathrectangle{\pgfqpoint{0.688192in}{0.643904in}}{\pgfqpoint{6.200000in}{4.620000in}}%
\pgfusepath{clip}%
\pgfsetrectcap%
\pgfsetroundjoin%
\pgfsetlinewidth{0.803000pt}%
\definecolor{currentstroke}{rgb}{0.690196,0.690196,0.690196}%
\pgfsetstrokecolor{currentstroke}%
\pgfsetdash{}{0pt}%
\pgfpathmoveto{\pgfqpoint{0.688192in}{3.541350in}}%
\pgfpathlineto{\pgfqpoint{6.888192in}{3.541350in}}%
\pgfusepath{stroke}%
\end{pgfscope}%
\begin{pgfscope}%
\pgfsetbuttcap%
\pgfsetroundjoin%
\definecolor{currentfill}{rgb}{0.000000,0.000000,0.000000}%
\pgfsetfillcolor{currentfill}%
\pgfsetlinewidth{0.803000pt}%
\definecolor{currentstroke}{rgb}{0.000000,0.000000,0.000000}%
\pgfsetstrokecolor{currentstroke}%
\pgfsetdash{}{0pt}%
\pgfsys@defobject{currentmarker}{\pgfqpoint{-0.048611in}{0.000000in}}{\pgfqpoint{-0.000000in}{0.000000in}}{%
\pgfpathmoveto{\pgfqpoint{-0.000000in}{0.000000in}}%
\pgfpathlineto{\pgfqpoint{-0.048611in}{0.000000in}}%
\pgfusepath{stroke,fill}%
}%
\begin{pgfscope}%
\pgfsys@transformshift{0.688192in}{3.541350in}%
\pgfsys@useobject{currentmarker}{}%
\end{pgfscope}%
\end{pgfscope}%
\begin{pgfscope}%
\definecolor{textcolor}{rgb}{0.000000,0.000000,0.000000}%
\pgfsetstrokecolor{textcolor}%
\pgfsetfillcolor{textcolor}%
\pgftext[x=0.395138in, y=3.471906in, left, base]{\color{textcolor}{\rmfamily\fontsize{14.000000}{16.800000}\selectfont\catcode`\^=\active\def^{\ifmmode\sp\else\^{}\fi}\catcode`\%=\active\def%{\%}$\mathdefault{20}$}}%
\end{pgfscope}%
\begin{pgfscope}%
\pgfpathrectangle{\pgfqpoint{0.688192in}{0.643904in}}{\pgfqpoint{6.200000in}{4.620000in}}%
\pgfusepath{clip}%
\pgfsetrectcap%
\pgfsetroundjoin%
\pgfsetlinewidth{0.803000pt}%
\definecolor{currentstroke}{rgb}{0.690196,0.690196,0.690196}%
\pgfsetstrokecolor{currentstroke}%
\pgfsetdash{}{0pt}%
\pgfpathmoveto{\pgfqpoint{0.688192in}{4.349094in}}%
\pgfpathlineto{\pgfqpoint{6.888192in}{4.349094in}}%
\pgfusepath{stroke}%
\end{pgfscope}%
\begin{pgfscope}%
\pgfsetbuttcap%
\pgfsetroundjoin%
\definecolor{currentfill}{rgb}{0.000000,0.000000,0.000000}%
\pgfsetfillcolor{currentfill}%
\pgfsetlinewidth{0.803000pt}%
\definecolor{currentstroke}{rgb}{0.000000,0.000000,0.000000}%
\pgfsetstrokecolor{currentstroke}%
\pgfsetdash{}{0pt}%
\pgfsys@defobject{currentmarker}{\pgfqpoint{-0.048611in}{0.000000in}}{\pgfqpoint{-0.000000in}{0.000000in}}{%
\pgfpathmoveto{\pgfqpoint{-0.000000in}{0.000000in}}%
\pgfpathlineto{\pgfqpoint{-0.048611in}{0.000000in}}%
\pgfusepath{stroke,fill}%
}%
\begin{pgfscope}%
\pgfsys@transformshift{0.688192in}{4.349094in}%
\pgfsys@useobject{currentmarker}{}%
\end{pgfscope}%
\end{pgfscope}%
\begin{pgfscope}%
\definecolor{textcolor}{rgb}{0.000000,0.000000,0.000000}%
\pgfsetstrokecolor{textcolor}%
\pgfsetfillcolor{textcolor}%
\pgftext[x=0.395138in, y=4.279650in, left, base]{\color{textcolor}{\rmfamily\fontsize{14.000000}{16.800000}\selectfont\catcode`\^=\active\def^{\ifmmode\sp\else\^{}\fi}\catcode`\%=\active\def%{\%}$\mathdefault{25}$}}%
\end{pgfscope}%
\begin{pgfscope}%
\pgfpathrectangle{\pgfqpoint{0.688192in}{0.643904in}}{\pgfqpoint{6.200000in}{4.620000in}}%
\pgfusepath{clip}%
\pgfsetrectcap%
\pgfsetroundjoin%
\pgfsetlinewidth{0.803000pt}%
\definecolor{currentstroke}{rgb}{0.690196,0.690196,0.690196}%
\pgfsetstrokecolor{currentstroke}%
\pgfsetdash{}{0pt}%
\pgfpathmoveto{\pgfqpoint{0.688192in}{5.156838in}}%
\pgfpathlineto{\pgfqpoint{6.888192in}{5.156838in}}%
\pgfusepath{stroke}%
\end{pgfscope}%
\begin{pgfscope}%
\pgfsetbuttcap%
\pgfsetroundjoin%
\definecolor{currentfill}{rgb}{0.000000,0.000000,0.000000}%
\pgfsetfillcolor{currentfill}%
\pgfsetlinewidth{0.803000pt}%
\definecolor{currentstroke}{rgb}{0.000000,0.000000,0.000000}%
\pgfsetstrokecolor{currentstroke}%
\pgfsetdash{}{0pt}%
\pgfsys@defobject{currentmarker}{\pgfqpoint{-0.048611in}{0.000000in}}{\pgfqpoint{-0.000000in}{0.000000in}}{%
\pgfpathmoveto{\pgfqpoint{-0.000000in}{0.000000in}}%
\pgfpathlineto{\pgfqpoint{-0.048611in}{0.000000in}}%
\pgfusepath{stroke,fill}%
}%
\begin{pgfscope}%
\pgfsys@transformshift{0.688192in}{5.156838in}%
\pgfsys@useobject{currentmarker}{}%
\end{pgfscope}%
\end{pgfscope}%
\begin{pgfscope}%
\definecolor{textcolor}{rgb}{0.000000,0.000000,0.000000}%
\pgfsetstrokecolor{textcolor}%
\pgfsetfillcolor{textcolor}%
\pgftext[x=0.395138in, y=5.087394in, left, base]{\color{textcolor}{\rmfamily\fontsize{14.000000}{16.800000}\selectfont\catcode`\^=\active\def^{\ifmmode\sp\else\^{}\fi}\catcode`\%=\active\def%{\%}$\mathdefault{30}$}}%
\end{pgfscope}%
\begin{pgfscope}%
\pgfpathrectangle{\pgfqpoint{0.688192in}{0.643904in}}{\pgfqpoint{6.200000in}{4.620000in}}%
\pgfusepath{clip}%
\pgfsetbuttcap%
\pgfsetroundjoin%
\pgfsetlinewidth{0.803000pt}%
\definecolor{currentstroke}{rgb}{0.690196,0.690196,0.690196}%
\pgfsetstrokecolor{currentstroke}%
\pgfsetstrokeopacity{0.200000}%
\pgfsetdash{{2.960000pt}{1.280000pt}}{0.000000pt}%
\pgfpathmoveto{\pgfqpoint{0.688192in}{0.795020in}}%
\pgfpathlineto{\pgfqpoint{6.888192in}{0.795020in}}%
\pgfusepath{stroke}%
\end{pgfscope}%
\begin{pgfscope}%
\pgfsetbuttcap%
\pgfsetroundjoin%
\definecolor{currentfill}{rgb}{0.000000,0.000000,0.000000}%
\pgfsetfillcolor{currentfill}%
\pgfsetlinewidth{0.602250pt}%
\definecolor{currentstroke}{rgb}{0.000000,0.000000,0.000000}%
\pgfsetstrokecolor{currentstroke}%
\pgfsetdash{}{0pt}%
\pgfsys@defobject{currentmarker}{\pgfqpoint{-0.027778in}{0.000000in}}{\pgfqpoint{-0.000000in}{0.000000in}}{%
\pgfpathmoveto{\pgfqpoint{-0.000000in}{0.000000in}}%
\pgfpathlineto{\pgfqpoint{-0.027778in}{0.000000in}}%
\pgfusepath{stroke,fill}%
}%
\begin{pgfscope}%
\pgfsys@transformshift{0.688192in}{0.795020in}%
\pgfsys@useobject{currentmarker}{}%
\end{pgfscope}%
\end{pgfscope}%
\begin{pgfscope}%
\pgfpathrectangle{\pgfqpoint{0.688192in}{0.643904in}}{\pgfqpoint{6.200000in}{4.620000in}}%
\pgfusepath{clip}%
\pgfsetbuttcap%
\pgfsetroundjoin%
\pgfsetlinewidth{0.803000pt}%
\definecolor{currentstroke}{rgb}{0.690196,0.690196,0.690196}%
\pgfsetstrokecolor{currentstroke}%
\pgfsetstrokeopacity{0.200000}%
\pgfsetdash{{2.960000pt}{1.280000pt}}{0.000000pt}%
\pgfpathmoveto{\pgfqpoint{0.688192in}{0.956569in}}%
\pgfpathlineto{\pgfqpoint{6.888192in}{0.956569in}}%
\pgfusepath{stroke}%
\end{pgfscope}%
\begin{pgfscope}%
\pgfsetbuttcap%
\pgfsetroundjoin%
\definecolor{currentfill}{rgb}{0.000000,0.000000,0.000000}%
\pgfsetfillcolor{currentfill}%
\pgfsetlinewidth{0.602250pt}%
\definecolor{currentstroke}{rgb}{0.000000,0.000000,0.000000}%
\pgfsetstrokecolor{currentstroke}%
\pgfsetdash{}{0pt}%
\pgfsys@defobject{currentmarker}{\pgfqpoint{-0.027778in}{0.000000in}}{\pgfqpoint{-0.000000in}{0.000000in}}{%
\pgfpathmoveto{\pgfqpoint{-0.000000in}{0.000000in}}%
\pgfpathlineto{\pgfqpoint{-0.027778in}{0.000000in}}%
\pgfusepath{stroke,fill}%
}%
\begin{pgfscope}%
\pgfsys@transformshift{0.688192in}{0.956569in}%
\pgfsys@useobject{currentmarker}{}%
\end{pgfscope}%
\end{pgfscope}%
\begin{pgfscope}%
\pgfpathrectangle{\pgfqpoint{0.688192in}{0.643904in}}{\pgfqpoint{6.200000in}{4.620000in}}%
\pgfusepath{clip}%
\pgfsetbuttcap%
\pgfsetroundjoin%
\pgfsetlinewidth{0.803000pt}%
\definecolor{currentstroke}{rgb}{0.690196,0.690196,0.690196}%
\pgfsetstrokecolor{currentstroke}%
\pgfsetstrokeopacity{0.200000}%
\pgfsetdash{{2.960000pt}{1.280000pt}}{0.000000pt}%
\pgfpathmoveto{\pgfqpoint{0.688192in}{1.279667in}}%
\pgfpathlineto{\pgfqpoint{6.888192in}{1.279667in}}%
\pgfusepath{stroke}%
\end{pgfscope}%
\begin{pgfscope}%
\pgfsetbuttcap%
\pgfsetroundjoin%
\definecolor{currentfill}{rgb}{0.000000,0.000000,0.000000}%
\pgfsetfillcolor{currentfill}%
\pgfsetlinewidth{0.602250pt}%
\definecolor{currentstroke}{rgb}{0.000000,0.000000,0.000000}%
\pgfsetstrokecolor{currentstroke}%
\pgfsetdash{}{0pt}%
\pgfsys@defobject{currentmarker}{\pgfqpoint{-0.027778in}{0.000000in}}{\pgfqpoint{-0.000000in}{0.000000in}}{%
\pgfpathmoveto{\pgfqpoint{-0.000000in}{0.000000in}}%
\pgfpathlineto{\pgfqpoint{-0.027778in}{0.000000in}}%
\pgfusepath{stroke,fill}%
}%
\begin{pgfscope}%
\pgfsys@transformshift{0.688192in}{1.279667in}%
\pgfsys@useobject{currentmarker}{}%
\end{pgfscope}%
\end{pgfscope}%
\begin{pgfscope}%
\pgfpathrectangle{\pgfqpoint{0.688192in}{0.643904in}}{\pgfqpoint{6.200000in}{4.620000in}}%
\pgfusepath{clip}%
\pgfsetbuttcap%
\pgfsetroundjoin%
\pgfsetlinewidth{0.803000pt}%
\definecolor{currentstroke}{rgb}{0.690196,0.690196,0.690196}%
\pgfsetstrokecolor{currentstroke}%
\pgfsetstrokeopacity{0.200000}%
\pgfsetdash{{2.960000pt}{1.280000pt}}{0.000000pt}%
\pgfpathmoveto{\pgfqpoint{0.688192in}{1.441215in}}%
\pgfpathlineto{\pgfqpoint{6.888192in}{1.441215in}}%
\pgfusepath{stroke}%
\end{pgfscope}%
\begin{pgfscope}%
\pgfsetbuttcap%
\pgfsetroundjoin%
\definecolor{currentfill}{rgb}{0.000000,0.000000,0.000000}%
\pgfsetfillcolor{currentfill}%
\pgfsetlinewidth{0.602250pt}%
\definecolor{currentstroke}{rgb}{0.000000,0.000000,0.000000}%
\pgfsetstrokecolor{currentstroke}%
\pgfsetdash{}{0pt}%
\pgfsys@defobject{currentmarker}{\pgfqpoint{-0.027778in}{0.000000in}}{\pgfqpoint{-0.000000in}{0.000000in}}{%
\pgfpathmoveto{\pgfqpoint{-0.000000in}{0.000000in}}%
\pgfpathlineto{\pgfqpoint{-0.027778in}{0.000000in}}%
\pgfusepath{stroke,fill}%
}%
\begin{pgfscope}%
\pgfsys@transformshift{0.688192in}{1.441215in}%
\pgfsys@useobject{currentmarker}{}%
\end{pgfscope}%
\end{pgfscope}%
\begin{pgfscope}%
\pgfpathrectangle{\pgfqpoint{0.688192in}{0.643904in}}{\pgfqpoint{6.200000in}{4.620000in}}%
\pgfusepath{clip}%
\pgfsetbuttcap%
\pgfsetroundjoin%
\pgfsetlinewidth{0.803000pt}%
\definecolor{currentstroke}{rgb}{0.690196,0.690196,0.690196}%
\pgfsetstrokecolor{currentstroke}%
\pgfsetstrokeopacity{0.200000}%
\pgfsetdash{{2.960000pt}{1.280000pt}}{0.000000pt}%
\pgfpathmoveto{\pgfqpoint{0.688192in}{1.602764in}}%
\pgfpathlineto{\pgfqpoint{6.888192in}{1.602764in}}%
\pgfusepath{stroke}%
\end{pgfscope}%
\begin{pgfscope}%
\pgfsetbuttcap%
\pgfsetroundjoin%
\definecolor{currentfill}{rgb}{0.000000,0.000000,0.000000}%
\pgfsetfillcolor{currentfill}%
\pgfsetlinewidth{0.602250pt}%
\definecolor{currentstroke}{rgb}{0.000000,0.000000,0.000000}%
\pgfsetstrokecolor{currentstroke}%
\pgfsetdash{}{0pt}%
\pgfsys@defobject{currentmarker}{\pgfqpoint{-0.027778in}{0.000000in}}{\pgfqpoint{-0.000000in}{0.000000in}}{%
\pgfpathmoveto{\pgfqpoint{-0.000000in}{0.000000in}}%
\pgfpathlineto{\pgfqpoint{-0.027778in}{0.000000in}}%
\pgfusepath{stroke,fill}%
}%
\begin{pgfscope}%
\pgfsys@transformshift{0.688192in}{1.602764in}%
\pgfsys@useobject{currentmarker}{}%
\end{pgfscope}%
\end{pgfscope}%
\begin{pgfscope}%
\pgfpathrectangle{\pgfqpoint{0.688192in}{0.643904in}}{\pgfqpoint{6.200000in}{4.620000in}}%
\pgfusepath{clip}%
\pgfsetbuttcap%
\pgfsetroundjoin%
\pgfsetlinewidth{0.803000pt}%
\definecolor{currentstroke}{rgb}{0.690196,0.690196,0.690196}%
\pgfsetstrokecolor{currentstroke}%
\pgfsetstrokeopacity{0.200000}%
\pgfsetdash{{2.960000pt}{1.280000pt}}{0.000000pt}%
\pgfpathmoveto{\pgfqpoint{0.688192in}{1.764313in}}%
\pgfpathlineto{\pgfqpoint{6.888192in}{1.764313in}}%
\pgfusepath{stroke}%
\end{pgfscope}%
\begin{pgfscope}%
\pgfsetbuttcap%
\pgfsetroundjoin%
\definecolor{currentfill}{rgb}{0.000000,0.000000,0.000000}%
\pgfsetfillcolor{currentfill}%
\pgfsetlinewidth{0.602250pt}%
\definecolor{currentstroke}{rgb}{0.000000,0.000000,0.000000}%
\pgfsetstrokecolor{currentstroke}%
\pgfsetdash{}{0pt}%
\pgfsys@defobject{currentmarker}{\pgfqpoint{-0.027778in}{0.000000in}}{\pgfqpoint{-0.000000in}{0.000000in}}{%
\pgfpathmoveto{\pgfqpoint{-0.000000in}{0.000000in}}%
\pgfpathlineto{\pgfqpoint{-0.027778in}{0.000000in}}%
\pgfusepath{stroke,fill}%
}%
\begin{pgfscope}%
\pgfsys@transformshift{0.688192in}{1.764313in}%
\pgfsys@useobject{currentmarker}{}%
\end{pgfscope}%
\end{pgfscope}%
\begin{pgfscope}%
\pgfpathrectangle{\pgfqpoint{0.688192in}{0.643904in}}{\pgfqpoint{6.200000in}{4.620000in}}%
\pgfusepath{clip}%
\pgfsetbuttcap%
\pgfsetroundjoin%
\pgfsetlinewidth{0.803000pt}%
\definecolor{currentstroke}{rgb}{0.690196,0.690196,0.690196}%
\pgfsetstrokecolor{currentstroke}%
\pgfsetstrokeopacity{0.200000}%
\pgfsetdash{{2.960000pt}{1.280000pt}}{0.000000pt}%
\pgfpathmoveto{\pgfqpoint{0.688192in}{2.087411in}}%
\pgfpathlineto{\pgfqpoint{6.888192in}{2.087411in}}%
\pgfusepath{stroke}%
\end{pgfscope}%
\begin{pgfscope}%
\pgfsetbuttcap%
\pgfsetroundjoin%
\definecolor{currentfill}{rgb}{0.000000,0.000000,0.000000}%
\pgfsetfillcolor{currentfill}%
\pgfsetlinewidth{0.602250pt}%
\definecolor{currentstroke}{rgb}{0.000000,0.000000,0.000000}%
\pgfsetstrokecolor{currentstroke}%
\pgfsetdash{}{0pt}%
\pgfsys@defobject{currentmarker}{\pgfqpoint{-0.027778in}{0.000000in}}{\pgfqpoint{-0.000000in}{0.000000in}}{%
\pgfpathmoveto{\pgfqpoint{-0.000000in}{0.000000in}}%
\pgfpathlineto{\pgfqpoint{-0.027778in}{0.000000in}}%
\pgfusepath{stroke,fill}%
}%
\begin{pgfscope}%
\pgfsys@transformshift{0.688192in}{2.087411in}%
\pgfsys@useobject{currentmarker}{}%
\end{pgfscope}%
\end{pgfscope}%
\begin{pgfscope}%
\pgfpathrectangle{\pgfqpoint{0.688192in}{0.643904in}}{\pgfqpoint{6.200000in}{4.620000in}}%
\pgfusepath{clip}%
\pgfsetbuttcap%
\pgfsetroundjoin%
\pgfsetlinewidth{0.803000pt}%
\definecolor{currentstroke}{rgb}{0.690196,0.690196,0.690196}%
\pgfsetstrokecolor{currentstroke}%
\pgfsetstrokeopacity{0.200000}%
\pgfsetdash{{2.960000pt}{1.280000pt}}{0.000000pt}%
\pgfpathmoveto{\pgfqpoint{0.688192in}{2.248960in}}%
\pgfpathlineto{\pgfqpoint{6.888192in}{2.248960in}}%
\pgfusepath{stroke}%
\end{pgfscope}%
\begin{pgfscope}%
\pgfsetbuttcap%
\pgfsetroundjoin%
\definecolor{currentfill}{rgb}{0.000000,0.000000,0.000000}%
\pgfsetfillcolor{currentfill}%
\pgfsetlinewidth{0.602250pt}%
\definecolor{currentstroke}{rgb}{0.000000,0.000000,0.000000}%
\pgfsetstrokecolor{currentstroke}%
\pgfsetdash{}{0pt}%
\pgfsys@defobject{currentmarker}{\pgfqpoint{-0.027778in}{0.000000in}}{\pgfqpoint{-0.000000in}{0.000000in}}{%
\pgfpathmoveto{\pgfqpoint{-0.000000in}{0.000000in}}%
\pgfpathlineto{\pgfqpoint{-0.027778in}{0.000000in}}%
\pgfusepath{stroke,fill}%
}%
\begin{pgfscope}%
\pgfsys@transformshift{0.688192in}{2.248960in}%
\pgfsys@useobject{currentmarker}{}%
\end{pgfscope}%
\end{pgfscope}%
\begin{pgfscope}%
\pgfpathrectangle{\pgfqpoint{0.688192in}{0.643904in}}{\pgfqpoint{6.200000in}{4.620000in}}%
\pgfusepath{clip}%
\pgfsetbuttcap%
\pgfsetroundjoin%
\pgfsetlinewidth{0.803000pt}%
\definecolor{currentstroke}{rgb}{0.690196,0.690196,0.690196}%
\pgfsetstrokecolor{currentstroke}%
\pgfsetstrokeopacity{0.200000}%
\pgfsetdash{{2.960000pt}{1.280000pt}}{0.000000pt}%
\pgfpathmoveto{\pgfqpoint{0.688192in}{2.410508in}}%
\pgfpathlineto{\pgfqpoint{6.888192in}{2.410508in}}%
\pgfusepath{stroke}%
\end{pgfscope}%
\begin{pgfscope}%
\pgfsetbuttcap%
\pgfsetroundjoin%
\definecolor{currentfill}{rgb}{0.000000,0.000000,0.000000}%
\pgfsetfillcolor{currentfill}%
\pgfsetlinewidth{0.602250pt}%
\definecolor{currentstroke}{rgb}{0.000000,0.000000,0.000000}%
\pgfsetstrokecolor{currentstroke}%
\pgfsetdash{}{0pt}%
\pgfsys@defobject{currentmarker}{\pgfqpoint{-0.027778in}{0.000000in}}{\pgfqpoint{-0.000000in}{0.000000in}}{%
\pgfpathmoveto{\pgfqpoint{-0.000000in}{0.000000in}}%
\pgfpathlineto{\pgfqpoint{-0.027778in}{0.000000in}}%
\pgfusepath{stroke,fill}%
}%
\begin{pgfscope}%
\pgfsys@transformshift{0.688192in}{2.410508in}%
\pgfsys@useobject{currentmarker}{}%
\end{pgfscope}%
\end{pgfscope}%
\begin{pgfscope}%
\pgfpathrectangle{\pgfqpoint{0.688192in}{0.643904in}}{\pgfqpoint{6.200000in}{4.620000in}}%
\pgfusepath{clip}%
\pgfsetbuttcap%
\pgfsetroundjoin%
\pgfsetlinewidth{0.803000pt}%
\definecolor{currentstroke}{rgb}{0.690196,0.690196,0.690196}%
\pgfsetstrokecolor{currentstroke}%
\pgfsetstrokeopacity{0.200000}%
\pgfsetdash{{2.960000pt}{1.280000pt}}{0.000000pt}%
\pgfpathmoveto{\pgfqpoint{0.688192in}{2.572057in}}%
\pgfpathlineto{\pgfqpoint{6.888192in}{2.572057in}}%
\pgfusepath{stroke}%
\end{pgfscope}%
\begin{pgfscope}%
\pgfsetbuttcap%
\pgfsetroundjoin%
\definecolor{currentfill}{rgb}{0.000000,0.000000,0.000000}%
\pgfsetfillcolor{currentfill}%
\pgfsetlinewidth{0.602250pt}%
\definecolor{currentstroke}{rgb}{0.000000,0.000000,0.000000}%
\pgfsetstrokecolor{currentstroke}%
\pgfsetdash{}{0pt}%
\pgfsys@defobject{currentmarker}{\pgfqpoint{-0.027778in}{0.000000in}}{\pgfqpoint{-0.000000in}{0.000000in}}{%
\pgfpathmoveto{\pgfqpoint{-0.000000in}{0.000000in}}%
\pgfpathlineto{\pgfqpoint{-0.027778in}{0.000000in}}%
\pgfusepath{stroke,fill}%
}%
\begin{pgfscope}%
\pgfsys@transformshift{0.688192in}{2.572057in}%
\pgfsys@useobject{currentmarker}{}%
\end{pgfscope}%
\end{pgfscope}%
\begin{pgfscope}%
\pgfpathrectangle{\pgfqpoint{0.688192in}{0.643904in}}{\pgfqpoint{6.200000in}{4.620000in}}%
\pgfusepath{clip}%
\pgfsetbuttcap%
\pgfsetroundjoin%
\pgfsetlinewidth{0.803000pt}%
\definecolor{currentstroke}{rgb}{0.690196,0.690196,0.690196}%
\pgfsetstrokecolor{currentstroke}%
\pgfsetstrokeopacity{0.200000}%
\pgfsetdash{{2.960000pt}{1.280000pt}}{0.000000pt}%
\pgfpathmoveto{\pgfqpoint{0.688192in}{2.895155in}}%
\pgfpathlineto{\pgfqpoint{6.888192in}{2.895155in}}%
\pgfusepath{stroke}%
\end{pgfscope}%
\begin{pgfscope}%
\pgfsetbuttcap%
\pgfsetroundjoin%
\definecolor{currentfill}{rgb}{0.000000,0.000000,0.000000}%
\pgfsetfillcolor{currentfill}%
\pgfsetlinewidth{0.602250pt}%
\definecolor{currentstroke}{rgb}{0.000000,0.000000,0.000000}%
\pgfsetstrokecolor{currentstroke}%
\pgfsetdash{}{0pt}%
\pgfsys@defobject{currentmarker}{\pgfqpoint{-0.027778in}{0.000000in}}{\pgfqpoint{-0.000000in}{0.000000in}}{%
\pgfpathmoveto{\pgfqpoint{-0.000000in}{0.000000in}}%
\pgfpathlineto{\pgfqpoint{-0.027778in}{0.000000in}}%
\pgfusepath{stroke,fill}%
}%
\begin{pgfscope}%
\pgfsys@transformshift{0.688192in}{2.895155in}%
\pgfsys@useobject{currentmarker}{}%
\end{pgfscope}%
\end{pgfscope}%
\begin{pgfscope}%
\pgfpathrectangle{\pgfqpoint{0.688192in}{0.643904in}}{\pgfqpoint{6.200000in}{4.620000in}}%
\pgfusepath{clip}%
\pgfsetbuttcap%
\pgfsetroundjoin%
\pgfsetlinewidth{0.803000pt}%
\definecolor{currentstroke}{rgb}{0.690196,0.690196,0.690196}%
\pgfsetstrokecolor{currentstroke}%
\pgfsetstrokeopacity{0.200000}%
\pgfsetdash{{2.960000pt}{1.280000pt}}{0.000000pt}%
\pgfpathmoveto{\pgfqpoint{0.688192in}{3.056704in}}%
\pgfpathlineto{\pgfqpoint{6.888192in}{3.056704in}}%
\pgfusepath{stroke}%
\end{pgfscope}%
\begin{pgfscope}%
\pgfsetbuttcap%
\pgfsetroundjoin%
\definecolor{currentfill}{rgb}{0.000000,0.000000,0.000000}%
\pgfsetfillcolor{currentfill}%
\pgfsetlinewidth{0.602250pt}%
\definecolor{currentstroke}{rgb}{0.000000,0.000000,0.000000}%
\pgfsetstrokecolor{currentstroke}%
\pgfsetdash{}{0pt}%
\pgfsys@defobject{currentmarker}{\pgfqpoint{-0.027778in}{0.000000in}}{\pgfqpoint{-0.000000in}{0.000000in}}{%
\pgfpathmoveto{\pgfqpoint{-0.000000in}{0.000000in}}%
\pgfpathlineto{\pgfqpoint{-0.027778in}{0.000000in}}%
\pgfusepath{stroke,fill}%
}%
\begin{pgfscope}%
\pgfsys@transformshift{0.688192in}{3.056704in}%
\pgfsys@useobject{currentmarker}{}%
\end{pgfscope}%
\end{pgfscope}%
\begin{pgfscope}%
\pgfpathrectangle{\pgfqpoint{0.688192in}{0.643904in}}{\pgfqpoint{6.200000in}{4.620000in}}%
\pgfusepath{clip}%
\pgfsetbuttcap%
\pgfsetroundjoin%
\pgfsetlinewidth{0.803000pt}%
\definecolor{currentstroke}{rgb}{0.690196,0.690196,0.690196}%
\pgfsetstrokecolor{currentstroke}%
\pgfsetstrokeopacity{0.200000}%
\pgfsetdash{{2.960000pt}{1.280000pt}}{0.000000pt}%
\pgfpathmoveto{\pgfqpoint{0.688192in}{3.218253in}}%
\pgfpathlineto{\pgfqpoint{6.888192in}{3.218253in}}%
\pgfusepath{stroke}%
\end{pgfscope}%
\begin{pgfscope}%
\pgfsetbuttcap%
\pgfsetroundjoin%
\definecolor{currentfill}{rgb}{0.000000,0.000000,0.000000}%
\pgfsetfillcolor{currentfill}%
\pgfsetlinewidth{0.602250pt}%
\definecolor{currentstroke}{rgb}{0.000000,0.000000,0.000000}%
\pgfsetstrokecolor{currentstroke}%
\pgfsetdash{}{0pt}%
\pgfsys@defobject{currentmarker}{\pgfqpoint{-0.027778in}{0.000000in}}{\pgfqpoint{-0.000000in}{0.000000in}}{%
\pgfpathmoveto{\pgfqpoint{-0.000000in}{0.000000in}}%
\pgfpathlineto{\pgfqpoint{-0.027778in}{0.000000in}}%
\pgfusepath{stroke,fill}%
}%
\begin{pgfscope}%
\pgfsys@transformshift{0.688192in}{3.218253in}%
\pgfsys@useobject{currentmarker}{}%
\end{pgfscope}%
\end{pgfscope}%
\begin{pgfscope}%
\pgfpathrectangle{\pgfqpoint{0.688192in}{0.643904in}}{\pgfqpoint{6.200000in}{4.620000in}}%
\pgfusepath{clip}%
\pgfsetbuttcap%
\pgfsetroundjoin%
\pgfsetlinewidth{0.803000pt}%
\definecolor{currentstroke}{rgb}{0.690196,0.690196,0.690196}%
\pgfsetstrokecolor{currentstroke}%
\pgfsetstrokeopacity{0.200000}%
\pgfsetdash{{2.960000pt}{1.280000pt}}{0.000000pt}%
\pgfpathmoveto{\pgfqpoint{0.688192in}{3.379801in}}%
\pgfpathlineto{\pgfqpoint{6.888192in}{3.379801in}}%
\pgfusepath{stroke}%
\end{pgfscope}%
\begin{pgfscope}%
\pgfsetbuttcap%
\pgfsetroundjoin%
\definecolor{currentfill}{rgb}{0.000000,0.000000,0.000000}%
\pgfsetfillcolor{currentfill}%
\pgfsetlinewidth{0.602250pt}%
\definecolor{currentstroke}{rgb}{0.000000,0.000000,0.000000}%
\pgfsetstrokecolor{currentstroke}%
\pgfsetdash{}{0pt}%
\pgfsys@defobject{currentmarker}{\pgfqpoint{-0.027778in}{0.000000in}}{\pgfqpoint{-0.000000in}{0.000000in}}{%
\pgfpathmoveto{\pgfqpoint{-0.000000in}{0.000000in}}%
\pgfpathlineto{\pgfqpoint{-0.027778in}{0.000000in}}%
\pgfusepath{stroke,fill}%
}%
\begin{pgfscope}%
\pgfsys@transformshift{0.688192in}{3.379801in}%
\pgfsys@useobject{currentmarker}{}%
\end{pgfscope}%
\end{pgfscope}%
\begin{pgfscope}%
\pgfpathrectangle{\pgfqpoint{0.688192in}{0.643904in}}{\pgfqpoint{6.200000in}{4.620000in}}%
\pgfusepath{clip}%
\pgfsetbuttcap%
\pgfsetroundjoin%
\pgfsetlinewidth{0.803000pt}%
\definecolor{currentstroke}{rgb}{0.690196,0.690196,0.690196}%
\pgfsetstrokecolor{currentstroke}%
\pgfsetstrokeopacity{0.200000}%
\pgfsetdash{{2.960000pt}{1.280000pt}}{0.000000pt}%
\pgfpathmoveto{\pgfqpoint{0.688192in}{3.702899in}}%
\pgfpathlineto{\pgfqpoint{6.888192in}{3.702899in}}%
\pgfusepath{stroke}%
\end{pgfscope}%
\begin{pgfscope}%
\pgfsetbuttcap%
\pgfsetroundjoin%
\definecolor{currentfill}{rgb}{0.000000,0.000000,0.000000}%
\pgfsetfillcolor{currentfill}%
\pgfsetlinewidth{0.602250pt}%
\definecolor{currentstroke}{rgb}{0.000000,0.000000,0.000000}%
\pgfsetstrokecolor{currentstroke}%
\pgfsetdash{}{0pt}%
\pgfsys@defobject{currentmarker}{\pgfqpoint{-0.027778in}{0.000000in}}{\pgfqpoint{-0.000000in}{0.000000in}}{%
\pgfpathmoveto{\pgfqpoint{-0.000000in}{0.000000in}}%
\pgfpathlineto{\pgfqpoint{-0.027778in}{0.000000in}}%
\pgfusepath{stroke,fill}%
}%
\begin{pgfscope}%
\pgfsys@transformshift{0.688192in}{3.702899in}%
\pgfsys@useobject{currentmarker}{}%
\end{pgfscope}%
\end{pgfscope}%
\begin{pgfscope}%
\pgfpathrectangle{\pgfqpoint{0.688192in}{0.643904in}}{\pgfqpoint{6.200000in}{4.620000in}}%
\pgfusepath{clip}%
\pgfsetbuttcap%
\pgfsetroundjoin%
\pgfsetlinewidth{0.803000pt}%
\definecolor{currentstroke}{rgb}{0.690196,0.690196,0.690196}%
\pgfsetstrokecolor{currentstroke}%
\pgfsetstrokeopacity{0.200000}%
\pgfsetdash{{2.960000pt}{1.280000pt}}{0.000000pt}%
\pgfpathmoveto{\pgfqpoint{0.688192in}{3.864448in}}%
\pgfpathlineto{\pgfqpoint{6.888192in}{3.864448in}}%
\pgfusepath{stroke}%
\end{pgfscope}%
\begin{pgfscope}%
\pgfsetbuttcap%
\pgfsetroundjoin%
\definecolor{currentfill}{rgb}{0.000000,0.000000,0.000000}%
\pgfsetfillcolor{currentfill}%
\pgfsetlinewidth{0.602250pt}%
\definecolor{currentstroke}{rgb}{0.000000,0.000000,0.000000}%
\pgfsetstrokecolor{currentstroke}%
\pgfsetdash{}{0pt}%
\pgfsys@defobject{currentmarker}{\pgfqpoint{-0.027778in}{0.000000in}}{\pgfqpoint{-0.000000in}{0.000000in}}{%
\pgfpathmoveto{\pgfqpoint{-0.000000in}{0.000000in}}%
\pgfpathlineto{\pgfqpoint{-0.027778in}{0.000000in}}%
\pgfusepath{stroke,fill}%
}%
\begin{pgfscope}%
\pgfsys@transformshift{0.688192in}{3.864448in}%
\pgfsys@useobject{currentmarker}{}%
\end{pgfscope}%
\end{pgfscope}%
\begin{pgfscope}%
\pgfpathrectangle{\pgfqpoint{0.688192in}{0.643904in}}{\pgfqpoint{6.200000in}{4.620000in}}%
\pgfusepath{clip}%
\pgfsetbuttcap%
\pgfsetroundjoin%
\pgfsetlinewidth{0.803000pt}%
\definecolor{currentstroke}{rgb}{0.690196,0.690196,0.690196}%
\pgfsetstrokecolor{currentstroke}%
\pgfsetstrokeopacity{0.200000}%
\pgfsetdash{{2.960000pt}{1.280000pt}}{0.000000pt}%
\pgfpathmoveto{\pgfqpoint{0.688192in}{4.025997in}}%
\pgfpathlineto{\pgfqpoint{6.888192in}{4.025997in}}%
\pgfusepath{stroke}%
\end{pgfscope}%
\begin{pgfscope}%
\pgfsetbuttcap%
\pgfsetroundjoin%
\definecolor{currentfill}{rgb}{0.000000,0.000000,0.000000}%
\pgfsetfillcolor{currentfill}%
\pgfsetlinewidth{0.602250pt}%
\definecolor{currentstroke}{rgb}{0.000000,0.000000,0.000000}%
\pgfsetstrokecolor{currentstroke}%
\pgfsetdash{}{0pt}%
\pgfsys@defobject{currentmarker}{\pgfqpoint{-0.027778in}{0.000000in}}{\pgfqpoint{-0.000000in}{0.000000in}}{%
\pgfpathmoveto{\pgfqpoint{-0.000000in}{0.000000in}}%
\pgfpathlineto{\pgfqpoint{-0.027778in}{0.000000in}}%
\pgfusepath{stroke,fill}%
}%
\begin{pgfscope}%
\pgfsys@transformshift{0.688192in}{4.025997in}%
\pgfsys@useobject{currentmarker}{}%
\end{pgfscope}%
\end{pgfscope}%
\begin{pgfscope}%
\pgfpathrectangle{\pgfqpoint{0.688192in}{0.643904in}}{\pgfqpoint{6.200000in}{4.620000in}}%
\pgfusepath{clip}%
\pgfsetbuttcap%
\pgfsetroundjoin%
\pgfsetlinewidth{0.803000pt}%
\definecolor{currentstroke}{rgb}{0.690196,0.690196,0.690196}%
\pgfsetstrokecolor{currentstroke}%
\pgfsetstrokeopacity{0.200000}%
\pgfsetdash{{2.960000pt}{1.280000pt}}{0.000000pt}%
\pgfpathmoveto{\pgfqpoint{0.688192in}{4.187545in}}%
\pgfpathlineto{\pgfqpoint{6.888192in}{4.187545in}}%
\pgfusepath{stroke}%
\end{pgfscope}%
\begin{pgfscope}%
\pgfsetbuttcap%
\pgfsetroundjoin%
\definecolor{currentfill}{rgb}{0.000000,0.000000,0.000000}%
\pgfsetfillcolor{currentfill}%
\pgfsetlinewidth{0.602250pt}%
\definecolor{currentstroke}{rgb}{0.000000,0.000000,0.000000}%
\pgfsetstrokecolor{currentstroke}%
\pgfsetdash{}{0pt}%
\pgfsys@defobject{currentmarker}{\pgfqpoint{-0.027778in}{0.000000in}}{\pgfqpoint{-0.000000in}{0.000000in}}{%
\pgfpathmoveto{\pgfqpoint{-0.000000in}{0.000000in}}%
\pgfpathlineto{\pgfqpoint{-0.027778in}{0.000000in}}%
\pgfusepath{stroke,fill}%
}%
\begin{pgfscope}%
\pgfsys@transformshift{0.688192in}{4.187545in}%
\pgfsys@useobject{currentmarker}{}%
\end{pgfscope}%
\end{pgfscope}%
\begin{pgfscope}%
\pgfpathrectangle{\pgfqpoint{0.688192in}{0.643904in}}{\pgfqpoint{6.200000in}{4.620000in}}%
\pgfusepath{clip}%
\pgfsetbuttcap%
\pgfsetroundjoin%
\pgfsetlinewidth{0.803000pt}%
\definecolor{currentstroke}{rgb}{0.690196,0.690196,0.690196}%
\pgfsetstrokecolor{currentstroke}%
\pgfsetstrokeopacity{0.200000}%
\pgfsetdash{{2.960000pt}{1.280000pt}}{0.000000pt}%
\pgfpathmoveto{\pgfqpoint{0.688192in}{4.510643in}}%
\pgfpathlineto{\pgfqpoint{6.888192in}{4.510643in}}%
\pgfusepath{stroke}%
\end{pgfscope}%
\begin{pgfscope}%
\pgfsetbuttcap%
\pgfsetroundjoin%
\definecolor{currentfill}{rgb}{0.000000,0.000000,0.000000}%
\pgfsetfillcolor{currentfill}%
\pgfsetlinewidth{0.602250pt}%
\definecolor{currentstroke}{rgb}{0.000000,0.000000,0.000000}%
\pgfsetstrokecolor{currentstroke}%
\pgfsetdash{}{0pt}%
\pgfsys@defobject{currentmarker}{\pgfqpoint{-0.027778in}{0.000000in}}{\pgfqpoint{-0.000000in}{0.000000in}}{%
\pgfpathmoveto{\pgfqpoint{-0.000000in}{0.000000in}}%
\pgfpathlineto{\pgfqpoint{-0.027778in}{0.000000in}}%
\pgfusepath{stroke,fill}%
}%
\begin{pgfscope}%
\pgfsys@transformshift{0.688192in}{4.510643in}%
\pgfsys@useobject{currentmarker}{}%
\end{pgfscope}%
\end{pgfscope}%
\begin{pgfscope}%
\pgfpathrectangle{\pgfqpoint{0.688192in}{0.643904in}}{\pgfqpoint{6.200000in}{4.620000in}}%
\pgfusepath{clip}%
\pgfsetbuttcap%
\pgfsetroundjoin%
\pgfsetlinewidth{0.803000pt}%
\definecolor{currentstroke}{rgb}{0.690196,0.690196,0.690196}%
\pgfsetstrokecolor{currentstroke}%
\pgfsetstrokeopacity{0.200000}%
\pgfsetdash{{2.960000pt}{1.280000pt}}{0.000000pt}%
\pgfpathmoveto{\pgfqpoint{0.688192in}{4.672192in}}%
\pgfpathlineto{\pgfqpoint{6.888192in}{4.672192in}}%
\pgfusepath{stroke}%
\end{pgfscope}%
\begin{pgfscope}%
\pgfsetbuttcap%
\pgfsetroundjoin%
\definecolor{currentfill}{rgb}{0.000000,0.000000,0.000000}%
\pgfsetfillcolor{currentfill}%
\pgfsetlinewidth{0.602250pt}%
\definecolor{currentstroke}{rgb}{0.000000,0.000000,0.000000}%
\pgfsetstrokecolor{currentstroke}%
\pgfsetdash{}{0pt}%
\pgfsys@defobject{currentmarker}{\pgfqpoint{-0.027778in}{0.000000in}}{\pgfqpoint{-0.000000in}{0.000000in}}{%
\pgfpathmoveto{\pgfqpoint{-0.000000in}{0.000000in}}%
\pgfpathlineto{\pgfqpoint{-0.027778in}{0.000000in}}%
\pgfusepath{stroke,fill}%
}%
\begin{pgfscope}%
\pgfsys@transformshift{0.688192in}{4.672192in}%
\pgfsys@useobject{currentmarker}{}%
\end{pgfscope}%
\end{pgfscope}%
\begin{pgfscope}%
\pgfpathrectangle{\pgfqpoint{0.688192in}{0.643904in}}{\pgfqpoint{6.200000in}{4.620000in}}%
\pgfusepath{clip}%
\pgfsetbuttcap%
\pgfsetroundjoin%
\pgfsetlinewidth{0.803000pt}%
\definecolor{currentstroke}{rgb}{0.690196,0.690196,0.690196}%
\pgfsetstrokecolor{currentstroke}%
\pgfsetstrokeopacity{0.200000}%
\pgfsetdash{{2.960000pt}{1.280000pt}}{0.000000pt}%
\pgfpathmoveto{\pgfqpoint{0.688192in}{4.833741in}}%
\pgfpathlineto{\pgfqpoint{6.888192in}{4.833741in}}%
\pgfusepath{stroke}%
\end{pgfscope}%
\begin{pgfscope}%
\pgfsetbuttcap%
\pgfsetroundjoin%
\definecolor{currentfill}{rgb}{0.000000,0.000000,0.000000}%
\pgfsetfillcolor{currentfill}%
\pgfsetlinewidth{0.602250pt}%
\definecolor{currentstroke}{rgb}{0.000000,0.000000,0.000000}%
\pgfsetstrokecolor{currentstroke}%
\pgfsetdash{}{0pt}%
\pgfsys@defobject{currentmarker}{\pgfqpoint{-0.027778in}{0.000000in}}{\pgfqpoint{-0.000000in}{0.000000in}}{%
\pgfpathmoveto{\pgfqpoint{-0.000000in}{0.000000in}}%
\pgfpathlineto{\pgfqpoint{-0.027778in}{0.000000in}}%
\pgfusepath{stroke,fill}%
}%
\begin{pgfscope}%
\pgfsys@transformshift{0.688192in}{4.833741in}%
\pgfsys@useobject{currentmarker}{}%
\end{pgfscope}%
\end{pgfscope}%
\begin{pgfscope}%
\pgfpathrectangle{\pgfqpoint{0.688192in}{0.643904in}}{\pgfqpoint{6.200000in}{4.620000in}}%
\pgfusepath{clip}%
\pgfsetbuttcap%
\pgfsetroundjoin%
\pgfsetlinewidth{0.803000pt}%
\definecolor{currentstroke}{rgb}{0.690196,0.690196,0.690196}%
\pgfsetstrokecolor{currentstroke}%
\pgfsetstrokeopacity{0.200000}%
\pgfsetdash{{2.960000pt}{1.280000pt}}{0.000000pt}%
\pgfpathmoveto{\pgfqpoint{0.688192in}{4.995290in}}%
\pgfpathlineto{\pgfqpoint{6.888192in}{4.995290in}}%
\pgfusepath{stroke}%
\end{pgfscope}%
\begin{pgfscope}%
\pgfsetbuttcap%
\pgfsetroundjoin%
\definecolor{currentfill}{rgb}{0.000000,0.000000,0.000000}%
\pgfsetfillcolor{currentfill}%
\pgfsetlinewidth{0.602250pt}%
\definecolor{currentstroke}{rgb}{0.000000,0.000000,0.000000}%
\pgfsetstrokecolor{currentstroke}%
\pgfsetdash{}{0pt}%
\pgfsys@defobject{currentmarker}{\pgfqpoint{-0.027778in}{0.000000in}}{\pgfqpoint{-0.000000in}{0.000000in}}{%
\pgfpathmoveto{\pgfqpoint{-0.000000in}{0.000000in}}%
\pgfpathlineto{\pgfqpoint{-0.027778in}{0.000000in}}%
\pgfusepath{stroke,fill}%
}%
\begin{pgfscope}%
\pgfsys@transformshift{0.688192in}{4.995290in}%
\pgfsys@useobject{currentmarker}{}%
\end{pgfscope}%
\end{pgfscope}%
\begin{pgfscope}%
\definecolor{textcolor}{rgb}{0.000000,0.000000,0.000000}%
\pgfsetstrokecolor{textcolor}%
\pgfsetfillcolor{textcolor}%
\pgftext[x=0.339583in,y=2.953904in,,bottom,rotate=90.000000]{\color{textcolor}{\rmfamily\fontsize{18.000000}{21.600000}\selectfont\catcode`\^=\active\def^{\ifmmode\sp\else\^{}\fi}\catcode`\%=\active\def%{\%}Time [seconds]}}%
\end{pgfscope}%
\begin{pgfscope}%
\pgfpathrectangle{\pgfqpoint{0.688192in}{0.643904in}}{\pgfqpoint{6.200000in}{4.620000in}}%
\pgfusepath{clip}%
\pgfsetrectcap%
\pgfsetroundjoin%
\pgfsetlinewidth{1.505625pt}%
\definecolor{currentstroke}{rgb}{0.007843,0.243137,1.000000}%
\pgfsetstrokecolor{currentstroke}%
\pgfsetdash{}{0pt}%
\pgfpathmoveto{\pgfqpoint{0.688192in}{0.944593in}}%
\pgfpathlineto{\pgfqpoint{1.573906in}{1.613312in}}%
\pgfpathlineto{\pgfqpoint{3.345334in}{2.774081in}}%
\pgfpathlineto{\pgfqpoint{6.888192in}{5.053904in}}%
\pgfusepath{stroke}%
\end{pgfscope}%
\begin{pgfscope}%
\pgfpathrectangle{\pgfqpoint{0.688192in}{0.643904in}}{\pgfqpoint{6.200000in}{4.620000in}}%
\pgfusepath{clip}%
\pgfsetbuttcap%
\pgfsetroundjoin%
\definecolor{currentfill}{rgb}{0.007843,0.243137,1.000000}%
\pgfsetfillcolor{currentfill}%
\pgfsetlinewidth{0.752812pt}%
\definecolor{currentstroke}{rgb}{1.000000,1.000000,1.000000}%
\pgfsetstrokecolor{currentstroke}%
\pgfsetdash{}{0pt}%
\pgfsys@defobject{currentmarker}{\pgfqpoint{-0.041667in}{-0.041667in}}{\pgfqpoint{0.041667in}{0.041667in}}{%
\pgfpathmoveto{\pgfqpoint{0.000000in}{-0.041667in}}%
\pgfpathcurveto{\pgfqpoint{0.011050in}{-0.041667in}}{\pgfqpoint{0.021649in}{-0.037276in}}{\pgfqpoint{0.029463in}{-0.029463in}}%
\pgfpathcurveto{\pgfqpoint{0.037276in}{-0.021649in}}{\pgfqpoint{0.041667in}{-0.011050in}}{\pgfqpoint{0.041667in}{0.000000in}}%
\pgfpathcurveto{\pgfqpoint{0.041667in}{0.011050in}}{\pgfqpoint{0.037276in}{0.021649in}}{\pgfqpoint{0.029463in}{0.029463in}}%
\pgfpathcurveto{\pgfqpoint{0.021649in}{0.037276in}}{\pgfqpoint{0.011050in}{0.041667in}}{\pgfqpoint{0.000000in}{0.041667in}}%
\pgfpathcurveto{\pgfqpoint{-0.011050in}{0.041667in}}{\pgfqpoint{-0.021649in}{0.037276in}}{\pgfqpoint{-0.029463in}{0.029463in}}%
\pgfpathcurveto{\pgfqpoint{-0.037276in}{0.021649in}}{\pgfqpoint{-0.041667in}{0.011050in}}{\pgfqpoint{-0.041667in}{0.000000in}}%
\pgfpathcurveto{\pgfqpoint{-0.041667in}{-0.011050in}}{\pgfqpoint{-0.037276in}{-0.021649in}}{\pgfqpoint{-0.029463in}{-0.029463in}}%
\pgfpathcurveto{\pgfqpoint{-0.021649in}{-0.037276in}}{\pgfqpoint{-0.011050in}{-0.041667in}}{\pgfqpoint{0.000000in}{-0.041667in}}%
\pgfpathlineto{\pgfqpoint{0.000000in}{-0.041667in}}%
\pgfpathclose%
\pgfusepath{stroke,fill}%
}%
\begin{pgfscope}%
\pgfsys@transformshift{0.688192in}{0.944593in}%
\pgfsys@useobject{currentmarker}{}%
\end{pgfscope}%
\begin{pgfscope}%
\pgfsys@transformshift{1.573906in}{1.613312in}%
\pgfsys@useobject{currentmarker}{}%
\end{pgfscope}%
\begin{pgfscope}%
\pgfsys@transformshift{3.345334in}{2.774081in}%
\pgfsys@useobject{currentmarker}{}%
\end{pgfscope}%
\begin{pgfscope}%
\pgfsys@transformshift{6.888192in}{5.053904in}%
\pgfsys@useobject{currentmarker}{}%
\end{pgfscope}%
\end{pgfscope}%
\begin{pgfscope}%
\pgfpathrectangle{\pgfqpoint{0.688192in}{0.643904in}}{\pgfqpoint{6.200000in}{4.620000in}}%
\pgfusepath{clip}%
\pgfsetbuttcap%
\pgfsetroundjoin%
\pgfsetlinewidth{1.505625pt}%
\definecolor{currentstroke}{rgb}{1.000000,0.486275,0.000000}%
\pgfsetstrokecolor{currentstroke}%
\pgfsetdash{{6.000000pt}{2.250000pt}}{0.000000pt}%
\pgfpathmoveto{\pgfqpoint{0.688192in}{0.937993in}}%
\pgfpathlineto{\pgfqpoint{1.573906in}{1.570977in}}%
\pgfpathlineto{\pgfqpoint{3.345334in}{2.596766in}}%
\pgfpathlineto{\pgfqpoint{6.888192in}{4.725093in}}%
\pgfusepath{stroke}%
\end{pgfscope}%
\begin{pgfscope}%
\pgfpathrectangle{\pgfqpoint{0.688192in}{0.643904in}}{\pgfqpoint{6.200000in}{4.620000in}}%
\pgfusepath{clip}%
\pgfsetbuttcap%
\pgfsetroundjoin%
\definecolor{currentfill}{rgb}{1.000000,0.486275,0.000000}%
\pgfsetfillcolor{currentfill}%
\pgfsetlinewidth{0.752812pt}%
\definecolor{currentstroke}{rgb}{1.000000,1.000000,1.000000}%
\pgfsetstrokecolor{currentstroke}%
\pgfsetdash{}{0pt}%
\pgfsys@defobject{currentmarker}{\pgfqpoint{-0.041667in}{-0.041667in}}{\pgfqpoint{0.041667in}{0.041667in}}{%
\pgfpathmoveto{\pgfqpoint{0.000000in}{-0.041667in}}%
\pgfpathcurveto{\pgfqpoint{0.011050in}{-0.041667in}}{\pgfqpoint{0.021649in}{-0.037276in}}{\pgfqpoint{0.029463in}{-0.029463in}}%
\pgfpathcurveto{\pgfqpoint{0.037276in}{-0.021649in}}{\pgfqpoint{0.041667in}{-0.011050in}}{\pgfqpoint{0.041667in}{0.000000in}}%
\pgfpathcurveto{\pgfqpoint{0.041667in}{0.011050in}}{\pgfqpoint{0.037276in}{0.021649in}}{\pgfqpoint{0.029463in}{0.029463in}}%
\pgfpathcurveto{\pgfqpoint{0.021649in}{0.037276in}}{\pgfqpoint{0.011050in}{0.041667in}}{\pgfqpoint{0.000000in}{0.041667in}}%
\pgfpathcurveto{\pgfqpoint{-0.011050in}{0.041667in}}{\pgfqpoint{-0.021649in}{0.037276in}}{\pgfqpoint{-0.029463in}{0.029463in}}%
\pgfpathcurveto{\pgfqpoint{-0.037276in}{0.021649in}}{\pgfqpoint{-0.041667in}{0.011050in}}{\pgfqpoint{-0.041667in}{0.000000in}}%
\pgfpathcurveto{\pgfqpoint{-0.041667in}{-0.011050in}}{\pgfqpoint{-0.037276in}{-0.021649in}}{\pgfqpoint{-0.029463in}{-0.029463in}}%
\pgfpathcurveto{\pgfqpoint{-0.021649in}{-0.037276in}}{\pgfqpoint{-0.011050in}{-0.041667in}}{\pgfqpoint{0.000000in}{-0.041667in}}%
\pgfpathlineto{\pgfqpoint{0.000000in}{-0.041667in}}%
\pgfpathclose%
\pgfusepath{stroke,fill}%
}%
\begin{pgfscope}%
\pgfsys@transformshift{0.688192in}{0.937993in}%
\pgfsys@useobject{currentmarker}{}%
\end{pgfscope}%
\begin{pgfscope}%
\pgfsys@transformshift{1.573906in}{1.570977in}%
\pgfsys@useobject{currentmarker}{}%
\end{pgfscope}%
\begin{pgfscope}%
\pgfsys@transformshift{3.345334in}{2.596766in}%
\pgfsys@useobject{currentmarker}{}%
\end{pgfscope}%
\begin{pgfscope}%
\pgfsys@transformshift{6.888192in}{4.725093in}%
\pgfsys@useobject{currentmarker}{}%
\end{pgfscope}%
\end{pgfscope}%
\begin{pgfscope}%
\pgfpathrectangle{\pgfqpoint{0.688192in}{0.643904in}}{\pgfqpoint{6.200000in}{4.620000in}}%
\pgfusepath{clip}%
\pgfsetbuttcap%
\pgfsetroundjoin%
\pgfsetlinewidth{1.505625pt}%
\definecolor{currentstroke}{rgb}{0.101961,0.788235,0.219608}%
\pgfsetstrokecolor{currentstroke}%
\pgfsetdash{{1.500000pt}{1.500000pt}}{0.000000pt}%
\pgfpathmoveto{\pgfqpoint{0.688192in}{0.853904in}}%
\pgfpathlineto{\pgfqpoint{1.573906in}{1.519257in}}%
\pgfpathlineto{\pgfqpoint{3.345334in}{2.702531in}}%
\pgfpathlineto{\pgfqpoint{6.888192in}{4.463327in}}%
\pgfusepath{stroke}%
\end{pgfscope}%
\begin{pgfscope}%
\pgfpathrectangle{\pgfqpoint{0.688192in}{0.643904in}}{\pgfqpoint{6.200000in}{4.620000in}}%
\pgfusepath{clip}%
\pgfsetbuttcap%
\pgfsetroundjoin%
\definecolor{currentfill}{rgb}{0.101961,0.788235,0.219608}%
\pgfsetfillcolor{currentfill}%
\pgfsetlinewidth{0.752812pt}%
\definecolor{currentstroke}{rgb}{1.000000,1.000000,1.000000}%
\pgfsetstrokecolor{currentstroke}%
\pgfsetdash{}{0pt}%
\pgfsys@defobject{currentmarker}{\pgfqpoint{-0.041667in}{-0.041667in}}{\pgfqpoint{0.041667in}{0.041667in}}{%
\pgfpathmoveto{\pgfqpoint{0.000000in}{-0.041667in}}%
\pgfpathcurveto{\pgfqpoint{0.011050in}{-0.041667in}}{\pgfqpoint{0.021649in}{-0.037276in}}{\pgfqpoint{0.029463in}{-0.029463in}}%
\pgfpathcurveto{\pgfqpoint{0.037276in}{-0.021649in}}{\pgfqpoint{0.041667in}{-0.011050in}}{\pgfqpoint{0.041667in}{0.000000in}}%
\pgfpathcurveto{\pgfqpoint{0.041667in}{0.011050in}}{\pgfqpoint{0.037276in}{0.021649in}}{\pgfqpoint{0.029463in}{0.029463in}}%
\pgfpathcurveto{\pgfqpoint{0.021649in}{0.037276in}}{\pgfqpoint{0.011050in}{0.041667in}}{\pgfqpoint{0.000000in}{0.041667in}}%
\pgfpathcurveto{\pgfqpoint{-0.011050in}{0.041667in}}{\pgfqpoint{-0.021649in}{0.037276in}}{\pgfqpoint{-0.029463in}{0.029463in}}%
\pgfpathcurveto{\pgfqpoint{-0.037276in}{0.021649in}}{\pgfqpoint{-0.041667in}{0.011050in}}{\pgfqpoint{-0.041667in}{0.000000in}}%
\pgfpathcurveto{\pgfqpoint{-0.041667in}{-0.011050in}}{\pgfqpoint{-0.037276in}{-0.021649in}}{\pgfqpoint{-0.029463in}{-0.029463in}}%
\pgfpathcurveto{\pgfqpoint{-0.021649in}{-0.037276in}}{\pgfqpoint{-0.011050in}{-0.041667in}}{\pgfqpoint{0.000000in}{-0.041667in}}%
\pgfpathlineto{\pgfqpoint{0.000000in}{-0.041667in}}%
\pgfpathclose%
\pgfusepath{stroke,fill}%
}%
\begin{pgfscope}%
\pgfsys@transformshift{0.688192in}{0.853904in}%
\pgfsys@useobject{currentmarker}{}%
\end{pgfscope}%
\begin{pgfscope}%
\pgfsys@transformshift{1.573906in}{1.519257in}%
\pgfsys@useobject{currentmarker}{}%
\end{pgfscope}%
\begin{pgfscope}%
\pgfsys@transformshift{3.345334in}{2.702531in}%
\pgfsys@useobject{currentmarker}{}%
\end{pgfscope}%
\begin{pgfscope}%
\pgfsys@transformshift{6.888192in}{4.463327in}%
\pgfsys@useobject{currentmarker}{}%
\end{pgfscope}%
\end{pgfscope}%
\begin{pgfscope}%
\pgfpathrectangle{\pgfqpoint{0.688192in}{0.643904in}}{\pgfqpoint{6.200000in}{4.620000in}}%
\pgfusepath{clip}%
\pgfsetbuttcap%
\pgfsetroundjoin%
\pgfsetlinewidth{1.505625pt}%
\definecolor{currentstroke}{rgb}{0.909804,0.000000,0.043137}%
\pgfsetstrokecolor{currentstroke}%
\pgfsetdash{{4.500000pt}{1.875000pt}{2.250000pt}{1.875000pt}}{0.000000pt}%
\pgfpathmoveto{\pgfqpoint{0.688192in}{0.858156in}}%
\pgfpathlineto{\pgfqpoint{1.573906in}{1.407561in}}%
\pgfpathlineto{\pgfqpoint{3.345334in}{2.117402in}}%
\pgfpathlineto{\pgfqpoint{6.888192in}{4.376203in}}%
\pgfusepath{stroke}%
\end{pgfscope}%
\begin{pgfscope}%
\pgfpathrectangle{\pgfqpoint{0.688192in}{0.643904in}}{\pgfqpoint{6.200000in}{4.620000in}}%
\pgfusepath{clip}%
\pgfsetbuttcap%
\pgfsetroundjoin%
\definecolor{currentfill}{rgb}{0.909804,0.000000,0.043137}%
\pgfsetfillcolor{currentfill}%
\pgfsetlinewidth{0.752812pt}%
\definecolor{currentstroke}{rgb}{1.000000,1.000000,1.000000}%
\pgfsetstrokecolor{currentstroke}%
\pgfsetdash{}{0pt}%
\pgfsys@defobject{currentmarker}{\pgfqpoint{-0.041667in}{-0.041667in}}{\pgfqpoint{0.041667in}{0.041667in}}{%
\pgfpathmoveto{\pgfqpoint{0.000000in}{-0.041667in}}%
\pgfpathcurveto{\pgfqpoint{0.011050in}{-0.041667in}}{\pgfqpoint{0.021649in}{-0.037276in}}{\pgfqpoint{0.029463in}{-0.029463in}}%
\pgfpathcurveto{\pgfqpoint{0.037276in}{-0.021649in}}{\pgfqpoint{0.041667in}{-0.011050in}}{\pgfqpoint{0.041667in}{0.000000in}}%
\pgfpathcurveto{\pgfqpoint{0.041667in}{0.011050in}}{\pgfqpoint{0.037276in}{0.021649in}}{\pgfqpoint{0.029463in}{0.029463in}}%
\pgfpathcurveto{\pgfqpoint{0.021649in}{0.037276in}}{\pgfqpoint{0.011050in}{0.041667in}}{\pgfqpoint{0.000000in}{0.041667in}}%
\pgfpathcurveto{\pgfqpoint{-0.011050in}{0.041667in}}{\pgfqpoint{-0.021649in}{0.037276in}}{\pgfqpoint{-0.029463in}{0.029463in}}%
\pgfpathcurveto{\pgfqpoint{-0.037276in}{0.021649in}}{\pgfqpoint{-0.041667in}{0.011050in}}{\pgfqpoint{-0.041667in}{0.000000in}}%
\pgfpathcurveto{\pgfqpoint{-0.041667in}{-0.011050in}}{\pgfqpoint{-0.037276in}{-0.021649in}}{\pgfqpoint{-0.029463in}{-0.029463in}}%
\pgfpathcurveto{\pgfqpoint{-0.021649in}{-0.037276in}}{\pgfqpoint{-0.011050in}{-0.041667in}}{\pgfqpoint{0.000000in}{-0.041667in}}%
\pgfpathlineto{\pgfqpoint{0.000000in}{-0.041667in}}%
\pgfpathclose%
\pgfusepath{stroke,fill}%
}%
\begin{pgfscope}%
\pgfsys@transformshift{0.688192in}{0.858156in}%
\pgfsys@useobject{currentmarker}{}%
\end{pgfscope}%
\begin{pgfscope}%
\pgfsys@transformshift{1.573906in}{1.407561in}%
\pgfsys@useobject{currentmarker}{}%
\end{pgfscope}%
\begin{pgfscope}%
\pgfsys@transformshift{3.345334in}{2.117402in}%
\pgfsys@useobject{currentmarker}{}%
\end{pgfscope}%
\begin{pgfscope}%
\pgfsys@transformshift{6.888192in}{4.376203in}%
\pgfsys@useobject{currentmarker}{}%
\end{pgfscope}%
\end{pgfscope}%
\begin{pgfscope}%
\pgfpathrectangle{\pgfqpoint{0.688192in}{0.643904in}}{\pgfqpoint{6.200000in}{4.620000in}}%
\pgfusepath{clip}%
\pgfsetbuttcap%
\pgfsetroundjoin%
\pgfsetlinewidth{1.505625pt}%
\definecolor{currentstroke}{rgb}{0.545098,0.168627,0.886275}%
\pgfsetstrokecolor{currentstroke}%
\pgfsetdash{{7.500000pt}{1.500000pt}{1.500000pt}{1.500000pt}}{0.000000pt}%
\pgfpathmoveto{\pgfqpoint{0.688192in}{0.876810in}}%
\pgfpathlineto{\pgfqpoint{1.573906in}{1.419924in}}%
\pgfpathlineto{\pgfqpoint{3.345334in}{2.445519in}}%
\pgfpathlineto{\pgfqpoint{6.888192in}{4.301315in}}%
\pgfusepath{stroke}%
\end{pgfscope}%
\begin{pgfscope}%
\pgfpathrectangle{\pgfqpoint{0.688192in}{0.643904in}}{\pgfqpoint{6.200000in}{4.620000in}}%
\pgfusepath{clip}%
\pgfsetbuttcap%
\pgfsetroundjoin%
\definecolor{currentfill}{rgb}{0.545098,0.168627,0.886275}%
\pgfsetfillcolor{currentfill}%
\pgfsetlinewidth{0.752812pt}%
\definecolor{currentstroke}{rgb}{1.000000,1.000000,1.000000}%
\pgfsetstrokecolor{currentstroke}%
\pgfsetdash{}{0pt}%
\pgfsys@defobject{currentmarker}{\pgfqpoint{-0.041667in}{-0.041667in}}{\pgfqpoint{0.041667in}{0.041667in}}{%
\pgfpathmoveto{\pgfqpoint{0.000000in}{-0.041667in}}%
\pgfpathcurveto{\pgfqpoint{0.011050in}{-0.041667in}}{\pgfqpoint{0.021649in}{-0.037276in}}{\pgfqpoint{0.029463in}{-0.029463in}}%
\pgfpathcurveto{\pgfqpoint{0.037276in}{-0.021649in}}{\pgfqpoint{0.041667in}{-0.011050in}}{\pgfqpoint{0.041667in}{0.000000in}}%
\pgfpathcurveto{\pgfqpoint{0.041667in}{0.011050in}}{\pgfqpoint{0.037276in}{0.021649in}}{\pgfqpoint{0.029463in}{0.029463in}}%
\pgfpathcurveto{\pgfqpoint{0.021649in}{0.037276in}}{\pgfqpoint{0.011050in}{0.041667in}}{\pgfqpoint{0.000000in}{0.041667in}}%
\pgfpathcurveto{\pgfqpoint{-0.011050in}{0.041667in}}{\pgfqpoint{-0.021649in}{0.037276in}}{\pgfqpoint{-0.029463in}{0.029463in}}%
\pgfpathcurveto{\pgfqpoint{-0.037276in}{0.021649in}}{\pgfqpoint{-0.041667in}{0.011050in}}{\pgfqpoint{-0.041667in}{0.000000in}}%
\pgfpathcurveto{\pgfqpoint{-0.041667in}{-0.011050in}}{\pgfqpoint{-0.037276in}{-0.021649in}}{\pgfqpoint{-0.029463in}{-0.029463in}}%
\pgfpathcurveto{\pgfqpoint{-0.021649in}{-0.037276in}}{\pgfqpoint{-0.011050in}{-0.041667in}}{\pgfqpoint{0.000000in}{-0.041667in}}%
\pgfpathlineto{\pgfqpoint{0.000000in}{-0.041667in}}%
\pgfpathclose%
\pgfusepath{stroke,fill}%
}%
\begin{pgfscope}%
\pgfsys@transformshift{0.688192in}{0.876810in}%
\pgfsys@useobject{currentmarker}{}%
\end{pgfscope}%
\begin{pgfscope}%
\pgfsys@transformshift{1.573906in}{1.419924in}%
\pgfsys@useobject{currentmarker}{}%
\end{pgfscope}%
\begin{pgfscope}%
\pgfsys@transformshift{3.345334in}{2.445519in}%
\pgfsys@useobject{currentmarker}{}%
\end{pgfscope}%
\begin{pgfscope}%
\pgfsys@transformshift{6.888192in}{4.301315in}%
\pgfsys@useobject{currentmarker}{}%
\end{pgfscope}%
\end{pgfscope}%
\begin{pgfscope}%
\pgfsetrectcap%
\pgfsetmiterjoin%
\pgfsetlinewidth{0.803000pt}%
\definecolor{currentstroke}{rgb}{0.000000,0.000000,0.000000}%
\pgfsetstrokecolor{currentstroke}%
\pgfsetdash{}{0pt}%
\pgfpathmoveto{\pgfqpoint{0.688192in}{0.643904in}}%
\pgfpathlineto{\pgfqpoint{0.688192in}{5.263904in}}%
\pgfusepath{stroke}%
\end{pgfscope}%
\begin{pgfscope}%
\pgfsetrectcap%
\pgfsetmiterjoin%
\pgfsetlinewidth{0.803000pt}%
\definecolor{currentstroke}{rgb}{0.000000,0.000000,0.000000}%
\pgfsetstrokecolor{currentstroke}%
\pgfsetdash{}{0pt}%
\pgfpathmoveto{\pgfqpoint{6.888192in}{0.643904in}}%
\pgfpathlineto{\pgfqpoint{6.888192in}{5.263904in}}%
\pgfusepath{stroke}%
\end{pgfscope}%
\begin{pgfscope}%
\pgfsetrectcap%
\pgfsetmiterjoin%
\pgfsetlinewidth{0.803000pt}%
\definecolor{currentstroke}{rgb}{0.000000,0.000000,0.000000}%
\pgfsetstrokecolor{currentstroke}%
\pgfsetdash{}{0pt}%
\pgfpathmoveto{\pgfqpoint{0.688192in}{0.643904in}}%
\pgfpathlineto{\pgfqpoint{6.888192in}{0.643904in}}%
\pgfusepath{stroke}%
\end{pgfscope}%
\begin{pgfscope}%
\pgfsetrectcap%
\pgfsetmiterjoin%
\pgfsetlinewidth{0.803000pt}%
\definecolor{currentstroke}{rgb}{0.000000,0.000000,0.000000}%
\pgfsetstrokecolor{currentstroke}%
\pgfsetdash{}{0pt}%
\pgfpathmoveto{\pgfqpoint{0.688192in}{5.263904in}}%
\pgfpathlineto{\pgfqpoint{6.888192in}{5.263904in}}%
\pgfusepath{stroke}%
\end{pgfscope}%
\begin{pgfscope}%
\pgfsetbuttcap%
\pgfsetmiterjoin%
\definecolor{currentfill}{rgb}{1.000000,1.000000,1.000000}%
\pgfsetfillcolor{currentfill}%
\pgfsetfillopacity{0.800000}%
\pgfsetlinewidth{1.003750pt}%
\definecolor{currentstroke}{rgb}{0.800000,0.800000,0.800000}%
\pgfsetstrokecolor{currentstroke}%
\pgfsetstrokeopacity{0.800000}%
\pgfsetdash{}{0pt}%
\pgfpathmoveto{\pgfqpoint{0.824303in}{3.458351in}}%
\pgfpathlineto{\pgfqpoint{2.578541in}{3.458351in}}%
\pgfpathquadraticcurveto{\pgfqpoint{2.617430in}{3.458351in}}{\pgfqpoint{2.617430in}{3.497240in}}%
\pgfpathlineto{\pgfqpoint{2.617430in}{5.127793in}}%
\pgfpathquadraticcurveto{\pgfqpoint{2.617430in}{5.166682in}}{\pgfqpoint{2.578541in}{5.166682in}}%
\pgfpathlineto{\pgfqpoint{0.824303in}{5.166682in}}%
\pgfpathquadraticcurveto{\pgfqpoint{0.785414in}{5.166682in}}{\pgfqpoint{0.785414in}{5.127793in}}%
\pgfpathlineto{\pgfqpoint{0.785414in}{3.497240in}}%
\pgfpathquadraticcurveto{\pgfqpoint{0.785414in}{3.458351in}}{\pgfqpoint{0.824303in}{3.458351in}}%
\pgfpathlineto{\pgfqpoint{0.824303in}{3.458351in}}%
\pgfpathclose%
\pgfusepath{stroke,fill}%
\end{pgfscope}%
\begin{pgfscope}%
\definecolor{textcolor}{rgb}{0.000000,0.000000,0.000000}%
\pgfsetstrokecolor{textcolor}%
\pgfsetfillcolor{textcolor}%
\pgftext[x=0.863192in,y=4.950015in,left,base]{\color{textcolor}{\rmfamily\fontsize{14.000000}{16.800000}\selectfont\catcode`\^=\active\def^{\ifmmode\sp\else\^{}\fi}\catcode`\%=\active\def%{\%}Number of Threads}}%
\end{pgfscope}%
\begin{pgfscope}%
\pgfsetrectcap%
\pgfsetroundjoin%
\pgfsetlinewidth{1.505625pt}%
\definecolor{currentstroke}{rgb}{0.007843,0.243137,1.000000}%
\pgfsetstrokecolor{currentstroke}%
\pgfsetdash{}{0pt}%
\pgfpathmoveto{\pgfqpoint{1.331284in}{4.743071in}}%
\pgfpathlineto{\pgfqpoint{1.525729in}{4.743071in}}%
\pgfpathlineto{\pgfqpoint{1.720173in}{4.743071in}}%
\pgfusepath{stroke}%
\end{pgfscope}%
\begin{pgfscope}%
\pgfsetbuttcap%
\pgfsetroundjoin%
\definecolor{currentfill}{rgb}{0.007843,0.243137,1.000000}%
\pgfsetfillcolor{currentfill}%
\pgfsetlinewidth{0.752812pt}%
\definecolor{currentstroke}{rgb}{1.000000,1.000000,1.000000}%
\pgfsetstrokecolor{currentstroke}%
\pgfsetdash{}{0pt}%
\pgfsys@defobject{currentmarker}{\pgfqpoint{-0.041667in}{-0.041667in}}{\pgfqpoint{0.041667in}{0.041667in}}{%
\pgfpathmoveto{\pgfqpoint{0.000000in}{-0.041667in}}%
\pgfpathcurveto{\pgfqpoint{0.011050in}{-0.041667in}}{\pgfqpoint{0.021649in}{-0.037276in}}{\pgfqpoint{0.029463in}{-0.029463in}}%
\pgfpathcurveto{\pgfqpoint{0.037276in}{-0.021649in}}{\pgfqpoint{0.041667in}{-0.011050in}}{\pgfqpoint{0.041667in}{0.000000in}}%
\pgfpathcurveto{\pgfqpoint{0.041667in}{0.011050in}}{\pgfqpoint{0.037276in}{0.021649in}}{\pgfqpoint{0.029463in}{0.029463in}}%
\pgfpathcurveto{\pgfqpoint{0.021649in}{0.037276in}}{\pgfqpoint{0.011050in}{0.041667in}}{\pgfqpoint{0.000000in}{0.041667in}}%
\pgfpathcurveto{\pgfqpoint{-0.011050in}{0.041667in}}{\pgfqpoint{-0.021649in}{0.037276in}}{\pgfqpoint{-0.029463in}{0.029463in}}%
\pgfpathcurveto{\pgfqpoint{-0.037276in}{0.021649in}}{\pgfqpoint{-0.041667in}{0.011050in}}{\pgfqpoint{-0.041667in}{0.000000in}}%
\pgfpathcurveto{\pgfqpoint{-0.041667in}{-0.011050in}}{\pgfqpoint{-0.037276in}{-0.021649in}}{\pgfqpoint{-0.029463in}{-0.029463in}}%
\pgfpathcurveto{\pgfqpoint{-0.021649in}{-0.037276in}}{\pgfqpoint{-0.011050in}{-0.041667in}}{\pgfqpoint{0.000000in}{-0.041667in}}%
\pgfpathlineto{\pgfqpoint{0.000000in}{-0.041667in}}%
\pgfpathclose%
\pgfusepath{stroke,fill}%
}%
\begin{pgfscope}%
\pgfsys@transformshift{1.525729in}{4.743071in}%
\pgfsys@useobject{currentmarker}{}%
\end{pgfscope}%
\end{pgfscope}%
\begin{pgfscope}%
\definecolor{textcolor}{rgb}{0.000000,0.000000,0.000000}%
\pgfsetstrokecolor{textcolor}%
\pgfsetfillcolor{textcolor}%
\pgftext[x=1.875729in,y=4.675016in,left,base]{\color{textcolor}{\rmfamily\fontsize{14.000000}{16.800000}\selectfont\catcode`\^=\active\def^{\ifmmode\sp\else\^{}\fi}\catcode`\%=\active\def%{\%}1}}%
\end{pgfscope}%
\begin{pgfscope}%
\pgfsetbuttcap%
\pgfsetroundjoin%
\pgfsetlinewidth{1.505625pt}%
\definecolor{currentstroke}{rgb}{1.000000,0.486275,0.000000}%
\pgfsetstrokecolor{currentstroke}%
\pgfsetdash{{6.000000pt}{2.250000pt}}{0.000000pt}%
\pgfpathmoveto{\pgfqpoint{1.331284in}{4.468072in}}%
\pgfpathlineto{\pgfqpoint{1.525729in}{4.468072in}}%
\pgfpathlineto{\pgfqpoint{1.720173in}{4.468072in}}%
\pgfusepath{stroke}%
\end{pgfscope}%
\begin{pgfscope}%
\pgfsetbuttcap%
\pgfsetroundjoin%
\definecolor{currentfill}{rgb}{1.000000,0.486275,0.000000}%
\pgfsetfillcolor{currentfill}%
\pgfsetlinewidth{0.752812pt}%
\definecolor{currentstroke}{rgb}{1.000000,1.000000,1.000000}%
\pgfsetstrokecolor{currentstroke}%
\pgfsetdash{}{0pt}%
\pgfsys@defobject{currentmarker}{\pgfqpoint{-0.041667in}{-0.041667in}}{\pgfqpoint{0.041667in}{0.041667in}}{%
\pgfpathmoveto{\pgfqpoint{0.000000in}{-0.041667in}}%
\pgfpathcurveto{\pgfqpoint{0.011050in}{-0.041667in}}{\pgfqpoint{0.021649in}{-0.037276in}}{\pgfqpoint{0.029463in}{-0.029463in}}%
\pgfpathcurveto{\pgfqpoint{0.037276in}{-0.021649in}}{\pgfqpoint{0.041667in}{-0.011050in}}{\pgfqpoint{0.041667in}{0.000000in}}%
\pgfpathcurveto{\pgfqpoint{0.041667in}{0.011050in}}{\pgfqpoint{0.037276in}{0.021649in}}{\pgfqpoint{0.029463in}{0.029463in}}%
\pgfpathcurveto{\pgfqpoint{0.021649in}{0.037276in}}{\pgfqpoint{0.011050in}{0.041667in}}{\pgfqpoint{0.000000in}{0.041667in}}%
\pgfpathcurveto{\pgfqpoint{-0.011050in}{0.041667in}}{\pgfqpoint{-0.021649in}{0.037276in}}{\pgfqpoint{-0.029463in}{0.029463in}}%
\pgfpathcurveto{\pgfqpoint{-0.037276in}{0.021649in}}{\pgfqpoint{-0.041667in}{0.011050in}}{\pgfqpoint{-0.041667in}{0.000000in}}%
\pgfpathcurveto{\pgfqpoint{-0.041667in}{-0.011050in}}{\pgfqpoint{-0.037276in}{-0.021649in}}{\pgfqpoint{-0.029463in}{-0.029463in}}%
\pgfpathcurveto{\pgfqpoint{-0.021649in}{-0.037276in}}{\pgfqpoint{-0.011050in}{-0.041667in}}{\pgfqpoint{0.000000in}{-0.041667in}}%
\pgfpathlineto{\pgfqpoint{0.000000in}{-0.041667in}}%
\pgfpathclose%
\pgfusepath{stroke,fill}%
}%
\begin{pgfscope}%
\pgfsys@transformshift{1.525729in}{4.468072in}%
\pgfsys@useobject{currentmarker}{}%
\end{pgfscope}%
\end{pgfscope}%
\begin{pgfscope}%
\definecolor{textcolor}{rgb}{0.000000,0.000000,0.000000}%
\pgfsetstrokecolor{textcolor}%
\pgfsetfillcolor{textcolor}%
\pgftext[x=1.875729in,y=4.400016in,left,base]{\color{textcolor}{\rmfamily\fontsize{14.000000}{16.800000}\selectfont\catcode`\^=\active\def^{\ifmmode\sp\else\^{}\fi}\catcode`\%=\active\def%{\%}2}}%
\end{pgfscope}%
\begin{pgfscope}%
\pgfsetbuttcap%
\pgfsetroundjoin%
\pgfsetlinewidth{1.505625pt}%
\definecolor{currentstroke}{rgb}{0.101961,0.788235,0.219608}%
\pgfsetstrokecolor{currentstroke}%
\pgfsetdash{{1.500000pt}{1.500000pt}}{0.000000pt}%
\pgfpathmoveto{\pgfqpoint{1.331284in}{4.193072in}}%
\pgfpathlineto{\pgfqpoint{1.525729in}{4.193072in}}%
\pgfpathlineto{\pgfqpoint{1.720173in}{4.193072in}}%
\pgfusepath{stroke}%
\end{pgfscope}%
\begin{pgfscope}%
\pgfsetbuttcap%
\pgfsetroundjoin%
\definecolor{currentfill}{rgb}{0.101961,0.788235,0.219608}%
\pgfsetfillcolor{currentfill}%
\pgfsetlinewidth{0.752812pt}%
\definecolor{currentstroke}{rgb}{1.000000,1.000000,1.000000}%
\pgfsetstrokecolor{currentstroke}%
\pgfsetdash{}{0pt}%
\pgfsys@defobject{currentmarker}{\pgfqpoint{-0.041667in}{-0.041667in}}{\pgfqpoint{0.041667in}{0.041667in}}{%
\pgfpathmoveto{\pgfqpoint{0.000000in}{-0.041667in}}%
\pgfpathcurveto{\pgfqpoint{0.011050in}{-0.041667in}}{\pgfqpoint{0.021649in}{-0.037276in}}{\pgfqpoint{0.029463in}{-0.029463in}}%
\pgfpathcurveto{\pgfqpoint{0.037276in}{-0.021649in}}{\pgfqpoint{0.041667in}{-0.011050in}}{\pgfqpoint{0.041667in}{0.000000in}}%
\pgfpathcurveto{\pgfqpoint{0.041667in}{0.011050in}}{\pgfqpoint{0.037276in}{0.021649in}}{\pgfqpoint{0.029463in}{0.029463in}}%
\pgfpathcurveto{\pgfqpoint{0.021649in}{0.037276in}}{\pgfqpoint{0.011050in}{0.041667in}}{\pgfqpoint{0.000000in}{0.041667in}}%
\pgfpathcurveto{\pgfqpoint{-0.011050in}{0.041667in}}{\pgfqpoint{-0.021649in}{0.037276in}}{\pgfqpoint{-0.029463in}{0.029463in}}%
\pgfpathcurveto{\pgfqpoint{-0.037276in}{0.021649in}}{\pgfqpoint{-0.041667in}{0.011050in}}{\pgfqpoint{-0.041667in}{0.000000in}}%
\pgfpathcurveto{\pgfqpoint{-0.041667in}{-0.011050in}}{\pgfqpoint{-0.037276in}{-0.021649in}}{\pgfqpoint{-0.029463in}{-0.029463in}}%
\pgfpathcurveto{\pgfqpoint{-0.021649in}{-0.037276in}}{\pgfqpoint{-0.011050in}{-0.041667in}}{\pgfqpoint{0.000000in}{-0.041667in}}%
\pgfpathlineto{\pgfqpoint{0.000000in}{-0.041667in}}%
\pgfpathclose%
\pgfusepath{stroke,fill}%
}%
\begin{pgfscope}%
\pgfsys@transformshift{1.525729in}{4.193072in}%
\pgfsys@useobject{currentmarker}{}%
\end{pgfscope}%
\end{pgfscope}%
\begin{pgfscope}%
\definecolor{textcolor}{rgb}{0.000000,0.000000,0.000000}%
\pgfsetstrokecolor{textcolor}%
\pgfsetfillcolor{textcolor}%
\pgftext[x=1.875729in,y=4.125016in,left,base]{\color{textcolor}{\rmfamily\fontsize{14.000000}{16.800000}\selectfont\catcode`\^=\active\def^{\ifmmode\sp\else\^{}\fi}\catcode`\%=\active\def%{\%}4}}%
\end{pgfscope}%
\begin{pgfscope}%
\pgfsetbuttcap%
\pgfsetroundjoin%
\pgfsetlinewidth{1.505625pt}%
\definecolor{currentstroke}{rgb}{0.909804,0.000000,0.043137}%
\pgfsetstrokecolor{currentstroke}%
\pgfsetdash{{4.500000pt}{1.875000pt}{2.250000pt}{1.875000pt}}{0.000000pt}%
\pgfpathmoveto{\pgfqpoint{1.331284in}{3.918072in}}%
\pgfpathlineto{\pgfqpoint{1.525729in}{3.918072in}}%
\pgfpathlineto{\pgfqpoint{1.720173in}{3.918072in}}%
\pgfusepath{stroke}%
\end{pgfscope}%
\begin{pgfscope}%
\pgfsetbuttcap%
\pgfsetroundjoin%
\definecolor{currentfill}{rgb}{0.909804,0.000000,0.043137}%
\pgfsetfillcolor{currentfill}%
\pgfsetlinewidth{0.752812pt}%
\definecolor{currentstroke}{rgb}{1.000000,1.000000,1.000000}%
\pgfsetstrokecolor{currentstroke}%
\pgfsetdash{}{0pt}%
\pgfsys@defobject{currentmarker}{\pgfqpoint{-0.041667in}{-0.041667in}}{\pgfqpoint{0.041667in}{0.041667in}}{%
\pgfpathmoveto{\pgfqpoint{0.000000in}{-0.041667in}}%
\pgfpathcurveto{\pgfqpoint{0.011050in}{-0.041667in}}{\pgfqpoint{0.021649in}{-0.037276in}}{\pgfqpoint{0.029463in}{-0.029463in}}%
\pgfpathcurveto{\pgfqpoint{0.037276in}{-0.021649in}}{\pgfqpoint{0.041667in}{-0.011050in}}{\pgfqpoint{0.041667in}{0.000000in}}%
\pgfpathcurveto{\pgfqpoint{0.041667in}{0.011050in}}{\pgfqpoint{0.037276in}{0.021649in}}{\pgfqpoint{0.029463in}{0.029463in}}%
\pgfpathcurveto{\pgfqpoint{0.021649in}{0.037276in}}{\pgfqpoint{0.011050in}{0.041667in}}{\pgfqpoint{0.000000in}{0.041667in}}%
\pgfpathcurveto{\pgfqpoint{-0.011050in}{0.041667in}}{\pgfqpoint{-0.021649in}{0.037276in}}{\pgfqpoint{-0.029463in}{0.029463in}}%
\pgfpathcurveto{\pgfqpoint{-0.037276in}{0.021649in}}{\pgfqpoint{-0.041667in}{0.011050in}}{\pgfqpoint{-0.041667in}{0.000000in}}%
\pgfpathcurveto{\pgfqpoint{-0.041667in}{-0.011050in}}{\pgfqpoint{-0.037276in}{-0.021649in}}{\pgfqpoint{-0.029463in}{-0.029463in}}%
\pgfpathcurveto{\pgfqpoint{-0.021649in}{-0.037276in}}{\pgfqpoint{-0.011050in}{-0.041667in}}{\pgfqpoint{0.000000in}{-0.041667in}}%
\pgfpathlineto{\pgfqpoint{0.000000in}{-0.041667in}}%
\pgfpathclose%
\pgfusepath{stroke,fill}%
}%
\begin{pgfscope}%
\pgfsys@transformshift{1.525729in}{3.918072in}%
\pgfsys@useobject{currentmarker}{}%
\end{pgfscope}%
\end{pgfscope}%
\begin{pgfscope}%
\definecolor{textcolor}{rgb}{0.000000,0.000000,0.000000}%
\pgfsetstrokecolor{textcolor}%
\pgfsetfillcolor{textcolor}%
\pgftext[x=1.875729in,y=3.850017in,left,base]{\color{textcolor}{\rmfamily\fontsize{14.000000}{16.800000}\selectfont\catcode`\^=\active\def^{\ifmmode\sp\else\^{}\fi}\catcode`\%=\active\def%{\%}8}}%
\end{pgfscope}%
\begin{pgfscope}%
\pgfsetbuttcap%
\pgfsetroundjoin%
\pgfsetlinewidth{1.505625pt}%
\definecolor{currentstroke}{rgb}{0.545098,0.168627,0.886275}%
\pgfsetstrokecolor{currentstroke}%
\pgfsetdash{{7.500000pt}{1.500000pt}{1.500000pt}{1.500000pt}}{0.000000pt}%
\pgfpathmoveto{\pgfqpoint{1.331284in}{3.643073in}}%
\pgfpathlineto{\pgfqpoint{1.525729in}{3.643073in}}%
\pgfpathlineto{\pgfqpoint{1.720173in}{3.643073in}}%
\pgfusepath{stroke}%
\end{pgfscope}%
\begin{pgfscope}%
\pgfsetbuttcap%
\pgfsetroundjoin%
\definecolor{currentfill}{rgb}{0.545098,0.168627,0.886275}%
\pgfsetfillcolor{currentfill}%
\pgfsetlinewidth{0.752812pt}%
\definecolor{currentstroke}{rgb}{1.000000,1.000000,1.000000}%
\pgfsetstrokecolor{currentstroke}%
\pgfsetdash{}{0pt}%
\pgfsys@defobject{currentmarker}{\pgfqpoint{-0.041667in}{-0.041667in}}{\pgfqpoint{0.041667in}{0.041667in}}{%
\pgfpathmoveto{\pgfqpoint{0.000000in}{-0.041667in}}%
\pgfpathcurveto{\pgfqpoint{0.011050in}{-0.041667in}}{\pgfqpoint{0.021649in}{-0.037276in}}{\pgfqpoint{0.029463in}{-0.029463in}}%
\pgfpathcurveto{\pgfqpoint{0.037276in}{-0.021649in}}{\pgfqpoint{0.041667in}{-0.011050in}}{\pgfqpoint{0.041667in}{0.000000in}}%
\pgfpathcurveto{\pgfqpoint{0.041667in}{0.011050in}}{\pgfqpoint{0.037276in}{0.021649in}}{\pgfqpoint{0.029463in}{0.029463in}}%
\pgfpathcurveto{\pgfqpoint{0.021649in}{0.037276in}}{\pgfqpoint{0.011050in}{0.041667in}}{\pgfqpoint{0.000000in}{0.041667in}}%
\pgfpathcurveto{\pgfqpoint{-0.011050in}{0.041667in}}{\pgfqpoint{-0.021649in}{0.037276in}}{\pgfqpoint{-0.029463in}{0.029463in}}%
\pgfpathcurveto{\pgfqpoint{-0.037276in}{0.021649in}}{\pgfqpoint{-0.041667in}{0.011050in}}{\pgfqpoint{-0.041667in}{0.000000in}}%
\pgfpathcurveto{\pgfqpoint{-0.041667in}{-0.011050in}}{\pgfqpoint{-0.037276in}{-0.021649in}}{\pgfqpoint{-0.029463in}{-0.029463in}}%
\pgfpathcurveto{\pgfqpoint{-0.021649in}{-0.037276in}}{\pgfqpoint{-0.011050in}{-0.041667in}}{\pgfqpoint{0.000000in}{-0.041667in}}%
\pgfpathlineto{\pgfqpoint{0.000000in}{-0.041667in}}%
\pgfpathclose%
\pgfusepath{stroke,fill}%
}%
\begin{pgfscope}%
\pgfsys@transformshift{1.525729in}{3.643073in}%
\pgfsys@useobject{currentmarker}{}%
\end{pgfscope}%
\end{pgfscope}%
\begin{pgfscope}%
\definecolor{textcolor}{rgb}{0.000000,0.000000,0.000000}%
\pgfsetstrokecolor{textcolor}%
\pgfsetfillcolor{textcolor}%
\pgftext[x=1.875729in,y=3.575017in,left,base]{\color{textcolor}{\rmfamily\fontsize{14.000000}{16.800000}\selectfont\catcode`\^=\active\def^{\ifmmode\sp\else\^{}\fi}\catcode`\%=\active\def%{\%}12}}%
\end{pgfscope}%
\end{pgfpicture}%
\makeatother%
\endgroup%
}
    \caption{Time scaling of a capacity expansion problem using a range of
    threads for parallelization.}
    \label{fig:thread-scaling}
\end{figure}

All simulations perform similarly when the problem size is small
because multithreading has some overhead. Multiple threads outperformed the
single threaded simulation in every case. Eight threads outperformed twelve
threads in the middle of the population range, but otherwise performed
similarly. Simulations with two and four threads performed similarly until the
higher end of the population range where four threads proved faster. The overall
speed improvement from multithreading was modest with a roughly four second
improvement at best. This suggests that further code optimization could improve
performance and that better computer architecture might be needed to fully
realize the parallelization enhancement.

\FloatBarrier
\section{Benchmarking \ac{osier}}
\subsection{\acs{temoa} and \acs{pygen}}
\label{section:temoa}

This thesis uses the tools \ac{temoa} and \ac{pygen} to establish benchmark
results for a typical \ac{esom}. \ac{temoa} is an open-source \ac{esom}
developed at North Carolina State University that uses \ac{milp} to develop
capacity-expansion scenarios \cite{decarolis_temoa_2010}. The key benefits of
\ac{temoa} are its open-source code, open data, and built-in uncertainty
analysis capabilities. These features address the need for greater transparency
in \ac{esom} modeling and robust assessment of future uncertainties
\cite{hunter_modeling_2013, fattahi_systemic_2020}. \ac{pygen} is another
open-source code, developed by this thesis' author, that facilitates rapid
development of \ac{temoa} models and enables sensitivity analyses using a
templated approach \cite{dotson_influence_2022, dotson_python_2021}. These
features of \ac{pygen} reduce the learning curve and the cost of producing
unique models in \ac{temoa} \cite{dotson_influence_2022}.

A single \ac{temoa} run minimizes total system cost \cite{decarolis_temoa_2010},

\begin{align}
  C_{total} &= C_{loans} + C_{fixed} + C_{variable}
  \intertext{where}
  C_{loans} &= \text{the sum of all investment loan costs},\nonumber\\
  C_{fixed} &= \text{the sum of all fixed operating costs},\nonumber\\
  C_{variable} &= \text{the sum of all variable operating costs}.\nonumber
\end{align}

Each of these terms is amortized over the model time horizon. The decision
variables include the generation from each technology at time, $t$, and the
capacity of each technology in year, $y$. The dispatch model deviates slightly
from the model described in Section \ref{section:dispatch} by making the initial
storage level for energy storage technologies a decision variable, whereas the
dispatch model used in this thesis does not optimize initial storage and assumes
energy storage starts at zero. The detailed formulation of \ac{temoa}'s
constraints and equations are available online \cite{decarolis_temoa_2010}.
% (\textcolor{red}{maybe in an appendix?}). 

\subsection{\acl{mga}}
\label{section:mga}
\ac{temoa}'s built-in method for uncertainty analysis is the \ac{hsj}
formulation of \ac{mga}. This algorithm is designed to handle
\textit{structural} uncertainty, which presumes to account for unmodeled
objectives. The steps for \ac{hsj} are \cite{decarolis_using_2011,
dotson_influence_2022}:
\begin{enumerate}
  \item obtain an optimal solution by any method,
  \item add a user-specified amount of slack to the objective function value
  from the first step,
  \item use the adjusted objective function value as an upper bound constraint,
  \item generate a new objective function that minimizes the sum of all decision
  variables,
  \item iterate the procedure,
  \item stop the \ac{mga} when no significant changes are observed.
\end{enumerate}
The mathematical formulation of this algorithm is to
\begin{align}
  \intertext{minimize:}
  p &= \sum_{k\in K} x_k,
  \intertext{subject to:}
  f_j\left(\vec{x}\right) &\leq T_j \quad\forall \quad j,\\
  \vec{x}&\in X,
  \intertext{where}
  p &= \text{the new objective function}\nonumber,\\
  x_k &= \text{the $k^{th}$ decision variable with a nonzero value in previous solutions}\nonumber,\\
  f_j\left(\vec{x}\right) &= \text{the $j^{th}$ original objective function},\nonumber\\
  T_j &= \text{the slack-adjusted target value},\nonumber\\
  X &= \text{the set of all feasible solutions}.\nonumber
\end{align}

Figure \ref{fig:standard_mga} illustrates this algorithm for a simple \ac{lp}
with two decision variables and a slack value of 10\%.

\begin{align}
  \intertext{Minimize:}
  f(x_1, x_2) &= c_1x_1 + c_2x_2,
  \intertext{subject to:}
  x_1 &+ x_2 = 1,
  \intertext{where}
  x_{k} &= \text{the $k^{th}$ decision variable,} \nonumber \\
  c_{k} &= \text{the $k^{th}$ cost.}\nonumber
\end{align}

The optimal solution occurs where the objective and constraint functions
intersect at $\left(1,0\right)$. Relaxing the objective function by 10\% gives a
new constraint shown by the dashed line. Since the constraint is written as an
equality, the feasible space is given by all points on the constraint curve. The
new \ac{mga} solution is now at the intersection between the \ac{mga} constraint and the
original constraint, $\left(0.6, 0.4\right)$.

\begin{figure}[h]
  \centering
  \resizebox{0.75\columnwidth}{!}{%% Creator: Matplotlib, PGF backend
%%
%% To include the figure in your LaTeX document, write
%%   \input{<filename>.pgf}
%%
%% Make sure the required packages are loaded in your preamble
%%   \usepackage{pgf}
%%
%% Also ensure that all the required font packages are loaded; for instance,
%% the lmodern package is sometimes necessary when using math font.
%%   \usepackage{lmodern}
%%
%% Figures using additional raster images can only be included by \input if
%% they are in the same directory as the main LaTeX file. For loading figures
%% from other directories you can use the `import` package
%%   \usepackage{import}
%%
%% and then include the figures with
%%   \import{<path to file>}{<filename>.pgf}
%%
%% Matplotlib used the following preamble
%%   \def\mathdefault#1{#1}
%%   \everymath=\expandafter{\the\everymath\displaystyle}
%%   \IfFileExists{scrextend.sty}{
%%     \usepackage[fontsize=10.000000pt]{scrextend}
%%   }{
%%     \renewcommand{\normalsize}{\fontsize{10.000000}{12.000000}\selectfont}
%%     \normalsize
%%   }
%%   
%%   \ifdefined\pdftexversion\else  % non-pdftex case.
%%     \usepackage{fontspec}
%%     \setmainfont{DejaVuSerif.ttf}[Path=\detokenize{/Users/samdotson/miniforge3/envs/2025-dotson-thesis/lib/python3.11/site-packages/matplotlib/mpl-data/fonts/ttf/}]
%%     \setsansfont{DejaVuSans.ttf}[Path=\detokenize{/Users/samdotson/miniforge3/envs/2025-dotson-thesis/lib/python3.11/site-packages/matplotlib/mpl-data/fonts/ttf/}]
%%     \setmonofont{DejaVuSansMono.ttf}[Path=\detokenize{/Users/samdotson/miniforge3/envs/2025-dotson-thesis/lib/python3.11/site-packages/matplotlib/mpl-data/fonts/ttf/}]
%%   \fi
%%   \makeatletter\@ifpackageloaded{underscore}{}{\usepackage[strings]{underscore}}\makeatother
%%
\begingroup%
\makeatletter%
\begin{pgfpicture}%
\pgfpathrectangle{\pgfpointorigin}{\pgfqpoint{7.930676in}{5.968802in}}%
\pgfusepath{use as bounding box, clip}%
\begin{pgfscope}%
\pgfsetbuttcap%
\pgfsetmiterjoin%
\definecolor{currentfill}{rgb}{1.000000,1.000000,1.000000}%
\pgfsetfillcolor{currentfill}%
\pgfsetlinewidth{0.000000pt}%
\definecolor{currentstroke}{rgb}{0.000000,0.000000,0.000000}%
\pgfsetstrokecolor{currentstroke}%
\pgfsetdash{}{0pt}%
\pgfpathmoveto{\pgfqpoint{0.000000in}{0.000000in}}%
\pgfpathlineto{\pgfqpoint{7.930676in}{0.000000in}}%
\pgfpathlineto{\pgfqpoint{7.930676in}{5.968802in}}%
\pgfpathlineto{\pgfqpoint{0.000000in}{5.968802in}}%
\pgfpathlineto{\pgfqpoint{0.000000in}{0.000000in}}%
\pgfpathclose%
\pgfusepath{fill}%
\end{pgfscope}%
\begin{pgfscope}%
\pgfsetbuttcap%
\pgfsetmiterjoin%
\definecolor{currentfill}{rgb}{1.000000,1.000000,1.000000}%
\pgfsetfillcolor{currentfill}%
\pgfsetlinewidth{0.000000pt}%
\definecolor{currentstroke}{rgb}{0.000000,0.000000,0.000000}%
\pgfsetstrokecolor{currentstroke}%
\pgfsetstrokeopacity{0.000000}%
\pgfsetdash{}{0pt}%
\pgfpathmoveto{\pgfqpoint{0.771832in}{0.709782in}}%
\pgfpathlineto{\pgfqpoint{7.830676in}{0.709782in}}%
\pgfpathlineto{\pgfqpoint{7.830676in}{5.118077in}}%
\pgfpathlineto{\pgfqpoint{0.771832in}{5.118077in}}%
\pgfpathlineto{\pgfqpoint{0.771832in}{0.709782in}}%
\pgfpathclose%
\pgfusepath{fill}%
\end{pgfscope}%
\begin{pgfscope}%
\pgfpathrectangle{\pgfqpoint{0.771832in}{0.709782in}}{\pgfqpoint{7.058844in}{4.408296in}}%
\pgfusepath{clip}%
\pgfsetbuttcap%
\pgfsetroundjoin%
\definecolor{currentfill}{rgb}{0.000000,0.000000,0.000000}%
\pgfsetfillcolor{currentfill}%
\pgfsetlinewidth{1.003750pt}%
\definecolor{currentstroke}{rgb}{0.000000,0.000000,0.000000}%
\pgfsetstrokecolor{currentstroke}%
\pgfsetdash{}{0pt}%
\pgfsys@defobject{currentmarker}{\pgfqpoint{-0.065881in}{-0.065881in}}{\pgfqpoint{0.065881in}{0.065881in}}{%
\pgfpathmoveto{\pgfqpoint{0.000000in}{-0.065881in}}%
\pgfpathcurveto{\pgfqpoint{0.017472in}{-0.065881in}}{\pgfqpoint{0.034230in}{-0.058939in}}{\pgfqpoint{0.046585in}{-0.046585in}}%
\pgfpathcurveto{\pgfqpoint{0.058939in}{-0.034230in}}{\pgfqpoint{0.065881in}{-0.017472in}}{\pgfqpoint{0.065881in}{0.000000in}}%
\pgfpathcurveto{\pgfqpoint{0.065881in}{0.017472in}}{\pgfqpoint{0.058939in}{0.034230in}}{\pgfqpoint{0.046585in}{0.046585in}}%
\pgfpathcurveto{\pgfqpoint{0.034230in}{0.058939in}}{\pgfqpoint{0.017472in}{0.065881in}}{\pgfqpoint{0.000000in}{0.065881in}}%
\pgfpathcurveto{\pgfqpoint{-0.017472in}{0.065881in}}{\pgfqpoint{-0.034230in}{0.058939in}}{\pgfqpoint{-0.046585in}{0.046585in}}%
\pgfpathcurveto{\pgfqpoint{-0.058939in}{0.034230in}}{\pgfqpoint{-0.065881in}{0.017472in}}{\pgfqpoint{-0.065881in}{0.000000in}}%
\pgfpathcurveto{\pgfqpoint{-0.065881in}{-0.017472in}}{\pgfqpoint{-0.058939in}{-0.034230in}}{\pgfqpoint{-0.046585in}{-0.046585in}}%
\pgfpathcurveto{\pgfqpoint{-0.034230in}{-0.058939in}}{\pgfqpoint{-0.017472in}{-0.065881in}}{\pgfqpoint{0.000000in}{-0.065881in}}%
\pgfpathlineto{\pgfqpoint{0.000000in}{-0.065881in}}%
\pgfpathclose%
\pgfusepath{stroke,fill}%
}%
\begin{pgfscope}%
\pgfsys@transformshift{4.622110in}{2.312798in}%
\pgfsys@useobject{currentmarker}{}%
\end{pgfscope}%
\end{pgfscope}%
\begin{pgfscope}%
\pgfpathrectangle{\pgfqpoint{0.771832in}{0.709782in}}{\pgfqpoint{7.058844in}{4.408296in}}%
\pgfusepath{clip}%
\pgfsetbuttcap%
\pgfsetroundjoin%
\definecolor{currentfill}{rgb}{1.000000,1.000000,1.000000}%
\pgfsetfillcolor{currentfill}%
\pgfsetlinewidth{1.003750pt}%
\definecolor{currentstroke}{rgb}{0.000000,0.000000,0.000000}%
\pgfsetstrokecolor{currentstroke}%
\pgfsetdash{}{0pt}%
\pgfsys@defobject{currentmarker}{\pgfqpoint{-0.065881in}{-0.065881in}}{\pgfqpoint{0.065881in}{0.065881in}}{%
\pgfpathmoveto{\pgfqpoint{0.000000in}{-0.065881in}}%
\pgfpathcurveto{\pgfqpoint{0.017472in}{-0.065881in}}{\pgfqpoint{0.034230in}{-0.058939in}}{\pgfqpoint{0.046585in}{-0.046585in}}%
\pgfpathcurveto{\pgfqpoint{0.058939in}{-0.034230in}}{\pgfqpoint{0.065881in}{-0.017472in}}{\pgfqpoint{0.065881in}{0.000000in}}%
\pgfpathcurveto{\pgfqpoint{0.065881in}{0.017472in}}{\pgfqpoint{0.058939in}{0.034230in}}{\pgfqpoint{0.046585in}{0.046585in}}%
\pgfpathcurveto{\pgfqpoint{0.034230in}{0.058939in}}{\pgfqpoint{0.017472in}{0.065881in}}{\pgfqpoint{0.000000in}{0.065881in}}%
\pgfpathcurveto{\pgfqpoint{-0.017472in}{0.065881in}}{\pgfqpoint{-0.034230in}{0.058939in}}{\pgfqpoint{-0.046585in}{0.046585in}}%
\pgfpathcurveto{\pgfqpoint{-0.058939in}{0.034230in}}{\pgfqpoint{-0.065881in}{0.017472in}}{\pgfqpoint{-0.065881in}{0.000000in}}%
\pgfpathcurveto{\pgfqpoint{-0.065881in}{-0.017472in}}{\pgfqpoint{-0.058939in}{-0.034230in}}{\pgfqpoint{-0.046585in}{-0.046585in}}%
\pgfpathcurveto{\pgfqpoint{-0.034230in}{-0.058939in}}{\pgfqpoint{-0.017472in}{-0.065881in}}{\pgfqpoint{0.000000in}{-0.065881in}}%
\pgfpathlineto{\pgfqpoint{0.000000in}{-0.065881in}}%
\pgfpathclose%
\pgfusepath{stroke,fill}%
}%
\begin{pgfscope}%
\pgfsys@transformshift{7.188963in}{0.709782in}%
\pgfsys@useobject{currentmarker}{}%
\end{pgfscope}%
\end{pgfscope}%
\begin{pgfscope}%
\pgfpathrectangle{\pgfqpoint{0.771832in}{0.709782in}}{\pgfqpoint{7.058844in}{4.408296in}}%
\pgfusepath{clip}%
\pgfsetrectcap%
\pgfsetroundjoin%
\pgfsetlinewidth{0.803000pt}%
\definecolor{currentstroke}{rgb}{0.501961,0.501961,0.501961}%
\pgfsetstrokecolor{currentstroke}%
\pgfsetstrokeopacity{0.300000}%
\pgfsetdash{}{0pt}%
\pgfpathmoveto{\pgfqpoint{0.771832in}{0.709782in}}%
\pgfpathlineto{\pgfqpoint{0.771832in}{5.118077in}}%
\pgfusepath{stroke}%
\end{pgfscope}%
\begin{pgfscope}%
\pgfsetbuttcap%
\pgfsetroundjoin%
\definecolor{currentfill}{rgb}{0.000000,0.000000,0.000000}%
\pgfsetfillcolor{currentfill}%
\pgfsetlinewidth{0.803000pt}%
\definecolor{currentstroke}{rgb}{0.000000,0.000000,0.000000}%
\pgfsetstrokecolor{currentstroke}%
\pgfsetdash{}{0pt}%
\pgfsys@defobject{currentmarker}{\pgfqpoint{0.000000in}{-0.048611in}}{\pgfqpoint{0.000000in}{0.000000in}}{%
\pgfpathmoveto{\pgfqpoint{0.000000in}{0.000000in}}%
\pgfpathlineto{\pgfqpoint{0.000000in}{-0.048611in}}%
\pgfusepath{stroke,fill}%
}%
\begin{pgfscope}%
\pgfsys@transformshift{0.771832in}{0.709782in}%
\pgfsys@useobject{currentmarker}{}%
\end{pgfscope}%
\end{pgfscope}%
\begin{pgfscope}%
\definecolor{textcolor}{rgb}{0.000000,0.000000,0.000000}%
\pgfsetstrokecolor{textcolor}%
\pgfsetfillcolor{textcolor}%
\pgftext[x=0.771832in,y=0.612559in,,top]{\color{textcolor}{\rmfamily\fontsize{14.000000}{16.800000}\selectfont\catcode`\^=\active\def^{\ifmmode\sp\else\^{}\fi}\catcode`\%=\active\def%{\%}$\mathdefault{0.0}$}}%
\end{pgfscope}%
\begin{pgfscope}%
\pgfpathrectangle{\pgfqpoint{0.771832in}{0.709782in}}{\pgfqpoint{7.058844in}{4.408296in}}%
\pgfusepath{clip}%
\pgfsetrectcap%
\pgfsetroundjoin%
\pgfsetlinewidth{0.803000pt}%
\definecolor{currentstroke}{rgb}{0.501961,0.501961,0.501961}%
\pgfsetstrokecolor{currentstroke}%
\pgfsetstrokeopacity{0.300000}%
\pgfsetdash{}{0pt}%
\pgfpathmoveto{\pgfqpoint{2.055258in}{0.709782in}}%
\pgfpathlineto{\pgfqpoint{2.055258in}{5.118077in}}%
\pgfusepath{stroke}%
\end{pgfscope}%
\begin{pgfscope}%
\pgfsetbuttcap%
\pgfsetroundjoin%
\definecolor{currentfill}{rgb}{0.000000,0.000000,0.000000}%
\pgfsetfillcolor{currentfill}%
\pgfsetlinewidth{0.803000pt}%
\definecolor{currentstroke}{rgb}{0.000000,0.000000,0.000000}%
\pgfsetstrokecolor{currentstroke}%
\pgfsetdash{}{0pt}%
\pgfsys@defobject{currentmarker}{\pgfqpoint{0.000000in}{-0.048611in}}{\pgfqpoint{0.000000in}{0.000000in}}{%
\pgfpathmoveto{\pgfqpoint{0.000000in}{0.000000in}}%
\pgfpathlineto{\pgfqpoint{0.000000in}{-0.048611in}}%
\pgfusepath{stroke,fill}%
}%
\begin{pgfscope}%
\pgfsys@transformshift{2.055258in}{0.709782in}%
\pgfsys@useobject{currentmarker}{}%
\end{pgfscope}%
\end{pgfscope}%
\begin{pgfscope}%
\definecolor{textcolor}{rgb}{0.000000,0.000000,0.000000}%
\pgfsetstrokecolor{textcolor}%
\pgfsetfillcolor{textcolor}%
\pgftext[x=2.055258in,y=0.612559in,,top]{\color{textcolor}{\rmfamily\fontsize{14.000000}{16.800000}\selectfont\catcode`\^=\active\def^{\ifmmode\sp\else\^{}\fi}\catcode`\%=\active\def%{\%}$\mathdefault{0.2}$}}%
\end{pgfscope}%
\begin{pgfscope}%
\pgfpathrectangle{\pgfqpoint{0.771832in}{0.709782in}}{\pgfqpoint{7.058844in}{4.408296in}}%
\pgfusepath{clip}%
\pgfsetrectcap%
\pgfsetroundjoin%
\pgfsetlinewidth{0.803000pt}%
\definecolor{currentstroke}{rgb}{0.501961,0.501961,0.501961}%
\pgfsetstrokecolor{currentstroke}%
\pgfsetstrokeopacity{0.300000}%
\pgfsetdash{}{0pt}%
\pgfpathmoveto{\pgfqpoint{3.338684in}{0.709782in}}%
\pgfpathlineto{\pgfqpoint{3.338684in}{5.118077in}}%
\pgfusepath{stroke}%
\end{pgfscope}%
\begin{pgfscope}%
\pgfsetbuttcap%
\pgfsetroundjoin%
\definecolor{currentfill}{rgb}{0.000000,0.000000,0.000000}%
\pgfsetfillcolor{currentfill}%
\pgfsetlinewidth{0.803000pt}%
\definecolor{currentstroke}{rgb}{0.000000,0.000000,0.000000}%
\pgfsetstrokecolor{currentstroke}%
\pgfsetdash{}{0pt}%
\pgfsys@defobject{currentmarker}{\pgfqpoint{0.000000in}{-0.048611in}}{\pgfqpoint{0.000000in}{0.000000in}}{%
\pgfpathmoveto{\pgfqpoint{0.000000in}{0.000000in}}%
\pgfpathlineto{\pgfqpoint{0.000000in}{-0.048611in}}%
\pgfusepath{stroke,fill}%
}%
\begin{pgfscope}%
\pgfsys@transformshift{3.338684in}{0.709782in}%
\pgfsys@useobject{currentmarker}{}%
\end{pgfscope}%
\end{pgfscope}%
\begin{pgfscope}%
\definecolor{textcolor}{rgb}{0.000000,0.000000,0.000000}%
\pgfsetstrokecolor{textcolor}%
\pgfsetfillcolor{textcolor}%
\pgftext[x=3.338684in,y=0.612559in,,top]{\color{textcolor}{\rmfamily\fontsize{14.000000}{16.800000}\selectfont\catcode`\^=\active\def^{\ifmmode\sp\else\^{}\fi}\catcode`\%=\active\def%{\%}$\mathdefault{0.4}$}}%
\end{pgfscope}%
\begin{pgfscope}%
\pgfpathrectangle{\pgfqpoint{0.771832in}{0.709782in}}{\pgfqpoint{7.058844in}{4.408296in}}%
\pgfusepath{clip}%
\pgfsetrectcap%
\pgfsetroundjoin%
\pgfsetlinewidth{0.803000pt}%
\definecolor{currentstroke}{rgb}{0.501961,0.501961,0.501961}%
\pgfsetstrokecolor{currentstroke}%
\pgfsetstrokeopacity{0.300000}%
\pgfsetdash{}{0pt}%
\pgfpathmoveto{\pgfqpoint{4.622110in}{0.709782in}}%
\pgfpathlineto{\pgfqpoint{4.622110in}{5.118077in}}%
\pgfusepath{stroke}%
\end{pgfscope}%
\begin{pgfscope}%
\pgfsetbuttcap%
\pgfsetroundjoin%
\definecolor{currentfill}{rgb}{0.000000,0.000000,0.000000}%
\pgfsetfillcolor{currentfill}%
\pgfsetlinewidth{0.803000pt}%
\definecolor{currentstroke}{rgb}{0.000000,0.000000,0.000000}%
\pgfsetstrokecolor{currentstroke}%
\pgfsetdash{}{0pt}%
\pgfsys@defobject{currentmarker}{\pgfqpoint{0.000000in}{-0.048611in}}{\pgfqpoint{0.000000in}{0.000000in}}{%
\pgfpathmoveto{\pgfqpoint{0.000000in}{0.000000in}}%
\pgfpathlineto{\pgfqpoint{0.000000in}{-0.048611in}}%
\pgfusepath{stroke,fill}%
}%
\begin{pgfscope}%
\pgfsys@transformshift{4.622110in}{0.709782in}%
\pgfsys@useobject{currentmarker}{}%
\end{pgfscope}%
\end{pgfscope}%
\begin{pgfscope}%
\definecolor{textcolor}{rgb}{0.000000,0.000000,0.000000}%
\pgfsetstrokecolor{textcolor}%
\pgfsetfillcolor{textcolor}%
\pgftext[x=4.622110in,y=0.612559in,,top]{\color{textcolor}{\rmfamily\fontsize{14.000000}{16.800000}\selectfont\catcode`\^=\active\def^{\ifmmode\sp\else\^{}\fi}\catcode`\%=\active\def%{\%}$\mathdefault{0.6}$}}%
\end{pgfscope}%
\begin{pgfscope}%
\pgfpathrectangle{\pgfqpoint{0.771832in}{0.709782in}}{\pgfqpoint{7.058844in}{4.408296in}}%
\pgfusepath{clip}%
\pgfsetrectcap%
\pgfsetroundjoin%
\pgfsetlinewidth{0.803000pt}%
\definecolor{currentstroke}{rgb}{0.501961,0.501961,0.501961}%
\pgfsetstrokecolor{currentstroke}%
\pgfsetstrokeopacity{0.300000}%
\pgfsetdash{}{0pt}%
\pgfpathmoveto{\pgfqpoint{5.905537in}{0.709782in}}%
\pgfpathlineto{\pgfqpoint{5.905537in}{5.118077in}}%
\pgfusepath{stroke}%
\end{pgfscope}%
\begin{pgfscope}%
\pgfsetbuttcap%
\pgfsetroundjoin%
\definecolor{currentfill}{rgb}{0.000000,0.000000,0.000000}%
\pgfsetfillcolor{currentfill}%
\pgfsetlinewidth{0.803000pt}%
\definecolor{currentstroke}{rgb}{0.000000,0.000000,0.000000}%
\pgfsetstrokecolor{currentstroke}%
\pgfsetdash{}{0pt}%
\pgfsys@defobject{currentmarker}{\pgfqpoint{0.000000in}{-0.048611in}}{\pgfqpoint{0.000000in}{0.000000in}}{%
\pgfpathmoveto{\pgfqpoint{0.000000in}{0.000000in}}%
\pgfpathlineto{\pgfqpoint{0.000000in}{-0.048611in}}%
\pgfusepath{stroke,fill}%
}%
\begin{pgfscope}%
\pgfsys@transformshift{5.905537in}{0.709782in}%
\pgfsys@useobject{currentmarker}{}%
\end{pgfscope}%
\end{pgfscope}%
\begin{pgfscope}%
\definecolor{textcolor}{rgb}{0.000000,0.000000,0.000000}%
\pgfsetstrokecolor{textcolor}%
\pgfsetfillcolor{textcolor}%
\pgftext[x=5.905537in,y=0.612559in,,top]{\color{textcolor}{\rmfamily\fontsize{14.000000}{16.800000}\selectfont\catcode`\^=\active\def^{\ifmmode\sp\else\^{}\fi}\catcode`\%=\active\def%{\%}$\mathdefault{0.8}$}}%
\end{pgfscope}%
\begin{pgfscope}%
\pgfpathrectangle{\pgfqpoint{0.771832in}{0.709782in}}{\pgfqpoint{7.058844in}{4.408296in}}%
\pgfusepath{clip}%
\pgfsetrectcap%
\pgfsetroundjoin%
\pgfsetlinewidth{0.803000pt}%
\definecolor{currentstroke}{rgb}{0.501961,0.501961,0.501961}%
\pgfsetstrokecolor{currentstroke}%
\pgfsetstrokeopacity{0.300000}%
\pgfsetdash{}{0pt}%
\pgfpathmoveto{\pgfqpoint{7.188963in}{0.709782in}}%
\pgfpathlineto{\pgfqpoint{7.188963in}{5.118077in}}%
\pgfusepath{stroke}%
\end{pgfscope}%
\begin{pgfscope}%
\pgfsetbuttcap%
\pgfsetroundjoin%
\definecolor{currentfill}{rgb}{0.000000,0.000000,0.000000}%
\pgfsetfillcolor{currentfill}%
\pgfsetlinewidth{0.803000pt}%
\definecolor{currentstroke}{rgb}{0.000000,0.000000,0.000000}%
\pgfsetstrokecolor{currentstroke}%
\pgfsetdash{}{0pt}%
\pgfsys@defobject{currentmarker}{\pgfqpoint{0.000000in}{-0.048611in}}{\pgfqpoint{0.000000in}{0.000000in}}{%
\pgfpathmoveto{\pgfqpoint{0.000000in}{0.000000in}}%
\pgfpathlineto{\pgfqpoint{0.000000in}{-0.048611in}}%
\pgfusepath{stroke,fill}%
}%
\begin{pgfscope}%
\pgfsys@transformshift{7.188963in}{0.709782in}%
\pgfsys@useobject{currentmarker}{}%
\end{pgfscope}%
\end{pgfscope}%
\begin{pgfscope}%
\definecolor{textcolor}{rgb}{0.000000,0.000000,0.000000}%
\pgfsetstrokecolor{textcolor}%
\pgfsetfillcolor{textcolor}%
\pgftext[x=7.188963in,y=0.612559in,,top]{\color{textcolor}{\rmfamily\fontsize{14.000000}{16.800000}\selectfont\catcode`\^=\active\def^{\ifmmode\sp\else\^{}\fi}\catcode`\%=\active\def%{\%}$\mathdefault{1.0}$}}%
\end{pgfscope}%
\begin{pgfscope}%
\definecolor{textcolor}{rgb}{0.000000,0.000000,0.000000}%
\pgfsetstrokecolor{textcolor}%
\pgfsetfillcolor{textcolor}%
\pgftext[x=4.301254in,y=0.368826in,,top]{\color{textcolor}{\rmfamily\fontsize{20.000000}{24.000000}\selectfont\catcode`\^=\active\def^{\ifmmode\sp\else\^{}\fi}\catcode`\%=\active\def%{\%}x$_1$}}%
\end{pgfscope}%
\begin{pgfscope}%
\pgfpathrectangle{\pgfqpoint{0.771832in}{0.709782in}}{\pgfqpoint{7.058844in}{4.408296in}}%
\pgfusepath{clip}%
\pgfsetrectcap%
\pgfsetroundjoin%
\pgfsetlinewidth{0.803000pt}%
\definecolor{currentstroke}{rgb}{0.501961,0.501961,0.501961}%
\pgfsetstrokecolor{currentstroke}%
\pgfsetstrokeopacity{0.300000}%
\pgfsetdash{}{0pt}%
\pgfpathmoveto{\pgfqpoint{0.771832in}{0.709782in}}%
\pgfpathlineto{\pgfqpoint{7.830676in}{0.709782in}}%
\pgfusepath{stroke}%
\end{pgfscope}%
\begin{pgfscope}%
\pgfsetbuttcap%
\pgfsetroundjoin%
\definecolor{currentfill}{rgb}{0.000000,0.000000,0.000000}%
\pgfsetfillcolor{currentfill}%
\pgfsetlinewidth{0.803000pt}%
\definecolor{currentstroke}{rgb}{0.000000,0.000000,0.000000}%
\pgfsetstrokecolor{currentstroke}%
\pgfsetdash{}{0pt}%
\pgfsys@defobject{currentmarker}{\pgfqpoint{-0.048611in}{0.000000in}}{\pgfqpoint{-0.000000in}{0.000000in}}{%
\pgfpathmoveto{\pgfqpoint{-0.000000in}{0.000000in}}%
\pgfpathlineto{\pgfqpoint{-0.048611in}{0.000000in}}%
\pgfusepath{stroke,fill}%
}%
\begin{pgfscope}%
\pgfsys@transformshift{0.771832in}{0.709782in}%
\pgfsys@useobject{currentmarker}{}%
\end{pgfscope}%
\end{pgfscope}%
\begin{pgfscope}%
\definecolor{textcolor}{rgb}{0.000000,0.000000,0.000000}%
\pgfsetstrokecolor{textcolor}%
\pgfsetfillcolor{textcolor}%
\pgftext[x=0.424381in, y=0.635915in, left, base]{\color{textcolor}{\rmfamily\fontsize{14.000000}{16.800000}\selectfont\catcode`\^=\active\def^{\ifmmode\sp\else\^{}\fi}\catcode`\%=\active\def%{\%}$\mathdefault{0.0}$}}%
\end{pgfscope}%
\begin{pgfscope}%
\pgfpathrectangle{\pgfqpoint{0.771832in}{0.709782in}}{\pgfqpoint{7.058844in}{4.408296in}}%
\pgfusepath{clip}%
\pgfsetrectcap%
\pgfsetroundjoin%
\pgfsetlinewidth{0.803000pt}%
\definecolor{currentstroke}{rgb}{0.501961,0.501961,0.501961}%
\pgfsetstrokecolor{currentstroke}%
\pgfsetstrokeopacity{0.300000}%
\pgfsetdash{}{0pt}%
\pgfpathmoveto{\pgfqpoint{0.771832in}{1.511290in}}%
\pgfpathlineto{\pgfqpoint{7.830676in}{1.511290in}}%
\pgfusepath{stroke}%
\end{pgfscope}%
\begin{pgfscope}%
\pgfsetbuttcap%
\pgfsetroundjoin%
\definecolor{currentfill}{rgb}{0.000000,0.000000,0.000000}%
\pgfsetfillcolor{currentfill}%
\pgfsetlinewidth{0.803000pt}%
\definecolor{currentstroke}{rgb}{0.000000,0.000000,0.000000}%
\pgfsetstrokecolor{currentstroke}%
\pgfsetdash{}{0pt}%
\pgfsys@defobject{currentmarker}{\pgfqpoint{-0.048611in}{0.000000in}}{\pgfqpoint{-0.000000in}{0.000000in}}{%
\pgfpathmoveto{\pgfqpoint{-0.000000in}{0.000000in}}%
\pgfpathlineto{\pgfqpoint{-0.048611in}{0.000000in}}%
\pgfusepath{stroke,fill}%
}%
\begin{pgfscope}%
\pgfsys@transformshift{0.771832in}{1.511290in}%
\pgfsys@useobject{currentmarker}{}%
\end{pgfscope}%
\end{pgfscope}%
\begin{pgfscope}%
\definecolor{textcolor}{rgb}{0.000000,0.000000,0.000000}%
\pgfsetstrokecolor{textcolor}%
\pgfsetfillcolor{textcolor}%
\pgftext[x=0.424381in, y=1.437424in, left, base]{\color{textcolor}{\rmfamily\fontsize{14.000000}{16.800000}\selectfont\catcode`\^=\active\def^{\ifmmode\sp\else\^{}\fi}\catcode`\%=\active\def%{\%}$\mathdefault{0.2}$}}%
\end{pgfscope}%
\begin{pgfscope}%
\pgfpathrectangle{\pgfqpoint{0.771832in}{0.709782in}}{\pgfqpoint{7.058844in}{4.408296in}}%
\pgfusepath{clip}%
\pgfsetrectcap%
\pgfsetroundjoin%
\pgfsetlinewidth{0.803000pt}%
\definecolor{currentstroke}{rgb}{0.501961,0.501961,0.501961}%
\pgfsetstrokecolor{currentstroke}%
\pgfsetstrokeopacity{0.300000}%
\pgfsetdash{}{0pt}%
\pgfpathmoveto{\pgfqpoint{0.771832in}{2.312798in}}%
\pgfpathlineto{\pgfqpoint{7.830676in}{2.312798in}}%
\pgfusepath{stroke}%
\end{pgfscope}%
\begin{pgfscope}%
\pgfsetbuttcap%
\pgfsetroundjoin%
\definecolor{currentfill}{rgb}{0.000000,0.000000,0.000000}%
\pgfsetfillcolor{currentfill}%
\pgfsetlinewidth{0.803000pt}%
\definecolor{currentstroke}{rgb}{0.000000,0.000000,0.000000}%
\pgfsetstrokecolor{currentstroke}%
\pgfsetdash{}{0pt}%
\pgfsys@defobject{currentmarker}{\pgfqpoint{-0.048611in}{0.000000in}}{\pgfqpoint{-0.000000in}{0.000000in}}{%
\pgfpathmoveto{\pgfqpoint{-0.000000in}{0.000000in}}%
\pgfpathlineto{\pgfqpoint{-0.048611in}{0.000000in}}%
\pgfusepath{stroke,fill}%
}%
\begin{pgfscope}%
\pgfsys@transformshift{0.771832in}{2.312798in}%
\pgfsys@useobject{currentmarker}{}%
\end{pgfscope}%
\end{pgfscope}%
\begin{pgfscope}%
\definecolor{textcolor}{rgb}{0.000000,0.000000,0.000000}%
\pgfsetstrokecolor{textcolor}%
\pgfsetfillcolor{textcolor}%
\pgftext[x=0.424381in, y=2.238932in, left, base]{\color{textcolor}{\rmfamily\fontsize{14.000000}{16.800000}\selectfont\catcode`\^=\active\def^{\ifmmode\sp\else\^{}\fi}\catcode`\%=\active\def%{\%}$\mathdefault{0.4}$}}%
\end{pgfscope}%
\begin{pgfscope}%
\pgfpathrectangle{\pgfqpoint{0.771832in}{0.709782in}}{\pgfqpoint{7.058844in}{4.408296in}}%
\pgfusepath{clip}%
\pgfsetrectcap%
\pgfsetroundjoin%
\pgfsetlinewidth{0.803000pt}%
\definecolor{currentstroke}{rgb}{0.501961,0.501961,0.501961}%
\pgfsetstrokecolor{currentstroke}%
\pgfsetstrokeopacity{0.300000}%
\pgfsetdash{}{0pt}%
\pgfpathmoveto{\pgfqpoint{0.771832in}{3.114307in}}%
\pgfpathlineto{\pgfqpoint{7.830676in}{3.114307in}}%
\pgfusepath{stroke}%
\end{pgfscope}%
\begin{pgfscope}%
\pgfsetbuttcap%
\pgfsetroundjoin%
\definecolor{currentfill}{rgb}{0.000000,0.000000,0.000000}%
\pgfsetfillcolor{currentfill}%
\pgfsetlinewidth{0.803000pt}%
\definecolor{currentstroke}{rgb}{0.000000,0.000000,0.000000}%
\pgfsetstrokecolor{currentstroke}%
\pgfsetdash{}{0pt}%
\pgfsys@defobject{currentmarker}{\pgfqpoint{-0.048611in}{0.000000in}}{\pgfqpoint{-0.000000in}{0.000000in}}{%
\pgfpathmoveto{\pgfqpoint{-0.000000in}{0.000000in}}%
\pgfpathlineto{\pgfqpoint{-0.048611in}{0.000000in}}%
\pgfusepath{stroke,fill}%
}%
\begin{pgfscope}%
\pgfsys@transformshift{0.771832in}{3.114307in}%
\pgfsys@useobject{currentmarker}{}%
\end{pgfscope}%
\end{pgfscope}%
\begin{pgfscope}%
\definecolor{textcolor}{rgb}{0.000000,0.000000,0.000000}%
\pgfsetstrokecolor{textcolor}%
\pgfsetfillcolor{textcolor}%
\pgftext[x=0.424381in, y=3.040440in, left, base]{\color{textcolor}{\rmfamily\fontsize{14.000000}{16.800000}\selectfont\catcode`\^=\active\def^{\ifmmode\sp\else\^{}\fi}\catcode`\%=\active\def%{\%}$\mathdefault{0.6}$}}%
\end{pgfscope}%
\begin{pgfscope}%
\pgfpathrectangle{\pgfqpoint{0.771832in}{0.709782in}}{\pgfqpoint{7.058844in}{4.408296in}}%
\pgfusepath{clip}%
\pgfsetrectcap%
\pgfsetroundjoin%
\pgfsetlinewidth{0.803000pt}%
\definecolor{currentstroke}{rgb}{0.501961,0.501961,0.501961}%
\pgfsetstrokecolor{currentstroke}%
\pgfsetstrokeopacity{0.300000}%
\pgfsetdash{}{0pt}%
\pgfpathmoveto{\pgfqpoint{0.771832in}{3.915815in}}%
\pgfpathlineto{\pgfqpoint{7.830676in}{3.915815in}}%
\pgfusepath{stroke}%
\end{pgfscope}%
\begin{pgfscope}%
\pgfsetbuttcap%
\pgfsetroundjoin%
\definecolor{currentfill}{rgb}{0.000000,0.000000,0.000000}%
\pgfsetfillcolor{currentfill}%
\pgfsetlinewidth{0.803000pt}%
\definecolor{currentstroke}{rgb}{0.000000,0.000000,0.000000}%
\pgfsetstrokecolor{currentstroke}%
\pgfsetdash{}{0pt}%
\pgfsys@defobject{currentmarker}{\pgfqpoint{-0.048611in}{0.000000in}}{\pgfqpoint{-0.000000in}{0.000000in}}{%
\pgfpathmoveto{\pgfqpoint{-0.000000in}{0.000000in}}%
\pgfpathlineto{\pgfqpoint{-0.048611in}{0.000000in}}%
\pgfusepath{stroke,fill}%
}%
\begin{pgfscope}%
\pgfsys@transformshift{0.771832in}{3.915815in}%
\pgfsys@useobject{currentmarker}{}%
\end{pgfscope}%
\end{pgfscope}%
\begin{pgfscope}%
\definecolor{textcolor}{rgb}{0.000000,0.000000,0.000000}%
\pgfsetstrokecolor{textcolor}%
\pgfsetfillcolor{textcolor}%
\pgftext[x=0.424381in, y=3.841949in, left, base]{\color{textcolor}{\rmfamily\fontsize{14.000000}{16.800000}\selectfont\catcode`\^=\active\def^{\ifmmode\sp\else\^{}\fi}\catcode`\%=\active\def%{\%}$\mathdefault{0.8}$}}%
\end{pgfscope}%
\begin{pgfscope}%
\pgfpathrectangle{\pgfqpoint{0.771832in}{0.709782in}}{\pgfqpoint{7.058844in}{4.408296in}}%
\pgfusepath{clip}%
\pgfsetrectcap%
\pgfsetroundjoin%
\pgfsetlinewidth{0.803000pt}%
\definecolor{currentstroke}{rgb}{0.501961,0.501961,0.501961}%
\pgfsetstrokecolor{currentstroke}%
\pgfsetstrokeopacity{0.300000}%
\pgfsetdash{}{0pt}%
\pgfpathmoveto{\pgfqpoint{0.771832in}{4.717323in}}%
\pgfpathlineto{\pgfqpoint{7.830676in}{4.717323in}}%
\pgfusepath{stroke}%
\end{pgfscope}%
\begin{pgfscope}%
\pgfsetbuttcap%
\pgfsetroundjoin%
\definecolor{currentfill}{rgb}{0.000000,0.000000,0.000000}%
\pgfsetfillcolor{currentfill}%
\pgfsetlinewidth{0.803000pt}%
\definecolor{currentstroke}{rgb}{0.000000,0.000000,0.000000}%
\pgfsetstrokecolor{currentstroke}%
\pgfsetdash{}{0pt}%
\pgfsys@defobject{currentmarker}{\pgfqpoint{-0.048611in}{0.000000in}}{\pgfqpoint{-0.000000in}{0.000000in}}{%
\pgfpathmoveto{\pgfqpoint{-0.000000in}{0.000000in}}%
\pgfpathlineto{\pgfqpoint{-0.048611in}{0.000000in}}%
\pgfusepath{stroke,fill}%
}%
\begin{pgfscope}%
\pgfsys@transformshift{0.771832in}{4.717323in}%
\pgfsys@useobject{currentmarker}{}%
\end{pgfscope}%
\end{pgfscope}%
\begin{pgfscope}%
\definecolor{textcolor}{rgb}{0.000000,0.000000,0.000000}%
\pgfsetstrokecolor{textcolor}%
\pgfsetfillcolor{textcolor}%
\pgftext[x=0.424381in, y=4.643457in, left, base]{\color{textcolor}{\rmfamily\fontsize{14.000000}{16.800000}\selectfont\catcode`\^=\active\def^{\ifmmode\sp\else\^{}\fi}\catcode`\%=\active\def%{\%}$\mathdefault{1.0}$}}%
\end{pgfscope}%
\begin{pgfscope}%
\definecolor{textcolor}{rgb}{0.000000,0.000000,0.000000}%
\pgfsetstrokecolor{textcolor}%
\pgfsetfillcolor{textcolor}%
\pgftext[x=0.368826in,y=2.913929in,,bottom,rotate=90.000000]{\color{textcolor}{\rmfamily\fontsize{20.000000}{24.000000}\selectfont\catcode`\^=\active\def^{\ifmmode\sp\else\^{}\fi}\catcode`\%=\active\def%{\%}x$_2$}}%
\end{pgfscope}%
\begin{pgfscope}%
\pgfpathrectangle{\pgfqpoint{0.771832in}{0.709782in}}{\pgfqpoint{7.058844in}{4.408296in}}%
\pgfusepath{clip}%
\pgfsetrectcap%
\pgfsetroundjoin%
\pgfsetlinewidth{1.505625pt}%
\definecolor{currentstroke}{rgb}{0.000000,0.000000,0.000000}%
\pgfsetstrokecolor{currentstroke}%
\pgfsetdash{}{0pt}%
\pgfpathmoveto{\pgfqpoint{7.188963in}{0.709782in}}%
\pgfpathlineto{\pgfqpoint{0.771832in}{4.717323in}}%
\pgfusepath{stroke}%
\end{pgfscope}%
\begin{pgfscope}%
\pgfpathrectangle{\pgfqpoint{0.771832in}{0.709782in}}{\pgfqpoint{7.058844in}{4.408296in}}%
\pgfusepath{clip}%
\pgfsetrectcap%
\pgfsetroundjoin%
\pgfsetlinewidth{1.505625pt}%
\definecolor{currentstroke}{rgb}{0.501961,0.501961,0.501961}%
\pgfsetstrokecolor{currentstroke}%
\pgfsetdash{}{0pt}%
\pgfpathmoveto{\pgfqpoint{7.188963in}{0.709782in}}%
\pgfpathlineto{\pgfqpoint{0.771832in}{3.915815in}}%
\pgfusepath{stroke}%
\end{pgfscope}%
\begin{pgfscope}%
\pgfpathrectangle{\pgfqpoint{0.771832in}{0.709782in}}{\pgfqpoint{7.058844in}{4.408296in}}%
\pgfusepath{clip}%
\pgfsetbuttcap%
\pgfsetroundjoin%
\pgfsetlinewidth{1.505625pt}%
\definecolor{currentstroke}{rgb}{0.501961,0.501961,0.501961}%
\pgfsetstrokecolor{currentstroke}%
\pgfsetdash{{5.550000pt}{2.400000pt}}{0.000000pt}%
\pgfpathmoveto{\pgfqpoint{7.830676in}{0.709782in}}%
\pgfpathlineto{\pgfqpoint{0.771832in}{4.236418in}}%
\pgfusepath{stroke}%
\end{pgfscope}%
\begin{pgfscope}%
\pgfsetrectcap%
\pgfsetmiterjoin%
\pgfsetlinewidth{0.803000pt}%
\definecolor{currentstroke}{rgb}{0.000000,0.000000,0.000000}%
\pgfsetstrokecolor{currentstroke}%
\pgfsetdash{}{0pt}%
\pgfpathmoveto{\pgfqpoint{0.771832in}{0.709782in}}%
\pgfpathlineto{\pgfqpoint{0.771832in}{5.118077in}}%
\pgfusepath{stroke}%
\end{pgfscope}%
\begin{pgfscope}%
\pgfsetrectcap%
\pgfsetmiterjoin%
\pgfsetlinewidth{0.803000pt}%
\definecolor{currentstroke}{rgb}{0.000000,0.000000,0.000000}%
\pgfsetstrokecolor{currentstroke}%
\pgfsetdash{}{0pt}%
\pgfpathmoveto{\pgfqpoint{7.830676in}{0.709782in}}%
\pgfpathlineto{\pgfqpoint{7.830676in}{5.118077in}}%
\pgfusepath{stroke}%
\end{pgfscope}%
\begin{pgfscope}%
\pgfsetrectcap%
\pgfsetmiterjoin%
\pgfsetlinewidth{0.803000pt}%
\definecolor{currentstroke}{rgb}{0.000000,0.000000,0.000000}%
\pgfsetstrokecolor{currentstroke}%
\pgfsetdash{}{0pt}%
\pgfpathmoveto{\pgfqpoint{0.771832in}{0.709782in}}%
\pgfpathlineto{\pgfqpoint{7.830676in}{0.709782in}}%
\pgfusepath{stroke}%
\end{pgfscope}%
\begin{pgfscope}%
\pgfsetrectcap%
\pgfsetmiterjoin%
\pgfsetlinewidth{0.803000pt}%
\definecolor{currentstroke}{rgb}{0.000000,0.000000,0.000000}%
\pgfsetstrokecolor{currentstroke}%
\pgfsetdash{}{0pt}%
\pgfpathmoveto{\pgfqpoint{0.771832in}{5.118077in}}%
\pgfpathlineto{\pgfqpoint{7.830676in}{5.118077in}}%
\pgfusepath{stroke}%
\end{pgfscope}%
\begin{pgfscope}%
\pgfsetroundcap%
\pgfsetroundjoin%
\pgfsetlinewidth{1.003750pt}%
\definecolor{currentstroke}{rgb}{0.000000,0.000000,0.000000}%
\pgfsetstrokecolor{currentstroke}%
\pgfsetdash{}{0pt}%
\pgfpathmoveto{\pgfqpoint{5.449588in}{1.037174in}}%
\pgfpathquadraticcurveto{\pgfqpoint{6.305624in}{0.876048in}}{\pgfqpoint{7.151295in}{0.716871in}}%
\pgfusepath{stroke}%
\end{pgfscope}%
\begin{pgfscope}%
\pgfsetroundcap%
\pgfsetroundjoin%
\pgfsetlinewidth{1.003750pt}%
\definecolor{currentstroke}{rgb}{0.000000,0.000000,0.000000}%
\pgfsetstrokecolor{currentstroke}%
\pgfsetdash{}{0pt}%
\pgfpathmoveto{\pgfqpoint{7.087448in}{0.798140in}}%
\pgfpathlineto{\pgfqpoint{7.151295in}{0.716871in}}%
\pgfpathlineto{\pgfqpoint{7.062271in}{0.664377in}}%
\pgfusepath{stroke}%
\end{pgfscope}%
\begin{pgfscope}%
\definecolor{textcolor}{rgb}{0.000000,0.000000,0.000000}%
\pgfsetstrokecolor{textcolor}%
\pgfsetfillcolor{textcolor}%
\pgftext[x=3.980397in,y=1.110536in,left,base]{\color{textcolor}{\rmfamily\fontsize{14.000000}{16.800000}\selectfont\catcode`\^=\active\def^{\ifmmode\sp\else\^{}\fi}\catcode`\%=\active\def%{\%}Optimum: (1, 0)}}%
\end{pgfscope}%
\begin{pgfscope}%
\pgfsetroundcap%
\pgfsetroundjoin%
\pgfsetlinewidth{1.003750pt}%
\definecolor{currentstroke}{rgb}{0.000000,0.000000,0.000000}%
\pgfsetstrokecolor{currentstroke}%
\pgfsetdash{}{0pt}%
\pgfpathmoveto{\pgfqpoint{5.949744in}{2.638706in}}%
\pgfpathquadraticcurveto{\pgfqpoint{5.299414in}{2.479063in}}{\pgfqpoint{4.659326in}{2.321934in}}%
\pgfusepath{stroke}%
\end{pgfscope}%
\begin{pgfscope}%
\pgfsetroundcap%
\pgfsetroundjoin%
\pgfsetlinewidth{1.003750pt}%
\definecolor{currentstroke}{rgb}{0.000000,0.000000,0.000000}%
\pgfsetstrokecolor{currentstroke}%
\pgfsetdash{}{0pt}%
\pgfpathmoveto{\pgfqpoint{4.751085in}{2.274383in}}%
\pgfpathlineto{\pgfqpoint{4.659326in}{2.321934in}}%
\pgfpathlineto{\pgfqpoint{4.718636in}{2.406569in}}%
\pgfusepath{stroke}%
\end{pgfscope}%
\begin{pgfscope}%
\definecolor{textcolor}{rgb}{0.000000,0.000000,0.000000}%
\pgfsetstrokecolor{textcolor}%
\pgfsetfillcolor{textcolor}%
\pgftext[x=5.263823in,y=2.713552in,left,base]{\color{textcolor}{\rmfamily\fontsize{14.000000}{16.800000}\selectfont\catcode`\^=\active\def^{\ifmmode\sp\else\^{}\fi}\catcode`\%=\active\def%{\%}MGA Solution: (0.6, 0.4)}}%
\end{pgfscope}%
\begin{pgfscope}%
\definecolor{textcolor}{rgb}{0.000000,0.000000,0.000000}%
\pgfsetstrokecolor{textcolor}%
\pgfsetfillcolor{textcolor}%
\pgftext[x=4.301254in,y=5.201411in,,base]{\color{textcolor}{\rmfamily\fontsize{20.000000}{24.000000}\selectfont\catcode`\^=\active\def^{\ifmmode\sp\else\^{}\fi}\catcode`\%=\active\def%{\%}Design Space}}%
\end{pgfscope}%
\begin{pgfscope}%
\pgfsetbuttcap%
\pgfsetmiterjoin%
\definecolor{currentfill}{rgb}{0.300000,0.300000,0.300000}%
\pgfsetfillcolor{currentfill}%
\pgfsetfillopacity{0.500000}%
\pgfsetlinewidth{1.003750pt}%
\definecolor{currentstroke}{rgb}{0.300000,0.300000,0.300000}%
\pgfsetstrokecolor{currentstroke}%
\pgfsetstrokeopacity{0.500000}%
\pgfsetdash{}{0pt}%
\pgfpathmoveto{\pgfqpoint{4.681326in}{3.934007in}}%
\pgfpathlineto{\pgfqpoint{7.702898in}{3.934007in}}%
\pgfpathquadraticcurveto{\pgfqpoint{7.747343in}{3.934007in}}{\pgfqpoint{7.747343in}{3.978451in}}%
\pgfpathlineto{\pgfqpoint{7.747343in}{4.934744in}}%
\pgfpathquadraticcurveto{\pgfqpoint{7.747343in}{4.979189in}}{\pgfqpoint{7.702898in}{4.979189in}}%
\pgfpathlineto{\pgfqpoint{4.681326in}{4.979189in}}%
\pgfpathquadraticcurveto{\pgfqpoint{4.636882in}{4.979189in}}{\pgfqpoint{4.636882in}{4.934744in}}%
\pgfpathlineto{\pgfqpoint{4.636882in}{3.978451in}}%
\pgfpathquadraticcurveto{\pgfqpoint{4.636882in}{3.934007in}}{\pgfqpoint{4.681326in}{3.934007in}}%
\pgfpathlineto{\pgfqpoint{4.681326in}{3.934007in}}%
\pgfpathclose%
\pgfusepath{stroke,fill}%
\end{pgfscope}%
\begin{pgfscope}%
\pgfsetbuttcap%
\pgfsetmiterjoin%
\definecolor{currentfill}{rgb}{1.000000,1.000000,1.000000}%
\pgfsetfillcolor{currentfill}%
\pgfsetlinewidth{1.003750pt}%
\definecolor{currentstroke}{rgb}{0.800000,0.800000,0.800000}%
\pgfsetstrokecolor{currentstroke}%
\pgfsetdash{}{0pt}%
\pgfpathmoveto{\pgfqpoint{4.653548in}{3.961785in}}%
\pgfpathlineto{\pgfqpoint{7.675120in}{3.961785in}}%
\pgfpathquadraticcurveto{\pgfqpoint{7.719565in}{3.961785in}}{\pgfqpoint{7.719565in}{4.006229in}}%
\pgfpathlineto{\pgfqpoint{7.719565in}{4.962522in}}%
\pgfpathquadraticcurveto{\pgfqpoint{7.719565in}{5.006966in}}{\pgfqpoint{7.675120in}{5.006966in}}%
\pgfpathlineto{\pgfqpoint{4.653548in}{5.006966in}}%
\pgfpathquadraticcurveto{\pgfqpoint{4.609104in}{5.006966in}}{\pgfqpoint{4.609104in}{4.962522in}}%
\pgfpathlineto{\pgfqpoint{4.609104in}{4.006229in}}%
\pgfpathquadraticcurveto{\pgfqpoint{4.609104in}{3.961785in}}{\pgfqpoint{4.653548in}{3.961785in}}%
\pgfpathlineto{\pgfqpoint{4.653548in}{3.961785in}}%
\pgfpathclose%
\pgfusepath{stroke,fill}%
\end{pgfscope}%
\begin{pgfscope}%
\pgfsetrectcap%
\pgfsetroundjoin%
\pgfsetlinewidth{1.505625pt}%
\definecolor{currentstroke}{rgb}{0.000000,0.000000,0.000000}%
\pgfsetstrokecolor{currentstroke}%
\pgfsetdash{}{0pt}%
\pgfpathmoveto{\pgfqpoint{4.697993in}{4.827018in}}%
\pgfpathlineto{\pgfqpoint{4.920215in}{4.827018in}}%
\pgfpathlineto{\pgfqpoint{5.142437in}{4.827018in}}%
\pgfusepath{stroke}%
\end{pgfscope}%
\begin{pgfscope}%
\definecolor{textcolor}{rgb}{0.000000,0.000000,0.000000}%
\pgfsetstrokecolor{textcolor}%
\pgfsetfillcolor{textcolor}%
\pgftext[x=5.320215in,y=4.749241in,left,base]{\color{textcolor}{\rmfamily\fontsize{16.000000}{19.200000}\selectfont\catcode`\^=\active\def^{\ifmmode\sp\else\^{}\fi}\catcode`\%=\active\def%{\%}x$_1$ + x$_2$ = 1}}%
\end{pgfscope}%
\begin{pgfscope}%
\pgfsetrectcap%
\pgfsetroundjoin%
\pgfsetlinewidth{1.505625pt}%
\definecolor{currentstroke}{rgb}{0.501961,0.501961,0.501961}%
\pgfsetstrokecolor{currentstroke}%
\pgfsetdash{}{0pt}%
\pgfpathmoveto{\pgfqpoint{4.697993in}{4.500847in}}%
\pgfpathlineto{\pgfqpoint{4.920215in}{4.500847in}}%
\pgfpathlineto{\pgfqpoint{5.142437in}{4.500847in}}%
\pgfusepath{stroke}%
\end{pgfscope}%
\begin{pgfscope}%
\definecolor{textcolor}{rgb}{0.000000,0.000000,0.000000}%
\pgfsetstrokecolor{textcolor}%
\pgfsetfillcolor{textcolor}%
\pgftext[x=5.320215in,y=4.423069in,left,base]{\color{textcolor}{\rmfamily\fontsize{16.000000}{19.200000}\selectfont\catcode`\^=\active\def^{\ifmmode\sp\else\^{}\fi}\catcode`\%=\active\def%{\%}min(c$_1$x$_1$ + c$_2$x$_2$)}}%
\end{pgfscope}%
\begin{pgfscope}%
\pgfsetbuttcap%
\pgfsetroundjoin%
\pgfsetlinewidth{1.505625pt}%
\definecolor{currentstroke}{rgb}{0.501961,0.501961,0.501961}%
\pgfsetstrokecolor{currentstroke}%
\pgfsetdash{{5.550000pt}{2.400000pt}}{0.000000pt}%
\pgfpathmoveto{\pgfqpoint{4.697993in}{4.174675in}}%
\pgfpathlineto{\pgfqpoint{4.920215in}{4.174675in}}%
\pgfpathlineto{\pgfqpoint{5.142437in}{4.174675in}}%
\pgfusepath{stroke}%
\end{pgfscope}%
\begin{pgfscope}%
\definecolor{textcolor}{rgb}{0.000000,0.000000,0.000000}%
\pgfsetstrokecolor{textcolor}%
\pgfsetfillcolor{textcolor}%
\pgftext[x=5.320215in,y=4.096897in,left,base]{\color{textcolor}{\rmfamily\fontsize{16.000000}{19.200000}\selectfont\catcode`\^=\active\def^{\ifmmode\sp\else\^{}\fi}\catcode`\%=\active\def%{\%}c$_1$x$_1$ + c$_2$x$_2$ $\leq c_1\cdot$slack}}%
\end{pgfscope}%
\begin{pgfscope}%
\definecolor{textcolor}{rgb}{0.000000,0.000000,0.000000}%
\pgfsetstrokecolor{textcolor}%
\pgfsetfillcolor{textcolor}%
\pgftext[x=3.980676in,y=5.868802in,,top]{\color{textcolor}{\rmfamily\fontsize{24.000000}{28.800000}\selectfont\catcode`\^=\active\def^{\ifmmode\sp\else\^{}\fi}\catcode`\%=\active\def%{\%}Modeling-to-Generate-Alternatives}}%
\end{pgfscope}%
\end{pgfpicture}%
\makeatother%
\endgroup%
}
  \caption{Simple demonstration of the standard \ac{mga} algorithm.}
  \label{fig:standard_mga}
\end{figure}

This procedure results in a small set of maximally different solutions for
modelers to interpret. In this way, \ac{mga} efficiently proposes alternatives
that may capture unmodeled objectives, such as political expediency or social
acceptance. However, this method depends on a single objective function which
does not guarantee that these alternative solutions will be optimal or
near-optimal for any other measurable objective.


\ac{pygen} was an initial exploration on repeatable analysis and functionality
extension for an existing \ac{esom}. While successful in that regard, \ac{pygen}
could not overcome \ac{temoa}'s inherent limitations on optimizing multiple
objectives and the inability to modify its objective function. Addressing these
limits led to the development of \ac{osier}.


\subsection{Data for benchmark problems}
 In order to verify \ac{osier}'s accuracy, this section analyzes an energy
system and compare the results against a representative \ac{esom}, \ac{temoa}.
For this problem, I chose to model the state of Illinois broadly and using
weather data from the Champaign-Urbana region due to its geographic centrality.
Chapter \ref{chapter:benchmark-results} presents the results from this problem,
with a variety of optimization criteria. This section describes the data used in
both models. The basic inputs for \ac{osier} and \ac{temoa} are
\begin{enumerate}
    \item Time series data for
    \begin{itemize}
      \item electricity demand
      \item \ac{vre} production (e.g., solar or wind),
    \end{itemize} 
    \item and technology data.
\end{enumerate}
\noindent
The time series data for electricity demand, wind energy, and solar energy, come
from \ac{uiuc}. All of the time series are averaged across several years to
simulate a ``typical'' year. I re-scaled the demand data by the total energy
demand for Illinois in order for the hourly demand to be on the same scale as
the default power units (MW) for \ac{osier} technologies. However, this
normalization choice is somewhat arbitrary. \ac{osier} automatically normalizes
the \ac{vre} time series because \ac{vre} capacity is a decision variable.
Figure \ref{fig:normalized_ldc} shows the normalized demand and load duration
curves.


 \begin{figure}[h]
  \centering
  \resizebox{1\columnwidth}{!}{%% Creator: Matplotlib, PGF backend
%%
%% To include the figure in your LaTeX document, write
%%   \input{<filename>.pgf}
%%
%% Make sure the required packages are loaded in your preamble
%%   \usepackage{pgf}
%%
%% Also ensure that all the required font packages are loaded; for instance,
%% the lmodern package is sometimes necessary when using math font.
%%   \usepackage{lmodern}
%%
%% Figures using additional raster images can only be included by \input if
%% they are in the same directory as the main LaTeX file. For loading figures
%% from other directories you can use the `import` package
%%   \usepackage{import}
%%
%% and then include the figures with
%%   \import{<path to file>}{<filename>.pgf}
%%
%% Matplotlib used the following preamble
%%   \def\mathdefault#1{#1}
%%   \everymath=\expandafter{\the\everymath\displaystyle}
%%   \IfFileExists{scrextend.sty}{
%%     \usepackage[fontsize=10.000000pt]{scrextend}
%%   }{
%%     \renewcommand{\normalsize}{\fontsize{10.000000}{12.000000}\selectfont}
%%     \normalsize
%%   }
%%   
%%   \makeatletter\@ifpackageloaded{underscore}{}{\usepackage[strings]{underscore}}\makeatother
%%
\begingroup%
\makeatletter%
\begin{pgfpicture}%
\pgfpathrectangle{\pgfpointorigin}{\pgfqpoint{11.893610in}{5.900000in}}%
\pgfusepath{use as bounding box, clip}%
\begin{pgfscope}%
\pgfsetbuttcap%
\pgfsetmiterjoin%
\definecolor{currentfill}{rgb}{1.000000,1.000000,1.000000}%
\pgfsetfillcolor{currentfill}%
\pgfsetlinewidth{0.000000pt}%
\definecolor{currentstroke}{rgb}{0.000000,0.000000,0.000000}%
\pgfsetstrokecolor{currentstroke}%
\pgfsetdash{}{0pt}%
\pgfpathmoveto{\pgfqpoint{0.000000in}{0.000000in}}%
\pgfpathlineto{\pgfqpoint{11.893610in}{0.000000in}}%
\pgfpathlineto{\pgfqpoint{11.893610in}{5.900000in}}%
\pgfpathlineto{\pgfqpoint{0.000000in}{5.900000in}}%
\pgfpathlineto{\pgfqpoint{0.000000in}{0.000000in}}%
\pgfpathclose%
\pgfusepath{fill}%
\end{pgfscope}%
\begin{pgfscope}%
\pgfsetbuttcap%
\pgfsetmiterjoin%
\definecolor{currentfill}{rgb}{1.000000,1.000000,1.000000}%
\pgfsetfillcolor{currentfill}%
\pgfsetlinewidth{0.000000pt}%
\definecolor{currentstroke}{rgb}{0.000000,0.000000,0.000000}%
\pgfsetstrokecolor{currentstroke}%
\pgfsetstrokeopacity{0.000000}%
\pgfsetdash{}{0pt}%
\pgfpathmoveto{\pgfqpoint{0.742589in}{0.670138in}}%
\pgfpathlineto{\pgfqpoint{8.410881in}{0.670138in}}%
\pgfpathlineto{\pgfqpoint{8.410881in}{5.516628in}}%
\pgfpathlineto{\pgfqpoint{0.742589in}{5.516628in}}%
\pgfpathlineto{\pgfqpoint{0.742589in}{0.670138in}}%
\pgfpathclose%
\pgfusepath{fill}%
\end{pgfscope}%
\begin{pgfscope}%
\pgfpathrectangle{\pgfqpoint{0.742589in}{0.670138in}}{\pgfqpoint{7.668292in}{4.846490in}}%
\pgfusepath{clip}%
\pgfsetrectcap%
\pgfsetroundjoin%
\pgfsetlinewidth{0.803000pt}%
\definecolor{currentstroke}{rgb}{0.690196,0.690196,0.690196}%
\pgfsetstrokecolor{currentstroke}%
\pgfsetdash{}{0pt}%
\pgfpathmoveto{\pgfqpoint{0.742589in}{0.670138in}}%
\pgfpathlineto{\pgfqpoint{0.742589in}{5.516628in}}%
\pgfusepath{stroke}%
\end{pgfscope}%
\begin{pgfscope}%
\pgfsetbuttcap%
\pgfsetroundjoin%
\definecolor{currentfill}{rgb}{0.000000,0.000000,0.000000}%
\pgfsetfillcolor{currentfill}%
\pgfsetlinewidth{0.803000pt}%
\definecolor{currentstroke}{rgb}{0.000000,0.000000,0.000000}%
\pgfsetstrokecolor{currentstroke}%
\pgfsetdash{}{0pt}%
\pgfsys@defobject{currentmarker}{\pgfqpoint{0.000000in}{-0.048611in}}{\pgfqpoint{0.000000in}{0.000000in}}{%
\pgfpathmoveto{\pgfqpoint{0.000000in}{0.000000in}}%
\pgfpathlineto{\pgfqpoint{0.000000in}{-0.048611in}}%
\pgfusepath{stroke,fill}%
}%
\begin{pgfscope}%
\pgfsys@transformshift{0.742589in}{0.670138in}%
\pgfsys@useobject{currentmarker}{}%
\end{pgfscope}%
\end{pgfscope}%
\begin{pgfscope}%
\definecolor{textcolor}{rgb}{0.000000,0.000000,0.000000}%
\pgfsetstrokecolor{textcolor}%
\pgfsetfillcolor{textcolor}%
\pgftext[x=0.742589in,y=0.572916in,,top]{\color{textcolor}{\rmfamily\fontsize{14.000000}{16.800000}\selectfont\catcode`\^=\active\def^{\ifmmode\sp\else\^{}\fi}\catcode`\%=\active\def%{\%}$\mathdefault{0}$}}%
\end{pgfscope}%
\begin{pgfscope}%
\pgfpathrectangle{\pgfqpoint{0.742589in}{0.670138in}}{\pgfqpoint{7.668292in}{4.846490in}}%
\pgfusepath{clip}%
\pgfsetrectcap%
\pgfsetroundjoin%
\pgfsetlinewidth{0.803000pt}%
\definecolor{currentstroke}{rgb}{0.690196,0.690196,0.690196}%
\pgfsetstrokecolor{currentstroke}%
\pgfsetdash{}{0pt}%
\pgfpathmoveto{\pgfqpoint{1.617965in}{0.670138in}}%
\pgfpathlineto{\pgfqpoint{1.617965in}{5.516628in}}%
\pgfusepath{stroke}%
\end{pgfscope}%
\begin{pgfscope}%
\pgfsetbuttcap%
\pgfsetroundjoin%
\definecolor{currentfill}{rgb}{0.000000,0.000000,0.000000}%
\pgfsetfillcolor{currentfill}%
\pgfsetlinewidth{0.803000pt}%
\definecolor{currentstroke}{rgb}{0.000000,0.000000,0.000000}%
\pgfsetstrokecolor{currentstroke}%
\pgfsetdash{}{0pt}%
\pgfsys@defobject{currentmarker}{\pgfqpoint{0.000000in}{-0.048611in}}{\pgfqpoint{0.000000in}{0.000000in}}{%
\pgfpathmoveto{\pgfqpoint{0.000000in}{0.000000in}}%
\pgfpathlineto{\pgfqpoint{0.000000in}{-0.048611in}}%
\pgfusepath{stroke,fill}%
}%
\begin{pgfscope}%
\pgfsys@transformshift{1.617965in}{0.670138in}%
\pgfsys@useobject{currentmarker}{}%
\end{pgfscope}%
\end{pgfscope}%
\begin{pgfscope}%
\definecolor{textcolor}{rgb}{0.000000,0.000000,0.000000}%
\pgfsetstrokecolor{textcolor}%
\pgfsetfillcolor{textcolor}%
\pgftext[x=1.617965in,y=0.572916in,,top]{\color{textcolor}{\rmfamily\fontsize{14.000000}{16.800000}\selectfont\catcode`\^=\active\def^{\ifmmode\sp\else\^{}\fi}\catcode`\%=\active\def%{\%}$\mathdefault{1000}$}}%
\end{pgfscope}%
\begin{pgfscope}%
\pgfpathrectangle{\pgfqpoint{0.742589in}{0.670138in}}{\pgfqpoint{7.668292in}{4.846490in}}%
\pgfusepath{clip}%
\pgfsetrectcap%
\pgfsetroundjoin%
\pgfsetlinewidth{0.803000pt}%
\definecolor{currentstroke}{rgb}{0.690196,0.690196,0.690196}%
\pgfsetstrokecolor{currentstroke}%
\pgfsetdash{}{0pt}%
\pgfpathmoveto{\pgfqpoint{2.493341in}{0.670138in}}%
\pgfpathlineto{\pgfqpoint{2.493341in}{5.516628in}}%
\pgfusepath{stroke}%
\end{pgfscope}%
\begin{pgfscope}%
\pgfsetbuttcap%
\pgfsetroundjoin%
\definecolor{currentfill}{rgb}{0.000000,0.000000,0.000000}%
\pgfsetfillcolor{currentfill}%
\pgfsetlinewidth{0.803000pt}%
\definecolor{currentstroke}{rgb}{0.000000,0.000000,0.000000}%
\pgfsetstrokecolor{currentstroke}%
\pgfsetdash{}{0pt}%
\pgfsys@defobject{currentmarker}{\pgfqpoint{0.000000in}{-0.048611in}}{\pgfqpoint{0.000000in}{0.000000in}}{%
\pgfpathmoveto{\pgfqpoint{0.000000in}{0.000000in}}%
\pgfpathlineto{\pgfqpoint{0.000000in}{-0.048611in}}%
\pgfusepath{stroke,fill}%
}%
\begin{pgfscope}%
\pgfsys@transformshift{2.493341in}{0.670138in}%
\pgfsys@useobject{currentmarker}{}%
\end{pgfscope}%
\end{pgfscope}%
\begin{pgfscope}%
\definecolor{textcolor}{rgb}{0.000000,0.000000,0.000000}%
\pgfsetstrokecolor{textcolor}%
\pgfsetfillcolor{textcolor}%
\pgftext[x=2.493341in,y=0.572916in,,top]{\color{textcolor}{\rmfamily\fontsize{14.000000}{16.800000}\selectfont\catcode`\^=\active\def^{\ifmmode\sp\else\^{}\fi}\catcode`\%=\active\def%{\%}$\mathdefault{2000}$}}%
\end{pgfscope}%
\begin{pgfscope}%
\pgfpathrectangle{\pgfqpoint{0.742589in}{0.670138in}}{\pgfqpoint{7.668292in}{4.846490in}}%
\pgfusepath{clip}%
\pgfsetrectcap%
\pgfsetroundjoin%
\pgfsetlinewidth{0.803000pt}%
\definecolor{currentstroke}{rgb}{0.690196,0.690196,0.690196}%
\pgfsetstrokecolor{currentstroke}%
\pgfsetdash{}{0pt}%
\pgfpathmoveto{\pgfqpoint{3.368716in}{0.670138in}}%
\pgfpathlineto{\pgfqpoint{3.368716in}{5.516628in}}%
\pgfusepath{stroke}%
\end{pgfscope}%
\begin{pgfscope}%
\pgfsetbuttcap%
\pgfsetroundjoin%
\definecolor{currentfill}{rgb}{0.000000,0.000000,0.000000}%
\pgfsetfillcolor{currentfill}%
\pgfsetlinewidth{0.803000pt}%
\definecolor{currentstroke}{rgb}{0.000000,0.000000,0.000000}%
\pgfsetstrokecolor{currentstroke}%
\pgfsetdash{}{0pt}%
\pgfsys@defobject{currentmarker}{\pgfqpoint{0.000000in}{-0.048611in}}{\pgfqpoint{0.000000in}{0.000000in}}{%
\pgfpathmoveto{\pgfqpoint{0.000000in}{0.000000in}}%
\pgfpathlineto{\pgfqpoint{0.000000in}{-0.048611in}}%
\pgfusepath{stroke,fill}%
}%
\begin{pgfscope}%
\pgfsys@transformshift{3.368716in}{0.670138in}%
\pgfsys@useobject{currentmarker}{}%
\end{pgfscope}%
\end{pgfscope}%
\begin{pgfscope}%
\definecolor{textcolor}{rgb}{0.000000,0.000000,0.000000}%
\pgfsetstrokecolor{textcolor}%
\pgfsetfillcolor{textcolor}%
\pgftext[x=3.368716in,y=0.572916in,,top]{\color{textcolor}{\rmfamily\fontsize{14.000000}{16.800000}\selectfont\catcode`\^=\active\def^{\ifmmode\sp\else\^{}\fi}\catcode`\%=\active\def%{\%}$\mathdefault{3000}$}}%
\end{pgfscope}%
\begin{pgfscope}%
\pgfpathrectangle{\pgfqpoint{0.742589in}{0.670138in}}{\pgfqpoint{7.668292in}{4.846490in}}%
\pgfusepath{clip}%
\pgfsetrectcap%
\pgfsetroundjoin%
\pgfsetlinewidth{0.803000pt}%
\definecolor{currentstroke}{rgb}{0.690196,0.690196,0.690196}%
\pgfsetstrokecolor{currentstroke}%
\pgfsetdash{}{0pt}%
\pgfpathmoveto{\pgfqpoint{4.244092in}{0.670138in}}%
\pgfpathlineto{\pgfqpoint{4.244092in}{5.516628in}}%
\pgfusepath{stroke}%
\end{pgfscope}%
\begin{pgfscope}%
\pgfsetbuttcap%
\pgfsetroundjoin%
\definecolor{currentfill}{rgb}{0.000000,0.000000,0.000000}%
\pgfsetfillcolor{currentfill}%
\pgfsetlinewidth{0.803000pt}%
\definecolor{currentstroke}{rgb}{0.000000,0.000000,0.000000}%
\pgfsetstrokecolor{currentstroke}%
\pgfsetdash{}{0pt}%
\pgfsys@defobject{currentmarker}{\pgfqpoint{0.000000in}{-0.048611in}}{\pgfqpoint{0.000000in}{0.000000in}}{%
\pgfpathmoveto{\pgfqpoint{0.000000in}{0.000000in}}%
\pgfpathlineto{\pgfqpoint{0.000000in}{-0.048611in}}%
\pgfusepath{stroke,fill}%
}%
\begin{pgfscope}%
\pgfsys@transformshift{4.244092in}{0.670138in}%
\pgfsys@useobject{currentmarker}{}%
\end{pgfscope}%
\end{pgfscope}%
\begin{pgfscope}%
\definecolor{textcolor}{rgb}{0.000000,0.000000,0.000000}%
\pgfsetstrokecolor{textcolor}%
\pgfsetfillcolor{textcolor}%
\pgftext[x=4.244092in,y=0.572916in,,top]{\color{textcolor}{\rmfamily\fontsize{14.000000}{16.800000}\selectfont\catcode`\^=\active\def^{\ifmmode\sp\else\^{}\fi}\catcode`\%=\active\def%{\%}$\mathdefault{4000}$}}%
\end{pgfscope}%
\begin{pgfscope}%
\pgfpathrectangle{\pgfqpoint{0.742589in}{0.670138in}}{\pgfqpoint{7.668292in}{4.846490in}}%
\pgfusepath{clip}%
\pgfsetrectcap%
\pgfsetroundjoin%
\pgfsetlinewidth{0.803000pt}%
\definecolor{currentstroke}{rgb}{0.690196,0.690196,0.690196}%
\pgfsetstrokecolor{currentstroke}%
\pgfsetdash{}{0pt}%
\pgfpathmoveto{\pgfqpoint{5.119468in}{0.670138in}}%
\pgfpathlineto{\pgfqpoint{5.119468in}{5.516628in}}%
\pgfusepath{stroke}%
\end{pgfscope}%
\begin{pgfscope}%
\pgfsetbuttcap%
\pgfsetroundjoin%
\definecolor{currentfill}{rgb}{0.000000,0.000000,0.000000}%
\pgfsetfillcolor{currentfill}%
\pgfsetlinewidth{0.803000pt}%
\definecolor{currentstroke}{rgb}{0.000000,0.000000,0.000000}%
\pgfsetstrokecolor{currentstroke}%
\pgfsetdash{}{0pt}%
\pgfsys@defobject{currentmarker}{\pgfqpoint{0.000000in}{-0.048611in}}{\pgfqpoint{0.000000in}{0.000000in}}{%
\pgfpathmoveto{\pgfqpoint{0.000000in}{0.000000in}}%
\pgfpathlineto{\pgfqpoint{0.000000in}{-0.048611in}}%
\pgfusepath{stroke,fill}%
}%
\begin{pgfscope}%
\pgfsys@transformshift{5.119468in}{0.670138in}%
\pgfsys@useobject{currentmarker}{}%
\end{pgfscope}%
\end{pgfscope}%
\begin{pgfscope}%
\definecolor{textcolor}{rgb}{0.000000,0.000000,0.000000}%
\pgfsetstrokecolor{textcolor}%
\pgfsetfillcolor{textcolor}%
\pgftext[x=5.119468in,y=0.572916in,,top]{\color{textcolor}{\rmfamily\fontsize{14.000000}{16.800000}\selectfont\catcode`\^=\active\def^{\ifmmode\sp\else\^{}\fi}\catcode`\%=\active\def%{\%}$\mathdefault{5000}$}}%
\end{pgfscope}%
\begin{pgfscope}%
\pgfpathrectangle{\pgfqpoint{0.742589in}{0.670138in}}{\pgfqpoint{7.668292in}{4.846490in}}%
\pgfusepath{clip}%
\pgfsetrectcap%
\pgfsetroundjoin%
\pgfsetlinewidth{0.803000pt}%
\definecolor{currentstroke}{rgb}{0.690196,0.690196,0.690196}%
\pgfsetstrokecolor{currentstroke}%
\pgfsetdash{}{0pt}%
\pgfpathmoveto{\pgfqpoint{5.994844in}{0.670138in}}%
\pgfpathlineto{\pgfqpoint{5.994844in}{5.516628in}}%
\pgfusepath{stroke}%
\end{pgfscope}%
\begin{pgfscope}%
\pgfsetbuttcap%
\pgfsetroundjoin%
\definecolor{currentfill}{rgb}{0.000000,0.000000,0.000000}%
\pgfsetfillcolor{currentfill}%
\pgfsetlinewidth{0.803000pt}%
\definecolor{currentstroke}{rgb}{0.000000,0.000000,0.000000}%
\pgfsetstrokecolor{currentstroke}%
\pgfsetdash{}{0pt}%
\pgfsys@defobject{currentmarker}{\pgfqpoint{0.000000in}{-0.048611in}}{\pgfqpoint{0.000000in}{0.000000in}}{%
\pgfpathmoveto{\pgfqpoint{0.000000in}{0.000000in}}%
\pgfpathlineto{\pgfqpoint{0.000000in}{-0.048611in}}%
\pgfusepath{stroke,fill}%
}%
\begin{pgfscope}%
\pgfsys@transformshift{5.994844in}{0.670138in}%
\pgfsys@useobject{currentmarker}{}%
\end{pgfscope}%
\end{pgfscope}%
\begin{pgfscope}%
\definecolor{textcolor}{rgb}{0.000000,0.000000,0.000000}%
\pgfsetstrokecolor{textcolor}%
\pgfsetfillcolor{textcolor}%
\pgftext[x=5.994844in,y=0.572916in,,top]{\color{textcolor}{\rmfamily\fontsize{14.000000}{16.800000}\selectfont\catcode`\^=\active\def^{\ifmmode\sp\else\^{}\fi}\catcode`\%=\active\def%{\%}$\mathdefault{6000}$}}%
\end{pgfscope}%
\begin{pgfscope}%
\pgfpathrectangle{\pgfqpoint{0.742589in}{0.670138in}}{\pgfqpoint{7.668292in}{4.846490in}}%
\pgfusepath{clip}%
\pgfsetrectcap%
\pgfsetroundjoin%
\pgfsetlinewidth{0.803000pt}%
\definecolor{currentstroke}{rgb}{0.690196,0.690196,0.690196}%
\pgfsetstrokecolor{currentstroke}%
\pgfsetdash{}{0pt}%
\pgfpathmoveto{\pgfqpoint{6.870220in}{0.670138in}}%
\pgfpathlineto{\pgfqpoint{6.870220in}{5.516628in}}%
\pgfusepath{stroke}%
\end{pgfscope}%
\begin{pgfscope}%
\pgfsetbuttcap%
\pgfsetroundjoin%
\definecolor{currentfill}{rgb}{0.000000,0.000000,0.000000}%
\pgfsetfillcolor{currentfill}%
\pgfsetlinewidth{0.803000pt}%
\definecolor{currentstroke}{rgb}{0.000000,0.000000,0.000000}%
\pgfsetstrokecolor{currentstroke}%
\pgfsetdash{}{0pt}%
\pgfsys@defobject{currentmarker}{\pgfqpoint{0.000000in}{-0.048611in}}{\pgfqpoint{0.000000in}{0.000000in}}{%
\pgfpathmoveto{\pgfqpoint{0.000000in}{0.000000in}}%
\pgfpathlineto{\pgfqpoint{0.000000in}{-0.048611in}}%
\pgfusepath{stroke,fill}%
}%
\begin{pgfscope}%
\pgfsys@transformshift{6.870220in}{0.670138in}%
\pgfsys@useobject{currentmarker}{}%
\end{pgfscope}%
\end{pgfscope}%
\begin{pgfscope}%
\definecolor{textcolor}{rgb}{0.000000,0.000000,0.000000}%
\pgfsetstrokecolor{textcolor}%
\pgfsetfillcolor{textcolor}%
\pgftext[x=6.870220in,y=0.572916in,,top]{\color{textcolor}{\rmfamily\fontsize{14.000000}{16.800000}\selectfont\catcode`\^=\active\def^{\ifmmode\sp\else\^{}\fi}\catcode`\%=\active\def%{\%}$\mathdefault{7000}$}}%
\end{pgfscope}%
\begin{pgfscope}%
\pgfpathrectangle{\pgfqpoint{0.742589in}{0.670138in}}{\pgfqpoint{7.668292in}{4.846490in}}%
\pgfusepath{clip}%
\pgfsetrectcap%
\pgfsetroundjoin%
\pgfsetlinewidth{0.803000pt}%
\definecolor{currentstroke}{rgb}{0.690196,0.690196,0.690196}%
\pgfsetstrokecolor{currentstroke}%
\pgfsetdash{}{0pt}%
\pgfpathmoveto{\pgfqpoint{7.745595in}{0.670138in}}%
\pgfpathlineto{\pgfqpoint{7.745595in}{5.516628in}}%
\pgfusepath{stroke}%
\end{pgfscope}%
\begin{pgfscope}%
\pgfsetbuttcap%
\pgfsetroundjoin%
\definecolor{currentfill}{rgb}{0.000000,0.000000,0.000000}%
\pgfsetfillcolor{currentfill}%
\pgfsetlinewidth{0.803000pt}%
\definecolor{currentstroke}{rgb}{0.000000,0.000000,0.000000}%
\pgfsetstrokecolor{currentstroke}%
\pgfsetdash{}{0pt}%
\pgfsys@defobject{currentmarker}{\pgfqpoint{0.000000in}{-0.048611in}}{\pgfqpoint{0.000000in}{0.000000in}}{%
\pgfpathmoveto{\pgfqpoint{0.000000in}{0.000000in}}%
\pgfpathlineto{\pgfqpoint{0.000000in}{-0.048611in}}%
\pgfusepath{stroke,fill}%
}%
\begin{pgfscope}%
\pgfsys@transformshift{7.745595in}{0.670138in}%
\pgfsys@useobject{currentmarker}{}%
\end{pgfscope}%
\end{pgfscope}%
\begin{pgfscope}%
\definecolor{textcolor}{rgb}{0.000000,0.000000,0.000000}%
\pgfsetstrokecolor{textcolor}%
\pgfsetfillcolor{textcolor}%
\pgftext[x=7.745595in,y=0.572916in,,top]{\color{textcolor}{\rmfamily\fontsize{14.000000}{16.800000}\selectfont\catcode`\^=\active\def^{\ifmmode\sp\else\^{}\fi}\catcode`\%=\active\def%{\%}$\mathdefault{8000}$}}%
\end{pgfscope}%
\begin{pgfscope}%
\pgfsetbuttcap%
\pgfsetroundjoin%
\definecolor{currentfill}{rgb}{0.000000,0.000000,0.000000}%
\pgfsetfillcolor{currentfill}%
\pgfsetlinewidth{0.602250pt}%
\definecolor{currentstroke}{rgb}{0.000000,0.000000,0.000000}%
\pgfsetstrokecolor{currentstroke}%
\pgfsetdash{}{0pt}%
\pgfsys@defobject{currentmarker}{\pgfqpoint{0.000000in}{-0.027778in}}{\pgfqpoint{0.000000in}{0.000000in}}{%
\pgfpathmoveto{\pgfqpoint{0.000000in}{0.000000in}}%
\pgfpathlineto{\pgfqpoint{0.000000in}{-0.027778in}}%
\pgfusepath{stroke,fill}%
}%
\begin{pgfscope}%
\pgfsys@transformshift{0.917664in}{0.670138in}%
\pgfsys@useobject{currentmarker}{}%
\end{pgfscope}%
\end{pgfscope}%
\begin{pgfscope}%
\pgfsetbuttcap%
\pgfsetroundjoin%
\definecolor{currentfill}{rgb}{0.000000,0.000000,0.000000}%
\pgfsetfillcolor{currentfill}%
\pgfsetlinewidth{0.602250pt}%
\definecolor{currentstroke}{rgb}{0.000000,0.000000,0.000000}%
\pgfsetstrokecolor{currentstroke}%
\pgfsetdash{}{0pt}%
\pgfsys@defobject{currentmarker}{\pgfqpoint{0.000000in}{-0.027778in}}{\pgfqpoint{0.000000in}{0.000000in}}{%
\pgfpathmoveto{\pgfqpoint{0.000000in}{0.000000in}}%
\pgfpathlineto{\pgfqpoint{0.000000in}{-0.027778in}}%
\pgfusepath{stroke,fill}%
}%
\begin{pgfscope}%
\pgfsys@transformshift{1.092739in}{0.670138in}%
\pgfsys@useobject{currentmarker}{}%
\end{pgfscope}%
\end{pgfscope}%
\begin{pgfscope}%
\pgfsetbuttcap%
\pgfsetroundjoin%
\definecolor{currentfill}{rgb}{0.000000,0.000000,0.000000}%
\pgfsetfillcolor{currentfill}%
\pgfsetlinewidth{0.602250pt}%
\definecolor{currentstroke}{rgb}{0.000000,0.000000,0.000000}%
\pgfsetstrokecolor{currentstroke}%
\pgfsetdash{}{0pt}%
\pgfsys@defobject{currentmarker}{\pgfqpoint{0.000000in}{-0.027778in}}{\pgfqpoint{0.000000in}{0.000000in}}{%
\pgfpathmoveto{\pgfqpoint{0.000000in}{0.000000in}}%
\pgfpathlineto{\pgfqpoint{0.000000in}{-0.027778in}}%
\pgfusepath{stroke,fill}%
}%
\begin{pgfscope}%
\pgfsys@transformshift{1.267814in}{0.670138in}%
\pgfsys@useobject{currentmarker}{}%
\end{pgfscope}%
\end{pgfscope}%
\begin{pgfscope}%
\pgfsetbuttcap%
\pgfsetroundjoin%
\definecolor{currentfill}{rgb}{0.000000,0.000000,0.000000}%
\pgfsetfillcolor{currentfill}%
\pgfsetlinewidth{0.602250pt}%
\definecolor{currentstroke}{rgb}{0.000000,0.000000,0.000000}%
\pgfsetstrokecolor{currentstroke}%
\pgfsetdash{}{0pt}%
\pgfsys@defobject{currentmarker}{\pgfqpoint{0.000000in}{-0.027778in}}{\pgfqpoint{0.000000in}{0.000000in}}{%
\pgfpathmoveto{\pgfqpoint{0.000000in}{0.000000in}}%
\pgfpathlineto{\pgfqpoint{0.000000in}{-0.027778in}}%
\pgfusepath{stroke,fill}%
}%
\begin{pgfscope}%
\pgfsys@transformshift{1.442890in}{0.670138in}%
\pgfsys@useobject{currentmarker}{}%
\end{pgfscope}%
\end{pgfscope}%
\begin{pgfscope}%
\pgfsetbuttcap%
\pgfsetroundjoin%
\definecolor{currentfill}{rgb}{0.000000,0.000000,0.000000}%
\pgfsetfillcolor{currentfill}%
\pgfsetlinewidth{0.602250pt}%
\definecolor{currentstroke}{rgb}{0.000000,0.000000,0.000000}%
\pgfsetstrokecolor{currentstroke}%
\pgfsetdash{}{0pt}%
\pgfsys@defobject{currentmarker}{\pgfqpoint{0.000000in}{-0.027778in}}{\pgfqpoint{0.000000in}{0.000000in}}{%
\pgfpathmoveto{\pgfqpoint{0.000000in}{0.000000in}}%
\pgfpathlineto{\pgfqpoint{0.000000in}{-0.027778in}}%
\pgfusepath{stroke,fill}%
}%
\begin{pgfscope}%
\pgfsys@transformshift{1.793040in}{0.670138in}%
\pgfsys@useobject{currentmarker}{}%
\end{pgfscope}%
\end{pgfscope}%
\begin{pgfscope}%
\pgfsetbuttcap%
\pgfsetroundjoin%
\definecolor{currentfill}{rgb}{0.000000,0.000000,0.000000}%
\pgfsetfillcolor{currentfill}%
\pgfsetlinewidth{0.602250pt}%
\definecolor{currentstroke}{rgb}{0.000000,0.000000,0.000000}%
\pgfsetstrokecolor{currentstroke}%
\pgfsetdash{}{0pt}%
\pgfsys@defobject{currentmarker}{\pgfqpoint{0.000000in}{-0.027778in}}{\pgfqpoint{0.000000in}{0.000000in}}{%
\pgfpathmoveto{\pgfqpoint{0.000000in}{0.000000in}}%
\pgfpathlineto{\pgfqpoint{0.000000in}{-0.027778in}}%
\pgfusepath{stroke,fill}%
}%
\begin{pgfscope}%
\pgfsys@transformshift{1.968115in}{0.670138in}%
\pgfsys@useobject{currentmarker}{}%
\end{pgfscope}%
\end{pgfscope}%
\begin{pgfscope}%
\pgfsetbuttcap%
\pgfsetroundjoin%
\definecolor{currentfill}{rgb}{0.000000,0.000000,0.000000}%
\pgfsetfillcolor{currentfill}%
\pgfsetlinewidth{0.602250pt}%
\definecolor{currentstroke}{rgb}{0.000000,0.000000,0.000000}%
\pgfsetstrokecolor{currentstroke}%
\pgfsetdash{}{0pt}%
\pgfsys@defobject{currentmarker}{\pgfqpoint{0.000000in}{-0.027778in}}{\pgfqpoint{0.000000in}{0.000000in}}{%
\pgfpathmoveto{\pgfqpoint{0.000000in}{0.000000in}}%
\pgfpathlineto{\pgfqpoint{0.000000in}{-0.027778in}}%
\pgfusepath{stroke,fill}%
}%
\begin{pgfscope}%
\pgfsys@transformshift{2.143190in}{0.670138in}%
\pgfsys@useobject{currentmarker}{}%
\end{pgfscope}%
\end{pgfscope}%
\begin{pgfscope}%
\pgfsetbuttcap%
\pgfsetroundjoin%
\definecolor{currentfill}{rgb}{0.000000,0.000000,0.000000}%
\pgfsetfillcolor{currentfill}%
\pgfsetlinewidth{0.602250pt}%
\definecolor{currentstroke}{rgb}{0.000000,0.000000,0.000000}%
\pgfsetstrokecolor{currentstroke}%
\pgfsetdash{}{0pt}%
\pgfsys@defobject{currentmarker}{\pgfqpoint{0.000000in}{-0.027778in}}{\pgfqpoint{0.000000in}{0.000000in}}{%
\pgfpathmoveto{\pgfqpoint{0.000000in}{0.000000in}}%
\pgfpathlineto{\pgfqpoint{0.000000in}{-0.027778in}}%
\pgfusepath{stroke,fill}%
}%
\begin{pgfscope}%
\pgfsys@transformshift{2.318265in}{0.670138in}%
\pgfsys@useobject{currentmarker}{}%
\end{pgfscope}%
\end{pgfscope}%
\begin{pgfscope}%
\pgfsetbuttcap%
\pgfsetroundjoin%
\definecolor{currentfill}{rgb}{0.000000,0.000000,0.000000}%
\pgfsetfillcolor{currentfill}%
\pgfsetlinewidth{0.602250pt}%
\definecolor{currentstroke}{rgb}{0.000000,0.000000,0.000000}%
\pgfsetstrokecolor{currentstroke}%
\pgfsetdash{}{0pt}%
\pgfsys@defobject{currentmarker}{\pgfqpoint{0.000000in}{-0.027778in}}{\pgfqpoint{0.000000in}{0.000000in}}{%
\pgfpathmoveto{\pgfqpoint{0.000000in}{0.000000in}}%
\pgfpathlineto{\pgfqpoint{0.000000in}{-0.027778in}}%
\pgfusepath{stroke,fill}%
}%
\begin{pgfscope}%
\pgfsys@transformshift{2.668416in}{0.670138in}%
\pgfsys@useobject{currentmarker}{}%
\end{pgfscope}%
\end{pgfscope}%
\begin{pgfscope}%
\pgfsetbuttcap%
\pgfsetroundjoin%
\definecolor{currentfill}{rgb}{0.000000,0.000000,0.000000}%
\pgfsetfillcolor{currentfill}%
\pgfsetlinewidth{0.602250pt}%
\definecolor{currentstroke}{rgb}{0.000000,0.000000,0.000000}%
\pgfsetstrokecolor{currentstroke}%
\pgfsetdash{}{0pt}%
\pgfsys@defobject{currentmarker}{\pgfqpoint{0.000000in}{-0.027778in}}{\pgfqpoint{0.000000in}{0.000000in}}{%
\pgfpathmoveto{\pgfqpoint{0.000000in}{0.000000in}}%
\pgfpathlineto{\pgfqpoint{0.000000in}{-0.027778in}}%
\pgfusepath{stroke,fill}%
}%
\begin{pgfscope}%
\pgfsys@transformshift{2.843491in}{0.670138in}%
\pgfsys@useobject{currentmarker}{}%
\end{pgfscope}%
\end{pgfscope}%
\begin{pgfscope}%
\pgfsetbuttcap%
\pgfsetroundjoin%
\definecolor{currentfill}{rgb}{0.000000,0.000000,0.000000}%
\pgfsetfillcolor{currentfill}%
\pgfsetlinewidth{0.602250pt}%
\definecolor{currentstroke}{rgb}{0.000000,0.000000,0.000000}%
\pgfsetstrokecolor{currentstroke}%
\pgfsetdash{}{0pt}%
\pgfsys@defobject{currentmarker}{\pgfqpoint{0.000000in}{-0.027778in}}{\pgfqpoint{0.000000in}{0.000000in}}{%
\pgfpathmoveto{\pgfqpoint{0.000000in}{0.000000in}}%
\pgfpathlineto{\pgfqpoint{0.000000in}{-0.027778in}}%
\pgfusepath{stroke,fill}%
}%
\begin{pgfscope}%
\pgfsys@transformshift{3.018566in}{0.670138in}%
\pgfsys@useobject{currentmarker}{}%
\end{pgfscope}%
\end{pgfscope}%
\begin{pgfscope}%
\pgfsetbuttcap%
\pgfsetroundjoin%
\definecolor{currentfill}{rgb}{0.000000,0.000000,0.000000}%
\pgfsetfillcolor{currentfill}%
\pgfsetlinewidth{0.602250pt}%
\definecolor{currentstroke}{rgb}{0.000000,0.000000,0.000000}%
\pgfsetstrokecolor{currentstroke}%
\pgfsetdash{}{0pt}%
\pgfsys@defobject{currentmarker}{\pgfqpoint{0.000000in}{-0.027778in}}{\pgfqpoint{0.000000in}{0.000000in}}{%
\pgfpathmoveto{\pgfqpoint{0.000000in}{0.000000in}}%
\pgfpathlineto{\pgfqpoint{0.000000in}{-0.027778in}}%
\pgfusepath{stroke,fill}%
}%
\begin{pgfscope}%
\pgfsys@transformshift{3.193641in}{0.670138in}%
\pgfsys@useobject{currentmarker}{}%
\end{pgfscope}%
\end{pgfscope}%
\begin{pgfscope}%
\pgfsetbuttcap%
\pgfsetroundjoin%
\definecolor{currentfill}{rgb}{0.000000,0.000000,0.000000}%
\pgfsetfillcolor{currentfill}%
\pgfsetlinewidth{0.602250pt}%
\definecolor{currentstroke}{rgb}{0.000000,0.000000,0.000000}%
\pgfsetstrokecolor{currentstroke}%
\pgfsetdash{}{0pt}%
\pgfsys@defobject{currentmarker}{\pgfqpoint{0.000000in}{-0.027778in}}{\pgfqpoint{0.000000in}{0.000000in}}{%
\pgfpathmoveto{\pgfqpoint{0.000000in}{0.000000in}}%
\pgfpathlineto{\pgfqpoint{0.000000in}{-0.027778in}}%
\pgfusepath{stroke,fill}%
}%
\begin{pgfscope}%
\pgfsys@transformshift{3.543792in}{0.670138in}%
\pgfsys@useobject{currentmarker}{}%
\end{pgfscope}%
\end{pgfscope}%
\begin{pgfscope}%
\pgfsetbuttcap%
\pgfsetroundjoin%
\definecolor{currentfill}{rgb}{0.000000,0.000000,0.000000}%
\pgfsetfillcolor{currentfill}%
\pgfsetlinewidth{0.602250pt}%
\definecolor{currentstroke}{rgb}{0.000000,0.000000,0.000000}%
\pgfsetstrokecolor{currentstroke}%
\pgfsetdash{}{0pt}%
\pgfsys@defobject{currentmarker}{\pgfqpoint{0.000000in}{-0.027778in}}{\pgfqpoint{0.000000in}{0.000000in}}{%
\pgfpathmoveto{\pgfqpoint{0.000000in}{0.000000in}}%
\pgfpathlineto{\pgfqpoint{0.000000in}{-0.027778in}}%
\pgfusepath{stroke,fill}%
}%
\begin{pgfscope}%
\pgfsys@transformshift{3.718867in}{0.670138in}%
\pgfsys@useobject{currentmarker}{}%
\end{pgfscope}%
\end{pgfscope}%
\begin{pgfscope}%
\pgfsetbuttcap%
\pgfsetroundjoin%
\definecolor{currentfill}{rgb}{0.000000,0.000000,0.000000}%
\pgfsetfillcolor{currentfill}%
\pgfsetlinewidth{0.602250pt}%
\definecolor{currentstroke}{rgb}{0.000000,0.000000,0.000000}%
\pgfsetstrokecolor{currentstroke}%
\pgfsetdash{}{0pt}%
\pgfsys@defobject{currentmarker}{\pgfqpoint{0.000000in}{-0.027778in}}{\pgfqpoint{0.000000in}{0.000000in}}{%
\pgfpathmoveto{\pgfqpoint{0.000000in}{0.000000in}}%
\pgfpathlineto{\pgfqpoint{0.000000in}{-0.027778in}}%
\pgfusepath{stroke,fill}%
}%
\begin{pgfscope}%
\pgfsys@transformshift{3.893942in}{0.670138in}%
\pgfsys@useobject{currentmarker}{}%
\end{pgfscope}%
\end{pgfscope}%
\begin{pgfscope}%
\pgfsetbuttcap%
\pgfsetroundjoin%
\definecolor{currentfill}{rgb}{0.000000,0.000000,0.000000}%
\pgfsetfillcolor{currentfill}%
\pgfsetlinewidth{0.602250pt}%
\definecolor{currentstroke}{rgb}{0.000000,0.000000,0.000000}%
\pgfsetstrokecolor{currentstroke}%
\pgfsetdash{}{0pt}%
\pgfsys@defobject{currentmarker}{\pgfqpoint{0.000000in}{-0.027778in}}{\pgfqpoint{0.000000in}{0.000000in}}{%
\pgfpathmoveto{\pgfqpoint{0.000000in}{0.000000in}}%
\pgfpathlineto{\pgfqpoint{0.000000in}{-0.027778in}}%
\pgfusepath{stroke,fill}%
}%
\begin{pgfscope}%
\pgfsys@transformshift{4.069017in}{0.670138in}%
\pgfsys@useobject{currentmarker}{}%
\end{pgfscope}%
\end{pgfscope}%
\begin{pgfscope}%
\pgfsetbuttcap%
\pgfsetroundjoin%
\definecolor{currentfill}{rgb}{0.000000,0.000000,0.000000}%
\pgfsetfillcolor{currentfill}%
\pgfsetlinewidth{0.602250pt}%
\definecolor{currentstroke}{rgb}{0.000000,0.000000,0.000000}%
\pgfsetstrokecolor{currentstroke}%
\pgfsetdash{}{0pt}%
\pgfsys@defobject{currentmarker}{\pgfqpoint{0.000000in}{-0.027778in}}{\pgfqpoint{0.000000in}{0.000000in}}{%
\pgfpathmoveto{\pgfqpoint{0.000000in}{0.000000in}}%
\pgfpathlineto{\pgfqpoint{0.000000in}{-0.027778in}}%
\pgfusepath{stroke,fill}%
}%
\begin{pgfscope}%
\pgfsys@transformshift{4.419167in}{0.670138in}%
\pgfsys@useobject{currentmarker}{}%
\end{pgfscope}%
\end{pgfscope}%
\begin{pgfscope}%
\pgfsetbuttcap%
\pgfsetroundjoin%
\definecolor{currentfill}{rgb}{0.000000,0.000000,0.000000}%
\pgfsetfillcolor{currentfill}%
\pgfsetlinewidth{0.602250pt}%
\definecolor{currentstroke}{rgb}{0.000000,0.000000,0.000000}%
\pgfsetstrokecolor{currentstroke}%
\pgfsetdash{}{0pt}%
\pgfsys@defobject{currentmarker}{\pgfqpoint{0.000000in}{-0.027778in}}{\pgfqpoint{0.000000in}{0.000000in}}{%
\pgfpathmoveto{\pgfqpoint{0.000000in}{0.000000in}}%
\pgfpathlineto{\pgfqpoint{0.000000in}{-0.027778in}}%
\pgfusepath{stroke,fill}%
}%
\begin{pgfscope}%
\pgfsys@transformshift{4.594243in}{0.670138in}%
\pgfsys@useobject{currentmarker}{}%
\end{pgfscope}%
\end{pgfscope}%
\begin{pgfscope}%
\pgfsetbuttcap%
\pgfsetroundjoin%
\definecolor{currentfill}{rgb}{0.000000,0.000000,0.000000}%
\pgfsetfillcolor{currentfill}%
\pgfsetlinewidth{0.602250pt}%
\definecolor{currentstroke}{rgb}{0.000000,0.000000,0.000000}%
\pgfsetstrokecolor{currentstroke}%
\pgfsetdash{}{0pt}%
\pgfsys@defobject{currentmarker}{\pgfqpoint{0.000000in}{-0.027778in}}{\pgfqpoint{0.000000in}{0.000000in}}{%
\pgfpathmoveto{\pgfqpoint{0.000000in}{0.000000in}}%
\pgfpathlineto{\pgfqpoint{0.000000in}{-0.027778in}}%
\pgfusepath{stroke,fill}%
}%
\begin{pgfscope}%
\pgfsys@transformshift{4.769318in}{0.670138in}%
\pgfsys@useobject{currentmarker}{}%
\end{pgfscope}%
\end{pgfscope}%
\begin{pgfscope}%
\pgfsetbuttcap%
\pgfsetroundjoin%
\definecolor{currentfill}{rgb}{0.000000,0.000000,0.000000}%
\pgfsetfillcolor{currentfill}%
\pgfsetlinewidth{0.602250pt}%
\definecolor{currentstroke}{rgb}{0.000000,0.000000,0.000000}%
\pgfsetstrokecolor{currentstroke}%
\pgfsetdash{}{0pt}%
\pgfsys@defobject{currentmarker}{\pgfqpoint{0.000000in}{-0.027778in}}{\pgfqpoint{0.000000in}{0.000000in}}{%
\pgfpathmoveto{\pgfqpoint{0.000000in}{0.000000in}}%
\pgfpathlineto{\pgfqpoint{0.000000in}{-0.027778in}}%
\pgfusepath{stroke,fill}%
}%
\begin{pgfscope}%
\pgfsys@transformshift{4.944393in}{0.670138in}%
\pgfsys@useobject{currentmarker}{}%
\end{pgfscope}%
\end{pgfscope}%
\begin{pgfscope}%
\pgfsetbuttcap%
\pgfsetroundjoin%
\definecolor{currentfill}{rgb}{0.000000,0.000000,0.000000}%
\pgfsetfillcolor{currentfill}%
\pgfsetlinewidth{0.602250pt}%
\definecolor{currentstroke}{rgb}{0.000000,0.000000,0.000000}%
\pgfsetstrokecolor{currentstroke}%
\pgfsetdash{}{0pt}%
\pgfsys@defobject{currentmarker}{\pgfqpoint{0.000000in}{-0.027778in}}{\pgfqpoint{0.000000in}{0.000000in}}{%
\pgfpathmoveto{\pgfqpoint{0.000000in}{0.000000in}}%
\pgfpathlineto{\pgfqpoint{0.000000in}{-0.027778in}}%
\pgfusepath{stroke,fill}%
}%
\begin{pgfscope}%
\pgfsys@transformshift{5.294543in}{0.670138in}%
\pgfsys@useobject{currentmarker}{}%
\end{pgfscope}%
\end{pgfscope}%
\begin{pgfscope}%
\pgfsetbuttcap%
\pgfsetroundjoin%
\definecolor{currentfill}{rgb}{0.000000,0.000000,0.000000}%
\pgfsetfillcolor{currentfill}%
\pgfsetlinewidth{0.602250pt}%
\definecolor{currentstroke}{rgb}{0.000000,0.000000,0.000000}%
\pgfsetstrokecolor{currentstroke}%
\pgfsetdash{}{0pt}%
\pgfsys@defobject{currentmarker}{\pgfqpoint{0.000000in}{-0.027778in}}{\pgfqpoint{0.000000in}{0.000000in}}{%
\pgfpathmoveto{\pgfqpoint{0.000000in}{0.000000in}}%
\pgfpathlineto{\pgfqpoint{0.000000in}{-0.027778in}}%
\pgfusepath{stroke,fill}%
}%
\begin{pgfscope}%
\pgfsys@transformshift{5.469618in}{0.670138in}%
\pgfsys@useobject{currentmarker}{}%
\end{pgfscope}%
\end{pgfscope}%
\begin{pgfscope}%
\pgfsetbuttcap%
\pgfsetroundjoin%
\definecolor{currentfill}{rgb}{0.000000,0.000000,0.000000}%
\pgfsetfillcolor{currentfill}%
\pgfsetlinewidth{0.602250pt}%
\definecolor{currentstroke}{rgb}{0.000000,0.000000,0.000000}%
\pgfsetstrokecolor{currentstroke}%
\pgfsetdash{}{0pt}%
\pgfsys@defobject{currentmarker}{\pgfqpoint{0.000000in}{-0.027778in}}{\pgfqpoint{0.000000in}{0.000000in}}{%
\pgfpathmoveto{\pgfqpoint{0.000000in}{0.000000in}}%
\pgfpathlineto{\pgfqpoint{0.000000in}{-0.027778in}}%
\pgfusepath{stroke,fill}%
}%
\begin{pgfscope}%
\pgfsys@transformshift{5.644693in}{0.670138in}%
\pgfsys@useobject{currentmarker}{}%
\end{pgfscope}%
\end{pgfscope}%
\begin{pgfscope}%
\pgfsetbuttcap%
\pgfsetroundjoin%
\definecolor{currentfill}{rgb}{0.000000,0.000000,0.000000}%
\pgfsetfillcolor{currentfill}%
\pgfsetlinewidth{0.602250pt}%
\definecolor{currentstroke}{rgb}{0.000000,0.000000,0.000000}%
\pgfsetstrokecolor{currentstroke}%
\pgfsetdash{}{0pt}%
\pgfsys@defobject{currentmarker}{\pgfqpoint{0.000000in}{-0.027778in}}{\pgfqpoint{0.000000in}{0.000000in}}{%
\pgfpathmoveto{\pgfqpoint{0.000000in}{0.000000in}}%
\pgfpathlineto{\pgfqpoint{0.000000in}{-0.027778in}}%
\pgfusepath{stroke,fill}%
}%
\begin{pgfscope}%
\pgfsys@transformshift{5.819769in}{0.670138in}%
\pgfsys@useobject{currentmarker}{}%
\end{pgfscope}%
\end{pgfscope}%
\begin{pgfscope}%
\pgfsetbuttcap%
\pgfsetroundjoin%
\definecolor{currentfill}{rgb}{0.000000,0.000000,0.000000}%
\pgfsetfillcolor{currentfill}%
\pgfsetlinewidth{0.602250pt}%
\definecolor{currentstroke}{rgb}{0.000000,0.000000,0.000000}%
\pgfsetstrokecolor{currentstroke}%
\pgfsetdash{}{0pt}%
\pgfsys@defobject{currentmarker}{\pgfqpoint{0.000000in}{-0.027778in}}{\pgfqpoint{0.000000in}{0.000000in}}{%
\pgfpathmoveto{\pgfqpoint{0.000000in}{0.000000in}}%
\pgfpathlineto{\pgfqpoint{0.000000in}{-0.027778in}}%
\pgfusepath{stroke,fill}%
}%
\begin{pgfscope}%
\pgfsys@transformshift{6.169919in}{0.670138in}%
\pgfsys@useobject{currentmarker}{}%
\end{pgfscope}%
\end{pgfscope}%
\begin{pgfscope}%
\pgfsetbuttcap%
\pgfsetroundjoin%
\definecolor{currentfill}{rgb}{0.000000,0.000000,0.000000}%
\pgfsetfillcolor{currentfill}%
\pgfsetlinewidth{0.602250pt}%
\definecolor{currentstroke}{rgb}{0.000000,0.000000,0.000000}%
\pgfsetstrokecolor{currentstroke}%
\pgfsetdash{}{0pt}%
\pgfsys@defobject{currentmarker}{\pgfqpoint{0.000000in}{-0.027778in}}{\pgfqpoint{0.000000in}{0.000000in}}{%
\pgfpathmoveto{\pgfqpoint{0.000000in}{0.000000in}}%
\pgfpathlineto{\pgfqpoint{0.000000in}{-0.027778in}}%
\pgfusepath{stroke,fill}%
}%
\begin{pgfscope}%
\pgfsys@transformshift{6.344994in}{0.670138in}%
\pgfsys@useobject{currentmarker}{}%
\end{pgfscope}%
\end{pgfscope}%
\begin{pgfscope}%
\pgfsetbuttcap%
\pgfsetroundjoin%
\definecolor{currentfill}{rgb}{0.000000,0.000000,0.000000}%
\pgfsetfillcolor{currentfill}%
\pgfsetlinewidth{0.602250pt}%
\definecolor{currentstroke}{rgb}{0.000000,0.000000,0.000000}%
\pgfsetstrokecolor{currentstroke}%
\pgfsetdash{}{0pt}%
\pgfsys@defobject{currentmarker}{\pgfqpoint{0.000000in}{-0.027778in}}{\pgfqpoint{0.000000in}{0.000000in}}{%
\pgfpathmoveto{\pgfqpoint{0.000000in}{0.000000in}}%
\pgfpathlineto{\pgfqpoint{0.000000in}{-0.027778in}}%
\pgfusepath{stroke,fill}%
}%
\begin{pgfscope}%
\pgfsys@transformshift{6.520069in}{0.670138in}%
\pgfsys@useobject{currentmarker}{}%
\end{pgfscope}%
\end{pgfscope}%
\begin{pgfscope}%
\pgfsetbuttcap%
\pgfsetroundjoin%
\definecolor{currentfill}{rgb}{0.000000,0.000000,0.000000}%
\pgfsetfillcolor{currentfill}%
\pgfsetlinewidth{0.602250pt}%
\definecolor{currentstroke}{rgb}{0.000000,0.000000,0.000000}%
\pgfsetstrokecolor{currentstroke}%
\pgfsetdash{}{0pt}%
\pgfsys@defobject{currentmarker}{\pgfqpoint{0.000000in}{-0.027778in}}{\pgfqpoint{0.000000in}{0.000000in}}{%
\pgfpathmoveto{\pgfqpoint{0.000000in}{0.000000in}}%
\pgfpathlineto{\pgfqpoint{0.000000in}{-0.027778in}}%
\pgfusepath{stroke,fill}%
}%
\begin{pgfscope}%
\pgfsys@transformshift{6.695144in}{0.670138in}%
\pgfsys@useobject{currentmarker}{}%
\end{pgfscope}%
\end{pgfscope}%
\begin{pgfscope}%
\pgfsetbuttcap%
\pgfsetroundjoin%
\definecolor{currentfill}{rgb}{0.000000,0.000000,0.000000}%
\pgfsetfillcolor{currentfill}%
\pgfsetlinewidth{0.602250pt}%
\definecolor{currentstroke}{rgb}{0.000000,0.000000,0.000000}%
\pgfsetstrokecolor{currentstroke}%
\pgfsetdash{}{0pt}%
\pgfsys@defobject{currentmarker}{\pgfqpoint{0.000000in}{-0.027778in}}{\pgfqpoint{0.000000in}{0.000000in}}{%
\pgfpathmoveto{\pgfqpoint{0.000000in}{0.000000in}}%
\pgfpathlineto{\pgfqpoint{0.000000in}{-0.027778in}}%
\pgfusepath{stroke,fill}%
}%
\begin{pgfscope}%
\pgfsys@transformshift{7.045295in}{0.670138in}%
\pgfsys@useobject{currentmarker}{}%
\end{pgfscope}%
\end{pgfscope}%
\begin{pgfscope}%
\pgfsetbuttcap%
\pgfsetroundjoin%
\definecolor{currentfill}{rgb}{0.000000,0.000000,0.000000}%
\pgfsetfillcolor{currentfill}%
\pgfsetlinewidth{0.602250pt}%
\definecolor{currentstroke}{rgb}{0.000000,0.000000,0.000000}%
\pgfsetstrokecolor{currentstroke}%
\pgfsetdash{}{0pt}%
\pgfsys@defobject{currentmarker}{\pgfqpoint{0.000000in}{-0.027778in}}{\pgfqpoint{0.000000in}{0.000000in}}{%
\pgfpathmoveto{\pgfqpoint{0.000000in}{0.000000in}}%
\pgfpathlineto{\pgfqpoint{0.000000in}{-0.027778in}}%
\pgfusepath{stroke,fill}%
}%
\begin{pgfscope}%
\pgfsys@transformshift{7.220370in}{0.670138in}%
\pgfsys@useobject{currentmarker}{}%
\end{pgfscope}%
\end{pgfscope}%
\begin{pgfscope}%
\pgfsetbuttcap%
\pgfsetroundjoin%
\definecolor{currentfill}{rgb}{0.000000,0.000000,0.000000}%
\pgfsetfillcolor{currentfill}%
\pgfsetlinewidth{0.602250pt}%
\definecolor{currentstroke}{rgb}{0.000000,0.000000,0.000000}%
\pgfsetstrokecolor{currentstroke}%
\pgfsetdash{}{0pt}%
\pgfsys@defobject{currentmarker}{\pgfqpoint{0.000000in}{-0.027778in}}{\pgfqpoint{0.000000in}{0.000000in}}{%
\pgfpathmoveto{\pgfqpoint{0.000000in}{0.000000in}}%
\pgfpathlineto{\pgfqpoint{0.000000in}{-0.027778in}}%
\pgfusepath{stroke,fill}%
}%
\begin{pgfscope}%
\pgfsys@transformshift{7.395445in}{0.670138in}%
\pgfsys@useobject{currentmarker}{}%
\end{pgfscope}%
\end{pgfscope}%
\begin{pgfscope}%
\pgfsetbuttcap%
\pgfsetroundjoin%
\definecolor{currentfill}{rgb}{0.000000,0.000000,0.000000}%
\pgfsetfillcolor{currentfill}%
\pgfsetlinewidth{0.602250pt}%
\definecolor{currentstroke}{rgb}{0.000000,0.000000,0.000000}%
\pgfsetstrokecolor{currentstroke}%
\pgfsetdash{}{0pt}%
\pgfsys@defobject{currentmarker}{\pgfqpoint{0.000000in}{-0.027778in}}{\pgfqpoint{0.000000in}{0.000000in}}{%
\pgfpathmoveto{\pgfqpoint{0.000000in}{0.000000in}}%
\pgfpathlineto{\pgfqpoint{0.000000in}{-0.027778in}}%
\pgfusepath{stroke,fill}%
}%
\begin{pgfscope}%
\pgfsys@transformshift{7.570520in}{0.670138in}%
\pgfsys@useobject{currentmarker}{}%
\end{pgfscope}%
\end{pgfscope}%
\begin{pgfscope}%
\pgfsetbuttcap%
\pgfsetroundjoin%
\definecolor{currentfill}{rgb}{0.000000,0.000000,0.000000}%
\pgfsetfillcolor{currentfill}%
\pgfsetlinewidth{0.602250pt}%
\definecolor{currentstroke}{rgb}{0.000000,0.000000,0.000000}%
\pgfsetstrokecolor{currentstroke}%
\pgfsetdash{}{0pt}%
\pgfsys@defobject{currentmarker}{\pgfqpoint{0.000000in}{-0.027778in}}{\pgfqpoint{0.000000in}{0.000000in}}{%
\pgfpathmoveto{\pgfqpoint{0.000000in}{0.000000in}}%
\pgfpathlineto{\pgfqpoint{0.000000in}{-0.027778in}}%
\pgfusepath{stroke,fill}%
}%
\begin{pgfscope}%
\pgfsys@transformshift{7.920671in}{0.670138in}%
\pgfsys@useobject{currentmarker}{}%
\end{pgfscope}%
\end{pgfscope}%
\begin{pgfscope}%
\pgfsetbuttcap%
\pgfsetroundjoin%
\definecolor{currentfill}{rgb}{0.000000,0.000000,0.000000}%
\pgfsetfillcolor{currentfill}%
\pgfsetlinewidth{0.602250pt}%
\definecolor{currentstroke}{rgb}{0.000000,0.000000,0.000000}%
\pgfsetstrokecolor{currentstroke}%
\pgfsetdash{}{0pt}%
\pgfsys@defobject{currentmarker}{\pgfqpoint{0.000000in}{-0.027778in}}{\pgfqpoint{0.000000in}{0.000000in}}{%
\pgfpathmoveto{\pgfqpoint{0.000000in}{0.000000in}}%
\pgfpathlineto{\pgfqpoint{0.000000in}{-0.027778in}}%
\pgfusepath{stroke,fill}%
}%
\begin{pgfscope}%
\pgfsys@transformshift{8.095746in}{0.670138in}%
\pgfsys@useobject{currentmarker}{}%
\end{pgfscope}%
\end{pgfscope}%
\begin{pgfscope}%
\pgfsetbuttcap%
\pgfsetroundjoin%
\definecolor{currentfill}{rgb}{0.000000,0.000000,0.000000}%
\pgfsetfillcolor{currentfill}%
\pgfsetlinewidth{0.602250pt}%
\definecolor{currentstroke}{rgb}{0.000000,0.000000,0.000000}%
\pgfsetstrokecolor{currentstroke}%
\pgfsetdash{}{0pt}%
\pgfsys@defobject{currentmarker}{\pgfqpoint{0.000000in}{-0.027778in}}{\pgfqpoint{0.000000in}{0.000000in}}{%
\pgfpathmoveto{\pgfqpoint{0.000000in}{0.000000in}}%
\pgfpathlineto{\pgfqpoint{0.000000in}{-0.027778in}}%
\pgfusepath{stroke,fill}%
}%
\begin{pgfscope}%
\pgfsys@transformshift{8.270821in}{0.670138in}%
\pgfsys@useobject{currentmarker}{}%
\end{pgfscope}%
\end{pgfscope}%
\begin{pgfscope}%
\definecolor{textcolor}{rgb}{0.000000,0.000000,0.000000}%
\pgfsetstrokecolor{textcolor}%
\pgfsetfillcolor{textcolor}%
\pgftext[x=4.576735in,y=0.339583in,,top]{\color{textcolor}{\rmfamily\fontsize{18.000000}{21.600000}\selectfont\catcode`\^=\active\def^{\ifmmode\sp\else\^{}\fi}\catcode`\%=\active\def%{\%}Time [hours]}}%
\end{pgfscope}%
\begin{pgfscope}%
\pgfpathrectangle{\pgfqpoint{0.742589in}{0.670138in}}{\pgfqpoint{7.668292in}{4.846490in}}%
\pgfusepath{clip}%
\pgfsetrectcap%
\pgfsetroundjoin%
\pgfsetlinewidth{0.803000pt}%
\definecolor{currentstroke}{rgb}{0.690196,0.690196,0.690196}%
\pgfsetstrokecolor{currentstroke}%
\pgfsetdash{}{0pt}%
\pgfpathmoveto{\pgfqpoint{0.742589in}{1.314483in}}%
\pgfpathlineto{\pgfqpoint{8.410881in}{1.314483in}}%
\pgfusepath{stroke}%
\end{pgfscope}%
\begin{pgfscope}%
\pgfsetbuttcap%
\pgfsetroundjoin%
\definecolor{currentfill}{rgb}{0.000000,0.000000,0.000000}%
\pgfsetfillcolor{currentfill}%
\pgfsetlinewidth{0.803000pt}%
\definecolor{currentstroke}{rgb}{0.000000,0.000000,0.000000}%
\pgfsetstrokecolor{currentstroke}%
\pgfsetdash{}{0pt}%
\pgfsys@defobject{currentmarker}{\pgfqpoint{-0.048611in}{0.000000in}}{\pgfqpoint{-0.000000in}{0.000000in}}{%
\pgfpathmoveto{\pgfqpoint{-0.000000in}{0.000000in}}%
\pgfpathlineto{\pgfqpoint{-0.048611in}{0.000000in}}%
\pgfusepath{stroke,fill}%
}%
\begin{pgfscope}%
\pgfsys@transformshift{0.742589in}{1.314483in}%
\pgfsys@useobject{currentmarker}{}%
\end{pgfscope}%
\end{pgfscope}%
\begin{pgfscope}%
\definecolor{textcolor}{rgb}{0.000000,0.000000,0.000000}%
\pgfsetstrokecolor{textcolor}%
\pgfsetfillcolor{textcolor}%
\pgftext[x=0.395138in, y=1.245039in, left, base]{\color{textcolor}{\rmfamily\fontsize{14.000000}{16.800000}\selectfont\catcode`\^=\active\def^{\ifmmode\sp\else\^{}\fi}\catcode`\%=\active\def%{\%}$\mathdefault{0.6}$}}%
\end{pgfscope}%
\begin{pgfscope}%
\pgfpathrectangle{\pgfqpoint{0.742589in}{0.670138in}}{\pgfqpoint{7.668292in}{4.846490in}}%
\pgfusepath{clip}%
\pgfsetrectcap%
\pgfsetroundjoin%
\pgfsetlinewidth{0.803000pt}%
\definecolor{currentstroke}{rgb}{0.690196,0.690196,0.690196}%
\pgfsetstrokecolor{currentstroke}%
\pgfsetdash{}{0pt}%
\pgfpathmoveto{\pgfqpoint{0.742589in}{2.365019in}}%
\pgfpathlineto{\pgfqpoint{8.410881in}{2.365019in}}%
\pgfusepath{stroke}%
\end{pgfscope}%
\begin{pgfscope}%
\pgfsetbuttcap%
\pgfsetroundjoin%
\definecolor{currentfill}{rgb}{0.000000,0.000000,0.000000}%
\pgfsetfillcolor{currentfill}%
\pgfsetlinewidth{0.803000pt}%
\definecolor{currentstroke}{rgb}{0.000000,0.000000,0.000000}%
\pgfsetstrokecolor{currentstroke}%
\pgfsetdash{}{0pt}%
\pgfsys@defobject{currentmarker}{\pgfqpoint{-0.048611in}{0.000000in}}{\pgfqpoint{-0.000000in}{0.000000in}}{%
\pgfpathmoveto{\pgfqpoint{-0.000000in}{0.000000in}}%
\pgfpathlineto{\pgfqpoint{-0.048611in}{0.000000in}}%
\pgfusepath{stroke,fill}%
}%
\begin{pgfscope}%
\pgfsys@transformshift{0.742589in}{2.365019in}%
\pgfsys@useobject{currentmarker}{}%
\end{pgfscope}%
\end{pgfscope}%
\begin{pgfscope}%
\definecolor{textcolor}{rgb}{0.000000,0.000000,0.000000}%
\pgfsetstrokecolor{textcolor}%
\pgfsetfillcolor{textcolor}%
\pgftext[x=0.395138in, y=2.295575in, left, base]{\color{textcolor}{\rmfamily\fontsize{14.000000}{16.800000}\selectfont\catcode`\^=\active\def^{\ifmmode\sp\else\^{}\fi}\catcode`\%=\active\def%{\%}$\mathdefault{0.7}$}}%
\end{pgfscope}%
\begin{pgfscope}%
\pgfpathrectangle{\pgfqpoint{0.742589in}{0.670138in}}{\pgfqpoint{7.668292in}{4.846490in}}%
\pgfusepath{clip}%
\pgfsetrectcap%
\pgfsetroundjoin%
\pgfsetlinewidth{0.803000pt}%
\definecolor{currentstroke}{rgb}{0.690196,0.690196,0.690196}%
\pgfsetstrokecolor{currentstroke}%
\pgfsetdash{}{0pt}%
\pgfpathmoveto{\pgfqpoint{0.742589in}{3.415556in}}%
\pgfpathlineto{\pgfqpoint{8.410881in}{3.415556in}}%
\pgfusepath{stroke}%
\end{pgfscope}%
\begin{pgfscope}%
\pgfsetbuttcap%
\pgfsetroundjoin%
\definecolor{currentfill}{rgb}{0.000000,0.000000,0.000000}%
\pgfsetfillcolor{currentfill}%
\pgfsetlinewidth{0.803000pt}%
\definecolor{currentstroke}{rgb}{0.000000,0.000000,0.000000}%
\pgfsetstrokecolor{currentstroke}%
\pgfsetdash{}{0pt}%
\pgfsys@defobject{currentmarker}{\pgfqpoint{-0.048611in}{0.000000in}}{\pgfqpoint{-0.000000in}{0.000000in}}{%
\pgfpathmoveto{\pgfqpoint{-0.000000in}{0.000000in}}%
\pgfpathlineto{\pgfqpoint{-0.048611in}{0.000000in}}%
\pgfusepath{stroke,fill}%
}%
\begin{pgfscope}%
\pgfsys@transformshift{0.742589in}{3.415556in}%
\pgfsys@useobject{currentmarker}{}%
\end{pgfscope}%
\end{pgfscope}%
\begin{pgfscope}%
\definecolor{textcolor}{rgb}{0.000000,0.000000,0.000000}%
\pgfsetstrokecolor{textcolor}%
\pgfsetfillcolor{textcolor}%
\pgftext[x=0.395138in, y=3.346111in, left, base]{\color{textcolor}{\rmfamily\fontsize{14.000000}{16.800000}\selectfont\catcode`\^=\active\def^{\ifmmode\sp\else\^{}\fi}\catcode`\%=\active\def%{\%}$\mathdefault{0.8}$}}%
\end{pgfscope}%
\begin{pgfscope}%
\pgfpathrectangle{\pgfqpoint{0.742589in}{0.670138in}}{\pgfqpoint{7.668292in}{4.846490in}}%
\pgfusepath{clip}%
\pgfsetrectcap%
\pgfsetroundjoin%
\pgfsetlinewidth{0.803000pt}%
\definecolor{currentstroke}{rgb}{0.690196,0.690196,0.690196}%
\pgfsetstrokecolor{currentstroke}%
\pgfsetdash{}{0pt}%
\pgfpathmoveto{\pgfqpoint{0.742589in}{4.466092in}}%
\pgfpathlineto{\pgfqpoint{8.410881in}{4.466092in}}%
\pgfusepath{stroke}%
\end{pgfscope}%
\begin{pgfscope}%
\pgfsetbuttcap%
\pgfsetroundjoin%
\definecolor{currentfill}{rgb}{0.000000,0.000000,0.000000}%
\pgfsetfillcolor{currentfill}%
\pgfsetlinewidth{0.803000pt}%
\definecolor{currentstroke}{rgb}{0.000000,0.000000,0.000000}%
\pgfsetstrokecolor{currentstroke}%
\pgfsetdash{}{0pt}%
\pgfsys@defobject{currentmarker}{\pgfqpoint{-0.048611in}{0.000000in}}{\pgfqpoint{-0.000000in}{0.000000in}}{%
\pgfpathmoveto{\pgfqpoint{-0.000000in}{0.000000in}}%
\pgfpathlineto{\pgfqpoint{-0.048611in}{0.000000in}}%
\pgfusepath{stroke,fill}%
}%
\begin{pgfscope}%
\pgfsys@transformshift{0.742589in}{4.466092in}%
\pgfsys@useobject{currentmarker}{}%
\end{pgfscope}%
\end{pgfscope}%
\begin{pgfscope}%
\definecolor{textcolor}{rgb}{0.000000,0.000000,0.000000}%
\pgfsetstrokecolor{textcolor}%
\pgfsetfillcolor{textcolor}%
\pgftext[x=0.395138in, y=4.396648in, left, base]{\color{textcolor}{\rmfamily\fontsize{14.000000}{16.800000}\selectfont\catcode`\^=\active\def^{\ifmmode\sp\else\^{}\fi}\catcode`\%=\active\def%{\%}$\mathdefault{0.9}$}}%
\end{pgfscope}%
\begin{pgfscope}%
\pgfpathrectangle{\pgfqpoint{0.742589in}{0.670138in}}{\pgfqpoint{7.668292in}{4.846490in}}%
\pgfusepath{clip}%
\pgfsetrectcap%
\pgfsetroundjoin%
\pgfsetlinewidth{0.803000pt}%
\definecolor{currentstroke}{rgb}{0.690196,0.690196,0.690196}%
\pgfsetstrokecolor{currentstroke}%
\pgfsetdash{}{0pt}%
\pgfpathmoveto{\pgfqpoint{0.742589in}{5.516628in}}%
\pgfpathlineto{\pgfqpoint{8.410881in}{5.516628in}}%
\pgfusepath{stroke}%
\end{pgfscope}%
\begin{pgfscope}%
\pgfsetbuttcap%
\pgfsetroundjoin%
\definecolor{currentfill}{rgb}{0.000000,0.000000,0.000000}%
\pgfsetfillcolor{currentfill}%
\pgfsetlinewidth{0.803000pt}%
\definecolor{currentstroke}{rgb}{0.000000,0.000000,0.000000}%
\pgfsetstrokecolor{currentstroke}%
\pgfsetdash{}{0pt}%
\pgfsys@defobject{currentmarker}{\pgfqpoint{-0.048611in}{0.000000in}}{\pgfqpoint{-0.000000in}{0.000000in}}{%
\pgfpathmoveto{\pgfqpoint{-0.000000in}{0.000000in}}%
\pgfpathlineto{\pgfqpoint{-0.048611in}{0.000000in}}%
\pgfusepath{stroke,fill}%
}%
\begin{pgfscope}%
\pgfsys@transformshift{0.742589in}{5.516628in}%
\pgfsys@useobject{currentmarker}{}%
\end{pgfscope}%
\end{pgfscope}%
\begin{pgfscope}%
\definecolor{textcolor}{rgb}{0.000000,0.000000,0.000000}%
\pgfsetstrokecolor{textcolor}%
\pgfsetfillcolor{textcolor}%
\pgftext[x=0.395138in, y=5.447184in, left, base]{\color{textcolor}{\rmfamily\fontsize{14.000000}{16.800000}\selectfont\catcode`\^=\active\def^{\ifmmode\sp\else\^{}\fi}\catcode`\%=\active\def%{\%}$\mathdefault{1.0}$}}%
\end{pgfscope}%
\begin{pgfscope}%
\pgfsetbuttcap%
\pgfsetroundjoin%
\definecolor{currentfill}{rgb}{0.000000,0.000000,0.000000}%
\pgfsetfillcolor{currentfill}%
\pgfsetlinewidth{0.602250pt}%
\definecolor{currentstroke}{rgb}{0.000000,0.000000,0.000000}%
\pgfsetstrokecolor{currentstroke}%
\pgfsetdash{}{0pt}%
\pgfsys@defobject{currentmarker}{\pgfqpoint{-0.027778in}{0.000000in}}{\pgfqpoint{-0.000000in}{0.000000in}}{%
\pgfpathmoveto{\pgfqpoint{-0.000000in}{0.000000in}}%
\pgfpathlineto{\pgfqpoint{-0.027778in}{0.000000in}}%
\pgfusepath{stroke,fill}%
}%
\begin{pgfscope}%
\pgfsys@transformshift{0.742589in}{0.684161in}%
\pgfsys@useobject{currentmarker}{}%
\end{pgfscope}%
\end{pgfscope}%
\begin{pgfscope}%
\pgfsetbuttcap%
\pgfsetroundjoin%
\definecolor{currentfill}{rgb}{0.000000,0.000000,0.000000}%
\pgfsetfillcolor{currentfill}%
\pgfsetlinewidth{0.602250pt}%
\definecolor{currentstroke}{rgb}{0.000000,0.000000,0.000000}%
\pgfsetstrokecolor{currentstroke}%
\pgfsetdash{}{0pt}%
\pgfsys@defobject{currentmarker}{\pgfqpoint{-0.027778in}{0.000000in}}{\pgfqpoint{-0.000000in}{0.000000in}}{%
\pgfpathmoveto{\pgfqpoint{-0.000000in}{0.000000in}}%
\pgfpathlineto{\pgfqpoint{-0.027778in}{0.000000in}}%
\pgfusepath{stroke,fill}%
}%
\begin{pgfscope}%
\pgfsys@transformshift{0.742589in}{0.894269in}%
\pgfsys@useobject{currentmarker}{}%
\end{pgfscope}%
\end{pgfscope}%
\begin{pgfscope}%
\pgfsetbuttcap%
\pgfsetroundjoin%
\definecolor{currentfill}{rgb}{0.000000,0.000000,0.000000}%
\pgfsetfillcolor{currentfill}%
\pgfsetlinewidth{0.602250pt}%
\definecolor{currentstroke}{rgb}{0.000000,0.000000,0.000000}%
\pgfsetstrokecolor{currentstroke}%
\pgfsetdash{}{0pt}%
\pgfsys@defobject{currentmarker}{\pgfqpoint{-0.027778in}{0.000000in}}{\pgfqpoint{-0.000000in}{0.000000in}}{%
\pgfpathmoveto{\pgfqpoint{-0.000000in}{0.000000in}}%
\pgfpathlineto{\pgfqpoint{-0.027778in}{0.000000in}}%
\pgfusepath{stroke,fill}%
}%
\begin{pgfscope}%
\pgfsys@transformshift{0.742589in}{1.104376in}%
\pgfsys@useobject{currentmarker}{}%
\end{pgfscope}%
\end{pgfscope}%
\begin{pgfscope}%
\pgfsetbuttcap%
\pgfsetroundjoin%
\definecolor{currentfill}{rgb}{0.000000,0.000000,0.000000}%
\pgfsetfillcolor{currentfill}%
\pgfsetlinewidth{0.602250pt}%
\definecolor{currentstroke}{rgb}{0.000000,0.000000,0.000000}%
\pgfsetstrokecolor{currentstroke}%
\pgfsetdash{}{0pt}%
\pgfsys@defobject{currentmarker}{\pgfqpoint{-0.027778in}{0.000000in}}{\pgfqpoint{-0.000000in}{0.000000in}}{%
\pgfpathmoveto{\pgfqpoint{-0.000000in}{0.000000in}}%
\pgfpathlineto{\pgfqpoint{-0.027778in}{0.000000in}}%
\pgfusepath{stroke,fill}%
}%
\begin{pgfscope}%
\pgfsys@transformshift{0.742589in}{1.524590in}%
\pgfsys@useobject{currentmarker}{}%
\end{pgfscope}%
\end{pgfscope}%
\begin{pgfscope}%
\pgfsetbuttcap%
\pgfsetroundjoin%
\definecolor{currentfill}{rgb}{0.000000,0.000000,0.000000}%
\pgfsetfillcolor{currentfill}%
\pgfsetlinewidth{0.602250pt}%
\definecolor{currentstroke}{rgb}{0.000000,0.000000,0.000000}%
\pgfsetstrokecolor{currentstroke}%
\pgfsetdash{}{0pt}%
\pgfsys@defobject{currentmarker}{\pgfqpoint{-0.027778in}{0.000000in}}{\pgfqpoint{-0.000000in}{0.000000in}}{%
\pgfpathmoveto{\pgfqpoint{-0.000000in}{0.000000in}}%
\pgfpathlineto{\pgfqpoint{-0.027778in}{0.000000in}}%
\pgfusepath{stroke,fill}%
}%
\begin{pgfscope}%
\pgfsys@transformshift{0.742589in}{1.734698in}%
\pgfsys@useobject{currentmarker}{}%
\end{pgfscope}%
\end{pgfscope}%
\begin{pgfscope}%
\pgfsetbuttcap%
\pgfsetroundjoin%
\definecolor{currentfill}{rgb}{0.000000,0.000000,0.000000}%
\pgfsetfillcolor{currentfill}%
\pgfsetlinewidth{0.602250pt}%
\definecolor{currentstroke}{rgb}{0.000000,0.000000,0.000000}%
\pgfsetstrokecolor{currentstroke}%
\pgfsetdash{}{0pt}%
\pgfsys@defobject{currentmarker}{\pgfqpoint{-0.027778in}{0.000000in}}{\pgfqpoint{-0.000000in}{0.000000in}}{%
\pgfpathmoveto{\pgfqpoint{-0.000000in}{0.000000in}}%
\pgfpathlineto{\pgfqpoint{-0.027778in}{0.000000in}}%
\pgfusepath{stroke,fill}%
}%
\begin{pgfscope}%
\pgfsys@transformshift{0.742589in}{1.944805in}%
\pgfsys@useobject{currentmarker}{}%
\end{pgfscope}%
\end{pgfscope}%
\begin{pgfscope}%
\pgfsetbuttcap%
\pgfsetroundjoin%
\definecolor{currentfill}{rgb}{0.000000,0.000000,0.000000}%
\pgfsetfillcolor{currentfill}%
\pgfsetlinewidth{0.602250pt}%
\definecolor{currentstroke}{rgb}{0.000000,0.000000,0.000000}%
\pgfsetstrokecolor{currentstroke}%
\pgfsetdash{}{0pt}%
\pgfsys@defobject{currentmarker}{\pgfqpoint{-0.027778in}{0.000000in}}{\pgfqpoint{-0.000000in}{0.000000in}}{%
\pgfpathmoveto{\pgfqpoint{-0.000000in}{0.000000in}}%
\pgfpathlineto{\pgfqpoint{-0.027778in}{0.000000in}}%
\pgfusepath{stroke,fill}%
}%
\begin{pgfscope}%
\pgfsys@transformshift{0.742589in}{2.154912in}%
\pgfsys@useobject{currentmarker}{}%
\end{pgfscope}%
\end{pgfscope}%
\begin{pgfscope}%
\pgfsetbuttcap%
\pgfsetroundjoin%
\definecolor{currentfill}{rgb}{0.000000,0.000000,0.000000}%
\pgfsetfillcolor{currentfill}%
\pgfsetlinewidth{0.602250pt}%
\definecolor{currentstroke}{rgb}{0.000000,0.000000,0.000000}%
\pgfsetstrokecolor{currentstroke}%
\pgfsetdash{}{0pt}%
\pgfsys@defobject{currentmarker}{\pgfqpoint{-0.027778in}{0.000000in}}{\pgfqpoint{-0.000000in}{0.000000in}}{%
\pgfpathmoveto{\pgfqpoint{-0.000000in}{0.000000in}}%
\pgfpathlineto{\pgfqpoint{-0.027778in}{0.000000in}}%
\pgfusepath{stroke,fill}%
}%
\begin{pgfscope}%
\pgfsys@transformshift{0.742589in}{2.575127in}%
\pgfsys@useobject{currentmarker}{}%
\end{pgfscope}%
\end{pgfscope}%
\begin{pgfscope}%
\pgfsetbuttcap%
\pgfsetroundjoin%
\definecolor{currentfill}{rgb}{0.000000,0.000000,0.000000}%
\pgfsetfillcolor{currentfill}%
\pgfsetlinewidth{0.602250pt}%
\definecolor{currentstroke}{rgb}{0.000000,0.000000,0.000000}%
\pgfsetstrokecolor{currentstroke}%
\pgfsetdash{}{0pt}%
\pgfsys@defobject{currentmarker}{\pgfqpoint{-0.027778in}{0.000000in}}{\pgfqpoint{-0.000000in}{0.000000in}}{%
\pgfpathmoveto{\pgfqpoint{-0.000000in}{0.000000in}}%
\pgfpathlineto{\pgfqpoint{-0.027778in}{0.000000in}}%
\pgfusepath{stroke,fill}%
}%
\begin{pgfscope}%
\pgfsys@transformshift{0.742589in}{2.785234in}%
\pgfsys@useobject{currentmarker}{}%
\end{pgfscope}%
\end{pgfscope}%
\begin{pgfscope}%
\pgfsetbuttcap%
\pgfsetroundjoin%
\definecolor{currentfill}{rgb}{0.000000,0.000000,0.000000}%
\pgfsetfillcolor{currentfill}%
\pgfsetlinewidth{0.602250pt}%
\definecolor{currentstroke}{rgb}{0.000000,0.000000,0.000000}%
\pgfsetstrokecolor{currentstroke}%
\pgfsetdash{}{0pt}%
\pgfsys@defobject{currentmarker}{\pgfqpoint{-0.027778in}{0.000000in}}{\pgfqpoint{-0.000000in}{0.000000in}}{%
\pgfpathmoveto{\pgfqpoint{-0.000000in}{0.000000in}}%
\pgfpathlineto{\pgfqpoint{-0.027778in}{0.000000in}}%
\pgfusepath{stroke,fill}%
}%
\begin{pgfscope}%
\pgfsys@transformshift{0.742589in}{2.995341in}%
\pgfsys@useobject{currentmarker}{}%
\end{pgfscope}%
\end{pgfscope}%
\begin{pgfscope}%
\pgfsetbuttcap%
\pgfsetroundjoin%
\definecolor{currentfill}{rgb}{0.000000,0.000000,0.000000}%
\pgfsetfillcolor{currentfill}%
\pgfsetlinewidth{0.602250pt}%
\definecolor{currentstroke}{rgb}{0.000000,0.000000,0.000000}%
\pgfsetstrokecolor{currentstroke}%
\pgfsetdash{}{0pt}%
\pgfsys@defobject{currentmarker}{\pgfqpoint{-0.027778in}{0.000000in}}{\pgfqpoint{-0.000000in}{0.000000in}}{%
\pgfpathmoveto{\pgfqpoint{-0.000000in}{0.000000in}}%
\pgfpathlineto{\pgfqpoint{-0.027778in}{0.000000in}}%
\pgfusepath{stroke,fill}%
}%
\begin{pgfscope}%
\pgfsys@transformshift{0.742589in}{3.205448in}%
\pgfsys@useobject{currentmarker}{}%
\end{pgfscope}%
\end{pgfscope}%
\begin{pgfscope}%
\pgfsetbuttcap%
\pgfsetroundjoin%
\definecolor{currentfill}{rgb}{0.000000,0.000000,0.000000}%
\pgfsetfillcolor{currentfill}%
\pgfsetlinewidth{0.602250pt}%
\definecolor{currentstroke}{rgb}{0.000000,0.000000,0.000000}%
\pgfsetstrokecolor{currentstroke}%
\pgfsetdash{}{0pt}%
\pgfsys@defobject{currentmarker}{\pgfqpoint{-0.027778in}{0.000000in}}{\pgfqpoint{-0.000000in}{0.000000in}}{%
\pgfpathmoveto{\pgfqpoint{-0.000000in}{0.000000in}}%
\pgfpathlineto{\pgfqpoint{-0.027778in}{0.000000in}}%
\pgfusepath{stroke,fill}%
}%
\begin{pgfscope}%
\pgfsys@transformshift{0.742589in}{3.625663in}%
\pgfsys@useobject{currentmarker}{}%
\end{pgfscope}%
\end{pgfscope}%
\begin{pgfscope}%
\pgfsetbuttcap%
\pgfsetroundjoin%
\definecolor{currentfill}{rgb}{0.000000,0.000000,0.000000}%
\pgfsetfillcolor{currentfill}%
\pgfsetlinewidth{0.602250pt}%
\definecolor{currentstroke}{rgb}{0.000000,0.000000,0.000000}%
\pgfsetstrokecolor{currentstroke}%
\pgfsetdash{}{0pt}%
\pgfsys@defobject{currentmarker}{\pgfqpoint{-0.027778in}{0.000000in}}{\pgfqpoint{-0.000000in}{0.000000in}}{%
\pgfpathmoveto{\pgfqpoint{-0.000000in}{0.000000in}}%
\pgfpathlineto{\pgfqpoint{-0.027778in}{0.000000in}}%
\pgfusepath{stroke,fill}%
}%
\begin{pgfscope}%
\pgfsys@transformshift{0.742589in}{3.835770in}%
\pgfsys@useobject{currentmarker}{}%
\end{pgfscope}%
\end{pgfscope}%
\begin{pgfscope}%
\pgfsetbuttcap%
\pgfsetroundjoin%
\definecolor{currentfill}{rgb}{0.000000,0.000000,0.000000}%
\pgfsetfillcolor{currentfill}%
\pgfsetlinewidth{0.602250pt}%
\definecolor{currentstroke}{rgb}{0.000000,0.000000,0.000000}%
\pgfsetstrokecolor{currentstroke}%
\pgfsetdash{}{0pt}%
\pgfsys@defobject{currentmarker}{\pgfqpoint{-0.027778in}{0.000000in}}{\pgfqpoint{-0.000000in}{0.000000in}}{%
\pgfpathmoveto{\pgfqpoint{-0.000000in}{0.000000in}}%
\pgfpathlineto{\pgfqpoint{-0.027778in}{0.000000in}}%
\pgfusepath{stroke,fill}%
}%
\begin{pgfscope}%
\pgfsys@transformshift{0.742589in}{4.045877in}%
\pgfsys@useobject{currentmarker}{}%
\end{pgfscope}%
\end{pgfscope}%
\begin{pgfscope}%
\pgfsetbuttcap%
\pgfsetroundjoin%
\definecolor{currentfill}{rgb}{0.000000,0.000000,0.000000}%
\pgfsetfillcolor{currentfill}%
\pgfsetlinewidth{0.602250pt}%
\definecolor{currentstroke}{rgb}{0.000000,0.000000,0.000000}%
\pgfsetstrokecolor{currentstroke}%
\pgfsetdash{}{0pt}%
\pgfsys@defobject{currentmarker}{\pgfqpoint{-0.027778in}{0.000000in}}{\pgfqpoint{-0.000000in}{0.000000in}}{%
\pgfpathmoveto{\pgfqpoint{-0.000000in}{0.000000in}}%
\pgfpathlineto{\pgfqpoint{-0.027778in}{0.000000in}}%
\pgfusepath{stroke,fill}%
}%
\begin{pgfscope}%
\pgfsys@transformshift{0.742589in}{4.255985in}%
\pgfsys@useobject{currentmarker}{}%
\end{pgfscope}%
\end{pgfscope}%
\begin{pgfscope}%
\pgfsetbuttcap%
\pgfsetroundjoin%
\definecolor{currentfill}{rgb}{0.000000,0.000000,0.000000}%
\pgfsetfillcolor{currentfill}%
\pgfsetlinewidth{0.602250pt}%
\definecolor{currentstroke}{rgb}{0.000000,0.000000,0.000000}%
\pgfsetstrokecolor{currentstroke}%
\pgfsetdash{}{0pt}%
\pgfsys@defobject{currentmarker}{\pgfqpoint{-0.027778in}{0.000000in}}{\pgfqpoint{-0.000000in}{0.000000in}}{%
\pgfpathmoveto{\pgfqpoint{-0.000000in}{0.000000in}}%
\pgfpathlineto{\pgfqpoint{-0.027778in}{0.000000in}}%
\pgfusepath{stroke,fill}%
}%
\begin{pgfscope}%
\pgfsys@transformshift{0.742589in}{4.676199in}%
\pgfsys@useobject{currentmarker}{}%
\end{pgfscope}%
\end{pgfscope}%
\begin{pgfscope}%
\pgfsetbuttcap%
\pgfsetroundjoin%
\definecolor{currentfill}{rgb}{0.000000,0.000000,0.000000}%
\pgfsetfillcolor{currentfill}%
\pgfsetlinewidth{0.602250pt}%
\definecolor{currentstroke}{rgb}{0.000000,0.000000,0.000000}%
\pgfsetstrokecolor{currentstroke}%
\pgfsetdash{}{0pt}%
\pgfsys@defobject{currentmarker}{\pgfqpoint{-0.027778in}{0.000000in}}{\pgfqpoint{-0.000000in}{0.000000in}}{%
\pgfpathmoveto{\pgfqpoint{-0.000000in}{0.000000in}}%
\pgfpathlineto{\pgfqpoint{-0.027778in}{0.000000in}}%
\pgfusepath{stroke,fill}%
}%
\begin{pgfscope}%
\pgfsys@transformshift{0.742589in}{4.886306in}%
\pgfsys@useobject{currentmarker}{}%
\end{pgfscope}%
\end{pgfscope}%
\begin{pgfscope}%
\pgfsetbuttcap%
\pgfsetroundjoin%
\definecolor{currentfill}{rgb}{0.000000,0.000000,0.000000}%
\pgfsetfillcolor{currentfill}%
\pgfsetlinewidth{0.602250pt}%
\definecolor{currentstroke}{rgb}{0.000000,0.000000,0.000000}%
\pgfsetstrokecolor{currentstroke}%
\pgfsetdash{}{0pt}%
\pgfsys@defobject{currentmarker}{\pgfqpoint{-0.027778in}{0.000000in}}{\pgfqpoint{-0.000000in}{0.000000in}}{%
\pgfpathmoveto{\pgfqpoint{-0.000000in}{0.000000in}}%
\pgfpathlineto{\pgfqpoint{-0.027778in}{0.000000in}}%
\pgfusepath{stroke,fill}%
}%
\begin{pgfscope}%
\pgfsys@transformshift{0.742589in}{5.096414in}%
\pgfsys@useobject{currentmarker}{}%
\end{pgfscope}%
\end{pgfscope}%
\begin{pgfscope}%
\pgfsetbuttcap%
\pgfsetroundjoin%
\definecolor{currentfill}{rgb}{0.000000,0.000000,0.000000}%
\pgfsetfillcolor{currentfill}%
\pgfsetlinewidth{0.602250pt}%
\definecolor{currentstroke}{rgb}{0.000000,0.000000,0.000000}%
\pgfsetstrokecolor{currentstroke}%
\pgfsetdash{}{0pt}%
\pgfsys@defobject{currentmarker}{\pgfqpoint{-0.027778in}{0.000000in}}{\pgfqpoint{-0.000000in}{0.000000in}}{%
\pgfpathmoveto{\pgfqpoint{-0.000000in}{0.000000in}}%
\pgfpathlineto{\pgfqpoint{-0.027778in}{0.000000in}}%
\pgfusepath{stroke,fill}%
}%
\begin{pgfscope}%
\pgfsys@transformshift{0.742589in}{5.306521in}%
\pgfsys@useobject{currentmarker}{}%
\end{pgfscope}%
\end{pgfscope}%
\begin{pgfscope}%
\definecolor{textcolor}{rgb}{0.000000,0.000000,0.000000}%
\pgfsetstrokecolor{textcolor}%
\pgfsetfillcolor{textcolor}%
\pgftext[x=0.339583in,y=3.093383in,,bottom,rotate=90.000000]{\color{textcolor}{\rmfamily\fontsize{18.000000}{21.600000}\selectfont\catcode`\^=\active\def^{\ifmmode\sp\else\^{}\fi}\catcode`\%=\active\def%{\%}Demand (--)}}%
\end{pgfscope}%
\begin{pgfscope}%
\pgfpathrectangle{\pgfqpoint{0.742589in}{0.670138in}}{\pgfqpoint{7.668292in}{4.846490in}}%
\pgfusepath{clip}%
\pgfsetrectcap%
\pgfsetroundjoin%
\pgfsetlinewidth{1.505625pt}%
\definecolor{currentstroke}{rgb}{0.121569,0.466667,0.705882}%
\pgfsetstrokecolor{currentstroke}%
\pgfsetdash{}{0pt}%
\pgfpathmoveto{\pgfqpoint{0.742589in}{1.132052in}}%
\pgfpathlineto{\pgfqpoint{0.743464in}{1.245779in}}%
\pgfpathlineto{\pgfqpoint{0.744340in}{1.216338in}}%
\pgfpathlineto{\pgfqpoint{0.745215in}{1.221251in}}%
\pgfpathlineto{\pgfqpoint{0.746090in}{1.160756in}}%
\pgfpathlineto{\pgfqpoint{0.746966in}{1.165521in}}%
\pgfpathlineto{\pgfqpoint{0.748717in}{1.289336in}}%
\pgfpathlineto{\pgfqpoint{0.749592in}{1.302128in}}%
\pgfpathlineto{\pgfqpoint{0.750467in}{1.291270in}}%
\pgfpathlineto{\pgfqpoint{0.751343in}{1.247104in}}%
\pgfpathlineto{\pgfqpoint{0.752218in}{1.239478in}}%
\pgfpathlineto{\pgfqpoint{0.753093in}{1.237325in}}%
\pgfpathlineto{\pgfqpoint{0.753969in}{1.271787in}}%
\pgfpathlineto{\pgfqpoint{0.754844in}{1.335894in}}%
\pgfpathlineto{\pgfqpoint{0.755720in}{1.303995in}}%
\pgfpathlineto{\pgfqpoint{0.756595in}{1.314810in}}%
\pgfpathlineto{\pgfqpoint{0.757470in}{1.448070in}}%
\pgfpathlineto{\pgfqpoint{0.758346in}{1.390678in}}%
\pgfpathlineto{\pgfqpoint{0.759221in}{1.363946in}}%
\pgfpathlineto{\pgfqpoint{0.760972in}{1.267558in}}%
\pgfpathlineto{\pgfqpoint{0.761847in}{1.207115in}}%
\pgfpathlineto{\pgfqpoint{0.762723in}{1.193492in}}%
\pgfpathlineto{\pgfqpoint{0.763598in}{1.088857in}}%
\pgfpathlineto{\pgfqpoint{0.764473in}{1.044786in}}%
\pgfpathlineto{\pgfqpoint{0.765349in}{1.030287in}}%
\pgfpathlineto{\pgfqpoint{0.766224in}{1.001229in}}%
\pgfpathlineto{\pgfqpoint{0.767099in}{1.043669in}}%
\pgfpathlineto{\pgfqpoint{0.767975in}{1.102631in}}%
\pgfpathlineto{\pgfqpoint{0.768850in}{1.131996in}}%
\pgfpathlineto{\pgfqpoint{0.770601in}{1.488399in}}%
\pgfpathlineto{\pgfqpoint{0.771476in}{1.526882in}}%
\pgfpathlineto{\pgfqpoint{0.773227in}{1.686619in}}%
\pgfpathlineto{\pgfqpoint{0.774102in}{1.669855in}}%
\pgfpathlineto{\pgfqpoint{0.774978in}{1.703433in}}%
\pgfpathlineto{\pgfqpoint{0.775853in}{1.782670in}}%
\pgfpathlineto{\pgfqpoint{0.776729in}{1.788185in}}%
\pgfpathlineto{\pgfqpoint{0.777604in}{1.752532in}}%
\pgfpathlineto{\pgfqpoint{0.778479in}{1.775174in}}%
\pgfpathlineto{\pgfqpoint{0.779355in}{1.733376in}}%
\pgfpathlineto{\pgfqpoint{0.781105in}{1.579322in}}%
\pgfpathlineto{\pgfqpoint{0.781981in}{1.547479in}}%
\pgfpathlineto{\pgfqpoint{0.782856in}{1.572399in}}%
\pgfpathlineto{\pgfqpoint{0.783732in}{1.581552in}}%
\pgfpathlineto{\pgfqpoint{0.784607in}{1.445654in}}%
\pgfpathlineto{\pgfqpoint{0.785482in}{1.405966in}}%
\pgfpathlineto{\pgfqpoint{0.786358in}{1.385634in}}%
\pgfpathlineto{\pgfqpoint{0.787233in}{1.418710in}}%
\pgfpathlineto{\pgfqpoint{0.788108in}{1.359808in}}%
\pgfpathlineto{\pgfqpoint{0.788984in}{1.373249in}}%
\pgfpathlineto{\pgfqpoint{0.789859in}{1.376454in}}%
\pgfpathlineto{\pgfqpoint{0.790735in}{1.441425in}}%
\pgfpathlineto{\pgfqpoint{0.791610in}{1.612144in}}%
\pgfpathlineto{\pgfqpoint{0.792485in}{1.692428in}}%
\pgfpathlineto{\pgfqpoint{0.793361in}{1.740962in}}%
\pgfpathlineto{\pgfqpoint{0.794236in}{1.806357in}}%
\pgfpathlineto{\pgfqpoint{0.795111in}{1.771252in}}%
\pgfpathlineto{\pgfqpoint{0.795987in}{1.802901in}}%
\pgfpathlineto{\pgfqpoint{0.796862in}{1.872716in}}%
\pgfpathlineto{\pgfqpoint{0.797738in}{1.853225in}}%
\pgfpathlineto{\pgfqpoint{0.798613in}{1.815235in}}%
\pgfpathlineto{\pgfqpoint{0.799488in}{1.751100in}}%
\pgfpathlineto{\pgfqpoint{0.801239in}{1.442029in}}%
\pgfpathlineto{\pgfqpoint{0.802114in}{1.415689in}}%
\pgfpathlineto{\pgfqpoint{0.802990in}{1.335854in}}%
\pgfpathlineto{\pgfqpoint{0.803865in}{1.284292in}}%
\pgfpathlineto{\pgfqpoint{0.804741in}{1.259311in}}%
\pgfpathlineto{\pgfqpoint{0.805616in}{1.223547in}}%
\pgfpathlineto{\pgfqpoint{0.806491in}{1.174402in}}%
\pgfpathlineto{\pgfqpoint{0.807367in}{1.183040in}}%
\pgfpathlineto{\pgfqpoint{0.808242in}{1.208716in}}%
\pgfpathlineto{\pgfqpoint{0.809117in}{1.151535in}}%
\pgfpathlineto{\pgfqpoint{0.809993in}{1.139876in}}%
\pgfpathlineto{\pgfqpoint{0.810868in}{1.178661in}}%
\pgfpathlineto{\pgfqpoint{0.813494in}{1.825786in}}%
\pgfpathlineto{\pgfqpoint{0.814370in}{1.905465in}}%
\pgfpathlineto{\pgfqpoint{0.815245in}{1.705974in}}%
\pgfpathlineto{\pgfqpoint{0.816120in}{1.714410in}}%
\pgfpathlineto{\pgfqpoint{0.817871in}{1.975193in}}%
\pgfpathlineto{\pgfqpoint{0.818747in}{1.991798in}}%
\pgfpathlineto{\pgfqpoint{0.819622in}{1.961488in}}%
\pgfpathlineto{\pgfqpoint{0.820497in}{1.966440in}}%
\pgfpathlineto{\pgfqpoint{0.821373in}{1.720470in}}%
\pgfpathlineto{\pgfqpoint{0.823123in}{1.504556in}}%
\pgfpathlineto{\pgfqpoint{0.823999in}{1.496733in}}%
\pgfpathlineto{\pgfqpoint{0.825750in}{1.314287in}}%
\pgfpathlineto{\pgfqpoint{0.826625in}{1.208957in}}%
\pgfpathlineto{\pgfqpoint{0.827500in}{1.223305in}}%
\pgfpathlineto{\pgfqpoint{0.828376in}{1.186937in}}%
\pgfpathlineto{\pgfqpoint{0.829251in}{1.127219in}}%
\pgfpathlineto{\pgfqpoint{0.830126in}{1.125256in}}%
\pgfpathlineto{\pgfqpoint{0.831002in}{1.202946in}}%
\pgfpathlineto{\pgfqpoint{0.831877in}{1.313894in}}%
\pgfpathlineto{\pgfqpoint{0.832753in}{1.546150in}}%
\pgfpathlineto{\pgfqpoint{0.833628in}{1.659025in}}%
\pgfpathlineto{\pgfqpoint{0.834503in}{1.870682in}}%
\pgfpathlineto{\pgfqpoint{0.835379in}{1.966540in}}%
\pgfpathlineto{\pgfqpoint{0.836254in}{2.162565in}}%
\pgfpathlineto{\pgfqpoint{0.837129in}{2.150895in}}%
\pgfpathlineto{\pgfqpoint{0.838005in}{2.103478in}}%
\pgfpathlineto{\pgfqpoint{0.838880in}{2.089315in}}%
\pgfpathlineto{\pgfqpoint{0.839756in}{2.024607in}}%
\pgfpathlineto{\pgfqpoint{0.841506in}{1.806226in}}%
\pgfpathlineto{\pgfqpoint{0.842382in}{1.550681in}}%
\pgfpathlineto{\pgfqpoint{0.843257in}{1.407594in}}%
\pgfpathlineto{\pgfqpoint{0.844132in}{1.366483in}}%
\pgfpathlineto{\pgfqpoint{0.845883in}{1.323772in}}%
\pgfpathlineto{\pgfqpoint{0.846759in}{1.241369in}}%
\pgfpathlineto{\pgfqpoint{0.847634in}{1.208686in}}%
\pgfpathlineto{\pgfqpoint{0.848509in}{1.159329in}}%
\pgfpathlineto{\pgfqpoint{0.849385in}{1.165793in}}%
\pgfpathlineto{\pgfqpoint{0.850260in}{1.127219in}}%
\pgfpathlineto{\pgfqpoint{0.851135in}{1.158362in}}%
\pgfpathlineto{\pgfqpoint{0.852011in}{1.235720in}}%
\pgfpathlineto{\pgfqpoint{0.852886in}{1.247803in}}%
\pgfpathlineto{\pgfqpoint{0.853762in}{1.359777in}}%
\pgfpathlineto{\pgfqpoint{0.854637in}{1.706307in}}%
\pgfpathlineto{\pgfqpoint{0.855512in}{1.817685in}}%
\pgfpathlineto{\pgfqpoint{0.856388in}{1.759746in}}%
\pgfpathlineto{\pgfqpoint{0.857263in}{1.935605in}}%
\pgfpathlineto{\pgfqpoint{0.858138in}{1.936561in}}%
\pgfpathlineto{\pgfqpoint{0.859014in}{1.985307in}}%
\pgfpathlineto{\pgfqpoint{0.859889in}{1.965360in}}%
\pgfpathlineto{\pgfqpoint{0.860765in}{2.061121in}}%
\pgfpathlineto{\pgfqpoint{0.861640in}{1.969974in}}%
\pgfpathlineto{\pgfqpoint{0.862515in}{1.926024in}}%
\pgfpathlineto{\pgfqpoint{0.864266in}{1.618947in}}%
\pgfpathlineto{\pgfqpoint{0.865142in}{1.588016in}}%
\pgfpathlineto{\pgfqpoint{0.866017in}{1.529808in}}%
\pgfpathlineto{\pgfqpoint{0.866892in}{1.515732in}}%
\pgfpathlineto{\pgfqpoint{0.867768in}{1.449792in}}%
\pgfpathlineto{\pgfqpoint{0.868643in}{1.326943in}}%
\pgfpathlineto{\pgfqpoint{0.869518in}{1.352437in}}%
\pgfpathlineto{\pgfqpoint{0.870394in}{1.289699in}}%
\pgfpathlineto{\pgfqpoint{0.871269in}{1.284413in}}%
\pgfpathlineto{\pgfqpoint{0.872145in}{1.242909in}}%
\pgfpathlineto{\pgfqpoint{0.873020in}{1.255777in}}%
\pgfpathlineto{\pgfqpoint{0.876521in}{1.893999in}}%
\pgfpathlineto{\pgfqpoint{0.877397in}{1.869922in}}%
\pgfpathlineto{\pgfqpoint{0.878272in}{2.001243in}}%
\pgfpathlineto{\pgfqpoint{0.879148in}{2.006768in}}%
\pgfpathlineto{\pgfqpoint{0.880023in}{1.947204in}}%
\pgfpathlineto{\pgfqpoint{0.880898in}{2.027985in}}%
\pgfpathlineto{\pgfqpoint{0.881774in}{2.055950in}}%
\pgfpathlineto{\pgfqpoint{0.882649in}{2.049790in}}%
\pgfpathlineto{\pgfqpoint{0.883524in}{2.005708in}}%
\pgfpathlineto{\pgfqpoint{0.885275in}{1.641662in}}%
\pgfpathlineto{\pgfqpoint{0.886151in}{1.610943in}}%
\pgfpathlineto{\pgfqpoint{0.887901in}{1.428557in}}%
\pgfpathlineto{\pgfqpoint{0.888777in}{1.428557in}}%
\pgfpathlineto{\pgfqpoint{0.889652in}{1.407745in}}%
\pgfpathlineto{\pgfqpoint{0.891403in}{1.298187in}}%
\pgfpathlineto{\pgfqpoint{0.892278in}{1.319210in}}%
\pgfpathlineto{\pgfqpoint{0.893154in}{1.286950in}}%
\pgfpathlineto{\pgfqpoint{0.894029in}{1.242245in}}%
\pgfpathlineto{\pgfqpoint{0.894904in}{1.310420in}}%
\pgfpathlineto{\pgfqpoint{0.895780in}{1.470000in}}%
\pgfpathlineto{\pgfqpoint{0.896655in}{1.787871in}}%
\pgfpathlineto{\pgfqpoint{0.897530in}{1.968621in}}%
\pgfpathlineto{\pgfqpoint{0.899281in}{2.101663in}}%
\pgfpathlineto{\pgfqpoint{0.900157in}{2.092589in}}%
\pgfpathlineto{\pgfqpoint{0.901032in}{2.007860in}}%
\pgfpathlineto{\pgfqpoint{0.901907in}{2.148298in}}%
\pgfpathlineto{\pgfqpoint{0.902783in}{2.199252in}}%
\pgfpathlineto{\pgfqpoint{0.903658in}{2.203700in}}%
\pgfpathlineto{\pgfqpoint{0.904533in}{2.012928in}}%
\pgfpathlineto{\pgfqpoint{0.906284in}{1.786350in}}%
\pgfpathlineto{\pgfqpoint{0.907160in}{1.730559in}}%
\pgfpathlineto{\pgfqpoint{0.908910in}{1.570073in}}%
\pgfpathlineto{\pgfqpoint{0.909786in}{1.555816in}}%
\pgfpathlineto{\pgfqpoint{0.910661in}{1.442663in}}%
\pgfpathlineto{\pgfqpoint{0.912412in}{1.377539in}}%
\pgfpathlineto{\pgfqpoint{0.913287in}{1.303503in}}%
\pgfpathlineto{\pgfqpoint{0.914163in}{1.271122in}}%
\pgfpathlineto{\pgfqpoint{0.915913in}{1.427470in}}%
\pgfpathlineto{\pgfqpoint{0.916789in}{1.635772in}}%
\pgfpathlineto{\pgfqpoint{0.917664in}{1.973933in}}%
\pgfpathlineto{\pgfqpoint{0.918539in}{2.169993in}}%
\pgfpathlineto{\pgfqpoint{0.919415in}{2.138729in}}%
\pgfpathlineto{\pgfqpoint{0.920290in}{1.890687in}}%
\pgfpathlineto{\pgfqpoint{0.921166in}{2.062379in}}%
\pgfpathlineto{\pgfqpoint{0.922916in}{2.238429in}}%
\pgfpathlineto{\pgfqpoint{0.925542in}{2.067812in}}%
\pgfpathlineto{\pgfqpoint{0.927293in}{1.608435in}}%
\pgfpathlineto{\pgfqpoint{0.928169in}{1.592698in}}%
\pgfpathlineto{\pgfqpoint{0.929919in}{1.356424in}}%
\pgfpathlineto{\pgfqpoint{0.931670in}{1.410826in}}%
\pgfpathlineto{\pgfqpoint{0.933421in}{1.362013in}}%
\pgfpathlineto{\pgfqpoint{0.934296in}{1.329027in}}%
\pgfpathlineto{\pgfqpoint{0.935172in}{1.327306in}}%
\pgfpathlineto{\pgfqpoint{0.936047in}{1.364248in}}%
\pgfpathlineto{\pgfqpoint{0.936922in}{1.444838in}}%
\pgfpathlineto{\pgfqpoint{0.937798in}{1.608345in}}%
\pgfpathlineto{\pgfqpoint{0.938673in}{1.854410in}}%
\pgfpathlineto{\pgfqpoint{0.939548in}{1.985475in}}%
\pgfpathlineto{\pgfqpoint{0.940424in}{2.185401in}}%
\pgfpathlineto{\pgfqpoint{0.941299in}{2.174708in}}%
\pgfpathlineto{\pgfqpoint{0.942175in}{2.176110in}}%
\pgfpathlineto{\pgfqpoint{0.943050in}{2.143377in}}%
\pgfpathlineto{\pgfqpoint{0.943925in}{2.145558in}}%
\pgfpathlineto{\pgfqpoint{0.944801in}{2.273558in}}%
\pgfpathlineto{\pgfqpoint{0.945676in}{2.320585in}}%
\pgfpathlineto{\pgfqpoint{0.946551in}{2.262099in}}%
\pgfpathlineto{\pgfqpoint{0.947427in}{2.031384in}}%
\pgfpathlineto{\pgfqpoint{0.951804in}{1.466224in}}%
\pgfpathlineto{\pgfqpoint{0.953554in}{1.356968in}}%
\pgfpathlineto{\pgfqpoint{0.954430in}{1.343224in}}%
\pgfpathlineto{\pgfqpoint{0.955305in}{1.355186in}}%
\pgfpathlineto{\pgfqpoint{0.956181in}{1.384969in}}%
\pgfpathlineto{\pgfqpoint{0.957056in}{1.389772in}}%
\pgfpathlineto{\pgfqpoint{0.957931in}{1.480482in}}%
\pgfpathlineto{\pgfqpoint{0.960557in}{2.090452in}}%
\pgfpathlineto{\pgfqpoint{0.961433in}{2.180611in}}%
\pgfpathlineto{\pgfqpoint{0.962308in}{2.368521in}}%
\pgfpathlineto{\pgfqpoint{0.963184in}{2.365122in}}%
\pgfpathlineto{\pgfqpoint{0.964059in}{2.322674in}}%
\pgfpathlineto{\pgfqpoint{0.964934in}{2.404865in}}%
\pgfpathlineto{\pgfqpoint{0.965810in}{2.348286in}}%
\pgfpathlineto{\pgfqpoint{0.966685in}{2.353749in}}%
\pgfpathlineto{\pgfqpoint{0.967560in}{2.287654in}}%
\pgfpathlineto{\pgfqpoint{0.968436in}{1.964809in}}%
\pgfpathlineto{\pgfqpoint{0.969311in}{1.798070in}}%
\pgfpathlineto{\pgfqpoint{0.971062in}{1.709052in}}%
\pgfpathlineto{\pgfqpoint{0.971937in}{1.616500in}}%
\pgfpathlineto{\pgfqpoint{0.972813in}{1.424479in}}%
\pgfpathlineto{\pgfqpoint{0.973688in}{1.322171in}}%
\pgfpathlineto{\pgfqpoint{0.974563in}{1.336005in}}%
\pgfpathlineto{\pgfqpoint{0.975439in}{1.247168in}}%
\pgfpathlineto{\pgfqpoint{0.976314in}{1.270790in}}%
\pgfpathlineto{\pgfqpoint{0.977190in}{1.271817in}}%
\pgfpathlineto{\pgfqpoint{0.978065in}{1.222007in}}%
\pgfpathlineto{\pgfqpoint{0.978940in}{1.302959in}}%
\pgfpathlineto{\pgfqpoint{0.979816in}{1.445926in}}%
\pgfpathlineto{\pgfqpoint{0.980691in}{1.702280in}}%
\pgfpathlineto{\pgfqpoint{0.981566in}{2.068034in}}%
\pgfpathlineto{\pgfqpoint{0.982442in}{2.131220in}}%
\pgfpathlineto{\pgfqpoint{0.983317in}{2.233517in}}%
\pgfpathlineto{\pgfqpoint{0.984193in}{2.227409in}}%
\pgfpathlineto{\pgfqpoint{0.985068in}{2.194371in}}%
\pgfpathlineto{\pgfqpoint{0.985943in}{2.355786in}}%
\pgfpathlineto{\pgfqpoint{0.986819in}{2.327715in}}%
\pgfpathlineto{\pgfqpoint{0.987694in}{2.255116in}}%
\pgfpathlineto{\pgfqpoint{0.988569in}{2.150094in}}%
\pgfpathlineto{\pgfqpoint{0.989445in}{1.944571in}}%
\pgfpathlineto{\pgfqpoint{0.990320in}{1.920496in}}%
\pgfpathlineto{\pgfqpoint{0.991196in}{1.875096in}}%
\pgfpathlineto{\pgfqpoint{0.994697in}{1.600491in}}%
\pgfpathlineto{\pgfqpoint{0.997323in}{1.452541in}}%
\pgfpathlineto{\pgfqpoint{0.998199in}{1.433269in}}%
\pgfpathlineto{\pgfqpoint{0.999074in}{1.404936in}}%
\pgfpathlineto{\pgfqpoint{0.999949in}{1.428497in}}%
\pgfpathlineto{\pgfqpoint{1.000825in}{1.614597in}}%
\pgfpathlineto{\pgfqpoint{1.001700in}{1.883975in}}%
\pgfpathlineto{\pgfqpoint{1.002576in}{2.050410in}}%
\pgfpathlineto{\pgfqpoint{1.004326in}{2.272140in}}%
\pgfpathlineto{\pgfqpoint{1.005202in}{2.268959in}}%
\pgfpathlineto{\pgfqpoint{1.006077in}{2.224335in}}%
\pgfpathlineto{\pgfqpoint{1.006952in}{2.118933in}}%
\pgfpathlineto{\pgfqpoint{1.007828in}{2.231970in}}%
\pgfpathlineto{\pgfqpoint{1.008703in}{2.204189in}}%
\pgfpathlineto{\pgfqpoint{1.009579in}{2.114783in}}%
\pgfpathlineto{\pgfqpoint{1.010454in}{1.908958in}}%
\pgfpathlineto{\pgfqpoint{1.011329in}{1.786018in}}%
\pgfpathlineto{\pgfqpoint{1.012205in}{1.741524in}}%
\pgfpathlineto{\pgfqpoint{1.013080in}{1.588318in}}%
\pgfpathlineto{\pgfqpoint{1.014831in}{1.462267in}}%
\pgfpathlineto{\pgfqpoint{1.015706in}{1.337455in}}%
\pgfpathlineto{\pgfqpoint{1.016582in}{1.292115in}}%
\pgfpathlineto{\pgfqpoint{1.017457in}{1.298277in}}%
\pgfpathlineto{\pgfqpoint{1.020083in}{1.246897in}}%
\pgfpathlineto{\pgfqpoint{1.020958in}{1.302899in}}%
\pgfpathlineto{\pgfqpoint{1.021834in}{1.460304in}}%
\pgfpathlineto{\pgfqpoint{1.023585in}{1.943637in}}%
\pgfpathlineto{\pgfqpoint{1.025335in}{2.112871in}}%
\pgfpathlineto{\pgfqpoint{1.026211in}{2.126361in}}%
\pgfpathlineto{\pgfqpoint{1.027086in}{2.088747in}}%
\pgfpathlineto{\pgfqpoint{1.027961in}{2.184290in}}%
\pgfpathlineto{\pgfqpoint{1.028837in}{2.175365in}}%
\pgfpathlineto{\pgfqpoint{1.029712in}{2.021509in}}%
\pgfpathlineto{\pgfqpoint{1.030588in}{2.044946in}}%
\pgfpathlineto{\pgfqpoint{1.031463in}{1.862651in}}%
\pgfpathlineto{\pgfqpoint{1.032338in}{1.741766in}}%
\pgfpathlineto{\pgfqpoint{1.033214in}{1.413605in}}%
\pgfpathlineto{\pgfqpoint{1.034089in}{1.382674in}}%
\pgfpathlineto{\pgfqpoint{1.034964in}{1.314075in}}%
\pgfpathlineto{\pgfqpoint{1.035840in}{1.326128in}}%
\pgfpathlineto{\pgfqpoint{1.037591in}{1.178721in}}%
\pgfpathlineto{\pgfqpoint{1.038466in}{1.167001in}}%
\pgfpathlineto{\pgfqpoint{1.039341in}{1.115076in}}%
\pgfpathlineto{\pgfqpoint{1.040217in}{1.131297in}}%
\pgfpathlineto{\pgfqpoint{1.041092in}{1.138970in}}%
\pgfpathlineto{\pgfqpoint{1.041967in}{1.231522in}}%
\pgfpathlineto{\pgfqpoint{1.044594in}{1.869785in}}%
\pgfpathlineto{\pgfqpoint{1.046344in}{2.053998in}}%
\pgfpathlineto{\pgfqpoint{1.047220in}{2.030742in}}%
\pgfpathlineto{\pgfqpoint{1.048095in}{2.036268in}}%
\pgfpathlineto{\pgfqpoint{1.049846in}{2.237099in}}%
\pgfpathlineto{\pgfqpoint{1.050721in}{2.218435in}}%
\pgfpathlineto{\pgfqpoint{1.051597in}{2.155984in}}%
\pgfpathlineto{\pgfqpoint{1.052472in}{2.004591in}}%
\pgfpathlineto{\pgfqpoint{1.053347in}{1.960248in}}%
\pgfpathlineto{\pgfqpoint{1.054223in}{1.889354in}}%
\pgfpathlineto{\pgfqpoint{1.055098in}{1.842776in}}%
\pgfpathlineto{\pgfqpoint{1.055973in}{1.752066in}}%
\pgfpathlineto{\pgfqpoint{1.057724in}{1.430249in}}%
\pgfpathlineto{\pgfqpoint{1.058600in}{1.363583in}}%
\pgfpathlineto{\pgfqpoint{1.059475in}{1.367873in}}%
\pgfpathlineto{\pgfqpoint{1.060350in}{1.380197in}}%
\pgfpathlineto{\pgfqpoint{1.061226in}{1.283658in}}%
\pgfpathlineto{\pgfqpoint{1.062101in}{1.302446in}}%
\pgfpathlineto{\pgfqpoint{1.062976in}{1.184460in}}%
\pgfpathlineto{\pgfqpoint{1.063852in}{1.398472in}}%
\pgfpathlineto{\pgfqpoint{1.065603in}{2.037062in}}%
\pgfpathlineto{\pgfqpoint{1.066478in}{1.941026in}}%
\pgfpathlineto{\pgfqpoint{1.068229in}{2.036192in}}%
\pgfpathlineto{\pgfqpoint{1.070855in}{2.401842in}}%
\pgfpathlineto{\pgfqpoint{1.071730in}{2.382655in}}%
\pgfpathlineto{\pgfqpoint{1.073481in}{2.127198in}}%
\pgfpathlineto{\pgfqpoint{1.074356in}{2.096690in}}%
\pgfpathlineto{\pgfqpoint{1.075232in}{1.979972in}}%
\pgfpathlineto{\pgfqpoint{1.076107in}{1.983174in}}%
\pgfpathlineto{\pgfqpoint{1.076982in}{1.924121in}}%
\pgfpathlineto{\pgfqpoint{1.077858in}{1.758349in}}%
\pgfpathlineto{\pgfqpoint{1.079609in}{1.565331in}}%
\pgfpathlineto{\pgfqpoint{1.080484in}{1.569560in}}%
\pgfpathlineto{\pgfqpoint{1.081359in}{1.540683in}}%
\pgfpathlineto{\pgfqpoint{1.083110in}{1.425265in}}%
\pgfpathlineto{\pgfqpoint{1.083985in}{1.520837in}}%
\pgfpathlineto{\pgfqpoint{1.084861in}{1.659846in}}%
\pgfpathlineto{\pgfqpoint{1.086612in}{2.165905in}}%
\pgfpathlineto{\pgfqpoint{1.088362in}{2.366278in}}%
\pgfpathlineto{\pgfqpoint{1.089238in}{2.392085in}}%
\pgfpathlineto{\pgfqpoint{1.090113in}{2.397288in}}%
\pgfpathlineto{\pgfqpoint{1.090988in}{2.412930in}}%
\pgfpathlineto{\pgfqpoint{1.091864in}{2.447307in}}%
\pgfpathlineto{\pgfqpoint{1.092739in}{2.394622in}}%
\pgfpathlineto{\pgfqpoint{1.094490in}{2.137287in}}%
\pgfpathlineto{\pgfqpoint{1.096241in}{1.952575in}}%
\pgfpathlineto{\pgfqpoint{1.097116in}{1.913730in}}%
\pgfpathlineto{\pgfqpoint{1.098867in}{1.710533in}}%
\pgfpathlineto{\pgfqpoint{1.099742in}{1.522287in}}%
\pgfpathlineto{\pgfqpoint{1.100618in}{1.439522in}}%
\pgfpathlineto{\pgfqpoint{1.102368in}{1.326248in}}%
\pgfpathlineto{\pgfqpoint{1.103244in}{1.304047in}}%
\pgfpathlineto{\pgfqpoint{1.104119in}{1.324406in}}%
\pgfpathlineto{\pgfqpoint{1.104994in}{1.285107in}}%
\pgfpathlineto{\pgfqpoint{1.107621in}{2.063273in}}%
\pgfpathlineto{\pgfqpoint{1.108496in}{2.209091in}}%
\pgfpathlineto{\pgfqpoint{1.109371in}{2.188687in}}%
\pgfpathlineto{\pgfqpoint{1.110247in}{2.332839in}}%
\pgfpathlineto{\pgfqpoint{1.111122in}{2.312488in}}%
\pgfpathlineto{\pgfqpoint{1.111997in}{2.276186in}}%
\pgfpathlineto{\pgfqpoint{1.112873in}{2.284617in}}%
\pgfpathlineto{\pgfqpoint{1.113748in}{2.388046in}}%
\pgfpathlineto{\pgfqpoint{1.114624in}{2.307741in}}%
\pgfpathlineto{\pgfqpoint{1.119000in}{1.653050in}}%
\pgfpathlineto{\pgfqpoint{1.119876in}{1.548446in}}%
\pgfpathlineto{\pgfqpoint{1.120751in}{1.370501in}}%
\pgfpathlineto{\pgfqpoint{1.122502in}{1.203067in}}%
\pgfpathlineto{\pgfqpoint{1.123377in}{1.184581in}}%
\pgfpathlineto{\pgfqpoint{1.126003in}{1.381224in}}%
\pgfpathlineto{\pgfqpoint{1.129505in}{2.250066in}}%
\pgfpathlineto{\pgfqpoint{1.131256in}{2.386025in}}%
\pgfpathlineto{\pgfqpoint{1.132131in}{2.489330in}}%
\pgfpathlineto{\pgfqpoint{1.133007in}{2.527722in}}%
\pgfpathlineto{\pgfqpoint{1.133882in}{2.598767in}}%
\pgfpathlineto{\pgfqpoint{1.134757in}{2.576717in}}%
\pgfpathlineto{\pgfqpoint{1.135633in}{2.476190in}}%
\pgfpathlineto{\pgfqpoint{1.136508in}{2.297108in}}%
\pgfpathlineto{\pgfqpoint{1.138259in}{2.131245in}}%
\pgfpathlineto{\pgfqpoint{1.139134in}{2.102670in}}%
\pgfpathlineto{\pgfqpoint{1.140010in}{1.998398in}}%
\pgfpathlineto{\pgfqpoint{1.140885in}{1.772002in}}%
\pgfpathlineto{\pgfqpoint{1.142636in}{1.538296in}}%
\pgfpathlineto{\pgfqpoint{1.143511in}{1.493319in}}%
\pgfpathlineto{\pgfqpoint{1.144386in}{1.387084in}}%
\pgfpathlineto{\pgfqpoint{1.145262in}{1.391011in}}%
\pgfpathlineto{\pgfqpoint{1.146137in}{1.402761in}}%
\pgfpathlineto{\pgfqpoint{1.147013in}{1.514464in}}%
\pgfpathlineto{\pgfqpoint{1.148763in}{1.931793in}}%
\pgfpathlineto{\pgfqpoint{1.149639in}{2.160334in}}%
\pgfpathlineto{\pgfqpoint{1.151389in}{2.456964in}}%
\pgfpathlineto{\pgfqpoint{1.152265in}{2.461340in}}%
\pgfpathlineto{\pgfqpoint{1.154016in}{2.564399in}}%
\pgfpathlineto{\pgfqpoint{1.154891in}{2.587476in}}%
\pgfpathlineto{\pgfqpoint{1.155766in}{2.521299in}}%
\pgfpathlineto{\pgfqpoint{1.156642in}{2.405730in}}%
\pgfpathlineto{\pgfqpoint{1.157517in}{2.327043in}}%
\pgfpathlineto{\pgfqpoint{1.158392in}{2.151031in}}%
\pgfpathlineto{\pgfqpoint{1.159268in}{2.067087in}}%
\pgfpathlineto{\pgfqpoint{1.161894in}{1.705398in}}%
\pgfpathlineto{\pgfqpoint{1.162769in}{1.546513in}}%
\pgfpathlineto{\pgfqpoint{1.163645in}{1.490993in}}%
\pgfpathlineto{\pgfqpoint{1.164520in}{1.481932in}}%
\pgfpathlineto{\pgfqpoint{1.166271in}{1.371830in}}%
\pgfpathlineto{\pgfqpoint{1.167146in}{1.351501in}}%
\pgfpathlineto{\pgfqpoint{1.168897in}{1.582126in}}%
\pgfpathlineto{\pgfqpoint{1.170648in}{2.048673in}}%
\pgfpathlineto{\pgfqpoint{1.171523in}{2.101036in}}%
\pgfpathlineto{\pgfqpoint{1.172398in}{2.236910in}}%
\pgfpathlineto{\pgfqpoint{1.173274in}{2.300512in}}%
\pgfpathlineto{\pgfqpoint{1.175025in}{2.457204in}}%
\pgfpathlineto{\pgfqpoint{1.175900in}{2.498540in}}%
\pgfpathlineto{\pgfqpoint{1.176775in}{2.495110in}}%
\pgfpathlineto{\pgfqpoint{1.178526in}{2.380115in}}%
\pgfpathlineto{\pgfqpoint{1.179401in}{2.247751in}}%
\pgfpathlineto{\pgfqpoint{1.181152in}{2.125053in}}%
\pgfpathlineto{\pgfqpoint{1.184654in}{1.618132in}}%
\pgfpathlineto{\pgfqpoint{1.185529in}{1.532618in}}%
\pgfpathlineto{\pgfqpoint{1.186404in}{1.521260in}}%
\pgfpathlineto{\pgfqpoint{1.187280in}{1.480451in}}%
\pgfpathlineto{\pgfqpoint{1.189031in}{1.574755in}}%
\pgfpathlineto{\pgfqpoint{1.189906in}{1.789099in}}%
\pgfpathlineto{\pgfqpoint{1.190781in}{1.904215in}}%
\pgfpathlineto{\pgfqpoint{1.191657in}{2.331593in}}%
\pgfpathlineto{\pgfqpoint{1.192532in}{2.507770in}}%
\pgfpathlineto{\pgfqpoint{1.194283in}{2.543491in}}%
\pgfpathlineto{\pgfqpoint{1.195158in}{2.603805in}}%
\pgfpathlineto{\pgfqpoint{1.196034in}{2.834052in}}%
\pgfpathlineto{\pgfqpoint{1.197784in}{2.825567in}}%
\pgfpathlineto{\pgfqpoint{1.198660in}{2.763432in}}%
\pgfpathlineto{\pgfqpoint{1.200410in}{2.408479in}}%
\pgfpathlineto{\pgfqpoint{1.201286in}{2.286113in}}%
\pgfpathlineto{\pgfqpoint{1.203912in}{2.083791in}}%
\pgfpathlineto{\pgfqpoint{1.204787in}{1.892465in}}%
\pgfpathlineto{\pgfqpoint{1.206538in}{1.719202in}}%
\pgfpathlineto{\pgfqpoint{1.208289in}{1.624233in}}%
\pgfpathlineto{\pgfqpoint{1.209164in}{1.657732in}}%
\pgfpathlineto{\pgfqpoint{1.210040in}{1.679964in}}%
\pgfpathlineto{\pgfqpoint{1.210915in}{1.854798in}}%
\pgfpathlineto{\pgfqpoint{1.211790in}{2.279679in}}%
\pgfpathlineto{\pgfqpoint{1.213541in}{2.708003in}}%
\pgfpathlineto{\pgfqpoint{1.214416in}{2.695529in}}%
\pgfpathlineto{\pgfqpoint{1.215292in}{2.758293in}}%
\pgfpathlineto{\pgfqpoint{1.216167in}{2.777426in}}%
\pgfpathlineto{\pgfqpoint{1.217043in}{2.836386in}}%
\pgfpathlineto{\pgfqpoint{1.217918in}{2.809997in}}%
\pgfpathlineto{\pgfqpoint{1.218793in}{2.749882in}}%
\pgfpathlineto{\pgfqpoint{1.219669in}{2.661094in}}%
\pgfpathlineto{\pgfqpoint{1.221419in}{2.340847in}}%
\pgfpathlineto{\pgfqpoint{1.223170in}{2.113363in}}%
\pgfpathlineto{\pgfqpoint{1.224921in}{1.926870in}}%
\pgfpathlineto{\pgfqpoint{1.225796in}{1.733580in}}%
\pgfpathlineto{\pgfqpoint{1.226672in}{1.653715in}}%
\pgfpathlineto{\pgfqpoint{1.227547in}{1.601005in}}%
\pgfpathlineto{\pgfqpoint{1.228422in}{1.571221in}}%
\pgfpathlineto{\pgfqpoint{1.229298in}{1.507667in}}%
\pgfpathlineto{\pgfqpoint{1.230173in}{1.267467in}}%
\pgfpathlineto{\pgfqpoint{1.231924in}{1.770734in}}%
\pgfpathlineto{\pgfqpoint{1.234550in}{2.398739in}}%
\pgfpathlineto{\pgfqpoint{1.238052in}{2.664898in}}%
\pgfpathlineto{\pgfqpoint{1.238927in}{2.626800in}}%
\pgfpathlineto{\pgfqpoint{1.239802in}{2.568092in}}%
\pgfpathlineto{\pgfqpoint{1.240678in}{2.558882in}}%
\pgfpathlineto{\pgfqpoint{1.241553in}{2.493509in}}%
\pgfpathlineto{\pgfqpoint{1.242428in}{2.310822in}}%
\pgfpathlineto{\pgfqpoint{1.243304in}{2.256873in}}%
\pgfpathlineto{\pgfqpoint{1.244179in}{2.171178in}}%
\pgfpathlineto{\pgfqpoint{1.245055in}{2.133451in}}%
\pgfpathlineto{\pgfqpoint{1.245930in}{2.041261in}}%
\pgfpathlineto{\pgfqpoint{1.246805in}{1.869689in}}%
\pgfpathlineto{\pgfqpoint{1.247681in}{1.817523in}}%
\pgfpathlineto{\pgfqpoint{1.248556in}{1.725817in}}%
\pgfpathlineto{\pgfqpoint{1.249431in}{1.663411in}}%
\pgfpathlineto{\pgfqpoint{1.250307in}{1.644441in}}%
\pgfpathlineto{\pgfqpoint{1.251182in}{1.648338in}}%
\pgfpathlineto{\pgfqpoint{1.252058in}{1.770703in}}%
\pgfpathlineto{\pgfqpoint{1.254684in}{2.387132in}}%
\pgfpathlineto{\pgfqpoint{1.255559in}{2.459244in}}%
\pgfpathlineto{\pgfqpoint{1.256434in}{2.433984in}}%
\pgfpathlineto{\pgfqpoint{1.257310in}{2.546486in}}%
\pgfpathlineto{\pgfqpoint{1.258185in}{2.604012in}}%
\pgfpathlineto{\pgfqpoint{1.259061in}{2.815784in}}%
\pgfpathlineto{\pgfqpoint{1.259936in}{2.818991in}}%
\pgfpathlineto{\pgfqpoint{1.260811in}{2.862084in}}%
\pgfpathlineto{\pgfqpoint{1.261687in}{2.745611in}}%
\pgfpathlineto{\pgfqpoint{1.262562in}{2.532596in}}%
\pgfpathlineto{\pgfqpoint{1.263438in}{2.391533in}}%
\pgfpathlineto{\pgfqpoint{1.264313in}{2.388452in}}%
\pgfpathlineto{\pgfqpoint{1.265188in}{2.343414in}}%
\pgfpathlineto{\pgfqpoint{1.266064in}{2.217545in}}%
\pgfpathlineto{\pgfqpoint{1.267814in}{1.908897in}}%
\pgfpathlineto{\pgfqpoint{1.268690in}{1.874069in}}%
\pgfpathlineto{\pgfqpoint{1.269565in}{1.858755in}}%
\pgfpathlineto{\pgfqpoint{1.270441in}{1.775265in}}%
\pgfpathlineto{\pgfqpoint{1.271316in}{1.740286in}}%
\pgfpathlineto{\pgfqpoint{1.272191in}{1.777742in}}%
\pgfpathlineto{\pgfqpoint{1.273067in}{1.854888in}}%
\pgfpathlineto{\pgfqpoint{1.273942in}{2.004621in}}%
\pgfpathlineto{\pgfqpoint{1.275693in}{2.443479in}}%
\pgfpathlineto{\pgfqpoint{1.276568in}{2.507004in}}%
\pgfpathlineto{\pgfqpoint{1.277444in}{2.537042in}}%
\pgfpathlineto{\pgfqpoint{1.278319in}{2.588751in}}%
\pgfpathlineto{\pgfqpoint{1.279194in}{2.585810in}}%
\pgfpathlineto{\pgfqpoint{1.280070in}{2.783578in}}%
\pgfpathlineto{\pgfqpoint{1.280945in}{2.753740in}}%
\pgfpathlineto{\pgfqpoint{1.281820in}{2.677815in}}%
\pgfpathlineto{\pgfqpoint{1.282696in}{2.560960in}}%
\pgfpathlineto{\pgfqpoint{1.284447in}{2.261072in}}%
\pgfpathlineto{\pgfqpoint{1.285322in}{2.203318in}}%
\pgfpathlineto{\pgfqpoint{1.287073in}{2.039992in}}%
\pgfpathlineto{\pgfqpoint{1.287948in}{1.896543in}}%
\pgfpathlineto{\pgfqpoint{1.288823in}{1.636497in}}%
\pgfpathlineto{\pgfqpoint{1.289699in}{1.495283in}}%
\pgfpathlineto{\pgfqpoint{1.291450in}{1.403486in}}%
\pgfpathlineto{\pgfqpoint{1.292325in}{1.401794in}}%
\pgfpathlineto{\pgfqpoint{1.293200in}{1.373733in}}%
\pgfpathlineto{\pgfqpoint{1.294076in}{1.493108in}}%
\pgfpathlineto{\pgfqpoint{1.296702in}{2.034329in}}%
\pgfpathlineto{\pgfqpoint{1.297577in}{2.114814in}}%
\pgfpathlineto{\pgfqpoint{1.298453in}{2.070113in}}%
\pgfpathlineto{\pgfqpoint{1.299328in}{2.144057in}}%
\pgfpathlineto{\pgfqpoint{1.300203in}{2.192643in}}%
\pgfpathlineto{\pgfqpoint{1.301079in}{2.334897in}}%
\pgfpathlineto{\pgfqpoint{1.301954in}{2.376058in}}%
\pgfpathlineto{\pgfqpoint{1.302829in}{2.503671in}}%
\pgfpathlineto{\pgfqpoint{1.304580in}{2.285026in}}%
\pgfpathlineto{\pgfqpoint{1.305456in}{2.118952in}}%
\pgfpathlineto{\pgfqpoint{1.307206in}{1.928833in}}%
\pgfpathlineto{\pgfqpoint{1.308957in}{1.796711in}}%
\pgfpathlineto{\pgfqpoint{1.310708in}{1.520898in}}%
\pgfpathlineto{\pgfqpoint{1.312459in}{1.381345in}}%
\pgfpathlineto{\pgfqpoint{1.313334in}{1.388050in}}%
\pgfpathlineto{\pgfqpoint{1.314209in}{1.320842in}}%
\pgfpathlineto{\pgfqpoint{1.315960in}{1.526365in}}%
\pgfpathlineto{\pgfqpoint{1.316835in}{1.729330in}}%
\pgfpathlineto{\pgfqpoint{1.317711in}{1.999217in}}%
\pgfpathlineto{\pgfqpoint{1.319462in}{2.196725in}}%
\pgfpathlineto{\pgfqpoint{1.321212in}{2.240865in}}%
\pgfpathlineto{\pgfqpoint{1.322088in}{2.335717in}}%
\pgfpathlineto{\pgfqpoint{1.322963in}{2.492947in}}%
\pgfpathlineto{\pgfqpoint{1.323838in}{2.439360in}}%
\pgfpathlineto{\pgfqpoint{1.325589in}{2.285479in}}%
\pgfpathlineto{\pgfqpoint{1.327340in}{2.073582in}}%
\pgfpathlineto{\pgfqpoint{1.328215in}{1.997824in}}%
\pgfpathlineto{\pgfqpoint{1.329091in}{1.886001in}}%
\pgfpathlineto{\pgfqpoint{1.330841in}{1.572097in}}%
\pgfpathlineto{\pgfqpoint{1.333468in}{1.277073in}}%
\pgfpathlineto{\pgfqpoint{1.334343in}{1.265927in}}%
\pgfpathlineto{\pgfqpoint{1.335218in}{1.295710in}}%
\pgfpathlineto{\pgfqpoint{1.336969in}{1.600491in}}%
\pgfpathlineto{\pgfqpoint{1.337844in}{1.972545in}}%
\pgfpathlineto{\pgfqpoint{1.338720in}{2.184424in}}%
\pgfpathlineto{\pgfqpoint{1.339595in}{2.298645in}}%
\pgfpathlineto{\pgfqpoint{1.340471in}{2.353181in}}%
\pgfpathlineto{\pgfqpoint{1.341346in}{2.339428in}}%
\pgfpathlineto{\pgfqpoint{1.342221in}{2.445280in}}%
\pgfpathlineto{\pgfqpoint{1.343097in}{2.235501in}}%
\pgfpathlineto{\pgfqpoint{1.343972in}{2.525129in}}%
\pgfpathlineto{\pgfqpoint{1.344847in}{2.560900in}}%
\pgfpathlineto{\pgfqpoint{1.345723in}{2.558345in}}%
\pgfpathlineto{\pgfqpoint{1.346598in}{2.557577in}}%
\pgfpathlineto{\pgfqpoint{1.350100in}{2.168701in}}%
\pgfpathlineto{\pgfqpoint{1.350975in}{1.939828in}}%
\pgfpathlineto{\pgfqpoint{1.352726in}{1.692801in}}%
\pgfpathlineto{\pgfqpoint{1.353601in}{1.692560in}}%
\pgfpathlineto{\pgfqpoint{1.355352in}{1.559531in}}%
\pgfpathlineto{\pgfqpoint{1.356227in}{1.595779in}}%
\pgfpathlineto{\pgfqpoint{1.357103in}{1.694856in}}%
\pgfpathlineto{\pgfqpoint{1.357978in}{1.885064in}}%
\pgfpathlineto{\pgfqpoint{1.359729in}{2.423043in}}%
\pgfpathlineto{\pgfqpoint{1.360604in}{2.510258in}}%
\pgfpathlineto{\pgfqpoint{1.361480in}{2.502304in}}%
\pgfpathlineto{\pgfqpoint{1.362355in}{2.329403in}}%
\pgfpathlineto{\pgfqpoint{1.363230in}{2.447800in}}%
\pgfpathlineto{\pgfqpoint{1.364106in}{2.431330in}}%
\pgfpathlineto{\pgfqpoint{1.364981in}{2.473011in}}%
\pgfpathlineto{\pgfqpoint{1.365856in}{2.589654in}}%
\pgfpathlineto{\pgfqpoint{1.366732in}{2.491636in}}%
\pgfpathlineto{\pgfqpoint{1.367607in}{2.426693in}}%
\pgfpathlineto{\pgfqpoint{1.368483in}{2.266630in}}%
\pgfpathlineto{\pgfqpoint{1.370233in}{2.056153in}}%
\pgfpathlineto{\pgfqpoint{1.371109in}{2.083550in}}%
\pgfpathlineto{\pgfqpoint{1.373735in}{1.691835in}}%
\pgfpathlineto{\pgfqpoint{1.376361in}{1.527211in}}%
\pgfpathlineto{\pgfqpoint{1.377236in}{1.547721in}}%
\pgfpathlineto{\pgfqpoint{1.378112in}{1.578199in}}%
\pgfpathlineto{\pgfqpoint{1.378987in}{1.785595in}}%
\pgfpathlineto{\pgfqpoint{1.379862in}{2.142593in}}%
\pgfpathlineto{\pgfqpoint{1.380738in}{2.183843in}}%
\pgfpathlineto{\pgfqpoint{1.381613in}{2.210570in}}%
\pgfpathlineto{\pgfqpoint{1.383364in}{2.566630in}}%
\pgfpathlineto{\pgfqpoint{1.384239in}{2.562345in}}%
\pgfpathlineto{\pgfqpoint{1.385115in}{2.632943in}}%
\pgfpathlineto{\pgfqpoint{1.386865in}{2.587557in}}%
\pgfpathlineto{\pgfqpoint{1.388616in}{2.427629in}}%
\pgfpathlineto{\pgfqpoint{1.389492in}{2.241468in}}%
\pgfpathlineto{\pgfqpoint{1.390367in}{2.159307in}}%
\pgfpathlineto{\pgfqpoint{1.391242in}{2.036911in}}%
\pgfpathlineto{\pgfqpoint{1.392118in}{1.964628in}}%
\pgfpathlineto{\pgfqpoint{1.393869in}{1.637252in}}%
\pgfpathlineto{\pgfqpoint{1.394744in}{1.552101in}}%
\pgfpathlineto{\pgfqpoint{1.395619in}{1.511745in}}%
\pgfpathlineto{\pgfqpoint{1.397370in}{1.390618in}}%
\pgfpathlineto{\pgfqpoint{1.398245in}{1.385332in}}%
\pgfpathlineto{\pgfqpoint{1.399121in}{1.473142in}}%
\pgfpathlineto{\pgfqpoint{1.399996in}{1.632721in}}%
\pgfpathlineto{\pgfqpoint{1.401747in}{2.119699in}}%
\pgfpathlineto{\pgfqpoint{1.402622in}{2.061127in}}%
\pgfpathlineto{\pgfqpoint{1.403498in}{2.229131in}}%
\pgfpathlineto{\pgfqpoint{1.404373in}{2.338459in}}%
\pgfpathlineto{\pgfqpoint{1.405248in}{2.243211in}}%
\pgfpathlineto{\pgfqpoint{1.406124in}{2.302644in}}%
\pgfpathlineto{\pgfqpoint{1.406999in}{2.417308in}}%
\pgfpathlineto{\pgfqpoint{1.407875in}{2.411371in}}%
\pgfpathlineto{\pgfqpoint{1.408750in}{2.366204in}}%
\pgfpathlineto{\pgfqpoint{1.410501in}{2.126503in}}%
\pgfpathlineto{\pgfqpoint{1.411376in}{2.049658in}}%
\pgfpathlineto{\pgfqpoint{1.412251in}{1.944873in}}%
\pgfpathlineto{\pgfqpoint{1.414002in}{1.681353in}}%
\pgfpathlineto{\pgfqpoint{1.416628in}{1.498485in}}%
\pgfpathlineto{\pgfqpoint{1.417504in}{1.487731in}}%
\pgfpathlineto{\pgfqpoint{1.418379in}{1.455441in}}%
\pgfpathlineto{\pgfqpoint{1.419254in}{1.444144in}}%
\pgfpathlineto{\pgfqpoint{1.420130in}{1.476615in}}%
\pgfpathlineto{\pgfqpoint{1.422756in}{2.083608in}}%
\pgfpathlineto{\pgfqpoint{1.423631in}{2.190841in}}%
\pgfpathlineto{\pgfqpoint{1.424507in}{2.224664in}}%
\pgfpathlineto{\pgfqpoint{1.425382in}{2.270772in}}%
\pgfpathlineto{\pgfqpoint{1.426257in}{2.276648in}}%
\pgfpathlineto{\pgfqpoint{1.427133in}{2.322026in}}%
\pgfpathlineto{\pgfqpoint{1.428008in}{2.563352in}}%
\pgfpathlineto{\pgfqpoint{1.428884in}{2.592325in}}%
\pgfpathlineto{\pgfqpoint{1.430634in}{2.434063in}}%
\pgfpathlineto{\pgfqpoint{1.432385in}{2.198303in}}%
\pgfpathlineto{\pgfqpoint{1.433260in}{2.157011in}}%
\pgfpathlineto{\pgfqpoint{1.434136in}{2.049326in}}%
\pgfpathlineto{\pgfqpoint{1.435011in}{1.992780in}}%
\pgfpathlineto{\pgfqpoint{1.435887in}{1.665495in}}%
\pgfpathlineto{\pgfqpoint{1.436762in}{1.594118in}}%
\pgfpathlineto{\pgfqpoint{1.437637in}{1.577957in}}%
\pgfpathlineto{\pgfqpoint{1.438513in}{1.578018in}}%
\pgfpathlineto{\pgfqpoint{1.439388in}{1.515974in}}%
\pgfpathlineto{\pgfqpoint{1.440263in}{1.508332in}}%
\pgfpathlineto{\pgfqpoint{1.441139in}{1.489634in}}%
\pgfpathlineto{\pgfqpoint{1.442890in}{1.691973in}}%
\pgfpathlineto{\pgfqpoint{1.443765in}{1.609987in}}%
\pgfpathlineto{\pgfqpoint{1.444640in}{1.635109in}}%
\pgfpathlineto{\pgfqpoint{1.445516in}{1.716534in}}%
\pgfpathlineto{\pgfqpoint{1.446391in}{1.660605in}}%
\pgfpathlineto{\pgfqpoint{1.447266in}{1.634005in}}%
\pgfpathlineto{\pgfqpoint{1.448142in}{1.753848in}}%
\pgfpathlineto{\pgfqpoint{1.449017in}{1.778562in}}%
\pgfpathlineto{\pgfqpoint{1.449893in}{1.924935in}}%
\pgfpathlineto{\pgfqpoint{1.450768in}{1.981241in}}%
\pgfpathlineto{\pgfqpoint{1.451643in}{1.855976in}}%
\pgfpathlineto{\pgfqpoint{1.452519in}{1.848696in}}%
\pgfpathlineto{\pgfqpoint{1.454269in}{1.692801in}}%
\pgfpathlineto{\pgfqpoint{1.455145in}{1.714188in}}%
\pgfpathlineto{\pgfqpoint{1.456020in}{1.577867in}}%
\pgfpathlineto{\pgfqpoint{1.456896in}{1.541136in}}%
\pgfpathlineto{\pgfqpoint{1.457771in}{1.523375in}}%
\pgfpathlineto{\pgfqpoint{1.458646in}{1.560951in}}%
\pgfpathlineto{\pgfqpoint{1.459522in}{1.501445in}}%
\pgfpathlineto{\pgfqpoint{1.460397in}{1.465469in}}%
\pgfpathlineto{\pgfqpoint{1.461272in}{1.482505in}}%
\pgfpathlineto{\pgfqpoint{1.462148in}{1.516034in}}%
\pgfpathlineto{\pgfqpoint{1.463023in}{1.627888in}}%
\pgfpathlineto{\pgfqpoint{1.463899in}{1.697661in}}%
\pgfpathlineto{\pgfqpoint{1.464774in}{1.790114in}}%
\pgfpathlineto{\pgfqpoint{1.466525in}{2.065216in}}%
\pgfpathlineto{\pgfqpoint{1.468275in}{2.189421in}}%
\pgfpathlineto{\pgfqpoint{1.469151in}{2.701481in}}%
\pgfpathlineto{\pgfqpoint{1.470026in}{2.636690in}}%
\pgfpathlineto{\pgfqpoint{1.470902in}{2.667664in}}%
\pgfpathlineto{\pgfqpoint{1.471777in}{2.584932in}}%
\pgfpathlineto{\pgfqpoint{1.473528in}{2.355376in}}%
\pgfpathlineto{\pgfqpoint{1.476154in}{2.153538in}}%
\pgfpathlineto{\pgfqpoint{1.478780in}{1.811693in}}%
\pgfpathlineto{\pgfqpoint{1.479655in}{1.763363in}}%
\pgfpathlineto{\pgfqpoint{1.480531in}{1.683135in}}%
\pgfpathlineto{\pgfqpoint{1.481406in}{1.655829in}}%
\pgfpathlineto{\pgfqpoint{1.482281in}{1.676279in}}%
\pgfpathlineto{\pgfqpoint{1.483157in}{1.775748in}}%
\pgfpathlineto{\pgfqpoint{1.485783in}{2.281685in}}%
\pgfpathlineto{\pgfqpoint{1.486658in}{2.363910in}}%
\pgfpathlineto{\pgfqpoint{1.487534in}{2.393806in}}%
\pgfpathlineto{\pgfqpoint{1.488409in}{2.389369in}}%
\pgfpathlineto{\pgfqpoint{1.489284in}{2.467191in}}%
\pgfpathlineto{\pgfqpoint{1.490160in}{2.489987in}}%
\pgfpathlineto{\pgfqpoint{1.491035in}{2.607219in}}%
\pgfpathlineto{\pgfqpoint{1.491911in}{2.621698in}}%
\pgfpathlineto{\pgfqpoint{1.492786in}{2.554974in}}%
\pgfpathlineto{\pgfqpoint{1.493661in}{2.450918in}}%
\pgfpathlineto{\pgfqpoint{1.495412in}{2.201687in}}%
\pgfpathlineto{\pgfqpoint{1.496287in}{2.144385in}}%
\pgfpathlineto{\pgfqpoint{1.497163in}{2.049326in}}%
\pgfpathlineto{\pgfqpoint{1.498038in}{1.864554in}}%
\pgfpathlineto{\pgfqpoint{1.498914in}{1.827673in}}%
\pgfpathlineto{\pgfqpoint{1.500664in}{1.582337in}}%
\pgfpathlineto{\pgfqpoint{1.501540in}{1.576749in}}%
\pgfpathlineto{\pgfqpoint{1.502415in}{1.585690in}}%
\pgfpathlineto{\pgfqpoint{1.503290in}{1.583153in}}%
\pgfpathlineto{\pgfqpoint{1.504166in}{1.696034in}}%
\pgfpathlineto{\pgfqpoint{1.506792in}{2.540693in}}%
\pgfpathlineto{\pgfqpoint{1.507667in}{2.709123in}}%
\pgfpathlineto{\pgfqpoint{1.508543in}{2.711053in}}%
\pgfpathlineto{\pgfqpoint{1.509418in}{2.707889in}}%
\pgfpathlineto{\pgfqpoint{1.512044in}{2.843235in}}%
\pgfpathlineto{\pgfqpoint{1.512920in}{2.877969in}}%
\pgfpathlineto{\pgfqpoint{1.514670in}{2.542474in}}%
\pgfpathlineto{\pgfqpoint{1.515546in}{2.372473in}}%
\pgfpathlineto{\pgfqpoint{1.516421in}{2.325170in}}%
\pgfpathlineto{\pgfqpoint{1.517296in}{2.355014in}}%
\pgfpathlineto{\pgfqpoint{1.518172in}{2.254578in}}%
\pgfpathlineto{\pgfqpoint{1.519047in}{2.108742in}}%
\pgfpathlineto{\pgfqpoint{1.520798in}{1.922581in}}%
\pgfpathlineto{\pgfqpoint{1.521673in}{1.925692in}}%
\pgfpathlineto{\pgfqpoint{1.522549in}{1.903460in}}%
\pgfpathlineto{\pgfqpoint{1.523424in}{1.841447in}}%
\pgfpathlineto{\pgfqpoint{1.524299in}{1.813808in}}%
\pgfpathlineto{\pgfqpoint{1.525175in}{1.964416in}}%
\pgfpathlineto{\pgfqpoint{1.528676in}{2.814887in}}%
\pgfpathlineto{\pgfqpoint{1.529552in}{2.859939in}}%
\pgfpathlineto{\pgfqpoint{1.530427in}{2.881354in}}%
\pgfpathlineto{\pgfqpoint{1.532178in}{2.972975in}}%
\pgfpathlineto{\pgfqpoint{1.533053in}{2.962169in}}%
\pgfpathlineto{\pgfqpoint{1.533929in}{2.967402in}}%
\pgfpathlineto{\pgfqpoint{1.537430in}{2.368274in}}%
\pgfpathlineto{\pgfqpoint{1.538306in}{2.376762in}}%
\pgfpathlineto{\pgfqpoint{1.539181in}{2.267083in}}%
\pgfpathlineto{\pgfqpoint{1.540932in}{2.003956in}}%
\pgfpathlineto{\pgfqpoint{1.541807in}{1.907417in}}%
\pgfpathlineto{\pgfqpoint{1.542682in}{1.885669in}}%
\pgfpathlineto{\pgfqpoint{1.543558in}{1.824018in}}%
\pgfpathlineto{\pgfqpoint{1.544433in}{1.784266in}}%
\pgfpathlineto{\pgfqpoint{1.545309in}{1.791214in}}%
\pgfpathlineto{\pgfqpoint{1.546184in}{1.848636in}}%
\pgfpathlineto{\pgfqpoint{1.548810in}{2.471317in}}%
\pgfpathlineto{\pgfqpoint{1.549685in}{2.513088in}}%
\pgfpathlineto{\pgfqpoint{1.550561in}{2.696641in}}%
\pgfpathlineto{\pgfqpoint{1.551436in}{2.808739in}}%
\pgfpathlineto{\pgfqpoint{1.552312in}{2.870643in}}%
\pgfpathlineto{\pgfqpoint{1.553187in}{2.822904in}}%
\pgfpathlineto{\pgfqpoint{1.554062in}{2.839250in}}%
\pgfpathlineto{\pgfqpoint{1.555813in}{2.693602in}}%
\pgfpathlineto{\pgfqpoint{1.556688in}{2.582618in}}%
\pgfpathlineto{\pgfqpoint{1.557564in}{2.411983in}}%
\pgfpathlineto{\pgfqpoint{1.559315in}{2.195041in}}%
\pgfpathlineto{\pgfqpoint{1.560190in}{2.120764in}}%
\pgfpathlineto{\pgfqpoint{1.561065in}{1.956442in}}%
\pgfpathlineto{\pgfqpoint{1.563691in}{1.669815in}}%
\pgfpathlineto{\pgfqpoint{1.564567in}{1.616259in}}%
\pgfpathlineto{\pgfqpoint{1.565442in}{1.605113in}}%
\pgfpathlineto{\pgfqpoint{1.566318in}{1.604207in}}%
\pgfpathlineto{\pgfqpoint{1.567193in}{1.689449in}}%
\pgfpathlineto{\pgfqpoint{1.569819in}{2.434024in}}%
\pgfpathlineto{\pgfqpoint{1.570694in}{2.482933in}}%
\pgfpathlineto{\pgfqpoint{1.571570in}{2.550638in}}%
\pgfpathlineto{\pgfqpoint{1.572445in}{2.579700in}}%
\pgfpathlineto{\pgfqpoint{1.573321in}{2.535939in}}%
\pgfpathlineto{\pgfqpoint{1.574196in}{2.755471in}}%
\pgfpathlineto{\pgfqpoint{1.575071in}{2.802199in}}%
\pgfpathlineto{\pgfqpoint{1.575947in}{2.705074in}}%
\pgfpathlineto{\pgfqpoint{1.576822in}{2.709001in}}%
\pgfpathlineto{\pgfqpoint{1.577697in}{2.629981in}}%
\pgfpathlineto{\pgfqpoint{1.578573in}{2.469556in}}%
\pgfpathlineto{\pgfqpoint{1.579448in}{2.387485in}}%
\pgfpathlineto{\pgfqpoint{1.580324in}{2.335561in}}%
\pgfpathlineto{\pgfqpoint{1.581199in}{2.302515in}}%
\pgfpathlineto{\pgfqpoint{1.582074in}{2.232980in}}%
\pgfpathlineto{\pgfqpoint{1.582950in}{1.961577in}}%
\pgfpathlineto{\pgfqpoint{1.583825in}{1.948196in}}%
\pgfpathlineto{\pgfqpoint{1.584700in}{1.944118in}}%
\pgfpathlineto{\pgfqpoint{1.585576in}{1.887481in}}%
\pgfpathlineto{\pgfqpoint{1.586451in}{1.860658in}}%
\pgfpathlineto{\pgfqpoint{1.587327in}{2.002869in}}%
\pgfpathlineto{\pgfqpoint{1.588202in}{2.071014in}}%
\pgfpathlineto{\pgfqpoint{1.589077in}{2.188849in}}%
\pgfpathlineto{\pgfqpoint{1.589953in}{2.143705in}}%
\pgfpathlineto{\pgfqpoint{1.590828in}{2.374979in}}%
\pgfpathlineto{\pgfqpoint{1.591703in}{2.523482in}}%
\pgfpathlineto{\pgfqpoint{1.592579in}{2.599256in}}%
\pgfpathlineto{\pgfqpoint{1.593454in}{2.572890in}}%
\pgfpathlineto{\pgfqpoint{1.594330in}{2.504998in}}%
\pgfpathlineto{\pgfqpoint{1.595205in}{2.506143in}}%
\pgfpathlineto{\pgfqpoint{1.596080in}{2.582773in}}%
\pgfpathlineto{\pgfqpoint{1.596956in}{2.576181in}}%
\pgfpathlineto{\pgfqpoint{1.597831in}{2.515270in}}%
\pgfpathlineto{\pgfqpoint{1.598706in}{2.391201in}}%
\pgfpathlineto{\pgfqpoint{1.599582in}{2.221019in}}%
\pgfpathlineto{\pgfqpoint{1.600457in}{2.193863in}}%
\pgfpathlineto{\pgfqpoint{1.601333in}{2.142845in}}%
\pgfpathlineto{\pgfqpoint{1.602208in}{2.029843in}}%
\pgfpathlineto{\pgfqpoint{1.603083in}{1.880141in}}%
\pgfpathlineto{\pgfqpoint{1.607460in}{1.489755in}}%
\pgfpathlineto{\pgfqpoint{1.608336in}{1.473232in}}%
\pgfpathlineto{\pgfqpoint{1.609211in}{1.543764in}}%
\pgfpathlineto{\pgfqpoint{1.611837in}{2.096602in}}%
\pgfpathlineto{\pgfqpoint{1.613588in}{2.367093in}}%
\pgfpathlineto{\pgfqpoint{1.614463in}{2.355254in}}%
\pgfpathlineto{\pgfqpoint{1.615339in}{2.365940in}}%
\pgfpathlineto{\pgfqpoint{1.616214in}{2.413318in}}%
\pgfpathlineto{\pgfqpoint{1.617965in}{2.665881in}}%
\pgfpathlineto{\pgfqpoint{1.618840in}{2.640630in}}%
\pgfpathlineto{\pgfqpoint{1.619715in}{2.530844in}}%
\pgfpathlineto{\pgfqpoint{1.620591in}{2.140579in}}%
\pgfpathlineto{\pgfqpoint{1.621466in}{2.273306in}}%
\pgfpathlineto{\pgfqpoint{1.622342in}{2.170363in}}%
\pgfpathlineto{\pgfqpoint{1.624968in}{1.650483in}}%
\pgfpathlineto{\pgfqpoint{1.625843in}{1.626197in}}%
\pgfpathlineto{\pgfqpoint{1.626718in}{1.614024in}}%
\pgfpathlineto{\pgfqpoint{1.627594in}{1.567234in}}%
\pgfpathlineto{\pgfqpoint{1.628469in}{1.538991in}}%
\pgfpathlineto{\pgfqpoint{1.630220in}{1.658155in}}%
\pgfpathlineto{\pgfqpoint{1.631095in}{1.830572in}}%
\pgfpathlineto{\pgfqpoint{1.632846in}{2.286020in}}%
\pgfpathlineto{\pgfqpoint{1.633721in}{2.301611in}}%
\pgfpathlineto{\pgfqpoint{1.635472in}{2.443918in}}%
\pgfpathlineto{\pgfqpoint{1.636348in}{2.452909in}}%
\pgfpathlineto{\pgfqpoint{1.637223in}{2.568362in}}%
\pgfpathlineto{\pgfqpoint{1.638098in}{2.616677in}}%
\pgfpathlineto{\pgfqpoint{1.638974in}{2.739187in}}%
\pgfpathlineto{\pgfqpoint{1.640724in}{2.486139in}}%
\pgfpathlineto{\pgfqpoint{1.641600in}{2.414037in}}%
\pgfpathlineto{\pgfqpoint{1.642475in}{2.468227in}}%
\pgfpathlineto{\pgfqpoint{1.643351in}{2.448381in}}%
\pgfpathlineto{\pgfqpoint{1.644226in}{2.346042in}}%
\pgfpathlineto{\pgfqpoint{1.645977in}{1.998519in}}%
\pgfpathlineto{\pgfqpoint{1.646852in}{1.963087in}}%
\pgfpathlineto{\pgfqpoint{1.647727in}{1.824350in}}%
\pgfpathlineto{\pgfqpoint{1.648603in}{1.766293in}}%
\pgfpathlineto{\pgfqpoint{1.649478in}{1.752791in}}%
\pgfpathlineto{\pgfqpoint{1.650354in}{1.704280in}}%
\pgfpathlineto{\pgfqpoint{1.651229in}{1.801846in}}%
\pgfpathlineto{\pgfqpoint{1.652980in}{2.230885in}}%
\pgfpathlineto{\pgfqpoint{1.653855in}{2.348493in}}%
\pgfpathlineto{\pgfqpoint{1.654730in}{2.363511in}}%
\pgfpathlineto{\pgfqpoint{1.655606in}{2.553328in}}%
\pgfpathlineto{\pgfqpoint{1.656481in}{2.593796in}}%
\pgfpathlineto{\pgfqpoint{1.657357in}{2.464654in}}%
\pgfpathlineto{\pgfqpoint{1.658232in}{2.499636in}}%
\pgfpathlineto{\pgfqpoint{1.659107in}{2.601394in}}%
\pgfpathlineto{\pgfqpoint{1.659983in}{2.751042in}}%
\pgfpathlineto{\pgfqpoint{1.660858in}{2.675921in}}%
\pgfpathlineto{\pgfqpoint{1.662609in}{2.381082in}}%
\pgfpathlineto{\pgfqpoint{1.665235in}{2.026732in}}%
\pgfpathlineto{\pgfqpoint{1.666110in}{1.848787in}}%
\pgfpathlineto{\pgfqpoint{1.666986in}{1.732764in}}%
\pgfpathlineto{\pgfqpoint{1.667861in}{1.699477in}}%
\pgfpathlineto{\pgfqpoint{1.668737in}{1.528661in}}%
\pgfpathlineto{\pgfqpoint{1.669612in}{1.283204in}}%
\pgfpathlineto{\pgfqpoint{1.670487in}{1.218895in}}%
\pgfpathlineto{\pgfqpoint{1.671363in}{1.245930in}}%
\pgfpathlineto{\pgfqpoint{1.672238in}{1.414179in}}%
\pgfpathlineto{\pgfqpoint{1.673113in}{1.693708in}}%
\pgfpathlineto{\pgfqpoint{1.674864in}{2.096257in}}%
\pgfpathlineto{\pgfqpoint{1.675740in}{2.166599in}}%
\pgfpathlineto{\pgfqpoint{1.676615in}{2.121070in}}%
\pgfpathlineto{\pgfqpoint{1.677490in}{2.113532in}}%
\pgfpathlineto{\pgfqpoint{1.678366in}{2.164363in}}%
\pgfpathlineto{\pgfqpoint{1.679241in}{2.295529in}}%
\pgfpathlineto{\pgfqpoint{1.680116in}{2.335534in}}%
\pgfpathlineto{\pgfqpoint{1.680992in}{2.638846in}}%
\pgfpathlineto{\pgfqpoint{1.682743in}{2.408992in}}%
\pgfpathlineto{\pgfqpoint{1.683618in}{2.408932in}}%
\pgfpathlineto{\pgfqpoint{1.684493in}{2.286657in}}%
\pgfpathlineto{\pgfqpoint{1.685369in}{2.094031in}}%
\pgfpathlineto{\pgfqpoint{1.686244in}{2.128074in}}%
\pgfpathlineto{\pgfqpoint{1.688870in}{1.606291in}}%
\pgfpathlineto{\pgfqpoint{1.689746in}{1.506127in}}%
\pgfpathlineto{\pgfqpoint{1.690621in}{1.468973in}}%
\pgfpathlineto{\pgfqpoint{1.691496in}{1.448161in}}%
\pgfpathlineto{\pgfqpoint{1.694122in}{1.800396in}}%
\pgfpathlineto{\pgfqpoint{1.694998in}{1.907556in}}%
\pgfpathlineto{\pgfqpoint{1.695873in}{2.121987in}}%
\pgfpathlineto{\pgfqpoint{1.697624in}{2.313736in}}%
\pgfpathlineto{\pgfqpoint{1.698499in}{2.341638in}}%
\pgfpathlineto{\pgfqpoint{1.699375in}{2.381947in}}%
\pgfpathlineto{\pgfqpoint{1.700250in}{2.527780in}}%
\pgfpathlineto{\pgfqpoint{1.701125in}{2.207892in}}%
\pgfpathlineto{\pgfqpoint{1.702001in}{2.342311in}}%
\pgfpathlineto{\pgfqpoint{1.702876in}{2.352925in}}%
\pgfpathlineto{\pgfqpoint{1.703752in}{2.244459in}}%
\pgfpathlineto{\pgfqpoint{1.704627in}{2.012354in}}%
\pgfpathlineto{\pgfqpoint{1.705502in}{1.959493in}}%
\pgfpathlineto{\pgfqpoint{1.706378in}{1.850750in}}%
\pgfpathlineto{\pgfqpoint{1.707253in}{1.838577in}}%
\pgfpathlineto{\pgfqpoint{1.708128in}{1.636104in}}%
\pgfpathlineto{\pgfqpoint{1.709879in}{1.476525in}}%
\pgfpathlineto{\pgfqpoint{1.710755in}{1.404301in}}%
\pgfpathlineto{\pgfqpoint{1.711630in}{1.376240in}}%
\pgfpathlineto{\pgfqpoint{1.713381in}{1.344342in}}%
\pgfpathlineto{\pgfqpoint{1.714256in}{1.478307in}}%
\pgfpathlineto{\pgfqpoint{1.715131in}{1.655342in}}%
\pgfpathlineto{\pgfqpoint{1.716007in}{1.906307in}}%
\pgfpathlineto{\pgfqpoint{1.717758in}{2.244243in}}%
\pgfpathlineto{\pgfqpoint{1.718633in}{2.275571in}}%
\pgfpathlineto{\pgfqpoint{1.719508in}{2.317926in}}%
\pgfpathlineto{\pgfqpoint{1.720384in}{2.319911in}}%
\pgfpathlineto{\pgfqpoint{1.721259in}{2.253177in}}%
\pgfpathlineto{\pgfqpoint{1.722134in}{2.313147in}}%
\pgfpathlineto{\pgfqpoint{1.723010in}{2.259932in}}%
\pgfpathlineto{\pgfqpoint{1.723885in}{2.249666in}}%
\pgfpathlineto{\pgfqpoint{1.724761in}{2.134749in}}%
\pgfpathlineto{\pgfqpoint{1.725636in}{1.967165in}}%
\pgfpathlineto{\pgfqpoint{1.726511in}{1.858604in}}%
\pgfpathlineto{\pgfqpoint{1.728262in}{1.723521in}}%
\pgfpathlineto{\pgfqpoint{1.730013in}{1.487489in}}%
\pgfpathlineto{\pgfqpoint{1.732639in}{1.313622in}}%
\pgfpathlineto{\pgfqpoint{1.733514in}{1.285893in}}%
\pgfpathlineto{\pgfqpoint{1.734390in}{1.293535in}}%
\pgfpathlineto{\pgfqpoint{1.735265in}{1.358962in}}%
\pgfpathlineto{\pgfqpoint{1.737891in}{1.887858in}}%
\pgfpathlineto{\pgfqpoint{1.739642in}{1.983210in}}%
\pgfpathlineto{\pgfqpoint{1.740517in}{1.866654in}}%
\pgfpathlineto{\pgfqpoint{1.741393in}{1.933888in}}%
\pgfpathlineto{\pgfqpoint{1.742268in}{1.967501in}}%
\pgfpathlineto{\pgfqpoint{1.743143in}{2.062258in}}%
\pgfpathlineto{\pgfqpoint{1.744019in}{2.045103in}}%
\pgfpathlineto{\pgfqpoint{1.745770in}{1.872982in}}%
\pgfpathlineto{\pgfqpoint{1.746645in}{1.702075in}}%
\pgfpathlineto{\pgfqpoint{1.747520in}{1.600884in}}%
\pgfpathlineto{\pgfqpoint{1.749271in}{1.503348in}}%
\pgfpathlineto{\pgfqpoint{1.751022in}{1.234874in}}%
\pgfpathlineto{\pgfqpoint{1.751897in}{1.181512in}}%
\pgfpathlineto{\pgfqpoint{1.752773in}{1.110056in}}%
\pgfpathlineto{\pgfqpoint{1.754523in}{1.032903in}}%
\pgfpathlineto{\pgfqpoint{1.756274in}{1.133895in}}%
\pgfpathlineto{\pgfqpoint{1.758025in}{1.351508in}}%
\pgfpathlineto{\pgfqpoint{1.758900in}{1.562107in}}%
\pgfpathlineto{\pgfqpoint{1.759776in}{1.588510in}}%
\pgfpathlineto{\pgfqpoint{1.760651in}{1.703217in}}%
\pgfpathlineto{\pgfqpoint{1.761526in}{1.762834in}}%
\pgfpathlineto{\pgfqpoint{1.762402in}{1.761824in}}%
\pgfpathlineto{\pgfqpoint{1.763277in}{1.823516in}}%
\pgfpathlineto{\pgfqpoint{1.764152in}{1.998876in}}%
\pgfpathlineto{\pgfqpoint{1.765028in}{1.972674in}}%
\pgfpathlineto{\pgfqpoint{1.765903in}{1.990876in}}%
\pgfpathlineto{\pgfqpoint{1.766779in}{1.957620in}}%
\pgfpathlineto{\pgfqpoint{1.768529in}{1.830754in}}%
\pgfpathlineto{\pgfqpoint{1.769405in}{1.805894in}}%
\pgfpathlineto{\pgfqpoint{1.771155in}{1.622391in}}%
\pgfpathlineto{\pgfqpoint{1.772906in}{1.474773in}}%
\pgfpathlineto{\pgfqpoint{1.773782in}{1.406446in}}%
\pgfpathlineto{\pgfqpoint{1.774657in}{1.359898in}}%
\pgfpathlineto{\pgfqpoint{1.775532in}{1.337999in}}%
\pgfpathlineto{\pgfqpoint{1.776408in}{1.269581in}}%
\pgfpathlineto{\pgfqpoint{1.777283in}{1.367933in}}%
\pgfpathlineto{\pgfqpoint{1.779034in}{1.765511in}}%
\pgfpathlineto{\pgfqpoint{1.779909in}{2.006997in}}%
\pgfpathlineto{\pgfqpoint{1.780785in}{2.328871in}}%
\pgfpathlineto{\pgfqpoint{1.781660in}{2.369992in}}%
\pgfpathlineto{\pgfqpoint{1.782535in}{2.266769in}}%
\pgfpathlineto{\pgfqpoint{1.784286in}{2.488482in}}%
\pgfpathlineto{\pgfqpoint{1.785161in}{2.516358in}}%
\pgfpathlineto{\pgfqpoint{1.786037in}{2.714817in}}%
\pgfpathlineto{\pgfqpoint{1.787788in}{2.469374in}}%
\pgfpathlineto{\pgfqpoint{1.788663in}{2.405307in}}%
\pgfpathlineto{\pgfqpoint{1.789538in}{2.279679in}}%
\pgfpathlineto{\pgfqpoint{1.790414in}{2.293181in}}%
\pgfpathlineto{\pgfqpoint{1.791289in}{2.329097in}}%
\pgfpathlineto{\pgfqpoint{1.792164in}{2.316501in}}%
\pgfpathlineto{\pgfqpoint{1.793040in}{2.335772in}}%
\pgfpathlineto{\pgfqpoint{1.793915in}{2.156709in}}%
\pgfpathlineto{\pgfqpoint{1.794791in}{2.079925in}}%
\pgfpathlineto{\pgfqpoint{1.795666in}{2.065909in}}%
\pgfpathlineto{\pgfqpoint{1.796541in}{2.023530in}}%
\pgfpathlineto{\pgfqpoint{1.797417in}{2.008880in}}%
\pgfpathlineto{\pgfqpoint{1.798292in}{2.101190in}}%
\pgfpathlineto{\pgfqpoint{1.800918in}{2.965234in}}%
\pgfpathlineto{\pgfqpoint{1.801794in}{3.041317in}}%
\pgfpathlineto{\pgfqpoint{1.802669in}{3.169545in}}%
\pgfpathlineto{\pgfqpoint{1.803544in}{3.151692in}}%
\pgfpathlineto{\pgfqpoint{1.805295in}{3.270324in}}%
\pgfpathlineto{\pgfqpoint{1.806171in}{3.331307in}}%
\pgfpathlineto{\pgfqpoint{1.807046in}{3.292556in}}%
\pgfpathlineto{\pgfqpoint{1.808797in}{2.900629in}}%
\pgfpathlineto{\pgfqpoint{1.809672in}{2.548213in}}%
\pgfpathlineto{\pgfqpoint{1.810547in}{2.444636in}}%
\pgfpathlineto{\pgfqpoint{1.811423in}{2.484689in}}%
\pgfpathlineto{\pgfqpoint{1.812298in}{2.456023in}}%
\pgfpathlineto{\pgfqpoint{1.813174in}{2.265633in}}%
\pgfpathlineto{\pgfqpoint{1.816675in}{1.836372in}}%
\pgfpathlineto{\pgfqpoint{1.817550in}{1.773029in}}%
\pgfpathlineto{\pgfqpoint{1.818426in}{1.735725in}}%
\pgfpathlineto{\pgfqpoint{1.820177in}{1.993841in}}%
\pgfpathlineto{\pgfqpoint{1.822803in}{2.552883in}}%
\pgfpathlineto{\pgfqpoint{1.823678in}{2.564879in}}%
\pgfpathlineto{\pgfqpoint{1.824553in}{2.605270in}}%
\pgfpathlineto{\pgfqpoint{1.825429in}{2.623395in}}%
\pgfpathlineto{\pgfqpoint{1.826304in}{2.760541in}}%
\pgfpathlineto{\pgfqpoint{1.827180in}{2.770538in}}%
\pgfpathlineto{\pgfqpoint{1.828055in}{2.751455in}}%
\pgfpathlineto{\pgfqpoint{1.829806in}{2.522809in}}%
\pgfpathlineto{\pgfqpoint{1.832432in}{2.222378in}}%
\pgfpathlineto{\pgfqpoint{1.834183in}{1.932579in}}%
\pgfpathlineto{\pgfqpoint{1.835058in}{1.747837in}}%
\pgfpathlineto{\pgfqpoint{1.835933in}{1.680840in}}%
\pgfpathlineto{\pgfqpoint{1.836809in}{1.536242in}}%
\pgfpathlineto{\pgfqpoint{1.837684in}{1.506217in}}%
\pgfpathlineto{\pgfqpoint{1.838559in}{1.428044in}}%
\pgfpathlineto{\pgfqpoint{1.839435in}{1.432242in}}%
\pgfpathlineto{\pgfqpoint{1.840310in}{1.473987in}}%
\pgfpathlineto{\pgfqpoint{1.842936in}{2.126851in}}%
\pgfpathlineto{\pgfqpoint{1.844687in}{2.388733in}}%
\pgfpathlineto{\pgfqpoint{1.845562in}{2.136883in}}%
\pgfpathlineto{\pgfqpoint{1.847313in}{2.424413in}}%
\pgfpathlineto{\pgfqpoint{1.848189in}{2.580195in}}%
\pgfpathlineto{\pgfqpoint{1.849064in}{2.604482in}}%
\pgfpathlineto{\pgfqpoint{1.849939in}{2.545912in}}%
\pgfpathlineto{\pgfqpoint{1.851690in}{2.178881in}}%
\pgfpathlineto{\pgfqpoint{1.852565in}{2.026822in}}%
\pgfpathlineto{\pgfqpoint{1.853441in}{1.805894in}}%
\pgfpathlineto{\pgfqpoint{1.854316in}{1.709506in}}%
\pgfpathlineto{\pgfqpoint{1.856067in}{1.456981in}}%
\pgfpathlineto{\pgfqpoint{1.857818in}{1.226507in}}%
\pgfpathlineto{\pgfqpoint{1.858693in}{1.147548in}}%
\pgfpathlineto{\pgfqpoint{1.859568in}{1.299848in}}%
\pgfpathlineto{\pgfqpoint{1.860444in}{1.272663in}}%
\pgfpathlineto{\pgfqpoint{1.861319in}{1.359475in}}%
\pgfpathlineto{\pgfqpoint{1.863070in}{1.878717in}}%
\pgfpathlineto{\pgfqpoint{1.863945in}{2.218599in}}%
\pgfpathlineto{\pgfqpoint{1.866571in}{2.713679in}}%
\pgfpathlineto{\pgfqpoint{1.867447in}{2.677767in}}%
\pgfpathlineto{\pgfqpoint{1.868322in}{2.708800in}}%
\pgfpathlineto{\pgfqpoint{1.869198in}{2.761455in}}%
\pgfpathlineto{\pgfqpoint{1.870073in}{2.743284in}}%
\pgfpathlineto{\pgfqpoint{1.873574in}{2.046185in}}%
\pgfpathlineto{\pgfqpoint{1.874450in}{1.918835in}}%
\pgfpathlineto{\pgfqpoint{1.875325in}{1.887964in}}%
\pgfpathlineto{\pgfqpoint{1.877076in}{1.487701in}}%
\pgfpathlineto{\pgfqpoint{1.877951in}{1.378143in}}%
\pgfpathlineto{\pgfqpoint{1.878827in}{1.368869in}}%
\pgfpathlineto{\pgfqpoint{1.879702in}{1.284896in}}%
\pgfpathlineto{\pgfqpoint{1.880577in}{1.394726in}}%
\pgfpathlineto{\pgfqpoint{1.881453in}{1.330538in}}%
\pgfpathlineto{\pgfqpoint{1.882328in}{1.417109in}}%
\pgfpathlineto{\pgfqpoint{1.884079in}{1.870006in}}%
\pgfpathlineto{\pgfqpoint{1.884954in}{2.051716in}}%
\pgfpathlineto{\pgfqpoint{1.886705in}{2.230901in}}%
\pgfpathlineto{\pgfqpoint{1.887580in}{2.249852in}}%
\pgfpathlineto{\pgfqpoint{1.888456in}{2.237323in}}%
\pgfpathlineto{\pgfqpoint{1.889331in}{2.268725in}}%
\pgfpathlineto{\pgfqpoint{1.890207in}{2.459201in}}%
\pgfpathlineto{\pgfqpoint{1.891082in}{2.491905in}}%
\pgfpathlineto{\pgfqpoint{1.893708in}{2.126352in}}%
\pgfpathlineto{\pgfqpoint{1.894583in}{2.072162in}}%
\pgfpathlineto{\pgfqpoint{1.895459in}{1.869810in}}%
\pgfpathlineto{\pgfqpoint{1.896334in}{1.738806in}}%
\pgfpathlineto{\pgfqpoint{1.897210in}{1.508241in}}%
\pgfpathlineto{\pgfqpoint{1.899836in}{1.173979in}}%
\pgfpathlineto{\pgfqpoint{1.900711in}{1.101514in}}%
\pgfpathlineto{\pgfqpoint{1.901586in}{1.102994in}}%
\pgfpathlineto{\pgfqpoint{1.902462in}{1.060282in}}%
\pgfpathlineto{\pgfqpoint{1.903337in}{1.131690in}}%
\pgfpathlineto{\pgfqpoint{1.905963in}{1.692214in}}%
\pgfpathlineto{\pgfqpoint{1.906839in}{1.753525in}}%
\pgfpathlineto{\pgfqpoint{1.907714in}{1.934686in}}%
\pgfpathlineto{\pgfqpoint{1.908589in}{1.784105in}}%
\pgfpathlineto{\pgfqpoint{1.909465in}{1.768079in}}%
\pgfpathlineto{\pgfqpoint{1.910340in}{1.885379in}}%
\pgfpathlineto{\pgfqpoint{1.911216in}{2.050873in}}%
\pgfpathlineto{\pgfqpoint{1.912966in}{2.128826in}}%
\pgfpathlineto{\pgfqpoint{1.914717in}{1.866034in}}%
\pgfpathlineto{\pgfqpoint{1.915592in}{1.723914in}}%
\pgfpathlineto{\pgfqpoint{1.916468in}{1.714157in}}%
\pgfpathlineto{\pgfqpoint{1.917343in}{1.725696in}}%
\pgfpathlineto{\pgfqpoint{1.918219in}{1.489453in}}%
\pgfpathlineto{\pgfqpoint{1.919969in}{1.237774in}}%
\pgfpathlineto{\pgfqpoint{1.920845in}{1.151264in}}%
\pgfpathlineto{\pgfqpoint{1.921720in}{1.128035in}}%
\pgfpathlineto{\pgfqpoint{1.922595in}{1.094355in}}%
\pgfpathlineto{\pgfqpoint{1.923471in}{1.016272in}}%
\pgfpathlineto{\pgfqpoint{1.924346in}{1.250793in}}%
\pgfpathlineto{\pgfqpoint{1.925222in}{1.408538in}}%
\pgfpathlineto{\pgfqpoint{1.926097in}{1.431085in}}%
\pgfpathlineto{\pgfqpoint{1.926972in}{1.606103in}}%
\pgfpathlineto{\pgfqpoint{1.927848in}{1.668917in}}%
\pgfpathlineto{\pgfqpoint{1.928723in}{1.689679in}}%
\pgfpathlineto{\pgfqpoint{1.929599in}{1.667428in}}%
\pgfpathlineto{\pgfqpoint{1.930474in}{1.784291in}}%
\pgfpathlineto{\pgfqpoint{1.933100in}{2.205726in}}%
\pgfpathlineto{\pgfqpoint{1.933975in}{2.143595in}}%
\pgfpathlineto{\pgfqpoint{1.935726in}{1.770311in}}%
\pgfpathlineto{\pgfqpoint{1.936602in}{1.690325in}}%
\pgfpathlineto{\pgfqpoint{1.937477in}{1.744001in}}%
\pgfpathlineto{\pgfqpoint{1.939228in}{1.390860in}}%
\pgfpathlineto{\pgfqpoint{1.940978in}{1.139815in}}%
\pgfpathlineto{\pgfqpoint{1.941854in}{1.118731in}}%
\pgfpathlineto{\pgfqpoint{1.942729in}{1.026844in}}%
\pgfpathlineto{\pgfqpoint{1.943605in}{1.010442in}}%
\pgfpathlineto{\pgfqpoint{1.944480in}{1.071217in}}%
\pgfpathlineto{\pgfqpoint{1.945355in}{1.210347in}}%
\pgfpathlineto{\pgfqpoint{1.947981in}{1.895076in}}%
\pgfpathlineto{\pgfqpoint{1.949732in}{2.196211in}}%
\pgfpathlineto{\pgfqpoint{1.950608in}{2.214983in}}%
\pgfpathlineto{\pgfqpoint{1.951483in}{2.339220in}}%
\pgfpathlineto{\pgfqpoint{1.952358in}{2.624415in}}%
\pgfpathlineto{\pgfqpoint{1.953234in}{2.791791in}}%
\pgfpathlineto{\pgfqpoint{1.954109in}{2.750552in}}%
\pgfpathlineto{\pgfqpoint{1.954984in}{2.660083in}}%
\pgfpathlineto{\pgfqpoint{1.956735in}{2.260770in}}%
\pgfpathlineto{\pgfqpoint{1.958486in}{2.126141in}}%
\pgfpathlineto{\pgfqpoint{1.959361in}{2.010874in}}%
\pgfpathlineto{\pgfqpoint{1.960237in}{1.768740in}}%
\pgfpathlineto{\pgfqpoint{1.961987in}{1.517061in}}%
\pgfpathlineto{\pgfqpoint{1.962863in}{1.445926in}}%
\pgfpathlineto{\pgfqpoint{1.963738in}{1.306856in}}%
\pgfpathlineto{\pgfqpoint{1.964614in}{1.300150in}}%
\pgfpathlineto{\pgfqpoint{1.965489in}{1.247259in}}%
\pgfpathlineto{\pgfqpoint{1.966364in}{1.345822in}}%
\pgfpathlineto{\pgfqpoint{1.968115in}{1.759818in}}%
\pgfpathlineto{\pgfqpoint{1.968990in}{2.062718in}}%
\pgfpathlineto{\pgfqpoint{1.969866in}{2.144706in}}%
\pgfpathlineto{\pgfqpoint{1.970741in}{2.337383in}}%
\pgfpathlineto{\pgfqpoint{1.972492in}{2.441346in}}%
\pgfpathlineto{\pgfqpoint{1.973367in}{2.614123in}}%
\pgfpathlineto{\pgfqpoint{1.974243in}{2.607319in}}%
\pgfpathlineto{\pgfqpoint{1.975118in}{2.701204in}}%
\pgfpathlineto{\pgfqpoint{1.976869in}{2.542927in}}%
\pgfpathlineto{\pgfqpoint{1.977744in}{2.384706in}}%
\pgfpathlineto{\pgfqpoint{1.978620in}{2.335198in}}%
\pgfpathlineto{\pgfqpoint{1.982121in}{1.803659in}}%
\pgfpathlineto{\pgfqpoint{1.982996in}{1.585146in}}%
\pgfpathlineto{\pgfqpoint{1.983872in}{1.438132in}}%
\pgfpathlineto{\pgfqpoint{1.984747in}{1.399408in}}%
\pgfpathlineto{\pgfqpoint{1.985623in}{1.308336in}}%
\pgfpathlineto{\pgfqpoint{1.986498in}{1.298912in}}%
\pgfpathlineto{\pgfqpoint{1.987373in}{1.297250in}}%
\pgfpathlineto{\pgfqpoint{1.988249in}{1.546494in}}%
\pgfpathlineto{\pgfqpoint{1.989124in}{1.673741in}}%
\pgfpathlineto{\pgfqpoint{1.989999in}{1.994155in}}%
\pgfpathlineto{\pgfqpoint{1.990875in}{1.998858in}}%
\pgfpathlineto{\pgfqpoint{1.992626in}{2.268683in}}%
\pgfpathlineto{\pgfqpoint{1.993501in}{2.239008in}}%
\pgfpathlineto{\pgfqpoint{1.994376in}{2.305051in}}%
\pgfpathlineto{\pgfqpoint{1.995252in}{2.419853in}}%
\pgfpathlineto{\pgfqpoint{1.996127in}{2.423574in}}%
\pgfpathlineto{\pgfqpoint{1.997002in}{2.403878in}}%
\pgfpathlineto{\pgfqpoint{1.997878in}{2.198364in}}%
\pgfpathlineto{\pgfqpoint{1.998753in}{1.604297in}}%
\pgfpathlineto{\pgfqpoint{1.999629in}{1.986437in}}%
\pgfpathlineto{\pgfqpoint{2.000504in}{1.963510in}}%
\pgfpathlineto{\pgfqpoint{2.001379in}{1.958224in}}%
\pgfpathlineto{\pgfqpoint{2.002255in}{1.713704in}}%
\pgfpathlineto{\pgfqpoint{2.004881in}{1.371618in}}%
\pgfpathlineto{\pgfqpoint{2.005756in}{1.349840in}}%
\pgfpathlineto{\pgfqpoint{2.006632in}{1.249343in}}%
\pgfpathlineto{\pgfqpoint{2.007507in}{1.260791in}}%
\pgfpathlineto{\pgfqpoint{2.008382in}{1.383157in}}%
\pgfpathlineto{\pgfqpoint{2.011008in}{1.952304in}}%
\pgfpathlineto{\pgfqpoint{2.011884in}{2.063484in}}%
\pgfpathlineto{\pgfqpoint{2.012759in}{2.122664in}}%
\pgfpathlineto{\pgfqpoint{2.013635in}{2.127828in}}%
\pgfpathlineto{\pgfqpoint{2.014510in}{2.130007in}}%
\pgfpathlineto{\pgfqpoint{2.015385in}{2.244006in}}%
\pgfpathlineto{\pgfqpoint{2.016261in}{2.270873in}}%
\pgfpathlineto{\pgfqpoint{2.017136in}{2.228701in}}%
\pgfpathlineto{\pgfqpoint{2.018011in}{2.158907in}}%
\pgfpathlineto{\pgfqpoint{2.021513in}{1.654862in}}%
\pgfpathlineto{\pgfqpoint{2.022388in}{1.612423in}}%
\pgfpathlineto{\pgfqpoint{2.024139in}{1.263570in}}%
\pgfpathlineto{\pgfqpoint{2.026765in}{0.959816in}}%
\pgfpathlineto{\pgfqpoint{2.027641in}{0.953775in}}%
\pgfpathlineto{\pgfqpoint{2.028516in}{0.963532in}}%
\pgfpathlineto{\pgfqpoint{2.029391in}{1.063696in}}%
\pgfpathlineto{\pgfqpoint{2.032017in}{1.626190in}}%
\pgfpathlineto{\pgfqpoint{2.032893in}{1.760807in}}%
\pgfpathlineto{\pgfqpoint{2.033768in}{1.809115in}}%
\pgfpathlineto{\pgfqpoint{2.034644in}{1.829268in}}%
\pgfpathlineto{\pgfqpoint{2.036394in}{1.931986in}}%
\pgfpathlineto{\pgfqpoint{2.037270in}{2.039513in}}%
\pgfpathlineto{\pgfqpoint{2.038145in}{2.083621in}}%
\pgfpathlineto{\pgfqpoint{2.039020in}{2.008113in}}%
\pgfpathlineto{\pgfqpoint{2.039896in}{2.035129in}}%
\pgfpathlineto{\pgfqpoint{2.041647in}{1.845736in}}%
\pgfpathlineto{\pgfqpoint{2.042522in}{1.790549in}}%
\pgfpathlineto{\pgfqpoint{2.043397in}{1.673772in}}%
\pgfpathlineto{\pgfqpoint{2.046023in}{1.175882in}}%
\pgfpathlineto{\pgfqpoint{2.046899in}{1.136553in}}%
\pgfpathlineto{\pgfqpoint{2.047774in}{1.126139in}}%
\pgfpathlineto{\pgfqpoint{2.048650in}{1.110983in}}%
\pgfpathlineto{\pgfqpoint{2.049525in}{1.175602in}}%
\pgfpathlineto{\pgfqpoint{2.050400in}{1.172378in}}%
\pgfpathlineto{\pgfqpoint{2.053026in}{1.678059in}}%
\pgfpathlineto{\pgfqpoint{2.053902in}{1.739525in}}%
\pgfpathlineto{\pgfqpoint{2.054777in}{1.876546in}}%
\pgfpathlineto{\pgfqpoint{2.055653in}{1.908191in}}%
\pgfpathlineto{\pgfqpoint{2.059154in}{2.301608in}}%
\pgfpathlineto{\pgfqpoint{2.060030in}{2.168274in}}%
\pgfpathlineto{\pgfqpoint{2.060905in}{2.105993in}}%
\pgfpathlineto{\pgfqpoint{2.061780in}{1.955868in}}%
\pgfpathlineto{\pgfqpoint{2.062656in}{1.867454in}}%
\pgfpathlineto{\pgfqpoint{2.063531in}{1.727146in}}%
\pgfpathlineto{\pgfqpoint{2.064406in}{1.643626in}}%
\pgfpathlineto{\pgfqpoint{2.067033in}{1.195183in}}%
\pgfpathlineto{\pgfqpoint{2.068783in}{1.101846in}}%
\pgfpathlineto{\pgfqpoint{2.069659in}{1.085082in}}%
\pgfpathlineto{\pgfqpoint{2.070534in}{1.097073in}}%
\pgfpathlineto{\pgfqpoint{2.071409in}{1.197358in}}%
\pgfpathlineto{\pgfqpoint{2.072285in}{1.379599in}}%
\pgfpathlineto{\pgfqpoint{2.074036in}{1.938147in}}%
\pgfpathlineto{\pgfqpoint{2.074911in}{2.060648in}}%
\pgfpathlineto{\pgfqpoint{2.075786in}{2.248779in}}%
\pgfpathlineto{\pgfqpoint{2.077537in}{2.363967in}}%
\pgfpathlineto{\pgfqpoint{2.078412in}{2.349481in}}%
\pgfpathlineto{\pgfqpoint{2.079288in}{2.527527in}}%
\pgfpathlineto{\pgfqpoint{2.080163in}{2.500592in}}%
\pgfpathlineto{\pgfqpoint{2.081914in}{2.186553in}}%
\pgfpathlineto{\pgfqpoint{2.082789in}{2.030508in}}%
\pgfpathlineto{\pgfqpoint{2.085415in}{1.748049in}}%
\pgfpathlineto{\pgfqpoint{2.086291in}{1.523858in}}%
\pgfpathlineto{\pgfqpoint{2.087166in}{1.415599in}}%
\pgfpathlineto{\pgfqpoint{2.088917in}{1.145434in}}%
\pgfpathlineto{\pgfqpoint{2.090668in}{1.061611in}}%
\pgfpathlineto{\pgfqpoint{2.091543in}{1.074268in}}%
\pgfpathlineto{\pgfqpoint{2.092418in}{1.196694in}}%
\pgfpathlineto{\pgfqpoint{2.097671in}{2.242800in}}%
\pgfpathlineto{\pgfqpoint{2.098546in}{2.213017in}}%
\pgfpathlineto{\pgfqpoint{2.099421in}{2.447048in}}%
\pgfpathlineto{\pgfqpoint{2.100297in}{2.462668in}}%
\pgfpathlineto{\pgfqpoint{2.101172in}{2.547274in}}%
\pgfpathlineto{\pgfqpoint{2.102048in}{2.466387in}}%
\pgfpathlineto{\pgfqpoint{2.103798in}{2.146983in}}%
\pgfpathlineto{\pgfqpoint{2.104674in}{2.101371in}}%
\pgfpathlineto{\pgfqpoint{2.105549in}{2.089319in}}%
\pgfpathlineto{\pgfqpoint{2.106424in}{2.043889in}}%
\pgfpathlineto{\pgfqpoint{2.108175in}{1.688814in}}%
\pgfpathlineto{\pgfqpoint{2.109926in}{1.452299in}}%
\pgfpathlineto{\pgfqpoint{2.110801in}{1.316884in}}%
\pgfpathlineto{\pgfqpoint{2.112552in}{1.202644in}}%
\pgfpathlineto{\pgfqpoint{2.113427in}{1.249464in}}%
\pgfpathlineto{\pgfqpoint{2.114303in}{1.460444in}}%
\pgfpathlineto{\pgfqpoint{2.115178in}{1.600673in}}%
\pgfpathlineto{\pgfqpoint{2.116054in}{1.677601in}}%
\pgfpathlineto{\pgfqpoint{2.116929in}{1.846310in}}%
\pgfpathlineto{\pgfqpoint{2.117804in}{1.869057in}}%
\pgfpathlineto{\pgfqpoint{2.119555in}{2.036213in}}%
\pgfpathlineto{\pgfqpoint{2.120430in}{2.265806in}}%
\pgfpathlineto{\pgfqpoint{2.122181in}{2.453402in}}%
\pgfpathlineto{\pgfqpoint{2.123932in}{2.238025in}}%
\pgfpathlineto{\pgfqpoint{2.124807in}{2.184227in}}%
\pgfpathlineto{\pgfqpoint{2.125683in}{1.993444in}}%
\pgfpathlineto{\pgfqpoint{2.126558in}{1.927806in}}%
\pgfpathlineto{\pgfqpoint{2.129184in}{1.459911in}}%
\pgfpathlineto{\pgfqpoint{2.130060in}{1.328393in}}%
\pgfpathlineto{\pgfqpoint{2.132686in}{1.116194in}}%
\pgfpathlineto{\pgfqpoint{2.133561in}{1.158422in}}%
\pgfpathlineto{\pgfqpoint{2.134436in}{1.253965in}}%
\pgfpathlineto{\pgfqpoint{2.136187in}{1.663885in}}%
\pgfpathlineto{\pgfqpoint{2.137938in}{2.074553in}}%
\pgfpathlineto{\pgfqpoint{2.138813in}{2.186802in}}%
\pgfpathlineto{\pgfqpoint{2.139689in}{2.202067in}}%
\pgfpathlineto{\pgfqpoint{2.140564in}{2.267234in}}%
\pgfpathlineto{\pgfqpoint{2.142315in}{2.510136in}}%
\pgfpathlineto{\pgfqpoint{2.143190in}{2.537762in}}%
\pgfpathlineto{\pgfqpoint{2.144066in}{2.409167in}}%
\pgfpathlineto{\pgfqpoint{2.146692in}{1.911888in}}%
\pgfpathlineto{\pgfqpoint{2.147567in}{1.821510in}}%
\pgfpathlineto{\pgfqpoint{2.151944in}{1.035332in}}%
\pgfpathlineto{\pgfqpoint{2.152819in}{0.968425in}}%
\pgfpathlineto{\pgfqpoint{2.153695in}{0.958487in}}%
\pgfpathlineto{\pgfqpoint{2.154570in}{0.961387in}}%
\pgfpathlineto{\pgfqpoint{2.155445in}{1.075204in}}%
\pgfpathlineto{\pgfqpoint{2.157196in}{1.487405in}}%
\pgfpathlineto{\pgfqpoint{2.158072in}{1.840056in}}%
\pgfpathlineto{\pgfqpoint{2.158947in}{1.776099in}}%
\pgfpathlineto{\pgfqpoint{2.159822in}{2.140471in}}%
\pgfpathlineto{\pgfqpoint{2.160698in}{2.172551in}}%
\pgfpathlineto{\pgfqpoint{2.161573in}{2.579465in}}%
\pgfpathlineto{\pgfqpoint{2.162448in}{2.723021in}}%
\pgfpathlineto{\pgfqpoint{2.163324in}{2.802434in}}%
\pgfpathlineto{\pgfqpoint{2.164199in}{2.792702in}}%
\pgfpathlineto{\pgfqpoint{2.165075in}{2.569755in}}%
\pgfpathlineto{\pgfqpoint{2.165950in}{2.271463in}}%
\pgfpathlineto{\pgfqpoint{2.167701in}{2.139522in}}%
\pgfpathlineto{\pgfqpoint{2.168576in}{2.058448in}}%
\pgfpathlineto{\pgfqpoint{2.169451in}{1.928410in}}%
\pgfpathlineto{\pgfqpoint{2.171202in}{1.543613in}}%
\pgfpathlineto{\pgfqpoint{2.172078in}{1.490027in}}%
\pgfpathlineto{\pgfqpoint{2.172953in}{1.373280in}}%
\pgfpathlineto{\pgfqpoint{2.173828in}{1.359264in}}%
\pgfpathlineto{\pgfqpoint{2.174704in}{1.275200in}}%
\pgfpathlineto{\pgfqpoint{2.175579in}{1.340627in}}%
\pgfpathlineto{\pgfqpoint{2.176454in}{1.468399in}}%
\pgfpathlineto{\pgfqpoint{2.178205in}{1.898606in}}%
\pgfpathlineto{\pgfqpoint{2.179081in}{1.973548in}}%
\pgfpathlineto{\pgfqpoint{2.179956in}{2.083500in}}%
\pgfpathlineto{\pgfqpoint{2.180831in}{2.090000in}}%
\pgfpathlineto{\pgfqpoint{2.181707in}{2.128631in}}%
\pgfpathlineto{\pgfqpoint{2.182582in}{2.338520in}}%
\pgfpathlineto{\pgfqpoint{2.183457in}{2.344386in}}%
\pgfpathlineto{\pgfqpoint{2.184333in}{2.384661in}}%
\pgfpathlineto{\pgfqpoint{2.185208in}{2.448108in}}%
\pgfpathlineto{\pgfqpoint{2.186959in}{2.264908in}}%
\pgfpathlineto{\pgfqpoint{2.187834in}{2.124419in}}%
\pgfpathlineto{\pgfqpoint{2.188710in}{2.058781in}}%
\pgfpathlineto{\pgfqpoint{2.189585in}{2.050172in}}%
\pgfpathlineto{\pgfqpoint{2.190460in}{1.922792in}}%
\pgfpathlineto{\pgfqpoint{2.193087in}{1.404392in}}%
\pgfpathlineto{\pgfqpoint{2.193962in}{1.249434in}}%
\pgfpathlineto{\pgfqpoint{2.195713in}{1.124259in}}%
\pgfpathlineto{\pgfqpoint{2.196588in}{1.175459in}}%
\pgfpathlineto{\pgfqpoint{2.198339in}{1.451278in}}%
\pgfpathlineto{\pgfqpoint{2.200090in}{1.907653in}}%
\pgfpathlineto{\pgfqpoint{2.200965in}{1.905986in}}%
\pgfpathlineto{\pgfqpoint{2.204467in}{2.302146in}}%
\pgfpathlineto{\pgfqpoint{2.205342in}{2.295374in}}%
\pgfpathlineto{\pgfqpoint{2.208843in}{1.998882in}}%
\pgfpathlineto{\pgfqpoint{2.209719in}{1.877150in}}%
\pgfpathlineto{\pgfqpoint{2.210594in}{1.706787in}}%
\pgfpathlineto{\pgfqpoint{2.211470in}{1.602485in}}%
\pgfpathlineto{\pgfqpoint{2.212345in}{1.395481in}}%
\pgfpathlineto{\pgfqpoint{2.214096in}{1.145917in}}%
\pgfpathlineto{\pgfqpoint{2.214971in}{1.096711in}}%
\pgfpathlineto{\pgfqpoint{2.215846in}{0.987727in}}%
\pgfpathlineto{\pgfqpoint{2.216722in}{1.014399in}}%
\pgfpathlineto{\pgfqpoint{2.217597in}{1.151264in}}%
\pgfpathlineto{\pgfqpoint{2.221099in}{1.962894in}}%
\pgfpathlineto{\pgfqpoint{2.221974in}{2.030223in}}%
\pgfpathlineto{\pgfqpoint{2.222849in}{2.042338in}}%
\pgfpathlineto{\pgfqpoint{2.223725in}{2.177820in}}%
\pgfpathlineto{\pgfqpoint{2.224600in}{2.236033in}}%
\pgfpathlineto{\pgfqpoint{2.226351in}{2.509705in}}%
\pgfpathlineto{\pgfqpoint{2.228977in}{2.110101in}}%
\pgfpathlineto{\pgfqpoint{2.229852in}{2.000483in}}%
\pgfpathlineto{\pgfqpoint{2.232479in}{1.564153in}}%
\pgfpathlineto{\pgfqpoint{2.234229in}{1.214908in}}%
\pgfpathlineto{\pgfqpoint{2.235105in}{1.104565in}}%
\pgfpathlineto{\pgfqpoint{2.235980in}{1.052278in}}%
\pgfpathlineto{\pgfqpoint{2.236855in}{1.033368in}}%
\pgfpathlineto{\pgfqpoint{2.237731in}{1.059527in}}%
\pgfpathlineto{\pgfqpoint{2.238606in}{1.193613in}}%
\pgfpathlineto{\pgfqpoint{2.240357in}{1.650977in}}%
\pgfpathlineto{\pgfqpoint{2.241232in}{1.966023in}}%
\pgfpathlineto{\pgfqpoint{2.242108in}{2.116189in}}%
\pgfpathlineto{\pgfqpoint{2.242983in}{2.171401in}}%
\pgfpathlineto{\pgfqpoint{2.243858in}{2.402027in}}%
\pgfpathlineto{\pgfqpoint{2.244734in}{2.488136in}}%
\pgfpathlineto{\pgfqpoint{2.245609in}{2.430658in}}%
\pgfpathlineto{\pgfqpoint{2.246485in}{2.675963in}}%
\pgfpathlineto{\pgfqpoint{2.247360in}{2.637068in}}%
\pgfpathlineto{\pgfqpoint{2.249111in}{2.347483in}}%
\pgfpathlineto{\pgfqpoint{2.249986in}{2.307439in}}%
\pgfpathlineto{\pgfqpoint{2.250861in}{2.360632in}}%
\pgfpathlineto{\pgfqpoint{2.251737in}{2.247509in}}%
\pgfpathlineto{\pgfqpoint{2.252612in}{2.207637in}}%
\pgfpathlineto{\pgfqpoint{2.254363in}{1.819577in}}%
\pgfpathlineto{\pgfqpoint{2.255238in}{1.720682in}}%
\pgfpathlineto{\pgfqpoint{2.256114in}{1.586929in}}%
\pgfpathlineto{\pgfqpoint{2.257864in}{1.449248in}}%
\pgfpathlineto{\pgfqpoint{2.259615in}{1.479062in}}%
\pgfpathlineto{\pgfqpoint{2.261366in}{1.787943in}}%
\pgfpathlineto{\pgfqpoint{2.262241in}{2.050078in}}%
\pgfpathlineto{\pgfqpoint{2.263992in}{2.349240in}}%
\pgfpathlineto{\pgfqpoint{2.266618in}{2.734063in}}%
\pgfpathlineto{\pgfqpoint{2.267494in}{2.766014in}}%
\pgfpathlineto{\pgfqpoint{2.268369in}{2.606641in}}%
\pgfpathlineto{\pgfqpoint{2.269244in}{2.523699in}}%
\pgfpathlineto{\pgfqpoint{2.270120in}{2.367152in}}%
\pgfpathlineto{\pgfqpoint{2.273621in}{1.985410in}}%
\pgfpathlineto{\pgfqpoint{2.276247in}{1.450124in}}%
\pgfpathlineto{\pgfqpoint{2.278873in}{1.216902in}}%
\pgfpathlineto{\pgfqpoint{2.279749in}{1.200923in}}%
\pgfpathlineto{\pgfqpoint{2.280624in}{1.356606in}}%
\pgfpathlineto{\pgfqpoint{2.283250in}{2.100342in}}%
\pgfpathlineto{\pgfqpoint{2.285001in}{2.395376in}}%
\pgfpathlineto{\pgfqpoint{2.285876in}{2.376620in}}%
\pgfpathlineto{\pgfqpoint{2.286752in}{2.406511in}}%
\pgfpathlineto{\pgfqpoint{2.287627in}{2.498897in}}%
\pgfpathlineto{\pgfqpoint{2.288503in}{2.485531in}}%
\pgfpathlineto{\pgfqpoint{2.289378in}{2.493088in}}%
\pgfpathlineto{\pgfqpoint{2.290253in}{2.392609in}}%
\pgfpathlineto{\pgfqpoint{2.291129in}{2.344132in}}%
\pgfpathlineto{\pgfqpoint{2.292004in}{2.251164in}}%
\pgfpathlineto{\pgfqpoint{2.292879in}{2.271977in}}%
\pgfpathlineto{\pgfqpoint{2.293755in}{2.145382in}}%
\pgfpathlineto{\pgfqpoint{2.294630in}{2.096780in}}%
\pgfpathlineto{\pgfqpoint{2.296381in}{1.806921in}}%
\pgfpathlineto{\pgfqpoint{2.297256in}{1.695278in}}%
\pgfpathlineto{\pgfqpoint{2.298132in}{1.651993in}}%
\pgfpathlineto{\pgfqpoint{2.299007in}{1.559743in}}%
\pgfpathlineto{\pgfqpoint{2.299882in}{1.499089in}}%
\pgfpathlineto{\pgfqpoint{2.300758in}{1.494920in}}%
\pgfpathlineto{\pgfqpoint{2.301633in}{1.591610in}}%
\pgfpathlineto{\pgfqpoint{2.302509in}{1.794808in}}%
\pgfpathlineto{\pgfqpoint{2.303384in}{1.916513in}}%
\pgfpathlineto{\pgfqpoint{2.305135in}{2.197004in}}%
\pgfpathlineto{\pgfqpoint{2.306010in}{2.260791in}}%
\pgfpathlineto{\pgfqpoint{2.307761in}{2.516582in}}%
\pgfpathlineto{\pgfqpoint{2.308636in}{2.499713in}}%
\pgfpathlineto{\pgfqpoint{2.309512in}{2.416415in}}%
\pgfpathlineto{\pgfqpoint{2.310387in}{2.505974in}}%
\pgfpathlineto{\pgfqpoint{2.311262in}{2.492688in}}%
\pgfpathlineto{\pgfqpoint{2.312138in}{2.381601in}}%
\pgfpathlineto{\pgfqpoint{2.313013in}{2.315292in}}%
\pgfpathlineto{\pgfqpoint{2.313888in}{2.212953in}}%
\pgfpathlineto{\pgfqpoint{2.315639in}{1.976771in}}%
\pgfpathlineto{\pgfqpoint{2.316515in}{1.950068in}}%
\pgfpathlineto{\pgfqpoint{2.318265in}{1.692107in}}%
\pgfpathlineto{\pgfqpoint{2.320891in}{1.482596in}}%
\pgfpathlineto{\pgfqpoint{2.321767in}{1.486100in}}%
\pgfpathlineto{\pgfqpoint{2.322642in}{1.573034in}}%
\pgfpathlineto{\pgfqpoint{2.323518in}{1.700504in}}%
\pgfpathlineto{\pgfqpoint{2.325268in}{2.057689in}}%
\pgfpathlineto{\pgfqpoint{2.326144in}{2.091480in}}%
\pgfpathlineto{\pgfqpoint{2.327019in}{2.212131in}}%
\pgfpathlineto{\pgfqpoint{2.327895in}{2.282691in}}%
\pgfpathlineto{\pgfqpoint{2.328770in}{2.199021in}}%
\pgfpathlineto{\pgfqpoint{2.329645in}{2.269803in}}%
\pgfpathlineto{\pgfqpoint{2.330521in}{2.162578in}}%
\pgfpathlineto{\pgfqpoint{2.331396in}{2.228200in}}%
\pgfpathlineto{\pgfqpoint{2.332271in}{2.117809in}}%
\pgfpathlineto{\pgfqpoint{2.333147in}{1.951628in}}%
\pgfpathlineto{\pgfqpoint{2.334022in}{1.913964in}}%
\pgfpathlineto{\pgfqpoint{2.334898in}{1.948437in}}%
\pgfpathlineto{\pgfqpoint{2.337524in}{1.726572in}}%
\pgfpathlineto{\pgfqpoint{2.339274in}{1.534762in}}%
\pgfpathlineto{\pgfqpoint{2.340150in}{1.510688in}}%
\pgfpathlineto{\pgfqpoint{2.341025in}{1.457706in}}%
\pgfpathlineto{\pgfqpoint{2.341901in}{1.471752in}}%
\pgfpathlineto{\pgfqpoint{2.342776in}{1.479243in}}%
\pgfpathlineto{\pgfqpoint{2.343651in}{1.521623in}}%
\pgfpathlineto{\pgfqpoint{2.345402in}{1.839941in}}%
\pgfpathlineto{\pgfqpoint{2.346277in}{1.892938in}}%
\pgfpathlineto{\pgfqpoint{2.347153in}{1.874234in}}%
\pgfpathlineto{\pgfqpoint{2.348028in}{1.834033in}}%
\pgfpathlineto{\pgfqpoint{2.348904in}{1.934780in}}%
\pgfpathlineto{\pgfqpoint{2.349779in}{1.844902in}}%
\pgfpathlineto{\pgfqpoint{2.350654in}{2.072609in}}%
\pgfpathlineto{\pgfqpoint{2.351530in}{2.077266in}}%
\pgfpathlineto{\pgfqpoint{2.352405in}{2.071900in}}%
\pgfpathlineto{\pgfqpoint{2.354156in}{1.764830in}}%
\pgfpathlineto{\pgfqpoint{2.355031in}{1.664740in}}%
\pgfpathlineto{\pgfqpoint{2.355907in}{1.669815in}}%
\pgfpathlineto{\pgfqpoint{2.357657in}{1.428557in}}%
\pgfpathlineto{\pgfqpoint{2.359408in}{1.150478in}}%
\pgfpathlineto{\pgfqpoint{2.361159in}{0.988270in}}%
\pgfpathlineto{\pgfqpoint{2.362034in}{0.957581in}}%
\pgfpathlineto{\pgfqpoint{2.362910in}{0.907076in}}%
\pgfpathlineto{\pgfqpoint{2.363785in}{0.928100in}}%
\pgfpathlineto{\pgfqpoint{2.364660in}{1.061944in}}%
\pgfpathlineto{\pgfqpoint{2.365536in}{1.263661in}}%
\pgfpathlineto{\pgfqpoint{2.367286in}{1.526714in}}%
\pgfpathlineto{\pgfqpoint{2.368162in}{1.606407in}}%
\pgfpathlineto{\pgfqpoint{2.369037in}{1.643667in}}%
\pgfpathlineto{\pgfqpoint{2.369913in}{1.575523in}}%
\pgfpathlineto{\pgfqpoint{2.371663in}{1.906369in}}%
\pgfpathlineto{\pgfqpoint{2.372539in}{1.899096in}}%
\pgfpathlineto{\pgfqpoint{2.373414in}{1.938371in}}%
\pgfpathlineto{\pgfqpoint{2.374289in}{1.826033in}}%
\pgfpathlineto{\pgfqpoint{2.375165in}{1.654473in}}%
\pgfpathlineto{\pgfqpoint{2.376916in}{1.449188in}}%
\pgfpathlineto{\pgfqpoint{2.378666in}{1.333951in}}%
\pgfpathlineto{\pgfqpoint{2.380417in}{1.097889in}}%
\pgfpathlineto{\pgfqpoint{2.381292in}{1.022373in}}%
\pgfpathlineto{\pgfqpoint{2.383043in}{0.936285in}}%
\pgfpathlineto{\pgfqpoint{2.383919in}{0.938219in}}%
\pgfpathlineto{\pgfqpoint{2.384794in}{0.955618in}}%
\pgfpathlineto{\pgfqpoint{2.385669in}{0.995309in}}%
\pgfpathlineto{\pgfqpoint{2.387420in}{1.416886in}}%
\pgfpathlineto{\pgfqpoint{2.388295in}{1.544533in}}%
\pgfpathlineto{\pgfqpoint{2.389171in}{2.029792in}}%
\pgfpathlineto{\pgfqpoint{2.390046in}{1.846085in}}%
\pgfpathlineto{\pgfqpoint{2.390922in}{1.880128in}}%
\pgfpathlineto{\pgfqpoint{2.391797in}{1.843209in}}%
\pgfpathlineto{\pgfqpoint{2.392672in}{1.824465in}}%
\pgfpathlineto{\pgfqpoint{2.393548in}{1.927086in}}%
\pgfpathlineto{\pgfqpoint{2.394423in}{1.914012in}}%
\pgfpathlineto{\pgfqpoint{2.395298in}{1.813456in}}%
\pgfpathlineto{\pgfqpoint{2.396174in}{1.655679in}}%
\pgfpathlineto{\pgfqpoint{2.397925in}{1.512833in}}%
\pgfpathlineto{\pgfqpoint{2.399675in}{1.291481in}}%
\pgfpathlineto{\pgfqpoint{2.400551in}{1.298549in}}%
\pgfpathlineto{\pgfqpoint{2.404052in}{1.024669in}}%
\pgfpathlineto{\pgfqpoint{2.404928in}{1.026149in}}%
\pgfpathlineto{\pgfqpoint{2.405803in}{0.975282in}}%
\pgfpathlineto{\pgfqpoint{2.407554in}{1.232730in}}%
\pgfpathlineto{\pgfqpoint{2.410180in}{1.589676in}}%
\pgfpathlineto{\pgfqpoint{2.411055in}{1.475900in}}%
\pgfpathlineto{\pgfqpoint{2.411931in}{1.478735in}}%
\pgfpathlineto{\pgfqpoint{2.412806in}{1.470893in}}%
\pgfpathlineto{\pgfqpoint{2.413681in}{1.509750in}}%
\pgfpathlineto{\pgfqpoint{2.414557in}{1.583348in}}%
\pgfpathlineto{\pgfqpoint{2.415432in}{1.583214in}}%
\pgfpathlineto{\pgfqpoint{2.417183in}{1.337888in}}%
\pgfpathlineto{\pgfqpoint{2.418058in}{1.330646in}}%
\pgfpathlineto{\pgfqpoint{2.418934in}{1.310632in}}%
\pgfpathlineto{\pgfqpoint{2.419809in}{1.271726in}}%
\pgfpathlineto{\pgfqpoint{2.420684in}{1.202524in}}%
\pgfpathlineto{\pgfqpoint{2.422435in}{1.027992in}}%
\pgfpathlineto{\pgfqpoint{2.424186in}{0.917920in}}%
\pgfpathlineto{\pgfqpoint{2.425061in}{0.912876in}}%
\pgfpathlineto{\pgfqpoint{2.425937in}{0.956282in}}%
\pgfpathlineto{\pgfqpoint{2.426812in}{0.926046in}}%
\pgfpathlineto{\pgfqpoint{2.427687in}{0.975916in}}%
\pgfpathlineto{\pgfqpoint{2.428563in}{1.244994in}}%
\pgfpathlineto{\pgfqpoint{2.430313in}{1.471441in}}%
\pgfpathlineto{\pgfqpoint{2.431189in}{1.429949in}}%
\pgfpathlineto{\pgfqpoint{2.432064in}{1.318951in}}%
\pgfpathlineto{\pgfqpoint{2.432940in}{1.391523in}}%
\pgfpathlineto{\pgfqpoint{2.433815in}{1.387406in}}%
\pgfpathlineto{\pgfqpoint{2.435566in}{1.605214in}}%
\pgfpathlineto{\pgfqpoint{2.436441in}{1.550503in}}%
\pgfpathlineto{\pgfqpoint{2.437316in}{1.458409in}}%
\pgfpathlineto{\pgfqpoint{2.438192in}{1.506682in}}%
\pgfpathlineto{\pgfqpoint{2.439067in}{1.162084in}}%
\pgfpathlineto{\pgfqpoint{2.439943in}{1.191215in}}%
\pgfpathlineto{\pgfqpoint{2.441693in}{0.973254in}}%
\pgfpathlineto{\pgfqpoint{2.445195in}{0.722859in}}%
\pgfpathlineto{\pgfqpoint{2.446070in}{0.685876in}}%
\pgfpathlineto{\pgfqpoint{2.446946in}{0.698732in}}%
\pgfpathlineto{\pgfqpoint{2.447821in}{0.724517in}}%
\pgfpathlineto{\pgfqpoint{2.448696in}{0.772020in}}%
\pgfpathlineto{\pgfqpoint{2.451322in}{1.207406in}}%
\pgfpathlineto{\pgfqpoint{2.452198in}{1.198464in}}%
\pgfpathlineto{\pgfqpoint{2.453073in}{1.313933in}}%
\pgfpathlineto{\pgfqpoint{2.454824in}{1.378655in}}%
\pgfpathlineto{\pgfqpoint{2.455699in}{1.901067in}}%
\pgfpathlineto{\pgfqpoint{2.456575in}{2.169344in}}%
\pgfpathlineto{\pgfqpoint{2.458325in}{2.024961in}}%
\pgfpathlineto{\pgfqpoint{2.459201in}{1.852658in}}%
\pgfpathlineto{\pgfqpoint{2.460076in}{1.764783in}}%
\pgfpathlineto{\pgfqpoint{2.460952in}{1.715728in}}%
\pgfpathlineto{\pgfqpoint{2.464453in}{1.264990in}}%
\pgfpathlineto{\pgfqpoint{2.465329in}{1.233727in}}%
\pgfpathlineto{\pgfqpoint{2.467079in}{1.137248in}}%
\pgfpathlineto{\pgfqpoint{2.467955in}{1.123504in}}%
\pgfpathlineto{\pgfqpoint{2.468830in}{1.090942in}}%
\pgfpathlineto{\pgfqpoint{2.469705in}{1.180715in}}%
\pgfpathlineto{\pgfqpoint{2.470581in}{1.405449in}}%
\pgfpathlineto{\pgfqpoint{2.472332in}{1.588754in}}%
\pgfpathlineto{\pgfqpoint{2.473207in}{1.729813in}}%
\pgfpathlineto{\pgfqpoint{2.474082in}{1.918579in}}%
\pgfpathlineto{\pgfqpoint{2.474958in}{2.013498in}}%
\pgfpathlineto{\pgfqpoint{2.475833in}{2.038460in}}%
\pgfpathlineto{\pgfqpoint{2.476708in}{2.140939in}}%
\pgfpathlineto{\pgfqpoint{2.477584in}{2.106414in}}%
\pgfpathlineto{\pgfqpoint{2.478459in}{2.154600in}}%
\pgfpathlineto{\pgfqpoint{2.479335in}{2.064384in}}%
\pgfpathlineto{\pgfqpoint{2.480210in}{1.844742in}}%
\pgfpathlineto{\pgfqpoint{2.481085in}{1.743669in}}%
\pgfpathlineto{\pgfqpoint{2.481961in}{1.834439in}}%
\pgfpathlineto{\pgfqpoint{2.482836in}{1.766112in}}%
\pgfpathlineto{\pgfqpoint{2.483711in}{1.803024in}}%
\pgfpathlineto{\pgfqpoint{2.484587in}{1.658094in}}%
\pgfpathlineto{\pgfqpoint{2.485462in}{1.383187in}}%
\pgfpathlineto{\pgfqpoint{2.486338in}{1.412215in}}%
\pgfpathlineto{\pgfqpoint{2.487213in}{1.315646in}}%
\pgfpathlineto{\pgfqpoint{2.488088in}{1.273690in}}%
\pgfpathlineto{\pgfqpoint{2.488964in}{1.346487in}}%
\pgfpathlineto{\pgfqpoint{2.489839in}{1.340808in}}%
\pgfpathlineto{\pgfqpoint{2.490714in}{1.343798in}}%
\pgfpathlineto{\pgfqpoint{2.493341in}{1.856340in}}%
\pgfpathlineto{\pgfqpoint{2.494216in}{2.012952in}}%
\pgfpathlineto{\pgfqpoint{2.495091in}{2.030323in}}%
\pgfpathlineto{\pgfqpoint{2.495967in}{2.174585in}}%
\pgfpathlineto{\pgfqpoint{2.496842in}{1.989090in}}%
\pgfpathlineto{\pgfqpoint{2.497717in}{1.869875in}}%
\pgfpathlineto{\pgfqpoint{2.498593in}{1.890608in}}%
\pgfpathlineto{\pgfqpoint{2.500344in}{1.689949in}}%
\pgfpathlineto{\pgfqpoint{2.501219in}{1.634185in}}%
\pgfpathlineto{\pgfqpoint{2.502094in}{1.540713in}}%
\pgfpathlineto{\pgfqpoint{2.502970in}{1.595930in}}%
\pgfpathlineto{\pgfqpoint{2.503845in}{1.492474in}}%
\pgfpathlineto{\pgfqpoint{2.504720in}{1.440217in}}%
\pgfpathlineto{\pgfqpoint{2.505596in}{1.262543in}}%
\pgfpathlineto{\pgfqpoint{2.506471in}{1.179869in}}%
\pgfpathlineto{\pgfqpoint{2.507347in}{1.126102in}}%
\pgfpathlineto{\pgfqpoint{2.508222in}{1.100940in}}%
\pgfpathlineto{\pgfqpoint{2.509097in}{1.052670in}}%
\pgfpathlineto{\pgfqpoint{2.509973in}{1.091667in}}%
\pgfpathlineto{\pgfqpoint{2.510848in}{1.012164in}}%
\pgfpathlineto{\pgfqpoint{2.511723in}{1.075204in}}%
\pgfpathlineto{\pgfqpoint{2.513474in}{1.420734in}}%
\pgfpathlineto{\pgfqpoint{2.514350in}{1.680581in}}%
\pgfpathlineto{\pgfqpoint{2.515225in}{1.727905in}}%
\pgfpathlineto{\pgfqpoint{2.516100in}{1.756578in}}%
\pgfpathlineto{\pgfqpoint{2.516976in}{1.914244in}}%
\pgfpathlineto{\pgfqpoint{2.517851in}{1.925364in}}%
\pgfpathlineto{\pgfqpoint{2.519602in}{2.005255in}}%
\pgfpathlineto{\pgfqpoint{2.520477in}{1.983425in}}%
\pgfpathlineto{\pgfqpoint{2.521353in}{1.936455in}}%
\pgfpathlineto{\pgfqpoint{2.522228in}{1.763016in}}%
\pgfpathlineto{\pgfqpoint{2.523103in}{1.738017in}}%
\pgfpathlineto{\pgfqpoint{2.523979in}{1.744545in}}%
\pgfpathlineto{\pgfqpoint{2.524854in}{1.743155in}}%
\pgfpathlineto{\pgfqpoint{2.525729in}{1.670237in}}%
\pgfpathlineto{\pgfqpoint{2.528356in}{1.304288in}}%
\pgfpathlineto{\pgfqpoint{2.529231in}{1.164917in}}%
\pgfpathlineto{\pgfqpoint{2.530106in}{1.155492in}}%
\pgfpathlineto{\pgfqpoint{2.530982in}{1.083420in}}%
\pgfpathlineto{\pgfqpoint{2.531857in}{0.973318in}}%
\pgfpathlineto{\pgfqpoint{2.532732in}{1.011650in}}%
\pgfpathlineto{\pgfqpoint{2.535359in}{1.861994in}}%
\pgfpathlineto{\pgfqpoint{2.536234in}{1.991523in}}%
\pgfpathlineto{\pgfqpoint{2.537109in}{2.044366in}}%
\pgfpathlineto{\pgfqpoint{2.539735in}{2.426885in}}%
\pgfpathlineto{\pgfqpoint{2.540611in}{2.401341in}}%
\pgfpathlineto{\pgfqpoint{2.541486in}{2.479444in}}%
\pgfpathlineto{\pgfqpoint{2.542362in}{2.344013in}}%
\pgfpathlineto{\pgfqpoint{2.543237in}{2.145883in}}%
\pgfpathlineto{\pgfqpoint{2.544988in}{2.044946in}}%
\pgfpathlineto{\pgfqpoint{2.545863in}{1.803115in}}%
\pgfpathlineto{\pgfqpoint{2.546738in}{1.671446in}}%
\pgfpathlineto{\pgfqpoint{2.548489in}{1.291270in}}%
\pgfpathlineto{\pgfqpoint{2.550240in}{1.045662in}}%
\pgfpathlineto{\pgfqpoint{2.551115in}{0.970570in}}%
\pgfpathlineto{\pgfqpoint{2.551991in}{0.670138in}}%
\pgfpathlineto{\pgfqpoint{2.552866in}{0.690467in}}%
\pgfpathlineto{\pgfqpoint{2.553741in}{0.760153in}}%
\pgfpathlineto{\pgfqpoint{2.554617in}{0.909801in}}%
\pgfpathlineto{\pgfqpoint{2.556368in}{1.476311in}}%
\pgfpathlineto{\pgfqpoint{2.557243in}{1.574462in}}%
\pgfpathlineto{\pgfqpoint{2.558118in}{1.594669in}}%
\pgfpathlineto{\pgfqpoint{2.558994in}{1.679750in}}%
\pgfpathlineto{\pgfqpoint{2.559869in}{1.626369in}}%
\pgfpathlineto{\pgfqpoint{2.561620in}{2.023476in}}%
\pgfpathlineto{\pgfqpoint{2.562495in}{1.981630in}}%
\pgfpathlineto{\pgfqpoint{2.564246in}{1.602877in}}%
\pgfpathlineto{\pgfqpoint{2.565121in}{1.532859in}}%
\pgfpathlineto{\pgfqpoint{2.565997in}{1.669059in}}%
\pgfpathlineto{\pgfqpoint{2.566872in}{1.683709in}}%
\pgfpathlineto{\pgfqpoint{2.567747in}{1.673530in}}%
\pgfpathlineto{\pgfqpoint{2.568623in}{1.523284in}}%
\pgfpathlineto{\pgfqpoint{2.569498in}{1.323711in}}%
\pgfpathlineto{\pgfqpoint{2.570374in}{1.258466in}}%
\pgfpathlineto{\pgfqpoint{2.571249in}{1.156640in}}%
\pgfpathlineto{\pgfqpoint{2.572124in}{1.203671in}}%
\pgfpathlineto{\pgfqpoint{2.573000in}{1.180231in}}%
\pgfpathlineto{\pgfqpoint{2.573875in}{1.132687in}}%
\pgfpathlineto{\pgfqpoint{2.574750in}{1.140812in}}%
\pgfpathlineto{\pgfqpoint{2.577377in}{2.096138in}}%
\pgfpathlineto{\pgfqpoint{2.578252in}{2.216741in}}%
\pgfpathlineto{\pgfqpoint{2.579127in}{2.091930in}}%
\pgfpathlineto{\pgfqpoint{2.580003in}{2.332278in}}%
\pgfpathlineto{\pgfqpoint{2.580878in}{2.345719in}}%
\pgfpathlineto{\pgfqpoint{2.581753in}{2.314280in}}%
\pgfpathlineto{\pgfqpoint{2.582629in}{2.350083in}}%
\pgfpathlineto{\pgfqpoint{2.583504in}{2.198111in}}%
\pgfpathlineto{\pgfqpoint{2.584380in}{2.129331in}}%
\pgfpathlineto{\pgfqpoint{2.586130in}{1.877396in}}%
\pgfpathlineto{\pgfqpoint{2.587006in}{1.906964in}}%
\pgfpathlineto{\pgfqpoint{2.590507in}{1.358841in}}%
\pgfpathlineto{\pgfqpoint{2.591383in}{1.308397in}}%
\pgfpathlineto{\pgfqpoint{2.594009in}{1.089884in}}%
\pgfpathlineto{\pgfqpoint{2.594884in}{0.995822in}}%
\pgfpathlineto{\pgfqpoint{2.595760in}{1.116103in}}%
\pgfpathlineto{\pgfqpoint{2.598386in}{1.712819in}}%
\pgfpathlineto{\pgfqpoint{2.599261in}{1.716687in}}%
\pgfpathlineto{\pgfqpoint{2.600136in}{2.196806in}}%
\pgfpathlineto{\pgfqpoint{2.601012in}{2.432141in}}%
\pgfpathlineto{\pgfqpoint{2.601887in}{2.549027in}}%
\pgfpathlineto{\pgfqpoint{2.602763in}{2.541756in}}%
\pgfpathlineto{\pgfqpoint{2.603638in}{2.730849in}}%
\pgfpathlineto{\pgfqpoint{2.604513in}{2.778299in}}%
\pgfpathlineto{\pgfqpoint{2.605389in}{2.612976in}}%
\pgfpathlineto{\pgfqpoint{2.606264in}{2.281945in}}%
\pgfpathlineto{\pgfqpoint{2.607139in}{2.100909in}}%
\pgfpathlineto{\pgfqpoint{2.608015in}{2.118559in}}%
\pgfpathlineto{\pgfqpoint{2.610641in}{1.513135in}}%
\pgfpathlineto{\pgfqpoint{2.613267in}{1.132536in}}%
\pgfpathlineto{\pgfqpoint{2.614142in}{1.095624in}}%
\pgfpathlineto{\pgfqpoint{2.615018in}{1.037416in}}%
\pgfpathlineto{\pgfqpoint{2.615893in}{0.919798in}}%
\pgfpathlineto{\pgfqpoint{2.616769in}{1.000967in}}%
\pgfpathlineto{\pgfqpoint{2.617644in}{1.169855in}}%
\pgfpathlineto{\pgfqpoint{2.619395in}{1.579274in}}%
\pgfpathlineto{\pgfqpoint{2.620270in}{1.723106in}}%
\pgfpathlineto{\pgfqpoint{2.621145in}{1.745337in}}%
\pgfpathlineto{\pgfqpoint{2.622021in}{2.022302in}}%
\pgfpathlineto{\pgfqpoint{2.622896in}{1.996270in}}%
\pgfpathlineto{\pgfqpoint{2.623772in}{2.156922in}}%
\pgfpathlineto{\pgfqpoint{2.624647in}{2.098544in}}%
\pgfpathlineto{\pgfqpoint{2.625522in}{2.136420in}}%
\pgfpathlineto{\pgfqpoint{2.627273in}{1.883890in}}%
\pgfpathlineto{\pgfqpoint{2.628148in}{1.997643in}}%
\pgfpathlineto{\pgfqpoint{2.629024in}{1.975019in}}%
\pgfpathlineto{\pgfqpoint{2.629899in}{1.939979in}}%
\pgfpathlineto{\pgfqpoint{2.630775in}{1.799188in}}%
\pgfpathlineto{\pgfqpoint{2.631650in}{1.578229in}}%
\pgfpathlineto{\pgfqpoint{2.632525in}{1.475709in}}%
\pgfpathlineto{\pgfqpoint{2.633401in}{1.275834in}}%
\pgfpathlineto{\pgfqpoint{2.635151in}{1.137157in}}%
\pgfpathlineto{\pgfqpoint{2.636027in}{1.125679in}}%
\pgfpathlineto{\pgfqpoint{2.636902in}{1.109005in}}%
\pgfpathlineto{\pgfqpoint{2.637778in}{1.196724in}}%
\pgfpathlineto{\pgfqpoint{2.638653in}{1.427893in}}%
\pgfpathlineto{\pgfqpoint{2.639528in}{1.522511in}}%
\pgfpathlineto{\pgfqpoint{2.640404in}{1.726354in}}%
\pgfpathlineto{\pgfqpoint{2.641279in}{1.814441in}}%
\pgfpathlineto{\pgfqpoint{2.642154in}{1.968899in}}%
\pgfpathlineto{\pgfqpoint{2.643030in}{2.173801in}}%
\pgfpathlineto{\pgfqpoint{2.643905in}{2.241024in}}%
\pgfpathlineto{\pgfqpoint{2.644781in}{2.154750in}}%
\pgfpathlineto{\pgfqpoint{2.645656in}{2.173570in}}%
\pgfpathlineto{\pgfqpoint{2.646531in}{2.471496in}}%
\pgfpathlineto{\pgfqpoint{2.647407in}{2.357028in}}%
\pgfpathlineto{\pgfqpoint{2.649157in}{2.050716in}}%
\pgfpathlineto{\pgfqpoint{2.650033in}{2.100012in}}%
\pgfpathlineto{\pgfqpoint{2.650908in}{1.953089in}}%
\pgfpathlineto{\pgfqpoint{2.652659in}{1.595235in}}%
\pgfpathlineto{\pgfqpoint{2.656160in}{1.062699in}}%
\pgfpathlineto{\pgfqpoint{2.657036in}{1.049166in}}%
\pgfpathlineto{\pgfqpoint{2.657911in}{1.009083in}}%
\pgfpathlineto{\pgfqpoint{2.659662in}{1.281301in}}%
\pgfpathlineto{\pgfqpoint{2.661413in}{1.870324in}}%
\pgfpathlineto{\pgfqpoint{2.662288in}{2.065171in}}%
\pgfpathlineto{\pgfqpoint{2.663163in}{2.161635in}}%
\pgfpathlineto{\pgfqpoint{2.664039in}{2.084639in}}%
\pgfpathlineto{\pgfqpoint{2.664914in}{2.201680in}}%
\pgfpathlineto{\pgfqpoint{2.666665in}{2.538145in}}%
\pgfpathlineto{\pgfqpoint{2.667540in}{2.447677in}}%
\pgfpathlineto{\pgfqpoint{2.668416in}{2.432382in}}%
\pgfpathlineto{\pgfqpoint{2.671042in}{2.013743in}}%
\pgfpathlineto{\pgfqpoint{2.671917in}{1.950310in}}%
\pgfpathlineto{\pgfqpoint{2.672793in}{1.791032in}}%
\pgfpathlineto{\pgfqpoint{2.674543in}{1.306509in}}%
\pgfpathlineto{\pgfqpoint{2.676294in}{1.124308in}}%
\pgfpathlineto{\pgfqpoint{2.677169in}{1.055698in}}%
\pgfpathlineto{\pgfqpoint{2.678045in}{1.045144in}}%
\pgfpathlineto{\pgfqpoint{2.678920in}{1.073164in}}%
\pgfpathlineto{\pgfqpoint{2.679796in}{1.156099in}}%
\pgfpathlineto{\pgfqpoint{2.681546in}{1.681060in}}%
\pgfpathlineto{\pgfqpoint{2.682422in}{1.856590in}}%
\pgfpathlineto{\pgfqpoint{2.683297in}{1.944402in}}%
\pgfpathlineto{\pgfqpoint{2.685048in}{2.169550in}}%
\pgfpathlineto{\pgfqpoint{2.686799in}{2.705277in}}%
\pgfpathlineto{\pgfqpoint{2.687674in}{2.764540in}}%
\pgfpathlineto{\pgfqpoint{2.688549in}{2.862470in}}%
\pgfpathlineto{\pgfqpoint{2.690300in}{2.412044in}}%
\pgfpathlineto{\pgfqpoint{2.691175in}{2.211991in}}%
\pgfpathlineto{\pgfqpoint{2.692051in}{2.203348in}}%
\pgfpathlineto{\pgfqpoint{2.692926in}{2.034706in}}%
\pgfpathlineto{\pgfqpoint{2.693802in}{1.920043in}}%
\pgfpathlineto{\pgfqpoint{2.694677in}{1.671385in}}%
\pgfpathlineto{\pgfqpoint{2.696428in}{1.401429in}}%
\pgfpathlineto{\pgfqpoint{2.697303in}{1.346921in}}%
\pgfpathlineto{\pgfqpoint{2.698178in}{1.317634in}}%
\pgfpathlineto{\pgfqpoint{2.699929in}{1.276551in}}%
\pgfpathlineto{\pgfqpoint{2.700805in}{1.336313in}}%
\pgfpathlineto{\pgfqpoint{2.703431in}{1.810502in}}%
\pgfpathlineto{\pgfqpoint{2.705181in}{2.016512in}}%
\pgfpathlineto{\pgfqpoint{2.706057in}{2.004383in}}%
\pgfpathlineto{\pgfqpoint{2.706932in}{2.079426in}}%
\pgfpathlineto{\pgfqpoint{2.707808in}{2.299548in}}%
\pgfpathlineto{\pgfqpoint{2.708683in}{2.421341in}}%
\pgfpathlineto{\pgfqpoint{2.709558in}{2.449160in}}%
\pgfpathlineto{\pgfqpoint{2.710434in}{2.312380in}}%
\pgfpathlineto{\pgfqpoint{2.711309in}{2.248598in}}%
\pgfpathlineto{\pgfqpoint{2.712184in}{2.111513in}}%
\pgfpathlineto{\pgfqpoint{2.713060in}{2.073068in}}%
\pgfpathlineto{\pgfqpoint{2.714811in}{1.835466in}}%
\pgfpathlineto{\pgfqpoint{2.716561in}{1.477401in}}%
\pgfpathlineto{\pgfqpoint{2.718312in}{1.265111in}}%
\pgfpathlineto{\pgfqpoint{2.719187in}{1.248316in}}%
\pgfpathlineto{\pgfqpoint{2.720063in}{1.133895in}}%
\pgfpathlineto{\pgfqpoint{2.720938in}{1.180511in}}%
\pgfpathlineto{\pgfqpoint{2.721814in}{1.292523in}}%
\pgfpathlineto{\pgfqpoint{2.725315in}{1.965919in}}%
\pgfpathlineto{\pgfqpoint{2.726191in}{1.912793in}}%
\pgfpathlineto{\pgfqpoint{2.727066in}{2.043545in}}%
\pgfpathlineto{\pgfqpoint{2.727941in}{1.990284in}}%
\pgfpathlineto{\pgfqpoint{2.729692in}{2.234197in}}%
\pgfpathlineto{\pgfqpoint{2.730567in}{2.280349in}}%
\pgfpathlineto{\pgfqpoint{2.733194in}{1.885224in}}%
\pgfpathlineto{\pgfqpoint{2.734069in}{1.900802in}}%
\pgfpathlineto{\pgfqpoint{2.734944in}{1.873012in}}%
\pgfpathlineto{\pgfqpoint{2.735820in}{1.611396in}}%
\pgfpathlineto{\pgfqpoint{2.736695in}{1.568926in}}%
\pgfpathlineto{\pgfqpoint{2.738446in}{1.325775in}}%
\pgfpathlineto{\pgfqpoint{2.739321in}{1.264733in}}%
\pgfpathlineto{\pgfqpoint{2.741947in}{1.160898in}}%
\pgfpathlineto{\pgfqpoint{2.742823in}{1.266564in}}%
\pgfpathlineto{\pgfqpoint{2.743698in}{1.316440in}}%
\pgfpathlineto{\pgfqpoint{2.744573in}{1.298774in}}%
\pgfpathlineto{\pgfqpoint{2.746324in}{1.590327in}}%
\pgfpathlineto{\pgfqpoint{2.747200in}{1.613661in}}%
\pgfpathlineto{\pgfqpoint{2.748075in}{1.778507in}}%
\pgfpathlineto{\pgfqpoint{2.748950in}{1.762331in}}%
\pgfpathlineto{\pgfqpoint{2.749826in}{1.786994in}}%
\pgfpathlineto{\pgfqpoint{2.750701in}{1.996710in}}%
\pgfpathlineto{\pgfqpoint{2.751576in}{2.025471in}}%
\pgfpathlineto{\pgfqpoint{2.752452in}{2.007827in}}%
\pgfpathlineto{\pgfqpoint{2.753327in}{1.825007in}}%
\pgfpathlineto{\pgfqpoint{2.754203in}{1.839100in}}%
\pgfpathlineto{\pgfqpoint{2.755078in}{1.823353in}}%
\pgfpathlineto{\pgfqpoint{2.756829in}{1.382885in}}%
\pgfpathlineto{\pgfqpoint{2.757704in}{1.278825in}}%
\pgfpathlineto{\pgfqpoint{2.758579in}{1.054392in}}%
\pgfpathlineto{\pgfqpoint{2.759455in}{1.130959in}}%
\pgfpathlineto{\pgfqpoint{2.760330in}{1.099025in}}%
\pgfpathlineto{\pgfqpoint{2.761206in}{1.089022in}}%
\pgfpathlineto{\pgfqpoint{2.762956in}{1.048987in}}%
\pgfpathlineto{\pgfqpoint{2.763832in}{1.052485in}}%
\pgfpathlineto{\pgfqpoint{2.764707in}{1.305487in}}%
\pgfpathlineto{\pgfqpoint{2.765582in}{1.352150in}}%
\pgfpathlineto{\pgfqpoint{2.766458in}{1.509203in}}%
\pgfpathlineto{\pgfqpoint{2.767333in}{1.574595in}}%
\pgfpathlineto{\pgfqpoint{2.768209in}{1.589728in}}%
\pgfpathlineto{\pgfqpoint{2.769084in}{1.735900in}}%
\pgfpathlineto{\pgfqpoint{2.769959in}{1.769004in}}%
\pgfpathlineto{\pgfqpoint{2.770835in}{1.816348in}}%
\pgfpathlineto{\pgfqpoint{2.771710in}{1.776633in}}%
\pgfpathlineto{\pgfqpoint{2.772585in}{1.988179in}}%
\pgfpathlineto{\pgfqpoint{2.773461in}{2.008123in}}%
\pgfpathlineto{\pgfqpoint{2.774336in}{1.895301in}}%
\pgfpathlineto{\pgfqpoint{2.775212in}{1.925360in}}%
\pgfpathlineto{\pgfqpoint{2.776962in}{1.954146in}}%
\pgfpathlineto{\pgfqpoint{2.777838in}{1.874130in}}%
\pgfpathlineto{\pgfqpoint{2.778713in}{1.703525in}}%
\pgfpathlineto{\pgfqpoint{2.780464in}{1.566418in}}%
\pgfpathlineto{\pgfqpoint{2.781339in}{1.421247in}}%
\pgfpathlineto{\pgfqpoint{2.782215in}{1.431155in}}%
\pgfpathlineto{\pgfqpoint{2.783090in}{1.384667in}}%
\pgfpathlineto{\pgfqpoint{2.783965in}{1.291451in}}%
\pgfpathlineto{\pgfqpoint{2.784841in}{1.261396in}}%
\pgfpathlineto{\pgfqpoint{2.785716in}{1.252469in}}%
\pgfpathlineto{\pgfqpoint{2.786591in}{1.232725in}}%
\pgfpathlineto{\pgfqpoint{2.787467in}{1.524610in}}%
\pgfpathlineto{\pgfqpoint{2.789218in}{1.680750in}}%
\pgfpathlineto{\pgfqpoint{2.791844in}{2.119756in}}%
\pgfpathlineto{\pgfqpoint{2.793594in}{2.738315in}}%
\pgfpathlineto{\pgfqpoint{2.794470in}{2.659063in}}%
\pgfpathlineto{\pgfqpoint{2.795345in}{2.687586in}}%
\pgfpathlineto{\pgfqpoint{2.796221in}{2.591392in}}%
\pgfpathlineto{\pgfqpoint{2.797096in}{2.636234in}}%
\pgfpathlineto{\pgfqpoint{2.797971in}{2.534378in}}%
\pgfpathlineto{\pgfqpoint{2.798847in}{2.540691in}}%
\pgfpathlineto{\pgfqpoint{2.800597in}{2.096961in}}%
\pgfpathlineto{\pgfqpoint{2.802348in}{1.787679in}}%
\pgfpathlineto{\pgfqpoint{2.803224in}{1.731345in}}%
\pgfpathlineto{\pgfqpoint{2.804099in}{1.705730in}}%
\pgfpathlineto{\pgfqpoint{2.804974in}{1.700595in}}%
\pgfpathlineto{\pgfqpoint{2.805850in}{1.723008in}}%
\pgfpathlineto{\pgfqpoint{2.807600in}{2.003886in}}%
\pgfpathlineto{\pgfqpoint{2.808476in}{2.364815in}}%
\pgfpathlineto{\pgfqpoint{2.809351in}{2.589013in}}%
\pgfpathlineto{\pgfqpoint{2.811102in}{2.764201in}}%
\pgfpathlineto{\pgfqpoint{2.811977in}{2.753940in}}%
\pgfpathlineto{\pgfqpoint{2.812853in}{2.954457in}}%
\pgfpathlineto{\pgfqpoint{2.813728in}{2.996178in}}%
\pgfpathlineto{\pgfqpoint{2.814603in}{3.012048in}}%
\pgfpathlineto{\pgfqpoint{2.815479in}{2.965348in}}%
\pgfpathlineto{\pgfqpoint{2.817230in}{2.689931in}}%
\pgfpathlineto{\pgfqpoint{2.818105in}{2.582950in}}%
\pgfpathlineto{\pgfqpoint{2.818980in}{2.519064in}}%
\pgfpathlineto{\pgfqpoint{2.819856in}{2.515288in}}%
\pgfpathlineto{\pgfqpoint{2.820731in}{2.231258in}}%
\pgfpathlineto{\pgfqpoint{2.821606in}{2.143268in}}%
\pgfpathlineto{\pgfqpoint{2.822482in}{1.922007in}}%
\pgfpathlineto{\pgfqpoint{2.824233in}{1.677819in}}%
\pgfpathlineto{\pgfqpoint{2.825108in}{1.457102in}}%
\pgfpathlineto{\pgfqpoint{2.826859in}{1.572399in}}%
\pgfpathlineto{\pgfqpoint{2.829485in}{2.290883in}}%
\pgfpathlineto{\pgfqpoint{2.830360in}{2.390963in}}%
\pgfpathlineto{\pgfqpoint{2.831236in}{2.428457in}}%
\pgfpathlineto{\pgfqpoint{2.832986in}{2.576807in}}%
\pgfpathlineto{\pgfqpoint{2.834737in}{2.817962in}}%
\pgfpathlineto{\pgfqpoint{2.835612in}{2.652849in}}%
\pgfpathlineto{\pgfqpoint{2.836488in}{2.681271in}}%
\pgfpathlineto{\pgfqpoint{2.837363in}{2.552166in}}%
\pgfpathlineto{\pgfqpoint{2.838239in}{2.478483in}}%
\pgfpathlineto{\pgfqpoint{2.839114in}{2.469042in}}%
\pgfpathlineto{\pgfqpoint{2.839989in}{2.342418in}}%
\pgfpathlineto{\pgfqpoint{2.843491in}{1.657188in}}%
\pgfpathlineto{\pgfqpoint{2.844366in}{1.592094in}}%
\pgfpathlineto{\pgfqpoint{2.846117in}{1.406174in}}%
\pgfpathlineto{\pgfqpoint{2.846992in}{1.376089in}}%
\pgfpathlineto{\pgfqpoint{2.847868in}{1.460787in}}%
\pgfpathlineto{\pgfqpoint{2.848743in}{1.592521in}}%
\pgfpathlineto{\pgfqpoint{2.849618in}{1.868633in}}%
\pgfpathlineto{\pgfqpoint{2.850494in}{1.887882in}}%
\pgfpathlineto{\pgfqpoint{2.851369in}{2.206145in}}%
\pgfpathlineto{\pgfqpoint{2.852245in}{2.319763in}}%
\pgfpathlineto{\pgfqpoint{2.853120in}{2.503499in}}%
\pgfpathlineto{\pgfqpoint{2.853995in}{2.603807in}}%
\pgfpathlineto{\pgfqpoint{2.854871in}{2.814667in}}%
\pgfpathlineto{\pgfqpoint{2.855746in}{2.795524in}}%
\pgfpathlineto{\pgfqpoint{2.856621in}{2.930738in}}%
\pgfpathlineto{\pgfqpoint{2.858372in}{2.682795in}}%
\pgfpathlineto{\pgfqpoint{2.859248in}{2.752951in}}%
\pgfpathlineto{\pgfqpoint{2.860998in}{2.611676in}}%
\pgfpathlineto{\pgfqpoint{2.861874in}{2.429381in}}%
\pgfpathlineto{\pgfqpoint{2.863625in}{1.871864in}}%
\pgfpathlineto{\pgfqpoint{2.864500in}{1.713281in}}%
\pgfpathlineto{\pgfqpoint{2.867126in}{1.460183in}}%
\pgfpathlineto{\pgfqpoint{2.868001in}{1.461875in}}%
\pgfpathlineto{\pgfqpoint{2.868877in}{1.535034in}}%
\pgfpathlineto{\pgfqpoint{2.870628in}{1.986770in}}%
\pgfpathlineto{\pgfqpoint{2.871503in}{2.213081in}}%
\pgfpathlineto{\pgfqpoint{2.875004in}{2.642145in}}%
\pgfpathlineto{\pgfqpoint{2.875880in}{2.670830in}}%
\pgfpathlineto{\pgfqpoint{2.876755in}{2.973025in}}%
\pgfpathlineto{\pgfqpoint{2.877631in}{2.951439in}}%
\pgfpathlineto{\pgfqpoint{2.878506in}{2.880278in}}%
\pgfpathlineto{\pgfqpoint{2.879381in}{2.923560in}}%
\pgfpathlineto{\pgfqpoint{2.880257in}{2.849724in}}%
\pgfpathlineto{\pgfqpoint{2.881132in}{2.835867in}}%
\pgfpathlineto{\pgfqpoint{2.882007in}{2.777448in}}%
\pgfpathlineto{\pgfqpoint{2.882883in}{2.641520in}}%
\pgfpathlineto{\pgfqpoint{2.884634in}{2.114602in}}%
\pgfpathlineto{\pgfqpoint{2.885509in}{1.945507in}}%
\pgfpathlineto{\pgfqpoint{2.886384in}{1.900711in}}%
\pgfpathlineto{\pgfqpoint{2.887260in}{1.798010in}}%
\pgfpathlineto{\pgfqpoint{2.888135in}{1.742461in}}%
\pgfpathlineto{\pgfqpoint{2.889010in}{1.744394in}}%
\pgfpathlineto{\pgfqpoint{2.889886in}{1.781457in}}%
\pgfpathlineto{\pgfqpoint{2.891637in}{2.187560in}}%
\pgfpathlineto{\pgfqpoint{2.892512in}{2.550052in}}%
\pgfpathlineto{\pgfqpoint{2.894263in}{2.779886in}}%
\pgfpathlineto{\pgfqpoint{2.895138in}{3.105963in}}%
\pgfpathlineto{\pgfqpoint{2.896013in}{3.157606in}}%
\pgfpathlineto{\pgfqpoint{2.896889in}{3.170729in}}%
\pgfpathlineto{\pgfqpoint{2.897764in}{3.253669in}}%
\pgfpathlineto{\pgfqpoint{2.898640in}{3.142004in}}%
\pgfpathlineto{\pgfqpoint{2.899515in}{3.095455in}}%
\pgfpathlineto{\pgfqpoint{2.900390in}{3.134908in}}%
\pgfpathlineto{\pgfqpoint{2.901266in}{3.030527in}}%
\pgfpathlineto{\pgfqpoint{2.902141in}{2.988379in}}%
\pgfpathlineto{\pgfqpoint{2.903016in}{2.961284in}}%
\pgfpathlineto{\pgfqpoint{2.904767in}{2.690031in}}%
\pgfpathlineto{\pgfqpoint{2.905643in}{2.472758in}}%
\pgfpathlineto{\pgfqpoint{2.907393in}{2.203831in}}%
\pgfpathlineto{\pgfqpoint{2.908269in}{2.155924in}}%
\pgfpathlineto{\pgfqpoint{2.909144in}{1.988128in}}%
\pgfpathlineto{\pgfqpoint{2.910019in}{1.877392in}}%
\pgfpathlineto{\pgfqpoint{2.910895in}{1.830089in}}%
\pgfpathlineto{\pgfqpoint{2.911770in}{1.933972in}}%
\pgfpathlineto{\pgfqpoint{2.912646in}{2.155501in}}%
\pgfpathlineto{\pgfqpoint{2.913521in}{2.478715in}}%
\pgfpathlineto{\pgfqpoint{2.915272in}{2.786747in}}%
\pgfpathlineto{\pgfqpoint{2.916147in}{2.694162in}}%
\pgfpathlineto{\pgfqpoint{2.917898in}{3.116364in}}%
\pgfpathlineto{\pgfqpoint{2.918773in}{3.179170in}}%
\pgfpathlineto{\pgfqpoint{2.919649in}{3.094860in}}%
\pgfpathlineto{\pgfqpoint{2.920524in}{3.122632in}}%
\pgfpathlineto{\pgfqpoint{2.921399in}{3.014357in}}%
\pgfpathlineto{\pgfqpoint{2.922275in}{2.830130in}}%
\pgfpathlineto{\pgfqpoint{2.923150in}{2.799680in}}%
\pgfpathlineto{\pgfqpoint{2.924901in}{2.404250in}}%
\pgfpathlineto{\pgfqpoint{2.925776in}{2.056425in}}%
\pgfpathlineto{\pgfqpoint{2.928402in}{1.504043in}}%
\pgfpathlineto{\pgfqpoint{2.930153in}{1.413394in}}%
\pgfpathlineto{\pgfqpoint{2.931028in}{1.411551in}}%
\pgfpathlineto{\pgfqpoint{2.931904in}{1.447949in}}%
\pgfpathlineto{\pgfqpoint{2.932779in}{1.595672in}}%
\pgfpathlineto{\pgfqpoint{2.933655in}{1.828767in}}%
\pgfpathlineto{\pgfqpoint{2.934530in}{2.138621in}}%
\pgfpathlineto{\pgfqpoint{2.936281in}{2.302806in}}%
\pgfpathlineto{\pgfqpoint{2.937156in}{2.414445in}}%
\pgfpathlineto{\pgfqpoint{2.938031in}{2.634760in}}%
\pgfpathlineto{\pgfqpoint{2.938907in}{2.640429in}}%
\pgfpathlineto{\pgfqpoint{2.939782in}{2.815761in}}%
\pgfpathlineto{\pgfqpoint{2.940658in}{2.838175in}}%
\pgfpathlineto{\pgfqpoint{2.941533in}{2.819918in}}%
\pgfpathlineto{\pgfqpoint{2.943284in}{2.634516in}}%
\pgfpathlineto{\pgfqpoint{2.944159in}{2.670639in}}%
\pgfpathlineto{\pgfqpoint{2.945910in}{2.472788in}}%
\pgfpathlineto{\pgfqpoint{2.947661in}{1.972814in}}%
\pgfpathlineto{\pgfqpoint{2.949411in}{1.690355in}}%
\pgfpathlineto{\pgfqpoint{2.950287in}{1.662746in}}%
\pgfpathlineto{\pgfqpoint{2.951162in}{1.603935in}}%
\pgfpathlineto{\pgfqpoint{2.952037in}{1.569651in}}%
\pgfpathlineto{\pgfqpoint{2.952913in}{1.611516in}}%
\pgfpathlineto{\pgfqpoint{2.953788in}{1.815089in}}%
\pgfpathlineto{\pgfqpoint{2.954664in}{2.156427in}}%
\pgfpathlineto{\pgfqpoint{2.955539in}{2.626107in}}%
\pgfpathlineto{\pgfqpoint{2.956414in}{2.674789in}}%
\pgfpathlineto{\pgfqpoint{2.958165in}{3.226749in}}%
\pgfpathlineto{\pgfqpoint{2.959040in}{3.250911in}}%
\pgfpathlineto{\pgfqpoint{2.959916in}{3.464768in}}%
\pgfpathlineto{\pgfqpoint{2.960791in}{3.568298in}}%
\pgfpathlineto{\pgfqpoint{2.961667in}{3.707431in}}%
\pgfpathlineto{\pgfqpoint{2.963417in}{3.341443in}}%
\pgfpathlineto{\pgfqpoint{2.964293in}{3.262311in}}%
\pgfpathlineto{\pgfqpoint{2.965168in}{3.256701in}}%
\pgfpathlineto{\pgfqpoint{2.966919in}{2.839673in}}%
\pgfpathlineto{\pgfqpoint{2.968670in}{2.228510in}}%
\pgfpathlineto{\pgfqpoint{2.969545in}{1.918654in}}%
\pgfpathlineto{\pgfqpoint{2.972171in}{1.636165in}}%
\pgfpathlineto{\pgfqpoint{2.973046in}{1.559834in}}%
\pgfpathlineto{\pgfqpoint{2.973922in}{1.624566in}}%
\pgfpathlineto{\pgfqpoint{2.974797in}{1.877350in}}%
\pgfpathlineto{\pgfqpoint{2.975673in}{2.263025in}}%
\pgfpathlineto{\pgfqpoint{2.978299in}{3.005243in}}%
\pgfpathlineto{\pgfqpoint{2.979174in}{3.026074in}}%
\pgfpathlineto{\pgfqpoint{2.981800in}{3.356245in}}%
\pgfpathlineto{\pgfqpoint{2.982676in}{3.396951in}}%
\pgfpathlineto{\pgfqpoint{2.983551in}{3.362889in}}%
\pgfpathlineto{\pgfqpoint{2.985302in}{2.955864in}}%
\pgfpathlineto{\pgfqpoint{2.986177in}{3.138927in}}%
\pgfpathlineto{\pgfqpoint{2.987052in}{3.048972in}}%
\pgfpathlineto{\pgfqpoint{2.987928in}{3.028251in}}%
\pgfpathlineto{\pgfqpoint{2.989679in}{2.454483in}}%
\pgfpathlineto{\pgfqpoint{2.990554in}{2.199270in}}%
\pgfpathlineto{\pgfqpoint{2.992305in}{1.982933in}}%
\pgfpathlineto{\pgfqpoint{2.993180in}{1.918322in}}%
\pgfpathlineto{\pgfqpoint{2.994056in}{1.810757in}}%
\pgfpathlineto{\pgfqpoint{2.995806in}{1.992273in}}%
\pgfpathlineto{\pgfqpoint{2.996682in}{2.458746in}}%
\pgfpathlineto{\pgfqpoint{2.998432in}{2.824567in}}%
\pgfpathlineto{\pgfqpoint{2.999308in}{2.816340in}}%
\pgfpathlineto{\pgfqpoint{3.000183in}{3.009957in}}%
\pgfpathlineto{\pgfqpoint{3.001059in}{3.102287in}}%
\pgfpathlineto{\pgfqpoint{3.001934in}{3.319235in}}%
\pgfpathlineto{\pgfqpoint{3.002809in}{3.342899in}}%
\pgfpathlineto{\pgfqpoint{3.003685in}{3.305977in}}%
\pgfpathlineto{\pgfqpoint{3.006311in}{3.027890in}}%
\pgfpathlineto{\pgfqpoint{3.007186in}{3.138594in}}%
\pgfpathlineto{\pgfqpoint{3.008062in}{3.175537in}}%
\pgfpathlineto{\pgfqpoint{3.008937in}{3.046979in}}%
\pgfpathlineto{\pgfqpoint{3.010688in}{2.548032in}}%
\pgfpathlineto{\pgfqpoint{3.011563in}{2.328553in}}%
\pgfpathlineto{\pgfqpoint{3.013314in}{2.132182in}}%
\pgfpathlineto{\pgfqpoint{3.014189in}{2.061801in}}%
\pgfpathlineto{\pgfqpoint{3.015065in}{2.019452in}}%
\pgfpathlineto{\pgfqpoint{3.015940in}{2.027517in}}%
\pgfpathlineto{\pgfqpoint{3.016815in}{2.114740in}}%
\pgfpathlineto{\pgfqpoint{3.017691in}{2.270290in}}%
\pgfpathlineto{\pgfqpoint{3.019441in}{3.000378in}}%
\pgfpathlineto{\pgfqpoint{3.020317in}{3.128637in}}%
\pgfpathlineto{\pgfqpoint{3.021192in}{3.144763in}}%
\pgfpathlineto{\pgfqpoint{3.022068in}{3.183781in}}%
\pgfpathlineto{\pgfqpoint{3.022943in}{3.337884in}}%
\pgfpathlineto{\pgfqpoint{3.023818in}{3.346808in}}%
\pgfpathlineto{\pgfqpoint{3.025569in}{3.157323in}}%
\pgfpathlineto{\pgfqpoint{3.026444in}{3.105588in}}%
\pgfpathlineto{\pgfqpoint{3.027320in}{2.978936in}}%
\pgfpathlineto{\pgfqpoint{3.028195in}{3.021364in}}%
\pgfpathlineto{\pgfqpoint{3.029071in}{3.027979in}}%
\pgfpathlineto{\pgfqpoint{3.029946in}{2.860274in}}%
\pgfpathlineto{\pgfqpoint{3.031697in}{2.289375in}}%
\pgfpathlineto{\pgfqpoint{3.032572in}{2.111823in}}%
\pgfpathlineto{\pgfqpoint{3.035198in}{1.883161in}}%
\pgfpathlineto{\pgfqpoint{3.036074in}{1.793690in}}%
\pgfpathlineto{\pgfqpoint{3.036949in}{1.835164in}}%
\pgfpathlineto{\pgfqpoint{3.038700in}{2.257778in}}%
\pgfpathlineto{\pgfqpoint{3.039575in}{2.604348in}}%
\pgfpathlineto{\pgfqpoint{3.040450in}{2.410888in}}%
\pgfpathlineto{\pgfqpoint{3.043952in}{3.147041in}}%
\pgfpathlineto{\pgfqpoint{3.044827in}{3.131713in}}%
\pgfpathlineto{\pgfqpoint{3.047453in}{2.603969in}}%
\pgfpathlineto{\pgfqpoint{3.048329in}{2.580589in}}%
\pgfpathlineto{\pgfqpoint{3.049204in}{2.587390in}}%
\pgfpathlineto{\pgfqpoint{3.050080in}{2.408207in}}%
\pgfpathlineto{\pgfqpoint{3.050955in}{2.303874in}}%
\pgfpathlineto{\pgfqpoint{3.051830in}{1.906209in}}%
\pgfpathlineto{\pgfqpoint{3.052706in}{1.661145in}}%
\pgfpathlineto{\pgfqpoint{3.053581in}{1.632510in}}%
\pgfpathlineto{\pgfqpoint{3.056207in}{1.376904in}}%
\pgfpathlineto{\pgfqpoint{3.057958in}{1.349809in}}%
\pgfpathlineto{\pgfqpoint{3.058833in}{1.394486in}}%
\pgfpathlineto{\pgfqpoint{3.059709in}{1.601829in}}%
\pgfpathlineto{\pgfqpoint{3.060584in}{2.005086in}}%
\pgfpathlineto{\pgfqpoint{3.061459in}{2.125258in}}%
\pgfpathlineto{\pgfqpoint{3.062335in}{2.191784in}}%
\pgfpathlineto{\pgfqpoint{3.063210in}{2.321873in}}%
\pgfpathlineto{\pgfqpoint{3.064086in}{2.341375in}}%
\pgfpathlineto{\pgfqpoint{3.064961in}{2.523006in}}%
\pgfpathlineto{\pgfqpoint{3.065836in}{2.563056in}}%
\pgfpathlineto{\pgfqpoint{3.066712in}{2.647871in}}%
\pgfpathlineto{\pgfqpoint{3.067587in}{2.605803in}}%
\pgfpathlineto{\pgfqpoint{3.068462in}{2.495870in}}%
\pgfpathlineto{\pgfqpoint{3.069338in}{2.474570in}}%
\pgfpathlineto{\pgfqpoint{3.070213in}{2.469737in}}%
\pgfpathlineto{\pgfqpoint{3.071089in}{2.578600in}}%
\pgfpathlineto{\pgfqpoint{3.071964in}{2.450526in}}%
\pgfpathlineto{\pgfqpoint{3.072839in}{2.169940in}}%
\pgfpathlineto{\pgfqpoint{3.073715in}{1.979036in}}%
\pgfpathlineto{\pgfqpoint{3.074590in}{1.715003in}}%
\pgfpathlineto{\pgfqpoint{3.075465in}{1.563639in}}%
\pgfpathlineto{\pgfqpoint{3.077216in}{1.460334in}}%
\pgfpathlineto{\pgfqpoint{3.078092in}{1.439945in}}%
\pgfpathlineto{\pgfqpoint{3.078967in}{1.471178in}}%
\pgfpathlineto{\pgfqpoint{3.079842in}{1.537531in}}%
\pgfpathlineto{\pgfqpoint{3.080718in}{1.766511in}}%
\pgfpathlineto{\pgfqpoint{3.082468in}{2.047040in}}%
\pgfpathlineto{\pgfqpoint{3.083344in}{2.352569in}}%
\pgfpathlineto{\pgfqpoint{3.084219in}{2.498348in}}%
\pgfpathlineto{\pgfqpoint{3.085095in}{2.470944in}}%
\pgfpathlineto{\pgfqpoint{3.085970in}{2.668972in}}%
\pgfpathlineto{\pgfqpoint{3.087721in}{2.896889in}}%
\pgfpathlineto{\pgfqpoint{3.089471in}{2.812989in}}%
\pgfpathlineto{\pgfqpoint{3.090347in}{2.587028in}}%
\pgfpathlineto{\pgfqpoint{3.091222in}{2.575338in}}%
\pgfpathlineto{\pgfqpoint{3.092973in}{2.300642in}}%
\pgfpathlineto{\pgfqpoint{3.093848in}{1.982540in}}%
\pgfpathlineto{\pgfqpoint{3.096474in}{1.467281in}}%
\pgfpathlineto{\pgfqpoint{3.097350in}{1.385423in}}%
\pgfpathlineto{\pgfqpoint{3.098225in}{1.331021in}}%
\pgfpathlineto{\pgfqpoint{3.099101in}{1.241792in}}%
\pgfpathlineto{\pgfqpoint{3.099976in}{1.300150in}}%
\pgfpathlineto{\pgfqpoint{3.102602in}{2.045357in}}%
\pgfpathlineto{\pgfqpoint{3.103477in}{2.355623in}}%
\pgfpathlineto{\pgfqpoint{3.104353in}{2.483243in}}%
\pgfpathlineto{\pgfqpoint{3.105228in}{2.694030in}}%
\pgfpathlineto{\pgfqpoint{3.106104in}{2.621330in}}%
\pgfpathlineto{\pgfqpoint{3.107854in}{2.912414in}}%
\pgfpathlineto{\pgfqpoint{3.111356in}{2.486204in}}%
\pgfpathlineto{\pgfqpoint{3.112231in}{2.436812in}}%
\pgfpathlineto{\pgfqpoint{3.113107in}{2.296806in}}%
\pgfpathlineto{\pgfqpoint{3.113982in}{2.341753in}}%
\pgfpathlineto{\pgfqpoint{3.114857in}{2.148856in}}%
\pgfpathlineto{\pgfqpoint{3.116608in}{1.615896in}}%
\pgfpathlineto{\pgfqpoint{3.117483in}{1.442905in}}%
\pgfpathlineto{\pgfqpoint{3.118359in}{1.419948in}}%
\pgfpathlineto{\pgfqpoint{3.119234in}{1.334465in}}%
\pgfpathlineto{\pgfqpoint{3.120110in}{1.191921in}}%
\pgfpathlineto{\pgfqpoint{3.120985in}{1.192646in}}%
\pgfpathlineto{\pgfqpoint{3.121860in}{1.379359in}}%
\pgfpathlineto{\pgfqpoint{3.123611in}{1.958472in}}%
\pgfpathlineto{\pgfqpoint{3.124486in}{2.080138in}}%
\pgfpathlineto{\pgfqpoint{3.125362in}{2.392502in}}%
\pgfpathlineto{\pgfqpoint{3.126237in}{2.490640in}}%
\pgfpathlineto{\pgfqpoint{3.127113in}{2.666365in}}%
\pgfpathlineto{\pgfqpoint{3.127988in}{2.900551in}}%
\pgfpathlineto{\pgfqpoint{3.128863in}{2.850716in}}%
\pgfpathlineto{\pgfqpoint{3.129739in}{2.875143in}}%
\pgfpathlineto{\pgfqpoint{3.130614in}{2.825186in}}%
\pgfpathlineto{\pgfqpoint{3.131490in}{2.619222in}}%
\pgfpathlineto{\pgfqpoint{3.132365in}{2.612397in}}%
\pgfpathlineto{\pgfqpoint{3.133240in}{2.756243in}}%
\pgfpathlineto{\pgfqpoint{3.134116in}{2.744191in}}%
\pgfpathlineto{\pgfqpoint{3.134991in}{2.693686in}}%
\pgfpathlineto{\pgfqpoint{3.135866in}{2.367277in}}%
\pgfpathlineto{\pgfqpoint{3.138493in}{1.787770in}}%
\pgfpathlineto{\pgfqpoint{3.140243in}{1.650724in}}%
\pgfpathlineto{\pgfqpoint{3.141994in}{1.457374in}}%
\pgfpathlineto{\pgfqpoint{3.142869in}{1.514866in}}%
\pgfpathlineto{\pgfqpoint{3.144620in}{2.022004in}}%
\pgfpathlineto{\pgfqpoint{3.145496in}{2.184685in}}%
\pgfpathlineto{\pgfqpoint{3.147246in}{2.560289in}}%
\pgfpathlineto{\pgfqpoint{3.148997in}{2.718196in}}%
\pgfpathlineto{\pgfqpoint{3.149872in}{2.874791in}}%
\pgfpathlineto{\pgfqpoint{3.150748in}{2.928283in}}%
\pgfpathlineto{\pgfqpoint{3.151623in}{2.829922in}}%
\pgfpathlineto{\pgfqpoint{3.152499in}{2.671696in}}%
\pgfpathlineto{\pgfqpoint{3.153374in}{2.636639in}}%
\pgfpathlineto{\pgfqpoint{3.154249in}{2.714529in}}%
\pgfpathlineto{\pgfqpoint{3.155125in}{2.627716in}}%
\pgfpathlineto{\pgfqpoint{3.156000in}{2.572891in}}%
\pgfpathlineto{\pgfqpoint{3.157751in}{1.985440in}}%
\pgfpathlineto{\pgfqpoint{3.158626in}{1.809277in}}%
\pgfpathlineto{\pgfqpoint{3.159502in}{1.695158in}}%
\pgfpathlineto{\pgfqpoint{3.160377in}{1.617709in}}%
\pgfpathlineto{\pgfqpoint{3.161252in}{1.511352in}}%
\pgfpathlineto{\pgfqpoint{3.162128in}{1.505674in}}%
\pgfpathlineto{\pgfqpoint{3.163003in}{1.517031in}}%
\pgfpathlineto{\pgfqpoint{3.164754in}{2.026563in}}%
\pgfpathlineto{\pgfqpoint{3.165629in}{2.346277in}}%
\pgfpathlineto{\pgfqpoint{3.167380in}{2.535989in}}%
\pgfpathlineto{\pgfqpoint{3.169131in}{2.797465in}}%
\pgfpathlineto{\pgfqpoint{3.170006in}{2.954165in}}%
\pgfpathlineto{\pgfqpoint{3.170881in}{2.987346in}}%
\pgfpathlineto{\pgfqpoint{3.171757in}{3.310575in}}%
\pgfpathlineto{\pgfqpoint{3.172632in}{3.242640in}}%
\pgfpathlineto{\pgfqpoint{3.173508in}{3.114042in}}%
\pgfpathlineto{\pgfqpoint{3.175258in}{3.014598in}}%
\pgfpathlineto{\pgfqpoint{3.176134in}{2.914554in}}%
\pgfpathlineto{\pgfqpoint{3.177884in}{2.438927in}}%
\pgfpathlineto{\pgfqpoint{3.179635in}{1.975109in}}%
\pgfpathlineto{\pgfqpoint{3.181386in}{1.706515in}}%
\pgfpathlineto{\pgfqpoint{3.182261in}{1.647704in}}%
\pgfpathlineto{\pgfqpoint{3.183137in}{1.459881in}}%
\pgfpathlineto{\pgfqpoint{3.184887in}{1.816518in}}%
\pgfpathlineto{\pgfqpoint{3.185763in}{2.241825in}}%
\pgfpathlineto{\pgfqpoint{3.187514in}{2.706330in}}%
\pgfpathlineto{\pgfqpoint{3.189264in}{3.027375in}}%
\pgfpathlineto{\pgfqpoint{3.190140in}{3.038331in}}%
\pgfpathlineto{\pgfqpoint{3.191890in}{3.196490in}}%
\pgfpathlineto{\pgfqpoint{3.193641in}{2.958426in}}%
\pgfpathlineto{\pgfqpoint{3.197143in}{2.500910in}}%
\pgfpathlineto{\pgfqpoint{3.198018in}{2.320367in}}%
\pgfpathlineto{\pgfqpoint{3.199769in}{1.790609in}}%
\pgfpathlineto{\pgfqpoint{3.201520in}{1.479545in}}%
\pgfpathlineto{\pgfqpoint{3.202395in}{1.471631in}}%
\pgfpathlineto{\pgfqpoint{3.203270in}{1.498092in}}%
\pgfpathlineto{\pgfqpoint{3.204146in}{1.496914in}}%
\pgfpathlineto{\pgfqpoint{3.205021in}{1.582398in}}%
\pgfpathlineto{\pgfqpoint{3.205896in}{1.779630in}}%
\pgfpathlineto{\pgfqpoint{3.206772in}{1.893413in}}%
\pgfpathlineto{\pgfqpoint{3.208523in}{2.224795in}}%
\pgfpathlineto{\pgfqpoint{3.212024in}{2.689690in}}%
\pgfpathlineto{\pgfqpoint{3.212899in}{2.915441in}}%
\pgfpathlineto{\pgfqpoint{3.214650in}{2.782046in}}%
\pgfpathlineto{\pgfqpoint{3.216401in}{2.608142in}}%
\pgfpathlineto{\pgfqpoint{3.217276in}{2.586152in}}%
\pgfpathlineto{\pgfqpoint{3.218152in}{2.557033in}}%
\pgfpathlineto{\pgfqpoint{3.220778in}{1.771791in}}%
\pgfpathlineto{\pgfqpoint{3.221653in}{1.675886in}}%
\pgfpathlineto{\pgfqpoint{3.223404in}{1.383127in}}%
\pgfpathlineto{\pgfqpoint{3.224279in}{1.308034in}}%
\pgfpathlineto{\pgfqpoint{3.225155in}{1.317458in}}%
\pgfpathlineto{\pgfqpoint{3.226030in}{1.330417in}}%
\pgfpathlineto{\pgfqpoint{3.226905in}{1.470688in}}%
\pgfpathlineto{\pgfqpoint{3.227781in}{1.662358in}}%
\pgfpathlineto{\pgfqpoint{3.228656in}{2.061945in}}%
\pgfpathlineto{\pgfqpoint{3.232158in}{2.629906in}}%
\pgfpathlineto{\pgfqpoint{3.233033in}{2.716069in}}%
\pgfpathlineto{\pgfqpoint{3.233908in}{3.026258in}}%
\pgfpathlineto{\pgfqpoint{3.234784in}{3.055121in}}%
\pgfpathlineto{\pgfqpoint{3.236535in}{2.840014in}}%
\pgfpathlineto{\pgfqpoint{3.237410in}{2.753369in}}%
\pgfpathlineto{\pgfqpoint{3.238285in}{2.794334in}}%
\pgfpathlineto{\pgfqpoint{3.239161in}{2.707400in}}%
\pgfpathlineto{\pgfqpoint{3.240036in}{2.488797in}}%
\pgfpathlineto{\pgfqpoint{3.241787in}{1.913609in}}%
\pgfpathlineto{\pgfqpoint{3.243538in}{1.575873in}}%
\pgfpathlineto{\pgfqpoint{3.244413in}{1.453809in}}%
\pgfpathlineto{\pgfqpoint{3.245288in}{1.385120in}}%
\pgfpathlineto{\pgfqpoint{3.246164in}{1.355398in}}%
\pgfpathlineto{\pgfqpoint{3.247039in}{1.398019in}}%
\pgfpathlineto{\pgfqpoint{3.250541in}{2.613251in}}%
\pgfpathlineto{\pgfqpoint{3.252291in}{2.927044in}}%
\pgfpathlineto{\pgfqpoint{3.253167in}{2.981193in}}%
\pgfpathlineto{\pgfqpoint{3.254042in}{2.999511in}}%
\pgfpathlineto{\pgfqpoint{3.254917in}{3.383699in}}%
\pgfpathlineto{\pgfqpoint{3.255793in}{3.356040in}}%
\pgfpathlineto{\pgfqpoint{3.258419in}{2.833775in}}%
\pgfpathlineto{\pgfqpoint{3.259294in}{2.807050in}}%
\pgfpathlineto{\pgfqpoint{3.260170in}{2.796297in}}%
\pgfpathlineto{\pgfqpoint{3.261045in}{2.679610in}}%
\pgfpathlineto{\pgfqpoint{3.261921in}{2.518973in}}%
\pgfpathlineto{\pgfqpoint{3.263671in}{2.024708in}}%
\pgfpathlineto{\pgfqpoint{3.264547in}{1.880684in}}%
\pgfpathlineto{\pgfqpoint{3.265422in}{1.814805in}}%
\pgfpathlineto{\pgfqpoint{3.266297in}{1.791969in}}%
\pgfpathlineto{\pgfqpoint{3.267173in}{1.738775in}}%
\pgfpathlineto{\pgfqpoint{3.268048in}{1.718054in}}%
\pgfpathlineto{\pgfqpoint{3.268924in}{1.934850in}}%
\pgfpathlineto{\pgfqpoint{3.270674in}{2.638867in}}%
\pgfpathlineto{\pgfqpoint{3.271550in}{2.697900in}}%
\pgfpathlineto{\pgfqpoint{3.272425in}{2.985260in}}%
\pgfpathlineto{\pgfqpoint{3.273300in}{3.034258in}}%
\pgfpathlineto{\pgfqpoint{3.274176in}{3.119762in}}%
\pgfpathlineto{\pgfqpoint{3.275051in}{3.155152in}}%
\pgfpathlineto{\pgfqpoint{3.276802in}{3.255527in}}%
\pgfpathlineto{\pgfqpoint{3.277677in}{3.225783in}}%
\pgfpathlineto{\pgfqpoint{3.279428in}{2.892525in}}%
\pgfpathlineto{\pgfqpoint{3.280303in}{2.933796in}}%
\pgfpathlineto{\pgfqpoint{3.282054in}{3.127599in}}%
\pgfpathlineto{\pgfqpoint{3.282930in}{2.980797in}}%
\pgfpathlineto{\pgfqpoint{3.283805in}{2.597298in}}%
\pgfpathlineto{\pgfqpoint{3.284680in}{2.388180in}}%
\pgfpathlineto{\pgfqpoint{3.286431in}{2.164412in}}%
\pgfpathlineto{\pgfqpoint{3.287306in}{2.052075in}}%
\pgfpathlineto{\pgfqpoint{3.288182in}{1.900590in}}%
\pgfpathlineto{\pgfqpoint{3.289057in}{1.855734in}}%
\pgfpathlineto{\pgfqpoint{3.289933in}{1.982180in}}%
\pgfpathlineto{\pgfqpoint{3.290808in}{2.380049in}}%
\pgfpathlineto{\pgfqpoint{3.292559in}{2.758517in}}%
\pgfpathlineto{\pgfqpoint{3.293434in}{2.928443in}}%
\pgfpathlineto{\pgfqpoint{3.294309in}{2.971959in}}%
\pgfpathlineto{\pgfqpoint{3.295185in}{3.203039in}}%
\pgfpathlineto{\pgfqpoint{3.296936in}{3.498546in}}%
\pgfpathlineto{\pgfqpoint{3.297811in}{3.520035in}}%
\pgfpathlineto{\pgfqpoint{3.298686in}{3.431399in}}%
\pgfpathlineto{\pgfqpoint{3.300437in}{2.993999in}}%
\pgfpathlineto{\pgfqpoint{3.301312in}{2.870151in}}%
\pgfpathlineto{\pgfqpoint{3.302188in}{2.791373in}}%
\pgfpathlineto{\pgfqpoint{3.303063in}{2.787356in}}%
\pgfpathlineto{\pgfqpoint{3.304814in}{2.177038in}}%
\pgfpathlineto{\pgfqpoint{3.305689in}{2.143811in}}%
\pgfpathlineto{\pgfqpoint{3.307440in}{1.953965in}}%
\pgfpathlineto{\pgfqpoint{3.308315in}{1.939436in}}%
\pgfpathlineto{\pgfqpoint{3.309191in}{1.850629in}}%
\pgfpathlineto{\pgfqpoint{3.310066in}{1.868058in}}%
\pgfpathlineto{\pgfqpoint{3.310942in}{2.068825in}}%
\pgfpathlineto{\pgfqpoint{3.312692in}{2.725028in}}%
\pgfpathlineto{\pgfqpoint{3.313568in}{2.850371in}}%
\pgfpathlineto{\pgfqpoint{3.314443in}{3.051747in}}%
\pgfpathlineto{\pgfqpoint{3.315318in}{3.126504in}}%
\pgfpathlineto{\pgfqpoint{3.316194in}{3.364836in}}%
\pgfpathlineto{\pgfqpoint{3.317069in}{3.392710in}}%
\pgfpathlineto{\pgfqpoint{3.317945in}{3.504629in}}%
\pgfpathlineto{\pgfqpoint{3.318820in}{3.446226in}}%
\pgfpathlineto{\pgfqpoint{3.321446in}{2.874722in}}%
\pgfpathlineto{\pgfqpoint{3.322321in}{2.920686in}}%
\pgfpathlineto{\pgfqpoint{3.323197in}{2.783006in}}%
\pgfpathlineto{\pgfqpoint{3.324072in}{2.812065in}}%
\pgfpathlineto{\pgfqpoint{3.325823in}{2.347341in}}%
\pgfpathlineto{\pgfqpoint{3.326698in}{2.324535in}}%
\pgfpathlineto{\pgfqpoint{3.328449in}{2.127893in}}%
\pgfpathlineto{\pgfqpoint{3.329324in}{2.087597in}}%
\pgfpathlineto{\pgfqpoint{3.330200in}{2.008819in}}%
\pgfpathlineto{\pgfqpoint{3.331075in}{2.077025in}}%
\pgfpathlineto{\pgfqpoint{3.331951in}{2.283508in}}%
\pgfpathlineto{\pgfqpoint{3.334577in}{3.113969in}}%
\pgfpathlineto{\pgfqpoint{3.335452in}{3.239416in}}%
\pgfpathlineto{\pgfqpoint{3.336327in}{3.177879in}}%
\pgfpathlineto{\pgfqpoint{3.337203in}{3.400195in}}%
\pgfpathlineto{\pgfqpoint{3.338078in}{3.349469in}}%
\pgfpathlineto{\pgfqpoint{3.338954in}{3.471544in}}%
\pgfpathlineto{\pgfqpoint{3.339829in}{3.299429in}}%
\pgfpathlineto{\pgfqpoint{3.340704in}{3.198394in}}%
\pgfpathlineto{\pgfqpoint{3.343330in}{2.701177in}}%
\pgfpathlineto{\pgfqpoint{3.344206in}{2.682661in}}%
\pgfpathlineto{\pgfqpoint{3.345081in}{2.628350in}}%
\pgfpathlineto{\pgfqpoint{3.346832in}{2.068779in}}%
\pgfpathlineto{\pgfqpoint{3.347707in}{1.855432in}}%
\pgfpathlineto{\pgfqpoint{3.348583in}{1.721920in}}%
\pgfpathlineto{\pgfqpoint{3.351209in}{1.552191in}}%
\pgfpathlineto{\pgfqpoint{3.352084in}{1.625593in}}%
\pgfpathlineto{\pgfqpoint{3.355586in}{2.625228in}}%
\pgfpathlineto{\pgfqpoint{3.356461in}{2.627165in}}%
\pgfpathlineto{\pgfqpoint{3.357336in}{2.858483in}}%
\pgfpathlineto{\pgfqpoint{3.358212in}{2.850787in}}%
\pgfpathlineto{\pgfqpoint{3.359963in}{3.087614in}}%
\pgfpathlineto{\pgfqpoint{3.360838in}{3.144002in}}%
\pgfpathlineto{\pgfqpoint{3.361713in}{3.093002in}}%
\pgfpathlineto{\pgfqpoint{3.362589in}{3.014807in}}%
\pgfpathlineto{\pgfqpoint{3.363464in}{2.895061in}}%
\pgfpathlineto{\pgfqpoint{3.364339in}{2.874773in}}%
\pgfpathlineto{\pgfqpoint{3.365215in}{2.731746in}}%
\pgfpathlineto{\pgfqpoint{3.366090in}{2.540994in}}%
\pgfpathlineto{\pgfqpoint{3.367841in}{2.081012in}}%
\pgfpathlineto{\pgfqpoint{3.369592in}{1.691986in}}%
\pgfpathlineto{\pgfqpoint{3.370467in}{1.619612in}}%
\pgfpathlineto{\pgfqpoint{3.371342in}{1.507033in}}%
\pgfpathlineto{\pgfqpoint{3.373093in}{1.729381in}}%
\pgfpathlineto{\pgfqpoint{3.373969in}{2.028027in}}%
\pgfpathlineto{\pgfqpoint{3.374844in}{2.227458in}}%
\pgfpathlineto{\pgfqpoint{3.376595in}{2.800569in}}%
\pgfpathlineto{\pgfqpoint{3.377470in}{2.849903in}}%
\pgfpathlineto{\pgfqpoint{3.378345in}{3.063113in}}%
\pgfpathlineto{\pgfqpoint{3.379221in}{3.170864in}}%
\pgfpathlineto{\pgfqpoint{3.380096in}{3.144978in}}%
\pgfpathlineto{\pgfqpoint{3.380972in}{3.389418in}}%
\pgfpathlineto{\pgfqpoint{3.381847in}{3.357852in}}%
\pgfpathlineto{\pgfqpoint{3.382722in}{3.264820in}}%
\pgfpathlineto{\pgfqpoint{3.383598in}{3.266776in}}%
\pgfpathlineto{\pgfqpoint{3.384473in}{3.231855in}}%
\pgfpathlineto{\pgfqpoint{3.385348in}{3.117359in}}%
\pgfpathlineto{\pgfqpoint{3.386224in}{3.057581in}}%
\pgfpathlineto{\pgfqpoint{3.387099in}{2.865530in}}%
\pgfpathlineto{\pgfqpoint{3.387975in}{2.538637in}}%
\pgfpathlineto{\pgfqpoint{3.388850in}{2.413644in}}%
\pgfpathlineto{\pgfqpoint{3.390601in}{2.031806in}}%
\pgfpathlineto{\pgfqpoint{3.392352in}{1.805139in}}%
\pgfpathlineto{\pgfqpoint{3.393227in}{1.731375in}}%
\pgfpathlineto{\pgfqpoint{3.394102in}{1.718718in}}%
\pgfpathlineto{\pgfqpoint{3.394978in}{1.924114in}}%
\pgfpathlineto{\pgfqpoint{3.395853in}{2.390120in}}%
\pgfpathlineto{\pgfqpoint{3.396728in}{2.688233in}}%
\pgfpathlineto{\pgfqpoint{3.397604in}{2.725419in}}%
\pgfpathlineto{\pgfqpoint{3.401981in}{3.434050in}}%
\pgfpathlineto{\pgfqpoint{3.402856in}{3.631053in}}%
\pgfpathlineto{\pgfqpoint{3.403731in}{3.588965in}}%
\pgfpathlineto{\pgfqpoint{3.404607in}{3.507529in}}%
\pgfpathlineto{\pgfqpoint{3.406358in}{3.183692in}}%
\pgfpathlineto{\pgfqpoint{3.407233in}{3.216043in}}%
\pgfpathlineto{\pgfqpoint{3.408108in}{3.133580in}}%
\pgfpathlineto{\pgfqpoint{3.410734in}{2.571230in}}%
\pgfpathlineto{\pgfqpoint{3.412485in}{2.344985in}}%
\pgfpathlineto{\pgfqpoint{3.413361in}{2.316198in}}%
\pgfpathlineto{\pgfqpoint{3.415111in}{2.241166in}}%
\pgfpathlineto{\pgfqpoint{3.415987in}{2.341984in}}%
\pgfpathlineto{\pgfqpoint{3.416862in}{2.652933in}}%
\pgfpathlineto{\pgfqpoint{3.417737in}{2.762194in}}%
\pgfpathlineto{\pgfqpoint{3.418613in}{3.262996in}}%
\pgfpathlineto{\pgfqpoint{3.419488in}{3.294599in}}%
\pgfpathlineto{\pgfqpoint{3.420364in}{3.433727in}}%
\pgfpathlineto{\pgfqpoint{3.421239in}{3.498044in}}%
\pgfpathlineto{\pgfqpoint{3.422114in}{3.452797in}}%
\pgfpathlineto{\pgfqpoint{3.422990in}{3.549778in}}%
\pgfpathlineto{\pgfqpoint{3.423865in}{3.876829in}}%
\pgfpathlineto{\pgfqpoint{3.424740in}{3.899418in}}%
\pgfpathlineto{\pgfqpoint{3.426491in}{3.618769in}}%
\pgfpathlineto{\pgfqpoint{3.427367in}{3.739125in}}%
\pgfpathlineto{\pgfqpoint{3.428242in}{3.791684in}}%
\pgfpathlineto{\pgfqpoint{3.429117in}{3.752023in}}%
\pgfpathlineto{\pgfqpoint{3.430868in}{3.482402in}}%
\pgfpathlineto{\pgfqpoint{3.431743in}{3.282014in}}%
\pgfpathlineto{\pgfqpoint{3.432619in}{3.215802in}}%
\pgfpathlineto{\pgfqpoint{3.433494in}{3.095641in}}%
\pgfpathlineto{\pgfqpoint{3.434370in}{3.046616in}}%
\pgfpathlineto{\pgfqpoint{3.435245in}{2.960861in}}%
\pgfpathlineto{\pgfqpoint{3.436120in}{2.903952in}}%
\pgfpathlineto{\pgfqpoint{3.436996in}{2.981401in}}%
\pgfpathlineto{\pgfqpoint{3.437871in}{3.199688in}}%
\pgfpathlineto{\pgfqpoint{3.439622in}{3.437955in}}%
\pgfpathlineto{\pgfqpoint{3.440497in}{3.712133in}}%
\pgfpathlineto{\pgfqpoint{3.441373in}{3.825019in}}%
\pgfpathlineto{\pgfqpoint{3.442248in}{3.794291in}}%
\pgfpathlineto{\pgfqpoint{3.444874in}{4.185233in}}%
\pgfpathlineto{\pgfqpoint{3.445749in}{3.914189in}}%
\pgfpathlineto{\pgfqpoint{3.448376in}{3.598636in}}%
\pgfpathlineto{\pgfqpoint{3.449251in}{3.630383in}}%
\pgfpathlineto{\pgfqpoint{3.450126in}{3.642677in}}%
\pgfpathlineto{\pgfqpoint{3.451002in}{3.293583in}}%
\pgfpathlineto{\pgfqpoint{3.451877in}{3.089026in}}%
\pgfpathlineto{\pgfqpoint{3.454503in}{2.755881in}}%
\pgfpathlineto{\pgfqpoint{3.455379in}{2.652424in}}%
\pgfpathlineto{\pgfqpoint{3.456254in}{2.651246in}}%
\pgfpathlineto{\pgfqpoint{3.457129in}{2.657378in}}%
\pgfpathlineto{\pgfqpoint{3.458005in}{2.845180in}}%
\pgfpathlineto{\pgfqpoint{3.458880in}{3.094993in}}%
\pgfpathlineto{\pgfqpoint{3.459755in}{3.433825in}}%
\pgfpathlineto{\pgfqpoint{3.460631in}{3.621860in}}%
\pgfpathlineto{\pgfqpoint{3.461506in}{3.734153in}}%
\pgfpathlineto{\pgfqpoint{3.462382in}{3.785422in}}%
\pgfpathlineto{\pgfqpoint{3.464132in}{4.349083in}}%
\pgfpathlineto{\pgfqpoint{3.465008in}{4.399287in}}%
\pgfpathlineto{\pgfqpoint{3.465883in}{4.414606in}}%
\pgfpathlineto{\pgfqpoint{3.466758in}{4.269492in}}%
\pgfpathlineto{\pgfqpoint{3.468509in}{3.920695in}}%
\pgfpathlineto{\pgfqpoint{3.469385in}{3.863273in}}%
\pgfpathlineto{\pgfqpoint{3.470260in}{3.749395in}}%
\pgfpathlineto{\pgfqpoint{3.471135in}{3.459022in}}%
\pgfpathlineto{\pgfqpoint{3.472011in}{3.300862in}}%
\pgfpathlineto{\pgfqpoint{3.472886in}{3.029671in}}%
\pgfpathlineto{\pgfqpoint{3.474637in}{2.620980in}}%
\pgfpathlineto{\pgfqpoint{3.475512in}{2.576244in}}%
\pgfpathlineto{\pgfqpoint{3.476388in}{2.494234in}}%
\pgfpathlineto{\pgfqpoint{3.477263in}{2.467260in}}%
\pgfpathlineto{\pgfqpoint{3.478138in}{2.579537in}}%
\pgfpathlineto{\pgfqpoint{3.479889in}{3.133373in}}%
\pgfpathlineto{\pgfqpoint{3.480764in}{3.424032in}}%
\pgfpathlineto{\pgfqpoint{3.482515in}{3.777409in}}%
\pgfpathlineto{\pgfqpoint{3.483391in}{3.893276in}}%
\pgfpathlineto{\pgfqpoint{3.484266in}{3.875523in}}%
\pgfpathlineto{\pgfqpoint{3.486017in}{3.950917in}}%
\pgfpathlineto{\pgfqpoint{3.490394in}{3.324272in}}%
\pgfpathlineto{\pgfqpoint{3.491269in}{3.473824in}}%
\pgfpathlineto{\pgfqpoint{3.492144in}{3.342336in}}%
\pgfpathlineto{\pgfqpoint{3.494770in}{2.521722in}}%
\pgfpathlineto{\pgfqpoint{3.495646in}{2.370207in}}%
\pgfpathlineto{\pgfqpoint{3.496521in}{2.289194in}}%
\pgfpathlineto{\pgfqpoint{3.497397in}{2.312725in}}%
\pgfpathlineto{\pgfqpoint{3.498272in}{2.255605in}}%
\pgfpathlineto{\pgfqpoint{3.499147in}{2.260259in}}%
\pgfpathlineto{\pgfqpoint{3.500023in}{2.383651in}}%
\pgfpathlineto{\pgfqpoint{3.500898in}{2.727976in}}%
\pgfpathlineto{\pgfqpoint{3.503524in}{3.090199in}}%
\pgfpathlineto{\pgfqpoint{3.504400in}{3.310860in}}%
\pgfpathlineto{\pgfqpoint{3.506150in}{3.476474in}}%
\pgfpathlineto{\pgfqpoint{3.507026in}{3.458785in}}%
\pgfpathlineto{\pgfqpoint{3.507901in}{3.371334in}}%
\pgfpathlineto{\pgfqpoint{3.509652in}{2.681251in}}%
\pgfpathlineto{\pgfqpoint{3.510527in}{2.809039in}}%
\pgfpathlineto{\pgfqpoint{3.511403in}{2.722926in}}%
\pgfpathlineto{\pgfqpoint{3.512278in}{2.795512in}}%
\pgfpathlineto{\pgfqpoint{3.514904in}{2.412315in}}%
\pgfpathlineto{\pgfqpoint{3.515779in}{2.210809in}}%
\pgfpathlineto{\pgfqpoint{3.516655in}{2.136834in}}%
\pgfpathlineto{\pgfqpoint{3.517530in}{2.087779in}}%
\pgfpathlineto{\pgfqpoint{3.518406in}{2.000724in}}%
\pgfpathlineto{\pgfqpoint{3.519281in}{1.849904in}}%
\pgfpathlineto{\pgfqpoint{3.520156in}{1.905999in}}%
\pgfpathlineto{\pgfqpoint{3.521032in}{2.068814in}}%
\pgfpathlineto{\pgfqpoint{3.521907in}{2.328339in}}%
\pgfpathlineto{\pgfqpoint{3.523658in}{2.704060in}}%
\pgfpathlineto{\pgfqpoint{3.524533in}{2.908120in}}%
\pgfpathlineto{\pgfqpoint{3.525409in}{3.014978in}}%
\pgfpathlineto{\pgfqpoint{3.526284in}{3.019582in}}%
\pgfpathlineto{\pgfqpoint{3.528035in}{3.307190in}}%
\pgfpathlineto{\pgfqpoint{3.528910in}{3.272283in}}%
\pgfpathlineto{\pgfqpoint{3.529786in}{3.299133in}}%
\pgfpathlineto{\pgfqpoint{3.531536in}{3.007391in}}%
\pgfpathlineto{\pgfqpoint{3.533287in}{2.784607in}}%
\pgfpathlineto{\pgfqpoint{3.535038in}{2.469405in}}%
\pgfpathlineto{\pgfqpoint{3.535913in}{2.273789in}}%
\pgfpathlineto{\pgfqpoint{3.536789in}{2.012928in}}%
\pgfpathlineto{\pgfqpoint{3.537664in}{1.840269in}}%
\pgfpathlineto{\pgfqpoint{3.539415in}{1.768800in}}%
\pgfpathlineto{\pgfqpoint{3.540290in}{1.873465in}}%
\pgfpathlineto{\pgfqpoint{3.541165in}{1.929403in}}%
\pgfpathlineto{\pgfqpoint{3.542916in}{2.437462in}}%
\pgfpathlineto{\pgfqpoint{3.543792in}{2.913677in}}%
\pgfpathlineto{\pgfqpoint{3.544667in}{3.033416in}}%
\pgfpathlineto{\pgfqpoint{3.546418in}{3.176065in}}%
\pgfpathlineto{\pgfqpoint{3.547293in}{3.342126in}}%
\pgfpathlineto{\pgfqpoint{3.548168in}{3.406184in}}%
\pgfpathlineto{\pgfqpoint{3.549044in}{3.608331in}}%
\pgfpathlineto{\pgfqpoint{3.549919in}{3.542190in}}%
\pgfpathlineto{\pgfqpoint{3.551670in}{3.250232in}}%
\pgfpathlineto{\pgfqpoint{3.552545in}{3.076218in}}%
\pgfpathlineto{\pgfqpoint{3.553421in}{2.994903in}}%
\pgfpathlineto{\pgfqpoint{3.554296in}{2.990010in}}%
\pgfpathlineto{\pgfqpoint{3.556047in}{2.775968in}}%
\pgfpathlineto{\pgfqpoint{3.556922in}{2.525135in}}%
\pgfpathlineto{\pgfqpoint{3.558673in}{2.298800in}}%
\pgfpathlineto{\pgfqpoint{3.559548in}{2.264727in}}%
\pgfpathlineto{\pgfqpoint{3.560424in}{2.163838in}}%
\pgfpathlineto{\pgfqpoint{3.561299in}{2.170846in}}%
\pgfpathlineto{\pgfqpoint{3.562174in}{2.237617in}}%
\pgfpathlineto{\pgfqpoint{3.564801in}{3.024519in}}%
\pgfpathlineto{\pgfqpoint{3.565676in}{3.104970in}}%
\pgfpathlineto{\pgfqpoint{3.566551in}{3.261544in}}%
\pgfpathlineto{\pgfqpoint{3.567427in}{3.322046in}}%
\pgfpathlineto{\pgfqpoint{3.568302in}{3.554218in}}%
\pgfpathlineto{\pgfqpoint{3.569177in}{3.626772in}}%
\pgfpathlineto{\pgfqpoint{3.570053in}{3.638608in}}%
\pgfpathlineto{\pgfqpoint{3.570928in}{3.600108in}}%
\pgfpathlineto{\pgfqpoint{3.571804in}{3.544113in}}%
\pgfpathlineto{\pgfqpoint{3.573554in}{2.973861in}}%
\pgfpathlineto{\pgfqpoint{3.574430in}{2.856860in}}%
\pgfpathlineto{\pgfqpoint{3.575305in}{2.855411in}}%
\pgfpathlineto{\pgfqpoint{3.576180in}{2.898394in}}%
\pgfpathlineto{\pgfqpoint{3.577056in}{2.826413in}}%
\pgfpathlineto{\pgfqpoint{3.577931in}{2.595576in}}%
\pgfpathlineto{\pgfqpoint{3.580557in}{2.203831in}}%
\pgfpathlineto{\pgfqpoint{3.581433in}{2.167704in}}%
\pgfpathlineto{\pgfqpoint{3.582308in}{2.087356in}}%
\pgfpathlineto{\pgfqpoint{3.583183in}{2.191635in}}%
\pgfpathlineto{\pgfqpoint{3.585810in}{2.790899in}}%
\pgfpathlineto{\pgfqpoint{3.586685in}{2.787645in}}%
\pgfpathlineto{\pgfqpoint{3.587560in}{2.979633in}}%
\pgfpathlineto{\pgfqpoint{3.588436in}{3.050078in}}%
\pgfpathlineto{\pgfqpoint{3.589311in}{3.085286in}}%
\pgfpathlineto{\pgfqpoint{3.590186in}{3.202059in}}%
\pgfpathlineto{\pgfqpoint{3.591062in}{3.185860in}}%
\pgfpathlineto{\pgfqpoint{3.591937in}{3.180565in}}%
\pgfpathlineto{\pgfqpoint{3.592813in}{3.123072in}}%
\pgfpathlineto{\pgfqpoint{3.593688in}{3.131678in}}%
\pgfpathlineto{\pgfqpoint{3.594563in}{3.009673in}}%
\pgfpathlineto{\pgfqpoint{3.595439in}{2.962613in}}%
\pgfpathlineto{\pgfqpoint{3.596314in}{3.146055in}}%
\pgfpathlineto{\pgfqpoint{3.597189in}{3.088120in}}%
\pgfpathlineto{\pgfqpoint{3.598065in}{3.103525in}}%
\pgfpathlineto{\pgfqpoint{3.598940in}{2.953279in}}%
\pgfpathlineto{\pgfqpoint{3.599816in}{2.695619in}}%
\pgfpathlineto{\pgfqpoint{3.601566in}{2.540873in}}%
\pgfpathlineto{\pgfqpoint{3.603317in}{2.477953in}}%
\pgfpathlineto{\pgfqpoint{3.604192in}{2.528982in}}%
\pgfpathlineto{\pgfqpoint{3.605943in}{3.024254in}}%
\pgfpathlineto{\pgfqpoint{3.606819in}{3.324179in}}%
\pgfpathlineto{\pgfqpoint{3.607694in}{3.355883in}}%
\pgfpathlineto{\pgfqpoint{3.608569in}{3.665226in}}%
\pgfpathlineto{\pgfqpoint{3.609445in}{3.531777in}}%
\pgfpathlineto{\pgfqpoint{3.610320in}{3.548526in}}%
\pgfpathlineto{\pgfqpoint{3.611195in}{3.625584in}}%
\pgfpathlineto{\pgfqpoint{3.612071in}{3.675911in}}%
\pgfpathlineto{\pgfqpoint{3.613822in}{3.501655in}}%
\pgfpathlineto{\pgfqpoint{3.615572in}{3.052665in}}%
\pgfpathlineto{\pgfqpoint{3.616448in}{2.930564in}}%
\pgfpathlineto{\pgfqpoint{3.617323in}{3.170009in}}%
\pgfpathlineto{\pgfqpoint{3.618198in}{3.115879in}}%
\pgfpathlineto{\pgfqpoint{3.619949in}{2.875286in}}%
\pgfpathlineto{\pgfqpoint{3.620825in}{2.745430in}}%
\pgfpathlineto{\pgfqpoint{3.621700in}{2.733710in}}%
\pgfpathlineto{\pgfqpoint{3.622575in}{2.718032in}}%
\pgfpathlineto{\pgfqpoint{3.623451in}{2.722412in}}%
\pgfpathlineto{\pgfqpoint{3.624326in}{2.635962in}}%
\pgfpathlineto{\pgfqpoint{3.625201in}{2.402129in}}%
\pgfpathlineto{\pgfqpoint{3.626077in}{2.591387in}}%
\pgfpathlineto{\pgfqpoint{3.626952in}{2.686402in}}%
\pgfpathlineto{\pgfqpoint{3.627828in}{2.851317in}}%
\pgfpathlineto{\pgfqpoint{3.628703in}{2.945393in}}%
\pgfpathlineto{\pgfqpoint{3.629578in}{3.102176in}}%
\pgfpathlineto{\pgfqpoint{3.630454in}{3.149801in}}%
\pgfpathlineto{\pgfqpoint{3.631329in}{3.165962in}}%
\pgfpathlineto{\pgfqpoint{3.632204in}{3.229302in}}%
\pgfpathlineto{\pgfqpoint{3.633080in}{3.267055in}}%
\pgfpathlineto{\pgfqpoint{3.633955in}{3.263959in}}%
\pgfpathlineto{\pgfqpoint{3.634831in}{3.058639in}}%
\pgfpathlineto{\pgfqpoint{3.635706in}{3.025185in}}%
\pgfpathlineto{\pgfqpoint{3.636581in}{2.829750in}}%
\pgfpathlineto{\pgfqpoint{3.637457in}{2.875951in}}%
\pgfpathlineto{\pgfqpoint{3.638332in}{2.891084in}}%
\pgfpathlineto{\pgfqpoint{3.639207in}{2.834296in}}%
\pgfpathlineto{\pgfqpoint{3.640958in}{2.666652in}}%
\pgfpathlineto{\pgfqpoint{3.642709in}{2.332933in}}%
\pgfpathlineto{\pgfqpoint{3.643584in}{2.317920in}}%
\pgfpathlineto{\pgfqpoint{3.646210in}{2.032723in}}%
\pgfpathlineto{\pgfqpoint{3.647961in}{2.249167in}}%
\pgfpathlineto{\pgfqpoint{3.648837in}{2.463867in}}%
\pgfpathlineto{\pgfqpoint{3.649712in}{2.523161in}}%
\pgfpathlineto{\pgfqpoint{3.650587in}{2.690699in}}%
\pgfpathlineto{\pgfqpoint{3.651463in}{2.725482in}}%
\pgfpathlineto{\pgfqpoint{3.652338in}{2.927310in}}%
\pgfpathlineto{\pgfqpoint{3.653213in}{3.009565in}}%
\pgfpathlineto{\pgfqpoint{3.654089in}{2.854399in}}%
\pgfpathlineto{\pgfqpoint{3.654964in}{2.851147in}}%
\pgfpathlineto{\pgfqpoint{3.655840in}{2.810115in}}%
\pgfpathlineto{\pgfqpoint{3.657590in}{2.505559in}}%
\pgfpathlineto{\pgfqpoint{3.658466in}{2.522725in}}%
\pgfpathlineto{\pgfqpoint{3.659341in}{2.474479in}}%
\pgfpathlineto{\pgfqpoint{3.664593in}{1.551889in}}%
\pgfpathlineto{\pgfqpoint{3.665469in}{1.581250in}}%
\pgfpathlineto{\pgfqpoint{3.666344in}{1.565452in}}%
\pgfpathlineto{\pgfqpoint{3.667220in}{1.531405in}}%
\pgfpathlineto{\pgfqpoint{3.668970in}{1.877633in}}%
\pgfpathlineto{\pgfqpoint{3.669846in}{2.369219in}}%
\pgfpathlineto{\pgfqpoint{3.671596in}{2.553514in}}%
\pgfpathlineto{\pgfqpoint{3.672472in}{2.723233in}}%
\pgfpathlineto{\pgfqpoint{3.673347in}{2.652260in}}%
\pgfpathlineto{\pgfqpoint{3.674223in}{2.836883in}}%
\pgfpathlineto{\pgfqpoint{3.675098in}{2.827431in}}%
\pgfpathlineto{\pgfqpoint{3.675973in}{2.968795in}}%
\pgfpathlineto{\pgfqpoint{3.676849in}{2.616384in}}%
\pgfpathlineto{\pgfqpoint{3.677724in}{2.555206in}}%
\pgfpathlineto{\pgfqpoint{3.678599in}{2.299508in}}%
\pgfpathlineto{\pgfqpoint{3.679475in}{2.280434in}}%
\pgfpathlineto{\pgfqpoint{3.680350in}{2.365163in}}%
\pgfpathlineto{\pgfqpoint{3.682101in}{2.154232in}}%
\pgfpathlineto{\pgfqpoint{3.683852in}{1.824259in}}%
\pgfpathlineto{\pgfqpoint{3.684727in}{1.701531in}}%
\pgfpathlineto{\pgfqpoint{3.687353in}{1.563639in}}%
\pgfpathlineto{\pgfqpoint{3.689104in}{1.910878in}}%
\pgfpathlineto{\pgfqpoint{3.689979in}{2.124065in}}%
\pgfpathlineto{\pgfqpoint{3.690855in}{2.411298in}}%
\pgfpathlineto{\pgfqpoint{3.691730in}{2.413650in}}%
\pgfpathlineto{\pgfqpoint{3.693481in}{2.718952in}}%
\pgfpathlineto{\pgfqpoint{3.694356in}{2.733124in}}%
\pgfpathlineto{\pgfqpoint{3.695232in}{2.828674in}}%
\pgfpathlineto{\pgfqpoint{3.696107in}{2.872754in}}%
\pgfpathlineto{\pgfqpoint{3.696982in}{2.875518in}}%
\pgfpathlineto{\pgfqpoint{3.697858in}{2.714906in}}%
\pgfpathlineto{\pgfqpoint{3.699608in}{2.259049in}}%
\pgfpathlineto{\pgfqpoint{3.700484in}{2.250953in}}%
\pgfpathlineto{\pgfqpoint{3.701359in}{2.389388in}}%
\pgfpathlineto{\pgfqpoint{3.703985in}{2.186432in}}%
\pgfpathlineto{\pgfqpoint{3.704861in}{2.089893in}}%
\pgfpathlineto{\pgfqpoint{3.705736in}{1.915633in}}%
\pgfpathlineto{\pgfqpoint{3.707487in}{1.843621in}}%
\pgfpathlineto{\pgfqpoint{3.708362in}{1.860235in}}%
\pgfpathlineto{\pgfqpoint{3.709238in}{1.857185in}}%
\pgfpathlineto{\pgfqpoint{3.710988in}{2.373483in}}%
\pgfpathlineto{\pgfqpoint{3.711864in}{2.475090in}}%
\pgfpathlineto{\pgfqpoint{3.713614in}{2.762389in}}%
\pgfpathlineto{\pgfqpoint{3.714490in}{2.766606in}}%
\pgfpathlineto{\pgfqpoint{3.715365in}{2.862901in}}%
\pgfpathlineto{\pgfqpoint{3.716241in}{2.907627in}}%
\pgfpathlineto{\pgfqpoint{3.717116in}{3.090629in}}%
\pgfpathlineto{\pgfqpoint{3.717991in}{3.136740in}}%
\pgfpathlineto{\pgfqpoint{3.718867in}{3.043928in}}%
\pgfpathlineto{\pgfqpoint{3.720617in}{2.683942in}}%
\pgfpathlineto{\pgfqpoint{3.721493in}{2.667574in}}%
\pgfpathlineto{\pgfqpoint{3.722368in}{2.705678in}}%
\pgfpathlineto{\pgfqpoint{3.723244in}{2.567545in}}%
\pgfpathlineto{\pgfqpoint{3.724119in}{2.664718in}}%
\pgfpathlineto{\pgfqpoint{3.727620in}{2.062375in}}%
\pgfpathlineto{\pgfqpoint{3.728496in}{2.072978in}}%
\pgfpathlineto{\pgfqpoint{3.729371in}{2.100828in}}%
\pgfpathlineto{\pgfqpoint{3.730247in}{2.162971in}}%
\pgfpathlineto{\pgfqpoint{3.731122in}{2.175755in}}%
\pgfpathlineto{\pgfqpoint{3.732873in}{2.715905in}}%
\pgfpathlineto{\pgfqpoint{3.733748in}{2.750697in}}%
\pgfpathlineto{\pgfqpoint{3.734623in}{2.743572in}}%
\pgfpathlineto{\pgfqpoint{3.735499in}{2.786117in}}%
\pgfpathlineto{\pgfqpoint{3.736374in}{2.792582in}}%
\pgfpathlineto{\pgfqpoint{3.737250in}{2.957819in}}%
\pgfpathlineto{\pgfqpoint{3.739000in}{2.871179in}}%
\pgfpathlineto{\pgfqpoint{3.739876in}{2.735781in}}%
\pgfpathlineto{\pgfqpoint{3.740751in}{2.752749in}}%
\pgfpathlineto{\pgfqpoint{3.741626in}{2.628502in}}%
\pgfpathlineto{\pgfqpoint{3.742502in}{2.603814in}}%
\pgfpathlineto{\pgfqpoint{3.743377in}{2.683144in}}%
\pgfpathlineto{\pgfqpoint{3.744253in}{2.589807in}}%
\pgfpathlineto{\pgfqpoint{3.745128in}{2.541175in}}%
\pgfpathlineto{\pgfqpoint{3.746003in}{2.388029in}}%
\pgfpathlineto{\pgfqpoint{3.749505in}{2.093639in}}%
\pgfpathlineto{\pgfqpoint{3.750380in}{2.065879in}}%
\pgfpathlineto{\pgfqpoint{3.752131in}{2.265583in}}%
\pgfpathlineto{\pgfqpoint{3.753006in}{2.449660in}}%
\pgfpathlineto{\pgfqpoint{3.753882in}{2.765774in}}%
\pgfpathlineto{\pgfqpoint{3.754757in}{2.888836in}}%
\pgfpathlineto{\pgfqpoint{3.755632in}{2.936472in}}%
\pgfpathlineto{\pgfqpoint{3.756508in}{2.890292in}}%
\pgfpathlineto{\pgfqpoint{3.758259in}{3.076320in}}%
\pgfpathlineto{\pgfqpoint{3.759134in}{2.946374in}}%
\pgfpathlineto{\pgfqpoint{3.760009in}{3.055600in}}%
\pgfpathlineto{\pgfqpoint{3.762635in}{2.702590in}}%
\pgfpathlineto{\pgfqpoint{3.764386in}{2.867614in}}%
\pgfpathlineto{\pgfqpoint{3.765262in}{2.885768in}}%
\pgfpathlineto{\pgfqpoint{3.766137in}{2.848524in}}%
\pgfpathlineto{\pgfqpoint{3.767012in}{2.826835in}}%
\pgfpathlineto{\pgfqpoint{3.767888in}{2.740325in}}%
\pgfpathlineto{\pgfqpoint{3.769638in}{2.527401in}}%
\pgfpathlineto{\pgfqpoint{3.770514in}{2.517282in}}%
\pgfpathlineto{\pgfqpoint{3.771389in}{2.503810in}}%
\pgfpathlineto{\pgfqpoint{3.772265in}{2.594831in}}%
\pgfpathlineto{\pgfqpoint{3.773140in}{2.764003in}}%
\pgfpathlineto{\pgfqpoint{3.774015in}{2.727586in}}%
\pgfpathlineto{\pgfqpoint{3.775766in}{2.943641in}}%
\pgfpathlineto{\pgfqpoint{3.776641in}{2.917277in}}%
\pgfpathlineto{\pgfqpoint{3.777517in}{2.774821in}}%
\pgfpathlineto{\pgfqpoint{3.778392in}{2.767678in}}%
\pgfpathlineto{\pgfqpoint{3.779268in}{2.882446in}}%
\pgfpathlineto{\pgfqpoint{3.780143in}{2.776939in}}%
\pgfpathlineto{\pgfqpoint{3.781018in}{2.919423in}}%
\pgfpathlineto{\pgfqpoint{3.781894in}{2.921778in}}%
\pgfpathlineto{\pgfqpoint{3.782769in}{2.893404in}}%
\pgfpathlineto{\pgfqpoint{3.783644in}{2.749912in}}%
\pgfpathlineto{\pgfqpoint{3.784520in}{2.781740in}}%
\pgfpathlineto{\pgfqpoint{3.785395in}{2.838858in}}%
\pgfpathlineto{\pgfqpoint{3.786271in}{2.862932in}}%
\pgfpathlineto{\pgfqpoint{3.787146in}{2.902079in}}%
\pgfpathlineto{\pgfqpoint{3.789772in}{2.528126in}}%
\pgfpathlineto{\pgfqpoint{3.791523in}{2.477107in}}%
\pgfpathlineto{\pgfqpoint{3.792398in}{2.501091in}}%
\pgfpathlineto{\pgfqpoint{3.793274in}{2.585363in}}%
\pgfpathlineto{\pgfqpoint{3.794149in}{2.840666in}}%
\pgfpathlineto{\pgfqpoint{3.795024in}{2.873790in}}%
\pgfpathlineto{\pgfqpoint{3.795900in}{2.876191in}}%
\pgfpathlineto{\pgfqpoint{3.796775in}{2.956560in}}%
\pgfpathlineto{\pgfqpoint{3.797651in}{2.995692in}}%
\pgfpathlineto{\pgfqpoint{3.798526in}{3.016304in}}%
\pgfpathlineto{\pgfqpoint{3.799401in}{3.159938in}}%
\pgfpathlineto{\pgfqpoint{3.800277in}{3.145635in}}%
\pgfpathlineto{\pgfqpoint{3.802027in}{3.281585in}}%
\pgfpathlineto{\pgfqpoint{3.802903in}{3.225582in}}%
\pgfpathlineto{\pgfqpoint{3.803778in}{3.009032in}}%
\pgfpathlineto{\pgfqpoint{3.804654in}{2.886725in}}%
\pgfpathlineto{\pgfqpoint{3.806404in}{3.058064in}}%
\pgfpathlineto{\pgfqpoint{3.807280in}{3.083891in}}%
\pgfpathlineto{\pgfqpoint{3.808155in}{3.051812in}}%
\pgfpathlineto{\pgfqpoint{3.809030in}{2.944761in}}%
\pgfpathlineto{\pgfqpoint{3.809906in}{2.770289in}}%
\pgfpathlineto{\pgfqpoint{3.810781in}{2.665776in}}%
\pgfpathlineto{\pgfqpoint{3.811657in}{2.619620in}}%
\pgfpathlineto{\pgfqpoint{3.812532in}{2.639496in}}%
\pgfpathlineto{\pgfqpoint{3.813407in}{2.610589in}}%
\pgfpathlineto{\pgfqpoint{3.814283in}{2.506445in}}%
\pgfpathlineto{\pgfqpoint{3.815158in}{2.708186in}}%
\pgfpathlineto{\pgfqpoint{3.816033in}{2.716864in}}%
\pgfpathlineto{\pgfqpoint{3.817784in}{2.922077in}}%
\pgfpathlineto{\pgfqpoint{3.818660in}{2.964556in}}%
\pgfpathlineto{\pgfqpoint{3.819535in}{3.055542in}}%
\pgfpathlineto{\pgfqpoint{3.820410in}{3.039574in}}%
\pgfpathlineto{\pgfqpoint{3.821286in}{2.940557in}}%
\pgfpathlineto{\pgfqpoint{3.822161in}{3.005181in}}%
\pgfpathlineto{\pgfqpoint{3.823036in}{3.101964in}}%
\pgfpathlineto{\pgfqpoint{3.823912in}{3.133315in}}%
\pgfpathlineto{\pgfqpoint{3.826538in}{2.774126in}}%
\pgfpathlineto{\pgfqpoint{3.827413in}{2.861663in}}%
\pgfpathlineto{\pgfqpoint{3.828289in}{2.819012in}}%
\pgfpathlineto{\pgfqpoint{3.829164in}{2.903439in}}%
\pgfpathlineto{\pgfqpoint{3.830915in}{2.683235in}}%
\pgfpathlineto{\pgfqpoint{3.831790in}{2.611827in}}%
\pgfpathlineto{\pgfqpoint{3.832666in}{2.563105in}}%
\pgfpathlineto{\pgfqpoint{3.833541in}{2.464390in}}%
\pgfpathlineto{\pgfqpoint{3.834416in}{2.483179in}}%
\pgfpathlineto{\pgfqpoint{3.835292in}{2.700159in}}%
\pgfpathlineto{\pgfqpoint{3.837042in}{3.283798in}}%
\pgfpathlineto{\pgfqpoint{3.837918in}{3.365954in}}%
\pgfpathlineto{\pgfqpoint{3.838793in}{3.398288in}}%
\pgfpathlineto{\pgfqpoint{3.839669in}{3.354237in}}%
\pgfpathlineto{\pgfqpoint{3.840544in}{3.444196in}}%
\pgfpathlineto{\pgfqpoint{3.841419in}{3.488827in}}%
\pgfpathlineto{\pgfqpoint{3.842295in}{3.443011in}}%
\pgfpathlineto{\pgfqpoint{3.843170in}{3.544266in}}%
\pgfpathlineto{\pgfqpoint{3.844045in}{3.433721in}}%
\pgfpathlineto{\pgfqpoint{3.844921in}{3.431541in}}%
\pgfpathlineto{\pgfqpoint{3.845796in}{3.248732in}}%
\pgfpathlineto{\pgfqpoint{3.846672in}{3.273406in}}%
\pgfpathlineto{\pgfqpoint{3.847547in}{3.352394in}}%
\pgfpathlineto{\pgfqpoint{3.848422in}{3.474730in}}%
\pgfpathlineto{\pgfqpoint{3.849298in}{3.300409in}}%
\pgfpathlineto{\pgfqpoint{3.850173in}{3.218732in}}%
\pgfpathlineto{\pgfqpoint{3.851048in}{3.001911in}}%
\pgfpathlineto{\pgfqpoint{3.851924in}{2.983787in}}%
\pgfpathlineto{\pgfqpoint{3.852799in}{3.015866in}}%
\pgfpathlineto{\pgfqpoint{3.853675in}{2.954094in}}%
\pgfpathlineto{\pgfqpoint{3.854550in}{2.920717in}}%
\pgfpathlineto{\pgfqpoint{3.855425in}{2.870514in}}%
\pgfpathlineto{\pgfqpoint{3.856301in}{3.018278in}}%
\pgfpathlineto{\pgfqpoint{3.857176in}{3.353030in}}%
\pgfpathlineto{\pgfqpoint{3.858051in}{3.505862in}}%
\pgfpathlineto{\pgfqpoint{3.858927in}{3.596131in}}%
\pgfpathlineto{\pgfqpoint{3.860678in}{3.433236in}}%
\pgfpathlineto{\pgfqpoint{3.861553in}{3.672752in}}%
\pgfpathlineto{\pgfqpoint{3.863304in}{3.763064in}}%
\pgfpathlineto{\pgfqpoint{3.864179in}{3.177439in}}%
\pgfpathlineto{\pgfqpoint{3.865930in}{2.984416in}}%
\pgfpathlineto{\pgfqpoint{3.866805in}{2.933484in}}%
\pgfpathlineto{\pgfqpoint{3.867681in}{2.758963in}}%
\pgfpathlineto{\pgfqpoint{3.868556in}{2.710297in}}%
\pgfpathlineto{\pgfqpoint{3.869431in}{2.854535in}}%
\pgfpathlineto{\pgfqpoint{3.871182in}{2.693052in}}%
\pgfpathlineto{\pgfqpoint{3.872057in}{2.525165in}}%
\pgfpathlineto{\pgfqpoint{3.872933in}{2.287744in}}%
\pgfpathlineto{\pgfqpoint{3.873808in}{2.272188in}}%
\pgfpathlineto{\pgfqpoint{3.874684in}{2.291641in}}%
\pgfpathlineto{\pgfqpoint{3.875559in}{2.278471in}}%
\pgfpathlineto{\pgfqpoint{3.876434in}{2.250590in}}%
\pgfpathlineto{\pgfqpoint{3.877310in}{2.305576in}}%
\pgfpathlineto{\pgfqpoint{3.878185in}{2.428400in}}%
\pgfpathlineto{\pgfqpoint{3.879060in}{2.427185in}}%
\pgfpathlineto{\pgfqpoint{3.879936in}{2.713466in}}%
\pgfpathlineto{\pgfqpoint{3.880811in}{2.334767in}}%
\pgfpathlineto{\pgfqpoint{3.881687in}{2.415658in}}%
\pgfpathlineto{\pgfqpoint{3.883437in}{2.636792in}}%
\pgfpathlineto{\pgfqpoint{3.884313in}{2.737146in}}%
\pgfpathlineto{\pgfqpoint{3.885188in}{2.577818in}}%
\pgfpathlineto{\pgfqpoint{3.886063in}{2.584909in}}%
\pgfpathlineto{\pgfqpoint{3.886939in}{2.637539in}}%
\pgfpathlineto{\pgfqpoint{3.887814in}{2.178193in}}%
\pgfpathlineto{\pgfqpoint{3.888690in}{2.048994in}}%
\pgfpathlineto{\pgfqpoint{3.889565in}{2.034335in}}%
\pgfpathlineto{\pgfqpoint{3.890440in}{2.045973in}}%
\pgfpathlineto{\pgfqpoint{3.891316in}{2.035703in}}%
\pgfpathlineto{\pgfqpoint{3.892191in}{1.953270in}}%
\pgfpathlineto{\pgfqpoint{3.893942in}{1.857849in}}%
\pgfpathlineto{\pgfqpoint{3.894817in}{1.768952in}}%
\pgfpathlineto{\pgfqpoint{3.895693in}{1.763605in}}%
\pgfpathlineto{\pgfqpoint{3.897443in}{1.659273in}}%
\pgfpathlineto{\pgfqpoint{3.898319in}{1.686442in}}%
\pgfpathlineto{\pgfqpoint{3.899194in}{1.887919in}}%
\pgfpathlineto{\pgfqpoint{3.900069in}{1.891483in}}%
\pgfpathlineto{\pgfqpoint{3.900945in}{2.114882in}}%
\pgfpathlineto{\pgfqpoint{3.901820in}{2.229335in}}%
\pgfpathlineto{\pgfqpoint{3.902696in}{2.227906in}}%
\pgfpathlineto{\pgfqpoint{3.903571in}{2.106956in}}%
\pgfpathlineto{\pgfqpoint{3.904446in}{2.166408in}}%
\pgfpathlineto{\pgfqpoint{3.905322in}{2.289833in}}%
\pgfpathlineto{\pgfqpoint{3.906197in}{2.480165in}}%
\pgfpathlineto{\pgfqpoint{3.907072in}{2.509364in}}%
\pgfpathlineto{\pgfqpoint{3.907948in}{2.461877in}}%
\pgfpathlineto{\pgfqpoint{3.908823in}{2.106406in}}%
\pgfpathlineto{\pgfqpoint{3.909699in}{1.882110in}}%
\pgfpathlineto{\pgfqpoint{3.910574in}{1.863618in}}%
\pgfpathlineto{\pgfqpoint{3.911449in}{1.929015in}}%
\pgfpathlineto{\pgfqpoint{3.912325in}{1.801544in}}%
\pgfpathlineto{\pgfqpoint{3.913200in}{1.744968in}}%
\pgfpathlineto{\pgfqpoint{3.914075in}{1.728838in}}%
\pgfpathlineto{\pgfqpoint{3.918452in}{1.523858in}}%
\pgfpathlineto{\pgfqpoint{3.919328in}{1.574034in}}%
\pgfpathlineto{\pgfqpoint{3.921078in}{1.869771in}}%
\pgfpathlineto{\pgfqpoint{3.921954in}{2.029091in}}%
\pgfpathlineto{\pgfqpoint{3.922829in}{2.126436in}}%
\pgfpathlineto{\pgfqpoint{3.923705in}{2.270652in}}%
\pgfpathlineto{\pgfqpoint{3.924580in}{2.368461in}}%
\pgfpathlineto{\pgfqpoint{3.925455in}{2.528686in}}%
\pgfpathlineto{\pgfqpoint{3.926331in}{2.850126in}}%
\pgfpathlineto{\pgfqpoint{3.927206in}{2.898507in}}%
\pgfpathlineto{\pgfqpoint{3.928082in}{2.864524in}}%
\pgfpathlineto{\pgfqpoint{3.928957in}{2.643514in}}%
\pgfpathlineto{\pgfqpoint{3.929832in}{2.555844in}}%
\pgfpathlineto{\pgfqpoint{3.930708in}{2.295728in}}%
\pgfpathlineto{\pgfqpoint{3.931583in}{2.295326in}}%
\pgfpathlineto{\pgfqpoint{3.932458in}{2.366734in}}%
\pgfpathlineto{\pgfqpoint{3.934209in}{2.167251in}}%
\pgfpathlineto{\pgfqpoint{3.935085in}{2.011870in}}%
\pgfpathlineto{\pgfqpoint{3.935960in}{1.951156in}}%
\pgfpathlineto{\pgfqpoint{3.938586in}{1.847065in}}%
\pgfpathlineto{\pgfqpoint{3.939461in}{1.800064in}}%
\pgfpathlineto{\pgfqpoint{3.940337in}{1.720771in}}%
\pgfpathlineto{\pgfqpoint{3.942088in}{2.189537in}}%
\pgfpathlineto{\pgfqpoint{3.943838in}{2.353696in}}%
\pgfpathlineto{\pgfqpoint{3.945589in}{2.432784in}}%
\pgfpathlineto{\pgfqpoint{3.946464in}{2.630877in}}%
\pgfpathlineto{\pgfqpoint{3.947340in}{2.562016in}}%
\pgfpathlineto{\pgfqpoint{3.948215in}{2.675533in}}%
\pgfpathlineto{\pgfqpoint{3.949966in}{2.575983in}}%
\pgfpathlineto{\pgfqpoint{3.951717in}{2.410706in}}%
\pgfpathlineto{\pgfqpoint{3.952592in}{2.433324in}}%
\pgfpathlineto{\pgfqpoint{3.953467in}{2.496173in}}%
\pgfpathlineto{\pgfqpoint{3.955218in}{2.319755in}}%
\pgfpathlineto{\pgfqpoint{3.956094in}{2.177824in}}%
\pgfpathlineto{\pgfqpoint{3.956969in}{2.089380in}}%
\pgfpathlineto{\pgfqpoint{3.957844in}{1.889203in}}%
\pgfpathlineto{\pgfqpoint{3.958720in}{1.799007in}}%
\pgfpathlineto{\pgfqpoint{3.959595in}{1.760554in}}%
\pgfpathlineto{\pgfqpoint{3.960470in}{1.739682in}}%
\pgfpathlineto{\pgfqpoint{3.961346in}{1.680847in}}%
\pgfpathlineto{\pgfqpoint{3.963972in}{2.510453in}}%
\pgfpathlineto{\pgfqpoint{3.964847in}{2.515750in}}%
\pgfpathlineto{\pgfqpoint{3.965723in}{2.515863in}}%
\pgfpathlineto{\pgfqpoint{3.966598in}{2.533533in}}%
\pgfpathlineto{\pgfqpoint{3.967473in}{2.614519in}}%
\pgfpathlineto{\pgfqpoint{3.968349in}{2.654948in}}%
\pgfpathlineto{\pgfqpoint{3.969224in}{2.736828in}}%
\pgfpathlineto{\pgfqpoint{3.970975in}{2.652631in}}%
\pgfpathlineto{\pgfqpoint{3.971850in}{2.472989in}}%
\pgfpathlineto{\pgfqpoint{3.972726in}{2.355921in}}%
\pgfpathlineto{\pgfqpoint{3.974476in}{2.461219in}}%
\pgfpathlineto{\pgfqpoint{3.975352in}{2.545110in}}%
\pgfpathlineto{\pgfqpoint{3.976227in}{2.523129in}}%
\pgfpathlineto{\pgfqpoint{3.977103in}{2.420527in}}%
\pgfpathlineto{\pgfqpoint{3.977978in}{2.390299in}}%
\pgfpathlineto{\pgfqpoint{3.978853in}{2.307905in}}%
\pgfpathlineto{\pgfqpoint{3.979729in}{2.321325in}}%
\pgfpathlineto{\pgfqpoint{3.980604in}{2.284391in}}%
\pgfpathlineto{\pgfqpoint{3.981479in}{2.333183in}}%
\pgfpathlineto{\pgfqpoint{3.982355in}{2.399587in}}%
\pgfpathlineto{\pgfqpoint{3.984106in}{2.973018in}}%
\pgfpathlineto{\pgfqpoint{3.985856in}{3.310977in}}%
\pgfpathlineto{\pgfqpoint{3.986732in}{3.597117in}}%
\pgfpathlineto{\pgfqpoint{3.987607in}{3.576362in}}%
\pgfpathlineto{\pgfqpoint{3.989358in}{3.720812in}}%
\pgfpathlineto{\pgfqpoint{3.990233in}{3.889355in}}%
\pgfpathlineto{\pgfqpoint{3.991109in}{3.777422in}}%
\pgfpathlineto{\pgfqpoint{3.991984in}{3.744284in}}%
\pgfpathlineto{\pgfqpoint{3.993735in}{3.385943in}}%
\pgfpathlineto{\pgfqpoint{3.994610in}{3.375314in}}%
\pgfpathlineto{\pgfqpoint{3.995485in}{3.534448in}}%
\pgfpathlineto{\pgfqpoint{3.996361in}{3.498351in}}%
\pgfpathlineto{\pgfqpoint{3.997236in}{3.660438in}}%
\pgfpathlineto{\pgfqpoint{3.998112in}{3.566919in}}%
\pgfpathlineto{\pgfqpoint{3.999862in}{3.269629in}}%
\pgfpathlineto{\pgfqpoint{4.000738in}{3.173181in}}%
\pgfpathlineto{\pgfqpoint{4.001613in}{3.154996in}}%
\pgfpathlineto{\pgfqpoint{4.002488in}{3.204142in}}%
\pgfpathlineto{\pgfqpoint{4.003364in}{3.119172in}}%
\pgfpathlineto{\pgfqpoint{4.005990in}{3.818849in}}%
\pgfpathlineto{\pgfqpoint{4.006865in}{3.774555in}}%
\pgfpathlineto{\pgfqpoint{4.008616in}{3.862049in}}%
\pgfpathlineto{\pgfqpoint{4.009491in}{4.065783in}}%
\pgfpathlineto{\pgfqpoint{4.010367in}{4.067057in}}%
\pgfpathlineto{\pgfqpoint{4.011242in}{3.931821in}}%
\pgfpathlineto{\pgfqpoint{4.012118in}{4.006346in}}%
\pgfpathlineto{\pgfqpoint{4.012993in}{4.134537in}}%
\pgfpathlineto{\pgfqpoint{4.013868in}{3.882751in}}%
\pgfpathlineto{\pgfqpoint{4.015619in}{3.626840in}}%
\pgfpathlineto{\pgfqpoint{4.016494in}{3.581902in}}%
\pgfpathlineto{\pgfqpoint{4.018245in}{3.330646in}}%
\pgfpathlineto{\pgfqpoint{4.019121in}{3.165871in}}%
\pgfpathlineto{\pgfqpoint{4.019996in}{3.194083in}}%
\pgfpathlineto{\pgfqpoint{4.020871in}{3.056796in}}%
\pgfpathlineto{\pgfqpoint{4.022622in}{2.881086in}}%
\pgfpathlineto{\pgfqpoint{4.023497in}{2.854384in}}%
\pgfpathlineto{\pgfqpoint{4.024373in}{2.699334in}}%
\pgfpathlineto{\pgfqpoint{4.026124in}{3.092468in}}%
\pgfpathlineto{\pgfqpoint{4.026999in}{3.376599in}}%
\pgfpathlineto{\pgfqpoint{4.028750in}{3.602019in}}%
\pgfpathlineto{\pgfqpoint{4.029625in}{3.656684in}}%
\pgfpathlineto{\pgfqpoint{4.030500in}{3.613676in}}%
\pgfpathlineto{\pgfqpoint{4.031376in}{3.619716in}}%
\pgfpathlineto{\pgfqpoint{4.032251in}{3.736304in}}%
\pgfpathlineto{\pgfqpoint{4.033127in}{3.733061in}}%
\pgfpathlineto{\pgfqpoint{4.034877in}{3.320390in}}%
\pgfpathlineto{\pgfqpoint{4.035753in}{3.315865in}}%
\pgfpathlineto{\pgfqpoint{4.036628in}{3.386618in}}%
\pgfpathlineto{\pgfqpoint{4.037503in}{3.346655in}}%
\pgfpathlineto{\pgfqpoint{4.038379in}{3.140195in}}%
\pgfpathlineto{\pgfqpoint{4.039254in}{3.300711in}}%
\pgfpathlineto{\pgfqpoint{4.041880in}{2.974574in}}%
\pgfpathlineto{\pgfqpoint{4.044506in}{2.754884in}}%
\pgfpathlineto{\pgfqpoint{4.045382in}{2.780886in}}%
\pgfpathlineto{\pgfqpoint{4.046257in}{3.089697in}}%
\pgfpathlineto{\pgfqpoint{4.047133in}{3.519299in}}%
\pgfpathlineto{\pgfqpoint{4.048008in}{3.426638in}}%
\pgfpathlineto{\pgfqpoint{4.049759in}{3.854129in}}%
\pgfpathlineto{\pgfqpoint{4.051509in}{3.693991in}}%
\pgfpathlineto{\pgfqpoint{4.052385in}{3.548530in}}%
\pgfpathlineto{\pgfqpoint{4.053260in}{3.727511in}}%
\pgfpathlineto{\pgfqpoint{4.054136in}{3.778131in}}%
\pgfpathlineto{\pgfqpoint{4.055011in}{3.428366in}}%
\pgfpathlineto{\pgfqpoint{4.055886in}{3.346588in}}%
\pgfpathlineto{\pgfqpoint{4.056762in}{3.327550in}}%
\pgfpathlineto{\pgfqpoint{4.057637in}{3.274837in}}%
\pgfpathlineto{\pgfqpoint{4.058513in}{3.423349in}}%
\pgfpathlineto{\pgfqpoint{4.060263in}{3.188163in}}%
\pgfpathlineto{\pgfqpoint{4.062014in}{2.822727in}}%
\pgfpathlineto{\pgfqpoint{4.062889in}{2.726098in}}%
\pgfpathlineto{\pgfqpoint{4.064640in}{2.612311in}}%
\pgfpathlineto{\pgfqpoint{4.065516in}{2.602252in}}%
\pgfpathlineto{\pgfqpoint{4.066391in}{2.598890in}}%
\pgfpathlineto{\pgfqpoint{4.067266in}{2.827412in}}%
\pgfpathlineto{\pgfqpoint{4.068142in}{2.982659in}}%
\pgfpathlineto{\pgfqpoint{4.069892in}{3.564746in}}%
\pgfpathlineto{\pgfqpoint{4.070768in}{3.604083in}}%
\pgfpathlineto{\pgfqpoint{4.071643in}{3.867553in}}%
\pgfpathlineto{\pgfqpoint{4.072519in}{3.675681in}}%
\pgfpathlineto{\pgfqpoint{4.073394in}{3.786305in}}%
\pgfpathlineto{\pgfqpoint{4.074269in}{3.975376in}}%
\pgfpathlineto{\pgfqpoint{4.075145in}{3.933631in}}%
\pgfpathlineto{\pgfqpoint{4.076020in}{3.628332in}}%
\pgfpathlineto{\pgfqpoint{4.077771in}{3.273017in}}%
\pgfpathlineto{\pgfqpoint{4.078646in}{3.301237in}}%
\pgfpathlineto{\pgfqpoint{4.079522in}{3.364084in}}%
\pgfpathlineto{\pgfqpoint{4.083898in}{2.675955in}}%
\pgfpathlineto{\pgfqpoint{4.084774in}{2.625027in}}%
\pgfpathlineto{\pgfqpoint{4.085649in}{2.529183in}}%
\pgfpathlineto{\pgfqpoint{4.086525in}{2.519849in}}%
\pgfpathlineto{\pgfqpoint{4.087400in}{2.580225in}}%
\pgfpathlineto{\pgfqpoint{4.090026in}{3.487772in}}%
\pgfpathlineto{\pgfqpoint{4.090901in}{3.479279in}}%
\pgfpathlineto{\pgfqpoint{4.091777in}{3.411123in}}%
\pgfpathlineto{\pgfqpoint{4.092652in}{3.405547in}}%
\pgfpathlineto{\pgfqpoint{4.094403in}{3.611349in}}%
\pgfpathlineto{\pgfqpoint{4.095278in}{3.594965in}}%
\pgfpathlineto{\pgfqpoint{4.096154in}{3.693632in}}%
\pgfpathlineto{\pgfqpoint{4.097029in}{3.566328in}}%
\pgfpathlineto{\pgfqpoint{4.097904in}{3.491649in}}%
\pgfpathlineto{\pgfqpoint{4.098780in}{3.280790in}}%
\pgfpathlineto{\pgfqpoint{4.099655in}{3.236357in}}%
\pgfpathlineto{\pgfqpoint{4.100531in}{3.322762in}}%
\pgfpathlineto{\pgfqpoint{4.101406in}{3.336113in}}%
\pgfpathlineto{\pgfqpoint{4.102281in}{3.269992in}}%
\pgfpathlineto{\pgfqpoint{4.104032in}{2.897820in}}%
\pgfpathlineto{\pgfqpoint{4.104907in}{2.809346in}}%
\pgfpathlineto{\pgfqpoint{4.105783in}{2.757301in}}%
\pgfpathlineto{\pgfqpoint{4.107534in}{2.540601in}}%
\pgfpathlineto{\pgfqpoint{4.108409in}{2.477364in}}%
\pgfpathlineto{\pgfqpoint{4.109284in}{2.817620in}}%
\pgfpathlineto{\pgfqpoint{4.111035in}{3.162058in}}%
\pgfpathlineto{\pgfqpoint{4.111910in}{3.163527in}}%
\pgfpathlineto{\pgfqpoint{4.112786in}{3.273663in}}%
\pgfpathlineto{\pgfqpoint{4.113661in}{3.274922in}}%
\pgfpathlineto{\pgfqpoint{4.114537in}{3.379137in}}%
\pgfpathlineto{\pgfqpoint{4.115412in}{3.646028in}}%
\pgfpathlineto{\pgfqpoint{4.116287in}{3.570244in}}%
\pgfpathlineto{\pgfqpoint{4.117163in}{3.665432in}}%
\pgfpathlineto{\pgfqpoint{4.118038in}{3.555481in}}%
\pgfpathlineto{\pgfqpoint{4.119789in}{3.246678in}}%
\pgfpathlineto{\pgfqpoint{4.120664in}{3.164899in}}%
\pgfpathlineto{\pgfqpoint{4.121540in}{3.370579in}}%
\pgfpathlineto{\pgfqpoint{4.122415in}{3.202269in}}%
\pgfpathlineto{\pgfqpoint{4.123290in}{3.206075in}}%
\pgfpathlineto{\pgfqpoint{4.126792in}{2.688793in}}%
\pgfpathlineto{\pgfqpoint{4.127667in}{2.750383in}}%
\pgfpathlineto{\pgfqpoint{4.128543in}{2.693898in}}%
\pgfpathlineto{\pgfqpoint{4.129418in}{2.858297in}}%
\pgfpathlineto{\pgfqpoint{4.132044in}{3.950914in}}%
\pgfpathlineto{\pgfqpoint{4.132919in}{3.867207in}}%
\pgfpathlineto{\pgfqpoint{4.133795in}{4.034599in}}%
\pgfpathlineto{\pgfqpoint{4.134670in}{4.077814in}}%
\pgfpathlineto{\pgfqpoint{4.135546in}{4.060515in}}%
\pgfpathlineto{\pgfqpoint{4.136421in}{4.149206in}}%
\pgfpathlineto{\pgfqpoint{4.137296in}{4.155464in}}%
\pgfpathlineto{\pgfqpoint{4.138172in}{4.125051in}}%
\pgfpathlineto{\pgfqpoint{4.139922in}{3.752904in}}%
\pgfpathlineto{\pgfqpoint{4.140798in}{3.659095in}}%
\pgfpathlineto{\pgfqpoint{4.141673in}{3.720548in}}%
\pgfpathlineto{\pgfqpoint{4.142549in}{4.090515in}}%
\pgfpathlineto{\pgfqpoint{4.143424in}{4.019741in}}%
\pgfpathlineto{\pgfqpoint{4.145175in}{3.778454in}}%
\pgfpathlineto{\pgfqpoint{4.146050in}{3.559881in}}%
\pgfpathlineto{\pgfqpoint{4.146925in}{3.475334in}}%
\pgfpathlineto{\pgfqpoint{4.147801in}{3.347894in}}%
\pgfpathlineto{\pgfqpoint{4.148676in}{3.029912in}}%
\pgfpathlineto{\pgfqpoint{4.149552in}{2.969198in}}%
\pgfpathlineto{\pgfqpoint{4.150427in}{2.961672in}}%
\pgfpathlineto{\pgfqpoint{4.152178in}{3.666071in}}%
\pgfpathlineto{\pgfqpoint{4.153053in}{3.957783in}}%
\pgfpathlineto{\pgfqpoint{4.153928in}{4.091871in}}%
\pgfpathlineto{\pgfqpoint{4.154804in}{4.295794in}}%
\pgfpathlineto{\pgfqpoint{4.155679in}{4.369021in}}%
\pgfpathlineto{\pgfqpoint{4.156555in}{4.321622in}}%
\pgfpathlineto{\pgfqpoint{4.157430in}{4.453244in}}%
\pgfpathlineto{\pgfqpoint{4.158305in}{4.519467in}}%
\pgfpathlineto{\pgfqpoint{4.159181in}{4.475909in}}%
\pgfpathlineto{\pgfqpoint{4.160056in}{4.144774in}}%
\pgfpathlineto{\pgfqpoint{4.161807in}{3.759851in}}%
\pgfpathlineto{\pgfqpoint{4.162682in}{3.705884in}}%
\pgfpathlineto{\pgfqpoint{4.163558in}{3.836631in}}%
\pgfpathlineto{\pgfqpoint{4.164433in}{3.841011in}}%
\pgfpathlineto{\pgfqpoint{4.168810in}{3.259057in}}%
\pgfpathlineto{\pgfqpoint{4.169685in}{3.174510in}}%
\pgfpathlineto{\pgfqpoint{4.170561in}{3.164662in}}%
\pgfpathlineto{\pgfqpoint{4.171436in}{3.287005in}}%
\pgfpathlineto{\pgfqpoint{4.173187in}{3.790848in}}%
\pgfpathlineto{\pgfqpoint{4.174062in}{3.887615in}}%
\pgfpathlineto{\pgfqpoint{4.174937in}{3.845506in}}%
\pgfpathlineto{\pgfqpoint{4.175813in}{4.054455in}}%
\pgfpathlineto{\pgfqpoint{4.176688in}{4.041626in}}%
\pgfpathlineto{\pgfqpoint{4.179314in}{3.768724in}}%
\pgfpathlineto{\pgfqpoint{4.180190in}{3.863697in}}%
\pgfpathlineto{\pgfqpoint{4.181940in}{3.534113in}}%
\pgfpathlineto{\pgfqpoint{4.182816in}{3.443375in}}%
\pgfpathlineto{\pgfqpoint{4.183691in}{3.400907in}}%
\pgfpathlineto{\pgfqpoint{4.184567in}{3.328139in}}%
\pgfpathlineto{\pgfqpoint{4.185442in}{3.302645in}}%
\pgfpathlineto{\pgfqpoint{4.186317in}{3.222991in}}%
\pgfpathlineto{\pgfqpoint{4.187193in}{3.005898in}}%
\pgfpathlineto{\pgfqpoint{4.188068in}{2.896854in}}%
\pgfpathlineto{\pgfqpoint{4.188943in}{2.870423in}}%
\pgfpathlineto{\pgfqpoint{4.189819in}{2.778536in}}%
\pgfpathlineto{\pgfqpoint{4.190694in}{2.754945in}}%
\pgfpathlineto{\pgfqpoint{4.191570in}{2.683325in}}%
\pgfpathlineto{\pgfqpoint{4.192445in}{2.789989in}}%
\pgfpathlineto{\pgfqpoint{4.193320in}{3.115817in}}%
\pgfpathlineto{\pgfqpoint{4.195071in}{3.476329in}}%
\pgfpathlineto{\pgfqpoint{4.195947in}{3.509489in}}%
\pgfpathlineto{\pgfqpoint{4.196822in}{3.715192in}}%
\pgfpathlineto{\pgfqpoint{4.197697in}{3.720653in}}%
\pgfpathlineto{\pgfqpoint{4.198573in}{3.624131in}}%
\pgfpathlineto{\pgfqpoint{4.199448in}{3.761838in}}%
\pgfpathlineto{\pgfqpoint{4.200323in}{3.729690in}}%
\pgfpathlineto{\pgfqpoint{4.201199in}{3.858414in}}%
\pgfpathlineto{\pgfqpoint{4.202074in}{3.563128in}}%
\pgfpathlineto{\pgfqpoint{4.202950in}{3.152265in}}%
\pgfpathlineto{\pgfqpoint{4.203825in}{3.460927in}}%
\pgfpathlineto{\pgfqpoint{4.204700in}{3.007299in}}%
\pgfpathlineto{\pgfqpoint{4.206451in}{3.259903in}}%
\pgfpathlineto{\pgfqpoint{4.208202in}{3.165931in}}%
\pgfpathlineto{\pgfqpoint{4.209077in}{3.064559in}}%
\pgfpathlineto{\pgfqpoint{4.209953in}{3.000824in}}%
\pgfpathlineto{\pgfqpoint{4.210828in}{2.960347in}}%
\pgfpathlineto{\pgfqpoint{4.211703in}{2.955363in}}%
\pgfpathlineto{\pgfqpoint{4.212579in}{2.937572in}}%
\pgfpathlineto{\pgfqpoint{4.213454in}{3.064341in}}%
\pgfpathlineto{\pgfqpoint{4.214329in}{3.470410in}}%
\pgfpathlineto{\pgfqpoint{4.216956in}{3.943233in}}%
\pgfpathlineto{\pgfqpoint{4.217831in}{4.017074in}}%
\pgfpathlineto{\pgfqpoint{4.218706in}{3.931228in}}%
\pgfpathlineto{\pgfqpoint{4.219582in}{3.995517in}}%
\pgfpathlineto{\pgfqpoint{4.220457in}{3.996533in}}%
\pgfpathlineto{\pgfqpoint{4.221332in}{4.070882in}}%
\pgfpathlineto{\pgfqpoint{4.225709in}{3.516120in}}%
\pgfpathlineto{\pgfqpoint{4.226585in}{3.539129in}}%
\pgfpathlineto{\pgfqpoint{4.227460in}{3.653732in}}%
\pgfpathlineto{\pgfqpoint{4.228335in}{3.532726in}}%
\pgfpathlineto{\pgfqpoint{4.229211in}{3.467601in}}%
\pgfpathlineto{\pgfqpoint{4.230962in}{3.175235in}}%
\pgfpathlineto{\pgfqpoint{4.232712in}{2.969832in}}%
\pgfpathlineto{\pgfqpoint{4.233588in}{2.979951in}}%
\pgfpathlineto{\pgfqpoint{4.234463in}{3.107119in}}%
\pgfpathlineto{\pgfqpoint{4.235338in}{3.307578in}}%
\pgfpathlineto{\pgfqpoint{4.236214in}{3.367943in}}%
\pgfpathlineto{\pgfqpoint{4.237089in}{3.505302in}}%
\pgfpathlineto{\pgfqpoint{4.237965in}{3.706268in}}%
\pgfpathlineto{\pgfqpoint{4.238840in}{3.610711in}}%
\pgfpathlineto{\pgfqpoint{4.239715in}{3.573620in}}%
\pgfpathlineto{\pgfqpoint{4.241466in}{3.748426in}}%
\pgfpathlineto{\pgfqpoint{4.242341in}{3.770643in}}%
\pgfpathlineto{\pgfqpoint{4.243217in}{3.782856in}}%
\pgfpathlineto{\pgfqpoint{4.244968in}{3.335376in}}%
\pgfpathlineto{\pgfqpoint{4.245843in}{3.267138in}}%
\pgfpathlineto{\pgfqpoint{4.246718in}{3.283297in}}%
\pgfpathlineto{\pgfqpoint{4.247594in}{3.379066in}}%
\pgfpathlineto{\pgfqpoint{4.248469in}{3.399607in}}%
\pgfpathlineto{\pgfqpoint{4.249344in}{3.389367in}}%
\pgfpathlineto{\pgfqpoint{4.250220in}{3.085008in}}%
\pgfpathlineto{\pgfqpoint{4.251095in}{3.029610in}}%
\pgfpathlineto{\pgfqpoint{4.251971in}{2.942103in}}%
\pgfpathlineto{\pgfqpoint{4.252846in}{2.818438in}}%
\pgfpathlineto{\pgfqpoint{4.253721in}{2.759264in}}%
\pgfpathlineto{\pgfqpoint{4.254597in}{2.747242in}}%
\pgfpathlineto{\pgfqpoint{4.255472in}{2.766524in}}%
\pgfpathlineto{\pgfqpoint{4.256347in}{3.120390in}}%
\pgfpathlineto{\pgfqpoint{4.257223in}{3.341180in}}%
\pgfpathlineto{\pgfqpoint{4.258098in}{3.488334in}}%
\pgfpathlineto{\pgfqpoint{4.258974in}{3.510161in}}%
\pgfpathlineto{\pgfqpoint{4.259849in}{3.521285in}}%
\pgfpathlineto{\pgfqpoint{4.260724in}{3.542641in}}%
\pgfpathlineto{\pgfqpoint{4.261600in}{3.471031in}}%
\pgfpathlineto{\pgfqpoint{4.262475in}{3.579187in}}%
\pgfpathlineto{\pgfqpoint{4.263350in}{3.758892in}}%
\pgfpathlineto{\pgfqpoint{4.264226in}{3.580330in}}%
\pgfpathlineto{\pgfqpoint{4.265101in}{3.553482in}}%
\pgfpathlineto{\pgfqpoint{4.266852in}{3.249422in}}%
\pgfpathlineto{\pgfqpoint{4.268603in}{3.526564in}}%
\pgfpathlineto{\pgfqpoint{4.269478in}{3.469504in}}%
\pgfpathlineto{\pgfqpoint{4.270353in}{3.483943in}}%
\pgfpathlineto{\pgfqpoint{4.271229in}{3.345689in}}%
\pgfpathlineto{\pgfqpoint{4.272104in}{3.143156in}}%
\pgfpathlineto{\pgfqpoint{4.272980in}{3.068637in}}%
\pgfpathlineto{\pgfqpoint{4.273855in}{3.040605in}}%
\pgfpathlineto{\pgfqpoint{4.274730in}{3.035712in}}%
\pgfpathlineto{\pgfqpoint{4.275606in}{3.061236in}}%
\pgfpathlineto{\pgfqpoint{4.276481in}{3.060288in}}%
\pgfpathlineto{\pgfqpoint{4.277356in}{3.465923in}}%
\pgfpathlineto{\pgfqpoint{4.279107in}{3.734500in}}%
\pgfpathlineto{\pgfqpoint{4.279983in}{3.801747in}}%
\pgfpathlineto{\pgfqpoint{4.280858in}{3.762652in}}%
\pgfpathlineto{\pgfqpoint{4.281733in}{3.764853in}}%
\pgfpathlineto{\pgfqpoint{4.282609in}{3.731625in}}%
\pgfpathlineto{\pgfqpoint{4.283484in}{3.549532in}}%
\pgfpathlineto{\pgfqpoint{4.284359in}{3.702311in}}%
\pgfpathlineto{\pgfqpoint{4.285235in}{3.704018in}}%
\pgfpathlineto{\pgfqpoint{4.286110in}{3.579022in}}%
\pgfpathlineto{\pgfqpoint{4.287861in}{3.023191in}}%
\pgfpathlineto{\pgfqpoint{4.288736in}{3.027306in}}%
\pgfpathlineto{\pgfqpoint{4.289612in}{3.344903in}}%
\pgfpathlineto{\pgfqpoint{4.290487in}{3.256761in}}%
\pgfpathlineto{\pgfqpoint{4.291362in}{3.480862in}}%
\pgfpathlineto{\pgfqpoint{4.293113in}{3.171459in}}%
\pgfpathlineto{\pgfqpoint{4.294864in}{2.928238in}}%
\pgfpathlineto{\pgfqpoint{4.295739in}{2.924674in}}%
\pgfpathlineto{\pgfqpoint{4.296615in}{2.924069in}}%
\pgfpathlineto{\pgfqpoint{4.297490in}{2.844767in}}%
\pgfpathlineto{\pgfqpoint{4.300116in}{3.665599in}}%
\pgfpathlineto{\pgfqpoint{4.300992in}{3.679347in}}%
\pgfpathlineto{\pgfqpoint{4.301867in}{3.819699in}}%
\pgfpathlineto{\pgfqpoint{4.302742in}{3.685058in}}%
\pgfpathlineto{\pgfqpoint{4.303618in}{3.760958in}}%
\pgfpathlineto{\pgfqpoint{4.304493in}{3.799272in}}%
\pgfpathlineto{\pgfqpoint{4.306244in}{3.654061in}}%
\pgfpathlineto{\pgfqpoint{4.308870in}{2.962850in}}%
\pgfpathlineto{\pgfqpoint{4.309745in}{2.852168in}}%
\pgfpathlineto{\pgfqpoint{4.310621in}{3.044925in}}%
\pgfpathlineto{\pgfqpoint{4.311496in}{2.866617in}}%
\pgfpathlineto{\pgfqpoint{4.312371in}{2.821278in}}%
\pgfpathlineto{\pgfqpoint{4.313247in}{2.803244in}}%
\pgfpathlineto{\pgfqpoint{4.314122in}{2.636898in}}%
\pgfpathlineto{\pgfqpoint{4.314998in}{2.614485in}}%
\pgfpathlineto{\pgfqpoint{4.315873in}{2.551082in}}%
\pgfpathlineto{\pgfqpoint{4.316748in}{2.518943in}}%
\pgfpathlineto{\pgfqpoint{4.317624in}{2.466444in}}%
\pgfpathlineto{\pgfqpoint{4.318499in}{2.601630in}}%
\pgfpathlineto{\pgfqpoint{4.319374in}{2.874584in}}%
\pgfpathlineto{\pgfqpoint{4.320250in}{2.845946in}}%
\pgfpathlineto{\pgfqpoint{4.321125in}{3.070017in}}%
\pgfpathlineto{\pgfqpoint{4.322001in}{3.197247in}}%
\pgfpathlineto{\pgfqpoint{4.322876in}{3.417838in}}%
\pgfpathlineto{\pgfqpoint{4.323751in}{3.380757in}}%
\pgfpathlineto{\pgfqpoint{4.324627in}{3.306537in}}%
\pgfpathlineto{\pgfqpoint{4.326378in}{3.484385in}}%
\pgfpathlineto{\pgfqpoint{4.327253in}{3.475923in}}%
\pgfpathlineto{\pgfqpoint{4.328128in}{3.266042in}}%
\pgfpathlineto{\pgfqpoint{4.330754in}{2.929416in}}%
\pgfpathlineto{\pgfqpoint{4.331630in}{3.103253in}}%
\pgfpathlineto{\pgfqpoint{4.332505in}{3.004690in}}%
\pgfpathlineto{\pgfqpoint{4.333381in}{3.025291in}}%
\pgfpathlineto{\pgfqpoint{4.334256in}{2.851121in}}%
\pgfpathlineto{\pgfqpoint{4.337757in}{2.544226in}}%
\pgfpathlineto{\pgfqpoint{4.338633in}{2.525498in}}%
\pgfpathlineto{\pgfqpoint{4.339508in}{2.620255in}}%
\pgfpathlineto{\pgfqpoint{4.340384in}{2.992248in}}%
\pgfpathlineto{\pgfqpoint{4.341259in}{3.019394in}}%
\pgfpathlineto{\pgfqpoint{4.342134in}{3.187741in}}%
\pgfpathlineto{\pgfqpoint{4.343010in}{3.300421in}}%
\pgfpathlineto{\pgfqpoint{4.344760in}{3.452403in}}%
\pgfpathlineto{\pgfqpoint{4.345636in}{3.581533in}}%
\pgfpathlineto{\pgfqpoint{4.346511in}{3.780397in}}%
\pgfpathlineto{\pgfqpoint{4.347387in}{4.338173in}}%
\pgfpathlineto{\pgfqpoint{4.348262in}{4.160091in}}%
\pgfpathlineto{\pgfqpoint{4.349137in}{3.845220in}}%
\pgfpathlineto{\pgfqpoint{4.350888in}{3.576129in}}%
\pgfpathlineto{\pgfqpoint{4.351763in}{3.641033in}}%
\pgfpathlineto{\pgfqpoint{4.352639in}{3.661707in}}%
\pgfpathlineto{\pgfqpoint{4.354390in}{3.518348in}}%
\pgfpathlineto{\pgfqpoint{4.355265in}{3.415465in}}%
\pgfpathlineto{\pgfqpoint{4.357016in}{3.040998in}}%
\pgfpathlineto{\pgfqpoint{4.357891in}{2.929929in}}%
\pgfpathlineto{\pgfqpoint{4.359642in}{2.815901in}}%
\pgfpathlineto{\pgfqpoint{4.360517in}{2.829327in}}%
\pgfpathlineto{\pgfqpoint{4.361393in}{3.245580in}}%
\pgfpathlineto{\pgfqpoint{4.363143in}{3.772711in}}%
\pgfpathlineto{\pgfqpoint{4.364019in}{3.759542in}}%
\pgfpathlineto{\pgfqpoint{4.365769in}{4.080502in}}%
\pgfpathlineto{\pgfqpoint{4.366645in}{4.137187in}}%
\pgfpathlineto{\pgfqpoint{4.367520in}{4.372728in}}%
\pgfpathlineto{\pgfqpoint{4.368396in}{4.344229in}}%
\pgfpathlineto{\pgfqpoint{4.369271in}{4.106499in}}%
\pgfpathlineto{\pgfqpoint{4.370146in}{4.027970in}}%
\pgfpathlineto{\pgfqpoint{4.371897in}{3.595697in}}%
\pgfpathlineto{\pgfqpoint{4.372772in}{3.540074in}}%
\pgfpathlineto{\pgfqpoint{4.373648in}{3.732872in}}%
\pgfpathlineto{\pgfqpoint{4.374523in}{3.761659in}}%
\pgfpathlineto{\pgfqpoint{4.375399in}{3.669681in}}%
\pgfpathlineto{\pgfqpoint{4.376274in}{3.626486in}}%
\pgfpathlineto{\pgfqpoint{4.378900in}{3.140739in}}%
\pgfpathlineto{\pgfqpoint{4.379775in}{3.146297in}}%
\pgfpathlineto{\pgfqpoint{4.380651in}{3.131587in}}%
\pgfpathlineto{\pgfqpoint{4.381526in}{3.108902in}}%
\pgfpathlineto{\pgfqpoint{4.382402in}{3.459536in}}%
\pgfpathlineto{\pgfqpoint{4.383277in}{3.406896in}}%
\pgfpathlineto{\pgfqpoint{4.384152in}{3.456926in}}%
\pgfpathlineto{\pgfqpoint{4.385028in}{3.764502in}}%
\pgfpathlineto{\pgfqpoint{4.385903in}{3.747720in}}%
\pgfpathlineto{\pgfqpoint{4.386778in}{3.910498in}}%
\pgfpathlineto{\pgfqpoint{4.387654in}{3.876903in}}%
\pgfpathlineto{\pgfqpoint{4.388529in}{3.784843in}}%
\pgfpathlineto{\pgfqpoint{4.389405in}{3.778469in}}%
\pgfpathlineto{\pgfqpoint{4.390280in}{3.830141in}}%
\pgfpathlineto{\pgfqpoint{4.391155in}{3.839957in}}%
\pgfpathlineto{\pgfqpoint{4.392031in}{3.857642in}}%
\pgfpathlineto{\pgfqpoint{4.392906in}{3.626515in}}%
\pgfpathlineto{\pgfqpoint{4.393781in}{3.678543in}}%
\pgfpathlineto{\pgfqpoint{4.394657in}{3.796880in}}%
\pgfpathlineto{\pgfqpoint{4.395532in}{3.776732in}}%
\pgfpathlineto{\pgfqpoint{4.396408in}{3.675692in}}%
\pgfpathlineto{\pgfqpoint{4.397283in}{3.659079in}}%
\pgfpathlineto{\pgfqpoint{4.399034in}{3.409605in}}%
\pgfpathlineto{\pgfqpoint{4.399909in}{3.433287in}}%
\pgfpathlineto{\pgfqpoint{4.400784in}{3.401963in}}%
\pgfpathlineto{\pgfqpoint{4.401660in}{3.317264in}}%
\pgfpathlineto{\pgfqpoint{4.402535in}{3.458153in}}%
\pgfpathlineto{\pgfqpoint{4.403411in}{3.942678in}}%
\pgfpathlineto{\pgfqpoint{4.404286in}{4.213807in}}%
\pgfpathlineto{\pgfqpoint{4.405161in}{4.288979in}}%
\pgfpathlineto{\pgfqpoint{4.406037in}{4.335226in}}%
\pgfpathlineto{\pgfqpoint{4.406912in}{4.255569in}}%
\pgfpathlineto{\pgfqpoint{4.407787in}{4.253677in}}%
\pgfpathlineto{\pgfqpoint{4.408663in}{4.141370in}}%
\pgfpathlineto{\pgfqpoint{4.409538in}{4.279366in}}%
\pgfpathlineto{\pgfqpoint{4.411289in}{4.119724in}}%
\pgfpathlineto{\pgfqpoint{4.412164in}{4.007157in}}%
\pgfpathlineto{\pgfqpoint{4.413040in}{3.803607in}}%
\pgfpathlineto{\pgfqpoint{4.413915in}{3.686512in}}%
\pgfpathlineto{\pgfqpoint{4.414790in}{3.644951in}}%
\pgfpathlineto{\pgfqpoint{4.415666in}{3.861340in}}%
\pgfpathlineto{\pgfqpoint{4.416541in}{3.780991in}}%
\pgfpathlineto{\pgfqpoint{4.418292in}{3.572507in}}%
\pgfpathlineto{\pgfqpoint{4.420043in}{3.417459in}}%
\pgfpathlineto{\pgfqpoint{4.420918in}{2.702718in}}%
\pgfpathlineto{\pgfqpoint{4.421793in}{2.843902in}}%
\pgfpathlineto{\pgfqpoint{4.422669in}{2.738965in}}%
\pgfpathlineto{\pgfqpoint{4.423544in}{2.831163in}}%
\pgfpathlineto{\pgfqpoint{4.424420in}{3.289116in}}%
\pgfpathlineto{\pgfqpoint{4.426170in}{3.921102in}}%
\pgfpathlineto{\pgfqpoint{4.427046in}{4.051705in}}%
\pgfpathlineto{\pgfqpoint{4.427921in}{4.080831in}}%
\pgfpathlineto{\pgfqpoint{4.428796in}{4.262681in}}%
\pgfpathlineto{\pgfqpoint{4.430547in}{4.361863in}}%
\pgfpathlineto{\pgfqpoint{4.431423in}{4.531230in}}%
\pgfpathlineto{\pgfqpoint{4.432298in}{4.502303in}}%
\pgfpathlineto{\pgfqpoint{4.434924in}{3.902125in}}%
\pgfpathlineto{\pgfqpoint{4.435799in}{3.895201in}}%
\pgfpathlineto{\pgfqpoint{4.436675in}{3.983796in}}%
\pgfpathlineto{\pgfqpoint{4.437550in}{3.925981in}}%
\pgfpathlineto{\pgfqpoint{4.438426in}{3.665513in}}%
\pgfpathlineto{\pgfqpoint{4.439301in}{3.570121in}}%
\pgfpathlineto{\pgfqpoint{4.440176in}{3.408216in}}%
\pgfpathlineto{\pgfqpoint{4.441927in}{3.240903in}}%
\pgfpathlineto{\pgfqpoint{4.442802in}{3.248092in}}%
\pgfpathlineto{\pgfqpoint{4.443678in}{3.241658in}}%
\pgfpathlineto{\pgfqpoint{4.444553in}{3.380626in}}%
\pgfpathlineto{\pgfqpoint{4.446304in}{4.088152in}}%
\pgfpathlineto{\pgfqpoint{4.448055in}{4.510343in}}%
\pgfpathlineto{\pgfqpoint{4.448930in}{4.455167in}}%
\pgfpathlineto{\pgfqpoint{4.449805in}{4.521988in}}%
\pgfpathlineto{\pgfqpoint{4.450681in}{4.339688in}}%
\pgfpathlineto{\pgfqpoint{4.451556in}{4.385821in}}%
\pgfpathlineto{\pgfqpoint{4.452432in}{3.959164in}}%
\pgfpathlineto{\pgfqpoint{4.453307in}{4.010031in}}%
\pgfpathlineto{\pgfqpoint{4.454182in}{3.921709in}}%
\pgfpathlineto{\pgfqpoint{4.455058in}{3.632618in}}%
\pgfpathlineto{\pgfqpoint{4.455933in}{3.468428in}}%
\pgfpathlineto{\pgfqpoint{4.456808in}{3.596206in}}%
\pgfpathlineto{\pgfqpoint{4.457684in}{3.609087in}}%
\pgfpathlineto{\pgfqpoint{4.460310in}{3.261655in}}%
\pgfpathlineto{\pgfqpoint{4.461185in}{3.079390in}}%
\pgfpathlineto{\pgfqpoint{4.462061in}{2.960830in}}%
\pgfpathlineto{\pgfqpoint{4.462936in}{2.880421in}}%
\pgfpathlineto{\pgfqpoint{4.463812in}{2.828436in}}%
\pgfpathlineto{\pgfqpoint{4.464687in}{2.824328in}}%
\pgfpathlineto{\pgfqpoint{4.465562in}{2.897484in}}%
\pgfpathlineto{\pgfqpoint{4.466438in}{3.117517in}}%
\pgfpathlineto{\pgfqpoint{4.467313in}{3.236780in}}%
\pgfpathlineto{\pgfqpoint{4.468188in}{3.459202in}}%
\pgfpathlineto{\pgfqpoint{4.469064in}{3.600521in}}%
\pgfpathlineto{\pgfqpoint{4.469939in}{3.601012in}}%
\pgfpathlineto{\pgfqpoint{4.470815in}{3.605083in}}%
\pgfpathlineto{\pgfqpoint{4.471690in}{3.619068in}}%
\pgfpathlineto{\pgfqpoint{4.472565in}{3.745345in}}%
\pgfpathlineto{\pgfqpoint{4.473441in}{3.723787in}}%
\pgfpathlineto{\pgfqpoint{4.474316in}{3.679390in}}%
\pgfpathlineto{\pgfqpoint{4.476942in}{3.248094in}}%
\pgfpathlineto{\pgfqpoint{4.477818in}{3.277992in}}%
\pgfpathlineto{\pgfqpoint{4.478693in}{3.486571in}}%
\pgfpathlineto{\pgfqpoint{4.480444in}{3.346685in}}%
\pgfpathlineto{\pgfqpoint{4.483070in}{3.010671in}}%
\pgfpathlineto{\pgfqpoint{4.483945in}{2.947358in}}%
\pgfpathlineto{\pgfqpoint{4.484821in}{2.821912in}}%
\pgfpathlineto{\pgfqpoint{4.485696in}{2.785423in}}%
\pgfpathlineto{\pgfqpoint{4.486571in}{2.898680in}}%
\pgfpathlineto{\pgfqpoint{4.488322in}{3.562590in}}%
\pgfpathlineto{\pgfqpoint{4.489197in}{3.779913in}}%
\pgfpathlineto{\pgfqpoint{4.490073in}{3.802600in}}%
\pgfpathlineto{\pgfqpoint{4.491824in}{3.757004in}}%
\pgfpathlineto{\pgfqpoint{4.494450in}{4.091888in}}%
\pgfpathlineto{\pgfqpoint{4.495325in}{4.153099in}}%
\pgfpathlineto{\pgfqpoint{4.496200in}{3.983828in}}%
\pgfpathlineto{\pgfqpoint{4.497076in}{3.705635in}}%
\pgfpathlineto{\pgfqpoint{4.497951in}{3.609899in}}%
\pgfpathlineto{\pgfqpoint{4.498827in}{3.588149in}}%
\pgfpathlineto{\pgfqpoint{4.499702in}{3.741330in}}%
\pgfpathlineto{\pgfqpoint{4.500577in}{3.676417in}}%
\pgfpathlineto{\pgfqpoint{4.501453in}{3.710188in}}%
\pgfpathlineto{\pgfqpoint{4.503203in}{3.538586in}}%
\pgfpathlineto{\pgfqpoint{4.504954in}{3.267756in}}%
\pgfpathlineto{\pgfqpoint{4.506705in}{3.295667in}}%
\pgfpathlineto{\pgfqpoint{4.507580in}{3.465520in}}%
\pgfpathlineto{\pgfqpoint{4.508456in}{3.881603in}}%
\pgfpathlineto{\pgfqpoint{4.510206in}{4.245281in}}%
\pgfpathlineto{\pgfqpoint{4.511957in}{4.156242in}}%
\pgfpathlineto{\pgfqpoint{4.512833in}{4.189615in}}%
\pgfpathlineto{\pgfqpoint{4.513708in}{4.193966in}}%
\pgfpathlineto{\pgfqpoint{4.514583in}{4.147625in}}%
\pgfpathlineto{\pgfqpoint{4.515459in}{4.342651in}}%
\pgfpathlineto{\pgfqpoint{4.516334in}{4.352917in}}%
\pgfpathlineto{\pgfqpoint{4.518085in}{3.833732in}}%
\pgfpathlineto{\pgfqpoint{4.518960in}{3.760394in}}%
\pgfpathlineto{\pgfqpoint{4.519836in}{3.782271in}}%
\pgfpathlineto{\pgfqpoint{4.520711in}{3.852248in}}%
\pgfpathlineto{\pgfqpoint{4.521586in}{3.709221in}}%
\pgfpathlineto{\pgfqpoint{4.523337in}{3.632437in}}%
\pgfpathlineto{\pgfqpoint{4.524212in}{3.454159in}}%
\pgfpathlineto{\pgfqpoint{4.525088in}{3.390424in}}%
\pgfpathlineto{\pgfqpoint{4.525963in}{3.238305in}}%
\pgfpathlineto{\pgfqpoint{4.526839in}{3.210153in}}%
\pgfpathlineto{\pgfqpoint{4.527714in}{3.269810in}}%
\pgfpathlineto{\pgfqpoint{4.528589in}{3.351720in}}%
\pgfpathlineto{\pgfqpoint{4.529465in}{3.664884in}}%
\pgfpathlineto{\pgfqpoint{4.530340in}{3.756019in}}%
\pgfpathlineto{\pgfqpoint{4.531215in}{4.052270in}}%
\pgfpathlineto{\pgfqpoint{4.532091in}{4.196897in}}%
\pgfpathlineto{\pgfqpoint{4.532966in}{4.126529in}}%
\pgfpathlineto{\pgfqpoint{4.533842in}{4.191550in}}%
\pgfpathlineto{\pgfqpoint{4.534717in}{4.004426in}}%
\pgfpathlineto{\pgfqpoint{4.535592in}{4.190351in}}%
\pgfpathlineto{\pgfqpoint{4.536468in}{4.250669in}}%
\pgfpathlineto{\pgfqpoint{4.537343in}{4.237253in}}%
\pgfpathlineto{\pgfqpoint{4.539094in}{3.609012in}}%
\pgfpathlineto{\pgfqpoint{4.539969in}{3.443598in}}%
\pgfpathlineto{\pgfqpoint{4.540845in}{3.672161in}}%
\pgfpathlineto{\pgfqpoint{4.541720in}{3.753292in}}%
\pgfpathlineto{\pgfqpoint{4.542595in}{3.734534in}}%
\pgfpathlineto{\pgfqpoint{4.543471in}{3.576676in}}%
\pgfpathlineto{\pgfqpoint{4.544346in}{3.642707in}}%
\pgfpathlineto{\pgfqpoint{4.545221in}{3.426279in}}%
\pgfpathlineto{\pgfqpoint{4.546097in}{3.348724in}}%
\pgfpathlineto{\pgfqpoint{4.546972in}{3.239113in}}%
\pgfpathlineto{\pgfqpoint{4.547848in}{3.211021in}}%
\pgfpathlineto{\pgfqpoint{4.548723in}{3.219479in}}%
\pgfpathlineto{\pgfqpoint{4.549598in}{3.157099in}}%
\pgfpathlineto{\pgfqpoint{4.550474in}{3.391611in}}%
\pgfpathlineto{\pgfqpoint{4.551349in}{3.510132in}}%
\pgfpathlineto{\pgfqpoint{4.552224in}{3.704812in}}%
\pgfpathlineto{\pgfqpoint{4.553100in}{3.616056in}}%
\pgfpathlineto{\pgfqpoint{4.553975in}{3.661420in}}%
\pgfpathlineto{\pgfqpoint{4.554851in}{3.553529in}}%
\pgfpathlineto{\pgfqpoint{4.555726in}{3.710890in}}%
\pgfpathlineto{\pgfqpoint{4.556601in}{3.763264in}}%
\pgfpathlineto{\pgfqpoint{4.557477in}{3.651309in}}%
\pgfpathlineto{\pgfqpoint{4.558352in}{3.592822in}}%
\pgfpathlineto{\pgfqpoint{4.559227in}{3.507394in}}%
\pgfpathlineto{\pgfqpoint{4.560978in}{3.243116in}}%
\pgfpathlineto{\pgfqpoint{4.561854in}{3.275880in}}%
\pgfpathlineto{\pgfqpoint{4.562729in}{3.505646in}}%
\pgfpathlineto{\pgfqpoint{4.563604in}{3.541214in}}%
\pgfpathlineto{\pgfqpoint{4.564480in}{3.509988in}}%
\pgfpathlineto{\pgfqpoint{4.566230in}{3.217629in}}%
\pgfpathlineto{\pgfqpoint{4.567106in}{3.165599in}}%
\pgfpathlineto{\pgfqpoint{4.567981in}{3.051948in}}%
\pgfpathlineto{\pgfqpoint{4.568857in}{3.057158in}}%
\pgfpathlineto{\pgfqpoint{4.570607in}{3.174589in}}%
\pgfpathlineto{\pgfqpoint{4.571483in}{3.387122in}}%
\pgfpathlineto{\pgfqpoint{4.572358in}{3.418623in}}%
\pgfpathlineto{\pgfqpoint{4.573233in}{3.599363in}}%
\pgfpathlineto{\pgfqpoint{4.574109in}{3.398885in}}%
\pgfpathlineto{\pgfqpoint{4.574984in}{3.452081in}}%
\pgfpathlineto{\pgfqpoint{4.575860in}{3.439800in}}%
\pgfpathlineto{\pgfqpoint{4.576735in}{3.539688in}}%
\pgfpathlineto{\pgfqpoint{4.577610in}{3.732963in}}%
\pgfpathlineto{\pgfqpoint{4.578486in}{3.756053in}}%
\pgfpathlineto{\pgfqpoint{4.579361in}{3.676922in}}%
\pgfpathlineto{\pgfqpoint{4.581987in}{3.130170in}}%
\pgfpathlineto{\pgfqpoint{4.582863in}{3.149927in}}%
\pgfpathlineto{\pgfqpoint{4.583738in}{3.214042in}}%
\pgfpathlineto{\pgfqpoint{4.584613in}{3.303264in}}%
\pgfpathlineto{\pgfqpoint{4.585489in}{3.458373in}}%
\pgfpathlineto{\pgfqpoint{4.587239in}{3.108169in}}%
\pgfpathlineto{\pgfqpoint{4.588115in}{3.056214in}}%
\pgfpathlineto{\pgfqpoint{4.589866in}{2.913641in}}%
\pgfpathlineto{\pgfqpoint{4.590741in}{2.965407in}}%
\pgfpathlineto{\pgfqpoint{4.591616in}{2.977740in}}%
\pgfpathlineto{\pgfqpoint{4.592492in}{3.368115in}}%
\pgfpathlineto{\pgfqpoint{4.593367in}{3.279672in}}%
\pgfpathlineto{\pgfqpoint{4.594243in}{3.451260in}}%
\pgfpathlineto{\pgfqpoint{4.595118in}{3.494939in}}%
\pgfpathlineto{\pgfqpoint{4.595993in}{3.491347in}}%
\pgfpathlineto{\pgfqpoint{4.596869in}{3.496047in}}%
\pgfpathlineto{\pgfqpoint{4.597744in}{3.672688in}}%
\pgfpathlineto{\pgfqpoint{4.598619in}{3.579287in}}%
\pgfpathlineto{\pgfqpoint{4.599495in}{3.598507in}}%
\pgfpathlineto{\pgfqpoint{4.601246in}{3.673410in}}%
\pgfpathlineto{\pgfqpoint{4.602996in}{3.279067in}}%
\pgfpathlineto{\pgfqpoint{4.603872in}{3.181864in}}%
\pgfpathlineto{\pgfqpoint{4.604747in}{3.491449in}}%
\pgfpathlineto{\pgfqpoint{4.605622in}{3.623564in}}%
\pgfpathlineto{\pgfqpoint{4.606498in}{3.583578in}}%
\pgfpathlineto{\pgfqpoint{4.608249in}{3.225445in}}%
\pgfpathlineto{\pgfqpoint{4.609124in}{3.152157in}}%
\pgfpathlineto{\pgfqpoint{4.609999in}{3.107225in}}%
\pgfpathlineto{\pgfqpoint{4.611750in}{2.975413in}}%
\pgfpathlineto{\pgfqpoint{4.612625in}{2.941204in}}%
\pgfpathlineto{\pgfqpoint{4.613501in}{3.127823in}}%
\pgfpathlineto{\pgfqpoint{4.614376in}{3.135845in}}%
\pgfpathlineto{\pgfqpoint{4.615252in}{3.104318in}}%
\pgfpathlineto{\pgfqpoint{4.617002in}{2.946517in}}%
\pgfpathlineto{\pgfqpoint{4.617878in}{3.028913in}}%
\pgfpathlineto{\pgfqpoint{4.618753in}{3.145501in}}%
\pgfpathlineto{\pgfqpoint{4.619628in}{3.136070in}}%
\pgfpathlineto{\pgfqpoint{4.620504in}{3.072843in}}%
\pgfpathlineto{\pgfqpoint{4.621379in}{3.278484in}}%
\pgfpathlineto{\pgfqpoint{4.622255in}{3.197472in}}%
\pgfpathlineto{\pgfqpoint{4.623130in}{3.188012in}}%
\pgfpathlineto{\pgfqpoint{4.624005in}{3.139537in}}%
\pgfpathlineto{\pgfqpoint{4.624881in}{3.141114in}}%
\pgfpathlineto{\pgfqpoint{4.625756in}{3.347214in}}%
\pgfpathlineto{\pgfqpoint{4.626631in}{3.486993in}}%
\pgfpathlineto{\pgfqpoint{4.627507in}{3.553976in}}%
\pgfpathlineto{\pgfqpoint{4.628382in}{3.498245in}}%
\pgfpathlineto{\pgfqpoint{4.630133in}{3.243342in}}%
\pgfpathlineto{\pgfqpoint{4.631008in}{3.224086in}}%
\pgfpathlineto{\pgfqpoint{4.631884in}{3.181306in}}%
\pgfpathlineto{\pgfqpoint{4.632759in}{3.164391in}}%
\pgfpathlineto{\pgfqpoint{4.633634in}{3.253769in}}%
\pgfpathlineto{\pgfqpoint{4.634510in}{3.607855in}}%
\pgfpathlineto{\pgfqpoint{4.636261in}{3.786307in}}%
\pgfpathlineto{\pgfqpoint{4.637136in}{3.836997in}}%
\pgfpathlineto{\pgfqpoint{4.638011in}{3.871081in}}%
\pgfpathlineto{\pgfqpoint{4.638887in}{3.987020in}}%
\pgfpathlineto{\pgfqpoint{4.639762in}{4.061360in}}%
\pgfpathlineto{\pgfqpoint{4.640637in}{4.064095in}}%
\pgfpathlineto{\pgfqpoint{4.642388in}{4.001045in}}%
\pgfpathlineto{\pgfqpoint{4.643264in}{3.845994in}}%
\pgfpathlineto{\pgfqpoint{4.645014in}{3.456515in}}%
\pgfpathlineto{\pgfqpoint{4.645890in}{3.419708in}}%
\pgfpathlineto{\pgfqpoint{4.646765in}{3.427676in}}%
\pgfpathlineto{\pgfqpoint{4.647640in}{3.831269in}}%
\pgfpathlineto{\pgfqpoint{4.648516in}{3.814316in}}%
\pgfpathlineto{\pgfqpoint{4.650267in}{3.540459in}}%
\pgfpathlineto{\pgfqpoint{4.651142in}{3.449953in}}%
\pgfpathlineto{\pgfqpoint{4.652017in}{3.436851in}}%
\pgfpathlineto{\pgfqpoint{4.652893in}{3.395619in}}%
\pgfpathlineto{\pgfqpoint{4.653768in}{3.433000in}}%
\pgfpathlineto{\pgfqpoint{4.654643in}{3.535245in}}%
\pgfpathlineto{\pgfqpoint{4.656394in}{4.380028in}}%
\pgfpathlineto{\pgfqpoint{4.657270in}{4.593510in}}%
\pgfpathlineto{\pgfqpoint{4.658145in}{4.492895in}}%
\pgfpathlineto{\pgfqpoint{4.659020in}{4.358034in}}%
\pgfpathlineto{\pgfqpoint{4.659896in}{4.435115in}}%
\pgfpathlineto{\pgfqpoint{4.660771in}{4.614073in}}%
\pgfpathlineto{\pgfqpoint{4.661646in}{4.473567in}}%
\pgfpathlineto{\pgfqpoint{4.662522in}{4.509994in}}%
\pgfpathlineto{\pgfqpoint{4.663397in}{4.572964in}}%
\pgfpathlineto{\pgfqpoint{4.664273in}{4.272599in}}%
\pgfpathlineto{\pgfqpoint{4.665148in}{4.145509in}}%
\pgfpathlineto{\pgfqpoint{4.666023in}{3.927650in}}%
\pgfpathlineto{\pgfqpoint{4.666899in}{3.794603in}}%
\pgfpathlineto{\pgfqpoint{4.667774in}{3.814241in}}%
\pgfpathlineto{\pgfqpoint{4.668649in}{3.942844in}}%
\pgfpathlineto{\pgfqpoint{4.670400in}{3.591620in}}%
\pgfpathlineto{\pgfqpoint{4.671276in}{3.572930in}}%
\pgfpathlineto{\pgfqpoint{4.672151in}{3.522977in}}%
\pgfpathlineto{\pgfqpoint{4.673026in}{3.491826in}}%
\pgfpathlineto{\pgfqpoint{4.673902in}{3.489825in}}%
\pgfpathlineto{\pgfqpoint{4.675652in}{3.333772in}}%
\pgfpathlineto{\pgfqpoint{4.676528in}{3.820018in}}%
\pgfpathlineto{\pgfqpoint{4.677403in}{3.861872in}}%
\pgfpathlineto{\pgfqpoint{4.678279in}{4.102569in}}%
\pgfpathlineto{\pgfqpoint{4.679154in}{4.151282in}}%
\pgfpathlineto{\pgfqpoint{4.680905in}{3.832242in}}%
\pgfpathlineto{\pgfqpoint{4.681780in}{3.777479in}}%
\pgfpathlineto{\pgfqpoint{4.682655in}{3.836576in}}%
\pgfpathlineto{\pgfqpoint{4.683531in}{4.178931in}}%
\pgfpathlineto{\pgfqpoint{4.684406in}{3.958763in}}%
\pgfpathlineto{\pgfqpoint{4.685282in}{3.954096in}}%
\pgfpathlineto{\pgfqpoint{4.687032in}{3.425868in}}%
\pgfpathlineto{\pgfqpoint{4.687908in}{3.446960in}}%
\pgfpathlineto{\pgfqpoint{4.688783in}{3.444289in}}%
\pgfpathlineto{\pgfqpoint{4.689658in}{3.242398in}}%
\pgfpathlineto{\pgfqpoint{4.691409in}{3.080191in}}%
\pgfpathlineto{\pgfqpoint{4.692285in}{2.942979in}}%
\pgfpathlineto{\pgfqpoint{4.693160in}{2.865499in}}%
\pgfpathlineto{\pgfqpoint{4.694035in}{2.760495in}}%
\pgfpathlineto{\pgfqpoint{4.695786in}{2.634384in}}%
\pgfpathlineto{\pgfqpoint{4.696661in}{2.541543in}}%
\pgfpathlineto{\pgfqpoint{4.697537in}{2.760663in}}%
\pgfpathlineto{\pgfqpoint{4.698412in}{2.859298in}}%
\pgfpathlineto{\pgfqpoint{4.699288in}{3.134436in}}%
\pgfpathlineto{\pgfqpoint{4.700163in}{3.232807in}}%
\pgfpathlineto{\pgfqpoint{4.701038in}{3.449011in}}%
\pgfpathlineto{\pgfqpoint{4.701914in}{3.181507in}}%
\pgfpathlineto{\pgfqpoint{4.702789in}{3.176172in}}%
\pgfpathlineto{\pgfqpoint{4.703664in}{3.238673in}}%
\pgfpathlineto{\pgfqpoint{4.704540in}{3.392633in}}%
\pgfpathlineto{\pgfqpoint{4.705415in}{3.423893in}}%
\pgfpathlineto{\pgfqpoint{4.706291in}{3.308063in}}%
\pgfpathlineto{\pgfqpoint{4.707166in}{3.271285in}}%
\pgfpathlineto{\pgfqpoint{4.708041in}{3.143557in}}%
\pgfpathlineto{\pgfqpoint{4.708917in}{3.146536in}}%
\pgfpathlineto{\pgfqpoint{4.709792in}{3.244815in}}%
\pgfpathlineto{\pgfqpoint{4.710667in}{3.203621in}}%
\pgfpathlineto{\pgfqpoint{4.711543in}{3.194106in}}%
\pgfpathlineto{\pgfqpoint{4.714169in}{2.948076in}}%
\pgfpathlineto{\pgfqpoint{4.715920in}{2.807654in}}%
\pgfpathlineto{\pgfqpoint{4.716795in}{2.766763in}}%
\pgfpathlineto{\pgfqpoint{4.717670in}{2.815145in}}%
\pgfpathlineto{\pgfqpoint{4.719421in}{3.638942in}}%
\pgfpathlineto{\pgfqpoint{4.720297in}{3.948323in}}%
\pgfpathlineto{\pgfqpoint{4.721172in}{3.905532in}}%
\pgfpathlineto{\pgfqpoint{4.722047in}{3.930907in}}%
\pgfpathlineto{\pgfqpoint{4.722923in}{4.027446in}}%
\pgfpathlineto{\pgfqpoint{4.723798in}{3.837500in}}%
\pgfpathlineto{\pgfqpoint{4.724674in}{4.176944in}}%
\pgfpathlineto{\pgfqpoint{4.726424in}{4.285676in}}%
\pgfpathlineto{\pgfqpoint{4.728175in}{3.782667in}}%
\pgfpathlineto{\pgfqpoint{4.729050in}{3.546587in}}%
\pgfpathlineto{\pgfqpoint{4.729926in}{3.637701in}}%
\pgfpathlineto{\pgfqpoint{4.730801in}{3.960024in}}%
\pgfpathlineto{\pgfqpoint{4.731677in}{3.883375in}}%
\pgfpathlineto{\pgfqpoint{4.732552in}{3.904444in}}%
\pgfpathlineto{\pgfqpoint{4.733427in}{3.771876in}}%
\pgfpathlineto{\pgfqpoint{4.734303in}{3.556430in}}%
\pgfpathlineto{\pgfqpoint{4.735178in}{3.501077in}}%
\pgfpathlineto{\pgfqpoint{4.736053in}{3.363374in}}%
\pgfpathlineto{\pgfqpoint{4.736929in}{3.532907in}}%
\pgfpathlineto{\pgfqpoint{4.737804in}{3.449878in}}%
\pgfpathlineto{\pgfqpoint{4.738680in}{3.561815in}}%
\pgfpathlineto{\pgfqpoint{4.740430in}{4.009904in}}%
\pgfpathlineto{\pgfqpoint{4.741306in}{4.112190in}}%
\pgfpathlineto{\pgfqpoint{4.742181in}{4.045867in}}%
\pgfpathlineto{\pgfqpoint{4.743932in}{4.206401in}}%
\pgfpathlineto{\pgfqpoint{4.744807in}{4.123855in}}%
\pgfpathlineto{\pgfqpoint{4.747433in}{4.500754in}}%
\pgfpathlineto{\pgfqpoint{4.748309in}{4.135730in}}%
\pgfpathlineto{\pgfqpoint{4.750059in}{3.638828in}}%
\pgfpathlineto{\pgfqpoint{4.750935in}{3.644351in}}%
\pgfpathlineto{\pgfqpoint{4.752686in}{3.517540in}}%
\pgfpathlineto{\pgfqpoint{4.754436in}{3.235904in}}%
\pgfpathlineto{\pgfqpoint{4.755312in}{2.886417in}}%
\pgfpathlineto{\pgfqpoint{4.757062in}{2.763780in}}%
\pgfpathlineto{\pgfqpoint{4.757938in}{2.778468in}}%
\pgfpathlineto{\pgfqpoint{4.758813in}{2.823475in}}%
\pgfpathlineto{\pgfqpoint{4.759689in}{2.836801in}}%
\pgfpathlineto{\pgfqpoint{4.760564in}{3.081383in}}%
\pgfpathlineto{\pgfqpoint{4.762315in}{3.723379in}}%
\pgfpathlineto{\pgfqpoint{4.763190in}{3.784188in}}%
\pgfpathlineto{\pgfqpoint{4.764065in}{3.704196in}}%
\pgfpathlineto{\pgfqpoint{4.764941in}{3.670203in}}%
\pgfpathlineto{\pgfqpoint{4.766692in}{3.975937in}}%
\pgfpathlineto{\pgfqpoint{4.767567in}{3.990884in}}%
\pgfpathlineto{\pgfqpoint{4.768442in}{4.103800in}}%
\pgfpathlineto{\pgfqpoint{4.770193in}{3.956684in}}%
\pgfpathlineto{\pgfqpoint{4.771068in}{3.785412in}}%
\pgfpathlineto{\pgfqpoint{4.771944in}{3.754388in}}%
\pgfpathlineto{\pgfqpoint{4.772819in}{3.795286in}}%
\pgfpathlineto{\pgfqpoint{4.773695in}{3.695606in}}%
\pgfpathlineto{\pgfqpoint{4.774570in}{3.648937in}}%
\pgfpathlineto{\pgfqpoint{4.776321in}{3.325805in}}%
\pgfpathlineto{\pgfqpoint{4.777196in}{3.167373in}}%
\pgfpathlineto{\pgfqpoint{4.778947in}{3.028500in}}%
\pgfpathlineto{\pgfqpoint{4.779822in}{3.058933in}}%
\pgfpathlineto{\pgfqpoint{4.780698in}{3.034430in}}%
\pgfpathlineto{\pgfqpoint{4.783324in}{4.476046in}}%
\pgfpathlineto{\pgfqpoint{4.784199in}{4.647500in}}%
\pgfpathlineto{\pgfqpoint{4.785074in}{4.627801in}}%
\pgfpathlineto{\pgfqpoint{4.786825in}{4.701780in}}%
\pgfpathlineto{\pgfqpoint{4.787701in}{4.919848in}}%
\pgfpathlineto{\pgfqpoint{4.788576in}{5.056253in}}%
\pgfpathlineto{\pgfqpoint{4.792077in}{4.348379in}}%
\pgfpathlineto{\pgfqpoint{4.792953in}{4.105693in}}%
\pgfpathlineto{\pgfqpoint{4.793828in}{4.011865in}}%
\pgfpathlineto{\pgfqpoint{4.794704in}{4.107581in}}%
\pgfpathlineto{\pgfqpoint{4.796454in}{3.886207in}}%
\pgfpathlineto{\pgfqpoint{4.797330in}{3.690168in}}%
\pgfpathlineto{\pgfqpoint{4.798205in}{3.725661in}}%
\pgfpathlineto{\pgfqpoint{4.799080in}{3.667023in}}%
\pgfpathlineto{\pgfqpoint{4.799956in}{3.539175in}}%
\pgfpathlineto{\pgfqpoint{4.800831in}{3.565756in}}%
\pgfpathlineto{\pgfqpoint{4.801707in}{3.800576in}}%
\pgfpathlineto{\pgfqpoint{4.802582in}{4.430044in}}%
\pgfpathlineto{\pgfqpoint{4.805208in}{5.224722in}}%
\pgfpathlineto{\pgfqpoint{4.806083in}{4.907883in}}%
\pgfpathlineto{\pgfqpoint{4.806959in}{4.982463in}}%
\pgfpathlineto{\pgfqpoint{4.807834in}{4.946237in}}%
\pgfpathlineto{\pgfqpoint{4.808710in}{5.072376in}}%
\pgfpathlineto{\pgfqpoint{4.809585in}{5.255919in}}%
\pgfpathlineto{\pgfqpoint{4.811336in}{4.846430in}}%
\pgfpathlineto{\pgfqpoint{4.813086in}{4.256862in}}%
\pgfpathlineto{\pgfqpoint{4.814837in}{4.418630in}}%
\pgfpathlineto{\pgfqpoint{4.815713in}{4.602586in}}%
\pgfpathlineto{\pgfqpoint{4.816588in}{4.473795in}}%
\pgfpathlineto{\pgfqpoint{4.817463in}{4.296408in}}%
\pgfpathlineto{\pgfqpoint{4.818339in}{4.032443in}}%
\pgfpathlineto{\pgfqpoint{4.819214in}{3.899045in}}%
\pgfpathlineto{\pgfqpoint{4.820965in}{3.738196in}}%
\pgfpathlineto{\pgfqpoint{4.821840in}{3.754394in}}%
\pgfpathlineto{\pgfqpoint{4.822716in}{3.759204in}}%
\pgfpathlineto{\pgfqpoint{4.825342in}{4.825389in}}%
\pgfpathlineto{\pgfqpoint{4.826217in}{4.671875in}}%
\pgfpathlineto{\pgfqpoint{4.827092in}{4.299400in}}%
\pgfpathlineto{\pgfqpoint{4.827968in}{4.525477in}}%
\pgfpathlineto{\pgfqpoint{4.828843in}{4.441921in}}%
\pgfpathlineto{\pgfqpoint{4.829719in}{4.074084in}}%
\pgfpathlineto{\pgfqpoint{4.830594in}{4.139129in}}%
\pgfpathlineto{\pgfqpoint{4.831469in}{4.373131in}}%
\pgfpathlineto{\pgfqpoint{4.832345in}{4.176866in}}%
\pgfpathlineto{\pgfqpoint{4.833220in}{4.094857in}}%
\pgfpathlineto{\pgfqpoint{4.834095in}{3.806425in}}%
\pgfpathlineto{\pgfqpoint{4.834971in}{3.705876in}}%
\pgfpathlineto{\pgfqpoint{4.835846in}{3.698249in}}%
\pgfpathlineto{\pgfqpoint{4.836722in}{3.597473in}}%
\pgfpathlineto{\pgfqpoint{4.837597in}{3.603174in}}%
\pgfpathlineto{\pgfqpoint{4.838472in}{3.519125in}}%
\pgfpathlineto{\pgfqpoint{4.839348in}{3.285178in}}%
\pgfpathlineto{\pgfqpoint{4.841098in}{3.062709in}}%
\pgfpathlineto{\pgfqpoint{4.841974in}{3.049493in}}%
\pgfpathlineto{\pgfqpoint{4.842849in}{3.028500in}}%
\pgfpathlineto{\pgfqpoint{4.843725in}{2.952059in}}%
\pgfpathlineto{\pgfqpoint{4.844600in}{3.455594in}}%
\pgfpathlineto{\pgfqpoint{4.845475in}{3.600380in}}%
\pgfpathlineto{\pgfqpoint{4.846351in}{3.691674in}}%
\pgfpathlineto{\pgfqpoint{4.848101in}{3.782358in}}%
\pgfpathlineto{\pgfqpoint{4.848977in}{3.813174in}}%
\pgfpathlineto{\pgfqpoint{4.849852in}{4.113318in}}%
\pgfpathlineto{\pgfqpoint{4.851603in}{3.930351in}}%
\pgfpathlineto{\pgfqpoint{4.852478in}{4.107727in}}%
\pgfpathlineto{\pgfqpoint{4.855980in}{3.400648in}}%
\pgfpathlineto{\pgfqpoint{4.856855in}{3.516520in}}%
\pgfpathlineto{\pgfqpoint{4.857731in}{3.969274in}}%
\pgfpathlineto{\pgfqpoint{4.858606in}{3.935103in}}%
\pgfpathlineto{\pgfqpoint{4.859481in}{3.719922in}}%
\pgfpathlineto{\pgfqpoint{4.861232in}{3.491977in}}%
\pgfpathlineto{\pgfqpoint{4.862108in}{3.272869in}}%
\pgfpathlineto{\pgfqpoint{4.862983in}{3.209511in}}%
\pgfpathlineto{\pgfqpoint{4.863858in}{3.237867in}}%
\pgfpathlineto{\pgfqpoint{4.864734in}{3.349752in}}%
\pgfpathlineto{\pgfqpoint{4.865609in}{3.735064in}}%
\pgfpathlineto{\pgfqpoint{4.866484in}{3.842398in}}%
\pgfpathlineto{\pgfqpoint{4.867360in}{4.209587in}}%
\pgfpathlineto{\pgfqpoint{4.868235in}{4.323558in}}%
\pgfpathlineto{\pgfqpoint{4.869111in}{3.992147in}}%
\pgfpathlineto{\pgfqpoint{4.869986in}{4.044880in}}%
\pgfpathlineto{\pgfqpoint{4.870861in}{4.251546in}}%
\pgfpathlineto{\pgfqpoint{4.871737in}{4.220517in}}%
\pgfpathlineto{\pgfqpoint{4.872612in}{4.405149in}}%
\pgfpathlineto{\pgfqpoint{4.873487in}{4.194840in}}%
\pgfpathlineto{\pgfqpoint{4.875238in}{3.948092in}}%
\pgfpathlineto{\pgfqpoint{4.876114in}{3.736799in}}%
\pgfpathlineto{\pgfqpoint{4.876989in}{3.777464in}}%
\pgfpathlineto{\pgfqpoint{4.877864in}{3.761115in}}%
\pgfpathlineto{\pgfqpoint{4.878740in}{4.045281in}}%
\pgfpathlineto{\pgfqpoint{4.879615in}{4.023268in}}%
\pgfpathlineto{\pgfqpoint{4.880490in}{3.951906in}}%
\pgfpathlineto{\pgfqpoint{4.881366in}{3.670950in}}%
\pgfpathlineto{\pgfqpoint{4.882241in}{3.483973in}}%
\pgfpathlineto{\pgfqpoint{4.883117in}{3.416613in}}%
\pgfpathlineto{\pgfqpoint{4.883992in}{3.372059in}}%
\pgfpathlineto{\pgfqpoint{4.884867in}{3.404455in}}%
\pgfpathlineto{\pgfqpoint{4.887493in}{3.873936in}}%
\pgfpathlineto{\pgfqpoint{4.888369in}{3.921888in}}%
\pgfpathlineto{\pgfqpoint{4.889244in}{4.044790in}}%
\pgfpathlineto{\pgfqpoint{4.890120in}{4.002879in}}%
\pgfpathlineto{\pgfqpoint{4.890995in}{4.051662in}}%
\pgfpathlineto{\pgfqpoint{4.891870in}{3.946242in}}%
\pgfpathlineto{\pgfqpoint{4.892746in}{3.950546in}}%
\pgfpathlineto{\pgfqpoint{4.893621in}{3.929213in}}%
\pgfpathlineto{\pgfqpoint{4.894496in}{3.971313in}}%
\pgfpathlineto{\pgfqpoint{4.895372in}{3.800648in}}%
\pgfpathlineto{\pgfqpoint{4.896247in}{3.722867in}}%
\pgfpathlineto{\pgfqpoint{4.897123in}{3.618202in}}%
\pgfpathlineto{\pgfqpoint{4.897998in}{3.963422in}}%
\pgfpathlineto{\pgfqpoint{4.899749in}{4.082132in}}%
\pgfpathlineto{\pgfqpoint{4.900624in}{3.830967in}}%
\pgfpathlineto{\pgfqpoint{4.901499in}{3.476006in}}%
\pgfpathlineto{\pgfqpoint{4.904126in}{2.909676in}}%
\pgfpathlineto{\pgfqpoint{4.905001in}{2.831366in}}%
\pgfpathlineto{\pgfqpoint{4.905876in}{2.925950in}}%
\pgfpathlineto{\pgfqpoint{4.906752in}{2.965822in}}%
\pgfpathlineto{\pgfqpoint{4.907627in}{3.358050in}}%
\pgfpathlineto{\pgfqpoint{4.908502in}{3.493337in}}%
\pgfpathlineto{\pgfqpoint{4.909378in}{3.804839in}}%
\pgfpathlineto{\pgfqpoint{4.910253in}{3.765986in}}%
\pgfpathlineto{\pgfqpoint{4.911129in}{3.815034in}}%
\pgfpathlineto{\pgfqpoint{4.912004in}{3.882696in}}%
\pgfpathlineto{\pgfqpoint{4.912879in}{3.647011in}}%
\pgfpathlineto{\pgfqpoint{4.913755in}{3.800421in}}%
\pgfpathlineto{\pgfqpoint{4.914630in}{3.757566in}}%
\pgfpathlineto{\pgfqpoint{4.915505in}{3.864798in}}%
\pgfpathlineto{\pgfqpoint{4.916381in}{3.836140in}}%
\pgfpathlineto{\pgfqpoint{4.917256in}{3.560281in}}%
\pgfpathlineto{\pgfqpoint{4.918132in}{3.557676in}}%
\pgfpathlineto{\pgfqpoint{4.919882in}{3.991627in}}%
\pgfpathlineto{\pgfqpoint{4.920758in}{3.876654in}}%
\pgfpathlineto{\pgfqpoint{4.921633in}{3.811296in}}%
\pgfpathlineto{\pgfqpoint{4.922508in}{3.683750in}}%
\pgfpathlineto{\pgfqpoint{4.923384in}{3.647880in}}%
\pgfpathlineto{\pgfqpoint{4.924259in}{3.553070in}}%
\pgfpathlineto{\pgfqpoint{4.925135in}{3.579991in}}%
\pgfpathlineto{\pgfqpoint{4.926010in}{3.562962in}}%
\pgfpathlineto{\pgfqpoint{4.926885in}{3.613558in}}%
\pgfpathlineto{\pgfqpoint{4.927761in}{3.569872in}}%
\pgfpathlineto{\pgfqpoint{4.928636in}{4.108790in}}%
\pgfpathlineto{\pgfqpoint{4.929511in}{4.185174in}}%
\pgfpathlineto{\pgfqpoint{4.930387in}{4.218136in}}%
\pgfpathlineto{\pgfqpoint{4.931262in}{4.331070in}}%
\pgfpathlineto{\pgfqpoint{4.933013in}{4.028441in}}%
\pgfpathlineto{\pgfqpoint{4.933888in}{4.062272in}}%
\pgfpathlineto{\pgfqpoint{4.934764in}{4.128499in}}%
\pgfpathlineto{\pgfqpoint{4.935639in}{4.338206in}}%
\pgfpathlineto{\pgfqpoint{4.938265in}{3.970294in}}%
\pgfpathlineto{\pgfqpoint{4.939141in}{3.988304in}}%
\pgfpathlineto{\pgfqpoint{4.940016in}{4.018397in}}%
\pgfpathlineto{\pgfqpoint{4.940891in}{4.251174in}}%
\pgfpathlineto{\pgfqpoint{4.941767in}{4.220213in}}%
\pgfpathlineto{\pgfqpoint{4.942642in}{4.277529in}}%
\pgfpathlineto{\pgfqpoint{4.943517in}{4.209339in}}%
\pgfpathlineto{\pgfqpoint{4.944393in}{3.933518in}}%
\pgfpathlineto{\pgfqpoint{4.945268in}{3.766213in}}%
\pgfpathlineto{\pgfqpoint{4.946144in}{3.684542in}}%
\pgfpathlineto{\pgfqpoint{4.947019in}{3.542535in}}%
\pgfpathlineto{\pgfqpoint{4.947894in}{3.625527in}}%
\pgfpathlineto{\pgfqpoint{4.948770in}{3.753413in}}%
\pgfpathlineto{\pgfqpoint{4.951396in}{5.087057in}}%
\pgfpathlineto{\pgfqpoint{4.952271in}{4.975445in}}%
\pgfpathlineto{\pgfqpoint{4.953147in}{4.914882in}}%
\pgfpathlineto{\pgfqpoint{4.954022in}{4.913145in}}%
\pgfpathlineto{\pgfqpoint{4.954897in}{4.878445in}}%
\pgfpathlineto{\pgfqpoint{4.955773in}{4.945239in}}%
\pgfpathlineto{\pgfqpoint{4.956648in}{4.689392in}}%
\pgfpathlineto{\pgfqpoint{4.957523in}{4.561506in}}%
\pgfpathlineto{\pgfqpoint{4.958399in}{4.495694in}}%
\pgfpathlineto{\pgfqpoint{4.961025in}{4.096178in}}%
\pgfpathlineto{\pgfqpoint{4.961900in}{4.099916in}}%
\pgfpathlineto{\pgfqpoint{4.962776in}{3.972899in}}%
\pgfpathlineto{\pgfqpoint{4.963651in}{4.010921in}}%
\pgfpathlineto{\pgfqpoint{4.965402in}{3.772065in}}%
\pgfpathlineto{\pgfqpoint{4.968028in}{3.471211in}}%
\pgfpathlineto{\pgfqpoint{4.968903in}{3.413781in}}%
\pgfpathlineto{\pgfqpoint{4.969779in}{3.475100in}}%
\pgfpathlineto{\pgfqpoint{4.970654in}{3.887491in}}%
\pgfpathlineto{\pgfqpoint{4.972405in}{4.280739in}}%
\pgfpathlineto{\pgfqpoint{4.973280in}{4.283117in}}%
\pgfpathlineto{\pgfqpoint{4.975031in}{4.330315in}}%
\pgfpathlineto{\pgfqpoint{4.975906in}{4.534887in}}%
\pgfpathlineto{\pgfqpoint{4.976782in}{4.555654in}}%
\pgfpathlineto{\pgfqpoint{4.977657in}{4.291386in}}%
\pgfpathlineto{\pgfqpoint{4.978532in}{4.231842in}}%
\pgfpathlineto{\pgfqpoint{4.980283in}{3.869669in}}%
\pgfpathlineto{\pgfqpoint{4.981159in}{3.514632in}}%
\pgfpathlineto{\pgfqpoint{4.982034in}{3.697984in}}%
\pgfpathlineto{\pgfqpoint{4.983785in}{3.600154in}}%
\pgfpathlineto{\pgfqpoint{4.984660in}{3.500851in}}%
\pgfpathlineto{\pgfqpoint{4.985535in}{3.465547in}}%
\pgfpathlineto{\pgfqpoint{4.986411in}{3.466982in}}%
\pgfpathlineto{\pgfqpoint{4.987286in}{3.436096in}}%
\pgfpathlineto{\pgfqpoint{4.989037in}{3.203055in}}%
\pgfpathlineto{\pgfqpoint{4.989912in}{3.152006in}}%
\pgfpathlineto{\pgfqpoint{4.993414in}{4.190800in}}%
\pgfpathlineto{\pgfqpoint{4.994289in}{4.149606in}}%
\pgfpathlineto{\pgfqpoint{4.996040in}{4.016849in}}%
\pgfpathlineto{\pgfqpoint{4.996915in}{4.200277in}}%
\pgfpathlineto{\pgfqpoint{4.997791in}{4.224517in}}%
\pgfpathlineto{\pgfqpoint{4.998666in}{4.405113in}}%
\pgfpathlineto{\pgfqpoint{4.999542in}{4.434338in}}%
\pgfpathlineto{\pgfqpoint{5.000417in}{4.353423in}}%
\pgfpathlineto{\pgfqpoint{5.002168in}{3.905917in}}%
\pgfpathlineto{\pgfqpoint{5.003043in}{4.066878in}}%
\pgfpathlineto{\pgfqpoint{5.003918in}{4.071825in}}%
\pgfpathlineto{\pgfqpoint{5.004794in}{4.064839in}}%
\pgfpathlineto{\pgfqpoint{5.005669in}{4.036710in}}%
\pgfpathlineto{\pgfqpoint{5.006545in}{4.095046in}}%
\pgfpathlineto{\pgfqpoint{5.008295in}{3.550502in}}%
\pgfpathlineto{\pgfqpoint{5.010046in}{3.404040in}}%
\pgfpathlineto{\pgfqpoint{5.010921in}{3.450821in}}%
\pgfpathlineto{\pgfqpoint{5.011797in}{3.554807in}}%
\pgfpathlineto{\pgfqpoint{5.015298in}{4.183890in}}%
\pgfpathlineto{\pgfqpoint{5.016174in}{4.159838in}}%
\pgfpathlineto{\pgfqpoint{5.017049in}{4.266655in}}%
\pgfpathlineto{\pgfqpoint{5.017924in}{4.066312in}}%
\pgfpathlineto{\pgfqpoint{5.018800in}{4.172865in}}%
\pgfpathlineto{\pgfqpoint{5.019675in}{4.235127in}}%
\pgfpathlineto{\pgfqpoint{5.022301in}{3.920642in}}%
\pgfpathlineto{\pgfqpoint{5.023177in}{3.852754in}}%
\pgfpathlineto{\pgfqpoint{5.024052in}{3.857662in}}%
\pgfpathlineto{\pgfqpoint{5.024927in}{3.968557in}}%
\pgfpathlineto{\pgfqpoint{5.026678in}{3.937256in}}%
\pgfpathlineto{\pgfqpoint{5.027554in}{3.953189in}}%
\pgfpathlineto{\pgfqpoint{5.028429in}{3.755112in}}%
\pgfpathlineto{\pgfqpoint{5.030180in}{3.537967in}}%
\pgfpathlineto{\pgfqpoint{5.031055in}{3.435681in}}%
\pgfpathlineto{\pgfqpoint{5.032806in}{3.727133in}}%
\pgfpathlineto{\pgfqpoint{5.035432in}{4.498979in}}%
\pgfpathlineto{\pgfqpoint{5.036307in}{4.474889in}}%
\pgfpathlineto{\pgfqpoint{5.037183in}{4.470056in}}%
\pgfpathlineto{\pgfqpoint{5.039809in}{4.213265in}}%
\pgfpathlineto{\pgfqpoint{5.041560in}{4.315438in}}%
\pgfpathlineto{\pgfqpoint{5.043310in}{3.938200in}}%
\pgfpathlineto{\pgfqpoint{5.044186in}{3.798684in}}%
\pgfpathlineto{\pgfqpoint{5.045061in}{3.782675in}}%
\pgfpathlineto{\pgfqpoint{5.045936in}{3.891871in}}%
\pgfpathlineto{\pgfqpoint{5.046812in}{3.874087in}}%
\pgfpathlineto{\pgfqpoint{5.047687in}{3.889605in}}%
\pgfpathlineto{\pgfqpoint{5.048563in}{3.816166in}}%
\pgfpathlineto{\pgfqpoint{5.050313in}{3.592678in}}%
\pgfpathlineto{\pgfqpoint{5.051189in}{3.580859in}}%
\pgfpathlineto{\pgfqpoint{5.052064in}{3.526941in}}%
\pgfpathlineto{\pgfqpoint{5.052939in}{3.501077in}}%
\pgfpathlineto{\pgfqpoint{5.053815in}{3.436322in}}%
\pgfpathlineto{\pgfqpoint{5.054690in}{3.698664in}}%
\pgfpathlineto{\pgfqpoint{5.055566in}{3.729890in}}%
\pgfpathlineto{\pgfqpoint{5.056441in}{3.796797in}}%
\pgfpathlineto{\pgfqpoint{5.057316in}{3.901159in}}%
\pgfpathlineto{\pgfqpoint{5.058192in}{3.866007in}}%
\pgfpathlineto{\pgfqpoint{5.059067in}{3.894325in}}%
\pgfpathlineto{\pgfqpoint{5.059942in}{3.803404in}}%
\pgfpathlineto{\pgfqpoint{5.060818in}{4.190158in}}%
\pgfpathlineto{\pgfqpoint{5.061693in}{4.255441in}}%
\pgfpathlineto{\pgfqpoint{5.062569in}{4.075449in}}%
\pgfpathlineto{\pgfqpoint{5.063444in}{4.133294in}}%
\pgfpathlineto{\pgfqpoint{5.065195in}{3.895835in}}%
\pgfpathlineto{\pgfqpoint{5.066070in}{3.828211in}}%
\pgfpathlineto{\pgfqpoint{5.066945in}{3.975127in}}%
\pgfpathlineto{\pgfqpoint{5.067821in}{3.736006in}}%
\pgfpathlineto{\pgfqpoint{5.068696in}{3.786602in}}%
\pgfpathlineto{\pgfqpoint{5.069572in}{3.820508in}}%
\pgfpathlineto{\pgfqpoint{5.070447in}{3.821113in}}%
\pgfpathlineto{\pgfqpoint{5.072198in}{3.663436in}}%
\pgfpathlineto{\pgfqpoint{5.073073in}{3.605629in}}%
\pgfpathlineto{\pgfqpoint{5.073948in}{3.598681in}}%
\pgfpathlineto{\pgfqpoint{5.074824in}{3.640479in}}%
\pgfpathlineto{\pgfqpoint{5.077450in}{4.415761in}}%
\pgfpathlineto{\pgfqpoint{5.078325in}{4.466243in}}%
\pgfpathlineto{\pgfqpoint{5.079201in}{4.642761in}}%
\pgfpathlineto{\pgfqpoint{5.080076in}{4.689052in}}%
\pgfpathlineto{\pgfqpoint{5.082702in}{4.492409in}}%
\pgfpathlineto{\pgfqpoint{5.083578in}{4.734361in}}%
\pgfpathlineto{\pgfqpoint{5.084453in}{4.532319in}}%
\pgfpathlineto{\pgfqpoint{5.085328in}{4.215871in}}%
\pgfpathlineto{\pgfqpoint{5.086204in}{4.089797in}}%
\pgfpathlineto{\pgfqpoint{5.087079in}{4.061441in}}%
\pgfpathlineto{\pgfqpoint{5.087954in}{4.155081in}}%
\pgfpathlineto{\pgfqpoint{5.088830in}{4.079489in}}%
\pgfpathlineto{\pgfqpoint{5.089705in}{4.075147in}}%
\pgfpathlineto{\pgfqpoint{5.090581in}{3.877221in}}%
\pgfpathlineto{\pgfqpoint{5.093207in}{3.729550in}}%
\pgfpathlineto{\pgfqpoint{5.094082in}{3.682692in}}%
\pgfpathlineto{\pgfqpoint{5.094957in}{3.684014in}}%
\pgfpathlineto{\pgfqpoint{5.095833in}{3.701269in}}%
\pgfpathlineto{\pgfqpoint{5.096708in}{3.966896in}}%
\pgfpathlineto{\pgfqpoint{5.098459in}{4.654352in}}%
\pgfpathlineto{\pgfqpoint{5.099334in}{4.491390in}}%
\pgfpathlineto{\pgfqpoint{5.100210in}{4.627129in}}%
\pgfpathlineto{\pgfqpoint{5.101085in}{4.696415in}}%
\pgfpathlineto{\pgfqpoint{5.101960in}{4.719107in}}%
\pgfpathlineto{\pgfqpoint{5.102836in}{4.512987in}}%
\pgfpathlineto{\pgfqpoint{5.104587in}{4.289914in}}%
\pgfpathlineto{\pgfqpoint{5.105462in}{4.175130in}}%
\pgfpathlineto{\pgfqpoint{5.106337in}{3.846108in}}%
\pgfpathlineto{\pgfqpoint{5.108088in}{3.595434in}}%
\pgfpathlineto{\pgfqpoint{5.108963in}{3.934575in}}%
\pgfpathlineto{\pgfqpoint{5.109839in}{3.819036in}}%
\pgfpathlineto{\pgfqpoint{5.110714in}{3.775803in}}%
\pgfpathlineto{\pgfqpoint{5.111590in}{3.802007in}}%
\pgfpathlineto{\pgfqpoint{5.112465in}{3.624621in}}%
\pgfpathlineto{\pgfqpoint{5.114216in}{3.474458in}}%
\pgfpathlineto{\pgfqpoint{5.115091in}{3.435227in}}%
\pgfpathlineto{\pgfqpoint{5.116842in}{3.491600in}}%
\pgfpathlineto{\pgfqpoint{5.117717in}{3.952661in}}%
\pgfpathlineto{\pgfqpoint{5.119468in}{4.546176in}}%
\pgfpathlineto{\pgfqpoint{5.120343in}{4.561884in}}%
\pgfpathlineto{\pgfqpoint{5.121219in}{4.450120in}}%
\pgfpathlineto{\pgfqpoint{5.122094in}{4.383553in}}%
\pgfpathlineto{\pgfqpoint{5.122969in}{4.266731in}}%
\pgfpathlineto{\pgfqpoint{5.123845in}{4.022135in}}%
\pgfpathlineto{\pgfqpoint{5.124720in}{4.129217in}}%
\pgfpathlineto{\pgfqpoint{5.127346in}{3.358881in}}%
\pgfpathlineto{\pgfqpoint{5.128222in}{3.313119in}}%
\pgfpathlineto{\pgfqpoint{5.129097in}{3.323238in}}%
\pgfpathlineto{\pgfqpoint{5.129973in}{3.380252in}}%
\pgfpathlineto{\pgfqpoint{5.130848in}{3.273020in}}%
\pgfpathlineto{\pgfqpoint{5.131723in}{3.390182in}}%
\pgfpathlineto{\pgfqpoint{5.132599in}{3.423334in}}%
\pgfpathlineto{\pgfqpoint{5.133474in}{3.331129in}}%
\pgfpathlineto{\pgfqpoint{5.134349in}{3.321086in}}%
\pgfpathlineto{\pgfqpoint{5.135225in}{3.337208in}}%
\pgfpathlineto{\pgfqpoint{5.136100in}{3.381234in}}%
\pgfpathlineto{\pgfqpoint{5.136976in}{3.374551in}}%
\pgfpathlineto{\pgfqpoint{5.137851in}{3.431678in}}%
\pgfpathlineto{\pgfqpoint{5.138726in}{3.730607in}}%
\pgfpathlineto{\pgfqpoint{5.139602in}{3.785394in}}%
\pgfpathlineto{\pgfqpoint{5.140477in}{3.721583in}}%
\pgfpathlineto{\pgfqpoint{5.141352in}{3.766213in}}%
\pgfpathlineto{\pgfqpoint{5.142228in}{3.742387in}}%
\pgfpathlineto{\pgfqpoint{5.143103in}{3.703874in}}%
\pgfpathlineto{\pgfqpoint{5.143979in}{3.845693in}}%
\pgfpathlineto{\pgfqpoint{5.144854in}{3.651014in}}%
\pgfpathlineto{\pgfqpoint{5.145729in}{3.689338in}}%
\pgfpathlineto{\pgfqpoint{5.146605in}{3.711615in}}%
\pgfpathlineto{\pgfqpoint{5.147480in}{3.539288in}}%
\pgfpathlineto{\pgfqpoint{5.148355in}{3.293787in}}%
\pgfpathlineto{\pgfqpoint{5.149231in}{3.163598in}}%
\pgfpathlineto{\pgfqpoint{5.150106in}{3.141962in}}%
\pgfpathlineto{\pgfqpoint{5.151857in}{3.412120in}}%
\pgfpathlineto{\pgfqpoint{5.152732in}{3.530679in}}%
\pgfpathlineto{\pgfqpoint{5.153608in}{3.471060in}}%
\pgfpathlineto{\pgfqpoint{5.154483in}{3.445611in}}%
\pgfpathlineto{\pgfqpoint{5.155358in}{3.357409in}}%
\pgfpathlineto{\pgfqpoint{5.156234in}{3.319764in}}%
\pgfpathlineto{\pgfqpoint{5.157109in}{3.203470in}}%
\pgfpathlineto{\pgfqpoint{5.157985in}{3.187461in}}%
\pgfpathlineto{\pgfqpoint{5.158860in}{3.164239in}}%
\pgfpathlineto{\pgfqpoint{5.161486in}{3.921737in}}%
\pgfpathlineto{\pgfqpoint{5.162361in}{4.018850in}}%
\pgfpathlineto{\pgfqpoint{5.163237in}{3.911467in}}%
\pgfpathlineto{\pgfqpoint{5.164112in}{3.717769in}}%
\pgfpathlineto{\pgfqpoint{5.164988in}{3.699457in}}%
\pgfpathlineto{\pgfqpoint{5.165863in}{3.739556in}}%
\pgfpathlineto{\pgfqpoint{5.167614in}{3.928609in}}%
\pgfpathlineto{\pgfqpoint{5.170240in}{3.263618in}}%
\pgfpathlineto{\pgfqpoint{5.171115in}{3.370926in}}%
\pgfpathlineto{\pgfqpoint{5.171991in}{3.400490in}}%
\pgfpathlineto{\pgfqpoint{5.172866in}{3.418085in}}%
\pgfpathlineto{\pgfqpoint{5.173741in}{3.570740in}}%
\pgfpathlineto{\pgfqpoint{5.174617in}{3.547972in}}%
\pgfpathlineto{\pgfqpoint{5.176367in}{3.303528in}}%
\pgfpathlineto{\pgfqpoint{5.177243in}{3.283554in}}%
\pgfpathlineto{\pgfqpoint{5.178118in}{3.288010in}}%
\pgfpathlineto{\pgfqpoint{5.178994in}{3.244513in}}%
\pgfpathlineto{\pgfqpoint{5.179869in}{3.423938in}}%
\pgfpathlineto{\pgfqpoint{5.180744in}{3.691981in}}%
\pgfpathlineto{\pgfqpoint{5.181620in}{3.814505in}}%
\pgfpathlineto{\pgfqpoint{5.183370in}{4.126271in}}%
\pgfpathlineto{\pgfqpoint{5.184246in}{4.110866in}}%
\pgfpathlineto{\pgfqpoint{5.185121in}{3.925211in}}%
\pgfpathlineto{\pgfqpoint{5.185997in}{3.902330in}}%
\pgfpathlineto{\pgfqpoint{5.186872in}{3.928798in}}%
\pgfpathlineto{\pgfqpoint{5.188623in}{3.840029in}}%
\pgfpathlineto{\pgfqpoint{5.189498in}{3.642631in}}%
\pgfpathlineto{\pgfqpoint{5.191249in}{3.354539in}}%
\pgfpathlineto{\pgfqpoint{5.192124in}{3.351103in}}%
\pgfpathlineto{\pgfqpoint{5.193000in}{3.539779in}}%
\pgfpathlineto{\pgfqpoint{5.193875in}{3.491336in}}%
\pgfpathlineto{\pgfqpoint{5.194750in}{3.558771in}}%
\pgfpathlineto{\pgfqpoint{5.195626in}{3.504513in}}%
\pgfpathlineto{\pgfqpoint{5.198252in}{3.096351in}}%
\pgfpathlineto{\pgfqpoint{5.199127in}{3.129729in}}%
\pgfpathlineto{\pgfqpoint{5.200003in}{3.119194in}}%
\pgfpathlineto{\pgfqpoint{5.200878in}{3.190783in}}%
\pgfpathlineto{\pgfqpoint{5.201753in}{3.427034in}}%
\pgfpathlineto{\pgfqpoint{5.202629in}{3.422126in}}%
\pgfpathlineto{\pgfqpoint{5.203504in}{3.618126in}}%
\pgfpathlineto{\pgfqpoint{5.204379in}{3.482160in}}%
\pgfpathlineto{\pgfqpoint{5.205255in}{3.400755in}}%
\pgfpathlineto{\pgfqpoint{5.207006in}{3.454220in}}%
\pgfpathlineto{\pgfqpoint{5.207881in}{3.599096in}}%
\pgfpathlineto{\pgfqpoint{5.208756in}{3.686846in}}%
\pgfpathlineto{\pgfqpoint{5.209632in}{3.727926in}}%
\pgfpathlineto{\pgfqpoint{5.211382in}{3.382895in}}%
\pgfpathlineto{\pgfqpoint{5.212258in}{3.378666in}}%
\pgfpathlineto{\pgfqpoint{5.213133in}{3.353331in}}%
\pgfpathlineto{\pgfqpoint{5.214009in}{3.584673in}}%
\pgfpathlineto{\pgfqpoint{5.214884in}{3.587203in}}%
\pgfpathlineto{\pgfqpoint{5.215759in}{3.556808in}}%
\pgfpathlineto{\pgfqpoint{5.216635in}{3.434208in}}%
\pgfpathlineto{\pgfqpoint{5.217510in}{3.384368in}}%
\pgfpathlineto{\pgfqpoint{5.218385in}{3.267771in}}%
\pgfpathlineto{\pgfqpoint{5.219261in}{3.032087in}}%
\pgfpathlineto{\pgfqpoint{5.220136in}{3.061312in}}%
\pgfpathlineto{\pgfqpoint{5.221012in}{3.054515in}}%
\pgfpathlineto{\pgfqpoint{5.221887in}{3.179758in}}%
\pgfpathlineto{\pgfqpoint{5.223638in}{3.819187in}}%
\pgfpathlineto{\pgfqpoint{5.225388in}{4.240602in}}%
\pgfpathlineto{\pgfqpoint{5.226264in}{4.057439in}}%
\pgfpathlineto{\pgfqpoint{5.227139in}{4.250683in}}%
\pgfpathlineto{\pgfqpoint{5.228015in}{4.310341in}}%
\pgfpathlineto{\pgfqpoint{5.228890in}{4.255856in}}%
\pgfpathlineto{\pgfqpoint{5.229765in}{4.421802in}}%
\pgfpathlineto{\pgfqpoint{5.230641in}{4.240338in}}%
\pgfpathlineto{\pgfqpoint{5.231516in}{4.393144in}}%
\pgfpathlineto{\pgfqpoint{5.232391in}{4.185891in}}%
\pgfpathlineto{\pgfqpoint{5.233267in}{4.102333in}}%
\pgfpathlineto{\pgfqpoint{5.234142in}{4.120532in}}%
\pgfpathlineto{\pgfqpoint{5.235018in}{4.206998in}}%
\pgfpathlineto{\pgfqpoint{5.235893in}{4.219495in}}%
\pgfpathlineto{\pgfqpoint{5.236768in}{4.272205in}}%
\pgfpathlineto{\pgfqpoint{5.237644in}{4.215229in}}%
\pgfpathlineto{\pgfqpoint{5.238519in}{4.042487in}}%
\pgfpathlineto{\pgfqpoint{5.239394in}{3.799477in}}%
\pgfpathlineto{\pgfqpoint{5.240270in}{3.724792in}}%
\pgfpathlineto{\pgfqpoint{5.241145in}{3.753451in}}%
\pgfpathlineto{\pgfqpoint{5.242021in}{3.688809in}}%
\pgfpathlineto{\pgfqpoint{5.242896in}{3.867517in}}%
\pgfpathlineto{\pgfqpoint{5.243771in}{4.294029in}}%
\pgfpathlineto{\pgfqpoint{5.245522in}{4.611460in}}%
\pgfpathlineto{\pgfqpoint{5.246397in}{4.557957in}}%
\pgfpathlineto{\pgfqpoint{5.248148in}{4.853903in}}%
\pgfpathlineto{\pgfqpoint{5.249024in}{4.858434in}}%
\pgfpathlineto{\pgfqpoint{5.249899in}{4.912125in}}%
\pgfpathlineto{\pgfqpoint{5.250774in}{4.836156in}}%
\pgfpathlineto{\pgfqpoint{5.251650in}{4.634832in}}%
\pgfpathlineto{\pgfqpoint{5.252525in}{4.595375in}}%
\pgfpathlineto{\pgfqpoint{5.254276in}{4.062574in}}%
\pgfpathlineto{\pgfqpoint{5.255151in}{3.905803in}}%
\pgfpathlineto{\pgfqpoint{5.256027in}{3.883677in}}%
\pgfpathlineto{\pgfqpoint{5.256902in}{4.239016in}}%
\pgfpathlineto{\pgfqpoint{5.257777in}{4.240036in}}%
\pgfpathlineto{\pgfqpoint{5.258653in}{4.163803in}}%
\pgfpathlineto{\pgfqpoint{5.259528in}{3.888963in}}%
\pgfpathlineto{\pgfqpoint{5.260404in}{3.728946in}}%
\pgfpathlineto{\pgfqpoint{5.261279in}{3.685260in}}%
\pgfpathlineto{\pgfqpoint{5.262154in}{3.572326in}}%
\pgfpathlineto{\pgfqpoint{5.263030in}{3.517351in}}%
\pgfpathlineto{\pgfqpoint{5.263905in}{3.481481in}}%
\pgfpathlineto{\pgfqpoint{5.264780in}{3.737517in}}%
\pgfpathlineto{\pgfqpoint{5.265656in}{3.833761in}}%
\pgfpathlineto{\pgfqpoint{5.266531in}{3.876126in}}%
\pgfpathlineto{\pgfqpoint{5.267407in}{3.382178in}}%
\pgfpathlineto{\pgfqpoint{5.268282in}{3.471928in}}%
\pgfpathlineto{\pgfqpoint{5.269157in}{3.442930in}}%
\pgfpathlineto{\pgfqpoint{5.270033in}{3.554731in}}%
\pgfpathlineto{\pgfqpoint{5.270908in}{3.559564in}}%
\pgfpathlineto{\pgfqpoint{5.271783in}{3.521542in}}%
\pgfpathlineto{\pgfqpoint{5.272659in}{3.424844in}}%
\pgfpathlineto{\pgfqpoint{5.273534in}{3.550880in}}%
\pgfpathlineto{\pgfqpoint{5.276160in}{3.277475in}}%
\pgfpathlineto{\pgfqpoint{5.277036in}{3.292314in}}%
\pgfpathlineto{\pgfqpoint{5.277911in}{3.509724in}}%
\pgfpathlineto{\pgfqpoint{5.279662in}{3.313232in}}%
\pgfpathlineto{\pgfqpoint{5.280537in}{3.242247in}}%
\pgfpathlineto{\pgfqpoint{5.281413in}{3.205849in}}%
\pgfpathlineto{\pgfqpoint{5.282288in}{3.086911in}}%
\pgfpathlineto{\pgfqpoint{5.283163in}{3.066409in}}%
\pgfpathlineto{\pgfqpoint{5.284039in}{3.024687in}}%
\pgfpathlineto{\pgfqpoint{5.284914in}{3.085099in}}%
\pgfpathlineto{\pgfqpoint{5.285789in}{3.297336in}}%
\pgfpathlineto{\pgfqpoint{5.286665in}{3.367679in}}%
\pgfpathlineto{\pgfqpoint{5.287540in}{3.468719in}}%
\pgfpathlineto{\pgfqpoint{5.288416in}{3.476006in}}%
\pgfpathlineto{\pgfqpoint{5.289291in}{3.388823in}}%
\pgfpathlineto{\pgfqpoint{5.290166in}{3.358353in}}%
\pgfpathlineto{\pgfqpoint{5.291042in}{3.412913in}}%
\pgfpathlineto{\pgfqpoint{5.291917in}{3.397734in}}%
\pgfpathlineto{\pgfqpoint{5.293668in}{3.573912in}}%
\pgfpathlineto{\pgfqpoint{5.294543in}{3.610537in}}%
\pgfpathlineto{\pgfqpoint{5.295419in}{3.449198in}}%
\pgfpathlineto{\pgfqpoint{5.296294in}{3.473401in}}%
\pgfpathlineto{\pgfqpoint{5.297169in}{3.565114in}}%
\pgfpathlineto{\pgfqpoint{5.298045in}{3.546122in}}%
\pgfpathlineto{\pgfqpoint{5.298920in}{3.859890in}}%
\pgfpathlineto{\pgfqpoint{5.299795in}{3.894665in}}%
\pgfpathlineto{\pgfqpoint{5.300671in}{3.777049in}}%
\pgfpathlineto{\pgfqpoint{5.301546in}{3.543404in}}%
\pgfpathlineto{\pgfqpoint{5.302422in}{3.569947in}}%
\pgfpathlineto{\pgfqpoint{5.303297in}{3.556053in}}%
\pgfpathlineto{\pgfqpoint{5.304172in}{3.513915in}}%
\pgfpathlineto{\pgfqpoint{5.305048in}{3.451501in}}%
\pgfpathlineto{\pgfqpoint{5.305923in}{3.573044in}}%
\pgfpathlineto{\pgfqpoint{5.306798in}{3.909126in}}%
\pgfpathlineto{\pgfqpoint{5.307674in}{3.938200in}}%
\pgfpathlineto{\pgfqpoint{5.308549in}{3.954284in}}%
\pgfpathlineto{\pgfqpoint{5.309425in}{3.891418in}}%
\pgfpathlineto{\pgfqpoint{5.310300in}{3.734421in}}%
\pgfpathlineto{\pgfqpoint{5.311175in}{3.966745in}}%
\pgfpathlineto{\pgfqpoint{5.312051in}{3.919774in}}%
\pgfpathlineto{\pgfqpoint{5.312926in}{4.032405in}}%
\pgfpathlineto{\pgfqpoint{5.313801in}{4.067785in}}%
\pgfpathlineto{\pgfqpoint{5.314677in}{4.236524in}}%
\pgfpathlineto{\pgfqpoint{5.315552in}{3.997970in}}%
\pgfpathlineto{\pgfqpoint{5.316428in}{3.642291in}}%
\pgfpathlineto{\pgfqpoint{5.317303in}{3.542233in}}%
\pgfpathlineto{\pgfqpoint{5.318178in}{3.483029in}}%
\pgfpathlineto{\pgfqpoint{5.319929in}{3.651240in}}%
\pgfpathlineto{\pgfqpoint{5.320804in}{3.688243in}}%
\pgfpathlineto{\pgfqpoint{5.321680in}{3.595132in}}%
\pgfpathlineto{\pgfqpoint{5.322555in}{3.368887in}}%
\pgfpathlineto{\pgfqpoint{5.324306in}{3.196598in}}%
\pgfpathlineto{\pgfqpoint{5.325181in}{3.253726in}}%
\pgfpathlineto{\pgfqpoint{5.326057in}{3.151062in}}%
\pgfpathlineto{\pgfqpoint{5.326932in}{3.253197in}}%
\pgfpathlineto{\pgfqpoint{5.327807in}{3.397432in}}%
\pgfpathlineto{\pgfqpoint{5.330434in}{3.551446in}}%
\pgfpathlineto{\pgfqpoint{5.331309in}{3.366886in}}%
\pgfpathlineto{\pgfqpoint{5.332184in}{3.646822in}}%
\pgfpathlineto{\pgfqpoint{5.333060in}{3.762286in}}%
\pgfpathlineto{\pgfqpoint{5.333935in}{3.704063in}}%
\pgfpathlineto{\pgfqpoint{5.334810in}{3.760813in}}%
\pgfpathlineto{\pgfqpoint{5.335686in}{3.690093in}}%
\pgfpathlineto{\pgfqpoint{5.337437in}{3.162238in}}%
\pgfpathlineto{\pgfqpoint{5.338312in}{3.052476in}}%
\pgfpathlineto{\pgfqpoint{5.340063in}{3.296543in}}%
\pgfpathlineto{\pgfqpoint{5.340938in}{3.516520in}}%
\pgfpathlineto{\pgfqpoint{5.341813in}{3.464792in}}%
\pgfpathlineto{\pgfqpoint{5.342689in}{3.375041in}}%
\pgfpathlineto{\pgfqpoint{5.344440in}{3.088044in}}%
\pgfpathlineto{\pgfqpoint{5.345315in}{2.987533in}}%
\pgfpathlineto{\pgfqpoint{5.347066in}{2.852058in}}%
\pgfpathlineto{\pgfqpoint{5.347941in}{2.934256in}}%
\pgfpathlineto{\pgfqpoint{5.348816in}{3.201393in}}%
\pgfpathlineto{\pgfqpoint{5.349692in}{3.247760in}}%
\pgfpathlineto{\pgfqpoint{5.350567in}{3.349291in}}%
\pgfpathlineto{\pgfqpoint{5.351443in}{3.148155in}}%
\pgfpathlineto{\pgfqpoint{5.352318in}{3.361751in}}%
\pgfpathlineto{\pgfqpoint{5.353193in}{3.327920in}}%
\pgfpathlineto{\pgfqpoint{5.354069in}{3.248402in}}%
\pgfpathlineto{\pgfqpoint{5.354944in}{3.455428in}}%
\pgfpathlineto{\pgfqpoint{5.355819in}{3.423787in}}%
\pgfpathlineto{\pgfqpoint{5.356695in}{3.574667in}}%
\pgfpathlineto{\pgfqpoint{5.358446in}{3.204301in}}%
\pgfpathlineto{\pgfqpoint{5.359321in}{3.250176in}}%
\pgfpathlineto{\pgfqpoint{5.360196in}{3.224312in}}%
\pgfpathlineto{\pgfqpoint{5.361072in}{3.581048in}}%
\pgfpathlineto{\pgfqpoint{5.361947in}{3.565265in}}%
\pgfpathlineto{\pgfqpoint{5.362822in}{3.385614in}}%
\pgfpathlineto{\pgfqpoint{5.363698in}{3.359372in}}%
\pgfpathlineto{\pgfqpoint{5.365449in}{3.094425in}}%
\pgfpathlineto{\pgfqpoint{5.366324in}{3.043188in}}%
\pgfpathlineto{\pgfqpoint{5.368075in}{2.999578in}}%
\pgfpathlineto{\pgfqpoint{5.368950in}{3.104016in}}%
\pgfpathlineto{\pgfqpoint{5.369825in}{3.338190in}}%
\pgfpathlineto{\pgfqpoint{5.370701in}{3.425410in}}%
\pgfpathlineto{\pgfqpoint{5.371576in}{3.618957in}}%
\pgfpathlineto{\pgfqpoint{5.372452in}{3.631644in}}%
\pgfpathlineto{\pgfqpoint{5.373327in}{3.741481in}}%
\pgfpathlineto{\pgfqpoint{5.374202in}{3.433491in}}%
\pgfpathlineto{\pgfqpoint{5.376828in}{3.779617in}}%
\pgfpathlineto{\pgfqpoint{5.377704in}{3.805556in}}%
\pgfpathlineto{\pgfqpoint{5.379455in}{3.437531in}}%
\pgfpathlineto{\pgfqpoint{5.380330in}{3.455503in}}%
\pgfpathlineto{\pgfqpoint{5.381205in}{3.448594in}}%
\pgfpathlineto{\pgfqpoint{5.382956in}{3.719091in}}%
\pgfpathlineto{\pgfqpoint{5.383831in}{3.795853in}}%
\pgfpathlineto{\pgfqpoint{5.384707in}{3.683485in}}%
\pgfpathlineto{\pgfqpoint{5.386458in}{3.370511in}}%
\pgfpathlineto{\pgfqpoint{5.387333in}{3.309871in}}%
\pgfpathlineto{\pgfqpoint{5.388208in}{3.311873in}}%
\pgfpathlineto{\pgfqpoint{5.389084in}{3.226993in}}%
\pgfpathlineto{\pgfqpoint{5.389959in}{3.356238in}}%
\pgfpathlineto{\pgfqpoint{5.390835in}{3.578103in}}%
\pgfpathlineto{\pgfqpoint{5.392585in}{4.244869in}}%
\pgfpathlineto{\pgfqpoint{5.393461in}{4.332731in}}%
\pgfpathlineto{\pgfqpoint{5.394336in}{4.260614in}}%
\pgfpathlineto{\pgfqpoint{5.395211in}{4.392313in}}%
\pgfpathlineto{\pgfqpoint{5.396087in}{4.425125in}}%
\pgfpathlineto{\pgfqpoint{5.396962in}{4.609005in}}%
\pgfpathlineto{\pgfqpoint{5.397838in}{4.444872in}}%
\pgfpathlineto{\pgfqpoint{5.398713in}{4.499168in}}%
\pgfpathlineto{\pgfqpoint{5.399588in}{4.319969in}}%
\pgfpathlineto{\pgfqpoint{5.401339in}{3.834743in}}%
\pgfpathlineto{\pgfqpoint{5.402214in}{3.839727in}}%
\pgfpathlineto{\pgfqpoint{5.403090in}{3.916791in}}%
\pgfpathlineto{\pgfqpoint{5.403965in}{4.021569in}}%
\pgfpathlineto{\pgfqpoint{5.404841in}{3.886962in}}%
\pgfpathlineto{\pgfqpoint{5.405716in}{3.803442in}}%
\pgfpathlineto{\pgfqpoint{5.407467in}{3.467171in}}%
\pgfpathlineto{\pgfqpoint{5.408342in}{3.411327in}}%
\pgfpathlineto{\pgfqpoint{5.409217in}{3.499265in}}%
\pgfpathlineto{\pgfqpoint{5.410093in}{3.534493in}}%
\pgfpathlineto{\pgfqpoint{5.410968in}{3.646181in}}%
\pgfpathlineto{\pgfqpoint{5.411844in}{3.988418in}}%
\pgfpathlineto{\pgfqpoint{5.414470in}{4.372150in}}%
\pgfpathlineto{\pgfqpoint{5.415345in}{4.357689in}}%
\pgfpathlineto{\pgfqpoint{5.416220in}{4.305432in}}%
\pgfpathlineto{\pgfqpoint{5.417096in}{4.045621in}}%
\pgfpathlineto{\pgfqpoint{5.417971in}{4.018435in}}%
\pgfpathlineto{\pgfqpoint{5.418847in}{4.145377in}}%
\pgfpathlineto{\pgfqpoint{5.419722in}{4.060686in}}%
\pgfpathlineto{\pgfqpoint{5.422348in}{3.488164in}}%
\pgfpathlineto{\pgfqpoint{5.423223in}{3.520258in}}%
\pgfpathlineto{\pgfqpoint{5.424099in}{3.864119in}}%
\pgfpathlineto{\pgfqpoint{5.425850in}{3.626622in}}%
\pgfpathlineto{\pgfqpoint{5.426725in}{3.550049in}}%
\pgfpathlineto{\pgfqpoint{5.427600in}{3.354237in}}%
\pgfpathlineto{\pgfqpoint{5.428476in}{3.243946in}}%
\pgfpathlineto{\pgfqpoint{5.429351in}{3.086723in}}%
\pgfpathlineto{\pgfqpoint{5.430226in}{3.008413in}}%
\pgfpathlineto{\pgfqpoint{5.431102in}{2.970806in}}%
\pgfpathlineto{\pgfqpoint{5.431977in}{2.910658in}}%
\pgfpathlineto{\pgfqpoint{5.434603in}{3.452256in}}%
\pgfpathlineto{\pgfqpoint{5.435479in}{3.501379in}}%
\pgfpathlineto{\pgfqpoint{5.436354in}{3.674763in}}%
\pgfpathlineto{\pgfqpoint{5.437229in}{3.708934in}}%
\pgfpathlineto{\pgfqpoint{5.438105in}{3.525242in}}%
\pgfpathlineto{\pgfqpoint{5.438980in}{3.549369in}}%
\pgfpathlineto{\pgfqpoint{5.439856in}{3.762928in}}%
\pgfpathlineto{\pgfqpoint{5.440731in}{3.675481in}}%
\pgfpathlineto{\pgfqpoint{5.441606in}{3.424391in}}%
\pgfpathlineto{\pgfqpoint{5.442482in}{3.382933in}}%
\pgfpathlineto{\pgfqpoint{5.443357in}{3.368660in}}%
\pgfpathlineto{\pgfqpoint{5.444232in}{3.529698in}}%
\pgfpathlineto{\pgfqpoint{5.445108in}{3.637572in}}%
\pgfpathlineto{\pgfqpoint{5.445983in}{3.602986in}}%
\pgfpathlineto{\pgfqpoint{5.446859in}{3.518446in}}%
\pgfpathlineto{\pgfqpoint{5.447734in}{3.509233in}}%
\pgfpathlineto{\pgfqpoint{5.448609in}{3.310249in}}%
\pgfpathlineto{\pgfqpoint{5.449485in}{3.249723in}}%
\pgfpathlineto{\pgfqpoint{5.450360in}{3.160728in}}%
\pgfpathlineto{\pgfqpoint{5.451235in}{3.104620in}}%
\pgfpathlineto{\pgfqpoint{5.452111in}{3.072224in}}%
\pgfpathlineto{\pgfqpoint{5.452986in}{3.163145in}}%
\pgfpathlineto{\pgfqpoint{5.453862in}{3.594830in}}%
\pgfpathlineto{\pgfqpoint{5.454737in}{3.785507in}}%
\pgfpathlineto{\pgfqpoint{5.455612in}{3.854226in}}%
\pgfpathlineto{\pgfqpoint{5.456488in}{4.141488in}}%
\pgfpathlineto{\pgfqpoint{5.457363in}{3.975806in}}%
\pgfpathlineto{\pgfqpoint{5.459114in}{3.905388in}}%
\pgfpathlineto{\pgfqpoint{5.459989in}{4.076091in}}%
\pgfpathlineto{\pgfqpoint{5.460865in}{4.041279in}}%
\pgfpathlineto{\pgfqpoint{5.461740in}{3.936576in}}%
\pgfpathlineto{\pgfqpoint{5.463491in}{3.658301in}}%
\pgfpathlineto{\pgfqpoint{5.464366in}{3.437984in}}%
\pgfpathlineto{\pgfqpoint{5.465241in}{3.522788in}}%
\pgfpathlineto{\pgfqpoint{5.466117in}{3.739820in}}%
\pgfpathlineto{\pgfqpoint{5.466992in}{3.737630in}}%
\pgfpathlineto{\pgfqpoint{5.467868in}{3.586070in}}%
\pgfpathlineto{\pgfqpoint{5.468743in}{3.514481in}}%
\pgfpathlineto{\pgfqpoint{5.470494in}{3.275738in}}%
\pgfpathlineto{\pgfqpoint{5.471369in}{3.332979in}}%
\pgfpathlineto{\pgfqpoint{5.473120in}{3.300885in}}%
\pgfpathlineto{\pgfqpoint{5.473995in}{3.394109in}}%
\pgfpathlineto{\pgfqpoint{5.477497in}{4.329295in}}%
\pgfpathlineto{\pgfqpoint{5.479247in}{3.987360in}}%
\pgfpathlineto{\pgfqpoint{5.480123in}{3.989097in}}%
\pgfpathlineto{\pgfqpoint{5.480998in}{4.131671in}}%
\pgfpathlineto{\pgfqpoint{5.481874in}{4.159951in}}%
\pgfpathlineto{\pgfqpoint{5.482749in}{4.134767in}}%
\pgfpathlineto{\pgfqpoint{5.483624in}{4.045998in}}%
\pgfpathlineto{\pgfqpoint{5.484500in}{3.679407in}}%
\pgfpathlineto{\pgfqpoint{5.485375in}{3.629680in}}%
\pgfpathlineto{\pgfqpoint{5.486250in}{3.621751in}}%
\pgfpathlineto{\pgfqpoint{5.487126in}{3.749561in}}%
\pgfpathlineto{\pgfqpoint{5.488001in}{4.064953in}}%
\pgfpathlineto{\pgfqpoint{5.489752in}{4.004880in}}%
\pgfpathlineto{\pgfqpoint{5.491503in}{3.746314in}}%
\pgfpathlineto{\pgfqpoint{5.493253in}{3.561301in}}%
\pgfpathlineto{\pgfqpoint{5.494129in}{3.483860in}}%
\pgfpathlineto{\pgfqpoint{5.495004in}{3.593810in}}%
\pgfpathlineto{\pgfqpoint{5.495880in}{3.866611in}}%
\pgfpathlineto{\pgfqpoint{5.496755in}{3.970445in}}%
\pgfpathlineto{\pgfqpoint{5.497630in}{4.189931in}}%
\pgfpathlineto{\pgfqpoint{5.498506in}{4.178793in}}%
\pgfpathlineto{\pgfqpoint{5.500256in}{4.296786in}}%
\pgfpathlineto{\pgfqpoint{5.501132in}{4.256196in}}%
\pgfpathlineto{\pgfqpoint{5.502007in}{4.330466in}}%
\pgfpathlineto{\pgfqpoint{5.502883in}{4.378985in}}%
\pgfpathlineto{\pgfqpoint{5.503758in}{4.171619in}}%
\pgfpathlineto{\pgfqpoint{5.504633in}{4.084889in}}%
\pgfpathlineto{\pgfqpoint{5.505509in}{3.881601in}}%
\pgfpathlineto{\pgfqpoint{5.506384in}{3.783770in}}%
\pgfpathlineto{\pgfqpoint{5.507259in}{3.788867in}}%
\pgfpathlineto{\pgfqpoint{5.508135in}{4.133596in}}%
\pgfpathlineto{\pgfqpoint{5.509010in}{4.269638in}}%
\pgfpathlineto{\pgfqpoint{5.509886in}{4.091194in}}%
\pgfpathlineto{\pgfqpoint{5.510761in}{3.979695in}}%
\pgfpathlineto{\pgfqpoint{5.511636in}{3.658829in}}%
\pgfpathlineto{\pgfqpoint{5.512512in}{3.669590in}}%
\pgfpathlineto{\pgfqpoint{5.513387in}{3.555750in}}%
\pgfpathlineto{\pgfqpoint{5.514262in}{3.575724in}}%
\pgfpathlineto{\pgfqpoint{5.515138in}{3.460676in}}%
\pgfpathlineto{\pgfqpoint{5.516889in}{3.956437in}}%
\pgfpathlineto{\pgfqpoint{5.518639in}{4.595866in}}%
\pgfpathlineto{\pgfqpoint{5.519515in}{4.850127in}}%
\pgfpathlineto{\pgfqpoint{5.520390in}{4.737231in}}%
\pgfpathlineto{\pgfqpoint{5.521265in}{4.697887in}}%
\pgfpathlineto{\pgfqpoint{5.522141in}{4.556144in}}%
\pgfpathlineto{\pgfqpoint{5.523016in}{4.647405in}}%
\pgfpathlineto{\pgfqpoint{5.523892in}{4.673685in}}%
\pgfpathlineto{\pgfqpoint{5.524767in}{4.601492in}}%
\pgfpathlineto{\pgfqpoint{5.525642in}{4.471302in}}%
\pgfpathlineto{\pgfqpoint{5.526518in}{4.135333in}}%
\pgfpathlineto{\pgfqpoint{5.527393in}{3.979167in}}%
\pgfpathlineto{\pgfqpoint{5.528269in}{4.049812in}}%
\pgfpathlineto{\pgfqpoint{5.529144in}{4.470019in}}%
\pgfpathlineto{\pgfqpoint{5.530019in}{4.598395in}}%
\pgfpathlineto{\pgfqpoint{5.532645in}{3.996309in}}%
\pgfpathlineto{\pgfqpoint{5.533521in}{3.660302in}}%
\pgfpathlineto{\pgfqpoint{5.534396in}{3.643613in}}%
\pgfpathlineto{\pgfqpoint{5.535272in}{3.610462in}}%
\pgfpathlineto{\pgfqpoint{5.536147in}{3.588033in}}%
\pgfpathlineto{\pgfqpoint{5.537022in}{3.654600in}}%
\pgfpathlineto{\pgfqpoint{5.539648in}{4.864437in}}%
\pgfpathlineto{\pgfqpoint{5.540524in}{4.883694in}}%
\pgfpathlineto{\pgfqpoint{5.541399in}{5.055190in}}%
\pgfpathlineto{\pgfqpoint{5.542275in}{4.853110in}}%
\pgfpathlineto{\pgfqpoint{5.543150in}{4.858547in}}%
\pgfpathlineto{\pgfqpoint{5.544025in}{4.971745in}}%
\pgfpathlineto{\pgfqpoint{5.545776in}{4.931759in}}%
\pgfpathlineto{\pgfqpoint{5.548402in}{4.083265in}}%
\pgfpathlineto{\pgfqpoint{5.549278in}{4.077753in}}%
\pgfpathlineto{\pgfqpoint{5.550153in}{4.202278in}}%
\pgfpathlineto{\pgfqpoint{5.551028in}{4.257140in}}%
\pgfpathlineto{\pgfqpoint{5.551904in}{4.196010in}}%
\pgfpathlineto{\pgfqpoint{5.552779in}{4.068049in}}%
\pgfpathlineto{\pgfqpoint{5.553654in}{3.824473in}}%
\pgfpathlineto{\pgfqpoint{5.555405in}{3.702477in}}%
\pgfpathlineto{\pgfqpoint{5.556281in}{3.568022in}}%
\pgfpathlineto{\pgfqpoint{5.557156in}{3.584258in}}%
\pgfpathlineto{\pgfqpoint{5.558031in}{3.740273in}}%
\pgfpathlineto{\pgfqpoint{5.558907in}{4.105844in}}%
\pgfpathlineto{\pgfqpoint{5.560657in}{4.529638in}}%
\pgfpathlineto{\pgfqpoint{5.561533in}{4.591523in}}%
\pgfpathlineto{\pgfqpoint{5.562408in}{4.526202in}}%
\pgfpathlineto{\pgfqpoint{5.563284in}{4.428523in}}%
\pgfpathlineto{\pgfqpoint{5.564159in}{4.445740in}}%
\pgfpathlineto{\pgfqpoint{5.565034in}{4.493089in}}%
\pgfpathlineto{\pgfqpoint{5.565910in}{4.562941in}}%
\pgfpathlineto{\pgfqpoint{5.566785in}{4.712462in}}%
\pgfpathlineto{\pgfqpoint{5.567660in}{4.668738in}}%
\pgfpathlineto{\pgfqpoint{5.568536in}{4.310492in}}%
\pgfpathlineto{\pgfqpoint{5.569411in}{4.201409in}}%
\pgfpathlineto{\pgfqpoint{5.570287in}{4.166748in}}%
\pgfpathlineto{\pgfqpoint{5.571162in}{4.062725in}}%
\pgfpathlineto{\pgfqpoint{5.572037in}{4.360106in}}%
\pgfpathlineto{\pgfqpoint{5.572913in}{4.089760in}}%
\pgfpathlineto{\pgfqpoint{5.575539in}{3.635533in}}%
\pgfpathlineto{\pgfqpoint{5.576414in}{3.568475in}}%
\pgfpathlineto{\pgfqpoint{5.577290in}{3.599625in}}%
\pgfpathlineto{\pgfqpoint{5.578165in}{3.587089in}}%
\pgfpathlineto{\pgfqpoint{5.579040in}{3.613520in}}%
\pgfpathlineto{\pgfqpoint{5.579916in}{3.861325in}}%
\pgfpathlineto{\pgfqpoint{5.580791in}{3.931592in}}%
\pgfpathlineto{\pgfqpoint{5.581666in}{4.058194in}}%
\pgfpathlineto{\pgfqpoint{5.582542in}{4.075261in}}%
\pgfpathlineto{\pgfqpoint{5.583417in}{4.270808in}}%
\pgfpathlineto{\pgfqpoint{5.584293in}{4.081264in}}%
\pgfpathlineto{\pgfqpoint{5.585168in}{4.121099in}}%
\pgfpathlineto{\pgfqpoint{5.586043in}{4.086324in}}%
\pgfpathlineto{\pgfqpoint{5.586919in}{4.350968in}}%
\pgfpathlineto{\pgfqpoint{5.588669in}{4.163538in}}%
\pgfpathlineto{\pgfqpoint{5.589545in}{3.872237in}}%
\pgfpathlineto{\pgfqpoint{5.590420in}{3.776634in}}%
\pgfpathlineto{\pgfqpoint{5.591296in}{3.850866in}}%
\pgfpathlineto{\pgfqpoint{5.593046in}{4.183248in}}%
\pgfpathlineto{\pgfqpoint{5.593922in}{3.977166in}}%
\pgfpathlineto{\pgfqpoint{5.594797in}{3.857436in}}%
\pgfpathlineto{\pgfqpoint{5.595672in}{3.459468in}}%
\pgfpathlineto{\pgfqpoint{5.597423in}{3.306587in}}%
\pgfpathlineto{\pgfqpoint{5.598299in}{3.374022in}}%
\pgfpathlineto{\pgfqpoint{5.599174in}{3.371681in}}%
\pgfpathlineto{\pgfqpoint{5.600049in}{3.476006in}}%
\pgfpathlineto{\pgfqpoint{5.600925in}{3.897421in}}%
\pgfpathlineto{\pgfqpoint{5.601800in}{4.052681in}}%
\pgfpathlineto{\pgfqpoint{5.602675in}{4.336205in}}%
\pgfpathlineto{\pgfqpoint{5.603551in}{4.355915in}}%
\pgfpathlineto{\pgfqpoint{5.604426in}{4.591259in}}%
\pgfpathlineto{\pgfqpoint{5.605302in}{4.676894in}}%
\pgfpathlineto{\pgfqpoint{5.606177in}{4.590391in}}%
\pgfpathlineto{\pgfqpoint{5.607052in}{4.687428in}}%
\pgfpathlineto{\pgfqpoint{5.607928in}{4.708422in}}%
\pgfpathlineto{\pgfqpoint{5.609678in}{4.583594in}}%
\pgfpathlineto{\pgfqpoint{5.610554in}{4.214285in}}%
\pgfpathlineto{\pgfqpoint{5.611429in}{4.185136in}}%
\pgfpathlineto{\pgfqpoint{5.613180in}{4.339867in}}%
\pgfpathlineto{\pgfqpoint{5.614055in}{4.296295in}}%
\pgfpathlineto{\pgfqpoint{5.614931in}{4.158101in}}%
\pgfpathlineto{\pgfqpoint{5.615806in}{3.931970in}}%
\pgfpathlineto{\pgfqpoint{5.616681in}{3.784148in}}%
\pgfpathlineto{\pgfqpoint{5.617557in}{3.571193in}}%
\pgfpathlineto{\pgfqpoint{5.618432in}{3.564699in}}%
\pgfpathlineto{\pgfqpoint{5.619308in}{3.504249in}}%
\pgfpathlineto{\pgfqpoint{5.620183in}{3.545329in}}%
\pgfpathlineto{\pgfqpoint{5.621058in}{3.537627in}}%
\pgfpathlineto{\pgfqpoint{5.623684in}{4.502755in}}%
\pgfpathlineto{\pgfqpoint{5.624560in}{4.614707in}}%
\pgfpathlineto{\pgfqpoint{5.625435in}{4.545308in}}%
\pgfpathlineto{\pgfqpoint{5.626311in}{4.180378in}}%
\pgfpathlineto{\pgfqpoint{5.627186in}{4.138883in}}%
\pgfpathlineto{\pgfqpoint{5.628061in}{4.066161in}}%
\pgfpathlineto{\pgfqpoint{5.628937in}{4.289876in}}%
\pgfpathlineto{\pgfqpoint{5.629812in}{4.152249in}}%
\pgfpathlineto{\pgfqpoint{5.630687in}{4.141261in}}%
\pgfpathlineto{\pgfqpoint{5.631563in}{3.998348in}}%
\pgfpathlineto{\pgfqpoint{5.632438in}{3.802271in}}%
\pgfpathlineto{\pgfqpoint{5.633314in}{3.855925in}}%
\pgfpathlineto{\pgfqpoint{5.634189in}{3.875748in}}%
\pgfpathlineto{\pgfqpoint{5.635064in}{3.933555in}}%
\pgfpathlineto{\pgfqpoint{5.635940in}{3.801781in}}%
\pgfpathlineto{\pgfqpoint{5.636815in}{3.613105in}}%
\pgfpathlineto{\pgfqpoint{5.637690in}{3.364243in}}%
\pgfpathlineto{\pgfqpoint{5.638566in}{3.228088in}}%
\pgfpathlineto{\pgfqpoint{5.639441in}{3.257237in}}%
\pgfpathlineto{\pgfqpoint{5.640317in}{3.161294in}}%
\pgfpathlineto{\pgfqpoint{5.641192in}{3.097597in}}%
\pgfpathlineto{\pgfqpoint{5.642067in}{3.088271in}}%
\pgfpathlineto{\pgfqpoint{5.642943in}{3.175944in}}%
\pgfpathlineto{\pgfqpoint{5.643818in}{3.182137in}}%
\pgfpathlineto{\pgfqpoint{5.644693in}{3.374249in}}%
\pgfpathlineto{\pgfqpoint{5.645569in}{3.406267in}}%
\pgfpathlineto{\pgfqpoint{5.646444in}{3.556845in}}%
\pgfpathlineto{\pgfqpoint{5.647320in}{3.430772in}}%
\pgfpathlineto{\pgfqpoint{5.648195in}{3.511310in}}%
\pgfpathlineto{\pgfqpoint{5.649070in}{3.617787in}}%
\pgfpathlineto{\pgfqpoint{5.649946in}{3.903387in}}%
\pgfpathlineto{\pgfqpoint{5.650821in}{3.678350in}}%
\pgfpathlineto{\pgfqpoint{5.651696in}{3.716523in}}%
\pgfpathlineto{\pgfqpoint{5.652572in}{3.407551in}}%
\pgfpathlineto{\pgfqpoint{5.654323in}{3.264222in}}%
\pgfpathlineto{\pgfqpoint{5.656073in}{3.400113in}}%
\pgfpathlineto{\pgfqpoint{5.656949in}{3.375117in}}%
\pgfpathlineto{\pgfqpoint{5.659575in}{2.817283in}}%
\pgfpathlineto{\pgfqpoint{5.660450in}{2.793306in}}%
\pgfpathlineto{\pgfqpoint{5.662201in}{2.569818in}}%
\pgfpathlineto{\pgfqpoint{5.663076in}{2.637669in}}%
\pgfpathlineto{\pgfqpoint{5.663952in}{2.755133in}}%
\pgfpathlineto{\pgfqpoint{5.664827in}{2.956571in}}%
\pgfpathlineto{\pgfqpoint{5.666578in}{3.568211in}}%
\pgfpathlineto{\pgfqpoint{5.667453in}{3.499265in}}%
\pgfpathlineto{\pgfqpoint{5.668329in}{3.503531in}}%
\pgfpathlineto{\pgfqpoint{5.669204in}{3.532718in}}%
\pgfpathlineto{\pgfqpoint{5.670079in}{3.827833in}}%
\pgfpathlineto{\pgfqpoint{5.670955in}{3.681711in}}%
\pgfpathlineto{\pgfqpoint{5.671830in}{3.643651in}}%
\pgfpathlineto{\pgfqpoint{5.672706in}{3.579047in}}%
\pgfpathlineto{\pgfqpoint{5.673581in}{3.445535in}}%
\pgfpathlineto{\pgfqpoint{5.674456in}{3.358617in}}%
\pgfpathlineto{\pgfqpoint{5.675332in}{3.364167in}}%
\pgfpathlineto{\pgfqpoint{5.676207in}{3.485974in}}%
\pgfpathlineto{\pgfqpoint{5.677082in}{3.667363in}}%
\pgfpathlineto{\pgfqpoint{5.677958in}{3.605062in}}%
\pgfpathlineto{\pgfqpoint{5.679709in}{3.051344in}}%
\pgfpathlineto{\pgfqpoint{5.680584in}{2.936711in}}%
\pgfpathlineto{\pgfqpoint{5.681459in}{2.872900in}}%
\pgfpathlineto{\pgfqpoint{5.684085in}{2.741389in}}%
\pgfpathlineto{\pgfqpoint{5.686712in}{3.856907in}}%
\pgfpathlineto{\pgfqpoint{5.687587in}{3.843503in}}%
\pgfpathlineto{\pgfqpoint{5.688462in}{3.852942in}}%
\pgfpathlineto{\pgfqpoint{5.689338in}{3.999707in}}%
\pgfpathlineto{\pgfqpoint{5.690213in}{3.995478in}}%
\pgfpathlineto{\pgfqpoint{5.691964in}{4.414704in}}%
\pgfpathlineto{\pgfqpoint{5.692839in}{4.440379in}}%
\pgfpathlineto{\pgfqpoint{5.694590in}{3.919321in}}%
\pgfpathlineto{\pgfqpoint{5.695465in}{3.702591in}}%
\pgfpathlineto{\pgfqpoint{5.696341in}{3.818281in}}%
\pgfpathlineto{\pgfqpoint{5.697216in}{3.792870in}}%
\pgfpathlineto{\pgfqpoint{5.698091in}{3.712408in}}%
\pgfpathlineto{\pgfqpoint{5.701593in}{2.874901in}}%
\pgfpathlineto{\pgfqpoint{5.702468in}{2.858779in}}%
\pgfpathlineto{\pgfqpoint{5.703344in}{2.902502in}}%
\pgfpathlineto{\pgfqpoint{5.704219in}{2.889400in}}%
\pgfpathlineto{\pgfqpoint{5.705094in}{2.892836in}}%
\pgfpathlineto{\pgfqpoint{5.708596in}{4.221572in}}%
\pgfpathlineto{\pgfqpoint{5.709471in}{4.298825in}}%
\pgfpathlineto{\pgfqpoint{5.712097in}{4.768079in}}%
\pgfpathlineto{\pgfqpoint{5.712973in}{4.739950in}}%
\pgfpathlineto{\pgfqpoint{5.713848in}{4.837025in}}%
\pgfpathlineto{\pgfqpoint{5.714724in}{4.752938in}}%
\pgfpathlineto{\pgfqpoint{5.715599in}{4.487878in}}%
\pgfpathlineto{\pgfqpoint{5.716474in}{4.357312in}}%
\pgfpathlineto{\pgfqpoint{5.718225in}{4.461070in}}%
\pgfpathlineto{\pgfqpoint{5.722602in}{3.715693in}}%
\pgfpathlineto{\pgfqpoint{5.723477in}{3.674990in}}%
\pgfpathlineto{\pgfqpoint{5.724353in}{3.607970in}}%
\pgfpathlineto{\pgfqpoint{5.725228in}{3.407211in}}%
\pgfpathlineto{\pgfqpoint{5.726103in}{3.369642in}}%
\pgfpathlineto{\pgfqpoint{5.727854in}{4.004276in}}%
\pgfpathlineto{\pgfqpoint{5.728730in}{4.253440in}}%
\pgfpathlineto{\pgfqpoint{5.729605in}{4.245662in}}%
\pgfpathlineto{\pgfqpoint{5.730480in}{4.266202in}}%
\pgfpathlineto{\pgfqpoint{5.731356in}{4.346286in}}%
\pgfpathlineto{\pgfqpoint{5.732231in}{4.308528in}}%
\pgfpathlineto{\pgfqpoint{5.733106in}{4.531904in}}%
\pgfpathlineto{\pgfqpoint{5.733982in}{4.627998in}}%
\pgfpathlineto{\pgfqpoint{5.734857in}{4.509778in}}%
\pgfpathlineto{\pgfqpoint{5.736608in}{4.357161in}}%
\pgfpathlineto{\pgfqpoint{5.737483in}{4.229086in}}%
\pgfpathlineto{\pgfqpoint{5.738359in}{4.306263in}}%
\pgfpathlineto{\pgfqpoint{5.739234in}{4.487576in}}%
\pgfpathlineto{\pgfqpoint{5.740109in}{4.441776in}}%
\pgfpathlineto{\pgfqpoint{5.742736in}{3.660226in}}%
\pgfpathlineto{\pgfqpoint{5.743611in}{3.565152in}}%
\pgfpathlineto{\pgfqpoint{5.744486in}{3.437078in}}%
\pgfpathlineto{\pgfqpoint{5.745362in}{3.392335in}}%
\pgfpathlineto{\pgfqpoint{5.746237in}{3.328977in}}%
\pgfpathlineto{\pgfqpoint{5.747112in}{3.349630in}}%
\pgfpathlineto{\pgfqpoint{5.747988in}{3.418727in}}%
\pgfpathlineto{\pgfqpoint{5.748863in}{3.630398in}}%
\pgfpathlineto{\pgfqpoint{5.749739in}{4.244340in}}%
\pgfpathlineto{\pgfqpoint{5.750614in}{4.363089in}}%
\pgfpathlineto{\pgfqpoint{5.751489in}{4.588994in}}%
\pgfpathlineto{\pgfqpoint{5.752365in}{4.538776in}}%
\pgfpathlineto{\pgfqpoint{5.753240in}{4.570190in}}%
\pgfpathlineto{\pgfqpoint{5.754991in}{4.840234in}}%
\pgfpathlineto{\pgfqpoint{5.755866in}{4.736400in}}%
\pgfpathlineto{\pgfqpoint{5.756742in}{4.765096in}}%
\pgfpathlineto{\pgfqpoint{5.757617in}{4.427390in}}%
\pgfpathlineto{\pgfqpoint{5.758492in}{4.251741in}}%
\pgfpathlineto{\pgfqpoint{5.759368in}{4.403527in}}%
\pgfpathlineto{\pgfqpoint{5.761118in}{4.601038in}}%
\pgfpathlineto{\pgfqpoint{5.761994in}{4.544175in}}%
\pgfpathlineto{\pgfqpoint{5.762869in}{4.384308in}}%
\pgfpathlineto{\pgfqpoint{5.763745in}{4.069521in}}%
\pgfpathlineto{\pgfqpoint{5.764620in}{3.931101in}}%
\pgfpathlineto{\pgfqpoint{5.765495in}{3.869254in}}%
\pgfpathlineto{\pgfqpoint{5.766371in}{3.738763in}}%
\pgfpathlineto{\pgfqpoint{5.767246in}{3.723924in}}%
\pgfpathlineto{\pgfqpoint{5.768121in}{3.806009in}}%
\pgfpathlineto{\pgfqpoint{5.768997in}{4.083605in}}%
\pgfpathlineto{\pgfqpoint{5.769872in}{4.239696in}}%
\pgfpathlineto{\pgfqpoint{5.771623in}{4.811161in}}%
\pgfpathlineto{\pgfqpoint{5.772498in}{4.735230in}}%
\pgfpathlineto{\pgfqpoint{5.773374in}{4.758979in}}%
\pgfpathlineto{\pgfqpoint{5.774249in}{4.865268in}}%
\pgfpathlineto{\pgfqpoint{5.775124in}{5.195573in}}%
\pgfpathlineto{\pgfqpoint{5.777751in}{4.898646in}}%
\pgfpathlineto{\pgfqpoint{5.778626in}{4.605569in}}%
\pgfpathlineto{\pgfqpoint{5.779501in}{4.483763in}}%
\pgfpathlineto{\pgfqpoint{5.780377in}{4.466734in}}%
\pgfpathlineto{\pgfqpoint{5.782127in}{4.412627in}}%
\pgfpathlineto{\pgfqpoint{5.784754in}{3.873483in}}%
\pgfpathlineto{\pgfqpoint{5.785629in}{3.738083in}}%
\pgfpathlineto{\pgfqpoint{5.786504in}{3.649994in}}%
\pgfpathlineto{\pgfqpoint{5.787380in}{3.605364in}}%
\pgfpathlineto{\pgfqpoint{5.788255in}{3.577952in}}%
\pgfpathlineto{\pgfqpoint{5.789130in}{3.594188in}}%
\pgfpathlineto{\pgfqpoint{5.790881in}{3.724113in}}%
\pgfpathlineto{\pgfqpoint{5.791757in}{4.069333in}}%
\pgfpathlineto{\pgfqpoint{5.792632in}{4.212775in}}%
\pgfpathlineto{\pgfqpoint{5.793507in}{4.212737in}}%
\pgfpathlineto{\pgfqpoint{5.794383in}{4.485462in}}%
\pgfpathlineto{\pgfqpoint{5.795258in}{4.521785in}}%
\pgfpathlineto{\pgfqpoint{5.796134in}{4.516348in}}%
\pgfpathlineto{\pgfqpoint{5.797009in}{4.526958in}}%
\pgfpathlineto{\pgfqpoint{5.797884in}{4.652314in}}%
\pgfpathlineto{\pgfqpoint{5.798760in}{4.502491in}}%
\pgfpathlineto{\pgfqpoint{5.799635in}{4.495165in}}%
\pgfpathlineto{\pgfqpoint{5.800510in}{4.330806in}}%
\pgfpathlineto{\pgfqpoint{5.802261in}{4.673080in}}%
\pgfpathlineto{\pgfqpoint{5.803137in}{4.712537in}}%
\pgfpathlineto{\pgfqpoint{5.804887in}{4.324462in}}%
\pgfpathlineto{\pgfqpoint{5.805763in}{4.013602in}}%
\pgfpathlineto{\pgfqpoint{5.807513in}{3.794909in}}%
\pgfpathlineto{\pgfqpoint{5.808389in}{3.645312in}}%
\pgfpathlineto{\pgfqpoint{5.809264in}{3.616918in}}%
\pgfpathlineto{\pgfqpoint{5.810140in}{3.663436in}}%
\pgfpathlineto{\pgfqpoint{5.811015in}{3.825379in}}%
\pgfpathlineto{\pgfqpoint{5.811890in}{3.926079in}}%
\pgfpathlineto{\pgfqpoint{5.812766in}{4.256574in}}%
\pgfpathlineto{\pgfqpoint{5.814516in}{4.403489in}}%
\pgfpathlineto{\pgfqpoint{5.815392in}{4.590806in}}%
\pgfpathlineto{\pgfqpoint{5.816267in}{4.619502in}}%
\pgfpathlineto{\pgfqpoint{5.817143in}{4.878596in}}%
\pgfpathlineto{\pgfqpoint{5.818018in}{4.952413in}}%
\pgfpathlineto{\pgfqpoint{5.819769in}{4.850089in}}%
\pgfpathlineto{\pgfqpoint{5.820644in}{4.642081in}}%
\pgfpathlineto{\pgfqpoint{5.821519in}{4.636153in}}%
\pgfpathlineto{\pgfqpoint{5.822395in}{4.659034in}}%
\pgfpathlineto{\pgfqpoint{5.823270in}{4.872291in}}%
\pgfpathlineto{\pgfqpoint{5.824146in}{4.813540in}}%
\pgfpathlineto{\pgfqpoint{5.827647in}{3.950207in}}%
\pgfpathlineto{\pgfqpoint{5.828522in}{3.795626in}}%
\pgfpathlineto{\pgfqpoint{5.829398in}{3.769649in}}%
\pgfpathlineto{\pgfqpoint{5.830273in}{3.715466in}}%
\pgfpathlineto{\pgfqpoint{5.831149in}{3.805254in}}%
\pgfpathlineto{\pgfqpoint{5.832024in}{4.204657in}}%
\pgfpathlineto{\pgfqpoint{5.832899in}{4.389783in}}%
\pgfpathlineto{\pgfqpoint{5.833775in}{4.901289in}}%
\pgfpathlineto{\pgfqpoint{5.834650in}{5.208789in}}%
\pgfpathlineto{\pgfqpoint{5.835525in}{5.333616in}}%
\pgfpathlineto{\pgfqpoint{5.836401in}{5.409547in}}%
\pgfpathlineto{\pgfqpoint{5.837276in}{5.353741in}}%
\pgfpathlineto{\pgfqpoint{5.838152in}{5.489178in}}%
\pgfpathlineto{\pgfqpoint{5.839027in}{5.452515in}}%
\pgfpathlineto{\pgfqpoint{5.839902in}{5.516628in}}%
\pgfpathlineto{\pgfqpoint{5.840778in}{5.244017in}}%
\pgfpathlineto{\pgfqpoint{5.841653in}{4.864664in}}%
\pgfpathlineto{\pgfqpoint{5.842528in}{4.649633in}}%
\pgfpathlineto{\pgfqpoint{5.843404in}{4.887167in}}%
\pgfpathlineto{\pgfqpoint{5.844279in}{4.842160in}}%
\pgfpathlineto{\pgfqpoint{5.845155in}{4.867609in}}%
\pgfpathlineto{\pgfqpoint{5.846030in}{4.655712in}}%
\pgfpathlineto{\pgfqpoint{5.847781in}{4.144395in}}%
\pgfpathlineto{\pgfqpoint{5.849531in}{3.816959in}}%
\pgfpathlineto{\pgfqpoint{5.850407in}{3.718449in}}%
\pgfpathlineto{\pgfqpoint{5.852158in}{3.826172in}}%
\pgfpathlineto{\pgfqpoint{5.853908in}{4.075336in}}%
\pgfpathlineto{\pgfqpoint{5.855659in}{4.542929in}}%
\pgfpathlineto{\pgfqpoint{5.856534in}{4.592467in}}%
\pgfpathlineto{\pgfqpoint{5.857410in}{4.567019in}}%
\pgfpathlineto{\pgfqpoint{5.858285in}{4.733040in}}%
\pgfpathlineto{\pgfqpoint{5.859161in}{5.031553in}}%
\pgfpathlineto{\pgfqpoint{5.860036in}{5.023284in}}%
\pgfpathlineto{\pgfqpoint{5.860911in}{5.073804in}}%
\pgfpathlineto{\pgfqpoint{5.861787in}{4.943502in}}%
\pgfpathlineto{\pgfqpoint{5.862662in}{4.613083in}}%
\pgfpathlineto{\pgfqpoint{5.864413in}{4.354593in}}%
\pgfpathlineto{\pgfqpoint{5.865288in}{4.321781in}}%
\pgfpathlineto{\pgfqpoint{5.866164in}{4.240036in}}%
\pgfpathlineto{\pgfqpoint{5.867914in}{3.764476in}}%
\pgfpathlineto{\pgfqpoint{5.868790in}{3.474005in}}%
\pgfpathlineto{\pgfqpoint{5.869665in}{3.307719in}}%
\pgfpathlineto{\pgfqpoint{5.870540in}{3.249421in}}%
\pgfpathlineto{\pgfqpoint{5.871416in}{3.155178in}}%
\pgfpathlineto{\pgfqpoint{5.873167in}{3.087440in}}%
\pgfpathlineto{\pgfqpoint{5.874042in}{3.311457in}}%
\pgfpathlineto{\pgfqpoint{5.874917in}{3.355596in}}%
\pgfpathlineto{\pgfqpoint{5.875793in}{3.654034in}}%
\pgfpathlineto{\pgfqpoint{5.876668in}{3.749561in}}%
\pgfpathlineto{\pgfqpoint{5.877543in}{3.998461in}}%
\pgfpathlineto{\pgfqpoint{5.878419in}{4.027308in}}%
\pgfpathlineto{\pgfqpoint{5.879294in}{4.038447in}}%
\pgfpathlineto{\pgfqpoint{5.880170in}{4.280550in}}%
\pgfpathlineto{\pgfqpoint{5.881045in}{4.390425in}}%
\pgfpathlineto{\pgfqpoint{5.881920in}{4.191253in}}%
\pgfpathlineto{\pgfqpoint{5.882796in}{4.297390in}}%
\pgfpathlineto{\pgfqpoint{5.883671in}{4.183663in}}%
\pgfpathlineto{\pgfqpoint{5.884546in}{4.172978in}}%
\pgfpathlineto{\pgfqpoint{5.885422in}{4.172223in}}%
\pgfpathlineto{\pgfqpoint{5.886297in}{4.264276in}}%
\pgfpathlineto{\pgfqpoint{5.887173in}{4.205940in}}%
\pgfpathlineto{\pgfqpoint{5.889799in}{3.528036in}}%
\pgfpathlineto{\pgfqpoint{5.890674in}{3.428544in}}%
\pgfpathlineto{\pgfqpoint{5.891549in}{3.406720in}}%
\pgfpathlineto{\pgfqpoint{5.892425in}{3.344307in}}%
\pgfpathlineto{\pgfqpoint{5.893300in}{3.367716in}}%
\pgfpathlineto{\pgfqpoint{5.894176in}{3.340569in}}%
\pgfpathlineto{\pgfqpoint{5.896802in}{4.231465in}}%
\pgfpathlineto{\pgfqpoint{5.897677in}{4.414439in}}%
\pgfpathlineto{\pgfqpoint{5.899428in}{4.633359in}}%
\pgfpathlineto{\pgfqpoint{5.900303in}{4.587257in}}%
\pgfpathlineto{\pgfqpoint{5.901179in}{4.653862in}}%
\pgfpathlineto{\pgfqpoint{5.902054in}{4.751579in}}%
\pgfpathlineto{\pgfqpoint{5.902929in}{4.724393in}}%
\pgfpathlineto{\pgfqpoint{5.903805in}{4.574155in}}%
\pgfpathlineto{\pgfqpoint{5.904680in}{4.472058in}}%
\pgfpathlineto{\pgfqpoint{5.905555in}{4.333109in}}%
\pgfpathlineto{\pgfqpoint{5.906431in}{4.364750in}}%
\pgfpathlineto{\pgfqpoint{5.907306in}{4.362938in}}%
\pgfpathlineto{\pgfqpoint{5.908182in}{4.292632in}}%
\pgfpathlineto{\pgfqpoint{5.909932in}{3.861287in}}%
\pgfpathlineto{\pgfqpoint{5.910808in}{3.471966in}}%
\pgfpathlineto{\pgfqpoint{5.911683in}{3.371379in}}%
\pgfpathlineto{\pgfqpoint{5.913434in}{3.140830in}}%
\pgfpathlineto{\pgfqpoint{5.914309in}{3.111983in}}%
\pgfpathlineto{\pgfqpoint{5.915185in}{3.254216in}}%
\pgfpathlineto{\pgfqpoint{5.917811in}{4.076771in}}%
\pgfpathlineto{\pgfqpoint{5.918686in}{4.318006in}}%
\pgfpathlineto{\pgfqpoint{5.919561in}{4.472284in}}%
\pgfpathlineto{\pgfqpoint{5.921312in}{4.669154in}}%
\pgfpathlineto{\pgfqpoint{5.922188in}{4.493466in}}%
\pgfpathlineto{\pgfqpoint{5.923063in}{4.680783in}}%
\pgfpathlineto{\pgfqpoint{5.923938in}{4.674138in}}%
\pgfpathlineto{\pgfqpoint{5.924814in}{4.687768in}}%
\pgfpathlineto{\pgfqpoint{5.925689in}{4.403489in}}%
\pgfpathlineto{\pgfqpoint{5.926565in}{4.337413in}}%
\pgfpathlineto{\pgfqpoint{5.927440in}{4.408889in}}%
\pgfpathlineto{\pgfqpoint{5.928315in}{4.553615in}}%
\pgfpathlineto{\pgfqpoint{5.929191in}{4.565093in}}%
\pgfpathlineto{\pgfqpoint{5.930066in}{4.430297in}}%
\pgfpathlineto{\pgfqpoint{5.931817in}{3.550880in}}%
\pgfpathlineto{\pgfqpoint{5.932692in}{3.520636in}}%
\pgfpathlineto{\pgfqpoint{5.935318in}{3.243267in}}%
\pgfpathlineto{\pgfqpoint{5.936194in}{3.442855in}}%
\pgfpathlineto{\pgfqpoint{5.937069in}{3.397017in}}%
\pgfpathlineto{\pgfqpoint{5.937944in}{3.398149in}}%
\pgfpathlineto{\pgfqpoint{5.939695in}{3.964026in}}%
\pgfpathlineto{\pgfqpoint{5.940571in}{4.033425in}}%
\pgfpathlineto{\pgfqpoint{5.941446in}{4.147151in}}%
\pgfpathlineto{\pgfqpoint{5.942321in}{4.184418in}}%
\pgfpathlineto{\pgfqpoint{5.943197in}{4.256800in}}%
\pgfpathlineto{\pgfqpoint{5.944072in}{4.500225in}}%
\pgfpathlineto{\pgfqpoint{5.944947in}{4.465563in}}%
\pgfpathlineto{\pgfqpoint{5.946698in}{4.367166in}}%
\pgfpathlineto{\pgfqpoint{5.947574in}{4.267863in}}%
\pgfpathlineto{\pgfqpoint{5.948449in}{4.265673in}}%
\pgfpathlineto{\pgfqpoint{5.949324in}{4.310190in}}%
\pgfpathlineto{\pgfqpoint{5.950200in}{4.377965in}}%
\pgfpathlineto{\pgfqpoint{5.951950in}{3.841804in}}%
\pgfpathlineto{\pgfqpoint{5.952826in}{3.594566in}}%
\pgfpathlineto{\pgfqpoint{5.953701in}{3.461998in}}%
\pgfpathlineto{\pgfqpoint{5.955452in}{3.164315in}}%
\pgfpathlineto{\pgfqpoint{5.956327in}{3.182892in}}%
\pgfpathlineto{\pgfqpoint{5.957203in}{3.224274in}}%
\pgfpathlineto{\pgfqpoint{5.958078in}{3.537627in}}%
\pgfpathlineto{\pgfqpoint{5.961580in}{4.189667in}}%
\pgfpathlineto{\pgfqpoint{5.962455in}{4.259292in}}%
\pgfpathlineto{\pgfqpoint{5.963330in}{4.439171in}}%
\pgfpathlineto{\pgfqpoint{5.964206in}{4.474436in}}%
\pgfpathlineto{\pgfqpoint{5.965081in}{4.729755in}}%
\pgfpathlineto{\pgfqpoint{5.965956in}{4.635889in}}%
\pgfpathlineto{\pgfqpoint{5.966832in}{4.474399in}}%
\pgfpathlineto{\pgfqpoint{5.967707in}{4.377890in}}%
\pgfpathlineto{\pgfqpoint{5.968583in}{4.239054in}}%
\pgfpathlineto{\pgfqpoint{5.969458in}{4.351233in}}%
\pgfpathlineto{\pgfqpoint{5.971209in}{4.481761in}}%
\pgfpathlineto{\pgfqpoint{5.972959in}{4.008127in}}%
\pgfpathlineto{\pgfqpoint{5.973835in}{3.615597in}}%
\pgfpathlineto{\pgfqpoint{5.974710in}{3.556241in}}%
\pgfpathlineto{\pgfqpoint{5.975586in}{3.439872in}}%
\pgfpathlineto{\pgfqpoint{5.976461in}{3.390371in}}%
\pgfpathlineto{\pgfqpoint{5.977336in}{3.191048in}}%
\pgfpathlineto{\pgfqpoint{5.978212in}{3.495527in}}%
\pgfpathlineto{\pgfqpoint{5.980838in}{4.646650in}}%
\pgfpathlineto{\pgfqpoint{5.981713in}{4.729981in}}%
\pgfpathlineto{\pgfqpoint{5.982589in}{4.855942in}}%
\pgfpathlineto{\pgfqpoint{5.983464in}{4.786807in}}%
\pgfpathlineto{\pgfqpoint{5.984339in}{4.852166in}}%
\pgfpathlineto{\pgfqpoint{5.985215in}{4.816900in}}%
\pgfpathlineto{\pgfqpoint{5.986090in}{4.735192in}}%
\pgfpathlineto{\pgfqpoint{5.986965in}{4.736249in}}%
\pgfpathlineto{\pgfqpoint{5.987841in}{4.444306in}}%
\pgfpathlineto{\pgfqpoint{5.988716in}{4.349685in}}%
\pgfpathlineto{\pgfqpoint{5.989592in}{4.175998in}}%
\pgfpathlineto{\pgfqpoint{5.990467in}{4.278662in}}%
\pgfpathlineto{\pgfqpoint{5.992218in}{4.707478in}}%
\pgfpathlineto{\pgfqpoint{5.994844in}{3.751374in}}%
\pgfpathlineto{\pgfqpoint{5.997470in}{3.324446in}}%
\pgfpathlineto{\pgfqpoint{5.998345in}{3.317159in}}%
\pgfpathlineto{\pgfqpoint{5.999221in}{3.371681in}}%
\pgfpathlineto{\pgfqpoint{6.000971in}{3.796419in}}%
\pgfpathlineto{\pgfqpoint{6.002722in}{4.609043in}}%
\pgfpathlineto{\pgfqpoint{6.003598in}{4.630641in}}%
\pgfpathlineto{\pgfqpoint{6.004473in}{4.908236in}}%
\pgfpathlineto{\pgfqpoint{6.005348in}{4.784579in}}%
\pgfpathlineto{\pgfqpoint{6.006224in}{4.881994in}}%
\pgfpathlineto{\pgfqpoint{6.007099in}{4.814899in}}%
\pgfpathlineto{\pgfqpoint{6.007974in}{4.621730in}}%
\pgfpathlineto{\pgfqpoint{6.008850in}{4.571512in}}%
\pgfpathlineto{\pgfqpoint{6.010601in}{4.078432in}}%
\pgfpathlineto{\pgfqpoint{6.011476in}{4.080849in}}%
\pgfpathlineto{\pgfqpoint{6.012351in}{4.210509in}}%
\pgfpathlineto{\pgfqpoint{6.013227in}{4.268279in}}%
\pgfpathlineto{\pgfqpoint{6.014977in}{3.746616in}}%
\pgfpathlineto{\pgfqpoint{6.015853in}{3.484539in}}%
\pgfpathlineto{\pgfqpoint{6.017604in}{3.178361in}}%
\pgfpathlineto{\pgfqpoint{6.018479in}{3.156801in}}%
\pgfpathlineto{\pgfqpoint{6.019354in}{3.182854in}}%
\pgfpathlineto{\pgfqpoint{6.020230in}{3.304963in}}%
\pgfpathlineto{\pgfqpoint{6.021105in}{3.473212in}}%
\pgfpathlineto{\pgfqpoint{6.021980in}{3.553032in}}%
\pgfpathlineto{\pgfqpoint{6.022856in}{3.766288in}}%
\pgfpathlineto{\pgfqpoint{6.023731in}{3.872728in}}%
\pgfpathlineto{\pgfqpoint{6.024607in}{4.151041in}}%
\pgfpathlineto{\pgfqpoint{6.025482in}{4.092667in}}%
\pgfpathlineto{\pgfqpoint{6.027233in}{4.171694in}}%
\pgfpathlineto{\pgfqpoint{6.028983in}{4.565471in}}%
\pgfpathlineto{\pgfqpoint{6.030734in}{4.239318in}}%
\pgfpathlineto{\pgfqpoint{6.031610in}{4.013375in}}%
\pgfpathlineto{\pgfqpoint{6.033360in}{4.285874in}}%
\pgfpathlineto{\pgfqpoint{6.034236in}{4.211453in}}%
\pgfpathlineto{\pgfqpoint{6.035986in}{3.501002in}}%
\pgfpathlineto{\pgfqpoint{6.036862in}{3.145625in}}%
\pgfpathlineto{\pgfqpoint{6.037737in}{2.993008in}}%
\pgfpathlineto{\pgfqpoint{6.038613in}{2.791456in}}%
\pgfpathlineto{\pgfqpoint{6.039488in}{2.741578in}}%
\pgfpathlineto{\pgfqpoint{6.040363in}{2.724549in}}%
\pgfpathlineto{\pgfqpoint{6.041239in}{2.758267in}}%
\pgfpathlineto{\pgfqpoint{6.042989in}{3.223557in}}%
\pgfpathlineto{\pgfqpoint{6.043865in}{3.600267in}}%
\pgfpathlineto{\pgfqpoint{6.044740in}{3.857398in}}%
\pgfpathlineto{\pgfqpoint{6.046491in}{4.022324in}}%
\pgfpathlineto{\pgfqpoint{6.047366in}{3.980904in}}%
\pgfpathlineto{\pgfqpoint{6.048242in}{4.016925in}}%
\pgfpathlineto{\pgfqpoint{6.049117in}{4.200994in}}%
\pgfpathlineto{\pgfqpoint{6.049992in}{4.189063in}}%
\pgfpathlineto{\pgfqpoint{6.050868in}{4.163727in}}%
\pgfpathlineto{\pgfqpoint{6.052619in}{3.845051in}}%
\pgfpathlineto{\pgfqpoint{6.053494in}{4.047886in}}%
\pgfpathlineto{\pgfqpoint{6.054369in}{3.983585in}}%
\pgfpathlineto{\pgfqpoint{6.055245in}{3.846675in}}%
\pgfpathlineto{\pgfqpoint{6.057871in}{2.988288in}}%
\pgfpathlineto{\pgfqpoint{6.060497in}{2.723530in}}%
\pgfpathlineto{\pgfqpoint{6.061372in}{2.703556in}}%
\pgfpathlineto{\pgfqpoint{6.062248in}{2.757134in}}%
\pgfpathlineto{\pgfqpoint{6.063123in}{3.018985in}}%
\pgfpathlineto{\pgfqpoint{6.063999in}{3.196032in}}%
\pgfpathlineto{\pgfqpoint{6.064874in}{3.607819in}}%
\pgfpathlineto{\pgfqpoint{6.065749in}{3.842446in}}%
\pgfpathlineto{\pgfqpoint{6.066625in}{3.776105in}}%
\pgfpathlineto{\pgfqpoint{6.067500in}{3.961081in}}%
\pgfpathlineto{\pgfqpoint{6.068375in}{3.983962in}}%
\pgfpathlineto{\pgfqpoint{6.069251in}{3.994421in}}%
\pgfpathlineto{\pgfqpoint{6.070126in}{3.908258in}}%
\pgfpathlineto{\pgfqpoint{6.071002in}{3.952359in}}%
\pgfpathlineto{\pgfqpoint{6.071877in}{3.897459in}}%
\pgfpathlineto{\pgfqpoint{6.072752in}{3.749071in}}%
\pgfpathlineto{\pgfqpoint{6.073628in}{3.736799in}}%
\pgfpathlineto{\pgfqpoint{6.074503in}{3.784525in}}%
\pgfpathlineto{\pgfqpoint{6.075378in}{3.808426in}}%
\pgfpathlineto{\pgfqpoint{6.076254in}{3.772783in}}%
\pgfpathlineto{\pgfqpoint{6.077129in}{3.540459in}}%
\pgfpathlineto{\pgfqpoint{6.078005in}{3.175151in}}%
\pgfpathlineto{\pgfqpoint{6.079755in}{2.743844in}}%
\pgfpathlineto{\pgfqpoint{6.080631in}{2.605084in}}%
\pgfpathlineto{\pgfqpoint{6.081506in}{2.607462in}}%
\pgfpathlineto{\pgfqpoint{6.082381in}{2.580315in}}%
\pgfpathlineto{\pgfqpoint{6.083257in}{2.519524in}}%
\pgfpathlineto{\pgfqpoint{6.084132in}{2.540065in}}%
\pgfpathlineto{\pgfqpoint{6.085008in}{2.567930in}}%
\pgfpathlineto{\pgfqpoint{6.085883in}{2.870635in}}%
\pgfpathlineto{\pgfqpoint{6.086758in}{3.064294in}}%
\pgfpathlineto{\pgfqpoint{6.087634in}{3.077321in}}%
\pgfpathlineto{\pgfqpoint{6.088509in}{3.067315in}}%
\pgfpathlineto{\pgfqpoint{6.089384in}{3.072261in}}%
\pgfpathlineto{\pgfqpoint{6.090260in}{3.245872in}}%
\pgfpathlineto{\pgfqpoint{6.091135in}{3.481821in}}%
\pgfpathlineto{\pgfqpoint{6.092011in}{3.529244in}}%
\pgfpathlineto{\pgfqpoint{6.092886in}{3.454824in}}%
\pgfpathlineto{\pgfqpoint{6.093761in}{3.415669in}}%
\pgfpathlineto{\pgfqpoint{6.094637in}{3.406116in}}%
\pgfpathlineto{\pgfqpoint{6.095512in}{3.474345in}}%
\pgfpathlineto{\pgfqpoint{6.096387in}{3.365527in}}%
\pgfpathlineto{\pgfqpoint{6.097263in}{3.399244in}}%
\pgfpathlineto{\pgfqpoint{6.098138in}{3.330034in}}%
\pgfpathlineto{\pgfqpoint{6.100764in}{2.690454in}}%
\pgfpathlineto{\pgfqpoint{6.101640in}{2.564305in}}%
\pgfpathlineto{\pgfqpoint{6.103390in}{2.422147in}}%
\pgfpathlineto{\pgfqpoint{6.104266in}{2.645409in}}%
\pgfpathlineto{\pgfqpoint{6.105141in}{2.751018in}}%
\pgfpathlineto{\pgfqpoint{6.106017in}{2.943507in}}%
\pgfpathlineto{\pgfqpoint{6.106892in}{3.373758in}}%
\pgfpathlineto{\pgfqpoint{6.107767in}{3.537136in}}%
\pgfpathlineto{\pgfqpoint{6.108643in}{3.615597in}}%
\pgfpathlineto{\pgfqpoint{6.109518in}{3.613898in}}%
\pgfpathlineto{\pgfqpoint{6.110393in}{3.666494in}}%
\pgfpathlineto{\pgfqpoint{6.112144in}{3.990645in}}%
\pgfpathlineto{\pgfqpoint{6.113020in}{3.999179in}}%
\pgfpathlineto{\pgfqpoint{6.114770in}{3.689413in}}%
\pgfpathlineto{\pgfqpoint{6.115646in}{3.637081in}}%
\pgfpathlineto{\pgfqpoint{6.116521in}{3.782826in}}%
\pgfpathlineto{\pgfqpoint{6.117396in}{3.746881in}}%
\pgfpathlineto{\pgfqpoint{6.118272in}{3.837197in}}%
\pgfpathlineto{\pgfqpoint{6.119147in}{3.710369in}}%
\pgfpathlineto{\pgfqpoint{6.120023in}{3.452558in}}%
\pgfpathlineto{\pgfqpoint{6.120898in}{3.533171in}}%
\pgfpathlineto{\pgfqpoint{6.122649in}{3.247307in}}%
\pgfpathlineto{\pgfqpoint{6.123524in}{3.232166in}}%
\pgfpathlineto{\pgfqpoint{6.124399in}{3.209209in}}%
\pgfpathlineto{\pgfqpoint{6.125275in}{3.389427in}}%
\pgfpathlineto{\pgfqpoint{6.127901in}{4.299391in}}%
\pgfpathlineto{\pgfqpoint{6.128776in}{4.398166in}}%
\pgfpathlineto{\pgfqpoint{6.129652in}{4.058005in}}%
\pgfpathlineto{\pgfqpoint{6.130527in}{4.211642in}}%
\pgfpathlineto{\pgfqpoint{6.131402in}{4.209565in}}%
\pgfpathlineto{\pgfqpoint{6.132278in}{4.401639in}}%
\pgfpathlineto{\pgfqpoint{6.133153in}{4.506002in}}%
\pgfpathlineto{\pgfqpoint{6.134029in}{4.673609in}}%
\pgfpathlineto{\pgfqpoint{6.134904in}{4.424558in}}%
\pgfpathlineto{\pgfqpoint{6.135779in}{4.260047in}}%
\pgfpathlineto{\pgfqpoint{6.136655in}{4.163689in}}%
\pgfpathlineto{\pgfqpoint{6.137530in}{4.330315in}}%
\pgfpathlineto{\pgfqpoint{6.138405in}{4.348854in}}%
\pgfpathlineto{\pgfqpoint{6.139281in}{4.304224in}}%
\pgfpathlineto{\pgfqpoint{6.141032in}{3.912260in}}%
\pgfpathlineto{\pgfqpoint{6.141907in}{3.564888in}}%
\pgfpathlineto{\pgfqpoint{6.142782in}{3.472608in}}%
\pgfpathlineto{\pgfqpoint{6.143658in}{3.300319in}}%
\pgfpathlineto{\pgfqpoint{6.145408in}{3.130824in}}%
\pgfpathlineto{\pgfqpoint{6.146284in}{3.234016in}}%
\pgfpathlineto{\pgfqpoint{6.147159in}{3.480310in}}%
\pgfpathlineto{\pgfqpoint{6.148035in}{3.534304in}}%
\pgfpathlineto{\pgfqpoint{6.148910in}{3.937558in}}%
\pgfpathlineto{\pgfqpoint{6.150661in}{4.174035in}}%
\pgfpathlineto{\pgfqpoint{6.151536in}{4.130916in}}%
\pgfpathlineto{\pgfqpoint{6.152411in}{4.167578in}}%
\pgfpathlineto{\pgfqpoint{6.153287in}{4.238563in}}%
\pgfpathlineto{\pgfqpoint{6.154162in}{4.261633in}}%
\pgfpathlineto{\pgfqpoint{6.155038in}{4.431808in}}%
\pgfpathlineto{\pgfqpoint{6.156788in}{4.072957in}}%
\pgfpathlineto{\pgfqpoint{6.157664in}{3.974447in}}%
\pgfpathlineto{\pgfqpoint{6.158539in}{3.995289in}}%
\pgfpathlineto{\pgfqpoint{6.159414in}{4.035955in}}%
\pgfpathlineto{\pgfqpoint{6.160290in}{4.159423in}}%
\pgfpathlineto{\pgfqpoint{6.162041in}{3.701571in}}%
\pgfpathlineto{\pgfqpoint{6.162916in}{3.410949in}}%
\pgfpathlineto{\pgfqpoint{6.164667in}{3.087402in}}%
\pgfpathlineto{\pgfqpoint{6.165542in}{3.130484in}}%
\pgfpathlineto{\pgfqpoint{6.167293in}{3.092160in}}%
\pgfpathlineto{\pgfqpoint{6.168168in}{3.178738in}}%
\pgfpathlineto{\pgfqpoint{6.169044in}{3.177077in}}%
\pgfpathlineto{\pgfqpoint{6.171670in}{4.050605in}}%
\pgfpathlineto{\pgfqpoint{6.172545in}{3.780674in}}%
\pgfpathlineto{\pgfqpoint{6.173420in}{4.094706in}}%
\pgfpathlineto{\pgfqpoint{6.174296in}{4.038522in}}%
\pgfpathlineto{\pgfqpoint{6.175171in}{4.062612in}}%
\pgfpathlineto{\pgfqpoint{6.176047in}{4.223120in}}%
\pgfpathlineto{\pgfqpoint{6.176922in}{4.127857in}}%
\pgfpathlineto{\pgfqpoint{6.177797in}{4.103503in}}%
\pgfpathlineto{\pgfqpoint{6.178673in}{4.067407in}}%
\pgfpathlineto{\pgfqpoint{6.180423in}{4.407227in}}%
\pgfpathlineto{\pgfqpoint{6.181299in}{4.403829in}}%
\pgfpathlineto{\pgfqpoint{6.182174in}{4.223385in}}%
\pgfpathlineto{\pgfqpoint{6.183050in}{3.970860in}}%
\pgfpathlineto{\pgfqpoint{6.183925in}{3.615219in}}%
\pgfpathlineto{\pgfqpoint{6.184800in}{3.553787in}}%
\pgfpathlineto{\pgfqpoint{6.186551in}{3.303075in}}%
\pgfpathlineto{\pgfqpoint{6.187426in}{3.240322in}}%
\pgfpathlineto{\pgfqpoint{6.188302in}{3.447310in}}%
\pgfpathlineto{\pgfqpoint{6.189177in}{3.742123in}}%
\pgfpathlineto{\pgfqpoint{6.190053in}{3.741368in}}%
\pgfpathlineto{\pgfqpoint{6.190928in}{4.201296in}}%
\pgfpathlineto{\pgfqpoint{6.192679in}{4.664321in}}%
\pgfpathlineto{\pgfqpoint{6.193554in}{4.571210in}}%
\pgfpathlineto{\pgfqpoint{6.194430in}{4.671683in}}%
\pgfpathlineto{\pgfqpoint{6.195305in}{4.573400in}}%
\pgfpathlineto{\pgfqpoint{6.196180in}{4.669191in}}%
\pgfpathlineto{\pgfqpoint{6.197931in}{4.512723in}}%
\pgfpathlineto{\pgfqpoint{6.198806in}{4.314456in}}%
\pgfpathlineto{\pgfqpoint{6.199682in}{4.282966in}}%
\pgfpathlineto{\pgfqpoint{6.200557in}{4.315627in}}%
\pgfpathlineto{\pgfqpoint{6.202308in}{4.661375in}}%
\pgfpathlineto{\pgfqpoint{6.203183in}{4.440001in}}%
\pgfpathlineto{\pgfqpoint{6.205809in}{3.571042in}}%
\pgfpathlineto{\pgfqpoint{6.206685in}{3.533247in}}%
\pgfpathlineto{\pgfqpoint{6.207560in}{3.454862in}}%
\pgfpathlineto{\pgfqpoint{6.208436in}{3.427261in}}%
\pgfpathlineto{\pgfqpoint{6.209311in}{3.588562in}}%
\pgfpathlineto{\pgfqpoint{6.210186in}{3.912789in}}%
\pgfpathlineto{\pgfqpoint{6.211062in}{3.966216in}}%
\pgfpathlineto{\pgfqpoint{6.211937in}{4.050756in}}%
\pgfpathlineto{\pgfqpoint{6.212812in}{4.175810in}}%
\pgfpathlineto{\pgfqpoint{6.213688in}{4.352856in}}%
\pgfpathlineto{\pgfqpoint{6.214563in}{4.446949in}}%
\pgfpathlineto{\pgfqpoint{6.215439in}{4.670966in}}%
\pgfpathlineto{\pgfqpoint{6.216314in}{4.653409in}}%
\pgfpathlineto{\pgfqpoint{6.217189in}{4.848277in}}%
\pgfpathlineto{\pgfqpoint{6.218065in}{4.776877in}}%
\pgfpathlineto{\pgfqpoint{6.219815in}{4.448912in}}%
\pgfpathlineto{\pgfqpoint{6.220691in}{4.396731in}}%
\pgfpathlineto{\pgfqpoint{6.221566in}{4.497733in}}%
\pgfpathlineto{\pgfqpoint{6.223317in}{4.817353in}}%
\pgfpathlineto{\pgfqpoint{6.227694in}{3.532076in}}%
\pgfpathlineto{\pgfqpoint{6.228569in}{3.576555in}}%
\pgfpathlineto{\pgfqpoint{6.229445in}{3.437946in}}%
\pgfpathlineto{\pgfqpoint{6.230320in}{3.496697in}}%
\pgfpathlineto{\pgfqpoint{6.231195in}{3.663700in}}%
\pgfpathlineto{\pgfqpoint{6.232071in}{3.595207in}}%
\pgfpathlineto{\pgfqpoint{6.232946in}{3.669137in}}%
\pgfpathlineto{\pgfqpoint{6.233821in}{3.612614in}}%
\pgfpathlineto{\pgfqpoint{6.234697in}{3.839840in}}%
\pgfpathlineto{\pgfqpoint{6.235572in}{3.892248in}}%
\pgfpathlineto{\pgfqpoint{6.236448in}{3.879222in}}%
\pgfpathlineto{\pgfqpoint{6.237323in}{3.880845in}}%
\pgfpathlineto{\pgfqpoint{6.238198in}{3.805292in}}%
\pgfpathlineto{\pgfqpoint{6.239074in}{3.862231in}}%
\pgfpathlineto{\pgfqpoint{6.239949in}{4.016396in}}%
\pgfpathlineto{\pgfqpoint{6.241700in}{3.794833in}}%
\pgfpathlineto{\pgfqpoint{6.242575in}{3.900593in}}%
\pgfpathlineto{\pgfqpoint{6.244326in}{4.217645in}}%
\pgfpathlineto{\pgfqpoint{6.247827in}{3.220876in}}%
\pgfpathlineto{\pgfqpoint{6.248703in}{3.097484in}}%
\pgfpathlineto{\pgfqpoint{6.249578in}{3.051230in}}%
\pgfpathlineto{\pgfqpoint{6.250454in}{3.065616in}}%
\pgfpathlineto{\pgfqpoint{6.251329in}{3.186705in}}%
\pgfpathlineto{\pgfqpoint{6.252204in}{3.353444in}}%
\pgfpathlineto{\pgfqpoint{6.255706in}{4.453632in}}%
\pgfpathlineto{\pgfqpoint{6.256581in}{4.513214in}}%
\pgfpathlineto{\pgfqpoint{6.257457in}{4.506266in}}%
\pgfpathlineto{\pgfqpoint{6.258332in}{4.449252in}}%
\pgfpathlineto{\pgfqpoint{6.259207in}{4.640495in}}%
\pgfpathlineto{\pgfqpoint{6.260083in}{4.498903in}}%
\pgfpathlineto{\pgfqpoint{6.260958in}{4.439812in}}%
\pgfpathlineto{\pgfqpoint{6.262709in}{4.171430in}}%
\pgfpathlineto{\pgfqpoint{6.263584in}{4.222327in}}%
\pgfpathlineto{\pgfqpoint{6.264460in}{4.353611in}}%
\pgfpathlineto{\pgfqpoint{6.265335in}{4.561204in}}%
\pgfpathlineto{\pgfqpoint{6.267961in}{3.620543in}}%
\pgfpathlineto{\pgfqpoint{6.268836in}{3.503229in}}%
\pgfpathlineto{\pgfqpoint{6.269712in}{3.300356in}}%
\pgfpathlineto{\pgfqpoint{6.271463in}{3.089441in}}%
\pgfpathlineto{\pgfqpoint{6.272338in}{3.291785in}}%
\pgfpathlineto{\pgfqpoint{6.273213in}{3.782675in}}%
\pgfpathlineto{\pgfqpoint{6.274089in}{4.000160in}}%
\pgfpathlineto{\pgfqpoint{6.275839in}{4.931722in}}%
\pgfpathlineto{\pgfqpoint{6.276715in}{4.921753in}}%
\pgfpathlineto{\pgfqpoint{6.277590in}{4.995910in}}%
\pgfpathlineto{\pgfqpoint{6.278466in}{4.866061in}}%
\pgfpathlineto{\pgfqpoint{6.279341in}{5.080676in}}%
\pgfpathlineto{\pgfqpoint{6.280216in}{4.925982in}}%
\pgfpathlineto{\pgfqpoint{6.281092in}{4.948713in}}%
\pgfpathlineto{\pgfqpoint{6.281967in}{4.600510in}}%
\pgfpathlineto{\pgfqpoint{6.282842in}{4.390501in}}%
\pgfpathlineto{\pgfqpoint{6.283718in}{4.299165in}}%
\pgfpathlineto{\pgfqpoint{6.284593in}{4.317326in}}%
\pgfpathlineto{\pgfqpoint{6.285469in}{4.459749in}}%
\pgfpathlineto{\pgfqpoint{6.286344in}{4.364070in}}%
\pgfpathlineto{\pgfqpoint{6.288970in}{3.356389in}}%
\pgfpathlineto{\pgfqpoint{6.289845in}{3.133958in}}%
\pgfpathlineto{\pgfqpoint{6.291596in}{2.919078in}}%
\pgfpathlineto{\pgfqpoint{6.292472in}{2.929272in}}%
\pgfpathlineto{\pgfqpoint{6.293347in}{2.975262in}}%
\pgfpathlineto{\pgfqpoint{6.294222in}{3.106734in}}%
\pgfpathlineto{\pgfqpoint{6.295098in}{3.310891in}}%
\pgfpathlineto{\pgfqpoint{6.296848in}{4.001746in}}%
\pgfpathlineto{\pgfqpoint{6.297724in}{4.097689in}}%
\pgfpathlineto{\pgfqpoint{6.298599in}{4.047546in}}%
\pgfpathlineto{\pgfqpoint{6.299475in}{4.073637in}}%
\pgfpathlineto{\pgfqpoint{6.300350in}{4.243396in}}%
\pgfpathlineto{\pgfqpoint{6.301225in}{4.266580in}}%
\pgfpathlineto{\pgfqpoint{6.302101in}{4.350364in}}%
\pgfpathlineto{\pgfqpoint{6.302976in}{4.251363in}}%
\pgfpathlineto{\pgfqpoint{6.303851in}{4.114302in}}%
\pgfpathlineto{\pgfqpoint{6.304727in}{4.027195in}}%
\pgfpathlineto{\pgfqpoint{6.305602in}{3.983358in}}%
\pgfpathlineto{\pgfqpoint{6.306478in}{4.000160in}}%
\pgfpathlineto{\pgfqpoint{6.307353in}{3.878353in}}%
\pgfpathlineto{\pgfqpoint{6.309979in}{3.007431in}}%
\pgfpathlineto{\pgfqpoint{6.312605in}{2.720019in}}%
\pgfpathlineto{\pgfqpoint{6.313481in}{2.687094in}}%
\pgfpathlineto{\pgfqpoint{6.314356in}{2.665270in}}%
\pgfpathlineto{\pgfqpoint{6.315231in}{2.732063in}}%
\pgfpathlineto{\pgfqpoint{6.316107in}{2.879357in}}%
\pgfpathlineto{\pgfqpoint{6.316982in}{3.215288in}}%
\pgfpathlineto{\pgfqpoint{6.318733in}{3.620241in}}%
\pgfpathlineto{\pgfqpoint{6.319608in}{3.660831in}}%
\pgfpathlineto{\pgfqpoint{6.322234in}{3.894627in}}%
\pgfpathlineto{\pgfqpoint{6.323110in}{4.045319in}}%
\pgfpathlineto{\pgfqpoint{6.323985in}{4.005748in}}%
\pgfpathlineto{\pgfqpoint{6.324861in}{3.798873in}}%
\pgfpathlineto{\pgfqpoint{6.325736in}{3.732231in}}%
\pgfpathlineto{\pgfqpoint{6.326611in}{3.785318in}}%
\pgfpathlineto{\pgfqpoint{6.327487in}{3.788867in}}%
\pgfpathlineto{\pgfqpoint{6.328362in}{3.965461in}}%
\pgfpathlineto{\pgfqpoint{6.330988in}{3.109189in}}%
\pgfpathlineto{\pgfqpoint{6.331864in}{2.932784in}}%
\pgfpathlineto{\pgfqpoint{6.332739in}{2.868935in}}%
\pgfpathlineto{\pgfqpoint{6.333614in}{2.764875in}}%
\pgfpathlineto{\pgfqpoint{6.334490in}{2.701819in}}%
\pgfpathlineto{\pgfqpoint{6.335365in}{2.820945in}}%
\pgfpathlineto{\pgfqpoint{6.336240in}{3.118213in}}%
\pgfpathlineto{\pgfqpoint{6.337116in}{3.248666in}}%
\pgfpathlineto{\pgfqpoint{6.337991in}{3.591243in}}%
\pgfpathlineto{\pgfqpoint{6.338867in}{4.052002in}}%
\pgfpathlineto{\pgfqpoint{6.339742in}{4.266240in}}%
\pgfpathlineto{\pgfqpoint{6.340617in}{4.172789in}}%
\pgfpathlineto{\pgfqpoint{6.341493in}{4.239281in}}%
\pgfpathlineto{\pgfqpoint{6.342368in}{4.443211in}}%
\pgfpathlineto{\pgfqpoint{6.343243in}{4.389066in}}%
\pgfpathlineto{\pgfqpoint{6.344119in}{4.500074in}}%
\pgfpathlineto{\pgfqpoint{6.344994in}{4.386347in}}%
\pgfpathlineto{\pgfqpoint{6.345870in}{4.185853in}}%
\pgfpathlineto{\pgfqpoint{6.346745in}{4.134729in}}%
\pgfpathlineto{\pgfqpoint{6.347620in}{4.064500in}}%
\pgfpathlineto{\pgfqpoint{6.348496in}{4.213907in}}%
\pgfpathlineto{\pgfqpoint{6.349371in}{4.449705in}}%
\pgfpathlineto{\pgfqpoint{6.351122in}{3.930497in}}%
\pgfpathlineto{\pgfqpoint{6.351997in}{3.687110in}}%
\pgfpathlineto{\pgfqpoint{6.352873in}{3.576668in}}%
\pgfpathlineto{\pgfqpoint{6.353748in}{3.395695in}}%
\pgfpathlineto{\pgfqpoint{6.354623in}{3.368849in}}%
\pgfpathlineto{\pgfqpoint{6.355499in}{3.281553in}}%
\pgfpathlineto{\pgfqpoint{6.356374in}{3.418841in}}%
\pgfpathlineto{\pgfqpoint{6.357249in}{3.755074in}}%
\pgfpathlineto{\pgfqpoint{6.358125in}{3.945411in}}%
\pgfpathlineto{\pgfqpoint{6.359876in}{4.490144in}}%
\pgfpathlineto{\pgfqpoint{6.360751in}{4.483385in}}%
\pgfpathlineto{\pgfqpoint{6.361626in}{4.573928in}}%
\pgfpathlineto{\pgfqpoint{6.362502in}{4.503548in}}%
\pgfpathlineto{\pgfqpoint{6.363377in}{4.522427in}}%
\pgfpathlineto{\pgfqpoint{6.364252in}{4.752221in}}%
\pgfpathlineto{\pgfqpoint{6.365128in}{4.683388in}}%
\pgfpathlineto{\pgfqpoint{6.366003in}{4.590693in}}%
\pgfpathlineto{\pgfqpoint{6.366879in}{4.366260in}}%
\pgfpathlineto{\pgfqpoint{6.367754in}{4.229161in}}%
\pgfpathlineto{\pgfqpoint{6.368629in}{4.310718in}}%
\pgfpathlineto{\pgfqpoint{6.369505in}{4.328238in}}%
\pgfpathlineto{\pgfqpoint{6.370380in}{4.421387in}}%
\pgfpathlineto{\pgfqpoint{6.371255in}{4.154363in}}%
\pgfpathlineto{\pgfqpoint{6.373006in}{3.447763in}}%
\pgfpathlineto{\pgfqpoint{6.373882in}{3.324559in}}%
\pgfpathlineto{\pgfqpoint{6.374757in}{3.142566in}}%
\pgfpathlineto{\pgfqpoint{6.376508in}{3.001692in}}%
\pgfpathlineto{\pgfqpoint{6.377383in}{3.083853in}}%
\pgfpathlineto{\pgfqpoint{6.378258in}{3.136110in}}%
\pgfpathlineto{\pgfqpoint{6.379134in}{3.157481in}}%
\pgfpathlineto{\pgfqpoint{6.380009in}{3.400943in}}%
\pgfpathlineto{\pgfqpoint{6.381760in}{3.485068in}}%
\pgfpathlineto{\pgfqpoint{6.382635in}{3.534493in}}%
\pgfpathlineto{\pgfqpoint{6.383511in}{3.562811in}}%
\pgfpathlineto{\pgfqpoint{6.384386in}{3.825077in}}%
\pgfpathlineto{\pgfqpoint{6.385261in}{3.872916in}}%
\pgfpathlineto{\pgfqpoint{6.386137in}{3.903236in}}%
\pgfpathlineto{\pgfqpoint{6.387012in}{3.902821in}}%
\pgfpathlineto{\pgfqpoint{6.387888in}{3.763645in}}%
\pgfpathlineto{\pgfqpoint{6.388763in}{3.764174in}}%
\pgfpathlineto{\pgfqpoint{6.389638in}{3.760964in}}%
\pgfpathlineto{\pgfqpoint{6.390514in}{3.848563in}}%
\pgfpathlineto{\pgfqpoint{6.391389in}{4.062687in}}%
\pgfpathlineto{\pgfqpoint{6.392264in}{3.809445in}}%
\pgfpathlineto{\pgfqpoint{6.393140in}{3.387351in}}%
\pgfpathlineto{\pgfqpoint{6.394015in}{3.114966in}}%
\pgfpathlineto{\pgfqpoint{6.394891in}{3.009848in}}%
\pgfpathlineto{\pgfqpoint{6.396641in}{2.673916in}}%
\pgfpathlineto{\pgfqpoint{6.397517in}{2.608029in}}%
\pgfpathlineto{\pgfqpoint{6.398392in}{2.639859in}}%
\pgfpathlineto{\pgfqpoint{6.399267in}{2.951172in}}%
\pgfpathlineto{\pgfqpoint{6.400143in}{3.001201in}}%
\pgfpathlineto{\pgfqpoint{6.401018in}{3.116174in}}%
\pgfpathlineto{\pgfqpoint{6.401894in}{3.070298in}}%
\pgfpathlineto{\pgfqpoint{6.403644in}{3.351216in}}%
\pgfpathlineto{\pgfqpoint{6.404520in}{3.366584in}}%
\pgfpathlineto{\pgfqpoint{6.405395in}{3.525355in}}%
\pgfpathlineto{\pgfqpoint{6.406270in}{3.497188in}}%
\pgfpathlineto{\pgfqpoint{6.407146in}{3.558469in}}%
\pgfpathlineto{\pgfqpoint{6.408021in}{3.560961in}}%
\pgfpathlineto{\pgfqpoint{6.408897in}{3.454144in}}%
\pgfpathlineto{\pgfqpoint{6.409772in}{3.447234in}}%
\pgfpathlineto{\pgfqpoint{6.410647in}{3.271056in}}%
\pgfpathlineto{\pgfqpoint{6.411523in}{3.372247in}}%
\pgfpathlineto{\pgfqpoint{6.412398in}{3.360240in}}%
\pgfpathlineto{\pgfqpoint{6.414149in}{2.739992in}}%
\pgfpathlineto{\pgfqpoint{6.415024in}{2.625548in}}%
\pgfpathlineto{\pgfqpoint{6.415900in}{2.191371in}}%
\pgfpathlineto{\pgfqpoint{6.416775in}{2.228298in}}%
\pgfpathlineto{\pgfqpoint{6.418526in}{2.144514in}}%
\pgfpathlineto{\pgfqpoint{6.419401in}{2.222332in}}%
\pgfpathlineto{\pgfqpoint{6.420276in}{2.715903in}}%
\pgfpathlineto{\pgfqpoint{6.421152in}{2.915868in}}%
\pgfpathlineto{\pgfqpoint{6.422903in}{3.506137in}}%
\pgfpathlineto{\pgfqpoint{6.423778in}{3.524827in}}%
\pgfpathlineto{\pgfqpoint{6.424653in}{3.638063in}}%
\pgfpathlineto{\pgfqpoint{6.425529in}{3.671063in}}%
\pgfpathlineto{\pgfqpoint{6.426404in}{3.724717in}}%
\pgfpathlineto{\pgfqpoint{6.427279in}{3.856227in}}%
\pgfpathlineto{\pgfqpoint{6.428155in}{4.038296in}}%
\pgfpathlineto{\pgfqpoint{6.430781in}{3.412120in}}%
\pgfpathlineto{\pgfqpoint{6.432532in}{3.493714in}}%
\pgfpathlineto{\pgfqpoint{6.434282in}{3.290993in}}%
\pgfpathlineto{\pgfqpoint{6.436033in}{2.656850in}}%
\pgfpathlineto{\pgfqpoint{6.437784in}{2.435249in}}%
\pgfpathlineto{\pgfqpoint{6.438659in}{2.431398in}}%
\pgfpathlineto{\pgfqpoint{6.439535in}{2.423166in}}%
\pgfpathlineto{\pgfqpoint{6.440410in}{2.463869in}}%
\pgfpathlineto{\pgfqpoint{6.442161in}{2.666289in}}%
\pgfpathlineto{\pgfqpoint{6.443036in}{3.029368in}}%
\pgfpathlineto{\pgfqpoint{6.444787in}{2.979377in}}%
\pgfpathlineto{\pgfqpoint{6.445662in}{3.175227in}}%
\pgfpathlineto{\pgfqpoint{6.446538in}{3.257199in}}%
\pgfpathlineto{\pgfqpoint{6.447413in}{3.158840in}}%
\pgfpathlineto{\pgfqpoint{6.448288in}{3.289369in}}%
\pgfpathlineto{\pgfqpoint{6.449164in}{3.227069in}}%
\pgfpathlineto{\pgfqpoint{6.450039in}{3.197353in}}%
\pgfpathlineto{\pgfqpoint{6.450915in}{2.943620in}}%
\pgfpathlineto{\pgfqpoint{6.451790in}{2.839107in}}%
\pgfpathlineto{\pgfqpoint{6.452665in}{2.776995in}}%
\pgfpathlineto{\pgfqpoint{6.453541in}{2.806748in}}%
\pgfpathlineto{\pgfqpoint{6.454416in}{2.801500in}}%
\pgfpathlineto{\pgfqpoint{6.456167in}{2.389675in}}%
\pgfpathlineto{\pgfqpoint{6.457042in}{2.136433in}}%
\pgfpathlineto{\pgfqpoint{6.458793in}{2.078022in}}%
\pgfpathlineto{\pgfqpoint{6.460544in}{1.846869in}}%
\pgfpathlineto{\pgfqpoint{6.461419in}{1.953799in}}%
\pgfpathlineto{\pgfqpoint{6.462295in}{2.163279in}}%
\pgfpathlineto{\pgfqpoint{6.463170in}{2.112117in}}%
\pgfpathlineto{\pgfqpoint{6.465796in}{2.565665in}}%
\pgfpathlineto{\pgfqpoint{6.466671in}{2.674483in}}%
\pgfpathlineto{\pgfqpoint{6.467547in}{2.725380in}}%
\pgfpathlineto{\pgfqpoint{6.468422in}{3.063577in}}%
\pgfpathlineto{\pgfqpoint{6.469298in}{3.054402in}}%
\pgfpathlineto{\pgfqpoint{6.470173in}{3.035372in}}%
\pgfpathlineto{\pgfqpoint{6.471048in}{3.217478in}}%
\pgfpathlineto{\pgfqpoint{6.471924in}{2.935729in}}%
\pgfpathlineto{\pgfqpoint{6.472799in}{2.978169in}}%
\pgfpathlineto{\pgfqpoint{6.473674in}{2.984399in}}%
\pgfpathlineto{\pgfqpoint{6.474550in}{3.081588in}}%
\pgfpathlineto{\pgfqpoint{6.475425in}{3.123083in}}%
\pgfpathlineto{\pgfqpoint{6.476301in}{2.951550in}}%
\pgfpathlineto{\pgfqpoint{6.477176in}{2.671198in}}%
\pgfpathlineto{\pgfqpoint{6.479802in}{2.336210in}}%
\pgfpathlineto{\pgfqpoint{6.480677in}{2.229129in}}%
\pgfpathlineto{\pgfqpoint{6.481553in}{2.229620in}}%
\pgfpathlineto{\pgfqpoint{6.482428in}{2.174871in}}%
\pgfpathlineto{\pgfqpoint{6.484179in}{2.494038in}}%
\pgfpathlineto{\pgfqpoint{6.485930in}{3.107225in}}%
\pgfpathlineto{\pgfqpoint{6.487680in}{3.259691in}}%
\pgfpathlineto{\pgfqpoint{6.489431in}{3.354010in}}%
\pgfpathlineto{\pgfqpoint{6.490307in}{3.568022in}}%
\pgfpathlineto{\pgfqpoint{6.491182in}{3.563302in}}%
\pgfpathlineto{\pgfqpoint{6.492057in}{3.474722in}}%
\pgfpathlineto{\pgfqpoint{6.492933in}{3.276078in}}%
\pgfpathlineto{\pgfqpoint{6.493808in}{3.173868in}}%
\pgfpathlineto{\pgfqpoint{6.494683in}{3.117911in}}%
\pgfpathlineto{\pgfqpoint{6.495559in}{3.195578in}}%
\pgfpathlineto{\pgfqpoint{6.496434in}{3.417821in}}%
\pgfpathlineto{\pgfqpoint{6.497310in}{3.361033in}}%
\pgfpathlineto{\pgfqpoint{6.499060in}{2.852171in}}%
\pgfpathlineto{\pgfqpoint{6.499936in}{2.749092in}}%
\pgfpathlineto{\pgfqpoint{6.500811in}{2.734820in}}%
\pgfpathlineto{\pgfqpoint{6.501686in}{2.697855in}}%
\pgfpathlineto{\pgfqpoint{6.502562in}{2.635630in}}%
\pgfpathlineto{\pgfqpoint{6.503437in}{2.650544in}}%
\pgfpathlineto{\pgfqpoint{6.504313in}{2.888079in}}%
\pgfpathlineto{\pgfqpoint{6.505188in}{3.043452in}}%
\pgfpathlineto{\pgfqpoint{6.506939in}{3.489976in}}%
\pgfpathlineto{\pgfqpoint{6.507814in}{3.788565in}}%
\pgfpathlineto{\pgfqpoint{6.508689in}{3.589279in}}%
\pgfpathlineto{\pgfqpoint{6.509565in}{3.533285in}}%
\pgfpathlineto{\pgfqpoint{6.510440in}{3.614011in}}%
\pgfpathlineto{\pgfqpoint{6.512191in}{3.969878in}}%
\pgfpathlineto{\pgfqpoint{6.513066in}{3.780145in}}%
\pgfpathlineto{\pgfqpoint{6.513942in}{3.481481in}}%
\pgfpathlineto{\pgfqpoint{6.515692in}{3.199770in}}%
\pgfpathlineto{\pgfqpoint{6.516568in}{3.201846in}}%
\pgfpathlineto{\pgfqpoint{6.517443in}{3.226578in}}%
\pgfpathlineto{\pgfqpoint{6.518319in}{3.204489in}}%
\pgfpathlineto{\pgfqpoint{6.519194in}{3.089894in}}%
\pgfpathlineto{\pgfqpoint{6.520945in}{2.704613in}}%
\pgfpathlineto{\pgfqpoint{6.521820in}{2.641633in}}%
\pgfpathlineto{\pgfqpoint{6.522695in}{2.547918in}}%
\pgfpathlineto{\pgfqpoint{6.523571in}{2.569138in}}%
\pgfpathlineto{\pgfqpoint{6.524446in}{2.608482in}}%
\pgfpathlineto{\pgfqpoint{6.525322in}{2.855871in}}%
\pgfpathlineto{\pgfqpoint{6.526197in}{2.938561in}}%
\pgfpathlineto{\pgfqpoint{6.527072in}{3.128936in}}%
\pgfpathlineto{\pgfqpoint{6.527948in}{3.420426in}}%
\pgfpathlineto{\pgfqpoint{6.528823in}{3.587656in}}%
\pgfpathlineto{\pgfqpoint{6.529698in}{3.481632in}}%
\pgfpathlineto{\pgfqpoint{6.531449in}{3.608498in}}%
\pgfpathlineto{\pgfqpoint{6.532325in}{3.727322in}}%
\pgfpathlineto{\pgfqpoint{6.533200in}{3.716939in}}%
\pgfpathlineto{\pgfqpoint{6.534075in}{3.563680in}}%
\pgfpathlineto{\pgfqpoint{6.534951in}{3.356465in}}%
\pgfpathlineto{\pgfqpoint{6.535826in}{3.392184in}}%
\pgfpathlineto{\pgfqpoint{6.536701in}{3.392486in}}%
\pgfpathlineto{\pgfqpoint{6.537577in}{3.487409in}}%
\pgfpathlineto{\pgfqpoint{6.538452in}{3.517766in}}%
\pgfpathlineto{\pgfqpoint{6.539328in}{3.472494in}}%
\pgfpathlineto{\pgfqpoint{6.541954in}{2.922363in}}%
\pgfpathlineto{\pgfqpoint{6.542829in}{2.878941in}}%
\pgfpathlineto{\pgfqpoint{6.543704in}{2.851756in}}%
\pgfpathlineto{\pgfqpoint{6.544580in}{2.894649in}}%
\pgfpathlineto{\pgfqpoint{6.545455in}{2.917870in}}%
\pgfpathlineto{\pgfqpoint{6.546331in}{3.005015in}}%
\pgfpathlineto{\pgfqpoint{6.547206in}{3.187800in}}%
\pgfpathlineto{\pgfqpoint{6.548081in}{3.429035in}}%
\pgfpathlineto{\pgfqpoint{6.548957in}{3.456070in}}%
\pgfpathlineto{\pgfqpoint{6.549832in}{3.532190in}}%
\pgfpathlineto{\pgfqpoint{6.550707in}{3.575989in}}%
\pgfpathlineto{\pgfqpoint{6.551583in}{3.585126in}}%
\pgfpathlineto{\pgfqpoint{6.552458in}{3.690773in}}%
\pgfpathlineto{\pgfqpoint{6.553334in}{3.613407in}}%
\pgfpathlineto{\pgfqpoint{6.554209in}{3.704554in}}%
\pgfpathlineto{\pgfqpoint{6.555960in}{3.288236in}}%
\pgfpathlineto{\pgfqpoint{6.556835in}{3.184780in}}%
\pgfpathlineto{\pgfqpoint{6.557710in}{3.196711in}}%
\pgfpathlineto{\pgfqpoint{6.558586in}{3.187385in}}%
\pgfpathlineto{\pgfqpoint{6.559461in}{3.335698in}}%
\pgfpathlineto{\pgfqpoint{6.560337in}{3.258483in}}%
\pgfpathlineto{\pgfqpoint{6.562087in}{2.755133in}}%
\pgfpathlineto{\pgfqpoint{6.563838in}{2.434305in}}%
\pgfpathlineto{\pgfqpoint{6.564713in}{2.392809in}}%
\pgfpathlineto{\pgfqpoint{6.565589in}{2.379896in}}%
\pgfpathlineto{\pgfqpoint{6.566464in}{2.446387in}}%
\pgfpathlineto{\pgfqpoint{6.567340in}{2.857570in}}%
\pgfpathlineto{\pgfqpoint{6.568215in}{2.975903in}}%
\pgfpathlineto{\pgfqpoint{6.569090in}{3.424278in}}%
\pgfpathlineto{\pgfqpoint{6.571716in}{3.690131in}}%
\pgfpathlineto{\pgfqpoint{6.572592in}{4.022437in}}%
\pgfpathlineto{\pgfqpoint{6.574343in}{4.463902in}}%
\pgfpathlineto{\pgfqpoint{6.576093in}{4.350477in}}%
\pgfpathlineto{\pgfqpoint{6.576969in}{4.009184in}}%
\pgfpathlineto{\pgfqpoint{6.577844in}{3.913959in}}%
\pgfpathlineto{\pgfqpoint{6.578719in}{3.784676in}}%
\pgfpathlineto{\pgfqpoint{6.579595in}{3.859210in}}%
\pgfpathlineto{\pgfqpoint{6.580470in}{3.872237in}}%
\pgfpathlineto{\pgfqpoint{6.581346in}{3.753035in}}%
\pgfpathlineto{\pgfqpoint{6.582221in}{3.528867in}}%
\pgfpathlineto{\pgfqpoint{6.583096in}{3.459393in}}%
\pgfpathlineto{\pgfqpoint{6.583972in}{3.267620in}}%
\pgfpathlineto{\pgfqpoint{6.584847in}{3.264071in}}%
\pgfpathlineto{\pgfqpoint{6.586598in}{3.204905in}}%
\pgfpathlineto{\pgfqpoint{6.587473in}{3.369416in}}%
\pgfpathlineto{\pgfqpoint{6.588349in}{3.641121in}}%
\pgfpathlineto{\pgfqpoint{6.589224in}{3.672913in}}%
\pgfpathlineto{\pgfqpoint{6.590099in}{3.887642in}}%
\pgfpathlineto{\pgfqpoint{6.590975in}{4.004955in}}%
\pgfpathlineto{\pgfqpoint{6.591850in}{3.968293in}}%
\pgfpathlineto{\pgfqpoint{6.592726in}{4.170561in}}%
\pgfpathlineto{\pgfqpoint{6.593601in}{4.239092in}}%
\pgfpathlineto{\pgfqpoint{6.594476in}{4.144886in}}%
\pgfpathlineto{\pgfqpoint{6.595352in}{4.142092in}}%
\pgfpathlineto{\pgfqpoint{6.596227in}{4.125063in}}%
\pgfpathlineto{\pgfqpoint{6.597102in}{4.010921in}}%
\pgfpathlineto{\pgfqpoint{6.597978in}{3.807822in}}%
\pgfpathlineto{\pgfqpoint{6.598853in}{3.712559in}}%
\pgfpathlineto{\pgfqpoint{6.599729in}{3.663738in}}%
\pgfpathlineto{\pgfqpoint{6.600604in}{3.575045in}}%
\pgfpathlineto{\pgfqpoint{6.601479in}{3.599474in}}%
\pgfpathlineto{\pgfqpoint{6.604105in}{2.945357in}}%
\pgfpathlineto{\pgfqpoint{6.604981in}{2.872334in}}%
\pgfpathlineto{\pgfqpoint{6.605856in}{2.842958in}}%
\pgfpathlineto{\pgfqpoint{6.607607in}{2.696080in}}%
\pgfpathlineto{\pgfqpoint{6.610233in}{3.060821in}}%
\pgfpathlineto{\pgfqpoint{6.611108in}{3.429753in}}%
\pgfpathlineto{\pgfqpoint{6.612859in}{3.824926in}}%
\pgfpathlineto{\pgfqpoint{6.613735in}{3.730720in}}%
\pgfpathlineto{\pgfqpoint{6.614610in}{3.879977in}}%
\pgfpathlineto{\pgfqpoint{6.615485in}{4.180265in}}%
\pgfpathlineto{\pgfqpoint{6.616361in}{4.318572in}}%
\pgfpathlineto{\pgfqpoint{6.617236in}{4.214700in}}%
\pgfpathlineto{\pgfqpoint{6.618987in}{3.504664in}}%
\pgfpathlineto{\pgfqpoint{6.619862in}{3.562207in}}%
\pgfpathlineto{\pgfqpoint{6.620738in}{3.578179in}}%
\pgfpathlineto{\pgfqpoint{6.621613in}{3.682919in}}%
\pgfpathlineto{\pgfqpoint{6.623364in}{3.509875in}}%
\pgfpathlineto{\pgfqpoint{6.625990in}{2.903824in}}%
\pgfpathlineto{\pgfqpoint{6.626865in}{2.849113in}}%
\pgfpathlineto{\pgfqpoint{6.627741in}{2.756153in}}%
\pgfpathlineto{\pgfqpoint{6.628616in}{2.591868in}}%
\pgfpathlineto{\pgfqpoint{6.629491in}{2.797120in}}%
\pgfpathlineto{\pgfqpoint{6.630367in}{2.868747in}}%
\pgfpathlineto{\pgfqpoint{6.631242in}{3.061463in}}%
\pgfpathlineto{\pgfqpoint{6.632117in}{3.552466in}}%
\pgfpathlineto{\pgfqpoint{6.632993in}{3.747372in}}%
\pgfpathlineto{\pgfqpoint{6.633868in}{3.876088in}}%
\pgfpathlineto{\pgfqpoint{6.634744in}{3.768252in}}%
\pgfpathlineto{\pgfqpoint{6.635619in}{3.768138in}}%
\pgfpathlineto{\pgfqpoint{6.636494in}{3.939181in}}%
\pgfpathlineto{\pgfqpoint{6.637370in}{3.955040in}}%
\pgfpathlineto{\pgfqpoint{6.638245in}{3.835121in}}%
\pgfpathlineto{\pgfqpoint{6.639120in}{3.783544in}}%
\pgfpathlineto{\pgfqpoint{6.639996in}{3.600229in}}%
\pgfpathlineto{\pgfqpoint{6.640871in}{3.698135in}}%
\pgfpathlineto{\pgfqpoint{6.641747in}{3.610915in}}%
\pgfpathlineto{\pgfqpoint{6.642622in}{3.559337in}}%
\pgfpathlineto{\pgfqpoint{6.643497in}{3.683334in}}%
\pgfpathlineto{\pgfqpoint{6.644373in}{3.593810in}}%
\pgfpathlineto{\pgfqpoint{6.645248in}{3.390673in}}%
\pgfpathlineto{\pgfqpoint{6.646123in}{3.084797in}}%
\pgfpathlineto{\pgfqpoint{6.646999in}{2.895064in}}%
\pgfpathlineto{\pgfqpoint{6.648750in}{2.816037in}}%
\pgfpathlineto{\pgfqpoint{6.649625in}{2.739653in}}%
\pgfpathlineto{\pgfqpoint{6.653126in}{3.537249in}}%
\pgfpathlineto{\pgfqpoint{6.654002in}{3.562169in}}%
\pgfpathlineto{\pgfqpoint{6.654877in}{3.476006in}}%
\pgfpathlineto{\pgfqpoint{6.655753in}{3.586485in}}%
\pgfpathlineto{\pgfqpoint{6.656628in}{3.642442in}}%
\pgfpathlineto{\pgfqpoint{6.657503in}{3.786677in}}%
\pgfpathlineto{\pgfqpoint{6.658379in}{3.837311in}}%
\pgfpathlineto{\pgfqpoint{6.659254in}{3.837348in}}%
\pgfpathlineto{\pgfqpoint{6.660129in}{3.902896in}}%
\pgfpathlineto{\pgfqpoint{6.661005in}{3.635042in}}%
\pgfpathlineto{\pgfqpoint{6.662756in}{3.388521in}}%
\pgfpathlineto{\pgfqpoint{6.663631in}{3.441722in}}%
\pgfpathlineto{\pgfqpoint{6.664506in}{3.441307in}}%
\pgfpathlineto{\pgfqpoint{6.666257in}{3.028840in}}%
\pgfpathlineto{\pgfqpoint{6.668008in}{2.576992in}}%
\pgfpathlineto{\pgfqpoint{6.668883in}{2.479916in}}%
\pgfpathlineto{\pgfqpoint{6.669759in}{2.341987in}}%
\pgfpathlineto{\pgfqpoint{6.670634in}{2.328696in}}%
\pgfpathlineto{\pgfqpoint{6.671509in}{2.513143in}}%
\pgfpathlineto{\pgfqpoint{6.672385in}{2.865386in}}%
\pgfpathlineto{\pgfqpoint{6.673260in}{2.801840in}}%
\pgfpathlineto{\pgfqpoint{6.674135in}{3.053534in}}%
\pgfpathlineto{\pgfqpoint{6.675011in}{3.163711in}}%
\pgfpathlineto{\pgfqpoint{6.675886in}{3.230618in}}%
\pgfpathlineto{\pgfqpoint{6.676762in}{3.458071in}}%
\pgfpathlineto{\pgfqpoint{6.677637in}{3.514632in}}%
\pgfpathlineto{\pgfqpoint{6.678512in}{3.682617in}}%
\pgfpathlineto{\pgfqpoint{6.680263in}{3.759983in}}%
\pgfpathlineto{\pgfqpoint{6.681138in}{3.724755in}}%
\pgfpathlineto{\pgfqpoint{6.682014in}{3.502852in}}%
\pgfpathlineto{\pgfqpoint{6.682889in}{3.395997in}}%
\pgfpathlineto{\pgfqpoint{6.683765in}{3.195427in}}%
\pgfpathlineto{\pgfqpoint{6.684640in}{3.121686in}}%
\pgfpathlineto{\pgfqpoint{6.685515in}{2.999993in}}%
\pgfpathlineto{\pgfqpoint{6.688141in}{2.153273in}}%
\pgfpathlineto{\pgfqpoint{6.689892in}{1.770522in}}%
\pgfpathlineto{\pgfqpoint{6.690768in}{1.703993in}}%
\pgfpathlineto{\pgfqpoint{6.691643in}{1.713772in}}%
\pgfpathlineto{\pgfqpoint{6.692518in}{1.789401in}}%
\pgfpathlineto{\pgfqpoint{6.693394in}{2.011040in}}%
\pgfpathlineto{\pgfqpoint{6.694269in}{2.005527in}}%
\pgfpathlineto{\pgfqpoint{6.696020in}{2.687320in}}%
\pgfpathlineto{\pgfqpoint{6.698646in}{3.054364in}}%
\pgfpathlineto{\pgfqpoint{6.699521in}{3.233261in}}%
\pgfpathlineto{\pgfqpoint{6.700397in}{3.325390in}}%
\pgfpathlineto{\pgfqpoint{6.701272in}{3.353897in}}%
\pgfpathlineto{\pgfqpoint{6.702147in}{3.275814in}}%
\pgfpathlineto{\pgfqpoint{6.703023in}{3.130635in}}%
\pgfpathlineto{\pgfqpoint{6.703898in}{3.141132in}}%
\pgfpathlineto{\pgfqpoint{6.704774in}{2.934521in}}%
\pgfpathlineto{\pgfqpoint{6.705649in}{2.859836in}}%
\pgfpathlineto{\pgfqpoint{6.706524in}{2.886455in}}%
\pgfpathlineto{\pgfqpoint{6.707400in}{2.687207in}}%
\pgfpathlineto{\pgfqpoint{6.709150in}{2.186198in}}%
\pgfpathlineto{\pgfqpoint{6.710901in}{1.915890in}}%
\pgfpathlineto{\pgfqpoint{6.712652in}{1.821571in}}%
\pgfpathlineto{\pgfqpoint{6.713527in}{1.941830in}}%
\pgfpathlineto{\pgfqpoint{6.717904in}{3.283970in}}%
\pgfpathlineto{\pgfqpoint{6.718780in}{3.469021in}}%
\pgfpathlineto{\pgfqpoint{6.719655in}{3.545556in}}%
\pgfpathlineto{\pgfqpoint{6.720530in}{3.447574in}}%
\pgfpathlineto{\pgfqpoint{6.721406in}{3.521806in}}%
\pgfpathlineto{\pgfqpoint{6.723157in}{3.309683in}}%
\pgfpathlineto{\pgfqpoint{6.724032in}{2.996632in}}%
\pgfpathlineto{\pgfqpoint{6.725783in}{2.851189in}}%
\pgfpathlineto{\pgfqpoint{6.726658in}{2.788322in}}%
\pgfpathlineto{\pgfqpoint{6.727533in}{2.605575in}}%
\pgfpathlineto{\pgfqpoint{6.728409in}{2.490526in}}%
\pgfpathlineto{\pgfqpoint{6.730160in}{1.998504in}}%
\pgfpathlineto{\pgfqpoint{6.731035in}{1.855779in}}%
\pgfpathlineto{\pgfqpoint{6.732786in}{1.763839in}}%
\pgfpathlineto{\pgfqpoint{6.733661in}{1.770598in}}%
\pgfpathlineto{\pgfqpoint{6.734536in}{1.849361in}}%
\pgfpathlineto{\pgfqpoint{6.736287in}{2.224787in}}%
\pgfpathlineto{\pgfqpoint{6.737163in}{2.675200in}}%
\pgfpathlineto{\pgfqpoint{6.738038in}{2.969145in}}%
\pgfpathlineto{\pgfqpoint{6.739789in}{3.040545in}}%
\pgfpathlineto{\pgfqpoint{6.740664in}{3.043528in}}%
\pgfpathlineto{\pgfqpoint{6.741539in}{3.195465in}}%
\pgfpathlineto{\pgfqpoint{6.742415in}{3.423447in}}%
\pgfpathlineto{\pgfqpoint{6.743290in}{3.479971in}}%
\pgfpathlineto{\pgfqpoint{6.745041in}{2.947510in}}%
\pgfpathlineto{\pgfqpoint{6.746792in}{2.722624in}}%
\pgfpathlineto{\pgfqpoint{6.747667in}{2.613806in}}%
\pgfpathlineto{\pgfqpoint{6.748542in}{2.603045in}}%
\pgfpathlineto{\pgfqpoint{6.752044in}{2.004394in}}%
\pgfpathlineto{\pgfqpoint{6.752919in}{1.918948in}}%
\pgfpathlineto{\pgfqpoint{6.753795in}{1.878849in}}%
\pgfpathlineto{\pgfqpoint{6.754670in}{1.866578in}}%
\pgfpathlineto{\pgfqpoint{6.755545in}{1.872695in}}%
\pgfpathlineto{\pgfqpoint{6.757296in}{2.060653in}}%
\pgfpathlineto{\pgfqpoint{6.758172in}{2.099884in}}%
\pgfpathlineto{\pgfqpoint{6.759047in}{2.165205in}}%
\pgfpathlineto{\pgfqpoint{6.759922in}{2.269756in}}%
\pgfpathlineto{\pgfqpoint{6.760798in}{2.415766in}}%
\pgfpathlineto{\pgfqpoint{6.761673in}{2.463152in}}%
\pgfpathlineto{\pgfqpoint{6.762548in}{2.776391in}}%
\pgfpathlineto{\pgfqpoint{6.764299in}{3.015134in}}%
\pgfpathlineto{\pgfqpoint{6.765175in}{2.953664in}}%
\pgfpathlineto{\pgfqpoint{6.766050in}{2.801575in}}%
\pgfpathlineto{\pgfqpoint{6.766925in}{2.778203in}}%
\pgfpathlineto{\pgfqpoint{6.767801in}{2.663118in}}%
\pgfpathlineto{\pgfqpoint{6.768676in}{2.644239in}}%
\pgfpathlineto{\pgfqpoint{6.769551in}{2.655490in}}%
\pgfpathlineto{\pgfqpoint{6.770427in}{2.482144in}}%
\pgfpathlineto{\pgfqpoint{6.771302in}{2.180006in}}%
\pgfpathlineto{\pgfqpoint{6.773053in}{1.904940in}}%
\pgfpathlineto{\pgfqpoint{6.773928in}{1.700821in}}%
\pgfpathlineto{\pgfqpoint{6.774804in}{1.692628in}}%
\pgfpathlineto{\pgfqpoint{6.775679in}{1.631422in}}%
\pgfpathlineto{\pgfqpoint{6.776554in}{1.694327in}}%
\pgfpathlineto{\pgfqpoint{6.777430in}{1.872468in}}%
\pgfpathlineto{\pgfqpoint{6.778305in}{1.956668in}}%
\pgfpathlineto{\pgfqpoint{6.779181in}{2.137906in}}%
\pgfpathlineto{\pgfqpoint{6.780056in}{1.879152in}}%
\pgfpathlineto{\pgfqpoint{6.781807in}{2.341043in}}%
\pgfpathlineto{\pgfqpoint{6.782682in}{2.320767in}}%
\pgfpathlineto{\pgfqpoint{6.783557in}{2.337909in}}%
\pgfpathlineto{\pgfqpoint{6.784433in}{2.463076in}}%
\pgfpathlineto{\pgfqpoint{6.785308in}{2.533457in}}%
\pgfpathlineto{\pgfqpoint{6.786184in}{2.512388in}}%
\pgfpathlineto{\pgfqpoint{6.787059in}{2.426451in}}%
\pgfpathlineto{\pgfqpoint{6.787934in}{2.436608in}}%
\pgfpathlineto{\pgfqpoint{6.788810in}{2.283462in}}%
\pgfpathlineto{\pgfqpoint{6.789685in}{2.360111in}}%
\pgfpathlineto{\pgfqpoint{6.790560in}{2.349576in}}%
\pgfpathlineto{\pgfqpoint{6.791436in}{2.219614in}}%
\pgfpathlineto{\pgfqpoint{6.792311in}{1.918344in}}%
\pgfpathlineto{\pgfqpoint{6.793187in}{1.801521in}}%
\pgfpathlineto{\pgfqpoint{6.794062in}{1.726119in}}%
\pgfpathlineto{\pgfqpoint{6.794937in}{1.691382in}}%
\pgfpathlineto{\pgfqpoint{6.796688in}{1.578335in}}%
\pgfpathlineto{\pgfqpoint{6.797563in}{1.697083in}}%
\pgfpathlineto{\pgfqpoint{6.800190in}{2.371249in}}%
\pgfpathlineto{\pgfqpoint{6.801065in}{2.374836in}}%
\pgfpathlineto{\pgfqpoint{6.801940in}{2.464360in}}%
\pgfpathlineto{\pgfqpoint{6.802816in}{2.450390in}}%
\pgfpathlineto{\pgfqpoint{6.804566in}{2.643483in}}%
\pgfpathlineto{\pgfqpoint{6.805442in}{2.716960in}}%
\pgfpathlineto{\pgfqpoint{6.806317in}{2.919116in}}%
\pgfpathlineto{\pgfqpoint{6.807193in}{2.870597in}}%
\pgfpathlineto{\pgfqpoint{6.808943in}{2.477009in}}%
\pgfpathlineto{\pgfqpoint{6.809819in}{2.360451in}}%
\pgfpathlineto{\pgfqpoint{6.810694in}{2.418145in}}%
\pgfpathlineto{\pgfqpoint{6.811569in}{2.452051in}}%
\pgfpathlineto{\pgfqpoint{6.813320in}{1.970412in}}%
\pgfpathlineto{\pgfqpoint{6.815071in}{1.831954in}}%
\pgfpathlineto{\pgfqpoint{6.815946in}{1.716000in}}%
\pgfpathlineto{\pgfqpoint{6.816822in}{1.704522in}}%
\pgfpathlineto{\pgfqpoint{6.817697in}{1.652227in}}%
\pgfpathlineto{\pgfqpoint{6.818572in}{1.881870in}}%
\pgfpathlineto{\pgfqpoint{6.819448in}{1.979701in}}%
\pgfpathlineto{\pgfqpoint{6.820323in}{2.145420in}}%
\pgfpathlineto{\pgfqpoint{6.822074in}{2.194996in}}%
\pgfpathlineto{\pgfqpoint{6.822949in}{2.294299in}}%
\pgfpathlineto{\pgfqpoint{6.823825in}{2.284482in}}%
\pgfpathlineto{\pgfqpoint{6.824700in}{2.347084in}}%
\pgfpathlineto{\pgfqpoint{6.826451in}{2.607538in}}%
\pgfpathlineto{\pgfqpoint{6.827326in}{2.705633in}}%
\pgfpathlineto{\pgfqpoint{6.829077in}{2.504044in}}%
\pgfpathlineto{\pgfqpoint{6.829952in}{2.471157in}}%
\pgfpathlineto{\pgfqpoint{6.830828in}{2.369173in}}%
\pgfpathlineto{\pgfqpoint{6.831703in}{2.331302in}}%
\pgfpathlineto{\pgfqpoint{6.832578in}{2.372760in}}%
\pgfpathlineto{\pgfqpoint{6.833454in}{2.252048in}}%
\pgfpathlineto{\pgfqpoint{6.834329in}{2.033015in}}%
\pgfpathlineto{\pgfqpoint{6.836080in}{1.814586in}}%
\pgfpathlineto{\pgfqpoint{6.836955in}{1.717397in}}%
\pgfpathlineto{\pgfqpoint{6.837831in}{1.680885in}}%
\pgfpathlineto{\pgfqpoint{6.838706in}{1.660420in}}%
\pgfpathlineto{\pgfqpoint{6.839581in}{1.765651in}}%
\pgfpathlineto{\pgfqpoint{6.841332in}{2.272437in}}%
\pgfpathlineto{\pgfqpoint{6.842208in}{2.470099in}}%
\pgfpathlineto{\pgfqpoint{6.843083in}{2.530965in}}%
\pgfpathlineto{\pgfqpoint{6.843958in}{2.567552in}}%
\pgfpathlineto{\pgfqpoint{6.844834in}{2.649411in}}%
\pgfpathlineto{\pgfqpoint{6.845709in}{2.823173in}}%
\pgfpathlineto{\pgfqpoint{6.846584in}{2.830725in}}%
\pgfpathlineto{\pgfqpoint{6.847460in}{3.253046in}}%
\pgfpathlineto{\pgfqpoint{6.848335in}{3.413554in}}%
\pgfpathlineto{\pgfqpoint{6.850086in}{3.127388in}}%
\pgfpathlineto{\pgfqpoint{6.850961in}{2.906920in}}%
\pgfpathlineto{\pgfqpoint{6.852712in}{2.797158in}}%
\pgfpathlineto{\pgfqpoint{6.853587in}{2.776580in}}%
\pgfpathlineto{\pgfqpoint{6.854463in}{2.606028in}}%
\pgfpathlineto{\pgfqpoint{6.856214in}{2.123860in}}%
\pgfpathlineto{\pgfqpoint{6.857964in}{1.946360in}}%
\pgfpathlineto{\pgfqpoint{6.858840in}{1.996692in}}%
\pgfpathlineto{\pgfqpoint{6.859715in}{1.938016in}}%
\pgfpathlineto{\pgfqpoint{6.860591in}{1.991406in}}%
\pgfpathlineto{\pgfqpoint{6.863217in}{2.937164in}}%
\pgfpathlineto{\pgfqpoint{6.864092in}{3.043037in}}%
\pgfpathlineto{\pgfqpoint{6.865843in}{3.478762in}}%
\pgfpathlineto{\pgfqpoint{6.866718in}{3.594868in}}%
\pgfpathlineto{\pgfqpoint{6.867594in}{3.789169in}}%
\pgfpathlineto{\pgfqpoint{6.868469in}{3.833459in}}%
\pgfpathlineto{\pgfqpoint{6.869344in}{3.779805in}}%
\pgfpathlineto{\pgfqpoint{6.870220in}{3.663209in}}%
\pgfpathlineto{\pgfqpoint{6.871095in}{3.374551in}}%
\pgfpathlineto{\pgfqpoint{6.871970in}{3.239491in}}%
\pgfpathlineto{\pgfqpoint{6.872846in}{3.001277in}}%
\pgfpathlineto{\pgfqpoint{6.873721in}{2.919795in}}%
\pgfpathlineto{\pgfqpoint{6.874597in}{2.770539in}}%
\pgfpathlineto{\pgfqpoint{6.877223in}{1.981777in}}%
\pgfpathlineto{\pgfqpoint{6.878973in}{1.698933in}}%
\pgfpathlineto{\pgfqpoint{6.879849in}{1.613789in}}%
\pgfpathlineto{\pgfqpoint{6.880724in}{1.496891in}}%
\pgfpathlineto{\pgfqpoint{6.881600in}{1.679111in}}%
\pgfpathlineto{\pgfqpoint{6.885976in}{3.138300in}}%
\pgfpathlineto{\pgfqpoint{6.886852in}{3.125991in}}%
\pgfpathlineto{\pgfqpoint{6.887727in}{3.191123in}}%
\pgfpathlineto{\pgfqpoint{6.889478in}{3.492770in}}%
\pgfpathlineto{\pgfqpoint{6.890353in}{3.361298in}}%
\pgfpathlineto{\pgfqpoint{6.891229in}{3.294504in}}%
\pgfpathlineto{\pgfqpoint{6.893855in}{2.736783in}}%
\pgfpathlineto{\pgfqpoint{6.894730in}{2.743391in}}%
\pgfpathlineto{\pgfqpoint{6.895606in}{2.637480in}}%
\pgfpathlineto{\pgfqpoint{6.896481in}{2.484863in}}%
\pgfpathlineto{\pgfqpoint{6.898232in}{1.967618in}}%
\pgfpathlineto{\pgfqpoint{6.899982in}{1.834371in}}%
\pgfpathlineto{\pgfqpoint{6.900858in}{1.720531in}}%
\pgfpathlineto{\pgfqpoint{6.901733in}{1.694214in}}%
\pgfpathlineto{\pgfqpoint{6.902609in}{1.753531in}}%
\pgfpathlineto{\pgfqpoint{6.904359in}{2.040453in}}%
\pgfpathlineto{\pgfqpoint{6.906110in}{2.495322in}}%
\pgfpathlineto{\pgfqpoint{6.906985in}{2.580956in}}%
\pgfpathlineto{\pgfqpoint{6.907861in}{2.743693in}}%
\pgfpathlineto{\pgfqpoint{6.909612in}{3.164919in}}%
\pgfpathlineto{\pgfqpoint{6.910487in}{3.172433in}}%
\pgfpathlineto{\pgfqpoint{6.911362in}{3.242209in}}%
\pgfpathlineto{\pgfqpoint{6.912238in}{3.130069in}}%
\pgfpathlineto{\pgfqpoint{6.913113in}{3.103072in}}%
\pgfpathlineto{\pgfqpoint{6.913988in}{3.011320in}}%
\pgfpathlineto{\pgfqpoint{6.917490in}{2.262545in}}%
\pgfpathlineto{\pgfqpoint{6.919241in}{1.793781in}}%
\pgfpathlineto{\pgfqpoint{6.920116in}{1.668350in}}%
\pgfpathlineto{\pgfqpoint{6.921867in}{1.612997in}}%
\pgfpathlineto{\pgfqpoint{6.922742in}{1.599706in}}%
\pgfpathlineto{\pgfqpoint{6.923618in}{1.734426in}}%
\pgfpathlineto{\pgfqpoint{6.925368in}{2.328130in}}%
\pgfpathlineto{\pgfqpoint{6.926244in}{2.395679in}}%
\pgfpathlineto{\pgfqpoint{6.927119in}{2.400172in}}%
\pgfpathlineto{\pgfqpoint{6.928870in}{2.831593in}}%
\pgfpathlineto{\pgfqpoint{6.930621in}{3.185648in}}%
\pgfpathlineto{\pgfqpoint{6.931496in}{3.064559in}}%
\pgfpathlineto{\pgfqpoint{6.932371in}{3.115343in}}%
\pgfpathlineto{\pgfqpoint{6.934122in}{2.731421in}}%
\pgfpathlineto{\pgfqpoint{6.934997in}{2.618828in}}%
\pgfpathlineto{\pgfqpoint{6.935873in}{2.631363in}}%
\pgfpathlineto{\pgfqpoint{6.936748in}{2.568647in}}%
\pgfpathlineto{\pgfqpoint{6.938499in}{2.338438in}}%
\pgfpathlineto{\pgfqpoint{6.941125in}{1.736200in}}%
\pgfpathlineto{\pgfqpoint{6.942000in}{1.677185in}}%
\pgfpathlineto{\pgfqpoint{6.942876in}{1.575163in}}%
\pgfpathlineto{\pgfqpoint{6.943751in}{1.532346in}}%
\pgfpathlineto{\pgfqpoint{6.944627in}{1.552357in}}%
\pgfpathlineto{\pgfqpoint{6.945502in}{1.736767in}}%
\pgfpathlineto{\pgfqpoint{6.947253in}{2.158484in}}%
\pgfpathlineto{\pgfqpoint{6.948128in}{2.253030in}}%
\pgfpathlineto{\pgfqpoint{6.949003in}{2.223541in}}%
\pgfpathlineto{\pgfqpoint{6.949879in}{2.280480in}}%
\pgfpathlineto{\pgfqpoint{6.950754in}{2.439214in}}%
\pgfpathlineto{\pgfqpoint{6.951630in}{2.531192in}}%
\pgfpathlineto{\pgfqpoint{6.952505in}{2.567968in}}%
\pgfpathlineto{\pgfqpoint{6.953380in}{2.685206in}}%
\pgfpathlineto{\pgfqpoint{6.954256in}{2.665081in}}%
\pgfpathlineto{\pgfqpoint{6.956006in}{2.601912in}}%
\pgfpathlineto{\pgfqpoint{6.956882in}{2.460962in}}%
\pgfpathlineto{\pgfqpoint{6.957757in}{2.485731in}}%
\pgfpathlineto{\pgfqpoint{6.958633in}{2.418787in}}%
\pgfpathlineto{\pgfqpoint{6.960383in}{2.011644in}}%
\pgfpathlineto{\pgfqpoint{6.961259in}{1.853929in}}%
\pgfpathlineto{\pgfqpoint{6.962134in}{1.751303in}}%
\pgfpathlineto{\pgfqpoint{6.963009in}{1.681905in}}%
\pgfpathlineto{\pgfqpoint{6.963885in}{1.644109in}}%
\pgfpathlineto{\pgfqpoint{6.964760in}{1.642712in}}%
\pgfpathlineto{\pgfqpoint{6.965636in}{1.922686in}}%
\pgfpathlineto{\pgfqpoint{6.966511in}{2.028786in}}%
\pgfpathlineto{\pgfqpoint{6.967386in}{2.238266in}}%
\pgfpathlineto{\pgfqpoint{6.969137in}{2.426149in}}%
\pgfpathlineto{\pgfqpoint{6.970888in}{2.804370in}}%
\pgfpathlineto{\pgfqpoint{6.971763in}{2.679655in}}%
\pgfpathlineto{\pgfqpoint{6.972639in}{2.675578in}}%
\pgfpathlineto{\pgfqpoint{6.973514in}{2.773710in}}%
\pgfpathlineto{\pgfqpoint{6.974389in}{2.831895in}}%
\pgfpathlineto{\pgfqpoint{6.975265in}{2.787605in}}%
\pgfpathlineto{\pgfqpoint{6.976140in}{2.763629in}}%
\pgfpathlineto{\pgfqpoint{6.977015in}{2.754567in}}%
\pgfpathlineto{\pgfqpoint{6.977891in}{2.617053in}}%
\pgfpathlineto{\pgfqpoint{6.978766in}{2.631401in}}%
\pgfpathlineto{\pgfqpoint{6.979642in}{2.713222in}}%
\pgfpathlineto{\pgfqpoint{6.981392in}{2.352597in}}%
\pgfpathlineto{\pgfqpoint{6.983143in}{1.969997in}}%
\pgfpathlineto{\pgfqpoint{6.984018in}{1.903996in}}%
\pgfpathlineto{\pgfqpoint{6.985769in}{1.694667in}}%
\pgfpathlineto{\pgfqpoint{6.988395in}{1.916419in}}%
\pgfpathlineto{\pgfqpoint{6.989271in}{2.136094in}}%
\pgfpathlineto{\pgfqpoint{6.990146in}{2.088254in}}%
\pgfpathlineto{\pgfqpoint{6.991022in}{2.192466in}}%
\pgfpathlineto{\pgfqpoint{6.991897in}{2.353390in}}%
\pgfpathlineto{\pgfqpoint{6.992772in}{2.380047in}}%
\pgfpathlineto{\pgfqpoint{6.993648in}{2.446614in}}%
\pgfpathlineto{\pgfqpoint{6.994523in}{2.769708in}}%
\pgfpathlineto{\pgfqpoint{6.996274in}{2.883208in}}%
\pgfpathlineto{\pgfqpoint{6.997149in}{2.736141in}}%
\pgfpathlineto{\pgfqpoint{6.998025in}{2.712052in}}%
\pgfpathlineto{\pgfqpoint{6.999775in}{2.296300in}}%
\pgfpathlineto{\pgfqpoint{7.000651in}{2.248876in}}%
\pgfpathlineto{\pgfqpoint{7.001526in}{2.118159in}}%
\pgfpathlineto{\pgfqpoint{7.002401in}{1.786456in}}%
\pgfpathlineto{\pgfqpoint{7.004152in}{1.497155in}}%
\pgfpathlineto{\pgfqpoint{7.005028in}{1.527135in}}%
\pgfpathlineto{\pgfqpoint{7.006778in}{1.514524in}}%
\pgfpathlineto{\pgfqpoint{7.007654in}{1.595099in}}%
\pgfpathlineto{\pgfqpoint{7.008529in}{1.800389in}}%
\pgfpathlineto{\pgfqpoint{7.009404in}{2.156558in}}%
\pgfpathlineto{\pgfqpoint{7.010280in}{2.169245in}}%
\pgfpathlineto{\pgfqpoint{7.012031in}{2.479086in}}%
\pgfpathlineto{\pgfqpoint{7.013781in}{3.092160in}}%
\pgfpathlineto{\pgfqpoint{7.015532in}{3.332639in}}%
\pgfpathlineto{\pgfqpoint{7.016407in}{3.324068in}}%
\pgfpathlineto{\pgfqpoint{7.020784in}{2.428339in}}%
\pgfpathlineto{\pgfqpoint{7.021660in}{2.426716in}}%
\pgfpathlineto{\pgfqpoint{7.024286in}{1.839090in}}%
\pgfpathlineto{\pgfqpoint{7.025161in}{1.749529in}}%
\pgfpathlineto{\pgfqpoint{7.026037in}{1.743978in}}%
\pgfpathlineto{\pgfqpoint{7.026912in}{1.631385in}}%
\pgfpathlineto{\pgfqpoint{7.027787in}{1.667406in}}%
\pgfpathlineto{\pgfqpoint{7.028663in}{1.796311in}}%
\pgfpathlineto{\pgfqpoint{7.030413in}{2.382992in}}%
\pgfpathlineto{\pgfqpoint{7.031289in}{2.568043in}}%
\pgfpathlineto{\pgfqpoint{7.032164in}{2.651337in}}%
\pgfpathlineto{\pgfqpoint{7.033040in}{2.604102in}}%
\pgfpathlineto{\pgfqpoint{7.035666in}{2.989496in}}%
\pgfpathlineto{\pgfqpoint{7.036541in}{3.079096in}}%
\pgfpathlineto{\pgfqpoint{7.037416in}{3.210870in}}%
\pgfpathlineto{\pgfqpoint{7.039167in}{2.950455in}}%
\pgfpathlineto{\pgfqpoint{7.040043in}{2.880980in}}%
\pgfpathlineto{\pgfqpoint{7.040918in}{2.738407in}}%
\pgfpathlineto{\pgfqpoint{7.043544in}{2.610143in}}%
\pgfpathlineto{\pgfqpoint{7.045295in}{2.250991in}}%
\pgfpathlineto{\pgfqpoint{7.046170in}{2.143117in}}%
\pgfpathlineto{\pgfqpoint{7.047046in}{2.131223in}}%
\pgfpathlineto{\pgfqpoint{7.047921in}{1.938809in}}%
\pgfpathlineto{\pgfqpoint{7.048796in}{1.915852in}}%
\pgfpathlineto{\pgfqpoint{7.049672in}{2.055934in}}%
\pgfpathlineto{\pgfqpoint{7.052298in}{2.665081in}}%
\pgfpathlineto{\pgfqpoint{7.053173in}{2.555923in}}%
\pgfpathlineto{\pgfqpoint{7.054049in}{2.670065in}}%
\pgfpathlineto{\pgfqpoint{7.054924in}{2.734669in}}%
\pgfpathlineto{\pgfqpoint{7.055799in}{2.759249in}}%
\pgfpathlineto{\pgfqpoint{7.056675in}{2.747922in}}%
\pgfpathlineto{\pgfqpoint{7.057550in}{2.908166in}}%
\pgfpathlineto{\pgfqpoint{7.058425in}{2.834802in}}%
\pgfpathlineto{\pgfqpoint{7.060176in}{2.665836in}}%
\pgfpathlineto{\pgfqpoint{7.061052in}{2.641444in}}%
\pgfpathlineto{\pgfqpoint{7.061927in}{2.481049in}}%
\pgfpathlineto{\pgfqpoint{7.062802in}{2.517825in}}%
\pgfpathlineto{\pgfqpoint{7.063678in}{2.580541in}}%
\pgfpathlineto{\pgfqpoint{7.066304in}{2.001487in}}%
\pgfpathlineto{\pgfqpoint{7.068930in}{1.770220in}}%
\pgfpathlineto{\pgfqpoint{7.069805in}{1.734086in}}%
\pgfpathlineto{\pgfqpoint{7.070681in}{1.850833in}}%
\pgfpathlineto{\pgfqpoint{7.071556in}{2.090482in}}%
\pgfpathlineto{\pgfqpoint{7.072431in}{2.231168in}}%
\pgfpathlineto{\pgfqpoint{7.073307in}{2.205757in}}%
\pgfpathlineto{\pgfqpoint{7.074182in}{2.103018in}}%
\pgfpathlineto{\pgfqpoint{7.075058in}{2.188577in}}%
\pgfpathlineto{\pgfqpoint{7.075933in}{2.349954in}}%
\pgfpathlineto{\pgfqpoint{7.076808in}{2.442989in}}%
\pgfpathlineto{\pgfqpoint{7.077684in}{2.640954in}}%
\pgfpathlineto{\pgfqpoint{7.078559in}{2.703443in}}%
\pgfpathlineto{\pgfqpoint{7.079434in}{2.726098in}}%
\pgfpathlineto{\pgfqpoint{7.080310in}{2.851605in}}%
\pgfpathlineto{\pgfqpoint{7.081185in}{2.787983in}}%
\pgfpathlineto{\pgfqpoint{7.082936in}{2.580239in}}%
\pgfpathlineto{\pgfqpoint{7.083811in}{2.582278in}}%
\pgfpathlineto{\pgfqpoint{7.084687in}{2.686301in}}%
\pgfpathlineto{\pgfqpoint{7.086437in}{2.510576in}}%
\pgfpathlineto{\pgfqpoint{7.088188in}{2.417578in}}%
\pgfpathlineto{\pgfqpoint{7.089939in}{2.318426in}}%
\pgfpathlineto{\pgfqpoint{7.090814in}{2.333492in}}%
\pgfpathlineto{\pgfqpoint{7.091690in}{2.436873in}}%
\pgfpathlineto{\pgfqpoint{7.093440in}{2.768462in}}%
\pgfpathlineto{\pgfqpoint{7.094316in}{2.994896in}}%
\pgfpathlineto{\pgfqpoint{7.095191in}{2.944451in}}%
\pgfpathlineto{\pgfqpoint{7.096067in}{3.066522in}}%
\pgfpathlineto{\pgfqpoint{7.096942in}{3.138791in}}%
\pgfpathlineto{\pgfqpoint{7.097817in}{3.270037in}}%
\pgfpathlineto{\pgfqpoint{7.098693in}{3.251309in}}%
\pgfpathlineto{\pgfqpoint{7.099568in}{3.338228in}}%
\pgfpathlineto{\pgfqpoint{7.100443in}{3.265808in}}%
\pgfpathlineto{\pgfqpoint{7.101319in}{3.113908in}}%
\pgfpathlineto{\pgfqpoint{7.103070in}{2.912508in}}%
\pgfpathlineto{\pgfqpoint{7.103945in}{2.925421in}}%
\pgfpathlineto{\pgfqpoint{7.104820in}{2.909450in}}%
\pgfpathlineto{\pgfqpoint{7.106571in}{2.740672in}}%
\pgfpathlineto{\pgfqpoint{7.107446in}{2.549580in}}%
\pgfpathlineto{\pgfqpoint{7.109197in}{2.370834in}}%
\pgfpathlineto{\pgfqpoint{7.110073in}{2.338174in}}%
\pgfpathlineto{\pgfqpoint{7.110948in}{2.231092in}}%
\pgfpathlineto{\pgfqpoint{7.111823in}{2.221766in}}%
\pgfpathlineto{\pgfqpoint{7.112699in}{2.254691in}}%
\pgfpathlineto{\pgfqpoint{7.113574in}{2.347122in}}%
\pgfpathlineto{\pgfqpoint{7.114449in}{2.392658in}}%
\pgfpathlineto{\pgfqpoint{7.115325in}{2.556565in}}%
\pgfpathlineto{\pgfqpoint{7.116200in}{2.625737in}}%
\pgfpathlineto{\pgfqpoint{7.117076in}{2.648958in}}%
\pgfpathlineto{\pgfqpoint{7.117951in}{2.752679in}}%
\pgfpathlineto{\pgfqpoint{7.118826in}{2.705935in}}%
\pgfpathlineto{\pgfqpoint{7.119702in}{2.768386in}}%
\pgfpathlineto{\pgfqpoint{7.120577in}{2.797498in}}%
\pgfpathlineto{\pgfqpoint{7.121453in}{2.882981in}}%
\pgfpathlineto{\pgfqpoint{7.122328in}{2.892836in}}%
\pgfpathlineto{\pgfqpoint{7.123203in}{2.699138in}}%
\pgfpathlineto{\pgfqpoint{7.124079in}{2.655264in}}%
\pgfpathlineto{\pgfqpoint{7.125829in}{2.670782in}}%
\pgfpathlineto{\pgfqpoint{7.126705in}{2.858137in}}%
\pgfpathlineto{\pgfqpoint{7.129331in}{2.489922in}}%
\pgfpathlineto{\pgfqpoint{7.130206in}{2.482333in}}%
\pgfpathlineto{\pgfqpoint{7.131082in}{2.429963in}}%
\pgfpathlineto{\pgfqpoint{7.131957in}{2.413651in}}%
\pgfpathlineto{\pgfqpoint{7.132832in}{2.439704in}}%
\pgfpathlineto{\pgfqpoint{7.133708in}{2.312914in}}%
\pgfpathlineto{\pgfqpoint{7.134583in}{2.386843in}}%
\pgfpathlineto{\pgfqpoint{7.137209in}{2.764233in}}%
\pgfpathlineto{\pgfqpoint{7.138960in}{3.053005in}}%
\pgfpathlineto{\pgfqpoint{7.139835in}{3.021553in}}%
\pgfpathlineto{\pgfqpoint{7.140711in}{3.157745in}}%
\pgfpathlineto{\pgfqpoint{7.141586in}{3.228201in}}%
\pgfpathlineto{\pgfqpoint{7.142462in}{3.180400in}}%
\pgfpathlineto{\pgfqpoint{7.143337in}{3.224010in}}%
\pgfpathlineto{\pgfqpoint{7.145088in}{2.854663in}}%
\pgfpathlineto{\pgfqpoint{7.145963in}{2.812714in}}%
\pgfpathlineto{\pgfqpoint{7.146838in}{2.892761in}}%
\pgfpathlineto{\pgfqpoint{7.147714in}{2.766347in}}%
\pgfpathlineto{\pgfqpoint{7.148589in}{2.721340in}}%
\pgfpathlineto{\pgfqpoint{7.149465in}{2.506951in}}%
\pgfpathlineto{\pgfqpoint{7.150340in}{2.374006in}}%
\pgfpathlineto{\pgfqpoint{7.151215in}{2.311630in}}%
\pgfpathlineto{\pgfqpoint{7.152091in}{2.149800in}}%
\pgfpathlineto{\pgfqpoint{7.152966in}{2.128164in}}%
\pgfpathlineto{\pgfqpoint{7.153841in}{2.206134in}}%
\pgfpathlineto{\pgfqpoint{7.154717in}{2.340628in}}%
\pgfpathlineto{\pgfqpoint{7.155592in}{2.644276in}}%
\pgfpathlineto{\pgfqpoint{7.156468in}{2.789493in}}%
\pgfpathlineto{\pgfqpoint{7.157343in}{3.218762in}}%
\pgfpathlineto{\pgfqpoint{7.158218in}{3.325805in}}%
\pgfpathlineto{\pgfqpoint{7.159094in}{3.620052in}}%
\pgfpathlineto{\pgfqpoint{7.159969in}{3.707197in}}%
\pgfpathlineto{\pgfqpoint{7.160844in}{3.844069in}}%
\pgfpathlineto{\pgfqpoint{7.161720in}{3.923587in}}%
\pgfpathlineto{\pgfqpoint{7.162595in}{4.125818in}}%
\pgfpathlineto{\pgfqpoint{7.163471in}{4.218514in}}%
\pgfpathlineto{\pgfqpoint{7.164346in}{4.063631in}}%
\pgfpathlineto{\pgfqpoint{7.166972in}{3.224879in}}%
\pgfpathlineto{\pgfqpoint{7.167847in}{3.209473in}}%
\pgfpathlineto{\pgfqpoint{7.169598in}{2.864291in}}%
\pgfpathlineto{\pgfqpoint{7.171349in}{2.356524in}}%
\pgfpathlineto{\pgfqpoint{7.172224in}{2.191371in}}%
\pgfpathlineto{\pgfqpoint{7.173100in}{2.205304in}}%
\pgfpathlineto{\pgfqpoint{7.175726in}{2.141795in}}%
\pgfpathlineto{\pgfqpoint{7.176601in}{2.302115in}}%
\pgfpathlineto{\pgfqpoint{7.177477in}{2.516164in}}%
\pgfpathlineto{\pgfqpoint{7.178352in}{2.892043in}}%
\pgfpathlineto{\pgfqpoint{7.179227in}{3.083929in}}%
\pgfpathlineto{\pgfqpoint{7.180978in}{3.287217in}}%
\pgfpathlineto{\pgfqpoint{7.181853in}{3.476837in}}%
\pgfpathlineto{\pgfqpoint{7.182729in}{3.547708in}}%
\pgfpathlineto{\pgfqpoint{7.183604in}{3.575800in}}%
\pgfpathlineto{\pgfqpoint{7.184480in}{3.672196in}}%
\pgfpathlineto{\pgfqpoint{7.186230in}{3.283743in}}%
\pgfpathlineto{\pgfqpoint{7.187106in}{3.056743in}}%
\pgfpathlineto{\pgfqpoint{7.189732in}{2.876449in}}%
\pgfpathlineto{\pgfqpoint{7.192358in}{2.328281in}}%
\pgfpathlineto{\pgfqpoint{7.194109in}{2.161316in}}%
\pgfpathlineto{\pgfqpoint{7.194984in}{2.087726in}}%
\pgfpathlineto{\pgfqpoint{7.195859in}{2.090482in}}%
\pgfpathlineto{\pgfqpoint{7.196735in}{2.168339in}}%
\pgfpathlineto{\pgfqpoint{7.197610in}{2.292600in}}%
\pgfpathlineto{\pgfqpoint{7.198486in}{2.333680in}}%
\pgfpathlineto{\pgfqpoint{7.199361in}{2.414822in}}%
\pgfpathlineto{\pgfqpoint{7.200236in}{2.546295in}}%
\pgfpathlineto{\pgfqpoint{7.201112in}{2.606556in}}%
\pgfpathlineto{\pgfqpoint{7.201987in}{2.638952in}}%
\pgfpathlineto{\pgfqpoint{7.202862in}{2.711523in}}%
\pgfpathlineto{\pgfqpoint{7.203738in}{2.824457in}}%
\pgfpathlineto{\pgfqpoint{7.204613in}{2.758947in}}%
\pgfpathlineto{\pgfqpoint{7.206364in}{2.850057in}}%
\pgfpathlineto{\pgfqpoint{7.207239in}{2.830725in}}%
\pgfpathlineto{\pgfqpoint{7.208115in}{3.102505in}}%
\pgfpathlineto{\pgfqpoint{7.208990in}{2.863800in}}%
\pgfpathlineto{\pgfqpoint{7.209865in}{2.954646in}}%
\pgfpathlineto{\pgfqpoint{7.210741in}{2.950832in}}%
\pgfpathlineto{\pgfqpoint{7.211616in}{2.849830in}}%
\pgfpathlineto{\pgfqpoint{7.212492in}{2.640123in}}%
\pgfpathlineto{\pgfqpoint{7.213367in}{2.552940in}}%
\pgfpathlineto{\pgfqpoint{7.215118in}{2.255408in}}%
\pgfpathlineto{\pgfqpoint{7.215993in}{2.196091in}}%
\pgfpathlineto{\pgfqpoint{7.216868in}{2.168943in}}%
\pgfpathlineto{\pgfqpoint{7.217744in}{2.006886in}}%
\pgfpathlineto{\pgfqpoint{7.218619in}{1.997787in}}%
\pgfpathlineto{\pgfqpoint{7.219495in}{2.187520in}}%
\pgfpathlineto{\pgfqpoint{7.220370in}{2.313065in}}%
\pgfpathlineto{\pgfqpoint{7.221245in}{2.866934in}}%
\pgfpathlineto{\pgfqpoint{7.222121in}{3.027216in}}%
\pgfpathlineto{\pgfqpoint{7.222996in}{3.091858in}}%
\pgfpathlineto{\pgfqpoint{7.223871in}{3.073885in}}%
\pgfpathlineto{\pgfqpoint{7.224747in}{3.202262in}}%
\pgfpathlineto{\pgfqpoint{7.225622in}{3.218384in}}%
\pgfpathlineto{\pgfqpoint{7.226498in}{3.290200in}}%
\pgfpathlineto{\pgfqpoint{7.227373in}{3.226766in}}%
\pgfpathlineto{\pgfqpoint{7.228248in}{3.253197in}}%
\pgfpathlineto{\pgfqpoint{7.229124in}{2.805653in}}%
\pgfpathlineto{\pgfqpoint{7.229999in}{2.651262in}}%
\pgfpathlineto{\pgfqpoint{7.231750in}{2.841976in}}%
\pgfpathlineto{\pgfqpoint{7.232625in}{2.631439in}}%
\pgfpathlineto{\pgfqpoint{7.233501in}{2.544407in}}%
\pgfpathlineto{\pgfqpoint{7.234376in}{2.592019in}}%
\pgfpathlineto{\pgfqpoint{7.235251in}{2.264206in}}%
\pgfpathlineto{\pgfqpoint{7.236127in}{2.208438in}}%
\pgfpathlineto{\pgfqpoint{7.237002in}{2.191107in}}%
\pgfpathlineto{\pgfqpoint{7.237877in}{2.055971in}}%
\pgfpathlineto{\pgfqpoint{7.238753in}{1.875187in}}%
\pgfpathlineto{\pgfqpoint{7.239628in}{2.026596in}}%
\pgfpathlineto{\pgfqpoint{7.240504in}{2.124880in}}%
\pgfpathlineto{\pgfqpoint{7.241379in}{2.398360in}}%
\pgfpathlineto{\pgfqpoint{7.243130in}{2.776580in}}%
\pgfpathlineto{\pgfqpoint{7.244005in}{2.864631in}}%
\pgfpathlineto{\pgfqpoint{7.244880in}{2.746336in}}%
\pgfpathlineto{\pgfqpoint{7.246631in}{2.971675in}}%
\pgfpathlineto{\pgfqpoint{7.247507in}{2.951210in}}%
\pgfpathlineto{\pgfqpoint{7.248382in}{2.890457in}}%
\pgfpathlineto{\pgfqpoint{7.249257in}{2.874486in}}%
\pgfpathlineto{\pgfqpoint{7.250133in}{2.737312in}}%
\pgfpathlineto{\pgfqpoint{7.251008in}{2.548636in}}%
\pgfpathlineto{\pgfqpoint{7.251883in}{2.557924in}}%
\pgfpathlineto{\pgfqpoint{7.253634in}{2.326167in}}%
\pgfpathlineto{\pgfqpoint{7.254510in}{2.046683in}}%
\pgfpathlineto{\pgfqpoint{7.255385in}{2.075266in}}%
\pgfpathlineto{\pgfqpoint{7.256260in}{2.024066in}}%
\pgfpathlineto{\pgfqpoint{7.257136in}{2.010436in}}%
\pgfpathlineto{\pgfqpoint{7.258887in}{1.863973in}}%
\pgfpathlineto{\pgfqpoint{7.259762in}{1.836598in}}%
\pgfpathlineto{\pgfqpoint{7.260637in}{1.894746in}}%
\pgfpathlineto{\pgfqpoint{7.261513in}{1.890479in}}%
\pgfpathlineto{\pgfqpoint{7.262388in}{2.138133in}}%
\pgfpathlineto{\pgfqpoint{7.263263in}{2.210137in}}%
\pgfpathlineto{\pgfqpoint{7.264139in}{2.399379in}}%
\pgfpathlineto{\pgfqpoint{7.265014in}{2.447407in}}%
\pgfpathlineto{\pgfqpoint{7.266765in}{2.664477in}}%
\pgfpathlineto{\pgfqpoint{7.267640in}{2.732554in}}%
\pgfpathlineto{\pgfqpoint{7.268516in}{2.871503in}}%
\pgfpathlineto{\pgfqpoint{7.269391in}{2.828988in}}%
\pgfpathlineto{\pgfqpoint{7.270266in}{2.821247in}}%
\pgfpathlineto{\pgfqpoint{7.272017in}{2.587111in}}%
\pgfpathlineto{\pgfqpoint{7.272893in}{2.499928in}}%
\pgfpathlineto{\pgfqpoint{7.273768in}{2.507442in}}%
\pgfpathlineto{\pgfqpoint{7.274643in}{2.386239in}}%
\pgfpathlineto{\pgfqpoint{7.275519in}{2.340552in}}%
\pgfpathlineto{\pgfqpoint{7.276394in}{2.121859in}}%
\pgfpathlineto{\pgfqpoint{7.277269in}{2.022745in}}%
\pgfpathlineto{\pgfqpoint{7.278145in}{2.007377in}}%
\pgfpathlineto{\pgfqpoint{7.279896in}{1.805750in}}%
\pgfpathlineto{\pgfqpoint{7.280771in}{1.776903in}}%
\pgfpathlineto{\pgfqpoint{7.281646in}{1.868957in}}%
\pgfpathlineto{\pgfqpoint{7.282522in}{1.999184in}}%
\pgfpathlineto{\pgfqpoint{7.283397in}{1.976604in}}%
\pgfpathlineto{\pgfqpoint{7.284272in}{2.316501in}}%
\pgfpathlineto{\pgfqpoint{7.285148in}{2.536629in}}%
\pgfpathlineto{\pgfqpoint{7.286023in}{2.513181in}}%
\pgfpathlineto{\pgfqpoint{7.286899in}{2.568270in}}%
\pgfpathlineto{\pgfqpoint{7.287774in}{2.704122in}}%
\pgfpathlineto{\pgfqpoint{7.288649in}{2.744750in}}%
\pgfpathlineto{\pgfqpoint{7.289525in}{2.739728in}}%
\pgfpathlineto{\pgfqpoint{7.290400in}{2.949284in}}%
\pgfpathlineto{\pgfqpoint{7.291275in}{2.852416in}}%
\pgfpathlineto{\pgfqpoint{7.293026in}{2.461264in}}%
\pgfpathlineto{\pgfqpoint{7.293902in}{3.581917in}}%
\pgfpathlineto{\pgfqpoint{7.294777in}{3.707348in}}%
\pgfpathlineto{\pgfqpoint{7.295652in}{3.614955in}}%
\pgfpathlineto{\pgfqpoint{7.296528in}{3.562849in}}%
\pgfpathlineto{\pgfqpoint{7.299154in}{3.218762in}}%
\pgfpathlineto{\pgfqpoint{7.300029in}{3.271321in}}%
\pgfpathlineto{\pgfqpoint{7.300905in}{3.075395in}}%
\pgfpathlineto{\pgfqpoint{7.301780in}{2.967030in}}%
\pgfpathlineto{\pgfqpoint{7.302655in}{3.266676in}}%
\pgfpathlineto{\pgfqpoint{7.303531in}{3.464452in}}%
\pgfpathlineto{\pgfqpoint{7.304406in}{3.475081in}}%
\pgfpathlineto{\pgfqpoint{7.305281in}{3.936425in}}%
\pgfpathlineto{\pgfqpoint{7.306157in}{2.582542in}}%
\pgfpathlineto{\pgfqpoint{7.307032in}{3.311080in}}%
\pgfpathlineto{\pgfqpoint{7.307908in}{2.890004in}}%
\pgfpathlineto{\pgfqpoint{7.308783in}{2.789191in}}%
\pgfpathlineto{\pgfqpoint{7.309658in}{2.906580in}}%
\pgfpathlineto{\pgfqpoint{7.310534in}{2.892043in}}%
\pgfpathlineto{\pgfqpoint{7.312284in}{2.662362in}}%
\pgfpathlineto{\pgfqpoint{7.314035in}{2.181063in}}%
\pgfpathlineto{\pgfqpoint{7.314911in}{2.227365in}}%
\pgfpathlineto{\pgfqpoint{7.315786in}{2.321582in}}%
\pgfpathlineto{\pgfqpoint{7.316661in}{2.199823in}}%
\pgfpathlineto{\pgfqpoint{7.317537in}{2.167136in}}%
\pgfpathlineto{\pgfqpoint{7.320163in}{2.455676in}}%
\pgfpathlineto{\pgfqpoint{7.321038in}{2.502771in}}%
\pgfpathlineto{\pgfqpoint{7.321914in}{2.613941in}}%
\pgfpathlineto{\pgfqpoint{7.322789in}{2.593222in}}%
\pgfpathlineto{\pgfqpoint{7.323664in}{2.742150in}}%
\pgfpathlineto{\pgfqpoint{7.325415in}{3.219770in}}%
\pgfpathlineto{\pgfqpoint{7.327166in}{3.592602in}}%
\pgfpathlineto{\pgfqpoint{7.328041in}{2.565627in}}%
\pgfpathlineto{\pgfqpoint{7.328917in}{2.641444in}}%
\pgfpathlineto{\pgfqpoint{7.329792in}{2.634535in}}%
\pgfpathlineto{\pgfqpoint{7.331543in}{2.967672in}}%
\pgfpathlineto{\pgfqpoint{7.332418in}{2.927913in}}%
\pgfpathlineto{\pgfqpoint{7.333293in}{2.949586in}}%
\pgfpathlineto{\pgfqpoint{7.335044in}{2.515258in}}%
\pgfpathlineto{\pgfqpoint{7.339421in}{1.884022in}}%
\pgfpathlineto{\pgfqpoint{7.340296in}{1.740694in}}%
\pgfpathlineto{\pgfqpoint{7.341172in}{1.778602in}}%
\pgfpathlineto{\pgfqpoint{7.342047in}{1.598044in}}%
\pgfpathlineto{\pgfqpoint{7.342923in}{1.557115in}}%
\pgfpathlineto{\pgfqpoint{7.344673in}{1.778753in}}%
\pgfpathlineto{\pgfqpoint{7.346424in}{2.125710in}}%
\pgfpathlineto{\pgfqpoint{7.347299in}{2.311932in}}%
\pgfpathlineto{\pgfqpoint{7.348175in}{2.369475in}}%
\pgfpathlineto{\pgfqpoint{7.349926in}{2.704727in}}%
\pgfpathlineto{\pgfqpoint{7.350801in}{2.701744in}}%
\pgfpathlineto{\pgfqpoint{7.351676in}{2.848508in}}%
\pgfpathlineto{\pgfqpoint{7.353427in}{2.989156in}}%
\pgfpathlineto{\pgfqpoint{7.354302in}{2.977527in}}%
\pgfpathlineto{\pgfqpoint{7.355178in}{2.661456in}}%
\pgfpathlineto{\pgfqpoint{7.357804in}{2.278780in}}%
\pgfpathlineto{\pgfqpoint{7.358679in}{2.308307in}}%
\pgfpathlineto{\pgfqpoint{7.360430in}{1.998882in}}%
\pgfpathlineto{\pgfqpoint{7.361305in}{1.780037in}}%
\pgfpathlineto{\pgfqpoint{7.363056in}{1.647016in}}%
\pgfpathlineto{\pgfqpoint{7.363932in}{1.543069in}}%
\pgfpathlineto{\pgfqpoint{7.364807in}{1.591475in}}%
\pgfpathlineto{\pgfqpoint{7.365682in}{1.685718in}}%
\pgfpathlineto{\pgfqpoint{7.367433in}{2.087499in}}%
\pgfpathlineto{\pgfqpoint{7.368308in}{2.216631in}}%
\pgfpathlineto{\pgfqpoint{7.369184in}{2.441706in}}%
\pgfpathlineto{\pgfqpoint{7.370059in}{2.578653in}}%
\pgfpathlineto{\pgfqpoint{7.370935in}{2.572046in}}%
\pgfpathlineto{\pgfqpoint{7.371810in}{2.733838in}}%
\pgfpathlineto{\pgfqpoint{7.372685in}{2.699894in}}%
\pgfpathlineto{\pgfqpoint{7.373561in}{2.832763in}}%
\pgfpathlineto{\pgfqpoint{7.374436in}{2.920513in}}%
\pgfpathlineto{\pgfqpoint{7.375311in}{2.928970in}}%
\pgfpathlineto{\pgfqpoint{7.377062in}{2.503817in}}%
\pgfpathlineto{\pgfqpoint{7.378813in}{2.363547in}}%
\pgfpathlineto{\pgfqpoint{7.379688in}{2.396736in}}%
\pgfpathlineto{\pgfqpoint{7.382314in}{1.908414in}}%
\pgfpathlineto{\pgfqpoint{7.383190in}{1.896294in}}%
\pgfpathlineto{\pgfqpoint{7.384065in}{1.876508in}}%
\pgfpathlineto{\pgfqpoint{7.385816in}{1.809148in}}%
\pgfpathlineto{\pgfqpoint{7.386691in}{1.881493in}}%
\pgfpathlineto{\pgfqpoint{7.387567in}{2.132658in}}%
\pgfpathlineto{\pgfqpoint{7.388442in}{2.200018in}}%
\pgfpathlineto{\pgfqpoint{7.389318in}{2.314952in}}%
\pgfpathlineto{\pgfqpoint{7.390193in}{2.346027in}}%
\pgfpathlineto{\pgfqpoint{7.391068in}{2.459489in}}%
\pgfpathlineto{\pgfqpoint{7.391944in}{2.404174in}}%
\pgfpathlineto{\pgfqpoint{7.392819in}{2.424979in}}%
\pgfpathlineto{\pgfqpoint{7.393694in}{2.526774in}}%
\pgfpathlineto{\pgfqpoint{7.394570in}{2.461793in}}%
\pgfpathlineto{\pgfqpoint{7.396321in}{2.689963in}}%
\pgfpathlineto{\pgfqpoint{7.398071in}{2.427433in}}%
\pgfpathlineto{\pgfqpoint{7.398947in}{2.354145in}}%
\pgfpathlineto{\pgfqpoint{7.400697in}{2.107322in}}%
\pgfpathlineto{\pgfqpoint{7.402448in}{1.738994in}}%
\pgfpathlineto{\pgfqpoint{7.405074in}{1.590229in}}%
\pgfpathlineto{\pgfqpoint{7.405950in}{1.552660in}}%
\pgfpathlineto{\pgfqpoint{7.406825in}{1.488509in}}%
\pgfpathlineto{\pgfqpoint{7.407700in}{1.547185in}}%
\pgfpathlineto{\pgfqpoint{7.409451in}{1.710336in}}%
\pgfpathlineto{\pgfqpoint{7.410327in}{1.689834in}}%
\pgfpathlineto{\pgfqpoint{7.411202in}{1.804202in}}%
\pgfpathlineto{\pgfqpoint{7.412077in}{1.984269in}}%
\pgfpathlineto{\pgfqpoint{7.415579in}{2.923684in}}%
\pgfpathlineto{\pgfqpoint{7.416454in}{3.000333in}}%
\pgfpathlineto{\pgfqpoint{7.417330in}{3.011962in}}%
\pgfpathlineto{\pgfqpoint{7.418205in}{2.869691in}}%
\pgfpathlineto{\pgfqpoint{7.419080in}{2.663420in}}%
\pgfpathlineto{\pgfqpoint{7.419956in}{2.579899in}}%
\pgfpathlineto{\pgfqpoint{7.422582in}{2.135678in}}%
\pgfpathlineto{\pgfqpoint{7.423457in}{1.950703in}}%
\pgfpathlineto{\pgfqpoint{7.424333in}{1.887383in}}%
\pgfpathlineto{\pgfqpoint{7.426083in}{1.691004in}}%
\pgfpathlineto{\pgfqpoint{7.426959in}{1.672578in}}%
\pgfpathlineto{\pgfqpoint{7.427834in}{1.623153in}}%
\pgfpathlineto{\pgfqpoint{7.433086in}{2.642011in}}%
\pgfpathlineto{\pgfqpoint{7.433962in}{2.681808in}}%
\pgfpathlineto{\pgfqpoint{7.435712in}{2.966275in}}%
\pgfpathlineto{\pgfqpoint{7.436588in}{3.144605in}}%
\pgfpathlineto{\pgfqpoint{7.437463in}{3.254254in}}%
\pgfpathlineto{\pgfqpoint{7.438339in}{3.247835in}}%
\pgfpathlineto{\pgfqpoint{7.440089in}{2.769783in}}%
\pgfpathlineto{\pgfqpoint{7.440965in}{2.699101in}}%
\pgfpathlineto{\pgfqpoint{7.441840in}{2.520317in}}%
\pgfpathlineto{\pgfqpoint{7.442715in}{2.455751in}}%
\pgfpathlineto{\pgfqpoint{7.443591in}{2.238417in}}%
\pgfpathlineto{\pgfqpoint{7.444466in}{2.135112in}}%
\pgfpathlineto{\pgfqpoint{7.445342in}{1.977851in}}%
\pgfpathlineto{\pgfqpoint{7.447092in}{1.785701in}}%
\pgfpathlineto{\pgfqpoint{7.447968in}{1.718492in}}%
\pgfpathlineto{\pgfqpoint{7.448843in}{1.717624in}}%
\pgfpathlineto{\pgfqpoint{7.449718in}{1.783738in}}%
\pgfpathlineto{\pgfqpoint{7.453220in}{3.013133in}}%
\pgfpathlineto{\pgfqpoint{7.454095in}{2.974129in}}%
\pgfpathlineto{\pgfqpoint{7.454971in}{3.067655in}}%
\pgfpathlineto{\pgfqpoint{7.455846in}{3.038770in}}%
\pgfpathlineto{\pgfqpoint{7.456721in}{3.223935in}}%
\pgfpathlineto{\pgfqpoint{7.457597in}{3.312968in}}%
\pgfpathlineto{\pgfqpoint{7.458472in}{3.261579in}}%
\pgfpathlineto{\pgfqpoint{7.459348in}{3.164466in}}%
\pgfpathlineto{\pgfqpoint{7.461974in}{2.649751in}}%
\pgfpathlineto{\pgfqpoint{7.462849in}{2.563097in}}%
\pgfpathlineto{\pgfqpoint{7.463724in}{2.436193in}}%
\pgfpathlineto{\pgfqpoint{7.464600in}{2.205757in}}%
\pgfpathlineto{\pgfqpoint{7.465475in}{2.098185in}}%
\pgfpathlineto{\pgfqpoint{7.466351in}{1.950136in}}%
\pgfpathlineto{\pgfqpoint{7.468977in}{1.807638in}}%
\pgfpathlineto{\pgfqpoint{7.469852in}{1.829651in}}%
\pgfpathlineto{\pgfqpoint{7.470727in}{1.943529in}}%
\pgfpathlineto{\pgfqpoint{7.471603in}{2.110834in}}%
\pgfpathlineto{\pgfqpoint{7.472478in}{2.355051in}}%
\pgfpathlineto{\pgfqpoint{7.473354in}{2.503100in}}%
\pgfpathlineto{\pgfqpoint{7.475104in}{2.867425in}}%
\pgfpathlineto{\pgfqpoint{7.475980in}{2.958648in}}%
\pgfpathlineto{\pgfqpoint{7.476855in}{2.988250in}}%
\pgfpathlineto{\pgfqpoint{7.478606in}{3.263014in}}%
\pgfpathlineto{\pgfqpoint{7.479481in}{3.243342in}}%
\pgfpathlineto{\pgfqpoint{7.480357in}{3.284914in}}%
\pgfpathlineto{\pgfqpoint{7.482107in}{2.812752in}}%
\pgfpathlineto{\pgfqpoint{7.482983in}{2.690945in}}%
\pgfpathlineto{\pgfqpoint{7.483858in}{2.618450in}}%
\pgfpathlineto{\pgfqpoint{7.485609in}{2.267038in}}%
\pgfpathlineto{\pgfqpoint{7.488235in}{1.928501in}}%
\pgfpathlineto{\pgfqpoint{7.489110in}{1.871638in}}%
\pgfpathlineto{\pgfqpoint{7.489986in}{1.939451in}}%
\pgfpathlineto{\pgfqpoint{7.490861in}{1.855251in}}%
\pgfpathlineto{\pgfqpoint{7.492612in}{2.229620in}}%
\pgfpathlineto{\pgfqpoint{7.493487in}{2.291278in}}%
\pgfpathlineto{\pgfqpoint{7.495238in}{2.616109in}}%
\pgfpathlineto{\pgfqpoint{7.496113in}{2.732177in}}%
\pgfpathlineto{\pgfqpoint{7.496989in}{2.772464in}}%
\pgfpathlineto{\pgfqpoint{7.497864in}{2.762043in}}%
\pgfpathlineto{\pgfqpoint{7.498739in}{2.803463in}}%
\pgfpathlineto{\pgfqpoint{7.499615in}{2.782130in}}%
\pgfpathlineto{\pgfqpoint{7.500490in}{2.795006in}}%
\pgfpathlineto{\pgfqpoint{7.501366in}{2.731497in}}%
\pgfpathlineto{\pgfqpoint{7.502241in}{2.427660in}}%
\pgfpathlineto{\pgfqpoint{7.503992in}{2.091955in}}%
\pgfpathlineto{\pgfqpoint{7.504867in}{2.050799in}}%
\pgfpathlineto{\pgfqpoint{7.505742in}{1.936883in}}%
\pgfpathlineto{\pgfqpoint{7.507493in}{1.661666in}}%
\pgfpathlineto{\pgfqpoint{7.510119in}{1.391056in}}%
\pgfpathlineto{\pgfqpoint{7.510995in}{1.349447in}}%
\pgfpathlineto{\pgfqpoint{7.511870in}{1.277745in}}%
\pgfpathlineto{\pgfqpoint{7.512745in}{1.405706in}}%
\pgfpathlineto{\pgfqpoint{7.515372in}{1.950891in}}%
\pgfpathlineto{\pgfqpoint{7.516247in}{1.971016in}}%
\pgfpathlineto{\pgfqpoint{7.517998in}{2.453524in}}%
\pgfpathlineto{\pgfqpoint{7.518873in}{2.527718in}}%
\pgfpathlineto{\pgfqpoint{7.519748in}{2.475612in}}%
\pgfpathlineto{\pgfqpoint{7.520624in}{2.726362in}}%
\pgfpathlineto{\pgfqpoint{7.521499in}{2.376649in}}%
\pgfpathlineto{\pgfqpoint{7.522375in}{2.551581in}}%
\pgfpathlineto{\pgfqpoint{7.524125in}{1.896143in}}%
\pgfpathlineto{\pgfqpoint{7.525001in}{1.853967in}}%
\pgfpathlineto{\pgfqpoint{7.526752in}{1.629950in}}%
\pgfpathlineto{\pgfqpoint{7.527627in}{1.563420in}}%
\pgfpathlineto{\pgfqpoint{7.528502in}{1.404045in}}%
\pgfpathlineto{\pgfqpoint{7.529378in}{1.303269in}}%
\pgfpathlineto{\pgfqpoint{7.530253in}{1.284994in}}%
\pgfpathlineto{\pgfqpoint{7.531128in}{1.285712in}}%
\pgfpathlineto{\pgfqpoint{7.532004in}{1.260489in}}%
\pgfpathlineto{\pgfqpoint{7.532879in}{1.263170in}}%
\pgfpathlineto{\pgfqpoint{7.533755in}{1.352279in}}%
\pgfpathlineto{\pgfqpoint{7.536381in}{1.731707in}}%
\pgfpathlineto{\pgfqpoint{7.537256in}{1.746961in}}%
\pgfpathlineto{\pgfqpoint{7.538131in}{1.798539in}}%
\pgfpathlineto{\pgfqpoint{7.539007in}{1.869146in}}%
\pgfpathlineto{\pgfqpoint{7.540758in}{1.910302in}}%
\pgfpathlineto{\pgfqpoint{7.541633in}{1.986195in}}%
\pgfpathlineto{\pgfqpoint{7.542508in}{1.966372in}}%
\pgfpathlineto{\pgfqpoint{7.543384in}{1.870580in}}%
\pgfpathlineto{\pgfqpoint{7.544259in}{1.857479in}}%
\pgfpathlineto{\pgfqpoint{7.546010in}{1.670049in}}%
\pgfpathlineto{\pgfqpoint{7.546885in}{1.692628in}}%
\pgfpathlineto{\pgfqpoint{7.548636in}{1.488660in}}%
\pgfpathlineto{\pgfqpoint{7.549511in}{1.420847in}}%
\pgfpathlineto{\pgfqpoint{7.550387in}{1.383504in}}%
\pgfpathlineto{\pgfqpoint{7.551262in}{1.369761in}}%
\pgfpathlineto{\pgfqpoint{7.552137in}{1.281823in}}%
\pgfpathlineto{\pgfqpoint{7.553013in}{1.311727in}}%
\pgfpathlineto{\pgfqpoint{7.553888in}{1.314936in}}%
\pgfpathlineto{\pgfqpoint{7.554764in}{1.344387in}}%
\pgfpathlineto{\pgfqpoint{7.555639in}{1.407858in}}%
\pgfpathlineto{\pgfqpoint{7.556514in}{1.392793in}}%
\pgfpathlineto{\pgfqpoint{7.557390in}{1.495494in}}%
\pgfpathlineto{\pgfqpoint{7.558265in}{1.468082in}}%
\pgfpathlineto{\pgfqpoint{7.559140in}{1.553037in}}%
\pgfpathlineto{\pgfqpoint{7.560016in}{1.529552in}}%
\pgfpathlineto{\pgfqpoint{7.560891in}{1.555265in}}%
\pgfpathlineto{\pgfqpoint{7.561767in}{1.660269in}}%
\pgfpathlineto{\pgfqpoint{7.562642in}{1.709732in}}%
\pgfpathlineto{\pgfqpoint{7.563517in}{1.706523in}}%
\pgfpathlineto{\pgfqpoint{7.564393in}{1.710940in}}%
\pgfpathlineto{\pgfqpoint{7.565268in}{1.554736in}}%
\pgfpathlineto{\pgfqpoint{7.566143in}{1.530609in}}%
\pgfpathlineto{\pgfqpoint{7.567019in}{1.489075in}}%
\pgfpathlineto{\pgfqpoint{7.567894in}{1.535517in}}%
\pgfpathlineto{\pgfqpoint{7.568770in}{1.448108in}}%
\pgfpathlineto{\pgfqpoint{7.569645in}{1.407896in}}%
\pgfpathlineto{\pgfqpoint{7.570520in}{1.258564in}}%
\pgfpathlineto{\pgfqpoint{7.571396in}{1.223562in}}%
\pgfpathlineto{\pgfqpoint{7.572271in}{1.239458in}}%
\pgfpathlineto{\pgfqpoint{7.573146in}{1.338422in}}%
\pgfpathlineto{\pgfqpoint{7.574022in}{1.378709in}}%
\pgfpathlineto{\pgfqpoint{7.574897in}{1.396908in}}%
\pgfpathlineto{\pgfqpoint{7.575773in}{1.425000in}}%
\pgfpathlineto{\pgfqpoint{7.577523in}{1.772146in}}%
\pgfpathlineto{\pgfqpoint{7.578399in}{1.821042in}}%
\pgfpathlineto{\pgfqpoint{7.579274in}{1.821269in}}%
\pgfpathlineto{\pgfqpoint{7.580149in}{1.826404in}}%
\pgfpathlineto{\pgfqpoint{7.581025in}{1.904072in}}%
\pgfpathlineto{\pgfqpoint{7.581900in}{1.938658in}}%
\pgfpathlineto{\pgfqpoint{7.582776in}{1.985704in}}%
\pgfpathlineto{\pgfqpoint{7.583651in}{2.093012in}}%
\pgfpathlineto{\pgfqpoint{7.584526in}{2.077078in}}%
\pgfpathlineto{\pgfqpoint{7.585402in}{2.109437in}}%
\pgfpathlineto{\pgfqpoint{7.586277in}{1.983363in}}%
\pgfpathlineto{\pgfqpoint{7.587152in}{1.934693in}}%
\pgfpathlineto{\pgfqpoint{7.588028in}{1.843886in}}%
\pgfpathlineto{\pgfqpoint{7.588903in}{1.818777in}}%
\pgfpathlineto{\pgfqpoint{7.589779in}{1.738919in}}%
\pgfpathlineto{\pgfqpoint{7.590654in}{1.685114in}}%
\pgfpathlineto{\pgfqpoint{7.591529in}{1.614092in}}%
\pgfpathlineto{\pgfqpoint{7.592405in}{1.504669in}}%
\pgfpathlineto{\pgfqpoint{7.593280in}{1.580714in}}%
\pgfpathlineto{\pgfqpoint{7.594155in}{1.503914in}}%
\pgfpathlineto{\pgfqpoint{7.595031in}{1.556964in}}%
\pgfpathlineto{\pgfqpoint{7.595906in}{1.504820in}}%
\pgfpathlineto{\pgfqpoint{7.596782in}{1.596157in}}%
\pgfpathlineto{\pgfqpoint{7.598532in}{1.871487in}}%
\pgfpathlineto{\pgfqpoint{7.599408in}{1.929747in}}%
\pgfpathlineto{\pgfqpoint{7.600283in}{1.948701in}}%
\pgfpathlineto{\pgfqpoint{7.601158in}{1.956782in}}%
\pgfpathlineto{\pgfqpoint{7.602034in}{2.075606in}}%
\pgfpathlineto{\pgfqpoint{7.603785in}{1.966863in}}%
\pgfpathlineto{\pgfqpoint{7.604660in}{2.137075in}}%
\pgfpathlineto{\pgfqpoint{7.605535in}{2.205077in}}%
\pgfpathlineto{\pgfqpoint{7.606411in}{2.207569in}}%
\pgfpathlineto{\pgfqpoint{7.607286in}{1.993973in}}%
\pgfpathlineto{\pgfqpoint{7.609912in}{1.783586in}}%
\pgfpathlineto{\pgfqpoint{7.610788in}{1.638861in}}%
\pgfpathlineto{\pgfqpoint{7.611663in}{1.600763in}}%
\pgfpathlineto{\pgfqpoint{7.612538in}{1.525096in}}%
\pgfpathlineto{\pgfqpoint{7.613414in}{1.478579in}}%
\pgfpathlineto{\pgfqpoint{7.614289in}{1.461625in}}%
\pgfpathlineto{\pgfqpoint{7.615164in}{1.429607in}}%
\pgfpathlineto{\pgfqpoint{7.616915in}{1.551716in}}%
\pgfpathlineto{\pgfqpoint{7.617791in}{1.588718in}}%
\pgfpathlineto{\pgfqpoint{7.618666in}{1.753493in}}%
\pgfpathlineto{\pgfqpoint{7.619541in}{1.997334in}}%
\pgfpathlineto{\pgfqpoint{7.620417in}{2.039018in}}%
\pgfpathlineto{\pgfqpoint{7.621292in}{2.050195in}}%
\pgfpathlineto{\pgfqpoint{7.622167in}{2.017119in}}%
\pgfpathlineto{\pgfqpoint{7.623043in}{2.058161in}}%
\pgfpathlineto{\pgfqpoint{7.624794in}{2.344366in}}%
\pgfpathlineto{\pgfqpoint{7.625669in}{2.384351in}}%
\pgfpathlineto{\pgfqpoint{7.626544in}{2.370457in}}%
\pgfpathlineto{\pgfqpoint{7.627420in}{2.315557in}}%
\pgfpathlineto{\pgfqpoint{7.630046in}{1.900258in}}%
\pgfpathlineto{\pgfqpoint{7.630921in}{1.862425in}}%
\pgfpathlineto{\pgfqpoint{7.631797in}{1.873941in}}%
\pgfpathlineto{\pgfqpoint{7.632672in}{1.734312in}}%
\pgfpathlineto{\pgfqpoint{7.633547in}{1.642901in}}%
\pgfpathlineto{\pgfqpoint{7.634423in}{1.610014in}}%
\pgfpathlineto{\pgfqpoint{7.635298in}{1.551149in}}%
\pgfpathlineto{\pgfqpoint{7.636173in}{1.537103in}}%
\pgfpathlineto{\pgfqpoint{7.637049in}{1.514373in}}%
\pgfpathlineto{\pgfqpoint{7.637924in}{1.503839in}}%
\pgfpathlineto{\pgfqpoint{7.638800in}{1.563911in}}%
\pgfpathlineto{\pgfqpoint{7.639675in}{1.686209in}}%
\pgfpathlineto{\pgfqpoint{7.641426in}{1.786456in}}%
\pgfpathlineto{\pgfqpoint{7.642301in}{1.770409in}}%
\pgfpathlineto{\pgfqpoint{7.644052in}{1.917816in}}%
\pgfpathlineto{\pgfqpoint{7.644927in}{2.037546in}}%
\pgfpathlineto{\pgfqpoint{7.645803in}{2.069753in}}%
\pgfpathlineto{\pgfqpoint{7.646678in}{2.056613in}}%
\pgfpathlineto{\pgfqpoint{7.647553in}{2.123596in}}%
\pgfpathlineto{\pgfqpoint{7.648429in}{2.106303in}}%
\pgfpathlineto{\pgfqpoint{7.649304in}{2.001147in}}%
\pgfpathlineto{\pgfqpoint{7.650179in}{1.840714in}}%
\pgfpathlineto{\pgfqpoint{7.651055in}{1.791478in}}%
\pgfpathlineto{\pgfqpoint{7.651930in}{1.666839in}}%
\pgfpathlineto{\pgfqpoint{7.652806in}{1.723854in}}%
\pgfpathlineto{\pgfqpoint{7.653681in}{1.619264in}}%
\pgfpathlineto{\pgfqpoint{7.654556in}{1.566064in}}%
\pgfpathlineto{\pgfqpoint{7.655432in}{1.529929in}}%
\pgfpathlineto{\pgfqpoint{7.656307in}{1.564251in}}%
\pgfpathlineto{\pgfqpoint{7.657183in}{1.486583in}}%
\pgfpathlineto{\pgfqpoint{7.658058in}{1.525172in}}%
\pgfpathlineto{\pgfqpoint{7.658933in}{1.530609in}}%
\pgfpathlineto{\pgfqpoint{7.659809in}{1.628364in}}%
\pgfpathlineto{\pgfqpoint{7.660684in}{1.779848in}}%
\pgfpathlineto{\pgfqpoint{7.661559in}{1.819532in}}%
\pgfpathlineto{\pgfqpoint{7.662435in}{1.885419in}}%
\pgfpathlineto{\pgfqpoint{7.663310in}{1.845094in}}%
\pgfpathlineto{\pgfqpoint{7.664186in}{1.862878in}}%
\pgfpathlineto{\pgfqpoint{7.665061in}{1.947946in}}%
\pgfpathlineto{\pgfqpoint{7.665936in}{1.997900in}}%
\pgfpathlineto{\pgfqpoint{7.666812in}{2.109286in}}%
\pgfpathlineto{\pgfqpoint{7.667687in}{2.153500in}}%
\pgfpathlineto{\pgfqpoint{7.668562in}{2.162411in}}%
\pgfpathlineto{\pgfqpoint{7.669438in}{2.192957in}}%
\pgfpathlineto{\pgfqpoint{7.670313in}{2.143909in}}%
\pgfpathlineto{\pgfqpoint{7.671189in}{2.019988in}}%
\pgfpathlineto{\pgfqpoint{7.672064in}{1.985402in}}%
\pgfpathlineto{\pgfqpoint{7.672939in}{1.878623in}}%
\pgfpathlineto{\pgfqpoint{7.673815in}{1.902184in}}%
\pgfpathlineto{\pgfqpoint{7.674690in}{1.886892in}}%
\pgfpathlineto{\pgfqpoint{7.675565in}{1.780943in}}%
\pgfpathlineto{\pgfqpoint{7.676441in}{1.732840in}}%
\pgfpathlineto{\pgfqpoint{7.677316in}{1.701086in}}%
\pgfpathlineto{\pgfqpoint{7.679067in}{1.624513in}}%
\pgfpathlineto{\pgfqpoint{7.679942in}{1.592116in}}%
\pgfpathlineto{\pgfqpoint{7.680818in}{1.626854in}}%
\pgfpathlineto{\pgfqpoint{7.681693in}{1.686020in}}%
\pgfpathlineto{\pgfqpoint{7.682568in}{1.982608in}}%
\pgfpathlineto{\pgfqpoint{7.683444in}{2.073038in}}%
\pgfpathlineto{\pgfqpoint{7.684319in}{2.077003in}}%
\pgfpathlineto{\pgfqpoint{7.685195in}{2.153727in}}%
\pgfpathlineto{\pgfqpoint{7.686070in}{2.274023in}}%
\pgfpathlineto{\pgfqpoint{7.686945in}{2.256881in}}%
\pgfpathlineto{\pgfqpoint{7.687821in}{2.380727in}}%
\pgfpathlineto{\pgfqpoint{7.688696in}{2.462170in}}%
\pgfpathlineto{\pgfqpoint{7.689571in}{2.445444in}}%
\pgfpathlineto{\pgfqpoint{7.690447in}{2.392809in}}%
\pgfpathlineto{\pgfqpoint{7.692198in}{2.142701in}}%
\pgfpathlineto{\pgfqpoint{7.693073in}{2.064165in}}%
\pgfpathlineto{\pgfqpoint{7.693948in}{2.101319in}}%
\pgfpathlineto{\pgfqpoint{7.694824in}{2.045286in}}%
\pgfpathlineto{\pgfqpoint{7.697450in}{1.760403in}}%
\pgfpathlineto{\pgfqpoint{7.698325in}{1.731330in}}%
\pgfpathlineto{\pgfqpoint{7.699201in}{1.634141in}}%
\pgfpathlineto{\pgfqpoint{7.700076in}{1.610769in}}%
\pgfpathlineto{\pgfqpoint{7.700951in}{1.636029in}}%
\pgfpathlineto{\pgfqpoint{7.701827in}{1.772939in}}%
\pgfpathlineto{\pgfqpoint{7.702702in}{1.856950in}}%
\pgfpathlineto{\pgfqpoint{7.703577in}{1.857176in}}%
\pgfpathlineto{\pgfqpoint{7.705328in}{2.250122in}}%
\pgfpathlineto{\pgfqpoint{7.706204in}{2.420108in}}%
\pgfpathlineto{\pgfqpoint{7.707079in}{2.533155in}}%
\pgfpathlineto{\pgfqpoint{7.707954in}{2.603385in}}%
\pgfpathlineto{\pgfqpoint{7.708830in}{2.615958in}}%
\pgfpathlineto{\pgfqpoint{7.709705in}{2.642426in}}%
\pgfpathlineto{\pgfqpoint{7.710580in}{2.699894in}}%
\pgfpathlineto{\pgfqpoint{7.711456in}{2.723077in}}%
\pgfpathlineto{\pgfqpoint{7.713207in}{2.506536in}}%
\pgfpathlineto{\pgfqpoint{7.714082in}{2.418409in}}%
\pgfpathlineto{\pgfqpoint{7.714957in}{2.254087in}}%
\pgfpathlineto{\pgfqpoint{7.715833in}{2.213950in}}%
\pgfpathlineto{\pgfqpoint{7.719334in}{1.729706in}}%
\pgfpathlineto{\pgfqpoint{7.721085in}{1.652793in}}%
\pgfpathlineto{\pgfqpoint{7.721960in}{1.652302in}}%
\pgfpathlineto{\pgfqpoint{7.723711in}{1.816700in}}%
\pgfpathlineto{\pgfqpoint{7.726337in}{2.411046in}}%
\pgfpathlineto{\pgfqpoint{7.727213in}{2.432757in}}%
\pgfpathlineto{\pgfqpoint{7.728088in}{2.526736in}}%
\pgfpathlineto{\pgfqpoint{7.728963in}{2.476669in}}%
\pgfpathlineto{\pgfqpoint{7.730714in}{2.702008in}}%
\pgfpathlineto{\pgfqpoint{7.731589in}{2.691172in}}%
\pgfpathlineto{\pgfqpoint{7.732465in}{2.649713in}}%
\pgfpathlineto{\pgfqpoint{7.734216in}{2.282594in}}%
\pgfpathlineto{\pgfqpoint{7.735091in}{2.264244in}}%
\pgfpathlineto{\pgfqpoint{7.736842in}{2.182838in}}%
\pgfpathlineto{\pgfqpoint{7.737717in}{2.029843in}}%
\pgfpathlineto{\pgfqpoint{7.741219in}{1.598120in}}%
\pgfpathlineto{\pgfqpoint{7.742094in}{1.574484in}}%
\pgfpathlineto{\pgfqpoint{7.742969in}{1.569084in}}%
\pgfpathlineto{\pgfqpoint{7.743845in}{1.690967in}}%
\pgfpathlineto{\pgfqpoint{7.745595in}{2.176797in}}%
\pgfpathlineto{\pgfqpoint{7.746471in}{2.531229in}}%
\pgfpathlineto{\pgfqpoint{7.747346in}{2.437212in}}%
\pgfpathlineto{\pgfqpoint{7.749097in}{2.788511in}}%
\pgfpathlineto{\pgfqpoint{7.749972in}{2.776391in}}%
\pgfpathlineto{\pgfqpoint{7.751723in}{2.998407in}}%
\pgfpathlineto{\pgfqpoint{7.752598in}{3.044962in}}%
\pgfpathlineto{\pgfqpoint{7.753474in}{3.005052in}}%
\pgfpathlineto{\pgfqpoint{7.754349in}{2.664439in}}%
\pgfpathlineto{\pgfqpoint{7.755225in}{2.537384in}}%
\pgfpathlineto{\pgfqpoint{7.756100in}{2.575104in}}%
\pgfpathlineto{\pgfqpoint{7.756975in}{2.527227in}}%
\pgfpathlineto{\pgfqpoint{7.757851in}{2.504723in}}%
\pgfpathlineto{\pgfqpoint{7.760477in}{1.978304in}}%
\pgfpathlineto{\pgfqpoint{7.762228in}{1.836296in}}%
\pgfpathlineto{\pgfqpoint{7.763978in}{1.744696in}}%
\pgfpathlineto{\pgfqpoint{7.764854in}{1.755985in}}%
\pgfpathlineto{\pgfqpoint{7.766604in}{2.219312in}}%
\pgfpathlineto{\pgfqpoint{7.767480in}{2.445972in}}%
\pgfpathlineto{\pgfqpoint{7.768355in}{2.545842in}}%
\pgfpathlineto{\pgfqpoint{7.769231in}{2.502571in}}%
\pgfpathlineto{\pgfqpoint{7.770106in}{2.663759in}}%
\pgfpathlineto{\pgfqpoint{7.770981in}{2.717489in}}%
\pgfpathlineto{\pgfqpoint{7.771857in}{2.845412in}}%
\pgfpathlineto{\pgfqpoint{7.772732in}{2.927887in}}%
\pgfpathlineto{\pgfqpoint{7.773607in}{2.890067in}}%
\pgfpathlineto{\pgfqpoint{7.774483in}{2.817623in}}%
\pgfpathlineto{\pgfqpoint{7.775358in}{2.608746in}}%
\pgfpathlineto{\pgfqpoint{7.777984in}{2.366001in}}%
\pgfpathlineto{\pgfqpoint{7.778860in}{2.195902in}}%
\pgfpathlineto{\pgfqpoint{7.779735in}{2.100979in}}%
\pgfpathlineto{\pgfqpoint{7.780610in}{1.903166in}}%
\pgfpathlineto{\pgfqpoint{7.781486in}{1.782869in}}%
\pgfpathlineto{\pgfqpoint{7.783237in}{1.676543in}}%
\pgfpathlineto{\pgfqpoint{7.784112in}{1.638974in}}%
\pgfpathlineto{\pgfqpoint{7.784987in}{1.648376in}}%
\pgfpathlineto{\pgfqpoint{7.785863in}{1.710147in}}%
\pgfpathlineto{\pgfqpoint{7.787614in}{2.050976in}}%
\pgfpathlineto{\pgfqpoint{7.788489in}{2.281329in}}%
\pgfpathlineto{\pgfqpoint{7.789364in}{2.374818in}}%
\pgfpathlineto{\pgfqpoint{7.791115in}{2.696311in}}%
\pgfpathlineto{\pgfqpoint{7.791990in}{2.679697in}}%
\pgfpathlineto{\pgfqpoint{7.792866in}{2.807732in}}%
\pgfpathlineto{\pgfqpoint{7.794617in}{2.852264in}}%
\pgfpathlineto{\pgfqpoint{7.795492in}{2.781186in}}%
\pgfpathlineto{\pgfqpoint{7.797243in}{2.354145in}}%
\pgfpathlineto{\pgfqpoint{7.798118in}{2.302039in}}%
\pgfpathlineto{\pgfqpoint{7.798993in}{2.273230in}}%
\pgfpathlineto{\pgfqpoint{7.799869in}{2.346405in}}%
\pgfpathlineto{\pgfqpoint{7.802495in}{1.878245in}}%
\pgfpathlineto{\pgfqpoint{7.803370in}{1.780490in}}%
\pgfpathlineto{\pgfqpoint{7.804246in}{1.644373in}}%
\pgfpathlineto{\pgfqpoint{7.805121in}{1.570519in}}%
\pgfpathlineto{\pgfqpoint{7.805996in}{1.714301in}}%
\pgfpathlineto{\pgfqpoint{7.806872in}{1.687493in}}%
\pgfpathlineto{\pgfqpoint{7.807747in}{1.989555in}}%
\pgfpathlineto{\pgfqpoint{7.808623in}{2.105450in}}%
\pgfpathlineto{\pgfqpoint{7.809498in}{2.389841in}}%
\pgfpathlineto{\pgfqpoint{7.810373in}{2.497482in}}%
\pgfpathlineto{\pgfqpoint{7.811249in}{2.655143in}}%
\pgfpathlineto{\pgfqpoint{7.812124in}{2.697332in}}%
\pgfpathlineto{\pgfqpoint{7.812999in}{2.722559in}}%
\pgfpathlineto{\pgfqpoint{7.813875in}{2.773337in}}%
\pgfpathlineto{\pgfqpoint{7.814750in}{2.704582in}}%
\pgfpathlineto{\pgfqpoint{7.815626in}{2.701564in}}%
\pgfpathlineto{\pgfqpoint{7.816501in}{2.631816in}}%
\pgfpathlineto{\pgfqpoint{7.817376in}{2.601572in}}%
\pgfpathlineto{\pgfqpoint{7.818252in}{2.488563in}}%
\pgfpathlineto{\pgfqpoint{7.819127in}{2.461830in}}%
\pgfpathlineto{\pgfqpoint{7.820878in}{2.363094in}}%
\pgfpathlineto{\pgfqpoint{7.821753in}{2.224107in}}%
\pgfpathlineto{\pgfqpoint{7.822629in}{2.201150in}}%
\pgfpathlineto{\pgfqpoint{7.823504in}{2.205681in}}%
\pgfpathlineto{\pgfqpoint{7.824379in}{2.247895in}}%
\pgfpathlineto{\pgfqpoint{7.825255in}{2.163883in}}%
\pgfpathlineto{\pgfqpoint{7.826130in}{2.129222in}}%
\pgfpathlineto{\pgfqpoint{7.827005in}{2.163695in}}%
\pgfpathlineto{\pgfqpoint{7.827881in}{2.176646in}}%
\pgfpathlineto{\pgfqpoint{7.828756in}{2.254955in}}%
\pgfpathlineto{\pgfqpoint{7.829632in}{2.427303in}}%
\pgfpathlineto{\pgfqpoint{7.830507in}{2.754819in}}%
\pgfpathlineto{\pgfqpoint{7.831382in}{2.779514in}}%
\pgfpathlineto{\pgfqpoint{7.832258in}{2.878119in}}%
\pgfpathlineto{\pgfqpoint{7.833133in}{2.858773in}}%
\pgfpathlineto{\pgfqpoint{7.834008in}{2.979041in}}%
\pgfpathlineto{\pgfqpoint{7.834884in}{2.990364in}}%
\pgfpathlineto{\pgfqpoint{7.835759in}{3.059299in}}%
\pgfpathlineto{\pgfqpoint{7.836635in}{3.013862in}}%
\pgfpathlineto{\pgfqpoint{7.838385in}{2.802104in}}%
\pgfpathlineto{\pgfqpoint{7.839261in}{2.786925in}}%
\pgfpathlineto{\pgfqpoint{7.840136in}{2.689623in}}%
\pgfpathlineto{\pgfqpoint{7.841011in}{2.832348in}}%
\pgfpathlineto{\pgfqpoint{7.842762in}{2.588923in}}%
\pgfpathlineto{\pgfqpoint{7.843638in}{2.339155in}}%
\pgfpathlineto{\pgfqpoint{7.844513in}{2.244987in}}%
\pgfpathlineto{\pgfqpoint{7.845388in}{2.201528in}}%
\pgfpathlineto{\pgfqpoint{7.846264in}{2.184159in}}%
\pgfpathlineto{\pgfqpoint{7.847139in}{2.197450in}}%
\pgfpathlineto{\pgfqpoint{7.848014in}{2.276477in}}%
\pgfpathlineto{\pgfqpoint{7.848890in}{2.268661in}}%
\pgfpathlineto{\pgfqpoint{7.849765in}{2.365397in}}%
\pgfpathlineto{\pgfqpoint{7.851516in}{2.718334in}}%
\pgfpathlineto{\pgfqpoint{7.852391in}{2.878859in}}%
\pgfpathlineto{\pgfqpoint{7.853267in}{2.983421in}}%
\pgfpathlineto{\pgfqpoint{7.854142in}{2.943003in}}%
\pgfpathlineto{\pgfqpoint{7.855017in}{3.013882in}}%
\pgfpathlineto{\pgfqpoint{7.855893in}{2.918754in}}%
\pgfpathlineto{\pgfqpoint{7.856768in}{3.006536in}}%
\pgfpathlineto{\pgfqpoint{7.857644in}{3.170050in}}%
\pgfpathlineto{\pgfqpoint{7.858519in}{3.034692in}}%
\pgfpathlineto{\pgfqpoint{7.859394in}{2.977640in}}%
\pgfpathlineto{\pgfqpoint{7.860270in}{2.940298in}}%
\pgfpathlineto{\pgfqpoint{7.861145in}{2.928140in}}%
\pgfpathlineto{\pgfqpoint{7.862020in}{2.955401in}}%
\pgfpathlineto{\pgfqpoint{7.862896in}{2.783867in}}%
\pgfpathlineto{\pgfqpoint{7.863771in}{2.744825in}}%
\pgfpathlineto{\pgfqpoint{7.864647in}{2.652319in}}%
\pgfpathlineto{\pgfqpoint{7.865522in}{2.487846in}}%
\pgfpathlineto{\pgfqpoint{7.866397in}{2.494982in}}%
\pgfpathlineto{\pgfqpoint{7.867273in}{2.438081in}}%
\pgfpathlineto{\pgfqpoint{7.868148in}{2.343648in}}%
\pgfpathlineto{\pgfqpoint{7.869023in}{2.324807in}}%
\pgfpathlineto{\pgfqpoint{7.869899in}{2.366605in}}%
\pgfpathlineto{\pgfqpoint{7.870774in}{2.582731in}}%
\pgfpathlineto{\pgfqpoint{7.871650in}{2.554667in}}%
\pgfpathlineto{\pgfqpoint{7.872525in}{2.760338in}}%
\pgfpathlineto{\pgfqpoint{7.873400in}{2.657739in}}%
\pgfpathlineto{\pgfqpoint{7.874276in}{2.609776in}}%
\pgfpathlineto{\pgfqpoint{7.875151in}{2.784919in}}%
\pgfpathlineto{\pgfqpoint{7.876026in}{2.897543in}}%
\pgfpathlineto{\pgfqpoint{7.876902in}{2.933175in}}%
\pgfpathlineto{\pgfqpoint{7.877777in}{2.928509in}}%
\pgfpathlineto{\pgfqpoint{7.878653in}{3.112625in}}%
\pgfpathlineto{\pgfqpoint{7.879528in}{3.006638in}}%
\pgfpathlineto{\pgfqpoint{7.880403in}{2.815999in}}%
\pgfpathlineto{\pgfqpoint{7.881279in}{2.690190in}}%
\pgfpathlineto{\pgfqpoint{7.882154in}{2.638613in}}%
\pgfpathlineto{\pgfqpoint{7.883029in}{2.637065in}}%
\pgfpathlineto{\pgfqpoint{7.883905in}{2.546030in}}%
\pgfpathlineto{\pgfqpoint{7.884780in}{2.598401in}}%
\pgfpathlineto{\pgfqpoint{7.885656in}{2.676975in}}%
\pgfpathlineto{\pgfqpoint{7.887406in}{2.556829in}}%
\pgfpathlineto{\pgfqpoint{7.889157in}{2.368304in}}%
\pgfpathlineto{\pgfqpoint{7.890032in}{2.358752in}}%
\pgfpathlineto{\pgfqpoint{7.891783in}{2.442838in}}%
\pgfpathlineto{\pgfqpoint{7.892659in}{2.627285in}}%
\pgfpathlineto{\pgfqpoint{7.893534in}{2.891401in}}%
\pgfpathlineto{\pgfqpoint{7.894409in}{3.038695in}}%
\pgfpathlineto{\pgfqpoint{7.895285in}{3.069203in}}%
\pgfpathlineto{\pgfqpoint{7.896160in}{3.146002in}}%
\pgfpathlineto{\pgfqpoint{7.897035in}{3.074489in}}%
\pgfpathlineto{\pgfqpoint{7.897911in}{3.311759in}}%
\pgfpathlineto{\pgfqpoint{7.898786in}{3.350801in}}%
\pgfpathlineto{\pgfqpoint{7.899662in}{2.934710in}}%
\pgfpathlineto{\pgfqpoint{7.902288in}{2.556791in}}%
\pgfpathlineto{\pgfqpoint{7.903163in}{2.520846in}}%
\pgfpathlineto{\pgfqpoint{7.904038in}{2.887739in}}%
\pgfpathlineto{\pgfqpoint{7.904914in}{2.813696in}}%
\pgfpathlineto{\pgfqpoint{7.905789in}{2.770123in}}%
\pgfpathlineto{\pgfqpoint{7.906665in}{2.599005in}}%
\pgfpathlineto{\pgfqpoint{7.907540in}{2.533948in}}%
\pgfpathlineto{\pgfqpoint{7.908415in}{2.521186in}}%
\pgfpathlineto{\pgfqpoint{7.909291in}{2.236680in}}%
\pgfpathlineto{\pgfqpoint{7.910166in}{2.147157in}}%
\pgfpathlineto{\pgfqpoint{7.911041in}{2.105925in}}%
\pgfpathlineto{\pgfqpoint{7.911917in}{2.188955in}}%
\pgfpathlineto{\pgfqpoint{7.912792in}{2.368040in}}%
\pgfpathlineto{\pgfqpoint{7.913668in}{2.631325in}}%
\pgfpathlineto{\pgfqpoint{7.914543in}{2.501967in}}%
\pgfpathlineto{\pgfqpoint{7.915418in}{2.604631in}}%
\pgfpathlineto{\pgfqpoint{7.916294in}{2.499286in}}%
\pgfpathlineto{\pgfqpoint{7.917169in}{2.572046in}}%
\pgfpathlineto{\pgfqpoint{7.918044in}{2.684639in}}%
\pgfpathlineto{\pgfqpoint{7.918920in}{2.898802in}}%
\pgfpathlineto{\pgfqpoint{7.919795in}{2.879508in}}%
\pgfpathlineto{\pgfqpoint{7.920671in}{2.952569in}}%
\pgfpathlineto{\pgfqpoint{7.921546in}{2.947132in}}%
\pgfpathlineto{\pgfqpoint{7.922421in}{2.706463in}}%
\pgfpathlineto{\pgfqpoint{7.924172in}{2.481993in}}%
\pgfpathlineto{\pgfqpoint{7.925048in}{2.490300in}}%
\pgfpathlineto{\pgfqpoint{7.929424in}{1.959689in}}%
\pgfpathlineto{\pgfqpoint{7.930300in}{1.924990in}}%
\pgfpathlineto{\pgfqpoint{7.931175in}{1.854194in}}%
\pgfpathlineto{\pgfqpoint{7.932051in}{1.826781in}}%
\pgfpathlineto{\pgfqpoint{7.932926in}{1.872053in}}%
\pgfpathlineto{\pgfqpoint{7.933801in}{1.940772in}}%
\pgfpathlineto{\pgfqpoint{7.935552in}{2.329112in}}%
\pgfpathlineto{\pgfqpoint{7.937303in}{2.376838in}}%
\pgfpathlineto{\pgfqpoint{7.938178in}{2.410971in}}%
\pgfpathlineto{\pgfqpoint{7.941680in}{2.834387in}}%
\pgfpathlineto{\pgfqpoint{7.942555in}{2.817925in}}%
\pgfpathlineto{\pgfqpoint{7.943430in}{2.655113in}}%
\pgfpathlineto{\pgfqpoint{7.944306in}{2.553695in}}%
\pgfpathlineto{\pgfqpoint{7.945181in}{2.527416in}}%
\pgfpathlineto{\pgfqpoint{7.946057in}{2.543614in}}%
\pgfpathlineto{\pgfqpoint{7.946932in}{2.345725in}}%
\pgfpathlineto{\pgfqpoint{7.950433in}{1.861179in}}%
\pgfpathlineto{\pgfqpoint{7.951309in}{1.786834in}}%
\pgfpathlineto{\pgfqpoint{7.952184in}{1.779093in}}%
\pgfpathlineto{\pgfqpoint{7.953060in}{1.751530in}}%
\pgfpathlineto{\pgfqpoint{7.953935in}{1.771844in}}%
\pgfpathlineto{\pgfqpoint{7.957436in}{2.376536in}}%
\pgfpathlineto{\pgfqpoint{7.958312in}{2.430831in}}%
\pgfpathlineto{\pgfqpoint{7.959187in}{2.511180in}}%
\pgfpathlineto{\pgfqpoint{7.960063in}{2.617959in}}%
\pgfpathlineto{\pgfqpoint{7.960938in}{2.668819in}}%
\pgfpathlineto{\pgfqpoint{7.962689in}{2.811619in}}%
\pgfpathlineto{\pgfqpoint{7.963564in}{2.876034in}}%
\pgfpathlineto{\pgfqpoint{7.964439in}{2.698799in}}%
\pgfpathlineto{\pgfqpoint{7.965315in}{2.646995in}}%
\pgfpathlineto{\pgfqpoint{7.966190in}{2.523678in}}%
\pgfpathlineto{\pgfqpoint{7.967066in}{2.471308in}}%
\pgfpathlineto{\pgfqpoint{7.968816in}{2.275458in}}%
\pgfpathlineto{\pgfqpoint{7.969692in}{2.117932in}}%
\pgfpathlineto{\pgfqpoint{7.970567in}{2.056991in}}%
\pgfpathlineto{\pgfqpoint{7.972318in}{1.851551in}}%
\pgfpathlineto{\pgfqpoint{7.973193in}{1.817191in}}%
\pgfpathlineto{\pgfqpoint{7.974069in}{1.837693in}}%
\pgfpathlineto{\pgfqpoint{7.974944in}{1.888138in}}%
\pgfpathlineto{\pgfqpoint{7.975819in}{2.000014in}}%
\pgfpathlineto{\pgfqpoint{7.977570in}{2.381369in}}%
\pgfpathlineto{\pgfqpoint{7.978445in}{2.513030in}}%
\pgfpathlineto{\pgfqpoint{7.979321in}{2.572763in}}%
\pgfpathlineto{\pgfqpoint{7.980196in}{2.588969in}}%
\pgfpathlineto{\pgfqpoint{7.981072in}{2.633486in}}%
\pgfpathlineto{\pgfqpoint{7.981947in}{2.803400in}}%
\pgfpathlineto{\pgfqpoint{7.982822in}{2.770236in}}%
\pgfpathlineto{\pgfqpoint{7.983698in}{2.819402in}}%
\pgfpathlineto{\pgfqpoint{7.984573in}{2.808938in}}%
\pgfpathlineto{\pgfqpoint{7.985448in}{2.627587in}}%
\pgfpathlineto{\pgfqpoint{7.986324in}{2.574424in}}%
\pgfpathlineto{\pgfqpoint{7.987199in}{2.474857in}}%
\pgfpathlineto{\pgfqpoint{7.988075in}{2.427773in}}%
\pgfpathlineto{\pgfqpoint{7.989825in}{2.208664in}}%
\pgfpathlineto{\pgfqpoint{7.990701in}{1.972678in}}%
\pgfpathlineto{\pgfqpoint{7.993327in}{1.735370in}}%
\pgfpathlineto{\pgfqpoint{7.994202in}{1.897577in}}%
\pgfpathlineto{\pgfqpoint{7.995078in}{1.789401in}}%
\pgfpathlineto{\pgfqpoint{7.996828in}{2.035847in}}%
\pgfpathlineto{\pgfqpoint{7.997704in}{2.092386in}}%
\pgfpathlineto{\pgfqpoint{7.998579in}{2.504931in}}%
\pgfpathlineto{\pgfqpoint{7.999454in}{2.660574in}}%
\pgfpathlineto{\pgfqpoint{8.001205in}{2.820629in}}%
\pgfpathlineto{\pgfqpoint{8.002081in}{2.952222in}}%
\pgfpathlineto{\pgfqpoint{8.002956in}{2.637196in}}%
\pgfpathlineto{\pgfqpoint{8.003831in}{2.682501in}}%
\pgfpathlineto{\pgfqpoint{8.004707in}{2.637776in}}%
\pgfpathlineto{\pgfqpoint{8.005582in}{2.709748in}}%
\pgfpathlineto{\pgfqpoint{8.006457in}{2.473762in}}%
\pgfpathlineto{\pgfqpoint{8.007333in}{2.343271in}}%
\pgfpathlineto{\pgfqpoint{8.008208in}{2.399870in}}%
\pgfpathlineto{\pgfqpoint{8.009084in}{2.346820in}}%
\pgfpathlineto{\pgfqpoint{8.013460in}{1.644600in}}%
\pgfpathlineto{\pgfqpoint{8.014336in}{1.610316in}}%
\pgfpathlineto{\pgfqpoint{8.016087in}{1.510635in}}%
\pgfpathlineto{\pgfqpoint{8.016962in}{1.480315in}}%
\pgfpathlineto{\pgfqpoint{8.017837in}{1.674164in}}%
\pgfpathlineto{\pgfqpoint{8.019588in}{2.200667in}}%
\pgfpathlineto{\pgfqpoint{8.020463in}{2.267054in}}%
\pgfpathlineto{\pgfqpoint{8.021339in}{2.410131in}}%
\pgfpathlineto{\pgfqpoint{8.022214in}{2.348068in}}%
\pgfpathlineto{\pgfqpoint{8.023090in}{2.431649in}}%
\pgfpathlineto{\pgfqpoint{8.023965in}{2.635960in}}%
\pgfpathlineto{\pgfqpoint{8.024840in}{2.647491in}}%
\pgfpathlineto{\pgfqpoint{8.025716in}{2.605197in}}%
\pgfpathlineto{\pgfqpoint{8.026591in}{2.593265in}}%
\pgfpathlineto{\pgfqpoint{8.027466in}{2.596022in}}%
\pgfpathlineto{\pgfqpoint{8.029217in}{2.364529in}}%
\pgfpathlineto{\pgfqpoint{8.030093in}{2.272852in}}%
\pgfpathlineto{\pgfqpoint{8.030968in}{2.237624in}}%
\pgfpathlineto{\pgfqpoint{8.033594in}{1.815076in}}%
\pgfpathlineto{\pgfqpoint{8.035345in}{1.571048in}}%
\pgfpathlineto{\pgfqpoint{8.037096in}{1.501535in}}%
\pgfpathlineto{\pgfqpoint{8.037971in}{1.618358in}}%
\pgfpathlineto{\pgfqpoint{8.039722in}{2.186465in}}%
\pgfpathlineto{\pgfqpoint{8.041472in}{2.829076in}}%
\pgfpathlineto{\pgfqpoint{8.042348in}{2.861975in}}%
\pgfpathlineto{\pgfqpoint{8.043223in}{2.951585in}}%
\pgfpathlineto{\pgfqpoint{8.044099in}{2.901337in}}%
\pgfpathlineto{\pgfqpoint{8.044974in}{2.947403in}}%
\pgfpathlineto{\pgfqpoint{8.045849in}{2.938437in}}%
\pgfpathlineto{\pgfqpoint{8.046725in}{2.820086in}}%
\pgfpathlineto{\pgfqpoint{8.050226in}{2.121632in}}%
\pgfpathlineto{\pgfqpoint{8.051102in}{1.972225in}}%
\pgfpathlineto{\pgfqpoint{8.051977in}{1.952893in}}%
\pgfpathlineto{\pgfqpoint{8.054603in}{1.492096in}}%
\pgfpathlineto{\pgfqpoint{8.055479in}{1.444748in}}%
\pgfpathlineto{\pgfqpoint{8.056354in}{1.378558in}}%
\pgfpathlineto{\pgfqpoint{8.058105in}{1.292508in}}%
\pgfpathlineto{\pgfqpoint{8.058980in}{1.367231in}}%
\pgfpathlineto{\pgfqpoint{8.061606in}{2.114050in}}%
\pgfpathlineto{\pgfqpoint{8.062482in}{2.183368in}}%
\pgfpathlineto{\pgfqpoint{8.063357in}{2.293151in}}%
\pgfpathlineto{\pgfqpoint{8.064232in}{2.290775in}}%
\pgfpathlineto{\pgfqpoint{8.065108in}{2.313000in}}%
\pgfpathlineto{\pgfqpoint{8.065983in}{2.321447in}}%
\pgfpathlineto{\pgfqpoint{8.066858in}{2.387540in}}%
\pgfpathlineto{\pgfqpoint{8.067734in}{2.421092in}}%
\pgfpathlineto{\pgfqpoint{8.068609in}{2.328583in}}%
\pgfpathlineto{\pgfqpoint{8.069485in}{2.095202in}}%
\pgfpathlineto{\pgfqpoint{8.070360in}{1.984269in}}%
\pgfpathlineto{\pgfqpoint{8.071235in}{1.955158in}}%
\pgfpathlineto{\pgfqpoint{8.072111in}{1.959198in}}%
\pgfpathlineto{\pgfqpoint{8.072986in}{1.785701in}}%
\pgfpathlineto{\pgfqpoint{8.073861in}{1.667594in}}%
\pgfpathlineto{\pgfqpoint{8.075612in}{1.366740in}}%
\pgfpathlineto{\pgfqpoint{8.076488in}{1.324942in}}%
\pgfpathlineto{\pgfqpoint{8.077363in}{1.249200in}}%
\pgfpathlineto{\pgfqpoint{8.078238in}{1.147631in}}%
\pgfpathlineto{\pgfqpoint{8.079114in}{1.202569in}}%
\pgfpathlineto{\pgfqpoint{8.079989in}{1.283031in}}%
\pgfpathlineto{\pgfqpoint{8.082615in}{1.812989in}}%
\pgfpathlineto{\pgfqpoint{8.083491in}{1.840330in}}%
\pgfpathlineto{\pgfqpoint{8.085241in}{2.002768in}}%
\pgfpathlineto{\pgfqpoint{8.086117in}{2.034842in}}%
\pgfpathlineto{\pgfqpoint{8.086992in}{2.289754in}}%
\pgfpathlineto{\pgfqpoint{8.087867in}{2.354821in}}%
\pgfpathlineto{\pgfqpoint{8.088743in}{2.318683in}}%
\pgfpathlineto{\pgfqpoint{8.089618in}{2.260355in}}%
\pgfpathlineto{\pgfqpoint{8.091369in}{1.948248in}}%
\pgfpathlineto{\pgfqpoint{8.093995in}{1.752096in}}%
\pgfpathlineto{\pgfqpoint{8.094870in}{1.649282in}}%
\pgfpathlineto{\pgfqpoint{8.095746in}{1.500742in}}%
\pgfpathlineto{\pgfqpoint{8.098372in}{1.266833in}}%
\pgfpathlineto{\pgfqpoint{8.099247in}{1.214651in}}%
\pgfpathlineto{\pgfqpoint{8.100123in}{1.220957in}}%
\pgfpathlineto{\pgfqpoint{8.100998in}{1.282238in}}%
\pgfpathlineto{\pgfqpoint{8.102749in}{1.831069in}}%
\pgfpathlineto{\pgfqpoint{8.104500in}{2.173719in}}%
\pgfpathlineto{\pgfqpoint{8.105375in}{2.252081in}}%
\pgfpathlineto{\pgfqpoint{8.106250in}{2.260781in}}%
\pgfpathlineto{\pgfqpoint{8.107126in}{2.258157in}}%
\pgfpathlineto{\pgfqpoint{8.108001in}{2.393751in}}%
\pgfpathlineto{\pgfqpoint{8.109752in}{2.434723in}}%
\pgfpathlineto{\pgfqpoint{8.110627in}{2.368682in}}%
\pgfpathlineto{\pgfqpoint{8.111503in}{2.161769in}}%
\pgfpathlineto{\pgfqpoint{8.115004in}{1.714867in}}%
\pgfpathlineto{\pgfqpoint{8.115879in}{1.560513in}}%
\pgfpathlineto{\pgfqpoint{8.117630in}{1.358962in}}%
\pgfpathlineto{\pgfqpoint{8.118506in}{1.322714in}}%
\pgfpathlineto{\pgfqpoint{8.119381in}{1.253655in}}%
\pgfpathlineto{\pgfqpoint{8.120256in}{1.232360in}}%
\pgfpathlineto{\pgfqpoint{8.121132in}{1.272308in}}%
\pgfpathlineto{\pgfqpoint{8.122007in}{1.413031in}}%
\pgfpathlineto{\pgfqpoint{8.124633in}{2.128809in}}%
\pgfpathlineto{\pgfqpoint{8.125509in}{2.196227in}}%
\pgfpathlineto{\pgfqpoint{8.126384in}{2.211843in}}%
\pgfpathlineto{\pgfqpoint{8.127259in}{2.209364in}}%
\pgfpathlineto{\pgfqpoint{8.128135in}{2.224018in}}%
\pgfpathlineto{\pgfqpoint{8.129885in}{2.317230in}}%
\pgfpathlineto{\pgfqpoint{8.130761in}{2.286251in}}%
\pgfpathlineto{\pgfqpoint{8.131636in}{2.244874in}}%
\pgfpathlineto{\pgfqpoint{8.132512in}{2.017081in}}%
\pgfpathlineto{\pgfqpoint{8.134262in}{1.778338in}}%
\pgfpathlineto{\pgfqpoint{8.135138in}{1.734539in}}%
\pgfpathlineto{\pgfqpoint{8.136013in}{1.676316in}}%
\pgfpathlineto{\pgfqpoint{8.136888in}{1.582904in}}%
\pgfpathlineto{\pgfqpoint{8.137764in}{1.431985in}}%
\pgfpathlineto{\pgfqpoint{8.138639in}{1.381956in}}%
\pgfpathlineto{\pgfqpoint{8.139515in}{1.298021in}}%
\pgfpathlineto{\pgfqpoint{8.140390in}{1.287260in}}%
\pgfpathlineto{\pgfqpoint{8.141265in}{1.271137in}}%
\pgfpathlineto{\pgfqpoint{8.142141in}{1.278424in}}%
\pgfpathlineto{\pgfqpoint{8.143016in}{1.371271in}}%
\pgfpathlineto{\pgfqpoint{8.143891in}{1.510295in}}%
\pgfpathlineto{\pgfqpoint{8.144767in}{1.696911in}}%
\pgfpathlineto{\pgfqpoint{8.145642in}{1.800969in}}%
\pgfpathlineto{\pgfqpoint{8.146518in}{1.718070in}}%
\pgfpathlineto{\pgfqpoint{8.147393in}{1.827424in}}%
\pgfpathlineto{\pgfqpoint{8.148268in}{1.817742in}}%
\pgfpathlineto{\pgfqpoint{8.149144in}{1.859593in}}%
\pgfpathlineto{\pgfqpoint{8.150019in}{1.966242in}}%
\pgfpathlineto{\pgfqpoint{8.150894in}{1.999903in}}%
\pgfpathlineto{\pgfqpoint{8.152645in}{1.913851in}}%
\pgfpathlineto{\pgfqpoint{8.153521in}{1.839808in}}%
\pgfpathlineto{\pgfqpoint{8.155271in}{1.551262in}}%
\pgfpathlineto{\pgfqpoint{8.156147in}{1.624135in}}%
\pgfpathlineto{\pgfqpoint{8.157022in}{1.496816in}}%
\pgfpathlineto{\pgfqpoint{8.157897in}{1.461399in}}%
\pgfpathlineto{\pgfqpoint{8.158773in}{1.306780in}}%
\pgfpathlineto{\pgfqpoint{8.160524in}{1.149670in}}%
\pgfpathlineto{\pgfqpoint{8.161399in}{1.164056in}}%
\pgfpathlineto{\pgfqpoint{8.162274in}{1.130867in}}%
\pgfpathlineto{\pgfqpoint{8.163150in}{1.156164in}}%
\pgfpathlineto{\pgfqpoint{8.164025in}{1.235758in}}%
\pgfpathlineto{\pgfqpoint{8.164900in}{1.386525in}}%
\pgfpathlineto{\pgfqpoint{8.166651in}{1.740753in}}%
\pgfpathlineto{\pgfqpoint{8.167527in}{1.801278in}}%
\pgfpathlineto{\pgfqpoint{8.168402in}{1.971537in}}%
\pgfpathlineto{\pgfqpoint{8.169277in}{1.959511in}}%
\pgfpathlineto{\pgfqpoint{8.170153in}{1.984646in}}%
\pgfpathlineto{\pgfqpoint{8.171903in}{2.205456in}}%
\pgfpathlineto{\pgfqpoint{8.172779in}{2.136862in}}%
\pgfpathlineto{\pgfqpoint{8.173654in}{1.972791in}}%
\pgfpathlineto{\pgfqpoint{8.175405in}{1.388337in}}%
\pgfpathlineto{\pgfqpoint{8.176280in}{1.150803in}}%
\pgfpathlineto{\pgfqpoint{8.177156in}{1.203928in}}%
\pgfpathlineto{\pgfqpoint{8.179782in}{0.876794in}}%
\pgfpathlineto{\pgfqpoint{8.182408in}{0.746756in}}%
\pgfpathlineto{\pgfqpoint{8.184159in}{1.158732in}}%
\pgfpathlineto{\pgfqpoint{8.185034in}{1.178404in}}%
\pgfpathlineto{\pgfqpoint{8.187660in}{2.080224in}}%
\pgfpathlineto{\pgfqpoint{8.188536in}{2.107525in}}%
\pgfpathlineto{\pgfqpoint{8.189411in}{2.124130in}}%
\pgfpathlineto{\pgfqpoint{8.190286in}{2.171529in}}%
\pgfpathlineto{\pgfqpoint{8.191162in}{2.195731in}}%
\pgfpathlineto{\pgfqpoint{8.192037in}{2.247867in}}%
\pgfpathlineto{\pgfqpoint{8.192913in}{2.254634in}}%
\pgfpathlineto{\pgfqpoint{8.193788in}{2.209660in}}%
\pgfpathlineto{\pgfqpoint{8.194663in}{2.087952in}}%
\pgfpathlineto{\pgfqpoint{8.195539in}{1.790458in}}%
\pgfpathlineto{\pgfqpoint{8.197289in}{1.509162in}}%
\pgfpathlineto{\pgfqpoint{8.198165in}{1.384901in}}%
\pgfpathlineto{\pgfqpoint{8.200791in}{1.157901in}}%
\pgfpathlineto{\pgfqpoint{8.202542in}{1.041381in}}%
\pgfpathlineto{\pgfqpoint{8.203417in}{1.017367in}}%
\pgfpathlineto{\pgfqpoint{8.205168in}{1.106022in}}%
\pgfpathlineto{\pgfqpoint{8.206043in}{1.138947in}}%
\pgfpathlineto{\pgfqpoint{8.207794in}{1.522039in}}%
\pgfpathlineto{\pgfqpoint{8.208669in}{1.591297in}}%
\pgfpathlineto{\pgfqpoint{8.209545in}{1.730928in}}%
\pgfpathlineto{\pgfqpoint{8.210420in}{1.699511in}}%
\pgfpathlineto{\pgfqpoint{8.211295in}{1.606213in}}%
\pgfpathlineto{\pgfqpoint{8.212171in}{1.717672in}}%
\pgfpathlineto{\pgfqpoint{8.213046in}{1.792112in}}%
\pgfpathlineto{\pgfqpoint{8.213922in}{1.907210in}}%
\pgfpathlineto{\pgfqpoint{8.214797in}{1.821607in}}%
\pgfpathlineto{\pgfqpoint{8.215672in}{1.768181in}}%
\pgfpathlineto{\pgfqpoint{8.216548in}{1.580865in}}%
\pgfpathlineto{\pgfqpoint{8.218298in}{1.410350in}}%
\pgfpathlineto{\pgfqpoint{8.219174in}{1.448184in}}%
\pgfpathlineto{\pgfqpoint{8.220049in}{1.377312in}}%
\pgfpathlineto{\pgfqpoint{8.220925in}{1.358849in}}%
\pgfpathlineto{\pgfqpoint{8.221800in}{1.270080in}}%
\pgfpathlineto{\pgfqpoint{8.223551in}{1.201889in}}%
\pgfpathlineto{\pgfqpoint{8.224426in}{1.202418in}}%
\pgfpathlineto{\pgfqpoint{8.225301in}{1.189656in}}%
\pgfpathlineto{\pgfqpoint{8.226177in}{1.210271in}}%
\pgfpathlineto{\pgfqpoint{8.227928in}{1.354016in}}%
\pgfpathlineto{\pgfqpoint{8.228803in}{1.569943in}}%
\pgfpathlineto{\pgfqpoint{8.229678in}{1.656005in}}%
\pgfpathlineto{\pgfqpoint{8.230554in}{1.701036in}}%
\pgfpathlineto{\pgfqpoint{8.231429in}{1.701596in}}%
\pgfpathlineto{\pgfqpoint{8.232304in}{1.597385in}}%
\pgfpathlineto{\pgfqpoint{8.233180in}{1.576400in}}%
\pgfpathlineto{\pgfqpoint{8.234055in}{1.515978in}}%
\pgfpathlineto{\pgfqpoint{8.234931in}{1.520404in}}%
\pgfpathlineto{\pgfqpoint{8.235806in}{1.478029in}}%
\pgfpathlineto{\pgfqpoint{8.236681in}{1.604275in}}%
\pgfpathlineto{\pgfqpoint{8.238432in}{1.406990in}}%
\pgfpathlineto{\pgfqpoint{8.239307in}{1.350202in}}%
\pgfpathlineto{\pgfqpoint{8.241058in}{1.321128in}}%
\pgfpathlineto{\pgfqpoint{8.241934in}{1.218880in}}%
\pgfpathlineto{\pgfqpoint{8.244560in}{1.059726in}}%
\pgfpathlineto{\pgfqpoint{8.246310in}{1.016234in}}%
\pgfpathlineto{\pgfqpoint{8.248061in}{1.118822in}}%
\pgfpathlineto{\pgfqpoint{8.248937in}{1.221297in}}%
\pgfpathlineto{\pgfqpoint{8.249812in}{1.230850in}}%
\pgfpathlineto{\pgfqpoint{8.250687in}{1.176208in}}%
\pgfpathlineto{\pgfqpoint{8.251563in}{1.163316in}}%
\pgfpathlineto{\pgfqpoint{8.252438in}{1.156664in}}%
\pgfpathlineto{\pgfqpoint{8.253313in}{1.111279in}}%
\pgfpathlineto{\pgfqpoint{8.254189in}{1.177298in}}%
\pgfpathlineto{\pgfqpoint{8.256815in}{1.296862in}}%
\pgfpathlineto{\pgfqpoint{8.257690in}{1.428889in}}%
\pgfpathlineto{\pgfqpoint{8.258566in}{1.405215in}}%
\pgfpathlineto{\pgfqpoint{8.260316in}{1.199624in}}%
\pgfpathlineto{\pgfqpoint{8.262067in}{1.136002in}}%
\pgfpathlineto{\pgfqpoint{8.263818in}{0.950195in}}%
\pgfpathlineto{\pgfqpoint{8.264693in}{0.907454in}}%
\pgfpathlineto{\pgfqpoint{8.266444in}{0.961749in}}%
\pgfpathlineto{\pgfqpoint{8.268195in}{0.946457in}}%
\pgfpathlineto{\pgfqpoint{8.269070in}{1.005473in}}%
\pgfpathlineto{\pgfqpoint{8.270821in}{1.207204in}}%
\pgfpathlineto{\pgfqpoint{8.271696in}{1.168570in}}%
\pgfpathlineto{\pgfqpoint{8.272572in}{1.264327in}}%
\pgfpathlineto{\pgfqpoint{8.273447in}{1.137746in}}%
\pgfpathlineto{\pgfqpoint{8.274322in}{1.136466in}}%
\pgfpathlineto{\pgfqpoint{8.275198in}{1.184134in}}%
\pgfpathlineto{\pgfqpoint{8.276073in}{1.184611in}}%
\pgfpathlineto{\pgfqpoint{8.276949in}{1.183935in}}%
\pgfpathlineto{\pgfqpoint{8.278699in}{1.336609in}}%
\pgfpathlineto{\pgfqpoint{8.279575in}{1.284994in}}%
\pgfpathlineto{\pgfqpoint{8.281325in}{1.154730in}}%
\pgfpathlineto{\pgfqpoint{8.282201in}{1.145743in}}%
\pgfpathlineto{\pgfqpoint{8.284827in}{0.984140in}}%
\pgfpathlineto{\pgfqpoint{8.285702in}{1.011854in}}%
\pgfpathlineto{\pgfqpoint{8.286578in}{1.015307in}}%
\pgfpathlineto{\pgfqpoint{8.287453in}{0.952661in}}%
\pgfpathlineto{\pgfqpoint{8.288328in}{0.915415in}}%
\pgfpathlineto{\pgfqpoint{8.289204in}{0.949717in}}%
\pgfpathlineto{\pgfqpoint{8.290079in}{0.999369in}}%
\pgfpathlineto{\pgfqpoint{8.290955in}{1.119728in}}%
\pgfpathlineto{\pgfqpoint{8.291830in}{1.141203in}}%
\pgfpathlineto{\pgfqpoint{8.292705in}{1.054128in}}%
\pgfpathlineto{\pgfqpoint{8.294456in}{1.202324in}}%
\pgfpathlineto{\pgfqpoint{8.295331in}{1.257946in}}%
\pgfpathlineto{\pgfqpoint{8.296207in}{1.255954in}}%
\pgfpathlineto{\pgfqpoint{8.297082in}{1.282685in}}%
\pgfpathlineto{\pgfqpoint{8.297958in}{1.359975in}}%
\pgfpathlineto{\pgfqpoint{8.298833in}{1.332645in}}%
\pgfpathlineto{\pgfqpoint{8.299708in}{1.418015in}}%
\pgfpathlineto{\pgfqpoint{8.300584in}{1.357603in}}%
\pgfpathlineto{\pgfqpoint{8.301459in}{1.361001in}}%
\pgfpathlineto{\pgfqpoint{8.303210in}{1.284277in}}%
\pgfpathlineto{\pgfqpoint{8.304085in}{1.211895in}}%
\pgfpathlineto{\pgfqpoint{8.304961in}{1.185918in}}%
\pgfpathlineto{\pgfqpoint{8.305836in}{1.101227in}}%
\pgfpathlineto{\pgfqpoint{8.306711in}{1.053690in}}%
\pgfpathlineto{\pgfqpoint{8.307587in}{1.082943in}}%
\pgfpathlineto{\pgfqpoint{8.308462in}{1.086728in}}%
\pgfpathlineto{\pgfqpoint{8.309337in}{1.096507in}}%
\pgfpathlineto{\pgfqpoint{8.310213in}{1.034773in}}%
\pgfpathlineto{\pgfqpoint{8.312839in}{1.341498in}}%
\pgfpathlineto{\pgfqpoint{8.313714in}{1.346027in}}%
\pgfpathlineto{\pgfqpoint{8.315465in}{1.310953in}}%
\pgfpathlineto{\pgfqpoint{8.316340in}{1.354670in}}%
\pgfpathlineto{\pgfqpoint{8.317216in}{1.309474in}}%
\pgfpathlineto{\pgfqpoint{8.318091in}{1.222118in}}%
\pgfpathlineto{\pgfqpoint{8.318967in}{1.260306in}}%
\pgfpathlineto{\pgfqpoint{8.319842in}{1.157318in}}%
\pgfpathlineto{\pgfqpoint{8.320717in}{1.132188in}}%
\pgfpathlineto{\pgfqpoint{8.321593in}{1.262717in}}%
\pgfpathlineto{\pgfqpoint{8.322468in}{1.266644in}}%
\pgfpathlineto{\pgfqpoint{8.323344in}{1.252220in}}%
\pgfpathlineto{\pgfqpoint{8.324219in}{1.206798in}}%
\pgfpathlineto{\pgfqpoint{8.325094in}{1.179272in}}%
\pgfpathlineto{\pgfqpoint{8.325970in}{1.126147in}}%
\pgfpathlineto{\pgfqpoint{8.326845in}{1.172551in}}%
\pgfpathlineto{\pgfqpoint{8.327720in}{1.187806in}}%
\pgfpathlineto{\pgfqpoint{8.328596in}{1.160054in}}%
\pgfpathlineto{\pgfqpoint{8.329471in}{1.185993in}}%
\pgfpathlineto{\pgfqpoint{8.330347in}{1.126494in}}%
\pgfpathlineto{\pgfqpoint{8.331222in}{1.102473in}}%
\pgfpathlineto{\pgfqpoint{8.332097in}{1.128714in}}%
\pgfpathlineto{\pgfqpoint{8.332973in}{1.318674in}}%
\pgfpathlineto{\pgfqpoint{8.334723in}{1.470285in}}%
\pgfpathlineto{\pgfqpoint{8.335599in}{1.475158in}}%
\pgfpathlineto{\pgfqpoint{8.336474in}{1.501520in}}%
\pgfpathlineto{\pgfqpoint{8.337350in}{1.551236in}}%
\pgfpathlineto{\pgfqpoint{8.338225in}{1.504933in}}%
\pgfpathlineto{\pgfqpoint{8.339100in}{1.499808in}}%
\pgfpathlineto{\pgfqpoint{8.339976in}{1.523669in}}%
\pgfpathlineto{\pgfqpoint{8.340851in}{1.497138in}}%
\pgfpathlineto{\pgfqpoint{8.341726in}{1.530307in}}%
\pgfpathlineto{\pgfqpoint{8.344353in}{1.267210in}}%
\pgfpathlineto{\pgfqpoint{8.345228in}{1.277141in}}%
\pgfpathlineto{\pgfqpoint{8.346103in}{1.262188in}}%
\pgfpathlineto{\pgfqpoint{8.346979in}{1.231756in}}%
\pgfpathlineto{\pgfqpoint{8.347854in}{1.161941in}}%
\pgfpathlineto{\pgfqpoint{8.348729in}{1.171909in}}%
\pgfpathlineto{\pgfqpoint{8.349605in}{1.127808in}}%
\pgfpathlineto{\pgfqpoint{8.350480in}{1.137437in}}%
\pgfpathlineto{\pgfqpoint{8.351356in}{1.068906in}}%
\pgfpathlineto{\pgfqpoint{8.352231in}{1.029676in}}%
\pgfpathlineto{\pgfqpoint{8.353106in}{1.044639in}}%
\pgfpathlineto{\pgfqpoint{8.353982in}{1.170399in}}%
\pgfpathlineto{\pgfqpoint{8.354857in}{1.250432in}}%
\pgfpathlineto{\pgfqpoint{8.355732in}{1.242246in}}%
\pgfpathlineto{\pgfqpoint{8.356608in}{1.244824in}}%
\pgfpathlineto{\pgfqpoint{8.357483in}{1.259565in}}%
\pgfpathlineto{\pgfqpoint{8.358359in}{1.257112in}}%
\pgfpathlineto{\pgfqpoint{8.359234in}{1.302081in}}%
\pgfpathlineto{\pgfqpoint{8.360109in}{1.569119in}}%
\pgfpathlineto{\pgfqpoint{8.360985in}{1.417817in}}%
\pgfpathlineto{\pgfqpoint{8.361860in}{1.493793in}}%
\pgfpathlineto{\pgfqpoint{8.362735in}{1.502895in}}%
\pgfpathlineto{\pgfqpoint{8.363611in}{1.419110in}}%
\pgfpathlineto{\pgfqpoint{8.364486in}{1.274875in}}%
\pgfpathlineto{\pgfqpoint{8.365362in}{1.221712in}}%
\pgfpathlineto{\pgfqpoint{8.366237in}{1.150161in}}%
\pgfpathlineto{\pgfqpoint{8.367112in}{1.128752in}}%
\pgfpathlineto{\pgfqpoint{8.367988in}{1.114820in}}%
\pgfpathlineto{\pgfqpoint{8.368863in}{1.115008in}}%
\pgfpathlineto{\pgfqpoint{8.369738in}{1.065772in}}%
\pgfpathlineto{\pgfqpoint{8.370614in}{0.981383in}}%
\pgfpathlineto{\pgfqpoint{8.371489in}{0.981383in}}%
\pgfpathlineto{\pgfqpoint{8.372365in}{1.006228in}}%
\pgfpathlineto{\pgfqpoint{8.374115in}{1.079290in}}%
\pgfpathlineto{\pgfqpoint{8.374991in}{1.173420in}}%
\pgfpathlineto{\pgfqpoint{8.375866in}{1.235334in}}%
\pgfpathlineto{\pgfqpoint{8.376741in}{1.266615in}}%
\pgfpathlineto{\pgfqpoint{8.377617in}{1.229582in}}%
\pgfpathlineto{\pgfqpoint{8.378492in}{1.277424in}}%
\pgfpathlineto{\pgfqpoint{8.379368in}{1.215108in}}%
\pgfpathlineto{\pgfqpoint{8.381994in}{1.463947in}}%
\pgfpathlineto{\pgfqpoint{8.383744in}{1.533176in}}%
\pgfpathlineto{\pgfqpoint{8.385495in}{1.345067in}}%
\pgfpathlineto{\pgfqpoint{8.387246in}{1.202795in}}%
\pgfpathlineto{\pgfqpoint{8.388121in}{1.165944in}}%
\pgfpathlineto{\pgfqpoint{8.388997in}{1.102322in}}%
\pgfpathlineto{\pgfqpoint{8.389872in}{0.997128in}}%
\pgfpathlineto{\pgfqpoint{8.390747in}{1.004718in}}%
\pgfpathlineto{\pgfqpoint{8.391623in}{0.997166in}}%
\pgfpathlineto{\pgfqpoint{8.392498in}{1.019141in}}%
\pgfpathlineto{\pgfqpoint{8.393374in}{0.950611in}}%
\pgfpathlineto{\pgfqpoint{8.394249in}{0.953367in}}%
\pgfpathlineto{\pgfqpoint{8.395124in}{0.976777in}}%
\pgfpathlineto{\pgfqpoint{8.396875in}{1.238648in}}%
\pgfpathlineto{\pgfqpoint{8.397750in}{1.237114in}}%
\pgfpathlineto{\pgfqpoint{8.398626in}{1.232433in}}%
\pgfpathlineto{\pgfqpoint{8.399501in}{1.358153in}}%
\pgfpathlineto{\pgfqpoint{8.400377in}{1.255675in}}%
\pgfpathlineto{\pgfqpoint{8.401252in}{1.348313in}}%
\pgfpathlineto{\pgfqpoint{8.402127in}{1.368120in}}%
\pgfpathlineto{\pgfqpoint{8.403003in}{1.342558in}}%
\pgfpathlineto{\pgfqpoint{8.403878in}{1.336031in}}%
\pgfpathlineto{\pgfqpoint{8.404753in}{1.466345in}}%
\pgfpathlineto{\pgfqpoint{8.405629in}{1.425038in}}%
\pgfpathlineto{\pgfqpoint{8.406504in}{1.322714in}}%
\pgfpathlineto{\pgfqpoint{8.407380in}{1.270231in}}%
\pgfpathlineto{\pgfqpoint{8.408255in}{1.104361in}}%
\pgfpathlineto{\pgfqpoint{8.409130in}{1.149255in}}%
\pgfpathlineto{\pgfqpoint{8.410006in}{1.058636in}}%
\pgfpathlineto{\pgfqpoint{8.410881in}{1.220164in}}%
\pgfpathlineto{\pgfqpoint{8.411756in}{1.159751in}}%
\pgfpathlineto{\pgfqpoint{8.411756in}{1.159751in}}%
\pgfusepath{stroke}%
\end{pgfscope}%
\begin{pgfscope}%
\pgfsetrectcap%
\pgfsetmiterjoin%
\pgfsetlinewidth{0.803000pt}%
\definecolor{currentstroke}{rgb}{0.000000,0.000000,0.000000}%
\pgfsetstrokecolor{currentstroke}%
\pgfsetdash{}{0pt}%
\pgfpathmoveto{\pgfqpoint{0.742589in}{0.670138in}}%
\pgfpathlineto{\pgfqpoint{0.742589in}{5.516628in}}%
\pgfusepath{stroke}%
\end{pgfscope}%
\begin{pgfscope}%
\pgfsetrectcap%
\pgfsetmiterjoin%
\pgfsetlinewidth{0.803000pt}%
\definecolor{currentstroke}{rgb}{0.000000,0.000000,0.000000}%
\pgfsetstrokecolor{currentstroke}%
\pgfsetdash{}{0pt}%
\pgfpathmoveto{\pgfqpoint{8.410881in}{0.670138in}}%
\pgfpathlineto{\pgfqpoint{8.410881in}{5.516628in}}%
\pgfusepath{stroke}%
\end{pgfscope}%
\begin{pgfscope}%
\pgfsetrectcap%
\pgfsetmiterjoin%
\pgfsetlinewidth{0.803000pt}%
\definecolor{currentstroke}{rgb}{0.000000,0.000000,0.000000}%
\pgfsetstrokecolor{currentstroke}%
\pgfsetdash{}{0pt}%
\pgfpathmoveto{\pgfqpoint{0.742589in}{0.670138in}}%
\pgfpathlineto{\pgfqpoint{8.410881in}{0.670138in}}%
\pgfusepath{stroke}%
\end{pgfscope}%
\begin{pgfscope}%
\pgfsetrectcap%
\pgfsetmiterjoin%
\pgfsetlinewidth{0.803000pt}%
\definecolor{currentstroke}{rgb}{0.000000,0.000000,0.000000}%
\pgfsetstrokecolor{currentstroke}%
\pgfsetdash{}{0pt}%
\pgfpathmoveto{\pgfqpoint{0.742589in}{5.516628in}}%
\pgfpathlineto{\pgfqpoint{8.410881in}{5.516628in}}%
\pgfusepath{stroke}%
\end{pgfscope}%
\begin{pgfscope}%
\definecolor{textcolor}{rgb}{0.000000,0.000000,0.000000}%
\pgfsetstrokecolor{textcolor}%
\pgfsetfillcolor{textcolor}%
\pgftext[x=4.576735in,y=5.599962in,,base]{\color{textcolor}{\rmfamily\fontsize{20.000000}{24.000000}\selectfont\catcode`\^=\active\def^{\ifmmode\sp\else\^{}\fi}\catcode`\%=\active\def%{\%}Normalized Demand Curve}}%
\end{pgfscope}%
\begin{pgfscope}%
\pgfsetbuttcap%
\pgfsetmiterjoin%
\definecolor{currentfill}{rgb}{1.000000,1.000000,1.000000}%
\pgfsetfillcolor{currentfill}%
\pgfsetlinewidth{0.000000pt}%
\definecolor{currentstroke}{rgb}{0.000000,0.000000,0.000000}%
\pgfsetstrokecolor{currentstroke}%
\pgfsetstrokeopacity{0.000000}%
\pgfsetdash{}{0pt}%
\pgfpathmoveto{\pgfqpoint{8.609492in}{0.670138in}}%
\pgfpathlineto{\pgfqpoint{11.676809in}{0.670138in}}%
\pgfpathlineto{\pgfqpoint{11.676809in}{5.516628in}}%
\pgfpathlineto{\pgfqpoint{8.609492in}{5.516628in}}%
\pgfpathlineto{\pgfqpoint{8.609492in}{0.670138in}}%
\pgfpathclose%
\pgfusepath{fill}%
\end{pgfscope}%
\begin{pgfscope}%
\pgfpathrectangle{\pgfqpoint{8.609492in}{0.670138in}}{\pgfqpoint{3.067317in}{4.846490in}}%
\pgfusepath{clip}%
\pgfsetrectcap%
\pgfsetroundjoin%
\pgfsetlinewidth{0.803000pt}%
\definecolor{currentstroke}{rgb}{0.690196,0.690196,0.690196}%
\pgfsetstrokecolor{currentstroke}%
\pgfsetdash{}{0pt}%
\pgfpathmoveto{\pgfqpoint{8.639564in}{0.670138in}}%
\pgfpathlineto{\pgfqpoint{8.639564in}{5.516628in}}%
\pgfusepath{stroke}%
\end{pgfscope}%
\begin{pgfscope}%
\pgfsetbuttcap%
\pgfsetroundjoin%
\definecolor{currentfill}{rgb}{0.000000,0.000000,0.000000}%
\pgfsetfillcolor{currentfill}%
\pgfsetlinewidth{0.803000pt}%
\definecolor{currentstroke}{rgb}{0.000000,0.000000,0.000000}%
\pgfsetstrokecolor{currentstroke}%
\pgfsetdash{}{0pt}%
\pgfsys@defobject{currentmarker}{\pgfqpoint{0.000000in}{-0.048611in}}{\pgfqpoint{0.000000in}{0.000000in}}{%
\pgfpathmoveto{\pgfqpoint{0.000000in}{0.000000in}}%
\pgfpathlineto{\pgfqpoint{0.000000in}{-0.048611in}}%
\pgfusepath{stroke,fill}%
}%
\begin{pgfscope}%
\pgfsys@transformshift{8.639564in}{0.670138in}%
\pgfsys@useobject{currentmarker}{}%
\end{pgfscope}%
\end{pgfscope}%
\begin{pgfscope}%
\definecolor{textcolor}{rgb}{0.000000,0.000000,0.000000}%
\pgfsetstrokecolor{textcolor}%
\pgfsetfillcolor{textcolor}%
\pgftext[x=8.639564in,y=0.572916in,,top]{\color{textcolor}{\rmfamily\fontsize{14.000000}{16.800000}\selectfont\catcode`\^=\active\def^{\ifmmode\sp\else\^{}\fi}\catcode`\%=\active\def%{\%}$\mathdefault{0}$}}%
\end{pgfscope}%
\begin{pgfscope}%
\pgfpathrectangle{\pgfqpoint{8.609492in}{0.670138in}}{\pgfqpoint{3.067317in}{4.846490in}}%
\pgfusepath{clip}%
\pgfsetrectcap%
\pgfsetroundjoin%
\pgfsetlinewidth{0.803000pt}%
\definecolor{currentstroke}{rgb}{0.690196,0.690196,0.690196}%
\pgfsetstrokecolor{currentstroke}%
\pgfsetdash{}{0pt}%
\pgfpathmoveto{\pgfqpoint{9.391357in}{0.670138in}}%
\pgfpathlineto{\pgfqpoint{9.391357in}{5.516628in}}%
\pgfusepath{stroke}%
\end{pgfscope}%
\begin{pgfscope}%
\pgfsetbuttcap%
\pgfsetroundjoin%
\definecolor{currentfill}{rgb}{0.000000,0.000000,0.000000}%
\pgfsetfillcolor{currentfill}%
\pgfsetlinewidth{0.803000pt}%
\definecolor{currentstroke}{rgb}{0.000000,0.000000,0.000000}%
\pgfsetstrokecolor{currentstroke}%
\pgfsetdash{}{0pt}%
\pgfsys@defobject{currentmarker}{\pgfqpoint{0.000000in}{-0.048611in}}{\pgfqpoint{0.000000in}{0.000000in}}{%
\pgfpathmoveto{\pgfqpoint{0.000000in}{0.000000in}}%
\pgfpathlineto{\pgfqpoint{0.000000in}{-0.048611in}}%
\pgfusepath{stroke,fill}%
}%
\begin{pgfscope}%
\pgfsys@transformshift{9.391357in}{0.670138in}%
\pgfsys@useobject{currentmarker}{}%
\end{pgfscope}%
\end{pgfscope}%
\begin{pgfscope}%
\definecolor{textcolor}{rgb}{0.000000,0.000000,0.000000}%
\pgfsetstrokecolor{textcolor}%
\pgfsetfillcolor{textcolor}%
\pgftext[x=9.391357in,y=0.572916in,,top]{\color{textcolor}{\rmfamily\fontsize{14.000000}{16.800000}\selectfont\catcode`\^=\active\def^{\ifmmode\sp\else\^{}\fi}\catcode`\%=\active\def%{\%}$\mathdefault{25}$}}%
\end{pgfscope}%
\begin{pgfscope}%
\pgfpathrectangle{\pgfqpoint{8.609492in}{0.670138in}}{\pgfqpoint{3.067317in}{4.846490in}}%
\pgfusepath{clip}%
\pgfsetrectcap%
\pgfsetroundjoin%
\pgfsetlinewidth{0.803000pt}%
\definecolor{currentstroke}{rgb}{0.690196,0.690196,0.690196}%
\pgfsetstrokecolor{currentstroke}%
\pgfsetdash{}{0pt}%
\pgfpathmoveto{\pgfqpoint{10.143151in}{0.670138in}}%
\pgfpathlineto{\pgfqpoint{10.143151in}{5.516628in}}%
\pgfusepath{stroke}%
\end{pgfscope}%
\begin{pgfscope}%
\pgfsetbuttcap%
\pgfsetroundjoin%
\definecolor{currentfill}{rgb}{0.000000,0.000000,0.000000}%
\pgfsetfillcolor{currentfill}%
\pgfsetlinewidth{0.803000pt}%
\definecolor{currentstroke}{rgb}{0.000000,0.000000,0.000000}%
\pgfsetstrokecolor{currentstroke}%
\pgfsetdash{}{0pt}%
\pgfsys@defobject{currentmarker}{\pgfqpoint{0.000000in}{-0.048611in}}{\pgfqpoint{0.000000in}{0.000000in}}{%
\pgfpathmoveto{\pgfqpoint{0.000000in}{0.000000in}}%
\pgfpathlineto{\pgfqpoint{0.000000in}{-0.048611in}}%
\pgfusepath{stroke,fill}%
}%
\begin{pgfscope}%
\pgfsys@transformshift{10.143151in}{0.670138in}%
\pgfsys@useobject{currentmarker}{}%
\end{pgfscope}%
\end{pgfscope}%
\begin{pgfscope}%
\definecolor{textcolor}{rgb}{0.000000,0.000000,0.000000}%
\pgfsetstrokecolor{textcolor}%
\pgfsetfillcolor{textcolor}%
\pgftext[x=10.143151in,y=0.572916in,,top]{\color{textcolor}{\rmfamily\fontsize{14.000000}{16.800000}\selectfont\catcode`\^=\active\def^{\ifmmode\sp\else\^{}\fi}\catcode`\%=\active\def%{\%}$\mathdefault{50}$}}%
\end{pgfscope}%
\begin{pgfscope}%
\pgfpathrectangle{\pgfqpoint{8.609492in}{0.670138in}}{\pgfqpoint{3.067317in}{4.846490in}}%
\pgfusepath{clip}%
\pgfsetrectcap%
\pgfsetroundjoin%
\pgfsetlinewidth{0.803000pt}%
\definecolor{currentstroke}{rgb}{0.690196,0.690196,0.690196}%
\pgfsetstrokecolor{currentstroke}%
\pgfsetdash{}{0pt}%
\pgfpathmoveto{\pgfqpoint{10.894944in}{0.670138in}}%
\pgfpathlineto{\pgfqpoint{10.894944in}{5.516628in}}%
\pgfusepath{stroke}%
\end{pgfscope}%
\begin{pgfscope}%
\pgfsetbuttcap%
\pgfsetroundjoin%
\definecolor{currentfill}{rgb}{0.000000,0.000000,0.000000}%
\pgfsetfillcolor{currentfill}%
\pgfsetlinewidth{0.803000pt}%
\definecolor{currentstroke}{rgb}{0.000000,0.000000,0.000000}%
\pgfsetstrokecolor{currentstroke}%
\pgfsetdash{}{0pt}%
\pgfsys@defobject{currentmarker}{\pgfqpoint{0.000000in}{-0.048611in}}{\pgfqpoint{0.000000in}{0.000000in}}{%
\pgfpathmoveto{\pgfqpoint{0.000000in}{0.000000in}}%
\pgfpathlineto{\pgfqpoint{0.000000in}{-0.048611in}}%
\pgfusepath{stroke,fill}%
}%
\begin{pgfscope}%
\pgfsys@transformshift{10.894944in}{0.670138in}%
\pgfsys@useobject{currentmarker}{}%
\end{pgfscope}%
\end{pgfscope}%
\begin{pgfscope}%
\definecolor{textcolor}{rgb}{0.000000,0.000000,0.000000}%
\pgfsetstrokecolor{textcolor}%
\pgfsetfillcolor{textcolor}%
\pgftext[x=10.894944in,y=0.572916in,,top]{\color{textcolor}{\rmfamily\fontsize{14.000000}{16.800000}\selectfont\catcode`\^=\active\def^{\ifmmode\sp\else\^{}\fi}\catcode`\%=\active\def%{\%}$\mathdefault{75}$}}%
\end{pgfscope}%
\begin{pgfscope}%
\pgfpathrectangle{\pgfqpoint{8.609492in}{0.670138in}}{\pgfqpoint{3.067317in}{4.846490in}}%
\pgfusepath{clip}%
\pgfsetrectcap%
\pgfsetroundjoin%
\pgfsetlinewidth{0.803000pt}%
\definecolor{currentstroke}{rgb}{0.690196,0.690196,0.690196}%
\pgfsetstrokecolor{currentstroke}%
\pgfsetdash{}{0pt}%
\pgfpathmoveto{\pgfqpoint{11.646737in}{0.670138in}}%
\pgfpathlineto{\pgfqpoint{11.646737in}{5.516628in}}%
\pgfusepath{stroke}%
\end{pgfscope}%
\begin{pgfscope}%
\pgfsetbuttcap%
\pgfsetroundjoin%
\definecolor{currentfill}{rgb}{0.000000,0.000000,0.000000}%
\pgfsetfillcolor{currentfill}%
\pgfsetlinewidth{0.803000pt}%
\definecolor{currentstroke}{rgb}{0.000000,0.000000,0.000000}%
\pgfsetstrokecolor{currentstroke}%
\pgfsetdash{}{0pt}%
\pgfsys@defobject{currentmarker}{\pgfqpoint{0.000000in}{-0.048611in}}{\pgfqpoint{0.000000in}{0.000000in}}{%
\pgfpathmoveto{\pgfqpoint{0.000000in}{0.000000in}}%
\pgfpathlineto{\pgfqpoint{0.000000in}{-0.048611in}}%
\pgfusepath{stroke,fill}%
}%
\begin{pgfscope}%
\pgfsys@transformshift{11.646737in}{0.670138in}%
\pgfsys@useobject{currentmarker}{}%
\end{pgfscope}%
\end{pgfscope}%
\begin{pgfscope}%
\definecolor{textcolor}{rgb}{0.000000,0.000000,0.000000}%
\pgfsetstrokecolor{textcolor}%
\pgfsetfillcolor{textcolor}%
\pgftext[x=11.646737in,y=0.572916in,,top]{\color{textcolor}{\rmfamily\fontsize{14.000000}{16.800000}\selectfont\catcode`\^=\active\def^{\ifmmode\sp\else\^{}\fi}\catcode`\%=\active\def%{\%}$\mathdefault{100}$}}%
\end{pgfscope}%
\begin{pgfscope}%
\pgfsetbuttcap%
\pgfsetroundjoin%
\definecolor{currentfill}{rgb}{0.000000,0.000000,0.000000}%
\pgfsetfillcolor{currentfill}%
\pgfsetlinewidth{0.602250pt}%
\definecolor{currentstroke}{rgb}{0.000000,0.000000,0.000000}%
\pgfsetstrokecolor{currentstroke}%
\pgfsetdash{}{0pt}%
\pgfsys@defobject{currentmarker}{\pgfqpoint{0.000000in}{-0.027778in}}{\pgfqpoint{0.000000in}{0.000000in}}{%
\pgfpathmoveto{\pgfqpoint{0.000000in}{0.000000in}}%
\pgfpathlineto{\pgfqpoint{0.000000in}{-0.027778in}}%
\pgfusepath{stroke,fill}%
}%
\begin{pgfscope}%
\pgfsys@transformshift{8.789923in}{0.670138in}%
\pgfsys@useobject{currentmarker}{}%
\end{pgfscope}%
\end{pgfscope}%
\begin{pgfscope}%
\pgfsetbuttcap%
\pgfsetroundjoin%
\definecolor{currentfill}{rgb}{0.000000,0.000000,0.000000}%
\pgfsetfillcolor{currentfill}%
\pgfsetlinewidth{0.602250pt}%
\definecolor{currentstroke}{rgb}{0.000000,0.000000,0.000000}%
\pgfsetstrokecolor{currentstroke}%
\pgfsetdash{}{0pt}%
\pgfsys@defobject{currentmarker}{\pgfqpoint{0.000000in}{-0.027778in}}{\pgfqpoint{0.000000in}{0.000000in}}{%
\pgfpathmoveto{\pgfqpoint{0.000000in}{0.000000in}}%
\pgfpathlineto{\pgfqpoint{0.000000in}{-0.027778in}}%
\pgfusepath{stroke,fill}%
}%
\begin{pgfscope}%
\pgfsys@transformshift{8.940281in}{0.670138in}%
\pgfsys@useobject{currentmarker}{}%
\end{pgfscope}%
\end{pgfscope}%
\begin{pgfscope}%
\pgfsetbuttcap%
\pgfsetroundjoin%
\definecolor{currentfill}{rgb}{0.000000,0.000000,0.000000}%
\pgfsetfillcolor{currentfill}%
\pgfsetlinewidth{0.602250pt}%
\definecolor{currentstroke}{rgb}{0.000000,0.000000,0.000000}%
\pgfsetstrokecolor{currentstroke}%
\pgfsetdash{}{0pt}%
\pgfsys@defobject{currentmarker}{\pgfqpoint{0.000000in}{-0.027778in}}{\pgfqpoint{0.000000in}{0.000000in}}{%
\pgfpathmoveto{\pgfqpoint{0.000000in}{0.000000in}}%
\pgfpathlineto{\pgfqpoint{0.000000in}{-0.027778in}}%
\pgfusepath{stroke,fill}%
}%
\begin{pgfscope}%
\pgfsys@transformshift{9.090640in}{0.670138in}%
\pgfsys@useobject{currentmarker}{}%
\end{pgfscope}%
\end{pgfscope}%
\begin{pgfscope}%
\pgfsetbuttcap%
\pgfsetroundjoin%
\definecolor{currentfill}{rgb}{0.000000,0.000000,0.000000}%
\pgfsetfillcolor{currentfill}%
\pgfsetlinewidth{0.602250pt}%
\definecolor{currentstroke}{rgb}{0.000000,0.000000,0.000000}%
\pgfsetstrokecolor{currentstroke}%
\pgfsetdash{}{0pt}%
\pgfsys@defobject{currentmarker}{\pgfqpoint{0.000000in}{-0.027778in}}{\pgfqpoint{0.000000in}{0.000000in}}{%
\pgfpathmoveto{\pgfqpoint{0.000000in}{0.000000in}}%
\pgfpathlineto{\pgfqpoint{0.000000in}{-0.027778in}}%
\pgfusepath{stroke,fill}%
}%
\begin{pgfscope}%
\pgfsys@transformshift{9.240999in}{0.670138in}%
\pgfsys@useobject{currentmarker}{}%
\end{pgfscope}%
\end{pgfscope}%
\begin{pgfscope}%
\pgfsetbuttcap%
\pgfsetroundjoin%
\definecolor{currentfill}{rgb}{0.000000,0.000000,0.000000}%
\pgfsetfillcolor{currentfill}%
\pgfsetlinewidth{0.602250pt}%
\definecolor{currentstroke}{rgb}{0.000000,0.000000,0.000000}%
\pgfsetstrokecolor{currentstroke}%
\pgfsetdash{}{0pt}%
\pgfsys@defobject{currentmarker}{\pgfqpoint{0.000000in}{-0.027778in}}{\pgfqpoint{0.000000in}{0.000000in}}{%
\pgfpathmoveto{\pgfqpoint{0.000000in}{0.000000in}}%
\pgfpathlineto{\pgfqpoint{0.000000in}{-0.027778in}}%
\pgfusepath{stroke,fill}%
}%
\begin{pgfscope}%
\pgfsys@transformshift{9.541716in}{0.670138in}%
\pgfsys@useobject{currentmarker}{}%
\end{pgfscope}%
\end{pgfscope}%
\begin{pgfscope}%
\pgfsetbuttcap%
\pgfsetroundjoin%
\definecolor{currentfill}{rgb}{0.000000,0.000000,0.000000}%
\pgfsetfillcolor{currentfill}%
\pgfsetlinewidth{0.602250pt}%
\definecolor{currentstroke}{rgb}{0.000000,0.000000,0.000000}%
\pgfsetstrokecolor{currentstroke}%
\pgfsetdash{}{0pt}%
\pgfsys@defobject{currentmarker}{\pgfqpoint{0.000000in}{-0.027778in}}{\pgfqpoint{0.000000in}{0.000000in}}{%
\pgfpathmoveto{\pgfqpoint{0.000000in}{0.000000in}}%
\pgfpathlineto{\pgfqpoint{0.000000in}{-0.027778in}}%
\pgfusepath{stroke,fill}%
}%
\begin{pgfscope}%
\pgfsys@transformshift{9.692075in}{0.670138in}%
\pgfsys@useobject{currentmarker}{}%
\end{pgfscope}%
\end{pgfscope}%
\begin{pgfscope}%
\pgfsetbuttcap%
\pgfsetroundjoin%
\definecolor{currentfill}{rgb}{0.000000,0.000000,0.000000}%
\pgfsetfillcolor{currentfill}%
\pgfsetlinewidth{0.602250pt}%
\definecolor{currentstroke}{rgb}{0.000000,0.000000,0.000000}%
\pgfsetstrokecolor{currentstroke}%
\pgfsetdash{}{0pt}%
\pgfsys@defobject{currentmarker}{\pgfqpoint{0.000000in}{-0.027778in}}{\pgfqpoint{0.000000in}{0.000000in}}{%
\pgfpathmoveto{\pgfqpoint{0.000000in}{0.000000in}}%
\pgfpathlineto{\pgfqpoint{0.000000in}{-0.027778in}}%
\pgfusepath{stroke,fill}%
}%
\begin{pgfscope}%
\pgfsys@transformshift{9.842433in}{0.670138in}%
\pgfsys@useobject{currentmarker}{}%
\end{pgfscope}%
\end{pgfscope}%
\begin{pgfscope}%
\pgfsetbuttcap%
\pgfsetroundjoin%
\definecolor{currentfill}{rgb}{0.000000,0.000000,0.000000}%
\pgfsetfillcolor{currentfill}%
\pgfsetlinewidth{0.602250pt}%
\definecolor{currentstroke}{rgb}{0.000000,0.000000,0.000000}%
\pgfsetstrokecolor{currentstroke}%
\pgfsetdash{}{0pt}%
\pgfsys@defobject{currentmarker}{\pgfqpoint{0.000000in}{-0.027778in}}{\pgfqpoint{0.000000in}{0.000000in}}{%
\pgfpathmoveto{\pgfqpoint{0.000000in}{0.000000in}}%
\pgfpathlineto{\pgfqpoint{0.000000in}{-0.027778in}}%
\pgfusepath{stroke,fill}%
}%
\begin{pgfscope}%
\pgfsys@transformshift{9.992792in}{0.670138in}%
\pgfsys@useobject{currentmarker}{}%
\end{pgfscope}%
\end{pgfscope}%
\begin{pgfscope}%
\pgfsetbuttcap%
\pgfsetroundjoin%
\definecolor{currentfill}{rgb}{0.000000,0.000000,0.000000}%
\pgfsetfillcolor{currentfill}%
\pgfsetlinewidth{0.602250pt}%
\definecolor{currentstroke}{rgb}{0.000000,0.000000,0.000000}%
\pgfsetstrokecolor{currentstroke}%
\pgfsetdash{}{0pt}%
\pgfsys@defobject{currentmarker}{\pgfqpoint{0.000000in}{-0.027778in}}{\pgfqpoint{0.000000in}{0.000000in}}{%
\pgfpathmoveto{\pgfqpoint{0.000000in}{0.000000in}}%
\pgfpathlineto{\pgfqpoint{0.000000in}{-0.027778in}}%
\pgfusepath{stroke,fill}%
}%
\begin{pgfscope}%
\pgfsys@transformshift{10.293509in}{0.670138in}%
\pgfsys@useobject{currentmarker}{}%
\end{pgfscope}%
\end{pgfscope}%
\begin{pgfscope}%
\pgfsetbuttcap%
\pgfsetroundjoin%
\definecolor{currentfill}{rgb}{0.000000,0.000000,0.000000}%
\pgfsetfillcolor{currentfill}%
\pgfsetlinewidth{0.602250pt}%
\definecolor{currentstroke}{rgb}{0.000000,0.000000,0.000000}%
\pgfsetstrokecolor{currentstroke}%
\pgfsetdash{}{0pt}%
\pgfsys@defobject{currentmarker}{\pgfqpoint{0.000000in}{-0.027778in}}{\pgfqpoint{0.000000in}{0.000000in}}{%
\pgfpathmoveto{\pgfqpoint{0.000000in}{0.000000in}}%
\pgfpathlineto{\pgfqpoint{0.000000in}{-0.027778in}}%
\pgfusepath{stroke,fill}%
}%
\begin{pgfscope}%
\pgfsys@transformshift{10.443868in}{0.670138in}%
\pgfsys@useobject{currentmarker}{}%
\end{pgfscope}%
\end{pgfscope}%
\begin{pgfscope}%
\pgfsetbuttcap%
\pgfsetroundjoin%
\definecolor{currentfill}{rgb}{0.000000,0.000000,0.000000}%
\pgfsetfillcolor{currentfill}%
\pgfsetlinewidth{0.602250pt}%
\definecolor{currentstroke}{rgb}{0.000000,0.000000,0.000000}%
\pgfsetstrokecolor{currentstroke}%
\pgfsetdash{}{0pt}%
\pgfsys@defobject{currentmarker}{\pgfqpoint{0.000000in}{-0.027778in}}{\pgfqpoint{0.000000in}{0.000000in}}{%
\pgfpathmoveto{\pgfqpoint{0.000000in}{0.000000in}}%
\pgfpathlineto{\pgfqpoint{0.000000in}{-0.027778in}}%
\pgfusepath{stroke,fill}%
}%
\begin{pgfscope}%
\pgfsys@transformshift{10.594227in}{0.670138in}%
\pgfsys@useobject{currentmarker}{}%
\end{pgfscope}%
\end{pgfscope}%
\begin{pgfscope}%
\pgfsetbuttcap%
\pgfsetroundjoin%
\definecolor{currentfill}{rgb}{0.000000,0.000000,0.000000}%
\pgfsetfillcolor{currentfill}%
\pgfsetlinewidth{0.602250pt}%
\definecolor{currentstroke}{rgb}{0.000000,0.000000,0.000000}%
\pgfsetstrokecolor{currentstroke}%
\pgfsetdash{}{0pt}%
\pgfsys@defobject{currentmarker}{\pgfqpoint{0.000000in}{-0.027778in}}{\pgfqpoint{0.000000in}{0.000000in}}{%
\pgfpathmoveto{\pgfqpoint{0.000000in}{0.000000in}}%
\pgfpathlineto{\pgfqpoint{0.000000in}{-0.027778in}}%
\pgfusepath{stroke,fill}%
}%
\begin{pgfscope}%
\pgfsys@transformshift{10.744585in}{0.670138in}%
\pgfsys@useobject{currentmarker}{}%
\end{pgfscope}%
\end{pgfscope}%
\begin{pgfscope}%
\pgfsetbuttcap%
\pgfsetroundjoin%
\definecolor{currentfill}{rgb}{0.000000,0.000000,0.000000}%
\pgfsetfillcolor{currentfill}%
\pgfsetlinewidth{0.602250pt}%
\definecolor{currentstroke}{rgb}{0.000000,0.000000,0.000000}%
\pgfsetstrokecolor{currentstroke}%
\pgfsetdash{}{0pt}%
\pgfsys@defobject{currentmarker}{\pgfqpoint{0.000000in}{-0.027778in}}{\pgfqpoint{0.000000in}{0.000000in}}{%
\pgfpathmoveto{\pgfqpoint{0.000000in}{0.000000in}}%
\pgfpathlineto{\pgfqpoint{0.000000in}{-0.027778in}}%
\pgfusepath{stroke,fill}%
}%
\begin{pgfscope}%
\pgfsys@transformshift{11.045303in}{0.670138in}%
\pgfsys@useobject{currentmarker}{}%
\end{pgfscope}%
\end{pgfscope}%
\begin{pgfscope}%
\pgfsetbuttcap%
\pgfsetroundjoin%
\definecolor{currentfill}{rgb}{0.000000,0.000000,0.000000}%
\pgfsetfillcolor{currentfill}%
\pgfsetlinewidth{0.602250pt}%
\definecolor{currentstroke}{rgb}{0.000000,0.000000,0.000000}%
\pgfsetstrokecolor{currentstroke}%
\pgfsetdash{}{0pt}%
\pgfsys@defobject{currentmarker}{\pgfqpoint{0.000000in}{-0.027778in}}{\pgfqpoint{0.000000in}{0.000000in}}{%
\pgfpathmoveto{\pgfqpoint{0.000000in}{0.000000in}}%
\pgfpathlineto{\pgfqpoint{0.000000in}{-0.027778in}}%
\pgfusepath{stroke,fill}%
}%
\begin{pgfscope}%
\pgfsys@transformshift{11.195661in}{0.670138in}%
\pgfsys@useobject{currentmarker}{}%
\end{pgfscope}%
\end{pgfscope}%
\begin{pgfscope}%
\pgfsetbuttcap%
\pgfsetroundjoin%
\definecolor{currentfill}{rgb}{0.000000,0.000000,0.000000}%
\pgfsetfillcolor{currentfill}%
\pgfsetlinewidth{0.602250pt}%
\definecolor{currentstroke}{rgb}{0.000000,0.000000,0.000000}%
\pgfsetstrokecolor{currentstroke}%
\pgfsetdash{}{0pt}%
\pgfsys@defobject{currentmarker}{\pgfqpoint{0.000000in}{-0.027778in}}{\pgfqpoint{0.000000in}{0.000000in}}{%
\pgfpathmoveto{\pgfqpoint{0.000000in}{0.000000in}}%
\pgfpathlineto{\pgfqpoint{0.000000in}{-0.027778in}}%
\pgfusepath{stroke,fill}%
}%
\begin{pgfscope}%
\pgfsys@transformshift{11.346020in}{0.670138in}%
\pgfsys@useobject{currentmarker}{}%
\end{pgfscope}%
\end{pgfscope}%
\begin{pgfscope}%
\pgfsetbuttcap%
\pgfsetroundjoin%
\definecolor{currentfill}{rgb}{0.000000,0.000000,0.000000}%
\pgfsetfillcolor{currentfill}%
\pgfsetlinewidth{0.602250pt}%
\definecolor{currentstroke}{rgb}{0.000000,0.000000,0.000000}%
\pgfsetstrokecolor{currentstroke}%
\pgfsetdash{}{0pt}%
\pgfsys@defobject{currentmarker}{\pgfqpoint{0.000000in}{-0.027778in}}{\pgfqpoint{0.000000in}{0.000000in}}{%
\pgfpathmoveto{\pgfqpoint{0.000000in}{0.000000in}}%
\pgfpathlineto{\pgfqpoint{0.000000in}{-0.027778in}}%
\pgfusepath{stroke,fill}%
}%
\begin{pgfscope}%
\pgfsys@transformshift{11.496379in}{0.670138in}%
\pgfsys@useobject{currentmarker}{}%
\end{pgfscope}%
\end{pgfscope}%
\begin{pgfscope}%
\definecolor{textcolor}{rgb}{0.000000,0.000000,0.000000}%
\pgfsetstrokecolor{textcolor}%
\pgfsetfillcolor{textcolor}%
\pgftext[x=10.143151in,y=0.339583in,,top]{\color{textcolor}{\rmfamily\fontsize{18.000000}{21.600000}\selectfont\catcode`\^=\active\def^{\ifmmode\sp\else\^{}\fi}\catcode`\%=\active\def%{\%}Time [\%]}}%
\end{pgfscope}%
\begin{pgfscope}%
\pgfpathrectangle{\pgfqpoint{8.609492in}{0.670138in}}{\pgfqpoint{3.067317in}{4.846490in}}%
\pgfusepath{clip}%
\pgfsetrectcap%
\pgfsetroundjoin%
\pgfsetlinewidth{0.803000pt}%
\definecolor{currentstroke}{rgb}{0.690196,0.690196,0.690196}%
\pgfsetstrokecolor{currentstroke}%
\pgfsetdash{}{0pt}%
\pgfpathmoveto{\pgfqpoint{8.609492in}{1.314483in}}%
\pgfpathlineto{\pgfqpoint{11.676809in}{1.314483in}}%
\pgfusepath{stroke}%
\end{pgfscope}%
\begin{pgfscope}%
\pgfsetbuttcap%
\pgfsetroundjoin%
\definecolor{currentfill}{rgb}{0.000000,0.000000,0.000000}%
\pgfsetfillcolor{currentfill}%
\pgfsetlinewidth{0.803000pt}%
\definecolor{currentstroke}{rgb}{0.000000,0.000000,0.000000}%
\pgfsetstrokecolor{currentstroke}%
\pgfsetdash{}{0pt}%
\pgfsys@defobject{currentmarker}{\pgfqpoint{-0.048611in}{0.000000in}}{\pgfqpoint{-0.000000in}{0.000000in}}{%
\pgfpathmoveto{\pgfqpoint{-0.000000in}{0.000000in}}%
\pgfpathlineto{\pgfqpoint{-0.048611in}{0.000000in}}%
\pgfusepath{stroke,fill}%
}%
\begin{pgfscope}%
\pgfsys@transformshift{8.609492in}{1.314483in}%
\pgfsys@useobject{currentmarker}{}%
\end{pgfscope}%
\end{pgfscope}%
\begin{pgfscope}%
\pgfpathrectangle{\pgfqpoint{8.609492in}{0.670138in}}{\pgfqpoint{3.067317in}{4.846490in}}%
\pgfusepath{clip}%
\pgfsetrectcap%
\pgfsetroundjoin%
\pgfsetlinewidth{0.803000pt}%
\definecolor{currentstroke}{rgb}{0.690196,0.690196,0.690196}%
\pgfsetstrokecolor{currentstroke}%
\pgfsetdash{}{0pt}%
\pgfpathmoveto{\pgfqpoint{8.609492in}{2.365019in}}%
\pgfpathlineto{\pgfqpoint{11.676809in}{2.365019in}}%
\pgfusepath{stroke}%
\end{pgfscope}%
\begin{pgfscope}%
\pgfsetbuttcap%
\pgfsetroundjoin%
\definecolor{currentfill}{rgb}{0.000000,0.000000,0.000000}%
\pgfsetfillcolor{currentfill}%
\pgfsetlinewidth{0.803000pt}%
\definecolor{currentstroke}{rgb}{0.000000,0.000000,0.000000}%
\pgfsetstrokecolor{currentstroke}%
\pgfsetdash{}{0pt}%
\pgfsys@defobject{currentmarker}{\pgfqpoint{-0.048611in}{0.000000in}}{\pgfqpoint{-0.000000in}{0.000000in}}{%
\pgfpathmoveto{\pgfqpoint{-0.000000in}{0.000000in}}%
\pgfpathlineto{\pgfqpoint{-0.048611in}{0.000000in}}%
\pgfusepath{stroke,fill}%
}%
\begin{pgfscope}%
\pgfsys@transformshift{8.609492in}{2.365019in}%
\pgfsys@useobject{currentmarker}{}%
\end{pgfscope}%
\end{pgfscope}%
\begin{pgfscope}%
\pgfpathrectangle{\pgfqpoint{8.609492in}{0.670138in}}{\pgfqpoint{3.067317in}{4.846490in}}%
\pgfusepath{clip}%
\pgfsetrectcap%
\pgfsetroundjoin%
\pgfsetlinewidth{0.803000pt}%
\definecolor{currentstroke}{rgb}{0.690196,0.690196,0.690196}%
\pgfsetstrokecolor{currentstroke}%
\pgfsetdash{}{0pt}%
\pgfpathmoveto{\pgfqpoint{8.609492in}{3.415556in}}%
\pgfpathlineto{\pgfqpoint{11.676809in}{3.415556in}}%
\pgfusepath{stroke}%
\end{pgfscope}%
\begin{pgfscope}%
\pgfsetbuttcap%
\pgfsetroundjoin%
\definecolor{currentfill}{rgb}{0.000000,0.000000,0.000000}%
\pgfsetfillcolor{currentfill}%
\pgfsetlinewidth{0.803000pt}%
\definecolor{currentstroke}{rgb}{0.000000,0.000000,0.000000}%
\pgfsetstrokecolor{currentstroke}%
\pgfsetdash{}{0pt}%
\pgfsys@defobject{currentmarker}{\pgfqpoint{-0.048611in}{0.000000in}}{\pgfqpoint{-0.000000in}{0.000000in}}{%
\pgfpathmoveto{\pgfqpoint{-0.000000in}{0.000000in}}%
\pgfpathlineto{\pgfqpoint{-0.048611in}{0.000000in}}%
\pgfusepath{stroke,fill}%
}%
\begin{pgfscope}%
\pgfsys@transformshift{8.609492in}{3.415556in}%
\pgfsys@useobject{currentmarker}{}%
\end{pgfscope}%
\end{pgfscope}%
\begin{pgfscope}%
\pgfpathrectangle{\pgfqpoint{8.609492in}{0.670138in}}{\pgfqpoint{3.067317in}{4.846490in}}%
\pgfusepath{clip}%
\pgfsetrectcap%
\pgfsetroundjoin%
\pgfsetlinewidth{0.803000pt}%
\definecolor{currentstroke}{rgb}{0.690196,0.690196,0.690196}%
\pgfsetstrokecolor{currentstroke}%
\pgfsetdash{}{0pt}%
\pgfpathmoveto{\pgfqpoint{8.609492in}{4.466092in}}%
\pgfpathlineto{\pgfqpoint{11.676809in}{4.466092in}}%
\pgfusepath{stroke}%
\end{pgfscope}%
\begin{pgfscope}%
\pgfsetbuttcap%
\pgfsetroundjoin%
\definecolor{currentfill}{rgb}{0.000000,0.000000,0.000000}%
\pgfsetfillcolor{currentfill}%
\pgfsetlinewidth{0.803000pt}%
\definecolor{currentstroke}{rgb}{0.000000,0.000000,0.000000}%
\pgfsetstrokecolor{currentstroke}%
\pgfsetdash{}{0pt}%
\pgfsys@defobject{currentmarker}{\pgfqpoint{-0.048611in}{0.000000in}}{\pgfqpoint{-0.000000in}{0.000000in}}{%
\pgfpathmoveto{\pgfqpoint{-0.000000in}{0.000000in}}%
\pgfpathlineto{\pgfqpoint{-0.048611in}{0.000000in}}%
\pgfusepath{stroke,fill}%
}%
\begin{pgfscope}%
\pgfsys@transformshift{8.609492in}{4.466092in}%
\pgfsys@useobject{currentmarker}{}%
\end{pgfscope}%
\end{pgfscope}%
\begin{pgfscope}%
\pgfpathrectangle{\pgfqpoint{8.609492in}{0.670138in}}{\pgfqpoint{3.067317in}{4.846490in}}%
\pgfusepath{clip}%
\pgfsetrectcap%
\pgfsetroundjoin%
\pgfsetlinewidth{0.803000pt}%
\definecolor{currentstroke}{rgb}{0.690196,0.690196,0.690196}%
\pgfsetstrokecolor{currentstroke}%
\pgfsetdash{}{0pt}%
\pgfpathmoveto{\pgfqpoint{8.609492in}{5.516628in}}%
\pgfpathlineto{\pgfqpoint{11.676809in}{5.516628in}}%
\pgfusepath{stroke}%
\end{pgfscope}%
\begin{pgfscope}%
\pgfsetbuttcap%
\pgfsetroundjoin%
\definecolor{currentfill}{rgb}{0.000000,0.000000,0.000000}%
\pgfsetfillcolor{currentfill}%
\pgfsetlinewidth{0.803000pt}%
\definecolor{currentstroke}{rgb}{0.000000,0.000000,0.000000}%
\pgfsetstrokecolor{currentstroke}%
\pgfsetdash{}{0pt}%
\pgfsys@defobject{currentmarker}{\pgfqpoint{-0.048611in}{0.000000in}}{\pgfqpoint{-0.000000in}{0.000000in}}{%
\pgfpathmoveto{\pgfqpoint{-0.000000in}{0.000000in}}%
\pgfpathlineto{\pgfqpoint{-0.048611in}{0.000000in}}%
\pgfusepath{stroke,fill}%
}%
\begin{pgfscope}%
\pgfsys@transformshift{8.609492in}{5.516628in}%
\pgfsys@useobject{currentmarker}{}%
\end{pgfscope}%
\end{pgfscope}%
\begin{pgfscope}%
\pgfsetbuttcap%
\pgfsetroundjoin%
\definecolor{currentfill}{rgb}{0.000000,0.000000,0.000000}%
\pgfsetfillcolor{currentfill}%
\pgfsetlinewidth{0.602250pt}%
\definecolor{currentstroke}{rgb}{0.000000,0.000000,0.000000}%
\pgfsetstrokecolor{currentstroke}%
\pgfsetdash{}{0pt}%
\pgfsys@defobject{currentmarker}{\pgfqpoint{-0.027778in}{0.000000in}}{\pgfqpoint{-0.000000in}{0.000000in}}{%
\pgfpathmoveto{\pgfqpoint{-0.000000in}{0.000000in}}%
\pgfpathlineto{\pgfqpoint{-0.027778in}{0.000000in}}%
\pgfusepath{stroke,fill}%
}%
\begin{pgfscope}%
\pgfsys@transformshift{8.609492in}{0.684161in}%
\pgfsys@useobject{currentmarker}{}%
\end{pgfscope}%
\end{pgfscope}%
\begin{pgfscope}%
\pgfsetbuttcap%
\pgfsetroundjoin%
\definecolor{currentfill}{rgb}{0.000000,0.000000,0.000000}%
\pgfsetfillcolor{currentfill}%
\pgfsetlinewidth{0.602250pt}%
\definecolor{currentstroke}{rgb}{0.000000,0.000000,0.000000}%
\pgfsetstrokecolor{currentstroke}%
\pgfsetdash{}{0pt}%
\pgfsys@defobject{currentmarker}{\pgfqpoint{-0.027778in}{0.000000in}}{\pgfqpoint{-0.000000in}{0.000000in}}{%
\pgfpathmoveto{\pgfqpoint{-0.000000in}{0.000000in}}%
\pgfpathlineto{\pgfqpoint{-0.027778in}{0.000000in}}%
\pgfusepath{stroke,fill}%
}%
\begin{pgfscope}%
\pgfsys@transformshift{8.609492in}{0.894269in}%
\pgfsys@useobject{currentmarker}{}%
\end{pgfscope}%
\end{pgfscope}%
\begin{pgfscope}%
\pgfsetbuttcap%
\pgfsetroundjoin%
\definecolor{currentfill}{rgb}{0.000000,0.000000,0.000000}%
\pgfsetfillcolor{currentfill}%
\pgfsetlinewidth{0.602250pt}%
\definecolor{currentstroke}{rgb}{0.000000,0.000000,0.000000}%
\pgfsetstrokecolor{currentstroke}%
\pgfsetdash{}{0pt}%
\pgfsys@defobject{currentmarker}{\pgfqpoint{-0.027778in}{0.000000in}}{\pgfqpoint{-0.000000in}{0.000000in}}{%
\pgfpathmoveto{\pgfqpoint{-0.000000in}{0.000000in}}%
\pgfpathlineto{\pgfqpoint{-0.027778in}{0.000000in}}%
\pgfusepath{stroke,fill}%
}%
\begin{pgfscope}%
\pgfsys@transformshift{8.609492in}{1.104376in}%
\pgfsys@useobject{currentmarker}{}%
\end{pgfscope}%
\end{pgfscope}%
\begin{pgfscope}%
\pgfsetbuttcap%
\pgfsetroundjoin%
\definecolor{currentfill}{rgb}{0.000000,0.000000,0.000000}%
\pgfsetfillcolor{currentfill}%
\pgfsetlinewidth{0.602250pt}%
\definecolor{currentstroke}{rgb}{0.000000,0.000000,0.000000}%
\pgfsetstrokecolor{currentstroke}%
\pgfsetdash{}{0pt}%
\pgfsys@defobject{currentmarker}{\pgfqpoint{-0.027778in}{0.000000in}}{\pgfqpoint{-0.000000in}{0.000000in}}{%
\pgfpathmoveto{\pgfqpoint{-0.000000in}{0.000000in}}%
\pgfpathlineto{\pgfqpoint{-0.027778in}{0.000000in}}%
\pgfusepath{stroke,fill}%
}%
\begin{pgfscope}%
\pgfsys@transformshift{8.609492in}{1.524590in}%
\pgfsys@useobject{currentmarker}{}%
\end{pgfscope}%
\end{pgfscope}%
\begin{pgfscope}%
\pgfsetbuttcap%
\pgfsetroundjoin%
\definecolor{currentfill}{rgb}{0.000000,0.000000,0.000000}%
\pgfsetfillcolor{currentfill}%
\pgfsetlinewidth{0.602250pt}%
\definecolor{currentstroke}{rgb}{0.000000,0.000000,0.000000}%
\pgfsetstrokecolor{currentstroke}%
\pgfsetdash{}{0pt}%
\pgfsys@defobject{currentmarker}{\pgfqpoint{-0.027778in}{0.000000in}}{\pgfqpoint{-0.000000in}{0.000000in}}{%
\pgfpathmoveto{\pgfqpoint{-0.000000in}{0.000000in}}%
\pgfpathlineto{\pgfqpoint{-0.027778in}{0.000000in}}%
\pgfusepath{stroke,fill}%
}%
\begin{pgfscope}%
\pgfsys@transformshift{8.609492in}{1.734698in}%
\pgfsys@useobject{currentmarker}{}%
\end{pgfscope}%
\end{pgfscope}%
\begin{pgfscope}%
\pgfsetbuttcap%
\pgfsetroundjoin%
\definecolor{currentfill}{rgb}{0.000000,0.000000,0.000000}%
\pgfsetfillcolor{currentfill}%
\pgfsetlinewidth{0.602250pt}%
\definecolor{currentstroke}{rgb}{0.000000,0.000000,0.000000}%
\pgfsetstrokecolor{currentstroke}%
\pgfsetdash{}{0pt}%
\pgfsys@defobject{currentmarker}{\pgfqpoint{-0.027778in}{0.000000in}}{\pgfqpoint{-0.000000in}{0.000000in}}{%
\pgfpathmoveto{\pgfqpoint{-0.000000in}{0.000000in}}%
\pgfpathlineto{\pgfqpoint{-0.027778in}{0.000000in}}%
\pgfusepath{stroke,fill}%
}%
\begin{pgfscope}%
\pgfsys@transformshift{8.609492in}{1.944805in}%
\pgfsys@useobject{currentmarker}{}%
\end{pgfscope}%
\end{pgfscope}%
\begin{pgfscope}%
\pgfsetbuttcap%
\pgfsetroundjoin%
\definecolor{currentfill}{rgb}{0.000000,0.000000,0.000000}%
\pgfsetfillcolor{currentfill}%
\pgfsetlinewidth{0.602250pt}%
\definecolor{currentstroke}{rgb}{0.000000,0.000000,0.000000}%
\pgfsetstrokecolor{currentstroke}%
\pgfsetdash{}{0pt}%
\pgfsys@defobject{currentmarker}{\pgfqpoint{-0.027778in}{0.000000in}}{\pgfqpoint{-0.000000in}{0.000000in}}{%
\pgfpathmoveto{\pgfqpoint{-0.000000in}{0.000000in}}%
\pgfpathlineto{\pgfqpoint{-0.027778in}{0.000000in}}%
\pgfusepath{stroke,fill}%
}%
\begin{pgfscope}%
\pgfsys@transformshift{8.609492in}{2.154912in}%
\pgfsys@useobject{currentmarker}{}%
\end{pgfscope}%
\end{pgfscope}%
\begin{pgfscope}%
\pgfsetbuttcap%
\pgfsetroundjoin%
\definecolor{currentfill}{rgb}{0.000000,0.000000,0.000000}%
\pgfsetfillcolor{currentfill}%
\pgfsetlinewidth{0.602250pt}%
\definecolor{currentstroke}{rgb}{0.000000,0.000000,0.000000}%
\pgfsetstrokecolor{currentstroke}%
\pgfsetdash{}{0pt}%
\pgfsys@defobject{currentmarker}{\pgfqpoint{-0.027778in}{0.000000in}}{\pgfqpoint{-0.000000in}{0.000000in}}{%
\pgfpathmoveto{\pgfqpoint{-0.000000in}{0.000000in}}%
\pgfpathlineto{\pgfqpoint{-0.027778in}{0.000000in}}%
\pgfusepath{stroke,fill}%
}%
\begin{pgfscope}%
\pgfsys@transformshift{8.609492in}{2.575127in}%
\pgfsys@useobject{currentmarker}{}%
\end{pgfscope}%
\end{pgfscope}%
\begin{pgfscope}%
\pgfsetbuttcap%
\pgfsetroundjoin%
\definecolor{currentfill}{rgb}{0.000000,0.000000,0.000000}%
\pgfsetfillcolor{currentfill}%
\pgfsetlinewidth{0.602250pt}%
\definecolor{currentstroke}{rgb}{0.000000,0.000000,0.000000}%
\pgfsetstrokecolor{currentstroke}%
\pgfsetdash{}{0pt}%
\pgfsys@defobject{currentmarker}{\pgfqpoint{-0.027778in}{0.000000in}}{\pgfqpoint{-0.000000in}{0.000000in}}{%
\pgfpathmoveto{\pgfqpoint{-0.000000in}{0.000000in}}%
\pgfpathlineto{\pgfqpoint{-0.027778in}{0.000000in}}%
\pgfusepath{stroke,fill}%
}%
\begin{pgfscope}%
\pgfsys@transformshift{8.609492in}{2.785234in}%
\pgfsys@useobject{currentmarker}{}%
\end{pgfscope}%
\end{pgfscope}%
\begin{pgfscope}%
\pgfsetbuttcap%
\pgfsetroundjoin%
\definecolor{currentfill}{rgb}{0.000000,0.000000,0.000000}%
\pgfsetfillcolor{currentfill}%
\pgfsetlinewidth{0.602250pt}%
\definecolor{currentstroke}{rgb}{0.000000,0.000000,0.000000}%
\pgfsetstrokecolor{currentstroke}%
\pgfsetdash{}{0pt}%
\pgfsys@defobject{currentmarker}{\pgfqpoint{-0.027778in}{0.000000in}}{\pgfqpoint{-0.000000in}{0.000000in}}{%
\pgfpathmoveto{\pgfqpoint{-0.000000in}{0.000000in}}%
\pgfpathlineto{\pgfqpoint{-0.027778in}{0.000000in}}%
\pgfusepath{stroke,fill}%
}%
\begin{pgfscope}%
\pgfsys@transformshift{8.609492in}{2.995341in}%
\pgfsys@useobject{currentmarker}{}%
\end{pgfscope}%
\end{pgfscope}%
\begin{pgfscope}%
\pgfsetbuttcap%
\pgfsetroundjoin%
\definecolor{currentfill}{rgb}{0.000000,0.000000,0.000000}%
\pgfsetfillcolor{currentfill}%
\pgfsetlinewidth{0.602250pt}%
\definecolor{currentstroke}{rgb}{0.000000,0.000000,0.000000}%
\pgfsetstrokecolor{currentstroke}%
\pgfsetdash{}{0pt}%
\pgfsys@defobject{currentmarker}{\pgfqpoint{-0.027778in}{0.000000in}}{\pgfqpoint{-0.000000in}{0.000000in}}{%
\pgfpathmoveto{\pgfqpoint{-0.000000in}{0.000000in}}%
\pgfpathlineto{\pgfqpoint{-0.027778in}{0.000000in}}%
\pgfusepath{stroke,fill}%
}%
\begin{pgfscope}%
\pgfsys@transformshift{8.609492in}{3.205448in}%
\pgfsys@useobject{currentmarker}{}%
\end{pgfscope}%
\end{pgfscope}%
\begin{pgfscope}%
\pgfsetbuttcap%
\pgfsetroundjoin%
\definecolor{currentfill}{rgb}{0.000000,0.000000,0.000000}%
\pgfsetfillcolor{currentfill}%
\pgfsetlinewidth{0.602250pt}%
\definecolor{currentstroke}{rgb}{0.000000,0.000000,0.000000}%
\pgfsetstrokecolor{currentstroke}%
\pgfsetdash{}{0pt}%
\pgfsys@defobject{currentmarker}{\pgfqpoint{-0.027778in}{0.000000in}}{\pgfqpoint{-0.000000in}{0.000000in}}{%
\pgfpathmoveto{\pgfqpoint{-0.000000in}{0.000000in}}%
\pgfpathlineto{\pgfqpoint{-0.027778in}{0.000000in}}%
\pgfusepath{stroke,fill}%
}%
\begin{pgfscope}%
\pgfsys@transformshift{8.609492in}{3.625663in}%
\pgfsys@useobject{currentmarker}{}%
\end{pgfscope}%
\end{pgfscope}%
\begin{pgfscope}%
\pgfsetbuttcap%
\pgfsetroundjoin%
\definecolor{currentfill}{rgb}{0.000000,0.000000,0.000000}%
\pgfsetfillcolor{currentfill}%
\pgfsetlinewidth{0.602250pt}%
\definecolor{currentstroke}{rgb}{0.000000,0.000000,0.000000}%
\pgfsetstrokecolor{currentstroke}%
\pgfsetdash{}{0pt}%
\pgfsys@defobject{currentmarker}{\pgfqpoint{-0.027778in}{0.000000in}}{\pgfqpoint{-0.000000in}{0.000000in}}{%
\pgfpathmoveto{\pgfqpoint{-0.000000in}{0.000000in}}%
\pgfpathlineto{\pgfqpoint{-0.027778in}{0.000000in}}%
\pgfusepath{stroke,fill}%
}%
\begin{pgfscope}%
\pgfsys@transformshift{8.609492in}{3.835770in}%
\pgfsys@useobject{currentmarker}{}%
\end{pgfscope}%
\end{pgfscope}%
\begin{pgfscope}%
\pgfsetbuttcap%
\pgfsetroundjoin%
\definecolor{currentfill}{rgb}{0.000000,0.000000,0.000000}%
\pgfsetfillcolor{currentfill}%
\pgfsetlinewidth{0.602250pt}%
\definecolor{currentstroke}{rgb}{0.000000,0.000000,0.000000}%
\pgfsetstrokecolor{currentstroke}%
\pgfsetdash{}{0pt}%
\pgfsys@defobject{currentmarker}{\pgfqpoint{-0.027778in}{0.000000in}}{\pgfqpoint{-0.000000in}{0.000000in}}{%
\pgfpathmoveto{\pgfqpoint{-0.000000in}{0.000000in}}%
\pgfpathlineto{\pgfqpoint{-0.027778in}{0.000000in}}%
\pgfusepath{stroke,fill}%
}%
\begin{pgfscope}%
\pgfsys@transformshift{8.609492in}{4.045877in}%
\pgfsys@useobject{currentmarker}{}%
\end{pgfscope}%
\end{pgfscope}%
\begin{pgfscope}%
\pgfsetbuttcap%
\pgfsetroundjoin%
\definecolor{currentfill}{rgb}{0.000000,0.000000,0.000000}%
\pgfsetfillcolor{currentfill}%
\pgfsetlinewidth{0.602250pt}%
\definecolor{currentstroke}{rgb}{0.000000,0.000000,0.000000}%
\pgfsetstrokecolor{currentstroke}%
\pgfsetdash{}{0pt}%
\pgfsys@defobject{currentmarker}{\pgfqpoint{-0.027778in}{0.000000in}}{\pgfqpoint{-0.000000in}{0.000000in}}{%
\pgfpathmoveto{\pgfqpoint{-0.000000in}{0.000000in}}%
\pgfpathlineto{\pgfqpoint{-0.027778in}{0.000000in}}%
\pgfusepath{stroke,fill}%
}%
\begin{pgfscope}%
\pgfsys@transformshift{8.609492in}{4.255985in}%
\pgfsys@useobject{currentmarker}{}%
\end{pgfscope}%
\end{pgfscope}%
\begin{pgfscope}%
\pgfsetbuttcap%
\pgfsetroundjoin%
\definecolor{currentfill}{rgb}{0.000000,0.000000,0.000000}%
\pgfsetfillcolor{currentfill}%
\pgfsetlinewidth{0.602250pt}%
\definecolor{currentstroke}{rgb}{0.000000,0.000000,0.000000}%
\pgfsetstrokecolor{currentstroke}%
\pgfsetdash{}{0pt}%
\pgfsys@defobject{currentmarker}{\pgfqpoint{-0.027778in}{0.000000in}}{\pgfqpoint{-0.000000in}{0.000000in}}{%
\pgfpathmoveto{\pgfqpoint{-0.000000in}{0.000000in}}%
\pgfpathlineto{\pgfqpoint{-0.027778in}{0.000000in}}%
\pgfusepath{stroke,fill}%
}%
\begin{pgfscope}%
\pgfsys@transformshift{8.609492in}{4.676199in}%
\pgfsys@useobject{currentmarker}{}%
\end{pgfscope}%
\end{pgfscope}%
\begin{pgfscope}%
\pgfsetbuttcap%
\pgfsetroundjoin%
\definecolor{currentfill}{rgb}{0.000000,0.000000,0.000000}%
\pgfsetfillcolor{currentfill}%
\pgfsetlinewidth{0.602250pt}%
\definecolor{currentstroke}{rgb}{0.000000,0.000000,0.000000}%
\pgfsetstrokecolor{currentstroke}%
\pgfsetdash{}{0pt}%
\pgfsys@defobject{currentmarker}{\pgfqpoint{-0.027778in}{0.000000in}}{\pgfqpoint{-0.000000in}{0.000000in}}{%
\pgfpathmoveto{\pgfqpoint{-0.000000in}{0.000000in}}%
\pgfpathlineto{\pgfqpoint{-0.027778in}{0.000000in}}%
\pgfusepath{stroke,fill}%
}%
\begin{pgfscope}%
\pgfsys@transformshift{8.609492in}{4.886306in}%
\pgfsys@useobject{currentmarker}{}%
\end{pgfscope}%
\end{pgfscope}%
\begin{pgfscope}%
\pgfsetbuttcap%
\pgfsetroundjoin%
\definecolor{currentfill}{rgb}{0.000000,0.000000,0.000000}%
\pgfsetfillcolor{currentfill}%
\pgfsetlinewidth{0.602250pt}%
\definecolor{currentstroke}{rgb}{0.000000,0.000000,0.000000}%
\pgfsetstrokecolor{currentstroke}%
\pgfsetdash{}{0pt}%
\pgfsys@defobject{currentmarker}{\pgfqpoint{-0.027778in}{0.000000in}}{\pgfqpoint{-0.000000in}{0.000000in}}{%
\pgfpathmoveto{\pgfqpoint{-0.000000in}{0.000000in}}%
\pgfpathlineto{\pgfqpoint{-0.027778in}{0.000000in}}%
\pgfusepath{stroke,fill}%
}%
\begin{pgfscope}%
\pgfsys@transformshift{8.609492in}{5.096414in}%
\pgfsys@useobject{currentmarker}{}%
\end{pgfscope}%
\end{pgfscope}%
\begin{pgfscope}%
\pgfsetbuttcap%
\pgfsetroundjoin%
\definecolor{currentfill}{rgb}{0.000000,0.000000,0.000000}%
\pgfsetfillcolor{currentfill}%
\pgfsetlinewidth{0.602250pt}%
\definecolor{currentstroke}{rgb}{0.000000,0.000000,0.000000}%
\pgfsetstrokecolor{currentstroke}%
\pgfsetdash{}{0pt}%
\pgfsys@defobject{currentmarker}{\pgfqpoint{-0.027778in}{0.000000in}}{\pgfqpoint{-0.000000in}{0.000000in}}{%
\pgfpathmoveto{\pgfqpoint{-0.000000in}{0.000000in}}%
\pgfpathlineto{\pgfqpoint{-0.027778in}{0.000000in}}%
\pgfusepath{stroke,fill}%
}%
\begin{pgfscope}%
\pgfsys@transformshift{8.609492in}{5.306521in}%
\pgfsys@useobject{currentmarker}{}%
\end{pgfscope}%
\end{pgfscope}%
\begin{pgfscope}%
\pgfpathrectangle{\pgfqpoint{8.609492in}{0.670138in}}{\pgfqpoint{3.067317in}{4.846490in}}%
\pgfusepath{clip}%
\pgfsetrectcap%
\pgfsetroundjoin%
\pgfsetlinewidth{1.505625pt}%
\definecolor{currentstroke}{rgb}{0.839216,0.152941,0.156863}%
\pgfsetstrokecolor{currentstroke}%
\pgfsetdash{}{0pt}%
\pgfpathmoveto{\pgfqpoint{8.639906in}{5.516628in}}%
\pgfpathlineto{\pgfqpoint{8.640591in}{5.452515in}}%
\pgfpathlineto{\pgfqpoint{8.642645in}{5.224722in}}%
\pgfpathlineto{\pgfqpoint{8.643330in}{5.195573in}}%
\pgfpathlineto{\pgfqpoint{8.643672in}{5.095553in}}%
\pgfpathlineto{\pgfqpoint{8.644699in}{5.074292in}}%
\pgfpathlineto{\pgfqpoint{8.645041in}{5.073804in}}%
\pgfpathlineto{\pgfqpoint{8.645384in}{5.072376in}}%
\pgfpathlineto{\pgfqpoint{8.645726in}{5.056253in}}%
\pgfpathlineto{\pgfqpoint{8.646069in}{5.055301in}}%
\pgfpathlineto{\pgfqpoint{8.646411in}{5.055190in}}%
\pgfpathlineto{\pgfqpoint{8.647438in}{4.995910in}}%
\pgfpathlineto{\pgfqpoint{8.648123in}{4.982463in}}%
\pgfpathlineto{\pgfqpoint{8.648807in}{4.971745in}}%
\pgfpathlineto{\pgfqpoint{8.649150in}{4.952413in}}%
\pgfpathlineto{\pgfqpoint{8.649492in}{4.951695in}}%
\pgfpathlineto{\pgfqpoint{8.650177in}{4.946237in}}%
\pgfpathlineto{\pgfqpoint{8.650861in}{4.943502in}}%
\pgfpathlineto{\pgfqpoint{8.651204in}{4.931759in}}%
\pgfpathlineto{\pgfqpoint{8.651546in}{4.931722in}}%
\pgfpathlineto{\pgfqpoint{8.652231in}{4.921753in}}%
\pgfpathlineto{\pgfqpoint{8.652573in}{4.919848in}}%
\pgfpathlineto{\pgfqpoint{8.653258in}{4.913145in}}%
\pgfpathlineto{\pgfqpoint{8.653600in}{4.912125in}}%
\pgfpathlineto{\pgfqpoint{8.653942in}{4.908236in}}%
\pgfpathlineto{\pgfqpoint{8.654285in}{4.907883in}}%
\pgfpathlineto{\pgfqpoint{8.654970in}{4.898646in}}%
\pgfpathlineto{\pgfqpoint{8.655312in}{4.897815in}}%
\pgfpathlineto{\pgfqpoint{8.655997in}{4.887167in}}%
\pgfpathlineto{\pgfqpoint{8.657024in}{4.878596in}}%
\pgfpathlineto{\pgfqpoint{8.657366in}{4.878445in}}%
\pgfpathlineto{\pgfqpoint{8.658051in}{4.867609in}}%
\pgfpathlineto{\pgfqpoint{8.658735in}{4.865268in}}%
\pgfpathlineto{\pgfqpoint{8.659420in}{4.864437in}}%
\pgfpathlineto{\pgfqpoint{8.659762in}{4.858547in}}%
\pgfpathlineto{\pgfqpoint{8.660105in}{4.858434in}}%
\pgfpathlineto{\pgfqpoint{8.660789in}{4.853903in}}%
\pgfpathlineto{\pgfqpoint{8.661474in}{4.852166in}}%
\pgfpathlineto{\pgfqpoint{8.661816in}{4.850127in}}%
\pgfpathlineto{\pgfqpoint{8.662159in}{4.850089in}}%
\pgfpathlineto{\pgfqpoint{8.662844in}{4.846430in}}%
\pgfpathlineto{\pgfqpoint{8.663871in}{4.837025in}}%
\pgfpathlineto{\pgfqpoint{8.664213in}{4.836156in}}%
\pgfpathlineto{\pgfqpoint{8.664898in}{4.817353in}}%
\pgfpathlineto{\pgfqpoint{8.665240in}{4.816900in}}%
\pgfpathlineto{\pgfqpoint{8.666267in}{4.811161in}}%
\pgfpathlineto{\pgfqpoint{8.666609in}{4.786807in}}%
\pgfpathlineto{\pgfqpoint{8.666952in}{4.784579in}}%
\pgfpathlineto{\pgfqpoint{8.668321in}{4.758979in}}%
\pgfpathlineto{\pgfqpoint{8.668663in}{4.752938in}}%
\pgfpathlineto{\pgfqpoint{8.669348in}{4.751579in}}%
\pgfpathlineto{\pgfqpoint{8.670033in}{4.737231in}}%
\pgfpathlineto{\pgfqpoint{8.671060in}{4.735230in}}%
\pgfpathlineto{\pgfqpoint{8.671402in}{4.735192in}}%
\pgfpathlineto{\pgfqpoint{8.672087in}{4.733040in}}%
\pgfpathlineto{\pgfqpoint{8.672429in}{4.729981in}}%
\pgfpathlineto{\pgfqpoint{8.672772in}{4.729755in}}%
\pgfpathlineto{\pgfqpoint{8.673456in}{4.719107in}}%
\pgfpathlineto{\pgfqpoint{8.673799in}{4.718277in}}%
\pgfpathlineto{\pgfqpoint{8.674141in}{4.715829in}}%
\pgfpathlineto{\pgfqpoint{8.674483in}{4.715558in}}%
\pgfpathlineto{\pgfqpoint{8.674826in}{4.712537in}}%
\pgfpathlineto{\pgfqpoint{8.675168in}{4.712462in}}%
\pgfpathlineto{\pgfqpoint{8.676537in}{4.697887in}}%
\pgfpathlineto{\pgfqpoint{8.676880in}{4.696415in}}%
\pgfpathlineto{\pgfqpoint{8.677564in}{4.689392in}}%
\pgfpathlineto{\pgfqpoint{8.677907in}{4.689052in}}%
\pgfpathlineto{\pgfqpoint{8.678591in}{4.687428in}}%
\pgfpathlineto{\pgfqpoint{8.679961in}{4.674138in}}%
\pgfpathlineto{\pgfqpoint{8.682015in}{4.670966in}}%
\pgfpathlineto{\pgfqpoint{8.682357in}{4.669191in}}%
\pgfpathlineto{\pgfqpoint{8.682700in}{4.669154in}}%
\pgfpathlineto{\pgfqpoint{8.683384in}{4.668398in}}%
\pgfpathlineto{\pgfqpoint{8.684069in}{4.661690in}}%
\pgfpathlineto{\pgfqpoint{8.684411in}{4.661375in}}%
\pgfpathlineto{\pgfqpoint{8.685781in}{4.653862in}}%
\pgfpathlineto{\pgfqpoint{8.686123in}{4.653409in}}%
\pgfpathlineto{\pgfqpoint{8.686465in}{4.652314in}}%
\pgfpathlineto{\pgfqpoint{8.687492in}{4.647500in}}%
\pgfpathlineto{\pgfqpoint{8.687835in}{4.647405in}}%
\pgfpathlineto{\pgfqpoint{8.688177in}{4.646650in}}%
\pgfpathlineto{\pgfqpoint{8.688520in}{4.642761in}}%
\pgfpathlineto{\pgfqpoint{8.688862in}{4.642081in}}%
\pgfpathlineto{\pgfqpoint{8.691258in}{4.630641in}}%
\pgfpathlineto{\pgfqpoint{8.691601in}{4.627998in}}%
\pgfpathlineto{\pgfqpoint{8.691943in}{4.627801in}}%
\pgfpathlineto{\pgfqpoint{8.692285in}{4.627129in}}%
\pgfpathlineto{\pgfqpoint{8.692970in}{4.619502in}}%
\pgfpathlineto{\pgfqpoint{8.693655in}{4.616519in}}%
\pgfpathlineto{\pgfqpoint{8.694339in}{4.614073in}}%
\pgfpathlineto{\pgfqpoint{8.695366in}{4.610402in}}%
\pgfpathlineto{\pgfqpoint{8.695709in}{4.609043in}}%
\pgfpathlineto{\pgfqpoint{8.696051in}{4.609005in}}%
\pgfpathlineto{\pgfqpoint{8.697078in}{4.601492in}}%
\pgfpathlineto{\pgfqpoint{8.698105in}{4.600321in}}%
\pgfpathlineto{\pgfqpoint{8.699132in}{4.595866in}}%
\pgfpathlineto{\pgfqpoint{8.699475in}{4.595375in}}%
\pgfpathlineto{\pgfqpoint{8.700159in}{4.592467in}}%
\pgfpathlineto{\pgfqpoint{8.700844in}{4.591259in}}%
\pgfpathlineto{\pgfqpoint{8.701529in}{4.590693in}}%
\pgfpathlineto{\pgfqpoint{8.701871in}{4.590391in}}%
\pgfpathlineto{\pgfqpoint{8.702898in}{4.587257in}}%
\pgfpathlineto{\pgfqpoint{8.703240in}{4.583594in}}%
\pgfpathlineto{\pgfqpoint{8.703925in}{4.574155in}}%
\pgfpathlineto{\pgfqpoint{8.704610in}{4.573400in}}%
\pgfpathlineto{\pgfqpoint{8.704952in}{4.572964in}}%
\pgfpathlineto{\pgfqpoint{8.705637in}{4.571210in}}%
\pgfpathlineto{\pgfqpoint{8.706322in}{4.568982in}}%
\pgfpathlineto{\pgfqpoint{8.707006in}{4.565471in}}%
\pgfpathlineto{\pgfqpoint{8.707349in}{4.565093in}}%
\pgfpathlineto{\pgfqpoint{8.708033in}{4.561884in}}%
\pgfpathlineto{\pgfqpoint{8.708718in}{4.561204in}}%
\pgfpathlineto{\pgfqpoint{8.709745in}{4.557957in}}%
\pgfpathlineto{\pgfqpoint{8.710772in}{4.553615in}}%
\pgfpathlineto{\pgfqpoint{8.711114in}{4.546176in}}%
\pgfpathlineto{\pgfqpoint{8.712484in}{4.542929in}}%
\pgfpathlineto{\pgfqpoint{8.713511in}{4.534887in}}%
\pgfpathlineto{\pgfqpoint{8.714538in}{4.531904in}}%
\pgfpathlineto{\pgfqpoint{8.714880in}{4.531230in}}%
\pgfpathlineto{\pgfqpoint{8.716250in}{4.525825in}}%
\pgfpathlineto{\pgfqpoint{8.716592in}{4.525477in}}%
\pgfpathlineto{\pgfqpoint{8.716934in}{4.522427in}}%
\pgfpathlineto{\pgfqpoint{8.717961in}{4.520916in}}%
\pgfpathlineto{\pgfqpoint{8.718304in}{4.519467in}}%
\pgfpathlineto{\pgfqpoint{8.718988in}{4.513214in}}%
\pgfpathlineto{\pgfqpoint{8.719673in}{4.512723in}}%
\pgfpathlineto{\pgfqpoint{8.720015in}{4.510343in}}%
\pgfpathlineto{\pgfqpoint{8.720700in}{4.509778in}}%
\pgfpathlineto{\pgfqpoint{8.721042in}{4.506266in}}%
\pgfpathlineto{\pgfqpoint{8.721727in}{4.505700in}}%
\pgfpathlineto{\pgfqpoint{8.722412in}{4.502755in}}%
\pgfpathlineto{\pgfqpoint{8.723097in}{4.502303in}}%
\pgfpathlineto{\pgfqpoint{8.723781in}{4.500225in}}%
\pgfpathlineto{\pgfqpoint{8.724124in}{4.500074in}}%
\pgfpathlineto{\pgfqpoint{8.724808in}{4.498979in}}%
\pgfpathlineto{\pgfqpoint{8.725151in}{4.498903in}}%
\pgfpathlineto{\pgfqpoint{8.726862in}{4.493466in}}%
\pgfpathlineto{\pgfqpoint{8.727889in}{4.492409in}}%
\pgfpathlineto{\pgfqpoint{8.728574in}{4.490144in}}%
\pgfpathlineto{\pgfqpoint{8.728916in}{4.487954in}}%
\pgfpathlineto{\pgfqpoint{8.729601in}{4.487576in}}%
\pgfpathlineto{\pgfqpoint{8.730286in}{4.483763in}}%
\pgfpathlineto{\pgfqpoint{8.730628in}{4.483385in}}%
\pgfpathlineto{\pgfqpoint{8.731313in}{4.480591in}}%
\pgfpathlineto{\pgfqpoint{8.731655in}{4.479987in}}%
\pgfpathlineto{\pgfqpoint{8.731998in}{4.476046in}}%
\pgfpathlineto{\pgfqpoint{8.732340in}{4.475909in}}%
\pgfpathlineto{\pgfqpoint{8.733025in}{4.474436in}}%
\pgfpathlineto{\pgfqpoint{8.733367in}{4.474399in}}%
\pgfpathlineto{\pgfqpoint{8.736106in}{4.470019in}}%
\pgfpathlineto{\pgfqpoint{8.736448in}{4.466734in}}%
\pgfpathlineto{\pgfqpoint{8.737133in}{4.465563in}}%
\pgfpathlineto{\pgfqpoint{8.738160in}{4.459749in}}%
\pgfpathlineto{\pgfqpoint{8.738502in}{4.455167in}}%
\pgfpathlineto{\pgfqpoint{8.738844in}{4.455142in}}%
\pgfpathlineto{\pgfqpoint{8.740214in}{4.449705in}}%
\pgfpathlineto{\pgfqpoint{8.740899in}{4.448912in}}%
\pgfpathlineto{\pgfqpoint{8.741583in}{4.445740in}}%
\pgfpathlineto{\pgfqpoint{8.743295in}{4.441921in}}%
\pgfpathlineto{\pgfqpoint{8.743637in}{4.441776in}}%
\pgfpathlineto{\pgfqpoint{8.744322in}{4.440001in}}%
\pgfpathlineto{\pgfqpoint{8.744664in}{4.439812in}}%
\pgfpathlineto{\pgfqpoint{8.745007in}{4.439171in}}%
\pgfpathlineto{\pgfqpoint{8.745349in}{4.435115in}}%
\pgfpathlineto{\pgfqpoint{8.746034in}{4.434338in}}%
\pgfpathlineto{\pgfqpoint{8.746718in}{4.430297in}}%
\pgfpathlineto{\pgfqpoint{8.747061in}{4.430044in}}%
\pgfpathlineto{\pgfqpoint{8.748088in}{4.427390in}}%
\pgfpathlineto{\pgfqpoint{8.748773in}{4.424558in}}%
\pgfpathlineto{\pgfqpoint{8.749115in}{4.424407in}}%
\pgfpathlineto{\pgfqpoint{8.749457in}{4.421802in}}%
\pgfpathlineto{\pgfqpoint{8.750142in}{4.421387in}}%
\pgfpathlineto{\pgfqpoint{8.750827in}{4.415761in}}%
\pgfpathlineto{\pgfqpoint{8.751169in}{4.415572in}}%
\pgfpathlineto{\pgfqpoint{8.751511in}{4.414704in}}%
\pgfpathlineto{\pgfqpoint{8.752196in}{4.414439in}}%
\pgfpathlineto{\pgfqpoint{8.753908in}{4.406132in}}%
\pgfpathlineto{\pgfqpoint{8.754250in}{4.405149in}}%
\pgfpathlineto{\pgfqpoint{8.754592in}{4.405113in}}%
\pgfpathlineto{\pgfqpoint{8.755962in}{4.403489in}}%
\pgfpathlineto{\pgfqpoint{8.756304in}{4.403489in}}%
\pgfpathlineto{\pgfqpoint{8.758016in}{4.396731in}}%
\pgfpathlineto{\pgfqpoint{8.758701in}{4.392313in}}%
\pgfpathlineto{\pgfqpoint{8.759043in}{4.390501in}}%
\pgfpathlineto{\pgfqpoint{8.759385in}{4.390425in}}%
\pgfpathlineto{\pgfqpoint{8.760070in}{4.389066in}}%
\pgfpathlineto{\pgfqpoint{8.760755in}{4.385821in}}%
\pgfpathlineto{\pgfqpoint{8.764178in}{4.374982in}}%
\pgfpathlineto{\pgfqpoint{8.764863in}{4.372728in}}%
\pgfpathlineto{\pgfqpoint{8.765205in}{4.372150in}}%
\pgfpathlineto{\pgfqpoint{8.766232in}{4.367166in}}%
\pgfpathlineto{\pgfqpoint{8.767602in}{4.364070in}}%
\pgfpathlineto{\pgfqpoint{8.768286in}{4.363315in}}%
\pgfpathlineto{\pgfqpoint{8.769313in}{4.362938in}}%
\pgfpathlineto{\pgfqpoint{8.769998in}{4.360106in}}%
\pgfpathlineto{\pgfqpoint{8.770340in}{4.358034in}}%
\pgfpathlineto{\pgfqpoint{8.771367in}{4.357161in}}%
\pgfpathlineto{\pgfqpoint{8.772394in}{4.353611in}}%
\pgfpathlineto{\pgfqpoint{8.773421in}{4.352856in}}%
\pgfpathlineto{\pgfqpoint{8.773764in}{4.351233in}}%
\pgfpathlineto{\pgfqpoint{8.774791in}{4.350364in}}%
\pgfpathlineto{\pgfqpoint{8.775476in}{4.349571in}}%
\pgfpathlineto{\pgfqpoint{8.776503in}{4.348379in}}%
\pgfpathlineto{\pgfqpoint{8.776845in}{4.346437in}}%
\pgfpathlineto{\pgfqpoint{8.777530in}{4.346173in}}%
\pgfpathlineto{\pgfqpoint{8.778899in}{4.339688in}}%
\pgfpathlineto{\pgfqpoint{8.779241in}{4.338206in}}%
\pgfpathlineto{\pgfqpoint{8.779584in}{4.338173in}}%
\pgfpathlineto{\pgfqpoint{8.780953in}{4.334732in}}%
\pgfpathlineto{\pgfqpoint{8.781980in}{4.331170in}}%
\pgfpathlineto{\pgfqpoint{8.782665in}{4.330806in}}%
\pgfpathlineto{\pgfqpoint{8.783350in}{4.330315in}}%
\pgfpathlineto{\pgfqpoint{8.783692in}{4.330315in}}%
\pgfpathlineto{\pgfqpoint{8.785061in}{4.328238in}}%
\pgfpathlineto{\pgfqpoint{8.785746in}{4.324462in}}%
\pgfpathlineto{\pgfqpoint{8.786088in}{4.323558in}}%
\pgfpathlineto{\pgfqpoint{8.786431in}{4.321781in}}%
\pgfpathlineto{\pgfqpoint{8.786773in}{4.321622in}}%
\pgfpathlineto{\pgfqpoint{8.787458in}{4.318572in}}%
\pgfpathlineto{\pgfqpoint{8.788485in}{4.317326in}}%
\pgfpathlineto{\pgfqpoint{8.788827in}{4.315627in}}%
\pgfpathlineto{\pgfqpoint{8.789169in}{4.315438in}}%
\pgfpathlineto{\pgfqpoint{8.789512in}{4.314456in}}%
\pgfpathlineto{\pgfqpoint{8.789854in}{4.310756in}}%
\pgfpathlineto{\pgfqpoint{8.790539in}{4.310492in}}%
\pgfpathlineto{\pgfqpoint{8.791224in}{4.310190in}}%
\pgfpathlineto{\pgfqpoint{8.792251in}{4.307622in}}%
\pgfpathlineto{\pgfqpoint{8.792935in}{4.305659in}}%
\pgfpathlineto{\pgfqpoint{8.793278in}{4.305432in}}%
\pgfpathlineto{\pgfqpoint{8.793620in}{4.304224in}}%
\pgfpathlineto{\pgfqpoint{8.793962in}{4.299400in}}%
\pgfpathlineto{\pgfqpoint{8.794647in}{4.299165in}}%
\pgfpathlineto{\pgfqpoint{8.795332in}{4.298825in}}%
\pgfpathlineto{\pgfqpoint{8.796016in}{4.296786in}}%
\pgfpathlineto{\pgfqpoint{8.797043in}{4.296295in}}%
\pgfpathlineto{\pgfqpoint{8.797386in}{4.295794in}}%
\pgfpathlineto{\pgfqpoint{8.797728in}{4.294218in}}%
\pgfpathlineto{\pgfqpoint{8.798070in}{4.294029in}}%
\pgfpathlineto{\pgfqpoint{8.799097in}{4.289914in}}%
\pgfpathlineto{\pgfqpoint{8.799440in}{4.289876in}}%
\pgfpathlineto{\pgfqpoint{8.799782in}{4.288979in}}%
\pgfpathlineto{\pgfqpoint{8.800125in}{4.285874in}}%
\pgfpathlineto{\pgfqpoint{8.800467in}{4.285676in}}%
\pgfpathlineto{\pgfqpoint{8.800809in}{4.283117in}}%
\pgfpathlineto{\pgfqpoint{8.801152in}{4.282966in}}%
\pgfpathlineto{\pgfqpoint{8.801494in}{4.282359in}}%
\pgfpathlineto{\pgfqpoint{8.801836in}{4.280739in}}%
\pgfpathlineto{\pgfqpoint{8.802179in}{4.280550in}}%
\pgfpathlineto{\pgfqpoint{8.803206in}{4.277529in}}%
\pgfpathlineto{\pgfqpoint{8.803890in}{4.272599in}}%
\pgfpathlineto{\pgfqpoint{8.804233in}{4.272205in}}%
\pgfpathlineto{\pgfqpoint{8.805260in}{4.269638in}}%
\pgfpathlineto{\pgfqpoint{8.805602in}{4.269492in}}%
\pgfpathlineto{\pgfqpoint{8.806629in}{4.266731in}}%
\pgfpathlineto{\pgfqpoint{8.807999in}{4.266202in}}%
\pgfpathlineto{\pgfqpoint{8.808341in}{4.265673in}}%
\pgfpathlineto{\pgfqpoint{8.808683in}{4.264390in}}%
\pgfpathlineto{\pgfqpoint{8.809026in}{4.264276in}}%
\pgfpathlineto{\pgfqpoint{8.809710in}{4.261935in}}%
\pgfpathlineto{\pgfqpoint{8.810053in}{4.261633in}}%
\pgfpathlineto{\pgfqpoint{8.811764in}{4.256862in}}%
\pgfpathlineto{\pgfqpoint{8.812449in}{4.256574in}}%
\pgfpathlineto{\pgfqpoint{8.813818in}{4.255441in}}%
\pgfpathlineto{\pgfqpoint{8.814161in}{4.253677in}}%
\pgfpathlineto{\pgfqpoint{8.814503in}{4.253440in}}%
\pgfpathlineto{\pgfqpoint{8.815188in}{4.251741in}}%
\pgfpathlineto{\pgfqpoint{8.816900in}{4.250669in}}%
\pgfpathlineto{\pgfqpoint{8.817242in}{4.250117in}}%
\pgfpathlineto{\pgfqpoint{8.817584in}{4.245662in}}%
\pgfpathlineto{\pgfqpoint{8.818269in}{4.245281in}}%
\pgfpathlineto{\pgfqpoint{8.818611in}{4.245246in}}%
\pgfpathlineto{\pgfqpoint{8.820323in}{4.243396in}}%
\pgfpathlineto{\pgfqpoint{8.820665in}{4.240602in}}%
\pgfpathlineto{\pgfqpoint{8.823062in}{4.239092in}}%
\pgfpathlineto{\pgfqpoint{8.823746in}{4.239016in}}%
\pgfpathlineto{\pgfqpoint{8.824089in}{4.238563in}}%
\pgfpathlineto{\pgfqpoint{8.824431in}{4.237279in}}%
\pgfpathlineto{\pgfqpoint{8.824773in}{4.237253in}}%
\pgfpathlineto{\pgfqpoint{8.825801in}{4.235127in}}%
\pgfpathlineto{\pgfqpoint{8.826143in}{4.231842in}}%
\pgfpathlineto{\pgfqpoint{8.827170in}{4.231238in}}%
\pgfpathlineto{\pgfqpoint{8.827855in}{4.229161in}}%
\pgfpathlineto{\pgfqpoint{8.828197in}{4.229086in}}%
\pgfpathlineto{\pgfqpoint{8.828882in}{4.223385in}}%
\pgfpathlineto{\pgfqpoint{8.829224in}{4.223120in}}%
\pgfpathlineto{\pgfqpoint{8.830593in}{4.220517in}}%
\pgfpathlineto{\pgfqpoint{8.830936in}{4.220213in}}%
\pgfpathlineto{\pgfqpoint{8.832990in}{4.216173in}}%
\pgfpathlineto{\pgfqpoint{8.834017in}{4.215871in}}%
\pgfpathlineto{\pgfqpoint{8.835044in}{4.214285in}}%
\pgfpathlineto{\pgfqpoint{8.836071in}{4.213492in}}%
\pgfpathlineto{\pgfqpoint{8.837098in}{4.212737in}}%
\pgfpathlineto{\pgfqpoint{8.837440in}{4.211642in}}%
\pgfpathlineto{\pgfqpoint{8.838125in}{4.211453in}}%
\pgfpathlineto{\pgfqpoint{8.838810in}{4.209587in}}%
\pgfpathlineto{\pgfqpoint{8.839494in}{4.209339in}}%
\pgfpathlineto{\pgfqpoint{8.840179in}{4.206401in}}%
\pgfpathlineto{\pgfqpoint{8.840521in}{4.205940in}}%
\pgfpathlineto{\pgfqpoint{8.840864in}{4.204657in}}%
\pgfpathlineto{\pgfqpoint{8.841548in}{4.201409in}}%
\pgfpathlineto{\pgfqpoint{8.842233in}{4.200994in}}%
\pgfpathlineto{\pgfqpoint{8.843603in}{4.198578in}}%
\pgfpathlineto{\pgfqpoint{8.843945in}{4.198487in}}%
\pgfpathlineto{\pgfqpoint{8.844972in}{4.194840in}}%
\pgfpathlineto{\pgfqpoint{8.845314in}{4.193966in}}%
\pgfpathlineto{\pgfqpoint{8.845657in}{4.191550in}}%
\pgfpathlineto{\pgfqpoint{8.848395in}{4.189063in}}%
\pgfpathlineto{\pgfqpoint{8.848738in}{4.186118in}}%
\pgfpathlineto{\pgfqpoint{8.850449in}{4.185136in}}%
\pgfpathlineto{\pgfqpoint{8.851134in}{4.183890in}}%
\pgfpathlineto{\pgfqpoint{8.851819in}{4.183248in}}%
\pgfpathlineto{\pgfqpoint{8.852504in}{4.180378in}}%
\pgfpathlineto{\pgfqpoint{8.852846in}{4.180265in}}%
\pgfpathlineto{\pgfqpoint{8.853188in}{4.178931in}}%
\pgfpathlineto{\pgfqpoint{8.853531in}{4.178793in}}%
\pgfpathlineto{\pgfqpoint{8.853873in}{4.176944in}}%
\pgfpathlineto{\pgfqpoint{8.854215in}{4.176866in}}%
\pgfpathlineto{\pgfqpoint{8.854900in}{4.175810in}}%
\pgfpathlineto{\pgfqpoint{8.855585in}{4.174035in}}%
\pgfpathlineto{\pgfqpoint{8.855927in}{4.172978in}}%
\pgfpathlineto{\pgfqpoint{8.856612in}{4.172789in}}%
\pgfpathlineto{\pgfqpoint{8.857296in}{4.172147in}}%
\pgfpathlineto{\pgfqpoint{8.857981in}{4.171619in}}%
\pgfpathlineto{\pgfqpoint{8.858323in}{4.171430in}}%
\pgfpathlineto{\pgfqpoint{8.858666in}{4.170561in}}%
\pgfpathlineto{\pgfqpoint{8.859693in}{4.163803in}}%
\pgfpathlineto{\pgfqpoint{8.860720in}{4.163538in}}%
\pgfpathlineto{\pgfqpoint{8.861747in}{4.159951in}}%
\pgfpathlineto{\pgfqpoint{8.862089in}{4.159838in}}%
\pgfpathlineto{\pgfqpoint{8.862432in}{4.159423in}}%
\pgfpathlineto{\pgfqpoint{8.862774in}{4.158215in}}%
\pgfpathlineto{\pgfqpoint{8.863116in}{4.158101in}}%
\pgfpathlineto{\pgfqpoint{8.863801in}{4.155464in}}%
\pgfpathlineto{\pgfqpoint{8.864828in}{4.154363in}}%
\pgfpathlineto{\pgfqpoint{8.865513in}{4.152249in}}%
\pgfpathlineto{\pgfqpoint{8.866540in}{4.151041in}}%
\pgfpathlineto{\pgfqpoint{8.867567in}{4.147625in}}%
\pgfpathlineto{\pgfqpoint{8.867909in}{4.147151in}}%
\pgfpathlineto{\pgfqpoint{8.868252in}{4.145603in}}%
\pgfpathlineto{\pgfqpoint{8.868936in}{4.145377in}}%
\pgfpathlineto{\pgfqpoint{8.869621in}{4.144774in}}%
\pgfpathlineto{\pgfqpoint{8.869963in}{4.144395in}}%
\pgfpathlineto{\pgfqpoint{8.870306in}{4.142092in}}%
\pgfpathlineto{\pgfqpoint{8.871675in}{4.141261in}}%
\pgfpathlineto{\pgfqpoint{8.872017in}{4.139129in}}%
\pgfpathlineto{\pgfqpoint{8.872360in}{4.138883in}}%
\pgfpathlineto{\pgfqpoint{8.873044in}{4.135730in}}%
\pgfpathlineto{\pgfqpoint{8.875098in}{4.133294in}}%
\pgfpathlineto{\pgfqpoint{8.875783in}{4.130916in}}%
\pgfpathlineto{\pgfqpoint{8.876126in}{4.130878in}}%
\pgfpathlineto{\pgfqpoint{8.876468in}{4.130463in}}%
\pgfpathlineto{\pgfqpoint{8.876810in}{4.129368in}}%
\pgfpathlineto{\pgfqpoint{8.877153in}{4.129217in}}%
\pgfpathlineto{\pgfqpoint{8.878864in}{4.126271in}}%
\pgfpathlineto{\pgfqpoint{8.879891in}{4.125051in}}%
\pgfpathlineto{\pgfqpoint{8.881603in}{4.119971in}}%
\pgfpathlineto{\pgfqpoint{8.882630in}{4.119362in}}%
\pgfpathlineto{\pgfqpoint{8.882972in}{4.114906in}}%
\pgfpathlineto{\pgfqpoint{8.883657in}{4.113318in}}%
\pgfpathlineto{\pgfqpoint{8.884684in}{4.110866in}}%
\pgfpathlineto{\pgfqpoint{8.885369in}{4.107727in}}%
\pgfpathlineto{\pgfqpoint{8.885711in}{4.107581in}}%
\pgfpathlineto{\pgfqpoint{8.886396in}{4.105844in}}%
\pgfpathlineto{\pgfqpoint{8.886738in}{4.105693in}}%
\pgfpathlineto{\pgfqpoint{8.888450in}{4.102333in}}%
\pgfpathlineto{\pgfqpoint{8.889477in}{4.097689in}}%
\pgfpathlineto{\pgfqpoint{8.889819in}{4.096254in}}%
\pgfpathlineto{\pgfqpoint{8.890162in}{4.096178in}}%
\pgfpathlineto{\pgfqpoint{8.890504in}{4.095046in}}%
\pgfpathlineto{\pgfqpoint{8.891531in}{4.094706in}}%
\pgfpathlineto{\pgfqpoint{8.892216in}{4.091888in}}%
\pgfpathlineto{\pgfqpoint{8.892900in}{4.091836in}}%
\pgfpathlineto{\pgfqpoint{8.894612in}{4.089760in}}%
\pgfpathlineto{\pgfqpoint{8.895982in}{4.083605in}}%
\pgfpathlineto{\pgfqpoint{8.896324in}{4.083265in}}%
\pgfpathlineto{\pgfqpoint{8.897009in}{4.081264in}}%
\pgfpathlineto{\pgfqpoint{8.897693in}{4.080831in}}%
\pgfpathlineto{\pgfqpoint{8.898036in}{4.080502in}}%
\pgfpathlineto{\pgfqpoint{8.898720in}{4.078621in}}%
\pgfpathlineto{\pgfqpoint{8.899063in}{4.078432in}}%
\pgfpathlineto{\pgfqpoint{8.899747in}{4.077753in}}%
\pgfpathlineto{\pgfqpoint{8.900090in}{4.077660in}}%
\pgfpathlineto{\pgfqpoint{8.901117in}{4.075449in}}%
\pgfpathlineto{\pgfqpoint{8.902486in}{4.074958in}}%
\pgfpathlineto{\pgfqpoint{8.903171in}{4.073826in}}%
\pgfpathlineto{\pgfqpoint{8.903513in}{4.073637in}}%
\pgfpathlineto{\pgfqpoint{8.904198in}{4.071825in}}%
\pgfpathlineto{\pgfqpoint{8.905225in}{4.069521in}}%
\pgfpathlineto{\pgfqpoint{8.905567in}{4.069333in}}%
\pgfpathlineto{\pgfqpoint{8.906252in}{4.067785in}}%
\pgfpathlineto{\pgfqpoint{8.907279in}{4.066878in}}%
\pgfpathlineto{\pgfqpoint{8.909333in}{4.065783in}}%
\pgfpathlineto{\pgfqpoint{8.910018in}{4.064839in}}%
\pgfpathlineto{\pgfqpoint{8.911045in}{4.063631in}}%
\pgfpathlineto{\pgfqpoint{8.911387in}{4.062725in}}%
\pgfpathlineto{\pgfqpoint{8.912414in}{4.062574in}}%
\pgfpathlineto{\pgfqpoint{8.912757in}{4.062272in}}%
\pgfpathlineto{\pgfqpoint{8.913099in}{4.061441in}}%
\pgfpathlineto{\pgfqpoint{8.913441in}{4.061360in}}%
\pgfpathlineto{\pgfqpoint{8.914126in}{4.060515in}}%
\pgfpathlineto{\pgfqpoint{8.914468in}{4.058194in}}%
\pgfpathlineto{\pgfqpoint{8.915153in}{4.058005in}}%
\pgfpathlineto{\pgfqpoint{8.915495in}{4.057439in}}%
\pgfpathlineto{\pgfqpoint{8.916180in}{4.052681in}}%
\pgfpathlineto{\pgfqpoint{8.917207in}{4.051705in}}%
\pgfpathlineto{\pgfqpoint{8.917549in}{4.051662in}}%
\pgfpathlineto{\pgfqpoint{8.918234in}{4.050605in}}%
\pgfpathlineto{\pgfqpoint{8.918576in}{4.049812in}}%
\pgfpathlineto{\pgfqpoint{8.919261in}{4.047546in}}%
\pgfpathlineto{\pgfqpoint{8.920288in}{4.046829in}}%
\pgfpathlineto{\pgfqpoint{8.920973in}{4.045867in}}%
\pgfpathlineto{\pgfqpoint{8.922685in}{4.044790in}}%
\pgfpathlineto{\pgfqpoint{8.923027in}{4.042525in}}%
\pgfpathlineto{\pgfqpoint{8.923369in}{4.042487in}}%
\pgfpathlineto{\pgfqpoint{8.924054in}{4.041279in}}%
\pgfpathlineto{\pgfqpoint{8.924396in}{4.038522in}}%
\pgfpathlineto{\pgfqpoint{8.925081in}{4.038296in}}%
\pgfpathlineto{\pgfqpoint{8.925766in}{4.037050in}}%
\pgfpathlineto{\pgfqpoint{8.926793in}{4.035955in}}%
\pgfpathlineto{\pgfqpoint{8.927820in}{4.032443in}}%
\pgfpathlineto{\pgfqpoint{8.928162in}{4.032405in}}%
\pgfpathlineto{\pgfqpoint{8.928847in}{4.031046in}}%
\pgfpathlineto{\pgfqpoint{8.929189in}{4.030712in}}%
\pgfpathlineto{\pgfqpoint{8.929874in}{4.027970in}}%
\pgfpathlineto{\pgfqpoint{8.930559in}{4.027446in}}%
\pgfpathlineto{\pgfqpoint{8.931243in}{4.027195in}}%
\pgfpathlineto{\pgfqpoint{8.931586in}{4.025345in}}%
\pgfpathlineto{\pgfqpoint{8.931928in}{4.025295in}}%
\pgfpathlineto{\pgfqpoint{8.932613in}{4.023598in}}%
\pgfpathlineto{\pgfqpoint{8.934667in}{4.021569in}}%
\pgfpathlineto{\pgfqpoint{8.935009in}{4.019741in}}%
\pgfpathlineto{\pgfqpoint{8.935351in}{4.019681in}}%
\pgfpathlineto{\pgfqpoint{8.936036in}{4.018473in}}%
\pgfpathlineto{\pgfqpoint{8.936721in}{4.018397in}}%
\pgfpathlineto{\pgfqpoint{8.937063in}{4.017074in}}%
\pgfpathlineto{\pgfqpoint{8.938090in}{4.016396in}}%
\pgfpathlineto{\pgfqpoint{8.938775in}{4.013602in}}%
\pgfpathlineto{\pgfqpoint{8.939117in}{4.013375in}}%
\pgfpathlineto{\pgfqpoint{8.939802in}{4.010921in}}%
\pgfpathlineto{\pgfqpoint{8.940487in}{4.010675in}}%
\pgfpathlineto{\pgfqpoint{8.941856in}{4.009184in}}%
\pgfpathlineto{\pgfqpoint{8.943225in}{4.005748in}}%
\pgfpathlineto{\pgfqpoint{8.943568in}{4.004955in}}%
\pgfpathlineto{\pgfqpoint{8.943910in}{4.004880in}}%
\pgfpathlineto{\pgfqpoint{8.944595in}{4.004276in}}%
\pgfpathlineto{\pgfqpoint{8.944937in}{4.004087in}}%
\pgfpathlineto{\pgfqpoint{8.945622in}{4.002463in}}%
\pgfpathlineto{\pgfqpoint{8.945964in}{4.001767in}}%
\pgfpathlineto{\pgfqpoint{8.946649in}{4.001557in}}%
\pgfpathlineto{\pgfqpoint{8.948018in}{3.999858in}}%
\pgfpathlineto{\pgfqpoint{8.948361in}{3.999707in}}%
\pgfpathlineto{\pgfqpoint{8.949388in}{3.998461in}}%
\pgfpathlineto{\pgfqpoint{8.950072in}{3.997970in}}%
\pgfpathlineto{\pgfqpoint{8.950415in}{3.996533in}}%
\pgfpathlineto{\pgfqpoint{8.952469in}{3.995289in}}%
\pgfpathlineto{\pgfqpoint{8.954523in}{3.990645in}}%
\pgfpathlineto{\pgfqpoint{8.954865in}{3.990456in}}%
\pgfpathlineto{\pgfqpoint{8.955550in}{3.988418in}}%
\pgfpathlineto{\pgfqpoint{8.955892in}{3.988304in}}%
\pgfpathlineto{\pgfqpoint{8.956577in}{3.987058in}}%
\pgfpathlineto{\pgfqpoint{8.956919in}{3.987020in}}%
\pgfpathlineto{\pgfqpoint{8.957262in}{3.983962in}}%
\pgfpathlineto{\pgfqpoint{8.958631in}{3.983358in}}%
\pgfpathlineto{\pgfqpoint{8.959316in}{3.980337in}}%
\pgfpathlineto{\pgfqpoint{8.960000in}{3.979167in}}%
\pgfpathlineto{\pgfqpoint{8.960685in}{3.975937in}}%
\pgfpathlineto{\pgfqpoint{8.961027in}{3.975806in}}%
\pgfpathlineto{\pgfqpoint{8.962055in}{3.974447in}}%
\pgfpathlineto{\pgfqpoint{8.963082in}{3.970860in}}%
\pgfpathlineto{\pgfqpoint{8.963766in}{3.970369in}}%
\pgfpathlineto{\pgfqpoint{8.964109in}{3.970294in}}%
\pgfpathlineto{\pgfqpoint{8.964793in}{3.969274in}}%
\pgfpathlineto{\pgfqpoint{8.966163in}{3.966896in}}%
\pgfpathlineto{\pgfqpoint{8.966505in}{3.966745in}}%
\pgfpathlineto{\pgfqpoint{8.967532in}{3.965385in}}%
\pgfpathlineto{\pgfqpoint{8.969244in}{3.959164in}}%
\pgfpathlineto{\pgfqpoint{8.970271in}{3.957783in}}%
\pgfpathlineto{\pgfqpoint{8.971640in}{3.954284in}}%
\pgfpathlineto{\pgfqpoint{8.971983in}{3.954096in}}%
\pgfpathlineto{\pgfqpoint{8.972667in}{3.952736in}}%
\pgfpathlineto{\pgfqpoint{8.973352in}{3.952359in}}%
\pgfpathlineto{\pgfqpoint{8.974379in}{3.951415in}}%
\pgfpathlineto{\pgfqpoint{8.974721in}{3.950917in}}%
\pgfpathlineto{\pgfqpoint{8.975064in}{3.950914in}}%
\pgfpathlineto{\pgfqpoint{8.975748in}{3.950207in}}%
\pgfpathlineto{\pgfqpoint{8.976091in}{3.948432in}}%
\pgfpathlineto{\pgfqpoint{8.976775in}{3.948092in}}%
\pgfpathlineto{\pgfqpoint{8.977460in}{3.945411in}}%
\pgfpathlineto{\pgfqpoint{8.979514in}{3.942678in}}%
\pgfpathlineto{\pgfqpoint{8.980199in}{3.939181in}}%
\pgfpathlineto{\pgfqpoint{8.980541in}{3.938200in}}%
\pgfpathlineto{\pgfqpoint{8.980884in}{3.938200in}}%
\pgfpathlineto{\pgfqpoint{8.981911in}{3.936576in}}%
\pgfpathlineto{\pgfqpoint{8.982595in}{3.936425in}}%
\pgfpathlineto{\pgfqpoint{8.983280in}{3.934575in}}%
\pgfpathlineto{\pgfqpoint{8.983622in}{3.933631in}}%
\pgfpathlineto{\pgfqpoint{8.984307in}{3.933518in}}%
\pgfpathlineto{\pgfqpoint{8.984649in}{3.931970in}}%
\pgfpathlineto{\pgfqpoint{8.986019in}{3.931101in}}%
\pgfpathlineto{\pgfqpoint{8.987046in}{3.930351in}}%
\pgfpathlineto{\pgfqpoint{8.987731in}{3.928798in}}%
\pgfpathlineto{\pgfqpoint{8.988415in}{3.928115in}}%
\pgfpathlineto{\pgfqpoint{8.990127in}{3.925981in}}%
\pgfpathlineto{\pgfqpoint{8.990469in}{3.925887in}}%
\pgfpathlineto{\pgfqpoint{8.991496in}{3.923587in}}%
\pgfpathlineto{\pgfqpoint{8.991839in}{3.921888in}}%
\pgfpathlineto{\pgfqpoint{8.992523in}{3.921709in}}%
\pgfpathlineto{\pgfqpoint{8.993208in}{3.920695in}}%
\pgfpathlineto{\pgfqpoint{8.993893in}{3.920387in}}%
\pgfpathlineto{\pgfqpoint{8.994577in}{3.919321in}}%
\pgfpathlineto{\pgfqpoint{8.995262in}{3.917400in}}%
\pgfpathlineto{\pgfqpoint{8.995604in}{3.917282in}}%
\pgfpathlineto{\pgfqpoint{8.996289in}{3.916362in}}%
\pgfpathlineto{\pgfqpoint{8.996632in}{3.914189in}}%
\pgfpathlineto{\pgfqpoint{8.996974in}{3.913959in}}%
\pgfpathlineto{\pgfqpoint{8.997659in}{3.912260in}}%
\pgfpathlineto{\pgfqpoint{8.998001in}{3.912147in}}%
\pgfpathlineto{\pgfqpoint{9.000055in}{3.908258in}}%
\pgfpathlineto{\pgfqpoint{9.000397in}{3.905917in}}%
\pgfpathlineto{\pgfqpoint{9.001424in}{3.905388in}}%
\pgfpathlineto{\pgfqpoint{9.002451in}{3.903236in}}%
\pgfpathlineto{\pgfqpoint{9.003136in}{3.902896in}}%
\pgfpathlineto{\pgfqpoint{9.003478in}{3.902821in}}%
\pgfpathlineto{\pgfqpoint{9.005875in}{3.899045in}}%
\pgfpathlineto{\pgfqpoint{9.006560in}{3.897459in}}%
\pgfpathlineto{\pgfqpoint{9.006902in}{3.897421in}}%
\pgfpathlineto{\pgfqpoint{9.007587in}{3.895201in}}%
\pgfpathlineto{\pgfqpoint{9.008271in}{3.894627in}}%
\pgfpathlineto{\pgfqpoint{9.008614in}{3.894325in}}%
\pgfpathlineto{\pgfqpoint{9.008956in}{3.893276in}}%
\pgfpathlineto{\pgfqpoint{9.009298in}{3.893192in}}%
\pgfpathlineto{\pgfqpoint{9.009983in}{3.891871in}}%
\pgfpathlineto{\pgfqpoint{9.010325in}{3.891418in}}%
\pgfpathlineto{\pgfqpoint{9.010668in}{3.889605in}}%
\pgfpathlineto{\pgfqpoint{9.011352in}{3.888963in}}%
\pgfpathlineto{\pgfqpoint{9.012037in}{3.887642in}}%
\pgfpathlineto{\pgfqpoint{9.013407in}{3.887415in}}%
\pgfpathlineto{\pgfqpoint{9.014091in}{3.886207in}}%
\pgfpathlineto{\pgfqpoint{9.014776in}{3.883677in}}%
\pgfpathlineto{\pgfqpoint{9.015803in}{3.882696in}}%
\pgfpathlineto{\pgfqpoint{9.016145in}{3.881603in}}%
\pgfpathlineto{\pgfqpoint{9.016488in}{3.881601in}}%
\pgfpathlineto{\pgfqpoint{9.017515in}{3.880586in}}%
\pgfpathlineto{\pgfqpoint{9.017857in}{3.880468in}}%
\pgfpathlineto{\pgfqpoint{9.018542in}{3.879222in}}%
\pgfpathlineto{\pgfqpoint{9.019569in}{3.877221in}}%
\pgfpathlineto{\pgfqpoint{9.020253in}{3.876829in}}%
\pgfpathlineto{\pgfqpoint{9.020596in}{3.876654in}}%
\pgfpathlineto{\pgfqpoint{9.021281in}{3.876088in}}%
\pgfpathlineto{\pgfqpoint{9.021965in}{3.875523in}}%
\pgfpathlineto{\pgfqpoint{9.022308in}{3.874087in}}%
\pgfpathlineto{\pgfqpoint{9.022992in}{3.873483in}}%
\pgfpathlineto{\pgfqpoint{9.024019in}{3.872237in}}%
\pgfpathlineto{\pgfqpoint{9.024362in}{3.872237in}}%
\pgfpathlineto{\pgfqpoint{9.025389in}{3.869254in}}%
\pgfpathlineto{\pgfqpoint{9.026073in}{3.868440in}}%
\pgfpathlineto{\pgfqpoint{9.027785in}{3.866762in}}%
\pgfpathlineto{\pgfqpoint{9.028127in}{3.866611in}}%
\pgfpathlineto{\pgfqpoint{9.028470in}{3.866007in}}%
\pgfpathlineto{\pgfqpoint{9.029154in}{3.864119in}}%
\pgfpathlineto{\pgfqpoint{9.029839in}{3.863273in}}%
\pgfpathlineto{\pgfqpoint{9.030524in}{3.862049in}}%
\pgfpathlineto{\pgfqpoint{9.030866in}{3.861872in}}%
\pgfpathlineto{\pgfqpoint{9.031209in}{3.861340in}}%
\pgfpathlineto{\pgfqpoint{9.031893in}{3.861287in}}%
\pgfpathlineto{\pgfqpoint{9.032578in}{3.859210in}}%
\pgfpathlineto{\pgfqpoint{9.033263in}{3.857662in}}%
\pgfpathlineto{\pgfqpoint{9.034290in}{3.857398in}}%
\pgfpathlineto{\pgfqpoint{9.035317in}{3.855925in}}%
\pgfpathlineto{\pgfqpoint{9.035659in}{3.854226in}}%
\pgfpathlineto{\pgfqpoint{9.036001in}{3.854129in}}%
\pgfpathlineto{\pgfqpoint{9.036686in}{3.852754in}}%
\pgfpathlineto{\pgfqpoint{9.037028in}{3.852248in}}%
\pgfpathlineto{\pgfqpoint{9.038740in}{3.846108in}}%
\pgfpathlineto{\pgfqpoint{9.039767in}{3.845506in}}%
\pgfpathlineto{\pgfqpoint{9.040452in}{3.845051in}}%
\pgfpathlineto{\pgfqpoint{9.041479in}{3.842446in}}%
\pgfpathlineto{\pgfqpoint{9.041821in}{3.842398in}}%
\pgfpathlineto{\pgfqpoint{9.042506in}{3.841011in}}%
\pgfpathlineto{\pgfqpoint{9.042848in}{3.840029in}}%
\pgfpathlineto{\pgfqpoint{9.044218in}{3.839727in}}%
\pgfpathlineto{\pgfqpoint{9.044560in}{3.837500in}}%
\pgfpathlineto{\pgfqpoint{9.046957in}{3.836140in}}%
\pgfpathlineto{\pgfqpoint{9.047984in}{3.833761in}}%
\pgfpathlineto{\pgfqpoint{9.048668in}{3.833459in}}%
\pgfpathlineto{\pgfqpoint{9.049353in}{3.831269in}}%
\pgfpathlineto{\pgfqpoint{9.049695in}{3.830967in}}%
\pgfpathlineto{\pgfqpoint{9.050380in}{3.830008in}}%
\pgfpathlineto{\pgfqpoint{9.053119in}{3.825077in}}%
\pgfpathlineto{\pgfqpoint{9.053803in}{3.824926in}}%
\pgfpathlineto{\pgfqpoint{9.054488in}{3.824360in}}%
\pgfpathlineto{\pgfqpoint{9.054830in}{3.821113in}}%
\pgfpathlineto{\pgfqpoint{9.055515in}{3.820508in}}%
\pgfpathlineto{\pgfqpoint{9.056200in}{3.820018in}}%
\pgfpathlineto{\pgfqpoint{9.056885in}{3.819699in}}%
\pgfpathlineto{\pgfqpoint{9.057569in}{3.819036in}}%
\pgfpathlineto{\pgfqpoint{9.058596in}{3.818281in}}%
\pgfpathlineto{\pgfqpoint{9.058939in}{3.816959in}}%
\pgfpathlineto{\pgfqpoint{9.059281in}{3.816846in}}%
\pgfpathlineto{\pgfqpoint{9.060650in}{3.814505in}}%
\pgfpathlineto{\pgfqpoint{9.061335in}{3.814241in}}%
\pgfpathlineto{\pgfqpoint{9.063731in}{3.806765in}}%
\pgfpathlineto{\pgfqpoint{9.064759in}{3.806349in}}%
\pgfpathlineto{\pgfqpoint{9.066470in}{3.804839in}}%
\pgfpathlineto{\pgfqpoint{9.066813in}{3.803607in}}%
\pgfpathlineto{\pgfqpoint{9.067497in}{3.803404in}}%
\pgfpathlineto{\pgfqpoint{9.068182in}{3.802271in}}%
\pgfpathlineto{\pgfqpoint{9.070578in}{3.800421in}}%
\pgfpathlineto{\pgfqpoint{9.071263in}{3.799272in}}%
\pgfpathlineto{\pgfqpoint{9.071948in}{3.798684in}}%
\pgfpathlineto{\pgfqpoint{9.072290in}{3.796880in}}%
\pgfpathlineto{\pgfqpoint{9.072633in}{3.796797in}}%
\pgfpathlineto{\pgfqpoint{9.074687in}{3.794909in}}%
\pgfpathlineto{\pgfqpoint{9.075371in}{3.794603in}}%
\pgfpathlineto{\pgfqpoint{9.075714in}{3.794291in}}%
\pgfpathlineto{\pgfqpoint{9.076741in}{3.790848in}}%
\pgfpathlineto{\pgfqpoint{9.077425in}{3.790148in}}%
\pgfpathlineto{\pgfqpoint{9.078110in}{3.788867in}}%
\pgfpathlineto{\pgfqpoint{9.078452in}{3.788867in}}%
\pgfpathlineto{\pgfqpoint{9.078795in}{3.788565in}}%
\pgfpathlineto{\pgfqpoint{9.079137in}{3.786677in}}%
\pgfpathlineto{\pgfqpoint{9.080164in}{3.786305in}}%
\pgfpathlineto{\pgfqpoint{9.080506in}{3.785507in}}%
\pgfpathlineto{\pgfqpoint{9.082218in}{3.784843in}}%
\pgfpathlineto{\pgfqpoint{9.084615in}{3.783544in}}%
\pgfpathlineto{\pgfqpoint{9.084957in}{3.782856in}}%
\pgfpathlineto{\pgfqpoint{9.086326in}{3.782667in}}%
\pgfpathlineto{\pgfqpoint{9.087353in}{3.781468in}}%
\pgfpathlineto{\pgfqpoint{9.088038in}{3.780900in}}%
\pgfpathlineto{\pgfqpoint{9.090435in}{3.779013in}}%
\pgfpathlineto{\pgfqpoint{9.090777in}{3.778469in}}%
\pgfpathlineto{\pgfqpoint{9.091119in}{3.778454in}}%
\pgfpathlineto{\pgfqpoint{9.092489in}{3.777422in}}%
\pgfpathlineto{\pgfqpoint{9.093173in}{3.777162in}}%
\pgfpathlineto{\pgfqpoint{9.094200in}{3.776634in}}%
\pgfpathlineto{\pgfqpoint{9.094885in}{3.776105in}}%
\pgfpathlineto{\pgfqpoint{9.095227in}{3.775803in}}%
\pgfpathlineto{\pgfqpoint{9.096254in}{3.772711in}}%
\pgfpathlineto{\pgfqpoint{9.097624in}{3.770643in}}%
\pgfpathlineto{\pgfqpoint{9.098309in}{3.769649in}}%
\pgfpathlineto{\pgfqpoint{9.098651in}{3.768724in}}%
\pgfpathlineto{\pgfqpoint{9.099336in}{3.768252in}}%
\pgfpathlineto{\pgfqpoint{9.099678in}{3.768138in}}%
\pgfpathlineto{\pgfqpoint{9.100020in}{3.766288in}}%
\pgfpathlineto{\pgfqpoint{9.101047in}{3.765986in}}%
\pgfpathlineto{\pgfqpoint{9.101732in}{3.764502in}}%
\pgfpathlineto{\pgfqpoint{9.102759in}{3.764174in}}%
\pgfpathlineto{\pgfqpoint{9.103444in}{3.763526in}}%
\pgfpathlineto{\pgfqpoint{9.109264in}{3.758892in}}%
\pgfpathlineto{\pgfqpoint{9.109606in}{3.757701in}}%
\pgfpathlineto{\pgfqpoint{9.109948in}{3.757566in}}%
\pgfpathlineto{\pgfqpoint{9.110975in}{3.756053in}}%
\pgfpathlineto{\pgfqpoint{9.111318in}{3.756019in}}%
\pgfpathlineto{\pgfqpoint{9.114056in}{3.753292in}}%
\pgfpathlineto{\pgfqpoint{9.115426in}{3.752023in}}%
\pgfpathlineto{\pgfqpoint{9.115768in}{3.751374in}}%
\pgfpathlineto{\pgfqpoint{9.116111in}{3.749561in}}%
\pgfpathlineto{\pgfqpoint{9.116795in}{3.749395in}}%
\pgfpathlineto{\pgfqpoint{9.117480in}{3.748426in}}%
\pgfpathlineto{\pgfqpoint{9.118165in}{3.747598in}}%
\pgfpathlineto{\pgfqpoint{9.119192in}{3.746616in}}%
\pgfpathlineto{\pgfqpoint{9.119534in}{3.746314in}}%
\pgfpathlineto{\pgfqpoint{9.121246in}{3.742123in}}%
\pgfpathlineto{\pgfqpoint{9.121588in}{3.741481in}}%
\pgfpathlineto{\pgfqpoint{9.122957in}{3.741330in}}%
\pgfpathlineto{\pgfqpoint{9.123642in}{3.739820in}}%
\pgfpathlineto{\pgfqpoint{9.124669in}{3.738763in}}%
\pgfpathlineto{\pgfqpoint{9.125354in}{3.738083in}}%
\pgfpathlineto{\pgfqpoint{9.126381in}{3.736799in}}%
\pgfpathlineto{\pgfqpoint{9.126723in}{3.736799in}}%
\pgfpathlineto{\pgfqpoint{9.127408in}{3.736130in}}%
\pgfpathlineto{\pgfqpoint{9.127750in}{3.736006in}}%
\pgfpathlineto{\pgfqpoint{9.128435in}{3.734534in}}%
\pgfpathlineto{\pgfqpoint{9.129462in}{3.734153in}}%
\pgfpathlineto{\pgfqpoint{9.129804in}{3.733061in}}%
\pgfpathlineto{\pgfqpoint{9.130831in}{3.732872in}}%
\pgfpathlineto{\pgfqpoint{9.131516in}{3.731815in}}%
\pgfpathlineto{\pgfqpoint{9.132201in}{3.731625in}}%
\pgfpathlineto{\pgfqpoint{9.132543in}{3.730720in}}%
\pgfpathlineto{\pgfqpoint{9.132886in}{3.730607in}}%
\pgfpathlineto{\pgfqpoint{9.133570in}{3.729690in}}%
\pgfpathlineto{\pgfqpoint{9.133913in}{3.729550in}}%
\pgfpathlineto{\pgfqpoint{9.135282in}{3.727322in}}%
\pgfpathlineto{\pgfqpoint{9.135624in}{3.727133in}}%
\pgfpathlineto{\pgfqpoint{9.136309in}{3.724792in}}%
\pgfpathlineto{\pgfqpoint{9.136994in}{3.724717in}}%
\pgfpathlineto{\pgfqpoint{9.137678in}{3.723924in}}%
\pgfpathlineto{\pgfqpoint{9.138363in}{3.723379in}}%
\pgfpathlineto{\pgfqpoint{9.138705in}{3.722867in}}%
\pgfpathlineto{\pgfqpoint{9.139390in}{3.720812in}}%
\pgfpathlineto{\pgfqpoint{9.140759in}{3.719922in}}%
\pgfpathlineto{\pgfqpoint{9.142129in}{3.717256in}}%
\pgfpathlineto{\pgfqpoint{9.144183in}{3.715192in}}%
\pgfpathlineto{\pgfqpoint{9.144525in}{3.712769in}}%
\pgfpathlineto{\pgfqpoint{9.145895in}{3.711615in}}%
\pgfpathlineto{\pgfqpoint{9.146579in}{3.710369in}}%
\pgfpathlineto{\pgfqpoint{9.146922in}{3.710188in}}%
\pgfpathlineto{\pgfqpoint{9.147949in}{3.708103in}}%
\pgfpathlineto{\pgfqpoint{9.148633in}{3.707348in}}%
\pgfpathlineto{\pgfqpoint{9.149661in}{3.707159in}}%
\pgfpathlineto{\pgfqpoint{9.150003in}{3.706555in}}%
\pgfpathlineto{\pgfqpoint{9.150688in}{3.706268in}}%
\pgfpathlineto{\pgfqpoint{9.152057in}{3.705635in}}%
\pgfpathlineto{\pgfqpoint{9.152742in}{3.704683in}}%
\pgfpathlineto{\pgfqpoint{9.154453in}{3.703874in}}%
\pgfpathlineto{\pgfqpoint{9.154796in}{3.702591in}}%
\pgfpathlineto{\pgfqpoint{9.155480in}{3.702311in}}%
\pgfpathlineto{\pgfqpoint{9.157192in}{3.698665in}}%
\pgfpathlineto{\pgfqpoint{9.157534in}{3.698664in}}%
\pgfpathlineto{\pgfqpoint{9.158219in}{3.698135in}}%
\pgfpathlineto{\pgfqpoint{9.158904in}{3.697963in}}%
\pgfpathlineto{\pgfqpoint{9.159589in}{3.695606in}}%
\pgfpathlineto{\pgfqpoint{9.160616in}{3.691981in}}%
\pgfpathlineto{\pgfqpoint{9.160958in}{3.691674in}}%
\pgfpathlineto{\pgfqpoint{9.161643in}{3.690168in}}%
\pgfpathlineto{\pgfqpoint{9.162327in}{3.690093in}}%
\pgfpathlineto{\pgfqpoint{9.163012in}{3.689338in}}%
\pgfpathlineto{\pgfqpoint{9.163697in}{3.688243in}}%
\pgfpathlineto{\pgfqpoint{9.164381in}{3.686883in}}%
\pgfpathlineto{\pgfqpoint{9.164724in}{3.686846in}}%
\pgfpathlineto{\pgfqpoint{9.165066in}{3.686512in}}%
\pgfpathlineto{\pgfqpoint{9.165751in}{3.685058in}}%
\pgfpathlineto{\pgfqpoint{9.167120in}{3.683485in}}%
\pgfpathlineto{\pgfqpoint{9.167805in}{3.682919in}}%
\pgfpathlineto{\pgfqpoint{9.168490in}{3.682617in}}%
\pgfpathlineto{\pgfqpoint{9.168832in}{3.681711in}}%
\pgfpathlineto{\pgfqpoint{9.169174in}{3.679407in}}%
\pgfpathlineto{\pgfqpoint{9.169859in}{3.679347in}}%
\pgfpathlineto{\pgfqpoint{9.172940in}{3.675692in}}%
\pgfpathlineto{\pgfqpoint{9.173625in}{3.675481in}}%
\pgfpathlineto{\pgfqpoint{9.174309in}{3.674763in}}%
\pgfpathlineto{\pgfqpoint{9.174994in}{3.672913in}}%
\pgfpathlineto{\pgfqpoint{9.177048in}{3.672161in}}%
\pgfpathlineto{\pgfqpoint{9.177733in}{3.671063in}}%
\pgfpathlineto{\pgfqpoint{9.178075in}{3.670950in}}%
\pgfpathlineto{\pgfqpoint{9.179102in}{3.669681in}}%
\pgfpathlineto{\pgfqpoint{9.179445in}{3.669590in}}%
\pgfpathlineto{\pgfqpoint{9.179787in}{3.669137in}}%
\pgfpathlineto{\pgfqpoint{9.180472in}{3.667023in}}%
\pgfpathlineto{\pgfqpoint{9.180814in}{3.666910in}}%
\pgfpathlineto{\pgfqpoint{9.182183in}{3.665513in}}%
\pgfpathlineto{\pgfqpoint{9.183210in}{3.665150in}}%
\pgfpathlineto{\pgfqpoint{9.183553in}{3.664884in}}%
\pgfpathlineto{\pgfqpoint{9.183895in}{3.663738in}}%
\pgfpathlineto{\pgfqpoint{9.184922in}{3.663436in}}%
\pgfpathlineto{\pgfqpoint{9.185265in}{3.663209in}}%
\pgfpathlineto{\pgfqpoint{9.185607in}{3.661718in}}%
\pgfpathlineto{\pgfqpoint{9.186292in}{3.661420in}}%
\pgfpathlineto{\pgfqpoint{9.186976in}{3.660438in}}%
\pgfpathlineto{\pgfqpoint{9.188003in}{3.659773in}}%
\pgfpathlineto{\pgfqpoint{9.188346in}{3.659095in}}%
\pgfpathlineto{\pgfqpoint{9.189030in}{3.658829in}}%
\pgfpathlineto{\pgfqpoint{9.189373in}{3.658746in}}%
\pgfpathlineto{\pgfqpoint{9.189715in}{3.658301in}}%
\pgfpathlineto{\pgfqpoint{9.190742in}{3.654298in}}%
\pgfpathlineto{\pgfqpoint{9.191769in}{3.653921in}}%
\pgfpathlineto{\pgfqpoint{9.192112in}{3.653732in}}%
\pgfpathlineto{\pgfqpoint{9.193481in}{3.651240in}}%
\pgfpathlineto{\pgfqpoint{9.193823in}{3.651014in}}%
\pgfpathlineto{\pgfqpoint{9.194850in}{3.648106in}}%
\pgfpathlineto{\pgfqpoint{9.195193in}{3.647880in}}%
\pgfpathlineto{\pgfqpoint{9.195877in}{3.646822in}}%
\pgfpathlineto{\pgfqpoint{9.196904in}{3.645312in}}%
\pgfpathlineto{\pgfqpoint{9.197589in}{3.644351in}}%
\pgfpathlineto{\pgfqpoint{9.197931in}{3.643651in}}%
\pgfpathlineto{\pgfqpoint{9.198958in}{3.643424in}}%
\pgfpathlineto{\pgfqpoint{9.199301in}{3.643349in}}%
\pgfpathlineto{\pgfqpoint{9.199643in}{3.642707in}}%
\pgfpathlineto{\pgfqpoint{9.201013in}{3.642442in}}%
\pgfpathlineto{\pgfqpoint{9.201355in}{3.642291in}}%
\pgfpathlineto{\pgfqpoint{9.201697in}{3.641121in}}%
\pgfpathlineto{\pgfqpoint{9.202724in}{3.640743in}}%
\pgfpathlineto{\pgfqpoint{9.203067in}{3.640479in}}%
\pgfpathlineto{\pgfqpoint{9.203751in}{3.638942in}}%
\pgfpathlineto{\pgfqpoint{9.204436in}{3.638608in}}%
\pgfpathlineto{\pgfqpoint{9.205121in}{3.637761in}}%
\pgfpathlineto{\pgfqpoint{9.206490in}{3.637380in}}%
\pgfpathlineto{\pgfqpoint{9.207175in}{3.636918in}}%
\pgfpathlineto{\pgfqpoint{9.207859in}{3.636817in}}%
\pgfpathlineto{\pgfqpoint{9.208544in}{3.635533in}}%
\pgfpathlineto{\pgfqpoint{9.208886in}{3.635042in}}%
\pgfpathlineto{\pgfqpoint{9.209229in}{3.632618in}}%
\pgfpathlineto{\pgfqpoint{9.209571in}{3.632437in}}%
\pgfpathlineto{\pgfqpoint{9.210256in}{3.631153in}}%
\pgfpathlineto{\pgfqpoint{9.210598in}{3.631053in}}%
\pgfpathlineto{\pgfqpoint{9.210941in}{3.630398in}}%
\pgfpathlineto{\pgfqpoint{9.211283in}{3.630383in}}%
\pgfpathlineto{\pgfqpoint{9.212995in}{3.627339in}}%
\pgfpathlineto{\pgfqpoint{9.213679in}{3.626772in}}%
\pgfpathlineto{\pgfqpoint{9.214706in}{3.626486in}}%
\pgfpathlineto{\pgfqpoint{9.215049in}{3.625584in}}%
\pgfpathlineto{\pgfqpoint{9.215391in}{3.625527in}}%
\pgfpathlineto{\pgfqpoint{9.217445in}{3.620543in}}%
\pgfpathlineto{\pgfqpoint{9.220526in}{3.617787in}}%
\pgfpathlineto{\pgfqpoint{9.221553in}{3.615792in}}%
\pgfpathlineto{\pgfqpoint{9.222923in}{3.614955in}}%
\pgfpathlineto{\pgfqpoint{9.223607in}{3.614162in}}%
\pgfpathlineto{\pgfqpoint{9.226004in}{3.613105in}}%
\pgfpathlineto{\pgfqpoint{9.226346in}{3.612614in}}%
\pgfpathlineto{\pgfqpoint{9.227031in}{3.610915in}}%
\pgfpathlineto{\pgfqpoint{9.228400in}{3.609899in}}%
\pgfpathlineto{\pgfqpoint{9.228743in}{3.609087in}}%
\pgfpathlineto{\pgfqpoint{9.229085in}{3.609012in}}%
\pgfpathlineto{\pgfqpoint{9.230112in}{3.607970in}}%
\pgfpathlineto{\pgfqpoint{9.231139in}{3.607819in}}%
\pgfpathlineto{\pgfqpoint{9.231824in}{3.605947in}}%
\pgfpathlineto{\pgfqpoint{9.232851in}{3.605083in}}%
\pgfpathlineto{\pgfqpoint{9.233535in}{3.604873in}}%
\pgfpathlineto{\pgfqpoint{9.234562in}{3.602986in}}%
\pgfpathlineto{\pgfqpoint{9.235247in}{3.602019in}}%
\pgfpathlineto{\pgfqpoint{9.235590in}{3.601012in}}%
\pgfpathlineto{\pgfqpoint{9.235932in}{3.600947in}}%
\pgfpathlineto{\pgfqpoint{9.236617in}{3.600380in}}%
\pgfpathlineto{\pgfqpoint{9.239013in}{3.599363in}}%
\pgfpathlineto{\pgfqpoint{9.240382in}{3.598507in}}%
\pgfpathlineto{\pgfqpoint{9.241067in}{3.597133in}}%
\pgfpathlineto{\pgfqpoint{9.241409in}{3.597117in}}%
\pgfpathlineto{\pgfqpoint{9.241752in}{3.596206in}}%
\pgfpathlineto{\pgfqpoint{9.242094in}{3.596131in}}%
\pgfpathlineto{\pgfqpoint{9.243121in}{3.595207in}}%
\pgfpathlineto{\pgfqpoint{9.245175in}{3.594566in}}%
\pgfpathlineto{\pgfqpoint{9.247229in}{3.592678in}}%
\pgfpathlineto{\pgfqpoint{9.247914in}{3.592602in}}%
\pgfpathlineto{\pgfqpoint{9.248599in}{3.591432in}}%
\pgfpathlineto{\pgfqpoint{9.248941in}{3.591243in}}%
\pgfpathlineto{\pgfqpoint{9.249626in}{3.589279in}}%
\pgfpathlineto{\pgfqpoint{9.252365in}{3.586485in}}%
\pgfpathlineto{\pgfqpoint{9.252707in}{3.586070in}}%
\pgfpathlineto{\pgfqpoint{9.253392in}{3.584673in}}%
\pgfpathlineto{\pgfqpoint{9.254076in}{3.583578in}}%
\pgfpathlineto{\pgfqpoint{9.254419in}{3.581917in}}%
\pgfpathlineto{\pgfqpoint{9.254761in}{3.581902in}}%
\pgfpathlineto{\pgfqpoint{9.257500in}{3.579047in}}%
\pgfpathlineto{\pgfqpoint{9.257842in}{3.579022in}}%
\pgfpathlineto{\pgfqpoint{9.258184in}{3.578179in}}%
\pgfpathlineto{\pgfqpoint{9.258869in}{3.577952in}}%
\pgfpathlineto{\pgfqpoint{9.259554in}{3.576676in}}%
\pgfpathlineto{\pgfqpoint{9.260238in}{3.576555in}}%
\pgfpathlineto{\pgfqpoint{9.262635in}{3.575045in}}%
\pgfpathlineto{\pgfqpoint{9.263320in}{3.574667in}}%
\pgfpathlineto{\pgfqpoint{9.263662in}{3.574589in}}%
\pgfpathlineto{\pgfqpoint{9.264689in}{3.573044in}}%
\pgfpathlineto{\pgfqpoint{9.265031in}{3.572930in}}%
\pgfpathlineto{\pgfqpoint{9.265716in}{3.572326in}}%
\pgfpathlineto{\pgfqpoint{9.266058in}{3.571193in}}%
\pgfpathlineto{\pgfqpoint{9.266743in}{3.570740in}}%
\pgfpathlineto{\pgfqpoint{9.267428in}{3.570121in}}%
\pgfpathlineto{\pgfqpoint{9.268455in}{3.569625in}}%
\pgfpathlineto{\pgfqpoint{9.269140in}{3.568298in}}%
\pgfpathlineto{\pgfqpoint{9.270167in}{3.568022in}}%
\pgfpathlineto{\pgfqpoint{9.270509in}{3.567040in}}%
\pgfpathlineto{\pgfqpoint{9.270851in}{3.566919in}}%
\pgfpathlineto{\pgfqpoint{9.271878in}{3.565265in}}%
\pgfpathlineto{\pgfqpoint{9.273590in}{3.564699in}}%
\pgfpathlineto{\pgfqpoint{9.274275in}{3.563415in}}%
\pgfpathlineto{\pgfqpoint{9.276329in}{3.562590in}}%
\pgfpathlineto{\pgfqpoint{9.278041in}{3.561008in}}%
\pgfpathlineto{\pgfqpoint{9.278383in}{3.560961in}}%
\pgfpathlineto{\pgfqpoint{9.279068in}{3.559881in}}%
\pgfpathlineto{\pgfqpoint{9.280437in}{3.558469in}}%
\pgfpathlineto{\pgfqpoint{9.281122in}{3.557525in}}%
\pgfpathlineto{\pgfqpoint{9.281464in}{3.556845in}}%
\pgfpathlineto{\pgfqpoint{9.281806in}{3.556808in}}%
\pgfpathlineto{\pgfqpoint{9.282833in}{3.556053in}}%
\pgfpathlineto{\pgfqpoint{9.287284in}{3.551861in}}%
\pgfpathlineto{\pgfqpoint{9.289338in}{3.549778in}}%
\pgfpathlineto{\pgfqpoint{9.290023in}{3.549369in}}%
\pgfpathlineto{\pgfqpoint{9.290365in}{3.548530in}}%
\pgfpathlineto{\pgfqpoint{9.290707in}{3.548526in}}%
\pgfpathlineto{\pgfqpoint{9.292761in}{3.545556in}}%
\pgfpathlineto{\pgfqpoint{9.293104in}{3.545329in}}%
\pgfpathlineto{\pgfqpoint{9.293788in}{3.544266in}}%
\pgfpathlineto{\pgfqpoint{9.294131in}{3.544113in}}%
\pgfpathlineto{\pgfqpoint{9.295158in}{3.542535in}}%
\pgfpathlineto{\pgfqpoint{9.295843in}{3.542190in}}%
\pgfpathlineto{\pgfqpoint{9.296527in}{3.540647in}}%
\pgfpathlineto{\pgfqpoint{9.299266in}{3.539129in}}%
\pgfpathlineto{\pgfqpoint{9.300293in}{3.537627in}}%
\pgfpathlineto{\pgfqpoint{9.300635in}{3.537627in}}%
\pgfpathlineto{\pgfqpoint{9.301320in}{3.537136in}}%
\pgfpathlineto{\pgfqpoint{9.301662in}{3.537048in}}%
\pgfpathlineto{\pgfqpoint{9.303032in}{3.534493in}}%
\pgfpathlineto{\pgfqpoint{9.304401in}{3.534113in}}%
\pgfpathlineto{\pgfqpoint{9.304744in}{3.533285in}}%
\pgfpathlineto{\pgfqpoint{9.305771in}{3.532907in}}%
\pgfpathlineto{\pgfqpoint{9.308167in}{3.531777in}}%
\pgfpathlineto{\pgfqpoint{9.309194in}{3.529244in}}%
\pgfpathlineto{\pgfqpoint{9.310221in}{3.528036in}}%
\pgfpathlineto{\pgfqpoint{9.310563in}{3.527112in}}%
\pgfpathlineto{\pgfqpoint{9.311248in}{3.526941in}}%
\pgfpathlineto{\pgfqpoint{9.311591in}{3.526564in}}%
\pgfpathlineto{\pgfqpoint{9.311933in}{3.525355in}}%
\pgfpathlineto{\pgfqpoint{9.312275in}{3.525242in}}%
\pgfpathlineto{\pgfqpoint{9.312618in}{3.524827in}}%
\pgfpathlineto{\pgfqpoint{9.312960in}{3.522977in}}%
\pgfpathlineto{\pgfqpoint{9.313302in}{3.522788in}}%
\pgfpathlineto{\pgfqpoint{9.313987in}{3.521542in}}%
\pgfpathlineto{\pgfqpoint{9.314329in}{3.521285in}}%
\pgfpathlineto{\pgfqpoint{9.315014in}{3.520258in}}%
\pgfpathlineto{\pgfqpoint{9.315356in}{3.520035in}}%
\pgfpathlineto{\pgfqpoint{9.316041in}{3.519125in}}%
\pgfpathlineto{\pgfqpoint{9.316383in}{3.519088in}}%
\pgfpathlineto{\pgfqpoint{9.317068in}{3.518348in}}%
\pgfpathlineto{\pgfqpoint{9.317410in}{3.518332in}}%
\pgfpathlineto{\pgfqpoint{9.318095in}{3.517540in}}%
\pgfpathlineto{\pgfqpoint{9.318437in}{3.517351in}}%
\pgfpathlineto{\pgfqpoint{9.318780in}{3.516520in}}%
\pgfpathlineto{\pgfqpoint{9.319122in}{3.516520in}}%
\pgfpathlineto{\pgfqpoint{9.319464in}{3.516120in}}%
\pgfpathlineto{\pgfqpoint{9.319807in}{3.514632in}}%
\pgfpathlineto{\pgfqpoint{9.320492in}{3.514481in}}%
\pgfpathlineto{\pgfqpoint{9.320834in}{3.513915in}}%
\pgfpathlineto{\pgfqpoint{9.321861in}{3.510161in}}%
\pgfpathlineto{\pgfqpoint{9.323230in}{3.509875in}}%
\pgfpathlineto{\pgfqpoint{9.324257in}{3.509233in}}%
\pgfpathlineto{\pgfqpoint{9.324600in}{3.507529in}}%
\pgfpathlineto{\pgfqpoint{9.324942in}{3.507394in}}%
\pgfpathlineto{\pgfqpoint{9.325627in}{3.505862in}}%
\pgfpathlineto{\pgfqpoint{9.328023in}{3.504629in}}%
\pgfpathlineto{\pgfqpoint{9.329393in}{3.504030in}}%
\pgfpathlineto{\pgfqpoint{9.331447in}{3.501077in}}%
\pgfpathlineto{\pgfqpoint{9.332474in}{3.500851in}}%
\pgfpathlineto{\pgfqpoint{9.333158in}{3.500322in}}%
\pgfpathlineto{\pgfqpoint{9.333843in}{3.500095in}}%
\pgfpathlineto{\pgfqpoint{9.334185in}{3.499265in}}%
\pgfpathlineto{\pgfqpoint{9.334528in}{3.499265in}}%
\pgfpathlineto{\pgfqpoint{9.335212in}{3.498351in}}%
\pgfpathlineto{\pgfqpoint{9.336239in}{3.497981in}}%
\pgfpathlineto{\pgfqpoint{9.336924in}{3.496697in}}%
\pgfpathlineto{\pgfqpoint{9.337609in}{3.496047in}}%
\pgfpathlineto{\pgfqpoint{9.338294in}{3.494939in}}%
\pgfpathlineto{\pgfqpoint{9.338978in}{3.493337in}}%
\pgfpathlineto{\pgfqpoint{9.340005in}{3.491826in}}%
\pgfpathlineto{\pgfqpoint{9.341375in}{3.491347in}}%
\pgfpathlineto{\pgfqpoint{9.341717in}{3.491336in}}%
\pgfpathlineto{\pgfqpoint{9.342059in}{3.489976in}}%
\pgfpathlineto{\pgfqpoint{9.342402in}{3.489825in}}%
\pgfpathlineto{\pgfqpoint{9.343086in}{3.488827in}}%
\pgfpathlineto{\pgfqpoint{9.344798in}{3.487409in}}%
\pgfpathlineto{\pgfqpoint{9.345825in}{3.486571in}}%
\pgfpathlineto{\pgfqpoint{9.346510in}{3.485974in}}%
\pgfpathlineto{\pgfqpoint{9.347195in}{3.484615in}}%
\pgfpathlineto{\pgfqpoint{9.347879in}{3.484385in}}%
\pgfpathlineto{\pgfqpoint{9.348564in}{3.483943in}}%
\pgfpathlineto{\pgfqpoint{9.348906in}{3.483860in}}%
\pgfpathlineto{\pgfqpoint{9.349591in}{3.482402in}}%
\pgfpathlineto{\pgfqpoint{9.351645in}{3.480862in}}%
\pgfpathlineto{\pgfqpoint{9.354726in}{3.476329in}}%
\pgfpathlineto{\pgfqpoint{9.355411in}{3.476006in}}%
\pgfpathlineto{\pgfqpoint{9.356438in}{3.475923in}}%
\pgfpathlineto{\pgfqpoint{9.357123in}{3.475100in}}%
\pgfpathlineto{\pgfqpoint{9.357807in}{3.475016in}}%
\pgfpathlineto{\pgfqpoint{9.358834in}{3.474458in}}%
\pgfpathlineto{\pgfqpoint{9.359519in}{3.474005in}}%
\pgfpathlineto{\pgfqpoint{9.360888in}{3.473099in}}%
\pgfpathlineto{\pgfqpoint{9.361915in}{3.471966in}}%
\pgfpathlineto{\pgfqpoint{9.362258in}{3.471928in}}%
\pgfpathlineto{\pgfqpoint{9.363285in}{3.471211in}}%
\pgfpathlineto{\pgfqpoint{9.363970in}{3.471031in}}%
\pgfpathlineto{\pgfqpoint{9.366366in}{3.467171in}}%
\pgfpathlineto{\pgfqpoint{9.366708in}{3.466982in}}%
\pgfpathlineto{\pgfqpoint{9.367393in}{3.465547in}}%
\pgfpathlineto{\pgfqpoint{9.368078in}{3.465336in}}%
\pgfpathlineto{\pgfqpoint{9.368762in}{3.464768in}}%
\pgfpathlineto{\pgfqpoint{9.369447in}{3.463763in}}%
\pgfpathlineto{\pgfqpoint{9.371501in}{3.459468in}}%
\pgfpathlineto{\pgfqpoint{9.372528in}{3.459022in}}%
\pgfpathlineto{\pgfqpoint{9.374240in}{3.458071in}}%
\pgfpathlineto{\pgfqpoint{9.374925in}{3.456515in}}%
\pgfpathlineto{\pgfqpoint{9.375952in}{3.455503in}}%
\pgfpathlineto{\pgfqpoint{9.376294in}{3.455428in}}%
\pgfpathlineto{\pgfqpoint{9.377321in}{3.454220in}}%
\pgfpathlineto{\pgfqpoint{9.378006in}{3.454144in}}%
\pgfpathlineto{\pgfqpoint{9.378690in}{3.452558in}}%
\pgfpathlineto{\pgfqpoint{9.380402in}{3.451501in}}%
\pgfpathlineto{\pgfqpoint{9.381429in}{3.450821in}}%
\pgfpathlineto{\pgfqpoint{9.381772in}{3.449953in}}%
\pgfpathlineto{\pgfqpoint{9.382456in}{3.449878in}}%
\pgfpathlineto{\pgfqpoint{9.383141in}{3.449122in}}%
\pgfpathlineto{\pgfqpoint{9.383483in}{3.449011in}}%
\pgfpathlineto{\pgfqpoint{9.385195in}{3.447234in}}%
\pgfpathlineto{\pgfqpoint{9.385537in}{3.446960in}}%
\pgfpathlineto{\pgfqpoint{9.386222in}{3.445611in}}%
\pgfpathlineto{\pgfqpoint{9.386564in}{3.445535in}}%
\pgfpathlineto{\pgfqpoint{9.386907in}{3.444289in}}%
\pgfpathlineto{\pgfqpoint{9.387249in}{3.444196in}}%
\pgfpathlineto{\pgfqpoint{9.387934in}{3.443375in}}%
\pgfpathlineto{\pgfqpoint{9.388619in}{3.442930in}}%
\pgfpathlineto{\pgfqpoint{9.388961in}{3.442855in}}%
\pgfpathlineto{\pgfqpoint{9.389646in}{3.441835in}}%
\pgfpathlineto{\pgfqpoint{9.389988in}{3.441722in}}%
\pgfpathlineto{\pgfqpoint{9.390330in}{3.441307in}}%
\pgfpathlineto{\pgfqpoint{9.390673in}{3.439872in}}%
\pgfpathlineto{\pgfqpoint{9.391357in}{3.439645in}}%
\pgfpathlineto{\pgfqpoint{9.392042in}{3.437984in}}%
\pgfpathlineto{\pgfqpoint{9.392727in}{3.437946in}}%
\pgfpathlineto{\pgfqpoint{9.395123in}{3.435303in}}%
\pgfpathlineto{\pgfqpoint{9.395465in}{3.435227in}}%
\pgfpathlineto{\pgfqpoint{9.396492in}{3.434050in}}%
\pgfpathlineto{\pgfqpoint{9.397177in}{3.433727in}}%
\pgfpathlineto{\pgfqpoint{9.397862in}{3.433491in}}%
\pgfpathlineto{\pgfqpoint{9.398889in}{3.433000in}}%
\pgfpathlineto{\pgfqpoint{9.399231in}{3.431678in}}%
\pgfpathlineto{\pgfqpoint{9.399916in}{3.431399in}}%
\pgfpathlineto{\pgfqpoint{9.401285in}{3.429035in}}%
\pgfpathlineto{\pgfqpoint{9.402312in}{3.428366in}}%
\pgfpathlineto{\pgfqpoint{9.402997in}{3.427261in}}%
\pgfpathlineto{\pgfqpoint{9.404024in}{3.426279in}}%
\pgfpathlineto{\pgfqpoint{9.405736in}{3.424391in}}%
\pgfpathlineto{\pgfqpoint{9.409159in}{3.423114in}}%
\pgfpathlineto{\pgfqpoint{9.410871in}{3.418727in}}%
\pgfpathlineto{\pgfqpoint{9.411213in}{3.418623in}}%
\pgfpathlineto{\pgfqpoint{9.411898in}{3.417838in}}%
\pgfpathlineto{\pgfqpoint{9.412240in}{3.417821in}}%
\pgfpathlineto{\pgfqpoint{9.412583in}{3.417459in}}%
\pgfpathlineto{\pgfqpoint{9.413610in}{3.415556in}}%
\pgfpathlineto{\pgfqpoint{9.413952in}{3.415465in}}%
\pgfpathlineto{\pgfqpoint{9.414295in}{3.413781in}}%
\pgfpathlineto{\pgfqpoint{9.415664in}{3.412724in}}%
\pgfpathlineto{\pgfqpoint{9.416006in}{3.412120in}}%
\pgfpathlineto{\pgfqpoint{9.416349in}{3.412120in}}%
\pgfpathlineto{\pgfqpoint{9.417033in}{3.411123in}}%
\pgfpathlineto{\pgfqpoint{9.417376in}{3.410949in}}%
\pgfpathlineto{\pgfqpoint{9.418060in}{3.409605in}}%
\pgfpathlineto{\pgfqpoint{9.418403in}{3.409439in}}%
\pgfpathlineto{\pgfqpoint{9.419087in}{3.407551in}}%
\pgfpathlineto{\pgfqpoint{9.420799in}{3.406184in}}%
\pgfpathlineto{\pgfqpoint{9.421484in}{3.405776in}}%
\pgfpathlineto{\pgfqpoint{9.421826in}{3.405547in}}%
\pgfpathlineto{\pgfqpoint{9.423196in}{3.401963in}}%
\pgfpathlineto{\pgfqpoint{9.423538in}{3.401661in}}%
\pgfpathlineto{\pgfqpoint{9.423880in}{3.400943in}}%
\pgfpathlineto{\pgfqpoint{9.424907in}{3.400648in}}%
\pgfpathlineto{\pgfqpoint{9.427646in}{3.399244in}}%
\pgfpathlineto{\pgfqpoint{9.430042in}{3.397003in}}%
\pgfpathlineto{\pgfqpoint{9.430727in}{3.396728in}}%
\pgfpathlineto{\pgfqpoint{9.432097in}{3.395619in}}%
\pgfpathlineto{\pgfqpoint{9.432781in}{3.394321in}}%
\pgfpathlineto{\pgfqpoint{9.433124in}{3.394109in}}%
\pgfpathlineto{\pgfqpoint{9.433808in}{3.392710in}}%
\pgfpathlineto{\pgfqpoint{9.435178in}{3.392184in}}%
\pgfpathlineto{\pgfqpoint{9.436547in}{3.390371in}}%
\pgfpathlineto{\pgfqpoint{9.436889in}{3.390182in}}%
\pgfpathlineto{\pgfqpoint{9.437232in}{3.389427in}}%
\pgfpathlineto{\pgfqpoint{9.438259in}{3.389125in}}%
\pgfpathlineto{\pgfqpoint{9.439286in}{3.388219in}}%
\pgfpathlineto{\pgfqpoint{9.439971in}{3.387351in}}%
\pgfpathlineto{\pgfqpoint{9.440655in}{3.386618in}}%
\pgfpathlineto{\pgfqpoint{9.441340in}{3.385614in}}%
\pgfpathlineto{\pgfqpoint{9.441682in}{3.385521in}}%
\pgfpathlineto{\pgfqpoint{9.442025in}{3.384489in}}%
\pgfpathlineto{\pgfqpoint{9.442367in}{3.384368in}}%
\pgfpathlineto{\pgfqpoint{9.443394in}{3.382895in}}%
\pgfpathlineto{\pgfqpoint{9.444079in}{3.382178in}}%
\pgfpathlineto{\pgfqpoint{9.444421in}{3.381982in}}%
\pgfpathlineto{\pgfqpoint{9.445106in}{3.380757in}}%
\pgfpathlineto{\pgfqpoint{9.445790in}{3.380252in}}%
\pgfpathlineto{\pgfqpoint{9.446133in}{3.379137in}}%
\pgfpathlineto{\pgfqpoint{9.446475in}{3.379066in}}%
\pgfpathlineto{\pgfqpoint{9.447160in}{3.377987in}}%
\pgfpathlineto{\pgfqpoint{9.447844in}{3.375948in}}%
\pgfpathlineto{\pgfqpoint{9.448529in}{3.375344in}}%
\pgfpathlineto{\pgfqpoint{9.449556in}{3.375041in}}%
\pgfpathlineto{\pgfqpoint{9.452295in}{3.373758in}}%
\pgfpathlineto{\pgfqpoint{9.452637in}{3.372247in}}%
\pgfpathlineto{\pgfqpoint{9.454691in}{3.370926in}}%
\pgfpathlineto{\pgfqpoint{9.457088in}{3.368698in}}%
\pgfpathlineto{\pgfqpoint{9.457430in}{3.368660in}}%
\pgfpathlineto{\pgfqpoint{9.458115in}{3.367943in}}%
\pgfpathlineto{\pgfqpoint{9.458800in}{3.367716in}}%
\pgfpathlineto{\pgfqpoint{9.459142in}{3.367679in}}%
\pgfpathlineto{\pgfqpoint{9.460169in}{3.365954in}}%
\pgfpathlineto{\pgfqpoint{9.461881in}{3.364167in}}%
\pgfpathlineto{\pgfqpoint{9.462565in}{3.363827in}}%
\pgfpathlineto{\pgfqpoint{9.468385in}{3.356389in}}%
\pgfpathlineto{\pgfqpoint{9.470097in}{3.355596in}}%
\pgfpathlineto{\pgfqpoint{9.470782in}{3.354237in}}%
\pgfpathlineto{\pgfqpoint{9.471466in}{3.354010in}}%
\pgfpathlineto{\pgfqpoint{9.471809in}{3.353897in}}%
\pgfpathlineto{\pgfqpoint{9.472493in}{3.353331in}}%
\pgfpathlineto{\pgfqpoint{9.473520in}{3.352394in}}%
\pgfpathlineto{\pgfqpoint{9.474205in}{3.351216in}}%
\pgfpathlineto{\pgfqpoint{9.474890in}{3.350801in}}%
\pgfpathlineto{\pgfqpoint{9.475232in}{3.349752in}}%
\pgfpathlineto{\pgfqpoint{9.476259in}{3.349291in}}%
\pgfpathlineto{\pgfqpoint{9.477971in}{3.346685in}}%
\pgfpathlineto{\pgfqpoint{9.478656in}{3.346588in}}%
\pgfpathlineto{\pgfqpoint{9.479340in}{3.345689in}}%
\pgfpathlineto{\pgfqpoint{9.481052in}{3.342336in}}%
\pgfpathlineto{\pgfqpoint{9.481394in}{3.342126in}}%
\pgfpathlineto{\pgfqpoint{9.482764in}{3.340002in}}%
\pgfpathlineto{\pgfqpoint{9.483106in}{3.339964in}}%
\pgfpathlineto{\pgfqpoint{9.483449in}{3.338228in}}%
\pgfpathlineto{\pgfqpoint{9.484133in}{3.337884in}}%
\pgfpathlineto{\pgfqpoint{9.489268in}{3.328139in}}%
\pgfpathlineto{\pgfqpoint{9.490295in}{3.327550in}}%
\pgfpathlineto{\pgfqpoint{9.490980in}{3.325805in}}%
\pgfpathlineto{\pgfqpoint{9.491323in}{3.325805in}}%
\pgfpathlineto{\pgfqpoint{9.492692in}{3.324272in}}%
\pgfpathlineto{\pgfqpoint{9.493377in}{3.324068in}}%
\pgfpathlineto{\pgfqpoint{9.494061in}{3.323064in}}%
\pgfpathlineto{\pgfqpoint{9.494404in}{3.322762in}}%
\pgfpathlineto{\pgfqpoint{9.494746in}{3.322046in}}%
\pgfpathlineto{\pgfqpoint{9.495088in}{3.322029in}}%
\pgfpathlineto{\pgfqpoint{9.496115in}{3.320390in}}%
\pgfpathlineto{\pgfqpoint{9.497142in}{3.319235in}}%
\pgfpathlineto{\pgfqpoint{9.497827in}{3.317264in}}%
\pgfpathlineto{\pgfqpoint{9.498169in}{3.317159in}}%
\pgfpathlineto{\pgfqpoint{9.498512in}{3.315865in}}%
\pgfpathlineto{\pgfqpoint{9.498854in}{3.315837in}}%
\pgfpathlineto{\pgfqpoint{9.499539in}{3.313232in}}%
\pgfpathlineto{\pgfqpoint{9.500566in}{3.312779in}}%
\pgfpathlineto{\pgfqpoint{9.500908in}{3.311873in}}%
\pgfpathlineto{\pgfqpoint{9.501935in}{3.311382in}}%
\pgfpathlineto{\pgfqpoint{9.502620in}{3.310977in}}%
\pgfpathlineto{\pgfqpoint{9.503647in}{3.310575in}}%
\pgfpathlineto{\pgfqpoint{9.504674in}{3.309871in}}%
\pgfpathlineto{\pgfqpoint{9.505016in}{3.309683in}}%
\pgfpathlineto{\pgfqpoint{9.505701in}{3.307719in}}%
\pgfpathlineto{\pgfqpoint{9.506386in}{3.307190in}}%
\pgfpathlineto{\pgfqpoint{9.507070in}{3.306537in}}%
\pgfpathlineto{\pgfqpoint{9.507755in}{3.304963in}}%
\pgfpathlineto{\pgfqpoint{9.508440in}{3.303264in}}%
\pgfpathlineto{\pgfqpoint{9.509125in}{3.302645in}}%
\pgfpathlineto{\pgfqpoint{9.510836in}{3.300885in}}%
\pgfpathlineto{\pgfqpoint{9.511863in}{3.300711in}}%
\pgfpathlineto{\pgfqpoint{9.512548in}{3.300409in}}%
\pgfpathlineto{\pgfqpoint{9.513575in}{3.300319in}}%
\pgfpathlineto{\pgfqpoint{9.513917in}{3.299450in}}%
\pgfpathlineto{\pgfqpoint{9.514602in}{3.299133in}}%
\pgfpathlineto{\pgfqpoint{9.515287in}{3.297597in}}%
\pgfpathlineto{\pgfqpoint{9.515629in}{3.297336in}}%
\pgfpathlineto{\pgfqpoint{9.517341in}{3.293787in}}%
\pgfpathlineto{\pgfqpoint{9.518026in}{3.293583in}}%
\pgfpathlineto{\pgfqpoint{9.518710in}{3.292314in}}%
\pgfpathlineto{\pgfqpoint{9.519053in}{3.292314in}}%
\pgfpathlineto{\pgfqpoint{9.519737in}{3.290993in}}%
\pgfpathlineto{\pgfqpoint{9.521107in}{3.288236in}}%
\pgfpathlineto{\pgfqpoint{9.521449in}{3.288010in}}%
\pgfpathlineto{\pgfqpoint{9.522134in}{3.287005in}}%
\pgfpathlineto{\pgfqpoint{9.522476in}{3.286792in}}%
\pgfpathlineto{\pgfqpoint{9.523161in}{3.285178in}}%
\pgfpathlineto{\pgfqpoint{9.523845in}{3.284876in}}%
\pgfpathlineto{\pgfqpoint{9.524530in}{3.283798in}}%
\pgfpathlineto{\pgfqpoint{9.525215in}{3.283554in}}%
\pgfpathlineto{\pgfqpoint{9.525900in}{3.282799in}}%
\pgfpathlineto{\pgfqpoint{9.526584in}{3.281620in}}%
\pgfpathlineto{\pgfqpoint{9.527269in}{3.281553in}}%
\pgfpathlineto{\pgfqpoint{9.527611in}{3.280805in}}%
\pgfpathlineto{\pgfqpoint{9.527954in}{3.280790in}}%
\pgfpathlineto{\pgfqpoint{9.528638in}{3.279401in}}%
\pgfpathlineto{\pgfqpoint{9.530692in}{3.277475in}}%
\pgfpathlineto{\pgfqpoint{9.531035in}{3.276078in}}%
\pgfpathlineto{\pgfqpoint{9.533089in}{3.274837in}}%
\pgfpathlineto{\pgfqpoint{9.533774in}{3.273734in}}%
\pgfpathlineto{\pgfqpoint{9.534458in}{3.273406in}}%
\pgfpathlineto{\pgfqpoint{9.535143in}{3.273017in}}%
\pgfpathlineto{\pgfqpoint{9.535485in}{3.272869in}}%
\pgfpathlineto{\pgfqpoint{9.537197in}{3.270324in}}%
\pgfpathlineto{\pgfqpoint{9.538224in}{3.269992in}}%
\pgfpathlineto{\pgfqpoint{9.539251in}{3.269169in}}%
\pgfpathlineto{\pgfqpoint{9.539593in}{3.267771in}}%
\pgfpathlineto{\pgfqpoint{9.540278in}{3.267620in}}%
\pgfpathlineto{\pgfqpoint{9.540963in}{3.267055in}}%
\pgfpathlineto{\pgfqpoint{9.542332in}{3.265808in}}%
\pgfpathlineto{\pgfqpoint{9.543017in}{3.264222in}}%
\pgfpathlineto{\pgfqpoint{9.544044in}{3.263618in}}%
\pgfpathlineto{\pgfqpoint{9.544386in}{3.263014in}}%
\pgfpathlineto{\pgfqpoint{9.545071in}{3.262874in}}%
\pgfpathlineto{\pgfqpoint{9.546098in}{3.261579in}}%
\pgfpathlineto{\pgfqpoint{9.546440in}{3.261544in}}%
\pgfpathlineto{\pgfqpoint{9.546783in}{3.259903in}}%
\pgfpathlineto{\pgfqpoint{9.547125in}{3.259691in}}%
\pgfpathlineto{\pgfqpoint{9.547810in}{3.258867in}}%
\pgfpathlineto{\pgfqpoint{9.548494in}{3.258483in}}%
\pgfpathlineto{\pgfqpoint{9.548837in}{3.257237in}}%
\pgfpathlineto{\pgfqpoint{9.549179in}{3.257199in}}%
\pgfpathlineto{\pgfqpoint{9.550548in}{3.255527in}}%
\pgfpathlineto{\pgfqpoint{9.550891in}{3.254254in}}%
\pgfpathlineto{\pgfqpoint{9.551233in}{3.254216in}}%
\pgfpathlineto{\pgfqpoint{9.551918in}{3.253726in}}%
\pgfpathlineto{\pgfqpoint{9.552260in}{3.253669in}}%
\pgfpathlineto{\pgfqpoint{9.552945in}{3.253197in}}%
\pgfpathlineto{\pgfqpoint{9.553287in}{3.253046in}}%
\pgfpathlineto{\pgfqpoint{9.553972in}{3.250911in}}%
\pgfpathlineto{\pgfqpoint{9.554314in}{3.250232in}}%
\pgfpathlineto{\pgfqpoint{9.554657in}{3.250176in}}%
\pgfpathlineto{\pgfqpoint{9.555341in}{3.249723in}}%
\pgfpathlineto{\pgfqpoint{9.559450in}{3.247269in}}%
\pgfpathlineto{\pgfqpoint{9.561846in}{3.243516in}}%
\pgfpathlineto{\pgfqpoint{9.565269in}{3.241658in}}%
\pgfpathlineto{\pgfqpoint{9.566296in}{3.239491in}}%
\pgfpathlineto{\pgfqpoint{9.567323in}{3.239113in}}%
\pgfpathlineto{\pgfqpoint{9.568008in}{3.238509in}}%
\pgfpathlineto{\pgfqpoint{9.568693in}{3.237867in}}%
\pgfpathlineto{\pgfqpoint{9.569378in}{3.236795in}}%
\pgfpathlineto{\pgfqpoint{9.570062in}{3.236513in}}%
\pgfpathlineto{\pgfqpoint{9.570747in}{3.235904in}}%
\pgfpathlineto{\pgfqpoint{9.571432in}{3.233261in}}%
\pgfpathlineto{\pgfqpoint{9.572459in}{3.231855in}}%
\pgfpathlineto{\pgfqpoint{9.572801in}{3.230769in}}%
\pgfpathlineto{\pgfqpoint{9.573143in}{3.230618in}}%
\pgfpathlineto{\pgfqpoint{9.574170in}{3.228201in}}%
\pgfpathlineto{\pgfqpoint{9.574513in}{3.228088in}}%
\pgfpathlineto{\pgfqpoint{9.574855in}{3.227069in}}%
\pgfpathlineto{\pgfqpoint{9.576567in}{3.226578in}}%
\pgfpathlineto{\pgfqpoint{9.577252in}{3.225582in}}%
\pgfpathlineto{\pgfqpoint{9.577594in}{3.225445in}}%
\pgfpathlineto{\pgfqpoint{9.578621in}{3.224274in}}%
\pgfpathlineto{\pgfqpoint{9.579990in}{3.223935in}}%
\pgfpathlineto{\pgfqpoint{9.580675in}{3.222991in}}%
\pgfpathlineto{\pgfqpoint{9.581702in}{3.219819in}}%
\pgfpathlineto{\pgfqpoint{9.582387in}{3.219479in}}%
\pgfpathlineto{\pgfqpoint{9.582729in}{3.218762in}}%
\pgfpathlineto{\pgfqpoint{9.583414in}{3.218732in}}%
\pgfpathlineto{\pgfqpoint{9.585126in}{3.216708in}}%
\pgfpathlineto{\pgfqpoint{9.586153in}{3.215288in}}%
\pgfpathlineto{\pgfqpoint{9.586495in}{3.215024in}}%
\pgfpathlineto{\pgfqpoint{9.586837in}{3.214042in}}%
\pgfpathlineto{\pgfqpoint{9.587180in}{3.211021in}}%
\pgfpathlineto{\pgfqpoint{9.587864in}{3.210830in}}%
\pgfpathlineto{\pgfqpoint{9.588549in}{3.210028in}}%
\pgfpathlineto{\pgfqpoint{9.589234in}{3.209473in}}%
\pgfpathlineto{\pgfqpoint{9.589918in}{3.208879in}}%
\pgfpathlineto{\pgfqpoint{9.590603in}{3.206075in}}%
\pgfpathlineto{\pgfqpoint{9.590945in}{3.205849in}}%
\pgfpathlineto{\pgfqpoint{9.591630in}{3.204905in}}%
\pgfpathlineto{\pgfqpoint{9.592315in}{3.204301in}}%
\pgfpathlineto{\pgfqpoint{9.592657in}{3.204142in}}%
\pgfpathlineto{\pgfqpoint{9.593342in}{3.203545in}}%
\pgfpathlineto{\pgfqpoint{9.594027in}{3.203470in}}%
\pgfpathlineto{\pgfqpoint{9.595054in}{3.202429in}}%
\pgfpathlineto{\pgfqpoint{9.596765in}{3.201393in}}%
\pgfpathlineto{\pgfqpoint{9.597108in}{3.199770in}}%
\pgfpathlineto{\pgfqpoint{9.597450in}{3.199688in}}%
\pgfpathlineto{\pgfqpoint{9.598135in}{3.197472in}}%
\pgfpathlineto{\pgfqpoint{9.598819in}{3.197247in}}%
\pgfpathlineto{\pgfqpoint{9.599504in}{3.196598in}}%
\pgfpathlineto{\pgfqpoint{9.599846in}{3.196490in}}%
\pgfpathlineto{\pgfqpoint{9.600873in}{3.195465in}}%
\pgfpathlineto{\pgfqpoint{9.601216in}{3.195427in}}%
\pgfpathlineto{\pgfqpoint{9.601558in}{3.194257in}}%
\pgfpathlineto{\pgfqpoint{9.602243in}{3.194083in}}%
\pgfpathlineto{\pgfqpoint{9.602928in}{3.191123in}}%
\pgfpathlineto{\pgfqpoint{9.603955in}{3.190783in}}%
\pgfpathlineto{\pgfqpoint{9.604639in}{3.189726in}}%
\pgfpathlineto{\pgfqpoint{9.604982in}{3.188163in}}%
\pgfpathlineto{\pgfqpoint{9.607036in}{3.187385in}}%
\pgfpathlineto{\pgfqpoint{9.609432in}{3.182892in}}%
\pgfpathlineto{\pgfqpoint{9.609774in}{3.182854in}}%
\pgfpathlineto{\pgfqpoint{9.610459in}{3.181864in}}%
\pgfpathlineto{\pgfqpoint{9.616964in}{3.175235in}}%
\pgfpathlineto{\pgfqpoint{9.617648in}{3.175151in}}%
\pgfpathlineto{\pgfqpoint{9.621072in}{3.170009in}}%
\pgfpathlineto{\pgfqpoint{9.621414in}{3.169545in}}%
\pgfpathlineto{\pgfqpoint{9.622099in}{3.165962in}}%
\pgfpathlineto{\pgfqpoint{9.623126in}{3.165599in}}%
\pgfpathlineto{\pgfqpoint{9.623468in}{3.164919in}}%
\pgfpathlineto{\pgfqpoint{9.624495in}{3.164662in}}%
\pgfpathlineto{\pgfqpoint{9.625522in}{3.164315in}}%
\pgfpathlineto{\pgfqpoint{9.626207in}{3.164164in}}%
\pgfpathlineto{\pgfqpoint{9.626892in}{3.163598in}}%
\pgfpathlineto{\pgfqpoint{9.627234in}{3.163527in}}%
\pgfpathlineto{\pgfqpoint{9.627577in}{3.163145in}}%
\pgfpathlineto{\pgfqpoint{9.628261in}{3.162058in}}%
\pgfpathlineto{\pgfqpoint{9.628946in}{3.160728in}}%
\pgfpathlineto{\pgfqpoint{9.629288in}{3.160652in}}%
\pgfpathlineto{\pgfqpoint{9.629973in}{3.159938in}}%
\pgfpathlineto{\pgfqpoint{9.630658in}{3.157745in}}%
\pgfpathlineto{\pgfqpoint{9.632027in}{3.157323in}}%
\pgfpathlineto{\pgfqpoint{9.632712in}{3.156801in}}%
\pgfpathlineto{\pgfqpoint{9.633396in}{3.155178in}}%
\pgfpathlineto{\pgfqpoint{9.634081in}{3.154996in}}%
\pgfpathlineto{\pgfqpoint{9.634423in}{3.154309in}}%
\pgfpathlineto{\pgfqpoint{9.634766in}{3.152265in}}%
\pgfpathlineto{\pgfqpoint{9.635793in}{3.151692in}}%
\pgfpathlineto{\pgfqpoint{9.637505in}{3.148155in}}%
\pgfpathlineto{\pgfqpoint{9.638189in}{3.146536in}}%
\pgfpathlineto{\pgfqpoint{9.639216in}{3.146055in}}%
\pgfpathlineto{\pgfqpoint{9.639559in}{3.146002in}}%
\pgfpathlineto{\pgfqpoint{9.640243in}{3.145625in}}%
\pgfpathlineto{\pgfqpoint{9.640586in}{3.145501in}}%
\pgfpathlineto{\pgfqpoint{9.641270in}{3.144763in}}%
\pgfpathlineto{\pgfqpoint{9.641613in}{3.144605in}}%
\pgfpathlineto{\pgfqpoint{9.643324in}{3.142004in}}%
\pgfpathlineto{\pgfqpoint{9.643667in}{3.141962in}}%
\pgfpathlineto{\pgfqpoint{9.644009in}{3.141132in}}%
\pgfpathlineto{\pgfqpoint{9.644694in}{3.140830in}}%
\pgfpathlineto{\pgfqpoint{9.645036in}{3.140739in}}%
\pgfpathlineto{\pgfqpoint{9.646406in}{3.138791in}}%
\pgfpathlineto{\pgfqpoint{9.647090in}{3.138300in}}%
\pgfpathlineto{\pgfqpoint{9.647775in}{3.136110in}}%
\pgfpathlineto{\pgfqpoint{9.648460in}{3.135845in}}%
\pgfpathlineto{\pgfqpoint{9.649144in}{3.134436in}}%
\pgfpathlineto{\pgfqpoint{9.649829in}{3.133657in}}%
\pgfpathlineto{\pgfqpoint{9.650856in}{3.133315in}}%
\pgfpathlineto{\pgfqpoint{9.651198in}{3.132976in}}%
\pgfpathlineto{\pgfqpoint{9.651541in}{3.131713in}}%
\pgfpathlineto{\pgfqpoint{9.652225in}{3.131587in}}%
\pgfpathlineto{\pgfqpoint{9.652910in}{3.130635in}}%
\pgfpathlineto{\pgfqpoint{9.654280in}{3.129729in}}%
\pgfpathlineto{\pgfqpoint{9.655649in}{3.128636in}}%
\pgfpathlineto{\pgfqpoint{9.656334in}{3.127599in}}%
\pgfpathlineto{\pgfqpoint{9.656676in}{3.127388in}}%
\pgfpathlineto{\pgfqpoint{9.657361in}{3.126029in}}%
\pgfpathlineto{\pgfqpoint{9.657703in}{3.125991in}}%
\pgfpathlineto{\pgfqpoint{9.658045in}{3.123083in}}%
\pgfpathlineto{\pgfqpoint{9.658388in}{3.123072in}}%
\pgfpathlineto{\pgfqpoint{9.658730in}{3.122632in}}%
\pgfpathlineto{\pgfqpoint{9.659072in}{3.121686in}}%
\pgfpathlineto{\pgfqpoint{9.659415in}{3.121686in}}%
\pgfpathlineto{\pgfqpoint{9.660099in}{3.120063in}}%
\pgfpathlineto{\pgfqpoint{9.660784in}{3.119762in}}%
\pgfpathlineto{\pgfqpoint{9.661126in}{3.119194in}}%
\pgfpathlineto{\pgfqpoint{9.661469in}{3.119172in}}%
\pgfpathlineto{\pgfqpoint{9.661811in}{3.118666in}}%
\pgfpathlineto{\pgfqpoint{9.662154in}{3.118666in}}%
\pgfpathlineto{\pgfqpoint{9.663181in}{3.117517in}}%
\pgfpathlineto{\pgfqpoint{9.663523in}{3.117359in}}%
\pgfpathlineto{\pgfqpoint{9.664208in}{3.116174in}}%
\pgfpathlineto{\pgfqpoint{9.665577in}{3.114966in}}%
\pgfpathlineto{\pgfqpoint{9.665919in}{3.114042in}}%
\pgfpathlineto{\pgfqpoint{9.666604in}{3.113908in}}%
\pgfpathlineto{\pgfqpoint{9.667289in}{3.112342in}}%
\pgfpathlineto{\pgfqpoint{9.667631in}{3.111983in}}%
\pgfpathlineto{\pgfqpoint{9.668316in}{3.109743in}}%
\pgfpathlineto{\pgfqpoint{9.669000in}{3.109189in}}%
\pgfpathlineto{\pgfqpoint{9.670370in}{3.108169in}}%
\pgfpathlineto{\pgfqpoint{9.670712in}{3.107225in}}%
\pgfpathlineto{\pgfqpoint{9.671397in}{3.107119in}}%
\pgfpathlineto{\pgfqpoint{9.675163in}{3.103253in}}%
\pgfpathlineto{\pgfqpoint{9.677901in}{3.101675in}}%
\pgfpathlineto{\pgfqpoint{9.678929in}{3.097484in}}%
\pgfpathlineto{\pgfqpoint{9.679613in}{3.095641in}}%
\pgfpathlineto{\pgfqpoint{9.681325in}{3.094425in}}%
\pgfpathlineto{\pgfqpoint{9.682010in}{3.092468in}}%
\pgfpathlineto{\pgfqpoint{9.683037in}{3.091858in}}%
\pgfpathlineto{\pgfqpoint{9.683721in}{3.090199in}}%
\pgfpathlineto{\pgfqpoint{9.685091in}{3.089026in}}%
\pgfpathlineto{\pgfqpoint{9.685775in}{3.088120in}}%
\pgfpathlineto{\pgfqpoint{9.686118in}{3.088044in}}%
\pgfpathlineto{\pgfqpoint{9.686802in}{3.087440in}}%
\pgfpathlineto{\pgfqpoint{9.687145in}{3.087402in}}%
\pgfpathlineto{\pgfqpoint{9.687830in}{3.086723in}}%
\pgfpathlineto{\pgfqpoint{9.688172in}{3.085286in}}%
\pgfpathlineto{\pgfqpoint{9.689541in}{3.084495in}}%
\pgfpathlineto{\pgfqpoint{9.690226in}{3.083891in}}%
\pgfpathlineto{\pgfqpoint{9.690568in}{3.083853in}}%
\pgfpathlineto{\pgfqpoint{9.690911in}{3.081588in}}%
\pgfpathlineto{\pgfqpoint{9.691253in}{3.081383in}}%
\pgfpathlineto{\pgfqpoint{9.691938in}{3.079390in}}%
\pgfpathlineto{\pgfqpoint{9.692280in}{3.079096in}}%
\pgfpathlineto{\pgfqpoint{9.692622in}{3.077321in}}%
\pgfpathlineto{\pgfqpoint{9.692965in}{3.077167in}}%
\pgfpathlineto{\pgfqpoint{9.693307in}{3.076320in}}%
\pgfpathlineto{\pgfqpoint{9.693649in}{3.076218in}}%
\pgfpathlineto{\pgfqpoint{9.695361in}{3.072261in}}%
\pgfpathlineto{\pgfqpoint{9.695703in}{3.072224in}}%
\pgfpathlineto{\pgfqpoint{9.696388in}{3.070449in}}%
\pgfpathlineto{\pgfqpoint{9.697073in}{3.070017in}}%
\pgfpathlineto{\pgfqpoint{9.698100in}{3.068002in}}%
\pgfpathlineto{\pgfqpoint{9.698785in}{3.067315in}}%
\pgfpathlineto{\pgfqpoint{9.699469in}{3.066409in}}%
\pgfpathlineto{\pgfqpoint{9.699812in}{3.065676in}}%
\pgfpathlineto{\pgfqpoint{9.700154in}{3.065616in}}%
\pgfpathlineto{\pgfqpoint{9.700839in}{3.064559in}}%
\pgfpathlineto{\pgfqpoint{9.701866in}{3.064294in}}%
\pgfpathlineto{\pgfqpoint{9.702550in}{3.063381in}}%
\pgfpathlineto{\pgfqpoint{9.703235in}{3.062709in}}%
\pgfpathlineto{\pgfqpoint{9.703577in}{3.061463in}}%
\pgfpathlineto{\pgfqpoint{9.704947in}{3.060821in}}%
\pgfpathlineto{\pgfqpoint{9.707343in}{3.057581in}}%
\pgfpathlineto{\pgfqpoint{9.708713in}{3.056214in}}%
\pgfpathlineto{\pgfqpoint{9.709055in}{3.056177in}}%
\pgfpathlineto{\pgfqpoint{9.709740in}{3.055542in}}%
\pgfpathlineto{\pgfqpoint{9.710767in}{3.054440in}}%
\pgfpathlineto{\pgfqpoint{9.711451in}{3.054364in}}%
\pgfpathlineto{\pgfqpoint{9.712136in}{3.053005in}}%
\pgfpathlineto{\pgfqpoint{9.713506in}{3.051812in}}%
\pgfpathlineto{\pgfqpoint{9.713848in}{3.051747in}}%
\pgfpathlineto{\pgfqpoint{9.715902in}{3.048972in}}%
\pgfpathlineto{\pgfqpoint{9.716587in}{3.046616in}}%
\pgfpathlineto{\pgfqpoint{9.717271in}{3.044962in}}%
\pgfpathlineto{\pgfqpoint{9.717614in}{3.044925in}}%
\pgfpathlineto{\pgfqpoint{9.718298in}{3.043528in}}%
\pgfpathlineto{\pgfqpoint{9.719325in}{3.043037in}}%
\pgfpathlineto{\pgfqpoint{9.720010in}{3.041317in}}%
\pgfpathlineto{\pgfqpoint{9.721037in}{3.040545in}}%
\pgfpathlineto{\pgfqpoint{9.721722in}{3.038846in}}%
\pgfpathlineto{\pgfqpoint{9.722407in}{3.038695in}}%
\pgfpathlineto{\pgfqpoint{9.722749in}{3.038331in}}%
\pgfpathlineto{\pgfqpoint{9.723776in}{3.035825in}}%
\pgfpathlineto{\pgfqpoint{9.724461in}{3.035372in}}%
\pgfpathlineto{\pgfqpoint{9.725830in}{3.034258in}}%
\pgfpathlineto{\pgfqpoint{9.726515in}{3.032087in}}%
\pgfpathlineto{\pgfqpoint{9.727199in}{3.029912in}}%
\pgfpathlineto{\pgfqpoint{9.728226in}{3.029406in}}%
\pgfpathlineto{\pgfqpoint{9.728569in}{3.029368in}}%
\pgfpathlineto{\pgfqpoint{9.729253in}{3.028840in}}%
\pgfpathlineto{\pgfqpoint{9.729938in}{3.028500in}}%
\pgfpathlineto{\pgfqpoint{9.731992in}{3.027224in}}%
\pgfpathlineto{\pgfqpoint{9.732677in}{3.027076in}}%
\pgfpathlineto{\pgfqpoint{9.733704in}{3.025291in}}%
\pgfpathlineto{\pgfqpoint{9.734046in}{3.025185in}}%
\pgfpathlineto{\pgfqpoint{9.734731in}{3.024519in}}%
\pgfpathlineto{\pgfqpoint{9.735073in}{3.024254in}}%
\pgfpathlineto{\pgfqpoint{9.736100in}{3.021553in}}%
\pgfpathlineto{\pgfqpoint{9.736785in}{3.021024in}}%
\pgfpathlineto{\pgfqpoint{9.737127in}{3.019601in}}%
\pgfpathlineto{\pgfqpoint{9.737812in}{3.019394in}}%
\pgfpathlineto{\pgfqpoint{9.738497in}{3.018985in}}%
\pgfpathlineto{\pgfqpoint{9.738839in}{3.018910in}}%
\pgfpathlineto{\pgfqpoint{9.739182in}{3.018292in}}%
\pgfpathlineto{\pgfqpoint{9.739866in}{3.018278in}}%
\pgfpathlineto{\pgfqpoint{9.740551in}{3.015927in}}%
\pgfpathlineto{\pgfqpoint{9.740893in}{3.015866in}}%
\pgfpathlineto{\pgfqpoint{9.741578in}{3.014978in}}%
\pgfpathlineto{\pgfqpoint{9.742605in}{3.014357in}}%
\pgfpathlineto{\pgfqpoint{9.746028in}{3.009848in}}%
\pgfpathlineto{\pgfqpoint{9.747056in}{3.009032in}}%
\pgfpathlineto{\pgfqpoint{9.748083in}{3.008413in}}%
\pgfpathlineto{\pgfqpoint{9.748425in}{3.007529in}}%
\pgfpathlineto{\pgfqpoint{9.749452in}{3.007299in}}%
\pgfpathlineto{\pgfqpoint{9.750479in}{3.005898in}}%
\pgfpathlineto{\pgfqpoint{9.750821in}{3.005767in}}%
\pgfpathlineto{\pgfqpoint{9.751506in}{3.005181in}}%
\pgfpathlineto{\pgfqpoint{9.752875in}{3.004675in}}%
\pgfpathlineto{\pgfqpoint{9.753560in}{3.002032in}}%
\pgfpathlineto{\pgfqpoint{9.754245in}{3.001692in}}%
\pgfpathlineto{\pgfqpoint{9.755614in}{3.000378in}}%
\pgfpathlineto{\pgfqpoint{9.755957in}{3.000333in}}%
\pgfpathlineto{\pgfqpoint{9.756984in}{2.999578in}}%
\pgfpathlineto{\pgfqpoint{9.757326in}{2.999511in}}%
\pgfpathlineto{\pgfqpoint{9.758011in}{2.998143in}}%
\pgfpathlineto{\pgfqpoint{9.759722in}{2.994904in}}%
\pgfpathlineto{\pgfqpoint{9.760749in}{2.994896in}}%
\pgfpathlineto{\pgfqpoint{9.762803in}{2.990010in}}%
\pgfpathlineto{\pgfqpoint{9.764173in}{2.988379in}}%
\pgfpathlineto{\pgfqpoint{9.764858in}{2.988250in}}%
\pgfpathlineto{\pgfqpoint{9.765542in}{2.987346in}}%
\pgfpathlineto{\pgfqpoint{9.765885in}{2.986853in}}%
\pgfpathlineto{\pgfqpoint{9.766569in}{2.984416in}}%
\pgfpathlineto{\pgfqpoint{9.767254in}{2.984210in}}%
\pgfpathlineto{\pgfqpoint{9.767939in}{2.983647in}}%
\pgfpathlineto{\pgfqpoint{9.768281in}{2.983421in}}%
\pgfpathlineto{\pgfqpoint{9.769308in}{2.981401in}}%
\pgfpathlineto{\pgfqpoint{9.770335in}{2.980797in}}%
\pgfpathlineto{\pgfqpoint{9.771020in}{2.979633in}}%
\pgfpathlineto{\pgfqpoint{9.772389in}{2.978936in}}%
\pgfpathlineto{\pgfqpoint{9.773074in}{2.977740in}}%
\pgfpathlineto{\pgfqpoint{9.773759in}{2.977527in}}%
\pgfpathlineto{\pgfqpoint{9.774443in}{2.975903in}}%
\pgfpathlineto{\pgfqpoint{9.775128in}{2.975292in}}%
\pgfpathlineto{\pgfqpoint{9.775813in}{2.975027in}}%
\pgfpathlineto{\pgfqpoint{9.777524in}{2.973018in}}%
\pgfpathlineto{\pgfqpoint{9.778209in}{2.972958in}}%
\pgfpathlineto{\pgfqpoint{9.779236in}{2.970806in}}%
\pgfpathlineto{\pgfqpoint{9.779921in}{2.969832in}}%
\pgfpathlineto{\pgfqpoint{9.780605in}{2.969145in}}%
\pgfpathlineto{\pgfqpoint{9.781290in}{2.968261in}}%
\pgfpathlineto{\pgfqpoint{9.782660in}{2.966315in}}%
\pgfpathlineto{\pgfqpoint{9.783002in}{2.966275in}}%
\pgfpathlineto{\pgfqpoint{9.784029in}{2.965348in}}%
\pgfpathlineto{\pgfqpoint{9.784371in}{2.965234in}}%
\pgfpathlineto{\pgfqpoint{9.786083in}{2.962169in}}%
\pgfpathlineto{\pgfqpoint{9.788479in}{2.960347in}}%
\pgfpathlineto{\pgfqpoint{9.788822in}{2.958648in}}%
\pgfpathlineto{\pgfqpoint{9.789164in}{2.958426in}}%
\pgfpathlineto{\pgfqpoint{9.789506in}{2.957819in}}%
\pgfpathlineto{\pgfqpoint{9.789849in}{2.956571in}}%
\pgfpathlineto{\pgfqpoint{9.790191in}{2.956560in}}%
\pgfpathlineto{\pgfqpoint{9.790876in}{2.955401in}}%
\pgfpathlineto{\pgfqpoint{9.791218in}{2.955363in}}%
\pgfpathlineto{\pgfqpoint{9.791903in}{2.954457in}}%
\pgfpathlineto{\pgfqpoint{9.793272in}{2.953279in}}%
\pgfpathlineto{\pgfqpoint{9.793957in}{2.952222in}}%
\pgfpathlineto{\pgfqpoint{9.797038in}{2.950455in}}%
\pgfpathlineto{\pgfqpoint{9.797723in}{2.949624in}}%
\pgfpathlineto{\pgfqpoint{9.798408in}{2.949284in}}%
\pgfpathlineto{\pgfqpoint{9.799092in}{2.947510in}}%
\pgfpathlineto{\pgfqpoint{9.800119in}{2.947132in}}%
\pgfpathlineto{\pgfqpoint{9.801146in}{2.945393in}}%
\pgfpathlineto{\pgfqpoint{9.801489in}{2.945357in}}%
\pgfpathlineto{\pgfqpoint{9.802516in}{2.943641in}}%
\pgfpathlineto{\pgfqpoint{9.803543in}{2.943311in}}%
\pgfpathlineto{\pgfqpoint{9.804570in}{2.942103in}}%
\pgfpathlineto{\pgfqpoint{9.805254in}{2.940557in}}%
\pgfpathlineto{\pgfqpoint{9.805597in}{2.940298in}}%
\pgfpathlineto{\pgfqpoint{9.807309in}{2.937209in}}%
\pgfpathlineto{\pgfqpoint{9.807993in}{2.937051in}}%
\pgfpathlineto{\pgfqpoint{9.809363in}{2.935643in}}%
\pgfpathlineto{\pgfqpoint{9.810047in}{2.934521in}}%
\pgfpathlineto{\pgfqpoint{9.812101in}{2.932784in}}%
\pgfpathlineto{\pgfqpoint{9.812786in}{2.930738in}}%
\pgfpathlineto{\pgfqpoint{9.813813in}{2.929929in}}%
\pgfpathlineto{\pgfqpoint{9.814498in}{2.929416in}}%
\pgfpathlineto{\pgfqpoint{9.815182in}{2.928970in}}%
\pgfpathlineto{\pgfqpoint{9.815867in}{2.928509in}}%
\pgfpathlineto{\pgfqpoint{9.818264in}{2.927887in}}%
\pgfpathlineto{\pgfqpoint{9.820660in}{2.923911in}}%
\pgfpathlineto{\pgfqpoint{9.821687in}{2.923560in}}%
\pgfpathlineto{\pgfqpoint{9.822029in}{2.923344in}}%
\pgfpathlineto{\pgfqpoint{9.822714in}{2.922363in}}%
\pgfpathlineto{\pgfqpoint{9.823399in}{2.921778in}}%
\pgfpathlineto{\pgfqpoint{9.823741in}{2.920717in}}%
\pgfpathlineto{\pgfqpoint{9.824426in}{2.920513in}}%
\pgfpathlineto{\pgfqpoint{9.825111in}{2.919795in}}%
\pgfpathlineto{\pgfqpoint{9.826138in}{2.919078in}}%
\pgfpathlineto{\pgfqpoint{9.826480in}{2.918754in}}%
\pgfpathlineto{\pgfqpoint{9.828192in}{2.914773in}}%
\pgfpathlineto{\pgfqpoint{9.828876in}{2.914509in}}%
\pgfpathlineto{\pgfqpoint{9.829219in}{2.913677in}}%
\pgfpathlineto{\pgfqpoint{9.829561in}{2.913641in}}%
\pgfpathlineto{\pgfqpoint{9.829903in}{2.912508in}}%
\pgfpathlineto{\pgfqpoint{9.830246in}{2.912414in}}%
\pgfpathlineto{\pgfqpoint{9.830930in}{2.911081in}}%
\pgfpathlineto{\pgfqpoint{9.831273in}{2.910658in}}%
\pgfpathlineto{\pgfqpoint{9.831615in}{2.909676in}}%
\pgfpathlineto{\pgfqpoint{9.832300in}{2.909450in}}%
\pgfpathlineto{\pgfqpoint{9.832642in}{2.908166in}}%
\pgfpathlineto{\pgfqpoint{9.832985in}{2.908120in}}%
\pgfpathlineto{\pgfqpoint{9.834354in}{2.906580in}}%
\pgfpathlineto{\pgfqpoint{9.834696in}{2.903952in}}%
\pgfpathlineto{\pgfqpoint{9.835381in}{2.903439in}}%
\pgfpathlineto{\pgfqpoint{9.836066in}{2.902079in}}%
\pgfpathlineto{\pgfqpoint{9.837093in}{2.900629in}}%
\pgfpathlineto{\pgfqpoint{9.837435in}{2.900551in}}%
\pgfpathlineto{\pgfqpoint{9.838120in}{2.898802in}}%
\pgfpathlineto{\pgfqpoint{9.839147in}{2.898394in}}%
\pgfpathlineto{\pgfqpoint{9.839831in}{2.897543in}}%
\pgfpathlineto{\pgfqpoint{9.840174in}{2.897484in}}%
\pgfpathlineto{\pgfqpoint{9.840516in}{2.896889in}}%
\pgfpathlineto{\pgfqpoint{9.840858in}{2.896854in}}%
\pgfpathlineto{\pgfqpoint{9.841201in}{2.895064in}}%
\pgfpathlineto{\pgfqpoint{9.842228in}{2.894800in}}%
\pgfpathlineto{\pgfqpoint{9.842913in}{2.894316in}}%
\pgfpathlineto{\pgfqpoint{9.843597in}{2.892836in}}%
\pgfpathlineto{\pgfqpoint{9.844282in}{2.892761in}}%
\pgfpathlineto{\pgfqpoint{9.845651in}{2.892043in}}%
\pgfpathlineto{\pgfqpoint{9.845994in}{2.892043in}}%
\pgfpathlineto{\pgfqpoint{9.847021in}{2.890457in}}%
\pgfpathlineto{\pgfqpoint{9.848048in}{2.890004in}}%
\pgfpathlineto{\pgfqpoint{9.850102in}{2.886455in}}%
\pgfpathlineto{\pgfqpoint{9.850444in}{2.886417in}}%
\pgfpathlineto{\pgfqpoint{9.851129in}{2.884794in}}%
\pgfpathlineto{\pgfqpoint{9.851471in}{2.883208in}}%
\pgfpathlineto{\pgfqpoint{9.852498in}{2.882446in}}%
\pgfpathlineto{\pgfqpoint{9.853183in}{2.881086in}}%
\pgfpathlineto{\pgfqpoint{9.853525in}{2.880980in}}%
\pgfpathlineto{\pgfqpoint{9.854210in}{2.880278in}}%
\pgfpathlineto{\pgfqpoint{9.855579in}{2.878941in}}%
\pgfpathlineto{\pgfqpoint{9.855922in}{2.878859in}}%
\pgfpathlineto{\pgfqpoint{9.856606in}{2.877969in}}%
\pgfpathlineto{\pgfqpoint{9.856949in}{2.877642in}}%
\pgfpathlineto{\pgfqpoint{9.857633in}{2.876449in}}%
\pgfpathlineto{\pgfqpoint{9.860030in}{2.874901in}}%
\pgfpathlineto{\pgfqpoint{9.861742in}{2.874486in}}%
\pgfpathlineto{\pgfqpoint{9.862769in}{2.872754in}}%
\pgfpathlineto{\pgfqpoint{9.864823in}{2.870597in}}%
\pgfpathlineto{\pgfqpoint{9.865850in}{2.870151in}}%
\pgfpathlineto{\pgfqpoint{9.868246in}{2.867199in}}%
\pgfpathlineto{\pgfqpoint{9.868931in}{2.866617in}}%
\pgfpathlineto{\pgfqpoint{9.869616in}{2.865530in}}%
\pgfpathlineto{\pgfqpoint{9.870300in}{2.865386in}}%
\pgfpathlineto{\pgfqpoint{9.870985in}{2.864524in}}%
\pgfpathlineto{\pgfqpoint{9.871670in}{2.863800in}}%
\pgfpathlineto{\pgfqpoint{9.872012in}{2.862932in}}%
\pgfpathlineto{\pgfqpoint{9.872354in}{2.862901in}}%
\pgfpathlineto{\pgfqpoint{9.873381in}{2.861975in}}%
\pgfpathlineto{\pgfqpoint{9.873724in}{2.861663in}}%
\pgfpathlineto{\pgfqpoint{9.874408in}{2.859939in}}%
\pgfpathlineto{\pgfqpoint{9.874751in}{2.859836in}}%
\pgfpathlineto{\pgfqpoint{9.875436in}{2.858779in}}%
\pgfpathlineto{\pgfqpoint{9.876120in}{2.858514in}}%
\pgfpathlineto{\pgfqpoint{9.877490in}{2.858137in}}%
\pgfpathlineto{\pgfqpoint{9.878859in}{2.856777in}}%
\pgfpathlineto{\pgfqpoint{9.879544in}{2.855411in}}%
\pgfpathlineto{\pgfqpoint{9.879886in}{2.855224in}}%
\pgfpathlineto{\pgfqpoint{9.880571in}{2.854535in}}%
\pgfpathlineto{\pgfqpoint{9.881255in}{2.854384in}}%
\pgfpathlineto{\pgfqpoint{9.881598in}{2.852416in}}%
\pgfpathlineto{\pgfqpoint{9.885021in}{2.851147in}}%
\pgfpathlineto{\pgfqpoint{9.885364in}{2.851121in}}%
\pgfpathlineto{\pgfqpoint{9.886733in}{2.850126in}}%
\pgfpathlineto{\pgfqpoint{9.888102in}{2.849724in}}%
\pgfpathlineto{\pgfqpoint{9.889129in}{2.848524in}}%
\pgfpathlineto{\pgfqpoint{9.889814in}{2.848469in}}%
\pgfpathlineto{\pgfqpoint{9.890156in}{2.845946in}}%
\pgfpathlineto{\pgfqpoint{9.890841in}{2.845412in}}%
\pgfpathlineto{\pgfqpoint{9.891526in}{2.844767in}}%
\pgfpathlineto{\pgfqpoint{9.892211in}{2.843638in}}%
\pgfpathlineto{\pgfqpoint{9.893580in}{2.842958in}}%
\pgfpathlineto{\pgfqpoint{9.894949in}{2.839673in}}%
\pgfpathlineto{\pgfqpoint{9.895634in}{2.839107in}}%
\pgfpathlineto{\pgfqpoint{9.895976in}{2.838858in}}%
\pgfpathlineto{\pgfqpoint{9.898373in}{2.834500in}}%
\pgfpathlineto{\pgfqpoint{9.900427in}{2.833775in}}%
\pgfpathlineto{\pgfqpoint{9.901454in}{2.832348in}}%
\pgfpathlineto{\pgfqpoint{9.903508in}{2.830824in}}%
\pgfpathlineto{\pgfqpoint{9.904193in}{2.830725in}}%
\pgfpathlineto{\pgfqpoint{9.904877in}{2.829922in}}%
\pgfpathlineto{\pgfqpoint{9.907274in}{2.828436in}}%
\pgfpathlineto{\pgfqpoint{9.907958in}{2.827431in}}%
\pgfpathlineto{\pgfqpoint{9.908301in}{2.827412in}}%
\pgfpathlineto{\pgfqpoint{9.910355in}{2.824457in}}%
\pgfpathlineto{\pgfqpoint{9.910697in}{2.824328in}}%
\pgfpathlineto{\pgfqpoint{9.911382in}{2.823173in}}%
\pgfpathlineto{\pgfqpoint{9.914463in}{2.820629in}}%
\pgfpathlineto{\pgfqpoint{9.915148in}{2.819943in}}%
\pgfpathlineto{\pgfqpoint{9.915832in}{2.819812in}}%
\pgfpathlineto{\pgfqpoint{9.916859in}{2.819012in}}%
\pgfpathlineto{\pgfqpoint{9.917202in}{2.818991in}}%
\pgfpathlineto{\pgfqpoint{9.917887in}{2.817962in}}%
\pgfpathlineto{\pgfqpoint{9.918914in}{2.817620in}}%
\pgfpathlineto{\pgfqpoint{9.919256in}{2.817283in}}%
\pgfpathlineto{\pgfqpoint{9.919941in}{2.816037in}}%
\pgfpathlineto{\pgfqpoint{9.921652in}{2.815761in}}%
\pgfpathlineto{\pgfqpoint{9.922337in}{2.814887in}}%
\pgfpathlineto{\pgfqpoint{9.923022in}{2.814564in}}%
\pgfpathlineto{\pgfqpoint{9.923364in}{2.813733in}}%
\pgfpathlineto{\pgfqpoint{9.923706in}{2.813696in}}%
\pgfpathlineto{\pgfqpoint{9.924391in}{2.812903in}}%
\pgfpathlineto{\pgfqpoint{9.925760in}{2.812065in}}%
\pgfpathlineto{\pgfqpoint{9.926445in}{2.811368in}}%
\pgfpathlineto{\pgfqpoint{9.926788in}{2.810115in}}%
\pgfpathlineto{\pgfqpoint{9.927472in}{2.809860in}}%
\pgfpathlineto{\pgfqpoint{9.928157in}{2.809039in}}%
\pgfpathlineto{\pgfqpoint{9.929526in}{2.808447in}}%
\pgfpathlineto{\pgfqpoint{9.930211in}{2.807654in}}%
\pgfpathlineto{\pgfqpoint{9.931923in}{2.805011in}}%
\pgfpathlineto{\pgfqpoint{9.932607in}{2.804370in}}%
\pgfpathlineto{\pgfqpoint{9.933292in}{2.803463in}}%
\pgfpathlineto{\pgfqpoint{9.933977in}{2.803244in}}%
\pgfpathlineto{\pgfqpoint{9.934661in}{2.802434in}}%
\pgfpathlineto{\pgfqpoint{9.937400in}{2.800569in}}%
\pgfpathlineto{\pgfqpoint{9.937743in}{2.799680in}}%
\pgfpathlineto{\pgfqpoint{9.938085in}{2.797498in}}%
\pgfpathlineto{\pgfqpoint{9.939454in}{2.797120in}}%
\pgfpathlineto{\pgfqpoint{9.940139in}{2.795524in}}%
\pgfpathlineto{\pgfqpoint{9.940481in}{2.795512in}}%
\pgfpathlineto{\pgfqpoint{9.942535in}{2.792924in}}%
\pgfpathlineto{\pgfqpoint{9.943905in}{2.791791in}}%
\pgfpathlineto{\pgfqpoint{9.944932in}{2.790899in}}%
\pgfpathlineto{\pgfqpoint{9.945617in}{2.789493in}}%
\pgfpathlineto{\pgfqpoint{9.945959in}{2.789191in}}%
\pgfpathlineto{\pgfqpoint{9.946644in}{2.788322in}}%
\pgfpathlineto{\pgfqpoint{9.947671in}{2.787605in}}%
\pgfpathlineto{\pgfqpoint{9.949725in}{2.786117in}}%
\pgfpathlineto{\pgfqpoint{9.950067in}{2.785423in}}%
\pgfpathlineto{\pgfqpoint{9.950409in}{2.785377in}}%
\pgfpathlineto{\pgfqpoint{9.953148in}{2.782046in}}%
\pgfpathlineto{\pgfqpoint{9.954518in}{2.780886in}}%
\pgfpathlineto{\pgfqpoint{9.955545in}{2.778536in}}%
\pgfpathlineto{\pgfqpoint{9.956572in}{2.778203in}}%
\pgfpathlineto{\pgfqpoint{9.956914in}{2.777448in}}%
\pgfpathlineto{\pgfqpoint{9.957256in}{2.777426in}}%
\pgfpathlineto{\pgfqpoint{9.957941in}{2.776939in}}%
\pgfpathlineto{\pgfqpoint{9.958968in}{2.776580in}}%
\pgfpathlineto{\pgfqpoint{9.959995in}{2.776391in}}%
\pgfpathlineto{\pgfqpoint{9.963076in}{2.772464in}}%
\pgfpathlineto{\pgfqpoint{9.963419in}{2.770539in}}%
\pgfpathlineto{\pgfqpoint{9.964446in}{2.770289in}}%
\pgfpathlineto{\pgfqpoint{9.965130in}{2.770123in}}%
\pgfpathlineto{\pgfqpoint{9.965815in}{2.769708in}}%
\pgfpathlineto{\pgfqpoint{9.966157in}{2.768462in}}%
\pgfpathlineto{\pgfqpoint{9.967184in}{2.768315in}}%
\pgfpathlineto{\pgfqpoint{9.967869in}{2.767791in}}%
\pgfpathlineto{\pgfqpoint{9.968211in}{2.767678in}}%
\pgfpathlineto{\pgfqpoint{9.968896in}{2.766606in}}%
\pgfpathlineto{\pgfqpoint{9.969581in}{2.766347in}}%
\pgfpathlineto{\pgfqpoint{9.970266in}{2.765909in}}%
\pgfpathlineto{\pgfqpoint{9.970608in}{2.765774in}}%
\pgfpathlineto{\pgfqpoint{9.971635in}{2.764540in}}%
\pgfpathlineto{\pgfqpoint{9.972320in}{2.764201in}}%
\pgfpathlineto{\pgfqpoint{9.974374in}{2.762949in}}%
\pgfpathlineto{\pgfqpoint{9.975058in}{2.762194in}}%
\pgfpathlineto{\pgfqpoint{9.975401in}{2.762043in}}%
\pgfpathlineto{\pgfqpoint{9.976428in}{2.760541in}}%
\pgfpathlineto{\pgfqpoint{9.977112in}{2.760338in}}%
\pgfpathlineto{\pgfqpoint{9.977455in}{2.759264in}}%
\pgfpathlineto{\pgfqpoint{9.978482in}{2.758947in}}%
\pgfpathlineto{\pgfqpoint{9.979167in}{2.758293in}}%
\pgfpathlineto{\pgfqpoint{9.979509in}{2.758267in}}%
\pgfpathlineto{\pgfqpoint{9.980194in}{2.757134in}}%
\pgfpathlineto{\pgfqpoint{9.980536in}{2.756243in}}%
\pgfpathlineto{\pgfqpoint{9.981563in}{2.755881in}}%
\pgfpathlineto{\pgfqpoint{9.982248in}{2.755133in}}%
\pgfpathlineto{\pgfqpoint{9.983275in}{2.754884in}}%
\pgfpathlineto{\pgfqpoint{9.984302in}{2.754567in}}%
\pgfpathlineto{\pgfqpoint{9.984986in}{2.753740in}}%
\pgfpathlineto{\pgfqpoint{9.986014in}{2.752749in}}%
\pgfpathlineto{\pgfqpoint{9.986356in}{2.752679in}}%
\pgfpathlineto{\pgfqpoint{9.987041in}{2.751042in}}%
\pgfpathlineto{\pgfqpoint{9.988068in}{2.750697in}}%
\pgfpathlineto{\pgfqpoint{9.988752in}{2.750383in}}%
\pgfpathlineto{\pgfqpoint{9.990806in}{2.747242in}}%
\pgfpathlineto{\pgfqpoint{9.991491in}{2.745611in}}%
\pgfpathlineto{\pgfqpoint{9.991833in}{2.745430in}}%
\pgfpathlineto{\pgfqpoint{9.992518in}{2.744750in}}%
\pgfpathlineto{\pgfqpoint{9.993203in}{2.743844in}}%
\pgfpathlineto{\pgfqpoint{9.994915in}{2.743284in}}%
\pgfpathlineto{\pgfqpoint{9.995599in}{2.741578in}}%
\pgfpathlineto{\pgfqpoint{9.995942in}{2.741389in}}%
\pgfpathlineto{\pgfqpoint{9.996626in}{2.740325in}}%
\pgfpathlineto{\pgfqpoint{10.001419in}{2.735650in}}%
\pgfpathlineto{\pgfqpoint{10.002104in}{2.734669in}}%
\pgfpathlineto{\pgfqpoint{10.002446in}{2.734088in}}%
\pgfpathlineto{\pgfqpoint{10.003131in}{2.733838in}}%
\pgfpathlineto{\pgfqpoint{10.003816in}{2.733368in}}%
\pgfpathlineto{\pgfqpoint{10.004500in}{2.732554in}}%
\pgfpathlineto{\pgfqpoint{10.005527in}{2.731746in}}%
\pgfpathlineto{\pgfqpoint{10.006554in}{2.731421in}}%
\pgfpathlineto{\pgfqpoint{10.007581in}{2.729760in}}%
\pgfpathlineto{\pgfqpoint{10.008266in}{2.727586in}}%
\pgfpathlineto{\pgfqpoint{10.008951in}{2.726098in}}%
\pgfpathlineto{\pgfqpoint{10.009293in}{2.726098in}}%
\pgfpathlineto{\pgfqpoint{10.009978in}{2.725419in}}%
\pgfpathlineto{\pgfqpoint{10.010320in}{2.725380in}}%
\pgfpathlineto{\pgfqpoint{10.011005in}{2.724549in}}%
\pgfpathlineto{\pgfqpoint{10.011690in}{2.723233in}}%
\pgfpathlineto{\pgfqpoint{10.013744in}{2.722412in}}%
\pgfpathlineto{\pgfqpoint{10.015113in}{2.718952in}}%
\pgfpathlineto{\pgfqpoint{10.015798in}{2.718196in}}%
\pgfpathlineto{\pgfqpoint{10.016140in}{2.718032in}}%
\pgfpathlineto{\pgfqpoint{10.017509in}{2.716069in}}%
\pgfpathlineto{\pgfqpoint{10.018194in}{2.715903in}}%
\pgfpathlineto{\pgfqpoint{10.018536in}{2.714906in}}%
\pgfpathlineto{\pgfqpoint{10.019221in}{2.714529in}}%
\pgfpathlineto{\pgfqpoint{10.019906in}{2.713466in}}%
\pgfpathlineto{\pgfqpoint{10.020248in}{2.713222in}}%
\pgfpathlineto{\pgfqpoint{10.020933in}{2.712052in}}%
\pgfpathlineto{\pgfqpoint{10.021618in}{2.711475in}}%
\pgfpathlineto{\pgfqpoint{10.022302in}{2.710297in}}%
\pgfpathlineto{\pgfqpoint{10.022987in}{2.709416in}}%
\pgfpathlineto{\pgfqpoint{10.024699in}{2.708800in}}%
\pgfpathlineto{\pgfqpoint{10.026410in}{2.707400in}}%
\pgfpathlineto{\pgfqpoint{10.027095in}{2.706330in}}%
\pgfpathlineto{\pgfqpoint{10.029149in}{2.705074in}}%
\pgfpathlineto{\pgfqpoint{10.029834in}{2.704613in}}%
\pgfpathlineto{\pgfqpoint{10.030176in}{2.704582in}}%
\pgfpathlineto{\pgfqpoint{10.031203in}{2.703556in}}%
\pgfpathlineto{\pgfqpoint{10.031546in}{2.703443in}}%
\pgfpathlineto{\pgfqpoint{10.032230in}{2.702590in}}%
\pgfpathlineto{\pgfqpoint{10.032915in}{2.701819in}}%
\pgfpathlineto{\pgfqpoint{10.033942in}{2.701481in}}%
\pgfpathlineto{\pgfqpoint{10.034627in}{2.701177in}}%
\pgfpathlineto{\pgfqpoint{10.034969in}{2.701117in}}%
\pgfpathlineto{\pgfqpoint{10.035654in}{2.699894in}}%
\pgfpathlineto{\pgfqpoint{10.036338in}{2.699856in}}%
\pgfpathlineto{\pgfqpoint{10.037023in}{2.699138in}}%
\pgfpathlineto{\pgfqpoint{10.038050in}{2.698799in}}%
\pgfpathlineto{\pgfqpoint{10.038393in}{2.697900in}}%
\pgfpathlineto{\pgfqpoint{10.038735in}{2.697855in}}%
\pgfpathlineto{\pgfqpoint{10.040104in}{2.696080in}}%
\pgfpathlineto{\pgfqpoint{10.041131in}{2.695619in}}%
\pgfpathlineto{\pgfqpoint{10.041474in}{2.695529in}}%
\pgfpathlineto{\pgfqpoint{10.041816in}{2.694162in}}%
\pgfpathlineto{\pgfqpoint{10.043185in}{2.693602in}}%
\pgfpathlineto{\pgfqpoint{10.044897in}{2.690983in}}%
\pgfpathlineto{\pgfqpoint{10.045582in}{2.690699in}}%
\pgfpathlineto{\pgfqpoint{10.046951in}{2.689963in}}%
\pgfpathlineto{\pgfqpoint{10.047978in}{2.689623in}}%
\pgfpathlineto{\pgfqpoint{10.049005in}{2.688233in}}%
\pgfpathlineto{\pgfqpoint{10.049348in}{2.688113in}}%
\pgfpathlineto{\pgfqpoint{10.050032in}{2.687320in}}%
\pgfpathlineto{\pgfqpoint{10.050717in}{2.687094in}}%
\pgfpathlineto{\pgfqpoint{10.051402in}{2.686301in}}%
\pgfpathlineto{\pgfqpoint{10.051744in}{2.685963in}}%
\pgfpathlineto{\pgfqpoint{10.052771in}{2.683942in}}%
\pgfpathlineto{\pgfqpoint{10.054483in}{2.683021in}}%
\pgfpathlineto{\pgfqpoint{10.055852in}{2.682501in}}%
\pgfpathlineto{\pgfqpoint{10.056537in}{2.681550in}}%
\pgfpathlineto{\pgfqpoint{10.057222in}{2.681251in}}%
\pgfpathlineto{\pgfqpoint{10.057906in}{2.679697in}}%
\pgfpathlineto{\pgfqpoint{10.058591in}{2.679610in}}%
\pgfpathlineto{\pgfqpoint{10.058933in}{2.677815in}}%
\pgfpathlineto{\pgfqpoint{10.059618in}{2.677541in}}%
\pgfpathlineto{\pgfqpoint{10.060645in}{2.675963in}}%
\pgfpathlineto{\pgfqpoint{10.061330in}{2.675921in}}%
\pgfpathlineto{\pgfqpoint{10.063726in}{2.673649in}}%
\pgfpathlineto{\pgfqpoint{10.064411in}{2.671899in}}%
\pgfpathlineto{\pgfqpoint{10.065096in}{2.671198in}}%
\pgfpathlineto{\pgfqpoint{10.065780in}{2.670782in}}%
\pgfpathlineto{\pgfqpoint{10.066123in}{2.670639in}}%
\pgfpathlineto{\pgfqpoint{10.067150in}{2.669267in}}%
\pgfpathlineto{\pgfqpoint{10.068177in}{2.668819in}}%
\pgfpathlineto{\pgfqpoint{10.068519in}{2.667664in}}%
\pgfpathlineto{\pgfqpoint{10.068861in}{2.667574in}}%
\pgfpathlineto{\pgfqpoint{10.069546in}{2.666365in}}%
\pgfpathlineto{\pgfqpoint{10.069888in}{2.666289in}}%
\pgfpathlineto{\pgfqpoint{10.070573in}{2.665836in}}%
\pgfpathlineto{\pgfqpoint{10.071258in}{2.665776in}}%
\pgfpathlineto{\pgfqpoint{10.071943in}{2.665081in}}%
\pgfpathlineto{\pgfqpoint{10.072627in}{2.664898in}}%
\pgfpathlineto{\pgfqpoint{10.073654in}{2.664439in}}%
\pgfpathlineto{\pgfqpoint{10.074339in}{2.663457in}}%
\pgfpathlineto{\pgfqpoint{10.075024in}{2.663118in}}%
\pgfpathlineto{\pgfqpoint{10.078105in}{2.659063in}}%
\pgfpathlineto{\pgfqpoint{10.078789in}{2.657378in}}%
\pgfpathlineto{\pgfqpoint{10.080159in}{2.655566in}}%
\pgfpathlineto{\pgfqpoint{10.081871in}{2.654948in}}%
\pgfpathlineto{\pgfqpoint{10.082213in}{2.652933in}}%
\pgfpathlineto{\pgfqpoint{10.083925in}{2.652260in}}%
\pgfpathlineto{\pgfqpoint{10.084267in}{2.651337in}}%
\pgfpathlineto{\pgfqpoint{10.084952in}{2.651246in}}%
\pgfpathlineto{\pgfqpoint{10.086321in}{2.649736in}}%
\pgfpathlineto{\pgfqpoint{10.086663in}{2.649713in}}%
\pgfpathlineto{\pgfqpoint{10.087690in}{2.648958in}}%
\pgfpathlineto{\pgfqpoint{10.088375in}{2.647491in}}%
\pgfpathlineto{\pgfqpoint{10.088718in}{2.646995in}}%
\pgfpathlineto{\pgfqpoint{10.089402in}{2.644276in}}%
\pgfpathlineto{\pgfqpoint{10.089745in}{2.644239in}}%
\pgfpathlineto{\pgfqpoint{10.090087in}{2.643514in}}%
\pgfpathlineto{\pgfqpoint{10.090429in}{2.643483in}}%
\pgfpathlineto{\pgfqpoint{10.091114in}{2.642145in}}%
\pgfpathlineto{\pgfqpoint{10.093853in}{2.640954in}}%
\pgfpathlineto{\pgfqpoint{10.095564in}{2.639594in}}%
\pgfpathlineto{\pgfqpoint{10.096249in}{2.639368in}}%
\pgfpathlineto{\pgfqpoint{10.096934in}{2.638867in}}%
\pgfpathlineto{\pgfqpoint{10.097619in}{2.638613in}}%
\pgfpathlineto{\pgfqpoint{10.098303in}{2.637669in}}%
\pgfpathlineto{\pgfqpoint{10.101384in}{2.636792in}}%
\pgfpathlineto{\pgfqpoint{10.102411in}{2.636234in}}%
\pgfpathlineto{\pgfqpoint{10.103438in}{2.635630in}}%
\pgfpathlineto{\pgfqpoint{10.104123in}{2.634535in}}%
\pgfpathlineto{\pgfqpoint{10.104808in}{2.634384in}}%
\pgfpathlineto{\pgfqpoint{10.107547in}{2.631439in}}%
\pgfpathlineto{\pgfqpoint{10.108574in}{2.631325in}}%
\pgfpathlineto{\pgfqpoint{10.109601in}{2.629906in}}%
\pgfpathlineto{\pgfqpoint{10.109943in}{2.628502in}}%
\pgfpathlineto{\pgfqpoint{10.110285in}{2.628350in}}%
\pgfpathlineto{\pgfqpoint{10.110970in}{2.627587in}}%
\pgfpathlineto{\pgfqpoint{10.111655in}{2.627210in}}%
\pgfpathlineto{\pgfqpoint{10.111997in}{2.627165in}}%
\pgfpathlineto{\pgfqpoint{10.115421in}{2.623395in}}%
\pgfpathlineto{\pgfqpoint{10.116105in}{2.621372in}}%
\pgfpathlineto{\pgfqpoint{10.116448in}{2.621330in}}%
\pgfpathlineto{\pgfqpoint{10.118159in}{2.619222in}}%
\pgfpathlineto{\pgfqpoint{10.119186in}{2.617959in}}%
\pgfpathlineto{\pgfqpoint{10.119871in}{2.616677in}}%
\pgfpathlineto{\pgfqpoint{10.120898in}{2.615958in}}%
\pgfpathlineto{\pgfqpoint{10.121583in}{2.614519in}}%
\pgfpathlineto{\pgfqpoint{10.121925in}{2.614485in}}%
\pgfpathlineto{\pgfqpoint{10.122610in}{2.614123in}}%
\pgfpathlineto{\pgfqpoint{10.124664in}{2.612976in}}%
\pgfpathlineto{\pgfqpoint{10.125349in}{2.612311in}}%
\pgfpathlineto{\pgfqpoint{10.127060in}{2.610172in}}%
\pgfpathlineto{\pgfqpoint{10.127403in}{2.610143in}}%
\pgfpathlineto{\pgfqpoint{10.129799in}{2.607666in}}%
\pgfpathlineto{\pgfqpoint{10.130826in}{2.607462in}}%
\pgfpathlineto{\pgfqpoint{10.133565in}{2.606028in}}%
\pgfpathlineto{\pgfqpoint{10.136988in}{2.604055in}}%
\pgfpathlineto{\pgfqpoint{10.138700in}{2.603805in}}%
\pgfpathlineto{\pgfqpoint{10.140754in}{2.601630in}}%
\pgfpathlineto{\pgfqpoint{10.142124in}{2.601394in}}%
\pgfpathlineto{\pgfqpoint{10.143151in}{2.599005in}}%
\pgfpathlineto{\pgfqpoint{10.144178in}{2.598401in}}%
\pgfpathlineto{\pgfqpoint{10.144862in}{2.597108in}}%
\pgfpathlineto{\pgfqpoint{10.145205in}{2.596097in}}%
\pgfpathlineto{\pgfqpoint{10.145547in}{2.596022in}}%
\pgfpathlineto{\pgfqpoint{10.146916in}{2.593796in}}%
\pgfpathlineto{\pgfqpoint{10.148286in}{2.592019in}}%
\pgfpathlineto{\pgfqpoint{10.148628in}{2.591868in}}%
\pgfpathlineto{\pgfqpoint{10.151025in}{2.588969in}}%
\pgfpathlineto{\pgfqpoint{10.151709in}{2.588751in}}%
\pgfpathlineto{\pgfqpoint{10.152394in}{2.588281in}}%
\pgfpathlineto{\pgfqpoint{10.153079in}{2.587476in}}%
\pgfpathlineto{\pgfqpoint{10.154106in}{2.587028in}}%
\pgfpathlineto{\pgfqpoint{10.154448in}{2.586167in}}%
\pgfpathlineto{\pgfqpoint{10.155475in}{2.585903in}}%
\pgfpathlineto{\pgfqpoint{10.155817in}{2.585810in}}%
\pgfpathlineto{\pgfqpoint{10.156845in}{2.584909in}}%
\pgfpathlineto{\pgfqpoint{10.157187in}{2.582950in}}%
\pgfpathlineto{\pgfqpoint{10.158556in}{2.582542in}}%
\pgfpathlineto{\pgfqpoint{10.159241in}{2.582278in}}%
\pgfpathlineto{\pgfqpoint{10.159926in}{2.580956in}}%
\pgfpathlineto{\pgfqpoint{10.160610in}{2.580541in}}%
\pgfpathlineto{\pgfqpoint{10.161295in}{2.580239in}}%
\pgfpathlineto{\pgfqpoint{10.161980in}{2.580195in}}%
\pgfpathlineto{\pgfqpoint{10.162664in}{2.579899in}}%
\pgfpathlineto{\pgfqpoint{10.163691in}{2.579537in}}%
\pgfpathlineto{\pgfqpoint{10.164034in}{2.579465in}}%
\pgfpathlineto{\pgfqpoint{10.164376in}{2.578653in}}%
\pgfpathlineto{\pgfqpoint{10.164718in}{2.578600in}}%
\pgfpathlineto{\pgfqpoint{10.165746in}{2.576807in}}%
\pgfpathlineto{\pgfqpoint{10.166088in}{2.576717in}}%
\pgfpathlineto{\pgfqpoint{10.166773in}{2.576181in}}%
\pgfpathlineto{\pgfqpoint{10.167115in}{2.575983in}}%
\pgfpathlineto{\pgfqpoint{10.168827in}{2.572890in}}%
\pgfpathlineto{\pgfqpoint{10.169511in}{2.572385in}}%
\pgfpathlineto{\pgfqpoint{10.170538in}{2.571230in}}%
\pgfpathlineto{\pgfqpoint{10.170881in}{2.569818in}}%
\pgfpathlineto{\pgfqpoint{10.171223in}{2.569755in}}%
\pgfpathlineto{\pgfqpoint{10.172250in}{2.568362in}}%
\pgfpathlineto{\pgfqpoint{10.175331in}{2.567545in}}%
\pgfpathlineto{\pgfqpoint{10.176701in}{2.565665in}}%
\pgfpathlineto{\pgfqpoint{10.177043in}{2.565627in}}%
\pgfpathlineto{\pgfqpoint{10.177728in}{2.564399in}}%
\pgfpathlineto{\pgfqpoint{10.178412in}{2.563960in}}%
\pgfpathlineto{\pgfqpoint{10.179097in}{2.563105in}}%
\pgfpathlineto{\pgfqpoint{10.179782in}{2.563056in}}%
\pgfpathlineto{\pgfqpoint{10.180466in}{2.562016in}}%
\pgfpathlineto{\pgfqpoint{10.181836in}{2.560289in}}%
\pgfpathlineto{\pgfqpoint{10.182521in}{2.558944in}}%
\pgfpathlineto{\pgfqpoint{10.183548in}{2.558542in}}%
\pgfpathlineto{\pgfqpoint{10.184232in}{2.557924in}}%
\pgfpathlineto{\pgfqpoint{10.186971in}{2.555490in}}%
\pgfpathlineto{\pgfqpoint{10.187998in}{2.554667in}}%
\pgfpathlineto{\pgfqpoint{10.188683in}{2.553695in}}%
\pgfpathlineto{\pgfqpoint{10.190052in}{2.552883in}}%
\pgfpathlineto{\pgfqpoint{10.191079in}{2.551203in}}%
\pgfpathlineto{\pgfqpoint{10.191422in}{2.551082in}}%
\pgfpathlineto{\pgfqpoint{10.194845in}{2.546974in}}%
\pgfpathlineto{\pgfqpoint{10.196214in}{2.546030in}}%
\pgfpathlineto{\pgfqpoint{10.196899in}{2.545842in}}%
\pgfpathlineto{\pgfqpoint{10.197241in}{2.545110in}}%
\pgfpathlineto{\pgfqpoint{10.197584in}{2.545086in}}%
\pgfpathlineto{\pgfqpoint{10.198268in}{2.544226in}}%
\pgfpathlineto{\pgfqpoint{10.199980in}{2.542474in}}%
\pgfpathlineto{\pgfqpoint{10.200665in}{2.541543in}}%
\pgfpathlineto{\pgfqpoint{10.202377in}{2.540693in}}%
\pgfpathlineto{\pgfqpoint{10.203061in}{2.540601in}}%
\pgfpathlineto{\pgfqpoint{10.203404in}{2.540065in}}%
\pgfpathlineto{\pgfqpoint{10.204088in}{2.538145in}}%
\pgfpathlineto{\pgfqpoint{10.205115in}{2.537384in}}%
\pgfpathlineto{\pgfqpoint{10.205800in}{2.537042in}}%
\pgfpathlineto{\pgfqpoint{10.207512in}{2.535269in}}%
\pgfpathlineto{\pgfqpoint{10.208197in}{2.534378in}}%
\pgfpathlineto{\pgfqpoint{10.209224in}{2.533948in}}%
\pgfpathlineto{\pgfqpoint{10.209908in}{2.533457in}}%
\pgfpathlineto{\pgfqpoint{10.211278in}{2.532596in}}%
\pgfpathlineto{\pgfqpoint{10.211620in}{2.531229in}}%
\pgfpathlineto{\pgfqpoint{10.212647in}{2.530844in}}%
\pgfpathlineto{\pgfqpoint{10.212989in}{2.529183in}}%
\pgfpathlineto{\pgfqpoint{10.214359in}{2.528126in}}%
\pgfpathlineto{\pgfqpoint{10.215043in}{2.527722in}}%
\pgfpathlineto{\pgfqpoint{10.215728in}{2.527527in}}%
\pgfpathlineto{\pgfqpoint{10.217782in}{2.526736in}}%
\pgfpathlineto{\pgfqpoint{10.218467in}{2.525498in}}%
\pgfpathlineto{\pgfqpoint{10.219152in}{2.525165in}}%
\pgfpathlineto{\pgfqpoint{10.219836in}{2.525129in}}%
\pgfpathlineto{\pgfqpoint{10.220179in}{2.524206in}}%
\pgfpathlineto{\pgfqpoint{10.220521in}{2.524129in}}%
\pgfpathlineto{\pgfqpoint{10.221206in}{2.523678in}}%
\pgfpathlineto{\pgfqpoint{10.223260in}{2.522725in}}%
\pgfpathlineto{\pgfqpoint{10.223944in}{2.521299in}}%
\pgfpathlineto{\pgfqpoint{10.224971in}{2.520846in}}%
\pgfpathlineto{\pgfqpoint{10.225656in}{2.520100in}}%
\pgfpathlineto{\pgfqpoint{10.227710in}{2.518460in}}%
\pgfpathlineto{\pgfqpoint{10.229080in}{2.516358in}}%
\pgfpathlineto{\pgfqpoint{10.231134in}{2.515258in}}%
\pgfpathlineto{\pgfqpoint{10.231818in}{2.513181in}}%
\pgfpathlineto{\pgfqpoint{10.233188in}{2.513030in}}%
\pgfpathlineto{\pgfqpoint{10.233873in}{2.512388in}}%
\pgfpathlineto{\pgfqpoint{10.234215in}{2.511180in}}%
\pgfpathlineto{\pgfqpoint{10.234557in}{2.511150in}}%
\pgfpathlineto{\pgfqpoint{10.235242in}{2.510453in}}%
\pgfpathlineto{\pgfqpoint{10.237296in}{2.509364in}}%
\pgfpathlineto{\pgfqpoint{10.237981in}{2.507770in}}%
\pgfpathlineto{\pgfqpoint{10.240035in}{2.506143in}}%
\pgfpathlineto{\pgfqpoint{10.242774in}{2.504595in}}%
\pgfpathlineto{\pgfqpoint{10.243116in}{2.504044in}}%
\pgfpathlineto{\pgfqpoint{10.244143in}{2.503810in}}%
\pgfpathlineto{\pgfqpoint{10.245170in}{2.503100in}}%
\pgfpathlineto{\pgfqpoint{10.247566in}{2.500592in}}%
\pgfpathlineto{\pgfqpoint{10.248251in}{2.499713in}}%
\pgfpathlineto{\pgfqpoint{10.249620in}{2.499286in}}%
\pgfpathlineto{\pgfqpoint{10.250647in}{2.498540in}}%
\pgfpathlineto{\pgfqpoint{10.250990in}{2.498348in}}%
\pgfpathlineto{\pgfqpoint{10.252359in}{2.495322in}}%
\pgfpathlineto{\pgfqpoint{10.254071in}{2.494038in}}%
\pgfpathlineto{\pgfqpoint{10.255098in}{2.492947in}}%
\pgfpathlineto{\pgfqpoint{10.255440in}{2.492688in}}%
\pgfpathlineto{\pgfqpoint{10.256125in}{2.491636in}}%
\pgfpathlineto{\pgfqpoint{10.256810in}{2.490640in}}%
\pgfpathlineto{\pgfqpoint{10.258521in}{2.489922in}}%
\pgfpathlineto{\pgfqpoint{10.259549in}{2.488563in}}%
\pgfpathlineto{\pgfqpoint{10.260233in}{2.488136in}}%
\pgfpathlineto{\pgfqpoint{10.260576in}{2.487846in}}%
\pgfpathlineto{\pgfqpoint{10.260918in}{2.486204in}}%
\pgfpathlineto{\pgfqpoint{10.261260in}{2.486139in}}%
\pgfpathlineto{\pgfqpoint{10.261945in}{2.485731in}}%
\pgfpathlineto{\pgfqpoint{10.262287in}{2.485531in}}%
\pgfpathlineto{\pgfqpoint{10.262972in}{2.484689in}}%
\pgfpathlineto{\pgfqpoint{10.263314in}{2.484598in}}%
\pgfpathlineto{\pgfqpoint{10.263999in}{2.483243in}}%
\pgfpathlineto{\pgfqpoint{10.264684in}{2.482933in}}%
\pgfpathlineto{\pgfqpoint{10.265368in}{2.482144in}}%
\pgfpathlineto{\pgfqpoint{10.265711in}{2.481993in}}%
\pgfpathlineto{\pgfqpoint{10.266395in}{2.480709in}}%
\pgfpathlineto{\pgfqpoint{10.266738in}{2.480181in}}%
\pgfpathlineto{\pgfqpoint{10.267422in}{2.479916in}}%
\pgfpathlineto{\pgfqpoint{10.268450in}{2.478715in}}%
\pgfpathlineto{\pgfqpoint{10.272215in}{2.476669in}}%
\pgfpathlineto{\pgfqpoint{10.273242in}{2.475929in}}%
\pgfpathlineto{\pgfqpoint{10.273927in}{2.475612in}}%
\pgfpathlineto{\pgfqpoint{10.274612in}{2.474857in}}%
\pgfpathlineto{\pgfqpoint{10.275296in}{2.474479in}}%
\pgfpathlineto{\pgfqpoint{10.275639in}{2.474442in}}%
\pgfpathlineto{\pgfqpoint{10.276666in}{2.472989in}}%
\pgfpathlineto{\pgfqpoint{10.277351in}{2.472758in}}%
\pgfpathlineto{\pgfqpoint{10.277693in}{2.471610in}}%
\pgfpathlineto{\pgfqpoint{10.279405in}{2.470944in}}%
\pgfpathlineto{\pgfqpoint{10.280089in}{2.470099in}}%
\pgfpathlineto{\pgfqpoint{10.281116in}{2.469405in}}%
\pgfpathlineto{\pgfqpoint{10.281459in}{2.469374in}}%
\pgfpathlineto{\pgfqpoint{10.282486in}{2.468227in}}%
\pgfpathlineto{\pgfqpoint{10.282828in}{2.467260in}}%
\pgfpathlineto{\pgfqpoint{10.283170in}{2.467191in}}%
\pgfpathlineto{\pgfqpoint{10.283513in}{2.466444in}}%
\pgfpathlineto{\pgfqpoint{10.283855in}{2.466387in}}%
\pgfpathlineto{\pgfqpoint{10.284540in}{2.464654in}}%
\pgfpathlineto{\pgfqpoint{10.285225in}{2.464390in}}%
\pgfpathlineto{\pgfqpoint{10.285567in}{2.464360in}}%
\pgfpathlineto{\pgfqpoint{10.286252in}{2.463907in}}%
\pgfpathlineto{\pgfqpoint{10.287279in}{2.463681in}}%
\pgfpathlineto{\pgfqpoint{10.287963in}{2.463076in}}%
\pgfpathlineto{\pgfqpoint{10.289333in}{2.461830in}}%
\pgfpathlineto{\pgfqpoint{10.289675in}{2.461793in}}%
\pgfpathlineto{\pgfqpoint{10.290360in}{2.461264in}}%
\pgfpathlineto{\pgfqpoint{10.291387in}{2.460962in}}%
\pgfpathlineto{\pgfqpoint{10.292756in}{2.459201in}}%
\pgfpathlineto{\pgfqpoint{10.293441in}{2.458746in}}%
\pgfpathlineto{\pgfqpoint{10.293783in}{2.457204in}}%
\pgfpathlineto{\pgfqpoint{10.294468in}{2.456537in}}%
\pgfpathlineto{\pgfqpoint{10.295153in}{2.455751in}}%
\pgfpathlineto{\pgfqpoint{10.295495in}{2.455676in}}%
\pgfpathlineto{\pgfqpoint{10.296180in}{2.453524in}}%
\pgfpathlineto{\pgfqpoint{10.297549in}{2.452909in}}%
\pgfpathlineto{\pgfqpoint{10.298576in}{2.450526in}}%
\pgfpathlineto{\pgfqpoint{10.298918in}{2.450390in}}%
\pgfpathlineto{\pgfqpoint{10.299603in}{2.449660in}}%
\pgfpathlineto{\pgfqpoint{10.300972in}{2.447800in}}%
\pgfpathlineto{\pgfqpoint{10.302342in}{2.447048in}}%
\pgfpathlineto{\pgfqpoint{10.303027in}{2.446387in}}%
\pgfpathlineto{\pgfqpoint{10.303369in}{2.446387in}}%
\pgfpathlineto{\pgfqpoint{10.304054in}{2.445897in}}%
\pgfpathlineto{\pgfqpoint{10.305081in}{2.445280in}}%
\pgfpathlineto{\pgfqpoint{10.306108in}{2.443782in}}%
\pgfpathlineto{\pgfqpoint{10.309189in}{2.441132in}}%
\pgfpathlineto{\pgfqpoint{10.309873in}{2.439360in}}%
\pgfpathlineto{\pgfqpoint{10.310558in}{2.438927in}}%
\pgfpathlineto{\pgfqpoint{10.312270in}{2.436873in}}%
\pgfpathlineto{\pgfqpoint{10.312955in}{2.436608in}}%
\pgfpathlineto{\pgfqpoint{10.313297in}{2.436193in}}%
\pgfpathlineto{\pgfqpoint{10.313639in}{2.435316in}}%
\pgfpathlineto{\pgfqpoint{10.313982in}{2.435249in}}%
\pgfpathlineto{\pgfqpoint{10.315009in}{2.434063in}}%
\pgfpathlineto{\pgfqpoint{10.315693in}{2.433984in}}%
\pgfpathlineto{\pgfqpoint{10.316378in}{2.432784in}}%
\pgfpathlineto{\pgfqpoint{10.317063in}{2.432455in}}%
\pgfpathlineto{\pgfqpoint{10.317747in}{2.432141in}}%
\pgfpathlineto{\pgfqpoint{10.318432in}{2.431398in}}%
\pgfpathlineto{\pgfqpoint{10.318774in}{2.431330in}}%
\pgfpathlineto{\pgfqpoint{10.319459in}{2.430658in}}%
\pgfpathlineto{\pgfqpoint{10.319802in}{2.430499in}}%
\pgfpathlineto{\pgfqpoint{10.321171in}{2.428400in}}%
\pgfpathlineto{\pgfqpoint{10.321513in}{2.428339in}}%
\pgfpathlineto{\pgfqpoint{10.322198in}{2.427660in}}%
\pgfpathlineto{\pgfqpoint{10.323225in}{2.427303in}}%
\pgfpathlineto{\pgfqpoint{10.325279in}{2.426451in}}%
\pgfpathlineto{\pgfqpoint{10.325621in}{2.426149in}}%
\pgfpathlineto{\pgfqpoint{10.325964in}{2.424979in}}%
\pgfpathlineto{\pgfqpoint{10.326306in}{2.424903in}}%
\pgfpathlineto{\pgfqpoint{10.327676in}{2.423166in}}%
\pgfpathlineto{\pgfqpoint{10.328018in}{2.423043in}}%
\pgfpathlineto{\pgfqpoint{10.329045in}{2.421092in}}%
\pgfpathlineto{\pgfqpoint{10.331441in}{2.417578in}}%
\pgfpathlineto{\pgfqpoint{10.332811in}{2.416415in}}%
\pgfpathlineto{\pgfqpoint{10.333495in}{2.415658in}}%
\pgfpathlineto{\pgfqpoint{10.333838in}{2.415493in}}%
\pgfpathlineto{\pgfqpoint{10.334865in}{2.414037in}}%
\pgfpathlineto{\pgfqpoint{10.335549in}{2.413650in}}%
\pgfpathlineto{\pgfqpoint{10.335892in}{2.413644in}}%
\pgfpathlineto{\pgfqpoint{10.338973in}{2.410971in}}%
\pgfpathlineto{\pgfqpoint{10.339658in}{2.410706in}}%
\pgfpathlineto{\pgfqpoint{10.340685in}{2.409158in}}%
\pgfpathlineto{\pgfqpoint{10.343081in}{2.407895in}}%
\pgfpathlineto{\pgfqpoint{10.343766in}{2.405867in}}%
\pgfpathlineto{\pgfqpoint{10.344450in}{2.405307in}}%
\pgfpathlineto{\pgfqpoint{10.345820in}{2.403878in}}%
\pgfpathlineto{\pgfqpoint{10.346505in}{2.402129in}}%
\pgfpathlineto{\pgfqpoint{10.347532in}{2.401639in}}%
\pgfpathlineto{\pgfqpoint{10.347874in}{2.401341in}}%
\pgfpathlineto{\pgfqpoint{10.348559in}{2.400172in}}%
\pgfpathlineto{\pgfqpoint{10.351640in}{2.397288in}}%
\pgfpathlineto{\pgfqpoint{10.352667in}{2.396094in}}%
\pgfpathlineto{\pgfqpoint{10.353352in}{2.395679in}}%
\pgfpathlineto{\pgfqpoint{10.354379in}{2.394622in}}%
\pgfpathlineto{\pgfqpoint{10.354721in}{2.393806in}}%
\pgfpathlineto{\pgfqpoint{10.355063in}{2.393751in}}%
\pgfpathlineto{\pgfqpoint{10.355406in}{2.392809in}}%
\pgfpathlineto{\pgfqpoint{10.356775in}{2.392609in}}%
\pgfpathlineto{\pgfqpoint{10.358144in}{2.392085in}}%
\pgfpathlineto{\pgfqpoint{10.358487in}{2.391533in}}%
\pgfpathlineto{\pgfqpoint{10.358829in}{2.391527in}}%
\pgfpathlineto{\pgfqpoint{10.359514in}{2.390997in}}%
\pgfpathlineto{\pgfqpoint{10.360198in}{2.390899in}}%
\pgfpathlineto{\pgfqpoint{10.360883in}{2.390299in}}%
\pgfpathlineto{\pgfqpoint{10.362595in}{2.389369in}}%
\pgfpathlineto{\pgfqpoint{10.362937in}{2.388758in}}%
\pgfpathlineto{\pgfqpoint{10.363622in}{2.388452in}}%
\pgfpathlineto{\pgfqpoint{10.364307in}{2.388046in}}%
\pgfpathlineto{\pgfqpoint{10.364649in}{2.388029in}}%
\pgfpathlineto{\pgfqpoint{10.365334in}{2.387485in}}%
\pgfpathlineto{\pgfqpoint{10.366361in}{2.386768in}}%
\pgfpathlineto{\pgfqpoint{10.368415in}{2.385635in}}%
\pgfpathlineto{\pgfqpoint{10.369099in}{2.384706in}}%
\pgfpathlineto{\pgfqpoint{10.369784in}{2.384351in}}%
\pgfpathlineto{\pgfqpoint{10.371154in}{2.381947in}}%
\pgfpathlineto{\pgfqpoint{10.372181in}{2.381180in}}%
\pgfpathlineto{\pgfqpoint{10.372865in}{2.380727in}}%
\pgfpathlineto{\pgfqpoint{10.373550in}{2.380115in}}%
\pgfpathlineto{\pgfqpoint{10.374577in}{2.379896in}}%
\pgfpathlineto{\pgfqpoint{10.375262in}{2.376838in}}%
\pgfpathlineto{\pgfqpoint{10.376631in}{2.376536in}}%
\pgfpathlineto{\pgfqpoint{10.376973in}{2.376058in}}%
\pgfpathlineto{\pgfqpoint{10.377316in}{2.374979in}}%
\pgfpathlineto{\pgfqpoint{10.378000in}{2.374818in}}%
\pgfpathlineto{\pgfqpoint{10.379028in}{2.373073in}}%
\pgfpathlineto{\pgfqpoint{10.381082in}{2.371249in}}%
\pgfpathlineto{\pgfqpoint{10.382793in}{2.369475in}}%
\pgfpathlineto{\pgfqpoint{10.384505in}{2.368521in}}%
\pgfpathlineto{\pgfqpoint{10.386217in}{2.368040in}}%
\pgfpathlineto{\pgfqpoint{10.386901in}{2.367152in}}%
\pgfpathlineto{\pgfqpoint{10.387244in}{2.367093in}}%
\pgfpathlineto{\pgfqpoint{10.388271in}{2.366278in}}%
\pgfpathlineto{\pgfqpoint{10.389298in}{2.365940in}}%
\pgfpathlineto{\pgfqpoint{10.389983in}{2.365163in}}%
\pgfpathlineto{\pgfqpoint{10.390667in}{2.364815in}}%
\pgfpathlineto{\pgfqpoint{10.392379in}{2.363638in}}%
\pgfpathlineto{\pgfqpoint{10.393406in}{2.363094in}}%
\pgfpathlineto{\pgfqpoint{10.394091in}{2.361767in}}%
\pgfpathlineto{\pgfqpoint{10.394433in}{2.361659in}}%
\pgfpathlineto{\pgfqpoint{10.395118in}{2.360451in}}%
\pgfpathlineto{\pgfqpoint{10.395803in}{2.359847in}}%
\pgfpathlineto{\pgfqpoint{10.397172in}{2.355921in}}%
\pgfpathlineto{\pgfqpoint{10.399568in}{2.354821in}}%
\pgfpathlineto{\pgfqpoint{10.399911in}{2.354145in}}%
\pgfpathlineto{\pgfqpoint{10.400253in}{2.354145in}}%
\pgfpathlineto{\pgfqpoint{10.400938in}{2.353696in}}%
\pgfpathlineto{\pgfqpoint{10.401622in}{2.353314in}}%
\pgfpathlineto{\pgfqpoint{10.402307in}{2.352925in}}%
\pgfpathlineto{\pgfqpoint{10.403334in}{2.352204in}}%
\pgfpathlineto{\pgfqpoint{10.403676in}{2.350083in}}%
\pgfpathlineto{\pgfqpoint{10.404704in}{2.349481in}}%
\pgfpathlineto{\pgfqpoint{10.409839in}{2.346042in}}%
\pgfpathlineto{\pgfqpoint{10.410181in}{2.346027in}}%
\pgfpathlineto{\pgfqpoint{10.411208in}{2.345234in}}%
\pgfpathlineto{\pgfqpoint{10.411893in}{2.344985in}}%
\pgfpathlineto{\pgfqpoint{10.412235in}{2.344386in}}%
\pgfpathlineto{\pgfqpoint{10.413262in}{2.344109in}}%
\pgfpathlineto{\pgfqpoint{10.413947in}{2.343648in}}%
\pgfpathlineto{\pgfqpoint{10.414632in}{2.343271in}}%
\pgfpathlineto{\pgfqpoint{10.414974in}{2.342418in}}%
\pgfpathlineto{\pgfqpoint{10.416686in}{2.341638in}}%
\pgfpathlineto{\pgfqpoint{10.418397in}{2.340628in}}%
\pgfpathlineto{\pgfqpoint{10.418740in}{2.340552in}}%
\pgfpathlineto{\pgfqpoint{10.419424in}{2.339220in}}%
\pgfpathlineto{\pgfqpoint{10.419767in}{2.339155in}}%
\pgfpathlineto{\pgfqpoint{10.420451in}{2.338459in}}%
\pgfpathlineto{\pgfqpoint{10.421136in}{2.338174in}}%
\pgfpathlineto{\pgfqpoint{10.422506in}{2.337116in}}%
\pgfpathlineto{\pgfqpoint{10.423190in}{2.335772in}}%
\pgfpathlineto{\pgfqpoint{10.424560in}{2.335534in}}%
\pgfpathlineto{\pgfqpoint{10.425587in}{2.334897in}}%
\pgfpathlineto{\pgfqpoint{10.425929in}{2.334767in}}%
\pgfpathlineto{\pgfqpoint{10.426614in}{2.333492in}}%
\pgfpathlineto{\pgfqpoint{10.429010in}{2.331302in}}%
\pgfpathlineto{\pgfqpoint{10.429352in}{2.329403in}}%
\pgfpathlineto{\pgfqpoint{10.430380in}{2.328871in}}%
\pgfpathlineto{\pgfqpoint{10.432776in}{2.327715in}}%
\pgfpathlineto{\pgfqpoint{10.433118in}{2.327043in}}%
\pgfpathlineto{\pgfqpoint{10.433461in}{2.327043in}}%
\pgfpathlineto{\pgfqpoint{10.434488in}{2.325170in}}%
\pgfpathlineto{\pgfqpoint{10.435172in}{2.324535in}}%
\pgfpathlineto{\pgfqpoint{10.436199in}{2.322026in}}%
\pgfpathlineto{\pgfqpoint{10.437911in}{2.321183in}}%
\pgfpathlineto{\pgfqpoint{10.438596in}{2.320585in}}%
\pgfpathlineto{\pgfqpoint{10.440308in}{2.319755in}}%
\pgfpathlineto{\pgfqpoint{10.440992in}{2.318426in}}%
\pgfpathlineto{\pgfqpoint{10.441335in}{2.318426in}}%
\pgfpathlineto{\pgfqpoint{10.441677in}{2.317926in}}%
\pgfpathlineto{\pgfqpoint{10.442019in}{2.317920in}}%
\pgfpathlineto{\pgfqpoint{10.442704in}{2.316501in}}%
\pgfpathlineto{\pgfqpoint{10.443389in}{2.316411in}}%
\pgfpathlineto{\pgfqpoint{10.445785in}{2.314952in}}%
\pgfpathlineto{\pgfqpoint{10.446812in}{2.313147in}}%
\pgfpathlineto{\pgfqpoint{10.449551in}{2.312380in}}%
\pgfpathlineto{\pgfqpoint{10.451263in}{2.309704in}}%
\pgfpathlineto{\pgfqpoint{10.451947in}{2.308307in}}%
\pgfpathlineto{\pgfqpoint{10.452974in}{2.307499in}}%
\pgfpathlineto{\pgfqpoint{10.453317in}{2.307439in}}%
\pgfpathlineto{\pgfqpoint{10.454001in}{2.305576in}}%
\pgfpathlineto{\pgfqpoint{10.454344in}{2.305051in}}%
\pgfpathlineto{\pgfqpoint{10.455028in}{2.302806in}}%
\pgfpathlineto{\pgfqpoint{10.457425in}{2.301608in}}%
\pgfpathlineto{\pgfqpoint{10.459137in}{2.299508in}}%
\pgfpathlineto{\pgfqpoint{10.459479in}{2.299399in}}%
\pgfpathlineto{\pgfqpoint{10.461875in}{2.295728in}}%
\pgfpathlineto{\pgfqpoint{10.463245in}{2.295130in}}%
\pgfpathlineto{\pgfqpoint{10.464272in}{2.293151in}}%
\pgfpathlineto{\pgfqpoint{10.465984in}{2.290883in}}%
\pgfpathlineto{\pgfqpoint{10.466326in}{2.290775in}}%
\pgfpathlineto{\pgfqpoint{10.467011in}{2.289833in}}%
\pgfpathlineto{\pgfqpoint{10.467353in}{2.289754in}}%
\pgfpathlineto{\pgfqpoint{10.468038in}{2.289194in}}%
\pgfpathlineto{\pgfqpoint{10.468380in}{2.287744in}}%
\pgfpathlineto{\pgfqpoint{10.469065in}{2.287472in}}%
\pgfpathlineto{\pgfqpoint{10.470776in}{2.286294in}}%
\pgfpathlineto{\pgfqpoint{10.471803in}{2.286020in}}%
\pgfpathlineto{\pgfqpoint{10.472146in}{2.285479in}}%
\pgfpathlineto{\pgfqpoint{10.472488in}{2.285464in}}%
\pgfpathlineto{\pgfqpoint{10.473515in}{2.284520in}}%
\pgfpathlineto{\pgfqpoint{10.474200in}{2.284391in}}%
\pgfpathlineto{\pgfqpoint{10.474542in}{2.283508in}}%
\pgfpathlineto{\pgfqpoint{10.474885in}{2.283462in}}%
\pgfpathlineto{\pgfqpoint{10.475569in}{2.282594in}}%
\pgfpathlineto{\pgfqpoint{10.476254in}{2.281884in}}%
\pgfpathlineto{\pgfqpoint{10.477281in}{2.281329in}}%
\pgfpathlineto{\pgfqpoint{10.477966in}{2.280480in}}%
\pgfpathlineto{\pgfqpoint{10.478993in}{2.280253in}}%
\pgfpathlineto{\pgfqpoint{10.479335in}{2.279679in}}%
\pgfpathlineto{\pgfqpoint{10.479677in}{2.279679in}}%
\pgfpathlineto{\pgfqpoint{10.481047in}{2.276477in}}%
\pgfpathlineto{\pgfqpoint{10.482074in}{2.275458in}}%
\pgfpathlineto{\pgfqpoint{10.482759in}{2.274023in}}%
\pgfpathlineto{\pgfqpoint{10.486182in}{2.271977in}}%
\pgfpathlineto{\pgfqpoint{10.486867in}{2.271463in}}%
\pgfpathlineto{\pgfqpoint{10.487551in}{2.270772in}}%
\pgfpathlineto{\pgfqpoint{10.488236in}{2.270290in}}%
\pgfpathlineto{\pgfqpoint{10.489263in}{2.269288in}}%
\pgfpathlineto{\pgfqpoint{10.490290in}{2.268683in}}%
\pgfpathlineto{\pgfqpoint{10.490975in}{2.268533in}}%
\pgfpathlineto{\pgfqpoint{10.491660in}{2.267234in}}%
\pgfpathlineto{\pgfqpoint{10.493714in}{2.266056in}}%
\pgfpathlineto{\pgfqpoint{10.496110in}{2.264574in}}%
\pgfpathlineto{\pgfqpoint{10.497822in}{2.263549in}}%
\pgfpathlineto{\pgfqpoint{10.498849in}{2.263025in}}%
\pgfpathlineto{\pgfqpoint{10.499534in}{2.262545in}}%
\pgfpathlineto{\pgfqpoint{10.500903in}{2.260791in}}%
\pgfpathlineto{\pgfqpoint{10.501588in}{2.260770in}}%
\pgfpathlineto{\pgfqpoint{10.503984in}{2.258233in}}%
\pgfpathlineto{\pgfqpoint{10.504669in}{2.258157in}}%
\pgfpathlineto{\pgfqpoint{10.505011in}{2.257778in}}%
\pgfpathlineto{\pgfqpoint{10.505353in}{2.256902in}}%
\pgfpathlineto{\pgfqpoint{10.506038in}{2.256873in}}%
\pgfpathlineto{\pgfqpoint{10.506380in}{2.255605in}}%
\pgfpathlineto{\pgfqpoint{10.508777in}{2.254087in}}%
\pgfpathlineto{\pgfqpoint{10.509119in}{2.253177in}}%
\pgfpathlineto{\pgfqpoint{10.510146in}{2.252614in}}%
\pgfpathlineto{\pgfqpoint{10.511516in}{2.250991in}}%
\pgfpathlineto{\pgfqpoint{10.512200in}{2.250915in}}%
\pgfpathlineto{\pgfqpoint{10.514254in}{2.249167in}}%
\pgfpathlineto{\pgfqpoint{10.515966in}{2.248594in}}%
\pgfpathlineto{\pgfqpoint{10.516309in}{2.247895in}}%
\pgfpathlineto{\pgfqpoint{10.516993in}{2.247751in}}%
\pgfpathlineto{\pgfqpoint{10.517336in}{2.247509in}}%
\pgfpathlineto{\pgfqpoint{10.518020in}{2.244987in}}%
\pgfpathlineto{\pgfqpoint{10.518363in}{2.244874in}}%
\pgfpathlineto{\pgfqpoint{10.519047in}{2.244430in}}%
\pgfpathlineto{\pgfqpoint{10.519732in}{2.244006in}}%
\pgfpathlineto{\pgfqpoint{10.521101in}{2.241468in}}%
\pgfpathlineto{\pgfqpoint{10.522128in}{2.240865in}}%
\pgfpathlineto{\pgfqpoint{10.522813in}{2.238429in}}%
\pgfpathlineto{\pgfqpoint{10.523498in}{2.238266in}}%
\pgfpathlineto{\pgfqpoint{10.524867in}{2.237617in}}%
\pgfpathlineto{\pgfqpoint{10.527264in}{2.236665in}}%
\pgfpathlineto{\pgfqpoint{10.528291in}{2.235623in}}%
\pgfpathlineto{\pgfqpoint{10.528633in}{2.235501in}}%
\pgfpathlineto{\pgfqpoint{10.529660in}{2.234075in}}%
\pgfpathlineto{\pgfqpoint{10.530345in}{2.233282in}}%
\pgfpathlineto{\pgfqpoint{10.530687in}{2.232980in}}%
\pgfpathlineto{\pgfqpoint{10.531029in}{2.232044in}}%
\pgfpathlineto{\pgfqpoint{10.531714in}{2.231847in}}%
\pgfpathlineto{\pgfqpoint{10.532399in}{2.231168in}}%
\pgfpathlineto{\pgfqpoint{10.533426in}{2.230885in}}%
\pgfpathlineto{\pgfqpoint{10.533768in}{2.229620in}}%
\pgfpathlineto{\pgfqpoint{10.534453in}{2.229506in}}%
\pgfpathlineto{\pgfqpoint{10.537534in}{2.227672in}}%
\pgfpathlineto{\pgfqpoint{10.538219in}{2.227409in}}%
\pgfpathlineto{\pgfqpoint{10.538903in}{2.227241in}}%
\pgfpathlineto{\pgfqpoint{10.539588in}{2.226305in}}%
\pgfpathlineto{\pgfqpoint{10.539930in}{2.224795in}}%
\pgfpathlineto{\pgfqpoint{10.540615in}{2.224664in}}%
\pgfpathlineto{\pgfqpoint{10.541642in}{2.224018in}}%
\pgfpathlineto{\pgfqpoint{10.542669in}{2.223541in}}%
\pgfpathlineto{\pgfqpoint{10.543012in}{2.222982in}}%
\pgfpathlineto{\pgfqpoint{10.543696in}{2.222710in}}%
\pgfpathlineto{\pgfqpoint{10.544723in}{2.222332in}}%
\pgfpathlineto{\pgfqpoint{10.545750in}{2.221019in}}%
\pgfpathlineto{\pgfqpoint{10.546093in}{2.220938in}}%
\pgfpathlineto{\pgfqpoint{10.546777in}{2.219614in}}%
\pgfpathlineto{\pgfqpoint{10.547120in}{2.219312in}}%
\pgfpathlineto{\pgfqpoint{10.547804in}{2.218435in}}%
\pgfpathlineto{\pgfqpoint{10.548147in}{2.218240in}}%
\pgfpathlineto{\pgfqpoint{10.549174in}{2.216741in}}%
\pgfpathlineto{\pgfqpoint{10.549859in}{2.216500in}}%
\pgfpathlineto{\pgfqpoint{10.550543in}{2.214983in}}%
\pgfpathlineto{\pgfqpoint{10.550886in}{2.214977in}}%
\pgfpathlineto{\pgfqpoint{10.551228in}{2.213950in}}%
\pgfpathlineto{\pgfqpoint{10.551570in}{2.213888in}}%
\pgfpathlineto{\pgfqpoint{10.551913in}{2.213081in}}%
\pgfpathlineto{\pgfqpoint{10.553282in}{2.212802in}}%
\pgfpathlineto{\pgfqpoint{10.553967in}{2.211991in}}%
\pgfpathlineto{\pgfqpoint{10.554309in}{2.211843in}}%
\pgfpathlineto{\pgfqpoint{10.554651in}{2.210809in}}%
\pgfpathlineto{\pgfqpoint{10.555678in}{2.210570in}}%
\pgfpathlineto{\pgfqpoint{10.557390in}{2.208664in}}%
\pgfpathlineto{\pgfqpoint{10.558075in}{2.207892in}}%
\pgfpathlineto{\pgfqpoint{10.558760in}{2.207577in}}%
\pgfpathlineto{\pgfqpoint{10.559102in}{2.207569in}}%
\pgfpathlineto{\pgfqpoint{10.559444in}{2.206145in}}%
\pgfpathlineto{\pgfqpoint{10.559787in}{2.206134in}}%
\pgfpathlineto{\pgfqpoint{10.560471in}{2.205757in}}%
\pgfpathlineto{\pgfqpoint{10.561498in}{2.205456in}}%
\pgfpathlineto{\pgfqpoint{10.562183in}{2.205077in}}%
\pgfpathlineto{\pgfqpoint{10.562868in}{2.203831in}}%
\pgfpathlineto{\pgfqpoint{10.563895in}{2.203567in}}%
\pgfpathlineto{\pgfqpoint{10.564579in}{2.203318in}}%
\pgfpathlineto{\pgfqpoint{10.565264in}{2.202067in}}%
\pgfpathlineto{\pgfqpoint{10.566976in}{2.201150in}}%
\pgfpathlineto{\pgfqpoint{10.568003in}{2.200018in}}%
\pgfpathlineto{\pgfqpoint{10.569715in}{2.199021in}}%
\pgfpathlineto{\pgfqpoint{10.570399in}{2.198303in}}%
\pgfpathlineto{\pgfqpoint{10.570742in}{2.198111in}}%
\pgfpathlineto{\pgfqpoint{10.571426in}{2.197004in}}%
\pgfpathlineto{\pgfqpoint{10.574850in}{2.194996in}}%
\pgfpathlineto{\pgfqpoint{10.575535in}{2.193863in}}%
\pgfpathlineto{\pgfqpoint{10.575877in}{2.193712in}}%
\pgfpathlineto{\pgfqpoint{10.576562in}{2.192643in}}%
\pgfpathlineto{\pgfqpoint{10.577246in}{2.192164in}}%
\pgfpathlineto{\pgfqpoint{10.578273in}{2.191371in}}%
\pgfpathlineto{\pgfqpoint{10.578958in}{2.191107in}}%
\pgfpathlineto{\pgfqpoint{10.579300in}{2.190841in}}%
\pgfpathlineto{\pgfqpoint{10.579643in}{2.189537in}}%
\pgfpathlineto{\pgfqpoint{10.581697in}{2.188577in}}%
\pgfpathlineto{\pgfqpoint{10.582381in}{2.188003in}}%
\pgfpathlineto{\pgfqpoint{10.583751in}{2.186553in}}%
\pgfpathlineto{\pgfqpoint{10.585805in}{2.185828in}}%
\pgfpathlineto{\pgfqpoint{10.586832in}{2.185401in}}%
\pgfpathlineto{\pgfqpoint{10.587517in}{2.184575in}}%
\pgfpathlineto{\pgfqpoint{10.589228in}{2.183843in}}%
\pgfpathlineto{\pgfqpoint{10.590255in}{2.182282in}}%
\pgfpathlineto{\pgfqpoint{10.590940in}{2.181063in}}%
\pgfpathlineto{\pgfqpoint{10.591625in}{2.180421in}}%
\pgfpathlineto{\pgfqpoint{10.592310in}{2.180006in}}%
\pgfpathlineto{\pgfqpoint{10.592994in}{2.178533in}}%
\pgfpathlineto{\pgfqpoint{10.595733in}{2.176110in}}%
\pgfpathlineto{\pgfqpoint{10.597102in}{2.175365in}}%
\pgfpathlineto{\pgfqpoint{10.597787in}{2.174708in}}%
\pgfpathlineto{\pgfqpoint{10.598472in}{2.174566in}}%
\pgfpathlineto{\pgfqpoint{10.598814in}{2.173801in}}%
\pgfpathlineto{\pgfqpoint{10.599499in}{2.173570in}}%
\pgfpathlineto{\pgfqpoint{10.600183in}{2.171529in}}%
\pgfpathlineto{\pgfqpoint{10.602580in}{2.170363in}}%
\pgfpathlineto{\pgfqpoint{10.603607in}{2.169623in}}%
\pgfpathlineto{\pgfqpoint{10.604634in}{2.169245in}}%
\pgfpathlineto{\pgfqpoint{10.606003in}{2.168339in}}%
\pgfpathlineto{\pgfqpoint{10.606346in}{2.168274in}}%
\pgfpathlineto{\pgfqpoint{10.607030in}{2.167704in}}%
\pgfpathlineto{\pgfqpoint{10.609769in}{2.165205in}}%
\pgfpathlineto{\pgfqpoint{10.610112in}{2.164412in}}%
\pgfpathlineto{\pgfqpoint{10.610454in}{2.164363in}}%
\pgfpathlineto{\pgfqpoint{10.611139in}{2.163838in}}%
\pgfpathlineto{\pgfqpoint{10.614220in}{2.162411in}}%
\pgfpathlineto{\pgfqpoint{10.614904in}{2.161635in}}%
\pgfpathlineto{\pgfqpoint{10.616274in}{2.160002in}}%
\pgfpathlineto{\pgfqpoint{10.617643in}{2.158484in}}%
\pgfpathlineto{\pgfqpoint{10.617986in}{2.157125in}}%
\pgfpathlineto{\pgfqpoint{10.619697in}{2.156427in}}%
\pgfpathlineto{\pgfqpoint{10.621409in}{2.154600in}}%
\pgfpathlineto{\pgfqpoint{10.622094in}{2.154232in}}%
\pgfpathlineto{\pgfqpoint{10.622778in}{2.153538in}}%
\pgfpathlineto{\pgfqpoint{10.623805in}{2.153273in}}%
\pgfpathlineto{\pgfqpoint{10.624148in}{2.153036in}}%
\pgfpathlineto{\pgfqpoint{10.624490in}{2.151031in}}%
\pgfpathlineto{\pgfqpoint{10.625175in}{2.150895in}}%
\pgfpathlineto{\pgfqpoint{10.625859in}{2.150094in}}%
\pgfpathlineto{\pgfqpoint{10.627914in}{2.148298in}}%
\pgfpathlineto{\pgfqpoint{10.628598in}{2.147157in}}%
\pgfpathlineto{\pgfqpoint{10.629625in}{2.146815in}}%
\pgfpathlineto{\pgfqpoint{10.629968in}{2.145956in}}%
\pgfpathlineto{\pgfqpoint{10.630652in}{2.145558in}}%
\pgfpathlineto{\pgfqpoint{10.631337in}{2.145382in}}%
\pgfpathlineto{\pgfqpoint{10.632022in}{2.144514in}}%
\pgfpathlineto{\pgfqpoint{10.635103in}{2.143117in}}%
\pgfpathlineto{\pgfqpoint{10.636472in}{2.141795in}}%
\pgfpathlineto{\pgfqpoint{10.637157in}{2.140579in}}%
\pgfpathlineto{\pgfqpoint{10.637842in}{2.140303in}}%
\pgfpathlineto{\pgfqpoint{10.638869in}{2.138621in}}%
\pgfpathlineto{\pgfqpoint{10.639553in}{2.138064in}}%
\pgfpathlineto{\pgfqpoint{10.639896in}{2.137906in}}%
\pgfpathlineto{\pgfqpoint{10.640580in}{2.137075in}}%
\pgfpathlineto{\pgfqpoint{10.641265in}{2.136862in}}%
\pgfpathlineto{\pgfqpoint{10.641607in}{2.136834in}}%
\pgfpathlineto{\pgfqpoint{10.642292in}{2.136420in}}%
\pgfpathlineto{\pgfqpoint{10.644346in}{2.134201in}}%
\pgfpathlineto{\pgfqpoint{10.644689in}{2.133451in}}%
\pgfpathlineto{\pgfqpoint{10.645031in}{2.133399in}}%
\pgfpathlineto{\pgfqpoint{10.646058in}{2.131245in}}%
\pgfpathlineto{\pgfqpoint{10.647085in}{2.131072in}}%
\pgfpathlineto{\pgfqpoint{10.647770in}{2.129331in}}%
\pgfpathlineto{\pgfqpoint{10.649481in}{2.128631in}}%
\pgfpathlineto{\pgfqpoint{10.650166in}{2.128074in}}%
\pgfpathlineto{\pgfqpoint{10.650851in}{2.127828in}}%
\pgfpathlineto{\pgfqpoint{10.651878in}{2.126503in}}%
\pgfpathlineto{\pgfqpoint{10.653590in}{2.126141in}}%
\pgfpathlineto{\pgfqpoint{10.654617in}{2.125258in}}%
\pgfpathlineto{\pgfqpoint{10.657355in}{2.123890in}}%
\pgfpathlineto{\pgfqpoint{10.658040in}{2.123596in}}%
\pgfpathlineto{\pgfqpoint{10.658725in}{2.121987in}}%
\pgfpathlineto{\pgfqpoint{10.659409in}{2.121632in}}%
\pgfpathlineto{\pgfqpoint{10.660779in}{2.119756in}}%
\pgfpathlineto{\pgfqpoint{10.661121in}{2.119699in}}%
\pgfpathlineto{\pgfqpoint{10.661464in}{2.118952in}}%
\pgfpathlineto{\pgfqpoint{10.661806in}{2.118933in}}%
\pgfpathlineto{\pgfqpoint{10.662491in}{2.118574in}}%
\pgfpathlineto{\pgfqpoint{10.662833in}{2.118559in}}%
\pgfpathlineto{\pgfqpoint{10.663518in}{2.117932in}}%
\pgfpathlineto{\pgfqpoint{10.663860in}{2.117809in}}%
\pgfpathlineto{\pgfqpoint{10.664202in}{2.116189in}}%
\pgfpathlineto{\pgfqpoint{10.664887in}{2.116044in}}%
\pgfpathlineto{\pgfqpoint{10.665229in}{2.114995in}}%
\pgfpathlineto{\pgfqpoint{10.667626in}{2.114050in}}%
\pgfpathlineto{\pgfqpoint{10.668995in}{2.112117in}}%
\pgfpathlineto{\pgfqpoint{10.669680in}{2.111823in}}%
\pgfpathlineto{\pgfqpoint{10.672761in}{2.108742in}}%
\pgfpathlineto{\pgfqpoint{10.673446in}{2.107322in}}%
\pgfpathlineto{\pgfqpoint{10.675500in}{2.106054in}}%
\pgfpathlineto{\pgfqpoint{10.676184in}{2.105925in}}%
\pgfpathlineto{\pgfqpoint{10.676527in}{2.105450in}}%
\pgfpathlineto{\pgfqpoint{10.677211in}{2.103478in}}%
\pgfpathlineto{\pgfqpoint{10.678923in}{2.101583in}}%
\pgfpathlineto{\pgfqpoint{10.680293in}{2.101036in}}%
\pgfpathlineto{\pgfqpoint{10.681320in}{2.100828in}}%
\pgfpathlineto{\pgfqpoint{10.682004in}{2.100148in}}%
\pgfpathlineto{\pgfqpoint{10.682689in}{2.099884in}}%
\pgfpathlineto{\pgfqpoint{10.683716in}{2.096961in}}%
\pgfpathlineto{\pgfqpoint{10.685428in}{2.096138in}}%
\pgfpathlineto{\pgfqpoint{10.686455in}{2.093639in}}%
\pgfpathlineto{\pgfqpoint{10.686797in}{2.093585in}}%
\pgfpathlineto{\pgfqpoint{10.687482in}{2.092589in}}%
\pgfpathlineto{\pgfqpoint{10.688851in}{2.091930in}}%
\pgfpathlineto{\pgfqpoint{10.690221in}{2.090482in}}%
\pgfpathlineto{\pgfqpoint{10.690563in}{2.090452in}}%
\pgfpathlineto{\pgfqpoint{10.691590in}{2.089380in}}%
\pgfpathlineto{\pgfqpoint{10.692617in}{2.089156in}}%
\pgfpathlineto{\pgfqpoint{10.693986in}{2.087779in}}%
\pgfpathlineto{\pgfqpoint{10.695698in}{2.087356in}}%
\pgfpathlineto{\pgfqpoint{10.696041in}{2.084969in}}%
\pgfpathlineto{\pgfqpoint{10.696383in}{2.084639in}}%
\pgfpathlineto{\pgfqpoint{10.697068in}{2.083621in}}%
\pgfpathlineto{\pgfqpoint{10.698095in}{2.083500in}}%
\pgfpathlineto{\pgfqpoint{10.699806in}{2.080590in}}%
\pgfpathlineto{\pgfqpoint{10.700491in}{2.080138in}}%
\pgfpathlineto{\pgfqpoint{10.701176in}{2.079925in}}%
\pgfpathlineto{\pgfqpoint{10.701518in}{2.079426in}}%
\pgfpathlineto{\pgfqpoint{10.702203in}{2.077266in}}%
\pgfpathlineto{\pgfqpoint{10.703230in}{2.077003in}}%
\pgfpathlineto{\pgfqpoint{10.703915in}{2.075606in}}%
\pgfpathlineto{\pgfqpoint{10.704599in}{2.075266in}}%
\pgfpathlineto{\pgfqpoint{10.705969in}{2.073068in}}%
\pgfpathlineto{\pgfqpoint{10.706996in}{2.072706in}}%
\pgfpathlineto{\pgfqpoint{10.708023in}{2.072162in}}%
\pgfpathlineto{\pgfqpoint{10.708365in}{2.071900in}}%
\pgfpathlineto{\pgfqpoint{10.709734in}{2.068825in}}%
\pgfpathlineto{\pgfqpoint{10.710761in}{2.068557in}}%
\pgfpathlineto{\pgfqpoint{10.711446in}{2.067812in}}%
\pgfpathlineto{\pgfqpoint{10.711789in}{2.067746in}}%
\pgfpathlineto{\pgfqpoint{10.712816in}{2.065909in}}%
\pgfpathlineto{\pgfqpoint{10.713158in}{2.065879in}}%
\pgfpathlineto{\pgfqpoint{10.713500in}{2.065216in}}%
\pgfpathlineto{\pgfqpoint{10.713843in}{2.065171in}}%
\pgfpathlineto{\pgfqpoint{10.714870in}{2.063484in}}%
\pgfpathlineto{\pgfqpoint{10.715897in}{2.063273in}}%
\pgfpathlineto{\pgfqpoint{10.716581in}{2.062390in}}%
\pgfpathlineto{\pgfqpoint{10.717608in}{2.062258in}}%
\pgfpathlineto{\pgfqpoint{10.719662in}{2.060653in}}%
\pgfpathlineto{\pgfqpoint{10.720005in}{2.060648in}}%
\pgfpathlineto{\pgfqpoint{10.720347in}{2.060314in}}%
\pgfpathlineto{\pgfqpoint{10.721032in}{2.058448in}}%
\pgfpathlineto{\pgfqpoint{10.721717in}{2.057689in}}%
\pgfpathlineto{\pgfqpoint{10.722059in}{2.056991in}}%
\pgfpathlineto{\pgfqpoint{10.722401in}{2.056953in}}%
\pgfpathlineto{\pgfqpoint{10.723771in}{2.055971in}}%
\pgfpathlineto{\pgfqpoint{10.724455in}{2.055934in}}%
\pgfpathlineto{\pgfqpoint{10.724798in}{2.053998in}}%
\pgfpathlineto{\pgfqpoint{10.725140in}{2.053797in}}%
\pgfpathlineto{\pgfqpoint{10.725825in}{2.053223in}}%
\pgfpathlineto{\pgfqpoint{10.727879in}{2.050873in}}%
\pgfpathlineto{\pgfqpoint{10.728906in}{2.050410in}}%
\pgfpathlineto{\pgfqpoint{10.729933in}{2.050172in}}%
\pgfpathlineto{\pgfqpoint{10.730960in}{2.049658in}}%
\pgfpathlineto{\pgfqpoint{10.731987in}{2.048994in}}%
\pgfpathlineto{\pgfqpoint{10.732329in}{2.048994in}}%
\pgfpathlineto{\pgfqpoint{10.733014in}{2.048673in}}%
\pgfpathlineto{\pgfqpoint{10.733699in}{2.047212in}}%
\pgfpathlineto{\pgfqpoint{10.734726in}{2.046683in}}%
\pgfpathlineto{\pgfqpoint{10.735410in}{2.045973in}}%
\pgfpathlineto{\pgfqpoint{10.737122in}{2.045103in}}%
\pgfpathlineto{\pgfqpoint{10.738149in}{2.044366in}}%
\pgfpathlineto{\pgfqpoint{10.738834in}{2.043545in}}%
\pgfpathlineto{\pgfqpoint{10.739519in}{2.041737in}}%
\pgfpathlineto{\pgfqpoint{10.740888in}{2.040506in}}%
\pgfpathlineto{\pgfqpoint{10.741230in}{2.040453in}}%
\pgfpathlineto{\pgfqpoint{10.742600in}{2.038965in}}%
\pgfpathlineto{\pgfqpoint{10.742942in}{2.038875in}}%
\pgfpathlineto{\pgfqpoint{10.743284in}{2.038460in}}%
\pgfpathlineto{\pgfqpoint{10.743969in}{2.037285in}}%
\pgfpathlineto{\pgfqpoint{10.748420in}{2.034329in}}%
\pgfpathlineto{\pgfqpoint{10.749104in}{2.032723in}}%
\pgfpathlineto{\pgfqpoint{10.751501in}{2.030323in}}%
\pgfpathlineto{\pgfqpoint{10.753212in}{2.029632in}}%
\pgfpathlineto{\pgfqpoint{10.753897in}{2.028786in}}%
\pgfpathlineto{\pgfqpoint{10.755267in}{2.027985in}}%
\pgfpathlineto{\pgfqpoint{10.756294in}{2.026732in}}%
\pgfpathlineto{\pgfqpoint{10.756978in}{2.026563in}}%
\pgfpathlineto{\pgfqpoint{10.757663in}{2.025471in}}%
\pgfpathlineto{\pgfqpoint{10.758348in}{2.024708in}}%
\pgfpathlineto{\pgfqpoint{10.759032in}{2.024497in}}%
\pgfpathlineto{\pgfqpoint{10.761086in}{2.022302in}}%
\pgfpathlineto{\pgfqpoint{10.762113in}{2.021509in}}%
\pgfpathlineto{\pgfqpoint{10.762798in}{2.019452in}}%
\pgfpathlineto{\pgfqpoint{10.763141in}{2.019393in}}%
\pgfpathlineto{\pgfqpoint{10.766564in}{2.012952in}}%
\pgfpathlineto{\pgfqpoint{10.767249in}{2.012928in}}%
\pgfpathlineto{\pgfqpoint{10.768618in}{2.011040in}}%
\pgfpathlineto{\pgfqpoint{10.769303in}{2.010436in}}%
\pgfpathlineto{\pgfqpoint{10.769645in}{2.010118in}}%
\pgfpathlineto{\pgfqpoint{10.770330in}{2.008880in}}%
\pgfpathlineto{\pgfqpoint{10.770672in}{2.008819in}}%
\pgfpathlineto{\pgfqpoint{10.771014in}{2.008123in}}%
\pgfpathlineto{\pgfqpoint{10.772042in}{2.007827in}}%
\pgfpathlineto{\pgfqpoint{10.773069in}{2.006886in}}%
\pgfpathlineto{\pgfqpoint{10.773411in}{2.006768in}}%
\pgfpathlineto{\pgfqpoint{10.774096in}{2.005708in}}%
\pgfpathlineto{\pgfqpoint{10.775123in}{2.005086in}}%
\pgfpathlineto{\pgfqpoint{10.775807in}{2.004591in}}%
\pgfpathlineto{\pgfqpoint{10.777177in}{2.003886in}}%
\pgfpathlineto{\pgfqpoint{10.777519in}{2.002869in}}%
\pgfpathlineto{\pgfqpoint{10.778204in}{2.002431in}}%
\pgfpathlineto{\pgfqpoint{10.779573in}{2.001479in}}%
\pgfpathlineto{\pgfqpoint{10.781627in}{1.999903in}}%
\pgfpathlineto{\pgfqpoint{10.781970in}{1.999217in}}%
\pgfpathlineto{\pgfqpoint{10.783681in}{1.998858in}}%
\pgfpathlineto{\pgfqpoint{10.784366in}{1.998504in}}%
\pgfpathlineto{\pgfqpoint{10.784708in}{1.998398in}}%
\pgfpathlineto{\pgfqpoint{10.785393in}{1.997824in}}%
\pgfpathlineto{\pgfqpoint{10.786078in}{1.997643in}}%
\pgfpathlineto{\pgfqpoint{10.787447in}{1.996270in}}%
\pgfpathlineto{\pgfqpoint{10.788132in}{1.994502in}}%
\pgfpathlineto{\pgfqpoint{10.789159in}{1.993841in}}%
\pgfpathlineto{\pgfqpoint{10.789844in}{1.993444in}}%
\pgfpathlineto{\pgfqpoint{10.790528in}{1.992535in}}%
\pgfpathlineto{\pgfqpoint{10.792582in}{1.990876in}}%
\pgfpathlineto{\pgfqpoint{10.793609in}{1.989555in}}%
\pgfpathlineto{\pgfqpoint{10.794979in}{1.988128in}}%
\pgfpathlineto{\pgfqpoint{10.795321in}{1.987856in}}%
\pgfpathlineto{\pgfqpoint{10.796006in}{1.986437in}}%
\pgfpathlineto{\pgfqpoint{10.797718in}{1.985440in}}%
\pgfpathlineto{\pgfqpoint{10.798745in}{1.985307in}}%
\pgfpathlineto{\pgfqpoint{10.799429in}{1.984269in}}%
\pgfpathlineto{\pgfqpoint{10.800114in}{1.984232in}}%
\pgfpathlineto{\pgfqpoint{10.800799in}{1.983425in}}%
\pgfpathlineto{\pgfqpoint{10.802168in}{1.982933in}}%
\pgfpathlineto{\pgfqpoint{10.802853in}{1.982540in}}%
\pgfpathlineto{\pgfqpoint{10.803195in}{1.982532in}}%
\pgfpathlineto{\pgfqpoint{10.804564in}{1.981241in}}%
\pgfpathlineto{\pgfqpoint{10.805249in}{1.979701in}}%
\pgfpathlineto{\pgfqpoint{10.806276in}{1.977851in}}%
\pgfpathlineto{\pgfqpoint{10.806619in}{1.977851in}}%
\pgfpathlineto{\pgfqpoint{10.806961in}{1.976771in}}%
\pgfpathlineto{\pgfqpoint{10.807646in}{1.976604in}}%
\pgfpathlineto{\pgfqpoint{10.807988in}{1.975193in}}%
\pgfpathlineto{\pgfqpoint{10.808673in}{1.975019in}}%
\pgfpathlineto{\pgfqpoint{10.809015in}{1.973961in}}%
\pgfpathlineto{\pgfqpoint{10.809700in}{1.973933in}}%
\pgfpathlineto{\pgfqpoint{10.811069in}{1.972791in}}%
\pgfpathlineto{\pgfqpoint{10.812438in}{1.972225in}}%
\pgfpathlineto{\pgfqpoint{10.813465in}{1.970412in}}%
\pgfpathlineto{\pgfqpoint{10.813808in}{1.970397in}}%
\pgfpathlineto{\pgfqpoint{10.816547in}{1.967618in}}%
\pgfpathlineto{\pgfqpoint{10.817231in}{1.967165in}}%
\pgfpathlineto{\pgfqpoint{10.818258in}{1.966440in}}%
\pgfpathlineto{\pgfqpoint{10.819970in}{1.965919in}}%
\pgfpathlineto{\pgfqpoint{10.820655in}{1.965360in}}%
\pgfpathlineto{\pgfqpoint{10.821339in}{1.964628in}}%
\pgfpathlineto{\pgfqpoint{10.822024in}{1.964170in}}%
\pgfpathlineto{\pgfqpoint{10.822709in}{1.963087in}}%
\pgfpathlineto{\pgfqpoint{10.823394in}{1.962894in}}%
\pgfpathlineto{\pgfqpoint{10.824078in}{1.961577in}}%
\pgfpathlineto{\pgfqpoint{10.824421in}{1.961488in}}%
\pgfpathlineto{\pgfqpoint{10.825105in}{1.960248in}}%
\pgfpathlineto{\pgfqpoint{10.825790in}{1.959511in}}%
\pgfpathlineto{\pgfqpoint{10.826475in}{1.959198in}}%
\pgfpathlineto{\pgfqpoint{10.827159in}{1.958224in}}%
\pgfpathlineto{\pgfqpoint{10.827502in}{1.958141in}}%
\pgfpathlineto{\pgfqpoint{10.828871in}{1.956442in}}%
\pgfpathlineto{\pgfqpoint{10.829213in}{1.956381in}}%
\pgfpathlineto{\pgfqpoint{10.830583in}{1.954554in}}%
\pgfpathlineto{\pgfqpoint{10.833664in}{1.952893in}}%
\pgfpathlineto{\pgfqpoint{10.836745in}{1.950310in}}%
\pgfpathlineto{\pgfqpoint{10.837430in}{1.950068in}}%
\pgfpathlineto{\pgfqpoint{10.838114in}{1.948437in}}%
\pgfpathlineto{\pgfqpoint{10.839484in}{1.948019in}}%
\pgfpathlineto{\pgfqpoint{10.840169in}{1.947893in}}%
\pgfpathlineto{\pgfqpoint{10.841538in}{1.945687in}}%
\pgfpathlineto{\pgfqpoint{10.842223in}{1.945507in}}%
\pgfpathlineto{\pgfqpoint{10.842907in}{1.944571in}}%
\pgfpathlineto{\pgfqpoint{10.843934in}{1.944118in}}%
\pgfpathlineto{\pgfqpoint{10.844961in}{1.942925in}}%
\pgfpathlineto{\pgfqpoint{10.845304in}{1.942909in}}%
\pgfpathlineto{\pgfqpoint{10.845988in}{1.941263in}}%
\pgfpathlineto{\pgfqpoint{10.849070in}{1.939451in}}%
\pgfpathlineto{\pgfqpoint{10.849754in}{1.939436in}}%
\pgfpathlineto{\pgfqpoint{10.850439in}{1.938809in}}%
\pgfpathlineto{\pgfqpoint{10.852151in}{1.937774in}}%
\pgfpathlineto{\pgfqpoint{10.852835in}{1.936561in}}%
\pgfpathlineto{\pgfqpoint{10.853178in}{1.936455in}}%
\pgfpathlineto{\pgfqpoint{10.853520in}{1.935826in}}%
\pgfpathlineto{\pgfqpoint{10.854547in}{1.935605in}}%
\pgfpathlineto{\pgfqpoint{10.854889in}{1.934850in}}%
\pgfpathlineto{\pgfqpoint{10.855916in}{1.934686in}}%
\pgfpathlineto{\pgfqpoint{10.857286in}{1.931986in}}%
\pgfpathlineto{\pgfqpoint{10.857628in}{1.931793in}}%
\pgfpathlineto{\pgfqpoint{10.857971in}{1.929747in}}%
\pgfpathlineto{\pgfqpoint{10.860025in}{1.928018in}}%
\pgfpathlineto{\pgfqpoint{10.861736in}{1.926870in}}%
\pgfpathlineto{\pgfqpoint{10.862763in}{1.925692in}}%
\pgfpathlineto{\pgfqpoint{10.863790in}{1.925364in}}%
\pgfpathlineto{\pgfqpoint{10.864133in}{1.925360in}}%
\pgfpathlineto{\pgfqpoint{10.864817in}{1.924935in}}%
\pgfpathlineto{\pgfqpoint{10.865160in}{1.924846in}}%
\pgfpathlineto{\pgfqpoint{10.865845in}{1.924121in}}%
\pgfpathlineto{\pgfqpoint{10.866187in}{1.924114in}}%
\pgfpathlineto{\pgfqpoint{10.866529in}{1.922792in}}%
\pgfpathlineto{\pgfqpoint{10.867214in}{1.922581in}}%
\pgfpathlineto{\pgfqpoint{10.868926in}{1.920043in}}%
\pgfpathlineto{\pgfqpoint{10.869268in}{1.919763in}}%
\pgfpathlineto{\pgfqpoint{10.869953in}{1.918835in}}%
\pgfpathlineto{\pgfqpoint{10.871664in}{1.917816in}}%
\pgfpathlineto{\pgfqpoint{10.872349in}{1.916798in}}%
\pgfpathlineto{\pgfqpoint{10.874061in}{1.916388in}}%
\pgfpathlineto{\pgfqpoint{10.874746in}{1.915890in}}%
\pgfpathlineto{\pgfqpoint{10.875430in}{1.915633in}}%
\pgfpathlineto{\pgfqpoint{10.875773in}{1.914289in}}%
\pgfpathlineto{\pgfqpoint{10.877484in}{1.913851in}}%
\pgfpathlineto{\pgfqpoint{10.878169in}{1.913609in}}%
\pgfpathlineto{\pgfqpoint{10.879196in}{1.911847in}}%
\pgfpathlineto{\pgfqpoint{10.879538in}{1.911699in}}%
\pgfpathlineto{\pgfqpoint{10.880223in}{1.910770in}}%
\pgfpathlineto{\pgfqpoint{10.880565in}{1.910302in}}%
\pgfpathlineto{\pgfqpoint{10.880908in}{1.908958in}}%
\pgfpathlineto{\pgfqpoint{10.881592in}{1.908897in}}%
\pgfpathlineto{\pgfqpoint{10.882620in}{1.907848in}}%
\pgfpathlineto{\pgfqpoint{10.886043in}{1.906209in}}%
\pgfpathlineto{\pgfqpoint{10.887755in}{1.905960in}}%
\pgfpathlineto{\pgfqpoint{10.889124in}{1.904215in}}%
\pgfpathlineto{\pgfqpoint{10.889809in}{1.903996in}}%
\pgfpathlineto{\pgfqpoint{10.890493in}{1.903166in}}%
\pgfpathlineto{\pgfqpoint{10.890836in}{1.903143in}}%
\pgfpathlineto{\pgfqpoint{10.891178in}{1.902184in}}%
\pgfpathlineto{\pgfqpoint{10.891521in}{1.902108in}}%
\pgfpathlineto{\pgfqpoint{10.891863in}{1.901134in}}%
\pgfpathlineto{\pgfqpoint{10.893232in}{1.900590in}}%
\pgfpathlineto{\pgfqpoint{10.894259in}{1.900145in}}%
\pgfpathlineto{\pgfqpoint{10.894944in}{1.898606in}}%
\pgfpathlineto{\pgfqpoint{10.896998in}{1.896294in}}%
\pgfpathlineto{\pgfqpoint{10.897340in}{1.896143in}}%
\pgfpathlineto{\pgfqpoint{10.898025in}{1.895076in}}%
\pgfpathlineto{\pgfqpoint{10.900422in}{1.892465in}}%
\pgfpathlineto{\pgfqpoint{10.902133in}{1.890608in}}%
\pgfpathlineto{\pgfqpoint{10.902818in}{1.890479in}}%
\pgfpathlineto{\pgfqpoint{10.903160in}{1.889354in}}%
\pgfpathlineto{\pgfqpoint{10.903503in}{1.889203in}}%
\pgfpathlineto{\pgfqpoint{10.904187in}{1.887964in}}%
\pgfpathlineto{\pgfqpoint{10.905899in}{1.887481in}}%
\pgfpathlineto{\pgfqpoint{10.906241in}{1.887383in}}%
\pgfpathlineto{\pgfqpoint{10.907611in}{1.885669in}}%
\pgfpathlineto{\pgfqpoint{10.908296in}{1.885379in}}%
\pgfpathlineto{\pgfqpoint{10.909323in}{1.885064in}}%
\pgfpathlineto{\pgfqpoint{10.909665in}{1.884022in}}%
\pgfpathlineto{\pgfqpoint{10.910350in}{1.883890in}}%
\pgfpathlineto{\pgfqpoint{10.911034in}{1.882852in}}%
\pgfpathlineto{\pgfqpoint{10.913088in}{1.881493in}}%
\pgfpathlineto{\pgfqpoint{10.913773in}{1.880684in}}%
\pgfpathlineto{\pgfqpoint{10.914800in}{1.880128in}}%
\pgfpathlineto{\pgfqpoint{10.915142in}{1.879152in}}%
\pgfpathlineto{\pgfqpoint{10.917539in}{1.878170in}}%
\pgfpathlineto{\pgfqpoint{10.918224in}{1.877633in}}%
\pgfpathlineto{\pgfqpoint{10.918908in}{1.877392in}}%
\pgfpathlineto{\pgfqpoint{10.919593in}{1.877150in}}%
\pgfpathlineto{\pgfqpoint{10.919935in}{1.876546in}}%
\pgfpathlineto{\pgfqpoint{10.920278in}{1.876508in}}%
\pgfpathlineto{\pgfqpoint{10.920962in}{1.875187in}}%
\pgfpathlineto{\pgfqpoint{10.921305in}{1.875096in}}%
\pgfpathlineto{\pgfqpoint{10.921647in}{1.874234in}}%
\pgfpathlineto{\pgfqpoint{10.923701in}{1.873465in}}%
\pgfpathlineto{\pgfqpoint{10.924386in}{1.872982in}}%
\pgfpathlineto{\pgfqpoint{10.926440in}{1.871638in}}%
\pgfpathlineto{\pgfqpoint{10.926782in}{1.871487in}}%
\pgfpathlineto{\pgfqpoint{10.927467in}{1.870682in}}%
\pgfpathlineto{\pgfqpoint{10.931233in}{1.869659in}}%
\pgfpathlineto{\pgfqpoint{10.931917in}{1.869057in}}%
\pgfpathlineto{\pgfqpoint{10.932602in}{1.868633in}}%
\pgfpathlineto{\pgfqpoint{10.933972in}{1.866654in}}%
\pgfpathlineto{\pgfqpoint{10.934314in}{1.866578in}}%
\pgfpathlineto{\pgfqpoint{10.936026in}{1.864291in}}%
\pgfpathlineto{\pgfqpoint{10.938422in}{1.862425in}}%
\pgfpathlineto{\pgfqpoint{10.942873in}{1.857176in}}%
\pgfpathlineto{\pgfqpoint{10.946638in}{1.854888in}}%
\pgfpathlineto{\pgfqpoint{10.947665in}{1.854410in}}%
\pgfpathlineto{\pgfqpoint{10.950747in}{1.852419in}}%
\pgfpathlineto{\pgfqpoint{10.952116in}{1.850750in}}%
\pgfpathlineto{\pgfqpoint{10.952458in}{1.850629in}}%
\pgfpathlineto{\pgfqpoint{10.953485in}{1.848968in}}%
\pgfpathlineto{\pgfqpoint{10.954512in}{1.848636in}}%
\pgfpathlineto{\pgfqpoint{10.955197in}{1.848303in}}%
\pgfpathlineto{\pgfqpoint{10.955882in}{1.846869in}}%
\pgfpathlineto{\pgfqpoint{10.956566in}{1.846085in}}%
\pgfpathlineto{\pgfqpoint{10.957936in}{1.844742in}}%
\pgfpathlineto{\pgfqpoint{10.958620in}{1.843621in}}%
\pgfpathlineto{\pgfqpoint{10.959648in}{1.842300in}}%
\pgfpathlineto{\pgfqpoint{10.960332in}{1.841265in}}%
\pgfpathlineto{\pgfqpoint{10.961017in}{1.840330in}}%
\pgfpathlineto{\pgfqpoint{10.962386in}{1.839808in}}%
\pgfpathlineto{\pgfqpoint{10.962729in}{1.839100in}}%
\pgfpathlineto{\pgfqpoint{10.963071in}{1.839090in}}%
\pgfpathlineto{\pgfqpoint{10.963756in}{1.837693in}}%
\pgfpathlineto{\pgfqpoint{10.964440in}{1.836598in}}%
\pgfpathlineto{\pgfqpoint{10.965125in}{1.836296in}}%
\pgfpathlineto{\pgfqpoint{10.966152in}{1.834439in}}%
\pgfpathlineto{\pgfqpoint{10.966837in}{1.834033in}}%
\pgfpathlineto{\pgfqpoint{10.967179in}{1.833644in}}%
\pgfpathlineto{\pgfqpoint{10.967864in}{1.831954in}}%
\pgfpathlineto{\pgfqpoint{10.968206in}{1.831841in}}%
\pgfpathlineto{\pgfqpoint{10.968891in}{1.830754in}}%
\pgfpathlineto{\pgfqpoint{10.969576in}{1.830089in}}%
\pgfpathlineto{\pgfqpoint{10.970603in}{1.828767in}}%
\pgfpathlineto{\pgfqpoint{10.971287in}{1.827424in}}%
\pgfpathlineto{\pgfqpoint{10.972314in}{1.826464in}}%
\pgfpathlineto{\pgfqpoint{10.972657in}{1.826404in}}%
\pgfpathlineto{\pgfqpoint{10.974711in}{1.824350in}}%
\pgfpathlineto{\pgfqpoint{10.975738in}{1.824018in}}%
\pgfpathlineto{\pgfqpoint{10.976423in}{1.823496in}}%
\pgfpathlineto{\pgfqpoint{10.977450in}{1.823111in}}%
\pgfpathlineto{\pgfqpoint{10.978134in}{1.821722in}}%
\pgfpathlineto{\pgfqpoint{10.980188in}{1.821042in}}%
\pgfpathlineto{\pgfqpoint{10.980873in}{1.819796in}}%
\pgfpathlineto{\pgfqpoint{10.981558in}{1.819532in}}%
\pgfpathlineto{\pgfqpoint{10.981900in}{1.818792in}}%
\pgfpathlineto{\pgfqpoint{10.982242in}{1.818777in}}%
\pgfpathlineto{\pgfqpoint{10.982585in}{1.817742in}}%
\pgfpathlineto{\pgfqpoint{10.983269in}{1.817523in}}%
\pgfpathlineto{\pgfqpoint{10.984639in}{1.816518in}}%
\pgfpathlineto{\pgfqpoint{10.984981in}{1.816348in}}%
\pgfpathlineto{\pgfqpoint{10.985666in}{1.815235in}}%
\pgfpathlineto{\pgfqpoint{10.987720in}{1.814441in}}%
\pgfpathlineto{\pgfqpoint{10.990801in}{1.809277in}}%
\pgfpathlineto{\pgfqpoint{10.991486in}{1.809115in}}%
\pgfpathlineto{\pgfqpoint{10.991828in}{1.807751in}}%
\pgfpathlineto{\pgfqpoint{10.992170in}{1.807638in}}%
\pgfpathlineto{\pgfqpoint{10.992855in}{1.806357in}}%
\pgfpathlineto{\pgfqpoint{10.994225in}{1.805750in}}%
\pgfpathlineto{\pgfqpoint{10.995936in}{1.803115in}}%
\pgfpathlineto{\pgfqpoint{10.996621in}{1.802901in}}%
\pgfpathlineto{\pgfqpoint{10.996963in}{1.801846in}}%
\pgfpathlineto{\pgfqpoint{10.997990in}{1.801521in}}%
\pgfpathlineto{\pgfqpoint{11.000387in}{1.799581in}}%
\pgfpathlineto{\pgfqpoint{11.001071in}{1.799134in}}%
\pgfpathlineto{\pgfqpoint{11.001414in}{1.799007in}}%
\pgfpathlineto{\pgfqpoint{11.002441in}{1.798070in}}%
\pgfpathlineto{\pgfqpoint{11.003810in}{1.797708in}}%
\pgfpathlineto{\pgfqpoint{11.004153in}{1.797406in}}%
\pgfpathlineto{\pgfqpoint{11.005522in}{1.794899in}}%
\pgfpathlineto{\pgfqpoint{11.006207in}{1.794763in}}%
\pgfpathlineto{\pgfqpoint{11.006549in}{1.793781in}}%
\pgfpathlineto{\pgfqpoint{11.006891in}{1.793690in}}%
\pgfpathlineto{\pgfqpoint{11.007576in}{1.792112in}}%
\pgfpathlineto{\pgfqpoint{11.009288in}{1.791478in}}%
\pgfpathlineto{\pgfqpoint{11.014423in}{1.787770in}}%
\pgfpathlineto{\pgfqpoint{11.014765in}{1.787679in}}%
\pgfpathlineto{\pgfqpoint{11.015450in}{1.786834in}}%
\pgfpathlineto{\pgfqpoint{11.016135in}{1.786456in}}%
\pgfpathlineto{\pgfqpoint{11.016819in}{1.786018in}}%
\pgfpathlineto{\pgfqpoint{11.017504in}{1.785701in}}%
\pgfpathlineto{\pgfqpoint{11.017846in}{1.785595in}}%
\pgfpathlineto{\pgfqpoint{11.018531in}{1.784291in}}%
\pgfpathlineto{\pgfqpoint{11.019216in}{1.784105in}}%
\pgfpathlineto{\pgfqpoint{11.020928in}{1.782670in}}%
\pgfpathlineto{\pgfqpoint{11.021270in}{1.782514in}}%
\pgfpathlineto{\pgfqpoint{11.021955in}{1.780943in}}%
\pgfpathlineto{\pgfqpoint{11.022982in}{1.780037in}}%
\pgfpathlineto{\pgfqpoint{11.023666in}{1.779630in}}%
\pgfpathlineto{\pgfqpoint{11.024351in}{1.778753in}}%
\pgfpathlineto{\pgfqpoint{11.026405in}{1.777658in}}%
\pgfpathlineto{\pgfqpoint{11.027090in}{1.776633in}}%
\pgfpathlineto{\pgfqpoint{11.028117in}{1.775265in}}%
\pgfpathlineto{\pgfqpoint{11.028459in}{1.775174in}}%
\pgfpathlineto{\pgfqpoint{11.029144in}{1.774112in}}%
\pgfpathlineto{\pgfqpoint{11.029486in}{1.773029in}}%
\pgfpathlineto{\pgfqpoint{11.029829in}{1.772939in}}%
\pgfpathlineto{\pgfqpoint{11.030513in}{1.772002in}}%
\pgfpathlineto{\pgfqpoint{11.031883in}{1.771126in}}%
\pgfpathlineto{\pgfqpoint{11.032567in}{1.770703in}}%
\pgfpathlineto{\pgfqpoint{11.034279in}{1.770220in}}%
\pgfpathlineto{\pgfqpoint{11.035306in}{1.769004in}}%
\pgfpathlineto{\pgfqpoint{11.037018in}{1.768529in}}%
\pgfpathlineto{\pgfqpoint{11.037703in}{1.768079in}}%
\pgfpathlineto{\pgfqpoint{11.038045in}{1.766511in}}%
\pgfpathlineto{\pgfqpoint{11.038730in}{1.766293in}}%
\pgfpathlineto{\pgfqpoint{11.039414in}{1.766112in}}%
\pgfpathlineto{\pgfqpoint{11.040441in}{1.765177in}}%
\pgfpathlineto{\pgfqpoint{11.042838in}{1.762834in}}%
\pgfpathlineto{\pgfqpoint{11.043180in}{1.762820in}}%
\pgfpathlineto{\pgfqpoint{11.043865in}{1.762331in}}%
\pgfpathlineto{\pgfqpoint{11.044207in}{1.762246in}}%
\pgfpathlineto{\pgfqpoint{11.046261in}{1.759818in}}%
\pgfpathlineto{\pgfqpoint{11.046604in}{1.759746in}}%
\pgfpathlineto{\pgfqpoint{11.047973in}{1.756174in}}%
\pgfpathlineto{\pgfqpoint{11.048658in}{1.755985in}}%
\pgfpathlineto{\pgfqpoint{11.049000in}{1.754109in}}%
\pgfpathlineto{\pgfqpoint{11.050027in}{1.753525in}}%
\pgfpathlineto{\pgfqpoint{11.050369in}{1.753493in}}%
\pgfpathlineto{\pgfqpoint{11.051054in}{1.752532in}}%
\pgfpathlineto{\pgfqpoint{11.052081in}{1.751530in}}%
\pgfpathlineto{\pgfqpoint{11.053108in}{1.750538in}}%
\pgfpathlineto{\pgfqpoint{11.053451in}{1.749537in}}%
\pgfpathlineto{\pgfqpoint{11.053793in}{1.749529in}}%
\pgfpathlineto{\pgfqpoint{11.054135in}{1.748049in}}%
\pgfpathlineto{\pgfqpoint{11.054478in}{1.747837in}}%
\pgfpathlineto{\pgfqpoint{11.055847in}{1.744968in}}%
\pgfpathlineto{\pgfqpoint{11.056874in}{1.744394in}}%
\pgfpathlineto{\pgfqpoint{11.058243in}{1.743978in}}%
\pgfpathlineto{\pgfqpoint{11.058928in}{1.743155in}}%
\pgfpathlineto{\pgfqpoint{11.060297in}{1.741524in}}%
\pgfpathlineto{\pgfqpoint{11.061667in}{1.740753in}}%
\pgfpathlineto{\pgfqpoint{11.062009in}{1.740694in}}%
\pgfpathlineto{\pgfqpoint{11.063721in}{1.738919in}}%
\pgfpathlineto{\pgfqpoint{11.064406in}{1.738775in}}%
\pgfpathlineto{\pgfqpoint{11.065775in}{1.736200in}}%
\pgfpathlineto{\pgfqpoint{11.066802in}{1.735370in}}%
\pgfpathlineto{\pgfqpoint{11.067487in}{1.734426in}}%
\pgfpathlineto{\pgfqpoint{11.068514in}{1.734086in}}%
\pgfpathlineto{\pgfqpoint{11.069198in}{1.733376in}}%
\pgfpathlineto{\pgfqpoint{11.070568in}{1.732553in}}%
\pgfpathlineto{\pgfqpoint{11.071253in}{1.731375in}}%
\pgfpathlineto{\pgfqpoint{11.071937in}{1.731330in}}%
\pgfpathlineto{\pgfqpoint{11.074334in}{1.728868in}}%
\pgfpathlineto{\pgfqpoint{11.074676in}{1.728838in}}%
\pgfpathlineto{\pgfqpoint{11.075361in}{1.727905in}}%
\pgfpathlineto{\pgfqpoint{11.075703in}{1.727888in}}%
\pgfpathlineto{\pgfqpoint{11.076388in}{1.726648in}}%
\pgfpathlineto{\pgfqpoint{11.077757in}{1.726119in}}%
\pgfpathlineto{\pgfqpoint{11.078442in}{1.725696in}}%
\pgfpathlineto{\pgfqpoint{11.078784in}{1.724276in}}%
\pgfpathlineto{\pgfqpoint{11.080154in}{1.723521in}}%
\pgfpathlineto{\pgfqpoint{11.081181in}{1.721920in}}%
\pgfpathlineto{\pgfqpoint{11.081865in}{1.720771in}}%
\pgfpathlineto{\pgfqpoint{11.082892in}{1.720470in}}%
\pgfpathlineto{\pgfqpoint{11.085289in}{1.718054in}}%
\pgfpathlineto{\pgfqpoint{11.086316in}{1.717624in}}%
\pgfpathlineto{\pgfqpoint{11.086658in}{1.717397in}}%
\pgfpathlineto{\pgfqpoint{11.087343in}{1.716534in}}%
\pgfpathlineto{\pgfqpoint{11.087685in}{1.716491in}}%
\pgfpathlineto{\pgfqpoint{11.088370in}{1.715728in}}%
\pgfpathlineto{\pgfqpoint{11.088712in}{1.715728in}}%
\pgfpathlineto{\pgfqpoint{11.089397in}{1.714867in}}%
\pgfpathlineto{\pgfqpoint{11.090082in}{1.714301in}}%
\pgfpathlineto{\pgfqpoint{11.091793in}{1.713704in}}%
\pgfpathlineto{\pgfqpoint{11.092820in}{1.712710in}}%
\pgfpathlineto{\pgfqpoint{11.093163in}{1.712677in}}%
\pgfpathlineto{\pgfqpoint{11.093505in}{1.712164in}}%
\pgfpathlineto{\pgfqpoint{11.094190in}{1.710533in}}%
\pgfpathlineto{\pgfqpoint{11.096244in}{1.709052in}}%
\pgfpathlineto{\pgfqpoint{11.096586in}{1.706787in}}%
\pgfpathlineto{\pgfqpoint{11.101379in}{1.703313in}}%
\pgfpathlineto{\pgfqpoint{11.103091in}{1.702868in}}%
\pgfpathlineto{\pgfqpoint{11.104118in}{1.701596in}}%
\pgfpathlineto{\pgfqpoint{11.104803in}{1.701230in}}%
\pgfpathlineto{\pgfqpoint{11.106514in}{1.700504in}}%
\pgfpathlineto{\pgfqpoint{11.106857in}{1.699511in}}%
\pgfpathlineto{\pgfqpoint{11.107199in}{1.699477in}}%
\pgfpathlineto{\pgfqpoint{11.108226in}{1.698192in}}%
\pgfpathlineto{\pgfqpoint{11.110965in}{1.696034in}}%
\pgfpathlineto{\pgfqpoint{11.111649in}{1.695875in}}%
\pgfpathlineto{\pgfqpoint{11.112334in}{1.695158in}}%
\pgfpathlineto{\pgfqpoint{11.114046in}{1.693708in}}%
\pgfpathlineto{\pgfqpoint{11.114388in}{1.692801in}}%
\pgfpathlineto{\pgfqpoint{11.115758in}{1.692560in}}%
\pgfpathlineto{\pgfqpoint{11.118496in}{1.691382in}}%
\pgfpathlineto{\pgfqpoint{11.119523in}{1.690355in}}%
\pgfpathlineto{\pgfqpoint{11.119866in}{1.690325in}}%
\pgfpathlineto{\pgfqpoint{11.120550in}{1.689834in}}%
\pgfpathlineto{\pgfqpoint{11.121578in}{1.689449in}}%
\pgfpathlineto{\pgfqpoint{11.121920in}{1.688814in}}%
\pgfpathlineto{\pgfqpoint{11.122605in}{1.686964in}}%
\pgfpathlineto{\pgfqpoint{11.123632in}{1.686209in}}%
\pgfpathlineto{\pgfqpoint{11.124659in}{1.685718in}}%
\pgfpathlineto{\pgfqpoint{11.125343in}{1.685114in}}%
\pgfpathlineto{\pgfqpoint{11.126028in}{1.683135in}}%
\pgfpathlineto{\pgfqpoint{11.126370in}{1.683037in}}%
\pgfpathlineto{\pgfqpoint{11.127055in}{1.681353in}}%
\pgfpathlineto{\pgfqpoint{11.127740in}{1.680958in}}%
\pgfpathlineto{\pgfqpoint{11.129794in}{1.680357in}}%
\pgfpathlineto{\pgfqpoint{11.130821in}{1.679111in}}%
\pgfpathlineto{\pgfqpoint{11.131506in}{1.677819in}}%
\pgfpathlineto{\pgfqpoint{11.133217in}{1.677366in}}%
\pgfpathlineto{\pgfqpoint{11.133902in}{1.677185in}}%
\pgfpathlineto{\pgfqpoint{11.134587in}{1.676316in}}%
\pgfpathlineto{\pgfqpoint{11.134929in}{1.676279in}}%
\pgfpathlineto{\pgfqpoint{11.135614in}{1.675739in}}%
\pgfpathlineto{\pgfqpoint{11.135956in}{1.674345in}}%
\pgfpathlineto{\pgfqpoint{11.137325in}{1.673530in}}%
\pgfpathlineto{\pgfqpoint{11.138695in}{1.670237in}}%
\pgfpathlineto{\pgfqpoint{11.139722in}{1.669815in}}%
\pgfpathlineto{\pgfqpoint{11.140064in}{1.669815in}}%
\pgfpathlineto{\pgfqpoint{11.140749in}{1.668917in}}%
\pgfpathlineto{\pgfqpoint{11.141776in}{1.667428in}}%
\pgfpathlineto{\pgfqpoint{11.142118in}{1.667406in}}%
\pgfpathlineto{\pgfqpoint{11.142461in}{1.666839in}}%
\pgfpathlineto{\pgfqpoint{11.143145in}{1.664740in}}%
\pgfpathlineto{\pgfqpoint{11.143488in}{1.664528in}}%
\pgfpathlineto{\pgfqpoint{11.144172in}{1.663411in}}%
\pgfpathlineto{\pgfqpoint{11.144515in}{1.663328in}}%
\pgfpathlineto{\pgfqpoint{11.145884in}{1.661206in}}%
\pgfpathlineto{\pgfqpoint{11.146226in}{1.661145in}}%
\pgfpathlineto{\pgfqpoint{11.146911in}{1.660605in}}%
\pgfpathlineto{\pgfqpoint{11.147938in}{1.659846in}}%
\pgfpathlineto{\pgfqpoint{11.148623in}{1.659025in}}%
\pgfpathlineto{\pgfqpoint{11.149308in}{1.658850in}}%
\pgfpathlineto{\pgfqpoint{11.149650in}{1.658155in}}%
\pgfpathlineto{\pgfqpoint{11.150677in}{1.657732in}}%
\pgfpathlineto{\pgfqpoint{11.151704in}{1.657188in}}%
\pgfpathlineto{\pgfqpoint{11.152046in}{1.657067in}}%
\pgfpathlineto{\pgfqpoint{11.152731in}{1.655829in}}%
\pgfpathlineto{\pgfqpoint{11.153758in}{1.655164in}}%
\pgfpathlineto{\pgfqpoint{11.154785in}{1.654618in}}%
\pgfpathlineto{\pgfqpoint{11.155127in}{1.654473in}}%
\pgfpathlineto{\pgfqpoint{11.155470in}{1.653715in}}%
\pgfpathlineto{\pgfqpoint{11.156155in}{1.653503in}}%
\pgfpathlineto{\pgfqpoint{11.157182in}{1.652302in}}%
\pgfpathlineto{\pgfqpoint{11.157866in}{1.651993in}}%
\pgfpathlineto{\pgfqpoint{11.158551in}{1.650724in}}%
\pgfpathlineto{\pgfqpoint{11.158893in}{1.650483in}}%
\pgfpathlineto{\pgfqpoint{11.159578in}{1.648376in}}%
\pgfpathlineto{\pgfqpoint{11.160263in}{1.648111in}}%
\pgfpathlineto{\pgfqpoint{11.160947in}{1.647704in}}%
\pgfpathlineto{\pgfqpoint{11.161290in}{1.647696in}}%
\pgfpathlineto{\pgfqpoint{11.161974in}{1.647016in}}%
\pgfpathlineto{\pgfqpoint{11.162317in}{1.645015in}}%
\pgfpathlineto{\pgfqpoint{11.163344in}{1.644441in}}%
\pgfpathlineto{\pgfqpoint{11.164028in}{1.644109in}}%
\pgfpathlineto{\pgfqpoint{11.165740in}{1.641662in}}%
\pgfpathlineto{\pgfqpoint{11.166767in}{1.639314in}}%
\pgfpathlineto{\pgfqpoint{11.167452in}{1.638861in}}%
\pgfpathlineto{\pgfqpoint{11.167794in}{1.638823in}}%
\pgfpathlineto{\pgfqpoint{11.168479in}{1.636497in}}%
\pgfpathlineto{\pgfqpoint{11.169848in}{1.636029in}}%
\pgfpathlineto{\pgfqpoint{11.172930in}{1.634005in}}%
\pgfpathlineto{\pgfqpoint{11.173614in}{1.632510in}}%
\pgfpathlineto{\pgfqpoint{11.173957in}{1.632291in}}%
\pgfpathlineto{\pgfqpoint{11.174299in}{1.631422in}}%
\pgfpathlineto{\pgfqpoint{11.174641in}{1.631385in}}%
\pgfpathlineto{\pgfqpoint{11.175326in}{1.629534in}}%
\pgfpathlineto{\pgfqpoint{11.175668in}{1.628364in}}%
\pgfpathlineto{\pgfqpoint{11.176011in}{1.628276in}}%
\pgfpathlineto{\pgfqpoint{11.176353in}{1.627888in}}%
\pgfpathlineto{\pgfqpoint{11.177038in}{1.626854in}}%
\pgfpathlineto{\pgfqpoint{11.177722in}{1.626197in}}%
\pgfpathlineto{\pgfqpoint{11.178065in}{1.626190in}}%
\pgfpathlineto{\pgfqpoint{11.178749in}{1.625532in}}%
\pgfpathlineto{\pgfqpoint{11.179092in}{1.625260in}}%
\pgfpathlineto{\pgfqpoint{11.179434in}{1.624566in}}%
\pgfpathlineto{\pgfqpoint{11.181146in}{1.624082in}}%
\pgfpathlineto{\pgfqpoint{11.181488in}{1.623153in}}%
\pgfpathlineto{\pgfqpoint{11.181831in}{1.623085in}}%
\pgfpathlineto{\pgfqpoint{11.182515in}{1.621484in}}%
\pgfpathlineto{\pgfqpoint{11.182858in}{1.619612in}}%
\pgfpathlineto{\pgfqpoint{11.185254in}{1.617709in}}%
\pgfpathlineto{\pgfqpoint{11.185939in}{1.616500in}}%
\pgfpathlineto{\pgfqpoint{11.186966in}{1.615685in}}%
\pgfpathlineto{\pgfqpoint{11.187650in}{1.614092in}}%
\pgfpathlineto{\pgfqpoint{11.188677in}{1.613661in}}%
\pgfpathlineto{\pgfqpoint{11.189704in}{1.612151in}}%
\pgfpathlineto{\pgfqpoint{11.190389in}{1.611949in}}%
\pgfpathlineto{\pgfqpoint{11.191074in}{1.611396in}}%
\pgfpathlineto{\pgfqpoint{11.191416in}{1.611335in}}%
\pgfpathlineto{\pgfqpoint{11.192101in}{1.610852in}}%
\pgfpathlineto{\pgfqpoint{11.192443in}{1.610769in}}%
\pgfpathlineto{\pgfqpoint{11.193128in}{1.610157in}}%
\pgfpathlineto{\pgfqpoint{11.193813in}{1.609987in}}%
\pgfpathlineto{\pgfqpoint{11.194155in}{1.608435in}}%
\pgfpathlineto{\pgfqpoint{11.194497in}{1.608345in}}%
\pgfpathlineto{\pgfqpoint{11.194840in}{1.606407in}}%
\pgfpathlineto{\pgfqpoint{11.196209in}{1.605656in}}%
\pgfpathlineto{\pgfqpoint{11.196894in}{1.605214in}}%
\pgfpathlineto{\pgfqpoint{11.197236in}{1.605113in}}%
\pgfpathlineto{\pgfqpoint{11.197578in}{1.604297in}}%
\pgfpathlineto{\pgfqpoint{11.198263in}{1.604207in}}%
\pgfpathlineto{\pgfqpoint{11.198606in}{1.603935in}}%
\pgfpathlineto{\pgfqpoint{11.199290in}{1.602485in}}%
\pgfpathlineto{\pgfqpoint{11.200317in}{1.601095in}}%
\pgfpathlineto{\pgfqpoint{11.202371in}{1.600491in}}%
\pgfpathlineto{\pgfqpoint{11.203398in}{1.599464in}}%
\pgfpathlineto{\pgfqpoint{11.203741in}{1.598120in}}%
\pgfpathlineto{\pgfqpoint{11.204425in}{1.597907in}}%
\pgfpathlineto{\pgfqpoint{11.205110in}{1.597410in}}%
\pgfpathlineto{\pgfqpoint{11.205452in}{1.597385in}}%
\pgfpathlineto{\pgfqpoint{11.205795in}{1.596157in}}%
\pgfpathlineto{\pgfqpoint{11.206822in}{1.595779in}}%
\pgfpathlineto{\pgfqpoint{11.207164in}{1.595672in}}%
\pgfpathlineto{\pgfqpoint{11.208191in}{1.594669in}}%
\pgfpathlineto{\pgfqpoint{11.208876in}{1.594118in}}%
\pgfpathlineto{\pgfqpoint{11.209561in}{1.592698in}}%
\pgfpathlineto{\pgfqpoint{11.210588in}{1.592094in}}%
\pgfpathlineto{\pgfqpoint{11.211272in}{1.591475in}}%
\pgfpathlineto{\pgfqpoint{11.211615in}{1.591297in}}%
\pgfpathlineto{\pgfqpoint{11.211957in}{1.590327in}}%
\pgfpathlineto{\pgfqpoint{11.212299in}{1.590229in}}%
\pgfpathlineto{\pgfqpoint{11.213669in}{1.588718in}}%
\pgfpathlineto{\pgfqpoint{11.215723in}{1.587291in}}%
\pgfpathlineto{\pgfqpoint{11.216065in}{1.586929in}}%
\pgfpathlineto{\pgfqpoint{11.217092in}{1.583348in}}%
\pgfpathlineto{\pgfqpoint{11.219831in}{1.582035in}}%
\pgfpathlineto{\pgfqpoint{11.220858in}{1.581008in}}%
\pgfpathlineto{\pgfqpoint{11.221543in}{1.580714in}}%
\pgfpathlineto{\pgfqpoint{11.221885in}{1.580260in}}%
\pgfpathlineto{\pgfqpoint{11.222227in}{1.579322in}}%
\pgfpathlineto{\pgfqpoint{11.223254in}{1.579045in}}%
\pgfpathlineto{\pgfqpoint{11.223939in}{1.578335in}}%
\pgfpathlineto{\pgfqpoint{11.225651in}{1.577867in}}%
\pgfpathlineto{\pgfqpoint{11.226336in}{1.576400in}}%
\pgfpathlineto{\pgfqpoint{11.227705in}{1.574755in}}%
\pgfpathlineto{\pgfqpoint{11.229759in}{1.573911in}}%
\pgfpathlineto{\pgfqpoint{11.230444in}{1.572792in}}%
\pgfpathlineto{\pgfqpoint{11.231128in}{1.572399in}}%
\pgfpathlineto{\pgfqpoint{11.231471in}{1.572097in}}%
\pgfpathlineto{\pgfqpoint{11.232155in}{1.571048in}}%
\pgfpathlineto{\pgfqpoint{11.233183in}{1.569943in}}%
\pgfpathlineto{\pgfqpoint{11.234552in}{1.569084in}}%
\pgfpathlineto{\pgfqpoint{11.234894in}{1.568926in}}%
\pgfpathlineto{\pgfqpoint{11.235579in}{1.567234in}}%
\pgfpathlineto{\pgfqpoint{11.236264in}{1.566418in}}%
\pgfpathlineto{\pgfqpoint{11.239002in}{1.563639in}}%
\pgfpathlineto{\pgfqpoint{11.239687in}{1.563420in}}%
\pgfpathlineto{\pgfqpoint{11.240029in}{1.562107in}}%
\pgfpathlineto{\pgfqpoint{11.240372in}{1.561948in}}%
\pgfpathlineto{\pgfqpoint{11.241741in}{1.559834in}}%
\pgfpathlineto{\pgfqpoint{11.242426in}{1.559531in}}%
\pgfpathlineto{\pgfqpoint{11.242768in}{1.557115in}}%
\pgfpathlineto{\pgfqpoint{11.243453in}{1.556511in}}%
\pgfpathlineto{\pgfqpoint{11.244480in}{1.554736in}}%
\pgfpathlineto{\pgfqpoint{11.245165in}{1.554427in}}%
\pgfpathlineto{\pgfqpoint{11.245507in}{1.554245in}}%
\pgfpathlineto{\pgfqpoint{11.246192in}{1.553037in}}%
\pgfpathlineto{\pgfqpoint{11.247219in}{1.552191in}}%
\pgfpathlineto{\pgfqpoint{11.248246in}{1.551716in}}%
\pgfpathlineto{\pgfqpoint{11.248930in}{1.551236in}}%
\pgfpathlineto{\pgfqpoint{11.249273in}{1.551149in}}%
\pgfpathlineto{\pgfqpoint{11.249958in}{1.550681in}}%
\pgfpathlineto{\pgfqpoint{11.250300in}{1.550503in}}%
\pgfpathlineto{\pgfqpoint{11.250985in}{1.548446in}}%
\pgfpathlineto{\pgfqpoint{11.251669in}{1.547611in}}%
\pgfpathlineto{\pgfqpoint{11.252696in}{1.547185in}}%
\pgfpathlineto{\pgfqpoint{11.253039in}{1.546513in}}%
\pgfpathlineto{\pgfqpoint{11.253723in}{1.546301in}}%
\pgfpathlineto{\pgfqpoint{11.254066in}{1.546150in}}%
\pgfpathlineto{\pgfqpoint{11.255093in}{1.543764in}}%
\pgfpathlineto{\pgfqpoint{11.255777in}{1.543613in}}%
\pgfpathlineto{\pgfqpoint{11.256804in}{1.542525in}}%
\pgfpathlineto{\pgfqpoint{11.257147in}{1.542495in}}%
\pgfpathlineto{\pgfqpoint{11.258174in}{1.541136in}}%
\pgfpathlineto{\pgfqpoint{11.258859in}{1.540683in}}%
\pgfpathlineto{\pgfqpoint{11.259201in}{1.540350in}}%
\pgfpathlineto{\pgfqpoint{11.259886in}{1.538296in}}%
\pgfpathlineto{\pgfqpoint{11.260913in}{1.536242in}}%
\pgfpathlineto{\pgfqpoint{11.261255in}{1.536046in}}%
\pgfpathlineto{\pgfqpoint{11.262282in}{1.534762in}}%
\pgfpathlineto{\pgfqpoint{11.262967in}{1.534521in}}%
\pgfpathlineto{\pgfqpoint{11.263651in}{1.532859in}}%
\pgfpathlineto{\pgfqpoint{11.264336in}{1.532346in}}%
\pgfpathlineto{\pgfqpoint{11.265021in}{1.531051in}}%
\pgfpathlineto{\pgfqpoint{11.265363in}{1.530979in}}%
\pgfpathlineto{\pgfqpoint{11.266048in}{1.530609in}}%
\pgfpathlineto{\pgfqpoint{11.267417in}{1.529552in}}%
\pgfpathlineto{\pgfqpoint{11.267760in}{1.529537in}}%
\pgfpathlineto{\pgfqpoint{11.268787in}{1.527211in}}%
\pgfpathlineto{\pgfqpoint{11.269471in}{1.527135in}}%
\pgfpathlineto{\pgfqpoint{11.270498in}{1.526365in}}%
\pgfpathlineto{\pgfqpoint{11.270841in}{1.525172in}}%
\pgfpathlineto{\pgfqpoint{11.271183in}{1.525096in}}%
\pgfpathlineto{\pgfqpoint{11.272210in}{1.523858in}}%
\pgfpathlineto{\pgfqpoint{11.273579in}{1.523375in}}%
\pgfpathlineto{\pgfqpoint{11.273922in}{1.523284in}}%
\pgfpathlineto{\pgfqpoint{11.274949in}{1.522287in}}%
\pgfpathlineto{\pgfqpoint{11.278030in}{1.520052in}}%
\pgfpathlineto{\pgfqpoint{11.278715in}{1.517061in}}%
\pgfpathlineto{\pgfqpoint{11.279057in}{1.517031in}}%
\pgfpathlineto{\pgfqpoint{11.280084in}{1.515978in}}%
\pgfpathlineto{\pgfqpoint{11.280769in}{1.515732in}}%
\pgfpathlineto{\pgfqpoint{11.281453in}{1.514524in}}%
\pgfpathlineto{\pgfqpoint{11.282138in}{1.514373in}}%
\pgfpathlineto{\pgfqpoint{11.282823in}{1.512833in}}%
\pgfpathlineto{\pgfqpoint{11.283507in}{1.512138in}}%
\pgfpathlineto{\pgfqpoint{11.284877in}{1.510688in}}%
\pgfpathlineto{\pgfqpoint{11.285219in}{1.510635in}}%
\pgfpathlineto{\pgfqpoint{11.286931in}{1.509203in}}%
\pgfpathlineto{\pgfqpoint{11.287273in}{1.509162in}}%
\pgfpathlineto{\pgfqpoint{11.287616in}{1.508332in}}%
\pgfpathlineto{\pgfqpoint{11.287958in}{1.508241in}}%
\pgfpathlineto{\pgfqpoint{11.289327in}{1.506217in}}%
\pgfpathlineto{\pgfqpoint{11.289670in}{1.506127in}}%
\pgfpathlineto{\pgfqpoint{11.291039in}{1.504669in}}%
\pgfpathlineto{\pgfqpoint{11.292066in}{1.504043in}}%
\pgfpathlineto{\pgfqpoint{11.292751in}{1.503839in}}%
\pgfpathlineto{\pgfqpoint{11.293436in}{1.503272in}}%
\pgfpathlineto{\pgfqpoint{11.293778in}{1.502895in}}%
\pgfpathlineto{\pgfqpoint{11.294120in}{1.501535in}}%
\pgfpathlineto{\pgfqpoint{11.294805in}{1.501445in}}%
\pgfpathlineto{\pgfqpoint{11.296174in}{1.499111in}}%
\pgfpathlineto{\pgfqpoint{11.296517in}{1.499089in}}%
\pgfpathlineto{\pgfqpoint{11.297886in}{1.497138in}}%
\pgfpathlineto{\pgfqpoint{11.298571in}{1.496891in}}%
\pgfpathlineto{\pgfqpoint{11.299255in}{1.496733in}}%
\pgfpathlineto{\pgfqpoint{11.300282in}{1.495494in}}%
\pgfpathlineto{\pgfqpoint{11.301310in}{1.494849in}}%
\pgfpathlineto{\pgfqpoint{11.301994in}{1.493493in}}%
\pgfpathlineto{\pgfqpoint{11.302679in}{1.493108in}}%
\pgfpathlineto{\pgfqpoint{11.303364in}{1.492323in}}%
\pgfpathlineto{\pgfqpoint{11.304391in}{1.491507in}}%
\pgfpathlineto{\pgfqpoint{11.305760in}{1.489634in}}%
\pgfpathlineto{\pgfqpoint{11.308156in}{1.488399in}}%
\pgfpathlineto{\pgfqpoint{11.308499in}{1.487731in}}%
\pgfpathlineto{\pgfqpoint{11.309526in}{1.487405in}}%
\pgfpathlineto{\pgfqpoint{11.310211in}{1.486100in}}%
\pgfpathlineto{\pgfqpoint{11.310553in}{1.483865in}}%
\pgfpathlineto{\pgfqpoint{11.310895in}{1.483683in}}%
\pgfpathlineto{\pgfqpoint{11.311238in}{1.482596in}}%
\pgfpathlineto{\pgfqpoint{11.311922in}{1.482505in}}%
\pgfpathlineto{\pgfqpoint{11.312265in}{1.481932in}}%
\pgfpathlineto{\pgfqpoint{11.312607in}{1.480633in}}%
\pgfpathlineto{\pgfqpoint{11.313634in}{1.480315in}}%
\pgfpathlineto{\pgfqpoint{11.314319in}{1.479243in}}%
\pgfpathlineto{\pgfqpoint{11.316030in}{1.478029in}}%
\pgfpathlineto{\pgfqpoint{11.317057in}{1.476525in}}%
\pgfpathlineto{\pgfqpoint{11.318085in}{1.475709in}}%
\pgfpathlineto{\pgfqpoint{11.318769in}{1.475158in}}%
\pgfpathlineto{\pgfqpoint{11.319454in}{1.473987in}}%
\pgfpathlineto{\pgfqpoint{11.320823in}{1.471631in}}%
\pgfpathlineto{\pgfqpoint{11.322193in}{1.470688in}}%
\pgfpathlineto{\pgfqpoint{11.324589in}{1.468082in}}%
\pgfpathlineto{\pgfqpoint{11.324931in}{1.468054in}}%
\pgfpathlineto{\pgfqpoint{11.325958in}{1.466224in}}%
\pgfpathlineto{\pgfqpoint{11.327328in}{1.464175in}}%
\pgfpathlineto{\pgfqpoint{11.328013in}{1.463947in}}%
\pgfpathlineto{\pgfqpoint{11.328355in}{1.463204in}}%
\pgfpathlineto{\pgfqpoint{11.328697in}{1.463143in}}%
\pgfpathlineto{\pgfqpoint{11.329382in}{1.461875in}}%
\pgfpathlineto{\pgfqpoint{11.330409in}{1.460787in}}%
\pgfpathlineto{\pgfqpoint{11.331094in}{1.460334in}}%
\pgfpathlineto{\pgfqpoint{11.331778in}{1.460183in}}%
\pgfpathlineto{\pgfqpoint{11.332805in}{1.459126in}}%
\pgfpathlineto{\pgfqpoint{11.333490in}{1.457827in}}%
\pgfpathlineto{\pgfqpoint{11.335544in}{1.456981in}}%
\pgfpathlineto{\pgfqpoint{11.336229in}{1.454527in}}%
\pgfpathlineto{\pgfqpoint{11.337598in}{1.452299in}}%
\pgfpathlineto{\pgfqpoint{11.338283in}{1.451997in}}%
\pgfpathlineto{\pgfqpoint{11.339652in}{1.449248in}}%
\pgfpathlineto{\pgfqpoint{11.339995in}{1.449188in}}%
\pgfpathlineto{\pgfqpoint{11.340679in}{1.448184in}}%
\pgfpathlineto{\pgfqpoint{11.342049in}{1.447949in}}%
\pgfpathlineto{\pgfqpoint{11.342391in}{1.445926in}}%
\pgfpathlineto{\pgfqpoint{11.343076in}{1.445654in}}%
\pgfpathlineto{\pgfqpoint{11.343418in}{1.444838in}}%
\pgfpathlineto{\pgfqpoint{11.343761in}{1.444748in}}%
\pgfpathlineto{\pgfqpoint{11.345130in}{1.442663in}}%
\pgfpathlineto{\pgfqpoint{11.346157in}{1.442029in}}%
\pgfpathlineto{\pgfqpoint{11.348211in}{1.438818in}}%
\pgfpathlineto{\pgfqpoint{11.348896in}{1.438442in}}%
\pgfpathlineto{\pgfqpoint{11.349580in}{1.437518in}}%
\pgfpathlineto{\pgfqpoint{11.350265in}{1.433541in}}%
\pgfpathlineto{\pgfqpoint{11.350950in}{1.433183in}}%
\pgfpathlineto{\pgfqpoint{11.351977in}{1.431155in}}%
\pgfpathlineto{\pgfqpoint{11.352319in}{1.431085in}}%
\pgfpathlineto{\pgfqpoint{11.353004in}{1.429949in}}%
\pgfpathlineto{\pgfqpoint{11.354716in}{1.428557in}}%
\pgfpathlineto{\pgfqpoint{11.355058in}{1.428497in}}%
\pgfpathlineto{\pgfqpoint{11.356427in}{1.426322in}}%
\pgfpathlineto{\pgfqpoint{11.357112in}{1.425038in}}%
\pgfpathlineto{\pgfqpoint{11.357797in}{1.424963in}}%
\pgfpathlineto{\pgfqpoint{11.358824in}{1.423543in}}%
\pgfpathlineto{\pgfqpoint{11.359166in}{1.421247in}}%
\pgfpathlineto{\pgfqpoint{11.362932in}{1.417109in}}%
\pgfpathlineto{\pgfqpoint{11.363617in}{1.416886in}}%
\pgfpathlineto{\pgfqpoint{11.363959in}{1.415689in}}%
\pgfpathlineto{\pgfqpoint{11.364301in}{1.415599in}}%
\pgfpathlineto{\pgfqpoint{11.364986in}{1.413605in}}%
\pgfpathlineto{\pgfqpoint{11.367040in}{1.412215in}}%
\pgfpathlineto{\pgfqpoint{11.368067in}{1.410350in}}%
\pgfpathlineto{\pgfqpoint{11.368409in}{1.408538in}}%
\pgfpathlineto{\pgfqpoint{11.370806in}{1.406990in}}%
\pgfpathlineto{\pgfqpoint{11.371491in}{1.406174in}}%
\pgfpathlineto{\pgfqpoint{11.375599in}{1.403486in}}%
\pgfpathlineto{\pgfqpoint{11.377310in}{1.399408in}}%
\pgfpathlineto{\pgfqpoint{11.380392in}{1.391056in}}%
\pgfpathlineto{\pgfqpoint{11.381761in}{1.390618in}}%
\pgfpathlineto{\pgfqpoint{11.383130in}{1.388337in}}%
\pgfpathlineto{\pgfqpoint{11.384157in}{1.387406in}}%
\pgfpathlineto{\pgfqpoint{11.384842in}{1.386525in}}%
\pgfpathlineto{\pgfqpoint{11.385527in}{1.385423in}}%
\pgfpathlineto{\pgfqpoint{11.387581in}{1.384667in}}%
\pgfpathlineto{\pgfqpoint{11.387923in}{1.383504in}}%
\pgfpathlineto{\pgfqpoint{11.389977in}{1.382674in}}%
\pgfpathlineto{\pgfqpoint{11.390662in}{1.381345in}}%
\pgfpathlineto{\pgfqpoint{11.391004in}{1.381224in}}%
\pgfpathlineto{\pgfqpoint{11.391689in}{1.379599in}}%
\pgfpathlineto{\pgfqpoint{11.394428in}{1.377539in}}%
\pgfpathlineto{\pgfqpoint{11.395113in}{1.376904in}}%
\pgfpathlineto{\pgfqpoint{11.395797in}{1.376240in}}%
\pgfpathlineto{\pgfqpoint{11.396140in}{1.376089in}}%
\pgfpathlineto{\pgfqpoint{11.396482in}{1.373733in}}%
\pgfpathlineto{\pgfqpoint{11.397851in}{1.373249in}}%
\pgfpathlineto{\pgfqpoint{11.398194in}{1.371951in}}%
\pgfpathlineto{\pgfqpoint{11.398878in}{1.371618in}}%
\pgfpathlineto{\pgfqpoint{11.399221in}{1.371271in}}%
\pgfpathlineto{\pgfqpoint{11.400932in}{1.367933in}}%
\pgfpathlineto{\pgfqpoint{11.401275in}{1.367873in}}%
\pgfpathlineto{\pgfqpoint{11.402302in}{1.366483in}}%
\pgfpathlineto{\pgfqpoint{11.402644in}{1.364248in}}%
\pgfpathlineto{\pgfqpoint{11.403329in}{1.363583in}}%
\pgfpathlineto{\pgfqpoint{11.403671in}{1.362103in}}%
\pgfpathlineto{\pgfqpoint{11.404014in}{1.362013in}}%
\pgfpathlineto{\pgfqpoint{11.404698in}{1.359975in}}%
\pgfpathlineto{\pgfqpoint{11.406068in}{1.359475in}}%
\pgfpathlineto{\pgfqpoint{11.407095in}{1.358962in}}%
\pgfpathlineto{\pgfqpoint{11.408122in}{1.358841in}}%
\pgfpathlineto{\pgfqpoint{11.409149in}{1.357935in}}%
\pgfpathlineto{\pgfqpoint{11.410860in}{1.356606in}}%
\pgfpathlineto{\pgfqpoint{11.411203in}{1.356424in}}%
\pgfpathlineto{\pgfqpoint{11.411888in}{1.355186in}}%
\pgfpathlineto{\pgfqpoint{11.413942in}{1.352279in}}%
\pgfpathlineto{\pgfqpoint{11.414284in}{1.352150in}}%
\pgfpathlineto{\pgfqpoint{11.414626in}{1.351508in}}%
\pgfpathlineto{\pgfqpoint{11.414969in}{1.351501in}}%
\pgfpathlineto{\pgfqpoint{11.415311in}{1.350274in}}%
\pgfpathlineto{\pgfqpoint{11.417023in}{1.349809in}}%
\pgfpathlineto{\pgfqpoint{11.417365in}{1.349447in}}%
\pgfpathlineto{\pgfqpoint{11.418392in}{1.346487in}}%
\pgfpathlineto{\pgfqpoint{11.418734in}{1.346482in}}%
\pgfpathlineto{\pgfqpoint{11.419419in}{1.345852in}}%
\pgfpathlineto{\pgfqpoint{11.419761in}{1.345822in}}%
\pgfpathlineto{\pgfqpoint{11.420446in}{1.344387in}}%
\pgfpathlineto{\pgfqpoint{11.421131in}{1.344100in}}%
\pgfpathlineto{\pgfqpoint{11.422500in}{1.342558in}}%
\pgfpathlineto{\pgfqpoint{11.423185in}{1.341498in}}%
\pgfpathlineto{\pgfqpoint{11.423870in}{1.340627in}}%
\pgfpathlineto{\pgfqpoint{11.424212in}{1.338610in}}%
\pgfpathlineto{\pgfqpoint{11.425924in}{1.337455in}}%
\pgfpathlineto{\pgfqpoint{11.426608in}{1.336313in}}%
\pgfpathlineto{\pgfqpoint{11.427293in}{1.336005in}}%
\pgfpathlineto{\pgfqpoint{11.427978in}{1.335854in}}%
\pgfpathlineto{\pgfqpoint{11.429347in}{1.332645in}}%
\pgfpathlineto{\pgfqpoint{11.430032in}{1.330646in}}%
\pgfpathlineto{\pgfqpoint{11.431059in}{1.330417in}}%
\pgfpathlineto{\pgfqpoint{11.432086in}{1.327396in}}%
\pgfpathlineto{\pgfqpoint{11.432771in}{1.326943in}}%
\pgfpathlineto{\pgfqpoint{11.433455in}{1.326128in}}%
\pgfpathlineto{\pgfqpoint{11.434140in}{1.325675in}}%
\pgfpathlineto{\pgfqpoint{11.434482in}{1.325648in}}%
\pgfpathlineto{\pgfqpoint{11.435509in}{1.323772in}}%
\pgfpathlineto{\pgfqpoint{11.435852in}{1.323711in}}%
\pgfpathlineto{\pgfqpoint{11.436194in}{1.322714in}}%
\pgfpathlineto{\pgfqpoint{11.436879in}{1.322684in}}%
\pgfpathlineto{\pgfqpoint{11.442699in}{1.314287in}}%
\pgfpathlineto{\pgfqpoint{11.443726in}{1.313954in}}%
\pgfpathlineto{\pgfqpoint{11.444753in}{1.313622in}}%
\pgfpathlineto{\pgfqpoint{11.445437in}{1.310953in}}%
\pgfpathlineto{\pgfqpoint{11.446122in}{1.310420in}}%
\pgfpathlineto{\pgfqpoint{11.447149in}{1.308397in}}%
\pgfpathlineto{\pgfqpoint{11.447834in}{1.308034in}}%
\pgfpathlineto{\pgfqpoint{11.448176in}{1.306856in}}%
\pgfpathlineto{\pgfqpoint{11.448861in}{1.306509in}}%
\pgfpathlineto{\pgfqpoint{11.449546in}{1.305338in}}%
\pgfpathlineto{\pgfqpoint{11.450230in}{1.304288in}}%
\pgfpathlineto{\pgfqpoint{11.450915in}{1.304024in}}%
\pgfpathlineto{\pgfqpoint{11.451257in}{1.303995in}}%
\pgfpathlineto{\pgfqpoint{11.452284in}{1.302959in}}%
\pgfpathlineto{\pgfqpoint{11.452627in}{1.302899in}}%
\pgfpathlineto{\pgfqpoint{11.453311in}{1.302128in}}%
\pgfpathlineto{\pgfqpoint{11.453654in}{1.302081in}}%
\pgfpathlineto{\pgfqpoint{11.455366in}{1.298912in}}%
\pgfpathlineto{\pgfqpoint{11.458104in}{1.297551in}}%
\pgfpathlineto{\pgfqpoint{11.458789in}{1.296862in}}%
\pgfpathlineto{\pgfqpoint{11.460843in}{1.292523in}}%
\pgfpathlineto{\pgfqpoint{11.461185in}{1.292508in}}%
\pgfpathlineto{\pgfqpoint{11.462555in}{1.291270in}}%
\pgfpathlineto{\pgfqpoint{11.462897in}{1.291270in}}%
\pgfpathlineto{\pgfqpoint{11.463582in}{1.289336in}}%
\pgfpathlineto{\pgfqpoint{11.464267in}{1.287260in}}%
\pgfpathlineto{\pgfqpoint{11.464609in}{1.286950in}}%
\pgfpathlineto{\pgfqpoint{11.465294in}{1.285712in}}%
\pgfpathlineto{\pgfqpoint{11.465978in}{1.284994in}}%
\pgfpathlineto{\pgfqpoint{11.466663in}{1.284896in}}%
\pgfpathlineto{\pgfqpoint{11.467348in}{1.284292in}}%
\pgfpathlineto{\pgfqpoint{11.467690in}{1.284277in}}%
\pgfpathlineto{\pgfqpoint{11.468375in}{1.283204in}}%
\pgfpathlineto{\pgfqpoint{11.469059in}{1.282685in}}%
\pgfpathlineto{\pgfqpoint{11.470086in}{1.281301in}}%
\pgfpathlineto{\pgfqpoint{11.470429in}{1.279519in}}%
\pgfpathlineto{\pgfqpoint{11.471113in}{1.278825in}}%
\pgfpathlineto{\pgfqpoint{11.472825in}{1.277141in}}%
\pgfpathlineto{\pgfqpoint{11.473168in}{1.277073in}}%
\pgfpathlineto{\pgfqpoint{11.475222in}{1.273690in}}%
\pgfpathlineto{\pgfqpoint{11.475906in}{1.272663in}}%
\pgfpathlineto{\pgfqpoint{11.476933in}{1.271787in}}%
\pgfpathlineto{\pgfqpoint{11.477276in}{1.271726in}}%
\pgfpathlineto{\pgfqpoint{11.477618in}{1.271137in}}%
\pgfpathlineto{\pgfqpoint{11.477960in}{1.271122in}}%
\pgfpathlineto{\pgfqpoint{11.479330in}{1.269581in}}%
\pgfpathlineto{\pgfqpoint{11.480699in}{1.267467in}}%
\pgfpathlineto{\pgfqpoint{11.482069in}{1.266615in}}%
\pgfpathlineto{\pgfqpoint{11.482411in}{1.266564in}}%
\pgfpathlineto{\pgfqpoint{11.483438in}{1.264990in}}%
\pgfpathlineto{\pgfqpoint{11.484123in}{1.264327in}}%
\pgfpathlineto{\pgfqpoint{11.484807in}{1.263570in}}%
\pgfpathlineto{\pgfqpoint{11.485834in}{1.262574in}}%
\pgfpathlineto{\pgfqpoint{11.486177in}{1.262543in}}%
\pgfpathlineto{\pgfqpoint{11.486519in}{1.262188in}}%
\pgfpathlineto{\pgfqpoint{11.487546in}{1.260489in}}%
\pgfpathlineto{\pgfqpoint{11.487888in}{1.260306in}}%
\pgfpathlineto{\pgfqpoint{11.488573in}{1.259402in}}%
\pgfpathlineto{\pgfqpoint{11.489258in}{1.259130in}}%
\pgfpathlineto{\pgfqpoint{11.489943in}{1.258466in}}%
\pgfpathlineto{\pgfqpoint{11.490970in}{1.257112in}}%
\pgfpathlineto{\pgfqpoint{11.491312in}{1.255954in}}%
\pgfpathlineto{\pgfqpoint{11.492339in}{1.255675in}}%
\pgfpathlineto{\pgfqpoint{11.493024in}{1.253965in}}%
\pgfpathlineto{\pgfqpoint{11.493708in}{1.253572in}}%
\pgfpathlineto{\pgfqpoint{11.494735in}{1.250793in}}%
\pgfpathlineto{\pgfqpoint{11.495420in}{1.250432in}}%
\pgfpathlineto{\pgfqpoint{11.495762in}{1.249464in}}%
\pgfpathlineto{\pgfqpoint{11.496789in}{1.249200in}}%
\pgfpathlineto{\pgfqpoint{11.497474in}{1.247803in}}%
\pgfpathlineto{\pgfqpoint{11.498159in}{1.247709in}}%
\pgfpathlineto{\pgfqpoint{11.498844in}{1.247259in}}%
\pgfpathlineto{\pgfqpoint{11.499871in}{1.246897in}}%
\pgfpathlineto{\pgfqpoint{11.500555in}{1.245779in}}%
\pgfpathlineto{\pgfqpoint{11.501240in}{1.244824in}}%
\pgfpathlineto{\pgfqpoint{11.501582in}{1.244571in}}%
\pgfpathlineto{\pgfqpoint{11.502609in}{1.242246in}}%
\pgfpathlineto{\pgfqpoint{11.502952in}{1.242245in}}%
\pgfpathlineto{\pgfqpoint{11.503636in}{1.241369in}}%
\pgfpathlineto{\pgfqpoint{11.503979in}{1.239478in}}%
\pgfpathlineto{\pgfqpoint{11.504321in}{1.239458in}}%
\pgfpathlineto{\pgfqpoint{11.505348in}{1.237774in}}%
\pgfpathlineto{\pgfqpoint{11.506375in}{1.237114in}}%
\pgfpathlineto{\pgfqpoint{11.507060in}{1.235758in}}%
\pgfpathlineto{\pgfqpoint{11.507402in}{1.235720in}}%
\pgfpathlineto{\pgfqpoint{11.508429in}{1.234874in}}%
\pgfpathlineto{\pgfqpoint{11.508772in}{1.234703in}}%
\pgfpathlineto{\pgfqpoint{11.509114in}{1.233727in}}%
\pgfpathlineto{\pgfqpoint{11.509456in}{1.233727in}}%
\pgfpathlineto{\pgfqpoint{11.509799in}{1.232730in}}%
\pgfpathlineto{\pgfqpoint{11.510141in}{1.232725in}}%
\pgfpathlineto{\pgfqpoint{11.512195in}{1.230850in}}%
\pgfpathlineto{\pgfqpoint{11.514249in}{1.223547in}}%
\pgfpathlineto{\pgfqpoint{11.514592in}{1.223305in}}%
\pgfpathlineto{\pgfqpoint{11.514934in}{1.222118in}}%
\pgfpathlineto{\pgfqpoint{11.515619in}{1.221712in}}%
\pgfpathlineto{\pgfqpoint{11.516303in}{1.221251in}}%
\pgfpathlineto{\pgfqpoint{11.517330in}{1.220164in}}%
\pgfpathlineto{\pgfqpoint{11.517673in}{1.218895in}}%
\pgfpathlineto{\pgfqpoint{11.518015in}{1.218880in}}%
\pgfpathlineto{\pgfqpoint{11.518700in}{1.216338in}}%
\pgfpathlineto{\pgfqpoint{11.519042in}{1.216199in}}%
\pgfpathlineto{\pgfqpoint{11.519384in}{1.215108in}}%
\pgfpathlineto{\pgfqpoint{11.520411in}{1.214651in}}%
\pgfpathlineto{\pgfqpoint{11.521438in}{1.212069in}}%
\pgfpathlineto{\pgfqpoint{11.521781in}{1.211895in}}%
\pgfpathlineto{\pgfqpoint{11.522123in}{1.210347in}}%
\pgfpathlineto{\pgfqpoint{11.522465in}{1.210271in}}%
\pgfpathlineto{\pgfqpoint{11.523150in}{1.208716in}}%
\pgfpathlineto{\pgfqpoint{11.523493in}{1.208686in}}%
\pgfpathlineto{\pgfqpoint{11.523835in}{1.207406in}}%
\pgfpathlineto{\pgfqpoint{11.524862in}{1.206798in}}%
\pgfpathlineto{\pgfqpoint{11.525204in}{1.203928in}}%
\pgfpathlineto{\pgfqpoint{11.527601in}{1.202569in}}%
\pgfpathlineto{\pgfqpoint{11.528970in}{1.202324in}}%
\pgfpathlineto{\pgfqpoint{11.529312in}{1.201889in}}%
\pgfpathlineto{\pgfqpoint{11.531367in}{1.196694in}}%
\pgfpathlineto{\pgfqpoint{11.531709in}{1.196331in}}%
\pgfpathlineto{\pgfqpoint{11.532736in}{1.193613in}}%
\pgfpathlineto{\pgfqpoint{11.533078in}{1.193492in}}%
\pgfpathlineto{\pgfqpoint{11.536159in}{1.185993in}}%
\pgfpathlineto{\pgfqpoint{11.536502in}{1.185918in}}%
\pgfpathlineto{\pgfqpoint{11.537186in}{1.184611in}}%
\pgfpathlineto{\pgfqpoint{11.537871in}{1.184460in}}%
\pgfpathlineto{\pgfqpoint{11.538556in}{1.183935in}}%
\pgfpathlineto{\pgfqpoint{11.540268in}{1.180231in}}%
\pgfpathlineto{\pgfqpoint{11.540952in}{1.179272in}}%
\pgfpathlineto{\pgfqpoint{11.541637in}{1.178661in}}%
\pgfpathlineto{\pgfqpoint{11.542322in}{1.178404in}}%
\pgfpathlineto{\pgfqpoint{11.543006in}{1.176425in}}%
\pgfpathlineto{\pgfqpoint{11.544376in}{1.175459in}}%
\pgfpathlineto{\pgfqpoint{11.545060in}{1.173979in}}%
\pgfpathlineto{\pgfqpoint{11.545745in}{1.173420in}}%
\pgfpathlineto{\pgfqpoint{11.546430in}{1.172378in}}%
\pgfpathlineto{\pgfqpoint{11.546772in}{1.171909in}}%
\pgfpathlineto{\pgfqpoint{11.547799in}{1.168570in}}%
\pgfpathlineto{\pgfqpoint{11.548826in}{1.165944in}}%
\pgfpathlineto{\pgfqpoint{11.549511in}{1.165521in}}%
\pgfpathlineto{\pgfqpoint{11.552250in}{1.160898in}}%
\pgfpathlineto{\pgfqpoint{11.552592in}{1.160756in}}%
\pgfpathlineto{\pgfqpoint{11.553277in}{1.159751in}}%
\pgfpathlineto{\pgfqpoint{11.554646in}{1.158422in}}%
\pgfpathlineto{\pgfqpoint{11.554988in}{1.158362in}}%
\pgfpathlineto{\pgfqpoint{11.557043in}{1.156164in}}%
\pgfpathlineto{\pgfqpoint{11.557385in}{1.156099in}}%
\pgfpathlineto{\pgfqpoint{11.558754in}{1.153589in}}%
\pgfpathlineto{\pgfqpoint{11.559097in}{1.151535in}}%
\pgfpathlineto{\pgfqpoint{11.561151in}{1.149670in}}%
\pgfpathlineto{\pgfqpoint{11.561493in}{1.149255in}}%
\pgfpathlineto{\pgfqpoint{11.562178in}{1.147631in}}%
\pgfpathlineto{\pgfqpoint{11.562520in}{1.147548in}}%
\pgfpathlineto{\pgfqpoint{11.562862in}{1.145917in}}%
\pgfpathlineto{\pgfqpoint{11.563547in}{1.145434in}}%
\pgfpathlineto{\pgfqpoint{11.563889in}{1.141203in}}%
\pgfpathlineto{\pgfqpoint{11.564232in}{1.140812in}}%
\pgfpathlineto{\pgfqpoint{11.564574in}{1.139876in}}%
\pgfpathlineto{\pgfqpoint{11.564916in}{1.139815in}}%
\pgfpathlineto{\pgfqpoint{11.565259in}{1.138970in}}%
\pgfpathlineto{\pgfqpoint{11.565601in}{1.138947in}}%
\pgfpathlineto{\pgfqpoint{11.566628in}{1.137746in}}%
\pgfpathlineto{\pgfqpoint{11.568682in}{1.136466in}}%
\pgfpathlineto{\pgfqpoint{11.569709in}{1.134982in}}%
\pgfpathlineto{\pgfqpoint{11.570052in}{1.133925in}}%
\pgfpathlineto{\pgfqpoint{11.570736in}{1.133895in}}%
\pgfpathlineto{\pgfqpoint{11.571079in}{1.132717in}}%
\pgfpathlineto{\pgfqpoint{11.571763in}{1.132536in}}%
\pgfpathlineto{\pgfqpoint{11.572448in}{1.132052in}}%
\pgfpathlineto{\pgfqpoint{11.573133in}{1.131690in}}%
\pgfpathlineto{\pgfqpoint{11.574160in}{1.130867in}}%
\pgfpathlineto{\pgfqpoint{11.574502in}{1.128752in}}%
\pgfpathlineto{\pgfqpoint{11.574845in}{1.128714in}}%
\pgfpathlineto{\pgfqpoint{11.575872in}{1.127219in}}%
\pgfpathlineto{\pgfqpoint{11.576214in}{1.127219in}}%
\pgfpathlineto{\pgfqpoint{11.576899in}{1.126147in}}%
\pgfpathlineto{\pgfqpoint{11.577583in}{1.126102in}}%
\pgfpathlineto{\pgfqpoint{11.578268in}{1.125256in}}%
\pgfpathlineto{\pgfqpoint{11.578610in}{1.124308in}}%
\pgfpathlineto{\pgfqpoint{11.578953in}{1.124259in}}%
\pgfpathlineto{\pgfqpoint{11.579295in}{1.123504in}}%
\pgfpathlineto{\pgfqpoint{11.580322in}{1.118822in}}%
\pgfpathlineto{\pgfqpoint{11.580664in}{1.118731in}}%
\pgfpathlineto{\pgfqpoint{11.581007in}{1.116194in}}%
\pgfpathlineto{\pgfqpoint{11.581349in}{1.116103in}}%
\pgfpathlineto{\pgfqpoint{11.581691in}{1.115076in}}%
\pgfpathlineto{\pgfqpoint{11.582376in}{1.114820in}}%
\pgfpathlineto{\pgfqpoint{11.582719in}{1.111279in}}%
\pgfpathlineto{\pgfqpoint{11.583061in}{1.110983in}}%
\pgfpathlineto{\pgfqpoint{11.583746in}{1.109231in}}%
\pgfpathlineto{\pgfqpoint{11.584088in}{1.109005in}}%
\pgfpathlineto{\pgfqpoint{11.584773in}{1.104565in}}%
\pgfpathlineto{\pgfqpoint{11.585115in}{1.104361in}}%
\pgfpathlineto{\pgfqpoint{11.585800in}{1.102631in}}%
\pgfpathlineto{\pgfqpoint{11.586484in}{1.102322in}}%
\pgfpathlineto{\pgfqpoint{11.587511in}{1.101227in}}%
\pgfpathlineto{\pgfqpoint{11.587854in}{1.100940in}}%
\pgfpathlineto{\pgfqpoint{11.588538in}{1.099025in}}%
\pgfpathlineto{\pgfqpoint{11.588881in}{1.098735in}}%
\pgfpathlineto{\pgfqpoint{11.589565in}{1.097527in}}%
\pgfpathlineto{\pgfqpoint{11.590250in}{1.097073in}}%
\pgfpathlineto{\pgfqpoint{11.591277in}{1.095624in}}%
\pgfpathlineto{\pgfqpoint{11.591620in}{1.094355in}}%
\pgfpathlineto{\pgfqpoint{11.591962in}{1.091667in}}%
\pgfpathlineto{\pgfqpoint{11.592304in}{1.091523in}}%
\pgfpathlineto{\pgfqpoint{11.595385in}{1.086426in}}%
\pgfpathlineto{\pgfqpoint{11.595728in}{1.085082in}}%
\pgfpathlineto{\pgfqpoint{11.596412in}{1.085029in}}%
\pgfpathlineto{\pgfqpoint{11.597439in}{1.079290in}}%
\pgfpathlineto{\pgfqpoint{11.598124in}{1.075204in}}%
\pgfpathlineto{\pgfqpoint{11.598466in}{1.075204in}}%
\pgfpathlineto{\pgfqpoint{11.599151in}{1.073164in}}%
\pgfpathlineto{\pgfqpoint{11.600178in}{1.068906in}}%
\pgfpathlineto{\pgfqpoint{11.600863in}{1.067358in}}%
\pgfpathlineto{\pgfqpoint{11.602575in}{1.063129in}}%
\pgfpathlineto{\pgfqpoint{11.603602in}{1.061611in}}%
\pgfpathlineto{\pgfqpoint{11.604286in}{1.060010in}}%
\pgfpathlineto{\pgfqpoint{11.604971in}{1.059527in}}%
\pgfpathlineto{\pgfqpoint{11.605313in}{1.058636in}}%
\pgfpathlineto{\pgfqpoint{11.605998in}{1.054392in}}%
\pgfpathlineto{\pgfqpoint{11.606683in}{1.053690in}}%
\pgfpathlineto{\pgfqpoint{11.607367in}{1.052485in}}%
\pgfpathlineto{\pgfqpoint{11.607710in}{1.052278in}}%
\pgfpathlineto{\pgfqpoint{11.608052in}{1.049166in}}%
\pgfpathlineto{\pgfqpoint{11.608737in}{1.048647in}}%
\pgfpathlineto{\pgfqpoint{11.609079in}{1.045662in}}%
\pgfpathlineto{\pgfqpoint{11.609764in}{1.044786in}}%
\pgfpathlineto{\pgfqpoint{11.610106in}{1.044639in}}%
\pgfpathlineto{\pgfqpoint{11.611476in}{1.041313in}}%
\pgfpathlineto{\pgfqpoint{11.612160in}{1.040095in}}%
\pgfpathlineto{\pgfqpoint{11.612503in}{1.037718in}}%
\pgfpathlineto{\pgfqpoint{11.612845in}{1.037416in}}%
\pgfpathlineto{\pgfqpoint{11.613187in}{1.035332in}}%
\pgfpathlineto{\pgfqpoint{11.613872in}{1.034773in}}%
\pgfpathlineto{\pgfqpoint{11.615241in}{1.030287in}}%
\pgfpathlineto{\pgfqpoint{11.615584in}{1.029676in}}%
\pgfpathlineto{\pgfqpoint{11.616268in}{1.026844in}}%
\pgfpathlineto{\pgfqpoint{11.616611in}{1.026149in}}%
\pgfpathlineto{\pgfqpoint{11.617296in}{1.022373in}}%
\pgfpathlineto{\pgfqpoint{11.617980in}{1.017367in}}%
\pgfpathlineto{\pgfqpoint{11.618323in}{1.016272in}}%
\pgfpathlineto{\pgfqpoint{11.618665in}{1.016234in}}%
\pgfpathlineto{\pgfqpoint{11.621404in}{1.010442in}}%
\pgfpathlineto{\pgfqpoint{11.621746in}{1.009083in}}%
\pgfpathlineto{\pgfqpoint{11.622431in}{1.005473in}}%
\pgfpathlineto{\pgfqpoint{11.622773in}{1.004718in}}%
\pgfpathlineto{\pgfqpoint{11.623115in}{1.001229in}}%
\pgfpathlineto{\pgfqpoint{11.623458in}{1.000967in}}%
\pgfpathlineto{\pgfqpoint{11.624827in}{0.995822in}}%
\pgfpathlineto{\pgfqpoint{11.625169in}{0.995309in}}%
\pgfpathlineto{\pgfqpoint{11.625854in}{0.988270in}}%
\pgfpathlineto{\pgfqpoint{11.626197in}{0.987727in}}%
\pgfpathlineto{\pgfqpoint{11.626881in}{0.981383in}}%
\pgfpathlineto{\pgfqpoint{11.627224in}{0.981383in}}%
\pgfpathlineto{\pgfqpoint{11.627566in}{0.980507in}}%
\pgfpathlineto{\pgfqpoint{11.628251in}{0.975916in}}%
\pgfpathlineto{\pgfqpoint{11.628593in}{0.975282in}}%
\pgfpathlineto{\pgfqpoint{11.628935in}{0.973318in}}%
\pgfpathlineto{\pgfqpoint{11.629278in}{0.973254in}}%
\pgfpathlineto{\pgfqpoint{11.629962in}{0.969301in}}%
\pgfpathlineto{\pgfqpoint{11.630305in}{0.968425in}}%
\pgfpathlineto{\pgfqpoint{11.630989in}{0.961749in}}%
\pgfpathlineto{\pgfqpoint{11.631332in}{0.961387in}}%
\pgfpathlineto{\pgfqpoint{11.632701in}{0.956282in}}%
\pgfpathlineto{\pgfqpoint{11.634413in}{0.952661in}}%
\pgfpathlineto{\pgfqpoint{11.635098in}{0.950195in}}%
\pgfpathlineto{\pgfqpoint{11.635440in}{0.949717in}}%
\pgfpathlineto{\pgfqpoint{11.635782in}{0.946457in}}%
\pgfpathlineto{\pgfqpoint{11.636467in}{0.936285in}}%
\pgfpathlineto{\pgfqpoint{11.636809in}{0.934318in}}%
\pgfpathlineto{\pgfqpoint{11.637152in}{0.934111in}}%
\pgfpathlineto{\pgfqpoint{11.637494in}{0.933204in}}%
\pgfpathlineto{\pgfqpoint{11.638521in}{0.920654in}}%
\pgfpathlineto{\pgfqpoint{11.638863in}{0.919798in}}%
\pgfpathlineto{\pgfqpoint{11.639890in}{0.912876in}}%
\pgfpathlineto{\pgfqpoint{11.640575in}{0.907454in}}%
\pgfpathlineto{\pgfqpoint{11.641260in}{0.907076in}}%
\pgfpathlineto{\pgfqpoint{11.642972in}{0.772020in}}%
\pgfpathlineto{\pgfqpoint{11.643656in}{0.757638in}}%
\pgfpathlineto{\pgfqpoint{11.643999in}{0.757102in}}%
\pgfpathlineto{\pgfqpoint{11.644683in}{0.746756in}}%
\pgfpathlineto{\pgfqpoint{11.645026in}{0.724517in}}%
\pgfpathlineto{\pgfqpoint{11.645368in}{0.722859in}}%
\pgfpathlineto{\pgfqpoint{11.646053in}{0.690467in}}%
\pgfpathlineto{\pgfqpoint{11.646395in}{0.685876in}}%
\pgfpathlineto{\pgfqpoint{11.646737in}{0.670138in}}%
\pgfpathlineto{\pgfqpoint{11.646737in}{0.670138in}}%
\pgfusepath{stroke}%
\end{pgfscope}%
\begin{pgfscope}%
\pgfsetrectcap%
\pgfsetmiterjoin%
\pgfsetlinewidth{0.803000pt}%
\definecolor{currentstroke}{rgb}{0.000000,0.000000,0.000000}%
\pgfsetstrokecolor{currentstroke}%
\pgfsetdash{}{0pt}%
\pgfpathmoveto{\pgfqpoint{8.609492in}{0.670138in}}%
\pgfpathlineto{\pgfqpoint{8.609492in}{5.516628in}}%
\pgfusepath{stroke}%
\end{pgfscope}%
\begin{pgfscope}%
\pgfsetrectcap%
\pgfsetmiterjoin%
\pgfsetlinewidth{0.803000pt}%
\definecolor{currentstroke}{rgb}{0.000000,0.000000,0.000000}%
\pgfsetstrokecolor{currentstroke}%
\pgfsetdash{}{0pt}%
\pgfpathmoveto{\pgfqpoint{11.676809in}{0.670138in}}%
\pgfpathlineto{\pgfqpoint{11.676809in}{5.516628in}}%
\pgfusepath{stroke}%
\end{pgfscope}%
\begin{pgfscope}%
\pgfsetrectcap%
\pgfsetmiterjoin%
\pgfsetlinewidth{0.803000pt}%
\definecolor{currentstroke}{rgb}{0.000000,0.000000,0.000000}%
\pgfsetstrokecolor{currentstroke}%
\pgfsetdash{}{0pt}%
\pgfpathmoveto{\pgfqpoint{8.609492in}{0.670138in}}%
\pgfpathlineto{\pgfqpoint{11.676809in}{0.670138in}}%
\pgfusepath{stroke}%
\end{pgfscope}%
\begin{pgfscope}%
\pgfsetrectcap%
\pgfsetmiterjoin%
\pgfsetlinewidth{0.803000pt}%
\definecolor{currentstroke}{rgb}{0.000000,0.000000,0.000000}%
\pgfsetstrokecolor{currentstroke}%
\pgfsetdash{}{0pt}%
\pgfpathmoveto{\pgfqpoint{8.609492in}{5.516628in}}%
\pgfpathlineto{\pgfqpoint{11.676809in}{5.516628in}}%
\pgfusepath{stroke}%
\end{pgfscope}%
\begin{pgfscope}%
\definecolor{textcolor}{rgb}{0.000000,0.000000,0.000000}%
\pgfsetstrokecolor{textcolor}%
\pgfsetfillcolor{textcolor}%
\pgftext[x=10.143151in,y=5.599962in,,base]{\color{textcolor}{\rmfamily\fontsize{20.000000}{24.000000}\selectfont\catcode`\^=\active\def^{\ifmmode\sp\else\^{}\fi}\catcode`\%=\active\def%{\%}Load Duration Curve}}%
\end{pgfscope}%
\end{pgfpicture}%
\makeatother%
\endgroup%
}
  \caption{The normalized demand and load duration curves that are used in this thesis.}
  \label{fig:normalized_ldc}
\end{figure}

\FloatBarrier
\subsection{Exercise 1: Deciding Among Evolutionary Algorithms}

\ac{osier} allows users to choose among a variety of \ac{moo} methods. This is
motivated by the desire for flexibility. However, Exercises 1 and 2 use just one
algorithm, \ac{unsga3} as implemented by \ac{pymoo}, which should be justified
by comparing the results against different algorithms. As an important aside,
although I used the \ac{deap} implementations of \ac{nsga2} and \ac{nsga3},
these algorithms are not exclusive to \ac{deap}. \ac{pymoo} also implements
them, I simply wanted to show the breadth of support for different tools in
\ac{osier}. Figure \ref{fig:algorithm-comparison} provides the justification for
choosing \ac{unsga3} by comparing the results of three \ac{moo} algorithms by
showing the respective scatter plots and a density plot of the points on each
axis. Since it took \ac{unsga3} 128 generations to reach its convergence
criterion, the other two algorithms were also stopped after 128 generations,
before converging. The density plot above the scatter plot shows the density of
points along the ``total cost'' objective. Similarly, the density plot to the
right shows the distribution of points for the ``emissions'' objective.

There are a few notable features of Figure \ref{fig:algorithm-comparison}.
First, all three algorithms identified very similar Pareto fronts, the main
differences involve the distribution of points and the extent of their
respective solution sets. Second, the two \ac{deap} algorithms have a greater
number of points along the bottom part of the Pareto front, indicating a greater
sampling over the cost objective. This is further supported by the higher
concentration of points along the lower half of the emission objective's range.
Third, the algorithms implemented by \ac{deap} both have more extreme values
along both axes. All of these features can be attributed to the fact that
neither \ac{nsga2} nor \ac{nsga3} fully converged. Thus, choosing \ac{unsga3}
will be used for the remaining exercises for its faster convergence.

\begin{figure}[ht]
  \centering
  \resizebox{0.75\columnwidth}{!}{\input{figures/04_benchmark_chapter/algorithm_comparison_kde.pgf}}
  \caption{Compares the \ac{moo} algorithms.}
  \label{fig:algorithm-comparison}
\end{figure}

\FloatBarrier

\subsection{Exercise 2: Exploring objective space}
Since structural uncertainty persists regardless of the number of objectives
used, it's important to check the near-optimal objective space for alternative
solutions. In the first benchmark exercise, I used \ac{temoa} to calculate the
least-cost solution. Then I generated 30 alternative solutions with \ac{mga} as
described in Section \ref{section:mga} with a 10\% slack variable added to
\ac{temoa}'s objective function. Figure \ref{fig:temoa-benchmark-01} shows the
points from \ac{temoa} in red and \ac{osier}'s Pareto-front for the same problem
in black. The red- and gray-shaded regions are the sub-optimal spaces (i.e.,
within 10\% of any objective) for \ac{temoa} and \ac{osier}, respectively.
The solid black points indicates points along the Pareto-front, while the open
black points are points tested in early generations of the \ac{osier} simulation.
\footnote{There are many more tested points than shown in Figure \ref{fig:temoa-benchmark-01}. 
For simplicity and clarity, Figure \ref{fig:temoa-benchmark-01} shows a random subset
of points.}

\begin{figure}[h]
  \centering
  % \resizebox{0.6\columnwidth}{!}{%% Creator: Matplotlib, PGF backend
%%
%% To include the figure in your LaTeX document, write
%%   \input{<filename>.pgf}
%%
%% Make sure the required packages are loaded in your preamble
%%   \usepackage{pgf}
%%
%% Also ensure that all the required font packages are loaded; for instance,
%% the lmodern package is sometimes necessary when using math font.
%%   \usepackage{lmodern}
%%
%% Figures using additional raster images can only be included by \input if
%% they are in the same directory as the main LaTeX file. For loading figures
%% from other directories you can use the `import` package
%%   \usepackage{import}
%%
%% and then include the figures with
%%   \import{<path to file>}{<filename>.pgf}
%%
%% Matplotlib used the following preamble
%%   \def\mathdefault#1{#1}
%%   \everymath=\expandafter{\the\everymath\displaystyle}
%%   \IfFileExists{scrextend.sty}{
%%     \usepackage[fontsize=10.000000pt]{scrextend}
%%   }{
%%     \renewcommand{\normalsize}{\fontsize{10.000000}{12.000000}\selectfont}
%%     \normalsize
%%   }
%%   
%%   \makeatletter\@ifpackageloaded{underscore}{}{\usepackage[strings]{underscore}}\makeatother
%%
\begingroup%
\makeatletter%
\begin{pgfpicture}%
\pgfpathrectangle{\pgfpointorigin}{\pgfqpoint{6.988192in}{5.458470in}}%
\pgfusepath{use as bounding box, clip}%
\begin{pgfscope}%
\pgfsetbuttcap%
\pgfsetmiterjoin%
\definecolor{currentfill}{rgb}{1.000000,1.000000,1.000000}%
\pgfsetfillcolor{currentfill}%
\pgfsetlinewidth{0.000000pt}%
\definecolor{currentstroke}{rgb}{0.000000,0.000000,0.000000}%
\pgfsetstrokecolor{currentstroke}%
\pgfsetdash{}{0pt}%
\pgfpathmoveto{\pgfqpoint{0.000000in}{0.000000in}}%
\pgfpathlineto{\pgfqpoint{6.988192in}{0.000000in}}%
\pgfpathlineto{\pgfqpoint{6.988192in}{5.458470in}}%
\pgfpathlineto{\pgfqpoint{0.000000in}{5.458470in}}%
\pgfpathlineto{\pgfqpoint{0.000000in}{0.000000in}}%
\pgfpathclose%
\pgfusepath{fill}%
\end{pgfscope}%
\begin{pgfscope}%
\pgfsetbuttcap%
\pgfsetmiterjoin%
\definecolor{currentfill}{rgb}{1.000000,1.000000,1.000000}%
\pgfsetfillcolor{currentfill}%
\pgfsetlinewidth{0.000000pt}%
\definecolor{currentstroke}{rgb}{0.000000,0.000000,0.000000}%
\pgfsetstrokecolor{currentstroke}%
\pgfsetstrokeopacity{0.000000}%
\pgfsetdash{}{0pt}%
\pgfpathmoveto{\pgfqpoint{0.688192in}{0.670138in}}%
\pgfpathlineto{\pgfqpoint{6.888192in}{0.670138in}}%
\pgfpathlineto{\pgfqpoint{6.888192in}{5.290138in}}%
\pgfpathlineto{\pgfqpoint{0.688192in}{5.290138in}}%
\pgfpathlineto{\pgfqpoint{0.688192in}{0.670138in}}%
\pgfpathclose%
\pgfusepath{fill}%
\end{pgfscope}%
\begin{pgfscope}%
\pgfpathrectangle{\pgfqpoint{0.688192in}{0.670138in}}{\pgfqpoint{6.200000in}{4.620000in}}%
\pgfusepath{clip}%
\pgfsetbuttcap%
\pgfsetmiterjoin%
\definecolor{currentfill}{rgb}{0.121569,0.466667,0.705882}%
\pgfsetfillcolor{currentfill}%
\pgfsetfillopacity{0.500000}%
\pgfsetlinewidth{1.003750pt}%
\definecolor{currentstroke}{rgb}{0.121569,0.466667,0.705882}%
\pgfsetstrokecolor{currentstroke}%
\pgfsetstrokeopacity{0.500000}%
\pgfsetdash{}{0pt}%
\pgfpathmoveto{\pgfqpoint{0.741425in}{1.377543in}}%
\pgfpathlineto{\pgfqpoint{0.758703in}{0.955032in}}%
\pgfpathlineto{\pgfqpoint{0.768198in}{0.875033in}}%
\pgfpathlineto{\pgfqpoint{0.774746in}{0.828781in}}%
\pgfpathlineto{\pgfqpoint{0.778243in}{0.822495in}}%
\pgfpathlineto{\pgfqpoint{0.782159in}{0.789611in}}%
\pgfpathlineto{\pgfqpoint{0.786516in}{0.779881in}}%
\pgfpathlineto{\pgfqpoint{0.792538in}{0.779145in}}%
\pgfpathlineto{\pgfqpoint{0.794668in}{0.758056in}}%
\pgfpathlineto{\pgfqpoint{0.799837in}{0.752930in}}%
\pgfpathlineto{\pgfqpoint{0.809370in}{0.751978in}}%
\pgfpathlineto{\pgfqpoint{0.812629in}{0.743975in}}%
\pgfpathlineto{\pgfqpoint{0.815972in}{0.742575in}}%
\pgfpathlineto{\pgfqpoint{0.822987in}{0.738477in}}%
\pgfpathlineto{\pgfqpoint{0.828825in}{0.734937in}}%
\pgfpathlineto{\pgfqpoint{0.829214in}{0.733319in}}%
\pgfpathlineto{\pgfqpoint{0.833044in}{0.730858in}}%
\pgfpathlineto{\pgfqpoint{0.848459in}{0.726329in}}%
\pgfpathlineto{\pgfqpoint{0.864854in}{0.720019in}}%
\pgfpathlineto{\pgfqpoint{0.887104in}{0.715517in}}%
\pgfpathlineto{\pgfqpoint{0.907479in}{0.714004in}}%
\pgfpathlineto{\pgfqpoint{0.908310in}{0.712008in}}%
\pgfpathlineto{\pgfqpoint{0.909513in}{0.708525in}}%
\pgfpathlineto{\pgfqpoint{0.912740in}{0.707284in}}%
\pgfpathlineto{\pgfqpoint{0.920440in}{0.706723in}}%
\pgfpathlineto{\pgfqpoint{0.925670in}{0.705238in}}%
\pgfpathlineto{\pgfqpoint{0.948903in}{0.702931in}}%
\pgfpathlineto{\pgfqpoint{0.951945in}{0.701707in}}%
\pgfpathlineto{\pgfqpoint{0.952035in}{0.700391in}}%
\pgfpathlineto{\pgfqpoint{0.957029in}{0.700173in}}%
\pgfpathlineto{\pgfqpoint{0.968828in}{0.697963in}}%
\pgfpathlineto{\pgfqpoint{0.974412in}{0.697738in}}%
\pgfpathlineto{\pgfqpoint{0.975275in}{0.696914in}}%
\pgfpathlineto{\pgfqpoint{1.021767in}{0.694795in}}%
\pgfpathlineto{\pgfqpoint{1.025407in}{0.690657in}}%
\pgfpathlineto{\pgfqpoint{1.027475in}{0.690338in}}%
\pgfpathlineto{\pgfqpoint{1.034837in}{0.689784in}}%
\pgfpathlineto{\pgfqpoint{1.049406in}{0.687676in}}%
\pgfpathlineto{\pgfqpoint{1.054714in}{0.687138in}}%
\pgfpathlineto{\pgfqpoint{1.059617in}{0.686467in}}%
\pgfpathlineto{\pgfqpoint{1.072141in}{0.685078in}}%
\pgfpathlineto{\pgfqpoint{1.092208in}{0.684413in}}%
\pgfpathlineto{\pgfqpoint{1.115209in}{0.684111in}}%
\pgfpathlineto{\pgfqpoint{1.131834in}{0.684071in}}%
\pgfpathlineto{\pgfqpoint{1.152628in}{0.684059in}}%
\pgfpathlineto{\pgfqpoint{1.251312in}{0.683263in}}%
\pgfpathlineto{\pgfqpoint{1.277476in}{0.683159in}}%
\pgfpathlineto{\pgfqpoint{1.314870in}{0.682855in}}%
\pgfpathlineto{\pgfqpoint{1.369253in}{0.682756in}}%
\pgfpathlineto{\pgfqpoint{1.398687in}{0.682288in}}%
\pgfpathlineto{\pgfqpoint{1.467852in}{0.682134in}}%
\pgfpathlineto{\pgfqpoint{1.557026in}{0.681680in}}%
\pgfpathlineto{\pgfqpoint{1.627242in}{0.680913in}}%
\pgfpathlineto{\pgfqpoint{1.737728in}{0.680478in}}%
\pgfpathlineto{\pgfqpoint{1.887036in}{0.679610in}}%
\pgfpathlineto{\pgfqpoint{2.037481in}{0.678826in}}%
\pgfpathlineto{\pgfqpoint{2.258348in}{0.677741in}}%
\pgfpathlineto{\pgfqpoint{2.626338in}{0.676361in}}%
\pgfpathlineto{\pgfqpoint{3.263784in}{0.674352in}}%
\pgfpathlineto{\pgfqpoint{5.322800in}{0.670138in}}%
\pgfpathlineto{\pgfqpoint{6.888192in}{0.683471in}}%
\pgfpathlineto{\pgfqpoint{4.623274in}{0.688107in}}%
\pgfpathlineto{\pgfqpoint{3.922083in}{0.690316in}}%
\pgfpathlineto{\pgfqpoint{3.517294in}{0.691835in}}%
\pgfpathlineto{\pgfqpoint{3.274341in}{0.693028in}}%
\pgfpathlineto{\pgfqpoint{3.108851in}{0.693891in}}%
\pgfpathlineto{\pgfqpoint{2.944612in}{0.694845in}}%
\pgfpathlineto{\pgfqpoint{2.823078in}{0.695324in}}%
\pgfpathlineto{\pgfqpoint{2.745840in}{0.696168in}}%
\pgfpathlineto{\pgfqpoint{2.647748in}{0.696667in}}%
\pgfpathlineto{\pgfqpoint{2.571667in}{0.696836in}}%
\pgfpathlineto{\pgfqpoint{2.539290in}{0.697351in}}%
\pgfpathlineto{\pgfqpoint{2.479468in}{0.697460in}}%
\pgfpathlineto{\pgfqpoint{2.438334in}{0.697795in}}%
\pgfpathlineto{\pgfqpoint{2.409554in}{0.697909in}}%
\pgfpathlineto{\pgfqpoint{2.301002in}{0.698784in}}%
\pgfpathlineto{\pgfqpoint{2.278129in}{0.698798in}}%
\pgfpathlineto{\pgfqpoint{2.259841in}{0.698842in}}%
\pgfpathlineto{\pgfqpoint{2.234540in}{0.699174in}}%
\pgfpathlineto{\pgfqpoint{2.212467in}{0.699905in}}%
\pgfpathlineto{\pgfqpoint{2.198690in}{0.701434in}}%
\pgfpathlineto{\pgfqpoint{2.193297in}{0.702172in}}%
\pgfpathlineto{\pgfqpoint{2.187458in}{0.702763in}}%
\pgfpathlineto{\pgfqpoint{2.171432in}{0.705082in}}%
\pgfpathlineto{\pgfqpoint{2.163334in}{0.705692in}}%
\pgfpathlineto{\pgfqpoint{2.161059in}{0.706043in}}%
\pgfpathlineto{\pgfqpoint{2.157055in}{0.710594in}}%
\pgfpathlineto{\pgfqpoint{2.105914in}{0.712924in}}%
\pgfpathlineto{\pgfqpoint{2.104964in}{0.713831in}}%
\pgfpathlineto{\pgfqpoint{2.098822in}{0.714079in}}%
\pgfpathlineto{\pgfqpoint{2.085843in}{0.716510in}}%
\pgfpathlineto{\pgfqpoint{2.080349in}{0.716749in}}%
\pgfpathlineto{\pgfqpoint{2.080251in}{0.718197in}}%
\pgfpathlineto{\pgfqpoint{2.076904in}{0.719544in}}%
\pgfpathlineto{\pgfqpoint{2.051349in}{0.722081in}}%
\pgfpathlineto{\pgfqpoint{2.045595in}{0.723715in}}%
\pgfpathlineto{\pgfqpoint{2.037125in}{0.724332in}}%
\pgfpathlineto{\pgfqpoint{2.033576in}{0.725697in}}%
\pgfpathlineto{\pgfqpoint{2.032252in}{0.729528in}}%
\pgfpathlineto{\pgfqpoint{2.031338in}{0.731724in}}%
\pgfpathlineto{\pgfqpoint{2.008926in}{0.733388in}}%
\pgfpathlineto{\pgfqpoint{1.984450in}{0.738340in}}%
\pgfpathlineto{\pgfqpoint{1.966417in}{0.745282in}}%
\pgfpathlineto{\pgfqpoint{1.949459in}{0.750264in}}%
\pgfpathlineto{\pgfqpoint{1.945246in}{0.752971in}}%
\pgfpathlineto{\pgfqpoint{1.944819in}{0.754750in}}%
\pgfpathlineto{\pgfqpoint{1.938397in}{0.758644in}}%
\pgfpathlineto{\pgfqpoint{1.930681in}{0.763152in}}%
\pgfpathlineto{\pgfqpoint{1.927003in}{0.764692in}}%
\pgfpathlineto{\pgfqpoint{1.923418in}{0.773495in}}%
\pgfpathlineto{\pgfqpoint{1.912932in}{0.774543in}}%
\pgfpathlineto{\pgfqpoint{1.907246in}{0.780181in}}%
\pgfpathlineto{\pgfqpoint{1.904902in}{0.803379in}}%
\pgfpathlineto{\pgfqpoint{1.898279in}{0.804188in}}%
\pgfpathlineto{\pgfqpoint{1.893486in}{0.814892in}}%
\pgfpathlineto{\pgfqpoint{1.889178in}{0.851064in}}%
\pgfpathlineto{\pgfqpoint{1.885331in}{0.857978in}}%
\pgfpathlineto{\pgfqpoint{1.878129in}{0.908856in}}%
\pgfpathlineto{\pgfqpoint{1.867684in}{0.996854in}}%
\pgfpathlineto{\pgfqpoint{1.848679in}{1.461617in}}%
\pgfpathlineto{\pgfqpoint{0.741425in}{1.377543in}}%
\pgfpathclose%
\pgfusepath{stroke,fill}%
\end{pgfscope}%
\begin{pgfscope}%
\pgfpathrectangle{\pgfqpoint{0.688192in}{0.670138in}}{\pgfqpoint{6.200000in}{4.620000in}}%
\pgfusepath{clip}%
\pgfsetbuttcap%
\pgfsetroundjoin%
\pgfsetlinewidth{1.003750pt}%
\definecolor{currentstroke}{rgb}{1.000000,0.000000,0.000000}%
\pgfsetstrokecolor{currentstroke}%
\pgfsetdash{}{0pt}%
\pgfpathmoveto{\pgfqpoint{1.380312in}{4.826960in}}%
\pgfpathcurveto{\pgfqpoint{1.388548in}{4.826960in}}{\pgfqpoint{1.396449in}{4.830233in}}{\pgfqpoint{1.402272in}{4.836057in}}%
\pgfpathcurveto{\pgfqpoint{1.408096in}{4.841881in}}{\pgfqpoint{1.411369in}{4.849781in}}{\pgfqpoint{1.411369in}{4.858017in}}%
\pgfpathcurveto{\pgfqpoint{1.411369in}{4.866253in}}{\pgfqpoint{1.408096in}{4.874153in}}{\pgfqpoint{1.402272in}{4.879977in}}%
\pgfpathcurveto{\pgfqpoint{1.396449in}{4.885801in}}{\pgfqpoint{1.388548in}{4.889073in}}{\pgfqpoint{1.380312in}{4.889073in}}%
\pgfpathcurveto{\pgfqpoint{1.372076in}{4.889073in}}{\pgfqpoint{1.364176in}{4.885801in}}{\pgfqpoint{1.358352in}{4.879977in}}%
\pgfpathcurveto{\pgfqpoint{1.352528in}{4.874153in}}{\pgfqpoint{1.349256in}{4.866253in}}{\pgfqpoint{1.349256in}{4.858017in}}%
\pgfpathcurveto{\pgfqpoint{1.349256in}{4.849781in}}{\pgfqpoint{1.352528in}{4.841881in}}{\pgfqpoint{1.358352in}{4.836057in}}%
\pgfpathcurveto{\pgfqpoint{1.364176in}{4.830233in}}{\pgfqpoint{1.372076in}{4.826960in}}{\pgfqpoint{1.380312in}{4.826960in}}%
\pgfpathlineto{\pgfqpoint{1.380312in}{4.826960in}}%
\pgfpathclose%
\pgfusepath{stroke}%
\end{pgfscope}%
\begin{pgfscope}%
\pgfpathrectangle{\pgfqpoint{0.688192in}{0.670138in}}{\pgfqpoint{6.200000in}{4.620000in}}%
\pgfusepath{clip}%
\pgfsetbuttcap%
\pgfsetroundjoin%
\pgfsetlinewidth{1.003750pt}%
\definecolor{currentstroke}{rgb}{1.000000,0.000000,0.000000}%
\pgfsetstrokecolor{currentstroke}%
\pgfsetdash{}{0pt}%
\pgfpathmoveto{\pgfqpoint{1.103776in}{2.239473in}}%
\pgfpathcurveto{\pgfqpoint{1.112013in}{2.239473in}}{\pgfqpoint{1.119913in}{2.242745in}}{\pgfqpoint{1.125737in}{2.248569in}}%
\pgfpathcurveto{\pgfqpoint{1.131560in}{2.254393in}}{\pgfqpoint{1.134833in}{2.262293in}}{\pgfqpoint{1.134833in}{2.270529in}}%
\pgfpathcurveto{\pgfqpoint{1.134833in}{2.278765in}}{\pgfqpoint{1.131560in}{2.286665in}}{\pgfqpoint{1.125737in}{2.292489in}}%
\pgfpathcurveto{\pgfqpoint{1.119913in}{2.298313in}}{\pgfqpoint{1.112013in}{2.301586in}}{\pgfqpoint{1.103776in}{2.301586in}}%
\pgfpathcurveto{\pgfqpoint{1.095540in}{2.301586in}}{\pgfqpoint{1.087640in}{2.298313in}}{\pgfqpoint{1.081816in}{2.292489in}}%
\pgfpathcurveto{\pgfqpoint{1.075992in}{2.286665in}}{\pgfqpoint{1.072720in}{2.278765in}}{\pgfqpoint{1.072720in}{2.270529in}}%
\pgfpathcurveto{\pgfqpoint{1.072720in}{2.262293in}}{\pgfqpoint{1.075992in}{2.254393in}}{\pgfqpoint{1.081816in}{2.248569in}}%
\pgfpathcurveto{\pgfqpoint{1.087640in}{2.242745in}}{\pgfqpoint{1.095540in}{2.239473in}}{\pgfqpoint{1.103776in}{2.239473in}}%
\pgfpathlineto{\pgfqpoint{1.103776in}{2.239473in}}%
\pgfpathclose%
\pgfusepath{stroke}%
\end{pgfscope}%
\begin{pgfscope}%
\pgfpathrectangle{\pgfqpoint{0.688192in}{0.670138in}}{\pgfqpoint{6.200000in}{4.620000in}}%
\pgfusepath{clip}%
\pgfsetbuttcap%
\pgfsetroundjoin%
\pgfsetlinewidth{1.003750pt}%
\definecolor{currentstroke}{rgb}{1.000000,0.000000,0.000000}%
\pgfsetstrokecolor{currentstroke}%
\pgfsetdash{}{0pt}%
\pgfpathmoveto{\pgfqpoint{1.146295in}{2.309659in}}%
\pgfpathcurveto{\pgfqpoint{1.154531in}{2.309659in}}{\pgfqpoint{1.162431in}{2.312932in}}{\pgfqpoint{1.168255in}{2.318756in}}%
\pgfpathcurveto{\pgfqpoint{1.174079in}{2.324579in}}{\pgfqpoint{1.177352in}{2.332480in}}{\pgfqpoint{1.177352in}{2.340716in}}%
\pgfpathcurveto{\pgfqpoint{1.177352in}{2.348952in}}{\pgfqpoint{1.174079in}{2.356852in}}{\pgfqpoint{1.168255in}{2.362676in}}%
\pgfpathcurveto{\pgfqpoint{1.162431in}{2.368500in}}{\pgfqpoint{1.154531in}{2.371772in}}{\pgfqpoint{1.146295in}{2.371772in}}%
\pgfpathcurveto{\pgfqpoint{1.138059in}{2.371772in}}{\pgfqpoint{1.130159in}{2.368500in}}{\pgfqpoint{1.124335in}{2.362676in}}%
\pgfpathcurveto{\pgfqpoint{1.118511in}{2.356852in}}{\pgfqpoint{1.115239in}{2.348952in}}{\pgfqpoint{1.115239in}{2.340716in}}%
\pgfpathcurveto{\pgfqpoint{1.115239in}{2.332480in}}{\pgfqpoint{1.118511in}{2.324579in}}{\pgfqpoint{1.124335in}{2.318756in}}%
\pgfpathcurveto{\pgfqpoint{1.130159in}{2.312932in}}{\pgfqpoint{1.138059in}{2.309659in}}{\pgfqpoint{1.146295in}{2.309659in}}%
\pgfpathlineto{\pgfqpoint{1.146295in}{2.309659in}}%
\pgfpathclose%
\pgfusepath{stroke}%
\end{pgfscope}%
\begin{pgfscope}%
\pgfpathrectangle{\pgfqpoint{0.688192in}{0.670138in}}{\pgfqpoint{6.200000in}{4.620000in}}%
\pgfusepath{clip}%
\pgfsetbuttcap%
\pgfsetroundjoin%
\pgfsetlinewidth{1.003750pt}%
\definecolor{currentstroke}{rgb}{1.000000,0.000000,0.000000}%
\pgfsetstrokecolor{currentstroke}%
\pgfsetdash{}{0pt}%
\pgfpathmoveto{\pgfqpoint{1.283972in}{2.337286in}}%
\pgfpathcurveto{\pgfqpoint{1.292208in}{2.337286in}}{\pgfqpoint{1.300108in}{2.340558in}}{\pgfqpoint{1.305932in}{2.346382in}}%
\pgfpathcurveto{\pgfqpoint{1.311756in}{2.352206in}}{\pgfqpoint{1.315028in}{2.360106in}}{\pgfqpoint{1.315028in}{2.368343in}}%
\pgfpathcurveto{\pgfqpoint{1.315028in}{2.376579in}}{\pgfqpoint{1.311756in}{2.384479in}}{\pgfqpoint{1.305932in}{2.390303in}}%
\pgfpathcurveto{\pgfqpoint{1.300108in}{2.396127in}}{\pgfqpoint{1.292208in}{2.399399in}}{\pgfqpoint{1.283972in}{2.399399in}}%
\pgfpathcurveto{\pgfqpoint{1.275735in}{2.399399in}}{\pgfqpoint{1.267835in}{2.396127in}}{\pgfqpoint{1.262011in}{2.390303in}}%
\pgfpathcurveto{\pgfqpoint{1.256187in}{2.384479in}}{\pgfqpoint{1.252915in}{2.376579in}}{\pgfqpoint{1.252915in}{2.368343in}}%
\pgfpathcurveto{\pgfqpoint{1.252915in}{2.360106in}}{\pgfqpoint{1.256187in}{2.352206in}}{\pgfqpoint{1.262011in}{2.346382in}}%
\pgfpathcurveto{\pgfqpoint{1.267835in}{2.340558in}}{\pgfqpoint{1.275735in}{2.337286in}}{\pgfqpoint{1.283972in}{2.337286in}}%
\pgfpathlineto{\pgfqpoint{1.283972in}{2.337286in}}%
\pgfpathclose%
\pgfusepath{stroke}%
\end{pgfscope}%
\begin{pgfscope}%
\pgfpathrectangle{\pgfqpoint{0.688192in}{0.670138in}}{\pgfqpoint{6.200000in}{4.620000in}}%
\pgfusepath{clip}%
\pgfsetbuttcap%
\pgfsetroundjoin%
\pgfsetlinewidth{1.003750pt}%
\definecolor{currentstroke}{rgb}{1.000000,0.000000,0.000000}%
\pgfsetstrokecolor{currentstroke}%
\pgfsetdash{}{0pt}%
\pgfpathmoveto{\pgfqpoint{1.194044in}{2.459126in}}%
\pgfpathcurveto{\pgfqpoint{1.202281in}{2.459126in}}{\pgfqpoint{1.210181in}{2.462398in}}{\pgfqpoint{1.216005in}{2.468222in}}%
\pgfpathcurveto{\pgfqpoint{1.221828in}{2.474046in}}{\pgfqpoint{1.225101in}{2.481946in}}{\pgfqpoint{1.225101in}{2.490182in}}%
\pgfpathcurveto{\pgfqpoint{1.225101in}{2.498419in}}{\pgfqpoint{1.221828in}{2.506319in}}{\pgfqpoint{1.216005in}{2.512143in}}%
\pgfpathcurveto{\pgfqpoint{1.210181in}{2.517967in}}{\pgfqpoint{1.202281in}{2.521239in}}{\pgfqpoint{1.194044in}{2.521239in}}%
\pgfpathcurveto{\pgfqpoint{1.185808in}{2.521239in}}{\pgfqpoint{1.177908in}{2.517967in}}{\pgfqpoint{1.172084in}{2.512143in}}%
\pgfpathcurveto{\pgfqpoint{1.166260in}{2.506319in}}{\pgfqpoint{1.162988in}{2.498419in}}{\pgfqpoint{1.162988in}{2.490182in}}%
\pgfpathcurveto{\pgfqpoint{1.162988in}{2.481946in}}{\pgfqpoint{1.166260in}{2.474046in}}{\pgfqpoint{1.172084in}{2.468222in}}%
\pgfpathcurveto{\pgfqpoint{1.177908in}{2.462398in}}{\pgfqpoint{1.185808in}{2.459126in}}{\pgfqpoint{1.194044in}{2.459126in}}%
\pgfpathlineto{\pgfqpoint{1.194044in}{2.459126in}}%
\pgfpathclose%
\pgfusepath{stroke}%
\end{pgfscope}%
\begin{pgfscope}%
\pgfpathrectangle{\pgfqpoint{0.688192in}{0.670138in}}{\pgfqpoint{6.200000in}{4.620000in}}%
\pgfusepath{clip}%
\pgfsetbuttcap%
\pgfsetroundjoin%
\pgfsetlinewidth{1.003750pt}%
\definecolor{currentstroke}{rgb}{1.000000,0.000000,0.000000}%
\pgfsetstrokecolor{currentstroke}%
\pgfsetdash{}{0pt}%
\pgfpathmoveto{\pgfqpoint{1.126156in}{2.039611in}}%
\pgfpathcurveto{\pgfqpoint{1.134392in}{2.039611in}}{\pgfqpoint{1.142292in}{2.042884in}}{\pgfqpoint{1.148116in}{2.048708in}}%
\pgfpathcurveto{\pgfqpoint{1.153940in}{2.054532in}}{\pgfqpoint{1.157212in}{2.062432in}}{\pgfqpoint{1.157212in}{2.070668in}}%
\pgfpathcurveto{\pgfqpoint{1.157212in}{2.078904in}}{\pgfqpoint{1.153940in}{2.086804in}}{\pgfqpoint{1.148116in}{2.092628in}}%
\pgfpathcurveto{\pgfqpoint{1.142292in}{2.098452in}}{\pgfqpoint{1.134392in}{2.101724in}}{\pgfqpoint{1.126156in}{2.101724in}}%
\pgfpathcurveto{\pgfqpoint{1.117920in}{2.101724in}}{\pgfqpoint{1.110020in}{2.098452in}}{\pgfqpoint{1.104196in}{2.092628in}}%
\pgfpathcurveto{\pgfqpoint{1.098372in}{2.086804in}}{\pgfqpoint{1.095099in}{2.078904in}}{\pgfqpoint{1.095099in}{2.070668in}}%
\pgfpathcurveto{\pgfqpoint{1.095099in}{2.062432in}}{\pgfqpoint{1.098372in}{2.054532in}}{\pgfqpoint{1.104196in}{2.048708in}}%
\pgfpathcurveto{\pgfqpoint{1.110020in}{2.042884in}}{\pgfqpoint{1.117920in}{2.039611in}}{\pgfqpoint{1.126156in}{2.039611in}}%
\pgfpathlineto{\pgfqpoint{1.126156in}{2.039611in}}%
\pgfpathclose%
\pgfusepath{stroke}%
\end{pgfscope}%
\begin{pgfscope}%
\pgfpathrectangle{\pgfqpoint{0.688192in}{0.670138in}}{\pgfqpoint{6.200000in}{4.620000in}}%
\pgfusepath{clip}%
\pgfsetbuttcap%
\pgfsetroundjoin%
\pgfsetlinewidth{1.003750pt}%
\definecolor{currentstroke}{rgb}{1.000000,0.000000,0.000000}%
\pgfsetstrokecolor{currentstroke}%
\pgfsetdash{}{0pt}%
\pgfpathmoveto{\pgfqpoint{1.073783in}{1.736453in}}%
\pgfpathcurveto{\pgfqpoint{1.082020in}{1.736453in}}{\pgfqpoint{1.089920in}{1.739725in}}{\pgfqpoint{1.095744in}{1.745549in}}%
\pgfpathcurveto{\pgfqpoint{1.101568in}{1.751373in}}{\pgfqpoint{1.104840in}{1.759273in}}{\pgfqpoint{1.104840in}{1.767510in}}%
\pgfpathcurveto{\pgfqpoint{1.104840in}{1.775746in}}{\pgfqpoint{1.101568in}{1.783646in}}{\pgfqpoint{1.095744in}{1.789470in}}%
\pgfpathcurveto{\pgfqpoint{1.089920in}{1.795294in}}{\pgfqpoint{1.082020in}{1.798566in}}{\pgfqpoint{1.073783in}{1.798566in}}%
\pgfpathcurveto{\pgfqpoint{1.065547in}{1.798566in}}{\pgfqpoint{1.057647in}{1.795294in}}{\pgfqpoint{1.051823in}{1.789470in}}%
\pgfpathcurveto{\pgfqpoint{1.045999in}{1.783646in}}{\pgfqpoint{1.042727in}{1.775746in}}{\pgfqpoint{1.042727in}{1.767510in}}%
\pgfpathcurveto{\pgfqpoint{1.042727in}{1.759273in}}{\pgfqpoint{1.045999in}{1.751373in}}{\pgfqpoint{1.051823in}{1.745549in}}%
\pgfpathcurveto{\pgfqpoint{1.057647in}{1.739725in}}{\pgfqpoint{1.065547in}{1.736453in}}{\pgfqpoint{1.073783in}{1.736453in}}%
\pgfpathlineto{\pgfqpoint{1.073783in}{1.736453in}}%
\pgfpathclose%
\pgfusepath{stroke}%
\end{pgfscope}%
\begin{pgfscope}%
\pgfpathrectangle{\pgfqpoint{0.688192in}{0.670138in}}{\pgfqpoint{6.200000in}{4.620000in}}%
\pgfusepath{clip}%
\pgfsetbuttcap%
\pgfsetroundjoin%
\pgfsetlinewidth{1.003750pt}%
\definecolor{currentstroke}{rgb}{1.000000,0.000000,0.000000}%
\pgfsetstrokecolor{currentstroke}%
\pgfsetdash{}{0pt}%
\pgfpathmoveto{\pgfqpoint{1.071350in}{1.723563in}}%
\pgfpathcurveto{\pgfqpoint{1.079586in}{1.723563in}}{\pgfqpoint{1.087486in}{1.726835in}}{\pgfqpoint{1.093310in}{1.732659in}}%
\pgfpathcurveto{\pgfqpoint{1.099134in}{1.738483in}}{\pgfqpoint{1.102407in}{1.746383in}}{\pgfqpoint{1.102407in}{1.754620in}}%
\pgfpathcurveto{\pgfqpoint{1.102407in}{1.762856in}}{\pgfqpoint{1.099134in}{1.770756in}}{\pgfqpoint{1.093310in}{1.776580in}}%
\pgfpathcurveto{\pgfqpoint{1.087486in}{1.782404in}}{\pgfqpoint{1.079586in}{1.785676in}}{\pgfqpoint{1.071350in}{1.785676in}}%
\pgfpathcurveto{\pgfqpoint{1.063114in}{1.785676in}}{\pgfqpoint{1.055214in}{1.782404in}}{\pgfqpoint{1.049390in}{1.776580in}}%
\pgfpathcurveto{\pgfqpoint{1.043566in}{1.770756in}}{\pgfqpoint{1.040294in}{1.762856in}}{\pgfqpoint{1.040294in}{1.754620in}}%
\pgfpathcurveto{\pgfqpoint{1.040294in}{1.746383in}}{\pgfqpoint{1.043566in}{1.738483in}}{\pgfqpoint{1.049390in}{1.732659in}}%
\pgfpathcurveto{\pgfqpoint{1.055214in}{1.726835in}}{\pgfqpoint{1.063114in}{1.723563in}}{\pgfqpoint{1.071350in}{1.723563in}}%
\pgfpathlineto{\pgfqpoint{1.071350in}{1.723563in}}%
\pgfpathclose%
\pgfusepath{stroke}%
\end{pgfscope}%
\begin{pgfscope}%
\pgfpathrectangle{\pgfqpoint{0.688192in}{0.670138in}}{\pgfqpoint{6.200000in}{4.620000in}}%
\pgfusepath{clip}%
\pgfsetbuttcap%
\pgfsetroundjoin%
\pgfsetlinewidth{1.003750pt}%
\definecolor{currentstroke}{rgb}{1.000000,0.000000,0.000000}%
\pgfsetstrokecolor{currentstroke}%
\pgfsetdash{}{0pt}%
\pgfpathmoveto{\pgfqpoint{1.176638in}{2.187985in}}%
\pgfpathcurveto{\pgfqpoint{1.184874in}{2.187985in}}{\pgfqpoint{1.192774in}{2.191257in}}{\pgfqpoint{1.198598in}{2.197081in}}%
\pgfpathcurveto{\pgfqpoint{1.204422in}{2.202905in}}{\pgfqpoint{1.207694in}{2.210805in}}{\pgfqpoint{1.207694in}{2.219041in}}%
\pgfpathcurveto{\pgfqpoint{1.207694in}{2.227277in}}{\pgfqpoint{1.204422in}{2.235178in}}{\pgfqpoint{1.198598in}{2.241001in}}%
\pgfpathcurveto{\pgfqpoint{1.192774in}{2.246825in}}{\pgfqpoint{1.184874in}{2.250098in}}{\pgfqpoint{1.176638in}{2.250098in}}%
\pgfpathcurveto{\pgfqpoint{1.168401in}{2.250098in}}{\pgfqpoint{1.160501in}{2.246825in}}{\pgfqpoint{1.154677in}{2.241001in}}%
\pgfpathcurveto{\pgfqpoint{1.148853in}{2.235178in}}{\pgfqpoint{1.145581in}{2.227277in}}{\pgfqpoint{1.145581in}{2.219041in}}%
\pgfpathcurveto{\pgfqpoint{1.145581in}{2.210805in}}{\pgfqpoint{1.148853in}{2.202905in}}{\pgfqpoint{1.154677in}{2.197081in}}%
\pgfpathcurveto{\pgfqpoint{1.160501in}{2.191257in}}{\pgfqpoint{1.168401in}{2.187985in}}{\pgfqpoint{1.176638in}{2.187985in}}%
\pgfpathlineto{\pgfqpoint{1.176638in}{2.187985in}}%
\pgfpathclose%
\pgfusepath{stroke}%
\end{pgfscope}%
\begin{pgfscope}%
\pgfpathrectangle{\pgfqpoint{0.688192in}{0.670138in}}{\pgfqpoint{6.200000in}{4.620000in}}%
\pgfusepath{clip}%
\pgfsetbuttcap%
\pgfsetroundjoin%
\pgfsetlinewidth{1.003750pt}%
\definecolor{currentstroke}{rgb}{1.000000,0.000000,0.000000}%
\pgfsetstrokecolor{currentstroke}%
\pgfsetdash{}{0pt}%
\pgfpathmoveto{\pgfqpoint{1.160265in}{1.939369in}}%
\pgfpathcurveto{\pgfqpoint{1.168501in}{1.939369in}}{\pgfqpoint{1.176401in}{1.942641in}}{\pgfqpoint{1.182225in}{1.948465in}}%
\pgfpathcurveto{\pgfqpoint{1.188049in}{1.954289in}}{\pgfqpoint{1.191321in}{1.962189in}}{\pgfqpoint{1.191321in}{1.970426in}}%
\pgfpathcurveto{\pgfqpoint{1.191321in}{1.978662in}}{\pgfqpoint{1.188049in}{1.986562in}}{\pgfqpoint{1.182225in}{1.992386in}}%
\pgfpathcurveto{\pgfqpoint{1.176401in}{1.998210in}}{\pgfqpoint{1.168501in}{2.001482in}}{\pgfqpoint{1.160265in}{2.001482in}}%
\pgfpathcurveto{\pgfqpoint{1.152029in}{2.001482in}}{\pgfqpoint{1.144129in}{1.998210in}}{\pgfqpoint{1.138305in}{1.992386in}}%
\pgfpathcurveto{\pgfqpoint{1.132481in}{1.986562in}}{\pgfqpoint{1.129208in}{1.978662in}}{\pgfqpoint{1.129208in}{1.970426in}}%
\pgfpathcurveto{\pgfqpoint{1.129208in}{1.962189in}}{\pgfqpoint{1.132481in}{1.954289in}}{\pgfqpoint{1.138305in}{1.948465in}}%
\pgfpathcurveto{\pgfqpoint{1.144129in}{1.942641in}}{\pgfqpoint{1.152029in}{1.939369in}}{\pgfqpoint{1.160265in}{1.939369in}}%
\pgfpathlineto{\pgfqpoint{1.160265in}{1.939369in}}%
\pgfpathclose%
\pgfusepath{stroke}%
\end{pgfscope}%
\begin{pgfscope}%
\pgfpathrectangle{\pgfqpoint{0.688192in}{0.670138in}}{\pgfqpoint{6.200000in}{4.620000in}}%
\pgfusepath{clip}%
\pgfsetbuttcap%
\pgfsetroundjoin%
\pgfsetlinewidth{1.003750pt}%
\definecolor{currentstroke}{rgb}{1.000000,0.000000,0.000000}%
\pgfsetstrokecolor{currentstroke}%
\pgfsetdash{}{0pt}%
\pgfpathmoveto{\pgfqpoint{1.160053in}{1.935902in}}%
\pgfpathcurveto{\pgfqpoint{1.168289in}{1.935902in}}{\pgfqpoint{1.176189in}{1.939174in}}{\pgfqpoint{1.182013in}{1.944998in}}%
\pgfpathcurveto{\pgfqpoint{1.187837in}{1.950822in}}{\pgfqpoint{1.191109in}{1.958722in}}{\pgfqpoint{1.191109in}{1.966959in}}%
\pgfpathcurveto{\pgfqpoint{1.191109in}{1.975195in}}{\pgfqpoint{1.187837in}{1.983095in}}{\pgfqpoint{1.182013in}{1.988919in}}%
\pgfpathcurveto{\pgfqpoint{1.176189in}{1.994743in}}{\pgfqpoint{1.168289in}{1.998015in}}{\pgfqpoint{1.160053in}{1.998015in}}%
\pgfpathcurveto{\pgfqpoint{1.151817in}{1.998015in}}{\pgfqpoint{1.143916in}{1.994743in}}{\pgfqpoint{1.138093in}{1.988919in}}%
\pgfpathcurveto{\pgfqpoint{1.132269in}{1.983095in}}{\pgfqpoint{1.128996in}{1.975195in}}{\pgfqpoint{1.128996in}{1.966959in}}%
\pgfpathcurveto{\pgfqpoint{1.128996in}{1.958722in}}{\pgfqpoint{1.132269in}{1.950822in}}{\pgfqpoint{1.138093in}{1.944998in}}%
\pgfpathcurveto{\pgfqpoint{1.143916in}{1.939174in}}{\pgfqpoint{1.151817in}{1.935902in}}{\pgfqpoint{1.160053in}{1.935902in}}%
\pgfpathlineto{\pgfqpoint{1.160053in}{1.935902in}}%
\pgfpathclose%
\pgfusepath{stroke}%
\end{pgfscope}%
\begin{pgfscope}%
\pgfpathrectangle{\pgfqpoint{0.688192in}{0.670138in}}{\pgfqpoint{6.200000in}{4.620000in}}%
\pgfusepath{clip}%
\pgfsetbuttcap%
\pgfsetroundjoin%
\pgfsetlinewidth{1.003750pt}%
\definecolor{currentstroke}{rgb}{1.000000,0.000000,0.000000}%
\pgfsetstrokecolor{currentstroke}%
\pgfsetdash{}{0pt}%
\pgfpathmoveto{\pgfqpoint{1.157986in}{1.610198in}}%
\pgfpathcurveto{\pgfqpoint{1.166223in}{1.610198in}}{\pgfqpoint{1.174123in}{1.613470in}}{\pgfqpoint{1.179947in}{1.619294in}}%
\pgfpathcurveto{\pgfqpoint{1.185771in}{1.625118in}}{\pgfqpoint{1.189043in}{1.633018in}}{\pgfqpoint{1.189043in}{1.641254in}}%
\pgfpathcurveto{\pgfqpoint{1.189043in}{1.649491in}}{\pgfqpoint{1.185771in}{1.657391in}}{\pgfqpoint{1.179947in}{1.663215in}}%
\pgfpathcurveto{\pgfqpoint{1.174123in}{1.669038in}}{\pgfqpoint{1.166223in}{1.672311in}}{\pgfqpoint{1.157986in}{1.672311in}}%
\pgfpathcurveto{\pgfqpoint{1.149750in}{1.672311in}}{\pgfqpoint{1.141850in}{1.669038in}}{\pgfqpoint{1.136026in}{1.663215in}}%
\pgfpathcurveto{\pgfqpoint{1.130202in}{1.657391in}}{\pgfqpoint{1.126930in}{1.649491in}}{\pgfqpoint{1.126930in}{1.641254in}}%
\pgfpathcurveto{\pgfqpoint{1.126930in}{1.633018in}}{\pgfqpoint{1.130202in}{1.625118in}}{\pgfqpoint{1.136026in}{1.619294in}}%
\pgfpathcurveto{\pgfqpoint{1.141850in}{1.613470in}}{\pgfqpoint{1.149750in}{1.610198in}}{\pgfqpoint{1.157986in}{1.610198in}}%
\pgfpathlineto{\pgfqpoint{1.157986in}{1.610198in}}%
\pgfpathclose%
\pgfusepath{stroke}%
\end{pgfscope}%
\begin{pgfscope}%
\pgfpathrectangle{\pgfqpoint{0.688192in}{0.670138in}}{\pgfqpoint{6.200000in}{4.620000in}}%
\pgfusepath{clip}%
\pgfsetbuttcap%
\pgfsetroundjoin%
\pgfsetlinewidth{1.003750pt}%
\definecolor{currentstroke}{rgb}{1.000000,0.000000,0.000000}%
\pgfsetstrokecolor{currentstroke}%
\pgfsetdash{}{0pt}%
\pgfpathmoveto{\pgfqpoint{1.155798in}{1.612752in}}%
\pgfpathcurveto{\pgfqpoint{1.164035in}{1.612752in}}{\pgfqpoint{1.171935in}{1.616024in}}{\pgfqpoint{1.177759in}{1.621848in}}%
\pgfpathcurveto{\pgfqpoint{1.183582in}{1.627672in}}{\pgfqpoint{1.186855in}{1.635572in}}{\pgfqpoint{1.186855in}{1.643808in}}%
\pgfpathcurveto{\pgfqpoint{1.186855in}{1.652044in}}{\pgfqpoint{1.183582in}{1.659945in}}{\pgfqpoint{1.177759in}{1.665768in}}%
\pgfpathcurveto{\pgfqpoint{1.171935in}{1.671592in}}{\pgfqpoint{1.164035in}{1.674865in}}{\pgfqpoint{1.155798in}{1.674865in}}%
\pgfpathcurveto{\pgfqpoint{1.147562in}{1.674865in}}{\pgfqpoint{1.139662in}{1.671592in}}{\pgfqpoint{1.133838in}{1.665768in}}%
\pgfpathcurveto{\pgfqpoint{1.128014in}{1.659945in}}{\pgfqpoint{1.124742in}{1.652044in}}{\pgfqpoint{1.124742in}{1.643808in}}%
\pgfpathcurveto{\pgfqpoint{1.124742in}{1.635572in}}{\pgfqpoint{1.128014in}{1.627672in}}{\pgfqpoint{1.133838in}{1.621848in}}%
\pgfpathcurveto{\pgfqpoint{1.139662in}{1.616024in}}{\pgfqpoint{1.147562in}{1.612752in}}{\pgfqpoint{1.155798in}{1.612752in}}%
\pgfpathlineto{\pgfqpoint{1.155798in}{1.612752in}}%
\pgfpathclose%
\pgfusepath{stroke}%
\end{pgfscope}%
\begin{pgfscope}%
\pgfpathrectangle{\pgfqpoint{0.688192in}{0.670138in}}{\pgfqpoint{6.200000in}{4.620000in}}%
\pgfusepath{clip}%
\pgfsetbuttcap%
\pgfsetroundjoin%
\pgfsetlinewidth{1.003750pt}%
\definecolor{currentstroke}{rgb}{1.000000,0.000000,0.000000}%
\pgfsetstrokecolor{currentstroke}%
\pgfsetdash{}{0pt}%
\pgfpathmoveto{\pgfqpoint{1.295347in}{1.895177in}}%
\pgfpathcurveto{\pgfqpoint{1.303583in}{1.895177in}}{\pgfqpoint{1.311483in}{1.898449in}}{\pgfqpoint{1.317307in}{1.904273in}}%
\pgfpathcurveto{\pgfqpoint{1.323131in}{1.910097in}}{\pgfqpoint{1.326403in}{1.917997in}}{\pgfqpoint{1.326403in}{1.926233in}}%
\pgfpathcurveto{\pgfqpoint{1.326403in}{1.934470in}}{\pgfqpoint{1.323131in}{1.942370in}}{\pgfqpoint{1.317307in}{1.948193in}}%
\pgfpathcurveto{\pgfqpoint{1.311483in}{1.954017in}}{\pgfqpoint{1.303583in}{1.957290in}}{\pgfqpoint{1.295347in}{1.957290in}}%
\pgfpathcurveto{\pgfqpoint{1.287111in}{1.957290in}}{\pgfqpoint{1.279211in}{1.954017in}}{\pgfqpoint{1.273387in}{1.948193in}}%
\pgfpathcurveto{\pgfqpoint{1.267563in}{1.942370in}}{\pgfqpoint{1.264290in}{1.934470in}}{\pgfqpoint{1.264290in}{1.926233in}}%
\pgfpathcurveto{\pgfqpoint{1.264290in}{1.917997in}}{\pgfqpoint{1.267563in}{1.910097in}}{\pgfqpoint{1.273387in}{1.904273in}}%
\pgfpathcurveto{\pgfqpoint{1.279211in}{1.898449in}}{\pgfqpoint{1.287111in}{1.895177in}}{\pgfqpoint{1.295347in}{1.895177in}}%
\pgfpathlineto{\pgfqpoint{1.295347in}{1.895177in}}%
\pgfpathclose%
\pgfusepath{stroke}%
\end{pgfscope}%
\begin{pgfscope}%
\pgfpathrectangle{\pgfqpoint{0.688192in}{0.670138in}}{\pgfqpoint{6.200000in}{4.620000in}}%
\pgfusepath{clip}%
\pgfsetbuttcap%
\pgfsetroundjoin%
\pgfsetlinewidth{1.003750pt}%
\definecolor{currentstroke}{rgb}{1.000000,0.000000,0.000000}%
\pgfsetstrokecolor{currentstroke}%
\pgfsetdash{}{0pt}%
\pgfpathmoveto{\pgfqpoint{1.160535in}{1.595409in}}%
\pgfpathcurveto{\pgfqpoint{1.168772in}{1.595409in}}{\pgfqpoint{1.176672in}{1.598682in}}{\pgfqpoint{1.182496in}{1.604506in}}%
\pgfpathcurveto{\pgfqpoint{1.188320in}{1.610330in}}{\pgfqpoint{1.191592in}{1.618230in}}{\pgfqpoint{1.191592in}{1.626466in}}%
\pgfpathcurveto{\pgfqpoint{1.191592in}{1.634702in}}{\pgfqpoint{1.188320in}{1.642602in}}{\pgfqpoint{1.182496in}{1.648426in}}%
\pgfpathcurveto{\pgfqpoint{1.176672in}{1.654250in}}{\pgfqpoint{1.168772in}{1.657522in}}{\pgfqpoint{1.160535in}{1.657522in}}%
\pgfpathcurveto{\pgfqpoint{1.152299in}{1.657522in}}{\pgfqpoint{1.144399in}{1.654250in}}{\pgfqpoint{1.138575in}{1.648426in}}%
\pgfpathcurveto{\pgfqpoint{1.132751in}{1.642602in}}{\pgfqpoint{1.129479in}{1.634702in}}{\pgfqpoint{1.129479in}{1.626466in}}%
\pgfpathcurveto{\pgfqpoint{1.129479in}{1.618230in}}{\pgfqpoint{1.132751in}{1.610330in}}{\pgfqpoint{1.138575in}{1.604506in}}%
\pgfpathcurveto{\pgfqpoint{1.144399in}{1.598682in}}{\pgfqpoint{1.152299in}{1.595409in}}{\pgfqpoint{1.160535in}{1.595409in}}%
\pgfpathlineto{\pgfqpoint{1.160535in}{1.595409in}}%
\pgfpathclose%
\pgfusepath{stroke}%
\end{pgfscope}%
\begin{pgfscope}%
\pgfpathrectangle{\pgfqpoint{0.688192in}{0.670138in}}{\pgfqpoint{6.200000in}{4.620000in}}%
\pgfusepath{clip}%
\pgfsetbuttcap%
\pgfsetroundjoin%
\pgfsetlinewidth{1.003750pt}%
\definecolor{currentstroke}{rgb}{1.000000,0.000000,0.000000}%
\pgfsetstrokecolor{currentstroke}%
\pgfsetdash{}{0pt}%
\pgfpathmoveto{\pgfqpoint{1.265571in}{1.691299in}}%
\pgfpathcurveto{\pgfqpoint{1.273807in}{1.691299in}}{\pgfqpoint{1.281707in}{1.694572in}}{\pgfqpoint{1.287531in}{1.700396in}}%
\pgfpathcurveto{\pgfqpoint{1.293355in}{1.706220in}}{\pgfqpoint{1.296627in}{1.714120in}}{\pgfqpoint{1.296627in}{1.722356in}}%
\pgfpathcurveto{\pgfqpoint{1.296627in}{1.730592in}}{\pgfqpoint{1.293355in}{1.738492in}}{\pgfqpoint{1.287531in}{1.744316in}}%
\pgfpathcurveto{\pgfqpoint{1.281707in}{1.750140in}}{\pgfqpoint{1.273807in}{1.753412in}}{\pgfqpoint{1.265571in}{1.753412in}}%
\pgfpathcurveto{\pgfqpoint{1.257335in}{1.753412in}}{\pgfqpoint{1.249435in}{1.750140in}}{\pgfqpoint{1.243611in}{1.744316in}}%
\pgfpathcurveto{\pgfqpoint{1.237787in}{1.738492in}}{\pgfqpoint{1.234514in}{1.730592in}}{\pgfqpoint{1.234514in}{1.722356in}}%
\pgfpathcurveto{\pgfqpoint{1.234514in}{1.714120in}}{\pgfqpoint{1.237787in}{1.706220in}}{\pgfqpoint{1.243611in}{1.700396in}}%
\pgfpathcurveto{\pgfqpoint{1.249435in}{1.694572in}}{\pgfqpoint{1.257335in}{1.691299in}}{\pgfqpoint{1.265571in}{1.691299in}}%
\pgfpathlineto{\pgfqpoint{1.265571in}{1.691299in}}%
\pgfpathclose%
\pgfusepath{stroke}%
\end{pgfscope}%
\begin{pgfscope}%
\pgfpathrectangle{\pgfqpoint{0.688192in}{0.670138in}}{\pgfqpoint{6.200000in}{4.620000in}}%
\pgfusepath{clip}%
\pgfsetbuttcap%
\pgfsetroundjoin%
\pgfsetlinewidth{1.003750pt}%
\definecolor{currentstroke}{rgb}{1.000000,0.000000,0.000000}%
\pgfsetstrokecolor{currentstroke}%
\pgfsetdash{}{0pt}%
\pgfpathmoveto{\pgfqpoint{1.297728in}{1.636949in}}%
\pgfpathcurveto{\pgfqpoint{1.305964in}{1.636949in}}{\pgfqpoint{1.313864in}{1.640222in}}{\pgfqpoint{1.319688in}{1.646045in}}%
\pgfpathcurveto{\pgfqpoint{1.325512in}{1.651869in}}{\pgfqpoint{1.328784in}{1.659769in}}{\pgfqpoint{1.328784in}{1.668006in}}%
\pgfpathcurveto{\pgfqpoint{1.328784in}{1.676242in}}{\pgfqpoint{1.325512in}{1.684142in}}{\pgfqpoint{1.319688in}{1.689966in}}%
\pgfpathcurveto{\pgfqpoint{1.313864in}{1.695790in}}{\pgfqpoint{1.305964in}{1.699062in}}{\pgfqpoint{1.297728in}{1.699062in}}%
\pgfpathcurveto{\pgfqpoint{1.289491in}{1.699062in}}{\pgfqpoint{1.281591in}{1.695790in}}{\pgfqpoint{1.275767in}{1.689966in}}%
\pgfpathcurveto{\pgfqpoint{1.269943in}{1.684142in}}{\pgfqpoint{1.266671in}{1.676242in}}{\pgfqpoint{1.266671in}{1.668006in}}%
\pgfpathcurveto{\pgfqpoint{1.266671in}{1.659769in}}{\pgfqpoint{1.269943in}{1.651869in}}{\pgfqpoint{1.275767in}{1.646045in}}%
\pgfpathcurveto{\pgfqpoint{1.281591in}{1.640222in}}{\pgfqpoint{1.289491in}{1.636949in}}{\pgfqpoint{1.297728in}{1.636949in}}%
\pgfpathlineto{\pgfqpoint{1.297728in}{1.636949in}}%
\pgfpathclose%
\pgfusepath{stroke}%
\end{pgfscope}%
\begin{pgfscope}%
\pgfpathrectangle{\pgfqpoint{0.688192in}{0.670138in}}{\pgfqpoint{6.200000in}{4.620000in}}%
\pgfusepath{clip}%
\pgfsetbuttcap%
\pgfsetroundjoin%
\pgfsetlinewidth{1.003750pt}%
\definecolor{currentstroke}{rgb}{1.000000,0.000000,0.000000}%
\pgfsetstrokecolor{currentstroke}%
\pgfsetdash{}{0pt}%
\pgfpathmoveto{\pgfqpoint{1.147178in}{1.734556in}}%
\pgfpathcurveto{\pgfqpoint{1.155415in}{1.734556in}}{\pgfqpoint{1.163315in}{1.737828in}}{\pgfqpoint{1.169139in}{1.743652in}}%
\pgfpathcurveto{\pgfqpoint{1.174962in}{1.749476in}}{\pgfqpoint{1.178235in}{1.757376in}}{\pgfqpoint{1.178235in}{1.765612in}}%
\pgfpathcurveto{\pgfqpoint{1.178235in}{1.773848in}}{\pgfqpoint{1.174962in}{1.781749in}}{\pgfqpoint{1.169139in}{1.787572in}}%
\pgfpathcurveto{\pgfqpoint{1.163315in}{1.793396in}}{\pgfqpoint{1.155415in}{1.796669in}}{\pgfqpoint{1.147178in}{1.796669in}}%
\pgfpathcurveto{\pgfqpoint{1.138942in}{1.796669in}}{\pgfqpoint{1.131042in}{1.793396in}}{\pgfqpoint{1.125218in}{1.787572in}}%
\pgfpathcurveto{\pgfqpoint{1.119394in}{1.781749in}}{\pgfqpoint{1.116122in}{1.773848in}}{\pgfqpoint{1.116122in}{1.765612in}}%
\pgfpathcurveto{\pgfqpoint{1.116122in}{1.757376in}}{\pgfqpoint{1.119394in}{1.749476in}}{\pgfqpoint{1.125218in}{1.743652in}}%
\pgfpathcurveto{\pgfqpoint{1.131042in}{1.737828in}}{\pgfqpoint{1.138942in}{1.734556in}}{\pgfqpoint{1.147178in}{1.734556in}}%
\pgfpathlineto{\pgfqpoint{1.147178in}{1.734556in}}%
\pgfpathclose%
\pgfusepath{stroke}%
\end{pgfscope}%
\begin{pgfscope}%
\pgfpathrectangle{\pgfqpoint{0.688192in}{0.670138in}}{\pgfqpoint{6.200000in}{4.620000in}}%
\pgfusepath{clip}%
\pgfsetbuttcap%
\pgfsetroundjoin%
\pgfsetlinewidth{1.003750pt}%
\definecolor{currentstroke}{rgb}{1.000000,0.000000,0.000000}%
\pgfsetstrokecolor{currentstroke}%
\pgfsetdash{}{0pt}%
\pgfpathmoveto{\pgfqpoint{1.240097in}{1.549585in}}%
\pgfpathcurveto{\pgfqpoint{1.248334in}{1.549585in}}{\pgfqpoint{1.256234in}{1.552857in}}{\pgfqpoint{1.262058in}{1.558681in}}%
\pgfpathcurveto{\pgfqpoint{1.267881in}{1.564505in}}{\pgfqpoint{1.271154in}{1.572405in}}{\pgfqpoint{1.271154in}{1.580642in}}%
\pgfpathcurveto{\pgfqpoint{1.271154in}{1.588878in}}{\pgfqpoint{1.267881in}{1.596778in}}{\pgfqpoint{1.262058in}{1.602602in}}%
\pgfpathcurveto{\pgfqpoint{1.256234in}{1.608426in}}{\pgfqpoint{1.248334in}{1.611698in}}{\pgfqpoint{1.240097in}{1.611698in}}%
\pgfpathcurveto{\pgfqpoint{1.231861in}{1.611698in}}{\pgfqpoint{1.223961in}{1.608426in}}{\pgfqpoint{1.218137in}{1.602602in}}%
\pgfpathcurveto{\pgfqpoint{1.212313in}{1.596778in}}{\pgfqpoint{1.209041in}{1.588878in}}{\pgfqpoint{1.209041in}{1.580642in}}%
\pgfpathcurveto{\pgfqpoint{1.209041in}{1.572405in}}{\pgfqpoint{1.212313in}{1.564505in}}{\pgfqpoint{1.218137in}{1.558681in}}%
\pgfpathcurveto{\pgfqpoint{1.223961in}{1.552857in}}{\pgfqpoint{1.231861in}{1.549585in}}{\pgfqpoint{1.240097in}{1.549585in}}%
\pgfpathlineto{\pgfqpoint{1.240097in}{1.549585in}}%
\pgfpathclose%
\pgfusepath{stroke}%
\end{pgfscope}%
\begin{pgfscope}%
\pgfpathrectangle{\pgfqpoint{0.688192in}{0.670138in}}{\pgfqpoint{6.200000in}{4.620000in}}%
\pgfusepath{clip}%
\pgfsetbuttcap%
\pgfsetroundjoin%
\pgfsetlinewidth{1.003750pt}%
\definecolor{currentstroke}{rgb}{1.000000,0.000000,0.000000}%
\pgfsetstrokecolor{currentstroke}%
\pgfsetdash{}{0pt}%
\pgfpathmoveto{\pgfqpoint{1.275017in}{1.474367in}}%
\pgfpathcurveto{\pgfqpoint{1.283253in}{1.474367in}}{\pgfqpoint{1.291153in}{1.477639in}}{\pgfqpoint{1.296977in}{1.483463in}}%
\pgfpathcurveto{\pgfqpoint{1.302801in}{1.489287in}}{\pgfqpoint{1.306073in}{1.497187in}}{\pgfqpoint{1.306073in}{1.505423in}}%
\pgfpathcurveto{\pgfqpoint{1.306073in}{1.513660in}}{\pgfqpoint{1.302801in}{1.521560in}}{\pgfqpoint{1.296977in}{1.527384in}}%
\pgfpathcurveto{\pgfqpoint{1.291153in}{1.533208in}}{\pgfqpoint{1.283253in}{1.536480in}}{\pgfqpoint{1.275017in}{1.536480in}}%
\pgfpathcurveto{\pgfqpoint{1.266780in}{1.536480in}}{\pgfqpoint{1.258880in}{1.533208in}}{\pgfqpoint{1.253056in}{1.527384in}}%
\pgfpathcurveto{\pgfqpoint{1.247232in}{1.521560in}}{\pgfqpoint{1.243960in}{1.513660in}}{\pgfqpoint{1.243960in}{1.505423in}}%
\pgfpathcurveto{\pgfqpoint{1.243960in}{1.497187in}}{\pgfqpoint{1.247232in}{1.489287in}}{\pgfqpoint{1.253056in}{1.483463in}}%
\pgfpathcurveto{\pgfqpoint{1.258880in}{1.477639in}}{\pgfqpoint{1.266780in}{1.474367in}}{\pgfqpoint{1.275017in}{1.474367in}}%
\pgfpathlineto{\pgfqpoint{1.275017in}{1.474367in}}%
\pgfpathclose%
\pgfusepath{stroke}%
\end{pgfscope}%
\begin{pgfscope}%
\pgfpathrectangle{\pgfqpoint{0.688192in}{0.670138in}}{\pgfqpoint{6.200000in}{4.620000in}}%
\pgfusepath{clip}%
\pgfsetbuttcap%
\pgfsetroundjoin%
\pgfsetlinewidth{1.003750pt}%
\definecolor{currentstroke}{rgb}{1.000000,0.000000,0.000000}%
\pgfsetstrokecolor{currentstroke}%
\pgfsetdash{}{0pt}%
\pgfpathmoveto{\pgfqpoint{1.274070in}{1.506532in}}%
\pgfpathcurveto{\pgfqpoint{1.282306in}{1.506532in}}{\pgfqpoint{1.290206in}{1.509804in}}{\pgfqpoint{1.296030in}{1.515628in}}%
\pgfpathcurveto{\pgfqpoint{1.301854in}{1.521452in}}{\pgfqpoint{1.305126in}{1.529352in}}{\pgfqpoint{1.305126in}{1.537589in}}%
\pgfpathcurveto{\pgfqpoint{1.305126in}{1.545825in}}{\pgfqpoint{1.301854in}{1.553725in}}{\pgfqpoint{1.296030in}{1.559549in}}%
\pgfpathcurveto{\pgfqpoint{1.290206in}{1.565373in}}{\pgfqpoint{1.282306in}{1.568645in}}{\pgfqpoint{1.274070in}{1.568645in}}%
\pgfpathcurveto{\pgfqpoint{1.265833in}{1.568645in}}{\pgfqpoint{1.257933in}{1.565373in}}{\pgfqpoint{1.252109in}{1.559549in}}%
\pgfpathcurveto{\pgfqpoint{1.246285in}{1.553725in}}{\pgfqpoint{1.243013in}{1.545825in}}{\pgfqpoint{1.243013in}{1.537589in}}%
\pgfpathcurveto{\pgfqpoint{1.243013in}{1.529352in}}{\pgfqpoint{1.246285in}{1.521452in}}{\pgfqpoint{1.252109in}{1.515628in}}%
\pgfpathcurveto{\pgfqpoint{1.257933in}{1.509804in}}{\pgfqpoint{1.265833in}{1.506532in}}{\pgfqpoint{1.274070in}{1.506532in}}%
\pgfpathlineto{\pgfqpoint{1.274070in}{1.506532in}}%
\pgfpathclose%
\pgfusepath{stroke}%
\end{pgfscope}%
\begin{pgfscope}%
\pgfpathrectangle{\pgfqpoint{0.688192in}{0.670138in}}{\pgfqpoint{6.200000in}{4.620000in}}%
\pgfusepath{clip}%
\pgfsetbuttcap%
\pgfsetroundjoin%
\pgfsetlinewidth{1.003750pt}%
\definecolor{currentstroke}{rgb}{1.000000,0.000000,0.000000}%
\pgfsetstrokecolor{currentstroke}%
\pgfsetdash{}{0pt}%
\pgfpathmoveto{\pgfqpoint{1.147538in}{1.722969in}}%
\pgfpathcurveto{\pgfqpoint{1.155775in}{1.722969in}}{\pgfqpoint{1.163675in}{1.726242in}}{\pgfqpoint{1.169499in}{1.732066in}}%
\pgfpathcurveto{\pgfqpoint{1.175323in}{1.737890in}}{\pgfqpoint{1.178595in}{1.745790in}}{\pgfqpoint{1.178595in}{1.754026in}}%
\pgfpathcurveto{\pgfqpoint{1.178595in}{1.762262in}}{\pgfqpoint{1.175323in}{1.770162in}}{\pgfqpoint{1.169499in}{1.775986in}}%
\pgfpathcurveto{\pgfqpoint{1.163675in}{1.781810in}}{\pgfqpoint{1.155775in}{1.785082in}}{\pgfqpoint{1.147538in}{1.785082in}}%
\pgfpathcurveto{\pgfqpoint{1.139302in}{1.785082in}}{\pgfqpoint{1.131402in}{1.781810in}}{\pgfqpoint{1.125578in}{1.775986in}}%
\pgfpathcurveto{\pgfqpoint{1.119754in}{1.770162in}}{\pgfqpoint{1.116482in}{1.762262in}}{\pgfqpoint{1.116482in}{1.754026in}}%
\pgfpathcurveto{\pgfqpoint{1.116482in}{1.745790in}}{\pgfqpoint{1.119754in}{1.737890in}}{\pgfqpoint{1.125578in}{1.732066in}}%
\pgfpathcurveto{\pgfqpoint{1.131402in}{1.726242in}}{\pgfqpoint{1.139302in}{1.722969in}}{\pgfqpoint{1.147538in}{1.722969in}}%
\pgfpathlineto{\pgfqpoint{1.147538in}{1.722969in}}%
\pgfpathclose%
\pgfusepath{stroke}%
\end{pgfscope}%
\begin{pgfscope}%
\pgfpathrectangle{\pgfqpoint{0.688192in}{0.670138in}}{\pgfqpoint{6.200000in}{4.620000in}}%
\pgfusepath{clip}%
\pgfsetbuttcap%
\pgfsetroundjoin%
\pgfsetlinewidth{1.003750pt}%
\definecolor{currentstroke}{rgb}{1.000000,0.000000,0.000000}%
\pgfsetstrokecolor{currentstroke}%
\pgfsetdash{}{0pt}%
\pgfpathmoveto{\pgfqpoint{1.194960in}{1.895058in}}%
\pgfpathcurveto{\pgfqpoint{1.203197in}{1.895058in}}{\pgfqpoint{1.211097in}{1.898330in}}{\pgfqpoint{1.216921in}{1.904154in}}%
\pgfpathcurveto{\pgfqpoint{1.222745in}{1.909978in}}{\pgfqpoint{1.226017in}{1.917878in}}{\pgfqpoint{1.226017in}{1.926114in}}%
\pgfpathcurveto{\pgfqpoint{1.226017in}{1.934351in}}{\pgfqpoint{1.222745in}{1.942251in}}{\pgfqpoint{1.216921in}{1.948075in}}%
\pgfpathcurveto{\pgfqpoint{1.211097in}{1.953898in}}{\pgfqpoint{1.203197in}{1.957171in}}{\pgfqpoint{1.194960in}{1.957171in}}%
\pgfpathcurveto{\pgfqpoint{1.186724in}{1.957171in}}{\pgfqpoint{1.178824in}{1.953898in}}{\pgfqpoint{1.173000in}{1.948075in}}%
\pgfpathcurveto{\pgfqpoint{1.167176in}{1.942251in}}{\pgfqpoint{1.163904in}{1.934351in}}{\pgfqpoint{1.163904in}{1.926114in}}%
\pgfpathcurveto{\pgfqpoint{1.163904in}{1.917878in}}{\pgfqpoint{1.167176in}{1.909978in}}{\pgfqpoint{1.173000in}{1.904154in}}%
\pgfpathcurveto{\pgfqpoint{1.178824in}{1.898330in}}{\pgfqpoint{1.186724in}{1.895058in}}{\pgfqpoint{1.194960in}{1.895058in}}%
\pgfpathlineto{\pgfqpoint{1.194960in}{1.895058in}}%
\pgfpathclose%
\pgfusepath{stroke}%
\end{pgfscope}%
\begin{pgfscope}%
\pgfpathrectangle{\pgfqpoint{0.688192in}{0.670138in}}{\pgfqpoint{6.200000in}{4.620000in}}%
\pgfusepath{clip}%
\pgfsetbuttcap%
\pgfsetroundjoin%
\pgfsetlinewidth{1.003750pt}%
\definecolor{currentstroke}{rgb}{1.000000,0.000000,0.000000}%
\pgfsetstrokecolor{currentstroke}%
\pgfsetdash{}{0pt}%
\pgfpathmoveto{\pgfqpoint{1.164778in}{1.905975in}}%
\pgfpathcurveto{\pgfqpoint{1.173015in}{1.905975in}}{\pgfqpoint{1.180915in}{1.909248in}}{\pgfqpoint{1.186738in}{1.915072in}}%
\pgfpathcurveto{\pgfqpoint{1.192562in}{1.920896in}}{\pgfqpoint{1.195835in}{1.928796in}}{\pgfqpoint{1.195835in}{1.937032in}}%
\pgfpathcurveto{\pgfqpoint{1.195835in}{1.945268in}}{\pgfqpoint{1.192562in}{1.953168in}}{\pgfqpoint{1.186738in}{1.958992in}}%
\pgfpathcurveto{\pgfqpoint{1.180915in}{1.964816in}}{\pgfqpoint{1.173015in}{1.968088in}}{\pgfqpoint{1.164778in}{1.968088in}}%
\pgfpathcurveto{\pgfqpoint{1.156542in}{1.968088in}}{\pgfqpoint{1.148642in}{1.964816in}}{\pgfqpoint{1.142818in}{1.958992in}}%
\pgfpathcurveto{\pgfqpoint{1.136994in}{1.953168in}}{\pgfqpoint{1.133722in}{1.945268in}}{\pgfqpoint{1.133722in}{1.937032in}}%
\pgfpathcurveto{\pgfqpoint{1.133722in}{1.928796in}}{\pgfqpoint{1.136994in}{1.920896in}}{\pgfqpoint{1.142818in}{1.915072in}}%
\pgfpathcurveto{\pgfqpoint{1.148642in}{1.909248in}}{\pgfqpoint{1.156542in}{1.905975in}}{\pgfqpoint{1.164778in}{1.905975in}}%
\pgfpathlineto{\pgfqpoint{1.164778in}{1.905975in}}%
\pgfpathclose%
\pgfusepath{stroke}%
\end{pgfscope}%
\begin{pgfscope}%
\pgfpathrectangle{\pgfqpoint{0.688192in}{0.670138in}}{\pgfqpoint{6.200000in}{4.620000in}}%
\pgfusepath{clip}%
\pgfsetbuttcap%
\pgfsetroundjoin%
\pgfsetlinewidth{1.003750pt}%
\definecolor{currentstroke}{rgb}{1.000000,0.000000,0.000000}%
\pgfsetstrokecolor{currentstroke}%
\pgfsetdash{}{0pt}%
\pgfpathmoveto{\pgfqpoint{1.143899in}{1.731794in}}%
\pgfpathcurveto{\pgfqpoint{1.152135in}{1.731794in}}{\pgfqpoint{1.160035in}{1.735067in}}{\pgfqpoint{1.165859in}{1.740890in}}%
\pgfpathcurveto{\pgfqpoint{1.171683in}{1.746714in}}{\pgfqpoint{1.174956in}{1.754614in}}{\pgfqpoint{1.174956in}{1.762851in}}%
\pgfpathcurveto{\pgfqpoint{1.174956in}{1.771087in}}{\pgfqpoint{1.171683in}{1.778987in}}{\pgfqpoint{1.165859in}{1.784811in}}%
\pgfpathcurveto{\pgfqpoint{1.160035in}{1.790635in}}{\pgfqpoint{1.152135in}{1.793907in}}{\pgfqpoint{1.143899in}{1.793907in}}%
\pgfpathcurveto{\pgfqpoint{1.135663in}{1.793907in}}{\pgfqpoint{1.127763in}{1.790635in}}{\pgfqpoint{1.121939in}{1.784811in}}%
\pgfpathcurveto{\pgfqpoint{1.116115in}{1.778987in}}{\pgfqpoint{1.112843in}{1.771087in}}{\pgfqpoint{1.112843in}{1.762851in}}%
\pgfpathcurveto{\pgfqpoint{1.112843in}{1.754614in}}{\pgfqpoint{1.116115in}{1.746714in}}{\pgfqpoint{1.121939in}{1.740890in}}%
\pgfpathcurveto{\pgfqpoint{1.127763in}{1.735067in}}{\pgfqpoint{1.135663in}{1.731794in}}{\pgfqpoint{1.143899in}{1.731794in}}%
\pgfpathlineto{\pgfqpoint{1.143899in}{1.731794in}}%
\pgfpathclose%
\pgfusepath{stroke}%
\end{pgfscope}%
\begin{pgfscope}%
\pgfpathrectangle{\pgfqpoint{0.688192in}{0.670138in}}{\pgfqpoint{6.200000in}{4.620000in}}%
\pgfusepath{clip}%
\pgfsetbuttcap%
\pgfsetroundjoin%
\pgfsetlinewidth{1.003750pt}%
\definecolor{currentstroke}{rgb}{1.000000,0.000000,0.000000}%
\pgfsetstrokecolor{currentstroke}%
\pgfsetdash{}{0pt}%
\pgfpathmoveto{\pgfqpoint{0.688192in}{1.225706in}}%
\pgfpathcurveto{\pgfqpoint{0.696428in}{1.225706in}}{\pgfqpoint{0.704328in}{1.228978in}}{\pgfqpoint{0.710152in}{1.234802in}}%
\pgfpathcurveto{\pgfqpoint{0.715976in}{1.240626in}}{\pgfqpoint{0.719248in}{1.248526in}}{\pgfqpoint{0.719248in}{1.256762in}}%
\pgfpathcurveto{\pgfqpoint{0.719248in}{1.264998in}}{\pgfqpoint{0.715976in}{1.272898in}}{\pgfqpoint{0.710152in}{1.278722in}}%
\pgfpathcurveto{\pgfqpoint{0.704328in}{1.284546in}}{\pgfqpoint{0.696428in}{1.287819in}}{\pgfqpoint{0.688192in}{1.287819in}}%
\pgfpathcurveto{\pgfqpoint{0.679955in}{1.287819in}}{\pgfqpoint{0.672055in}{1.284546in}}{\pgfqpoint{0.666231in}{1.278722in}}%
\pgfpathcurveto{\pgfqpoint{0.660407in}{1.272898in}}{\pgfqpoint{0.657135in}{1.264998in}}{\pgfqpoint{0.657135in}{1.256762in}}%
\pgfpathcurveto{\pgfqpoint{0.657135in}{1.248526in}}{\pgfqpoint{0.660407in}{1.240626in}}{\pgfqpoint{0.666231in}{1.234802in}}%
\pgfpathcurveto{\pgfqpoint{0.672055in}{1.228978in}}{\pgfqpoint{0.679955in}{1.225706in}}{\pgfqpoint{0.688192in}{1.225706in}}%
\pgfpathlineto{\pgfqpoint{0.688192in}{1.225706in}}%
\pgfpathclose%
\pgfusepath{stroke}%
\end{pgfscope}%
\begin{pgfscope}%
\pgfpathrectangle{\pgfqpoint{0.688192in}{0.670138in}}{\pgfqpoint{6.200000in}{4.620000in}}%
\pgfusepath{clip}%
\pgfsetbuttcap%
\pgfsetmiterjoin%
\definecolor{currentfill}{rgb}{0.839216,0.152941,0.156863}%
\pgfsetfillcolor{currentfill}%
\pgfsetfillopacity{0.200000}%
\pgfsetlinewidth{1.003750pt}%
\definecolor{currentstroke}{rgb}{0.839216,0.152941,0.156863}%
\pgfsetstrokecolor{currentstroke}%
\pgfsetstrokeopacity{0.200000}%
\pgfsetdash{}{0pt}%
\pgfpathmoveto{\pgfqpoint{0.688192in}{0.670138in}}%
\pgfpathlineto{\pgfqpoint{1.790122in}{0.670138in}}%
\pgfpathlineto{\pgfqpoint{1.790122in}{5.290138in}}%
\pgfpathlineto{\pgfqpoint{0.688192in}{5.290138in}}%
\pgfpathlineto{\pgfqpoint{0.688192in}{0.670138in}}%
\pgfpathclose%
\pgfusepath{stroke,fill}%
\end{pgfscope}%
\begin{pgfscope}%
\pgfsetbuttcap%
\pgfsetmiterjoin%
\definecolor{currentfill}{rgb}{0.839216,0.152941,0.156863}%
\pgfsetfillcolor{currentfill}%
\pgfsetfillopacity{0.200000}%
\pgfsetlinewidth{1.003750pt}%
\definecolor{currentstroke}{rgb}{0.839216,0.152941,0.156863}%
\pgfsetstrokecolor{currentstroke}%
\pgfsetstrokeopacity{0.200000}%
\pgfsetdash{}{0pt}%
\pgfpathrectangle{\pgfqpoint{0.688192in}{0.670138in}}{\pgfqpoint{6.200000in}{4.620000in}}%
\pgfusepath{clip}%
\pgfpathmoveto{\pgfqpoint{0.688192in}{0.670138in}}%
\pgfpathlineto{\pgfqpoint{1.790122in}{0.670138in}}%
\pgfpathlineto{\pgfqpoint{1.790122in}{5.290138in}}%
\pgfpathlineto{\pgfqpoint{0.688192in}{5.290138in}}%
\pgfpathlineto{\pgfqpoint{0.688192in}{0.670138in}}%
\pgfpathclose%
\pgfusepath{clip}%
\pgfsys@defobject{currentpattern}{\pgfqpoint{0in}{0in}}{\pgfqpoint{1in}{1in}}{%
\begin{pgfscope}%
\pgfpathrectangle{\pgfqpoint{0in}{0in}}{\pgfqpoint{1in}{1in}}%
\pgfusepath{clip}%
\pgfpathmoveto{\pgfqpoint{-0.500000in}{0.500000in}}%
\pgfpathlineto{\pgfqpoint{0.500000in}{1.500000in}}%
\pgfpathmoveto{\pgfqpoint{-0.333333in}{0.333333in}}%
\pgfpathlineto{\pgfqpoint{0.666667in}{1.333333in}}%
\pgfpathmoveto{\pgfqpoint{-0.166667in}{0.166667in}}%
\pgfpathlineto{\pgfqpoint{0.833333in}{1.166667in}}%
\pgfpathmoveto{\pgfqpoint{0.000000in}{0.000000in}}%
\pgfpathlineto{\pgfqpoint{1.000000in}{1.000000in}}%
\pgfpathmoveto{\pgfqpoint{0.166667in}{-0.166667in}}%
\pgfpathlineto{\pgfqpoint{1.166667in}{0.833333in}}%
\pgfpathmoveto{\pgfqpoint{0.333333in}{-0.333333in}}%
\pgfpathlineto{\pgfqpoint{1.333333in}{0.666667in}}%
\pgfpathmoveto{\pgfqpoint{0.500000in}{-0.500000in}}%
\pgfpathlineto{\pgfqpoint{1.500000in}{0.500000in}}%
\pgfusepath{stroke}%
\end{pgfscope}%
}%
\pgfsys@transformshift{0.688192in}{0.670138in}%
\pgfsys@useobject{currentpattern}{}%
\pgfsys@transformshift{1in}{0in}%
\pgfsys@useobject{currentpattern}{}%
\pgfsys@transformshift{1in}{0in}%
\pgfsys@transformshift{-2in}{0in}%
\pgfsys@transformshift{0in}{1in}%
\pgfsys@useobject{currentpattern}{}%
\pgfsys@transformshift{1in}{0in}%
\pgfsys@useobject{currentpattern}{}%
\pgfsys@transformshift{1in}{0in}%
\pgfsys@transformshift{-2in}{0in}%
\pgfsys@transformshift{0in}{1in}%
\pgfsys@useobject{currentpattern}{}%
\pgfsys@transformshift{1in}{0in}%
\pgfsys@useobject{currentpattern}{}%
\pgfsys@transformshift{1in}{0in}%
\pgfsys@transformshift{-2in}{0in}%
\pgfsys@transformshift{0in}{1in}%
\pgfsys@useobject{currentpattern}{}%
\pgfsys@transformshift{1in}{0in}%
\pgfsys@useobject{currentpattern}{}%
\pgfsys@transformshift{1in}{0in}%
\pgfsys@transformshift{-2in}{0in}%
\pgfsys@transformshift{0in}{1in}%
\pgfsys@useobject{currentpattern}{}%
\pgfsys@transformshift{1in}{0in}%
\pgfsys@useobject{currentpattern}{}%
\pgfsys@transformshift{1in}{0in}%
\pgfsys@transformshift{-2in}{0in}%
\pgfsys@transformshift{0in}{1in}%
\end{pgfscope}%
\begin{pgfscope}%
\pgfpathrectangle{\pgfqpoint{0.688192in}{0.670138in}}{\pgfqpoint{6.200000in}{4.620000in}}%
\pgfusepath{clip}%
\pgfsetrectcap%
\pgfsetroundjoin%
\pgfsetlinewidth{0.803000pt}%
\definecolor{currentstroke}{rgb}{0.690196,0.690196,0.690196}%
\pgfsetstrokecolor{currentstroke}%
\pgfsetdash{}{0pt}%
\pgfpathmoveto{\pgfqpoint{1.122474in}{0.670138in}}%
\pgfpathlineto{\pgfqpoint{1.122474in}{5.290138in}}%
\pgfusepath{stroke}%
\end{pgfscope}%
\begin{pgfscope}%
\pgfsetbuttcap%
\pgfsetroundjoin%
\definecolor{currentfill}{rgb}{0.000000,0.000000,0.000000}%
\pgfsetfillcolor{currentfill}%
\pgfsetlinewidth{0.803000pt}%
\definecolor{currentstroke}{rgb}{0.000000,0.000000,0.000000}%
\pgfsetstrokecolor{currentstroke}%
\pgfsetdash{}{0pt}%
\pgfsys@defobject{currentmarker}{\pgfqpoint{0.000000in}{-0.048611in}}{\pgfqpoint{0.000000in}{0.000000in}}{%
\pgfpathmoveto{\pgfqpoint{0.000000in}{0.000000in}}%
\pgfpathlineto{\pgfqpoint{0.000000in}{-0.048611in}}%
\pgfusepath{stroke,fill}%
}%
\begin{pgfscope}%
\pgfsys@transformshift{1.122474in}{0.670138in}%
\pgfsys@useobject{currentmarker}{}%
\end{pgfscope}%
\end{pgfscope}%
\begin{pgfscope}%
\definecolor{textcolor}{rgb}{0.000000,0.000000,0.000000}%
\pgfsetstrokecolor{textcolor}%
\pgfsetfillcolor{textcolor}%
\pgftext[x=1.122474in,y=0.572916in,,top]{\color{textcolor}{\rmfamily\fontsize{14.000000}{16.800000}\selectfont\catcode`\^=\active\def^{\ifmmode\sp\else\^{}\fi}\catcode`\%=\active\def%{\%}$\mathdefault{5500}$}}%
\end{pgfscope}%
\begin{pgfscope}%
\pgfpathrectangle{\pgfqpoint{0.688192in}{0.670138in}}{\pgfqpoint{6.200000in}{4.620000in}}%
\pgfusepath{clip}%
\pgfsetrectcap%
\pgfsetroundjoin%
\pgfsetlinewidth{0.803000pt}%
\definecolor{currentstroke}{rgb}{0.690196,0.690196,0.690196}%
\pgfsetstrokecolor{currentstroke}%
\pgfsetdash{}{0pt}%
\pgfpathmoveto{\pgfqpoint{2.163709in}{0.670138in}}%
\pgfpathlineto{\pgfqpoint{2.163709in}{5.290138in}}%
\pgfusepath{stroke}%
\end{pgfscope}%
\begin{pgfscope}%
\pgfsetbuttcap%
\pgfsetroundjoin%
\definecolor{currentfill}{rgb}{0.000000,0.000000,0.000000}%
\pgfsetfillcolor{currentfill}%
\pgfsetlinewidth{0.803000pt}%
\definecolor{currentstroke}{rgb}{0.000000,0.000000,0.000000}%
\pgfsetstrokecolor{currentstroke}%
\pgfsetdash{}{0pt}%
\pgfsys@defobject{currentmarker}{\pgfqpoint{0.000000in}{-0.048611in}}{\pgfqpoint{0.000000in}{0.000000in}}{%
\pgfpathmoveto{\pgfqpoint{0.000000in}{0.000000in}}%
\pgfpathlineto{\pgfqpoint{0.000000in}{-0.048611in}}%
\pgfusepath{stroke,fill}%
}%
\begin{pgfscope}%
\pgfsys@transformshift{2.163709in}{0.670138in}%
\pgfsys@useobject{currentmarker}{}%
\end{pgfscope}%
\end{pgfscope}%
\begin{pgfscope}%
\definecolor{textcolor}{rgb}{0.000000,0.000000,0.000000}%
\pgfsetstrokecolor{textcolor}%
\pgfsetfillcolor{textcolor}%
\pgftext[x=2.163709in,y=0.572916in,,top]{\color{textcolor}{\rmfamily\fontsize{14.000000}{16.800000}\selectfont\catcode`\^=\active\def^{\ifmmode\sp\else\^{}\fi}\catcode`\%=\active\def%{\%}$\mathdefault{6000}$}}%
\end{pgfscope}%
\begin{pgfscope}%
\pgfpathrectangle{\pgfqpoint{0.688192in}{0.670138in}}{\pgfqpoint{6.200000in}{4.620000in}}%
\pgfusepath{clip}%
\pgfsetrectcap%
\pgfsetroundjoin%
\pgfsetlinewidth{0.803000pt}%
\definecolor{currentstroke}{rgb}{0.690196,0.690196,0.690196}%
\pgfsetstrokecolor{currentstroke}%
\pgfsetdash{}{0pt}%
\pgfpathmoveto{\pgfqpoint{3.204944in}{0.670138in}}%
\pgfpathlineto{\pgfqpoint{3.204944in}{5.290138in}}%
\pgfusepath{stroke}%
\end{pgfscope}%
\begin{pgfscope}%
\pgfsetbuttcap%
\pgfsetroundjoin%
\definecolor{currentfill}{rgb}{0.000000,0.000000,0.000000}%
\pgfsetfillcolor{currentfill}%
\pgfsetlinewidth{0.803000pt}%
\definecolor{currentstroke}{rgb}{0.000000,0.000000,0.000000}%
\pgfsetstrokecolor{currentstroke}%
\pgfsetdash{}{0pt}%
\pgfsys@defobject{currentmarker}{\pgfqpoint{0.000000in}{-0.048611in}}{\pgfqpoint{0.000000in}{0.000000in}}{%
\pgfpathmoveto{\pgfqpoint{0.000000in}{0.000000in}}%
\pgfpathlineto{\pgfqpoint{0.000000in}{-0.048611in}}%
\pgfusepath{stroke,fill}%
}%
\begin{pgfscope}%
\pgfsys@transformshift{3.204944in}{0.670138in}%
\pgfsys@useobject{currentmarker}{}%
\end{pgfscope}%
\end{pgfscope}%
\begin{pgfscope}%
\definecolor{textcolor}{rgb}{0.000000,0.000000,0.000000}%
\pgfsetstrokecolor{textcolor}%
\pgfsetfillcolor{textcolor}%
\pgftext[x=3.204944in,y=0.572916in,,top]{\color{textcolor}{\rmfamily\fontsize{14.000000}{16.800000}\selectfont\catcode`\^=\active\def^{\ifmmode\sp\else\^{}\fi}\catcode`\%=\active\def%{\%}$\mathdefault{6500}$}}%
\end{pgfscope}%
\begin{pgfscope}%
\pgfpathrectangle{\pgfqpoint{0.688192in}{0.670138in}}{\pgfqpoint{6.200000in}{4.620000in}}%
\pgfusepath{clip}%
\pgfsetrectcap%
\pgfsetroundjoin%
\pgfsetlinewidth{0.803000pt}%
\definecolor{currentstroke}{rgb}{0.690196,0.690196,0.690196}%
\pgfsetstrokecolor{currentstroke}%
\pgfsetdash{}{0pt}%
\pgfpathmoveto{\pgfqpoint{4.246179in}{0.670138in}}%
\pgfpathlineto{\pgfqpoint{4.246179in}{5.290138in}}%
\pgfusepath{stroke}%
\end{pgfscope}%
\begin{pgfscope}%
\pgfsetbuttcap%
\pgfsetroundjoin%
\definecolor{currentfill}{rgb}{0.000000,0.000000,0.000000}%
\pgfsetfillcolor{currentfill}%
\pgfsetlinewidth{0.803000pt}%
\definecolor{currentstroke}{rgb}{0.000000,0.000000,0.000000}%
\pgfsetstrokecolor{currentstroke}%
\pgfsetdash{}{0pt}%
\pgfsys@defobject{currentmarker}{\pgfqpoint{0.000000in}{-0.048611in}}{\pgfqpoint{0.000000in}{0.000000in}}{%
\pgfpathmoveto{\pgfqpoint{0.000000in}{0.000000in}}%
\pgfpathlineto{\pgfqpoint{0.000000in}{-0.048611in}}%
\pgfusepath{stroke,fill}%
}%
\begin{pgfscope}%
\pgfsys@transformshift{4.246179in}{0.670138in}%
\pgfsys@useobject{currentmarker}{}%
\end{pgfscope}%
\end{pgfscope}%
\begin{pgfscope}%
\definecolor{textcolor}{rgb}{0.000000,0.000000,0.000000}%
\pgfsetstrokecolor{textcolor}%
\pgfsetfillcolor{textcolor}%
\pgftext[x=4.246179in,y=0.572916in,,top]{\color{textcolor}{\rmfamily\fontsize{14.000000}{16.800000}\selectfont\catcode`\^=\active\def^{\ifmmode\sp\else\^{}\fi}\catcode`\%=\active\def%{\%}$\mathdefault{7000}$}}%
\end{pgfscope}%
\begin{pgfscope}%
\pgfpathrectangle{\pgfqpoint{0.688192in}{0.670138in}}{\pgfqpoint{6.200000in}{4.620000in}}%
\pgfusepath{clip}%
\pgfsetrectcap%
\pgfsetroundjoin%
\pgfsetlinewidth{0.803000pt}%
\definecolor{currentstroke}{rgb}{0.690196,0.690196,0.690196}%
\pgfsetstrokecolor{currentstroke}%
\pgfsetdash{}{0pt}%
\pgfpathmoveto{\pgfqpoint{5.287414in}{0.670138in}}%
\pgfpathlineto{\pgfqpoint{5.287414in}{5.290138in}}%
\pgfusepath{stroke}%
\end{pgfscope}%
\begin{pgfscope}%
\pgfsetbuttcap%
\pgfsetroundjoin%
\definecolor{currentfill}{rgb}{0.000000,0.000000,0.000000}%
\pgfsetfillcolor{currentfill}%
\pgfsetlinewidth{0.803000pt}%
\definecolor{currentstroke}{rgb}{0.000000,0.000000,0.000000}%
\pgfsetstrokecolor{currentstroke}%
\pgfsetdash{}{0pt}%
\pgfsys@defobject{currentmarker}{\pgfqpoint{0.000000in}{-0.048611in}}{\pgfqpoint{0.000000in}{0.000000in}}{%
\pgfpathmoveto{\pgfqpoint{0.000000in}{0.000000in}}%
\pgfpathlineto{\pgfqpoint{0.000000in}{-0.048611in}}%
\pgfusepath{stroke,fill}%
}%
\begin{pgfscope}%
\pgfsys@transformshift{5.287414in}{0.670138in}%
\pgfsys@useobject{currentmarker}{}%
\end{pgfscope}%
\end{pgfscope}%
\begin{pgfscope}%
\definecolor{textcolor}{rgb}{0.000000,0.000000,0.000000}%
\pgfsetstrokecolor{textcolor}%
\pgfsetfillcolor{textcolor}%
\pgftext[x=5.287414in,y=0.572916in,,top]{\color{textcolor}{\rmfamily\fontsize{14.000000}{16.800000}\selectfont\catcode`\^=\active\def^{\ifmmode\sp\else\^{}\fi}\catcode`\%=\active\def%{\%}$\mathdefault{7500}$}}%
\end{pgfscope}%
\begin{pgfscope}%
\pgfpathrectangle{\pgfqpoint{0.688192in}{0.670138in}}{\pgfqpoint{6.200000in}{4.620000in}}%
\pgfusepath{clip}%
\pgfsetrectcap%
\pgfsetroundjoin%
\pgfsetlinewidth{0.803000pt}%
\definecolor{currentstroke}{rgb}{0.690196,0.690196,0.690196}%
\pgfsetstrokecolor{currentstroke}%
\pgfsetdash{}{0pt}%
\pgfpathmoveto{\pgfqpoint{6.328649in}{0.670138in}}%
\pgfpathlineto{\pgfqpoint{6.328649in}{5.290138in}}%
\pgfusepath{stroke}%
\end{pgfscope}%
\begin{pgfscope}%
\pgfsetbuttcap%
\pgfsetroundjoin%
\definecolor{currentfill}{rgb}{0.000000,0.000000,0.000000}%
\pgfsetfillcolor{currentfill}%
\pgfsetlinewidth{0.803000pt}%
\definecolor{currentstroke}{rgb}{0.000000,0.000000,0.000000}%
\pgfsetstrokecolor{currentstroke}%
\pgfsetdash{}{0pt}%
\pgfsys@defobject{currentmarker}{\pgfqpoint{0.000000in}{-0.048611in}}{\pgfqpoint{0.000000in}{0.000000in}}{%
\pgfpathmoveto{\pgfqpoint{0.000000in}{0.000000in}}%
\pgfpathlineto{\pgfqpoint{0.000000in}{-0.048611in}}%
\pgfusepath{stroke,fill}%
}%
\begin{pgfscope}%
\pgfsys@transformshift{6.328649in}{0.670138in}%
\pgfsys@useobject{currentmarker}{}%
\end{pgfscope}%
\end{pgfscope}%
\begin{pgfscope}%
\definecolor{textcolor}{rgb}{0.000000,0.000000,0.000000}%
\pgfsetstrokecolor{textcolor}%
\pgfsetfillcolor{textcolor}%
\pgftext[x=6.328649in,y=0.572916in,,top]{\color{textcolor}{\rmfamily\fontsize{14.000000}{16.800000}\selectfont\catcode`\^=\active\def^{\ifmmode\sp\else\^{}\fi}\catcode`\%=\active\def%{\%}$\mathdefault{8000}$}}%
\end{pgfscope}%
\begin{pgfscope}%
\definecolor{textcolor}{rgb}{0.000000,0.000000,0.000000}%
\pgfsetstrokecolor{textcolor}%
\pgfsetfillcolor{textcolor}%
\pgftext[x=3.788192in,y=0.339583in,,top]{\color{textcolor}{\rmfamily\fontsize{18.000000}{21.600000}\selectfont\catcode`\^=\active\def^{\ifmmode\sp\else\^{}\fi}\catcode`\%=\active\def%{\%}Total Cost (M\$)}}%
\end{pgfscope}%
\begin{pgfscope}%
\pgfpathrectangle{\pgfqpoint{0.688192in}{0.670138in}}{\pgfqpoint{6.200000in}{4.620000in}}%
\pgfusepath{clip}%
\pgfsetrectcap%
\pgfsetroundjoin%
\pgfsetlinewidth{0.803000pt}%
\definecolor{currentstroke}{rgb}{0.690196,0.690196,0.690196}%
\pgfsetstrokecolor{currentstroke}%
\pgfsetdash{}{0pt}%
\pgfpathmoveto{\pgfqpoint{0.688192in}{1.130833in}}%
\pgfpathlineto{\pgfqpoint{6.888192in}{1.130833in}}%
\pgfusepath{stroke}%
\end{pgfscope}%
\begin{pgfscope}%
\pgfsetbuttcap%
\pgfsetroundjoin%
\definecolor{currentfill}{rgb}{0.000000,0.000000,0.000000}%
\pgfsetfillcolor{currentfill}%
\pgfsetlinewidth{0.803000pt}%
\definecolor{currentstroke}{rgb}{0.000000,0.000000,0.000000}%
\pgfsetstrokecolor{currentstroke}%
\pgfsetdash{}{0pt}%
\pgfsys@defobject{currentmarker}{\pgfqpoint{-0.048611in}{0.000000in}}{\pgfqpoint{-0.000000in}{0.000000in}}{%
\pgfpathmoveto{\pgfqpoint{-0.000000in}{0.000000in}}%
\pgfpathlineto{\pgfqpoint{-0.048611in}{0.000000in}}%
\pgfusepath{stroke,fill}%
}%
\begin{pgfscope}%
\pgfsys@transformshift{0.688192in}{1.130833in}%
\pgfsys@useobject{currentmarker}{}%
\end{pgfscope}%
\end{pgfscope}%
\begin{pgfscope}%
\definecolor{textcolor}{rgb}{0.000000,0.000000,0.000000}%
\pgfsetstrokecolor{textcolor}%
\pgfsetfillcolor{textcolor}%
\pgftext[x=0.395138in, y=1.061389in, left, base]{\color{textcolor}{\rmfamily\fontsize{14.000000}{16.800000}\selectfont\catcode`\^=\active\def^{\ifmmode\sp\else\^{}\fi}\catcode`\%=\active\def%{\%}$\mathdefault{10}$}}%
\end{pgfscope}%
\begin{pgfscope}%
\pgfpathrectangle{\pgfqpoint{0.688192in}{0.670138in}}{\pgfqpoint{6.200000in}{4.620000in}}%
\pgfusepath{clip}%
\pgfsetrectcap%
\pgfsetroundjoin%
\pgfsetlinewidth{0.803000pt}%
\definecolor{currentstroke}{rgb}{0.690196,0.690196,0.690196}%
\pgfsetstrokecolor{currentstroke}%
\pgfsetdash{}{0pt}%
\pgfpathmoveto{\pgfqpoint{0.688192in}{1.724860in}}%
\pgfpathlineto{\pgfqpoint{6.888192in}{1.724860in}}%
\pgfusepath{stroke}%
\end{pgfscope}%
\begin{pgfscope}%
\pgfsetbuttcap%
\pgfsetroundjoin%
\definecolor{currentfill}{rgb}{0.000000,0.000000,0.000000}%
\pgfsetfillcolor{currentfill}%
\pgfsetlinewidth{0.803000pt}%
\definecolor{currentstroke}{rgb}{0.000000,0.000000,0.000000}%
\pgfsetstrokecolor{currentstroke}%
\pgfsetdash{}{0pt}%
\pgfsys@defobject{currentmarker}{\pgfqpoint{-0.048611in}{0.000000in}}{\pgfqpoint{-0.000000in}{0.000000in}}{%
\pgfpathmoveto{\pgfqpoint{-0.000000in}{0.000000in}}%
\pgfpathlineto{\pgfqpoint{-0.048611in}{0.000000in}}%
\pgfusepath{stroke,fill}%
}%
\begin{pgfscope}%
\pgfsys@transformshift{0.688192in}{1.724860in}%
\pgfsys@useobject{currentmarker}{}%
\end{pgfscope}%
\end{pgfscope}%
\begin{pgfscope}%
\definecolor{textcolor}{rgb}{0.000000,0.000000,0.000000}%
\pgfsetstrokecolor{textcolor}%
\pgfsetfillcolor{textcolor}%
\pgftext[x=0.395138in, y=1.655416in, left, base]{\color{textcolor}{\rmfamily\fontsize{14.000000}{16.800000}\selectfont\catcode`\^=\active\def^{\ifmmode\sp\else\^{}\fi}\catcode`\%=\active\def%{\%}$\mathdefault{20}$}}%
\end{pgfscope}%
\begin{pgfscope}%
\pgfpathrectangle{\pgfqpoint{0.688192in}{0.670138in}}{\pgfqpoint{6.200000in}{4.620000in}}%
\pgfusepath{clip}%
\pgfsetrectcap%
\pgfsetroundjoin%
\pgfsetlinewidth{0.803000pt}%
\definecolor{currentstroke}{rgb}{0.690196,0.690196,0.690196}%
\pgfsetstrokecolor{currentstroke}%
\pgfsetdash{}{0pt}%
\pgfpathmoveto{\pgfqpoint{0.688192in}{2.318888in}}%
\pgfpathlineto{\pgfqpoint{6.888192in}{2.318888in}}%
\pgfusepath{stroke}%
\end{pgfscope}%
\begin{pgfscope}%
\pgfsetbuttcap%
\pgfsetroundjoin%
\definecolor{currentfill}{rgb}{0.000000,0.000000,0.000000}%
\pgfsetfillcolor{currentfill}%
\pgfsetlinewidth{0.803000pt}%
\definecolor{currentstroke}{rgb}{0.000000,0.000000,0.000000}%
\pgfsetstrokecolor{currentstroke}%
\pgfsetdash{}{0pt}%
\pgfsys@defobject{currentmarker}{\pgfqpoint{-0.048611in}{0.000000in}}{\pgfqpoint{-0.000000in}{0.000000in}}{%
\pgfpathmoveto{\pgfqpoint{-0.000000in}{0.000000in}}%
\pgfpathlineto{\pgfqpoint{-0.048611in}{0.000000in}}%
\pgfusepath{stroke,fill}%
}%
\begin{pgfscope}%
\pgfsys@transformshift{0.688192in}{2.318888in}%
\pgfsys@useobject{currentmarker}{}%
\end{pgfscope}%
\end{pgfscope}%
\begin{pgfscope}%
\definecolor{textcolor}{rgb}{0.000000,0.000000,0.000000}%
\pgfsetstrokecolor{textcolor}%
\pgfsetfillcolor{textcolor}%
\pgftext[x=0.395138in, y=2.249444in, left, base]{\color{textcolor}{\rmfamily\fontsize{14.000000}{16.800000}\selectfont\catcode`\^=\active\def^{\ifmmode\sp\else\^{}\fi}\catcode`\%=\active\def%{\%}$\mathdefault{30}$}}%
\end{pgfscope}%
\begin{pgfscope}%
\pgfpathrectangle{\pgfqpoint{0.688192in}{0.670138in}}{\pgfqpoint{6.200000in}{4.620000in}}%
\pgfusepath{clip}%
\pgfsetrectcap%
\pgfsetroundjoin%
\pgfsetlinewidth{0.803000pt}%
\definecolor{currentstroke}{rgb}{0.690196,0.690196,0.690196}%
\pgfsetstrokecolor{currentstroke}%
\pgfsetdash{}{0pt}%
\pgfpathmoveto{\pgfqpoint{0.688192in}{2.912915in}}%
\pgfpathlineto{\pgfqpoint{6.888192in}{2.912915in}}%
\pgfusepath{stroke}%
\end{pgfscope}%
\begin{pgfscope}%
\pgfsetbuttcap%
\pgfsetroundjoin%
\definecolor{currentfill}{rgb}{0.000000,0.000000,0.000000}%
\pgfsetfillcolor{currentfill}%
\pgfsetlinewidth{0.803000pt}%
\definecolor{currentstroke}{rgb}{0.000000,0.000000,0.000000}%
\pgfsetstrokecolor{currentstroke}%
\pgfsetdash{}{0pt}%
\pgfsys@defobject{currentmarker}{\pgfqpoint{-0.048611in}{0.000000in}}{\pgfqpoint{-0.000000in}{0.000000in}}{%
\pgfpathmoveto{\pgfqpoint{-0.000000in}{0.000000in}}%
\pgfpathlineto{\pgfqpoint{-0.048611in}{0.000000in}}%
\pgfusepath{stroke,fill}%
}%
\begin{pgfscope}%
\pgfsys@transformshift{0.688192in}{2.912915in}%
\pgfsys@useobject{currentmarker}{}%
\end{pgfscope}%
\end{pgfscope}%
\begin{pgfscope}%
\definecolor{textcolor}{rgb}{0.000000,0.000000,0.000000}%
\pgfsetstrokecolor{textcolor}%
\pgfsetfillcolor{textcolor}%
\pgftext[x=0.395138in, y=2.843471in, left, base]{\color{textcolor}{\rmfamily\fontsize{14.000000}{16.800000}\selectfont\catcode`\^=\active\def^{\ifmmode\sp\else\^{}\fi}\catcode`\%=\active\def%{\%}$\mathdefault{40}$}}%
\end{pgfscope}%
\begin{pgfscope}%
\pgfpathrectangle{\pgfqpoint{0.688192in}{0.670138in}}{\pgfqpoint{6.200000in}{4.620000in}}%
\pgfusepath{clip}%
\pgfsetrectcap%
\pgfsetroundjoin%
\pgfsetlinewidth{0.803000pt}%
\definecolor{currentstroke}{rgb}{0.690196,0.690196,0.690196}%
\pgfsetstrokecolor{currentstroke}%
\pgfsetdash{}{0pt}%
\pgfpathmoveto{\pgfqpoint{0.688192in}{3.506943in}}%
\pgfpathlineto{\pgfqpoint{6.888192in}{3.506943in}}%
\pgfusepath{stroke}%
\end{pgfscope}%
\begin{pgfscope}%
\pgfsetbuttcap%
\pgfsetroundjoin%
\definecolor{currentfill}{rgb}{0.000000,0.000000,0.000000}%
\pgfsetfillcolor{currentfill}%
\pgfsetlinewidth{0.803000pt}%
\definecolor{currentstroke}{rgb}{0.000000,0.000000,0.000000}%
\pgfsetstrokecolor{currentstroke}%
\pgfsetdash{}{0pt}%
\pgfsys@defobject{currentmarker}{\pgfqpoint{-0.048611in}{0.000000in}}{\pgfqpoint{-0.000000in}{0.000000in}}{%
\pgfpathmoveto{\pgfqpoint{-0.000000in}{0.000000in}}%
\pgfpathlineto{\pgfqpoint{-0.048611in}{0.000000in}}%
\pgfusepath{stroke,fill}%
}%
\begin{pgfscope}%
\pgfsys@transformshift{0.688192in}{3.506943in}%
\pgfsys@useobject{currentmarker}{}%
\end{pgfscope}%
\end{pgfscope}%
\begin{pgfscope}%
\definecolor{textcolor}{rgb}{0.000000,0.000000,0.000000}%
\pgfsetstrokecolor{textcolor}%
\pgfsetfillcolor{textcolor}%
\pgftext[x=0.395138in, y=3.437499in, left, base]{\color{textcolor}{\rmfamily\fontsize{14.000000}{16.800000}\selectfont\catcode`\^=\active\def^{\ifmmode\sp\else\^{}\fi}\catcode`\%=\active\def%{\%}$\mathdefault{50}$}}%
\end{pgfscope}%
\begin{pgfscope}%
\pgfpathrectangle{\pgfqpoint{0.688192in}{0.670138in}}{\pgfqpoint{6.200000in}{4.620000in}}%
\pgfusepath{clip}%
\pgfsetrectcap%
\pgfsetroundjoin%
\pgfsetlinewidth{0.803000pt}%
\definecolor{currentstroke}{rgb}{0.690196,0.690196,0.690196}%
\pgfsetstrokecolor{currentstroke}%
\pgfsetdash{}{0pt}%
\pgfpathmoveto{\pgfqpoint{0.688192in}{4.100970in}}%
\pgfpathlineto{\pgfqpoint{6.888192in}{4.100970in}}%
\pgfusepath{stroke}%
\end{pgfscope}%
\begin{pgfscope}%
\pgfsetbuttcap%
\pgfsetroundjoin%
\definecolor{currentfill}{rgb}{0.000000,0.000000,0.000000}%
\pgfsetfillcolor{currentfill}%
\pgfsetlinewidth{0.803000pt}%
\definecolor{currentstroke}{rgb}{0.000000,0.000000,0.000000}%
\pgfsetstrokecolor{currentstroke}%
\pgfsetdash{}{0pt}%
\pgfsys@defobject{currentmarker}{\pgfqpoint{-0.048611in}{0.000000in}}{\pgfqpoint{-0.000000in}{0.000000in}}{%
\pgfpathmoveto{\pgfqpoint{-0.000000in}{0.000000in}}%
\pgfpathlineto{\pgfqpoint{-0.048611in}{0.000000in}}%
\pgfusepath{stroke,fill}%
}%
\begin{pgfscope}%
\pgfsys@transformshift{0.688192in}{4.100970in}%
\pgfsys@useobject{currentmarker}{}%
\end{pgfscope}%
\end{pgfscope}%
\begin{pgfscope}%
\definecolor{textcolor}{rgb}{0.000000,0.000000,0.000000}%
\pgfsetstrokecolor{textcolor}%
\pgfsetfillcolor{textcolor}%
\pgftext[x=0.395138in, y=4.031526in, left, base]{\color{textcolor}{\rmfamily\fontsize{14.000000}{16.800000}\selectfont\catcode`\^=\active\def^{\ifmmode\sp\else\^{}\fi}\catcode`\%=\active\def%{\%}$\mathdefault{60}$}}%
\end{pgfscope}%
\begin{pgfscope}%
\pgfpathrectangle{\pgfqpoint{0.688192in}{0.670138in}}{\pgfqpoint{6.200000in}{4.620000in}}%
\pgfusepath{clip}%
\pgfsetrectcap%
\pgfsetroundjoin%
\pgfsetlinewidth{0.803000pt}%
\definecolor{currentstroke}{rgb}{0.690196,0.690196,0.690196}%
\pgfsetstrokecolor{currentstroke}%
\pgfsetdash{}{0pt}%
\pgfpathmoveto{\pgfqpoint{0.688192in}{4.694998in}}%
\pgfpathlineto{\pgfqpoint{6.888192in}{4.694998in}}%
\pgfusepath{stroke}%
\end{pgfscope}%
\begin{pgfscope}%
\pgfsetbuttcap%
\pgfsetroundjoin%
\definecolor{currentfill}{rgb}{0.000000,0.000000,0.000000}%
\pgfsetfillcolor{currentfill}%
\pgfsetlinewidth{0.803000pt}%
\definecolor{currentstroke}{rgb}{0.000000,0.000000,0.000000}%
\pgfsetstrokecolor{currentstroke}%
\pgfsetdash{}{0pt}%
\pgfsys@defobject{currentmarker}{\pgfqpoint{-0.048611in}{0.000000in}}{\pgfqpoint{-0.000000in}{0.000000in}}{%
\pgfpathmoveto{\pgfqpoint{-0.000000in}{0.000000in}}%
\pgfpathlineto{\pgfqpoint{-0.048611in}{0.000000in}}%
\pgfusepath{stroke,fill}%
}%
\begin{pgfscope}%
\pgfsys@transformshift{0.688192in}{4.694998in}%
\pgfsys@useobject{currentmarker}{}%
\end{pgfscope}%
\end{pgfscope}%
\begin{pgfscope}%
\definecolor{textcolor}{rgb}{0.000000,0.000000,0.000000}%
\pgfsetstrokecolor{textcolor}%
\pgfsetfillcolor{textcolor}%
\pgftext[x=0.395138in, y=4.625553in, left, base]{\color{textcolor}{\rmfamily\fontsize{14.000000}{16.800000}\selectfont\catcode`\^=\active\def^{\ifmmode\sp\else\^{}\fi}\catcode`\%=\active\def%{\%}$\mathdefault{70}$}}%
\end{pgfscope}%
\begin{pgfscope}%
\pgfpathrectangle{\pgfqpoint{0.688192in}{0.670138in}}{\pgfqpoint{6.200000in}{4.620000in}}%
\pgfusepath{clip}%
\pgfsetrectcap%
\pgfsetroundjoin%
\pgfsetlinewidth{0.803000pt}%
\definecolor{currentstroke}{rgb}{0.690196,0.690196,0.690196}%
\pgfsetstrokecolor{currentstroke}%
\pgfsetdash{}{0pt}%
\pgfpathmoveto{\pgfqpoint{0.688192in}{5.289025in}}%
\pgfpathlineto{\pgfqpoint{6.888192in}{5.289025in}}%
\pgfusepath{stroke}%
\end{pgfscope}%
\begin{pgfscope}%
\pgfsetbuttcap%
\pgfsetroundjoin%
\definecolor{currentfill}{rgb}{0.000000,0.000000,0.000000}%
\pgfsetfillcolor{currentfill}%
\pgfsetlinewidth{0.803000pt}%
\definecolor{currentstroke}{rgb}{0.000000,0.000000,0.000000}%
\pgfsetstrokecolor{currentstroke}%
\pgfsetdash{}{0pt}%
\pgfsys@defobject{currentmarker}{\pgfqpoint{-0.048611in}{0.000000in}}{\pgfqpoint{-0.000000in}{0.000000in}}{%
\pgfpathmoveto{\pgfqpoint{-0.000000in}{0.000000in}}%
\pgfpathlineto{\pgfqpoint{-0.048611in}{0.000000in}}%
\pgfusepath{stroke,fill}%
}%
\begin{pgfscope}%
\pgfsys@transformshift{0.688192in}{5.289025in}%
\pgfsys@useobject{currentmarker}{}%
\end{pgfscope}%
\end{pgfscope}%
\begin{pgfscope}%
\definecolor{textcolor}{rgb}{0.000000,0.000000,0.000000}%
\pgfsetstrokecolor{textcolor}%
\pgfsetfillcolor{textcolor}%
\pgftext[x=0.395138in, y=5.219581in, left, base]{\color{textcolor}{\rmfamily\fontsize{14.000000}{16.800000}\selectfont\catcode`\^=\active\def^{\ifmmode\sp\else\^{}\fi}\catcode`\%=\active\def%{\%}$\mathdefault{80}$}}%
\end{pgfscope}%
\begin{pgfscope}%
\definecolor{textcolor}{rgb}{0.000000,0.000000,0.000000}%
\pgfsetstrokecolor{textcolor}%
\pgfsetfillcolor{textcolor}%
\pgftext[x=0.339583in,y=2.980138in,,bottom,rotate=90.000000]{\color{textcolor}{\rmfamily\fontsize{18.000000}{21.600000}\selectfont\catcode`\^=\active\def^{\ifmmode\sp\else\^{}\fi}\catcode`\%=\active\def%{\%}CO2 emissions (MT CO2)}}%
\end{pgfscope}%
\begin{pgfscope}%
\pgfpathrectangle{\pgfqpoint{0.688192in}{0.670138in}}{\pgfqpoint{6.200000in}{4.620000in}}%
\pgfusepath{clip}%
\pgfsetrectcap%
\pgfsetroundjoin%
\pgfsetlinewidth{1.505625pt}%
\definecolor{currentstroke}{rgb}{0.000000,0.000000,1.000000}%
\pgfsetstrokecolor{currentstroke}%
\pgfsetdash{}{0pt}%
\pgfpathmoveto{\pgfqpoint{0.741425in}{1.377543in}}%
\pgfpathlineto{\pgfqpoint{0.758703in}{0.955032in}}%
\pgfpathlineto{\pgfqpoint{0.768198in}{0.875033in}}%
\pgfpathlineto{\pgfqpoint{0.774746in}{0.828781in}}%
\pgfpathlineto{\pgfqpoint{0.778243in}{0.822495in}}%
\pgfpathlineto{\pgfqpoint{0.782159in}{0.789611in}}%
\pgfpathlineto{\pgfqpoint{0.786516in}{0.779881in}}%
\pgfpathlineto{\pgfqpoint{0.792538in}{0.779145in}}%
\pgfpathlineto{\pgfqpoint{0.794668in}{0.758056in}}%
\pgfpathlineto{\pgfqpoint{0.799837in}{0.752930in}}%
\pgfpathlineto{\pgfqpoint{0.809370in}{0.751978in}}%
\pgfpathlineto{\pgfqpoint{0.812629in}{0.743975in}}%
\pgfpathlineto{\pgfqpoint{0.815972in}{0.742575in}}%
\pgfpathlineto{\pgfqpoint{0.822987in}{0.738477in}}%
\pgfpathlineto{\pgfqpoint{0.828825in}{0.734937in}}%
\pgfpathlineto{\pgfqpoint{0.829214in}{0.733319in}}%
\pgfpathlineto{\pgfqpoint{0.833044in}{0.730858in}}%
\pgfpathlineto{\pgfqpoint{0.848459in}{0.726329in}}%
\pgfpathlineto{\pgfqpoint{0.864854in}{0.720019in}}%
\pgfpathlineto{\pgfqpoint{0.887104in}{0.715517in}}%
\pgfpathlineto{\pgfqpoint{0.907479in}{0.714004in}}%
\pgfpathlineto{\pgfqpoint{0.908310in}{0.712008in}}%
\pgfpathlineto{\pgfqpoint{0.909513in}{0.708525in}}%
\pgfpathlineto{\pgfqpoint{0.912740in}{0.707284in}}%
\pgfpathlineto{\pgfqpoint{0.920440in}{0.706723in}}%
\pgfpathlineto{\pgfqpoint{0.925670in}{0.705238in}}%
\pgfpathlineto{\pgfqpoint{0.948903in}{0.702931in}}%
\pgfpathlineto{\pgfqpoint{0.951945in}{0.701707in}}%
\pgfpathlineto{\pgfqpoint{0.952035in}{0.700391in}}%
\pgfpathlineto{\pgfqpoint{0.957029in}{0.700173in}}%
\pgfpathlineto{\pgfqpoint{0.968828in}{0.697963in}}%
\pgfpathlineto{\pgfqpoint{0.974412in}{0.697738in}}%
\pgfpathlineto{\pgfqpoint{0.975275in}{0.696914in}}%
\pgfpathlineto{\pgfqpoint{1.021767in}{0.694795in}}%
\pgfpathlineto{\pgfqpoint{1.025407in}{0.690657in}}%
\pgfpathlineto{\pgfqpoint{1.027475in}{0.690338in}}%
\pgfpathlineto{\pgfqpoint{1.034837in}{0.689784in}}%
\pgfpathlineto{\pgfqpoint{1.049406in}{0.687676in}}%
\pgfpathlineto{\pgfqpoint{1.054714in}{0.687138in}}%
\pgfpathlineto{\pgfqpoint{1.059617in}{0.686467in}}%
\pgfpathlineto{\pgfqpoint{1.072141in}{0.685078in}}%
\pgfpathlineto{\pgfqpoint{1.092208in}{0.684413in}}%
\pgfpathlineto{\pgfqpoint{1.115209in}{0.684111in}}%
\pgfpathlineto{\pgfqpoint{1.131834in}{0.684071in}}%
\pgfpathlineto{\pgfqpoint{1.152628in}{0.684059in}}%
\pgfpathlineto{\pgfqpoint{1.251312in}{0.683263in}}%
\pgfpathlineto{\pgfqpoint{1.277476in}{0.683159in}}%
\pgfpathlineto{\pgfqpoint{1.314870in}{0.682855in}}%
\pgfpathlineto{\pgfqpoint{1.369253in}{0.682756in}}%
\pgfpathlineto{\pgfqpoint{1.398687in}{0.682288in}}%
\pgfpathlineto{\pgfqpoint{1.467852in}{0.682134in}}%
\pgfpathlineto{\pgfqpoint{1.557026in}{0.681680in}}%
\pgfpathlineto{\pgfqpoint{1.627242in}{0.680913in}}%
\pgfpathlineto{\pgfqpoint{1.737728in}{0.680478in}}%
\pgfpathlineto{\pgfqpoint{1.887036in}{0.679610in}}%
\pgfpathlineto{\pgfqpoint{2.037481in}{0.678826in}}%
\pgfpathlineto{\pgfqpoint{2.258348in}{0.677741in}}%
\pgfpathlineto{\pgfqpoint{2.626338in}{0.676361in}}%
\pgfpathlineto{\pgfqpoint{3.263784in}{0.674352in}}%
\pgfpathlineto{\pgfqpoint{5.322800in}{0.670138in}}%
\pgfusepath{stroke}%
\end{pgfscope}%
\begin{pgfscope}%
\pgfpathrectangle{\pgfqpoint{0.688192in}{0.670138in}}{\pgfqpoint{6.200000in}{4.620000in}}%
\pgfusepath{clip}%
\pgfsetbuttcap%
\pgfsetroundjoin%
\definecolor{currentfill}{rgb}{0.000000,0.000000,1.000000}%
\pgfsetfillcolor{currentfill}%
\pgfsetlinewidth{1.003750pt}%
\definecolor{currentstroke}{rgb}{0.000000,0.000000,1.000000}%
\pgfsetstrokecolor{currentstroke}%
\pgfsetdash{}{0pt}%
\pgfsys@defobject{currentmarker}{\pgfqpoint{-0.006944in}{-0.006944in}}{\pgfqpoint{0.006944in}{0.006944in}}{%
\pgfpathmoveto{\pgfqpoint{0.000000in}{-0.006944in}}%
\pgfpathcurveto{\pgfqpoint{0.001842in}{-0.006944in}}{\pgfqpoint{0.003608in}{-0.006213in}}{\pgfqpoint{0.004910in}{-0.004910in}}%
\pgfpathcurveto{\pgfqpoint{0.006213in}{-0.003608in}}{\pgfqpoint{0.006944in}{-0.001842in}}{\pgfqpoint{0.006944in}{0.000000in}}%
\pgfpathcurveto{\pgfqpoint{0.006944in}{0.001842in}}{\pgfqpoint{0.006213in}{0.003608in}}{\pgfqpoint{0.004910in}{0.004910in}}%
\pgfpathcurveto{\pgfqpoint{0.003608in}{0.006213in}}{\pgfqpoint{0.001842in}{0.006944in}}{\pgfqpoint{0.000000in}{0.006944in}}%
\pgfpathcurveto{\pgfqpoint{-0.001842in}{0.006944in}}{\pgfqpoint{-0.003608in}{0.006213in}}{\pgfqpoint{-0.004910in}{0.004910in}}%
\pgfpathcurveto{\pgfqpoint{-0.006213in}{0.003608in}}{\pgfqpoint{-0.006944in}{0.001842in}}{\pgfqpoint{-0.006944in}{0.000000in}}%
\pgfpathcurveto{\pgfqpoint{-0.006944in}{-0.001842in}}{\pgfqpoint{-0.006213in}{-0.003608in}}{\pgfqpoint{-0.004910in}{-0.004910in}}%
\pgfpathcurveto{\pgfqpoint{-0.003608in}{-0.006213in}}{\pgfqpoint{-0.001842in}{-0.006944in}}{\pgfqpoint{0.000000in}{-0.006944in}}%
\pgfpathlineto{\pgfqpoint{0.000000in}{-0.006944in}}%
\pgfpathclose%
\pgfusepath{stroke,fill}%
}%
\begin{pgfscope}%
\pgfsys@transformshift{0.741425in}{1.377543in}%
\pgfsys@useobject{currentmarker}{}%
\end{pgfscope}%
\begin{pgfscope}%
\pgfsys@transformshift{0.758703in}{0.955032in}%
\pgfsys@useobject{currentmarker}{}%
\end{pgfscope}%
\begin{pgfscope}%
\pgfsys@transformshift{0.768198in}{0.875033in}%
\pgfsys@useobject{currentmarker}{}%
\end{pgfscope}%
\begin{pgfscope}%
\pgfsys@transformshift{0.774746in}{0.828781in}%
\pgfsys@useobject{currentmarker}{}%
\end{pgfscope}%
\begin{pgfscope}%
\pgfsys@transformshift{0.778243in}{0.822495in}%
\pgfsys@useobject{currentmarker}{}%
\end{pgfscope}%
\begin{pgfscope}%
\pgfsys@transformshift{0.782159in}{0.789611in}%
\pgfsys@useobject{currentmarker}{}%
\end{pgfscope}%
\begin{pgfscope}%
\pgfsys@transformshift{0.786516in}{0.779881in}%
\pgfsys@useobject{currentmarker}{}%
\end{pgfscope}%
\begin{pgfscope}%
\pgfsys@transformshift{0.792538in}{0.779145in}%
\pgfsys@useobject{currentmarker}{}%
\end{pgfscope}%
\begin{pgfscope}%
\pgfsys@transformshift{0.794668in}{0.758056in}%
\pgfsys@useobject{currentmarker}{}%
\end{pgfscope}%
\begin{pgfscope}%
\pgfsys@transformshift{0.799837in}{0.752930in}%
\pgfsys@useobject{currentmarker}{}%
\end{pgfscope}%
\begin{pgfscope}%
\pgfsys@transformshift{0.809370in}{0.751978in}%
\pgfsys@useobject{currentmarker}{}%
\end{pgfscope}%
\begin{pgfscope}%
\pgfsys@transformshift{0.812629in}{0.743975in}%
\pgfsys@useobject{currentmarker}{}%
\end{pgfscope}%
\begin{pgfscope}%
\pgfsys@transformshift{0.815972in}{0.742575in}%
\pgfsys@useobject{currentmarker}{}%
\end{pgfscope}%
\begin{pgfscope}%
\pgfsys@transformshift{0.822987in}{0.738477in}%
\pgfsys@useobject{currentmarker}{}%
\end{pgfscope}%
\begin{pgfscope}%
\pgfsys@transformshift{0.828825in}{0.734937in}%
\pgfsys@useobject{currentmarker}{}%
\end{pgfscope}%
\begin{pgfscope}%
\pgfsys@transformshift{0.829214in}{0.733319in}%
\pgfsys@useobject{currentmarker}{}%
\end{pgfscope}%
\begin{pgfscope}%
\pgfsys@transformshift{0.833044in}{0.730858in}%
\pgfsys@useobject{currentmarker}{}%
\end{pgfscope}%
\begin{pgfscope}%
\pgfsys@transformshift{0.848459in}{0.726329in}%
\pgfsys@useobject{currentmarker}{}%
\end{pgfscope}%
\begin{pgfscope}%
\pgfsys@transformshift{0.864854in}{0.720019in}%
\pgfsys@useobject{currentmarker}{}%
\end{pgfscope}%
\begin{pgfscope}%
\pgfsys@transformshift{0.887104in}{0.715517in}%
\pgfsys@useobject{currentmarker}{}%
\end{pgfscope}%
\begin{pgfscope}%
\pgfsys@transformshift{0.907479in}{0.714004in}%
\pgfsys@useobject{currentmarker}{}%
\end{pgfscope}%
\begin{pgfscope}%
\pgfsys@transformshift{0.908310in}{0.712008in}%
\pgfsys@useobject{currentmarker}{}%
\end{pgfscope}%
\begin{pgfscope}%
\pgfsys@transformshift{0.909513in}{0.708525in}%
\pgfsys@useobject{currentmarker}{}%
\end{pgfscope}%
\begin{pgfscope}%
\pgfsys@transformshift{0.912740in}{0.707284in}%
\pgfsys@useobject{currentmarker}{}%
\end{pgfscope}%
\begin{pgfscope}%
\pgfsys@transformshift{0.920440in}{0.706723in}%
\pgfsys@useobject{currentmarker}{}%
\end{pgfscope}%
\begin{pgfscope}%
\pgfsys@transformshift{0.925670in}{0.705238in}%
\pgfsys@useobject{currentmarker}{}%
\end{pgfscope}%
\begin{pgfscope}%
\pgfsys@transformshift{0.948903in}{0.702931in}%
\pgfsys@useobject{currentmarker}{}%
\end{pgfscope}%
\begin{pgfscope}%
\pgfsys@transformshift{0.951945in}{0.701707in}%
\pgfsys@useobject{currentmarker}{}%
\end{pgfscope}%
\begin{pgfscope}%
\pgfsys@transformshift{0.952035in}{0.700391in}%
\pgfsys@useobject{currentmarker}{}%
\end{pgfscope}%
\begin{pgfscope}%
\pgfsys@transformshift{0.957029in}{0.700173in}%
\pgfsys@useobject{currentmarker}{}%
\end{pgfscope}%
\begin{pgfscope}%
\pgfsys@transformshift{0.968828in}{0.697963in}%
\pgfsys@useobject{currentmarker}{}%
\end{pgfscope}%
\begin{pgfscope}%
\pgfsys@transformshift{0.974412in}{0.697738in}%
\pgfsys@useobject{currentmarker}{}%
\end{pgfscope}%
\begin{pgfscope}%
\pgfsys@transformshift{0.975275in}{0.696914in}%
\pgfsys@useobject{currentmarker}{}%
\end{pgfscope}%
\begin{pgfscope}%
\pgfsys@transformshift{1.021767in}{0.694795in}%
\pgfsys@useobject{currentmarker}{}%
\end{pgfscope}%
\begin{pgfscope}%
\pgfsys@transformshift{1.025407in}{0.690657in}%
\pgfsys@useobject{currentmarker}{}%
\end{pgfscope}%
\begin{pgfscope}%
\pgfsys@transformshift{1.027475in}{0.690338in}%
\pgfsys@useobject{currentmarker}{}%
\end{pgfscope}%
\begin{pgfscope}%
\pgfsys@transformshift{1.034837in}{0.689784in}%
\pgfsys@useobject{currentmarker}{}%
\end{pgfscope}%
\begin{pgfscope}%
\pgfsys@transformshift{1.049406in}{0.687676in}%
\pgfsys@useobject{currentmarker}{}%
\end{pgfscope}%
\begin{pgfscope}%
\pgfsys@transformshift{1.054714in}{0.687138in}%
\pgfsys@useobject{currentmarker}{}%
\end{pgfscope}%
\begin{pgfscope}%
\pgfsys@transformshift{1.059617in}{0.686467in}%
\pgfsys@useobject{currentmarker}{}%
\end{pgfscope}%
\begin{pgfscope}%
\pgfsys@transformshift{1.072141in}{0.685078in}%
\pgfsys@useobject{currentmarker}{}%
\end{pgfscope}%
\begin{pgfscope}%
\pgfsys@transformshift{1.092208in}{0.684413in}%
\pgfsys@useobject{currentmarker}{}%
\end{pgfscope}%
\begin{pgfscope}%
\pgfsys@transformshift{1.115209in}{0.684111in}%
\pgfsys@useobject{currentmarker}{}%
\end{pgfscope}%
\begin{pgfscope}%
\pgfsys@transformshift{1.131834in}{0.684071in}%
\pgfsys@useobject{currentmarker}{}%
\end{pgfscope}%
\begin{pgfscope}%
\pgfsys@transformshift{1.152628in}{0.684059in}%
\pgfsys@useobject{currentmarker}{}%
\end{pgfscope}%
\begin{pgfscope}%
\pgfsys@transformshift{1.251312in}{0.683263in}%
\pgfsys@useobject{currentmarker}{}%
\end{pgfscope}%
\begin{pgfscope}%
\pgfsys@transformshift{1.277476in}{0.683159in}%
\pgfsys@useobject{currentmarker}{}%
\end{pgfscope}%
\begin{pgfscope}%
\pgfsys@transformshift{1.314870in}{0.682855in}%
\pgfsys@useobject{currentmarker}{}%
\end{pgfscope}%
\begin{pgfscope}%
\pgfsys@transformshift{1.369253in}{0.682756in}%
\pgfsys@useobject{currentmarker}{}%
\end{pgfscope}%
\begin{pgfscope}%
\pgfsys@transformshift{1.398687in}{0.682288in}%
\pgfsys@useobject{currentmarker}{}%
\end{pgfscope}%
\begin{pgfscope}%
\pgfsys@transformshift{1.467852in}{0.682134in}%
\pgfsys@useobject{currentmarker}{}%
\end{pgfscope}%
\begin{pgfscope}%
\pgfsys@transformshift{1.557026in}{0.681680in}%
\pgfsys@useobject{currentmarker}{}%
\end{pgfscope}%
\begin{pgfscope}%
\pgfsys@transformshift{1.627242in}{0.680913in}%
\pgfsys@useobject{currentmarker}{}%
\end{pgfscope}%
\begin{pgfscope}%
\pgfsys@transformshift{1.737728in}{0.680478in}%
\pgfsys@useobject{currentmarker}{}%
\end{pgfscope}%
\begin{pgfscope}%
\pgfsys@transformshift{1.887036in}{0.679610in}%
\pgfsys@useobject{currentmarker}{}%
\end{pgfscope}%
\begin{pgfscope}%
\pgfsys@transformshift{2.037481in}{0.678826in}%
\pgfsys@useobject{currentmarker}{}%
\end{pgfscope}%
\begin{pgfscope}%
\pgfsys@transformshift{2.258348in}{0.677741in}%
\pgfsys@useobject{currentmarker}{}%
\end{pgfscope}%
\begin{pgfscope}%
\pgfsys@transformshift{2.626338in}{0.676361in}%
\pgfsys@useobject{currentmarker}{}%
\end{pgfscope}%
\begin{pgfscope}%
\pgfsys@transformshift{3.263784in}{0.674352in}%
\pgfsys@useobject{currentmarker}{}%
\end{pgfscope}%
\begin{pgfscope}%
\pgfsys@transformshift{5.322800in}{0.670138in}%
\pgfsys@useobject{currentmarker}{}%
\end{pgfscope}%
\end{pgfscope}%
\begin{pgfscope}%
\pgfpathrectangle{\pgfqpoint{0.688192in}{0.670138in}}{\pgfqpoint{6.200000in}{4.620000in}}%
\pgfusepath{clip}%
\pgfsetrectcap%
\pgfsetroundjoin%
\pgfsetlinewidth{1.505625pt}%
\definecolor{currentstroke}{rgb}{0.121569,0.466667,0.705882}%
\pgfsetstrokecolor{currentstroke}%
\pgfsetstrokeopacity{0.500000}%
\pgfsetdash{}{0pt}%
\pgfpathmoveto{\pgfqpoint{1.848679in}{1.461617in}}%
\pgfpathlineto{\pgfqpoint{1.867684in}{0.996854in}}%
\pgfpathlineto{\pgfqpoint{1.878129in}{0.908856in}}%
\pgfpathlineto{\pgfqpoint{1.885331in}{0.857978in}}%
\pgfpathlineto{\pgfqpoint{1.889178in}{0.851064in}}%
\pgfpathlineto{\pgfqpoint{1.893486in}{0.814892in}}%
\pgfpathlineto{\pgfqpoint{1.898279in}{0.804188in}}%
\pgfpathlineto{\pgfqpoint{1.904902in}{0.803379in}}%
\pgfpathlineto{\pgfqpoint{1.907246in}{0.780181in}}%
\pgfpathlineto{\pgfqpoint{1.912932in}{0.774543in}}%
\pgfpathlineto{\pgfqpoint{1.923418in}{0.773495in}}%
\pgfpathlineto{\pgfqpoint{1.927003in}{0.764692in}}%
\pgfpathlineto{\pgfqpoint{1.930681in}{0.763152in}}%
\pgfpathlineto{\pgfqpoint{1.938397in}{0.758644in}}%
\pgfpathlineto{\pgfqpoint{1.944819in}{0.754750in}}%
\pgfpathlineto{\pgfqpoint{1.945246in}{0.752971in}}%
\pgfpathlineto{\pgfqpoint{1.949459in}{0.750264in}}%
\pgfpathlineto{\pgfqpoint{1.966417in}{0.745282in}}%
\pgfpathlineto{\pgfqpoint{1.984450in}{0.738340in}}%
\pgfpathlineto{\pgfqpoint{2.008926in}{0.733388in}}%
\pgfpathlineto{\pgfqpoint{2.031338in}{0.731724in}}%
\pgfpathlineto{\pgfqpoint{2.032252in}{0.729528in}}%
\pgfpathlineto{\pgfqpoint{2.033576in}{0.725697in}}%
\pgfpathlineto{\pgfqpoint{2.037125in}{0.724332in}}%
\pgfpathlineto{\pgfqpoint{2.045595in}{0.723715in}}%
\pgfpathlineto{\pgfqpoint{2.051349in}{0.722081in}}%
\pgfpathlineto{\pgfqpoint{2.076904in}{0.719544in}}%
\pgfpathlineto{\pgfqpoint{2.080251in}{0.718197in}}%
\pgfpathlineto{\pgfqpoint{2.080349in}{0.716749in}}%
\pgfpathlineto{\pgfqpoint{2.085843in}{0.716510in}}%
\pgfpathlineto{\pgfqpoint{2.098822in}{0.714079in}}%
\pgfpathlineto{\pgfqpoint{2.104964in}{0.713831in}}%
\pgfpathlineto{\pgfqpoint{2.105914in}{0.712924in}}%
\pgfpathlineto{\pgfqpoint{2.157055in}{0.710594in}}%
\pgfpathlineto{\pgfqpoint{2.161059in}{0.706043in}}%
\pgfpathlineto{\pgfqpoint{2.163334in}{0.705692in}}%
\pgfpathlineto{\pgfqpoint{2.171432in}{0.705082in}}%
\pgfpathlineto{\pgfqpoint{2.187458in}{0.702763in}}%
\pgfpathlineto{\pgfqpoint{2.193297in}{0.702172in}}%
\pgfpathlineto{\pgfqpoint{2.198690in}{0.701434in}}%
\pgfpathlineto{\pgfqpoint{2.212467in}{0.699905in}}%
\pgfpathlineto{\pgfqpoint{2.234540in}{0.699174in}}%
\pgfpathlineto{\pgfqpoint{2.259841in}{0.698842in}}%
\pgfpathlineto{\pgfqpoint{2.278129in}{0.698798in}}%
\pgfpathlineto{\pgfqpoint{2.301002in}{0.698784in}}%
\pgfpathlineto{\pgfqpoint{2.409554in}{0.697909in}}%
\pgfpathlineto{\pgfqpoint{2.438334in}{0.697795in}}%
\pgfpathlineto{\pgfqpoint{2.479468in}{0.697460in}}%
\pgfpathlineto{\pgfqpoint{2.539290in}{0.697351in}}%
\pgfpathlineto{\pgfqpoint{2.571667in}{0.696836in}}%
\pgfpathlineto{\pgfqpoint{2.647748in}{0.696667in}}%
\pgfpathlineto{\pgfqpoint{2.745840in}{0.696168in}}%
\pgfpathlineto{\pgfqpoint{2.823078in}{0.695324in}}%
\pgfpathlineto{\pgfqpoint{2.944612in}{0.694845in}}%
\pgfpathlineto{\pgfqpoint{3.108851in}{0.693891in}}%
\pgfpathlineto{\pgfqpoint{3.274341in}{0.693028in}}%
\pgfpathlineto{\pgfqpoint{3.517294in}{0.691835in}}%
\pgfpathlineto{\pgfqpoint{3.922083in}{0.690316in}}%
\pgfpathlineto{\pgfqpoint{4.623274in}{0.688107in}}%
\pgfpathlineto{\pgfqpoint{6.888192in}{0.683471in}}%
\pgfusepath{stroke}%
\end{pgfscope}%
\begin{pgfscope}%
\pgfsetrectcap%
\pgfsetmiterjoin%
\pgfsetlinewidth{0.803000pt}%
\definecolor{currentstroke}{rgb}{0.000000,0.000000,0.000000}%
\pgfsetstrokecolor{currentstroke}%
\pgfsetdash{}{0pt}%
\pgfpathmoveto{\pgfqpoint{0.688192in}{0.670138in}}%
\pgfpathlineto{\pgfqpoint{0.688192in}{5.290138in}}%
\pgfusepath{stroke}%
\end{pgfscope}%
\begin{pgfscope}%
\pgfsetrectcap%
\pgfsetmiterjoin%
\pgfsetlinewidth{0.803000pt}%
\definecolor{currentstroke}{rgb}{0.000000,0.000000,0.000000}%
\pgfsetstrokecolor{currentstroke}%
\pgfsetdash{}{0pt}%
\pgfpathmoveto{\pgfqpoint{6.888192in}{0.670138in}}%
\pgfpathlineto{\pgfqpoint{6.888192in}{5.290138in}}%
\pgfusepath{stroke}%
\end{pgfscope}%
\begin{pgfscope}%
\pgfsetrectcap%
\pgfsetmiterjoin%
\pgfsetlinewidth{0.803000pt}%
\definecolor{currentstroke}{rgb}{0.000000,0.000000,0.000000}%
\pgfsetstrokecolor{currentstroke}%
\pgfsetdash{}{0pt}%
\pgfpathmoveto{\pgfqpoint{0.688192in}{0.670138in}}%
\pgfpathlineto{\pgfqpoint{6.888192in}{0.670138in}}%
\pgfusepath{stroke}%
\end{pgfscope}%
\begin{pgfscope}%
\pgfsetrectcap%
\pgfsetmiterjoin%
\pgfsetlinewidth{0.803000pt}%
\definecolor{currentstroke}{rgb}{0.000000,0.000000,0.000000}%
\pgfsetstrokecolor{currentstroke}%
\pgfsetdash{}{0pt}%
\pgfpathmoveto{\pgfqpoint{0.688192in}{5.290138in}}%
\pgfpathlineto{\pgfqpoint{6.888192in}{5.290138in}}%
\pgfusepath{stroke}%
\end{pgfscope}%
\begin{pgfscope}%
\pgfsetbuttcap%
\pgfsetmiterjoin%
\pgfsetlinewidth{1.003750pt}%
\definecolor{currentstroke}{rgb}{0.000000,0.000000,0.000000}%
\pgfsetstrokecolor{currentstroke}%
\pgfsetstrokeopacity{0.500000}%
\pgfsetdash{}{0pt}%
\pgfpathmoveto{\pgfqpoint{0.646542in}{1.071430in}}%
\pgfpathlineto{\pgfqpoint{0.810103in}{1.071430in}}%
\pgfpathlineto{\pgfqpoint{0.810103in}{1.434970in}}%
\pgfpathlineto{\pgfqpoint{0.646542in}{1.434970in}}%
\pgfpathlineto{\pgfqpoint{0.646542in}{1.071430in}}%
\pgfpathclose%
\pgfpathmoveto{\pgfqpoint{3.788192in}{5.151538in}}%
\pgfpathquadraticcurveto{\pgfqpoint{2.217367in}{3.293254in}}{\pgfqpoint{0.646542in}{1.434970in}}%
\pgfpathmoveto{\pgfqpoint{6.702192in}{2.980138in}}%
\pgfpathquadraticcurveto{\pgfqpoint{3.756147in}{2.025784in}}{\pgfqpoint{0.810103in}{1.071430in}}%
\pgfusepath{stroke}%
\end{pgfscope}%
\begin{pgfscope}%
\pgfsetbuttcap%
\pgfsetmiterjoin%
\definecolor{currentfill}{rgb}{1.000000,1.000000,1.000000}%
\pgfsetfillcolor{currentfill}%
\pgfsetlinewidth{0.000000pt}%
\definecolor{currentstroke}{rgb}{0.000000,0.000000,0.000000}%
\pgfsetstrokecolor{currentstroke}%
\pgfsetstrokeopacity{0.000000}%
\pgfsetdash{}{0pt}%
\pgfpathmoveto{\pgfqpoint{3.788192in}{2.980138in}}%
\pgfpathlineto{\pgfqpoint{6.702192in}{2.980138in}}%
\pgfpathlineto{\pgfqpoint{6.702192in}{5.151538in}}%
\pgfpathlineto{\pgfqpoint{3.788192in}{5.151538in}}%
\pgfpathlineto{\pgfqpoint{3.788192in}{2.980138in}}%
\pgfpathclose%
\pgfusepath{fill}%
\end{pgfscope}%
\begin{pgfscope}%
\pgfpathrectangle{\pgfqpoint{3.788192in}{2.980138in}}{\pgfqpoint{2.914000in}{2.171400in}}%
\pgfusepath{clip}%
\pgfsetbuttcap%
\pgfsetmiterjoin%
\definecolor{currentfill}{rgb}{0.121569,0.466667,0.705882}%
\pgfsetfillcolor{currentfill}%
\pgfsetfillopacity{0.500000}%
\pgfsetlinewidth{1.003750pt}%
\definecolor{currentstroke}{rgb}{0.121569,0.466667,0.705882}%
\pgfsetstrokecolor{currentstroke}%
\pgfsetstrokeopacity{0.500000}%
\pgfsetdash{}{0pt}%
\pgfpathmoveto{\pgfqpoint{5.478622in}{4.808529in}}%
\pgfpathlineto{\pgfqpoint{5.786444in}{2.284897in}}%
\pgfpathlineto{\pgfqpoint{5.955602in}{1.807070in}}%
\pgfpathlineto{\pgfqpoint{6.072257in}{1.530809in}}%
\pgfpathlineto{\pgfqpoint{6.134561in}{1.493263in}}%
\pgfpathlineto{\pgfqpoint{6.204334in}{1.296852in}}%
\pgfpathlineto{\pgfqpoint{6.281957in}{1.238734in}}%
\pgfpathlineto{\pgfqpoint{6.389240in}{1.234341in}}%
\pgfpathlineto{\pgfqpoint{6.427201in}{1.108377in}}%
\pgfpathlineto{\pgfqpoint{6.519285in}{1.077759in}}%
\pgfpathlineto{\pgfqpoint{6.689124in}{1.072073in}}%
\pgfpathlineto{\pgfqpoint{6.747190in}{1.024269in}}%
\pgfpathlineto{\pgfqpoint{6.806750in}{1.015907in}}%
\pgfpathlineto{\pgfqpoint{6.931730in}{0.991431in}}%
\pgfpathlineto{\pgfqpoint{7.035734in}{0.970288in}}%
\pgfpathlineto{\pgfqpoint{7.042663in}{0.960624in}}%
\pgfpathlineto{\pgfqpoint{7.110896in}{0.945926in}}%
\pgfpathlineto{\pgfqpoint{7.385541in}{0.918874in}}%
\pgfpathlineto{\pgfqpoint{7.677622in}{0.881182in}}%
\pgfpathlineto{\pgfqpoint{8.074033in}{0.854291in}}%
\pgfpathlineto{\pgfqpoint{8.437037in}{0.845259in}}%
\pgfpathlineto{\pgfqpoint{8.451833in}{0.833333in}}%
\pgfpathlineto{\pgfqpoint{8.473270in}{0.812531in}}%
\pgfpathlineto{\pgfqpoint{8.530759in}{0.805118in}}%
\pgfpathlineto{\pgfqpoint{8.667946in}{0.801766in}}%
\pgfpathlineto{\pgfqpoint{8.761129in}{0.792896in}}%
\pgfpathlineto{\pgfqpoint{9.175032in}{0.779118in}}%
\pgfpathlineto{\pgfqpoint{9.229242in}{0.771806in}}%
\pgfpathlineto{\pgfqpoint{9.230836in}{0.763945in}}%
\pgfpathlineto{\pgfqpoint{9.319803in}{0.762647in}}%
\pgfpathlineto{\pgfqpoint{9.530028in}{0.749444in}}%
\pgfpathlineto{\pgfqpoint{9.629502in}{0.748101in}}%
\pgfpathlineto{\pgfqpoint{9.644888in}{0.743176in}}%
\pgfpathlineto{\pgfqpoint{10.473184in}{0.730520in}}%
\pgfpathlineto{\pgfqpoint{10.538033in}{0.705808in}}%
\pgfpathlineto{\pgfqpoint{10.574876in}{0.703903in}}%
\pgfpathlineto{\pgfqpoint{10.706030in}{0.700593in}}%
\pgfpathlineto{\pgfqpoint{10.965602in}{0.687998in}}%
\pgfpathlineto{\pgfqpoint{11.060169in}{0.684789in}}%
\pgfpathlineto{\pgfqpoint{11.147519in}{0.680781in}}%
\pgfpathlineto{\pgfqpoint{11.370645in}{0.672484in}}%
\pgfpathlineto{\pgfqpoint{11.728159in}{0.668514in}}%
\pgfpathlineto{\pgfqpoint{12.137942in}{0.666709in}}%
\pgfpathlineto{\pgfqpoint{12.434130in}{0.666468in}}%
\pgfpathlineto{\pgfqpoint{12.804602in}{0.666397in}}%
\pgfpathlineto{\pgfqpoint{14.562739in}{0.661643in}}%
\pgfpathlineto{\pgfqpoint{15.028873in}{0.661022in}}%
\pgfpathlineto{\pgfqpoint{15.695083in}{0.659204in}}%
\pgfpathlineto{\pgfqpoint{16.663977in}{0.658612in}}%
\pgfpathlineto{\pgfqpoint{17.188373in}{0.655818in}}%
\pgfpathlineto{\pgfqpoint{18.420605in}{0.654898in}}%
\pgfpathlineto{\pgfqpoint{20.009331in}{0.652189in}}%
\pgfpathlineto{\pgfqpoint{21.260300in}{0.647607in}}%
\pgfpathlineto{\pgfqpoint{23.228701in}{0.645006in}}%
\pgfpathlineto{\pgfqpoint{25.888770in}{0.639824in}}%
\pgfpathlineto{\pgfqpoint{28.569097in}{0.635141in}}%
\pgfpathlineto{\pgfqpoint{32.504050in}{0.628662in}}%
\pgfpathlineto{\pgfqpoint{39.060135in}{0.620415in}}%
\pgfpathlineto{\pgfqpoint{50.416853in}{0.608419in}}%
\pgfpathlineto{\pgfqpoint{87.100182in}{0.583248in}}%
\pgfpathlineto{\pgfqpoint{114.989116in}{0.662886in}}%
\pgfpathlineto{\pgfqpoint{74.637454in}{0.690574in}}%
\pgfpathlineto{\pgfqpoint{62.145064in}{0.703770in}}%
\pgfpathlineto{\pgfqpoint{54.933371in}{0.712842in}}%
\pgfpathlineto{\pgfqpoint{50.604922in}{0.719969in}}%
\pgfpathlineto{\pgfqpoint{47.656562in}{0.725121in}}%
\pgfpathlineto{\pgfqpoint{44.730486in}{0.730820in}}%
\pgfpathlineto{\pgfqpoint{42.565245in}{0.733682in}}%
\pgfpathlineto{\pgfqpoint{41.189180in}{0.738722in}}%
\pgfpathlineto{\pgfqpoint{39.441581in}{0.741702in}}%
\pgfpathlineto{\pgfqpoint{38.086125in}{0.742714in}}%
\pgfpathlineto{\pgfqpoint{37.509290in}{0.745787in}}%
\pgfpathlineto{\pgfqpoint{36.443507in}{0.746439in}}%
\pgfpathlineto{\pgfqpoint{35.710675in}{0.748438in}}%
\pgfpathlineto{\pgfqpoint{35.197928in}{0.749121in}}%
\pgfpathlineto{\pgfqpoint{33.263977in}{0.754350in}}%
\pgfpathlineto{\pgfqpoint{32.856458in}{0.754429in}}%
\pgfpathlineto{\pgfqpoint{32.530652in}{0.754694in}}%
\pgfpathlineto{\pgfqpoint{32.079890in}{0.756679in}}%
\pgfpathlineto{\pgfqpoint{31.686625in}{0.761046in}}%
\pgfpathlineto{\pgfqpoint{31.441186in}{0.770173in}}%
\pgfpathlineto{\pgfqpoint{31.345101in}{0.774582in}}%
\pgfpathlineto{\pgfqpoint{31.241077in}{0.778112in}}%
\pgfpathlineto{\pgfqpoint{30.955549in}{0.791966in}}%
\pgfpathlineto{\pgfqpoint{30.811278in}{0.795608in}}%
\pgfpathlineto{\pgfqpoint{30.770752in}{0.797703in}}%
\pgfpathlineto{\pgfqpoint{30.699417in}{0.824886in}}%
\pgfpathlineto{\pgfqpoint{29.788292in}{0.838808in}}%
\pgfpathlineto{\pgfqpoint{29.771368in}{0.844225in}}%
\pgfpathlineto{\pgfqpoint{29.661946in}{0.845702in}}%
\pgfpathlineto{\pgfqpoint{29.430699in}{0.860225in}}%
\pgfpathlineto{\pgfqpoint{29.332834in}{0.861654in}}%
\pgfpathlineto{\pgfqpoint{29.331081in}{0.870301in}}%
\pgfpathlineto{\pgfqpoint{29.271451in}{0.878344in}}%
\pgfpathlineto{\pgfqpoint{28.816157in}{0.893499in}}%
\pgfpathlineto{\pgfqpoint{28.713656in}{0.903257in}}%
\pgfpathlineto{\pgfqpoint{28.562750in}{0.906943in}}%
\pgfpathlineto{\pgfqpoint{28.499512in}{0.915098in}}%
\pgfpathlineto{\pgfqpoint{28.475931in}{0.937980in}}%
\pgfpathlineto{\pgfqpoint{28.459655in}{0.951098in}}%
\pgfpathlineto{\pgfqpoint{28.060351in}{0.961034in}}%
\pgfpathlineto{\pgfqpoint{27.624299in}{0.990614in}}%
\pgfpathlineto{\pgfqpoint{27.303010in}{1.032076in}}%
\pgfpathlineto{\pgfqpoint{27.000901in}{1.061832in}}%
\pgfpathlineto{\pgfqpoint{26.925844in}{1.078000in}}%
\pgfpathlineto{\pgfqpoint{26.918222in}{1.088630in}}%
\pgfpathlineto{\pgfqpoint{26.803818in}{1.111888in}}%
\pgfpathlineto{\pgfqpoint{26.666341in}{1.138811in}}%
\pgfpathlineto{\pgfqpoint{26.600824in}{1.148010in}}%
\pgfpathlineto{\pgfqpoint{26.536952in}{1.200594in}}%
\pgfpathlineto{\pgfqpoint{26.350128in}{1.206849in}}%
\pgfpathlineto{\pgfqpoint{26.248836in}{1.240528in}}%
\pgfpathlineto{\pgfqpoint{26.207079in}{1.379089in}}%
\pgfpathlineto{\pgfqpoint{26.089068in}{1.383921in}}%
\pgfpathlineto{\pgfqpoint{26.003683in}{1.447851in}}%
\pgfpathlineto{\pgfqpoint{25.926932in}{1.663903in}}%
\pgfpathlineto{\pgfqpoint{25.858398in}{1.705204in}}%
\pgfpathlineto{\pgfqpoint{25.730077in}{2.009091in}}%
\pgfpathlineto{\pgfqpoint{25.544004in}{2.534701in}}%
\pgfpathlineto{\pgfqpoint{25.205400in}{5.310696in}}%
\pgfpathlineto{\pgfqpoint{5.478622in}{4.808529in}}%
\pgfpathclose%
\pgfusepath{stroke,fill}%
\end{pgfscope}%
\begin{pgfscope}%
\pgfpathrectangle{\pgfqpoint{3.788192in}{2.980138in}}{\pgfqpoint{2.914000in}{2.171400in}}%
\pgfusepath{clip}%
\pgfsetbuttcap%
\pgfsetroundjoin%
\pgfsetlinewidth{1.003750pt}%
\definecolor{currentstroke}{rgb}{1.000000,0.000000,0.000000}%
\pgfsetstrokecolor{currentstroke}%
\pgfsetdash{}{0pt}%
\pgfpathmoveto{\pgfqpoint{16.861003in}{25.566105in}}%
\pgfpathcurveto{\pgfqpoint{16.869240in}{25.566105in}}{\pgfqpoint{16.877140in}{25.569377in}}{\pgfqpoint{16.882964in}{25.575201in}}%
\pgfpathcurveto{\pgfqpoint{16.888788in}{25.581025in}}{\pgfqpoint{16.892060in}{25.588925in}}{\pgfqpoint{16.892060in}{25.597161in}}%
\pgfpathcurveto{\pgfqpoint{16.892060in}{25.605397in}}{\pgfqpoint{16.888788in}{25.613297in}}{\pgfqpoint{16.882964in}{25.619121in}}%
\pgfpathcurveto{\pgfqpoint{16.877140in}{25.624945in}}{\pgfqpoint{16.869240in}{25.628218in}}{\pgfqpoint{16.861003in}{25.628218in}}%
\pgfpathcurveto{\pgfqpoint{16.852767in}{25.628218in}}{\pgfqpoint{16.844867in}{25.624945in}}{\pgfqpoint{16.839043in}{25.619121in}}%
\pgfpathcurveto{\pgfqpoint{16.833219in}{25.613297in}}{\pgfqpoint{16.829947in}{25.605397in}}{\pgfqpoint{16.829947in}{25.597161in}}%
\pgfpathcurveto{\pgfqpoint{16.829947in}{25.588925in}}{\pgfqpoint{16.833219in}{25.581025in}}{\pgfqpoint{16.839043in}{25.575201in}}%
\pgfpathcurveto{\pgfqpoint{16.844867in}{25.569377in}}{\pgfqpoint{16.852767in}{25.566105in}}{\pgfqpoint{16.861003in}{25.566105in}}%
\pgfusepath{stroke}%
\end{pgfscope}%
\begin{pgfscope}%
\pgfpathrectangle{\pgfqpoint{3.788192in}{2.980138in}}{\pgfqpoint{2.914000in}{2.171400in}}%
\pgfusepath{clip}%
\pgfsetbuttcap%
\pgfsetroundjoin%
\pgfsetlinewidth{1.003750pt}%
\definecolor{currentstroke}{rgb}{1.000000,0.000000,0.000000}%
\pgfsetstrokecolor{currentstroke}%
\pgfsetdash{}{0pt}%
\pgfpathmoveto{\pgfqpoint{11.934253in}{10.111217in}}%
\pgfpathcurveto{\pgfqpoint{11.942489in}{10.111217in}}{\pgfqpoint{11.950389in}{10.114489in}}{\pgfqpoint{11.956213in}{10.120313in}}%
\pgfpathcurveto{\pgfqpoint{11.962037in}{10.126137in}}{\pgfqpoint{11.965309in}{10.134037in}}{\pgfqpoint{11.965309in}{10.142274in}}%
\pgfpathcurveto{\pgfqpoint{11.965309in}{10.150510in}}{\pgfqpoint{11.962037in}{10.158410in}}{\pgfqpoint{11.956213in}{10.164234in}}%
\pgfpathcurveto{\pgfqpoint{11.950389in}{10.170058in}}{\pgfqpoint{11.942489in}{10.173330in}}{\pgfqpoint{11.934253in}{10.173330in}}%
\pgfpathcurveto{\pgfqpoint{11.926016in}{10.173330in}}{\pgfqpoint{11.918116in}{10.170058in}}{\pgfqpoint{11.912292in}{10.164234in}}%
\pgfpathcurveto{\pgfqpoint{11.906469in}{10.158410in}}{\pgfqpoint{11.903196in}{10.150510in}}{\pgfqpoint{11.903196in}{10.142274in}}%
\pgfpathcurveto{\pgfqpoint{11.903196in}{10.134037in}}{\pgfqpoint{11.906469in}{10.126137in}}{\pgfqpoint{11.912292in}{10.120313in}}%
\pgfpathcurveto{\pgfqpoint{11.918116in}{10.114489in}}{\pgfqpoint{11.926016in}{10.111217in}}{\pgfqpoint{11.934253in}{10.111217in}}%
\pgfusepath{stroke}%
\end{pgfscope}%
\begin{pgfscope}%
\pgfpathrectangle{\pgfqpoint{3.788192in}{2.980138in}}{\pgfqpoint{2.914000in}{2.171400in}}%
\pgfusepath{clip}%
\pgfsetbuttcap%
\pgfsetroundjoin%
\pgfsetlinewidth{1.003750pt}%
\definecolor{currentstroke}{rgb}{1.000000,0.000000,0.000000}%
\pgfsetstrokecolor{currentstroke}%
\pgfsetdash{}{0pt}%
\pgfpathmoveto{\pgfqpoint{12.691767in}{10.530438in}}%
\pgfpathcurveto{\pgfqpoint{12.700003in}{10.530438in}}{\pgfqpoint{12.707903in}{10.533710in}}{\pgfqpoint{12.713727in}{10.539534in}}%
\pgfpathcurveto{\pgfqpoint{12.719551in}{10.545358in}}{\pgfqpoint{12.722823in}{10.553258in}}{\pgfqpoint{12.722823in}{10.561494in}}%
\pgfpathcurveto{\pgfqpoint{12.722823in}{10.569730in}}{\pgfqpoint{12.719551in}{10.577630in}}{\pgfqpoint{12.713727in}{10.583454in}}%
\pgfpathcurveto{\pgfqpoint{12.707903in}{10.589278in}}{\pgfqpoint{12.700003in}{10.592551in}}{\pgfqpoint{12.691767in}{10.592551in}}%
\pgfpathcurveto{\pgfqpoint{12.683530in}{10.592551in}}{\pgfqpoint{12.675630in}{10.589278in}}{\pgfqpoint{12.669806in}{10.583454in}}%
\pgfpathcurveto{\pgfqpoint{12.663983in}{10.577630in}}{\pgfqpoint{12.660710in}{10.569730in}}{\pgfqpoint{12.660710in}{10.561494in}}%
\pgfpathcurveto{\pgfqpoint{12.660710in}{10.553258in}}{\pgfqpoint{12.663983in}{10.545358in}}{\pgfqpoint{12.669806in}{10.539534in}}%
\pgfpathcurveto{\pgfqpoint{12.675630in}{10.533710in}}{\pgfqpoint{12.683530in}{10.530438in}}{\pgfqpoint{12.691767in}{10.530438in}}%
\pgfusepath{stroke}%
\end{pgfscope}%
\begin{pgfscope}%
\pgfpathrectangle{\pgfqpoint{3.788192in}{2.980138in}}{\pgfqpoint{2.914000in}{2.171400in}}%
\pgfusepath{clip}%
\pgfsetbuttcap%
\pgfsetroundjoin%
\pgfsetlinewidth{1.003750pt}%
\definecolor{currentstroke}{rgb}{1.000000,0.000000,0.000000}%
\pgfsetstrokecolor{currentstroke}%
\pgfsetdash{}{0pt}%
\pgfpathmoveto{\pgfqpoint{15.144604in}{10.695451in}}%
\pgfpathcurveto{\pgfqpoint{15.152841in}{10.695451in}}{\pgfqpoint{15.160741in}{10.698723in}}{\pgfqpoint{15.166565in}{10.704547in}}%
\pgfpathcurveto{\pgfqpoint{15.172389in}{10.710371in}}{\pgfqpoint{15.175661in}{10.718271in}}{\pgfqpoint{15.175661in}{10.726507in}}%
\pgfpathcurveto{\pgfqpoint{15.175661in}{10.734743in}}{\pgfqpoint{15.172389in}{10.742643in}}{\pgfqpoint{15.166565in}{10.748467in}}%
\pgfpathcurveto{\pgfqpoint{15.160741in}{10.754291in}}{\pgfqpoint{15.152841in}{10.757564in}}{\pgfqpoint{15.144604in}{10.757564in}}%
\pgfpathcurveto{\pgfqpoint{15.136368in}{10.757564in}}{\pgfqpoint{15.128468in}{10.754291in}}{\pgfqpoint{15.122644in}{10.748467in}}%
\pgfpathcurveto{\pgfqpoint{15.116820in}{10.742643in}}{\pgfqpoint{15.113548in}{10.734743in}}{\pgfqpoint{15.113548in}{10.726507in}}%
\pgfpathcurveto{\pgfqpoint{15.113548in}{10.718271in}}{\pgfqpoint{15.116820in}{10.710371in}}{\pgfqpoint{15.122644in}{10.704547in}}%
\pgfpathcurveto{\pgfqpoint{15.128468in}{10.698723in}}{\pgfqpoint{15.136368in}{10.695451in}}{\pgfqpoint{15.144604in}{10.695451in}}%
\pgfusepath{stroke}%
\end{pgfscope}%
\begin{pgfscope}%
\pgfpathrectangle{\pgfqpoint{3.788192in}{2.980138in}}{\pgfqpoint{2.914000in}{2.171400in}}%
\pgfusepath{clip}%
\pgfsetbuttcap%
\pgfsetroundjoin%
\pgfsetlinewidth{1.003750pt}%
\definecolor{currentstroke}{rgb}{1.000000,0.000000,0.000000}%
\pgfsetstrokecolor{currentstroke}%
\pgfsetdash{}{0pt}%
\pgfpathmoveto{\pgfqpoint{13.542463in}{11.423191in}}%
\pgfpathcurveto{\pgfqpoint{13.550699in}{11.423191in}}{\pgfqpoint{13.558599in}{11.426464in}}{\pgfqpoint{13.564423in}{11.432288in}}%
\pgfpathcurveto{\pgfqpoint{13.570247in}{11.438112in}}{\pgfqpoint{13.573519in}{11.446012in}}{\pgfqpoint{13.573519in}{11.454248in}}%
\pgfpathcurveto{\pgfqpoint{13.573519in}{11.462484in}}{\pgfqpoint{13.570247in}{11.470384in}}{\pgfqpoint{13.564423in}{11.476208in}}%
\pgfpathcurveto{\pgfqpoint{13.558599in}{11.482032in}}{\pgfqpoint{13.550699in}{11.485304in}}{\pgfqpoint{13.542463in}{11.485304in}}%
\pgfpathcurveto{\pgfqpoint{13.534227in}{11.485304in}}{\pgfqpoint{13.526327in}{11.482032in}}{\pgfqpoint{13.520503in}{11.476208in}}%
\pgfpathcurveto{\pgfqpoint{13.514679in}{11.470384in}}{\pgfqpoint{13.511406in}{11.462484in}}{\pgfqpoint{13.511406in}{11.454248in}}%
\pgfpathcurveto{\pgfqpoint{13.511406in}{11.446012in}}{\pgfqpoint{13.514679in}{11.438112in}}{\pgfqpoint{13.520503in}{11.432288in}}%
\pgfpathcurveto{\pgfqpoint{13.526327in}{11.426464in}}{\pgfqpoint{13.534227in}{11.423191in}}{\pgfqpoint{13.542463in}{11.423191in}}%
\pgfusepath{stroke}%
\end{pgfscope}%
\begin{pgfscope}%
\pgfpathrectangle{\pgfqpoint{3.788192in}{2.980138in}}{\pgfqpoint{2.914000in}{2.171400in}}%
\pgfusepath{clip}%
\pgfsetbuttcap%
\pgfsetroundjoin%
\pgfsetlinewidth{1.003750pt}%
\definecolor{currentstroke}{rgb}{1.000000,0.000000,0.000000}%
\pgfsetstrokecolor{currentstroke}%
\pgfsetdash{}{0pt}%
\pgfpathmoveto{\pgfqpoint{12.332968in}{8.917460in}}%
\pgfpathcurveto{\pgfqpoint{12.341204in}{8.917460in}}{\pgfqpoint{12.349104in}{8.920732in}}{\pgfqpoint{12.354928in}{8.926556in}}%
\pgfpathcurveto{\pgfqpoint{12.360752in}{8.932380in}}{\pgfqpoint{12.364025in}{8.940280in}}{\pgfqpoint{12.364025in}{8.948517in}}%
\pgfpathcurveto{\pgfqpoint{12.364025in}{8.956753in}}{\pgfqpoint{12.360752in}{8.964653in}}{\pgfqpoint{12.354928in}{8.970477in}}%
\pgfpathcurveto{\pgfqpoint{12.349104in}{8.976301in}}{\pgfqpoint{12.341204in}{8.979573in}}{\pgfqpoint{12.332968in}{8.979573in}}%
\pgfpathcurveto{\pgfqpoint{12.324732in}{8.979573in}}{\pgfqpoint{12.316832in}{8.976301in}}{\pgfqpoint{12.311008in}{8.970477in}}%
\pgfpathcurveto{\pgfqpoint{12.305184in}{8.964653in}}{\pgfqpoint{12.301912in}{8.956753in}}{\pgfqpoint{12.301912in}{8.948517in}}%
\pgfpathcurveto{\pgfqpoint{12.301912in}{8.940280in}}{\pgfqpoint{12.305184in}{8.932380in}}{\pgfqpoint{12.311008in}{8.926556in}}%
\pgfpathcurveto{\pgfqpoint{12.316832in}{8.920732in}}{\pgfqpoint{12.324732in}{8.917460in}}{\pgfqpoint{12.332968in}{8.917460in}}%
\pgfusepath{stroke}%
\end{pgfscope}%
\begin{pgfscope}%
\pgfpathrectangle{\pgfqpoint{3.788192in}{2.980138in}}{\pgfqpoint{2.914000in}{2.171400in}}%
\pgfusepath{clip}%
\pgfsetbuttcap%
\pgfsetroundjoin%
\pgfsetlinewidth{1.003750pt}%
\definecolor{currentstroke}{rgb}{1.000000,0.000000,0.000000}%
\pgfsetstrokecolor{currentstroke}%
\pgfsetdash{}{0pt}%
\pgfpathmoveto{\pgfqpoint{11.399902in}{7.106716in}}%
\pgfpathcurveto{\pgfqpoint{11.408139in}{7.106716in}}{\pgfqpoint{11.416039in}{7.109988in}}{\pgfqpoint{11.421863in}{7.115812in}}%
\pgfpathcurveto{\pgfqpoint{11.427686in}{7.121636in}}{\pgfqpoint{11.430959in}{7.129536in}}{\pgfqpoint{11.430959in}{7.137773in}}%
\pgfpathcurveto{\pgfqpoint{11.430959in}{7.146009in}}{\pgfqpoint{11.427686in}{7.153909in}}{\pgfqpoint{11.421863in}{7.159733in}}%
\pgfpathcurveto{\pgfqpoint{11.416039in}{7.165557in}}{\pgfqpoint{11.408139in}{7.168829in}}{\pgfqpoint{11.399902in}{7.168829in}}%
\pgfpathcurveto{\pgfqpoint{11.391666in}{7.168829in}}{\pgfqpoint{11.383766in}{7.165557in}}{\pgfqpoint{11.377942in}{7.159733in}}%
\pgfpathcurveto{\pgfqpoint{11.372118in}{7.153909in}}{\pgfqpoint{11.368846in}{7.146009in}}{\pgfqpoint{11.368846in}{7.137773in}}%
\pgfpathcurveto{\pgfqpoint{11.368846in}{7.129536in}}{\pgfqpoint{11.372118in}{7.121636in}}{\pgfqpoint{11.377942in}{7.115812in}}%
\pgfpathcurveto{\pgfqpoint{11.383766in}{7.109988in}}{\pgfqpoint{11.391666in}{7.106716in}}{\pgfqpoint{11.399902in}{7.106716in}}%
\pgfusepath{stroke}%
\end{pgfscope}%
\begin{pgfscope}%
\pgfpathrectangle{\pgfqpoint{3.788192in}{2.980138in}}{\pgfqpoint{2.914000in}{2.171400in}}%
\pgfusepath{clip}%
\pgfsetbuttcap%
\pgfsetroundjoin%
\pgfsetlinewidth{1.003750pt}%
\definecolor{currentstroke}{rgb}{1.000000,0.000000,0.000000}%
\pgfsetstrokecolor{currentstroke}%
\pgfsetdash{}{0pt}%
\pgfpathmoveto{\pgfqpoint{11.356549in}{7.029726in}}%
\pgfpathcurveto{\pgfqpoint{11.364785in}{7.029726in}}{\pgfqpoint{11.372685in}{7.032998in}}{\pgfqpoint{11.378509in}{7.038822in}}%
\pgfpathcurveto{\pgfqpoint{11.384333in}{7.044646in}}{\pgfqpoint{11.387605in}{7.052546in}}{\pgfqpoint{11.387605in}{7.060782in}}%
\pgfpathcurveto{\pgfqpoint{11.387605in}{7.069018in}}{\pgfqpoint{11.384333in}{7.076918in}}{\pgfqpoint{11.378509in}{7.082742in}}%
\pgfpathcurveto{\pgfqpoint{11.372685in}{7.088566in}}{\pgfqpoint{11.364785in}{7.091839in}}{\pgfqpoint{11.356549in}{7.091839in}}%
\pgfpathcurveto{\pgfqpoint{11.348313in}{7.091839in}}{\pgfqpoint{11.340412in}{7.088566in}}{\pgfqpoint{11.334589in}{7.082742in}}%
\pgfpathcurveto{\pgfqpoint{11.328765in}{7.076918in}}{\pgfqpoint{11.325492in}{7.069018in}}{\pgfqpoint{11.325492in}{7.060782in}}%
\pgfpathcurveto{\pgfqpoint{11.325492in}{7.052546in}}{\pgfqpoint{11.328765in}{7.044646in}}{\pgfqpoint{11.334589in}{7.038822in}}%
\pgfpathcurveto{\pgfqpoint{11.340412in}{7.032998in}}{\pgfqpoint{11.348313in}{7.029726in}}{\pgfqpoint{11.356549in}{7.029726in}}%
\pgfusepath{stroke}%
\end{pgfscope}%
\begin{pgfscope}%
\pgfpathrectangle{\pgfqpoint{3.788192in}{2.980138in}}{\pgfqpoint{2.914000in}{2.171400in}}%
\pgfusepath{clip}%
\pgfsetbuttcap%
\pgfsetroundjoin%
\pgfsetlinewidth{1.003750pt}%
\definecolor{currentstroke}{rgb}{1.000000,0.000000,0.000000}%
\pgfsetstrokecolor{currentstroke}%
\pgfsetdash{}{0pt}%
\pgfpathmoveto{\pgfqpoint{13.232346in}{9.803683in}}%
\pgfpathcurveto{\pgfqpoint{13.240582in}{9.803683in}}{\pgfqpoint{13.248482in}{9.806956in}}{\pgfqpoint{13.254306in}{9.812780in}}%
\pgfpathcurveto{\pgfqpoint{13.260130in}{9.818604in}}{\pgfqpoint{13.263402in}{9.826504in}}{\pgfqpoint{13.263402in}{9.834740in}}%
\pgfpathcurveto{\pgfqpoint{13.263402in}{9.842976in}}{\pgfqpoint{13.260130in}{9.850876in}}{\pgfqpoint{13.254306in}{9.856700in}}%
\pgfpathcurveto{\pgfqpoint{13.248482in}{9.862524in}}{\pgfqpoint{13.240582in}{9.865796in}}{\pgfqpoint{13.232346in}{9.865796in}}%
\pgfpathcurveto{\pgfqpoint{13.224110in}{9.865796in}}{\pgfqpoint{13.216210in}{9.862524in}}{\pgfqpoint{13.210386in}{9.856700in}}%
\pgfpathcurveto{\pgfqpoint{13.204562in}{9.850876in}}{\pgfqpoint{13.201289in}{9.842976in}}{\pgfqpoint{13.201289in}{9.834740in}}%
\pgfpathcurveto{\pgfqpoint{13.201289in}{9.826504in}}{\pgfqpoint{13.204562in}{9.818604in}}{\pgfqpoint{13.210386in}{9.812780in}}%
\pgfpathcurveto{\pgfqpoint{13.216210in}{9.806956in}}{\pgfqpoint{13.224110in}{9.803683in}}{\pgfqpoint{13.232346in}{9.803683in}}%
\pgfusepath{stroke}%
\end{pgfscope}%
\begin{pgfscope}%
\pgfpathrectangle{\pgfqpoint{3.788192in}{2.980138in}}{\pgfqpoint{2.914000in}{2.171400in}}%
\pgfusepath{clip}%
\pgfsetbuttcap%
\pgfsetroundjoin%
\pgfsetlinewidth{1.003750pt}%
\definecolor{currentstroke}{rgb}{1.000000,0.000000,0.000000}%
\pgfsetstrokecolor{currentstroke}%
\pgfsetdash{}{0pt}%
\pgfpathmoveto{\pgfqpoint{12.940652in}{8.318719in}}%
\pgfpathcurveto{\pgfqpoint{12.948888in}{8.318719in}}{\pgfqpoint{12.956788in}{8.321992in}}{\pgfqpoint{12.962612in}{8.327816in}}%
\pgfpathcurveto{\pgfqpoint{12.968436in}{8.333640in}}{\pgfqpoint{12.971708in}{8.341540in}}{\pgfqpoint{12.971708in}{8.349776in}}%
\pgfpathcurveto{\pgfqpoint{12.971708in}{8.358012in}}{\pgfqpoint{12.968436in}{8.365912in}}{\pgfqpoint{12.962612in}{8.371736in}}%
\pgfpathcurveto{\pgfqpoint{12.956788in}{8.377560in}}{\pgfqpoint{12.948888in}{8.380832in}}{\pgfqpoint{12.940652in}{8.380832in}}%
\pgfpathcurveto{\pgfqpoint{12.932415in}{8.380832in}}{\pgfqpoint{12.924515in}{8.377560in}}{\pgfqpoint{12.918691in}{8.371736in}}%
\pgfpathcurveto{\pgfqpoint{12.912867in}{8.365912in}}{\pgfqpoint{12.909595in}{8.358012in}}{\pgfqpoint{12.909595in}{8.349776in}}%
\pgfpathcurveto{\pgfqpoint{12.909595in}{8.341540in}}{\pgfqpoint{12.912867in}{8.333640in}}{\pgfqpoint{12.918691in}{8.327816in}}%
\pgfpathcurveto{\pgfqpoint{12.924515in}{8.321992in}}{\pgfqpoint{12.932415in}{8.318719in}}{\pgfqpoint{12.940652in}{8.318719in}}%
\pgfusepath{stroke}%
\end{pgfscope}%
\begin{pgfscope}%
\pgfpathrectangle{\pgfqpoint{3.788192in}{2.980138in}}{\pgfqpoint{2.914000in}{2.171400in}}%
\pgfusepath{clip}%
\pgfsetbuttcap%
\pgfsetroundjoin%
\pgfsetlinewidth{1.003750pt}%
\definecolor{currentstroke}{rgb}{1.000000,0.000000,0.000000}%
\pgfsetstrokecolor{currentstroke}%
\pgfsetdash{}{0pt}%
\pgfpathmoveto{\pgfqpoint{12.936873in}{8.298012in}}%
\pgfpathcurveto{\pgfqpoint{12.945109in}{8.298012in}}{\pgfqpoint{12.953009in}{8.301284in}}{\pgfqpoint{12.958833in}{8.307108in}}%
\pgfpathcurveto{\pgfqpoint{12.964657in}{8.312932in}}{\pgfqpoint{12.967929in}{8.320832in}}{\pgfqpoint{12.967929in}{8.329068in}}%
\pgfpathcurveto{\pgfqpoint{12.967929in}{8.337305in}}{\pgfqpoint{12.964657in}{8.345205in}}{\pgfqpoint{12.958833in}{8.351028in}}%
\pgfpathcurveto{\pgfqpoint{12.953009in}{8.356852in}}{\pgfqpoint{12.945109in}{8.360125in}}{\pgfqpoint{12.936873in}{8.360125in}}%
\pgfpathcurveto{\pgfqpoint{12.928637in}{8.360125in}}{\pgfqpoint{12.920737in}{8.356852in}}{\pgfqpoint{12.914913in}{8.351028in}}%
\pgfpathcurveto{\pgfqpoint{12.909089in}{8.345205in}}{\pgfqpoint{12.905816in}{8.337305in}}{\pgfqpoint{12.905816in}{8.329068in}}%
\pgfpathcurveto{\pgfqpoint{12.905816in}{8.320832in}}{\pgfqpoint{12.909089in}{8.312932in}}{\pgfqpoint{12.914913in}{8.307108in}}%
\pgfpathcurveto{\pgfqpoint{12.920737in}{8.301284in}}{\pgfqpoint{12.928637in}{8.298012in}}{\pgfqpoint{12.936873in}{8.298012in}}%
\pgfusepath{stroke}%
\end{pgfscope}%
\begin{pgfscope}%
\pgfpathrectangle{\pgfqpoint{3.788192in}{2.980138in}}{\pgfqpoint{2.914000in}{2.171400in}}%
\pgfusepath{clip}%
\pgfsetbuttcap%
\pgfsetroundjoin%
\pgfsetlinewidth{1.003750pt}%
\definecolor{currentstroke}{rgb}{1.000000,0.000000,0.000000}%
\pgfsetstrokecolor{currentstroke}%
\pgfsetdash{}{0pt}%
\pgfpathmoveto{\pgfqpoint{12.900060in}{6.352602in}}%
\pgfpathcurveto{\pgfqpoint{12.908296in}{6.352602in}}{\pgfqpoint{12.916196in}{6.355874in}}{\pgfqpoint{12.922020in}{6.361698in}}%
\pgfpathcurveto{\pgfqpoint{12.927844in}{6.367522in}}{\pgfqpoint{12.931116in}{6.375422in}}{\pgfqpoint{12.931116in}{6.383658in}}%
\pgfpathcurveto{\pgfqpoint{12.931116in}{6.391894in}}{\pgfqpoint{12.927844in}{6.399794in}}{\pgfqpoint{12.922020in}{6.405618in}}%
\pgfpathcurveto{\pgfqpoint{12.916196in}{6.411442in}}{\pgfqpoint{12.908296in}{6.414715in}}{\pgfqpoint{12.900060in}{6.414715in}}%
\pgfpathcurveto{\pgfqpoint{12.891823in}{6.414715in}}{\pgfqpoint{12.883923in}{6.411442in}}{\pgfqpoint{12.878099in}{6.405618in}}%
\pgfpathcurveto{\pgfqpoint{12.872275in}{6.399794in}}{\pgfqpoint{12.869003in}{6.391894in}}{\pgfqpoint{12.869003in}{6.383658in}}%
\pgfpathcurveto{\pgfqpoint{12.869003in}{6.375422in}}{\pgfqpoint{12.872275in}{6.367522in}}{\pgfqpoint{12.878099in}{6.361698in}}%
\pgfpathcurveto{\pgfqpoint{12.883923in}{6.355874in}}{\pgfqpoint{12.891823in}{6.352602in}}{\pgfqpoint{12.900060in}{6.352602in}}%
\pgfusepath{stroke}%
\end{pgfscope}%
\begin{pgfscope}%
\pgfpathrectangle{\pgfqpoint{3.788192in}{2.980138in}}{\pgfqpoint{2.914000in}{2.171400in}}%
\pgfusepath{clip}%
\pgfsetbuttcap%
\pgfsetroundjoin%
\pgfsetlinewidth{1.003750pt}%
\definecolor{currentstroke}{rgb}{1.000000,0.000000,0.000000}%
\pgfsetstrokecolor{currentstroke}%
\pgfsetdash{}{0pt}%
\pgfpathmoveto{\pgfqpoint{12.861075in}{6.367856in}}%
\pgfpathcurveto{\pgfqpoint{12.869311in}{6.367856in}}{\pgfqpoint{12.877211in}{6.371128in}}{\pgfqpoint{12.883035in}{6.376952in}}%
\pgfpathcurveto{\pgfqpoint{12.888859in}{6.382776in}}{\pgfqpoint{12.892131in}{6.390676in}}{\pgfqpoint{12.892131in}{6.398913in}}%
\pgfpathcurveto{\pgfqpoint{12.892131in}{6.407149in}}{\pgfqpoint{12.888859in}{6.415049in}}{\pgfqpoint{12.883035in}{6.420873in}}%
\pgfpathcurveto{\pgfqpoint{12.877211in}{6.426697in}}{\pgfqpoint{12.869311in}{6.429969in}}{\pgfqpoint{12.861075in}{6.429969in}}%
\pgfpathcurveto{\pgfqpoint{12.852838in}{6.429969in}}{\pgfqpoint{12.844938in}{6.426697in}}{\pgfqpoint{12.839114in}{6.420873in}}%
\pgfpathcurveto{\pgfqpoint{12.833291in}{6.415049in}}{\pgfqpoint{12.830018in}{6.407149in}}{\pgfqpoint{12.830018in}{6.398913in}}%
\pgfpathcurveto{\pgfqpoint{12.830018in}{6.390676in}}{\pgfqpoint{12.833291in}{6.382776in}}{\pgfqpoint{12.839114in}{6.376952in}}%
\pgfpathcurveto{\pgfqpoint{12.844938in}{6.371128in}}{\pgfqpoint{12.852838in}{6.367856in}}{\pgfqpoint{12.861075in}{6.367856in}}%
\pgfusepath{stroke}%
\end{pgfscope}%
\begin{pgfscope}%
\pgfpathrectangle{\pgfqpoint{3.788192in}{2.980138in}}{\pgfqpoint{2.914000in}{2.171400in}}%
\pgfusepath{clip}%
\pgfsetbuttcap%
\pgfsetroundjoin%
\pgfsetlinewidth{1.003750pt}%
\definecolor{currentstroke}{rgb}{1.000000,0.000000,0.000000}%
\pgfsetstrokecolor{currentstroke}%
\pgfsetdash{}{0pt}%
\pgfpathmoveto{\pgfqpoint{15.347266in}{8.054762in}}%
\pgfpathcurveto{\pgfqpoint{15.355502in}{8.054762in}}{\pgfqpoint{15.363402in}{8.058034in}}{\pgfqpoint{15.369226in}{8.063858in}}%
\pgfpathcurveto{\pgfqpoint{15.375050in}{8.069682in}}{\pgfqpoint{15.378323in}{8.077582in}}{\pgfqpoint{15.378323in}{8.085818in}}%
\pgfpathcurveto{\pgfqpoint{15.378323in}{8.094054in}}{\pgfqpoint{15.375050in}{8.101954in}}{\pgfqpoint{15.369226in}{8.107778in}}%
\pgfpathcurveto{\pgfqpoint{15.363402in}{8.113602in}}{\pgfqpoint{15.355502in}{8.116875in}}{\pgfqpoint{15.347266in}{8.116875in}}%
\pgfpathcurveto{\pgfqpoint{15.339030in}{8.116875in}}{\pgfqpoint{15.331130in}{8.113602in}}{\pgfqpoint{15.325306in}{8.107778in}}%
\pgfpathcurveto{\pgfqpoint{15.319482in}{8.101954in}}{\pgfqpoint{15.316210in}{8.094054in}}{\pgfqpoint{15.316210in}{8.085818in}}%
\pgfpathcurveto{\pgfqpoint{15.316210in}{8.077582in}}{\pgfqpoint{15.319482in}{8.069682in}}{\pgfqpoint{15.325306in}{8.063858in}}%
\pgfpathcurveto{\pgfqpoint{15.331130in}{8.058034in}}{\pgfqpoint{15.339030in}{8.054762in}}{\pgfqpoint{15.347266in}{8.054762in}}%
\pgfusepath{stroke}%
\end{pgfscope}%
\begin{pgfscope}%
\pgfpathrectangle{\pgfqpoint{3.788192in}{2.980138in}}{\pgfqpoint{2.914000in}{2.171400in}}%
\pgfusepath{clip}%
\pgfsetbuttcap%
\pgfsetroundjoin%
\pgfsetlinewidth{1.003750pt}%
\definecolor{currentstroke}{rgb}{1.000000,0.000000,0.000000}%
\pgfsetstrokecolor{currentstroke}%
\pgfsetdash{}{0pt}%
\pgfpathmoveto{\pgfqpoint{12.945472in}{6.264272in}}%
\pgfpathcurveto{\pgfqpoint{12.953708in}{6.264272in}}{\pgfqpoint{12.961608in}{6.267545in}}{\pgfqpoint{12.967432in}{6.273369in}}%
\pgfpathcurveto{\pgfqpoint{12.973256in}{6.279193in}}{\pgfqpoint{12.976528in}{6.287093in}}{\pgfqpoint{12.976528in}{6.295329in}}%
\pgfpathcurveto{\pgfqpoint{12.976528in}{6.303565in}}{\pgfqpoint{12.973256in}{6.311465in}}{\pgfqpoint{12.967432in}{6.317289in}}%
\pgfpathcurveto{\pgfqpoint{12.961608in}{6.323113in}}{\pgfqpoint{12.953708in}{6.326385in}}{\pgfqpoint{12.945472in}{6.326385in}}%
\pgfpathcurveto{\pgfqpoint{12.937236in}{6.326385in}}{\pgfqpoint{12.929336in}{6.323113in}}{\pgfqpoint{12.923512in}{6.317289in}}%
\pgfpathcurveto{\pgfqpoint{12.917688in}{6.311465in}}{\pgfqpoint{12.914415in}{6.303565in}}{\pgfqpoint{12.914415in}{6.295329in}}%
\pgfpathcurveto{\pgfqpoint{12.914415in}{6.287093in}}{\pgfqpoint{12.917688in}{6.279193in}}{\pgfqpoint{12.923512in}{6.273369in}}%
\pgfpathcurveto{\pgfqpoint{12.929336in}{6.267545in}}{\pgfqpoint{12.937236in}{6.264272in}}{\pgfqpoint{12.945472in}{6.264272in}}%
\pgfusepath{stroke}%
\end{pgfscope}%
\begin{pgfscope}%
\pgfpathrectangle{\pgfqpoint{3.788192in}{2.980138in}}{\pgfqpoint{2.914000in}{2.171400in}}%
\pgfusepath{clip}%
\pgfsetbuttcap%
\pgfsetroundjoin%
\pgfsetlinewidth{1.003750pt}%
\definecolor{currentstroke}{rgb}{1.000000,0.000000,0.000000}%
\pgfsetstrokecolor{currentstroke}%
\pgfsetdash{}{0pt}%
\pgfpathmoveto{\pgfqpoint{14.816780in}{6.837016in}}%
\pgfpathcurveto{\pgfqpoint{14.825016in}{6.837016in}}{\pgfqpoint{14.832916in}{6.840288in}}{\pgfqpoint{14.838740in}{6.846112in}}%
\pgfpathcurveto{\pgfqpoint{14.844564in}{6.851936in}}{\pgfqpoint{14.847836in}{6.859836in}}{\pgfqpoint{14.847836in}{6.868073in}}%
\pgfpathcurveto{\pgfqpoint{14.847836in}{6.876309in}}{\pgfqpoint{14.844564in}{6.884209in}}{\pgfqpoint{14.838740in}{6.890033in}}%
\pgfpathcurveto{\pgfqpoint{14.832916in}{6.895857in}}{\pgfqpoint{14.825016in}{6.899129in}}{\pgfqpoint{14.816780in}{6.899129in}}%
\pgfpathcurveto{\pgfqpoint{14.808543in}{6.899129in}}{\pgfqpoint{14.800643in}{6.895857in}}{\pgfqpoint{14.794819in}{6.890033in}}%
\pgfpathcurveto{\pgfqpoint{14.788995in}{6.884209in}}{\pgfqpoint{14.785723in}{6.876309in}}{\pgfqpoint{14.785723in}{6.868073in}}%
\pgfpathcurveto{\pgfqpoint{14.785723in}{6.859836in}}{\pgfqpoint{14.788995in}{6.851936in}}{\pgfqpoint{14.794819in}{6.846112in}}%
\pgfpathcurveto{\pgfqpoint{14.800643in}{6.840288in}}{\pgfqpoint{14.808543in}{6.837016in}}{\pgfqpoint{14.816780in}{6.837016in}}%
\pgfusepath{stroke}%
\end{pgfscope}%
\begin{pgfscope}%
\pgfpathrectangle{\pgfqpoint{3.788192in}{2.980138in}}{\pgfqpoint{2.914000in}{2.171400in}}%
\pgfusepath{clip}%
\pgfsetbuttcap%
\pgfsetroundjoin%
\pgfsetlinewidth{1.003750pt}%
\definecolor{currentstroke}{rgb}{1.000000,0.000000,0.000000}%
\pgfsetstrokecolor{currentstroke}%
\pgfsetdash{}{0pt}%
\pgfpathmoveto{\pgfqpoint{15.389681in}{6.512387in}}%
\pgfpathcurveto{\pgfqpoint{15.397917in}{6.512387in}}{\pgfqpoint{15.405817in}{6.515659in}}{\pgfqpoint{15.411641in}{6.521483in}}%
\pgfpathcurveto{\pgfqpoint{15.417465in}{6.527307in}}{\pgfqpoint{15.420738in}{6.535207in}}{\pgfqpoint{15.420738in}{6.543443in}}%
\pgfpathcurveto{\pgfqpoint{15.420738in}{6.551679in}}{\pgfqpoint{15.417465in}{6.559579in}}{\pgfqpoint{15.411641in}{6.565403in}}%
\pgfpathcurveto{\pgfqpoint{15.405817in}{6.571227in}}{\pgfqpoint{15.397917in}{6.574500in}}{\pgfqpoint{15.389681in}{6.574500in}}%
\pgfpathcurveto{\pgfqpoint{15.381445in}{6.574500in}}{\pgfqpoint{15.373545in}{6.571227in}}{\pgfqpoint{15.367721in}{6.565403in}}%
\pgfpathcurveto{\pgfqpoint{15.361897in}{6.559579in}}{\pgfqpoint{15.358625in}{6.551679in}}{\pgfqpoint{15.358625in}{6.543443in}}%
\pgfpathcurveto{\pgfqpoint{15.358625in}{6.535207in}}{\pgfqpoint{15.361897in}{6.527307in}}{\pgfqpoint{15.367721in}{6.521483in}}%
\pgfpathcurveto{\pgfqpoint{15.373545in}{6.515659in}}{\pgfqpoint{15.381445in}{6.512387in}}{\pgfqpoint{15.389681in}{6.512387in}}%
\pgfusepath{stroke}%
\end{pgfscope}%
\begin{pgfscope}%
\pgfpathrectangle{\pgfqpoint{3.788192in}{2.980138in}}{\pgfqpoint{2.914000in}{2.171400in}}%
\pgfusepath{clip}%
\pgfsetbuttcap%
\pgfsetroundjoin%
\pgfsetlinewidth{1.003750pt}%
\definecolor{currentstroke}{rgb}{1.000000,0.000000,0.000000}%
\pgfsetstrokecolor{currentstroke}%
\pgfsetdash{}{0pt}%
\pgfpathmoveto{\pgfqpoint{12.707501in}{7.095383in}}%
\pgfpathcurveto{\pgfqpoint{12.715737in}{7.095383in}}{\pgfqpoint{12.723637in}{7.098656in}}{\pgfqpoint{12.729461in}{7.104479in}}%
\pgfpathcurveto{\pgfqpoint{12.735285in}{7.110303in}}{\pgfqpoint{12.738557in}{7.118203in}}{\pgfqpoint{12.738557in}{7.126440in}}%
\pgfpathcurveto{\pgfqpoint{12.738557in}{7.134676in}}{\pgfqpoint{12.735285in}{7.142576in}}{\pgfqpoint{12.729461in}{7.148400in}}%
\pgfpathcurveto{\pgfqpoint{12.723637in}{7.154224in}}{\pgfqpoint{12.715737in}{7.157496in}}{\pgfqpoint{12.707501in}{7.157496in}}%
\pgfpathcurveto{\pgfqpoint{12.699265in}{7.157496in}}{\pgfqpoint{12.691365in}{7.154224in}}{\pgfqpoint{12.685541in}{7.148400in}}%
\pgfpathcurveto{\pgfqpoint{12.679717in}{7.142576in}}{\pgfqpoint{12.676444in}{7.134676in}}{\pgfqpoint{12.676444in}{7.126440in}}%
\pgfpathcurveto{\pgfqpoint{12.676444in}{7.118203in}}{\pgfqpoint{12.679717in}{7.110303in}}{\pgfqpoint{12.685541in}{7.104479in}}%
\pgfpathcurveto{\pgfqpoint{12.691365in}{7.098656in}}{\pgfqpoint{12.699265in}{7.095383in}}{\pgfqpoint{12.707501in}{7.095383in}}%
\pgfusepath{stroke}%
\end{pgfscope}%
\begin{pgfscope}%
\pgfpathrectangle{\pgfqpoint{3.788192in}{2.980138in}}{\pgfqpoint{2.914000in}{2.171400in}}%
\pgfusepath{clip}%
\pgfsetbuttcap%
\pgfsetroundjoin%
\pgfsetlinewidth{1.003750pt}%
\definecolor{currentstroke}{rgb}{1.000000,0.000000,0.000000}%
\pgfsetstrokecolor{currentstroke}%
\pgfsetdash{}{0pt}%
\pgfpathmoveto{\pgfqpoint{14.362942in}{5.990566in}}%
\pgfpathcurveto{\pgfqpoint{14.371178in}{5.990566in}}{\pgfqpoint{14.379078in}{5.993838in}}{\pgfqpoint{14.384902in}{5.999662in}}%
\pgfpathcurveto{\pgfqpoint{14.390726in}{6.005486in}}{\pgfqpoint{14.393998in}{6.013386in}}{\pgfqpoint{14.393998in}{6.021622in}}%
\pgfpathcurveto{\pgfqpoint{14.393998in}{6.029859in}}{\pgfqpoint{14.390726in}{6.037759in}}{\pgfqpoint{14.384902in}{6.043583in}}%
\pgfpathcurveto{\pgfqpoint{14.379078in}{6.049407in}}{\pgfqpoint{14.371178in}{6.052679in}}{\pgfqpoint{14.362942in}{6.052679in}}%
\pgfpathcurveto{\pgfqpoint{14.354705in}{6.052679in}}{\pgfqpoint{14.346805in}{6.049407in}}{\pgfqpoint{14.340982in}{6.043583in}}%
\pgfpathcurveto{\pgfqpoint{14.335158in}{6.037759in}}{\pgfqpoint{14.331885in}{6.029859in}}{\pgfqpoint{14.331885in}{6.021622in}}%
\pgfpathcurveto{\pgfqpoint{14.331885in}{6.013386in}}{\pgfqpoint{14.335158in}{6.005486in}}{\pgfqpoint{14.340982in}{5.999662in}}%
\pgfpathcurveto{\pgfqpoint{14.346805in}{5.993838in}}{\pgfqpoint{14.354705in}{5.990566in}}{\pgfqpoint{14.362942in}{5.990566in}}%
\pgfusepath{stroke}%
\end{pgfscope}%
\begin{pgfscope}%
\pgfpathrectangle{\pgfqpoint{3.788192in}{2.980138in}}{\pgfqpoint{2.914000in}{2.171400in}}%
\pgfusepath{clip}%
\pgfsetbuttcap%
\pgfsetroundjoin%
\pgfsetlinewidth{1.003750pt}%
\definecolor{currentstroke}{rgb}{1.000000,0.000000,0.000000}%
\pgfsetstrokecolor{currentstroke}%
\pgfsetdash{}{0pt}%
\pgfpathmoveto{\pgfqpoint{14.985061in}{5.541293in}}%
\pgfpathcurveto{\pgfqpoint{14.993297in}{5.541293in}}{\pgfqpoint{15.001197in}{5.544566in}}{\pgfqpoint{15.007021in}{5.550390in}}%
\pgfpathcurveto{\pgfqpoint{15.012845in}{5.556213in}}{\pgfqpoint{15.016118in}{5.564113in}}{\pgfqpoint{15.016118in}{5.572350in}}%
\pgfpathcurveto{\pgfqpoint{15.016118in}{5.580586in}}{\pgfqpoint{15.012845in}{5.588486in}}{\pgfqpoint{15.007021in}{5.594310in}}%
\pgfpathcurveto{\pgfqpoint{15.001197in}{5.600134in}}{\pgfqpoint{14.993297in}{5.603406in}}{\pgfqpoint{14.985061in}{5.603406in}}%
\pgfpathcurveto{\pgfqpoint{14.976825in}{5.603406in}}{\pgfqpoint{14.968925in}{5.600134in}}{\pgfqpoint{14.963101in}{5.594310in}}%
\pgfpathcurveto{\pgfqpoint{14.957277in}{5.588486in}}{\pgfqpoint{14.954005in}{5.580586in}}{\pgfqpoint{14.954005in}{5.572350in}}%
\pgfpathcurveto{\pgfqpoint{14.954005in}{5.564113in}}{\pgfqpoint{14.957277in}{5.556213in}}{\pgfqpoint{14.963101in}{5.550390in}}%
\pgfpathcurveto{\pgfqpoint{14.968925in}{5.544566in}}{\pgfqpoint{14.976825in}{5.541293in}}{\pgfqpoint{14.985061in}{5.541293in}}%
\pgfusepath{stroke}%
\end{pgfscope}%
\begin{pgfscope}%
\pgfpathrectangle{\pgfqpoint{3.788192in}{2.980138in}}{\pgfqpoint{2.914000in}{2.171400in}}%
\pgfusepath{clip}%
\pgfsetbuttcap%
\pgfsetroundjoin%
\pgfsetlinewidth{1.003750pt}%
\definecolor{currentstroke}{rgb}{1.000000,0.000000,0.000000}%
\pgfsetstrokecolor{currentstroke}%
\pgfsetdash{}{0pt}%
\pgfpathmoveto{\pgfqpoint{14.968190in}{5.733414in}}%
\pgfpathcurveto{\pgfqpoint{14.976426in}{5.733414in}}{\pgfqpoint{14.984326in}{5.736686in}}{\pgfqpoint{14.990150in}{5.742510in}}%
\pgfpathcurveto{\pgfqpoint{14.995974in}{5.748334in}}{\pgfqpoint{14.999246in}{5.756234in}}{\pgfqpoint{14.999246in}{5.764470in}}%
\pgfpathcurveto{\pgfqpoint{14.999246in}{5.772706in}}{\pgfqpoint{14.995974in}{5.780606in}}{\pgfqpoint{14.990150in}{5.786430in}}%
\pgfpathcurveto{\pgfqpoint{14.984326in}{5.792254in}}{\pgfqpoint{14.976426in}{5.795527in}}{\pgfqpoint{14.968190in}{5.795527in}}%
\pgfpathcurveto{\pgfqpoint{14.959954in}{5.795527in}}{\pgfqpoint{14.952053in}{5.792254in}}{\pgfqpoint{14.946230in}{5.786430in}}%
\pgfpathcurveto{\pgfqpoint{14.940406in}{5.780606in}}{\pgfqpoint{14.937133in}{5.772706in}}{\pgfqpoint{14.937133in}{5.764470in}}%
\pgfpathcurveto{\pgfqpoint{14.937133in}{5.756234in}}{\pgfqpoint{14.940406in}{5.748334in}}{\pgfqpoint{14.946230in}{5.742510in}}%
\pgfpathcurveto{\pgfqpoint{14.952053in}{5.736686in}}{\pgfqpoint{14.959954in}{5.733414in}}{\pgfqpoint{14.968190in}{5.733414in}}%
\pgfusepath{stroke}%
\end{pgfscope}%
\begin{pgfscope}%
\pgfpathrectangle{\pgfqpoint{3.788192in}{2.980138in}}{\pgfqpoint{2.914000in}{2.171400in}}%
\pgfusepath{clip}%
\pgfsetbuttcap%
\pgfsetroundjoin%
\pgfsetlinewidth{1.003750pt}%
\definecolor{currentstroke}{rgb}{1.000000,0.000000,0.000000}%
\pgfsetstrokecolor{currentstroke}%
\pgfsetdash{}{0pt}%
\pgfpathmoveto{\pgfqpoint{12.713917in}{7.026179in}}%
\pgfpathcurveto{\pgfqpoint{12.722154in}{7.026179in}}{\pgfqpoint{12.730054in}{7.029451in}}{\pgfqpoint{12.735878in}{7.035275in}}%
\pgfpathcurveto{\pgfqpoint{12.741701in}{7.041099in}}{\pgfqpoint{12.744974in}{7.048999in}}{\pgfqpoint{12.744974in}{7.057235in}}%
\pgfpathcurveto{\pgfqpoint{12.744974in}{7.065472in}}{\pgfqpoint{12.741701in}{7.073372in}}{\pgfqpoint{12.735878in}{7.079196in}}%
\pgfpathcurveto{\pgfqpoint{12.730054in}{7.085019in}}{\pgfqpoint{12.722154in}{7.088292in}}{\pgfqpoint{12.713917in}{7.088292in}}%
\pgfpathcurveto{\pgfqpoint{12.705681in}{7.088292in}}{\pgfqpoint{12.697781in}{7.085019in}}{\pgfqpoint{12.691957in}{7.079196in}}%
\pgfpathcurveto{\pgfqpoint{12.686133in}{7.073372in}}{\pgfqpoint{12.682861in}{7.065472in}}{\pgfqpoint{12.682861in}{7.057235in}}%
\pgfpathcurveto{\pgfqpoint{12.682861in}{7.048999in}}{\pgfqpoint{12.686133in}{7.041099in}}{\pgfqpoint{12.691957in}{7.035275in}}%
\pgfpathcurveto{\pgfqpoint{12.697781in}{7.029451in}}{\pgfqpoint{12.705681in}{7.026179in}}{\pgfqpoint{12.713917in}{7.026179in}}%
\pgfusepath{stroke}%
\end{pgfscope}%
\begin{pgfscope}%
\pgfpathrectangle{\pgfqpoint{3.788192in}{2.980138in}}{\pgfqpoint{2.914000in}{2.171400in}}%
\pgfusepath{clip}%
\pgfsetbuttcap%
\pgfsetroundjoin%
\pgfsetlinewidth{1.003750pt}%
\definecolor{currentstroke}{rgb}{1.000000,0.000000,0.000000}%
\pgfsetstrokecolor{currentstroke}%
\pgfsetdash{}{0pt}%
\pgfpathmoveto{\pgfqpoint{13.558784in}{8.054051in}}%
\pgfpathcurveto{\pgfqpoint{13.567020in}{8.054051in}}{\pgfqpoint{13.574920in}{8.057323in}}{\pgfqpoint{13.580744in}{8.063147in}}%
\pgfpathcurveto{\pgfqpoint{13.586568in}{8.068971in}}{\pgfqpoint{13.589840in}{8.076871in}}{\pgfqpoint{13.589840in}{8.085108in}}%
\pgfpathcurveto{\pgfqpoint{13.589840in}{8.093344in}}{\pgfqpoint{13.586568in}{8.101244in}}{\pgfqpoint{13.580744in}{8.107068in}}%
\pgfpathcurveto{\pgfqpoint{13.574920in}{8.112892in}}{\pgfqpoint{13.567020in}{8.116164in}}{\pgfqpoint{13.558784in}{8.116164in}}%
\pgfpathcurveto{\pgfqpoint{13.550548in}{8.116164in}}{\pgfqpoint{13.542648in}{8.112892in}}{\pgfqpoint{13.536824in}{8.107068in}}%
\pgfpathcurveto{\pgfqpoint{13.531000in}{8.101244in}}{\pgfqpoint{13.527727in}{8.093344in}}{\pgfqpoint{13.527727in}{8.085108in}}%
\pgfpathcurveto{\pgfqpoint{13.527727in}{8.076871in}}{\pgfqpoint{13.531000in}{8.068971in}}{\pgfqpoint{13.536824in}{8.063147in}}%
\pgfpathcurveto{\pgfqpoint{13.542648in}{8.057323in}}{\pgfqpoint{13.550548in}{8.054051in}}{\pgfqpoint{13.558784in}{8.054051in}}%
\pgfusepath{stroke}%
\end{pgfscope}%
\begin{pgfscope}%
\pgfpathrectangle{\pgfqpoint{3.788192in}{2.980138in}}{\pgfqpoint{2.914000in}{2.171400in}}%
\pgfusepath{clip}%
\pgfsetbuttcap%
\pgfsetroundjoin%
\pgfsetlinewidth{1.003750pt}%
\definecolor{currentstroke}{rgb}{1.000000,0.000000,0.000000}%
\pgfsetstrokecolor{currentstroke}%
\pgfsetdash{}{0pt}%
\pgfpathmoveto{\pgfqpoint{13.021061in}{8.119261in}}%
\pgfpathcurveto{\pgfqpoint{13.029297in}{8.119261in}}{\pgfqpoint{13.037197in}{8.122533in}}{\pgfqpoint{13.043021in}{8.128357in}}%
\pgfpathcurveto{\pgfqpoint{13.048845in}{8.134181in}}{\pgfqpoint{13.052117in}{8.142081in}}{\pgfqpoint{13.052117in}{8.150317in}}%
\pgfpathcurveto{\pgfqpoint{13.052117in}{8.158554in}}{\pgfqpoint{13.048845in}{8.166454in}}{\pgfqpoint{13.043021in}{8.172278in}}%
\pgfpathcurveto{\pgfqpoint{13.037197in}{8.178102in}}{\pgfqpoint{13.029297in}{8.181374in}}{\pgfqpoint{13.021061in}{8.181374in}}%
\pgfpathcurveto{\pgfqpoint{13.012825in}{8.181374in}}{\pgfqpoint{13.004925in}{8.178102in}}{\pgfqpoint{12.999101in}{8.172278in}}%
\pgfpathcurveto{\pgfqpoint{12.993277in}{8.166454in}}{\pgfqpoint{12.990004in}{8.158554in}}{\pgfqpoint{12.990004in}{8.150317in}}%
\pgfpathcurveto{\pgfqpoint{12.990004in}{8.142081in}}{\pgfqpoint{12.993277in}{8.134181in}}{\pgfqpoint{12.999101in}{8.128357in}}%
\pgfpathcurveto{\pgfqpoint{13.004925in}{8.122533in}}{\pgfqpoint{13.012825in}{8.119261in}}{\pgfqpoint{13.021061in}{8.119261in}}%
\pgfusepath{stroke}%
\end{pgfscope}%
\begin{pgfscope}%
\pgfpathrectangle{\pgfqpoint{3.788192in}{2.980138in}}{\pgfqpoint{2.914000in}{2.171400in}}%
\pgfusepath{clip}%
\pgfsetbuttcap%
\pgfsetroundjoin%
\pgfsetlinewidth{1.003750pt}%
\definecolor{currentstroke}{rgb}{1.000000,0.000000,0.000000}%
\pgfsetstrokecolor{currentstroke}%
\pgfsetdash{}{0pt}%
\pgfpathmoveto{\pgfqpoint{12.649080in}{7.078889in}}%
\pgfpathcurveto{\pgfqpoint{12.657316in}{7.078889in}}{\pgfqpoint{12.665216in}{7.082161in}}{\pgfqpoint{12.671040in}{7.087985in}}%
\pgfpathcurveto{\pgfqpoint{12.676864in}{7.093809in}}{\pgfqpoint{12.680136in}{7.101709in}}{\pgfqpoint{12.680136in}{7.109946in}}%
\pgfpathcurveto{\pgfqpoint{12.680136in}{7.118182in}}{\pgfqpoint{12.676864in}{7.126082in}}{\pgfqpoint{12.671040in}{7.131906in}}%
\pgfpathcurveto{\pgfqpoint{12.665216in}{7.137730in}}{\pgfqpoint{12.657316in}{7.141002in}}{\pgfqpoint{12.649080in}{7.141002in}}%
\pgfpathcurveto{\pgfqpoint{12.640843in}{7.141002in}}{\pgfqpoint{12.632943in}{7.137730in}}{\pgfqpoint{12.627119in}{7.131906in}}%
\pgfpathcurveto{\pgfqpoint{12.621295in}{7.126082in}}{\pgfqpoint{12.618023in}{7.118182in}}{\pgfqpoint{12.618023in}{7.109946in}}%
\pgfpathcurveto{\pgfqpoint{12.618023in}{7.101709in}}{\pgfqpoint{12.621295in}{7.093809in}}{\pgfqpoint{12.627119in}{7.087985in}}%
\pgfpathcurveto{\pgfqpoint{12.632943in}{7.082161in}}{\pgfqpoint{12.640843in}{7.078889in}}{\pgfqpoint{12.649080in}{7.078889in}}%
\pgfusepath{stroke}%
\end{pgfscope}%
\begin{pgfscope}%
\pgfpathrectangle{\pgfqpoint{3.788192in}{2.980138in}}{\pgfqpoint{2.914000in}{2.171400in}}%
\pgfusepath{clip}%
\pgfsetbuttcap%
\pgfsetroundjoin%
\pgfsetlinewidth{1.003750pt}%
\definecolor{currentstroke}{rgb}{1.000000,0.000000,0.000000}%
\pgfsetstrokecolor{currentstroke}%
\pgfsetdash{}{0pt}%
\pgfpathmoveto{\pgfqpoint{4.530215in}{4.056056in}}%
\pgfpathcurveto{\pgfqpoint{4.538452in}{4.056056in}}{\pgfqpoint{4.546352in}{4.059328in}}{\pgfqpoint{4.552176in}{4.065152in}}%
\pgfpathcurveto{\pgfqpoint{4.557999in}{4.070976in}}{\pgfqpoint{4.561272in}{4.078876in}}{\pgfqpoint{4.561272in}{4.087112in}}%
\pgfpathcurveto{\pgfqpoint{4.561272in}{4.095349in}}{\pgfqpoint{4.557999in}{4.103249in}}{\pgfqpoint{4.552176in}{4.109073in}}%
\pgfpathcurveto{\pgfqpoint{4.546352in}{4.114897in}}{\pgfqpoint{4.538452in}{4.118169in}}{\pgfqpoint{4.530215in}{4.118169in}}%
\pgfpathcurveto{\pgfqpoint{4.521979in}{4.118169in}}{\pgfqpoint{4.514079in}{4.114897in}}{\pgfqpoint{4.508255in}{4.109073in}}%
\pgfpathcurveto{\pgfqpoint{4.502431in}{4.103249in}}{\pgfqpoint{4.499159in}{4.095349in}}{\pgfqpoint{4.499159in}{4.087112in}}%
\pgfpathcurveto{\pgfqpoint{4.499159in}{4.078876in}}{\pgfqpoint{4.502431in}{4.070976in}}{\pgfqpoint{4.508255in}{4.065152in}}%
\pgfpathcurveto{\pgfqpoint{4.514079in}{4.059328in}}{\pgfqpoint{4.521979in}{4.056056in}}{\pgfqpoint{4.530215in}{4.056056in}}%
\pgfpathlineto{\pgfqpoint{4.530215in}{4.056056in}}%
\pgfpathclose%
\pgfusepath{stroke}%
\end{pgfscope}%
\begin{pgfscope}%
\pgfpathrectangle{\pgfqpoint{3.788192in}{2.980138in}}{\pgfqpoint{2.914000in}{2.171400in}}%
\pgfusepath{clip}%
\pgfsetbuttcap%
\pgfsetmiterjoin%
\definecolor{currentfill}{rgb}{0.839216,0.152941,0.156863}%
\pgfsetfillcolor{currentfill}%
\pgfsetfillopacity{0.200000}%
\pgfsetlinewidth{1.003750pt}%
\definecolor{currentstroke}{rgb}{0.839216,0.152941,0.156863}%
\pgfsetstrokecolor{currentstroke}%
\pgfsetstrokeopacity{0.200000}%
\pgfsetdash{}{0pt}%
\pgfpathmoveto{\pgfqpoint{4.530215in}{2.980138in}}%
\pgfpathlineto{\pgfqpoint{24.162152in}{2.980138in}}%
\pgfpathlineto{\pgfqpoint{24.162152in}{5.151538in}}%
\pgfpathlineto{\pgfqpoint{4.530215in}{5.151538in}}%
\pgfpathlineto{\pgfqpoint{4.530215in}{2.980138in}}%
\pgfpathclose%
\pgfusepath{stroke,fill}%
\end{pgfscope}%
\begin{pgfscope}%
\pgfsetbuttcap%
\pgfsetmiterjoin%
\definecolor{currentfill}{rgb}{0.839216,0.152941,0.156863}%
\pgfsetfillcolor{currentfill}%
\pgfsetfillopacity{0.200000}%
\pgfsetlinewidth{1.003750pt}%
\definecolor{currentstroke}{rgb}{0.839216,0.152941,0.156863}%
\pgfsetstrokecolor{currentstroke}%
\pgfsetstrokeopacity{0.200000}%
\pgfsetdash{}{0pt}%
\pgfpathrectangle{\pgfqpoint{3.788192in}{2.980138in}}{\pgfqpoint{2.914000in}{2.171400in}}%
\pgfusepath{clip}%
\pgfpathmoveto{\pgfqpoint{4.530215in}{2.980138in}}%
\pgfpathlineto{\pgfqpoint{24.162152in}{2.980138in}}%
\pgfpathlineto{\pgfqpoint{24.162152in}{5.151538in}}%
\pgfpathlineto{\pgfqpoint{4.530215in}{5.151538in}}%
\pgfpathlineto{\pgfqpoint{4.530215in}{2.980138in}}%
\pgfpathclose%
\pgfusepath{clip}%
\pgfsys@defobject{currentpattern}{\pgfqpoint{0in}{0in}}{\pgfqpoint{1in}{1in}}{%
\begin{pgfscope}%
\pgfpathrectangle{\pgfqpoint{0in}{0in}}{\pgfqpoint{1in}{1in}}%
\pgfusepath{clip}%
\pgfpathmoveto{\pgfqpoint{-0.500000in}{0.500000in}}%
\pgfpathlineto{\pgfqpoint{0.500000in}{1.500000in}}%
\pgfpathmoveto{\pgfqpoint{-0.333333in}{0.333333in}}%
\pgfpathlineto{\pgfqpoint{0.666667in}{1.333333in}}%
\pgfpathmoveto{\pgfqpoint{-0.166667in}{0.166667in}}%
\pgfpathlineto{\pgfqpoint{0.833333in}{1.166667in}}%
\pgfpathmoveto{\pgfqpoint{0.000000in}{0.000000in}}%
\pgfpathlineto{\pgfqpoint{1.000000in}{1.000000in}}%
\pgfpathmoveto{\pgfqpoint{0.166667in}{-0.166667in}}%
\pgfpathlineto{\pgfqpoint{1.166667in}{0.833333in}}%
\pgfpathmoveto{\pgfqpoint{0.333333in}{-0.333333in}}%
\pgfpathlineto{\pgfqpoint{1.333333in}{0.666667in}}%
\pgfpathmoveto{\pgfqpoint{0.500000in}{-0.500000in}}%
\pgfpathlineto{\pgfqpoint{1.500000in}{0.500000in}}%
\pgfusepath{stroke}%
\end{pgfscope}%
}%
\pgfsys@transformshift{4.530215in}{2.980138in}%
\pgfsys@useobject{currentpattern}{}%
\pgfsys@transformshift{1in}{0in}%
\pgfsys@useobject{currentpattern}{}%
\pgfsys@transformshift{1in}{0in}%
\pgfsys@useobject{currentpattern}{}%
\pgfsys@transformshift{1in}{0in}%
\pgfsys@useobject{currentpattern}{}%
\pgfsys@transformshift{1in}{0in}%
\pgfsys@useobject{currentpattern}{}%
\pgfsys@transformshift{1in}{0in}%
\pgfsys@useobject{currentpattern}{}%
\pgfsys@transformshift{1in}{0in}%
\pgfsys@useobject{currentpattern}{}%
\pgfsys@transformshift{1in}{0in}%
\pgfsys@useobject{currentpattern}{}%
\pgfsys@transformshift{1in}{0in}%
\pgfsys@useobject{currentpattern}{}%
\pgfsys@transformshift{1in}{0in}%
\pgfsys@useobject{currentpattern}{}%
\pgfsys@transformshift{1in}{0in}%
\pgfsys@useobject{currentpattern}{}%
\pgfsys@transformshift{1in}{0in}%
\pgfsys@useobject{currentpattern}{}%
\pgfsys@transformshift{1in}{0in}%
\pgfsys@useobject{currentpattern}{}%
\pgfsys@transformshift{1in}{0in}%
\pgfsys@useobject{currentpattern}{}%
\pgfsys@transformshift{1in}{0in}%
\pgfsys@useobject{currentpattern}{}%
\pgfsys@transformshift{1in}{0in}%
\pgfsys@useobject{currentpattern}{}%
\pgfsys@transformshift{1in}{0in}%
\pgfsys@useobject{currentpattern}{}%
\pgfsys@transformshift{1in}{0in}%
\pgfsys@useobject{currentpattern}{}%
\pgfsys@transformshift{1in}{0in}%
\pgfsys@useobject{currentpattern}{}%
\pgfsys@transformshift{1in}{0in}%
\pgfsys@useobject{currentpattern}{}%
\pgfsys@transformshift{1in}{0in}%
\pgfsys@transformshift{-20in}{0in}%
\pgfsys@transformshift{0in}{1in}%
\pgfsys@useobject{currentpattern}{}%
\pgfsys@transformshift{1in}{0in}%
\pgfsys@useobject{currentpattern}{}%
\pgfsys@transformshift{1in}{0in}%
\pgfsys@useobject{currentpattern}{}%
\pgfsys@transformshift{1in}{0in}%
\pgfsys@useobject{currentpattern}{}%
\pgfsys@transformshift{1in}{0in}%
\pgfsys@useobject{currentpattern}{}%
\pgfsys@transformshift{1in}{0in}%
\pgfsys@useobject{currentpattern}{}%
\pgfsys@transformshift{1in}{0in}%
\pgfsys@useobject{currentpattern}{}%
\pgfsys@transformshift{1in}{0in}%
\pgfsys@useobject{currentpattern}{}%
\pgfsys@transformshift{1in}{0in}%
\pgfsys@useobject{currentpattern}{}%
\pgfsys@transformshift{1in}{0in}%
\pgfsys@useobject{currentpattern}{}%
\pgfsys@transformshift{1in}{0in}%
\pgfsys@useobject{currentpattern}{}%
\pgfsys@transformshift{1in}{0in}%
\pgfsys@useobject{currentpattern}{}%
\pgfsys@transformshift{1in}{0in}%
\pgfsys@useobject{currentpattern}{}%
\pgfsys@transformshift{1in}{0in}%
\pgfsys@useobject{currentpattern}{}%
\pgfsys@transformshift{1in}{0in}%
\pgfsys@useobject{currentpattern}{}%
\pgfsys@transformshift{1in}{0in}%
\pgfsys@useobject{currentpattern}{}%
\pgfsys@transformshift{1in}{0in}%
\pgfsys@useobject{currentpattern}{}%
\pgfsys@transformshift{1in}{0in}%
\pgfsys@useobject{currentpattern}{}%
\pgfsys@transformshift{1in}{0in}%
\pgfsys@useobject{currentpattern}{}%
\pgfsys@transformshift{1in}{0in}%
\pgfsys@useobject{currentpattern}{}%
\pgfsys@transformshift{1in}{0in}%
\pgfsys@transformshift{-20in}{0in}%
\pgfsys@transformshift{0in}{1in}%
\pgfsys@useobject{currentpattern}{}%
\pgfsys@transformshift{1in}{0in}%
\pgfsys@useobject{currentpattern}{}%
\pgfsys@transformshift{1in}{0in}%
\pgfsys@useobject{currentpattern}{}%
\pgfsys@transformshift{1in}{0in}%
\pgfsys@useobject{currentpattern}{}%
\pgfsys@transformshift{1in}{0in}%
\pgfsys@useobject{currentpattern}{}%
\pgfsys@transformshift{1in}{0in}%
\pgfsys@useobject{currentpattern}{}%
\pgfsys@transformshift{1in}{0in}%
\pgfsys@useobject{currentpattern}{}%
\pgfsys@transformshift{1in}{0in}%
\pgfsys@useobject{currentpattern}{}%
\pgfsys@transformshift{1in}{0in}%
\pgfsys@useobject{currentpattern}{}%
\pgfsys@transformshift{1in}{0in}%
\pgfsys@useobject{currentpattern}{}%
\pgfsys@transformshift{1in}{0in}%
\pgfsys@useobject{currentpattern}{}%
\pgfsys@transformshift{1in}{0in}%
\pgfsys@useobject{currentpattern}{}%
\pgfsys@transformshift{1in}{0in}%
\pgfsys@useobject{currentpattern}{}%
\pgfsys@transformshift{1in}{0in}%
\pgfsys@useobject{currentpattern}{}%
\pgfsys@transformshift{1in}{0in}%
\pgfsys@useobject{currentpattern}{}%
\pgfsys@transformshift{1in}{0in}%
\pgfsys@useobject{currentpattern}{}%
\pgfsys@transformshift{1in}{0in}%
\pgfsys@useobject{currentpattern}{}%
\pgfsys@transformshift{1in}{0in}%
\pgfsys@useobject{currentpattern}{}%
\pgfsys@transformshift{1in}{0in}%
\pgfsys@useobject{currentpattern}{}%
\pgfsys@transformshift{1in}{0in}%
\pgfsys@useobject{currentpattern}{}%
\pgfsys@transformshift{1in}{0in}%
\pgfsys@transformshift{-20in}{0in}%
\pgfsys@transformshift{0in}{1in}%
\end{pgfscope}%
\begin{pgfscope}%
\pgfpathrectangle{\pgfqpoint{3.788192in}{2.980138in}}{\pgfqpoint{2.914000in}{2.171400in}}%
\pgfusepath{clip}%
\pgfsetrectcap%
\pgfsetroundjoin%
\pgfsetlinewidth{0.803000pt}%
\definecolor{currentstroke}{rgb}{0.690196,0.690196,0.690196}%
\pgfsetstrokecolor{currentstroke}%
\pgfsetdash{}{0pt}%
\pgfpathmoveto{\pgfqpoint{4.105109in}{2.980138in}}%
\pgfpathlineto{\pgfqpoint{4.105109in}{5.151538in}}%
\pgfusepath{stroke}%
\end{pgfscope}%
\begin{pgfscope}%
\pgfsetbuttcap%
\pgfsetroundjoin%
\definecolor{currentfill}{rgb}{0.000000,0.000000,0.000000}%
\pgfsetfillcolor{currentfill}%
\pgfsetlinewidth{0.803000pt}%
\definecolor{currentstroke}{rgb}{0.000000,0.000000,0.000000}%
\pgfsetstrokecolor{currentstroke}%
\pgfsetdash{}{0pt}%
\pgfsys@defobject{currentmarker}{\pgfqpoint{0.000000in}{-0.048611in}}{\pgfqpoint{0.000000in}{0.000000in}}{%
\pgfpathmoveto{\pgfqpoint{0.000000in}{0.000000in}}%
\pgfpathlineto{\pgfqpoint{0.000000in}{-0.048611in}}%
\pgfusepath{stroke,fill}%
}%
\begin{pgfscope}%
\pgfsys@transformshift{4.105109in}{2.980138in}%
\pgfsys@useobject{currentmarker}{}%
\end{pgfscope}%
\end{pgfscope}%
\begin{pgfscope}%
\definecolor{textcolor}{rgb}{0.000000,0.000000,0.000000}%
\pgfsetstrokecolor{textcolor}%
\pgfsetfillcolor{textcolor}%
\pgftext[x=4.105109in,y=2.882916in,,top]{\color{textcolor}{\rmfamily\fontsize{14.000000}{16.800000}\selectfont\catcode`\^=\active\def^{\ifmmode\sp\else\^{}\fi}\catcode`\%=\active\def%{\%}$\mathdefault{5280}$}}%
\end{pgfscope}%
\begin{pgfscope}%
\pgfpathrectangle{\pgfqpoint{3.788192in}{2.980138in}}{\pgfqpoint{2.914000in}{2.171400in}}%
\pgfusepath{clip}%
\pgfsetrectcap%
\pgfsetroundjoin%
\pgfsetlinewidth{0.803000pt}%
\definecolor{currentstroke}{rgb}{0.690196,0.690196,0.690196}%
\pgfsetstrokecolor{currentstroke}%
\pgfsetdash{}{0pt}%
\pgfpathmoveto{\pgfqpoint{4.847132in}{2.980138in}}%
\pgfpathlineto{\pgfqpoint{4.847132in}{5.151538in}}%
\pgfusepath{stroke}%
\end{pgfscope}%
\begin{pgfscope}%
\pgfsetbuttcap%
\pgfsetroundjoin%
\definecolor{currentfill}{rgb}{0.000000,0.000000,0.000000}%
\pgfsetfillcolor{currentfill}%
\pgfsetlinewidth{0.803000pt}%
\definecolor{currentstroke}{rgb}{0.000000,0.000000,0.000000}%
\pgfsetstrokecolor{currentstroke}%
\pgfsetdash{}{0pt}%
\pgfsys@defobject{currentmarker}{\pgfqpoint{0.000000in}{-0.048611in}}{\pgfqpoint{0.000000in}{0.000000in}}{%
\pgfpathmoveto{\pgfqpoint{0.000000in}{0.000000in}}%
\pgfpathlineto{\pgfqpoint{0.000000in}{-0.048611in}}%
\pgfusepath{stroke,fill}%
}%
\begin{pgfscope}%
\pgfsys@transformshift{4.847132in}{2.980138in}%
\pgfsys@useobject{currentmarker}{}%
\end{pgfscope}%
\end{pgfscope}%
\begin{pgfscope}%
\definecolor{textcolor}{rgb}{0.000000,0.000000,0.000000}%
\pgfsetstrokecolor{textcolor}%
\pgfsetfillcolor{textcolor}%
\pgftext[x=4.847132in,y=2.882916in,,top]{\color{textcolor}{\rmfamily\fontsize{14.000000}{16.800000}\selectfont\catcode`\^=\active\def^{\ifmmode\sp\else\^{}\fi}\catcode`\%=\active\def%{\%}$\mathdefault{5300}$}}%
\end{pgfscope}%
\begin{pgfscope}%
\pgfpathrectangle{\pgfqpoint{3.788192in}{2.980138in}}{\pgfqpoint{2.914000in}{2.171400in}}%
\pgfusepath{clip}%
\pgfsetrectcap%
\pgfsetroundjoin%
\pgfsetlinewidth{0.803000pt}%
\definecolor{currentstroke}{rgb}{0.690196,0.690196,0.690196}%
\pgfsetstrokecolor{currentstroke}%
\pgfsetdash{}{0pt}%
\pgfpathmoveto{\pgfqpoint{5.589156in}{2.980138in}}%
\pgfpathlineto{\pgfqpoint{5.589156in}{5.151538in}}%
\pgfusepath{stroke}%
\end{pgfscope}%
\begin{pgfscope}%
\pgfsetbuttcap%
\pgfsetroundjoin%
\definecolor{currentfill}{rgb}{0.000000,0.000000,0.000000}%
\pgfsetfillcolor{currentfill}%
\pgfsetlinewidth{0.803000pt}%
\definecolor{currentstroke}{rgb}{0.000000,0.000000,0.000000}%
\pgfsetstrokecolor{currentstroke}%
\pgfsetdash{}{0pt}%
\pgfsys@defobject{currentmarker}{\pgfqpoint{0.000000in}{-0.048611in}}{\pgfqpoint{0.000000in}{0.000000in}}{%
\pgfpathmoveto{\pgfqpoint{0.000000in}{0.000000in}}%
\pgfpathlineto{\pgfqpoint{0.000000in}{-0.048611in}}%
\pgfusepath{stroke,fill}%
}%
\begin{pgfscope}%
\pgfsys@transformshift{5.589156in}{2.980138in}%
\pgfsys@useobject{currentmarker}{}%
\end{pgfscope}%
\end{pgfscope}%
\begin{pgfscope}%
\definecolor{textcolor}{rgb}{0.000000,0.000000,0.000000}%
\pgfsetstrokecolor{textcolor}%
\pgfsetfillcolor{textcolor}%
\pgftext[x=5.589156in,y=2.882916in,,top]{\color{textcolor}{\rmfamily\fontsize{14.000000}{16.800000}\selectfont\catcode`\^=\active\def^{\ifmmode\sp\else\^{}\fi}\catcode`\%=\active\def%{\%}$\mathdefault{5320}$}}%
\end{pgfscope}%
\begin{pgfscope}%
\pgfpathrectangle{\pgfqpoint{3.788192in}{2.980138in}}{\pgfqpoint{2.914000in}{2.171400in}}%
\pgfusepath{clip}%
\pgfsetrectcap%
\pgfsetroundjoin%
\pgfsetlinewidth{0.803000pt}%
\definecolor{currentstroke}{rgb}{0.690196,0.690196,0.690196}%
\pgfsetstrokecolor{currentstroke}%
\pgfsetdash{}{0pt}%
\pgfpathmoveto{\pgfqpoint{6.331180in}{2.980138in}}%
\pgfpathlineto{\pgfqpoint{6.331180in}{5.151538in}}%
\pgfusepath{stroke}%
\end{pgfscope}%
\begin{pgfscope}%
\pgfsetbuttcap%
\pgfsetroundjoin%
\definecolor{currentfill}{rgb}{0.000000,0.000000,0.000000}%
\pgfsetfillcolor{currentfill}%
\pgfsetlinewidth{0.803000pt}%
\definecolor{currentstroke}{rgb}{0.000000,0.000000,0.000000}%
\pgfsetstrokecolor{currentstroke}%
\pgfsetdash{}{0pt}%
\pgfsys@defobject{currentmarker}{\pgfqpoint{0.000000in}{-0.048611in}}{\pgfqpoint{0.000000in}{0.000000in}}{%
\pgfpathmoveto{\pgfqpoint{0.000000in}{0.000000in}}%
\pgfpathlineto{\pgfqpoint{0.000000in}{-0.048611in}}%
\pgfusepath{stroke,fill}%
}%
\begin{pgfscope}%
\pgfsys@transformshift{6.331180in}{2.980138in}%
\pgfsys@useobject{currentmarker}{}%
\end{pgfscope}%
\end{pgfscope}%
\begin{pgfscope}%
\definecolor{textcolor}{rgb}{0.000000,0.000000,0.000000}%
\pgfsetstrokecolor{textcolor}%
\pgfsetfillcolor{textcolor}%
\pgftext[x=6.331180in,y=2.882916in,,top]{\color{textcolor}{\rmfamily\fontsize{14.000000}{16.800000}\selectfont\catcode`\^=\active\def^{\ifmmode\sp\else\^{}\fi}\catcode`\%=\active\def%{\%}$\mathdefault{5340}$}}%
\end{pgfscope}%
\begin{pgfscope}%
\pgfpathrectangle{\pgfqpoint{3.788192in}{2.980138in}}{\pgfqpoint{2.914000in}{2.171400in}}%
\pgfusepath{clip}%
\pgfsetrectcap%
\pgfsetroundjoin%
\pgfsetlinewidth{0.803000pt}%
\definecolor{currentstroke}{rgb}{0.690196,0.690196,0.690196}%
\pgfsetstrokecolor{currentstroke}%
\pgfsetdash{}{0pt}%
\pgfpathmoveto{\pgfqpoint{3.788192in}{3.334947in}}%
\pgfpathlineto{\pgfqpoint{6.702192in}{3.334947in}}%
\pgfusepath{stroke}%
\end{pgfscope}%
\begin{pgfscope}%
\pgfsetbuttcap%
\pgfsetroundjoin%
\definecolor{currentfill}{rgb}{0.000000,0.000000,0.000000}%
\pgfsetfillcolor{currentfill}%
\pgfsetlinewidth{0.803000pt}%
\definecolor{currentstroke}{rgb}{0.000000,0.000000,0.000000}%
\pgfsetstrokecolor{currentstroke}%
\pgfsetdash{}{0pt}%
\pgfsys@defobject{currentmarker}{\pgfqpoint{-0.048611in}{0.000000in}}{\pgfqpoint{-0.000000in}{0.000000in}}{%
\pgfpathmoveto{\pgfqpoint{-0.000000in}{0.000000in}}%
\pgfpathlineto{\pgfqpoint{-0.048611in}{0.000000in}}%
\pgfusepath{stroke,fill}%
}%
\begin{pgfscope}%
\pgfsys@transformshift{3.788192in}{3.334947in}%
\pgfsys@useobject{currentmarker}{}%
\end{pgfscope}%
\end{pgfscope}%
\begin{pgfscope}%
\definecolor{textcolor}{rgb}{0.000000,0.000000,0.000000}%
\pgfsetstrokecolor{textcolor}%
\pgfsetfillcolor{textcolor}%
\pgftext[x=3.495138in, y=3.265502in, left, base]{\color{textcolor}{\rmfamily\fontsize{14.000000}{16.800000}\selectfont\catcode`\^=\active\def^{\ifmmode\sp\else\^{}\fi}\catcode`\%=\active\def%{\%}$\mathdefault{10}$}}%
\end{pgfscope}%
\begin{pgfscope}%
\pgfpathrectangle{\pgfqpoint{3.788192in}{2.980138in}}{\pgfqpoint{2.914000in}{2.171400in}}%
\pgfusepath{clip}%
\pgfsetrectcap%
\pgfsetroundjoin%
\pgfsetlinewidth{0.803000pt}%
\definecolor{currentstroke}{rgb}{0.690196,0.690196,0.690196}%
\pgfsetstrokecolor{currentstroke}%
\pgfsetdash{}{0pt}%
\pgfpathmoveto{\pgfqpoint{3.788192in}{4.044564in}}%
\pgfpathlineto{\pgfqpoint{6.702192in}{4.044564in}}%
\pgfusepath{stroke}%
\end{pgfscope}%
\begin{pgfscope}%
\pgfsetbuttcap%
\pgfsetroundjoin%
\definecolor{currentfill}{rgb}{0.000000,0.000000,0.000000}%
\pgfsetfillcolor{currentfill}%
\pgfsetlinewidth{0.803000pt}%
\definecolor{currentstroke}{rgb}{0.000000,0.000000,0.000000}%
\pgfsetstrokecolor{currentstroke}%
\pgfsetdash{}{0pt}%
\pgfsys@defobject{currentmarker}{\pgfqpoint{-0.048611in}{0.000000in}}{\pgfqpoint{-0.000000in}{0.000000in}}{%
\pgfpathmoveto{\pgfqpoint{-0.000000in}{0.000000in}}%
\pgfpathlineto{\pgfqpoint{-0.048611in}{0.000000in}}%
\pgfusepath{stroke,fill}%
}%
\begin{pgfscope}%
\pgfsys@transformshift{3.788192in}{4.044564in}%
\pgfsys@useobject{currentmarker}{}%
\end{pgfscope}%
\end{pgfscope}%
\begin{pgfscope}%
\definecolor{textcolor}{rgb}{0.000000,0.000000,0.000000}%
\pgfsetstrokecolor{textcolor}%
\pgfsetfillcolor{textcolor}%
\pgftext[x=3.495138in, y=3.975119in, left, base]{\color{textcolor}{\rmfamily\fontsize{14.000000}{16.800000}\selectfont\catcode`\^=\active\def^{\ifmmode\sp\else\^{}\fi}\catcode`\%=\active\def%{\%}$\mathdefault{12}$}}%
\end{pgfscope}%
\begin{pgfscope}%
\pgfpathrectangle{\pgfqpoint{3.788192in}{2.980138in}}{\pgfqpoint{2.914000in}{2.171400in}}%
\pgfusepath{clip}%
\pgfsetrectcap%
\pgfsetroundjoin%
\pgfsetlinewidth{0.803000pt}%
\definecolor{currentstroke}{rgb}{0.690196,0.690196,0.690196}%
\pgfsetstrokecolor{currentstroke}%
\pgfsetdash{}{0pt}%
\pgfpathmoveto{\pgfqpoint{3.788192in}{4.754181in}}%
\pgfpathlineto{\pgfqpoint{6.702192in}{4.754181in}}%
\pgfusepath{stroke}%
\end{pgfscope}%
\begin{pgfscope}%
\pgfsetbuttcap%
\pgfsetroundjoin%
\definecolor{currentfill}{rgb}{0.000000,0.000000,0.000000}%
\pgfsetfillcolor{currentfill}%
\pgfsetlinewidth{0.803000pt}%
\definecolor{currentstroke}{rgb}{0.000000,0.000000,0.000000}%
\pgfsetstrokecolor{currentstroke}%
\pgfsetdash{}{0pt}%
\pgfsys@defobject{currentmarker}{\pgfqpoint{-0.048611in}{0.000000in}}{\pgfqpoint{-0.000000in}{0.000000in}}{%
\pgfpathmoveto{\pgfqpoint{-0.000000in}{0.000000in}}%
\pgfpathlineto{\pgfqpoint{-0.048611in}{0.000000in}}%
\pgfusepath{stroke,fill}%
}%
\begin{pgfscope}%
\pgfsys@transformshift{3.788192in}{4.754181in}%
\pgfsys@useobject{currentmarker}{}%
\end{pgfscope}%
\end{pgfscope}%
\begin{pgfscope}%
\definecolor{textcolor}{rgb}{0.000000,0.000000,0.000000}%
\pgfsetstrokecolor{textcolor}%
\pgfsetfillcolor{textcolor}%
\pgftext[x=3.495138in, y=4.684736in, left, base]{\color{textcolor}{\rmfamily\fontsize{14.000000}{16.800000}\selectfont\catcode`\^=\active\def^{\ifmmode\sp\else\^{}\fi}\catcode`\%=\active\def%{\%}$\mathdefault{14}$}}%
\end{pgfscope}%
\begin{pgfscope}%
\pgfpathrectangle{\pgfqpoint{3.788192in}{2.980138in}}{\pgfqpoint{2.914000in}{2.171400in}}%
\pgfusepath{clip}%
\pgfsetrectcap%
\pgfsetroundjoin%
\pgfsetlinewidth{1.505625pt}%
\definecolor{currentstroke}{rgb}{0.000000,0.000000,1.000000}%
\pgfsetstrokecolor{currentstroke}%
\pgfsetdash{}{0pt}%
\pgfpathmoveto{\pgfqpoint{5.478622in}{4.808529in}}%
\pgfpathlineto{\pgfqpoint{5.701947in}{2.977638in}}%
\pgfusepath{stroke}%
\end{pgfscope}%
\begin{pgfscope}%
\pgfpathrectangle{\pgfqpoint{3.788192in}{2.980138in}}{\pgfqpoint{2.914000in}{2.171400in}}%
\pgfusepath{clip}%
\pgfsetbuttcap%
\pgfsetroundjoin%
\definecolor{currentfill}{rgb}{0.000000,0.000000,1.000000}%
\pgfsetfillcolor{currentfill}%
\pgfsetlinewidth{1.003750pt}%
\definecolor{currentstroke}{rgb}{0.000000,0.000000,1.000000}%
\pgfsetstrokecolor{currentstroke}%
\pgfsetdash{}{0pt}%
\pgfsys@defobject{currentmarker}{\pgfqpoint{-0.041667in}{-0.041667in}}{\pgfqpoint{0.041667in}{0.041667in}}{%
\pgfpathmoveto{\pgfqpoint{0.000000in}{-0.041667in}}%
\pgfpathcurveto{\pgfqpoint{0.011050in}{-0.041667in}}{\pgfqpoint{0.021649in}{-0.037276in}}{\pgfqpoint{0.029463in}{-0.029463in}}%
\pgfpathcurveto{\pgfqpoint{0.037276in}{-0.021649in}}{\pgfqpoint{0.041667in}{-0.011050in}}{\pgfqpoint{0.041667in}{0.000000in}}%
\pgfpathcurveto{\pgfqpoint{0.041667in}{0.011050in}}{\pgfqpoint{0.037276in}{0.021649in}}{\pgfqpoint{0.029463in}{0.029463in}}%
\pgfpathcurveto{\pgfqpoint{0.021649in}{0.037276in}}{\pgfqpoint{0.011050in}{0.041667in}}{\pgfqpoint{0.000000in}{0.041667in}}%
\pgfpathcurveto{\pgfqpoint{-0.011050in}{0.041667in}}{\pgfqpoint{-0.021649in}{0.037276in}}{\pgfqpoint{-0.029463in}{0.029463in}}%
\pgfpathcurveto{\pgfqpoint{-0.037276in}{0.021649in}}{\pgfqpoint{-0.041667in}{0.011050in}}{\pgfqpoint{-0.041667in}{0.000000in}}%
\pgfpathcurveto{\pgfqpoint{-0.041667in}{-0.011050in}}{\pgfqpoint{-0.037276in}{-0.021649in}}{\pgfqpoint{-0.029463in}{-0.029463in}}%
\pgfpathcurveto{\pgfqpoint{-0.021649in}{-0.037276in}}{\pgfqpoint{-0.011050in}{-0.041667in}}{\pgfqpoint{0.000000in}{-0.041667in}}%
\pgfpathlineto{\pgfqpoint{0.000000in}{-0.041667in}}%
\pgfpathclose%
\pgfusepath{stroke,fill}%
}%
\begin{pgfscope}%
\pgfsys@transformshift{5.478622in}{4.808529in}%
\pgfsys@useobject{currentmarker}{}%
\end{pgfscope}%
\begin{pgfscope}%
\pgfsys@transformshift{5.786444in}{2.284897in}%
\pgfsys@useobject{currentmarker}{}%
\end{pgfscope}%
\begin{pgfscope}%
\pgfsys@transformshift{5.955602in}{1.807070in}%
\pgfsys@useobject{currentmarker}{}%
\end{pgfscope}%
\begin{pgfscope}%
\pgfsys@transformshift{6.072257in}{1.530809in}%
\pgfsys@useobject{currentmarker}{}%
\end{pgfscope}%
\begin{pgfscope}%
\pgfsys@transformshift{6.134561in}{1.493263in}%
\pgfsys@useobject{currentmarker}{}%
\end{pgfscope}%
\begin{pgfscope}%
\pgfsys@transformshift{6.204334in}{1.296852in}%
\pgfsys@useobject{currentmarker}{}%
\end{pgfscope}%
\begin{pgfscope}%
\pgfsys@transformshift{6.281957in}{1.238734in}%
\pgfsys@useobject{currentmarker}{}%
\end{pgfscope}%
\begin{pgfscope}%
\pgfsys@transformshift{6.389240in}{1.234341in}%
\pgfsys@useobject{currentmarker}{}%
\end{pgfscope}%
\begin{pgfscope}%
\pgfsys@transformshift{6.427201in}{1.108377in}%
\pgfsys@useobject{currentmarker}{}%
\end{pgfscope}%
\begin{pgfscope}%
\pgfsys@transformshift{6.519285in}{1.077759in}%
\pgfsys@useobject{currentmarker}{}%
\end{pgfscope}%
\begin{pgfscope}%
\pgfsys@transformshift{6.689124in}{1.072073in}%
\pgfsys@useobject{currentmarker}{}%
\end{pgfscope}%
\begin{pgfscope}%
\pgfsys@transformshift{6.747190in}{1.024269in}%
\pgfsys@useobject{currentmarker}{}%
\end{pgfscope}%
\begin{pgfscope}%
\pgfsys@transformshift{6.806750in}{1.015907in}%
\pgfsys@useobject{currentmarker}{}%
\end{pgfscope}%
\begin{pgfscope}%
\pgfsys@transformshift{6.931730in}{0.991431in}%
\pgfsys@useobject{currentmarker}{}%
\end{pgfscope}%
\begin{pgfscope}%
\pgfsys@transformshift{7.035734in}{0.970288in}%
\pgfsys@useobject{currentmarker}{}%
\end{pgfscope}%
\begin{pgfscope}%
\pgfsys@transformshift{7.042663in}{0.960624in}%
\pgfsys@useobject{currentmarker}{}%
\end{pgfscope}%
\begin{pgfscope}%
\pgfsys@transformshift{7.110896in}{0.945926in}%
\pgfsys@useobject{currentmarker}{}%
\end{pgfscope}%
\begin{pgfscope}%
\pgfsys@transformshift{7.385541in}{0.918874in}%
\pgfsys@useobject{currentmarker}{}%
\end{pgfscope}%
\begin{pgfscope}%
\pgfsys@transformshift{7.677622in}{0.881182in}%
\pgfsys@useobject{currentmarker}{}%
\end{pgfscope}%
\begin{pgfscope}%
\pgfsys@transformshift{8.074033in}{0.854291in}%
\pgfsys@useobject{currentmarker}{}%
\end{pgfscope}%
\begin{pgfscope}%
\pgfsys@transformshift{8.437037in}{0.845259in}%
\pgfsys@useobject{currentmarker}{}%
\end{pgfscope}%
\begin{pgfscope}%
\pgfsys@transformshift{8.451833in}{0.833333in}%
\pgfsys@useobject{currentmarker}{}%
\end{pgfscope}%
\begin{pgfscope}%
\pgfsys@transformshift{8.473270in}{0.812531in}%
\pgfsys@useobject{currentmarker}{}%
\end{pgfscope}%
\begin{pgfscope}%
\pgfsys@transformshift{8.530759in}{0.805118in}%
\pgfsys@useobject{currentmarker}{}%
\end{pgfscope}%
\begin{pgfscope}%
\pgfsys@transformshift{8.667946in}{0.801766in}%
\pgfsys@useobject{currentmarker}{}%
\end{pgfscope}%
\begin{pgfscope}%
\pgfsys@transformshift{8.761129in}{0.792896in}%
\pgfsys@useobject{currentmarker}{}%
\end{pgfscope}%
\begin{pgfscope}%
\pgfsys@transformshift{9.175032in}{0.779118in}%
\pgfsys@useobject{currentmarker}{}%
\end{pgfscope}%
\begin{pgfscope}%
\pgfsys@transformshift{9.229242in}{0.771806in}%
\pgfsys@useobject{currentmarker}{}%
\end{pgfscope}%
\begin{pgfscope}%
\pgfsys@transformshift{9.230836in}{0.763945in}%
\pgfsys@useobject{currentmarker}{}%
\end{pgfscope}%
\begin{pgfscope}%
\pgfsys@transformshift{9.319803in}{0.762647in}%
\pgfsys@useobject{currentmarker}{}%
\end{pgfscope}%
\begin{pgfscope}%
\pgfsys@transformshift{9.530028in}{0.749444in}%
\pgfsys@useobject{currentmarker}{}%
\end{pgfscope}%
\begin{pgfscope}%
\pgfsys@transformshift{9.629502in}{0.748101in}%
\pgfsys@useobject{currentmarker}{}%
\end{pgfscope}%
\begin{pgfscope}%
\pgfsys@transformshift{9.644888in}{0.743176in}%
\pgfsys@useobject{currentmarker}{}%
\end{pgfscope}%
\begin{pgfscope}%
\pgfsys@transformshift{10.473184in}{0.730520in}%
\pgfsys@useobject{currentmarker}{}%
\end{pgfscope}%
\begin{pgfscope}%
\pgfsys@transformshift{10.538033in}{0.705808in}%
\pgfsys@useobject{currentmarker}{}%
\end{pgfscope}%
\begin{pgfscope}%
\pgfsys@transformshift{10.574876in}{0.703903in}%
\pgfsys@useobject{currentmarker}{}%
\end{pgfscope}%
\begin{pgfscope}%
\pgfsys@transformshift{10.706030in}{0.700593in}%
\pgfsys@useobject{currentmarker}{}%
\end{pgfscope}%
\begin{pgfscope}%
\pgfsys@transformshift{10.965602in}{0.687998in}%
\pgfsys@useobject{currentmarker}{}%
\end{pgfscope}%
\begin{pgfscope}%
\pgfsys@transformshift{11.060169in}{0.684789in}%
\pgfsys@useobject{currentmarker}{}%
\end{pgfscope}%
\begin{pgfscope}%
\pgfsys@transformshift{11.147519in}{0.680781in}%
\pgfsys@useobject{currentmarker}{}%
\end{pgfscope}%
\begin{pgfscope}%
\pgfsys@transformshift{11.370645in}{0.672484in}%
\pgfsys@useobject{currentmarker}{}%
\end{pgfscope}%
\begin{pgfscope}%
\pgfsys@transformshift{11.728159in}{0.668514in}%
\pgfsys@useobject{currentmarker}{}%
\end{pgfscope}%
\begin{pgfscope}%
\pgfsys@transformshift{12.137942in}{0.666709in}%
\pgfsys@useobject{currentmarker}{}%
\end{pgfscope}%
\begin{pgfscope}%
\pgfsys@transformshift{12.434130in}{0.666468in}%
\pgfsys@useobject{currentmarker}{}%
\end{pgfscope}%
\begin{pgfscope}%
\pgfsys@transformshift{12.804602in}{0.666397in}%
\pgfsys@useobject{currentmarker}{}%
\end{pgfscope}%
\begin{pgfscope}%
\pgfsys@transformshift{14.562739in}{0.661643in}%
\pgfsys@useobject{currentmarker}{}%
\end{pgfscope}%
\begin{pgfscope}%
\pgfsys@transformshift{15.028873in}{0.661022in}%
\pgfsys@useobject{currentmarker}{}%
\end{pgfscope}%
\begin{pgfscope}%
\pgfsys@transformshift{15.695083in}{0.659204in}%
\pgfsys@useobject{currentmarker}{}%
\end{pgfscope}%
\begin{pgfscope}%
\pgfsys@transformshift{16.663977in}{0.658612in}%
\pgfsys@useobject{currentmarker}{}%
\end{pgfscope}%
\begin{pgfscope}%
\pgfsys@transformshift{17.188373in}{0.655818in}%
\pgfsys@useobject{currentmarker}{}%
\end{pgfscope}%
\begin{pgfscope}%
\pgfsys@transformshift{18.420605in}{0.654898in}%
\pgfsys@useobject{currentmarker}{}%
\end{pgfscope}%
\begin{pgfscope}%
\pgfsys@transformshift{20.009331in}{0.652189in}%
\pgfsys@useobject{currentmarker}{}%
\end{pgfscope}%
\begin{pgfscope}%
\pgfsys@transformshift{21.260300in}{0.647607in}%
\pgfsys@useobject{currentmarker}{}%
\end{pgfscope}%
\begin{pgfscope}%
\pgfsys@transformshift{23.228701in}{0.645006in}%
\pgfsys@useobject{currentmarker}{}%
\end{pgfscope}%
\begin{pgfscope}%
\pgfsys@transformshift{25.888770in}{0.639824in}%
\pgfsys@useobject{currentmarker}{}%
\end{pgfscope}%
\begin{pgfscope}%
\pgfsys@transformshift{28.569097in}{0.635141in}%
\pgfsys@useobject{currentmarker}{}%
\end{pgfscope}%
\begin{pgfscope}%
\pgfsys@transformshift{32.504050in}{0.628662in}%
\pgfsys@useobject{currentmarker}{}%
\end{pgfscope}%
\begin{pgfscope}%
\pgfsys@transformshift{39.060135in}{0.620415in}%
\pgfsys@useobject{currentmarker}{}%
\end{pgfscope}%
\begin{pgfscope}%
\pgfsys@transformshift{50.416853in}{0.608419in}%
\pgfsys@useobject{currentmarker}{}%
\end{pgfscope}%
\begin{pgfscope}%
\pgfsys@transformshift{87.100182in}{0.583248in}%
\pgfsys@useobject{currentmarker}{}%
\end{pgfscope}%
\end{pgfscope}%
\begin{pgfscope}%
\pgfpathrectangle{\pgfqpoint{3.788192in}{2.980138in}}{\pgfqpoint{2.914000in}{2.171400in}}%
\pgfusepath{clip}%
\pgfsetrectcap%
\pgfsetroundjoin%
\pgfsetlinewidth{1.505625pt}%
\definecolor{currentstroke}{rgb}{0.121569,0.466667,0.705882}%
\pgfsetstrokecolor{currentstroke}%
\pgfsetstrokeopacity{0.500000}%
\pgfsetdash{}{0pt}%
\pgfusepath{stroke}%
\end{pgfscope}%
\begin{pgfscope}%
\pgfsetrectcap%
\pgfsetmiterjoin%
\pgfsetlinewidth{0.803000pt}%
\definecolor{currentstroke}{rgb}{0.000000,0.000000,0.000000}%
\pgfsetstrokecolor{currentstroke}%
\pgfsetdash{}{0pt}%
\pgfpathmoveto{\pgfqpoint{3.788192in}{2.980138in}}%
\pgfpathlineto{\pgfqpoint{3.788192in}{5.151538in}}%
\pgfusepath{stroke}%
\end{pgfscope}%
\begin{pgfscope}%
\pgfsetrectcap%
\pgfsetmiterjoin%
\pgfsetlinewidth{0.803000pt}%
\definecolor{currentstroke}{rgb}{0.000000,0.000000,0.000000}%
\pgfsetstrokecolor{currentstroke}%
\pgfsetdash{}{0pt}%
\pgfpathmoveto{\pgfqpoint{6.702192in}{2.980138in}}%
\pgfpathlineto{\pgfqpoint{6.702192in}{5.151538in}}%
\pgfusepath{stroke}%
\end{pgfscope}%
\begin{pgfscope}%
\pgfsetrectcap%
\pgfsetmiterjoin%
\pgfsetlinewidth{0.803000pt}%
\definecolor{currentstroke}{rgb}{0.000000,0.000000,0.000000}%
\pgfsetstrokecolor{currentstroke}%
\pgfsetdash{}{0pt}%
\pgfpathmoveto{\pgfqpoint{3.788192in}{2.980138in}}%
\pgfpathlineto{\pgfqpoint{6.702192in}{2.980138in}}%
\pgfusepath{stroke}%
\end{pgfscope}%
\begin{pgfscope}%
\pgfsetrectcap%
\pgfsetmiterjoin%
\pgfsetlinewidth{0.803000pt}%
\definecolor{currentstroke}{rgb}{0.000000,0.000000,0.000000}%
\pgfsetstrokecolor{currentstroke}%
\pgfsetdash{}{0pt}%
\pgfpathmoveto{\pgfqpoint{3.788192in}{5.151538in}}%
\pgfpathlineto{\pgfqpoint{6.702192in}{5.151538in}}%
\pgfusepath{stroke}%
\end{pgfscope}%
\begin{pgfscope}%
\pgfsetbuttcap%
\pgfsetmiterjoin%
\definecolor{currentfill}{rgb}{1.000000,1.000000,1.000000}%
\pgfsetfillcolor{currentfill}%
\pgfsetfillopacity{0.800000}%
\pgfsetlinewidth{1.003750pt}%
\definecolor{currentstroke}{rgb}{0.800000,0.800000,0.800000}%
\pgfsetstrokecolor{currentstroke}%
\pgfsetstrokeopacity{0.800000}%
\pgfsetdash{}{0pt}%
\pgfpathmoveto{\pgfqpoint{3.327765in}{0.781249in}}%
\pgfpathlineto{\pgfqpoint{6.732636in}{0.781249in}}%
\pgfpathquadraticcurveto{\pgfqpoint{6.777080in}{0.781249in}}{\pgfqpoint{6.777080in}{0.825694in}}%
\pgfpathlineto{\pgfqpoint{6.777080in}{2.153779in}}%
\pgfpathquadraticcurveto{\pgfqpoint{6.777080in}{2.198223in}}{\pgfqpoint{6.732636in}{2.198223in}}%
\pgfpathlineto{\pgfqpoint{3.327765in}{2.198223in}}%
\pgfpathquadraticcurveto{\pgfqpoint{3.283320in}{2.198223in}}{\pgfqpoint{3.283320in}{2.153779in}}%
\pgfpathlineto{\pgfqpoint{3.283320in}{0.825694in}}%
\pgfpathquadraticcurveto{\pgfqpoint{3.283320in}{0.781249in}}{\pgfqpoint{3.327765in}{0.781249in}}%
\pgfpathlineto{\pgfqpoint{3.327765in}{0.781249in}}%
\pgfpathclose%
\pgfusepath{stroke,fill}%
\end{pgfscope}%
\begin{pgfscope}%
\pgfsetrectcap%
\pgfsetroundjoin%
\pgfsetlinewidth{1.505625pt}%
\definecolor{currentstroke}{rgb}{0.000000,0.000000,1.000000}%
\pgfsetstrokecolor{currentstroke}%
\pgfsetdash{}{0pt}%
\pgfpathmoveto{\pgfqpoint{3.372209in}{2.020446in}}%
\pgfpathlineto{\pgfqpoint{3.594431in}{2.020446in}}%
\pgfpathlineto{\pgfqpoint{3.816654in}{2.020446in}}%
\pgfusepath{stroke}%
\end{pgfscope}%
\begin{pgfscope}%
\pgfsetbuttcap%
\pgfsetroundjoin%
\definecolor{currentfill}{rgb}{0.000000,0.000000,1.000000}%
\pgfsetfillcolor{currentfill}%
\pgfsetlinewidth{1.003750pt}%
\definecolor{currentstroke}{rgb}{0.000000,0.000000,1.000000}%
\pgfsetstrokecolor{currentstroke}%
\pgfsetdash{}{0pt}%
\pgfsys@defobject{currentmarker}{\pgfqpoint{-0.006944in}{-0.006944in}}{\pgfqpoint{0.006944in}{0.006944in}}{%
\pgfpathmoveto{\pgfqpoint{0.000000in}{-0.006944in}}%
\pgfpathcurveto{\pgfqpoint{0.001842in}{-0.006944in}}{\pgfqpoint{0.003608in}{-0.006213in}}{\pgfqpoint{0.004910in}{-0.004910in}}%
\pgfpathcurveto{\pgfqpoint{0.006213in}{-0.003608in}}{\pgfqpoint{0.006944in}{-0.001842in}}{\pgfqpoint{0.006944in}{0.000000in}}%
\pgfpathcurveto{\pgfqpoint{0.006944in}{0.001842in}}{\pgfqpoint{0.006213in}{0.003608in}}{\pgfqpoint{0.004910in}{0.004910in}}%
\pgfpathcurveto{\pgfqpoint{0.003608in}{0.006213in}}{\pgfqpoint{0.001842in}{0.006944in}}{\pgfqpoint{0.000000in}{0.006944in}}%
\pgfpathcurveto{\pgfqpoint{-0.001842in}{0.006944in}}{\pgfqpoint{-0.003608in}{0.006213in}}{\pgfqpoint{-0.004910in}{0.004910in}}%
\pgfpathcurveto{\pgfqpoint{-0.006213in}{0.003608in}}{\pgfqpoint{-0.006944in}{0.001842in}}{\pgfqpoint{-0.006944in}{0.000000in}}%
\pgfpathcurveto{\pgfqpoint{-0.006944in}{-0.001842in}}{\pgfqpoint{-0.006213in}{-0.003608in}}{\pgfqpoint{-0.004910in}{-0.004910in}}%
\pgfpathcurveto{\pgfqpoint{-0.003608in}{-0.006213in}}{\pgfqpoint{-0.001842in}{-0.006944in}}{\pgfqpoint{0.000000in}{-0.006944in}}%
\pgfpathlineto{\pgfqpoint{0.000000in}{-0.006944in}}%
\pgfpathclose%
\pgfusepath{stroke,fill}%
}%
\begin{pgfscope}%
\pgfsys@transformshift{3.594431in}{2.020446in}%
\pgfsys@useobject{currentmarker}{}%
\end{pgfscope}%
\end{pgfscope}%
\begin{pgfscope}%
\definecolor{textcolor}{rgb}{0.000000,0.000000,0.000000}%
\pgfsetstrokecolor{textcolor}%
\pgfsetfillcolor{textcolor}%
\pgftext[x=3.994431in,y=1.942668in,left,base]{\color{textcolor}{\rmfamily\fontsize{16.000000}{19.200000}\selectfont\catcode`\^=\active\def^{\ifmmode\sp\else\^{}\fi}\catcode`\%=\active\def%{\%}osier}}%
\end{pgfscope}%
\begin{pgfscope}%
\pgfsetrectcap%
\pgfsetroundjoin%
\pgfsetlinewidth{1.505625pt}%
\definecolor{currentstroke}{rgb}{0.121569,0.466667,0.705882}%
\pgfsetstrokecolor{currentstroke}%
\pgfsetstrokeopacity{0.500000}%
\pgfsetdash{}{0pt}%
\pgfpathmoveto{\pgfqpoint{3.372209in}{1.682869in}}%
\pgfpathlineto{\pgfqpoint{3.594431in}{1.682869in}}%
\pgfpathlineto{\pgfqpoint{3.816654in}{1.682869in}}%
\pgfusepath{stroke}%
\end{pgfscope}%
\begin{pgfscope}%
\definecolor{textcolor}{rgb}{0.000000,0.000000,0.000000}%
\pgfsetstrokecolor{textcolor}%
\pgfsetfillcolor{textcolor}%
\pgftext[x=3.994431in,y=1.605091in,left,base]{\color{textcolor}{\rmfamily\fontsize{16.000000}{19.200000}\selectfont\catcode`\^=\active\def^{\ifmmode\sp\else\^{}\fi}\catcode`\%=\active\def%{\%}near-optimal space (osier)}}%
\end{pgfscope}%
\begin{pgfscope}%
\pgfsetbuttcap%
\pgfsetroundjoin%
\pgfsetlinewidth{1.003750pt}%
\definecolor{currentstroke}{rgb}{1.000000,0.000000,0.000000}%
\pgfsetstrokecolor{currentstroke}%
\pgfsetdash{}{0pt}%
\pgfpathmoveto{\pgfqpoint{3.594431in}{1.294791in}}%
\pgfpathcurveto{\pgfqpoint{3.602668in}{1.294791in}}{\pgfqpoint{3.610568in}{1.298063in}}{\pgfqpoint{3.616392in}{1.303887in}}%
\pgfpathcurveto{\pgfqpoint{3.622216in}{1.309711in}}{\pgfqpoint{3.625488in}{1.317611in}}{\pgfqpoint{3.625488in}{1.325847in}}%
\pgfpathcurveto{\pgfqpoint{3.625488in}{1.334084in}}{\pgfqpoint{3.622216in}{1.341984in}}{\pgfqpoint{3.616392in}{1.347808in}}%
\pgfpathcurveto{\pgfqpoint{3.610568in}{1.353632in}}{\pgfqpoint{3.602668in}{1.356904in}}{\pgfqpoint{3.594431in}{1.356904in}}%
\pgfpathcurveto{\pgfqpoint{3.586195in}{1.356904in}}{\pgfqpoint{3.578295in}{1.353632in}}{\pgfqpoint{3.572471in}{1.347808in}}%
\pgfpathcurveto{\pgfqpoint{3.566647in}{1.341984in}}{\pgfqpoint{3.563375in}{1.334084in}}{\pgfqpoint{3.563375in}{1.325847in}}%
\pgfpathcurveto{\pgfqpoint{3.563375in}{1.317611in}}{\pgfqpoint{3.566647in}{1.309711in}}{\pgfqpoint{3.572471in}{1.303887in}}%
\pgfpathcurveto{\pgfqpoint{3.578295in}{1.298063in}}{\pgfqpoint{3.586195in}{1.294791in}}{\pgfqpoint{3.594431in}{1.294791in}}%
\pgfpathlineto{\pgfqpoint{3.594431in}{1.294791in}}%
\pgfpathclose%
\pgfusepath{stroke}%
\end{pgfscope}%
\begin{pgfscope}%
\definecolor{textcolor}{rgb}{0.000000,0.000000,0.000000}%
\pgfsetstrokecolor{textcolor}%
\pgfsetfillcolor{textcolor}%
\pgftext[x=3.994431in,y=1.267514in,left,base]{\color{textcolor}{\rmfamily\fontsize{16.000000}{19.200000}\selectfont\catcode`\^=\active\def^{\ifmmode\sp\else\^{}\fi}\catcode`\%=\active\def%{\%}temoa+mga}}%
\end{pgfscope}%
\begin{pgfscope}%
\pgfsetbuttcap%
\pgfsetmiterjoin%
\definecolor{currentfill}{rgb}{0.839216,0.152941,0.156863}%
\pgfsetfillcolor{currentfill}%
\pgfsetfillopacity{0.200000}%
\pgfsetlinewidth{1.003750pt}%
\definecolor{currentstroke}{rgb}{0.839216,0.152941,0.156863}%
\pgfsetstrokecolor{currentstroke}%
\pgfsetstrokeopacity{0.200000}%
\pgfsetdash{}{0pt}%
\pgfpathmoveto{\pgfqpoint{3.372209in}{0.929937in}}%
\pgfpathlineto{\pgfqpoint{3.816654in}{0.929937in}}%
\pgfpathlineto{\pgfqpoint{3.816654in}{1.085493in}}%
\pgfpathlineto{\pgfqpoint{3.372209in}{1.085493in}}%
\pgfpathlineto{\pgfqpoint{3.372209in}{0.929937in}}%
\pgfpathclose%
\pgfusepath{stroke,fill}%
\end{pgfscope}%
\begin{pgfscope}%
\pgfsetbuttcap%
\pgfsetmiterjoin%
\definecolor{currentfill}{rgb}{0.839216,0.152941,0.156863}%
\pgfsetfillcolor{currentfill}%
\pgfsetfillopacity{0.200000}%
\pgfsetlinewidth{1.003750pt}%
\definecolor{currentstroke}{rgb}{0.839216,0.152941,0.156863}%
\pgfsetstrokecolor{currentstroke}%
\pgfsetstrokeopacity{0.200000}%
\pgfsetdash{}{0pt}%
\pgfpathmoveto{\pgfqpoint{3.372209in}{0.929937in}}%
\pgfpathlineto{\pgfqpoint{3.816654in}{0.929937in}}%
\pgfpathlineto{\pgfqpoint{3.816654in}{1.085493in}}%
\pgfpathlineto{\pgfqpoint{3.372209in}{1.085493in}}%
\pgfpathlineto{\pgfqpoint{3.372209in}{0.929937in}}%
\pgfpathclose%
\pgfusepath{clip}%
\pgfsys@defobject{currentpattern}{\pgfqpoint{0in}{0in}}{\pgfqpoint{1in}{1in}}{%
\begin{pgfscope}%
\pgfpathrectangle{\pgfqpoint{0in}{0in}}{\pgfqpoint{1in}{1in}}%
\pgfusepath{clip}%
\pgfpathmoveto{\pgfqpoint{-0.500000in}{0.500000in}}%
\pgfpathlineto{\pgfqpoint{0.500000in}{1.500000in}}%
\pgfpathmoveto{\pgfqpoint{-0.333333in}{0.333333in}}%
\pgfpathlineto{\pgfqpoint{0.666667in}{1.333333in}}%
\pgfpathmoveto{\pgfqpoint{-0.166667in}{0.166667in}}%
\pgfpathlineto{\pgfqpoint{0.833333in}{1.166667in}}%
\pgfpathmoveto{\pgfqpoint{0.000000in}{0.000000in}}%
\pgfpathlineto{\pgfqpoint{1.000000in}{1.000000in}}%
\pgfpathmoveto{\pgfqpoint{0.166667in}{-0.166667in}}%
\pgfpathlineto{\pgfqpoint{1.166667in}{0.833333in}}%
\pgfpathmoveto{\pgfqpoint{0.333333in}{-0.333333in}}%
\pgfpathlineto{\pgfqpoint{1.333333in}{0.666667in}}%
\pgfpathmoveto{\pgfqpoint{0.500000in}{-0.500000in}}%
\pgfpathlineto{\pgfqpoint{1.500000in}{0.500000in}}%
\pgfusepath{stroke}%
\end{pgfscope}%
}%
\pgfsys@transformshift{3.372209in}{0.929937in}%
\pgfsys@useobject{currentpattern}{}%
\pgfsys@transformshift{1in}{0in}%
\pgfsys@transformshift{-1in}{0in}%
\pgfsys@transformshift{0in}{1in}%
\end{pgfscope}%
\begin{pgfscope}%
\definecolor{textcolor}{rgb}{0.000000,0.000000,0.000000}%
\pgfsetstrokecolor{textcolor}%
\pgfsetfillcolor{textcolor}%
\pgftext[x=3.994431in,y=0.929937in,left,base]{\color{textcolor}{\rmfamily\fontsize{16.000000}{19.200000}\selectfont\catcode`\^=\active\def^{\ifmmode\sp\else\^{}\fi}\catcode`\%=\active\def%{\%}near-optimal space (Temoa)}}%
\end{pgfscope}%
\end{pgfpicture}%
\makeatother%
\endgroup%
}
  \resizebox{0.75\columnwidth}{!}{%% Creator: Matplotlib, PGF backend
%%
%% To include the figure in your LaTeX document, write
%%   \input{<filename>.pgf}
%%
%% Make sure the required packages are loaded in your preamble
%%   \usepackage{pgf}
%%
%% Also ensure that all the required font packages are loaded; for instance,
%% the lmodern package is sometimes necessary when using math font.
%%   \usepackage{lmodern}
%%
%% Figures using additional raster images can only be included by \input if
%% they are in the same directory as the main LaTeX file. For loading figures
%% from other directories you can use the `import` package
%%   \usepackage{import}
%%
%% and then include the figures with
%%   \import{<path to file>}{<filename>.pgf}
%%
%% Matplotlib used the following preamble
%%   \def\mathdefault#1{#1}
%%   \everymath=\expandafter{\the\everymath\displaystyle}
%%   \IfFileExists{scrextend.sty}{
%%     \usepackage[fontsize=10.000000pt]{scrextend}
%%   }{
%%     \renewcommand{\normalsize}{\fontsize{10.000000}{12.000000}\selectfont}
%%     \normalsize
%%   }
%%   
%%   \makeatletter\@ifpackageloaded{underscore}{}{\usepackage[strings]{underscore}}\makeatother
%%
\begingroup%
\makeatletter%
\begin{pgfpicture}%
\pgfpathrectangle{\pgfpointorigin}{\pgfqpoint{6.988192in}{5.458470in}}%
\pgfusepath{use as bounding box, clip}%
\begin{pgfscope}%
\pgfsetbuttcap%
\pgfsetmiterjoin%
\definecolor{currentfill}{rgb}{1.000000,1.000000,1.000000}%
\pgfsetfillcolor{currentfill}%
\pgfsetlinewidth{0.000000pt}%
\definecolor{currentstroke}{rgb}{0.000000,0.000000,0.000000}%
\pgfsetstrokecolor{currentstroke}%
\pgfsetdash{}{0pt}%
\pgfpathmoveto{\pgfqpoint{0.000000in}{0.000000in}}%
\pgfpathlineto{\pgfqpoint{6.988192in}{0.000000in}}%
\pgfpathlineto{\pgfqpoint{6.988192in}{5.458470in}}%
\pgfpathlineto{\pgfqpoint{0.000000in}{5.458470in}}%
\pgfpathlineto{\pgfqpoint{0.000000in}{0.000000in}}%
\pgfpathclose%
\pgfusepath{fill}%
\end{pgfscope}%
\begin{pgfscope}%
\pgfsetbuttcap%
\pgfsetmiterjoin%
\definecolor{currentfill}{rgb}{1.000000,1.000000,1.000000}%
\pgfsetfillcolor{currentfill}%
\pgfsetlinewidth{0.000000pt}%
\definecolor{currentstroke}{rgb}{0.000000,0.000000,0.000000}%
\pgfsetstrokecolor{currentstroke}%
\pgfsetstrokeopacity{0.000000}%
\pgfsetdash{}{0pt}%
\pgfpathmoveto{\pgfqpoint{0.688192in}{0.670138in}}%
\pgfpathlineto{\pgfqpoint{6.888192in}{0.670138in}}%
\pgfpathlineto{\pgfqpoint{6.888192in}{5.290138in}}%
\pgfpathlineto{\pgfqpoint{0.688192in}{5.290138in}}%
\pgfpathlineto{\pgfqpoint{0.688192in}{0.670138in}}%
\pgfpathclose%
\pgfusepath{fill}%
\end{pgfscope}%
\begin{pgfscope}%
\pgfpathrectangle{\pgfqpoint{0.688192in}{0.670138in}}{\pgfqpoint{6.200000in}{4.620000in}}%
\pgfusepath{clip}%
\pgfsetbuttcap%
\pgfsetmiterjoin%
\definecolor{currentfill}{rgb}{0.121569,0.466667,0.705882}%
\pgfsetfillcolor{currentfill}%
\pgfsetfillopacity{0.500000}%
\pgfsetlinewidth{1.003750pt}%
\definecolor{currentstroke}{rgb}{0.121569,0.466667,0.705882}%
\pgfsetstrokecolor{currentstroke}%
\pgfsetstrokeopacity{0.500000}%
\pgfsetdash{}{0pt}%
\pgfpathmoveto{\pgfqpoint{0.741425in}{1.377543in}}%
\pgfpathlineto{\pgfqpoint{0.758703in}{0.955032in}}%
\pgfpathlineto{\pgfqpoint{0.768198in}{0.875033in}}%
\pgfpathlineto{\pgfqpoint{0.774746in}{0.828781in}}%
\pgfpathlineto{\pgfqpoint{0.778243in}{0.822495in}}%
\pgfpathlineto{\pgfqpoint{0.782159in}{0.789611in}}%
\pgfpathlineto{\pgfqpoint{0.786516in}{0.779881in}}%
\pgfpathlineto{\pgfqpoint{0.792538in}{0.779145in}}%
\pgfpathlineto{\pgfqpoint{0.794668in}{0.758056in}}%
\pgfpathlineto{\pgfqpoint{0.799837in}{0.752930in}}%
\pgfpathlineto{\pgfqpoint{0.809370in}{0.751978in}}%
\pgfpathlineto{\pgfqpoint{0.812629in}{0.743975in}}%
\pgfpathlineto{\pgfqpoint{0.815972in}{0.742575in}}%
\pgfpathlineto{\pgfqpoint{0.822987in}{0.738477in}}%
\pgfpathlineto{\pgfqpoint{0.828825in}{0.734937in}}%
\pgfpathlineto{\pgfqpoint{0.829214in}{0.733319in}}%
\pgfpathlineto{\pgfqpoint{0.833044in}{0.730858in}}%
\pgfpathlineto{\pgfqpoint{0.848459in}{0.726329in}}%
\pgfpathlineto{\pgfqpoint{0.864854in}{0.720019in}}%
\pgfpathlineto{\pgfqpoint{0.887104in}{0.715517in}}%
\pgfpathlineto{\pgfqpoint{0.907479in}{0.714004in}}%
\pgfpathlineto{\pgfqpoint{0.908310in}{0.712008in}}%
\pgfpathlineto{\pgfqpoint{0.909513in}{0.708525in}}%
\pgfpathlineto{\pgfqpoint{0.912740in}{0.707284in}}%
\pgfpathlineto{\pgfqpoint{0.920440in}{0.706723in}}%
\pgfpathlineto{\pgfqpoint{0.925670in}{0.705238in}}%
\pgfpathlineto{\pgfqpoint{0.948903in}{0.702931in}}%
\pgfpathlineto{\pgfqpoint{0.951945in}{0.701707in}}%
\pgfpathlineto{\pgfqpoint{0.952035in}{0.700391in}}%
\pgfpathlineto{\pgfqpoint{0.957029in}{0.700173in}}%
\pgfpathlineto{\pgfqpoint{0.968828in}{0.697963in}}%
\pgfpathlineto{\pgfqpoint{0.974412in}{0.697738in}}%
\pgfpathlineto{\pgfqpoint{0.975275in}{0.696914in}}%
\pgfpathlineto{\pgfqpoint{1.021767in}{0.694795in}}%
\pgfpathlineto{\pgfqpoint{1.025407in}{0.690657in}}%
\pgfpathlineto{\pgfqpoint{1.027475in}{0.690338in}}%
\pgfpathlineto{\pgfqpoint{1.034837in}{0.689784in}}%
\pgfpathlineto{\pgfqpoint{1.049406in}{0.687676in}}%
\pgfpathlineto{\pgfqpoint{1.054714in}{0.687138in}}%
\pgfpathlineto{\pgfqpoint{1.059617in}{0.686467in}}%
\pgfpathlineto{\pgfqpoint{1.072141in}{0.685078in}}%
\pgfpathlineto{\pgfqpoint{1.092208in}{0.684413in}}%
\pgfpathlineto{\pgfqpoint{1.115209in}{0.684111in}}%
\pgfpathlineto{\pgfqpoint{1.131834in}{0.684071in}}%
\pgfpathlineto{\pgfqpoint{1.152628in}{0.684059in}}%
\pgfpathlineto{\pgfqpoint{1.251312in}{0.683263in}}%
\pgfpathlineto{\pgfqpoint{1.277476in}{0.683159in}}%
\pgfpathlineto{\pgfqpoint{1.314870in}{0.682855in}}%
\pgfpathlineto{\pgfqpoint{1.369253in}{0.682756in}}%
\pgfpathlineto{\pgfqpoint{1.398687in}{0.682288in}}%
\pgfpathlineto{\pgfqpoint{1.467852in}{0.682134in}}%
\pgfpathlineto{\pgfqpoint{1.557026in}{0.681680in}}%
\pgfpathlineto{\pgfqpoint{1.627242in}{0.680913in}}%
\pgfpathlineto{\pgfqpoint{1.737728in}{0.680478in}}%
\pgfpathlineto{\pgfqpoint{1.887036in}{0.679610in}}%
\pgfpathlineto{\pgfqpoint{2.037481in}{0.678826in}}%
\pgfpathlineto{\pgfqpoint{2.258348in}{0.677741in}}%
\pgfpathlineto{\pgfqpoint{2.626338in}{0.676361in}}%
\pgfpathlineto{\pgfqpoint{3.263784in}{0.674352in}}%
\pgfpathlineto{\pgfqpoint{5.322800in}{0.670138in}}%
\pgfpathlineto{\pgfqpoint{6.888192in}{0.683471in}}%
\pgfpathlineto{\pgfqpoint{4.623274in}{0.688107in}}%
\pgfpathlineto{\pgfqpoint{3.922083in}{0.690316in}}%
\pgfpathlineto{\pgfqpoint{3.517294in}{0.691835in}}%
\pgfpathlineto{\pgfqpoint{3.274341in}{0.693028in}}%
\pgfpathlineto{\pgfqpoint{3.108851in}{0.693891in}}%
\pgfpathlineto{\pgfqpoint{2.944612in}{0.694845in}}%
\pgfpathlineto{\pgfqpoint{2.823078in}{0.695324in}}%
\pgfpathlineto{\pgfqpoint{2.745840in}{0.696168in}}%
\pgfpathlineto{\pgfqpoint{2.647748in}{0.696667in}}%
\pgfpathlineto{\pgfqpoint{2.571667in}{0.696836in}}%
\pgfpathlineto{\pgfqpoint{2.539290in}{0.697351in}}%
\pgfpathlineto{\pgfqpoint{2.479468in}{0.697460in}}%
\pgfpathlineto{\pgfqpoint{2.438334in}{0.697795in}}%
\pgfpathlineto{\pgfqpoint{2.409554in}{0.697909in}}%
\pgfpathlineto{\pgfqpoint{2.301002in}{0.698784in}}%
\pgfpathlineto{\pgfqpoint{2.278129in}{0.698798in}}%
\pgfpathlineto{\pgfqpoint{2.259841in}{0.698842in}}%
\pgfpathlineto{\pgfqpoint{2.234540in}{0.699174in}}%
\pgfpathlineto{\pgfqpoint{2.212467in}{0.699905in}}%
\pgfpathlineto{\pgfqpoint{2.198690in}{0.701434in}}%
\pgfpathlineto{\pgfqpoint{2.193297in}{0.702172in}}%
\pgfpathlineto{\pgfqpoint{2.187458in}{0.702763in}}%
\pgfpathlineto{\pgfqpoint{2.171432in}{0.705082in}}%
\pgfpathlineto{\pgfqpoint{2.163334in}{0.705692in}}%
\pgfpathlineto{\pgfqpoint{2.161059in}{0.706043in}}%
\pgfpathlineto{\pgfqpoint{2.157055in}{0.710594in}}%
\pgfpathlineto{\pgfqpoint{2.105914in}{0.712924in}}%
\pgfpathlineto{\pgfqpoint{2.104964in}{0.713831in}}%
\pgfpathlineto{\pgfqpoint{2.098822in}{0.714079in}}%
\pgfpathlineto{\pgfqpoint{2.085843in}{0.716510in}}%
\pgfpathlineto{\pgfqpoint{2.080349in}{0.716749in}}%
\pgfpathlineto{\pgfqpoint{2.080251in}{0.718197in}}%
\pgfpathlineto{\pgfqpoint{2.076904in}{0.719544in}}%
\pgfpathlineto{\pgfqpoint{2.051349in}{0.722081in}}%
\pgfpathlineto{\pgfqpoint{2.045595in}{0.723715in}}%
\pgfpathlineto{\pgfqpoint{2.037125in}{0.724332in}}%
\pgfpathlineto{\pgfqpoint{2.033576in}{0.725697in}}%
\pgfpathlineto{\pgfqpoint{2.032252in}{0.729528in}}%
\pgfpathlineto{\pgfqpoint{2.031338in}{0.731724in}}%
\pgfpathlineto{\pgfqpoint{2.008926in}{0.733388in}}%
\pgfpathlineto{\pgfqpoint{1.984450in}{0.738340in}}%
\pgfpathlineto{\pgfqpoint{1.966417in}{0.745282in}}%
\pgfpathlineto{\pgfqpoint{1.949459in}{0.750264in}}%
\pgfpathlineto{\pgfqpoint{1.945246in}{0.752971in}}%
\pgfpathlineto{\pgfqpoint{1.944819in}{0.754750in}}%
\pgfpathlineto{\pgfqpoint{1.938397in}{0.758644in}}%
\pgfpathlineto{\pgfqpoint{1.930681in}{0.763152in}}%
\pgfpathlineto{\pgfqpoint{1.927003in}{0.764692in}}%
\pgfpathlineto{\pgfqpoint{1.923418in}{0.773495in}}%
\pgfpathlineto{\pgfqpoint{1.912932in}{0.774543in}}%
\pgfpathlineto{\pgfqpoint{1.907246in}{0.780181in}}%
\pgfpathlineto{\pgfqpoint{1.904902in}{0.803379in}}%
\pgfpathlineto{\pgfqpoint{1.898279in}{0.804188in}}%
\pgfpathlineto{\pgfqpoint{1.893486in}{0.814892in}}%
\pgfpathlineto{\pgfqpoint{1.889178in}{0.851064in}}%
\pgfpathlineto{\pgfqpoint{1.885331in}{0.857978in}}%
\pgfpathlineto{\pgfqpoint{1.878129in}{0.908856in}}%
\pgfpathlineto{\pgfqpoint{1.867684in}{0.996854in}}%
\pgfpathlineto{\pgfqpoint{1.848679in}{1.461617in}}%
\pgfpathlineto{\pgfqpoint{0.741425in}{1.377543in}}%
\pgfpathclose%
\pgfusepath{stroke,fill}%
\end{pgfscope}%
\begin{pgfscope}%
\pgfpathrectangle{\pgfqpoint{0.688192in}{0.670138in}}{\pgfqpoint{6.200000in}{4.620000in}}%
\pgfusepath{clip}%
\pgfsetbuttcap%
\pgfsetroundjoin%
\pgfsetlinewidth{1.003750pt}%
\definecolor{currentstroke}{rgb}{1.000000,0.000000,0.000000}%
\pgfsetstrokecolor{currentstroke}%
\pgfsetdash{}{0pt}%
\pgfpathmoveto{\pgfqpoint{1.380312in}{4.826960in}}%
\pgfpathcurveto{\pgfqpoint{1.388548in}{4.826960in}}{\pgfqpoint{1.396449in}{4.830233in}}{\pgfqpoint{1.402272in}{4.836057in}}%
\pgfpathcurveto{\pgfqpoint{1.408096in}{4.841881in}}{\pgfqpoint{1.411369in}{4.849781in}}{\pgfqpoint{1.411369in}{4.858017in}}%
\pgfpathcurveto{\pgfqpoint{1.411369in}{4.866253in}}{\pgfqpoint{1.408096in}{4.874153in}}{\pgfqpoint{1.402272in}{4.879977in}}%
\pgfpathcurveto{\pgfqpoint{1.396449in}{4.885801in}}{\pgfqpoint{1.388548in}{4.889073in}}{\pgfqpoint{1.380312in}{4.889073in}}%
\pgfpathcurveto{\pgfqpoint{1.372076in}{4.889073in}}{\pgfqpoint{1.364176in}{4.885801in}}{\pgfqpoint{1.358352in}{4.879977in}}%
\pgfpathcurveto{\pgfqpoint{1.352528in}{4.874153in}}{\pgfqpoint{1.349256in}{4.866253in}}{\pgfqpoint{1.349256in}{4.858017in}}%
\pgfpathcurveto{\pgfqpoint{1.349256in}{4.849781in}}{\pgfqpoint{1.352528in}{4.841881in}}{\pgfqpoint{1.358352in}{4.836057in}}%
\pgfpathcurveto{\pgfqpoint{1.364176in}{4.830233in}}{\pgfqpoint{1.372076in}{4.826960in}}{\pgfqpoint{1.380312in}{4.826960in}}%
\pgfpathlineto{\pgfqpoint{1.380312in}{4.826960in}}%
\pgfpathclose%
\pgfusepath{stroke}%
\end{pgfscope}%
\begin{pgfscope}%
\pgfpathrectangle{\pgfqpoint{0.688192in}{0.670138in}}{\pgfqpoint{6.200000in}{4.620000in}}%
\pgfusepath{clip}%
\pgfsetbuttcap%
\pgfsetroundjoin%
\pgfsetlinewidth{1.003750pt}%
\definecolor{currentstroke}{rgb}{1.000000,0.000000,0.000000}%
\pgfsetstrokecolor{currentstroke}%
\pgfsetdash{}{0pt}%
\pgfpathmoveto{\pgfqpoint{1.103776in}{2.239473in}}%
\pgfpathcurveto{\pgfqpoint{1.112013in}{2.239473in}}{\pgfqpoint{1.119913in}{2.242745in}}{\pgfqpoint{1.125737in}{2.248569in}}%
\pgfpathcurveto{\pgfqpoint{1.131560in}{2.254393in}}{\pgfqpoint{1.134833in}{2.262293in}}{\pgfqpoint{1.134833in}{2.270529in}}%
\pgfpathcurveto{\pgfqpoint{1.134833in}{2.278765in}}{\pgfqpoint{1.131560in}{2.286665in}}{\pgfqpoint{1.125737in}{2.292489in}}%
\pgfpathcurveto{\pgfqpoint{1.119913in}{2.298313in}}{\pgfqpoint{1.112013in}{2.301586in}}{\pgfqpoint{1.103776in}{2.301586in}}%
\pgfpathcurveto{\pgfqpoint{1.095540in}{2.301586in}}{\pgfqpoint{1.087640in}{2.298313in}}{\pgfqpoint{1.081816in}{2.292489in}}%
\pgfpathcurveto{\pgfqpoint{1.075992in}{2.286665in}}{\pgfqpoint{1.072720in}{2.278765in}}{\pgfqpoint{1.072720in}{2.270529in}}%
\pgfpathcurveto{\pgfqpoint{1.072720in}{2.262293in}}{\pgfqpoint{1.075992in}{2.254393in}}{\pgfqpoint{1.081816in}{2.248569in}}%
\pgfpathcurveto{\pgfqpoint{1.087640in}{2.242745in}}{\pgfqpoint{1.095540in}{2.239473in}}{\pgfqpoint{1.103776in}{2.239473in}}%
\pgfpathlineto{\pgfqpoint{1.103776in}{2.239473in}}%
\pgfpathclose%
\pgfusepath{stroke}%
\end{pgfscope}%
\begin{pgfscope}%
\pgfpathrectangle{\pgfqpoint{0.688192in}{0.670138in}}{\pgfqpoint{6.200000in}{4.620000in}}%
\pgfusepath{clip}%
\pgfsetbuttcap%
\pgfsetroundjoin%
\pgfsetlinewidth{1.003750pt}%
\definecolor{currentstroke}{rgb}{1.000000,0.000000,0.000000}%
\pgfsetstrokecolor{currentstroke}%
\pgfsetdash{}{0pt}%
\pgfpathmoveto{\pgfqpoint{1.146295in}{2.309659in}}%
\pgfpathcurveto{\pgfqpoint{1.154531in}{2.309659in}}{\pgfqpoint{1.162431in}{2.312932in}}{\pgfqpoint{1.168255in}{2.318756in}}%
\pgfpathcurveto{\pgfqpoint{1.174079in}{2.324579in}}{\pgfqpoint{1.177352in}{2.332480in}}{\pgfqpoint{1.177352in}{2.340716in}}%
\pgfpathcurveto{\pgfqpoint{1.177352in}{2.348952in}}{\pgfqpoint{1.174079in}{2.356852in}}{\pgfqpoint{1.168255in}{2.362676in}}%
\pgfpathcurveto{\pgfqpoint{1.162431in}{2.368500in}}{\pgfqpoint{1.154531in}{2.371772in}}{\pgfqpoint{1.146295in}{2.371772in}}%
\pgfpathcurveto{\pgfqpoint{1.138059in}{2.371772in}}{\pgfqpoint{1.130159in}{2.368500in}}{\pgfqpoint{1.124335in}{2.362676in}}%
\pgfpathcurveto{\pgfqpoint{1.118511in}{2.356852in}}{\pgfqpoint{1.115239in}{2.348952in}}{\pgfqpoint{1.115239in}{2.340716in}}%
\pgfpathcurveto{\pgfqpoint{1.115239in}{2.332480in}}{\pgfqpoint{1.118511in}{2.324579in}}{\pgfqpoint{1.124335in}{2.318756in}}%
\pgfpathcurveto{\pgfqpoint{1.130159in}{2.312932in}}{\pgfqpoint{1.138059in}{2.309659in}}{\pgfqpoint{1.146295in}{2.309659in}}%
\pgfpathlineto{\pgfqpoint{1.146295in}{2.309659in}}%
\pgfpathclose%
\pgfusepath{stroke}%
\end{pgfscope}%
\begin{pgfscope}%
\pgfpathrectangle{\pgfqpoint{0.688192in}{0.670138in}}{\pgfqpoint{6.200000in}{4.620000in}}%
\pgfusepath{clip}%
\pgfsetbuttcap%
\pgfsetroundjoin%
\pgfsetlinewidth{1.003750pt}%
\definecolor{currentstroke}{rgb}{1.000000,0.000000,0.000000}%
\pgfsetstrokecolor{currentstroke}%
\pgfsetdash{}{0pt}%
\pgfpathmoveto{\pgfqpoint{1.283972in}{2.337286in}}%
\pgfpathcurveto{\pgfqpoint{1.292208in}{2.337286in}}{\pgfqpoint{1.300108in}{2.340558in}}{\pgfqpoint{1.305932in}{2.346382in}}%
\pgfpathcurveto{\pgfqpoint{1.311756in}{2.352206in}}{\pgfqpoint{1.315028in}{2.360106in}}{\pgfqpoint{1.315028in}{2.368343in}}%
\pgfpathcurveto{\pgfqpoint{1.315028in}{2.376579in}}{\pgfqpoint{1.311756in}{2.384479in}}{\pgfqpoint{1.305932in}{2.390303in}}%
\pgfpathcurveto{\pgfqpoint{1.300108in}{2.396127in}}{\pgfqpoint{1.292208in}{2.399399in}}{\pgfqpoint{1.283972in}{2.399399in}}%
\pgfpathcurveto{\pgfqpoint{1.275735in}{2.399399in}}{\pgfqpoint{1.267835in}{2.396127in}}{\pgfqpoint{1.262011in}{2.390303in}}%
\pgfpathcurveto{\pgfqpoint{1.256187in}{2.384479in}}{\pgfqpoint{1.252915in}{2.376579in}}{\pgfqpoint{1.252915in}{2.368343in}}%
\pgfpathcurveto{\pgfqpoint{1.252915in}{2.360106in}}{\pgfqpoint{1.256187in}{2.352206in}}{\pgfqpoint{1.262011in}{2.346382in}}%
\pgfpathcurveto{\pgfqpoint{1.267835in}{2.340558in}}{\pgfqpoint{1.275735in}{2.337286in}}{\pgfqpoint{1.283972in}{2.337286in}}%
\pgfpathlineto{\pgfqpoint{1.283972in}{2.337286in}}%
\pgfpathclose%
\pgfusepath{stroke}%
\end{pgfscope}%
\begin{pgfscope}%
\pgfpathrectangle{\pgfqpoint{0.688192in}{0.670138in}}{\pgfqpoint{6.200000in}{4.620000in}}%
\pgfusepath{clip}%
\pgfsetbuttcap%
\pgfsetroundjoin%
\pgfsetlinewidth{1.003750pt}%
\definecolor{currentstroke}{rgb}{1.000000,0.000000,0.000000}%
\pgfsetstrokecolor{currentstroke}%
\pgfsetdash{}{0pt}%
\pgfpathmoveto{\pgfqpoint{1.194044in}{2.459126in}}%
\pgfpathcurveto{\pgfqpoint{1.202281in}{2.459126in}}{\pgfqpoint{1.210181in}{2.462398in}}{\pgfqpoint{1.216005in}{2.468222in}}%
\pgfpathcurveto{\pgfqpoint{1.221828in}{2.474046in}}{\pgfqpoint{1.225101in}{2.481946in}}{\pgfqpoint{1.225101in}{2.490182in}}%
\pgfpathcurveto{\pgfqpoint{1.225101in}{2.498419in}}{\pgfqpoint{1.221828in}{2.506319in}}{\pgfqpoint{1.216005in}{2.512143in}}%
\pgfpathcurveto{\pgfqpoint{1.210181in}{2.517967in}}{\pgfqpoint{1.202281in}{2.521239in}}{\pgfqpoint{1.194044in}{2.521239in}}%
\pgfpathcurveto{\pgfqpoint{1.185808in}{2.521239in}}{\pgfqpoint{1.177908in}{2.517967in}}{\pgfqpoint{1.172084in}{2.512143in}}%
\pgfpathcurveto{\pgfqpoint{1.166260in}{2.506319in}}{\pgfqpoint{1.162988in}{2.498419in}}{\pgfqpoint{1.162988in}{2.490182in}}%
\pgfpathcurveto{\pgfqpoint{1.162988in}{2.481946in}}{\pgfqpoint{1.166260in}{2.474046in}}{\pgfqpoint{1.172084in}{2.468222in}}%
\pgfpathcurveto{\pgfqpoint{1.177908in}{2.462398in}}{\pgfqpoint{1.185808in}{2.459126in}}{\pgfqpoint{1.194044in}{2.459126in}}%
\pgfpathlineto{\pgfqpoint{1.194044in}{2.459126in}}%
\pgfpathclose%
\pgfusepath{stroke}%
\end{pgfscope}%
\begin{pgfscope}%
\pgfpathrectangle{\pgfqpoint{0.688192in}{0.670138in}}{\pgfqpoint{6.200000in}{4.620000in}}%
\pgfusepath{clip}%
\pgfsetbuttcap%
\pgfsetroundjoin%
\pgfsetlinewidth{1.003750pt}%
\definecolor{currentstroke}{rgb}{1.000000,0.000000,0.000000}%
\pgfsetstrokecolor{currentstroke}%
\pgfsetdash{}{0pt}%
\pgfpathmoveto{\pgfqpoint{1.126156in}{2.039611in}}%
\pgfpathcurveto{\pgfqpoint{1.134392in}{2.039611in}}{\pgfqpoint{1.142292in}{2.042884in}}{\pgfqpoint{1.148116in}{2.048708in}}%
\pgfpathcurveto{\pgfqpoint{1.153940in}{2.054532in}}{\pgfqpoint{1.157212in}{2.062432in}}{\pgfqpoint{1.157212in}{2.070668in}}%
\pgfpathcurveto{\pgfqpoint{1.157212in}{2.078904in}}{\pgfqpoint{1.153940in}{2.086804in}}{\pgfqpoint{1.148116in}{2.092628in}}%
\pgfpathcurveto{\pgfqpoint{1.142292in}{2.098452in}}{\pgfqpoint{1.134392in}{2.101724in}}{\pgfqpoint{1.126156in}{2.101724in}}%
\pgfpathcurveto{\pgfqpoint{1.117920in}{2.101724in}}{\pgfqpoint{1.110020in}{2.098452in}}{\pgfqpoint{1.104196in}{2.092628in}}%
\pgfpathcurveto{\pgfqpoint{1.098372in}{2.086804in}}{\pgfqpoint{1.095099in}{2.078904in}}{\pgfqpoint{1.095099in}{2.070668in}}%
\pgfpathcurveto{\pgfqpoint{1.095099in}{2.062432in}}{\pgfqpoint{1.098372in}{2.054532in}}{\pgfqpoint{1.104196in}{2.048708in}}%
\pgfpathcurveto{\pgfqpoint{1.110020in}{2.042884in}}{\pgfqpoint{1.117920in}{2.039611in}}{\pgfqpoint{1.126156in}{2.039611in}}%
\pgfpathlineto{\pgfqpoint{1.126156in}{2.039611in}}%
\pgfpathclose%
\pgfusepath{stroke}%
\end{pgfscope}%
\begin{pgfscope}%
\pgfpathrectangle{\pgfqpoint{0.688192in}{0.670138in}}{\pgfqpoint{6.200000in}{4.620000in}}%
\pgfusepath{clip}%
\pgfsetbuttcap%
\pgfsetroundjoin%
\pgfsetlinewidth{1.003750pt}%
\definecolor{currentstroke}{rgb}{1.000000,0.000000,0.000000}%
\pgfsetstrokecolor{currentstroke}%
\pgfsetdash{}{0pt}%
\pgfpathmoveto{\pgfqpoint{1.073783in}{1.736453in}}%
\pgfpathcurveto{\pgfqpoint{1.082020in}{1.736453in}}{\pgfqpoint{1.089920in}{1.739725in}}{\pgfqpoint{1.095744in}{1.745549in}}%
\pgfpathcurveto{\pgfqpoint{1.101568in}{1.751373in}}{\pgfqpoint{1.104840in}{1.759273in}}{\pgfqpoint{1.104840in}{1.767510in}}%
\pgfpathcurveto{\pgfqpoint{1.104840in}{1.775746in}}{\pgfqpoint{1.101568in}{1.783646in}}{\pgfqpoint{1.095744in}{1.789470in}}%
\pgfpathcurveto{\pgfqpoint{1.089920in}{1.795294in}}{\pgfqpoint{1.082020in}{1.798566in}}{\pgfqpoint{1.073783in}{1.798566in}}%
\pgfpathcurveto{\pgfqpoint{1.065547in}{1.798566in}}{\pgfqpoint{1.057647in}{1.795294in}}{\pgfqpoint{1.051823in}{1.789470in}}%
\pgfpathcurveto{\pgfqpoint{1.045999in}{1.783646in}}{\pgfqpoint{1.042727in}{1.775746in}}{\pgfqpoint{1.042727in}{1.767510in}}%
\pgfpathcurveto{\pgfqpoint{1.042727in}{1.759273in}}{\pgfqpoint{1.045999in}{1.751373in}}{\pgfqpoint{1.051823in}{1.745549in}}%
\pgfpathcurveto{\pgfqpoint{1.057647in}{1.739725in}}{\pgfqpoint{1.065547in}{1.736453in}}{\pgfqpoint{1.073783in}{1.736453in}}%
\pgfpathlineto{\pgfqpoint{1.073783in}{1.736453in}}%
\pgfpathclose%
\pgfusepath{stroke}%
\end{pgfscope}%
\begin{pgfscope}%
\pgfpathrectangle{\pgfqpoint{0.688192in}{0.670138in}}{\pgfqpoint{6.200000in}{4.620000in}}%
\pgfusepath{clip}%
\pgfsetbuttcap%
\pgfsetroundjoin%
\pgfsetlinewidth{1.003750pt}%
\definecolor{currentstroke}{rgb}{1.000000,0.000000,0.000000}%
\pgfsetstrokecolor{currentstroke}%
\pgfsetdash{}{0pt}%
\pgfpathmoveto{\pgfqpoint{1.071350in}{1.723563in}}%
\pgfpathcurveto{\pgfqpoint{1.079586in}{1.723563in}}{\pgfqpoint{1.087486in}{1.726835in}}{\pgfqpoint{1.093310in}{1.732659in}}%
\pgfpathcurveto{\pgfqpoint{1.099134in}{1.738483in}}{\pgfqpoint{1.102407in}{1.746383in}}{\pgfqpoint{1.102407in}{1.754620in}}%
\pgfpathcurveto{\pgfqpoint{1.102407in}{1.762856in}}{\pgfqpoint{1.099134in}{1.770756in}}{\pgfqpoint{1.093310in}{1.776580in}}%
\pgfpathcurveto{\pgfqpoint{1.087486in}{1.782404in}}{\pgfqpoint{1.079586in}{1.785676in}}{\pgfqpoint{1.071350in}{1.785676in}}%
\pgfpathcurveto{\pgfqpoint{1.063114in}{1.785676in}}{\pgfqpoint{1.055214in}{1.782404in}}{\pgfqpoint{1.049390in}{1.776580in}}%
\pgfpathcurveto{\pgfqpoint{1.043566in}{1.770756in}}{\pgfqpoint{1.040294in}{1.762856in}}{\pgfqpoint{1.040294in}{1.754620in}}%
\pgfpathcurveto{\pgfqpoint{1.040294in}{1.746383in}}{\pgfqpoint{1.043566in}{1.738483in}}{\pgfqpoint{1.049390in}{1.732659in}}%
\pgfpathcurveto{\pgfqpoint{1.055214in}{1.726835in}}{\pgfqpoint{1.063114in}{1.723563in}}{\pgfqpoint{1.071350in}{1.723563in}}%
\pgfpathlineto{\pgfqpoint{1.071350in}{1.723563in}}%
\pgfpathclose%
\pgfusepath{stroke}%
\end{pgfscope}%
\begin{pgfscope}%
\pgfpathrectangle{\pgfqpoint{0.688192in}{0.670138in}}{\pgfqpoint{6.200000in}{4.620000in}}%
\pgfusepath{clip}%
\pgfsetbuttcap%
\pgfsetroundjoin%
\pgfsetlinewidth{1.003750pt}%
\definecolor{currentstroke}{rgb}{1.000000,0.000000,0.000000}%
\pgfsetstrokecolor{currentstroke}%
\pgfsetdash{}{0pt}%
\pgfpathmoveto{\pgfqpoint{1.176638in}{2.187985in}}%
\pgfpathcurveto{\pgfqpoint{1.184874in}{2.187985in}}{\pgfqpoint{1.192774in}{2.191257in}}{\pgfqpoint{1.198598in}{2.197081in}}%
\pgfpathcurveto{\pgfqpoint{1.204422in}{2.202905in}}{\pgfqpoint{1.207694in}{2.210805in}}{\pgfqpoint{1.207694in}{2.219041in}}%
\pgfpathcurveto{\pgfqpoint{1.207694in}{2.227277in}}{\pgfqpoint{1.204422in}{2.235178in}}{\pgfqpoint{1.198598in}{2.241001in}}%
\pgfpathcurveto{\pgfqpoint{1.192774in}{2.246825in}}{\pgfqpoint{1.184874in}{2.250098in}}{\pgfqpoint{1.176638in}{2.250098in}}%
\pgfpathcurveto{\pgfqpoint{1.168401in}{2.250098in}}{\pgfqpoint{1.160501in}{2.246825in}}{\pgfqpoint{1.154677in}{2.241001in}}%
\pgfpathcurveto{\pgfqpoint{1.148853in}{2.235178in}}{\pgfqpoint{1.145581in}{2.227277in}}{\pgfqpoint{1.145581in}{2.219041in}}%
\pgfpathcurveto{\pgfqpoint{1.145581in}{2.210805in}}{\pgfqpoint{1.148853in}{2.202905in}}{\pgfqpoint{1.154677in}{2.197081in}}%
\pgfpathcurveto{\pgfqpoint{1.160501in}{2.191257in}}{\pgfqpoint{1.168401in}{2.187985in}}{\pgfqpoint{1.176638in}{2.187985in}}%
\pgfpathlineto{\pgfqpoint{1.176638in}{2.187985in}}%
\pgfpathclose%
\pgfusepath{stroke}%
\end{pgfscope}%
\begin{pgfscope}%
\pgfpathrectangle{\pgfqpoint{0.688192in}{0.670138in}}{\pgfqpoint{6.200000in}{4.620000in}}%
\pgfusepath{clip}%
\pgfsetbuttcap%
\pgfsetroundjoin%
\pgfsetlinewidth{1.003750pt}%
\definecolor{currentstroke}{rgb}{1.000000,0.000000,0.000000}%
\pgfsetstrokecolor{currentstroke}%
\pgfsetdash{}{0pt}%
\pgfpathmoveto{\pgfqpoint{1.160265in}{1.939369in}}%
\pgfpathcurveto{\pgfqpoint{1.168501in}{1.939369in}}{\pgfqpoint{1.176401in}{1.942641in}}{\pgfqpoint{1.182225in}{1.948465in}}%
\pgfpathcurveto{\pgfqpoint{1.188049in}{1.954289in}}{\pgfqpoint{1.191321in}{1.962189in}}{\pgfqpoint{1.191321in}{1.970426in}}%
\pgfpathcurveto{\pgfqpoint{1.191321in}{1.978662in}}{\pgfqpoint{1.188049in}{1.986562in}}{\pgfqpoint{1.182225in}{1.992386in}}%
\pgfpathcurveto{\pgfqpoint{1.176401in}{1.998210in}}{\pgfqpoint{1.168501in}{2.001482in}}{\pgfqpoint{1.160265in}{2.001482in}}%
\pgfpathcurveto{\pgfqpoint{1.152029in}{2.001482in}}{\pgfqpoint{1.144129in}{1.998210in}}{\pgfqpoint{1.138305in}{1.992386in}}%
\pgfpathcurveto{\pgfqpoint{1.132481in}{1.986562in}}{\pgfqpoint{1.129208in}{1.978662in}}{\pgfqpoint{1.129208in}{1.970426in}}%
\pgfpathcurveto{\pgfqpoint{1.129208in}{1.962189in}}{\pgfqpoint{1.132481in}{1.954289in}}{\pgfqpoint{1.138305in}{1.948465in}}%
\pgfpathcurveto{\pgfqpoint{1.144129in}{1.942641in}}{\pgfqpoint{1.152029in}{1.939369in}}{\pgfqpoint{1.160265in}{1.939369in}}%
\pgfpathlineto{\pgfqpoint{1.160265in}{1.939369in}}%
\pgfpathclose%
\pgfusepath{stroke}%
\end{pgfscope}%
\begin{pgfscope}%
\pgfpathrectangle{\pgfqpoint{0.688192in}{0.670138in}}{\pgfqpoint{6.200000in}{4.620000in}}%
\pgfusepath{clip}%
\pgfsetbuttcap%
\pgfsetroundjoin%
\pgfsetlinewidth{1.003750pt}%
\definecolor{currentstroke}{rgb}{1.000000,0.000000,0.000000}%
\pgfsetstrokecolor{currentstroke}%
\pgfsetdash{}{0pt}%
\pgfpathmoveto{\pgfqpoint{1.160053in}{1.935902in}}%
\pgfpathcurveto{\pgfqpoint{1.168289in}{1.935902in}}{\pgfqpoint{1.176189in}{1.939174in}}{\pgfqpoint{1.182013in}{1.944998in}}%
\pgfpathcurveto{\pgfqpoint{1.187837in}{1.950822in}}{\pgfqpoint{1.191109in}{1.958722in}}{\pgfqpoint{1.191109in}{1.966959in}}%
\pgfpathcurveto{\pgfqpoint{1.191109in}{1.975195in}}{\pgfqpoint{1.187837in}{1.983095in}}{\pgfqpoint{1.182013in}{1.988919in}}%
\pgfpathcurveto{\pgfqpoint{1.176189in}{1.994743in}}{\pgfqpoint{1.168289in}{1.998015in}}{\pgfqpoint{1.160053in}{1.998015in}}%
\pgfpathcurveto{\pgfqpoint{1.151817in}{1.998015in}}{\pgfqpoint{1.143916in}{1.994743in}}{\pgfqpoint{1.138093in}{1.988919in}}%
\pgfpathcurveto{\pgfqpoint{1.132269in}{1.983095in}}{\pgfqpoint{1.128996in}{1.975195in}}{\pgfqpoint{1.128996in}{1.966959in}}%
\pgfpathcurveto{\pgfqpoint{1.128996in}{1.958722in}}{\pgfqpoint{1.132269in}{1.950822in}}{\pgfqpoint{1.138093in}{1.944998in}}%
\pgfpathcurveto{\pgfqpoint{1.143916in}{1.939174in}}{\pgfqpoint{1.151817in}{1.935902in}}{\pgfqpoint{1.160053in}{1.935902in}}%
\pgfpathlineto{\pgfqpoint{1.160053in}{1.935902in}}%
\pgfpathclose%
\pgfusepath{stroke}%
\end{pgfscope}%
\begin{pgfscope}%
\pgfpathrectangle{\pgfqpoint{0.688192in}{0.670138in}}{\pgfqpoint{6.200000in}{4.620000in}}%
\pgfusepath{clip}%
\pgfsetbuttcap%
\pgfsetroundjoin%
\pgfsetlinewidth{1.003750pt}%
\definecolor{currentstroke}{rgb}{1.000000,0.000000,0.000000}%
\pgfsetstrokecolor{currentstroke}%
\pgfsetdash{}{0pt}%
\pgfpathmoveto{\pgfqpoint{1.157986in}{1.610198in}}%
\pgfpathcurveto{\pgfqpoint{1.166223in}{1.610198in}}{\pgfqpoint{1.174123in}{1.613470in}}{\pgfqpoint{1.179947in}{1.619294in}}%
\pgfpathcurveto{\pgfqpoint{1.185771in}{1.625118in}}{\pgfqpoint{1.189043in}{1.633018in}}{\pgfqpoint{1.189043in}{1.641254in}}%
\pgfpathcurveto{\pgfqpoint{1.189043in}{1.649491in}}{\pgfqpoint{1.185771in}{1.657391in}}{\pgfqpoint{1.179947in}{1.663215in}}%
\pgfpathcurveto{\pgfqpoint{1.174123in}{1.669038in}}{\pgfqpoint{1.166223in}{1.672311in}}{\pgfqpoint{1.157986in}{1.672311in}}%
\pgfpathcurveto{\pgfqpoint{1.149750in}{1.672311in}}{\pgfqpoint{1.141850in}{1.669038in}}{\pgfqpoint{1.136026in}{1.663215in}}%
\pgfpathcurveto{\pgfqpoint{1.130202in}{1.657391in}}{\pgfqpoint{1.126930in}{1.649491in}}{\pgfqpoint{1.126930in}{1.641254in}}%
\pgfpathcurveto{\pgfqpoint{1.126930in}{1.633018in}}{\pgfqpoint{1.130202in}{1.625118in}}{\pgfqpoint{1.136026in}{1.619294in}}%
\pgfpathcurveto{\pgfqpoint{1.141850in}{1.613470in}}{\pgfqpoint{1.149750in}{1.610198in}}{\pgfqpoint{1.157986in}{1.610198in}}%
\pgfpathlineto{\pgfqpoint{1.157986in}{1.610198in}}%
\pgfpathclose%
\pgfusepath{stroke}%
\end{pgfscope}%
\begin{pgfscope}%
\pgfpathrectangle{\pgfqpoint{0.688192in}{0.670138in}}{\pgfqpoint{6.200000in}{4.620000in}}%
\pgfusepath{clip}%
\pgfsetbuttcap%
\pgfsetroundjoin%
\pgfsetlinewidth{1.003750pt}%
\definecolor{currentstroke}{rgb}{1.000000,0.000000,0.000000}%
\pgfsetstrokecolor{currentstroke}%
\pgfsetdash{}{0pt}%
\pgfpathmoveto{\pgfqpoint{1.155798in}{1.612752in}}%
\pgfpathcurveto{\pgfqpoint{1.164035in}{1.612752in}}{\pgfqpoint{1.171935in}{1.616024in}}{\pgfqpoint{1.177759in}{1.621848in}}%
\pgfpathcurveto{\pgfqpoint{1.183582in}{1.627672in}}{\pgfqpoint{1.186855in}{1.635572in}}{\pgfqpoint{1.186855in}{1.643808in}}%
\pgfpathcurveto{\pgfqpoint{1.186855in}{1.652044in}}{\pgfqpoint{1.183582in}{1.659945in}}{\pgfqpoint{1.177759in}{1.665768in}}%
\pgfpathcurveto{\pgfqpoint{1.171935in}{1.671592in}}{\pgfqpoint{1.164035in}{1.674865in}}{\pgfqpoint{1.155798in}{1.674865in}}%
\pgfpathcurveto{\pgfqpoint{1.147562in}{1.674865in}}{\pgfqpoint{1.139662in}{1.671592in}}{\pgfqpoint{1.133838in}{1.665768in}}%
\pgfpathcurveto{\pgfqpoint{1.128014in}{1.659945in}}{\pgfqpoint{1.124742in}{1.652044in}}{\pgfqpoint{1.124742in}{1.643808in}}%
\pgfpathcurveto{\pgfqpoint{1.124742in}{1.635572in}}{\pgfqpoint{1.128014in}{1.627672in}}{\pgfqpoint{1.133838in}{1.621848in}}%
\pgfpathcurveto{\pgfqpoint{1.139662in}{1.616024in}}{\pgfqpoint{1.147562in}{1.612752in}}{\pgfqpoint{1.155798in}{1.612752in}}%
\pgfpathlineto{\pgfqpoint{1.155798in}{1.612752in}}%
\pgfpathclose%
\pgfusepath{stroke}%
\end{pgfscope}%
\begin{pgfscope}%
\pgfpathrectangle{\pgfqpoint{0.688192in}{0.670138in}}{\pgfqpoint{6.200000in}{4.620000in}}%
\pgfusepath{clip}%
\pgfsetbuttcap%
\pgfsetroundjoin%
\pgfsetlinewidth{1.003750pt}%
\definecolor{currentstroke}{rgb}{1.000000,0.000000,0.000000}%
\pgfsetstrokecolor{currentstroke}%
\pgfsetdash{}{0pt}%
\pgfpathmoveto{\pgfqpoint{1.295347in}{1.895177in}}%
\pgfpathcurveto{\pgfqpoint{1.303583in}{1.895177in}}{\pgfqpoint{1.311483in}{1.898449in}}{\pgfqpoint{1.317307in}{1.904273in}}%
\pgfpathcurveto{\pgfqpoint{1.323131in}{1.910097in}}{\pgfqpoint{1.326403in}{1.917997in}}{\pgfqpoint{1.326403in}{1.926233in}}%
\pgfpathcurveto{\pgfqpoint{1.326403in}{1.934470in}}{\pgfqpoint{1.323131in}{1.942370in}}{\pgfqpoint{1.317307in}{1.948193in}}%
\pgfpathcurveto{\pgfqpoint{1.311483in}{1.954017in}}{\pgfqpoint{1.303583in}{1.957290in}}{\pgfqpoint{1.295347in}{1.957290in}}%
\pgfpathcurveto{\pgfqpoint{1.287111in}{1.957290in}}{\pgfqpoint{1.279211in}{1.954017in}}{\pgfqpoint{1.273387in}{1.948193in}}%
\pgfpathcurveto{\pgfqpoint{1.267563in}{1.942370in}}{\pgfqpoint{1.264290in}{1.934470in}}{\pgfqpoint{1.264290in}{1.926233in}}%
\pgfpathcurveto{\pgfqpoint{1.264290in}{1.917997in}}{\pgfqpoint{1.267563in}{1.910097in}}{\pgfqpoint{1.273387in}{1.904273in}}%
\pgfpathcurveto{\pgfqpoint{1.279211in}{1.898449in}}{\pgfqpoint{1.287111in}{1.895177in}}{\pgfqpoint{1.295347in}{1.895177in}}%
\pgfpathlineto{\pgfqpoint{1.295347in}{1.895177in}}%
\pgfpathclose%
\pgfusepath{stroke}%
\end{pgfscope}%
\begin{pgfscope}%
\pgfpathrectangle{\pgfqpoint{0.688192in}{0.670138in}}{\pgfqpoint{6.200000in}{4.620000in}}%
\pgfusepath{clip}%
\pgfsetbuttcap%
\pgfsetroundjoin%
\pgfsetlinewidth{1.003750pt}%
\definecolor{currentstroke}{rgb}{1.000000,0.000000,0.000000}%
\pgfsetstrokecolor{currentstroke}%
\pgfsetdash{}{0pt}%
\pgfpathmoveto{\pgfqpoint{1.160535in}{1.595409in}}%
\pgfpathcurveto{\pgfqpoint{1.168772in}{1.595409in}}{\pgfqpoint{1.176672in}{1.598682in}}{\pgfqpoint{1.182496in}{1.604506in}}%
\pgfpathcurveto{\pgfqpoint{1.188320in}{1.610330in}}{\pgfqpoint{1.191592in}{1.618230in}}{\pgfqpoint{1.191592in}{1.626466in}}%
\pgfpathcurveto{\pgfqpoint{1.191592in}{1.634702in}}{\pgfqpoint{1.188320in}{1.642602in}}{\pgfqpoint{1.182496in}{1.648426in}}%
\pgfpathcurveto{\pgfqpoint{1.176672in}{1.654250in}}{\pgfqpoint{1.168772in}{1.657522in}}{\pgfqpoint{1.160535in}{1.657522in}}%
\pgfpathcurveto{\pgfqpoint{1.152299in}{1.657522in}}{\pgfqpoint{1.144399in}{1.654250in}}{\pgfqpoint{1.138575in}{1.648426in}}%
\pgfpathcurveto{\pgfqpoint{1.132751in}{1.642602in}}{\pgfqpoint{1.129479in}{1.634702in}}{\pgfqpoint{1.129479in}{1.626466in}}%
\pgfpathcurveto{\pgfqpoint{1.129479in}{1.618230in}}{\pgfqpoint{1.132751in}{1.610330in}}{\pgfqpoint{1.138575in}{1.604506in}}%
\pgfpathcurveto{\pgfqpoint{1.144399in}{1.598682in}}{\pgfqpoint{1.152299in}{1.595409in}}{\pgfqpoint{1.160535in}{1.595409in}}%
\pgfpathlineto{\pgfqpoint{1.160535in}{1.595409in}}%
\pgfpathclose%
\pgfusepath{stroke}%
\end{pgfscope}%
\begin{pgfscope}%
\pgfpathrectangle{\pgfqpoint{0.688192in}{0.670138in}}{\pgfqpoint{6.200000in}{4.620000in}}%
\pgfusepath{clip}%
\pgfsetbuttcap%
\pgfsetroundjoin%
\pgfsetlinewidth{1.003750pt}%
\definecolor{currentstroke}{rgb}{1.000000,0.000000,0.000000}%
\pgfsetstrokecolor{currentstroke}%
\pgfsetdash{}{0pt}%
\pgfpathmoveto{\pgfqpoint{1.265571in}{1.691299in}}%
\pgfpathcurveto{\pgfqpoint{1.273807in}{1.691299in}}{\pgfqpoint{1.281707in}{1.694572in}}{\pgfqpoint{1.287531in}{1.700396in}}%
\pgfpathcurveto{\pgfqpoint{1.293355in}{1.706220in}}{\pgfqpoint{1.296627in}{1.714120in}}{\pgfqpoint{1.296627in}{1.722356in}}%
\pgfpathcurveto{\pgfqpoint{1.296627in}{1.730592in}}{\pgfqpoint{1.293355in}{1.738492in}}{\pgfqpoint{1.287531in}{1.744316in}}%
\pgfpathcurveto{\pgfqpoint{1.281707in}{1.750140in}}{\pgfqpoint{1.273807in}{1.753412in}}{\pgfqpoint{1.265571in}{1.753412in}}%
\pgfpathcurveto{\pgfqpoint{1.257335in}{1.753412in}}{\pgfqpoint{1.249435in}{1.750140in}}{\pgfqpoint{1.243611in}{1.744316in}}%
\pgfpathcurveto{\pgfqpoint{1.237787in}{1.738492in}}{\pgfqpoint{1.234514in}{1.730592in}}{\pgfqpoint{1.234514in}{1.722356in}}%
\pgfpathcurveto{\pgfqpoint{1.234514in}{1.714120in}}{\pgfqpoint{1.237787in}{1.706220in}}{\pgfqpoint{1.243611in}{1.700396in}}%
\pgfpathcurveto{\pgfqpoint{1.249435in}{1.694572in}}{\pgfqpoint{1.257335in}{1.691299in}}{\pgfqpoint{1.265571in}{1.691299in}}%
\pgfpathlineto{\pgfqpoint{1.265571in}{1.691299in}}%
\pgfpathclose%
\pgfusepath{stroke}%
\end{pgfscope}%
\begin{pgfscope}%
\pgfpathrectangle{\pgfqpoint{0.688192in}{0.670138in}}{\pgfqpoint{6.200000in}{4.620000in}}%
\pgfusepath{clip}%
\pgfsetbuttcap%
\pgfsetroundjoin%
\pgfsetlinewidth{1.003750pt}%
\definecolor{currentstroke}{rgb}{1.000000,0.000000,0.000000}%
\pgfsetstrokecolor{currentstroke}%
\pgfsetdash{}{0pt}%
\pgfpathmoveto{\pgfqpoint{1.297728in}{1.636949in}}%
\pgfpathcurveto{\pgfqpoint{1.305964in}{1.636949in}}{\pgfqpoint{1.313864in}{1.640222in}}{\pgfqpoint{1.319688in}{1.646045in}}%
\pgfpathcurveto{\pgfqpoint{1.325512in}{1.651869in}}{\pgfqpoint{1.328784in}{1.659769in}}{\pgfqpoint{1.328784in}{1.668006in}}%
\pgfpathcurveto{\pgfqpoint{1.328784in}{1.676242in}}{\pgfqpoint{1.325512in}{1.684142in}}{\pgfqpoint{1.319688in}{1.689966in}}%
\pgfpathcurveto{\pgfqpoint{1.313864in}{1.695790in}}{\pgfqpoint{1.305964in}{1.699062in}}{\pgfqpoint{1.297728in}{1.699062in}}%
\pgfpathcurveto{\pgfqpoint{1.289491in}{1.699062in}}{\pgfqpoint{1.281591in}{1.695790in}}{\pgfqpoint{1.275767in}{1.689966in}}%
\pgfpathcurveto{\pgfqpoint{1.269943in}{1.684142in}}{\pgfqpoint{1.266671in}{1.676242in}}{\pgfqpoint{1.266671in}{1.668006in}}%
\pgfpathcurveto{\pgfqpoint{1.266671in}{1.659769in}}{\pgfqpoint{1.269943in}{1.651869in}}{\pgfqpoint{1.275767in}{1.646045in}}%
\pgfpathcurveto{\pgfqpoint{1.281591in}{1.640222in}}{\pgfqpoint{1.289491in}{1.636949in}}{\pgfqpoint{1.297728in}{1.636949in}}%
\pgfpathlineto{\pgfqpoint{1.297728in}{1.636949in}}%
\pgfpathclose%
\pgfusepath{stroke}%
\end{pgfscope}%
\begin{pgfscope}%
\pgfpathrectangle{\pgfqpoint{0.688192in}{0.670138in}}{\pgfqpoint{6.200000in}{4.620000in}}%
\pgfusepath{clip}%
\pgfsetbuttcap%
\pgfsetroundjoin%
\pgfsetlinewidth{1.003750pt}%
\definecolor{currentstroke}{rgb}{1.000000,0.000000,0.000000}%
\pgfsetstrokecolor{currentstroke}%
\pgfsetdash{}{0pt}%
\pgfpathmoveto{\pgfqpoint{1.147178in}{1.734556in}}%
\pgfpathcurveto{\pgfqpoint{1.155415in}{1.734556in}}{\pgfqpoint{1.163315in}{1.737828in}}{\pgfqpoint{1.169139in}{1.743652in}}%
\pgfpathcurveto{\pgfqpoint{1.174962in}{1.749476in}}{\pgfqpoint{1.178235in}{1.757376in}}{\pgfqpoint{1.178235in}{1.765612in}}%
\pgfpathcurveto{\pgfqpoint{1.178235in}{1.773848in}}{\pgfqpoint{1.174962in}{1.781749in}}{\pgfqpoint{1.169139in}{1.787572in}}%
\pgfpathcurveto{\pgfqpoint{1.163315in}{1.793396in}}{\pgfqpoint{1.155415in}{1.796669in}}{\pgfqpoint{1.147178in}{1.796669in}}%
\pgfpathcurveto{\pgfqpoint{1.138942in}{1.796669in}}{\pgfqpoint{1.131042in}{1.793396in}}{\pgfqpoint{1.125218in}{1.787572in}}%
\pgfpathcurveto{\pgfqpoint{1.119394in}{1.781749in}}{\pgfqpoint{1.116122in}{1.773848in}}{\pgfqpoint{1.116122in}{1.765612in}}%
\pgfpathcurveto{\pgfqpoint{1.116122in}{1.757376in}}{\pgfqpoint{1.119394in}{1.749476in}}{\pgfqpoint{1.125218in}{1.743652in}}%
\pgfpathcurveto{\pgfqpoint{1.131042in}{1.737828in}}{\pgfqpoint{1.138942in}{1.734556in}}{\pgfqpoint{1.147178in}{1.734556in}}%
\pgfpathlineto{\pgfqpoint{1.147178in}{1.734556in}}%
\pgfpathclose%
\pgfusepath{stroke}%
\end{pgfscope}%
\begin{pgfscope}%
\pgfpathrectangle{\pgfqpoint{0.688192in}{0.670138in}}{\pgfqpoint{6.200000in}{4.620000in}}%
\pgfusepath{clip}%
\pgfsetbuttcap%
\pgfsetroundjoin%
\pgfsetlinewidth{1.003750pt}%
\definecolor{currentstroke}{rgb}{1.000000,0.000000,0.000000}%
\pgfsetstrokecolor{currentstroke}%
\pgfsetdash{}{0pt}%
\pgfpathmoveto{\pgfqpoint{1.240097in}{1.549585in}}%
\pgfpathcurveto{\pgfqpoint{1.248334in}{1.549585in}}{\pgfqpoint{1.256234in}{1.552857in}}{\pgfqpoint{1.262058in}{1.558681in}}%
\pgfpathcurveto{\pgfqpoint{1.267881in}{1.564505in}}{\pgfqpoint{1.271154in}{1.572405in}}{\pgfqpoint{1.271154in}{1.580642in}}%
\pgfpathcurveto{\pgfqpoint{1.271154in}{1.588878in}}{\pgfqpoint{1.267881in}{1.596778in}}{\pgfqpoint{1.262058in}{1.602602in}}%
\pgfpathcurveto{\pgfqpoint{1.256234in}{1.608426in}}{\pgfqpoint{1.248334in}{1.611698in}}{\pgfqpoint{1.240097in}{1.611698in}}%
\pgfpathcurveto{\pgfqpoint{1.231861in}{1.611698in}}{\pgfqpoint{1.223961in}{1.608426in}}{\pgfqpoint{1.218137in}{1.602602in}}%
\pgfpathcurveto{\pgfqpoint{1.212313in}{1.596778in}}{\pgfqpoint{1.209041in}{1.588878in}}{\pgfqpoint{1.209041in}{1.580642in}}%
\pgfpathcurveto{\pgfqpoint{1.209041in}{1.572405in}}{\pgfqpoint{1.212313in}{1.564505in}}{\pgfqpoint{1.218137in}{1.558681in}}%
\pgfpathcurveto{\pgfqpoint{1.223961in}{1.552857in}}{\pgfqpoint{1.231861in}{1.549585in}}{\pgfqpoint{1.240097in}{1.549585in}}%
\pgfpathlineto{\pgfqpoint{1.240097in}{1.549585in}}%
\pgfpathclose%
\pgfusepath{stroke}%
\end{pgfscope}%
\begin{pgfscope}%
\pgfpathrectangle{\pgfqpoint{0.688192in}{0.670138in}}{\pgfqpoint{6.200000in}{4.620000in}}%
\pgfusepath{clip}%
\pgfsetbuttcap%
\pgfsetroundjoin%
\pgfsetlinewidth{1.003750pt}%
\definecolor{currentstroke}{rgb}{1.000000,0.000000,0.000000}%
\pgfsetstrokecolor{currentstroke}%
\pgfsetdash{}{0pt}%
\pgfpathmoveto{\pgfqpoint{1.275017in}{1.474367in}}%
\pgfpathcurveto{\pgfqpoint{1.283253in}{1.474367in}}{\pgfqpoint{1.291153in}{1.477639in}}{\pgfqpoint{1.296977in}{1.483463in}}%
\pgfpathcurveto{\pgfqpoint{1.302801in}{1.489287in}}{\pgfqpoint{1.306073in}{1.497187in}}{\pgfqpoint{1.306073in}{1.505423in}}%
\pgfpathcurveto{\pgfqpoint{1.306073in}{1.513660in}}{\pgfqpoint{1.302801in}{1.521560in}}{\pgfqpoint{1.296977in}{1.527384in}}%
\pgfpathcurveto{\pgfqpoint{1.291153in}{1.533208in}}{\pgfqpoint{1.283253in}{1.536480in}}{\pgfqpoint{1.275017in}{1.536480in}}%
\pgfpathcurveto{\pgfqpoint{1.266780in}{1.536480in}}{\pgfqpoint{1.258880in}{1.533208in}}{\pgfqpoint{1.253056in}{1.527384in}}%
\pgfpathcurveto{\pgfqpoint{1.247232in}{1.521560in}}{\pgfqpoint{1.243960in}{1.513660in}}{\pgfqpoint{1.243960in}{1.505423in}}%
\pgfpathcurveto{\pgfqpoint{1.243960in}{1.497187in}}{\pgfqpoint{1.247232in}{1.489287in}}{\pgfqpoint{1.253056in}{1.483463in}}%
\pgfpathcurveto{\pgfqpoint{1.258880in}{1.477639in}}{\pgfqpoint{1.266780in}{1.474367in}}{\pgfqpoint{1.275017in}{1.474367in}}%
\pgfpathlineto{\pgfqpoint{1.275017in}{1.474367in}}%
\pgfpathclose%
\pgfusepath{stroke}%
\end{pgfscope}%
\begin{pgfscope}%
\pgfpathrectangle{\pgfqpoint{0.688192in}{0.670138in}}{\pgfqpoint{6.200000in}{4.620000in}}%
\pgfusepath{clip}%
\pgfsetbuttcap%
\pgfsetroundjoin%
\pgfsetlinewidth{1.003750pt}%
\definecolor{currentstroke}{rgb}{1.000000,0.000000,0.000000}%
\pgfsetstrokecolor{currentstroke}%
\pgfsetdash{}{0pt}%
\pgfpathmoveto{\pgfqpoint{1.274070in}{1.506532in}}%
\pgfpathcurveto{\pgfqpoint{1.282306in}{1.506532in}}{\pgfqpoint{1.290206in}{1.509804in}}{\pgfqpoint{1.296030in}{1.515628in}}%
\pgfpathcurveto{\pgfqpoint{1.301854in}{1.521452in}}{\pgfqpoint{1.305126in}{1.529352in}}{\pgfqpoint{1.305126in}{1.537589in}}%
\pgfpathcurveto{\pgfqpoint{1.305126in}{1.545825in}}{\pgfqpoint{1.301854in}{1.553725in}}{\pgfqpoint{1.296030in}{1.559549in}}%
\pgfpathcurveto{\pgfqpoint{1.290206in}{1.565373in}}{\pgfqpoint{1.282306in}{1.568645in}}{\pgfqpoint{1.274070in}{1.568645in}}%
\pgfpathcurveto{\pgfqpoint{1.265833in}{1.568645in}}{\pgfqpoint{1.257933in}{1.565373in}}{\pgfqpoint{1.252109in}{1.559549in}}%
\pgfpathcurveto{\pgfqpoint{1.246285in}{1.553725in}}{\pgfqpoint{1.243013in}{1.545825in}}{\pgfqpoint{1.243013in}{1.537589in}}%
\pgfpathcurveto{\pgfqpoint{1.243013in}{1.529352in}}{\pgfqpoint{1.246285in}{1.521452in}}{\pgfqpoint{1.252109in}{1.515628in}}%
\pgfpathcurveto{\pgfqpoint{1.257933in}{1.509804in}}{\pgfqpoint{1.265833in}{1.506532in}}{\pgfqpoint{1.274070in}{1.506532in}}%
\pgfpathlineto{\pgfqpoint{1.274070in}{1.506532in}}%
\pgfpathclose%
\pgfusepath{stroke}%
\end{pgfscope}%
\begin{pgfscope}%
\pgfpathrectangle{\pgfqpoint{0.688192in}{0.670138in}}{\pgfqpoint{6.200000in}{4.620000in}}%
\pgfusepath{clip}%
\pgfsetbuttcap%
\pgfsetroundjoin%
\pgfsetlinewidth{1.003750pt}%
\definecolor{currentstroke}{rgb}{1.000000,0.000000,0.000000}%
\pgfsetstrokecolor{currentstroke}%
\pgfsetdash{}{0pt}%
\pgfpathmoveto{\pgfqpoint{1.147538in}{1.722969in}}%
\pgfpathcurveto{\pgfqpoint{1.155775in}{1.722969in}}{\pgfqpoint{1.163675in}{1.726242in}}{\pgfqpoint{1.169499in}{1.732066in}}%
\pgfpathcurveto{\pgfqpoint{1.175323in}{1.737890in}}{\pgfqpoint{1.178595in}{1.745790in}}{\pgfqpoint{1.178595in}{1.754026in}}%
\pgfpathcurveto{\pgfqpoint{1.178595in}{1.762262in}}{\pgfqpoint{1.175323in}{1.770162in}}{\pgfqpoint{1.169499in}{1.775986in}}%
\pgfpathcurveto{\pgfqpoint{1.163675in}{1.781810in}}{\pgfqpoint{1.155775in}{1.785082in}}{\pgfqpoint{1.147538in}{1.785082in}}%
\pgfpathcurveto{\pgfqpoint{1.139302in}{1.785082in}}{\pgfqpoint{1.131402in}{1.781810in}}{\pgfqpoint{1.125578in}{1.775986in}}%
\pgfpathcurveto{\pgfqpoint{1.119754in}{1.770162in}}{\pgfqpoint{1.116482in}{1.762262in}}{\pgfqpoint{1.116482in}{1.754026in}}%
\pgfpathcurveto{\pgfqpoint{1.116482in}{1.745790in}}{\pgfqpoint{1.119754in}{1.737890in}}{\pgfqpoint{1.125578in}{1.732066in}}%
\pgfpathcurveto{\pgfqpoint{1.131402in}{1.726242in}}{\pgfqpoint{1.139302in}{1.722969in}}{\pgfqpoint{1.147538in}{1.722969in}}%
\pgfpathlineto{\pgfqpoint{1.147538in}{1.722969in}}%
\pgfpathclose%
\pgfusepath{stroke}%
\end{pgfscope}%
\begin{pgfscope}%
\pgfpathrectangle{\pgfqpoint{0.688192in}{0.670138in}}{\pgfqpoint{6.200000in}{4.620000in}}%
\pgfusepath{clip}%
\pgfsetbuttcap%
\pgfsetroundjoin%
\pgfsetlinewidth{1.003750pt}%
\definecolor{currentstroke}{rgb}{1.000000,0.000000,0.000000}%
\pgfsetstrokecolor{currentstroke}%
\pgfsetdash{}{0pt}%
\pgfpathmoveto{\pgfqpoint{1.194960in}{1.895058in}}%
\pgfpathcurveto{\pgfqpoint{1.203197in}{1.895058in}}{\pgfqpoint{1.211097in}{1.898330in}}{\pgfqpoint{1.216921in}{1.904154in}}%
\pgfpathcurveto{\pgfqpoint{1.222745in}{1.909978in}}{\pgfqpoint{1.226017in}{1.917878in}}{\pgfqpoint{1.226017in}{1.926114in}}%
\pgfpathcurveto{\pgfqpoint{1.226017in}{1.934351in}}{\pgfqpoint{1.222745in}{1.942251in}}{\pgfqpoint{1.216921in}{1.948075in}}%
\pgfpathcurveto{\pgfqpoint{1.211097in}{1.953898in}}{\pgfqpoint{1.203197in}{1.957171in}}{\pgfqpoint{1.194960in}{1.957171in}}%
\pgfpathcurveto{\pgfqpoint{1.186724in}{1.957171in}}{\pgfqpoint{1.178824in}{1.953898in}}{\pgfqpoint{1.173000in}{1.948075in}}%
\pgfpathcurveto{\pgfqpoint{1.167176in}{1.942251in}}{\pgfqpoint{1.163904in}{1.934351in}}{\pgfqpoint{1.163904in}{1.926114in}}%
\pgfpathcurveto{\pgfqpoint{1.163904in}{1.917878in}}{\pgfqpoint{1.167176in}{1.909978in}}{\pgfqpoint{1.173000in}{1.904154in}}%
\pgfpathcurveto{\pgfqpoint{1.178824in}{1.898330in}}{\pgfqpoint{1.186724in}{1.895058in}}{\pgfqpoint{1.194960in}{1.895058in}}%
\pgfpathlineto{\pgfqpoint{1.194960in}{1.895058in}}%
\pgfpathclose%
\pgfusepath{stroke}%
\end{pgfscope}%
\begin{pgfscope}%
\pgfpathrectangle{\pgfqpoint{0.688192in}{0.670138in}}{\pgfqpoint{6.200000in}{4.620000in}}%
\pgfusepath{clip}%
\pgfsetbuttcap%
\pgfsetroundjoin%
\pgfsetlinewidth{1.003750pt}%
\definecolor{currentstroke}{rgb}{1.000000,0.000000,0.000000}%
\pgfsetstrokecolor{currentstroke}%
\pgfsetdash{}{0pt}%
\pgfpathmoveto{\pgfqpoint{1.164778in}{1.905975in}}%
\pgfpathcurveto{\pgfqpoint{1.173015in}{1.905975in}}{\pgfqpoint{1.180915in}{1.909248in}}{\pgfqpoint{1.186738in}{1.915072in}}%
\pgfpathcurveto{\pgfqpoint{1.192562in}{1.920896in}}{\pgfqpoint{1.195835in}{1.928796in}}{\pgfqpoint{1.195835in}{1.937032in}}%
\pgfpathcurveto{\pgfqpoint{1.195835in}{1.945268in}}{\pgfqpoint{1.192562in}{1.953168in}}{\pgfqpoint{1.186738in}{1.958992in}}%
\pgfpathcurveto{\pgfqpoint{1.180915in}{1.964816in}}{\pgfqpoint{1.173015in}{1.968088in}}{\pgfqpoint{1.164778in}{1.968088in}}%
\pgfpathcurveto{\pgfqpoint{1.156542in}{1.968088in}}{\pgfqpoint{1.148642in}{1.964816in}}{\pgfqpoint{1.142818in}{1.958992in}}%
\pgfpathcurveto{\pgfqpoint{1.136994in}{1.953168in}}{\pgfqpoint{1.133722in}{1.945268in}}{\pgfqpoint{1.133722in}{1.937032in}}%
\pgfpathcurveto{\pgfqpoint{1.133722in}{1.928796in}}{\pgfqpoint{1.136994in}{1.920896in}}{\pgfqpoint{1.142818in}{1.915072in}}%
\pgfpathcurveto{\pgfqpoint{1.148642in}{1.909248in}}{\pgfqpoint{1.156542in}{1.905975in}}{\pgfqpoint{1.164778in}{1.905975in}}%
\pgfpathlineto{\pgfqpoint{1.164778in}{1.905975in}}%
\pgfpathclose%
\pgfusepath{stroke}%
\end{pgfscope}%
\begin{pgfscope}%
\pgfpathrectangle{\pgfqpoint{0.688192in}{0.670138in}}{\pgfqpoint{6.200000in}{4.620000in}}%
\pgfusepath{clip}%
\pgfsetbuttcap%
\pgfsetroundjoin%
\pgfsetlinewidth{1.003750pt}%
\definecolor{currentstroke}{rgb}{1.000000,0.000000,0.000000}%
\pgfsetstrokecolor{currentstroke}%
\pgfsetdash{}{0pt}%
\pgfpathmoveto{\pgfqpoint{1.143899in}{1.731794in}}%
\pgfpathcurveto{\pgfqpoint{1.152135in}{1.731794in}}{\pgfqpoint{1.160035in}{1.735067in}}{\pgfqpoint{1.165859in}{1.740890in}}%
\pgfpathcurveto{\pgfqpoint{1.171683in}{1.746714in}}{\pgfqpoint{1.174956in}{1.754614in}}{\pgfqpoint{1.174956in}{1.762851in}}%
\pgfpathcurveto{\pgfqpoint{1.174956in}{1.771087in}}{\pgfqpoint{1.171683in}{1.778987in}}{\pgfqpoint{1.165859in}{1.784811in}}%
\pgfpathcurveto{\pgfqpoint{1.160035in}{1.790635in}}{\pgfqpoint{1.152135in}{1.793907in}}{\pgfqpoint{1.143899in}{1.793907in}}%
\pgfpathcurveto{\pgfqpoint{1.135663in}{1.793907in}}{\pgfqpoint{1.127763in}{1.790635in}}{\pgfqpoint{1.121939in}{1.784811in}}%
\pgfpathcurveto{\pgfqpoint{1.116115in}{1.778987in}}{\pgfqpoint{1.112843in}{1.771087in}}{\pgfqpoint{1.112843in}{1.762851in}}%
\pgfpathcurveto{\pgfqpoint{1.112843in}{1.754614in}}{\pgfqpoint{1.116115in}{1.746714in}}{\pgfqpoint{1.121939in}{1.740890in}}%
\pgfpathcurveto{\pgfqpoint{1.127763in}{1.735067in}}{\pgfqpoint{1.135663in}{1.731794in}}{\pgfqpoint{1.143899in}{1.731794in}}%
\pgfpathlineto{\pgfqpoint{1.143899in}{1.731794in}}%
\pgfpathclose%
\pgfusepath{stroke}%
\end{pgfscope}%
\begin{pgfscope}%
\pgfpathrectangle{\pgfqpoint{0.688192in}{0.670138in}}{\pgfqpoint{6.200000in}{4.620000in}}%
\pgfusepath{clip}%
\pgfsetbuttcap%
\pgfsetroundjoin%
\pgfsetlinewidth{1.003750pt}%
\definecolor{currentstroke}{rgb}{1.000000,0.000000,0.000000}%
\pgfsetstrokecolor{currentstroke}%
\pgfsetdash{}{0pt}%
\pgfpathmoveto{\pgfqpoint{0.688192in}{1.225706in}}%
\pgfpathcurveto{\pgfqpoint{0.696428in}{1.225706in}}{\pgfqpoint{0.704328in}{1.228978in}}{\pgfqpoint{0.710152in}{1.234802in}}%
\pgfpathcurveto{\pgfqpoint{0.715976in}{1.240626in}}{\pgfqpoint{0.719248in}{1.248526in}}{\pgfqpoint{0.719248in}{1.256762in}}%
\pgfpathcurveto{\pgfqpoint{0.719248in}{1.264998in}}{\pgfqpoint{0.715976in}{1.272898in}}{\pgfqpoint{0.710152in}{1.278722in}}%
\pgfpathcurveto{\pgfqpoint{0.704328in}{1.284546in}}{\pgfqpoint{0.696428in}{1.287819in}}{\pgfqpoint{0.688192in}{1.287819in}}%
\pgfpathcurveto{\pgfqpoint{0.679955in}{1.287819in}}{\pgfqpoint{0.672055in}{1.284546in}}{\pgfqpoint{0.666231in}{1.278722in}}%
\pgfpathcurveto{\pgfqpoint{0.660407in}{1.272898in}}{\pgfqpoint{0.657135in}{1.264998in}}{\pgfqpoint{0.657135in}{1.256762in}}%
\pgfpathcurveto{\pgfqpoint{0.657135in}{1.248526in}}{\pgfqpoint{0.660407in}{1.240626in}}{\pgfqpoint{0.666231in}{1.234802in}}%
\pgfpathcurveto{\pgfqpoint{0.672055in}{1.228978in}}{\pgfqpoint{0.679955in}{1.225706in}}{\pgfqpoint{0.688192in}{1.225706in}}%
\pgfpathlineto{\pgfqpoint{0.688192in}{1.225706in}}%
\pgfpathclose%
\pgfusepath{stroke}%
\end{pgfscope}%
\begin{pgfscope}%
\pgfpathrectangle{\pgfqpoint{0.688192in}{0.670138in}}{\pgfqpoint{6.200000in}{4.620000in}}%
\pgfusepath{clip}%
\pgfsetbuttcap%
\pgfsetmiterjoin%
\definecolor{currentfill}{rgb}{0.839216,0.152941,0.156863}%
\pgfsetfillcolor{currentfill}%
\pgfsetfillopacity{0.200000}%
\pgfsetlinewidth{1.003750pt}%
\definecolor{currentstroke}{rgb}{0.839216,0.152941,0.156863}%
\pgfsetstrokecolor{currentstroke}%
\pgfsetstrokeopacity{0.200000}%
\pgfsetdash{}{0pt}%
\pgfpathmoveto{\pgfqpoint{0.688192in}{0.670138in}}%
\pgfpathlineto{\pgfqpoint{1.790122in}{0.670138in}}%
\pgfpathlineto{\pgfqpoint{1.790122in}{5.290138in}}%
\pgfpathlineto{\pgfqpoint{0.688192in}{5.290138in}}%
\pgfpathlineto{\pgfqpoint{0.688192in}{0.670138in}}%
\pgfpathclose%
\pgfusepath{stroke,fill}%
\end{pgfscope}%
\begin{pgfscope}%
\pgfsetbuttcap%
\pgfsetmiterjoin%
\definecolor{currentfill}{rgb}{0.839216,0.152941,0.156863}%
\pgfsetfillcolor{currentfill}%
\pgfsetfillopacity{0.200000}%
\pgfsetlinewidth{1.003750pt}%
\definecolor{currentstroke}{rgb}{0.839216,0.152941,0.156863}%
\pgfsetstrokecolor{currentstroke}%
\pgfsetstrokeopacity{0.200000}%
\pgfsetdash{}{0pt}%
\pgfpathrectangle{\pgfqpoint{0.688192in}{0.670138in}}{\pgfqpoint{6.200000in}{4.620000in}}%
\pgfusepath{clip}%
\pgfpathmoveto{\pgfqpoint{0.688192in}{0.670138in}}%
\pgfpathlineto{\pgfqpoint{1.790122in}{0.670138in}}%
\pgfpathlineto{\pgfqpoint{1.790122in}{5.290138in}}%
\pgfpathlineto{\pgfqpoint{0.688192in}{5.290138in}}%
\pgfpathlineto{\pgfqpoint{0.688192in}{0.670138in}}%
\pgfpathclose%
\pgfusepath{clip}%
\pgfsys@defobject{currentpattern}{\pgfqpoint{0in}{0in}}{\pgfqpoint{1in}{1in}}{%
\begin{pgfscope}%
\pgfpathrectangle{\pgfqpoint{0in}{0in}}{\pgfqpoint{1in}{1in}}%
\pgfusepath{clip}%
\pgfpathmoveto{\pgfqpoint{-0.500000in}{0.500000in}}%
\pgfpathlineto{\pgfqpoint{0.500000in}{1.500000in}}%
\pgfpathmoveto{\pgfqpoint{-0.333333in}{0.333333in}}%
\pgfpathlineto{\pgfqpoint{0.666667in}{1.333333in}}%
\pgfpathmoveto{\pgfqpoint{-0.166667in}{0.166667in}}%
\pgfpathlineto{\pgfqpoint{0.833333in}{1.166667in}}%
\pgfpathmoveto{\pgfqpoint{0.000000in}{0.000000in}}%
\pgfpathlineto{\pgfqpoint{1.000000in}{1.000000in}}%
\pgfpathmoveto{\pgfqpoint{0.166667in}{-0.166667in}}%
\pgfpathlineto{\pgfqpoint{1.166667in}{0.833333in}}%
\pgfpathmoveto{\pgfqpoint{0.333333in}{-0.333333in}}%
\pgfpathlineto{\pgfqpoint{1.333333in}{0.666667in}}%
\pgfpathmoveto{\pgfqpoint{0.500000in}{-0.500000in}}%
\pgfpathlineto{\pgfqpoint{1.500000in}{0.500000in}}%
\pgfusepath{stroke}%
\end{pgfscope}%
}%
\pgfsys@transformshift{0.688192in}{0.670138in}%
\pgfsys@useobject{currentpattern}{}%
\pgfsys@transformshift{1in}{0in}%
\pgfsys@useobject{currentpattern}{}%
\pgfsys@transformshift{1in}{0in}%
\pgfsys@transformshift{-2in}{0in}%
\pgfsys@transformshift{0in}{1in}%
\pgfsys@useobject{currentpattern}{}%
\pgfsys@transformshift{1in}{0in}%
\pgfsys@useobject{currentpattern}{}%
\pgfsys@transformshift{1in}{0in}%
\pgfsys@transformshift{-2in}{0in}%
\pgfsys@transformshift{0in}{1in}%
\pgfsys@useobject{currentpattern}{}%
\pgfsys@transformshift{1in}{0in}%
\pgfsys@useobject{currentpattern}{}%
\pgfsys@transformshift{1in}{0in}%
\pgfsys@transformshift{-2in}{0in}%
\pgfsys@transformshift{0in}{1in}%
\pgfsys@useobject{currentpattern}{}%
\pgfsys@transformshift{1in}{0in}%
\pgfsys@useobject{currentpattern}{}%
\pgfsys@transformshift{1in}{0in}%
\pgfsys@transformshift{-2in}{0in}%
\pgfsys@transformshift{0in}{1in}%
\pgfsys@useobject{currentpattern}{}%
\pgfsys@transformshift{1in}{0in}%
\pgfsys@useobject{currentpattern}{}%
\pgfsys@transformshift{1in}{0in}%
\pgfsys@transformshift{-2in}{0in}%
\pgfsys@transformshift{0in}{1in}%
\end{pgfscope}%
\begin{pgfscope}%
\pgfpathrectangle{\pgfqpoint{0.688192in}{0.670138in}}{\pgfqpoint{6.200000in}{4.620000in}}%
\pgfusepath{clip}%
\pgfsetrectcap%
\pgfsetroundjoin%
\pgfsetlinewidth{0.803000pt}%
\definecolor{currentstroke}{rgb}{0.690196,0.690196,0.690196}%
\pgfsetstrokecolor{currentstroke}%
\pgfsetdash{}{0pt}%
\pgfpathmoveto{\pgfqpoint{1.122474in}{0.670138in}}%
\pgfpathlineto{\pgfqpoint{1.122474in}{5.290138in}}%
\pgfusepath{stroke}%
\end{pgfscope}%
\begin{pgfscope}%
\pgfsetbuttcap%
\pgfsetroundjoin%
\definecolor{currentfill}{rgb}{0.000000,0.000000,0.000000}%
\pgfsetfillcolor{currentfill}%
\pgfsetlinewidth{0.803000pt}%
\definecolor{currentstroke}{rgb}{0.000000,0.000000,0.000000}%
\pgfsetstrokecolor{currentstroke}%
\pgfsetdash{}{0pt}%
\pgfsys@defobject{currentmarker}{\pgfqpoint{0.000000in}{-0.048611in}}{\pgfqpoint{0.000000in}{0.000000in}}{%
\pgfpathmoveto{\pgfqpoint{0.000000in}{0.000000in}}%
\pgfpathlineto{\pgfqpoint{0.000000in}{-0.048611in}}%
\pgfusepath{stroke,fill}%
}%
\begin{pgfscope}%
\pgfsys@transformshift{1.122474in}{0.670138in}%
\pgfsys@useobject{currentmarker}{}%
\end{pgfscope}%
\end{pgfscope}%
\begin{pgfscope}%
\definecolor{textcolor}{rgb}{0.000000,0.000000,0.000000}%
\pgfsetstrokecolor{textcolor}%
\pgfsetfillcolor{textcolor}%
\pgftext[x=1.122474in,y=0.572916in,,top]{\color{textcolor}{\rmfamily\fontsize{14.000000}{16.800000}\selectfont\catcode`\^=\active\def^{\ifmmode\sp\else\^{}\fi}\catcode`\%=\active\def%{\%}$\mathdefault{5500}$}}%
\end{pgfscope}%
\begin{pgfscope}%
\pgfpathrectangle{\pgfqpoint{0.688192in}{0.670138in}}{\pgfqpoint{6.200000in}{4.620000in}}%
\pgfusepath{clip}%
\pgfsetrectcap%
\pgfsetroundjoin%
\pgfsetlinewidth{0.803000pt}%
\definecolor{currentstroke}{rgb}{0.690196,0.690196,0.690196}%
\pgfsetstrokecolor{currentstroke}%
\pgfsetdash{}{0pt}%
\pgfpathmoveto{\pgfqpoint{2.163709in}{0.670138in}}%
\pgfpathlineto{\pgfqpoint{2.163709in}{5.290138in}}%
\pgfusepath{stroke}%
\end{pgfscope}%
\begin{pgfscope}%
\pgfsetbuttcap%
\pgfsetroundjoin%
\definecolor{currentfill}{rgb}{0.000000,0.000000,0.000000}%
\pgfsetfillcolor{currentfill}%
\pgfsetlinewidth{0.803000pt}%
\definecolor{currentstroke}{rgb}{0.000000,0.000000,0.000000}%
\pgfsetstrokecolor{currentstroke}%
\pgfsetdash{}{0pt}%
\pgfsys@defobject{currentmarker}{\pgfqpoint{0.000000in}{-0.048611in}}{\pgfqpoint{0.000000in}{0.000000in}}{%
\pgfpathmoveto{\pgfqpoint{0.000000in}{0.000000in}}%
\pgfpathlineto{\pgfqpoint{0.000000in}{-0.048611in}}%
\pgfusepath{stroke,fill}%
}%
\begin{pgfscope}%
\pgfsys@transformshift{2.163709in}{0.670138in}%
\pgfsys@useobject{currentmarker}{}%
\end{pgfscope}%
\end{pgfscope}%
\begin{pgfscope}%
\definecolor{textcolor}{rgb}{0.000000,0.000000,0.000000}%
\pgfsetstrokecolor{textcolor}%
\pgfsetfillcolor{textcolor}%
\pgftext[x=2.163709in,y=0.572916in,,top]{\color{textcolor}{\rmfamily\fontsize{14.000000}{16.800000}\selectfont\catcode`\^=\active\def^{\ifmmode\sp\else\^{}\fi}\catcode`\%=\active\def%{\%}$\mathdefault{6000}$}}%
\end{pgfscope}%
\begin{pgfscope}%
\pgfpathrectangle{\pgfqpoint{0.688192in}{0.670138in}}{\pgfqpoint{6.200000in}{4.620000in}}%
\pgfusepath{clip}%
\pgfsetrectcap%
\pgfsetroundjoin%
\pgfsetlinewidth{0.803000pt}%
\definecolor{currentstroke}{rgb}{0.690196,0.690196,0.690196}%
\pgfsetstrokecolor{currentstroke}%
\pgfsetdash{}{0pt}%
\pgfpathmoveto{\pgfqpoint{3.204944in}{0.670138in}}%
\pgfpathlineto{\pgfqpoint{3.204944in}{5.290138in}}%
\pgfusepath{stroke}%
\end{pgfscope}%
\begin{pgfscope}%
\pgfsetbuttcap%
\pgfsetroundjoin%
\definecolor{currentfill}{rgb}{0.000000,0.000000,0.000000}%
\pgfsetfillcolor{currentfill}%
\pgfsetlinewidth{0.803000pt}%
\definecolor{currentstroke}{rgb}{0.000000,0.000000,0.000000}%
\pgfsetstrokecolor{currentstroke}%
\pgfsetdash{}{0pt}%
\pgfsys@defobject{currentmarker}{\pgfqpoint{0.000000in}{-0.048611in}}{\pgfqpoint{0.000000in}{0.000000in}}{%
\pgfpathmoveto{\pgfqpoint{0.000000in}{0.000000in}}%
\pgfpathlineto{\pgfqpoint{0.000000in}{-0.048611in}}%
\pgfusepath{stroke,fill}%
}%
\begin{pgfscope}%
\pgfsys@transformshift{3.204944in}{0.670138in}%
\pgfsys@useobject{currentmarker}{}%
\end{pgfscope}%
\end{pgfscope}%
\begin{pgfscope}%
\definecolor{textcolor}{rgb}{0.000000,0.000000,0.000000}%
\pgfsetstrokecolor{textcolor}%
\pgfsetfillcolor{textcolor}%
\pgftext[x=3.204944in,y=0.572916in,,top]{\color{textcolor}{\rmfamily\fontsize{14.000000}{16.800000}\selectfont\catcode`\^=\active\def^{\ifmmode\sp\else\^{}\fi}\catcode`\%=\active\def%{\%}$\mathdefault{6500}$}}%
\end{pgfscope}%
\begin{pgfscope}%
\pgfpathrectangle{\pgfqpoint{0.688192in}{0.670138in}}{\pgfqpoint{6.200000in}{4.620000in}}%
\pgfusepath{clip}%
\pgfsetrectcap%
\pgfsetroundjoin%
\pgfsetlinewidth{0.803000pt}%
\definecolor{currentstroke}{rgb}{0.690196,0.690196,0.690196}%
\pgfsetstrokecolor{currentstroke}%
\pgfsetdash{}{0pt}%
\pgfpathmoveto{\pgfqpoint{4.246179in}{0.670138in}}%
\pgfpathlineto{\pgfqpoint{4.246179in}{5.290138in}}%
\pgfusepath{stroke}%
\end{pgfscope}%
\begin{pgfscope}%
\pgfsetbuttcap%
\pgfsetroundjoin%
\definecolor{currentfill}{rgb}{0.000000,0.000000,0.000000}%
\pgfsetfillcolor{currentfill}%
\pgfsetlinewidth{0.803000pt}%
\definecolor{currentstroke}{rgb}{0.000000,0.000000,0.000000}%
\pgfsetstrokecolor{currentstroke}%
\pgfsetdash{}{0pt}%
\pgfsys@defobject{currentmarker}{\pgfqpoint{0.000000in}{-0.048611in}}{\pgfqpoint{0.000000in}{0.000000in}}{%
\pgfpathmoveto{\pgfqpoint{0.000000in}{0.000000in}}%
\pgfpathlineto{\pgfqpoint{0.000000in}{-0.048611in}}%
\pgfusepath{stroke,fill}%
}%
\begin{pgfscope}%
\pgfsys@transformshift{4.246179in}{0.670138in}%
\pgfsys@useobject{currentmarker}{}%
\end{pgfscope}%
\end{pgfscope}%
\begin{pgfscope}%
\definecolor{textcolor}{rgb}{0.000000,0.000000,0.000000}%
\pgfsetstrokecolor{textcolor}%
\pgfsetfillcolor{textcolor}%
\pgftext[x=4.246179in,y=0.572916in,,top]{\color{textcolor}{\rmfamily\fontsize{14.000000}{16.800000}\selectfont\catcode`\^=\active\def^{\ifmmode\sp\else\^{}\fi}\catcode`\%=\active\def%{\%}$\mathdefault{7000}$}}%
\end{pgfscope}%
\begin{pgfscope}%
\pgfpathrectangle{\pgfqpoint{0.688192in}{0.670138in}}{\pgfqpoint{6.200000in}{4.620000in}}%
\pgfusepath{clip}%
\pgfsetrectcap%
\pgfsetroundjoin%
\pgfsetlinewidth{0.803000pt}%
\definecolor{currentstroke}{rgb}{0.690196,0.690196,0.690196}%
\pgfsetstrokecolor{currentstroke}%
\pgfsetdash{}{0pt}%
\pgfpathmoveto{\pgfqpoint{5.287414in}{0.670138in}}%
\pgfpathlineto{\pgfqpoint{5.287414in}{5.290138in}}%
\pgfusepath{stroke}%
\end{pgfscope}%
\begin{pgfscope}%
\pgfsetbuttcap%
\pgfsetroundjoin%
\definecolor{currentfill}{rgb}{0.000000,0.000000,0.000000}%
\pgfsetfillcolor{currentfill}%
\pgfsetlinewidth{0.803000pt}%
\definecolor{currentstroke}{rgb}{0.000000,0.000000,0.000000}%
\pgfsetstrokecolor{currentstroke}%
\pgfsetdash{}{0pt}%
\pgfsys@defobject{currentmarker}{\pgfqpoint{0.000000in}{-0.048611in}}{\pgfqpoint{0.000000in}{0.000000in}}{%
\pgfpathmoveto{\pgfqpoint{0.000000in}{0.000000in}}%
\pgfpathlineto{\pgfqpoint{0.000000in}{-0.048611in}}%
\pgfusepath{stroke,fill}%
}%
\begin{pgfscope}%
\pgfsys@transformshift{5.287414in}{0.670138in}%
\pgfsys@useobject{currentmarker}{}%
\end{pgfscope}%
\end{pgfscope}%
\begin{pgfscope}%
\definecolor{textcolor}{rgb}{0.000000,0.000000,0.000000}%
\pgfsetstrokecolor{textcolor}%
\pgfsetfillcolor{textcolor}%
\pgftext[x=5.287414in,y=0.572916in,,top]{\color{textcolor}{\rmfamily\fontsize{14.000000}{16.800000}\selectfont\catcode`\^=\active\def^{\ifmmode\sp\else\^{}\fi}\catcode`\%=\active\def%{\%}$\mathdefault{7500}$}}%
\end{pgfscope}%
\begin{pgfscope}%
\pgfpathrectangle{\pgfqpoint{0.688192in}{0.670138in}}{\pgfqpoint{6.200000in}{4.620000in}}%
\pgfusepath{clip}%
\pgfsetrectcap%
\pgfsetroundjoin%
\pgfsetlinewidth{0.803000pt}%
\definecolor{currentstroke}{rgb}{0.690196,0.690196,0.690196}%
\pgfsetstrokecolor{currentstroke}%
\pgfsetdash{}{0pt}%
\pgfpathmoveto{\pgfqpoint{6.328649in}{0.670138in}}%
\pgfpathlineto{\pgfqpoint{6.328649in}{5.290138in}}%
\pgfusepath{stroke}%
\end{pgfscope}%
\begin{pgfscope}%
\pgfsetbuttcap%
\pgfsetroundjoin%
\definecolor{currentfill}{rgb}{0.000000,0.000000,0.000000}%
\pgfsetfillcolor{currentfill}%
\pgfsetlinewidth{0.803000pt}%
\definecolor{currentstroke}{rgb}{0.000000,0.000000,0.000000}%
\pgfsetstrokecolor{currentstroke}%
\pgfsetdash{}{0pt}%
\pgfsys@defobject{currentmarker}{\pgfqpoint{0.000000in}{-0.048611in}}{\pgfqpoint{0.000000in}{0.000000in}}{%
\pgfpathmoveto{\pgfqpoint{0.000000in}{0.000000in}}%
\pgfpathlineto{\pgfqpoint{0.000000in}{-0.048611in}}%
\pgfusepath{stroke,fill}%
}%
\begin{pgfscope}%
\pgfsys@transformshift{6.328649in}{0.670138in}%
\pgfsys@useobject{currentmarker}{}%
\end{pgfscope}%
\end{pgfscope}%
\begin{pgfscope}%
\definecolor{textcolor}{rgb}{0.000000,0.000000,0.000000}%
\pgfsetstrokecolor{textcolor}%
\pgfsetfillcolor{textcolor}%
\pgftext[x=6.328649in,y=0.572916in,,top]{\color{textcolor}{\rmfamily\fontsize{14.000000}{16.800000}\selectfont\catcode`\^=\active\def^{\ifmmode\sp\else\^{}\fi}\catcode`\%=\active\def%{\%}$\mathdefault{8000}$}}%
\end{pgfscope}%
\begin{pgfscope}%
\definecolor{textcolor}{rgb}{0.000000,0.000000,0.000000}%
\pgfsetstrokecolor{textcolor}%
\pgfsetfillcolor{textcolor}%
\pgftext[x=3.788192in,y=0.339583in,,top]{\color{textcolor}{\rmfamily\fontsize{18.000000}{21.600000}\selectfont\catcode`\^=\active\def^{\ifmmode\sp\else\^{}\fi}\catcode`\%=\active\def%{\%}Total Cost (M\$)}}%
\end{pgfscope}%
\begin{pgfscope}%
\pgfpathrectangle{\pgfqpoint{0.688192in}{0.670138in}}{\pgfqpoint{6.200000in}{4.620000in}}%
\pgfusepath{clip}%
\pgfsetrectcap%
\pgfsetroundjoin%
\pgfsetlinewidth{0.803000pt}%
\definecolor{currentstroke}{rgb}{0.690196,0.690196,0.690196}%
\pgfsetstrokecolor{currentstroke}%
\pgfsetdash{}{0pt}%
\pgfpathmoveto{\pgfqpoint{0.688192in}{1.130833in}}%
\pgfpathlineto{\pgfqpoint{6.888192in}{1.130833in}}%
\pgfusepath{stroke}%
\end{pgfscope}%
\begin{pgfscope}%
\pgfsetbuttcap%
\pgfsetroundjoin%
\definecolor{currentfill}{rgb}{0.000000,0.000000,0.000000}%
\pgfsetfillcolor{currentfill}%
\pgfsetlinewidth{0.803000pt}%
\definecolor{currentstroke}{rgb}{0.000000,0.000000,0.000000}%
\pgfsetstrokecolor{currentstroke}%
\pgfsetdash{}{0pt}%
\pgfsys@defobject{currentmarker}{\pgfqpoint{-0.048611in}{0.000000in}}{\pgfqpoint{-0.000000in}{0.000000in}}{%
\pgfpathmoveto{\pgfqpoint{-0.000000in}{0.000000in}}%
\pgfpathlineto{\pgfqpoint{-0.048611in}{0.000000in}}%
\pgfusepath{stroke,fill}%
}%
\begin{pgfscope}%
\pgfsys@transformshift{0.688192in}{1.130833in}%
\pgfsys@useobject{currentmarker}{}%
\end{pgfscope}%
\end{pgfscope}%
\begin{pgfscope}%
\definecolor{textcolor}{rgb}{0.000000,0.000000,0.000000}%
\pgfsetstrokecolor{textcolor}%
\pgfsetfillcolor{textcolor}%
\pgftext[x=0.395138in, y=1.061389in, left, base]{\color{textcolor}{\rmfamily\fontsize{14.000000}{16.800000}\selectfont\catcode`\^=\active\def^{\ifmmode\sp\else\^{}\fi}\catcode`\%=\active\def%{\%}$\mathdefault{10}$}}%
\end{pgfscope}%
\begin{pgfscope}%
\pgfpathrectangle{\pgfqpoint{0.688192in}{0.670138in}}{\pgfqpoint{6.200000in}{4.620000in}}%
\pgfusepath{clip}%
\pgfsetrectcap%
\pgfsetroundjoin%
\pgfsetlinewidth{0.803000pt}%
\definecolor{currentstroke}{rgb}{0.690196,0.690196,0.690196}%
\pgfsetstrokecolor{currentstroke}%
\pgfsetdash{}{0pt}%
\pgfpathmoveto{\pgfqpoint{0.688192in}{1.724860in}}%
\pgfpathlineto{\pgfqpoint{6.888192in}{1.724860in}}%
\pgfusepath{stroke}%
\end{pgfscope}%
\begin{pgfscope}%
\pgfsetbuttcap%
\pgfsetroundjoin%
\definecolor{currentfill}{rgb}{0.000000,0.000000,0.000000}%
\pgfsetfillcolor{currentfill}%
\pgfsetlinewidth{0.803000pt}%
\definecolor{currentstroke}{rgb}{0.000000,0.000000,0.000000}%
\pgfsetstrokecolor{currentstroke}%
\pgfsetdash{}{0pt}%
\pgfsys@defobject{currentmarker}{\pgfqpoint{-0.048611in}{0.000000in}}{\pgfqpoint{-0.000000in}{0.000000in}}{%
\pgfpathmoveto{\pgfqpoint{-0.000000in}{0.000000in}}%
\pgfpathlineto{\pgfqpoint{-0.048611in}{0.000000in}}%
\pgfusepath{stroke,fill}%
}%
\begin{pgfscope}%
\pgfsys@transformshift{0.688192in}{1.724860in}%
\pgfsys@useobject{currentmarker}{}%
\end{pgfscope}%
\end{pgfscope}%
\begin{pgfscope}%
\definecolor{textcolor}{rgb}{0.000000,0.000000,0.000000}%
\pgfsetstrokecolor{textcolor}%
\pgfsetfillcolor{textcolor}%
\pgftext[x=0.395138in, y=1.655416in, left, base]{\color{textcolor}{\rmfamily\fontsize{14.000000}{16.800000}\selectfont\catcode`\^=\active\def^{\ifmmode\sp\else\^{}\fi}\catcode`\%=\active\def%{\%}$\mathdefault{20}$}}%
\end{pgfscope}%
\begin{pgfscope}%
\pgfpathrectangle{\pgfqpoint{0.688192in}{0.670138in}}{\pgfqpoint{6.200000in}{4.620000in}}%
\pgfusepath{clip}%
\pgfsetrectcap%
\pgfsetroundjoin%
\pgfsetlinewidth{0.803000pt}%
\definecolor{currentstroke}{rgb}{0.690196,0.690196,0.690196}%
\pgfsetstrokecolor{currentstroke}%
\pgfsetdash{}{0pt}%
\pgfpathmoveto{\pgfqpoint{0.688192in}{2.318888in}}%
\pgfpathlineto{\pgfqpoint{6.888192in}{2.318888in}}%
\pgfusepath{stroke}%
\end{pgfscope}%
\begin{pgfscope}%
\pgfsetbuttcap%
\pgfsetroundjoin%
\definecolor{currentfill}{rgb}{0.000000,0.000000,0.000000}%
\pgfsetfillcolor{currentfill}%
\pgfsetlinewidth{0.803000pt}%
\definecolor{currentstroke}{rgb}{0.000000,0.000000,0.000000}%
\pgfsetstrokecolor{currentstroke}%
\pgfsetdash{}{0pt}%
\pgfsys@defobject{currentmarker}{\pgfqpoint{-0.048611in}{0.000000in}}{\pgfqpoint{-0.000000in}{0.000000in}}{%
\pgfpathmoveto{\pgfqpoint{-0.000000in}{0.000000in}}%
\pgfpathlineto{\pgfqpoint{-0.048611in}{0.000000in}}%
\pgfusepath{stroke,fill}%
}%
\begin{pgfscope}%
\pgfsys@transformshift{0.688192in}{2.318888in}%
\pgfsys@useobject{currentmarker}{}%
\end{pgfscope}%
\end{pgfscope}%
\begin{pgfscope}%
\definecolor{textcolor}{rgb}{0.000000,0.000000,0.000000}%
\pgfsetstrokecolor{textcolor}%
\pgfsetfillcolor{textcolor}%
\pgftext[x=0.395138in, y=2.249444in, left, base]{\color{textcolor}{\rmfamily\fontsize{14.000000}{16.800000}\selectfont\catcode`\^=\active\def^{\ifmmode\sp\else\^{}\fi}\catcode`\%=\active\def%{\%}$\mathdefault{30}$}}%
\end{pgfscope}%
\begin{pgfscope}%
\pgfpathrectangle{\pgfqpoint{0.688192in}{0.670138in}}{\pgfqpoint{6.200000in}{4.620000in}}%
\pgfusepath{clip}%
\pgfsetrectcap%
\pgfsetroundjoin%
\pgfsetlinewidth{0.803000pt}%
\definecolor{currentstroke}{rgb}{0.690196,0.690196,0.690196}%
\pgfsetstrokecolor{currentstroke}%
\pgfsetdash{}{0pt}%
\pgfpathmoveto{\pgfqpoint{0.688192in}{2.912915in}}%
\pgfpathlineto{\pgfqpoint{6.888192in}{2.912915in}}%
\pgfusepath{stroke}%
\end{pgfscope}%
\begin{pgfscope}%
\pgfsetbuttcap%
\pgfsetroundjoin%
\definecolor{currentfill}{rgb}{0.000000,0.000000,0.000000}%
\pgfsetfillcolor{currentfill}%
\pgfsetlinewidth{0.803000pt}%
\definecolor{currentstroke}{rgb}{0.000000,0.000000,0.000000}%
\pgfsetstrokecolor{currentstroke}%
\pgfsetdash{}{0pt}%
\pgfsys@defobject{currentmarker}{\pgfqpoint{-0.048611in}{0.000000in}}{\pgfqpoint{-0.000000in}{0.000000in}}{%
\pgfpathmoveto{\pgfqpoint{-0.000000in}{0.000000in}}%
\pgfpathlineto{\pgfqpoint{-0.048611in}{0.000000in}}%
\pgfusepath{stroke,fill}%
}%
\begin{pgfscope}%
\pgfsys@transformshift{0.688192in}{2.912915in}%
\pgfsys@useobject{currentmarker}{}%
\end{pgfscope}%
\end{pgfscope}%
\begin{pgfscope}%
\definecolor{textcolor}{rgb}{0.000000,0.000000,0.000000}%
\pgfsetstrokecolor{textcolor}%
\pgfsetfillcolor{textcolor}%
\pgftext[x=0.395138in, y=2.843471in, left, base]{\color{textcolor}{\rmfamily\fontsize{14.000000}{16.800000}\selectfont\catcode`\^=\active\def^{\ifmmode\sp\else\^{}\fi}\catcode`\%=\active\def%{\%}$\mathdefault{40}$}}%
\end{pgfscope}%
\begin{pgfscope}%
\pgfpathrectangle{\pgfqpoint{0.688192in}{0.670138in}}{\pgfqpoint{6.200000in}{4.620000in}}%
\pgfusepath{clip}%
\pgfsetrectcap%
\pgfsetroundjoin%
\pgfsetlinewidth{0.803000pt}%
\definecolor{currentstroke}{rgb}{0.690196,0.690196,0.690196}%
\pgfsetstrokecolor{currentstroke}%
\pgfsetdash{}{0pt}%
\pgfpathmoveto{\pgfqpoint{0.688192in}{3.506943in}}%
\pgfpathlineto{\pgfqpoint{6.888192in}{3.506943in}}%
\pgfusepath{stroke}%
\end{pgfscope}%
\begin{pgfscope}%
\pgfsetbuttcap%
\pgfsetroundjoin%
\definecolor{currentfill}{rgb}{0.000000,0.000000,0.000000}%
\pgfsetfillcolor{currentfill}%
\pgfsetlinewidth{0.803000pt}%
\definecolor{currentstroke}{rgb}{0.000000,0.000000,0.000000}%
\pgfsetstrokecolor{currentstroke}%
\pgfsetdash{}{0pt}%
\pgfsys@defobject{currentmarker}{\pgfqpoint{-0.048611in}{0.000000in}}{\pgfqpoint{-0.000000in}{0.000000in}}{%
\pgfpathmoveto{\pgfqpoint{-0.000000in}{0.000000in}}%
\pgfpathlineto{\pgfqpoint{-0.048611in}{0.000000in}}%
\pgfusepath{stroke,fill}%
}%
\begin{pgfscope}%
\pgfsys@transformshift{0.688192in}{3.506943in}%
\pgfsys@useobject{currentmarker}{}%
\end{pgfscope}%
\end{pgfscope}%
\begin{pgfscope}%
\definecolor{textcolor}{rgb}{0.000000,0.000000,0.000000}%
\pgfsetstrokecolor{textcolor}%
\pgfsetfillcolor{textcolor}%
\pgftext[x=0.395138in, y=3.437499in, left, base]{\color{textcolor}{\rmfamily\fontsize{14.000000}{16.800000}\selectfont\catcode`\^=\active\def^{\ifmmode\sp\else\^{}\fi}\catcode`\%=\active\def%{\%}$\mathdefault{50}$}}%
\end{pgfscope}%
\begin{pgfscope}%
\pgfpathrectangle{\pgfqpoint{0.688192in}{0.670138in}}{\pgfqpoint{6.200000in}{4.620000in}}%
\pgfusepath{clip}%
\pgfsetrectcap%
\pgfsetroundjoin%
\pgfsetlinewidth{0.803000pt}%
\definecolor{currentstroke}{rgb}{0.690196,0.690196,0.690196}%
\pgfsetstrokecolor{currentstroke}%
\pgfsetdash{}{0pt}%
\pgfpathmoveto{\pgfqpoint{0.688192in}{4.100970in}}%
\pgfpathlineto{\pgfqpoint{6.888192in}{4.100970in}}%
\pgfusepath{stroke}%
\end{pgfscope}%
\begin{pgfscope}%
\pgfsetbuttcap%
\pgfsetroundjoin%
\definecolor{currentfill}{rgb}{0.000000,0.000000,0.000000}%
\pgfsetfillcolor{currentfill}%
\pgfsetlinewidth{0.803000pt}%
\definecolor{currentstroke}{rgb}{0.000000,0.000000,0.000000}%
\pgfsetstrokecolor{currentstroke}%
\pgfsetdash{}{0pt}%
\pgfsys@defobject{currentmarker}{\pgfqpoint{-0.048611in}{0.000000in}}{\pgfqpoint{-0.000000in}{0.000000in}}{%
\pgfpathmoveto{\pgfqpoint{-0.000000in}{0.000000in}}%
\pgfpathlineto{\pgfqpoint{-0.048611in}{0.000000in}}%
\pgfusepath{stroke,fill}%
}%
\begin{pgfscope}%
\pgfsys@transformshift{0.688192in}{4.100970in}%
\pgfsys@useobject{currentmarker}{}%
\end{pgfscope}%
\end{pgfscope}%
\begin{pgfscope}%
\definecolor{textcolor}{rgb}{0.000000,0.000000,0.000000}%
\pgfsetstrokecolor{textcolor}%
\pgfsetfillcolor{textcolor}%
\pgftext[x=0.395138in, y=4.031526in, left, base]{\color{textcolor}{\rmfamily\fontsize{14.000000}{16.800000}\selectfont\catcode`\^=\active\def^{\ifmmode\sp\else\^{}\fi}\catcode`\%=\active\def%{\%}$\mathdefault{60}$}}%
\end{pgfscope}%
\begin{pgfscope}%
\pgfpathrectangle{\pgfqpoint{0.688192in}{0.670138in}}{\pgfqpoint{6.200000in}{4.620000in}}%
\pgfusepath{clip}%
\pgfsetrectcap%
\pgfsetroundjoin%
\pgfsetlinewidth{0.803000pt}%
\definecolor{currentstroke}{rgb}{0.690196,0.690196,0.690196}%
\pgfsetstrokecolor{currentstroke}%
\pgfsetdash{}{0pt}%
\pgfpathmoveto{\pgfqpoint{0.688192in}{4.694998in}}%
\pgfpathlineto{\pgfqpoint{6.888192in}{4.694998in}}%
\pgfusepath{stroke}%
\end{pgfscope}%
\begin{pgfscope}%
\pgfsetbuttcap%
\pgfsetroundjoin%
\definecolor{currentfill}{rgb}{0.000000,0.000000,0.000000}%
\pgfsetfillcolor{currentfill}%
\pgfsetlinewidth{0.803000pt}%
\definecolor{currentstroke}{rgb}{0.000000,0.000000,0.000000}%
\pgfsetstrokecolor{currentstroke}%
\pgfsetdash{}{0pt}%
\pgfsys@defobject{currentmarker}{\pgfqpoint{-0.048611in}{0.000000in}}{\pgfqpoint{-0.000000in}{0.000000in}}{%
\pgfpathmoveto{\pgfqpoint{-0.000000in}{0.000000in}}%
\pgfpathlineto{\pgfqpoint{-0.048611in}{0.000000in}}%
\pgfusepath{stroke,fill}%
}%
\begin{pgfscope}%
\pgfsys@transformshift{0.688192in}{4.694998in}%
\pgfsys@useobject{currentmarker}{}%
\end{pgfscope}%
\end{pgfscope}%
\begin{pgfscope}%
\definecolor{textcolor}{rgb}{0.000000,0.000000,0.000000}%
\pgfsetstrokecolor{textcolor}%
\pgfsetfillcolor{textcolor}%
\pgftext[x=0.395138in, y=4.625553in, left, base]{\color{textcolor}{\rmfamily\fontsize{14.000000}{16.800000}\selectfont\catcode`\^=\active\def^{\ifmmode\sp\else\^{}\fi}\catcode`\%=\active\def%{\%}$\mathdefault{70}$}}%
\end{pgfscope}%
\begin{pgfscope}%
\pgfpathrectangle{\pgfqpoint{0.688192in}{0.670138in}}{\pgfqpoint{6.200000in}{4.620000in}}%
\pgfusepath{clip}%
\pgfsetrectcap%
\pgfsetroundjoin%
\pgfsetlinewidth{0.803000pt}%
\definecolor{currentstroke}{rgb}{0.690196,0.690196,0.690196}%
\pgfsetstrokecolor{currentstroke}%
\pgfsetdash{}{0pt}%
\pgfpathmoveto{\pgfqpoint{0.688192in}{5.289025in}}%
\pgfpathlineto{\pgfqpoint{6.888192in}{5.289025in}}%
\pgfusepath{stroke}%
\end{pgfscope}%
\begin{pgfscope}%
\pgfsetbuttcap%
\pgfsetroundjoin%
\definecolor{currentfill}{rgb}{0.000000,0.000000,0.000000}%
\pgfsetfillcolor{currentfill}%
\pgfsetlinewidth{0.803000pt}%
\definecolor{currentstroke}{rgb}{0.000000,0.000000,0.000000}%
\pgfsetstrokecolor{currentstroke}%
\pgfsetdash{}{0pt}%
\pgfsys@defobject{currentmarker}{\pgfqpoint{-0.048611in}{0.000000in}}{\pgfqpoint{-0.000000in}{0.000000in}}{%
\pgfpathmoveto{\pgfqpoint{-0.000000in}{0.000000in}}%
\pgfpathlineto{\pgfqpoint{-0.048611in}{0.000000in}}%
\pgfusepath{stroke,fill}%
}%
\begin{pgfscope}%
\pgfsys@transformshift{0.688192in}{5.289025in}%
\pgfsys@useobject{currentmarker}{}%
\end{pgfscope}%
\end{pgfscope}%
\begin{pgfscope}%
\definecolor{textcolor}{rgb}{0.000000,0.000000,0.000000}%
\pgfsetstrokecolor{textcolor}%
\pgfsetfillcolor{textcolor}%
\pgftext[x=0.395138in, y=5.219581in, left, base]{\color{textcolor}{\rmfamily\fontsize{14.000000}{16.800000}\selectfont\catcode`\^=\active\def^{\ifmmode\sp\else\^{}\fi}\catcode`\%=\active\def%{\%}$\mathdefault{80}$}}%
\end{pgfscope}%
\begin{pgfscope}%
\definecolor{textcolor}{rgb}{0.000000,0.000000,0.000000}%
\pgfsetstrokecolor{textcolor}%
\pgfsetfillcolor{textcolor}%
\pgftext[x=0.339583in,y=2.980138in,,bottom,rotate=90.000000]{\color{textcolor}{\rmfamily\fontsize{18.000000}{21.600000}\selectfont\catcode`\^=\active\def^{\ifmmode\sp\else\^{}\fi}\catcode`\%=\active\def%{\%}CO2 emissions (MT CO2)}}%
\end{pgfscope}%
\begin{pgfscope}%
\pgfpathrectangle{\pgfqpoint{0.688192in}{0.670138in}}{\pgfqpoint{6.200000in}{4.620000in}}%
\pgfusepath{clip}%
\pgfsetrectcap%
\pgfsetroundjoin%
\pgfsetlinewidth{1.505625pt}%
\definecolor{currentstroke}{rgb}{0.000000,0.000000,1.000000}%
\pgfsetstrokecolor{currentstroke}%
\pgfsetdash{}{0pt}%
\pgfpathmoveto{\pgfqpoint{0.741425in}{1.377543in}}%
\pgfpathlineto{\pgfqpoint{0.758703in}{0.955032in}}%
\pgfpathlineto{\pgfqpoint{0.768198in}{0.875033in}}%
\pgfpathlineto{\pgfqpoint{0.774746in}{0.828781in}}%
\pgfpathlineto{\pgfqpoint{0.778243in}{0.822495in}}%
\pgfpathlineto{\pgfqpoint{0.782159in}{0.789611in}}%
\pgfpathlineto{\pgfqpoint{0.786516in}{0.779881in}}%
\pgfpathlineto{\pgfqpoint{0.792538in}{0.779145in}}%
\pgfpathlineto{\pgfqpoint{0.794668in}{0.758056in}}%
\pgfpathlineto{\pgfqpoint{0.799837in}{0.752930in}}%
\pgfpathlineto{\pgfqpoint{0.809370in}{0.751978in}}%
\pgfpathlineto{\pgfqpoint{0.812629in}{0.743975in}}%
\pgfpathlineto{\pgfqpoint{0.815972in}{0.742575in}}%
\pgfpathlineto{\pgfqpoint{0.822987in}{0.738477in}}%
\pgfpathlineto{\pgfqpoint{0.828825in}{0.734937in}}%
\pgfpathlineto{\pgfqpoint{0.829214in}{0.733319in}}%
\pgfpathlineto{\pgfqpoint{0.833044in}{0.730858in}}%
\pgfpathlineto{\pgfqpoint{0.848459in}{0.726329in}}%
\pgfpathlineto{\pgfqpoint{0.864854in}{0.720019in}}%
\pgfpathlineto{\pgfqpoint{0.887104in}{0.715517in}}%
\pgfpathlineto{\pgfqpoint{0.907479in}{0.714004in}}%
\pgfpathlineto{\pgfqpoint{0.908310in}{0.712008in}}%
\pgfpathlineto{\pgfqpoint{0.909513in}{0.708525in}}%
\pgfpathlineto{\pgfqpoint{0.912740in}{0.707284in}}%
\pgfpathlineto{\pgfqpoint{0.920440in}{0.706723in}}%
\pgfpathlineto{\pgfqpoint{0.925670in}{0.705238in}}%
\pgfpathlineto{\pgfqpoint{0.948903in}{0.702931in}}%
\pgfpathlineto{\pgfqpoint{0.951945in}{0.701707in}}%
\pgfpathlineto{\pgfqpoint{0.952035in}{0.700391in}}%
\pgfpathlineto{\pgfqpoint{0.957029in}{0.700173in}}%
\pgfpathlineto{\pgfqpoint{0.968828in}{0.697963in}}%
\pgfpathlineto{\pgfqpoint{0.974412in}{0.697738in}}%
\pgfpathlineto{\pgfqpoint{0.975275in}{0.696914in}}%
\pgfpathlineto{\pgfqpoint{1.021767in}{0.694795in}}%
\pgfpathlineto{\pgfqpoint{1.025407in}{0.690657in}}%
\pgfpathlineto{\pgfqpoint{1.027475in}{0.690338in}}%
\pgfpathlineto{\pgfqpoint{1.034837in}{0.689784in}}%
\pgfpathlineto{\pgfqpoint{1.049406in}{0.687676in}}%
\pgfpathlineto{\pgfqpoint{1.054714in}{0.687138in}}%
\pgfpathlineto{\pgfqpoint{1.059617in}{0.686467in}}%
\pgfpathlineto{\pgfqpoint{1.072141in}{0.685078in}}%
\pgfpathlineto{\pgfqpoint{1.092208in}{0.684413in}}%
\pgfpathlineto{\pgfqpoint{1.115209in}{0.684111in}}%
\pgfpathlineto{\pgfqpoint{1.131834in}{0.684071in}}%
\pgfpathlineto{\pgfqpoint{1.152628in}{0.684059in}}%
\pgfpathlineto{\pgfqpoint{1.251312in}{0.683263in}}%
\pgfpathlineto{\pgfqpoint{1.277476in}{0.683159in}}%
\pgfpathlineto{\pgfqpoint{1.314870in}{0.682855in}}%
\pgfpathlineto{\pgfqpoint{1.369253in}{0.682756in}}%
\pgfpathlineto{\pgfqpoint{1.398687in}{0.682288in}}%
\pgfpathlineto{\pgfqpoint{1.467852in}{0.682134in}}%
\pgfpathlineto{\pgfqpoint{1.557026in}{0.681680in}}%
\pgfpathlineto{\pgfqpoint{1.627242in}{0.680913in}}%
\pgfpathlineto{\pgfqpoint{1.737728in}{0.680478in}}%
\pgfpathlineto{\pgfqpoint{1.887036in}{0.679610in}}%
\pgfpathlineto{\pgfqpoint{2.037481in}{0.678826in}}%
\pgfpathlineto{\pgfqpoint{2.258348in}{0.677741in}}%
\pgfpathlineto{\pgfqpoint{2.626338in}{0.676361in}}%
\pgfpathlineto{\pgfqpoint{3.263784in}{0.674352in}}%
\pgfpathlineto{\pgfqpoint{5.322800in}{0.670138in}}%
\pgfusepath{stroke}%
\end{pgfscope}%
\begin{pgfscope}%
\pgfpathrectangle{\pgfqpoint{0.688192in}{0.670138in}}{\pgfqpoint{6.200000in}{4.620000in}}%
\pgfusepath{clip}%
\pgfsetbuttcap%
\pgfsetroundjoin%
\definecolor{currentfill}{rgb}{0.000000,0.000000,1.000000}%
\pgfsetfillcolor{currentfill}%
\pgfsetlinewidth{1.003750pt}%
\definecolor{currentstroke}{rgb}{0.000000,0.000000,1.000000}%
\pgfsetstrokecolor{currentstroke}%
\pgfsetdash{}{0pt}%
\pgfsys@defobject{currentmarker}{\pgfqpoint{-0.006944in}{-0.006944in}}{\pgfqpoint{0.006944in}{0.006944in}}{%
\pgfpathmoveto{\pgfqpoint{0.000000in}{-0.006944in}}%
\pgfpathcurveto{\pgfqpoint{0.001842in}{-0.006944in}}{\pgfqpoint{0.003608in}{-0.006213in}}{\pgfqpoint{0.004910in}{-0.004910in}}%
\pgfpathcurveto{\pgfqpoint{0.006213in}{-0.003608in}}{\pgfqpoint{0.006944in}{-0.001842in}}{\pgfqpoint{0.006944in}{0.000000in}}%
\pgfpathcurveto{\pgfqpoint{0.006944in}{0.001842in}}{\pgfqpoint{0.006213in}{0.003608in}}{\pgfqpoint{0.004910in}{0.004910in}}%
\pgfpathcurveto{\pgfqpoint{0.003608in}{0.006213in}}{\pgfqpoint{0.001842in}{0.006944in}}{\pgfqpoint{0.000000in}{0.006944in}}%
\pgfpathcurveto{\pgfqpoint{-0.001842in}{0.006944in}}{\pgfqpoint{-0.003608in}{0.006213in}}{\pgfqpoint{-0.004910in}{0.004910in}}%
\pgfpathcurveto{\pgfqpoint{-0.006213in}{0.003608in}}{\pgfqpoint{-0.006944in}{0.001842in}}{\pgfqpoint{-0.006944in}{0.000000in}}%
\pgfpathcurveto{\pgfqpoint{-0.006944in}{-0.001842in}}{\pgfqpoint{-0.006213in}{-0.003608in}}{\pgfqpoint{-0.004910in}{-0.004910in}}%
\pgfpathcurveto{\pgfqpoint{-0.003608in}{-0.006213in}}{\pgfqpoint{-0.001842in}{-0.006944in}}{\pgfqpoint{0.000000in}{-0.006944in}}%
\pgfpathlineto{\pgfqpoint{0.000000in}{-0.006944in}}%
\pgfpathclose%
\pgfusepath{stroke,fill}%
}%
\begin{pgfscope}%
\pgfsys@transformshift{0.741425in}{1.377543in}%
\pgfsys@useobject{currentmarker}{}%
\end{pgfscope}%
\begin{pgfscope}%
\pgfsys@transformshift{0.758703in}{0.955032in}%
\pgfsys@useobject{currentmarker}{}%
\end{pgfscope}%
\begin{pgfscope}%
\pgfsys@transformshift{0.768198in}{0.875033in}%
\pgfsys@useobject{currentmarker}{}%
\end{pgfscope}%
\begin{pgfscope}%
\pgfsys@transformshift{0.774746in}{0.828781in}%
\pgfsys@useobject{currentmarker}{}%
\end{pgfscope}%
\begin{pgfscope}%
\pgfsys@transformshift{0.778243in}{0.822495in}%
\pgfsys@useobject{currentmarker}{}%
\end{pgfscope}%
\begin{pgfscope}%
\pgfsys@transformshift{0.782159in}{0.789611in}%
\pgfsys@useobject{currentmarker}{}%
\end{pgfscope}%
\begin{pgfscope}%
\pgfsys@transformshift{0.786516in}{0.779881in}%
\pgfsys@useobject{currentmarker}{}%
\end{pgfscope}%
\begin{pgfscope}%
\pgfsys@transformshift{0.792538in}{0.779145in}%
\pgfsys@useobject{currentmarker}{}%
\end{pgfscope}%
\begin{pgfscope}%
\pgfsys@transformshift{0.794668in}{0.758056in}%
\pgfsys@useobject{currentmarker}{}%
\end{pgfscope}%
\begin{pgfscope}%
\pgfsys@transformshift{0.799837in}{0.752930in}%
\pgfsys@useobject{currentmarker}{}%
\end{pgfscope}%
\begin{pgfscope}%
\pgfsys@transformshift{0.809370in}{0.751978in}%
\pgfsys@useobject{currentmarker}{}%
\end{pgfscope}%
\begin{pgfscope}%
\pgfsys@transformshift{0.812629in}{0.743975in}%
\pgfsys@useobject{currentmarker}{}%
\end{pgfscope}%
\begin{pgfscope}%
\pgfsys@transformshift{0.815972in}{0.742575in}%
\pgfsys@useobject{currentmarker}{}%
\end{pgfscope}%
\begin{pgfscope}%
\pgfsys@transformshift{0.822987in}{0.738477in}%
\pgfsys@useobject{currentmarker}{}%
\end{pgfscope}%
\begin{pgfscope}%
\pgfsys@transformshift{0.828825in}{0.734937in}%
\pgfsys@useobject{currentmarker}{}%
\end{pgfscope}%
\begin{pgfscope}%
\pgfsys@transformshift{0.829214in}{0.733319in}%
\pgfsys@useobject{currentmarker}{}%
\end{pgfscope}%
\begin{pgfscope}%
\pgfsys@transformshift{0.833044in}{0.730858in}%
\pgfsys@useobject{currentmarker}{}%
\end{pgfscope}%
\begin{pgfscope}%
\pgfsys@transformshift{0.848459in}{0.726329in}%
\pgfsys@useobject{currentmarker}{}%
\end{pgfscope}%
\begin{pgfscope}%
\pgfsys@transformshift{0.864854in}{0.720019in}%
\pgfsys@useobject{currentmarker}{}%
\end{pgfscope}%
\begin{pgfscope}%
\pgfsys@transformshift{0.887104in}{0.715517in}%
\pgfsys@useobject{currentmarker}{}%
\end{pgfscope}%
\begin{pgfscope}%
\pgfsys@transformshift{0.907479in}{0.714004in}%
\pgfsys@useobject{currentmarker}{}%
\end{pgfscope}%
\begin{pgfscope}%
\pgfsys@transformshift{0.908310in}{0.712008in}%
\pgfsys@useobject{currentmarker}{}%
\end{pgfscope}%
\begin{pgfscope}%
\pgfsys@transformshift{0.909513in}{0.708525in}%
\pgfsys@useobject{currentmarker}{}%
\end{pgfscope}%
\begin{pgfscope}%
\pgfsys@transformshift{0.912740in}{0.707284in}%
\pgfsys@useobject{currentmarker}{}%
\end{pgfscope}%
\begin{pgfscope}%
\pgfsys@transformshift{0.920440in}{0.706723in}%
\pgfsys@useobject{currentmarker}{}%
\end{pgfscope}%
\begin{pgfscope}%
\pgfsys@transformshift{0.925670in}{0.705238in}%
\pgfsys@useobject{currentmarker}{}%
\end{pgfscope}%
\begin{pgfscope}%
\pgfsys@transformshift{0.948903in}{0.702931in}%
\pgfsys@useobject{currentmarker}{}%
\end{pgfscope}%
\begin{pgfscope}%
\pgfsys@transformshift{0.951945in}{0.701707in}%
\pgfsys@useobject{currentmarker}{}%
\end{pgfscope}%
\begin{pgfscope}%
\pgfsys@transformshift{0.952035in}{0.700391in}%
\pgfsys@useobject{currentmarker}{}%
\end{pgfscope}%
\begin{pgfscope}%
\pgfsys@transformshift{0.957029in}{0.700173in}%
\pgfsys@useobject{currentmarker}{}%
\end{pgfscope}%
\begin{pgfscope}%
\pgfsys@transformshift{0.968828in}{0.697963in}%
\pgfsys@useobject{currentmarker}{}%
\end{pgfscope}%
\begin{pgfscope}%
\pgfsys@transformshift{0.974412in}{0.697738in}%
\pgfsys@useobject{currentmarker}{}%
\end{pgfscope}%
\begin{pgfscope}%
\pgfsys@transformshift{0.975275in}{0.696914in}%
\pgfsys@useobject{currentmarker}{}%
\end{pgfscope}%
\begin{pgfscope}%
\pgfsys@transformshift{1.021767in}{0.694795in}%
\pgfsys@useobject{currentmarker}{}%
\end{pgfscope}%
\begin{pgfscope}%
\pgfsys@transformshift{1.025407in}{0.690657in}%
\pgfsys@useobject{currentmarker}{}%
\end{pgfscope}%
\begin{pgfscope}%
\pgfsys@transformshift{1.027475in}{0.690338in}%
\pgfsys@useobject{currentmarker}{}%
\end{pgfscope}%
\begin{pgfscope}%
\pgfsys@transformshift{1.034837in}{0.689784in}%
\pgfsys@useobject{currentmarker}{}%
\end{pgfscope}%
\begin{pgfscope}%
\pgfsys@transformshift{1.049406in}{0.687676in}%
\pgfsys@useobject{currentmarker}{}%
\end{pgfscope}%
\begin{pgfscope}%
\pgfsys@transformshift{1.054714in}{0.687138in}%
\pgfsys@useobject{currentmarker}{}%
\end{pgfscope}%
\begin{pgfscope}%
\pgfsys@transformshift{1.059617in}{0.686467in}%
\pgfsys@useobject{currentmarker}{}%
\end{pgfscope}%
\begin{pgfscope}%
\pgfsys@transformshift{1.072141in}{0.685078in}%
\pgfsys@useobject{currentmarker}{}%
\end{pgfscope}%
\begin{pgfscope}%
\pgfsys@transformshift{1.092208in}{0.684413in}%
\pgfsys@useobject{currentmarker}{}%
\end{pgfscope}%
\begin{pgfscope}%
\pgfsys@transformshift{1.115209in}{0.684111in}%
\pgfsys@useobject{currentmarker}{}%
\end{pgfscope}%
\begin{pgfscope}%
\pgfsys@transformshift{1.131834in}{0.684071in}%
\pgfsys@useobject{currentmarker}{}%
\end{pgfscope}%
\begin{pgfscope}%
\pgfsys@transformshift{1.152628in}{0.684059in}%
\pgfsys@useobject{currentmarker}{}%
\end{pgfscope}%
\begin{pgfscope}%
\pgfsys@transformshift{1.251312in}{0.683263in}%
\pgfsys@useobject{currentmarker}{}%
\end{pgfscope}%
\begin{pgfscope}%
\pgfsys@transformshift{1.277476in}{0.683159in}%
\pgfsys@useobject{currentmarker}{}%
\end{pgfscope}%
\begin{pgfscope}%
\pgfsys@transformshift{1.314870in}{0.682855in}%
\pgfsys@useobject{currentmarker}{}%
\end{pgfscope}%
\begin{pgfscope}%
\pgfsys@transformshift{1.369253in}{0.682756in}%
\pgfsys@useobject{currentmarker}{}%
\end{pgfscope}%
\begin{pgfscope}%
\pgfsys@transformshift{1.398687in}{0.682288in}%
\pgfsys@useobject{currentmarker}{}%
\end{pgfscope}%
\begin{pgfscope}%
\pgfsys@transformshift{1.467852in}{0.682134in}%
\pgfsys@useobject{currentmarker}{}%
\end{pgfscope}%
\begin{pgfscope}%
\pgfsys@transformshift{1.557026in}{0.681680in}%
\pgfsys@useobject{currentmarker}{}%
\end{pgfscope}%
\begin{pgfscope}%
\pgfsys@transformshift{1.627242in}{0.680913in}%
\pgfsys@useobject{currentmarker}{}%
\end{pgfscope}%
\begin{pgfscope}%
\pgfsys@transformshift{1.737728in}{0.680478in}%
\pgfsys@useobject{currentmarker}{}%
\end{pgfscope}%
\begin{pgfscope}%
\pgfsys@transformshift{1.887036in}{0.679610in}%
\pgfsys@useobject{currentmarker}{}%
\end{pgfscope}%
\begin{pgfscope}%
\pgfsys@transformshift{2.037481in}{0.678826in}%
\pgfsys@useobject{currentmarker}{}%
\end{pgfscope}%
\begin{pgfscope}%
\pgfsys@transformshift{2.258348in}{0.677741in}%
\pgfsys@useobject{currentmarker}{}%
\end{pgfscope}%
\begin{pgfscope}%
\pgfsys@transformshift{2.626338in}{0.676361in}%
\pgfsys@useobject{currentmarker}{}%
\end{pgfscope}%
\begin{pgfscope}%
\pgfsys@transformshift{3.263784in}{0.674352in}%
\pgfsys@useobject{currentmarker}{}%
\end{pgfscope}%
\begin{pgfscope}%
\pgfsys@transformshift{5.322800in}{0.670138in}%
\pgfsys@useobject{currentmarker}{}%
\end{pgfscope}%
\end{pgfscope}%
\begin{pgfscope}%
\pgfpathrectangle{\pgfqpoint{0.688192in}{0.670138in}}{\pgfqpoint{6.200000in}{4.620000in}}%
\pgfusepath{clip}%
\pgfsetrectcap%
\pgfsetroundjoin%
\pgfsetlinewidth{1.505625pt}%
\definecolor{currentstroke}{rgb}{0.121569,0.466667,0.705882}%
\pgfsetstrokecolor{currentstroke}%
\pgfsetstrokeopacity{0.500000}%
\pgfsetdash{}{0pt}%
\pgfpathmoveto{\pgfqpoint{1.848679in}{1.461617in}}%
\pgfpathlineto{\pgfqpoint{1.867684in}{0.996854in}}%
\pgfpathlineto{\pgfqpoint{1.878129in}{0.908856in}}%
\pgfpathlineto{\pgfqpoint{1.885331in}{0.857978in}}%
\pgfpathlineto{\pgfqpoint{1.889178in}{0.851064in}}%
\pgfpathlineto{\pgfqpoint{1.893486in}{0.814892in}}%
\pgfpathlineto{\pgfqpoint{1.898279in}{0.804188in}}%
\pgfpathlineto{\pgfqpoint{1.904902in}{0.803379in}}%
\pgfpathlineto{\pgfqpoint{1.907246in}{0.780181in}}%
\pgfpathlineto{\pgfqpoint{1.912932in}{0.774543in}}%
\pgfpathlineto{\pgfqpoint{1.923418in}{0.773495in}}%
\pgfpathlineto{\pgfqpoint{1.927003in}{0.764692in}}%
\pgfpathlineto{\pgfqpoint{1.930681in}{0.763152in}}%
\pgfpathlineto{\pgfqpoint{1.938397in}{0.758644in}}%
\pgfpathlineto{\pgfqpoint{1.944819in}{0.754750in}}%
\pgfpathlineto{\pgfqpoint{1.945246in}{0.752971in}}%
\pgfpathlineto{\pgfqpoint{1.949459in}{0.750264in}}%
\pgfpathlineto{\pgfqpoint{1.966417in}{0.745282in}}%
\pgfpathlineto{\pgfqpoint{1.984450in}{0.738340in}}%
\pgfpathlineto{\pgfqpoint{2.008926in}{0.733388in}}%
\pgfpathlineto{\pgfqpoint{2.031338in}{0.731724in}}%
\pgfpathlineto{\pgfqpoint{2.032252in}{0.729528in}}%
\pgfpathlineto{\pgfqpoint{2.033576in}{0.725697in}}%
\pgfpathlineto{\pgfqpoint{2.037125in}{0.724332in}}%
\pgfpathlineto{\pgfqpoint{2.045595in}{0.723715in}}%
\pgfpathlineto{\pgfqpoint{2.051349in}{0.722081in}}%
\pgfpathlineto{\pgfqpoint{2.076904in}{0.719544in}}%
\pgfpathlineto{\pgfqpoint{2.080251in}{0.718197in}}%
\pgfpathlineto{\pgfqpoint{2.080349in}{0.716749in}}%
\pgfpathlineto{\pgfqpoint{2.085843in}{0.716510in}}%
\pgfpathlineto{\pgfqpoint{2.098822in}{0.714079in}}%
\pgfpathlineto{\pgfqpoint{2.104964in}{0.713831in}}%
\pgfpathlineto{\pgfqpoint{2.105914in}{0.712924in}}%
\pgfpathlineto{\pgfqpoint{2.157055in}{0.710594in}}%
\pgfpathlineto{\pgfqpoint{2.161059in}{0.706043in}}%
\pgfpathlineto{\pgfqpoint{2.163334in}{0.705692in}}%
\pgfpathlineto{\pgfqpoint{2.171432in}{0.705082in}}%
\pgfpathlineto{\pgfqpoint{2.187458in}{0.702763in}}%
\pgfpathlineto{\pgfqpoint{2.193297in}{0.702172in}}%
\pgfpathlineto{\pgfqpoint{2.198690in}{0.701434in}}%
\pgfpathlineto{\pgfqpoint{2.212467in}{0.699905in}}%
\pgfpathlineto{\pgfqpoint{2.234540in}{0.699174in}}%
\pgfpathlineto{\pgfqpoint{2.259841in}{0.698842in}}%
\pgfpathlineto{\pgfqpoint{2.278129in}{0.698798in}}%
\pgfpathlineto{\pgfqpoint{2.301002in}{0.698784in}}%
\pgfpathlineto{\pgfqpoint{2.409554in}{0.697909in}}%
\pgfpathlineto{\pgfqpoint{2.438334in}{0.697795in}}%
\pgfpathlineto{\pgfqpoint{2.479468in}{0.697460in}}%
\pgfpathlineto{\pgfqpoint{2.539290in}{0.697351in}}%
\pgfpathlineto{\pgfqpoint{2.571667in}{0.696836in}}%
\pgfpathlineto{\pgfqpoint{2.647748in}{0.696667in}}%
\pgfpathlineto{\pgfqpoint{2.745840in}{0.696168in}}%
\pgfpathlineto{\pgfqpoint{2.823078in}{0.695324in}}%
\pgfpathlineto{\pgfqpoint{2.944612in}{0.694845in}}%
\pgfpathlineto{\pgfqpoint{3.108851in}{0.693891in}}%
\pgfpathlineto{\pgfqpoint{3.274341in}{0.693028in}}%
\pgfpathlineto{\pgfqpoint{3.517294in}{0.691835in}}%
\pgfpathlineto{\pgfqpoint{3.922083in}{0.690316in}}%
\pgfpathlineto{\pgfqpoint{4.623274in}{0.688107in}}%
\pgfpathlineto{\pgfqpoint{6.888192in}{0.683471in}}%
\pgfusepath{stroke}%
\end{pgfscope}%
\begin{pgfscope}%
\pgfsetrectcap%
\pgfsetmiterjoin%
\pgfsetlinewidth{0.803000pt}%
\definecolor{currentstroke}{rgb}{0.000000,0.000000,0.000000}%
\pgfsetstrokecolor{currentstroke}%
\pgfsetdash{}{0pt}%
\pgfpathmoveto{\pgfqpoint{0.688192in}{0.670138in}}%
\pgfpathlineto{\pgfqpoint{0.688192in}{5.290138in}}%
\pgfusepath{stroke}%
\end{pgfscope}%
\begin{pgfscope}%
\pgfsetrectcap%
\pgfsetmiterjoin%
\pgfsetlinewidth{0.803000pt}%
\definecolor{currentstroke}{rgb}{0.000000,0.000000,0.000000}%
\pgfsetstrokecolor{currentstroke}%
\pgfsetdash{}{0pt}%
\pgfpathmoveto{\pgfqpoint{6.888192in}{0.670138in}}%
\pgfpathlineto{\pgfqpoint{6.888192in}{5.290138in}}%
\pgfusepath{stroke}%
\end{pgfscope}%
\begin{pgfscope}%
\pgfsetrectcap%
\pgfsetmiterjoin%
\pgfsetlinewidth{0.803000pt}%
\definecolor{currentstroke}{rgb}{0.000000,0.000000,0.000000}%
\pgfsetstrokecolor{currentstroke}%
\pgfsetdash{}{0pt}%
\pgfpathmoveto{\pgfqpoint{0.688192in}{0.670138in}}%
\pgfpathlineto{\pgfqpoint{6.888192in}{0.670138in}}%
\pgfusepath{stroke}%
\end{pgfscope}%
\begin{pgfscope}%
\pgfsetrectcap%
\pgfsetmiterjoin%
\pgfsetlinewidth{0.803000pt}%
\definecolor{currentstroke}{rgb}{0.000000,0.000000,0.000000}%
\pgfsetstrokecolor{currentstroke}%
\pgfsetdash{}{0pt}%
\pgfpathmoveto{\pgfqpoint{0.688192in}{5.290138in}}%
\pgfpathlineto{\pgfqpoint{6.888192in}{5.290138in}}%
\pgfusepath{stroke}%
\end{pgfscope}%
\begin{pgfscope}%
\pgfsetbuttcap%
\pgfsetmiterjoin%
\pgfsetlinewidth{1.003750pt}%
\definecolor{currentstroke}{rgb}{0.000000,0.000000,0.000000}%
\pgfsetstrokecolor{currentstroke}%
\pgfsetstrokeopacity{0.500000}%
\pgfsetdash{}{0pt}%
\pgfpathmoveto{\pgfqpoint{0.646542in}{1.071430in}}%
\pgfpathlineto{\pgfqpoint{0.810103in}{1.071430in}}%
\pgfpathlineto{\pgfqpoint{0.810103in}{1.434970in}}%
\pgfpathlineto{\pgfqpoint{0.646542in}{1.434970in}}%
\pgfpathlineto{\pgfqpoint{0.646542in}{1.071430in}}%
\pgfpathclose%
\pgfpathmoveto{\pgfqpoint{3.788192in}{5.151538in}}%
\pgfpathquadraticcurveto{\pgfqpoint{2.217367in}{3.293254in}}{\pgfqpoint{0.646542in}{1.434970in}}%
\pgfpathmoveto{\pgfqpoint{6.702192in}{2.980138in}}%
\pgfpathquadraticcurveto{\pgfqpoint{3.756147in}{2.025784in}}{\pgfqpoint{0.810103in}{1.071430in}}%
\pgfusepath{stroke}%
\end{pgfscope}%
\begin{pgfscope}%
\pgfsetbuttcap%
\pgfsetmiterjoin%
\definecolor{currentfill}{rgb}{1.000000,1.000000,1.000000}%
\pgfsetfillcolor{currentfill}%
\pgfsetlinewidth{0.000000pt}%
\definecolor{currentstroke}{rgb}{0.000000,0.000000,0.000000}%
\pgfsetstrokecolor{currentstroke}%
\pgfsetstrokeopacity{0.000000}%
\pgfsetdash{}{0pt}%
\pgfpathmoveto{\pgfqpoint{3.788192in}{2.980138in}}%
\pgfpathlineto{\pgfqpoint{6.702192in}{2.980138in}}%
\pgfpathlineto{\pgfqpoint{6.702192in}{5.151538in}}%
\pgfpathlineto{\pgfqpoint{3.788192in}{5.151538in}}%
\pgfpathlineto{\pgfqpoint{3.788192in}{2.980138in}}%
\pgfpathclose%
\pgfusepath{fill}%
\end{pgfscope}%
\begin{pgfscope}%
\pgfpathrectangle{\pgfqpoint{3.788192in}{2.980138in}}{\pgfqpoint{2.914000in}{2.171400in}}%
\pgfusepath{clip}%
\pgfsetbuttcap%
\pgfsetmiterjoin%
\definecolor{currentfill}{rgb}{0.121569,0.466667,0.705882}%
\pgfsetfillcolor{currentfill}%
\pgfsetfillopacity{0.500000}%
\pgfsetlinewidth{1.003750pt}%
\definecolor{currentstroke}{rgb}{0.121569,0.466667,0.705882}%
\pgfsetstrokecolor{currentstroke}%
\pgfsetstrokeopacity{0.500000}%
\pgfsetdash{}{0pt}%
\pgfpathmoveto{\pgfqpoint{5.478622in}{4.808529in}}%
\pgfpathlineto{\pgfqpoint{5.786444in}{2.284897in}}%
\pgfpathlineto{\pgfqpoint{5.955602in}{1.807070in}}%
\pgfpathlineto{\pgfqpoint{6.072257in}{1.530809in}}%
\pgfpathlineto{\pgfqpoint{6.134561in}{1.493263in}}%
\pgfpathlineto{\pgfqpoint{6.204334in}{1.296852in}}%
\pgfpathlineto{\pgfqpoint{6.281957in}{1.238734in}}%
\pgfpathlineto{\pgfqpoint{6.389240in}{1.234341in}}%
\pgfpathlineto{\pgfqpoint{6.427201in}{1.108377in}}%
\pgfpathlineto{\pgfqpoint{6.519285in}{1.077759in}}%
\pgfpathlineto{\pgfqpoint{6.689124in}{1.072073in}}%
\pgfpathlineto{\pgfqpoint{6.747190in}{1.024269in}}%
\pgfpathlineto{\pgfqpoint{6.806750in}{1.015907in}}%
\pgfpathlineto{\pgfqpoint{6.931730in}{0.991431in}}%
\pgfpathlineto{\pgfqpoint{7.035734in}{0.970288in}}%
\pgfpathlineto{\pgfqpoint{7.042663in}{0.960624in}}%
\pgfpathlineto{\pgfqpoint{7.110896in}{0.945926in}}%
\pgfpathlineto{\pgfqpoint{7.385541in}{0.918874in}}%
\pgfpathlineto{\pgfqpoint{7.677622in}{0.881182in}}%
\pgfpathlineto{\pgfqpoint{8.074033in}{0.854291in}}%
\pgfpathlineto{\pgfqpoint{8.437037in}{0.845259in}}%
\pgfpathlineto{\pgfqpoint{8.451833in}{0.833333in}}%
\pgfpathlineto{\pgfqpoint{8.473270in}{0.812531in}}%
\pgfpathlineto{\pgfqpoint{8.530759in}{0.805118in}}%
\pgfpathlineto{\pgfqpoint{8.667946in}{0.801766in}}%
\pgfpathlineto{\pgfqpoint{8.761129in}{0.792896in}}%
\pgfpathlineto{\pgfqpoint{9.175032in}{0.779118in}}%
\pgfpathlineto{\pgfqpoint{9.229242in}{0.771806in}}%
\pgfpathlineto{\pgfqpoint{9.230836in}{0.763945in}}%
\pgfpathlineto{\pgfqpoint{9.319803in}{0.762647in}}%
\pgfpathlineto{\pgfqpoint{9.530028in}{0.749444in}}%
\pgfpathlineto{\pgfqpoint{9.629502in}{0.748101in}}%
\pgfpathlineto{\pgfqpoint{9.644888in}{0.743176in}}%
\pgfpathlineto{\pgfqpoint{10.473184in}{0.730520in}}%
\pgfpathlineto{\pgfqpoint{10.538033in}{0.705808in}}%
\pgfpathlineto{\pgfqpoint{10.574876in}{0.703903in}}%
\pgfpathlineto{\pgfqpoint{10.706030in}{0.700593in}}%
\pgfpathlineto{\pgfqpoint{10.965602in}{0.687998in}}%
\pgfpathlineto{\pgfqpoint{11.060169in}{0.684789in}}%
\pgfpathlineto{\pgfqpoint{11.147519in}{0.680781in}}%
\pgfpathlineto{\pgfqpoint{11.370645in}{0.672484in}}%
\pgfpathlineto{\pgfqpoint{11.728159in}{0.668514in}}%
\pgfpathlineto{\pgfqpoint{12.137942in}{0.666709in}}%
\pgfpathlineto{\pgfqpoint{12.434130in}{0.666468in}}%
\pgfpathlineto{\pgfqpoint{12.804602in}{0.666397in}}%
\pgfpathlineto{\pgfqpoint{14.562739in}{0.661643in}}%
\pgfpathlineto{\pgfqpoint{15.028873in}{0.661022in}}%
\pgfpathlineto{\pgfqpoint{15.695083in}{0.659204in}}%
\pgfpathlineto{\pgfqpoint{16.663977in}{0.658612in}}%
\pgfpathlineto{\pgfqpoint{17.188373in}{0.655818in}}%
\pgfpathlineto{\pgfqpoint{18.420605in}{0.654898in}}%
\pgfpathlineto{\pgfqpoint{20.009331in}{0.652189in}}%
\pgfpathlineto{\pgfqpoint{21.260300in}{0.647607in}}%
\pgfpathlineto{\pgfqpoint{23.228701in}{0.645006in}}%
\pgfpathlineto{\pgfqpoint{25.888770in}{0.639824in}}%
\pgfpathlineto{\pgfqpoint{28.569097in}{0.635141in}}%
\pgfpathlineto{\pgfqpoint{32.504050in}{0.628662in}}%
\pgfpathlineto{\pgfqpoint{39.060135in}{0.620415in}}%
\pgfpathlineto{\pgfqpoint{50.416853in}{0.608419in}}%
\pgfpathlineto{\pgfqpoint{87.100182in}{0.583248in}}%
\pgfpathlineto{\pgfqpoint{114.989116in}{0.662886in}}%
\pgfpathlineto{\pgfqpoint{74.637454in}{0.690574in}}%
\pgfpathlineto{\pgfqpoint{62.145064in}{0.703770in}}%
\pgfpathlineto{\pgfqpoint{54.933371in}{0.712842in}}%
\pgfpathlineto{\pgfqpoint{50.604922in}{0.719969in}}%
\pgfpathlineto{\pgfqpoint{47.656562in}{0.725121in}}%
\pgfpathlineto{\pgfqpoint{44.730486in}{0.730820in}}%
\pgfpathlineto{\pgfqpoint{42.565245in}{0.733682in}}%
\pgfpathlineto{\pgfqpoint{41.189180in}{0.738722in}}%
\pgfpathlineto{\pgfqpoint{39.441581in}{0.741702in}}%
\pgfpathlineto{\pgfqpoint{38.086125in}{0.742714in}}%
\pgfpathlineto{\pgfqpoint{37.509290in}{0.745787in}}%
\pgfpathlineto{\pgfqpoint{36.443507in}{0.746439in}}%
\pgfpathlineto{\pgfqpoint{35.710675in}{0.748438in}}%
\pgfpathlineto{\pgfqpoint{35.197928in}{0.749121in}}%
\pgfpathlineto{\pgfqpoint{33.263977in}{0.754350in}}%
\pgfpathlineto{\pgfqpoint{32.856458in}{0.754429in}}%
\pgfpathlineto{\pgfqpoint{32.530652in}{0.754694in}}%
\pgfpathlineto{\pgfqpoint{32.079890in}{0.756679in}}%
\pgfpathlineto{\pgfqpoint{31.686625in}{0.761046in}}%
\pgfpathlineto{\pgfqpoint{31.441186in}{0.770173in}}%
\pgfpathlineto{\pgfqpoint{31.345101in}{0.774582in}}%
\pgfpathlineto{\pgfqpoint{31.241077in}{0.778112in}}%
\pgfpathlineto{\pgfqpoint{30.955549in}{0.791966in}}%
\pgfpathlineto{\pgfqpoint{30.811278in}{0.795608in}}%
\pgfpathlineto{\pgfqpoint{30.770752in}{0.797703in}}%
\pgfpathlineto{\pgfqpoint{30.699417in}{0.824886in}}%
\pgfpathlineto{\pgfqpoint{29.788292in}{0.838808in}}%
\pgfpathlineto{\pgfqpoint{29.771368in}{0.844225in}}%
\pgfpathlineto{\pgfqpoint{29.661946in}{0.845702in}}%
\pgfpathlineto{\pgfqpoint{29.430699in}{0.860225in}}%
\pgfpathlineto{\pgfqpoint{29.332834in}{0.861654in}}%
\pgfpathlineto{\pgfqpoint{29.331081in}{0.870301in}}%
\pgfpathlineto{\pgfqpoint{29.271451in}{0.878344in}}%
\pgfpathlineto{\pgfqpoint{28.816157in}{0.893499in}}%
\pgfpathlineto{\pgfqpoint{28.713656in}{0.903257in}}%
\pgfpathlineto{\pgfqpoint{28.562750in}{0.906943in}}%
\pgfpathlineto{\pgfqpoint{28.499512in}{0.915098in}}%
\pgfpathlineto{\pgfqpoint{28.475931in}{0.937980in}}%
\pgfpathlineto{\pgfqpoint{28.459655in}{0.951098in}}%
\pgfpathlineto{\pgfqpoint{28.060351in}{0.961034in}}%
\pgfpathlineto{\pgfqpoint{27.624299in}{0.990614in}}%
\pgfpathlineto{\pgfqpoint{27.303010in}{1.032076in}}%
\pgfpathlineto{\pgfqpoint{27.000901in}{1.061832in}}%
\pgfpathlineto{\pgfqpoint{26.925844in}{1.078000in}}%
\pgfpathlineto{\pgfqpoint{26.918222in}{1.088630in}}%
\pgfpathlineto{\pgfqpoint{26.803818in}{1.111888in}}%
\pgfpathlineto{\pgfqpoint{26.666341in}{1.138811in}}%
\pgfpathlineto{\pgfqpoint{26.600824in}{1.148010in}}%
\pgfpathlineto{\pgfqpoint{26.536952in}{1.200594in}}%
\pgfpathlineto{\pgfqpoint{26.350128in}{1.206849in}}%
\pgfpathlineto{\pgfqpoint{26.248836in}{1.240528in}}%
\pgfpathlineto{\pgfqpoint{26.207079in}{1.379089in}}%
\pgfpathlineto{\pgfqpoint{26.089068in}{1.383921in}}%
\pgfpathlineto{\pgfqpoint{26.003683in}{1.447851in}}%
\pgfpathlineto{\pgfqpoint{25.926932in}{1.663903in}}%
\pgfpathlineto{\pgfqpoint{25.858398in}{1.705204in}}%
\pgfpathlineto{\pgfqpoint{25.730077in}{2.009091in}}%
\pgfpathlineto{\pgfqpoint{25.544004in}{2.534701in}}%
\pgfpathlineto{\pgfqpoint{25.205400in}{5.310696in}}%
\pgfpathlineto{\pgfqpoint{5.478622in}{4.808529in}}%
\pgfpathclose%
\pgfusepath{stroke,fill}%
\end{pgfscope}%
\begin{pgfscope}%
\pgfpathrectangle{\pgfqpoint{3.788192in}{2.980138in}}{\pgfqpoint{2.914000in}{2.171400in}}%
\pgfusepath{clip}%
\pgfsetbuttcap%
\pgfsetroundjoin%
\pgfsetlinewidth{1.003750pt}%
\definecolor{currentstroke}{rgb}{1.000000,0.000000,0.000000}%
\pgfsetstrokecolor{currentstroke}%
\pgfsetdash{}{0pt}%
\pgfpathmoveto{\pgfqpoint{16.861003in}{25.566105in}}%
\pgfpathcurveto{\pgfqpoint{16.869240in}{25.566105in}}{\pgfqpoint{16.877140in}{25.569377in}}{\pgfqpoint{16.882964in}{25.575201in}}%
\pgfpathcurveto{\pgfqpoint{16.888788in}{25.581025in}}{\pgfqpoint{16.892060in}{25.588925in}}{\pgfqpoint{16.892060in}{25.597161in}}%
\pgfpathcurveto{\pgfqpoint{16.892060in}{25.605397in}}{\pgfqpoint{16.888788in}{25.613297in}}{\pgfqpoint{16.882964in}{25.619121in}}%
\pgfpathcurveto{\pgfqpoint{16.877140in}{25.624945in}}{\pgfqpoint{16.869240in}{25.628218in}}{\pgfqpoint{16.861003in}{25.628218in}}%
\pgfpathcurveto{\pgfqpoint{16.852767in}{25.628218in}}{\pgfqpoint{16.844867in}{25.624945in}}{\pgfqpoint{16.839043in}{25.619121in}}%
\pgfpathcurveto{\pgfqpoint{16.833219in}{25.613297in}}{\pgfqpoint{16.829947in}{25.605397in}}{\pgfqpoint{16.829947in}{25.597161in}}%
\pgfpathcurveto{\pgfqpoint{16.829947in}{25.588925in}}{\pgfqpoint{16.833219in}{25.581025in}}{\pgfqpoint{16.839043in}{25.575201in}}%
\pgfpathcurveto{\pgfqpoint{16.844867in}{25.569377in}}{\pgfqpoint{16.852767in}{25.566105in}}{\pgfqpoint{16.861003in}{25.566105in}}%
\pgfusepath{stroke}%
\end{pgfscope}%
\begin{pgfscope}%
\pgfpathrectangle{\pgfqpoint{3.788192in}{2.980138in}}{\pgfqpoint{2.914000in}{2.171400in}}%
\pgfusepath{clip}%
\pgfsetbuttcap%
\pgfsetroundjoin%
\pgfsetlinewidth{1.003750pt}%
\definecolor{currentstroke}{rgb}{1.000000,0.000000,0.000000}%
\pgfsetstrokecolor{currentstroke}%
\pgfsetdash{}{0pt}%
\pgfpathmoveto{\pgfqpoint{11.934253in}{10.111217in}}%
\pgfpathcurveto{\pgfqpoint{11.942489in}{10.111217in}}{\pgfqpoint{11.950389in}{10.114489in}}{\pgfqpoint{11.956213in}{10.120313in}}%
\pgfpathcurveto{\pgfqpoint{11.962037in}{10.126137in}}{\pgfqpoint{11.965309in}{10.134037in}}{\pgfqpoint{11.965309in}{10.142274in}}%
\pgfpathcurveto{\pgfqpoint{11.965309in}{10.150510in}}{\pgfqpoint{11.962037in}{10.158410in}}{\pgfqpoint{11.956213in}{10.164234in}}%
\pgfpathcurveto{\pgfqpoint{11.950389in}{10.170058in}}{\pgfqpoint{11.942489in}{10.173330in}}{\pgfqpoint{11.934253in}{10.173330in}}%
\pgfpathcurveto{\pgfqpoint{11.926016in}{10.173330in}}{\pgfqpoint{11.918116in}{10.170058in}}{\pgfqpoint{11.912292in}{10.164234in}}%
\pgfpathcurveto{\pgfqpoint{11.906469in}{10.158410in}}{\pgfqpoint{11.903196in}{10.150510in}}{\pgfqpoint{11.903196in}{10.142274in}}%
\pgfpathcurveto{\pgfqpoint{11.903196in}{10.134037in}}{\pgfqpoint{11.906469in}{10.126137in}}{\pgfqpoint{11.912292in}{10.120313in}}%
\pgfpathcurveto{\pgfqpoint{11.918116in}{10.114489in}}{\pgfqpoint{11.926016in}{10.111217in}}{\pgfqpoint{11.934253in}{10.111217in}}%
\pgfusepath{stroke}%
\end{pgfscope}%
\begin{pgfscope}%
\pgfpathrectangle{\pgfqpoint{3.788192in}{2.980138in}}{\pgfqpoint{2.914000in}{2.171400in}}%
\pgfusepath{clip}%
\pgfsetbuttcap%
\pgfsetroundjoin%
\pgfsetlinewidth{1.003750pt}%
\definecolor{currentstroke}{rgb}{1.000000,0.000000,0.000000}%
\pgfsetstrokecolor{currentstroke}%
\pgfsetdash{}{0pt}%
\pgfpathmoveto{\pgfqpoint{12.691767in}{10.530438in}}%
\pgfpathcurveto{\pgfqpoint{12.700003in}{10.530438in}}{\pgfqpoint{12.707903in}{10.533710in}}{\pgfqpoint{12.713727in}{10.539534in}}%
\pgfpathcurveto{\pgfqpoint{12.719551in}{10.545358in}}{\pgfqpoint{12.722823in}{10.553258in}}{\pgfqpoint{12.722823in}{10.561494in}}%
\pgfpathcurveto{\pgfqpoint{12.722823in}{10.569730in}}{\pgfqpoint{12.719551in}{10.577630in}}{\pgfqpoint{12.713727in}{10.583454in}}%
\pgfpathcurveto{\pgfqpoint{12.707903in}{10.589278in}}{\pgfqpoint{12.700003in}{10.592551in}}{\pgfqpoint{12.691767in}{10.592551in}}%
\pgfpathcurveto{\pgfqpoint{12.683530in}{10.592551in}}{\pgfqpoint{12.675630in}{10.589278in}}{\pgfqpoint{12.669806in}{10.583454in}}%
\pgfpathcurveto{\pgfqpoint{12.663983in}{10.577630in}}{\pgfqpoint{12.660710in}{10.569730in}}{\pgfqpoint{12.660710in}{10.561494in}}%
\pgfpathcurveto{\pgfqpoint{12.660710in}{10.553258in}}{\pgfqpoint{12.663983in}{10.545358in}}{\pgfqpoint{12.669806in}{10.539534in}}%
\pgfpathcurveto{\pgfqpoint{12.675630in}{10.533710in}}{\pgfqpoint{12.683530in}{10.530438in}}{\pgfqpoint{12.691767in}{10.530438in}}%
\pgfusepath{stroke}%
\end{pgfscope}%
\begin{pgfscope}%
\pgfpathrectangle{\pgfqpoint{3.788192in}{2.980138in}}{\pgfqpoint{2.914000in}{2.171400in}}%
\pgfusepath{clip}%
\pgfsetbuttcap%
\pgfsetroundjoin%
\pgfsetlinewidth{1.003750pt}%
\definecolor{currentstroke}{rgb}{1.000000,0.000000,0.000000}%
\pgfsetstrokecolor{currentstroke}%
\pgfsetdash{}{0pt}%
\pgfpathmoveto{\pgfqpoint{15.144604in}{10.695451in}}%
\pgfpathcurveto{\pgfqpoint{15.152841in}{10.695451in}}{\pgfqpoint{15.160741in}{10.698723in}}{\pgfqpoint{15.166565in}{10.704547in}}%
\pgfpathcurveto{\pgfqpoint{15.172389in}{10.710371in}}{\pgfqpoint{15.175661in}{10.718271in}}{\pgfqpoint{15.175661in}{10.726507in}}%
\pgfpathcurveto{\pgfqpoint{15.175661in}{10.734743in}}{\pgfqpoint{15.172389in}{10.742643in}}{\pgfqpoint{15.166565in}{10.748467in}}%
\pgfpathcurveto{\pgfqpoint{15.160741in}{10.754291in}}{\pgfqpoint{15.152841in}{10.757564in}}{\pgfqpoint{15.144604in}{10.757564in}}%
\pgfpathcurveto{\pgfqpoint{15.136368in}{10.757564in}}{\pgfqpoint{15.128468in}{10.754291in}}{\pgfqpoint{15.122644in}{10.748467in}}%
\pgfpathcurveto{\pgfqpoint{15.116820in}{10.742643in}}{\pgfqpoint{15.113548in}{10.734743in}}{\pgfqpoint{15.113548in}{10.726507in}}%
\pgfpathcurveto{\pgfqpoint{15.113548in}{10.718271in}}{\pgfqpoint{15.116820in}{10.710371in}}{\pgfqpoint{15.122644in}{10.704547in}}%
\pgfpathcurveto{\pgfqpoint{15.128468in}{10.698723in}}{\pgfqpoint{15.136368in}{10.695451in}}{\pgfqpoint{15.144604in}{10.695451in}}%
\pgfusepath{stroke}%
\end{pgfscope}%
\begin{pgfscope}%
\pgfpathrectangle{\pgfqpoint{3.788192in}{2.980138in}}{\pgfqpoint{2.914000in}{2.171400in}}%
\pgfusepath{clip}%
\pgfsetbuttcap%
\pgfsetroundjoin%
\pgfsetlinewidth{1.003750pt}%
\definecolor{currentstroke}{rgb}{1.000000,0.000000,0.000000}%
\pgfsetstrokecolor{currentstroke}%
\pgfsetdash{}{0pt}%
\pgfpathmoveto{\pgfqpoint{13.542463in}{11.423191in}}%
\pgfpathcurveto{\pgfqpoint{13.550699in}{11.423191in}}{\pgfqpoint{13.558599in}{11.426464in}}{\pgfqpoint{13.564423in}{11.432288in}}%
\pgfpathcurveto{\pgfqpoint{13.570247in}{11.438112in}}{\pgfqpoint{13.573519in}{11.446012in}}{\pgfqpoint{13.573519in}{11.454248in}}%
\pgfpathcurveto{\pgfqpoint{13.573519in}{11.462484in}}{\pgfqpoint{13.570247in}{11.470384in}}{\pgfqpoint{13.564423in}{11.476208in}}%
\pgfpathcurveto{\pgfqpoint{13.558599in}{11.482032in}}{\pgfqpoint{13.550699in}{11.485304in}}{\pgfqpoint{13.542463in}{11.485304in}}%
\pgfpathcurveto{\pgfqpoint{13.534227in}{11.485304in}}{\pgfqpoint{13.526327in}{11.482032in}}{\pgfqpoint{13.520503in}{11.476208in}}%
\pgfpathcurveto{\pgfqpoint{13.514679in}{11.470384in}}{\pgfqpoint{13.511406in}{11.462484in}}{\pgfqpoint{13.511406in}{11.454248in}}%
\pgfpathcurveto{\pgfqpoint{13.511406in}{11.446012in}}{\pgfqpoint{13.514679in}{11.438112in}}{\pgfqpoint{13.520503in}{11.432288in}}%
\pgfpathcurveto{\pgfqpoint{13.526327in}{11.426464in}}{\pgfqpoint{13.534227in}{11.423191in}}{\pgfqpoint{13.542463in}{11.423191in}}%
\pgfusepath{stroke}%
\end{pgfscope}%
\begin{pgfscope}%
\pgfpathrectangle{\pgfqpoint{3.788192in}{2.980138in}}{\pgfqpoint{2.914000in}{2.171400in}}%
\pgfusepath{clip}%
\pgfsetbuttcap%
\pgfsetroundjoin%
\pgfsetlinewidth{1.003750pt}%
\definecolor{currentstroke}{rgb}{1.000000,0.000000,0.000000}%
\pgfsetstrokecolor{currentstroke}%
\pgfsetdash{}{0pt}%
\pgfpathmoveto{\pgfqpoint{12.332968in}{8.917460in}}%
\pgfpathcurveto{\pgfqpoint{12.341204in}{8.917460in}}{\pgfqpoint{12.349104in}{8.920732in}}{\pgfqpoint{12.354928in}{8.926556in}}%
\pgfpathcurveto{\pgfqpoint{12.360752in}{8.932380in}}{\pgfqpoint{12.364025in}{8.940280in}}{\pgfqpoint{12.364025in}{8.948517in}}%
\pgfpathcurveto{\pgfqpoint{12.364025in}{8.956753in}}{\pgfqpoint{12.360752in}{8.964653in}}{\pgfqpoint{12.354928in}{8.970477in}}%
\pgfpathcurveto{\pgfqpoint{12.349104in}{8.976301in}}{\pgfqpoint{12.341204in}{8.979573in}}{\pgfqpoint{12.332968in}{8.979573in}}%
\pgfpathcurveto{\pgfqpoint{12.324732in}{8.979573in}}{\pgfqpoint{12.316832in}{8.976301in}}{\pgfqpoint{12.311008in}{8.970477in}}%
\pgfpathcurveto{\pgfqpoint{12.305184in}{8.964653in}}{\pgfqpoint{12.301912in}{8.956753in}}{\pgfqpoint{12.301912in}{8.948517in}}%
\pgfpathcurveto{\pgfqpoint{12.301912in}{8.940280in}}{\pgfqpoint{12.305184in}{8.932380in}}{\pgfqpoint{12.311008in}{8.926556in}}%
\pgfpathcurveto{\pgfqpoint{12.316832in}{8.920732in}}{\pgfqpoint{12.324732in}{8.917460in}}{\pgfqpoint{12.332968in}{8.917460in}}%
\pgfusepath{stroke}%
\end{pgfscope}%
\begin{pgfscope}%
\pgfpathrectangle{\pgfqpoint{3.788192in}{2.980138in}}{\pgfqpoint{2.914000in}{2.171400in}}%
\pgfusepath{clip}%
\pgfsetbuttcap%
\pgfsetroundjoin%
\pgfsetlinewidth{1.003750pt}%
\definecolor{currentstroke}{rgb}{1.000000,0.000000,0.000000}%
\pgfsetstrokecolor{currentstroke}%
\pgfsetdash{}{0pt}%
\pgfpathmoveto{\pgfqpoint{11.399902in}{7.106716in}}%
\pgfpathcurveto{\pgfqpoint{11.408139in}{7.106716in}}{\pgfqpoint{11.416039in}{7.109988in}}{\pgfqpoint{11.421863in}{7.115812in}}%
\pgfpathcurveto{\pgfqpoint{11.427686in}{7.121636in}}{\pgfqpoint{11.430959in}{7.129536in}}{\pgfqpoint{11.430959in}{7.137773in}}%
\pgfpathcurveto{\pgfqpoint{11.430959in}{7.146009in}}{\pgfqpoint{11.427686in}{7.153909in}}{\pgfqpoint{11.421863in}{7.159733in}}%
\pgfpathcurveto{\pgfqpoint{11.416039in}{7.165557in}}{\pgfqpoint{11.408139in}{7.168829in}}{\pgfqpoint{11.399902in}{7.168829in}}%
\pgfpathcurveto{\pgfqpoint{11.391666in}{7.168829in}}{\pgfqpoint{11.383766in}{7.165557in}}{\pgfqpoint{11.377942in}{7.159733in}}%
\pgfpathcurveto{\pgfqpoint{11.372118in}{7.153909in}}{\pgfqpoint{11.368846in}{7.146009in}}{\pgfqpoint{11.368846in}{7.137773in}}%
\pgfpathcurveto{\pgfqpoint{11.368846in}{7.129536in}}{\pgfqpoint{11.372118in}{7.121636in}}{\pgfqpoint{11.377942in}{7.115812in}}%
\pgfpathcurveto{\pgfqpoint{11.383766in}{7.109988in}}{\pgfqpoint{11.391666in}{7.106716in}}{\pgfqpoint{11.399902in}{7.106716in}}%
\pgfusepath{stroke}%
\end{pgfscope}%
\begin{pgfscope}%
\pgfpathrectangle{\pgfqpoint{3.788192in}{2.980138in}}{\pgfqpoint{2.914000in}{2.171400in}}%
\pgfusepath{clip}%
\pgfsetbuttcap%
\pgfsetroundjoin%
\pgfsetlinewidth{1.003750pt}%
\definecolor{currentstroke}{rgb}{1.000000,0.000000,0.000000}%
\pgfsetstrokecolor{currentstroke}%
\pgfsetdash{}{0pt}%
\pgfpathmoveto{\pgfqpoint{11.356549in}{7.029726in}}%
\pgfpathcurveto{\pgfqpoint{11.364785in}{7.029726in}}{\pgfqpoint{11.372685in}{7.032998in}}{\pgfqpoint{11.378509in}{7.038822in}}%
\pgfpathcurveto{\pgfqpoint{11.384333in}{7.044646in}}{\pgfqpoint{11.387605in}{7.052546in}}{\pgfqpoint{11.387605in}{7.060782in}}%
\pgfpathcurveto{\pgfqpoint{11.387605in}{7.069018in}}{\pgfqpoint{11.384333in}{7.076918in}}{\pgfqpoint{11.378509in}{7.082742in}}%
\pgfpathcurveto{\pgfqpoint{11.372685in}{7.088566in}}{\pgfqpoint{11.364785in}{7.091839in}}{\pgfqpoint{11.356549in}{7.091839in}}%
\pgfpathcurveto{\pgfqpoint{11.348313in}{7.091839in}}{\pgfqpoint{11.340412in}{7.088566in}}{\pgfqpoint{11.334589in}{7.082742in}}%
\pgfpathcurveto{\pgfqpoint{11.328765in}{7.076918in}}{\pgfqpoint{11.325492in}{7.069018in}}{\pgfqpoint{11.325492in}{7.060782in}}%
\pgfpathcurveto{\pgfqpoint{11.325492in}{7.052546in}}{\pgfqpoint{11.328765in}{7.044646in}}{\pgfqpoint{11.334589in}{7.038822in}}%
\pgfpathcurveto{\pgfqpoint{11.340412in}{7.032998in}}{\pgfqpoint{11.348313in}{7.029726in}}{\pgfqpoint{11.356549in}{7.029726in}}%
\pgfusepath{stroke}%
\end{pgfscope}%
\begin{pgfscope}%
\pgfpathrectangle{\pgfqpoint{3.788192in}{2.980138in}}{\pgfqpoint{2.914000in}{2.171400in}}%
\pgfusepath{clip}%
\pgfsetbuttcap%
\pgfsetroundjoin%
\pgfsetlinewidth{1.003750pt}%
\definecolor{currentstroke}{rgb}{1.000000,0.000000,0.000000}%
\pgfsetstrokecolor{currentstroke}%
\pgfsetdash{}{0pt}%
\pgfpathmoveto{\pgfqpoint{13.232346in}{9.803683in}}%
\pgfpathcurveto{\pgfqpoint{13.240582in}{9.803683in}}{\pgfqpoint{13.248482in}{9.806956in}}{\pgfqpoint{13.254306in}{9.812780in}}%
\pgfpathcurveto{\pgfqpoint{13.260130in}{9.818604in}}{\pgfqpoint{13.263402in}{9.826504in}}{\pgfqpoint{13.263402in}{9.834740in}}%
\pgfpathcurveto{\pgfqpoint{13.263402in}{9.842976in}}{\pgfqpoint{13.260130in}{9.850876in}}{\pgfqpoint{13.254306in}{9.856700in}}%
\pgfpathcurveto{\pgfqpoint{13.248482in}{9.862524in}}{\pgfqpoint{13.240582in}{9.865796in}}{\pgfqpoint{13.232346in}{9.865796in}}%
\pgfpathcurveto{\pgfqpoint{13.224110in}{9.865796in}}{\pgfqpoint{13.216210in}{9.862524in}}{\pgfqpoint{13.210386in}{9.856700in}}%
\pgfpathcurveto{\pgfqpoint{13.204562in}{9.850876in}}{\pgfqpoint{13.201289in}{9.842976in}}{\pgfqpoint{13.201289in}{9.834740in}}%
\pgfpathcurveto{\pgfqpoint{13.201289in}{9.826504in}}{\pgfqpoint{13.204562in}{9.818604in}}{\pgfqpoint{13.210386in}{9.812780in}}%
\pgfpathcurveto{\pgfqpoint{13.216210in}{9.806956in}}{\pgfqpoint{13.224110in}{9.803683in}}{\pgfqpoint{13.232346in}{9.803683in}}%
\pgfusepath{stroke}%
\end{pgfscope}%
\begin{pgfscope}%
\pgfpathrectangle{\pgfqpoint{3.788192in}{2.980138in}}{\pgfqpoint{2.914000in}{2.171400in}}%
\pgfusepath{clip}%
\pgfsetbuttcap%
\pgfsetroundjoin%
\pgfsetlinewidth{1.003750pt}%
\definecolor{currentstroke}{rgb}{1.000000,0.000000,0.000000}%
\pgfsetstrokecolor{currentstroke}%
\pgfsetdash{}{0pt}%
\pgfpathmoveto{\pgfqpoint{12.940652in}{8.318719in}}%
\pgfpathcurveto{\pgfqpoint{12.948888in}{8.318719in}}{\pgfqpoint{12.956788in}{8.321992in}}{\pgfqpoint{12.962612in}{8.327816in}}%
\pgfpathcurveto{\pgfqpoint{12.968436in}{8.333640in}}{\pgfqpoint{12.971708in}{8.341540in}}{\pgfqpoint{12.971708in}{8.349776in}}%
\pgfpathcurveto{\pgfqpoint{12.971708in}{8.358012in}}{\pgfqpoint{12.968436in}{8.365912in}}{\pgfqpoint{12.962612in}{8.371736in}}%
\pgfpathcurveto{\pgfqpoint{12.956788in}{8.377560in}}{\pgfqpoint{12.948888in}{8.380832in}}{\pgfqpoint{12.940652in}{8.380832in}}%
\pgfpathcurveto{\pgfqpoint{12.932415in}{8.380832in}}{\pgfqpoint{12.924515in}{8.377560in}}{\pgfqpoint{12.918691in}{8.371736in}}%
\pgfpathcurveto{\pgfqpoint{12.912867in}{8.365912in}}{\pgfqpoint{12.909595in}{8.358012in}}{\pgfqpoint{12.909595in}{8.349776in}}%
\pgfpathcurveto{\pgfqpoint{12.909595in}{8.341540in}}{\pgfqpoint{12.912867in}{8.333640in}}{\pgfqpoint{12.918691in}{8.327816in}}%
\pgfpathcurveto{\pgfqpoint{12.924515in}{8.321992in}}{\pgfqpoint{12.932415in}{8.318719in}}{\pgfqpoint{12.940652in}{8.318719in}}%
\pgfusepath{stroke}%
\end{pgfscope}%
\begin{pgfscope}%
\pgfpathrectangle{\pgfqpoint{3.788192in}{2.980138in}}{\pgfqpoint{2.914000in}{2.171400in}}%
\pgfusepath{clip}%
\pgfsetbuttcap%
\pgfsetroundjoin%
\pgfsetlinewidth{1.003750pt}%
\definecolor{currentstroke}{rgb}{1.000000,0.000000,0.000000}%
\pgfsetstrokecolor{currentstroke}%
\pgfsetdash{}{0pt}%
\pgfpathmoveto{\pgfqpoint{12.936873in}{8.298012in}}%
\pgfpathcurveto{\pgfqpoint{12.945109in}{8.298012in}}{\pgfqpoint{12.953009in}{8.301284in}}{\pgfqpoint{12.958833in}{8.307108in}}%
\pgfpathcurveto{\pgfqpoint{12.964657in}{8.312932in}}{\pgfqpoint{12.967929in}{8.320832in}}{\pgfqpoint{12.967929in}{8.329068in}}%
\pgfpathcurveto{\pgfqpoint{12.967929in}{8.337305in}}{\pgfqpoint{12.964657in}{8.345205in}}{\pgfqpoint{12.958833in}{8.351028in}}%
\pgfpathcurveto{\pgfqpoint{12.953009in}{8.356852in}}{\pgfqpoint{12.945109in}{8.360125in}}{\pgfqpoint{12.936873in}{8.360125in}}%
\pgfpathcurveto{\pgfqpoint{12.928637in}{8.360125in}}{\pgfqpoint{12.920737in}{8.356852in}}{\pgfqpoint{12.914913in}{8.351028in}}%
\pgfpathcurveto{\pgfqpoint{12.909089in}{8.345205in}}{\pgfqpoint{12.905816in}{8.337305in}}{\pgfqpoint{12.905816in}{8.329068in}}%
\pgfpathcurveto{\pgfqpoint{12.905816in}{8.320832in}}{\pgfqpoint{12.909089in}{8.312932in}}{\pgfqpoint{12.914913in}{8.307108in}}%
\pgfpathcurveto{\pgfqpoint{12.920737in}{8.301284in}}{\pgfqpoint{12.928637in}{8.298012in}}{\pgfqpoint{12.936873in}{8.298012in}}%
\pgfusepath{stroke}%
\end{pgfscope}%
\begin{pgfscope}%
\pgfpathrectangle{\pgfqpoint{3.788192in}{2.980138in}}{\pgfqpoint{2.914000in}{2.171400in}}%
\pgfusepath{clip}%
\pgfsetbuttcap%
\pgfsetroundjoin%
\pgfsetlinewidth{1.003750pt}%
\definecolor{currentstroke}{rgb}{1.000000,0.000000,0.000000}%
\pgfsetstrokecolor{currentstroke}%
\pgfsetdash{}{0pt}%
\pgfpathmoveto{\pgfqpoint{12.900060in}{6.352602in}}%
\pgfpathcurveto{\pgfqpoint{12.908296in}{6.352602in}}{\pgfqpoint{12.916196in}{6.355874in}}{\pgfqpoint{12.922020in}{6.361698in}}%
\pgfpathcurveto{\pgfqpoint{12.927844in}{6.367522in}}{\pgfqpoint{12.931116in}{6.375422in}}{\pgfqpoint{12.931116in}{6.383658in}}%
\pgfpathcurveto{\pgfqpoint{12.931116in}{6.391894in}}{\pgfqpoint{12.927844in}{6.399794in}}{\pgfqpoint{12.922020in}{6.405618in}}%
\pgfpathcurveto{\pgfqpoint{12.916196in}{6.411442in}}{\pgfqpoint{12.908296in}{6.414715in}}{\pgfqpoint{12.900060in}{6.414715in}}%
\pgfpathcurveto{\pgfqpoint{12.891823in}{6.414715in}}{\pgfqpoint{12.883923in}{6.411442in}}{\pgfqpoint{12.878099in}{6.405618in}}%
\pgfpathcurveto{\pgfqpoint{12.872275in}{6.399794in}}{\pgfqpoint{12.869003in}{6.391894in}}{\pgfqpoint{12.869003in}{6.383658in}}%
\pgfpathcurveto{\pgfqpoint{12.869003in}{6.375422in}}{\pgfqpoint{12.872275in}{6.367522in}}{\pgfqpoint{12.878099in}{6.361698in}}%
\pgfpathcurveto{\pgfqpoint{12.883923in}{6.355874in}}{\pgfqpoint{12.891823in}{6.352602in}}{\pgfqpoint{12.900060in}{6.352602in}}%
\pgfusepath{stroke}%
\end{pgfscope}%
\begin{pgfscope}%
\pgfpathrectangle{\pgfqpoint{3.788192in}{2.980138in}}{\pgfqpoint{2.914000in}{2.171400in}}%
\pgfusepath{clip}%
\pgfsetbuttcap%
\pgfsetroundjoin%
\pgfsetlinewidth{1.003750pt}%
\definecolor{currentstroke}{rgb}{1.000000,0.000000,0.000000}%
\pgfsetstrokecolor{currentstroke}%
\pgfsetdash{}{0pt}%
\pgfpathmoveto{\pgfqpoint{12.861075in}{6.367856in}}%
\pgfpathcurveto{\pgfqpoint{12.869311in}{6.367856in}}{\pgfqpoint{12.877211in}{6.371128in}}{\pgfqpoint{12.883035in}{6.376952in}}%
\pgfpathcurveto{\pgfqpoint{12.888859in}{6.382776in}}{\pgfqpoint{12.892131in}{6.390676in}}{\pgfqpoint{12.892131in}{6.398913in}}%
\pgfpathcurveto{\pgfqpoint{12.892131in}{6.407149in}}{\pgfqpoint{12.888859in}{6.415049in}}{\pgfqpoint{12.883035in}{6.420873in}}%
\pgfpathcurveto{\pgfqpoint{12.877211in}{6.426697in}}{\pgfqpoint{12.869311in}{6.429969in}}{\pgfqpoint{12.861075in}{6.429969in}}%
\pgfpathcurveto{\pgfqpoint{12.852838in}{6.429969in}}{\pgfqpoint{12.844938in}{6.426697in}}{\pgfqpoint{12.839114in}{6.420873in}}%
\pgfpathcurveto{\pgfqpoint{12.833291in}{6.415049in}}{\pgfqpoint{12.830018in}{6.407149in}}{\pgfqpoint{12.830018in}{6.398913in}}%
\pgfpathcurveto{\pgfqpoint{12.830018in}{6.390676in}}{\pgfqpoint{12.833291in}{6.382776in}}{\pgfqpoint{12.839114in}{6.376952in}}%
\pgfpathcurveto{\pgfqpoint{12.844938in}{6.371128in}}{\pgfqpoint{12.852838in}{6.367856in}}{\pgfqpoint{12.861075in}{6.367856in}}%
\pgfusepath{stroke}%
\end{pgfscope}%
\begin{pgfscope}%
\pgfpathrectangle{\pgfqpoint{3.788192in}{2.980138in}}{\pgfqpoint{2.914000in}{2.171400in}}%
\pgfusepath{clip}%
\pgfsetbuttcap%
\pgfsetroundjoin%
\pgfsetlinewidth{1.003750pt}%
\definecolor{currentstroke}{rgb}{1.000000,0.000000,0.000000}%
\pgfsetstrokecolor{currentstroke}%
\pgfsetdash{}{0pt}%
\pgfpathmoveto{\pgfqpoint{15.347266in}{8.054762in}}%
\pgfpathcurveto{\pgfqpoint{15.355502in}{8.054762in}}{\pgfqpoint{15.363402in}{8.058034in}}{\pgfqpoint{15.369226in}{8.063858in}}%
\pgfpathcurveto{\pgfqpoint{15.375050in}{8.069682in}}{\pgfqpoint{15.378323in}{8.077582in}}{\pgfqpoint{15.378323in}{8.085818in}}%
\pgfpathcurveto{\pgfqpoint{15.378323in}{8.094054in}}{\pgfqpoint{15.375050in}{8.101954in}}{\pgfqpoint{15.369226in}{8.107778in}}%
\pgfpathcurveto{\pgfqpoint{15.363402in}{8.113602in}}{\pgfqpoint{15.355502in}{8.116875in}}{\pgfqpoint{15.347266in}{8.116875in}}%
\pgfpathcurveto{\pgfqpoint{15.339030in}{8.116875in}}{\pgfqpoint{15.331130in}{8.113602in}}{\pgfqpoint{15.325306in}{8.107778in}}%
\pgfpathcurveto{\pgfqpoint{15.319482in}{8.101954in}}{\pgfqpoint{15.316210in}{8.094054in}}{\pgfqpoint{15.316210in}{8.085818in}}%
\pgfpathcurveto{\pgfqpoint{15.316210in}{8.077582in}}{\pgfqpoint{15.319482in}{8.069682in}}{\pgfqpoint{15.325306in}{8.063858in}}%
\pgfpathcurveto{\pgfqpoint{15.331130in}{8.058034in}}{\pgfqpoint{15.339030in}{8.054762in}}{\pgfqpoint{15.347266in}{8.054762in}}%
\pgfusepath{stroke}%
\end{pgfscope}%
\begin{pgfscope}%
\pgfpathrectangle{\pgfqpoint{3.788192in}{2.980138in}}{\pgfqpoint{2.914000in}{2.171400in}}%
\pgfusepath{clip}%
\pgfsetbuttcap%
\pgfsetroundjoin%
\pgfsetlinewidth{1.003750pt}%
\definecolor{currentstroke}{rgb}{1.000000,0.000000,0.000000}%
\pgfsetstrokecolor{currentstroke}%
\pgfsetdash{}{0pt}%
\pgfpathmoveto{\pgfqpoint{12.945472in}{6.264272in}}%
\pgfpathcurveto{\pgfqpoint{12.953708in}{6.264272in}}{\pgfqpoint{12.961608in}{6.267545in}}{\pgfqpoint{12.967432in}{6.273369in}}%
\pgfpathcurveto{\pgfqpoint{12.973256in}{6.279193in}}{\pgfqpoint{12.976528in}{6.287093in}}{\pgfqpoint{12.976528in}{6.295329in}}%
\pgfpathcurveto{\pgfqpoint{12.976528in}{6.303565in}}{\pgfqpoint{12.973256in}{6.311465in}}{\pgfqpoint{12.967432in}{6.317289in}}%
\pgfpathcurveto{\pgfqpoint{12.961608in}{6.323113in}}{\pgfqpoint{12.953708in}{6.326385in}}{\pgfqpoint{12.945472in}{6.326385in}}%
\pgfpathcurveto{\pgfqpoint{12.937236in}{6.326385in}}{\pgfqpoint{12.929336in}{6.323113in}}{\pgfqpoint{12.923512in}{6.317289in}}%
\pgfpathcurveto{\pgfqpoint{12.917688in}{6.311465in}}{\pgfqpoint{12.914415in}{6.303565in}}{\pgfqpoint{12.914415in}{6.295329in}}%
\pgfpathcurveto{\pgfqpoint{12.914415in}{6.287093in}}{\pgfqpoint{12.917688in}{6.279193in}}{\pgfqpoint{12.923512in}{6.273369in}}%
\pgfpathcurveto{\pgfqpoint{12.929336in}{6.267545in}}{\pgfqpoint{12.937236in}{6.264272in}}{\pgfqpoint{12.945472in}{6.264272in}}%
\pgfusepath{stroke}%
\end{pgfscope}%
\begin{pgfscope}%
\pgfpathrectangle{\pgfqpoint{3.788192in}{2.980138in}}{\pgfqpoint{2.914000in}{2.171400in}}%
\pgfusepath{clip}%
\pgfsetbuttcap%
\pgfsetroundjoin%
\pgfsetlinewidth{1.003750pt}%
\definecolor{currentstroke}{rgb}{1.000000,0.000000,0.000000}%
\pgfsetstrokecolor{currentstroke}%
\pgfsetdash{}{0pt}%
\pgfpathmoveto{\pgfqpoint{14.816780in}{6.837016in}}%
\pgfpathcurveto{\pgfqpoint{14.825016in}{6.837016in}}{\pgfqpoint{14.832916in}{6.840288in}}{\pgfqpoint{14.838740in}{6.846112in}}%
\pgfpathcurveto{\pgfqpoint{14.844564in}{6.851936in}}{\pgfqpoint{14.847836in}{6.859836in}}{\pgfqpoint{14.847836in}{6.868073in}}%
\pgfpathcurveto{\pgfqpoint{14.847836in}{6.876309in}}{\pgfqpoint{14.844564in}{6.884209in}}{\pgfqpoint{14.838740in}{6.890033in}}%
\pgfpathcurveto{\pgfqpoint{14.832916in}{6.895857in}}{\pgfqpoint{14.825016in}{6.899129in}}{\pgfqpoint{14.816780in}{6.899129in}}%
\pgfpathcurveto{\pgfqpoint{14.808543in}{6.899129in}}{\pgfqpoint{14.800643in}{6.895857in}}{\pgfqpoint{14.794819in}{6.890033in}}%
\pgfpathcurveto{\pgfqpoint{14.788995in}{6.884209in}}{\pgfqpoint{14.785723in}{6.876309in}}{\pgfqpoint{14.785723in}{6.868073in}}%
\pgfpathcurveto{\pgfqpoint{14.785723in}{6.859836in}}{\pgfqpoint{14.788995in}{6.851936in}}{\pgfqpoint{14.794819in}{6.846112in}}%
\pgfpathcurveto{\pgfqpoint{14.800643in}{6.840288in}}{\pgfqpoint{14.808543in}{6.837016in}}{\pgfqpoint{14.816780in}{6.837016in}}%
\pgfusepath{stroke}%
\end{pgfscope}%
\begin{pgfscope}%
\pgfpathrectangle{\pgfqpoint{3.788192in}{2.980138in}}{\pgfqpoint{2.914000in}{2.171400in}}%
\pgfusepath{clip}%
\pgfsetbuttcap%
\pgfsetroundjoin%
\pgfsetlinewidth{1.003750pt}%
\definecolor{currentstroke}{rgb}{1.000000,0.000000,0.000000}%
\pgfsetstrokecolor{currentstroke}%
\pgfsetdash{}{0pt}%
\pgfpathmoveto{\pgfqpoint{15.389681in}{6.512387in}}%
\pgfpathcurveto{\pgfqpoint{15.397917in}{6.512387in}}{\pgfqpoint{15.405817in}{6.515659in}}{\pgfqpoint{15.411641in}{6.521483in}}%
\pgfpathcurveto{\pgfqpoint{15.417465in}{6.527307in}}{\pgfqpoint{15.420738in}{6.535207in}}{\pgfqpoint{15.420738in}{6.543443in}}%
\pgfpathcurveto{\pgfqpoint{15.420738in}{6.551679in}}{\pgfqpoint{15.417465in}{6.559579in}}{\pgfqpoint{15.411641in}{6.565403in}}%
\pgfpathcurveto{\pgfqpoint{15.405817in}{6.571227in}}{\pgfqpoint{15.397917in}{6.574500in}}{\pgfqpoint{15.389681in}{6.574500in}}%
\pgfpathcurveto{\pgfqpoint{15.381445in}{6.574500in}}{\pgfqpoint{15.373545in}{6.571227in}}{\pgfqpoint{15.367721in}{6.565403in}}%
\pgfpathcurveto{\pgfqpoint{15.361897in}{6.559579in}}{\pgfqpoint{15.358625in}{6.551679in}}{\pgfqpoint{15.358625in}{6.543443in}}%
\pgfpathcurveto{\pgfqpoint{15.358625in}{6.535207in}}{\pgfqpoint{15.361897in}{6.527307in}}{\pgfqpoint{15.367721in}{6.521483in}}%
\pgfpathcurveto{\pgfqpoint{15.373545in}{6.515659in}}{\pgfqpoint{15.381445in}{6.512387in}}{\pgfqpoint{15.389681in}{6.512387in}}%
\pgfusepath{stroke}%
\end{pgfscope}%
\begin{pgfscope}%
\pgfpathrectangle{\pgfqpoint{3.788192in}{2.980138in}}{\pgfqpoint{2.914000in}{2.171400in}}%
\pgfusepath{clip}%
\pgfsetbuttcap%
\pgfsetroundjoin%
\pgfsetlinewidth{1.003750pt}%
\definecolor{currentstroke}{rgb}{1.000000,0.000000,0.000000}%
\pgfsetstrokecolor{currentstroke}%
\pgfsetdash{}{0pt}%
\pgfpathmoveto{\pgfqpoint{12.707501in}{7.095383in}}%
\pgfpathcurveto{\pgfqpoint{12.715737in}{7.095383in}}{\pgfqpoint{12.723637in}{7.098656in}}{\pgfqpoint{12.729461in}{7.104479in}}%
\pgfpathcurveto{\pgfqpoint{12.735285in}{7.110303in}}{\pgfqpoint{12.738557in}{7.118203in}}{\pgfqpoint{12.738557in}{7.126440in}}%
\pgfpathcurveto{\pgfqpoint{12.738557in}{7.134676in}}{\pgfqpoint{12.735285in}{7.142576in}}{\pgfqpoint{12.729461in}{7.148400in}}%
\pgfpathcurveto{\pgfqpoint{12.723637in}{7.154224in}}{\pgfqpoint{12.715737in}{7.157496in}}{\pgfqpoint{12.707501in}{7.157496in}}%
\pgfpathcurveto{\pgfqpoint{12.699265in}{7.157496in}}{\pgfqpoint{12.691365in}{7.154224in}}{\pgfqpoint{12.685541in}{7.148400in}}%
\pgfpathcurveto{\pgfqpoint{12.679717in}{7.142576in}}{\pgfqpoint{12.676444in}{7.134676in}}{\pgfqpoint{12.676444in}{7.126440in}}%
\pgfpathcurveto{\pgfqpoint{12.676444in}{7.118203in}}{\pgfqpoint{12.679717in}{7.110303in}}{\pgfqpoint{12.685541in}{7.104479in}}%
\pgfpathcurveto{\pgfqpoint{12.691365in}{7.098656in}}{\pgfqpoint{12.699265in}{7.095383in}}{\pgfqpoint{12.707501in}{7.095383in}}%
\pgfusepath{stroke}%
\end{pgfscope}%
\begin{pgfscope}%
\pgfpathrectangle{\pgfqpoint{3.788192in}{2.980138in}}{\pgfqpoint{2.914000in}{2.171400in}}%
\pgfusepath{clip}%
\pgfsetbuttcap%
\pgfsetroundjoin%
\pgfsetlinewidth{1.003750pt}%
\definecolor{currentstroke}{rgb}{1.000000,0.000000,0.000000}%
\pgfsetstrokecolor{currentstroke}%
\pgfsetdash{}{0pt}%
\pgfpathmoveto{\pgfqpoint{14.362942in}{5.990566in}}%
\pgfpathcurveto{\pgfqpoint{14.371178in}{5.990566in}}{\pgfqpoint{14.379078in}{5.993838in}}{\pgfqpoint{14.384902in}{5.999662in}}%
\pgfpathcurveto{\pgfqpoint{14.390726in}{6.005486in}}{\pgfqpoint{14.393998in}{6.013386in}}{\pgfqpoint{14.393998in}{6.021622in}}%
\pgfpathcurveto{\pgfqpoint{14.393998in}{6.029859in}}{\pgfqpoint{14.390726in}{6.037759in}}{\pgfqpoint{14.384902in}{6.043583in}}%
\pgfpathcurveto{\pgfqpoint{14.379078in}{6.049407in}}{\pgfqpoint{14.371178in}{6.052679in}}{\pgfqpoint{14.362942in}{6.052679in}}%
\pgfpathcurveto{\pgfqpoint{14.354705in}{6.052679in}}{\pgfqpoint{14.346805in}{6.049407in}}{\pgfqpoint{14.340982in}{6.043583in}}%
\pgfpathcurveto{\pgfqpoint{14.335158in}{6.037759in}}{\pgfqpoint{14.331885in}{6.029859in}}{\pgfqpoint{14.331885in}{6.021622in}}%
\pgfpathcurveto{\pgfqpoint{14.331885in}{6.013386in}}{\pgfqpoint{14.335158in}{6.005486in}}{\pgfqpoint{14.340982in}{5.999662in}}%
\pgfpathcurveto{\pgfqpoint{14.346805in}{5.993838in}}{\pgfqpoint{14.354705in}{5.990566in}}{\pgfqpoint{14.362942in}{5.990566in}}%
\pgfusepath{stroke}%
\end{pgfscope}%
\begin{pgfscope}%
\pgfpathrectangle{\pgfqpoint{3.788192in}{2.980138in}}{\pgfqpoint{2.914000in}{2.171400in}}%
\pgfusepath{clip}%
\pgfsetbuttcap%
\pgfsetroundjoin%
\pgfsetlinewidth{1.003750pt}%
\definecolor{currentstroke}{rgb}{1.000000,0.000000,0.000000}%
\pgfsetstrokecolor{currentstroke}%
\pgfsetdash{}{0pt}%
\pgfpathmoveto{\pgfqpoint{14.985061in}{5.541293in}}%
\pgfpathcurveto{\pgfqpoint{14.993297in}{5.541293in}}{\pgfqpoint{15.001197in}{5.544566in}}{\pgfqpoint{15.007021in}{5.550390in}}%
\pgfpathcurveto{\pgfqpoint{15.012845in}{5.556213in}}{\pgfqpoint{15.016118in}{5.564113in}}{\pgfqpoint{15.016118in}{5.572350in}}%
\pgfpathcurveto{\pgfqpoint{15.016118in}{5.580586in}}{\pgfqpoint{15.012845in}{5.588486in}}{\pgfqpoint{15.007021in}{5.594310in}}%
\pgfpathcurveto{\pgfqpoint{15.001197in}{5.600134in}}{\pgfqpoint{14.993297in}{5.603406in}}{\pgfqpoint{14.985061in}{5.603406in}}%
\pgfpathcurveto{\pgfqpoint{14.976825in}{5.603406in}}{\pgfqpoint{14.968925in}{5.600134in}}{\pgfqpoint{14.963101in}{5.594310in}}%
\pgfpathcurveto{\pgfqpoint{14.957277in}{5.588486in}}{\pgfqpoint{14.954005in}{5.580586in}}{\pgfqpoint{14.954005in}{5.572350in}}%
\pgfpathcurveto{\pgfqpoint{14.954005in}{5.564113in}}{\pgfqpoint{14.957277in}{5.556213in}}{\pgfqpoint{14.963101in}{5.550390in}}%
\pgfpathcurveto{\pgfqpoint{14.968925in}{5.544566in}}{\pgfqpoint{14.976825in}{5.541293in}}{\pgfqpoint{14.985061in}{5.541293in}}%
\pgfusepath{stroke}%
\end{pgfscope}%
\begin{pgfscope}%
\pgfpathrectangle{\pgfqpoint{3.788192in}{2.980138in}}{\pgfqpoint{2.914000in}{2.171400in}}%
\pgfusepath{clip}%
\pgfsetbuttcap%
\pgfsetroundjoin%
\pgfsetlinewidth{1.003750pt}%
\definecolor{currentstroke}{rgb}{1.000000,0.000000,0.000000}%
\pgfsetstrokecolor{currentstroke}%
\pgfsetdash{}{0pt}%
\pgfpathmoveto{\pgfqpoint{14.968190in}{5.733414in}}%
\pgfpathcurveto{\pgfqpoint{14.976426in}{5.733414in}}{\pgfqpoint{14.984326in}{5.736686in}}{\pgfqpoint{14.990150in}{5.742510in}}%
\pgfpathcurveto{\pgfqpoint{14.995974in}{5.748334in}}{\pgfqpoint{14.999246in}{5.756234in}}{\pgfqpoint{14.999246in}{5.764470in}}%
\pgfpathcurveto{\pgfqpoint{14.999246in}{5.772706in}}{\pgfqpoint{14.995974in}{5.780606in}}{\pgfqpoint{14.990150in}{5.786430in}}%
\pgfpathcurveto{\pgfqpoint{14.984326in}{5.792254in}}{\pgfqpoint{14.976426in}{5.795527in}}{\pgfqpoint{14.968190in}{5.795527in}}%
\pgfpathcurveto{\pgfqpoint{14.959954in}{5.795527in}}{\pgfqpoint{14.952053in}{5.792254in}}{\pgfqpoint{14.946230in}{5.786430in}}%
\pgfpathcurveto{\pgfqpoint{14.940406in}{5.780606in}}{\pgfqpoint{14.937133in}{5.772706in}}{\pgfqpoint{14.937133in}{5.764470in}}%
\pgfpathcurveto{\pgfqpoint{14.937133in}{5.756234in}}{\pgfqpoint{14.940406in}{5.748334in}}{\pgfqpoint{14.946230in}{5.742510in}}%
\pgfpathcurveto{\pgfqpoint{14.952053in}{5.736686in}}{\pgfqpoint{14.959954in}{5.733414in}}{\pgfqpoint{14.968190in}{5.733414in}}%
\pgfusepath{stroke}%
\end{pgfscope}%
\begin{pgfscope}%
\pgfpathrectangle{\pgfqpoint{3.788192in}{2.980138in}}{\pgfqpoint{2.914000in}{2.171400in}}%
\pgfusepath{clip}%
\pgfsetbuttcap%
\pgfsetroundjoin%
\pgfsetlinewidth{1.003750pt}%
\definecolor{currentstroke}{rgb}{1.000000,0.000000,0.000000}%
\pgfsetstrokecolor{currentstroke}%
\pgfsetdash{}{0pt}%
\pgfpathmoveto{\pgfqpoint{12.713917in}{7.026179in}}%
\pgfpathcurveto{\pgfqpoint{12.722154in}{7.026179in}}{\pgfqpoint{12.730054in}{7.029451in}}{\pgfqpoint{12.735878in}{7.035275in}}%
\pgfpathcurveto{\pgfqpoint{12.741701in}{7.041099in}}{\pgfqpoint{12.744974in}{7.048999in}}{\pgfqpoint{12.744974in}{7.057235in}}%
\pgfpathcurveto{\pgfqpoint{12.744974in}{7.065472in}}{\pgfqpoint{12.741701in}{7.073372in}}{\pgfqpoint{12.735878in}{7.079196in}}%
\pgfpathcurveto{\pgfqpoint{12.730054in}{7.085019in}}{\pgfqpoint{12.722154in}{7.088292in}}{\pgfqpoint{12.713917in}{7.088292in}}%
\pgfpathcurveto{\pgfqpoint{12.705681in}{7.088292in}}{\pgfqpoint{12.697781in}{7.085019in}}{\pgfqpoint{12.691957in}{7.079196in}}%
\pgfpathcurveto{\pgfqpoint{12.686133in}{7.073372in}}{\pgfqpoint{12.682861in}{7.065472in}}{\pgfqpoint{12.682861in}{7.057235in}}%
\pgfpathcurveto{\pgfqpoint{12.682861in}{7.048999in}}{\pgfqpoint{12.686133in}{7.041099in}}{\pgfqpoint{12.691957in}{7.035275in}}%
\pgfpathcurveto{\pgfqpoint{12.697781in}{7.029451in}}{\pgfqpoint{12.705681in}{7.026179in}}{\pgfqpoint{12.713917in}{7.026179in}}%
\pgfusepath{stroke}%
\end{pgfscope}%
\begin{pgfscope}%
\pgfpathrectangle{\pgfqpoint{3.788192in}{2.980138in}}{\pgfqpoint{2.914000in}{2.171400in}}%
\pgfusepath{clip}%
\pgfsetbuttcap%
\pgfsetroundjoin%
\pgfsetlinewidth{1.003750pt}%
\definecolor{currentstroke}{rgb}{1.000000,0.000000,0.000000}%
\pgfsetstrokecolor{currentstroke}%
\pgfsetdash{}{0pt}%
\pgfpathmoveto{\pgfqpoint{13.558784in}{8.054051in}}%
\pgfpathcurveto{\pgfqpoint{13.567020in}{8.054051in}}{\pgfqpoint{13.574920in}{8.057323in}}{\pgfqpoint{13.580744in}{8.063147in}}%
\pgfpathcurveto{\pgfqpoint{13.586568in}{8.068971in}}{\pgfqpoint{13.589840in}{8.076871in}}{\pgfqpoint{13.589840in}{8.085108in}}%
\pgfpathcurveto{\pgfqpoint{13.589840in}{8.093344in}}{\pgfqpoint{13.586568in}{8.101244in}}{\pgfqpoint{13.580744in}{8.107068in}}%
\pgfpathcurveto{\pgfqpoint{13.574920in}{8.112892in}}{\pgfqpoint{13.567020in}{8.116164in}}{\pgfqpoint{13.558784in}{8.116164in}}%
\pgfpathcurveto{\pgfqpoint{13.550548in}{8.116164in}}{\pgfqpoint{13.542648in}{8.112892in}}{\pgfqpoint{13.536824in}{8.107068in}}%
\pgfpathcurveto{\pgfqpoint{13.531000in}{8.101244in}}{\pgfqpoint{13.527727in}{8.093344in}}{\pgfqpoint{13.527727in}{8.085108in}}%
\pgfpathcurveto{\pgfqpoint{13.527727in}{8.076871in}}{\pgfqpoint{13.531000in}{8.068971in}}{\pgfqpoint{13.536824in}{8.063147in}}%
\pgfpathcurveto{\pgfqpoint{13.542648in}{8.057323in}}{\pgfqpoint{13.550548in}{8.054051in}}{\pgfqpoint{13.558784in}{8.054051in}}%
\pgfusepath{stroke}%
\end{pgfscope}%
\begin{pgfscope}%
\pgfpathrectangle{\pgfqpoint{3.788192in}{2.980138in}}{\pgfqpoint{2.914000in}{2.171400in}}%
\pgfusepath{clip}%
\pgfsetbuttcap%
\pgfsetroundjoin%
\pgfsetlinewidth{1.003750pt}%
\definecolor{currentstroke}{rgb}{1.000000,0.000000,0.000000}%
\pgfsetstrokecolor{currentstroke}%
\pgfsetdash{}{0pt}%
\pgfpathmoveto{\pgfqpoint{13.021061in}{8.119261in}}%
\pgfpathcurveto{\pgfqpoint{13.029297in}{8.119261in}}{\pgfqpoint{13.037197in}{8.122533in}}{\pgfqpoint{13.043021in}{8.128357in}}%
\pgfpathcurveto{\pgfqpoint{13.048845in}{8.134181in}}{\pgfqpoint{13.052117in}{8.142081in}}{\pgfqpoint{13.052117in}{8.150317in}}%
\pgfpathcurveto{\pgfqpoint{13.052117in}{8.158554in}}{\pgfqpoint{13.048845in}{8.166454in}}{\pgfqpoint{13.043021in}{8.172278in}}%
\pgfpathcurveto{\pgfqpoint{13.037197in}{8.178102in}}{\pgfqpoint{13.029297in}{8.181374in}}{\pgfqpoint{13.021061in}{8.181374in}}%
\pgfpathcurveto{\pgfqpoint{13.012825in}{8.181374in}}{\pgfqpoint{13.004925in}{8.178102in}}{\pgfqpoint{12.999101in}{8.172278in}}%
\pgfpathcurveto{\pgfqpoint{12.993277in}{8.166454in}}{\pgfqpoint{12.990004in}{8.158554in}}{\pgfqpoint{12.990004in}{8.150317in}}%
\pgfpathcurveto{\pgfqpoint{12.990004in}{8.142081in}}{\pgfqpoint{12.993277in}{8.134181in}}{\pgfqpoint{12.999101in}{8.128357in}}%
\pgfpathcurveto{\pgfqpoint{13.004925in}{8.122533in}}{\pgfqpoint{13.012825in}{8.119261in}}{\pgfqpoint{13.021061in}{8.119261in}}%
\pgfusepath{stroke}%
\end{pgfscope}%
\begin{pgfscope}%
\pgfpathrectangle{\pgfqpoint{3.788192in}{2.980138in}}{\pgfqpoint{2.914000in}{2.171400in}}%
\pgfusepath{clip}%
\pgfsetbuttcap%
\pgfsetroundjoin%
\pgfsetlinewidth{1.003750pt}%
\definecolor{currentstroke}{rgb}{1.000000,0.000000,0.000000}%
\pgfsetstrokecolor{currentstroke}%
\pgfsetdash{}{0pt}%
\pgfpathmoveto{\pgfqpoint{12.649080in}{7.078889in}}%
\pgfpathcurveto{\pgfqpoint{12.657316in}{7.078889in}}{\pgfqpoint{12.665216in}{7.082161in}}{\pgfqpoint{12.671040in}{7.087985in}}%
\pgfpathcurveto{\pgfqpoint{12.676864in}{7.093809in}}{\pgfqpoint{12.680136in}{7.101709in}}{\pgfqpoint{12.680136in}{7.109946in}}%
\pgfpathcurveto{\pgfqpoint{12.680136in}{7.118182in}}{\pgfqpoint{12.676864in}{7.126082in}}{\pgfqpoint{12.671040in}{7.131906in}}%
\pgfpathcurveto{\pgfqpoint{12.665216in}{7.137730in}}{\pgfqpoint{12.657316in}{7.141002in}}{\pgfqpoint{12.649080in}{7.141002in}}%
\pgfpathcurveto{\pgfqpoint{12.640843in}{7.141002in}}{\pgfqpoint{12.632943in}{7.137730in}}{\pgfqpoint{12.627119in}{7.131906in}}%
\pgfpathcurveto{\pgfqpoint{12.621295in}{7.126082in}}{\pgfqpoint{12.618023in}{7.118182in}}{\pgfqpoint{12.618023in}{7.109946in}}%
\pgfpathcurveto{\pgfqpoint{12.618023in}{7.101709in}}{\pgfqpoint{12.621295in}{7.093809in}}{\pgfqpoint{12.627119in}{7.087985in}}%
\pgfpathcurveto{\pgfqpoint{12.632943in}{7.082161in}}{\pgfqpoint{12.640843in}{7.078889in}}{\pgfqpoint{12.649080in}{7.078889in}}%
\pgfusepath{stroke}%
\end{pgfscope}%
\begin{pgfscope}%
\pgfpathrectangle{\pgfqpoint{3.788192in}{2.980138in}}{\pgfqpoint{2.914000in}{2.171400in}}%
\pgfusepath{clip}%
\pgfsetbuttcap%
\pgfsetroundjoin%
\pgfsetlinewidth{1.003750pt}%
\definecolor{currentstroke}{rgb}{1.000000,0.000000,0.000000}%
\pgfsetstrokecolor{currentstroke}%
\pgfsetdash{}{0pt}%
\pgfpathmoveto{\pgfqpoint{4.530215in}{4.056056in}}%
\pgfpathcurveto{\pgfqpoint{4.538452in}{4.056056in}}{\pgfqpoint{4.546352in}{4.059328in}}{\pgfqpoint{4.552176in}{4.065152in}}%
\pgfpathcurveto{\pgfqpoint{4.557999in}{4.070976in}}{\pgfqpoint{4.561272in}{4.078876in}}{\pgfqpoint{4.561272in}{4.087112in}}%
\pgfpathcurveto{\pgfqpoint{4.561272in}{4.095349in}}{\pgfqpoint{4.557999in}{4.103249in}}{\pgfqpoint{4.552176in}{4.109073in}}%
\pgfpathcurveto{\pgfqpoint{4.546352in}{4.114897in}}{\pgfqpoint{4.538452in}{4.118169in}}{\pgfqpoint{4.530215in}{4.118169in}}%
\pgfpathcurveto{\pgfqpoint{4.521979in}{4.118169in}}{\pgfqpoint{4.514079in}{4.114897in}}{\pgfqpoint{4.508255in}{4.109073in}}%
\pgfpathcurveto{\pgfqpoint{4.502431in}{4.103249in}}{\pgfqpoint{4.499159in}{4.095349in}}{\pgfqpoint{4.499159in}{4.087112in}}%
\pgfpathcurveto{\pgfqpoint{4.499159in}{4.078876in}}{\pgfqpoint{4.502431in}{4.070976in}}{\pgfqpoint{4.508255in}{4.065152in}}%
\pgfpathcurveto{\pgfqpoint{4.514079in}{4.059328in}}{\pgfqpoint{4.521979in}{4.056056in}}{\pgfqpoint{4.530215in}{4.056056in}}%
\pgfpathlineto{\pgfqpoint{4.530215in}{4.056056in}}%
\pgfpathclose%
\pgfusepath{stroke}%
\end{pgfscope}%
\begin{pgfscope}%
\pgfpathrectangle{\pgfqpoint{3.788192in}{2.980138in}}{\pgfqpoint{2.914000in}{2.171400in}}%
\pgfusepath{clip}%
\pgfsetbuttcap%
\pgfsetmiterjoin%
\definecolor{currentfill}{rgb}{0.839216,0.152941,0.156863}%
\pgfsetfillcolor{currentfill}%
\pgfsetfillopacity{0.200000}%
\pgfsetlinewidth{1.003750pt}%
\definecolor{currentstroke}{rgb}{0.839216,0.152941,0.156863}%
\pgfsetstrokecolor{currentstroke}%
\pgfsetstrokeopacity{0.200000}%
\pgfsetdash{}{0pt}%
\pgfpathmoveto{\pgfqpoint{4.530215in}{2.980138in}}%
\pgfpathlineto{\pgfqpoint{24.162152in}{2.980138in}}%
\pgfpathlineto{\pgfqpoint{24.162152in}{5.151538in}}%
\pgfpathlineto{\pgfqpoint{4.530215in}{5.151538in}}%
\pgfpathlineto{\pgfqpoint{4.530215in}{2.980138in}}%
\pgfpathclose%
\pgfusepath{stroke,fill}%
\end{pgfscope}%
\begin{pgfscope}%
\pgfsetbuttcap%
\pgfsetmiterjoin%
\definecolor{currentfill}{rgb}{0.839216,0.152941,0.156863}%
\pgfsetfillcolor{currentfill}%
\pgfsetfillopacity{0.200000}%
\pgfsetlinewidth{1.003750pt}%
\definecolor{currentstroke}{rgb}{0.839216,0.152941,0.156863}%
\pgfsetstrokecolor{currentstroke}%
\pgfsetstrokeopacity{0.200000}%
\pgfsetdash{}{0pt}%
\pgfpathrectangle{\pgfqpoint{3.788192in}{2.980138in}}{\pgfqpoint{2.914000in}{2.171400in}}%
\pgfusepath{clip}%
\pgfpathmoveto{\pgfqpoint{4.530215in}{2.980138in}}%
\pgfpathlineto{\pgfqpoint{24.162152in}{2.980138in}}%
\pgfpathlineto{\pgfqpoint{24.162152in}{5.151538in}}%
\pgfpathlineto{\pgfqpoint{4.530215in}{5.151538in}}%
\pgfpathlineto{\pgfqpoint{4.530215in}{2.980138in}}%
\pgfpathclose%
\pgfusepath{clip}%
\pgfsys@defobject{currentpattern}{\pgfqpoint{0in}{0in}}{\pgfqpoint{1in}{1in}}{%
\begin{pgfscope}%
\pgfpathrectangle{\pgfqpoint{0in}{0in}}{\pgfqpoint{1in}{1in}}%
\pgfusepath{clip}%
\pgfpathmoveto{\pgfqpoint{-0.500000in}{0.500000in}}%
\pgfpathlineto{\pgfqpoint{0.500000in}{1.500000in}}%
\pgfpathmoveto{\pgfqpoint{-0.333333in}{0.333333in}}%
\pgfpathlineto{\pgfqpoint{0.666667in}{1.333333in}}%
\pgfpathmoveto{\pgfqpoint{-0.166667in}{0.166667in}}%
\pgfpathlineto{\pgfqpoint{0.833333in}{1.166667in}}%
\pgfpathmoveto{\pgfqpoint{0.000000in}{0.000000in}}%
\pgfpathlineto{\pgfqpoint{1.000000in}{1.000000in}}%
\pgfpathmoveto{\pgfqpoint{0.166667in}{-0.166667in}}%
\pgfpathlineto{\pgfqpoint{1.166667in}{0.833333in}}%
\pgfpathmoveto{\pgfqpoint{0.333333in}{-0.333333in}}%
\pgfpathlineto{\pgfqpoint{1.333333in}{0.666667in}}%
\pgfpathmoveto{\pgfqpoint{0.500000in}{-0.500000in}}%
\pgfpathlineto{\pgfqpoint{1.500000in}{0.500000in}}%
\pgfusepath{stroke}%
\end{pgfscope}%
}%
\pgfsys@transformshift{4.530215in}{2.980138in}%
\pgfsys@useobject{currentpattern}{}%
\pgfsys@transformshift{1in}{0in}%
\pgfsys@useobject{currentpattern}{}%
\pgfsys@transformshift{1in}{0in}%
\pgfsys@useobject{currentpattern}{}%
\pgfsys@transformshift{1in}{0in}%
\pgfsys@useobject{currentpattern}{}%
\pgfsys@transformshift{1in}{0in}%
\pgfsys@useobject{currentpattern}{}%
\pgfsys@transformshift{1in}{0in}%
\pgfsys@useobject{currentpattern}{}%
\pgfsys@transformshift{1in}{0in}%
\pgfsys@useobject{currentpattern}{}%
\pgfsys@transformshift{1in}{0in}%
\pgfsys@useobject{currentpattern}{}%
\pgfsys@transformshift{1in}{0in}%
\pgfsys@useobject{currentpattern}{}%
\pgfsys@transformshift{1in}{0in}%
\pgfsys@useobject{currentpattern}{}%
\pgfsys@transformshift{1in}{0in}%
\pgfsys@useobject{currentpattern}{}%
\pgfsys@transformshift{1in}{0in}%
\pgfsys@useobject{currentpattern}{}%
\pgfsys@transformshift{1in}{0in}%
\pgfsys@useobject{currentpattern}{}%
\pgfsys@transformshift{1in}{0in}%
\pgfsys@useobject{currentpattern}{}%
\pgfsys@transformshift{1in}{0in}%
\pgfsys@useobject{currentpattern}{}%
\pgfsys@transformshift{1in}{0in}%
\pgfsys@useobject{currentpattern}{}%
\pgfsys@transformshift{1in}{0in}%
\pgfsys@useobject{currentpattern}{}%
\pgfsys@transformshift{1in}{0in}%
\pgfsys@useobject{currentpattern}{}%
\pgfsys@transformshift{1in}{0in}%
\pgfsys@useobject{currentpattern}{}%
\pgfsys@transformshift{1in}{0in}%
\pgfsys@useobject{currentpattern}{}%
\pgfsys@transformshift{1in}{0in}%
\pgfsys@transformshift{-20in}{0in}%
\pgfsys@transformshift{0in}{1in}%
\pgfsys@useobject{currentpattern}{}%
\pgfsys@transformshift{1in}{0in}%
\pgfsys@useobject{currentpattern}{}%
\pgfsys@transformshift{1in}{0in}%
\pgfsys@useobject{currentpattern}{}%
\pgfsys@transformshift{1in}{0in}%
\pgfsys@useobject{currentpattern}{}%
\pgfsys@transformshift{1in}{0in}%
\pgfsys@useobject{currentpattern}{}%
\pgfsys@transformshift{1in}{0in}%
\pgfsys@useobject{currentpattern}{}%
\pgfsys@transformshift{1in}{0in}%
\pgfsys@useobject{currentpattern}{}%
\pgfsys@transformshift{1in}{0in}%
\pgfsys@useobject{currentpattern}{}%
\pgfsys@transformshift{1in}{0in}%
\pgfsys@useobject{currentpattern}{}%
\pgfsys@transformshift{1in}{0in}%
\pgfsys@useobject{currentpattern}{}%
\pgfsys@transformshift{1in}{0in}%
\pgfsys@useobject{currentpattern}{}%
\pgfsys@transformshift{1in}{0in}%
\pgfsys@useobject{currentpattern}{}%
\pgfsys@transformshift{1in}{0in}%
\pgfsys@useobject{currentpattern}{}%
\pgfsys@transformshift{1in}{0in}%
\pgfsys@useobject{currentpattern}{}%
\pgfsys@transformshift{1in}{0in}%
\pgfsys@useobject{currentpattern}{}%
\pgfsys@transformshift{1in}{0in}%
\pgfsys@useobject{currentpattern}{}%
\pgfsys@transformshift{1in}{0in}%
\pgfsys@useobject{currentpattern}{}%
\pgfsys@transformshift{1in}{0in}%
\pgfsys@useobject{currentpattern}{}%
\pgfsys@transformshift{1in}{0in}%
\pgfsys@useobject{currentpattern}{}%
\pgfsys@transformshift{1in}{0in}%
\pgfsys@useobject{currentpattern}{}%
\pgfsys@transformshift{1in}{0in}%
\pgfsys@transformshift{-20in}{0in}%
\pgfsys@transformshift{0in}{1in}%
\pgfsys@useobject{currentpattern}{}%
\pgfsys@transformshift{1in}{0in}%
\pgfsys@useobject{currentpattern}{}%
\pgfsys@transformshift{1in}{0in}%
\pgfsys@useobject{currentpattern}{}%
\pgfsys@transformshift{1in}{0in}%
\pgfsys@useobject{currentpattern}{}%
\pgfsys@transformshift{1in}{0in}%
\pgfsys@useobject{currentpattern}{}%
\pgfsys@transformshift{1in}{0in}%
\pgfsys@useobject{currentpattern}{}%
\pgfsys@transformshift{1in}{0in}%
\pgfsys@useobject{currentpattern}{}%
\pgfsys@transformshift{1in}{0in}%
\pgfsys@useobject{currentpattern}{}%
\pgfsys@transformshift{1in}{0in}%
\pgfsys@useobject{currentpattern}{}%
\pgfsys@transformshift{1in}{0in}%
\pgfsys@useobject{currentpattern}{}%
\pgfsys@transformshift{1in}{0in}%
\pgfsys@useobject{currentpattern}{}%
\pgfsys@transformshift{1in}{0in}%
\pgfsys@useobject{currentpattern}{}%
\pgfsys@transformshift{1in}{0in}%
\pgfsys@useobject{currentpattern}{}%
\pgfsys@transformshift{1in}{0in}%
\pgfsys@useobject{currentpattern}{}%
\pgfsys@transformshift{1in}{0in}%
\pgfsys@useobject{currentpattern}{}%
\pgfsys@transformshift{1in}{0in}%
\pgfsys@useobject{currentpattern}{}%
\pgfsys@transformshift{1in}{0in}%
\pgfsys@useobject{currentpattern}{}%
\pgfsys@transformshift{1in}{0in}%
\pgfsys@useobject{currentpattern}{}%
\pgfsys@transformshift{1in}{0in}%
\pgfsys@useobject{currentpattern}{}%
\pgfsys@transformshift{1in}{0in}%
\pgfsys@useobject{currentpattern}{}%
\pgfsys@transformshift{1in}{0in}%
\pgfsys@transformshift{-20in}{0in}%
\pgfsys@transformshift{0in}{1in}%
\end{pgfscope}%
\begin{pgfscope}%
\pgfpathrectangle{\pgfqpoint{3.788192in}{2.980138in}}{\pgfqpoint{2.914000in}{2.171400in}}%
\pgfusepath{clip}%
\pgfsetrectcap%
\pgfsetroundjoin%
\pgfsetlinewidth{0.803000pt}%
\definecolor{currentstroke}{rgb}{0.690196,0.690196,0.690196}%
\pgfsetstrokecolor{currentstroke}%
\pgfsetdash{}{0pt}%
\pgfpathmoveto{\pgfqpoint{4.105109in}{2.980138in}}%
\pgfpathlineto{\pgfqpoint{4.105109in}{5.151538in}}%
\pgfusepath{stroke}%
\end{pgfscope}%
\begin{pgfscope}%
\pgfsetbuttcap%
\pgfsetroundjoin%
\definecolor{currentfill}{rgb}{0.000000,0.000000,0.000000}%
\pgfsetfillcolor{currentfill}%
\pgfsetlinewidth{0.803000pt}%
\definecolor{currentstroke}{rgb}{0.000000,0.000000,0.000000}%
\pgfsetstrokecolor{currentstroke}%
\pgfsetdash{}{0pt}%
\pgfsys@defobject{currentmarker}{\pgfqpoint{0.000000in}{-0.048611in}}{\pgfqpoint{0.000000in}{0.000000in}}{%
\pgfpathmoveto{\pgfqpoint{0.000000in}{0.000000in}}%
\pgfpathlineto{\pgfqpoint{0.000000in}{-0.048611in}}%
\pgfusepath{stroke,fill}%
}%
\begin{pgfscope}%
\pgfsys@transformshift{4.105109in}{2.980138in}%
\pgfsys@useobject{currentmarker}{}%
\end{pgfscope}%
\end{pgfscope}%
\begin{pgfscope}%
\definecolor{textcolor}{rgb}{0.000000,0.000000,0.000000}%
\pgfsetstrokecolor{textcolor}%
\pgfsetfillcolor{textcolor}%
\pgftext[x=4.105109in,y=2.882916in,,top]{\color{textcolor}{\rmfamily\fontsize{14.000000}{16.800000}\selectfont\catcode`\^=\active\def^{\ifmmode\sp\else\^{}\fi}\catcode`\%=\active\def%{\%}$\mathdefault{5280}$}}%
\end{pgfscope}%
\begin{pgfscope}%
\pgfpathrectangle{\pgfqpoint{3.788192in}{2.980138in}}{\pgfqpoint{2.914000in}{2.171400in}}%
\pgfusepath{clip}%
\pgfsetrectcap%
\pgfsetroundjoin%
\pgfsetlinewidth{0.803000pt}%
\definecolor{currentstroke}{rgb}{0.690196,0.690196,0.690196}%
\pgfsetstrokecolor{currentstroke}%
\pgfsetdash{}{0pt}%
\pgfpathmoveto{\pgfqpoint{4.847132in}{2.980138in}}%
\pgfpathlineto{\pgfqpoint{4.847132in}{5.151538in}}%
\pgfusepath{stroke}%
\end{pgfscope}%
\begin{pgfscope}%
\pgfsetbuttcap%
\pgfsetroundjoin%
\definecolor{currentfill}{rgb}{0.000000,0.000000,0.000000}%
\pgfsetfillcolor{currentfill}%
\pgfsetlinewidth{0.803000pt}%
\definecolor{currentstroke}{rgb}{0.000000,0.000000,0.000000}%
\pgfsetstrokecolor{currentstroke}%
\pgfsetdash{}{0pt}%
\pgfsys@defobject{currentmarker}{\pgfqpoint{0.000000in}{-0.048611in}}{\pgfqpoint{0.000000in}{0.000000in}}{%
\pgfpathmoveto{\pgfqpoint{0.000000in}{0.000000in}}%
\pgfpathlineto{\pgfqpoint{0.000000in}{-0.048611in}}%
\pgfusepath{stroke,fill}%
}%
\begin{pgfscope}%
\pgfsys@transformshift{4.847132in}{2.980138in}%
\pgfsys@useobject{currentmarker}{}%
\end{pgfscope}%
\end{pgfscope}%
\begin{pgfscope}%
\definecolor{textcolor}{rgb}{0.000000,0.000000,0.000000}%
\pgfsetstrokecolor{textcolor}%
\pgfsetfillcolor{textcolor}%
\pgftext[x=4.847132in,y=2.882916in,,top]{\color{textcolor}{\rmfamily\fontsize{14.000000}{16.800000}\selectfont\catcode`\^=\active\def^{\ifmmode\sp\else\^{}\fi}\catcode`\%=\active\def%{\%}$\mathdefault{5300}$}}%
\end{pgfscope}%
\begin{pgfscope}%
\pgfpathrectangle{\pgfqpoint{3.788192in}{2.980138in}}{\pgfqpoint{2.914000in}{2.171400in}}%
\pgfusepath{clip}%
\pgfsetrectcap%
\pgfsetroundjoin%
\pgfsetlinewidth{0.803000pt}%
\definecolor{currentstroke}{rgb}{0.690196,0.690196,0.690196}%
\pgfsetstrokecolor{currentstroke}%
\pgfsetdash{}{0pt}%
\pgfpathmoveto{\pgfqpoint{5.589156in}{2.980138in}}%
\pgfpathlineto{\pgfqpoint{5.589156in}{5.151538in}}%
\pgfusepath{stroke}%
\end{pgfscope}%
\begin{pgfscope}%
\pgfsetbuttcap%
\pgfsetroundjoin%
\definecolor{currentfill}{rgb}{0.000000,0.000000,0.000000}%
\pgfsetfillcolor{currentfill}%
\pgfsetlinewidth{0.803000pt}%
\definecolor{currentstroke}{rgb}{0.000000,0.000000,0.000000}%
\pgfsetstrokecolor{currentstroke}%
\pgfsetdash{}{0pt}%
\pgfsys@defobject{currentmarker}{\pgfqpoint{0.000000in}{-0.048611in}}{\pgfqpoint{0.000000in}{0.000000in}}{%
\pgfpathmoveto{\pgfqpoint{0.000000in}{0.000000in}}%
\pgfpathlineto{\pgfqpoint{0.000000in}{-0.048611in}}%
\pgfusepath{stroke,fill}%
}%
\begin{pgfscope}%
\pgfsys@transformshift{5.589156in}{2.980138in}%
\pgfsys@useobject{currentmarker}{}%
\end{pgfscope}%
\end{pgfscope}%
\begin{pgfscope}%
\definecolor{textcolor}{rgb}{0.000000,0.000000,0.000000}%
\pgfsetstrokecolor{textcolor}%
\pgfsetfillcolor{textcolor}%
\pgftext[x=5.589156in,y=2.882916in,,top]{\color{textcolor}{\rmfamily\fontsize{14.000000}{16.800000}\selectfont\catcode`\^=\active\def^{\ifmmode\sp\else\^{}\fi}\catcode`\%=\active\def%{\%}$\mathdefault{5320}$}}%
\end{pgfscope}%
\begin{pgfscope}%
\pgfpathrectangle{\pgfqpoint{3.788192in}{2.980138in}}{\pgfqpoint{2.914000in}{2.171400in}}%
\pgfusepath{clip}%
\pgfsetrectcap%
\pgfsetroundjoin%
\pgfsetlinewidth{0.803000pt}%
\definecolor{currentstroke}{rgb}{0.690196,0.690196,0.690196}%
\pgfsetstrokecolor{currentstroke}%
\pgfsetdash{}{0pt}%
\pgfpathmoveto{\pgfqpoint{6.331180in}{2.980138in}}%
\pgfpathlineto{\pgfqpoint{6.331180in}{5.151538in}}%
\pgfusepath{stroke}%
\end{pgfscope}%
\begin{pgfscope}%
\pgfsetbuttcap%
\pgfsetroundjoin%
\definecolor{currentfill}{rgb}{0.000000,0.000000,0.000000}%
\pgfsetfillcolor{currentfill}%
\pgfsetlinewidth{0.803000pt}%
\definecolor{currentstroke}{rgb}{0.000000,0.000000,0.000000}%
\pgfsetstrokecolor{currentstroke}%
\pgfsetdash{}{0pt}%
\pgfsys@defobject{currentmarker}{\pgfqpoint{0.000000in}{-0.048611in}}{\pgfqpoint{0.000000in}{0.000000in}}{%
\pgfpathmoveto{\pgfqpoint{0.000000in}{0.000000in}}%
\pgfpathlineto{\pgfqpoint{0.000000in}{-0.048611in}}%
\pgfusepath{stroke,fill}%
}%
\begin{pgfscope}%
\pgfsys@transformshift{6.331180in}{2.980138in}%
\pgfsys@useobject{currentmarker}{}%
\end{pgfscope}%
\end{pgfscope}%
\begin{pgfscope}%
\definecolor{textcolor}{rgb}{0.000000,0.000000,0.000000}%
\pgfsetstrokecolor{textcolor}%
\pgfsetfillcolor{textcolor}%
\pgftext[x=6.331180in,y=2.882916in,,top]{\color{textcolor}{\rmfamily\fontsize{14.000000}{16.800000}\selectfont\catcode`\^=\active\def^{\ifmmode\sp\else\^{}\fi}\catcode`\%=\active\def%{\%}$\mathdefault{5340}$}}%
\end{pgfscope}%
\begin{pgfscope}%
\pgfpathrectangle{\pgfqpoint{3.788192in}{2.980138in}}{\pgfqpoint{2.914000in}{2.171400in}}%
\pgfusepath{clip}%
\pgfsetrectcap%
\pgfsetroundjoin%
\pgfsetlinewidth{0.803000pt}%
\definecolor{currentstroke}{rgb}{0.690196,0.690196,0.690196}%
\pgfsetstrokecolor{currentstroke}%
\pgfsetdash{}{0pt}%
\pgfpathmoveto{\pgfqpoint{3.788192in}{3.334947in}}%
\pgfpathlineto{\pgfqpoint{6.702192in}{3.334947in}}%
\pgfusepath{stroke}%
\end{pgfscope}%
\begin{pgfscope}%
\pgfsetbuttcap%
\pgfsetroundjoin%
\definecolor{currentfill}{rgb}{0.000000,0.000000,0.000000}%
\pgfsetfillcolor{currentfill}%
\pgfsetlinewidth{0.803000pt}%
\definecolor{currentstroke}{rgb}{0.000000,0.000000,0.000000}%
\pgfsetstrokecolor{currentstroke}%
\pgfsetdash{}{0pt}%
\pgfsys@defobject{currentmarker}{\pgfqpoint{-0.048611in}{0.000000in}}{\pgfqpoint{-0.000000in}{0.000000in}}{%
\pgfpathmoveto{\pgfqpoint{-0.000000in}{0.000000in}}%
\pgfpathlineto{\pgfqpoint{-0.048611in}{0.000000in}}%
\pgfusepath{stroke,fill}%
}%
\begin{pgfscope}%
\pgfsys@transformshift{3.788192in}{3.334947in}%
\pgfsys@useobject{currentmarker}{}%
\end{pgfscope}%
\end{pgfscope}%
\begin{pgfscope}%
\definecolor{textcolor}{rgb}{0.000000,0.000000,0.000000}%
\pgfsetstrokecolor{textcolor}%
\pgfsetfillcolor{textcolor}%
\pgftext[x=3.495138in, y=3.265502in, left, base]{\color{textcolor}{\rmfamily\fontsize{14.000000}{16.800000}\selectfont\catcode`\^=\active\def^{\ifmmode\sp\else\^{}\fi}\catcode`\%=\active\def%{\%}$\mathdefault{10}$}}%
\end{pgfscope}%
\begin{pgfscope}%
\pgfpathrectangle{\pgfqpoint{3.788192in}{2.980138in}}{\pgfqpoint{2.914000in}{2.171400in}}%
\pgfusepath{clip}%
\pgfsetrectcap%
\pgfsetroundjoin%
\pgfsetlinewidth{0.803000pt}%
\definecolor{currentstroke}{rgb}{0.690196,0.690196,0.690196}%
\pgfsetstrokecolor{currentstroke}%
\pgfsetdash{}{0pt}%
\pgfpathmoveto{\pgfqpoint{3.788192in}{4.044564in}}%
\pgfpathlineto{\pgfqpoint{6.702192in}{4.044564in}}%
\pgfusepath{stroke}%
\end{pgfscope}%
\begin{pgfscope}%
\pgfsetbuttcap%
\pgfsetroundjoin%
\definecolor{currentfill}{rgb}{0.000000,0.000000,0.000000}%
\pgfsetfillcolor{currentfill}%
\pgfsetlinewidth{0.803000pt}%
\definecolor{currentstroke}{rgb}{0.000000,0.000000,0.000000}%
\pgfsetstrokecolor{currentstroke}%
\pgfsetdash{}{0pt}%
\pgfsys@defobject{currentmarker}{\pgfqpoint{-0.048611in}{0.000000in}}{\pgfqpoint{-0.000000in}{0.000000in}}{%
\pgfpathmoveto{\pgfqpoint{-0.000000in}{0.000000in}}%
\pgfpathlineto{\pgfqpoint{-0.048611in}{0.000000in}}%
\pgfusepath{stroke,fill}%
}%
\begin{pgfscope}%
\pgfsys@transformshift{3.788192in}{4.044564in}%
\pgfsys@useobject{currentmarker}{}%
\end{pgfscope}%
\end{pgfscope}%
\begin{pgfscope}%
\definecolor{textcolor}{rgb}{0.000000,0.000000,0.000000}%
\pgfsetstrokecolor{textcolor}%
\pgfsetfillcolor{textcolor}%
\pgftext[x=3.495138in, y=3.975119in, left, base]{\color{textcolor}{\rmfamily\fontsize{14.000000}{16.800000}\selectfont\catcode`\^=\active\def^{\ifmmode\sp\else\^{}\fi}\catcode`\%=\active\def%{\%}$\mathdefault{12}$}}%
\end{pgfscope}%
\begin{pgfscope}%
\pgfpathrectangle{\pgfqpoint{3.788192in}{2.980138in}}{\pgfqpoint{2.914000in}{2.171400in}}%
\pgfusepath{clip}%
\pgfsetrectcap%
\pgfsetroundjoin%
\pgfsetlinewidth{0.803000pt}%
\definecolor{currentstroke}{rgb}{0.690196,0.690196,0.690196}%
\pgfsetstrokecolor{currentstroke}%
\pgfsetdash{}{0pt}%
\pgfpathmoveto{\pgfqpoint{3.788192in}{4.754181in}}%
\pgfpathlineto{\pgfqpoint{6.702192in}{4.754181in}}%
\pgfusepath{stroke}%
\end{pgfscope}%
\begin{pgfscope}%
\pgfsetbuttcap%
\pgfsetroundjoin%
\definecolor{currentfill}{rgb}{0.000000,0.000000,0.000000}%
\pgfsetfillcolor{currentfill}%
\pgfsetlinewidth{0.803000pt}%
\definecolor{currentstroke}{rgb}{0.000000,0.000000,0.000000}%
\pgfsetstrokecolor{currentstroke}%
\pgfsetdash{}{0pt}%
\pgfsys@defobject{currentmarker}{\pgfqpoint{-0.048611in}{0.000000in}}{\pgfqpoint{-0.000000in}{0.000000in}}{%
\pgfpathmoveto{\pgfqpoint{-0.000000in}{0.000000in}}%
\pgfpathlineto{\pgfqpoint{-0.048611in}{0.000000in}}%
\pgfusepath{stroke,fill}%
}%
\begin{pgfscope}%
\pgfsys@transformshift{3.788192in}{4.754181in}%
\pgfsys@useobject{currentmarker}{}%
\end{pgfscope}%
\end{pgfscope}%
\begin{pgfscope}%
\definecolor{textcolor}{rgb}{0.000000,0.000000,0.000000}%
\pgfsetstrokecolor{textcolor}%
\pgfsetfillcolor{textcolor}%
\pgftext[x=3.495138in, y=4.684736in, left, base]{\color{textcolor}{\rmfamily\fontsize{14.000000}{16.800000}\selectfont\catcode`\^=\active\def^{\ifmmode\sp\else\^{}\fi}\catcode`\%=\active\def%{\%}$\mathdefault{14}$}}%
\end{pgfscope}%
\begin{pgfscope}%
\pgfpathrectangle{\pgfqpoint{3.788192in}{2.980138in}}{\pgfqpoint{2.914000in}{2.171400in}}%
\pgfusepath{clip}%
\pgfsetrectcap%
\pgfsetroundjoin%
\pgfsetlinewidth{1.505625pt}%
\definecolor{currentstroke}{rgb}{0.000000,0.000000,1.000000}%
\pgfsetstrokecolor{currentstroke}%
\pgfsetdash{}{0pt}%
\pgfpathmoveto{\pgfqpoint{5.478622in}{4.808529in}}%
\pgfpathlineto{\pgfqpoint{5.701947in}{2.977638in}}%
\pgfusepath{stroke}%
\end{pgfscope}%
\begin{pgfscope}%
\pgfpathrectangle{\pgfqpoint{3.788192in}{2.980138in}}{\pgfqpoint{2.914000in}{2.171400in}}%
\pgfusepath{clip}%
\pgfsetbuttcap%
\pgfsetroundjoin%
\definecolor{currentfill}{rgb}{0.000000,0.000000,1.000000}%
\pgfsetfillcolor{currentfill}%
\pgfsetlinewidth{1.003750pt}%
\definecolor{currentstroke}{rgb}{0.000000,0.000000,1.000000}%
\pgfsetstrokecolor{currentstroke}%
\pgfsetdash{}{0pt}%
\pgfsys@defobject{currentmarker}{\pgfqpoint{-0.041667in}{-0.041667in}}{\pgfqpoint{0.041667in}{0.041667in}}{%
\pgfpathmoveto{\pgfqpoint{0.000000in}{-0.041667in}}%
\pgfpathcurveto{\pgfqpoint{0.011050in}{-0.041667in}}{\pgfqpoint{0.021649in}{-0.037276in}}{\pgfqpoint{0.029463in}{-0.029463in}}%
\pgfpathcurveto{\pgfqpoint{0.037276in}{-0.021649in}}{\pgfqpoint{0.041667in}{-0.011050in}}{\pgfqpoint{0.041667in}{0.000000in}}%
\pgfpathcurveto{\pgfqpoint{0.041667in}{0.011050in}}{\pgfqpoint{0.037276in}{0.021649in}}{\pgfqpoint{0.029463in}{0.029463in}}%
\pgfpathcurveto{\pgfqpoint{0.021649in}{0.037276in}}{\pgfqpoint{0.011050in}{0.041667in}}{\pgfqpoint{0.000000in}{0.041667in}}%
\pgfpathcurveto{\pgfqpoint{-0.011050in}{0.041667in}}{\pgfqpoint{-0.021649in}{0.037276in}}{\pgfqpoint{-0.029463in}{0.029463in}}%
\pgfpathcurveto{\pgfqpoint{-0.037276in}{0.021649in}}{\pgfqpoint{-0.041667in}{0.011050in}}{\pgfqpoint{-0.041667in}{0.000000in}}%
\pgfpathcurveto{\pgfqpoint{-0.041667in}{-0.011050in}}{\pgfqpoint{-0.037276in}{-0.021649in}}{\pgfqpoint{-0.029463in}{-0.029463in}}%
\pgfpathcurveto{\pgfqpoint{-0.021649in}{-0.037276in}}{\pgfqpoint{-0.011050in}{-0.041667in}}{\pgfqpoint{0.000000in}{-0.041667in}}%
\pgfpathlineto{\pgfqpoint{0.000000in}{-0.041667in}}%
\pgfpathclose%
\pgfusepath{stroke,fill}%
}%
\begin{pgfscope}%
\pgfsys@transformshift{5.478622in}{4.808529in}%
\pgfsys@useobject{currentmarker}{}%
\end{pgfscope}%
\begin{pgfscope}%
\pgfsys@transformshift{5.786444in}{2.284897in}%
\pgfsys@useobject{currentmarker}{}%
\end{pgfscope}%
\begin{pgfscope}%
\pgfsys@transformshift{5.955602in}{1.807070in}%
\pgfsys@useobject{currentmarker}{}%
\end{pgfscope}%
\begin{pgfscope}%
\pgfsys@transformshift{6.072257in}{1.530809in}%
\pgfsys@useobject{currentmarker}{}%
\end{pgfscope}%
\begin{pgfscope}%
\pgfsys@transformshift{6.134561in}{1.493263in}%
\pgfsys@useobject{currentmarker}{}%
\end{pgfscope}%
\begin{pgfscope}%
\pgfsys@transformshift{6.204334in}{1.296852in}%
\pgfsys@useobject{currentmarker}{}%
\end{pgfscope}%
\begin{pgfscope}%
\pgfsys@transformshift{6.281957in}{1.238734in}%
\pgfsys@useobject{currentmarker}{}%
\end{pgfscope}%
\begin{pgfscope}%
\pgfsys@transformshift{6.389240in}{1.234341in}%
\pgfsys@useobject{currentmarker}{}%
\end{pgfscope}%
\begin{pgfscope}%
\pgfsys@transformshift{6.427201in}{1.108377in}%
\pgfsys@useobject{currentmarker}{}%
\end{pgfscope}%
\begin{pgfscope}%
\pgfsys@transformshift{6.519285in}{1.077759in}%
\pgfsys@useobject{currentmarker}{}%
\end{pgfscope}%
\begin{pgfscope}%
\pgfsys@transformshift{6.689124in}{1.072073in}%
\pgfsys@useobject{currentmarker}{}%
\end{pgfscope}%
\begin{pgfscope}%
\pgfsys@transformshift{6.747190in}{1.024269in}%
\pgfsys@useobject{currentmarker}{}%
\end{pgfscope}%
\begin{pgfscope}%
\pgfsys@transformshift{6.806750in}{1.015907in}%
\pgfsys@useobject{currentmarker}{}%
\end{pgfscope}%
\begin{pgfscope}%
\pgfsys@transformshift{6.931730in}{0.991431in}%
\pgfsys@useobject{currentmarker}{}%
\end{pgfscope}%
\begin{pgfscope}%
\pgfsys@transformshift{7.035734in}{0.970288in}%
\pgfsys@useobject{currentmarker}{}%
\end{pgfscope}%
\begin{pgfscope}%
\pgfsys@transformshift{7.042663in}{0.960624in}%
\pgfsys@useobject{currentmarker}{}%
\end{pgfscope}%
\begin{pgfscope}%
\pgfsys@transformshift{7.110896in}{0.945926in}%
\pgfsys@useobject{currentmarker}{}%
\end{pgfscope}%
\begin{pgfscope}%
\pgfsys@transformshift{7.385541in}{0.918874in}%
\pgfsys@useobject{currentmarker}{}%
\end{pgfscope}%
\begin{pgfscope}%
\pgfsys@transformshift{7.677622in}{0.881182in}%
\pgfsys@useobject{currentmarker}{}%
\end{pgfscope}%
\begin{pgfscope}%
\pgfsys@transformshift{8.074033in}{0.854291in}%
\pgfsys@useobject{currentmarker}{}%
\end{pgfscope}%
\begin{pgfscope}%
\pgfsys@transformshift{8.437037in}{0.845259in}%
\pgfsys@useobject{currentmarker}{}%
\end{pgfscope}%
\begin{pgfscope}%
\pgfsys@transformshift{8.451833in}{0.833333in}%
\pgfsys@useobject{currentmarker}{}%
\end{pgfscope}%
\begin{pgfscope}%
\pgfsys@transformshift{8.473270in}{0.812531in}%
\pgfsys@useobject{currentmarker}{}%
\end{pgfscope}%
\begin{pgfscope}%
\pgfsys@transformshift{8.530759in}{0.805118in}%
\pgfsys@useobject{currentmarker}{}%
\end{pgfscope}%
\begin{pgfscope}%
\pgfsys@transformshift{8.667946in}{0.801766in}%
\pgfsys@useobject{currentmarker}{}%
\end{pgfscope}%
\begin{pgfscope}%
\pgfsys@transformshift{8.761129in}{0.792896in}%
\pgfsys@useobject{currentmarker}{}%
\end{pgfscope}%
\begin{pgfscope}%
\pgfsys@transformshift{9.175032in}{0.779118in}%
\pgfsys@useobject{currentmarker}{}%
\end{pgfscope}%
\begin{pgfscope}%
\pgfsys@transformshift{9.229242in}{0.771806in}%
\pgfsys@useobject{currentmarker}{}%
\end{pgfscope}%
\begin{pgfscope}%
\pgfsys@transformshift{9.230836in}{0.763945in}%
\pgfsys@useobject{currentmarker}{}%
\end{pgfscope}%
\begin{pgfscope}%
\pgfsys@transformshift{9.319803in}{0.762647in}%
\pgfsys@useobject{currentmarker}{}%
\end{pgfscope}%
\begin{pgfscope}%
\pgfsys@transformshift{9.530028in}{0.749444in}%
\pgfsys@useobject{currentmarker}{}%
\end{pgfscope}%
\begin{pgfscope}%
\pgfsys@transformshift{9.629502in}{0.748101in}%
\pgfsys@useobject{currentmarker}{}%
\end{pgfscope}%
\begin{pgfscope}%
\pgfsys@transformshift{9.644888in}{0.743176in}%
\pgfsys@useobject{currentmarker}{}%
\end{pgfscope}%
\begin{pgfscope}%
\pgfsys@transformshift{10.473184in}{0.730520in}%
\pgfsys@useobject{currentmarker}{}%
\end{pgfscope}%
\begin{pgfscope}%
\pgfsys@transformshift{10.538033in}{0.705808in}%
\pgfsys@useobject{currentmarker}{}%
\end{pgfscope}%
\begin{pgfscope}%
\pgfsys@transformshift{10.574876in}{0.703903in}%
\pgfsys@useobject{currentmarker}{}%
\end{pgfscope}%
\begin{pgfscope}%
\pgfsys@transformshift{10.706030in}{0.700593in}%
\pgfsys@useobject{currentmarker}{}%
\end{pgfscope}%
\begin{pgfscope}%
\pgfsys@transformshift{10.965602in}{0.687998in}%
\pgfsys@useobject{currentmarker}{}%
\end{pgfscope}%
\begin{pgfscope}%
\pgfsys@transformshift{11.060169in}{0.684789in}%
\pgfsys@useobject{currentmarker}{}%
\end{pgfscope}%
\begin{pgfscope}%
\pgfsys@transformshift{11.147519in}{0.680781in}%
\pgfsys@useobject{currentmarker}{}%
\end{pgfscope}%
\begin{pgfscope}%
\pgfsys@transformshift{11.370645in}{0.672484in}%
\pgfsys@useobject{currentmarker}{}%
\end{pgfscope}%
\begin{pgfscope}%
\pgfsys@transformshift{11.728159in}{0.668514in}%
\pgfsys@useobject{currentmarker}{}%
\end{pgfscope}%
\begin{pgfscope}%
\pgfsys@transformshift{12.137942in}{0.666709in}%
\pgfsys@useobject{currentmarker}{}%
\end{pgfscope}%
\begin{pgfscope}%
\pgfsys@transformshift{12.434130in}{0.666468in}%
\pgfsys@useobject{currentmarker}{}%
\end{pgfscope}%
\begin{pgfscope}%
\pgfsys@transformshift{12.804602in}{0.666397in}%
\pgfsys@useobject{currentmarker}{}%
\end{pgfscope}%
\begin{pgfscope}%
\pgfsys@transformshift{14.562739in}{0.661643in}%
\pgfsys@useobject{currentmarker}{}%
\end{pgfscope}%
\begin{pgfscope}%
\pgfsys@transformshift{15.028873in}{0.661022in}%
\pgfsys@useobject{currentmarker}{}%
\end{pgfscope}%
\begin{pgfscope}%
\pgfsys@transformshift{15.695083in}{0.659204in}%
\pgfsys@useobject{currentmarker}{}%
\end{pgfscope}%
\begin{pgfscope}%
\pgfsys@transformshift{16.663977in}{0.658612in}%
\pgfsys@useobject{currentmarker}{}%
\end{pgfscope}%
\begin{pgfscope}%
\pgfsys@transformshift{17.188373in}{0.655818in}%
\pgfsys@useobject{currentmarker}{}%
\end{pgfscope}%
\begin{pgfscope}%
\pgfsys@transformshift{18.420605in}{0.654898in}%
\pgfsys@useobject{currentmarker}{}%
\end{pgfscope}%
\begin{pgfscope}%
\pgfsys@transformshift{20.009331in}{0.652189in}%
\pgfsys@useobject{currentmarker}{}%
\end{pgfscope}%
\begin{pgfscope}%
\pgfsys@transformshift{21.260300in}{0.647607in}%
\pgfsys@useobject{currentmarker}{}%
\end{pgfscope}%
\begin{pgfscope}%
\pgfsys@transformshift{23.228701in}{0.645006in}%
\pgfsys@useobject{currentmarker}{}%
\end{pgfscope}%
\begin{pgfscope}%
\pgfsys@transformshift{25.888770in}{0.639824in}%
\pgfsys@useobject{currentmarker}{}%
\end{pgfscope}%
\begin{pgfscope}%
\pgfsys@transformshift{28.569097in}{0.635141in}%
\pgfsys@useobject{currentmarker}{}%
\end{pgfscope}%
\begin{pgfscope}%
\pgfsys@transformshift{32.504050in}{0.628662in}%
\pgfsys@useobject{currentmarker}{}%
\end{pgfscope}%
\begin{pgfscope}%
\pgfsys@transformshift{39.060135in}{0.620415in}%
\pgfsys@useobject{currentmarker}{}%
\end{pgfscope}%
\begin{pgfscope}%
\pgfsys@transformshift{50.416853in}{0.608419in}%
\pgfsys@useobject{currentmarker}{}%
\end{pgfscope}%
\begin{pgfscope}%
\pgfsys@transformshift{87.100182in}{0.583248in}%
\pgfsys@useobject{currentmarker}{}%
\end{pgfscope}%
\end{pgfscope}%
\begin{pgfscope}%
\pgfpathrectangle{\pgfqpoint{3.788192in}{2.980138in}}{\pgfqpoint{2.914000in}{2.171400in}}%
\pgfusepath{clip}%
\pgfsetrectcap%
\pgfsetroundjoin%
\pgfsetlinewidth{1.505625pt}%
\definecolor{currentstroke}{rgb}{0.121569,0.466667,0.705882}%
\pgfsetstrokecolor{currentstroke}%
\pgfsetstrokeopacity{0.500000}%
\pgfsetdash{}{0pt}%
\pgfusepath{stroke}%
\end{pgfscope}%
\begin{pgfscope}%
\pgfsetrectcap%
\pgfsetmiterjoin%
\pgfsetlinewidth{0.803000pt}%
\definecolor{currentstroke}{rgb}{0.000000,0.000000,0.000000}%
\pgfsetstrokecolor{currentstroke}%
\pgfsetdash{}{0pt}%
\pgfpathmoveto{\pgfqpoint{3.788192in}{2.980138in}}%
\pgfpathlineto{\pgfqpoint{3.788192in}{5.151538in}}%
\pgfusepath{stroke}%
\end{pgfscope}%
\begin{pgfscope}%
\pgfsetrectcap%
\pgfsetmiterjoin%
\pgfsetlinewidth{0.803000pt}%
\definecolor{currentstroke}{rgb}{0.000000,0.000000,0.000000}%
\pgfsetstrokecolor{currentstroke}%
\pgfsetdash{}{0pt}%
\pgfpathmoveto{\pgfqpoint{6.702192in}{2.980138in}}%
\pgfpathlineto{\pgfqpoint{6.702192in}{5.151538in}}%
\pgfusepath{stroke}%
\end{pgfscope}%
\begin{pgfscope}%
\pgfsetrectcap%
\pgfsetmiterjoin%
\pgfsetlinewidth{0.803000pt}%
\definecolor{currentstroke}{rgb}{0.000000,0.000000,0.000000}%
\pgfsetstrokecolor{currentstroke}%
\pgfsetdash{}{0pt}%
\pgfpathmoveto{\pgfqpoint{3.788192in}{2.980138in}}%
\pgfpathlineto{\pgfqpoint{6.702192in}{2.980138in}}%
\pgfusepath{stroke}%
\end{pgfscope}%
\begin{pgfscope}%
\pgfsetrectcap%
\pgfsetmiterjoin%
\pgfsetlinewidth{0.803000pt}%
\definecolor{currentstroke}{rgb}{0.000000,0.000000,0.000000}%
\pgfsetstrokecolor{currentstroke}%
\pgfsetdash{}{0pt}%
\pgfpathmoveto{\pgfqpoint{3.788192in}{5.151538in}}%
\pgfpathlineto{\pgfqpoint{6.702192in}{5.151538in}}%
\pgfusepath{stroke}%
\end{pgfscope}%
\begin{pgfscope}%
\pgfsetbuttcap%
\pgfsetmiterjoin%
\definecolor{currentfill}{rgb}{1.000000,1.000000,1.000000}%
\pgfsetfillcolor{currentfill}%
\pgfsetfillopacity{0.800000}%
\pgfsetlinewidth{1.003750pt}%
\definecolor{currentstroke}{rgb}{0.800000,0.800000,0.800000}%
\pgfsetstrokecolor{currentstroke}%
\pgfsetstrokeopacity{0.800000}%
\pgfsetdash{}{0pt}%
\pgfpathmoveto{\pgfqpoint{3.327765in}{0.781249in}}%
\pgfpathlineto{\pgfqpoint{6.732636in}{0.781249in}}%
\pgfpathquadraticcurveto{\pgfqpoint{6.777080in}{0.781249in}}{\pgfqpoint{6.777080in}{0.825694in}}%
\pgfpathlineto{\pgfqpoint{6.777080in}{2.153779in}}%
\pgfpathquadraticcurveto{\pgfqpoint{6.777080in}{2.198223in}}{\pgfqpoint{6.732636in}{2.198223in}}%
\pgfpathlineto{\pgfqpoint{3.327765in}{2.198223in}}%
\pgfpathquadraticcurveto{\pgfqpoint{3.283320in}{2.198223in}}{\pgfqpoint{3.283320in}{2.153779in}}%
\pgfpathlineto{\pgfqpoint{3.283320in}{0.825694in}}%
\pgfpathquadraticcurveto{\pgfqpoint{3.283320in}{0.781249in}}{\pgfqpoint{3.327765in}{0.781249in}}%
\pgfpathlineto{\pgfqpoint{3.327765in}{0.781249in}}%
\pgfpathclose%
\pgfusepath{stroke,fill}%
\end{pgfscope}%
\begin{pgfscope}%
\pgfsetrectcap%
\pgfsetroundjoin%
\pgfsetlinewidth{1.505625pt}%
\definecolor{currentstroke}{rgb}{0.000000,0.000000,1.000000}%
\pgfsetstrokecolor{currentstroke}%
\pgfsetdash{}{0pt}%
\pgfpathmoveto{\pgfqpoint{3.372209in}{2.020446in}}%
\pgfpathlineto{\pgfqpoint{3.594431in}{2.020446in}}%
\pgfpathlineto{\pgfqpoint{3.816654in}{2.020446in}}%
\pgfusepath{stroke}%
\end{pgfscope}%
\begin{pgfscope}%
\pgfsetbuttcap%
\pgfsetroundjoin%
\definecolor{currentfill}{rgb}{0.000000,0.000000,1.000000}%
\pgfsetfillcolor{currentfill}%
\pgfsetlinewidth{1.003750pt}%
\definecolor{currentstroke}{rgb}{0.000000,0.000000,1.000000}%
\pgfsetstrokecolor{currentstroke}%
\pgfsetdash{}{0pt}%
\pgfsys@defobject{currentmarker}{\pgfqpoint{-0.006944in}{-0.006944in}}{\pgfqpoint{0.006944in}{0.006944in}}{%
\pgfpathmoveto{\pgfqpoint{0.000000in}{-0.006944in}}%
\pgfpathcurveto{\pgfqpoint{0.001842in}{-0.006944in}}{\pgfqpoint{0.003608in}{-0.006213in}}{\pgfqpoint{0.004910in}{-0.004910in}}%
\pgfpathcurveto{\pgfqpoint{0.006213in}{-0.003608in}}{\pgfqpoint{0.006944in}{-0.001842in}}{\pgfqpoint{0.006944in}{0.000000in}}%
\pgfpathcurveto{\pgfqpoint{0.006944in}{0.001842in}}{\pgfqpoint{0.006213in}{0.003608in}}{\pgfqpoint{0.004910in}{0.004910in}}%
\pgfpathcurveto{\pgfqpoint{0.003608in}{0.006213in}}{\pgfqpoint{0.001842in}{0.006944in}}{\pgfqpoint{0.000000in}{0.006944in}}%
\pgfpathcurveto{\pgfqpoint{-0.001842in}{0.006944in}}{\pgfqpoint{-0.003608in}{0.006213in}}{\pgfqpoint{-0.004910in}{0.004910in}}%
\pgfpathcurveto{\pgfqpoint{-0.006213in}{0.003608in}}{\pgfqpoint{-0.006944in}{0.001842in}}{\pgfqpoint{-0.006944in}{0.000000in}}%
\pgfpathcurveto{\pgfqpoint{-0.006944in}{-0.001842in}}{\pgfqpoint{-0.006213in}{-0.003608in}}{\pgfqpoint{-0.004910in}{-0.004910in}}%
\pgfpathcurveto{\pgfqpoint{-0.003608in}{-0.006213in}}{\pgfqpoint{-0.001842in}{-0.006944in}}{\pgfqpoint{0.000000in}{-0.006944in}}%
\pgfpathlineto{\pgfqpoint{0.000000in}{-0.006944in}}%
\pgfpathclose%
\pgfusepath{stroke,fill}%
}%
\begin{pgfscope}%
\pgfsys@transformshift{3.594431in}{2.020446in}%
\pgfsys@useobject{currentmarker}{}%
\end{pgfscope}%
\end{pgfscope}%
\begin{pgfscope}%
\definecolor{textcolor}{rgb}{0.000000,0.000000,0.000000}%
\pgfsetstrokecolor{textcolor}%
\pgfsetfillcolor{textcolor}%
\pgftext[x=3.994431in,y=1.942668in,left,base]{\color{textcolor}{\rmfamily\fontsize{16.000000}{19.200000}\selectfont\catcode`\^=\active\def^{\ifmmode\sp\else\^{}\fi}\catcode`\%=\active\def%{\%}osier}}%
\end{pgfscope}%
\begin{pgfscope}%
\pgfsetrectcap%
\pgfsetroundjoin%
\pgfsetlinewidth{1.505625pt}%
\definecolor{currentstroke}{rgb}{0.121569,0.466667,0.705882}%
\pgfsetstrokecolor{currentstroke}%
\pgfsetstrokeopacity{0.500000}%
\pgfsetdash{}{0pt}%
\pgfpathmoveto{\pgfqpoint{3.372209in}{1.682869in}}%
\pgfpathlineto{\pgfqpoint{3.594431in}{1.682869in}}%
\pgfpathlineto{\pgfqpoint{3.816654in}{1.682869in}}%
\pgfusepath{stroke}%
\end{pgfscope}%
\begin{pgfscope}%
\definecolor{textcolor}{rgb}{0.000000,0.000000,0.000000}%
\pgfsetstrokecolor{textcolor}%
\pgfsetfillcolor{textcolor}%
\pgftext[x=3.994431in,y=1.605091in,left,base]{\color{textcolor}{\rmfamily\fontsize{16.000000}{19.200000}\selectfont\catcode`\^=\active\def^{\ifmmode\sp\else\^{}\fi}\catcode`\%=\active\def%{\%}near-optimal space (osier)}}%
\end{pgfscope}%
\begin{pgfscope}%
\pgfsetbuttcap%
\pgfsetroundjoin%
\pgfsetlinewidth{1.003750pt}%
\definecolor{currentstroke}{rgb}{1.000000,0.000000,0.000000}%
\pgfsetstrokecolor{currentstroke}%
\pgfsetdash{}{0pt}%
\pgfpathmoveto{\pgfqpoint{3.594431in}{1.294791in}}%
\pgfpathcurveto{\pgfqpoint{3.602668in}{1.294791in}}{\pgfqpoint{3.610568in}{1.298063in}}{\pgfqpoint{3.616392in}{1.303887in}}%
\pgfpathcurveto{\pgfqpoint{3.622216in}{1.309711in}}{\pgfqpoint{3.625488in}{1.317611in}}{\pgfqpoint{3.625488in}{1.325847in}}%
\pgfpathcurveto{\pgfqpoint{3.625488in}{1.334084in}}{\pgfqpoint{3.622216in}{1.341984in}}{\pgfqpoint{3.616392in}{1.347808in}}%
\pgfpathcurveto{\pgfqpoint{3.610568in}{1.353632in}}{\pgfqpoint{3.602668in}{1.356904in}}{\pgfqpoint{3.594431in}{1.356904in}}%
\pgfpathcurveto{\pgfqpoint{3.586195in}{1.356904in}}{\pgfqpoint{3.578295in}{1.353632in}}{\pgfqpoint{3.572471in}{1.347808in}}%
\pgfpathcurveto{\pgfqpoint{3.566647in}{1.341984in}}{\pgfqpoint{3.563375in}{1.334084in}}{\pgfqpoint{3.563375in}{1.325847in}}%
\pgfpathcurveto{\pgfqpoint{3.563375in}{1.317611in}}{\pgfqpoint{3.566647in}{1.309711in}}{\pgfqpoint{3.572471in}{1.303887in}}%
\pgfpathcurveto{\pgfqpoint{3.578295in}{1.298063in}}{\pgfqpoint{3.586195in}{1.294791in}}{\pgfqpoint{3.594431in}{1.294791in}}%
\pgfpathlineto{\pgfqpoint{3.594431in}{1.294791in}}%
\pgfpathclose%
\pgfusepath{stroke}%
\end{pgfscope}%
\begin{pgfscope}%
\definecolor{textcolor}{rgb}{0.000000,0.000000,0.000000}%
\pgfsetstrokecolor{textcolor}%
\pgfsetfillcolor{textcolor}%
\pgftext[x=3.994431in,y=1.267514in,left,base]{\color{textcolor}{\rmfamily\fontsize{16.000000}{19.200000}\selectfont\catcode`\^=\active\def^{\ifmmode\sp\else\^{}\fi}\catcode`\%=\active\def%{\%}temoa+mga}}%
\end{pgfscope}%
\begin{pgfscope}%
\pgfsetbuttcap%
\pgfsetmiterjoin%
\definecolor{currentfill}{rgb}{0.839216,0.152941,0.156863}%
\pgfsetfillcolor{currentfill}%
\pgfsetfillopacity{0.200000}%
\pgfsetlinewidth{1.003750pt}%
\definecolor{currentstroke}{rgb}{0.839216,0.152941,0.156863}%
\pgfsetstrokecolor{currentstroke}%
\pgfsetstrokeopacity{0.200000}%
\pgfsetdash{}{0pt}%
\pgfpathmoveto{\pgfqpoint{3.372209in}{0.929937in}}%
\pgfpathlineto{\pgfqpoint{3.816654in}{0.929937in}}%
\pgfpathlineto{\pgfqpoint{3.816654in}{1.085493in}}%
\pgfpathlineto{\pgfqpoint{3.372209in}{1.085493in}}%
\pgfpathlineto{\pgfqpoint{3.372209in}{0.929937in}}%
\pgfpathclose%
\pgfusepath{stroke,fill}%
\end{pgfscope}%
\begin{pgfscope}%
\pgfsetbuttcap%
\pgfsetmiterjoin%
\definecolor{currentfill}{rgb}{0.839216,0.152941,0.156863}%
\pgfsetfillcolor{currentfill}%
\pgfsetfillopacity{0.200000}%
\pgfsetlinewidth{1.003750pt}%
\definecolor{currentstroke}{rgb}{0.839216,0.152941,0.156863}%
\pgfsetstrokecolor{currentstroke}%
\pgfsetstrokeopacity{0.200000}%
\pgfsetdash{}{0pt}%
\pgfpathmoveto{\pgfqpoint{3.372209in}{0.929937in}}%
\pgfpathlineto{\pgfqpoint{3.816654in}{0.929937in}}%
\pgfpathlineto{\pgfqpoint{3.816654in}{1.085493in}}%
\pgfpathlineto{\pgfqpoint{3.372209in}{1.085493in}}%
\pgfpathlineto{\pgfqpoint{3.372209in}{0.929937in}}%
\pgfpathclose%
\pgfusepath{clip}%
\pgfsys@defobject{currentpattern}{\pgfqpoint{0in}{0in}}{\pgfqpoint{1in}{1in}}{%
\begin{pgfscope}%
\pgfpathrectangle{\pgfqpoint{0in}{0in}}{\pgfqpoint{1in}{1in}}%
\pgfusepath{clip}%
\pgfpathmoveto{\pgfqpoint{-0.500000in}{0.500000in}}%
\pgfpathlineto{\pgfqpoint{0.500000in}{1.500000in}}%
\pgfpathmoveto{\pgfqpoint{-0.333333in}{0.333333in}}%
\pgfpathlineto{\pgfqpoint{0.666667in}{1.333333in}}%
\pgfpathmoveto{\pgfqpoint{-0.166667in}{0.166667in}}%
\pgfpathlineto{\pgfqpoint{0.833333in}{1.166667in}}%
\pgfpathmoveto{\pgfqpoint{0.000000in}{0.000000in}}%
\pgfpathlineto{\pgfqpoint{1.000000in}{1.000000in}}%
\pgfpathmoveto{\pgfqpoint{0.166667in}{-0.166667in}}%
\pgfpathlineto{\pgfqpoint{1.166667in}{0.833333in}}%
\pgfpathmoveto{\pgfqpoint{0.333333in}{-0.333333in}}%
\pgfpathlineto{\pgfqpoint{1.333333in}{0.666667in}}%
\pgfpathmoveto{\pgfqpoint{0.500000in}{-0.500000in}}%
\pgfpathlineto{\pgfqpoint{1.500000in}{0.500000in}}%
\pgfusepath{stroke}%
\end{pgfscope}%
}%
\pgfsys@transformshift{3.372209in}{0.929937in}%
\pgfsys@useobject{currentpattern}{}%
\pgfsys@transformshift{1in}{0in}%
\pgfsys@transformshift{-1in}{0in}%
\pgfsys@transformshift{0in}{1in}%
\end{pgfscope}%
\begin{pgfscope}%
\definecolor{textcolor}{rgb}{0.000000,0.000000,0.000000}%
\pgfsetstrokecolor{textcolor}%
\pgfsetfillcolor{textcolor}%
\pgftext[x=3.994431in,y=0.929937in,left,base]{\color{textcolor}{\rmfamily\fontsize{16.000000}{19.200000}\selectfont\catcode`\^=\active\def^{\ifmmode\sp\else\^{}\fi}\catcode`\%=\active\def%{\%}near-optimal space (Temoa)}}%
\end{pgfscope}%
\end{pgfpicture}%
\makeatother%
\endgroup%
}
  \caption{Compares the least-cost solutions between \acs{temoa}
  and \acs{osier} as well as their sub-optimal spaces. The least-cost solutions
  for \ac{osier} and \ac{temoa} are within 0.5\% of each other.}
  \label{fig:temoa-benchmark-01}
\end{figure}

First, \ac{temoa}'s least-cost solution is slightly better (within 0.5\%) than
\ac{osier}'s in terms of both cost and emissions. This happens because
\ac{temoa} optimizes energy dispatch slightly differently than \ac{osier}. In
particular, the initial storage value for energy storage technologies is a
decision variable in \ac{temoa} and not in \ac{osier}. A second reason for this
discrepancy has to do with convergence. \ac{osier}'s Pareto-front could likely
be improved with a lower convergence tolerance, but this would use additional
computational resources. Although, \ac{temoa} calculated an optimal solution
with slightly lower cost than \ac{osier}, modelers should not place too much
importance on this fact because \acp{esom} should be used to generate insight
rather than answers, due to the nature of the systems being modeled
\cite{decarolis_using_2011}.

Next, the sub-optimal spaces mostly overlap, indicating that \ac{temoa} could
find a solution with lower carbon emissions after sufficient iterations.
However, none of \ac{temoa}'s \ac{mga} solutions fall within \ac{osier}'s
sub-optimal space. This point highlights the necessity for \acl{moo}. The
objective of \ac{mga} is to produce a \textit{diverse subset} of points in the
sub-optimal region. \ac{mga} may capture appealing alternatives for some
unmodeled objective in the original problem, but it cannot guarantee that those
solutions will be an improvement along any other objective axis. This is
especially apparent here, where the least-cost solution happens also to be the
lowest carbon solution, for \ac{temoa}. The relatively small area where the two
\acp{esom} do not overlap is fully explained by the difference in their
least-cost solutions.

Even though \ac{moo} reduces structural uncertainty, it will always exist, as
discussed in Section \ref{section:uncertainty}. Therefore, identifying
alternative solutions by sampling points in the inferior region is still useful.
Figure \ref{fig:temoa-benchmark-02} focuses on the near-optimal space presented
in \ref{fig:temoa-benchmark-01} and shows both the complete set of near-optimal
solutions (green) and some randomly selected points, highlighted in red.

\begin{figure}[h]
  \centering
  \resizebox{0.75\columnwidth}{!}{\input{figures/04_benchmark_chapter/osier_mga_subset_01.pgf}}
  % \includegraphics[width=0.6\columnwidth]{figures/results/osier_mga_subset_01.png}
  % \resizebox{0.6\columnwidth}{!}{\input{figures/results/osier_mga_subset_01.png}}
  \caption{Points within \ac{osier}'s sub-optimal space.}
  \label{fig:temoa-benchmark-02}
\end{figure}

Both Figure \ref{fig:temoa-benchmark-01} and Figure \ref{fig:temoa-benchmark-02}
present solutions in the objective space. However, in order to be prescriptive,
the policy solutions must be formulated according to the decision space. In
other words, described according to the mix of technologies that produced a
solution. Figure \ref{fig:temoa-benchmark-03} presents the spread of results in
the decision space for each model. Figure \ref{fig:temoa-benchmark-03}a shows
the spread of each technology present in \ac{osier}'s Pareto front. Figure
\ref{fig:temoa-benchmark-03}b shows the same, but also includes the randomly
selected points from \ac{osier}'s near-optimal space. Lastly, Figure
\ref{fig:temoa-benchmark-03}c shows the same kind of distribution for
\ac{temoa}'s \ac{mga} solutions. Presented in this way, the design space results
indicate which technologies are always or usually present. Technologies that are
absent in all cases, including the near-optimal solutions, may be safely
ignored. For \ac{osier}, these technologies include both types of coal, biomass,
and largely ignores wind energy. In \ac{temoa}'s results, there are no
technologies that are totally absent. This result is due to the imperative built
into standard \ac{mga} to identify solutions that are maximally different in
design space, whereas \ac{osier} randomly selected points in its inferior
region. This suggests one avenue for improving \ac{osier}.

\newpage
\begin{figure}[ht!]
  \centering
  \resizebox{\columnwidth}{!}{\input{figures/04_benchmark_chapter/temoa_osier_mga_comparison1x3.pgf}}
  \caption{The design spaces for a) points on the Pareto-front in Figure
  \ref{fig:temoa-benchmark-01}, b) selected points in \ac{osier}'s sub-optimal
  space, identified in Figure \ref{fig:temoa-benchmark-02}, and c) points
  generated by \ac{temoa}'s \ac{mga} algorithm shown in Figure
  \ref{fig:temoa-benchmark-01}.}
  \label{fig:temoa-benchmark-03}
\end{figure}

% Natural gas with \ac{ccs} shows up in the randomly selected points in
% \ac{osier}'s sub-optimal region. A geo-political locus for energy
% infrastructure, described in Section \ref{section:energy-system-boundaries}
% offers one possible explanation for this technology since states with
% significant natural gas resources might seek to maintain their influence by
% developing low-carbon technology that still uses natural gas.
 
\FloatBarrier

\subsection{Exercise 3: Four Simultaneous Objectives}
Chapter \ref{chapter:lit-review} showed that conventional \acp{esom} virtually
always model a single objective and that objective is uniformly cost (or a
similar aggregated economic indicator). Further, Section
\ref{section:moo-in-energy} showed that the existing literature employing
\ac{moo} never model more than three objectives simultaneously. The purpose of
this final exercise is to demonstrate that \ac{osier} can optimize many
objectives, thereby providing more context and confidence for the tool. This
exercise minimized four objectives simultaneously: total system cost, lifecycle
carbon emissions, land-use change, and percentage of total energy from
non-renewable energy sources. Renewable energy sources include solar, wind, and
biomass. Although batteries are often used in conjunction with \acp{vre}, they
are not considered ``renewable'' (nor are they a true energy ``source'' since
they store energy from other sources rather than producing their own). For
clarity, the ``percent non-renewable'' objective refers to the penetration of
non-renewable sources as a percentage of the energy produced rather than as a
percentage of the systems total installed capacity. Figure
\ref{fig:4-obj-pareto} shows the objective-space Pareto front for this
4-dimensional problem.

\begin{noteBox}
\textbf{Reading \Aclp{pcp}:} Visualizing the Pareto front for this problem
presents a challenge due to its high dimensionality. Therefore, I present the
results with a novel plot, called a \ac{pcp}. This plot is helpful for
highlighting differences among a small set of solutions with a potentially large
number of dimensions. Figure \ref{fig:4-obj-pareto} and Figure
\ref{fig:4-obj-design} are both \acp{pcp}. Although \acp{pcp} show continuous
lines, they do not show a ``trend''. That is, for a given solution, each
objective takes on a single value that is plotted on its respective vertical
axis. The lines connecting these points simply emphasize that these points
belong to the same solution. Additionally, each objective axis has its own upper
and lower bound because each objective is scaled differently. The \ac{mga}
solutions presented in Figure \ref{fig:4-obj-design-mga} using a boxplot due to
the larger number of solutions included in \ac{mga}. 
\end{noteBox}


\begin{figure}[h]
  \centering
  \resizebox{\columnwidth}{!}{\input{figures/04_benchmark_chapter/4_obj_objective_space.pgf}}
  \caption{The Pareto front for a four objective problem. Extreme values for
  each objective are colored. The gray lines represent solutions on the Pareto
  front that are not extremum.}
  \label{fig:4-obj-pareto}
\end{figure}

Each of the colored lines in Figure \ref{fig:4-obj-pareto} belongs to a solution
with an `extreme' value on the Pareto-front. For instance, the blue line labeled
``Highest Renewable'' has the lowest percentage of non-renewable energy sources
of any solution. The gray lines are simply other points along the Pareto-front.
Figure \ref{fig:4-obj-pareto} shows that minimizing land-use change and
renewable energy maximization are strongly competing objectives, since the other
three extremum are grouped together on those two axes and diametrically opposed
to the ``highest renewable'' solution. Figure \ref{fig:4-obj-design} illustrates
the design space for each extreme solution. 


\begin{figure}[h]
  \centering
  \resizebox{\columnwidth}{!}{\input{figures/04_benchmark_chapter/4_obj_design_space.pgf}}
  \caption{The design space for a four objective problem.}
  \label{fig:4-obj-design}
\end{figure}

Figure \ref{fig:4-obj-design} shows that conventional coal and advanced coal
technologies are largely uninteresting because they make up at most 7\% and 4\%
of a solution's peak demand, respectively. The ``highest renewable'' solution
achieves its goal of reaching approximately 100\% renewable energy (by
percentage of energy produced) with a significant overbuild of wind energy and
batteries, with natural gas and a small amount of coal for reliability.
Interestingly, this solution uses no solar energy, even though solar and wind
are frequently assumed to complement each other.

Figure \ref{fig:4-obj-design-mga} extends the design space results to include
the \ac{mga} solutions. This plot indicates the design preferences for a
middling solution, but hides the relationship among energy technologies. The
most popular technologies in Figure \ref{fig:4-obj-design-mga} are conventional
nuclear, battery storage, and solar panels. The least popular technologies are
wind turbines, biomass, and ``advanced'' coal plants.

\begin{figure}[h]
  \centering
  \resizebox{\columnwidth}{!}{\input{figures/04_benchmark_chapter/4-obj-mga-design-space.pgf}}
  \caption{The design space for a four objective problem including alternative solutions suggested by MGA.}
  \label{fig:4-obj-design-mga}
\end{figure}
