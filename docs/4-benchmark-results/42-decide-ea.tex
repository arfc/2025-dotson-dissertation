\subsection{Exercise 4: Deciding Among Evolutionary Algorithms}

\ac{osier} allows users to choose among a variety of \ac{moo} methods. This is
motivated by the desire for flexibility. However, Exercises 5 and 6 use just one
algorithm, \ac{unsga3} as implemented by \ac{pymoo}. Figure
\ref{fig:algorithm-comparison} justifies this choice by comparing the results of
three \ac{moo} algorithms by showing the respective scatter plots and a density
plot of the points on each axis. The three algorithms are \ac{nsga2},
\ac{nsga3}, and \ac{unsga3}. The first two algorithms used the \ac{deap}
implementation to show the breadth \ac{osier}'s support for different tools.
Since it took \ac{unsga3} 128 generations to converge, I also stopped the other
two algorithms after 128 generations, before converging. The density plot above
the scatter plot shows the density of points along the ``total cost'' objective.
Similarly, the density plot to the right in Figure
\ref{fig:algorithm-comparison} shows the distribution of points for the
``emissions'' objective.

There are a few notable features of Figure \ref{fig:algorithm-comparison}.
First, all three algorithms identified very similar Pareto fronts, the main
differences involve the distribution of points and the extent of their
respective solution sets. Second, the two \ac{deap} algorithms have a greater
number of points along the bottom part of the Pareto front, indicating a greater
sampling over the cost objective. This is further supported by the higher
concentration of points along the lower half of the emission objective's range.
Third, the algorithms implemented by \ac{deap} both have more extreme values
along both axes. All of these features can be attributed to the fact that
neither \ac{nsga2} nor \ac{nsga3} fully converged. Thus, choosing \ac{unsga3}
will be used for the remaining exercises for its faster convergence.

\begin{figure}[ht]
  \centering
  \resizebox{0.75\columnwidth}{!}{\input{figures/04_benchmark_chapter/algorithm_comparison_kde.pgf}}
  \caption{Compares the \ac{moo} algorithms.}
  \label{fig:algorithm-comparison}
\end{figure}

\FloatBarrier
