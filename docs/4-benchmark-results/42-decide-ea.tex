\subsection{Exercise 4: Deciding Among Evolutionary Algorithms}

\ac{osier} allows users to choose among a variety of \ac{moo} methods. This is
motivated by the desire for flexibility. However, Exercises 5 and 6 use just one
algorithm, \ac{unsga3} as implemented by \ac{pymoo} \cite{blank_pymoo_2020}. Figure
\ref{fig:algorithm-comparison} justifies this choice by comparing the results of
three \ac{moo} algorithms by showing the respective scatter plots and a density
plot of the points on each axis. The three algorithms are \ac{nsga2},
\ac{nsga3}, and \ac{unsga3}. The first two algorithms used the \ac{deap} \cite{fortin_deap_2012}
implementation to show the breadth \ac{osier}'s support for different tools.
Since it took \ac{unsga3} 128 generations to converge, I also stopped the other
two algorithms after 128 generations, before converging. The density plot above
the scatter plot shows the density of points along the ``total cost'' objective.
Similarly, the density plot to the right in Figure
\ref{fig:algorithm-comparison} shows the distribution of points for the
``emissions'' objective.

There are a few notable features of Figure \ref{fig:algorithm-comparison}.
First, all three algorithms identified very similar Pareto fronts, the main
differences involve the distribution of points and the extent of their
respective solution sets. Second, the two \ac{deap} algorithms have a greater
number of points along the bottom part of the Pareto front, indicating a greater
sampling over the cost objective. This is further supported by the higher
concentration of points along the lower half of the emission objective's range.
Third, the algorithms implemented by \ac{deap} both have more extreme values
along both axes. All of these features can be attributed to the fact that
neither \ac{nsga2} nor \ac{nsga3} fully converged. Thus, choosing \ac{unsga3}
will be used for the remaining exercises for its faster convergence.

\begin{figure}[ht]
  \centering
  \resizebox{0.75\columnwidth}{!}{%% Creator: Matplotlib, PGF backend
%%
%% To include the figure in your LaTeX document, write
%%   \input{<filename>.pgf}
%%
%% Make sure the required packages are loaded in your preamble
%%   \usepackage{pgf}
%%
%% Also ensure that all the required font packages are loaded; for instance,
%% the lmodern package is sometimes necessary when using math font.
%%   \usepackage{lmodern}
%%
%% Figures using additional raster images can only be included by \input if
%% they are in the same directory as the main LaTeX file. For loading figures
%% from other directories you can use the `import` package
%%   \usepackage{import}
%%
%% and then include the figures with
%%   \import{<path to file>}{<filename>.pgf}
%%
%% Matplotlib used the following preamble
%%   \def\mathdefault#1{#1}
%%   \everymath=\expandafter{\the\everymath\displaystyle}
%%   \IfFileExists{scrextend.sty}{
%%     \usepackage[fontsize=10.000000pt]{scrextend}
%%   }{
%%     \renewcommand{\normalsize}{\fontsize{10.000000}{12.000000}\selectfont}
%%     \normalsize
%%   }
%%   
%%   \makeatletter\@ifpackageloaded{underscore}{}{\usepackage[strings]{underscore}}\makeatother
%%
\begingroup%
\makeatletter%
\begin{pgfpicture}%
\pgfpathrectangle{\pgfpointorigin}{\pgfqpoint{8.052313in}{5.900000in}}%
\pgfusepath{use as bounding box, clip}%
\begin{pgfscope}%
\pgfsetbuttcap%
\pgfsetmiterjoin%
\definecolor{currentfill}{rgb}{1.000000,1.000000,1.000000}%
\pgfsetfillcolor{currentfill}%
\pgfsetlinewidth{0.000000pt}%
\definecolor{currentstroke}{rgb}{0.000000,0.000000,0.000000}%
\pgfsetstrokecolor{currentstroke}%
\pgfsetdash{}{0pt}%
\pgfpathmoveto{\pgfqpoint{0.000000in}{0.000000in}}%
\pgfpathlineto{\pgfqpoint{8.052313in}{0.000000in}}%
\pgfpathlineto{\pgfqpoint{8.052313in}{5.900000in}}%
\pgfpathlineto{\pgfqpoint{0.000000in}{5.900000in}}%
\pgfpathlineto{\pgfqpoint{0.000000in}{0.000000in}}%
\pgfpathclose%
\pgfusepath{fill}%
\end{pgfscope}%
\begin{pgfscope}%
\pgfsetbuttcap%
\pgfsetmiterjoin%
\definecolor{currentfill}{rgb}{1.000000,1.000000,1.000000}%
\pgfsetfillcolor{currentfill}%
\pgfsetlinewidth{0.000000pt}%
\definecolor{currentstroke}{rgb}{0.000000,0.000000,0.000000}%
\pgfsetstrokecolor{currentstroke}%
\pgfsetstrokeopacity{0.000000}%
\pgfsetdash{}{0pt}%
\pgfpathmoveto{\pgfqpoint{0.840504in}{0.670138in}}%
\pgfpathlineto{\pgfqpoint{6.752313in}{0.670138in}}%
\pgfpathlineto{\pgfqpoint{6.752313in}{4.600000in}}%
\pgfpathlineto{\pgfqpoint{0.840504in}{4.600000in}}%
\pgfpathlineto{\pgfqpoint{0.840504in}{0.670138in}}%
\pgfpathclose%
\pgfusepath{fill}%
\end{pgfscope}%
\begin{pgfscope}%
\pgfpathrectangle{\pgfqpoint{0.840504in}{0.670138in}}{\pgfqpoint{5.911808in}{3.929862in}}%
\pgfusepath{clip}%
\pgfsetbuttcap%
\pgfsetroundjoin%
\definecolor{currentfill}{rgb}{0.121569,0.466667,0.705882}%
\pgfsetfillcolor{currentfill}%
\pgfsetlinewidth{1.003750pt}%
\definecolor{currentstroke}{rgb}{0.121569,0.466667,0.705882}%
\pgfsetstrokecolor{currentstroke}%
\pgfsetdash{}{0pt}%
\pgfsys@defobject{currentmarker}{\pgfqpoint{-0.038036in}{-0.038036in}}{\pgfqpoint{0.038036in}{0.038036in}}{%
\pgfpathmoveto{\pgfqpoint{-0.038036in}{-0.038036in}}%
\pgfpathlineto{\pgfqpoint{0.038036in}{-0.038036in}}%
\pgfpathlineto{\pgfqpoint{0.038036in}{0.038036in}}%
\pgfpathlineto{\pgfqpoint{-0.038036in}{0.038036in}}%
\pgfpathlineto{\pgfqpoint{-0.038036in}{-0.038036in}}%
\pgfpathclose%
\pgfusepath{stroke,fill}%
}%
\begin{pgfscope}%
\pgfsys@transformshift{1.418516in}{1.089795in}%
\pgfsys@useobject{currentmarker}{}%
\end{pgfscope}%
\begin{pgfscope}%
\pgfsys@transformshift{2.030903in}{1.071083in}%
\pgfsys@useobject{currentmarker}{}%
\end{pgfscope}%
\begin{pgfscope}%
\pgfsys@transformshift{0.994362in}{1.927786in}%
\pgfsys@useobject{currentmarker}{}%
\end{pgfscope}%
\begin{pgfscope}%
\pgfsys@transformshift{1.039403in}{1.161402in}%
\pgfsys@useobject{currentmarker}{}%
\end{pgfscope}%
\begin{pgfscope}%
\pgfsys@transformshift{1.062702in}{1.120187in}%
\pgfsys@useobject{currentmarker}{}%
\end{pgfscope}%
\begin{pgfscope}%
\pgfsys@transformshift{1.031093in}{1.181097in}%
\pgfsys@useobject{currentmarker}{}%
\end{pgfscope}%
\begin{pgfscope}%
\pgfsys@transformshift{1.003703in}{1.320047in}%
\pgfsys@useobject{currentmarker}{}%
\end{pgfscope}%
\begin{pgfscope}%
\pgfsys@transformshift{1.061588in}{1.122204in}%
\pgfsys@useobject{currentmarker}{}%
\end{pgfscope}%
\begin{pgfscope}%
\pgfsys@transformshift{1.133022in}{1.109025in}%
\pgfsys@useobject{currentmarker}{}%
\end{pgfscope}%
\begin{pgfscope}%
\pgfsys@transformshift{1.106237in}{1.110551in}%
\pgfsys@useobject{currentmarker}{}%
\end{pgfscope}%
\begin{pgfscope}%
\pgfsys@transformshift{1.120671in}{1.109323in}%
\pgfsys@useobject{currentmarker}{}%
\end{pgfscope}%
\begin{pgfscope}%
\pgfsys@transformshift{1.155415in}{1.107156in}%
\pgfsys@useobject{currentmarker}{}%
\end{pgfscope}%
\begin{pgfscope}%
\pgfsys@transformshift{1.250615in}{1.099567in}%
\pgfsys@useobject{currentmarker}{}%
\end{pgfscope}%
\begin{pgfscope}%
\pgfsys@transformshift{1.139707in}{1.107619in}%
\pgfsys@useobject{currentmarker}{}%
\end{pgfscope}%
\begin{pgfscope}%
\pgfsys@transformshift{1.054106in}{1.145228in}%
\pgfsys@useobject{currentmarker}{}%
\end{pgfscope}%
\begin{pgfscope}%
\pgfsys@transformshift{1.018469in}{1.221079in}%
\pgfsys@useobject{currentmarker}{}%
\end{pgfscope}%
\begin{pgfscope}%
\pgfsys@transformshift{1.002530in}{1.335461in}%
\pgfsys@useobject{currentmarker}{}%
\end{pgfscope}%
\begin{pgfscope}%
\pgfsys@transformshift{1.175667in}{1.105791in}%
\pgfsys@useobject{currentmarker}{}%
\end{pgfscope}%
\begin{pgfscope}%
\pgfsys@transformshift{1.112179in}{1.110239in}%
\pgfsys@useobject{currentmarker}{}%
\end{pgfscope}%
\begin{pgfscope}%
\pgfsys@transformshift{1.043547in}{1.151600in}%
\pgfsys@useobject{currentmarker}{}%
\end{pgfscope}%
\begin{pgfscope}%
\pgfsys@transformshift{1.028150in}{1.202993in}%
\pgfsys@useobject{currentmarker}{}%
\end{pgfscope}%
\begin{pgfscope}%
\pgfsys@transformshift{1.060383in}{1.123820in}%
\pgfsys@useobject{currentmarker}{}%
\end{pgfscope}%
\begin{pgfscope}%
\pgfsys@transformshift{1.284782in}{1.097209in}%
\pgfsys@useobject{currentmarker}{}%
\end{pgfscope}%
\begin{pgfscope}%
\pgfsys@transformshift{1.008961in}{1.276585in}%
\pgfsys@useobject{currentmarker}{}%
\end{pgfscope}%
\begin{pgfscope}%
\pgfsys@transformshift{1.191614in}{1.103485in}%
\pgfsys@useobject{currentmarker}{}%
\end{pgfscope}%
\begin{pgfscope}%
\pgfsys@transformshift{0.999689in}{1.430350in}%
\pgfsys@useobject{currentmarker}{}%
\end{pgfscope}%
\begin{pgfscope}%
\pgfsys@transformshift{0.990438in}{3.198321in}%
\pgfsys@useobject{currentmarker}{}%
\end{pgfscope}%
\begin{pgfscope}%
\pgfsys@transformshift{1.334942in}{1.093947in}%
\pgfsys@useobject{currentmarker}{}%
\end{pgfscope}%
\begin{pgfscope}%
\pgfsys@transformshift{1.010375in}{1.261075in}%
\pgfsys@useobject{currentmarker}{}%
\end{pgfscope}%
\begin{pgfscope}%
\pgfsys@transformshift{1.216706in}{1.102175in}%
\pgfsys@useobject{currentmarker}{}%
\end{pgfscope}%
\begin{pgfscope}%
\pgfsys@transformshift{1.563285in}{1.083755in}%
\pgfsys@useobject{currentmarker}{}%
\end{pgfscope}%
\begin{pgfscope}%
\pgfsys@transformshift{1.007368in}{1.288907in}%
\pgfsys@useobject{currentmarker}{}%
\end{pgfscope}%
\begin{pgfscope}%
\pgfsys@transformshift{1.011245in}{1.253675in}%
\pgfsys@useobject{currentmarker}{}%
\end{pgfscope}%
\begin{pgfscope}%
\pgfsys@transformshift{1.055402in}{1.131827in}%
\pgfsys@useobject{currentmarker}{}%
\end{pgfscope}%
\begin{pgfscope}%
\pgfsys@transformshift{1.083825in}{1.112945in}%
\pgfsys@useobject{currentmarker}{}%
\end{pgfscope}%
\begin{pgfscope}%
\pgfsys@transformshift{1.038248in}{1.166014in}%
\pgfsys@useobject{currentmarker}{}%
\end{pgfscope}%
\begin{pgfscope}%
\pgfsys@transformshift{1.010287in}{1.265940in}%
\pgfsys@useobject{currentmarker}{}%
\end{pgfscope}%
\begin{pgfscope}%
\pgfsys@transformshift{1.006609in}{1.293117in}%
\pgfsys@useobject{currentmarker}{}%
\end{pgfscope}%
\begin{pgfscope}%
\pgfsys@transformshift{1.037557in}{1.169695in}%
\pgfsys@useobject{currentmarker}{}%
\end{pgfscope}%
\begin{pgfscope}%
\pgfsys@transformshift{1.079103in}{1.112980in}%
\pgfsys@useobject{currentmarker}{}%
\end{pgfscope}%
\begin{pgfscope}%
\pgfsys@transformshift{1.070103in}{1.114010in}%
\pgfsys@useobject{currentmarker}{}%
\end{pgfscope}%
\begin{pgfscope}%
\pgfsys@transformshift{1.042083in}{1.154755in}%
\pgfsys@useobject{currentmarker}{}%
\end{pgfscope}%
\begin{pgfscope}%
\pgfsys@transformshift{1.043351in}{1.154079in}%
\pgfsys@useobject{currentmarker}{}%
\end{pgfscope}%
\begin{pgfscope}%
\pgfsys@transformshift{1.014746in}{1.240056in}%
\pgfsys@useobject{currentmarker}{}%
\end{pgfscope}%
\begin{pgfscope}%
\pgfsys@transformshift{1.023523in}{1.207541in}%
\pgfsys@useobject{currentmarker}{}%
\end{pgfscope}%
\begin{pgfscope}%
\pgfsys@transformshift{1.057074in}{1.130161in}%
\pgfsys@useobject{currentmarker}{}%
\end{pgfscope}%
\begin{pgfscope}%
\pgfsys@transformshift{1.005869in}{1.317184in}%
\pgfsys@useobject{currentmarker}{}%
\end{pgfscope}%
\begin{pgfscope}%
\pgfsys@transformshift{1.075327in}{1.113102in}%
\pgfsys@useobject{currentmarker}{}%
\end{pgfscope}%
\begin{pgfscope}%
\pgfsys@transformshift{1.028339in}{1.196989in}%
\pgfsys@useobject{currentmarker}{}%
\end{pgfscope}%
\begin{pgfscope}%
\pgfsys@transformshift{1.029345in}{1.182784in}%
\pgfsys@useobject{currentmarker}{}%
\end{pgfscope}%
\begin{pgfscope}%
\pgfsys@transformshift{1.054932in}{1.132787in}%
\pgfsys@useobject{currentmarker}{}%
\end{pgfscope}%
\begin{pgfscope}%
\pgfsys@transformshift{1.065546in}{1.116009in}%
\pgfsys@useobject{currentmarker}{}%
\end{pgfscope}%
\begin{pgfscope}%
\pgfsys@transformshift{1.032281in}{1.176631in}%
\pgfsys@useobject{currentmarker}{}%
\end{pgfscope}%
\begin{pgfscope}%
\pgfsys@transformshift{1.038269in}{1.162056in}%
\pgfsys@useobject{currentmarker}{}%
\end{pgfscope}%
\begin{pgfscope}%
\pgfsys@transformshift{0.996518in}{1.687222in}%
\pgfsys@useobject{currentmarker}{}%
\end{pgfscope}%
\begin{pgfscope}%
\pgfsys@transformshift{1.028612in}{1.186517in}%
\pgfsys@useobject{currentmarker}{}%
\end{pgfscope}%
\begin{pgfscope}%
\pgfsys@transformshift{1.000678in}{1.401090in}%
\pgfsys@useobject{currentmarker}{}%
\end{pgfscope}%
\begin{pgfscope}%
\pgfsys@transformshift{1.002046in}{1.398879in}%
\pgfsys@useobject{currentmarker}{}%
\end{pgfscope}%
\begin{pgfscope}%
\pgfsys@transformshift{0.998005in}{1.548137in}%
\pgfsys@useobject{currentmarker}{}%
\end{pgfscope}%
\begin{pgfscope}%
\pgfsys@transformshift{0.998799in}{1.529234in}%
\pgfsys@useobject{currentmarker}{}%
\end{pgfscope}%
\end{pgfscope}%
\begin{pgfscope}%
\pgfpathrectangle{\pgfqpoint{0.840504in}{0.670138in}}{\pgfqpoint{5.911808in}{3.929862in}}%
\pgfusepath{clip}%
\pgfsetbuttcap%
\pgfsetroundjoin%
\definecolor{currentfill}{rgb}{1.000000,0.498039,0.054902}%
\pgfsetfillcolor{currentfill}%
\pgfsetlinewidth{1.003750pt}%
\definecolor{currentstroke}{rgb}{1.000000,0.498039,0.054902}%
\pgfsetstrokecolor{currentstroke}%
\pgfsetdash{}{0pt}%
\pgfsys@defobject{currentmarker}{\pgfqpoint{-0.036175in}{-0.030772in}}{\pgfqpoint{0.036175in}{0.038036in}}{%
\pgfpathmoveto{\pgfqpoint{0.000000in}{0.038036in}}%
\pgfpathlineto{\pgfqpoint{-0.008540in}{0.011754in}}%
\pgfpathlineto{\pgfqpoint{-0.036175in}{0.011754in}}%
\pgfpathlineto{\pgfqpoint{-0.013817in}{-0.004490in}}%
\pgfpathlineto{\pgfqpoint{-0.022357in}{-0.030772in}}%
\pgfpathlineto{\pgfqpoint{-0.000000in}{-0.014529in}}%
\pgfpathlineto{\pgfqpoint{0.022357in}{-0.030772in}}%
\pgfpathlineto{\pgfqpoint{0.013817in}{-0.004490in}}%
\pgfpathlineto{\pgfqpoint{0.036175in}{0.011754in}}%
\pgfpathlineto{\pgfqpoint{0.008540in}{0.011754in}}%
\pgfpathlineto{\pgfqpoint{0.000000in}{0.038036in}}%
\pgfpathclose%
\pgfusepath{stroke,fill}%
}%
\begin{pgfscope}%
\pgfsys@transformshift{0.963227in}{4.274111in}%
\pgfsys@useobject{currentmarker}{}%
\end{pgfscope}%
\begin{pgfscope}%
\pgfsys@transformshift{0.963263in}{4.274093in}%
\pgfsys@useobject{currentmarker}{}%
\end{pgfscope}%
\begin{pgfscope}%
\pgfsys@transformshift{0.963558in}{4.274091in}%
\pgfsys@useobject{currentmarker}{}%
\end{pgfscope}%
\begin{pgfscope}%
\pgfsys@transformshift{0.964599in}{4.262938in}%
\pgfsys@useobject{currentmarker}{}%
\end{pgfscope}%
\begin{pgfscope}%
\pgfsys@transformshift{0.964929in}{4.262918in}%
\pgfsys@useobject{currentmarker}{}%
\end{pgfscope}%
\begin{pgfscope}%
\pgfsys@transformshift{0.969198in}{4.149343in}%
\pgfsys@useobject{currentmarker}{}%
\end{pgfscope}%
\begin{pgfscope}%
\pgfsys@transformshift{0.970574in}{3.888316in}%
\pgfsys@useobject{currentmarker}{}%
\end{pgfscope}%
\begin{pgfscope}%
\pgfsys@transformshift{0.978103in}{3.111215in}%
\pgfsys@useobject{currentmarker}{}%
\end{pgfscope}%
\begin{pgfscope}%
\pgfsys@transformshift{0.984934in}{2.907321in}%
\pgfsys@useobject{currentmarker}{}%
\end{pgfscope}%
\begin{pgfscope}%
\pgfsys@transformshift{0.986801in}{2.652616in}%
\pgfsys@useobject{currentmarker}{}%
\end{pgfscope}%
\begin{pgfscope}%
\pgfsys@transformshift{0.995218in}{1.185608in}%
\pgfsys@useobject{currentmarker}{}%
\end{pgfscope}%
\begin{pgfscope}%
\pgfsys@transformshift{1.023323in}{1.120602in}%
\pgfsys@useobject{currentmarker}{}%
\end{pgfscope}%
\begin{pgfscope}%
\pgfsys@transformshift{1.038589in}{1.114620in}%
\pgfsys@useobject{currentmarker}{}%
\end{pgfscope}%
\begin{pgfscope}%
\pgfsys@transformshift{1.082132in}{1.114532in}%
\pgfsys@useobject{currentmarker}{}%
\end{pgfscope}%
\begin{pgfscope}%
\pgfsys@transformshift{1.264828in}{1.101357in}%
\pgfsys@useobject{currentmarker}{}%
\end{pgfscope}%
\begin{pgfscope}%
\pgfsys@transformshift{1.303005in}{1.100144in}%
\pgfsys@useobject{currentmarker}{}%
\end{pgfscope}%
\begin{pgfscope}%
\pgfsys@transformshift{1.324570in}{1.100020in}%
\pgfsys@useobject{currentmarker}{}%
\end{pgfscope}%
\begin{pgfscope}%
\pgfsys@transformshift{1.334574in}{1.098925in}%
\pgfsys@useobject{currentmarker}{}%
\end{pgfscope}%
\begin{pgfscope}%
\pgfsys@transformshift{1.346285in}{1.096790in}%
\pgfsys@useobject{currentmarker}{}%
\end{pgfscope}%
\begin{pgfscope}%
\pgfsys@transformshift{1.377598in}{1.094824in}%
\pgfsys@useobject{currentmarker}{}%
\end{pgfscope}%
\begin{pgfscope}%
\pgfsys@transformshift{1.391323in}{1.090427in}%
\pgfsys@useobject{currentmarker}{}%
\end{pgfscope}%
\begin{pgfscope}%
\pgfsys@transformshift{1.403777in}{1.090427in}%
\pgfsys@useobject{currentmarker}{}%
\end{pgfscope}%
\begin{pgfscope}%
\pgfsys@transformshift{1.448572in}{1.086756in}%
\pgfsys@useobject{currentmarker}{}%
\end{pgfscope}%
\begin{pgfscope}%
\pgfsys@transformshift{1.450602in}{1.086629in}%
\pgfsys@useobject{currentmarker}{}%
\end{pgfscope}%
\begin{pgfscope}%
\pgfsys@transformshift{1.452464in}{1.086616in}%
\pgfsys@useobject{currentmarker}{}%
\end{pgfscope}%
\begin{pgfscope}%
\pgfsys@transformshift{1.455736in}{1.086278in}%
\pgfsys@useobject{currentmarker}{}%
\end{pgfscope}%
\begin{pgfscope}%
\pgfsys@transformshift{1.459934in}{1.086265in}%
\pgfsys@useobject{currentmarker}{}%
\end{pgfscope}%
\begin{pgfscope}%
\pgfsys@transformshift{1.463183in}{1.084664in}%
\pgfsys@useobject{currentmarker}{}%
\end{pgfscope}%
\begin{pgfscope}%
\pgfsys@transformshift{1.477616in}{1.084222in}%
\pgfsys@useobject{currentmarker}{}%
\end{pgfscope}%
\begin{pgfscope}%
\pgfsys@transformshift{1.478603in}{1.084106in}%
\pgfsys@useobject{currentmarker}{}%
\end{pgfscope}%
\begin{pgfscope}%
\pgfsys@transformshift{1.478878in}{1.084106in}%
\pgfsys@useobject{currentmarker}{}%
\end{pgfscope}%
\begin{pgfscope}%
\pgfsys@transformshift{1.480012in}{1.083823in}%
\pgfsys@useobject{currentmarker}{}%
\end{pgfscope}%
\begin{pgfscope}%
\pgfsys@transformshift{1.482451in}{1.083812in}%
\pgfsys@useobject{currentmarker}{}%
\end{pgfscope}%
\begin{pgfscope}%
\pgfsys@transformshift{1.484093in}{1.082866in}%
\pgfsys@useobject{currentmarker}{}%
\end{pgfscope}%
\begin{pgfscope}%
\pgfsys@transformshift{1.484692in}{1.082830in}%
\pgfsys@useobject{currentmarker}{}%
\end{pgfscope}%
\begin{pgfscope}%
\pgfsys@transformshift{1.505045in}{1.082204in}%
\pgfsys@useobject{currentmarker}{}%
\end{pgfscope}%
\begin{pgfscope}%
\pgfsys@transformshift{1.568922in}{1.081462in}%
\pgfsys@useobject{currentmarker}{}%
\end{pgfscope}%
\begin{pgfscope}%
\pgfsys@transformshift{1.568923in}{1.081462in}%
\pgfsys@useobject{currentmarker}{}%
\end{pgfscope}%
\begin{pgfscope}%
\pgfsys@transformshift{1.574843in}{1.081462in}%
\pgfsys@useobject{currentmarker}{}%
\end{pgfscope}%
\begin{pgfscope}%
\pgfsys@transformshift{1.592629in}{1.080844in}%
\pgfsys@useobject{currentmarker}{}%
\end{pgfscope}%
\begin{pgfscope}%
\pgfsys@transformshift{1.597367in}{1.080817in}%
\pgfsys@useobject{currentmarker}{}%
\end{pgfscope}%
\begin{pgfscope}%
\pgfsys@transformshift{1.620609in}{1.078873in}%
\pgfsys@useobject{currentmarker}{}%
\end{pgfscope}%
\begin{pgfscope}%
\pgfsys@transformshift{1.658402in}{1.077487in}%
\pgfsys@useobject{currentmarker}{}%
\end{pgfscope}%
\begin{pgfscope}%
\pgfsys@transformshift{1.660546in}{1.077031in}%
\pgfsys@useobject{currentmarker}{}%
\end{pgfscope}%
\begin{pgfscope}%
\pgfsys@transformshift{1.686963in}{1.076647in}%
\pgfsys@useobject{currentmarker}{}%
\end{pgfscope}%
\begin{pgfscope}%
\pgfsys@transformshift{1.697321in}{1.076367in}%
\pgfsys@useobject{currentmarker}{}%
\end{pgfscope}%
\begin{pgfscope}%
\pgfsys@transformshift{1.716525in}{1.076115in}%
\pgfsys@useobject{currentmarker}{}%
\end{pgfscope}%
\begin{pgfscope}%
\pgfsys@transformshift{1.724449in}{1.075900in}%
\pgfsys@useobject{currentmarker}{}%
\end{pgfscope}%
\begin{pgfscope}%
\pgfsys@transformshift{1.728245in}{1.075803in}%
\pgfsys@useobject{currentmarker}{}%
\end{pgfscope}%
\begin{pgfscope}%
\pgfsys@transformshift{1.766500in}{1.075792in}%
\pgfsys@useobject{currentmarker}{}%
\end{pgfscope}%
\begin{pgfscope}%
\pgfsys@transformshift{1.766992in}{1.075791in}%
\pgfsys@useobject{currentmarker}{}%
\end{pgfscope}%
\begin{pgfscope}%
\pgfsys@transformshift{1.767290in}{1.075768in}%
\pgfsys@useobject{currentmarker}{}%
\end{pgfscope}%
\begin{pgfscope}%
\pgfsys@transformshift{1.926935in}{1.074691in}%
\pgfsys@useobject{currentmarker}{}%
\end{pgfscope}%
\begin{pgfscope}%
\pgfsys@transformshift{1.947114in}{1.072368in}%
\pgfsys@useobject{currentmarker}{}%
\end{pgfscope}%
\begin{pgfscope}%
\pgfsys@transformshift{1.970932in}{1.072278in}%
\pgfsys@useobject{currentmarker}{}%
\end{pgfscope}%
\begin{pgfscope}%
\pgfsys@transformshift{1.984762in}{1.072114in}%
\pgfsys@useobject{currentmarker}{}%
\end{pgfscope}%
\begin{pgfscope}%
\pgfsys@transformshift{2.075979in}{1.072084in}%
\pgfsys@useobject{currentmarker}{}%
\end{pgfscope}%
\begin{pgfscope}%
\pgfsys@transformshift{2.078268in}{1.071282in}%
\pgfsys@useobject{currentmarker}{}%
\end{pgfscope}%
\begin{pgfscope}%
\pgfsys@transformshift{2.080214in}{1.071232in}%
\pgfsys@useobject{currentmarker}{}%
\end{pgfscope}%
\begin{pgfscope}%
\pgfsys@transformshift{2.080220in}{1.071231in}%
\pgfsys@useobject{currentmarker}{}%
\end{pgfscope}%
\begin{pgfscope}%
\pgfsys@transformshift{2.080251in}{1.071231in}%
\pgfsys@useobject{currentmarker}{}%
\end{pgfscope}%
\begin{pgfscope}%
\pgfsys@transformshift{2.084391in}{1.071045in}%
\pgfsys@useobject{currentmarker}{}%
\end{pgfscope}%
\begin{pgfscope}%
\pgfsys@transformshift{2.086304in}{1.071044in}%
\pgfsys@useobject{currentmarker}{}%
\end{pgfscope}%
\begin{pgfscope}%
\pgfsys@transformshift{2.109970in}{1.071044in}%
\pgfsys@useobject{currentmarker}{}%
\end{pgfscope}%
\begin{pgfscope}%
\pgfsys@transformshift{2.205747in}{1.069156in}%
\pgfsys@useobject{currentmarker}{}%
\end{pgfscope}%
\begin{pgfscope}%
\pgfsys@transformshift{2.215174in}{1.069102in}%
\pgfsys@useobject{currentmarker}{}%
\end{pgfscope}%
\begin{pgfscope}%
\pgfsys@transformshift{2.223618in}{1.069042in}%
\pgfsys@useobject{currentmarker}{}%
\end{pgfscope}%
\begin{pgfscope}%
\pgfsys@transformshift{2.383143in}{1.069030in}%
\pgfsys@useobject{currentmarker}{}%
\end{pgfscope}%
\begin{pgfscope}%
\pgfsys@transformshift{2.741740in}{1.067330in}%
\pgfsys@useobject{currentmarker}{}%
\end{pgfscope}%
\begin{pgfscope}%
\pgfsys@transformshift{2.750342in}{1.067281in}%
\pgfsys@useobject{currentmarker}{}%
\end{pgfscope}%
\begin{pgfscope}%
\pgfsys@transformshift{2.753656in}{1.067224in}%
\pgfsys@useobject{currentmarker}{}%
\end{pgfscope}%
\begin{pgfscope}%
\pgfsys@transformshift{2.762932in}{1.066945in}%
\pgfsys@useobject{currentmarker}{}%
\end{pgfscope}%
\begin{pgfscope}%
\pgfsys@transformshift{2.771464in}{1.066897in}%
\pgfsys@useobject{currentmarker}{}%
\end{pgfscope}%
\begin{pgfscope}%
\pgfsys@transformshift{2.772052in}{1.066868in}%
\pgfsys@useobject{currentmarker}{}%
\end{pgfscope}%
\begin{pgfscope}%
\pgfsys@transformshift{2.772052in}{1.066868in}%
\pgfsys@useobject{currentmarker}{}%
\end{pgfscope}%
\begin{pgfscope}%
\pgfsys@transformshift{2.772094in}{1.066867in}%
\pgfsys@useobject{currentmarker}{}%
\end{pgfscope}%
\begin{pgfscope}%
\pgfsys@transformshift{2.772118in}{1.066866in}%
\pgfsys@useobject{currentmarker}{}%
\end{pgfscope}%
\begin{pgfscope}%
\pgfsys@transformshift{2.772119in}{1.066866in}%
\pgfsys@useobject{currentmarker}{}%
\end{pgfscope}%
\begin{pgfscope}%
\pgfsys@transformshift{2.777491in}{1.066689in}%
\pgfsys@useobject{currentmarker}{}%
\end{pgfscope}%
\begin{pgfscope}%
\pgfsys@transformshift{2.787731in}{1.066552in}%
\pgfsys@useobject{currentmarker}{}%
\end{pgfscope}%
\begin{pgfscope}%
\pgfsys@transformshift{2.792244in}{1.066469in}%
\pgfsys@useobject{currentmarker}{}%
\end{pgfscope}%
\begin{pgfscope}%
\pgfsys@transformshift{2.792342in}{1.066469in}%
\pgfsys@useobject{currentmarker}{}%
\end{pgfscope}%
\begin{pgfscope}%
\pgfsys@transformshift{2.813119in}{1.066068in}%
\pgfsys@useobject{currentmarker}{}%
\end{pgfscope}%
\begin{pgfscope}%
\pgfsys@transformshift{2.813477in}{1.066052in}%
\pgfsys@useobject{currentmarker}{}%
\end{pgfscope}%
\begin{pgfscope}%
\pgfsys@transformshift{2.822480in}{1.065696in}%
\pgfsys@useobject{currentmarker}{}%
\end{pgfscope}%
\begin{pgfscope}%
\pgfsys@transformshift{2.827835in}{1.065420in}%
\pgfsys@useobject{currentmarker}{}%
\end{pgfscope}%
\begin{pgfscope}%
\pgfsys@transformshift{2.847035in}{1.065253in}%
\pgfsys@useobject{currentmarker}{}%
\end{pgfscope}%
\begin{pgfscope}%
\pgfsys@transformshift{2.854910in}{1.065119in}%
\pgfsys@useobject{currentmarker}{}%
\end{pgfscope}%
\begin{pgfscope}%
\pgfsys@transformshift{2.874033in}{1.064841in}%
\pgfsys@useobject{currentmarker}{}%
\end{pgfscope}%
\begin{pgfscope}%
\pgfsys@transformshift{2.874954in}{1.064829in}%
\pgfsys@useobject{currentmarker}{}%
\end{pgfscope}%
\begin{pgfscope}%
\pgfsys@transformshift{2.876587in}{1.064769in}%
\pgfsys@useobject{currentmarker}{}%
\end{pgfscope}%
\begin{pgfscope}%
\pgfsys@transformshift{2.882375in}{1.064711in}%
\pgfsys@useobject{currentmarker}{}%
\end{pgfscope}%
\begin{pgfscope}%
\pgfsys@transformshift{2.882476in}{1.064681in}%
\pgfsys@useobject{currentmarker}{}%
\end{pgfscope}%
\begin{pgfscope}%
\pgfsys@transformshift{2.920528in}{1.064610in}%
\pgfsys@useobject{currentmarker}{}%
\end{pgfscope}%
\begin{pgfscope}%
\pgfsys@transformshift{2.934019in}{1.064103in}%
\pgfsys@useobject{currentmarker}{}%
\end{pgfscope}%
\begin{pgfscope}%
\pgfsys@transformshift{2.963978in}{1.064094in}%
\pgfsys@useobject{currentmarker}{}%
\end{pgfscope}%
\begin{pgfscope}%
\pgfsys@transformshift{2.964684in}{1.064083in}%
\pgfsys@useobject{currentmarker}{}%
\end{pgfscope}%
\begin{pgfscope}%
\pgfsys@transformshift{2.964719in}{1.064052in}%
\pgfsys@useobject{currentmarker}{}%
\end{pgfscope}%
\begin{pgfscope}%
\pgfsys@transformshift{2.976951in}{1.064039in}%
\pgfsys@useobject{currentmarker}{}%
\end{pgfscope}%
\begin{pgfscope}%
\pgfsys@transformshift{2.979762in}{1.063973in}%
\pgfsys@useobject{currentmarker}{}%
\end{pgfscope}%
\begin{pgfscope}%
\pgfsys@transformshift{3.004863in}{1.063810in}%
\pgfsys@useobject{currentmarker}{}%
\end{pgfscope}%
\begin{pgfscope}%
\pgfsys@transformshift{3.031188in}{1.063803in}%
\pgfsys@useobject{currentmarker}{}%
\end{pgfscope}%
\begin{pgfscope}%
\pgfsys@transformshift{3.047864in}{1.063636in}%
\pgfsys@useobject{currentmarker}{}%
\end{pgfscope}%
\begin{pgfscope}%
\pgfsys@transformshift{3.054666in}{1.063531in}%
\pgfsys@useobject{currentmarker}{}%
\end{pgfscope}%
\begin{pgfscope}%
\pgfsys@transformshift{3.055743in}{1.063529in}%
\pgfsys@useobject{currentmarker}{}%
\end{pgfscope}%
\begin{pgfscope}%
\pgfsys@transformshift{3.059460in}{1.063500in}%
\pgfsys@useobject{currentmarker}{}%
\end{pgfscope}%
\begin{pgfscope}%
\pgfsys@transformshift{3.076386in}{1.063276in}%
\pgfsys@useobject{currentmarker}{}%
\end{pgfscope}%
\begin{pgfscope}%
\pgfsys@transformshift{3.079713in}{1.063270in}%
\pgfsys@useobject{currentmarker}{}%
\end{pgfscope}%
\begin{pgfscope}%
\pgfsys@transformshift{3.132051in}{1.062669in}%
\pgfsys@useobject{currentmarker}{}%
\end{pgfscope}%
\begin{pgfscope}%
\pgfsys@transformshift{3.181935in}{1.062543in}%
\pgfsys@useobject{currentmarker}{}%
\end{pgfscope}%
\begin{pgfscope}%
\pgfsys@transformshift{3.208867in}{1.062431in}%
\pgfsys@useobject{currentmarker}{}%
\end{pgfscope}%
\begin{pgfscope}%
\pgfsys@transformshift{3.298038in}{1.062430in}%
\pgfsys@useobject{currentmarker}{}%
\end{pgfscope}%
\begin{pgfscope}%
\pgfsys@transformshift{3.321491in}{1.062427in}%
\pgfsys@useobject{currentmarker}{}%
\end{pgfscope}%
\begin{pgfscope}%
\pgfsys@transformshift{3.380661in}{1.062269in}%
\pgfsys@useobject{currentmarker}{}%
\end{pgfscope}%
\begin{pgfscope}%
\pgfsys@transformshift{3.452729in}{1.061372in}%
\pgfsys@useobject{currentmarker}{}%
\end{pgfscope}%
\begin{pgfscope}%
\pgfsys@transformshift{3.457815in}{1.060951in}%
\pgfsys@useobject{currentmarker}{}%
\end{pgfscope}%
\begin{pgfscope}%
\pgfsys@transformshift{3.477407in}{1.060772in}%
\pgfsys@useobject{currentmarker}{}%
\end{pgfscope}%
\begin{pgfscope}%
\pgfsys@transformshift{3.482615in}{1.060764in}%
\pgfsys@useobject{currentmarker}{}%
\end{pgfscope}%
\begin{pgfscope}%
\pgfsys@transformshift{3.490248in}{1.060727in}%
\pgfsys@useobject{currentmarker}{}%
\end{pgfscope}%
\begin{pgfscope}%
\pgfsys@transformshift{3.496699in}{1.060537in}%
\pgfsys@useobject{currentmarker}{}%
\end{pgfscope}%
\begin{pgfscope}%
\pgfsys@transformshift{3.499237in}{1.060491in}%
\pgfsys@useobject{currentmarker}{}%
\end{pgfscope}%
\begin{pgfscope}%
\pgfsys@transformshift{3.510539in}{1.060449in}%
\pgfsys@useobject{currentmarker}{}%
\end{pgfscope}%
\begin{pgfscope}%
\pgfsys@transformshift{3.512564in}{1.060424in}%
\pgfsys@useobject{currentmarker}{}%
\end{pgfscope}%
\begin{pgfscope}%
\pgfsys@transformshift{3.516999in}{1.060309in}%
\pgfsys@useobject{currentmarker}{}%
\end{pgfscope}%
\begin{pgfscope}%
\pgfsys@transformshift{3.518716in}{1.060286in}%
\pgfsys@useobject{currentmarker}{}%
\end{pgfscope}%
\begin{pgfscope}%
\pgfsys@transformshift{3.534539in}{1.060285in}%
\pgfsys@useobject{currentmarker}{}%
\end{pgfscope}%
\begin{pgfscope}%
\pgfsys@transformshift{3.539920in}{1.060116in}%
\pgfsys@useobject{currentmarker}{}%
\end{pgfscope}%
\begin{pgfscope}%
\pgfsys@transformshift{3.551229in}{1.060084in}%
\pgfsys@useobject{currentmarker}{}%
\end{pgfscope}%
\begin{pgfscope}%
\pgfsys@transformshift{3.559556in}{1.060065in}%
\pgfsys@useobject{currentmarker}{}%
\end{pgfscope}%
\begin{pgfscope}%
\pgfsys@transformshift{3.575304in}{1.059947in}%
\pgfsys@useobject{currentmarker}{}%
\end{pgfscope}%
\begin{pgfscope}%
\pgfsys@transformshift{3.633265in}{1.059791in}%
\pgfsys@useobject{currentmarker}{}%
\end{pgfscope}%
\begin{pgfscope}%
\pgfsys@transformshift{3.641775in}{1.059759in}%
\pgfsys@useobject{currentmarker}{}%
\end{pgfscope}%
\begin{pgfscope}%
\pgfsys@transformshift{3.645452in}{1.059759in}%
\pgfsys@useobject{currentmarker}{}%
\end{pgfscope}%
\begin{pgfscope}%
\pgfsys@transformshift{3.664500in}{1.059750in}%
\pgfsys@useobject{currentmarker}{}%
\end{pgfscope}%
\begin{pgfscope}%
\pgfsys@transformshift{3.664825in}{1.059750in}%
\pgfsys@useobject{currentmarker}{}%
\end{pgfscope}%
\begin{pgfscope}%
\pgfsys@transformshift{3.760256in}{1.059746in}%
\pgfsys@useobject{currentmarker}{}%
\end{pgfscope}%
\begin{pgfscope}%
\pgfsys@transformshift{3.762750in}{1.059255in}%
\pgfsys@useobject{currentmarker}{}%
\end{pgfscope}%
\begin{pgfscope}%
\pgfsys@transformshift{3.789989in}{1.059233in}%
\pgfsys@useobject{currentmarker}{}%
\end{pgfscope}%
\begin{pgfscope}%
\pgfsys@transformshift{3.791286in}{1.059158in}%
\pgfsys@useobject{currentmarker}{}%
\end{pgfscope}%
\begin{pgfscope}%
\pgfsys@transformshift{3.814587in}{1.059126in}%
\pgfsys@useobject{currentmarker}{}%
\end{pgfscope}%
\begin{pgfscope}%
\pgfsys@transformshift{3.827245in}{1.059125in}%
\pgfsys@useobject{currentmarker}{}%
\end{pgfscope}%
\begin{pgfscope}%
\pgfsys@transformshift{3.830442in}{1.059125in}%
\pgfsys@useobject{currentmarker}{}%
\end{pgfscope}%
\begin{pgfscope}%
\pgfsys@transformshift{3.843727in}{1.059124in}%
\pgfsys@useobject{currentmarker}{}%
\end{pgfscope}%
\begin{pgfscope}%
\pgfsys@transformshift{3.932646in}{1.059074in}%
\pgfsys@useobject{currentmarker}{}%
\end{pgfscope}%
\begin{pgfscope}%
\pgfsys@transformshift{4.435247in}{1.058621in}%
\pgfsys@useobject{currentmarker}{}%
\end{pgfscope}%
\begin{pgfscope}%
\pgfsys@transformshift{4.503505in}{1.058621in}%
\pgfsys@useobject{currentmarker}{}%
\end{pgfscope}%
\begin{pgfscope}%
\pgfsys@transformshift{4.757331in}{1.058556in}%
\pgfsys@useobject{currentmarker}{}%
\end{pgfscope}%
\begin{pgfscope}%
\pgfsys@transformshift{5.212460in}{1.058552in}%
\pgfsys@useobject{currentmarker}{}%
\end{pgfscope}%
\begin{pgfscope}%
\pgfsys@transformshift{6.502961in}{1.058473in}%
\pgfsys@useobject{currentmarker}{}%
\end{pgfscope}%
\begin{pgfscope}%
\pgfsys@transformshift{6.715212in}{1.058457in}%
\pgfsys@useobject{currentmarker}{}%
\end{pgfscope}%
\end{pgfscope}%
\begin{pgfscope}%
\pgfpathrectangle{\pgfqpoint{0.840504in}{0.670138in}}{\pgfqpoint{5.911808in}{3.929862in}}%
\pgfusepath{clip}%
\pgfsetbuttcap%
\pgfsetroundjoin%
\definecolor{currentfill}{rgb}{0.580392,0.403922,0.741176}%
\pgfsetfillcolor{currentfill}%
\pgfsetlinewidth{1.003750pt}%
\definecolor{currentstroke}{rgb}{0.580392,0.403922,0.741176}%
\pgfsetstrokecolor{currentstroke}%
\pgfsetdash{}{0pt}%
\pgfsys@defobject{currentmarker}{\pgfqpoint{-0.038036in}{-0.038036in}}{\pgfqpoint{0.038036in}{0.038036in}}{%
\pgfpathmoveto{\pgfqpoint{0.000000in}{-0.038036in}}%
\pgfpathcurveto{\pgfqpoint{0.010087in}{-0.038036in}}{\pgfqpoint{0.019763in}{-0.034029in}}{\pgfqpoint{0.026896in}{-0.026896in}}%
\pgfpathcurveto{\pgfqpoint{0.034029in}{-0.019763in}}{\pgfqpoint{0.038036in}{-0.010087in}}{\pgfqpoint{0.038036in}{0.000000in}}%
\pgfpathcurveto{\pgfqpoint{0.038036in}{0.010087in}}{\pgfqpoint{0.034029in}{0.019763in}}{\pgfqpoint{0.026896in}{0.026896in}}%
\pgfpathcurveto{\pgfqpoint{0.019763in}{0.034029in}}{\pgfqpoint{0.010087in}{0.038036in}}{\pgfqpoint{0.000000in}{0.038036in}}%
\pgfpathcurveto{\pgfqpoint{-0.010087in}{0.038036in}}{\pgfqpoint{-0.019763in}{0.034029in}}{\pgfqpoint{-0.026896in}{0.026896in}}%
\pgfpathcurveto{\pgfqpoint{-0.034029in}{0.019763in}}{\pgfqpoint{-0.038036in}{0.010087in}}{\pgfqpoint{-0.038036in}{0.000000in}}%
\pgfpathcurveto{\pgfqpoint{-0.038036in}{-0.010087in}}{\pgfqpoint{-0.034029in}{-0.019763in}}{\pgfqpoint{-0.026896in}{-0.026896in}}%
\pgfpathcurveto{\pgfqpoint{-0.019763in}{-0.034029in}}{\pgfqpoint{-0.010087in}{-0.038036in}}{\pgfqpoint{0.000000in}{-0.038036in}}%
\pgfpathlineto{\pgfqpoint{0.000000in}{-0.038036in}}%
\pgfpathclose%
\pgfusepath{stroke,fill}%
}%
\begin{pgfscope}%
\pgfsys@transformshift{0.990991in}{3.992660in}%
\pgfsys@useobject{currentmarker}{}%
\end{pgfscope}%
\begin{pgfscope}%
\pgfsys@transformshift{1.015744in}{3.540547in}%
\pgfsys@useobject{currentmarker}{}%
\end{pgfscope}%
\begin{pgfscope}%
\pgfsys@transformshift{1.016253in}{2.084036in}%
\pgfsys@useobject{currentmarker}{}%
\end{pgfscope}%
\begin{pgfscope}%
\pgfsys@transformshift{1.024811in}{1.182798in}%
\pgfsys@useobject{currentmarker}{}%
\end{pgfscope}%
\begin{pgfscope}%
\pgfsys@transformshift{1.025718in}{1.182250in}%
\pgfsys@useobject{currentmarker}{}%
\end{pgfscope}%
\begin{pgfscope}%
\pgfsys@transformshift{1.104725in}{1.117735in}%
\pgfsys@useobject{currentmarker}{}%
\end{pgfscope}%
\begin{pgfscope}%
\pgfsys@transformshift{1.153053in}{1.109846in}%
\pgfsys@useobject{currentmarker}{}%
\end{pgfscope}%
\begin{pgfscope}%
\pgfsys@transformshift{1.165893in}{1.109510in}%
\pgfsys@useobject{currentmarker}{}%
\end{pgfscope}%
\begin{pgfscope}%
\pgfsys@transformshift{1.167729in}{1.109501in}%
\pgfsys@useobject{currentmarker}{}%
\end{pgfscope}%
\begin{pgfscope}%
\pgfsys@transformshift{1.171321in}{1.109453in}%
\pgfsys@useobject{currentmarker}{}%
\end{pgfscope}%
\begin{pgfscope}%
\pgfsys@transformshift{1.180938in}{1.108448in}%
\pgfsys@useobject{currentmarker}{}%
\end{pgfscope}%
\begin{pgfscope}%
\pgfsys@transformshift{1.182870in}{1.106946in}%
\pgfsys@useobject{currentmarker}{}%
\end{pgfscope}%
\begin{pgfscope}%
\pgfsys@transformshift{1.186252in}{1.106844in}%
\pgfsys@useobject{currentmarker}{}%
\end{pgfscope}%
\begin{pgfscope}%
\pgfsys@transformshift{1.187889in}{1.106749in}%
\pgfsys@useobject{currentmarker}{}%
\end{pgfscope}%
\begin{pgfscope}%
\pgfsys@transformshift{1.198560in}{1.105123in}%
\pgfsys@useobject{currentmarker}{}%
\end{pgfscope}%
\begin{pgfscope}%
\pgfsys@transformshift{1.201589in}{1.105081in}%
\pgfsys@useobject{currentmarker}{}%
\end{pgfscope}%
\begin{pgfscope}%
\pgfsys@transformshift{1.217148in}{1.104189in}%
\pgfsys@useobject{currentmarker}{}%
\end{pgfscope}%
\begin{pgfscope}%
\pgfsys@transformshift{1.219352in}{1.102858in}%
\pgfsys@useobject{currentmarker}{}%
\end{pgfscope}%
\begin{pgfscope}%
\pgfsys@transformshift{1.253596in}{1.101418in}%
\pgfsys@useobject{currentmarker}{}%
\end{pgfscope}%
\begin{pgfscope}%
\pgfsys@transformshift{1.268964in}{1.098088in}%
\pgfsys@useobject{currentmarker}{}%
\end{pgfscope}%
\begin{pgfscope}%
\pgfsys@transformshift{1.270722in}{1.096864in}%
\pgfsys@useobject{currentmarker}{}%
\end{pgfscope}%
\begin{pgfscope}%
\pgfsys@transformshift{1.401213in}{1.090419in}%
\pgfsys@useobject{currentmarker}{}%
\end{pgfscope}%
\begin{pgfscope}%
\pgfsys@transformshift{1.413547in}{1.089803in}%
\pgfsys@useobject{currentmarker}{}%
\end{pgfscope}%
\begin{pgfscope}%
\pgfsys@transformshift{1.642186in}{1.086053in}%
\pgfsys@useobject{currentmarker}{}%
\end{pgfscope}%
\begin{pgfscope}%
\pgfsys@transformshift{1.663156in}{1.084699in}%
\pgfsys@useobject{currentmarker}{}%
\end{pgfscope}%
\begin{pgfscope}%
\pgfsys@transformshift{1.669991in}{1.083908in}%
\pgfsys@useobject{currentmarker}{}%
\end{pgfscope}%
\begin{pgfscope}%
\pgfsys@transformshift{1.698362in}{1.083500in}%
\pgfsys@useobject{currentmarker}{}%
\end{pgfscope}%
\begin{pgfscope}%
\pgfsys@transformshift{1.705456in}{1.083409in}%
\pgfsys@useobject{currentmarker}{}%
\end{pgfscope}%
\begin{pgfscope}%
\pgfsys@transformshift{1.708150in}{1.083076in}%
\pgfsys@useobject{currentmarker}{}%
\end{pgfscope}%
\begin{pgfscope}%
\pgfsys@transformshift{1.765570in}{1.078953in}%
\pgfsys@useobject{currentmarker}{}%
\end{pgfscope}%
\begin{pgfscope}%
\pgfsys@transformshift{1.765965in}{1.076734in}%
\pgfsys@useobject{currentmarker}{}%
\end{pgfscope}%
\begin{pgfscope}%
\pgfsys@transformshift{1.838005in}{1.075869in}%
\pgfsys@useobject{currentmarker}{}%
\end{pgfscope}%
\begin{pgfscope}%
\pgfsys@transformshift{1.901245in}{1.074793in}%
\pgfsys@useobject{currentmarker}{}%
\end{pgfscope}%
\begin{pgfscope}%
\pgfsys@transformshift{1.926446in}{1.074207in}%
\pgfsys@useobject{currentmarker}{}%
\end{pgfscope}%
\begin{pgfscope}%
\pgfsys@transformshift{1.931223in}{1.073403in}%
\pgfsys@useobject{currentmarker}{}%
\end{pgfscope}%
\begin{pgfscope}%
\pgfsys@transformshift{1.939128in}{1.072930in}%
\pgfsys@useobject{currentmarker}{}%
\end{pgfscope}%
\begin{pgfscope}%
\pgfsys@transformshift{1.952583in}{1.072800in}%
\pgfsys@useobject{currentmarker}{}%
\end{pgfscope}%
\begin{pgfscope}%
\pgfsys@transformshift{1.961654in}{1.072471in}%
\pgfsys@useobject{currentmarker}{}%
\end{pgfscope}%
\begin{pgfscope}%
\pgfsys@transformshift{1.969006in}{1.072287in}%
\pgfsys@useobject{currentmarker}{}%
\end{pgfscope}%
\begin{pgfscope}%
\pgfsys@transformshift{1.985707in}{1.071769in}%
\pgfsys@useobject{currentmarker}{}%
\end{pgfscope}%
\begin{pgfscope}%
\pgfsys@transformshift{2.025269in}{1.071765in}%
\pgfsys@useobject{currentmarker}{}%
\end{pgfscope}%
\begin{pgfscope}%
\pgfsys@transformshift{2.036417in}{1.071749in}%
\pgfsys@useobject{currentmarker}{}%
\end{pgfscope}%
\begin{pgfscope}%
\pgfsys@transformshift{2.039048in}{1.071682in}%
\pgfsys@useobject{currentmarker}{}%
\end{pgfscope}%
\begin{pgfscope}%
\pgfsys@transformshift{2.041188in}{1.071310in}%
\pgfsys@useobject{currentmarker}{}%
\end{pgfscope}%
\begin{pgfscope}%
\pgfsys@transformshift{2.098575in}{1.070884in}%
\pgfsys@useobject{currentmarker}{}%
\end{pgfscope}%
\begin{pgfscope}%
\pgfsys@transformshift{2.150972in}{1.070419in}%
\pgfsys@useobject{currentmarker}{}%
\end{pgfscope}%
\begin{pgfscope}%
\pgfsys@transformshift{2.209427in}{1.070296in}%
\pgfsys@useobject{currentmarker}{}%
\end{pgfscope}%
\begin{pgfscope}%
\pgfsys@transformshift{2.640377in}{1.067949in}%
\pgfsys@useobject{currentmarker}{}%
\end{pgfscope}%
\begin{pgfscope}%
\pgfsys@transformshift{2.674239in}{1.067006in}%
\pgfsys@useobject{currentmarker}{}%
\end{pgfscope}%
\begin{pgfscope}%
\pgfsys@transformshift{2.693311in}{1.066862in}%
\pgfsys@useobject{currentmarker}{}%
\end{pgfscope}%
\begin{pgfscope}%
\pgfsys@transformshift{2.710732in}{1.066849in}%
\pgfsys@useobject{currentmarker}{}%
\end{pgfscope}%
\begin{pgfscope}%
\pgfsys@transformshift{2.720203in}{1.066631in}%
\pgfsys@useobject{currentmarker}{}%
\end{pgfscope}%
\begin{pgfscope}%
\pgfsys@transformshift{2.755449in}{1.066576in}%
\pgfsys@useobject{currentmarker}{}%
\end{pgfscope}%
\begin{pgfscope}%
\pgfsys@transformshift{2.761613in}{1.066463in}%
\pgfsys@useobject{currentmarker}{}%
\end{pgfscope}%
\begin{pgfscope}%
\pgfsys@transformshift{2.831252in}{1.066389in}%
\pgfsys@useobject{currentmarker}{}%
\end{pgfscope}%
\begin{pgfscope}%
\pgfsys@transformshift{2.845869in}{1.066308in}%
\pgfsys@useobject{currentmarker}{}%
\end{pgfscope}%
\begin{pgfscope}%
\pgfsys@transformshift{2.885251in}{1.066304in}%
\pgfsys@useobject{currentmarker}{}%
\end{pgfscope}%
\begin{pgfscope}%
\pgfsys@transformshift{2.885383in}{1.066254in}%
\pgfsys@useobject{currentmarker}{}%
\end{pgfscope}%
\begin{pgfscope}%
\pgfsys@transformshift{2.960181in}{1.065054in}%
\pgfsys@useobject{currentmarker}{}%
\end{pgfscope}%
\begin{pgfscope}%
\pgfsys@transformshift{3.001270in}{1.064785in}%
\pgfsys@useobject{currentmarker}{}%
\end{pgfscope}%
\begin{pgfscope}%
\pgfsys@transformshift{3.007176in}{1.064394in}%
\pgfsys@useobject{currentmarker}{}%
\end{pgfscope}%
\begin{pgfscope}%
\pgfsys@transformshift{3.016127in}{1.064087in}%
\pgfsys@useobject{currentmarker}{}%
\end{pgfscope}%
\begin{pgfscope}%
\pgfsys@transformshift{3.044689in}{1.063798in}%
\pgfsys@useobject{currentmarker}{}%
\end{pgfscope}%
\begin{pgfscope}%
\pgfsys@transformshift{3.066459in}{1.063659in}%
\pgfsys@useobject{currentmarker}{}%
\end{pgfscope}%
\begin{pgfscope}%
\pgfsys@transformshift{3.121624in}{1.063656in}%
\pgfsys@useobject{currentmarker}{}%
\end{pgfscope}%
\begin{pgfscope}%
\pgfsys@transformshift{3.135076in}{1.063541in}%
\pgfsys@useobject{currentmarker}{}%
\end{pgfscope}%
\begin{pgfscope}%
\pgfsys@transformshift{3.492654in}{1.062394in}%
\pgfsys@useobject{currentmarker}{}%
\end{pgfscope}%
\begin{pgfscope}%
\pgfsys@transformshift{3.557749in}{1.062342in}%
\pgfsys@useobject{currentmarker}{}%
\end{pgfscope}%
\begin{pgfscope}%
\pgfsys@transformshift{3.581077in}{1.062001in}%
\pgfsys@useobject{currentmarker}{}%
\end{pgfscope}%
\begin{pgfscope}%
\pgfsys@transformshift{3.600516in}{1.061645in}%
\pgfsys@useobject{currentmarker}{}%
\end{pgfscope}%
\begin{pgfscope}%
\pgfsys@transformshift{3.606495in}{1.060670in}%
\pgfsys@useobject{currentmarker}{}%
\end{pgfscope}%
\begin{pgfscope}%
\pgfsys@transformshift{3.649355in}{1.060274in}%
\pgfsys@useobject{currentmarker}{}%
\end{pgfscope}%
\begin{pgfscope}%
\pgfsys@transformshift{3.740558in}{1.059271in}%
\pgfsys@useobject{currentmarker}{}%
\end{pgfscope}%
\begin{pgfscope}%
\pgfsys@transformshift{3.748745in}{1.059238in}%
\pgfsys@useobject{currentmarker}{}%
\end{pgfscope}%
\begin{pgfscope}%
\pgfsys@transformshift{3.766323in}{1.059150in}%
\pgfsys@useobject{currentmarker}{}%
\end{pgfscope}%
\begin{pgfscope}%
\pgfsys@transformshift{3.885808in}{1.059149in}%
\pgfsys@useobject{currentmarker}{}%
\end{pgfscope}%
\begin{pgfscope}%
\pgfsys@transformshift{3.933797in}{1.059148in}%
\pgfsys@useobject{currentmarker}{}%
\end{pgfscope}%
\begin{pgfscope}%
\pgfsys@transformshift{4.070163in}{1.059145in}%
\pgfsys@useobject{currentmarker}{}%
\end{pgfscope}%
\begin{pgfscope}%
\pgfsys@transformshift{4.120452in}{1.058684in}%
\pgfsys@useobject{currentmarker}{}%
\end{pgfscope}%
\begin{pgfscope}%
\pgfsys@transformshift{4.232474in}{1.058611in}%
\pgfsys@useobject{currentmarker}{}%
\end{pgfscope}%
\begin{pgfscope}%
\pgfsys@transformshift{4.393612in}{1.058602in}%
\pgfsys@useobject{currentmarker}{}%
\end{pgfscope}%
\begin{pgfscope}%
\pgfsys@transformshift{4.407199in}{1.058573in}%
\pgfsys@useobject{currentmarker}{}%
\end{pgfscope}%
\begin{pgfscope}%
\pgfsys@transformshift{4.484258in}{1.058494in}%
\pgfsys@useobject{currentmarker}{}%
\end{pgfscope}%
\begin{pgfscope}%
\pgfsys@transformshift{4.485771in}{1.058481in}%
\pgfsys@useobject{currentmarker}{}%
\end{pgfscope}%
\begin{pgfscope}%
\pgfsys@transformshift{4.646808in}{1.058454in}%
\pgfsys@useobject{currentmarker}{}%
\end{pgfscope}%
\begin{pgfscope}%
\pgfsys@transformshift{4.733640in}{1.058446in}%
\pgfsys@useobject{currentmarker}{}%
\end{pgfscope}%
\begin{pgfscope}%
\pgfsys@transformshift{4.919563in}{1.058436in}%
\pgfsys@useobject{currentmarker}{}%
\end{pgfscope}%
\begin{pgfscope}%
\pgfsys@transformshift{5.798801in}{1.058412in}%
\pgfsys@useobject{currentmarker}{}%
\end{pgfscope}%
\begin{pgfscope}%
\pgfsys@transformshift{6.068379in}{1.058412in}%
\pgfsys@useobject{currentmarker}{}%
\end{pgfscope}%
\end{pgfscope}%
\begin{pgfscope}%
\pgfpathrectangle{\pgfqpoint{0.840504in}{0.670138in}}{\pgfqpoint{5.911808in}{3.929862in}}%
\pgfusepath{clip}%
\pgfsetrectcap%
\pgfsetroundjoin%
\pgfsetlinewidth{0.803000pt}%
\definecolor{currentstroke}{rgb}{0.690196,0.690196,0.690196}%
\pgfsetstrokecolor{currentstroke}%
\pgfsetstrokeopacity{0.200000}%
\pgfsetdash{}{0pt}%
\pgfpathmoveto{\pgfqpoint{1.313449in}{0.670138in}}%
\pgfpathlineto{\pgfqpoint{1.313449in}{4.600000in}}%
\pgfusepath{stroke}%
\end{pgfscope}%
\begin{pgfscope}%
\pgfsetbuttcap%
\pgfsetroundjoin%
\definecolor{currentfill}{rgb}{0.000000,0.000000,0.000000}%
\pgfsetfillcolor{currentfill}%
\pgfsetlinewidth{0.803000pt}%
\definecolor{currentstroke}{rgb}{0.000000,0.000000,0.000000}%
\pgfsetstrokecolor{currentstroke}%
\pgfsetdash{}{0pt}%
\pgfsys@defobject{currentmarker}{\pgfqpoint{0.000000in}{-0.048611in}}{\pgfqpoint{0.000000in}{0.000000in}}{%
\pgfpathmoveto{\pgfqpoint{0.000000in}{0.000000in}}%
\pgfpathlineto{\pgfqpoint{0.000000in}{-0.048611in}}%
\pgfusepath{stroke,fill}%
}%
\begin{pgfscope}%
\pgfsys@transformshift{1.313449in}{0.670138in}%
\pgfsys@useobject{currentmarker}{}%
\end{pgfscope}%
\end{pgfscope}%
\begin{pgfscope}%
\definecolor{textcolor}{rgb}{0.000000,0.000000,0.000000}%
\pgfsetstrokecolor{textcolor}%
\pgfsetfillcolor{textcolor}%
\pgftext[x=1.313449in,y=0.572916in,,top]{\color{textcolor}{\rmfamily\fontsize{14.000000}{16.800000}\selectfont\catcode`\^=\active\def^{\ifmmode\sp\else\^{}\fi}\catcode`\%=\active\def%{\%}$\mathdefault{6000}$}}%
\end{pgfscope}%
\begin{pgfscope}%
\pgfpathrectangle{\pgfqpoint{0.840504in}{0.670138in}}{\pgfqpoint{5.911808in}{3.929862in}}%
\pgfusepath{clip}%
\pgfsetrectcap%
\pgfsetroundjoin%
\pgfsetlinewidth{0.803000pt}%
\definecolor{currentstroke}{rgb}{0.690196,0.690196,0.690196}%
\pgfsetstrokecolor{currentstroke}%
\pgfsetstrokeopacity{0.200000}%
\pgfsetdash{}{0pt}%
\pgfpathmoveto{\pgfqpoint{2.259338in}{0.670138in}}%
\pgfpathlineto{\pgfqpoint{2.259338in}{4.600000in}}%
\pgfusepath{stroke}%
\end{pgfscope}%
\begin{pgfscope}%
\pgfsetbuttcap%
\pgfsetroundjoin%
\definecolor{currentfill}{rgb}{0.000000,0.000000,0.000000}%
\pgfsetfillcolor{currentfill}%
\pgfsetlinewidth{0.803000pt}%
\definecolor{currentstroke}{rgb}{0.000000,0.000000,0.000000}%
\pgfsetstrokecolor{currentstroke}%
\pgfsetdash{}{0pt}%
\pgfsys@defobject{currentmarker}{\pgfqpoint{0.000000in}{-0.048611in}}{\pgfqpoint{0.000000in}{0.000000in}}{%
\pgfpathmoveto{\pgfqpoint{0.000000in}{0.000000in}}%
\pgfpathlineto{\pgfqpoint{0.000000in}{-0.048611in}}%
\pgfusepath{stroke,fill}%
}%
\begin{pgfscope}%
\pgfsys@transformshift{2.259338in}{0.670138in}%
\pgfsys@useobject{currentmarker}{}%
\end{pgfscope}%
\end{pgfscope}%
\begin{pgfscope}%
\definecolor{textcolor}{rgb}{0.000000,0.000000,0.000000}%
\pgfsetstrokecolor{textcolor}%
\pgfsetfillcolor{textcolor}%
\pgftext[x=2.259338in,y=0.572916in,,top]{\color{textcolor}{\rmfamily\fontsize{14.000000}{16.800000}\selectfont\catcode`\^=\active\def^{\ifmmode\sp\else\^{}\fi}\catcode`\%=\active\def%{\%}$\mathdefault{8000}$}}%
\end{pgfscope}%
\begin{pgfscope}%
\pgfpathrectangle{\pgfqpoint{0.840504in}{0.670138in}}{\pgfqpoint{5.911808in}{3.929862in}}%
\pgfusepath{clip}%
\pgfsetrectcap%
\pgfsetroundjoin%
\pgfsetlinewidth{0.803000pt}%
\definecolor{currentstroke}{rgb}{0.690196,0.690196,0.690196}%
\pgfsetstrokecolor{currentstroke}%
\pgfsetstrokeopacity{0.200000}%
\pgfsetdash{}{0pt}%
\pgfpathmoveto{\pgfqpoint{3.205228in}{0.670138in}}%
\pgfpathlineto{\pgfqpoint{3.205228in}{4.600000in}}%
\pgfusepath{stroke}%
\end{pgfscope}%
\begin{pgfscope}%
\pgfsetbuttcap%
\pgfsetroundjoin%
\definecolor{currentfill}{rgb}{0.000000,0.000000,0.000000}%
\pgfsetfillcolor{currentfill}%
\pgfsetlinewidth{0.803000pt}%
\definecolor{currentstroke}{rgb}{0.000000,0.000000,0.000000}%
\pgfsetstrokecolor{currentstroke}%
\pgfsetdash{}{0pt}%
\pgfsys@defobject{currentmarker}{\pgfqpoint{0.000000in}{-0.048611in}}{\pgfqpoint{0.000000in}{0.000000in}}{%
\pgfpathmoveto{\pgfqpoint{0.000000in}{0.000000in}}%
\pgfpathlineto{\pgfqpoint{0.000000in}{-0.048611in}}%
\pgfusepath{stroke,fill}%
}%
\begin{pgfscope}%
\pgfsys@transformshift{3.205228in}{0.670138in}%
\pgfsys@useobject{currentmarker}{}%
\end{pgfscope}%
\end{pgfscope}%
\begin{pgfscope}%
\definecolor{textcolor}{rgb}{0.000000,0.000000,0.000000}%
\pgfsetstrokecolor{textcolor}%
\pgfsetfillcolor{textcolor}%
\pgftext[x=3.205228in,y=0.572916in,,top]{\color{textcolor}{\rmfamily\fontsize{14.000000}{16.800000}\selectfont\catcode`\^=\active\def^{\ifmmode\sp\else\^{}\fi}\catcode`\%=\active\def%{\%}$\mathdefault{10000}$}}%
\end{pgfscope}%
\begin{pgfscope}%
\pgfpathrectangle{\pgfqpoint{0.840504in}{0.670138in}}{\pgfqpoint{5.911808in}{3.929862in}}%
\pgfusepath{clip}%
\pgfsetrectcap%
\pgfsetroundjoin%
\pgfsetlinewidth{0.803000pt}%
\definecolor{currentstroke}{rgb}{0.690196,0.690196,0.690196}%
\pgfsetstrokecolor{currentstroke}%
\pgfsetstrokeopacity{0.200000}%
\pgfsetdash{}{0pt}%
\pgfpathmoveto{\pgfqpoint{4.151117in}{0.670138in}}%
\pgfpathlineto{\pgfqpoint{4.151117in}{4.600000in}}%
\pgfusepath{stroke}%
\end{pgfscope}%
\begin{pgfscope}%
\pgfsetbuttcap%
\pgfsetroundjoin%
\definecolor{currentfill}{rgb}{0.000000,0.000000,0.000000}%
\pgfsetfillcolor{currentfill}%
\pgfsetlinewidth{0.803000pt}%
\definecolor{currentstroke}{rgb}{0.000000,0.000000,0.000000}%
\pgfsetstrokecolor{currentstroke}%
\pgfsetdash{}{0pt}%
\pgfsys@defobject{currentmarker}{\pgfqpoint{0.000000in}{-0.048611in}}{\pgfqpoint{0.000000in}{0.000000in}}{%
\pgfpathmoveto{\pgfqpoint{0.000000in}{0.000000in}}%
\pgfpathlineto{\pgfqpoint{0.000000in}{-0.048611in}}%
\pgfusepath{stroke,fill}%
}%
\begin{pgfscope}%
\pgfsys@transformshift{4.151117in}{0.670138in}%
\pgfsys@useobject{currentmarker}{}%
\end{pgfscope}%
\end{pgfscope}%
\begin{pgfscope}%
\definecolor{textcolor}{rgb}{0.000000,0.000000,0.000000}%
\pgfsetstrokecolor{textcolor}%
\pgfsetfillcolor{textcolor}%
\pgftext[x=4.151117in,y=0.572916in,,top]{\color{textcolor}{\rmfamily\fontsize{14.000000}{16.800000}\selectfont\catcode`\^=\active\def^{\ifmmode\sp\else\^{}\fi}\catcode`\%=\active\def%{\%}$\mathdefault{12000}$}}%
\end{pgfscope}%
\begin{pgfscope}%
\pgfpathrectangle{\pgfqpoint{0.840504in}{0.670138in}}{\pgfqpoint{5.911808in}{3.929862in}}%
\pgfusepath{clip}%
\pgfsetrectcap%
\pgfsetroundjoin%
\pgfsetlinewidth{0.803000pt}%
\definecolor{currentstroke}{rgb}{0.690196,0.690196,0.690196}%
\pgfsetstrokecolor{currentstroke}%
\pgfsetstrokeopacity{0.200000}%
\pgfsetdash{}{0pt}%
\pgfpathmoveto{\pgfqpoint{5.097007in}{0.670138in}}%
\pgfpathlineto{\pgfqpoint{5.097007in}{4.600000in}}%
\pgfusepath{stroke}%
\end{pgfscope}%
\begin{pgfscope}%
\pgfsetbuttcap%
\pgfsetroundjoin%
\definecolor{currentfill}{rgb}{0.000000,0.000000,0.000000}%
\pgfsetfillcolor{currentfill}%
\pgfsetlinewidth{0.803000pt}%
\definecolor{currentstroke}{rgb}{0.000000,0.000000,0.000000}%
\pgfsetstrokecolor{currentstroke}%
\pgfsetdash{}{0pt}%
\pgfsys@defobject{currentmarker}{\pgfqpoint{0.000000in}{-0.048611in}}{\pgfqpoint{0.000000in}{0.000000in}}{%
\pgfpathmoveto{\pgfqpoint{0.000000in}{0.000000in}}%
\pgfpathlineto{\pgfqpoint{0.000000in}{-0.048611in}}%
\pgfusepath{stroke,fill}%
}%
\begin{pgfscope}%
\pgfsys@transformshift{5.097007in}{0.670138in}%
\pgfsys@useobject{currentmarker}{}%
\end{pgfscope}%
\end{pgfscope}%
\begin{pgfscope}%
\definecolor{textcolor}{rgb}{0.000000,0.000000,0.000000}%
\pgfsetstrokecolor{textcolor}%
\pgfsetfillcolor{textcolor}%
\pgftext[x=5.097007in,y=0.572916in,,top]{\color{textcolor}{\rmfamily\fontsize{14.000000}{16.800000}\selectfont\catcode`\^=\active\def^{\ifmmode\sp\else\^{}\fi}\catcode`\%=\active\def%{\%}$\mathdefault{14000}$}}%
\end{pgfscope}%
\begin{pgfscope}%
\pgfpathrectangle{\pgfqpoint{0.840504in}{0.670138in}}{\pgfqpoint{5.911808in}{3.929862in}}%
\pgfusepath{clip}%
\pgfsetrectcap%
\pgfsetroundjoin%
\pgfsetlinewidth{0.803000pt}%
\definecolor{currentstroke}{rgb}{0.690196,0.690196,0.690196}%
\pgfsetstrokecolor{currentstroke}%
\pgfsetstrokeopacity{0.200000}%
\pgfsetdash{}{0pt}%
\pgfpathmoveto{\pgfqpoint{6.042896in}{0.670138in}}%
\pgfpathlineto{\pgfqpoint{6.042896in}{4.600000in}}%
\pgfusepath{stroke}%
\end{pgfscope}%
\begin{pgfscope}%
\pgfsetbuttcap%
\pgfsetroundjoin%
\definecolor{currentfill}{rgb}{0.000000,0.000000,0.000000}%
\pgfsetfillcolor{currentfill}%
\pgfsetlinewidth{0.803000pt}%
\definecolor{currentstroke}{rgb}{0.000000,0.000000,0.000000}%
\pgfsetstrokecolor{currentstroke}%
\pgfsetdash{}{0pt}%
\pgfsys@defobject{currentmarker}{\pgfqpoint{0.000000in}{-0.048611in}}{\pgfqpoint{0.000000in}{0.000000in}}{%
\pgfpathmoveto{\pgfqpoint{0.000000in}{0.000000in}}%
\pgfpathlineto{\pgfqpoint{0.000000in}{-0.048611in}}%
\pgfusepath{stroke,fill}%
}%
\begin{pgfscope}%
\pgfsys@transformshift{6.042896in}{0.670138in}%
\pgfsys@useobject{currentmarker}{}%
\end{pgfscope}%
\end{pgfscope}%
\begin{pgfscope}%
\definecolor{textcolor}{rgb}{0.000000,0.000000,0.000000}%
\pgfsetstrokecolor{textcolor}%
\pgfsetfillcolor{textcolor}%
\pgftext[x=6.042896in,y=0.572916in,,top]{\color{textcolor}{\rmfamily\fontsize{14.000000}{16.800000}\selectfont\catcode`\^=\active\def^{\ifmmode\sp\else\^{}\fi}\catcode`\%=\active\def%{\%}$\mathdefault{16000}$}}%
\end{pgfscope}%
\begin{pgfscope}%
\definecolor{textcolor}{rgb}{0.000000,0.000000,0.000000}%
\pgfsetstrokecolor{textcolor}%
\pgfsetfillcolor{textcolor}%
\pgftext[x=3.796409in,y=0.339583in,,top]{\color{textcolor}{\rmfamily\fontsize{18.000000}{21.600000}\selectfont\catcode`\^=\active\def^{\ifmmode\sp\else\^{}\fi}\catcode`\%=\active\def%{\%}Total Cost [M\$]}}%
\end{pgfscope}%
\begin{pgfscope}%
\pgfpathrectangle{\pgfqpoint{0.840504in}{0.670138in}}{\pgfqpoint{5.911808in}{3.929862in}}%
\pgfusepath{clip}%
\pgfsetrectcap%
\pgfsetroundjoin%
\pgfsetlinewidth{0.803000pt}%
\definecolor{currentstroke}{rgb}{0.690196,0.690196,0.690196}%
\pgfsetstrokecolor{currentstroke}%
\pgfsetstrokeopacity{0.200000}%
\pgfsetdash{}{0pt}%
\pgfpathmoveto{\pgfqpoint{0.840504in}{0.670138in}}%
\pgfpathlineto{\pgfqpoint{6.752313in}{0.670138in}}%
\pgfusepath{stroke}%
\end{pgfscope}%
\begin{pgfscope}%
\pgfsetbuttcap%
\pgfsetroundjoin%
\definecolor{currentfill}{rgb}{0.000000,0.000000,0.000000}%
\pgfsetfillcolor{currentfill}%
\pgfsetlinewidth{0.803000pt}%
\definecolor{currentstroke}{rgb}{0.000000,0.000000,0.000000}%
\pgfsetstrokecolor{currentstroke}%
\pgfsetdash{}{0pt}%
\pgfsys@defobject{currentmarker}{\pgfqpoint{-0.048611in}{0.000000in}}{\pgfqpoint{-0.000000in}{0.000000in}}{%
\pgfpathmoveto{\pgfqpoint{-0.000000in}{0.000000in}}%
\pgfpathlineto{\pgfqpoint{-0.048611in}{0.000000in}}%
\pgfusepath{stroke,fill}%
}%
\begin{pgfscope}%
\pgfsys@transformshift{0.840504in}{0.670138in}%
\pgfsys@useobject{currentmarker}{}%
\end{pgfscope}%
\end{pgfscope}%
\begin{pgfscope}%
\definecolor{textcolor}{rgb}{0.000000,0.000000,0.000000}%
\pgfsetstrokecolor{textcolor}%
\pgfsetfillcolor{textcolor}%
\pgftext[x=0.493054in, y=0.600694in, left, base]{\color{textcolor}{\rmfamily\fontsize{14.000000}{16.800000}\selectfont\catcode`\^=\active\def^{\ifmmode\sp\else\^{}\fi}\catcode`\%=\active\def%{\%}$\mathdefault{0.0}$}}%
\end{pgfscope}%
\begin{pgfscope}%
\pgfpathrectangle{\pgfqpoint{0.840504in}{0.670138in}}{\pgfqpoint{5.911808in}{3.929862in}}%
\pgfusepath{clip}%
\pgfsetrectcap%
\pgfsetroundjoin%
\pgfsetlinewidth{0.803000pt}%
\definecolor{currentstroke}{rgb}{0.690196,0.690196,0.690196}%
\pgfsetstrokecolor{currentstroke}%
\pgfsetstrokeopacity{0.200000}%
\pgfsetdash{}{0pt}%
\pgfpathmoveto{\pgfqpoint{0.840504in}{1.116713in}}%
\pgfpathlineto{\pgfqpoint{6.752313in}{1.116713in}}%
\pgfusepath{stroke}%
\end{pgfscope}%
\begin{pgfscope}%
\pgfsetbuttcap%
\pgfsetroundjoin%
\definecolor{currentfill}{rgb}{0.000000,0.000000,0.000000}%
\pgfsetfillcolor{currentfill}%
\pgfsetlinewidth{0.803000pt}%
\definecolor{currentstroke}{rgb}{0.000000,0.000000,0.000000}%
\pgfsetstrokecolor{currentstroke}%
\pgfsetdash{}{0pt}%
\pgfsys@defobject{currentmarker}{\pgfqpoint{-0.048611in}{0.000000in}}{\pgfqpoint{-0.000000in}{0.000000in}}{%
\pgfpathmoveto{\pgfqpoint{-0.000000in}{0.000000in}}%
\pgfpathlineto{\pgfqpoint{-0.048611in}{0.000000in}}%
\pgfusepath{stroke,fill}%
}%
\begin{pgfscope}%
\pgfsys@transformshift{0.840504in}{1.116713in}%
\pgfsys@useobject{currentmarker}{}%
\end{pgfscope}%
\end{pgfscope}%
\begin{pgfscope}%
\definecolor{textcolor}{rgb}{0.000000,0.000000,0.000000}%
\pgfsetstrokecolor{textcolor}%
\pgfsetfillcolor{textcolor}%
\pgftext[x=0.493054in, y=1.047269in, left, base]{\color{textcolor}{\rmfamily\fontsize{14.000000}{16.800000}\selectfont\catcode`\^=\active\def^{\ifmmode\sp\else\^{}\fi}\catcode`\%=\active\def%{\%}$\mathdefault{2.5}$}}%
\end{pgfscope}%
\begin{pgfscope}%
\pgfpathrectangle{\pgfqpoint{0.840504in}{0.670138in}}{\pgfqpoint{5.911808in}{3.929862in}}%
\pgfusepath{clip}%
\pgfsetrectcap%
\pgfsetroundjoin%
\pgfsetlinewidth{0.803000pt}%
\definecolor{currentstroke}{rgb}{0.690196,0.690196,0.690196}%
\pgfsetstrokecolor{currentstroke}%
\pgfsetstrokeopacity{0.200000}%
\pgfsetdash{}{0pt}%
\pgfpathmoveto{\pgfqpoint{0.840504in}{1.563288in}}%
\pgfpathlineto{\pgfqpoint{6.752313in}{1.563288in}}%
\pgfusepath{stroke}%
\end{pgfscope}%
\begin{pgfscope}%
\pgfsetbuttcap%
\pgfsetroundjoin%
\definecolor{currentfill}{rgb}{0.000000,0.000000,0.000000}%
\pgfsetfillcolor{currentfill}%
\pgfsetlinewidth{0.803000pt}%
\definecolor{currentstroke}{rgb}{0.000000,0.000000,0.000000}%
\pgfsetstrokecolor{currentstroke}%
\pgfsetdash{}{0pt}%
\pgfsys@defobject{currentmarker}{\pgfqpoint{-0.048611in}{0.000000in}}{\pgfqpoint{-0.000000in}{0.000000in}}{%
\pgfpathmoveto{\pgfqpoint{-0.000000in}{0.000000in}}%
\pgfpathlineto{\pgfqpoint{-0.048611in}{0.000000in}}%
\pgfusepath{stroke,fill}%
}%
\begin{pgfscope}%
\pgfsys@transformshift{0.840504in}{1.563288in}%
\pgfsys@useobject{currentmarker}{}%
\end{pgfscope}%
\end{pgfscope}%
\begin{pgfscope}%
\definecolor{textcolor}{rgb}{0.000000,0.000000,0.000000}%
\pgfsetstrokecolor{textcolor}%
\pgfsetfillcolor{textcolor}%
\pgftext[x=0.493054in, y=1.493844in, left, base]{\color{textcolor}{\rmfamily\fontsize{14.000000}{16.800000}\selectfont\catcode`\^=\active\def^{\ifmmode\sp\else\^{}\fi}\catcode`\%=\active\def%{\%}$\mathdefault{5.0}$}}%
\end{pgfscope}%
\begin{pgfscope}%
\pgfpathrectangle{\pgfqpoint{0.840504in}{0.670138in}}{\pgfqpoint{5.911808in}{3.929862in}}%
\pgfusepath{clip}%
\pgfsetrectcap%
\pgfsetroundjoin%
\pgfsetlinewidth{0.803000pt}%
\definecolor{currentstroke}{rgb}{0.690196,0.690196,0.690196}%
\pgfsetstrokecolor{currentstroke}%
\pgfsetstrokeopacity{0.200000}%
\pgfsetdash{}{0pt}%
\pgfpathmoveto{\pgfqpoint{0.840504in}{2.009864in}}%
\pgfpathlineto{\pgfqpoint{6.752313in}{2.009864in}}%
\pgfusepath{stroke}%
\end{pgfscope}%
\begin{pgfscope}%
\pgfsetbuttcap%
\pgfsetroundjoin%
\definecolor{currentfill}{rgb}{0.000000,0.000000,0.000000}%
\pgfsetfillcolor{currentfill}%
\pgfsetlinewidth{0.803000pt}%
\definecolor{currentstroke}{rgb}{0.000000,0.000000,0.000000}%
\pgfsetstrokecolor{currentstroke}%
\pgfsetdash{}{0pt}%
\pgfsys@defobject{currentmarker}{\pgfqpoint{-0.048611in}{0.000000in}}{\pgfqpoint{-0.000000in}{0.000000in}}{%
\pgfpathmoveto{\pgfqpoint{-0.000000in}{0.000000in}}%
\pgfpathlineto{\pgfqpoint{-0.048611in}{0.000000in}}%
\pgfusepath{stroke,fill}%
}%
\begin{pgfscope}%
\pgfsys@transformshift{0.840504in}{2.009864in}%
\pgfsys@useobject{currentmarker}{}%
\end{pgfscope}%
\end{pgfscope}%
\begin{pgfscope}%
\definecolor{textcolor}{rgb}{0.000000,0.000000,0.000000}%
\pgfsetstrokecolor{textcolor}%
\pgfsetfillcolor{textcolor}%
\pgftext[x=0.493054in, y=1.940419in, left, base]{\color{textcolor}{\rmfamily\fontsize{14.000000}{16.800000}\selectfont\catcode`\^=\active\def^{\ifmmode\sp\else\^{}\fi}\catcode`\%=\active\def%{\%}$\mathdefault{7.5}$}}%
\end{pgfscope}%
\begin{pgfscope}%
\pgfpathrectangle{\pgfqpoint{0.840504in}{0.670138in}}{\pgfqpoint{5.911808in}{3.929862in}}%
\pgfusepath{clip}%
\pgfsetrectcap%
\pgfsetroundjoin%
\pgfsetlinewidth{0.803000pt}%
\definecolor{currentstroke}{rgb}{0.690196,0.690196,0.690196}%
\pgfsetstrokecolor{currentstroke}%
\pgfsetstrokeopacity{0.200000}%
\pgfsetdash{}{0pt}%
\pgfpathmoveto{\pgfqpoint{0.840504in}{2.456439in}}%
\pgfpathlineto{\pgfqpoint{6.752313in}{2.456439in}}%
\pgfusepath{stroke}%
\end{pgfscope}%
\begin{pgfscope}%
\pgfsetbuttcap%
\pgfsetroundjoin%
\definecolor{currentfill}{rgb}{0.000000,0.000000,0.000000}%
\pgfsetfillcolor{currentfill}%
\pgfsetlinewidth{0.803000pt}%
\definecolor{currentstroke}{rgb}{0.000000,0.000000,0.000000}%
\pgfsetstrokecolor{currentstroke}%
\pgfsetdash{}{0pt}%
\pgfsys@defobject{currentmarker}{\pgfqpoint{-0.048611in}{0.000000in}}{\pgfqpoint{-0.000000in}{0.000000in}}{%
\pgfpathmoveto{\pgfqpoint{-0.000000in}{0.000000in}}%
\pgfpathlineto{\pgfqpoint{-0.048611in}{0.000000in}}%
\pgfusepath{stroke,fill}%
}%
\begin{pgfscope}%
\pgfsys@transformshift{0.840504in}{2.456439in}%
\pgfsys@useobject{currentmarker}{}%
\end{pgfscope}%
\end{pgfscope}%
\begin{pgfscope}%
\definecolor{textcolor}{rgb}{0.000000,0.000000,0.000000}%
\pgfsetstrokecolor{textcolor}%
\pgfsetfillcolor{textcolor}%
\pgftext[x=0.395138in, y=2.386995in, left, base]{\color{textcolor}{\rmfamily\fontsize{14.000000}{16.800000}\selectfont\catcode`\^=\active\def^{\ifmmode\sp\else\^{}\fi}\catcode`\%=\active\def%{\%}$\mathdefault{10.0}$}}%
\end{pgfscope}%
\begin{pgfscope}%
\pgfpathrectangle{\pgfqpoint{0.840504in}{0.670138in}}{\pgfqpoint{5.911808in}{3.929862in}}%
\pgfusepath{clip}%
\pgfsetrectcap%
\pgfsetroundjoin%
\pgfsetlinewidth{0.803000pt}%
\definecolor{currentstroke}{rgb}{0.690196,0.690196,0.690196}%
\pgfsetstrokecolor{currentstroke}%
\pgfsetstrokeopacity{0.200000}%
\pgfsetdash{}{0pt}%
\pgfpathmoveto{\pgfqpoint{0.840504in}{2.903014in}}%
\pgfpathlineto{\pgfqpoint{6.752313in}{2.903014in}}%
\pgfusepath{stroke}%
\end{pgfscope}%
\begin{pgfscope}%
\pgfsetbuttcap%
\pgfsetroundjoin%
\definecolor{currentfill}{rgb}{0.000000,0.000000,0.000000}%
\pgfsetfillcolor{currentfill}%
\pgfsetlinewidth{0.803000pt}%
\definecolor{currentstroke}{rgb}{0.000000,0.000000,0.000000}%
\pgfsetstrokecolor{currentstroke}%
\pgfsetdash{}{0pt}%
\pgfsys@defobject{currentmarker}{\pgfqpoint{-0.048611in}{0.000000in}}{\pgfqpoint{-0.000000in}{0.000000in}}{%
\pgfpathmoveto{\pgfqpoint{-0.000000in}{0.000000in}}%
\pgfpathlineto{\pgfqpoint{-0.048611in}{0.000000in}}%
\pgfusepath{stroke,fill}%
}%
\begin{pgfscope}%
\pgfsys@transformshift{0.840504in}{2.903014in}%
\pgfsys@useobject{currentmarker}{}%
\end{pgfscope}%
\end{pgfscope}%
\begin{pgfscope}%
\definecolor{textcolor}{rgb}{0.000000,0.000000,0.000000}%
\pgfsetstrokecolor{textcolor}%
\pgfsetfillcolor{textcolor}%
\pgftext[x=0.395138in, y=2.833570in, left, base]{\color{textcolor}{\rmfamily\fontsize{14.000000}{16.800000}\selectfont\catcode`\^=\active\def^{\ifmmode\sp\else\^{}\fi}\catcode`\%=\active\def%{\%}$\mathdefault{12.5}$}}%
\end{pgfscope}%
\begin{pgfscope}%
\pgfpathrectangle{\pgfqpoint{0.840504in}{0.670138in}}{\pgfqpoint{5.911808in}{3.929862in}}%
\pgfusepath{clip}%
\pgfsetrectcap%
\pgfsetroundjoin%
\pgfsetlinewidth{0.803000pt}%
\definecolor{currentstroke}{rgb}{0.690196,0.690196,0.690196}%
\pgfsetstrokecolor{currentstroke}%
\pgfsetstrokeopacity{0.200000}%
\pgfsetdash{}{0pt}%
\pgfpathmoveto{\pgfqpoint{0.840504in}{3.349589in}}%
\pgfpathlineto{\pgfqpoint{6.752313in}{3.349589in}}%
\pgfusepath{stroke}%
\end{pgfscope}%
\begin{pgfscope}%
\pgfsetbuttcap%
\pgfsetroundjoin%
\definecolor{currentfill}{rgb}{0.000000,0.000000,0.000000}%
\pgfsetfillcolor{currentfill}%
\pgfsetlinewidth{0.803000pt}%
\definecolor{currentstroke}{rgb}{0.000000,0.000000,0.000000}%
\pgfsetstrokecolor{currentstroke}%
\pgfsetdash{}{0pt}%
\pgfsys@defobject{currentmarker}{\pgfqpoint{-0.048611in}{0.000000in}}{\pgfqpoint{-0.000000in}{0.000000in}}{%
\pgfpathmoveto{\pgfqpoint{-0.000000in}{0.000000in}}%
\pgfpathlineto{\pgfqpoint{-0.048611in}{0.000000in}}%
\pgfusepath{stroke,fill}%
}%
\begin{pgfscope}%
\pgfsys@transformshift{0.840504in}{3.349589in}%
\pgfsys@useobject{currentmarker}{}%
\end{pgfscope}%
\end{pgfscope}%
\begin{pgfscope}%
\definecolor{textcolor}{rgb}{0.000000,0.000000,0.000000}%
\pgfsetstrokecolor{textcolor}%
\pgfsetfillcolor{textcolor}%
\pgftext[x=0.395138in, y=3.280145in, left, base]{\color{textcolor}{\rmfamily\fontsize{14.000000}{16.800000}\selectfont\catcode`\^=\active\def^{\ifmmode\sp\else\^{}\fi}\catcode`\%=\active\def%{\%}$\mathdefault{15.0}$}}%
\end{pgfscope}%
\begin{pgfscope}%
\pgfpathrectangle{\pgfqpoint{0.840504in}{0.670138in}}{\pgfqpoint{5.911808in}{3.929862in}}%
\pgfusepath{clip}%
\pgfsetrectcap%
\pgfsetroundjoin%
\pgfsetlinewidth{0.803000pt}%
\definecolor{currentstroke}{rgb}{0.690196,0.690196,0.690196}%
\pgfsetstrokecolor{currentstroke}%
\pgfsetstrokeopacity{0.200000}%
\pgfsetdash{}{0pt}%
\pgfpathmoveto{\pgfqpoint{0.840504in}{3.796165in}}%
\pgfpathlineto{\pgfqpoint{6.752313in}{3.796165in}}%
\pgfusepath{stroke}%
\end{pgfscope}%
\begin{pgfscope}%
\pgfsetbuttcap%
\pgfsetroundjoin%
\definecolor{currentfill}{rgb}{0.000000,0.000000,0.000000}%
\pgfsetfillcolor{currentfill}%
\pgfsetlinewidth{0.803000pt}%
\definecolor{currentstroke}{rgb}{0.000000,0.000000,0.000000}%
\pgfsetstrokecolor{currentstroke}%
\pgfsetdash{}{0pt}%
\pgfsys@defobject{currentmarker}{\pgfqpoint{-0.048611in}{0.000000in}}{\pgfqpoint{-0.000000in}{0.000000in}}{%
\pgfpathmoveto{\pgfqpoint{-0.000000in}{0.000000in}}%
\pgfpathlineto{\pgfqpoint{-0.048611in}{0.000000in}}%
\pgfusepath{stroke,fill}%
}%
\begin{pgfscope}%
\pgfsys@transformshift{0.840504in}{3.796165in}%
\pgfsys@useobject{currentmarker}{}%
\end{pgfscope}%
\end{pgfscope}%
\begin{pgfscope}%
\definecolor{textcolor}{rgb}{0.000000,0.000000,0.000000}%
\pgfsetstrokecolor{textcolor}%
\pgfsetfillcolor{textcolor}%
\pgftext[x=0.395138in, y=3.726720in, left, base]{\color{textcolor}{\rmfamily\fontsize{14.000000}{16.800000}\selectfont\catcode`\^=\active\def^{\ifmmode\sp\else\^{}\fi}\catcode`\%=\active\def%{\%}$\mathdefault{17.5}$}}%
\end{pgfscope}%
\begin{pgfscope}%
\pgfpathrectangle{\pgfqpoint{0.840504in}{0.670138in}}{\pgfqpoint{5.911808in}{3.929862in}}%
\pgfusepath{clip}%
\pgfsetrectcap%
\pgfsetroundjoin%
\pgfsetlinewidth{0.803000pt}%
\definecolor{currentstroke}{rgb}{0.690196,0.690196,0.690196}%
\pgfsetstrokecolor{currentstroke}%
\pgfsetstrokeopacity{0.200000}%
\pgfsetdash{}{0pt}%
\pgfpathmoveto{\pgfqpoint{0.840504in}{4.242740in}}%
\pgfpathlineto{\pgfqpoint{6.752313in}{4.242740in}}%
\pgfusepath{stroke}%
\end{pgfscope}%
\begin{pgfscope}%
\pgfsetbuttcap%
\pgfsetroundjoin%
\definecolor{currentfill}{rgb}{0.000000,0.000000,0.000000}%
\pgfsetfillcolor{currentfill}%
\pgfsetlinewidth{0.803000pt}%
\definecolor{currentstroke}{rgb}{0.000000,0.000000,0.000000}%
\pgfsetstrokecolor{currentstroke}%
\pgfsetdash{}{0pt}%
\pgfsys@defobject{currentmarker}{\pgfqpoint{-0.048611in}{0.000000in}}{\pgfqpoint{-0.000000in}{0.000000in}}{%
\pgfpathmoveto{\pgfqpoint{-0.000000in}{0.000000in}}%
\pgfpathlineto{\pgfqpoint{-0.048611in}{0.000000in}}%
\pgfusepath{stroke,fill}%
}%
\begin{pgfscope}%
\pgfsys@transformshift{0.840504in}{4.242740in}%
\pgfsys@useobject{currentmarker}{}%
\end{pgfscope}%
\end{pgfscope}%
\begin{pgfscope}%
\definecolor{textcolor}{rgb}{0.000000,0.000000,0.000000}%
\pgfsetstrokecolor{textcolor}%
\pgfsetfillcolor{textcolor}%
\pgftext[x=0.395138in, y=4.173296in, left, base]{\color{textcolor}{\rmfamily\fontsize{14.000000}{16.800000}\selectfont\catcode`\^=\active\def^{\ifmmode\sp\else\^{}\fi}\catcode`\%=\active\def%{\%}$\mathdefault{20.0}$}}%
\end{pgfscope}%
\begin{pgfscope}%
\definecolor{textcolor}{rgb}{0.000000,0.000000,0.000000}%
\pgfsetstrokecolor{textcolor}%
\pgfsetfillcolor{textcolor}%
\pgftext[x=0.339583in,y=2.635069in,,bottom,rotate=90.000000]{\color{textcolor}{\rmfamily\fontsize{18.000000}{21.600000}\selectfont\catcode`\^=\active\def^{\ifmmode\sp\else\^{}\fi}\catcode`\%=\active\def%{\%}Emissions [MT CO2eq]}}%
\end{pgfscope}%
\begin{pgfscope}%
\pgfsetrectcap%
\pgfsetmiterjoin%
\pgfsetlinewidth{0.803000pt}%
\definecolor{currentstroke}{rgb}{0.000000,0.000000,0.000000}%
\pgfsetstrokecolor{currentstroke}%
\pgfsetdash{}{0pt}%
\pgfpathmoveto{\pgfqpoint{0.840504in}{0.670138in}}%
\pgfpathlineto{\pgfqpoint{0.840504in}{4.600000in}}%
\pgfusepath{stroke}%
\end{pgfscope}%
\begin{pgfscope}%
\pgfsetrectcap%
\pgfsetmiterjoin%
\pgfsetlinewidth{0.803000pt}%
\definecolor{currentstroke}{rgb}{0.000000,0.000000,0.000000}%
\pgfsetstrokecolor{currentstroke}%
\pgfsetdash{}{0pt}%
\pgfpathmoveto{\pgfqpoint{6.752313in}{0.670138in}}%
\pgfpathlineto{\pgfqpoint{6.752313in}{4.600000in}}%
\pgfusepath{stroke}%
\end{pgfscope}%
\begin{pgfscope}%
\pgfsetrectcap%
\pgfsetmiterjoin%
\pgfsetlinewidth{0.803000pt}%
\definecolor{currentstroke}{rgb}{0.000000,0.000000,0.000000}%
\pgfsetstrokecolor{currentstroke}%
\pgfsetdash{}{0pt}%
\pgfpathmoveto{\pgfqpoint{0.840504in}{0.670138in}}%
\pgfpathlineto{\pgfqpoint{6.752313in}{0.670138in}}%
\pgfusepath{stroke}%
\end{pgfscope}%
\begin{pgfscope}%
\pgfsetrectcap%
\pgfsetmiterjoin%
\pgfsetlinewidth{0.803000pt}%
\definecolor{currentstroke}{rgb}{0.000000,0.000000,0.000000}%
\pgfsetstrokecolor{currentstroke}%
\pgfsetdash{}{0pt}%
\pgfpathmoveto{\pgfqpoint{0.840504in}{4.600000in}}%
\pgfpathlineto{\pgfqpoint{6.752313in}{4.600000in}}%
\pgfusepath{stroke}%
\end{pgfscope}%
\begin{pgfscope}%
\pgfsetbuttcap%
\pgfsetmiterjoin%
\definecolor{currentfill}{rgb}{0.300000,0.300000,0.300000}%
\pgfsetfillcolor{currentfill}%
\pgfsetfillopacity{0.500000}%
\pgfsetlinewidth{1.003750pt}%
\definecolor{currentstroke}{rgb}{0.300000,0.300000,0.300000}%
\pgfsetstrokecolor{currentstroke}%
\pgfsetstrokeopacity{0.500000}%
\pgfsetdash{}{0pt}%
\pgfpathmoveto{\pgfqpoint{4.218378in}{3.342362in}}%
\pgfpathlineto{\pgfqpoint{6.624535in}{3.342362in}}%
\pgfpathquadraticcurveto{\pgfqpoint{6.668980in}{3.342362in}}{\pgfqpoint{6.668980in}{3.386807in}}%
\pgfpathlineto{\pgfqpoint{6.668980in}{4.416667in}}%
\pgfpathquadraticcurveto{\pgfqpoint{6.668980in}{4.461111in}}{\pgfqpoint{6.624535in}{4.461111in}}%
\pgfpathlineto{\pgfqpoint{4.218378in}{4.461111in}}%
\pgfpathquadraticcurveto{\pgfqpoint{4.173933in}{4.461111in}}{\pgfqpoint{4.173933in}{4.416667in}}%
\pgfpathlineto{\pgfqpoint{4.173933in}{3.386807in}}%
\pgfpathquadraticcurveto{\pgfqpoint{4.173933in}{3.342362in}}{\pgfqpoint{4.218378in}{3.342362in}}%
\pgfpathlineto{\pgfqpoint{4.218378in}{3.342362in}}%
\pgfpathclose%
\pgfusepath{stroke,fill}%
\end{pgfscope}%
\begin{pgfscope}%
\pgfsetbuttcap%
\pgfsetmiterjoin%
\definecolor{currentfill}{rgb}{1.000000,1.000000,1.000000}%
\pgfsetfillcolor{currentfill}%
\pgfsetlinewidth{1.003750pt}%
\definecolor{currentstroke}{rgb}{0.800000,0.800000,0.800000}%
\pgfsetstrokecolor{currentstroke}%
\pgfsetdash{}{0pt}%
\pgfpathmoveto{\pgfqpoint{4.190600in}{3.370140in}}%
\pgfpathlineto{\pgfqpoint{6.596757in}{3.370140in}}%
\pgfpathquadraticcurveto{\pgfqpoint{6.641202in}{3.370140in}}{\pgfqpoint{6.641202in}{3.414585in}}%
\pgfpathlineto{\pgfqpoint{6.641202in}{4.444444in}}%
\pgfpathquadraticcurveto{\pgfqpoint{6.641202in}{4.488889in}}{\pgfqpoint{6.596757in}{4.488889in}}%
\pgfpathlineto{\pgfqpoint{4.190600in}{4.488889in}}%
\pgfpathquadraticcurveto{\pgfqpoint{4.146156in}{4.488889in}}{\pgfqpoint{4.146156in}{4.444444in}}%
\pgfpathlineto{\pgfqpoint{4.146156in}{3.414585in}}%
\pgfpathquadraticcurveto{\pgfqpoint{4.146156in}{3.370140in}}{\pgfqpoint{4.190600in}{3.370140in}}%
\pgfpathlineto{\pgfqpoint{4.190600in}{3.370140in}}%
\pgfpathclose%
\pgfusepath{stroke,fill}%
\end{pgfscope}%
\begin{pgfscope}%
\pgfsetbuttcap%
\pgfsetroundjoin%
\definecolor{currentfill}{rgb}{0.121569,0.466667,0.705882}%
\pgfsetfillcolor{currentfill}%
\pgfsetlinewidth{1.003750pt}%
\definecolor{currentstroke}{rgb}{0.121569,0.466667,0.705882}%
\pgfsetstrokecolor{currentstroke}%
\pgfsetdash{}{0pt}%
\pgfsys@defobject{currentmarker}{\pgfqpoint{-0.038036in}{-0.038036in}}{\pgfqpoint{0.038036in}{0.038036in}}{%
\pgfpathmoveto{\pgfqpoint{-0.038036in}{-0.038036in}}%
\pgfpathlineto{\pgfqpoint{0.038036in}{-0.038036in}}%
\pgfpathlineto{\pgfqpoint{0.038036in}{0.038036in}}%
\pgfpathlineto{\pgfqpoint{-0.038036in}{0.038036in}}%
\pgfpathlineto{\pgfqpoint{-0.038036in}{-0.038036in}}%
\pgfpathclose%
\pgfusepath{stroke,fill}%
}%
\begin{pgfscope}%
\pgfsys@transformshift{4.457267in}{4.278550in}%
\pgfsys@useobject{currentmarker}{}%
\end{pgfscope}%
\end{pgfscope}%
\begin{pgfscope}%
\definecolor{textcolor}{rgb}{0.000000,0.000000,0.000000}%
\pgfsetstrokecolor{textcolor}%
\pgfsetfillcolor{textcolor}%
\pgftext[x=4.857267in,y=4.220216in,left,base]{\color{textcolor}{\rmfamily\fontsize{16.000000}{19.200000}\selectfont\catcode`\^=\active\def^{\ifmmode\sp\else\^{}\fi}\catcode`\%=\active\def%{\%}Pymoo/UNSGA3}}%
\end{pgfscope}%
\begin{pgfscope}%
\pgfsetbuttcap%
\pgfsetroundjoin%
\definecolor{currentfill}{rgb}{1.000000,0.498039,0.054902}%
\pgfsetfillcolor{currentfill}%
\pgfsetlinewidth{1.003750pt}%
\definecolor{currentstroke}{rgb}{1.000000,0.498039,0.054902}%
\pgfsetstrokecolor{currentstroke}%
\pgfsetdash{}{0pt}%
\pgfsys@defobject{currentmarker}{\pgfqpoint{-0.036175in}{-0.030772in}}{\pgfqpoint{0.036175in}{0.038036in}}{%
\pgfpathmoveto{\pgfqpoint{0.000000in}{0.038036in}}%
\pgfpathlineto{\pgfqpoint{-0.008540in}{0.011754in}}%
\pgfpathlineto{\pgfqpoint{-0.036175in}{0.011754in}}%
\pgfpathlineto{\pgfqpoint{-0.013817in}{-0.004490in}}%
\pgfpathlineto{\pgfqpoint{-0.022357in}{-0.030772in}}%
\pgfpathlineto{\pgfqpoint{-0.000000in}{-0.014529in}}%
\pgfpathlineto{\pgfqpoint{0.022357in}{-0.030772in}}%
\pgfpathlineto{\pgfqpoint{0.013817in}{-0.004490in}}%
\pgfpathlineto{\pgfqpoint{0.036175in}{0.011754in}}%
\pgfpathlineto{\pgfqpoint{0.008540in}{0.011754in}}%
\pgfpathlineto{\pgfqpoint{0.000000in}{0.038036in}}%
\pgfpathclose%
\pgfusepath{stroke,fill}%
}%
\begin{pgfscope}%
\pgfsys@transformshift{4.457267in}{3.927856in}%
\pgfsys@useobject{currentmarker}{}%
\end{pgfscope}%
\end{pgfscope}%
\begin{pgfscope}%
\definecolor{textcolor}{rgb}{0.000000,0.000000,0.000000}%
\pgfsetstrokecolor{textcolor}%
\pgfsetfillcolor{textcolor}%
\pgftext[x=4.857267in,y=3.869522in,left,base]{\color{textcolor}{\rmfamily\fontsize{16.000000}{19.200000}\selectfont\catcode`\^=\active\def^{\ifmmode\sp\else\^{}\fi}\catcode`\%=\active\def%{\%}DEAP/NSGA3}}%
\end{pgfscope}%
\begin{pgfscope}%
\pgfsetbuttcap%
\pgfsetroundjoin%
\definecolor{currentfill}{rgb}{0.580392,0.403922,0.741176}%
\pgfsetfillcolor{currentfill}%
\pgfsetlinewidth{1.003750pt}%
\definecolor{currentstroke}{rgb}{0.580392,0.403922,0.741176}%
\pgfsetstrokecolor{currentstroke}%
\pgfsetdash{}{0pt}%
\pgfsys@defobject{currentmarker}{\pgfqpoint{-0.038036in}{-0.038036in}}{\pgfqpoint{0.038036in}{0.038036in}}{%
\pgfpathmoveto{\pgfqpoint{0.000000in}{-0.038036in}}%
\pgfpathcurveto{\pgfqpoint{0.010087in}{-0.038036in}}{\pgfqpoint{0.019763in}{-0.034029in}}{\pgfqpoint{0.026896in}{-0.026896in}}%
\pgfpathcurveto{\pgfqpoint{0.034029in}{-0.019763in}}{\pgfqpoint{0.038036in}{-0.010087in}}{\pgfqpoint{0.038036in}{0.000000in}}%
\pgfpathcurveto{\pgfqpoint{0.038036in}{0.010087in}}{\pgfqpoint{0.034029in}{0.019763in}}{\pgfqpoint{0.026896in}{0.026896in}}%
\pgfpathcurveto{\pgfqpoint{0.019763in}{0.034029in}}{\pgfqpoint{0.010087in}{0.038036in}}{\pgfqpoint{0.000000in}{0.038036in}}%
\pgfpathcurveto{\pgfqpoint{-0.010087in}{0.038036in}}{\pgfqpoint{-0.019763in}{0.034029in}}{\pgfqpoint{-0.026896in}{0.026896in}}%
\pgfpathcurveto{\pgfqpoint{-0.034029in}{0.019763in}}{\pgfqpoint{-0.038036in}{0.010087in}}{\pgfqpoint{-0.038036in}{0.000000in}}%
\pgfpathcurveto{\pgfqpoint{-0.038036in}{-0.010087in}}{\pgfqpoint{-0.034029in}{-0.019763in}}{\pgfqpoint{-0.026896in}{-0.026896in}}%
\pgfpathcurveto{\pgfqpoint{-0.019763in}{-0.034029in}}{\pgfqpoint{-0.010087in}{-0.038036in}}{\pgfqpoint{0.000000in}{-0.038036in}}%
\pgfpathlineto{\pgfqpoint{0.000000in}{-0.038036in}}%
\pgfpathclose%
\pgfusepath{stroke,fill}%
}%
\begin{pgfscope}%
\pgfsys@transformshift{4.457267in}{3.577162in}%
\pgfsys@useobject{currentmarker}{}%
\end{pgfscope}%
\end{pgfscope}%
\begin{pgfscope}%
\definecolor{textcolor}{rgb}{0.000000,0.000000,0.000000}%
\pgfsetstrokecolor{textcolor}%
\pgfsetfillcolor{textcolor}%
\pgftext[x=4.857267in,y=3.518828in,left,base]{\color{textcolor}{\rmfamily\fontsize{16.000000}{19.200000}\selectfont\catcode`\^=\active\def^{\ifmmode\sp\else\^{}\fi}\catcode`\%=\active\def%{\%}DEAP/NSGA2}}%
\end{pgfscope}%
\begin{pgfscope}%
\pgfsetbuttcap%
\pgfsetmiterjoin%
\definecolor{currentfill}{rgb}{1.000000,1.000000,1.000000}%
\pgfsetfillcolor{currentfill}%
\pgfsetlinewidth{0.000000pt}%
\definecolor{currentstroke}{rgb}{0.000000,0.000000,0.000000}%
\pgfsetstrokecolor{currentstroke}%
\pgfsetstrokeopacity{0.000000}%
\pgfsetdash{}{0pt}%
\pgfpathmoveto{\pgfqpoint{0.840504in}{4.600000in}}%
\pgfpathlineto{\pgfqpoint{6.752313in}{4.600000in}}%
\pgfpathlineto{\pgfqpoint{6.752313in}{5.800000in}}%
\pgfpathlineto{\pgfqpoint{0.840504in}{5.800000in}}%
\pgfpathlineto{\pgfqpoint{0.840504in}{4.600000in}}%
\pgfpathclose%
\pgfusepath{fill}%
\end{pgfscope}%
\begin{pgfscope}%
\pgfpathrectangle{\pgfqpoint{0.840504in}{4.600000in}}{\pgfqpoint{5.911808in}{1.200000in}}%
\pgfusepath{clip}%
\pgfsetbuttcap%
\pgfsetroundjoin%
\definecolor{currentfill}{rgb}{0.121569,0.466667,0.705882}%
\pgfsetfillcolor{currentfill}%
\pgfsetfillopacity{0.300000}%
\pgfsetlinewidth{1.003750pt}%
\definecolor{currentstroke}{rgb}{0.121569,0.466667,0.705882}%
\pgfsetstrokecolor{currentstroke}%
\pgfsetstrokeopacity{0.300000}%
\pgfsetdash{}{0pt}%
\pgfsys@defobject{currentmarker}{\pgfqpoint{0.990438in}{4.654545in}}{\pgfqpoint{2.030903in}{5.745455in}}{%
\pgfpathmoveto{\pgfqpoint{0.990438in}{5.580241in}}%
\pgfpathlineto{\pgfqpoint{0.990438in}{4.654545in}}%
\pgfpathlineto{\pgfqpoint{0.994362in}{4.654545in}}%
\pgfpathlineto{\pgfqpoint{0.996518in}{4.654545in}}%
\pgfpathlineto{\pgfqpoint{0.998005in}{4.654545in}}%
\pgfpathlineto{\pgfqpoint{0.998799in}{4.654545in}}%
\pgfpathlineto{\pgfqpoint{0.999689in}{4.654545in}}%
\pgfpathlineto{\pgfqpoint{1.000678in}{4.654545in}}%
\pgfpathlineto{\pgfqpoint{1.002046in}{4.654545in}}%
\pgfpathlineto{\pgfqpoint{1.002530in}{4.654545in}}%
\pgfpathlineto{\pgfqpoint{1.003703in}{4.654545in}}%
\pgfpathlineto{\pgfqpoint{1.005869in}{4.654545in}}%
\pgfpathlineto{\pgfqpoint{1.006609in}{4.654545in}}%
\pgfpathlineto{\pgfqpoint{1.007368in}{4.654545in}}%
\pgfpathlineto{\pgfqpoint{1.008961in}{4.654545in}}%
\pgfpathlineto{\pgfqpoint{1.010287in}{4.654545in}}%
\pgfpathlineto{\pgfqpoint{1.010375in}{4.654545in}}%
\pgfpathlineto{\pgfqpoint{1.011245in}{4.654545in}}%
\pgfpathlineto{\pgfqpoint{1.014746in}{4.654545in}}%
\pgfpathlineto{\pgfqpoint{1.018469in}{4.654545in}}%
\pgfpathlineto{\pgfqpoint{1.023523in}{4.654545in}}%
\pgfpathlineto{\pgfqpoint{1.028150in}{4.654545in}}%
\pgfpathlineto{\pgfqpoint{1.028339in}{4.654545in}}%
\pgfpathlineto{\pgfqpoint{1.028612in}{4.654545in}}%
\pgfpathlineto{\pgfqpoint{1.029345in}{4.654545in}}%
\pgfpathlineto{\pgfqpoint{1.031093in}{4.654545in}}%
\pgfpathlineto{\pgfqpoint{1.032281in}{4.654545in}}%
\pgfpathlineto{\pgfqpoint{1.037557in}{4.654545in}}%
\pgfpathlineto{\pgfqpoint{1.038248in}{4.654545in}}%
\pgfpathlineto{\pgfqpoint{1.038269in}{4.654545in}}%
\pgfpathlineto{\pgfqpoint{1.039403in}{4.654545in}}%
\pgfpathlineto{\pgfqpoint{1.042083in}{4.654545in}}%
\pgfpathlineto{\pgfqpoint{1.043351in}{4.654545in}}%
\pgfpathlineto{\pgfqpoint{1.043547in}{4.654545in}}%
\pgfpathlineto{\pgfqpoint{1.054106in}{4.654545in}}%
\pgfpathlineto{\pgfqpoint{1.054932in}{4.654545in}}%
\pgfpathlineto{\pgfqpoint{1.055402in}{4.654545in}}%
\pgfpathlineto{\pgfqpoint{1.057074in}{4.654545in}}%
\pgfpathlineto{\pgfqpoint{1.060383in}{4.654545in}}%
\pgfpathlineto{\pgfqpoint{1.061588in}{4.654545in}}%
\pgfpathlineto{\pgfqpoint{1.062702in}{4.654545in}}%
\pgfpathlineto{\pgfqpoint{1.065546in}{4.654545in}}%
\pgfpathlineto{\pgfqpoint{1.070103in}{4.654545in}}%
\pgfpathlineto{\pgfqpoint{1.075327in}{4.654545in}}%
\pgfpathlineto{\pgfqpoint{1.079103in}{4.654545in}}%
\pgfpathlineto{\pgfqpoint{1.083825in}{4.654545in}}%
\pgfpathlineto{\pgfqpoint{1.106237in}{4.654545in}}%
\pgfpathlineto{\pgfqpoint{1.112179in}{4.654545in}}%
\pgfpathlineto{\pgfqpoint{1.120671in}{4.654545in}}%
\pgfpathlineto{\pgfqpoint{1.133022in}{4.654545in}}%
\pgfpathlineto{\pgfqpoint{1.139707in}{4.654545in}}%
\pgfpathlineto{\pgfqpoint{1.155415in}{4.654545in}}%
\pgfpathlineto{\pgfqpoint{1.175667in}{4.654545in}}%
\pgfpathlineto{\pgfqpoint{1.191614in}{4.654545in}}%
\pgfpathlineto{\pgfqpoint{1.216706in}{4.654545in}}%
\pgfpathlineto{\pgfqpoint{1.250615in}{4.654545in}}%
\pgfpathlineto{\pgfqpoint{1.284782in}{4.654545in}}%
\pgfpathlineto{\pgfqpoint{1.334942in}{4.654545in}}%
\pgfpathlineto{\pgfqpoint{1.418516in}{4.654545in}}%
\pgfpathlineto{\pgfqpoint{1.563285in}{4.654545in}}%
\pgfpathlineto{\pgfqpoint{2.030903in}{4.654545in}}%
\pgfpathlineto{\pgfqpoint{2.030903in}{4.678717in}}%
\pgfpathlineto{\pgfqpoint{2.030903in}{4.678717in}}%
\pgfpathlineto{\pgfqpoint{1.563285in}{4.682113in}}%
\pgfpathlineto{\pgfqpoint{1.418516in}{4.701013in}}%
\pgfpathlineto{\pgfqpoint{1.334942in}{4.737357in}}%
\pgfpathlineto{\pgfqpoint{1.284782in}{4.776373in}}%
\pgfpathlineto{\pgfqpoint{1.250615in}{4.819297in}}%
\pgfpathlineto{\pgfqpoint{1.216706in}{4.887691in}}%
\pgfpathlineto{\pgfqpoint{1.191614in}{4.965040in}}%
\pgfpathlineto{\pgfqpoint{1.175667in}{5.029865in}}%
\pgfpathlineto{\pgfqpoint{1.155415in}{5.131779in}}%
\pgfpathlineto{\pgfqpoint{1.139707in}{5.225244in}}%
\pgfpathlineto{\pgfqpoint{1.133022in}{5.268199in}}%
\pgfpathlineto{\pgfqpoint{1.120671in}{5.351010in}}%
\pgfpathlineto{\pgfqpoint{1.112179in}{5.409264in}}%
\pgfpathlineto{\pgfqpoint{1.106237in}{5.449929in}}%
\pgfpathlineto{\pgfqpoint{1.083825in}{5.593645in}}%
\pgfpathlineto{\pgfqpoint{1.079103in}{5.620210in}}%
\pgfpathlineto{\pgfqpoint{1.075327in}{5.640093in}}%
\pgfpathlineto{\pgfqpoint{1.070103in}{5.665346in}}%
\pgfpathlineto{\pgfqpoint{1.065546in}{5.684995in}}%
\pgfpathlineto{\pgfqpoint{1.062702in}{5.696032in}}%
\pgfpathlineto{\pgfqpoint{1.061588in}{5.700083in}}%
\pgfpathlineto{\pgfqpoint{1.060383in}{5.704292in}}%
\pgfpathlineto{\pgfqpoint{1.057074in}{5.714874in}}%
\pgfpathlineto{\pgfqpoint{1.055402in}{5.719664in}}%
\pgfpathlineto{\pgfqpoint{1.054932in}{5.720941in}}%
\pgfpathlineto{\pgfqpoint{1.054106in}{5.723113in}}%
\pgfpathlineto{\pgfqpoint{1.043547in}{5.742168in}}%
\pgfpathlineto{\pgfqpoint{1.043351in}{5.742364in}}%
\pgfpathlineto{\pgfqpoint{1.042083in}{5.743488in}}%
\pgfpathlineto{\pgfqpoint{1.039403in}{5.745040in}}%
\pgfpathlineto{\pgfqpoint{1.038269in}{5.745358in}}%
\pgfpathlineto{\pgfqpoint{1.038248in}{5.745362in}}%
\pgfpathlineto{\pgfqpoint{1.037557in}{5.745455in}}%
\pgfpathlineto{\pgfqpoint{1.032281in}{5.743659in}}%
\pgfpathlineto{\pgfqpoint{1.031093in}{5.742642in}}%
\pgfpathlineto{\pgfqpoint{1.029345in}{5.740734in}}%
\pgfpathlineto{\pgfqpoint{1.028612in}{5.739790in}}%
\pgfpathlineto{\pgfqpoint{1.028339in}{5.739415in}}%
\pgfpathlineto{\pgfqpoint{1.028150in}{5.739150in}}%
\pgfpathlineto{\pgfqpoint{1.023523in}{5.730868in}}%
\pgfpathlineto{\pgfqpoint{1.018469in}{5.717969in}}%
\pgfpathlineto{\pgfqpoint{1.014746in}{5.705943in}}%
\pgfpathlineto{\pgfqpoint{1.011245in}{5.692740in}}%
\pgfpathlineto{\pgfqpoint{1.010375in}{5.689182in}}%
\pgfpathlineto{\pgfqpoint{1.010287in}{5.688814in}}%
\pgfpathlineto{\pgfqpoint{1.008961in}{5.683164in}}%
\pgfpathlineto{\pgfqpoint{1.007368in}{5.676047in}}%
\pgfpathlineto{\pgfqpoint{1.006609in}{5.672530in}}%
\pgfpathlineto{\pgfqpoint{1.005869in}{5.669026in}}%
\pgfpathlineto{\pgfqpoint{1.003703in}{5.658349in}}%
\pgfpathlineto{\pgfqpoint{1.002530in}{5.652301in}}%
\pgfpathlineto{\pgfqpoint{1.002046in}{5.649755in}}%
\pgfpathlineto{\pgfqpoint{1.000678in}{5.642399in}}%
\pgfpathlineto{\pgfqpoint{0.999689in}{5.636930in}}%
\pgfpathlineto{\pgfqpoint{0.998799in}{5.631910in}}%
\pgfpathlineto{\pgfqpoint{0.998005in}{5.627346in}}%
\pgfpathlineto{\pgfqpoint{0.996518in}{5.618599in}}%
\pgfpathlineto{\pgfqpoint{0.994362in}{5.605459in}}%
\pgfpathlineto{\pgfqpoint{0.990438in}{5.580241in}}%
\pgfpathlineto{\pgfqpoint{0.990438in}{5.580241in}}%
\pgfpathclose%
\pgfusepath{stroke,fill}%
}%
\begin{pgfscope}%
\pgfsys@transformshift{0.000000in}{0.000000in}%
\pgfsys@useobject{currentmarker}{}%
\end{pgfscope}%
\end{pgfscope}%
\begin{pgfscope}%
\pgfpathrectangle{\pgfqpoint{0.840504in}{4.600000in}}{\pgfqpoint{5.911808in}{1.200000in}}%
\pgfusepath{clip}%
\pgfsetbuttcap%
\pgfsetroundjoin%
\definecolor{currentfill}{rgb}{1.000000,0.498039,0.054902}%
\pgfsetfillcolor{currentfill}%
\pgfsetfillopacity{0.300000}%
\pgfsetlinewidth{1.003750pt}%
\definecolor{currentstroke}{rgb}{1.000000,0.498039,0.054902}%
\pgfsetstrokecolor{currentstroke}%
\pgfsetstrokeopacity{0.300000}%
\pgfsetdash{}{0pt}%
\pgfsys@defobject{currentmarker}{\pgfqpoint{0.963227in}{4.654545in}}{\pgfqpoint{6.715212in}{4.747495in}}{%
\pgfpathmoveto{\pgfqpoint{0.963227in}{4.705307in}}%
\pgfpathlineto{\pgfqpoint{0.963227in}{4.654545in}}%
\pgfpathlineto{\pgfqpoint{0.963263in}{4.654545in}}%
\pgfpathlineto{\pgfqpoint{0.963558in}{4.654545in}}%
\pgfpathlineto{\pgfqpoint{0.964599in}{4.654545in}}%
\pgfpathlineto{\pgfqpoint{0.964929in}{4.654545in}}%
\pgfpathlineto{\pgfqpoint{0.969198in}{4.654545in}}%
\pgfpathlineto{\pgfqpoint{0.970574in}{4.654545in}}%
\pgfpathlineto{\pgfqpoint{0.978103in}{4.654545in}}%
\pgfpathlineto{\pgfqpoint{0.984934in}{4.654545in}}%
\pgfpathlineto{\pgfqpoint{0.986801in}{4.654545in}}%
\pgfpathlineto{\pgfqpoint{0.995218in}{4.654545in}}%
\pgfpathlineto{\pgfqpoint{1.023323in}{4.654545in}}%
\pgfpathlineto{\pgfqpoint{1.038589in}{4.654545in}}%
\pgfpathlineto{\pgfqpoint{1.082132in}{4.654545in}}%
\pgfpathlineto{\pgfqpoint{1.264828in}{4.654545in}}%
\pgfpathlineto{\pgfqpoint{1.303005in}{4.654545in}}%
\pgfpathlineto{\pgfqpoint{1.324570in}{4.654545in}}%
\pgfpathlineto{\pgfqpoint{1.334574in}{4.654545in}}%
\pgfpathlineto{\pgfqpoint{1.346285in}{4.654545in}}%
\pgfpathlineto{\pgfqpoint{1.377598in}{4.654545in}}%
\pgfpathlineto{\pgfqpoint{1.391323in}{4.654545in}}%
\pgfpathlineto{\pgfqpoint{1.403777in}{4.654545in}}%
\pgfpathlineto{\pgfqpoint{1.448572in}{4.654545in}}%
\pgfpathlineto{\pgfqpoint{1.450602in}{4.654545in}}%
\pgfpathlineto{\pgfqpoint{1.452464in}{4.654545in}}%
\pgfpathlineto{\pgfqpoint{1.455736in}{4.654545in}}%
\pgfpathlineto{\pgfqpoint{1.459934in}{4.654545in}}%
\pgfpathlineto{\pgfqpoint{1.463183in}{4.654545in}}%
\pgfpathlineto{\pgfqpoint{1.477616in}{4.654545in}}%
\pgfpathlineto{\pgfqpoint{1.478603in}{4.654545in}}%
\pgfpathlineto{\pgfqpoint{1.478878in}{4.654545in}}%
\pgfpathlineto{\pgfqpoint{1.480012in}{4.654545in}}%
\pgfpathlineto{\pgfqpoint{1.482451in}{4.654545in}}%
\pgfpathlineto{\pgfqpoint{1.484093in}{4.654545in}}%
\pgfpathlineto{\pgfqpoint{1.484692in}{4.654545in}}%
\pgfpathlineto{\pgfqpoint{1.505045in}{4.654545in}}%
\pgfpathlineto{\pgfqpoint{1.568922in}{4.654545in}}%
\pgfpathlineto{\pgfqpoint{1.568923in}{4.654545in}}%
\pgfpathlineto{\pgfqpoint{1.574843in}{4.654545in}}%
\pgfpathlineto{\pgfqpoint{1.592629in}{4.654545in}}%
\pgfpathlineto{\pgfqpoint{1.597367in}{4.654545in}}%
\pgfpathlineto{\pgfqpoint{1.620609in}{4.654545in}}%
\pgfpathlineto{\pgfqpoint{1.658402in}{4.654545in}}%
\pgfpathlineto{\pgfqpoint{1.660546in}{4.654545in}}%
\pgfpathlineto{\pgfqpoint{1.686963in}{4.654545in}}%
\pgfpathlineto{\pgfqpoint{1.697321in}{4.654545in}}%
\pgfpathlineto{\pgfqpoint{1.716525in}{4.654545in}}%
\pgfpathlineto{\pgfqpoint{1.724449in}{4.654545in}}%
\pgfpathlineto{\pgfqpoint{1.728245in}{4.654545in}}%
\pgfpathlineto{\pgfqpoint{1.766500in}{4.654545in}}%
\pgfpathlineto{\pgfqpoint{1.766992in}{4.654545in}}%
\pgfpathlineto{\pgfqpoint{1.767290in}{4.654545in}}%
\pgfpathlineto{\pgfqpoint{1.926935in}{4.654545in}}%
\pgfpathlineto{\pgfqpoint{1.947114in}{4.654545in}}%
\pgfpathlineto{\pgfqpoint{1.970932in}{4.654545in}}%
\pgfpathlineto{\pgfqpoint{1.984762in}{4.654545in}}%
\pgfpathlineto{\pgfqpoint{2.075979in}{4.654545in}}%
\pgfpathlineto{\pgfqpoint{2.078268in}{4.654545in}}%
\pgfpathlineto{\pgfqpoint{2.080214in}{4.654545in}}%
\pgfpathlineto{\pgfqpoint{2.080220in}{4.654545in}}%
\pgfpathlineto{\pgfqpoint{2.080251in}{4.654545in}}%
\pgfpathlineto{\pgfqpoint{2.084391in}{4.654545in}}%
\pgfpathlineto{\pgfqpoint{2.086304in}{4.654545in}}%
\pgfpathlineto{\pgfqpoint{2.109970in}{4.654545in}}%
\pgfpathlineto{\pgfqpoint{2.205747in}{4.654545in}}%
\pgfpathlineto{\pgfqpoint{2.215174in}{4.654545in}}%
\pgfpathlineto{\pgfqpoint{2.223618in}{4.654545in}}%
\pgfpathlineto{\pgfqpoint{2.383143in}{4.654545in}}%
\pgfpathlineto{\pgfqpoint{2.741740in}{4.654545in}}%
\pgfpathlineto{\pgfqpoint{2.750342in}{4.654545in}}%
\pgfpathlineto{\pgfqpoint{2.753656in}{4.654545in}}%
\pgfpathlineto{\pgfqpoint{2.762932in}{4.654545in}}%
\pgfpathlineto{\pgfqpoint{2.771464in}{4.654545in}}%
\pgfpathlineto{\pgfqpoint{2.772052in}{4.654545in}}%
\pgfpathlineto{\pgfqpoint{2.772052in}{4.654545in}}%
\pgfpathlineto{\pgfqpoint{2.772094in}{4.654545in}}%
\pgfpathlineto{\pgfqpoint{2.772118in}{4.654545in}}%
\pgfpathlineto{\pgfqpoint{2.772119in}{4.654545in}}%
\pgfpathlineto{\pgfqpoint{2.777491in}{4.654545in}}%
\pgfpathlineto{\pgfqpoint{2.787731in}{4.654545in}}%
\pgfpathlineto{\pgfqpoint{2.792244in}{4.654545in}}%
\pgfpathlineto{\pgfqpoint{2.792342in}{4.654545in}}%
\pgfpathlineto{\pgfqpoint{2.813119in}{4.654545in}}%
\pgfpathlineto{\pgfqpoint{2.813477in}{4.654545in}}%
\pgfpathlineto{\pgfqpoint{2.822480in}{4.654545in}}%
\pgfpathlineto{\pgfqpoint{2.827835in}{4.654545in}}%
\pgfpathlineto{\pgfqpoint{2.847035in}{4.654545in}}%
\pgfpathlineto{\pgfqpoint{2.854910in}{4.654545in}}%
\pgfpathlineto{\pgfqpoint{2.874033in}{4.654545in}}%
\pgfpathlineto{\pgfqpoint{2.874954in}{4.654545in}}%
\pgfpathlineto{\pgfqpoint{2.876587in}{4.654545in}}%
\pgfpathlineto{\pgfqpoint{2.882375in}{4.654545in}}%
\pgfpathlineto{\pgfqpoint{2.882476in}{4.654545in}}%
\pgfpathlineto{\pgfqpoint{2.920528in}{4.654545in}}%
\pgfpathlineto{\pgfqpoint{2.934019in}{4.654545in}}%
\pgfpathlineto{\pgfqpoint{2.963978in}{4.654545in}}%
\pgfpathlineto{\pgfqpoint{2.964684in}{4.654545in}}%
\pgfpathlineto{\pgfqpoint{2.964719in}{4.654545in}}%
\pgfpathlineto{\pgfqpoint{2.976951in}{4.654545in}}%
\pgfpathlineto{\pgfqpoint{2.979762in}{4.654545in}}%
\pgfpathlineto{\pgfqpoint{3.004863in}{4.654545in}}%
\pgfpathlineto{\pgfqpoint{3.031188in}{4.654545in}}%
\pgfpathlineto{\pgfqpoint{3.047864in}{4.654545in}}%
\pgfpathlineto{\pgfqpoint{3.054666in}{4.654545in}}%
\pgfpathlineto{\pgfqpoint{3.055743in}{4.654545in}}%
\pgfpathlineto{\pgfqpoint{3.059460in}{4.654545in}}%
\pgfpathlineto{\pgfqpoint{3.076386in}{4.654545in}}%
\pgfpathlineto{\pgfqpoint{3.079713in}{4.654545in}}%
\pgfpathlineto{\pgfqpoint{3.132051in}{4.654545in}}%
\pgfpathlineto{\pgfqpoint{3.181935in}{4.654545in}}%
\pgfpathlineto{\pgfqpoint{3.208867in}{4.654545in}}%
\pgfpathlineto{\pgfqpoint{3.298038in}{4.654545in}}%
\pgfpathlineto{\pgfqpoint{3.321491in}{4.654545in}}%
\pgfpathlineto{\pgfqpoint{3.380661in}{4.654545in}}%
\pgfpathlineto{\pgfqpoint{3.452729in}{4.654545in}}%
\pgfpathlineto{\pgfqpoint{3.457815in}{4.654545in}}%
\pgfpathlineto{\pgfqpoint{3.477407in}{4.654545in}}%
\pgfpathlineto{\pgfqpoint{3.482615in}{4.654545in}}%
\pgfpathlineto{\pgfqpoint{3.490248in}{4.654545in}}%
\pgfpathlineto{\pgfqpoint{3.496699in}{4.654545in}}%
\pgfpathlineto{\pgfqpoint{3.499237in}{4.654545in}}%
\pgfpathlineto{\pgfqpoint{3.510539in}{4.654545in}}%
\pgfpathlineto{\pgfqpoint{3.512564in}{4.654545in}}%
\pgfpathlineto{\pgfqpoint{3.516999in}{4.654545in}}%
\pgfpathlineto{\pgfqpoint{3.518716in}{4.654545in}}%
\pgfpathlineto{\pgfqpoint{3.534539in}{4.654545in}}%
\pgfpathlineto{\pgfqpoint{3.539920in}{4.654545in}}%
\pgfpathlineto{\pgfqpoint{3.551229in}{4.654545in}}%
\pgfpathlineto{\pgfqpoint{3.559556in}{4.654545in}}%
\pgfpathlineto{\pgfqpoint{3.575304in}{4.654545in}}%
\pgfpathlineto{\pgfqpoint{3.633265in}{4.654545in}}%
\pgfpathlineto{\pgfqpoint{3.641775in}{4.654545in}}%
\pgfpathlineto{\pgfqpoint{3.645452in}{4.654545in}}%
\pgfpathlineto{\pgfqpoint{3.664500in}{4.654545in}}%
\pgfpathlineto{\pgfqpoint{3.664825in}{4.654545in}}%
\pgfpathlineto{\pgfqpoint{3.760256in}{4.654545in}}%
\pgfpathlineto{\pgfqpoint{3.762750in}{4.654545in}}%
\pgfpathlineto{\pgfqpoint{3.789989in}{4.654545in}}%
\pgfpathlineto{\pgfqpoint{3.791286in}{4.654545in}}%
\pgfpathlineto{\pgfqpoint{3.814587in}{4.654545in}}%
\pgfpathlineto{\pgfqpoint{3.827245in}{4.654545in}}%
\pgfpathlineto{\pgfqpoint{3.830442in}{4.654545in}}%
\pgfpathlineto{\pgfqpoint{3.843727in}{4.654545in}}%
\pgfpathlineto{\pgfqpoint{3.932646in}{4.654545in}}%
\pgfpathlineto{\pgfqpoint{4.435247in}{4.654545in}}%
\pgfpathlineto{\pgfqpoint{4.503505in}{4.654545in}}%
\pgfpathlineto{\pgfqpoint{4.757331in}{4.654545in}}%
\pgfpathlineto{\pgfqpoint{5.212460in}{4.654545in}}%
\pgfpathlineto{\pgfqpoint{6.502961in}{4.654545in}}%
\pgfpathlineto{\pgfqpoint{6.715212in}{4.654545in}}%
\pgfpathlineto{\pgfqpoint{6.715212in}{4.657830in}}%
\pgfpathlineto{\pgfqpoint{6.715212in}{4.657830in}}%
\pgfpathlineto{\pgfqpoint{6.502961in}{4.657838in}}%
\pgfpathlineto{\pgfqpoint{5.212460in}{4.657878in}}%
\pgfpathlineto{\pgfqpoint{4.757331in}{4.661292in}}%
\pgfpathlineto{\pgfqpoint{4.503505in}{4.665639in}}%
\pgfpathlineto{\pgfqpoint{4.435247in}{4.667618in}}%
\pgfpathlineto{\pgfqpoint{3.932646in}{4.699182in}}%
\pgfpathlineto{\pgfqpoint{3.843727in}{4.706931in}}%
\pgfpathlineto{\pgfqpoint{3.830442in}{4.708085in}}%
\pgfpathlineto{\pgfqpoint{3.827245in}{4.708362in}}%
\pgfpathlineto{\pgfqpoint{3.814587in}{4.709456in}}%
\pgfpathlineto{\pgfqpoint{3.791286in}{4.711454in}}%
\pgfpathlineto{\pgfqpoint{3.789989in}{4.711565in}}%
\pgfpathlineto{\pgfqpoint{3.762750in}{4.713867in}}%
\pgfpathlineto{\pgfqpoint{3.760256in}{4.714076in}}%
\pgfpathlineto{\pgfqpoint{3.664825in}{4.721734in}}%
\pgfpathlineto{\pgfqpoint{3.664500in}{4.721758in}}%
\pgfpathlineto{\pgfqpoint{3.645452in}{4.723195in}}%
\pgfpathlineto{\pgfqpoint{3.641775in}{4.723469in}}%
\pgfpathlineto{\pgfqpoint{3.633265in}{4.724097in}}%
\pgfpathlineto{\pgfqpoint{3.575304in}{4.728184in}}%
\pgfpathlineto{\pgfqpoint{3.559556in}{4.729234in}}%
\pgfpathlineto{\pgfqpoint{3.551229in}{4.729779in}}%
\pgfpathlineto{\pgfqpoint{3.539920in}{4.730508in}}%
\pgfpathlineto{\pgfqpoint{3.534539in}{4.730850in}}%
\pgfpathlineto{\pgfqpoint{3.518716in}{4.731837in}}%
\pgfpathlineto{\pgfqpoint{3.516999in}{4.731943in}}%
\pgfpathlineto{\pgfqpoint{3.512564in}{4.732214in}}%
\pgfpathlineto{\pgfqpoint{3.510539in}{4.732337in}}%
\pgfpathlineto{\pgfqpoint{3.499237in}{4.733016in}}%
\pgfpathlineto{\pgfqpoint{3.496699in}{4.733166in}}%
\pgfpathlineto{\pgfqpoint{3.490248in}{4.733546in}}%
\pgfpathlineto{\pgfqpoint{3.482615in}{4.733989in}}%
\pgfpathlineto{\pgfqpoint{3.477407in}{4.734288in}}%
\pgfpathlineto{\pgfqpoint{3.457815in}{4.735387in}}%
\pgfpathlineto{\pgfqpoint{3.452729in}{4.735665in}}%
\pgfpathlineto{\pgfqpoint{3.380661in}{4.739306in}}%
\pgfpathlineto{\pgfqpoint{3.321491in}{4.741862in}}%
\pgfpathlineto{\pgfqpoint{3.298038in}{4.742762in}}%
\pgfpathlineto{\pgfqpoint{3.208867in}{4.745546in}}%
\pgfpathlineto{\pgfqpoint{3.181935in}{4.746169in}}%
\pgfpathlineto{\pgfqpoint{3.132051in}{4.747023in}}%
\pgfpathlineto{\pgfqpoint{3.079713in}{4.747461in}}%
\pgfpathlineto{\pgfqpoint{3.076386in}{4.747472in}}%
\pgfpathlineto{\pgfqpoint{3.059460in}{4.747495in}}%
\pgfpathlineto{\pgfqpoint{3.055743in}{4.747493in}}%
\pgfpathlineto{\pgfqpoint{3.054666in}{4.747492in}}%
\pgfpathlineto{\pgfqpoint{3.047864in}{4.747480in}}%
\pgfpathlineto{\pgfqpoint{3.031188in}{4.747410in}}%
\pgfpathlineto{\pgfqpoint{3.004863in}{4.747188in}}%
\pgfpathlineto{\pgfqpoint{2.979762in}{4.746844in}}%
\pgfpathlineto{\pgfqpoint{2.976951in}{4.746797in}}%
\pgfpathlineto{\pgfqpoint{2.964719in}{4.746576in}}%
\pgfpathlineto{\pgfqpoint{2.964684in}{4.746575in}}%
\pgfpathlineto{\pgfqpoint{2.963978in}{4.746561in}}%
\pgfpathlineto{\pgfqpoint{2.934019in}{4.745883in}}%
\pgfpathlineto{\pgfqpoint{2.920528in}{4.745517in}}%
\pgfpathlineto{\pgfqpoint{2.882476in}{4.744289in}}%
\pgfpathlineto{\pgfqpoint{2.882375in}{4.744285in}}%
\pgfpathlineto{\pgfqpoint{2.876587in}{4.744073in}}%
\pgfpathlineto{\pgfqpoint{2.874954in}{4.744013in}}%
\pgfpathlineto{\pgfqpoint{2.874033in}{4.743978in}}%
\pgfpathlineto{\pgfqpoint{2.854910in}{4.743224in}}%
\pgfpathlineto{\pgfqpoint{2.847035in}{4.742894in}}%
\pgfpathlineto{\pgfqpoint{2.827835in}{4.742043in}}%
\pgfpathlineto{\pgfqpoint{2.822480in}{4.741795in}}%
\pgfpathlineto{\pgfqpoint{2.813477in}{4.741366in}}%
\pgfpathlineto{\pgfqpoint{2.813119in}{4.741349in}}%
\pgfpathlineto{\pgfqpoint{2.792342in}{4.740311in}}%
\pgfpathlineto{\pgfqpoint{2.792244in}{4.740306in}}%
\pgfpathlineto{\pgfqpoint{2.787731in}{4.740072in}}%
\pgfpathlineto{\pgfqpoint{2.777491in}{4.739530in}}%
\pgfpathlineto{\pgfqpoint{2.772119in}{4.739240in}}%
\pgfpathlineto{\pgfqpoint{2.772118in}{4.739240in}}%
\pgfpathlineto{\pgfqpoint{2.772094in}{4.739239in}}%
\pgfpathlineto{\pgfqpoint{2.772052in}{4.739237in}}%
\pgfpathlineto{\pgfqpoint{2.772052in}{4.739237in}}%
\pgfpathlineto{\pgfqpoint{2.771464in}{4.739205in}}%
\pgfpathlineto{\pgfqpoint{2.762932in}{4.738736in}}%
\pgfpathlineto{\pgfqpoint{2.753656in}{4.738216in}}%
\pgfpathlineto{\pgfqpoint{2.750342in}{4.738028in}}%
\pgfpathlineto{\pgfqpoint{2.741740in}{4.737535in}}%
\pgfpathlineto{\pgfqpoint{2.383143in}{4.717887in}}%
\pgfpathlineto{\pgfqpoint{2.223618in}{4.715908in}}%
\pgfpathlineto{\pgfqpoint{2.215174in}{4.715972in}}%
\pgfpathlineto{\pgfqpoint{2.205747in}{4.716062in}}%
\pgfpathlineto{\pgfqpoint{2.109970in}{4.718048in}}%
\pgfpathlineto{\pgfqpoint{2.086304in}{4.718806in}}%
\pgfpathlineto{\pgfqpoint{2.084391in}{4.718871in}}%
\pgfpathlineto{\pgfqpoint{2.080251in}{4.719014in}}%
\pgfpathlineto{\pgfqpoint{2.080220in}{4.719015in}}%
\pgfpathlineto{\pgfqpoint{2.080214in}{4.719015in}}%
\pgfpathlineto{\pgfqpoint{2.078268in}{4.719083in}}%
\pgfpathlineto{\pgfqpoint{2.075979in}{4.719164in}}%
\pgfpathlineto{\pgfqpoint{1.984762in}{4.722928in}}%
\pgfpathlineto{\pgfqpoint{1.970932in}{4.723568in}}%
\pgfpathlineto{\pgfqpoint{1.947114in}{4.724698in}}%
\pgfpathlineto{\pgfqpoint{1.926935in}{4.725673in}}%
\pgfpathlineto{\pgfqpoint{1.767290in}{4.733109in}}%
\pgfpathlineto{\pgfqpoint{1.766992in}{4.733121in}}%
\pgfpathlineto{\pgfqpoint{1.766500in}{4.733141in}}%
\pgfpathlineto{\pgfqpoint{1.728245in}{4.734608in}}%
\pgfpathlineto{\pgfqpoint{1.724449in}{4.734743in}}%
\pgfpathlineto{\pgfqpoint{1.716525in}{4.735017in}}%
\pgfpathlineto{\pgfqpoint{1.697321in}{4.735642in}}%
\pgfpathlineto{\pgfqpoint{1.686963in}{4.735954in}}%
\pgfpathlineto{\pgfqpoint{1.660546in}{4.736666in}}%
\pgfpathlineto{\pgfqpoint{1.658402in}{4.736718in}}%
\pgfpathlineto{\pgfqpoint{1.620609in}{4.737493in}}%
\pgfpathlineto{\pgfqpoint{1.597367in}{4.737827in}}%
\pgfpathlineto{\pgfqpoint{1.592629in}{4.737881in}}%
\pgfpathlineto{\pgfqpoint{1.574843in}{4.738041in}}%
\pgfpathlineto{\pgfqpoint{1.568923in}{4.738078in}}%
\pgfpathlineto{\pgfqpoint{1.568922in}{4.738078in}}%
\pgfpathlineto{\pgfqpoint{1.505045in}{4.737982in}}%
\pgfpathlineto{\pgfqpoint{1.484692in}{4.737753in}}%
\pgfpathlineto{\pgfqpoint{1.484093in}{4.737745in}}%
\pgfpathlineto{\pgfqpoint{1.482451in}{4.737722in}}%
\pgfpathlineto{\pgfqpoint{1.480012in}{4.737687in}}%
\pgfpathlineto{\pgfqpoint{1.478878in}{4.737670in}}%
\pgfpathlineto{\pgfqpoint{1.478603in}{4.737666in}}%
\pgfpathlineto{\pgfqpoint{1.477616in}{4.737651in}}%
\pgfpathlineto{\pgfqpoint{1.463183in}{4.737406in}}%
\pgfpathlineto{\pgfqpoint{1.459934in}{4.737344in}}%
\pgfpathlineto{\pgfqpoint{1.455736in}{4.737261in}}%
\pgfpathlineto{\pgfqpoint{1.452464in}{4.737193in}}%
\pgfpathlineto{\pgfqpoint{1.450602in}{4.737153in}}%
\pgfpathlineto{\pgfqpoint{1.448572in}{4.737109in}}%
\pgfpathlineto{\pgfqpoint{1.403777in}{4.735890in}}%
\pgfpathlineto{\pgfqpoint{1.391323in}{4.735470in}}%
\pgfpathlineto{\pgfqpoint{1.377598in}{4.734967in}}%
\pgfpathlineto{\pgfqpoint{1.346285in}{4.733666in}}%
\pgfpathlineto{\pgfqpoint{1.334574in}{4.733126in}}%
\pgfpathlineto{\pgfqpoint{1.324570in}{4.732642in}}%
\pgfpathlineto{\pgfqpoint{1.303005in}{4.731531in}}%
\pgfpathlineto{\pgfqpoint{1.264828in}{4.729344in}}%
\pgfpathlineto{\pgfqpoint{1.082132in}{4.715785in}}%
\pgfpathlineto{\pgfqpoint{1.038589in}{4.712030in}}%
\pgfpathlineto{\pgfqpoint{1.023323in}{4.710686in}}%
\pgfpathlineto{\pgfqpoint{0.995218in}{4.708185in}}%
\pgfpathlineto{\pgfqpoint{0.986801in}{4.707430in}}%
\pgfpathlineto{\pgfqpoint{0.984934in}{4.707263in}}%
\pgfpathlineto{\pgfqpoint{0.978103in}{4.706648in}}%
\pgfpathlineto{\pgfqpoint{0.970574in}{4.705970in}}%
\pgfpathlineto{\pgfqpoint{0.969198in}{4.705846in}}%
\pgfpathlineto{\pgfqpoint{0.964929in}{4.705461in}}%
\pgfpathlineto{\pgfqpoint{0.964599in}{4.705431in}}%
\pgfpathlineto{\pgfqpoint{0.963558in}{4.705337in}}%
\pgfpathlineto{\pgfqpoint{0.963263in}{4.705311in}}%
\pgfpathlineto{\pgfqpoint{0.963227in}{4.705307in}}%
\pgfpathlineto{\pgfqpoint{0.963227in}{4.705307in}}%
\pgfpathclose%
\pgfusepath{stroke,fill}%
}%
\begin{pgfscope}%
\pgfsys@transformshift{0.000000in}{0.000000in}%
\pgfsys@useobject{currentmarker}{}%
\end{pgfscope}%
\end{pgfscope}%
\begin{pgfscope}%
\pgfpathrectangle{\pgfqpoint{0.840504in}{4.600000in}}{\pgfqpoint{5.911808in}{1.200000in}}%
\pgfusepath{clip}%
\pgfsetbuttcap%
\pgfsetroundjoin%
\definecolor{currentfill}{rgb}{0.580392,0.403922,0.741176}%
\pgfsetfillcolor{currentfill}%
\pgfsetfillopacity{0.300000}%
\pgfsetlinewidth{1.003750pt}%
\definecolor{currentstroke}{rgb}{0.580392,0.403922,0.741176}%
\pgfsetstrokecolor{currentstroke}%
\pgfsetstrokeopacity{0.300000}%
\pgfsetdash{}{0pt}%
\pgfsys@defobject{currentmarker}{\pgfqpoint{0.990991in}{4.654545in}}{\pgfqpoint{6.068379in}{4.739519in}}{%
\pgfpathmoveto{\pgfqpoint{0.990991in}{4.716237in}}%
\pgfpathlineto{\pgfqpoint{0.990991in}{4.654545in}}%
\pgfpathlineto{\pgfqpoint{1.015744in}{4.654545in}}%
\pgfpathlineto{\pgfqpoint{1.016253in}{4.654545in}}%
\pgfpathlineto{\pgfqpoint{1.024811in}{4.654545in}}%
\pgfpathlineto{\pgfqpoint{1.025718in}{4.654545in}}%
\pgfpathlineto{\pgfqpoint{1.104725in}{4.654545in}}%
\pgfpathlineto{\pgfqpoint{1.153053in}{4.654545in}}%
\pgfpathlineto{\pgfqpoint{1.165893in}{4.654545in}}%
\pgfpathlineto{\pgfqpoint{1.167729in}{4.654545in}}%
\pgfpathlineto{\pgfqpoint{1.171321in}{4.654545in}}%
\pgfpathlineto{\pgfqpoint{1.180938in}{4.654545in}}%
\pgfpathlineto{\pgfqpoint{1.182870in}{4.654545in}}%
\pgfpathlineto{\pgfqpoint{1.186252in}{4.654545in}}%
\pgfpathlineto{\pgfqpoint{1.187889in}{4.654545in}}%
\pgfpathlineto{\pgfqpoint{1.198560in}{4.654545in}}%
\pgfpathlineto{\pgfqpoint{1.201589in}{4.654545in}}%
\pgfpathlineto{\pgfqpoint{1.217148in}{4.654545in}}%
\pgfpathlineto{\pgfqpoint{1.219352in}{4.654545in}}%
\pgfpathlineto{\pgfqpoint{1.253596in}{4.654545in}}%
\pgfpathlineto{\pgfqpoint{1.268964in}{4.654545in}}%
\pgfpathlineto{\pgfqpoint{1.270722in}{4.654545in}}%
\pgfpathlineto{\pgfqpoint{1.401213in}{4.654545in}}%
\pgfpathlineto{\pgfqpoint{1.413547in}{4.654545in}}%
\pgfpathlineto{\pgfqpoint{1.642186in}{4.654545in}}%
\pgfpathlineto{\pgfqpoint{1.663156in}{4.654545in}}%
\pgfpathlineto{\pgfqpoint{1.669991in}{4.654545in}}%
\pgfpathlineto{\pgfqpoint{1.698362in}{4.654545in}}%
\pgfpathlineto{\pgfqpoint{1.705456in}{4.654545in}}%
\pgfpathlineto{\pgfqpoint{1.708150in}{4.654545in}}%
\pgfpathlineto{\pgfqpoint{1.765570in}{4.654545in}}%
\pgfpathlineto{\pgfqpoint{1.765965in}{4.654545in}}%
\pgfpathlineto{\pgfqpoint{1.838005in}{4.654545in}}%
\pgfpathlineto{\pgfqpoint{1.901245in}{4.654545in}}%
\pgfpathlineto{\pgfqpoint{1.926446in}{4.654545in}}%
\pgfpathlineto{\pgfqpoint{1.931223in}{4.654545in}}%
\pgfpathlineto{\pgfqpoint{1.939128in}{4.654545in}}%
\pgfpathlineto{\pgfqpoint{1.952583in}{4.654545in}}%
\pgfpathlineto{\pgfqpoint{1.961654in}{4.654545in}}%
\pgfpathlineto{\pgfqpoint{1.969006in}{4.654545in}}%
\pgfpathlineto{\pgfqpoint{1.985707in}{4.654545in}}%
\pgfpathlineto{\pgfqpoint{2.025269in}{4.654545in}}%
\pgfpathlineto{\pgfqpoint{2.036417in}{4.654545in}}%
\pgfpathlineto{\pgfqpoint{2.039048in}{4.654545in}}%
\pgfpathlineto{\pgfqpoint{2.041188in}{4.654545in}}%
\pgfpathlineto{\pgfqpoint{2.098575in}{4.654545in}}%
\pgfpathlineto{\pgfqpoint{2.150972in}{4.654545in}}%
\pgfpathlineto{\pgfqpoint{2.209427in}{4.654545in}}%
\pgfpathlineto{\pgfqpoint{2.640377in}{4.654545in}}%
\pgfpathlineto{\pgfqpoint{2.674239in}{4.654545in}}%
\pgfpathlineto{\pgfqpoint{2.693311in}{4.654545in}}%
\pgfpathlineto{\pgfqpoint{2.710732in}{4.654545in}}%
\pgfpathlineto{\pgfqpoint{2.720203in}{4.654545in}}%
\pgfpathlineto{\pgfqpoint{2.755449in}{4.654545in}}%
\pgfpathlineto{\pgfqpoint{2.761613in}{4.654545in}}%
\pgfpathlineto{\pgfqpoint{2.831252in}{4.654545in}}%
\pgfpathlineto{\pgfqpoint{2.845869in}{4.654545in}}%
\pgfpathlineto{\pgfqpoint{2.885251in}{4.654545in}}%
\pgfpathlineto{\pgfqpoint{2.885383in}{4.654545in}}%
\pgfpathlineto{\pgfqpoint{2.960181in}{4.654545in}}%
\pgfpathlineto{\pgfqpoint{3.001270in}{4.654545in}}%
\pgfpathlineto{\pgfqpoint{3.007176in}{4.654545in}}%
\pgfpathlineto{\pgfqpoint{3.016127in}{4.654545in}}%
\pgfpathlineto{\pgfqpoint{3.044689in}{4.654545in}}%
\pgfpathlineto{\pgfqpoint{3.066459in}{4.654545in}}%
\pgfpathlineto{\pgfqpoint{3.121624in}{4.654545in}}%
\pgfpathlineto{\pgfqpoint{3.135076in}{4.654545in}}%
\pgfpathlineto{\pgfqpoint{3.492654in}{4.654545in}}%
\pgfpathlineto{\pgfqpoint{3.557749in}{4.654545in}}%
\pgfpathlineto{\pgfqpoint{3.581077in}{4.654545in}}%
\pgfpathlineto{\pgfqpoint{3.600516in}{4.654545in}}%
\pgfpathlineto{\pgfqpoint{3.606495in}{4.654545in}}%
\pgfpathlineto{\pgfqpoint{3.649355in}{4.654545in}}%
\pgfpathlineto{\pgfqpoint{3.740558in}{4.654545in}}%
\pgfpathlineto{\pgfqpoint{3.748745in}{4.654545in}}%
\pgfpathlineto{\pgfqpoint{3.766323in}{4.654545in}}%
\pgfpathlineto{\pgfqpoint{3.885808in}{4.654545in}}%
\pgfpathlineto{\pgfqpoint{3.933797in}{4.654545in}}%
\pgfpathlineto{\pgfqpoint{4.070163in}{4.654545in}}%
\pgfpathlineto{\pgfqpoint{4.120452in}{4.654545in}}%
\pgfpathlineto{\pgfqpoint{4.232474in}{4.654545in}}%
\pgfpathlineto{\pgfqpoint{4.393612in}{4.654545in}}%
\pgfpathlineto{\pgfqpoint{4.407199in}{4.654545in}}%
\pgfpathlineto{\pgfqpoint{4.484258in}{4.654545in}}%
\pgfpathlineto{\pgfqpoint{4.485771in}{4.654545in}}%
\pgfpathlineto{\pgfqpoint{4.646808in}{4.654545in}}%
\pgfpathlineto{\pgfqpoint{4.733640in}{4.654545in}}%
\pgfpathlineto{\pgfqpoint{4.919563in}{4.654545in}}%
\pgfpathlineto{\pgfqpoint{5.798801in}{4.654545in}}%
\pgfpathlineto{\pgfqpoint{6.068379in}{4.654545in}}%
\pgfpathlineto{\pgfqpoint{6.068379in}{4.659267in}}%
\pgfpathlineto{\pgfqpoint{6.068379in}{4.659267in}}%
\pgfpathlineto{\pgfqpoint{5.798801in}{4.660113in}}%
\pgfpathlineto{\pgfqpoint{4.919563in}{4.671340in}}%
\pgfpathlineto{\pgfqpoint{4.733640in}{4.676454in}}%
\pgfpathlineto{\pgfqpoint{4.646808in}{4.678922in}}%
\pgfpathlineto{\pgfqpoint{4.485771in}{4.683416in}}%
\pgfpathlineto{\pgfqpoint{4.484258in}{4.683457in}}%
\pgfpathlineto{\pgfqpoint{4.407199in}{4.685516in}}%
\pgfpathlineto{\pgfqpoint{4.393612in}{4.685871in}}%
\pgfpathlineto{\pgfqpoint{4.232474in}{4.689882in}}%
\pgfpathlineto{\pgfqpoint{4.120452in}{4.692452in}}%
\pgfpathlineto{\pgfqpoint{4.070163in}{4.693552in}}%
\pgfpathlineto{\pgfqpoint{3.933797in}{4.696393in}}%
\pgfpathlineto{\pgfqpoint{3.885808in}{4.697352in}}%
\pgfpathlineto{\pgfqpoint{3.766323in}{4.699683in}}%
\pgfpathlineto{\pgfqpoint{3.748745in}{4.700021in}}%
\pgfpathlineto{\pgfqpoint{3.740558in}{4.700178in}}%
\pgfpathlineto{\pgfqpoint{3.649355in}{4.701925in}}%
\pgfpathlineto{\pgfqpoint{3.606495in}{4.702746in}}%
\pgfpathlineto{\pgfqpoint{3.600516in}{4.702860in}}%
\pgfpathlineto{\pgfqpoint{3.581077in}{4.703232in}}%
\pgfpathlineto{\pgfqpoint{3.557749in}{4.703680in}}%
\pgfpathlineto{\pgfqpoint{3.492654in}{4.704928in}}%
\pgfpathlineto{\pgfqpoint{3.135076in}{4.711352in}}%
\pgfpathlineto{\pgfqpoint{3.121624in}{4.711558in}}%
\pgfpathlineto{\pgfqpoint{3.066459in}{4.712360in}}%
\pgfpathlineto{\pgfqpoint{3.044689in}{4.712658in}}%
\pgfpathlineto{\pgfqpoint{3.016127in}{4.713032in}}%
\pgfpathlineto{\pgfqpoint{3.007176in}{4.713146in}}%
\pgfpathlineto{\pgfqpoint{3.001270in}{4.713219in}}%
\pgfpathlineto{\pgfqpoint{2.960181in}{4.713712in}}%
\pgfpathlineto{\pgfqpoint{2.885383in}{4.714526in}}%
\pgfpathlineto{\pgfqpoint{2.885251in}{4.714528in}}%
\pgfpathlineto{\pgfqpoint{2.845869in}{4.714923in}}%
\pgfpathlineto{\pgfqpoint{2.831252in}{4.715066in}}%
\pgfpathlineto{\pgfqpoint{2.761613in}{4.715735in}}%
\pgfpathlineto{\pgfqpoint{2.755449in}{4.715795in}}%
\pgfpathlineto{\pgfqpoint{2.720203in}{4.716140in}}%
\pgfpathlineto{\pgfqpoint{2.710732in}{4.716235in}}%
\pgfpathlineto{\pgfqpoint{2.693311in}{4.716413in}}%
\pgfpathlineto{\pgfqpoint{2.674239in}{4.716613in}}%
\pgfpathlineto{\pgfqpoint{2.640377in}{4.716987in}}%
\pgfpathlineto{\pgfqpoint{2.209427in}{4.725694in}}%
\pgfpathlineto{\pgfqpoint{2.150972in}{4.727446in}}%
\pgfpathlineto{\pgfqpoint{2.098575in}{4.729065in}}%
\pgfpathlineto{\pgfqpoint{2.041188in}{4.730848in}}%
\pgfpathlineto{\pgfqpoint{2.039048in}{4.730914in}}%
\pgfpathlineto{\pgfqpoint{2.036417in}{4.730995in}}%
\pgfpathlineto{\pgfqpoint{2.025269in}{4.731338in}}%
\pgfpathlineto{\pgfqpoint{1.985707in}{4.732537in}}%
\pgfpathlineto{\pgfqpoint{1.969006in}{4.733031in}}%
\pgfpathlineto{\pgfqpoint{1.961654in}{4.733246in}}%
\pgfpathlineto{\pgfqpoint{1.952583in}{4.733510in}}%
\pgfpathlineto{\pgfqpoint{1.939128in}{4.733894in}}%
\pgfpathlineto{\pgfqpoint{1.931223in}{4.734117in}}%
\pgfpathlineto{\pgfqpoint{1.926446in}{4.734251in}}%
\pgfpathlineto{\pgfqpoint{1.901245in}{4.734938in}}%
\pgfpathlineto{\pgfqpoint{1.838005in}{4.736519in}}%
\pgfpathlineto{\pgfqpoint{1.765965in}{4.738003in}}%
\pgfpathlineto{\pgfqpoint{1.765570in}{4.738010in}}%
\pgfpathlineto{\pgfqpoint{1.708150in}{4.738892in}}%
\pgfpathlineto{\pgfqpoint{1.705456in}{4.738926in}}%
\pgfpathlineto{\pgfqpoint{1.698362in}{4.739012in}}%
\pgfpathlineto{\pgfqpoint{1.669991in}{4.739308in}}%
\pgfpathlineto{\pgfqpoint{1.663156in}{4.739368in}}%
\pgfpathlineto{\pgfqpoint{1.642186in}{4.739519in}}%
\pgfpathlineto{\pgfqpoint{1.413547in}{4.737780in}}%
\pgfpathlineto{\pgfqpoint{1.401213in}{4.737492in}}%
\pgfpathlineto{\pgfqpoint{1.270722in}{4.733125in}}%
\pgfpathlineto{\pgfqpoint{1.268964in}{4.733050in}}%
\pgfpathlineto{\pgfqpoint{1.253596in}{4.732373in}}%
\pgfpathlineto{\pgfqpoint{1.219352in}{4.730749in}}%
\pgfpathlineto{\pgfqpoint{1.217148in}{4.730639in}}%
\pgfpathlineto{\pgfqpoint{1.201589in}{4.729844in}}%
\pgfpathlineto{\pgfqpoint{1.198560in}{4.729686in}}%
\pgfpathlineto{\pgfqpoint{1.187889in}{4.729117in}}%
\pgfpathlineto{\pgfqpoint{1.186252in}{4.729029in}}%
\pgfpathlineto{\pgfqpoint{1.182870in}{4.728845in}}%
\pgfpathlineto{\pgfqpoint{1.180938in}{4.728739in}}%
\pgfpathlineto{\pgfqpoint{1.171321in}{4.728205in}}%
\pgfpathlineto{\pgfqpoint{1.167729in}{4.728003in}}%
\pgfpathlineto{\pgfqpoint{1.165893in}{4.727899in}}%
\pgfpathlineto{\pgfqpoint{1.153053in}{4.727159in}}%
\pgfpathlineto{\pgfqpoint{1.104725in}{4.724196in}}%
\pgfpathlineto{\pgfqpoint{1.025718in}{4.718796in}}%
\pgfpathlineto{\pgfqpoint{1.024811in}{4.718731in}}%
\pgfpathlineto{\pgfqpoint{1.016253in}{4.718109in}}%
\pgfpathlineto{\pgfqpoint{1.015744in}{4.718071in}}%
\pgfpathlineto{\pgfqpoint{0.990991in}{4.716237in}}%
\pgfpathlineto{\pgfqpoint{0.990991in}{4.716237in}}%
\pgfpathclose%
\pgfusepath{stroke,fill}%
}%
\begin{pgfscope}%
\pgfsys@transformshift{0.000000in}{0.000000in}%
\pgfsys@useobject{currentmarker}{}%
\end{pgfscope}%
\end{pgfscope}%
\begin{pgfscope}%
\pgfpathrectangle{\pgfqpoint{0.840504in}{4.600000in}}{\pgfqpoint{5.911808in}{1.200000in}}%
\pgfusepath{clip}%
\pgfsetrectcap%
\pgfsetroundjoin%
\pgfsetlinewidth{0.803000pt}%
\definecolor{currentstroke}{rgb}{0.690196,0.690196,0.690196}%
\pgfsetstrokecolor{currentstroke}%
\pgfsetstrokeopacity{0.200000}%
\pgfsetdash{}{0pt}%
\pgfpathmoveto{\pgfqpoint{1.313449in}{4.600000in}}%
\pgfpathlineto{\pgfqpoint{1.313449in}{5.800000in}}%
\pgfusepath{stroke}%
\end{pgfscope}%
\begin{pgfscope}%
\pgfsetbuttcap%
\pgfsetroundjoin%
\definecolor{currentfill}{rgb}{0.000000,0.000000,0.000000}%
\pgfsetfillcolor{currentfill}%
\pgfsetlinewidth{0.803000pt}%
\definecolor{currentstroke}{rgb}{0.000000,0.000000,0.000000}%
\pgfsetstrokecolor{currentstroke}%
\pgfsetdash{}{0pt}%
\pgfsys@defobject{currentmarker}{\pgfqpoint{0.000000in}{-0.048611in}}{\pgfqpoint{0.000000in}{0.000000in}}{%
\pgfpathmoveto{\pgfqpoint{0.000000in}{0.000000in}}%
\pgfpathlineto{\pgfqpoint{0.000000in}{-0.048611in}}%
\pgfusepath{stroke,fill}%
}%
\begin{pgfscope}%
\pgfsys@transformshift{1.313449in}{4.600000in}%
\pgfsys@useobject{currentmarker}{}%
\end{pgfscope}%
\end{pgfscope}%
\begin{pgfscope}%
\pgfpathrectangle{\pgfqpoint{0.840504in}{4.600000in}}{\pgfqpoint{5.911808in}{1.200000in}}%
\pgfusepath{clip}%
\pgfsetrectcap%
\pgfsetroundjoin%
\pgfsetlinewidth{0.803000pt}%
\definecolor{currentstroke}{rgb}{0.690196,0.690196,0.690196}%
\pgfsetstrokecolor{currentstroke}%
\pgfsetstrokeopacity{0.200000}%
\pgfsetdash{}{0pt}%
\pgfpathmoveto{\pgfqpoint{2.259338in}{4.600000in}}%
\pgfpathlineto{\pgfqpoint{2.259338in}{5.800000in}}%
\pgfusepath{stroke}%
\end{pgfscope}%
\begin{pgfscope}%
\pgfsetbuttcap%
\pgfsetroundjoin%
\definecolor{currentfill}{rgb}{0.000000,0.000000,0.000000}%
\pgfsetfillcolor{currentfill}%
\pgfsetlinewidth{0.803000pt}%
\definecolor{currentstroke}{rgb}{0.000000,0.000000,0.000000}%
\pgfsetstrokecolor{currentstroke}%
\pgfsetdash{}{0pt}%
\pgfsys@defobject{currentmarker}{\pgfqpoint{0.000000in}{-0.048611in}}{\pgfqpoint{0.000000in}{0.000000in}}{%
\pgfpathmoveto{\pgfqpoint{0.000000in}{0.000000in}}%
\pgfpathlineto{\pgfqpoint{0.000000in}{-0.048611in}}%
\pgfusepath{stroke,fill}%
}%
\begin{pgfscope}%
\pgfsys@transformshift{2.259338in}{4.600000in}%
\pgfsys@useobject{currentmarker}{}%
\end{pgfscope}%
\end{pgfscope}%
\begin{pgfscope}%
\pgfpathrectangle{\pgfqpoint{0.840504in}{4.600000in}}{\pgfqpoint{5.911808in}{1.200000in}}%
\pgfusepath{clip}%
\pgfsetrectcap%
\pgfsetroundjoin%
\pgfsetlinewidth{0.803000pt}%
\definecolor{currentstroke}{rgb}{0.690196,0.690196,0.690196}%
\pgfsetstrokecolor{currentstroke}%
\pgfsetstrokeopacity{0.200000}%
\pgfsetdash{}{0pt}%
\pgfpathmoveto{\pgfqpoint{3.205228in}{4.600000in}}%
\pgfpathlineto{\pgfqpoint{3.205228in}{5.800000in}}%
\pgfusepath{stroke}%
\end{pgfscope}%
\begin{pgfscope}%
\pgfsetbuttcap%
\pgfsetroundjoin%
\definecolor{currentfill}{rgb}{0.000000,0.000000,0.000000}%
\pgfsetfillcolor{currentfill}%
\pgfsetlinewidth{0.803000pt}%
\definecolor{currentstroke}{rgb}{0.000000,0.000000,0.000000}%
\pgfsetstrokecolor{currentstroke}%
\pgfsetdash{}{0pt}%
\pgfsys@defobject{currentmarker}{\pgfqpoint{0.000000in}{-0.048611in}}{\pgfqpoint{0.000000in}{0.000000in}}{%
\pgfpathmoveto{\pgfqpoint{0.000000in}{0.000000in}}%
\pgfpathlineto{\pgfqpoint{0.000000in}{-0.048611in}}%
\pgfusepath{stroke,fill}%
}%
\begin{pgfscope}%
\pgfsys@transformshift{3.205228in}{4.600000in}%
\pgfsys@useobject{currentmarker}{}%
\end{pgfscope}%
\end{pgfscope}%
\begin{pgfscope}%
\pgfpathrectangle{\pgfqpoint{0.840504in}{4.600000in}}{\pgfqpoint{5.911808in}{1.200000in}}%
\pgfusepath{clip}%
\pgfsetrectcap%
\pgfsetroundjoin%
\pgfsetlinewidth{0.803000pt}%
\definecolor{currentstroke}{rgb}{0.690196,0.690196,0.690196}%
\pgfsetstrokecolor{currentstroke}%
\pgfsetstrokeopacity{0.200000}%
\pgfsetdash{}{0pt}%
\pgfpathmoveto{\pgfqpoint{4.151117in}{4.600000in}}%
\pgfpathlineto{\pgfqpoint{4.151117in}{5.800000in}}%
\pgfusepath{stroke}%
\end{pgfscope}%
\begin{pgfscope}%
\pgfsetbuttcap%
\pgfsetroundjoin%
\definecolor{currentfill}{rgb}{0.000000,0.000000,0.000000}%
\pgfsetfillcolor{currentfill}%
\pgfsetlinewidth{0.803000pt}%
\definecolor{currentstroke}{rgb}{0.000000,0.000000,0.000000}%
\pgfsetstrokecolor{currentstroke}%
\pgfsetdash{}{0pt}%
\pgfsys@defobject{currentmarker}{\pgfqpoint{0.000000in}{-0.048611in}}{\pgfqpoint{0.000000in}{0.000000in}}{%
\pgfpathmoveto{\pgfqpoint{0.000000in}{0.000000in}}%
\pgfpathlineto{\pgfqpoint{0.000000in}{-0.048611in}}%
\pgfusepath{stroke,fill}%
}%
\begin{pgfscope}%
\pgfsys@transformshift{4.151117in}{4.600000in}%
\pgfsys@useobject{currentmarker}{}%
\end{pgfscope}%
\end{pgfscope}%
\begin{pgfscope}%
\pgfpathrectangle{\pgfqpoint{0.840504in}{4.600000in}}{\pgfqpoint{5.911808in}{1.200000in}}%
\pgfusepath{clip}%
\pgfsetrectcap%
\pgfsetroundjoin%
\pgfsetlinewidth{0.803000pt}%
\definecolor{currentstroke}{rgb}{0.690196,0.690196,0.690196}%
\pgfsetstrokecolor{currentstroke}%
\pgfsetstrokeopacity{0.200000}%
\pgfsetdash{}{0pt}%
\pgfpathmoveto{\pgfqpoint{5.097007in}{4.600000in}}%
\pgfpathlineto{\pgfqpoint{5.097007in}{5.800000in}}%
\pgfusepath{stroke}%
\end{pgfscope}%
\begin{pgfscope}%
\pgfsetbuttcap%
\pgfsetroundjoin%
\definecolor{currentfill}{rgb}{0.000000,0.000000,0.000000}%
\pgfsetfillcolor{currentfill}%
\pgfsetlinewidth{0.803000pt}%
\definecolor{currentstroke}{rgb}{0.000000,0.000000,0.000000}%
\pgfsetstrokecolor{currentstroke}%
\pgfsetdash{}{0pt}%
\pgfsys@defobject{currentmarker}{\pgfqpoint{0.000000in}{-0.048611in}}{\pgfqpoint{0.000000in}{0.000000in}}{%
\pgfpathmoveto{\pgfqpoint{0.000000in}{0.000000in}}%
\pgfpathlineto{\pgfqpoint{0.000000in}{-0.048611in}}%
\pgfusepath{stroke,fill}%
}%
\begin{pgfscope}%
\pgfsys@transformshift{5.097007in}{4.600000in}%
\pgfsys@useobject{currentmarker}{}%
\end{pgfscope}%
\end{pgfscope}%
\begin{pgfscope}%
\pgfpathrectangle{\pgfqpoint{0.840504in}{4.600000in}}{\pgfqpoint{5.911808in}{1.200000in}}%
\pgfusepath{clip}%
\pgfsetrectcap%
\pgfsetroundjoin%
\pgfsetlinewidth{0.803000pt}%
\definecolor{currentstroke}{rgb}{0.690196,0.690196,0.690196}%
\pgfsetstrokecolor{currentstroke}%
\pgfsetstrokeopacity{0.200000}%
\pgfsetdash{}{0pt}%
\pgfpathmoveto{\pgfqpoint{6.042896in}{4.600000in}}%
\pgfpathlineto{\pgfqpoint{6.042896in}{5.800000in}}%
\pgfusepath{stroke}%
\end{pgfscope}%
\begin{pgfscope}%
\pgfsetbuttcap%
\pgfsetroundjoin%
\definecolor{currentfill}{rgb}{0.000000,0.000000,0.000000}%
\pgfsetfillcolor{currentfill}%
\pgfsetlinewidth{0.803000pt}%
\definecolor{currentstroke}{rgb}{0.000000,0.000000,0.000000}%
\pgfsetstrokecolor{currentstroke}%
\pgfsetdash{}{0pt}%
\pgfsys@defobject{currentmarker}{\pgfqpoint{0.000000in}{-0.048611in}}{\pgfqpoint{0.000000in}{0.000000in}}{%
\pgfpathmoveto{\pgfqpoint{0.000000in}{0.000000in}}%
\pgfpathlineto{\pgfqpoint{0.000000in}{-0.048611in}}%
\pgfusepath{stroke,fill}%
}%
\begin{pgfscope}%
\pgfsys@transformshift{6.042896in}{4.600000in}%
\pgfsys@useobject{currentmarker}{}%
\end{pgfscope}%
\end{pgfscope}%
\begin{pgfscope}%
\pgfpathrectangle{\pgfqpoint{0.840504in}{4.600000in}}{\pgfqpoint{5.911808in}{1.200000in}}%
\pgfusepath{clip}%
\pgfsetrectcap%
\pgfsetroundjoin%
\pgfsetlinewidth{0.803000pt}%
\definecolor{currentstroke}{rgb}{0.690196,0.690196,0.690196}%
\pgfsetstrokecolor{currentstroke}%
\pgfsetstrokeopacity{0.200000}%
\pgfsetdash{}{0pt}%
\pgfpathmoveto{\pgfqpoint{0.840504in}{4.654545in}}%
\pgfpathlineto{\pgfqpoint{6.752313in}{4.654545in}}%
\pgfusepath{stroke}%
\end{pgfscope}%
\begin{pgfscope}%
\pgfsetbuttcap%
\pgfsetroundjoin%
\definecolor{currentfill}{rgb}{0.000000,0.000000,0.000000}%
\pgfsetfillcolor{currentfill}%
\pgfsetlinewidth{0.803000pt}%
\definecolor{currentstroke}{rgb}{0.000000,0.000000,0.000000}%
\pgfsetstrokecolor{currentstroke}%
\pgfsetdash{}{0pt}%
\pgfsys@defobject{currentmarker}{\pgfqpoint{-0.048611in}{0.000000in}}{\pgfqpoint{-0.000000in}{0.000000in}}{%
\pgfpathmoveto{\pgfqpoint{-0.000000in}{0.000000in}}%
\pgfpathlineto{\pgfqpoint{-0.048611in}{0.000000in}}%
\pgfusepath{stroke,fill}%
}%
\begin{pgfscope}%
\pgfsys@transformshift{0.840504in}{4.654545in}%
\pgfsys@useobject{currentmarker}{}%
\end{pgfscope}%
\end{pgfscope}%
\begin{pgfscope}%
\pgfpathrectangle{\pgfqpoint{0.840504in}{4.600000in}}{\pgfqpoint{5.911808in}{1.200000in}}%
\pgfusepath{clip}%
\pgfsetrectcap%
\pgfsetroundjoin%
\pgfsetlinewidth{0.803000pt}%
\definecolor{currentstroke}{rgb}{0.690196,0.690196,0.690196}%
\pgfsetstrokecolor{currentstroke}%
\pgfsetstrokeopacity{0.200000}%
\pgfsetdash{}{0pt}%
\pgfpathmoveto{\pgfqpoint{0.840504in}{5.208707in}}%
\pgfpathlineto{\pgfqpoint{6.752313in}{5.208707in}}%
\pgfusepath{stroke}%
\end{pgfscope}%
\begin{pgfscope}%
\pgfsetbuttcap%
\pgfsetroundjoin%
\definecolor{currentfill}{rgb}{0.000000,0.000000,0.000000}%
\pgfsetfillcolor{currentfill}%
\pgfsetlinewidth{0.803000pt}%
\definecolor{currentstroke}{rgb}{0.000000,0.000000,0.000000}%
\pgfsetstrokecolor{currentstroke}%
\pgfsetdash{}{0pt}%
\pgfsys@defobject{currentmarker}{\pgfqpoint{-0.048611in}{0.000000in}}{\pgfqpoint{-0.000000in}{0.000000in}}{%
\pgfpathmoveto{\pgfqpoint{-0.000000in}{0.000000in}}%
\pgfpathlineto{\pgfqpoint{-0.048611in}{0.000000in}}%
\pgfusepath{stroke,fill}%
}%
\begin{pgfscope}%
\pgfsys@transformshift{0.840504in}{5.208707in}%
\pgfsys@useobject{currentmarker}{}%
\end{pgfscope}%
\end{pgfscope}%
\begin{pgfscope}%
\pgfpathrectangle{\pgfqpoint{0.840504in}{4.600000in}}{\pgfqpoint{5.911808in}{1.200000in}}%
\pgfusepath{clip}%
\pgfsetrectcap%
\pgfsetroundjoin%
\pgfsetlinewidth{0.803000pt}%
\definecolor{currentstroke}{rgb}{0.690196,0.690196,0.690196}%
\pgfsetstrokecolor{currentstroke}%
\pgfsetstrokeopacity{0.200000}%
\pgfsetdash{}{0pt}%
\pgfpathmoveto{\pgfqpoint{0.840504in}{5.762868in}}%
\pgfpathlineto{\pgfqpoint{6.752313in}{5.762868in}}%
\pgfusepath{stroke}%
\end{pgfscope}%
\begin{pgfscope}%
\pgfsetbuttcap%
\pgfsetroundjoin%
\definecolor{currentfill}{rgb}{0.000000,0.000000,0.000000}%
\pgfsetfillcolor{currentfill}%
\pgfsetlinewidth{0.803000pt}%
\definecolor{currentstroke}{rgb}{0.000000,0.000000,0.000000}%
\pgfsetstrokecolor{currentstroke}%
\pgfsetdash{}{0pt}%
\pgfsys@defobject{currentmarker}{\pgfqpoint{-0.048611in}{0.000000in}}{\pgfqpoint{-0.000000in}{0.000000in}}{%
\pgfpathmoveto{\pgfqpoint{-0.000000in}{0.000000in}}%
\pgfpathlineto{\pgfqpoint{-0.048611in}{0.000000in}}%
\pgfusepath{stroke,fill}%
}%
\begin{pgfscope}%
\pgfsys@transformshift{0.840504in}{5.762868in}%
\pgfsys@useobject{currentmarker}{}%
\end{pgfscope}%
\end{pgfscope}%
\begin{pgfscope}%
\pgfpathrectangle{\pgfqpoint{0.840504in}{4.600000in}}{\pgfqpoint{5.911808in}{1.200000in}}%
\pgfusepath{clip}%
\pgfsetrectcap%
\pgfsetroundjoin%
\pgfsetlinewidth{1.505625pt}%
\definecolor{currentstroke}{rgb}{0.121569,0.466667,0.705882}%
\pgfsetstrokecolor{currentstroke}%
\pgfsetdash{}{0pt}%
\pgfpathmoveto{\pgfqpoint{0.990438in}{5.580241in}}%
\pgfpathlineto{\pgfqpoint{0.994362in}{5.605459in}}%
\pgfpathlineto{\pgfqpoint{0.996518in}{5.618599in}}%
\pgfpathlineto{\pgfqpoint{0.998005in}{5.627346in}}%
\pgfpathlineto{\pgfqpoint{0.998799in}{5.631910in}}%
\pgfpathlineto{\pgfqpoint{0.999689in}{5.636930in}}%
\pgfpathlineto{\pgfqpoint{1.000678in}{5.642399in}}%
\pgfpathlineto{\pgfqpoint{1.002046in}{5.649755in}}%
\pgfpathlineto{\pgfqpoint{1.002530in}{5.652301in}}%
\pgfpathlineto{\pgfqpoint{1.003703in}{5.658349in}}%
\pgfpathlineto{\pgfqpoint{1.005869in}{5.669026in}}%
\pgfpathlineto{\pgfqpoint{1.006609in}{5.672530in}}%
\pgfpathlineto{\pgfqpoint{1.007368in}{5.676047in}}%
\pgfpathlineto{\pgfqpoint{1.008961in}{5.683164in}}%
\pgfpathlineto{\pgfqpoint{1.010287in}{5.688814in}}%
\pgfpathlineto{\pgfqpoint{1.010375in}{5.689182in}}%
\pgfpathlineto{\pgfqpoint{1.011245in}{5.692740in}}%
\pgfpathlineto{\pgfqpoint{1.014746in}{5.705943in}}%
\pgfpathlineto{\pgfqpoint{1.018469in}{5.717969in}}%
\pgfpathlineto{\pgfqpoint{1.023523in}{5.730868in}}%
\pgfpathlineto{\pgfqpoint{1.028150in}{5.739150in}}%
\pgfpathlineto{\pgfqpoint{1.028339in}{5.739415in}}%
\pgfpathlineto{\pgfqpoint{1.028612in}{5.739790in}}%
\pgfpathlineto{\pgfqpoint{1.029345in}{5.740734in}}%
\pgfpathlineto{\pgfqpoint{1.031093in}{5.742642in}}%
\pgfpathlineto{\pgfqpoint{1.032281in}{5.743659in}}%
\pgfpathlineto{\pgfqpoint{1.037557in}{5.745455in}}%
\pgfpathlineto{\pgfqpoint{1.038248in}{5.745362in}}%
\pgfpathlineto{\pgfqpoint{1.038269in}{5.745358in}}%
\pgfpathlineto{\pgfqpoint{1.039403in}{5.745040in}}%
\pgfpathlineto{\pgfqpoint{1.042083in}{5.743488in}}%
\pgfpathlineto{\pgfqpoint{1.043351in}{5.742364in}}%
\pgfpathlineto{\pgfqpoint{1.043547in}{5.742168in}}%
\pgfpathlineto{\pgfqpoint{1.054106in}{5.723113in}}%
\pgfpathlineto{\pgfqpoint{1.054932in}{5.720941in}}%
\pgfpathlineto{\pgfqpoint{1.055402in}{5.719664in}}%
\pgfpathlineto{\pgfqpoint{1.057074in}{5.714874in}}%
\pgfpathlineto{\pgfqpoint{1.060383in}{5.704292in}}%
\pgfpathlineto{\pgfqpoint{1.061588in}{5.700083in}}%
\pgfpathlineto{\pgfqpoint{1.062702in}{5.696032in}}%
\pgfpathlineto{\pgfqpoint{1.065546in}{5.684995in}}%
\pgfpathlineto{\pgfqpoint{1.070103in}{5.665346in}}%
\pgfpathlineto{\pgfqpoint{1.075327in}{5.640093in}}%
\pgfpathlineto{\pgfqpoint{1.079103in}{5.620210in}}%
\pgfpathlineto{\pgfqpoint{1.083825in}{5.593645in}}%
\pgfpathlineto{\pgfqpoint{1.106237in}{5.449929in}}%
\pgfpathlineto{\pgfqpoint{1.112179in}{5.409264in}}%
\pgfpathlineto{\pgfqpoint{1.120671in}{5.351010in}}%
\pgfpathlineto{\pgfqpoint{1.133022in}{5.268199in}}%
\pgfpathlineto{\pgfqpoint{1.139707in}{5.225244in}}%
\pgfpathlineto{\pgfqpoint{1.155415in}{5.131779in}}%
\pgfpathlineto{\pgfqpoint{1.175667in}{5.029865in}}%
\pgfpathlineto{\pgfqpoint{1.191614in}{4.965040in}}%
\pgfpathlineto{\pgfqpoint{1.216706in}{4.887691in}}%
\pgfpathlineto{\pgfqpoint{1.250615in}{4.819297in}}%
\pgfpathlineto{\pgfqpoint{1.284782in}{4.776373in}}%
\pgfpathlineto{\pgfqpoint{1.334942in}{4.737357in}}%
\pgfpathlineto{\pgfqpoint{1.418516in}{4.701013in}}%
\pgfpathlineto{\pgfqpoint{1.563285in}{4.682113in}}%
\pgfpathlineto{\pgfqpoint{2.030903in}{4.678717in}}%
\pgfusepath{stroke}%
\end{pgfscope}%
\begin{pgfscope}%
\pgfpathrectangle{\pgfqpoint{0.840504in}{4.600000in}}{\pgfqpoint{5.911808in}{1.200000in}}%
\pgfusepath{clip}%
\pgfsetrectcap%
\pgfsetroundjoin%
\pgfsetlinewidth{1.505625pt}%
\definecolor{currentstroke}{rgb}{1.000000,0.498039,0.054902}%
\pgfsetstrokecolor{currentstroke}%
\pgfsetdash{}{0pt}%
\pgfpathmoveto{\pgfqpoint{0.963227in}{4.705307in}}%
\pgfpathlineto{\pgfqpoint{1.082132in}{4.715785in}}%
\pgfpathlineto{\pgfqpoint{1.264828in}{4.729344in}}%
\pgfpathlineto{\pgfqpoint{1.334574in}{4.733126in}}%
\pgfpathlineto{\pgfqpoint{1.391323in}{4.735470in}}%
\pgfpathlineto{\pgfqpoint{1.403777in}{4.735890in}}%
\pgfpathlineto{\pgfqpoint{1.480012in}{4.737687in}}%
\pgfpathlineto{\pgfqpoint{1.505045in}{4.737982in}}%
\pgfpathlineto{\pgfqpoint{1.592629in}{4.737881in}}%
\pgfpathlineto{\pgfqpoint{1.620609in}{4.737493in}}%
\pgfpathlineto{\pgfqpoint{1.697321in}{4.735642in}}%
\pgfpathlineto{\pgfqpoint{1.767290in}{4.733109in}}%
\pgfpathlineto{\pgfqpoint{1.984762in}{4.722928in}}%
\pgfpathlineto{\pgfqpoint{2.086304in}{4.718806in}}%
\pgfpathlineto{\pgfqpoint{2.109970in}{4.718048in}}%
\pgfpathlineto{\pgfqpoint{2.223618in}{4.715908in}}%
\pgfpathlineto{\pgfqpoint{2.383143in}{4.717887in}}%
\pgfpathlineto{\pgfqpoint{2.827835in}{4.742043in}}%
\pgfpathlineto{\pgfqpoint{2.882476in}{4.744289in}}%
\pgfpathlineto{\pgfqpoint{2.934019in}{4.745883in}}%
\pgfpathlineto{\pgfqpoint{2.979762in}{4.746844in}}%
\pgfpathlineto{\pgfqpoint{3.031188in}{4.747410in}}%
\pgfpathlineto{\pgfqpoint{3.079713in}{4.747461in}}%
\pgfpathlineto{\pgfqpoint{3.132051in}{4.747023in}}%
\pgfpathlineto{\pgfqpoint{3.208867in}{4.745546in}}%
\pgfpathlineto{\pgfqpoint{3.321491in}{4.741862in}}%
\pgfpathlineto{\pgfqpoint{3.380661in}{4.739306in}}%
\pgfpathlineto{\pgfqpoint{3.496699in}{4.733166in}}%
\pgfpathlineto{\pgfqpoint{3.559556in}{4.729234in}}%
\pgfpathlineto{\pgfqpoint{3.633265in}{4.724097in}}%
\pgfpathlineto{\pgfqpoint{3.664825in}{4.721734in}}%
\pgfpathlineto{\pgfqpoint{3.814587in}{4.709456in}}%
\pgfpathlineto{\pgfqpoint{3.932646in}{4.699182in}}%
\pgfpathlineto{\pgfqpoint{4.435247in}{4.667618in}}%
\pgfpathlineto{\pgfqpoint{4.503505in}{4.665639in}}%
\pgfpathlineto{\pgfqpoint{4.757331in}{4.661292in}}%
\pgfpathlineto{\pgfqpoint{5.212460in}{4.657878in}}%
\pgfpathlineto{\pgfqpoint{6.715212in}{4.657830in}}%
\pgfpathlineto{\pgfqpoint{6.715212in}{4.657830in}}%
\pgfusepath{stroke}%
\end{pgfscope}%
\begin{pgfscope}%
\pgfpathrectangle{\pgfqpoint{0.840504in}{4.600000in}}{\pgfqpoint{5.911808in}{1.200000in}}%
\pgfusepath{clip}%
\pgfsetrectcap%
\pgfsetroundjoin%
\pgfsetlinewidth{1.505625pt}%
\definecolor{currentstroke}{rgb}{0.580392,0.403922,0.741176}%
\pgfsetstrokecolor{currentstroke}%
\pgfsetdash{}{0pt}%
\pgfpathmoveto{\pgfqpoint{0.990991in}{4.716237in}}%
\pgfpathlineto{\pgfqpoint{1.015744in}{4.718071in}}%
\pgfpathlineto{\pgfqpoint{1.016253in}{4.718109in}}%
\pgfpathlineto{\pgfqpoint{1.024811in}{4.718731in}}%
\pgfpathlineto{\pgfqpoint{1.025718in}{4.718796in}}%
\pgfpathlineto{\pgfqpoint{1.104725in}{4.724196in}}%
\pgfpathlineto{\pgfqpoint{1.153053in}{4.727159in}}%
\pgfpathlineto{\pgfqpoint{1.165893in}{4.727899in}}%
\pgfpathlineto{\pgfqpoint{1.167729in}{4.728003in}}%
\pgfpathlineto{\pgfqpoint{1.171321in}{4.728205in}}%
\pgfpathlineto{\pgfqpoint{1.180938in}{4.728739in}}%
\pgfpathlineto{\pgfqpoint{1.182870in}{4.728845in}}%
\pgfpathlineto{\pgfqpoint{1.186252in}{4.729029in}}%
\pgfpathlineto{\pgfqpoint{1.187889in}{4.729117in}}%
\pgfpathlineto{\pgfqpoint{1.198560in}{4.729686in}}%
\pgfpathlineto{\pgfqpoint{1.201589in}{4.729844in}}%
\pgfpathlineto{\pgfqpoint{1.217148in}{4.730639in}}%
\pgfpathlineto{\pgfqpoint{1.219352in}{4.730749in}}%
\pgfpathlineto{\pgfqpoint{1.253596in}{4.732373in}}%
\pgfpathlineto{\pgfqpoint{1.268964in}{4.733050in}}%
\pgfpathlineto{\pgfqpoint{1.270722in}{4.733125in}}%
\pgfpathlineto{\pgfqpoint{1.401213in}{4.737492in}}%
\pgfpathlineto{\pgfqpoint{1.413547in}{4.737780in}}%
\pgfpathlineto{\pgfqpoint{1.642186in}{4.739519in}}%
\pgfpathlineto{\pgfqpoint{1.663156in}{4.739368in}}%
\pgfpathlineto{\pgfqpoint{1.669991in}{4.739308in}}%
\pgfpathlineto{\pgfqpoint{1.698362in}{4.739012in}}%
\pgfpathlineto{\pgfqpoint{1.705456in}{4.738926in}}%
\pgfpathlineto{\pgfqpoint{1.708150in}{4.738892in}}%
\pgfpathlineto{\pgfqpoint{1.765570in}{4.738010in}}%
\pgfpathlineto{\pgfqpoint{1.765965in}{4.738003in}}%
\pgfpathlineto{\pgfqpoint{1.838005in}{4.736519in}}%
\pgfpathlineto{\pgfqpoint{1.901245in}{4.734938in}}%
\pgfpathlineto{\pgfqpoint{1.926446in}{4.734251in}}%
\pgfpathlineto{\pgfqpoint{1.931223in}{4.734117in}}%
\pgfpathlineto{\pgfqpoint{1.939128in}{4.733894in}}%
\pgfpathlineto{\pgfqpoint{1.952583in}{4.733510in}}%
\pgfpathlineto{\pgfqpoint{1.961654in}{4.733246in}}%
\pgfpathlineto{\pgfqpoint{1.969006in}{4.733031in}}%
\pgfpathlineto{\pgfqpoint{1.985707in}{4.732537in}}%
\pgfpathlineto{\pgfqpoint{2.025269in}{4.731338in}}%
\pgfpathlineto{\pgfqpoint{2.036417in}{4.730995in}}%
\pgfpathlineto{\pgfqpoint{2.039048in}{4.730914in}}%
\pgfpathlineto{\pgfqpoint{2.041188in}{4.730848in}}%
\pgfpathlineto{\pgfqpoint{2.098575in}{4.729065in}}%
\pgfpathlineto{\pgfqpoint{2.150972in}{4.727446in}}%
\pgfpathlineto{\pgfqpoint{2.209427in}{4.725694in}}%
\pgfpathlineto{\pgfqpoint{2.640377in}{4.716987in}}%
\pgfpathlineto{\pgfqpoint{2.674239in}{4.716613in}}%
\pgfpathlineto{\pgfqpoint{2.693311in}{4.716413in}}%
\pgfpathlineto{\pgfqpoint{2.710732in}{4.716235in}}%
\pgfpathlineto{\pgfqpoint{2.720203in}{4.716140in}}%
\pgfpathlineto{\pgfqpoint{2.755449in}{4.715795in}}%
\pgfpathlineto{\pgfqpoint{2.761613in}{4.715735in}}%
\pgfpathlineto{\pgfqpoint{2.831252in}{4.715066in}}%
\pgfpathlineto{\pgfqpoint{2.845869in}{4.714923in}}%
\pgfpathlineto{\pgfqpoint{2.885251in}{4.714528in}}%
\pgfpathlineto{\pgfqpoint{2.885383in}{4.714526in}}%
\pgfpathlineto{\pgfqpoint{2.960181in}{4.713712in}}%
\pgfpathlineto{\pgfqpoint{3.001270in}{4.713219in}}%
\pgfpathlineto{\pgfqpoint{3.007176in}{4.713146in}}%
\pgfpathlineto{\pgfqpoint{3.016127in}{4.713032in}}%
\pgfpathlineto{\pgfqpoint{3.044689in}{4.712658in}}%
\pgfpathlineto{\pgfqpoint{3.066459in}{4.712360in}}%
\pgfpathlineto{\pgfqpoint{3.121624in}{4.711558in}}%
\pgfpathlineto{\pgfqpoint{3.135076in}{4.711352in}}%
\pgfpathlineto{\pgfqpoint{3.492654in}{4.704928in}}%
\pgfpathlineto{\pgfqpoint{3.557749in}{4.703680in}}%
\pgfpathlineto{\pgfqpoint{3.581077in}{4.703232in}}%
\pgfpathlineto{\pgfqpoint{3.600516in}{4.702860in}}%
\pgfpathlineto{\pgfqpoint{3.606495in}{4.702746in}}%
\pgfpathlineto{\pgfqpoint{3.649355in}{4.701925in}}%
\pgfpathlineto{\pgfqpoint{3.740558in}{4.700178in}}%
\pgfpathlineto{\pgfqpoint{3.748745in}{4.700021in}}%
\pgfpathlineto{\pgfqpoint{3.766323in}{4.699683in}}%
\pgfpathlineto{\pgfqpoint{3.885808in}{4.697352in}}%
\pgfpathlineto{\pgfqpoint{3.933797in}{4.696393in}}%
\pgfpathlineto{\pgfqpoint{4.070163in}{4.693552in}}%
\pgfpathlineto{\pgfqpoint{4.120452in}{4.692452in}}%
\pgfpathlineto{\pgfqpoint{4.232474in}{4.689882in}}%
\pgfpathlineto{\pgfqpoint{4.393612in}{4.685871in}}%
\pgfpathlineto{\pgfqpoint{4.407199in}{4.685516in}}%
\pgfpathlineto{\pgfqpoint{4.484258in}{4.683457in}}%
\pgfpathlineto{\pgfqpoint{4.485771in}{4.683416in}}%
\pgfpathlineto{\pgfqpoint{4.646808in}{4.678922in}}%
\pgfpathlineto{\pgfqpoint{4.733640in}{4.676454in}}%
\pgfpathlineto{\pgfqpoint{4.919563in}{4.671340in}}%
\pgfpathlineto{\pgfqpoint{5.798801in}{4.660113in}}%
\pgfpathlineto{\pgfqpoint{6.068379in}{4.659267in}}%
\pgfusepath{stroke}%
\end{pgfscope}%
\begin{pgfscope}%
\pgfsetrectcap%
\pgfsetmiterjoin%
\pgfsetlinewidth{0.803000pt}%
\definecolor{currentstroke}{rgb}{0.000000,0.000000,0.000000}%
\pgfsetstrokecolor{currentstroke}%
\pgfsetdash{}{0pt}%
\pgfpathmoveto{\pgfqpoint{0.840504in}{4.600000in}}%
\pgfpathlineto{\pgfqpoint{0.840504in}{5.800000in}}%
\pgfusepath{stroke}%
\end{pgfscope}%
\begin{pgfscope}%
\pgfsetrectcap%
\pgfsetmiterjoin%
\pgfsetlinewidth{0.803000pt}%
\definecolor{currentstroke}{rgb}{0.000000,0.000000,0.000000}%
\pgfsetstrokecolor{currentstroke}%
\pgfsetdash{}{0pt}%
\pgfpathmoveto{\pgfqpoint{6.752313in}{4.600000in}}%
\pgfpathlineto{\pgfqpoint{6.752313in}{5.800000in}}%
\pgfusepath{stroke}%
\end{pgfscope}%
\begin{pgfscope}%
\pgfsetrectcap%
\pgfsetmiterjoin%
\pgfsetlinewidth{0.803000pt}%
\definecolor{currentstroke}{rgb}{0.000000,0.000000,0.000000}%
\pgfsetstrokecolor{currentstroke}%
\pgfsetdash{}{0pt}%
\pgfpathmoveto{\pgfqpoint{0.840504in}{4.600000in}}%
\pgfpathlineto{\pgfqpoint{6.752313in}{4.600000in}}%
\pgfusepath{stroke}%
\end{pgfscope}%
\begin{pgfscope}%
\pgfsetrectcap%
\pgfsetmiterjoin%
\pgfsetlinewidth{0.803000pt}%
\definecolor{currentstroke}{rgb}{0.000000,0.000000,0.000000}%
\pgfsetstrokecolor{currentstroke}%
\pgfsetdash{}{0pt}%
\pgfpathmoveto{\pgfqpoint{0.840504in}{5.800000in}}%
\pgfpathlineto{\pgfqpoint{6.752313in}{5.800000in}}%
\pgfusepath{stroke}%
\end{pgfscope}%
\begin{pgfscope}%
\pgfsetbuttcap%
\pgfsetmiterjoin%
\definecolor{currentfill}{rgb}{1.000000,1.000000,1.000000}%
\pgfsetfillcolor{currentfill}%
\pgfsetlinewidth{0.000000pt}%
\definecolor{currentstroke}{rgb}{0.000000,0.000000,0.000000}%
\pgfsetstrokecolor{currentstroke}%
\pgfsetstrokeopacity{0.000000}%
\pgfsetdash{}{0pt}%
\pgfpathmoveto{\pgfqpoint{6.752313in}{0.670138in}}%
\pgfpathlineto{\pgfqpoint{7.952313in}{0.670138in}}%
\pgfpathlineto{\pgfqpoint{7.952313in}{4.600000in}}%
\pgfpathlineto{\pgfqpoint{6.752313in}{4.600000in}}%
\pgfpathlineto{\pgfqpoint{6.752313in}{0.670138in}}%
\pgfpathclose%
\pgfusepath{fill}%
\end{pgfscope}%
\begin{pgfscope}%
\pgfpathrectangle{\pgfqpoint{6.752313in}{0.670138in}}{\pgfqpoint{1.200000in}{3.929862in}}%
\pgfusepath{clip}%
\pgfsetbuttcap%
\pgfsetroundjoin%
\definecolor{currentfill}{rgb}{0.121569,0.466667,0.705882}%
\pgfsetfillcolor{currentfill}%
\pgfsetfillopacity{0.300000}%
\pgfsetlinewidth{1.003750pt}%
\definecolor{currentstroke}{rgb}{0.121569,0.466667,0.705882}%
\pgfsetstrokecolor{currentstroke}%
\pgfsetstrokeopacity{0.300000}%
\pgfsetdash{}{0pt}%
\pgfsys@defobject{currentmarker}{\pgfqpoint{6.821626in}{1.058457in}}{\pgfqpoint{7.897767in}{3.198321in}}{%
\pgfpathmoveto{\pgfqpoint{7.727259in}{1.058457in}}%
\pgfpathlineto{\pgfqpoint{7.727259in}{1.071083in}}%
\pgfpathlineto{\pgfqpoint{7.771291in}{1.083755in}}%
\pgfpathlineto{\pgfqpoint{7.790344in}{1.089795in}}%
\pgfpathlineto{\pgfqpoint{7.802657in}{1.093947in}}%
\pgfpathlineto{\pgfqpoint{7.811865in}{1.097209in}}%
\pgfpathlineto{\pgfqpoint{7.818258in}{1.099567in}}%
\pgfpathlineto{\pgfqpoint{7.825069in}{1.102175in}}%
\pgfpathlineto{\pgfqpoint{7.828383in}{1.103485in}}%
\pgfpathlineto{\pgfqpoint{7.834046in}{1.105791in}}%
\pgfpathlineto{\pgfqpoint{7.837290in}{1.107156in}}%
\pgfpathlineto{\pgfqpoint{7.838373in}{1.107619in}}%
\pgfpathlineto{\pgfqpoint{7.841608in}{1.109025in}}%
\pgfpathlineto{\pgfqpoint{7.842282in}{1.109323in}}%
\pgfpathlineto{\pgfqpoint{7.844329in}{1.110239in}}%
\pgfpathlineto{\pgfqpoint{7.845019in}{1.110551in}}%
\pgfpathlineto{\pgfqpoint{7.850168in}{1.112945in}}%
\pgfpathlineto{\pgfqpoint{7.850243in}{1.112980in}}%
\pgfpathlineto{\pgfqpoint{7.850497in}{1.113102in}}%
\pgfpathlineto{\pgfqpoint{7.852380in}{1.114010in}}%
\pgfpathlineto{\pgfqpoint{7.856393in}{1.116009in}}%
\pgfpathlineto{\pgfqpoint{7.864204in}{1.120187in}}%
\pgfpathlineto{\pgfqpoint{7.867694in}{1.122204in}}%
\pgfpathlineto{\pgfqpoint{7.870355in}{1.123820in}}%
\pgfpathlineto{\pgfqpoint{7.879637in}{1.130161in}}%
\pgfpathlineto{\pgfqpoint{7.881767in}{1.131827in}}%
\pgfpathlineto{\pgfqpoint{7.882934in}{1.132787in}}%
\pgfpathlineto{\pgfqpoint{7.894140in}{1.145228in}}%
\pgfpathlineto{\pgfqpoint{7.897041in}{1.151600in}}%
\pgfpathlineto{\pgfqpoint{7.897651in}{1.154079in}}%
\pgfpathlineto{\pgfqpoint{7.897767in}{1.154755in}}%
\pgfpathlineto{\pgfqpoint{7.897766in}{1.161402in}}%
\pgfpathlineto{\pgfqpoint{7.897654in}{1.162056in}}%
\pgfpathlineto{\pgfqpoint{7.896555in}{1.166014in}}%
\pgfpathlineto{\pgfqpoint{7.894887in}{1.169695in}}%
\pgfpathlineto{\pgfqpoint{7.890078in}{1.176631in}}%
\pgfpathlineto{\pgfqpoint{7.885854in}{1.181097in}}%
\pgfpathlineto{\pgfqpoint{7.884032in}{1.182784in}}%
\pgfpathlineto{\pgfqpoint{7.879571in}{1.186517in}}%
\pgfpathlineto{\pgfqpoint{7.863978in}{1.196989in}}%
\pgfpathlineto{\pgfqpoint{7.853085in}{1.202993in}}%
\pgfpathlineto{\pgfqpoint{7.843940in}{1.207541in}}%
\pgfpathlineto{\pgfqpoint{7.812469in}{1.221079in}}%
\pgfpathlineto{\pgfqpoint{7.759069in}{1.240056in}}%
\pgfpathlineto{\pgfqpoint{7.715392in}{1.253675in}}%
\pgfpathlineto{\pgfqpoint{7.690200in}{1.261075in}}%
\pgfpathlineto{\pgfqpoint{7.673181in}{1.265940in}}%
\pgfpathlineto{\pgfqpoint{7.634929in}{1.276585in}}%
\pgfpathlineto{\pgfqpoint{7.589407in}{1.288907in}}%
\pgfpathlineto{\pgfqpoint{7.573662in}{1.293117in}}%
\pgfpathlineto{\pgfqpoint{7.483339in}{1.317184in}}%
\pgfpathlineto{\pgfqpoint{7.472679in}{1.320047in}}%
\pgfpathlineto{\pgfqpoint{7.416122in}{1.335461in}}%
\pgfpathlineto{\pgfqpoint{7.210627in}{1.398879in}}%
\pgfpathlineto{\pgfqpoint{7.204477in}{1.401090in}}%
\pgfpathlineto{\pgfqpoint{7.130173in}{1.430350in}}%
\pgfpathlineto{\pgfqpoint{6.965793in}{1.529234in}}%
\pgfpathlineto{\pgfqpoint{6.946167in}{1.548137in}}%
\pgfpathlineto{\pgfqpoint{6.862650in}{1.687222in}}%
\pgfpathlineto{\pgfqpoint{6.826849in}{1.927786in}}%
\pgfpathlineto{\pgfqpoint{6.821626in}{3.198321in}}%
\pgfpathlineto{\pgfqpoint{6.821626in}{1.058457in}}%
\pgfpathlineto{\pgfqpoint{6.821626in}{1.058457in}}%
\pgfpathlineto{\pgfqpoint{6.826849in}{1.058457in}}%
\pgfpathlineto{\pgfqpoint{6.862650in}{1.058457in}}%
\pgfpathlineto{\pgfqpoint{6.946167in}{1.058457in}}%
\pgfpathlineto{\pgfqpoint{6.965793in}{1.058457in}}%
\pgfpathlineto{\pgfqpoint{7.130173in}{1.058457in}}%
\pgfpathlineto{\pgfqpoint{7.204477in}{1.058457in}}%
\pgfpathlineto{\pgfqpoint{7.210627in}{1.058457in}}%
\pgfpathlineto{\pgfqpoint{7.416122in}{1.058457in}}%
\pgfpathlineto{\pgfqpoint{7.472679in}{1.058457in}}%
\pgfpathlineto{\pgfqpoint{7.483339in}{1.058457in}}%
\pgfpathlineto{\pgfqpoint{7.573662in}{1.058457in}}%
\pgfpathlineto{\pgfqpoint{7.589407in}{1.058457in}}%
\pgfpathlineto{\pgfqpoint{7.634929in}{1.058457in}}%
\pgfpathlineto{\pgfqpoint{7.673181in}{1.058457in}}%
\pgfpathlineto{\pgfqpoint{7.690200in}{1.058457in}}%
\pgfpathlineto{\pgfqpoint{7.715392in}{1.058457in}}%
\pgfpathlineto{\pgfqpoint{7.759069in}{1.058457in}}%
\pgfpathlineto{\pgfqpoint{7.812469in}{1.058457in}}%
\pgfpathlineto{\pgfqpoint{7.843940in}{1.058457in}}%
\pgfpathlineto{\pgfqpoint{7.853085in}{1.058457in}}%
\pgfpathlineto{\pgfqpoint{7.863978in}{1.058457in}}%
\pgfpathlineto{\pgfqpoint{7.879571in}{1.058457in}}%
\pgfpathlineto{\pgfqpoint{7.884032in}{1.058457in}}%
\pgfpathlineto{\pgfqpoint{7.885854in}{1.058457in}}%
\pgfpathlineto{\pgfqpoint{7.890078in}{1.058457in}}%
\pgfpathlineto{\pgfqpoint{7.894887in}{1.058457in}}%
\pgfpathlineto{\pgfqpoint{7.896555in}{1.058457in}}%
\pgfpathlineto{\pgfqpoint{7.897654in}{1.058457in}}%
\pgfpathlineto{\pgfqpoint{7.897766in}{1.058457in}}%
\pgfpathlineto{\pgfqpoint{7.897767in}{1.058457in}}%
\pgfpathlineto{\pgfqpoint{7.897651in}{1.058457in}}%
\pgfpathlineto{\pgfqpoint{7.897041in}{1.058457in}}%
\pgfpathlineto{\pgfqpoint{7.894140in}{1.058457in}}%
\pgfpathlineto{\pgfqpoint{7.882934in}{1.058457in}}%
\pgfpathlineto{\pgfqpoint{7.881767in}{1.058457in}}%
\pgfpathlineto{\pgfqpoint{7.879637in}{1.058457in}}%
\pgfpathlineto{\pgfqpoint{7.870355in}{1.058457in}}%
\pgfpathlineto{\pgfqpoint{7.867694in}{1.058457in}}%
\pgfpathlineto{\pgfqpoint{7.864204in}{1.058457in}}%
\pgfpathlineto{\pgfqpoint{7.856393in}{1.058457in}}%
\pgfpathlineto{\pgfqpoint{7.852380in}{1.058457in}}%
\pgfpathlineto{\pgfqpoint{7.850497in}{1.058457in}}%
\pgfpathlineto{\pgfqpoint{7.850243in}{1.058457in}}%
\pgfpathlineto{\pgfqpoint{7.850168in}{1.058457in}}%
\pgfpathlineto{\pgfqpoint{7.845019in}{1.058457in}}%
\pgfpathlineto{\pgfqpoint{7.844329in}{1.058457in}}%
\pgfpathlineto{\pgfqpoint{7.842282in}{1.058457in}}%
\pgfpathlineto{\pgfqpoint{7.841608in}{1.058457in}}%
\pgfpathlineto{\pgfqpoint{7.838373in}{1.058457in}}%
\pgfpathlineto{\pgfqpoint{7.837290in}{1.058457in}}%
\pgfpathlineto{\pgfqpoint{7.834046in}{1.058457in}}%
\pgfpathlineto{\pgfqpoint{7.828383in}{1.058457in}}%
\pgfpathlineto{\pgfqpoint{7.825069in}{1.058457in}}%
\pgfpathlineto{\pgfqpoint{7.818258in}{1.058457in}}%
\pgfpathlineto{\pgfqpoint{7.811865in}{1.058457in}}%
\pgfpathlineto{\pgfqpoint{7.802657in}{1.058457in}}%
\pgfpathlineto{\pgfqpoint{7.790344in}{1.058457in}}%
\pgfpathlineto{\pgfqpoint{7.771291in}{1.058457in}}%
\pgfpathlineto{\pgfqpoint{7.727259in}{1.058457in}}%
\pgfpathlineto{\pgfqpoint{7.727259in}{1.058457in}}%
\pgfpathclose%
\pgfusepath{stroke,fill}%
}%
\begin{pgfscope}%
\pgfsys@transformshift{0.000000in}{0.000000in}%
\pgfsys@useobject{currentmarker}{}%
\end{pgfscope}%
\end{pgfscope}%
\begin{pgfscope}%
\pgfpathrectangle{\pgfqpoint{6.752313in}{0.670138in}}{\pgfqpoint{1.200000in}{3.929862in}}%
\pgfusepath{clip}%
\pgfsetbuttcap%
\pgfsetroundjoin%
\definecolor{currentfill}{rgb}{1.000000,0.498039,0.054902}%
\pgfsetfillcolor{currentfill}%
\pgfsetfillopacity{0.300000}%
\pgfsetlinewidth{1.003750pt}%
\definecolor{currentstroke}{rgb}{1.000000,0.498039,0.054902}%
\pgfsetstrokecolor{currentstroke}%
\pgfsetstrokeopacity{0.300000}%
\pgfsetdash{}{0pt}%
\pgfsys@defobject{currentmarker}{\pgfqpoint{6.806858in}{1.058457in}}{\pgfqpoint{7.474973in}{4.274111in}}{%
\pgfpathmoveto{\pgfqpoint{7.474120in}{1.058457in}}%
\pgfpathlineto{\pgfqpoint{7.474120in}{1.058457in}}%
\pgfpathlineto{\pgfqpoint{7.474122in}{1.058473in}}%
\pgfpathlineto{\pgfqpoint{7.474132in}{1.058552in}}%
\pgfpathlineto{\pgfqpoint{7.474133in}{1.058556in}}%
\pgfpathlineto{\pgfqpoint{7.474141in}{1.058621in}}%
\pgfpathlineto{\pgfqpoint{7.474141in}{1.058621in}}%
\pgfpathlineto{\pgfqpoint{7.474199in}{1.059074in}}%
\pgfpathlineto{\pgfqpoint{7.474205in}{1.059124in}}%
\pgfpathlineto{\pgfqpoint{7.474205in}{1.059125in}}%
\pgfpathlineto{\pgfqpoint{7.474205in}{1.059125in}}%
\pgfpathlineto{\pgfqpoint{7.474206in}{1.059126in}}%
\pgfpathlineto{\pgfqpoint{7.474209in}{1.059158in}}%
\pgfpathlineto{\pgfqpoint{7.474219in}{1.059233in}}%
\pgfpathlineto{\pgfqpoint{7.474221in}{1.059255in}}%
\pgfpathlineto{\pgfqpoint{7.474281in}{1.059746in}}%
\pgfpathlineto{\pgfqpoint{7.474281in}{1.059750in}}%
\pgfpathlineto{\pgfqpoint{7.474281in}{1.059750in}}%
\pgfpathlineto{\pgfqpoint{7.474282in}{1.059759in}}%
\pgfpathlineto{\pgfqpoint{7.474282in}{1.059759in}}%
\pgfpathlineto{\pgfqpoint{7.474286in}{1.059791in}}%
\pgfpathlineto{\pgfqpoint{7.474304in}{1.059947in}}%
\pgfpathlineto{\pgfqpoint{7.474318in}{1.060065in}}%
\pgfpathlineto{\pgfqpoint{7.474320in}{1.060084in}}%
\pgfpathlineto{\pgfqpoint{7.474324in}{1.060116in}}%
\pgfpathlineto{\pgfqpoint{7.474343in}{1.060285in}}%
\pgfpathlineto{\pgfqpoint{7.474343in}{1.060286in}}%
\pgfpathlineto{\pgfqpoint{7.474346in}{1.060309in}}%
\pgfpathlineto{\pgfqpoint{7.474359in}{1.060424in}}%
\pgfpathlineto{\pgfqpoint{7.474361in}{1.060449in}}%
\pgfpathlineto{\pgfqpoint{7.474366in}{1.060491in}}%
\pgfpathlineto{\pgfqpoint{7.474371in}{1.060537in}}%
\pgfpathlineto{\pgfqpoint{7.474392in}{1.060727in}}%
\pgfpathlineto{\pgfqpoint{7.474396in}{1.060764in}}%
\pgfpathlineto{\pgfqpoint{7.474397in}{1.060772in}}%
\pgfpathlineto{\pgfqpoint{7.474416in}{1.060951in}}%
\pgfpathlineto{\pgfqpoint{7.474460in}{1.061372in}}%
\pgfpathlineto{\pgfqpoint{7.474547in}{1.062269in}}%
\pgfpathlineto{\pgfqpoint{7.474562in}{1.062427in}}%
\pgfpathlineto{\pgfqpoint{7.474562in}{1.062430in}}%
\pgfpathlineto{\pgfqpoint{7.474562in}{1.062431in}}%
\pgfpathlineto{\pgfqpoint{7.474572in}{1.062543in}}%
\pgfpathlineto{\pgfqpoint{7.474584in}{1.062669in}}%
\pgfpathlineto{\pgfqpoint{7.474635in}{1.063270in}}%
\pgfpathlineto{\pgfqpoint{7.474636in}{1.063276in}}%
\pgfpathlineto{\pgfqpoint{7.474654in}{1.063500in}}%
\pgfpathlineto{\pgfqpoint{7.474656in}{1.063529in}}%
\pgfpathlineto{\pgfqpoint{7.474656in}{1.063531in}}%
\pgfpathlineto{\pgfqpoint{7.474665in}{1.063636in}}%
\pgfpathlineto{\pgfqpoint{7.474678in}{1.063803in}}%
\pgfpathlineto{\pgfqpoint{7.474678in}{1.063810in}}%
\pgfpathlineto{\pgfqpoint{7.474691in}{1.063973in}}%
\pgfpathlineto{\pgfqpoint{7.474696in}{1.064039in}}%
\pgfpathlineto{\pgfqpoint{7.474697in}{1.064052in}}%
\pgfpathlineto{\pgfqpoint{7.474699in}{1.064083in}}%
\pgfpathlineto{\pgfqpoint{7.474700in}{1.064094in}}%
\pgfpathlineto{\pgfqpoint{7.474701in}{1.064103in}}%
\pgfpathlineto{\pgfqpoint{7.474737in}{1.064610in}}%
\pgfpathlineto{\pgfqpoint{7.474742in}{1.064681in}}%
\pgfpathlineto{\pgfqpoint{7.474744in}{1.064711in}}%
\pgfpathlineto{\pgfqpoint{7.474748in}{1.064769in}}%
\pgfpathlineto{\pgfqpoint{7.474752in}{1.064829in}}%
\pgfpathlineto{\pgfqpoint{7.474753in}{1.064841in}}%
\pgfpathlineto{\pgfqpoint{7.474771in}{1.065119in}}%
\pgfpathlineto{\pgfqpoint{7.474779in}{1.065253in}}%
\pgfpathlineto{\pgfqpoint{7.474790in}{1.065420in}}%
\pgfpathlineto{\pgfqpoint{7.474806in}{1.065696in}}%
\pgfpathlineto{\pgfqpoint{7.474826in}{1.066052in}}%
\pgfpathlineto{\pgfqpoint{7.474827in}{1.066068in}}%
\pgfpathlineto{\pgfqpoint{7.474848in}{1.066469in}}%
\pgfpathlineto{\pgfqpoint{7.474848in}{1.066469in}}%
\pgfpathlineto{\pgfqpoint{7.474852in}{1.066552in}}%
\pgfpathlineto{\pgfqpoint{7.474859in}{1.066689in}}%
\pgfpathlineto{\pgfqpoint{7.474867in}{1.066866in}}%
\pgfpathlineto{\pgfqpoint{7.474867in}{1.066866in}}%
\pgfpathlineto{\pgfqpoint{7.474867in}{1.066867in}}%
\pgfpathlineto{\pgfqpoint{7.474867in}{1.066868in}}%
\pgfpathlineto{\pgfqpoint{7.474867in}{1.066868in}}%
\pgfpathlineto{\pgfqpoint{7.474869in}{1.066897in}}%
\pgfpathlineto{\pgfqpoint{7.474871in}{1.066945in}}%
\pgfpathlineto{\pgfqpoint{7.474883in}{1.067224in}}%
\pgfpathlineto{\pgfqpoint{7.474886in}{1.067281in}}%
\pgfpathlineto{\pgfqpoint{7.474888in}{1.067330in}}%
\pgfpathlineto{\pgfqpoint{7.474944in}{1.069030in}}%
\pgfpathlineto{\pgfqpoint{7.474944in}{1.069042in}}%
\pgfpathlineto{\pgfqpoint{7.474946in}{1.069102in}}%
\pgfpathlineto{\pgfqpoint{7.474947in}{1.069156in}}%
\pgfpathlineto{\pgfqpoint{7.474972in}{1.071044in}}%
\pgfpathlineto{\pgfqpoint{7.474972in}{1.071044in}}%
\pgfpathlineto{\pgfqpoint{7.474972in}{1.071045in}}%
\pgfpathlineto{\pgfqpoint{7.474973in}{1.071231in}}%
\pgfpathlineto{\pgfqpoint{7.474973in}{1.071231in}}%
\pgfpathlineto{\pgfqpoint{7.474973in}{1.071232in}}%
\pgfpathlineto{\pgfqpoint{7.474973in}{1.071282in}}%
\pgfpathlineto{\pgfqpoint{7.474971in}{1.072084in}}%
\pgfpathlineto{\pgfqpoint{7.474971in}{1.072114in}}%
\pgfpathlineto{\pgfqpoint{7.474970in}{1.072278in}}%
\pgfpathlineto{\pgfqpoint{7.474969in}{1.072368in}}%
\pgfpathlineto{\pgfqpoint{7.474920in}{1.074691in}}%
\pgfpathlineto{\pgfqpoint{7.474879in}{1.075768in}}%
\pgfpathlineto{\pgfqpoint{7.474878in}{1.075791in}}%
\pgfpathlineto{\pgfqpoint{7.474877in}{1.075792in}}%
\pgfpathlineto{\pgfqpoint{7.474877in}{1.075803in}}%
\pgfpathlineto{\pgfqpoint{7.474873in}{1.075900in}}%
\pgfpathlineto{\pgfqpoint{7.474863in}{1.076115in}}%
\pgfpathlineto{\pgfqpoint{7.474850in}{1.076367in}}%
\pgfpathlineto{\pgfqpoint{7.474836in}{1.076647in}}%
\pgfpathlineto{\pgfqpoint{7.474815in}{1.077031in}}%
\pgfpathlineto{\pgfqpoint{7.474788in}{1.077487in}}%
\pgfpathlineto{\pgfqpoint{7.474694in}{1.078873in}}%
\pgfpathlineto{\pgfqpoint{7.474528in}{1.080817in}}%
\pgfpathlineto{\pgfqpoint{7.474526in}{1.080844in}}%
\pgfpathlineto{\pgfqpoint{7.474465in}{1.081462in}}%
\pgfpathlineto{\pgfqpoint{7.474465in}{1.081462in}}%
\pgfpathlineto{\pgfqpoint{7.474465in}{1.081462in}}%
\pgfpathlineto{\pgfqpoint{7.474387in}{1.082204in}}%
\pgfpathlineto{\pgfqpoint{7.474317in}{1.082830in}}%
\pgfpathlineto{\pgfqpoint{7.474313in}{1.082866in}}%
\pgfpathlineto{\pgfqpoint{7.474199in}{1.083812in}}%
\pgfpathlineto{\pgfqpoint{7.474197in}{1.083823in}}%
\pgfpathlineto{\pgfqpoint{7.474162in}{1.084106in}}%
\pgfpathlineto{\pgfqpoint{7.474162in}{1.084106in}}%
\pgfpathlineto{\pgfqpoint{7.474147in}{1.084222in}}%
\pgfpathlineto{\pgfqpoint{7.474088in}{1.084664in}}%
\pgfpathlineto{\pgfqpoint{7.473861in}{1.086265in}}%
\pgfpathlineto{\pgfqpoint{7.473860in}{1.086278in}}%
\pgfpathlineto{\pgfqpoint{7.473808in}{1.086616in}}%
\pgfpathlineto{\pgfqpoint{7.473806in}{1.086629in}}%
\pgfpathlineto{\pgfqpoint{7.473787in}{1.086756in}}%
\pgfpathlineto{\pgfqpoint{7.473150in}{1.090427in}}%
\pgfpathlineto{\pgfqpoint{7.473150in}{1.090427in}}%
\pgfpathlineto{\pgfqpoint{7.472210in}{1.094824in}}%
\pgfpathlineto{\pgfqpoint{7.471727in}{1.096790in}}%
\pgfpathlineto{\pgfqpoint{7.471159in}{1.098925in}}%
\pgfpathlineto{\pgfqpoint{7.470850in}{1.100020in}}%
\pgfpathlineto{\pgfqpoint{7.470814in}{1.100144in}}%
\pgfpathlineto{\pgfqpoint{7.470456in}{1.101357in}}%
\pgfpathlineto{\pgfqpoint{7.465635in}{1.114532in}}%
\pgfpathlineto{\pgfqpoint{7.465597in}{1.114620in}}%
\pgfpathlineto{\pgfqpoint{7.462844in}{1.120602in}}%
\pgfpathlineto{\pgfqpoint{7.412085in}{1.185608in}}%
\pgfpathlineto{\pgfqpoint{6.806858in}{2.652616in}}%
\pgfpathlineto{\pgfqpoint{6.809409in}{2.907321in}}%
\pgfpathlineto{\pgfqpoint{6.807482in}{3.111215in}}%
\pgfpathlineto{\pgfqpoint{6.813975in}{3.888316in}}%
\pgfpathlineto{\pgfqpoint{6.827512in}{4.149343in}}%
\pgfpathlineto{\pgfqpoint{6.828332in}{4.262918in}}%
\pgfpathlineto{\pgfqpoint{6.828332in}{4.262938in}}%
\pgfpathlineto{\pgfqpoint{6.828145in}{4.274091in}}%
\pgfpathlineto{\pgfqpoint{6.828145in}{4.274093in}}%
\pgfpathlineto{\pgfqpoint{6.828145in}{4.274111in}}%
\pgfpathlineto{\pgfqpoint{6.828145in}{1.058457in}}%
\pgfpathlineto{\pgfqpoint{6.828145in}{1.058457in}}%
\pgfpathlineto{\pgfqpoint{6.828145in}{1.058457in}}%
\pgfpathlineto{\pgfqpoint{6.828145in}{1.058457in}}%
\pgfpathlineto{\pgfqpoint{6.828332in}{1.058457in}}%
\pgfpathlineto{\pgfqpoint{6.828332in}{1.058457in}}%
\pgfpathlineto{\pgfqpoint{6.827512in}{1.058457in}}%
\pgfpathlineto{\pgfqpoint{6.813975in}{1.058457in}}%
\pgfpathlineto{\pgfqpoint{6.807482in}{1.058457in}}%
\pgfpathlineto{\pgfqpoint{6.809409in}{1.058457in}}%
\pgfpathlineto{\pgfqpoint{6.806858in}{1.058457in}}%
\pgfpathlineto{\pgfqpoint{7.412085in}{1.058457in}}%
\pgfpathlineto{\pgfqpoint{7.462844in}{1.058457in}}%
\pgfpathlineto{\pgfqpoint{7.465597in}{1.058457in}}%
\pgfpathlineto{\pgfqpoint{7.465635in}{1.058457in}}%
\pgfpathlineto{\pgfqpoint{7.470456in}{1.058457in}}%
\pgfpathlineto{\pgfqpoint{7.470814in}{1.058457in}}%
\pgfpathlineto{\pgfqpoint{7.470850in}{1.058457in}}%
\pgfpathlineto{\pgfqpoint{7.471159in}{1.058457in}}%
\pgfpathlineto{\pgfqpoint{7.471727in}{1.058457in}}%
\pgfpathlineto{\pgfqpoint{7.472210in}{1.058457in}}%
\pgfpathlineto{\pgfqpoint{7.473150in}{1.058457in}}%
\pgfpathlineto{\pgfqpoint{7.473150in}{1.058457in}}%
\pgfpathlineto{\pgfqpoint{7.473787in}{1.058457in}}%
\pgfpathlineto{\pgfqpoint{7.473806in}{1.058457in}}%
\pgfpathlineto{\pgfqpoint{7.473808in}{1.058457in}}%
\pgfpathlineto{\pgfqpoint{7.473860in}{1.058457in}}%
\pgfpathlineto{\pgfqpoint{7.473861in}{1.058457in}}%
\pgfpathlineto{\pgfqpoint{7.474088in}{1.058457in}}%
\pgfpathlineto{\pgfqpoint{7.474147in}{1.058457in}}%
\pgfpathlineto{\pgfqpoint{7.474162in}{1.058457in}}%
\pgfpathlineto{\pgfqpoint{7.474162in}{1.058457in}}%
\pgfpathlineto{\pgfqpoint{7.474197in}{1.058457in}}%
\pgfpathlineto{\pgfqpoint{7.474199in}{1.058457in}}%
\pgfpathlineto{\pgfqpoint{7.474313in}{1.058457in}}%
\pgfpathlineto{\pgfqpoint{7.474317in}{1.058457in}}%
\pgfpathlineto{\pgfqpoint{7.474387in}{1.058457in}}%
\pgfpathlineto{\pgfqpoint{7.474465in}{1.058457in}}%
\pgfpathlineto{\pgfqpoint{7.474465in}{1.058457in}}%
\pgfpathlineto{\pgfqpoint{7.474465in}{1.058457in}}%
\pgfpathlineto{\pgfqpoint{7.474526in}{1.058457in}}%
\pgfpathlineto{\pgfqpoint{7.474528in}{1.058457in}}%
\pgfpathlineto{\pgfqpoint{7.474694in}{1.058457in}}%
\pgfpathlineto{\pgfqpoint{7.474788in}{1.058457in}}%
\pgfpathlineto{\pgfqpoint{7.474815in}{1.058457in}}%
\pgfpathlineto{\pgfqpoint{7.474836in}{1.058457in}}%
\pgfpathlineto{\pgfqpoint{7.474850in}{1.058457in}}%
\pgfpathlineto{\pgfqpoint{7.474863in}{1.058457in}}%
\pgfpathlineto{\pgfqpoint{7.474873in}{1.058457in}}%
\pgfpathlineto{\pgfqpoint{7.474877in}{1.058457in}}%
\pgfpathlineto{\pgfqpoint{7.474877in}{1.058457in}}%
\pgfpathlineto{\pgfqpoint{7.474878in}{1.058457in}}%
\pgfpathlineto{\pgfqpoint{7.474879in}{1.058457in}}%
\pgfpathlineto{\pgfqpoint{7.474920in}{1.058457in}}%
\pgfpathlineto{\pgfqpoint{7.474969in}{1.058457in}}%
\pgfpathlineto{\pgfqpoint{7.474970in}{1.058457in}}%
\pgfpathlineto{\pgfqpoint{7.474971in}{1.058457in}}%
\pgfpathlineto{\pgfqpoint{7.474971in}{1.058457in}}%
\pgfpathlineto{\pgfqpoint{7.474973in}{1.058457in}}%
\pgfpathlineto{\pgfqpoint{7.474973in}{1.058457in}}%
\pgfpathlineto{\pgfqpoint{7.474973in}{1.058457in}}%
\pgfpathlineto{\pgfqpoint{7.474973in}{1.058457in}}%
\pgfpathlineto{\pgfqpoint{7.474972in}{1.058457in}}%
\pgfpathlineto{\pgfqpoint{7.474972in}{1.058457in}}%
\pgfpathlineto{\pgfqpoint{7.474972in}{1.058457in}}%
\pgfpathlineto{\pgfqpoint{7.474947in}{1.058457in}}%
\pgfpathlineto{\pgfqpoint{7.474946in}{1.058457in}}%
\pgfpathlineto{\pgfqpoint{7.474944in}{1.058457in}}%
\pgfpathlineto{\pgfqpoint{7.474944in}{1.058457in}}%
\pgfpathlineto{\pgfqpoint{7.474888in}{1.058457in}}%
\pgfpathlineto{\pgfqpoint{7.474886in}{1.058457in}}%
\pgfpathlineto{\pgfqpoint{7.474883in}{1.058457in}}%
\pgfpathlineto{\pgfqpoint{7.474871in}{1.058457in}}%
\pgfpathlineto{\pgfqpoint{7.474869in}{1.058457in}}%
\pgfpathlineto{\pgfqpoint{7.474867in}{1.058457in}}%
\pgfpathlineto{\pgfqpoint{7.474867in}{1.058457in}}%
\pgfpathlineto{\pgfqpoint{7.474867in}{1.058457in}}%
\pgfpathlineto{\pgfqpoint{7.474867in}{1.058457in}}%
\pgfpathlineto{\pgfqpoint{7.474867in}{1.058457in}}%
\pgfpathlineto{\pgfqpoint{7.474859in}{1.058457in}}%
\pgfpathlineto{\pgfqpoint{7.474852in}{1.058457in}}%
\pgfpathlineto{\pgfqpoint{7.474848in}{1.058457in}}%
\pgfpathlineto{\pgfqpoint{7.474848in}{1.058457in}}%
\pgfpathlineto{\pgfqpoint{7.474827in}{1.058457in}}%
\pgfpathlineto{\pgfqpoint{7.474826in}{1.058457in}}%
\pgfpathlineto{\pgfqpoint{7.474806in}{1.058457in}}%
\pgfpathlineto{\pgfqpoint{7.474790in}{1.058457in}}%
\pgfpathlineto{\pgfqpoint{7.474779in}{1.058457in}}%
\pgfpathlineto{\pgfqpoint{7.474771in}{1.058457in}}%
\pgfpathlineto{\pgfqpoint{7.474753in}{1.058457in}}%
\pgfpathlineto{\pgfqpoint{7.474752in}{1.058457in}}%
\pgfpathlineto{\pgfqpoint{7.474748in}{1.058457in}}%
\pgfpathlineto{\pgfqpoint{7.474744in}{1.058457in}}%
\pgfpathlineto{\pgfqpoint{7.474742in}{1.058457in}}%
\pgfpathlineto{\pgfqpoint{7.474737in}{1.058457in}}%
\pgfpathlineto{\pgfqpoint{7.474701in}{1.058457in}}%
\pgfpathlineto{\pgfqpoint{7.474700in}{1.058457in}}%
\pgfpathlineto{\pgfqpoint{7.474699in}{1.058457in}}%
\pgfpathlineto{\pgfqpoint{7.474697in}{1.058457in}}%
\pgfpathlineto{\pgfqpoint{7.474696in}{1.058457in}}%
\pgfpathlineto{\pgfqpoint{7.474691in}{1.058457in}}%
\pgfpathlineto{\pgfqpoint{7.474678in}{1.058457in}}%
\pgfpathlineto{\pgfqpoint{7.474678in}{1.058457in}}%
\pgfpathlineto{\pgfqpoint{7.474665in}{1.058457in}}%
\pgfpathlineto{\pgfqpoint{7.474656in}{1.058457in}}%
\pgfpathlineto{\pgfqpoint{7.474656in}{1.058457in}}%
\pgfpathlineto{\pgfqpoint{7.474654in}{1.058457in}}%
\pgfpathlineto{\pgfqpoint{7.474636in}{1.058457in}}%
\pgfpathlineto{\pgfqpoint{7.474635in}{1.058457in}}%
\pgfpathlineto{\pgfqpoint{7.474584in}{1.058457in}}%
\pgfpathlineto{\pgfqpoint{7.474572in}{1.058457in}}%
\pgfpathlineto{\pgfqpoint{7.474562in}{1.058457in}}%
\pgfpathlineto{\pgfqpoint{7.474562in}{1.058457in}}%
\pgfpathlineto{\pgfqpoint{7.474562in}{1.058457in}}%
\pgfpathlineto{\pgfqpoint{7.474547in}{1.058457in}}%
\pgfpathlineto{\pgfqpoint{7.474460in}{1.058457in}}%
\pgfpathlineto{\pgfqpoint{7.474416in}{1.058457in}}%
\pgfpathlineto{\pgfqpoint{7.474397in}{1.058457in}}%
\pgfpathlineto{\pgfqpoint{7.474396in}{1.058457in}}%
\pgfpathlineto{\pgfqpoint{7.474392in}{1.058457in}}%
\pgfpathlineto{\pgfqpoint{7.474371in}{1.058457in}}%
\pgfpathlineto{\pgfqpoint{7.474366in}{1.058457in}}%
\pgfpathlineto{\pgfqpoint{7.474361in}{1.058457in}}%
\pgfpathlineto{\pgfqpoint{7.474359in}{1.058457in}}%
\pgfpathlineto{\pgfqpoint{7.474346in}{1.058457in}}%
\pgfpathlineto{\pgfqpoint{7.474343in}{1.058457in}}%
\pgfpathlineto{\pgfqpoint{7.474343in}{1.058457in}}%
\pgfpathlineto{\pgfqpoint{7.474324in}{1.058457in}}%
\pgfpathlineto{\pgfqpoint{7.474320in}{1.058457in}}%
\pgfpathlineto{\pgfqpoint{7.474318in}{1.058457in}}%
\pgfpathlineto{\pgfqpoint{7.474304in}{1.058457in}}%
\pgfpathlineto{\pgfqpoint{7.474286in}{1.058457in}}%
\pgfpathlineto{\pgfqpoint{7.474282in}{1.058457in}}%
\pgfpathlineto{\pgfqpoint{7.474282in}{1.058457in}}%
\pgfpathlineto{\pgfqpoint{7.474281in}{1.058457in}}%
\pgfpathlineto{\pgfqpoint{7.474281in}{1.058457in}}%
\pgfpathlineto{\pgfqpoint{7.474281in}{1.058457in}}%
\pgfpathlineto{\pgfqpoint{7.474221in}{1.058457in}}%
\pgfpathlineto{\pgfqpoint{7.474219in}{1.058457in}}%
\pgfpathlineto{\pgfqpoint{7.474209in}{1.058457in}}%
\pgfpathlineto{\pgfqpoint{7.474206in}{1.058457in}}%
\pgfpathlineto{\pgfqpoint{7.474205in}{1.058457in}}%
\pgfpathlineto{\pgfqpoint{7.474205in}{1.058457in}}%
\pgfpathlineto{\pgfqpoint{7.474205in}{1.058457in}}%
\pgfpathlineto{\pgfqpoint{7.474199in}{1.058457in}}%
\pgfpathlineto{\pgfqpoint{7.474141in}{1.058457in}}%
\pgfpathlineto{\pgfqpoint{7.474141in}{1.058457in}}%
\pgfpathlineto{\pgfqpoint{7.474133in}{1.058457in}}%
\pgfpathlineto{\pgfqpoint{7.474132in}{1.058457in}}%
\pgfpathlineto{\pgfqpoint{7.474122in}{1.058457in}}%
\pgfpathlineto{\pgfqpoint{7.474120in}{1.058457in}}%
\pgfpathlineto{\pgfqpoint{7.474120in}{1.058457in}}%
\pgfpathclose%
\pgfusepath{stroke,fill}%
}%
\begin{pgfscope}%
\pgfsys@transformshift{0.000000in}{0.000000in}%
\pgfsys@useobject{currentmarker}{}%
\end{pgfscope}%
\end{pgfscope}%
\begin{pgfscope}%
\pgfpathrectangle{\pgfqpoint{6.752313in}{0.670138in}}{\pgfqpoint{1.200000in}{3.929862in}}%
\pgfusepath{clip}%
\pgfsetbuttcap%
\pgfsetroundjoin%
\definecolor{currentfill}{rgb}{0.580392,0.403922,0.741176}%
\pgfsetfillcolor{currentfill}%
\pgfsetfillopacity{0.300000}%
\pgfsetlinewidth{1.003750pt}%
\definecolor{currentstroke}{rgb}{0.580392,0.403922,0.741176}%
\pgfsetstrokecolor{currentstroke}%
\pgfsetstrokeopacity{0.300000}%
\pgfsetdash{}{0pt}%
\pgfsys@defobject{currentmarker}{\pgfqpoint{6.810494in}{1.058412in}}{\pgfqpoint{7.859376in}{3.992660in}}{%
\pgfpathmoveto{\pgfqpoint{7.852994in}{1.058457in}}%
\pgfpathlineto{\pgfqpoint{7.852994in}{1.058412in}}%
\pgfpathlineto{\pgfqpoint{7.852994in}{1.058412in}}%
\pgfpathlineto{\pgfqpoint{7.853010in}{1.058436in}}%
\pgfpathlineto{\pgfqpoint{7.853017in}{1.058446in}}%
\pgfpathlineto{\pgfqpoint{7.853023in}{1.058454in}}%
\pgfpathlineto{\pgfqpoint{7.853041in}{1.058481in}}%
\pgfpathlineto{\pgfqpoint{7.853050in}{1.058494in}}%
\pgfpathlineto{\pgfqpoint{7.853103in}{1.058573in}}%
\pgfpathlineto{\pgfqpoint{7.853123in}{1.058602in}}%
\pgfpathlineto{\pgfqpoint{7.853129in}{1.058611in}}%
\pgfpathlineto{\pgfqpoint{7.853178in}{1.058684in}}%
\pgfpathlineto{\pgfqpoint{7.853482in}{1.059145in}}%
\pgfpathlineto{\pgfqpoint{7.853485in}{1.059148in}}%
\pgfpathlineto{\pgfqpoint{7.853485in}{1.059149in}}%
\pgfpathlineto{\pgfqpoint{7.853486in}{1.059150in}}%
\pgfpathlineto{\pgfqpoint{7.853543in}{1.059238in}}%
\pgfpathlineto{\pgfqpoint{7.853564in}{1.059271in}}%
\pgfpathlineto{\pgfqpoint{7.854197in}{1.060274in}}%
\pgfpathlineto{\pgfqpoint{7.854436in}{1.060670in}}%
\pgfpathlineto{\pgfqpoint{7.855003in}{1.061645in}}%
\pgfpathlineto{\pgfqpoint{7.855201in}{1.062001in}}%
\pgfpathlineto{\pgfqpoint{7.855387in}{1.062342in}}%
\pgfpathlineto{\pgfqpoint{7.855414in}{1.062394in}}%
\pgfpathlineto{\pgfqpoint{7.856005in}{1.063541in}}%
\pgfpathlineto{\pgfqpoint{7.856062in}{1.063656in}}%
\pgfpathlineto{\pgfqpoint{7.856064in}{1.063659in}}%
\pgfpathlineto{\pgfqpoint{7.856132in}{1.063798in}}%
\pgfpathlineto{\pgfqpoint{7.856271in}{1.064087in}}%
\pgfpathlineto{\pgfqpoint{7.856415in}{1.064394in}}%
\pgfpathlineto{\pgfqpoint{7.856593in}{1.064785in}}%
\pgfpathlineto{\pgfqpoint{7.856713in}{1.065054in}}%
\pgfpathlineto{\pgfqpoint{7.857215in}{1.066254in}}%
\pgfpathlineto{\pgfqpoint{7.857235in}{1.066304in}}%
\pgfpathlineto{\pgfqpoint{7.857236in}{1.066308in}}%
\pgfpathlineto{\pgfqpoint{7.857268in}{1.066389in}}%
\pgfpathlineto{\pgfqpoint{7.857297in}{1.066463in}}%
\pgfpathlineto{\pgfqpoint{7.857341in}{1.066576in}}%
\pgfpathlineto{\pgfqpoint{7.857362in}{1.066631in}}%
\pgfpathlineto{\pgfqpoint{7.857444in}{1.066849in}}%
\pgfpathlineto{\pgfqpoint{7.857449in}{1.066862in}}%
\pgfpathlineto{\pgfqpoint{7.857503in}{1.067006in}}%
\pgfpathlineto{\pgfqpoint{7.857835in}{1.067949in}}%
\pgfpathlineto{\pgfqpoint{7.858522in}{1.070296in}}%
\pgfpathlineto{\pgfqpoint{7.858552in}{1.070419in}}%
\pgfpathlineto{\pgfqpoint{7.858663in}{1.070884in}}%
\pgfpathlineto{\pgfqpoint{7.858757in}{1.071310in}}%
\pgfpathlineto{\pgfqpoint{7.858833in}{1.071682in}}%
\pgfpathlineto{\pgfqpoint{7.858847in}{1.071749in}}%
\pgfpathlineto{\pgfqpoint{7.858850in}{1.071765in}}%
\pgfpathlineto{\pgfqpoint{7.858850in}{1.071769in}}%
\pgfpathlineto{\pgfqpoint{7.858947in}{1.072287in}}%
\pgfpathlineto{\pgfqpoint{7.858979in}{1.072471in}}%
\pgfpathlineto{\pgfqpoint{7.859034in}{1.072800in}}%
\pgfpathlineto{\pgfqpoint{7.859054in}{1.072930in}}%
\pgfpathlineto{\pgfqpoint{7.859123in}{1.073403in}}%
\pgfpathlineto{\pgfqpoint{7.859221in}{1.074207in}}%
\pgfpathlineto{\pgfqpoint{7.859278in}{1.074793in}}%
\pgfpathlineto{\pgfqpoint{7.859349in}{1.075869in}}%
\pgfpathlineto{\pgfqpoint{7.859376in}{1.076734in}}%
\pgfpathlineto{\pgfqpoint{7.859321in}{1.078953in}}%
\pgfpathlineto{\pgfqpoint{7.858742in}{1.083076in}}%
\pgfpathlineto{\pgfqpoint{7.858668in}{1.083409in}}%
\pgfpathlineto{\pgfqpoint{7.858648in}{1.083500in}}%
\pgfpathlineto{\pgfqpoint{7.858550in}{1.083908in}}%
\pgfpathlineto{\pgfqpoint{7.858345in}{1.084699in}}%
\pgfpathlineto{\pgfqpoint{7.857940in}{1.086053in}}%
\pgfpathlineto{\pgfqpoint{7.856472in}{1.089803in}}%
\pgfpathlineto{\pgfqpoint{7.856182in}{1.090419in}}%
\pgfpathlineto{\pgfqpoint{7.852330in}{1.096864in}}%
\pgfpathlineto{\pgfqpoint{7.851431in}{1.098088in}}%
\pgfpathlineto{\pgfqpoint{7.848714in}{1.101418in}}%
\pgfpathlineto{\pgfqpoint{7.847418in}{1.102858in}}%
\pgfpathlineto{\pgfqpoint{7.846155in}{1.104189in}}%
\pgfpathlineto{\pgfqpoint{7.845274in}{1.105081in}}%
\pgfpathlineto{\pgfqpoint{7.845232in}{1.105123in}}%
\pgfpathlineto{\pgfqpoint{7.843551in}{1.106749in}}%
\pgfpathlineto{\pgfqpoint{7.843450in}{1.106844in}}%
\pgfpathlineto{\pgfqpoint{7.843342in}{1.106946in}}%
\pgfpathlineto{\pgfqpoint{7.841698in}{1.108448in}}%
\pgfpathlineto{\pgfqpoint{7.840553in}{1.109453in}}%
\pgfpathlineto{\pgfqpoint{7.840498in}{1.109501in}}%
\pgfpathlineto{\pgfqpoint{7.840488in}{1.109510in}}%
\pgfpathlineto{\pgfqpoint{7.840097in}{1.109846in}}%
\pgfpathlineto{\pgfqpoint{7.829815in}{1.117735in}}%
\pgfpathlineto{\pgfqpoint{7.676279in}{1.182250in}}%
\pgfpathlineto{\pgfqpoint{7.674540in}{1.182798in}}%
\pgfpathlineto{\pgfqpoint{6.810494in}{2.084036in}}%
\pgfpathlineto{\pgfqpoint{6.810844in}{3.540547in}}%
\pgfpathlineto{\pgfqpoint{6.810844in}{3.992660in}}%
\pgfpathlineto{\pgfqpoint{6.810844in}{1.058457in}}%
\pgfpathlineto{\pgfqpoint{6.810844in}{1.058457in}}%
\pgfpathlineto{\pgfqpoint{6.810844in}{1.058457in}}%
\pgfpathlineto{\pgfqpoint{6.810494in}{1.058457in}}%
\pgfpathlineto{\pgfqpoint{7.674540in}{1.058457in}}%
\pgfpathlineto{\pgfqpoint{7.676279in}{1.058457in}}%
\pgfpathlineto{\pgfqpoint{7.829815in}{1.058457in}}%
\pgfpathlineto{\pgfqpoint{7.840097in}{1.058457in}}%
\pgfpathlineto{\pgfqpoint{7.840488in}{1.058457in}}%
\pgfpathlineto{\pgfqpoint{7.840498in}{1.058457in}}%
\pgfpathlineto{\pgfqpoint{7.840553in}{1.058457in}}%
\pgfpathlineto{\pgfqpoint{7.841698in}{1.058457in}}%
\pgfpathlineto{\pgfqpoint{7.843342in}{1.058457in}}%
\pgfpathlineto{\pgfqpoint{7.843450in}{1.058457in}}%
\pgfpathlineto{\pgfqpoint{7.843551in}{1.058457in}}%
\pgfpathlineto{\pgfqpoint{7.845232in}{1.058457in}}%
\pgfpathlineto{\pgfqpoint{7.845274in}{1.058457in}}%
\pgfpathlineto{\pgfqpoint{7.846155in}{1.058457in}}%
\pgfpathlineto{\pgfqpoint{7.847418in}{1.058457in}}%
\pgfpathlineto{\pgfqpoint{7.848714in}{1.058457in}}%
\pgfpathlineto{\pgfqpoint{7.851431in}{1.058457in}}%
\pgfpathlineto{\pgfqpoint{7.852330in}{1.058457in}}%
\pgfpathlineto{\pgfqpoint{7.856182in}{1.058457in}}%
\pgfpathlineto{\pgfqpoint{7.856472in}{1.058457in}}%
\pgfpathlineto{\pgfqpoint{7.857940in}{1.058457in}}%
\pgfpathlineto{\pgfqpoint{7.858345in}{1.058457in}}%
\pgfpathlineto{\pgfqpoint{7.858550in}{1.058457in}}%
\pgfpathlineto{\pgfqpoint{7.858648in}{1.058457in}}%
\pgfpathlineto{\pgfqpoint{7.858668in}{1.058457in}}%
\pgfpathlineto{\pgfqpoint{7.858742in}{1.058457in}}%
\pgfpathlineto{\pgfqpoint{7.859321in}{1.058457in}}%
\pgfpathlineto{\pgfqpoint{7.859376in}{1.058457in}}%
\pgfpathlineto{\pgfqpoint{7.859349in}{1.058457in}}%
\pgfpathlineto{\pgfqpoint{7.859278in}{1.058457in}}%
\pgfpathlineto{\pgfqpoint{7.859221in}{1.058457in}}%
\pgfpathlineto{\pgfqpoint{7.859123in}{1.058457in}}%
\pgfpathlineto{\pgfqpoint{7.859054in}{1.058457in}}%
\pgfpathlineto{\pgfqpoint{7.859034in}{1.058457in}}%
\pgfpathlineto{\pgfqpoint{7.858979in}{1.058457in}}%
\pgfpathlineto{\pgfqpoint{7.858947in}{1.058457in}}%
\pgfpathlineto{\pgfqpoint{7.858850in}{1.058457in}}%
\pgfpathlineto{\pgfqpoint{7.858850in}{1.058457in}}%
\pgfpathlineto{\pgfqpoint{7.858847in}{1.058457in}}%
\pgfpathlineto{\pgfqpoint{7.858833in}{1.058457in}}%
\pgfpathlineto{\pgfqpoint{7.858757in}{1.058457in}}%
\pgfpathlineto{\pgfqpoint{7.858663in}{1.058457in}}%
\pgfpathlineto{\pgfqpoint{7.858552in}{1.058457in}}%
\pgfpathlineto{\pgfqpoint{7.858522in}{1.058457in}}%
\pgfpathlineto{\pgfqpoint{7.857835in}{1.058457in}}%
\pgfpathlineto{\pgfqpoint{7.857503in}{1.058457in}}%
\pgfpathlineto{\pgfqpoint{7.857449in}{1.058457in}}%
\pgfpathlineto{\pgfqpoint{7.857444in}{1.058457in}}%
\pgfpathlineto{\pgfqpoint{7.857362in}{1.058457in}}%
\pgfpathlineto{\pgfqpoint{7.857341in}{1.058457in}}%
\pgfpathlineto{\pgfqpoint{7.857297in}{1.058457in}}%
\pgfpathlineto{\pgfqpoint{7.857268in}{1.058457in}}%
\pgfpathlineto{\pgfqpoint{7.857236in}{1.058457in}}%
\pgfpathlineto{\pgfqpoint{7.857235in}{1.058457in}}%
\pgfpathlineto{\pgfqpoint{7.857215in}{1.058457in}}%
\pgfpathlineto{\pgfqpoint{7.856713in}{1.058457in}}%
\pgfpathlineto{\pgfqpoint{7.856593in}{1.058457in}}%
\pgfpathlineto{\pgfqpoint{7.856415in}{1.058457in}}%
\pgfpathlineto{\pgfqpoint{7.856271in}{1.058457in}}%
\pgfpathlineto{\pgfqpoint{7.856132in}{1.058457in}}%
\pgfpathlineto{\pgfqpoint{7.856064in}{1.058457in}}%
\pgfpathlineto{\pgfqpoint{7.856062in}{1.058457in}}%
\pgfpathlineto{\pgfqpoint{7.856005in}{1.058457in}}%
\pgfpathlineto{\pgfqpoint{7.855414in}{1.058457in}}%
\pgfpathlineto{\pgfqpoint{7.855387in}{1.058457in}}%
\pgfpathlineto{\pgfqpoint{7.855201in}{1.058457in}}%
\pgfpathlineto{\pgfqpoint{7.855003in}{1.058457in}}%
\pgfpathlineto{\pgfqpoint{7.854436in}{1.058457in}}%
\pgfpathlineto{\pgfqpoint{7.854197in}{1.058457in}}%
\pgfpathlineto{\pgfqpoint{7.853564in}{1.058457in}}%
\pgfpathlineto{\pgfqpoint{7.853543in}{1.058457in}}%
\pgfpathlineto{\pgfqpoint{7.853486in}{1.058457in}}%
\pgfpathlineto{\pgfqpoint{7.853485in}{1.058457in}}%
\pgfpathlineto{\pgfqpoint{7.853485in}{1.058457in}}%
\pgfpathlineto{\pgfqpoint{7.853482in}{1.058457in}}%
\pgfpathlineto{\pgfqpoint{7.853178in}{1.058457in}}%
\pgfpathlineto{\pgfqpoint{7.853129in}{1.058457in}}%
\pgfpathlineto{\pgfqpoint{7.853123in}{1.058457in}}%
\pgfpathlineto{\pgfqpoint{7.853103in}{1.058457in}}%
\pgfpathlineto{\pgfqpoint{7.853050in}{1.058457in}}%
\pgfpathlineto{\pgfqpoint{7.853041in}{1.058457in}}%
\pgfpathlineto{\pgfqpoint{7.853023in}{1.058457in}}%
\pgfpathlineto{\pgfqpoint{7.853017in}{1.058457in}}%
\pgfpathlineto{\pgfqpoint{7.853010in}{1.058457in}}%
\pgfpathlineto{\pgfqpoint{7.852994in}{1.058457in}}%
\pgfpathlineto{\pgfqpoint{7.852994in}{1.058457in}}%
\pgfpathlineto{\pgfqpoint{7.852994in}{1.058457in}}%
\pgfpathclose%
\pgfusepath{stroke,fill}%
}%
\begin{pgfscope}%
\pgfsys@transformshift{0.000000in}{0.000000in}%
\pgfsys@useobject{currentmarker}{}%
\end{pgfscope}%
\end{pgfscope}%
\begin{pgfscope}%
\pgfpathrectangle{\pgfqpoint{6.752313in}{0.670138in}}{\pgfqpoint{1.200000in}{3.929862in}}%
\pgfusepath{clip}%
\pgfsetrectcap%
\pgfsetroundjoin%
\pgfsetlinewidth{0.803000pt}%
\definecolor{currentstroke}{rgb}{0.690196,0.690196,0.690196}%
\pgfsetstrokecolor{currentstroke}%
\pgfsetstrokeopacity{0.200000}%
\pgfsetdash{}{0pt}%
\pgfpathmoveto{\pgfqpoint{6.798039in}{0.670138in}}%
\pgfpathlineto{\pgfqpoint{6.798039in}{4.600000in}}%
\pgfusepath{stroke}%
\end{pgfscope}%
\begin{pgfscope}%
\pgfsetbuttcap%
\pgfsetroundjoin%
\definecolor{currentfill}{rgb}{0.000000,0.000000,0.000000}%
\pgfsetfillcolor{currentfill}%
\pgfsetlinewidth{0.803000pt}%
\definecolor{currentstroke}{rgb}{0.000000,0.000000,0.000000}%
\pgfsetstrokecolor{currentstroke}%
\pgfsetdash{}{0pt}%
\pgfsys@defobject{currentmarker}{\pgfqpoint{0.000000in}{-0.048611in}}{\pgfqpoint{0.000000in}{0.000000in}}{%
\pgfpathmoveto{\pgfqpoint{0.000000in}{0.000000in}}%
\pgfpathlineto{\pgfqpoint{0.000000in}{-0.048611in}}%
\pgfusepath{stroke,fill}%
}%
\begin{pgfscope}%
\pgfsys@transformshift{6.798039in}{0.670138in}%
\pgfsys@useobject{currentmarker}{}%
\end{pgfscope}%
\end{pgfscope}%
\begin{pgfscope}%
\pgfpathrectangle{\pgfqpoint{6.752313in}{0.670138in}}{\pgfqpoint{1.200000in}{3.929862in}}%
\pgfusepath{clip}%
\pgfsetrectcap%
\pgfsetroundjoin%
\pgfsetlinewidth{0.803000pt}%
\definecolor{currentstroke}{rgb}{0.690196,0.690196,0.690196}%
\pgfsetstrokecolor{currentstroke}%
\pgfsetstrokeopacity{0.200000}%
\pgfsetdash{}{0pt}%
\pgfpathmoveto{\pgfqpoint{7.455827in}{0.670138in}}%
\pgfpathlineto{\pgfqpoint{7.455827in}{4.600000in}}%
\pgfusepath{stroke}%
\end{pgfscope}%
\begin{pgfscope}%
\pgfsetbuttcap%
\pgfsetroundjoin%
\definecolor{currentfill}{rgb}{0.000000,0.000000,0.000000}%
\pgfsetfillcolor{currentfill}%
\pgfsetlinewidth{0.803000pt}%
\definecolor{currentstroke}{rgb}{0.000000,0.000000,0.000000}%
\pgfsetstrokecolor{currentstroke}%
\pgfsetdash{}{0pt}%
\pgfsys@defobject{currentmarker}{\pgfqpoint{0.000000in}{-0.048611in}}{\pgfqpoint{0.000000in}{0.000000in}}{%
\pgfpathmoveto{\pgfqpoint{0.000000in}{0.000000in}}%
\pgfpathlineto{\pgfqpoint{0.000000in}{-0.048611in}}%
\pgfusepath{stroke,fill}%
}%
\begin{pgfscope}%
\pgfsys@transformshift{7.455827in}{0.670138in}%
\pgfsys@useobject{currentmarker}{}%
\end{pgfscope}%
\end{pgfscope}%
\begin{pgfscope}%
\pgfpathrectangle{\pgfqpoint{6.752313in}{0.670138in}}{\pgfqpoint{1.200000in}{3.929862in}}%
\pgfusepath{clip}%
\pgfsetrectcap%
\pgfsetroundjoin%
\pgfsetlinewidth{0.803000pt}%
\definecolor{currentstroke}{rgb}{0.690196,0.690196,0.690196}%
\pgfsetstrokecolor{currentstroke}%
\pgfsetstrokeopacity{0.200000}%
\pgfsetdash{}{0pt}%
\pgfpathmoveto{\pgfqpoint{6.752313in}{0.670138in}}%
\pgfpathlineto{\pgfqpoint{7.952313in}{0.670138in}}%
\pgfusepath{stroke}%
\end{pgfscope}%
\begin{pgfscope}%
\pgfsetbuttcap%
\pgfsetroundjoin%
\definecolor{currentfill}{rgb}{0.000000,0.000000,0.000000}%
\pgfsetfillcolor{currentfill}%
\pgfsetlinewidth{0.803000pt}%
\definecolor{currentstroke}{rgb}{0.000000,0.000000,0.000000}%
\pgfsetstrokecolor{currentstroke}%
\pgfsetdash{}{0pt}%
\pgfsys@defobject{currentmarker}{\pgfqpoint{-0.048611in}{0.000000in}}{\pgfqpoint{-0.000000in}{0.000000in}}{%
\pgfpathmoveto{\pgfqpoint{-0.000000in}{0.000000in}}%
\pgfpathlineto{\pgfqpoint{-0.048611in}{0.000000in}}%
\pgfusepath{stroke,fill}%
}%
\begin{pgfscope}%
\pgfsys@transformshift{6.752313in}{0.670138in}%
\pgfsys@useobject{currentmarker}{}%
\end{pgfscope}%
\end{pgfscope}%
\begin{pgfscope}%
\pgfpathrectangle{\pgfqpoint{6.752313in}{0.670138in}}{\pgfqpoint{1.200000in}{3.929862in}}%
\pgfusepath{clip}%
\pgfsetrectcap%
\pgfsetroundjoin%
\pgfsetlinewidth{0.803000pt}%
\definecolor{currentstroke}{rgb}{0.690196,0.690196,0.690196}%
\pgfsetstrokecolor{currentstroke}%
\pgfsetstrokeopacity{0.200000}%
\pgfsetdash{}{0pt}%
\pgfpathmoveto{\pgfqpoint{6.752313in}{1.116713in}}%
\pgfpathlineto{\pgfqpoint{7.952313in}{1.116713in}}%
\pgfusepath{stroke}%
\end{pgfscope}%
\begin{pgfscope}%
\pgfsetbuttcap%
\pgfsetroundjoin%
\definecolor{currentfill}{rgb}{0.000000,0.000000,0.000000}%
\pgfsetfillcolor{currentfill}%
\pgfsetlinewidth{0.803000pt}%
\definecolor{currentstroke}{rgb}{0.000000,0.000000,0.000000}%
\pgfsetstrokecolor{currentstroke}%
\pgfsetdash{}{0pt}%
\pgfsys@defobject{currentmarker}{\pgfqpoint{-0.048611in}{0.000000in}}{\pgfqpoint{-0.000000in}{0.000000in}}{%
\pgfpathmoveto{\pgfqpoint{-0.000000in}{0.000000in}}%
\pgfpathlineto{\pgfqpoint{-0.048611in}{0.000000in}}%
\pgfusepath{stroke,fill}%
}%
\begin{pgfscope}%
\pgfsys@transformshift{6.752313in}{1.116713in}%
\pgfsys@useobject{currentmarker}{}%
\end{pgfscope}%
\end{pgfscope}%
\begin{pgfscope}%
\pgfpathrectangle{\pgfqpoint{6.752313in}{0.670138in}}{\pgfqpoint{1.200000in}{3.929862in}}%
\pgfusepath{clip}%
\pgfsetrectcap%
\pgfsetroundjoin%
\pgfsetlinewidth{0.803000pt}%
\definecolor{currentstroke}{rgb}{0.690196,0.690196,0.690196}%
\pgfsetstrokecolor{currentstroke}%
\pgfsetstrokeopacity{0.200000}%
\pgfsetdash{}{0pt}%
\pgfpathmoveto{\pgfqpoint{6.752313in}{1.563288in}}%
\pgfpathlineto{\pgfqpoint{7.952313in}{1.563288in}}%
\pgfusepath{stroke}%
\end{pgfscope}%
\begin{pgfscope}%
\pgfsetbuttcap%
\pgfsetroundjoin%
\definecolor{currentfill}{rgb}{0.000000,0.000000,0.000000}%
\pgfsetfillcolor{currentfill}%
\pgfsetlinewidth{0.803000pt}%
\definecolor{currentstroke}{rgb}{0.000000,0.000000,0.000000}%
\pgfsetstrokecolor{currentstroke}%
\pgfsetdash{}{0pt}%
\pgfsys@defobject{currentmarker}{\pgfqpoint{-0.048611in}{0.000000in}}{\pgfqpoint{-0.000000in}{0.000000in}}{%
\pgfpathmoveto{\pgfqpoint{-0.000000in}{0.000000in}}%
\pgfpathlineto{\pgfqpoint{-0.048611in}{0.000000in}}%
\pgfusepath{stroke,fill}%
}%
\begin{pgfscope}%
\pgfsys@transformshift{6.752313in}{1.563288in}%
\pgfsys@useobject{currentmarker}{}%
\end{pgfscope}%
\end{pgfscope}%
\begin{pgfscope}%
\pgfpathrectangle{\pgfqpoint{6.752313in}{0.670138in}}{\pgfqpoint{1.200000in}{3.929862in}}%
\pgfusepath{clip}%
\pgfsetrectcap%
\pgfsetroundjoin%
\pgfsetlinewidth{0.803000pt}%
\definecolor{currentstroke}{rgb}{0.690196,0.690196,0.690196}%
\pgfsetstrokecolor{currentstroke}%
\pgfsetstrokeopacity{0.200000}%
\pgfsetdash{}{0pt}%
\pgfpathmoveto{\pgfqpoint{6.752313in}{2.009864in}}%
\pgfpathlineto{\pgfqpoint{7.952313in}{2.009864in}}%
\pgfusepath{stroke}%
\end{pgfscope}%
\begin{pgfscope}%
\pgfsetbuttcap%
\pgfsetroundjoin%
\definecolor{currentfill}{rgb}{0.000000,0.000000,0.000000}%
\pgfsetfillcolor{currentfill}%
\pgfsetlinewidth{0.803000pt}%
\definecolor{currentstroke}{rgb}{0.000000,0.000000,0.000000}%
\pgfsetstrokecolor{currentstroke}%
\pgfsetdash{}{0pt}%
\pgfsys@defobject{currentmarker}{\pgfqpoint{-0.048611in}{0.000000in}}{\pgfqpoint{-0.000000in}{0.000000in}}{%
\pgfpathmoveto{\pgfqpoint{-0.000000in}{0.000000in}}%
\pgfpathlineto{\pgfqpoint{-0.048611in}{0.000000in}}%
\pgfusepath{stroke,fill}%
}%
\begin{pgfscope}%
\pgfsys@transformshift{6.752313in}{2.009864in}%
\pgfsys@useobject{currentmarker}{}%
\end{pgfscope}%
\end{pgfscope}%
\begin{pgfscope}%
\pgfpathrectangle{\pgfqpoint{6.752313in}{0.670138in}}{\pgfqpoint{1.200000in}{3.929862in}}%
\pgfusepath{clip}%
\pgfsetrectcap%
\pgfsetroundjoin%
\pgfsetlinewidth{0.803000pt}%
\definecolor{currentstroke}{rgb}{0.690196,0.690196,0.690196}%
\pgfsetstrokecolor{currentstroke}%
\pgfsetstrokeopacity{0.200000}%
\pgfsetdash{}{0pt}%
\pgfpathmoveto{\pgfqpoint{6.752313in}{2.456439in}}%
\pgfpathlineto{\pgfqpoint{7.952313in}{2.456439in}}%
\pgfusepath{stroke}%
\end{pgfscope}%
\begin{pgfscope}%
\pgfsetbuttcap%
\pgfsetroundjoin%
\definecolor{currentfill}{rgb}{0.000000,0.000000,0.000000}%
\pgfsetfillcolor{currentfill}%
\pgfsetlinewidth{0.803000pt}%
\definecolor{currentstroke}{rgb}{0.000000,0.000000,0.000000}%
\pgfsetstrokecolor{currentstroke}%
\pgfsetdash{}{0pt}%
\pgfsys@defobject{currentmarker}{\pgfqpoint{-0.048611in}{0.000000in}}{\pgfqpoint{-0.000000in}{0.000000in}}{%
\pgfpathmoveto{\pgfqpoint{-0.000000in}{0.000000in}}%
\pgfpathlineto{\pgfqpoint{-0.048611in}{0.000000in}}%
\pgfusepath{stroke,fill}%
}%
\begin{pgfscope}%
\pgfsys@transformshift{6.752313in}{2.456439in}%
\pgfsys@useobject{currentmarker}{}%
\end{pgfscope}%
\end{pgfscope}%
\begin{pgfscope}%
\pgfpathrectangle{\pgfqpoint{6.752313in}{0.670138in}}{\pgfqpoint{1.200000in}{3.929862in}}%
\pgfusepath{clip}%
\pgfsetrectcap%
\pgfsetroundjoin%
\pgfsetlinewidth{0.803000pt}%
\definecolor{currentstroke}{rgb}{0.690196,0.690196,0.690196}%
\pgfsetstrokecolor{currentstroke}%
\pgfsetstrokeopacity{0.200000}%
\pgfsetdash{}{0pt}%
\pgfpathmoveto{\pgfqpoint{6.752313in}{2.903014in}}%
\pgfpathlineto{\pgfqpoint{7.952313in}{2.903014in}}%
\pgfusepath{stroke}%
\end{pgfscope}%
\begin{pgfscope}%
\pgfsetbuttcap%
\pgfsetroundjoin%
\definecolor{currentfill}{rgb}{0.000000,0.000000,0.000000}%
\pgfsetfillcolor{currentfill}%
\pgfsetlinewidth{0.803000pt}%
\definecolor{currentstroke}{rgb}{0.000000,0.000000,0.000000}%
\pgfsetstrokecolor{currentstroke}%
\pgfsetdash{}{0pt}%
\pgfsys@defobject{currentmarker}{\pgfqpoint{-0.048611in}{0.000000in}}{\pgfqpoint{-0.000000in}{0.000000in}}{%
\pgfpathmoveto{\pgfqpoint{-0.000000in}{0.000000in}}%
\pgfpathlineto{\pgfqpoint{-0.048611in}{0.000000in}}%
\pgfusepath{stroke,fill}%
}%
\begin{pgfscope}%
\pgfsys@transformshift{6.752313in}{2.903014in}%
\pgfsys@useobject{currentmarker}{}%
\end{pgfscope}%
\end{pgfscope}%
\begin{pgfscope}%
\pgfpathrectangle{\pgfqpoint{6.752313in}{0.670138in}}{\pgfqpoint{1.200000in}{3.929862in}}%
\pgfusepath{clip}%
\pgfsetrectcap%
\pgfsetroundjoin%
\pgfsetlinewidth{0.803000pt}%
\definecolor{currentstroke}{rgb}{0.690196,0.690196,0.690196}%
\pgfsetstrokecolor{currentstroke}%
\pgfsetstrokeopacity{0.200000}%
\pgfsetdash{}{0pt}%
\pgfpathmoveto{\pgfqpoint{6.752313in}{3.349589in}}%
\pgfpathlineto{\pgfqpoint{7.952313in}{3.349589in}}%
\pgfusepath{stroke}%
\end{pgfscope}%
\begin{pgfscope}%
\pgfsetbuttcap%
\pgfsetroundjoin%
\definecolor{currentfill}{rgb}{0.000000,0.000000,0.000000}%
\pgfsetfillcolor{currentfill}%
\pgfsetlinewidth{0.803000pt}%
\definecolor{currentstroke}{rgb}{0.000000,0.000000,0.000000}%
\pgfsetstrokecolor{currentstroke}%
\pgfsetdash{}{0pt}%
\pgfsys@defobject{currentmarker}{\pgfqpoint{-0.048611in}{0.000000in}}{\pgfqpoint{-0.000000in}{0.000000in}}{%
\pgfpathmoveto{\pgfqpoint{-0.000000in}{0.000000in}}%
\pgfpathlineto{\pgfqpoint{-0.048611in}{0.000000in}}%
\pgfusepath{stroke,fill}%
}%
\begin{pgfscope}%
\pgfsys@transformshift{6.752313in}{3.349589in}%
\pgfsys@useobject{currentmarker}{}%
\end{pgfscope}%
\end{pgfscope}%
\begin{pgfscope}%
\pgfpathrectangle{\pgfqpoint{6.752313in}{0.670138in}}{\pgfqpoint{1.200000in}{3.929862in}}%
\pgfusepath{clip}%
\pgfsetrectcap%
\pgfsetroundjoin%
\pgfsetlinewidth{0.803000pt}%
\definecolor{currentstroke}{rgb}{0.690196,0.690196,0.690196}%
\pgfsetstrokecolor{currentstroke}%
\pgfsetstrokeopacity{0.200000}%
\pgfsetdash{}{0pt}%
\pgfpathmoveto{\pgfqpoint{6.752313in}{3.796165in}}%
\pgfpathlineto{\pgfqpoint{7.952313in}{3.796165in}}%
\pgfusepath{stroke}%
\end{pgfscope}%
\begin{pgfscope}%
\pgfsetbuttcap%
\pgfsetroundjoin%
\definecolor{currentfill}{rgb}{0.000000,0.000000,0.000000}%
\pgfsetfillcolor{currentfill}%
\pgfsetlinewidth{0.803000pt}%
\definecolor{currentstroke}{rgb}{0.000000,0.000000,0.000000}%
\pgfsetstrokecolor{currentstroke}%
\pgfsetdash{}{0pt}%
\pgfsys@defobject{currentmarker}{\pgfqpoint{-0.048611in}{0.000000in}}{\pgfqpoint{-0.000000in}{0.000000in}}{%
\pgfpathmoveto{\pgfqpoint{-0.000000in}{0.000000in}}%
\pgfpathlineto{\pgfqpoint{-0.048611in}{0.000000in}}%
\pgfusepath{stroke,fill}%
}%
\begin{pgfscope}%
\pgfsys@transformshift{6.752313in}{3.796165in}%
\pgfsys@useobject{currentmarker}{}%
\end{pgfscope}%
\end{pgfscope}%
\begin{pgfscope}%
\pgfpathrectangle{\pgfqpoint{6.752313in}{0.670138in}}{\pgfqpoint{1.200000in}{3.929862in}}%
\pgfusepath{clip}%
\pgfsetrectcap%
\pgfsetroundjoin%
\pgfsetlinewidth{0.803000pt}%
\definecolor{currentstroke}{rgb}{0.690196,0.690196,0.690196}%
\pgfsetstrokecolor{currentstroke}%
\pgfsetstrokeopacity{0.200000}%
\pgfsetdash{}{0pt}%
\pgfpathmoveto{\pgfqpoint{6.752313in}{4.242740in}}%
\pgfpathlineto{\pgfqpoint{7.952313in}{4.242740in}}%
\pgfusepath{stroke}%
\end{pgfscope}%
\begin{pgfscope}%
\pgfsetbuttcap%
\pgfsetroundjoin%
\definecolor{currentfill}{rgb}{0.000000,0.000000,0.000000}%
\pgfsetfillcolor{currentfill}%
\pgfsetlinewidth{0.803000pt}%
\definecolor{currentstroke}{rgb}{0.000000,0.000000,0.000000}%
\pgfsetstrokecolor{currentstroke}%
\pgfsetdash{}{0pt}%
\pgfsys@defobject{currentmarker}{\pgfqpoint{-0.048611in}{0.000000in}}{\pgfqpoint{-0.000000in}{0.000000in}}{%
\pgfpathmoveto{\pgfqpoint{-0.000000in}{0.000000in}}%
\pgfpathlineto{\pgfqpoint{-0.048611in}{0.000000in}}%
\pgfusepath{stroke,fill}%
}%
\begin{pgfscope}%
\pgfsys@transformshift{6.752313in}{4.242740in}%
\pgfsys@useobject{currentmarker}{}%
\end{pgfscope}%
\end{pgfscope}%
\begin{pgfscope}%
\pgfpathrectangle{\pgfqpoint{6.752313in}{0.670138in}}{\pgfqpoint{1.200000in}{3.929862in}}%
\pgfusepath{clip}%
\pgfsetrectcap%
\pgfsetroundjoin%
\pgfsetlinewidth{1.505625pt}%
\definecolor{currentstroke}{rgb}{0.121569,0.466667,0.705882}%
\pgfsetstrokecolor{currentstroke}%
\pgfsetdash{}{0pt}%
\pgfpathmoveto{\pgfqpoint{7.727259in}{1.071083in}}%
\pgfpathlineto{\pgfqpoint{7.771291in}{1.083755in}}%
\pgfpathlineto{\pgfqpoint{7.790344in}{1.089795in}}%
\pgfpathlineto{\pgfqpoint{7.802657in}{1.093947in}}%
\pgfpathlineto{\pgfqpoint{7.811865in}{1.097209in}}%
\pgfpathlineto{\pgfqpoint{7.818258in}{1.099567in}}%
\pgfpathlineto{\pgfqpoint{7.825069in}{1.102175in}}%
\pgfpathlineto{\pgfqpoint{7.828383in}{1.103485in}}%
\pgfpathlineto{\pgfqpoint{7.834046in}{1.105791in}}%
\pgfpathlineto{\pgfqpoint{7.837290in}{1.107156in}}%
\pgfpathlineto{\pgfqpoint{7.838373in}{1.107619in}}%
\pgfpathlineto{\pgfqpoint{7.841608in}{1.109025in}}%
\pgfpathlineto{\pgfqpoint{7.842282in}{1.109323in}}%
\pgfpathlineto{\pgfqpoint{7.844329in}{1.110239in}}%
\pgfpathlineto{\pgfqpoint{7.845019in}{1.110551in}}%
\pgfpathlineto{\pgfqpoint{7.850168in}{1.112945in}}%
\pgfpathlineto{\pgfqpoint{7.850243in}{1.112980in}}%
\pgfpathlineto{\pgfqpoint{7.850497in}{1.113102in}}%
\pgfpathlineto{\pgfqpoint{7.852380in}{1.114010in}}%
\pgfpathlineto{\pgfqpoint{7.856393in}{1.116009in}}%
\pgfpathlineto{\pgfqpoint{7.864204in}{1.120187in}}%
\pgfpathlineto{\pgfqpoint{7.867694in}{1.122204in}}%
\pgfpathlineto{\pgfqpoint{7.870355in}{1.123820in}}%
\pgfpathlineto{\pgfqpoint{7.879637in}{1.130161in}}%
\pgfpathlineto{\pgfqpoint{7.881767in}{1.131827in}}%
\pgfpathlineto{\pgfqpoint{7.882934in}{1.132787in}}%
\pgfpathlineto{\pgfqpoint{7.894140in}{1.145228in}}%
\pgfpathlineto{\pgfqpoint{7.897041in}{1.151600in}}%
\pgfpathlineto{\pgfqpoint{7.897651in}{1.154079in}}%
\pgfpathlineto{\pgfqpoint{7.897767in}{1.154755in}}%
\pgfpathlineto{\pgfqpoint{7.897766in}{1.161402in}}%
\pgfpathlineto{\pgfqpoint{7.897654in}{1.162056in}}%
\pgfpathlineto{\pgfqpoint{7.896555in}{1.166014in}}%
\pgfpathlineto{\pgfqpoint{7.894887in}{1.169695in}}%
\pgfpathlineto{\pgfqpoint{7.890078in}{1.176631in}}%
\pgfpathlineto{\pgfqpoint{7.885854in}{1.181097in}}%
\pgfpathlineto{\pgfqpoint{7.884032in}{1.182784in}}%
\pgfpathlineto{\pgfqpoint{7.879571in}{1.186517in}}%
\pgfpathlineto{\pgfqpoint{7.863978in}{1.196989in}}%
\pgfpathlineto{\pgfqpoint{7.853085in}{1.202993in}}%
\pgfpathlineto{\pgfqpoint{7.843940in}{1.207541in}}%
\pgfpathlineto{\pgfqpoint{7.812469in}{1.221079in}}%
\pgfpathlineto{\pgfqpoint{7.759069in}{1.240056in}}%
\pgfpathlineto{\pgfqpoint{7.715392in}{1.253675in}}%
\pgfpathlineto{\pgfqpoint{7.690200in}{1.261075in}}%
\pgfpathlineto{\pgfqpoint{7.673181in}{1.265940in}}%
\pgfpathlineto{\pgfqpoint{7.634929in}{1.276585in}}%
\pgfpathlineto{\pgfqpoint{7.589407in}{1.288907in}}%
\pgfpathlineto{\pgfqpoint{7.573662in}{1.293117in}}%
\pgfpathlineto{\pgfqpoint{7.483339in}{1.317184in}}%
\pgfpathlineto{\pgfqpoint{7.472679in}{1.320047in}}%
\pgfpathlineto{\pgfqpoint{7.416122in}{1.335461in}}%
\pgfpathlineto{\pgfqpoint{7.210627in}{1.398879in}}%
\pgfpathlineto{\pgfqpoint{7.204477in}{1.401090in}}%
\pgfpathlineto{\pgfqpoint{7.130173in}{1.430350in}}%
\pgfpathlineto{\pgfqpoint{6.965793in}{1.529234in}}%
\pgfpathlineto{\pgfqpoint{6.946167in}{1.548137in}}%
\pgfpathlineto{\pgfqpoint{6.862650in}{1.687222in}}%
\pgfpathlineto{\pgfqpoint{6.826849in}{1.927786in}}%
\pgfpathlineto{\pgfqpoint{6.821626in}{3.198321in}}%
\pgfusepath{stroke}%
\end{pgfscope}%
\begin{pgfscope}%
\pgfpathrectangle{\pgfqpoint{6.752313in}{0.670138in}}{\pgfqpoint{1.200000in}{3.929862in}}%
\pgfusepath{clip}%
\pgfsetrectcap%
\pgfsetroundjoin%
\pgfsetlinewidth{1.505625pt}%
\definecolor{currentstroke}{rgb}{1.000000,0.498039,0.054902}%
\pgfsetstrokecolor{currentstroke}%
\pgfsetdash{}{0pt}%
\pgfpathmoveto{\pgfqpoint{7.474120in}{1.058457in}}%
\pgfpathlineto{\pgfqpoint{7.474806in}{1.065696in}}%
\pgfpathlineto{\pgfqpoint{7.474969in}{1.072368in}}%
\pgfpathlineto{\pgfqpoint{7.474526in}{1.080844in}}%
\pgfpathlineto{\pgfqpoint{7.473787in}{1.086756in}}%
\pgfpathlineto{\pgfqpoint{7.472210in}{1.094824in}}%
\pgfpathlineto{\pgfqpoint{7.470456in}{1.101357in}}%
\pgfpathlineto{\pgfqpoint{7.465597in}{1.114620in}}%
\pgfpathlineto{\pgfqpoint{7.462844in}{1.120602in}}%
\pgfpathlineto{\pgfqpoint{7.412085in}{1.185608in}}%
\pgfpathlineto{\pgfqpoint{6.806858in}{2.652616in}}%
\pgfpathlineto{\pgfqpoint{6.809409in}{2.907321in}}%
\pgfpathlineto{\pgfqpoint{6.807482in}{3.111215in}}%
\pgfpathlineto{\pgfqpoint{6.813975in}{3.888316in}}%
\pgfpathlineto{\pgfqpoint{6.827512in}{4.149343in}}%
\pgfpathlineto{\pgfqpoint{6.828145in}{4.274111in}}%
\pgfpathlineto{\pgfqpoint{6.828145in}{4.274111in}}%
\pgfusepath{stroke}%
\end{pgfscope}%
\begin{pgfscope}%
\pgfpathrectangle{\pgfqpoint{6.752313in}{0.670138in}}{\pgfqpoint{1.200000in}{3.929862in}}%
\pgfusepath{clip}%
\pgfsetrectcap%
\pgfsetroundjoin%
\pgfsetlinewidth{1.505625pt}%
\definecolor{currentstroke}{rgb}{0.580392,0.403922,0.741176}%
\pgfsetstrokecolor{currentstroke}%
\pgfsetdash{}{0pt}%
\pgfpathmoveto{\pgfqpoint{7.852994in}{1.058412in}}%
\pgfpathlineto{\pgfqpoint{7.852994in}{1.058412in}}%
\pgfpathlineto{\pgfqpoint{7.853010in}{1.058436in}}%
\pgfpathlineto{\pgfqpoint{7.853017in}{1.058446in}}%
\pgfpathlineto{\pgfqpoint{7.853023in}{1.058454in}}%
\pgfpathlineto{\pgfqpoint{7.853041in}{1.058481in}}%
\pgfpathlineto{\pgfqpoint{7.853050in}{1.058494in}}%
\pgfpathlineto{\pgfqpoint{7.853103in}{1.058573in}}%
\pgfpathlineto{\pgfqpoint{7.853123in}{1.058602in}}%
\pgfpathlineto{\pgfqpoint{7.853129in}{1.058611in}}%
\pgfpathlineto{\pgfqpoint{7.853178in}{1.058684in}}%
\pgfpathlineto{\pgfqpoint{7.853482in}{1.059145in}}%
\pgfpathlineto{\pgfqpoint{7.853485in}{1.059148in}}%
\pgfpathlineto{\pgfqpoint{7.853485in}{1.059149in}}%
\pgfpathlineto{\pgfqpoint{7.853486in}{1.059150in}}%
\pgfpathlineto{\pgfqpoint{7.853543in}{1.059238in}}%
\pgfpathlineto{\pgfqpoint{7.853564in}{1.059271in}}%
\pgfpathlineto{\pgfqpoint{7.854197in}{1.060274in}}%
\pgfpathlineto{\pgfqpoint{7.854436in}{1.060670in}}%
\pgfpathlineto{\pgfqpoint{7.855003in}{1.061645in}}%
\pgfpathlineto{\pgfqpoint{7.855201in}{1.062001in}}%
\pgfpathlineto{\pgfqpoint{7.855387in}{1.062342in}}%
\pgfpathlineto{\pgfqpoint{7.855414in}{1.062394in}}%
\pgfpathlineto{\pgfqpoint{7.856005in}{1.063541in}}%
\pgfpathlineto{\pgfqpoint{7.856062in}{1.063656in}}%
\pgfpathlineto{\pgfqpoint{7.856064in}{1.063659in}}%
\pgfpathlineto{\pgfqpoint{7.856132in}{1.063798in}}%
\pgfpathlineto{\pgfqpoint{7.856271in}{1.064087in}}%
\pgfpathlineto{\pgfqpoint{7.856415in}{1.064394in}}%
\pgfpathlineto{\pgfqpoint{7.856593in}{1.064785in}}%
\pgfpathlineto{\pgfqpoint{7.856713in}{1.065054in}}%
\pgfpathlineto{\pgfqpoint{7.857215in}{1.066254in}}%
\pgfpathlineto{\pgfqpoint{7.857235in}{1.066304in}}%
\pgfpathlineto{\pgfqpoint{7.857236in}{1.066308in}}%
\pgfpathlineto{\pgfqpoint{7.857268in}{1.066389in}}%
\pgfpathlineto{\pgfqpoint{7.857297in}{1.066463in}}%
\pgfpathlineto{\pgfqpoint{7.857341in}{1.066576in}}%
\pgfpathlineto{\pgfqpoint{7.857362in}{1.066631in}}%
\pgfpathlineto{\pgfqpoint{7.857444in}{1.066849in}}%
\pgfpathlineto{\pgfqpoint{7.857449in}{1.066862in}}%
\pgfpathlineto{\pgfqpoint{7.857503in}{1.067006in}}%
\pgfpathlineto{\pgfqpoint{7.857835in}{1.067949in}}%
\pgfpathlineto{\pgfqpoint{7.858522in}{1.070296in}}%
\pgfpathlineto{\pgfqpoint{7.858552in}{1.070419in}}%
\pgfpathlineto{\pgfqpoint{7.858663in}{1.070884in}}%
\pgfpathlineto{\pgfqpoint{7.858757in}{1.071310in}}%
\pgfpathlineto{\pgfqpoint{7.858833in}{1.071682in}}%
\pgfpathlineto{\pgfqpoint{7.858847in}{1.071749in}}%
\pgfpathlineto{\pgfqpoint{7.858850in}{1.071765in}}%
\pgfpathlineto{\pgfqpoint{7.858850in}{1.071769in}}%
\pgfpathlineto{\pgfqpoint{7.858947in}{1.072287in}}%
\pgfpathlineto{\pgfqpoint{7.858979in}{1.072471in}}%
\pgfpathlineto{\pgfqpoint{7.859034in}{1.072800in}}%
\pgfpathlineto{\pgfqpoint{7.859054in}{1.072930in}}%
\pgfpathlineto{\pgfqpoint{7.859123in}{1.073403in}}%
\pgfpathlineto{\pgfqpoint{7.859221in}{1.074207in}}%
\pgfpathlineto{\pgfqpoint{7.859278in}{1.074793in}}%
\pgfpathlineto{\pgfqpoint{7.859349in}{1.075869in}}%
\pgfpathlineto{\pgfqpoint{7.859376in}{1.076734in}}%
\pgfpathlineto{\pgfqpoint{7.859321in}{1.078953in}}%
\pgfpathlineto{\pgfqpoint{7.858742in}{1.083076in}}%
\pgfpathlineto{\pgfqpoint{7.858668in}{1.083409in}}%
\pgfpathlineto{\pgfqpoint{7.858648in}{1.083500in}}%
\pgfpathlineto{\pgfqpoint{7.858550in}{1.083908in}}%
\pgfpathlineto{\pgfqpoint{7.858345in}{1.084699in}}%
\pgfpathlineto{\pgfqpoint{7.857940in}{1.086053in}}%
\pgfpathlineto{\pgfqpoint{7.856472in}{1.089803in}}%
\pgfpathlineto{\pgfqpoint{7.856182in}{1.090419in}}%
\pgfpathlineto{\pgfqpoint{7.852330in}{1.096864in}}%
\pgfpathlineto{\pgfqpoint{7.851431in}{1.098088in}}%
\pgfpathlineto{\pgfqpoint{7.848714in}{1.101418in}}%
\pgfpathlineto{\pgfqpoint{7.847418in}{1.102858in}}%
\pgfpathlineto{\pgfqpoint{7.846155in}{1.104189in}}%
\pgfpathlineto{\pgfqpoint{7.845274in}{1.105081in}}%
\pgfpathlineto{\pgfqpoint{7.845232in}{1.105123in}}%
\pgfpathlineto{\pgfqpoint{7.843551in}{1.106749in}}%
\pgfpathlineto{\pgfqpoint{7.843450in}{1.106844in}}%
\pgfpathlineto{\pgfqpoint{7.843342in}{1.106946in}}%
\pgfpathlineto{\pgfqpoint{7.841698in}{1.108448in}}%
\pgfpathlineto{\pgfqpoint{7.840553in}{1.109453in}}%
\pgfpathlineto{\pgfqpoint{7.840498in}{1.109501in}}%
\pgfpathlineto{\pgfqpoint{7.840488in}{1.109510in}}%
\pgfpathlineto{\pgfqpoint{7.840097in}{1.109846in}}%
\pgfpathlineto{\pgfqpoint{7.829815in}{1.117735in}}%
\pgfpathlineto{\pgfqpoint{7.676279in}{1.182250in}}%
\pgfpathlineto{\pgfqpoint{7.674540in}{1.182798in}}%
\pgfpathlineto{\pgfqpoint{6.810494in}{2.084036in}}%
\pgfpathlineto{\pgfqpoint{6.810844in}{3.540547in}}%
\pgfpathlineto{\pgfqpoint{6.810844in}{3.992660in}}%
\pgfusepath{stroke}%
\end{pgfscope}%
\begin{pgfscope}%
\pgfsetrectcap%
\pgfsetmiterjoin%
\pgfsetlinewidth{0.803000pt}%
\definecolor{currentstroke}{rgb}{0.000000,0.000000,0.000000}%
\pgfsetstrokecolor{currentstroke}%
\pgfsetdash{}{0pt}%
\pgfpathmoveto{\pgfqpoint{6.752313in}{0.670138in}}%
\pgfpathlineto{\pgfqpoint{6.752313in}{4.600000in}}%
\pgfusepath{stroke}%
\end{pgfscope}%
\begin{pgfscope}%
\pgfsetrectcap%
\pgfsetmiterjoin%
\pgfsetlinewidth{0.803000pt}%
\definecolor{currentstroke}{rgb}{0.000000,0.000000,0.000000}%
\pgfsetstrokecolor{currentstroke}%
\pgfsetdash{}{0pt}%
\pgfpathmoveto{\pgfqpoint{7.952313in}{0.670138in}}%
\pgfpathlineto{\pgfqpoint{7.952313in}{4.600000in}}%
\pgfusepath{stroke}%
\end{pgfscope}%
\begin{pgfscope}%
\pgfsetrectcap%
\pgfsetmiterjoin%
\pgfsetlinewidth{0.803000pt}%
\definecolor{currentstroke}{rgb}{0.000000,0.000000,0.000000}%
\pgfsetstrokecolor{currentstroke}%
\pgfsetdash{}{0pt}%
\pgfpathmoveto{\pgfqpoint{6.752313in}{0.670138in}}%
\pgfpathlineto{\pgfqpoint{7.952313in}{0.670138in}}%
\pgfusepath{stroke}%
\end{pgfscope}%
\begin{pgfscope}%
\pgfsetrectcap%
\pgfsetmiterjoin%
\pgfsetlinewidth{0.803000pt}%
\definecolor{currentstroke}{rgb}{0.000000,0.000000,0.000000}%
\pgfsetstrokecolor{currentstroke}%
\pgfsetdash{}{0pt}%
\pgfpathmoveto{\pgfqpoint{6.752313in}{4.600000in}}%
\pgfpathlineto{\pgfqpoint{7.952313in}{4.600000in}}%
\pgfusepath{stroke}%
\end{pgfscope}%
\end{pgfpicture}%
\makeatother%
\endgroup%
}
  \caption{Compares the \ac{moo} algorithms.}
  \label{fig:algorithm-comparison}
\end{figure}

\FloatBarrier
