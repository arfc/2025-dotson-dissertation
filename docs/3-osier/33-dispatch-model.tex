\section{Dispatch Model}
\ac{osier} offers two models for dispatching energy. The first is a merit order
dispatch model that optimally dispatches electricity according to the marginal
cost of generation with perfect foresight. This model is formulated as a
\acf{lp} and written in \ac{pyomo}. Similarly, the second dispatch model follows
merit-order but electricity is dispatched according to a hierarchy of rules and
is myopic. This approach facilitates parallelization and reduces the overhead
cost of problem setup. However, dispatch solutions may be sub-optimal compared 
to the \ac{lp} formulation.

\subsection{Economic/Merit-order dispatch}
\label{section:merit_order}

 The economic dispatch model minimizes the generation cost subject to physical
 constraints but does not optimize capacity investments. The complete set of
 equations for the model is detailed below.
\begin{align}
    \intertext{Minimize: }
    \label{eq:dispatch-objective}
    &\left(\sum_t^T\sum_g^G \left[C_{g,t}^{fuel} + C_{g,t}^{vom}\right]x_{g,t}
    \right)+\left(\sum_t^T\sum_g^S x_{g,t}c_{g,t}\pi\right)\\
    \intertext{such that,}
    \intertext{1. The generation meets demand, less the amount of energy stored or curtailed, 
    within a user-specified tolerance (undersupply and oversupply),}
    \left[\sum_g^Gx_{g,t}-\sum_g^S c_{g,t}\right] &\geq \left(1-\text{undersupply}\right)\text{D}_t\quad \forall \quad t \in T, S, \\
    \left[\sum_g^Gx_{g,t}-\sum_g^S c_{g,t}\right] &\leq \left(1+\text{oversupply}\right)\text{D}_t \quad \forall \quad t \in T, S,
    \intertext{2. A generator's production, $x_{g}$ does not exceed its capacity at any time, $t$}
    x_{g,t} &\leq \textbf{CAP}_{g}\Delta \tau \quad \forall \quad g,t \in G,T
    \intertext{3. A generator's ramping rate is never exceeded,}
    \frac{x_{r,t} - x_{r,t-1}}{\Delta \tau} = \Delta P_{r,t} &\leq
        \rho^{up}_g\textbf{CAP}_g\Delta\tau \quad \forall \quad r,t
        \in R, T,\\
    \frac{x_{r,t} - x_{r,t-1}}{\Delta \tau} = \Delta P_{r,t} &\leq
        -\rho^{down}_g\textbf{CAP}_g\Delta\tau \quad \forall \quad r,t
        \in R, T,
    \intertext{4. Storage capacity for each storage technology is never exceeded}
    \textbf{SOC}_{s,t} &\leq \textbf{CAP}^S_{s} \quad \forall \quad s,t \in S,T,
    \intertext{5. Storage discharge cannot exceed stored energy.}
    x_{s,t} &\leq \textbf{SOC}_{s,t} \quad \forall \quad s,t \in S,T,
    \intertext{6. Storage charge rate cannot exceed unit capacity}
    c_{s,t} &\leq \textbf{CAP}_{s}\Delta \tau \quad \forall \quad s,t \in S,T,
    \intertext{where,}
    G &= \text{ the set of all generating technologies},\nonumber\\
    R &= \text{ the set of all ramping technologies}, \quad R \subset G,\nonumber\\
    S &= \text{ the set of all storage technologies}, \quad S \subset G,\nonumber\\
    T &= \text{ the set of all time periods in the model},\nonumber\\
    D_t &= \text{ the demand at each time period, \textit{t}},\nonumber\\
    \textbf{CAP}_g &= \text{ the capacity of the \textit{g-th} technology}\quad \left[MW\right],\nonumber\\
    \textbf{CAP}^S_g &= \text{ the storage capacity of the \textit{g-th} technology}\quad \left[MWh\right],\nonumber\\
    \textbf{SOC}_{s,t} &= \text{ the state of charge of the \textit{g-th} technology at time \textit{t}}\quad \left[MWh\right]\nonumber,\\
    \Delta\tau &= t_{i+1} - t_i \quad \forall \quad t_i \in T \quad \left[h\right],\nonumber\\
    x_{g,t} &= \text{ the energy produced by generator, \textit{g}, at time, \textit{t}}\quad \left[MWh\right]\nonumber,\\
    c_{s,t} &= \text{ the energy stored by storage technology, \textit{s}, at time, \textit{t}}\quad \left[MWh\right]\nonumber,\\
    \rho_g &= \text{ the up/down ramp rate for technology, \textit{g}} \quad \left[-\right]\nonumber,\\
    \pi &= \text{ A small penalty for simultaneous charging and discharging.}\nonumber
\end{align}
The second term in the objective function, Equation \ref{eq:dispatch-objective},
represents a minor penalty to prevent the unphysical behavior of simultaneous
charging and discharging from storage technologies. I used this approach because
constraining this behavior requires a binary variable that makes the problem
non-convex and therefore requires a more sophisticated solver. A small but
sufficiently large $\pi$ will always nullify the penalty term. This dispatch
model reflects the minimum physical constraints for an energy system without
considering fine-scale operational details such as frequency control. Equation
\ref{eq:dispatch-objective} assumes that the retail cost for generating
electricity is identical to the marginal cost of producing electricity. 

\subsection{Hierarchical dispatch}

Similar to the \ac{lp} formulation described in Section
\ref{section:merit_order}, the hierarchical dispatch follows a set of rules to
dispatch energy. Technologies are first sorted by their marginal cost. If two
technologies have the same marginal cost the tie is broken by comparing the 
technology efficiencies. Then, model loops through each value in the provided 
electricity demand time series. A second loop runs through the sorted list of 
technology objects and calculates the power output for each technology. There
are three broad types of technologies in \ac{osier}: Standard technologies,
ramping technologies, and storage technologies. The power output for a standard
technology is given by

\begin{align}
    x_{g,t} &= \text{max}\left(0, \text{min}\left(D_t, \textbf{CAP}_g\right)\right)\label{eq:std_tech_output}.
\end{align}

\noindent
This equation guarantees that a standard technology will produce energy between
zero and its rated capacity. In addition to Equation \ref{eq:std_tech_output},
a ``ramping'' technology will calculate the maximum and minimum attainable power
levels given its ramp rates and the problem's time resolution. The power output
is given by

\begin{align}
    x_{g,t} &= \begin{cases}
         \text{max}\left(D_t, P_{min}\right) & D_t < x_{g,t-1}\\
         \text{min}\left(D_t, P_{max}\right) & x_{g,t-1} \leq D_t \leq \textbf{CAP}_g \\
         P_{max} & x_{g,t-1} \leq \textbf{CAP}_g \leq D_t
     \end{cases}
     \intertext{where}
     P_{min} &= \text{max}\left(0, \left(x_{g,t-1}-\rho_g^{down}\Delta t\right)\right),\nonumber\\
     P_{max} &= \text{min}\left(\left(x_{g,t-1}+\rho_g^{up}\Delta t\right), \textbf{CAP}_g\right).\nonumber
\end{align}

\textcolor{red}{$\rho_g$ is not the same $rho_g$ as in the LP formulation! Maybe
choose another variable and explain its meaning... It should be the "ramp up"
which is the ramp rate times the capacity.}

Finally, a storage technology has a charging and discharging mode, depending on
whether the current demand level is positive or negative.


Figure \ref{fig:hierarchy_algorithm} illustrates the algorithm.

\begin{figure}[H]
    \centering
    \begin{tikzpicture}[node distance=1.7cm]
            \tikzstyle{every node}=[font=\small] 
            \node (1) [lbblock]{\textbf{Sort technologies by marginal cost}}; 
            \node(2) [lbblock, below of =1] {\textbf{Start dispatch loop}};
            \node(3) [lbblock, below of=2] {\textbf{Calculate power output for
            current technology}};
            \node(4) [lbblock, below of=3] {\textbf{Decrement current demand
            value \\ by power output}};
            \node(5) [lbblock, below of=4] {\textbf{Reached last technology?}};
            \node (7) [lbblock, below of=5] {\textbf{Reached the end of the
            demand time series?}}; 
            \node (8) [loblock,below of=7] {\textbf{Done}}; 
            \draw [arrow] (1) -- (2); 
            \draw [arrow] (2) -- (3);
            \draw [arrow] (3) -- (4);
            \draw [arrow] (4) -- (5);
            \draw [arrow] (5) -- node[anchor=east]{yes} (7); 
            \draw [arrow] (5) -- ([shift={(0.5cm,0cm)}]5.east) -- node[anchor=west] 
            {no} ([shift={(0.5cm,0cm)}]3.east)--(3);
            \draw [arrow] (7) -- (8); 
            \draw [arrow] (7) -- node[anchor=east]{yes} (8); 
            \draw [arrow] (7) -- ([shift={(1.15cm,0cm)}]7.east) -- node[anchor=west] 
            {no} ([shift={(1.15cm,0cm)}]2.east)--(2);
    \end{tikzpicture}
    \caption{The hierarchical dispatch flow.}
    \label{fig:hierarchy_algorithm}
\end{figure}

\subsubsection{Technology output calculation}

Each technology object 