\section{Inputs}

% \textcolor{red}{General thoughts:}

% Accurate input data are essential but curating data represents a challenging step in the
% modeling process. \ac{osier} attempts to lower this barrier by providing a
% variety of technology data from ``reliable'' sources.

% This section connects to the normative and descriptive portions of modeling. 

This section describes the input data and parameters that users must provide to run an \ac{osier} simulation. Broadly, \ac{osier} needs
technology data and some objectives to optimize. 

\subsection{Technology Data}
\ac{osier} only requires marginal costs and technology names in order to run successfully.
These data are needed to run the dispatch model. \ac{osier} also accepts operational data, such
as ramp rates and storage capacities. Additionally, \ac{osier} will understand any bespoke 
piece of data (e.g., ``popularity,'' ``technology-readiness score,'' or anything else) that
might be needed for a user-defined objective. All of these data are passed to \ac{osier}
through an \texttt{osier.Technology} object. Code listing \ref{listing:user-defined-technologies}
shows how users can create a simple technology object.

\begin{listing}[!ht]
    \caption{A basic technology object in \ac{osier}.}
    \label{listing:user-defined-technologies}
    \begin{minted}
    [ frame=lines, framesep=2mm, baselinestretch=1.2, bgcolor=LightGray,
    fontsize=\footnotesize, linenos ] {python} 
    from osier import ThermalTechnology

    fusion = ThermalTechnology(technology_name="Fusion",
                               dispatchable=True,
                               renewable=False,
                               fuel_cost=10*(GWh)**-1,
                               lifecycle_co2_rate=0.0,
                               )
    \end{minted}
    \end{listing}

\subsection{Objectives}
\label{section:osier_objectives}

There are many possible objectives to optimize. This section summarizes a few of
them and how they may be calculated in \ac{osier}. Due to \ac{pymoo}'s
structure, all objectives are minimized. Therefore, if users wish to maximize
some quantity, it must be reformatted with a reversal in sign to be an equivalent minimization objective.

\subsubsection{Per-unit-capacity}

Some quantities of interest depend on the \textit{capacity} of each technology.
For example, land use of different energy producers is often reported as a land
density $\text{km}^2/\text{MW}$. A generalized specific \textit{density} with respect to power 
may be $\text{unit}/\text{MW}$. The objective function for these quantities reads
\begin{align} 
    \mathcal{K} &= \sum_g^G \textbf{CAP}_g \kappa_g ,
    \intertext{Where}
    \kappa &= \text{the power density of the \textit{g$^{th}$} technology} \quad \left[\frac{-}{MW}\right].
\end{align}

Table \ref{tab:objectives-per-capacity} lists some example objectives that could be
minimized or maximized.

\begin{table}[h]
    \centering
    \caption{Example objectives on a per-unit-capacity basis.}
    \begin{tabular}{cc}
       \toprule
       Quantity  & Units (per MW)\\
       \midrule
        Land Use & $\left[\text{km$^2$}\right]$\\
        Employment & $\left[\text{jobs}\right]$\\
        Capital Cost & $\left[\text{\$}\right]$\\
        Fixed O\&M Cost & $\left[\text{\$ / year}\right]$\\
        \bottomrule
    \end{tabular}
    \label{tab:objectives-per-capacity}
\end{table}

\subsubsection{Per-unit-energy}

Some quantities of interest depend on the \textit{amount of energy produced} by
each technology. For example, carbon emissions only occur when a coal or natural
gas plant burns fuel. A generalized specific \textit{intensity} with respect to energy 
may be in $\text{unit}/\text{MWh}$. The objective function for these quantities reads
\begin{align}
    \mathcal{E} &= \sum_g^G \xi_g \sum_t^T x_{g,t},
    \intertext{where}
    \xi_g &= \text{the energy density of the \textit{g$^{th}$} technology}\quad
    \left[\frac{-}{MWh}\right].
\end{align}

\begin{table}[h]
    \centering
    \caption{Example objectives on a per-unit-energy basis.}
    \begin{tabular}{cc}
       \toprule
       Quantity  & Units (per MWh)\\
       \midrule
        \acs{ghg} Emissions & $\left[\text{kg}\right]$ \\
        Water Use & $\left[\text{L}\right]$\\
        ``Safety'' & $\left[\text{deaths}\right]$\\
        Fuel Cost & $\left[\text{\$}\right]$\\
        Variable O\&M Cost & $\left[\text{\$}\right]$\\
        \bottomrule
    \end{tabular}
    \label{tab:objectives-per-energy}
\end{table}

\subsubsection{Reliability and Predictability}

Reliability has many definitions in the literature and it also depends heavily
on the dispatch method. A hierarchical flow, which dispatches energy based on a
set of rules (as opposed to true cost minimization), may simply report the
fraction of hours when electricity demand was not met by the model
\cite{donado_hyres_2020,bilil_multiobjective_2014,kamjoo_multi-objective_2016,riou_multi-objective_2021}.
\acs{lp} or \acs{milp} formulations typically have an energy balance constraint requiring 
electricity demand to be satisfied at all times, or within some specified tolerance. 
Thus reliability may be translated into a cost by determining consumers'
\ac{wtp} for electricity \cite{gorman_quest_2022, najafi_value_2021}. However,
this thesis relates system reliability to price volatility and net demand
predictability. Since the price of electricity is determined by matching supply
and demand, the price will spike when supply and demand are out of phase. For
instance, geopolitics may cause the supply of natural gas to drop, increasing
the spot price of electricity. Or, more commonly, the availability of solar and
wind resources may fall unexpectedly, leading to a greater demand for backup
energy. Both of those examples are difficult to predict; otherwise, fuel
reserves could be deployed, avoiding the price shock. Thus, I propose that
measuring the predictability and volatility of an energy system is an
appropriate proxy for reliability. Additionally, minimal price volatility is
considered an aspect of energy justice \cite{sovacool_energy_2015,
van_uffelen_revisiting_2022}.

In this thesis, I measure the  predictability of hourly electricity prices and
net demand using a measure from complexity science, \ac{wpe}
\cite{fadlallah_weighted-permutation_2013}. Permutation entropy, the precursor
to \ac{wpe}, is essentially the Shannon entropy for particular sequences of
values called \textit{motifs} \cite{bandt_permutation_2002}. \ac{wpe} expands on this
concept by weighting each instance of a motif by its variance
\cite{fadlallah_weighted-permutation_2013,garland_model-free_2014}. \ac{wpe} is
defined as
\begin{align}
    H_w(m) &= -\sum_{\pi \in \Pi} P_w(\pi)\log_2(P_w(\pi))
    \intertext{where}
    \pi &= \text{a particular motif,}\nonumber\\
    P_w &= \text{the probability of a given motif, $\pi$,}\nonumber\\
    &= \frac{\mathlarger{\sum\limits_{j\leq N}} w\left(x_j^{(m, \tau)}\right)\cdot\delta\left(\phi\left(x_j^{(m, \tau)}\right), \pi_i\right)}{\mathlarger{\sum\limits_{j\leq N}} w\left(x_j^{(m, \tau)}\right)}
    \intertext{and}
    w\left(x_j^{(m, \tau)}\right) &= \text{the weight of a particular vector}\nonumber\\
      &= \frac{1}{m}{\sum_{j}^m} \left(x_j^{(m,\tau)} - \Bar{x}\right)^2,\\\
     \phi(\cdot) &= \text{the ordinal pattern of a vector,}\nonumber\\
     \delta(\cdot) &= \text{Kronecker delta,}\nonumber\\
     m &= \text{the embedding dimension,}\nonumber\\
     \tau &= \text{the time delay}\nonumber.
\end{align}

There are other reliability metrics in the literature, frequently employing some
variation on the ``spread'' of data through standard deviation  or mean squared
error \cite{galvani_optimal_2021, galvani_unified_2014,
delsole_predictability_2004}. However, these metrics are unbounded and do not
contain any information about the underlying dynamics that produce a certain
distribution. Whereas \ac{wpe} can indicate a theoretical ceiling on
predictability \cite{garland_model-free_2014}. Importantly, \ac{wpe} works for
systems where the underlying dynamics are unknown. The Hurst exponent is another
measure of predictability, but it too has drawbacks, such as computational
expense and a stationarity requirement \cite{mesa_hurst_1993,
chandrasekaran_investigation_2019}. This thesis uses the \ac{wpe} implementation
I contributed to the open source package \texttt{PyEntropy}
\cite{donets_pyentropy_2023}.

\subsubsection{User-defined Objectives}

A key feature of \ac{osier} is the ability for users to define their own
objectives relatively easily. This feature is required
because modelers cannot know \textit{a priori} every objective that users might
be interested in optimizing. While \ac{osier} ships with some standard objective
functions, allowing users to create their own objectives makes every model
bespoke. With requisite user-supplied data \textit{any quantitative metric may be used as an objective in
\ac{osier}.} Every objective function has at least two arguments, the list of
technologies used in the model and the solved dispatch model. Users will never
have to pass these arguments manually since \ac{osier} will automatically call
the function during a simulation. One example of a user-defined objective might
be technology readiness. This objective is independent from the energy produced
and could be weighted by the capacity but is not a per-unit-capacity objective.
The values of the readiness parameter must be passed to each \texttt{Technology}
object, which can be accessed at run-time. Code listing
\ref{listing:user-defined-objective} shows the basic approach to creating a new
objective. 

\begin{listing}[!ht]
\caption{The fundamental way to create a novel objective in \ac{osier}.}
\label{listing:user-defined-objective}
\begin{minted}
[ frame=lines, framesep=2mm, baselinestretch=1.2, bgcolor=LightGray,
fontsize=\footnotesize, linenos ] {python} 

nuclear.readiness = 9
fusion.readiness = 3

technology_list = [nuclear, fusion]

def osier_objective(technology_list, solved_dispatch_model): 
    """ 
        Calculate the capacity-weighted technology readiness 
        score for this energy mix. 
    """

    total_capacity = np.array([t.capacity for t in technology_list]).sum()
    
    objective_value = np.array([t.readiness*t.capacity 
                                for t in technology_list]).sum()

    return objective_value / total_capacity
\end{minted}
\end{listing}
\noindent
Importantly, because all technologies in \ac{osier} are Python objects, users
can add attributes at will. Such as the technology readiness level as shown in
Code listing \ref{listing:user-defined-objective}. 

\subsection{Constraints}
\label{section:constraints}

Besides the physical constraints defined in Section \ref{section:dispatch_model},
\ac{osier} does not have any default constraints. This is because each
additional constraint corresponds to an additional assumption and will affect
the tradeoff analysis that makes \ac{moo} so powerful. However, there are some
circumstances where the optimal solutions are still
infeasible in practice. For instance, if a community wants to determine the best
energy mix according to their unique objectives, this community might not have
the budget for even a least-cost solution because the capital requirements are
too high. Therefore, they must constrain the capital cost for their modeling
problem. Thus, \ac{osier} enables the following:
\begin{enumerate}
    \item Users may define their own constraints.
    \item Any objective function may be transformed into a constraint.
\end{enumerate}
This feature makes \ac{osier} unique among \acp{esom}.
Single-objective \acp{esom} can never account for unique situations such as the
one suggested above, nor any other bespoke considerations. In the case above,
the capital cost may constrain the problem while still minimizing the total
cost. The solutions under these conditions will have a higher total cost but
could be achievable in the near term due to meeting capital cost requirements.