\section{Limitations of \ac{osier}}

Although \ac{osier} is the first \ac{esom} to allow arbitrary multi-objective
optimization it still has some limitations. This section describes some of the limitations in \ac{osier}'s modeling details
and scope.

\begin{enumerate}
    \item \ac{osier} has a limited time horizon and is not currently set up to
    optimize multiple decades into the future.
    \item \ac{osier} does not model interactions between the environment and the
    energy system. Therefore, temperature feedbacks and geoengineering
    technologies cannot be modeled adequately. Although, \ac{osier} could handle
    a negative emissions technology.
    \item Currently, \ac{osier}'s dispatch models can only model one energy carrier. That is, technologies
    could produce heat (measured in MW$_{th}$) or electricity. But there is no method
    to convert between the two endogenously.
    \item There is no concept of transmission or distribution within \ac{osier}'s dispatch
    models. 
    \item Demand for energy is treated as perfectly inelastic. Therefore, \ac{osier} cannot model 
    optional or deferrable loads.
    \item Ancillary services, capacity auctions, and other markets are not included in \ac{osier}. The dispatch
    model is strictly day-ahead or real time for the \ac{lp} and hierarchical formulations respectively.
\end{enumerate}

Many of these limitations are due to the underlying dispatch model. These limitations could be addressed 
by coupling \ac{osier}'s genetic algorithm to an external dispatch model, such as \ac{pypsa} or \ac{temoa}.
Other features, such as geoengineering and environmental interactions, are much more challenging to enclose.
Avenues of future work are detailed in Section \ref{section:future-work}.

% This responds to some of Cliff's comments on \ac{osier}.