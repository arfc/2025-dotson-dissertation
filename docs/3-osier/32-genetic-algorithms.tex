\section{Genetic Algorithms}
\label{section:genetic-algorithms}

Rather than rely on \ac{lp} to model future capacity requirements, in this
thesis, \acp{ga} assume the role of investment optimizer. \acp{ga} share a
fundamental algorithmic structure, which is \cite{blank_pymoo_2020}
\begin{enumerate}
    \item \textbf{Initialize} a starting population of $N_p$ individuals, where
    each individual has a set of ``genes'' that are randomly chosen from the
    bounds of the decision variables.
    \item Each individual in the population is \textbf{evaluated} for
    ``fitness.'' 
    \item The \textbf{fittest}, $N_f$ individuals ``survive'' and persist in the
    next generation.
    \item A ``selection'' operator \textbf{chooses} among the surviving
    individuals to mate.
    \item The parents are \textbf{combined} using a ``crossover'' operator,
    thereby filling the remaining $N_p - N_f$ individuals for the next
    generation.
    \item The offspring are finally \textbf{mutated} with some probability,
    $\mu$, to improve genetic diversity.
\end{enumerate}
\noindent
Figure \ref{fig:genetic-alg} illustrates the flow of these steps applied to an
energy systems model.

\begin{figure}[ht]
        \centering
        \begin{tikzpicture}[node distance=1.7cm]
                \tikzstyle{every node}=[font=\small] \node (1) [lbblock]
                {\textbf{Create initial population\\ of capacity sets}}; \node
                (2) [lbblock, below of=1] {\textbf{Evaluate dispatch model and
                calculate objectives}}; \node (3) [lbblock, below of=2]
                {\textbf{Survival}}; \node (4) [lbblock, below of=3]
                {\textbf{Selection}}; \node (5) [lbblock, below of=4]
                {\textbf{Crossover}}; \node (6) [lbblock, below of=5]
                {\textbf{Mutation}}; \node (7) [lbblock, below of=6] {\textbf{Is
                the termination \\ criteria satisfied?}}; \node (8) [loblock,
                below of=7] {\textbf{Done}}; \draw [arrow] (1) -- (2); \draw
                [arrow] (2) -- (3); \draw [arrow] (3) -- (4); \draw [arrow] (4)
                -- (5); \draw [arrow] (5) -- (6); \draw [arrow] (6) -- (7);
                \draw [arrow] (7) -- (8); \draw [arrow] (7) -- node[anchor=east]
                {yes} (8); \draw [arrow] (7) -- ([shift={(0.5cm,0cm)}]7.east)--
                node[anchor=west] {no} ([shift={(0.5cm,0cm)}]2.east)--(2);
        \end{tikzpicture}
        \caption{The basic flow of the \ac{ga} used in this thesis.}
        \label{fig:genetic-alg}
\end{figure}

\subsection{Specific \Aclp{ga}} The variety of \acp{ga} comes from different
types of operators being applied to the selection, crossover, and mutation
steps. Section \ref{section:moo-in-energy} showed that \ac{nsga2} is a popular
genetic algorithm choice. However, this algorithm performs poorly with greater
than three objectives \cite{deb_fast_2002, seada_unified_2016}. This thesis
uses a more modern algorithm, \ac{unsga3}. \ac{unsga3} builds on its
predecessors \ac{nsga2} and \ac{nsga3} by unifying efficient solutions of mono-,
multi-, and many-objective problems in a single algorithm.


\ac{nsga2} improves on the basic \ac{ga} by introducing a more sophisticated
mating and selection algorithms. Instead of random selection, the individuals
are sorted by rank (i.e. fitness) and crowding distance in binary tournament
mating selection. The crowding distance is simply the Manhattan distance between
individuals. A greater crowding distance is desirable to preserve diversity and
since the extreme points are maximally diverse they should always persist and
are therefore assigned a crowding distance of infinity \cite{deb_fast_2002}.

The successor to \ac{nsga2}, \ac{nsga3}, enhances the many-objective
capabilities of the former by introducing reference directions. Reference
directions are used for initialization and the survival steps. In addition to
fitness, individuals are chosen based on their proximity to a reference line,
thus ensuring population diversity which greatly important for many-objective
problems. Since diversity is handled by reference directions, individuals are
selected randomly for mating. References directions are rays passing through
uniformly spaced points on the unit simplex \cite{seada_unified_2016,
blank_generating_2021}. In this thesis, I use the Riesz s- Energy method
described by Blank et al. to calculate these points for a problem with an
arbitrary number of objectives \cite{blank_generating_2021}. Figure
\ref{fig:ref-dirs} illustrates a set of initialized reference directions used
in such methods.

\begin{figure}[h]
  \centering
  \resizebox{0.6\columnwidth}{!}{%% Creator: Matplotlib, PGF backend
%%
%% To include the figure in your LaTeX document, write
%%   \input{<filename>.pgf}
%%
%% Make sure the required packages are loaded in your preamble
%%   \usepackage{pgf}
%%
%% Also ensure that all the required font packages are loaded; for instance,
%% the lmodern package is sometimes necessary when using math font.
%%   \usepackage{lmodern}
%%
%% Figures using additional raster images can only be included by \input if
%% they are in the same directory as the main LaTeX file. For loading figures
%% from other directories you can use the `import` package
%%   \usepackage{import}
%%
%% and then include the figures with
%%   \import{<path to file>}{<filename>.pgf}
%%
%% Matplotlib used the following preamble
%%   \def\mathdefault#1{#1}
%%   \everymath=\expandafter{\the\everymath\displaystyle}
%%   \IfFileExists{scrextend.sty}{
%%     \usepackage[fontsize=10.000000pt]{scrextend}
%%   }{
%%     \renewcommand{\normalsize}{\fontsize{10.000000}{12.000000}\selectfont}
%%     \normalsize
%%   }
%%   
%%   \makeatletter\@ifpackageloaded{underscore}{}{\usepackage[strings]{underscore}}\makeatother
%%
\begingroup%
\makeatletter%
\begin{pgfpicture}%
\pgfpathrectangle{\pgfpointorigin}{\pgfqpoint{6.454833in}{6.506946in}}%
\pgfusepath{use as bounding box, clip}%
\begin{pgfscope}%
\pgfsetbuttcap%
\pgfsetmiterjoin%
\definecolor{currentfill}{rgb}{1.000000,1.000000,1.000000}%
\pgfsetfillcolor{currentfill}%
\pgfsetlinewidth{0.000000pt}%
\definecolor{currentstroke}{rgb}{0.000000,0.000000,0.000000}%
\pgfsetstrokecolor{currentstroke}%
\pgfsetdash{}{0pt}%
\pgfpathmoveto{\pgfqpoint{0.000000in}{0.000000in}}%
\pgfpathlineto{\pgfqpoint{6.454833in}{0.000000in}}%
\pgfpathlineto{\pgfqpoint{6.454833in}{6.506946in}}%
\pgfpathlineto{\pgfqpoint{0.000000in}{6.506946in}}%
\pgfpathlineto{\pgfqpoint{0.000000in}{0.000000in}}%
\pgfpathclose%
\pgfusepath{fill}%
\end{pgfscope}%
\begin{pgfscope}%
\pgfsetbuttcap%
\pgfsetmiterjoin%
\definecolor{currentfill}{rgb}{1.000000,1.000000,1.000000}%
\pgfsetfillcolor{currentfill}%
\pgfsetlinewidth{0.000000pt}%
\definecolor{currentstroke}{rgb}{0.000000,0.000000,0.000000}%
\pgfsetstrokecolor{currentstroke}%
\pgfsetstrokeopacity{0.000000}%
\pgfsetdash{}{0pt}%
\pgfpathmoveto{\pgfqpoint{0.194833in}{0.246946in}}%
\pgfpathlineto{\pgfqpoint{6.354833in}{0.246946in}}%
\pgfpathlineto{\pgfqpoint{6.354833in}{6.406946in}}%
\pgfpathlineto{\pgfqpoint{0.194833in}{6.406946in}}%
\pgfpathlineto{\pgfqpoint{0.194833in}{0.246946in}}%
\pgfpathclose%
\pgfusepath{fill}%
\end{pgfscope}%
\begin{pgfscope}%
\pgfsetbuttcap%
\pgfsetmiterjoin%
\definecolor{currentfill}{rgb}{0.950000,0.950000,0.950000}%
\pgfsetfillcolor{currentfill}%
\pgfsetfillopacity{0.500000}%
\pgfsetlinewidth{1.003750pt}%
\definecolor{currentstroke}{rgb}{0.950000,0.950000,0.950000}%
\pgfsetstrokecolor{currentstroke}%
\pgfsetstrokeopacity{0.500000}%
\pgfsetdash{}{0pt}%
\pgfpathmoveto{\pgfqpoint{3.358076in}{4.263351in}}%
\pgfpathlineto{\pgfqpoint{6.075248in}{2.391249in}}%
\pgfpathlineto{\pgfqpoint{6.252387in}{4.495556in}}%
\pgfpathlineto{\pgfqpoint{3.358076in}{6.360787in}}%
\pgfusepath{stroke,fill}%
\end{pgfscope}%
\begin{pgfscope}%
\pgfsetbuttcap%
\pgfsetmiterjoin%
\definecolor{currentfill}{rgb}{0.900000,0.900000,0.900000}%
\pgfsetfillcolor{currentfill}%
\pgfsetfillopacity{0.500000}%
\pgfsetlinewidth{1.003750pt}%
\definecolor{currentstroke}{rgb}{0.900000,0.900000,0.900000}%
\pgfsetstrokecolor{currentstroke}%
\pgfsetstrokeopacity{0.500000}%
\pgfsetdash{}{0pt}%
\pgfpathmoveto{\pgfqpoint{3.358076in}{4.263351in}}%
\pgfpathlineto{\pgfqpoint{0.640904in}{2.391249in}}%
\pgfpathlineto{\pgfqpoint{0.463765in}{4.495556in}}%
\pgfpathlineto{\pgfqpoint{3.358076in}{6.360787in}}%
\pgfusepath{stroke,fill}%
\end{pgfscope}%
\begin{pgfscope}%
\pgfsetbuttcap%
\pgfsetmiterjoin%
\definecolor{currentfill}{rgb}{0.925000,0.925000,0.925000}%
\pgfsetfillcolor{currentfill}%
\pgfsetfillopacity{0.500000}%
\pgfsetlinewidth{1.003750pt}%
\definecolor{currentstroke}{rgb}{0.925000,0.925000,0.925000}%
\pgfsetstrokecolor{currentstroke}%
\pgfsetstrokeopacity{0.500000}%
\pgfsetdash{}{0pt}%
\pgfpathmoveto{\pgfqpoint{3.358076in}{4.263351in}}%
\pgfpathlineto{\pgfqpoint{0.640904in}{2.391249in}}%
\pgfpathlineto{\pgfqpoint{3.358076in}{0.289869in}}%
\pgfpathlineto{\pgfqpoint{6.075248in}{2.391249in}}%
\pgfusepath{stroke,fill}%
\end{pgfscope}%
\begin{pgfscope}%
\pgfsetbuttcap%
\pgfsetroundjoin%
\pgfsetlinewidth{0.803000pt}%
\definecolor{currentstroke}{rgb}{0.690196,0.690196,0.690196}%
\pgfsetstrokecolor{currentstroke}%
\pgfsetdash{}{0pt}%
\pgfpathmoveto{\pgfqpoint{5.911714in}{2.264777in}}%
\pgfpathlineto{\pgfqpoint{3.194029in}{4.150324in}}%
\pgfpathlineto{\pgfqpoint{3.183916in}{6.248550in}}%
\pgfusepath{stroke}%
\end{pgfscope}%
\begin{pgfscope}%
\pgfsetbuttcap%
\pgfsetroundjoin%
\pgfsetlinewidth{0.803000pt}%
\definecolor{currentstroke}{rgb}{0.690196,0.690196,0.690196}%
\pgfsetstrokecolor{currentstroke}%
\pgfsetdash{}{0pt}%
\pgfpathmoveto{\pgfqpoint{5.456913in}{1.913047in}}%
\pgfpathlineto{\pgfqpoint{2.738146in}{3.836226in}}%
\pgfpathlineto{\pgfqpoint{2.699539in}{5.936395in}}%
\pgfusepath{stroke}%
\end{pgfscope}%
\begin{pgfscope}%
\pgfsetbuttcap%
\pgfsetroundjoin%
\pgfsetlinewidth{0.803000pt}%
\definecolor{currentstroke}{rgb}{0.690196,0.690196,0.690196}%
\pgfsetstrokecolor{currentstroke}%
\pgfsetdash{}{0pt}%
\pgfpathmoveto{\pgfqpoint{4.992727in}{1.554059in}}%
\pgfpathlineto{\pgfqpoint{2.273382in}{3.516009in}}%
\pgfpathlineto{\pgfqpoint{2.205132in}{5.617775in}}%
\pgfusepath{stroke}%
\end{pgfscope}%
\begin{pgfscope}%
\pgfsetbuttcap%
\pgfsetroundjoin%
\pgfsetlinewidth{0.803000pt}%
\definecolor{currentstroke}{rgb}{0.690196,0.690196,0.690196}%
\pgfsetstrokecolor{currentstroke}%
\pgfsetdash{}{0pt}%
\pgfpathmoveto{\pgfqpoint{4.518863in}{1.187587in}}%
\pgfpathlineto{\pgfqpoint{1.799474in}{3.189491in}}%
\pgfpathlineto{\pgfqpoint{1.700378in}{5.292488in}}%
\pgfusepath{stroke}%
\end{pgfscope}%
\begin{pgfscope}%
\pgfsetbuttcap%
\pgfsetroundjoin%
\pgfsetlinewidth{0.803000pt}%
\definecolor{currentstroke}{rgb}{0.690196,0.690196,0.690196}%
\pgfsetstrokecolor{currentstroke}%
\pgfsetdash{}{0pt}%
\pgfpathmoveto{\pgfqpoint{4.035014in}{0.813393in}}%
\pgfpathlineto{\pgfqpoint{1.316148in}{2.856485in}}%
\pgfpathlineto{\pgfqpoint{1.184951in}{4.960322in}}%
\pgfusepath{stroke}%
\end{pgfscope}%
\begin{pgfscope}%
\pgfsetbuttcap%
\pgfsetroundjoin%
\pgfsetlinewidth{0.803000pt}%
\definecolor{currentstroke}{rgb}{0.690196,0.690196,0.690196}%
\pgfsetstrokecolor{currentstroke}%
\pgfsetdash{}{0pt}%
\pgfpathmoveto{\pgfqpoint{3.540863in}{0.431231in}}%
\pgfpathlineto{\pgfqpoint{0.823123in}{2.516796in}}%
\pgfpathlineto{\pgfqpoint{0.658507in}{4.621057in}}%
\pgfusepath{stroke}%
\end{pgfscope}%
\begin{pgfscope}%
\pgfsetbuttcap%
\pgfsetroundjoin%
\pgfsetlinewidth{0.803000pt}%
\definecolor{currentstroke}{rgb}{0.690196,0.690196,0.690196}%
\pgfsetstrokecolor{currentstroke}%
\pgfsetdash{}{0pt}%
\pgfpathmoveto{\pgfqpoint{3.532236in}{6.248550in}}%
\pgfpathlineto{\pgfqpoint{3.522123in}{4.150324in}}%
\pgfpathlineto{\pgfqpoint{0.804438in}{2.264777in}}%
\pgfusepath{stroke}%
\end{pgfscope}%
\begin{pgfscope}%
\pgfsetbuttcap%
\pgfsetroundjoin%
\pgfsetlinewidth{0.803000pt}%
\definecolor{currentstroke}{rgb}{0.690196,0.690196,0.690196}%
\pgfsetstrokecolor{currentstroke}%
\pgfsetdash{}{0pt}%
\pgfpathmoveto{\pgfqpoint{4.016613in}{5.936395in}}%
\pgfpathlineto{\pgfqpoint{3.978006in}{3.836226in}}%
\pgfpathlineto{\pgfqpoint{1.259239in}{1.913047in}}%
\pgfusepath{stroke}%
\end{pgfscope}%
\begin{pgfscope}%
\pgfsetbuttcap%
\pgfsetroundjoin%
\pgfsetlinewidth{0.803000pt}%
\definecolor{currentstroke}{rgb}{0.690196,0.690196,0.690196}%
\pgfsetstrokecolor{currentstroke}%
\pgfsetdash{}{0pt}%
\pgfpathmoveto{\pgfqpoint{4.511020in}{5.617775in}}%
\pgfpathlineto{\pgfqpoint{4.442770in}{3.516009in}}%
\pgfpathlineto{\pgfqpoint{1.723425in}{1.554059in}}%
\pgfusepath{stroke}%
\end{pgfscope}%
\begin{pgfscope}%
\pgfsetbuttcap%
\pgfsetroundjoin%
\pgfsetlinewidth{0.803000pt}%
\definecolor{currentstroke}{rgb}{0.690196,0.690196,0.690196}%
\pgfsetstrokecolor{currentstroke}%
\pgfsetdash{}{0pt}%
\pgfpathmoveto{\pgfqpoint{5.015774in}{5.292488in}}%
\pgfpathlineto{\pgfqpoint{4.916678in}{3.189491in}}%
\pgfpathlineto{\pgfqpoint{2.197289in}{1.187587in}}%
\pgfusepath{stroke}%
\end{pgfscope}%
\begin{pgfscope}%
\pgfsetbuttcap%
\pgfsetroundjoin%
\pgfsetlinewidth{0.803000pt}%
\definecolor{currentstroke}{rgb}{0.690196,0.690196,0.690196}%
\pgfsetstrokecolor{currentstroke}%
\pgfsetdash{}{0pt}%
\pgfpathmoveto{\pgfqpoint{5.531201in}{4.960322in}}%
\pgfpathlineto{\pgfqpoint{5.400004in}{2.856485in}}%
\pgfpathlineto{\pgfqpoint{2.681138in}{0.813393in}}%
\pgfusepath{stroke}%
\end{pgfscope}%
\begin{pgfscope}%
\pgfsetbuttcap%
\pgfsetroundjoin%
\pgfsetlinewidth{0.803000pt}%
\definecolor{currentstroke}{rgb}{0.690196,0.690196,0.690196}%
\pgfsetstrokecolor{currentstroke}%
\pgfsetdash{}{0pt}%
\pgfpathmoveto{\pgfqpoint{6.057645in}{4.621057in}}%
\pgfpathlineto{\pgfqpoint{5.893029in}{2.516796in}}%
\pgfpathlineto{\pgfqpoint{3.175289in}{0.431231in}}%
\pgfusepath{stroke}%
\end{pgfscope}%
\begin{pgfscope}%
\pgfsetbuttcap%
\pgfsetroundjoin%
\pgfsetlinewidth{0.803000pt}%
\definecolor{currentstroke}{rgb}{0.690196,0.690196,0.690196}%
\pgfsetstrokecolor{currentstroke}%
\pgfsetdash{}{0pt}%
\pgfpathmoveto{\pgfqpoint{0.630280in}{2.517456in}}%
\pgfpathlineto{\pgfqpoint{3.358076in}{4.389566in}}%
\pgfpathlineto{\pgfqpoint{6.085872in}{2.517456in}}%
\pgfusepath{stroke}%
\end{pgfscope}%
\begin{pgfscope}%
\pgfsetbuttcap%
\pgfsetroundjoin%
\pgfsetlinewidth{0.803000pt}%
\definecolor{currentstroke}{rgb}{0.690196,0.690196,0.690196}%
\pgfsetstrokecolor{currentstroke}%
\pgfsetdash{}{0pt}%
\pgfpathmoveto{\pgfqpoint{0.600709in}{2.868742in}}%
\pgfpathlineto{\pgfqpoint{3.358076in}{4.740592in}}%
\pgfpathlineto{\pgfqpoint{6.115443in}{2.868742in}}%
\pgfusepath{stroke}%
\end{pgfscope}%
\begin{pgfscope}%
\pgfsetbuttcap%
\pgfsetroundjoin%
\pgfsetlinewidth{0.803000pt}%
\definecolor{currentstroke}{rgb}{0.690196,0.690196,0.690196}%
\pgfsetstrokecolor{currentstroke}%
\pgfsetdash{}{0pt}%
\pgfpathmoveto{\pgfqpoint{0.570490in}{3.227729in}}%
\pgfpathlineto{\pgfqpoint{3.358076in}{5.098883in}}%
\pgfpathlineto{\pgfqpoint{6.145662in}{3.227729in}}%
\pgfusepath{stroke}%
\end{pgfscope}%
\begin{pgfscope}%
\pgfsetbuttcap%
\pgfsetroundjoin%
\pgfsetlinewidth{0.803000pt}%
\definecolor{currentstroke}{rgb}{0.690196,0.690196,0.690196}%
\pgfsetstrokecolor{currentstroke}%
\pgfsetdash{}{0pt}%
\pgfpathmoveto{\pgfqpoint{0.539601in}{3.594671in}}%
\pgfpathlineto{\pgfqpoint{3.358076in}{5.464666in}}%
\pgfpathlineto{\pgfqpoint{6.176551in}{3.594671in}}%
\pgfusepath{stroke}%
\end{pgfscope}%
\begin{pgfscope}%
\pgfsetbuttcap%
\pgfsetroundjoin%
\pgfsetlinewidth{0.803000pt}%
\definecolor{currentstroke}{rgb}{0.690196,0.690196,0.690196}%
\pgfsetstrokecolor{currentstroke}%
\pgfsetdash{}{0pt}%
\pgfpathmoveto{\pgfqpoint{0.508019in}{3.969836in}}%
\pgfpathlineto{\pgfqpoint{3.358076in}{5.838179in}}%
\pgfpathlineto{\pgfqpoint{6.208133in}{3.969836in}}%
\pgfusepath{stroke}%
\end{pgfscope}%
\begin{pgfscope}%
\pgfsetbuttcap%
\pgfsetroundjoin%
\pgfsetlinewidth{0.803000pt}%
\definecolor{currentstroke}{rgb}{0.690196,0.690196,0.690196}%
\pgfsetstrokecolor{currentstroke}%
\pgfsetdash{}{0pt}%
\pgfpathmoveto{\pgfqpoint{0.475722in}{4.353505in}}%
\pgfpathlineto{\pgfqpoint{3.358076in}{6.219668in}}%
\pgfpathlineto{\pgfqpoint{6.240430in}{4.353505in}}%
\pgfusepath{stroke}%
\end{pgfscope}%
\begin{pgfscope}%
\pgfsetrectcap%
\pgfsetroundjoin%
\pgfsetlinewidth{0.803000pt}%
\definecolor{currentstroke}{rgb}{0.000000,0.000000,0.000000}%
\pgfsetstrokecolor{currentstroke}%
\pgfsetdash{}{0pt}%
\pgfpathmoveto{\pgfqpoint{6.075248in}{2.391249in}}%
\pgfpathlineto{\pgfqpoint{3.358076in}{0.289869in}}%
\pgfusepath{stroke}%
\end{pgfscope}%
\begin{pgfscope}%
\pgfsetrectcap%
\pgfsetroundjoin%
\pgfsetlinewidth{0.803000pt}%
\definecolor{currentstroke}{rgb}{0.000000,0.000000,0.000000}%
\pgfsetstrokecolor{currentstroke}%
\pgfsetdash{}{0pt}%
\pgfpathmoveto{\pgfqpoint{5.888724in}{2.280728in}}%
\pgfpathlineto{\pgfqpoint{5.957758in}{2.232831in}}%
\pgfusepath{stroke}%
\end{pgfscope}%
\begin{pgfscope}%
\definecolor{textcolor}{rgb}{0.000000,0.000000,0.000000}%
\pgfsetstrokecolor{textcolor}%
\pgfsetfillcolor{textcolor}%
\pgftext[x=6.047013in,y=2.067182in,,top]{\color{textcolor}{\rmfamily\fontsize{10.000000}{12.000000}\selectfont\catcode`\^=\active\def^{\ifmmode\sp\else\^{}\fi}\catcode`\%=\active\def%{\%}$\mathdefault{0.0}$}}%
\end{pgfscope}%
\begin{pgfscope}%
\pgfsetrectcap%
\pgfsetroundjoin%
\pgfsetlinewidth{0.803000pt}%
\definecolor{currentstroke}{rgb}{0.000000,0.000000,0.000000}%
\pgfsetstrokecolor{currentstroke}%
\pgfsetdash{}{0pt}%
\pgfpathmoveto{\pgfqpoint{5.433901in}{1.929325in}}%
\pgfpathlineto{\pgfqpoint{5.503001in}{1.880446in}}%
\pgfusepath{stroke}%
\end{pgfscope}%
\begin{pgfscope}%
\definecolor{textcolor}{rgb}{0.000000,0.000000,0.000000}%
\pgfsetstrokecolor{textcolor}%
\pgfsetfillcolor{textcolor}%
\pgftext[x=5.593673in,y=1.713497in,,top]{\color{textcolor}{\rmfamily\fontsize{10.000000}{12.000000}\selectfont\catcode`\^=\active\def^{\ifmmode\sp\else\^{}\fi}\catcode`\%=\active\def%{\%}$\mathdefault{0.2}$}}%
\end{pgfscope}%
\begin{pgfscope}%
\pgfsetrectcap%
\pgfsetroundjoin%
\pgfsetlinewidth{0.803000pt}%
\definecolor{currentstroke}{rgb}{0.000000,0.000000,0.000000}%
\pgfsetstrokecolor{currentstroke}%
\pgfsetdash{}{0pt}%
\pgfpathmoveto{\pgfqpoint{4.969697in}{1.570675in}}%
\pgfpathlineto{\pgfqpoint{5.038852in}{1.520781in}}%
\pgfusepath{stroke}%
\end{pgfscope}%
\begin{pgfscope}%
\definecolor{textcolor}{rgb}{0.000000,0.000000,0.000000}%
\pgfsetstrokecolor{textcolor}%
\pgfsetfillcolor{textcolor}%
\pgftext[x=5.130981in,y=1.352515in,,top]{\color{textcolor}{\rmfamily\fontsize{10.000000}{12.000000}\selectfont\catcode`\^=\active\def^{\ifmmode\sp\else\^{}\fi}\catcode`\%=\active\def%{\%}$\mathdefault{0.4}$}}%
\end{pgfscope}%
\begin{pgfscope}%
\pgfsetrectcap%
\pgfsetroundjoin%
\pgfsetlinewidth{0.803000pt}%
\definecolor{currentstroke}{rgb}{0.000000,0.000000,0.000000}%
\pgfsetstrokecolor{currentstroke}%
\pgfsetdash{}{0pt}%
\pgfpathmoveto{\pgfqpoint{4.495819in}{1.204551in}}%
\pgfpathlineto{\pgfqpoint{4.565016in}{1.153611in}}%
\pgfusepath{stroke}%
\end{pgfscope}%
\begin{pgfscope}%
\definecolor{textcolor}{rgb}{0.000000,0.000000,0.000000}%
\pgfsetstrokecolor{textcolor}%
\pgfsetfillcolor{textcolor}%
\pgftext[x=4.658643in,y=0.984008in,,top]{\color{textcolor}{\rmfamily\fontsize{10.000000}{12.000000}\selectfont\catcode`\^=\active\def^{\ifmmode\sp\else\^{}\fi}\catcode`\%=\active\def%{\%}$\mathdefault{0.6}$}}%
\end{pgfscope}%
\begin{pgfscope}%
\pgfsetrectcap%
\pgfsetroundjoin%
\pgfsetlinewidth{0.803000pt}%
\definecolor{currentstroke}{rgb}{0.000000,0.000000,0.000000}%
\pgfsetstrokecolor{currentstroke}%
\pgfsetdash{}{0pt}%
\pgfpathmoveto{\pgfqpoint{4.011962in}{0.830716in}}%
\pgfpathlineto{\pgfqpoint{4.081186in}{0.778697in}}%
\pgfusepath{stroke}%
\end{pgfscope}%
\begin{pgfscope}%
\definecolor{textcolor}{rgb}{0.000000,0.000000,0.000000}%
\pgfsetstrokecolor{textcolor}%
\pgfsetfillcolor{textcolor}%
\pgftext[x=4.176355in,y=0.607737in,,top]{\color{textcolor}{\rmfamily\fontsize{10.000000}{12.000000}\selectfont\catcode`\^=\active\def^{\ifmmode\sp\else\^{}\fi}\catcode`\%=\active\def%{\%}$\mathdefault{0.8}$}}%
\end{pgfscope}%
\begin{pgfscope}%
\pgfsetrectcap%
\pgfsetroundjoin%
\pgfsetlinewidth{0.803000pt}%
\definecolor{currentstroke}{rgb}{0.000000,0.000000,0.000000}%
\pgfsetstrokecolor{currentstroke}%
\pgfsetdash{}{0pt}%
\pgfpathmoveto{\pgfqpoint{3.517806in}{0.448925in}}%
\pgfpathlineto{\pgfqpoint{3.587044in}{0.395792in}}%
\pgfusepath{stroke}%
\end{pgfscope}%
\begin{pgfscope}%
\definecolor{textcolor}{rgb}{0.000000,0.000000,0.000000}%
\pgfsetstrokecolor{textcolor}%
\pgfsetfillcolor{textcolor}%
\pgftext[x=3.683800in,y=0.223457in,,top]{\color{textcolor}{\rmfamily\fontsize{10.000000}{12.000000}\selectfont\catcode`\^=\active\def^{\ifmmode\sp\else\^{}\fi}\catcode`\%=\active\def%{\%}$\mathdefault{1.0}$}}%
\end{pgfscope}%
\begin{pgfscope}%
\definecolor{textcolor}{rgb}{0.000000,0.000000,0.000000}%
\pgfsetstrokecolor{textcolor}%
\pgfsetfillcolor{textcolor}%
\pgftext[x=5.059162in,y=0.931525in,,,rotate=37.717305]{\color{textcolor}{\rmfamily\fontsize{14.000000}{16.800000}\selectfont\catcode`\^=\active\def^{\ifmmode\sp\else\^{}\fi}\catcode`\%=\active\def%{\%}$f_1$}}%
\end{pgfscope}%
\begin{pgfscope}%
\pgfsetrectcap%
\pgfsetroundjoin%
\pgfsetlinewidth{0.803000pt}%
\definecolor{currentstroke}{rgb}{0.000000,0.000000,0.000000}%
\pgfsetstrokecolor{currentstroke}%
\pgfsetdash{}{0pt}%
\pgfpathmoveto{\pgfqpoint{0.640904in}{2.391249in}}%
\pgfpathlineto{\pgfqpoint{3.358076in}{0.289869in}}%
\pgfusepath{stroke}%
\end{pgfscope}%
\begin{pgfscope}%
\pgfsetrectcap%
\pgfsetroundjoin%
\pgfsetlinewidth{0.803000pt}%
\definecolor{currentstroke}{rgb}{0.000000,0.000000,0.000000}%
\pgfsetstrokecolor{currentstroke}%
\pgfsetdash{}{0pt}%
\pgfpathmoveto{\pgfqpoint{0.827428in}{2.280728in}}%
\pgfpathlineto{\pgfqpoint{0.758394in}{2.232831in}}%
\pgfusepath{stroke}%
\end{pgfscope}%
\begin{pgfscope}%
\definecolor{textcolor}{rgb}{0.000000,0.000000,0.000000}%
\pgfsetstrokecolor{textcolor}%
\pgfsetfillcolor{textcolor}%
\pgftext[x=0.669139in,y=2.067182in,,top]{\color{textcolor}{\rmfamily\fontsize{10.000000}{12.000000}\selectfont\catcode`\^=\active\def^{\ifmmode\sp\else\^{}\fi}\catcode`\%=\active\def%{\%}$\mathdefault{0.0}$}}%
\end{pgfscope}%
\begin{pgfscope}%
\pgfsetrectcap%
\pgfsetroundjoin%
\pgfsetlinewidth{0.803000pt}%
\definecolor{currentstroke}{rgb}{0.000000,0.000000,0.000000}%
\pgfsetstrokecolor{currentstroke}%
\pgfsetdash{}{0pt}%
\pgfpathmoveto{\pgfqpoint{1.282251in}{1.929325in}}%
\pgfpathlineto{\pgfqpoint{1.213151in}{1.880446in}}%
\pgfusepath{stroke}%
\end{pgfscope}%
\begin{pgfscope}%
\definecolor{textcolor}{rgb}{0.000000,0.000000,0.000000}%
\pgfsetstrokecolor{textcolor}%
\pgfsetfillcolor{textcolor}%
\pgftext[x=1.122479in,y=1.713497in,,top]{\color{textcolor}{\rmfamily\fontsize{10.000000}{12.000000}\selectfont\catcode`\^=\active\def^{\ifmmode\sp\else\^{}\fi}\catcode`\%=\active\def%{\%}$\mathdefault{0.2}$}}%
\end{pgfscope}%
\begin{pgfscope}%
\pgfsetrectcap%
\pgfsetroundjoin%
\pgfsetlinewidth{0.803000pt}%
\definecolor{currentstroke}{rgb}{0.000000,0.000000,0.000000}%
\pgfsetstrokecolor{currentstroke}%
\pgfsetdash{}{0pt}%
\pgfpathmoveto{\pgfqpoint{1.746455in}{1.570675in}}%
\pgfpathlineto{\pgfqpoint{1.677300in}{1.520781in}}%
\pgfusepath{stroke}%
\end{pgfscope}%
\begin{pgfscope}%
\definecolor{textcolor}{rgb}{0.000000,0.000000,0.000000}%
\pgfsetstrokecolor{textcolor}%
\pgfsetfillcolor{textcolor}%
\pgftext[x=1.585171in,y=1.352515in,,top]{\color{textcolor}{\rmfamily\fontsize{10.000000}{12.000000}\selectfont\catcode`\^=\active\def^{\ifmmode\sp\else\^{}\fi}\catcode`\%=\active\def%{\%}$\mathdefault{0.4}$}}%
\end{pgfscope}%
\begin{pgfscope}%
\pgfsetrectcap%
\pgfsetroundjoin%
\pgfsetlinewidth{0.803000pt}%
\definecolor{currentstroke}{rgb}{0.000000,0.000000,0.000000}%
\pgfsetstrokecolor{currentstroke}%
\pgfsetdash{}{0pt}%
\pgfpathmoveto{\pgfqpoint{2.220333in}{1.204551in}}%
\pgfpathlineto{\pgfqpoint{2.151136in}{1.153611in}}%
\pgfusepath{stroke}%
\end{pgfscope}%
\begin{pgfscope}%
\definecolor{textcolor}{rgb}{0.000000,0.000000,0.000000}%
\pgfsetstrokecolor{textcolor}%
\pgfsetfillcolor{textcolor}%
\pgftext[x=2.057509in,y=0.984008in,,top]{\color{textcolor}{\rmfamily\fontsize{10.000000}{12.000000}\selectfont\catcode`\^=\active\def^{\ifmmode\sp\else\^{}\fi}\catcode`\%=\active\def%{\%}$\mathdefault{0.6}$}}%
\end{pgfscope}%
\begin{pgfscope}%
\pgfsetrectcap%
\pgfsetroundjoin%
\pgfsetlinewidth{0.803000pt}%
\definecolor{currentstroke}{rgb}{0.000000,0.000000,0.000000}%
\pgfsetstrokecolor{currentstroke}%
\pgfsetdash{}{0pt}%
\pgfpathmoveto{\pgfqpoint{2.704190in}{0.830716in}}%
\pgfpathlineto{\pgfqpoint{2.634966in}{0.778697in}}%
\pgfusepath{stroke}%
\end{pgfscope}%
\begin{pgfscope}%
\definecolor{textcolor}{rgb}{0.000000,0.000000,0.000000}%
\pgfsetstrokecolor{textcolor}%
\pgfsetfillcolor{textcolor}%
\pgftext[x=2.539797in,y=0.607737in,,top]{\color{textcolor}{\rmfamily\fontsize{10.000000}{12.000000}\selectfont\catcode`\^=\active\def^{\ifmmode\sp\else\^{}\fi}\catcode`\%=\active\def%{\%}$\mathdefault{0.8}$}}%
\end{pgfscope}%
\begin{pgfscope}%
\pgfsetrectcap%
\pgfsetroundjoin%
\pgfsetlinewidth{0.803000pt}%
\definecolor{currentstroke}{rgb}{0.000000,0.000000,0.000000}%
\pgfsetstrokecolor{currentstroke}%
\pgfsetdash{}{0pt}%
\pgfpathmoveto{\pgfqpoint{3.198346in}{0.448925in}}%
\pgfpathlineto{\pgfqpoint{3.129108in}{0.395792in}}%
\pgfusepath{stroke}%
\end{pgfscope}%
\begin{pgfscope}%
\definecolor{textcolor}{rgb}{0.000000,0.000000,0.000000}%
\pgfsetstrokecolor{textcolor}%
\pgfsetfillcolor{textcolor}%
\pgftext[x=3.032352in,y=0.223457in,,top]{\color{textcolor}{\rmfamily\fontsize{10.000000}{12.000000}\selectfont\catcode`\^=\active\def^{\ifmmode\sp\else\^{}\fi}\catcode`\%=\active\def%{\%}$\mathdefault{1.0}$}}%
\end{pgfscope}%
\begin{pgfscope}%
\definecolor{textcolor}{rgb}{0.000000,0.000000,0.000000}%
\pgfsetstrokecolor{textcolor}%
\pgfsetfillcolor{textcolor}%
\pgftext[x=1.656990in,y=0.931525in,,,rotate=322.282695]{\color{textcolor}{\rmfamily\fontsize{14.000000}{16.800000}\selectfont\catcode`\^=\active\def^{\ifmmode\sp\else\^{}\fi}\catcode`\%=\active\def%{\%}$f_2$}}%
\end{pgfscope}%
\begin{pgfscope}%
\pgfsetrectcap%
\pgfsetroundjoin%
\pgfsetlinewidth{0.803000pt}%
\definecolor{currentstroke}{rgb}{0.000000,0.000000,0.000000}%
\pgfsetstrokecolor{currentstroke}%
\pgfsetdash{}{0pt}%
\pgfpathmoveto{\pgfqpoint{0.640904in}{2.391249in}}%
\pgfpathlineto{\pgfqpoint{0.463765in}{4.495556in}}%
\pgfusepath{stroke}%
\end{pgfscope}%
\begin{pgfscope}%
\pgfsetrectcap%
\pgfsetroundjoin%
\pgfsetlinewidth{0.803000pt}%
\definecolor{currentstroke}{rgb}{0.000000,0.000000,0.000000}%
\pgfsetstrokecolor{currentstroke}%
\pgfsetdash{}{0pt}%
\pgfpathmoveto{\pgfqpoint{0.653356in}{2.533293in}}%
\pgfpathlineto{\pgfqpoint{0.584064in}{2.485737in}}%
\pgfusepath{stroke}%
\end{pgfscope}%
\begin{pgfscope}%
\definecolor{textcolor}{rgb}{0.000000,0.000000,0.000000}%
\pgfsetstrokecolor{textcolor}%
\pgfsetfillcolor{textcolor}%
\pgftext[x=0.358681in,y=2.517456in,,top]{\color{textcolor}{\rmfamily\fontsize{10.000000}{12.000000}\selectfont\catcode`\^=\active\def^{\ifmmode\sp\else\^{}\fi}\catcode`\%=\active\def%{\%}$\mathdefault{0.0}$}}%
\end{pgfscope}%
\begin{pgfscope}%
\pgfsetrectcap%
\pgfsetroundjoin%
\pgfsetlinewidth{0.803000pt}%
\definecolor{currentstroke}{rgb}{0.000000,0.000000,0.000000}%
\pgfsetstrokecolor{currentstroke}%
\pgfsetdash{}{0pt}%
\pgfpathmoveto{\pgfqpoint{0.624048in}{2.884587in}}%
\pgfpathlineto{\pgfqpoint{0.553964in}{2.837009in}}%
\pgfusepath{stroke}%
\end{pgfscope}%
\begin{pgfscope}%
\definecolor{textcolor}{rgb}{0.000000,0.000000,0.000000}%
\pgfsetstrokecolor{textcolor}%
\pgfsetfillcolor{textcolor}%
\pgftext[x=0.326166in,y=2.868742in,,top]{\color{textcolor}{\rmfamily\fontsize{10.000000}{12.000000}\selectfont\catcode`\^=\active\def^{\ifmmode\sp\else\^{}\fi}\catcode`\%=\active\def%{\%}$\mathdefault{0.2}$}}%
\end{pgfscope}%
\begin{pgfscope}%
\pgfsetrectcap%
\pgfsetroundjoin%
\pgfsetlinewidth{0.803000pt}%
\definecolor{currentstroke}{rgb}{0.000000,0.000000,0.000000}%
\pgfsetstrokecolor{currentstroke}%
\pgfsetdash{}{0pt}%
\pgfpathmoveto{\pgfqpoint{0.594099in}{3.243577in}}%
\pgfpathlineto{\pgfqpoint{0.523203in}{3.195988in}}%
\pgfusepath{stroke}%
\end{pgfscope}%
\begin{pgfscope}%
\definecolor{textcolor}{rgb}{0.000000,0.000000,0.000000}%
\pgfsetstrokecolor{textcolor}%
\pgfsetfillcolor{textcolor}%
\pgftext[x=0.292938in,y=3.227729in,,top]{\color{textcolor}{\rmfamily\fontsize{10.000000}{12.000000}\selectfont\catcode`\^=\active\def^{\ifmmode\sp\else\^{}\fi}\catcode`\%=\active\def%{\%}$\mathdefault{0.4}$}}%
\end{pgfscope}%
\begin{pgfscope}%
\pgfsetrectcap%
\pgfsetroundjoin%
\pgfsetlinewidth{0.803000pt}%
\definecolor{currentstroke}{rgb}{0.000000,0.000000,0.000000}%
\pgfsetstrokecolor{currentstroke}%
\pgfsetdash{}{0pt}%
\pgfpathmoveto{\pgfqpoint{0.563487in}{3.610519in}}%
\pgfpathlineto{\pgfqpoint{0.491760in}{3.562930in}}%
\pgfusepath{stroke}%
\end{pgfscope}%
\begin{pgfscope}%
\definecolor{textcolor}{rgb}{0.000000,0.000000,0.000000}%
\pgfsetstrokecolor{textcolor}%
\pgfsetfillcolor{textcolor}%
\pgftext[x=0.258973in,y=3.594671in,,top]{\color{textcolor}{\rmfamily\fontsize{10.000000}{12.000000}\selectfont\catcode`\^=\active\def^{\ifmmode\sp\else\^{}\fi}\catcode`\%=\active\def%{\%}$\mathdefault{0.6}$}}%
\end{pgfscope}%
\begin{pgfscope}%
\pgfsetrectcap%
\pgfsetroundjoin%
\pgfsetlinewidth{0.803000pt}%
\definecolor{currentstroke}{rgb}{0.000000,0.000000,0.000000}%
\pgfsetstrokecolor{currentstroke}%
\pgfsetdash{}{0pt}%
\pgfpathmoveto{\pgfqpoint{0.532188in}{3.985680in}}%
\pgfpathlineto{\pgfqpoint{0.459612in}{3.938103in}}%
\pgfusepath{stroke}%
\end{pgfscope}%
\begin{pgfscope}%
\definecolor{textcolor}{rgb}{0.000000,0.000000,0.000000}%
\pgfsetstrokecolor{textcolor}%
\pgfsetfillcolor{textcolor}%
\pgftext[x=0.224248in,y=3.969836in,,top]{\color{textcolor}{\rmfamily\fontsize{10.000000}{12.000000}\selectfont\catcode`\^=\active\def^{\ifmmode\sp\else\^{}\fi}\catcode`\%=\active\def%{\%}$\mathdefault{0.8}$}}%
\end{pgfscope}%
\begin{pgfscope}%
\pgfsetrectcap%
\pgfsetroundjoin%
\pgfsetlinewidth{0.803000pt}%
\definecolor{currentstroke}{rgb}{0.000000,0.000000,0.000000}%
\pgfsetstrokecolor{currentstroke}%
\pgfsetdash{}{0pt}%
\pgfpathmoveto{\pgfqpoint{0.500181in}{4.369340in}}%
\pgfpathlineto{\pgfqpoint{0.426734in}{4.321787in}}%
\pgfusepath{stroke}%
\end{pgfscope}%
\begin{pgfscope}%
\definecolor{textcolor}{rgb}{0.000000,0.000000,0.000000}%
\pgfsetstrokecolor{textcolor}%
\pgfsetfillcolor{textcolor}%
\pgftext[x=0.188735in,y=4.353505in,,top]{\color{textcolor}{\rmfamily\fontsize{10.000000}{12.000000}\selectfont\catcode`\^=\active\def^{\ifmmode\sp\else\^{}\fi}\catcode`\%=\active\def%{\%}$\mathdefault{1.0}$}}%
\end{pgfscope}%
\begin{pgfscope}%
\definecolor{textcolor}{rgb}{0.000000,0.000000,0.000000}%
\pgfsetstrokecolor{textcolor}%
\pgfsetfillcolor{textcolor}%
\pgftext[x=-0.051568in,y=3.410189in,,,rotate=274.811779]{\color{textcolor}{\rmfamily\fontsize{14.000000}{16.800000}\selectfont\catcode`\^=\active\def^{\ifmmode\sp\else\^{}\fi}\catcode`\%=\active\def%{\%}$f_3$}}%
\end{pgfscope}%
\begin{pgfscope}%
\pgfpathrectangle{\pgfqpoint{0.194833in}{0.246946in}}{\pgfqpoint{6.160000in}{6.160000in}}%
\pgfusepath{clip}%
\pgfsetbuttcap%
\pgfsetroundjoin%
\definecolor{currentfill}{rgb}{0.121569,0.466667,0.705882}%
\pgfsetfillcolor{currentfill}%
\pgfsetlinewidth{1.003750pt}%
\definecolor{currentstroke}{rgb}{0.121569,0.466667,0.705882}%
\pgfsetstrokecolor{currentstroke}%
\pgfsetdash{}{0pt}%
\pgfpathmoveto{\pgfqpoint{0.977454in}{2.468351in}}%
\pgfpathcurveto{\pgfqpoint{0.990477in}{2.468351in}}{\pgfqpoint{1.002968in}{2.473525in}}{\pgfqpoint{1.012176in}{2.482733in}}%
\pgfpathcurveto{\pgfqpoint{1.021385in}{2.491942in}}{\pgfqpoint{1.026559in}{2.504433in}}{\pgfqpoint{1.026559in}{2.517456in}}%
\pgfpathcurveto{\pgfqpoint{1.026559in}{2.530478in}}{\pgfqpoint{1.021385in}{2.542969in}}{\pgfqpoint{1.012176in}{2.552178in}}%
\pgfpathcurveto{\pgfqpoint{1.002968in}{2.561386in}}{\pgfqpoint{0.990477in}{2.566560in}}{\pgfqpoint{0.977454in}{2.566560in}}%
\pgfpathcurveto{\pgfqpoint{0.964431in}{2.566560in}}{\pgfqpoint{0.951940in}{2.561386in}}{\pgfqpoint{0.942732in}{2.552178in}}%
\pgfpathcurveto{\pgfqpoint{0.933523in}{2.542969in}}{\pgfqpoint{0.928349in}{2.530478in}}{\pgfqpoint{0.928349in}{2.517456in}}%
\pgfpathcurveto{\pgfqpoint{0.928349in}{2.504433in}}{\pgfqpoint{0.933523in}{2.491942in}}{\pgfqpoint{0.942732in}{2.482733in}}%
\pgfpathcurveto{\pgfqpoint{0.951940in}{2.473525in}}{\pgfqpoint{0.964431in}{2.468351in}}{\pgfqpoint{0.977454in}{2.468351in}}%
\pgfpathlineto{\pgfqpoint{0.977454in}{2.468351in}}%
\pgfpathclose%
\pgfusepath{stroke,fill}%
\end{pgfscope}%
\begin{pgfscope}%
\pgfpathrectangle{\pgfqpoint{0.194833in}{0.246946in}}{\pgfqpoint{6.160000in}{6.160000in}}%
\pgfusepath{clip}%
\pgfsetbuttcap%
\pgfsetroundjoin%
\definecolor{currentfill}{rgb}{0.121569,0.466667,0.705882}%
\pgfsetfillcolor{currentfill}%
\pgfsetlinewidth{1.003750pt}%
\definecolor{currentstroke}{rgb}{0.121569,0.466667,0.705882}%
\pgfsetstrokecolor{currentstroke}%
\pgfsetdash{}{0pt}%
\pgfpathmoveto{\pgfqpoint{2.279284in}{2.468351in}}%
\pgfpathcurveto{\pgfqpoint{2.292306in}{2.468351in}}{\pgfqpoint{2.304797in}{2.473525in}}{\pgfqpoint{2.314006in}{2.482733in}}%
\pgfpathcurveto{\pgfqpoint{2.323214in}{2.491942in}}{\pgfqpoint{2.328388in}{2.504433in}}{\pgfqpoint{2.328388in}{2.517456in}}%
\pgfpathcurveto{\pgfqpoint{2.328388in}{2.530478in}}{\pgfqpoint{2.323214in}{2.542969in}}{\pgfqpoint{2.314006in}{2.552178in}}%
\pgfpathcurveto{\pgfqpoint{2.304797in}{2.561386in}}{\pgfqpoint{2.292306in}{2.566560in}}{\pgfqpoint{2.279284in}{2.566560in}}%
\pgfpathcurveto{\pgfqpoint{2.266261in}{2.566560in}}{\pgfqpoint{2.253770in}{2.561386in}}{\pgfqpoint{2.244561in}{2.552178in}}%
\pgfpathcurveto{\pgfqpoint{2.235353in}{2.542969in}}{\pgfqpoint{2.230179in}{2.530478in}}{\pgfqpoint{2.230179in}{2.517456in}}%
\pgfpathcurveto{\pgfqpoint{2.230179in}{2.504433in}}{\pgfqpoint{2.235353in}{2.491942in}}{\pgfqpoint{2.244561in}{2.482733in}}%
\pgfpathcurveto{\pgfqpoint{2.253770in}{2.473525in}}{\pgfqpoint{2.266261in}{2.468351in}}{\pgfqpoint{2.279284in}{2.468351in}}%
\pgfpathlineto{\pgfqpoint{2.279284in}{2.468351in}}%
\pgfpathclose%
\pgfusepath{stroke,fill}%
\end{pgfscope}%
\begin{pgfscope}%
\pgfpathrectangle{\pgfqpoint{0.194833in}{0.246946in}}{\pgfqpoint{6.160000in}{6.160000in}}%
\pgfusepath{clip}%
\pgfsetbuttcap%
\pgfsetroundjoin%
\definecolor{currentfill}{rgb}{0.121569,0.466667,0.705882}%
\pgfsetfillcolor{currentfill}%
\pgfsetlinewidth{1.003750pt}%
\definecolor{currentstroke}{rgb}{0.121569,0.466667,0.705882}%
\pgfsetstrokecolor{currentstroke}%
\pgfsetdash{}{0pt}%
\pgfpathmoveto{\pgfqpoint{4.461082in}{2.468351in}}%
\pgfpathcurveto{\pgfqpoint{4.474105in}{2.468351in}}{\pgfqpoint{4.486596in}{2.473525in}}{\pgfqpoint{4.495804in}{2.482733in}}%
\pgfpathcurveto{\pgfqpoint{4.505013in}{2.491942in}}{\pgfqpoint{4.510187in}{2.504433in}}{\pgfqpoint{4.510187in}{2.517456in}}%
\pgfpathcurveto{\pgfqpoint{4.510187in}{2.530478in}}{\pgfqpoint{4.505013in}{2.542969in}}{\pgfqpoint{4.495804in}{2.552178in}}%
\pgfpathcurveto{\pgfqpoint{4.486596in}{2.561386in}}{\pgfqpoint{4.474105in}{2.566560in}}{\pgfqpoint{4.461082in}{2.566560in}}%
\pgfpathcurveto{\pgfqpoint{4.448059in}{2.566560in}}{\pgfqpoint{4.435568in}{2.561386in}}{\pgfqpoint{4.426360in}{2.552178in}}%
\pgfpathcurveto{\pgfqpoint{4.417151in}{2.542969in}}{\pgfqpoint{4.411977in}{2.530478in}}{\pgfqpoint{4.411977in}{2.517456in}}%
\pgfpathcurveto{\pgfqpoint{4.411977in}{2.504433in}}{\pgfqpoint{4.417151in}{2.491942in}}{\pgfqpoint{4.426360in}{2.482733in}}%
\pgfpathcurveto{\pgfqpoint{4.435568in}{2.473525in}}{\pgfqpoint{4.448059in}{2.468351in}}{\pgfqpoint{4.461082in}{2.468351in}}%
\pgfpathlineto{\pgfqpoint{4.461082in}{2.468351in}}%
\pgfpathclose%
\pgfusepath{stroke,fill}%
\end{pgfscope}%
\begin{pgfscope}%
\pgfpathrectangle{\pgfqpoint{0.194833in}{0.246946in}}{\pgfqpoint{6.160000in}{6.160000in}}%
\pgfusepath{clip}%
\pgfsetbuttcap%
\pgfsetroundjoin%
\definecolor{currentfill}{rgb}{0.121569,0.466667,0.705882}%
\pgfsetfillcolor{currentfill}%
\pgfsetlinewidth{1.003750pt}%
\definecolor{currentstroke}{rgb}{0.121569,0.466667,0.705882}%
\pgfsetstrokecolor{currentstroke}%
\pgfsetdash{}{0pt}%
\pgfpathmoveto{\pgfqpoint{4.885244in}{2.468351in}}%
\pgfpathcurveto{\pgfqpoint{4.898267in}{2.468351in}}{\pgfqpoint{4.910758in}{2.473525in}}{\pgfqpoint{4.919967in}{2.482733in}}%
\pgfpathcurveto{\pgfqpoint{4.929175in}{2.491942in}}{\pgfqpoint{4.934349in}{2.504433in}}{\pgfqpoint{4.934349in}{2.517456in}}%
\pgfpathcurveto{\pgfqpoint{4.934349in}{2.530478in}}{\pgfqpoint{4.929175in}{2.542969in}}{\pgfqpoint{4.919967in}{2.552178in}}%
\pgfpathcurveto{\pgfqpoint{4.910758in}{2.561386in}}{\pgfqpoint{4.898267in}{2.566560in}}{\pgfqpoint{4.885244in}{2.566560in}}%
\pgfpathcurveto{\pgfqpoint{4.872222in}{2.566560in}}{\pgfqpoint{4.859731in}{2.561386in}}{\pgfqpoint{4.850522in}{2.552178in}}%
\pgfpathcurveto{\pgfqpoint{4.841314in}{2.542969in}}{\pgfqpoint{4.836140in}{2.530478in}}{\pgfqpoint{4.836140in}{2.517456in}}%
\pgfpathcurveto{\pgfqpoint{4.836140in}{2.504433in}}{\pgfqpoint{4.841314in}{2.491942in}}{\pgfqpoint{4.850522in}{2.482733in}}%
\pgfpathcurveto{\pgfqpoint{4.859731in}{2.473525in}}{\pgfqpoint{4.872222in}{2.468351in}}{\pgfqpoint{4.885244in}{2.468351in}}%
\pgfpathlineto{\pgfqpoint{4.885244in}{2.468351in}}%
\pgfpathclose%
\pgfusepath{stroke,fill}%
\end{pgfscope}%
\begin{pgfscope}%
\pgfpathrectangle{\pgfqpoint{0.194833in}{0.246946in}}{\pgfqpoint{6.160000in}{6.160000in}}%
\pgfusepath{clip}%
\pgfsetbuttcap%
\pgfsetroundjoin%
\definecolor{currentfill}{rgb}{0.121569,0.466667,0.705882}%
\pgfsetfillcolor{currentfill}%
\pgfsetlinewidth{1.003750pt}%
\definecolor{currentstroke}{rgb}{0.121569,0.466667,0.705882}%
\pgfsetstrokecolor{currentstroke}%
\pgfsetdash{}{0pt}%
\pgfpathmoveto{\pgfqpoint{4.080163in}{2.468351in}}%
\pgfpathcurveto{\pgfqpoint{4.093185in}{2.468351in}}{\pgfqpoint{4.105676in}{2.473525in}}{\pgfqpoint{4.114885in}{2.482733in}}%
\pgfpathcurveto{\pgfqpoint{4.124093in}{2.491942in}}{\pgfqpoint{4.129267in}{2.504433in}}{\pgfqpoint{4.129267in}{2.517456in}}%
\pgfpathcurveto{\pgfqpoint{4.129267in}{2.530478in}}{\pgfqpoint{4.124093in}{2.542969in}}{\pgfqpoint{4.114885in}{2.552178in}}%
\pgfpathcurveto{\pgfqpoint{4.105676in}{2.561386in}}{\pgfqpoint{4.093185in}{2.566560in}}{\pgfqpoint{4.080163in}{2.566560in}}%
\pgfpathcurveto{\pgfqpoint{4.067140in}{2.566560in}}{\pgfqpoint{4.054649in}{2.561386in}}{\pgfqpoint{4.045440in}{2.552178in}}%
\pgfpathcurveto{\pgfqpoint{4.036232in}{2.542969in}}{\pgfqpoint{4.031058in}{2.530478in}}{\pgfqpoint{4.031058in}{2.517456in}}%
\pgfpathcurveto{\pgfqpoint{4.031058in}{2.504433in}}{\pgfqpoint{4.036232in}{2.491942in}}{\pgfqpoint{4.045440in}{2.482733in}}%
\pgfpathcurveto{\pgfqpoint{4.054649in}{2.473525in}}{\pgfqpoint{4.067140in}{2.468351in}}{\pgfqpoint{4.080163in}{2.468351in}}%
\pgfpathlineto{\pgfqpoint{4.080163in}{2.468351in}}%
\pgfpathclose%
\pgfusepath{stroke,fill}%
\end{pgfscope}%
\begin{pgfscope}%
\pgfpathrectangle{\pgfqpoint{0.194833in}{0.246946in}}{\pgfqpoint{6.160000in}{6.160000in}}%
\pgfusepath{clip}%
\pgfsetbuttcap%
\pgfsetroundjoin%
\definecolor{currentfill}{rgb}{0.121569,0.466667,0.705882}%
\pgfsetfillcolor{currentfill}%
\pgfsetlinewidth{1.003750pt}%
\definecolor{currentstroke}{rgb}{0.121569,0.466667,0.705882}%
\pgfsetstrokecolor{currentstroke}%
\pgfsetdash{}{0pt}%
\pgfpathmoveto{\pgfqpoint{5.316558in}{2.468351in}}%
\pgfpathcurveto{\pgfqpoint{5.329581in}{2.468351in}}{\pgfqpoint{5.342072in}{2.473525in}}{\pgfqpoint{5.351280in}{2.482733in}}%
\pgfpathcurveto{\pgfqpoint{5.360488in}{2.491942in}}{\pgfqpoint{5.365662in}{2.504433in}}{\pgfqpoint{5.365662in}{2.517456in}}%
\pgfpathcurveto{\pgfqpoint{5.365662in}{2.530478in}}{\pgfqpoint{5.360488in}{2.542969in}}{\pgfqpoint{5.351280in}{2.552178in}}%
\pgfpathcurveto{\pgfqpoint{5.342072in}{2.561386in}}{\pgfqpoint{5.329581in}{2.566560in}}{\pgfqpoint{5.316558in}{2.566560in}}%
\pgfpathcurveto{\pgfqpoint{5.303535in}{2.566560in}}{\pgfqpoint{5.291044in}{2.561386in}}{\pgfqpoint{5.281836in}{2.552178in}}%
\pgfpathcurveto{\pgfqpoint{5.272627in}{2.542969in}}{\pgfqpoint{5.267453in}{2.530478in}}{\pgfqpoint{5.267453in}{2.517456in}}%
\pgfpathcurveto{\pgfqpoint{5.267453in}{2.504433in}}{\pgfqpoint{5.272627in}{2.491942in}}{\pgfqpoint{5.281836in}{2.482733in}}%
\pgfpathcurveto{\pgfqpoint{5.291044in}{2.473525in}}{\pgfqpoint{5.303535in}{2.468351in}}{\pgfqpoint{5.316558in}{2.468351in}}%
\pgfpathlineto{\pgfqpoint{5.316558in}{2.468351in}}%
\pgfpathclose%
\pgfusepath{stroke,fill}%
\end{pgfscope}%
\begin{pgfscope}%
\pgfpathrectangle{\pgfqpoint{0.194833in}{0.246946in}}{\pgfqpoint{6.160000in}{6.160000in}}%
\pgfusepath{clip}%
\pgfsetbuttcap%
\pgfsetroundjoin%
\definecolor{currentfill}{rgb}{0.121569,0.466667,0.705882}%
\pgfsetfillcolor{currentfill}%
\pgfsetlinewidth{1.003750pt}%
\definecolor{currentstroke}{rgb}{0.121569,0.466667,0.705882}%
\pgfsetstrokecolor{currentstroke}%
\pgfsetdash{}{0pt}%
\pgfpathmoveto{\pgfqpoint{3.654994in}{2.468351in}}%
\pgfpathcurveto{\pgfqpoint{3.668016in}{2.468351in}}{\pgfqpoint{3.680507in}{2.473525in}}{\pgfqpoint{3.689716in}{2.482733in}}%
\pgfpathcurveto{\pgfqpoint{3.698924in}{2.491942in}}{\pgfqpoint{3.704098in}{2.504433in}}{\pgfqpoint{3.704098in}{2.517456in}}%
\pgfpathcurveto{\pgfqpoint{3.704098in}{2.530478in}}{\pgfqpoint{3.698924in}{2.542969in}}{\pgfqpoint{3.689716in}{2.552178in}}%
\pgfpathcurveto{\pgfqpoint{3.680507in}{2.561386in}}{\pgfqpoint{3.668016in}{2.566560in}}{\pgfqpoint{3.654994in}{2.566560in}}%
\pgfpathcurveto{\pgfqpoint{3.641971in}{2.566560in}}{\pgfqpoint{3.629480in}{2.561386in}}{\pgfqpoint{3.620271in}{2.552178in}}%
\pgfpathcurveto{\pgfqpoint{3.611063in}{2.542969in}}{\pgfqpoint{3.605889in}{2.530478in}}{\pgfqpoint{3.605889in}{2.517456in}}%
\pgfpathcurveto{\pgfqpoint{3.605889in}{2.504433in}}{\pgfqpoint{3.611063in}{2.491942in}}{\pgfqpoint{3.620271in}{2.482733in}}%
\pgfpathcurveto{\pgfqpoint{3.629480in}{2.473525in}}{\pgfqpoint{3.641971in}{2.468351in}}{\pgfqpoint{3.654994in}{2.468351in}}%
\pgfpathlineto{\pgfqpoint{3.654994in}{2.468351in}}%
\pgfpathclose%
\pgfusepath{stroke,fill}%
\end{pgfscope}%
\begin{pgfscope}%
\pgfpathrectangle{\pgfqpoint{0.194833in}{0.246946in}}{\pgfqpoint{6.160000in}{6.160000in}}%
\pgfusepath{clip}%
\pgfsetbuttcap%
\pgfsetroundjoin%
\definecolor{currentfill}{rgb}{0.121569,0.466667,0.705882}%
\pgfsetfillcolor{currentfill}%
\pgfsetlinewidth{1.003750pt}%
\definecolor{currentstroke}{rgb}{0.121569,0.466667,0.705882}%
\pgfsetstrokecolor{currentstroke}%
\pgfsetdash{}{0pt}%
\pgfpathmoveto{\pgfqpoint{3.208301in}{2.468351in}}%
\pgfpathcurveto{\pgfqpoint{3.221323in}{2.468351in}}{\pgfqpoint{3.233815in}{2.473525in}}{\pgfqpoint{3.243023in}{2.482733in}}%
\pgfpathcurveto{\pgfqpoint{3.252231in}{2.491942in}}{\pgfqpoint{3.257405in}{2.504433in}}{\pgfqpoint{3.257405in}{2.517456in}}%
\pgfpathcurveto{\pgfqpoint{3.257405in}{2.530478in}}{\pgfqpoint{3.252231in}{2.542969in}}{\pgfqpoint{3.243023in}{2.552178in}}%
\pgfpathcurveto{\pgfqpoint{3.233815in}{2.561386in}}{\pgfqpoint{3.221323in}{2.566560in}}{\pgfqpoint{3.208301in}{2.566560in}}%
\pgfpathcurveto{\pgfqpoint{3.195278in}{2.566560in}}{\pgfqpoint{3.182787in}{2.561386in}}{\pgfqpoint{3.173579in}{2.552178in}}%
\pgfpathcurveto{\pgfqpoint{3.164370in}{2.542969in}}{\pgfqpoint{3.159196in}{2.530478in}}{\pgfqpoint{3.159196in}{2.517456in}}%
\pgfpathcurveto{\pgfqpoint{3.159196in}{2.504433in}}{\pgfqpoint{3.164370in}{2.491942in}}{\pgfqpoint{3.173579in}{2.482733in}}%
\pgfpathcurveto{\pgfqpoint{3.182787in}{2.473525in}}{\pgfqpoint{3.195278in}{2.468351in}}{\pgfqpoint{3.208301in}{2.468351in}}%
\pgfpathlineto{\pgfqpoint{3.208301in}{2.468351in}}%
\pgfpathclose%
\pgfusepath{stroke,fill}%
\end{pgfscope}%
\begin{pgfscope}%
\pgfpathrectangle{\pgfqpoint{0.194833in}{0.246946in}}{\pgfqpoint{6.160000in}{6.160000in}}%
\pgfusepath{clip}%
\pgfsetbuttcap%
\pgfsetroundjoin%
\definecolor{currentfill}{rgb}{0.121569,0.466667,0.705882}%
\pgfsetfillcolor{currentfill}%
\pgfsetlinewidth{1.003750pt}%
\definecolor{currentstroke}{rgb}{0.121569,0.466667,0.705882}%
\pgfsetstrokecolor{currentstroke}%
\pgfsetdash{}{0pt}%
\pgfpathmoveto{\pgfqpoint{2.712527in}{2.468351in}}%
\pgfpathcurveto{\pgfqpoint{2.725550in}{2.468351in}}{\pgfqpoint{2.738041in}{2.473525in}}{\pgfqpoint{2.747250in}{2.482733in}}%
\pgfpathcurveto{\pgfqpoint{2.756458in}{2.491942in}}{\pgfqpoint{2.761632in}{2.504433in}}{\pgfqpoint{2.761632in}{2.517456in}}%
\pgfpathcurveto{\pgfqpoint{2.761632in}{2.530478in}}{\pgfqpoint{2.756458in}{2.542969in}}{\pgfqpoint{2.747250in}{2.552178in}}%
\pgfpathcurveto{\pgfqpoint{2.738041in}{2.561386in}}{\pgfqpoint{2.725550in}{2.566560in}}{\pgfqpoint{2.712527in}{2.566560in}}%
\pgfpathcurveto{\pgfqpoint{2.699505in}{2.566560in}}{\pgfqpoint{2.687014in}{2.561386in}}{\pgfqpoint{2.677805in}{2.552178in}}%
\pgfpathcurveto{\pgfqpoint{2.668597in}{2.542969in}}{\pgfqpoint{2.663423in}{2.530478in}}{\pgfqpoint{2.663423in}{2.517456in}}%
\pgfpathcurveto{\pgfqpoint{2.663423in}{2.504433in}}{\pgfqpoint{2.668597in}{2.491942in}}{\pgfqpoint{2.677805in}{2.482733in}}%
\pgfpathcurveto{\pgfqpoint{2.687014in}{2.473525in}}{\pgfqpoint{2.699505in}{2.468351in}}{\pgfqpoint{2.712527in}{2.468351in}}%
\pgfpathlineto{\pgfqpoint{2.712527in}{2.468351in}}%
\pgfpathclose%
\pgfusepath{stroke,fill}%
\end{pgfscope}%
\begin{pgfscope}%
\pgfpathrectangle{\pgfqpoint{0.194833in}{0.246946in}}{\pgfqpoint{6.160000in}{6.160000in}}%
\pgfusepath{clip}%
\pgfsetbuttcap%
\pgfsetroundjoin%
\definecolor{currentfill}{rgb}{0.121569,0.466667,0.705882}%
\pgfsetfillcolor{currentfill}%
\pgfsetlinewidth{1.003750pt}%
\definecolor{currentstroke}{rgb}{0.121569,0.466667,0.705882}%
\pgfsetstrokecolor{currentstroke}%
\pgfsetdash{}{0pt}%
\pgfpathmoveto{\pgfqpoint{5.738698in}{2.468351in}}%
\pgfpathcurveto{\pgfqpoint{5.751721in}{2.468351in}}{\pgfqpoint{5.764212in}{2.473525in}}{\pgfqpoint{5.773420in}{2.482733in}}%
\pgfpathcurveto{\pgfqpoint{5.782629in}{2.491942in}}{\pgfqpoint{5.787803in}{2.504433in}}{\pgfqpoint{5.787803in}{2.517456in}}%
\pgfpathcurveto{\pgfqpoint{5.787803in}{2.530478in}}{\pgfqpoint{5.782629in}{2.542969in}}{\pgfqpoint{5.773420in}{2.552178in}}%
\pgfpathcurveto{\pgfqpoint{5.764212in}{2.561386in}}{\pgfqpoint{5.751721in}{2.566560in}}{\pgfqpoint{5.738698in}{2.566560in}}%
\pgfpathcurveto{\pgfqpoint{5.725675in}{2.566560in}}{\pgfqpoint{5.713184in}{2.561386in}}{\pgfqpoint{5.703976in}{2.552178in}}%
\pgfpathcurveto{\pgfqpoint{5.694767in}{2.542969in}}{\pgfqpoint{5.689593in}{2.530478in}}{\pgfqpoint{5.689593in}{2.517456in}}%
\pgfpathcurveto{\pgfqpoint{5.689593in}{2.504433in}}{\pgfqpoint{5.694767in}{2.491942in}}{\pgfqpoint{5.703976in}{2.482733in}}%
\pgfpathcurveto{\pgfqpoint{5.713184in}{2.473525in}}{\pgfqpoint{5.725675in}{2.468351in}}{\pgfqpoint{5.738698in}{2.468351in}}%
\pgfpathlineto{\pgfqpoint{5.738698in}{2.468351in}}%
\pgfpathclose%
\pgfusepath{stroke,fill}%
\end{pgfscope}%
\begin{pgfscope}%
\pgfpathrectangle{\pgfqpoint{0.194833in}{0.246946in}}{\pgfqpoint{6.160000in}{6.160000in}}%
\pgfusepath{clip}%
\pgfsetbuttcap%
\pgfsetroundjoin%
\definecolor{currentfill}{rgb}{0.121569,0.466667,0.705882}%
\pgfsetfillcolor{currentfill}%
\pgfsetlinewidth{1.003750pt}%
\definecolor{currentstroke}{rgb}{0.121569,0.466667,0.705882}%
\pgfsetstrokecolor{currentstroke}%
\pgfsetdash{}{0pt}%
\pgfpathmoveto{\pgfqpoint{1.433009in}{2.468351in}}%
\pgfpathcurveto{\pgfqpoint{1.446031in}{2.468351in}}{\pgfqpoint{1.458522in}{2.473525in}}{\pgfqpoint{1.467731in}{2.482733in}}%
\pgfpathcurveto{\pgfqpoint{1.476939in}{2.491942in}}{\pgfqpoint{1.482113in}{2.504433in}}{\pgfqpoint{1.482113in}{2.517456in}}%
\pgfpathcurveto{\pgfqpoint{1.482113in}{2.530478in}}{\pgfqpoint{1.476939in}{2.542969in}}{\pgfqpoint{1.467731in}{2.552178in}}%
\pgfpathcurveto{\pgfqpoint{1.458522in}{2.561386in}}{\pgfqpoint{1.446031in}{2.566560in}}{\pgfqpoint{1.433009in}{2.566560in}}%
\pgfpathcurveto{\pgfqpoint{1.419986in}{2.566560in}}{\pgfqpoint{1.407495in}{2.561386in}}{\pgfqpoint{1.398286in}{2.552178in}}%
\pgfpathcurveto{\pgfqpoint{1.389078in}{2.542969in}}{\pgfqpoint{1.383904in}{2.530478in}}{\pgfqpoint{1.383904in}{2.517456in}}%
\pgfpathcurveto{\pgfqpoint{1.383904in}{2.504433in}}{\pgfqpoint{1.389078in}{2.491942in}}{\pgfqpoint{1.398286in}{2.482733in}}%
\pgfpathcurveto{\pgfqpoint{1.407495in}{2.473525in}}{\pgfqpoint{1.419986in}{2.468351in}}{\pgfqpoint{1.433009in}{2.468351in}}%
\pgfpathlineto{\pgfqpoint{1.433009in}{2.468351in}}%
\pgfpathclose%
\pgfusepath{stroke,fill}%
\end{pgfscope}%
\begin{pgfscope}%
\pgfpathrectangle{\pgfqpoint{0.194833in}{0.246946in}}{\pgfqpoint{6.160000in}{6.160000in}}%
\pgfusepath{clip}%
\pgfsetbuttcap%
\pgfsetroundjoin%
\definecolor{currentfill}{rgb}{0.121569,0.466667,0.705882}%
\pgfsetfillcolor{currentfill}%
\pgfsetlinewidth{1.003750pt}%
\definecolor{currentstroke}{rgb}{0.121569,0.466667,0.705882}%
\pgfsetstrokecolor{currentstroke}%
\pgfsetdash{}{0pt}%
\pgfpathmoveto{\pgfqpoint{1.868054in}{2.468351in}}%
\pgfpathcurveto{\pgfqpoint{1.881077in}{2.468351in}}{\pgfqpoint{1.893568in}{2.473525in}}{\pgfqpoint{1.902776in}{2.482733in}}%
\pgfpathcurveto{\pgfqpoint{1.911985in}{2.491942in}}{\pgfqpoint{1.917159in}{2.504433in}}{\pgfqpoint{1.917159in}{2.517456in}}%
\pgfpathcurveto{\pgfqpoint{1.917159in}{2.530478in}}{\pgfqpoint{1.911985in}{2.542969in}}{\pgfqpoint{1.902776in}{2.552178in}}%
\pgfpathcurveto{\pgfqpoint{1.893568in}{2.561386in}}{\pgfqpoint{1.881077in}{2.566560in}}{\pgfqpoint{1.868054in}{2.566560in}}%
\pgfpathcurveto{\pgfqpoint{1.855031in}{2.566560in}}{\pgfqpoint{1.842540in}{2.561386in}}{\pgfqpoint{1.833332in}{2.552178in}}%
\pgfpathcurveto{\pgfqpoint{1.824123in}{2.542969in}}{\pgfqpoint{1.818949in}{2.530478in}}{\pgfqpoint{1.818949in}{2.517456in}}%
\pgfpathcurveto{\pgfqpoint{1.818949in}{2.504433in}}{\pgfqpoint{1.824123in}{2.491942in}}{\pgfqpoint{1.833332in}{2.482733in}}%
\pgfpathcurveto{\pgfqpoint{1.842540in}{2.473525in}}{\pgfqpoint{1.855031in}{2.468351in}}{\pgfqpoint{1.868054in}{2.468351in}}%
\pgfpathlineto{\pgfqpoint{1.868054in}{2.468351in}}%
\pgfpathclose%
\pgfusepath{stroke,fill}%
\end{pgfscope}%
\begin{pgfscope}%
\pgfpathrectangle{\pgfqpoint{0.194833in}{0.246946in}}{\pgfqpoint{6.160000in}{6.160000in}}%
\pgfusepath{clip}%
\pgfsetbuttcap%
\pgfsetroundjoin%
\definecolor{currentfill}{rgb}{0.121569,0.466667,0.705882}%
\pgfsetfillcolor{currentfill}%
\pgfsetlinewidth{1.003750pt}%
\definecolor{currentstroke}{rgb}{0.121569,0.466667,0.705882}%
\pgfsetstrokecolor{currentstroke}%
\pgfsetdash{}{0pt}%
\pgfpathmoveto{\pgfqpoint{2.963709in}{2.720698in}}%
\pgfpathcurveto{\pgfqpoint{2.976731in}{2.720698in}}{\pgfqpoint{2.989222in}{2.725872in}}{\pgfqpoint{2.998431in}{2.735080in}}%
\pgfpathcurveto{\pgfqpoint{3.007639in}{2.744289in}}{\pgfqpoint{3.012813in}{2.756780in}}{\pgfqpoint{3.012813in}{2.769803in}}%
\pgfpathcurveto{\pgfqpoint{3.012813in}{2.782825in}}{\pgfqpoint{3.007639in}{2.795316in}}{\pgfqpoint{2.998431in}{2.804525in}}%
\pgfpathcurveto{\pgfqpoint{2.989222in}{2.813733in}}{\pgfqpoint{2.976731in}{2.818907in}}{\pgfqpoint{2.963709in}{2.818907in}}%
\pgfpathcurveto{\pgfqpoint{2.950686in}{2.818907in}}{\pgfqpoint{2.938195in}{2.813733in}}{\pgfqpoint{2.928986in}{2.804525in}}%
\pgfpathcurveto{\pgfqpoint{2.919778in}{2.795316in}}{\pgfqpoint{2.914604in}{2.782825in}}{\pgfqpoint{2.914604in}{2.769803in}}%
\pgfpathcurveto{\pgfqpoint{2.914604in}{2.756780in}}{\pgfqpoint{2.919778in}{2.744289in}}{\pgfqpoint{2.928986in}{2.735080in}}%
\pgfpathcurveto{\pgfqpoint{2.938195in}{2.725872in}}{\pgfqpoint{2.950686in}{2.720698in}}{\pgfqpoint{2.963709in}{2.720698in}}%
\pgfpathlineto{\pgfqpoint{2.963709in}{2.720698in}}%
\pgfpathclose%
\pgfusepath{stroke,fill}%
\end{pgfscope}%
\begin{pgfscope}%
\pgfpathrectangle{\pgfqpoint{0.194833in}{0.246946in}}{\pgfqpoint{6.160000in}{6.160000in}}%
\pgfusepath{clip}%
\pgfsetbuttcap%
\pgfsetroundjoin%
\definecolor{currentfill}{rgb}{0.121569,0.466667,0.705882}%
\pgfsetfillcolor{currentfill}%
\pgfsetlinewidth{1.003750pt}%
\definecolor{currentstroke}{rgb}{0.121569,0.466667,0.705882}%
\pgfsetstrokecolor{currentstroke}%
\pgfsetdash{}{0pt}%
\pgfpathmoveto{\pgfqpoint{3.429662in}{2.733538in}}%
\pgfpathcurveto{\pgfqpoint{3.442684in}{2.733538in}}{\pgfqpoint{3.455175in}{2.738712in}}{\pgfqpoint{3.464384in}{2.747920in}}%
\pgfpathcurveto{\pgfqpoint{3.473592in}{2.757129in}}{\pgfqpoint{3.478766in}{2.769620in}}{\pgfqpoint{3.478766in}{2.782642in}}%
\pgfpathcurveto{\pgfqpoint{3.478766in}{2.795665in}}{\pgfqpoint{3.473592in}{2.808156in}}{\pgfqpoint{3.464384in}{2.817365in}}%
\pgfpathcurveto{\pgfqpoint{3.455175in}{2.826573in}}{\pgfqpoint{3.442684in}{2.831747in}}{\pgfqpoint{3.429662in}{2.831747in}}%
\pgfpathcurveto{\pgfqpoint{3.416639in}{2.831747in}}{\pgfqpoint{3.404148in}{2.826573in}}{\pgfqpoint{3.394939in}{2.817365in}}%
\pgfpathcurveto{\pgfqpoint{3.385731in}{2.808156in}}{\pgfqpoint{3.380557in}{2.795665in}}{\pgfqpoint{3.380557in}{2.782642in}}%
\pgfpathcurveto{\pgfqpoint{3.380557in}{2.769620in}}{\pgfqpoint{3.385731in}{2.757129in}}{\pgfqpoint{3.394939in}{2.747920in}}%
\pgfpathcurveto{\pgfqpoint{3.404148in}{2.738712in}}{\pgfqpoint{3.416639in}{2.733538in}}{\pgfqpoint{3.429662in}{2.733538in}}%
\pgfpathlineto{\pgfqpoint{3.429662in}{2.733538in}}%
\pgfpathclose%
\pgfusepath{stroke,fill}%
\end{pgfscope}%
\begin{pgfscope}%
\pgfpathrectangle{\pgfqpoint{0.194833in}{0.246946in}}{\pgfqpoint{6.160000in}{6.160000in}}%
\pgfusepath{clip}%
\pgfsetbuttcap%
\pgfsetroundjoin%
\definecolor{currentfill}{rgb}{0.121569,0.466667,0.705882}%
\pgfsetfillcolor{currentfill}%
\pgfsetlinewidth{1.003750pt}%
\definecolor{currentstroke}{rgb}{0.121569,0.466667,0.705882}%
\pgfsetstrokecolor{currentstroke}%
\pgfsetdash{}{0pt}%
\pgfpathmoveto{\pgfqpoint{5.126949in}{2.739552in}}%
\pgfpathcurveto{\pgfqpoint{5.139971in}{2.739552in}}{\pgfqpoint{5.152463in}{2.744726in}}{\pgfqpoint{5.161671in}{2.753935in}}%
\pgfpathcurveto{\pgfqpoint{5.170879in}{2.763143in}}{\pgfqpoint{5.176053in}{2.775634in}}{\pgfqpoint{5.176053in}{2.788657in}}%
\pgfpathcurveto{\pgfqpoint{5.176053in}{2.801680in}}{\pgfqpoint{5.170879in}{2.814171in}}{\pgfqpoint{5.161671in}{2.823379in}}%
\pgfpathcurveto{\pgfqpoint{5.152463in}{2.832588in}}{\pgfqpoint{5.139971in}{2.837762in}}{\pgfqpoint{5.126949in}{2.837762in}}%
\pgfpathcurveto{\pgfqpoint{5.113926in}{2.837762in}}{\pgfqpoint{5.101435in}{2.832588in}}{\pgfqpoint{5.092227in}{2.823379in}}%
\pgfpathcurveto{\pgfqpoint{5.083018in}{2.814171in}}{\pgfqpoint{5.077844in}{2.801680in}}{\pgfqpoint{5.077844in}{2.788657in}}%
\pgfpathcurveto{\pgfqpoint{5.077844in}{2.775634in}}{\pgfqpoint{5.083018in}{2.763143in}}{\pgfqpoint{5.092227in}{2.753935in}}%
\pgfpathcurveto{\pgfqpoint{5.101435in}{2.744726in}}{\pgfqpoint{5.113926in}{2.739552in}}{\pgfqpoint{5.126949in}{2.739552in}}%
\pgfpathlineto{\pgfqpoint{5.126949in}{2.739552in}}%
\pgfpathclose%
\pgfusepath{stroke,fill}%
\end{pgfscope}%
\begin{pgfscope}%
\pgfpathrectangle{\pgfqpoint{0.194833in}{0.246946in}}{\pgfqpoint{6.160000in}{6.160000in}}%
\pgfusepath{clip}%
\pgfsetbuttcap%
\pgfsetroundjoin%
\definecolor{currentfill}{rgb}{0.121569,0.466667,0.705882}%
\pgfsetfillcolor{currentfill}%
\pgfsetlinewidth{1.003750pt}%
\definecolor{currentstroke}{rgb}{0.121569,0.466667,0.705882}%
\pgfsetstrokecolor{currentstroke}%
\pgfsetdash{}{0pt}%
\pgfpathmoveto{\pgfqpoint{3.866526in}{2.741249in}}%
\pgfpathcurveto{\pgfqpoint{3.879549in}{2.741249in}}{\pgfqpoint{3.892040in}{2.746423in}}{\pgfqpoint{3.901248in}{2.755631in}}%
\pgfpathcurveto{\pgfqpoint{3.910457in}{2.764840in}}{\pgfqpoint{3.915631in}{2.777331in}}{\pgfqpoint{3.915631in}{2.790353in}}%
\pgfpathcurveto{\pgfqpoint{3.915631in}{2.803376in}}{\pgfqpoint{3.910457in}{2.815867in}}{\pgfqpoint{3.901248in}{2.825076in}}%
\pgfpathcurveto{\pgfqpoint{3.892040in}{2.834284in}}{\pgfqpoint{3.879549in}{2.839458in}}{\pgfqpoint{3.866526in}{2.839458in}}%
\pgfpathcurveto{\pgfqpoint{3.853503in}{2.839458in}}{\pgfqpoint{3.841012in}{2.834284in}}{\pgfqpoint{3.831804in}{2.825076in}}%
\pgfpathcurveto{\pgfqpoint{3.822595in}{2.815867in}}{\pgfqpoint{3.817421in}{2.803376in}}{\pgfqpoint{3.817421in}{2.790353in}}%
\pgfpathcurveto{\pgfqpoint{3.817421in}{2.777331in}}{\pgfqpoint{3.822595in}{2.764840in}}{\pgfqpoint{3.831804in}{2.755631in}}%
\pgfpathcurveto{\pgfqpoint{3.841012in}{2.746423in}}{\pgfqpoint{3.853503in}{2.741249in}}{\pgfqpoint{3.866526in}{2.741249in}}%
\pgfpathlineto{\pgfqpoint{3.866526in}{2.741249in}}%
\pgfpathclose%
\pgfusepath{stroke,fill}%
\end{pgfscope}%
\begin{pgfscope}%
\pgfpathrectangle{\pgfqpoint{0.194833in}{0.246946in}}{\pgfqpoint{6.160000in}{6.160000in}}%
\pgfusepath{clip}%
\pgfsetbuttcap%
\pgfsetroundjoin%
\definecolor{currentfill}{rgb}{0.121569,0.466667,0.705882}%
\pgfsetfillcolor{currentfill}%
\pgfsetlinewidth{1.003750pt}%
\definecolor{currentstroke}{rgb}{0.121569,0.466667,0.705882}%
\pgfsetstrokecolor{currentstroke}%
\pgfsetdash{}{0pt}%
\pgfpathmoveto{\pgfqpoint{4.701088in}{2.744822in}}%
\pgfpathcurveto{\pgfqpoint{4.714111in}{2.744822in}}{\pgfqpoint{4.726602in}{2.749996in}}{\pgfqpoint{4.735810in}{2.759204in}}%
\pgfpathcurveto{\pgfqpoint{4.745019in}{2.768412in}}{\pgfqpoint{4.750193in}{2.780904in}}{\pgfqpoint{4.750193in}{2.793926in}}%
\pgfpathcurveto{\pgfqpoint{4.750193in}{2.806949in}}{\pgfqpoint{4.745019in}{2.819440in}}{\pgfqpoint{4.735810in}{2.828648in}}%
\pgfpathcurveto{\pgfqpoint{4.726602in}{2.837857in}}{\pgfqpoint{4.714111in}{2.843031in}}{\pgfqpoint{4.701088in}{2.843031in}}%
\pgfpathcurveto{\pgfqpoint{4.688065in}{2.843031in}}{\pgfqpoint{4.675574in}{2.837857in}}{\pgfqpoint{4.666366in}{2.828648in}}%
\pgfpathcurveto{\pgfqpoint{4.657157in}{2.819440in}}{\pgfqpoint{4.651983in}{2.806949in}}{\pgfqpoint{4.651983in}{2.793926in}}%
\pgfpathcurveto{\pgfqpoint{4.651983in}{2.780904in}}{\pgfqpoint{4.657157in}{2.768412in}}{\pgfqpoint{4.666366in}{2.759204in}}%
\pgfpathcurveto{\pgfqpoint{4.675574in}{2.749996in}}{\pgfqpoint{4.688065in}{2.744822in}}{\pgfqpoint{4.701088in}{2.744822in}}%
\pgfpathlineto{\pgfqpoint{4.701088in}{2.744822in}}%
\pgfpathclose%
\pgfusepath{stroke,fill}%
\end{pgfscope}%
\begin{pgfscope}%
\pgfpathrectangle{\pgfqpoint{0.194833in}{0.246946in}}{\pgfqpoint{6.160000in}{6.160000in}}%
\pgfusepath{clip}%
\pgfsetbuttcap%
\pgfsetroundjoin%
\definecolor{currentfill}{rgb}{0.121569,0.466667,0.705882}%
\pgfsetfillcolor{currentfill}%
\pgfsetlinewidth{1.003750pt}%
\definecolor{currentstroke}{rgb}{0.121569,0.466667,0.705882}%
\pgfsetstrokecolor{currentstroke}%
\pgfsetdash{}{0pt}%
\pgfpathmoveto{\pgfqpoint{2.498856in}{2.746591in}}%
\pgfpathcurveto{\pgfqpoint{2.511879in}{2.746591in}}{\pgfqpoint{2.524370in}{2.751765in}}{\pgfqpoint{2.533578in}{2.760973in}}%
\pgfpathcurveto{\pgfqpoint{2.542787in}{2.770182in}}{\pgfqpoint{2.547961in}{2.782673in}}{\pgfqpoint{2.547961in}{2.795696in}}%
\pgfpathcurveto{\pgfqpoint{2.547961in}{2.808718in}}{\pgfqpoint{2.542787in}{2.821209in}}{\pgfqpoint{2.533578in}{2.830418in}}%
\pgfpathcurveto{\pgfqpoint{2.524370in}{2.839626in}}{\pgfqpoint{2.511879in}{2.844800in}}{\pgfqpoint{2.498856in}{2.844800in}}%
\pgfpathcurveto{\pgfqpoint{2.485833in}{2.844800in}}{\pgfqpoint{2.473342in}{2.839626in}}{\pgfqpoint{2.464134in}{2.830418in}}%
\pgfpathcurveto{\pgfqpoint{2.454925in}{2.821209in}}{\pgfqpoint{2.449751in}{2.808718in}}{\pgfqpoint{2.449751in}{2.795696in}}%
\pgfpathcurveto{\pgfqpoint{2.449751in}{2.782673in}}{\pgfqpoint{2.454925in}{2.770182in}}{\pgfqpoint{2.464134in}{2.760973in}}%
\pgfpathcurveto{\pgfqpoint{2.473342in}{2.751765in}}{\pgfqpoint{2.485833in}{2.746591in}}{\pgfqpoint{2.498856in}{2.746591in}}%
\pgfpathlineto{\pgfqpoint{2.498856in}{2.746591in}}%
\pgfpathclose%
\pgfusepath{stroke,fill}%
\end{pgfscope}%
\begin{pgfscope}%
\pgfpathrectangle{\pgfqpoint{0.194833in}{0.246946in}}{\pgfqpoint{6.160000in}{6.160000in}}%
\pgfusepath{clip}%
\pgfsetbuttcap%
\pgfsetroundjoin%
\definecolor{currentfill}{rgb}{0.121569,0.466667,0.705882}%
\pgfsetfillcolor{currentfill}%
\pgfsetlinewidth{1.003750pt}%
\definecolor{currentstroke}{rgb}{0.121569,0.466667,0.705882}%
\pgfsetstrokecolor{currentstroke}%
\pgfsetdash{}{0pt}%
\pgfpathmoveto{\pgfqpoint{1.608630in}{2.752723in}}%
\pgfpathcurveto{\pgfqpoint{1.621653in}{2.752723in}}{\pgfqpoint{1.634144in}{2.757897in}}{\pgfqpoint{1.643353in}{2.767105in}}%
\pgfpathcurveto{\pgfqpoint{1.652561in}{2.776314in}}{\pgfqpoint{1.657735in}{2.788805in}}{\pgfqpoint{1.657735in}{2.801827in}}%
\pgfpathcurveto{\pgfqpoint{1.657735in}{2.814850in}}{\pgfqpoint{1.652561in}{2.827341in}}{\pgfqpoint{1.643353in}{2.836550in}}%
\pgfpathcurveto{\pgfqpoint{1.634144in}{2.845758in}}{\pgfqpoint{1.621653in}{2.850932in}}{\pgfqpoint{1.608630in}{2.850932in}}%
\pgfpathcurveto{\pgfqpoint{1.595608in}{2.850932in}}{\pgfqpoint{1.583117in}{2.845758in}}{\pgfqpoint{1.573908in}{2.836550in}}%
\pgfpathcurveto{\pgfqpoint{1.564700in}{2.827341in}}{\pgfqpoint{1.559526in}{2.814850in}}{\pgfqpoint{1.559526in}{2.801827in}}%
\pgfpathcurveto{\pgfqpoint{1.559526in}{2.788805in}}{\pgfqpoint{1.564700in}{2.776314in}}{\pgfqpoint{1.573908in}{2.767105in}}%
\pgfpathcurveto{\pgfqpoint{1.583117in}{2.757897in}}{\pgfqpoint{1.595608in}{2.752723in}}{\pgfqpoint{1.608630in}{2.752723in}}%
\pgfpathlineto{\pgfqpoint{1.608630in}{2.752723in}}%
\pgfpathclose%
\pgfusepath{stroke,fill}%
\end{pgfscope}%
\begin{pgfscope}%
\pgfpathrectangle{\pgfqpoint{0.194833in}{0.246946in}}{\pgfqpoint{6.160000in}{6.160000in}}%
\pgfusepath{clip}%
\pgfsetbuttcap%
\pgfsetroundjoin%
\definecolor{currentfill}{rgb}{0.121569,0.466667,0.705882}%
\pgfsetfillcolor{currentfill}%
\pgfsetlinewidth{1.003750pt}%
\definecolor{currentstroke}{rgb}{0.121569,0.466667,0.705882}%
\pgfsetstrokecolor{currentstroke}%
\pgfsetdash{}{0pt}%
\pgfpathmoveto{\pgfqpoint{4.285737in}{2.753257in}}%
\pgfpathcurveto{\pgfqpoint{4.298759in}{2.753257in}}{\pgfqpoint{4.311250in}{2.758431in}}{\pgfqpoint{4.320459in}{2.767639in}}%
\pgfpathcurveto{\pgfqpoint{4.329667in}{2.776848in}}{\pgfqpoint{4.334841in}{2.789339in}}{\pgfqpoint{4.334841in}{2.802361in}}%
\pgfpathcurveto{\pgfqpoint{4.334841in}{2.815384in}}{\pgfqpoint{4.329667in}{2.827875in}}{\pgfqpoint{4.320459in}{2.837084in}}%
\pgfpathcurveto{\pgfqpoint{4.311250in}{2.846292in}}{\pgfqpoint{4.298759in}{2.851466in}}{\pgfqpoint{4.285737in}{2.851466in}}%
\pgfpathcurveto{\pgfqpoint{4.272714in}{2.851466in}}{\pgfqpoint{4.260223in}{2.846292in}}{\pgfqpoint{4.251014in}{2.837084in}}%
\pgfpathcurveto{\pgfqpoint{4.241806in}{2.827875in}}{\pgfqpoint{4.236632in}{2.815384in}}{\pgfqpoint{4.236632in}{2.802361in}}%
\pgfpathcurveto{\pgfqpoint{4.236632in}{2.789339in}}{\pgfqpoint{4.241806in}{2.776848in}}{\pgfqpoint{4.251014in}{2.767639in}}%
\pgfpathcurveto{\pgfqpoint{4.260223in}{2.758431in}}{\pgfqpoint{4.272714in}{2.753257in}}{\pgfqpoint{4.285737in}{2.753257in}}%
\pgfpathlineto{\pgfqpoint{4.285737in}{2.753257in}}%
\pgfpathclose%
\pgfusepath{stroke,fill}%
\end{pgfscope}%
\begin{pgfscope}%
\pgfpathrectangle{\pgfqpoint{0.194833in}{0.246946in}}{\pgfqpoint{6.160000in}{6.160000in}}%
\pgfusepath{clip}%
\pgfsetbuttcap%
\pgfsetroundjoin%
\definecolor{currentfill}{rgb}{0.121569,0.466667,0.705882}%
\pgfsetfillcolor{currentfill}%
\pgfsetlinewidth{1.003750pt}%
\definecolor{currentstroke}{rgb}{0.121569,0.466667,0.705882}%
\pgfsetstrokecolor{currentstroke}%
\pgfsetdash{}{0pt}%
\pgfpathmoveto{\pgfqpoint{2.052955in}{2.754873in}}%
\pgfpathcurveto{\pgfqpoint{2.065977in}{2.754873in}}{\pgfqpoint{2.078468in}{2.760047in}}{\pgfqpoint{2.087677in}{2.769255in}}%
\pgfpathcurveto{\pgfqpoint{2.096885in}{2.778464in}}{\pgfqpoint{2.102059in}{2.790955in}}{\pgfqpoint{2.102059in}{2.803977in}}%
\pgfpathcurveto{\pgfqpoint{2.102059in}{2.817000in}}{\pgfqpoint{2.096885in}{2.829491in}}{\pgfqpoint{2.087677in}{2.838700in}}%
\pgfpathcurveto{\pgfqpoint{2.078468in}{2.847908in}}{\pgfqpoint{2.065977in}{2.853082in}}{\pgfqpoint{2.052955in}{2.853082in}}%
\pgfpathcurveto{\pgfqpoint{2.039932in}{2.853082in}}{\pgfqpoint{2.027441in}{2.847908in}}{\pgfqpoint{2.018232in}{2.838700in}}%
\pgfpathcurveto{\pgfqpoint{2.009024in}{2.829491in}}{\pgfqpoint{2.003850in}{2.817000in}}{\pgfqpoint{2.003850in}{2.803977in}}%
\pgfpathcurveto{\pgfqpoint{2.003850in}{2.790955in}}{\pgfqpoint{2.009024in}{2.778464in}}{\pgfqpoint{2.018232in}{2.769255in}}%
\pgfpathcurveto{\pgfqpoint{2.027441in}{2.760047in}}{\pgfqpoint{2.039932in}{2.754873in}}{\pgfqpoint{2.052955in}{2.754873in}}%
\pgfpathlineto{\pgfqpoint{2.052955in}{2.754873in}}%
\pgfpathclose%
\pgfusepath{stroke,fill}%
\end{pgfscope}%
\begin{pgfscope}%
\pgfpathrectangle{\pgfqpoint{0.194833in}{0.246946in}}{\pgfqpoint{6.160000in}{6.160000in}}%
\pgfusepath{clip}%
\pgfsetbuttcap%
\pgfsetroundjoin%
\definecolor{currentfill}{rgb}{0.121569,0.466667,0.705882}%
\pgfsetfillcolor{currentfill}%
\pgfsetlinewidth{1.003750pt}%
\definecolor{currentstroke}{rgb}{0.121569,0.466667,0.705882}%
\pgfsetstrokecolor{currentstroke}%
\pgfsetdash{}{0pt}%
\pgfpathmoveto{\pgfqpoint{5.540458in}{2.757868in}}%
\pgfpathcurveto{\pgfqpoint{5.553481in}{2.757868in}}{\pgfqpoint{5.565972in}{2.763042in}}{\pgfqpoint{5.575180in}{2.772250in}}%
\pgfpathcurveto{\pgfqpoint{5.584389in}{2.781458in}}{\pgfqpoint{5.589563in}{2.793950in}}{\pgfqpoint{5.589563in}{2.806972in}}%
\pgfpathcurveto{\pgfqpoint{5.589563in}{2.819995in}}{\pgfqpoint{5.584389in}{2.832486in}}{\pgfqpoint{5.575180in}{2.841694in}}%
\pgfpathcurveto{\pgfqpoint{5.565972in}{2.850903in}}{\pgfqpoint{5.553481in}{2.856077in}}{\pgfqpoint{5.540458in}{2.856077in}}%
\pgfpathcurveto{\pgfqpoint{5.527435in}{2.856077in}}{\pgfqpoint{5.514944in}{2.850903in}}{\pgfqpoint{5.505736in}{2.841694in}}%
\pgfpathcurveto{\pgfqpoint{5.496528in}{2.832486in}}{\pgfqpoint{5.491354in}{2.819995in}}{\pgfqpoint{5.491354in}{2.806972in}}%
\pgfpathcurveto{\pgfqpoint{5.491354in}{2.793950in}}{\pgfqpoint{5.496528in}{2.781458in}}{\pgfqpoint{5.505736in}{2.772250in}}%
\pgfpathcurveto{\pgfqpoint{5.514944in}{2.763042in}}{\pgfqpoint{5.527435in}{2.757868in}}{\pgfqpoint{5.540458in}{2.757868in}}%
\pgfpathlineto{\pgfqpoint{5.540458in}{2.757868in}}%
\pgfpathclose%
\pgfusepath{stroke,fill}%
\end{pgfscope}%
\begin{pgfscope}%
\pgfpathrectangle{\pgfqpoint{0.194833in}{0.246946in}}{\pgfqpoint{6.160000in}{6.160000in}}%
\pgfusepath{clip}%
\pgfsetbuttcap%
\pgfsetroundjoin%
\definecolor{currentfill}{rgb}{0.121569,0.466667,0.705882}%
\pgfsetfillcolor{currentfill}%
\pgfsetlinewidth{1.003750pt}%
\definecolor{currentstroke}{rgb}{0.121569,0.466667,0.705882}%
\pgfsetstrokecolor{currentstroke}%
\pgfsetdash{}{0pt}%
\pgfpathmoveto{\pgfqpoint{1.178852in}{2.762479in}}%
\pgfpathcurveto{\pgfqpoint{1.191874in}{2.762479in}}{\pgfqpoint{1.204365in}{2.767653in}}{\pgfqpoint{1.213574in}{2.776862in}}%
\pgfpathcurveto{\pgfqpoint{1.222782in}{2.786070in}}{\pgfqpoint{1.227956in}{2.798561in}}{\pgfqpoint{1.227956in}{2.811584in}}%
\pgfpathcurveto{\pgfqpoint{1.227956in}{2.824607in}}{\pgfqpoint{1.222782in}{2.837098in}}{\pgfqpoint{1.213574in}{2.846306in}}%
\pgfpathcurveto{\pgfqpoint{1.204365in}{2.855515in}}{\pgfqpoint{1.191874in}{2.860689in}}{\pgfqpoint{1.178852in}{2.860689in}}%
\pgfpathcurveto{\pgfqpoint{1.165829in}{2.860689in}}{\pgfqpoint{1.153338in}{2.855515in}}{\pgfqpoint{1.144129in}{2.846306in}}%
\pgfpathcurveto{\pgfqpoint{1.134921in}{2.837098in}}{\pgfqpoint{1.129747in}{2.824607in}}{\pgfqpoint{1.129747in}{2.811584in}}%
\pgfpathcurveto{\pgfqpoint{1.129747in}{2.798561in}}{\pgfqpoint{1.134921in}{2.786070in}}{\pgfqpoint{1.144129in}{2.776862in}}%
\pgfpathcurveto{\pgfqpoint{1.153338in}{2.767653in}}{\pgfqpoint{1.165829in}{2.762479in}}{\pgfqpoint{1.178852in}{2.762479in}}%
\pgfpathlineto{\pgfqpoint{1.178852in}{2.762479in}}%
\pgfpathclose%
\pgfusepath{stroke,fill}%
\end{pgfscope}%
\begin{pgfscope}%
\pgfpathrectangle{\pgfqpoint{0.194833in}{0.246946in}}{\pgfqpoint{6.160000in}{6.160000in}}%
\pgfusepath{clip}%
\pgfsetbuttcap%
\pgfsetroundjoin%
\definecolor{currentfill}{rgb}{0.121569,0.466667,0.705882}%
\pgfsetfillcolor{currentfill}%
\pgfsetlinewidth{1.003750pt}%
\definecolor{currentstroke}{rgb}{0.121569,0.466667,0.705882}%
\pgfsetstrokecolor{currentstroke}%
\pgfsetdash{}{0pt}%
\pgfpathmoveto{\pgfqpoint{3.201090in}{2.992094in}}%
\pgfpathcurveto{\pgfqpoint{3.214112in}{2.992094in}}{\pgfqpoint{3.226603in}{2.997268in}}{\pgfqpoint{3.235812in}{3.006476in}}%
\pgfpathcurveto{\pgfqpoint{3.245020in}{3.015685in}}{\pgfqpoint{3.250194in}{3.028176in}}{\pgfqpoint{3.250194in}{3.041198in}}%
\pgfpathcurveto{\pgfqpoint{3.250194in}{3.054221in}}{\pgfqpoint{3.245020in}{3.066712in}}{\pgfqpoint{3.235812in}{3.075921in}}%
\pgfpathcurveto{\pgfqpoint{3.226603in}{3.085129in}}{\pgfqpoint{3.214112in}{3.090303in}}{\pgfqpoint{3.201090in}{3.090303in}}%
\pgfpathcurveto{\pgfqpoint{3.188067in}{3.090303in}}{\pgfqpoint{3.175576in}{3.085129in}}{\pgfqpoint{3.166367in}{3.075921in}}%
\pgfpathcurveto{\pgfqpoint{3.157159in}{3.066712in}}{\pgfqpoint{3.151985in}{3.054221in}}{\pgfqpoint{3.151985in}{3.041198in}}%
\pgfpathcurveto{\pgfqpoint{3.151985in}{3.028176in}}{\pgfqpoint{3.157159in}{3.015685in}}{\pgfqpoint{3.166367in}{3.006476in}}%
\pgfpathcurveto{\pgfqpoint{3.175576in}{2.997268in}}{\pgfqpoint{3.188067in}{2.992094in}}{\pgfqpoint{3.201090in}{2.992094in}}%
\pgfpathlineto{\pgfqpoint{3.201090in}{2.992094in}}%
\pgfpathclose%
\pgfusepath{stroke,fill}%
\end{pgfscope}%
\begin{pgfscope}%
\pgfpathrectangle{\pgfqpoint{0.194833in}{0.246946in}}{\pgfqpoint{6.160000in}{6.160000in}}%
\pgfusepath{clip}%
\pgfsetbuttcap%
\pgfsetroundjoin%
\definecolor{currentfill}{rgb}{0.121569,0.466667,0.705882}%
\pgfsetfillcolor{currentfill}%
\pgfsetlinewidth{1.003750pt}%
\definecolor{currentstroke}{rgb}{0.121569,0.466667,0.705882}%
\pgfsetstrokecolor{currentstroke}%
\pgfsetdash{}{0pt}%
\pgfpathmoveto{\pgfqpoint{2.754077in}{2.998123in}}%
\pgfpathcurveto{\pgfqpoint{2.767100in}{2.998123in}}{\pgfqpoint{2.779591in}{3.003297in}}{\pgfqpoint{2.788800in}{3.012505in}}%
\pgfpathcurveto{\pgfqpoint{2.798008in}{3.021714in}}{\pgfqpoint{2.803182in}{3.034205in}}{\pgfqpoint{2.803182in}{3.047227in}}%
\pgfpathcurveto{\pgfqpoint{2.803182in}{3.060250in}}{\pgfqpoint{2.798008in}{3.072741in}}{\pgfqpoint{2.788800in}{3.081950in}}%
\pgfpathcurveto{\pgfqpoint{2.779591in}{3.091158in}}{\pgfqpoint{2.767100in}{3.096332in}}{\pgfqpoint{2.754077in}{3.096332in}}%
\pgfpathcurveto{\pgfqpoint{2.741055in}{3.096332in}}{\pgfqpoint{2.728564in}{3.091158in}}{\pgfqpoint{2.719355in}{3.081950in}}%
\pgfpathcurveto{\pgfqpoint{2.710147in}{3.072741in}}{\pgfqpoint{2.704973in}{3.060250in}}{\pgfqpoint{2.704973in}{3.047227in}}%
\pgfpathcurveto{\pgfqpoint{2.704973in}{3.034205in}}{\pgfqpoint{2.710147in}{3.021714in}}{\pgfqpoint{2.719355in}{3.012505in}}%
\pgfpathcurveto{\pgfqpoint{2.728564in}{3.003297in}}{\pgfqpoint{2.741055in}{2.998123in}}{\pgfqpoint{2.754077in}{2.998123in}}%
\pgfpathlineto{\pgfqpoint{2.754077in}{2.998123in}}%
\pgfpathclose%
\pgfusepath{stroke,fill}%
\end{pgfscope}%
\begin{pgfscope}%
\pgfpathrectangle{\pgfqpoint{0.194833in}{0.246946in}}{\pgfqpoint{6.160000in}{6.160000in}}%
\pgfusepath{clip}%
\pgfsetbuttcap%
\pgfsetroundjoin%
\definecolor{currentfill}{rgb}{0.121569,0.466667,0.705882}%
\pgfsetfillcolor{currentfill}%
\pgfsetlinewidth{1.003750pt}%
\definecolor{currentstroke}{rgb}{0.121569,0.466667,0.705882}%
\pgfsetstrokecolor{currentstroke}%
\pgfsetdash{}{0pt}%
\pgfpathmoveto{\pgfqpoint{3.650011in}{3.007005in}}%
\pgfpathcurveto{\pgfqpoint{3.663034in}{3.007005in}}{\pgfqpoint{3.675525in}{3.012179in}}{\pgfqpoint{3.684733in}{3.021387in}}%
\pgfpathcurveto{\pgfqpoint{3.693942in}{3.030596in}}{\pgfqpoint{3.699116in}{3.043087in}}{\pgfqpoint{3.699116in}{3.056110in}}%
\pgfpathcurveto{\pgfqpoint{3.699116in}{3.069132in}}{\pgfqpoint{3.693942in}{3.081623in}}{\pgfqpoint{3.684733in}{3.090832in}}%
\pgfpathcurveto{\pgfqpoint{3.675525in}{3.100040in}}{\pgfqpoint{3.663034in}{3.105214in}}{\pgfqpoint{3.650011in}{3.105214in}}%
\pgfpathcurveto{\pgfqpoint{3.636988in}{3.105214in}}{\pgfqpoint{3.624497in}{3.100040in}}{\pgfqpoint{3.615289in}{3.090832in}}%
\pgfpathcurveto{\pgfqpoint{3.606080in}{3.081623in}}{\pgfqpoint{3.600906in}{3.069132in}}{\pgfqpoint{3.600906in}{3.056110in}}%
\pgfpathcurveto{\pgfqpoint{3.600906in}{3.043087in}}{\pgfqpoint{3.606080in}{3.030596in}}{\pgfqpoint{3.615289in}{3.021387in}}%
\pgfpathcurveto{\pgfqpoint{3.624497in}{3.012179in}}{\pgfqpoint{3.636988in}{3.007005in}}{\pgfqpoint{3.650011in}{3.007005in}}%
\pgfpathlineto{\pgfqpoint{3.650011in}{3.007005in}}%
\pgfpathclose%
\pgfusepath{stroke,fill}%
\end{pgfscope}%
\begin{pgfscope}%
\pgfpathrectangle{\pgfqpoint{0.194833in}{0.246946in}}{\pgfqpoint{6.160000in}{6.160000in}}%
\pgfusepath{clip}%
\pgfsetbuttcap%
\pgfsetroundjoin%
\definecolor{currentfill}{rgb}{0.121569,0.466667,0.705882}%
\pgfsetfillcolor{currentfill}%
\pgfsetlinewidth{1.003750pt}%
\definecolor{currentstroke}{rgb}{0.121569,0.466667,0.705882}%
\pgfsetstrokecolor{currentstroke}%
\pgfsetdash{}{0pt}%
\pgfpathmoveto{\pgfqpoint{4.943998in}{3.013994in}}%
\pgfpathcurveto{\pgfqpoint{4.957021in}{3.013994in}}{\pgfqpoint{4.969512in}{3.019168in}}{\pgfqpoint{4.978720in}{3.028377in}}%
\pgfpathcurveto{\pgfqpoint{4.987929in}{3.037585in}}{\pgfqpoint{4.993102in}{3.050076in}}{\pgfqpoint{4.993102in}{3.063099in}}%
\pgfpathcurveto{\pgfqpoint{4.993102in}{3.076122in}}{\pgfqpoint{4.987929in}{3.088613in}}{\pgfqpoint{4.978720in}{3.097821in}}%
\pgfpathcurveto{\pgfqpoint{4.969512in}{3.107030in}}{\pgfqpoint{4.957021in}{3.112204in}}{\pgfqpoint{4.943998in}{3.112204in}}%
\pgfpathcurveto{\pgfqpoint{4.930975in}{3.112204in}}{\pgfqpoint{4.918484in}{3.107030in}}{\pgfqpoint{4.909276in}{3.097821in}}%
\pgfpathcurveto{\pgfqpoint{4.900067in}{3.088613in}}{\pgfqpoint{4.894893in}{3.076122in}}{\pgfqpoint{4.894893in}{3.063099in}}%
\pgfpathcurveto{\pgfqpoint{4.894893in}{3.050076in}}{\pgfqpoint{4.900067in}{3.037585in}}{\pgfqpoint{4.909276in}{3.028377in}}%
\pgfpathcurveto{\pgfqpoint{4.918484in}{3.019168in}}{\pgfqpoint{4.930975in}{3.013994in}}{\pgfqpoint{4.943998in}{3.013994in}}%
\pgfpathlineto{\pgfqpoint{4.943998in}{3.013994in}}%
\pgfpathclose%
\pgfusepath{stroke,fill}%
\end{pgfscope}%
\begin{pgfscope}%
\pgfpathrectangle{\pgfqpoint{0.194833in}{0.246946in}}{\pgfqpoint{6.160000in}{6.160000in}}%
\pgfusepath{clip}%
\pgfsetbuttcap%
\pgfsetroundjoin%
\definecolor{currentfill}{rgb}{0.121569,0.466667,0.705882}%
\pgfsetfillcolor{currentfill}%
\pgfsetlinewidth{1.003750pt}%
\definecolor{currentstroke}{rgb}{0.121569,0.466667,0.705882}%
\pgfsetstrokecolor{currentstroke}%
\pgfsetdash{}{0pt}%
\pgfpathmoveto{\pgfqpoint{2.277680in}{3.020784in}}%
\pgfpathcurveto{\pgfqpoint{2.290703in}{3.020784in}}{\pgfqpoint{2.303194in}{3.025958in}}{\pgfqpoint{2.312402in}{3.035166in}}%
\pgfpathcurveto{\pgfqpoint{2.321611in}{3.044374in}}{\pgfqpoint{2.326785in}{3.056866in}}{\pgfqpoint{2.326785in}{3.069888in}}%
\pgfpathcurveto{\pgfqpoint{2.326785in}{3.082911in}}{\pgfqpoint{2.321611in}{3.095402in}}{\pgfqpoint{2.312402in}{3.104610in}}%
\pgfpathcurveto{\pgfqpoint{2.303194in}{3.113819in}}{\pgfqpoint{2.290703in}{3.118993in}}{\pgfqpoint{2.277680in}{3.118993in}}%
\pgfpathcurveto{\pgfqpoint{2.264658in}{3.118993in}}{\pgfqpoint{2.252166in}{3.113819in}}{\pgfqpoint{2.242958in}{3.104610in}}%
\pgfpathcurveto{\pgfqpoint{2.233750in}{3.095402in}}{\pgfqpoint{2.228576in}{3.082911in}}{\pgfqpoint{2.228576in}{3.069888in}}%
\pgfpathcurveto{\pgfqpoint{2.228576in}{3.056866in}}{\pgfqpoint{2.233750in}{3.044374in}}{\pgfqpoint{2.242958in}{3.035166in}}%
\pgfpathcurveto{\pgfqpoint{2.252166in}{3.025958in}}{\pgfqpoint{2.264658in}{3.020784in}}{\pgfqpoint{2.277680in}{3.020784in}}%
\pgfpathlineto{\pgfqpoint{2.277680in}{3.020784in}}%
\pgfpathclose%
\pgfusepath{stroke,fill}%
\end{pgfscope}%
\begin{pgfscope}%
\pgfpathrectangle{\pgfqpoint{0.194833in}{0.246946in}}{\pgfqpoint{6.160000in}{6.160000in}}%
\pgfusepath{clip}%
\pgfsetbuttcap%
\pgfsetroundjoin%
\definecolor{currentfill}{rgb}{0.121569,0.466667,0.705882}%
\pgfsetfillcolor{currentfill}%
\pgfsetlinewidth{1.003750pt}%
\definecolor{currentstroke}{rgb}{0.121569,0.466667,0.705882}%
\pgfsetstrokecolor{currentstroke}%
\pgfsetdash{}{0pt}%
\pgfpathmoveto{\pgfqpoint{4.080792in}{3.021669in}}%
\pgfpathcurveto{\pgfqpoint{4.093814in}{3.021669in}}{\pgfqpoint{4.106305in}{3.026843in}}{\pgfqpoint{4.115514in}{3.036052in}}%
\pgfpathcurveto{\pgfqpoint{4.124722in}{3.045260in}}{\pgfqpoint{4.129896in}{3.057751in}}{\pgfqpoint{4.129896in}{3.070774in}}%
\pgfpathcurveto{\pgfqpoint{4.129896in}{3.083796in}}{\pgfqpoint{4.124722in}{3.096288in}}{\pgfqpoint{4.115514in}{3.105496in}}%
\pgfpathcurveto{\pgfqpoint{4.106305in}{3.114704in}}{\pgfqpoint{4.093814in}{3.119878in}}{\pgfqpoint{4.080792in}{3.119878in}}%
\pgfpathcurveto{\pgfqpoint{4.067769in}{3.119878in}}{\pgfqpoint{4.055278in}{3.114704in}}{\pgfqpoint{4.046069in}{3.105496in}}%
\pgfpathcurveto{\pgfqpoint{4.036861in}{3.096288in}}{\pgfqpoint{4.031687in}{3.083796in}}{\pgfqpoint{4.031687in}{3.070774in}}%
\pgfpathcurveto{\pgfqpoint{4.031687in}{3.057751in}}{\pgfqpoint{4.036861in}{3.045260in}}{\pgfqpoint{4.046069in}{3.036052in}}%
\pgfpathcurveto{\pgfqpoint{4.055278in}{3.026843in}}{\pgfqpoint{4.067769in}{3.021669in}}{\pgfqpoint{4.080792in}{3.021669in}}%
\pgfpathlineto{\pgfqpoint{4.080792in}{3.021669in}}%
\pgfpathclose%
\pgfusepath{stroke,fill}%
\end{pgfscope}%
\begin{pgfscope}%
\pgfpathrectangle{\pgfqpoint{0.194833in}{0.246946in}}{\pgfqpoint{6.160000in}{6.160000in}}%
\pgfusepath{clip}%
\pgfsetbuttcap%
\pgfsetroundjoin%
\definecolor{currentfill}{rgb}{0.121569,0.466667,0.705882}%
\pgfsetfillcolor{currentfill}%
\pgfsetlinewidth{1.003750pt}%
\definecolor{currentstroke}{rgb}{0.121569,0.466667,0.705882}%
\pgfsetstrokecolor{currentstroke}%
\pgfsetdash{}{0pt}%
\pgfpathmoveto{\pgfqpoint{4.512778in}{3.024454in}}%
\pgfpathcurveto{\pgfqpoint{4.525800in}{3.024454in}}{\pgfqpoint{4.538292in}{3.029628in}}{\pgfqpoint{4.547500in}{3.038837in}}%
\pgfpathcurveto{\pgfqpoint{4.556708in}{3.048045in}}{\pgfqpoint{4.561882in}{3.060536in}}{\pgfqpoint{4.561882in}{3.073559in}}%
\pgfpathcurveto{\pgfqpoint{4.561882in}{3.086582in}}{\pgfqpoint{4.556708in}{3.099073in}}{\pgfqpoint{4.547500in}{3.108281in}}%
\pgfpathcurveto{\pgfqpoint{4.538292in}{3.117490in}}{\pgfqpoint{4.525800in}{3.122663in}}{\pgfqpoint{4.512778in}{3.122663in}}%
\pgfpathcurveto{\pgfqpoint{4.499755in}{3.122663in}}{\pgfqpoint{4.487264in}{3.117490in}}{\pgfqpoint{4.478056in}{3.108281in}}%
\pgfpathcurveto{\pgfqpoint{4.468847in}{3.099073in}}{\pgfqpoint{4.463673in}{3.086582in}}{\pgfqpoint{4.463673in}{3.073559in}}%
\pgfpathcurveto{\pgfqpoint{4.463673in}{3.060536in}}{\pgfqpoint{4.468847in}{3.048045in}}{\pgfqpoint{4.478056in}{3.038837in}}%
\pgfpathcurveto{\pgfqpoint{4.487264in}{3.029628in}}{\pgfqpoint{4.499755in}{3.024454in}}{\pgfqpoint{4.512778in}{3.024454in}}%
\pgfpathlineto{\pgfqpoint{4.512778in}{3.024454in}}%
\pgfpathclose%
\pgfusepath{stroke,fill}%
\end{pgfscope}%
\begin{pgfscope}%
\pgfpathrectangle{\pgfqpoint{0.194833in}{0.246946in}}{\pgfqpoint{6.160000in}{6.160000in}}%
\pgfusepath{clip}%
\pgfsetbuttcap%
\pgfsetroundjoin%
\definecolor{currentfill}{rgb}{0.121569,0.466667,0.705882}%
\pgfsetfillcolor{currentfill}%
\pgfsetlinewidth{1.003750pt}%
\definecolor{currentstroke}{rgb}{0.121569,0.466667,0.705882}%
\pgfsetstrokecolor{currentstroke}%
\pgfsetdash{}{0pt}%
\pgfpathmoveto{\pgfqpoint{1.811886in}{3.025592in}}%
\pgfpathcurveto{\pgfqpoint{1.824909in}{3.025592in}}{\pgfqpoint{1.837400in}{3.030766in}}{\pgfqpoint{1.846608in}{3.039974in}}%
\pgfpathcurveto{\pgfqpoint{1.855817in}{3.049183in}}{\pgfqpoint{1.860991in}{3.061674in}}{\pgfqpoint{1.860991in}{3.074697in}}%
\pgfpathcurveto{\pgfqpoint{1.860991in}{3.087719in}}{\pgfqpoint{1.855817in}{3.100210in}}{\pgfqpoint{1.846608in}{3.109419in}}%
\pgfpathcurveto{\pgfqpoint{1.837400in}{3.118627in}}{\pgfqpoint{1.824909in}{3.123801in}}{\pgfqpoint{1.811886in}{3.123801in}}%
\pgfpathcurveto{\pgfqpoint{1.798863in}{3.123801in}}{\pgfqpoint{1.786372in}{3.118627in}}{\pgfqpoint{1.777164in}{3.109419in}}%
\pgfpathcurveto{\pgfqpoint{1.767955in}{3.100210in}}{\pgfqpoint{1.762781in}{3.087719in}}{\pgfqpoint{1.762781in}{3.074697in}}%
\pgfpathcurveto{\pgfqpoint{1.762781in}{3.061674in}}{\pgfqpoint{1.767955in}{3.049183in}}{\pgfqpoint{1.777164in}{3.039974in}}%
\pgfpathcurveto{\pgfqpoint{1.786372in}{3.030766in}}{\pgfqpoint{1.798863in}{3.025592in}}{\pgfqpoint{1.811886in}{3.025592in}}%
\pgfpathlineto{\pgfqpoint{1.811886in}{3.025592in}}%
\pgfpathclose%
\pgfusepath{stroke,fill}%
\end{pgfscope}%
\begin{pgfscope}%
\pgfpathrectangle{\pgfqpoint{0.194833in}{0.246946in}}{\pgfqpoint{6.160000in}{6.160000in}}%
\pgfusepath{clip}%
\pgfsetbuttcap%
\pgfsetroundjoin%
\definecolor{currentfill}{rgb}{0.121569,0.466667,0.705882}%
\pgfsetfillcolor{currentfill}%
\pgfsetlinewidth{1.003750pt}%
\definecolor{currentstroke}{rgb}{0.121569,0.466667,0.705882}%
\pgfsetstrokecolor{currentstroke}%
\pgfsetdash{}{0pt}%
\pgfpathmoveto{\pgfqpoint{5.353018in}{3.031612in}}%
\pgfpathcurveto{\pgfqpoint{5.366041in}{3.031612in}}{\pgfqpoint{5.378532in}{3.036786in}}{\pgfqpoint{5.387740in}{3.045994in}}%
\pgfpathcurveto{\pgfqpoint{5.396949in}{3.055203in}}{\pgfqpoint{5.402123in}{3.067694in}}{\pgfqpoint{5.402123in}{3.080716in}}%
\pgfpathcurveto{\pgfqpoint{5.402123in}{3.093739in}}{\pgfqpoint{5.396949in}{3.106230in}}{\pgfqpoint{5.387740in}{3.115439in}}%
\pgfpathcurveto{\pgfqpoint{5.378532in}{3.124647in}}{\pgfqpoint{5.366041in}{3.129821in}}{\pgfqpoint{5.353018in}{3.129821in}}%
\pgfpathcurveto{\pgfqpoint{5.339995in}{3.129821in}}{\pgfqpoint{5.327504in}{3.124647in}}{\pgfqpoint{5.318296in}{3.115439in}}%
\pgfpathcurveto{\pgfqpoint{5.309087in}{3.106230in}}{\pgfqpoint{5.303913in}{3.093739in}}{\pgfqpoint{5.303913in}{3.080716in}}%
\pgfpathcurveto{\pgfqpoint{5.303913in}{3.067694in}}{\pgfqpoint{5.309087in}{3.055203in}}{\pgfqpoint{5.318296in}{3.045994in}}%
\pgfpathcurveto{\pgfqpoint{5.327504in}{3.036786in}}{\pgfqpoint{5.339995in}{3.031612in}}{\pgfqpoint{5.353018in}{3.031612in}}%
\pgfpathlineto{\pgfqpoint{5.353018in}{3.031612in}}%
\pgfpathclose%
\pgfusepath{stroke,fill}%
\end{pgfscope}%
\begin{pgfscope}%
\pgfpathrectangle{\pgfqpoint{0.194833in}{0.246946in}}{\pgfqpoint{6.160000in}{6.160000in}}%
\pgfusepath{clip}%
\pgfsetbuttcap%
\pgfsetroundjoin%
\definecolor{currentfill}{rgb}{0.121569,0.466667,0.705882}%
\pgfsetfillcolor{currentfill}%
\pgfsetlinewidth{1.003750pt}%
\definecolor{currentstroke}{rgb}{0.121569,0.466667,0.705882}%
\pgfsetstrokecolor{currentstroke}%
\pgfsetdash{}{0pt}%
\pgfpathmoveto{\pgfqpoint{1.376996in}{3.051857in}}%
\pgfpathcurveto{\pgfqpoint{1.390019in}{3.051857in}}{\pgfqpoint{1.402510in}{3.057031in}}{\pgfqpoint{1.411718in}{3.066239in}}%
\pgfpathcurveto{\pgfqpoint{1.420927in}{3.075448in}}{\pgfqpoint{1.426101in}{3.087939in}}{\pgfqpoint{1.426101in}{3.100961in}}%
\pgfpathcurveto{\pgfqpoint{1.426101in}{3.113984in}}{\pgfqpoint{1.420927in}{3.126475in}}{\pgfqpoint{1.411718in}{3.135684in}}%
\pgfpathcurveto{\pgfqpoint{1.402510in}{3.144892in}}{\pgfqpoint{1.390019in}{3.150066in}}{\pgfqpoint{1.376996in}{3.150066in}}%
\pgfpathcurveto{\pgfqpoint{1.363973in}{3.150066in}}{\pgfqpoint{1.351482in}{3.144892in}}{\pgfqpoint{1.342274in}{3.135684in}}%
\pgfpathcurveto{\pgfqpoint{1.333066in}{3.126475in}}{\pgfqpoint{1.327892in}{3.113984in}}{\pgfqpoint{1.327892in}{3.100961in}}%
\pgfpathcurveto{\pgfqpoint{1.327892in}{3.087939in}}{\pgfqpoint{1.333066in}{3.075448in}}{\pgfqpoint{1.342274in}{3.066239in}}%
\pgfpathcurveto{\pgfqpoint{1.351482in}{3.057031in}}{\pgfqpoint{1.363973in}{3.051857in}}{\pgfqpoint{1.376996in}{3.051857in}}%
\pgfpathlineto{\pgfqpoint{1.376996in}{3.051857in}}%
\pgfpathclose%
\pgfusepath{stroke,fill}%
\end{pgfscope}%
\begin{pgfscope}%
\pgfpathrectangle{\pgfqpoint{0.194833in}{0.246946in}}{\pgfqpoint{6.160000in}{6.160000in}}%
\pgfusepath{clip}%
\pgfsetbuttcap%
\pgfsetroundjoin%
\definecolor{currentfill}{rgb}{0.121569,0.466667,0.705882}%
\pgfsetfillcolor{currentfill}%
\pgfsetlinewidth{1.003750pt}%
\definecolor{currentstroke}{rgb}{0.121569,0.466667,0.705882}%
\pgfsetstrokecolor{currentstroke}%
\pgfsetdash{}{0pt}%
\pgfpathmoveto{\pgfqpoint{4.755493in}{3.283472in}}%
\pgfpathcurveto{\pgfqpoint{4.768516in}{3.283472in}}{\pgfqpoint{4.781007in}{3.288646in}}{\pgfqpoint{4.790215in}{3.297854in}}%
\pgfpathcurveto{\pgfqpoint{4.799423in}{3.307063in}}{\pgfqpoint{4.804597in}{3.319554in}}{\pgfqpoint{4.804597in}{3.332576in}}%
\pgfpathcurveto{\pgfqpoint{4.804597in}{3.345599in}}{\pgfqpoint{4.799423in}{3.358090in}}{\pgfqpoint{4.790215in}{3.367299in}}%
\pgfpathcurveto{\pgfqpoint{4.781007in}{3.376507in}}{\pgfqpoint{4.768516in}{3.381681in}}{\pgfqpoint{4.755493in}{3.381681in}}%
\pgfpathcurveto{\pgfqpoint{4.742470in}{3.381681in}}{\pgfqpoint{4.729979in}{3.376507in}}{\pgfqpoint{4.720771in}{3.367299in}}%
\pgfpathcurveto{\pgfqpoint{4.711562in}{3.358090in}}{\pgfqpoint{4.706388in}{3.345599in}}{\pgfqpoint{4.706388in}{3.332576in}}%
\pgfpathcurveto{\pgfqpoint{4.706388in}{3.319554in}}{\pgfqpoint{4.711562in}{3.307063in}}{\pgfqpoint{4.720771in}{3.297854in}}%
\pgfpathcurveto{\pgfqpoint{4.729979in}{3.288646in}}{\pgfqpoint{4.742470in}{3.283472in}}{\pgfqpoint{4.755493in}{3.283472in}}%
\pgfpathlineto{\pgfqpoint{4.755493in}{3.283472in}}%
\pgfpathclose%
\pgfusepath{stroke,fill}%
\end{pgfscope}%
\begin{pgfscope}%
\pgfpathrectangle{\pgfqpoint{0.194833in}{0.246946in}}{\pgfqpoint{6.160000in}{6.160000in}}%
\pgfusepath{clip}%
\pgfsetbuttcap%
\pgfsetroundjoin%
\definecolor{currentfill}{rgb}{0.121569,0.466667,0.705882}%
\pgfsetfillcolor{currentfill}%
\pgfsetlinewidth{1.003750pt}%
\definecolor{currentstroke}{rgb}{0.121569,0.466667,0.705882}%
\pgfsetstrokecolor{currentstroke}%
\pgfsetdash{}{0pt}%
\pgfpathmoveto{\pgfqpoint{3.477448in}{3.287021in}}%
\pgfpathcurveto{\pgfqpoint{3.490470in}{3.287021in}}{\pgfqpoint{3.502961in}{3.292195in}}{\pgfqpoint{3.512170in}{3.301403in}}%
\pgfpathcurveto{\pgfqpoint{3.521378in}{3.310612in}}{\pgfqpoint{3.526552in}{3.323103in}}{\pgfqpoint{3.526552in}{3.336125in}}%
\pgfpathcurveto{\pgfqpoint{3.526552in}{3.349148in}}{\pgfqpoint{3.521378in}{3.361639in}}{\pgfqpoint{3.512170in}{3.370848in}}%
\pgfpathcurveto{\pgfqpoint{3.502961in}{3.380056in}}{\pgfqpoint{3.490470in}{3.385230in}}{\pgfqpoint{3.477448in}{3.385230in}}%
\pgfpathcurveto{\pgfqpoint{3.464425in}{3.385230in}}{\pgfqpoint{3.451934in}{3.380056in}}{\pgfqpoint{3.442725in}{3.370848in}}%
\pgfpathcurveto{\pgfqpoint{3.433517in}{3.361639in}}{\pgfqpoint{3.428343in}{3.349148in}}{\pgfqpoint{3.428343in}{3.336125in}}%
\pgfpathcurveto{\pgfqpoint{3.428343in}{3.323103in}}{\pgfqpoint{3.433517in}{3.310612in}}{\pgfqpoint{3.442725in}{3.301403in}}%
\pgfpathcurveto{\pgfqpoint{3.451934in}{3.292195in}}{\pgfqpoint{3.464425in}{3.287021in}}{\pgfqpoint{3.477448in}{3.287021in}}%
\pgfpathlineto{\pgfqpoint{3.477448in}{3.287021in}}%
\pgfpathclose%
\pgfusepath{stroke,fill}%
\end{pgfscope}%
\begin{pgfscope}%
\pgfpathrectangle{\pgfqpoint{0.194833in}{0.246946in}}{\pgfqpoint{6.160000in}{6.160000in}}%
\pgfusepath{clip}%
\pgfsetbuttcap%
\pgfsetroundjoin%
\definecolor{currentfill}{rgb}{0.121569,0.466667,0.705882}%
\pgfsetfillcolor{currentfill}%
\pgfsetlinewidth{1.003750pt}%
\definecolor{currentstroke}{rgb}{0.121569,0.466667,0.705882}%
\pgfsetstrokecolor{currentstroke}%
\pgfsetdash{}{0pt}%
\pgfpathmoveto{\pgfqpoint{3.087302in}{3.292845in}}%
\pgfpathcurveto{\pgfqpoint{3.100325in}{3.292845in}}{\pgfqpoint{3.112816in}{3.298019in}}{\pgfqpoint{3.122024in}{3.307228in}}%
\pgfpathcurveto{\pgfqpoint{3.131233in}{3.316436in}}{\pgfqpoint{3.136407in}{3.328927in}}{\pgfqpoint{3.136407in}{3.341950in}}%
\pgfpathcurveto{\pgfqpoint{3.136407in}{3.354973in}}{\pgfqpoint{3.131233in}{3.367464in}}{\pgfqpoint{3.122024in}{3.376672in}}%
\pgfpathcurveto{\pgfqpoint{3.112816in}{3.385880in}}{\pgfqpoint{3.100325in}{3.391054in}}{\pgfqpoint{3.087302in}{3.391054in}}%
\pgfpathcurveto{\pgfqpoint{3.074280in}{3.391054in}}{\pgfqpoint{3.061788in}{3.385880in}}{\pgfqpoint{3.052580in}{3.376672in}}%
\pgfpathcurveto{\pgfqpoint{3.043372in}{3.367464in}}{\pgfqpoint{3.038198in}{3.354973in}}{\pgfqpoint{3.038198in}{3.341950in}}%
\pgfpathcurveto{\pgfqpoint{3.038198in}{3.328927in}}{\pgfqpoint{3.043372in}{3.316436in}}{\pgfqpoint{3.052580in}{3.307228in}}%
\pgfpathcurveto{\pgfqpoint{3.061788in}{3.298019in}}{\pgfqpoint{3.074280in}{3.292845in}}{\pgfqpoint{3.087302in}{3.292845in}}%
\pgfpathlineto{\pgfqpoint{3.087302in}{3.292845in}}%
\pgfpathclose%
\pgfusepath{stroke,fill}%
\end{pgfscope}%
\begin{pgfscope}%
\pgfpathrectangle{\pgfqpoint{0.194833in}{0.246946in}}{\pgfqpoint{6.160000in}{6.160000in}}%
\pgfusepath{clip}%
\pgfsetbuttcap%
\pgfsetroundjoin%
\definecolor{currentfill}{rgb}{0.121569,0.466667,0.705882}%
\pgfsetfillcolor{currentfill}%
\pgfsetlinewidth{1.003750pt}%
\definecolor{currentstroke}{rgb}{0.121569,0.466667,0.705882}%
\pgfsetstrokecolor{currentstroke}%
\pgfsetdash{}{0pt}%
\pgfpathmoveto{\pgfqpoint{3.885905in}{3.293125in}}%
\pgfpathcurveto{\pgfqpoint{3.898928in}{3.293125in}}{\pgfqpoint{3.911419in}{3.298299in}}{\pgfqpoint{3.920628in}{3.307507in}}%
\pgfpathcurveto{\pgfqpoint{3.929836in}{3.316716in}}{\pgfqpoint{3.935010in}{3.329207in}}{\pgfqpoint{3.935010in}{3.342229in}}%
\pgfpathcurveto{\pgfqpoint{3.935010in}{3.355252in}}{\pgfqpoint{3.929836in}{3.367743in}}{\pgfqpoint{3.920628in}{3.376952in}}%
\pgfpathcurveto{\pgfqpoint{3.911419in}{3.386160in}}{\pgfqpoint{3.898928in}{3.391334in}}{\pgfqpoint{3.885905in}{3.391334in}}%
\pgfpathcurveto{\pgfqpoint{3.872883in}{3.391334in}}{\pgfqpoint{3.860392in}{3.386160in}}{\pgfqpoint{3.851183in}{3.376952in}}%
\pgfpathcurveto{\pgfqpoint{3.841975in}{3.367743in}}{\pgfqpoint{3.836801in}{3.355252in}}{\pgfqpoint{3.836801in}{3.342229in}}%
\pgfpathcurveto{\pgfqpoint{3.836801in}{3.329207in}}{\pgfqpoint{3.841975in}{3.316716in}}{\pgfqpoint{3.851183in}{3.307507in}}%
\pgfpathcurveto{\pgfqpoint{3.860392in}{3.298299in}}{\pgfqpoint{3.872883in}{3.293125in}}{\pgfqpoint{3.885905in}{3.293125in}}%
\pgfpathlineto{\pgfqpoint{3.885905in}{3.293125in}}%
\pgfpathclose%
\pgfusepath{stroke,fill}%
\end{pgfscope}%
\begin{pgfscope}%
\pgfpathrectangle{\pgfqpoint{0.194833in}{0.246946in}}{\pgfqpoint{6.160000in}{6.160000in}}%
\pgfusepath{clip}%
\pgfsetbuttcap%
\pgfsetroundjoin%
\definecolor{currentfill}{rgb}{0.121569,0.466667,0.705882}%
\pgfsetfillcolor{currentfill}%
\pgfsetlinewidth{1.003750pt}%
\definecolor{currentstroke}{rgb}{0.121569,0.466667,0.705882}%
\pgfsetstrokecolor{currentstroke}%
\pgfsetdash{}{0pt}%
\pgfpathmoveto{\pgfqpoint{4.310834in}{3.295879in}}%
\pgfpathcurveto{\pgfqpoint{4.323856in}{3.295879in}}{\pgfqpoint{4.336348in}{3.301053in}}{\pgfqpoint{4.345556in}{3.310261in}}%
\pgfpathcurveto{\pgfqpoint{4.354764in}{3.319470in}}{\pgfqpoint{4.359938in}{3.331961in}}{\pgfqpoint{4.359938in}{3.344984in}}%
\pgfpathcurveto{\pgfqpoint{4.359938in}{3.358006in}}{\pgfqpoint{4.354764in}{3.370497in}}{\pgfqpoint{4.345556in}{3.379706in}}%
\pgfpathcurveto{\pgfqpoint{4.336348in}{3.388914in}}{\pgfqpoint{4.323856in}{3.394088in}}{\pgfqpoint{4.310834in}{3.394088in}}%
\pgfpathcurveto{\pgfqpoint{4.297811in}{3.394088in}}{\pgfqpoint{4.285320in}{3.388914in}}{\pgfqpoint{4.276112in}{3.379706in}}%
\pgfpathcurveto{\pgfqpoint{4.266903in}{3.370497in}}{\pgfqpoint{4.261729in}{3.358006in}}{\pgfqpoint{4.261729in}{3.344984in}}%
\pgfpathcurveto{\pgfqpoint{4.261729in}{3.331961in}}{\pgfqpoint{4.266903in}{3.319470in}}{\pgfqpoint{4.276112in}{3.310261in}}%
\pgfpathcurveto{\pgfqpoint{4.285320in}{3.301053in}}{\pgfqpoint{4.297811in}{3.295879in}}{\pgfqpoint{4.310834in}{3.295879in}}%
\pgfpathlineto{\pgfqpoint{4.310834in}{3.295879in}}%
\pgfpathclose%
\pgfusepath{stroke,fill}%
\end{pgfscope}%
\begin{pgfscope}%
\pgfpathrectangle{\pgfqpoint{0.194833in}{0.246946in}}{\pgfqpoint{6.160000in}{6.160000in}}%
\pgfusepath{clip}%
\pgfsetbuttcap%
\pgfsetroundjoin%
\definecolor{currentfill}{rgb}{0.121569,0.466667,0.705882}%
\pgfsetfillcolor{currentfill}%
\pgfsetlinewidth{1.003750pt}%
\definecolor{currentstroke}{rgb}{0.121569,0.466667,0.705882}%
\pgfsetstrokecolor{currentstroke}%
\pgfsetdash{}{0pt}%
\pgfpathmoveto{\pgfqpoint{5.164221in}{3.307337in}}%
\pgfpathcurveto{\pgfqpoint{5.177244in}{3.307337in}}{\pgfqpoint{5.189735in}{3.312511in}}{\pgfqpoint{5.198943in}{3.321720in}}%
\pgfpathcurveto{\pgfqpoint{5.208152in}{3.330928in}}{\pgfqpoint{5.213326in}{3.343419in}}{\pgfqpoint{5.213326in}{3.356442in}}%
\pgfpathcurveto{\pgfqpoint{5.213326in}{3.369465in}}{\pgfqpoint{5.208152in}{3.381956in}}{\pgfqpoint{5.198943in}{3.391164in}}%
\pgfpathcurveto{\pgfqpoint{5.189735in}{3.400373in}}{\pgfqpoint{5.177244in}{3.405547in}}{\pgfqpoint{5.164221in}{3.405547in}}%
\pgfpathcurveto{\pgfqpoint{5.151198in}{3.405547in}}{\pgfqpoint{5.138707in}{3.400373in}}{\pgfqpoint{5.129499in}{3.391164in}}%
\pgfpathcurveto{\pgfqpoint{5.120291in}{3.381956in}}{\pgfqpoint{5.115117in}{3.369465in}}{\pgfqpoint{5.115117in}{3.356442in}}%
\pgfpathcurveto{\pgfqpoint{5.115117in}{3.343419in}}{\pgfqpoint{5.120291in}{3.330928in}}{\pgfqpoint{5.129499in}{3.321720in}}%
\pgfpathcurveto{\pgfqpoint{5.138707in}{3.312511in}}{\pgfqpoint{5.151198in}{3.307337in}}{\pgfqpoint{5.164221in}{3.307337in}}%
\pgfpathlineto{\pgfqpoint{5.164221in}{3.307337in}}%
\pgfpathclose%
\pgfusepath{stroke,fill}%
\end{pgfscope}%
\begin{pgfscope}%
\pgfpathrectangle{\pgfqpoint{0.194833in}{0.246946in}}{\pgfqpoint{6.160000in}{6.160000in}}%
\pgfusepath{clip}%
\pgfsetbuttcap%
\pgfsetroundjoin%
\definecolor{currentfill}{rgb}{0.121569,0.466667,0.705882}%
\pgfsetfillcolor{currentfill}%
\pgfsetlinewidth{1.003750pt}%
\definecolor{currentstroke}{rgb}{0.121569,0.466667,0.705882}%
\pgfsetstrokecolor{currentstroke}%
\pgfsetdash{}{0pt}%
\pgfpathmoveto{\pgfqpoint{2.709379in}{3.309746in}}%
\pgfpathcurveto{\pgfqpoint{2.722402in}{3.309746in}}{\pgfqpoint{2.734893in}{3.314920in}}{\pgfqpoint{2.744102in}{3.324129in}}%
\pgfpathcurveto{\pgfqpoint{2.753310in}{3.333337in}}{\pgfqpoint{2.758484in}{3.345828in}}{\pgfqpoint{2.758484in}{3.358851in}}%
\pgfpathcurveto{\pgfqpoint{2.758484in}{3.371873in}}{\pgfqpoint{2.753310in}{3.384365in}}{\pgfqpoint{2.744102in}{3.393573in}}%
\pgfpathcurveto{\pgfqpoint{2.734893in}{3.402781in}}{\pgfqpoint{2.722402in}{3.407955in}}{\pgfqpoint{2.709379in}{3.407955in}}%
\pgfpathcurveto{\pgfqpoint{2.696357in}{3.407955in}}{\pgfqpoint{2.683866in}{3.402781in}}{\pgfqpoint{2.674657in}{3.393573in}}%
\pgfpathcurveto{\pgfqpoint{2.665449in}{3.384365in}}{\pgfqpoint{2.660275in}{3.371873in}}{\pgfqpoint{2.660275in}{3.358851in}}%
\pgfpathcurveto{\pgfqpoint{2.660275in}{3.345828in}}{\pgfqpoint{2.665449in}{3.333337in}}{\pgfqpoint{2.674657in}{3.324129in}}%
\pgfpathcurveto{\pgfqpoint{2.683866in}{3.314920in}}{\pgfqpoint{2.696357in}{3.309746in}}{\pgfqpoint{2.709379in}{3.309746in}}%
\pgfpathlineto{\pgfqpoint{2.709379in}{3.309746in}}%
\pgfpathclose%
\pgfusepath{stroke,fill}%
\end{pgfscope}%
\begin{pgfscope}%
\pgfpathrectangle{\pgfqpoint{0.194833in}{0.246946in}}{\pgfqpoint{6.160000in}{6.160000in}}%
\pgfusepath{clip}%
\pgfsetbuttcap%
\pgfsetroundjoin%
\definecolor{currentfill}{rgb}{0.121569,0.466667,0.705882}%
\pgfsetfillcolor{currentfill}%
\pgfsetlinewidth{1.003750pt}%
\definecolor{currentstroke}{rgb}{0.121569,0.466667,0.705882}%
\pgfsetstrokecolor{currentstroke}%
\pgfsetdash{}{0pt}%
\pgfpathmoveto{\pgfqpoint{1.962010in}{3.314907in}}%
\pgfpathcurveto{\pgfqpoint{1.975032in}{3.314907in}}{\pgfqpoint{1.987524in}{3.320081in}}{\pgfqpoint{1.996732in}{3.329289in}}%
\pgfpathcurveto{\pgfqpoint{2.005940in}{3.338498in}}{\pgfqpoint{2.011114in}{3.350989in}}{\pgfqpoint{2.011114in}{3.364011in}}%
\pgfpathcurveto{\pgfqpoint{2.011114in}{3.377034in}}{\pgfqpoint{2.005940in}{3.389525in}}{\pgfqpoint{1.996732in}{3.398734in}}%
\pgfpathcurveto{\pgfqpoint{1.987524in}{3.407942in}}{\pgfqpoint{1.975032in}{3.413116in}}{\pgfqpoint{1.962010in}{3.413116in}}%
\pgfpathcurveto{\pgfqpoint{1.948987in}{3.413116in}}{\pgfqpoint{1.936496in}{3.407942in}}{\pgfqpoint{1.927288in}{3.398734in}}%
\pgfpathcurveto{\pgfqpoint{1.918079in}{3.389525in}}{\pgfqpoint{1.912905in}{3.377034in}}{\pgfqpoint{1.912905in}{3.364011in}}%
\pgfpathcurveto{\pgfqpoint{1.912905in}{3.350989in}}{\pgfqpoint{1.918079in}{3.338498in}}{\pgfqpoint{1.927288in}{3.329289in}}%
\pgfpathcurveto{\pgfqpoint{1.936496in}{3.320081in}}{\pgfqpoint{1.948987in}{3.314907in}}{\pgfqpoint{1.962010in}{3.314907in}}%
\pgfpathlineto{\pgfqpoint{1.962010in}{3.314907in}}%
\pgfpathclose%
\pgfusepath{stroke,fill}%
\end{pgfscope}%
\begin{pgfscope}%
\pgfpathrectangle{\pgfqpoint{0.194833in}{0.246946in}}{\pgfqpoint{6.160000in}{6.160000in}}%
\pgfusepath{clip}%
\pgfsetbuttcap%
\pgfsetroundjoin%
\definecolor{currentfill}{rgb}{0.121569,0.466667,0.705882}%
\pgfsetfillcolor{currentfill}%
\pgfsetlinewidth{1.003750pt}%
\definecolor{currentstroke}{rgb}{0.121569,0.466667,0.705882}%
\pgfsetstrokecolor{currentstroke}%
\pgfsetdash{}{0pt}%
\pgfpathmoveto{\pgfqpoint{1.565319in}{3.326890in}}%
\pgfpathcurveto{\pgfqpoint{1.578342in}{3.326890in}}{\pgfqpoint{1.590833in}{3.332064in}}{\pgfqpoint{1.600041in}{3.341273in}}%
\pgfpathcurveto{\pgfqpoint{1.609250in}{3.350481in}}{\pgfqpoint{1.614424in}{3.362972in}}{\pgfqpoint{1.614424in}{3.375995in}}%
\pgfpathcurveto{\pgfqpoint{1.614424in}{3.389018in}}{\pgfqpoint{1.609250in}{3.401509in}}{\pgfqpoint{1.600041in}{3.410717in}}%
\pgfpathcurveto{\pgfqpoint{1.590833in}{3.419926in}}{\pgfqpoint{1.578342in}{3.425099in}}{\pgfqpoint{1.565319in}{3.425099in}}%
\pgfpathcurveto{\pgfqpoint{1.552296in}{3.425099in}}{\pgfqpoint{1.539805in}{3.419926in}}{\pgfqpoint{1.530597in}{3.410717in}}%
\pgfpathcurveto{\pgfqpoint{1.521388in}{3.401509in}}{\pgfqpoint{1.516214in}{3.389018in}}{\pgfqpoint{1.516214in}{3.375995in}}%
\pgfpathcurveto{\pgfqpoint{1.516214in}{3.362972in}}{\pgfqpoint{1.521388in}{3.350481in}}{\pgfqpoint{1.530597in}{3.341273in}}%
\pgfpathcurveto{\pgfqpoint{1.539805in}{3.332064in}}{\pgfqpoint{1.552296in}{3.326890in}}{\pgfqpoint{1.565319in}{3.326890in}}%
\pgfpathlineto{\pgfqpoint{1.565319in}{3.326890in}}%
\pgfpathclose%
\pgfusepath{stroke,fill}%
\end{pgfscope}%
\begin{pgfscope}%
\pgfpathrectangle{\pgfqpoint{0.194833in}{0.246946in}}{\pgfqpoint{6.160000in}{6.160000in}}%
\pgfusepath{clip}%
\pgfsetbuttcap%
\pgfsetroundjoin%
\definecolor{currentfill}{rgb}{0.121569,0.466667,0.705882}%
\pgfsetfillcolor{currentfill}%
\pgfsetlinewidth{1.003750pt}%
\definecolor{currentstroke}{rgb}{0.121569,0.466667,0.705882}%
\pgfsetstrokecolor{currentstroke}%
\pgfsetdash{}{0pt}%
\pgfpathmoveto{\pgfqpoint{2.338836in}{3.329085in}}%
\pgfpathcurveto{\pgfqpoint{2.351859in}{3.329085in}}{\pgfqpoint{2.364350in}{3.334259in}}{\pgfqpoint{2.373559in}{3.343468in}}%
\pgfpathcurveto{\pgfqpoint{2.382767in}{3.352676in}}{\pgfqpoint{2.387941in}{3.365167in}}{\pgfqpoint{2.387941in}{3.378190in}}%
\pgfpathcurveto{\pgfqpoint{2.387941in}{3.391213in}}{\pgfqpoint{2.382767in}{3.403704in}}{\pgfqpoint{2.373559in}{3.412912in}}%
\pgfpathcurveto{\pgfqpoint{2.364350in}{3.422121in}}{\pgfqpoint{2.351859in}{3.427295in}}{\pgfqpoint{2.338836in}{3.427295in}}%
\pgfpathcurveto{\pgfqpoint{2.325814in}{3.427295in}}{\pgfqpoint{2.313323in}{3.422121in}}{\pgfqpoint{2.304114in}{3.412912in}}%
\pgfpathcurveto{\pgfqpoint{2.294906in}{3.403704in}}{\pgfqpoint{2.289732in}{3.391213in}}{\pgfqpoint{2.289732in}{3.378190in}}%
\pgfpathcurveto{\pgfqpoint{2.289732in}{3.365167in}}{\pgfqpoint{2.294906in}{3.352676in}}{\pgfqpoint{2.304114in}{3.343468in}}%
\pgfpathcurveto{\pgfqpoint{2.313323in}{3.334259in}}{\pgfqpoint{2.325814in}{3.329085in}}{\pgfqpoint{2.338836in}{3.329085in}}%
\pgfpathlineto{\pgfqpoint{2.338836in}{3.329085in}}%
\pgfpathclose%
\pgfusepath{stroke,fill}%
\end{pgfscope}%
\begin{pgfscope}%
\pgfpathrectangle{\pgfqpoint{0.194833in}{0.246946in}}{\pgfqpoint{6.160000in}{6.160000in}}%
\pgfusepath{clip}%
\pgfsetbuttcap%
\pgfsetroundjoin%
\definecolor{currentfill}{rgb}{0.121569,0.466667,0.705882}%
\pgfsetfillcolor{currentfill}%
\pgfsetlinewidth{1.003750pt}%
\definecolor{currentstroke}{rgb}{0.121569,0.466667,0.705882}%
\pgfsetstrokecolor{currentstroke}%
\pgfsetdash{}{0pt}%
\pgfpathmoveto{\pgfqpoint{4.568167in}{3.548094in}}%
\pgfpathcurveto{\pgfqpoint{4.581190in}{3.548094in}}{\pgfqpoint{4.593681in}{3.553268in}}{\pgfqpoint{4.602889in}{3.562477in}}%
\pgfpathcurveto{\pgfqpoint{4.612098in}{3.571685in}}{\pgfqpoint{4.617272in}{3.584176in}}{\pgfqpoint{4.617272in}{3.597199in}}%
\pgfpathcurveto{\pgfqpoint{4.617272in}{3.610221in}}{\pgfqpoint{4.612098in}{3.622713in}}{\pgfqpoint{4.602889in}{3.631921in}}%
\pgfpathcurveto{\pgfqpoint{4.593681in}{3.641129in}}{\pgfqpoint{4.581190in}{3.646303in}}{\pgfqpoint{4.568167in}{3.646303in}}%
\pgfpathcurveto{\pgfqpoint{4.555145in}{3.646303in}}{\pgfqpoint{4.542653in}{3.641129in}}{\pgfqpoint{4.533445in}{3.631921in}}%
\pgfpathcurveto{\pgfqpoint{4.524237in}{3.622713in}}{\pgfqpoint{4.519063in}{3.610221in}}{\pgfqpoint{4.519063in}{3.597199in}}%
\pgfpathcurveto{\pgfqpoint{4.519063in}{3.584176in}}{\pgfqpoint{4.524237in}{3.571685in}}{\pgfqpoint{4.533445in}{3.562477in}}%
\pgfpathcurveto{\pgfqpoint{4.542653in}{3.553268in}}{\pgfqpoint{4.555145in}{3.548094in}}{\pgfqpoint{4.568167in}{3.548094in}}%
\pgfpathlineto{\pgfqpoint{4.568167in}{3.548094in}}%
\pgfpathclose%
\pgfusepath{stroke,fill}%
\end{pgfscope}%
\begin{pgfscope}%
\pgfpathrectangle{\pgfqpoint{0.194833in}{0.246946in}}{\pgfqpoint{6.160000in}{6.160000in}}%
\pgfusepath{clip}%
\pgfsetbuttcap%
\pgfsetroundjoin%
\definecolor{currentfill}{rgb}{0.121569,0.466667,0.705882}%
\pgfsetfillcolor{currentfill}%
\pgfsetlinewidth{1.003750pt}%
\definecolor{currentstroke}{rgb}{0.121569,0.466667,0.705882}%
\pgfsetstrokecolor{currentstroke}%
\pgfsetdash{}{0pt}%
\pgfpathmoveto{\pgfqpoint{4.983798in}{3.570833in}}%
\pgfpathcurveto{\pgfqpoint{4.996821in}{3.570833in}}{\pgfqpoint{5.009312in}{3.576007in}}{\pgfqpoint{5.018521in}{3.585216in}}%
\pgfpathcurveto{\pgfqpoint{5.027729in}{3.594424in}}{\pgfqpoint{5.032903in}{3.606915in}}{\pgfqpoint{5.032903in}{3.619938in}}%
\pgfpathcurveto{\pgfqpoint{5.032903in}{3.632961in}}{\pgfqpoint{5.027729in}{3.645452in}}{\pgfqpoint{5.018521in}{3.654660in}}%
\pgfpathcurveto{\pgfqpoint{5.009312in}{3.663869in}}{\pgfqpoint{4.996821in}{3.669042in}}{\pgfqpoint{4.983798in}{3.669042in}}%
\pgfpathcurveto{\pgfqpoint{4.970776in}{3.669042in}}{\pgfqpoint{4.958285in}{3.663869in}}{\pgfqpoint{4.949076in}{3.654660in}}%
\pgfpathcurveto{\pgfqpoint{4.939868in}{3.645452in}}{\pgfqpoint{4.934694in}{3.632961in}}{\pgfqpoint{4.934694in}{3.619938in}}%
\pgfpathcurveto{\pgfqpoint{4.934694in}{3.606915in}}{\pgfqpoint{4.939868in}{3.594424in}}{\pgfqpoint{4.949076in}{3.585216in}}%
\pgfpathcurveto{\pgfqpoint{4.958285in}{3.576007in}}{\pgfqpoint{4.970776in}{3.570833in}}{\pgfqpoint{4.983798in}{3.570833in}}%
\pgfpathlineto{\pgfqpoint{4.983798in}{3.570833in}}%
\pgfpathclose%
\pgfusepath{stroke,fill}%
\end{pgfscope}%
\begin{pgfscope}%
\pgfpathrectangle{\pgfqpoint{0.194833in}{0.246946in}}{\pgfqpoint{6.160000in}{6.160000in}}%
\pgfusepath{clip}%
\pgfsetbuttcap%
\pgfsetroundjoin%
\definecolor{currentfill}{rgb}{0.121569,0.466667,0.705882}%
\pgfsetfillcolor{currentfill}%
\pgfsetlinewidth{1.003750pt}%
\definecolor{currentstroke}{rgb}{0.121569,0.466667,0.705882}%
\pgfsetstrokecolor{currentstroke}%
\pgfsetdash{}{0pt}%
\pgfpathmoveto{\pgfqpoint{4.120557in}{3.573930in}}%
\pgfpathcurveto{\pgfqpoint{4.133580in}{3.573930in}}{\pgfqpoint{4.146071in}{3.579104in}}{\pgfqpoint{4.155279in}{3.588313in}}%
\pgfpathcurveto{\pgfqpoint{4.164488in}{3.597521in}}{\pgfqpoint{4.169662in}{3.610012in}}{\pgfqpoint{4.169662in}{3.623035in}}%
\pgfpathcurveto{\pgfqpoint{4.169662in}{3.636058in}}{\pgfqpoint{4.164488in}{3.648549in}}{\pgfqpoint{4.155279in}{3.657757in}}%
\pgfpathcurveto{\pgfqpoint{4.146071in}{3.666966in}}{\pgfqpoint{4.133580in}{3.672140in}}{\pgfqpoint{4.120557in}{3.672140in}}%
\pgfpathcurveto{\pgfqpoint{4.107534in}{3.672140in}}{\pgfqpoint{4.095043in}{3.666966in}}{\pgfqpoint{4.085835in}{3.657757in}}%
\pgfpathcurveto{\pgfqpoint{4.076626in}{3.648549in}}{\pgfqpoint{4.071452in}{3.636058in}}{\pgfqpoint{4.071452in}{3.623035in}}%
\pgfpathcurveto{\pgfqpoint{4.071452in}{3.610012in}}{\pgfqpoint{4.076626in}{3.597521in}}{\pgfqpoint{4.085835in}{3.588313in}}%
\pgfpathcurveto{\pgfqpoint{4.095043in}{3.579104in}}{\pgfqpoint{4.107534in}{3.573930in}}{\pgfqpoint{4.120557in}{3.573930in}}%
\pgfpathlineto{\pgfqpoint{4.120557in}{3.573930in}}%
\pgfpathclose%
\pgfusepath{stroke,fill}%
\end{pgfscope}%
\begin{pgfscope}%
\pgfpathrectangle{\pgfqpoint{0.194833in}{0.246946in}}{\pgfqpoint{6.160000in}{6.160000in}}%
\pgfusepath{clip}%
\pgfsetbuttcap%
\pgfsetroundjoin%
\definecolor{currentfill}{rgb}{0.121569,0.466667,0.705882}%
\pgfsetfillcolor{currentfill}%
\pgfsetlinewidth{1.003750pt}%
\definecolor{currentstroke}{rgb}{0.121569,0.466667,0.705882}%
\pgfsetstrokecolor{currentstroke}%
\pgfsetdash{}{0pt}%
\pgfpathmoveto{\pgfqpoint{3.712259in}{3.579256in}}%
\pgfpathcurveto{\pgfqpoint{3.725282in}{3.579256in}}{\pgfqpoint{3.737773in}{3.584430in}}{\pgfqpoint{3.746981in}{3.593638in}}%
\pgfpathcurveto{\pgfqpoint{3.756190in}{3.602847in}}{\pgfqpoint{3.761364in}{3.615338in}}{\pgfqpoint{3.761364in}{3.628360in}}%
\pgfpathcurveto{\pgfqpoint{3.761364in}{3.641383in}}{\pgfqpoint{3.756190in}{3.653874in}}{\pgfqpoint{3.746981in}{3.663083in}}%
\pgfpathcurveto{\pgfqpoint{3.737773in}{3.672291in}}{\pgfqpoint{3.725282in}{3.677465in}}{\pgfqpoint{3.712259in}{3.677465in}}%
\pgfpathcurveto{\pgfqpoint{3.699236in}{3.677465in}}{\pgfqpoint{3.686745in}{3.672291in}}{\pgfqpoint{3.677537in}{3.663083in}}%
\pgfpathcurveto{\pgfqpoint{3.668329in}{3.653874in}}{\pgfqpoint{3.663155in}{3.641383in}}{\pgfqpoint{3.663155in}{3.628360in}}%
\pgfpathcurveto{\pgfqpoint{3.663155in}{3.615338in}}{\pgfqpoint{3.668329in}{3.602847in}}{\pgfqpoint{3.677537in}{3.593638in}}%
\pgfpathcurveto{\pgfqpoint{3.686745in}{3.584430in}}{\pgfqpoint{3.699236in}{3.579256in}}{\pgfqpoint{3.712259in}{3.579256in}}%
\pgfpathlineto{\pgfqpoint{3.712259in}{3.579256in}}%
\pgfpathclose%
\pgfusepath{stroke,fill}%
\end{pgfscope}%
\begin{pgfscope}%
\pgfpathrectangle{\pgfqpoint{0.194833in}{0.246946in}}{\pgfqpoint{6.160000in}{6.160000in}}%
\pgfusepath{clip}%
\pgfsetbuttcap%
\pgfsetroundjoin%
\definecolor{currentfill}{rgb}{0.121569,0.466667,0.705882}%
\pgfsetfillcolor{currentfill}%
\pgfsetlinewidth{1.003750pt}%
\definecolor{currentstroke}{rgb}{0.121569,0.466667,0.705882}%
\pgfsetstrokecolor{currentstroke}%
\pgfsetdash{}{0pt}%
\pgfpathmoveto{\pgfqpoint{3.317181in}{3.584652in}}%
\pgfpathcurveto{\pgfqpoint{3.330204in}{3.584652in}}{\pgfqpoint{3.342695in}{3.589826in}}{\pgfqpoint{3.351903in}{3.599035in}}%
\pgfpathcurveto{\pgfqpoint{3.361112in}{3.608243in}}{\pgfqpoint{3.366286in}{3.620734in}}{\pgfqpoint{3.366286in}{3.633757in}}%
\pgfpathcurveto{\pgfqpoint{3.366286in}{3.646780in}}{\pgfqpoint{3.361112in}{3.659271in}}{\pgfqpoint{3.351903in}{3.668479in}}%
\pgfpathcurveto{\pgfqpoint{3.342695in}{3.677687in}}{\pgfqpoint{3.330204in}{3.682861in}}{\pgfqpoint{3.317181in}{3.682861in}}%
\pgfpathcurveto{\pgfqpoint{3.304158in}{3.682861in}}{\pgfqpoint{3.291667in}{3.677687in}}{\pgfqpoint{3.282459in}{3.668479in}}%
\pgfpathcurveto{\pgfqpoint{3.273250in}{3.659271in}}{\pgfqpoint{3.268076in}{3.646780in}}{\pgfqpoint{3.268076in}{3.633757in}}%
\pgfpathcurveto{\pgfqpoint{3.268076in}{3.620734in}}{\pgfqpoint{3.273250in}{3.608243in}}{\pgfqpoint{3.282459in}{3.599035in}}%
\pgfpathcurveto{\pgfqpoint{3.291667in}{3.589826in}}{\pgfqpoint{3.304158in}{3.584652in}}{\pgfqpoint{3.317181in}{3.584652in}}%
\pgfpathlineto{\pgfqpoint{3.317181in}{3.584652in}}%
\pgfpathclose%
\pgfusepath{stroke,fill}%
\end{pgfscope}%
\begin{pgfscope}%
\pgfpathrectangle{\pgfqpoint{0.194833in}{0.246946in}}{\pgfqpoint{6.160000in}{6.160000in}}%
\pgfusepath{clip}%
\pgfsetbuttcap%
\pgfsetroundjoin%
\definecolor{currentfill}{rgb}{0.121569,0.466667,0.705882}%
\pgfsetfillcolor{currentfill}%
\pgfsetlinewidth{1.003750pt}%
\definecolor{currentstroke}{rgb}{0.121569,0.466667,0.705882}%
\pgfsetstrokecolor{currentstroke}%
\pgfsetdash{}{0pt}%
\pgfpathmoveto{\pgfqpoint{2.926952in}{3.597382in}}%
\pgfpathcurveto{\pgfqpoint{2.939974in}{3.597382in}}{\pgfqpoint{2.952466in}{3.602556in}}{\pgfqpoint{2.961674in}{3.611764in}}%
\pgfpathcurveto{\pgfqpoint{2.970882in}{3.620973in}}{\pgfqpoint{2.976056in}{3.633464in}}{\pgfqpoint{2.976056in}{3.646487in}}%
\pgfpathcurveto{\pgfqpoint{2.976056in}{3.659509in}}{\pgfqpoint{2.970882in}{3.672000in}}{\pgfqpoint{2.961674in}{3.681209in}}%
\pgfpathcurveto{\pgfqpoint{2.952466in}{3.690417in}}{\pgfqpoint{2.939974in}{3.695591in}}{\pgfqpoint{2.926952in}{3.695591in}}%
\pgfpathcurveto{\pgfqpoint{2.913929in}{3.695591in}}{\pgfqpoint{2.901438in}{3.690417in}}{\pgfqpoint{2.892230in}{3.681209in}}%
\pgfpathcurveto{\pgfqpoint{2.883021in}{3.672000in}}{\pgfqpoint{2.877847in}{3.659509in}}{\pgfqpoint{2.877847in}{3.646487in}}%
\pgfpathcurveto{\pgfqpoint{2.877847in}{3.633464in}}{\pgfqpoint{2.883021in}{3.620973in}}{\pgfqpoint{2.892230in}{3.611764in}}%
\pgfpathcurveto{\pgfqpoint{2.901438in}{3.602556in}}{\pgfqpoint{2.913929in}{3.597382in}}{\pgfqpoint{2.926952in}{3.597382in}}%
\pgfpathlineto{\pgfqpoint{2.926952in}{3.597382in}}%
\pgfpathclose%
\pgfusepath{stroke,fill}%
\end{pgfscope}%
\begin{pgfscope}%
\pgfpathrectangle{\pgfqpoint{0.194833in}{0.246946in}}{\pgfqpoint{6.160000in}{6.160000in}}%
\pgfusepath{clip}%
\pgfsetbuttcap%
\pgfsetroundjoin%
\definecolor{currentfill}{rgb}{0.121569,0.466667,0.705882}%
\pgfsetfillcolor{currentfill}%
\pgfsetlinewidth{1.003750pt}%
\definecolor{currentstroke}{rgb}{0.121569,0.466667,0.705882}%
\pgfsetstrokecolor{currentstroke}%
\pgfsetdash{}{0pt}%
\pgfpathmoveto{\pgfqpoint{1.759250in}{3.610114in}}%
\pgfpathcurveto{\pgfqpoint{1.772273in}{3.610114in}}{\pgfqpoint{1.784764in}{3.615288in}}{\pgfqpoint{1.793973in}{3.624497in}}%
\pgfpathcurveto{\pgfqpoint{1.803181in}{3.633705in}}{\pgfqpoint{1.808355in}{3.646196in}}{\pgfqpoint{1.808355in}{3.659219in}}%
\pgfpathcurveto{\pgfqpoint{1.808355in}{3.672242in}}{\pgfqpoint{1.803181in}{3.684733in}}{\pgfqpoint{1.793973in}{3.693941in}}%
\pgfpathcurveto{\pgfqpoint{1.784764in}{3.703150in}}{\pgfqpoint{1.772273in}{3.708324in}}{\pgfqpoint{1.759250in}{3.708324in}}%
\pgfpathcurveto{\pgfqpoint{1.746228in}{3.708324in}}{\pgfqpoint{1.733737in}{3.703150in}}{\pgfqpoint{1.724528in}{3.693941in}}%
\pgfpathcurveto{\pgfqpoint{1.715320in}{3.684733in}}{\pgfqpoint{1.710146in}{3.672242in}}{\pgfqpoint{1.710146in}{3.659219in}}%
\pgfpathcurveto{\pgfqpoint{1.710146in}{3.646196in}}{\pgfqpoint{1.715320in}{3.633705in}}{\pgfqpoint{1.724528in}{3.624497in}}%
\pgfpathcurveto{\pgfqpoint{1.733737in}{3.615288in}}{\pgfqpoint{1.746228in}{3.610114in}}{\pgfqpoint{1.759250in}{3.610114in}}%
\pgfpathlineto{\pgfqpoint{1.759250in}{3.610114in}}%
\pgfpathclose%
\pgfusepath{stroke,fill}%
\end{pgfscope}%
\begin{pgfscope}%
\pgfpathrectangle{\pgfqpoint{0.194833in}{0.246946in}}{\pgfqpoint{6.160000in}{6.160000in}}%
\pgfusepath{clip}%
\pgfsetbuttcap%
\pgfsetroundjoin%
\definecolor{currentfill}{rgb}{0.121569,0.466667,0.705882}%
\pgfsetfillcolor{currentfill}%
\pgfsetlinewidth{1.003750pt}%
\definecolor{currentstroke}{rgb}{0.121569,0.466667,0.705882}%
\pgfsetstrokecolor{currentstroke}%
\pgfsetdash{}{0pt}%
\pgfpathmoveto{\pgfqpoint{2.537892in}{3.615176in}}%
\pgfpathcurveto{\pgfqpoint{2.550915in}{3.615176in}}{\pgfqpoint{2.563406in}{3.620350in}}{\pgfqpoint{2.572614in}{3.629558in}}%
\pgfpathcurveto{\pgfqpoint{2.581823in}{3.638767in}}{\pgfqpoint{2.586997in}{3.651258in}}{\pgfqpoint{2.586997in}{3.664280in}}%
\pgfpathcurveto{\pgfqpoint{2.586997in}{3.677303in}}{\pgfqpoint{2.581823in}{3.689794in}}{\pgfqpoint{2.572614in}{3.699003in}}%
\pgfpathcurveto{\pgfqpoint{2.563406in}{3.708211in}}{\pgfqpoint{2.550915in}{3.713385in}}{\pgfqpoint{2.537892in}{3.713385in}}%
\pgfpathcurveto{\pgfqpoint{2.524869in}{3.713385in}}{\pgfqpoint{2.512378in}{3.708211in}}{\pgfqpoint{2.503170in}{3.699003in}}%
\pgfpathcurveto{\pgfqpoint{2.493962in}{3.689794in}}{\pgfqpoint{2.488788in}{3.677303in}}{\pgfqpoint{2.488788in}{3.664280in}}%
\pgfpathcurveto{\pgfqpoint{2.488788in}{3.651258in}}{\pgfqpoint{2.493962in}{3.638767in}}{\pgfqpoint{2.503170in}{3.629558in}}%
\pgfpathcurveto{\pgfqpoint{2.512378in}{3.620350in}}{\pgfqpoint{2.524869in}{3.615176in}}{\pgfqpoint{2.537892in}{3.615176in}}%
\pgfpathlineto{\pgfqpoint{2.537892in}{3.615176in}}%
\pgfpathclose%
\pgfusepath{stroke,fill}%
\end{pgfscope}%
\begin{pgfscope}%
\pgfpathrectangle{\pgfqpoint{0.194833in}{0.246946in}}{\pgfqpoint{6.160000in}{6.160000in}}%
\pgfusepath{clip}%
\pgfsetbuttcap%
\pgfsetroundjoin%
\definecolor{currentfill}{rgb}{0.121569,0.466667,0.705882}%
\pgfsetfillcolor{currentfill}%
\pgfsetlinewidth{1.003750pt}%
\definecolor{currentstroke}{rgb}{0.121569,0.466667,0.705882}%
\pgfsetstrokecolor{currentstroke}%
\pgfsetdash{}{0pt}%
\pgfpathmoveto{\pgfqpoint{2.144504in}{3.621256in}}%
\pgfpathcurveto{\pgfqpoint{2.157526in}{3.621256in}}{\pgfqpoint{2.170017in}{3.626430in}}{\pgfqpoint{2.179226in}{3.635639in}}%
\pgfpathcurveto{\pgfqpoint{2.188434in}{3.644847in}}{\pgfqpoint{2.193608in}{3.657338in}}{\pgfqpoint{2.193608in}{3.670361in}}%
\pgfpathcurveto{\pgfqpoint{2.193608in}{3.683383in}}{\pgfqpoint{2.188434in}{3.695875in}}{\pgfqpoint{2.179226in}{3.705083in}}%
\pgfpathcurveto{\pgfqpoint{2.170017in}{3.714291in}}{\pgfqpoint{2.157526in}{3.719465in}}{\pgfqpoint{2.144504in}{3.719465in}}%
\pgfpathcurveto{\pgfqpoint{2.131481in}{3.719465in}}{\pgfqpoint{2.118990in}{3.714291in}}{\pgfqpoint{2.109781in}{3.705083in}}%
\pgfpathcurveto{\pgfqpoint{2.100573in}{3.695875in}}{\pgfqpoint{2.095399in}{3.683383in}}{\pgfqpoint{2.095399in}{3.670361in}}%
\pgfpathcurveto{\pgfqpoint{2.095399in}{3.657338in}}{\pgfqpoint{2.100573in}{3.644847in}}{\pgfqpoint{2.109781in}{3.635639in}}%
\pgfpathcurveto{\pgfqpoint{2.118990in}{3.626430in}}{\pgfqpoint{2.131481in}{3.621256in}}{\pgfqpoint{2.144504in}{3.621256in}}%
\pgfpathlineto{\pgfqpoint{2.144504in}{3.621256in}}%
\pgfpathclose%
\pgfusepath{stroke,fill}%
\end{pgfscope}%
\begin{pgfscope}%
\pgfpathrectangle{\pgfqpoint{0.194833in}{0.246946in}}{\pgfqpoint{6.160000in}{6.160000in}}%
\pgfusepath{clip}%
\pgfsetbuttcap%
\pgfsetroundjoin%
\definecolor{currentfill}{rgb}{0.121569,0.466667,0.705882}%
\pgfsetfillcolor{currentfill}%
\pgfsetlinewidth{1.003750pt}%
\definecolor{currentstroke}{rgb}{0.121569,0.466667,0.705882}%
\pgfsetstrokecolor{currentstroke}%
\pgfsetdash{}{0pt}%
\pgfpathmoveto{\pgfqpoint{4.388937in}{3.813208in}}%
\pgfpathcurveto{\pgfqpoint{4.401960in}{3.813208in}}{\pgfqpoint{4.414451in}{3.818382in}}{\pgfqpoint{4.423660in}{3.827590in}}%
\pgfpathcurveto{\pgfqpoint{4.432868in}{3.836799in}}{\pgfqpoint{4.438042in}{3.849290in}}{\pgfqpoint{4.438042in}{3.862312in}}%
\pgfpathcurveto{\pgfqpoint{4.438042in}{3.875335in}}{\pgfqpoint{4.432868in}{3.887826in}}{\pgfqpoint{4.423660in}{3.897035in}}%
\pgfpathcurveto{\pgfqpoint{4.414451in}{3.906243in}}{\pgfqpoint{4.401960in}{3.911417in}}{\pgfqpoint{4.388937in}{3.911417in}}%
\pgfpathcurveto{\pgfqpoint{4.375915in}{3.911417in}}{\pgfqpoint{4.363424in}{3.906243in}}{\pgfqpoint{4.354215in}{3.897035in}}%
\pgfpathcurveto{\pgfqpoint{4.345007in}{3.887826in}}{\pgfqpoint{4.339833in}{3.875335in}}{\pgfqpoint{4.339833in}{3.862312in}}%
\pgfpathcurveto{\pgfqpoint{4.339833in}{3.849290in}}{\pgfqpoint{4.345007in}{3.836799in}}{\pgfqpoint{4.354215in}{3.827590in}}%
\pgfpathcurveto{\pgfqpoint{4.363424in}{3.818382in}}{\pgfqpoint{4.375915in}{3.813208in}}{\pgfqpoint{4.388937in}{3.813208in}}%
\pgfpathlineto{\pgfqpoint{4.388937in}{3.813208in}}%
\pgfpathclose%
\pgfusepath{stroke,fill}%
\end{pgfscope}%
\begin{pgfscope}%
\pgfpathrectangle{\pgfqpoint{0.194833in}{0.246946in}}{\pgfqpoint{6.160000in}{6.160000in}}%
\pgfusepath{clip}%
\pgfsetbuttcap%
\pgfsetroundjoin%
\definecolor{currentfill}{rgb}{0.121569,0.466667,0.705882}%
\pgfsetfillcolor{currentfill}%
\pgfsetlinewidth{1.003750pt}%
\definecolor{currentstroke}{rgb}{0.121569,0.466667,0.705882}%
\pgfsetstrokecolor{currentstroke}%
\pgfsetdash{}{0pt}%
\pgfpathmoveto{\pgfqpoint{4.802832in}{3.835122in}}%
\pgfpathcurveto{\pgfqpoint{4.815855in}{3.835122in}}{\pgfqpoint{4.828346in}{3.840296in}}{\pgfqpoint{4.837555in}{3.849505in}}%
\pgfpathcurveto{\pgfqpoint{4.846763in}{3.858713in}}{\pgfqpoint{4.851937in}{3.871204in}}{\pgfqpoint{4.851937in}{3.884227in}}%
\pgfpathcurveto{\pgfqpoint{4.851937in}{3.897250in}}{\pgfqpoint{4.846763in}{3.909741in}}{\pgfqpoint{4.837555in}{3.918949in}}%
\pgfpathcurveto{\pgfqpoint{4.828346in}{3.928158in}}{\pgfqpoint{4.815855in}{3.933332in}}{\pgfqpoint{4.802832in}{3.933332in}}%
\pgfpathcurveto{\pgfqpoint{4.789810in}{3.933332in}}{\pgfqpoint{4.777319in}{3.928158in}}{\pgfqpoint{4.768110in}{3.918949in}}%
\pgfpathcurveto{\pgfqpoint{4.758902in}{3.909741in}}{\pgfqpoint{4.753728in}{3.897250in}}{\pgfqpoint{4.753728in}{3.884227in}}%
\pgfpathcurveto{\pgfqpoint{4.753728in}{3.871204in}}{\pgfqpoint{4.758902in}{3.858713in}}{\pgfqpoint{4.768110in}{3.849505in}}%
\pgfpathcurveto{\pgfqpoint{4.777319in}{3.840296in}}{\pgfqpoint{4.789810in}{3.835122in}}{\pgfqpoint{4.802832in}{3.835122in}}%
\pgfpathlineto{\pgfqpoint{4.802832in}{3.835122in}}%
\pgfpathclose%
\pgfusepath{stroke,fill}%
\end{pgfscope}%
\begin{pgfscope}%
\pgfpathrectangle{\pgfqpoint{0.194833in}{0.246946in}}{\pgfqpoint{6.160000in}{6.160000in}}%
\pgfusepath{clip}%
\pgfsetbuttcap%
\pgfsetroundjoin%
\definecolor{currentfill}{rgb}{0.121569,0.466667,0.705882}%
\pgfsetfillcolor{currentfill}%
\pgfsetlinewidth{1.003750pt}%
\definecolor{currentstroke}{rgb}{0.121569,0.466667,0.705882}%
\pgfsetstrokecolor{currentstroke}%
\pgfsetdash{}{0pt}%
\pgfpathmoveto{\pgfqpoint{3.934856in}{3.862005in}}%
\pgfpathcurveto{\pgfqpoint{3.947878in}{3.862005in}}{\pgfqpoint{3.960370in}{3.867179in}}{\pgfqpoint{3.969578in}{3.876387in}}%
\pgfpathcurveto{\pgfqpoint{3.978786in}{3.885596in}}{\pgfqpoint{3.983960in}{3.898087in}}{\pgfqpoint{3.983960in}{3.911110in}}%
\pgfpathcurveto{\pgfqpoint{3.983960in}{3.924132in}}{\pgfqpoint{3.978786in}{3.936623in}}{\pgfqpoint{3.969578in}{3.945832in}}%
\pgfpathcurveto{\pgfqpoint{3.960370in}{3.955040in}}{\pgfqpoint{3.947878in}{3.960214in}}{\pgfqpoint{3.934856in}{3.960214in}}%
\pgfpathcurveto{\pgfqpoint{3.921833in}{3.960214in}}{\pgfqpoint{3.909342in}{3.955040in}}{\pgfqpoint{3.900134in}{3.945832in}}%
\pgfpathcurveto{\pgfqpoint{3.890925in}{3.936623in}}{\pgfqpoint{3.885751in}{3.924132in}}{\pgfqpoint{3.885751in}{3.911110in}}%
\pgfpathcurveto{\pgfqpoint{3.885751in}{3.898087in}}{\pgfqpoint{3.890925in}{3.885596in}}{\pgfqpoint{3.900134in}{3.876387in}}%
\pgfpathcurveto{\pgfqpoint{3.909342in}{3.867179in}}{\pgfqpoint{3.921833in}{3.862005in}}{\pgfqpoint{3.934856in}{3.862005in}}%
\pgfpathlineto{\pgfqpoint{3.934856in}{3.862005in}}%
\pgfpathclose%
\pgfusepath{stroke,fill}%
\end{pgfscope}%
\begin{pgfscope}%
\pgfpathrectangle{\pgfqpoint{0.194833in}{0.246946in}}{\pgfqpoint{6.160000in}{6.160000in}}%
\pgfusepath{clip}%
\pgfsetbuttcap%
\pgfsetroundjoin%
\definecolor{currentfill}{rgb}{0.121569,0.466667,0.705882}%
\pgfsetfillcolor{currentfill}%
\pgfsetlinewidth{1.003750pt}%
\definecolor{currentstroke}{rgb}{0.121569,0.466667,0.705882}%
\pgfsetstrokecolor{currentstroke}%
\pgfsetdash{}{0pt}%
\pgfpathmoveto{\pgfqpoint{3.534768in}{3.873168in}}%
\pgfpathcurveto{\pgfqpoint{3.547791in}{3.873168in}}{\pgfqpoint{3.560282in}{3.878342in}}{\pgfqpoint{3.569490in}{3.887550in}}%
\pgfpathcurveto{\pgfqpoint{3.578699in}{3.896759in}}{\pgfqpoint{3.583872in}{3.909250in}}{\pgfqpoint{3.583872in}{3.922272in}}%
\pgfpathcurveto{\pgfqpoint{3.583872in}{3.935295in}}{\pgfqpoint{3.578699in}{3.947786in}}{\pgfqpoint{3.569490in}{3.956995in}}%
\pgfpathcurveto{\pgfqpoint{3.560282in}{3.966203in}}{\pgfqpoint{3.547791in}{3.971377in}}{\pgfqpoint{3.534768in}{3.971377in}}%
\pgfpathcurveto{\pgfqpoint{3.521745in}{3.971377in}}{\pgfqpoint{3.509254in}{3.966203in}}{\pgfqpoint{3.500046in}{3.956995in}}%
\pgfpathcurveto{\pgfqpoint{3.490837in}{3.947786in}}{\pgfqpoint{3.485663in}{3.935295in}}{\pgfqpoint{3.485663in}{3.922272in}}%
\pgfpathcurveto{\pgfqpoint{3.485663in}{3.909250in}}{\pgfqpoint{3.490837in}{3.896759in}}{\pgfqpoint{3.500046in}{3.887550in}}%
\pgfpathcurveto{\pgfqpoint{3.509254in}{3.878342in}}{\pgfqpoint{3.521745in}{3.873168in}}{\pgfqpoint{3.534768in}{3.873168in}}%
\pgfpathlineto{\pgfqpoint{3.534768in}{3.873168in}}%
\pgfpathclose%
\pgfusepath{stroke,fill}%
\end{pgfscope}%
\begin{pgfscope}%
\pgfpathrectangle{\pgfqpoint{0.194833in}{0.246946in}}{\pgfqpoint{6.160000in}{6.160000in}}%
\pgfusepath{clip}%
\pgfsetbuttcap%
\pgfsetroundjoin%
\definecolor{currentfill}{rgb}{0.121569,0.466667,0.705882}%
\pgfsetfillcolor{currentfill}%
\pgfsetlinewidth{1.003750pt}%
\definecolor{currentstroke}{rgb}{0.121569,0.466667,0.705882}%
\pgfsetstrokecolor{currentstroke}%
\pgfsetdash{}{0pt}%
\pgfpathmoveto{\pgfqpoint{3.140420in}{3.883441in}}%
\pgfpathcurveto{\pgfqpoint{3.153443in}{3.883441in}}{\pgfqpoint{3.165934in}{3.888615in}}{\pgfqpoint{3.175142in}{3.897824in}}%
\pgfpathcurveto{\pgfqpoint{3.184351in}{3.907032in}}{\pgfqpoint{3.189525in}{3.919523in}}{\pgfqpoint{3.189525in}{3.932546in}}%
\pgfpathcurveto{\pgfqpoint{3.189525in}{3.945569in}}{\pgfqpoint{3.184351in}{3.958060in}}{\pgfqpoint{3.175142in}{3.967268in}}%
\pgfpathcurveto{\pgfqpoint{3.165934in}{3.976477in}}{\pgfqpoint{3.153443in}{3.981651in}}{\pgfqpoint{3.140420in}{3.981651in}}%
\pgfpathcurveto{\pgfqpoint{3.127397in}{3.981651in}}{\pgfqpoint{3.114906in}{3.976477in}}{\pgfqpoint{3.105698in}{3.967268in}}%
\pgfpathcurveto{\pgfqpoint{3.096489in}{3.958060in}}{\pgfqpoint{3.091315in}{3.945569in}}{\pgfqpoint{3.091315in}{3.932546in}}%
\pgfpathcurveto{\pgfqpoint{3.091315in}{3.919523in}}{\pgfqpoint{3.096489in}{3.907032in}}{\pgfqpoint{3.105698in}{3.897824in}}%
\pgfpathcurveto{\pgfqpoint{3.114906in}{3.888615in}}{\pgfqpoint{3.127397in}{3.883441in}}{\pgfqpoint{3.140420in}{3.883441in}}%
\pgfpathlineto{\pgfqpoint{3.140420in}{3.883441in}}%
\pgfpathclose%
\pgfusepath{stroke,fill}%
\end{pgfscope}%
\begin{pgfscope}%
\pgfpathrectangle{\pgfqpoint{0.194833in}{0.246946in}}{\pgfqpoint{6.160000in}{6.160000in}}%
\pgfusepath{clip}%
\pgfsetbuttcap%
\pgfsetroundjoin%
\definecolor{currentfill}{rgb}{0.121569,0.466667,0.705882}%
\pgfsetfillcolor{currentfill}%
\pgfsetlinewidth{1.003750pt}%
\definecolor{currentstroke}{rgb}{0.121569,0.466667,0.705882}%
\pgfsetstrokecolor{currentstroke}%
\pgfsetdash{}{0pt}%
\pgfpathmoveto{\pgfqpoint{2.748327in}{3.898058in}}%
\pgfpathcurveto{\pgfqpoint{2.761349in}{3.898058in}}{\pgfqpoint{2.773840in}{3.903232in}}{\pgfqpoint{2.783049in}{3.912440in}}%
\pgfpathcurveto{\pgfqpoint{2.792257in}{3.921649in}}{\pgfqpoint{2.797431in}{3.934140in}}{\pgfqpoint{2.797431in}{3.947163in}}%
\pgfpathcurveto{\pgfqpoint{2.797431in}{3.960185in}}{\pgfqpoint{2.792257in}{3.972676in}}{\pgfqpoint{2.783049in}{3.981885in}}%
\pgfpathcurveto{\pgfqpoint{2.773840in}{3.991093in}}{\pgfqpoint{2.761349in}{3.996267in}}{\pgfqpoint{2.748327in}{3.996267in}}%
\pgfpathcurveto{\pgfqpoint{2.735304in}{3.996267in}}{\pgfqpoint{2.722813in}{3.991093in}}{\pgfqpoint{2.713604in}{3.981885in}}%
\pgfpathcurveto{\pgfqpoint{2.704396in}{3.972676in}}{\pgfqpoint{2.699222in}{3.960185in}}{\pgfqpoint{2.699222in}{3.947163in}}%
\pgfpathcurveto{\pgfqpoint{2.699222in}{3.934140in}}{\pgfqpoint{2.704396in}{3.921649in}}{\pgfqpoint{2.713604in}{3.912440in}}%
\pgfpathcurveto{\pgfqpoint{2.722813in}{3.903232in}}{\pgfqpoint{2.735304in}{3.898058in}}{\pgfqpoint{2.748327in}{3.898058in}}%
\pgfpathlineto{\pgfqpoint{2.748327in}{3.898058in}}%
\pgfpathclose%
\pgfusepath{stroke,fill}%
\end{pgfscope}%
\begin{pgfscope}%
\pgfpathrectangle{\pgfqpoint{0.194833in}{0.246946in}}{\pgfqpoint{6.160000in}{6.160000in}}%
\pgfusepath{clip}%
\pgfsetbuttcap%
\pgfsetroundjoin%
\definecolor{currentfill}{rgb}{0.121569,0.466667,0.705882}%
\pgfsetfillcolor{currentfill}%
\pgfsetlinewidth{1.003750pt}%
\definecolor{currentstroke}{rgb}{0.121569,0.466667,0.705882}%
\pgfsetstrokecolor{currentstroke}%
\pgfsetdash{}{0pt}%
\pgfpathmoveto{\pgfqpoint{2.356478in}{3.910376in}}%
\pgfpathcurveto{\pgfqpoint{2.369500in}{3.910376in}}{\pgfqpoint{2.381992in}{3.915550in}}{\pgfqpoint{2.391200in}{3.924758in}}%
\pgfpathcurveto{\pgfqpoint{2.400408in}{3.933966in}}{\pgfqpoint{2.405582in}{3.946458in}}{\pgfqpoint{2.405582in}{3.959480in}}%
\pgfpathcurveto{\pgfqpoint{2.405582in}{3.972503in}}{\pgfqpoint{2.400408in}{3.984994in}}{\pgfqpoint{2.391200in}{3.994202in}}%
\pgfpathcurveto{\pgfqpoint{2.381992in}{4.003411in}}{\pgfqpoint{2.369500in}{4.008585in}}{\pgfqpoint{2.356478in}{4.008585in}}%
\pgfpathcurveto{\pgfqpoint{2.343455in}{4.008585in}}{\pgfqpoint{2.330964in}{4.003411in}}{\pgfqpoint{2.321756in}{3.994202in}}%
\pgfpathcurveto{\pgfqpoint{2.312547in}{3.984994in}}{\pgfqpoint{2.307373in}{3.972503in}}{\pgfqpoint{2.307373in}{3.959480in}}%
\pgfpathcurveto{\pgfqpoint{2.307373in}{3.946458in}}{\pgfqpoint{2.312547in}{3.933966in}}{\pgfqpoint{2.321756in}{3.924758in}}%
\pgfpathcurveto{\pgfqpoint{2.330964in}{3.915550in}}{\pgfqpoint{2.343455in}{3.910376in}}{\pgfqpoint{2.356478in}{3.910376in}}%
\pgfpathlineto{\pgfqpoint{2.356478in}{3.910376in}}%
\pgfpathclose%
\pgfusepath{stroke,fill}%
\end{pgfscope}%
\begin{pgfscope}%
\pgfpathrectangle{\pgfqpoint{0.194833in}{0.246946in}}{\pgfqpoint{6.160000in}{6.160000in}}%
\pgfusepath{clip}%
\pgfsetbuttcap%
\pgfsetroundjoin%
\definecolor{currentfill}{rgb}{0.121569,0.466667,0.705882}%
\pgfsetfillcolor{currentfill}%
\pgfsetlinewidth{1.003750pt}%
\definecolor{currentstroke}{rgb}{0.121569,0.466667,0.705882}%
\pgfsetstrokecolor{currentstroke}%
\pgfsetdash{}{0pt}%
\pgfpathmoveto{\pgfqpoint{1.971924in}{3.920711in}}%
\pgfpathcurveto{\pgfqpoint{1.984947in}{3.920711in}}{\pgfqpoint{1.997438in}{3.925885in}}{\pgfqpoint{2.006647in}{3.935093in}}%
\pgfpathcurveto{\pgfqpoint{2.015855in}{3.944302in}}{\pgfqpoint{2.021029in}{3.956793in}}{\pgfqpoint{2.021029in}{3.969816in}}%
\pgfpathcurveto{\pgfqpoint{2.021029in}{3.982838in}}{\pgfqpoint{2.015855in}{3.995329in}}{\pgfqpoint{2.006647in}{4.004538in}}%
\pgfpathcurveto{\pgfqpoint{1.997438in}{4.013746in}}{\pgfqpoint{1.984947in}{4.018920in}}{\pgfqpoint{1.971924in}{4.018920in}}%
\pgfpathcurveto{\pgfqpoint{1.958902in}{4.018920in}}{\pgfqpoint{1.946411in}{4.013746in}}{\pgfqpoint{1.937202in}{4.004538in}}%
\pgfpathcurveto{\pgfqpoint{1.927994in}{3.995329in}}{\pgfqpoint{1.922820in}{3.982838in}}{\pgfqpoint{1.922820in}{3.969816in}}%
\pgfpathcurveto{\pgfqpoint{1.922820in}{3.956793in}}{\pgfqpoint{1.927994in}{3.944302in}}{\pgfqpoint{1.937202in}{3.935093in}}%
\pgfpathcurveto{\pgfqpoint{1.946411in}{3.925885in}}{\pgfqpoint{1.958902in}{3.920711in}}{\pgfqpoint{1.971924in}{3.920711in}}%
\pgfpathlineto{\pgfqpoint{1.971924in}{3.920711in}}%
\pgfpathclose%
\pgfusepath{stroke,fill}%
\end{pgfscope}%
\begin{pgfscope}%
\pgfpathrectangle{\pgfqpoint{0.194833in}{0.246946in}}{\pgfqpoint{6.160000in}{6.160000in}}%
\pgfusepath{clip}%
\pgfsetbuttcap%
\pgfsetroundjoin%
\definecolor{currentfill}{rgb}{0.121569,0.466667,0.705882}%
\pgfsetfillcolor{currentfill}%
\pgfsetlinewidth{1.003750pt}%
\definecolor{currentstroke}{rgb}{0.121569,0.466667,0.705882}%
\pgfsetstrokecolor{currentstroke}%
\pgfsetdash{}{0pt}%
\pgfpathmoveto{\pgfqpoint{4.216315in}{4.078579in}}%
\pgfpathcurveto{\pgfqpoint{4.229338in}{4.078579in}}{\pgfqpoint{4.241829in}{4.083753in}}{\pgfqpoint{4.251038in}{4.092961in}}%
\pgfpathcurveto{\pgfqpoint{4.260246in}{4.102170in}}{\pgfqpoint{4.265420in}{4.114661in}}{\pgfqpoint{4.265420in}{4.127683in}}%
\pgfpathcurveto{\pgfqpoint{4.265420in}{4.140706in}}{\pgfqpoint{4.260246in}{4.153197in}}{\pgfqpoint{4.251038in}{4.162406in}}%
\pgfpathcurveto{\pgfqpoint{4.241829in}{4.171614in}}{\pgfqpoint{4.229338in}{4.176788in}}{\pgfqpoint{4.216315in}{4.176788in}}%
\pgfpathcurveto{\pgfqpoint{4.203293in}{4.176788in}}{\pgfqpoint{4.190802in}{4.171614in}}{\pgfqpoint{4.181593in}{4.162406in}}%
\pgfpathcurveto{\pgfqpoint{4.172385in}{4.153197in}}{\pgfqpoint{4.167211in}{4.140706in}}{\pgfqpoint{4.167211in}{4.127683in}}%
\pgfpathcurveto{\pgfqpoint{4.167211in}{4.114661in}}{\pgfqpoint{4.172385in}{4.102170in}}{\pgfqpoint{4.181593in}{4.092961in}}%
\pgfpathcurveto{\pgfqpoint{4.190802in}{4.083753in}}{\pgfqpoint{4.203293in}{4.078579in}}{\pgfqpoint{4.216315in}{4.078579in}}%
\pgfpathlineto{\pgfqpoint{4.216315in}{4.078579in}}%
\pgfpathclose%
\pgfusepath{stroke,fill}%
\end{pgfscope}%
\begin{pgfscope}%
\pgfpathrectangle{\pgfqpoint{0.194833in}{0.246946in}}{\pgfqpoint{6.160000in}{6.160000in}}%
\pgfusepath{clip}%
\pgfsetbuttcap%
\pgfsetroundjoin%
\definecolor{currentfill}{rgb}{0.121569,0.466667,0.705882}%
\pgfsetfillcolor{currentfill}%
\pgfsetlinewidth{1.003750pt}%
\definecolor{currentstroke}{rgb}{0.121569,0.466667,0.705882}%
\pgfsetstrokecolor{currentstroke}%
\pgfsetdash{}{0pt}%
\pgfpathmoveto{\pgfqpoint{4.621658in}{4.099715in}}%
\pgfpathcurveto{\pgfqpoint{4.634681in}{4.099715in}}{\pgfqpoint{4.647172in}{4.104889in}}{\pgfqpoint{4.656381in}{4.114098in}}%
\pgfpathcurveto{\pgfqpoint{4.665589in}{4.123306in}}{\pgfqpoint{4.670763in}{4.135797in}}{\pgfqpoint{4.670763in}{4.148820in}}%
\pgfpathcurveto{\pgfqpoint{4.670763in}{4.161843in}}{\pgfqpoint{4.665589in}{4.174334in}}{\pgfqpoint{4.656381in}{4.183542in}}%
\pgfpathcurveto{\pgfqpoint{4.647172in}{4.192751in}}{\pgfqpoint{4.634681in}{4.197925in}}{\pgfqpoint{4.621658in}{4.197925in}}%
\pgfpathcurveto{\pgfqpoint{4.608636in}{4.197925in}}{\pgfqpoint{4.596145in}{4.192751in}}{\pgfqpoint{4.586936in}{4.183542in}}%
\pgfpathcurveto{\pgfqpoint{4.577728in}{4.174334in}}{\pgfqpoint{4.572554in}{4.161843in}}{\pgfqpoint{4.572554in}{4.148820in}}%
\pgfpathcurveto{\pgfqpoint{4.572554in}{4.135797in}}{\pgfqpoint{4.577728in}{4.123306in}}{\pgfqpoint{4.586936in}{4.114098in}}%
\pgfpathcurveto{\pgfqpoint{4.596145in}{4.104889in}}{\pgfqpoint{4.608636in}{4.099715in}}{\pgfqpoint{4.621658in}{4.099715in}}%
\pgfpathlineto{\pgfqpoint{4.621658in}{4.099715in}}%
\pgfpathclose%
\pgfusepath{stroke,fill}%
\end{pgfscope}%
\begin{pgfscope}%
\pgfpathrectangle{\pgfqpoint{0.194833in}{0.246946in}}{\pgfqpoint{6.160000in}{6.160000in}}%
\pgfusepath{clip}%
\pgfsetbuttcap%
\pgfsetroundjoin%
\definecolor{currentfill}{rgb}{0.121569,0.466667,0.705882}%
\pgfsetfillcolor{currentfill}%
\pgfsetlinewidth{1.003750pt}%
\definecolor{currentstroke}{rgb}{0.121569,0.466667,0.705882}%
\pgfsetstrokecolor{currentstroke}%
\pgfsetdash{}{0pt}%
\pgfpathmoveto{\pgfqpoint{3.743097in}{4.158255in}}%
\pgfpathcurveto{\pgfqpoint{3.756120in}{4.158255in}}{\pgfqpoint{3.768611in}{4.163429in}}{\pgfqpoint{3.777819in}{4.172637in}}%
\pgfpathcurveto{\pgfqpoint{3.787028in}{4.181845in}}{\pgfqpoint{3.792202in}{4.194337in}}{\pgfqpoint{3.792202in}{4.207359in}}%
\pgfpathcurveto{\pgfqpoint{3.792202in}{4.220382in}}{\pgfqpoint{3.787028in}{4.232873in}}{\pgfqpoint{3.777819in}{4.242081in}}%
\pgfpathcurveto{\pgfqpoint{3.768611in}{4.251290in}}{\pgfqpoint{3.756120in}{4.256464in}}{\pgfqpoint{3.743097in}{4.256464in}}%
\pgfpathcurveto{\pgfqpoint{3.730075in}{4.256464in}}{\pgfqpoint{3.717583in}{4.251290in}}{\pgfqpoint{3.708375in}{4.242081in}}%
\pgfpathcurveto{\pgfqpoint{3.699167in}{4.232873in}}{\pgfqpoint{3.693993in}{4.220382in}}{\pgfqpoint{3.693993in}{4.207359in}}%
\pgfpathcurveto{\pgfqpoint{3.693993in}{4.194337in}}{\pgfqpoint{3.699167in}{4.181845in}}{\pgfqpoint{3.708375in}{4.172637in}}%
\pgfpathcurveto{\pgfqpoint{3.717583in}{4.163429in}}{\pgfqpoint{3.730075in}{4.158255in}}{\pgfqpoint{3.743097in}{4.158255in}}%
\pgfpathlineto{\pgfqpoint{3.743097in}{4.158255in}}%
\pgfpathclose%
\pgfusepath{stroke,fill}%
\end{pgfscope}%
\begin{pgfscope}%
\pgfpathrectangle{\pgfqpoint{0.194833in}{0.246946in}}{\pgfqpoint{6.160000in}{6.160000in}}%
\pgfusepath{clip}%
\pgfsetbuttcap%
\pgfsetroundjoin%
\definecolor{currentfill}{rgb}{0.121569,0.466667,0.705882}%
\pgfsetfillcolor{currentfill}%
\pgfsetlinewidth{1.003750pt}%
\definecolor{currentstroke}{rgb}{0.121569,0.466667,0.705882}%
\pgfsetstrokecolor{currentstroke}%
\pgfsetdash{}{0pt}%
\pgfpathmoveto{\pgfqpoint{3.346303in}{4.169378in}}%
\pgfpathcurveto{\pgfqpoint{3.359325in}{4.169378in}}{\pgfqpoint{3.371816in}{4.174552in}}{\pgfqpoint{3.381025in}{4.183760in}}%
\pgfpathcurveto{\pgfqpoint{3.390233in}{4.192969in}}{\pgfqpoint{3.395407in}{4.205460in}}{\pgfqpoint{3.395407in}{4.218482in}}%
\pgfpathcurveto{\pgfqpoint{3.395407in}{4.231505in}}{\pgfqpoint{3.390233in}{4.243996in}}{\pgfqpoint{3.381025in}{4.253205in}}%
\pgfpathcurveto{\pgfqpoint{3.371816in}{4.262413in}}{\pgfqpoint{3.359325in}{4.267587in}}{\pgfqpoint{3.346303in}{4.267587in}}%
\pgfpathcurveto{\pgfqpoint{3.333280in}{4.267587in}}{\pgfqpoint{3.320789in}{4.262413in}}{\pgfqpoint{3.311580in}{4.253205in}}%
\pgfpathcurveto{\pgfqpoint{3.302372in}{4.243996in}}{\pgfqpoint{3.297198in}{4.231505in}}{\pgfqpoint{3.297198in}{4.218482in}}%
\pgfpathcurveto{\pgfqpoint{3.297198in}{4.205460in}}{\pgfqpoint{3.302372in}{4.192969in}}{\pgfqpoint{3.311580in}{4.183760in}}%
\pgfpathcurveto{\pgfqpoint{3.320789in}{4.174552in}}{\pgfqpoint{3.333280in}{4.169378in}}{\pgfqpoint{3.346303in}{4.169378in}}%
\pgfpathlineto{\pgfqpoint{3.346303in}{4.169378in}}%
\pgfpathclose%
\pgfusepath{stroke,fill}%
\end{pgfscope}%
\begin{pgfscope}%
\pgfpathrectangle{\pgfqpoint{0.194833in}{0.246946in}}{\pgfqpoint{6.160000in}{6.160000in}}%
\pgfusepath{clip}%
\pgfsetbuttcap%
\pgfsetroundjoin%
\definecolor{currentfill}{rgb}{0.121569,0.466667,0.705882}%
\pgfsetfillcolor{currentfill}%
\pgfsetlinewidth{1.003750pt}%
\definecolor{currentstroke}{rgb}{0.121569,0.466667,0.705882}%
\pgfsetstrokecolor{currentstroke}%
\pgfsetdash{}{0pt}%
\pgfpathmoveto{\pgfqpoint{2.951564in}{4.182346in}}%
\pgfpathcurveto{\pgfqpoint{2.964587in}{4.182346in}}{\pgfqpoint{2.977078in}{4.187520in}}{\pgfqpoint{2.986286in}{4.196728in}}%
\pgfpathcurveto{\pgfqpoint{2.995495in}{4.205936in}}{\pgfqpoint{3.000669in}{4.218428in}}{\pgfqpoint{3.000669in}{4.231450in}}%
\pgfpathcurveto{\pgfqpoint{3.000669in}{4.244473in}}{\pgfqpoint{2.995495in}{4.256964in}}{\pgfqpoint{2.986286in}{4.266172in}}%
\pgfpathcurveto{\pgfqpoint{2.977078in}{4.275381in}}{\pgfqpoint{2.964587in}{4.280555in}}{\pgfqpoint{2.951564in}{4.280555in}}%
\pgfpathcurveto{\pgfqpoint{2.938542in}{4.280555in}}{\pgfqpoint{2.926050in}{4.275381in}}{\pgfqpoint{2.916842in}{4.266172in}}%
\pgfpathcurveto{\pgfqpoint{2.907634in}{4.256964in}}{\pgfqpoint{2.902460in}{4.244473in}}{\pgfqpoint{2.902460in}{4.231450in}}%
\pgfpathcurveto{\pgfqpoint{2.902460in}{4.218428in}}{\pgfqpoint{2.907634in}{4.205936in}}{\pgfqpoint{2.916842in}{4.196728in}}%
\pgfpathcurveto{\pgfqpoint{2.926050in}{4.187520in}}{\pgfqpoint{2.938542in}{4.182346in}}{\pgfqpoint{2.951564in}{4.182346in}}%
\pgfpathlineto{\pgfqpoint{2.951564in}{4.182346in}}%
\pgfpathclose%
\pgfusepath{stroke,fill}%
\end{pgfscope}%
\begin{pgfscope}%
\pgfpathrectangle{\pgfqpoint{0.194833in}{0.246946in}}{\pgfqpoint{6.160000in}{6.160000in}}%
\pgfusepath{clip}%
\pgfsetbuttcap%
\pgfsetroundjoin%
\definecolor{currentfill}{rgb}{0.121569,0.466667,0.705882}%
\pgfsetfillcolor{currentfill}%
\pgfsetlinewidth{1.003750pt}%
\definecolor{currentstroke}{rgb}{0.121569,0.466667,0.705882}%
\pgfsetstrokecolor{currentstroke}%
\pgfsetdash{}{0pt}%
\pgfpathmoveto{\pgfqpoint{2.558587in}{4.196275in}}%
\pgfpathcurveto{\pgfqpoint{2.571610in}{4.196275in}}{\pgfqpoint{2.584101in}{4.201449in}}{\pgfqpoint{2.593309in}{4.210658in}}%
\pgfpathcurveto{\pgfqpoint{2.602518in}{4.219866in}}{\pgfqpoint{2.607692in}{4.232357in}}{\pgfqpoint{2.607692in}{4.245380in}}%
\pgfpathcurveto{\pgfqpoint{2.607692in}{4.258403in}}{\pgfqpoint{2.602518in}{4.270894in}}{\pgfqpoint{2.593309in}{4.280102in}}%
\pgfpathcurveto{\pgfqpoint{2.584101in}{4.289311in}}{\pgfqpoint{2.571610in}{4.294485in}}{\pgfqpoint{2.558587in}{4.294485in}}%
\pgfpathcurveto{\pgfqpoint{2.545565in}{4.294485in}}{\pgfqpoint{2.533073in}{4.289311in}}{\pgfqpoint{2.523865in}{4.280102in}}%
\pgfpathcurveto{\pgfqpoint{2.514657in}{4.270894in}}{\pgfqpoint{2.509483in}{4.258403in}}{\pgfqpoint{2.509483in}{4.245380in}}%
\pgfpathcurveto{\pgfqpoint{2.509483in}{4.232357in}}{\pgfqpoint{2.514657in}{4.219866in}}{\pgfqpoint{2.523865in}{4.210658in}}%
\pgfpathcurveto{\pgfqpoint{2.533073in}{4.201449in}}{\pgfqpoint{2.545565in}{4.196275in}}{\pgfqpoint{2.558587in}{4.196275in}}%
\pgfpathlineto{\pgfqpoint{2.558587in}{4.196275in}}%
\pgfpathclose%
\pgfusepath{stroke,fill}%
\end{pgfscope}%
\begin{pgfscope}%
\pgfpathrectangle{\pgfqpoint{0.194833in}{0.246946in}}{\pgfqpoint{6.160000in}{6.160000in}}%
\pgfusepath{clip}%
\pgfsetbuttcap%
\pgfsetroundjoin%
\definecolor{currentfill}{rgb}{0.121569,0.466667,0.705882}%
\pgfsetfillcolor{currentfill}%
\pgfsetlinewidth{1.003750pt}%
\definecolor{currentstroke}{rgb}{0.121569,0.466667,0.705882}%
\pgfsetstrokecolor{currentstroke}%
\pgfsetdash{}{0pt}%
\pgfpathmoveto{\pgfqpoint{2.170319in}{4.210454in}}%
\pgfpathcurveto{\pgfqpoint{2.183342in}{4.210454in}}{\pgfqpoint{2.195833in}{4.215628in}}{\pgfqpoint{2.205042in}{4.224836in}}%
\pgfpathcurveto{\pgfqpoint{2.214250in}{4.234045in}}{\pgfqpoint{2.219424in}{4.246536in}}{\pgfqpoint{2.219424in}{4.259559in}}%
\pgfpathcurveto{\pgfqpoint{2.219424in}{4.272581in}}{\pgfqpoint{2.214250in}{4.285072in}}{\pgfqpoint{2.205042in}{4.294281in}}%
\pgfpathcurveto{\pgfqpoint{2.195833in}{4.303489in}}{\pgfqpoint{2.183342in}{4.308663in}}{\pgfqpoint{2.170319in}{4.308663in}}%
\pgfpathcurveto{\pgfqpoint{2.157297in}{4.308663in}}{\pgfqpoint{2.144806in}{4.303489in}}{\pgfqpoint{2.135597in}{4.294281in}}%
\pgfpathcurveto{\pgfqpoint{2.126389in}{4.285072in}}{\pgfqpoint{2.121215in}{4.272581in}}{\pgfqpoint{2.121215in}{4.259559in}}%
\pgfpathcurveto{\pgfqpoint{2.121215in}{4.246536in}}{\pgfqpoint{2.126389in}{4.234045in}}{\pgfqpoint{2.135597in}{4.224836in}}%
\pgfpathcurveto{\pgfqpoint{2.144806in}{4.215628in}}{\pgfqpoint{2.157297in}{4.210454in}}{\pgfqpoint{2.170319in}{4.210454in}}%
\pgfpathlineto{\pgfqpoint{2.170319in}{4.210454in}}%
\pgfpathclose%
\pgfusepath{stroke,fill}%
\end{pgfscope}%
\begin{pgfscope}%
\pgfpathrectangle{\pgfqpoint{0.194833in}{0.246946in}}{\pgfqpoint{6.160000in}{6.160000in}}%
\pgfusepath{clip}%
\pgfsetbuttcap%
\pgfsetroundjoin%
\definecolor{currentfill}{rgb}{0.121569,0.466667,0.705882}%
\pgfsetfillcolor{currentfill}%
\pgfsetlinewidth{1.003750pt}%
\definecolor{currentstroke}{rgb}{0.121569,0.466667,0.705882}%
\pgfsetstrokecolor{currentstroke}%
\pgfsetdash{}{0pt}%
\pgfpathmoveto{\pgfqpoint{4.047253in}{4.342537in}}%
\pgfpathcurveto{\pgfqpoint{4.060275in}{4.342537in}}{\pgfqpoint{4.072767in}{4.347711in}}{\pgfqpoint{4.081975in}{4.356920in}}%
\pgfpathcurveto{\pgfqpoint{4.091183in}{4.366128in}}{\pgfqpoint{4.096357in}{4.378619in}}{\pgfqpoint{4.096357in}{4.391642in}}%
\pgfpathcurveto{\pgfqpoint{4.096357in}{4.404665in}}{\pgfqpoint{4.091183in}{4.417156in}}{\pgfqpoint{4.081975in}{4.426364in}}%
\pgfpathcurveto{\pgfqpoint{4.072767in}{4.435573in}}{\pgfqpoint{4.060275in}{4.440747in}}{\pgfqpoint{4.047253in}{4.440747in}}%
\pgfpathcurveto{\pgfqpoint{4.034230in}{4.440747in}}{\pgfqpoint{4.021739in}{4.435573in}}{\pgfqpoint{4.012531in}{4.426364in}}%
\pgfpathcurveto{\pgfqpoint{4.003322in}{4.417156in}}{\pgfqpoint{3.998148in}{4.404665in}}{\pgfqpoint{3.998148in}{4.391642in}}%
\pgfpathcurveto{\pgfqpoint{3.998148in}{4.378619in}}{\pgfqpoint{4.003322in}{4.366128in}}{\pgfqpoint{4.012531in}{4.356920in}}%
\pgfpathcurveto{\pgfqpoint{4.021739in}{4.347711in}}{\pgfqpoint{4.034230in}{4.342537in}}{\pgfqpoint{4.047253in}{4.342537in}}%
\pgfpathlineto{\pgfqpoint{4.047253in}{4.342537in}}%
\pgfpathclose%
\pgfusepath{stroke,fill}%
\end{pgfscope}%
\begin{pgfscope}%
\pgfpathrectangle{\pgfqpoint{0.194833in}{0.246946in}}{\pgfqpoint{6.160000in}{6.160000in}}%
\pgfusepath{clip}%
\pgfsetbuttcap%
\pgfsetroundjoin%
\definecolor{currentfill}{rgb}{0.121569,0.466667,0.705882}%
\pgfsetfillcolor{currentfill}%
\pgfsetlinewidth{1.003750pt}%
\definecolor{currentstroke}{rgb}{0.121569,0.466667,0.705882}%
\pgfsetstrokecolor{currentstroke}%
\pgfsetdash{}{0pt}%
\pgfpathmoveto{\pgfqpoint{4.439248in}{4.366114in}}%
\pgfpathcurveto{\pgfqpoint{4.452271in}{4.366114in}}{\pgfqpoint{4.464762in}{4.371288in}}{\pgfqpoint{4.473970in}{4.380497in}}%
\pgfpathcurveto{\pgfqpoint{4.483178in}{4.389705in}}{\pgfqpoint{4.488352in}{4.402196in}}{\pgfqpoint{4.488352in}{4.415219in}}%
\pgfpathcurveto{\pgfqpoint{4.488352in}{4.428242in}}{\pgfqpoint{4.483178in}{4.440733in}}{\pgfqpoint{4.473970in}{4.449941in}}%
\pgfpathcurveto{\pgfqpoint{4.464762in}{4.459150in}}{\pgfqpoint{4.452271in}{4.464324in}}{\pgfqpoint{4.439248in}{4.464324in}}%
\pgfpathcurveto{\pgfqpoint{4.426225in}{4.464324in}}{\pgfqpoint{4.413734in}{4.459150in}}{\pgfqpoint{4.404526in}{4.449941in}}%
\pgfpathcurveto{\pgfqpoint{4.395317in}{4.440733in}}{\pgfqpoint{4.390143in}{4.428242in}}{\pgfqpoint{4.390143in}{4.415219in}}%
\pgfpathcurveto{\pgfqpoint{4.390143in}{4.402196in}}{\pgfqpoint{4.395317in}{4.389705in}}{\pgfqpoint{4.404526in}{4.380497in}}%
\pgfpathcurveto{\pgfqpoint{4.413734in}{4.371288in}}{\pgfqpoint{4.426225in}{4.366114in}}{\pgfqpoint{4.439248in}{4.366114in}}%
\pgfpathlineto{\pgfqpoint{4.439248in}{4.366114in}}%
\pgfpathclose%
\pgfusepath{stroke,fill}%
\end{pgfscope}%
\begin{pgfscope}%
\pgfpathrectangle{\pgfqpoint{0.194833in}{0.246946in}}{\pgfqpoint{6.160000in}{6.160000in}}%
\pgfusepath{clip}%
\pgfsetbuttcap%
\pgfsetroundjoin%
\definecolor{currentfill}{rgb}{0.121569,0.466667,0.705882}%
\pgfsetfillcolor{currentfill}%
\pgfsetlinewidth{1.003750pt}%
\definecolor{currentstroke}{rgb}{0.121569,0.466667,0.705882}%
\pgfsetstrokecolor{currentstroke}%
\pgfsetdash{}{0pt}%
\pgfpathmoveto{\pgfqpoint{3.540385in}{4.457200in}}%
\pgfpathcurveto{\pgfqpoint{3.553408in}{4.457200in}}{\pgfqpoint{3.565899in}{4.462374in}}{\pgfqpoint{3.575107in}{4.471583in}}%
\pgfpathcurveto{\pgfqpoint{3.584316in}{4.480791in}}{\pgfqpoint{3.589490in}{4.493282in}}{\pgfqpoint{3.589490in}{4.506305in}}%
\pgfpathcurveto{\pgfqpoint{3.589490in}{4.519328in}}{\pgfqpoint{3.584316in}{4.531819in}}{\pgfqpoint{3.575107in}{4.541027in}}%
\pgfpathcurveto{\pgfqpoint{3.565899in}{4.550236in}}{\pgfqpoint{3.553408in}{4.555410in}}{\pgfqpoint{3.540385in}{4.555410in}}%
\pgfpathcurveto{\pgfqpoint{3.527362in}{4.555410in}}{\pgfqpoint{3.514871in}{4.550236in}}{\pgfqpoint{3.505663in}{4.541027in}}%
\pgfpathcurveto{\pgfqpoint{3.496454in}{4.531819in}}{\pgfqpoint{3.491280in}{4.519328in}}{\pgfqpoint{3.491280in}{4.506305in}}%
\pgfpathcurveto{\pgfqpoint{3.491280in}{4.493282in}}{\pgfqpoint{3.496454in}{4.480791in}}{\pgfqpoint{3.505663in}{4.471583in}}%
\pgfpathcurveto{\pgfqpoint{3.514871in}{4.462374in}}{\pgfqpoint{3.527362in}{4.457200in}}{\pgfqpoint{3.540385in}{4.457200in}}%
\pgfpathlineto{\pgfqpoint{3.540385in}{4.457200in}}%
\pgfpathclose%
\pgfusepath{stroke,fill}%
\end{pgfscope}%
\begin{pgfscope}%
\pgfpathrectangle{\pgfqpoint{0.194833in}{0.246946in}}{\pgfqpoint{6.160000in}{6.160000in}}%
\pgfusepath{clip}%
\pgfsetbuttcap%
\pgfsetroundjoin%
\definecolor{currentfill}{rgb}{0.121569,0.466667,0.705882}%
\pgfsetfillcolor{currentfill}%
\pgfsetlinewidth{1.003750pt}%
\definecolor{currentstroke}{rgb}{0.121569,0.466667,0.705882}%
\pgfsetstrokecolor{currentstroke}%
\pgfsetdash{}{0pt}%
\pgfpathmoveto{\pgfqpoint{3.148957in}{4.467171in}}%
\pgfpathcurveto{\pgfqpoint{3.161980in}{4.467171in}}{\pgfqpoint{3.174471in}{4.472345in}}{\pgfqpoint{3.183679in}{4.481553in}}%
\pgfpathcurveto{\pgfqpoint{3.192888in}{4.490762in}}{\pgfqpoint{3.198062in}{4.503253in}}{\pgfqpoint{3.198062in}{4.516275in}}%
\pgfpathcurveto{\pgfqpoint{3.198062in}{4.529298in}}{\pgfqpoint{3.192888in}{4.541789in}}{\pgfqpoint{3.183679in}{4.550998in}}%
\pgfpathcurveto{\pgfqpoint{3.174471in}{4.560206in}}{\pgfqpoint{3.161980in}{4.565380in}}{\pgfqpoint{3.148957in}{4.565380in}}%
\pgfpathcurveto{\pgfqpoint{3.135934in}{4.565380in}}{\pgfqpoint{3.123443in}{4.560206in}}{\pgfqpoint{3.114235in}{4.550998in}}%
\pgfpathcurveto{\pgfqpoint{3.105026in}{4.541789in}}{\pgfqpoint{3.099852in}{4.529298in}}{\pgfqpoint{3.099852in}{4.516275in}}%
\pgfpathcurveto{\pgfqpoint{3.099852in}{4.503253in}}{\pgfqpoint{3.105026in}{4.490762in}}{\pgfqpoint{3.114235in}{4.481553in}}%
\pgfpathcurveto{\pgfqpoint{3.123443in}{4.472345in}}{\pgfqpoint{3.135934in}{4.467171in}}{\pgfqpoint{3.148957in}{4.467171in}}%
\pgfpathlineto{\pgfqpoint{3.148957in}{4.467171in}}%
\pgfpathclose%
\pgfusepath{stroke,fill}%
\end{pgfscope}%
\begin{pgfscope}%
\pgfpathrectangle{\pgfqpoint{0.194833in}{0.246946in}}{\pgfqpoint{6.160000in}{6.160000in}}%
\pgfusepath{clip}%
\pgfsetbuttcap%
\pgfsetroundjoin%
\definecolor{currentfill}{rgb}{0.121569,0.466667,0.705882}%
\pgfsetfillcolor{currentfill}%
\pgfsetlinewidth{1.003750pt}%
\definecolor{currentstroke}{rgb}{0.121569,0.466667,0.705882}%
\pgfsetstrokecolor{currentstroke}%
\pgfsetdash{}{0pt}%
\pgfpathmoveto{\pgfqpoint{2.755923in}{4.480817in}}%
\pgfpathcurveto{\pgfqpoint{2.768946in}{4.480817in}}{\pgfqpoint{2.781437in}{4.485991in}}{\pgfqpoint{2.790646in}{4.495200in}}%
\pgfpathcurveto{\pgfqpoint{2.799854in}{4.504408in}}{\pgfqpoint{2.805028in}{4.516899in}}{\pgfqpoint{2.805028in}{4.529922in}}%
\pgfpathcurveto{\pgfqpoint{2.805028in}{4.542945in}}{\pgfqpoint{2.799854in}{4.555436in}}{\pgfqpoint{2.790646in}{4.564644in}}%
\pgfpathcurveto{\pgfqpoint{2.781437in}{4.573853in}}{\pgfqpoint{2.768946in}{4.579027in}}{\pgfqpoint{2.755923in}{4.579027in}}%
\pgfpathcurveto{\pgfqpoint{2.742901in}{4.579027in}}{\pgfqpoint{2.730410in}{4.573853in}}{\pgfqpoint{2.721201in}{4.564644in}}%
\pgfpathcurveto{\pgfqpoint{2.711993in}{4.555436in}}{\pgfqpoint{2.706819in}{4.542945in}}{\pgfqpoint{2.706819in}{4.529922in}}%
\pgfpathcurveto{\pgfqpoint{2.706819in}{4.516899in}}{\pgfqpoint{2.711993in}{4.504408in}}{\pgfqpoint{2.721201in}{4.495200in}}%
\pgfpathcurveto{\pgfqpoint{2.730410in}{4.485991in}}{\pgfqpoint{2.742901in}{4.480817in}}{\pgfqpoint{2.755923in}{4.480817in}}%
\pgfpathlineto{\pgfqpoint{2.755923in}{4.480817in}}%
\pgfpathclose%
\pgfusepath{stroke,fill}%
\end{pgfscope}%
\begin{pgfscope}%
\pgfpathrectangle{\pgfqpoint{0.194833in}{0.246946in}}{\pgfqpoint{6.160000in}{6.160000in}}%
\pgfusepath{clip}%
\pgfsetbuttcap%
\pgfsetroundjoin%
\definecolor{currentfill}{rgb}{0.121569,0.466667,0.705882}%
\pgfsetfillcolor{currentfill}%
\pgfsetlinewidth{1.003750pt}%
\definecolor{currentstroke}{rgb}{0.121569,0.466667,0.705882}%
\pgfsetstrokecolor{currentstroke}%
\pgfsetdash{}{0pt}%
\pgfpathmoveto{\pgfqpoint{2.366397in}{4.496813in}}%
\pgfpathcurveto{\pgfqpoint{2.379420in}{4.496813in}}{\pgfqpoint{2.391911in}{4.501987in}}{\pgfqpoint{2.401120in}{4.511196in}}%
\pgfpathcurveto{\pgfqpoint{2.410328in}{4.520404in}}{\pgfqpoint{2.415502in}{4.532895in}}{\pgfqpoint{2.415502in}{4.545918in}}%
\pgfpathcurveto{\pgfqpoint{2.415502in}{4.558941in}}{\pgfqpoint{2.410328in}{4.571432in}}{\pgfqpoint{2.401120in}{4.580640in}}%
\pgfpathcurveto{\pgfqpoint{2.391911in}{4.589849in}}{\pgfqpoint{2.379420in}{4.595023in}}{\pgfqpoint{2.366397in}{4.595023in}}%
\pgfpathcurveto{\pgfqpoint{2.353375in}{4.595023in}}{\pgfqpoint{2.340884in}{4.589849in}}{\pgfqpoint{2.331675in}{4.580640in}}%
\pgfpathcurveto{\pgfqpoint{2.322467in}{4.571432in}}{\pgfqpoint{2.317293in}{4.558941in}}{\pgfqpoint{2.317293in}{4.545918in}}%
\pgfpathcurveto{\pgfqpoint{2.317293in}{4.532895in}}{\pgfqpoint{2.322467in}{4.520404in}}{\pgfqpoint{2.331675in}{4.511196in}}%
\pgfpathcurveto{\pgfqpoint{2.340884in}{4.501987in}}{\pgfqpoint{2.353375in}{4.496813in}}{\pgfqpoint{2.366397in}{4.496813in}}%
\pgfpathlineto{\pgfqpoint{2.366397in}{4.496813in}}%
\pgfpathclose%
\pgfusepath{stroke,fill}%
\end{pgfscope}%
\begin{pgfscope}%
\pgfpathrectangle{\pgfqpoint{0.194833in}{0.246946in}}{\pgfqpoint{6.160000in}{6.160000in}}%
\pgfusepath{clip}%
\pgfsetbuttcap%
\pgfsetroundjoin%
\definecolor{currentfill}{rgb}{0.121569,0.466667,0.705882}%
\pgfsetfillcolor{currentfill}%
\pgfsetlinewidth{1.003750pt}%
\definecolor{currentstroke}{rgb}{0.121569,0.466667,0.705882}%
\pgfsetstrokecolor{currentstroke}%
\pgfsetdash{}{0pt}%
\pgfpathmoveto{\pgfqpoint{3.869884in}{4.599537in}}%
\pgfpathcurveto{\pgfqpoint{3.882906in}{4.599537in}}{\pgfqpoint{3.895397in}{4.604711in}}{\pgfqpoint{3.904606in}{4.613920in}}%
\pgfpathcurveto{\pgfqpoint{3.913814in}{4.623128in}}{\pgfqpoint{3.918988in}{4.635619in}}{\pgfqpoint{3.918988in}{4.648642in}}%
\pgfpathcurveto{\pgfqpoint{3.918988in}{4.661665in}}{\pgfqpoint{3.913814in}{4.674156in}}{\pgfqpoint{3.904606in}{4.683364in}}%
\pgfpathcurveto{\pgfqpoint{3.895397in}{4.692573in}}{\pgfqpoint{3.882906in}{4.697747in}}{\pgfqpoint{3.869884in}{4.697747in}}%
\pgfpathcurveto{\pgfqpoint{3.856861in}{4.697747in}}{\pgfqpoint{3.844370in}{4.692573in}}{\pgfqpoint{3.835161in}{4.683364in}}%
\pgfpathcurveto{\pgfqpoint{3.825953in}{4.674156in}}{\pgfqpoint{3.820779in}{4.661665in}}{\pgfqpoint{3.820779in}{4.648642in}}%
\pgfpathcurveto{\pgfqpoint{3.820779in}{4.635619in}}{\pgfqpoint{3.825953in}{4.623128in}}{\pgfqpoint{3.835161in}{4.613920in}}%
\pgfpathcurveto{\pgfqpoint{3.844370in}{4.604711in}}{\pgfqpoint{3.856861in}{4.599537in}}{\pgfqpoint{3.869884in}{4.599537in}}%
\pgfpathlineto{\pgfqpoint{3.869884in}{4.599537in}}%
\pgfpathclose%
\pgfusepath{stroke,fill}%
\end{pgfscope}%
\begin{pgfscope}%
\pgfpathrectangle{\pgfqpoint{0.194833in}{0.246946in}}{\pgfqpoint{6.160000in}{6.160000in}}%
\pgfusepath{clip}%
\pgfsetbuttcap%
\pgfsetroundjoin%
\definecolor{currentfill}{rgb}{0.121569,0.466667,0.705882}%
\pgfsetfillcolor{currentfill}%
\pgfsetlinewidth{1.003750pt}%
\definecolor{currentstroke}{rgb}{0.121569,0.466667,0.705882}%
\pgfsetstrokecolor{currentstroke}%
\pgfsetdash{}{0pt}%
\pgfpathmoveto{\pgfqpoint{4.253926in}{4.636765in}}%
\pgfpathcurveto{\pgfqpoint{4.266948in}{4.636765in}}{\pgfqpoint{4.279440in}{4.641939in}}{\pgfqpoint{4.288648in}{4.651148in}}%
\pgfpathcurveto{\pgfqpoint{4.297856in}{4.660356in}}{\pgfqpoint{4.303030in}{4.672847in}}{\pgfqpoint{4.303030in}{4.685870in}}%
\pgfpathcurveto{\pgfqpoint{4.303030in}{4.698892in}}{\pgfqpoint{4.297856in}{4.711384in}}{\pgfqpoint{4.288648in}{4.720592in}}%
\pgfpathcurveto{\pgfqpoint{4.279440in}{4.729800in}}{\pgfqpoint{4.266948in}{4.734974in}}{\pgfqpoint{4.253926in}{4.734974in}}%
\pgfpathcurveto{\pgfqpoint{4.240903in}{4.734974in}}{\pgfqpoint{4.228412in}{4.729800in}}{\pgfqpoint{4.219204in}{4.720592in}}%
\pgfpathcurveto{\pgfqpoint{4.209995in}{4.711384in}}{\pgfqpoint{4.204821in}{4.698892in}}{\pgfqpoint{4.204821in}{4.685870in}}%
\pgfpathcurveto{\pgfqpoint{4.204821in}{4.672847in}}{\pgfqpoint{4.209995in}{4.660356in}}{\pgfqpoint{4.219204in}{4.651148in}}%
\pgfpathcurveto{\pgfqpoint{4.228412in}{4.641939in}}{\pgfqpoint{4.240903in}{4.636765in}}{\pgfqpoint{4.253926in}{4.636765in}}%
\pgfpathlineto{\pgfqpoint{4.253926in}{4.636765in}}%
\pgfpathclose%
\pgfusepath{stroke,fill}%
\end{pgfscope}%
\begin{pgfscope}%
\pgfpathrectangle{\pgfqpoint{0.194833in}{0.246946in}}{\pgfqpoint{6.160000in}{6.160000in}}%
\pgfusepath{clip}%
\pgfsetbuttcap%
\pgfsetroundjoin%
\definecolor{currentfill}{rgb}{0.121569,0.466667,0.705882}%
\pgfsetfillcolor{currentfill}%
\pgfsetlinewidth{1.003750pt}%
\definecolor{currentstroke}{rgb}{0.121569,0.466667,0.705882}%
\pgfsetstrokecolor{currentstroke}%
\pgfsetdash{}{0pt}%
\pgfpathmoveto{\pgfqpoint{3.338756in}{4.756535in}}%
\pgfpathcurveto{\pgfqpoint{3.351779in}{4.756535in}}{\pgfqpoint{3.364270in}{4.761709in}}{\pgfqpoint{3.373478in}{4.770917in}}%
\pgfpathcurveto{\pgfqpoint{3.382687in}{4.780126in}}{\pgfqpoint{3.387861in}{4.792617in}}{\pgfqpoint{3.387861in}{4.805640in}}%
\pgfpathcurveto{\pgfqpoint{3.387861in}{4.818662in}}{\pgfqpoint{3.382687in}{4.831153in}}{\pgfqpoint{3.373478in}{4.840362in}}%
\pgfpathcurveto{\pgfqpoint{3.364270in}{4.849570in}}{\pgfqpoint{3.351779in}{4.854744in}}{\pgfqpoint{3.338756in}{4.854744in}}%
\pgfpathcurveto{\pgfqpoint{3.325733in}{4.854744in}}{\pgfqpoint{3.313242in}{4.849570in}}{\pgfqpoint{3.304034in}{4.840362in}}%
\pgfpathcurveto{\pgfqpoint{3.294825in}{4.831153in}}{\pgfqpoint{3.289651in}{4.818662in}}{\pgfqpoint{3.289651in}{4.805640in}}%
\pgfpathcurveto{\pgfqpoint{3.289651in}{4.792617in}}{\pgfqpoint{3.294825in}{4.780126in}}{\pgfqpoint{3.304034in}{4.770917in}}%
\pgfpathcurveto{\pgfqpoint{3.313242in}{4.761709in}}{\pgfqpoint{3.325733in}{4.756535in}}{\pgfqpoint{3.338756in}{4.756535in}}%
\pgfpathlineto{\pgfqpoint{3.338756in}{4.756535in}}%
\pgfpathclose%
\pgfusepath{stroke,fill}%
\end{pgfscope}%
\begin{pgfscope}%
\pgfpathrectangle{\pgfqpoint{0.194833in}{0.246946in}}{\pgfqpoint{6.160000in}{6.160000in}}%
\pgfusepath{clip}%
\pgfsetbuttcap%
\pgfsetroundjoin%
\definecolor{currentfill}{rgb}{0.121569,0.466667,0.705882}%
\pgfsetfillcolor{currentfill}%
\pgfsetlinewidth{1.003750pt}%
\definecolor{currentstroke}{rgb}{0.121569,0.466667,0.705882}%
\pgfsetstrokecolor{currentstroke}%
\pgfsetdash{}{0pt}%
\pgfpathmoveto{\pgfqpoint{2.950563in}{4.764973in}}%
\pgfpathcurveto{\pgfqpoint{2.963586in}{4.764973in}}{\pgfqpoint{2.976077in}{4.770147in}}{\pgfqpoint{2.985285in}{4.779356in}}%
\pgfpathcurveto{\pgfqpoint{2.994494in}{4.788564in}}{\pgfqpoint{2.999668in}{4.801055in}}{\pgfqpoint{2.999668in}{4.814078in}}%
\pgfpathcurveto{\pgfqpoint{2.999668in}{4.827100in}}{\pgfqpoint{2.994494in}{4.839592in}}{\pgfqpoint{2.985285in}{4.848800in}}%
\pgfpathcurveto{\pgfqpoint{2.976077in}{4.858008in}}{\pgfqpoint{2.963586in}{4.863182in}}{\pgfqpoint{2.950563in}{4.863182in}}%
\pgfpathcurveto{\pgfqpoint{2.937540in}{4.863182in}}{\pgfqpoint{2.925049in}{4.858008in}}{\pgfqpoint{2.915841in}{4.848800in}}%
\pgfpathcurveto{\pgfqpoint{2.906632in}{4.839592in}}{\pgfqpoint{2.901458in}{4.827100in}}{\pgfqpoint{2.901458in}{4.814078in}}%
\pgfpathcurveto{\pgfqpoint{2.901458in}{4.801055in}}{\pgfqpoint{2.906632in}{4.788564in}}{\pgfqpoint{2.915841in}{4.779356in}}%
\pgfpathcurveto{\pgfqpoint{2.925049in}{4.770147in}}{\pgfqpoint{2.937540in}{4.764973in}}{\pgfqpoint{2.950563in}{4.764973in}}%
\pgfpathlineto{\pgfqpoint{2.950563in}{4.764973in}}%
\pgfpathclose%
\pgfusepath{stroke,fill}%
\end{pgfscope}%
\begin{pgfscope}%
\pgfpathrectangle{\pgfqpoint{0.194833in}{0.246946in}}{\pgfqpoint{6.160000in}{6.160000in}}%
\pgfusepath{clip}%
\pgfsetbuttcap%
\pgfsetroundjoin%
\definecolor{currentfill}{rgb}{0.121569,0.466667,0.705882}%
\pgfsetfillcolor{currentfill}%
\pgfsetlinewidth{1.003750pt}%
\definecolor{currentstroke}{rgb}{0.121569,0.466667,0.705882}%
\pgfsetstrokecolor{currentstroke}%
\pgfsetdash{}{0pt}%
\pgfpathmoveto{\pgfqpoint{2.562580in}{4.783325in}}%
\pgfpathcurveto{\pgfqpoint{2.575603in}{4.783325in}}{\pgfqpoint{2.588094in}{4.788499in}}{\pgfqpoint{2.597302in}{4.797708in}}%
\pgfpathcurveto{\pgfqpoint{2.606511in}{4.806916in}}{\pgfqpoint{2.611685in}{4.819407in}}{\pgfqpoint{2.611685in}{4.832430in}}%
\pgfpathcurveto{\pgfqpoint{2.611685in}{4.845453in}}{\pgfqpoint{2.606511in}{4.857944in}}{\pgfqpoint{2.597302in}{4.867152in}}%
\pgfpathcurveto{\pgfqpoint{2.588094in}{4.876361in}}{\pgfqpoint{2.575603in}{4.881535in}}{\pgfqpoint{2.562580in}{4.881535in}}%
\pgfpathcurveto{\pgfqpoint{2.549557in}{4.881535in}}{\pgfqpoint{2.537066in}{4.876361in}}{\pgfqpoint{2.527858in}{4.867152in}}%
\pgfpathcurveto{\pgfqpoint{2.518649in}{4.857944in}}{\pgfqpoint{2.513475in}{4.845453in}}{\pgfqpoint{2.513475in}{4.832430in}}%
\pgfpathcurveto{\pgfqpoint{2.513475in}{4.819407in}}{\pgfqpoint{2.518649in}{4.806916in}}{\pgfqpoint{2.527858in}{4.797708in}}%
\pgfpathcurveto{\pgfqpoint{2.537066in}{4.788499in}}{\pgfqpoint{2.549557in}{4.783325in}}{\pgfqpoint{2.562580in}{4.783325in}}%
\pgfpathlineto{\pgfqpoint{2.562580in}{4.783325in}}%
\pgfpathclose%
\pgfusepath{stroke,fill}%
\end{pgfscope}%
\begin{pgfscope}%
\pgfpathrectangle{\pgfqpoint{0.194833in}{0.246946in}}{\pgfqpoint{6.160000in}{6.160000in}}%
\pgfusepath{clip}%
\pgfsetbuttcap%
\pgfsetroundjoin%
\definecolor{currentfill}{rgb}{0.121569,0.466667,0.705882}%
\pgfsetfillcolor{currentfill}%
\pgfsetlinewidth{1.003750pt}%
\definecolor{currentstroke}{rgb}{0.121569,0.466667,0.705882}%
\pgfsetstrokecolor{currentstroke}%
\pgfsetdash{}{0pt}%
\pgfpathmoveto{\pgfqpoint{3.690612in}{4.854988in}}%
\pgfpathcurveto{\pgfqpoint{3.703634in}{4.854988in}}{\pgfqpoint{3.716125in}{4.860162in}}{\pgfqpoint{3.725334in}{4.869370in}}%
\pgfpathcurveto{\pgfqpoint{3.734542in}{4.878579in}}{\pgfqpoint{3.739716in}{4.891070in}}{\pgfqpoint{3.739716in}{4.904092in}}%
\pgfpathcurveto{\pgfqpoint{3.739716in}{4.917115in}}{\pgfqpoint{3.734542in}{4.929606in}}{\pgfqpoint{3.725334in}{4.938815in}}%
\pgfpathcurveto{\pgfqpoint{3.716125in}{4.948023in}}{\pgfqpoint{3.703634in}{4.953197in}}{\pgfqpoint{3.690612in}{4.953197in}}%
\pgfpathcurveto{\pgfqpoint{3.677589in}{4.953197in}}{\pgfqpoint{3.665098in}{4.948023in}}{\pgfqpoint{3.655889in}{4.938815in}}%
\pgfpathcurveto{\pgfqpoint{3.646681in}{4.929606in}}{\pgfqpoint{3.641507in}{4.917115in}}{\pgfqpoint{3.641507in}{4.904092in}}%
\pgfpathcurveto{\pgfqpoint{3.641507in}{4.891070in}}{\pgfqpoint{3.646681in}{4.878579in}}{\pgfqpoint{3.655889in}{4.869370in}}%
\pgfpathcurveto{\pgfqpoint{3.665098in}{4.860162in}}{\pgfqpoint{3.677589in}{4.854988in}}{\pgfqpoint{3.690612in}{4.854988in}}%
\pgfpathlineto{\pgfqpoint{3.690612in}{4.854988in}}%
\pgfpathclose%
\pgfusepath{stroke,fill}%
\end{pgfscope}%
\begin{pgfscope}%
\pgfpathrectangle{\pgfqpoint{0.194833in}{0.246946in}}{\pgfqpoint{6.160000in}{6.160000in}}%
\pgfusepath{clip}%
\pgfsetbuttcap%
\pgfsetroundjoin%
\definecolor{currentfill}{rgb}{0.121569,0.466667,0.705882}%
\pgfsetfillcolor{currentfill}%
\pgfsetlinewidth{1.003750pt}%
\definecolor{currentstroke}{rgb}{0.121569,0.466667,0.705882}%
\pgfsetstrokecolor{currentstroke}%
\pgfsetdash{}{0pt}%
\pgfpathmoveto{\pgfqpoint{4.073061in}{4.900907in}}%
\pgfpathcurveto{\pgfqpoint{4.086084in}{4.900907in}}{\pgfqpoint{4.098575in}{4.906081in}}{\pgfqpoint{4.107783in}{4.915289in}}%
\pgfpathcurveto{\pgfqpoint{4.116992in}{4.924497in}}{\pgfqpoint{4.122166in}{4.936989in}}{\pgfqpoint{4.122166in}{4.950011in}}%
\pgfpathcurveto{\pgfqpoint{4.122166in}{4.963034in}}{\pgfqpoint{4.116992in}{4.975525in}}{\pgfqpoint{4.107783in}{4.984733in}}%
\pgfpathcurveto{\pgfqpoint{4.098575in}{4.993942in}}{\pgfqpoint{4.086084in}{4.999116in}}{\pgfqpoint{4.073061in}{4.999116in}}%
\pgfpathcurveto{\pgfqpoint{4.060038in}{4.999116in}}{\pgfqpoint{4.047547in}{4.993942in}}{\pgfqpoint{4.038339in}{4.984733in}}%
\pgfpathcurveto{\pgfqpoint{4.029130in}{4.975525in}}{\pgfqpoint{4.023956in}{4.963034in}}{\pgfqpoint{4.023956in}{4.950011in}}%
\pgfpathcurveto{\pgfqpoint{4.023956in}{4.936989in}}{\pgfqpoint{4.029130in}{4.924497in}}{\pgfqpoint{4.038339in}{4.915289in}}%
\pgfpathcurveto{\pgfqpoint{4.047547in}{4.906081in}}{\pgfqpoint{4.060038in}{4.900907in}}{\pgfqpoint{4.073061in}{4.900907in}}%
\pgfpathlineto{\pgfqpoint{4.073061in}{4.900907in}}%
\pgfpathclose%
\pgfusepath{stroke,fill}%
\end{pgfscope}%
\begin{pgfscope}%
\pgfpathrectangle{\pgfqpoint{0.194833in}{0.246946in}}{\pgfqpoint{6.160000in}{6.160000in}}%
\pgfusepath{clip}%
\pgfsetbuttcap%
\pgfsetroundjoin%
\definecolor{currentfill}{rgb}{0.121569,0.466667,0.705882}%
\pgfsetfillcolor{currentfill}%
\pgfsetlinewidth{1.003750pt}%
\definecolor{currentstroke}{rgb}{0.121569,0.466667,0.705882}%
\pgfsetstrokecolor{currentstroke}%
\pgfsetdash{}{0pt}%
\pgfpathmoveto{\pgfqpoint{3.142786in}{5.052550in}}%
\pgfpathcurveto{\pgfqpoint{3.155809in}{5.052550in}}{\pgfqpoint{3.168300in}{5.057724in}}{\pgfqpoint{3.177508in}{5.066933in}}%
\pgfpathcurveto{\pgfqpoint{3.186717in}{5.076141in}}{\pgfqpoint{3.191891in}{5.088632in}}{\pgfqpoint{3.191891in}{5.101655in}}%
\pgfpathcurveto{\pgfqpoint{3.191891in}{5.114677in}}{\pgfqpoint{3.186717in}{5.127169in}}{\pgfqpoint{3.177508in}{5.136377in}}%
\pgfpathcurveto{\pgfqpoint{3.168300in}{5.145585in}}{\pgfqpoint{3.155809in}{5.150759in}}{\pgfqpoint{3.142786in}{5.150759in}}%
\pgfpathcurveto{\pgfqpoint{3.129763in}{5.150759in}}{\pgfqpoint{3.117272in}{5.145585in}}{\pgfqpoint{3.108064in}{5.136377in}}%
\pgfpathcurveto{\pgfqpoint{3.098855in}{5.127169in}}{\pgfqpoint{3.093682in}{5.114677in}}{\pgfqpoint{3.093682in}{5.101655in}}%
\pgfpathcurveto{\pgfqpoint{3.093682in}{5.088632in}}{\pgfqpoint{3.098855in}{5.076141in}}{\pgfqpoint{3.108064in}{5.066933in}}%
\pgfpathcurveto{\pgfqpoint{3.117272in}{5.057724in}}{\pgfqpoint{3.129763in}{5.052550in}}{\pgfqpoint{3.142786in}{5.052550in}}%
\pgfpathlineto{\pgfqpoint{3.142786in}{5.052550in}}%
\pgfpathclose%
\pgfusepath{stroke,fill}%
\end{pgfscope}%
\begin{pgfscope}%
\pgfpathrectangle{\pgfqpoint{0.194833in}{0.246946in}}{\pgfqpoint{6.160000in}{6.160000in}}%
\pgfusepath{clip}%
\pgfsetbuttcap%
\pgfsetroundjoin%
\definecolor{currentfill}{rgb}{0.121569,0.466667,0.705882}%
\pgfsetfillcolor{currentfill}%
\pgfsetlinewidth{1.003750pt}%
\definecolor{currentstroke}{rgb}{0.121569,0.466667,0.705882}%
\pgfsetstrokecolor{currentstroke}%
\pgfsetdash{}{0pt}%
\pgfpathmoveto{\pgfqpoint{2.754281in}{5.063293in}}%
\pgfpathcurveto{\pgfqpoint{2.767304in}{5.063293in}}{\pgfqpoint{2.779795in}{5.068467in}}{\pgfqpoint{2.789004in}{5.077675in}}%
\pgfpathcurveto{\pgfqpoint{2.798212in}{5.086884in}}{\pgfqpoint{2.803386in}{5.099375in}}{\pgfqpoint{2.803386in}{5.112398in}}%
\pgfpathcurveto{\pgfqpoint{2.803386in}{5.125420in}}{\pgfqpoint{2.798212in}{5.137911in}}{\pgfqpoint{2.789004in}{5.147120in}}%
\pgfpathcurveto{\pgfqpoint{2.779795in}{5.156328in}}{\pgfqpoint{2.767304in}{5.161502in}}{\pgfqpoint{2.754281in}{5.161502in}}%
\pgfpathcurveto{\pgfqpoint{2.741259in}{5.161502in}}{\pgfqpoint{2.728768in}{5.156328in}}{\pgfqpoint{2.719559in}{5.147120in}}%
\pgfpathcurveto{\pgfqpoint{2.710351in}{5.137911in}}{\pgfqpoint{2.705177in}{5.125420in}}{\pgfqpoint{2.705177in}{5.112398in}}%
\pgfpathcurveto{\pgfqpoint{2.705177in}{5.099375in}}{\pgfqpoint{2.710351in}{5.086884in}}{\pgfqpoint{2.719559in}{5.077675in}}%
\pgfpathcurveto{\pgfqpoint{2.728768in}{5.068467in}}{\pgfqpoint{2.741259in}{5.063293in}}{\pgfqpoint{2.754281in}{5.063293in}}%
\pgfpathlineto{\pgfqpoint{2.754281in}{5.063293in}}%
\pgfpathclose%
\pgfusepath{stroke,fill}%
\end{pgfscope}%
\begin{pgfscope}%
\pgfpathrectangle{\pgfqpoint{0.194833in}{0.246946in}}{\pgfqpoint{6.160000in}{6.160000in}}%
\pgfusepath{clip}%
\pgfsetbuttcap%
\pgfsetroundjoin%
\definecolor{currentfill}{rgb}{0.121569,0.466667,0.705882}%
\pgfsetfillcolor{currentfill}%
\pgfsetlinewidth{1.003750pt}%
\definecolor{currentstroke}{rgb}{0.121569,0.466667,0.705882}%
\pgfsetstrokecolor{currentstroke}%
\pgfsetdash{}{0pt}%
\pgfpathmoveto{\pgfqpoint{3.511269in}{5.112564in}}%
\pgfpathcurveto{\pgfqpoint{3.524291in}{5.112564in}}{\pgfqpoint{3.536783in}{5.117738in}}{\pgfqpoint{3.545991in}{5.126947in}}%
\pgfpathcurveto{\pgfqpoint{3.555199in}{5.136155in}}{\pgfqpoint{3.560373in}{5.148646in}}{\pgfqpoint{3.560373in}{5.161669in}}%
\pgfpathcurveto{\pgfqpoint{3.560373in}{5.174691in}}{\pgfqpoint{3.555199in}{5.187183in}}{\pgfqpoint{3.545991in}{5.196391in}}%
\pgfpathcurveto{\pgfqpoint{3.536783in}{5.205599in}}{\pgfqpoint{3.524291in}{5.210773in}}{\pgfqpoint{3.511269in}{5.210773in}}%
\pgfpathcurveto{\pgfqpoint{3.498246in}{5.210773in}}{\pgfqpoint{3.485755in}{5.205599in}}{\pgfqpoint{3.476547in}{5.196391in}}%
\pgfpathcurveto{\pgfqpoint{3.467338in}{5.187183in}}{\pgfqpoint{3.462164in}{5.174691in}}{\pgfqpoint{3.462164in}{5.161669in}}%
\pgfpathcurveto{\pgfqpoint{3.462164in}{5.148646in}}{\pgfqpoint{3.467338in}{5.136155in}}{\pgfqpoint{3.476547in}{5.126947in}}%
\pgfpathcurveto{\pgfqpoint{3.485755in}{5.117738in}}{\pgfqpoint{3.498246in}{5.112564in}}{\pgfqpoint{3.511269in}{5.112564in}}%
\pgfpathlineto{\pgfqpoint{3.511269in}{5.112564in}}%
\pgfpathclose%
\pgfusepath{stroke,fill}%
\end{pgfscope}%
\begin{pgfscope}%
\pgfpathrectangle{\pgfqpoint{0.194833in}{0.246946in}}{\pgfqpoint{6.160000in}{6.160000in}}%
\pgfusepath{clip}%
\pgfsetbuttcap%
\pgfsetroundjoin%
\definecolor{currentfill}{rgb}{0.121569,0.466667,0.705882}%
\pgfsetfillcolor{currentfill}%
\pgfsetlinewidth{1.003750pt}%
\definecolor{currentstroke}{rgb}{0.121569,0.466667,0.705882}%
\pgfsetstrokecolor{currentstroke}%
\pgfsetdash{}{0pt}%
\pgfpathmoveto{\pgfqpoint{3.896537in}{5.158708in}}%
\pgfpathcurveto{\pgfqpoint{3.909560in}{5.158708in}}{\pgfqpoint{3.922051in}{5.163882in}}{\pgfqpoint{3.931259in}{5.173091in}}%
\pgfpathcurveto{\pgfqpoint{3.940468in}{5.182299in}}{\pgfqpoint{3.945642in}{5.194790in}}{\pgfqpoint{3.945642in}{5.207813in}}%
\pgfpathcurveto{\pgfqpoint{3.945642in}{5.220836in}}{\pgfqpoint{3.940468in}{5.233327in}}{\pgfqpoint{3.931259in}{5.242535in}}%
\pgfpathcurveto{\pgfqpoint{3.922051in}{5.251744in}}{\pgfqpoint{3.909560in}{5.256918in}}{\pgfqpoint{3.896537in}{5.256918in}}%
\pgfpathcurveto{\pgfqpoint{3.883514in}{5.256918in}}{\pgfqpoint{3.871023in}{5.251744in}}{\pgfqpoint{3.861815in}{5.242535in}}%
\pgfpathcurveto{\pgfqpoint{3.852606in}{5.233327in}}{\pgfqpoint{3.847432in}{5.220836in}}{\pgfqpoint{3.847432in}{5.207813in}}%
\pgfpathcurveto{\pgfqpoint{3.847432in}{5.194790in}}{\pgfqpoint{3.852606in}{5.182299in}}{\pgfqpoint{3.861815in}{5.173091in}}%
\pgfpathcurveto{\pgfqpoint{3.871023in}{5.163882in}}{\pgfqpoint{3.883514in}{5.158708in}}{\pgfqpoint{3.896537in}{5.158708in}}%
\pgfpathlineto{\pgfqpoint{3.896537in}{5.158708in}}%
\pgfpathclose%
\pgfusepath{stroke,fill}%
\end{pgfscope}%
\begin{pgfscope}%
\pgfpathrectangle{\pgfqpoint{0.194833in}{0.246946in}}{\pgfqpoint{6.160000in}{6.160000in}}%
\pgfusepath{clip}%
\pgfsetbuttcap%
\pgfsetroundjoin%
\definecolor{currentfill}{rgb}{0.121569,0.466667,0.705882}%
\pgfsetfillcolor{currentfill}%
\pgfsetlinewidth{1.003750pt}%
\definecolor{currentstroke}{rgb}{0.121569,0.466667,0.705882}%
\pgfsetstrokecolor{currentstroke}%
\pgfsetdash{}{0pt}%
\pgfpathmoveto{\pgfqpoint{2.942462in}{5.338118in}}%
\pgfpathcurveto{\pgfqpoint{2.955485in}{5.338118in}}{\pgfqpoint{2.967976in}{5.343292in}}{\pgfqpoint{2.977184in}{5.352501in}}%
\pgfpathcurveto{\pgfqpoint{2.986393in}{5.361709in}}{\pgfqpoint{2.991567in}{5.374200in}}{\pgfqpoint{2.991567in}{5.387223in}}%
\pgfpathcurveto{\pgfqpoint{2.991567in}{5.400246in}}{\pgfqpoint{2.986393in}{5.412737in}}{\pgfqpoint{2.977184in}{5.421945in}}%
\pgfpathcurveto{\pgfqpoint{2.967976in}{5.431154in}}{\pgfqpoint{2.955485in}{5.436328in}}{\pgfqpoint{2.942462in}{5.436328in}}%
\pgfpathcurveto{\pgfqpoint{2.929439in}{5.436328in}}{\pgfqpoint{2.916948in}{5.431154in}}{\pgfqpoint{2.907740in}{5.421945in}}%
\pgfpathcurveto{\pgfqpoint{2.898531in}{5.412737in}}{\pgfqpoint{2.893357in}{5.400246in}}{\pgfqpoint{2.893357in}{5.387223in}}%
\pgfpathcurveto{\pgfqpoint{2.893357in}{5.374200in}}{\pgfqpoint{2.898531in}{5.361709in}}{\pgfqpoint{2.907740in}{5.352501in}}%
\pgfpathcurveto{\pgfqpoint{2.916948in}{5.343292in}}{\pgfqpoint{2.929439in}{5.338118in}}{\pgfqpoint{2.942462in}{5.338118in}}%
\pgfpathlineto{\pgfqpoint{2.942462in}{5.338118in}}%
\pgfpathclose%
\pgfusepath{stroke,fill}%
\end{pgfscope}%
\begin{pgfscope}%
\pgfpathrectangle{\pgfqpoint{0.194833in}{0.246946in}}{\pgfqpoint{6.160000in}{6.160000in}}%
\pgfusepath{clip}%
\pgfsetbuttcap%
\pgfsetroundjoin%
\definecolor{currentfill}{rgb}{0.121569,0.466667,0.705882}%
\pgfsetfillcolor{currentfill}%
\pgfsetlinewidth{1.003750pt}%
\definecolor{currentstroke}{rgb}{0.121569,0.466667,0.705882}%
\pgfsetstrokecolor{currentstroke}%
\pgfsetdash{}{0pt}%
\pgfpathmoveto{\pgfqpoint{3.329963in}{5.373944in}}%
\pgfpathcurveto{\pgfqpoint{3.342986in}{5.373944in}}{\pgfqpoint{3.355477in}{5.379118in}}{\pgfqpoint{3.364685in}{5.388326in}}%
\pgfpathcurveto{\pgfqpoint{3.373894in}{5.397535in}}{\pgfqpoint{3.379068in}{5.410026in}}{\pgfqpoint{3.379068in}{5.423048in}}%
\pgfpathcurveto{\pgfqpoint{3.379068in}{5.436071in}}{\pgfqpoint{3.373894in}{5.448562in}}{\pgfqpoint{3.364685in}{5.457771in}}%
\pgfpathcurveto{\pgfqpoint{3.355477in}{5.466979in}}{\pgfqpoint{3.342986in}{5.472153in}}{\pgfqpoint{3.329963in}{5.472153in}}%
\pgfpathcurveto{\pgfqpoint{3.316940in}{5.472153in}}{\pgfqpoint{3.304449in}{5.466979in}}{\pgfqpoint{3.295241in}{5.457771in}}%
\pgfpathcurveto{\pgfqpoint{3.286032in}{5.448562in}}{\pgfqpoint{3.280858in}{5.436071in}}{\pgfqpoint{3.280858in}{5.423048in}}%
\pgfpathcurveto{\pgfqpoint{3.280858in}{5.410026in}}{\pgfqpoint{3.286032in}{5.397535in}}{\pgfqpoint{3.295241in}{5.388326in}}%
\pgfpathcurveto{\pgfqpoint{3.304449in}{5.379118in}}{\pgfqpoint{3.316940in}{5.373944in}}{\pgfqpoint{3.329963in}{5.373944in}}%
\pgfpathlineto{\pgfqpoint{3.329963in}{5.373944in}}%
\pgfpathclose%
\pgfusepath{stroke,fill}%
\end{pgfscope}%
\begin{pgfscope}%
\pgfpathrectangle{\pgfqpoint{0.194833in}{0.246946in}}{\pgfqpoint{6.160000in}{6.160000in}}%
\pgfusepath{clip}%
\pgfsetbuttcap%
\pgfsetroundjoin%
\definecolor{currentfill}{rgb}{0.121569,0.466667,0.705882}%
\pgfsetfillcolor{currentfill}%
\pgfsetlinewidth{1.003750pt}%
\definecolor{currentstroke}{rgb}{0.121569,0.466667,0.705882}%
\pgfsetstrokecolor{currentstroke}%
\pgfsetdash{}{0pt}%
\pgfpathmoveto{\pgfqpoint{3.722911in}{5.412278in}}%
\pgfpathcurveto{\pgfqpoint{3.735934in}{5.412278in}}{\pgfqpoint{3.748425in}{5.417452in}}{\pgfqpoint{3.757633in}{5.426660in}}%
\pgfpathcurveto{\pgfqpoint{3.766842in}{5.435869in}}{\pgfqpoint{3.772016in}{5.448360in}}{\pgfqpoint{3.772016in}{5.461383in}}%
\pgfpathcurveto{\pgfqpoint{3.772016in}{5.474405in}}{\pgfqpoint{3.766842in}{5.486896in}}{\pgfqpoint{3.757633in}{5.496105in}}%
\pgfpathcurveto{\pgfqpoint{3.748425in}{5.505313in}}{\pgfqpoint{3.735934in}{5.510487in}}{\pgfqpoint{3.722911in}{5.510487in}}%
\pgfpathcurveto{\pgfqpoint{3.709888in}{5.510487in}}{\pgfqpoint{3.697397in}{5.505313in}}{\pgfqpoint{3.688189in}{5.496105in}}%
\pgfpathcurveto{\pgfqpoint{3.678980in}{5.486896in}}{\pgfqpoint{3.673806in}{5.474405in}}{\pgfqpoint{3.673806in}{5.461383in}}%
\pgfpathcurveto{\pgfqpoint{3.673806in}{5.448360in}}{\pgfqpoint{3.678980in}{5.435869in}}{\pgfqpoint{3.688189in}{5.426660in}}%
\pgfpathcurveto{\pgfqpoint{3.697397in}{5.417452in}}{\pgfqpoint{3.709888in}{5.412278in}}{\pgfqpoint{3.722911in}{5.412278in}}%
\pgfpathlineto{\pgfqpoint{3.722911in}{5.412278in}}%
\pgfpathclose%
\pgfusepath{stroke,fill}%
\end{pgfscope}%
\begin{pgfscope}%
\pgfpathrectangle{\pgfqpoint{0.194833in}{0.246946in}}{\pgfqpoint{6.160000in}{6.160000in}}%
\pgfusepath{clip}%
\pgfsetbuttcap%
\pgfsetroundjoin%
\definecolor{currentfill}{rgb}{0.121569,0.466667,0.705882}%
\pgfsetfillcolor{currentfill}%
\pgfsetlinewidth{1.003750pt}%
\definecolor{currentstroke}{rgb}{0.121569,0.466667,0.705882}%
\pgfsetstrokecolor{currentstroke}%
\pgfsetdash{}{0pt}%
\pgfpathmoveto{\pgfqpoint{3.146167in}{5.635617in}}%
\pgfpathcurveto{\pgfqpoint{3.159190in}{5.635617in}}{\pgfqpoint{3.171681in}{5.640791in}}{\pgfqpoint{3.180889in}{5.649999in}}%
\pgfpathcurveto{\pgfqpoint{3.190098in}{5.659208in}}{\pgfqpoint{3.195272in}{5.671699in}}{\pgfqpoint{3.195272in}{5.684721in}}%
\pgfpathcurveto{\pgfqpoint{3.195272in}{5.697744in}}{\pgfqpoint{3.190098in}{5.710235in}}{\pgfqpoint{3.180889in}{5.719444in}}%
\pgfpathcurveto{\pgfqpoint{3.171681in}{5.728652in}}{\pgfqpoint{3.159190in}{5.733826in}}{\pgfqpoint{3.146167in}{5.733826in}}%
\pgfpathcurveto{\pgfqpoint{3.133144in}{5.733826in}}{\pgfqpoint{3.120653in}{5.728652in}}{\pgfqpoint{3.111445in}{5.719444in}}%
\pgfpathcurveto{\pgfqpoint{3.102236in}{5.710235in}}{\pgfqpoint{3.097062in}{5.697744in}}{\pgfqpoint{3.097062in}{5.684721in}}%
\pgfpathcurveto{\pgfqpoint{3.097062in}{5.671699in}}{\pgfqpoint{3.102236in}{5.659208in}}{\pgfqpoint{3.111445in}{5.649999in}}%
\pgfpathcurveto{\pgfqpoint{3.120653in}{5.640791in}}{\pgfqpoint{3.133144in}{5.635617in}}{\pgfqpoint{3.146167in}{5.635617in}}%
\pgfpathlineto{\pgfqpoint{3.146167in}{5.635617in}}%
\pgfpathclose%
\pgfusepath{stroke,fill}%
\end{pgfscope}%
\begin{pgfscope}%
\pgfpathrectangle{\pgfqpoint{0.194833in}{0.246946in}}{\pgfqpoint{6.160000in}{6.160000in}}%
\pgfusepath{clip}%
\pgfsetbuttcap%
\pgfsetroundjoin%
\definecolor{currentfill}{rgb}{0.121569,0.466667,0.705882}%
\pgfsetfillcolor{currentfill}%
\pgfsetlinewidth{1.003750pt}%
\definecolor{currentstroke}{rgb}{0.121569,0.466667,0.705882}%
\pgfsetstrokecolor{currentstroke}%
\pgfsetdash{}{0pt}%
\pgfpathmoveto{\pgfqpoint{3.543693in}{5.674014in}}%
\pgfpathcurveto{\pgfqpoint{3.556716in}{5.674014in}}{\pgfqpoint{3.569207in}{5.679188in}}{\pgfqpoint{3.578415in}{5.688397in}}%
\pgfpathcurveto{\pgfqpoint{3.587624in}{5.697605in}}{\pgfqpoint{3.592798in}{5.710096in}}{\pgfqpoint{3.592798in}{5.723119in}}%
\pgfpathcurveto{\pgfqpoint{3.592798in}{5.736142in}}{\pgfqpoint{3.587624in}{5.748633in}}{\pgfqpoint{3.578415in}{5.757841in}}%
\pgfpathcurveto{\pgfqpoint{3.569207in}{5.767050in}}{\pgfqpoint{3.556716in}{5.772224in}}{\pgfqpoint{3.543693in}{5.772224in}}%
\pgfpathcurveto{\pgfqpoint{3.530670in}{5.772224in}}{\pgfqpoint{3.518179in}{5.767050in}}{\pgfqpoint{3.508971in}{5.757841in}}%
\pgfpathcurveto{\pgfqpoint{3.499762in}{5.748633in}}{\pgfqpoint{3.494588in}{5.736142in}}{\pgfqpoint{3.494588in}{5.723119in}}%
\pgfpathcurveto{\pgfqpoint{3.494588in}{5.710096in}}{\pgfqpoint{3.499762in}{5.697605in}}{\pgfqpoint{3.508971in}{5.688397in}}%
\pgfpathcurveto{\pgfqpoint{3.518179in}{5.679188in}}{\pgfqpoint{3.530670in}{5.674014in}}{\pgfqpoint{3.543693in}{5.674014in}}%
\pgfpathlineto{\pgfqpoint{3.543693in}{5.674014in}}%
\pgfpathclose%
\pgfusepath{stroke,fill}%
\end{pgfscope}%
\begin{pgfscope}%
\pgfpathrectangle{\pgfqpoint{0.194833in}{0.246946in}}{\pgfqpoint{6.160000in}{6.160000in}}%
\pgfusepath{clip}%
\pgfsetbuttcap%
\pgfsetroundjoin%
\definecolor{currentfill}{rgb}{0.121569,0.466667,0.705882}%
\pgfsetfillcolor{currentfill}%
\pgfsetlinewidth{1.003750pt}%
\definecolor{currentstroke}{rgb}{0.121569,0.466667,0.705882}%
\pgfsetstrokecolor{currentstroke}%
\pgfsetdash{}{0pt}%
\pgfpathmoveto{\pgfqpoint{3.358076in}{5.945096in}}%
\pgfpathcurveto{\pgfqpoint{3.371099in}{5.945096in}}{\pgfqpoint{3.383590in}{5.950270in}}{\pgfqpoint{3.392798in}{5.959478in}}%
\pgfpathcurveto{\pgfqpoint{3.402007in}{5.968687in}}{\pgfqpoint{3.407181in}{5.981178in}}{\pgfqpoint{3.407181in}{5.994201in}}%
\pgfpathcurveto{\pgfqpoint{3.407181in}{6.007223in}}{\pgfqpoint{3.402007in}{6.019714in}}{\pgfqpoint{3.392798in}{6.028923in}}%
\pgfpathcurveto{\pgfqpoint{3.383590in}{6.038131in}}{\pgfqpoint{3.371099in}{6.043305in}}{\pgfqpoint{3.358076in}{6.043305in}}%
\pgfpathcurveto{\pgfqpoint{3.345053in}{6.043305in}}{\pgfqpoint{3.332562in}{6.038131in}}{\pgfqpoint{3.323354in}{6.028923in}}%
\pgfpathcurveto{\pgfqpoint{3.314145in}{6.019714in}}{\pgfqpoint{3.308971in}{6.007223in}}{\pgfqpoint{3.308971in}{5.994201in}}%
\pgfpathcurveto{\pgfqpoint{3.308971in}{5.981178in}}{\pgfqpoint{3.314145in}{5.968687in}}{\pgfqpoint{3.323354in}{5.959478in}}%
\pgfpathcurveto{\pgfqpoint{3.332562in}{5.950270in}}{\pgfqpoint{3.345053in}{5.945096in}}{\pgfqpoint{3.358076in}{5.945096in}}%
\pgfpathlineto{\pgfqpoint{3.358076in}{5.945096in}}%
\pgfpathclose%
\pgfusepath{stroke,fill}%
\end{pgfscope}%
\end{pgfpicture}%
\makeatother%
\endgroup%
}
  \caption{A set of reference directions for a three-objective problem.}
  \label{fig:ref-dirs}
\end{figure}

\ac{nsga2} is useful for mono- and multi-objective functions while \ac{nsga3} is
better for many-objective problems. \ac{unsga3} can handle any number of
objectives by introducing the binary tournament from \ac{nsga2} and reducing to
the most efficient algorithm for the problem at hand \cite{seada_unified_2016}.
Chapter \ref{chapter:benchmark-results} demonstrates these three algorithms in 
Section \ref{section:4-exercise-1}.

\subsection{Hyperparameter Tuning}
Similar to other machine learning models, \acp{ga} have several hyperparameters
that must be tuned for optimal behavior. These hyperparameters include
probabilities for mutation, crossover, and selection, as well as the number of
parents, number of offspring, and population size. Determining ideal
hyperparameters is often performed using either a grid search or random sampling
\cite{bergstra_random_2012}. This thesis adopts the approach from Blank and Deb
\cite{blank_pymoo_2020} using a genetic algorithm to identify the ideal
hyperparameters. A multi-objective problem is converted into a single objective problem 
by choosing one objective and optimized with the target \ac{ga}, then a second \ac{ga} 
drives a new problem where the decision variables are hyperparameters of the desired algorithm.

\subsection{Convergence}
There are several ways to stop a simulation in \ac{pymoo}. A simulation may end
after reaching
\begin{enumerate}
    \item a specified end time (e.g., 100 minutes),
    \label{it:convergence1}
    \item a specified number of evaluations or iterations (e.g., 500 individual
    evaluations or 20 generations),
    \label{it:convergence2}
    \item a tolerance value in the design space,
    \label{it:convergence3}    
    \item a tolerance value in the objective space.
    \label{it:convergence4}
\end{enumerate}

It is possible that criteria \ref{it:convergence3} and \ref{it:convergence4}
will never be met; therefore, they are often combined with either of the first
two criteria. The fourth convergence criterion is the most interesting due to
the challenge of calculating an appropriate metric. This thesis uses the weakly
Pareto-compliant algorithm \ac{igdp} over the more common hypervolume
calculation due to its reduced computational requirements
\cite{ishibuchi_modified_2015}.

\subsection{\acs{pymoo} and \acs{deap}}

The \ac{esom} framework developed in this thesis is built on top of \ac{pymoo}
and \ac{deap}. \ac{pymoo} is an open-source library for \acp{ga} developed by
the creators of \ac{nsga2} and \ac{unsga3} \cite{blank_pymoo_2020}. This package
implements several \acp{ga} out-of-the-box and offers a set of visualization
tools and hyperparameter tuning. \ac{deap} is another open-source library
offering a toolkit for constructing \acp{ga} and therefore has fewer prepackaged
algorithms than \ac{pymoo}. There are robust reasons to use both libraries, so
\ac{osier} facilitates both. \ac{pymoo} provides a broader range of tools 
while \ac{deap} enables greater flexibility with \ac{ga} implementation and more
straightforward procedures for stopping a simulation and restarting at a later
time.




