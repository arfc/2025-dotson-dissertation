\section{\acs{mga} with \acl{moo}}
\label{section:mga-moo}

\ac{osier} addresses structural uncertainty in two ways. First, it uses 
\ac{moo} to identify tradeoffs. Second, since structural uncertainty will
persist regardless of the number of modeled objectives, \ac{osier} offers
a high-dimensional version of the classic \ac{mga} approach to handling
structural uncertainty.
This thesis applies some ideas from \ac{mga} to the analysis of the sub-optimal
space from a \acl{moo} problem. Due to their iterative process, \acp{ga}
naturally generate many samples in a problem's feasible space. However, this
does not lead to a ``limited set'' of solutions but rather a potentially
unbounded set. Some literature developed \acp{ga} that directly use \ac{mga} in
the iterative process
\cite{zechman_evolutionary_2004,zechman_evolutionary_2013}. However, existing
Python libraries such as \ac{pymoo} and \ac{deap} do not implement these
methods, and the challenge is not an inability to sample the sub-optimal space,
but rather to provide a comprehensible subset of solutions. The algorithm I
developed in this thesis to search the near-feasible space is the following:

\begin{enumerate}
    \item Obtain a set of Pareto-optimal solutions \textit{using any \ac{ga}}.
    \item Decide on a slack value (e.g., 10\% or 0.1), which represents an
    acceptable deviation from the Pareto front.
    \item Create a ``near-feasible front'' where the coordinates of each point
    are multiplied by unity plus the slack value. This is equivalent to relaxing
    the objective functions and converting them to a constraint. 
    \item Every individual is checked if all of its coordinates are
    \begin{itemize}
        \item below all of the coordinates for at least one point on the
        near-feasible front and
        \item above all of the coordinates for at least one point on the Pareto
        front.
    \end{itemize}  
    \item Lastly, the set of interior points may be sampled either randomly or
    with a farthest-first-traversal algorithm to restrict the number of analyzed
    solutions.
\end{enumerate}
\noindent
Figure \ref{fig:nd-mga} and Figure \ref{fig:3d-mga} show ``near-feasible
fronts'' and interior points with 20 percent slack for a 2-D and 3-D Pareto
front, respectively. Figure \ref{fig:nd-mga} shows clearly that only points
within the near-optimal space (gray) are considered. Illustrating this behavior
in three dimensions (and above) is considerably more difficult. The 3-D interior
points should be covered by both surfaces, obstructing their view. Figure
\ref{fig:3d-mga} shows that this is the case in three panels. First, a top view
of an opaque Pareto front (green) where no interior points can be observed.
Second, the same view with a translucent Pareto front, revealing interior points
and the near-optimal front (blue). Finally, the view from underneath the
near-optimal front once again obscures the interior points, except for two near
the edges of the sub-optimal space. The tested points are omitted for clarity.

\begin{figure}[h]
  \centering
  \resizebox{0.6\columnwidth}{!}{%% Creator: Matplotlib, PGF backend
%%
%% To include the figure in your LaTeX document, write
%%   \input{<filename>.pgf}
%%
%% Make sure the required packages are loaded in your preamble
%%   \usepackage{pgf}
%%
%% Also ensure that all the required font packages are loaded; for instance,
%% the lmodern package is sometimes necessary when using math font.
%%   \usepackage{lmodern}
%%
%% Figures using additional raster images can only be included by \input if
%% they are in the same directory as the main LaTeX file. For loading figures
%% from other directories you can use the `import` package
%%   \usepackage{import}
%%
%% and then include the figures with
%%   \import{<path to file>}{<filename>.pgf}
%%
%% Matplotlib used the following preamble
%%   \def\mathdefault#1{#1}
%%   \everymath=\expandafter{\the\everymath\displaystyle}
%%   \IfFileExists{scrextend.sty}{
%%     \usepackage[fontsize=10.000000pt]{scrextend}
%%   }{
%%     \renewcommand{\normalsize}{\fontsize{10.000000}{12.000000}\selectfont}
%%     \normalsize
%%   }
%%   
%%   \makeatletter\@ifpackageloaded{underscore}{}{\usepackage[strings]{underscore}}\makeatother
%%
\begingroup%
\makeatletter%
\begin{pgfpicture}%
\pgfpathrectangle{\pgfpointorigin}{\pgfqpoint{6.951690in}{5.397777in}}%
\pgfusepath{use as bounding box, clip}%
\begin{pgfscope}%
\pgfsetbuttcap%
\pgfsetmiterjoin%
\definecolor{currentfill}{rgb}{1.000000,1.000000,1.000000}%
\pgfsetfillcolor{currentfill}%
\pgfsetlinewidth{0.000000pt}%
\definecolor{currentstroke}{rgb}{0.000000,0.000000,0.000000}%
\pgfsetstrokecolor{currentstroke}%
\pgfsetdash{}{0pt}%
\pgfpathmoveto{\pgfqpoint{0.000000in}{0.000000in}}%
\pgfpathlineto{\pgfqpoint{6.951690in}{0.000000in}}%
\pgfpathlineto{\pgfqpoint{6.951690in}{5.397777in}}%
\pgfpathlineto{\pgfqpoint{0.000000in}{5.397777in}}%
\pgfpathlineto{\pgfqpoint{0.000000in}{0.000000in}}%
\pgfpathclose%
\pgfusepath{fill}%
\end{pgfscope}%
\begin{pgfscope}%
\pgfsetbuttcap%
\pgfsetmiterjoin%
\definecolor{currentfill}{rgb}{1.000000,1.000000,1.000000}%
\pgfsetfillcolor{currentfill}%
\pgfsetlinewidth{0.000000pt}%
\definecolor{currentstroke}{rgb}{0.000000,0.000000,0.000000}%
\pgfsetstrokecolor{currentstroke}%
\pgfsetstrokeopacity{0.000000}%
\pgfsetdash{}{0pt}%
\pgfpathmoveto{\pgfqpoint{0.626386in}{0.608332in}}%
\pgfpathlineto{\pgfqpoint{6.826386in}{0.608332in}}%
\pgfpathlineto{\pgfqpoint{6.826386in}{5.228333in}}%
\pgfpathlineto{\pgfqpoint{0.626386in}{5.228333in}}%
\pgfpathlineto{\pgfqpoint{0.626386in}{0.608332in}}%
\pgfpathclose%
\pgfusepath{fill}%
\end{pgfscope}%
\begin{pgfscope}%
\pgfpathrectangle{\pgfqpoint{0.626386in}{0.608332in}}{\pgfqpoint{6.200000in}{4.620000in}}%
\pgfusepath{clip}%
\pgfsetbuttcap%
\pgfsetroundjoin%
\definecolor{currentfill}{rgb}{0.121569,0.466667,0.705882}%
\pgfsetfillcolor{currentfill}%
\pgfsetlinewidth{1.003750pt}%
\definecolor{currentstroke}{rgb}{0.121569,0.466667,0.705882}%
\pgfsetstrokecolor{currentstroke}%
\pgfsetdash{}{0pt}%
\pgfsys@defobject{currentmarker}{\pgfqpoint{-0.012028in}{-0.012028in}}{\pgfqpoint{0.012028in}{0.012028in}}{%
\pgfpathmoveto{\pgfqpoint{0.000000in}{-0.012028in}}%
\pgfpathcurveto{\pgfqpoint{0.003190in}{-0.012028in}}{\pgfqpoint{0.006250in}{-0.010761in}}{\pgfqpoint{0.008505in}{-0.008505in}}%
\pgfpathcurveto{\pgfqpoint{0.010761in}{-0.006250in}}{\pgfqpoint{0.012028in}{-0.003190in}}{\pgfqpoint{0.012028in}{0.000000in}}%
\pgfpathcurveto{\pgfqpoint{0.012028in}{0.003190in}}{\pgfqpoint{0.010761in}{0.006250in}}{\pgfqpoint{0.008505in}{0.008505in}}%
\pgfpathcurveto{\pgfqpoint{0.006250in}{0.010761in}}{\pgfqpoint{0.003190in}{0.012028in}}{\pgfqpoint{0.000000in}{0.012028in}}%
\pgfpathcurveto{\pgfqpoint{-0.003190in}{0.012028in}}{\pgfqpoint{-0.006250in}{0.010761in}}{\pgfqpoint{-0.008505in}{0.008505in}}%
\pgfpathcurveto{\pgfqpoint{-0.010761in}{0.006250in}}{\pgfqpoint{-0.012028in}{0.003190in}}{\pgfqpoint{-0.012028in}{0.000000in}}%
\pgfpathcurveto{\pgfqpoint{-0.012028in}{-0.003190in}}{\pgfqpoint{-0.010761in}{-0.006250in}}{\pgfqpoint{-0.008505in}{-0.008505in}}%
\pgfpathcurveto{\pgfqpoint{-0.006250in}{-0.010761in}}{\pgfqpoint{-0.003190in}{-0.012028in}}{\pgfqpoint{0.000000in}{-0.012028in}}%
\pgfpathlineto{\pgfqpoint{0.000000in}{-0.012028in}}%
\pgfpathclose%
\pgfusepath{stroke,fill}%
}%
\begin{pgfscope}%
\pgfsys@transformshift{6.748714in}{2.274941in}%
\pgfsys@useobject{currentmarker}{}%
\end{pgfscope}%
\begin{pgfscope}%
\pgfsys@transformshift{6.413646in}{1.630257in}%
\pgfsys@useobject{currentmarker}{}%
\end{pgfscope}%
\begin{pgfscope}%
\pgfsys@transformshift{2.626608in}{0.849035in}%
\pgfsys@useobject{currentmarker}{}%
\end{pgfscope}%
\begin{pgfscope}%
\pgfsys@transformshift{2.141870in}{1.939487in}%
\pgfsys@useobject{currentmarker}{}%
\end{pgfscope}%
\begin{pgfscope}%
\pgfsys@transformshift{6.669603in}{1.640906in}%
\pgfsys@useobject{currentmarker}{}%
\end{pgfscope}%
\begin{pgfscope}%
\pgfsys@transformshift{3.390781in}{3.524423in}%
\pgfsys@useobject{currentmarker}{}%
\end{pgfscope}%
\begin{pgfscope}%
\pgfsys@transformshift{6.039894in}{4.905623in}%
\pgfsys@useobject{currentmarker}{}%
\end{pgfscope}%
\begin{pgfscope}%
\pgfsys@transformshift{4.856800in}{3.796453in}%
\pgfsys@useobject{currentmarker}{}%
\end{pgfscope}%
\begin{pgfscope}%
\pgfsys@transformshift{5.238397in}{1.418422in}%
\pgfsys@useobject{currentmarker}{}%
\end{pgfscope}%
\begin{pgfscope}%
\pgfsys@transformshift{1.704963in}{4.941840in}%
\pgfsys@useobject{currentmarker}{}%
\end{pgfscope}%
\begin{pgfscope}%
\pgfsys@transformshift{1.003358in}{3.926010in}%
\pgfsys@useobject{currentmarker}{}%
\end{pgfscope}%
\begin{pgfscope}%
\pgfsys@transformshift{4.833963in}{3.530664in}%
\pgfsys@useobject{currentmarker}{}%
\end{pgfscope}%
\begin{pgfscope}%
\pgfsys@transformshift{1.003350in}{5.525871in}%
\pgfsys@useobject{currentmarker}{}%
\end{pgfscope}%
\begin{pgfscope}%
\pgfsys@transformshift{3.377903in}{3.104026in}%
\pgfsys@useobject{currentmarker}{}%
\end{pgfscope}%
\begin{pgfscope}%
\pgfsys@transformshift{0.646020in}{1.573541in}%
\pgfsys@useobject{currentmarker}{}%
\end{pgfscope}%
\begin{pgfscope}%
\pgfsys@transformshift{6.007732in}{2.520309in}%
\pgfsys@useobject{currentmarker}{}%
\end{pgfscope}%
\begin{pgfscope}%
\pgfsys@transformshift{5.238781in}{5.222709in}%
\pgfsys@useobject{currentmarker}{}%
\end{pgfscope}%
\begin{pgfscope}%
\pgfsys@transformshift{1.588421in}{5.604699in}%
\pgfsys@useobject{currentmarker}{}%
\end{pgfscope}%
\begin{pgfscope}%
\pgfsys@transformshift{1.769311in}{5.321012in}%
\pgfsys@useobject{currentmarker}{}%
\end{pgfscope}%
\begin{pgfscope}%
\pgfsys@transformshift{1.171002in}{2.753761in}%
\pgfsys@useobject{currentmarker}{}%
\end{pgfscope}%
\begin{pgfscope}%
\pgfsys@transformshift{5.822807in}{1.735579in}%
\pgfsys@useobject{currentmarker}{}%
\end{pgfscope}%
\begin{pgfscope}%
\pgfsys@transformshift{6.074184in}{5.518512in}%
\pgfsys@useobject{currentmarker}{}%
\end{pgfscope}%
\begin{pgfscope}%
\pgfsys@transformshift{5.902569in}{2.648563in}%
\pgfsys@useobject{currentmarker}{}%
\end{pgfscope}%
\begin{pgfscope}%
\pgfsys@transformshift{2.948874in}{2.832420in}%
\pgfsys@useobject{currentmarker}{}%
\end{pgfscope}%
\begin{pgfscope}%
\pgfsys@transformshift{1.204215in}{1.439799in}%
\pgfsys@useobject{currentmarker}{}%
\end{pgfscope}%
\begin{pgfscope}%
\pgfsys@transformshift{3.989788in}{4.195533in}%
\pgfsys@useobject{currentmarker}{}%
\end{pgfscope}%
\begin{pgfscope}%
\pgfsys@transformshift{3.440330in}{1.868584in}%
\pgfsys@useobject{currentmarker}{}%
\end{pgfscope}%
\begin{pgfscope}%
\pgfsys@transformshift{6.564798in}{4.292381in}%
\pgfsys@useobject{currentmarker}{}%
\end{pgfscope}%
\begin{pgfscope}%
\pgfsys@transformshift{3.585366in}{2.694274in}%
\pgfsys@useobject{currentmarker}{}%
\end{pgfscope}%
\begin{pgfscope}%
\pgfsys@transformshift{5.304717in}{4.860155in}%
\pgfsys@useobject{currentmarker}{}%
\end{pgfscope}%
\begin{pgfscope}%
\pgfsys@transformshift{2.466166in}{4.195984in}%
\pgfsys@useobject{currentmarker}{}%
\end{pgfscope}%
\begin{pgfscope}%
\pgfsys@transformshift{5.696446in}{5.032539in}%
\pgfsys@useobject{currentmarker}{}%
\end{pgfscope}%
\begin{pgfscope}%
\pgfsys@transformshift{0.782050in}{2.994974in}%
\pgfsys@useobject{currentmarker}{}%
\end{pgfscope}%
\begin{pgfscope}%
\pgfsys@transformshift{4.467561in}{1.731753in}%
\pgfsys@useobject{currentmarker}{}%
\end{pgfscope}%
\begin{pgfscope}%
\pgfsys@transformshift{3.460521in}{3.096552in}%
\pgfsys@useobject{currentmarker}{}%
\end{pgfscope}%
\begin{pgfscope}%
\pgfsys@transformshift{6.823525in}{0.239116in}%
\pgfsys@useobject{currentmarker}{}%
\end{pgfscope}%
\begin{pgfscope}%
\pgfsys@transformshift{1.888138in}{1.604054in}%
\pgfsys@useobject{currentmarker}{}%
\end{pgfscope}%
\begin{pgfscope}%
\pgfsys@transformshift{2.762305in}{0.494936in}%
\pgfsys@useobject{currentmarker}{}%
\end{pgfscope}%
\begin{pgfscope}%
\pgfsys@transformshift{2.672978in}{1.285705in}%
\pgfsys@useobject{currentmarker}{}%
\end{pgfscope}%
\begin{pgfscope}%
\pgfsys@transformshift{5.824222in}{3.672674in}%
\pgfsys@useobject{currentmarker}{}%
\end{pgfscope}%
\begin{pgfscope}%
\pgfsys@transformshift{1.161961in}{1.404314in}%
\pgfsys@useobject{currentmarker}{}%
\end{pgfscope}%
\begin{pgfscope}%
\pgfsys@transformshift{4.896258in}{0.937813in}%
\pgfsys@useobject{currentmarker}{}%
\end{pgfscope}%
\begin{pgfscope}%
\pgfsys@transformshift{6.622180in}{0.949313in}%
\pgfsys@useobject{currentmarker}{}%
\end{pgfscope}%
\begin{pgfscope}%
\pgfsys@transformshift{5.626688in}{2.917418in}%
\pgfsys@useobject{currentmarker}{}%
\end{pgfscope}%
\begin{pgfscope}%
\pgfsys@transformshift{2.536411in}{1.480725in}%
\pgfsys@useobject{currentmarker}{}%
\end{pgfscope}%
\begin{pgfscope}%
\pgfsys@transformshift{6.224103in}{2.260474in}%
\pgfsys@useobject{currentmarker}{}%
\end{pgfscope}%
\begin{pgfscope}%
\pgfsys@transformshift{1.104948in}{0.390510in}%
\pgfsys@useobject{currentmarker}{}%
\end{pgfscope}%
\begin{pgfscope}%
\pgfsys@transformshift{3.519955in}{4.960386in}%
\pgfsys@useobject{currentmarker}{}%
\end{pgfscope}%
\begin{pgfscope}%
\pgfsys@transformshift{2.824140in}{4.686724in}%
\pgfsys@useobject{currentmarker}{}%
\end{pgfscope}%
\begin{pgfscope}%
\pgfsys@transformshift{6.563792in}{0.571309in}%
\pgfsys@useobject{currentmarker}{}%
\end{pgfscope}%
\begin{pgfscope}%
\pgfsys@transformshift{2.493362in}{3.574964in}%
\pgfsys@useobject{currentmarker}{}%
\end{pgfscope}%
\begin{pgfscope}%
\pgfsys@transformshift{2.517917in}{1.157796in}%
\pgfsys@useobject{currentmarker}{}%
\end{pgfscope}%
\begin{pgfscope}%
\pgfsys@transformshift{3.732402in}{4.344262in}%
\pgfsys@useobject{currentmarker}{}%
\end{pgfscope}%
\begin{pgfscope}%
\pgfsys@transformshift{4.934404in}{1.556609in}%
\pgfsys@useobject{currentmarker}{}%
\end{pgfscope}%
\begin{pgfscope}%
\pgfsys@transformshift{4.525612in}{1.923829in}%
\pgfsys@useobject{currentmarker}{}%
\end{pgfscope}%
\begin{pgfscope}%
\pgfsys@transformshift{2.364084in}{5.265262in}%
\pgfsys@useobject{currentmarker}{}%
\end{pgfscope}%
\begin{pgfscope}%
\pgfsys@transformshift{2.378996in}{5.368462in}%
\pgfsys@useobject{currentmarker}{}%
\end{pgfscope}%
\begin{pgfscope}%
\pgfsys@transformshift{1.892708in}{3.282919in}%
\pgfsys@useobject{currentmarker}{}%
\end{pgfscope}%
\begin{pgfscope}%
\pgfsys@transformshift{3.010271in}{4.602870in}%
\pgfsys@useobject{currentmarker}{}%
\end{pgfscope}%
\begin{pgfscope}%
\pgfsys@transformshift{1.425941in}{5.382087in}%
\pgfsys@useobject{currentmarker}{}%
\end{pgfscope}%
\begin{pgfscope}%
\pgfsys@transformshift{5.194955in}{1.161596in}%
\pgfsys@useobject{currentmarker}{}%
\end{pgfscope}%
\begin{pgfscope}%
\pgfsys@transformshift{4.968865in}{0.711625in}%
\pgfsys@useobject{currentmarker}{}%
\end{pgfscope}%
\begin{pgfscope}%
\pgfsys@transformshift{4.114124in}{0.764184in}%
\pgfsys@useobject{currentmarker}{}%
\end{pgfscope}%
\begin{pgfscope}%
\pgfsys@transformshift{0.900272in}{2.342915in}%
\pgfsys@useobject{currentmarker}{}%
\end{pgfscope}%
\begin{pgfscope}%
\pgfsys@transformshift{4.958177in}{4.161509in}%
\pgfsys@useobject{currentmarker}{}%
\end{pgfscope}%
\begin{pgfscope}%
\pgfsys@transformshift{4.820132in}{3.424780in}%
\pgfsys@useobject{currentmarker}{}%
\end{pgfscope}%
\begin{pgfscope}%
\pgfsys@transformshift{6.136298in}{3.131568in}%
\pgfsys@useobject{currentmarker}{}%
\end{pgfscope}%
\begin{pgfscope}%
\pgfsys@transformshift{1.239574in}{0.975273in}%
\pgfsys@useobject{currentmarker}{}%
\end{pgfscope}%
\begin{pgfscope}%
\pgfsys@transformshift{6.658326in}{4.548452in}%
\pgfsys@useobject{currentmarker}{}%
\end{pgfscope}%
\begin{pgfscope}%
\pgfsys@transformshift{0.668721in}{5.352014in}%
\pgfsys@useobject{currentmarker}{}%
\end{pgfscope}%
\begin{pgfscope}%
\pgfsys@transformshift{4.734825in}{5.160915in}%
\pgfsys@useobject{currentmarker}{}%
\end{pgfscope}%
\begin{pgfscope}%
\pgfsys@transformshift{1.582845in}{3.408881in}%
\pgfsys@useobject{currentmarker}{}%
\end{pgfscope}%
\begin{pgfscope}%
\pgfsys@transformshift{3.784181in}{1.152152in}%
\pgfsys@useobject{currentmarker}{}%
\end{pgfscope}%
\begin{pgfscope}%
\pgfsys@transformshift{1.020356in}{3.821881in}%
\pgfsys@useobject{currentmarker}{}%
\end{pgfscope}%
\begin{pgfscope}%
\pgfsys@transformshift{0.826380in}{1.502212in}%
\pgfsys@useobject{currentmarker}{}%
\end{pgfscope}%
\begin{pgfscope}%
\pgfsys@transformshift{3.857296in}{3.483551in}%
\pgfsys@useobject{currentmarker}{}%
\end{pgfscope}%
\begin{pgfscope}%
\pgfsys@transformshift{2.400085in}{4.793404in}%
\pgfsys@useobject{currentmarker}{}%
\end{pgfscope}%
\begin{pgfscope}%
\pgfsys@transformshift{6.234743in}{4.140911in}%
\pgfsys@useobject{currentmarker}{}%
\end{pgfscope}%
\begin{pgfscope}%
\pgfsys@transformshift{6.434302in}{0.527551in}%
\pgfsys@useobject{currentmarker}{}%
\end{pgfscope}%
\begin{pgfscope}%
\pgfsys@transformshift{0.940729in}{0.880528in}%
\pgfsys@useobject{currentmarker}{}%
\end{pgfscope}%
\begin{pgfscope}%
\pgfsys@transformshift{1.391633in}{3.096166in}%
\pgfsys@useobject{currentmarker}{}%
\end{pgfscope}%
\begin{pgfscope}%
\pgfsys@transformshift{6.350284in}{1.199255in}%
\pgfsys@useobject{currentmarker}{}%
\end{pgfscope}%
\begin{pgfscope}%
\pgfsys@transformshift{6.841479in}{4.993467in}%
\pgfsys@useobject{currentmarker}{}%
\end{pgfscope}%
\begin{pgfscope}%
\pgfsys@transformshift{2.650795in}{1.247392in}%
\pgfsys@useobject{currentmarker}{}%
\end{pgfscope}%
\begin{pgfscope}%
\pgfsys@transformshift{3.740935in}{1.917027in}%
\pgfsys@useobject{currentmarker}{}%
\end{pgfscope}%
\begin{pgfscope}%
\pgfsys@transformshift{0.816347in}{0.710485in}%
\pgfsys@useobject{currentmarker}{}%
\end{pgfscope}%
\begin{pgfscope}%
\pgfsys@transformshift{4.318591in}{3.655900in}%
\pgfsys@useobject{currentmarker}{}%
\end{pgfscope}%
\begin{pgfscope}%
\pgfsys@transformshift{6.683803in}{0.287542in}%
\pgfsys@useobject{currentmarker}{}%
\end{pgfscope}%
\begin{pgfscope}%
\pgfsys@transformshift{5.222113in}{3.649505in}%
\pgfsys@useobject{currentmarker}{}%
\end{pgfscope}%
\begin{pgfscope}%
\pgfsys@transformshift{3.771139in}{0.864515in}%
\pgfsys@useobject{currentmarker}{}%
\end{pgfscope}%
\begin{pgfscope}%
\pgfsys@transformshift{6.004284in}{4.027861in}%
\pgfsys@useobject{currentmarker}{}%
\end{pgfscope}%
\begin{pgfscope}%
\pgfsys@transformshift{4.190676in}{5.543986in}%
\pgfsys@useobject{currentmarker}{}%
\end{pgfscope}%
\begin{pgfscope}%
\pgfsys@transformshift{1.192751in}{4.162065in}%
\pgfsys@useobject{currentmarker}{}%
\end{pgfscope}%
\begin{pgfscope}%
\pgfsys@transformshift{3.689329in}{1.732570in}%
\pgfsys@useobject{currentmarker}{}%
\end{pgfscope}%
\begin{pgfscope}%
\pgfsys@transformshift{6.175975in}{4.600810in}%
\pgfsys@useobject{currentmarker}{}%
\end{pgfscope}%
\begin{pgfscope}%
\pgfsys@transformshift{3.381021in}{3.697299in}%
\pgfsys@useobject{currentmarker}{}%
\end{pgfscope}%
\begin{pgfscope}%
\pgfsys@transformshift{4.581980in}{0.252491in}%
\pgfsys@useobject{currentmarker}{}%
\end{pgfscope}%
\begin{pgfscope}%
\pgfsys@transformshift{2.015825in}{1.386537in}%
\pgfsys@useobject{currentmarker}{}%
\end{pgfscope}%
\begin{pgfscope}%
\pgfsys@transformshift{2.007116in}{4.238541in}%
\pgfsys@useobject{currentmarker}{}%
\end{pgfscope}%
\begin{pgfscope}%
\pgfsys@transformshift{6.528407in}{1.656070in}%
\pgfsys@useobject{currentmarker}{}%
\end{pgfscope}%
\begin{pgfscope}%
\pgfsys@transformshift{5.310581in}{0.217860in}%
\pgfsys@useobject{currentmarker}{}%
\end{pgfscope}%
\begin{pgfscope}%
\pgfsys@transformshift{5.699741in}{4.595809in}%
\pgfsys@useobject{currentmarker}{}%
\end{pgfscope}%
\begin{pgfscope}%
\pgfsys@transformshift{0.968218in}{0.744828in}%
\pgfsys@useobject{currentmarker}{}%
\end{pgfscope}%
\begin{pgfscope}%
\pgfsys@transformshift{4.094222in}{4.356993in}%
\pgfsys@useobject{currentmarker}{}%
\end{pgfscope}%
\begin{pgfscope}%
\pgfsys@transformshift{5.208142in}{3.919384in}%
\pgfsys@useobject{currentmarker}{}%
\end{pgfscope}%
\begin{pgfscope}%
\pgfsys@transformshift{5.406600in}{3.467528in}%
\pgfsys@useobject{currentmarker}{}%
\end{pgfscope}%
\begin{pgfscope}%
\pgfsys@transformshift{2.102448in}{0.359253in}%
\pgfsys@useobject{currentmarker}{}%
\end{pgfscope}%
\begin{pgfscope}%
\pgfsys@transformshift{6.341119in}{2.459282in}%
\pgfsys@useobject{currentmarker}{}%
\end{pgfscope}%
\begin{pgfscope}%
\pgfsys@transformshift{5.161335in}{1.653424in}%
\pgfsys@useobject{currentmarker}{}%
\end{pgfscope}%
\begin{pgfscope}%
\pgfsys@transformshift{6.771717in}{1.175442in}%
\pgfsys@useobject{currentmarker}{}%
\end{pgfscope}%
\begin{pgfscope}%
\pgfsys@transformshift{0.920489in}{2.894691in}%
\pgfsys@useobject{currentmarker}{}%
\end{pgfscope}%
\begin{pgfscope}%
\pgfsys@transformshift{2.949004in}{3.773060in}%
\pgfsys@useobject{currentmarker}{}%
\end{pgfscope}%
\begin{pgfscope}%
\pgfsys@transformshift{0.626675in}{0.984625in}%
\pgfsys@useobject{currentmarker}{}%
\end{pgfscope}%
\begin{pgfscope}%
\pgfsys@transformshift{6.010775in}{3.921228in}%
\pgfsys@useobject{currentmarker}{}%
\end{pgfscope}%
\begin{pgfscope}%
\pgfsys@transformshift{2.707698in}{0.341599in}%
\pgfsys@useobject{currentmarker}{}%
\end{pgfscope}%
\begin{pgfscope}%
\pgfsys@transformshift{0.890930in}{2.326007in}%
\pgfsys@useobject{currentmarker}{}%
\end{pgfscope}%
\begin{pgfscope}%
\pgfsys@transformshift{6.537541in}{5.204043in}%
\pgfsys@useobject{currentmarker}{}%
\end{pgfscope}%
\begin{pgfscope}%
\pgfsys@transformshift{2.599215in}{2.231556in}%
\pgfsys@useobject{currentmarker}{}%
\end{pgfscope}%
\begin{pgfscope}%
\pgfsys@transformshift{2.298475in}{0.978419in}%
\pgfsys@useobject{currentmarker}{}%
\end{pgfscope}%
\begin{pgfscope}%
\pgfsys@transformshift{4.334250in}{2.725847in}%
\pgfsys@useobject{currentmarker}{}%
\end{pgfscope}%
\begin{pgfscope}%
\pgfsys@transformshift{5.862556in}{2.901862in}%
\pgfsys@useobject{currentmarker}{}%
\end{pgfscope}%
\begin{pgfscope}%
\pgfsys@transformshift{0.737427in}{2.588836in}%
\pgfsys@useobject{currentmarker}{}%
\end{pgfscope}%
\begin{pgfscope}%
\pgfsys@transformshift{1.466590in}{2.014927in}%
\pgfsys@useobject{currentmarker}{}%
\end{pgfscope}%
\begin{pgfscope}%
\pgfsys@transformshift{3.768743in}{3.162637in}%
\pgfsys@useobject{currentmarker}{}%
\end{pgfscope}%
\begin{pgfscope}%
\pgfsys@transformshift{4.981965in}{5.452028in}%
\pgfsys@useobject{currentmarker}{}%
\end{pgfscope}%
\begin{pgfscope}%
\pgfsys@transformshift{2.976413in}{4.315850in}%
\pgfsys@useobject{currentmarker}{}%
\end{pgfscope}%
\begin{pgfscope}%
\pgfsys@transformshift{3.939099in}{4.910376in}%
\pgfsys@useobject{currentmarker}{}%
\end{pgfscope}%
\begin{pgfscope}%
\pgfsys@transformshift{4.142002in}{2.285504in}%
\pgfsys@useobject{currentmarker}{}%
\end{pgfscope}%
\begin{pgfscope}%
\pgfsys@transformshift{3.760338in}{3.088483in}%
\pgfsys@useobject{currentmarker}{}%
\end{pgfscope}%
\begin{pgfscope}%
\pgfsys@transformshift{6.804641in}{0.539977in}%
\pgfsys@useobject{currentmarker}{}%
\end{pgfscope}%
\begin{pgfscope}%
\pgfsys@transformshift{0.907503in}{1.263762in}%
\pgfsys@useobject{currentmarker}{}%
\end{pgfscope}%
\begin{pgfscope}%
\pgfsys@transformshift{0.650888in}{0.782229in}%
\pgfsys@useobject{currentmarker}{}%
\end{pgfscope}%
\begin{pgfscope}%
\pgfsys@transformshift{3.098873in}{1.320918in}%
\pgfsys@useobject{currentmarker}{}%
\end{pgfscope}%
\begin{pgfscope}%
\pgfsys@transformshift{3.647529in}{4.805754in}%
\pgfsys@useobject{currentmarker}{}%
\end{pgfscope}%
\begin{pgfscope}%
\pgfsys@transformshift{4.628549in}{4.381591in}%
\pgfsys@useobject{currentmarker}{}%
\end{pgfscope}%
\begin{pgfscope}%
\pgfsys@transformshift{5.832577in}{0.897172in}%
\pgfsys@useobject{currentmarker}{}%
\end{pgfscope}%
\begin{pgfscope}%
\pgfsys@transformshift{0.903870in}{2.644883in}%
\pgfsys@useobject{currentmarker}{}%
\end{pgfscope}%
\begin{pgfscope}%
\pgfsys@transformshift{3.550935in}{5.430657in}%
\pgfsys@useobject{currentmarker}{}%
\end{pgfscope}%
\begin{pgfscope}%
\pgfsys@transformshift{2.242411in}{2.622652in}%
\pgfsys@useobject{currentmarker}{}%
\end{pgfscope}%
\begin{pgfscope}%
\pgfsys@transformshift{4.130464in}{4.830717in}%
\pgfsys@useobject{currentmarker}{}%
\end{pgfscope}%
\begin{pgfscope}%
\pgfsys@transformshift{1.411996in}{5.257843in}%
\pgfsys@useobject{currentmarker}{}%
\end{pgfscope}%
\begin{pgfscope}%
\pgfsys@transformshift{4.260224in}{2.722876in}%
\pgfsys@useobject{currentmarker}{}%
\end{pgfscope}%
\begin{pgfscope}%
\pgfsys@transformshift{3.405747in}{0.478012in}%
\pgfsys@useobject{currentmarker}{}%
\end{pgfscope}%
\begin{pgfscope}%
\pgfsys@transformshift{2.350355in}{3.579143in}%
\pgfsys@useobject{currentmarker}{}%
\end{pgfscope}%
\begin{pgfscope}%
\pgfsys@transformshift{6.387106in}{5.215306in}%
\pgfsys@useobject{currentmarker}{}%
\end{pgfscope}%
\begin{pgfscope}%
\pgfsys@transformshift{2.231490in}{3.123346in}%
\pgfsys@useobject{currentmarker}{}%
\end{pgfscope}%
\begin{pgfscope}%
\pgfsys@transformshift{2.544036in}{2.207787in}%
\pgfsys@useobject{currentmarker}{}%
\end{pgfscope}%
\begin{pgfscope}%
\pgfsys@transformshift{2.827197in}{2.561162in}%
\pgfsys@useobject{currentmarker}{}%
\end{pgfscope}%
\begin{pgfscope}%
\pgfsys@transformshift{0.646834in}{2.681178in}%
\pgfsys@useobject{currentmarker}{}%
\end{pgfscope}%
\begin{pgfscope}%
\pgfsys@transformshift{2.594401in}{1.640905in}%
\pgfsys@useobject{currentmarker}{}%
\end{pgfscope}%
\begin{pgfscope}%
\pgfsys@transformshift{4.214490in}{2.877842in}%
\pgfsys@useobject{currentmarker}{}%
\end{pgfscope}%
\begin{pgfscope}%
\pgfsys@transformshift{5.137671in}{2.729692in}%
\pgfsys@useobject{currentmarker}{}%
\end{pgfscope}%
\begin{pgfscope}%
\pgfsys@transformshift{5.961491in}{4.495294in}%
\pgfsys@useobject{currentmarker}{}%
\end{pgfscope}%
\begin{pgfscope}%
\pgfsys@transformshift{4.017877in}{5.354591in}%
\pgfsys@useobject{currentmarker}{}%
\end{pgfscope}%
\begin{pgfscope}%
\pgfsys@transformshift{4.678181in}{2.499123in}%
\pgfsys@useobject{currentmarker}{}%
\end{pgfscope}%
\begin{pgfscope}%
\pgfsys@transformshift{2.045123in}{1.090396in}%
\pgfsys@useobject{currentmarker}{}%
\end{pgfscope}%
\begin{pgfscope}%
\pgfsys@transformshift{2.850108in}{1.542976in}%
\pgfsys@useobject{currentmarker}{}%
\end{pgfscope}%
\begin{pgfscope}%
\pgfsys@transformshift{3.559126in}{4.430788in}%
\pgfsys@useobject{currentmarker}{}%
\end{pgfscope}%
\begin{pgfscope}%
\pgfsys@transformshift{0.868814in}{4.461062in}%
\pgfsys@useobject{currentmarker}{}%
\end{pgfscope}%
\begin{pgfscope}%
\pgfsys@transformshift{1.680316in}{3.410044in}%
\pgfsys@useobject{currentmarker}{}%
\end{pgfscope}%
\begin{pgfscope}%
\pgfsys@transformshift{0.654024in}{4.739106in}%
\pgfsys@useobject{currentmarker}{}%
\end{pgfscope}%
\begin{pgfscope}%
\pgfsys@transformshift{5.124198in}{1.177305in}%
\pgfsys@useobject{currentmarker}{}%
\end{pgfscope}%
\begin{pgfscope}%
\pgfsys@transformshift{6.386361in}{2.124473in}%
\pgfsys@useobject{currentmarker}{}%
\end{pgfscope}%
\begin{pgfscope}%
\pgfsys@transformshift{4.465022in}{1.382993in}%
\pgfsys@useobject{currentmarker}{}%
\end{pgfscope}%
\begin{pgfscope}%
\pgfsys@transformshift{3.751574in}{3.923941in}%
\pgfsys@useobject{currentmarker}{}%
\end{pgfscope}%
\begin{pgfscope}%
\pgfsys@transformshift{1.865799in}{1.980264in}%
\pgfsys@useobject{currentmarker}{}%
\end{pgfscope}%
\begin{pgfscope}%
\pgfsys@transformshift{3.344360in}{3.226466in}%
\pgfsys@useobject{currentmarker}{}%
\end{pgfscope}%
\begin{pgfscope}%
\pgfsys@transformshift{2.451884in}{2.779128in}%
\pgfsys@useobject{currentmarker}{}%
\end{pgfscope}%
\begin{pgfscope}%
\pgfsys@transformshift{1.279225in}{2.588906in}%
\pgfsys@useobject{currentmarker}{}%
\end{pgfscope}%
\begin{pgfscope}%
\pgfsys@transformshift{0.759148in}{1.180563in}%
\pgfsys@useobject{currentmarker}{}%
\end{pgfscope}%
\begin{pgfscope}%
\pgfsys@transformshift{6.349859in}{1.331878in}%
\pgfsys@useobject{currentmarker}{}%
\end{pgfscope}%
\begin{pgfscope}%
\pgfsys@transformshift{2.908829in}{1.191898in}%
\pgfsys@useobject{currentmarker}{}%
\end{pgfscope}%
\begin{pgfscope}%
\pgfsys@transformshift{4.294240in}{3.406530in}%
\pgfsys@useobject{currentmarker}{}%
\end{pgfscope}%
\begin{pgfscope}%
\pgfsys@transformshift{6.223995in}{4.193718in}%
\pgfsys@useobject{currentmarker}{}%
\end{pgfscope}%
\begin{pgfscope}%
\pgfsys@transformshift{6.541431in}{1.266260in}%
\pgfsys@useobject{currentmarker}{}%
\end{pgfscope}%
\begin{pgfscope}%
\pgfsys@transformshift{2.137565in}{4.917569in}%
\pgfsys@useobject{currentmarker}{}%
\end{pgfscope}%
\begin{pgfscope}%
\pgfsys@transformshift{6.080457in}{0.934593in}%
\pgfsys@useobject{currentmarker}{}%
\end{pgfscope}%
\begin{pgfscope}%
\pgfsys@transformshift{2.901099in}{0.280599in}%
\pgfsys@useobject{currentmarker}{}%
\end{pgfscope}%
\begin{pgfscope}%
\pgfsys@transformshift{2.090748in}{2.087960in}%
\pgfsys@useobject{currentmarker}{}%
\end{pgfscope}%
\begin{pgfscope}%
\pgfsys@transformshift{2.560285in}{0.866113in}%
\pgfsys@useobject{currentmarker}{}%
\end{pgfscope}%
\begin{pgfscope}%
\pgfsys@transformshift{6.601782in}{2.153191in}%
\pgfsys@useobject{currentmarker}{}%
\end{pgfscope}%
\begin{pgfscope}%
\pgfsys@transformshift{3.588372in}{3.986237in}%
\pgfsys@useobject{currentmarker}{}%
\end{pgfscope}%
\begin{pgfscope}%
\pgfsys@transformshift{5.736898in}{5.045594in}%
\pgfsys@useobject{currentmarker}{}%
\end{pgfscope}%
\begin{pgfscope}%
\pgfsys@transformshift{4.309654in}{5.016243in}%
\pgfsys@useobject{currentmarker}{}%
\end{pgfscope}%
\begin{pgfscope}%
\pgfsys@transformshift{4.099087in}{4.534596in}%
\pgfsys@useobject{currentmarker}{}%
\end{pgfscope}%
\begin{pgfscope}%
\pgfsys@transformshift{2.145398in}{4.283886in}%
\pgfsys@useobject{currentmarker}{}%
\end{pgfscope}%
\begin{pgfscope}%
\pgfsys@transformshift{2.250270in}{3.345020in}%
\pgfsys@useobject{currentmarker}{}%
\end{pgfscope}%
\begin{pgfscope}%
\pgfsys@transformshift{6.491615in}{3.548322in}%
\pgfsys@useobject{currentmarker}{}%
\end{pgfscope}%
\begin{pgfscope}%
\pgfsys@transformshift{1.068840in}{5.064836in}%
\pgfsys@useobject{currentmarker}{}%
\end{pgfscope}%
\begin{pgfscope}%
\pgfsys@transformshift{3.883485in}{0.664182in}%
\pgfsys@useobject{currentmarker}{}%
\end{pgfscope}%
\begin{pgfscope}%
\pgfsys@transformshift{2.971665in}{4.192784in}%
\pgfsys@useobject{currentmarker}{}%
\end{pgfscope}%
\begin{pgfscope}%
\pgfsys@transformshift{1.209223in}{3.839503in}%
\pgfsys@useobject{currentmarker}{}%
\end{pgfscope}%
\begin{pgfscope}%
\pgfsys@transformshift{6.871656in}{1.235874in}%
\pgfsys@useobject{currentmarker}{}%
\end{pgfscope}%
\begin{pgfscope}%
\pgfsys@transformshift{4.623291in}{5.231267in}%
\pgfsys@useobject{currentmarker}{}%
\end{pgfscope}%
\begin{pgfscope}%
\pgfsys@transformshift{5.719362in}{1.486985in}%
\pgfsys@useobject{currentmarker}{}%
\end{pgfscope}%
\begin{pgfscope}%
\pgfsys@transformshift{4.762970in}{4.961946in}%
\pgfsys@useobject{currentmarker}{}%
\end{pgfscope}%
\begin{pgfscope}%
\pgfsys@transformshift{4.886365in}{1.147861in}%
\pgfsys@useobject{currentmarker}{}%
\end{pgfscope}%
\begin{pgfscope}%
\pgfsys@transformshift{5.786494in}{4.789347in}%
\pgfsys@useobject{currentmarker}{}%
\end{pgfscope}%
\begin{pgfscope}%
\pgfsys@transformshift{6.112708in}{0.869915in}%
\pgfsys@useobject{currentmarker}{}%
\end{pgfscope}%
\begin{pgfscope}%
\pgfsys@transformshift{3.058098in}{3.065259in}%
\pgfsys@useobject{currentmarker}{}%
\end{pgfscope}%
\begin{pgfscope}%
\pgfsys@transformshift{1.419679in}{2.355247in}%
\pgfsys@useobject{currentmarker}{}%
\end{pgfscope}%
\begin{pgfscope}%
\pgfsys@transformshift{6.365833in}{4.315201in}%
\pgfsys@useobject{currentmarker}{}%
\end{pgfscope}%
\begin{pgfscope}%
\pgfsys@transformshift{1.021013in}{4.198028in}%
\pgfsys@useobject{currentmarker}{}%
\end{pgfscope}%
\begin{pgfscope}%
\pgfsys@transformshift{3.973729in}{5.516341in}%
\pgfsys@useobject{currentmarker}{}%
\end{pgfscope}%
\begin{pgfscope}%
\pgfsys@transformshift{4.944733in}{5.586916in}%
\pgfsys@useobject{currentmarker}{}%
\end{pgfscope}%
\begin{pgfscope}%
\pgfsys@transformshift{3.107162in}{4.467529in}%
\pgfsys@useobject{currentmarker}{}%
\end{pgfscope}%
\begin{pgfscope}%
\pgfsys@transformshift{5.668124in}{0.924856in}%
\pgfsys@useobject{currentmarker}{}%
\end{pgfscope}%
\begin{pgfscope}%
\pgfsys@transformshift{1.849565in}{3.214984in}%
\pgfsys@useobject{currentmarker}{}%
\end{pgfscope}%
\begin{pgfscope}%
\pgfsys@transformshift{5.738722in}{1.604631in}%
\pgfsys@useobject{currentmarker}{}%
\end{pgfscope}%
\begin{pgfscope}%
\pgfsys@transformshift{0.750418in}{1.855157in}%
\pgfsys@useobject{currentmarker}{}%
\end{pgfscope}%
\begin{pgfscope}%
\pgfsys@transformshift{2.910595in}{3.519489in}%
\pgfsys@useobject{currentmarker}{}%
\end{pgfscope}%
\begin{pgfscope}%
\pgfsys@transformshift{1.803510in}{1.618952in}%
\pgfsys@useobject{currentmarker}{}%
\end{pgfscope}%
\begin{pgfscope}%
\pgfsys@transformshift{4.522746in}{0.923843in}%
\pgfsys@useobject{currentmarker}{}%
\end{pgfscope}%
\begin{pgfscope}%
\pgfsys@transformshift{3.232716in}{2.542740in}%
\pgfsys@useobject{currentmarker}{}%
\end{pgfscope}%
\begin{pgfscope}%
\pgfsys@transformshift{1.440932in}{1.007042in}%
\pgfsys@useobject{currentmarker}{}%
\end{pgfscope}%
\begin{pgfscope}%
\pgfsys@transformshift{3.097251in}{4.226800in}%
\pgfsys@useobject{currentmarker}{}%
\end{pgfscope}%
\begin{pgfscope}%
\pgfsys@transformshift{4.639182in}{4.513124in}%
\pgfsys@useobject{currentmarker}{}%
\end{pgfscope}%
\begin{pgfscope}%
\pgfsys@transformshift{4.729855in}{0.475843in}%
\pgfsys@useobject{currentmarker}{}%
\end{pgfscope}%
\begin{pgfscope}%
\pgfsys@transformshift{5.652688in}{1.824843in}%
\pgfsys@useobject{currentmarker}{}%
\end{pgfscope}%
\begin{pgfscope}%
\pgfsys@transformshift{3.226216in}{1.426385in}%
\pgfsys@useobject{currentmarker}{}%
\end{pgfscope}%
\begin{pgfscope}%
\pgfsys@transformshift{5.167749in}{4.568793in}%
\pgfsys@useobject{currentmarker}{}%
\end{pgfscope}%
\begin{pgfscope}%
\pgfsys@transformshift{6.010328in}{1.606239in}%
\pgfsys@useobject{currentmarker}{}%
\end{pgfscope}%
\begin{pgfscope}%
\pgfsys@transformshift{4.470339in}{1.975688in}%
\pgfsys@useobject{currentmarker}{}%
\end{pgfscope}%
\begin{pgfscope}%
\pgfsys@transformshift{5.772001in}{0.915793in}%
\pgfsys@useobject{currentmarker}{}%
\end{pgfscope}%
\begin{pgfscope}%
\pgfsys@transformshift{5.528211in}{3.267157in}%
\pgfsys@useobject{currentmarker}{}%
\end{pgfscope}%
\begin{pgfscope}%
\pgfsys@transformshift{1.920141in}{0.399767in}%
\pgfsys@useobject{currentmarker}{}%
\end{pgfscope}%
\begin{pgfscope}%
\pgfsys@transformshift{1.233618in}{4.969010in}%
\pgfsys@useobject{currentmarker}{}%
\end{pgfscope}%
\begin{pgfscope}%
\pgfsys@transformshift{5.169037in}{4.325928in}%
\pgfsys@useobject{currentmarker}{}%
\end{pgfscope}%
\begin{pgfscope}%
\pgfsys@transformshift{4.070491in}{2.299736in}%
\pgfsys@useobject{currentmarker}{}%
\end{pgfscope}%
\begin{pgfscope}%
\pgfsys@transformshift{3.110229in}{0.525900in}%
\pgfsys@useobject{currentmarker}{}%
\end{pgfscope}%
\begin{pgfscope}%
\pgfsys@transformshift{4.423411in}{3.893190in}%
\pgfsys@useobject{currentmarker}{}%
\end{pgfscope}%
\begin{pgfscope}%
\pgfsys@transformshift{1.296334in}{4.701201in}%
\pgfsys@useobject{currentmarker}{}%
\end{pgfscope}%
\begin{pgfscope}%
\pgfsys@transformshift{4.933271in}{5.359361in}%
\pgfsys@useobject{currentmarker}{}%
\end{pgfscope}%
\begin{pgfscope}%
\pgfsys@transformshift{6.616783in}{4.755727in}%
\pgfsys@useobject{currentmarker}{}%
\end{pgfscope}%
\begin{pgfscope}%
\pgfsys@transformshift{3.774014in}{2.483044in}%
\pgfsys@useobject{currentmarker}{}%
\end{pgfscope}%
\begin{pgfscope}%
\pgfsys@transformshift{0.714621in}{2.239258in}%
\pgfsys@useobject{currentmarker}{}%
\end{pgfscope}%
\begin{pgfscope}%
\pgfsys@transformshift{1.455364in}{0.361922in}%
\pgfsys@useobject{currentmarker}{}%
\end{pgfscope}%
\begin{pgfscope}%
\pgfsys@transformshift{4.502330in}{3.441172in}%
\pgfsys@useobject{currentmarker}{}%
\end{pgfscope}%
\begin{pgfscope}%
\pgfsys@transformshift{5.177961in}{3.455854in}%
\pgfsys@useobject{currentmarker}{}%
\end{pgfscope}%
\begin{pgfscope}%
\pgfsys@transformshift{4.047061in}{3.342995in}%
\pgfsys@useobject{currentmarker}{}%
\end{pgfscope}%
\begin{pgfscope}%
\pgfsys@transformshift{3.295777in}{4.783609in}%
\pgfsys@useobject{currentmarker}{}%
\end{pgfscope}%
\begin{pgfscope}%
\pgfsys@transformshift{1.063543in}{0.413376in}%
\pgfsys@useobject{currentmarker}{}%
\end{pgfscope}%
\begin{pgfscope}%
\pgfsys@transformshift{0.783832in}{2.739907in}%
\pgfsys@useobject{currentmarker}{}%
\end{pgfscope}%
\begin{pgfscope}%
\pgfsys@transformshift{5.078126in}{4.404576in}%
\pgfsys@useobject{currentmarker}{}%
\end{pgfscope}%
\begin{pgfscope}%
\pgfsys@transformshift{4.136988in}{4.658145in}%
\pgfsys@useobject{currentmarker}{}%
\end{pgfscope}%
\begin{pgfscope}%
\pgfsys@transformshift{5.736855in}{1.292936in}%
\pgfsys@useobject{currentmarker}{}%
\end{pgfscope}%
\begin{pgfscope}%
\pgfsys@transformshift{1.983580in}{4.886320in}%
\pgfsys@useobject{currentmarker}{}%
\end{pgfscope}%
\begin{pgfscope}%
\pgfsys@transformshift{6.873462in}{1.295461in}%
\pgfsys@useobject{currentmarker}{}%
\end{pgfscope}%
\begin{pgfscope}%
\pgfsys@transformshift{4.409221in}{1.675381in}%
\pgfsys@useobject{currentmarker}{}%
\end{pgfscope}%
\begin{pgfscope}%
\pgfsys@transformshift{4.047554in}{5.293867in}%
\pgfsys@useobject{currentmarker}{}%
\end{pgfscope}%
\begin{pgfscope}%
\pgfsys@transformshift{1.998868in}{0.481306in}%
\pgfsys@useobject{currentmarker}{}%
\end{pgfscope}%
\begin{pgfscope}%
\pgfsys@transformshift{4.261880in}{4.741855in}%
\pgfsys@useobject{currentmarker}{}%
\end{pgfscope}%
\begin{pgfscope}%
\pgfsys@transformshift{3.995381in}{3.332257in}%
\pgfsys@useobject{currentmarker}{}%
\end{pgfscope}%
\begin{pgfscope}%
\pgfsys@transformshift{4.523514in}{2.854564in}%
\pgfsys@useobject{currentmarker}{}%
\end{pgfscope}%
\begin{pgfscope}%
\pgfsys@transformshift{1.056228in}{2.867376in}%
\pgfsys@useobject{currentmarker}{}%
\end{pgfscope}%
\begin{pgfscope}%
\pgfsys@transformshift{4.359336in}{0.514376in}%
\pgfsys@useobject{currentmarker}{}%
\end{pgfscope}%
\begin{pgfscope}%
\pgfsys@transformshift{4.808662in}{4.480181in}%
\pgfsys@useobject{currentmarker}{}%
\end{pgfscope}%
\begin{pgfscope}%
\pgfsys@transformshift{6.488879in}{4.097878in}%
\pgfsys@useobject{currentmarker}{}%
\end{pgfscope}%
\begin{pgfscope}%
\pgfsys@transformshift{6.287794in}{4.801690in}%
\pgfsys@useobject{currentmarker}{}%
\end{pgfscope}%
\begin{pgfscope}%
\pgfsys@transformshift{4.998063in}{2.004002in}%
\pgfsys@useobject{currentmarker}{}%
\end{pgfscope}%
\begin{pgfscope}%
\pgfsys@transformshift{0.985787in}{5.511862in}%
\pgfsys@useobject{currentmarker}{}%
\end{pgfscope}%
\begin{pgfscope}%
\pgfsys@transformshift{4.566527in}{5.375120in}%
\pgfsys@useobject{currentmarker}{}%
\end{pgfscope}%
\begin{pgfscope}%
\pgfsys@transformshift{1.608226in}{1.416583in}%
\pgfsys@useobject{currentmarker}{}%
\end{pgfscope}%
\begin{pgfscope}%
\pgfsys@transformshift{4.195638in}{1.749215in}%
\pgfsys@useobject{currentmarker}{}%
\end{pgfscope}%
\begin{pgfscope}%
\pgfsys@transformshift{0.763440in}{3.773630in}%
\pgfsys@useobject{currentmarker}{}%
\end{pgfscope}%
\begin{pgfscope}%
\pgfsys@transformshift{1.843265in}{4.852228in}%
\pgfsys@useobject{currentmarker}{}%
\end{pgfscope}%
\begin{pgfscope}%
\pgfsys@transformshift{1.275953in}{2.071698in}%
\pgfsys@useobject{currentmarker}{}%
\end{pgfscope}%
\begin{pgfscope}%
\pgfsys@transformshift{2.621490in}{1.658213in}%
\pgfsys@useobject{currentmarker}{}%
\end{pgfscope}%
\begin{pgfscope}%
\pgfsys@transformshift{4.138927in}{3.125188in}%
\pgfsys@useobject{currentmarker}{}%
\end{pgfscope}%
\begin{pgfscope}%
\pgfsys@transformshift{1.196307in}{2.321273in}%
\pgfsys@useobject{currentmarker}{}%
\end{pgfscope}%
\begin{pgfscope}%
\pgfsys@transformshift{3.514381in}{4.916843in}%
\pgfsys@useobject{currentmarker}{}%
\end{pgfscope}%
\begin{pgfscope}%
\pgfsys@transformshift{5.521781in}{4.691254in}%
\pgfsys@useobject{currentmarker}{}%
\end{pgfscope}%
\begin{pgfscope}%
\pgfsys@transformshift{4.495839in}{2.987808in}%
\pgfsys@useobject{currentmarker}{}%
\end{pgfscope}%
\begin{pgfscope}%
\pgfsys@transformshift{6.548739in}{2.904807in}%
\pgfsys@useobject{currentmarker}{}%
\end{pgfscope}%
\begin{pgfscope}%
\pgfsys@transformshift{3.592530in}{3.343022in}%
\pgfsys@useobject{currentmarker}{}%
\end{pgfscope}%
\begin{pgfscope}%
\pgfsys@transformshift{2.527701in}{5.010536in}%
\pgfsys@useobject{currentmarker}{}%
\end{pgfscope}%
\begin{pgfscope}%
\pgfsys@transformshift{3.904015in}{3.203000in}%
\pgfsys@useobject{currentmarker}{}%
\end{pgfscope}%
\begin{pgfscope}%
\pgfsys@transformshift{2.004440in}{3.186267in}%
\pgfsys@useobject{currentmarker}{}%
\end{pgfscope}%
\begin{pgfscope}%
\pgfsys@transformshift{3.898349in}{1.531822in}%
\pgfsys@useobject{currentmarker}{}%
\end{pgfscope}%
\begin{pgfscope}%
\pgfsys@transformshift{3.307557in}{0.739362in}%
\pgfsys@useobject{currentmarker}{}%
\end{pgfscope}%
\begin{pgfscope}%
\pgfsys@transformshift{2.013122in}{5.165292in}%
\pgfsys@useobject{currentmarker}{}%
\end{pgfscope}%
\begin{pgfscope}%
\pgfsys@transformshift{6.131114in}{4.719844in}%
\pgfsys@useobject{currentmarker}{}%
\end{pgfscope}%
\begin{pgfscope}%
\pgfsys@transformshift{5.146627in}{1.968083in}%
\pgfsys@useobject{currentmarker}{}%
\end{pgfscope}%
\begin{pgfscope}%
\pgfsys@transformshift{1.105166in}{4.174348in}%
\pgfsys@useobject{currentmarker}{}%
\end{pgfscope}%
\begin{pgfscope}%
\pgfsys@transformshift{1.727846in}{3.988167in}%
\pgfsys@useobject{currentmarker}{}%
\end{pgfscope}%
\begin{pgfscope}%
\pgfsys@transformshift{5.477223in}{2.268647in}%
\pgfsys@useobject{currentmarker}{}%
\end{pgfscope}%
\begin{pgfscope}%
\pgfsys@transformshift{0.959381in}{3.622332in}%
\pgfsys@useobject{currentmarker}{}%
\end{pgfscope}%
\begin{pgfscope}%
\pgfsys@transformshift{3.248512in}{0.542400in}%
\pgfsys@useobject{currentmarker}{}%
\end{pgfscope}%
\begin{pgfscope}%
\pgfsys@transformshift{3.129105in}{5.246287in}%
\pgfsys@useobject{currentmarker}{}%
\end{pgfscope}%
\begin{pgfscope}%
\pgfsys@transformshift{1.943354in}{4.893697in}%
\pgfsys@useobject{currentmarker}{}%
\end{pgfscope}%
\begin{pgfscope}%
\pgfsys@transformshift{3.587215in}{3.580868in}%
\pgfsys@useobject{currentmarker}{}%
\end{pgfscope}%
\begin{pgfscope}%
\pgfsys@transformshift{2.342935in}{2.097338in}%
\pgfsys@useobject{currentmarker}{}%
\end{pgfscope}%
\begin{pgfscope}%
\pgfsys@transformshift{1.021497in}{0.690499in}%
\pgfsys@useobject{currentmarker}{}%
\end{pgfscope}%
\begin{pgfscope}%
\pgfsys@transformshift{3.817014in}{1.576762in}%
\pgfsys@useobject{currentmarker}{}%
\end{pgfscope}%
\begin{pgfscope}%
\pgfsys@transformshift{5.442389in}{2.100678in}%
\pgfsys@useobject{currentmarker}{}%
\end{pgfscope}%
\begin{pgfscope}%
\pgfsys@transformshift{6.058288in}{4.673716in}%
\pgfsys@useobject{currentmarker}{}%
\end{pgfscope}%
\begin{pgfscope}%
\pgfsys@transformshift{1.979398in}{0.840928in}%
\pgfsys@useobject{currentmarker}{}%
\end{pgfscope}%
\begin{pgfscope}%
\pgfsys@transformshift{2.648712in}{3.247655in}%
\pgfsys@useobject{currentmarker}{}%
\end{pgfscope}%
\begin{pgfscope}%
\pgfsys@transformshift{5.949050in}{5.533644in}%
\pgfsys@useobject{currentmarker}{}%
\end{pgfscope}%
\begin{pgfscope}%
\pgfsys@transformshift{5.550495in}{1.991920in}%
\pgfsys@useobject{currentmarker}{}%
\end{pgfscope}%
\begin{pgfscope}%
\pgfsys@transformshift{6.755225in}{5.103245in}%
\pgfsys@useobject{currentmarker}{}%
\end{pgfscope}%
\begin{pgfscope}%
\pgfsys@transformshift{3.865731in}{2.334149in}%
\pgfsys@useobject{currentmarker}{}%
\end{pgfscope}%
\begin{pgfscope}%
\pgfsys@transformshift{2.976125in}{5.561992in}%
\pgfsys@useobject{currentmarker}{}%
\end{pgfscope}%
\begin{pgfscope}%
\pgfsys@transformshift{3.081797in}{3.709330in}%
\pgfsys@useobject{currentmarker}{}%
\end{pgfscope}%
\begin{pgfscope}%
\pgfsys@transformshift{1.382073in}{0.821854in}%
\pgfsys@useobject{currentmarker}{}%
\end{pgfscope}%
\begin{pgfscope}%
\pgfsys@transformshift{5.184168in}{4.949794in}%
\pgfsys@useobject{currentmarker}{}%
\end{pgfscope}%
\begin{pgfscope}%
\pgfsys@transformshift{4.142106in}{4.831008in}%
\pgfsys@useobject{currentmarker}{}%
\end{pgfscope}%
\begin{pgfscope}%
\pgfsys@transformshift{0.655059in}{2.256314in}%
\pgfsys@useobject{currentmarker}{}%
\end{pgfscope}%
\begin{pgfscope}%
\pgfsys@transformshift{0.802744in}{2.327255in}%
\pgfsys@useobject{currentmarker}{}%
\end{pgfscope}%
\begin{pgfscope}%
\pgfsys@transformshift{2.579353in}{2.170771in}%
\pgfsys@useobject{currentmarker}{}%
\end{pgfscope}%
\begin{pgfscope}%
\pgfsys@transformshift{1.585746in}{4.271890in}%
\pgfsys@useobject{currentmarker}{}%
\end{pgfscope}%
\begin{pgfscope}%
\pgfsys@transformshift{6.865968in}{0.230952in}%
\pgfsys@useobject{currentmarker}{}%
\end{pgfscope}%
\begin{pgfscope}%
\pgfsys@transformshift{4.662028in}{2.002836in}%
\pgfsys@useobject{currentmarker}{}%
\end{pgfscope}%
\begin{pgfscope}%
\pgfsys@transformshift{4.027602in}{2.510529in}%
\pgfsys@useobject{currentmarker}{}%
\end{pgfscope}%
\begin{pgfscope}%
\pgfsys@transformshift{6.703161in}{1.286896in}%
\pgfsys@useobject{currentmarker}{}%
\end{pgfscope}%
\begin{pgfscope}%
\pgfsys@transformshift{4.412214in}{4.151845in}%
\pgfsys@useobject{currentmarker}{}%
\end{pgfscope}%
\begin{pgfscope}%
\pgfsys@transformshift{1.445357in}{3.910361in}%
\pgfsys@useobject{currentmarker}{}%
\end{pgfscope}%
\begin{pgfscope}%
\pgfsys@transformshift{5.808650in}{2.612827in}%
\pgfsys@useobject{currentmarker}{}%
\end{pgfscope}%
\begin{pgfscope}%
\pgfsys@transformshift{1.220703in}{0.896912in}%
\pgfsys@useobject{currentmarker}{}%
\end{pgfscope}%
\begin{pgfscope}%
\pgfsys@transformshift{3.603961in}{4.178180in}%
\pgfsys@useobject{currentmarker}{}%
\end{pgfscope}%
\begin{pgfscope}%
\pgfsys@transformshift{4.539430in}{0.774317in}%
\pgfsys@useobject{currentmarker}{}%
\end{pgfscope}%
\begin{pgfscope}%
\pgfsys@transformshift{2.716342in}{1.761126in}%
\pgfsys@useobject{currentmarker}{}%
\end{pgfscope}%
\begin{pgfscope}%
\pgfsys@transformshift{2.797144in}{0.838678in}%
\pgfsys@useobject{currentmarker}{}%
\end{pgfscope}%
\begin{pgfscope}%
\pgfsys@transformshift{3.762384in}{2.095525in}%
\pgfsys@useobject{currentmarker}{}%
\end{pgfscope}%
\begin{pgfscope}%
\pgfsys@transformshift{5.164025in}{0.672611in}%
\pgfsys@useobject{currentmarker}{}%
\end{pgfscope}%
\begin{pgfscope}%
\pgfsys@transformshift{1.870542in}{0.457542in}%
\pgfsys@useobject{currentmarker}{}%
\end{pgfscope}%
\begin{pgfscope}%
\pgfsys@transformshift{2.910121in}{4.505037in}%
\pgfsys@useobject{currentmarker}{}%
\end{pgfscope}%
\begin{pgfscope}%
\pgfsys@transformshift{1.411858in}{2.496224in}%
\pgfsys@useobject{currentmarker}{}%
\end{pgfscope}%
\begin{pgfscope}%
\pgfsys@transformshift{4.895000in}{0.989972in}%
\pgfsys@useobject{currentmarker}{}%
\end{pgfscope}%
\begin{pgfscope}%
\pgfsys@transformshift{0.738002in}{0.308557in}%
\pgfsys@useobject{currentmarker}{}%
\end{pgfscope}%
\begin{pgfscope}%
\pgfsys@transformshift{4.221218in}{1.777841in}%
\pgfsys@useobject{currentmarker}{}%
\end{pgfscope}%
\begin{pgfscope}%
\pgfsys@transformshift{4.841482in}{1.341076in}%
\pgfsys@useobject{currentmarker}{}%
\end{pgfscope}%
\begin{pgfscope}%
\pgfsys@transformshift{3.825316in}{5.064443in}%
\pgfsys@useobject{currentmarker}{}%
\end{pgfscope}%
\begin{pgfscope}%
\pgfsys@transformshift{4.349185in}{3.651545in}%
\pgfsys@useobject{currentmarker}{}%
\end{pgfscope}%
\begin{pgfscope}%
\pgfsys@transformshift{6.614449in}{2.078768in}%
\pgfsys@useobject{currentmarker}{}%
\end{pgfscope}%
\begin{pgfscope}%
\pgfsys@transformshift{3.440025in}{3.712999in}%
\pgfsys@useobject{currentmarker}{}%
\end{pgfscope}%
\begin{pgfscope}%
\pgfsys@transformshift{3.524073in}{3.416601in}%
\pgfsys@useobject{currentmarker}{}%
\end{pgfscope}%
\begin{pgfscope}%
\pgfsys@transformshift{0.656534in}{3.158901in}%
\pgfsys@useobject{currentmarker}{}%
\end{pgfscope}%
\begin{pgfscope}%
\pgfsys@transformshift{2.233883in}{2.221251in}%
\pgfsys@useobject{currentmarker}{}%
\end{pgfscope}%
\begin{pgfscope}%
\pgfsys@transformshift{5.568420in}{1.991570in}%
\pgfsys@useobject{currentmarker}{}%
\end{pgfscope}%
\begin{pgfscope}%
\pgfsys@transformshift{6.063286in}{3.907894in}%
\pgfsys@useobject{currentmarker}{}%
\end{pgfscope}%
\begin{pgfscope}%
\pgfsys@transformshift{5.342023in}{3.563392in}%
\pgfsys@useobject{currentmarker}{}%
\end{pgfscope}%
\begin{pgfscope}%
\pgfsys@transformshift{6.739944in}{5.157110in}%
\pgfsys@useobject{currentmarker}{}%
\end{pgfscope}%
\begin{pgfscope}%
\pgfsys@transformshift{6.204109in}{4.336997in}%
\pgfsys@useobject{currentmarker}{}%
\end{pgfscope}%
\begin{pgfscope}%
\pgfsys@transformshift{0.896520in}{0.404529in}%
\pgfsys@useobject{currentmarker}{}%
\end{pgfscope}%
\begin{pgfscope}%
\pgfsys@transformshift{5.692888in}{5.529288in}%
\pgfsys@useobject{currentmarker}{}%
\end{pgfscope}%
\begin{pgfscope}%
\pgfsys@transformshift{4.756665in}{2.321836in}%
\pgfsys@useobject{currentmarker}{}%
\end{pgfscope}%
\begin{pgfscope}%
\pgfsys@transformshift{4.620765in}{1.241046in}%
\pgfsys@useobject{currentmarker}{}%
\end{pgfscope}%
\begin{pgfscope}%
\pgfsys@transformshift{3.198662in}{2.226083in}%
\pgfsys@useobject{currentmarker}{}%
\end{pgfscope}%
\begin{pgfscope}%
\pgfsys@transformshift{2.414933in}{1.108528in}%
\pgfsys@useobject{currentmarker}{}%
\end{pgfscope}%
\begin{pgfscope}%
\pgfsys@transformshift{6.506654in}{5.601735in}%
\pgfsys@useobject{currentmarker}{}%
\end{pgfscope}%
\begin{pgfscope}%
\pgfsys@transformshift{4.142466in}{3.244470in}%
\pgfsys@useobject{currentmarker}{}%
\end{pgfscope}%
\begin{pgfscope}%
\pgfsys@transformshift{6.658868in}{3.905016in}%
\pgfsys@useobject{currentmarker}{}%
\end{pgfscope}%
\begin{pgfscope}%
\pgfsys@transformshift{0.934661in}{1.771176in}%
\pgfsys@useobject{currentmarker}{}%
\end{pgfscope}%
\begin{pgfscope}%
\pgfsys@transformshift{3.154612in}{2.342803in}%
\pgfsys@useobject{currentmarker}{}%
\end{pgfscope}%
\begin{pgfscope}%
\pgfsys@transformshift{2.002356in}{2.343359in}%
\pgfsys@useobject{currentmarker}{}%
\end{pgfscope}%
\begin{pgfscope}%
\pgfsys@transformshift{5.090432in}{4.517353in}%
\pgfsys@useobject{currentmarker}{}%
\end{pgfscope}%
\begin{pgfscope}%
\pgfsys@transformshift{6.080457in}{1.202822in}%
\pgfsys@useobject{currentmarker}{}%
\end{pgfscope}%
\begin{pgfscope}%
\pgfsys@transformshift{3.323354in}{1.022890in}%
\pgfsys@useobject{currentmarker}{}%
\end{pgfscope}%
\begin{pgfscope}%
\pgfsys@transformshift{2.354026in}{0.531485in}%
\pgfsys@useobject{currentmarker}{}%
\end{pgfscope}%
\begin{pgfscope}%
\pgfsys@transformshift{4.593579in}{1.800996in}%
\pgfsys@useobject{currentmarker}{}%
\end{pgfscope}%
\begin{pgfscope}%
\pgfsys@transformshift{6.272463in}{1.782078in}%
\pgfsys@useobject{currentmarker}{}%
\end{pgfscope}%
\begin{pgfscope}%
\pgfsys@transformshift{1.152253in}{2.650657in}%
\pgfsys@useobject{currentmarker}{}%
\end{pgfscope}%
\begin{pgfscope}%
\pgfsys@transformshift{1.944140in}{3.576537in}%
\pgfsys@useobject{currentmarker}{}%
\end{pgfscope}%
\begin{pgfscope}%
\pgfsys@transformshift{3.994197in}{5.242610in}%
\pgfsys@useobject{currentmarker}{}%
\end{pgfscope}%
\begin{pgfscope}%
\pgfsys@transformshift{2.653452in}{4.274385in}%
\pgfsys@useobject{currentmarker}{}%
\end{pgfscope}%
\begin{pgfscope}%
\pgfsys@transformshift{5.626037in}{0.483404in}%
\pgfsys@useobject{currentmarker}{}%
\end{pgfscope}%
\begin{pgfscope}%
\pgfsys@transformshift{2.255638in}{4.246827in}%
\pgfsys@useobject{currentmarker}{}%
\end{pgfscope}%
\begin{pgfscope}%
\pgfsys@transformshift{4.652513in}{3.859054in}%
\pgfsys@useobject{currentmarker}{}%
\end{pgfscope}%
\begin{pgfscope}%
\pgfsys@transformshift{3.277927in}{0.472024in}%
\pgfsys@useobject{currentmarker}{}%
\end{pgfscope}%
\begin{pgfscope}%
\pgfsys@transformshift{2.053643in}{2.791624in}%
\pgfsys@useobject{currentmarker}{}%
\end{pgfscope}%
\begin{pgfscope}%
\pgfsys@transformshift{3.166837in}{4.829729in}%
\pgfsys@useobject{currentmarker}{}%
\end{pgfscope}%
\begin{pgfscope}%
\pgfsys@transformshift{5.976142in}{5.391060in}%
\pgfsys@useobject{currentmarker}{}%
\end{pgfscope}%
\begin{pgfscope}%
\pgfsys@transformshift{6.825375in}{5.032743in}%
\pgfsys@useobject{currentmarker}{}%
\end{pgfscope}%
\begin{pgfscope}%
\pgfsys@transformshift{4.646199in}{2.772443in}%
\pgfsys@useobject{currentmarker}{}%
\end{pgfscope}%
\begin{pgfscope}%
\pgfsys@transformshift{1.824536in}{0.631393in}%
\pgfsys@useobject{currentmarker}{}%
\end{pgfscope}%
\begin{pgfscope}%
\pgfsys@transformshift{4.222637in}{3.845652in}%
\pgfsys@useobject{currentmarker}{}%
\end{pgfscope}%
\begin{pgfscope}%
\pgfsys@transformshift{4.493681in}{0.668277in}%
\pgfsys@useobject{currentmarker}{}%
\end{pgfscope}%
\begin{pgfscope}%
\pgfsys@transformshift{5.758562in}{0.396976in}%
\pgfsys@useobject{currentmarker}{}%
\end{pgfscope}%
\begin{pgfscope}%
\pgfsys@transformshift{4.139112in}{0.785824in}%
\pgfsys@useobject{currentmarker}{}%
\end{pgfscope}%
\begin{pgfscope}%
\pgfsys@transformshift{6.889347in}{3.604728in}%
\pgfsys@useobject{currentmarker}{}%
\end{pgfscope}%
\begin{pgfscope}%
\pgfsys@transformshift{2.493739in}{4.962428in}%
\pgfsys@useobject{currentmarker}{}%
\end{pgfscope}%
\begin{pgfscope}%
\pgfsys@transformshift{4.887080in}{1.597741in}%
\pgfsys@useobject{currentmarker}{}%
\end{pgfscope}%
\begin{pgfscope}%
\pgfsys@transformshift{6.166293in}{5.525876in}%
\pgfsys@useobject{currentmarker}{}%
\end{pgfscope}%
\begin{pgfscope}%
\pgfsys@transformshift{3.989516in}{3.609263in}%
\pgfsys@useobject{currentmarker}{}%
\end{pgfscope}%
\begin{pgfscope}%
\pgfsys@transformshift{3.344430in}{0.694691in}%
\pgfsys@useobject{currentmarker}{}%
\end{pgfscope}%
\begin{pgfscope}%
\pgfsys@transformshift{6.075360in}{0.974671in}%
\pgfsys@useobject{currentmarker}{}%
\end{pgfscope}%
\begin{pgfscope}%
\pgfsys@transformshift{4.786806in}{2.685359in}%
\pgfsys@useobject{currentmarker}{}%
\end{pgfscope}%
\begin{pgfscope}%
\pgfsys@transformshift{2.275116in}{4.214604in}%
\pgfsys@useobject{currentmarker}{}%
\end{pgfscope}%
\begin{pgfscope}%
\pgfsys@transformshift{5.820529in}{4.658387in}%
\pgfsys@useobject{currentmarker}{}%
\end{pgfscope}%
\begin{pgfscope}%
\pgfsys@transformshift{1.870196in}{3.240343in}%
\pgfsys@useobject{currentmarker}{}%
\end{pgfscope}%
\begin{pgfscope}%
\pgfsys@transformshift{1.672379in}{4.284370in}%
\pgfsys@useobject{currentmarker}{}%
\end{pgfscope}%
\begin{pgfscope}%
\pgfsys@transformshift{2.302752in}{4.962718in}%
\pgfsys@useobject{currentmarker}{}%
\end{pgfscope}%
\begin{pgfscope}%
\pgfsys@transformshift{0.673143in}{0.893077in}%
\pgfsys@useobject{currentmarker}{}%
\end{pgfscope}%
\begin{pgfscope}%
\pgfsys@transformshift{1.263833in}{1.611700in}%
\pgfsys@useobject{currentmarker}{}%
\end{pgfscope}%
\begin{pgfscope}%
\pgfsys@transformshift{4.282282in}{4.761581in}%
\pgfsys@useobject{currentmarker}{}%
\end{pgfscope}%
\begin{pgfscope}%
\pgfsys@transformshift{4.622965in}{3.963335in}%
\pgfsys@useobject{currentmarker}{}%
\end{pgfscope}%
\begin{pgfscope}%
\pgfsys@transformshift{1.301933in}{1.347133in}%
\pgfsys@useobject{currentmarker}{}%
\end{pgfscope}%
\begin{pgfscope}%
\pgfsys@transformshift{5.734062in}{3.964222in}%
\pgfsys@useobject{currentmarker}{}%
\end{pgfscope}%
\begin{pgfscope}%
\pgfsys@transformshift{6.543641in}{1.069784in}%
\pgfsys@useobject{currentmarker}{}%
\end{pgfscope}%
\begin{pgfscope}%
\pgfsys@transformshift{1.455448in}{1.633203in}%
\pgfsys@useobject{currentmarker}{}%
\end{pgfscope}%
\begin{pgfscope}%
\pgfsys@transformshift{5.604056in}{2.970101in}%
\pgfsys@useobject{currentmarker}{}%
\end{pgfscope}%
\begin{pgfscope}%
\pgfsys@transformshift{4.489020in}{1.154589in}%
\pgfsys@useobject{currentmarker}{}%
\end{pgfscope}%
\begin{pgfscope}%
\pgfsys@transformshift{4.421605in}{3.229968in}%
\pgfsys@useobject{currentmarker}{}%
\end{pgfscope}%
\begin{pgfscope}%
\pgfsys@transformshift{4.858586in}{2.609125in}%
\pgfsys@useobject{currentmarker}{}%
\end{pgfscope}%
\begin{pgfscope}%
\pgfsys@transformshift{4.648877in}{2.516551in}%
\pgfsys@useobject{currentmarker}{}%
\end{pgfscope}%
\begin{pgfscope}%
\pgfsys@transformshift{4.130094in}{1.571152in}%
\pgfsys@useobject{currentmarker}{}%
\end{pgfscope}%
\begin{pgfscope}%
\pgfsys@transformshift{0.896027in}{1.189427in}%
\pgfsys@useobject{currentmarker}{}%
\end{pgfscope}%
\begin{pgfscope}%
\pgfsys@transformshift{1.847530in}{0.612260in}%
\pgfsys@useobject{currentmarker}{}%
\end{pgfscope}%
\begin{pgfscope}%
\pgfsys@transformshift{6.351608in}{0.972429in}%
\pgfsys@useobject{currentmarker}{}%
\end{pgfscope}%
\begin{pgfscope}%
\pgfsys@transformshift{6.134735in}{5.352141in}%
\pgfsys@useobject{currentmarker}{}%
\end{pgfscope}%
\begin{pgfscope}%
\pgfsys@transformshift{6.182954in}{5.595977in}%
\pgfsys@useobject{currentmarker}{}%
\end{pgfscope}%
\begin{pgfscope}%
\pgfsys@transformshift{6.046383in}{4.049364in}%
\pgfsys@useobject{currentmarker}{}%
\end{pgfscope}%
\begin{pgfscope}%
\pgfsys@transformshift{5.741789in}{2.214999in}%
\pgfsys@useobject{currentmarker}{}%
\end{pgfscope}%
\begin{pgfscope}%
\pgfsys@transformshift{2.504074in}{4.901585in}%
\pgfsys@useobject{currentmarker}{}%
\end{pgfscope}%
\begin{pgfscope}%
\pgfsys@transformshift{4.351859in}{0.591350in}%
\pgfsys@useobject{currentmarker}{}%
\end{pgfscope}%
\begin{pgfscope}%
\pgfsys@transformshift{3.783058in}{2.728852in}%
\pgfsys@useobject{currentmarker}{}%
\end{pgfscope}%
\begin{pgfscope}%
\pgfsys@transformshift{6.770911in}{3.250488in}%
\pgfsys@useobject{currentmarker}{}%
\end{pgfscope}%
\begin{pgfscope}%
\pgfsys@transformshift{6.342129in}{0.486091in}%
\pgfsys@useobject{currentmarker}{}%
\end{pgfscope}%
\begin{pgfscope}%
\pgfsys@transformshift{2.417289in}{3.706067in}%
\pgfsys@useobject{currentmarker}{}%
\end{pgfscope}%
\begin{pgfscope}%
\pgfsys@transformshift{4.438185in}{3.896967in}%
\pgfsys@useobject{currentmarker}{}%
\end{pgfscope}%
\begin{pgfscope}%
\pgfsys@transformshift{1.602751in}{2.048253in}%
\pgfsys@useobject{currentmarker}{}%
\end{pgfscope}%
\begin{pgfscope}%
\pgfsys@transformshift{5.482158in}{0.576145in}%
\pgfsys@useobject{currentmarker}{}%
\end{pgfscope}%
\begin{pgfscope}%
\pgfsys@transformshift{2.180598in}{2.896309in}%
\pgfsys@useobject{currentmarker}{}%
\end{pgfscope}%
\begin{pgfscope}%
\pgfsys@transformshift{1.404368in}{0.948406in}%
\pgfsys@useobject{currentmarker}{}%
\end{pgfscope}%
\begin{pgfscope}%
\pgfsys@transformshift{4.519508in}{2.969509in}%
\pgfsys@useobject{currentmarker}{}%
\end{pgfscope}%
\begin{pgfscope}%
\pgfsys@transformshift{2.527479in}{4.051990in}%
\pgfsys@useobject{currentmarker}{}%
\end{pgfscope}%
\begin{pgfscope}%
\pgfsys@transformshift{3.248174in}{0.965477in}%
\pgfsys@useobject{currentmarker}{}%
\end{pgfscope}%
\begin{pgfscope}%
\pgfsys@transformshift{3.839753in}{0.806067in}%
\pgfsys@useobject{currentmarker}{}%
\end{pgfscope}%
\begin{pgfscope}%
\pgfsys@transformshift{6.845770in}{3.650208in}%
\pgfsys@useobject{currentmarker}{}%
\end{pgfscope}%
\begin{pgfscope}%
\pgfsys@transformshift{1.617632in}{3.341566in}%
\pgfsys@useobject{currentmarker}{}%
\end{pgfscope}%
\begin{pgfscope}%
\pgfsys@transformshift{2.896925in}{4.283457in}%
\pgfsys@useobject{currentmarker}{}%
\end{pgfscope}%
\begin{pgfscope}%
\pgfsys@transformshift{6.327943in}{3.286432in}%
\pgfsys@useobject{currentmarker}{}%
\end{pgfscope}%
\begin{pgfscope}%
\pgfsys@transformshift{3.189071in}{0.912943in}%
\pgfsys@useobject{currentmarker}{}%
\end{pgfscope}%
\begin{pgfscope}%
\pgfsys@transformshift{2.362344in}{2.960396in}%
\pgfsys@useobject{currentmarker}{}%
\end{pgfscope}%
\begin{pgfscope}%
\pgfsys@transformshift{2.429558in}{1.668869in}%
\pgfsys@useobject{currentmarker}{}%
\end{pgfscope}%
\begin{pgfscope}%
\pgfsys@transformshift{3.106341in}{1.308808in}%
\pgfsys@useobject{currentmarker}{}%
\end{pgfscope}%
\begin{pgfscope}%
\pgfsys@transformshift{3.037818in}{3.917910in}%
\pgfsys@useobject{currentmarker}{}%
\end{pgfscope}%
\begin{pgfscope}%
\pgfsys@transformshift{5.310423in}{0.359517in}%
\pgfsys@useobject{currentmarker}{}%
\end{pgfscope}%
\begin{pgfscope}%
\pgfsys@transformshift{5.540275in}{4.582562in}%
\pgfsys@useobject{currentmarker}{}%
\end{pgfscope}%
\begin{pgfscope}%
\pgfsys@transformshift{4.445540in}{2.355851in}%
\pgfsys@useobject{currentmarker}{}%
\end{pgfscope}%
\begin{pgfscope}%
\pgfsys@transformshift{6.144699in}{0.842884in}%
\pgfsys@useobject{currentmarker}{}%
\end{pgfscope}%
\begin{pgfscope}%
\pgfsys@transformshift{5.587596in}{4.316849in}%
\pgfsys@useobject{currentmarker}{}%
\end{pgfscope}%
\begin{pgfscope}%
\pgfsys@transformshift{5.860039in}{4.270332in}%
\pgfsys@useobject{currentmarker}{}%
\end{pgfscope}%
\begin{pgfscope}%
\pgfsys@transformshift{1.327747in}{0.342544in}%
\pgfsys@useobject{currentmarker}{}%
\end{pgfscope}%
\begin{pgfscope}%
\pgfsys@transformshift{5.284133in}{3.990551in}%
\pgfsys@useobject{currentmarker}{}%
\end{pgfscope}%
\begin{pgfscope}%
\pgfsys@transformshift{3.993120in}{0.899747in}%
\pgfsys@useobject{currentmarker}{}%
\end{pgfscope}%
\begin{pgfscope}%
\pgfsys@transformshift{1.820555in}{3.037024in}%
\pgfsys@useobject{currentmarker}{}%
\end{pgfscope}%
\begin{pgfscope}%
\pgfsys@transformshift{4.976421in}{5.467875in}%
\pgfsys@useobject{currentmarker}{}%
\end{pgfscope}%
\begin{pgfscope}%
\pgfsys@transformshift{4.518831in}{5.000925in}%
\pgfsys@useobject{currentmarker}{}%
\end{pgfscope}%
\begin{pgfscope}%
\pgfsys@transformshift{5.758990in}{0.597983in}%
\pgfsys@useobject{currentmarker}{}%
\end{pgfscope}%
\begin{pgfscope}%
\pgfsys@transformshift{5.322391in}{5.195192in}%
\pgfsys@useobject{currentmarker}{}%
\end{pgfscope}%
\begin{pgfscope}%
\pgfsys@transformshift{6.191893in}{3.808835in}%
\pgfsys@useobject{currentmarker}{}%
\end{pgfscope}%
\begin{pgfscope}%
\pgfsys@transformshift{4.112982in}{3.072140in}%
\pgfsys@useobject{currentmarker}{}%
\end{pgfscope}%
\begin{pgfscope}%
\pgfsys@transformshift{4.346249in}{2.073142in}%
\pgfsys@useobject{currentmarker}{}%
\end{pgfscope}%
\begin{pgfscope}%
\pgfsys@transformshift{5.031335in}{4.367788in}%
\pgfsys@useobject{currentmarker}{}%
\end{pgfscope}%
\begin{pgfscope}%
\pgfsys@transformshift{2.320682in}{1.898677in}%
\pgfsys@useobject{currentmarker}{}%
\end{pgfscope}%
\begin{pgfscope}%
\pgfsys@transformshift{2.639656in}{4.074777in}%
\pgfsys@useobject{currentmarker}{}%
\end{pgfscope}%
\begin{pgfscope}%
\pgfsys@transformshift{4.591237in}{3.148890in}%
\pgfsys@useobject{currentmarker}{}%
\end{pgfscope}%
\begin{pgfscope}%
\pgfsys@transformshift{5.707718in}{0.457513in}%
\pgfsys@useobject{currentmarker}{}%
\end{pgfscope}%
\begin{pgfscope}%
\pgfsys@transformshift{2.585349in}{5.586581in}%
\pgfsys@useobject{currentmarker}{}%
\end{pgfscope}%
\begin{pgfscope}%
\pgfsys@transformshift{1.537944in}{5.101799in}%
\pgfsys@useobject{currentmarker}{}%
\end{pgfscope}%
\begin{pgfscope}%
\pgfsys@transformshift{5.052617in}{2.372156in}%
\pgfsys@useobject{currentmarker}{}%
\end{pgfscope}%
\begin{pgfscope}%
\pgfsys@transformshift{3.531247in}{3.288660in}%
\pgfsys@useobject{currentmarker}{}%
\end{pgfscope}%
\begin{pgfscope}%
\pgfsys@transformshift{5.727482in}{2.427963in}%
\pgfsys@useobject{currentmarker}{}%
\end{pgfscope}%
\begin{pgfscope}%
\pgfsys@transformshift{1.090843in}{5.060215in}%
\pgfsys@useobject{currentmarker}{}%
\end{pgfscope}%
\begin{pgfscope}%
\pgfsys@transformshift{0.917944in}{4.411994in}%
\pgfsys@useobject{currentmarker}{}%
\end{pgfscope}%
\begin{pgfscope}%
\pgfsys@transformshift{5.944407in}{4.296100in}%
\pgfsys@useobject{currentmarker}{}%
\end{pgfscope}%
\begin{pgfscope}%
\pgfsys@transformshift{2.400886in}{0.241725in}%
\pgfsys@useobject{currentmarker}{}%
\end{pgfscope}%
\begin{pgfscope}%
\pgfsys@transformshift{3.933951in}{2.423825in}%
\pgfsys@useobject{currentmarker}{}%
\end{pgfscope}%
\begin{pgfscope}%
\pgfsys@transformshift{2.953373in}{0.403868in}%
\pgfsys@useobject{currentmarker}{}%
\end{pgfscope}%
\begin{pgfscope}%
\pgfsys@transformshift{3.684811in}{5.370307in}%
\pgfsys@useobject{currentmarker}{}%
\end{pgfscope}%
\begin{pgfscope}%
\pgfsys@transformshift{4.527017in}{2.440840in}%
\pgfsys@useobject{currentmarker}{}%
\end{pgfscope}%
\begin{pgfscope}%
\pgfsys@transformshift{1.597930in}{3.973651in}%
\pgfsys@useobject{currentmarker}{}%
\end{pgfscope}%
\begin{pgfscope}%
\pgfsys@transformshift{4.244140in}{5.201478in}%
\pgfsys@useobject{currentmarker}{}%
\end{pgfscope}%
\begin{pgfscope}%
\pgfsys@transformshift{2.731012in}{2.446474in}%
\pgfsys@useobject{currentmarker}{}%
\end{pgfscope}%
\begin{pgfscope}%
\pgfsys@transformshift{1.721705in}{3.275660in}%
\pgfsys@useobject{currentmarker}{}%
\end{pgfscope}%
\begin{pgfscope}%
\pgfsys@transformshift{6.883942in}{1.299839in}%
\pgfsys@useobject{currentmarker}{}%
\end{pgfscope}%
\begin{pgfscope}%
\pgfsys@transformshift{1.465171in}{0.970937in}%
\pgfsys@useobject{currentmarker}{}%
\end{pgfscope}%
\begin{pgfscope}%
\pgfsys@transformshift{2.667479in}{5.139195in}%
\pgfsys@useobject{currentmarker}{}%
\end{pgfscope}%
\begin{pgfscope}%
\pgfsys@transformshift{2.701238in}{2.707072in}%
\pgfsys@useobject{currentmarker}{}%
\end{pgfscope}%
\begin{pgfscope}%
\pgfsys@transformshift{6.547016in}{2.283006in}%
\pgfsys@useobject{currentmarker}{}%
\end{pgfscope}%
\begin{pgfscope}%
\pgfsys@transformshift{3.266648in}{5.347777in}%
\pgfsys@useobject{currentmarker}{}%
\end{pgfscope}%
\begin{pgfscope}%
\pgfsys@transformshift{2.378545in}{1.229894in}%
\pgfsys@useobject{currentmarker}{}%
\end{pgfscope}%
\begin{pgfscope}%
\pgfsys@transformshift{5.194616in}{2.453481in}%
\pgfsys@useobject{currentmarker}{}%
\end{pgfscope}%
\begin{pgfscope}%
\pgfsys@transformshift{4.251196in}{5.418875in}%
\pgfsys@useobject{currentmarker}{}%
\end{pgfscope}%
\begin{pgfscope}%
\pgfsys@transformshift{5.395600in}{2.679137in}%
\pgfsys@useobject{currentmarker}{}%
\end{pgfscope}%
\begin{pgfscope}%
\pgfsys@transformshift{1.518682in}{4.800976in}%
\pgfsys@useobject{currentmarker}{}%
\end{pgfscope}%
\begin{pgfscope}%
\pgfsys@transformshift{4.342434in}{0.678690in}%
\pgfsys@useobject{currentmarker}{}%
\end{pgfscope}%
\begin{pgfscope}%
\pgfsys@transformshift{6.703472in}{2.313088in}%
\pgfsys@useobject{currentmarker}{}%
\end{pgfscope}%
\begin{pgfscope}%
\pgfsys@transformshift{0.850509in}{5.345405in}%
\pgfsys@useobject{currentmarker}{}%
\end{pgfscope}%
\begin{pgfscope}%
\pgfsys@transformshift{4.473622in}{2.677575in}%
\pgfsys@useobject{currentmarker}{}%
\end{pgfscope}%
\begin{pgfscope}%
\pgfsys@transformshift{3.497820in}{5.397189in}%
\pgfsys@useobject{currentmarker}{}%
\end{pgfscope}%
\begin{pgfscope}%
\pgfsys@transformshift{2.507566in}{2.142928in}%
\pgfsys@useobject{currentmarker}{}%
\end{pgfscope}%
\begin{pgfscope}%
\pgfsys@transformshift{1.572349in}{1.477577in}%
\pgfsys@useobject{currentmarker}{}%
\end{pgfscope}%
\begin{pgfscope}%
\pgfsys@transformshift{0.918109in}{2.102665in}%
\pgfsys@useobject{currentmarker}{}%
\end{pgfscope}%
\begin{pgfscope}%
\pgfsys@transformshift{5.551334in}{2.724328in}%
\pgfsys@useobject{currentmarker}{}%
\end{pgfscope}%
\begin{pgfscope}%
\pgfsys@transformshift{3.804229in}{3.856476in}%
\pgfsys@useobject{currentmarker}{}%
\end{pgfscope}%
\begin{pgfscope}%
\pgfsys@transformshift{3.405157in}{1.312488in}%
\pgfsys@useobject{currentmarker}{}%
\end{pgfscope}%
\begin{pgfscope}%
\pgfsys@transformshift{6.210793in}{3.665697in}%
\pgfsys@useobject{currentmarker}{}%
\end{pgfscope}%
\begin{pgfscope}%
\pgfsys@transformshift{5.999147in}{3.448027in}%
\pgfsys@useobject{currentmarker}{}%
\end{pgfscope}%
\begin{pgfscope}%
\pgfsys@transformshift{3.648746in}{5.011550in}%
\pgfsys@useobject{currentmarker}{}%
\end{pgfscope}%
\begin{pgfscope}%
\pgfsys@transformshift{0.730912in}{5.317873in}%
\pgfsys@useobject{currentmarker}{}%
\end{pgfscope}%
\begin{pgfscope}%
\pgfsys@transformshift{1.721779in}{0.648734in}%
\pgfsys@useobject{currentmarker}{}%
\end{pgfscope}%
\begin{pgfscope}%
\pgfsys@transformshift{2.297439in}{3.943681in}%
\pgfsys@useobject{currentmarker}{}%
\end{pgfscope}%
\begin{pgfscope}%
\pgfsys@transformshift{4.635825in}{4.998243in}%
\pgfsys@useobject{currentmarker}{}%
\end{pgfscope}%
\begin{pgfscope}%
\pgfsys@transformshift{1.290406in}{1.104941in}%
\pgfsys@useobject{currentmarker}{}%
\end{pgfscope}%
\begin{pgfscope}%
\pgfsys@transformshift{0.954170in}{5.152561in}%
\pgfsys@useobject{currentmarker}{}%
\end{pgfscope}%
\begin{pgfscope}%
\pgfsys@transformshift{3.740532in}{1.895624in}%
\pgfsys@useobject{currentmarker}{}%
\end{pgfscope}%
\begin{pgfscope}%
\pgfsys@transformshift{1.671660in}{5.497159in}%
\pgfsys@useobject{currentmarker}{}%
\end{pgfscope}%
\begin{pgfscope}%
\pgfsys@transformshift{4.359756in}{3.536926in}%
\pgfsys@useobject{currentmarker}{}%
\end{pgfscope}%
\begin{pgfscope}%
\pgfsys@transformshift{4.913462in}{0.591345in}%
\pgfsys@useobject{currentmarker}{}%
\end{pgfscope}%
\begin{pgfscope}%
\pgfsys@transformshift{5.375068in}{1.908884in}%
\pgfsys@useobject{currentmarker}{}%
\end{pgfscope}%
\begin{pgfscope}%
\pgfsys@transformshift{4.581745in}{0.846596in}%
\pgfsys@useobject{currentmarker}{}%
\end{pgfscope}%
\begin{pgfscope}%
\pgfsys@transformshift{5.444748in}{5.453287in}%
\pgfsys@useobject{currentmarker}{}%
\end{pgfscope}%
\begin{pgfscope}%
\pgfsys@transformshift{3.442235in}{3.630797in}%
\pgfsys@useobject{currentmarker}{}%
\end{pgfscope}%
\begin{pgfscope}%
\pgfsys@transformshift{1.615806in}{4.731700in}%
\pgfsys@useobject{currentmarker}{}%
\end{pgfscope}%
\begin{pgfscope}%
\pgfsys@transformshift{3.854399in}{3.314784in}%
\pgfsys@useobject{currentmarker}{}%
\end{pgfscope}%
\begin{pgfscope}%
\pgfsys@transformshift{2.055857in}{1.570616in}%
\pgfsys@useobject{currentmarker}{}%
\end{pgfscope}%
\begin{pgfscope}%
\pgfsys@transformshift{1.436347in}{3.105986in}%
\pgfsys@useobject{currentmarker}{}%
\end{pgfscope}%
\begin{pgfscope}%
\pgfsys@transformshift{4.726454in}{3.149538in}%
\pgfsys@useobject{currentmarker}{}%
\end{pgfscope}%
\begin{pgfscope}%
\pgfsys@transformshift{4.935666in}{4.874458in}%
\pgfsys@useobject{currentmarker}{}%
\end{pgfscope}%
\begin{pgfscope}%
\pgfsys@transformshift{4.924599in}{2.981236in}%
\pgfsys@useobject{currentmarker}{}%
\end{pgfscope}%
\begin{pgfscope}%
\pgfsys@transformshift{5.646671in}{0.236834in}%
\pgfsys@useobject{currentmarker}{}%
\end{pgfscope}%
\begin{pgfscope}%
\pgfsys@transformshift{2.508063in}{5.385651in}%
\pgfsys@useobject{currentmarker}{}%
\end{pgfscope}%
\begin{pgfscope}%
\pgfsys@transformshift{4.318884in}{4.682130in}%
\pgfsys@useobject{currentmarker}{}%
\end{pgfscope}%
\begin{pgfscope}%
\pgfsys@transformshift{3.453664in}{5.476330in}%
\pgfsys@useobject{currentmarker}{}%
\end{pgfscope}%
\begin{pgfscope}%
\pgfsys@transformshift{6.008581in}{1.218234in}%
\pgfsys@useobject{currentmarker}{}%
\end{pgfscope}%
\begin{pgfscope}%
\pgfsys@transformshift{2.172534in}{1.988399in}%
\pgfsys@useobject{currentmarker}{}%
\end{pgfscope}%
\begin{pgfscope}%
\pgfsys@transformshift{4.574985in}{2.093693in}%
\pgfsys@useobject{currentmarker}{}%
\end{pgfscope}%
\begin{pgfscope}%
\pgfsys@transformshift{2.821949in}{0.915201in}%
\pgfsys@useobject{currentmarker}{}%
\end{pgfscope}%
\begin{pgfscope}%
\pgfsys@transformshift{2.291114in}{1.022960in}%
\pgfsys@useobject{currentmarker}{}%
\end{pgfscope}%
\begin{pgfscope}%
\pgfsys@transformshift{3.249111in}{3.451440in}%
\pgfsys@useobject{currentmarker}{}%
\end{pgfscope}%
\begin{pgfscope}%
\pgfsys@transformshift{3.418379in}{4.944464in}%
\pgfsys@useobject{currentmarker}{}%
\end{pgfscope}%
\begin{pgfscope}%
\pgfsys@transformshift{6.721855in}{4.742230in}%
\pgfsys@useobject{currentmarker}{}%
\end{pgfscope}%
\begin{pgfscope}%
\pgfsys@transformshift{3.195830in}{2.139203in}%
\pgfsys@useobject{currentmarker}{}%
\end{pgfscope}%
\begin{pgfscope}%
\pgfsys@transformshift{4.613722in}{3.598510in}%
\pgfsys@useobject{currentmarker}{}%
\end{pgfscope}%
\begin{pgfscope}%
\pgfsys@transformshift{4.571450in}{1.755305in}%
\pgfsys@useobject{currentmarker}{}%
\end{pgfscope}%
\begin{pgfscope}%
\pgfsys@transformshift{2.687764in}{4.679773in}%
\pgfsys@useobject{currentmarker}{}%
\end{pgfscope}%
\begin{pgfscope}%
\pgfsys@transformshift{4.919162in}{1.245477in}%
\pgfsys@useobject{currentmarker}{}%
\end{pgfscope}%
\begin{pgfscope}%
\pgfsys@transformshift{5.478505in}{3.023083in}%
\pgfsys@useobject{currentmarker}{}%
\end{pgfscope}%
\begin{pgfscope}%
\pgfsys@transformshift{5.286372in}{2.644489in}%
\pgfsys@useobject{currentmarker}{}%
\end{pgfscope}%
\begin{pgfscope}%
\pgfsys@transformshift{1.932215in}{3.910854in}%
\pgfsys@useobject{currentmarker}{}%
\end{pgfscope}%
\begin{pgfscope}%
\pgfsys@transformshift{5.826780in}{3.067734in}%
\pgfsys@useobject{currentmarker}{}%
\end{pgfscope}%
\begin{pgfscope}%
\pgfsys@transformshift{6.173009in}{5.541503in}%
\pgfsys@useobject{currentmarker}{}%
\end{pgfscope}%
\begin{pgfscope}%
\pgfsys@transformshift{4.563118in}{4.624819in}%
\pgfsys@useobject{currentmarker}{}%
\end{pgfscope}%
\begin{pgfscope}%
\pgfsys@transformshift{6.081851in}{5.283892in}%
\pgfsys@useobject{currentmarker}{}%
\end{pgfscope}%
\begin{pgfscope}%
\pgfsys@transformshift{3.816489in}{4.911076in}%
\pgfsys@useobject{currentmarker}{}%
\end{pgfscope}%
\begin{pgfscope}%
\pgfsys@transformshift{2.570919in}{0.951181in}%
\pgfsys@useobject{currentmarker}{}%
\end{pgfscope}%
\begin{pgfscope}%
\pgfsys@transformshift{6.077710in}{2.312357in}%
\pgfsys@useobject{currentmarker}{}%
\end{pgfscope}%
\begin{pgfscope}%
\pgfsys@transformshift{3.339419in}{1.969108in}%
\pgfsys@useobject{currentmarker}{}%
\end{pgfscope}%
\begin{pgfscope}%
\pgfsys@transformshift{6.293994in}{0.283070in}%
\pgfsys@useobject{currentmarker}{}%
\end{pgfscope}%
\begin{pgfscope}%
\pgfsys@transformshift{3.755276in}{2.885846in}%
\pgfsys@useobject{currentmarker}{}%
\end{pgfscope}%
\begin{pgfscope}%
\pgfsys@transformshift{6.775032in}{5.131919in}%
\pgfsys@useobject{currentmarker}{}%
\end{pgfscope}%
\begin{pgfscope}%
\pgfsys@transformshift{5.354899in}{1.732918in}%
\pgfsys@useobject{currentmarker}{}%
\end{pgfscope}%
\begin{pgfscope}%
\pgfsys@transformshift{5.345581in}{2.511129in}%
\pgfsys@useobject{currentmarker}{}%
\end{pgfscope}%
\begin{pgfscope}%
\pgfsys@transformshift{2.387974in}{3.463126in}%
\pgfsys@useobject{currentmarker}{}%
\end{pgfscope}%
\begin{pgfscope}%
\pgfsys@transformshift{6.222673in}{3.024721in}%
\pgfsys@useobject{currentmarker}{}%
\end{pgfscope}%
\begin{pgfscope}%
\pgfsys@transformshift{5.708974in}{3.028273in}%
\pgfsys@useobject{currentmarker}{}%
\end{pgfscope}%
\begin{pgfscope}%
\pgfsys@transformshift{2.121233in}{4.570489in}%
\pgfsys@useobject{currentmarker}{}%
\end{pgfscope}%
\begin{pgfscope}%
\pgfsys@transformshift{6.588359in}{2.987668in}%
\pgfsys@useobject{currentmarker}{}%
\end{pgfscope}%
\begin{pgfscope}%
\pgfsys@transformshift{0.660766in}{3.380144in}%
\pgfsys@useobject{currentmarker}{}%
\end{pgfscope}%
\begin{pgfscope}%
\pgfsys@transformshift{0.638422in}{0.377692in}%
\pgfsys@useobject{currentmarker}{}%
\end{pgfscope}%
\begin{pgfscope}%
\pgfsys@transformshift{4.883686in}{2.024349in}%
\pgfsys@useobject{currentmarker}{}%
\end{pgfscope}%
\begin{pgfscope}%
\pgfsys@transformshift{2.740257in}{3.626682in}%
\pgfsys@useobject{currentmarker}{}%
\end{pgfscope}%
\begin{pgfscope}%
\pgfsys@transformshift{0.959435in}{5.629592in}%
\pgfsys@useobject{currentmarker}{}%
\end{pgfscope}%
\begin{pgfscope}%
\pgfsys@transformshift{1.541526in}{4.402371in}%
\pgfsys@useobject{currentmarker}{}%
\end{pgfscope}%
\begin{pgfscope}%
\pgfsys@transformshift{6.276516in}{3.780366in}%
\pgfsys@useobject{currentmarker}{}%
\end{pgfscope}%
\begin{pgfscope}%
\pgfsys@transformshift{4.656231in}{5.099656in}%
\pgfsys@useobject{currentmarker}{}%
\end{pgfscope}%
\begin{pgfscope}%
\pgfsys@transformshift{1.403655in}{3.799710in}%
\pgfsys@useobject{currentmarker}{}%
\end{pgfscope}%
\begin{pgfscope}%
\pgfsys@transformshift{3.077661in}{4.281679in}%
\pgfsys@useobject{currentmarker}{}%
\end{pgfscope}%
\begin{pgfscope}%
\pgfsys@transformshift{3.978488in}{3.094930in}%
\pgfsys@useobject{currentmarker}{}%
\end{pgfscope}%
\begin{pgfscope}%
\pgfsys@transformshift{6.308239in}{1.314485in}%
\pgfsys@useobject{currentmarker}{}%
\end{pgfscope}%
\begin{pgfscope}%
\pgfsys@transformshift{5.665058in}{0.916599in}%
\pgfsys@useobject{currentmarker}{}%
\end{pgfscope}%
\begin{pgfscope}%
\pgfsys@transformshift{4.307675in}{1.537587in}%
\pgfsys@useobject{currentmarker}{}%
\end{pgfscope}%
\begin{pgfscope}%
\pgfsys@transformshift{1.783796in}{5.632632in}%
\pgfsys@useobject{currentmarker}{}%
\end{pgfscope}%
\begin{pgfscope}%
\pgfsys@transformshift{4.003026in}{4.943363in}%
\pgfsys@useobject{currentmarker}{}%
\end{pgfscope}%
\begin{pgfscope}%
\pgfsys@transformshift{5.169143in}{4.312399in}%
\pgfsys@useobject{currentmarker}{}%
\end{pgfscope}%
\begin{pgfscope}%
\pgfsys@transformshift{4.726654in}{2.312508in}%
\pgfsys@useobject{currentmarker}{}%
\end{pgfscope}%
\begin{pgfscope}%
\pgfsys@transformshift{2.266604in}{2.484720in}%
\pgfsys@useobject{currentmarker}{}%
\end{pgfscope}%
\begin{pgfscope}%
\pgfsys@transformshift{4.237641in}{4.877722in}%
\pgfsys@useobject{currentmarker}{}%
\end{pgfscope}%
\begin{pgfscope}%
\pgfsys@transformshift{6.127742in}{2.376750in}%
\pgfsys@useobject{currentmarker}{}%
\end{pgfscope}%
\begin{pgfscope}%
\pgfsys@transformshift{0.685222in}{2.435419in}%
\pgfsys@useobject{currentmarker}{}%
\end{pgfscope}%
\begin{pgfscope}%
\pgfsys@transformshift{3.995462in}{2.932399in}%
\pgfsys@useobject{currentmarker}{}%
\end{pgfscope}%
\begin{pgfscope}%
\pgfsys@transformshift{1.199929in}{4.919594in}%
\pgfsys@useobject{currentmarker}{}%
\end{pgfscope}%
\begin{pgfscope}%
\pgfsys@transformshift{4.786246in}{5.141406in}%
\pgfsys@useobject{currentmarker}{}%
\end{pgfscope}%
\begin{pgfscope}%
\pgfsys@transformshift{4.097274in}{4.381858in}%
\pgfsys@useobject{currentmarker}{}%
\end{pgfscope}%
\begin{pgfscope}%
\pgfsys@transformshift{1.456272in}{2.142124in}%
\pgfsys@useobject{currentmarker}{}%
\end{pgfscope}%
\begin{pgfscope}%
\pgfsys@transformshift{0.983666in}{3.334171in}%
\pgfsys@useobject{currentmarker}{}%
\end{pgfscope}%
\begin{pgfscope}%
\pgfsys@transformshift{4.540991in}{1.982214in}%
\pgfsys@useobject{currentmarker}{}%
\end{pgfscope}%
\begin{pgfscope}%
\pgfsys@transformshift{2.887081in}{2.480684in}%
\pgfsys@useobject{currentmarker}{}%
\end{pgfscope}%
\begin{pgfscope}%
\pgfsys@transformshift{1.974107in}{4.422461in}%
\pgfsys@useobject{currentmarker}{}%
\end{pgfscope}%
\begin{pgfscope}%
\pgfsys@transformshift{3.742329in}{1.808660in}%
\pgfsys@useobject{currentmarker}{}%
\end{pgfscope}%
\begin{pgfscope}%
\pgfsys@transformshift{5.625123in}{5.136882in}%
\pgfsys@useobject{currentmarker}{}%
\end{pgfscope}%
\begin{pgfscope}%
\pgfsys@transformshift{4.970478in}{2.101681in}%
\pgfsys@useobject{currentmarker}{}%
\end{pgfscope}%
\begin{pgfscope}%
\pgfsys@transformshift{4.449601in}{2.118021in}%
\pgfsys@useobject{currentmarker}{}%
\end{pgfscope}%
\begin{pgfscope}%
\pgfsys@transformshift{4.545370in}{1.625933in}%
\pgfsys@useobject{currentmarker}{}%
\end{pgfscope}%
\begin{pgfscope}%
\pgfsys@transformshift{1.701117in}{3.302885in}%
\pgfsys@useobject{currentmarker}{}%
\end{pgfscope}%
\begin{pgfscope}%
\pgfsys@transformshift{3.020819in}{0.698542in}%
\pgfsys@useobject{currentmarker}{}%
\end{pgfscope}%
\begin{pgfscope}%
\pgfsys@transformshift{2.882987in}{3.637747in}%
\pgfsys@useobject{currentmarker}{}%
\end{pgfscope}%
\begin{pgfscope}%
\pgfsys@transformshift{6.173915in}{5.168388in}%
\pgfsys@useobject{currentmarker}{}%
\end{pgfscope}%
\begin{pgfscope}%
\pgfsys@transformshift{6.016951in}{5.527557in}%
\pgfsys@useobject{currentmarker}{}%
\end{pgfscope}%
\begin{pgfscope}%
\pgfsys@transformshift{2.444148in}{5.354277in}%
\pgfsys@useobject{currentmarker}{}%
\end{pgfscope}%
\begin{pgfscope}%
\pgfsys@transformshift{3.541478in}{2.148269in}%
\pgfsys@useobject{currentmarker}{}%
\end{pgfscope}%
\begin{pgfscope}%
\pgfsys@transformshift{2.937770in}{5.471810in}%
\pgfsys@useobject{currentmarker}{}%
\end{pgfscope}%
\begin{pgfscope}%
\pgfsys@transformshift{6.844005in}{5.483905in}%
\pgfsys@useobject{currentmarker}{}%
\end{pgfscope}%
\begin{pgfscope}%
\pgfsys@transformshift{5.545303in}{0.229241in}%
\pgfsys@useobject{currentmarker}{}%
\end{pgfscope}%
\begin{pgfscope}%
\pgfsys@transformshift{2.322194in}{2.891862in}%
\pgfsys@useobject{currentmarker}{}%
\end{pgfscope}%
\begin{pgfscope}%
\pgfsys@transformshift{2.813031in}{1.667116in}%
\pgfsys@useobject{currentmarker}{}%
\end{pgfscope}%
\begin{pgfscope}%
\pgfsys@transformshift{6.512908in}{2.472316in}%
\pgfsys@useobject{currentmarker}{}%
\end{pgfscope}%
\begin{pgfscope}%
\pgfsys@transformshift{2.649575in}{3.713479in}%
\pgfsys@useobject{currentmarker}{}%
\end{pgfscope}%
\begin{pgfscope}%
\pgfsys@transformshift{2.685982in}{5.150939in}%
\pgfsys@useobject{currentmarker}{}%
\end{pgfscope}%
\begin{pgfscope}%
\pgfsys@transformshift{4.611924in}{2.043274in}%
\pgfsys@useobject{currentmarker}{}%
\end{pgfscope}%
\begin{pgfscope}%
\pgfsys@transformshift{1.008702in}{0.812180in}%
\pgfsys@useobject{currentmarker}{}%
\end{pgfscope}%
\begin{pgfscope}%
\pgfsys@transformshift{4.631452in}{4.610566in}%
\pgfsys@useobject{currentmarker}{}%
\end{pgfscope}%
\begin{pgfscope}%
\pgfsys@transformshift{3.566628in}{0.476345in}%
\pgfsys@useobject{currentmarker}{}%
\end{pgfscope}%
\begin{pgfscope}%
\pgfsys@transformshift{3.859542in}{1.606297in}%
\pgfsys@useobject{currentmarker}{}%
\end{pgfscope}%
\begin{pgfscope}%
\pgfsys@transformshift{3.126543in}{3.326074in}%
\pgfsys@useobject{currentmarker}{}%
\end{pgfscope}%
\begin{pgfscope}%
\pgfsys@transformshift{2.071983in}{0.819441in}%
\pgfsys@useobject{currentmarker}{}%
\end{pgfscope}%
\begin{pgfscope}%
\pgfsys@transformshift{2.557020in}{3.165861in}%
\pgfsys@useobject{currentmarker}{}%
\end{pgfscope}%
\begin{pgfscope}%
\pgfsys@transformshift{4.660763in}{3.961684in}%
\pgfsys@useobject{currentmarker}{}%
\end{pgfscope}%
\begin{pgfscope}%
\pgfsys@transformshift{6.095763in}{0.327096in}%
\pgfsys@useobject{currentmarker}{}%
\end{pgfscope}%
\begin{pgfscope}%
\pgfsys@transformshift{4.777143in}{5.321530in}%
\pgfsys@useobject{currentmarker}{}%
\end{pgfscope}%
\begin{pgfscope}%
\pgfsys@transformshift{5.019800in}{2.678550in}%
\pgfsys@useobject{currentmarker}{}%
\end{pgfscope}%
\begin{pgfscope}%
\pgfsys@transformshift{1.096184in}{4.373993in}%
\pgfsys@useobject{currentmarker}{}%
\end{pgfscope}%
\begin{pgfscope}%
\pgfsys@transformshift{4.817608in}{1.764063in}%
\pgfsys@useobject{currentmarker}{}%
\end{pgfscope}%
\begin{pgfscope}%
\pgfsys@transformshift{2.388446in}{2.056302in}%
\pgfsys@useobject{currentmarker}{}%
\end{pgfscope}%
\begin{pgfscope}%
\pgfsys@transformshift{6.638673in}{1.337319in}%
\pgfsys@useobject{currentmarker}{}%
\end{pgfscope}%
\begin{pgfscope}%
\pgfsys@transformshift{3.917515in}{4.306589in}%
\pgfsys@useobject{currentmarker}{}%
\end{pgfscope}%
\begin{pgfscope}%
\pgfsys@transformshift{5.262884in}{2.769688in}%
\pgfsys@useobject{currentmarker}{}%
\end{pgfscope}%
\begin{pgfscope}%
\pgfsys@transformshift{2.540800in}{2.195987in}%
\pgfsys@useobject{currentmarker}{}%
\end{pgfscope}%
\begin{pgfscope}%
\pgfsys@transformshift{2.278667in}{4.273157in}%
\pgfsys@useobject{currentmarker}{}%
\end{pgfscope}%
\begin{pgfscope}%
\pgfsys@transformshift{1.600388in}{5.157514in}%
\pgfsys@useobject{currentmarker}{}%
\end{pgfscope}%
\begin{pgfscope}%
\pgfsys@transformshift{1.256587in}{4.185291in}%
\pgfsys@useobject{currentmarker}{}%
\end{pgfscope}%
\begin{pgfscope}%
\pgfsys@transformshift{1.675103in}{2.367521in}%
\pgfsys@useobject{currentmarker}{}%
\end{pgfscope}%
\begin{pgfscope}%
\pgfsys@transformshift{1.140633in}{1.975087in}%
\pgfsys@useobject{currentmarker}{}%
\end{pgfscope}%
\begin{pgfscope}%
\pgfsys@transformshift{6.150431in}{1.526901in}%
\pgfsys@useobject{currentmarker}{}%
\end{pgfscope}%
\begin{pgfscope}%
\pgfsys@transformshift{2.754860in}{3.534710in}%
\pgfsys@useobject{currentmarker}{}%
\end{pgfscope}%
\begin{pgfscope}%
\pgfsys@transformshift{5.688736in}{3.249390in}%
\pgfsys@useobject{currentmarker}{}%
\end{pgfscope}%
\begin{pgfscope}%
\pgfsys@transformshift{3.462161in}{1.070390in}%
\pgfsys@useobject{currentmarker}{}%
\end{pgfscope}%
\begin{pgfscope}%
\pgfsys@transformshift{4.573148in}{2.042096in}%
\pgfsys@useobject{currentmarker}{}%
\end{pgfscope}%
\begin{pgfscope}%
\pgfsys@transformshift{1.477976in}{1.126700in}%
\pgfsys@useobject{currentmarker}{}%
\end{pgfscope}%
\begin{pgfscope}%
\pgfsys@transformshift{2.059752in}{5.603854in}%
\pgfsys@useobject{currentmarker}{}%
\end{pgfscope}%
\begin{pgfscope}%
\pgfsys@transformshift{3.761870in}{2.737511in}%
\pgfsys@useobject{currentmarker}{}%
\end{pgfscope}%
\begin{pgfscope}%
\pgfsys@transformshift{2.387502in}{2.618844in}%
\pgfsys@useobject{currentmarker}{}%
\end{pgfscope}%
\begin{pgfscope}%
\pgfsys@transformshift{1.177883in}{1.628458in}%
\pgfsys@useobject{currentmarker}{}%
\end{pgfscope}%
\begin{pgfscope}%
\pgfsys@transformshift{1.514227in}{1.497437in}%
\pgfsys@useobject{currentmarker}{}%
\end{pgfscope}%
\begin{pgfscope}%
\pgfsys@transformshift{1.995204in}{3.074388in}%
\pgfsys@useobject{currentmarker}{}%
\end{pgfscope}%
\begin{pgfscope}%
\pgfsys@transformshift{2.698425in}{5.229061in}%
\pgfsys@useobject{currentmarker}{}%
\end{pgfscope}%
\begin{pgfscope}%
\pgfsys@transformshift{2.848675in}{1.640935in}%
\pgfsys@useobject{currentmarker}{}%
\end{pgfscope}%
\begin{pgfscope}%
\pgfsys@transformshift{4.312361in}{4.661572in}%
\pgfsys@useobject{currentmarker}{}%
\end{pgfscope}%
\begin{pgfscope}%
\pgfsys@transformshift{6.711979in}{4.754123in}%
\pgfsys@useobject{currentmarker}{}%
\end{pgfscope}%
\begin{pgfscope}%
\pgfsys@transformshift{4.555861in}{1.916473in}%
\pgfsys@useobject{currentmarker}{}%
\end{pgfscope}%
\begin{pgfscope}%
\pgfsys@transformshift{4.456666in}{4.816556in}%
\pgfsys@useobject{currentmarker}{}%
\end{pgfscope}%
\begin{pgfscope}%
\pgfsys@transformshift{1.955766in}{4.913128in}%
\pgfsys@useobject{currentmarker}{}%
\end{pgfscope}%
\begin{pgfscope}%
\pgfsys@transformshift{4.702284in}{0.728509in}%
\pgfsys@useobject{currentmarker}{}%
\end{pgfscope}%
\begin{pgfscope}%
\pgfsys@transformshift{5.486862in}{1.687828in}%
\pgfsys@useobject{currentmarker}{}%
\end{pgfscope}%
\begin{pgfscope}%
\pgfsys@transformshift{6.662081in}{3.638517in}%
\pgfsys@useobject{currentmarker}{}%
\end{pgfscope}%
\begin{pgfscope}%
\pgfsys@transformshift{2.027746in}{5.261369in}%
\pgfsys@useobject{currentmarker}{}%
\end{pgfscope}%
\begin{pgfscope}%
\pgfsys@transformshift{3.231916in}{0.236334in}%
\pgfsys@useobject{currentmarker}{}%
\end{pgfscope}%
\begin{pgfscope}%
\pgfsys@transformshift{6.714576in}{3.312566in}%
\pgfsys@useobject{currentmarker}{}%
\end{pgfscope}%
\begin{pgfscope}%
\pgfsys@transformshift{2.461026in}{2.860383in}%
\pgfsys@useobject{currentmarker}{}%
\end{pgfscope}%
\begin{pgfscope}%
\pgfsys@transformshift{4.920990in}{3.284783in}%
\pgfsys@useobject{currentmarker}{}%
\end{pgfscope}%
\begin{pgfscope}%
\pgfsys@transformshift{0.633466in}{3.601195in}%
\pgfsys@useobject{currentmarker}{}%
\end{pgfscope}%
\begin{pgfscope}%
\pgfsys@transformshift{6.410720in}{4.771303in}%
\pgfsys@useobject{currentmarker}{}%
\end{pgfscope}%
\begin{pgfscope}%
\pgfsys@transformshift{0.737807in}{5.192324in}%
\pgfsys@useobject{currentmarker}{}%
\end{pgfscope}%
\begin{pgfscope}%
\pgfsys@transformshift{3.030773in}{1.004290in}%
\pgfsys@useobject{currentmarker}{}%
\end{pgfscope}%
\begin{pgfscope}%
\pgfsys@transformshift{5.721762in}{3.184300in}%
\pgfsys@useobject{currentmarker}{}%
\end{pgfscope}%
\begin{pgfscope}%
\pgfsys@transformshift{6.672696in}{1.345201in}%
\pgfsys@useobject{currentmarker}{}%
\end{pgfscope}%
\begin{pgfscope}%
\pgfsys@transformshift{4.262420in}{2.709797in}%
\pgfsys@useobject{currentmarker}{}%
\end{pgfscope}%
\begin{pgfscope}%
\pgfsys@transformshift{0.910133in}{1.575668in}%
\pgfsys@useobject{currentmarker}{}%
\end{pgfscope}%
\begin{pgfscope}%
\pgfsys@transformshift{3.702260in}{3.494981in}%
\pgfsys@useobject{currentmarker}{}%
\end{pgfscope}%
\begin{pgfscope}%
\pgfsys@transformshift{3.848048in}{4.539957in}%
\pgfsys@useobject{currentmarker}{}%
\end{pgfscope}%
\begin{pgfscope}%
\pgfsys@transformshift{0.823245in}{1.736091in}%
\pgfsys@useobject{currentmarker}{}%
\end{pgfscope}%
\begin{pgfscope}%
\pgfsys@transformshift{1.944397in}{5.536021in}%
\pgfsys@useobject{currentmarker}{}%
\end{pgfscope}%
\begin{pgfscope}%
\pgfsys@transformshift{5.748313in}{2.938554in}%
\pgfsys@useobject{currentmarker}{}%
\end{pgfscope}%
\begin{pgfscope}%
\pgfsys@transformshift{0.850308in}{2.263331in}%
\pgfsys@useobject{currentmarker}{}%
\end{pgfscope}%
\begin{pgfscope}%
\pgfsys@transformshift{5.533335in}{2.051328in}%
\pgfsys@useobject{currentmarker}{}%
\end{pgfscope}%
\begin{pgfscope}%
\pgfsys@transformshift{4.920908in}{1.178994in}%
\pgfsys@useobject{currentmarker}{}%
\end{pgfscope}%
\begin{pgfscope}%
\pgfsys@transformshift{3.078704in}{4.308053in}%
\pgfsys@useobject{currentmarker}{}%
\end{pgfscope}%
\begin{pgfscope}%
\pgfsys@transformshift{2.622395in}{4.775882in}%
\pgfsys@useobject{currentmarker}{}%
\end{pgfscope}%
\begin{pgfscope}%
\pgfsys@transformshift{3.070403in}{0.511672in}%
\pgfsys@useobject{currentmarker}{}%
\end{pgfscope}%
\begin{pgfscope}%
\pgfsys@transformshift{2.010159in}{4.587327in}%
\pgfsys@useobject{currentmarker}{}%
\end{pgfscope}%
\begin{pgfscope}%
\pgfsys@transformshift{4.615435in}{1.512341in}%
\pgfsys@useobject{currentmarker}{}%
\end{pgfscope}%
\begin{pgfscope}%
\pgfsys@transformshift{3.283848in}{1.923445in}%
\pgfsys@useobject{currentmarker}{}%
\end{pgfscope}%
\begin{pgfscope}%
\pgfsys@transformshift{3.516931in}{2.372846in}%
\pgfsys@useobject{currentmarker}{}%
\end{pgfscope}%
\begin{pgfscope}%
\pgfsys@transformshift{2.730136in}{4.537345in}%
\pgfsys@useobject{currentmarker}{}%
\end{pgfscope}%
\begin{pgfscope}%
\pgfsys@transformshift{4.197092in}{1.232488in}%
\pgfsys@useobject{currentmarker}{}%
\end{pgfscope}%
\begin{pgfscope}%
\pgfsys@transformshift{1.084735in}{5.441939in}%
\pgfsys@useobject{currentmarker}{}%
\end{pgfscope}%
\begin{pgfscope}%
\pgfsys@transformshift{5.055315in}{4.079259in}%
\pgfsys@useobject{currentmarker}{}%
\end{pgfscope}%
\begin{pgfscope}%
\pgfsys@transformshift{4.998613in}{2.542947in}%
\pgfsys@useobject{currentmarker}{}%
\end{pgfscope}%
\begin{pgfscope}%
\pgfsys@transformshift{3.753514in}{3.592069in}%
\pgfsys@useobject{currentmarker}{}%
\end{pgfscope}%
\begin{pgfscope}%
\pgfsys@transformshift{5.009085in}{1.180972in}%
\pgfsys@useobject{currentmarker}{}%
\end{pgfscope}%
\begin{pgfscope}%
\pgfsys@transformshift{3.916301in}{0.346932in}%
\pgfsys@useobject{currentmarker}{}%
\end{pgfscope}%
\begin{pgfscope}%
\pgfsys@transformshift{3.295930in}{0.419960in}%
\pgfsys@useobject{currentmarker}{}%
\end{pgfscope}%
\begin{pgfscope}%
\pgfsys@transformshift{2.941804in}{4.369959in}%
\pgfsys@useobject{currentmarker}{}%
\end{pgfscope}%
\begin{pgfscope}%
\pgfsys@transformshift{4.725180in}{3.528098in}%
\pgfsys@useobject{currentmarker}{}%
\end{pgfscope}%
\begin{pgfscope}%
\pgfsys@transformshift{2.788343in}{2.740159in}%
\pgfsys@useobject{currentmarker}{}%
\end{pgfscope}%
\begin{pgfscope}%
\pgfsys@transformshift{1.878187in}{1.349408in}%
\pgfsys@useobject{currentmarker}{}%
\end{pgfscope}%
\begin{pgfscope}%
\pgfsys@transformshift{1.664014in}{1.692202in}%
\pgfsys@useobject{currentmarker}{}%
\end{pgfscope}%
\begin{pgfscope}%
\pgfsys@transformshift{6.188220in}{4.797263in}%
\pgfsys@useobject{currentmarker}{}%
\end{pgfscope}%
\begin{pgfscope}%
\pgfsys@transformshift{6.450629in}{4.934804in}%
\pgfsys@useobject{currentmarker}{}%
\end{pgfscope}%
\begin{pgfscope}%
\pgfsys@transformshift{1.258554in}{4.447746in}%
\pgfsys@useobject{currentmarker}{}%
\end{pgfscope}%
\begin{pgfscope}%
\pgfsys@transformshift{3.528620in}{4.236590in}%
\pgfsys@useobject{currentmarker}{}%
\end{pgfscope}%
\begin{pgfscope}%
\pgfsys@transformshift{5.959390in}{4.101922in}%
\pgfsys@useobject{currentmarker}{}%
\end{pgfscope}%
\begin{pgfscope}%
\pgfsys@transformshift{5.097255in}{5.072753in}%
\pgfsys@useobject{currentmarker}{}%
\end{pgfscope}%
\begin{pgfscope}%
\pgfsys@transformshift{2.059371in}{1.976888in}%
\pgfsys@useobject{currentmarker}{}%
\end{pgfscope}%
\begin{pgfscope}%
\pgfsys@transformshift{6.460961in}{0.415587in}%
\pgfsys@useobject{currentmarker}{}%
\end{pgfscope}%
\begin{pgfscope}%
\pgfsys@transformshift{3.758362in}{5.370915in}%
\pgfsys@useobject{currentmarker}{}%
\end{pgfscope}%
\begin{pgfscope}%
\pgfsys@transformshift{3.653366in}{1.443312in}%
\pgfsys@useobject{currentmarker}{}%
\end{pgfscope}%
\begin{pgfscope}%
\pgfsys@transformshift{6.462242in}{4.811799in}%
\pgfsys@useobject{currentmarker}{}%
\end{pgfscope}%
\begin{pgfscope}%
\pgfsys@transformshift{1.189272in}{5.328609in}%
\pgfsys@useobject{currentmarker}{}%
\end{pgfscope}%
\begin{pgfscope}%
\pgfsys@transformshift{0.783360in}{3.137870in}%
\pgfsys@useobject{currentmarker}{}%
\end{pgfscope}%
\begin{pgfscope}%
\pgfsys@transformshift{3.754773in}{4.837084in}%
\pgfsys@useobject{currentmarker}{}%
\end{pgfscope}%
\begin{pgfscope}%
\pgfsys@transformshift{5.897790in}{3.739800in}%
\pgfsys@useobject{currentmarker}{}%
\end{pgfscope}%
\begin{pgfscope}%
\pgfsys@transformshift{3.670624in}{1.499253in}%
\pgfsys@useobject{currentmarker}{}%
\end{pgfscope}%
\begin{pgfscope}%
\pgfsys@transformshift{2.928574in}{1.683345in}%
\pgfsys@useobject{currentmarker}{}%
\end{pgfscope}%
\begin{pgfscope}%
\pgfsys@transformshift{4.163611in}{2.548539in}%
\pgfsys@useobject{currentmarker}{}%
\end{pgfscope}%
\begin{pgfscope}%
\pgfsys@transformshift{2.994857in}{2.965365in}%
\pgfsys@useobject{currentmarker}{}%
\end{pgfscope}%
\begin{pgfscope}%
\pgfsys@transformshift{1.687544in}{3.910411in}%
\pgfsys@useobject{currentmarker}{}%
\end{pgfscope}%
\begin{pgfscope}%
\pgfsys@transformshift{4.119993in}{3.466925in}%
\pgfsys@useobject{currentmarker}{}%
\end{pgfscope}%
\begin{pgfscope}%
\pgfsys@transformshift{2.882721in}{1.955034in}%
\pgfsys@useobject{currentmarker}{}%
\end{pgfscope}%
\begin{pgfscope}%
\pgfsys@transformshift{3.997017in}{3.791505in}%
\pgfsys@useobject{currentmarker}{}%
\end{pgfscope}%
\begin{pgfscope}%
\pgfsys@transformshift{6.666083in}{4.144635in}%
\pgfsys@useobject{currentmarker}{}%
\end{pgfscope}%
\begin{pgfscope}%
\pgfsys@transformshift{1.630066in}{2.230214in}%
\pgfsys@useobject{currentmarker}{}%
\end{pgfscope}%
\begin{pgfscope}%
\pgfsys@transformshift{3.940720in}{0.753932in}%
\pgfsys@useobject{currentmarker}{}%
\end{pgfscope}%
\begin{pgfscope}%
\pgfsys@transformshift{0.671154in}{1.163744in}%
\pgfsys@useobject{currentmarker}{}%
\end{pgfscope}%
\begin{pgfscope}%
\pgfsys@transformshift{3.659008in}{2.878941in}%
\pgfsys@useobject{currentmarker}{}%
\end{pgfscope}%
\begin{pgfscope}%
\pgfsys@transformshift{3.796370in}{0.920318in}%
\pgfsys@useobject{currentmarker}{}%
\end{pgfscope}%
\begin{pgfscope}%
\pgfsys@transformshift{5.193854in}{3.388259in}%
\pgfsys@useobject{currentmarker}{}%
\end{pgfscope}%
\begin{pgfscope}%
\pgfsys@transformshift{1.879139in}{4.064168in}%
\pgfsys@useobject{currentmarker}{}%
\end{pgfscope}%
\begin{pgfscope}%
\pgfsys@transformshift{5.683591in}{4.909548in}%
\pgfsys@useobject{currentmarker}{}%
\end{pgfscope}%
\begin{pgfscope}%
\pgfsys@transformshift{6.267351in}{4.062387in}%
\pgfsys@useobject{currentmarker}{}%
\end{pgfscope}%
\begin{pgfscope}%
\pgfsys@transformshift{5.605500in}{0.456673in}%
\pgfsys@useobject{currentmarker}{}%
\end{pgfscope}%
\begin{pgfscope}%
\pgfsys@transformshift{5.238620in}{2.651835in}%
\pgfsys@useobject{currentmarker}{}%
\end{pgfscope}%
\begin{pgfscope}%
\pgfsys@transformshift{1.665680in}{2.554861in}%
\pgfsys@useobject{currentmarker}{}%
\end{pgfscope}%
\begin{pgfscope}%
\pgfsys@transformshift{3.973932in}{3.794709in}%
\pgfsys@useobject{currentmarker}{}%
\end{pgfscope}%
\begin{pgfscope}%
\pgfsys@transformshift{1.749542in}{0.304289in}%
\pgfsys@useobject{currentmarker}{}%
\end{pgfscope}%
\begin{pgfscope}%
\pgfsys@transformshift{5.210576in}{5.541205in}%
\pgfsys@useobject{currentmarker}{}%
\end{pgfscope}%
\begin{pgfscope}%
\pgfsys@transformshift{3.657671in}{3.972159in}%
\pgfsys@useobject{currentmarker}{}%
\end{pgfscope}%
\begin{pgfscope}%
\pgfsys@transformshift{4.365775in}{1.854875in}%
\pgfsys@useobject{currentmarker}{}%
\end{pgfscope}%
\begin{pgfscope}%
\pgfsys@transformshift{4.581518in}{3.919286in}%
\pgfsys@useobject{currentmarker}{}%
\end{pgfscope}%
\begin{pgfscope}%
\pgfsys@transformshift{6.044030in}{3.977402in}%
\pgfsys@useobject{currentmarker}{}%
\end{pgfscope}%
\begin{pgfscope}%
\pgfsys@transformshift{3.257253in}{4.176009in}%
\pgfsys@useobject{currentmarker}{}%
\end{pgfscope}%
\begin{pgfscope}%
\pgfsys@transformshift{0.728425in}{2.306781in}%
\pgfsys@useobject{currentmarker}{}%
\end{pgfscope}%
\begin{pgfscope}%
\pgfsys@transformshift{0.699212in}{0.644844in}%
\pgfsys@useobject{currentmarker}{}%
\end{pgfscope}%
\begin{pgfscope}%
\pgfsys@transformshift{2.824735in}{3.322079in}%
\pgfsys@useobject{currentmarker}{}%
\end{pgfscope}%
\begin{pgfscope}%
\pgfsys@transformshift{5.575909in}{4.168814in}%
\pgfsys@useobject{currentmarker}{}%
\end{pgfscope}%
\begin{pgfscope}%
\pgfsys@transformshift{2.694993in}{0.745092in}%
\pgfsys@useobject{currentmarker}{}%
\end{pgfscope}%
\begin{pgfscope}%
\pgfsys@transformshift{2.827246in}{2.093276in}%
\pgfsys@useobject{currentmarker}{}%
\end{pgfscope}%
\begin{pgfscope}%
\pgfsys@transformshift{4.860671in}{1.087019in}%
\pgfsys@useobject{currentmarker}{}%
\end{pgfscope}%
\begin{pgfscope}%
\pgfsys@transformshift{2.873589in}{1.278336in}%
\pgfsys@useobject{currentmarker}{}%
\end{pgfscope}%
\begin{pgfscope}%
\pgfsys@transformshift{1.536825in}{0.259268in}%
\pgfsys@useobject{currentmarker}{}%
\end{pgfscope}%
\begin{pgfscope}%
\pgfsys@transformshift{2.764352in}{0.681245in}%
\pgfsys@useobject{currentmarker}{}%
\end{pgfscope}%
\begin{pgfscope}%
\pgfsys@transformshift{6.472715in}{3.118065in}%
\pgfsys@useobject{currentmarker}{}%
\end{pgfscope}%
\begin{pgfscope}%
\pgfsys@transformshift{5.451502in}{0.558177in}%
\pgfsys@useobject{currentmarker}{}%
\end{pgfscope}%
\begin{pgfscope}%
\pgfsys@transformshift{4.521637in}{4.721217in}%
\pgfsys@useobject{currentmarker}{}%
\end{pgfscope}%
\begin{pgfscope}%
\pgfsys@transformshift{5.541218in}{1.493272in}%
\pgfsys@useobject{currentmarker}{}%
\end{pgfscope}%
\begin{pgfscope}%
\pgfsys@transformshift{4.253324in}{3.800007in}%
\pgfsys@useobject{currentmarker}{}%
\end{pgfscope}%
\begin{pgfscope}%
\pgfsys@transformshift{5.895094in}{2.687622in}%
\pgfsys@useobject{currentmarker}{}%
\end{pgfscope}%
\begin{pgfscope}%
\pgfsys@transformshift{6.081673in}{1.079201in}%
\pgfsys@useobject{currentmarker}{}%
\end{pgfscope}%
\begin{pgfscope}%
\pgfsys@transformshift{2.484298in}{1.217911in}%
\pgfsys@useobject{currentmarker}{}%
\end{pgfscope}%
\begin{pgfscope}%
\pgfsys@transformshift{5.425424in}{5.125628in}%
\pgfsys@useobject{currentmarker}{}%
\end{pgfscope}%
\begin{pgfscope}%
\pgfsys@transformshift{2.759490in}{2.627771in}%
\pgfsys@useobject{currentmarker}{}%
\end{pgfscope}%
\begin{pgfscope}%
\pgfsys@transformshift{1.602078in}{5.464114in}%
\pgfsys@useobject{currentmarker}{}%
\end{pgfscope}%
\begin{pgfscope}%
\pgfsys@transformshift{2.916437in}{4.833342in}%
\pgfsys@useobject{currentmarker}{}%
\end{pgfscope}%
\begin{pgfscope}%
\pgfsys@transformshift{5.260685in}{3.002060in}%
\pgfsys@useobject{currentmarker}{}%
\end{pgfscope}%
\begin{pgfscope}%
\pgfsys@transformshift{1.180553in}{0.604540in}%
\pgfsys@useobject{currentmarker}{}%
\end{pgfscope}%
\begin{pgfscope}%
\pgfsys@transformshift{5.687892in}{4.590760in}%
\pgfsys@useobject{currentmarker}{}%
\end{pgfscope}%
\begin{pgfscope}%
\pgfsys@transformshift{0.761716in}{5.353630in}%
\pgfsys@useobject{currentmarker}{}%
\end{pgfscope}%
\begin{pgfscope}%
\pgfsys@transformshift{3.249295in}{1.538378in}%
\pgfsys@useobject{currentmarker}{}%
\end{pgfscope}%
\begin{pgfscope}%
\pgfsys@transformshift{5.306802in}{3.756128in}%
\pgfsys@useobject{currentmarker}{}%
\end{pgfscope}%
\begin{pgfscope}%
\pgfsys@transformshift{5.910930in}{2.450328in}%
\pgfsys@useobject{currentmarker}{}%
\end{pgfscope}%
\begin{pgfscope}%
\pgfsys@transformshift{4.532737in}{4.248805in}%
\pgfsys@useobject{currentmarker}{}%
\end{pgfscope}%
\begin{pgfscope}%
\pgfsys@transformshift{6.161062in}{4.110244in}%
\pgfsys@useobject{currentmarker}{}%
\end{pgfscope}%
\begin{pgfscope}%
\pgfsys@transformshift{0.787277in}{2.876043in}%
\pgfsys@useobject{currentmarker}{}%
\end{pgfscope}%
\begin{pgfscope}%
\pgfsys@transformshift{6.267104in}{3.324397in}%
\pgfsys@useobject{currentmarker}{}%
\end{pgfscope}%
\begin{pgfscope}%
\pgfsys@transformshift{2.836078in}{5.207461in}%
\pgfsys@useobject{currentmarker}{}%
\end{pgfscope}%
\begin{pgfscope}%
\pgfsys@transformshift{1.628938in}{3.940938in}%
\pgfsys@useobject{currentmarker}{}%
\end{pgfscope}%
\begin{pgfscope}%
\pgfsys@transformshift{1.324555in}{2.256327in}%
\pgfsys@useobject{currentmarker}{}%
\end{pgfscope}%
\begin{pgfscope}%
\pgfsys@transformshift{6.296936in}{0.536252in}%
\pgfsys@useobject{currentmarker}{}%
\end{pgfscope}%
\begin{pgfscope}%
\pgfsys@transformshift{1.336231in}{0.735668in}%
\pgfsys@useobject{currentmarker}{}%
\end{pgfscope}%
\begin{pgfscope}%
\pgfsys@transformshift{6.633723in}{4.558885in}%
\pgfsys@useobject{currentmarker}{}%
\end{pgfscope}%
\begin{pgfscope}%
\pgfsys@transformshift{0.911925in}{5.373046in}%
\pgfsys@useobject{currentmarker}{}%
\end{pgfscope}%
\begin{pgfscope}%
\pgfsys@transformshift{5.570592in}{2.806925in}%
\pgfsys@useobject{currentmarker}{}%
\end{pgfscope}%
\begin{pgfscope}%
\pgfsys@transformshift{5.204466in}{4.173171in}%
\pgfsys@useobject{currentmarker}{}%
\end{pgfscope}%
\begin{pgfscope}%
\pgfsys@transformshift{2.411060in}{1.472078in}%
\pgfsys@useobject{currentmarker}{}%
\end{pgfscope}%
\begin{pgfscope}%
\pgfsys@transformshift{4.028813in}{2.990967in}%
\pgfsys@useobject{currentmarker}{}%
\end{pgfscope}%
\begin{pgfscope}%
\pgfsys@transformshift{2.181372in}{1.436556in}%
\pgfsys@useobject{currentmarker}{}%
\end{pgfscope}%
\begin{pgfscope}%
\pgfsys@transformshift{2.042943in}{1.212113in}%
\pgfsys@useobject{currentmarker}{}%
\end{pgfscope}%
\begin{pgfscope}%
\pgfsys@transformshift{3.470396in}{1.039351in}%
\pgfsys@useobject{currentmarker}{}%
\end{pgfscope}%
\begin{pgfscope}%
\pgfsys@transformshift{4.279956in}{1.609999in}%
\pgfsys@useobject{currentmarker}{}%
\end{pgfscope}%
\begin{pgfscope}%
\pgfsys@transformshift{5.419420in}{5.133857in}%
\pgfsys@useobject{currentmarker}{}%
\end{pgfscope}%
\begin{pgfscope}%
\pgfsys@transformshift{1.923650in}{2.529221in}%
\pgfsys@useobject{currentmarker}{}%
\end{pgfscope}%
\begin{pgfscope}%
\pgfsys@transformshift{6.254481in}{2.411960in}%
\pgfsys@useobject{currentmarker}{}%
\end{pgfscope}%
\begin{pgfscope}%
\pgfsys@transformshift{1.386280in}{0.596923in}%
\pgfsys@useobject{currentmarker}{}%
\end{pgfscope}%
\begin{pgfscope}%
\pgfsys@transformshift{1.367933in}{5.261716in}%
\pgfsys@useobject{currentmarker}{}%
\end{pgfscope}%
\begin{pgfscope}%
\pgfsys@transformshift{5.902000in}{1.579038in}%
\pgfsys@useobject{currentmarker}{}%
\end{pgfscope}%
\begin{pgfscope}%
\pgfsys@transformshift{3.800574in}{4.788179in}%
\pgfsys@useobject{currentmarker}{}%
\end{pgfscope}%
\begin{pgfscope}%
\pgfsys@transformshift{3.371569in}{3.312704in}%
\pgfsys@useobject{currentmarker}{}%
\end{pgfscope}%
\begin{pgfscope}%
\pgfsys@transformshift{6.073693in}{4.359344in}%
\pgfsys@useobject{currentmarker}{}%
\end{pgfscope}%
\begin{pgfscope}%
\pgfsys@transformshift{4.193810in}{4.467048in}%
\pgfsys@useobject{currentmarker}{}%
\end{pgfscope}%
\begin{pgfscope}%
\pgfsys@transformshift{2.599471in}{4.124679in}%
\pgfsys@useobject{currentmarker}{}%
\end{pgfscope}%
\begin{pgfscope}%
\pgfsys@transformshift{4.514961in}{4.936874in}%
\pgfsys@useobject{currentmarker}{}%
\end{pgfscope}%
\begin{pgfscope}%
\pgfsys@transformshift{3.272950in}{3.195707in}%
\pgfsys@useobject{currentmarker}{}%
\end{pgfscope}%
\begin{pgfscope}%
\pgfsys@transformshift{1.510721in}{2.354600in}%
\pgfsys@useobject{currentmarker}{}%
\end{pgfscope}%
\begin{pgfscope}%
\pgfsys@transformshift{3.264662in}{0.622165in}%
\pgfsys@useobject{currentmarker}{}%
\end{pgfscope}%
\begin{pgfscope}%
\pgfsys@transformshift{3.625515in}{3.140835in}%
\pgfsys@useobject{currentmarker}{}%
\end{pgfscope}%
\begin{pgfscope}%
\pgfsys@transformshift{2.563533in}{1.864619in}%
\pgfsys@useobject{currentmarker}{}%
\end{pgfscope}%
\begin{pgfscope}%
\pgfsys@transformshift{4.278420in}{1.568148in}%
\pgfsys@useobject{currentmarker}{}%
\end{pgfscope}%
\begin{pgfscope}%
\pgfsys@transformshift{6.133139in}{1.593437in}%
\pgfsys@useobject{currentmarker}{}%
\end{pgfscope}%
\begin{pgfscope}%
\pgfsys@transformshift{4.848546in}{3.568652in}%
\pgfsys@useobject{currentmarker}{}%
\end{pgfscope}%
\begin{pgfscope}%
\pgfsys@transformshift{6.723860in}{3.848760in}%
\pgfsys@useobject{currentmarker}{}%
\end{pgfscope}%
\begin{pgfscope}%
\pgfsys@transformshift{1.457826in}{4.531374in}%
\pgfsys@useobject{currentmarker}{}%
\end{pgfscope}%
\begin{pgfscope}%
\pgfsys@transformshift{6.297737in}{1.887891in}%
\pgfsys@useobject{currentmarker}{}%
\end{pgfscope}%
\begin{pgfscope}%
\pgfsys@transformshift{2.346430in}{0.797244in}%
\pgfsys@useobject{currentmarker}{}%
\end{pgfscope}%
\begin{pgfscope}%
\pgfsys@transformshift{6.648502in}{3.096227in}%
\pgfsys@useobject{currentmarker}{}%
\end{pgfscope}%
\begin{pgfscope}%
\pgfsys@transformshift{4.173255in}{3.519895in}%
\pgfsys@useobject{currentmarker}{}%
\end{pgfscope}%
\begin{pgfscope}%
\pgfsys@transformshift{2.165725in}{1.381293in}%
\pgfsys@useobject{currentmarker}{}%
\end{pgfscope}%
\begin{pgfscope}%
\pgfsys@transformshift{2.721770in}{1.510947in}%
\pgfsys@useobject{currentmarker}{}%
\end{pgfscope}%
\begin{pgfscope}%
\pgfsys@transformshift{3.548642in}{1.282580in}%
\pgfsys@useobject{currentmarker}{}%
\end{pgfscope}%
\begin{pgfscope}%
\pgfsys@transformshift{2.475107in}{4.994021in}%
\pgfsys@useobject{currentmarker}{}%
\end{pgfscope}%
\begin{pgfscope}%
\pgfsys@transformshift{1.470327in}{2.482215in}%
\pgfsys@useobject{currentmarker}{}%
\end{pgfscope}%
\begin{pgfscope}%
\pgfsys@transformshift{6.669199in}{5.298142in}%
\pgfsys@useobject{currentmarker}{}%
\end{pgfscope}%
\begin{pgfscope}%
\pgfsys@transformshift{0.827830in}{1.636520in}%
\pgfsys@useobject{currentmarker}{}%
\end{pgfscope}%
\begin{pgfscope}%
\pgfsys@transformshift{5.032274in}{4.734021in}%
\pgfsys@useobject{currentmarker}{}%
\end{pgfscope}%
\begin{pgfscope}%
\pgfsys@transformshift{4.672835in}{2.191141in}%
\pgfsys@useobject{currentmarker}{}%
\end{pgfscope}%
\begin{pgfscope}%
\pgfsys@transformshift{3.169599in}{2.938631in}%
\pgfsys@useobject{currentmarker}{}%
\end{pgfscope}%
\begin{pgfscope}%
\pgfsys@transformshift{2.953476in}{1.578938in}%
\pgfsys@useobject{currentmarker}{}%
\end{pgfscope}%
\begin{pgfscope}%
\pgfsys@transformshift{2.521981in}{4.095214in}%
\pgfsys@useobject{currentmarker}{}%
\end{pgfscope}%
\begin{pgfscope}%
\pgfsys@transformshift{2.205016in}{0.421058in}%
\pgfsys@useobject{currentmarker}{}%
\end{pgfscope}%
\begin{pgfscope}%
\pgfsys@transformshift{5.352028in}{4.064159in}%
\pgfsys@useobject{currentmarker}{}%
\end{pgfscope}%
\begin{pgfscope}%
\pgfsys@transformshift{4.636411in}{5.380365in}%
\pgfsys@useobject{currentmarker}{}%
\end{pgfscope}%
\begin{pgfscope}%
\pgfsys@transformshift{2.770858in}{1.945693in}%
\pgfsys@useobject{currentmarker}{}%
\end{pgfscope}%
\begin{pgfscope}%
\pgfsys@transformshift{0.782278in}{5.103194in}%
\pgfsys@useobject{currentmarker}{}%
\end{pgfscope}%
\begin{pgfscope}%
\pgfsys@transformshift{1.775147in}{2.990035in}%
\pgfsys@useobject{currentmarker}{}%
\end{pgfscope}%
\begin{pgfscope}%
\pgfsys@transformshift{2.358027in}{3.214200in}%
\pgfsys@useobject{currentmarker}{}%
\end{pgfscope}%
\begin{pgfscope}%
\pgfsys@transformshift{3.922695in}{4.973715in}%
\pgfsys@useobject{currentmarker}{}%
\end{pgfscope}%
\begin{pgfscope}%
\pgfsys@transformshift{3.978582in}{4.103172in}%
\pgfsys@useobject{currentmarker}{}%
\end{pgfscope}%
\begin{pgfscope}%
\pgfsys@transformshift{4.368506in}{5.467180in}%
\pgfsys@useobject{currentmarker}{}%
\end{pgfscope}%
\begin{pgfscope}%
\pgfsys@transformshift{3.193502in}{1.561359in}%
\pgfsys@useobject{currentmarker}{}%
\end{pgfscope}%
\begin{pgfscope}%
\pgfsys@transformshift{6.288622in}{3.614347in}%
\pgfsys@useobject{currentmarker}{}%
\end{pgfscope}%
\begin{pgfscope}%
\pgfsys@transformshift{4.084287in}{3.312912in}%
\pgfsys@useobject{currentmarker}{}%
\end{pgfscope}%
\begin{pgfscope}%
\pgfsys@transformshift{1.347858in}{3.946126in}%
\pgfsys@useobject{currentmarker}{}%
\end{pgfscope}%
\begin{pgfscope}%
\pgfsys@transformshift{4.769809in}{3.591916in}%
\pgfsys@useobject{currentmarker}{}%
\end{pgfscope}%
\begin{pgfscope}%
\pgfsys@transformshift{3.773972in}{0.287757in}%
\pgfsys@useobject{currentmarker}{}%
\end{pgfscope}%
\begin{pgfscope}%
\pgfsys@transformshift{2.176257in}{4.675706in}%
\pgfsys@useobject{currentmarker}{}%
\end{pgfscope}%
\begin{pgfscope}%
\pgfsys@transformshift{4.512409in}{3.576521in}%
\pgfsys@useobject{currentmarker}{}%
\end{pgfscope}%
\begin{pgfscope}%
\pgfsys@transformshift{2.101242in}{4.547340in}%
\pgfsys@useobject{currentmarker}{}%
\end{pgfscope}%
\begin{pgfscope}%
\pgfsys@transformshift{1.753081in}{0.789429in}%
\pgfsys@useobject{currentmarker}{}%
\end{pgfscope}%
\begin{pgfscope}%
\pgfsys@transformshift{4.632241in}{0.212409in}%
\pgfsys@useobject{currentmarker}{}%
\end{pgfscope}%
\begin{pgfscope}%
\pgfsys@transformshift{0.632041in}{2.836243in}%
\pgfsys@useobject{currentmarker}{}%
\end{pgfscope}%
\begin{pgfscope}%
\pgfsys@transformshift{4.936656in}{4.301994in}%
\pgfsys@useobject{currentmarker}{}%
\end{pgfscope}%
\begin{pgfscope}%
\pgfsys@transformshift{6.689590in}{3.328316in}%
\pgfsys@useobject{currentmarker}{}%
\end{pgfscope}%
\begin{pgfscope}%
\pgfsys@transformshift{4.217622in}{4.612652in}%
\pgfsys@useobject{currentmarker}{}%
\end{pgfscope}%
\begin{pgfscope}%
\pgfsys@transformshift{2.619341in}{1.072985in}%
\pgfsys@useobject{currentmarker}{}%
\end{pgfscope}%
\begin{pgfscope}%
\pgfsys@transformshift{5.300125in}{5.145135in}%
\pgfsys@useobject{currentmarker}{}%
\end{pgfscope}%
\begin{pgfscope}%
\pgfsys@transformshift{3.358836in}{5.209199in}%
\pgfsys@useobject{currentmarker}{}%
\end{pgfscope}%
\begin{pgfscope}%
\pgfsys@transformshift{2.044740in}{2.143881in}%
\pgfsys@useobject{currentmarker}{}%
\end{pgfscope}%
\begin{pgfscope}%
\pgfsys@transformshift{5.852632in}{5.298387in}%
\pgfsys@useobject{currentmarker}{}%
\end{pgfscope}%
\begin{pgfscope}%
\pgfsys@transformshift{4.560976in}{5.361356in}%
\pgfsys@useobject{currentmarker}{}%
\end{pgfscope}%
\begin{pgfscope}%
\pgfsys@transformshift{2.888138in}{1.740370in}%
\pgfsys@useobject{currentmarker}{}%
\end{pgfscope}%
\begin{pgfscope}%
\pgfsys@transformshift{3.882673in}{5.078881in}%
\pgfsys@useobject{currentmarker}{}%
\end{pgfscope}%
\begin{pgfscope}%
\pgfsys@transformshift{1.316389in}{2.227657in}%
\pgfsys@useobject{currentmarker}{}%
\end{pgfscope}%
\begin{pgfscope}%
\pgfsys@transformshift{5.300434in}{3.557414in}%
\pgfsys@useobject{currentmarker}{}%
\end{pgfscope}%
\begin{pgfscope}%
\pgfsys@transformshift{1.980597in}{1.453799in}%
\pgfsys@useobject{currentmarker}{}%
\end{pgfscope}%
\begin{pgfscope}%
\pgfsys@transformshift{0.695441in}{3.262340in}%
\pgfsys@useobject{currentmarker}{}%
\end{pgfscope}%
\begin{pgfscope}%
\pgfsys@transformshift{6.706101in}{0.873051in}%
\pgfsys@useobject{currentmarker}{}%
\end{pgfscope}%
\begin{pgfscope}%
\pgfsys@transformshift{5.474031in}{1.664002in}%
\pgfsys@useobject{currentmarker}{}%
\end{pgfscope}%
\begin{pgfscope}%
\pgfsys@transformshift{4.718805in}{5.538386in}%
\pgfsys@useobject{currentmarker}{}%
\end{pgfscope}%
\begin{pgfscope}%
\pgfsys@transformshift{4.886382in}{2.207520in}%
\pgfsys@useobject{currentmarker}{}%
\end{pgfscope}%
\begin{pgfscope}%
\pgfsys@transformshift{2.099852in}{3.710927in}%
\pgfsys@useobject{currentmarker}{}%
\end{pgfscope}%
\begin{pgfscope}%
\pgfsys@transformshift{6.633788in}{4.370551in}%
\pgfsys@useobject{currentmarker}{}%
\end{pgfscope}%
\begin{pgfscope}%
\pgfsys@transformshift{6.665277in}{4.003804in}%
\pgfsys@useobject{currentmarker}{}%
\end{pgfscope}%
\begin{pgfscope}%
\pgfsys@transformshift{6.761160in}{5.179835in}%
\pgfsys@useobject{currentmarker}{}%
\end{pgfscope}%
\begin{pgfscope}%
\pgfsys@transformshift{0.733463in}{4.564859in}%
\pgfsys@useobject{currentmarker}{}%
\end{pgfscope}%
\begin{pgfscope}%
\pgfsys@transformshift{0.844439in}{3.911732in}%
\pgfsys@useobject{currentmarker}{}%
\end{pgfscope}%
\begin{pgfscope}%
\pgfsys@transformshift{5.381436in}{3.192049in}%
\pgfsys@useobject{currentmarker}{}%
\end{pgfscope}%
\begin{pgfscope}%
\pgfsys@transformshift{3.807021in}{0.858638in}%
\pgfsys@useobject{currentmarker}{}%
\end{pgfscope}%
\begin{pgfscope}%
\pgfsys@transformshift{2.391056in}{1.264711in}%
\pgfsys@useobject{currentmarker}{}%
\end{pgfscope}%
\begin{pgfscope}%
\pgfsys@transformshift{2.948632in}{0.706713in}%
\pgfsys@useobject{currentmarker}{}%
\end{pgfscope}%
\begin{pgfscope}%
\pgfsys@transformshift{3.242636in}{0.463250in}%
\pgfsys@useobject{currentmarker}{}%
\end{pgfscope}%
\begin{pgfscope}%
\pgfsys@transformshift{0.758804in}{5.598900in}%
\pgfsys@useobject{currentmarker}{}%
\end{pgfscope}%
\begin{pgfscope}%
\pgfsys@transformshift{2.299802in}{0.900808in}%
\pgfsys@useobject{currentmarker}{}%
\end{pgfscope}%
\begin{pgfscope}%
\pgfsys@transformshift{3.295763in}{1.279436in}%
\pgfsys@useobject{currentmarker}{}%
\end{pgfscope}%
\begin{pgfscope}%
\pgfsys@transformshift{6.553575in}{2.015242in}%
\pgfsys@useobject{currentmarker}{}%
\end{pgfscope}%
\begin{pgfscope}%
\pgfsys@transformshift{4.901427in}{2.011591in}%
\pgfsys@useobject{currentmarker}{}%
\end{pgfscope}%
\begin{pgfscope}%
\pgfsys@transformshift{3.869181in}{0.528128in}%
\pgfsys@useobject{currentmarker}{}%
\end{pgfscope}%
\begin{pgfscope}%
\pgfsys@transformshift{0.816007in}{3.698963in}%
\pgfsys@useobject{currentmarker}{}%
\end{pgfscope}%
\begin{pgfscope}%
\pgfsys@transformshift{3.384732in}{5.596170in}%
\pgfsys@useobject{currentmarker}{}%
\end{pgfscope}%
\begin{pgfscope}%
\pgfsys@transformshift{1.883288in}{5.003959in}%
\pgfsys@useobject{currentmarker}{}%
\end{pgfscope}%
\begin{pgfscope}%
\pgfsys@transformshift{1.174030in}{4.249498in}%
\pgfsys@useobject{currentmarker}{}%
\end{pgfscope}%
\begin{pgfscope}%
\pgfsys@transformshift{3.011708in}{4.125365in}%
\pgfsys@useobject{currentmarker}{}%
\end{pgfscope}%
\begin{pgfscope}%
\pgfsys@transformshift{2.049231in}{1.517104in}%
\pgfsys@useobject{currentmarker}{}%
\end{pgfscope}%
\begin{pgfscope}%
\pgfsys@transformshift{6.343334in}{1.037194in}%
\pgfsys@useobject{currentmarker}{}%
\end{pgfscope}%
\begin{pgfscope}%
\pgfsys@transformshift{6.000403in}{3.813646in}%
\pgfsys@useobject{currentmarker}{}%
\end{pgfscope}%
\begin{pgfscope}%
\pgfsys@transformshift{3.776016in}{5.515685in}%
\pgfsys@useobject{currentmarker}{}%
\end{pgfscope}%
\begin{pgfscope}%
\pgfsys@transformshift{4.647949in}{0.315952in}%
\pgfsys@useobject{currentmarker}{}%
\end{pgfscope}%
\begin{pgfscope}%
\pgfsys@transformshift{4.495272in}{5.387673in}%
\pgfsys@useobject{currentmarker}{}%
\end{pgfscope}%
\begin{pgfscope}%
\pgfsys@transformshift{4.731555in}{2.352484in}%
\pgfsys@useobject{currentmarker}{}%
\end{pgfscope}%
\begin{pgfscope}%
\pgfsys@transformshift{1.649823in}{1.237897in}%
\pgfsys@useobject{currentmarker}{}%
\end{pgfscope}%
\begin{pgfscope}%
\pgfsys@transformshift{1.616021in}{4.337931in}%
\pgfsys@useobject{currentmarker}{}%
\end{pgfscope}%
\begin{pgfscope}%
\pgfsys@transformshift{5.747019in}{4.866110in}%
\pgfsys@useobject{currentmarker}{}%
\end{pgfscope}%
\begin{pgfscope}%
\pgfsys@transformshift{0.883956in}{0.861120in}%
\pgfsys@useobject{currentmarker}{}%
\end{pgfscope}%
\begin{pgfscope}%
\pgfsys@transformshift{2.898364in}{4.087447in}%
\pgfsys@useobject{currentmarker}{}%
\end{pgfscope}%
\begin{pgfscope}%
\pgfsys@transformshift{3.156722in}{2.291898in}%
\pgfsys@useobject{currentmarker}{}%
\end{pgfscope}%
\begin{pgfscope}%
\pgfsys@transformshift{2.298231in}{4.039262in}%
\pgfsys@useobject{currentmarker}{}%
\end{pgfscope}%
\begin{pgfscope}%
\pgfsys@transformshift{1.889708in}{3.986759in}%
\pgfsys@useobject{currentmarker}{}%
\end{pgfscope}%
\begin{pgfscope}%
\pgfsys@transformshift{5.893202in}{3.444348in}%
\pgfsys@useobject{currentmarker}{}%
\end{pgfscope}%
\begin{pgfscope}%
\pgfsys@transformshift{5.171030in}{4.730581in}%
\pgfsys@useobject{currentmarker}{}%
\end{pgfscope}%
\begin{pgfscope}%
\pgfsys@transformshift{6.623684in}{4.946985in}%
\pgfsys@useobject{currentmarker}{}%
\end{pgfscope}%
\begin{pgfscope}%
\pgfsys@transformshift{2.202053in}{3.748833in}%
\pgfsys@useobject{currentmarker}{}%
\end{pgfscope}%
\begin{pgfscope}%
\pgfsys@transformshift{0.806785in}{2.518087in}%
\pgfsys@useobject{currentmarker}{}%
\end{pgfscope}%
\begin{pgfscope}%
\pgfsys@transformshift{6.119562in}{2.890613in}%
\pgfsys@useobject{currentmarker}{}%
\end{pgfscope}%
\begin{pgfscope}%
\pgfsys@transformshift{4.361458in}{2.021060in}%
\pgfsys@useobject{currentmarker}{}%
\end{pgfscope}%
\begin{pgfscope}%
\pgfsys@transformshift{4.079672in}{4.818521in}%
\pgfsys@useobject{currentmarker}{}%
\end{pgfscope}%
\begin{pgfscope}%
\pgfsys@transformshift{5.987951in}{1.287495in}%
\pgfsys@useobject{currentmarker}{}%
\end{pgfscope}%
\begin{pgfscope}%
\pgfsys@transformshift{4.502762in}{1.680264in}%
\pgfsys@useobject{currentmarker}{}%
\end{pgfscope}%
\begin{pgfscope}%
\pgfsys@transformshift{6.715823in}{0.617697in}%
\pgfsys@useobject{currentmarker}{}%
\end{pgfscope}%
\begin{pgfscope}%
\pgfsys@transformshift{4.070039in}{4.771017in}%
\pgfsys@useobject{currentmarker}{}%
\end{pgfscope}%
\begin{pgfscope}%
\pgfsys@transformshift{5.057400in}{2.521292in}%
\pgfsys@useobject{currentmarker}{}%
\end{pgfscope}%
\begin{pgfscope}%
\pgfsys@transformshift{4.514795in}{3.566155in}%
\pgfsys@useobject{currentmarker}{}%
\end{pgfscope}%
\begin{pgfscope}%
\pgfsys@transformshift{6.362345in}{1.868520in}%
\pgfsys@useobject{currentmarker}{}%
\end{pgfscope}%
\begin{pgfscope}%
\pgfsys@transformshift{1.841212in}{0.507564in}%
\pgfsys@useobject{currentmarker}{}%
\end{pgfscope}%
\begin{pgfscope}%
\pgfsys@transformshift{1.487338in}{4.673985in}%
\pgfsys@useobject{currentmarker}{}%
\end{pgfscope}%
\begin{pgfscope}%
\pgfsys@transformshift{6.597316in}{3.374792in}%
\pgfsys@useobject{currentmarker}{}%
\end{pgfscope}%
\begin{pgfscope}%
\pgfsys@transformshift{4.460198in}{3.902970in}%
\pgfsys@useobject{currentmarker}{}%
\end{pgfscope}%
\begin{pgfscope}%
\pgfsys@transformshift{1.318192in}{2.419967in}%
\pgfsys@useobject{currentmarker}{}%
\end{pgfscope}%
\begin{pgfscope}%
\pgfsys@transformshift{1.613930in}{0.413143in}%
\pgfsys@useobject{currentmarker}{}%
\end{pgfscope}%
\begin{pgfscope}%
\pgfsys@transformshift{5.257206in}{1.360614in}%
\pgfsys@useobject{currentmarker}{}%
\end{pgfscope}%
\begin{pgfscope}%
\pgfsys@transformshift{1.440141in}{1.772643in}%
\pgfsys@useobject{currentmarker}{}%
\end{pgfscope}%
\begin{pgfscope}%
\pgfsys@transformshift{2.953166in}{3.772788in}%
\pgfsys@useobject{currentmarker}{}%
\end{pgfscope}%
\begin{pgfscope}%
\pgfsys@transformshift{4.214887in}{4.188086in}%
\pgfsys@useobject{currentmarker}{}%
\end{pgfscope}%
\begin{pgfscope}%
\pgfsys@transformshift{4.789466in}{2.974967in}%
\pgfsys@useobject{currentmarker}{}%
\end{pgfscope}%
\begin{pgfscope}%
\pgfsys@transformshift{5.488053in}{3.383650in}%
\pgfsys@useobject{currentmarker}{}%
\end{pgfscope}%
\begin{pgfscope}%
\pgfsys@transformshift{4.578398in}{2.975727in}%
\pgfsys@useobject{currentmarker}{}%
\end{pgfscope}%
\begin{pgfscope}%
\pgfsys@transformshift{5.907316in}{1.755087in}%
\pgfsys@useobject{currentmarker}{}%
\end{pgfscope}%
\begin{pgfscope}%
\pgfsys@transformshift{0.794864in}{1.561967in}%
\pgfsys@useobject{currentmarker}{}%
\end{pgfscope}%
\begin{pgfscope}%
\pgfsys@transformshift{4.611415in}{0.866224in}%
\pgfsys@useobject{currentmarker}{}%
\end{pgfscope}%
\begin{pgfscope}%
\pgfsys@transformshift{1.024983in}{5.547819in}%
\pgfsys@useobject{currentmarker}{}%
\end{pgfscope}%
\begin{pgfscope}%
\pgfsys@transformshift{2.096234in}{0.799507in}%
\pgfsys@useobject{currentmarker}{}%
\end{pgfscope}%
\begin{pgfscope}%
\pgfsys@transformshift{1.569828in}{0.385262in}%
\pgfsys@useobject{currentmarker}{}%
\end{pgfscope}%
\begin{pgfscope}%
\pgfsys@transformshift{2.777036in}{5.405926in}%
\pgfsys@useobject{currentmarker}{}%
\end{pgfscope}%
\begin{pgfscope}%
\pgfsys@transformshift{2.785162in}{1.086089in}%
\pgfsys@useobject{currentmarker}{}%
\end{pgfscope}%
\begin{pgfscope}%
\pgfsys@transformshift{3.728300in}{1.947660in}%
\pgfsys@useobject{currentmarker}{}%
\end{pgfscope}%
\begin{pgfscope}%
\pgfsys@transformshift{5.521787in}{2.578131in}%
\pgfsys@useobject{currentmarker}{}%
\end{pgfscope}%
\begin{pgfscope}%
\pgfsys@transformshift{3.465046in}{3.551128in}%
\pgfsys@useobject{currentmarker}{}%
\end{pgfscope}%
\begin{pgfscope}%
\pgfsys@transformshift{6.422339in}{4.732718in}%
\pgfsys@useobject{currentmarker}{}%
\end{pgfscope}%
\begin{pgfscope}%
\pgfsys@transformshift{1.621743in}{4.623190in}%
\pgfsys@useobject{currentmarker}{}%
\end{pgfscope}%
\begin{pgfscope}%
\pgfsys@transformshift{0.890611in}{4.445590in}%
\pgfsys@useobject{currentmarker}{}%
\end{pgfscope}%
\begin{pgfscope}%
\pgfsys@transformshift{5.947438in}{4.060954in}%
\pgfsys@useobject{currentmarker}{}%
\end{pgfscope}%
\begin{pgfscope}%
\pgfsys@transformshift{2.512667in}{5.565949in}%
\pgfsys@useobject{currentmarker}{}%
\end{pgfscope}%
\begin{pgfscope}%
\pgfsys@transformshift{4.067232in}{5.013550in}%
\pgfsys@useobject{currentmarker}{}%
\end{pgfscope}%
\begin{pgfscope}%
\pgfsys@transformshift{2.448968in}{2.405912in}%
\pgfsys@useobject{currentmarker}{}%
\end{pgfscope}%
\begin{pgfscope}%
\pgfsys@transformshift{4.072426in}{5.401527in}%
\pgfsys@useobject{currentmarker}{}%
\end{pgfscope}%
\begin{pgfscope}%
\pgfsys@transformshift{0.857677in}{4.708843in}%
\pgfsys@useobject{currentmarker}{}%
\end{pgfscope}%
\begin{pgfscope}%
\pgfsys@transformshift{6.282338in}{2.736996in}%
\pgfsys@useobject{currentmarker}{}%
\end{pgfscope}%
\begin{pgfscope}%
\pgfsys@transformshift{5.093364in}{5.042830in}%
\pgfsys@useobject{currentmarker}{}%
\end{pgfscope}%
\begin{pgfscope}%
\pgfsys@transformshift{3.571786in}{1.453709in}%
\pgfsys@useobject{currentmarker}{}%
\end{pgfscope}%
\begin{pgfscope}%
\pgfsys@transformshift{1.562124in}{0.839372in}%
\pgfsys@useobject{currentmarker}{}%
\end{pgfscope}%
\begin{pgfscope}%
\pgfsys@transformshift{0.816111in}{3.179639in}%
\pgfsys@useobject{currentmarker}{}%
\end{pgfscope}%
\begin{pgfscope}%
\pgfsys@transformshift{2.160495in}{0.629547in}%
\pgfsys@useobject{currentmarker}{}%
\end{pgfscope}%
\begin{pgfscope}%
\pgfsys@transformshift{4.651263in}{3.714401in}%
\pgfsys@useobject{currentmarker}{}%
\end{pgfscope}%
\begin{pgfscope}%
\pgfsys@transformshift{4.577791in}{2.654746in}%
\pgfsys@useobject{currentmarker}{}%
\end{pgfscope}%
\begin{pgfscope}%
\pgfsys@transformshift{6.728982in}{2.689800in}%
\pgfsys@useobject{currentmarker}{}%
\end{pgfscope}%
\begin{pgfscope}%
\pgfsys@transformshift{4.891599in}{4.563249in}%
\pgfsys@useobject{currentmarker}{}%
\end{pgfscope}%
\begin{pgfscope}%
\pgfsys@transformshift{6.740999in}{3.849346in}%
\pgfsys@useobject{currentmarker}{}%
\end{pgfscope}%
\begin{pgfscope}%
\pgfsys@transformshift{4.099947in}{4.262015in}%
\pgfsys@useobject{currentmarker}{}%
\end{pgfscope}%
\begin{pgfscope}%
\pgfsys@transformshift{1.034388in}{0.344090in}%
\pgfsys@useobject{currentmarker}{}%
\end{pgfscope}%
\begin{pgfscope}%
\pgfsys@transformshift{4.420099in}{4.245184in}%
\pgfsys@useobject{currentmarker}{}%
\end{pgfscope}%
\begin{pgfscope}%
\pgfsys@transformshift{5.747745in}{4.768645in}%
\pgfsys@useobject{currentmarker}{}%
\end{pgfscope}%
\begin{pgfscope}%
\pgfsys@transformshift{3.239791in}{1.440911in}%
\pgfsys@useobject{currentmarker}{}%
\end{pgfscope}%
\begin{pgfscope}%
\pgfsys@transformshift{6.718639in}{1.029180in}%
\pgfsys@useobject{currentmarker}{}%
\end{pgfscope}%
\begin{pgfscope}%
\pgfsys@transformshift{6.744954in}{3.132132in}%
\pgfsys@useobject{currentmarker}{}%
\end{pgfscope}%
\begin{pgfscope}%
\pgfsys@transformshift{2.462530in}{4.768330in}%
\pgfsys@useobject{currentmarker}{}%
\end{pgfscope}%
\begin{pgfscope}%
\pgfsys@transformshift{4.109290in}{4.745653in}%
\pgfsys@useobject{currentmarker}{}%
\end{pgfscope}%
\begin{pgfscope}%
\pgfsys@transformshift{3.063840in}{4.490096in}%
\pgfsys@useobject{currentmarker}{}%
\end{pgfscope}%
\begin{pgfscope}%
\pgfsys@transformshift{3.621973in}{5.508516in}%
\pgfsys@useobject{currentmarker}{}%
\end{pgfscope}%
\begin{pgfscope}%
\pgfsys@transformshift{4.459954in}{4.106209in}%
\pgfsys@useobject{currentmarker}{}%
\end{pgfscope}%
\begin{pgfscope}%
\pgfsys@transformshift{6.170199in}{2.008509in}%
\pgfsys@useobject{currentmarker}{}%
\end{pgfscope}%
\begin{pgfscope}%
\pgfsys@transformshift{4.627067in}{0.828684in}%
\pgfsys@useobject{currentmarker}{}%
\end{pgfscope}%
\begin{pgfscope}%
\pgfsys@transformshift{4.432043in}{2.914979in}%
\pgfsys@useobject{currentmarker}{}%
\end{pgfscope}%
\begin{pgfscope}%
\pgfsys@transformshift{4.727994in}{4.194037in}%
\pgfsys@useobject{currentmarker}{}%
\end{pgfscope}%
\begin{pgfscope}%
\pgfsys@transformshift{6.576946in}{3.399158in}%
\pgfsys@useobject{currentmarker}{}%
\end{pgfscope}%
\begin{pgfscope}%
\pgfsys@transformshift{5.099498in}{0.575336in}%
\pgfsys@useobject{currentmarker}{}%
\end{pgfscope}%
\begin{pgfscope}%
\pgfsys@transformshift{3.780532in}{2.528300in}%
\pgfsys@useobject{currentmarker}{}%
\end{pgfscope}%
\begin{pgfscope}%
\pgfsys@transformshift{4.323725in}{2.496238in}%
\pgfsys@useobject{currentmarker}{}%
\end{pgfscope}%
\begin{pgfscope}%
\pgfsys@transformshift{3.260696in}{1.952726in}%
\pgfsys@useobject{currentmarker}{}%
\end{pgfscope}%
\begin{pgfscope}%
\pgfsys@transformshift{1.045688in}{1.410939in}%
\pgfsys@useobject{currentmarker}{}%
\end{pgfscope}%
\begin{pgfscope}%
\pgfsys@transformshift{5.821127in}{4.339093in}%
\pgfsys@useobject{currentmarker}{}%
\end{pgfscope}%
\begin{pgfscope}%
\pgfsys@transformshift{5.566394in}{4.342515in}%
\pgfsys@useobject{currentmarker}{}%
\end{pgfscope}%
\begin{pgfscope}%
\pgfsys@transformshift{3.639947in}{2.206803in}%
\pgfsys@useobject{currentmarker}{}%
\end{pgfscope}%
\begin{pgfscope}%
\pgfsys@transformshift{5.561833in}{4.281859in}%
\pgfsys@useobject{currentmarker}{}%
\end{pgfscope}%
\begin{pgfscope}%
\pgfsys@transformshift{1.847167in}{0.855194in}%
\pgfsys@useobject{currentmarker}{}%
\end{pgfscope}%
\begin{pgfscope}%
\pgfsys@transformshift{6.708460in}{3.253407in}%
\pgfsys@useobject{currentmarker}{}%
\end{pgfscope}%
\begin{pgfscope}%
\pgfsys@transformshift{3.909865in}{2.742576in}%
\pgfsys@useobject{currentmarker}{}%
\end{pgfscope}%
\begin{pgfscope}%
\pgfsys@transformshift{2.269464in}{3.395422in}%
\pgfsys@useobject{currentmarker}{}%
\end{pgfscope}%
\begin{pgfscope}%
\pgfsys@transformshift{6.875543in}{4.279453in}%
\pgfsys@useobject{currentmarker}{}%
\end{pgfscope}%
\begin{pgfscope}%
\pgfsys@transformshift{3.861631in}{2.252622in}%
\pgfsys@useobject{currentmarker}{}%
\end{pgfscope}%
\begin{pgfscope}%
\pgfsys@transformshift{2.729439in}{2.717699in}%
\pgfsys@useobject{currentmarker}{}%
\end{pgfscope}%
\end{pgfscope}%
\begin{pgfscope}%
\pgfpathrectangle{\pgfqpoint{0.626386in}{0.608332in}}{\pgfqpoint{6.200000in}{4.620000in}}%
\pgfusepath{clip}%
\pgfsetbuttcap%
\pgfsetroundjoin%
\definecolor{currentfill}{rgb}{0.839216,0.152941,0.156863}%
\pgfsetfillcolor{currentfill}%
\pgfsetlinewidth{1.003750pt}%
\definecolor{currentstroke}{rgb}{0.839216,0.152941,0.156863}%
\pgfsetstrokecolor{currentstroke}%
\pgfsetdash{}{0pt}%
\pgfsys@defobject{currentmarker}{\pgfqpoint{-0.031056in}{-0.031056in}}{\pgfqpoint{0.031056in}{0.031056in}}{%
\pgfpathmoveto{\pgfqpoint{0.000000in}{-0.031056in}}%
\pgfpathcurveto{\pgfqpoint{0.008236in}{-0.031056in}}{\pgfqpoint{0.016136in}{-0.027784in}}{\pgfqpoint{0.021960in}{-0.021960in}}%
\pgfpathcurveto{\pgfqpoint{0.027784in}{-0.016136in}}{\pgfqpoint{0.031056in}{-0.008236in}}{\pgfqpoint{0.031056in}{0.000000in}}%
\pgfpathcurveto{\pgfqpoint{0.031056in}{0.008236in}}{\pgfqpoint{0.027784in}{0.016136in}}{\pgfqpoint{0.021960in}{0.021960in}}%
\pgfpathcurveto{\pgfqpoint{0.016136in}{0.027784in}}{\pgfqpoint{0.008236in}{0.031056in}}{\pgfqpoint{0.000000in}{0.031056in}}%
\pgfpathcurveto{\pgfqpoint{-0.008236in}{0.031056in}}{\pgfqpoint{-0.016136in}{0.027784in}}{\pgfqpoint{-0.021960in}{0.021960in}}%
\pgfpathcurveto{\pgfqpoint{-0.027784in}{0.016136in}}{\pgfqpoint{-0.031056in}{0.008236in}}{\pgfqpoint{-0.031056in}{0.000000in}}%
\pgfpathcurveto{\pgfqpoint{-0.031056in}{-0.008236in}}{\pgfqpoint{-0.027784in}{-0.016136in}}{\pgfqpoint{-0.021960in}{-0.021960in}}%
\pgfpathcurveto{\pgfqpoint{-0.016136in}{-0.027784in}}{\pgfqpoint{-0.008236in}{-0.031056in}}{\pgfqpoint{0.000000in}{-0.031056in}}%
\pgfpathlineto{\pgfqpoint{0.000000in}{-0.031056in}}%
\pgfpathclose%
\pgfusepath{stroke,fill}%
}%
\begin{pgfscope}%
\pgfsys@transformshift{1.171002in}{2.753761in}%
\pgfsys@useobject{currentmarker}{}%
\end{pgfscope}%
\begin{pgfscope}%
\pgfsys@transformshift{2.672978in}{1.285705in}%
\pgfsys@useobject{currentmarker}{}%
\end{pgfscope}%
\begin{pgfscope}%
\pgfsys@transformshift{2.536411in}{1.480725in}%
\pgfsys@useobject{currentmarker}{}%
\end{pgfscope}%
\begin{pgfscope}%
\pgfsys@transformshift{6.563792in}{0.571309in}%
\pgfsys@useobject{currentmarker}{}%
\end{pgfscope}%
\begin{pgfscope}%
\pgfsys@transformshift{4.968865in}{0.711625in}%
\pgfsys@useobject{currentmarker}{}%
\end{pgfscope}%
\begin{pgfscope}%
\pgfsys@transformshift{4.114124in}{0.764184in}%
\pgfsys@useobject{currentmarker}{}%
\end{pgfscope}%
\begin{pgfscope}%
\pgfsys@transformshift{2.650795in}{1.247392in}%
\pgfsys@useobject{currentmarker}{}%
\end{pgfscope}%
\begin{pgfscope}%
\pgfsys@transformshift{3.771139in}{0.864515in}%
\pgfsys@useobject{currentmarker}{}%
\end{pgfscope}%
\begin{pgfscope}%
\pgfsys@transformshift{3.098873in}{1.320918in}%
\pgfsys@useobject{currentmarker}{}%
\end{pgfscope}%
\begin{pgfscope}%
\pgfsys@transformshift{2.594401in}{1.640905in}%
\pgfsys@useobject{currentmarker}{}%
\end{pgfscope}%
\begin{pgfscope}%
\pgfsys@transformshift{1.865799in}{1.980264in}%
\pgfsys@useobject{currentmarker}{}%
\end{pgfscope}%
\begin{pgfscope}%
\pgfsys@transformshift{1.279225in}{2.588906in}%
\pgfsys@useobject{currentmarker}{}%
\end{pgfscope}%
\begin{pgfscope}%
\pgfsys@transformshift{2.908829in}{1.191898in}%
\pgfsys@useobject{currentmarker}{}%
\end{pgfscope}%
\begin{pgfscope}%
\pgfsys@transformshift{1.419679in}{2.355247in}%
\pgfsys@useobject{currentmarker}{}%
\end{pgfscope}%
\begin{pgfscope}%
\pgfsys@transformshift{1.056228in}{2.867376in}%
\pgfsys@useobject{currentmarker}{}%
\end{pgfscope}%
\begin{pgfscope}%
\pgfsys@transformshift{0.763440in}{3.773630in}%
\pgfsys@useobject{currentmarker}{}%
\end{pgfscope}%
\begin{pgfscope}%
\pgfsys@transformshift{4.539430in}{0.774317in}%
\pgfsys@useobject{currentmarker}{}%
\end{pgfscope}%
\begin{pgfscope}%
\pgfsys@transformshift{5.164025in}{0.672611in}%
\pgfsys@useobject{currentmarker}{}%
\end{pgfscope}%
\begin{pgfscope}%
\pgfsys@transformshift{1.411858in}{2.496224in}%
\pgfsys@useobject{currentmarker}{}%
\end{pgfscope}%
\begin{pgfscope}%
\pgfsys@transformshift{3.323354in}{1.022890in}%
\pgfsys@useobject{currentmarker}{}%
\end{pgfscope}%
\begin{pgfscope}%
\pgfsys@transformshift{1.152253in}{2.650657in}%
\pgfsys@useobject{currentmarker}{}%
\end{pgfscope}%
\begin{pgfscope}%
\pgfsys@transformshift{4.493681in}{0.668277in}%
\pgfsys@useobject{currentmarker}{}%
\end{pgfscope}%
\begin{pgfscope}%
\pgfsys@transformshift{4.139112in}{0.785824in}%
\pgfsys@useobject{currentmarker}{}%
\end{pgfscope}%
\begin{pgfscope}%
\pgfsys@transformshift{1.602751in}{2.048253in}%
\pgfsys@useobject{currentmarker}{}%
\end{pgfscope}%
\begin{pgfscope}%
\pgfsys@transformshift{5.482158in}{0.576145in}%
\pgfsys@useobject{currentmarker}{}%
\end{pgfscope}%
\begin{pgfscope}%
\pgfsys@transformshift{3.248174in}{0.965477in}%
\pgfsys@useobject{currentmarker}{}%
\end{pgfscope}%
\begin{pgfscope}%
\pgfsys@transformshift{3.839753in}{0.806067in}%
\pgfsys@useobject{currentmarker}{}%
\end{pgfscope}%
\begin{pgfscope}%
\pgfsys@transformshift{2.429558in}{1.668869in}%
\pgfsys@useobject{currentmarker}{}%
\end{pgfscope}%
\begin{pgfscope}%
\pgfsys@transformshift{3.106341in}{1.308808in}%
\pgfsys@useobject{currentmarker}{}%
\end{pgfscope}%
\begin{pgfscope}%
\pgfsys@transformshift{3.993120in}{0.899747in}%
\pgfsys@useobject{currentmarker}{}%
\end{pgfscope}%
\begin{pgfscope}%
\pgfsys@transformshift{5.758990in}{0.597983in}%
\pgfsys@useobject{currentmarker}{}%
\end{pgfscope}%
\begin{pgfscope}%
\pgfsys@transformshift{4.342434in}{0.678690in}%
\pgfsys@useobject{currentmarker}{}%
\end{pgfscope}%
\begin{pgfscope}%
\pgfsys@transformshift{4.913462in}{0.591345in}%
\pgfsys@useobject{currentmarker}{}%
\end{pgfscope}%
\begin{pgfscope}%
\pgfsys@transformshift{1.456272in}{2.142124in}%
\pgfsys@useobject{currentmarker}{}%
\end{pgfscope}%
\begin{pgfscope}%
\pgfsys@transformshift{0.983666in}{3.334171in}%
\pgfsys@useobject{currentmarker}{}%
\end{pgfscope}%
\begin{pgfscope}%
\pgfsys@transformshift{1.675103in}{2.367521in}%
\pgfsys@useobject{currentmarker}{}%
\end{pgfscope}%
\begin{pgfscope}%
\pgfsys@transformshift{3.462161in}{1.070390in}%
\pgfsys@useobject{currentmarker}{}%
\end{pgfscope}%
\begin{pgfscope}%
\pgfsys@transformshift{4.702284in}{0.728509in}%
\pgfsys@useobject{currentmarker}{}%
\end{pgfscope}%
\begin{pgfscope}%
\pgfsys@transformshift{2.059371in}{1.976888in}%
\pgfsys@useobject{currentmarker}{}%
\end{pgfscope}%
\begin{pgfscope}%
\pgfsys@transformshift{1.630066in}{2.230214in}%
\pgfsys@useobject{currentmarker}{}%
\end{pgfscope}%
\begin{pgfscope}%
\pgfsys@transformshift{3.940720in}{0.753932in}%
\pgfsys@useobject{currentmarker}{}%
\end{pgfscope}%
\begin{pgfscope}%
\pgfsys@transformshift{3.796370in}{0.920318in}%
\pgfsys@useobject{currentmarker}{}%
\end{pgfscope}%
\begin{pgfscope}%
\pgfsys@transformshift{2.873589in}{1.278336in}%
\pgfsys@useobject{currentmarker}{}%
\end{pgfscope}%
\begin{pgfscope}%
\pgfsys@transformshift{5.451502in}{0.558177in}%
\pgfsys@useobject{currentmarker}{}%
\end{pgfscope}%
\begin{pgfscope}%
\pgfsys@transformshift{1.324555in}{2.256327in}%
\pgfsys@useobject{currentmarker}{}%
\end{pgfscope}%
\begin{pgfscope}%
\pgfsys@transformshift{2.411060in}{1.472078in}%
\pgfsys@useobject{currentmarker}{}%
\end{pgfscope}%
\begin{pgfscope}%
\pgfsys@transformshift{3.470396in}{1.039351in}%
\pgfsys@useobject{currentmarker}{}%
\end{pgfscope}%
\begin{pgfscope}%
\pgfsys@transformshift{1.510721in}{2.354600in}%
\pgfsys@useobject{currentmarker}{}%
\end{pgfscope}%
\begin{pgfscope}%
\pgfsys@transformshift{2.721770in}{1.510947in}%
\pgfsys@useobject{currentmarker}{}%
\end{pgfscope}%
\begin{pgfscope}%
\pgfsys@transformshift{1.470327in}{2.482215in}%
\pgfsys@useobject{currentmarker}{}%
\end{pgfscope}%
\begin{pgfscope}%
\pgfsys@transformshift{3.807021in}{0.858638in}%
\pgfsys@useobject{currentmarker}{}%
\end{pgfscope}%
\begin{pgfscope}%
\pgfsys@transformshift{0.816007in}{3.698963in}%
\pgfsys@useobject{currentmarker}{}%
\end{pgfscope}%
\begin{pgfscope}%
\pgfsys@transformshift{1.318192in}{2.419967in}%
\pgfsys@useobject{currentmarker}{}%
\end{pgfscope}%
\begin{pgfscope}%
\pgfsys@transformshift{5.099498in}{0.575336in}%
\pgfsys@useobject{currentmarker}{}%
\end{pgfscope}%
\end{pgfscope}%
\begin{pgfscope}%
\pgfpathrectangle{\pgfqpoint{0.626386in}{0.608332in}}{\pgfqpoint{6.200000in}{4.620000in}}%
\pgfusepath{clip}%
\pgfsetbuttcap%
\pgfsetmiterjoin%
\definecolor{currentfill}{rgb}{0.501961,0.501961,0.501961}%
\pgfsetfillcolor{currentfill}%
\pgfsetfillopacity{0.200000}%
\pgfsetlinewidth{1.003750pt}%
\definecolor{currentstroke}{rgb}{0.501961,0.501961,0.501961}%
\pgfsetstrokecolor{currentstroke}%
\pgfsetstrokeopacity{0.200000}%
\pgfsetdash{}{0pt}%
\pgfpathmoveto{\pgfqpoint{0.626386in}{4.391376in}}%
\pgfpathlineto{\pgfqpoint{0.627161in}{4.307262in}}%
\pgfpathlineto{\pgfqpoint{0.629487in}{4.224002in}}%
\pgfpathlineto{\pgfqpoint{0.633363in}{4.141596in}}%
\pgfpathlineto{\pgfqpoint{0.638790in}{4.060043in}}%
\pgfpathlineto{\pgfqpoint{0.645767in}{3.979345in}}%
\pgfpathlineto{\pgfqpoint{0.654294in}{3.899501in}}%
\pgfpathlineto{\pgfqpoint{0.664372in}{3.820510in}}%
\pgfpathlineto{\pgfqpoint{0.676001in}{3.742374in}}%
\pgfpathlineto{\pgfqpoint{0.689180in}{3.665091in}}%
\pgfpathlineto{\pgfqpoint{0.703909in}{3.588663in}}%
\pgfpathlineto{\pgfqpoint{0.720189in}{3.513088in}}%
\pgfpathlineto{\pgfqpoint{0.738019in}{3.438368in}}%
\pgfpathlineto{\pgfqpoint{0.757400in}{3.364501in}}%
\pgfpathlineto{\pgfqpoint{0.778331in}{3.291488in}}%
\pgfpathlineto{\pgfqpoint{0.800813in}{3.219329in}}%
\pgfpathlineto{\pgfqpoint{0.824845in}{3.148025in}}%
\pgfpathlineto{\pgfqpoint{0.850428in}{3.077574in}}%
\pgfpathlineto{\pgfqpoint{0.877561in}{3.007977in}}%
\pgfpathlineto{\pgfqpoint{0.906244in}{2.939234in}}%
\pgfpathlineto{\pgfqpoint{0.936478in}{2.871345in}}%
\pgfpathlineto{\pgfqpoint{0.968263in}{2.804310in}}%
\pgfpathlineto{\pgfqpoint{1.001598in}{2.738129in}}%
\pgfpathlineto{\pgfqpoint{1.036483in}{2.672801in}}%
\pgfpathlineto{\pgfqpoint{1.072919in}{2.608328in}}%
\pgfpathlineto{\pgfqpoint{1.110905in}{2.544709in}}%
\pgfpathlineto{\pgfqpoint{1.150442in}{2.481943in}}%
\pgfpathlineto{\pgfqpoint{1.191529in}{2.420032in}}%
\pgfpathlineto{\pgfqpoint{1.234167in}{2.358975in}}%
\pgfpathlineto{\pgfqpoint{1.278355in}{2.298771in}}%
\pgfpathlineto{\pgfqpoint{1.324094in}{2.239422in}}%
\pgfpathlineto{\pgfqpoint{1.371383in}{2.180926in}}%
\pgfpathlineto{\pgfqpoint{1.420223in}{2.123284in}}%
\pgfpathlineto{\pgfqpoint{1.470613in}{2.066497in}}%
\pgfpathlineto{\pgfqpoint{1.522553in}{2.010563in}}%
\pgfpathlineto{\pgfqpoint{1.576044in}{1.955483in}}%
\pgfpathlineto{\pgfqpoint{1.631085in}{1.901257in}}%
\pgfpathlineto{\pgfqpoint{1.687677in}{1.847886in}}%
\pgfpathlineto{\pgfqpoint{1.745820in}{1.795368in}}%
\pgfpathlineto{\pgfqpoint{1.805512in}{1.743704in}}%
\pgfpathlineto{\pgfqpoint{1.866756in}{1.692894in}}%
\pgfpathlineto{\pgfqpoint{1.929549in}{1.642937in}}%
\pgfpathlineto{\pgfqpoint{1.993893in}{1.593835in}}%
\pgfpathlineto{\pgfqpoint{2.059788in}{1.545587in}}%
\pgfpathlineto{\pgfqpoint{2.127233in}{1.498193in}}%
\pgfpathlineto{\pgfqpoint{2.196229in}{1.451653in}}%
\pgfpathlineto{\pgfqpoint{2.266775in}{1.405966in}}%
\pgfpathlineto{\pgfqpoint{2.338871in}{1.361134in}}%
\pgfpathlineto{\pgfqpoint{2.412518in}{1.317156in}}%
\pgfpathlineto{\pgfqpoint{2.487716in}{1.274031in}}%
\pgfpathlineto{\pgfqpoint{2.564464in}{1.231760in}}%
\pgfpathlineto{\pgfqpoint{2.642762in}{1.190344in}}%
\pgfpathlineto{\pgfqpoint{2.722611in}{1.149781in}}%
\pgfpathlineto{\pgfqpoint{2.804010in}{1.110073in}}%
\pgfpathlineto{\pgfqpoint{2.886960in}{1.071218in}}%
\pgfpathlineto{\pgfqpoint{2.971460in}{1.033217in}}%
\pgfpathlineto{\pgfqpoint{3.057510in}{0.996070in}}%
\pgfpathlineto{\pgfqpoint{3.145112in}{0.959777in}}%
\pgfpathlineto{\pgfqpoint{3.234263in}{0.924338in}}%
\pgfpathlineto{\pgfqpoint{3.324965in}{0.889753in}}%
\pgfpathlineto{\pgfqpoint{3.389449in}{0.866091in}}%
\pgfpathlineto{\pgfqpoint{3.436350in}{0.849652in}}%
\pgfpathlineto{\pgfqpoint{3.484027in}{0.833640in}}%
\pgfpathlineto{\pgfqpoint{3.532479in}{0.818056in}}%
\pgfpathlineto{\pgfqpoint{3.581706in}{0.802898in}}%
\pgfpathlineto{\pgfqpoint{3.631709in}{0.788168in}}%
\pgfpathlineto{\pgfqpoint{3.682486in}{0.773864in}}%
\pgfpathlineto{\pgfqpoint{3.734039in}{0.759987in}}%
\pgfpathlineto{\pgfqpoint{3.786367in}{0.746537in}}%
\pgfpathlineto{\pgfqpoint{3.839470in}{0.733515in}}%
\pgfpathlineto{\pgfqpoint{3.893349in}{0.720919in}}%
\pgfpathlineto{\pgfqpoint{3.948003in}{0.708750in}}%
\pgfpathlineto{\pgfqpoint{4.003432in}{0.697008in}}%
\pgfpathlineto{\pgfqpoint{4.059636in}{0.685694in}}%
\pgfpathlineto{\pgfqpoint{4.116616in}{0.674806in}}%
\pgfpathlineto{\pgfqpoint{4.174370in}{0.664345in}}%
\pgfpathlineto{\pgfqpoint{4.232900in}{0.654311in}}%
\pgfpathlineto{\pgfqpoint{4.292205in}{0.644704in}}%
\pgfpathlineto{\pgfqpoint{4.352286in}{0.635524in}}%
\pgfpathlineto{\pgfqpoint{4.413141in}{0.626771in}}%
\pgfpathlineto{\pgfqpoint{4.474772in}{0.618445in}}%
\pgfpathlineto{\pgfqpoint{4.537178in}{0.610546in}}%
\pgfpathlineto{\pgfqpoint{4.600360in}{0.603074in}}%
\pgfpathlineto{\pgfqpoint{4.664316in}{0.596029in}}%
\pgfpathlineto{\pgfqpoint{4.729048in}{0.589411in}}%
\pgfpathlineto{\pgfqpoint{4.794555in}{0.583220in}}%
\pgfpathlineto{\pgfqpoint{4.860837in}{0.577455in}}%
\pgfpathlineto{\pgfqpoint{4.927895in}{0.572118in}}%
\pgfpathlineto{\pgfqpoint{4.995728in}{0.567208in}}%
\pgfpathlineto{\pgfqpoint{5.064335in}{0.562725in}}%
\pgfpathlineto{\pgfqpoint{5.133719in}{0.558668in}}%
\pgfpathlineto{\pgfqpoint{5.203877in}{0.555039in}}%
\pgfpathlineto{\pgfqpoint{5.274811in}{0.551837in}}%
\pgfpathlineto{\pgfqpoint{5.346520in}{0.549062in}}%
\pgfpathlineto{\pgfqpoint{5.419004in}{0.546713in}}%
\pgfpathlineto{\pgfqpoint{5.492263in}{0.544792in}}%
\pgfpathlineto{\pgfqpoint{5.566298in}{0.543297in}}%
\pgfpathlineto{\pgfqpoint{5.641107in}{0.542230in}}%
\pgfpathlineto{\pgfqpoint{5.716692in}{0.541589in}}%
\pgfpathlineto{\pgfqpoint{5.793053in}{0.541376in}}%
\pgfpathlineto{\pgfqpoint{6.826386in}{0.608332in}}%
\pgfpathlineto{\pgfqpoint{6.734754in}{0.608589in}}%
\pgfpathlineto{\pgfqpoint{6.644052in}{0.609357in}}%
\pgfpathlineto{\pgfqpoint{6.554280in}{0.610638in}}%
\pgfpathlineto{\pgfqpoint{6.465438in}{0.612431in}}%
\pgfpathlineto{\pgfqpoint{6.377527in}{0.614737in}}%
\pgfpathlineto{\pgfqpoint{6.290546in}{0.617555in}}%
\pgfpathlineto{\pgfqpoint{6.204496in}{0.620886in}}%
\pgfpathlineto{\pgfqpoint{6.119375in}{0.624728in}}%
\pgfpathlineto{\pgfqpoint{6.035185in}{0.629083in}}%
\pgfpathlineto{\pgfqpoint{5.951925in}{0.633951in}}%
\pgfpathlineto{\pgfqpoint{5.869596in}{0.639331in}}%
\pgfpathlineto{\pgfqpoint{5.788197in}{0.645223in}}%
\pgfpathlineto{\pgfqpoint{5.707728in}{0.651628in}}%
\pgfpathlineto{\pgfqpoint{5.628189in}{0.658545in}}%
\pgfpathlineto{\pgfqpoint{5.549580in}{0.665974in}}%
\pgfpathlineto{\pgfqpoint{5.471902in}{0.673916in}}%
\pgfpathlineto{\pgfqpoint{5.395154in}{0.682370in}}%
\pgfpathlineto{\pgfqpoint{5.319337in}{0.691336in}}%
\pgfpathlineto{\pgfqpoint{5.244450in}{0.700815in}}%
\pgfpathlineto{\pgfqpoint{5.170493in}{0.710807in}}%
\pgfpathlineto{\pgfqpoint{5.097466in}{0.721310in}}%
\pgfpathlineto{\pgfqpoint{5.025369in}{0.732326in}}%
\pgfpathlineto{\pgfqpoint{4.954203in}{0.743854in}}%
\pgfpathlineto{\pgfqpoint{4.883967in}{0.755895in}}%
\pgfpathlineto{\pgfqpoint{4.814661in}{0.768448in}}%
\pgfpathlineto{\pgfqpoint{4.746286in}{0.781514in}}%
\pgfpathlineto{\pgfqpoint{4.678841in}{0.795091in}}%
\pgfpathlineto{\pgfqpoint{4.612326in}{0.809182in}}%
\pgfpathlineto{\pgfqpoint{4.546742in}{0.823784in}}%
\pgfpathlineto{\pgfqpoint{4.482087in}{0.838899in}}%
\pgfpathlineto{\pgfqpoint{4.418363in}{0.854526in}}%
\pgfpathlineto{\pgfqpoint{4.355570in}{0.870666in}}%
\pgfpathlineto{\pgfqpoint{4.293706in}{0.887318in}}%
\pgfpathlineto{\pgfqpoint{4.232773in}{0.904482in}}%
\pgfpathlineto{\pgfqpoint{4.172770in}{0.922159in}}%
\pgfpathlineto{\pgfqpoint{4.113698in}{0.940348in}}%
\pgfpathlineto{\pgfqpoint{4.055555in}{0.959050in}}%
\pgfpathlineto{\pgfqpoint{3.998343in}{0.978264in}}%
\pgfpathlineto{\pgfqpoint{3.942061in}{0.997990in}}%
\pgfpathlineto{\pgfqpoint{3.864681in}{1.026385in}}%
\pgfpathlineto{\pgfqpoint{3.755839in}{1.067887in}}%
\pgfpathlineto{\pgfqpoint{3.648857in}{1.110414in}}%
\pgfpathlineto{\pgfqpoint{3.543735in}{1.153966in}}%
\pgfpathlineto{\pgfqpoint{3.440475in}{1.198542in}}%
\pgfpathlineto{\pgfqpoint{3.339074in}{1.244143in}}%
\pgfpathlineto{\pgfqpoint{3.239535in}{1.290768in}}%
\pgfpathlineto{\pgfqpoint{3.141856in}{1.338419in}}%
\pgfpathlineto{\pgfqpoint{3.046037in}{1.387094in}}%
\pgfpathlineto{\pgfqpoint{2.952079in}{1.436794in}}%
\pgfpathlineto{\pgfqpoint{2.859982in}{1.487519in}}%
\pgfpathlineto{\pgfqpoint{2.769745in}{1.539268in}}%
\pgfpathlineto{\pgfqpoint{2.681368in}{1.592042in}}%
\pgfpathlineto{\pgfqpoint{2.594853in}{1.645841in}}%
\pgfpathlineto{\pgfqpoint{2.510197in}{1.700665in}}%
\pgfpathlineto{\pgfqpoint{2.427403in}{1.756513in}}%
\pgfpathlineto{\pgfqpoint{2.346469in}{1.813386in}}%
\pgfpathlineto{\pgfqpoint{2.267395in}{1.871284in}}%
\pgfpathlineto{\pgfqpoint{2.190182in}{1.930206in}}%
\pgfpathlineto{\pgfqpoint{2.114830in}{1.990154in}}%
\pgfpathlineto{\pgfqpoint{2.041338in}{2.051126in}}%
\pgfpathlineto{\pgfqpoint{1.969706in}{2.113122in}}%
\pgfpathlineto{\pgfqpoint{1.899935in}{2.176144in}}%
\pgfpathlineto{\pgfqpoint{1.832025in}{2.240190in}}%
\pgfpathlineto{\pgfqpoint{1.765976in}{2.305261in}}%
\pgfpathlineto{\pgfqpoint{1.701786in}{2.371357in}}%
\pgfpathlineto{\pgfqpoint{1.639458in}{2.438477in}}%
\pgfpathlineto{\pgfqpoint{1.578990in}{2.506623in}}%
\pgfpathlineto{\pgfqpoint{1.520382in}{2.575793in}}%
\pgfpathlineto{\pgfqpoint{1.463635in}{2.645987in}}%
\pgfpathlineto{\pgfqpoint{1.408749in}{2.717207in}}%
\pgfpathlineto{\pgfqpoint{1.355723in}{2.789451in}}%
\pgfpathlineto{\pgfqpoint{1.304558in}{2.862720in}}%
\pgfpathlineto{\pgfqpoint{1.255253in}{2.937014in}}%
\pgfpathlineto{\pgfqpoint{1.207809in}{3.012332in}}%
\pgfpathlineto{\pgfqpoint{1.162226in}{3.088675in}}%
\pgfpathlineto{\pgfqpoint{1.118503in}{3.166043in}}%
\pgfpathlineto{\pgfqpoint{1.076640in}{3.244436in}}%
\pgfpathlineto{\pgfqpoint{1.036638in}{3.323853in}}%
\pgfpathlineto{\pgfqpoint{0.998497in}{3.404295in}}%
\pgfpathlineto{\pgfqpoint{0.962216in}{3.485762in}}%
\pgfpathlineto{\pgfqpoint{0.927796in}{3.568253in}}%
\pgfpathlineto{\pgfqpoint{0.895236in}{3.651770in}}%
\pgfpathlineto{\pgfqpoint{0.864537in}{3.736311in}}%
\pgfpathlineto{\pgfqpoint{0.835698in}{3.821877in}}%
\pgfpathlineto{\pgfqpoint{0.808720in}{3.908467in}}%
\pgfpathlineto{\pgfqpoint{0.783603in}{3.996082in}}%
\pgfpathlineto{\pgfqpoint{0.760346in}{4.084722in}}%
\pgfpathlineto{\pgfqpoint{0.738950in}{4.174387in}}%
\pgfpathlineto{\pgfqpoint{0.719414in}{4.265077in}}%
\pgfpathlineto{\pgfqpoint{0.701738in}{4.356791in}}%
\pgfpathlineto{\pgfqpoint{0.685924in}{4.449530in}}%
\pgfpathlineto{\pgfqpoint{0.671970in}{4.543294in}}%
\pgfpathlineto{\pgfqpoint{0.659876in}{4.638082in}}%
\pgfpathlineto{\pgfqpoint{0.649643in}{4.733895in}}%
\pgfpathlineto{\pgfqpoint{0.641270in}{4.830733in}}%
\pgfpathlineto{\pgfqpoint{0.634758in}{4.928596in}}%
\pgfpathlineto{\pgfqpoint{0.630107in}{5.027483in}}%
\pgfpathlineto{\pgfqpoint{0.627316in}{5.127396in}}%
\pgfpathlineto{\pgfqpoint{0.626386in}{5.228333in}}%
\pgfpathlineto{\pgfqpoint{0.626386in}{4.391376in}}%
\pgfpathclose%
\pgfusepath{stroke,fill}%
\end{pgfscope}%
\begin{pgfscope}%
\pgfpathrectangle{\pgfqpoint{0.626386in}{0.608332in}}{\pgfqpoint{6.200000in}{4.620000in}}%
\pgfusepath{clip}%
\pgfsetbuttcap%
\pgfsetroundjoin%
\definecolor{currentfill}{rgb}{1.000000,1.000000,1.000000}%
\pgfsetfillcolor{currentfill}%
\pgfsetlinewidth{1.003750pt}%
\definecolor{currentstroke}{rgb}{1.000000,1.000000,1.000000}%
\pgfsetstrokecolor{currentstroke}%
\pgfsetdash{}{0pt}%
\pgfsys@defobject{currentmarker}{\pgfqpoint{0.626386in}{0.206593in}}{\pgfqpoint{5.689719in}{4.307680in}}{%
\pgfpathmoveto{\pgfqpoint{0.626386in}{0.206593in}}%
\pgfpathlineto{\pgfqpoint{0.626386in}{4.307680in}}%
\pgfpathlineto{\pgfqpoint{0.627146in}{4.225249in}}%
\pgfpathlineto{\pgfqpoint{0.629425in}{4.143654in}}%
\pgfpathlineto{\pgfqpoint{0.633224in}{4.062896in}}%
\pgfpathlineto{\pgfqpoint{0.638542in}{3.982974in}}%
\pgfpathlineto{\pgfqpoint{0.645379in}{3.903890in}}%
\pgfpathlineto{\pgfqpoint{0.653736in}{3.825643in}}%
\pgfpathlineto{\pgfqpoint{0.663613in}{3.748232in}}%
\pgfpathlineto{\pgfqpoint{0.675008in}{3.671658in}}%
\pgfpathlineto{\pgfqpoint{0.687924in}{3.595921in}}%
\pgfpathlineto{\pgfqpoint{0.702359in}{3.521021in}}%
\pgfpathlineto{\pgfqpoint{0.718313in}{3.446958in}}%
\pgfpathlineto{\pgfqpoint{0.735787in}{3.373732in}}%
\pgfpathlineto{\pgfqpoint{0.754780in}{3.301343in}}%
\pgfpathlineto{\pgfqpoint{0.775292in}{3.229790in}}%
\pgfpathlineto{\pgfqpoint{0.797324in}{3.159075in}}%
\pgfpathlineto{\pgfqpoint{0.820876in}{3.089196in}}%
\pgfpathlineto{\pgfqpoint{0.845947in}{3.020154in}}%
\pgfpathlineto{\pgfqpoint{0.872537in}{2.951949in}}%
\pgfpathlineto{\pgfqpoint{0.900647in}{2.884581in}}%
\pgfpathlineto{\pgfqpoint{0.930277in}{2.818050in}}%
\pgfpathlineto{\pgfqpoint{0.961425in}{2.752355in}}%
\pgfpathlineto{\pgfqpoint{0.994094in}{2.687498in}}%
\pgfpathlineto{\pgfqpoint{1.028281in}{2.623477in}}%
\pgfpathlineto{\pgfqpoint{1.063988in}{2.560293in}}%
\pgfpathlineto{\pgfqpoint{1.101215in}{2.497947in}}%
\pgfpathlineto{\pgfqpoint{1.139961in}{2.436436in}}%
\pgfpathlineto{\pgfqpoint{1.180227in}{2.375763in}}%
\pgfpathlineto{\pgfqpoint{1.222011in}{2.315927in}}%
\pgfpathlineto{\pgfqpoint{1.265316in}{2.256928in}}%
\pgfpathlineto{\pgfqpoint{1.310140in}{2.198765in}}%
\pgfpathlineto{\pgfqpoint{1.356483in}{2.141439in}}%
\pgfpathlineto{\pgfqpoint{1.404346in}{2.084951in}}%
\pgfpathlineto{\pgfqpoint{1.453728in}{2.029299in}}%
\pgfpathlineto{\pgfqpoint{1.504630in}{1.974484in}}%
\pgfpathlineto{\pgfqpoint{1.557051in}{1.920505in}}%
\pgfpathlineto{\pgfqpoint{1.610991in}{1.867364in}}%
\pgfpathlineto{\pgfqpoint{1.666451in}{1.815060in}}%
\pgfpathlineto{\pgfqpoint{1.723431in}{1.763592in}}%
\pgfpathlineto{\pgfqpoint{1.781930in}{1.712961in}}%
\pgfpathlineto{\pgfqpoint{1.841948in}{1.663168in}}%
\pgfpathlineto{\pgfqpoint{1.903486in}{1.614211in}}%
\pgfpathlineto{\pgfqpoint{1.966543in}{1.566091in}}%
\pgfpathlineto{\pgfqpoint{2.031120in}{1.518807in}}%
\pgfpathlineto{\pgfqpoint{2.097216in}{1.472361in}}%
\pgfpathlineto{\pgfqpoint{2.164832in}{1.426751in}}%
\pgfpathlineto{\pgfqpoint{2.233967in}{1.381979in}}%
\pgfpathlineto{\pgfqpoint{2.304622in}{1.338043in}}%
\pgfpathlineto{\pgfqpoint{2.376796in}{1.294944in}}%
\pgfpathlineto{\pgfqpoint{2.450489in}{1.252682in}}%
\pgfpathlineto{\pgfqpoint{2.525702in}{1.211257in}}%
\pgfpathlineto{\pgfqpoint{2.602434in}{1.170669in}}%
\pgfpathlineto{\pgfqpoint{2.680686in}{1.130918in}}%
\pgfpathlineto{\pgfqpoint{2.760457in}{1.092003in}}%
\pgfpathlineto{\pgfqpoint{2.841748in}{1.053925in}}%
\pgfpathlineto{\pgfqpoint{2.924558in}{1.016685in}}%
\pgfpathlineto{\pgfqpoint{3.008888in}{0.980281in}}%
\pgfpathlineto{\pgfqpoint{3.094737in}{0.944714in}}%
\pgfpathlineto{\pgfqpoint{3.182106in}{0.909984in}}%
\pgfpathlineto{\pgfqpoint{3.270994in}{0.876090in}}%
\pgfpathlineto{\pgfqpoint{3.334188in}{0.852901in}}%
\pgfpathlineto{\pgfqpoint{3.380151in}{0.836791in}}%
\pgfpathlineto{\pgfqpoint{3.426874in}{0.821099in}}%
\pgfpathlineto{\pgfqpoint{3.474357in}{0.805827in}}%
\pgfpathlineto{\pgfqpoint{3.522600in}{0.790972in}}%
\pgfpathlineto{\pgfqpoint{3.571602in}{0.776536in}}%
\pgfpathlineto{\pgfqpoint{3.621364in}{0.762518in}}%
\pgfpathlineto{\pgfqpoint{3.671886in}{0.748919in}}%
\pgfpathlineto{\pgfqpoint{3.723168in}{0.735739in}}%
\pgfpathlineto{\pgfqpoint{3.775209in}{0.722976in}}%
\pgfpathlineto{\pgfqpoint{3.828010in}{0.710632in}}%
\pgfpathlineto{\pgfqpoint{3.881570in}{0.698707in}}%
\pgfpathlineto{\pgfqpoint{3.935891in}{0.687200in}}%
\pgfpathlineto{\pgfqpoint{3.990971in}{0.676112in}}%
\pgfpathlineto{\pgfqpoint{4.046811in}{0.665441in}}%
\pgfpathlineto{\pgfqpoint{4.103411in}{0.655190in}}%
\pgfpathlineto{\pgfqpoint{4.160770in}{0.645357in}}%
\pgfpathlineto{\pgfqpoint{4.218889in}{0.635942in}}%
\pgfpathlineto{\pgfqpoint{4.277768in}{0.626945in}}%
\pgfpathlineto{\pgfqpoint{4.337406in}{0.618367in}}%
\pgfpathlineto{\pgfqpoint{4.397805in}{0.610208in}}%
\pgfpathlineto{\pgfqpoint{4.458963in}{0.602467in}}%
\pgfpathlineto{\pgfqpoint{4.520880in}{0.595144in}}%
\pgfpathlineto{\pgfqpoint{4.583558in}{0.588240in}}%
\pgfpathlineto{\pgfqpoint{4.646995in}{0.581754in}}%
\pgfpathlineto{\pgfqpoint{4.711192in}{0.575687in}}%
\pgfpathlineto{\pgfqpoint{4.776148in}{0.570038in}}%
\pgfpathlineto{\pgfqpoint{4.841865in}{0.564808in}}%
\pgfpathlineto{\pgfqpoint{4.908341in}{0.559996in}}%
\pgfpathlineto{\pgfqpoint{4.975576in}{0.555602in}}%
\pgfpathlineto{\pgfqpoint{5.043572in}{0.551627in}}%
\pgfpathlineto{\pgfqpoint{5.112327in}{0.548070in}}%
\pgfpathlineto{\pgfqpoint{5.181842in}{0.544932in}}%
\pgfpathlineto{\pgfqpoint{5.252117in}{0.542212in}}%
\pgfpathlineto{\pgfqpoint{5.323151in}{0.539911in}}%
\pgfpathlineto{\pgfqpoint{5.394945in}{0.538028in}}%
\pgfpathlineto{\pgfqpoint{5.467499in}{0.536563in}}%
\pgfpathlineto{\pgfqpoint{5.540813in}{0.535517in}}%
\pgfpathlineto{\pgfqpoint{5.614886in}{0.534890in}}%
\pgfpathlineto{\pgfqpoint{5.689719in}{0.534680in}}%
\pgfpathlineto{\pgfqpoint{5.689719in}{0.206593in}}%
\pgfpathlineto{\pgfqpoint{5.689719in}{0.206593in}}%
\pgfpathlineto{\pgfqpoint{5.614886in}{0.206593in}}%
\pgfpathlineto{\pgfqpoint{5.540813in}{0.206593in}}%
\pgfpathlineto{\pgfqpoint{5.467499in}{0.206593in}}%
\pgfpathlineto{\pgfqpoint{5.394945in}{0.206593in}}%
\pgfpathlineto{\pgfqpoint{5.323151in}{0.206593in}}%
\pgfpathlineto{\pgfqpoint{5.252117in}{0.206593in}}%
\pgfpathlineto{\pgfqpoint{5.181842in}{0.206593in}}%
\pgfpathlineto{\pgfqpoint{5.112327in}{0.206593in}}%
\pgfpathlineto{\pgfqpoint{5.043572in}{0.206593in}}%
\pgfpathlineto{\pgfqpoint{4.975576in}{0.206593in}}%
\pgfpathlineto{\pgfqpoint{4.908341in}{0.206593in}}%
\pgfpathlineto{\pgfqpoint{4.841865in}{0.206593in}}%
\pgfpathlineto{\pgfqpoint{4.776148in}{0.206593in}}%
\pgfpathlineto{\pgfqpoint{4.711192in}{0.206593in}}%
\pgfpathlineto{\pgfqpoint{4.646995in}{0.206593in}}%
\pgfpathlineto{\pgfqpoint{4.583558in}{0.206593in}}%
\pgfpathlineto{\pgfqpoint{4.520880in}{0.206593in}}%
\pgfpathlineto{\pgfqpoint{4.458963in}{0.206593in}}%
\pgfpathlineto{\pgfqpoint{4.397805in}{0.206593in}}%
\pgfpathlineto{\pgfqpoint{4.337406in}{0.206593in}}%
\pgfpathlineto{\pgfqpoint{4.277768in}{0.206593in}}%
\pgfpathlineto{\pgfqpoint{4.218889in}{0.206593in}}%
\pgfpathlineto{\pgfqpoint{4.160770in}{0.206593in}}%
\pgfpathlineto{\pgfqpoint{4.103411in}{0.206593in}}%
\pgfpathlineto{\pgfqpoint{4.046811in}{0.206593in}}%
\pgfpathlineto{\pgfqpoint{3.990971in}{0.206593in}}%
\pgfpathlineto{\pgfqpoint{3.935891in}{0.206593in}}%
\pgfpathlineto{\pgfqpoint{3.881570in}{0.206593in}}%
\pgfpathlineto{\pgfqpoint{3.828010in}{0.206593in}}%
\pgfpathlineto{\pgfqpoint{3.775209in}{0.206593in}}%
\pgfpathlineto{\pgfqpoint{3.723168in}{0.206593in}}%
\pgfpathlineto{\pgfqpoint{3.671886in}{0.206593in}}%
\pgfpathlineto{\pgfqpoint{3.621364in}{0.206593in}}%
\pgfpathlineto{\pgfqpoint{3.571602in}{0.206593in}}%
\pgfpathlineto{\pgfqpoint{3.522600in}{0.206593in}}%
\pgfpathlineto{\pgfqpoint{3.474357in}{0.206593in}}%
\pgfpathlineto{\pgfqpoint{3.426874in}{0.206593in}}%
\pgfpathlineto{\pgfqpoint{3.380151in}{0.206593in}}%
\pgfpathlineto{\pgfqpoint{3.334188in}{0.206593in}}%
\pgfpathlineto{\pgfqpoint{3.270994in}{0.206593in}}%
\pgfpathlineto{\pgfqpoint{3.182106in}{0.206593in}}%
\pgfpathlineto{\pgfqpoint{3.094737in}{0.206593in}}%
\pgfpathlineto{\pgfqpoint{3.008888in}{0.206593in}}%
\pgfpathlineto{\pgfqpoint{2.924558in}{0.206593in}}%
\pgfpathlineto{\pgfqpoint{2.841748in}{0.206593in}}%
\pgfpathlineto{\pgfqpoint{2.760457in}{0.206593in}}%
\pgfpathlineto{\pgfqpoint{2.680686in}{0.206593in}}%
\pgfpathlineto{\pgfqpoint{2.602434in}{0.206593in}}%
\pgfpathlineto{\pgfqpoint{2.525702in}{0.206593in}}%
\pgfpathlineto{\pgfqpoint{2.450489in}{0.206593in}}%
\pgfpathlineto{\pgfqpoint{2.376796in}{0.206593in}}%
\pgfpathlineto{\pgfqpoint{2.304622in}{0.206593in}}%
\pgfpathlineto{\pgfqpoint{2.233967in}{0.206593in}}%
\pgfpathlineto{\pgfqpoint{2.164832in}{0.206593in}}%
\pgfpathlineto{\pgfqpoint{2.097216in}{0.206593in}}%
\pgfpathlineto{\pgfqpoint{2.031120in}{0.206593in}}%
\pgfpathlineto{\pgfqpoint{1.966543in}{0.206593in}}%
\pgfpathlineto{\pgfqpoint{1.903486in}{0.206593in}}%
\pgfpathlineto{\pgfqpoint{1.841948in}{0.206593in}}%
\pgfpathlineto{\pgfqpoint{1.781930in}{0.206593in}}%
\pgfpathlineto{\pgfqpoint{1.723431in}{0.206593in}}%
\pgfpathlineto{\pgfqpoint{1.666451in}{0.206593in}}%
\pgfpathlineto{\pgfqpoint{1.610991in}{0.206593in}}%
\pgfpathlineto{\pgfqpoint{1.557051in}{0.206593in}}%
\pgfpathlineto{\pgfqpoint{1.504630in}{0.206593in}}%
\pgfpathlineto{\pgfqpoint{1.453728in}{0.206593in}}%
\pgfpathlineto{\pgfqpoint{1.404346in}{0.206593in}}%
\pgfpathlineto{\pgfqpoint{1.356483in}{0.206593in}}%
\pgfpathlineto{\pgfqpoint{1.310140in}{0.206593in}}%
\pgfpathlineto{\pgfqpoint{1.265316in}{0.206593in}}%
\pgfpathlineto{\pgfqpoint{1.222011in}{0.206593in}}%
\pgfpathlineto{\pgfqpoint{1.180227in}{0.206593in}}%
\pgfpathlineto{\pgfqpoint{1.139961in}{0.206593in}}%
\pgfpathlineto{\pgfqpoint{1.101215in}{0.206593in}}%
\pgfpathlineto{\pgfqpoint{1.063988in}{0.206593in}}%
\pgfpathlineto{\pgfqpoint{1.028281in}{0.206593in}}%
\pgfpathlineto{\pgfqpoint{0.994094in}{0.206593in}}%
\pgfpathlineto{\pgfqpoint{0.961425in}{0.206593in}}%
\pgfpathlineto{\pgfqpoint{0.930277in}{0.206593in}}%
\pgfpathlineto{\pgfqpoint{0.900647in}{0.206593in}}%
\pgfpathlineto{\pgfqpoint{0.872537in}{0.206593in}}%
\pgfpathlineto{\pgfqpoint{0.845947in}{0.206593in}}%
\pgfpathlineto{\pgfqpoint{0.820876in}{0.206593in}}%
\pgfpathlineto{\pgfqpoint{0.797324in}{0.206593in}}%
\pgfpathlineto{\pgfqpoint{0.775292in}{0.206593in}}%
\pgfpathlineto{\pgfqpoint{0.754780in}{0.206593in}}%
\pgfpathlineto{\pgfqpoint{0.735787in}{0.206593in}}%
\pgfpathlineto{\pgfqpoint{0.718313in}{0.206593in}}%
\pgfpathlineto{\pgfqpoint{0.702359in}{0.206593in}}%
\pgfpathlineto{\pgfqpoint{0.687924in}{0.206593in}}%
\pgfpathlineto{\pgfqpoint{0.675008in}{0.206593in}}%
\pgfpathlineto{\pgfqpoint{0.663613in}{0.206593in}}%
\pgfpathlineto{\pgfqpoint{0.653736in}{0.206593in}}%
\pgfpathlineto{\pgfqpoint{0.645379in}{0.206593in}}%
\pgfpathlineto{\pgfqpoint{0.638542in}{0.206593in}}%
\pgfpathlineto{\pgfqpoint{0.633224in}{0.206593in}}%
\pgfpathlineto{\pgfqpoint{0.629425in}{0.206593in}}%
\pgfpathlineto{\pgfqpoint{0.627146in}{0.206593in}}%
\pgfpathlineto{\pgfqpoint{0.626386in}{0.206593in}}%
\pgfpathlineto{\pgfqpoint{0.626386in}{0.206593in}}%
\pgfpathclose%
\pgfusepath{stroke,fill}%
}%
\begin{pgfscope}%
\pgfsys@transformshift{0.000000in}{0.000000in}%
\pgfsys@useobject{currentmarker}{}%
\end{pgfscope}%
\end{pgfscope}%
\begin{pgfscope}%
\pgfsetbuttcap%
\pgfsetroundjoin%
\definecolor{currentfill}{rgb}{0.000000,0.000000,0.000000}%
\pgfsetfillcolor{currentfill}%
\pgfsetlinewidth{0.803000pt}%
\definecolor{currentstroke}{rgb}{0.000000,0.000000,0.000000}%
\pgfsetstrokecolor{currentstroke}%
\pgfsetdash{}{0pt}%
\pgfsys@defobject{currentmarker}{\pgfqpoint{0.000000in}{-0.048611in}}{\pgfqpoint{0.000000in}{0.000000in}}{%
\pgfpathmoveto{\pgfqpoint{0.000000in}{0.000000in}}%
\pgfpathlineto{\pgfqpoint{0.000000in}{-0.048611in}}%
\pgfusepath{stroke,fill}%
}%
\begin{pgfscope}%
\pgfsys@transformshift{0.626386in}{0.608332in}%
\pgfsys@useobject{currentmarker}{}%
\end{pgfscope}%
\end{pgfscope}%
\begin{pgfscope}%
\definecolor{textcolor}{rgb}{0.000000,0.000000,0.000000}%
\pgfsetstrokecolor{textcolor}%
\pgfsetfillcolor{textcolor}%
\pgftext[x=0.626386in,y=0.511110in,,top]{\color{textcolor}{\rmfamily\fontsize{14.000000}{16.800000}\selectfont\catcode`\^=\active\def^{\ifmmode\sp\else\^{}\fi}\catcode`\%=\active\def%{\%}$\mathdefault{0}$}}%
\end{pgfscope}%
\begin{pgfscope}%
\pgfsetbuttcap%
\pgfsetroundjoin%
\definecolor{currentfill}{rgb}{0.000000,0.000000,0.000000}%
\pgfsetfillcolor{currentfill}%
\pgfsetlinewidth{0.803000pt}%
\definecolor{currentstroke}{rgb}{0.000000,0.000000,0.000000}%
\pgfsetstrokecolor{currentstroke}%
\pgfsetdash{}{0pt}%
\pgfsys@defobject{currentmarker}{\pgfqpoint{0.000000in}{-0.048611in}}{\pgfqpoint{0.000000in}{0.000000in}}{%
\pgfpathmoveto{\pgfqpoint{0.000000in}{0.000000in}}%
\pgfpathlineto{\pgfqpoint{0.000000in}{-0.048611in}}%
\pgfusepath{stroke,fill}%
}%
\begin{pgfscope}%
\pgfsys@transformshift{1.386190in}{0.608332in}%
\pgfsys@useobject{currentmarker}{}%
\end{pgfscope}%
\end{pgfscope}%
\begin{pgfscope}%
\definecolor{textcolor}{rgb}{0.000000,0.000000,0.000000}%
\pgfsetstrokecolor{textcolor}%
\pgfsetfillcolor{textcolor}%
\pgftext[x=1.386190in,y=0.511110in,,top]{\color{textcolor}{\rmfamily\fontsize{14.000000}{16.800000}\selectfont\catcode`\^=\active\def^{\ifmmode\sp\else\^{}\fi}\catcode`\%=\active\def%{\%}$\mathdefault{20}$}}%
\end{pgfscope}%
\begin{pgfscope}%
\pgfsetbuttcap%
\pgfsetroundjoin%
\definecolor{currentfill}{rgb}{0.000000,0.000000,0.000000}%
\pgfsetfillcolor{currentfill}%
\pgfsetlinewidth{0.803000pt}%
\definecolor{currentstroke}{rgb}{0.000000,0.000000,0.000000}%
\pgfsetstrokecolor{currentstroke}%
\pgfsetdash{}{0pt}%
\pgfsys@defobject{currentmarker}{\pgfqpoint{0.000000in}{-0.048611in}}{\pgfqpoint{0.000000in}{0.000000in}}{%
\pgfpathmoveto{\pgfqpoint{0.000000in}{0.000000in}}%
\pgfpathlineto{\pgfqpoint{0.000000in}{-0.048611in}}%
\pgfusepath{stroke,fill}%
}%
\begin{pgfscope}%
\pgfsys@transformshift{2.145994in}{0.608332in}%
\pgfsys@useobject{currentmarker}{}%
\end{pgfscope}%
\end{pgfscope}%
\begin{pgfscope}%
\definecolor{textcolor}{rgb}{0.000000,0.000000,0.000000}%
\pgfsetstrokecolor{textcolor}%
\pgfsetfillcolor{textcolor}%
\pgftext[x=2.145994in,y=0.511110in,,top]{\color{textcolor}{\rmfamily\fontsize{14.000000}{16.800000}\selectfont\catcode`\^=\active\def^{\ifmmode\sp\else\^{}\fi}\catcode`\%=\active\def%{\%}$\mathdefault{40}$}}%
\end{pgfscope}%
\begin{pgfscope}%
\pgfsetbuttcap%
\pgfsetroundjoin%
\definecolor{currentfill}{rgb}{0.000000,0.000000,0.000000}%
\pgfsetfillcolor{currentfill}%
\pgfsetlinewidth{0.803000pt}%
\definecolor{currentstroke}{rgb}{0.000000,0.000000,0.000000}%
\pgfsetstrokecolor{currentstroke}%
\pgfsetdash{}{0pt}%
\pgfsys@defobject{currentmarker}{\pgfqpoint{0.000000in}{-0.048611in}}{\pgfqpoint{0.000000in}{0.000000in}}{%
\pgfpathmoveto{\pgfqpoint{0.000000in}{0.000000in}}%
\pgfpathlineto{\pgfqpoint{0.000000in}{-0.048611in}}%
\pgfusepath{stroke,fill}%
}%
\begin{pgfscope}%
\pgfsys@transformshift{2.905798in}{0.608332in}%
\pgfsys@useobject{currentmarker}{}%
\end{pgfscope}%
\end{pgfscope}%
\begin{pgfscope}%
\definecolor{textcolor}{rgb}{0.000000,0.000000,0.000000}%
\pgfsetstrokecolor{textcolor}%
\pgfsetfillcolor{textcolor}%
\pgftext[x=2.905798in,y=0.511110in,,top]{\color{textcolor}{\rmfamily\fontsize{14.000000}{16.800000}\selectfont\catcode`\^=\active\def^{\ifmmode\sp\else\^{}\fi}\catcode`\%=\active\def%{\%}$\mathdefault{60}$}}%
\end{pgfscope}%
\begin{pgfscope}%
\pgfsetbuttcap%
\pgfsetroundjoin%
\definecolor{currentfill}{rgb}{0.000000,0.000000,0.000000}%
\pgfsetfillcolor{currentfill}%
\pgfsetlinewidth{0.803000pt}%
\definecolor{currentstroke}{rgb}{0.000000,0.000000,0.000000}%
\pgfsetstrokecolor{currentstroke}%
\pgfsetdash{}{0pt}%
\pgfsys@defobject{currentmarker}{\pgfqpoint{0.000000in}{-0.048611in}}{\pgfqpoint{0.000000in}{0.000000in}}{%
\pgfpathmoveto{\pgfqpoint{0.000000in}{0.000000in}}%
\pgfpathlineto{\pgfqpoint{0.000000in}{-0.048611in}}%
\pgfusepath{stroke,fill}%
}%
\begin{pgfscope}%
\pgfsys@transformshift{3.665602in}{0.608332in}%
\pgfsys@useobject{currentmarker}{}%
\end{pgfscope}%
\end{pgfscope}%
\begin{pgfscope}%
\definecolor{textcolor}{rgb}{0.000000,0.000000,0.000000}%
\pgfsetstrokecolor{textcolor}%
\pgfsetfillcolor{textcolor}%
\pgftext[x=3.665602in,y=0.511110in,,top]{\color{textcolor}{\rmfamily\fontsize{14.000000}{16.800000}\selectfont\catcode`\^=\active\def^{\ifmmode\sp\else\^{}\fi}\catcode`\%=\active\def%{\%}$\mathdefault{80}$}}%
\end{pgfscope}%
\begin{pgfscope}%
\pgfsetbuttcap%
\pgfsetroundjoin%
\definecolor{currentfill}{rgb}{0.000000,0.000000,0.000000}%
\pgfsetfillcolor{currentfill}%
\pgfsetlinewidth{0.803000pt}%
\definecolor{currentstroke}{rgb}{0.000000,0.000000,0.000000}%
\pgfsetstrokecolor{currentstroke}%
\pgfsetdash{}{0pt}%
\pgfsys@defobject{currentmarker}{\pgfqpoint{0.000000in}{-0.048611in}}{\pgfqpoint{0.000000in}{0.000000in}}{%
\pgfpathmoveto{\pgfqpoint{0.000000in}{0.000000in}}%
\pgfpathlineto{\pgfqpoint{0.000000in}{-0.048611in}}%
\pgfusepath{stroke,fill}%
}%
\begin{pgfscope}%
\pgfsys@transformshift{4.425406in}{0.608332in}%
\pgfsys@useobject{currentmarker}{}%
\end{pgfscope}%
\end{pgfscope}%
\begin{pgfscope}%
\definecolor{textcolor}{rgb}{0.000000,0.000000,0.000000}%
\pgfsetstrokecolor{textcolor}%
\pgfsetfillcolor{textcolor}%
\pgftext[x=4.425406in,y=0.511110in,,top]{\color{textcolor}{\rmfamily\fontsize{14.000000}{16.800000}\selectfont\catcode`\^=\active\def^{\ifmmode\sp\else\^{}\fi}\catcode`\%=\active\def%{\%}$\mathdefault{100}$}}%
\end{pgfscope}%
\begin{pgfscope}%
\pgfsetbuttcap%
\pgfsetroundjoin%
\definecolor{currentfill}{rgb}{0.000000,0.000000,0.000000}%
\pgfsetfillcolor{currentfill}%
\pgfsetlinewidth{0.803000pt}%
\definecolor{currentstroke}{rgb}{0.000000,0.000000,0.000000}%
\pgfsetstrokecolor{currentstroke}%
\pgfsetdash{}{0pt}%
\pgfsys@defobject{currentmarker}{\pgfqpoint{0.000000in}{-0.048611in}}{\pgfqpoint{0.000000in}{0.000000in}}{%
\pgfpathmoveto{\pgfqpoint{0.000000in}{0.000000in}}%
\pgfpathlineto{\pgfqpoint{0.000000in}{-0.048611in}}%
\pgfusepath{stroke,fill}%
}%
\begin{pgfscope}%
\pgfsys@transformshift{5.185210in}{0.608332in}%
\pgfsys@useobject{currentmarker}{}%
\end{pgfscope}%
\end{pgfscope}%
\begin{pgfscope}%
\definecolor{textcolor}{rgb}{0.000000,0.000000,0.000000}%
\pgfsetstrokecolor{textcolor}%
\pgfsetfillcolor{textcolor}%
\pgftext[x=5.185210in,y=0.511110in,,top]{\color{textcolor}{\rmfamily\fontsize{14.000000}{16.800000}\selectfont\catcode`\^=\active\def^{\ifmmode\sp\else\^{}\fi}\catcode`\%=\active\def%{\%}$\mathdefault{120}$}}%
\end{pgfscope}%
\begin{pgfscope}%
\pgfsetbuttcap%
\pgfsetroundjoin%
\definecolor{currentfill}{rgb}{0.000000,0.000000,0.000000}%
\pgfsetfillcolor{currentfill}%
\pgfsetlinewidth{0.803000pt}%
\definecolor{currentstroke}{rgb}{0.000000,0.000000,0.000000}%
\pgfsetstrokecolor{currentstroke}%
\pgfsetdash{}{0pt}%
\pgfsys@defobject{currentmarker}{\pgfqpoint{0.000000in}{-0.048611in}}{\pgfqpoint{0.000000in}{0.000000in}}{%
\pgfpathmoveto{\pgfqpoint{0.000000in}{0.000000in}}%
\pgfpathlineto{\pgfqpoint{0.000000in}{-0.048611in}}%
\pgfusepath{stroke,fill}%
}%
\begin{pgfscope}%
\pgfsys@transformshift{5.945013in}{0.608332in}%
\pgfsys@useobject{currentmarker}{}%
\end{pgfscope}%
\end{pgfscope}%
\begin{pgfscope}%
\definecolor{textcolor}{rgb}{0.000000,0.000000,0.000000}%
\pgfsetstrokecolor{textcolor}%
\pgfsetfillcolor{textcolor}%
\pgftext[x=5.945013in,y=0.511110in,,top]{\color{textcolor}{\rmfamily\fontsize{14.000000}{16.800000}\selectfont\catcode`\^=\active\def^{\ifmmode\sp\else\^{}\fi}\catcode`\%=\active\def%{\%}$\mathdefault{140}$}}%
\end{pgfscope}%
\begin{pgfscope}%
\pgfsetbuttcap%
\pgfsetroundjoin%
\definecolor{currentfill}{rgb}{0.000000,0.000000,0.000000}%
\pgfsetfillcolor{currentfill}%
\pgfsetlinewidth{0.803000pt}%
\definecolor{currentstroke}{rgb}{0.000000,0.000000,0.000000}%
\pgfsetstrokecolor{currentstroke}%
\pgfsetdash{}{0pt}%
\pgfsys@defobject{currentmarker}{\pgfqpoint{0.000000in}{-0.048611in}}{\pgfqpoint{0.000000in}{0.000000in}}{%
\pgfpathmoveto{\pgfqpoint{0.000000in}{0.000000in}}%
\pgfpathlineto{\pgfqpoint{0.000000in}{-0.048611in}}%
\pgfusepath{stroke,fill}%
}%
\begin{pgfscope}%
\pgfsys@transformshift{6.704817in}{0.608332in}%
\pgfsys@useobject{currentmarker}{}%
\end{pgfscope}%
\end{pgfscope}%
\begin{pgfscope}%
\definecolor{textcolor}{rgb}{0.000000,0.000000,0.000000}%
\pgfsetstrokecolor{textcolor}%
\pgfsetfillcolor{textcolor}%
\pgftext[x=6.704817in,y=0.511110in,,top]{\color{textcolor}{\rmfamily\fontsize{14.000000}{16.800000}\selectfont\catcode`\^=\active\def^{\ifmmode\sp\else\^{}\fi}\catcode`\%=\active\def%{\%}$\mathdefault{160}$}}%
\end{pgfscope}%
\begin{pgfscope}%
\definecolor{textcolor}{rgb}{0.000000,0.000000,0.000000}%
\pgfsetstrokecolor{textcolor}%
\pgfsetfillcolor{textcolor}%
\pgftext[x=3.726386in,y=0.277777in,,top]{\color{textcolor}{\rmfamily\fontsize{14.000000}{16.800000}\selectfont\catcode`\^=\active\def^{\ifmmode\sp\else\^{}\fi}\catcode`\%=\active\def%{\%}f1}}%
\end{pgfscope}%
\begin{pgfscope}%
\pgfsetbuttcap%
\pgfsetroundjoin%
\definecolor{currentfill}{rgb}{0.000000,0.000000,0.000000}%
\pgfsetfillcolor{currentfill}%
\pgfsetlinewidth{0.803000pt}%
\definecolor{currentstroke}{rgb}{0.000000,0.000000,0.000000}%
\pgfsetstrokecolor{currentstroke}%
\pgfsetdash{}{0pt}%
\pgfsys@defobject{currentmarker}{\pgfqpoint{-0.048611in}{0.000000in}}{\pgfqpoint{-0.000000in}{0.000000in}}{%
\pgfpathmoveto{\pgfqpoint{-0.000000in}{0.000000in}}%
\pgfpathlineto{\pgfqpoint{-0.048611in}{0.000000in}}%
\pgfusepath{stroke,fill}%
}%
\begin{pgfscope}%
\pgfsys@transformshift{0.626386in}{1.043550in}%
\pgfsys@useobject{currentmarker}{}%
\end{pgfscope}%
\end{pgfscope}%
\begin{pgfscope}%
\definecolor{textcolor}{rgb}{0.000000,0.000000,0.000000}%
\pgfsetstrokecolor{textcolor}%
\pgfsetfillcolor{textcolor}%
\pgftext[x=0.333333in, y=0.974106in, left, base]{\color{textcolor}{\rmfamily\fontsize{14.000000}{16.800000}\selectfont\catcode`\^=\active\def^{\ifmmode\sp\else\^{}\fi}\catcode`\%=\active\def%{\%}$\mathdefault{10}$}}%
\end{pgfscope}%
\begin{pgfscope}%
\pgfsetbuttcap%
\pgfsetroundjoin%
\definecolor{currentfill}{rgb}{0.000000,0.000000,0.000000}%
\pgfsetfillcolor{currentfill}%
\pgfsetlinewidth{0.803000pt}%
\definecolor{currentstroke}{rgb}{0.000000,0.000000,0.000000}%
\pgfsetstrokecolor{currentstroke}%
\pgfsetdash{}{0pt}%
\pgfsys@defobject{currentmarker}{\pgfqpoint{-0.048611in}{0.000000in}}{\pgfqpoint{-0.000000in}{0.000000in}}{%
\pgfpathmoveto{\pgfqpoint{-0.000000in}{0.000000in}}%
\pgfpathlineto{\pgfqpoint{-0.048611in}{0.000000in}}%
\pgfusepath{stroke,fill}%
}%
\begin{pgfscope}%
\pgfsys@transformshift{0.626386in}{1.880506in}%
\pgfsys@useobject{currentmarker}{}%
\end{pgfscope}%
\end{pgfscope}%
\begin{pgfscope}%
\definecolor{textcolor}{rgb}{0.000000,0.000000,0.000000}%
\pgfsetstrokecolor{textcolor}%
\pgfsetfillcolor{textcolor}%
\pgftext[x=0.333333in, y=1.811062in, left, base]{\color{textcolor}{\rmfamily\fontsize{14.000000}{16.800000}\selectfont\catcode`\^=\active\def^{\ifmmode\sp\else\^{}\fi}\catcode`\%=\active\def%{\%}$\mathdefault{20}$}}%
\end{pgfscope}%
\begin{pgfscope}%
\pgfsetbuttcap%
\pgfsetroundjoin%
\definecolor{currentfill}{rgb}{0.000000,0.000000,0.000000}%
\pgfsetfillcolor{currentfill}%
\pgfsetlinewidth{0.803000pt}%
\definecolor{currentstroke}{rgb}{0.000000,0.000000,0.000000}%
\pgfsetstrokecolor{currentstroke}%
\pgfsetdash{}{0pt}%
\pgfsys@defobject{currentmarker}{\pgfqpoint{-0.048611in}{0.000000in}}{\pgfqpoint{-0.000000in}{0.000000in}}{%
\pgfpathmoveto{\pgfqpoint{-0.000000in}{0.000000in}}%
\pgfpathlineto{\pgfqpoint{-0.048611in}{0.000000in}}%
\pgfusepath{stroke,fill}%
}%
\begin{pgfscope}%
\pgfsys@transformshift{0.626386in}{2.717463in}%
\pgfsys@useobject{currentmarker}{}%
\end{pgfscope}%
\end{pgfscope}%
\begin{pgfscope}%
\definecolor{textcolor}{rgb}{0.000000,0.000000,0.000000}%
\pgfsetstrokecolor{textcolor}%
\pgfsetfillcolor{textcolor}%
\pgftext[x=0.333333in, y=2.648019in, left, base]{\color{textcolor}{\rmfamily\fontsize{14.000000}{16.800000}\selectfont\catcode`\^=\active\def^{\ifmmode\sp\else\^{}\fi}\catcode`\%=\active\def%{\%}$\mathdefault{30}$}}%
\end{pgfscope}%
\begin{pgfscope}%
\pgfsetbuttcap%
\pgfsetroundjoin%
\definecolor{currentfill}{rgb}{0.000000,0.000000,0.000000}%
\pgfsetfillcolor{currentfill}%
\pgfsetlinewidth{0.803000pt}%
\definecolor{currentstroke}{rgb}{0.000000,0.000000,0.000000}%
\pgfsetstrokecolor{currentstroke}%
\pgfsetdash{}{0pt}%
\pgfsys@defobject{currentmarker}{\pgfqpoint{-0.048611in}{0.000000in}}{\pgfqpoint{-0.000000in}{0.000000in}}{%
\pgfpathmoveto{\pgfqpoint{-0.000000in}{0.000000in}}%
\pgfpathlineto{\pgfqpoint{-0.048611in}{0.000000in}}%
\pgfusepath{stroke,fill}%
}%
\begin{pgfscope}%
\pgfsys@transformshift{0.626386in}{3.554419in}%
\pgfsys@useobject{currentmarker}{}%
\end{pgfscope}%
\end{pgfscope}%
\begin{pgfscope}%
\definecolor{textcolor}{rgb}{0.000000,0.000000,0.000000}%
\pgfsetstrokecolor{textcolor}%
\pgfsetfillcolor{textcolor}%
\pgftext[x=0.333333in, y=3.484975in, left, base]{\color{textcolor}{\rmfamily\fontsize{14.000000}{16.800000}\selectfont\catcode`\^=\active\def^{\ifmmode\sp\else\^{}\fi}\catcode`\%=\active\def%{\%}$\mathdefault{40}$}}%
\end{pgfscope}%
\begin{pgfscope}%
\pgfsetbuttcap%
\pgfsetroundjoin%
\definecolor{currentfill}{rgb}{0.000000,0.000000,0.000000}%
\pgfsetfillcolor{currentfill}%
\pgfsetlinewidth{0.803000pt}%
\definecolor{currentstroke}{rgb}{0.000000,0.000000,0.000000}%
\pgfsetstrokecolor{currentstroke}%
\pgfsetdash{}{0pt}%
\pgfsys@defobject{currentmarker}{\pgfqpoint{-0.048611in}{0.000000in}}{\pgfqpoint{-0.000000in}{0.000000in}}{%
\pgfpathmoveto{\pgfqpoint{-0.000000in}{0.000000in}}%
\pgfpathlineto{\pgfqpoint{-0.048611in}{0.000000in}}%
\pgfusepath{stroke,fill}%
}%
\begin{pgfscope}%
\pgfsys@transformshift{0.626386in}{4.391376in}%
\pgfsys@useobject{currentmarker}{}%
\end{pgfscope}%
\end{pgfscope}%
\begin{pgfscope}%
\definecolor{textcolor}{rgb}{0.000000,0.000000,0.000000}%
\pgfsetstrokecolor{textcolor}%
\pgfsetfillcolor{textcolor}%
\pgftext[x=0.333333in, y=4.321932in, left, base]{\color{textcolor}{\rmfamily\fontsize{14.000000}{16.800000}\selectfont\catcode`\^=\active\def^{\ifmmode\sp\else\^{}\fi}\catcode`\%=\active\def%{\%}$\mathdefault{50}$}}%
\end{pgfscope}%
\begin{pgfscope}%
\pgfsetbuttcap%
\pgfsetroundjoin%
\definecolor{currentfill}{rgb}{0.000000,0.000000,0.000000}%
\pgfsetfillcolor{currentfill}%
\pgfsetlinewidth{0.803000pt}%
\definecolor{currentstroke}{rgb}{0.000000,0.000000,0.000000}%
\pgfsetstrokecolor{currentstroke}%
\pgfsetdash{}{0pt}%
\pgfsys@defobject{currentmarker}{\pgfqpoint{-0.048611in}{0.000000in}}{\pgfqpoint{-0.000000in}{0.000000in}}{%
\pgfpathmoveto{\pgfqpoint{-0.000000in}{0.000000in}}%
\pgfpathlineto{\pgfqpoint{-0.048611in}{0.000000in}}%
\pgfusepath{stroke,fill}%
}%
\begin{pgfscope}%
\pgfsys@transformshift{0.626386in}{5.228333in}%
\pgfsys@useobject{currentmarker}{}%
\end{pgfscope}%
\end{pgfscope}%
\begin{pgfscope}%
\definecolor{textcolor}{rgb}{0.000000,0.000000,0.000000}%
\pgfsetstrokecolor{textcolor}%
\pgfsetfillcolor{textcolor}%
\pgftext[x=0.333333in, y=5.158888in, left, base]{\color{textcolor}{\rmfamily\fontsize{14.000000}{16.800000}\selectfont\catcode`\^=\active\def^{\ifmmode\sp\else\^{}\fi}\catcode`\%=\active\def%{\%}$\mathdefault{60}$}}%
\end{pgfscope}%
\begin{pgfscope}%
\definecolor{textcolor}{rgb}{0.000000,0.000000,0.000000}%
\pgfsetstrokecolor{textcolor}%
\pgfsetfillcolor{textcolor}%
\pgftext[x=0.277777in,y=2.918332in,,bottom,rotate=90.000000]{\color{textcolor}{\rmfamily\fontsize{14.000000}{16.800000}\selectfont\catcode`\^=\active\def^{\ifmmode\sp\else\^{}\fi}\catcode`\%=\active\def%{\%}f2}}%
\end{pgfscope}%
\begin{pgfscope}%
\pgfpathrectangle{\pgfqpoint{0.626386in}{0.608332in}}{\pgfqpoint{6.200000in}{4.620000in}}%
\pgfusepath{clip}%
\pgfsetrectcap%
\pgfsetroundjoin%
\pgfsetlinewidth{3.011250pt}%
\definecolor{currentstroke}{rgb}{0.000000,0.000000,0.000000}%
\pgfsetstrokecolor{currentstroke}%
\pgfsetdash{}{0pt}%
\pgfpathmoveto{\pgfqpoint{0.626386in}{4.391376in}}%
\pgfpathlineto{\pgfqpoint{0.627161in}{4.307262in}}%
\pgfpathlineto{\pgfqpoint{0.629487in}{4.224002in}}%
\pgfpathlineto{\pgfqpoint{0.633363in}{4.141596in}}%
\pgfpathlineto{\pgfqpoint{0.638790in}{4.060043in}}%
\pgfpathlineto{\pgfqpoint{0.645767in}{3.979345in}}%
\pgfpathlineto{\pgfqpoint{0.654294in}{3.899501in}}%
\pgfpathlineto{\pgfqpoint{0.664372in}{3.820510in}}%
\pgfpathlineto{\pgfqpoint{0.676001in}{3.742374in}}%
\pgfpathlineto{\pgfqpoint{0.689180in}{3.665091in}}%
\pgfpathlineto{\pgfqpoint{0.703909in}{3.588663in}}%
\pgfpathlineto{\pgfqpoint{0.720189in}{3.513088in}}%
\pgfpathlineto{\pgfqpoint{0.738019in}{3.438368in}}%
\pgfpathlineto{\pgfqpoint{0.757400in}{3.364501in}}%
\pgfpathlineto{\pgfqpoint{0.778331in}{3.291488in}}%
\pgfpathlineto{\pgfqpoint{0.800813in}{3.219329in}}%
\pgfpathlineto{\pgfqpoint{0.824845in}{3.148025in}}%
\pgfpathlineto{\pgfqpoint{0.850428in}{3.077574in}}%
\pgfpathlineto{\pgfqpoint{0.877561in}{3.007977in}}%
\pgfpathlineto{\pgfqpoint{0.906244in}{2.939234in}}%
\pgfpathlineto{\pgfqpoint{0.936478in}{2.871345in}}%
\pgfpathlineto{\pgfqpoint{0.968263in}{2.804310in}}%
\pgfpathlineto{\pgfqpoint{1.001598in}{2.738129in}}%
\pgfpathlineto{\pgfqpoint{1.036483in}{2.672801in}}%
\pgfpathlineto{\pgfqpoint{1.072919in}{2.608328in}}%
\pgfpathlineto{\pgfqpoint{1.110905in}{2.544709in}}%
\pgfpathlineto{\pgfqpoint{1.150442in}{2.481943in}}%
\pgfpathlineto{\pgfqpoint{1.191529in}{2.420032in}}%
\pgfpathlineto{\pgfqpoint{1.234167in}{2.358975in}}%
\pgfpathlineto{\pgfqpoint{1.278355in}{2.298771in}}%
\pgfpathlineto{\pgfqpoint{1.324094in}{2.239422in}}%
\pgfpathlineto{\pgfqpoint{1.371383in}{2.180926in}}%
\pgfpathlineto{\pgfqpoint{1.420223in}{2.123284in}}%
\pgfpathlineto{\pgfqpoint{1.470613in}{2.066497in}}%
\pgfpathlineto{\pgfqpoint{1.522553in}{2.010563in}}%
\pgfpathlineto{\pgfqpoint{1.576044in}{1.955483in}}%
\pgfpathlineto{\pgfqpoint{1.631085in}{1.901257in}}%
\pgfpathlineto{\pgfqpoint{1.687677in}{1.847886in}}%
\pgfpathlineto{\pgfqpoint{1.745820in}{1.795368in}}%
\pgfpathlineto{\pgfqpoint{1.805512in}{1.743704in}}%
\pgfpathlineto{\pgfqpoint{1.866756in}{1.692894in}}%
\pgfpathlineto{\pgfqpoint{1.929549in}{1.642937in}}%
\pgfpathlineto{\pgfqpoint{1.993893in}{1.593835in}}%
\pgfpathlineto{\pgfqpoint{2.059788in}{1.545587in}}%
\pgfpathlineto{\pgfqpoint{2.127233in}{1.498193in}}%
\pgfpathlineto{\pgfqpoint{2.196229in}{1.451653in}}%
\pgfpathlineto{\pgfqpoint{2.266775in}{1.405966in}}%
\pgfpathlineto{\pgfqpoint{2.338871in}{1.361134in}}%
\pgfpathlineto{\pgfqpoint{2.412518in}{1.317156in}}%
\pgfpathlineto{\pgfqpoint{2.487716in}{1.274031in}}%
\pgfpathlineto{\pgfqpoint{2.564464in}{1.231760in}}%
\pgfpathlineto{\pgfqpoint{2.642762in}{1.190344in}}%
\pgfpathlineto{\pgfqpoint{2.722611in}{1.149781in}}%
\pgfpathlineto{\pgfqpoint{2.804010in}{1.110073in}}%
\pgfpathlineto{\pgfqpoint{2.886960in}{1.071218in}}%
\pgfpathlineto{\pgfqpoint{2.971460in}{1.033217in}}%
\pgfpathlineto{\pgfqpoint{3.057510in}{0.996070in}}%
\pgfpathlineto{\pgfqpoint{3.145112in}{0.959777in}}%
\pgfpathlineto{\pgfqpoint{3.234263in}{0.924338in}}%
\pgfpathlineto{\pgfqpoint{3.324965in}{0.889753in}}%
\pgfpathlineto{\pgfqpoint{3.389449in}{0.866091in}}%
\pgfpathlineto{\pgfqpoint{3.436350in}{0.849652in}}%
\pgfpathlineto{\pgfqpoint{3.484027in}{0.833640in}}%
\pgfpathlineto{\pgfqpoint{3.532479in}{0.818056in}}%
\pgfpathlineto{\pgfqpoint{3.581706in}{0.802898in}}%
\pgfpathlineto{\pgfqpoint{3.631709in}{0.788168in}}%
\pgfpathlineto{\pgfqpoint{3.682486in}{0.773864in}}%
\pgfpathlineto{\pgfqpoint{3.734039in}{0.759987in}}%
\pgfpathlineto{\pgfqpoint{3.786367in}{0.746537in}}%
\pgfpathlineto{\pgfqpoint{3.839470in}{0.733515in}}%
\pgfpathlineto{\pgfqpoint{3.893349in}{0.720919in}}%
\pgfpathlineto{\pgfqpoint{3.948003in}{0.708750in}}%
\pgfpathlineto{\pgfqpoint{4.003432in}{0.697008in}}%
\pgfpathlineto{\pgfqpoint{4.059636in}{0.685694in}}%
\pgfpathlineto{\pgfqpoint{4.116616in}{0.674806in}}%
\pgfpathlineto{\pgfqpoint{4.174370in}{0.664345in}}%
\pgfpathlineto{\pgfqpoint{4.232900in}{0.654311in}}%
\pgfpathlineto{\pgfqpoint{4.292205in}{0.644704in}}%
\pgfpathlineto{\pgfqpoint{4.352286in}{0.635524in}}%
\pgfpathlineto{\pgfqpoint{4.413141in}{0.626771in}}%
\pgfpathlineto{\pgfqpoint{4.474772in}{0.618445in}}%
\pgfpathlineto{\pgfqpoint{4.537178in}{0.610546in}}%
\pgfpathlineto{\pgfqpoint{4.569987in}{0.606666in}}%
\pgfusepath{stroke}%
\end{pgfscope}%
\begin{pgfscope}%
\pgfpathrectangle{\pgfqpoint{0.626386in}{0.608332in}}{\pgfqpoint{6.200000in}{4.620000in}}%
\pgfusepath{clip}%
\pgfsetrectcap%
\pgfsetroundjoin%
\pgfsetlinewidth{1.003750pt}%
\definecolor{currentstroke}{rgb}{0.000000,0.000000,0.000000}%
\pgfsetstrokecolor{currentstroke}%
\pgfsetstrokeopacity{0.200000}%
\pgfsetdash{}{0pt}%
\pgfpathmoveto{\pgfqpoint{0.626386in}{5.228333in}}%
\pgfpathlineto{\pgfqpoint{0.627316in}{5.127396in}}%
\pgfpathlineto{\pgfqpoint{0.630107in}{5.027483in}}%
\pgfpathlineto{\pgfqpoint{0.634758in}{4.928596in}}%
\pgfpathlineto{\pgfqpoint{0.641270in}{4.830733in}}%
\pgfpathlineto{\pgfqpoint{0.649643in}{4.733895in}}%
\pgfpathlineto{\pgfqpoint{0.659876in}{4.638082in}}%
\pgfpathlineto{\pgfqpoint{0.671970in}{4.543294in}}%
\pgfpathlineto{\pgfqpoint{0.685924in}{4.449530in}}%
\pgfpathlineto{\pgfqpoint{0.701738in}{4.356791in}}%
\pgfpathlineto{\pgfqpoint{0.719414in}{4.265077in}}%
\pgfpathlineto{\pgfqpoint{0.738950in}{4.174387in}}%
\pgfpathlineto{\pgfqpoint{0.760346in}{4.084722in}}%
\pgfpathlineto{\pgfqpoint{0.783603in}{3.996082in}}%
\pgfpathlineto{\pgfqpoint{0.808720in}{3.908467in}}%
\pgfpathlineto{\pgfqpoint{0.835698in}{3.821877in}}%
\pgfpathlineto{\pgfqpoint{0.864537in}{3.736311in}}%
\pgfpathlineto{\pgfqpoint{0.895236in}{3.651770in}}%
\pgfpathlineto{\pgfqpoint{0.927796in}{3.568253in}}%
\pgfpathlineto{\pgfqpoint{0.962216in}{3.485762in}}%
\pgfpathlineto{\pgfqpoint{0.998497in}{3.404295in}}%
\pgfpathlineto{\pgfqpoint{1.036638in}{3.323853in}}%
\pgfpathlineto{\pgfqpoint{1.076640in}{3.244436in}}%
\pgfpathlineto{\pgfqpoint{1.118503in}{3.166043in}}%
\pgfpathlineto{\pgfqpoint{1.162226in}{3.088675in}}%
\pgfpathlineto{\pgfqpoint{1.207809in}{3.012332in}}%
\pgfpathlineto{\pgfqpoint{1.255253in}{2.937014in}}%
\pgfpathlineto{\pgfqpoint{1.304558in}{2.862720in}}%
\pgfpathlineto{\pgfqpoint{1.355723in}{2.789451in}}%
\pgfpathlineto{\pgfqpoint{1.408749in}{2.717207in}}%
\pgfpathlineto{\pgfqpoint{1.463635in}{2.645987in}}%
\pgfpathlineto{\pgfqpoint{1.520382in}{2.575793in}}%
\pgfpathlineto{\pgfqpoint{1.578990in}{2.506623in}}%
\pgfpathlineto{\pgfqpoint{1.639458in}{2.438477in}}%
\pgfpathlineto{\pgfqpoint{1.701786in}{2.371357in}}%
\pgfpathlineto{\pgfqpoint{1.765976in}{2.305261in}}%
\pgfpathlineto{\pgfqpoint{1.832025in}{2.240190in}}%
\pgfpathlineto{\pgfqpoint{1.899935in}{2.176144in}}%
\pgfpathlineto{\pgfqpoint{1.969706in}{2.113122in}}%
\pgfpathlineto{\pgfqpoint{2.041338in}{2.051126in}}%
\pgfpathlineto{\pgfqpoint{2.114830in}{1.990154in}}%
\pgfpathlineto{\pgfqpoint{2.190182in}{1.930206in}}%
\pgfpathlineto{\pgfqpoint{2.267395in}{1.871284in}}%
\pgfpathlineto{\pgfqpoint{2.346469in}{1.813386in}}%
\pgfpathlineto{\pgfqpoint{2.427403in}{1.756513in}}%
\pgfpathlineto{\pgfqpoint{2.510197in}{1.700665in}}%
\pgfpathlineto{\pgfqpoint{2.594853in}{1.645841in}}%
\pgfpathlineto{\pgfqpoint{2.681368in}{1.592042in}}%
\pgfpathlineto{\pgfqpoint{2.769745in}{1.539268in}}%
\pgfpathlineto{\pgfqpoint{2.859982in}{1.487519in}}%
\pgfpathlineto{\pgfqpoint{2.952079in}{1.436794in}}%
\pgfpathlineto{\pgfqpoint{3.046037in}{1.387094in}}%
\pgfpathlineto{\pgfqpoint{3.141856in}{1.338419in}}%
\pgfpathlineto{\pgfqpoint{3.239535in}{1.290768in}}%
\pgfpathlineto{\pgfqpoint{3.339074in}{1.244143in}}%
\pgfpathlineto{\pgfqpoint{3.440475in}{1.198542in}}%
\pgfpathlineto{\pgfqpoint{3.543735in}{1.153966in}}%
\pgfpathlineto{\pgfqpoint{3.648857in}{1.110414in}}%
\pgfpathlineto{\pgfqpoint{3.755839in}{1.067887in}}%
\pgfpathlineto{\pgfqpoint{3.864681in}{1.026385in}}%
\pgfpathlineto{\pgfqpoint{3.942061in}{0.997990in}}%
\pgfpathlineto{\pgfqpoint{3.998343in}{0.978264in}}%
\pgfpathlineto{\pgfqpoint{4.055555in}{0.959050in}}%
\pgfpathlineto{\pgfqpoint{4.113698in}{0.940348in}}%
\pgfpathlineto{\pgfqpoint{4.172770in}{0.922159in}}%
\pgfpathlineto{\pgfqpoint{4.232773in}{0.904482in}}%
\pgfpathlineto{\pgfqpoint{4.293706in}{0.887318in}}%
\pgfpathlineto{\pgfqpoint{4.355570in}{0.870666in}}%
\pgfpathlineto{\pgfqpoint{4.418363in}{0.854526in}}%
\pgfpathlineto{\pgfqpoint{4.482087in}{0.838899in}}%
\pgfpathlineto{\pgfqpoint{4.546742in}{0.823784in}}%
\pgfpathlineto{\pgfqpoint{4.612326in}{0.809182in}}%
\pgfpathlineto{\pgfqpoint{4.678841in}{0.795091in}}%
\pgfpathlineto{\pgfqpoint{4.746286in}{0.781514in}}%
\pgfpathlineto{\pgfqpoint{4.814661in}{0.768448in}}%
\pgfpathlineto{\pgfqpoint{4.883967in}{0.755895in}}%
\pgfpathlineto{\pgfqpoint{4.954203in}{0.743854in}}%
\pgfpathlineto{\pgfqpoint{5.025369in}{0.732326in}}%
\pgfpathlineto{\pgfqpoint{5.097466in}{0.721310in}}%
\pgfpathlineto{\pgfqpoint{5.170493in}{0.710807in}}%
\pgfpathlineto{\pgfqpoint{5.244450in}{0.700815in}}%
\pgfpathlineto{\pgfqpoint{5.319337in}{0.691336in}}%
\pgfpathlineto{\pgfqpoint{5.395154in}{0.682370in}}%
\pgfpathlineto{\pgfqpoint{5.471902in}{0.673916in}}%
\pgfpathlineto{\pgfqpoint{5.549580in}{0.665974in}}%
\pgfpathlineto{\pgfqpoint{5.628189in}{0.658545in}}%
\pgfpathlineto{\pgfqpoint{5.707728in}{0.651628in}}%
\pgfpathlineto{\pgfqpoint{5.788197in}{0.645223in}}%
\pgfpathlineto{\pgfqpoint{5.869596in}{0.639331in}}%
\pgfpathlineto{\pgfqpoint{5.951925in}{0.633951in}}%
\pgfpathlineto{\pgfqpoint{6.035185in}{0.629083in}}%
\pgfpathlineto{\pgfqpoint{6.119375in}{0.624728in}}%
\pgfpathlineto{\pgfqpoint{6.204496in}{0.620886in}}%
\pgfpathlineto{\pgfqpoint{6.290546in}{0.617555in}}%
\pgfpathlineto{\pgfqpoint{6.377527in}{0.614737in}}%
\pgfpathlineto{\pgfqpoint{6.465438in}{0.612431in}}%
\pgfpathlineto{\pgfqpoint{6.554280in}{0.610638in}}%
\pgfpathlineto{\pgfqpoint{6.644052in}{0.609357in}}%
\pgfpathlineto{\pgfqpoint{6.734754in}{0.608589in}}%
\pgfpathlineto{\pgfqpoint{6.826386in}{0.608332in}}%
\pgfusepath{stroke}%
\end{pgfscope}%
\begin{pgfscope}%
\pgfsetrectcap%
\pgfsetmiterjoin%
\pgfsetlinewidth{0.803000pt}%
\definecolor{currentstroke}{rgb}{0.000000,0.000000,0.000000}%
\pgfsetstrokecolor{currentstroke}%
\pgfsetdash{}{0pt}%
\pgfpathmoveto{\pgfqpoint{0.626386in}{0.608332in}}%
\pgfpathlineto{\pgfqpoint{0.626386in}{5.228333in}}%
\pgfusepath{stroke}%
\end{pgfscope}%
\begin{pgfscope}%
\pgfsetrectcap%
\pgfsetmiterjoin%
\pgfsetlinewidth{0.803000pt}%
\definecolor{currentstroke}{rgb}{0.000000,0.000000,0.000000}%
\pgfsetstrokecolor{currentstroke}%
\pgfsetdash{}{0pt}%
\pgfpathmoveto{\pgfqpoint{6.826386in}{0.608332in}}%
\pgfpathlineto{\pgfqpoint{6.826386in}{5.228333in}}%
\pgfusepath{stroke}%
\end{pgfscope}%
\begin{pgfscope}%
\pgfsetrectcap%
\pgfsetmiterjoin%
\pgfsetlinewidth{0.803000pt}%
\definecolor{currentstroke}{rgb}{0.000000,0.000000,0.000000}%
\pgfsetstrokecolor{currentstroke}%
\pgfsetdash{}{0pt}%
\pgfpathmoveto{\pgfqpoint{0.626386in}{0.608332in}}%
\pgfpathlineto{\pgfqpoint{6.826386in}{0.608332in}}%
\pgfusepath{stroke}%
\end{pgfscope}%
\begin{pgfscope}%
\pgfsetrectcap%
\pgfsetmiterjoin%
\pgfsetlinewidth{0.803000pt}%
\definecolor{currentstroke}{rgb}{0.000000,0.000000,0.000000}%
\pgfsetstrokecolor{currentstroke}%
\pgfsetdash{}{0pt}%
\pgfpathmoveto{\pgfqpoint{0.626386in}{5.228333in}}%
\pgfpathlineto{\pgfqpoint{6.826386in}{5.228333in}}%
\pgfusepath{stroke}%
\end{pgfscope}%
\begin{pgfscope}%
\pgfsetbuttcap%
\pgfsetmiterjoin%
\definecolor{currentfill}{rgb}{0.300000,0.300000,0.300000}%
\pgfsetfillcolor{currentfill}%
\pgfsetfillopacity{0.500000}%
\pgfsetlinewidth{1.003750pt}%
\definecolor{currentstroke}{rgb}{0.300000,0.300000,0.300000}%
\pgfsetstrokecolor{currentstroke}%
\pgfsetstrokeopacity{0.500000}%
\pgfsetdash{}{0pt}%
\pgfpathmoveto{\pgfqpoint{4.314927in}{4.220000in}}%
\pgfpathlineto{\pgfqpoint{6.718053in}{4.220000in}}%
\pgfpathquadraticcurveto{\pgfqpoint{6.756942in}{4.220000in}}{\pgfqpoint{6.756942in}{4.258889in}}%
\pgfpathlineto{\pgfqpoint{6.756942in}{5.064444in}}%
\pgfpathquadraticcurveto{\pgfqpoint{6.756942in}{5.103333in}}{\pgfqpoint{6.718053in}{5.103333in}}%
\pgfpathlineto{\pgfqpoint{4.314927in}{5.103333in}}%
\pgfpathquadraticcurveto{\pgfqpoint{4.276038in}{5.103333in}}{\pgfqpoint{4.276038in}{5.064444in}}%
\pgfpathlineto{\pgfqpoint{4.276038in}{4.258889in}}%
\pgfpathquadraticcurveto{\pgfqpoint{4.276038in}{4.220000in}}{\pgfqpoint{4.314927in}{4.220000in}}%
\pgfpathlineto{\pgfqpoint{4.314927in}{4.220000in}}%
\pgfpathclose%
\pgfusepath{stroke,fill}%
\end{pgfscope}%
\begin{pgfscope}%
\pgfsetbuttcap%
\pgfsetmiterjoin%
\definecolor{currentfill}{rgb}{1.000000,1.000000,1.000000}%
\pgfsetfillcolor{currentfill}%
\pgfsetlinewidth{1.003750pt}%
\definecolor{currentstroke}{rgb}{0.800000,0.800000,0.800000}%
\pgfsetstrokecolor{currentstroke}%
\pgfsetdash{}{0pt}%
\pgfpathmoveto{\pgfqpoint{4.287149in}{4.247778in}}%
\pgfpathlineto{\pgfqpoint{6.690275in}{4.247778in}}%
\pgfpathquadraticcurveto{\pgfqpoint{6.729164in}{4.247778in}}{\pgfqpoint{6.729164in}{4.286667in}}%
\pgfpathlineto{\pgfqpoint{6.729164in}{5.092221in}}%
\pgfpathquadraticcurveto{\pgfqpoint{6.729164in}{5.131110in}}{\pgfqpoint{6.690275in}{5.131110in}}%
\pgfpathlineto{\pgfqpoint{4.287149in}{5.131110in}}%
\pgfpathquadraticcurveto{\pgfqpoint{4.248260in}{5.131110in}}{\pgfqpoint{4.248260in}{5.092221in}}%
\pgfpathlineto{\pgfqpoint{4.248260in}{4.286667in}}%
\pgfpathquadraticcurveto{\pgfqpoint{4.248260in}{4.247778in}}{\pgfqpoint{4.287149in}{4.247778in}}%
\pgfpathlineto{\pgfqpoint{4.287149in}{4.247778in}}%
\pgfpathclose%
\pgfusepath{stroke,fill}%
\end{pgfscope}%
\begin{pgfscope}%
\pgfsetrectcap%
\pgfsetroundjoin%
\pgfsetlinewidth{3.011250pt}%
\definecolor{currentstroke}{rgb}{0.000000,0.000000,0.000000}%
\pgfsetstrokecolor{currentstroke}%
\pgfsetdash{}{0pt}%
\pgfpathmoveto{\pgfqpoint{4.326038in}{4.982499in}}%
\pgfpathlineto{\pgfqpoint{4.520482in}{4.982499in}}%
\pgfpathlineto{\pgfqpoint{4.714927in}{4.982499in}}%
\pgfusepath{stroke}%
\end{pgfscope}%
\begin{pgfscope}%
\definecolor{textcolor}{rgb}{0.000000,0.000000,0.000000}%
\pgfsetstrokecolor{textcolor}%
\pgfsetfillcolor{textcolor}%
\pgftext[x=4.870482in,y=4.914444in,left,base]{\color{textcolor}{\rmfamily\fontsize{14.000000}{16.800000}\selectfont\catcode`\^=\active\def^{\ifmmode\sp\else\^{}\fi}\catcode`\%=\active\def%{\%}Pareto Front}}%
\end{pgfscope}%
\begin{pgfscope}%
\pgfsetbuttcap%
\pgfsetroundjoin%
\definecolor{currentfill}{rgb}{0.121569,0.466667,0.705882}%
\pgfsetfillcolor{currentfill}%
\pgfsetlinewidth{1.003750pt}%
\definecolor{currentstroke}{rgb}{0.121569,0.466667,0.705882}%
\pgfsetstrokecolor{currentstroke}%
\pgfsetdash{}{0pt}%
\pgfsys@defobject{currentmarker}{\pgfqpoint{-0.012028in}{-0.012028in}}{\pgfqpoint{0.012028in}{0.012028in}}{%
\pgfpathmoveto{\pgfqpoint{0.000000in}{-0.012028in}}%
\pgfpathcurveto{\pgfqpoint{0.003190in}{-0.012028in}}{\pgfqpoint{0.006250in}{-0.010761in}}{\pgfqpoint{0.008505in}{-0.008505in}}%
\pgfpathcurveto{\pgfqpoint{0.010761in}{-0.006250in}}{\pgfqpoint{0.012028in}{-0.003190in}}{\pgfqpoint{0.012028in}{0.000000in}}%
\pgfpathcurveto{\pgfqpoint{0.012028in}{0.003190in}}{\pgfqpoint{0.010761in}{0.006250in}}{\pgfqpoint{0.008505in}{0.008505in}}%
\pgfpathcurveto{\pgfqpoint{0.006250in}{0.010761in}}{\pgfqpoint{0.003190in}{0.012028in}}{\pgfqpoint{0.000000in}{0.012028in}}%
\pgfpathcurveto{\pgfqpoint{-0.003190in}{0.012028in}}{\pgfqpoint{-0.006250in}{0.010761in}}{\pgfqpoint{-0.008505in}{0.008505in}}%
\pgfpathcurveto{\pgfqpoint{-0.010761in}{0.006250in}}{\pgfqpoint{-0.012028in}{0.003190in}}{\pgfqpoint{-0.012028in}{0.000000in}}%
\pgfpathcurveto{\pgfqpoint{-0.012028in}{-0.003190in}}{\pgfqpoint{-0.010761in}{-0.006250in}}{\pgfqpoint{-0.008505in}{-0.008505in}}%
\pgfpathcurveto{\pgfqpoint{-0.006250in}{-0.010761in}}{\pgfqpoint{-0.003190in}{-0.012028in}}{\pgfqpoint{0.000000in}{-0.012028in}}%
\pgfpathlineto{\pgfqpoint{0.000000in}{-0.012028in}}%
\pgfpathclose%
\pgfusepath{stroke,fill}%
}%
\begin{pgfscope}%
\pgfsys@transformshift{4.520482in}{4.690486in}%
\pgfsys@useobject{currentmarker}{}%
\end{pgfscope}%
\end{pgfscope}%
\begin{pgfscope}%
\definecolor{textcolor}{rgb}{0.000000,0.000000,0.000000}%
\pgfsetstrokecolor{textcolor}%
\pgfsetfillcolor{textcolor}%
\pgftext[x=4.870482in,y=4.639444in,left,base]{\color{textcolor}{\rmfamily\fontsize{14.000000}{16.800000}\selectfont\catcode`\^=\active\def^{\ifmmode\sp\else\^{}\fi}\catcode`\%=\active\def%{\%}Tested points}}%
\end{pgfscope}%
\begin{pgfscope}%
\pgfsetbuttcap%
\pgfsetroundjoin%
\definecolor{currentfill}{rgb}{0.839216,0.152941,0.156863}%
\pgfsetfillcolor{currentfill}%
\pgfsetlinewidth{1.003750pt}%
\definecolor{currentstroke}{rgb}{0.839216,0.152941,0.156863}%
\pgfsetstrokecolor{currentstroke}%
\pgfsetdash{}{0pt}%
\pgfsys@defobject{currentmarker}{\pgfqpoint{-0.031056in}{-0.031056in}}{\pgfqpoint{0.031056in}{0.031056in}}{%
\pgfpathmoveto{\pgfqpoint{0.000000in}{-0.031056in}}%
\pgfpathcurveto{\pgfqpoint{0.008236in}{-0.031056in}}{\pgfqpoint{0.016136in}{-0.027784in}}{\pgfqpoint{0.021960in}{-0.021960in}}%
\pgfpathcurveto{\pgfqpoint{0.027784in}{-0.016136in}}{\pgfqpoint{0.031056in}{-0.008236in}}{\pgfqpoint{0.031056in}{0.000000in}}%
\pgfpathcurveto{\pgfqpoint{0.031056in}{0.008236in}}{\pgfqpoint{0.027784in}{0.016136in}}{\pgfqpoint{0.021960in}{0.021960in}}%
\pgfpathcurveto{\pgfqpoint{0.016136in}{0.027784in}}{\pgfqpoint{0.008236in}{0.031056in}}{\pgfqpoint{0.000000in}{0.031056in}}%
\pgfpathcurveto{\pgfqpoint{-0.008236in}{0.031056in}}{\pgfqpoint{-0.016136in}{0.027784in}}{\pgfqpoint{-0.021960in}{0.021960in}}%
\pgfpathcurveto{\pgfqpoint{-0.027784in}{0.016136in}}{\pgfqpoint{-0.031056in}{0.008236in}}{\pgfqpoint{-0.031056in}{0.000000in}}%
\pgfpathcurveto{\pgfqpoint{-0.031056in}{-0.008236in}}{\pgfqpoint{-0.027784in}{-0.016136in}}{\pgfqpoint{-0.021960in}{-0.021960in}}%
\pgfpathcurveto{\pgfqpoint{-0.016136in}{-0.027784in}}{\pgfqpoint{-0.008236in}{-0.031056in}}{\pgfqpoint{0.000000in}{-0.031056in}}%
\pgfpathlineto{\pgfqpoint{0.000000in}{-0.031056in}}%
\pgfpathclose%
\pgfusepath{stroke,fill}%
}%
\begin{pgfscope}%
\pgfsys@transformshift{4.520482in}{4.415486in}%
\pgfsys@useobject{currentmarker}{}%
\end{pgfscope}%
\end{pgfscope}%
\begin{pgfscope}%
\definecolor{textcolor}{rgb}{0.000000,0.000000,0.000000}%
\pgfsetstrokecolor{textcolor}%
\pgfsetfillcolor{textcolor}%
\pgftext[x=4.870482in,y=4.364445in,left,base]{\color{textcolor}{\rmfamily\fontsize{14.000000}{16.800000}\selectfont\catcode`\^=\active\def^{\ifmmode\sp\else\^{}\fi}\catcode`\%=\active\def%{\%}Alternative solutions}}%
\end{pgfscope}%
\end{pgfpicture}%
\makeatother%
\endgroup%
}
  \caption{All of the alternative points inside the near-feasible space selected
  using the algorithm described in Section \ref{section:mga-moo}.}
  \label{fig:nd-mga}
\end{figure}

\begin{figure}[H]
  \centering
  \resizebox{1\columnwidth}{!}{%% Creator: Matplotlib, PGF backend
%%
%% To include the figure in your LaTeX document, write
%%   \input{<filename>.pgf}
%%
%% Make sure the required packages are loaded in your preamble
%%   \usepackage{pgf}
%%
%% Also ensure that all the required font packages are loaded; for instance,
%% the lmodern package is sometimes necessary when using math font.
%%   \usepackage{lmodern}
%%
%% Figures using additional raster images can only be included by \input if
%% they are in the same directory as the main LaTeX file. For loading figures
%% from other directories you can use the `import` package
%%   \usepackage{import}
%%
%% and then include the figures with
%%   \import{<path to file>}{<filename>.pgf}
%%
%% Matplotlib used the following preamble
%%   \def\mathdefault#1{#1}
%%   \everymath=\expandafter{\the\everymath\displaystyle}
%%   \IfFileExists{scrextend.sty}{
%%     \usepackage[fontsize=10.000000pt]{scrextend}
%%   }{
%%     \renewcommand{\normalsize}{\fontsize{10.000000}{12.000000}\selectfont}
%%     \normalsize
%%   }
%%   
%%   \makeatletter\@ifpackageloaded{underscore}{}{\usepackage[strings]{underscore}}\makeatother
%%
\begingroup%
\makeatletter%
\begin{pgfpicture}%
\pgfpathrectangle{\pgfpointorigin}{\pgfqpoint{9.654231in}{3.182465in}}%
\pgfusepath{use as bounding box, clip}%
\begin{pgfscope}%
\pgfsetbuttcap%
\pgfsetmiterjoin%
\definecolor{currentfill}{rgb}{1.000000,1.000000,1.000000}%
\pgfsetfillcolor{currentfill}%
\pgfsetlinewidth{0.000000pt}%
\definecolor{currentstroke}{rgb}{0.000000,0.000000,0.000000}%
\pgfsetstrokecolor{currentstroke}%
\pgfsetdash{}{0pt}%
\pgfpathmoveto{\pgfqpoint{0.000000in}{0.000000in}}%
\pgfpathlineto{\pgfqpoint{9.654231in}{0.000000in}}%
\pgfpathlineto{\pgfqpoint{9.654231in}{3.182465in}}%
\pgfpathlineto{\pgfqpoint{0.000000in}{3.182465in}}%
\pgfpathlineto{\pgfqpoint{0.000000in}{0.000000in}}%
\pgfpathclose%
\pgfusepath{fill}%
\end{pgfscope}%
\begin{pgfscope}%
\pgfsetbuttcap%
\pgfsetmiterjoin%
\definecolor{currentfill}{rgb}{1.000000,1.000000,1.000000}%
\pgfsetfillcolor{currentfill}%
\pgfsetlinewidth{0.000000pt}%
\definecolor{currentstroke}{rgb}{0.000000,0.000000,0.000000}%
\pgfsetstrokecolor{currentstroke}%
\pgfsetstrokeopacity{0.000000}%
\pgfsetdash{}{0pt}%
\pgfpathmoveto{\pgfqpoint{3.536584in}{0.147348in}}%
\pgfpathlineto{\pgfqpoint{6.271879in}{0.147348in}}%
\pgfpathlineto{\pgfqpoint{6.271879in}{2.882642in}}%
\pgfpathlineto{\pgfqpoint{3.536584in}{2.882642in}}%
\pgfpathlineto{\pgfqpoint{3.536584in}{0.147348in}}%
\pgfpathclose%
\pgfusepath{fill}%
\end{pgfscope}%
\begin{pgfscope}%
\pgfsetbuttcap%
\pgfsetmiterjoin%
\definecolor{currentfill}{rgb}{0.950000,0.950000,0.950000}%
\pgfsetfillcolor{currentfill}%
\pgfsetfillopacity{0.500000}%
\pgfsetlinewidth{1.003750pt}%
\definecolor{currentstroke}{rgb}{0.950000,0.950000,0.950000}%
\pgfsetstrokecolor{currentstroke}%
\pgfsetstrokeopacity{0.500000}%
\pgfsetdash{}{0pt}%
\pgfpathmoveto{\pgfqpoint{4.941195in}{1.930798in}}%
\pgfpathlineto{\pgfqpoint{6.147731in}{1.099507in}}%
\pgfpathlineto{\pgfqpoint{6.226389in}{2.033906in}}%
\pgfpathlineto{\pgfqpoint{4.941195in}{2.862146in}}%
\pgfusepath{stroke,fill}%
\end{pgfscope}%
\begin{pgfscope}%
\pgfsetbuttcap%
\pgfsetmiterjoin%
\definecolor{currentfill}{rgb}{0.900000,0.900000,0.900000}%
\pgfsetfillcolor{currentfill}%
\pgfsetfillopacity{0.500000}%
\pgfsetlinewidth{1.003750pt}%
\definecolor{currentstroke}{rgb}{0.900000,0.900000,0.900000}%
\pgfsetstrokecolor{currentstroke}%
\pgfsetstrokeopacity{0.500000}%
\pgfsetdash{}{0pt}%
\pgfpathmoveto{\pgfqpoint{4.941195in}{1.930798in}}%
\pgfpathlineto{\pgfqpoint{3.734658in}{1.099507in}}%
\pgfpathlineto{\pgfqpoint{3.656001in}{2.033906in}}%
\pgfpathlineto{\pgfqpoint{4.941195in}{2.862146in}}%
\pgfusepath{stroke,fill}%
\end{pgfscope}%
\begin{pgfscope}%
\pgfsetbuttcap%
\pgfsetmiterjoin%
\definecolor{currentfill}{rgb}{0.925000,0.925000,0.925000}%
\pgfsetfillcolor{currentfill}%
\pgfsetfillopacity{0.500000}%
\pgfsetlinewidth{1.003750pt}%
\definecolor{currentstroke}{rgb}{0.925000,0.925000,0.925000}%
\pgfsetstrokecolor{currentstroke}%
\pgfsetstrokeopacity{0.500000}%
\pgfsetdash{}{0pt}%
\pgfpathmoveto{\pgfqpoint{4.941195in}{1.930798in}}%
\pgfpathlineto{\pgfqpoint{3.734658in}{1.099507in}}%
\pgfpathlineto{\pgfqpoint{4.941195in}{0.166408in}}%
\pgfpathlineto{\pgfqpoint{6.147731in}{1.099507in}}%
\pgfusepath{stroke,fill}%
\end{pgfscope}%
\begin{pgfscope}%
\pgfsetbuttcap%
\pgfsetroundjoin%
\pgfsetlinewidth{0.803000pt}%
\definecolor{currentstroke}{rgb}{0.690196,0.690196,0.690196}%
\pgfsetstrokecolor{currentstroke}%
\pgfsetdash{}{0pt}%
\pgfpathmoveto{\pgfqpoint{6.075116in}{1.043348in}}%
\pgfpathlineto{\pgfqpoint{4.868351in}{1.880609in}}%
\pgfpathlineto{\pgfqpoint{4.863861in}{2.812308in}}%
\pgfusepath{stroke}%
\end{pgfscope}%
\begin{pgfscope}%
\pgfsetbuttcap%
\pgfsetroundjoin%
\pgfsetlinewidth{0.803000pt}%
\definecolor{currentstroke}{rgb}{0.690196,0.690196,0.690196}%
\pgfsetstrokecolor{currentstroke}%
\pgfsetdash{}{0pt}%
\pgfpathmoveto{\pgfqpoint{5.845306in}{0.865621in}}%
\pgfpathlineto{\pgfqpoint{4.638013in}{1.721909in}}%
\pgfpathlineto{\pgfqpoint{4.619106in}{2.654576in}}%
\pgfusepath{stroke}%
\end{pgfscope}%
\begin{pgfscope}%
\pgfsetbuttcap%
\pgfsetroundjoin%
\pgfsetlinewidth{0.803000pt}%
\definecolor{currentstroke}{rgb}{0.690196,0.690196,0.690196}%
\pgfsetstrokecolor{currentstroke}%
\pgfsetdash{}{0pt}%
\pgfpathmoveto{\pgfqpoint{5.610093in}{0.683714in}}%
\pgfpathlineto{\pgfqpoint{4.402562in}{1.559686in}}%
\pgfpathlineto{\pgfqpoint{4.368575in}{2.493122in}}%
\pgfusepath{stroke}%
\end{pgfscope}%
\begin{pgfscope}%
\pgfsetbuttcap%
\pgfsetroundjoin%
\pgfsetlinewidth{0.803000pt}%
\definecolor{currentstroke}{rgb}{0.690196,0.690196,0.690196}%
\pgfsetstrokecolor{currentstroke}%
\pgfsetdash{}{0pt}%
\pgfpathmoveto{\pgfqpoint{5.369283in}{0.497478in}}%
\pgfpathlineto{\pgfqpoint{4.161826in}{1.393821in}}%
\pgfpathlineto{\pgfqpoint{4.112061in}{2.327813in}}%
\pgfusepath{stroke}%
\end{pgfscope}%
\begin{pgfscope}%
\pgfsetbuttcap%
\pgfsetroundjoin%
\pgfsetlinewidth{0.803000pt}%
\definecolor{currentstroke}{rgb}{0.690196,0.690196,0.690196}%
\pgfsetstrokecolor{currentstroke}%
\pgfsetdash{}{0pt}%
\pgfpathmoveto{\pgfqpoint{5.122674in}{0.306758in}}%
\pgfpathlineto{\pgfqpoint{3.915624in}{1.224191in}}%
\pgfpathlineto{\pgfqpoint{3.849347in}{2.158507in}}%
\pgfusepath{stroke}%
\end{pgfscope}%
\begin{pgfscope}%
\pgfsetbuttcap%
\pgfsetroundjoin%
\pgfsetlinewidth{0.803000pt}%
\definecolor{currentstroke}{rgb}{0.690196,0.690196,0.690196}%
\pgfsetstrokecolor{currentstroke}%
\pgfsetdash{}{0pt}%
\pgfpathmoveto{\pgfqpoint{5.018529in}{2.812308in}}%
\pgfpathlineto{\pgfqpoint{5.014039in}{1.880609in}}%
\pgfpathlineto{\pgfqpoint{3.807274in}{1.043348in}}%
\pgfusepath{stroke}%
\end{pgfscope}%
\begin{pgfscope}%
\pgfsetbuttcap%
\pgfsetroundjoin%
\pgfsetlinewidth{0.803000pt}%
\definecolor{currentstroke}{rgb}{0.690196,0.690196,0.690196}%
\pgfsetstrokecolor{currentstroke}%
\pgfsetdash{}{0pt}%
\pgfpathmoveto{\pgfqpoint{5.263284in}{2.654576in}}%
\pgfpathlineto{\pgfqpoint{5.244376in}{1.721909in}}%
\pgfpathlineto{\pgfqpoint{4.037084in}{0.865621in}}%
\pgfusepath{stroke}%
\end{pgfscope}%
\begin{pgfscope}%
\pgfsetbuttcap%
\pgfsetroundjoin%
\pgfsetlinewidth{0.803000pt}%
\definecolor{currentstroke}{rgb}{0.690196,0.690196,0.690196}%
\pgfsetstrokecolor{currentstroke}%
\pgfsetdash{}{0pt}%
\pgfpathmoveto{\pgfqpoint{5.513815in}{2.493122in}}%
\pgfpathlineto{\pgfqpoint{5.479827in}{1.559686in}}%
\pgfpathlineto{\pgfqpoint{4.272297in}{0.683714in}}%
\pgfusepath{stroke}%
\end{pgfscope}%
\begin{pgfscope}%
\pgfsetbuttcap%
\pgfsetroundjoin%
\pgfsetlinewidth{0.803000pt}%
\definecolor{currentstroke}{rgb}{0.690196,0.690196,0.690196}%
\pgfsetstrokecolor{currentstroke}%
\pgfsetdash{}{0pt}%
\pgfpathmoveto{\pgfqpoint{5.770329in}{2.327813in}}%
\pgfpathlineto{\pgfqpoint{5.720564in}{1.393821in}}%
\pgfpathlineto{\pgfqpoint{4.513107in}{0.497478in}}%
\pgfusepath{stroke}%
\end{pgfscope}%
\begin{pgfscope}%
\pgfsetbuttcap%
\pgfsetroundjoin%
\pgfsetlinewidth{0.803000pt}%
\definecolor{currentstroke}{rgb}{0.690196,0.690196,0.690196}%
\pgfsetstrokecolor{currentstroke}%
\pgfsetdash{}{0pt}%
\pgfpathmoveto{\pgfqpoint{6.033043in}{2.158507in}}%
\pgfpathlineto{\pgfqpoint{5.966766in}{1.224191in}}%
\pgfpathlineto{\pgfqpoint{4.759716in}{0.306758in}}%
\pgfusepath{stroke}%
\end{pgfscope}%
\begin{pgfscope}%
\pgfsetbuttcap%
\pgfsetroundjoin%
\pgfsetlinewidth{0.803000pt}%
\definecolor{currentstroke}{rgb}{0.690196,0.690196,0.690196}%
\pgfsetstrokecolor{currentstroke}%
\pgfsetdash{}{0pt}%
\pgfpathmoveto{\pgfqpoint{3.729941in}{1.155548in}}%
\pgfpathlineto{\pgfqpoint{4.941195in}{1.986842in}}%
\pgfpathlineto{\pgfqpoint{6.152449in}{1.155548in}}%
\pgfusepath{stroke}%
\end{pgfscope}%
\begin{pgfscope}%
\pgfsetbuttcap%
\pgfsetroundjoin%
\pgfsetlinewidth{0.803000pt}%
\definecolor{currentstroke}{rgb}{0.690196,0.690196,0.690196}%
\pgfsetstrokecolor{currentstroke}%
\pgfsetdash{}{0pt}%
\pgfpathmoveto{\pgfqpoint{3.714997in}{1.333067in}}%
\pgfpathlineto{\pgfqpoint{4.941195in}{2.164215in}}%
\pgfpathlineto{\pgfqpoint{6.167392in}{1.333067in}}%
\pgfusepath{stroke}%
\end{pgfscope}%
\begin{pgfscope}%
\pgfsetbuttcap%
\pgfsetroundjoin%
\pgfsetlinewidth{0.803000pt}%
\definecolor{currentstroke}{rgb}{0.690196,0.690196,0.690196}%
\pgfsetstrokecolor{currentstroke}%
\pgfsetdash{}{0pt}%
\pgfpathmoveto{\pgfqpoint{3.699681in}{1.515021in}}%
\pgfpathlineto{\pgfqpoint{4.941195in}{2.345771in}}%
\pgfpathlineto{\pgfqpoint{6.182709in}{1.515021in}}%
\pgfusepath{stroke}%
\end{pgfscope}%
\begin{pgfscope}%
\pgfsetbuttcap%
\pgfsetroundjoin%
\pgfsetlinewidth{0.803000pt}%
\definecolor{currentstroke}{rgb}{0.690196,0.690196,0.690196}%
\pgfsetstrokecolor{currentstroke}%
\pgfsetdash{}{0pt}%
\pgfpathmoveto{\pgfqpoint{3.683976in}{1.701578in}}%
\pgfpathlineto{\pgfqpoint{4.941195in}{2.531660in}}%
\pgfpathlineto{\pgfqpoint{6.198413in}{1.701578in}}%
\pgfusepath{stroke}%
\end{pgfscope}%
\begin{pgfscope}%
\pgfsetbuttcap%
\pgfsetroundjoin%
\pgfsetlinewidth{0.803000pt}%
\definecolor{currentstroke}{rgb}{0.690196,0.690196,0.690196}%
\pgfsetstrokecolor{currentstroke}%
\pgfsetdash{}{0pt}%
\pgfpathmoveto{\pgfqpoint{3.667870in}{1.892915in}}%
\pgfpathlineto{\pgfqpoint{4.941195in}{2.722038in}}%
\pgfpathlineto{\pgfqpoint{6.214520in}{1.892915in}}%
\pgfusepath{stroke}%
\end{pgfscope}%
\begin{pgfscope}%
\pgfsetrectcap%
\pgfsetroundjoin%
\pgfsetlinewidth{0.803000pt}%
\definecolor{currentstroke}{rgb}{0.000000,0.000000,0.000000}%
\pgfsetstrokecolor{currentstroke}%
\pgfsetdash{}{0pt}%
\pgfpathmoveto{\pgfqpoint{6.147731in}{1.099507in}}%
\pgfpathlineto{\pgfqpoint{4.941195in}{0.166408in}}%
\pgfusepath{stroke}%
\end{pgfscope}%
\begin{pgfscope}%
\pgfsetrectcap%
\pgfsetroundjoin%
\pgfsetlinewidth{0.803000pt}%
\definecolor{currentstroke}{rgb}{0.000000,0.000000,0.000000}%
\pgfsetstrokecolor{currentstroke}%
\pgfsetdash{}{0pt}%
\pgfpathmoveto{\pgfqpoint{6.064907in}{1.050431in}}%
\pgfpathlineto{\pgfqpoint{6.095561in}{1.029163in}}%
\pgfusepath{stroke}%
\end{pgfscope}%
\begin{pgfscope}%
\pgfsetrectcap%
\pgfsetroundjoin%
\pgfsetlinewidth{0.803000pt}%
\definecolor{currentstroke}{rgb}{0.000000,0.000000,0.000000}%
\pgfsetstrokecolor{currentstroke}%
\pgfsetdash{}{0pt}%
\pgfpathmoveto{\pgfqpoint{5.835087in}{0.872869in}}%
\pgfpathlineto{\pgfqpoint{5.865774in}{0.851104in}}%
\pgfusepath{stroke}%
\end{pgfscope}%
\begin{pgfscope}%
\pgfsetrectcap%
\pgfsetroundjoin%
\pgfsetlinewidth{0.803000pt}%
\definecolor{currentstroke}{rgb}{0.000000,0.000000,0.000000}%
\pgfsetstrokecolor{currentstroke}%
\pgfsetdash{}{0pt}%
\pgfpathmoveto{\pgfqpoint{5.599865in}{0.691133in}}%
\pgfpathlineto{\pgfqpoint{5.630578in}{0.668853in}}%
\pgfusepath{stroke}%
\end{pgfscope}%
\begin{pgfscope}%
\pgfsetrectcap%
\pgfsetroundjoin%
\pgfsetlinewidth{0.803000pt}%
\definecolor{currentstroke}{rgb}{0.000000,0.000000,0.000000}%
\pgfsetstrokecolor{currentstroke}%
\pgfsetdash{}{0pt}%
\pgfpathmoveto{\pgfqpoint{5.359049in}{0.505076in}}%
\pgfpathlineto{\pgfqpoint{5.389781in}{0.482262in}}%
\pgfusepath{stroke}%
\end{pgfscope}%
\begin{pgfscope}%
\pgfsetrectcap%
\pgfsetroundjoin%
\pgfsetlinewidth{0.803000pt}%
\definecolor{currentstroke}{rgb}{0.000000,0.000000,0.000000}%
\pgfsetstrokecolor{currentstroke}%
\pgfsetdash{}{0pt}%
\pgfpathmoveto{\pgfqpoint{5.112437in}{0.314540in}}%
\pgfpathlineto{\pgfqpoint{5.143179in}{0.291173in}}%
\pgfusepath{stroke}%
\end{pgfscope}%
\begin{pgfscope}%
\definecolor{textcolor}{rgb}{0.000000,0.000000,0.000000}%
\pgfsetstrokecolor{textcolor}%
\pgfsetfillcolor{textcolor}%
\pgftext[x=5.840237in,y=0.241958in,,]{\color{textcolor}{\rmfamily\fontsize{14.000000}{16.800000}\selectfont\catcode`\^=\active\def^{\ifmmode\sp\else\^{}\fi}\catcode`\%=\active\def%{\%}f1}}%
\end{pgfscope}%
\begin{pgfscope}%
\pgfsetrectcap%
\pgfsetroundjoin%
\pgfsetlinewidth{0.803000pt}%
\definecolor{currentstroke}{rgb}{0.000000,0.000000,0.000000}%
\pgfsetstrokecolor{currentstroke}%
\pgfsetdash{}{0pt}%
\pgfpathmoveto{\pgfqpoint{3.734658in}{1.099507in}}%
\pgfpathlineto{\pgfqpoint{4.941195in}{0.166408in}}%
\pgfusepath{stroke}%
\end{pgfscope}%
\begin{pgfscope}%
\pgfsetrectcap%
\pgfsetroundjoin%
\pgfsetlinewidth{0.803000pt}%
\definecolor{currentstroke}{rgb}{0.000000,0.000000,0.000000}%
\pgfsetstrokecolor{currentstroke}%
\pgfsetdash{}{0pt}%
\pgfpathmoveto{\pgfqpoint{3.817483in}{1.050431in}}%
\pgfpathlineto{\pgfqpoint{3.786829in}{1.029163in}}%
\pgfusepath{stroke}%
\end{pgfscope}%
\begin{pgfscope}%
\pgfsetrectcap%
\pgfsetroundjoin%
\pgfsetlinewidth{0.803000pt}%
\definecolor{currentstroke}{rgb}{0.000000,0.000000,0.000000}%
\pgfsetstrokecolor{currentstroke}%
\pgfsetdash{}{0pt}%
\pgfpathmoveto{\pgfqpoint{4.047303in}{0.872869in}}%
\pgfpathlineto{\pgfqpoint{4.016616in}{0.851104in}}%
\pgfusepath{stroke}%
\end{pgfscope}%
\begin{pgfscope}%
\pgfsetrectcap%
\pgfsetroundjoin%
\pgfsetlinewidth{0.803000pt}%
\definecolor{currentstroke}{rgb}{0.000000,0.000000,0.000000}%
\pgfsetstrokecolor{currentstroke}%
\pgfsetdash{}{0pt}%
\pgfpathmoveto{\pgfqpoint{4.282525in}{0.691133in}}%
\pgfpathlineto{\pgfqpoint{4.251812in}{0.668853in}}%
\pgfusepath{stroke}%
\end{pgfscope}%
\begin{pgfscope}%
\pgfsetrectcap%
\pgfsetroundjoin%
\pgfsetlinewidth{0.803000pt}%
\definecolor{currentstroke}{rgb}{0.000000,0.000000,0.000000}%
\pgfsetstrokecolor{currentstroke}%
\pgfsetdash{}{0pt}%
\pgfpathmoveto{\pgfqpoint{4.523341in}{0.505076in}}%
\pgfpathlineto{\pgfqpoint{4.492609in}{0.482262in}}%
\pgfusepath{stroke}%
\end{pgfscope}%
\begin{pgfscope}%
\pgfsetrectcap%
\pgfsetroundjoin%
\pgfsetlinewidth{0.803000pt}%
\definecolor{currentstroke}{rgb}{0.000000,0.000000,0.000000}%
\pgfsetstrokecolor{currentstroke}%
\pgfsetdash{}{0pt}%
\pgfpathmoveto{\pgfqpoint{4.769953in}{0.314540in}}%
\pgfpathlineto{\pgfqpoint{4.739210in}{0.291173in}}%
\pgfusepath{stroke}%
\end{pgfscope}%
\begin{pgfscope}%
\definecolor{textcolor}{rgb}{0.000000,0.000000,0.000000}%
\pgfsetstrokecolor{textcolor}%
\pgfsetfillcolor{textcolor}%
\pgftext[x=4.042153in,y=0.241958in,,]{\color{textcolor}{\rmfamily\fontsize{14.000000}{16.800000}\selectfont\catcode`\^=\active\def^{\ifmmode\sp\else\^{}\fi}\catcode`\%=\active\def%{\%}f2}}%
\end{pgfscope}%
\begin{pgfscope}%
\pgfsetrectcap%
\pgfsetroundjoin%
\pgfsetlinewidth{0.803000pt}%
\definecolor{currentstroke}{rgb}{0.000000,0.000000,0.000000}%
\pgfsetstrokecolor{currentstroke}%
\pgfsetdash{}{0pt}%
\pgfpathmoveto{\pgfqpoint{3.734658in}{1.099507in}}%
\pgfpathlineto{\pgfqpoint{3.656001in}{2.033906in}}%
\pgfusepath{stroke}%
\end{pgfscope}%
\begin{pgfscope}%
\pgfsetrectcap%
\pgfsetroundjoin%
\pgfsetlinewidth{0.803000pt}%
\definecolor{currentstroke}{rgb}{0.000000,0.000000,0.000000}%
\pgfsetstrokecolor{currentstroke}%
\pgfsetdash{}{0pt}%
\pgfpathmoveto{\pgfqpoint{3.740188in}{1.162580in}}%
\pgfpathlineto{\pgfqpoint{3.709419in}{1.141464in}}%
\pgfusepath{stroke}%
\end{pgfscope}%
\begin{pgfscope}%
\pgfsetrectcap%
\pgfsetroundjoin%
\pgfsetlinewidth{0.803000pt}%
\definecolor{currentstroke}{rgb}{0.000000,0.000000,0.000000}%
\pgfsetstrokecolor{currentstroke}%
\pgfsetdash{}{0pt}%
\pgfpathmoveto{\pgfqpoint{3.725377in}{1.340103in}}%
\pgfpathlineto{\pgfqpoint{3.694208in}{1.318976in}}%
\pgfusepath{stroke}%
\end{pgfscope}%
\begin{pgfscope}%
\pgfsetrectcap%
\pgfsetroundjoin%
\pgfsetlinewidth{0.803000pt}%
\definecolor{currentstroke}{rgb}{0.000000,0.000000,0.000000}%
\pgfsetstrokecolor{currentstroke}%
\pgfsetdash{}{0pt}%
\pgfpathmoveto{\pgfqpoint{3.710198in}{1.522058in}}%
\pgfpathlineto{\pgfqpoint{3.678617in}{1.500926in}}%
\pgfusepath{stroke}%
\end{pgfscope}%
\begin{pgfscope}%
\pgfsetrectcap%
\pgfsetroundjoin%
\pgfsetlinewidth{0.803000pt}%
\definecolor{currentstroke}{rgb}{0.000000,0.000000,0.000000}%
\pgfsetstrokecolor{currentstroke}%
\pgfsetdash{}{0pt}%
\pgfpathmoveto{\pgfqpoint{3.694634in}{1.708614in}}%
\pgfpathlineto{\pgfqpoint{3.662631in}{1.687484in}}%
\pgfusepath{stroke}%
\end{pgfscope}%
\begin{pgfscope}%
\pgfsetrectcap%
\pgfsetroundjoin%
\pgfsetlinewidth{0.803000pt}%
\definecolor{currentstroke}{rgb}{0.000000,0.000000,0.000000}%
\pgfsetstrokecolor{currentstroke}%
\pgfsetdash{}{0pt}%
\pgfpathmoveto{\pgfqpoint{3.678672in}{1.899948in}}%
\pgfpathlineto{\pgfqpoint{3.646235in}{1.878827in}}%
\pgfusepath{stroke}%
\end{pgfscope}%
\begin{pgfscope}%
\definecolor{textcolor}{rgb}{0.000000,0.000000,0.000000}%
\pgfsetstrokecolor{textcolor}%
\pgfsetfillcolor{textcolor}%
\pgftext[x=3.138409in,y=1.551958in,,]{\color{textcolor}{\rmfamily\fontsize{14.000000}{16.800000}\selectfont\catcode`\^=\active\def^{\ifmmode\sp\else\^{}\fi}\catcode`\%=\active\def%{\%}f3}}%
\end{pgfscope}%
\begin{pgfscope}%
\pgfpathrectangle{\pgfqpoint{3.536584in}{0.147348in}}{\pgfqpoint{2.735294in}{2.735294in}}%
\pgfusepath{clip}%
\pgfsetbuttcap%
\pgfsetroundjoin%
\definecolor{currentfill}{rgb}{0.050070,0.192203,0.290728}%
\pgfsetfillcolor{currentfill}%
\pgfsetlinewidth{0.000000pt}%
\definecolor{currentstroke}{rgb}{0.000000,0.000000,0.000000}%
\pgfsetstrokecolor{currentstroke}%
\pgfsetdash{}{0pt}%
\pgfpathmoveto{\pgfqpoint{5.897888in}{1.291641in}}%
\pgfpathlineto{\pgfqpoint{5.810696in}{1.165011in}}%
\pgfpathlineto{\pgfqpoint{5.897793in}{1.225003in}}%
\pgfpathlineto{\pgfqpoint{5.897888in}{1.291641in}}%
\pgfpathclose%
\pgfusepath{fill}%
\end{pgfscope}%
\begin{pgfscope}%
\pgfpathrectangle{\pgfqpoint{3.536584in}{0.147348in}}{\pgfqpoint{2.735294in}{2.735294in}}%
\pgfusepath{clip}%
\pgfsetbuttcap%
\pgfsetroundjoin%
\definecolor{currentfill}{rgb}{0.050070,0.192203,0.290728}%
\pgfsetfillcolor{currentfill}%
\pgfsetlinewidth{0.000000pt}%
\definecolor{currentstroke}{rgb}{0.000000,0.000000,0.000000}%
\pgfsetstrokecolor{currentstroke}%
\pgfsetdash{}{0pt}%
\pgfpathmoveto{\pgfqpoint{4.071693in}{1.165011in}}%
\pgfpathlineto{\pgfqpoint{3.984502in}{1.291641in}}%
\pgfpathlineto{\pgfqpoint{3.984596in}{1.225003in}}%
\pgfpathlineto{\pgfqpoint{4.071693in}{1.165011in}}%
\pgfpathclose%
\pgfusepath{fill}%
\end{pgfscope}%
\begin{pgfscope}%
\pgfpathrectangle{\pgfqpoint{3.536584in}{0.147348in}}{\pgfqpoint{2.735294in}{2.735294in}}%
\pgfusepath{clip}%
\pgfsetbuttcap%
\pgfsetroundjoin%
\definecolor{currentfill}{rgb}{0.090605,0.347808,0.526096}%
\pgfsetfillcolor{currentfill}%
\pgfsetlinewidth{0.000000pt}%
\definecolor{currentstroke}{rgb}{0.000000,0.000000,0.000000}%
\pgfsetstrokecolor{currentstroke}%
\pgfsetdash{}{0pt}%
\pgfpathmoveto{\pgfqpoint{4.854003in}{2.561465in}}%
\pgfpathlineto{\pgfqpoint{5.028387in}{2.561465in}}%
\pgfpathlineto{\pgfqpoint{4.941195in}{2.621838in}}%
\pgfpathlineto{\pgfqpoint{4.854003in}{2.561465in}}%
\pgfpathclose%
\pgfusepath{fill}%
\end{pgfscope}%
\begin{pgfscope}%
\pgfpathrectangle{\pgfqpoint{3.536584in}{0.147348in}}{\pgfqpoint{2.735294in}{2.735294in}}%
\pgfusepath{clip}%
\pgfsetbuttcap%
\pgfsetroundjoin%
\definecolor{currentfill}{rgb}{0.047548,0.182523,0.276086}%
\pgfsetfillcolor{currentfill}%
\pgfsetlinewidth{0.000000pt}%
\definecolor{currentstroke}{rgb}{0.000000,0.000000,0.000000}%
\pgfsetstrokecolor{currentstroke}%
\pgfsetdash{}{0pt}%
\pgfpathmoveto{\pgfqpoint{5.810696in}{1.165011in}}%
\pgfpathlineto{\pgfqpoint{5.801292in}{1.232896in}}%
\pgfpathlineto{\pgfqpoint{5.697728in}{1.101844in}}%
\pgfpathlineto{\pgfqpoint{5.810696in}{1.165011in}}%
\pgfpathclose%
\pgfusepath{fill}%
\end{pgfscope}%
\begin{pgfscope}%
\pgfpathrectangle{\pgfqpoint{3.536584in}{0.147348in}}{\pgfqpoint{2.735294in}{2.735294in}}%
\pgfusepath{clip}%
\pgfsetbuttcap%
\pgfsetroundjoin%
\definecolor{currentfill}{rgb}{0.047548,0.182523,0.276086}%
\pgfsetfillcolor{currentfill}%
\pgfsetlinewidth{0.000000pt}%
\definecolor{currentstroke}{rgb}{0.000000,0.000000,0.000000}%
\pgfsetstrokecolor{currentstroke}%
\pgfsetdash{}{0pt}%
\pgfpathmoveto{\pgfqpoint{4.184662in}{1.101844in}}%
\pgfpathlineto{\pgfqpoint{4.081098in}{1.232896in}}%
\pgfpathlineto{\pgfqpoint{4.071693in}{1.165011in}}%
\pgfpathlineto{\pgfqpoint{4.184662in}{1.101844in}}%
\pgfpathclose%
\pgfusepath{fill}%
\end{pgfscope}%
\begin{pgfscope}%
\pgfpathrectangle{\pgfqpoint{3.536584in}{0.147348in}}{\pgfqpoint{2.735294in}{2.735294in}}%
\pgfusepath{clip}%
\pgfsetbuttcap%
\pgfsetroundjoin%
\definecolor{currentfill}{rgb}{0.048960,0.187944,0.284285}%
\pgfsetfillcolor{currentfill}%
\pgfsetlinewidth{0.000000pt}%
\definecolor{currentstroke}{rgb}{0.000000,0.000000,0.000000}%
\pgfsetstrokecolor{currentstroke}%
\pgfsetdash{}{0pt}%
\pgfpathmoveto{\pgfqpoint{3.984502in}{1.291641in}}%
\pgfpathlineto{\pgfqpoint{4.071693in}{1.165011in}}%
\pgfpathlineto{\pgfqpoint{4.070905in}{1.618388in}}%
\pgfpathlineto{\pgfqpoint{3.984502in}{1.291641in}}%
\pgfpathclose%
\pgfusepath{fill}%
\end{pgfscope}%
\begin{pgfscope}%
\pgfpathrectangle{\pgfqpoint{3.536584in}{0.147348in}}{\pgfqpoint{2.735294in}{2.735294in}}%
\pgfusepath{clip}%
\pgfsetbuttcap%
\pgfsetroundjoin%
\definecolor{currentfill}{rgb}{0.048960,0.187944,0.284285}%
\pgfsetfillcolor{currentfill}%
\pgfsetlinewidth{0.000000pt}%
\definecolor{currentstroke}{rgb}{0.000000,0.000000,0.000000}%
\pgfsetstrokecolor{currentstroke}%
\pgfsetdash{}{0pt}%
\pgfpathmoveto{\pgfqpoint{5.897888in}{1.291641in}}%
\pgfpathlineto{\pgfqpoint{5.811485in}{1.618388in}}%
\pgfpathlineto{\pgfqpoint{5.810696in}{1.165011in}}%
\pgfpathlineto{\pgfqpoint{5.897888in}{1.291641in}}%
\pgfpathclose%
\pgfusepath{fill}%
\end{pgfscope}%
\begin{pgfscope}%
\pgfpathrectangle{\pgfqpoint{3.536584in}{0.147348in}}{\pgfqpoint{2.735294in}{2.735294in}}%
\pgfusepath{clip}%
\pgfsetbuttcap%
\pgfsetroundjoin%
\definecolor{currentfill}{rgb}{0.070254,0.269685,0.407928}%
\pgfsetfillcolor{currentfill}%
\pgfsetlinewidth{0.000000pt}%
\definecolor{currentstroke}{rgb}{0.000000,0.000000,0.000000}%
\pgfsetstrokecolor{currentstroke}%
\pgfsetdash{}{0pt}%
\pgfpathmoveto{\pgfqpoint{5.801292in}{1.232896in}}%
\pgfpathlineto{\pgfqpoint{5.810696in}{1.165011in}}%
\pgfpathlineto{\pgfqpoint{5.811485in}{1.618388in}}%
\pgfpathlineto{\pgfqpoint{5.801292in}{1.232896in}}%
\pgfpathclose%
\pgfusepath{fill}%
\end{pgfscope}%
\begin{pgfscope}%
\pgfpathrectangle{\pgfqpoint{3.536584in}{0.147348in}}{\pgfqpoint{2.735294in}{2.735294in}}%
\pgfusepath{clip}%
\pgfsetbuttcap%
\pgfsetroundjoin%
\definecolor{currentfill}{rgb}{0.070254,0.269685,0.407928}%
\pgfsetfillcolor{currentfill}%
\pgfsetlinewidth{0.000000pt}%
\definecolor{currentstroke}{rgb}{0.000000,0.000000,0.000000}%
\pgfsetstrokecolor{currentstroke}%
\pgfsetdash{}{0pt}%
\pgfpathmoveto{\pgfqpoint{4.070905in}{1.618388in}}%
\pgfpathlineto{\pgfqpoint{4.071693in}{1.165011in}}%
\pgfpathlineto{\pgfqpoint{4.081098in}{1.232896in}}%
\pgfpathlineto{\pgfqpoint{4.070905in}{1.618388in}}%
\pgfpathclose%
\pgfusepath{fill}%
\end{pgfscope}%
\begin{pgfscope}%
\pgfpathrectangle{\pgfqpoint{3.536584in}{0.147348in}}{\pgfqpoint{2.735294in}{2.735294in}}%
\pgfusepath{clip}%
\pgfsetbuttcap%
\pgfsetroundjoin%
\definecolor{currentfill}{rgb}{0.044978,0.172658,0.261163}%
\pgfsetfillcolor{currentfill}%
\pgfsetlinewidth{0.000000pt}%
\definecolor{currentstroke}{rgb}{0.000000,0.000000,0.000000}%
\pgfsetstrokecolor{currentstroke}%
\pgfsetdash{}{0pt}%
\pgfpathmoveto{\pgfqpoint{4.328767in}{1.039603in}}%
\pgfpathlineto{\pgfqpoint{4.209060in}{1.171211in}}%
\pgfpathlineto{\pgfqpoint{4.184662in}{1.101844in}}%
\pgfpathlineto{\pgfqpoint{4.328767in}{1.039603in}}%
\pgfpathclose%
\pgfusepath{fill}%
\end{pgfscope}%
\begin{pgfscope}%
\pgfpathrectangle{\pgfqpoint{3.536584in}{0.147348in}}{\pgfqpoint{2.735294in}{2.735294in}}%
\pgfusepath{clip}%
\pgfsetbuttcap%
\pgfsetroundjoin%
\definecolor{currentfill}{rgb}{0.044978,0.172658,0.261163}%
\pgfsetfillcolor{currentfill}%
\pgfsetlinewidth{0.000000pt}%
\definecolor{currentstroke}{rgb}{0.000000,0.000000,0.000000}%
\pgfsetstrokecolor{currentstroke}%
\pgfsetdash{}{0pt}%
\pgfpathmoveto{\pgfqpoint{5.697728in}{1.101844in}}%
\pgfpathlineto{\pgfqpoint{5.673330in}{1.171211in}}%
\pgfpathlineto{\pgfqpoint{5.553623in}{1.039603in}}%
\pgfpathlineto{\pgfqpoint{5.697728in}{1.101844in}}%
\pgfpathclose%
\pgfusepath{fill}%
\end{pgfscope}%
\begin{pgfscope}%
\pgfpathrectangle{\pgfqpoint{3.536584in}{0.147348in}}{\pgfqpoint{2.735294in}{2.735294in}}%
\pgfusepath{clip}%
\pgfsetbuttcap%
\pgfsetroundjoin%
\definecolor{currentfill}{rgb}{0.081954,0.314596,0.475860}%
\pgfsetfillcolor{currentfill}%
\pgfsetlinewidth{0.000000pt}%
\definecolor{currentstroke}{rgb}{0.000000,0.000000,0.000000}%
\pgfsetstrokecolor{currentstroke}%
\pgfsetdash{}{0pt}%
\pgfpathmoveto{\pgfqpoint{5.028387in}{2.561465in}}%
\pgfpathlineto{\pgfqpoint{4.854003in}{2.561465in}}%
\pgfpathlineto{\pgfqpoint{4.816407in}{2.124855in}}%
\pgfpathlineto{\pgfqpoint{5.028387in}{2.561465in}}%
\pgfpathclose%
\pgfusepath{fill}%
\end{pgfscope}%
\begin{pgfscope}%
\pgfpathrectangle{\pgfqpoint{3.536584in}{0.147348in}}{\pgfqpoint{2.735294in}{2.735294in}}%
\pgfusepath{clip}%
\pgfsetbuttcap%
\pgfsetroundjoin%
\definecolor{currentfill}{rgb}{0.047247,0.181368,0.274339}%
\pgfsetfillcolor{currentfill}%
\pgfsetlinewidth{0.000000pt}%
\definecolor{currentstroke}{rgb}{0.000000,0.000000,0.000000}%
\pgfsetstrokecolor{currentstroke}%
\pgfsetdash{}{0pt}%
\pgfpathmoveto{\pgfqpoint{5.664465in}{1.589448in}}%
\pgfpathlineto{\pgfqpoint{5.697728in}{1.101844in}}%
\pgfpathlineto{\pgfqpoint{5.801292in}{1.232896in}}%
\pgfpathlineto{\pgfqpoint{5.664465in}{1.589448in}}%
\pgfpathclose%
\pgfusepath{fill}%
\end{pgfscope}%
\begin{pgfscope}%
\pgfpathrectangle{\pgfqpoint{3.536584in}{0.147348in}}{\pgfqpoint{2.735294in}{2.735294in}}%
\pgfusepath{clip}%
\pgfsetbuttcap%
\pgfsetroundjoin%
\definecolor{currentfill}{rgb}{0.047247,0.181368,0.274339}%
\pgfsetfillcolor{currentfill}%
\pgfsetlinewidth{0.000000pt}%
\definecolor{currentstroke}{rgb}{0.000000,0.000000,0.000000}%
\pgfsetstrokecolor{currentstroke}%
\pgfsetdash{}{0pt}%
\pgfpathmoveto{\pgfqpoint{4.081098in}{1.232896in}}%
\pgfpathlineto{\pgfqpoint{4.184662in}{1.101844in}}%
\pgfpathlineto{\pgfqpoint{4.217925in}{1.589448in}}%
\pgfpathlineto{\pgfqpoint{4.081098in}{1.232896in}}%
\pgfpathclose%
\pgfusepath{fill}%
\end{pgfscope}%
\begin{pgfscope}%
\pgfpathrectangle{\pgfqpoint{3.536584in}{0.147348in}}{\pgfqpoint{2.735294in}{2.735294in}}%
\pgfusepath{clip}%
\pgfsetbuttcap%
\pgfsetroundjoin%
\definecolor{currentfill}{rgb}{0.067179,0.257880,0.390071}%
\pgfsetfillcolor{currentfill}%
\pgfsetlinewidth{0.000000pt}%
\definecolor{currentstroke}{rgb}{0.000000,0.000000,0.000000}%
\pgfsetstrokecolor{currentstroke}%
\pgfsetdash{}{0pt}%
\pgfpathmoveto{\pgfqpoint{4.217925in}{1.589448in}}%
\pgfpathlineto{\pgfqpoint{4.184662in}{1.101844in}}%
\pgfpathlineto{\pgfqpoint{4.209060in}{1.171211in}}%
\pgfpathlineto{\pgfqpoint{4.217925in}{1.589448in}}%
\pgfpathclose%
\pgfusepath{fill}%
\end{pgfscope}%
\begin{pgfscope}%
\pgfpathrectangle{\pgfqpoint{3.536584in}{0.147348in}}{\pgfqpoint{2.735294in}{2.735294in}}%
\pgfusepath{clip}%
\pgfsetbuttcap%
\pgfsetroundjoin%
\definecolor{currentfill}{rgb}{0.067179,0.257880,0.390071}%
\pgfsetfillcolor{currentfill}%
\pgfsetlinewidth{0.000000pt}%
\definecolor{currentstroke}{rgb}{0.000000,0.000000,0.000000}%
\pgfsetstrokecolor{currentstroke}%
\pgfsetdash{}{0pt}%
\pgfpathmoveto{\pgfqpoint{5.673330in}{1.171211in}}%
\pgfpathlineto{\pgfqpoint{5.697728in}{1.101844in}}%
\pgfpathlineto{\pgfqpoint{5.664465in}{1.589448in}}%
\pgfpathlineto{\pgfqpoint{5.673330in}{1.171211in}}%
\pgfpathclose%
\pgfusepath{fill}%
\end{pgfscope}%
\begin{pgfscope}%
\pgfpathrectangle{\pgfqpoint{3.536584in}{0.147348in}}{\pgfqpoint{2.735294in}{2.735294in}}%
\pgfusepath{clip}%
\pgfsetbuttcap%
\pgfsetroundjoin%
\definecolor{currentfill}{rgb}{0.042579,0.163449,0.247234}%
\pgfsetfillcolor{currentfill}%
\pgfsetlinewidth{0.000000pt}%
\definecolor{currentstroke}{rgb}{0.000000,0.000000,0.000000}%
\pgfsetstrokecolor{currentstroke}%
\pgfsetdash{}{0pt}%
\pgfpathmoveto{\pgfqpoint{4.328767in}{1.039603in}}%
\pgfpathlineto{\pgfqpoint{4.506598in}{0.985051in}}%
\pgfpathlineto{\pgfqpoint{4.374765in}{1.111775in}}%
\pgfpathlineto{\pgfqpoint{4.328767in}{1.039603in}}%
\pgfpathclose%
\pgfusepath{fill}%
\end{pgfscope}%
\begin{pgfscope}%
\pgfpathrectangle{\pgfqpoint{3.536584in}{0.147348in}}{\pgfqpoint{2.735294in}{2.735294in}}%
\pgfusepath{clip}%
\pgfsetbuttcap%
\pgfsetroundjoin%
\definecolor{currentfill}{rgb}{0.042579,0.163449,0.247234}%
\pgfsetfillcolor{currentfill}%
\pgfsetlinewidth{0.000000pt}%
\definecolor{currentstroke}{rgb}{0.000000,0.000000,0.000000}%
\pgfsetstrokecolor{currentstroke}%
\pgfsetdash{}{0pt}%
\pgfpathmoveto{\pgfqpoint{5.507624in}{1.111775in}}%
\pgfpathlineto{\pgfqpoint{5.375791in}{0.985051in}}%
\pgfpathlineto{\pgfqpoint{5.553623in}{1.039603in}}%
\pgfpathlineto{\pgfqpoint{5.507624in}{1.111775in}}%
\pgfpathclose%
\pgfusepath{fill}%
\end{pgfscope}%
\begin{pgfscope}%
\pgfpathrectangle{\pgfqpoint{3.536584in}{0.147348in}}{\pgfqpoint{2.735294in}{2.735294in}}%
\pgfusepath{clip}%
\pgfsetbuttcap%
\pgfsetroundjoin%
\definecolor{currentfill}{rgb}{0.052493,0.201505,0.304798}%
\pgfsetfillcolor{currentfill}%
\pgfsetlinewidth{0.000000pt}%
\definecolor{currentstroke}{rgb}{0.000000,0.000000,0.000000}%
\pgfsetstrokecolor{currentstroke}%
\pgfsetdash{}{0pt}%
\pgfpathmoveto{\pgfqpoint{5.664465in}{1.589448in}}%
\pgfpathlineto{\pgfqpoint{5.801292in}{1.232896in}}%
\pgfpathlineto{\pgfqpoint{5.811485in}{1.618388in}}%
\pgfpathlineto{\pgfqpoint{5.664465in}{1.589448in}}%
\pgfpathclose%
\pgfusepath{fill}%
\end{pgfscope}%
\begin{pgfscope}%
\pgfpathrectangle{\pgfqpoint{3.536584in}{0.147348in}}{\pgfqpoint{2.735294in}{2.735294in}}%
\pgfusepath{clip}%
\pgfsetbuttcap%
\pgfsetroundjoin%
\definecolor{currentfill}{rgb}{0.052493,0.201505,0.304798}%
\pgfsetfillcolor{currentfill}%
\pgfsetlinewidth{0.000000pt}%
\definecolor{currentstroke}{rgb}{0.000000,0.000000,0.000000}%
\pgfsetstrokecolor{currentstroke}%
\pgfsetdash{}{0pt}%
\pgfpathmoveto{\pgfqpoint{4.070905in}{1.618388in}}%
\pgfpathlineto{\pgfqpoint{4.081098in}{1.232896in}}%
\pgfpathlineto{\pgfqpoint{4.217925in}{1.589448in}}%
\pgfpathlineto{\pgfqpoint{4.070905in}{1.618388in}}%
\pgfpathclose%
\pgfusepath{fill}%
\end{pgfscope}%
\begin{pgfscope}%
\pgfpathrectangle{\pgfqpoint{3.536584in}{0.147348in}}{\pgfqpoint{2.735294in}{2.735294in}}%
\pgfusepath{clip}%
\pgfsetbuttcap%
\pgfsetroundjoin%
\definecolor{currentfill}{rgb}{0.082280,0.315849,0.477755}%
\pgfsetfillcolor{currentfill}%
\pgfsetlinewidth{0.000000pt}%
\definecolor{currentstroke}{rgb}{0.000000,0.000000,0.000000}%
\pgfsetstrokecolor{currentstroke}%
\pgfsetdash{}{0pt}%
\pgfpathmoveto{\pgfqpoint{5.376890in}{2.253326in}}%
\pgfpathlineto{\pgfqpoint{5.028387in}{2.561465in}}%
\pgfpathlineto{\pgfqpoint{5.065982in}{2.124855in}}%
\pgfpathlineto{\pgfqpoint{5.376890in}{2.253326in}}%
\pgfpathclose%
\pgfusepath{fill}%
\end{pgfscope}%
\begin{pgfscope}%
\pgfpathrectangle{\pgfqpoint{3.536584in}{0.147348in}}{\pgfqpoint{2.735294in}{2.735294in}}%
\pgfusepath{clip}%
\pgfsetbuttcap%
\pgfsetroundjoin%
\definecolor{currentfill}{rgb}{0.082280,0.315849,0.477755}%
\pgfsetfillcolor{currentfill}%
\pgfsetlinewidth{0.000000pt}%
\definecolor{currentstroke}{rgb}{0.000000,0.000000,0.000000}%
\pgfsetstrokecolor{currentstroke}%
\pgfsetdash{}{0pt}%
\pgfpathmoveto{\pgfqpoint{4.816407in}{2.124855in}}%
\pgfpathlineto{\pgfqpoint{4.854003in}{2.561465in}}%
\pgfpathlineto{\pgfqpoint{4.505500in}{2.253326in}}%
\pgfpathlineto{\pgfqpoint{4.816407in}{2.124855in}}%
\pgfpathclose%
\pgfusepath{fill}%
\end{pgfscope}%
\begin{pgfscope}%
\pgfpathrectangle{\pgfqpoint{3.536584in}{0.147348in}}{\pgfqpoint{2.735294in}{2.735294in}}%
\pgfusepath{clip}%
\pgfsetbuttcap%
\pgfsetroundjoin%
\definecolor{currentfill}{rgb}{0.045702,0.175435,0.265364}%
\pgfsetfillcolor{currentfill}%
\pgfsetlinewidth{0.000000pt}%
\definecolor{currentstroke}{rgb}{0.000000,0.000000,0.000000}%
\pgfsetstrokecolor{currentstroke}%
\pgfsetdash{}{0pt}%
\pgfpathmoveto{\pgfqpoint{4.209060in}{1.171211in}}%
\pgfpathlineto{\pgfqpoint{4.328767in}{1.039603in}}%
\pgfpathlineto{\pgfqpoint{4.414548in}{1.562087in}}%
\pgfpathlineto{\pgfqpoint{4.209060in}{1.171211in}}%
\pgfpathclose%
\pgfusepath{fill}%
\end{pgfscope}%
\begin{pgfscope}%
\pgfpathrectangle{\pgfqpoint{3.536584in}{0.147348in}}{\pgfqpoint{2.735294in}{2.735294in}}%
\pgfusepath{clip}%
\pgfsetbuttcap%
\pgfsetroundjoin%
\definecolor{currentfill}{rgb}{0.045702,0.175435,0.265364}%
\pgfsetfillcolor{currentfill}%
\pgfsetlinewidth{0.000000pt}%
\definecolor{currentstroke}{rgb}{0.000000,0.000000,0.000000}%
\pgfsetstrokecolor{currentstroke}%
\pgfsetdash{}{0pt}%
\pgfpathmoveto{\pgfqpoint{5.467842in}{1.562087in}}%
\pgfpathlineto{\pgfqpoint{5.553623in}{1.039603in}}%
\pgfpathlineto{\pgfqpoint{5.673330in}{1.171211in}}%
\pgfpathlineto{\pgfqpoint{5.467842in}{1.562087in}}%
\pgfpathclose%
\pgfusepath{fill}%
\end{pgfscope}%
\begin{pgfscope}%
\pgfpathrectangle{\pgfqpoint{3.536584in}{0.147348in}}{\pgfqpoint{2.735294in}{2.735294in}}%
\pgfusepath{clip}%
\pgfsetbuttcap%
\pgfsetroundjoin%
\definecolor{currentfill}{rgb}{0.040669,0.156116,0.236142}%
\pgfsetfillcolor{currentfill}%
\pgfsetlinewidth{0.000000pt}%
\definecolor{currentstroke}{rgb}{0.000000,0.000000,0.000000}%
\pgfsetstrokecolor{currentstroke}%
\pgfsetdash{}{0pt}%
\pgfpathmoveto{\pgfqpoint{4.579939in}{1.063020in}}%
\pgfpathlineto{\pgfqpoint{4.506598in}{0.985051in}}%
\pgfpathlineto{\pgfqpoint{4.714698in}{0.946905in}}%
\pgfpathlineto{\pgfqpoint{4.579939in}{1.063020in}}%
\pgfpathclose%
\pgfusepath{fill}%
\end{pgfscope}%
\begin{pgfscope}%
\pgfpathrectangle{\pgfqpoint{3.536584in}{0.147348in}}{\pgfqpoint{2.735294in}{2.735294in}}%
\pgfusepath{clip}%
\pgfsetbuttcap%
\pgfsetroundjoin%
\definecolor{currentfill}{rgb}{0.040669,0.156116,0.236142}%
\pgfsetfillcolor{currentfill}%
\pgfsetlinewidth{0.000000pt}%
\definecolor{currentstroke}{rgb}{0.000000,0.000000,0.000000}%
\pgfsetstrokecolor{currentstroke}%
\pgfsetdash{}{0pt}%
\pgfpathmoveto{\pgfqpoint{5.167692in}{0.946905in}}%
\pgfpathlineto{\pgfqpoint{5.375791in}{0.985051in}}%
\pgfpathlineto{\pgfqpoint{5.302451in}{1.063020in}}%
\pgfpathlineto{\pgfqpoint{5.167692in}{0.946905in}}%
\pgfpathclose%
\pgfusepath{fill}%
\end{pgfscope}%
\begin{pgfscope}%
\pgfpathrectangle{\pgfqpoint{3.536584in}{0.147348in}}{\pgfqpoint{2.735294in}{2.735294in}}%
\pgfusepath{clip}%
\pgfsetbuttcap%
\pgfsetroundjoin%
\definecolor{currentfill}{rgb}{0.063981,0.245604,0.371502}%
\pgfsetfillcolor{currentfill}%
\pgfsetlinewidth{0.000000pt}%
\definecolor{currentstroke}{rgb}{0.000000,0.000000,0.000000}%
\pgfsetstrokecolor{currentstroke}%
\pgfsetdash{}{0pt}%
\pgfpathmoveto{\pgfqpoint{4.414548in}{1.562087in}}%
\pgfpathlineto{\pgfqpoint{4.328767in}{1.039603in}}%
\pgfpathlineto{\pgfqpoint{4.374765in}{1.111775in}}%
\pgfpathlineto{\pgfqpoint{4.414548in}{1.562087in}}%
\pgfpathclose%
\pgfusepath{fill}%
\end{pgfscope}%
\begin{pgfscope}%
\pgfpathrectangle{\pgfqpoint{3.536584in}{0.147348in}}{\pgfqpoint{2.735294in}{2.735294in}}%
\pgfusepath{clip}%
\pgfsetbuttcap%
\pgfsetroundjoin%
\definecolor{currentfill}{rgb}{0.063981,0.245604,0.371502}%
\pgfsetfillcolor{currentfill}%
\pgfsetlinewidth{0.000000pt}%
\definecolor{currentstroke}{rgb}{0.000000,0.000000,0.000000}%
\pgfsetstrokecolor{currentstroke}%
\pgfsetdash{}{0pt}%
\pgfpathmoveto{\pgfqpoint{5.507624in}{1.111775in}}%
\pgfpathlineto{\pgfqpoint{5.553623in}{1.039603in}}%
\pgfpathlineto{\pgfqpoint{5.467842in}{1.562087in}}%
\pgfpathlineto{\pgfqpoint{5.507624in}{1.111775in}}%
\pgfpathclose%
\pgfusepath{fill}%
\end{pgfscope}%
\begin{pgfscope}%
\pgfpathrectangle{\pgfqpoint{3.536584in}{0.147348in}}{\pgfqpoint{2.735294in}{2.735294in}}%
\pgfusepath{clip}%
\pgfsetbuttcap%
\pgfsetroundjoin%
\definecolor{currentfill}{rgb}{0.060942,0.233938,0.353856}%
\pgfsetfillcolor{currentfill}%
\pgfsetlinewidth{0.000000pt}%
\definecolor{currentstroke}{rgb}{0.000000,0.000000,0.000000}%
\pgfsetstrokecolor{currentstroke}%
\pgfsetdash{}{0pt}%
\pgfpathmoveto{\pgfqpoint{4.217925in}{1.589448in}}%
\pgfpathlineto{\pgfqpoint{4.147150in}{1.772344in}}%
\pgfpathlineto{\pgfqpoint{4.070905in}{1.618388in}}%
\pgfpathlineto{\pgfqpoint{4.217925in}{1.589448in}}%
\pgfpathclose%
\pgfusepath{fill}%
\end{pgfscope}%
\begin{pgfscope}%
\pgfpathrectangle{\pgfqpoint{3.536584in}{0.147348in}}{\pgfqpoint{2.735294in}{2.735294in}}%
\pgfusepath{clip}%
\pgfsetbuttcap%
\pgfsetroundjoin%
\definecolor{currentfill}{rgb}{0.060942,0.233938,0.353856}%
\pgfsetfillcolor{currentfill}%
\pgfsetlinewidth{0.000000pt}%
\definecolor{currentstroke}{rgb}{0.000000,0.000000,0.000000}%
\pgfsetstrokecolor{currentstroke}%
\pgfsetdash{}{0pt}%
\pgfpathmoveto{\pgfqpoint{5.811485in}{1.618388in}}%
\pgfpathlineto{\pgfqpoint{5.735240in}{1.772344in}}%
\pgfpathlineto{\pgfqpoint{5.664465in}{1.589448in}}%
\pgfpathlineto{\pgfqpoint{5.811485in}{1.618388in}}%
\pgfpathclose%
\pgfusepath{fill}%
\end{pgfscope}%
\begin{pgfscope}%
\pgfpathrectangle{\pgfqpoint{3.536584in}{0.147348in}}{\pgfqpoint{2.735294in}{2.735294in}}%
\pgfusepath{clip}%
\pgfsetbuttcap%
\pgfsetroundjoin%
\definecolor{currentfill}{rgb}{0.081954,0.314596,0.475860}%
\pgfsetfillcolor{currentfill}%
\pgfsetlinewidth{0.000000pt}%
\definecolor{currentstroke}{rgb}{0.000000,0.000000,0.000000}%
\pgfsetstrokecolor{currentstroke}%
\pgfsetdash{}{0pt}%
\pgfpathmoveto{\pgfqpoint{4.816407in}{2.124855in}}%
\pgfpathlineto{\pgfqpoint{5.065982in}{2.124855in}}%
\pgfpathlineto{\pgfqpoint{5.028387in}{2.561465in}}%
\pgfpathlineto{\pgfqpoint{4.816407in}{2.124855in}}%
\pgfpathclose%
\pgfusepath{fill}%
\end{pgfscope}%
\begin{pgfscope}%
\pgfpathrectangle{\pgfqpoint{3.536584in}{0.147348in}}{\pgfqpoint{2.735294in}{2.735294in}}%
\pgfusepath{clip}%
\pgfsetbuttcap%
\pgfsetroundjoin%
\definecolor{currentfill}{rgb}{0.039595,0.151995,0.229908}%
\pgfsetfillcolor{currentfill}%
\pgfsetlinewidth{0.000000pt}%
\definecolor{currentstroke}{rgb}{0.000000,0.000000,0.000000}%
\pgfsetstrokecolor{currentstroke}%
\pgfsetdash{}{0pt}%
\pgfpathmoveto{\pgfqpoint{4.941195in}{0.933095in}}%
\pgfpathlineto{\pgfqpoint{4.816678in}{1.035006in}}%
\pgfpathlineto{\pgfqpoint{4.714698in}{0.946905in}}%
\pgfpathlineto{\pgfqpoint{4.941195in}{0.933095in}}%
\pgfpathclose%
\pgfusepath{fill}%
\end{pgfscope}%
\begin{pgfscope}%
\pgfpathrectangle{\pgfqpoint{3.536584in}{0.147348in}}{\pgfqpoint{2.735294in}{2.735294in}}%
\pgfusepath{clip}%
\pgfsetbuttcap%
\pgfsetroundjoin%
\definecolor{currentfill}{rgb}{0.039595,0.151995,0.229908}%
\pgfsetfillcolor{currentfill}%
\pgfsetlinewidth{0.000000pt}%
\definecolor{currentstroke}{rgb}{0.000000,0.000000,0.000000}%
\pgfsetstrokecolor{currentstroke}%
\pgfsetdash{}{0pt}%
\pgfpathmoveto{\pgfqpoint{5.167692in}{0.946905in}}%
\pgfpathlineto{\pgfqpoint{5.065712in}{1.035006in}}%
\pgfpathlineto{\pgfqpoint{4.941195in}{0.933095in}}%
\pgfpathlineto{\pgfqpoint{5.167692in}{0.946905in}}%
\pgfpathclose%
\pgfusepath{fill}%
\end{pgfscope}%
\begin{pgfscope}%
\pgfpathrectangle{\pgfqpoint{3.536584in}{0.147348in}}{\pgfqpoint{2.735294in}{2.735294in}}%
\pgfusepath{clip}%
\pgfsetbuttcap%
\pgfsetroundjoin%
\definecolor{currentfill}{rgb}{0.075436,0.289576,0.438014}%
\pgfsetfillcolor{currentfill}%
\pgfsetlinewidth{0.000000pt}%
\definecolor{currentstroke}{rgb}{0.000000,0.000000,0.000000}%
\pgfsetstrokecolor{currentstroke}%
\pgfsetdash{}{0pt}%
\pgfpathmoveto{\pgfqpoint{5.508743in}{2.103280in}}%
\pgfpathlineto{\pgfqpoint{5.376890in}{2.253326in}}%
\pgfpathlineto{\pgfqpoint{5.303210in}{2.116981in}}%
\pgfpathlineto{\pgfqpoint{5.508743in}{2.103280in}}%
\pgfpathclose%
\pgfusepath{fill}%
\end{pgfscope}%
\begin{pgfscope}%
\pgfpathrectangle{\pgfqpoint{3.536584in}{0.147348in}}{\pgfqpoint{2.735294in}{2.735294in}}%
\pgfusepath{clip}%
\pgfsetbuttcap%
\pgfsetroundjoin%
\definecolor{currentfill}{rgb}{0.075436,0.289576,0.438014}%
\pgfsetfillcolor{currentfill}%
\pgfsetlinewidth{0.000000pt}%
\definecolor{currentstroke}{rgb}{0.000000,0.000000,0.000000}%
\pgfsetstrokecolor{currentstroke}%
\pgfsetdash{}{0pt}%
\pgfpathmoveto{\pgfqpoint{4.579180in}{2.116981in}}%
\pgfpathlineto{\pgfqpoint{4.505500in}{2.253326in}}%
\pgfpathlineto{\pgfqpoint{4.373646in}{2.103280in}}%
\pgfpathlineto{\pgfqpoint{4.579180in}{2.116981in}}%
\pgfpathclose%
\pgfusepath{fill}%
\end{pgfscope}%
\begin{pgfscope}%
\pgfpathrectangle{\pgfqpoint{3.536584in}{0.147348in}}{\pgfqpoint{2.735294in}{2.735294in}}%
\pgfusepath{clip}%
\pgfsetbuttcap%
\pgfsetroundjoin%
\definecolor{currentfill}{rgb}{0.062760,0.240916,0.364410}%
\pgfsetfillcolor{currentfill}%
\pgfsetlinewidth{0.000000pt}%
\definecolor{currentstroke}{rgb}{0.000000,0.000000,0.000000}%
\pgfsetstrokecolor{currentstroke}%
\pgfsetdash{}{0pt}%
\pgfpathmoveto{\pgfqpoint{4.217925in}{1.589448in}}%
\pgfpathlineto{\pgfqpoint{4.373646in}{2.103280in}}%
\pgfpathlineto{\pgfqpoint{4.147150in}{1.772344in}}%
\pgfpathlineto{\pgfqpoint{4.217925in}{1.589448in}}%
\pgfpathclose%
\pgfusepath{fill}%
\end{pgfscope}%
\begin{pgfscope}%
\pgfpathrectangle{\pgfqpoint{3.536584in}{0.147348in}}{\pgfqpoint{2.735294in}{2.735294in}}%
\pgfusepath{clip}%
\pgfsetbuttcap%
\pgfsetroundjoin%
\definecolor{currentfill}{rgb}{0.062760,0.240916,0.364410}%
\pgfsetfillcolor{currentfill}%
\pgfsetlinewidth{0.000000pt}%
\definecolor{currentstroke}{rgb}{0.000000,0.000000,0.000000}%
\pgfsetstrokecolor{currentstroke}%
\pgfsetdash{}{0pt}%
\pgfpathmoveto{\pgfqpoint{5.735240in}{1.772344in}}%
\pgfpathlineto{\pgfqpoint{5.508743in}{2.103280in}}%
\pgfpathlineto{\pgfqpoint{5.664465in}{1.589448in}}%
\pgfpathlineto{\pgfqpoint{5.735240in}{1.772344in}}%
\pgfpathclose%
\pgfusepath{fill}%
\end{pgfscope}%
\begin{pgfscope}%
\pgfpathrectangle{\pgfqpoint{3.536584in}{0.147348in}}{\pgfqpoint{2.735294in}{2.735294in}}%
\pgfusepath{clip}%
\pgfsetbuttcap%
\pgfsetroundjoin%
\definecolor{currentfill}{rgb}{0.043508,0.167016,0.252629}%
\pgfsetfillcolor{currentfill}%
\pgfsetlinewidth{0.000000pt}%
\definecolor{currentstroke}{rgb}{0.000000,0.000000,0.000000}%
\pgfsetstrokecolor{currentstroke}%
\pgfsetdash{}{0pt}%
\pgfpathmoveto{\pgfqpoint{4.374765in}{1.111775in}}%
\pgfpathlineto{\pgfqpoint{4.506598in}{0.985051in}}%
\pgfpathlineto{\pgfqpoint{4.538549in}{1.360106in}}%
\pgfpathlineto{\pgfqpoint{4.374765in}{1.111775in}}%
\pgfpathclose%
\pgfusepath{fill}%
\end{pgfscope}%
\begin{pgfscope}%
\pgfpathrectangle{\pgfqpoint{3.536584in}{0.147348in}}{\pgfqpoint{2.735294in}{2.735294in}}%
\pgfusepath{clip}%
\pgfsetbuttcap%
\pgfsetroundjoin%
\definecolor{currentfill}{rgb}{0.043508,0.167016,0.252629}%
\pgfsetfillcolor{currentfill}%
\pgfsetlinewidth{0.000000pt}%
\definecolor{currentstroke}{rgb}{0.000000,0.000000,0.000000}%
\pgfsetstrokecolor{currentstroke}%
\pgfsetdash{}{0pt}%
\pgfpathmoveto{\pgfqpoint{5.343841in}{1.360106in}}%
\pgfpathlineto{\pgfqpoint{5.375791in}{0.985051in}}%
\pgfpathlineto{\pgfqpoint{5.507624in}{1.111775in}}%
\pgfpathlineto{\pgfqpoint{5.343841in}{1.360106in}}%
\pgfpathclose%
\pgfusepath{fill}%
\end{pgfscope}%
\begin{pgfscope}%
\pgfpathrectangle{\pgfqpoint{3.536584in}{0.147348in}}{\pgfqpoint{2.735294in}{2.735294in}}%
\pgfusepath{clip}%
\pgfsetbuttcap%
\pgfsetroundjoin%
\definecolor{currentfill}{rgb}{0.050011,0.191979,0.290388}%
\pgfsetfillcolor{currentfill}%
\pgfsetlinewidth{0.000000pt}%
\definecolor{currentstroke}{rgb}{0.000000,0.000000,0.000000}%
\pgfsetstrokecolor{currentstroke}%
\pgfsetdash{}{0pt}%
\pgfpathmoveto{\pgfqpoint{4.217925in}{1.589448in}}%
\pgfpathlineto{\pgfqpoint{4.209060in}{1.171211in}}%
\pgfpathlineto{\pgfqpoint{4.414548in}{1.562087in}}%
\pgfpathlineto{\pgfqpoint{4.217925in}{1.589448in}}%
\pgfpathclose%
\pgfusepath{fill}%
\end{pgfscope}%
\begin{pgfscope}%
\pgfpathrectangle{\pgfqpoint{3.536584in}{0.147348in}}{\pgfqpoint{2.735294in}{2.735294in}}%
\pgfusepath{clip}%
\pgfsetbuttcap%
\pgfsetroundjoin%
\definecolor{currentfill}{rgb}{0.050011,0.191979,0.290388}%
\pgfsetfillcolor{currentfill}%
\pgfsetlinewidth{0.000000pt}%
\definecolor{currentstroke}{rgb}{0.000000,0.000000,0.000000}%
\pgfsetstrokecolor{currentstroke}%
\pgfsetdash{}{0pt}%
\pgfpathmoveto{\pgfqpoint{5.467842in}{1.562087in}}%
\pgfpathlineto{\pgfqpoint{5.673330in}{1.171211in}}%
\pgfpathlineto{\pgfqpoint{5.664465in}{1.589448in}}%
\pgfpathlineto{\pgfqpoint{5.467842in}{1.562087in}}%
\pgfpathclose%
\pgfusepath{fill}%
\end{pgfscope}%
\begin{pgfscope}%
\pgfpathrectangle{\pgfqpoint{3.536584in}{0.147348in}}{\pgfqpoint{2.735294in}{2.735294in}}%
\pgfusepath{clip}%
\pgfsetbuttcap%
\pgfsetroundjoin%
\definecolor{currentfill}{rgb}{0.049941,0.191710,0.289982}%
\pgfsetfillcolor{currentfill}%
\pgfsetlinewidth{0.000000pt}%
\definecolor{currentstroke}{rgb}{0.000000,0.000000,0.000000}%
\pgfsetstrokecolor{currentstroke}%
\pgfsetdash{}{0pt}%
\pgfpathmoveto{\pgfqpoint{5.302451in}{1.063020in}}%
\pgfpathlineto{\pgfqpoint{5.375791in}{0.985051in}}%
\pgfpathlineto{\pgfqpoint{5.343841in}{1.360106in}}%
\pgfpathlineto{\pgfqpoint{5.302451in}{1.063020in}}%
\pgfpathclose%
\pgfusepath{fill}%
\end{pgfscope}%
\begin{pgfscope}%
\pgfpathrectangle{\pgfqpoint{3.536584in}{0.147348in}}{\pgfqpoint{2.735294in}{2.735294in}}%
\pgfusepath{clip}%
\pgfsetbuttcap%
\pgfsetroundjoin%
\definecolor{currentfill}{rgb}{0.049941,0.191710,0.289982}%
\pgfsetfillcolor{currentfill}%
\pgfsetlinewidth{0.000000pt}%
\definecolor{currentstroke}{rgb}{0.000000,0.000000,0.000000}%
\pgfsetstrokecolor{currentstroke}%
\pgfsetdash{}{0pt}%
\pgfpathmoveto{\pgfqpoint{4.538549in}{1.360106in}}%
\pgfpathlineto{\pgfqpoint{4.506598in}{0.985051in}}%
\pgfpathlineto{\pgfqpoint{4.579939in}{1.063020in}}%
\pgfpathlineto{\pgfqpoint{4.538549in}{1.360106in}}%
\pgfpathclose%
\pgfusepath{fill}%
\end{pgfscope}%
\begin{pgfscope}%
\pgfpathrectangle{\pgfqpoint{3.536584in}{0.147348in}}{\pgfqpoint{2.735294in}{2.735294in}}%
\pgfusepath{clip}%
\pgfsetbuttcap%
\pgfsetroundjoin%
\definecolor{currentfill}{rgb}{0.078663,0.301965,0.456754}%
\pgfsetfillcolor{currentfill}%
\pgfsetlinewidth{0.000000pt}%
\definecolor{currentstroke}{rgb}{0.000000,0.000000,0.000000}%
\pgfsetstrokecolor{currentstroke}%
\pgfsetdash{}{0pt}%
\pgfpathmoveto{\pgfqpoint{4.816407in}{2.124855in}}%
\pgfpathlineto{\pgfqpoint{4.505500in}{2.253326in}}%
\pgfpathlineto{\pgfqpoint{4.579180in}{2.116981in}}%
\pgfpathlineto{\pgfqpoint{4.816407in}{2.124855in}}%
\pgfpathclose%
\pgfusepath{fill}%
\end{pgfscope}%
\begin{pgfscope}%
\pgfpathrectangle{\pgfqpoint{3.536584in}{0.147348in}}{\pgfqpoint{2.735294in}{2.735294in}}%
\pgfusepath{clip}%
\pgfsetbuttcap%
\pgfsetroundjoin%
\definecolor{currentfill}{rgb}{0.078663,0.301965,0.456754}%
\pgfsetfillcolor{currentfill}%
\pgfsetlinewidth{0.000000pt}%
\definecolor{currentstroke}{rgb}{0.000000,0.000000,0.000000}%
\pgfsetstrokecolor{currentstroke}%
\pgfsetdash{}{0pt}%
\pgfpathmoveto{\pgfqpoint{5.303210in}{2.116981in}}%
\pgfpathlineto{\pgfqpoint{5.376890in}{2.253326in}}%
\pgfpathlineto{\pgfqpoint{5.065982in}{2.124855in}}%
\pgfpathlineto{\pgfqpoint{5.303210in}{2.116981in}}%
\pgfpathclose%
\pgfusepath{fill}%
\end{pgfscope}%
\begin{pgfscope}%
\pgfpathrectangle{\pgfqpoint{3.536584in}{0.147348in}}{\pgfqpoint{2.735294in}{2.735294in}}%
\pgfusepath{clip}%
\pgfsetbuttcap%
\pgfsetroundjoin%
\definecolor{currentfill}{rgb}{0.064954,0.249341,0.377155}%
\pgfsetfillcolor{currentfill}%
\pgfsetlinewidth{0.000000pt}%
\definecolor{currentstroke}{rgb}{0.000000,0.000000,0.000000}%
\pgfsetstrokecolor{currentstroke}%
\pgfsetdash{}{0pt}%
\pgfpathmoveto{\pgfqpoint{4.414548in}{1.562087in}}%
\pgfpathlineto{\pgfqpoint{4.373646in}{2.103280in}}%
\pgfpathlineto{\pgfqpoint{4.217925in}{1.589448in}}%
\pgfpathlineto{\pgfqpoint{4.414548in}{1.562087in}}%
\pgfpathclose%
\pgfusepath{fill}%
\end{pgfscope}%
\begin{pgfscope}%
\pgfpathrectangle{\pgfqpoint{3.536584in}{0.147348in}}{\pgfqpoint{2.735294in}{2.735294in}}%
\pgfusepath{clip}%
\pgfsetbuttcap%
\pgfsetroundjoin%
\definecolor{currentfill}{rgb}{0.064954,0.249341,0.377155}%
\pgfsetfillcolor{currentfill}%
\pgfsetlinewidth{0.000000pt}%
\definecolor{currentstroke}{rgb}{0.000000,0.000000,0.000000}%
\pgfsetstrokecolor{currentstroke}%
\pgfsetdash{}{0pt}%
\pgfpathmoveto{\pgfqpoint{5.664465in}{1.589448in}}%
\pgfpathlineto{\pgfqpoint{5.508743in}{2.103280in}}%
\pgfpathlineto{\pgfqpoint{5.467842in}{1.562087in}}%
\pgfpathlineto{\pgfqpoint{5.664465in}{1.589448in}}%
\pgfpathclose%
\pgfusepath{fill}%
\end{pgfscope}%
\begin{pgfscope}%
\pgfpathrectangle{\pgfqpoint{3.536584in}{0.147348in}}{\pgfqpoint{2.735294in}{2.735294in}}%
\pgfusepath{clip}%
\pgfsetbuttcap%
\pgfsetroundjoin%
\definecolor{currentfill}{rgb}{0.042669,0.163794,0.247755}%
\pgfsetfillcolor{currentfill}%
\pgfsetlinewidth{0.000000pt}%
\definecolor{currentstroke}{rgb}{0.000000,0.000000,0.000000}%
\pgfsetstrokecolor{currentstroke}%
\pgfsetdash{}{0pt}%
\pgfpathmoveto{\pgfqpoint{4.714698in}{0.946905in}}%
\pgfpathlineto{\pgfqpoint{4.801285in}{1.337788in}}%
\pgfpathlineto{\pgfqpoint{4.579939in}{1.063020in}}%
\pgfpathlineto{\pgfqpoint{4.714698in}{0.946905in}}%
\pgfpathclose%
\pgfusepath{fill}%
\end{pgfscope}%
\begin{pgfscope}%
\pgfpathrectangle{\pgfqpoint{3.536584in}{0.147348in}}{\pgfqpoint{2.735294in}{2.735294in}}%
\pgfusepath{clip}%
\pgfsetbuttcap%
\pgfsetroundjoin%
\definecolor{currentfill}{rgb}{0.042669,0.163794,0.247755}%
\pgfsetfillcolor{currentfill}%
\pgfsetlinewidth{0.000000pt}%
\definecolor{currentstroke}{rgb}{0.000000,0.000000,0.000000}%
\pgfsetstrokecolor{currentstroke}%
\pgfsetdash{}{0pt}%
\pgfpathmoveto{\pgfqpoint{5.302451in}{1.063020in}}%
\pgfpathlineto{\pgfqpoint{5.081105in}{1.337788in}}%
\pgfpathlineto{\pgfqpoint{5.167692in}{0.946905in}}%
\pgfpathlineto{\pgfqpoint{5.302451in}{1.063020in}}%
\pgfpathclose%
\pgfusepath{fill}%
\end{pgfscope}%
\begin{pgfscope}%
\pgfpathrectangle{\pgfqpoint{3.536584in}{0.147348in}}{\pgfqpoint{2.735294in}{2.735294in}}%
\pgfusepath{clip}%
\pgfsetbuttcap%
\pgfsetroundjoin%
\definecolor{currentfill}{rgb}{0.068541,0.263111,0.397982}%
\pgfsetfillcolor{currentfill}%
\pgfsetlinewidth{0.000000pt}%
\definecolor{currentstroke}{rgb}{0.000000,0.000000,0.000000}%
\pgfsetstrokecolor{currentstroke}%
\pgfsetdash{}{0pt}%
\pgfpathmoveto{\pgfqpoint{4.579180in}{2.116981in}}%
\pgfpathlineto{\pgfqpoint{4.373646in}{2.103280in}}%
\pgfpathlineto{\pgfqpoint{4.677670in}{1.946079in}}%
\pgfpathlineto{\pgfqpoint{4.579180in}{2.116981in}}%
\pgfpathclose%
\pgfusepath{fill}%
\end{pgfscope}%
\begin{pgfscope}%
\pgfpathrectangle{\pgfqpoint{3.536584in}{0.147348in}}{\pgfqpoint{2.735294in}{2.735294in}}%
\pgfusepath{clip}%
\pgfsetbuttcap%
\pgfsetroundjoin%
\definecolor{currentfill}{rgb}{0.068541,0.263111,0.397982}%
\pgfsetfillcolor{currentfill}%
\pgfsetlinewidth{0.000000pt}%
\definecolor{currentstroke}{rgb}{0.000000,0.000000,0.000000}%
\pgfsetstrokecolor{currentstroke}%
\pgfsetdash{}{0pt}%
\pgfpathmoveto{\pgfqpoint{5.204720in}{1.946079in}}%
\pgfpathlineto{\pgfqpoint{5.508743in}{2.103280in}}%
\pgfpathlineto{\pgfqpoint{5.303210in}{2.116981in}}%
\pgfpathlineto{\pgfqpoint{5.204720in}{1.946079in}}%
\pgfpathclose%
\pgfusepath{fill}%
\end{pgfscope}%
\begin{pgfscope}%
\pgfpathrectangle{\pgfqpoint{3.536584in}{0.147348in}}{\pgfqpoint{2.735294in}{2.735294in}}%
\pgfusepath{clip}%
\pgfsetbuttcap%
\pgfsetroundjoin%
\definecolor{currentfill}{rgb}{0.047555,0.182548,0.276123}%
\pgfsetfillcolor{currentfill}%
\pgfsetlinewidth{0.000000pt}%
\definecolor{currentstroke}{rgb}{0.000000,0.000000,0.000000}%
\pgfsetstrokecolor{currentstroke}%
\pgfsetdash{}{0pt}%
\pgfpathmoveto{\pgfqpoint{4.714698in}{0.946905in}}%
\pgfpathlineto{\pgfqpoint{4.816678in}{1.035006in}}%
\pgfpathlineto{\pgfqpoint{4.801285in}{1.337788in}}%
\pgfpathlineto{\pgfqpoint{4.714698in}{0.946905in}}%
\pgfpathclose%
\pgfusepath{fill}%
\end{pgfscope}%
\begin{pgfscope}%
\pgfpathrectangle{\pgfqpoint{3.536584in}{0.147348in}}{\pgfqpoint{2.735294in}{2.735294in}}%
\pgfusepath{clip}%
\pgfsetbuttcap%
\pgfsetroundjoin%
\definecolor{currentfill}{rgb}{0.047555,0.182548,0.276123}%
\pgfsetfillcolor{currentfill}%
\pgfsetlinewidth{0.000000pt}%
\definecolor{currentstroke}{rgb}{0.000000,0.000000,0.000000}%
\pgfsetstrokecolor{currentstroke}%
\pgfsetdash{}{0pt}%
\pgfpathmoveto{\pgfqpoint{5.081105in}{1.337788in}}%
\pgfpathlineto{\pgfqpoint{5.065712in}{1.035006in}}%
\pgfpathlineto{\pgfqpoint{5.167692in}{0.946905in}}%
\pgfpathlineto{\pgfqpoint{5.081105in}{1.337788in}}%
\pgfpathclose%
\pgfusepath{fill}%
\end{pgfscope}%
\begin{pgfscope}%
\pgfpathrectangle{\pgfqpoint{3.536584in}{0.147348in}}{\pgfqpoint{2.735294in}{2.735294in}}%
\pgfusepath{clip}%
\pgfsetbuttcap%
\pgfsetroundjoin%
\definecolor{currentfill}{rgb}{0.046101,0.176968,0.267683}%
\pgfsetfillcolor{currentfill}%
\pgfsetlinewidth{0.000000pt}%
\definecolor{currentstroke}{rgb}{0.000000,0.000000,0.000000}%
\pgfsetstrokecolor{currentstroke}%
\pgfsetdash{}{0pt}%
\pgfpathmoveto{\pgfqpoint{4.941195in}{0.933095in}}%
\pgfpathlineto{\pgfqpoint{4.801285in}{1.337788in}}%
\pgfpathlineto{\pgfqpoint{4.816678in}{1.035006in}}%
\pgfpathlineto{\pgfqpoint{4.941195in}{0.933095in}}%
\pgfpathclose%
\pgfusepath{fill}%
\end{pgfscope}%
\begin{pgfscope}%
\pgfpathrectangle{\pgfqpoint{3.536584in}{0.147348in}}{\pgfqpoint{2.735294in}{2.735294in}}%
\pgfusepath{clip}%
\pgfsetbuttcap%
\pgfsetroundjoin%
\definecolor{currentfill}{rgb}{0.046101,0.176968,0.267683}%
\pgfsetfillcolor{currentfill}%
\pgfsetlinewidth{0.000000pt}%
\definecolor{currentstroke}{rgb}{0.000000,0.000000,0.000000}%
\pgfsetstrokecolor{currentstroke}%
\pgfsetdash{}{0pt}%
\pgfpathmoveto{\pgfqpoint{5.065712in}{1.035006in}}%
\pgfpathlineto{\pgfqpoint{5.081105in}{1.337788in}}%
\pgfpathlineto{\pgfqpoint{4.941195in}{0.933095in}}%
\pgfpathlineto{\pgfqpoint{5.065712in}{1.035006in}}%
\pgfpathclose%
\pgfusepath{fill}%
\end{pgfscope}%
\begin{pgfscope}%
\pgfpathrectangle{\pgfqpoint{3.536584in}{0.147348in}}{\pgfqpoint{2.735294in}{2.735294in}}%
\pgfusepath{clip}%
\pgfsetbuttcap%
\pgfsetroundjoin%
\definecolor{currentfill}{rgb}{0.051850,0.199036,0.301063}%
\pgfsetfillcolor{currentfill}%
\pgfsetlinewidth{0.000000pt}%
\definecolor{currentstroke}{rgb}{0.000000,0.000000,0.000000}%
\pgfsetstrokecolor{currentstroke}%
\pgfsetdash{}{0pt}%
\pgfpathmoveto{\pgfqpoint{4.374765in}{1.111775in}}%
\pgfpathlineto{\pgfqpoint{4.538549in}{1.360106in}}%
\pgfpathlineto{\pgfqpoint{4.414548in}{1.562087in}}%
\pgfpathlineto{\pgfqpoint{4.374765in}{1.111775in}}%
\pgfpathclose%
\pgfusepath{fill}%
\end{pgfscope}%
\begin{pgfscope}%
\pgfpathrectangle{\pgfqpoint{3.536584in}{0.147348in}}{\pgfqpoint{2.735294in}{2.735294in}}%
\pgfusepath{clip}%
\pgfsetbuttcap%
\pgfsetroundjoin%
\definecolor{currentfill}{rgb}{0.051850,0.199036,0.301063}%
\pgfsetfillcolor{currentfill}%
\pgfsetlinewidth{0.000000pt}%
\definecolor{currentstroke}{rgb}{0.000000,0.000000,0.000000}%
\pgfsetstrokecolor{currentstroke}%
\pgfsetdash{}{0pt}%
\pgfpathmoveto{\pgfqpoint{5.467842in}{1.562087in}}%
\pgfpathlineto{\pgfqpoint{5.343841in}{1.360106in}}%
\pgfpathlineto{\pgfqpoint{5.507624in}{1.111775in}}%
\pgfpathlineto{\pgfqpoint{5.467842in}{1.562087in}}%
\pgfpathclose%
\pgfusepath{fill}%
\end{pgfscope}%
\begin{pgfscope}%
\pgfpathrectangle{\pgfqpoint{3.536584in}{0.147348in}}{\pgfqpoint{2.735294in}{2.735294in}}%
\pgfusepath{clip}%
\pgfsetbuttcap%
\pgfsetroundjoin%
\definecolor{currentfill}{rgb}{0.064759,0.248590,0.376018}%
\pgfsetfillcolor{currentfill}%
\pgfsetlinewidth{0.000000pt}%
\definecolor{currentstroke}{rgb}{0.000000,0.000000,0.000000}%
\pgfsetstrokecolor{currentstroke}%
\pgfsetdash{}{0pt}%
\pgfpathmoveto{\pgfqpoint{5.508743in}{2.103280in}}%
\pgfpathlineto{\pgfqpoint{5.204720in}{1.946079in}}%
\pgfpathlineto{\pgfqpoint{5.467842in}{1.562087in}}%
\pgfpathlineto{\pgfqpoint{5.508743in}{2.103280in}}%
\pgfpathclose%
\pgfusepath{fill}%
\end{pgfscope}%
\begin{pgfscope}%
\pgfpathrectangle{\pgfqpoint{3.536584in}{0.147348in}}{\pgfqpoint{2.735294in}{2.735294in}}%
\pgfusepath{clip}%
\pgfsetbuttcap%
\pgfsetroundjoin%
\definecolor{currentfill}{rgb}{0.064759,0.248590,0.376018}%
\pgfsetfillcolor{currentfill}%
\pgfsetlinewidth{0.000000pt}%
\definecolor{currentstroke}{rgb}{0.000000,0.000000,0.000000}%
\pgfsetstrokecolor{currentstroke}%
\pgfsetdash{}{0pt}%
\pgfpathmoveto{\pgfqpoint{4.414548in}{1.562087in}}%
\pgfpathlineto{\pgfqpoint{4.677670in}{1.946079in}}%
\pgfpathlineto{\pgfqpoint{4.373646in}{2.103280in}}%
\pgfpathlineto{\pgfqpoint{4.414548in}{1.562087in}}%
\pgfpathclose%
\pgfusepath{fill}%
\end{pgfscope}%
\begin{pgfscope}%
\pgfpathrectangle{\pgfqpoint{3.536584in}{0.147348in}}{\pgfqpoint{2.735294in}{2.735294in}}%
\pgfusepath{clip}%
\pgfsetbuttcap%
\pgfsetroundjoin%
\definecolor{currentfill}{rgb}{0.071694,0.275212,0.416288}%
\pgfsetfillcolor{currentfill}%
\pgfsetlinewidth{0.000000pt}%
\definecolor{currentstroke}{rgb}{0.000000,0.000000,0.000000}%
\pgfsetstrokecolor{currentstroke}%
\pgfsetdash{}{0pt}%
\pgfpathmoveto{\pgfqpoint{4.677670in}{1.946079in}}%
\pgfpathlineto{\pgfqpoint{4.816407in}{2.124855in}}%
\pgfpathlineto{\pgfqpoint{4.579180in}{2.116981in}}%
\pgfpathlineto{\pgfqpoint{4.677670in}{1.946079in}}%
\pgfpathclose%
\pgfusepath{fill}%
\end{pgfscope}%
\begin{pgfscope}%
\pgfpathrectangle{\pgfqpoint{3.536584in}{0.147348in}}{\pgfqpoint{2.735294in}{2.735294in}}%
\pgfusepath{clip}%
\pgfsetbuttcap%
\pgfsetroundjoin%
\definecolor{currentfill}{rgb}{0.071694,0.275212,0.416288}%
\pgfsetfillcolor{currentfill}%
\pgfsetlinewidth{0.000000pt}%
\definecolor{currentstroke}{rgb}{0.000000,0.000000,0.000000}%
\pgfsetstrokecolor{currentstroke}%
\pgfsetdash{}{0pt}%
\pgfpathmoveto{\pgfqpoint{5.303210in}{2.116981in}}%
\pgfpathlineto{\pgfqpoint{5.065982in}{2.124855in}}%
\pgfpathlineto{\pgfqpoint{5.204720in}{1.946079in}}%
\pgfpathlineto{\pgfqpoint{5.303210in}{2.116981in}}%
\pgfpathclose%
\pgfusepath{fill}%
\end{pgfscope}%
\begin{pgfscope}%
\pgfpathrectangle{\pgfqpoint{3.536584in}{0.147348in}}{\pgfqpoint{2.735294in}{2.735294in}}%
\pgfusepath{clip}%
\pgfsetbuttcap%
\pgfsetroundjoin%
\definecolor{currentfill}{rgb}{0.071636,0.274990,0.415951}%
\pgfsetfillcolor{currentfill}%
\pgfsetlinewidth{0.000000pt}%
\definecolor{currentstroke}{rgb}{0.000000,0.000000,0.000000}%
\pgfsetstrokecolor{currentstroke}%
\pgfsetdash{}{0pt}%
\pgfpathmoveto{\pgfqpoint{4.941195in}{1.947266in}}%
\pgfpathlineto{\pgfqpoint{5.065982in}{2.124855in}}%
\pgfpathlineto{\pgfqpoint{4.816407in}{2.124855in}}%
\pgfpathlineto{\pgfqpoint{4.941195in}{1.947266in}}%
\pgfpathclose%
\pgfusepath{fill}%
\end{pgfscope}%
\begin{pgfscope}%
\pgfpathrectangle{\pgfqpoint{3.536584in}{0.147348in}}{\pgfqpoint{2.735294in}{2.735294in}}%
\pgfusepath{clip}%
\pgfsetbuttcap%
\pgfsetroundjoin%
\definecolor{currentfill}{rgb}{0.045820,0.175891,0.266053}%
\pgfsetfillcolor{currentfill}%
\pgfsetlinewidth{0.000000pt}%
\definecolor{currentstroke}{rgb}{0.000000,0.000000,0.000000}%
\pgfsetstrokecolor{currentstroke}%
\pgfsetdash{}{0pt}%
\pgfpathmoveto{\pgfqpoint{4.579939in}{1.063020in}}%
\pgfpathlineto{\pgfqpoint{4.661262in}{1.541647in}}%
\pgfpathlineto{\pgfqpoint{4.538549in}{1.360106in}}%
\pgfpathlineto{\pgfqpoint{4.579939in}{1.063020in}}%
\pgfpathclose%
\pgfusepath{fill}%
\end{pgfscope}%
\begin{pgfscope}%
\pgfpathrectangle{\pgfqpoint{3.536584in}{0.147348in}}{\pgfqpoint{2.735294in}{2.735294in}}%
\pgfusepath{clip}%
\pgfsetbuttcap%
\pgfsetroundjoin%
\definecolor{currentfill}{rgb}{0.045820,0.175891,0.266053}%
\pgfsetfillcolor{currentfill}%
\pgfsetlinewidth{0.000000pt}%
\definecolor{currentstroke}{rgb}{0.000000,0.000000,0.000000}%
\pgfsetstrokecolor{currentstroke}%
\pgfsetdash{}{0pt}%
\pgfpathmoveto{\pgfqpoint{5.343841in}{1.360106in}}%
\pgfpathlineto{\pgfqpoint{5.221128in}{1.541647in}}%
\pgfpathlineto{\pgfqpoint{5.302451in}{1.063020in}}%
\pgfpathlineto{\pgfqpoint{5.343841in}{1.360106in}}%
\pgfpathclose%
\pgfusepath{fill}%
\end{pgfscope}%
\begin{pgfscope}%
\pgfpathrectangle{\pgfqpoint{3.536584in}{0.147348in}}{\pgfqpoint{2.735294in}{2.735294in}}%
\pgfusepath{clip}%
\pgfsetbuttcap%
\pgfsetroundjoin%
\definecolor{currentfill}{rgb}{0.046814,0.179706,0.271825}%
\pgfsetfillcolor{currentfill}%
\pgfsetlinewidth{0.000000pt}%
\definecolor{currentstroke}{rgb}{0.000000,0.000000,0.000000}%
\pgfsetstrokecolor{currentstroke}%
\pgfsetdash{}{0pt}%
\pgfpathmoveto{\pgfqpoint{4.941195in}{1.533948in}}%
\pgfpathlineto{\pgfqpoint{4.941195in}{0.933095in}}%
\pgfpathlineto{\pgfqpoint{5.081105in}{1.337788in}}%
\pgfpathlineto{\pgfqpoint{4.941195in}{1.533948in}}%
\pgfpathclose%
\pgfusepath{fill}%
\end{pgfscope}%
\begin{pgfscope}%
\pgfpathrectangle{\pgfqpoint{3.536584in}{0.147348in}}{\pgfqpoint{2.735294in}{2.735294in}}%
\pgfusepath{clip}%
\pgfsetbuttcap%
\pgfsetroundjoin%
\definecolor{currentfill}{rgb}{0.046814,0.179706,0.271825}%
\pgfsetfillcolor{currentfill}%
\pgfsetlinewidth{0.000000pt}%
\definecolor{currentstroke}{rgb}{0.000000,0.000000,0.000000}%
\pgfsetstrokecolor{currentstroke}%
\pgfsetdash{}{0pt}%
\pgfpathmoveto{\pgfqpoint{4.801285in}{1.337788in}}%
\pgfpathlineto{\pgfqpoint{4.941195in}{0.933095in}}%
\pgfpathlineto{\pgfqpoint{4.941195in}{1.533948in}}%
\pgfpathlineto{\pgfqpoint{4.801285in}{1.337788in}}%
\pgfpathclose%
\pgfusepath{fill}%
\end{pgfscope}%
\begin{pgfscope}%
\pgfpathrectangle{\pgfqpoint{3.536584in}{0.147348in}}{\pgfqpoint{2.735294in}{2.735294in}}%
\pgfusepath{clip}%
\pgfsetbuttcap%
\pgfsetroundjoin%
\definecolor{currentfill}{rgb}{0.069261,0.265872,0.402159}%
\pgfsetfillcolor{currentfill}%
\pgfsetlinewidth{0.000000pt}%
\definecolor{currentstroke}{rgb}{0.000000,0.000000,0.000000}%
\pgfsetstrokecolor{currentstroke}%
\pgfsetdash{}{0pt}%
\pgfpathmoveto{\pgfqpoint{5.204720in}{1.946079in}}%
\pgfpathlineto{\pgfqpoint{5.065982in}{2.124855in}}%
\pgfpathlineto{\pgfqpoint{4.941195in}{1.947266in}}%
\pgfpathlineto{\pgfqpoint{5.204720in}{1.946079in}}%
\pgfpathclose%
\pgfusepath{fill}%
\end{pgfscope}%
\begin{pgfscope}%
\pgfpathrectangle{\pgfqpoint{3.536584in}{0.147348in}}{\pgfqpoint{2.735294in}{2.735294in}}%
\pgfusepath{clip}%
\pgfsetbuttcap%
\pgfsetroundjoin%
\definecolor{currentfill}{rgb}{0.069261,0.265872,0.402159}%
\pgfsetfillcolor{currentfill}%
\pgfsetlinewidth{0.000000pt}%
\definecolor{currentstroke}{rgb}{0.000000,0.000000,0.000000}%
\pgfsetstrokecolor{currentstroke}%
\pgfsetdash{}{0pt}%
\pgfpathmoveto{\pgfqpoint{4.941195in}{1.947266in}}%
\pgfpathlineto{\pgfqpoint{4.816407in}{2.124855in}}%
\pgfpathlineto{\pgfqpoint{4.677670in}{1.946079in}}%
\pgfpathlineto{\pgfqpoint{4.941195in}{1.947266in}}%
\pgfpathclose%
\pgfusepath{fill}%
\end{pgfscope}%
\begin{pgfscope}%
\pgfpathrectangle{\pgfqpoint{3.536584in}{0.147348in}}{\pgfqpoint{2.735294in}{2.735294in}}%
\pgfusepath{clip}%
\pgfsetbuttcap%
\pgfsetroundjoin%
\definecolor{currentfill}{rgb}{0.049465,0.189883,0.287218}%
\pgfsetfillcolor{currentfill}%
\pgfsetlinewidth{0.000000pt}%
\definecolor{currentstroke}{rgb}{0.000000,0.000000,0.000000}%
\pgfsetstrokecolor{currentstroke}%
\pgfsetdash{}{0pt}%
\pgfpathmoveto{\pgfqpoint{4.579939in}{1.063020in}}%
\pgfpathlineto{\pgfqpoint{4.801285in}{1.337788in}}%
\pgfpathlineto{\pgfqpoint{4.661262in}{1.541647in}}%
\pgfpathlineto{\pgfqpoint{4.579939in}{1.063020in}}%
\pgfpathclose%
\pgfusepath{fill}%
\end{pgfscope}%
\begin{pgfscope}%
\pgfpathrectangle{\pgfqpoint{3.536584in}{0.147348in}}{\pgfqpoint{2.735294in}{2.735294in}}%
\pgfusepath{clip}%
\pgfsetbuttcap%
\pgfsetroundjoin%
\definecolor{currentfill}{rgb}{0.049465,0.189883,0.287218}%
\pgfsetfillcolor{currentfill}%
\pgfsetlinewidth{0.000000pt}%
\definecolor{currentstroke}{rgb}{0.000000,0.000000,0.000000}%
\pgfsetstrokecolor{currentstroke}%
\pgfsetdash{}{0pt}%
\pgfpathmoveto{\pgfqpoint{5.221128in}{1.541647in}}%
\pgfpathlineto{\pgfqpoint{5.081105in}{1.337788in}}%
\pgfpathlineto{\pgfqpoint{5.302451in}{1.063020in}}%
\pgfpathlineto{\pgfqpoint{5.221128in}{1.541647in}}%
\pgfpathclose%
\pgfusepath{fill}%
\end{pgfscope}%
\begin{pgfscope}%
\pgfpathrectangle{\pgfqpoint{3.536584in}{0.147348in}}{\pgfqpoint{2.735294in}{2.735294in}}%
\pgfusepath{clip}%
\pgfsetbuttcap%
\pgfsetroundjoin%
\definecolor{currentfill}{rgb}{0.061576,0.236373,0.357539}%
\pgfsetfillcolor{currentfill}%
\pgfsetlinewidth{0.000000pt}%
\definecolor{currentstroke}{rgb}{0.000000,0.000000,0.000000}%
\pgfsetstrokecolor{currentstroke}%
\pgfsetdash{}{0pt}%
\pgfpathmoveto{\pgfqpoint{4.661262in}{1.541647in}}%
\pgfpathlineto{\pgfqpoint{4.677670in}{1.946079in}}%
\pgfpathlineto{\pgfqpoint{4.414548in}{1.562087in}}%
\pgfpathlineto{\pgfqpoint{4.661262in}{1.541647in}}%
\pgfpathclose%
\pgfusepath{fill}%
\end{pgfscope}%
\begin{pgfscope}%
\pgfpathrectangle{\pgfqpoint{3.536584in}{0.147348in}}{\pgfqpoint{2.735294in}{2.735294in}}%
\pgfusepath{clip}%
\pgfsetbuttcap%
\pgfsetroundjoin%
\definecolor{currentfill}{rgb}{0.061576,0.236373,0.357539}%
\pgfsetfillcolor{currentfill}%
\pgfsetlinewidth{0.000000pt}%
\definecolor{currentstroke}{rgb}{0.000000,0.000000,0.000000}%
\pgfsetstrokecolor{currentstroke}%
\pgfsetdash{}{0pt}%
\pgfpathmoveto{\pgfqpoint{5.467842in}{1.562087in}}%
\pgfpathlineto{\pgfqpoint{5.204720in}{1.946079in}}%
\pgfpathlineto{\pgfqpoint{5.221128in}{1.541647in}}%
\pgfpathlineto{\pgfqpoint{5.467842in}{1.562087in}}%
\pgfpathclose%
\pgfusepath{fill}%
\end{pgfscope}%
\begin{pgfscope}%
\pgfpathrectangle{\pgfqpoint{3.536584in}{0.147348in}}{\pgfqpoint{2.735294in}{2.735294in}}%
\pgfusepath{clip}%
\pgfsetbuttcap%
\pgfsetroundjoin%
\definecolor{currentfill}{rgb}{0.053541,0.205528,0.310883}%
\pgfsetfillcolor{currentfill}%
\pgfsetlinewidth{0.000000pt}%
\definecolor{currentstroke}{rgb}{0.000000,0.000000,0.000000}%
\pgfsetstrokecolor{currentstroke}%
\pgfsetdash{}{0pt}%
\pgfpathmoveto{\pgfqpoint{4.414548in}{1.562087in}}%
\pgfpathlineto{\pgfqpoint{4.538549in}{1.360106in}}%
\pgfpathlineto{\pgfqpoint{4.661262in}{1.541647in}}%
\pgfpathlineto{\pgfqpoint{4.414548in}{1.562087in}}%
\pgfpathclose%
\pgfusepath{fill}%
\end{pgfscope}%
\begin{pgfscope}%
\pgfpathrectangle{\pgfqpoint{3.536584in}{0.147348in}}{\pgfqpoint{2.735294in}{2.735294in}}%
\pgfusepath{clip}%
\pgfsetbuttcap%
\pgfsetroundjoin%
\definecolor{currentfill}{rgb}{0.053541,0.205528,0.310883}%
\pgfsetfillcolor{currentfill}%
\pgfsetlinewidth{0.000000pt}%
\definecolor{currentstroke}{rgb}{0.000000,0.000000,0.000000}%
\pgfsetstrokecolor{currentstroke}%
\pgfsetdash{}{0pt}%
\pgfpathmoveto{\pgfqpoint{5.221128in}{1.541647in}}%
\pgfpathlineto{\pgfqpoint{5.343841in}{1.360106in}}%
\pgfpathlineto{\pgfqpoint{5.467842in}{1.562087in}}%
\pgfpathlineto{\pgfqpoint{5.221128in}{1.541647in}}%
\pgfpathclose%
\pgfusepath{fill}%
\end{pgfscope}%
\begin{pgfscope}%
\pgfpathrectangle{\pgfqpoint{3.536584in}{0.147348in}}{\pgfqpoint{2.735294in}{2.735294in}}%
\pgfusepath{clip}%
\pgfsetbuttcap%
\pgfsetroundjoin%
\definecolor{currentfill}{rgb}{0.060634,0.232757,0.352069}%
\pgfsetfillcolor{currentfill}%
\pgfsetlinewidth{0.000000pt}%
\definecolor{currentstroke}{rgb}{0.000000,0.000000,0.000000}%
\pgfsetstrokecolor{currentstroke}%
\pgfsetdash{}{0pt}%
\pgfpathmoveto{\pgfqpoint{4.677670in}{1.946079in}}%
\pgfpathlineto{\pgfqpoint{4.661262in}{1.541647in}}%
\pgfpathlineto{\pgfqpoint{4.941195in}{1.947266in}}%
\pgfpathlineto{\pgfqpoint{4.677670in}{1.946079in}}%
\pgfpathclose%
\pgfusepath{fill}%
\end{pgfscope}%
\begin{pgfscope}%
\pgfpathrectangle{\pgfqpoint{3.536584in}{0.147348in}}{\pgfqpoint{2.735294in}{2.735294in}}%
\pgfusepath{clip}%
\pgfsetbuttcap%
\pgfsetroundjoin%
\definecolor{currentfill}{rgb}{0.060634,0.232757,0.352069}%
\pgfsetfillcolor{currentfill}%
\pgfsetlinewidth{0.000000pt}%
\definecolor{currentstroke}{rgb}{0.000000,0.000000,0.000000}%
\pgfsetstrokecolor{currentstroke}%
\pgfsetdash{}{0pt}%
\pgfpathmoveto{\pgfqpoint{4.941195in}{1.947266in}}%
\pgfpathlineto{\pgfqpoint{5.221128in}{1.541647in}}%
\pgfpathlineto{\pgfqpoint{5.204720in}{1.946079in}}%
\pgfpathlineto{\pgfqpoint{4.941195in}{1.947266in}}%
\pgfpathclose%
\pgfusepath{fill}%
\end{pgfscope}%
\begin{pgfscope}%
\pgfpathrectangle{\pgfqpoint{3.536584in}{0.147348in}}{\pgfqpoint{2.735294in}{2.735294in}}%
\pgfusepath{clip}%
\pgfsetbuttcap%
\pgfsetroundjoin%
\definecolor{currentfill}{rgb}{0.060773,0.233289,0.352874}%
\pgfsetfillcolor{currentfill}%
\pgfsetlinewidth{0.000000pt}%
\definecolor{currentstroke}{rgb}{0.000000,0.000000,0.000000}%
\pgfsetstrokecolor{currentstroke}%
\pgfsetdash{}{0pt}%
\pgfpathmoveto{\pgfqpoint{4.941195in}{1.533948in}}%
\pgfpathlineto{\pgfqpoint{4.941195in}{1.947266in}}%
\pgfpathlineto{\pgfqpoint{4.661262in}{1.541647in}}%
\pgfpathlineto{\pgfqpoint{4.941195in}{1.533948in}}%
\pgfpathclose%
\pgfusepath{fill}%
\end{pgfscope}%
\begin{pgfscope}%
\pgfpathrectangle{\pgfqpoint{3.536584in}{0.147348in}}{\pgfqpoint{2.735294in}{2.735294in}}%
\pgfusepath{clip}%
\pgfsetbuttcap%
\pgfsetroundjoin%
\definecolor{currentfill}{rgb}{0.060773,0.233289,0.352874}%
\pgfsetfillcolor{currentfill}%
\pgfsetlinewidth{0.000000pt}%
\definecolor{currentstroke}{rgb}{0.000000,0.000000,0.000000}%
\pgfsetstrokecolor{currentstroke}%
\pgfsetdash{}{0pt}%
\pgfpathmoveto{\pgfqpoint{5.221128in}{1.541647in}}%
\pgfpathlineto{\pgfqpoint{4.941195in}{1.947266in}}%
\pgfpathlineto{\pgfqpoint{4.941195in}{1.533948in}}%
\pgfpathlineto{\pgfqpoint{5.221128in}{1.541647in}}%
\pgfpathclose%
\pgfusepath{fill}%
\end{pgfscope}%
\begin{pgfscope}%
\pgfpathrectangle{\pgfqpoint{3.536584in}{0.147348in}}{\pgfqpoint{2.735294in}{2.735294in}}%
\pgfusepath{clip}%
\pgfsetbuttcap%
\pgfsetroundjoin%
\definecolor{currentfill}{rgb}{0.052607,0.201942,0.305459}%
\pgfsetfillcolor{currentfill}%
\pgfsetlinewidth{0.000000pt}%
\definecolor{currentstroke}{rgb}{0.000000,0.000000,0.000000}%
\pgfsetstrokecolor{currentstroke}%
\pgfsetdash{}{0pt}%
\pgfpathmoveto{\pgfqpoint{4.661262in}{1.541647in}}%
\pgfpathlineto{\pgfqpoint{4.801285in}{1.337788in}}%
\pgfpathlineto{\pgfqpoint{4.941195in}{1.533948in}}%
\pgfpathlineto{\pgfqpoint{4.661262in}{1.541647in}}%
\pgfpathclose%
\pgfusepath{fill}%
\end{pgfscope}%
\begin{pgfscope}%
\pgfpathrectangle{\pgfqpoint{3.536584in}{0.147348in}}{\pgfqpoint{2.735294in}{2.735294in}}%
\pgfusepath{clip}%
\pgfsetbuttcap%
\pgfsetroundjoin%
\definecolor{currentfill}{rgb}{0.052607,0.201942,0.305459}%
\pgfsetfillcolor{currentfill}%
\pgfsetlinewidth{0.000000pt}%
\definecolor{currentstroke}{rgb}{0.000000,0.000000,0.000000}%
\pgfsetstrokecolor{currentstroke}%
\pgfsetdash{}{0pt}%
\pgfpathmoveto{\pgfqpoint{4.941195in}{1.533948in}}%
\pgfpathlineto{\pgfqpoint{5.081105in}{1.337788in}}%
\pgfpathlineto{\pgfqpoint{5.221128in}{1.541647in}}%
\pgfpathlineto{\pgfqpoint{4.941195in}{1.533948in}}%
\pgfpathclose%
\pgfusepath{fill}%
\end{pgfscope}%
\begin{pgfscope}%
\pgfpathrectangle{\pgfqpoint{3.536584in}{0.147348in}}{\pgfqpoint{2.735294in}{2.735294in}}%
\pgfusepath{clip}%
\pgfsetbuttcap%
\pgfsetroundjoin%
\definecolor{currentfill}{rgb}{0.839216,0.152941,0.156863}%
\pgfsetfillcolor{currentfill}%
\pgfsetfillopacity{0.300000}%
\pgfsetlinewidth{1.003750pt}%
\definecolor{currentstroke}{rgb}{0.839216,0.152941,0.156863}%
\pgfsetstrokecolor{currentstroke}%
\pgfsetstrokeopacity{0.300000}%
\pgfsetdash{}{0pt}%
\pgfpathmoveto{\pgfqpoint{4.347722in}{0.944929in}}%
\pgfpathcurveto{\pgfqpoint{4.357810in}{0.944929in}}{\pgfqpoint{4.367485in}{0.948937in}}{\pgfqpoint{4.374618in}{0.956070in}}%
\pgfpathcurveto{\pgfqpoint{4.381751in}{0.963203in}}{\pgfqpoint{4.385759in}{0.972878in}}{\pgfqpoint{4.385759in}{0.982966in}}%
\pgfpathcurveto{\pgfqpoint{4.385759in}{0.993053in}}{\pgfqpoint{4.381751in}{1.002729in}}{\pgfqpoint{4.374618in}{1.009861in}}%
\pgfpathcurveto{\pgfqpoint{4.367485in}{1.016994in}}{\pgfqpoint{4.357810in}{1.021002in}}{\pgfqpoint{4.347722in}{1.021002in}}%
\pgfpathcurveto{\pgfqpoint{4.337635in}{1.021002in}}{\pgfqpoint{4.327959in}{1.016994in}}{\pgfqpoint{4.320827in}{1.009861in}}%
\pgfpathcurveto{\pgfqpoint{4.313694in}{1.002729in}}{\pgfqpoint{4.309686in}{0.993053in}}{\pgfqpoint{4.309686in}{0.982966in}}%
\pgfpathcurveto{\pgfqpoint{4.309686in}{0.972878in}}{\pgfqpoint{4.313694in}{0.963203in}}{\pgfqpoint{4.320827in}{0.956070in}}%
\pgfpathcurveto{\pgfqpoint{4.327959in}{0.948937in}}{\pgfqpoint{4.337635in}{0.944929in}}{\pgfqpoint{4.347722in}{0.944929in}}%
\pgfpathlineto{\pgfqpoint{4.347722in}{0.944929in}}%
\pgfpathclose%
\pgfusepath{stroke,fill}%
\end{pgfscope}%
\begin{pgfscope}%
\pgfpathrectangle{\pgfqpoint{3.536584in}{0.147348in}}{\pgfqpoint{2.735294in}{2.735294in}}%
\pgfusepath{clip}%
\pgfsetbuttcap%
\pgfsetroundjoin%
\definecolor{currentfill}{rgb}{0.839216,0.152941,0.156863}%
\pgfsetfillcolor{currentfill}%
\pgfsetfillopacity{0.383610}%
\pgfsetlinewidth{1.003750pt}%
\definecolor{currentstroke}{rgb}{0.839216,0.152941,0.156863}%
\pgfsetstrokecolor{currentstroke}%
\pgfsetstrokeopacity{0.383610}%
\pgfsetdash{}{0pt}%
\pgfpathmoveto{\pgfqpoint{4.327859in}{1.009596in}}%
\pgfpathcurveto{\pgfqpoint{4.337946in}{1.009596in}}{\pgfqpoint{4.347622in}{1.013604in}}{\pgfqpoint{4.354755in}{1.020737in}}%
\pgfpathcurveto{\pgfqpoint{4.361888in}{1.027870in}}{\pgfqpoint{4.365895in}{1.037545in}}{\pgfqpoint{4.365895in}{1.047633in}}%
\pgfpathcurveto{\pgfqpoint{4.365895in}{1.057720in}}{\pgfqpoint{4.361888in}{1.067396in}}{\pgfqpoint{4.354755in}{1.074528in}}%
\pgfpathcurveto{\pgfqpoint{4.347622in}{1.081661in}}{\pgfqpoint{4.337946in}{1.085669in}}{\pgfqpoint{4.327859in}{1.085669in}}%
\pgfpathcurveto{\pgfqpoint{4.317772in}{1.085669in}}{\pgfqpoint{4.308096in}{1.081661in}}{\pgfqpoint{4.300963in}{1.074528in}}%
\pgfpathcurveto{\pgfqpoint{4.293831in}{1.067396in}}{\pgfqpoint{4.289823in}{1.057720in}}{\pgfqpoint{4.289823in}{1.047633in}}%
\pgfpathcurveto{\pgfqpoint{4.289823in}{1.037545in}}{\pgfqpoint{4.293831in}{1.027870in}}{\pgfqpoint{4.300963in}{1.020737in}}%
\pgfpathcurveto{\pgfqpoint{4.308096in}{1.013604in}}{\pgfqpoint{4.317772in}{1.009596in}}{\pgfqpoint{4.327859in}{1.009596in}}%
\pgfpathlineto{\pgfqpoint{4.327859in}{1.009596in}}%
\pgfpathclose%
\pgfusepath{stroke,fill}%
\end{pgfscope}%
\begin{pgfscope}%
\pgfpathrectangle{\pgfqpoint{3.536584in}{0.147348in}}{\pgfqpoint{2.735294in}{2.735294in}}%
\pgfusepath{clip}%
\pgfsetbuttcap%
\pgfsetroundjoin%
\definecolor{currentfill}{rgb}{0.839216,0.152941,0.156863}%
\pgfsetfillcolor{currentfill}%
\pgfsetfillopacity{0.457533}%
\pgfsetlinewidth{1.003750pt}%
\definecolor{currentstroke}{rgb}{0.839216,0.152941,0.156863}%
\pgfsetstrokecolor{currentstroke}%
\pgfsetstrokeopacity{0.457533}%
\pgfsetdash{}{0pt}%
\pgfpathmoveto{\pgfqpoint{4.483875in}{0.962534in}}%
\pgfpathcurveto{\pgfqpoint{4.493962in}{0.962534in}}{\pgfqpoint{4.503637in}{0.966542in}}{\pgfqpoint{4.510770in}{0.973675in}}%
\pgfpathcurveto{\pgfqpoint{4.517903in}{0.980808in}}{\pgfqpoint{4.521911in}{0.990483in}}{\pgfqpoint{4.521911in}{1.000570in}}%
\pgfpathcurveto{\pgfqpoint{4.521911in}{1.010658in}}{\pgfqpoint{4.517903in}{1.020333in}}{\pgfqpoint{4.510770in}{1.027466in}}%
\pgfpathcurveto{\pgfqpoint{4.503637in}{1.034599in}}{\pgfqpoint{4.493962in}{1.038607in}}{\pgfqpoint{4.483875in}{1.038607in}}%
\pgfpathcurveto{\pgfqpoint{4.473787in}{1.038607in}}{\pgfqpoint{4.464112in}{1.034599in}}{\pgfqpoint{4.456979in}{1.027466in}}%
\pgfpathcurveto{\pgfqpoint{4.449846in}{1.020333in}}{\pgfqpoint{4.445838in}{1.010658in}}{\pgfqpoint{4.445838in}{1.000570in}}%
\pgfpathcurveto{\pgfqpoint{4.445838in}{0.990483in}}{\pgfqpoint{4.449846in}{0.980808in}}{\pgfqpoint{4.456979in}{0.973675in}}%
\pgfpathcurveto{\pgfqpoint{4.464112in}{0.966542in}}{\pgfqpoint{4.473787in}{0.962534in}}{\pgfqpoint{4.483875in}{0.962534in}}%
\pgfpathlineto{\pgfqpoint{4.483875in}{0.962534in}}%
\pgfpathclose%
\pgfusepath{stroke,fill}%
\end{pgfscope}%
\begin{pgfscope}%
\pgfpathrectangle{\pgfqpoint{3.536584in}{0.147348in}}{\pgfqpoint{2.735294in}{2.735294in}}%
\pgfusepath{clip}%
\pgfsetbuttcap%
\pgfsetroundjoin%
\definecolor{currentfill}{rgb}{0.839216,0.152941,0.156863}%
\pgfsetfillcolor{currentfill}%
\pgfsetfillopacity{0.492303}%
\pgfsetlinewidth{1.003750pt}%
\definecolor{currentstroke}{rgb}{0.839216,0.152941,0.156863}%
\pgfsetstrokecolor{currentstroke}%
\pgfsetstrokeopacity{0.492303}%
\pgfsetdash{}{0pt}%
\pgfpathmoveto{\pgfqpoint{4.160550in}{1.511653in}}%
\pgfpathcurveto{\pgfqpoint{4.170638in}{1.511653in}}{\pgfqpoint{4.180313in}{1.515661in}}{\pgfqpoint{4.187446in}{1.522794in}}%
\pgfpathcurveto{\pgfqpoint{4.194579in}{1.529927in}}{\pgfqpoint{4.198587in}{1.539602in}}{\pgfqpoint{4.198587in}{1.549689in}}%
\pgfpathcurveto{\pgfqpoint{4.198587in}{1.559777in}}{\pgfqpoint{4.194579in}{1.569452in}}{\pgfqpoint{4.187446in}{1.576585in}}%
\pgfpathcurveto{\pgfqpoint{4.180313in}{1.583718in}}{\pgfqpoint{4.170638in}{1.587726in}}{\pgfqpoint{4.160550in}{1.587726in}}%
\pgfpathcurveto{\pgfqpoint{4.150463in}{1.587726in}}{\pgfqpoint{4.140788in}{1.583718in}}{\pgfqpoint{4.133655in}{1.576585in}}%
\pgfpathcurveto{\pgfqpoint{4.126522in}{1.569452in}}{\pgfqpoint{4.122514in}{1.559777in}}{\pgfqpoint{4.122514in}{1.549689in}}%
\pgfpathcurveto{\pgfqpoint{4.122514in}{1.539602in}}{\pgfqpoint{4.126522in}{1.529927in}}{\pgfqpoint{4.133655in}{1.522794in}}%
\pgfpathcurveto{\pgfqpoint{4.140788in}{1.515661in}}{\pgfqpoint{4.150463in}{1.511653in}}{\pgfqpoint{4.160550in}{1.511653in}}%
\pgfpathlineto{\pgfqpoint{4.160550in}{1.511653in}}%
\pgfpathclose%
\pgfusepath{stroke,fill}%
\end{pgfscope}%
\begin{pgfscope}%
\pgfpathrectangle{\pgfqpoint{3.536584in}{0.147348in}}{\pgfqpoint{2.735294in}{2.735294in}}%
\pgfusepath{clip}%
\pgfsetbuttcap%
\pgfsetroundjoin%
\definecolor{currentfill}{rgb}{0.839216,0.152941,0.156863}%
\pgfsetfillcolor{currentfill}%
\pgfsetfillopacity{0.498590}%
\pgfsetlinewidth{1.003750pt}%
\definecolor{currentstroke}{rgb}{0.839216,0.152941,0.156863}%
\pgfsetstrokecolor{currentstroke}%
\pgfsetstrokeopacity{0.498590}%
\pgfsetdash{}{0pt}%
\pgfpathmoveto{\pgfqpoint{5.807826in}{1.621636in}}%
\pgfpathcurveto{\pgfqpoint{5.817913in}{1.621636in}}{\pgfqpoint{5.827588in}{1.625643in}}{\pgfqpoint{5.834721in}{1.632776in}}%
\pgfpathcurveto{\pgfqpoint{5.841854in}{1.639909in}}{\pgfqpoint{5.845862in}{1.649585in}}{\pgfqpoint{5.845862in}{1.659672in}}%
\pgfpathcurveto{\pgfqpoint{5.845862in}{1.669759in}}{\pgfqpoint{5.841854in}{1.679435in}}{\pgfqpoint{5.834721in}{1.686568in}}%
\pgfpathcurveto{\pgfqpoint{5.827588in}{1.693701in}}{\pgfqpoint{5.817913in}{1.697708in}}{\pgfqpoint{5.807826in}{1.697708in}}%
\pgfpathcurveto{\pgfqpoint{5.797738in}{1.697708in}}{\pgfqpoint{5.788063in}{1.693701in}}{\pgfqpoint{5.780930in}{1.686568in}}%
\pgfpathcurveto{\pgfqpoint{5.773797in}{1.679435in}}{\pgfqpoint{5.769789in}{1.669759in}}{\pgfqpoint{5.769789in}{1.659672in}}%
\pgfpathcurveto{\pgfqpoint{5.769789in}{1.649585in}}{\pgfqpoint{5.773797in}{1.639909in}}{\pgfqpoint{5.780930in}{1.632776in}}%
\pgfpathcurveto{\pgfqpoint{5.788063in}{1.625643in}}{\pgfqpoint{5.797738in}{1.621636in}}{\pgfqpoint{5.807826in}{1.621636in}}%
\pgfpathlineto{\pgfqpoint{5.807826in}{1.621636in}}%
\pgfpathclose%
\pgfusepath{stroke,fill}%
\end{pgfscope}%
\begin{pgfscope}%
\pgfpathrectangle{\pgfqpoint{3.536584in}{0.147348in}}{\pgfqpoint{2.735294in}{2.735294in}}%
\pgfusepath{clip}%
\pgfsetbuttcap%
\pgfsetroundjoin%
\definecolor{currentfill}{rgb}{0.839216,0.152941,0.156863}%
\pgfsetfillcolor{currentfill}%
\pgfsetfillopacity{0.612876}%
\pgfsetlinewidth{1.003750pt}%
\definecolor{currentstroke}{rgb}{0.839216,0.152941,0.156863}%
\pgfsetstrokecolor{currentstroke}%
\pgfsetstrokeopacity{0.612876}%
\pgfsetdash{}{0pt}%
\pgfpathmoveto{\pgfqpoint{4.496123in}{0.979547in}}%
\pgfpathcurveto{\pgfqpoint{4.506210in}{0.979547in}}{\pgfqpoint{4.515886in}{0.983554in}}{\pgfqpoint{4.523019in}{0.990687in}}%
\pgfpathcurveto{\pgfqpoint{4.530151in}{0.997820in}}{\pgfqpoint{4.534159in}{1.007495in}}{\pgfqpoint{4.534159in}{1.017583in}}%
\pgfpathcurveto{\pgfqpoint{4.534159in}{1.027670in}}{\pgfqpoint{4.530151in}{1.037346in}}{\pgfqpoint{4.523019in}{1.044479in}}%
\pgfpathcurveto{\pgfqpoint{4.515886in}{1.051611in}}{\pgfqpoint{4.506210in}{1.055619in}}{\pgfqpoint{4.496123in}{1.055619in}}%
\pgfpathcurveto{\pgfqpoint{4.486036in}{1.055619in}}{\pgfqpoint{4.476360in}{1.051611in}}{\pgfqpoint{4.469227in}{1.044479in}}%
\pgfpathcurveto{\pgfqpoint{4.462094in}{1.037346in}}{\pgfqpoint{4.458087in}{1.027670in}}{\pgfqpoint{4.458087in}{1.017583in}}%
\pgfpathcurveto{\pgfqpoint{4.458087in}{1.007495in}}{\pgfqpoint{4.462094in}{0.997820in}}{\pgfqpoint{4.469227in}{0.990687in}}%
\pgfpathcurveto{\pgfqpoint{4.476360in}{0.983554in}}{\pgfqpoint{4.486036in}{0.979547in}}{\pgfqpoint{4.496123in}{0.979547in}}%
\pgfpathlineto{\pgfqpoint{4.496123in}{0.979547in}}%
\pgfpathclose%
\pgfusepath{stroke,fill}%
\end{pgfscope}%
\begin{pgfscope}%
\pgfpathrectangle{\pgfqpoint{3.536584in}{0.147348in}}{\pgfqpoint{2.735294in}{2.735294in}}%
\pgfusepath{clip}%
\pgfsetbuttcap%
\pgfsetroundjoin%
\definecolor{currentfill}{rgb}{0.839216,0.152941,0.156863}%
\pgfsetfillcolor{currentfill}%
\pgfsetfillopacity{0.625674}%
\pgfsetlinewidth{1.003750pt}%
\definecolor{currentstroke}{rgb}{0.839216,0.152941,0.156863}%
\pgfsetstrokecolor{currentstroke}%
\pgfsetstrokeopacity{0.625674}%
\pgfsetdash{}{0pt}%
\pgfpathmoveto{\pgfqpoint{5.375023in}{2.164239in}}%
\pgfpathcurveto{\pgfqpoint{5.385111in}{2.164239in}}{\pgfqpoint{5.394786in}{2.168247in}}{\pgfqpoint{5.401919in}{2.175380in}}%
\pgfpathcurveto{\pgfqpoint{5.409052in}{2.182513in}}{\pgfqpoint{5.413059in}{2.192188in}}{\pgfqpoint{5.413059in}{2.202275in}}%
\pgfpathcurveto{\pgfqpoint{5.413059in}{2.212363in}}{\pgfqpoint{5.409052in}{2.222038in}}{\pgfqpoint{5.401919in}{2.229171in}}%
\pgfpathcurveto{\pgfqpoint{5.394786in}{2.236304in}}{\pgfqpoint{5.385111in}{2.240312in}}{\pgfqpoint{5.375023in}{2.240312in}}%
\pgfpathcurveto{\pgfqpoint{5.364936in}{2.240312in}}{\pgfqpoint{5.355260in}{2.236304in}}{\pgfqpoint{5.348127in}{2.229171in}}%
\pgfpathcurveto{\pgfqpoint{5.340995in}{2.222038in}}{\pgfqpoint{5.336987in}{2.212363in}}{\pgfqpoint{5.336987in}{2.202275in}}%
\pgfpathcurveto{\pgfqpoint{5.336987in}{2.192188in}}{\pgfqpoint{5.340995in}{2.182513in}}{\pgfqpoint{5.348127in}{2.175380in}}%
\pgfpathcurveto{\pgfqpoint{5.355260in}{2.168247in}}{\pgfqpoint{5.364936in}{2.164239in}}{\pgfqpoint{5.375023in}{2.164239in}}%
\pgfpathlineto{\pgfqpoint{5.375023in}{2.164239in}}%
\pgfpathclose%
\pgfusepath{stroke,fill}%
\end{pgfscope}%
\begin{pgfscope}%
\pgfpathrectangle{\pgfqpoint{3.536584in}{0.147348in}}{\pgfqpoint{2.735294in}{2.735294in}}%
\pgfusepath{clip}%
\pgfsetbuttcap%
\pgfsetroundjoin%
\definecolor{currentfill}{rgb}{0.839216,0.152941,0.156863}%
\pgfsetfillcolor{currentfill}%
\pgfsetfillopacity{0.631635}%
\pgfsetlinewidth{1.003750pt}%
\definecolor{currentstroke}{rgb}{0.839216,0.152941,0.156863}%
\pgfsetstrokecolor{currentstroke}%
\pgfsetstrokeopacity{0.631635}%
\pgfsetdash{}{0pt}%
\pgfpathmoveto{\pgfqpoint{5.366125in}{2.104757in}}%
\pgfpathcurveto{\pgfqpoint{5.376212in}{2.104757in}}{\pgfqpoint{5.385888in}{2.108765in}}{\pgfqpoint{5.393021in}{2.115898in}}%
\pgfpathcurveto{\pgfqpoint{5.400154in}{2.123030in}}{\pgfqpoint{5.404161in}{2.132706in}}{\pgfqpoint{5.404161in}{2.142793in}}%
\pgfpathcurveto{\pgfqpoint{5.404161in}{2.152881in}}{\pgfqpoint{5.400154in}{2.162556in}}{\pgfqpoint{5.393021in}{2.169689in}}%
\pgfpathcurveto{\pgfqpoint{5.385888in}{2.176822in}}{\pgfqpoint{5.376212in}{2.180830in}}{\pgfqpoint{5.366125in}{2.180830in}}%
\pgfpathcurveto{\pgfqpoint{5.356038in}{2.180830in}}{\pgfqpoint{5.346362in}{2.176822in}}{\pgfqpoint{5.339229in}{2.169689in}}%
\pgfpathcurveto{\pgfqpoint{5.332097in}{2.162556in}}{\pgfqpoint{5.328089in}{2.152881in}}{\pgfqpoint{5.328089in}{2.142793in}}%
\pgfpathcurveto{\pgfqpoint{5.328089in}{2.132706in}}{\pgfqpoint{5.332097in}{2.123030in}}{\pgfqpoint{5.339229in}{2.115898in}}%
\pgfpathcurveto{\pgfqpoint{5.346362in}{2.108765in}}{\pgfqpoint{5.356038in}{2.104757in}}{\pgfqpoint{5.366125in}{2.104757in}}%
\pgfpathlineto{\pgfqpoint{5.366125in}{2.104757in}}%
\pgfpathclose%
\pgfusepath{stroke,fill}%
\end{pgfscope}%
\begin{pgfscope}%
\pgfpathrectangle{\pgfqpoint{3.536584in}{0.147348in}}{\pgfqpoint{2.735294in}{2.735294in}}%
\pgfusepath{clip}%
\pgfsetbuttcap%
\pgfsetroundjoin%
\definecolor{currentfill}{rgb}{0.839216,0.152941,0.156863}%
\pgfsetfillcolor{currentfill}%
\pgfsetfillopacity{0.634032}%
\pgfsetlinewidth{1.003750pt}%
\definecolor{currentstroke}{rgb}{0.839216,0.152941,0.156863}%
\pgfsetstrokecolor{currentstroke}%
\pgfsetstrokeopacity{0.634032}%
\pgfsetdash{}{0pt}%
\pgfpathmoveto{\pgfqpoint{5.637120in}{1.533362in}}%
\pgfpathcurveto{\pgfqpoint{5.647207in}{1.533362in}}{\pgfqpoint{5.656883in}{1.537370in}}{\pgfqpoint{5.664016in}{1.544503in}}%
\pgfpathcurveto{\pgfqpoint{5.671149in}{1.551635in}}{\pgfqpoint{5.675156in}{1.561311in}}{\pgfqpoint{5.675156in}{1.571398in}}%
\pgfpathcurveto{\pgfqpoint{5.675156in}{1.581486in}}{\pgfqpoint{5.671149in}{1.591161in}}{\pgfqpoint{5.664016in}{1.598294in}}%
\pgfpathcurveto{\pgfqpoint{5.656883in}{1.605427in}}{\pgfqpoint{5.647207in}{1.609435in}}{\pgfqpoint{5.637120in}{1.609435in}}%
\pgfpathcurveto{\pgfqpoint{5.627033in}{1.609435in}}{\pgfqpoint{5.617357in}{1.605427in}}{\pgfqpoint{5.610224in}{1.598294in}}%
\pgfpathcurveto{\pgfqpoint{5.603091in}{1.591161in}}{\pgfqpoint{5.599084in}{1.581486in}}{\pgfqpoint{5.599084in}{1.571398in}}%
\pgfpathcurveto{\pgfqpoint{5.599084in}{1.561311in}}{\pgfqpoint{5.603091in}{1.551635in}}{\pgfqpoint{5.610224in}{1.544503in}}%
\pgfpathcurveto{\pgfqpoint{5.617357in}{1.537370in}}{\pgfqpoint{5.627033in}{1.533362in}}{\pgfqpoint{5.637120in}{1.533362in}}%
\pgfpathlineto{\pgfqpoint{5.637120in}{1.533362in}}%
\pgfpathclose%
\pgfusepath{stroke,fill}%
\end{pgfscope}%
\begin{pgfscope}%
\pgfpathrectangle{\pgfqpoint{3.536584in}{0.147348in}}{\pgfqpoint{2.735294in}{2.735294in}}%
\pgfusepath{clip}%
\pgfsetbuttcap%
\pgfsetroundjoin%
\definecolor{currentfill}{rgb}{0.839216,0.152941,0.156863}%
\pgfsetfillcolor{currentfill}%
\pgfsetfillopacity{0.652064}%
\pgfsetlinewidth{1.003750pt}%
\definecolor{currentstroke}{rgb}{0.839216,0.152941,0.156863}%
\pgfsetstrokecolor{currentstroke}%
\pgfsetstrokeopacity{0.652064}%
\pgfsetdash{}{0pt}%
\pgfpathmoveto{\pgfqpoint{4.750408in}{0.952320in}}%
\pgfpathcurveto{\pgfqpoint{4.760495in}{0.952320in}}{\pgfqpoint{4.770171in}{0.956328in}}{\pgfqpoint{4.777304in}{0.963461in}}%
\pgfpathcurveto{\pgfqpoint{4.784437in}{0.970593in}}{\pgfqpoint{4.788444in}{0.980269in}}{\pgfqpoint{4.788444in}{0.990356in}}%
\pgfpathcurveto{\pgfqpoint{4.788444in}{1.000444in}}{\pgfqpoint{4.784437in}{1.010119in}}{\pgfqpoint{4.777304in}{1.017252in}}%
\pgfpathcurveto{\pgfqpoint{4.770171in}{1.024385in}}{\pgfqpoint{4.760495in}{1.028393in}}{\pgfqpoint{4.750408in}{1.028393in}}%
\pgfpathcurveto{\pgfqpoint{4.740321in}{1.028393in}}{\pgfqpoint{4.730645in}{1.024385in}}{\pgfqpoint{4.723512in}{1.017252in}}%
\pgfpathcurveto{\pgfqpoint{4.716379in}{1.010119in}}{\pgfqpoint{4.712372in}{1.000444in}}{\pgfqpoint{4.712372in}{0.990356in}}%
\pgfpathcurveto{\pgfqpoint{4.712372in}{0.980269in}}{\pgfqpoint{4.716379in}{0.970593in}}{\pgfqpoint{4.723512in}{0.963461in}}%
\pgfpathcurveto{\pgfqpoint{4.730645in}{0.956328in}}{\pgfqpoint{4.740321in}{0.952320in}}{\pgfqpoint{4.750408in}{0.952320in}}%
\pgfpathlineto{\pgfqpoint{4.750408in}{0.952320in}}%
\pgfpathclose%
\pgfusepath{stroke,fill}%
\end{pgfscope}%
\begin{pgfscope}%
\pgfpathrectangle{\pgfqpoint{3.536584in}{0.147348in}}{\pgfqpoint{2.735294in}{2.735294in}}%
\pgfusepath{clip}%
\pgfsetbuttcap%
\pgfsetroundjoin%
\definecolor{currentfill}{rgb}{0.839216,0.152941,0.156863}%
\pgfsetfillcolor{currentfill}%
\pgfsetfillopacity{0.738747}%
\pgfsetlinewidth{1.003750pt}%
\definecolor{currentstroke}{rgb}{0.839216,0.152941,0.156863}%
\pgfsetstrokecolor{currentstroke}%
\pgfsetstrokeopacity{0.738747}%
\pgfsetdash{}{0pt}%
\pgfpathmoveto{\pgfqpoint{4.140523in}{1.507716in}}%
\pgfpathcurveto{\pgfqpoint{4.150611in}{1.507716in}}{\pgfqpoint{4.160286in}{1.511723in}}{\pgfqpoint{4.167419in}{1.518856in}}%
\pgfpathcurveto{\pgfqpoint{4.174552in}{1.525989in}}{\pgfqpoint{4.178560in}{1.535665in}}{\pgfqpoint{4.178560in}{1.545752in}}%
\pgfpathcurveto{\pgfqpoint{4.178560in}{1.555839in}}{\pgfqpoint{4.174552in}{1.565515in}}{\pgfqpoint{4.167419in}{1.572648in}}%
\pgfpathcurveto{\pgfqpoint{4.160286in}{1.579780in}}{\pgfqpoint{4.150611in}{1.583788in}}{\pgfqpoint{4.140523in}{1.583788in}}%
\pgfpathcurveto{\pgfqpoint{4.130436in}{1.583788in}}{\pgfqpoint{4.120761in}{1.579780in}}{\pgfqpoint{4.113628in}{1.572648in}}%
\pgfpathcurveto{\pgfqpoint{4.106495in}{1.565515in}}{\pgfqpoint{4.102487in}{1.555839in}}{\pgfqpoint{4.102487in}{1.545752in}}%
\pgfpathcurveto{\pgfqpoint{4.102487in}{1.535665in}}{\pgfqpoint{4.106495in}{1.525989in}}{\pgfqpoint{4.113628in}{1.518856in}}%
\pgfpathcurveto{\pgfqpoint{4.120761in}{1.511723in}}{\pgfqpoint{4.130436in}{1.507716in}}{\pgfqpoint{4.140523in}{1.507716in}}%
\pgfpathlineto{\pgfqpoint{4.140523in}{1.507716in}}%
\pgfpathclose%
\pgfusepath{stroke,fill}%
\end{pgfscope}%
\begin{pgfscope}%
\pgfpathrectangle{\pgfqpoint{3.536584in}{0.147348in}}{\pgfqpoint{2.735294in}{2.735294in}}%
\pgfusepath{clip}%
\pgfsetbuttcap%
\pgfsetroundjoin%
\definecolor{currentfill}{rgb}{0.839216,0.152941,0.156863}%
\pgfsetfillcolor{currentfill}%
\pgfsetfillopacity{0.791813}%
\pgfsetlinewidth{1.003750pt}%
\definecolor{currentstroke}{rgb}{0.839216,0.152941,0.156863}%
\pgfsetstrokecolor{currentstroke}%
\pgfsetstrokeopacity{0.791813}%
\pgfsetdash{}{0pt}%
\pgfpathmoveto{\pgfqpoint{4.633661in}{2.014183in}}%
\pgfpathcurveto{\pgfqpoint{4.643748in}{2.014183in}}{\pgfqpoint{4.653424in}{2.018191in}}{\pgfqpoint{4.660557in}{2.025323in}}%
\pgfpathcurveto{\pgfqpoint{4.667690in}{2.032456in}}{\pgfqpoint{4.671697in}{2.042132in}}{\pgfqpoint{4.671697in}{2.052219in}}%
\pgfpathcurveto{\pgfqpoint{4.671697in}{2.062307in}}{\pgfqpoint{4.667690in}{2.071982in}}{\pgfqpoint{4.660557in}{2.079115in}}%
\pgfpathcurveto{\pgfqpoint{4.653424in}{2.086248in}}{\pgfqpoint{4.643748in}{2.090255in}}{\pgfqpoint{4.633661in}{2.090255in}}%
\pgfpathcurveto{\pgfqpoint{4.623574in}{2.090255in}}{\pgfqpoint{4.613898in}{2.086248in}}{\pgfqpoint{4.606765in}{2.079115in}}%
\pgfpathcurveto{\pgfqpoint{4.599632in}{2.071982in}}{\pgfqpoint{4.595625in}{2.062307in}}{\pgfqpoint{4.595625in}{2.052219in}}%
\pgfpathcurveto{\pgfqpoint{4.595625in}{2.042132in}}{\pgfqpoint{4.599632in}{2.032456in}}{\pgfqpoint{4.606765in}{2.025323in}}%
\pgfpathcurveto{\pgfqpoint{4.613898in}{2.018191in}}{\pgfqpoint{4.623574in}{2.014183in}}{\pgfqpoint{4.633661in}{2.014183in}}%
\pgfpathlineto{\pgfqpoint{4.633661in}{2.014183in}}%
\pgfpathclose%
\pgfusepath{stroke,fill}%
\end{pgfscope}%
\begin{pgfscope}%
\pgfpathrectangle{\pgfqpoint{3.536584in}{0.147348in}}{\pgfqpoint{2.735294in}{2.735294in}}%
\pgfusepath{clip}%
\pgfsetbuttcap%
\pgfsetroundjoin%
\definecolor{currentfill}{rgb}{0.839216,0.152941,0.156863}%
\pgfsetfillcolor{currentfill}%
\pgfsetfillopacity{0.864233}%
\pgfsetlinewidth{1.003750pt}%
\definecolor{currentstroke}{rgb}{0.839216,0.152941,0.156863}%
\pgfsetstrokecolor{currentstroke}%
\pgfsetstrokeopacity{0.864233}%
\pgfsetdash{}{0pt}%
\pgfpathmoveto{\pgfqpoint{5.299988in}{1.926870in}}%
\pgfpathcurveto{\pgfqpoint{5.310075in}{1.926870in}}{\pgfqpoint{5.319750in}{1.930878in}}{\pgfqpoint{5.326883in}{1.938011in}}%
\pgfpathcurveto{\pgfqpoint{5.334016in}{1.945143in}}{\pgfqpoint{5.338024in}{1.954819in}}{\pgfqpoint{5.338024in}{1.964906in}}%
\pgfpathcurveto{\pgfqpoint{5.338024in}{1.974994in}}{\pgfqpoint{5.334016in}{1.984669in}}{\pgfqpoint{5.326883in}{1.991802in}}%
\pgfpathcurveto{\pgfqpoint{5.319750in}{1.998935in}}{\pgfqpoint{5.310075in}{2.002943in}}{\pgfqpoint{5.299988in}{2.002943in}}%
\pgfpathcurveto{\pgfqpoint{5.289900in}{2.002943in}}{\pgfqpoint{5.280225in}{1.998935in}}{\pgfqpoint{5.273092in}{1.991802in}}%
\pgfpathcurveto{\pgfqpoint{5.265959in}{1.984669in}}{\pgfqpoint{5.261951in}{1.974994in}}{\pgfqpoint{5.261951in}{1.964906in}}%
\pgfpathcurveto{\pgfqpoint{5.261951in}{1.954819in}}{\pgfqpoint{5.265959in}{1.945143in}}{\pgfqpoint{5.273092in}{1.938011in}}%
\pgfpathcurveto{\pgfqpoint{5.280225in}{1.930878in}}{\pgfqpoint{5.289900in}{1.926870in}}{\pgfqpoint{5.299988in}{1.926870in}}%
\pgfpathlineto{\pgfqpoint{5.299988in}{1.926870in}}%
\pgfpathclose%
\pgfusepath{stroke,fill}%
\end{pgfscope}%
\begin{pgfscope}%
\pgfpathrectangle{\pgfqpoint{3.536584in}{0.147348in}}{\pgfqpoint{2.735294in}{2.735294in}}%
\pgfusepath{clip}%
\pgfsetbuttcap%
\pgfsetroundjoin%
\definecolor{currentfill}{rgb}{0.839216,0.152941,0.156863}%
\pgfsetfillcolor{currentfill}%
\pgfsetfillopacity{0.929084}%
\pgfsetlinewidth{1.003750pt}%
\definecolor{currentstroke}{rgb}{0.839216,0.152941,0.156863}%
\pgfsetstrokecolor{currentstroke}%
\pgfsetstrokeopacity{0.929084}%
\pgfsetdash{}{0pt}%
\pgfpathmoveto{\pgfqpoint{5.359400in}{1.288920in}}%
\pgfpathcurveto{\pgfqpoint{5.369488in}{1.288920in}}{\pgfqpoint{5.379163in}{1.292928in}}{\pgfqpoint{5.386296in}{1.300060in}}%
\pgfpathcurveto{\pgfqpoint{5.393429in}{1.307193in}}{\pgfqpoint{5.397437in}{1.316869in}}{\pgfqpoint{5.397437in}{1.326956in}}%
\pgfpathcurveto{\pgfqpoint{5.397437in}{1.337043in}}{\pgfqpoint{5.393429in}{1.346719in}}{\pgfqpoint{5.386296in}{1.353852in}}%
\pgfpathcurveto{\pgfqpoint{5.379163in}{1.360985in}}{\pgfqpoint{5.369488in}{1.364992in}}{\pgfqpoint{5.359400in}{1.364992in}}%
\pgfpathcurveto{\pgfqpoint{5.349313in}{1.364992in}}{\pgfqpoint{5.339637in}{1.360985in}}{\pgfqpoint{5.332505in}{1.353852in}}%
\pgfpathcurveto{\pgfqpoint{5.325372in}{1.346719in}}{\pgfqpoint{5.321364in}{1.337043in}}{\pgfqpoint{5.321364in}{1.326956in}}%
\pgfpathcurveto{\pgfqpoint{5.321364in}{1.316869in}}{\pgfqpoint{5.325372in}{1.307193in}}{\pgfqpoint{5.332505in}{1.300060in}}%
\pgfpathcurveto{\pgfqpoint{5.339637in}{1.292928in}}{\pgfqpoint{5.349313in}{1.288920in}}{\pgfqpoint{5.359400in}{1.288920in}}%
\pgfpathlineto{\pgfqpoint{5.359400in}{1.288920in}}%
\pgfpathclose%
\pgfusepath{stroke,fill}%
\end{pgfscope}%
\begin{pgfscope}%
\pgfpathrectangle{\pgfqpoint{3.536584in}{0.147348in}}{\pgfqpoint{2.735294in}{2.735294in}}%
\pgfusepath{clip}%
\pgfsetbuttcap%
\pgfsetroundjoin%
\definecolor{currentfill}{rgb}{0.839216,0.152941,0.156863}%
\pgfsetfillcolor{currentfill}%
\pgfsetlinewidth{1.003750pt}%
\definecolor{currentstroke}{rgb}{0.839216,0.152941,0.156863}%
\pgfsetstrokecolor{currentstroke}%
\pgfsetdash{}{0pt}%
\pgfpathmoveto{\pgfqpoint{5.239366in}{1.506712in}}%
\pgfpathcurveto{\pgfqpoint{5.249453in}{1.506712in}}{\pgfqpoint{5.259128in}{1.510720in}}{\pgfqpoint{5.266261in}{1.517853in}}%
\pgfpathcurveto{\pgfqpoint{5.273394in}{1.524986in}}{\pgfqpoint{5.277402in}{1.534661in}}{\pgfqpoint{5.277402in}{1.544748in}}%
\pgfpathcurveto{\pgfqpoint{5.277402in}{1.554836in}}{\pgfqpoint{5.273394in}{1.564511in}}{\pgfqpoint{5.266261in}{1.571644in}}%
\pgfpathcurveto{\pgfqpoint{5.259128in}{1.578777in}}{\pgfqpoint{5.249453in}{1.582785in}}{\pgfqpoint{5.239366in}{1.582785in}}%
\pgfpathcurveto{\pgfqpoint{5.229278in}{1.582785in}}{\pgfqpoint{5.219603in}{1.578777in}}{\pgfqpoint{5.212470in}{1.571644in}}%
\pgfpathcurveto{\pgfqpoint{5.205337in}{1.564511in}}{\pgfqpoint{5.201329in}{1.554836in}}{\pgfqpoint{5.201329in}{1.544748in}}%
\pgfpathcurveto{\pgfqpoint{5.201329in}{1.534661in}}{\pgfqpoint{5.205337in}{1.524986in}}{\pgfqpoint{5.212470in}{1.517853in}}%
\pgfpathcurveto{\pgfqpoint{5.219603in}{1.510720in}}{\pgfqpoint{5.229278in}{1.506712in}}{\pgfqpoint{5.239366in}{1.506712in}}%
\pgfpathlineto{\pgfqpoint{5.239366in}{1.506712in}}%
\pgfpathclose%
\pgfusepath{stroke,fill}%
\end{pgfscope}%
\begin{pgfscope}%
\pgfpathrectangle{\pgfqpoint{3.536584in}{0.147348in}}{\pgfqpoint{2.735294in}{2.735294in}}%
\pgfusepath{clip}%
\pgfsetbuttcap%
\pgfsetroundjoin%
\definecolor{currentfill}{rgb}{0.071067,0.258424,0.071067}%
\pgfsetfillcolor{currentfill}%
\pgfsetfillopacity{0.200000}%
\pgfsetlinewidth{0.000000pt}%
\definecolor{currentstroke}{rgb}{0.000000,0.000000,0.000000}%
\pgfsetstrokecolor{currentstroke}%
\pgfsetdash{}{0pt}%
\pgfpathmoveto{\pgfqpoint{3.979947in}{1.088921in}}%
\pgfpathlineto{\pgfqpoint{3.883527in}{1.228895in}}%
\pgfpathlineto{\pgfqpoint{3.884101in}{1.155548in}}%
\pgfpathlineto{\pgfqpoint{3.979947in}{1.088921in}}%
\pgfpathclose%
\pgfusepath{fill}%
\end{pgfscope}%
\begin{pgfscope}%
\pgfpathrectangle{\pgfqpoint{3.536584in}{0.147348in}}{\pgfqpoint{2.735294in}{2.735294in}}%
\pgfusepath{clip}%
\pgfsetbuttcap%
\pgfsetroundjoin%
\definecolor{currentfill}{rgb}{0.071067,0.258424,0.071067}%
\pgfsetfillcolor{currentfill}%
\pgfsetfillopacity{0.200000}%
\pgfsetlinewidth{0.000000pt}%
\definecolor{currentstroke}{rgb}{0.000000,0.000000,0.000000}%
\pgfsetstrokecolor{currentstroke}%
\pgfsetdash{}{0pt}%
\pgfpathmoveto{\pgfqpoint{5.998863in}{1.228895in}}%
\pgfpathlineto{\pgfqpoint{5.902443in}{1.088921in}}%
\pgfpathlineto{\pgfqpoint{5.998289in}{1.155548in}}%
\pgfpathlineto{\pgfqpoint{5.998863in}{1.228895in}}%
\pgfpathclose%
\pgfusepath{fill}%
\end{pgfscope}%
\begin{pgfscope}%
\pgfpathrectangle{\pgfqpoint{3.536584in}{0.147348in}}{\pgfqpoint{2.735294in}{2.735294in}}%
\pgfusepath{clip}%
\pgfsetbuttcap%
\pgfsetroundjoin%
\definecolor{currentfill}{rgb}{0.128601,0.467641,0.128601}%
\pgfsetfillcolor{currentfill}%
\pgfsetfillopacity{0.200000}%
\pgfsetlinewidth{0.000000pt}%
\definecolor{currentstroke}{rgb}{0.000000,0.000000,0.000000}%
\pgfsetstrokecolor{currentstroke}%
\pgfsetdash{}{0pt}%
\pgfpathmoveto{\pgfqpoint{4.844774in}{2.632943in}}%
\pgfpathlineto{\pgfqpoint{5.037616in}{2.632943in}}%
\pgfpathlineto{\pgfqpoint{4.941195in}{2.699366in}}%
\pgfpathlineto{\pgfqpoint{4.844774in}{2.632943in}}%
\pgfpathclose%
\pgfusepath{fill}%
\end{pgfscope}%
\begin{pgfscope}%
\pgfpathrectangle{\pgfqpoint{3.536584in}{0.147348in}}{\pgfqpoint{2.735294in}{2.735294in}}%
\pgfusepath{clip}%
\pgfsetbuttcap%
\pgfsetroundjoin%
\definecolor{currentfill}{rgb}{0.067488,0.245410,0.067488}%
\pgfsetfillcolor{currentfill}%
\pgfsetfillopacity{0.200000}%
\pgfsetlinewidth{0.000000pt}%
\definecolor{currentstroke}{rgb}{0.000000,0.000000,0.000000}%
\pgfsetstrokecolor{currentstroke}%
\pgfsetdash{}{0pt}%
\pgfpathmoveto{\pgfqpoint{4.104467in}{1.018707in}}%
\pgfpathlineto{\pgfqpoint{3.989844in}{1.163589in}}%
\pgfpathlineto{\pgfqpoint{3.979947in}{1.088921in}}%
\pgfpathlineto{\pgfqpoint{4.104467in}{1.018707in}}%
\pgfpathclose%
\pgfusepath{fill}%
\end{pgfscope}%
\begin{pgfscope}%
\pgfpathrectangle{\pgfqpoint{3.536584in}{0.147348in}}{\pgfqpoint{2.735294in}{2.735294in}}%
\pgfusepath{clip}%
\pgfsetbuttcap%
\pgfsetroundjoin%
\definecolor{currentfill}{rgb}{0.067488,0.245410,0.067488}%
\pgfsetfillcolor{currentfill}%
\pgfsetfillopacity{0.200000}%
\pgfsetlinewidth{0.000000pt}%
\definecolor{currentstroke}{rgb}{0.000000,0.000000,0.000000}%
\pgfsetstrokecolor{currentstroke}%
\pgfsetdash{}{0pt}%
\pgfpathmoveto{\pgfqpoint{5.902443in}{1.088921in}}%
\pgfpathlineto{\pgfqpoint{5.892546in}{1.163589in}}%
\pgfpathlineto{\pgfqpoint{5.777922in}{1.018707in}}%
\pgfpathlineto{\pgfqpoint{5.902443in}{1.088921in}}%
\pgfpathclose%
\pgfusepath{fill}%
\end{pgfscope}%
\begin{pgfscope}%
\pgfpathrectangle{\pgfqpoint{3.536584in}{0.147348in}}{\pgfqpoint{2.735294in}{2.735294in}}%
\pgfusepath{clip}%
\pgfsetbuttcap%
\pgfsetroundjoin%
\definecolor{currentfill}{rgb}{0.069492,0.252698,0.069492}%
\pgfsetfillcolor{currentfill}%
\pgfsetfillopacity{0.200000}%
\pgfsetlinewidth{0.000000pt}%
\definecolor{currentstroke}{rgb}{0.000000,0.000000,0.000000}%
\pgfsetstrokecolor{currentstroke}%
\pgfsetdash{}{0pt}%
\pgfpathmoveto{\pgfqpoint{3.883527in}{1.228895in}}%
\pgfpathlineto{\pgfqpoint{3.979947in}{1.088921in}}%
\pgfpathlineto{\pgfqpoint{3.977739in}{1.589726in}}%
\pgfpathlineto{\pgfqpoint{3.883527in}{1.228895in}}%
\pgfpathclose%
\pgfusepath{fill}%
\end{pgfscope}%
\begin{pgfscope}%
\pgfpathrectangle{\pgfqpoint{3.536584in}{0.147348in}}{\pgfqpoint{2.735294in}{2.735294in}}%
\pgfusepath{clip}%
\pgfsetbuttcap%
\pgfsetroundjoin%
\definecolor{currentfill}{rgb}{0.069492,0.252698,0.069492}%
\pgfsetfillcolor{currentfill}%
\pgfsetfillopacity{0.200000}%
\pgfsetlinewidth{0.000000pt}%
\definecolor{currentstroke}{rgb}{0.000000,0.000000,0.000000}%
\pgfsetstrokecolor{currentstroke}%
\pgfsetdash{}{0pt}%
\pgfpathmoveto{\pgfqpoint{5.998863in}{1.228895in}}%
\pgfpathlineto{\pgfqpoint{5.904651in}{1.589726in}}%
\pgfpathlineto{\pgfqpoint{5.902443in}{1.088921in}}%
\pgfpathlineto{\pgfqpoint{5.998863in}{1.228895in}}%
\pgfpathclose%
\pgfusepath{fill}%
\end{pgfscope}%
\begin{pgfscope}%
\pgfpathrectangle{\pgfqpoint{3.536584in}{0.147348in}}{\pgfqpoint{2.735294in}{2.735294in}}%
\pgfusepath{clip}%
\pgfsetbuttcap%
\pgfsetroundjoin%
\definecolor{currentfill}{rgb}{0.099716,0.362602,0.099716}%
\pgfsetfillcolor{currentfill}%
\pgfsetfillopacity{0.200000}%
\pgfsetlinewidth{0.000000pt}%
\definecolor{currentstroke}{rgb}{0.000000,0.000000,0.000000}%
\pgfsetstrokecolor{currentstroke}%
\pgfsetdash{}{0pt}%
\pgfpathmoveto{\pgfqpoint{3.977739in}{1.589726in}}%
\pgfpathlineto{\pgfqpoint{3.979947in}{1.088921in}}%
\pgfpathlineto{\pgfqpoint{3.989844in}{1.163589in}}%
\pgfpathlineto{\pgfqpoint{3.977739in}{1.589726in}}%
\pgfpathclose%
\pgfusepath{fill}%
\end{pgfscope}%
\begin{pgfscope}%
\pgfpathrectangle{\pgfqpoint{3.536584in}{0.147348in}}{\pgfqpoint{2.735294in}{2.735294in}}%
\pgfusepath{clip}%
\pgfsetbuttcap%
\pgfsetroundjoin%
\definecolor{currentfill}{rgb}{0.099716,0.362602,0.099716}%
\pgfsetfillcolor{currentfill}%
\pgfsetfillopacity{0.200000}%
\pgfsetlinewidth{0.000000pt}%
\definecolor{currentstroke}{rgb}{0.000000,0.000000,0.000000}%
\pgfsetstrokecolor{currentstroke}%
\pgfsetdash{}{0pt}%
\pgfpathmoveto{\pgfqpoint{5.892546in}{1.163589in}}%
\pgfpathlineto{\pgfqpoint{5.902443in}{1.088921in}}%
\pgfpathlineto{\pgfqpoint{5.904651in}{1.589726in}}%
\pgfpathlineto{\pgfqpoint{5.892546in}{1.163589in}}%
\pgfpathclose%
\pgfusepath{fill}%
\end{pgfscope}%
\begin{pgfscope}%
\pgfpathrectangle{\pgfqpoint{3.536584in}{0.147348in}}{\pgfqpoint{2.735294in}{2.735294in}}%
\pgfusepath{clip}%
\pgfsetbuttcap%
\pgfsetroundjoin%
\definecolor{currentfill}{rgb}{0.063840,0.232145,0.063840}%
\pgfsetfillcolor{currentfill}%
\pgfsetfillopacity{0.200000}%
\pgfsetlinewidth{0.000000pt}%
\definecolor{currentstroke}{rgb}{0.000000,0.000000,0.000000}%
\pgfsetstrokecolor{currentstroke}%
\pgfsetdash{}{0pt}%
\pgfpathmoveto{\pgfqpoint{4.263554in}{0.949464in}}%
\pgfpathlineto{\pgfqpoint{4.130959in}{1.094945in}}%
\pgfpathlineto{\pgfqpoint{4.104467in}{1.018707in}}%
\pgfpathlineto{\pgfqpoint{4.263554in}{0.949464in}}%
\pgfpathclose%
\pgfusepath{fill}%
\end{pgfscope}%
\begin{pgfscope}%
\pgfpathrectangle{\pgfqpoint{3.536584in}{0.147348in}}{\pgfqpoint{2.735294in}{2.735294in}}%
\pgfusepath{clip}%
\pgfsetbuttcap%
\pgfsetroundjoin%
\definecolor{currentfill}{rgb}{0.063840,0.232145,0.063840}%
\pgfsetfillcolor{currentfill}%
\pgfsetfillopacity{0.200000}%
\pgfsetlinewidth{0.000000pt}%
\definecolor{currentstroke}{rgb}{0.000000,0.000000,0.000000}%
\pgfsetstrokecolor{currentstroke}%
\pgfsetdash{}{0pt}%
\pgfpathmoveto{\pgfqpoint{5.777922in}{1.018707in}}%
\pgfpathlineto{\pgfqpoint{5.751431in}{1.094945in}}%
\pgfpathlineto{\pgfqpoint{5.618836in}{0.949464in}}%
\pgfpathlineto{\pgfqpoint{5.777922in}{1.018707in}}%
\pgfpathclose%
\pgfusepath{fill}%
\end{pgfscope}%
\begin{pgfscope}%
\pgfpathrectangle{\pgfqpoint{3.536584in}{0.147348in}}{\pgfqpoint{2.735294in}{2.735294in}}%
\pgfusepath{clip}%
\pgfsetbuttcap%
\pgfsetroundjoin%
\definecolor{currentfill}{rgb}{0.116321,0.422987,0.116321}%
\pgfsetfillcolor{currentfill}%
\pgfsetfillopacity{0.200000}%
\pgfsetlinewidth{0.000000pt}%
\definecolor{currentstroke}{rgb}{0.000000,0.000000,0.000000}%
\pgfsetstrokecolor{currentstroke}%
\pgfsetdash{}{0pt}%
\pgfpathmoveto{\pgfqpoint{5.037616in}{2.632943in}}%
\pgfpathlineto{\pgfqpoint{4.844774in}{2.632943in}}%
\pgfpathlineto{\pgfqpoint{4.802903in}{2.150689in}}%
\pgfpathlineto{\pgfqpoint{5.037616in}{2.632943in}}%
\pgfpathclose%
\pgfusepath{fill}%
\end{pgfscope}%
\begin{pgfscope}%
\pgfpathrectangle{\pgfqpoint{3.536584in}{0.147348in}}{\pgfqpoint{2.735294in}{2.735294in}}%
\pgfusepath{clip}%
\pgfsetbuttcap%
\pgfsetroundjoin%
\definecolor{currentfill}{rgb}{0.067061,0.243857,0.067061}%
\pgfsetfillcolor{currentfill}%
\pgfsetfillopacity{0.200000}%
\pgfsetlinewidth{0.000000pt}%
\definecolor{currentstroke}{rgb}{0.000000,0.000000,0.000000}%
\pgfsetstrokecolor{currentstroke}%
\pgfsetdash{}{0pt}%
\pgfpathmoveto{\pgfqpoint{3.989844in}{1.163589in}}%
\pgfpathlineto{\pgfqpoint{4.104467in}{1.018707in}}%
\pgfpathlineto{\pgfqpoint{4.139943in}{1.557460in}}%
\pgfpathlineto{\pgfqpoint{3.989844in}{1.163589in}}%
\pgfpathclose%
\pgfusepath{fill}%
\end{pgfscope}%
\begin{pgfscope}%
\pgfpathrectangle{\pgfqpoint{3.536584in}{0.147348in}}{\pgfqpoint{2.735294in}{2.735294in}}%
\pgfusepath{clip}%
\pgfsetbuttcap%
\pgfsetroundjoin%
\definecolor{currentfill}{rgb}{0.067061,0.243857,0.067061}%
\pgfsetfillcolor{currentfill}%
\pgfsetfillopacity{0.200000}%
\pgfsetlinewidth{0.000000pt}%
\definecolor{currentstroke}{rgb}{0.000000,0.000000,0.000000}%
\pgfsetstrokecolor{currentstroke}%
\pgfsetdash{}{0pt}%
\pgfpathmoveto{\pgfqpoint{5.742447in}{1.557460in}}%
\pgfpathlineto{\pgfqpoint{5.777922in}{1.018707in}}%
\pgfpathlineto{\pgfqpoint{5.892546in}{1.163589in}}%
\pgfpathlineto{\pgfqpoint{5.742447in}{1.557460in}}%
\pgfpathclose%
\pgfusepath{fill}%
\end{pgfscope}%
\begin{pgfscope}%
\pgfpathrectangle{\pgfqpoint{3.536584in}{0.147348in}}{\pgfqpoint{2.735294in}{2.735294in}}%
\pgfusepath{clip}%
\pgfsetbuttcap%
\pgfsetroundjoin%
\definecolor{currentfill}{rgb}{0.095351,0.346729,0.095351}%
\pgfsetfillcolor{currentfill}%
\pgfsetfillopacity{0.200000}%
\pgfsetlinewidth{0.000000pt}%
\definecolor{currentstroke}{rgb}{0.000000,0.000000,0.000000}%
\pgfsetstrokecolor{currentstroke}%
\pgfsetdash{}{0pt}%
\pgfpathmoveto{\pgfqpoint{4.139943in}{1.557460in}}%
\pgfpathlineto{\pgfqpoint{4.104467in}{1.018707in}}%
\pgfpathlineto{\pgfqpoint{4.130959in}{1.094945in}}%
\pgfpathlineto{\pgfqpoint{4.139943in}{1.557460in}}%
\pgfpathclose%
\pgfusepath{fill}%
\end{pgfscope}%
\begin{pgfscope}%
\pgfpathrectangle{\pgfqpoint{3.536584in}{0.147348in}}{\pgfqpoint{2.735294in}{2.735294in}}%
\pgfusepath{clip}%
\pgfsetbuttcap%
\pgfsetroundjoin%
\definecolor{currentfill}{rgb}{0.095351,0.346729,0.095351}%
\pgfsetfillcolor{currentfill}%
\pgfsetfillopacity{0.200000}%
\pgfsetlinewidth{0.000000pt}%
\definecolor{currentstroke}{rgb}{0.000000,0.000000,0.000000}%
\pgfsetstrokecolor{currentstroke}%
\pgfsetdash{}{0pt}%
\pgfpathmoveto{\pgfqpoint{5.751431in}{1.094945in}}%
\pgfpathlineto{\pgfqpoint{5.777922in}{1.018707in}}%
\pgfpathlineto{\pgfqpoint{5.742447in}{1.557460in}}%
\pgfpathlineto{\pgfqpoint{5.751431in}{1.094945in}}%
\pgfpathclose%
\pgfusepath{fill}%
\end{pgfscope}%
\begin{pgfscope}%
\pgfpathrectangle{\pgfqpoint{3.536584in}{0.147348in}}{\pgfqpoint{2.735294in}{2.735294in}}%
\pgfusepath{clip}%
\pgfsetbuttcap%
\pgfsetroundjoin%
\definecolor{currentfill}{rgb}{0.060435,0.219763,0.060435}%
\pgfsetfillcolor{currentfill}%
\pgfsetfillopacity{0.200000}%
\pgfsetlinewidth{0.000000pt}%
\definecolor{currentstroke}{rgb}{0.000000,0.000000,0.000000}%
\pgfsetstrokecolor{currentstroke}%
\pgfsetdash{}{0pt}%
\pgfpathmoveto{\pgfqpoint{5.568365in}{1.028737in}}%
\pgfpathlineto{\pgfqpoint{5.422251in}{0.888724in}}%
\pgfpathlineto{\pgfqpoint{5.618836in}{0.949464in}}%
\pgfpathlineto{\pgfqpoint{5.568365in}{1.028737in}}%
\pgfpathclose%
\pgfusepath{fill}%
\end{pgfscope}%
\begin{pgfscope}%
\pgfpathrectangle{\pgfqpoint{3.536584in}{0.147348in}}{\pgfqpoint{2.735294in}{2.735294in}}%
\pgfusepath{clip}%
\pgfsetbuttcap%
\pgfsetroundjoin%
\definecolor{currentfill}{rgb}{0.060435,0.219763,0.060435}%
\pgfsetfillcolor{currentfill}%
\pgfsetfillopacity{0.200000}%
\pgfsetlinewidth{0.000000pt}%
\definecolor{currentstroke}{rgb}{0.000000,0.000000,0.000000}%
\pgfsetstrokecolor{currentstroke}%
\pgfsetdash{}{0pt}%
\pgfpathmoveto{\pgfqpoint{4.263554in}{0.949464in}}%
\pgfpathlineto{\pgfqpoint{4.460139in}{0.888724in}}%
\pgfpathlineto{\pgfqpoint{4.314024in}{1.028737in}}%
\pgfpathlineto{\pgfqpoint{4.263554in}{0.949464in}}%
\pgfpathclose%
\pgfusepath{fill}%
\end{pgfscope}%
\begin{pgfscope}%
\pgfpathrectangle{\pgfqpoint{3.536584in}{0.147348in}}{\pgfqpoint{2.735294in}{2.735294in}}%
\pgfusepath{clip}%
\pgfsetbuttcap%
\pgfsetroundjoin%
\definecolor{currentfill}{rgb}{0.074506,0.270932,0.074506}%
\pgfsetfillcolor{currentfill}%
\pgfsetfillopacity{0.200000}%
\pgfsetlinewidth{0.000000pt}%
\definecolor{currentstroke}{rgb}{0.000000,0.000000,0.000000}%
\pgfsetstrokecolor{currentstroke}%
\pgfsetdash{}{0pt}%
\pgfpathmoveto{\pgfqpoint{3.977739in}{1.589726in}}%
\pgfpathlineto{\pgfqpoint{3.989844in}{1.163589in}}%
\pgfpathlineto{\pgfqpoint{4.139943in}{1.557460in}}%
\pgfpathlineto{\pgfqpoint{3.977739in}{1.589726in}}%
\pgfpathclose%
\pgfusepath{fill}%
\end{pgfscope}%
\begin{pgfscope}%
\pgfpathrectangle{\pgfqpoint{3.536584in}{0.147348in}}{\pgfqpoint{2.735294in}{2.735294in}}%
\pgfusepath{clip}%
\pgfsetbuttcap%
\pgfsetroundjoin%
\definecolor{currentfill}{rgb}{0.074506,0.270932,0.074506}%
\pgfsetfillcolor{currentfill}%
\pgfsetfillopacity{0.200000}%
\pgfsetlinewidth{0.000000pt}%
\definecolor{currentstroke}{rgb}{0.000000,0.000000,0.000000}%
\pgfsetstrokecolor{currentstroke}%
\pgfsetdash{}{0pt}%
\pgfpathmoveto{\pgfqpoint{5.742447in}{1.557460in}}%
\pgfpathlineto{\pgfqpoint{5.892546in}{1.163589in}}%
\pgfpathlineto{\pgfqpoint{5.904651in}{1.589726in}}%
\pgfpathlineto{\pgfqpoint{5.742447in}{1.557460in}}%
\pgfpathclose%
\pgfusepath{fill}%
\end{pgfscope}%
\begin{pgfscope}%
\pgfpathrectangle{\pgfqpoint{3.536584in}{0.147348in}}{\pgfqpoint{2.735294in}{2.735294in}}%
\pgfusepath{clip}%
\pgfsetbuttcap%
\pgfsetroundjoin%
\definecolor{currentfill}{rgb}{0.116785,0.424671,0.116785}%
\pgfsetfillcolor{currentfill}%
\pgfsetfillopacity{0.200000}%
\pgfsetlinewidth{0.000000pt}%
\definecolor{currentstroke}{rgb}{0.000000,0.000000,0.000000}%
\pgfsetstrokecolor{currentstroke}%
\pgfsetdash{}{0pt}%
\pgfpathmoveto{\pgfqpoint{5.423597in}{2.292690in}}%
\pgfpathlineto{\pgfqpoint{5.037616in}{2.632943in}}%
\pgfpathlineto{\pgfqpoint{5.079487in}{2.150689in}}%
\pgfpathlineto{\pgfqpoint{5.423597in}{2.292690in}}%
\pgfpathclose%
\pgfusepath{fill}%
\end{pgfscope}%
\begin{pgfscope}%
\pgfpathrectangle{\pgfqpoint{3.536584in}{0.147348in}}{\pgfqpoint{2.735294in}{2.735294in}}%
\pgfusepath{clip}%
\pgfsetbuttcap%
\pgfsetroundjoin%
\definecolor{currentfill}{rgb}{0.116785,0.424671,0.116785}%
\pgfsetfillcolor{currentfill}%
\pgfsetfillopacity{0.200000}%
\pgfsetlinewidth{0.000000pt}%
\definecolor{currentstroke}{rgb}{0.000000,0.000000,0.000000}%
\pgfsetstrokecolor{currentstroke}%
\pgfsetdash{}{0pt}%
\pgfpathmoveto{\pgfqpoint{4.802903in}{2.150689in}}%
\pgfpathlineto{\pgfqpoint{4.844774in}{2.632943in}}%
\pgfpathlineto{\pgfqpoint{4.458793in}{2.292690in}}%
\pgfpathlineto{\pgfqpoint{4.802903in}{2.150689in}}%
\pgfpathclose%
\pgfusepath{fill}%
\end{pgfscope}%
\begin{pgfscope}%
\pgfpathrectangle{\pgfqpoint{3.536584in}{0.147348in}}{\pgfqpoint{2.735294in}{2.735294in}}%
\pgfusepath{clip}%
\pgfsetbuttcap%
\pgfsetroundjoin%
\definecolor{currentfill}{rgb}{0.064867,0.235879,0.064867}%
\pgfsetfillcolor{currentfill}%
\pgfsetfillopacity{0.200000}%
\pgfsetlinewidth{0.000000pt}%
\definecolor{currentstroke}{rgb}{0.000000,0.000000,0.000000}%
\pgfsetstrokecolor{currentstroke}%
\pgfsetdash{}{0pt}%
\pgfpathmoveto{\pgfqpoint{4.130959in}{1.094945in}}%
\pgfpathlineto{\pgfqpoint{4.263554in}{0.949464in}}%
\pgfpathlineto{\pgfqpoint{4.357383in}{1.526914in}}%
\pgfpathlineto{\pgfqpoint{4.130959in}{1.094945in}}%
\pgfpathclose%
\pgfusepath{fill}%
\end{pgfscope}%
\begin{pgfscope}%
\pgfpathrectangle{\pgfqpoint{3.536584in}{0.147348in}}{\pgfqpoint{2.735294in}{2.735294in}}%
\pgfusepath{clip}%
\pgfsetbuttcap%
\pgfsetroundjoin%
\definecolor{currentfill}{rgb}{0.064867,0.235879,0.064867}%
\pgfsetfillcolor{currentfill}%
\pgfsetfillopacity{0.200000}%
\pgfsetlinewidth{0.000000pt}%
\definecolor{currentstroke}{rgb}{0.000000,0.000000,0.000000}%
\pgfsetstrokecolor{currentstroke}%
\pgfsetdash{}{0pt}%
\pgfpathmoveto{\pgfqpoint{5.525007in}{1.526914in}}%
\pgfpathlineto{\pgfqpoint{5.618836in}{0.949464in}}%
\pgfpathlineto{\pgfqpoint{5.751431in}{1.094945in}}%
\pgfpathlineto{\pgfqpoint{5.525007in}{1.526914in}}%
\pgfpathclose%
\pgfusepath{fill}%
\end{pgfscope}%
\begin{pgfscope}%
\pgfpathrectangle{\pgfqpoint{3.536584in}{0.147348in}}{\pgfqpoint{2.735294in}{2.735294in}}%
\pgfusepath{clip}%
\pgfsetbuttcap%
\pgfsetroundjoin%
\definecolor{currentfill}{rgb}{0.057724,0.209904,0.057724}%
\pgfsetfillcolor{currentfill}%
\pgfsetfillopacity{0.200000}%
\pgfsetlinewidth{0.000000pt}%
\definecolor{currentstroke}{rgb}{0.000000,0.000000,0.000000}%
\pgfsetstrokecolor{currentstroke}%
\pgfsetdash{}{0pt}%
\pgfpathmoveto{\pgfqpoint{4.541034in}{0.974376in}}%
\pgfpathlineto{\pgfqpoint{4.460139in}{0.888724in}}%
\pgfpathlineto{\pgfqpoint{4.690418in}{0.846223in}}%
\pgfpathlineto{\pgfqpoint{4.541034in}{0.974376in}}%
\pgfpathclose%
\pgfusepath{fill}%
\end{pgfscope}%
\begin{pgfscope}%
\pgfpathrectangle{\pgfqpoint{3.536584in}{0.147348in}}{\pgfqpoint{2.735294in}{2.735294in}}%
\pgfusepath{clip}%
\pgfsetbuttcap%
\pgfsetroundjoin%
\definecolor{currentfill}{rgb}{0.057724,0.209904,0.057724}%
\pgfsetfillcolor{currentfill}%
\pgfsetfillopacity{0.200000}%
\pgfsetlinewidth{0.000000pt}%
\definecolor{currentstroke}{rgb}{0.000000,0.000000,0.000000}%
\pgfsetstrokecolor{currentstroke}%
\pgfsetdash{}{0pt}%
\pgfpathmoveto{\pgfqpoint{5.191972in}{0.846223in}}%
\pgfpathlineto{\pgfqpoint{5.422251in}{0.888724in}}%
\pgfpathlineto{\pgfqpoint{5.341356in}{0.974376in}}%
\pgfpathlineto{\pgfqpoint{5.191972in}{0.846223in}}%
\pgfpathclose%
\pgfusepath{fill}%
\end{pgfscope}%
\begin{pgfscope}%
\pgfpathrectangle{\pgfqpoint{3.536584in}{0.147348in}}{\pgfqpoint{2.735294in}{2.735294in}}%
\pgfusepath{clip}%
\pgfsetbuttcap%
\pgfsetroundjoin%
\definecolor{currentfill}{rgb}{0.086498,0.314539,0.086498}%
\pgfsetfillcolor{currentfill}%
\pgfsetfillopacity{0.200000}%
\pgfsetlinewidth{0.000000pt}%
\definecolor{currentstroke}{rgb}{0.000000,0.000000,0.000000}%
\pgfsetstrokecolor{currentstroke}%
\pgfsetdash{}{0pt}%
\pgfpathmoveto{\pgfqpoint{4.139943in}{1.557460in}}%
\pgfpathlineto{\pgfqpoint{4.061876in}{1.760124in}}%
\pgfpathlineto{\pgfqpoint{3.977739in}{1.589726in}}%
\pgfpathlineto{\pgfqpoint{4.139943in}{1.557460in}}%
\pgfpathclose%
\pgfusepath{fill}%
\end{pgfscope}%
\begin{pgfscope}%
\pgfpathrectangle{\pgfqpoint{3.536584in}{0.147348in}}{\pgfqpoint{2.735294in}{2.735294in}}%
\pgfusepath{clip}%
\pgfsetbuttcap%
\pgfsetroundjoin%
\definecolor{currentfill}{rgb}{0.086498,0.314539,0.086498}%
\pgfsetfillcolor{currentfill}%
\pgfsetfillopacity{0.200000}%
\pgfsetlinewidth{0.000000pt}%
\definecolor{currentstroke}{rgb}{0.000000,0.000000,0.000000}%
\pgfsetstrokecolor{currentstroke}%
\pgfsetdash{}{0pt}%
\pgfpathmoveto{\pgfqpoint{5.904651in}{1.589726in}}%
\pgfpathlineto{\pgfqpoint{5.820514in}{1.760124in}}%
\pgfpathlineto{\pgfqpoint{5.742447in}{1.557460in}}%
\pgfpathlineto{\pgfqpoint{5.904651in}{1.589726in}}%
\pgfpathclose%
\pgfusepath{fill}%
\end{pgfscope}%
\begin{pgfscope}%
\pgfpathrectangle{\pgfqpoint{3.536584in}{0.147348in}}{\pgfqpoint{2.735294in}{2.735294in}}%
\pgfusepath{clip}%
\pgfsetbuttcap%
\pgfsetroundjoin%
\definecolor{currentfill}{rgb}{0.090812,0.330224,0.090812}%
\pgfsetfillcolor{currentfill}%
\pgfsetfillopacity{0.200000}%
\pgfsetlinewidth{0.000000pt}%
\definecolor{currentstroke}{rgb}{0.000000,0.000000,0.000000}%
\pgfsetstrokecolor{currentstroke}%
\pgfsetdash{}{0pt}%
\pgfpathmoveto{\pgfqpoint{4.357383in}{1.526914in}}%
\pgfpathlineto{\pgfqpoint{4.263554in}{0.949464in}}%
\pgfpathlineto{\pgfqpoint{4.314024in}{1.028737in}}%
\pgfpathlineto{\pgfqpoint{4.357383in}{1.526914in}}%
\pgfpathclose%
\pgfusepath{fill}%
\end{pgfscope}%
\begin{pgfscope}%
\pgfpathrectangle{\pgfqpoint{3.536584in}{0.147348in}}{\pgfqpoint{2.735294in}{2.735294in}}%
\pgfusepath{clip}%
\pgfsetbuttcap%
\pgfsetroundjoin%
\definecolor{currentfill}{rgb}{0.090812,0.330224,0.090812}%
\pgfsetfillcolor{currentfill}%
\pgfsetfillopacity{0.200000}%
\pgfsetlinewidth{0.000000pt}%
\definecolor{currentstroke}{rgb}{0.000000,0.000000,0.000000}%
\pgfsetstrokecolor{currentstroke}%
\pgfsetdash{}{0pt}%
\pgfpathmoveto{\pgfqpoint{5.568365in}{1.028737in}}%
\pgfpathlineto{\pgfqpoint{5.618836in}{0.949464in}}%
\pgfpathlineto{\pgfqpoint{5.525007in}{1.526914in}}%
\pgfpathlineto{\pgfqpoint{5.568365in}{1.028737in}}%
\pgfpathclose%
\pgfusepath{fill}%
\end{pgfscope}%
\begin{pgfscope}%
\pgfpathrectangle{\pgfqpoint{3.536584in}{0.147348in}}{\pgfqpoint{2.735294in}{2.735294in}}%
\pgfusepath{clip}%
\pgfsetbuttcap%
\pgfsetroundjoin%
\definecolor{currentfill}{rgb}{0.116321,0.422987,0.116321}%
\pgfsetfillcolor{currentfill}%
\pgfsetfillopacity{0.200000}%
\pgfsetlinewidth{0.000000pt}%
\definecolor{currentstroke}{rgb}{0.000000,0.000000,0.000000}%
\pgfsetstrokecolor{currentstroke}%
\pgfsetdash{}{0pt}%
\pgfpathmoveto{\pgfqpoint{4.802903in}{2.150689in}}%
\pgfpathlineto{\pgfqpoint{5.079487in}{2.150689in}}%
\pgfpathlineto{\pgfqpoint{5.037616in}{2.632943in}}%
\pgfpathlineto{\pgfqpoint{4.802903in}{2.150689in}}%
\pgfpathclose%
\pgfusepath{fill}%
\end{pgfscope}%
\begin{pgfscope}%
\pgfpathrectangle{\pgfqpoint{3.536584in}{0.147348in}}{\pgfqpoint{2.735294in}{2.735294in}}%
\pgfusepath{clip}%
\pgfsetbuttcap%
\pgfsetroundjoin%
\definecolor{currentfill}{rgb}{0.056200,0.204363,0.056200}%
\pgfsetfillcolor{currentfill}%
\pgfsetfillopacity{0.200000}%
\pgfsetlinewidth{0.000000pt}%
\definecolor{currentstroke}{rgb}{0.000000,0.000000,0.000000}%
\pgfsetstrokecolor{currentstroke}%
\pgfsetdash{}{0pt}%
\pgfpathmoveto{\pgfqpoint{4.941195in}{0.830831in}}%
\pgfpathlineto{\pgfqpoint{4.803235in}{0.943120in}}%
\pgfpathlineto{\pgfqpoint{4.690418in}{0.846223in}}%
\pgfpathlineto{\pgfqpoint{4.941195in}{0.830831in}}%
\pgfpathclose%
\pgfusepath{fill}%
\end{pgfscope}%
\begin{pgfscope}%
\pgfpathrectangle{\pgfqpoint{3.536584in}{0.147348in}}{\pgfqpoint{2.735294in}{2.735294in}}%
\pgfusepath{clip}%
\pgfsetbuttcap%
\pgfsetroundjoin%
\definecolor{currentfill}{rgb}{0.056200,0.204363,0.056200}%
\pgfsetfillcolor{currentfill}%
\pgfsetfillopacity{0.200000}%
\pgfsetlinewidth{0.000000pt}%
\definecolor{currentstroke}{rgb}{0.000000,0.000000,0.000000}%
\pgfsetstrokecolor{currentstroke}%
\pgfsetdash{}{0pt}%
\pgfpathmoveto{\pgfqpoint{5.191972in}{0.846223in}}%
\pgfpathlineto{\pgfqpoint{5.079154in}{0.943120in}}%
\pgfpathlineto{\pgfqpoint{4.941195in}{0.830831in}}%
\pgfpathlineto{\pgfqpoint{5.191972in}{0.846223in}}%
\pgfpathclose%
\pgfusepath{fill}%
\end{pgfscope}%
\begin{pgfscope}%
\pgfpathrectangle{\pgfqpoint{3.536584in}{0.147348in}}{\pgfqpoint{2.735294in}{2.735294in}}%
\pgfusepath{clip}%
\pgfsetbuttcap%
\pgfsetroundjoin%
\definecolor{currentfill}{rgb}{0.107070,0.389346,0.107070}%
\pgfsetfillcolor{currentfill}%
\pgfsetfillopacity{0.200000}%
\pgfsetlinewidth{0.000000pt}%
\definecolor{currentstroke}{rgb}{0.000000,0.000000,0.000000}%
\pgfsetstrokecolor{currentstroke}%
\pgfsetdash{}{0pt}%
\pgfpathmoveto{\pgfqpoint{4.540103in}{2.141900in}}%
\pgfpathlineto{\pgfqpoint{4.458793in}{2.292690in}}%
\pgfpathlineto{\pgfqpoint{4.312652in}{2.126617in}}%
\pgfpathlineto{\pgfqpoint{4.540103in}{2.141900in}}%
\pgfpathclose%
\pgfusepath{fill}%
\end{pgfscope}%
\begin{pgfscope}%
\pgfpathrectangle{\pgfqpoint{3.536584in}{0.147348in}}{\pgfqpoint{2.735294in}{2.735294in}}%
\pgfusepath{clip}%
\pgfsetbuttcap%
\pgfsetroundjoin%
\definecolor{currentfill}{rgb}{0.107070,0.389346,0.107070}%
\pgfsetfillcolor{currentfill}%
\pgfsetfillopacity{0.200000}%
\pgfsetlinewidth{0.000000pt}%
\definecolor{currentstroke}{rgb}{0.000000,0.000000,0.000000}%
\pgfsetstrokecolor{currentstroke}%
\pgfsetdash{}{0pt}%
\pgfpathmoveto{\pgfqpoint{5.569738in}{2.126617in}}%
\pgfpathlineto{\pgfqpoint{5.423597in}{2.292690in}}%
\pgfpathlineto{\pgfqpoint{5.342287in}{2.141900in}}%
\pgfpathlineto{\pgfqpoint{5.569738in}{2.126617in}}%
\pgfpathclose%
\pgfusepath{fill}%
\end{pgfscope}%
\begin{pgfscope}%
\pgfpathrectangle{\pgfqpoint{3.536584in}{0.147348in}}{\pgfqpoint{2.735294in}{2.735294in}}%
\pgfusepath{clip}%
\pgfsetbuttcap%
\pgfsetroundjoin%
\definecolor{currentfill}{rgb}{0.089078,0.323920,0.089078}%
\pgfsetfillcolor{currentfill}%
\pgfsetfillopacity{0.200000}%
\pgfsetlinewidth{0.000000pt}%
\definecolor{currentstroke}{rgb}{0.000000,0.000000,0.000000}%
\pgfsetstrokecolor{currentstroke}%
\pgfsetdash{}{0pt}%
\pgfpathmoveto{\pgfqpoint{4.139943in}{1.557460in}}%
\pgfpathlineto{\pgfqpoint{4.312652in}{2.126617in}}%
\pgfpathlineto{\pgfqpoint{4.061876in}{1.760124in}}%
\pgfpathlineto{\pgfqpoint{4.139943in}{1.557460in}}%
\pgfpathclose%
\pgfusepath{fill}%
\end{pgfscope}%
\begin{pgfscope}%
\pgfpathrectangle{\pgfqpoint{3.536584in}{0.147348in}}{\pgfqpoint{2.735294in}{2.735294in}}%
\pgfusepath{clip}%
\pgfsetbuttcap%
\pgfsetroundjoin%
\definecolor{currentfill}{rgb}{0.089078,0.323920,0.089078}%
\pgfsetfillcolor{currentfill}%
\pgfsetfillopacity{0.200000}%
\pgfsetlinewidth{0.000000pt}%
\definecolor{currentstroke}{rgb}{0.000000,0.000000,0.000000}%
\pgfsetstrokecolor{currentstroke}%
\pgfsetdash{}{0pt}%
\pgfpathmoveto{\pgfqpoint{5.820514in}{1.760124in}}%
\pgfpathlineto{\pgfqpoint{5.569738in}{2.126617in}}%
\pgfpathlineto{\pgfqpoint{5.742447in}{1.557460in}}%
\pgfpathlineto{\pgfqpoint{5.820514in}{1.760124in}}%
\pgfpathclose%
\pgfusepath{fill}%
\end{pgfscope}%
\begin{pgfscope}%
\pgfpathrectangle{\pgfqpoint{3.536584in}{0.147348in}}{\pgfqpoint{2.735294in}{2.735294in}}%
\pgfusepath{clip}%
\pgfsetbuttcap%
\pgfsetroundjoin%
\definecolor{currentfill}{rgb}{0.061754,0.224559,0.061754}%
\pgfsetfillcolor{currentfill}%
\pgfsetfillopacity{0.200000}%
\pgfsetlinewidth{0.000000pt}%
\definecolor{currentstroke}{rgb}{0.000000,0.000000,0.000000}%
\pgfsetstrokecolor{currentstroke}%
\pgfsetdash{}{0pt}%
\pgfpathmoveto{\pgfqpoint{4.314024in}{1.028737in}}%
\pgfpathlineto{\pgfqpoint{4.460139in}{0.888724in}}%
\pgfpathlineto{\pgfqpoint{4.494794in}{1.302943in}}%
\pgfpathlineto{\pgfqpoint{4.314024in}{1.028737in}}%
\pgfpathclose%
\pgfusepath{fill}%
\end{pgfscope}%
\begin{pgfscope}%
\pgfpathrectangle{\pgfqpoint{3.536584in}{0.147348in}}{\pgfqpoint{2.735294in}{2.735294in}}%
\pgfusepath{clip}%
\pgfsetbuttcap%
\pgfsetroundjoin%
\definecolor{currentfill}{rgb}{0.061754,0.224559,0.061754}%
\pgfsetfillcolor{currentfill}%
\pgfsetfillopacity{0.200000}%
\pgfsetlinewidth{0.000000pt}%
\definecolor{currentstroke}{rgb}{0.000000,0.000000,0.000000}%
\pgfsetstrokecolor{currentstroke}%
\pgfsetdash{}{0pt}%
\pgfpathmoveto{\pgfqpoint{5.387596in}{1.302943in}}%
\pgfpathlineto{\pgfqpoint{5.422251in}{0.888724in}}%
\pgfpathlineto{\pgfqpoint{5.568365in}{1.028737in}}%
\pgfpathlineto{\pgfqpoint{5.387596in}{1.302943in}}%
\pgfpathclose%
\pgfusepath{fill}%
\end{pgfscope}%
\begin{pgfscope}%
\pgfpathrectangle{\pgfqpoint{3.536584in}{0.147348in}}{\pgfqpoint{2.735294in}{2.735294in}}%
\pgfusepath{clip}%
\pgfsetbuttcap%
\pgfsetroundjoin%
\definecolor{currentfill}{rgb}{0.070984,0.258123,0.070984}%
\pgfsetfillcolor{currentfill}%
\pgfsetfillopacity{0.200000}%
\pgfsetlinewidth{0.000000pt}%
\definecolor{currentstroke}{rgb}{0.000000,0.000000,0.000000}%
\pgfsetstrokecolor{currentstroke}%
\pgfsetdash{}{0pt}%
\pgfpathmoveto{\pgfqpoint{4.139943in}{1.557460in}}%
\pgfpathlineto{\pgfqpoint{4.130959in}{1.094945in}}%
\pgfpathlineto{\pgfqpoint{4.357383in}{1.526914in}}%
\pgfpathlineto{\pgfqpoint{4.139943in}{1.557460in}}%
\pgfpathclose%
\pgfusepath{fill}%
\end{pgfscope}%
\begin{pgfscope}%
\pgfpathrectangle{\pgfqpoint{3.536584in}{0.147348in}}{\pgfqpoint{2.735294in}{2.735294in}}%
\pgfusepath{clip}%
\pgfsetbuttcap%
\pgfsetroundjoin%
\definecolor{currentfill}{rgb}{0.070984,0.258123,0.070984}%
\pgfsetfillcolor{currentfill}%
\pgfsetfillopacity{0.200000}%
\pgfsetlinewidth{0.000000pt}%
\definecolor{currentstroke}{rgb}{0.000000,0.000000,0.000000}%
\pgfsetstrokecolor{currentstroke}%
\pgfsetdash{}{0pt}%
\pgfpathmoveto{\pgfqpoint{5.525007in}{1.526914in}}%
\pgfpathlineto{\pgfqpoint{5.751431in}{1.094945in}}%
\pgfpathlineto{\pgfqpoint{5.742447in}{1.557460in}}%
\pgfpathlineto{\pgfqpoint{5.525007in}{1.526914in}}%
\pgfpathclose%
\pgfusepath{fill}%
\end{pgfscope}%
\begin{pgfscope}%
\pgfpathrectangle{\pgfqpoint{3.536584in}{0.147348in}}{\pgfqpoint{2.735294in}{2.735294in}}%
\pgfusepath{clip}%
\pgfsetbuttcap%
\pgfsetroundjoin%
\definecolor{currentfill}{rgb}{0.070885,0.257762,0.070885}%
\pgfsetfillcolor{currentfill}%
\pgfsetfillopacity{0.200000}%
\pgfsetlinewidth{0.000000pt}%
\definecolor{currentstroke}{rgb}{0.000000,0.000000,0.000000}%
\pgfsetstrokecolor{currentstroke}%
\pgfsetdash{}{0pt}%
\pgfpathmoveto{\pgfqpoint{5.341356in}{0.974376in}}%
\pgfpathlineto{\pgfqpoint{5.422251in}{0.888724in}}%
\pgfpathlineto{\pgfqpoint{5.387596in}{1.302943in}}%
\pgfpathlineto{\pgfqpoint{5.341356in}{0.974376in}}%
\pgfpathclose%
\pgfusepath{fill}%
\end{pgfscope}%
\begin{pgfscope}%
\pgfpathrectangle{\pgfqpoint{3.536584in}{0.147348in}}{\pgfqpoint{2.735294in}{2.735294in}}%
\pgfusepath{clip}%
\pgfsetbuttcap%
\pgfsetroundjoin%
\definecolor{currentfill}{rgb}{0.070885,0.257762,0.070885}%
\pgfsetfillcolor{currentfill}%
\pgfsetfillopacity{0.200000}%
\pgfsetlinewidth{0.000000pt}%
\definecolor{currentstroke}{rgb}{0.000000,0.000000,0.000000}%
\pgfsetstrokecolor{currentstroke}%
\pgfsetdash{}{0pt}%
\pgfpathmoveto{\pgfqpoint{4.494794in}{1.302943in}}%
\pgfpathlineto{\pgfqpoint{4.460139in}{0.888724in}}%
\pgfpathlineto{\pgfqpoint{4.541034in}{0.974376in}}%
\pgfpathlineto{\pgfqpoint{4.494794in}{1.302943in}}%
\pgfpathclose%
\pgfusepath{fill}%
\end{pgfscope}%
\begin{pgfscope}%
\pgfpathrectangle{\pgfqpoint{3.536584in}{0.147348in}}{\pgfqpoint{2.735294in}{2.735294in}}%
\pgfusepath{clip}%
\pgfsetbuttcap%
\pgfsetroundjoin%
\definecolor{currentfill}{rgb}{0.111651,0.406004,0.111651}%
\pgfsetfillcolor{currentfill}%
\pgfsetfillopacity{0.200000}%
\pgfsetlinewidth{0.000000pt}%
\definecolor{currentstroke}{rgb}{0.000000,0.000000,0.000000}%
\pgfsetstrokecolor{currentstroke}%
\pgfsetdash{}{0pt}%
\pgfpathmoveto{\pgfqpoint{5.342287in}{2.141900in}}%
\pgfpathlineto{\pgfqpoint{5.423597in}{2.292690in}}%
\pgfpathlineto{\pgfqpoint{5.079487in}{2.150689in}}%
\pgfpathlineto{\pgfqpoint{5.342287in}{2.141900in}}%
\pgfpathclose%
\pgfusepath{fill}%
\end{pgfscope}%
\begin{pgfscope}%
\pgfpathrectangle{\pgfqpoint{3.536584in}{0.147348in}}{\pgfqpoint{2.735294in}{2.735294in}}%
\pgfusepath{clip}%
\pgfsetbuttcap%
\pgfsetroundjoin%
\definecolor{currentfill}{rgb}{0.111651,0.406004,0.111651}%
\pgfsetfillcolor{currentfill}%
\pgfsetfillopacity{0.200000}%
\pgfsetlinewidth{0.000000pt}%
\definecolor{currentstroke}{rgb}{0.000000,0.000000,0.000000}%
\pgfsetstrokecolor{currentstroke}%
\pgfsetdash{}{0pt}%
\pgfpathmoveto{\pgfqpoint{4.802903in}{2.150689in}}%
\pgfpathlineto{\pgfqpoint{4.458793in}{2.292690in}}%
\pgfpathlineto{\pgfqpoint{4.540103in}{2.141900in}}%
\pgfpathlineto{\pgfqpoint{4.802903in}{2.150689in}}%
\pgfpathclose%
\pgfusepath{fill}%
\end{pgfscope}%
\begin{pgfscope}%
\pgfpathrectangle{\pgfqpoint{3.536584in}{0.147348in}}{\pgfqpoint{2.735294in}{2.735294in}}%
\pgfusepath{clip}%
\pgfsetbuttcap%
\pgfsetroundjoin%
\definecolor{currentfill}{rgb}{0.092193,0.335248,0.092193}%
\pgfsetfillcolor{currentfill}%
\pgfsetfillopacity{0.200000}%
\pgfsetlinewidth{0.000000pt}%
\definecolor{currentstroke}{rgb}{0.000000,0.000000,0.000000}%
\pgfsetstrokecolor{currentstroke}%
\pgfsetdash{}{0pt}%
\pgfpathmoveto{\pgfqpoint{4.357383in}{1.526914in}}%
\pgfpathlineto{\pgfqpoint{4.312652in}{2.126617in}}%
\pgfpathlineto{\pgfqpoint{4.139943in}{1.557460in}}%
\pgfpathlineto{\pgfqpoint{4.357383in}{1.526914in}}%
\pgfpathclose%
\pgfusepath{fill}%
\end{pgfscope}%
\begin{pgfscope}%
\pgfpathrectangle{\pgfqpoint{3.536584in}{0.147348in}}{\pgfqpoint{2.735294in}{2.735294in}}%
\pgfusepath{clip}%
\pgfsetbuttcap%
\pgfsetroundjoin%
\definecolor{currentfill}{rgb}{0.092193,0.335248,0.092193}%
\pgfsetfillcolor{currentfill}%
\pgfsetfillopacity{0.200000}%
\pgfsetlinewidth{0.000000pt}%
\definecolor{currentstroke}{rgb}{0.000000,0.000000,0.000000}%
\pgfsetstrokecolor{currentstroke}%
\pgfsetdash{}{0pt}%
\pgfpathmoveto{\pgfqpoint{5.742447in}{1.557460in}}%
\pgfpathlineto{\pgfqpoint{5.569738in}{2.126617in}}%
\pgfpathlineto{\pgfqpoint{5.525007in}{1.526914in}}%
\pgfpathlineto{\pgfqpoint{5.742447in}{1.557460in}}%
\pgfpathclose%
\pgfusepath{fill}%
\end{pgfscope}%
\begin{pgfscope}%
\pgfpathrectangle{\pgfqpoint{3.536584in}{0.147348in}}{\pgfqpoint{2.735294in}{2.735294in}}%
\pgfusepath{clip}%
\pgfsetbuttcap%
\pgfsetroundjoin%
\definecolor{currentfill}{rgb}{0.060562,0.220227,0.060562}%
\pgfsetfillcolor{currentfill}%
\pgfsetfillopacity{0.200000}%
\pgfsetlinewidth{0.000000pt}%
\definecolor{currentstroke}{rgb}{0.000000,0.000000,0.000000}%
\pgfsetstrokecolor{currentstroke}%
\pgfsetdash{}{0pt}%
\pgfpathmoveto{\pgfqpoint{5.341356in}{0.974376in}}%
\pgfpathlineto{\pgfqpoint{5.096361in}{1.277996in}}%
\pgfpathlineto{\pgfqpoint{5.191972in}{0.846223in}}%
\pgfpathlineto{\pgfqpoint{5.341356in}{0.974376in}}%
\pgfpathclose%
\pgfusepath{fill}%
\end{pgfscope}%
\begin{pgfscope}%
\pgfpathrectangle{\pgfqpoint{3.536584in}{0.147348in}}{\pgfqpoint{2.735294in}{2.735294in}}%
\pgfusepath{clip}%
\pgfsetbuttcap%
\pgfsetroundjoin%
\definecolor{currentfill}{rgb}{0.060562,0.220227,0.060562}%
\pgfsetfillcolor{currentfill}%
\pgfsetfillopacity{0.200000}%
\pgfsetlinewidth{0.000000pt}%
\definecolor{currentstroke}{rgb}{0.000000,0.000000,0.000000}%
\pgfsetstrokecolor{currentstroke}%
\pgfsetdash{}{0pt}%
\pgfpathmoveto{\pgfqpoint{4.690418in}{0.846223in}}%
\pgfpathlineto{\pgfqpoint{4.786029in}{1.277996in}}%
\pgfpathlineto{\pgfqpoint{4.541034in}{0.974376in}}%
\pgfpathlineto{\pgfqpoint{4.690418in}{0.846223in}}%
\pgfpathclose%
\pgfusepath{fill}%
\end{pgfscope}%
\begin{pgfscope}%
\pgfpathrectangle{\pgfqpoint{3.536584in}{0.147348in}}{\pgfqpoint{2.735294in}{2.735294in}}%
\pgfusepath{clip}%
\pgfsetbuttcap%
\pgfsetroundjoin%
\definecolor{currentfill}{rgb}{0.097285,0.353762,0.097285}%
\pgfsetfillcolor{currentfill}%
\pgfsetfillopacity{0.200000}%
\pgfsetlinewidth{0.000000pt}%
\definecolor{currentstroke}{rgb}{0.000000,0.000000,0.000000}%
\pgfsetstrokecolor{currentstroke}%
\pgfsetdash{}{0pt}%
\pgfpathmoveto{\pgfqpoint{4.540103in}{2.141900in}}%
\pgfpathlineto{\pgfqpoint{4.312652in}{2.126617in}}%
\pgfpathlineto{\pgfqpoint{4.649042in}{1.952591in}}%
\pgfpathlineto{\pgfqpoint{4.540103in}{2.141900in}}%
\pgfpathclose%
\pgfusepath{fill}%
\end{pgfscope}%
\begin{pgfscope}%
\pgfpathrectangle{\pgfqpoint{3.536584in}{0.147348in}}{\pgfqpoint{2.735294in}{2.735294in}}%
\pgfusepath{clip}%
\pgfsetbuttcap%
\pgfsetroundjoin%
\definecolor{currentfill}{rgb}{0.097285,0.353762,0.097285}%
\pgfsetfillcolor{currentfill}%
\pgfsetfillopacity{0.200000}%
\pgfsetlinewidth{0.000000pt}%
\definecolor{currentstroke}{rgb}{0.000000,0.000000,0.000000}%
\pgfsetstrokecolor{currentstroke}%
\pgfsetdash{}{0pt}%
\pgfpathmoveto{\pgfqpoint{5.233348in}{1.952591in}}%
\pgfpathlineto{\pgfqpoint{5.569738in}{2.126617in}}%
\pgfpathlineto{\pgfqpoint{5.342287in}{2.141900in}}%
\pgfpathlineto{\pgfqpoint{5.233348in}{1.952591in}}%
\pgfpathclose%
\pgfusepath{fill}%
\end{pgfscope}%
\begin{pgfscope}%
\pgfpathrectangle{\pgfqpoint{3.536584in}{0.147348in}}{\pgfqpoint{2.735294in}{2.735294in}}%
\pgfusepath{clip}%
\pgfsetbuttcap%
\pgfsetroundjoin%
\definecolor{currentfill}{rgb}{0.067497,0.245443,0.067497}%
\pgfsetfillcolor{currentfill}%
\pgfsetfillopacity{0.200000}%
\pgfsetlinewidth{0.000000pt}%
\definecolor{currentstroke}{rgb}{0.000000,0.000000,0.000000}%
\pgfsetstrokecolor{currentstroke}%
\pgfsetdash{}{0pt}%
\pgfpathmoveto{\pgfqpoint{4.690418in}{0.846223in}}%
\pgfpathlineto{\pgfqpoint{4.803235in}{0.943120in}}%
\pgfpathlineto{\pgfqpoint{4.786029in}{1.277996in}}%
\pgfpathlineto{\pgfqpoint{4.690418in}{0.846223in}}%
\pgfpathclose%
\pgfusepath{fill}%
\end{pgfscope}%
\begin{pgfscope}%
\pgfpathrectangle{\pgfqpoint{3.536584in}{0.147348in}}{\pgfqpoint{2.735294in}{2.735294in}}%
\pgfusepath{clip}%
\pgfsetbuttcap%
\pgfsetroundjoin%
\definecolor{currentfill}{rgb}{0.067497,0.245443,0.067497}%
\pgfsetfillcolor{currentfill}%
\pgfsetfillopacity{0.200000}%
\pgfsetlinewidth{0.000000pt}%
\definecolor{currentstroke}{rgb}{0.000000,0.000000,0.000000}%
\pgfsetstrokecolor{currentstroke}%
\pgfsetdash{}{0pt}%
\pgfpathmoveto{\pgfqpoint{5.096361in}{1.277996in}}%
\pgfpathlineto{\pgfqpoint{5.079154in}{0.943120in}}%
\pgfpathlineto{\pgfqpoint{5.191972in}{0.846223in}}%
\pgfpathlineto{\pgfqpoint{5.096361in}{1.277996in}}%
\pgfpathclose%
\pgfusepath{fill}%
\end{pgfscope}%
\begin{pgfscope}%
\pgfpathrectangle{\pgfqpoint{3.536584in}{0.147348in}}{\pgfqpoint{2.735294in}{2.735294in}}%
\pgfusepath{clip}%
\pgfsetbuttcap%
\pgfsetroundjoin%
\definecolor{currentfill}{rgb}{0.065434,0.237940,0.065434}%
\pgfsetfillcolor{currentfill}%
\pgfsetfillopacity{0.200000}%
\pgfsetlinewidth{0.000000pt}%
\definecolor{currentstroke}{rgb}{0.000000,0.000000,0.000000}%
\pgfsetstrokecolor{currentstroke}%
\pgfsetdash{}{0pt}%
\pgfpathmoveto{\pgfqpoint{4.941195in}{0.830831in}}%
\pgfpathlineto{\pgfqpoint{4.786029in}{1.277996in}}%
\pgfpathlineto{\pgfqpoint{4.803235in}{0.943120in}}%
\pgfpathlineto{\pgfqpoint{4.941195in}{0.830831in}}%
\pgfpathclose%
\pgfusepath{fill}%
\end{pgfscope}%
\begin{pgfscope}%
\pgfpathrectangle{\pgfqpoint{3.536584in}{0.147348in}}{\pgfqpoint{2.735294in}{2.735294in}}%
\pgfusepath{clip}%
\pgfsetbuttcap%
\pgfsetroundjoin%
\definecolor{currentfill}{rgb}{0.065434,0.237940,0.065434}%
\pgfsetfillcolor{currentfill}%
\pgfsetfillopacity{0.200000}%
\pgfsetlinewidth{0.000000pt}%
\definecolor{currentstroke}{rgb}{0.000000,0.000000,0.000000}%
\pgfsetstrokecolor{currentstroke}%
\pgfsetdash{}{0pt}%
\pgfpathmoveto{\pgfqpoint{5.079154in}{0.943120in}}%
\pgfpathlineto{\pgfqpoint{5.096361in}{1.277996in}}%
\pgfpathlineto{\pgfqpoint{4.941195in}{0.830831in}}%
\pgfpathlineto{\pgfqpoint{5.079154in}{0.943120in}}%
\pgfpathclose%
\pgfusepath{fill}%
\end{pgfscope}%
\begin{pgfscope}%
\pgfpathrectangle{\pgfqpoint{3.536584in}{0.147348in}}{\pgfqpoint{2.735294in}{2.735294in}}%
\pgfusepath{clip}%
\pgfsetbuttcap%
\pgfsetroundjoin%
\definecolor{currentfill}{rgb}{0.073593,0.267612,0.073593}%
\pgfsetfillcolor{currentfill}%
\pgfsetfillopacity{0.200000}%
\pgfsetlinewidth{0.000000pt}%
\definecolor{currentstroke}{rgb}{0.000000,0.000000,0.000000}%
\pgfsetstrokecolor{currentstroke}%
\pgfsetdash{}{0pt}%
\pgfpathmoveto{\pgfqpoint{4.314024in}{1.028737in}}%
\pgfpathlineto{\pgfqpoint{4.494794in}{1.302943in}}%
\pgfpathlineto{\pgfqpoint{4.357383in}{1.526914in}}%
\pgfpathlineto{\pgfqpoint{4.314024in}{1.028737in}}%
\pgfpathclose%
\pgfusepath{fill}%
\end{pgfscope}%
\begin{pgfscope}%
\pgfpathrectangle{\pgfqpoint{3.536584in}{0.147348in}}{\pgfqpoint{2.735294in}{2.735294in}}%
\pgfusepath{clip}%
\pgfsetbuttcap%
\pgfsetroundjoin%
\definecolor{currentfill}{rgb}{0.073593,0.267612,0.073593}%
\pgfsetfillcolor{currentfill}%
\pgfsetfillopacity{0.200000}%
\pgfsetlinewidth{0.000000pt}%
\definecolor{currentstroke}{rgb}{0.000000,0.000000,0.000000}%
\pgfsetstrokecolor{currentstroke}%
\pgfsetdash{}{0pt}%
\pgfpathmoveto{\pgfqpoint{5.525007in}{1.526914in}}%
\pgfpathlineto{\pgfqpoint{5.387596in}{1.302943in}}%
\pgfpathlineto{\pgfqpoint{5.568365in}{1.028737in}}%
\pgfpathlineto{\pgfqpoint{5.525007in}{1.526914in}}%
\pgfpathclose%
\pgfusepath{fill}%
\end{pgfscope}%
\begin{pgfscope}%
\pgfpathrectangle{\pgfqpoint{3.536584in}{0.147348in}}{\pgfqpoint{2.735294in}{2.735294in}}%
\pgfusepath{clip}%
\pgfsetbuttcap%
\pgfsetroundjoin%
\definecolor{currentfill}{rgb}{0.091915,0.334238,0.091915}%
\pgfsetfillcolor{currentfill}%
\pgfsetfillopacity{0.200000}%
\pgfsetlinewidth{0.000000pt}%
\definecolor{currentstroke}{rgb}{0.000000,0.000000,0.000000}%
\pgfsetstrokecolor{currentstroke}%
\pgfsetdash{}{0pt}%
\pgfpathmoveto{\pgfqpoint{4.357383in}{1.526914in}}%
\pgfpathlineto{\pgfqpoint{4.649042in}{1.952591in}}%
\pgfpathlineto{\pgfqpoint{4.312652in}{2.126617in}}%
\pgfpathlineto{\pgfqpoint{4.357383in}{1.526914in}}%
\pgfpathclose%
\pgfusepath{fill}%
\end{pgfscope}%
\begin{pgfscope}%
\pgfpathrectangle{\pgfqpoint{3.536584in}{0.147348in}}{\pgfqpoint{2.735294in}{2.735294in}}%
\pgfusepath{clip}%
\pgfsetbuttcap%
\pgfsetroundjoin%
\definecolor{currentfill}{rgb}{0.091915,0.334238,0.091915}%
\pgfsetfillcolor{currentfill}%
\pgfsetfillopacity{0.200000}%
\pgfsetlinewidth{0.000000pt}%
\definecolor{currentstroke}{rgb}{0.000000,0.000000,0.000000}%
\pgfsetstrokecolor{currentstroke}%
\pgfsetdash{}{0pt}%
\pgfpathmoveto{\pgfqpoint{5.569738in}{2.126617in}}%
\pgfpathlineto{\pgfqpoint{5.233348in}{1.952591in}}%
\pgfpathlineto{\pgfqpoint{5.525007in}{1.526914in}}%
\pgfpathlineto{\pgfqpoint{5.569738in}{2.126617in}}%
\pgfpathclose%
\pgfusepath{fill}%
\end{pgfscope}%
\begin{pgfscope}%
\pgfpathrectangle{\pgfqpoint{3.536584in}{0.147348in}}{\pgfqpoint{2.735294in}{2.735294in}}%
\pgfusepath{clip}%
\pgfsetbuttcap%
\pgfsetroundjoin%
\definecolor{currentfill}{rgb}{0.101759,0.370033,0.101759}%
\pgfsetfillcolor{currentfill}%
\pgfsetfillopacity{0.200000}%
\pgfsetlinewidth{0.000000pt}%
\definecolor{currentstroke}{rgb}{0.000000,0.000000,0.000000}%
\pgfsetstrokecolor{currentstroke}%
\pgfsetdash{}{0pt}%
\pgfpathmoveto{\pgfqpoint{4.649042in}{1.952591in}}%
\pgfpathlineto{\pgfqpoint{4.802903in}{2.150689in}}%
\pgfpathlineto{\pgfqpoint{4.540103in}{2.141900in}}%
\pgfpathlineto{\pgfqpoint{4.649042in}{1.952591in}}%
\pgfpathclose%
\pgfusepath{fill}%
\end{pgfscope}%
\begin{pgfscope}%
\pgfpathrectangle{\pgfqpoint{3.536584in}{0.147348in}}{\pgfqpoint{2.735294in}{2.735294in}}%
\pgfusepath{clip}%
\pgfsetbuttcap%
\pgfsetroundjoin%
\definecolor{currentfill}{rgb}{0.101759,0.370033,0.101759}%
\pgfsetfillcolor{currentfill}%
\pgfsetfillopacity{0.200000}%
\pgfsetlinewidth{0.000000pt}%
\definecolor{currentstroke}{rgb}{0.000000,0.000000,0.000000}%
\pgfsetstrokecolor{currentstroke}%
\pgfsetdash{}{0pt}%
\pgfpathmoveto{\pgfqpoint{5.342287in}{2.141900in}}%
\pgfpathlineto{\pgfqpoint{5.079487in}{2.150689in}}%
\pgfpathlineto{\pgfqpoint{5.233348in}{1.952591in}}%
\pgfpathlineto{\pgfqpoint{5.342287in}{2.141900in}}%
\pgfpathclose%
\pgfusepath{fill}%
\end{pgfscope}%
\begin{pgfscope}%
\pgfpathrectangle{\pgfqpoint{3.536584in}{0.147348in}}{\pgfqpoint{2.735294in}{2.735294in}}%
\pgfusepath{clip}%
\pgfsetbuttcap%
\pgfsetroundjoin%
\definecolor{currentfill}{rgb}{0.101677,0.369734,0.101677}%
\pgfsetfillcolor{currentfill}%
\pgfsetfillopacity{0.200000}%
\pgfsetlinewidth{0.000000pt}%
\definecolor{currentstroke}{rgb}{0.000000,0.000000,0.000000}%
\pgfsetstrokecolor{currentstroke}%
\pgfsetdash{}{0pt}%
\pgfpathmoveto{\pgfqpoint{4.941195in}{1.953918in}}%
\pgfpathlineto{\pgfqpoint{5.079487in}{2.150689in}}%
\pgfpathlineto{\pgfqpoint{4.802903in}{2.150689in}}%
\pgfpathlineto{\pgfqpoint{4.941195in}{1.953918in}}%
\pgfpathclose%
\pgfusepath{fill}%
\end{pgfscope}%
\begin{pgfscope}%
\pgfpathrectangle{\pgfqpoint{3.536584in}{0.147348in}}{\pgfqpoint{2.735294in}{2.735294in}}%
\pgfusepath{clip}%
\pgfsetbuttcap%
\pgfsetroundjoin%
\definecolor{currentfill}{rgb}{0.065035,0.236492,0.065035}%
\pgfsetfillcolor{currentfill}%
\pgfsetfillopacity{0.200000}%
\pgfsetlinewidth{0.000000pt}%
\definecolor{currentstroke}{rgb}{0.000000,0.000000,0.000000}%
\pgfsetstrokecolor{currentstroke}%
\pgfsetdash{}{0pt}%
\pgfpathmoveto{\pgfqpoint{4.541034in}{0.974376in}}%
\pgfpathlineto{\pgfqpoint{4.630724in}{1.504068in}}%
\pgfpathlineto{\pgfqpoint{4.494794in}{1.302943in}}%
\pgfpathlineto{\pgfqpoint{4.541034in}{0.974376in}}%
\pgfpathclose%
\pgfusepath{fill}%
\end{pgfscope}%
\begin{pgfscope}%
\pgfpathrectangle{\pgfqpoint{3.536584in}{0.147348in}}{\pgfqpoint{2.735294in}{2.735294in}}%
\pgfusepath{clip}%
\pgfsetbuttcap%
\pgfsetroundjoin%
\definecolor{currentfill}{rgb}{0.065035,0.236492,0.065035}%
\pgfsetfillcolor{currentfill}%
\pgfsetfillopacity{0.200000}%
\pgfsetlinewidth{0.000000pt}%
\definecolor{currentstroke}{rgb}{0.000000,0.000000,0.000000}%
\pgfsetstrokecolor{currentstroke}%
\pgfsetdash{}{0pt}%
\pgfpathmoveto{\pgfqpoint{5.387596in}{1.302943in}}%
\pgfpathlineto{\pgfqpoint{5.251666in}{1.504068in}}%
\pgfpathlineto{\pgfqpoint{5.341356in}{0.974376in}}%
\pgfpathlineto{\pgfqpoint{5.387596in}{1.302943in}}%
\pgfpathclose%
\pgfusepath{fill}%
\end{pgfscope}%
\begin{pgfscope}%
\pgfpathrectangle{\pgfqpoint{3.536584in}{0.147348in}}{\pgfqpoint{2.735294in}{2.735294in}}%
\pgfusepath{clip}%
\pgfsetbuttcap%
\pgfsetroundjoin%
\definecolor{currentfill}{rgb}{0.066446,0.241622,0.066446}%
\pgfsetfillcolor{currentfill}%
\pgfsetfillopacity{0.200000}%
\pgfsetlinewidth{0.000000pt}%
\definecolor{currentstroke}{rgb}{0.000000,0.000000,0.000000}%
\pgfsetstrokecolor{currentstroke}%
\pgfsetdash{}{0pt}%
\pgfpathmoveto{\pgfqpoint{4.786029in}{1.277996in}}%
\pgfpathlineto{\pgfqpoint{4.941195in}{0.830831in}}%
\pgfpathlineto{\pgfqpoint{4.941195in}{1.495457in}}%
\pgfpathlineto{\pgfqpoint{4.786029in}{1.277996in}}%
\pgfpathclose%
\pgfusepath{fill}%
\end{pgfscope}%
\begin{pgfscope}%
\pgfpathrectangle{\pgfqpoint{3.536584in}{0.147348in}}{\pgfqpoint{2.735294in}{2.735294in}}%
\pgfusepath{clip}%
\pgfsetbuttcap%
\pgfsetroundjoin%
\definecolor{currentfill}{rgb}{0.066446,0.241622,0.066446}%
\pgfsetfillcolor{currentfill}%
\pgfsetfillopacity{0.200000}%
\pgfsetlinewidth{0.000000pt}%
\definecolor{currentstroke}{rgb}{0.000000,0.000000,0.000000}%
\pgfsetstrokecolor{currentstroke}%
\pgfsetdash{}{0pt}%
\pgfpathmoveto{\pgfqpoint{4.941195in}{1.495457in}}%
\pgfpathlineto{\pgfqpoint{4.941195in}{0.830831in}}%
\pgfpathlineto{\pgfqpoint{5.096361in}{1.277996in}}%
\pgfpathlineto{\pgfqpoint{4.941195in}{1.495457in}}%
\pgfpathclose%
\pgfusepath{fill}%
\end{pgfscope}%
\begin{pgfscope}%
\pgfpathrectangle{\pgfqpoint{3.536584in}{0.147348in}}{\pgfqpoint{2.735294in}{2.735294in}}%
\pgfusepath{clip}%
\pgfsetbuttcap%
\pgfsetroundjoin%
\definecolor{currentfill}{rgb}{0.098306,0.357475,0.098306}%
\pgfsetfillcolor{currentfill}%
\pgfsetfillopacity{0.200000}%
\pgfsetlinewidth{0.000000pt}%
\definecolor{currentstroke}{rgb}{0.000000,0.000000,0.000000}%
\pgfsetstrokecolor{currentstroke}%
\pgfsetdash{}{0pt}%
\pgfpathmoveto{\pgfqpoint{4.941195in}{1.953918in}}%
\pgfpathlineto{\pgfqpoint{4.802903in}{2.150689in}}%
\pgfpathlineto{\pgfqpoint{4.649042in}{1.952591in}}%
\pgfpathlineto{\pgfqpoint{4.941195in}{1.953918in}}%
\pgfpathclose%
\pgfusepath{fill}%
\end{pgfscope}%
\begin{pgfscope}%
\pgfpathrectangle{\pgfqpoint{3.536584in}{0.147348in}}{\pgfqpoint{2.735294in}{2.735294in}}%
\pgfusepath{clip}%
\pgfsetbuttcap%
\pgfsetroundjoin%
\definecolor{currentfill}{rgb}{0.098306,0.357475,0.098306}%
\pgfsetfillcolor{currentfill}%
\pgfsetfillopacity{0.200000}%
\pgfsetlinewidth{0.000000pt}%
\definecolor{currentstroke}{rgb}{0.000000,0.000000,0.000000}%
\pgfsetstrokecolor{currentstroke}%
\pgfsetdash{}{0pt}%
\pgfpathmoveto{\pgfqpoint{5.233348in}{1.952591in}}%
\pgfpathlineto{\pgfqpoint{5.079487in}{2.150689in}}%
\pgfpathlineto{\pgfqpoint{4.941195in}{1.953918in}}%
\pgfpathlineto{\pgfqpoint{5.233348in}{1.952591in}}%
\pgfpathclose%
\pgfusepath{fill}%
\end{pgfscope}%
\begin{pgfscope}%
\pgfpathrectangle{\pgfqpoint{3.536584in}{0.147348in}}{\pgfqpoint{2.735294in}{2.735294in}}%
\pgfusepath{clip}%
\pgfsetbuttcap%
\pgfsetroundjoin%
\definecolor{currentfill}{rgb}{0.070209,0.255305,0.070209}%
\pgfsetfillcolor{currentfill}%
\pgfsetfillopacity{0.200000}%
\pgfsetlinewidth{0.000000pt}%
\definecolor{currentstroke}{rgb}{0.000000,0.000000,0.000000}%
\pgfsetstrokecolor{currentstroke}%
\pgfsetdash{}{0pt}%
\pgfpathmoveto{\pgfqpoint{4.541034in}{0.974376in}}%
\pgfpathlineto{\pgfqpoint{4.786029in}{1.277996in}}%
\pgfpathlineto{\pgfqpoint{4.630724in}{1.504068in}}%
\pgfpathlineto{\pgfqpoint{4.541034in}{0.974376in}}%
\pgfpathclose%
\pgfusepath{fill}%
\end{pgfscope}%
\begin{pgfscope}%
\pgfpathrectangle{\pgfqpoint{3.536584in}{0.147348in}}{\pgfqpoint{2.735294in}{2.735294in}}%
\pgfusepath{clip}%
\pgfsetbuttcap%
\pgfsetroundjoin%
\definecolor{currentfill}{rgb}{0.070209,0.255305,0.070209}%
\pgfsetfillcolor{currentfill}%
\pgfsetfillopacity{0.200000}%
\pgfsetlinewidth{0.000000pt}%
\definecolor{currentstroke}{rgb}{0.000000,0.000000,0.000000}%
\pgfsetstrokecolor{currentstroke}%
\pgfsetdash{}{0pt}%
\pgfpathmoveto{\pgfqpoint{5.251666in}{1.504068in}}%
\pgfpathlineto{\pgfqpoint{5.096361in}{1.277996in}}%
\pgfpathlineto{\pgfqpoint{5.341356in}{0.974376in}}%
\pgfpathlineto{\pgfqpoint{5.251666in}{1.504068in}}%
\pgfpathclose%
\pgfusepath{fill}%
\end{pgfscope}%
\begin{pgfscope}%
\pgfpathrectangle{\pgfqpoint{3.536584in}{0.147348in}}{\pgfqpoint{2.735294in}{2.735294in}}%
\pgfusepath{clip}%
\pgfsetbuttcap%
\pgfsetroundjoin%
\definecolor{currentfill}{rgb}{0.087398,0.317812,0.087398}%
\pgfsetfillcolor{currentfill}%
\pgfsetfillopacity{0.200000}%
\pgfsetlinewidth{0.000000pt}%
\definecolor{currentstroke}{rgb}{0.000000,0.000000,0.000000}%
\pgfsetstrokecolor{currentstroke}%
\pgfsetdash{}{0pt}%
\pgfpathmoveto{\pgfqpoint{5.525007in}{1.526914in}}%
\pgfpathlineto{\pgfqpoint{5.233348in}{1.952591in}}%
\pgfpathlineto{\pgfqpoint{5.251666in}{1.504068in}}%
\pgfpathlineto{\pgfqpoint{5.525007in}{1.526914in}}%
\pgfpathclose%
\pgfusepath{fill}%
\end{pgfscope}%
\begin{pgfscope}%
\pgfpathrectangle{\pgfqpoint{3.536584in}{0.147348in}}{\pgfqpoint{2.735294in}{2.735294in}}%
\pgfusepath{clip}%
\pgfsetbuttcap%
\pgfsetroundjoin%
\definecolor{currentfill}{rgb}{0.087398,0.317812,0.087398}%
\pgfsetfillcolor{currentfill}%
\pgfsetfillopacity{0.200000}%
\pgfsetlinewidth{0.000000pt}%
\definecolor{currentstroke}{rgb}{0.000000,0.000000,0.000000}%
\pgfsetstrokecolor{currentstroke}%
\pgfsetdash{}{0pt}%
\pgfpathmoveto{\pgfqpoint{4.630724in}{1.504068in}}%
\pgfpathlineto{\pgfqpoint{4.649042in}{1.952591in}}%
\pgfpathlineto{\pgfqpoint{4.357383in}{1.526914in}}%
\pgfpathlineto{\pgfqpoint{4.630724in}{1.504068in}}%
\pgfpathclose%
\pgfusepath{fill}%
\end{pgfscope}%
\begin{pgfscope}%
\pgfpathrectangle{\pgfqpoint{3.536584in}{0.147348in}}{\pgfqpoint{2.735294in}{2.735294in}}%
\pgfusepath{clip}%
\pgfsetbuttcap%
\pgfsetroundjoin%
\definecolor{currentfill}{rgb}{0.075994,0.276341,0.075994}%
\pgfsetfillcolor{currentfill}%
\pgfsetfillopacity{0.200000}%
\pgfsetlinewidth{0.000000pt}%
\definecolor{currentstroke}{rgb}{0.000000,0.000000,0.000000}%
\pgfsetstrokecolor{currentstroke}%
\pgfsetdash{}{0pt}%
\pgfpathmoveto{\pgfqpoint{4.357383in}{1.526914in}}%
\pgfpathlineto{\pgfqpoint{4.494794in}{1.302943in}}%
\pgfpathlineto{\pgfqpoint{4.630724in}{1.504068in}}%
\pgfpathlineto{\pgfqpoint{4.357383in}{1.526914in}}%
\pgfpathclose%
\pgfusepath{fill}%
\end{pgfscope}%
\begin{pgfscope}%
\pgfpathrectangle{\pgfqpoint{3.536584in}{0.147348in}}{\pgfqpoint{2.735294in}{2.735294in}}%
\pgfusepath{clip}%
\pgfsetbuttcap%
\pgfsetroundjoin%
\definecolor{currentfill}{rgb}{0.075994,0.276341,0.075994}%
\pgfsetfillcolor{currentfill}%
\pgfsetfillopacity{0.200000}%
\pgfsetlinewidth{0.000000pt}%
\definecolor{currentstroke}{rgb}{0.000000,0.000000,0.000000}%
\pgfsetstrokecolor{currentstroke}%
\pgfsetdash{}{0pt}%
\pgfpathmoveto{\pgfqpoint{5.251666in}{1.504068in}}%
\pgfpathlineto{\pgfqpoint{5.387596in}{1.302943in}}%
\pgfpathlineto{\pgfqpoint{5.525007in}{1.526914in}}%
\pgfpathlineto{\pgfqpoint{5.251666in}{1.504068in}}%
\pgfpathclose%
\pgfusepath{fill}%
\end{pgfscope}%
\begin{pgfscope}%
\pgfpathrectangle{\pgfqpoint{3.536584in}{0.147348in}}{\pgfqpoint{2.735294in}{2.735294in}}%
\pgfusepath{clip}%
\pgfsetbuttcap%
\pgfsetroundjoin%
\definecolor{currentfill}{rgb}{0.086061,0.312950,0.086061}%
\pgfsetfillcolor{currentfill}%
\pgfsetfillopacity{0.200000}%
\pgfsetlinewidth{0.000000pt}%
\definecolor{currentstroke}{rgb}{0.000000,0.000000,0.000000}%
\pgfsetstrokecolor{currentstroke}%
\pgfsetdash{}{0pt}%
\pgfpathmoveto{\pgfqpoint{4.649042in}{1.952591in}}%
\pgfpathlineto{\pgfqpoint{4.630724in}{1.504068in}}%
\pgfpathlineto{\pgfqpoint{4.941195in}{1.953918in}}%
\pgfpathlineto{\pgfqpoint{4.649042in}{1.952591in}}%
\pgfpathclose%
\pgfusepath{fill}%
\end{pgfscope}%
\begin{pgfscope}%
\pgfpathrectangle{\pgfqpoint{3.536584in}{0.147348in}}{\pgfqpoint{2.735294in}{2.735294in}}%
\pgfusepath{clip}%
\pgfsetbuttcap%
\pgfsetroundjoin%
\definecolor{currentfill}{rgb}{0.086061,0.312950,0.086061}%
\pgfsetfillcolor{currentfill}%
\pgfsetfillopacity{0.200000}%
\pgfsetlinewidth{0.000000pt}%
\definecolor{currentstroke}{rgb}{0.000000,0.000000,0.000000}%
\pgfsetstrokecolor{currentstroke}%
\pgfsetdash{}{0pt}%
\pgfpathmoveto{\pgfqpoint{4.941195in}{1.953918in}}%
\pgfpathlineto{\pgfqpoint{5.251666in}{1.504068in}}%
\pgfpathlineto{\pgfqpoint{5.233348in}{1.952591in}}%
\pgfpathlineto{\pgfqpoint{4.941195in}{1.953918in}}%
\pgfpathclose%
\pgfusepath{fill}%
\end{pgfscope}%
\begin{pgfscope}%
\pgfpathrectangle{\pgfqpoint{3.536584in}{0.147348in}}{\pgfqpoint{2.735294in}{2.735294in}}%
\pgfusepath{clip}%
\pgfsetbuttcap%
\pgfsetroundjoin%
\definecolor{currentfill}{rgb}{0.086258,0.313666,0.086258}%
\pgfsetfillcolor{currentfill}%
\pgfsetfillopacity{0.200000}%
\pgfsetlinewidth{0.000000pt}%
\definecolor{currentstroke}{rgb}{0.000000,0.000000,0.000000}%
\pgfsetstrokecolor{currentstroke}%
\pgfsetdash{}{0pt}%
\pgfpathmoveto{\pgfqpoint{4.941195in}{1.495457in}}%
\pgfpathlineto{\pgfqpoint{4.941195in}{1.953918in}}%
\pgfpathlineto{\pgfqpoint{4.630724in}{1.504068in}}%
\pgfpathlineto{\pgfqpoint{4.941195in}{1.495457in}}%
\pgfpathclose%
\pgfusepath{fill}%
\end{pgfscope}%
\begin{pgfscope}%
\pgfpathrectangle{\pgfqpoint{3.536584in}{0.147348in}}{\pgfqpoint{2.735294in}{2.735294in}}%
\pgfusepath{clip}%
\pgfsetbuttcap%
\pgfsetroundjoin%
\definecolor{currentfill}{rgb}{0.086258,0.313666,0.086258}%
\pgfsetfillcolor{currentfill}%
\pgfsetfillopacity{0.200000}%
\pgfsetlinewidth{0.000000pt}%
\definecolor{currentstroke}{rgb}{0.000000,0.000000,0.000000}%
\pgfsetstrokecolor{currentstroke}%
\pgfsetdash{}{0pt}%
\pgfpathmoveto{\pgfqpoint{5.251666in}{1.504068in}}%
\pgfpathlineto{\pgfqpoint{4.941195in}{1.953918in}}%
\pgfpathlineto{\pgfqpoint{4.941195in}{1.495457in}}%
\pgfpathlineto{\pgfqpoint{5.251666in}{1.504068in}}%
\pgfpathclose%
\pgfusepath{fill}%
\end{pgfscope}%
\begin{pgfscope}%
\pgfpathrectangle{\pgfqpoint{3.536584in}{0.147348in}}{\pgfqpoint{2.735294in}{2.735294in}}%
\pgfusepath{clip}%
\pgfsetbuttcap%
\pgfsetroundjoin%
\definecolor{currentfill}{rgb}{0.074668,0.271519,0.074668}%
\pgfsetfillcolor{currentfill}%
\pgfsetfillopacity{0.200000}%
\pgfsetlinewidth{0.000000pt}%
\definecolor{currentstroke}{rgb}{0.000000,0.000000,0.000000}%
\pgfsetstrokecolor{currentstroke}%
\pgfsetdash{}{0pt}%
\pgfpathmoveto{\pgfqpoint{4.630724in}{1.504068in}}%
\pgfpathlineto{\pgfqpoint{4.786029in}{1.277996in}}%
\pgfpathlineto{\pgfqpoint{4.941195in}{1.495457in}}%
\pgfpathlineto{\pgfqpoint{4.630724in}{1.504068in}}%
\pgfpathclose%
\pgfusepath{fill}%
\end{pgfscope}%
\begin{pgfscope}%
\pgfpathrectangle{\pgfqpoint{3.536584in}{0.147348in}}{\pgfqpoint{2.735294in}{2.735294in}}%
\pgfusepath{clip}%
\pgfsetbuttcap%
\pgfsetroundjoin%
\definecolor{currentfill}{rgb}{0.074668,0.271519,0.074668}%
\pgfsetfillcolor{currentfill}%
\pgfsetfillopacity{0.200000}%
\pgfsetlinewidth{0.000000pt}%
\definecolor{currentstroke}{rgb}{0.000000,0.000000,0.000000}%
\pgfsetstrokecolor{currentstroke}%
\pgfsetdash{}{0pt}%
\pgfpathmoveto{\pgfqpoint{4.941195in}{1.495457in}}%
\pgfpathlineto{\pgfqpoint{5.096361in}{1.277996in}}%
\pgfpathlineto{\pgfqpoint{5.251666in}{1.504068in}}%
\pgfpathlineto{\pgfqpoint{4.941195in}{1.495457in}}%
\pgfpathclose%
\pgfusepath{fill}%
\end{pgfscope}%
\begin{pgfscope}%
\pgfsetbuttcap%
\pgfsetmiterjoin%
\definecolor{currentfill}{rgb}{1.000000,1.000000,1.000000}%
\pgfsetfillcolor{currentfill}%
\pgfsetlinewidth{0.000000pt}%
\definecolor{currentstroke}{rgb}{0.000000,0.000000,0.000000}%
\pgfsetstrokecolor{currentstroke}%
\pgfsetstrokeopacity{0.000000}%
\pgfsetdash{}{0pt}%
\pgfpathmoveto{\pgfqpoint{6.818937in}{0.147348in}}%
\pgfpathlineto{\pgfqpoint{9.554231in}{0.147348in}}%
\pgfpathlineto{\pgfqpoint{9.554231in}{2.882642in}}%
\pgfpathlineto{\pgfqpoint{6.818937in}{2.882642in}}%
\pgfpathlineto{\pgfqpoint{6.818937in}{0.147348in}}%
\pgfpathclose%
\pgfusepath{fill}%
\end{pgfscope}%
\begin{pgfscope}%
\pgfsetbuttcap%
\pgfsetmiterjoin%
\definecolor{currentfill}{rgb}{0.950000,0.950000,0.950000}%
\pgfsetfillcolor{currentfill}%
\pgfsetfillopacity{0.500000}%
\pgfsetlinewidth{1.003750pt}%
\definecolor{currentstroke}{rgb}{0.950000,0.950000,0.950000}%
\pgfsetstrokecolor{currentstroke}%
\pgfsetstrokeopacity{0.500000}%
\pgfsetdash{}{0pt}%
\pgfpathmoveto{\pgfqpoint{9.508741in}{1.070011in}}%
\pgfpathlineto{\pgfqpoint{8.223548in}{0.241771in}}%
\pgfpathlineto{\pgfqpoint{8.223548in}{1.173119in}}%
\pgfpathlineto{\pgfqpoint{9.430084in}{2.004410in}}%
\pgfusepath{stroke,fill}%
\end{pgfscope}%
\begin{pgfscope}%
\pgfsetbuttcap%
\pgfsetmiterjoin%
\definecolor{currentfill}{rgb}{0.900000,0.900000,0.900000}%
\pgfsetfillcolor{currentfill}%
\pgfsetfillopacity{0.500000}%
\pgfsetlinewidth{1.003750pt}%
\definecolor{currentstroke}{rgb}{0.900000,0.900000,0.900000}%
\pgfsetstrokecolor{currentstroke}%
\pgfsetstrokeopacity{0.500000}%
\pgfsetdash{}{0pt}%
\pgfpathmoveto{\pgfqpoint{6.938354in}{1.070011in}}%
\pgfpathlineto{\pgfqpoint{8.223548in}{0.241771in}}%
\pgfpathlineto{\pgfqpoint{8.223548in}{1.173119in}}%
\pgfpathlineto{\pgfqpoint{7.017011in}{2.004410in}}%
\pgfusepath{stroke,fill}%
\end{pgfscope}%
\begin{pgfscope}%
\pgfsetbuttcap%
\pgfsetmiterjoin%
\definecolor{currentfill}{rgb}{0.925000,0.925000,0.925000}%
\pgfsetfillcolor{currentfill}%
\pgfsetfillopacity{0.500000}%
\pgfsetlinewidth{1.003750pt}%
\definecolor{currentstroke}{rgb}{0.925000,0.925000,0.925000}%
\pgfsetstrokecolor{currentstroke}%
\pgfsetstrokeopacity{0.500000}%
\pgfsetdash{}{0pt}%
\pgfpathmoveto{\pgfqpoint{8.223548in}{2.937509in}}%
\pgfpathlineto{\pgfqpoint{9.430084in}{2.004410in}}%
\pgfpathlineto{\pgfqpoint{8.223548in}{1.173119in}}%
\pgfpathlineto{\pgfqpoint{7.017011in}{2.004410in}}%
\pgfusepath{stroke,fill}%
\end{pgfscope}%
\begin{pgfscope}%
\pgfsetbuttcap%
\pgfsetroundjoin%
\pgfsetlinewidth{0.803000pt}%
\definecolor{currentstroke}{rgb}{0.690196,0.690196,0.690196}%
\pgfsetstrokecolor{currentstroke}%
\pgfsetdash{}{0pt}%
\pgfpathmoveto{\pgfqpoint{8.304713in}{2.874739in}}%
\pgfpathlineto{\pgfqpoint{7.097924in}{1.948662in}}%
\pgfpathlineto{\pgfqpoint{7.024828in}{1.014283in}}%
\pgfusepath{stroke}%
\end{pgfscope}%
\begin{pgfscope}%
\pgfsetbuttcap%
\pgfsetroundjoin%
\pgfsetlinewidth{0.803000pt}%
\definecolor{currentstroke}{rgb}{0.690196,0.690196,0.690196}%
\pgfsetstrokecolor{currentstroke}%
\pgfsetdash{}{0pt}%
\pgfpathmoveto{\pgfqpoint{8.553700in}{2.682179in}}%
\pgfpathlineto{\pgfqpoint{7.346364in}{1.777489in}}%
\pgfpathlineto{\pgfqpoint{7.290085in}{0.843339in}}%
\pgfusepath{stroke}%
\end{pgfscope}%
\begin{pgfscope}%
\pgfsetbuttcap%
\pgfsetroundjoin%
\pgfsetlinewidth{0.803000pt}%
\definecolor{currentstroke}{rgb}{0.690196,0.690196,0.690196}%
\pgfsetstrokecolor{currentstroke}%
\pgfsetdash{}{0pt}%
\pgfpathmoveto{\pgfqpoint{8.796805in}{2.494169in}}%
\pgfpathlineto{\pgfqpoint{7.589265in}{1.610133in}}%
\pgfpathlineto{\pgfqpoint{7.549052in}{0.676448in}}%
\pgfusepath{stroke}%
\end{pgfscope}%
\begin{pgfscope}%
\pgfsetbuttcap%
\pgfsetroundjoin%
\pgfsetlinewidth{0.803000pt}%
\definecolor{currentstroke}{rgb}{0.690196,0.690196,0.690196}%
\pgfsetstrokecolor{currentstroke}%
\pgfsetdash{}{0pt}%
\pgfpathmoveto{\pgfqpoint{9.034233in}{2.310549in}}%
\pgfpathlineto{\pgfqpoint{7.826809in}{1.446468in}}%
\pgfpathlineto{\pgfqpoint{7.801951in}{0.513468in}}%
\pgfusepath{stroke}%
\end{pgfscope}%
\begin{pgfscope}%
\pgfsetbuttcap%
\pgfsetroundjoin%
\pgfsetlinewidth{0.803000pt}%
\definecolor{currentstroke}{rgb}{0.690196,0.690196,0.690196}%
\pgfsetstrokecolor{currentstroke}%
\pgfsetdash{}{0pt}%
\pgfpathmoveto{\pgfqpoint{9.266181in}{2.131167in}}%
\pgfpathlineto{\pgfqpoint{8.059172in}{1.286373in}}%
\pgfpathlineto{\pgfqpoint{8.048992in}{0.354263in}}%
\pgfusepath{stroke}%
\end{pgfscope}%
\begin{pgfscope}%
\pgfsetbuttcap%
\pgfsetroundjoin%
\pgfsetlinewidth{0.803000pt}%
\definecolor{currentstroke}{rgb}{0.690196,0.690196,0.690196}%
\pgfsetstrokecolor{currentstroke}%
\pgfsetdash{}{0pt}%
\pgfpathmoveto{\pgfqpoint{9.422268in}{1.014283in}}%
\pgfpathlineto{\pgfqpoint{9.349172in}{1.948662in}}%
\pgfpathlineto{\pgfqpoint{8.142383in}{2.874739in}}%
\pgfusepath{stroke}%
\end{pgfscope}%
\begin{pgfscope}%
\pgfsetbuttcap%
\pgfsetroundjoin%
\pgfsetlinewidth{0.803000pt}%
\definecolor{currentstroke}{rgb}{0.690196,0.690196,0.690196}%
\pgfsetstrokecolor{currentstroke}%
\pgfsetdash{}{0pt}%
\pgfpathmoveto{\pgfqpoint{9.157011in}{0.843339in}}%
\pgfpathlineto{\pgfqpoint{9.100731in}{1.777489in}}%
\pgfpathlineto{\pgfqpoint{7.893396in}{2.682179in}}%
\pgfusepath{stroke}%
\end{pgfscope}%
\begin{pgfscope}%
\pgfsetbuttcap%
\pgfsetroundjoin%
\pgfsetlinewidth{0.803000pt}%
\definecolor{currentstroke}{rgb}{0.690196,0.690196,0.690196}%
\pgfsetstrokecolor{currentstroke}%
\pgfsetdash{}{0pt}%
\pgfpathmoveto{\pgfqpoint{8.898044in}{0.676448in}}%
\pgfpathlineto{\pgfqpoint{8.857831in}{1.610133in}}%
\pgfpathlineto{\pgfqpoint{7.650291in}{2.494169in}}%
\pgfusepath{stroke}%
\end{pgfscope}%
\begin{pgfscope}%
\pgfsetbuttcap%
\pgfsetroundjoin%
\pgfsetlinewidth{0.803000pt}%
\definecolor{currentstroke}{rgb}{0.690196,0.690196,0.690196}%
\pgfsetstrokecolor{currentstroke}%
\pgfsetdash{}{0pt}%
\pgfpathmoveto{\pgfqpoint{8.645145in}{0.513468in}}%
\pgfpathlineto{\pgfqpoint{8.620287in}{1.446468in}}%
\pgfpathlineto{\pgfqpoint{7.412863in}{2.310549in}}%
\pgfusepath{stroke}%
\end{pgfscope}%
\begin{pgfscope}%
\pgfsetbuttcap%
\pgfsetroundjoin%
\pgfsetlinewidth{0.803000pt}%
\definecolor{currentstroke}{rgb}{0.690196,0.690196,0.690196}%
\pgfsetstrokecolor{currentstroke}%
\pgfsetdash{}{0pt}%
\pgfpathmoveto{\pgfqpoint{8.398104in}{0.354263in}}%
\pgfpathlineto{\pgfqpoint{8.387924in}{1.286373in}}%
\pgfpathlineto{\pgfqpoint{7.180914in}{2.131167in}}%
\pgfusepath{stroke}%
\end{pgfscope}%
\begin{pgfscope}%
\pgfsetbuttcap%
\pgfsetroundjoin%
\pgfsetlinewidth{0.803000pt}%
\definecolor{currentstroke}{rgb}{0.690196,0.690196,0.690196}%
\pgfsetstrokecolor{currentstroke}%
\pgfsetdash{}{0pt}%
\pgfpathmoveto{\pgfqpoint{6.943664in}{1.133087in}}%
\pgfpathlineto{\pgfqpoint{8.223548in}{0.304434in}}%
\pgfpathlineto{\pgfqpoint{9.503432in}{1.133087in}}%
\pgfusepath{stroke}%
\end{pgfscope}%
\begin{pgfscope}%
\pgfsetbuttcap%
\pgfsetroundjoin%
\pgfsetlinewidth{0.803000pt}%
\definecolor{currentstroke}{rgb}{0.690196,0.690196,0.690196}%
\pgfsetstrokecolor{currentstroke}%
\pgfsetdash{}{0pt}%
\pgfpathmoveto{\pgfqpoint{6.959936in}{1.326388in}}%
\pgfpathlineto{\pgfqpoint{8.223548in}{0.496654in}}%
\pgfpathlineto{\pgfqpoint{9.487160in}{1.326388in}}%
\pgfusepath{stroke}%
\end{pgfscope}%
\begin{pgfscope}%
\pgfsetbuttcap%
\pgfsetroundjoin%
\pgfsetlinewidth{0.803000pt}%
\definecolor{currentstroke}{rgb}{0.690196,0.690196,0.690196}%
\pgfsetstrokecolor{currentstroke}%
\pgfsetdash{}{0pt}%
\pgfpathmoveto{\pgfqpoint{6.975799in}{1.514835in}}%
\pgfpathlineto{\pgfqpoint{8.223548in}{0.684319in}}%
\pgfpathlineto{\pgfqpoint{9.471296in}{1.514835in}}%
\pgfusepath{stroke}%
\end{pgfscope}%
\begin{pgfscope}%
\pgfsetbuttcap%
\pgfsetroundjoin%
\pgfsetlinewidth{0.803000pt}%
\definecolor{currentstroke}{rgb}{0.690196,0.690196,0.690196}%
\pgfsetstrokecolor{currentstroke}%
\pgfsetdash{}{0pt}%
\pgfpathmoveto{\pgfqpoint{6.991269in}{1.698610in}}%
\pgfpathlineto{\pgfqpoint{8.223548in}{0.867589in}}%
\pgfpathlineto{\pgfqpoint{9.455826in}{1.698610in}}%
\pgfusepath{stroke}%
\end{pgfscope}%
\begin{pgfscope}%
\pgfsetbuttcap%
\pgfsetroundjoin%
\pgfsetlinewidth{0.803000pt}%
\definecolor{currentstroke}{rgb}{0.690196,0.690196,0.690196}%
\pgfsetstrokecolor{currentstroke}%
\pgfsetdash{}{0pt}%
\pgfpathmoveto{\pgfqpoint{7.006360in}{1.877883in}}%
\pgfpathlineto{\pgfqpoint{8.223548in}{1.046618in}}%
\pgfpathlineto{\pgfqpoint{9.440735in}{1.877883in}}%
\pgfusepath{stroke}%
\end{pgfscope}%
\begin{pgfscope}%
\pgfsetrectcap%
\pgfsetroundjoin%
\pgfsetlinewidth{0.803000pt}%
\definecolor{currentstroke}{rgb}{0.000000,0.000000,0.000000}%
\pgfsetstrokecolor{currentstroke}%
\pgfsetdash{}{0pt}%
\pgfpathmoveto{\pgfqpoint{9.430084in}{2.004410in}}%
\pgfpathlineto{\pgfqpoint{8.223548in}{2.937509in}}%
\pgfusepath{stroke}%
\end{pgfscope}%
\begin{pgfscope}%
\pgfsetrectcap%
\pgfsetroundjoin%
\pgfsetlinewidth{0.803000pt}%
\definecolor{currentstroke}{rgb}{0.000000,0.000000,0.000000}%
\pgfsetstrokecolor{currentstroke}%
\pgfsetdash{}{0pt}%
\pgfpathmoveto{\pgfqpoint{8.294475in}{2.866882in}}%
\pgfpathlineto{\pgfqpoint{8.325219in}{2.890475in}}%
\pgfusepath{stroke}%
\end{pgfscope}%
\begin{pgfscope}%
\pgfsetrectcap%
\pgfsetroundjoin%
\pgfsetlinewidth{0.803000pt}%
\definecolor{currentstroke}{rgb}{0.000000,0.000000,0.000000}%
\pgfsetstrokecolor{currentstroke}%
\pgfsetdash{}{0pt}%
\pgfpathmoveto{\pgfqpoint{8.543464in}{2.674509in}}%
\pgfpathlineto{\pgfqpoint{8.574201in}{2.697542in}}%
\pgfusepath{stroke}%
\end{pgfscope}%
\begin{pgfscope}%
\pgfsetrectcap%
\pgfsetroundjoin%
\pgfsetlinewidth{0.803000pt}%
\definecolor{currentstroke}{rgb}{0.000000,0.000000,0.000000}%
\pgfsetstrokecolor{currentstroke}%
\pgfsetdash{}{0pt}%
\pgfpathmoveto{\pgfqpoint{8.786574in}{2.486679in}}%
\pgfpathlineto{\pgfqpoint{8.817296in}{2.509171in}}%
\pgfusepath{stroke}%
\end{pgfscope}%
\begin{pgfscope}%
\pgfsetrectcap%
\pgfsetroundjoin%
\pgfsetlinewidth{0.803000pt}%
\definecolor{currentstroke}{rgb}{0.000000,0.000000,0.000000}%
\pgfsetstrokecolor{currentstroke}%
\pgfsetdash{}{0pt}%
\pgfpathmoveto{\pgfqpoint{9.024010in}{2.303233in}}%
\pgfpathlineto{\pgfqpoint{9.054708in}{2.325202in}}%
\pgfusepath{stroke}%
\end{pgfscope}%
\begin{pgfscope}%
\pgfsetrectcap%
\pgfsetroundjoin%
\pgfsetlinewidth{0.803000pt}%
\definecolor{currentstroke}{rgb}{0.000000,0.000000,0.000000}%
\pgfsetstrokecolor{currentstroke}%
\pgfsetdash{}{0pt}%
\pgfpathmoveto{\pgfqpoint{9.255968in}{2.124019in}}%
\pgfpathlineto{\pgfqpoint{9.286636in}{2.145484in}}%
\pgfusepath{stroke}%
\end{pgfscope}%
\begin{pgfscope}%
\definecolor{textcolor}{rgb}{0.000000,0.000000,0.000000}%
\pgfsetstrokecolor{textcolor}%
\pgfsetfillcolor{textcolor}%
\pgftext[x=9.122590in,y=2.861959in,,]{\color{textcolor}{\rmfamily\fontsize{14.000000}{16.800000}\selectfont\catcode`\^=\active\def^{\ifmmode\sp\else\^{}\fi}\catcode`\%=\active\def%{\%}f1}}%
\end{pgfscope}%
\begin{pgfscope}%
\pgfsetrectcap%
\pgfsetroundjoin%
\pgfsetlinewidth{0.803000pt}%
\definecolor{currentstroke}{rgb}{0.000000,0.000000,0.000000}%
\pgfsetstrokecolor{currentstroke}%
\pgfsetdash{}{0pt}%
\pgfpathmoveto{\pgfqpoint{7.017011in}{2.004410in}}%
\pgfpathlineto{\pgfqpoint{8.223548in}{2.937509in}}%
\pgfusepath{stroke}%
\end{pgfscope}%
\begin{pgfscope}%
\pgfsetrectcap%
\pgfsetroundjoin%
\pgfsetlinewidth{0.803000pt}%
\definecolor{currentstroke}{rgb}{0.000000,0.000000,0.000000}%
\pgfsetstrokecolor{currentstroke}%
\pgfsetdash{}{0pt}%
\pgfpathmoveto{\pgfqpoint{8.152621in}{2.866882in}}%
\pgfpathlineto{\pgfqpoint{8.121876in}{2.890475in}}%
\pgfusepath{stroke}%
\end{pgfscope}%
\begin{pgfscope}%
\pgfsetrectcap%
\pgfsetroundjoin%
\pgfsetlinewidth{0.803000pt}%
\definecolor{currentstroke}{rgb}{0.000000,0.000000,0.000000}%
\pgfsetstrokecolor{currentstroke}%
\pgfsetdash{}{0pt}%
\pgfpathmoveto{\pgfqpoint{7.903631in}{2.674509in}}%
\pgfpathlineto{\pgfqpoint{7.872894in}{2.697542in}}%
\pgfusepath{stroke}%
\end{pgfscope}%
\begin{pgfscope}%
\pgfsetrectcap%
\pgfsetroundjoin%
\pgfsetlinewidth{0.803000pt}%
\definecolor{currentstroke}{rgb}{0.000000,0.000000,0.000000}%
\pgfsetstrokecolor{currentstroke}%
\pgfsetdash{}{0pt}%
\pgfpathmoveto{\pgfqpoint{7.660522in}{2.486679in}}%
\pgfpathlineto{\pgfqpoint{7.629800in}{2.509171in}}%
\pgfusepath{stroke}%
\end{pgfscope}%
\begin{pgfscope}%
\pgfsetrectcap%
\pgfsetroundjoin%
\pgfsetlinewidth{0.803000pt}%
\definecolor{currentstroke}{rgb}{0.000000,0.000000,0.000000}%
\pgfsetstrokecolor{currentstroke}%
\pgfsetdash{}{0pt}%
\pgfpathmoveto{\pgfqpoint{7.423086in}{2.303233in}}%
\pgfpathlineto{\pgfqpoint{7.392387in}{2.325202in}}%
\pgfusepath{stroke}%
\end{pgfscope}%
\begin{pgfscope}%
\pgfsetrectcap%
\pgfsetroundjoin%
\pgfsetlinewidth{0.803000pt}%
\definecolor{currentstroke}{rgb}{0.000000,0.000000,0.000000}%
\pgfsetstrokecolor{currentstroke}%
\pgfsetdash{}{0pt}%
\pgfpathmoveto{\pgfqpoint{7.191128in}{2.124019in}}%
\pgfpathlineto{\pgfqpoint{7.160460in}{2.145484in}}%
\pgfusepath{stroke}%
\end{pgfscope}%
\begin{pgfscope}%
\definecolor{textcolor}{rgb}{0.000000,0.000000,0.000000}%
\pgfsetstrokecolor{textcolor}%
\pgfsetfillcolor{textcolor}%
\pgftext[x=7.324506in,y=2.861959in,,]{\color{textcolor}{\rmfamily\fontsize{14.000000}{16.800000}\selectfont\catcode`\^=\active\def^{\ifmmode\sp\else\^{}\fi}\catcode`\%=\active\def%{\%}f2}}%
\end{pgfscope}%
\begin{pgfscope}%
\pgfsetrectcap%
\pgfsetroundjoin%
\pgfsetlinewidth{0.803000pt}%
\definecolor{currentstroke}{rgb}{0.000000,0.000000,0.000000}%
\pgfsetstrokecolor{currentstroke}%
\pgfsetdash{}{0pt}%
\pgfpathmoveto{\pgfqpoint{7.017011in}{2.004410in}}%
\pgfpathlineto{\pgfqpoint{6.938354in}{1.070011in}}%
\pgfusepath{stroke}%
\end{pgfscope}%
\begin{pgfscope}%
\pgfsetrectcap%
\pgfsetroundjoin%
\pgfsetlinewidth{0.803000pt}%
\definecolor{currentstroke}{rgb}{0.000000,0.000000,0.000000}%
\pgfsetstrokecolor{currentstroke}%
\pgfsetdash{}{0pt}%
\pgfpathmoveto{\pgfqpoint{6.954524in}{1.126056in}}%
\pgfpathlineto{\pgfqpoint{6.921911in}{1.147171in}}%
\pgfusepath{stroke}%
\end{pgfscope}%
\begin{pgfscope}%
\pgfsetrectcap%
\pgfsetroundjoin%
\pgfsetlinewidth{0.803000pt}%
\definecolor{currentstroke}{rgb}{0.000000,0.000000,0.000000}%
\pgfsetstrokecolor{currentstroke}%
\pgfsetdash{}{0pt}%
\pgfpathmoveto{\pgfqpoint{6.970651in}{1.319352in}}%
\pgfpathlineto{\pgfqpoint{6.938476in}{1.340480in}}%
\pgfusepath{stroke}%
\end{pgfscope}%
\begin{pgfscope}%
\pgfsetrectcap%
\pgfsetroundjoin%
\pgfsetlinewidth{0.803000pt}%
\definecolor{currentstroke}{rgb}{0.000000,0.000000,0.000000}%
\pgfsetstrokecolor{currentstroke}%
\pgfsetdash{}{0pt}%
\pgfpathmoveto{\pgfqpoint{6.986372in}{1.507798in}}%
\pgfpathlineto{\pgfqpoint{6.954624in}{1.528930in}}%
\pgfusepath{stroke}%
\end{pgfscope}%
\begin{pgfscope}%
\pgfsetrectcap%
\pgfsetroundjoin%
\pgfsetlinewidth{0.803000pt}%
\definecolor{currentstroke}{rgb}{0.000000,0.000000,0.000000}%
\pgfsetstrokecolor{currentstroke}%
\pgfsetdash{}{0pt}%
\pgfpathmoveto{\pgfqpoint{7.001704in}{1.691573in}}%
\pgfpathlineto{\pgfqpoint{6.970371in}{1.712703in}}%
\pgfusepath{stroke}%
\end{pgfscope}%
\begin{pgfscope}%
\pgfsetrectcap%
\pgfsetroundjoin%
\pgfsetlinewidth{0.803000pt}%
\definecolor{currentstroke}{rgb}{0.000000,0.000000,0.000000}%
\pgfsetstrokecolor{currentstroke}%
\pgfsetdash{}{0pt}%
\pgfpathmoveto{\pgfqpoint{7.016660in}{1.870849in}}%
\pgfpathlineto{\pgfqpoint{6.985733in}{1.891971in}}%
\pgfusepath{stroke}%
\end{pgfscope}%
\begin{pgfscope}%
\definecolor{textcolor}{rgb}{0.000000,0.000000,0.000000}%
\pgfsetstrokecolor{textcolor}%
\pgfsetfillcolor{textcolor}%
\pgftext[x=6.420762in,y=1.551958in,,]{\color{textcolor}{\rmfamily\fontsize{14.000000}{16.800000}\selectfont\catcode`\^=\active\def^{\ifmmode\sp\else\^{}\fi}\catcode`\%=\active\def%{\%}f3}}%
\end{pgfscope}%
\begin{pgfscope}%
\pgfpathrectangle{\pgfqpoint{6.818937in}{0.147348in}}{\pgfqpoint{2.735294in}{2.735294in}}%
\pgfusepath{clip}%
\pgfsetbuttcap%
\pgfsetroundjoin%
\definecolor{currentfill}{rgb}{0.839216,0.152941,0.156863}%
\pgfsetfillcolor{currentfill}%
\pgfsetfillopacity{0.300000}%
\pgfsetlinewidth{1.003750pt}%
\definecolor{currentstroke}{rgb}{0.839216,0.152941,0.156863}%
\pgfsetstrokecolor{currentstroke}%
\pgfsetstrokeopacity{0.300000}%
\pgfsetdash{}{0pt}%
\pgfpathmoveto{\pgfqpoint{7.927926in}{1.506774in}}%
\pgfpathcurveto{\pgfqpoint{7.938014in}{1.506774in}}{\pgfqpoint{7.947689in}{1.510782in}}{\pgfqpoint{7.954822in}{1.517914in}}%
\pgfpathcurveto{\pgfqpoint{7.961955in}{1.525047in}}{\pgfqpoint{7.965963in}{1.534723in}}{\pgfqpoint{7.965963in}{1.544810in}}%
\pgfpathcurveto{\pgfqpoint{7.965963in}{1.554897in}}{\pgfqpoint{7.961955in}{1.564573in}}{\pgfqpoint{7.954822in}{1.571706in}}%
\pgfpathcurveto{\pgfqpoint{7.947689in}{1.578839in}}{\pgfqpoint{7.938014in}{1.582846in}}{\pgfqpoint{7.927926in}{1.582846in}}%
\pgfpathcurveto{\pgfqpoint{7.917839in}{1.582846in}}{\pgfqpoint{7.908164in}{1.578839in}}{\pgfqpoint{7.901031in}{1.571706in}}%
\pgfpathcurveto{\pgfqpoint{7.893898in}{1.564573in}}{\pgfqpoint{7.889890in}{1.554897in}}{\pgfqpoint{7.889890in}{1.544810in}}%
\pgfpathcurveto{\pgfqpoint{7.889890in}{1.534723in}}{\pgfqpoint{7.893898in}{1.525047in}}{\pgfqpoint{7.901031in}{1.517914in}}%
\pgfpathcurveto{\pgfqpoint{7.908164in}{1.510782in}}{\pgfqpoint{7.917839in}{1.506774in}}{\pgfqpoint{7.927926in}{1.506774in}}%
\pgfpathlineto{\pgfqpoint{7.927926in}{1.506774in}}%
\pgfpathclose%
\pgfusepath{stroke,fill}%
\end{pgfscope}%
\begin{pgfscope}%
\pgfpathrectangle{\pgfqpoint{6.818937in}{0.147348in}}{\pgfqpoint{2.735294in}{2.735294in}}%
\pgfusepath{clip}%
\pgfsetbuttcap%
\pgfsetroundjoin%
\definecolor{currentfill}{rgb}{0.839216,0.152941,0.156863}%
\pgfsetfillcolor{currentfill}%
\pgfsetfillopacity{0.368519}%
\pgfsetlinewidth{1.003750pt}%
\definecolor{currentstroke}{rgb}{0.839216,0.152941,0.156863}%
\pgfsetstrokecolor{currentstroke}%
\pgfsetstrokeopacity{0.368519}%
\pgfsetdash{}{0pt}%
\pgfpathmoveto{\pgfqpoint{7.807338in}{1.289993in}}%
\pgfpathcurveto{\pgfqpoint{7.817425in}{1.289993in}}{\pgfqpoint{7.827101in}{1.294001in}}{\pgfqpoint{7.834233in}{1.301134in}}%
\pgfpathcurveto{\pgfqpoint{7.841366in}{1.308267in}}{\pgfqpoint{7.845374in}{1.317942in}}{\pgfqpoint{7.845374in}{1.328030in}}%
\pgfpathcurveto{\pgfqpoint{7.845374in}{1.338117in}}{\pgfqpoint{7.841366in}{1.347793in}}{\pgfqpoint{7.834233in}{1.354925in}}%
\pgfpathcurveto{\pgfqpoint{7.827101in}{1.362058in}}{\pgfqpoint{7.817425in}{1.366066in}}{\pgfqpoint{7.807338in}{1.366066in}}%
\pgfpathcurveto{\pgfqpoint{7.797250in}{1.366066in}}{\pgfqpoint{7.787575in}{1.362058in}}{\pgfqpoint{7.780442in}{1.354925in}}%
\pgfpathcurveto{\pgfqpoint{7.773309in}{1.347793in}}{\pgfqpoint{7.769301in}{1.338117in}}{\pgfqpoint{7.769301in}{1.328030in}}%
\pgfpathcurveto{\pgfqpoint{7.769301in}{1.317942in}}{\pgfqpoint{7.773309in}{1.308267in}}{\pgfqpoint{7.780442in}{1.301134in}}%
\pgfpathcurveto{\pgfqpoint{7.787575in}{1.294001in}}{\pgfqpoint{7.797250in}{1.289993in}}{\pgfqpoint{7.807338in}{1.289993in}}%
\pgfpathlineto{\pgfqpoint{7.807338in}{1.289993in}}%
\pgfpathclose%
\pgfusepath{stroke,fill}%
\end{pgfscope}%
\begin{pgfscope}%
\pgfpathrectangle{\pgfqpoint{6.818937in}{0.147348in}}{\pgfqpoint{2.735294in}{2.735294in}}%
\pgfusepath{clip}%
\pgfsetbuttcap%
\pgfsetroundjoin%
\definecolor{currentfill}{rgb}{0.839216,0.152941,0.156863}%
\pgfsetfillcolor{currentfill}%
\pgfsetfillopacity{0.431635}%
\pgfsetlinewidth{1.003750pt}%
\definecolor{currentstroke}{rgb}{0.839216,0.152941,0.156863}%
\pgfsetstrokecolor{currentstroke}%
\pgfsetstrokeopacity{0.431635}%
\pgfsetdash{}{0pt}%
\pgfpathmoveto{\pgfqpoint{7.865218in}{1.926337in}}%
\pgfpathcurveto{\pgfqpoint{7.875306in}{1.926337in}}{\pgfqpoint{7.884981in}{1.930345in}}{\pgfqpoint{7.892114in}{1.937478in}}%
\pgfpathcurveto{\pgfqpoint{7.899247in}{1.944610in}}{\pgfqpoint{7.903255in}{1.954286in}}{\pgfqpoint{7.903255in}{1.964373in}}%
\pgfpathcurveto{\pgfqpoint{7.903255in}{1.974461in}}{\pgfqpoint{7.899247in}{1.984136in}}{\pgfqpoint{7.892114in}{1.991269in}}%
\pgfpathcurveto{\pgfqpoint{7.884981in}{1.998402in}}{\pgfqpoint{7.875306in}{2.002410in}}{\pgfqpoint{7.865218in}{2.002410in}}%
\pgfpathcurveto{\pgfqpoint{7.855131in}{2.002410in}}{\pgfqpoint{7.845455in}{1.998402in}}{\pgfqpoint{7.838323in}{1.991269in}}%
\pgfpathcurveto{\pgfqpoint{7.831190in}{1.984136in}}{\pgfqpoint{7.827182in}{1.974461in}}{\pgfqpoint{7.827182in}{1.964373in}}%
\pgfpathcurveto{\pgfqpoint{7.827182in}{1.954286in}}{\pgfqpoint{7.831190in}{1.944610in}}{\pgfqpoint{7.838323in}{1.937478in}}%
\pgfpathcurveto{\pgfqpoint{7.845455in}{1.930345in}}{\pgfqpoint{7.855131in}{1.926337in}}{\pgfqpoint{7.865218in}{1.926337in}}%
\pgfpathlineto{\pgfqpoint{7.865218in}{1.926337in}}%
\pgfpathclose%
\pgfusepath{stroke,fill}%
\end{pgfscope}%
\begin{pgfscope}%
\pgfpathrectangle{\pgfqpoint{6.818937in}{0.147348in}}{\pgfqpoint{2.735294in}{2.735294in}}%
\pgfusepath{clip}%
\pgfsetbuttcap%
\pgfsetroundjoin%
\definecolor{currentfill}{rgb}{0.839216,0.152941,0.156863}%
\pgfsetfillcolor{currentfill}%
\pgfsetfillopacity{0.502644}%
\pgfsetlinewidth{1.003750pt}%
\definecolor{currentstroke}{rgb}{0.839216,0.152941,0.156863}%
\pgfsetstrokecolor{currentstroke}%
\pgfsetstrokeopacity{0.502644}%
\pgfsetdash{}{0pt}%
\pgfpathmoveto{\pgfqpoint{8.531889in}{2.015496in}}%
\pgfpathcurveto{\pgfqpoint{8.541976in}{2.015496in}}{\pgfqpoint{8.551652in}{2.019504in}}{\pgfqpoint{8.558785in}{2.026637in}}%
\pgfpathcurveto{\pgfqpoint{8.565918in}{2.033769in}}{\pgfqpoint{8.569925in}{2.043445in}}{\pgfqpoint{8.569925in}{2.053532in}}%
\pgfpathcurveto{\pgfqpoint{8.569925in}{2.063620in}}{\pgfqpoint{8.565918in}{2.073295in}}{\pgfqpoint{8.558785in}{2.080428in}}%
\pgfpathcurveto{\pgfqpoint{8.551652in}{2.087561in}}{\pgfqpoint{8.541976in}{2.091569in}}{\pgfqpoint{8.531889in}{2.091569in}}%
\pgfpathcurveto{\pgfqpoint{8.521802in}{2.091569in}}{\pgfqpoint{8.512126in}{2.087561in}}{\pgfqpoint{8.504993in}{2.080428in}}%
\pgfpathcurveto{\pgfqpoint{8.497860in}{2.073295in}}{\pgfqpoint{8.493853in}{2.063620in}}{\pgfqpoint{8.493853in}{2.053532in}}%
\pgfpathcurveto{\pgfqpoint{8.493853in}{2.043445in}}{\pgfqpoint{8.497860in}{2.033769in}}{\pgfqpoint{8.504993in}{2.026637in}}%
\pgfpathcurveto{\pgfqpoint{8.512126in}{2.019504in}}{\pgfqpoint{8.521802in}{2.015496in}}{\pgfqpoint{8.531889in}{2.015496in}}%
\pgfpathlineto{\pgfqpoint{8.531889in}{2.015496in}}%
\pgfpathclose%
\pgfusepath{stroke,fill}%
\end{pgfscope}%
\begin{pgfscope}%
\pgfpathrectangle{\pgfqpoint{6.818937in}{0.147348in}}{\pgfqpoint{2.735294in}{2.735294in}}%
\pgfusepath{clip}%
\pgfsetbuttcap%
\pgfsetroundjoin%
\definecolor{currentfill}{rgb}{0.839216,0.152941,0.156863}%
\pgfsetfillcolor{currentfill}%
\pgfsetfillopacity{0.555030}%
\pgfsetlinewidth{1.003750pt}%
\definecolor{currentstroke}{rgb}{0.839216,0.152941,0.156863}%
\pgfsetstrokecolor{currentstroke}%
\pgfsetstrokeopacity{0.555030}%
\pgfsetdash{}{0pt}%
\pgfpathmoveto{\pgfqpoint{9.028634in}{1.507681in}}%
\pgfpathcurveto{\pgfqpoint{9.038721in}{1.507681in}}{\pgfqpoint{9.048397in}{1.511689in}}{\pgfqpoint{9.055530in}{1.518822in}}%
\pgfpathcurveto{\pgfqpoint{9.062663in}{1.525955in}}{\pgfqpoint{9.066670in}{1.535630in}}{\pgfqpoint{9.066670in}{1.545718in}}%
\pgfpathcurveto{\pgfqpoint{9.066670in}{1.555805in}}{\pgfqpoint{9.062663in}{1.565481in}}{\pgfqpoint{9.055530in}{1.572613in}}%
\pgfpathcurveto{\pgfqpoint{9.048397in}{1.579746in}}{\pgfqpoint{9.038721in}{1.583754in}}{\pgfqpoint{9.028634in}{1.583754in}}%
\pgfpathcurveto{\pgfqpoint{9.018547in}{1.583754in}}{\pgfqpoint{9.008871in}{1.579746in}}{\pgfqpoint{9.001738in}{1.572613in}}%
\pgfpathcurveto{\pgfqpoint{8.994605in}{1.565481in}}{\pgfqpoint{8.990598in}{1.555805in}}{\pgfqpoint{8.990598in}{1.545718in}}%
\pgfpathcurveto{\pgfqpoint{8.990598in}{1.535630in}}{\pgfqpoint{8.994605in}{1.525955in}}{\pgfqpoint{9.001738in}{1.518822in}}%
\pgfpathcurveto{\pgfqpoint{9.008871in}{1.511689in}}{\pgfqpoint{9.018547in}{1.507681in}}{\pgfqpoint{9.028634in}{1.507681in}}%
\pgfpathlineto{\pgfqpoint{9.028634in}{1.507681in}}%
\pgfpathclose%
\pgfusepath{stroke,fill}%
\end{pgfscope}%
\begin{pgfscope}%
\pgfpathrectangle{\pgfqpoint{6.818937in}{0.147348in}}{\pgfqpoint{2.735294in}{2.735294in}}%
\pgfusepath{clip}%
\pgfsetbuttcap%
\pgfsetroundjoin%
\definecolor{currentfill}{rgb}{0.839216,0.152941,0.156863}%
\pgfsetfillcolor{currentfill}%
\pgfsetfillopacity{0.641254}%
\pgfsetlinewidth{1.003750pt}%
\definecolor{currentstroke}{rgb}{0.839216,0.152941,0.156863}%
\pgfsetstrokecolor{currentstroke}%
\pgfsetstrokeopacity{0.641254}%
\pgfsetdash{}{0pt}%
\pgfpathmoveto{\pgfqpoint{8.416294in}{0.946553in}}%
\pgfpathcurveto{\pgfqpoint{8.426381in}{0.946553in}}{\pgfqpoint{8.436057in}{0.950561in}}{\pgfqpoint{8.443190in}{0.957694in}}%
\pgfpathcurveto{\pgfqpoint{8.450322in}{0.964827in}}{\pgfqpoint{8.454330in}{0.974502in}}{\pgfqpoint{8.454330in}{0.984590in}}%
\pgfpathcurveto{\pgfqpoint{8.454330in}{0.994677in}}{\pgfqpoint{8.450322in}{1.004352in}}{\pgfqpoint{8.443190in}{1.011485in}}%
\pgfpathcurveto{\pgfqpoint{8.436057in}{1.018618in}}{\pgfqpoint{8.426381in}{1.022626in}}{\pgfqpoint{8.416294in}{1.022626in}}%
\pgfpathcurveto{\pgfqpoint{8.406207in}{1.022626in}}{\pgfqpoint{8.396531in}{1.018618in}}{\pgfqpoint{8.389398in}{1.011485in}}%
\pgfpathcurveto{\pgfqpoint{8.382265in}{1.004352in}}{\pgfqpoint{8.378258in}{0.994677in}}{\pgfqpoint{8.378258in}{0.984590in}}%
\pgfpathcurveto{\pgfqpoint{8.378258in}{0.974502in}}{\pgfqpoint{8.382265in}{0.964827in}}{\pgfqpoint{8.389398in}{0.957694in}}%
\pgfpathcurveto{\pgfqpoint{8.396531in}{0.950561in}}{\pgfqpoint{8.406207in}{0.946553in}}{\pgfqpoint{8.416294in}{0.946553in}}%
\pgfpathlineto{\pgfqpoint{8.416294in}{0.946553in}}%
\pgfpathclose%
\pgfusepath{stroke,fill}%
\end{pgfscope}%
\begin{pgfscope}%
\pgfpathrectangle{\pgfqpoint{6.818937in}{0.147348in}}{\pgfqpoint{2.735294in}{2.735294in}}%
\pgfusepath{clip}%
\pgfsetbuttcap%
\pgfsetroundjoin%
\definecolor{currentfill}{rgb}{0.839216,0.152941,0.156863}%
\pgfsetfillcolor{currentfill}%
\pgfsetfillopacity{0.659293}%
\pgfsetlinewidth{1.003750pt}%
\definecolor{currentstroke}{rgb}{0.839216,0.152941,0.156863}%
\pgfsetstrokecolor{currentstroke}%
\pgfsetstrokeopacity{0.659293}%
\pgfsetdash{}{0pt}%
\pgfpathmoveto{\pgfqpoint{7.519784in}{1.533581in}}%
\pgfpathcurveto{\pgfqpoint{7.529872in}{1.533581in}}{\pgfqpoint{7.539547in}{1.537589in}}{\pgfqpoint{7.546680in}{1.544722in}}%
\pgfpathcurveto{\pgfqpoint{7.553813in}{1.551854in}}{\pgfqpoint{7.557821in}{1.561530in}}{\pgfqpoint{7.557821in}{1.571617in}}%
\pgfpathcurveto{\pgfqpoint{7.557821in}{1.581705in}}{\pgfqpoint{7.553813in}{1.591380in}}{\pgfqpoint{7.546680in}{1.598513in}}%
\pgfpathcurveto{\pgfqpoint{7.539547in}{1.605646in}}{\pgfqpoint{7.529872in}{1.609654in}}{\pgfqpoint{7.519784in}{1.609654in}}%
\pgfpathcurveto{\pgfqpoint{7.509697in}{1.609654in}}{\pgfqpoint{7.500022in}{1.605646in}}{\pgfqpoint{7.492889in}{1.598513in}}%
\pgfpathcurveto{\pgfqpoint{7.485756in}{1.591380in}}{\pgfqpoint{7.481748in}{1.581705in}}{\pgfqpoint{7.481748in}{1.571617in}}%
\pgfpathcurveto{\pgfqpoint{7.481748in}{1.561530in}}{\pgfqpoint{7.485756in}{1.551854in}}{\pgfqpoint{7.492889in}{1.544722in}}%
\pgfpathcurveto{\pgfqpoint{7.500022in}{1.537589in}}{\pgfqpoint{7.509697in}{1.533581in}}{\pgfqpoint{7.519784in}{1.533581in}}%
\pgfpathlineto{\pgfqpoint{7.519784in}{1.533581in}}%
\pgfpathclose%
\pgfusepath{stroke,fill}%
\end{pgfscope}%
\begin{pgfscope}%
\pgfpathrectangle{\pgfqpoint{6.818937in}{0.147348in}}{\pgfqpoint{2.735294in}{2.735294in}}%
\pgfusepath{clip}%
\pgfsetbuttcap%
\pgfsetroundjoin%
\definecolor{currentfill}{rgb}{0.839216,0.152941,0.156863}%
\pgfsetfillcolor{currentfill}%
\pgfsetfillopacity{0.661693}%
\pgfsetlinewidth{1.003750pt}%
\definecolor{currentstroke}{rgb}{0.839216,0.152941,0.156863}%
\pgfsetstrokecolor{currentstroke}%
\pgfsetstrokeopacity{0.661693}%
\pgfsetdash{}{0pt}%
\pgfpathmoveto{\pgfqpoint{7.793775in}{2.111490in}}%
\pgfpathcurveto{\pgfqpoint{7.803863in}{2.111490in}}{\pgfqpoint{7.813538in}{2.115498in}}{\pgfqpoint{7.820671in}{2.122631in}}%
\pgfpathcurveto{\pgfqpoint{7.827804in}{2.129763in}}{\pgfqpoint{7.831812in}{2.139439in}}{\pgfqpoint{7.831812in}{2.149526in}}%
\pgfpathcurveto{\pgfqpoint{7.831812in}{2.159614in}}{\pgfqpoint{7.827804in}{2.169289in}}{\pgfqpoint{7.820671in}{2.176422in}}%
\pgfpathcurveto{\pgfqpoint{7.813538in}{2.183555in}}{\pgfqpoint{7.803863in}{2.187563in}}{\pgfqpoint{7.793775in}{2.187563in}}%
\pgfpathcurveto{\pgfqpoint{7.783688in}{2.187563in}}{\pgfqpoint{7.774013in}{2.183555in}}{\pgfqpoint{7.766880in}{2.176422in}}%
\pgfpathcurveto{\pgfqpoint{7.759747in}{2.169289in}}{\pgfqpoint{7.755739in}{2.159614in}}{\pgfqpoint{7.755739in}{2.149526in}}%
\pgfpathcurveto{\pgfqpoint{7.755739in}{2.139439in}}{\pgfqpoint{7.759747in}{2.129763in}}{\pgfqpoint{7.766880in}{2.122631in}}%
\pgfpathcurveto{\pgfqpoint{7.774013in}{2.115498in}}{\pgfqpoint{7.783688in}{2.111490in}}{\pgfqpoint{7.793775in}{2.111490in}}%
\pgfpathlineto{\pgfqpoint{7.793775in}{2.111490in}}%
\pgfpathclose%
\pgfusepath{stroke,fill}%
\end{pgfscope}%
\begin{pgfscope}%
\pgfpathrectangle{\pgfqpoint{6.818937in}{0.147348in}}{\pgfqpoint{2.735294in}{2.735294in}}%
\pgfusepath{clip}%
\pgfsetbuttcap%
\pgfsetroundjoin%
\definecolor{currentfill}{rgb}{0.839216,0.152941,0.156863}%
\pgfsetfillcolor{currentfill}%
\pgfsetfillopacity{0.667667}%
\pgfsetlinewidth{1.003750pt}%
\definecolor{currentstroke}{rgb}{0.839216,0.152941,0.156863}%
\pgfsetstrokecolor{currentstroke}%
\pgfsetstrokeopacity{0.667667}%
\pgfsetdash{}{0pt}%
\pgfpathmoveto{\pgfqpoint{7.784633in}{2.171864in}}%
\pgfpathcurveto{\pgfqpoint{7.794720in}{2.171864in}}{\pgfqpoint{7.804396in}{2.175872in}}{\pgfqpoint{7.811529in}{2.183005in}}%
\pgfpathcurveto{\pgfqpoint{7.818661in}{2.190137in}}{\pgfqpoint{7.822669in}{2.199813in}}{\pgfqpoint{7.822669in}{2.209900in}}%
\pgfpathcurveto{\pgfqpoint{7.822669in}{2.219988in}}{\pgfqpoint{7.818661in}{2.229663in}}{\pgfqpoint{7.811529in}{2.236796in}}%
\pgfpathcurveto{\pgfqpoint{7.804396in}{2.243929in}}{\pgfqpoint{7.794720in}{2.247937in}}{\pgfqpoint{7.784633in}{2.247937in}}%
\pgfpathcurveto{\pgfqpoint{7.774546in}{2.247937in}}{\pgfqpoint{7.764870in}{2.243929in}}{\pgfqpoint{7.757737in}{2.236796in}}%
\pgfpathcurveto{\pgfqpoint{7.750604in}{2.229663in}}{\pgfqpoint{7.746597in}{2.219988in}}{\pgfqpoint{7.746597in}{2.209900in}}%
\pgfpathcurveto{\pgfqpoint{7.746597in}{2.199813in}}{\pgfqpoint{7.750604in}{2.190137in}}{\pgfqpoint{7.757737in}{2.183005in}}%
\pgfpathcurveto{\pgfqpoint{7.764870in}{2.175872in}}{\pgfqpoint{7.774546in}{2.171864in}}{\pgfqpoint{7.784633in}{2.171864in}}%
\pgfpathlineto{\pgfqpoint{7.784633in}{2.171864in}}%
\pgfpathclose%
\pgfusepath{stroke,fill}%
\end{pgfscope}%
\begin{pgfscope}%
\pgfpathrectangle{\pgfqpoint{6.818937in}{0.147348in}}{\pgfqpoint{2.735294in}{2.735294in}}%
\pgfusepath{clip}%
\pgfsetbuttcap%
\pgfsetroundjoin%
\definecolor{currentfill}{rgb}{0.839216,0.152941,0.156863}%
\pgfsetfillcolor{currentfill}%
\pgfsetfillopacity{0.680503}%
\pgfsetlinewidth{1.003750pt}%
\definecolor{currentstroke}{rgb}{0.839216,0.152941,0.156863}%
\pgfsetstrokecolor{currentstroke}%
\pgfsetstrokeopacity{0.680503}%
\pgfsetdash{}{0pt}%
\pgfpathmoveto{\pgfqpoint{8.674153in}{0.972903in}}%
\pgfpathcurveto{\pgfqpoint{8.684241in}{0.972903in}}{\pgfqpoint{8.693916in}{0.976910in}}{\pgfqpoint{8.701049in}{0.984043in}}%
\pgfpathcurveto{\pgfqpoint{8.708182in}{0.991176in}}{\pgfqpoint{8.712190in}{1.000852in}}{\pgfqpoint{8.712190in}{1.010939in}}%
\pgfpathcurveto{\pgfqpoint{8.712190in}{1.021026in}}{\pgfqpoint{8.708182in}{1.030702in}}{\pgfqpoint{8.701049in}{1.037835in}}%
\pgfpathcurveto{\pgfqpoint{8.693916in}{1.044968in}}{\pgfqpoint{8.684241in}{1.048975in}}{\pgfqpoint{8.674153in}{1.048975in}}%
\pgfpathcurveto{\pgfqpoint{8.664066in}{1.048975in}}{\pgfqpoint{8.654390in}{1.044968in}}{\pgfqpoint{8.647258in}{1.037835in}}%
\pgfpathcurveto{\pgfqpoint{8.640125in}{1.030702in}}{\pgfqpoint{8.636117in}{1.021026in}}{\pgfqpoint{8.636117in}{1.010939in}}%
\pgfpathcurveto{\pgfqpoint{8.636117in}{1.000852in}}{\pgfqpoint{8.640125in}{0.991176in}}{\pgfqpoint{8.647258in}{0.984043in}}%
\pgfpathcurveto{\pgfqpoint{8.654390in}{0.976910in}}{\pgfqpoint{8.664066in}{0.972903in}}{\pgfqpoint{8.674153in}{0.972903in}}%
\pgfpathlineto{\pgfqpoint{8.674153in}{0.972903in}}%
\pgfpathclose%
\pgfusepath{stroke,fill}%
\end{pgfscope}%
\begin{pgfscope}%
\pgfpathrectangle{\pgfqpoint{6.818937in}{0.147348in}}{\pgfqpoint{2.735294in}{2.735294in}}%
\pgfusepath{clip}%
\pgfsetbuttcap%
\pgfsetroundjoin%
\definecolor{currentfill}{rgb}{0.839216,0.152941,0.156863}%
\pgfsetfillcolor{currentfill}%
\pgfsetfillopacity{0.795933}%
\pgfsetlinewidth{1.003750pt}%
\definecolor{currentstroke}{rgb}{0.839216,0.152941,0.156863}%
\pgfsetstrokecolor{currentstroke}%
\pgfsetstrokeopacity{0.795933}%
\pgfsetdash{}{0pt}%
\pgfpathmoveto{\pgfqpoint{7.340626in}{1.623661in}}%
\pgfpathcurveto{\pgfqpoint{7.350714in}{1.623661in}}{\pgfqpoint{7.360389in}{1.627668in}}{\pgfqpoint{7.367522in}{1.634801in}}%
\pgfpathcurveto{\pgfqpoint{7.374655in}{1.641934in}}{\pgfqpoint{7.378663in}{1.651609in}}{\pgfqpoint{7.378663in}{1.661697in}}%
\pgfpathcurveto{\pgfqpoint{7.378663in}{1.671784in}}{\pgfqpoint{7.374655in}{1.681460in}}{\pgfqpoint{7.367522in}{1.688593in}}%
\pgfpathcurveto{\pgfqpoint{7.360389in}{1.695725in}}{\pgfqpoint{7.350714in}{1.699733in}}{\pgfqpoint{7.340626in}{1.699733in}}%
\pgfpathcurveto{\pgfqpoint{7.330539in}{1.699733in}}{\pgfqpoint{7.320863in}{1.695725in}}{\pgfqpoint{7.313731in}{1.688593in}}%
\pgfpathcurveto{\pgfqpoint{7.306598in}{1.681460in}}{\pgfqpoint{7.302590in}{1.671784in}}{\pgfqpoint{7.302590in}{1.661697in}}%
\pgfpathcurveto{\pgfqpoint{7.302590in}{1.651609in}}{\pgfqpoint{7.306598in}{1.641934in}}{\pgfqpoint{7.313731in}{1.634801in}}%
\pgfpathcurveto{\pgfqpoint{7.320863in}{1.627668in}}{\pgfqpoint{7.330539in}{1.623661in}}{\pgfqpoint{7.340626in}{1.623661in}}%
\pgfpathlineto{\pgfqpoint{7.340626in}{1.623661in}}%
\pgfpathclose%
\pgfusepath{stroke,fill}%
\end{pgfscope}%
\begin{pgfscope}%
\pgfpathrectangle{\pgfqpoint{6.818937in}{0.147348in}}{\pgfqpoint{2.735294in}{2.735294in}}%
\pgfusepath{clip}%
\pgfsetbuttcap%
\pgfsetroundjoin%
\definecolor{currentfill}{rgb}{0.839216,0.152941,0.156863}%
\pgfsetfillcolor{currentfill}%
\pgfsetfillopacity{0.802324}%
\pgfsetlinewidth{1.003750pt}%
\definecolor{currentstroke}{rgb}{0.839216,0.152941,0.156863}%
\pgfsetstrokecolor{currentstroke}%
\pgfsetstrokeopacity{0.802324}%
\pgfsetdash{}{0pt}%
\pgfpathmoveto{\pgfqpoint{9.019142in}{1.511610in}}%
\pgfpathcurveto{\pgfqpoint{9.029229in}{1.511610in}}{\pgfqpoint{9.038905in}{1.515617in}}{\pgfqpoint{9.046038in}{1.522750in}}%
\pgfpathcurveto{\pgfqpoint{9.053171in}{1.529883in}}{\pgfqpoint{9.057178in}{1.539559in}}{\pgfqpoint{9.057178in}{1.549646in}}%
\pgfpathcurveto{\pgfqpoint{9.057178in}{1.559733in}}{\pgfqpoint{9.053171in}{1.569409in}}{\pgfqpoint{9.046038in}{1.576542in}}%
\pgfpathcurveto{\pgfqpoint{9.038905in}{1.583675in}}{\pgfqpoint{9.029229in}{1.587682in}}{\pgfqpoint{9.019142in}{1.587682in}}%
\pgfpathcurveto{\pgfqpoint{9.009055in}{1.587682in}}{\pgfqpoint{8.999379in}{1.583675in}}{\pgfqpoint{8.992246in}{1.576542in}}%
\pgfpathcurveto{\pgfqpoint{8.985114in}{1.569409in}}{\pgfqpoint{8.981106in}{1.559733in}}{\pgfqpoint{8.981106in}{1.549646in}}%
\pgfpathcurveto{\pgfqpoint{8.981106in}{1.539559in}}{\pgfqpoint{8.985114in}{1.529883in}}{\pgfqpoint{8.992246in}{1.522750in}}%
\pgfpathcurveto{\pgfqpoint{8.999379in}{1.515617in}}{\pgfqpoint{9.009055in}{1.511610in}}{\pgfqpoint{9.019142in}{1.511610in}}%
\pgfpathlineto{\pgfqpoint{9.019142in}{1.511610in}}%
\pgfpathclose%
\pgfusepath{stroke,fill}%
\end{pgfscope}%
\begin{pgfscope}%
\pgfpathrectangle{\pgfqpoint{6.818937in}{0.147348in}}{\pgfqpoint{2.735294in}{2.735294in}}%
\pgfusepath{clip}%
\pgfsetbuttcap%
\pgfsetroundjoin%
\definecolor{currentfill}{rgb}{0.839216,0.152941,0.156863}%
\pgfsetfillcolor{currentfill}%
\pgfsetfillopacity{0.837755}%
\pgfsetlinewidth{1.003750pt}%
\definecolor{currentstroke}{rgb}{0.839216,0.152941,0.156863}%
\pgfsetstrokecolor{currentstroke}%
\pgfsetstrokeopacity{0.837755}%
\pgfsetdash{}{0pt}%
\pgfpathmoveto{\pgfqpoint{8.690520in}{0.950898in}}%
\pgfpathcurveto{\pgfqpoint{8.700607in}{0.950898in}}{\pgfqpoint{8.710282in}{0.954905in}}{\pgfqpoint{8.717415in}{0.962038in}}%
\pgfpathcurveto{\pgfqpoint{8.724548in}{0.969171in}}{\pgfqpoint{8.728556in}{0.978846in}}{\pgfqpoint{8.728556in}{0.988934in}}%
\pgfpathcurveto{\pgfqpoint{8.728556in}{0.999021in}}{\pgfqpoint{8.724548in}{1.008697in}}{\pgfqpoint{8.717415in}{1.015830in}}%
\pgfpathcurveto{\pgfqpoint{8.710282in}{1.022962in}}{\pgfqpoint{8.700607in}{1.026970in}}{\pgfqpoint{8.690520in}{1.026970in}}%
\pgfpathcurveto{\pgfqpoint{8.680432in}{1.026970in}}{\pgfqpoint{8.670757in}{1.022962in}}{\pgfqpoint{8.663624in}{1.015830in}}%
\pgfpathcurveto{\pgfqpoint{8.656491in}{1.008697in}}{\pgfqpoint{8.652483in}{0.999021in}}{\pgfqpoint{8.652483in}{0.988934in}}%
\pgfpathcurveto{\pgfqpoint{8.652483in}{0.978846in}}{\pgfqpoint{8.656491in}{0.969171in}}{\pgfqpoint{8.663624in}{0.962038in}}%
\pgfpathcurveto{\pgfqpoint{8.670757in}{0.954905in}}{\pgfqpoint{8.680432in}{0.950898in}}{\pgfqpoint{8.690520in}{0.950898in}}%
\pgfpathlineto{\pgfqpoint{8.690520in}{0.950898in}}%
\pgfpathclose%
\pgfusepath{stroke,fill}%
\end{pgfscope}%
\begin{pgfscope}%
\pgfpathrectangle{\pgfqpoint{6.818937in}{0.147348in}}{\pgfqpoint{2.735294in}{2.735294in}}%
\pgfusepath{clip}%
\pgfsetbuttcap%
\pgfsetroundjoin%
\definecolor{currentfill}{rgb}{0.839216,0.152941,0.156863}%
\pgfsetfillcolor{currentfill}%
\pgfsetfillopacity{0.913537}%
\pgfsetlinewidth{1.003750pt}%
\definecolor{currentstroke}{rgb}{0.839216,0.152941,0.156863}%
\pgfsetstrokecolor{currentstroke}%
\pgfsetstrokeopacity{0.913537}%
\pgfsetdash{}{0pt}%
\pgfpathmoveto{\pgfqpoint{8.852391in}{0.996845in}}%
\pgfpathcurveto{\pgfqpoint{8.862478in}{0.996845in}}{\pgfqpoint{8.872154in}{1.000853in}}{\pgfqpoint{8.879286in}{1.007986in}}%
\pgfpathcurveto{\pgfqpoint{8.886419in}{1.015119in}}{\pgfqpoint{8.890427in}{1.024794in}}{\pgfqpoint{8.890427in}{1.034882in}}%
\pgfpathcurveto{\pgfqpoint{8.890427in}{1.044969in}}{\pgfqpoint{8.886419in}{1.054645in}}{\pgfqpoint{8.879286in}{1.061777in}}%
\pgfpathcurveto{\pgfqpoint{8.872154in}{1.068910in}}{\pgfqpoint{8.862478in}{1.072918in}}{\pgfqpoint{8.852391in}{1.072918in}}%
\pgfpathcurveto{\pgfqpoint{8.842303in}{1.072918in}}{\pgfqpoint{8.832628in}{1.068910in}}{\pgfqpoint{8.825495in}{1.061777in}}%
\pgfpathcurveto{\pgfqpoint{8.818362in}{1.054645in}}{\pgfqpoint{8.814354in}{1.044969in}}{\pgfqpoint{8.814354in}{1.034882in}}%
\pgfpathcurveto{\pgfqpoint{8.814354in}{1.024794in}}{\pgfqpoint{8.818362in}{1.015119in}}{\pgfqpoint{8.825495in}{1.007986in}}%
\pgfpathcurveto{\pgfqpoint{8.832628in}{1.000853in}}{\pgfqpoint{8.842303in}{0.996845in}}{\pgfqpoint{8.852391in}{0.996845in}}%
\pgfpathlineto{\pgfqpoint{8.852391in}{0.996845in}}%
\pgfpathclose%
\pgfusepath{stroke,fill}%
\end{pgfscope}%
\begin{pgfscope}%
\pgfpathrectangle{\pgfqpoint{6.818937in}{0.147348in}}{\pgfqpoint{2.735294in}{2.735294in}}%
\pgfusepath{clip}%
\pgfsetbuttcap%
\pgfsetroundjoin%
\definecolor{currentfill}{rgb}{0.839216,0.152941,0.156863}%
\pgfsetfillcolor{currentfill}%
\pgfsetlinewidth{1.003750pt}%
\definecolor{currentstroke}{rgb}{0.839216,0.152941,0.156863}%
\pgfsetstrokecolor{currentstroke}%
\pgfsetdash{}{0pt}%
\pgfpathmoveto{\pgfqpoint{8.834855in}{0.927831in}}%
\pgfpathcurveto{\pgfqpoint{8.844942in}{0.927831in}}{\pgfqpoint{8.854618in}{0.931838in}}{\pgfqpoint{8.861751in}{0.938971in}}%
\pgfpathcurveto{\pgfqpoint{8.868883in}{0.946104in}}{\pgfqpoint{8.872891in}{0.955779in}}{\pgfqpoint{8.872891in}{0.965867in}}%
\pgfpathcurveto{\pgfqpoint{8.872891in}{0.975954in}}{\pgfqpoint{8.868883in}{0.985630in}}{\pgfqpoint{8.861751in}{0.992763in}}%
\pgfpathcurveto{\pgfqpoint{8.854618in}{0.999895in}}{\pgfqpoint{8.844942in}{1.003903in}}{\pgfqpoint{8.834855in}{1.003903in}}%
\pgfpathcurveto{\pgfqpoint{8.824768in}{1.003903in}}{\pgfqpoint{8.815092in}{0.999895in}}{\pgfqpoint{8.807959in}{0.992763in}}%
\pgfpathcurveto{\pgfqpoint{8.800826in}{0.985630in}}{\pgfqpoint{8.796819in}{0.975954in}}{\pgfqpoint{8.796819in}{0.965867in}}%
\pgfpathcurveto{\pgfqpoint{8.796819in}{0.955779in}}{\pgfqpoint{8.800826in}{0.946104in}}{\pgfqpoint{8.807959in}{0.938971in}}%
\pgfpathcurveto{\pgfqpoint{8.815092in}{0.931838in}}{\pgfqpoint{8.824768in}{0.927831in}}{\pgfqpoint{8.834855in}{0.927831in}}%
\pgfpathlineto{\pgfqpoint{8.834855in}{0.927831in}}%
\pgfpathclose%
\pgfusepath{stroke,fill}%
\end{pgfscope}%
\begin{pgfscope}%
\pgfpathrectangle{\pgfqpoint{6.818937in}{0.147348in}}{\pgfqpoint{2.735294in}{2.735294in}}%
\pgfusepath{clip}%
\pgfsetbuttcap%
\pgfsetroundjoin%
\definecolor{currentfill}{rgb}{0.074668,0.271519,0.074668}%
\pgfsetfillcolor{currentfill}%
\pgfsetfillopacity{0.200000}%
\pgfsetlinewidth{0.000000pt}%
\definecolor{currentstroke}{rgb}{0.000000,0.000000,0.000000}%
\pgfsetstrokecolor{currentstroke}%
\pgfsetdash{}{0pt}%
\pgfpathmoveto{\pgfqpoint{8.527601in}{1.505058in}}%
\pgfpathlineto{\pgfqpoint{8.375641in}{1.283420in}}%
\pgfpathlineto{\pgfqpoint{8.223548in}{1.496830in}}%
\pgfpathlineto{\pgfqpoint{8.527601in}{1.505058in}}%
\pgfpathclose%
\pgfusepath{fill}%
\end{pgfscope}%
\begin{pgfscope}%
\pgfpathrectangle{\pgfqpoint{6.818937in}{0.147348in}}{\pgfqpoint{2.735294in}{2.735294in}}%
\pgfusepath{clip}%
\pgfsetbuttcap%
\pgfsetroundjoin%
\definecolor{currentfill}{rgb}{0.074668,0.271519,0.074668}%
\pgfsetfillcolor{currentfill}%
\pgfsetfillopacity{0.200000}%
\pgfsetlinewidth{0.000000pt}%
\definecolor{currentstroke}{rgb}{0.000000,0.000000,0.000000}%
\pgfsetstrokecolor{currentstroke}%
\pgfsetdash{}{0pt}%
\pgfpathmoveto{\pgfqpoint{8.223548in}{1.496830in}}%
\pgfpathlineto{\pgfqpoint{8.071454in}{1.283420in}}%
\pgfpathlineto{\pgfqpoint{7.919495in}{1.505058in}}%
\pgfpathlineto{\pgfqpoint{8.223548in}{1.496830in}}%
\pgfpathclose%
\pgfusepath{fill}%
\end{pgfscope}%
\begin{pgfscope}%
\pgfpathrectangle{\pgfqpoint{6.818937in}{0.147348in}}{\pgfqpoint{2.735294in}{2.735294in}}%
\pgfusepath{clip}%
\pgfsetbuttcap%
\pgfsetroundjoin%
\definecolor{currentfill}{rgb}{0.086258,0.313666,0.086258}%
\pgfsetfillcolor{currentfill}%
\pgfsetfillopacity{0.200000}%
\pgfsetlinewidth{0.000000pt}%
\definecolor{currentstroke}{rgb}{0.000000,0.000000,0.000000}%
\pgfsetstrokecolor{currentstroke}%
\pgfsetdash{}{0pt}%
\pgfpathmoveto{\pgfqpoint{7.919495in}{1.505058in}}%
\pgfpathlineto{\pgfqpoint{8.223548in}{1.947612in}}%
\pgfpathlineto{\pgfqpoint{8.223548in}{1.496830in}}%
\pgfpathlineto{\pgfqpoint{7.919495in}{1.505058in}}%
\pgfpathclose%
\pgfusepath{fill}%
\end{pgfscope}%
\begin{pgfscope}%
\pgfpathrectangle{\pgfqpoint{6.818937in}{0.147348in}}{\pgfqpoint{2.735294in}{2.735294in}}%
\pgfusepath{clip}%
\pgfsetbuttcap%
\pgfsetroundjoin%
\definecolor{currentfill}{rgb}{0.086258,0.313666,0.086258}%
\pgfsetfillcolor{currentfill}%
\pgfsetfillopacity{0.200000}%
\pgfsetlinewidth{0.000000pt}%
\definecolor{currentstroke}{rgb}{0.000000,0.000000,0.000000}%
\pgfsetstrokecolor{currentstroke}%
\pgfsetdash{}{0pt}%
\pgfpathmoveto{\pgfqpoint{8.223548in}{1.496830in}}%
\pgfpathlineto{\pgfqpoint{8.223548in}{1.947612in}}%
\pgfpathlineto{\pgfqpoint{8.527601in}{1.505058in}}%
\pgfpathlineto{\pgfqpoint{8.223548in}{1.496830in}}%
\pgfpathclose%
\pgfusepath{fill}%
\end{pgfscope}%
\begin{pgfscope}%
\pgfpathrectangle{\pgfqpoint{6.818937in}{0.147348in}}{\pgfqpoint{2.735294in}{2.735294in}}%
\pgfusepath{clip}%
\pgfsetbuttcap%
\pgfsetroundjoin%
\definecolor{currentfill}{rgb}{0.086061,0.312950,0.086061}%
\pgfsetfillcolor{currentfill}%
\pgfsetfillopacity{0.200000}%
\pgfsetlinewidth{0.000000pt}%
\definecolor{currentstroke}{rgb}{0.000000,0.000000,0.000000}%
\pgfsetstrokecolor{currentstroke}%
\pgfsetdash{}{0pt}%
\pgfpathmoveto{\pgfqpoint{8.223548in}{1.947612in}}%
\pgfpathlineto{\pgfqpoint{7.919495in}{1.505058in}}%
\pgfpathlineto{\pgfqpoint{7.935071in}{1.947550in}}%
\pgfpathlineto{\pgfqpoint{8.223548in}{1.947612in}}%
\pgfpathclose%
\pgfusepath{fill}%
\end{pgfscope}%
\begin{pgfscope}%
\pgfpathrectangle{\pgfqpoint{6.818937in}{0.147348in}}{\pgfqpoint{2.735294in}{2.735294in}}%
\pgfusepath{clip}%
\pgfsetbuttcap%
\pgfsetroundjoin%
\definecolor{currentfill}{rgb}{0.086061,0.312950,0.086061}%
\pgfsetfillcolor{currentfill}%
\pgfsetfillopacity{0.200000}%
\pgfsetlinewidth{0.000000pt}%
\definecolor{currentstroke}{rgb}{0.000000,0.000000,0.000000}%
\pgfsetstrokecolor{currentstroke}%
\pgfsetdash{}{0pt}%
\pgfpathmoveto{\pgfqpoint{8.512025in}{1.947550in}}%
\pgfpathlineto{\pgfqpoint{8.527601in}{1.505058in}}%
\pgfpathlineto{\pgfqpoint{8.223548in}{1.947612in}}%
\pgfpathlineto{\pgfqpoint{8.512025in}{1.947550in}}%
\pgfpathclose%
\pgfusepath{fill}%
\end{pgfscope}%
\begin{pgfscope}%
\pgfpathrectangle{\pgfqpoint{6.818937in}{0.147348in}}{\pgfqpoint{2.735294in}{2.735294in}}%
\pgfusepath{clip}%
\pgfsetbuttcap%
\pgfsetroundjoin%
\definecolor{currentfill}{rgb}{0.075994,0.276341,0.075994}%
\pgfsetfillcolor{currentfill}%
\pgfsetfillopacity{0.200000}%
\pgfsetlinewidth{0.000000pt}%
\definecolor{currentstroke}{rgb}{0.000000,0.000000,0.000000}%
\pgfsetstrokecolor{currentstroke}%
\pgfsetdash{}{0pt}%
\pgfpathmoveto{\pgfqpoint{8.800986in}{1.527187in}}%
\pgfpathlineto{\pgfqpoint{8.664078in}{1.306218in}}%
\pgfpathlineto{\pgfqpoint{8.527601in}{1.505058in}}%
\pgfpathlineto{\pgfqpoint{8.800986in}{1.527187in}}%
\pgfpathclose%
\pgfusepath{fill}%
\end{pgfscope}%
\begin{pgfscope}%
\pgfpathrectangle{\pgfqpoint{6.818937in}{0.147348in}}{\pgfqpoint{2.735294in}{2.735294in}}%
\pgfusepath{clip}%
\pgfsetbuttcap%
\pgfsetroundjoin%
\definecolor{currentfill}{rgb}{0.075994,0.276341,0.075994}%
\pgfsetfillcolor{currentfill}%
\pgfsetfillopacity{0.200000}%
\pgfsetlinewidth{0.000000pt}%
\definecolor{currentstroke}{rgb}{0.000000,0.000000,0.000000}%
\pgfsetstrokecolor{currentstroke}%
\pgfsetdash{}{0pt}%
\pgfpathmoveto{\pgfqpoint{7.919495in}{1.505058in}}%
\pgfpathlineto{\pgfqpoint{7.783018in}{1.306218in}}%
\pgfpathlineto{\pgfqpoint{7.646110in}{1.527187in}}%
\pgfpathlineto{\pgfqpoint{7.919495in}{1.505058in}}%
\pgfpathclose%
\pgfusepath{fill}%
\end{pgfscope}%
\begin{pgfscope}%
\pgfpathrectangle{\pgfqpoint{6.818937in}{0.147348in}}{\pgfqpoint{2.735294in}{2.735294in}}%
\pgfusepath{clip}%
\pgfsetbuttcap%
\pgfsetroundjoin%
\definecolor{currentfill}{rgb}{0.087398,0.317812,0.087398}%
\pgfsetfillcolor{currentfill}%
\pgfsetfillopacity{0.200000}%
\pgfsetlinewidth{0.000000pt}%
\definecolor{currentstroke}{rgb}{0.000000,0.000000,0.000000}%
\pgfsetstrokecolor{currentstroke}%
\pgfsetdash{}{0pt}%
\pgfpathmoveto{\pgfqpoint{8.527601in}{1.505058in}}%
\pgfpathlineto{\pgfqpoint{8.512025in}{1.947550in}}%
\pgfpathlineto{\pgfqpoint{8.800986in}{1.527187in}}%
\pgfpathlineto{\pgfqpoint{8.527601in}{1.505058in}}%
\pgfpathclose%
\pgfusepath{fill}%
\end{pgfscope}%
\begin{pgfscope}%
\pgfpathrectangle{\pgfqpoint{6.818937in}{0.147348in}}{\pgfqpoint{2.735294in}{2.735294in}}%
\pgfusepath{clip}%
\pgfsetbuttcap%
\pgfsetroundjoin%
\definecolor{currentfill}{rgb}{0.087398,0.317812,0.087398}%
\pgfsetfillcolor{currentfill}%
\pgfsetfillopacity{0.200000}%
\pgfsetlinewidth{0.000000pt}%
\definecolor{currentstroke}{rgb}{0.000000,0.000000,0.000000}%
\pgfsetstrokecolor{currentstroke}%
\pgfsetdash{}{0pt}%
\pgfpathmoveto{\pgfqpoint{7.646110in}{1.527187in}}%
\pgfpathlineto{\pgfqpoint{7.935071in}{1.947550in}}%
\pgfpathlineto{\pgfqpoint{7.919495in}{1.505058in}}%
\pgfpathlineto{\pgfqpoint{7.646110in}{1.527187in}}%
\pgfpathclose%
\pgfusepath{fill}%
\end{pgfscope}%
\begin{pgfscope}%
\pgfpathrectangle{\pgfqpoint{6.818937in}{0.147348in}}{\pgfqpoint{2.735294in}{2.735294in}}%
\pgfusepath{clip}%
\pgfsetbuttcap%
\pgfsetroundjoin%
\definecolor{currentfill}{rgb}{0.070209,0.255305,0.070209}%
\pgfsetfillcolor{currentfill}%
\pgfsetfillopacity{0.200000}%
\pgfsetlinewidth{0.000000pt}%
\definecolor{currentstroke}{rgb}{0.000000,0.000000,0.000000}%
\pgfsetstrokecolor{currentstroke}%
\pgfsetdash{}{0pt}%
\pgfpathmoveto{\pgfqpoint{8.625616in}{0.971623in}}%
\pgfpathlineto{\pgfqpoint{8.375641in}{1.283420in}}%
\pgfpathlineto{\pgfqpoint{8.527601in}{1.505058in}}%
\pgfpathlineto{\pgfqpoint{8.625616in}{0.971623in}}%
\pgfpathclose%
\pgfusepath{fill}%
\end{pgfscope}%
\begin{pgfscope}%
\pgfpathrectangle{\pgfqpoint{6.818937in}{0.147348in}}{\pgfqpoint{2.735294in}{2.735294in}}%
\pgfusepath{clip}%
\pgfsetbuttcap%
\pgfsetroundjoin%
\definecolor{currentfill}{rgb}{0.070209,0.255305,0.070209}%
\pgfsetfillcolor{currentfill}%
\pgfsetfillopacity{0.200000}%
\pgfsetlinewidth{0.000000pt}%
\definecolor{currentstroke}{rgb}{0.000000,0.000000,0.000000}%
\pgfsetstrokecolor{currentstroke}%
\pgfsetdash{}{0pt}%
\pgfpathmoveto{\pgfqpoint{7.919495in}{1.505058in}}%
\pgfpathlineto{\pgfqpoint{8.071454in}{1.283420in}}%
\pgfpathlineto{\pgfqpoint{7.821480in}{0.971623in}}%
\pgfpathlineto{\pgfqpoint{7.919495in}{1.505058in}}%
\pgfpathclose%
\pgfusepath{fill}%
\end{pgfscope}%
\begin{pgfscope}%
\pgfpathrectangle{\pgfqpoint{6.818937in}{0.147348in}}{\pgfqpoint{2.735294in}{2.735294in}}%
\pgfusepath{clip}%
\pgfsetbuttcap%
\pgfsetroundjoin%
\definecolor{currentfill}{rgb}{0.098306,0.357475,0.098306}%
\pgfsetfillcolor{currentfill}%
\pgfsetfillopacity{0.200000}%
\pgfsetlinewidth{0.000000pt}%
\definecolor{currentstroke}{rgb}{0.000000,0.000000,0.000000}%
\pgfsetstrokecolor{currentstroke}%
\pgfsetdash{}{0pt}%
\pgfpathmoveto{\pgfqpoint{8.223548in}{1.947612in}}%
\pgfpathlineto{\pgfqpoint{8.361157in}{2.147735in}}%
\pgfpathlineto{\pgfqpoint{8.512025in}{1.947550in}}%
\pgfpathlineto{\pgfqpoint{8.223548in}{1.947612in}}%
\pgfpathclose%
\pgfusepath{fill}%
\end{pgfscope}%
\begin{pgfscope}%
\pgfpathrectangle{\pgfqpoint{6.818937in}{0.147348in}}{\pgfqpoint{2.735294in}{2.735294in}}%
\pgfusepath{clip}%
\pgfsetbuttcap%
\pgfsetroundjoin%
\definecolor{currentfill}{rgb}{0.098306,0.357475,0.098306}%
\pgfsetfillcolor{currentfill}%
\pgfsetfillopacity{0.200000}%
\pgfsetlinewidth{0.000000pt}%
\definecolor{currentstroke}{rgb}{0.000000,0.000000,0.000000}%
\pgfsetstrokecolor{currentstroke}%
\pgfsetdash{}{0pt}%
\pgfpathmoveto{\pgfqpoint{7.935071in}{1.947550in}}%
\pgfpathlineto{\pgfqpoint{8.085938in}{2.147735in}}%
\pgfpathlineto{\pgfqpoint{8.223548in}{1.947612in}}%
\pgfpathlineto{\pgfqpoint{7.935071in}{1.947550in}}%
\pgfpathclose%
\pgfusepath{fill}%
\end{pgfscope}%
\begin{pgfscope}%
\pgfpathrectangle{\pgfqpoint{6.818937in}{0.147348in}}{\pgfqpoint{2.735294in}{2.735294in}}%
\pgfusepath{clip}%
\pgfsetbuttcap%
\pgfsetroundjoin%
\definecolor{currentfill}{rgb}{0.066446,0.241622,0.066446}%
\pgfsetfillcolor{currentfill}%
\pgfsetfillopacity{0.200000}%
\pgfsetlinewidth{0.000000pt}%
\definecolor{currentstroke}{rgb}{0.000000,0.000000,0.000000}%
\pgfsetstrokecolor{currentstroke}%
\pgfsetdash{}{0pt}%
\pgfpathmoveto{\pgfqpoint{8.375641in}{1.283420in}}%
\pgfpathlineto{\pgfqpoint{8.223548in}{0.822079in}}%
\pgfpathlineto{\pgfqpoint{8.223548in}{1.496830in}}%
\pgfpathlineto{\pgfqpoint{8.375641in}{1.283420in}}%
\pgfpathclose%
\pgfusepath{fill}%
\end{pgfscope}%
\begin{pgfscope}%
\pgfpathrectangle{\pgfqpoint{6.818937in}{0.147348in}}{\pgfqpoint{2.735294in}{2.735294in}}%
\pgfusepath{clip}%
\pgfsetbuttcap%
\pgfsetroundjoin%
\definecolor{currentfill}{rgb}{0.066446,0.241622,0.066446}%
\pgfsetfillcolor{currentfill}%
\pgfsetfillopacity{0.200000}%
\pgfsetlinewidth{0.000000pt}%
\definecolor{currentstroke}{rgb}{0.000000,0.000000,0.000000}%
\pgfsetstrokecolor{currentstroke}%
\pgfsetdash{}{0pt}%
\pgfpathmoveto{\pgfqpoint{8.223548in}{1.496830in}}%
\pgfpathlineto{\pgfqpoint{8.223548in}{0.822079in}}%
\pgfpathlineto{\pgfqpoint{8.071454in}{1.283420in}}%
\pgfpathlineto{\pgfqpoint{8.223548in}{1.496830in}}%
\pgfpathclose%
\pgfusepath{fill}%
\end{pgfscope}%
\begin{pgfscope}%
\pgfpathrectangle{\pgfqpoint{6.818937in}{0.147348in}}{\pgfqpoint{2.735294in}{2.735294in}}%
\pgfusepath{clip}%
\pgfsetbuttcap%
\pgfsetroundjoin%
\definecolor{currentfill}{rgb}{0.065035,0.236492,0.065035}%
\pgfsetfillcolor{currentfill}%
\pgfsetfillopacity{0.200000}%
\pgfsetlinewidth{0.000000pt}%
\definecolor{currentstroke}{rgb}{0.000000,0.000000,0.000000}%
\pgfsetstrokecolor{currentstroke}%
\pgfsetdash{}{0pt}%
\pgfpathmoveto{\pgfqpoint{8.625616in}{0.971623in}}%
\pgfpathlineto{\pgfqpoint{8.527601in}{1.505058in}}%
\pgfpathlineto{\pgfqpoint{8.664078in}{1.306218in}}%
\pgfpathlineto{\pgfqpoint{8.625616in}{0.971623in}}%
\pgfpathclose%
\pgfusepath{fill}%
\end{pgfscope}%
\begin{pgfscope}%
\pgfpathrectangle{\pgfqpoint{6.818937in}{0.147348in}}{\pgfqpoint{2.735294in}{2.735294in}}%
\pgfusepath{clip}%
\pgfsetbuttcap%
\pgfsetroundjoin%
\definecolor{currentfill}{rgb}{0.065035,0.236492,0.065035}%
\pgfsetfillcolor{currentfill}%
\pgfsetfillopacity{0.200000}%
\pgfsetlinewidth{0.000000pt}%
\definecolor{currentstroke}{rgb}{0.000000,0.000000,0.000000}%
\pgfsetstrokecolor{currentstroke}%
\pgfsetdash{}{0pt}%
\pgfpathmoveto{\pgfqpoint{7.783018in}{1.306218in}}%
\pgfpathlineto{\pgfqpoint{7.919495in}{1.505058in}}%
\pgfpathlineto{\pgfqpoint{7.821480in}{0.971623in}}%
\pgfpathlineto{\pgfqpoint{7.783018in}{1.306218in}}%
\pgfpathclose%
\pgfusepath{fill}%
\end{pgfscope}%
\begin{pgfscope}%
\pgfpathrectangle{\pgfqpoint{6.818937in}{0.147348in}}{\pgfqpoint{2.735294in}{2.735294in}}%
\pgfusepath{clip}%
\pgfsetbuttcap%
\pgfsetroundjoin%
\definecolor{currentfill}{rgb}{0.101677,0.369734,0.101677}%
\pgfsetfillcolor{currentfill}%
\pgfsetfillopacity{0.200000}%
\pgfsetlinewidth{0.000000pt}%
\definecolor{currentstroke}{rgb}{0.000000,0.000000,0.000000}%
\pgfsetstrokecolor{currentstroke}%
\pgfsetdash{}{0pt}%
\pgfpathmoveto{\pgfqpoint{8.223548in}{1.947612in}}%
\pgfpathlineto{\pgfqpoint{8.085938in}{2.147735in}}%
\pgfpathlineto{\pgfqpoint{8.361157in}{2.147735in}}%
\pgfpathlineto{\pgfqpoint{8.223548in}{1.947612in}}%
\pgfpathclose%
\pgfusepath{fill}%
\end{pgfscope}%
\begin{pgfscope}%
\pgfpathrectangle{\pgfqpoint{6.818937in}{0.147348in}}{\pgfqpoint{2.735294in}{2.735294in}}%
\pgfusepath{clip}%
\pgfsetbuttcap%
\pgfsetroundjoin%
\definecolor{currentfill}{rgb}{0.101759,0.370033,0.101759}%
\pgfsetfillcolor{currentfill}%
\pgfsetfillopacity{0.200000}%
\pgfsetlinewidth{0.000000pt}%
\definecolor{currentstroke}{rgb}{0.000000,0.000000,0.000000}%
\pgfsetstrokecolor{currentstroke}%
\pgfsetdash{}{0pt}%
\pgfpathmoveto{\pgfqpoint{8.512025in}{1.947550in}}%
\pgfpathlineto{\pgfqpoint{8.361157in}{2.147735in}}%
\pgfpathlineto{\pgfqpoint{8.624680in}{2.141959in}}%
\pgfpathlineto{\pgfqpoint{8.512025in}{1.947550in}}%
\pgfpathclose%
\pgfusepath{fill}%
\end{pgfscope}%
\begin{pgfscope}%
\pgfpathrectangle{\pgfqpoint{6.818937in}{0.147348in}}{\pgfqpoint{2.735294in}{2.735294in}}%
\pgfusepath{clip}%
\pgfsetbuttcap%
\pgfsetroundjoin%
\definecolor{currentfill}{rgb}{0.101759,0.370033,0.101759}%
\pgfsetfillcolor{currentfill}%
\pgfsetfillopacity{0.200000}%
\pgfsetlinewidth{0.000000pt}%
\definecolor{currentstroke}{rgb}{0.000000,0.000000,0.000000}%
\pgfsetstrokecolor{currentstroke}%
\pgfsetdash{}{0pt}%
\pgfpathmoveto{\pgfqpoint{7.822416in}{2.141959in}}%
\pgfpathlineto{\pgfqpoint{8.085938in}{2.147735in}}%
\pgfpathlineto{\pgfqpoint{7.935071in}{1.947550in}}%
\pgfpathlineto{\pgfqpoint{7.822416in}{2.141959in}}%
\pgfpathclose%
\pgfusepath{fill}%
\end{pgfscope}%
\begin{pgfscope}%
\pgfpathrectangle{\pgfqpoint{6.818937in}{0.147348in}}{\pgfqpoint{2.735294in}{2.735294in}}%
\pgfusepath{clip}%
\pgfsetbuttcap%
\pgfsetroundjoin%
\definecolor{currentfill}{rgb}{0.091915,0.334238,0.091915}%
\pgfsetfillcolor{currentfill}%
\pgfsetfillopacity{0.200000}%
\pgfsetlinewidth{0.000000pt}%
\definecolor{currentstroke}{rgb}{0.000000,0.000000,0.000000}%
\pgfsetstrokecolor{currentstroke}%
\pgfsetdash{}{0pt}%
\pgfpathmoveto{\pgfqpoint{8.800986in}{1.527187in}}%
\pgfpathlineto{\pgfqpoint{8.512025in}{1.947550in}}%
\pgfpathlineto{\pgfqpoint{8.857732in}{2.131775in}}%
\pgfpathlineto{\pgfqpoint{8.800986in}{1.527187in}}%
\pgfpathclose%
\pgfusepath{fill}%
\end{pgfscope}%
\begin{pgfscope}%
\pgfpathrectangle{\pgfqpoint{6.818937in}{0.147348in}}{\pgfqpoint{2.735294in}{2.735294in}}%
\pgfusepath{clip}%
\pgfsetbuttcap%
\pgfsetroundjoin%
\definecolor{currentfill}{rgb}{0.091915,0.334238,0.091915}%
\pgfsetfillcolor{currentfill}%
\pgfsetfillopacity{0.200000}%
\pgfsetlinewidth{0.000000pt}%
\definecolor{currentstroke}{rgb}{0.000000,0.000000,0.000000}%
\pgfsetstrokecolor{currentstroke}%
\pgfsetdash{}{0pt}%
\pgfpathmoveto{\pgfqpoint{7.589364in}{2.131775in}}%
\pgfpathlineto{\pgfqpoint{7.935071in}{1.947550in}}%
\pgfpathlineto{\pgfqpoint{7.646110in}{1.527187in}}%
\pgfpathlineto{\pgfqpoint{7.589364in}{2.131775in}}%
\pgfpathclose%
\pgfusepath{fill}%
\end{pgfscope}%
\begin{pgfscope}%
\pgfpathrectangle{\pgfqpoint{6.818937in}{0.147348in}}{\pgfqpoint{2.735294in}{2.735294in}}%
\pgfusepath{clip}%
\pgfsetbuttcap%
\pgfsetroundjoin%
\definecolor{currentfill}{rgb}{0.073593,0.267612,0.073593}%
\pgfsetfillcolor{currentfill}%
\pgfsetfillopacity{0.200000}%
\pgfsetlinewidth{0.000000pt}%
\definecolor{currentstroke}{rgb}{0.000000,0.000000,0.000000}%
\pgfsetstrokecolor{currentstroke}%
\pgfsetdash{}{0pt}%
\pgfpathmoveto{\pgfqpoint{8.859135in}{1.021716in}}%
\pgfpathlineto{\pgfqpoint{8.664078in}{1.306218in}}%
\pgfpathlineto{\pgfqpoint{8.800986in}{1.527187in}}%
\pgfpathlineto{\pgfqpoint{8.859135in}{1.021716in}}%
\pgfpathclose%
\pgfusepath{fill}%
\end{pgfscope}%
\begin{pgfscope}%
\pgfpathrectangle{\pgfqpoint{6.818937in}{0.147348in}}{\pgfqpoint{2.735294in}{2.735294in}}%
\pgfusepath{clip}%
\pgfsetbuttcap%
\pgfsetroundjoin%
\definecolor{currentfill}{rgb}{0.073593,0.267612,0.073593}%
\pgfsetfillcolor{currentfill}%
\pgfsetfillopacity{0.200000}%
\pgfsetlinewidth{0.000000pt}%
\definecolor{currentstroke}{rgb}{0.000000,0.000000,0.000000}%
\pgfsetstrokecolor{currentstroke}%
\pgfsetdash{}{0pt}%
\pgfpathmoveto{\pgfqpoint{7.646110in}{1.527187in}}%
\pgfpathlineto{\pgfqpoint{7.783018in}{1.306218in}}%
\pgfpathlineto{\pgfqpoint{7.587961in}{1.021716in}}%
\pgfpathlineto{\pgfqpoint{7.646110in}{1.527187in}}%
\pgfpathclose%
\pgfusepath{fill}%
\end{pgfscope}%
\begin{pgfscope}%
\pgfpathrectangle{\pgfqpoint{6.818937in}{0.147348in}}{\pgfqpoint{2.735294in}{2.735294in}}%
\pgfusepath{clip}%
\pgfsetbuttcap%
\pgfsetroundjoin%
\definecolor{currentfill}{rgb}{0.065434,0.237940,0.065434}%
\pgfsetfillcolor{currentfill}%
\pgfsetfillopacity{0.200000}%
\pgfsetlinewidth{0.000000pt}%
\definecolor{currentstroke}{rgb}{0.000000,0.000000,0.000000}%
\pgfsetstrokecolor{currentstroke}%
\pgfsetdash{}{0pt}%
\pgfpathmoveto{\pgfqpoint{8.223548in}{0.822079in}}%
\pgfpathlineto{\pgfqpoint{8.375641in}{1.283420in}}%
\pgfpathlineto{\pgfqpoint{8.361488in}{0.943207in}}%
\pgfpathlineto{\pgfqpoint{8.223548in}{0.822079in}}%
\pgfpathclose%
\pgfusepath{fill}%
\end{pgfscope}%
\begin{pgfscope}%
\pgfpathrectangle{\pgfqpoint{6.818937in}{0.147348in}}{\pgfqpoint{2.735294in}{2.735294in}}%
\pgfusepath{clip}%
\pgfsetbuttcap%
\pgfsetroundjoin%
\definecolor{currentfill}{rgb}{0.065434,0.237940,0.065434}%
\pgfsetfillcolor{currentfill}%
\pgfsetfillopacity{0.200000}%
\pgfsetlinewidth{0.000000pt}%
\definecolor{currentstroke}{rgb}{0.000000,0.000000,0.000000}%
\pgfsetstrokecolor{currentstroke}%
\pgfsetdash{}{0pt}%
\pgfpathmoveto{\pgfqpoint{8.085608in}{0.943207in}}%
\pgfpathlineto{\pgfqpoint{8.071454in}{1.283420in}}%
\pgfpathlineto{\pgfqpoint{8.223548in}{0.822079in}}%
\pgfpathlineto{\pgfqpoint{8.085608in}{0.943207in}}%
\pgfpathclose%
\pgfusepath{fill}%
\end{pgfscope}%
\begin{pgfscope}%
\pgfpathrectangle{\pgfqpoint{6.818937in}{0.147348in}}{\pgfqpoint{2.735294in}{2.735294in}}%
\pgfusepath{clip}%
\pgfsetbuttcap%
\pgfsetroundjoin%
\definecolor{currentfill}{rgb}{0.067497,0.245443,0.067497}%
\pgfsetfillcolor{currentfill}%
\pgfsetfillopacity{0.200000}%
\pgfsetlinewidth{0.000000pt}%
\definecolor{currentstroke}{rgb}{0.000000,0.000000,0.000000}%
\pgfsetstrokecolor{currentstroke}%
\pgfsetdash{}{0pt}%
\pgfpathmoveto{\pgfqpoint{8.477877in}{0.836228in}}%
\pgfpathlineto{\pgfqpoint{8.361488in}{0.943207in}}%
\pgfpathlineto{\pgfqpoint{8.375641in}{1.283420in}}%
\pgfpathlineto{\pgfqpoint{8.477877in}{0.836228in}}%
\pgfpathclose%
\pgfusepath{fill}%
\end{pgfscope}%
\begin{pgfscope}%
\pgfpathrectangle{\pgfqpoint{6.818937in}{0.147348in}}{\pgfqpoint{2.735294in}{2.735294in}}%
\pgfusepath{clip}%
\pgfsetbuttcap%
\pgfsetroundjoin%
\definecolor{currentfill}{rgb}{0.067497,0.245443,0.067497}%
\pgfsetfillcolor{currentfill}%
\pgfsetfillopacity{0.200000}%
\pgfsetlinewidth{0.000000pt}%
\definecolor{currentstroke}{rgb}{0.000000,0.000000,0.000000}%
\pgfsetstrokecolor{currentstroke}%
\pgfsetdash{}{0pt}%
\pgfpathmoveto{\pgfqpoint{8.071454in}{1.283420in}}%
\pgfpathlineto{\pgfqpoint{8.085608in}{0.943207in}}%
\pgfpathlineto{\pgfqpoint{7.969219in}{0.836228in}}%
\pgfpathlineto{\pgfqpoint{8.071454in}{1.283420in}}%
\pgfpathclose%
\pgfusepath{fill}%
\end{pgfscope}%
\begin{pgfscope}%
\pgfpathrectangle{\pgfqpoint{6.818937in}{0.147348in}}{\pgfqpoint{2.735294in}{2.735294in}}%
\pgfusepath{clip}%
\pgfsetbuttcap%
\pgfsetroundjoin%
\definecolor{currentfill}{rgb}{0.097285,0.353762,0.097285}%
\pgfsetfillcolor{currentfill}%
\pgfsetfillopacity{0.200000}%
\pgfsetlinewidth{0.000000pt}%
\definecolor{currentstroke}{rgb}{0.000000,0.000000,0.000000}%
\pgfsetstrokecolor{currentstroke}%
\pgfsetdash{}{0pt}%
\pgfpathmoveto{\pgfqpoint{8.624680in}{2.141959in}}%
\pgfpathlineto{\pgfqpoint{8.857732in}{2.131775in}}%
\pgfpathlineto{\pgfqpoint{8.512025in}{1.947550in}}%
\pgfpathlineto{\pgfqpoint{8.624680in}{2.141959in}}%
\pgfpathclose%
\pgfusepath{fill}%
\end{pgfscope}%
\begin{pgfscope}%
\pgfpathrectangle{\pgfqpoint{6.818937in}{0.147348in}}{\pgfqpoint{2.735294in}{2.735294in}}%
\pgfusepath{clip}%
\pgfsetbuttcap%
\pgfsetroundjoin%
\definecolor{currentfill}{rgb}{0.097285,0.353762,0.097285}%
\pgfsetfillcolor{currentfill}%
\pgfsetfillopacity{0.200000}%
\pgfsetlinewidth{0.000000pt}%
\definecolor{currentstroke}{rgb}{0.000000,0.000000,0.000000}%
\pgfsetstrokecolor{currentstroke}%
\pgfsetdash{}{0pt}%
\pgfpathmoveto{\pgfqpoint{7.935071in}{1.947550in}}%
\pgfpathlineto{\pgfqpoint{7.589364in}{2.131775in}}%
\pgfpathlineto{\pgfqpoint{7.822416in}{2.141959in}}%
\pgfpathlineto{\pgfqpoint{7.935071in}{1.947550in}}%
\pgfpathclose%
\pgfusepath{fill}%
\end{pgfscope}%
\begin{pgfscope}%
\pgfpathrectangle{\pgfqpoint{6.818937in}{0.147348in}}{\pgfqpoint{2.735294in}{2.735294in}}%
\pgfusepath{clip}%
\pgfsetbuttcap%
\pgfsetroundjoin%
\definecolor{currentfill}{rgb}{0.060562,0.220227,0.060562}%
\pgfsetfillcolor{currentfill}%
\pgfsetfillopacity{0.200000}%
\pgfsetlinewidth{0.000000pt}%
\definecolor{currentstroke}{rgb}{0.000000,0.000000,0.000000}%
\pgfsetstrokecolor{currentstroke}%
\pgfsetdash{}{0pt}%
\pgfpathmoveto{\pgfqpoint{8.477877in}{0.836228in}}%
\pgfpathlineto{\pgfqpoint{8.375641in}{1.283420in}}%
\pgfpathlineto{\pgfqpoint{8.625616in}{0.971623in}}%
\pgfpathlineto{\pgfqpoint{8.477877in}{0.836228in}}%
\pgfpathclose%
\pgfusepath{fill}%
\end{pgfscope}%
\begin{pgfscope}%
\pgfpathrectangle{\pgfqpoint{6.818937in}{0.147348in}}{\pgfqpoint{2.735294in}{2.735294in}}%
\pgfusepath{clip}%
\pgfsetbuttcap%
\pgfsetroundjoin%
\definecolor{currentfill}{rgb}{0.060562,0.220227,0.060562}%
\pgfsetfillcolor{currentfill}%
\pgfsetfillopacity{0.200000}%
\pgfsetlinewidth{0.000000pt}%
\definecolor{currentstroke}{rgb}{0.000000,0.000000,0.000000}%
\pgfsetstrokecolor{currentstroke}%
\pgfsetdash{}{0pt}%
\pgfpathmoveto{\pgfqpoint{7.821480in}{0.971623in}}%
\pgfpathlineto{\pgfqpoint{8.071454in}{1.283420in}}%
\pgfpathlineto{\pgfqpoint{7.969219in}{0.836228in}}%
\pgfpathlineto{\pgfqpoint{7.821480in}{0.971623in}}%
\pgfpathclose%
\pgfusepath{fill}%
\end{pgfscope}%
\begin{pgfscope}%
\pgfpathrectangle{\pgfqpoint{6.818937in}{0.147348in}}{\pgfqpoint{2.735294in}{2.735294in}}%
\pgfusepath{clip}%
\pgfsetbuttcap%
\pgfsetroundjoin%
\definecolor{currentfill}{rgb}{0.092193,0.335248,0.092193}%
\pgfsetfillcolor{currentfill}%
\pgfsetfillopacity{0.200000}%
\pgfsetlinewidth{0.000000pt}%
\definecolor{currentstroke}{rgb}{0.000000,0.000000,0.000000}%
\pgfsetstrokecolor{currentstroke}%
\pgfsetdash{}{0pt}%
\pgfpathmoveto{\pgfqpoint{8.800986in}{1.527187in}}%
\pgfpathlineto{\pgfqpoint{8.857732in}{2.131775in}}%
\pgfpathlineto{\pgfqpoint{9.026747in}{1.557473in}}%
\pgfpathlineto{\pgfqpoint{8.800986in}{1.527187in}}%
\pgfpathclose%
\pgfusepath{fill}%
\end{pgfscope}%
\begin{pgfscope}%
\pgfpathrectangle{\pgfqpoint{6.818937in}{0.147348in}}{\pgfqpoint{2.735294in}{2.735294in}}%
\pgfusepath{clip}%
\pgfsetbuttcap%
\pgfsetroundjoin%
\definecolor{currentfill}{rgb}{0.092193,0.335248,0.092193}%
\pgfsetfillcolor{currentfill}%
\pgfsetfillopacity{0.200000}%
\pgfsetlinewidth{0.000000pt}%
\definecolor{currentstroke}{rgb}{0.000000,0.000000,0.000000}%
\pgfsetstrokecolor{currentstroke}%
\pgfsetdash{}{0pt}%
\pgfpathmoveto{\pgfqpoint{7.420349in}{1.557473in}}%
\pgfpathlineto{\pgfqpoint{7.589364in}{2.131775in}}%
\pgfpathlineto{\pgfqpoint{7.646110in}{1.527187in}}%
\pgfpathlineto{\pgfqpoint{7.420349in}{1.557473in}}%
\pgfpathclose%
\pgfusepath{fill}%
\end{pgfscope}%
\begin{pgfscope}%
\pgfpathrectangle{\pgfqpoint{6.818937in}{0.147348in}}{\pgfqpoint{2.735294in}{2.735294in}}%
\pgfusepath{clip}%
\pgfsetbuttcap%
\pgfsetroundjoin%
\definecolor{currentfill}{rgb}{0.111651,0.406004,0.111651}%
\pgfsetfillcolor{currentfill}%
\pgfsetfillopacity{0.200000}%
\pgfsetlinewidth{0.000000pt}%
\definecolor{currentstroke}{rgb}{0.000000,0.000000,0.000000}%
\pgfsetstrokecolor{currentstroke}%
\pgfsetdash{}{0pt}%
\pgfpathmoveto{\pgfqpoint{7.822416in}{2.141959in}}%
\pgfpathlineto{\pgfqpoint{7.734361in}{2.303109in}}%
\pgfpathlineto{\pgfqpoint{8.085938in}{2.147735in}}%
\pgfpathlineto{\pgfqpoint{7.822416in}{2.141959in}}%
\pgfpathclose%
\pgfusepath{fill}%
\end{pgfscope}%
\begin{pgfscope}%
\pgfpathrectangle{\pgfqpoint{6.818937in}{0.147348in}}{\pgfqpoint{2.735294in}{2.735294in}}%
\pgfusepath{clip}%
\pgfsetbuttcap%
\pgfsetroundjoin%
\definecolor{currentfill}{rgb}{0.111651,0.406004,0.111651}%
\pgfsetfillcolor{currentfill}%
\pgfsetfillopacity{0.200000}%
\pgfsetlinewidth{0.000000pt}%
\definecolor{currentstroke}{rgb}{0.000000,0.000000,0.000000}%
\pgfsetstrokecolor{currentstroke}%
\pgfsetdash{}{0pt}%
\pgfpathmoveto{\pgfqpoint{8.361157in}{2.147735in}}%
\pgfpathlineto{\pgfqpoint{8.712735in}{2.303109in}}%
\pgfpathlineto{\pgfqpoint{8.624680in}{2.141959in}}%
\pgfpathlineto{\pgfqpoint{8.361157in}{2.147735in}}%
\pgfpathclose%
\pgfusepath{fill}%
\end{pgfscope}%
\begin{pgfscope}%
\pgfpathrectangle{\pgfqpoint{6.818937in}{0.147348in}}{\pgfqpoint{2.735294in}{2.735294in}}%
\pgfusepath{clip}%
\pgfsetbuttcap%
\pgfsetroundjoin%
\definecolor{currentfill}{rgb}{0.070885,0.257762,0.070885}%
\pgfsetfillcolor{currentfill}%
\pgfsetfillopacity{0.200000}%
\pgfsetlinewidth{0.000000pt}%
\definecolor{currentstroke}{rgb}{0.000000,0.000000,0.000000}%
\pgfsetstrokecolor{currentstroke}%
\pgfsetdash{}{0pt}%
\pgfpathmoveto{\pgfqpoint{8.664078in}{1.306218in}}%
\pgfpathlineto{\pgfqpoint{8.714128in}{0.875593in}}%
\pgfpathlineto{\pgfqpoint{8.625616in}{0.971623in}}%
\pgfpathlineto{\pgfqpoint{8.664078in}{1.306218in}}%
\pgfpathclose%
\pgfusepath{fill}%
\end{pgfscope}%
\begin{pgfscope}%
\pgfpathrectangle{\pgfqpoint{6.818937in}{0.147348in}}{\pgfqpoint{2.735294in}{2.735294in}}%
\pgfusepath{clip}%
\pgfsetbuttcap%
\pgfsetroundjoin%
\definecolor{currentfill}{rgb}{0.070885,0.257762,0.070885}%
\pgfsetfillcolor{currentfill}%
\pgfsetfillopacity{0.200000}%
\pgfsetlinewidth{0.000000pt}%
\definecolor{currentstroke}{rgb}{0.000000,0.000000,0.000000}%
\pgfsetstrokecolor{currentstroke}%
\pgfsetdash{}{0pt}%
\pgfpathmoveto{\pgfqpoint{7.821480in}{0.971623in}}%
\pgfpathlineto{\pgfqpoint{7.732968in}{0.875593in}}%
\pgfpathlineto{\pgfqpoint{7.783018in}{1.306218in}}%
\pgfpathlineto{\pgfqpoint{7.821480in}{0.971623in}}%
\pgfpathclose%
\pgfusepath{fill}%
\end{pgfscope}%
\begin{pgfscope}%
\pgfpathrectangle{\pgfqpoint{6.818937in}{0.147348in}}{\pgfqpoint{2.735294in}{2.735294in}}%
\pgfusepath{clip}%
\pgfsetbuttcap%
\pgfsetroundjoin%
\definecolor{currentfill}{rgb}{0.070984,0.258123,0.070984}%
\pgfsetfillcolor{currentfill}%
\pgfsetfillopacity{0.200000}%
\pgfsetlinewidth{0.000000pt}%
\definecolor{currentstroke}{rgb}{0.000000,0.000000,0.000000}%
\pgfsetstrokecolor{currentstroke}%
\pgfsetdash{}{0pt}%
\pgfpathmoveto{\pgfqpoint{9.026747in}{1.557473in}}%
\pgfpathlineto{\pgfqpoint{9.053362in}{1.083902in}}%
\pgfpathlineto{\pgfqpoint{8.800986in}{1.527187in}}%
\pgfpathlineto{\pgfqpoint{9.026747in}{1.557473in}}%
\pgfpathclose%
\pgfusepath{fill}%
\end{pgfscope}%
\begin{pgfscope}%
\pgfpathrectangle{\pgfqpoint{6.818937in}{0.147348in}}{\pgfqpoint{2.735294in}{2.735294in}}%
\pgfusepath{clip}%
\pgfsetbuttcap%
\pgfsetroundjoin%
\definecolor{currentfill}{rgb}{0.070984,0.258123,0.070984}%
\pgfsetfillcolor{currentfill}%
\pgfsetfillopacity{0.200000}%
\pgfsetlinewidth{0.000000pt}%
\definecolor{currentstroke}{rgb}{0.000000,0.000000,0.000000}%
\pgfsetstrokecolor{currentstroke}%
\pgfsetdash{}{0pt}%
\pgfpathmoveto{\pgfqpoint{7.646110in}{1.527187in}}%
\pgfpathlineto{\pgfqpoint{7.393734in}{1.083902in}}%
\pgfpathlineto{\pgfqpoint{7.420349in}{1.557473in}}%
\pgfpathlineto{\pgfqpoint{7.646110in}{1.527187in}}%
\pgfpathclose%
\pgfusepath{fill}%
\end{pgfscope}%
\begin{pgfscope}%
\pgfpathrectangle{\pgfqpoint{6.818937in}{0.147348in}}{\pgfqpoint{2.735294in}{2.735294in}}%
\pgfusepath{clip}%
\pgfsetbuttcap%
\pgfsetroundjoin%
\definecolor{currentfill}{rgb}{0.061754,0.224559,0.061754}%
\pgfsetfillcolor{currentfill}%
\pgfsetfillopacity{0.200000}%
\pgfsetlinewidth{0.000000pt}%
\definecolor{currentstroke}{rgb}{0.000000,0.000000,0.000000}%
\pgfsetstrokecolor{currentstroke}%
\pgfsetdash{}{0pt}%
\pgfpathmoveto{\pgfqpoint{8.859135in}{1.021716in}}%
\pgfpathlineto{\pgfqpoint{8.714128in}{0.875593in}}%
\pgfpathlineto{\pgfqpoint{8.664078in}{1.306218in}}%
\pgfpathlineto{\pgfqpoint{8.859135in}{1.021716in}}%
\pgfpathclose%
\pgfusepath{fill}%
\end{pgfscope}%
\begin{pgfscope}%
\pgfpathrectangle{\pgfqpoint{6.818937in}{0.147348in}}{\pgfqpoint{2.735294in}{2.735294in}}%
\pgfusepath{clip}%
\pgfsetbuttcap%
\pgfsetroundjoin%
\definecolor{currentfill}{rgb}{0.061754,0.224559,0.061754}%
\pgfsetfillcolor{currentfill}%
\pgfsetfillopacity{0.200000}%
\pgfsetlinewidth{0.000000pt}%
\definecolor{currentstroke}{rgb}{0.000000,0.000000,0.000000}%
\pgfsetstrokecolor{currentstroke}%
\pgfsetdash{}{0pt}%
\pgfpathmoveto{\pgfqpoint{7.783018in}{1.306218in}}%
\pgfpathlineto{\pgfqpoint{7.732968in}{0.875593in}}%
\pgfpathlineto{\pgfqpoint{7.587961in}{1.021716in}}%
\pgfpathlineto{\pgfqpoint{7.783018in}{1.306218in}}%
\pgfpathclose%
\pgfusepath{fill}%
\end{pgfscope}%
\begin{pgfscope}%
\pgfpathrectangle{\pgfqpoint{6.818937in}{0.147348in}}{\pgfqpoint{2.735294in}{2.735294in}}%
\pgfusepath{clip}%
\pgfsetbuttcap%
\pgfsetroundjoin%
\definecolor{currentfill}{rgb}{0.089078,0.323920,0.089078}%
\pgfsetfillcolor{currentfill}%
\pgfsetfillopacity{0.200000}%
\pgfsetlinewidth{0.000000pt}%
\definecolor{currentstroke}{rgb}{0.000000,0.000000,0.000000}%
\pgfsetstrokecolor{currentstroke}%
\pgfsetdash{}{0pt}%
\pgfpathmoveto{\pgfqpoint{9.026747in}{1.557473in}}%
\pgfpathlineto{\pgfqpoint{8.857732in}{2.131775in}}%
\pgfpathlineto{\pgfqpoint{9.112059in}{1.762300in}}%
\pgfpathlineto{\pgfqpoint{9.026747in}{1.557473in}}%
\pgfpathclose%
\pgfusepath{fill}%
\end{pgfscope}%
\begin{pgfscope}%
\pgfpathrectangle{\pgfqpoint{6.818937in}{0.147348in}}{\pgfqpoint{2.735294in}{2.735294in}}%
\pgfusepath{clip}%
\pgfsetbuttcap%
\pgfsetroundjoin%
\definecolor{currentfill}{rgb}{0.089078,0.323920,0.089078}%
\pgfsetfillcolor{currentfill}%
\pgfsetfillopacity{0.200000}%
\pgfsetlinewidth{0.000000pt}%
\definecolor{currentstroke}{rgb}{0.000000,0.000000,0.000000}%
\pgfsetstrokecolor{currentstroke}%
\pgfsetdash{}{0pt}%
\pgfpathmoveto{\pgfqpoint{7.335037in}{1.762300in}}%
\pgfpathlineto{\pgfqpoint{7.589364in}{2.131775in}}%
\pgfpathlineto{\pgfqpoint{7.420349in}{1.557473in}}%
\pgfpathlineto{\pgfqpoint{7.335037in}{1.762300in}}%
\pgfpathclose%
\pgfusepath{fill}%
\end{pgfscope}%
\begin{pgfscope}%
\pgfpathrectangle{\pgfqpoint{6.818937in}{0.147348in}}{\pgfqpoint{2.735294in}{2.735294in}}%
\pgfusepath{clip}%
\pgfsetbuttcap%
\pgfsetroundjoin%
\definecolor{currentfill}{rgb}{0.107070,0.389346,0.107070}%
\pgfsetfillcolor{currentfill}%
\pgfsetfillopacity{0.200000}%
\pgfsetlinewidth{0.000000pt}%
\definecolor{currentstroke}{rgb}{0.000000,0.000000,0.000000}%
\pgfsetstrokecolor{currentstroke}%
\pgfsetdash{}{0pt}%
\pgfpathmoveto{\pgfqpoint{8.624680in}{2.141959in}}%
\pgfpathlineto{\pgfqpoint{8.712735in}{2.303109in}}%
\pgfpathlineto{\pgfqpoint{8.857732in}{2.131775in}}%
\pgfpathlineto{\pgfqpoint{8.624680in}{2.141959in}}%
\pgfpathclose%
\pgfusepath{fill}%
\end{pgfscope}%
\begin{pgfscope}%
\pgfpathrectangle{\pgfqpoint{6.818937in}{0.147348in}}{\pgfqpoint{2.735294in}{2.735294in}}%
\pgfusepath{clip}%
\pgfsetbuttcap%
\pgfsetroundjoin%
\definecolor{currentfill}{rgb}{0.107070,0.389346,0.107070}%
\pgfsetfillcolor{currentfill}%
\pgfsetfillopacity{0.200000}%
\pgfsetlinewidth{0.000000pt}%
\definecolor{currentstroke}{rgb}{0.000000,0.000000,0.000000}%
\pgfsetstrokecolor{currentstroke}%
\pgfsetdash{}{0pt}%
\pgfpathmoveto{\pgfqpoint{7.589364in}{2.131775in}}%
\pgfpathlineto{\pgfqpoint{7.734361in}{2.303109in}}%
\pgfpathlineto{\pgfqpoint{7.822416in}{2.141959in}}%
\pgfpathlineto{\pgfqpoint{7.589364in}{2.131775in}}%
\pgfpathclose%
\pgfusepath{fill}%
\end{pgfscope}%
\begin{pgfscope}%
\pgfpathrectangle{\pgfqpoint{6.818937in}{0.147348in}}{\pgfqpoint{2.735294in}{2.735294in}}%
\pgfusepath{clip}%
\pgfsetbuttcap%
\pgfsetroundjoin%
\definecolor{currentfill}{rgb}{0.056200,0.204363,0.056200}%
\pgfsetfillcolor{currentfill}%
\pgfsetfillopacity{0.200000}%
\pgfsetlinewidth{0.000000pt}%
\definecolor{currentstroke}{rgb}{0.000000,0.000000,0.000000}%
\pgfsetstrokecolor{currentstroke}%
\pgfsetdash{}{0pt}%
\pgfpathmoveto{\pgfqpoint{7.969219in}{0.836228in}}%
\pgfpathlineto{\pgfqpoint{8.085608in}{0.943207in}}%
\pgfpathlineto{\pgfqpoint{8.223548in}{0.822079in}}%
\pgfpathlineto{\pgfqpoint{7.969219in}{0.836228in}}%
\pgfpathclose%
\pgfusepath{fill}%
\end{pgfscope}%
\begin{pgfscope}%
\pgfpathrectangle{\pgfqpoint{6.818937in}{0.147348in}}{\pgfqpoint{2.735294in}{2.735294in}}%
\pgfusepath{clip}%
\pgfsetbuttcap%
\pgfsetroundjoin%
\definecolor{currentfill}{rgb}{0.056200,0.204363,0.056200}%
\pgfsetfillcolor{currentfill}%
\pgfsetfillopacity{0.200000}%
\pgfsetlinewidth{0.000000pt}%
\definecolor{currentstroke}{rgb}{0.000000,0.000000,0.000000}%
\pgfsetstrokecolor{currentstroke}%
\pgfsetdash{}{0pt}%
\pgfpathmoveto{\pgfqpoint{8.223548in}{0.822079in}}%
\pgfpathlineto{\pgfqpoint{8.361488in}{0.943207in}}%
\pgfpathlineto{\pgfqpoint{8.477877in}{0.836228in}}%
\pgfpathlineto{\pgfqpoint{8.223548in}{0.822079in}}%
\pgfpathclose%
\pgfusepath{fill}%
\end{pgfscope}%
\begin{pgfscope}%
\pgfpathrectangle{\pgfqpoint{6.818937in}{0.147348in}}{\pgfqpoint{2.735294in}{2.735294in}}%
\pgfusepath{clip}%
\pgfsetbuttcap%
\pgfsetroundjoin%
\definecolor{currentfill}{rgb}{0.086498,0.314539,0.086498}%
\pgfsetfillcolor{currentfill}%
\pgfsetfillopacity{0.200000}%
\pgfsetlinewidth{0.000000pt}%
\definecolor{currentstroke}{rgb}{0.000000,0.000000,0.000000}%
\pgfsetstrokecolor{currentstroke}%
\pgfsetdash{}{0pt}%
\pgfpathmoveto{\pgfqpoint{9.026747in}{1.557473in}}%
\pgfpathlineto{\pgfqpoint{9.112059in}{1.762300in}}%
\pgfpathlineto{\pgfqpoint{9.203313in}{1.590365in}}%
\pgfpathlineto{\pgfqpoint{9.026747in}{1.557473in}}%
\pgfpathclose%
\pgfusepath{fill}%
\end{pgfscope}%
\begin{pgfscope}%
\pgfpathrectangle{\pgfqpoint{6.818937in}{0.147348in}}{\pgfqpoint{2.735294in}{2.735294in}}%
\pgfusepath{clip}%
\pgfsetbuttcap%
\pgfsetroundjoin%
\definecolor{currentfill}{rgb}{0.086498,0.314539,0.086498}%
\pgfsetfillcolor{currentfill}%
\pgfsetfillopacity{0.200000}%
\pgfsetlinewidth{0.000000pt}%
\definecolor{currentstroke}{rgb}{0.000000,0.000000,0.000000}%
\pgfsetstrokecolor{currentstroke}%
\pgfsetdash{}{0pt}%
\pgfpathmoveto{\pgfqpoint{7.243783in}{1.590365in}}%
\pgfpathlineto{\pgfqpoint{7.335037in}{1.762300in}}%
\pgfpathlineto{\pgfqpoint{7.420349in}{1.557473in}}%
\pgfpathlineto{\pgfqpoint{7.243783in}{1.590365in}}%
\pgfpathclose%
\pgfusepath{fill}%
\end{pgfscope}%
\begin{pgfscope}%
\pgfpathrectangle{\pgfqpoint{6.818937in}{0.147348in}}{\pgfqpoint{2.735294in}{2.735294in}}%
\pgfusepath{clip}%
\pgfsetbuttcap%
\pgfsetroundjoin%
\definecolor{currentfill}{rgb}{0.090812,0.330224,0.090812}%
\pgfsetfillcolor{currentfill}%
\pgfsetfillopacity{0.200000}%
\pgfsetlinewidth{0.000000pt}%
\definecolor{currentstroke}{rgb}{0.000000,0.000000,0.000000}%
\pgfsetstrokecolor{currentstroke}%
\pgfsetdash{}{0pt}%
\pgfpathmoveto{\pgfqpoint{8.800986in}{1.527187in}}%
\pgfpathlineto{\pgfqpoint{8.920136in}{0.932617in}}%
\pgfpathlineto{\pgfqpoint{8.859135in}{1.021716in}}%
\pgfpathlineto{\pgfqpoint{8.800986in}{1.527187in}}%
\pgfpathclose%
\pgfusepath{fill}%
\end{pgfscope}%
\begin{pgfscope}%
\pgfpathrectangle{\pgfqpoint{6.818937in}{0.147348in}}{\pgfqpoint{2.735294in}{2.735294in}}%
\pgfusepath{clip}%
\pgfsetbuttcap%
\pgfsetroundjoin%
\definecolor{currentfill}{rgb}{0.090812,0.330224,0.090812}%
\pgfsetfillcolor{currentfill}%
\pgfsetfillopacity{0.200000}%
\pgfsetlinewidth{0.000000pt}%
\definecolor{currentstroke}{rgb}{0.000000,0.000000,0.000000}%
\pgfsetstrokecolor{currentstroke}%
\pgfsetdash{}{0pt}%
\pgfpathmoveto{\pgfqpoint{7.587961in}{1.021716in}}%
\pgfpathlineto{\pgfqpoint{7.526959in}{0.932617in}}%
\pgfpathlineto{\pgfqpoint{7.646110in}{1.527187in}}%
\pgfpathlineto{\pgfqpoint{7.587961in}{1.021716in}}%
\pgfpathclose%
\pgfusepath{fill}%
\end{pgfscope}%
\begin{pgfscope}%
\pgfpathrectangle{\pgfqpoint{6.818937in}{0.147348in}}{\pgfqpoint{2.735294in}{2.735294in}}%
\pgfusepath{clip}%
\pgfsetbuttcap%
\pgfsetroundjoin%
\definecolor{currentfill}{rgb}{0.116321,0.422987,0.116321}%
\pgfsetfillcolor{currentfill}%
\pgfsetfillopacity{0.200000}%
\pgfsetlinewidth{0.000000pt}%
\definecolor{currentstroke}{rgb}{0.000000,0.000000,0.000000}%
\pgfsetstrokecolor{currentstroke}%
\pgfsetdash{}{0pt}%
\pgfpathmoveto{\pgfqpoint{8.361157in}{2.147735in}}%
\pgfpathlineto{\pgfqpoint{8.085938in}{2.147735in}}%
\pgfpathlineto{\pgfqpoint{8.123207in}{2.676888in}}%
\pgfpathlineto{\pgfqpoint{8.361157in}{2.147735in}}%
\pgfpathclose%
\pgfusepath{fill}%
\end{pgfscope}%
\begin{pgfscope}%
\pgfpathrectangle{\pgfqpoint{6.818937in}{0.147348in}}{\pgfqpoint{2.735294in}{2.735294in}}%
\pgfusepath{clip}%
\pgfsetbuttcap%
\pgfsetroundjoin%
\definecolor{currentfill}{rgb}{0.057724,0.209904,0.057724}%
\pgfsetfillcolor{currentfill}%
\pgfsetfillopacity{0.200000}%
\pgfsetlinewidth{0.000000pt}%
\definecolor{currentstroke}{rgb}{0.000000,0.000000,0.000000}%
\pgfsetstrokecolor{currentstroke}%
\pgfsetdash{}{0pt}%
\pgfpathmoveto{\pgfqpoint{7.969219in}{0.836228in}}%
\pgfpathlineto{\pgfqpoint{7.732968in}{0.875593in}}%
\pgfpathlineto{\pgfqpoint{7.821480in}{0.971623in}}%
\pgfpathlineto{\pgfqpoint{7.969219in}{0.836228in}}%
\pgfpathclose%
\pgfusepath{fill}%
\end{pgfscope}%
\begin{pgfscope}%
\pgfpathrectangle{\pgfqpoint{6.818937in}{0.147348in}}{\pgfqpoint{2.735294in}{2.735294in}}%
\pgfusepath{clip}%
\pgfsetbuttcap%
\pgfsetroundjoin%
\definecolor{currentfill}{rgb}{0.057724,0.209904,0.057724}%
\pgfsetfillcolor{currentfill}%
\pgfsetfillopacity{0.200000}%
\pgfsetlinewidth{0.000000pt}%
\definecolor{currentstroke}{rgb}{0.000000,0.000000,0.000000}%
\pgfsetstrokecolor{currentstroke}%
\pgfsetdash{}{0pt}%
\pgfpathmoveto{\pgfqpoint{8.625616in}{0.971623in}}%
\pgfpathlineto{\pgfqpoint{8.714128in}{0.875593in}}%
\pgfpathlineto{\pgfqpoint{8.477877in}{0.836228in}}%
\pgfpathlineto{\pgfqpoint{8.625616in}{0.971623in}}%
\pgfpathclose%
\pgfusepath{fill}%
\end{pgfscope}%
\begin{pgfscope}%
\pgfpathrectangle{\pgfqpoint{6.818937in}{0.147348in}}{\pgfqpoint{2.735294in}{2.735294in}}%
\pgfusepath{clip}%
\pgfsetbuttcap%
\pgfsetroundjoin%
\definecolor{currentfill}{rgb}{0.064867,0.235879,0.064867}%
\pgfsetfillcolor{currentfill}%
\pgfsetfillopacity{0.200000}%
\pgfsetlinewidth{0.000000pt}%
\definecolor{currentstroke}{rgb}{0.000000,0.000000,0.000000}%
\pgfsetstrokecolor{currentstroke}%
\pgfsetdash{}{0pt}%
\pgfpathmoveto{\pgfqpoint{9.053362in}{1.083902in}}%
\pgfpathlineto{\pgfqpoint{8.920136in}{0.932617in}}%
\pgfpathlineto{\pgfqpoint{8.800986in}{1.527187in}}%
\pgfpathlineto{\pgfqpoint{9.053362in}{1.083902in}}%
\pgfpathclose%
\pgfusepath{fill}%
\end{pgfscope}%
\begin{pgfscope}%
\pgfpathrectangle{\pgfqpoint{6.818937in}{0.147348in}}{\pgfqpoint{2.735294in}{2.735294in}}%
\pgfusepath{clip}%
\pgfsetbuttcap%
\pgfsetroundjoin%
\definecolor{currentfill}{rgb}{0.064867,0.235879,0.064867}%
\pgfsetfillcolor{currentfill}%
\pgfsetfillopacity{0.200000}%
\pgfsetlinewidth{0.000000pt}%
\definecolor{currentstroke}{rgb}{0.000000,0.000000,0.000000}%
\pgfsetstrokecolor{currentstroke}%
\pgfsetdash{}{0pt}%
\pgfpathmoveto{\pgfqpoint{7.646110in}{1.527187in}}%
\pgfpathlineto{\pgfqpoint{7.526959in}{0.932617in}}%
\pgfpathlineto{\pgfqpoint{7.393734in}{1.083902in}}%
\pgfpathlineto{\pgfqpoint{7.646110in}{1.527187in}}%
\pgfpathclose%
\pgfusepath{fill}%
\end{pgfscope}%
\begin{pgfscope}%
\pgfpathrectangle{\pgfqpoint{6.818937in}{0.147348in}}{\pgfqpoint{2.735294in}{2.735294in}}%
\pgfusepath{clip}%
\pgfsetbuttcap%
\pgfsetroundjoin%
\definecolor{currentfill}{rgb}{0.116785,0.424671,0.116785}%
\pgfsetfillcolor{currentfill}%
\pgfsetfillopacity{0.200000}%
\pgfsetlinewidth{0.000000pt}%
\definecolor{currentstroke}{rgb}{0.000000,0.000000,0.000000}%
\pgfsetstrokecolor{currentstroke}%
\pgfsetdash{}{0pt}%
\pgfpathmoveto{\pgfqpoint{8.361157in}{2.147735in}}%
\pgfpathlineto{\pgfqpoint{8.323888in}{2.676888in}}%
\pgfpathlineto{\pgfqpoint{8.712735in}{2.303109in}}%
\pgfpathlineto{\pgfqpoint{8.361157in}{2.147735in}}%
\pgfpathclose%
\pgfusepath{fill}%
\end{pgfscope}%
\begin{pgfscope}%
\pgfpathrectangle{\pgfqpoint{6.818937in}{0.147348in}}{\pgfqpoint{2.735294in}{2.735294in}}%
\pgfusepath{clip}%
\pgfsetbuttcap%
\pgfsetroundjoin%
\definecolor{currentfill}{rgb}{0.116785,0.424671,0.116785}%
\pgfsetfillcolor{currentfill}%
\pgfsetfillopacity{0.200000}%
\pgfsetlinewidth{0.000000pt}%
\definecolor{currentstroke}{rgb}{0.000000,0.000000,0.000000}%
\pgfsetstrokecolor{currentstroke}%
\pgfsetdash{}{0pt}%
\pgfpathmoveto{\pgfqpoint{7.734361in}{2.303109in}}%
\pgfpathlineto{\pgfqpoint{8.123207in}{2.676888in}}%
\pgfpathlineto{\pgfqpoint{8.085938in}{2.147735in}}%
\pgfpathlineto{\pgfqpoint{7.734361in}{2.303109in}}%
\pgfpathclose%
\pgfusepath{fill}%
\end{pgfscope}%
\begin{pgfscope}%
\pgfpathrectangle{\pgfqpoint{6.818937in}{0.147348in}}{\pgfqpoint{2.735294in}{2.735294in}}%
\pgfusepath{clip}%
\pgfsetbuttcap%
\pgfsetroundjoin%
\definecolor{currentfill}{rgb}{0.074506,0.270932,0.074506}%
\pgfsetfillcolor{currentfill}%
\pgfsetfillopacity{0.200000}%
\pgfsetlinewidth{0.000000pt}%
\definecolor{currentstroke}{rgb}{0.000000,0.000000,0.000000}%
\pgfsetstrokecolor{currentstroke}%
\pgfsetdash{}{0pt}%
\pgfpathmoveto{\pgfqpoint{9.203313in}{1.590365in}}%
\pgfpathlineto{\pgfqpoint{9.208715in}{1.149785in}}%
\pgfpathlineto{\pgfqpoint{9.026747in}{1.557473in}}%
\pgfpathlineto{\pgfqpoint{9.203313in}{1.590365in}}%
\pgfpathclose%
\pgfusepath{fill}%
\end{pgfscope}%
\begin{pgfscope}%
\pgfpathrectangle{\pgfqpoint{6.818937in}{0.147348in}}{\pgfqpoint{2.735294in}{2.735294in}}%
\pgfusepath{clip}%
\pgfsetbuttcap%
\pgfsetroundjoin%
\definecolor{currentfill}{rgb}{0.074506,0.270932,0.074506}%
\pgfsetfillcolor{currentfill}%
\pgfsetfillopacity{0.200000}%
\pgfsetlinewidth{0.000000pt}%
\definecolor{currentstroke}{rgb}{0.000000,0.000000,0.000000}%
\pgfsetstrokecolor{currentstroke}%
\pgfsetdash{}{0pt}%
\pgfpathmoveto{\pgfqpoint{7.420349in}{1.557473in}}%
\pgfpathlineto{\pgfqpoint{7.238381in}{1.149785in}}%
\pgfpathlineto{\pgfqpoint{7.243783in}{1.590365in}}%
\pgfpathlineto{\pgfqpoint{7.420349in}{1.557473in}}%
\pgfpathclose%
\pgfusepath{fill}%
\end{pgfscope}%
\begin{pgfscope}%
\pgfpathrectangle{\pgfqpoint{6.818937in}{0.147348in}}{\pgfqpoint{2.735294in}{2.735294in}}%
\pgfusepath{clip}%
\pgfsetbuttcap%
\pgfsetroundjoin%
\definecolor{currentfill}{rgb}{0.060435,0.219763,0.060435}%
\pgfsetfillcolor{currentfill}%
\pgfsetfillopacity{0.200000}%
\pgfsetlinewidth{0.000000pt}%
\definecolor{currentstroke}{rgb}{0.000000,0.000000,0.000000}%
\pgfsetstrokecolor{currentstroke}%
\pgfsetdash{}{0pt}%
\pgfpathmoveto{\pgfqpoint{8.920136in}{0.932617in}}%
\pgfpathlineto{\pgfqpoint{8.714128in}{0.875593in}}%
\pgfpathlineto{\pgfqpoint{8.859135in}{1.021716in}}%
\pgfpathlineto{\pgfqpoint{8.920136in}{0.932617in}}%
\pgfpathclose%
\pgfusepath{fill}%
\end{pgfscope}%
\begin{pgfscope}%
\pgfpathrectangle{\pgfqpoint{6.818937in}{0.147348in}}{\pgfqpoint{2.735294in}{2.735294in}}%
\pgfusepath{clip}%
\pgfsetbuttcap%
\pgfsetroundjoin%
\definecolor{currentfill}{rgb}{0.060435,0.219763,0.060435}%
\pgfsetfillcolor{currentfill}%
\pgfsetfillopacity{0.200000}%
\pgfsetlinewidth{0.000000pt}%
\definecolor{currentstroke}{rgb}{0.000000,0.000000,0.000000}%
\pgfsetstrokecolor{currentstroke}%
\pgfsetdash{}{0pt}%
\pgfpathmoveto{\pgfqpoint{7.587961in}{1.021716in}}%
\pgfpathlineto{\pgfqpoint{7.732968in}{0.875593in}}%
\pgfpathlineto{\pgfqpoint{7.526959in}{0.932617in}}%
\pgfpathlineto{\pgfqpoint{7.587961in}{1.021716in}}%
\pgfpathclose%
\pgfusepath{fill}%
\end{pgfscope}%
\begin{pgfscope}%
\pgfpathrectangle{\pgfqpoint{6.818937in}{0.147348in}}{\pgfqpoint{2.735294in}{2.735294in}}%
\pgfusepath{clip}%
\pgfsetbuttcap%
\pgfsetroundjoin%
\definecolor{currentfill}{rgb}{0.095351,0.346729,0.095351}%
\pgfsetfillcolor{currentfill}%
\pgfsetfillopacity{0.200000}%
\pgfsetlinewidth{0.000000pt}%
\definecolor{currentstroke}{rgb}{0.000000,0.000000,0.000000}%
\pgfsetstrokecolor{currentstroke}%
\pgfsetdash{}{0pt}%
\pgfpathmoveto{\pgfqpoint{9.026747in}{1.557473in}}%
\pgfpathlineto{\pgfqpoint{9.091593in}{0.998749in}}%
\pgfpathlineto{\pgfqpoint{9.053362in}{1.083902in}}%
\pgfpathlineto{\pgfqpoint{9.026747in}{1.557473in}}%
\pgfpathclose%
\pgfusepath{fill}%
\end{pgfscope}%
\begin{pgfscope}%
\pgfpathrectangle{\pgfqpoint{6.818937in}{0.147348in}}{\pgfqpoint{2.735294in}{2.735294in}}%
\pgfusepath{clip}%
\pgfsetbuttcap%
\pgfsetroundjoin%
\definecolor{currentfill}{rgb}{0.095351,0.346729,0.095351}%
\pgfsetfillcolor{currentfill}%
\pgfsetfillopacity{0.200000}%
\pgfsetlinewidth{0.000000pt}%
\definecolor{currentstroke}{rgb}{0.000000,0.000000,0.000000}%
\pgfsetstrokecolor{currentstroke}%
\pgfsetdash{}{0pt}%
\pgfpathmoveto{\pgfqpoint{7.393734in}{1.083902in}}%
\pgfpathlineto{\pgfqpoint{7.355503in}{0.998749in}}%
\pgfpathlineto{\pgfqpoint{7.420349in}{1.557473in}}%
\pgfpathlineto{\pgfqpoint{7.393734in}{1.083902in}}%
\pgfpathclose%
\pgfusepath{fill}%
\end{pgfscope}%
\begin{pgfscope}%
\pgfpathrectangle{\pgfqpoint{6.818937in}{0.147348in}}{\pgfqpoint{2.735294in}{2.735294in}}%
\pgfusepath{clip}%
\pgfsetbuttcap%
\pgfsetroundjoin%
\definecolor{currentfill}{rgb}{0.067061,0.243857,0.067061}%
\pgfsetfillcolor{currentfill}%
\pgfsetfillopacity{0.200000}%
\pgfsetlinewidth{0.000000pt}%
\definecolor{currentstroke}{rgb}{0.000000,0.000000,0.000000}%
\pgfsetstrokecolor{currentstroke}%
\pgfsetdash{}{0pt}%
\pgfpathmoveto{\pgfqpoint{9.208715in}{1.149785in}}%
\pgfpathlineto{\pgfqpoint{9.091593in}{0.998749in}}%
\pgfpathlineto{\pgfqpoint{9.026747in}{1.557473in}}%
\pgfpathlineto{\pgfqpoint{9.208715in}{1.149785in}}%
\pgfpathclose%
\pgfusepath{fill}%
\end{pgfscope}%
\begin{pgfscope}%
\pgfpathrectangle{\pgfqpoint{6.818937in}{0.147348in}}{\pgfqpoint{2.735294in}{2.735294in}}%
\pgfusepath{clip}%
\pgfsetbuttcap%
\pgfsetroundjoin%
\definecolor{currentfill}{rgb}{0.067061,0.243857,0.067061}%
\pgfsetfillcolor{currentfill}%
\pgfsetfillopacity{0.200000}%
\pgfsetlinewidth{0.000000pt}%
\definecolor{currentstroke}{rgb}{0.000000,0.000000,0.000000}%
\pgfsetstrokecolor{currentstroke}%
\pgfsetdash{}{0pt}%
\pgfpathmoveto{\pgfqpoint{7.420349in}{1.557473in}}%
\pgfpathlineto{\pgfqpoint{7.355503in}{0.998749in}}%
\pgfpathlineto{\pgfqpoint{7.238381in}{1.149785in}}%
\pgfpathlineto{\pgfqpoint{7.420349in}{1.557473in}}%
\pgfpathclose%
\pgfusepath{fill}%
\end{pgfscope}%
\begin{pgfscope}%
\pgfpathrectangle{\pgfqpoint{6.818937in}{0.147348in}}{\pgfqpoint{2.735294in}{2.735294in}}%
\pgfusepath{clip}%
\pgfsetbuttcap%
\pgfsetroundjoin%
\definecolor{currentfill}{rgb}{0.116321,0.422987,0.116321}%
\pgfsetfillcolor{currentfill}%
\pgfsetfillopacity{0.200000}%
\pgfsetlinewidth{0.000000pt}%
\definecolor{currentstroke}{rgb}{0.000000,0.000000,0.000000}%
\pgfsetstrokecolor{currentstroke}%
\pgfsetdash{}{0pt}%
\pgfpathmoveto{\pgfqpoint{8.123207in}{2.676888in}}%
\pgfpathlineto{\pgfqpoint{8.323888in}{2.676888in}}%
\pgfpathlineto{\pgfqpoint{8.361157in}{2.147735in}}%
\pgfpathlineto{\pgfqpoint{8.123207in}{2.676888in}}%
\pgfpathclose%
\pgfusepath{fill}%
\end{pgfscope}%
\begin{pgfscope}%
\pgfpathrectangle{\pgfqpoint{6.818937in}{0.147348in}}{\pgfqpoint{2.735294in}{2.735294in}}%
\pgfusepath{clip}%
\pgfsetbuttcap%
\pgfsetroundjoin%
\definecolor{currentfill}{rgb}{0.063840,0.232145,0.063840}%
\pgfsetfillcolor{currentfill}%
\pgfsetfillopacity{0.200000}%
\pgfsetlinewidth{0.000000pt}%
\definecolor{currentstroke}{rgb}{0.000000,0.000000,0.000000}%
\pgfsetstrokecolor{currentstroke}%
\pgfsetdash{}{0pt}%
\pgfpathmoveto{\pgfqpoint{8.920136in}{0.932617in}}%
\pgfpathlineto{\pgfqpoint{9.053362in}{1.083902in}}%
\pgfpathlineto{\pgfqpoint{9.091593in}{0.998749in}}%
\pgfpathlineto{\pgfqpoint{8.920136in}{0.932617in}}%
\pgfpathclose%
\pgfusepath{fill}%
\end{pgfscope}%
\begin{pgfscope}%
\pgfpathrectangle{\pgfqpoint{6.818937in}{0.147348in}}{\pgfqpoint{2.735294in}{2.735294in}}%
\pgfusepath{clip}%
\pgfsetbuttcap%
\pgfsetroundjoin%
\definecolor{currentfill}{rgb}{0.063840,0.232145,0.063840}%
\pgfsetfillcolor{currentfill}%
\pgfsetfillopacity{0.200000}%
\pgfsetlinewidth{0.000000pt}%
\definecolor{currentstroke}{rgb}{0.000000,0.000000,0.000000}%
\pgfsetstrokecolor{currentstroke}%
\pgfsetdash{}{0pt}%
\pgfpathmoveto{\pgfqpoint{7.355503in}{0.998749in}}%
\pgfpathlineto{\pgfqpoint{7.393734in}{1.083902in}}%
\pgfpathlineto{\pgfqpoint{7.526959in}{0.932617in}}%
\pgfpathlineto{\pgfqpoint{7.355503in}{0.998749in}}%
\pgfpathclose%
\pgfusepath{fill}%
\end{pgfscope}%
\begin{pgfscope}%
\pgfpathrectangle{\pgfqpoint{6.818937in}{0.147348in}}{\pgfqpoint{2.735294in}{2.735294in}}%
\pgfusepath{clip}%
\pgfsetbuttcap%
\pgfsetroundjoin%
\definecolor{currentfill}{rgb}{0.099716,0.362602,0.099716}%
\pgfsetfillcolor{currentfill}%
\pgfsetfillopacity{0.200000}%
\pgfsetlinewidth{0.000000pt}%
\definecolor{currentstroke}{rgb}{0.000000,0.000000,0.000000}%
\pgfsetstrokecolor{currentstroke}%
\pgfsetdash{}{0pt}%
\pgfpathmoveto{\pgfqpoint{9.203313in}{1.590365in}}%
\pgfpathlineto{\pgfqpoint{9.230175in}{1.067062in}}%
\pgfpathlineto{\pgfqpoint{9.208715in}{1.149785in}}%
\pgfpathlineto{\pgfqpoint{9.203313in}{1.590365in}}%
\pgfpathclose%
\pgfusepath{fill}%
\end{pgfscope}%
\begin{pgfscope}%
\pgfpathrectangle{\pgfqpoint{6.818937in}{0.147348in}}{\pgfqpoint{2.735294in}{2.735294in}}%
\pgfusepath{clip}%
\pgfsetbuttcap%
\pgfsetroundjoin%
\definecolor{currentfill}{rgb}{0.099716,0.362602,0.099716}%
\pgfsetfillcolor{currentfill}%
\pgfsetfillopacity{0.200000}%
\pgfsetlinewidth{0.000000pt}%
\definecolor{currentstroke}{rgb}{0.000000,0.000000,0.000000}%
\pgfsetstrokecolor{currentstroke}%
\pgfsetdash{}{0pt}%
\pgfpathmoveto{\pgfqpoint{7.238381in}{1.149785in}}%
\pgfpathlineto{\pgfqpoint{7.216921in}{1.067062in}}%
\pgfpathlineto{\pgfqpoint{7.243783in}{1.590365in}}%
\pgfpathlineto{\pgfqpoint{7.238381in}{1.149785in}}%
\pgfpathclose%
\pgfusepath{fill}%
\end{pgfscope}%
\begin{pgfscope}%
\pgfpathrectangle{\pgfqpoint{6.818937in}{0.147348in}}{\pgfqpoint{2.735294in}{2.735294in}}%
\pgfusepath{clip}%
\pgfsetbuttcap%
\pgfsetroundjoin%
\definecolor{currentfill}{rgb}{0.069492,0.252698,0.069492}%
\pgfsetfillcolor{currentfill}%
\pgfsetfillopacity{0.200000}%
\pgfsetlinewidth{0.000000pt}%
\definecolor{currentstroke}{rgb}{0.000000,0.000000,0.000000}%
\pgfsetstrokecolor{currentstroke}%
\pgfsetdash{}{0pt}%
\pgfpathmoveto{\pgfqpoint{9.330515in}{1.213837in}}%
\pgfpathlineto{\pgfqpoint{9.230175in}{1.067062in}}%
\pgfpathlineto{\pgfqpoint{9.203313in}{1.590365in}}%
\pgfpathlineto{\pgfqpoint{9.330515in}{1.213837in}}%
\pgfpathclose%
\pgfusepath{fill}%
\end{pgfscope}%
\begin{pgfscope}%
\pgfpathrectangle{\pgfqpoint{6.818937in}{0.147348in}}{\pgfqpoint{2.735294in}{2.735294in}}%
\pgfusepath{clip}%
\pgfsetbuttcap%
\pgfsetroundjoin%
\definecolor{currentfill}{rgb}{0.069492,0.252698,0.069492}%
\pgfsetfillcolor{currentfill}%
\pgfsetfillopacity{0.200000}%
\pgfsetlinewidth{0.000000pt}%
\definecolor{currentstroke}{rgb}{0.000000,0.000000,0.000000}%
\pgfsetstrokecolor{currentstroke}%
\pgfsetdash{}{0pt}%
\pgfpathmoveto{\pgfqpoint{7.116581in}{1.213837in}}%
\pgfpathlineto{\pgfqpoint{7.243783in}{1.590365in}}%
\pgfpathlineto{\pgfqpoint{7.216921in}{1.067062in}}%
\pgfpathlineto{\pgfqpoint{7.116581in}{1.213837in}}%
\pgfpathclose%
\pgfusepath{fill}%
\end{pgfscope}%
\begin{pgfscope}%
\pgfpathrectangle{\pgfqpoint{6.818937in}{0.147348in}}{\pgfqpoint{2.735294in}{2.735294in}}%
\pgfusepath{clip}%
\pgfsetbuttcap%
\pgfsetroundjoin%
\definecolor{currentfill}{rgb}{0.067488,0.245410,0.067488}%
\pgfsetfillcolor{currentfill}%
\pgfsetfillopacity{0.200000}%
\pgfsetlinewidth{0.000000pt}%
\definecolor{currentstroke}{rgb}{0.000000,0.000000,0.000000}%
\pgfsetstrokecolor{currentstroke}%
\pgfsetdash{}{0pt}%
\pgfpathmoveto{\pgfqpoint{9.091593in}{0.998749in}}%
\pgfpathlineto{\pgfqpoint{9.208715in}{1.149785in}}%
\pgfpathlineto{\pgfqpoint{9.230175in}{1.067062in}}%
\pgfpathlineto{\pgfqpoint{9.091593in}{0.998749in}}%
\pgfpathclose%
\pgfusepath{fill}%
\end{pgfscope}%
\begin{pgfscope}%
\pgfpathrectangle{\pgfqpoint{6.818937in}{0.147348in}}{\pgfqpoint{2.735294in}{2.735294in}}%
\pgfusepath{clip}%
\pgfsetbuttcap%
\pgfsetroundjoin%
\definecolor{currentfill}{rgb}{0.067488,0.245410,0.067488}%
\pgfsetfillcolor{currentfill}%
\pgfsetfillopacity{0.200000}%
\pgfsetlinewidth{0.000000pt}%
\definecolor{currentstroke}{rgb}{0.000000,0.000000,0.000000}%
\pgfsetstrokecolor{currentstroke}%
\pgfsetdash{}{0pt}%
\pgfpathmoveto{\pgfqpoint{7.216921in}{1.067062in}}%
\pgfpathlineto{\pgfqpoint{7.238381in}{1.149785in}}%
\pgfpathlineto{\pgfqpoint{7.355503in}{0.998749in}}%
\pgfpathlineto{\pgfqpoint{7.216921in}{1.067062in}}%
\pgfpathclose%
\pgfusepath{fill}%
\end{pgfscope}%
\begin{pgfscope}%
\pgfpathrectangle{\pgfqpoint{6.818937in}{0.147348in}}{\pgfqpoint{2.735294in}{2.735294in}}%
\pgfusepath{clip}%
\pgfsetbuttcap%
\pgfsetroundjoin%
\definecolor{currentfill}{rgb}{0.128601,0.467641,0.128601}%
\pgfsetfillcolor{currentfill}%
\pgfsetfillopacity{0.200000}%
\pgfsetlinewidth{0.000000pt}%
\definecolor{currentstroke}{rgb}{0.000000,0.000000,0.000000}%
\pgfsetstrokecolor{currentstroke}%
\pgfsetdash{}{0pt}%
\pgfpathmoveto{\pgfqpoint{8.323888in}{2.676888in}}%
\pgfpathlineto{\pgfqpoint{8.123207in}{2.676888in}}%
\pgfpathlineto{\pgfqpoint{8.223548in}{2.756756in}}%
\pgfpathlineto{\pgfqpoint{8.323888in}{2.676888in}}%
\pgfpathclose%
\pgfusepath{fill}%
\end{pgfscope}%
\begin{pgfscope}%
\pgfpathrectangle{\pgfqpoint{6.818937in}{0.147348in}}{\pgfqpoint{2.735294in}{2.735294in}}%
\pgfusepath{clip}%
\pgfsetbuttcap%
\pgfsetroundjoin%
\definecolor{currentfill}{rgb}{0.071067,0.258424,0.071067}%
\pgfsetfillcolor{currentfill}%
\pgfsetfillopacity{0.200000}%
\pgfsetlinewidth{0.000000pt}%
\definecolor{currentstroke}{rgb}{0.000000,0.000000,0.000000}%
\pgfsetstrokecolor{currentstroke}%
\pgfsetdash{}{0pt}%
\pgfpathmoveto{\pgfqpoint{9.230175in}{1.067062in}}%
\pgfpathlineto{\pgfqpoint{9.330515in}{1.213837in}}%
\pgfpathlineto{\pgfqpoint{9.340537in}{1.133087in}}%
\pgfpathlineto{\pgfqpoint{9.230175in}{1.067062in}}%
\pgfpathclose%
\pgfusepath{fill}%
\end{pgfscope}%
\begin{pgfscope}%
\pgfpathrectangle{\pgfqpoint{6.818937in}{0.147348in}}{\pgfqpoint{2.735294in}{2.735294in}}%
\pgfusepath{clip}%
\pgfsetbuttcap%
\pgfsetroundjoin%
\definecolor{currentfill}{rgb}{0.071067,0.258424,0.071067}%
\pgfsetfillcolor{currentfill}%
\pgfsetfillopacity{0.200000}%
\pgfsetlinewidth{0.000000pt}%
\definecolor{currentstroke}{rgb}{0.000000,0.000000,0.000000}%
\pgfsetstrokecolor{currentstroke}%
\pgfsetdash{}{0pt}%
\pgfpathmoveto{\pgfqpoint{7.116581in}{1.213837in}}%
\pgfpathlineto{\pgfqpoint{7.216921in}{1.067062in}}%
\pgfpathlineto{\pgfqpoint{7.106558in}{1.133087in}}%
\pgfpathlineto{\pgfqpoint{7.116581in}{1.213837in}}%
\pgfpathclose%
\pgfusepath{fill}%
\end{pgfscope}%
\begin{pgfscope}%
\pgfpathrectangle{\pgfqpoint{6.818937in}{0.147348in}}{\pgfqpoint{2.735294in}{2.735294in}}%
\pgfusepath{clip}%
\pgfsetbuttcap%
\pgfsetroundjoin%
\definecolor{currentfill}{rgb}{0.052607,0.201942,0.305459}%
\pgfsetfillcolor{currentfill}%
\pgfsetlinewidth{0.000000pt}%
\definecolor{currentstroke}{rgb}{0.000000,0.000000,0.000000}%
\pgfsetstrokecolor{currentstroke}%
\pgfsetdash{}{0pt}%
\pgfpathmoveto{\pgfqpoint{8.502215in}{1.541694in}}%
\pgfpathlineto{\pgfqpoint{8.362938in}{1.338584in}}%
\pgfpathlineto{\pgfqpoint{8.223548in}{1.534090in}}%
\pgfpathlineto{\pgfqpoint{8.502215in}{1.541694in}}%
\pgfpathclose%
\pgfusepath{fill}%
\end{pgfscope}%
\begin{pgfscope}%
\pgfpathrectangle{\pgfqpoint{6.818937in}{0.147348in}}{\pgfqpoint{2.735294in}{2.735294in}}%
\pgfusepath{clip}%
\pgfsetbuttcap%
\pgfsetroundjoin%
\definecolor{currentfill}{rgb}{0.052607,0.201942,0.305459}%
\pgfsetfillcolor{currentfill}%
\pgfsetlinewidth{0.000000pt}%
\definecolor{currentstroke}{rgb}{0.000000,0.000000,0.000000}%
\pgfsetstrokecolor{currentstroke}%
\pgfsetdash{}{0pt}%
\pgfpathmoveto{\pgfqpoint{8.223548in}{1.534090in}}%
\pgfpathlineto{\pgfqpoint{8.084158in}{1.338584in}}%
\pgfpathlineto{\pgfqpoint{7.944881in}{1.541694in}}%
\pgfpathlineto{\pgfqpoint{8.223548in}{1.534090in}}%
\pgfpathclose%
\pgfusepath{fill}%
\end{pgfscope}%
\begin{pgfscope}%
\pgfpathrectangle{\pgfqpoint{6.818937in}{0.147348in}}{\pgfqpoint{2.735294in}{2.735294in}}%
\pgfusepath{clip}%
\pgfsetbuttcap%
\pgfsetroundjoin%
\definecolor{currentfill}{rgb}{0.060773,0.233289,0.352874}%
\pgfsetfillcolor{currentfill}%
\pgfsetlinewidth{0.000000pt}%
\definecolor{currentstroke}{rgb}{0.000000,0.000000,0.000000}%
\pgfsetstrokecolor{currentstroke}%
\pgfsetdash{}{0pt}%
\pgfpathmoveto{\pgfqpoint{8.223548in}{1.534090in}}%
\pgfpathlineto{\pgfqpoint{8.223548in}{1.947297in}}%
\pgfpathlineto{\pgfqpoint{8.502215in}{1.541694in}}%
\pgfpathlineto{\pgfqpoint{8.223548in}{1.534090in}}%
\pgfpathclose%
\pgfusepath{fill}%
\end{pgfscope}%
\begin{pgfscope}%
\pgfpathrectangle{\pgfqpoint{6.818937in}{0.147348in}}{\pgfqpoint{2.735294in}{2.735294in}}%
\pgfusepath{clip}%
\pgfsetbuttcap%
\pgfsetroundjoin%
\definecolor{currentfill}{rgb}{0.060773,0.233289,0.352874}%
\pgfsetfillcolor{currentfill}%
\pgfsetlinewidth{0.000000pt}%
\definecolor{currentstroke}{rgb}{0.000000,0.000000,0.000000}%
\pgfsetstrokecolor{currentstroke}%
\pgfsetdash{}{0pt}%
\pgfpathmoveto{\pgfqpoint{7.944881in}{1.541694in}}%
\pgfpathlineto{\pgfqpoint{8.223548in}{1.947297in}}%
\pgfpathlineto{\pgfqpoint{8.223548in}{1.534090in}}%
\pgfpathlineto{\pgfqpoint{7.944881in}{1.541694in}}%
\pgfpathclose%
\pgfusepath{fill}%
\end{pgfscope}%
\begin{pgfscope}%
\pgfpathrectangle{\pgfqpoint{6.818937in}{0.147348in}}{\pgfqpoint{2.735294in}{2.735294in}}%
\pgfusepath{clip}%
\pgfsetbuttcap%
\pgfsetroundjoin%
\definecolor{currentfill}{rgb}{0.060634,0.232757,0.352069}%
\pgfsetfillcolor{currentfill}%
\pgfsetlinewidth{0.000000pt}%
\definecolor{currentstroke}{rgb}{0.000000,0.000000,0.000000}%
\pgfsetstrokecolor{currentstroke}%
\pgfsetdash{}{0pt}%
\pgfpathmoveto{\pgfqpoint{8.487849in}{1.947240in}}%
\pgfpathlineto{\pgfqpoint{8.502215in}{1.541694in}}%
\pgfpathlineto{\pgfqpoint{8.223548in}{1.947297in}}%
\pgfpathlineto{\pgfqpoint{8.487849in}{1.947240in}}%
\pgfpathclose%
\pgfusepath{fill}%
\end{pgfscope}%
\begin{pgfscope}%
\pgfpathrectangle{\pgfqpoint{6.818937in}{0.147348in}}{\pgfqpoint{2.735294in}{2.735294in}}%
\pgfusepath{clip}%
\pgfsetbuttcap%
\pgfsetroundjoin%
\definecolor{currentfill}{rgb}{0.060634,0.232757,0.352069}%
\pgfsetfillcolor{currentfill}%
\pgfsetlinewidth{0.000000pt}%
\definecolor{currentstroke}{rgb}{0.000000,0.000000,0.000000}%
\pgfsetstrokecolor{currentstroke}%
\pgfsetdash{}{0pt}%
\pgfpathmoveto{\pgfqpoint{8.223548in}{1.947297in}}%
\pgfpathlineto{\pgfqpoint{7.944881in}{1.541694in}}%
\pgfpathlineto{\pgfqpoint{7.959246in}{1.947240in}}%
\pgfpathlineto{\pgfqpoint{8.223548in}{1.947297in}}%
\pgfpathclose%
\pgfusepath{fill}%
\end{pgfscope}%
\begin{pgfscope}%
\pgfpathrectangle{\pgfqpoint{6.818937in}{0.147348in}}{\pgfqpoint{2.735294in}{2.735294in}}%
\pgfusepath{clip}%
\pgfsetbuttcap%
\pgfsetroundjoin%
\definecolor{currentfill}{rgb}{0.053541,0.205528,0.310883}%
\pgfsetfillcolor{currentfill}%
\pgfsetlinewidth{0.000000pt}%
\definecolor{currentstroke}{rgb}{0.000000,0.000000,0.000000}%
\pgfsetstrokecolor{currentstroke}%
\pgfsetdash{}{0pt}%
\pgfpathmoveto{\pgfqpoint{8.752556in}{1.562133in}}%
\pgfpathlineto{\pgfqpoint{8.627169in}{1.359642in}}%
\pgfpathlineto{\pgfqpoint{8.502215in}{1.541694in}}%
\pgfpathlineto{\pgfqpoint{8.752556in}{1.562133in}}%
\pgfpathclose%
\pgfusepath{fill}%
\end{pgfscope}%
\begin{pgfscope}%
\pgfpathrectangle{\pgfqpoint{6.818937in}{0.147348in}}{\pgfqpoint{2.735294in}{2.735294in}}%
\pgfusepath{clip}%
\pgfsetbuttcap%
\pgfsetroundjoin%
\definecolor{currentfill}{rgb}{0.053541,0.205528,0.310883}%
\pgfsetfillcolor{currentfill}%
\pgfsetlinewidth{0.000000pt}%
\definecolor{currentstroke}{rgb}{0.000000,0.000000,0.000000}%
\pgfsetstrokecolor{currentstroke}%
\pgfsetdash{}{0pt}%
\pgfpathmoveto{\pgfqpoint{7.944881in}{1.541694in}}%
\pgfpathlineto{\pgfqpoint{7.819927in}{1.359642in}}%
\pgfpathlineto{\pgfqpoint{7.694540in}{1.562133in}}%
\pgfpathlineto{\pgfqpoint{7.944881in}{1.541694in}}%
\pgfpathclose%
\pgfusepath{fill}%
\end{pgfscope}%
\begin{pgfscope}%
\pgfpathrectangle{\pgfqpoint{6.818937in}{0.147348in}}{\pgfqpoint{2.735294in}{2.735294in}}%
\pgfusepath{clip}%
\pgfsetbuttcap%
\pgfsetroundjoin%
\definecolor{currentfill}{rgb}{0.061576,0.236373,0.357539}%
\pgfsetfillcolor{currentfill}%
\pgfsetlinewidth{0.000000pt}%
\definecolor{currentstroke}{rgb}{0.000000,0.000000,0.000000}%
\pgfsetstrokecolor{currentstroke}%
\pgfsetdash{}{0pt}%
\pgfpathmoveto{\pgfqpoint{7.694540in}{1.562133in}}%
\pgfpathlineto{\pgfqpoint{7.959246in}{1.947240in}}%
\pgfpathlineto{\pgfqpoint{7.944881in}{1.541694in}}%
\pgfpathlineto{\pgfqpoint{7.694540in}{1.562133in}}%
\pgfpathclose%
\pgfusepath{fill}%
\end{pgfscope}%
\begin{pgfscope}%
\pgfpathrectangle{\pgfqpoint{6.818937in}{0.147348in}}{\pgfqpoint{2.735294in}{2.735294in}}%
\pgfusepath{clip}%
\pgfsetbuttcap%
\pgfsetroundjoin%
\definecolor{currentfill}{rgb}{0.061576,0.236373,0.357539}%
\pgfsetfillcolor{currentfill}%
\pgfsetlinewidth{0.000000pt}%
\definecolor{currentstroke}{rgb}{0.000000,0.000000,0.000000}%
\pgfsetstrokecolor{currentstroke}%
\pgfsetdash{}{0pt}%
\pgfpathmoveto{\pgfqpoint{8.502215in}{1.541694in}}%
\pgfpathlineto{\pgfqpoint{8.487849in}{1.947240in}}%
\pgfpathlineto{\pgfqpoint{8.752556in}{1.562133in}}%
\pgfpathlineto{\pgfqpoint{8.502215in}{1.541694in}}%
\pgfpathclose%
\pgfusepath{fill}%
\end{pgfscope}%
\begin{pgfscope}%
\pgfpathrectangle{\pgfqpoint{6.818937in}{0.147348in}}{\pgfqpoint{2.735294in}{2.735294in}}%
\pgfusepath{clip}%
\pgfsetbuttcap%
\pgfsetroundjoin%
\definecolor{currentfill}{rgb}{0.049465,0.189883,0.287218}%
\pgfsetfillcolor{currentfill}%
\pgfsetlinewidth{0.000000pt}%
\definecolor{currentstroke}{rgb}{0.000000,0.000000,0.000000}%
\pgfsetstrokecolor{currentstroke}%
\pgfsetdash{}{0pt}%
\pgfpathmoveto{\pgfqpoint{8.591651in}{1.053752in}}%
\pgfpathlineto{\pgfqpoint{8.362938in}{1.338584in}}%
\pgfpathlineto{\pgfqpoint{8.502215in}{1.541694in}}%
\pgfpathlineto{\pgfqpoint{8.591651in}{1.053752in}}%
\pgfpathclose%
\pgfusepath{fill}%
\end{pgfscope}%
\begin{pgfscope}%
\pgfpathrectangle{\pgfqpoint{6.818937in}{0.147348in}}{\pgfqpoint{2.735294in}{2.735294in}}%
\pgfusepath{clip}%
\pgfsetbuttcap%
\pgfsetroundjoin%
\definecolor{currentfill}{rgb}{0.049465,0.189883,0.287218}%
\pgfsetfillcolor{currentfill}%
\pgfsetlinewidth{0.000000pt}%
\definecolor{currentstroke}{rgb}{0.000000,0.000000,0.000000}%
\pgfsetstrokecolor{currentstroke}%
\pgfsetdash{}{0pt}%
\pgfpathmoveto{\pgfqpoint{7.944881in}{1.541694in}}%
\pgfpathlineto{\pgfqpoint{8.084158in}{1.338584in}}%
\pgfpathlineto{\pgfqpoint{7.855445in}{1.053752in}}%
\pgfpathlineto{\pgfqpoint{7.944881in}{1.541694in}}%
\pgfpathclose%
\pgfusepath{fill}%
\end{pgfscope}%
\begin{pgfscope}%
\pgfpathrectangle{\pgfqpoint{6.818937in}{0.147348in}}{\pgfqpoint{2.735294in}{2.735294in}}%
\pgfusepath{clip}%
\pgfsetbuttcap%
\pgfsetroundjoin%
\definecolor{currentfill}{rgb}{0.069261,0.265872,0.402159}%
\pgfsetfillcolor{currentfill}%
\pgfsetlinewidth{0.000000pt}%
\definecolor{currentstroke}{rgb}{0.000000,0.000000,0.000000}%
\pgfsetstrokecolor{currentstroke}%
\pgfsetdash{}{0pt}%
\pgfpathmoveto{\pgfqpoint{8.223548in}{1.947297in}}%
\pgfpathlineto{\pgfqpoint{8.349584in}{2.130589in}}%
\pgfpathlineto{\pgfqpoint{8.487849in}{1.947240in}}%
\pgfpathlineto{\pgfqpoint{8.223548in}{1.947297in}}%
\pgfpathclose%
\pgfusepath{fill}%
\end{pgfscope}%
\begin{pgfscope}%
\pgfpathrectangle{\pgfqpoint{6.818937in}{0.147348in}}{\pgfqpoint{2.735294in}{2.735294in}}%
\pgfusepath{clip}%
\pgfsetbuttcap%
\pgfsetroundjoin%
\definecolor{currentfill}{rgb}{0.069261,0.265872,0.402159}%
\pgfsetfillcolor{currentfill}%
\pgfsetlinewidth{0.000000pt}%
\definecolor{currentstroke}{rgb}{0.000000,0.000000,0.000000}%
\pgfsetstrokecolor{currentstroke}%
\pgfsetdash{}{0pt}%
\pgfpathmoveto{\pgfqpoint{7.959246in}{1.947240in}}%
\pgfpathlineto{\pgfqpoint{8.097511in}{2.130589in}}%
\pgfpathlineto{\pgfqpoint{8.223548in}{1.947297in}}%
\pgfpathlineto{\pgfqpoint{7.959246in}{1.947240in}}%
\pgfpathclose%
\pgfusepath{fill}%
\end{pgfscope}%
\begin{pgfscope}%
\pgfpathrectangle{\pgfqpoint{6.818937in}{0.147348in}}{\pgfqpoint{2.735294in}{2.735294in}}%
\pgfusepath{clip}%
\pgfsetbuttcap%
\pgfsetroundjoin%
\definecolor{currentfill}{rgb}{0.046814,0.179706,0.271825}%
\pgfsetfillcolor{currentfill}%
\pgfsetlinewidth{0.000000pt}%
\definecolor{currentstroke}{rgb}{0.000000,0.000000,0.000000}%
\pgfsetstrokecolor{currentstroke}%
\pgfsetdash{}{0pt}%
\pgfpathmoveto{\pgfqpoint{8.362938in}{1.338584in}}%
\pgfpathlineto{\pgfqpoint{8.223548in}{0.917160in}}%
\pgfpathlineto{\pgfqpoint{8.223548in}{1.534090in}}%
\pgfpathlineto{\pgfqpoint{8.362938in}{1.338584in}}%
\pgfpathclose%
\pgfusepath{fill}%
\end{pgfscope}%
\begin{pgfscope}%
\pgfpathrectangle{\pgfqpoint{6.818937in}{0.147348in}}{\pgfqpoint{2.735294in}{2.735294in}}%
\pgfusepath{clip}%
\pgfsetbuttcap%
\pgfsetroundjoin%
\definecolor{currentfill}{rgb}{0.046814,0.179706,0.271825}%
\pgfsetfillcolor{currentfill}%
\pgfsetlinewidth{0.000000pt}%
\definecolor{currentstroke}{rgb}{0.000000,0.000000,0.000000}%
\pgfsetstrokecolor{currentstroke}%
\pgfsetdash{}{0pt}%
\pgfpathmoveto{\pgfqpoint{8.223548in}{1.534090in}}%
\pgfpathlineto{\pgfqpoint{8.223548in}{0.917160in}}%
\pgfpathlineto{\pgfqpoint{8.084158in}{1.338584in}}%
\pgfpathlineto{\pgfqpoint{8.223548in}{1.534090in}}%
\pgfpathclose%
\pgfusepath{fill}%
\end{pgfscope}%
\begin{pgfscope}%
\pgfpathrectangle{\pgfqpoint{6.818937in}{0.147348in}}{\pgfqpoint{2.735294in}{2.735294in}}%
\pgfusepath{clip}%
\pgfsetbuttcap%
\pgfsetroundjoin%
\definecolor{currentfill}{rgb}{0.045820,0.175891,0.266053}%
\pgfsetfillcolor{currentfill}%
\pgfsetlinewidth{0.000000pt}%
\definecolor{currentstroke}{rgb}{0.000000,0.000000,0.000000}%
\pgfsetstrokecolor{currentstroke}%
\pgfsetdash{}{0pt}%
\pgfpathmoveto{\pgfqpoint{8.591651in}{1.053752in}}%
\pgfpathlineto{\pgfqpoint{8.502215in}{1.541694in}}%
\pgfpathlineto{\pgfqpoint{8.627169in}{1.359642in}}%
\pgfpathlineto{\pgfqpoint{8.591651in}{1.053752in}}%
\pgfpathclose%
\pgfusepath{fill}%
\end{pgfscope}%
\begin{pgfscope}%
\pgfpathrectangle{\pgfqpoint{6.818937in}{0.147348in}}{\pgfqpoint{2.735294in}{2.735294in}}%
\pgfusepath{clip}%
\pgfsetbuttcap%
\pgfsetroundjoin%
\definecolor{currentfill}{rgb}{0.045820,0.175891,0.266053}%
\pgfsetfillcolor{currentfill}%
\pgfsetlinewidth{0.000000pt}%
\definecolor{currentstroke}{rgb}{0.000000,0.000000,0.000000}%
\pgfsetstrokecolor{currentstroke}%
\pgfsetdash{}{0pt}%
\pgfpathmoveto{\pgfqpoint{7.819927in}{1.359642in}}%
\pgfpathlineto{\pgfqpoint{7.944881in}{1.541694in}}%
\pgfpathlineto{\pgfqpoint{7.855445in}{1.053752in}}%
\pgfpathlineto{\pgfqpoint{7.819927in}{1.359642in}}%
\pgfpathclose%
\pgfusepath{fill}%
\end{pgfscope}%
\begin{pgfscope}%
\pgfpathrectangle{\pgfqpoint{6.818937in}{0.147348in}}{\pgfqpoint{2.735294in}{2.735294in}}%
\pgfusepath{clip}%
\pgfsetbuttcap%
\pgfsetroundjoin%
\definecolor{currentfill}{rgb}{0.071636,0.274990,0.415951}%
\pgfsetfillcolor{currentfill}%
\pgfsetlinewidth{0.000000pt}%
\definecolor{currentstroke}{rgb}{0.000000,0.000000,0.000000}%
\pgfsetstrokecolor{currentstroke}%
\pgfsetdash{}{0pt}%
\pgfpathmoveto{\pgfqpoint{8.223548in}{1.947297in}}%
\pgfpathlineto{\pgfqpoint{8.097511in}{2.130589in}}%
\pgfpathlineto{\pgfqpoint{8.349584in}{2.130589in}}%
\pgfpathlineto{\pgfqpoint{8.223548in}{1.947297in}}%
\pgfpathclose%
\pgfusepath{fill}%
\end{pgfscope}%
\begin{pgfscope}%
\pgfpathrectangle{\pgfqpoint{6.818937in}{0.147348in}}{\pgfqpoint{2.735294in}{2.735294in}}%
\pgfusepath{clip}%
\pgfsetbuttcap%
\pgfsetroundjoin%
\definecolor{currentfill}{rgb}{0.071694,0.275212,0.416288}%
\pgfsetfillcolor{currentfill}%
\pgfsetlinewidth{0.000000pt}%
\definecolor{currentstroke}{rgb}{0.000000,0.000000,0.000000}%
\pgfsetstrokecolor{currentstroke}%
\pgfsetdash{}{0pt}%
\pgfpathmoveto{\pgfqpoint{8.487849in}{1.947240in}}%
\pgfpathlineto{\pgfqpoint{8.349584in}{2.130589in}}%
\pgfpathlineto{\pgfqpoint{8.590867in}{2.125260in}}%
\pgfpathlineto{\pgfqpoint{8.487849in}{1.947240in}}%
\pgfpathclose%
\pgfusepath{fill}%
\end{pgfscope}%
\begin{pgfscope}%
\pgfpathrectangle{\pgfqpoint{6.818937in}{0.147348in}}{\pgfqpoint{2.735294in}{2.735294in}}%
\pgfusepath{clip}%
\pgfsetbuttcap%
\pgfsetroundjoin%
\definecolor{currentfill}{rgb}{0.071694,0.275212,0.416288}%
\pgfsetfillcolor{currentfill}%
\pgfsetlinewidth{0.000000pt}%
\definecolor{currentstroke}{rgb}{0.000000,0.000000,0.000000}%
\pgfsetstrokecolor{currentstroke}%
\pgfsetdash{}{0pt}%
\pgfpathmoveto{\pgfqpoint{7.856229in}{2.125260in}}%
\pgfpathlineto{\pgfqpoint{8.097511in}{2.130589in}}%
\pgfpathlineto{\pgfqpoint{7.959246in}{1.947240in}}%
\pgfpathlineto{\pgfqpoint{7.856229in}{2.125260in}}%
\pgfpathclose%
\pgfusepath{fill}%
\end{pgfscope}%
\begin{pgfscope}%
\pgfpathrectangle{\pgfqpoint{6.818937in}{0.147348in}}{\pgfqpoint{2.735294in}{2.735294in}}%
\pgfusepath{clip}%
\pgfsetbuttcap%
\pgfsetroundjoin%
\definecolor{currentfill}{rgb}{0.064759,0.248590,0.376018}%
\pgfsetfillcolor{currentfill}%
\pgfsetlinewidth{0.000000pt}%
\definecolor{currentstroke}{rgb}{0.000000,0.000000,0.000000}%
\pgfsetstrokecolor{currentstroke}%
\pgfsetdash{}{0pt}%
\pgfpathmoveto{\pgfqpoint{8.752556in}{1.562133in}}%
\pgfpathlineto{\pgfqpoint{8.487849in}{1.947240in}}%
\pgfpathlineto{\pgfqpoint{8.804056in}{2.115869in}}%
\pgfpathlineto{\pgfqpoint{8.752556in}{1.562133in}}%
\pgfpathclose%
\pgfusepath{fill}%
\end{pgfscope}%
\begin{pgfscope}%
\pgfpathrectangle{\pgfqpoint{6.818937in}{0.147348in}}{\pgfqpoint{2.735294in}{2.735294in}}%
\pgfusepath{clip}%
\pgfsetbuttcap%
\pgfsetroundjoin%
\definecolor{currentfill}{rgb}{0.064759,0.248590,0.376018}%
\pgfsetfillcolor{currentfill}%
\pgfsetlinewidth{0.000000pt}%
\definecolor{currentstroke}{rgb}{0.000000,0.000000,0.000000}%
\pgfsetstrokecolor{currentstroke}%
\pgfsetdash{}{0pt}%
\pgfpathmoveto{\pgfqpoint{7.643040in}{2.115869in}}%
\pgfpathlineto{\pgfqpoint{7.959246in}{1.947240in}}%
\pgfpathlineto{\pgfqpoint{7.694540in}{1.562133in}}%
\pgfpathlineto{\pgfqpoint{7.643040in}{2.115869in}}%
\pgfpathclose%
\pgfusepath{fill}%
\end{pgfscope}%
\begin{pgfscope}%
\pgfpathrectangle{\pgfqpoint{6.818937in}{0.147348in}}{\pgfqpoint{2.735294in}{2.735294in}}%
\pgfusepath{clip}%
\pgfsetbuttcap%
\pgfsetroundjoin%
\definecolor{currentfill}{rgb}{0.051850,0.199036,0.301063}%
\pgfsetfillcolor{currentfill}%
\pgfsetlinewidth{0.000000pt}%
\definecolor{currentstroke}{rgb}{0.000000,0.000000,0.000000}%
\pgfsetstrokecolor{currentstroke}%
\pgfsetdash{}{0pt}%
\pgfpathmoveto{\pgfqpoint{8.805232in}{1.099921in}}%
\pgfpathlineto{\pgfqpoint{8.627169in}{1.359642in}}%
\pgfpathlineto{\pgfqpoint{8.752556in}{1.562133in}}%
\pgfpathlineto{\pgfqpoint{8.805232in}{1.099921in}}%
\pgfpathclose%
\pgfusepath{fill}%
\end{pgfscope}%
\begin{pgfscope}%
\pgfpathrectangle{\pgfqpoint{6.818937in}{0.147348in}}{\pgfqpoint{2.735294in}{2.735294in}}%
\pgfusepath{clip}%
\pgfsetbuttcap%
\pgfsetroundjoin%
\definecolor{currentfill}{rgb}{0.051850,0.199036,0.301063}%
\pgfsetfillcolor{currentfill}%
\pgfsetlinewidth{0.000000pt}%
\definecolor{currentstroke}{rgb}{0.000000,0.000000,0.000000}%
\pgfsetstrokecolor{currentstroke}%
\pgfsetdash{}{0pt}%
\pgfpathmoveto{\pgfqpoint{7.694540in}{1.562133in}}%
\pgfpathlineto{\pgfqpoint{7.819927in}{1.359642in}}%
\pgfpathlineto{\pgfqpoint{7.641864in}{1.099921in}}%
\pgfpathlineto{\pgfqpoint{7.694540in}{1.562133in}}%
\pgfpathclose%
\pgfusepath{fill}%
\end{pgfscope}%
\begin{pgfscope}%
\pgfpathrectangle{\pgfqpoint{6.818937in}{0.147348in}}{\pgfqpoint{2.735294in}{2.735294in}}%
\pgfusepath{clip}%
\pgfsetbuttcap%
\pgfsetroundjoin%
\definecolor{currentfill}{rgb}{0.046101,0.176968,0.267683}%
\pgfsetfillcolor{currentfill}%
\pgfsetlinewidth{0.000000pt}%
\definecolor{currentstroke}{rgb}{0.000000,0.000000,0.000000}%
\pgfsetstrokecolor{currentstroke}%
\pgfsetdash{}{0pt}%
\pgfpathmoveto{\pgfqpoint{8.223548in}{0.917160in}}%
\pgfpathlineto{\pgfqpoint{8.362938in}{1.338584in}}%
\pgfpathlineto{\pgfqpoint{8.349861in}{1.027548in}}%
\pgfpathlineto{\pgfqpoint{8.223548in}{0.917160in}}%
\pgfpathclose%
\pgfusepath{fill}%
\end{pgfscope}%
\begin{pgfscope}%
\pgfpathrectangle{\pgfqpoint{6.818937in}{0.147348in}}{\pgfqpoint{2.735294in}{2.735294in}}%
\pgfusepath{clip}%
\pgfsetbuttcap%
\pgfsetroundjoin%
\definecolor{currentfill}{rgb}{0.046101,0.176968,0.267683}%
\pgfsetfillcolor{currentfill}%
\pgfsetlinewidth{0.000000pt}%
\definecolor{currentstroke}{rgb}{0.000000,0.000000,0.000000}%
\pgfsetstrokecolor{currentstroke}%
\pgfsetdash{}{0pt}%
\pgfpathmoveto{\pgfqpoint{8.097234in}{1.027548in}}%
\pgfpathlineto{\pgfqpoint{8.084158in}{1.338584in}}%
\pgfpathlineto{\pgfqpoint{8.223548in}{0.917160in}}%
\pgfpathlineto{\pgfqpoint{8.097234in}{1.027548in}}%
\pgfpathclose%
\pgfusepath{fill}%
\end{pgfscope}%
\begin{pgfscope}%
\pgfpathrectangle{\pgfqpoint{6.818937in}{0.147348in}}{\pgfqpoint{2.735294in}{2.735294in}}%
\pgfusepath{clip}%
\pgfsetbuttcap%
\pgfsetroundjoin%
\definecolor{currentfill}{rgb}{0.047555,0.182548,0.276123}%
\pgfsetfillcolor{currentfill}%
\pgfsetlinewidth{0.000000pt}%
\definecolor{currentstroke}{rgb}{0.000000,0.000000,0.000000}%
\pgfsetstrokecolor{currentstroke}%
\pgfsetdash{}{0pt}%
\pgfpathmoveto{\pgfqpoint{8.456300in}{0.930196in}}%
\pgfpathlineto{\pgfqpoint{8.349861in}{1.027548in}}%
\pgfpathlineto{\pgfqpoint{8.362938in}{1.338584in}}%
\pgfpathlineto{\pgfqpoint{8.456300in}{0.930196in}}%
\pgfpathclose%
\pgfusepath{fill}%
\end{pgfscope}%
\begin{pgfscope}%
\pgfpathrectangle{\pgfqpoint{6.818937in}{0.147348in}}{\pgfqpoint{2.735294in}{2.735294in}}%
\pgfusepath{clip}%
\pgfsetbuttcap%
\pgfsetroundjoin%
\definecolor{currentfill}{rgb}{0.047555,0.182548,0.276123}%
\pgfsetfillcolor{currentfill}%
\pgfsetlinewidth{0.000000pt}%
\definecolor{currentstroke}{rgb}{0.000000,0.000000,0.000000}%
\pgfsetstrokecolor{currentstroke}%
\pgfsetdash{}{0pt}%
\pgfpathmoveto{\pgfqpoint{8.084158in}{1.338584in}}%
\pgfpathlineto{\pgfqpoint{8.097234in}{1.027548in}}%
\pgfpathlineto{\pgfqpoint{7.990796in}{0.930196in}}%
\pgfpathlineto{\pgfqpoint{8.084158in}{1.338584in}}%
\pgfpathclose%
\pgfusepath{fill}%
\end{pgfscope}%
\begin{pgfscope}%
\pgfpathrectangle{\pgfqpoint{6.818937in}{0.147348in}}{\pgfqpoint{2.735294in}{2.735294in}}%
\pgfusepath{clip}%
\pgfsetbuttcap%
\pgfsetroundjoin%
\definecolor{currentfill}{rgb}{0.068541,0.263111,0.397982}%
\pgfsetfillcolor{currentfill}%
\pgfsetlinewidth{0.000000pt}%
\definecolor{currentstroke}{rgb}{0.000000,0.000000,0.000000}%
\pgfsetstrokecolor{currentstroke}%
\pgfsetdash{}{0pt}%
\pgfpathmoveto{\pgfqpoint{8.590867in}{2.125260in}}%
\pgfpathlineto{\pgfqpoint{8.804056in}{2.115869in}}%
\pgfpathlineto{\pgfqpoint{8.487849in}{1.947240in}}%
\pgfpathlineto{\pgfqpoint{8.590867in}{2.125260in}}%
\pgfpathclose%
\pgfusepath{fill}%
\end{pgfscope}%
\begin{pgfscope}%
\pgfpathrectangle{\pgfqpoint{6.818937in}{0.147348in}}{\pgfqpoint{2.735294in}{2.735294in}}%
\pgfusepath{clip}%
\pgfsetbuttcap%
\pgfsetroundjoin%
\definecolor{currentfill}{rgb}{0.068541,0.263111,0.397982}%
\pgfsetfillcolor{currentfill}%
\pgfsetlinewidth{0.000000pt}%
\definecolor{currentstroke}{rgb}{0.000000,0.000000,0.000000}%
\pgfsetstrokecolor{currentstroke}%
\pgfsetdash{}{0pt}%
\pgfpathmoveto{\pgfqpoint{7.959246in}{1.947240in}}%
\pgfpathlineto{\pgfqpoint{7.643040in}{2.115869in}}%
\pgfpathlineto{\pgfqpoint{7.856229in}{2.125260in}}%
\pgfpathlineto{\pgfqpoint{7.959246in}{1.947240in}}%
\pgfpathclose%
\pgfusepath{fill}%
\end{pgfscope}%
\begin{pgfscope}%
\pgfpathrectangle{\pgfqpoint{6.818937in}{0.147348in}}{\pgfqpoint{2.735294in}{2.735294in}}%
\pgfusepath{clip}%
\pgfsetbuttcap%
\pgfsetroundjoin%
\definecolor{currentfill}{rgb}{0.042669,0.163794,0.247755}%
\pgfsetfillcolor{currentfill}%
\pgfsetlinewidth{0.000000pt}%
\definecolor{currentstroke}{rgb}{0.000000,0.000000,0.000000}%
\pgfsetstrokecolor{currentstroke}%
\pgfsetdash{}{0pt}%
\pgfpathmoveto{\pgfqpoint{8.456300in}{0.930196in}}%
\pgfpathlineto{\pgfqpoint{8.362938in}{1.338584in}}%
\pgfpathlineto{\pgfqpoint{8.591651in}{1.053752in}}%
\pgfpathlineto{\pgfqpoint{8.456300in}{0.930196in}}%
\pgfpathclose%
\pgfusepath{fill}%
\end{pgfscope}%
\begin{pgfscope}%
\pgfpathrectangle{\pgfqpoint{6.818937in}{0.147348in}}{\pgfqpoint{2.735294in}{2.735294in}}%
\pgfusepath{clip}%
\pgfsetbuttcap%
\pgfsetroundjoin%
\definecolor{currentfill}{rgb}{0.042669,0.163794,0.247755}%
\pgfsetfillcolor{currentfill}%
\pgfsetlinewidth{0.000000pt}%
\definecolor{currentstroke}{rgb}{0.000000,0.000000,0.000000}%
\pgfsetstrokecolor{currentstroke}%
\pgfsetdash{}{0pt}%
\pgfpathmoveto{\pgfqpoint{7.855445in}{1.053752in}}%
\pgfpathlineto{\pgfqpoint{8.084158in}{1.338584in}}%
\pgfpathlineto{\pgfqpoint{7.990796in}{0.930196in}}%
\pgfpathlineto{\pgfqpoint{7.855445in}{1.053752in}}%
\pgfpathclose%
\pgfusepath{fill}%
\end{pgfscope}%
\begin{pgfscope}%
\pgfpathrectangle{\pgfqpoint{6.818937in}{0.147348in}}{\pgfqpoint{2.735294in}{2.735294in}}%
\pgfusepath{clip}%
\pgfsetbuttcap%
\pgfsetroundjoin%
\definecolor{currentfill}{rgb}{0.064954,0.249341,0.377155}%
\pgfsetfillcolor{currentfill}%
\pgfsetlinewidth{0.000000pt}%
\definecolor{currentstroke}{rgb}{0.000000,0.000000,0.000000}%
\pgfsetstrokecolor{currentstroke}%
\pgfsetdash{}{0pt}%
\pgfpathmoveto{\pgfqpoint{8.752556in}{1.562133in}}%
\pgfpathlineto{\pgfqpoint{8.804056in}{2.115869in}}%
\pgfpathlineto{\pgfqpoint{8.958969in}{1.590078in}}%
\pgfpathlineto{\pgfqpoint{8.752556in}{1.562133in}}%
\pgfpathclose%
\pgfusepath{fill}%
\end{pgfscope}%
\begin{pgfscope}%
\pgfpathrectangle{\pgfqpoint{6.818937in}{0.147348in}}{\pgfqpoint{2.735294in}{2.735294in}}%
\pgfusepath{clip}%
\pgfsetbuttcap%
\pgfsetroundjoin%
\definecolor{currentfill}{rgb}{0.064954,0.249341,0.377155}%
\pgfsetfillcolor{currentfill}%
\pgfsetlinewidth{0.000000pt}%
\definecolor{currentstroke}{rgb}{0.000000,0.000000,0.000000}%
\pgfsetstrokecolor{currentstroke}%
\pgfsetdash{}{0pt}%
\pgfpathmoveto{\pgfqpoint{7.488127in}{1.590078in}}%
\pgfpathlineto{\pgfqpoint{7.643040in}{2.115869in}}%
\pgfpathlineto{\pgfqpoint{7.694540in}{1.562133in}}%
\pgfpathlineto{\pgfqpoint{7.488127in}{1.590078in}}%
\pgfpathclose%
\pgfusepath{fill}%
\end{pgfscope}%
\begin{pgfscope}%
\pgfpathrectangle{\pgfqpoint{6.818937in}{0.147348in}}{\pgfqpoint{2.735294in}{2.735294in}}%
\pgfusepath{clip}%
\pgfsetbuttcap%
\pgfsetroundjoin%
\definecolor{currentfill}{rgb}{0.078663,0.301965,0.456754}%
\pgfsetfillcolor{currentfill}%
\pgfsetlinewidth{0.000000pt}%
\definecolor{currentstroke}{rgb}{0.000000,0.000000,0.000000}%
\pgfsetstrokecolor{currentstroke}%
\pgfsetdash{}{0pt}%
\pgfpathmoveto{\pgfqpoint{7.856229in}{2.125260in}}%
\pgfpathlineto{\pgfqpoint{7.775860in}{2.272632in}}%
\pgfpathlineto{\pgfqpoint{8.097511in}{2.130589in}}%
\pgfpathlineto{\pgfqpoint{7.856229in}{2.125260in}}%
\pgfpathclose%
\pgfusepath{fill}%
\end{pgfscope}%
\begin{pgfscope}%
\pgfpathrectangle{\pgfqpoint{6.818937in}{0.147348in}}{\pgfqpoint{2.735294in}{2.735294in}}%
\pgfusepath{clip}%
\pgfsetbuttcap%
\pgfsetroundjoin%
\definecolor{currentfill}{rgb}{0.078663,0.301965,0.456754}%
\pgfsetfillcolor{currentfill}%
\pgfsetlinewidth{0.000000pt}%
\definecolor{currentstroke}{rgb}{0.000000,0.000000,0.000000}%
\pgfsetstrokecolor{currentstroke}%
\pgfsetdash{}{0pt}%
\pgfpathmoveto{\pgfqpoint{8.349584in}{2.130589in}}%
\pgfpathlineto{\pgfqpoint{8.671236in}{2.272632in}}%
\pgfpathlineto{\pgfqpoint{8.590867in}{2.125260in}}%
\pgfpathlineto{\pgfqpoint{8.349584in}{2.130589in}}%
\pgfpathclose%
\pgfusepath{fill}%
\end{pgfscope}%
\begin{pgfscope}%
\pgfpathrectangle{\pgfqpoint{6.818937in}{0.147348in}}{\pgfqpoint{2.735294in}{2.735294in}}%
\pgfusepath{clip}%
\pgfsetbuttcap%
\pgfsetroundjoin%
\definecolor{currentfill}{rgb}{0.049941,0.191710,0.289982}%
\pgfsetfillcolor{currentfill}%
\pgfsetlinewidth{0.000000pt}%
\definecolor{currentstroke}{rgb}{0.000000,0.000000,0.000000}%
\pgfsetstrokecolor{currentstroke}%
\pgfsetdash{}{0pt}%
\pgfpathmoveto{\pgfqpoint{7.855445in}{1.053752in}}%
\pgfpathlineto{\pgfqpoint{7.774694in}{0.966453in}}%
\pgfpathlineto{\pgfqpoint{7.819927in}{1.359642in}}%
\pgfpathlineto{\pgfqpoint{7.855445in}{1.053752in}}%
\pgfpathclose%
\pgfusepath{fill}%
\end{pgfscope}%
\begin{pgfscope}%
\pgfpathrectangle{\pgfqpoint{6.818937in}{0.147348in}}{\pgfqpoint{2.735294in}{2.735294in}}%
\pgfusepath{clip}%
\pgfsetbuttcap%
\pgfsetroundjoin%
\definecolor{currentfill}{rgb}{0.049941,0.191710,0.289982}%
\pgfsetfillcolor{currentfill}%
\pgfsetlinewidth{0.000000pt}%
\definecolor{currentstroke}{rgb}{0.000000,0.000000,0.000000}%
\pgfsetstrokecolor{currentstroke}%
\pgfsetdash{}{0pt}%
\pgfpathmoveto{\pgfqpoint{8.627169in}{1.359642in}}%
\pgfpathlineto{\pgfqpoint{8.672401in}{0.966453in}}%
\pgfpathlineto{\pgfqpoint{8.591651in}{1.053752in}}%
\pgfpathlineto{\pgfqpoint{8.627169in}{1.359642in}}%
\pgfpathclose%
\pgfusepath{fill}%
\end{pgfscope}%
\begin{pgfscope}%
\pgfpathrectangle{\pgfqpoint{6.818937in}{0.147348in}}{\pgfqpoint{2.735294in}{2.735294in}}%
\pgfusepath{clip}%
\pgfsetbuttcap%
\pgfsetroundjoin%
\definecolor{currentfill}{rgb}{0.050011,0.191979,0.290388}%
\pgfsetfillcolor{currentfill}%
\pgfsetlinewidth{0.000000pt}%
\definecolor{currentstroke}{rgb}{0.000000,0.000000,0.000000}%
\pgfsetstrokecolor{currentstroke}%
\pgfsetdash{}{0pt}%
\pgfpathmoveto{\pgfqpoint{8.958969in}{1.590078in}}%
\pgfpathlineto{\pgfqpoint{8.982643in}{1.157190in}}%
\pgfpathlineto{\pgfqpoint{8.752556in}{1.562133in}}%
\pgfpathlineto{\pgfqpoint{8.958969in}{1.590078in}}%
\pgfpathclose%
\pgfusepath{fill}%
\end{pgfscope}%
\begin{pgfscope}%
\pgfpathrectangle{\pgfqpoint{6.818937in}{0.147348in}}{\pgfqpoint{2.735294in}{2.735294in}}%
\pgfusepath{clip}%
\pgfsetbuttcap%
\pgfsetroundjoin%
\definecolor{currentfill}{rgb}{0.050011,0.191979,0.290388}%
\pgfsetfillcolor{currentfill}%
\pgfsetlinewidth{0.000000pt}%
\definecolor{currentstroke}{rgb}{0.000000,0.000000,0.000000}%
\pgfsetstrokecolor{currentstroke}%
\pgfsetdash{}{0pt}%
\pgfpathmoveto{\pgfqpoint{7.694540in}{1.562133in}}%
\pgfpathlineto{\pgfqpoint{7.464453in}{1.157190in}}%
\pgfpathlineto{\pgfqpoint{7.488127in}{1.590078in}}%
\pgfpathlineto{\pgfqpoint{7.694540in}{1.562133in}}%
\pgfpathclose%
\pgfusepath{fill}%
\end{pgfscope}%
\begin{pgfscope}%
\pgfpathrectangle{\pgfqpoint{6.818937in}{0.147348in}}{\pgfqpoint{2.735294in}{2.735294in}}%
\pgfusepath{clip}%
\pgfsetbuttcap%
\pgfsetroundjoin%
\definecolor{currentfill}{rgb}{0.043508,0.167016,0.252629}%
\pgfsetfillcolor{currentfill}%
\pgfsetlinewidth{0.000000pt}%
\definecolor{currentstroke}{rgb}{0.000000,0.000000,0.000000}%
\pgfsetstrokecolor{currentstroke}%
\pgfsetdash{}{0pt}%
\pgfpathmoveto{\pgfqpoint{8.805232in}{1.099921in}}%
\pgfpathlineto{\pgfqpoint{8.672401in}{0.966453in}}%
\pgfpathlineto{\pgfqpoint{8.627169in}{1.359642in}}%
\pgfpathlineto{\pgfqpoint{8.805232in}{1.099921in}}%
\pgfpathclose%
\pgfusepath{fill}%
\end{pgfscope}%
\begin{pgfscope}%
\pgfpathrectangle{\pgfqpoint{6.818937in}{0.147348in}}{\pgfqpoint{2.735294in}{2.735294in}}%
\pgfusepath{clip}%
\pgfsetbuttcap%
\pgfsetroundjoin%
\definecolor{currentfill}{rgb}{0.043508,0.167016,0.252629}%
\pgfsetfillcolor{currentfill}%
\pgfsetlinewidth{0.000000pt}%
\definecolor{currentstroke}{rgb}{0.000000,0.000000,0.000000}%
\pgfsetstrokecolor{currentstroke}%
\pgfsetdash{}{0pt}%
\pgfpathmoveto{\pgfqpoint{7.819927in}{1.359642in}}%
\pgfpathlineto{\pgfqpoint{7.774694in}{0.966453in}}%
\pgfpathlineto{\pgfqpoint{7.641864in}{1.099921in}}%
\pgfpathlineto{\pgfqpoint{7.819927in}{1.359642in}}%
\pgfpathclose%
\pgfusepath{fill}%
\end{pgfscope}%
\begin{pgfscope}%
\pgfpathrectangle{\pgfqpoint{6.818937in}{0.147348in}}{\pgfqpoint{2.735294in}{2.735294in}}%
\pgfusepath{clip}%
\pgfsetbuttcap%
\pgfsetroundjoin%
\definecolor{currentfill}{rgb}{0.062760,0.240916,0.364410}%
\pgfsetfillcolor{currentfill}%
\pgfsetlinewidth{0.000000pt}%
\definecolor{currentstroke}{rgb}{0.000000,0.000000,0.000000}%
\pgfsetstrokecolor{currentstroke}%
\pgfsetdash{}{0pt}%
\pgfpathmoveto{\pgfqpoint{8.958969in}{1.590078in}}%
\pgfpathlineto{\pgfqpoint{8.804056in}{2.115869in}}%
\pgfpathlineto{\pgfqpoint{9.036807in}{1.777676in}}%
\pgfpathlineto{\pgfqpoint{8.958969in}{1.590078in}}%
\pgfpathclose%
\pgfusepath{fill}%
\end{pgfscope}%
\begin{pgfscope}%
\pgfpathrectangle{\pgfqpoint{6.818937in}{0.147348in}}{\pgfqpoint{2.735294in}{2.735294in}}%
\pgfusepath{clip}%
\pgfsetbuttcap%
\pgfsetroundjoin%
\definecolor{currentfill}{rgb}{0.062760,0.240916,0.364410}%
\pgfsetfillcolor{currentfill}%
\pgfsetlinewidth{0.000000pt}%
\definecolor{currentstroke}{rgb}{0.000000,0.000000,0.000000}%
\pgfsetstrokecolor{currentstroke}%
\pgfsetdash{}{0pt}%
\pgfpathmoveto{\pgfqpoint{7.410289in}{1.777676in}}%
\pgfpathlineto{\pgfqpoint{7.643040in}{2.115869in}}%
\pgfpathlineto{\pgfqpoint{7.488127in}{1.590078in}}%
\pgfpathlineto{\pgfqpoint{7.410289in}{1.777676in}}%
\pgfpathclose%
\pgfusepath{fill}%
\end{pgfscope}%
\begin{pgfscope}%
\pgfpathrectangle{\pgfqpoint{6.818937in}{0.147348in}}{\pgfqpoint{2.735294in}{2.735294in}}%
\pgfusepath{clip}%
\pgfsetbuttcap%
\pgfsetroundjoin%
\definecolor{currentfill}{rgb}{0.075436,0.289576,0.438014}%
\pgfsetfillcolor{currentfill}%
\pgfsetlinewidth{0.000000pt}%
\definecolor{currentstroke}{rgb}{0.000000,0.000000,0.000000}%
\pgfsetstrokecolor{currentstroke}%
\pgfsetdash{}{0pt}%
\pgfpathmoveto{\pgfqpoint{8.590867in}{2.125260in}}%
\pgfpathlineto{\pgfqpoint{8.671236in}{2.272632in}}%
\pgfpathlineto{\pgfqpoint{8.804056in}{2.115869in}}%
\pgfpathlineto{\pgfqpoint{8.590867in}{2.125260in}}%
\pgfpathclose%
\pgfusepath{fill}%
\end{pgfscope}%
\begin{pgfscope}%
\pgfpathrectangle{\pgfqpoint{6.818937in}{0.147348in}}{\pgfqpoint{2.735294in}{2.735294in}}%
\pgfusepath{clip}%
\pgfsetbuttcap%
\pgfsetroundjoin%
\definecolor{currentfill}{rgb}{0.075436,0.289576,0.438014}%
\pgfsetfillcolor{currentfill}%
\pgfsetlinewidth{0.000000pt}%
\definecolor{currentstroke}{rgb}{0.000000,0.000000,0.000000}%
\pgfsetstrokecolor{currentstroke}%
\pgfsetdash{}{0pt}%
\pgfpathmoveto{\pgfqpoint{7.643040in}{2.115869in}}%
\pgfpathlineto{\pgfqpoint{7.775860in}{2.272632in}}%
\pgfpathlineto{\pgfqpoint{7.856229in}{2.125260in}}%
\pgfpathlineto{\pgfqpoint{7.643040in}{2.115869in}}%
\pgfpathclose%
\pgfusepath{fill}%
\end{pgfscope}%
\begin{pgfscope}%
\pgfpathrectangle{\pgfqpoint{6.818937in}{0.147348in}}{\pgfqpoint{2.735294in}{2.735294in}}%
\pgfusepath{clip}%
\pgfsetbuttcap%
\pgfsetroundjoin%
\definecolor{currentfill}{rgb}{0.039595,0.151995,0.229908}%
\pgfsetfillcolor{currentfill}%
\pgfsetlinewidth{0.000000pt}%
\definecolor{currentstroke}{rgb}{0.000000,0.000000,0.000000}%
\pgfsetstrokecolor{currentstroke}%
\pgfsetdash{}{0pt}%
\pgfpathmoveto{\pgfqpoint{8.223548in}{0.917160in}}%
\pgfpathlineto{\pgfqpoint{8.349861in}{1.027548in}}%
\pgfpathlineto{\pgfqpoint{8.456300in}{0.930196in}}%
\pgfpathlineto{\pgfqpoint{8.223548in}{0.917160in}}%
\pgfpathclose%
\pgfusepath{fill}%
\end{pgfscope}%
\begin{pgfscope}%
\pgfpathrectangle{\pgfqpoint{6.818937in}{0.147348in}}{\pgfqpoint{2.735294in}{2.735294in}}%
\pgfusepath{clip}%
\pgfsetbuttcap%
\pgfsetroundjoin%
\definecolor{currentfill}{rgb}{0.039595,0.151995,0.229908}%
\pgfsetfillcolor{currentfill}%
\pgfsetlinewidth{0.000000pt}%
\definecolor{currentstroke}{rgb}{0.000000,0.000000,0.000000}%
\pgfsetstrokecolor{currentstroke}%
\pgfsetdash{}{0pt}%
\pgfpathmoveto{\pgfqpoint{7.990796in}{0.930196in}}%
\pgfpathlineto{\pgfqpoint{8.097234in}{1.027548in}}%
\pgfpathlineto{\pgfqpoint{8.223548in}{0.917160in}}%
\pgfpathlineto{\pgfqpoint{7.990796in}{0.930196in}}%
\pgfpathclose%
\pgfusepath{fill}%
\end{pgfscope}%
\begin{pgfscope}%
\pgfpathrectangle{\pgfqpoint{6.818937in}{0.147348in}}{\pgfqpoint{2.735294in}{2.735294in}}%
\pgfusepath{clip}%
\pgfsetbuttcap%
\pgfsetroundjoin%
\definecolor{currentfill}{rgb}{0.060942,0.233938,0.353856}%
\pgfsetfillcolor{currentfill}%
\pgfsetlinewidth{0.000000pt}%
\definecolor{currentstroke}{rgb}{0.000000,0.000000,0.000000}%
\pgfsetstrokecolor{currentstroke}%
\pgfsetdash{}{0pt}%
\pgfpathmoveto{\pgfqpoint{7.327008in}{1.620392in}}%
\pgfpathlineto{\pgfqpoint{7.410289in}{1.777676in}}%
\pgfpathlineto{\pgfqpoint{7.488127in}{1.590078in}}%
\pgfpathlineto{\pgfqpoint{7.327008in}{1.620392in}}%
\pgfpathclose%
\pgfusepath{fill}%
\end{pgfscope}%
\begin{pgfscope}%
\pgfpathrectangle{\pgfqpoint{6.818937in}{0.147348in}}{\pgfqpoint{2.735294in}{2.735294in}}%
\pgfusepath{clip}%
\pgfsetbuttcap%
\pgfsetroundjoin%
\definecolor{currentfill}{rgb}{0.060942,0.233938,0.353856}%
\pgfsetfillcolor{currentfill}%
\pgfsetlinewidth{0.000000pt}%
\definecolor{currentstroke}{rgb}{0.000000,0.000000,0.000000}%
\pgfsetstrokecolor{currentstroke}%
\pgfsetdash{}{0pt}%
\pgfpathmoveto{\pgfqpoint{8.958969in}{1.590078in}}%
\pgfpathlineto{\pgfqpoint{9.036807in}{1.777676in}}%
\pgfpathlineto{\pgfqpoint{9.120088in}{1.620392in}}%
\pgfpathlineto{\pgfqpoint{8.958969in}{1.590078in}}%
\pgfpathclose%
\pgfusepath{fill}%
\end{pgfscope}%
\begin{pgfscope}%
\pgfpathrectangle{\pgfqpoint{6.818937in}{0.147348in}}{\pgfqpoint{2.735294in}{2.735294in}}%
\pgfusepath{clip}%
\pgfsetbuttcap%
\pgfsetroundjoin%
\definecolor{currentfill}{rgb}{0.063981,0.245604,0.371502}%
\pgfsetfillcolor{currentfill}%
\pgfsetlinewidth{0.000000pt}%
\definecolor{currentstroke}{rgb}{0.000000,0.000000,0.000000}%
\pgfsetstrokecolor{currentstroke}%
\pgfsetdash{}{0pt}%
\pgfpathmoveto{\pgfqpoint{7.641864in}{1.099921in}}%
\pgfpathlineto{\pgfqpoint{7.586426in}{1.018945in}}%
\pgfpathlineto{\pgfqpoint{7.694540in}{1.562133in}}%
\pgfpathlineto{\pgfqpoint{7.641864in}{1.099921in}}%
\pgfpathclose%
\pgfusepath{fill}%
\end{pgfscope}%
\begin{pgfscope}%
\pgfpathrectangle{\pgfqpoint{6.818937in}{0.147348in}}{\pgfqpoint{2.735294in}{2.735294in}}%
\pgfusepath{clip}%
\pgfsetbuttcap%
\pgfsetroundjoin%
\definecolor{currentfill}{rgb}{0.063981,0.245604,0.371502}%
\pgfsetfillcolor{currentfill}%
\pgfsetlinewidth{0.000000pt}%
\definecolor{currentstroke}{rgb}{0.000000,0.000000,0.000000}%
\pgfsetstrokecolor{currentstroke}%
\pgfsetdash{}{0pt}%
\pgfpathmoveto{\pgfqpoint{8.752556in}{1.562133in}}%
\pgfpathlineto{\pgfqpoint{8.860669in}{1.018945in}}%
\pgfpathlineto{\pgfqpoint{8.805232in}{1.099921in}}%
\pgfpathlineto{\pgfqpoint{8.752556in}{1.562133in}}%
\pgfpathclose%
\pgfusepath{fill}%
\end{pgfscope}%
\begin{pgfscope}%
\pgfpathrectangle{\pgfqpoint{6.818937in}{0.147348in}}{\pgfqpoint{2.735294in}{2.735294in}}%
\pgfusepath{clip}%
\pgfsetbuttcap%
\pgfsetroundjoin%
\definecolor{currentfill}{rgb}{0.081954,0.314596,0.475860}%
\pgfsetfillcolor{currentfill}%
\pgfsetlinewidth{0.000000pt}%
\definecolor{currentstroke}{rgb}{0.000000,0.000000,0.000000}%
\pgfsetstrokecolor{currentstroke}%
\pgfsetdash{}{0pt}%
\pgfpathmoveto{\pgfqpoint{8.349584in}{2.130589in}}%
\pgfpathlineto{\pgfqpoint{8.097511in}{2.130589in}}%
\pgfpathlineto{\pgfqpoint{8.131822in}{2.613951in}}%
\pgfpathlineto{\pgfqpoint{8.349584in}{2.130589in}}%
\pgfpathclose%
\pgfusepath{fill}%
\end{pgfscope}%
\begin{pgfscope}%
\pgfpathrectangle{\pgfqpoint{6.818937in}{0.147348in}}{\pgfqpoint{2.735294in}{2.735294in}}%
\pgfusepath{clip}%
\pgfsetbuttcap%
\pgfsetroundjoin%
\definecolor{currentfill}{rgb}{0.040669,0.156116,0.236142}%
\pgfsetfillcolor{currentfill}%
\pgfsetlinewidth{0.000000pt}%
\definecolor{currentstroke}{rgb}{0.000000,0.000000,0.000000}%
\pgfsetstrokecolor{currentstroke}%
\pgfsetdash{}{0pt}%
\pgfpathmoveto{\pgfqpoint{8.591651in}{1.053752in}}%
\pgfpathlineto{\pgfqpoint{8.672401in}{0.966453in}}%
\pgfpathlineto{\pgfqpoint{8.456300in}{0.930196in}}%
\pgfpathlineto{\pgfqpoint{8.591651in}{1.053752in}}%
\pgfpathclose%
\pgfusepath{fill}%
\end{pgfscope}%
\begin{pgfscope}%
\pgfpathrectangle{\pgfqpoint{6.818937in}{0.147348in}}{\pgfqpoint{2.735294in}{2.735294in}}%
\pgfusepath{clip}%
\pgfsetbuttcap%
\pgfsetroundjoin%
\definecolor{currentfill}{rgb}{0.040669,0.156116,0.236142}%
\pgfsetfillcolor{currentfill}%
\pgfsetlinewidth{0.000000pt}%
\definecolor{currentstroke}{rgb}{0.000000,0.000000,0.000000}%
\pgfsetstrokecolor{currentstroke}%
\pgfsetdash{}{0pt}%
\pgfpathmoveto{\pgfqpoint{7.990796in}{0.930196in}}%
\pgfpathlineto{\pgfqpoint{7.774694in}{0.966453in}}%
\pgfpathlineto{\pgfqpoint{7.855445in}{1.053752in}}%
\pgfpathlineto{\pgfqpoint{7.990796in}{0.930196in}}%
\pgfpathclose%
\pgfusepath{fill}%
\end{pgfscope}%
\begin{pgfscope}%
\pgfpathrectangle{\pgfqpoint{6.818937in}{0.147348in}}{\pgfqpoint{2.735294in}{2.735294in}}%
\pgfusepath{clip}%
\pgfsetbuttcap%
\pgfsetroundjoin%
\definecolor{currentfill}{rgb}{0.045702,0.175435,0.265364}%
\pgfsetfillcolor{currentfill}%
\pgfsetlinewidth{0.000000pt}%
\definecolor{currentstroke}{rgb}{0.000000,0.000000,0.000000}%
\pgfsetstrokecolor{currentstroke}%
\pgfsetdash{}{0pt}%
\pgfpathmoveto{\pgfqpoint{8.982643in}{1.157190in}}%
\pgfpathlineto{\pgfqpoint{8.860669in}{1.018945in}}%
\pgfpathlineto{\pgfqpoint{8.752556in}{1.562133in}}%
\pgfpathlineto{\pgfqpoint{8.982643in}{1.157190in}}%
\pgfpathclose%
\pgfusepath{fill}%
\end{pgfscope}%
\begin{pgfscope}%
\pgfpathrectangle{\pgfqpoint{6.818937in}{0.147348in}}{\pgfqpoint{2.735294in}{2.735294in}}%
\pgfusepath{clip}%
\pgfsetbuttcap%
\pgfsetroundjoin%
\definecolor{currentfill}{rgb}{0.045702,0.175435,0.265364}%
\pgfsetfillcolor{currentfill}%
\pgfsetlinewidth{0.000000pt}%
\definecolor{currentstroke}{rgb}{0.000000,0.000000,0.000000}%
\pgfsetstrokecolor{currentstroke}%
\pgfsetdash{}{0pt}%
\pgfpathmoveto{\pgfqpoint{7.694540in}{1.562133in}}%
\pgfpathlineto{\pgfqpoint{7.586426in}{1.018945in}}%
\pgfpathlineto{\pgfqpoint{7.464453in}{1.157190in}}%
\pgfpathlineto{\pgfqpoint{7.694540in}{1.562133in}}%
\pgfpathclose%
\pgfusepath{fill}%
\end{pgfscope}%
\begin{pgfscope}%
\pgfpathrectangle{\pgfqpoint{6.818937in}{0.147348in}}{\pgfqpoint{2.735294in}{2.735294in}}%
\pgfusepath{clip}%
\pgfsetbuttcap%
\pgfsetroundjoin%
\definecolor{currentfill}{rgb}{0.082280,0.315849,0.477755}%
\pgfsetfillcolor{currentfill}%
\pgfsetlinewidth{0.000000pt}%
\definecolor{currentstroke}{rgb}{0.000000,0.000000,0.000000}%
\pgfsetstrokecolor{currentstroke}%
\pgfsetdash{}{0pt}%
\pgfpathmoveto{\pgfqpoint{8.349584in}{2.130589in}}%
\pgfpathlineto{\pgfqpoint{8.315273in}{2.613951in}}%
\pgfpathlineto{\pgfqpoint{8.671236in}{2.272632in}}%
\pgfpathlineto{\pgfqpoint{8.349584in}{2.130589in}}%
\pgfpathclose%
\pgfusepath{fill}%
\end{pgfscope}%
\begin{pgfscope}%
\pgfpathrectangle{\pgfqpoint{6.818937in}{0.147348in}}{\pgfqpoint{2.735294in}{2.735294in}}%
\pgfusepath{clip}%
\pgfsetbuttcap%
\pgfsetroundjoin%
\definecolor{currentfill}{rgb}{0.082280,0.315849,0.477755}%
\pgfsetfillcolor{currentfill}%
\pgfsetlinewidth{0.000000pt}%
\definecolor{currentstroke}{rgb}{0.000000,0.000000,0.000000}%
\pgfsetstrokecolor{currentstroke}%
\pgfsetdash{}{0pt}%
\pgfpathmoveto{\pgfqpoint{7.775860in}{2.272632in}}%
\pgfpathlineto{\pgfqpoint{8.131822in}{2.613951in}}%
\pgfpathlineto{\pgfqpoint{8.097511in}{2.130589in}}%
\pgfpathlineto{\pgfqpoint{7.775860in}{2.272632in}}%
\pgfpathclose%
\pgfusepath{fill}%
\end{pgfscope}%
\begin{pgfscope}%
\pgfpathrectangle{\pgfqpoint{6.818937in}{0.147348in}}{\pgfqpoint{2.735294in}{2.735294in}}%
\pgfusepath{clip}%
\pgfsetbuttcap%
\pgfsetroundjoin%
\definecolor{currentfill}{rgb}{0.052493,0.201505,0.304798}%
\pgfsetfillcolor{currentfill}%
\pgfsetlinewidth{0.000000pt}%
\definecolor{currentstroke}{rgb}{0.000000,0.000000,0.000000}%
\pgfsetstrokecolor{currentstroke}%
\pgfsetdash{}{0pt}%
\pgfpathmoveto{\pgfqpoint{9.120088in}{1.620392in}}%
\pgfpathlineto{\pgfqpoint{9.124324in}{1.217806in}}%
\pgfpathlineto{\pgfqpoint{8.958969in}{1.590078in}}%
\pgfpathlineto{\pgfqpoint{9.120088in}{1.620392in}}%
\pgfpathclose%
\pgfusepath{fill}%
\end{pgfscope}%
\begin{pgfscope}%
\pgfpathrectangle{\pgfqpoint{6.818937in}{0.147348in}}{\pgfqpoint{2.735294in}{2.735294in}}%
\pgfusepath{clip}%
\pgfsetbuttcap%
\pgfsetroundjoin%
\definecolor{currentfill}{rgb}{0.052493,0.201505,0.304798}%
\pgfsetfillcolor{currentfill}%
\pgfsetlinewidth{0.000000pt}%
\definecolor{currentstroke}{rgb}{0.000000,0.000000,0.000000}%
\pgfsetstrokecolor{currentstroke}%
\pgfsetdash{}{0pt}%
\pgfpathmoveto{\pgfqpoint{7.488127in}{1.590078in}}%
\pgfpathlineto{\pgfqpoint{7.322771in}{1.217806in}}%
\pgfpathlineto{\pgfqpoint{7.327008in}{1.620392in}}%
\pgfpathlineto{\pgfqpoint{7.488127in}{1.590078in}}%
\pgfpathclose%
\pgfusepath{fill}%
\end{pgfscope}%
\begin{pgfscope}%
\pgfpathrectangle{\pgfqpoint{6.818937in}{0.147348in}}{\pgfqpoint{2.735294in}{2.735294in}}%
\pgfusepath{clip}%
\pgfsetbuttcap%
\pgfsetroundjoin%
\definecolor{currentfill}{rgb}{0.042579,0.163449,0.247234}%
\pgfsetfillcolor{currentfill}%
\pgfsetlinewidth{0.000000pt}%
\definecolor{currentstroke}{rgb}{0.000000,0.000000,0.000000}%
\pgfsetstrokecolor{currentstroke}%
\pgfsetdash{}{0pt}%
\pgfpathmoveto{\pgfqpoint{8.860669in}{1.018945in}}%
\pgfpathlineto{\pgfqpoint{8.672401in}{0.966453in}}%
\pgfpathlineto{\pgfqpoint{8.805232in}{1.099921in}}%
\pgfpathlineto{\pgfqpoint{8.860669in}{1.018945in}}%
\pgfpathclose%
\pgfusepath{fill}%
\end{pgfscope}%
\begin{pgfscope}%
\pgfpathrectangle{\pgfqpoint{6.818937in}{0.147348in}}{\pgfqpoint{2.735294in}{2.735294in}}%
\pgfusepath{clip}%
\pgfsetbuttcap%
\pgfsetroundjoin%
\definecolor{currentfill}{rgb}{0.042579,0.163449,0.247234}%
\pgfsetfillcolor{currentfill}%
\pgfsetlinewidth{0.000000pt}%
\definecolor{currentstroke}{rgb}{0.000000,0.000000,0.000000}%
\pgfsetstrokecolor{currentstroke}%
\pgfsetdash{}{0pt}%
\pgfpathmoveto{\pgfqpoint{7.641864in}{1.099921in}}%
\pgfpathlineto{\pgfqpoint{7.774694in}{0.966453in}}%
\pgfpathlineto{\pgfqpoint{7.586426in}{1.018945in}}%
\pgfpathlineto{\pgfqpoint{7.641864in}{1.099921in}}%
\pgfpathclose%
\pgfusepath{fill}%
\end{pgfscope}%
\begin{pgfscope}%
\pgfpathrectangle{\pgfqpoint{6.818937in}{0.147348in}}{\pgfqpoint{2.735294in}{2.735294in}}%
\pgfusepath{clip}%
\pgfsetbuttcap%
\pgfsetroundjoin%
\definecolor{currentfill}{rgb}{0.067179,0.257880,0.390071}%
\pgfsetfillcolor{currentfill}%
\pgfsetlinewidth{0.000000pt}%
\definecolor{currentstroke}{rgb}{0.000000,0.000000,0.000000}%
\pgfsetstrokecolor{currentstroke}%
\pgfsetdash{}{0pt}%
\pgfpathmoveto{\pgfqpoint{8.958969in}{1.590078in}}%
\pgfpathlineto{\pgfqpoint{9.017173in}{1.079776in}}%
\pgfpathlineto{\pgfqpoint{8.982643in}{1.157190in}}%
\pgfpathlineto{\pgfqpoint{8.958969in}{1.590078in}}%
\pgfpathclose%
\pgfusepath{fill}%
\end{pgfscope}%
\begin{pgfscope}%
\pgfpathrectangle{\pgfqpoint{6.818937in}{0.147348in}}{\pgfqpoint{2.735294in}{2.735294in}}%
\pgfusepath{clip}%
\pgfsetbuttcap%
\pgfsetroundjoin%
\definecolor{currentfill}{rgb}{0.067179,0.257880,0.390071}%
\pgfsetfillcolor{currentfill}%
\pgfsetlinewidth{0.000000pt}%
\definecolor{currentstroke}{rgb}{0.000000,0.000000,0.000000}%
\pgfsetstrokecolor{currentstroke}%
\pgfsetdash{}{0pt}%
\pgfpathmoveto{\pgfqpoint{7.464453in}{1.157190in}}%
\pgfpathlineto{\pgfqpoint{7.429923in}{1.079776in}}%
\pgfpathlineto{\pgfqpoint{7.488127in}{1.590078in}}%
\pgfpathlineto{\pgfqpoint{7.464453in}{1.157190in}}%
\pgfpathclose%
\pgfusepath{fill}%
\end{pgfscope}%
\begin{pgfscope}%
\pgfpathrectangle{\pgfqpoint{6.818937in}{0.147348in}}{\pgfqpoint{2.735294in}{2.735294in}}%
\pgfusepath{clip}%
\pgfsetbuttcap%
\pgfsetroundjoin%
\definecolor{currentfill}{rgb}{0.047247,0.181368,0.274339}%
\pgfsetfillcolor{currentfill}%
\pgfsetlinewidth{0.000000pt}%
\definecolor{currentstroke}{rgb}{0.000000,0.000000,0.000000}%
\pgfsetstrokecolor{currentstroke}%
\pgfsetdash{}{0pt}%
\pgfpathmoveto{\pgfqpoint{9.124324in}{1.217806in}}%
\pgfpathlineto{\pgfqpoint{9.017173in}{1.079776in}}%
\pgfpathlineto{\pgfqpoint{8.958969in}{1.590078in}}%
\pgfpathlineto{\pgfqpoint{9.124324in}{1.217806in}}%
\pgfpathclose%
\pgfusepath{fill}%
\end{pgfscope}%
\begin{pgfscope}%
\pgfpathrectangle{\pgfqpoint{6.818937in}{0.147348in}}{\pgfqpoint{2.735294in}{2.735294in}}%
\pgfusepath{clip}%
\pgfsetbuttcap%
\pgfsetroundjoin%
\definecolor{currentfill}{rgb}{0.047247,0.181368,0.274339}%
\pgfsetfillcolor{currentfill}%
\pgfsetlinewidth{0.000000pt}%
\definecolor{currentstroke}{rgb}{0.000000,0.000000,0.000000}%
\pgfsetstrokecolor{currentstroke}%
\pgfsetdash{}{0pt}%
\pgfpathmoveto{\pgfqpoint{7.488127in}{1.590078in}}%
\pgfpathlineto{\pgfqpoint{7.429923in}{1.079776in}}%
\pgfpathlineto{\pgfqpoint{7.322771in}{1.217806in}}%
\pgfpathlineto{\pgfqpoint{7.488127in}{1.590078in}}%
\pgfpathclose%
\pgfusepath{fill}%
\end{pgfscope}%
\begin{pgfscope}%
\pgfpathrectangle{\pgfqpoint{6.818937in}{0.147348in}}{\pgfqpoint{2.735294in}{2.735294in}}%
\pgfusepath{clip}%
\pgfsetbuttcap%
\pgfsetroundjoin%
\definecolor{currentfill}{rgb}{0.081954,0.314596,0.475860}%
\pgfsetfillcolor{currentfill}%
\pgfsetlinewidth{0.000000pt}%
\definecolor{currentstroke}{rgb}{0.000000,0.000000,0.000000}%
\pgfsetstrokecolor{currentstroke}%
\pgfsetdash{}{0pt}%
\pgfpathmoveto{\pgfqpoint{8.131822in}{2.613951in}}%
\pgfpathlineto{\pgfqpoint{8.315273in}{2.613951in}}%
\pgfpathlineto{\pgfqpoint{8.349584in}{2.130589in}}%
\pgfpathlineto{\pgfqpoint{8.131822in}{2.613951in}}%
\pgfpathclose%
\pgfusepath{fill}%
\end{pgfscope}%
\begin{pgfscope}%
\pgfpathrectangle{\pgfqpoint{6.818937in}{0.147348in}}{\pgfqpoint{2.735294in}{2.735294in}}%
\pgfusepath{clip}%
\pgfsetbuttcap%
\pgfsetroundjoin%
\definecolor{currentfill}{rgb}{0.044978,0.172658,0.261163}%
\pgfsetfillcolor{currentfill}%
\pgfsetlinewidth{0.000000pt}%
\definecolor{currentstroke}{rgb}{0.000000,0.000000,0.000000}%
\pgfsetstrokecolor{currentstroke}%
\pgfsetdash{}{0pt}%
\pgfpathmoveto{\pgfqpoint{8.860669in}{1.018945in}}%
\pgfpathlineto{\pgfqpoint{8.982643in}{1.157190in}}%
\pgfpathlineto{\pgfqpoint{9.017173in}{1.079776in}}%
\pgfpathlineto{\pgfqpoint{8.860669in}{1.018945in}}%
\pgfpathclose%
\pgfusepath{fill}%
\end{pgfscope}%
\begin{pgfscope}%
\pgfpathrectangle{\pgfqpoint{6.818937in}{0.147348in}}{\pgfqpoint{2.735294in}{2.735294in}}%
\pgfusepath{clip}%
\pgfsetbuttcap%
\pgfsetroundjoin%
\definecolor{currentfill}{rgb}{0.044978,0.172658,0.261163}%
\pgfsetfillcolor{currentfill}%
\pgfsetlinewidth{0.000000pt}%
\definecolor{currentstroke}{rgb}{0.000000,0.000000,0.000000}%
\pgfsetstrokecolor{currentstroke}%
\pgfsetdash{}{0pt}%
\pgfpathmoveto{\pgfqpoint{7.429923in}{1.079776in}}%
\pgfpathlineto{\pgfqpoint{7.464453in}{1.157190in}}%
\pgfpathlineto{\pgfqpoint{7.586426in}{1.018945in}}%
\pgfpathlineto{\pgfqpoint{7.429923in}{1.079776in}}%
\pgfpathclose%
\pgfusepath{fill}%
\end{pgfscope}%
\begin{pgfscope}%
\pgfpathrectangle{\pgfqpoint{6.818937in}{0.147348in}}{\pgfqpoint{2.735294in}{2.735294in}}%
\pgfusepath{clip}%
\pgfsetbuttcap%
\pgfsetroundjoin%
\definecolor{currentfill}{rgb}{0.070254,0.269685,0.407928}%
\pgfsetfillcolor{currentfill}%
\pgfsetlinewidth{0.000000pt}%
\definecolor{currentstroke}{rgb}{0.000000,0.000000,0.000000}%
\pgfsetstrokecolor{currentstroke}%
\pgfsetdash{}{0pt}%
\pgfpathmoveto{\pgfqpoint{9.120088in}{1.620392in}}%
\pgfpathlineto{\pgfqpoint{9.143495in}{1.142562in}}%
\pgfpathlineto{\pgfqpoint{9.124324in}{1.217806in}}%
\pgfpathlineto{\pgfqpoint{9.120088in}{1.620392in}}%
\pgfpathclose%
\pgfusepath{fill}%
\end{pgfscope}%
\begin{pgfscope}%
\pgfpathrectangle{\pgfqpoint{6.818937in}{0.147348in}}{\pgfqpoint{2.735294in}{2.735294in}}%
\pgfusepath{clip}%
\pgfsetbuttcap%
\pgfsetroundjoin%
\definecolor{currentfill}{rgb}{0.070254,0.269685,0.407928}%
\pgfsetfillcolor{currentfill}%
\pgfsetlinewidth{0.000000pt}%
\definecolor{currentstroke}{rgb}{0.000000,0.000000,0.000000}%
\pgfsetstrokecolor{currentstroke}%
\pgfsetdash{}{0pt}%
\pgfpathmoveto{\pgfqpoint{7.322771in}{1.217806in}}%
\pgfpathlineto{\pgfqpoint{7.303601in}{1.142562in}}%
\pgfpathlineto{\pgfqpoint{7.327008in}{1.620392in}}%
\pgfpathlineto{\pgfqpoint{7.322771in}{1.217806in}}%
\pgfpathclose%
\pgfusepath{fill}%
\end{pgfscope}%
\begin{pgfscope}%
\pgfpathrectangle{\pgfqpoint{6.818937in}{0.147348in}}{\pgfqpoint{2.735294in}{2.735294in}}%
\pgfusepath{clip}%
\pgfsetbuttcap%
\pgfsetroundjoin%
\definecolor{currentfill}{rgb}{0.048960,0.187944,0.284285}%
\pgfsetfillcolor{currentfill}%
\pgfsetlinewidth{0.000000pt}%
\definecolor{currentstroke}{rgb}{0.000000,0.000000,0.000000}%
\pgfsetstrokecolor{currentstroke}%
\pgfsetdash{}{0pt}%
\pgfpathmoveto{\pgfqpoint{9.235220in}{1.276682in}}%
\pgfpathlineto{\pgfqpoint{9.143495in}{1.142562in}}%
\pgfpathlineto{\pgfqpoint{9.120088in}{1.620392in}}%
\pgfpathlineto{\pgfqpoint{9.235220in}{1.276682in}}%
\pgfpathclose%
\pgfusepath{fill}%
\end{pgfscope}%
\begin{pgfscope}%
\pgfpathrectangle{\pgfqpoint{6.818937in}{0.147348in}}{\pgfqpoint{2.735294in}{2.735294in}}%
\pgfusepath{clip}%
\pgfsetbuttcap%
\pgfsetroundjoin%
\definecolor{currentfill}{rgb}{0.048960,0.187944,0.284285}%
\pgfsetfillcolor{currentfill}%
\pgfsetlinewidth{0.000000pt}%
\definecolor{currentstroke}{rgb}{0.000000,0.000000,0.000000}%
\pgfsetstrokecolor{currentstroke}%
\pgfsetdash{}{0pt}%
\pgfpathmoveto{\pgfqpoint{7.211876in}{1.276682in}}%
\pgfpathlineto{\pgfqpoint{7.327008in}{1.620392in}}%
\pgfpathlineto{\pgfqpoint{7.303601in}{1.142562in}}%
\pgfpathlineto{\pgfqpoint{7.211876in}{1.276682in}}%
\pgfpathclose%
\pgfusepath{fill}%
\end{pgfscope}%
\begin{pgfscope}%
\pgfpathrectangle{\pgfqpoint{6.818937in}{0.147348in}}{\pgfqpoint{2.735294in}{2.735294in}}%
\pgfusepath{clip}%
\pgfsetbuttcap%
\pgfsetroundjoin%
\definecolor{currentfill}{rgb}{0.047548,0.182523,0.276086}%
\pgfsetfillcolor{currentfill}%
\pgfsetlinewidth{0.000000pt}%
\definecolor{currentstroke}{rgb}{0.000000,0.000000,0.000000}%
\pgfsetstrokecolor{currentstroke}%
\pgfsetdash{}{0pt}%
\pgfpathmoveto{\pgfqpoint{9.017173in}{1.079776in}}%
\pgfpathlineto{\pgfqpoint{9.124324in}{1.217806in}}%
\pgfpathlineto{\pgfqpoint{9.143495in}{1.142562in}}%
\pgfpathlineto{\pgfqpoint{9.017173in}{1.079776in}}%
\pgfpathclose%
\pgfusepath{fill}%
\end{pgfscope}%
\begin{pgfscope}%
\pgfpathrectangle{\pgfqpoint{6.818937in}{0.147348in}}{\pgfqpoint{2.735294in}{2.735294in}}%
\pgfusepath{clip}%
\pgfsetbuttcap%
\pgfsetroundjoin%
\definecolor{currentfill}{rgb}{0.047548,0.182523,0.276086}%
\pgfsetfillcolor{currentfill}%
\pgfsetlinewidth{0.000000pt}%
\definecolor{currentstroke}{rgb}{0.000000,0.000000,0.000000}%
\pgfsetstrokecolor{currentstroke}%
\pgfsetdash{}{0pt}%
\pgfpathmoveto{\pgfqpoint{7.303601in}{1.142562in}}%
\pgfpathlineto{\pgfqpoint{7.322771in}{1.217806in}}%
\pgfpathlineto{\pgfqpoint{7.429923in}{1.079776in}}%
\pgfpathlineto{\pgfqpoint{7.303601in}{1.142562in}}%
\pgfpathclose%
\pgfusepath{fill}%
\end{pgfscope}%
\begin{pgfscope}%
\pgfpathrectangle{\pgfqpoint{6.818937in}{0.147348in}}{\pgfqpoint{2.735294in}{2.735294in}}%
\pgfusepath{clip}%
\pgfsetbuttcap%
\pgfsetroundjoin%
\definecolor{currentfill}{rgb}{0.090605,0.347808,0.526096}%
\pgfsetfillcolor{currentfill}%
\pgfsetlinewidth{0.000000pt}%
\definecolor{currentstroke}{rgb}{0.000000,0.000000,0.000000}%
\pgfsetstrokecolor{currentstroke}%
\pgfsetdash{}{0pt}%
\pgfpathmoveto{\pgfqpoint{8.315273in}{2.613951in}}%
\pgfpathlineto{\pgfqpoint{8.131822in}{2.613951in}}%
\pgfpathlineto{\pgfqpoint{8.223548in}{2.686670in}}%
\pgfpathlineto{\pgfqpoint{8.315273in}{2.613951in}}%
\pgfpathclose%
\pgfusepath{fill}%
\end{pgfscope}%
\begin{pgfscope}%
\pgfpathrectangle{\pgfqpoint{6.818937in}{0.147348in}}{\pgfqpoint{2.735294in}{2.735294in}}%
\pgfusepath{clip}%
\pgfsetbuttcap%
\pgfsetroundjoin%
\definecolor{currentfill}{rgb}{0.050070,0.192203,0.290728}%
\pgfsetfillcolor{currentfill}%
\pgfsetlinewidth{0.000000pt}%
\definecolor{currentstroke}{rgb}{0.000000,0.000000,0.000000}%
\pgfsetstrokecolor{currentstroke}%
\pgfsetdash{}{0pt}%
\pgfpathmoveto{\pgfqpoint{9.143495in}{1.142562in}}%
\pgfpathlineto{\pgfqpoint{9.235220in}{1.276682in}}%
\pgfpathlineto{\pgfqpoint{9.243948in}{1.203196in}}%
\pgfpathlineto{\pgfqpoint{9.143495in}{1.142562in}}%
\pgfpathclose%
\pgfusepath{fill}%
\end{pgfscope}%
\begin{pgfscope}%
\pgfpathrectangle{\pgfqpoint{6.818937in}{0.147348in}}{\pgfqpoint{2.735294in}{2.735294in}}%
\pgfusepath{clip}%
\pgfsetbuttcap%
\pgfsetroundjoin%
\definecolor{currentfill}{rgb}{0.050070,0.192203,0.290728}%
\pgfsetfillcolor{currentfill}%
\pgfsetlinewidth{0.000000pt}%
\definecolor{currentstroke}{rgb}{0.000000,0.000000,0.000000}%
\pgfsetstrokecolor{currentstroke}%
\pgfsetdash{}{0pt}%
\pgfpathmoveto{\pgfqpoint{7.211876in}{1.276682in}}%
\pgfpathlineto{\pgfqpoint{7.303601in}{1.142562in}}%
\pgfpathlineto{\pgfqpoint{7.203147in}{1.203196in}}%
\pgfpathlineto{\pgfqpoint{7.211876in}{1.276682in}}%
\pgfpathclose%
\pgfusepath{fill}%
\end{pgfscope}%
\begin{pgfscope}%
\pgfsetbuttcap%
\pgfsetmiterjoin%
\definecolor{currentfill}{rgb}{1.000000,1.000000,1.000000}%
\pgfsetfillcolor{currentfill}%
\pgfsetlinewidth{0.000000pt}%
\definecolor{currentstroke}{rgb}{0.000000,0.000000,0.000000}%
\pgfsetstrokecolor{currentstroke}%
\pgfsetstrokeopacity{0.000000}%
\pgfsetdash{}{0pt}%
\pgfpathmoveto{\pgfqpoint{0.254231in}{0.147348in}}%
\pgfpathlineto{\pgfqpoint{2.989526in}{0.147348in}}%
\pgfpathlineto{\pgfqpoint{2.989526in}{2.882642in}}%
\pgfpathlineto{\pgfqpoint{0.254231in}{2.882642in}}%
\pgfpathlineto{\pgfqpoint{0.254231in}{0.147348in}}%
\pgfpathclose%
\pgfusepath{fill}%
\end{pgfscope}%
\begin{pgfscope}%
\pgfsetbuttcap%
\pgfsetmiterjoin%
\definecolor{currentfill}{rgb}{0.950000,0.950000,0.950000}%
\pgfsetfillcolor{currentfill}%
\pgfsetfillopacity{0.500000}%
\pgfsetlinewidth{1.003750pt}%
\definecolor{currentstroke}{rgb}{0.950000,0.950000,0.950000}%
\pgfsetstrokecolor{currentstroke}%
\pgfsetstrokeopacity{0.500000}%
\pgfsetdash{}{0pt}%
\pgfpathmoveto{\pgfqpoint{1.658842in}{1.930798in}}%
\pgfpathlineto{\pgfqpoint{2.865378in}{1.099507in}}%
\pgfpathlineto{\pgfqpoint{2.944036in}{2.033906in}}%
\pgfpathlineto{\pgfqpoint{1.658842in}{2.862146in}}%
\pgfusepath{stroke,fill}%
\end{pgfscope}%
\begin{pgfscope}%
\pgfsetbuttcap%
\pgfsetmiterjoin%
\definecolor{currentfill}{rgb}{0.900000,0.900000,0.900000}%
\pgfsetfillcolor{currentfill}%
\pgfsetfillopacity{0.500000}%
\pgfsetlinewidth{1.003750pt}%
\definecolor{currentstroke}{rgb}{0.900000,0.900000,0.900000}%
\pgfsetstrokecolor{currentstroke}%
\pgfsetstrokeopacity{0.500000}%
\pgfsetdash{}{0pt}%
\pgfpathmoveto{\pgfqpoint{1.658842in}{1.930798in}}%
\pgfpathlineto{\pgfqpoint{0.452305in}{1.099507in}}%
\pgfpathlineto{\pgfqpoint{0.373648in}{2.033906in}}%
\pgfpathlineto{\pgfqpoint{1.658842in}{2.862146in}}%
\pgfusepath{stroke,fill}%
\end{pgfscope}%
\begin{pgfscope}%
\pgfsetbuttcap%
\pgfsetmiterjoin%
\definecolor{currentfill}{rgb}{0.925000,0.925000,0.925000}%
\pgfsetfillcolor{currentfill}%
\pgfsetfillopacity{0.500000}%
\pgfsetlinewidth{1.003750pt}%
\definecolor{currentstroke}{rgb}{0.925000,0.925000,0.925000}%
\pgfsetstrokecolor{currentstroke}%
\pgfsetstrokeopacity{0.500000}%
\pgfsetdash{}{0pt}%
\pgfpathmoveto{\pgfqpoint{1.658842in}{1.930798in}}%
\pgfpathlineto{\pgfqpoint{0.452305in}{1.099507in}}%
\pgfpathlineto{\pgfqpoint{1.658842in}{0.166408in}}%
\pgfpathlineto{\pgfqpoint{2.865378in}{1.099507in}}%
\pgfusepath{stroke,fill}%
\end{pgfscope}%
\begin{pgfscope}%
\pgfsetbuttcap%
\pgfsetroundjoin%
\pgfsetlinewidth{0.803000pt}%
\definecolor{currentstroke}{rgb}{0.690196,0.690196,0.690196}%
\pgfsetstrokecolor{currentstroke}%
\pgfsetdash{}{0pt}%
\pgfpathmoveto{\pgfqpoint{2.792763in}{1.043348in}}%
\pgfpathlineto{\pgfqpoint{1.585998in}{1.880609in}}%
\pgfpathlineto{\pgfqpoint{1.581508in}{2.812308in}}%
\pgfusepath{stroke}%
\end{pgfscope}%
\begin{pgfscope}%
\pgfsetbuttcap%
\pgfsetroundjoin%
\pgfsetlinewidth{0.803000pt}%
\definecolor{currentstroke}{rgb}{0.690196,0.690196,0.690196}%
\pgfsetstrokecolor{currentstroke}%
\pgfsetdash{}{0pt}%
\pgfpathmoveto{\pgfqpoint{2.562953in}{0.865621in}}%
\pgfpathlineto{\pgfqpoint{1.355660in}{1.721909in}}%
\pgfpathlineto{\pgfqpoint{1.336753in}{2.654576in}}%
\pgfusepath{stroke}%
\end{pgfscope}%
\begin{pgfscope}%
\pgfsetbuttcap%
\pgfsetroundjoin%
\pgfsetlinewidth{0.803000pt}%
\definecolor{currentstroke}{rgb}{0.690196,0.690196,0.690196}%
\pgfsetstrokecolor{currentstroke}%
\pgfsetdash{}{0pt}%
\pgfpathmoveto{\pgfqpoint{2.327740in}{0.683714in}}%
\pgfpathlineto{\pgfqpoint{1.120209in}{1.559686in}}%
\pgfpathlineto{\pgfqpoint{1.086222in}{2.493122in}}%
\pgfusepath{stroke}%
\end{pgfscope}%
\begin{pgfscope}%
\pgfsetbuttcap%
\pgfsetroundjoin%
\pgfsetlinewidth{0.803000pt}%
\definecolor{currentstroke}{rgb}{0.690196,0.690196,0.690196}%
\pgfsetstrokecolor{currentstroke}%
\pgfsetdash{}{0pt}%
\pgfpathmoveto{\pgfqpoint{2.086930in}{0.497478in}}%
\pgfpathlineto{\pgfqpoint{0.879473in}{1.393821in}}%
\pgfpathlineto{\pgfqpoint{0.829708in}{2.327813in}}%
\pgfusepath{stroke}%
\end{pgfscope}%
\begin{pgfscope}%
\pgfsetbuttcap%
\pgfsetroundjoin%
\pgfsetlinewidth{0.803000pt}%
\definecolor{currentstroke}{rgb}{0.690196,0.690196,0.690196}%
\pgfsetstrokecolor{currentstroke}%
\pgfsetdash{}{0pt}%
\pgfpathmoveto{\pgfqpoint{1.840321in}{0.306758in}}%
\pgfpathlineto{\pgfqpoint{0.633271in}{1.224191in}}%
\pgfpathlineto{\pgfqpoint{0.566994in}{2.158507in}}%
\pgfusepath{stroke}%
\end{pgfscope}%
\begin{pgfscope}%
\pgfsetbuttcap%
\pgfsetroundjoin%
\pgfsetlinewidth{0.803000pt}%
\definecolor{currentstroke}{rgb}{0.690196,0.690196,0.690196}%
\pgfsetstrokecolor{currentstroke}%
\pgfsetdash{}{0pt}%
\pgfpathmoveto{\pgfqpoint{1.736176in}{2.812308in}}%
\pgfpathlineto{\pgfqpoint{1.731686in}{1.880609in}}%
\pgfpathlineto{\pgfqpoint{0.524921in}{1.043348in}}%
\pgfusepath{stroke}%
\end{pgfscope}%
\begin{pgfscope}%
\pgfsetbuttcap%
\pgfsetroundjoin%
\pgfsetlinewidth{0.803000pt}%
\definecolor{currentstroke}{rgb}{0.690196,0.690196,0.690196}%
\pgfsetstrokecolor{currentstroke}%
\pgfsetdash{}{0pt}%
\pgfpathmoveto{\pgfqpoint{1.980931in}{2.654576in}}%
\pgfpathlineto{\pgfqpoint{1.962024in}{1.721909in}}%
\pgfpathlineto{\pgfqpoint{0.754731in}{0.865621in}}%
\pgfusepath{stroke}%
\end{pgfscope}%
\begin{pgfscope}%
\pgfsetbuttcap%
\pgfsetroundjoin%
\pgfsetlinewidth{0.803000pt}%
\definecolor{currentstroke}{rgb}{0.690196,0.690196,0.690196}%
\pgfsetstrokecolor{currentstroke}%
\pgfsetdash{}{0pt}%
\pgfpathmoveto{\pgfqpoint{2.231462in}{2.493122in}}%
\pgfpathlineto{\pgfqpoint{2.197475in}{1.559686in}}%
\pgfpathlineto{\pgfqpoint{0.989944in}{0.683714in}}%
\pgfusepath{stroke}%
\end{pgfscope}%
\begin{pgfscope}%
\pgfsetbuttcap%
\pgfsetroundjoin%
\pgfsetlinewidth{0.803000pt}%
\definecolor{currentstroke}{rgb}{0.690196,0.690196,0.690196}%
\pgfsetstrokecolor{currentstroke}%
\pgfsetdash{}{0pt}%
\pgfpathmoveto{\pgfqpoint{2.487976in}{2.327813in}}%
\pgfpathlineto{\pgfqpoint{2.438211in}{1.393821in}}%
\pgfpathlineto{\pgfqpoint{1.230754in}{0.497478in}}%
\pgfusepath{stroke}%
\end{pgfscope}%
\begin{pgfscope}%
\pgfsetbuttcap%
\pgfsetroundjoin%
\pgfsetlinewidth{0.803000pt}%
\definecolor{currentstroke}{rgb}{0.690196,0.690196,0.690196}%
\pgfsetstrokecolor{currentstroke}%
\pgfsetdash{}{0pt}%
\pgfpathmoveto{\pgfqpoint{2.750690in}{2.158507in}}%
\pgfpathlineto{\pgfqpoint{2.684413in}{1.224191in}}%
\pgfpathlineto{\pgfqpoint{1.477363in}{0.306758in}}%
\pgfusepath{stroke}%
\end{pgfscope}%
\begin{pgfscope}%
\pgfsetbuttcap%
\pgfsetroundjoin%
\pgfsetlinewidth{0.803000pt}%
\definecolor{currentstroke}{rgb}{0.690196,0.690196,0.690196}%
\pgfsetstrokecolor{currentstroke}%
\pgfsetdash{}{0pt}%
\pgfpathmoveto{\pgfqpoint{0.447588in}{1.155548in}}%
\pgfpathlineto{\pgfqpoint{1.658842in}{1.986842in}}%
\pgfpathlineto{\pgfqpoint{2.870096in}{1.155548in}}%
\pgfusepath{stroke}%
\end{pgfscope}%
\begin{pgfscope}%
\pgfsetbuttcap%
\pgfsetroundjoin%
\pgfsetlinewidth{0.803000pt}%
\definecolor{currentstroke}{rgb}{0.690196,0.690196,0.690196}%
\pgfsetstrokecolor{currentstroke}%
\pgfsetdash{}{0pt}%
\pgfpathmoveto{\pgfqpoint{0.432645in}{1.333067in}}%
\pgfpathlineto{\pgfqpoint{1.658842in}{2.164215in}}%
\pgfpathlineto{\pgfqpoint{2.885039in}{1.333067in}}%
\pgfusepath{stroke}%
\end{pgfscope}%
\begin{pgfscope}%
\pgfsetbuttcap%
\pgfsetroundjoin%
\pgfsetlinewidth{0.803000pt}%
\definecolor{currentstroke}{rgb}{0.690196,0.690196,0.690196}%
\pgfsetstrokecolor{currentstroke}%
\pgfsetdash{}{0pt}%
\pgfpathmoveto{\pgfqpoint{0.417328in}{1.515021in}}%
\pgfpathlineto{\pgfqpoint{1.658842in}{2.345771in}}%
\pgfpathlineto{\pgfqpoint{2.900356in}{1.515021in}}%
\pgfusepath{stroke}%
\end{pgfscope}%
\begin{pgfscope}%
\pgfsetbuttcap%
\pgfsetroundjoin%
\pgfsetlinewidth{0.803000pt}%
\definecolor{currentstroke}{rgb}{0.690196,0.690196,0.690196}%
\pgfsetstrokecolor{currentstroke}%
\pgfsetdash{}{0pt}%
\pgfpathmoveto{\pgfqpoint{0.401624in}{1.701578in}}%
\pgfpathlineto{\pgfqpoint{1.658842in}{2.531660in}}%
\pgfpathlineto{\pgfqpoint{2.916060in}{1.701578in}}%
\pgfusepath{stroke}%
\end{pgfscope}%
\begin{pgfscope}%
\pgfsetbuttcap%
\pgfsetroundjoin%
\pgfsetlinewidth{0.803000pt}%
\definecolor{currentstroke}{rgb}{0.690196,0.690196,0.690196}%
\pgfsetstrokecolor{currentstroke}%
\pgfsetdash{}{0pt}%
\pgfpathmoveto{\pgfqpoint{0.385517in}{1.892915in}}%
\pgfpathlineto{\pgfqpoint{1.658842in}{2.722038in}}%
\pgfpathlineto{\pgfqpoint{2.932167in}{1.892915in}}%
\pgfusepath{stroke}%
\end{pgfscope}%
\begin{pgfscope}%
\pgfsetrectcap%
\pgfsetroundjoin%
\pgfsetlinewidth{0.803000pt}%
\definecolor{currentstroke}{rgb}{0.000000,0.000000,0.000000}%
\pgfsetstrokecolor{currentstroke}%
\pgfsetdash{}{0pt}%
\pgfpathmoveto{\pgfqpoint{2.865378in}{1.099507in}}%
\pgfpathlineto{\pgfqpoint{1.658842in}{0.166408in}}%
\pgfusepath{stroke}%
\end{pgfscope}%
\begin{pgfscope}%
\pgfsetrectcap%
\pgfsetroundjoin%
\pgfsetlinewidth{0.803000pt}%
\definecolor{currentstroke}{rgb}{0.000000,0.000000,0.000000}%
\pgfsetstrokecolor{currentstroke}%
\pgfsetdash{}{0pt}%
\pgfpathmoveto{\pgfqpoint{2.782554in}{1.050431in}}%
\pgfpathlineto{\pgfqpoint{2.813208in}{1.029163in}}%
\pgfusepath{stroke}%
\end{pgfscope}%
\begin{pgfscope}%
\pgfsetrectcap%
\pgfsetroundjoin%
\pgfsetlinewidth{0.803000pt}%
\definecolor{currentstroke}{rgb}{0.000000,0.000000,0.000000}%
\pgfsetstrokecolor{currentstroke}%
\pgfsetdash{}{0pt}%
\pgfpathmoveto{\pgfqpoint{2.552734in}{0.872869in}}%
\pgfpathlineto{\pgfqpoint{2.583421in}{0.851104in}}%
\pgfusepath{stroke}%
\end{pgfscope}%
\begin{pgfscope}%
\pgfsetrectcap%
\pgfsetroundjoin%
\pgfsetlinewidth{0.803000pt}%
\definecolor{currentstroke}{rgb}{0.000000,0.000000,0.000000}%
\pgfsetstrokecolor{currentstroke}%
\pgfsetdash{}{0pt}%
\pgfpathmoveto{\pgfqpoint{2.317512in}{0.691133in}}%
\pgfpathlineto{\pgfqpoint{2.348225in}{0.668853in}}%
\pgfusepath{stroke}%
\end{pgfscope}%
\begin{pgfscope}%
\pgfsetrectcap%
\pgfsetroundjoin%
\pgfsetlinewidth{0.803000pt}%
\definecolor{currentstroke}{rgb}{0.000000,0.000000,0.000000}%
\pgfsetstrokecolor{currentstroke}%
\pgfsetdash{}{0pt}%
\pgfpathmoveto{\pgfqpoint{2.076696in}{0.505076in}}%
\pgfpathlineto{\pgfqpoint{2.107428in}{0.482262in}}%
\pgfusepath{stroke}%
\end{pgfscope}%
\begin{pgfscope}%
\pgfsetrectcap%
\pgfsetroundjoin%
\pgfsetlinewidth{0.803000pt}%
\definecolor{currentstroke}{rgb}{0.000000,0.000000,0.000000}%
\pgfsetstrokecolor{currentstroke}%
\pgfsetdash{}{0pt}%
\pgfpathmoveto{\pgfqpoint{1.830084in}{0.314540in}}%
\pgfpathlineto{\pgfqpoint{1.860826in}{0.291173in}}%
\pgfusepath{stroke}%
\end{pgfscope}%
\begin{pgfscope}%
\definecolor{textcolor}{rgb}{0.000000,0.000000,0.000000}%
\pgfsetstrokecolor{textcolor}%
\pgfsetfillcolor{textcolor}%
\pgftext[x=2.557884in,y=0.241958in,,]{\color{textcolor}{\rmfamily\fontsize{14.000000}{16.800000}\selectfont\catcode`\^=\active\def^{\ifmmode\sp\else\^{}\fi}\catcode`\%=\active\def%{\%}f1}}%
\end{pgfscope}%
\begin{pgfscope}%
\pgfsetrectcap%
\pgfsetroundjoin%
\pgfsetlinewidth{0.803000pt}%
\definecolor{currentstroke}{rgb}{0.000000,0.000000,0.000000}%
\pgfsetstrokecolor{currentstroke}%
\pgfsetdash{}{0pt}%
\pgfpathmoveto{\pgfqpoint{0.452305in}{1.099507in}}%
\pgfpathlineto{\pgfqpoint{1.658842in}{0.166408in}}%
\pgfusepath{stroke}%
\end{pgfscope}%
\begin{pgfscope}%
\pgfsetrectcap%
\pgfsetroundjoin%
\pgfsetlinewidth{0.803000pt}%
\definecolor{currentstroke}{rgb}{0.000000,0.000000,0.000000}%
\pgfsetstrokecolor{currentstroke}%
\pgfsetdash{}{0pt}%
\pgfpathmoveto{\pgfqpoint{0.535130in}{1.050431in}}%
\pgfpathlineto{\pgfqpoint{0.504476in}{1.029163in}}%
\pgfusepath{stroke}%
\end{pgfscope}%
\begin{pgfscope}%
\pgfsetrectcap%
\pgfsetroundjoin%
\pgfsetlinewidth{0.803000pt}%
\definecolor{currentstroke}{rgb}{0.000000,0.000000,0.000000}%
\pgfsetstrokecolor{currentstroke}%
\pgfsetdash{}{0pt}%
\pgfpathmoveto{\pgfqpoint{0.764950in}{0.872869in}}%
\pgfpathlineto{\pgfqpoint{0.734263in}{0.851104in}}%
\pgfusepath{stroke}%
\end{pgfscope}%
\begin{pgfscope}%
\pgfsetrectcap%
\pgfsetroundjoin%
\pgfsetlinewidth{0.803000pt}%
\definecolor{currentstroke}{rgb}{0.000000,0.000000,0.000000}%
\pgfsetstrokecolor{currentstroke}%
\pgfsetdash{}{0pt}%
\pgfpathmoveto{\pgfqpoint{1.000172in}{0.691133in}}%
\pgfpathlineto{\pgfqpoint{0.969459in}{0.668853in}}%
\pgfusepath{stroke}%
\end{pgfscope}%
\begin{pgfscope}%
\pgfsetrectcap%
\pgfsetroundjoin%
\pgfsetlinewidth{0.803000pt}%
\definecolor{currentstroke}{rgb}{0.000000,0.000000,0.000000}%
\pgfsetstrokecolor{currentstroke}%
\pgfsetdash{}{0pt}%
\pgfpathmoveto{\pgfqpoint{1.240988in}{0.505076in}}%
\pgfpathlineto{\pgfqpoint{1.210256in}{0.482262in}}%
\pgfusepath{stroke}%
\end{pgfscope}%
\begin{pgfscope}%
\pgfsetrectcap%
\pgfsetroundjoin%
\pgfsetlinewidth{0.803000pt}%
\definecolor{currentstroke}{rgb}{0.000000,0.000000,0.000000}%
\pgfsetstrokecolor{currentstroke}%
\pgfsetdash{}{0pt}%
\pgfpathmoveto{\pgfqpoint{1.487600in}{0.314540in}}%
\pgfpathlineto{\pgfqpoint{1.456858in}{0.291173in}}%
\pgfusepath{stroke}%
\end{pgfscope}%
\begin{pgfscope}%
\definecolor{textcolor}{rgb}{0.000000,0.000000,0.000000}%
\pgfsetstrokecolor{textcolor}%
\pgfsetfillcolor{textcolor}%
\pgftext[x=0.759800in,y=0.241958in,,]{\color{textcolor}{\rmfamily\fontsize{14.000000}{16.800000}\selectfont\catcode`\^=\active\def^{\ifmmode\sp\else\^{}\fi}\catcode`\%=\active\def%{\%}f2}}%
\end{pgfscope}%
\begin{pgfscope}%
\pgfsetrectcap%
\pgfsetroundjoin%
\pgfsetlinewidth{0.803000pt}%
\definecolor{currentstroke}{rgb}{0.000000,0.000000,0.000000}%
\pgfsetstrokecolor{currentstroke}%
\pgfsetdash{}{0pt}%
\pgfpathmoveto{\pgfqpoint{0.452305in}{1.099507in}}%
\pgfpathlineto{\pgfqpoint{0.373648in}{2.033906in}}%
\pgfusepath{stroke}%
\end{pgfscope}%
\begin{pgfscope}%
\pgfsetrectcap%
\pgfsetroundjoin%
\pgfsetlinewidth{0.803000pt}%
\definecolor{currentstroke}{rgb}{0.000000,0.000000,0.000000}%
\pgfsetstrokecolor{currentstroke}%
\pgfsetdash{}{0pt}%
\pgfpathmoveto{\pgfqpoint{0.457835in}{1.162580in}}%
\pgfpathlineto{\pgfqpoint{0.427066in}{1.141464in}}%
\pgfusepath{stroke}%
\end{pgfscope}%
\begin{pgfscope}%
\pgfsetrectcap%
\pgfsetroundjoin%
\pgfsetlinewidth{0.803000pt}%
\definecolor{currentstroke}{rgb}{0.000000,0.000000,0.000000}%
\pgfsetstrokecolor{currentstroke}%
\pgfsetdash{}{0pt}%
\pgfpathmoveto{\pgfqpoint{0.443025in}{1.340103in}}%
\pgfpathlineto{\pgfqpoint{0.411855in}{1.318976in}}%
\pgfusepath{stroke}%
\end{pgfscope}%
\begin{pgfscope}%
\pgfsetrectcap%
\pgfsetroundjoin%
\pgfsetlinewidth{0.803000pt}%
\definecolor{currentstroke}{rgb}{0.000000,0.000000,0.000000}%
\pgfsetstrokecolor{currentstroke}%
\pgfsetdash{}{0pt}%
\pgfpathmoveto{\pgfqpoint{0.427845in}{1.522058in}}%
\pgfpathlineto{\pgfqpoint{0.396264in}{1.500926in}}%
\pgfusepath{stroke}%
\end{pgfscope}%
\begin{pgfscope}%
\pgfsetrectcap%
\pgfsetroundjoin%
\pgfsetlinewidth{0.803000pt}%
\definecolor{currentstroke}{rgb}{0.000000,0.000000,0.000000}%
\pgfsetstrokecolor{currentstroke}%
\pgfsetdash{}{0pt}%
\pgfpathmoveto{\pgfqpoint{0.412281in}{1.708614in}}%
\pgfpathlineto{\pgfqpoint{0.380278in}{1.687484in}}%
\pgfusepath{stroke}%
\end{pgfscope}%
\begin{pgfscope}%
\pgfsetrectcap%
\pgfsetroundjoin%
\pgfsetlinewidth{0.803000pt}%
\definecolor{currentstroke}{rgb}{0.000000,0.000000,0.000000}%
\pgfsetstrokecolor{currentstroke}%
\pgfsetdash{}{0pt}%
\pgfpathmoveto{\pgfqpoint{0.396319in}{1.899948in}}%
\pgfpathlineto{\pgfqpoint{0.363882in}{1.878827in}}%
\pgfusepath{stroke}%
\end{pgfscope}%
\begin{pgfscope}%
\definecolor{textcolor}{rgb}{0.000000,0.000000,0.000000}%
\pgfsetstrokecolor{textcolor}%
\pgfsetfillcolor{textcolor}%
\pgftext[x=-0.143944in,y=1.551958in,,]{\color{textcolor}{\rmfamily\fontsize{14.000000}{16.800000}\selectfont\catcode`\^=\active\def^{\ifmmode\sp\else\^{}\fi}\catcode`\%=\active\def%{\%}f3}}%
\end{pgfscope}%
\begin{pgfscope}%
\pgfpathrectangle{\pgfqpoint{0.254231in}{0.147348in}}{\pgfqpoint{2.735294in}{2.735294in}}%
\pgfusepath{clip}%
\pgfsetbuttcap%
\pgfsetroundjoin%
\definecolor{currentfill}{rgb}{0.050070,0.192203,0.290728}%
\pgfsetfillcolor{currentfill}%
\pgfsetlinewidth{0.000000pt}%
\definecolor{currentstroke}{rgb}{0.000000,0.000000,0.000000}%
\pgfsetstrokecolor{currentstroke}%
\pgfsetdash{}{0pt}%
\pgfpathmoveto{\pgfqpoint{2.615535in}{1.291641in}}%
\pgfpathlineto{\pgfqpoint{2.528344in}{1.165011in}}%
\pgfpathlineto{\pgfqpoint{2.615440in}{1.225003in}}%
\pgfpathlineto{\pgfqpoint{2.615535in}{1.291641in}}%
\pgfpathclose%
\pgfusepath{fill}%
\end{pgfscope}%
\begin{pgfscope}%
\pgfpathrectangle{\pgfqpoint{0.254231in}{0.147348in}}{\pgfqpoint{2.735294in}{2.735294in}}%
\pgfusepath{clip}%
\pgfsetbuttcap%
\pgfsetroundjoin%
\definecolor{currentfill}{rgb}{0.050070,0.192203,0.290728}%
\pgfsetfillcolor{currentfill}%
\pgfsetlinewidth{0.000000pt}%
\definecolor{currentstroke}{rgb}{0.000000,0.000000,0.000000}%
\pgfsetstrokecolor{currentstroke}%
\pgfsetdash{}{0pt}%
\pgfpathmoveto{\pgfqpoint{0.789340in}{1.165011in}}%
\pgfpathlineto{\pgfqpoint{0.702149in}{1.291641in}}%
\pgfpathlineto{\pgfqpoint{0.702244in}{1.225003in}}%
\pgfpathlineto{\pgfqpoint{0.789340in}{1.165011in}}%
\pgfpathclose%
\pgfusepath{fill}%
\end{pgfscope}%
\begin{pgfscope}%
\pgfpathrectangle{\pgfqpoint{0.254231in}{0.147348in}}{\pgfqpoint{2.735294in}{2.735294in}}%
\pgfusepath{clip}%
\pgfsetbuttcap%
\pgfsetroundjoin%
\definecolor{currentfill}{rgb}{0.090605,0.347808,0.526096}%
\pgfsetfillcolor{currentfill}%
\pgfsetlinewidth{0.000000pt}%
\definecolor{currentstroke}{rgb}{0.000000,0.000000,0.000000}%
\pgfsetstrokecolor{currentstroke}%
\pgfsetdash{}{0pt}%
\pgfpathmoveto{\pgfqpoint{1.571650in}{2.561465in}}%
\pgfpathlineto{\pgfqpoint{1.746034in}{2.561465in}}%
\pgfpathlineto{\pgfqpoint{1.658842in}{2.621838in}}%
\pgfpathlineto{\pgfqpoint{1.571650in}{2.561465in}}%
\pgfpathclose%
\pgfusepath{fill}%
\end{pgfscope}%
\begin{pgfscope}%
\pgfpathrectangle{\pgfqpoint{0.254231in}{0.147348in}}{\pgfqpoint{2.735294in}{2.735294in}}%
\pgfusepath{clip}%
\pgfsetbuttcap%
\pgfsetroundjoin%
\definecolor{currentfill}{rgb}{0.047548,0.182523,0.276086}%
\pgfsetfillcolor{currentfill}%
\pgfsetlinewidth{0.000000pt}%
\definecolor{currentstroke}{rgb}{0.000000,0.000000,0.000000}%
\pgfsetstrokecolor{currentstroke}%
\pgfsetdash{}{0pt}%
\pgfpathmoveto{\pgfqpoint{2.528344in}{1.165011in}}%
\pgfpathlineto{\pgfqpoint{2.518939in}{1.232896in}}%
\pgfpathlineto{\pgfqpoint{2.415375in}{1.101844in}}%
\pgfpathlineto{\pgfqpoint{2.528344in}{1.165011in}}%
\pgfpathclose%
\pgfusepath{fill}%
\end{pgfscope}%
\begin{pgfscope}%
\pgfpathrectangle{\pgfqpoint{0.254231in}{0.147348in}}{\pgfqpoint{2.735294in}{2.735294in}}%
\pgfusepath{clip}%
\pgfsetbuttcap%
\pgfsetroundjoin%
\definecolor{currentfill}{rgb}{0.047548,0.182523,0.276086}%
\pgfsetfillcolor{currentfill}%
\pgfsetlinewidth{0.000000pt}%
\definecolor{currentstroke}{rgb}{0.000000,0.000000,0.000000}%
\pgfsetstrokecolor{currentstroke}%
\pgfsetdash{}{0pt}%
\pgfpathmoveto{\pgfqpoint{0.902309in}{1.101844in}}%
\pgfpathlineto{\pgfqpoint{0.798745in}{1.232896in}}%
\pgfpathlineto{\pgfqpoint{0.789340in}{1.165011in}}%
\pgfpathlineto{\pgfqpoint{0.902309in}{1.101844in}}%
\pgfpathclose%
\pgfusepath{fill}%
\end{pgfscope}%
\begin{pgfscope}%
\pgfpathrectangle{\pgfqpoint{0.254231in}{0.147348in}}{\pgfqpoint{2.735294in}{2.735294in}}%
\pgfusepath{clip}%
\pgfsetbuttcap%
\pgfsetroundjoin%
\definecolor{currentfill}{rgb}{0.048960,0.187944,0.284285}%
\pgfsetfillcolor{currentfill}%
\pgfsetlinewidth{0.000000pt}%
\definecolor{currentstroke}{rgb}{0.000000,0.000000,0.000000}%
\pgfsetstrokecolor{currentstroke}%
\pgfsetdash{}{0pt}%
\pgfpathmoveto{\pgfqpoint{0.702149in}{1.291641in}}%
\pgfpathlineto{\pgfqpoint{0.789340in}{1.165011in}}%
\pgfpathlineto{\pgfqpoint{0.788552in}{1.618388in}}%
\pgfpathlineto{\pgfqpoint{0.702149in}{1.291641in}}%
\pgfpathclose%
\pgfusepath{fill}%
\end{pgfscope}%
\begin{pgfscope}%
\pgfpathrectangle{\pgfqpoint{0.254231in}{0.147348in}}{\pgfqpoint{2.735294in}{2.735294in}}%
\pgfusepath{clip}%
\pgfsetbuttcap%
\pgfsetroundjoin%
\definecolor{currentfill}{rgb}{0.048960,0.187944,0.284285}%
\pgfsetfillcolor{currentfill}%
\pgfsetlinewidth{0.000000pt}%
\definecolor{currentstroke}{rgb}{0.000000,0.000000,0.000000}%
\pgfsetstrokecolor{currentstroke}%
\pgfsetdash{}{0pt}%
\pgfpathmoveto{\pgfqpoint{2.615535in}{1.291641in}}%
\pgfpathlineto{\pgfqpoint{2.529132in}{1.618388in}}%
\pgfpathlineto{\pgfqpoint{2.528344in}{1.165011in}}%
\pgfpathlineto{\pgfqpoint{2.615535in}{1.291641in}}%
\pgfpathclose%
\pgfusepath{fill}%
\end{pgfscope}%
\begin{pgfscope}%
\pgfpathrectangle{\pgfqpoint{0.254231in}{0.147348in}}{\pgfqpoint{2.735294in}{2.735294in}}%
\pgfusepath{clip}%
\pgfsetbuttcap%
\pgfsetroundjoin%
\definecolor{currentfill}{rgb}{0.070254,0.269685,0.407928}%
\pgfsetfillcolor{currentfill}%
\pgfsetlinewidth{0.000000pt}%
\definecolor{currentstroke}{rgb}{0.000000,0.000000,0.000000}%
\pgfsetstrokecolor{currentstroke}%
\pgfsetdash{}{0pt}%
\pgfpathmoveto{\pgfqpoint{2.518939in}{1.232896in}}%
\pgfpathlineto{\pgfqpoint{2.528344in}{1.165011in}}%
\pgfpathlineto{\pgfqpoint{2.529132in}{1.618388in}}%
\pgfpathlineto{\pgfqpoint{2.518939in}{1.232896in}}%
\pgfpathclose%
\pgfusepath{fill}%
\end{pgfscope}%
\begin{pgfscope}%
\pgfpathrectangle{\pgfqpoint{0.254231in}{0.147348in}}{\pgfqpoint{2.735294in}{2.735294in}}%
\pgfusepath{clip}%
\pgfsetbuttcap%
\pgfsetroundjoin%
\definecolor{currentfill}{rgb}{0.070254,0.269685,0.407928}%
\pgfsetfillcolor{currentfill}%
\pgfsetlinewidth{0.000000pt}%
\definecolor{currentstroke}{rgb}{0.000000,0.000000,0.000000}%
\pgfsetstrokecolor{currentstroke}%
\pgfsetdash{}{0pt}%
\pgfpathmoveto{\pgfqpoint{0.788552in}{1.618388in}}%
\pgfpathlineto{\pgfqpoint{0.789340in}{1.165011in}}%
\pgfpathlineto{\pgfqpoint{0.798745in}{1.232896in}}%
\pgfpathlineto{\pgfqpoint{0.788552in}{1.618388in}}%
\pgfpathclose%
\pgfusepath{fill}%
\end{pgfscope}%
\begin{pgfscope}%
\pgfpathrectangle{\pgfqpoint{0.254231in}{0.147348in}}{\pgfqpoint{2.735294in}{2.735294in}}%
\pgfusepath{clip}%
\pgfsetbuttcap%
\pgfsetroundjoin%
\definecolor{currentfill}{rgb}{0.044978,0.172658,0.261163}%
\pgfsetfillcolor{currentfill}%
\pgfsetlinewidth{0.000000pt}%
\definecolor{currentstroke}{rgb}{0.000000,0.000000,0.000000}%
\pgfsetstrokecolor{currentstroke}%
\pgfsetdash{}{0pt}%
\pgfpathmoveto{\pgfqpoint{1.046414in}{1.039603in}}%
\pgfpathlineto{\pgfqpoint{0.926707in}{1.171211in}}%
\pgfpathlineto{\pgfqpoint{0.902309in}{1.101844in}}%
\pgfpathlineto{\pgfqpoint{1.046414in}{1.039603in}}%
\pgfpathclose%
\pgfusepath{fill}%
\end{pgfscope}%
\begin{pgfscope}%
\pgfpathrectangle{\pgfqpoint{0.254231in}{0.147348in}}{\pgfqpoint{2.735294in}{2.735294in}}%
\pgfusepath{clip}%
\pgfsetbuttcap%
\pgfsetroundjoin%
\definecolor{currentfill}{rgb}{0.044978,0.172658,0.261163}%
\pgfsetfillcolor{currentfill}%
\pgfsetlinewidth{0.000000pt}%
\definecolor{currentstroke}{rgb}{0.000000,0.000000,0.000000}%
\pgfsetstrokecolor{currentstroke}%
\pgfsetdash{}{0pt}%
\pgfpathmoveto{\pgfqpoint{2.415375in}{1.101844in}}%
\pgfpathlineto{\pgfqpoint{2.390977in}{1.171211in}}%
\pgfpathlineto{\pgfqpoint{2.271270in}{1.039603in}}%
\pgfpathlineto{\pgfqpoint{2.415375in}{1.101844in}}%
\pgfpathclose%
\pgfusepath{fill}%
\end{pgfscope}%
\begin{pgfscope}%
\pgfpathrectangle{\pgfqpoint{0.254231in}{0.147348in}}{\pgfqpoint{2.735294in}{2.735294in}}%
\pgfusepath{clip}%
\pgfsetbuttcap%
\pgfsetroundjoin%
\definecolor{currentfill}{rgb}{0.081954,0.314596,0.475860}%
\pgfsetfillcolor{currentfill}%
\pgfsetlinewidth{0.000000pt}%
\definecolor{currentstroke}{rgb}{0.000000,0.000000,0.000000}%
\pgfsetstrokecolor{currentstroke}%
\pgfsetdash{}{0pt}%
\pgfpathmoveto{\pgfqpoint{1.746034in}{2.561465in}}%
\pgfpathlineto{\pgfqpoint{1.571650in}{2.561465in}}%
\pgfpathlineto{\pgfqpoint{1.534055in}{2.124855in}}%
\pgfpathlineto{\pgfqpoint{1.746034in}{2.561465in}}%
\pgfpathclose%
\pgfusepath{fill}%
\end{pgfscope}%
\begin{pgfscope}%
\pgfpathrectangle{\pgfqpoint{0.254231in}{0.147348in}}{\pgfqpoint{2.735294in}{2.735294in}}%
\pgfusepath{clip}%
\pgfsetbuttcap%
\pgfsetroundjoin%
\definecolor{currentfill}{rgb}{0.047247,0.181368,0.274339}%
\pgfsetfillcolor{currentfill}%
\pgfsetlinewidth{0.000000pt}%
\definecolor{currentstroke}{rgb}{0.000000,0.000000,0.000000}%
\pgfsetstrokecolor{currentstroke}%
\pgfsetdash{}{0pt}%
\pgfpathmoveto{\pgfqpoint{2.382112in}{1.589448in}}%
\pgfpathlineto{\pgfqpoint{2.415375in}{1.101844in}}%
\pgfpathlineto{\pgfqpoint{2.518939in}{1.232896in}}%
\pgfpathlineto{\pgfqpoint{2.382112in}{1.589448in}}%
\pgfpathclose%
\pgfusepath{fill}%
\end{pgfscope}%
\begin{pgfscope}%
\pgfpathrectangle{\pgfqpoint{0.254231in}{0.147348in}}{\pgfqpoint{2.735294in}{2.735294in}}%
\pgfusepath{clip}%
\pgfsetbuttcap%
\pgfsetroundjoin%
\definecolor{currentfill}{rgb}{0.047247,0.181368,0.274339}%
\pgfsetfillcolor{currentfill}%
\pgfsetlinewidth{0.000000pt}%
\definecolor{currentstroke}{rgb}{0.000000,0.000000,0.000000}%
\pgfsetstrokecolor{currentstroke}%
\pgfsetdash{}{0pt}%
\pgfpathmoveto{\pgfqpoint{0.798745in}{1.232896in}}%
\pgfpathlineto{\pgfqpoint{0.902309in}{1.101844in}}%
\pgfpathlineto{\pgfqpoint{0.935572in}{1.589448in}}%
\pgfpathlineto{\pgfqpoint{0.798745in}{1.232896in}}%
\pgfpathclose%
\pgfusepath{fill}%
\end{pgfscope}%
\begin{pgfscope}%
\pgfpathrectangle{\pgfqpoint{0.254231in}{0.147348in}}{\pgfqpoint{2.735294in}{2.735294in}}%
\pgfusepath{clip}%
\pgfsetbuttcap%
\pgfsetroundjoin%
\definecolor{currentfill}{rgb}{0.067179,0.257880,0.390071}%
\pgfsetfillcolor{currentfill}%
\pgfsetlinewidth{0.000000pt}%
\definecolor{currentstroke}{rgb}{0.000000,0.000000,0.000000}%
\pgfsetstrokecolor{currentstroke}%
\pgfsetdash{}{0pt}%
\pgfpathmoveto{\pgfqpoint{0.935572in}{1.589448in}}%
\pgfpathlineto{\pgfqpoint{0.902309in}{1.101844in}}%
\pgfpathlineto{\pgfqpoint{0.926707in}{1.171211in}}%
\pgfpathlineto{\pgfqpoint{0.935572in}{1.589448in}}%
\pgfpathclose%
\pgfusepath{fill}%
\end{pgfscope}%
\begin{pgfscope}%
\pgfpathrectangle{\pgfqpoint{0.254231in}{0.147348in}}{\pgfqpoint{2.735294in}{2.735294in}}%
\pgfusepath{clip}%
\pgfsetbuttcap%
\pgfsetroundjoin%
\definecolor{currentfill}{rgb}{0.067179,0.257880,0.390071}%
\pgfsetfillcolor{currentfill}%
\pgfsetlinewidth{0.000000pt}%
\definecolor{currentstroke}{rgb}{0.000000,0.000000,0.000000}%
\pgfsetstrokecolor{currentstroke}%
\pgfsetdash{}{0pt}%
\pgfpathmoveto{\pgfqpoint{2.390977in}{1.171211in}}%
\pgfpathlineto{\pgfqpoint{2.415375in}{1.101844in}}%
\pgfpathlineto{\pgfqpoint{2.382112in}{1.589448in}}%
\pgfpathlineto{\pgfqpoint{2.390977in}{1.171211in}}%
\pgfpathclose%
\pgfusepath{fill}%
\end{pgfscope}%
\begin{pgfscope}%
\pgfpathrectangle{\pgfqpoint{0.254231in}{0.147348in}}{\pgfqpoint{2.735294in}{2.735294in}}%
\pgfusepath{clip}%
\pgfsetbuttcap%
\pgfsetroundjoin%
\definecolor{currentfill}{rgb}{0.042579,0.163449,0.247234}%
\pgfsetfillcolor{currentfill}%
\pgfsetlinewidth{0.000000pt}%
\definecolor{currentstroke}{rgb}{0.000000,0.000000,0.000000}%
\pgfsetstrokecolor{currentstroke}%
\pgfsetdash{}{0pt}%
\pgfpathmoveto{\pgfqpoint{1.046414in}{1.039603in}}%
\pgfpathlineto{\pgfqpoint{1.224246in}{0.985051in}}%
\pgfpathlineto{\pgfqpoint{1.092413in}{1.111775in}}%
\pgfpathlineto{\pgfqpoint{1.046414in}{1.039603in}}%
\pgfpathclose%
\pgfusepath{fill}%
\end{pgfscope}%
\begin{pgfscope}%
\pgfpathrectangle{\pgfqpoint{0.254231in}{0.147348in}}{\pgfqpoint{2.735294in}{2.735294in}}%
\pgfusepath{clip}%
\pgfsetbuttcap%
\pgfsetroundjoin%
\definecolor{currentfill}{rgb}{0.042579,0.163449,0.247234}%
\pgfsetfillcolor{currentfill}%
\pgfsetlinewidth{0.000000pt}%
\definecolor{currentstroke}{rgb}{0.000000,0.000000,0.000000}%
\pgfsetstrokecolor{currentstroke}%
\pgfsetdash{}{0pt}%
\pgfpathmoveto{\pgfqpoint{2.225271in}{1.111775in}}%
\pgfpathlineto{\pgfqpoint{2.093438in}{0.985051in}}%
\pgfpathlineto{\pgfqpoint{2.271270in}{1.039603in}}%
\pgfpathlineto{\pgfqpoint{2.225271in}{1.111775in}}%
\pgfpathclose%
\pgfusepath{fill}%
\end{pgfscope}%
\begin{pgfscope}%
\pgfpathrectangle{\pgfqpoint{0.254231in}{0.147348in}}{\pgfqpoint{2.735294in}{2.735294in}}%
\pgfusepath{clip}%
\pgfsetbuttcap%
\pgfsetroundjoin%
\definecolor{currentfill}{rgb}{0.052493,0.201505,0.304798}%
\pgfsetfillcolor{currentfill}%
\pgfsetlinewidth{0.000000pt}%
\definecolor{currentstroke}{rgb}{0.000000,0.000000,0.000000}%
\pgfsetstrokecolor{currentstroke}%
\pgfsetdash{}{0pt}%
\pgfpathmoveto{\pgfqpoint{2.382112in}{1.589448in}}%
\pgfpathlineto{\pgfqpoint{2.518939in}{1.232896in}}%
\pgfpathlineto{\pgfqpoint{2.529132in}{1.618388in}}%
\pgfpathlineto{\pgfqpoint{2.382112in}{1.589448in}}%
\pgfpathclose%
\pgfusepath{fill}%
\end{pgfscope}%
\begin{pgfscope}%
\pgfpathrectangle{\pgfqpoint{0.254231in}{0.147348in}}{\pgfqpoint{2.735294in}{2.735294in}}%
\pgfusepath{clip}%
\pgfsetbuttcap%
\pgfsetroundjoin%
\definecolor{currentfill}{rgb}{0.052493,0.201505,0.304798}%
\pgfsetfillcolor{currentfill}%
\pgfsetlinewidth{0.000000pt}%
\definecolor{currentstroke}{rgb}{0.000000,0.000000,0.000000}%
\pgfsetstrokecolor{currentstroke}%
\pgfsetdash{}{0pt}%
\pgfpathmoveto{\pgfqpoint{0.788552in}{1.618388in}}%
\pgfpathlineto{\pgfqpoint{0.798745in}{1.232896in}}%
\pgfpathlineto{\pgfqpoint{0.935572in}{1.589448in}}%
\pgfpathlineto{\pgfqpoint{0.788552in}{1.618388in}}%
\pgfpathclose%
\pgfusepath{fill}%
\end{pgfscope}%
\begin{pgfscope}%
\pgfpathrectangle{\pgfqpoint{0.254231in}{0.147348in}}{\pgfqpoint{2.735294in}{2.735294in}}%
\pgfusepath{clip}%
\pgfsetbuttcap%
\pgfsetroundjoin%
\definecolor{currentfill}{rgb}{0.082280,0.315849,0.477755}%
\pgfsetfillcolor{currentfill}%
\pgfsetlinewidth{0.000000pt}%
\definecolor{currentstroke}{rgb}{0.000000,0.000000,0.000000}%
\pgfsetstrokecolor{currentstroke}%
\pgfsetdash{}{0pt}%
\pgfpathmoveto{\pgfqpoint{2.094537in}{2.253326in}}%
\pgfpathlineto{\pgfqpoint{1.746034in}{2.561465in}}%
\pgfpathlineto{\pgfqpoint{1.783629in}{2.124855in}}%
\pgfpathlineto{\pgfqpoint{2.094537in}{2.253326in}}%
\pgfpathclose%
\pgfusepath{fill}%
\end{pgfscope}%
\begin{pgfscope}%
\pgfpathrectangle{\pgfqpoint{0.254231in}{0.147348in}}{\pgfqpoint{2.735294in}{2.735294in}}%
\pgfusepath{clip}%
\pgfsetbuttcap%
\pgfsetroundjoin%
\definecolor{currentfill}{rgb}{0.082280,0.315849,0.477755}%
\pgfsetfillcolor{currentfill}%
\pgfsetlinewidth{0.000000pt}%
\definecolor{currentstroke}{rgb}{0.000000,0.000000,0.000000}%
\pgfsetstrokecolor{currentstroke}%
\pgfsetdash{}{0pt}%
\pgfpathmoveto{\pgfqpoint{1.534055in}{2.124855in}}%
\pgfpathlineto{\pgfqpoint{1.571650in}{2.561465in}}%
\pgfpathlineto{\pgfqpoint{1.223147in}{2.253326in}}%
\pgfpathlineto{\pgfqpoint{1.534055in}{2.124855in}}%
\pgfpathclose%
\pgfusepath{fill}%
\end{pgfscope}%
\begin{pgfscope}%
\pgfpathrectangle{\pgfqpoint{0.254231in}{0.147348in}}{\pgfqpoint{2.735294in}{2.735294in}}%
\pgfusepath{clip}%
\pgfsetbuttcap%
\pgfsetroundjoin%
\definecolor{currentfill}{rgb}{0.045702,0.175435,0.265364}%
\pgfsetfillcolor{currentfill}%
\pgfsetlinewidth{0.000000pt}%
\definecolor{currentstroke}{rgb}{0.000000,0.000000,0.000000}%
\pgfsetstrokecolor{currentstroke}%
\pgfsetdash{}{0pt}%
\pgfpathmoveto{\pgfqpoint{0.926707in}{1.171211in}}%
\pgfpathlineto{\pgfqpoint{1.046414in}{1.039603in}}%
\pgfpathlineto{\pgfqpoint{1.132195in}{1.562087in}}%
\pgfpathlineto{\pgfqpoint{0.926707in}{1.171211in}}%
\pgfpathclose%
\pgfusepath{fill}%
\end{pgfscope}%
\begin{pgfscope}%
\pgfpathrectangle{\pgfqpoint{0.254231in}{0.147348in}}{\pgfqpoint{2.735294in}{2.735294in}}%
\pgfusepath{clip}%
\pgfsetbuttcap%
\pgfsetroundjoin%
\definecolor{currentfill}{rgb}{0.045702,0.175435,0.265364}%
\pgfsetfillcolor{currentfill}%
\pgfsetlinewidth{0.000000pt}%
\definecolor{currentstroke}{rgb}{0.000000,0.000000,0.000000}%
\pgfsetstrokecolor{currentstroke}%
\pgfsetdash{}{0pt}%
\pgfpathmoveto{\pgfqpoint{2.185489in}{1.562087in}}%
\pgfpathlineto{\pgfqpoint{2.271270in}{1.039603in}}%
\pgfpathlineto{\pgfqpoint{2.390977in}{1.171211in}}%
\pgfpathlineto{\pgfqpoint{2.185489in}{1.562087in}}%
\pgfpathclose%
\pgfusepath{fill}%
\end{pgfscope}%
\begin{pgfscope}%
\pgfpathrectangle{\pgfqpoint{0.254231in}{0.147348in}}{\pgfqpoint{2.735294in}{2.735294in}}%
\pgfusepath{clip}%
\pgfsetbuttcap%
\pgfsetroundjoin%
\definecolor{currentfill}{rgb}{0.040669,0.156116,0.236142}%
\pgfsetfillcolor{currentfill}%
\pgfsetlinewidth{0.000000pt}%
\definecolor{currentstroke}{rgb}{0.000000,0.000000,0.000000}%
\pgfsetstrokecolor{currentstroke}%
\pgfsetdash{}{0pt}%
\pgfpathmoveto{\pgfqpoint{1.297586in}{1.063020in}}%
\pgfpathlineto{\pgfqpoint{1.224246in}{0.985051in}}%
\pgfpathlineto{\pgfqpoint{1.432345in}{0.946905in}}%
\pgfpathlineto{\pgfqpoint{1.297586in}{1.063020in}}%
\pgfpathclose%
\pgfusepath{fill}%
\end{pgfscope}%
\begin{pgfscope}%
\pgfpathrectangle{\pgfqpoint{0.254231in}{0.147348in}}{\pgfqpoint{2.735294in}{2.735294in}}%
\pgfusepath{clip}%
\pgfsetbuttcap%
\pgfsetroundjoin%
\definecolor{currentfill}{rgb}{0.040669,0.156116,0.236142}%
\pgfsetfillcolor{currentfill}%
\pgfsetlinewidth{0.000000pt}%
\definecolor{currentstroke}{rgb}{0.000000,0.000000,0.000000}%
\pgfsetstrokecolor{currentstroke}%
\pgfsetdash{}{0pt}%
\pgfpathmoveto{\pgfqpoint{1.885339in}{0.946905in}}%
\pgfpathlineto{\pgfqpoint{2.093438in}{0.985051in}}%
\pgfpathlineto{\pgfqpoint{2.020098in}{1.063020in}}%
\pgfpathlineto{\pgfqpoint{1.885339in}{0.946905in}}%
\pgfpathclose%
\pgfusepath{fill}%
\end{pgfscope}%
\begin{pgfscope}%
\pgfpathrectangle{\pgfqpoint{0.254231in}{0.147348in}}{\pgfqpoint{2.735294in}{2.735294in}}%
\pgfusepath{clip}%
\pgfsetbuttcap%
\pgfsetroundjoin%
\definecolor{currentfill}{rgb}{0.063981,0.245604,0.371502}%
\pgfsetfillcolor{currentfill}%
\pgfsetlinewidth{0.000000pt}%
\definecolor{currentstroke}{rgb}{0.000000,0.000000,0.000000}%
\pgfsetstrokecolor{currentstroke}%
\pgfsetdash{}{0pt}%
\pgfpathmoveto{\pgfqpoint{1.132195in}{1.562087in}}%
\pgfpathlineto{\pgfqpoint{1.046414in}{1.039603in}}%
\pgfpathlineto{\pgfqpoint{1.092413in}{1.111775in}}%
\pgfpathlineto{\pgfqpoint{1.132195in}{1.562087in}}%
\pgfpathclose%
\pgfusepath{fill}%
\end{pgfscope}%
\begin{pgfscope}%
\pgfpathrectangle{\pgfqpoint{0.254231in}{0.147348in}}{\pgfqpoint{2.735294in}{2.735294in}}%
\pgfusepath{clip}%
\pgfsetbuttcap%
\pgfsetroundjoin%
\definecolor{currentfill}{rgb}{0.063981,0.245604,0.371502}%
\pgfsetfillcolor{currentfill}%
\pgfsetlinewidth{0.000000pt}%
\definecolor{currentstroke}{rgb}{0.000000,0.000000,0.000000}%
\pgfsetstrokecolor{currentstroke}%
\pgfsetdash{}{0pt}%
\pgfpathmoveto{\pgfqpoint{2.225271in}{1.111775in}}%
\pgfpathlineto{\pgfqpoint{2.271270in}{1.039603in}}%
\pgfpathlineto{\pgfqpoint{2.185489in}{1.562087in}}%
\pgfpathlineto{\pgfqpoint{2.225271in}{1.111775in}}%
\pgfpathclose%
\pgfusepath{fill}%
\end{pgfscope}%
\begin{pgfscope}%
\pgfpathrectangle{\pgfqpoint{0.254231in}{0.147348in}}{\pgfqpoint{2.735294in}{2.735294in}}%
\pgfusepath{clip}%
\pgfsetbuttcap%
\pgfsetroundjoin%
\definecolor{currentfill}{rgb}{0.060942,0.233938,0.353856}%
\pgfsetfillcolor{currentfill}%
\pgfsetlinewidth{0.000000pt}%
\definecolor{currentstroke}{rgb}{0.000000,0.000000,0.000000}%
\pgfsetstrokecolor{currentstroke}%
\pgfsetdash{}{0pt}%
\pgfpathmoveto{\pgfqpoint{0.935572in}{1.589448in}}%
\pgfpathlineto{\pgfqpoint{0.864797in}{1.772344in}}%
\pgfpathlineto{\pgfqpoint{0.788552in}{1.618388in}}%
\pgfpathlineto{\pgfqpoint{0.935572in}{1.589448in}}%
\pgfpathclose%
\pgfusepath{fill}%
\end{pgfscope}%
\begin{pgfscope}%
\pgfpathrectangle{\pgfqpoint{0.254231in}{0.147348in}}{\pgfqpoint{2.735294in}{2.735294in}}%
\pgfusepath{clip}%
\pgfsetbuttcap%
\pgfsetroundjoin%
\definecolor{currentfill}{rgb}{0.060942,0.233938,0.353856}%
\pgfsetfillcolor{currentfill}%
\pgfsetlinewidth{0.000000pt}%
\definecolor{currentstroke}{rgb}{0.000000,0.000000,0.000000}%
\pgfsetstrokecolor{currentstroke}%
\pgfsetdash{}{0pt}%
\pgfpathmoveto{\pgfqpoint{2.529132in}{1.618388in}}%
\pgfpathlineto{\pgfqpoint{2.452887in}{1.772344in}}%
\pgfpathlineto{\pgfqpoint{2.382112in}{1.589448in}}%
\pgfpathlineto{\pgfqpoint{2.529132in}{1.618388in}}%
\pgfpathclose%
\pgfusepath{fill}%
\end{pgfscope}%
\begin{pgfscope}%
\pgfpathrectangle{\pgfqpoint{0.254231in}{0.147348in}}{\pgfqpoint{2.735294in}{2.735294in}}%
\pgfusepath{clip}%
\pgfsetbuttcap%
\pgfsetroundjoin%
\definecolor{currentfill}{rgb}{0.081954,0.314596,0.475860}%
\pgfsetfillcolor{currentfill}%
\pgfsetlinewidth{0.000000pt}%
\definecolor{currentstroke}{rgb}{0.000000,0.000000,0.000000}%
\pgfsetstrokecolor{currentstroke}%
\pgfsetdash{}{0pt}%
\pgfpathmoveto{\pgfqpoint{1.534055in}{2.124855in}}%
\pgfpathlineto{\pgfqpoint{1.783629in}{2.124855in}}%
\pgfpathlineto{\pgfqpoint{1.746034in}{2.561465in}}%
\pgfpathlineto{\pgfqpoint{1.534055in}{2.124855in}}%
\pgfpathclose%
\pgfusepath{fill}%
\end{pgfscope}%
\begin{pgfscope}%
\pgfpathrectangle{\pgfqpoint{0.254231in}{0.147348in}}{\pgfqpoint{2.735294in}{2.735294in}}%
\pgfusepath{clip}%
\pgfsetbuttcap%
\pgfsetroundjoin%
\definecolor{currentfill}{rgb}{0.039595,0.151995,0.229908}%
\pgfsetfillcolor{currentfill}%
\pgfsetlinewidth{0.000000pt}%
\definecolor{currentstroke}{rgb}{0.000000,0.000000,0.000000}%
\pgfsetstrokecolor{currentstroke}%
\pgfsetdash{}{0pt}%
\pgfpathmoveto{\pgfqpoint{1.658842in}{0.933095in}}%
\pgfpathlineto{\pgfqpoint{1.534325in}{1.035006in}}%
\pgfpathlineto{\pgfqpoint{1.432345in}{0.946905in}}%
\pgfpathlineto{\pgfqpoint{1.658842in}{0.933095in}}%
\pgfpathclose%
\pgfusepath{fill}%
\end{pgfscope}%
\begin{pgfscope}%
\pgfpathrectangle{\pgfqpoint{0.254231in}{0.147348in}}{\pgfqpoint{2.735294in}{2.735294in}}%
\pgfusepath{clip}%
\pgfsetbuttcap%
\pgfsetroundjoin%
\definecolor{currentfill}{rgb}{0.039595,0.151995,0.229908}%
\pgfsetfillcolor{currentfill}%
\pgfsetlinewidth{0.000000pt}%
\definecolor{currentstroke}{rgb}{0.000000,0.000000,0.000000}%
\pgfsetstrokecolor{currentstroke}%
\pgfsetdash{}{0pt}%
\pgfpathmoveto{\pgfqpoint{1.885339in}{0.946905in}}%
\pgfpathlineto{\pgfqpoint{1.783359in}{1.035006in}}%
\pgfpathlineto{\pgfqpoint{1.658842in}{0.933095in}}%
\pgfpathlineto{\pgfqpoint{1.885339in}{0.946905in}}%
\pgfpathclose%
\pgfusepath{fill}%
\end{pgfscope}%
\begin{pgfscope}%
\pgfpathrectangle{\pgfqpoint{0.254231in}{0.147348in}}{\pgfqpoint{2.735294in}{2.735294in}}%
\pgfusepath{clip}%
\pgfsetbuttcap%
\pgfsetroundjoin%
\definecolor{currentfill}{rgb}{0.075436,0.289576,0.438014}%
\pgfsetfillcolor{currentfill}%
\pgfsetlinewidth{0.000000pt}%
\definecolor{currentstroke}{rgb}{0.000000,0.000000,0.000000}%
\pgfsetstrokecolor{currentstroke}%
\pgfsetdash{}{0pt}%
\pgfpathmoveto{\pgfqpoint{2.226391in}{2.103280in}}%
\pgfpathlineto{\pgfqpoint{2.094537in}{2.253326in}}%
\pgfpathlineto{\pgfqpoint{2.020857in}{2.116981in}}%
\pgfpathlineto{\pgfqpoint{2.226391in}{2.103280in}}%
\pgfpathclose%
\pgfusepath{fill}%
\end{pgfscope}%
\begin{pgfscope}%
\pgfpathrectangle{\pgfqpoint{0.254231in}{0.147348in}}{\pgfqpoint{2.735294in}{2.735294in}}%
\pgfusepath{clip}%
\pgfsetbuttcap%
\pgfsetroundjoin%
\definecolor{currentfill}{rgb}{0.075436,0.289576,0.438014}%
\pgfsetfillcolor{currentfill}%
\pgfsetlinewidth{0.000000pt}%
\definecolor{currentstroke}{rgb}{0.000000,0.000000,0.000000}%
\pgfsetstrokecolor{currentstroke}%
\pgfsetdash{}{0pt}%
\pgfpathmoveto{\pgfqpoint{1.296827in}{2.116981in}}%
\pgfpathlineto{\pgfqpoint{1.223147in}{2.253326in}}%
\pgfpathlineto{\pgfqpoint{1.091293in}{2.103280in}}%
\pgfpathlineto{\pgfqpoint{1.296827in}{2.116981in}}%
\pgfpathclose%
\pgfusepath{fill}%
\end{pgfscope}%
\begin{pgfscope}%
\pgfpathrectangle{\pgfqpoint{0.254231in}{0.147348in}}{\pgfqpoint{2.735294in}{2.735294in}}%
\pgfusepath{clip}%
\pgfsetbuttcap%
\pgfsetroundjoin%
\definecolor{currentfill}{rgb}{0.062760,0.240916,0.364410}%
\pgfsetfillcolor{currentfill}%
\pgfsetlinewidth{0.000000pt}%
\definecolor{currentstroke}{rgb}{0.000000,0.000000,0.000000}%
\pgfsetstrokecolor{currentstroke}%
\pgfsetdash{}{0pt}%
\pgfpathmoveto{\pgfqpoint{0.935572in}{1.589448in}}%
\pgfpathlineto{\pgfqpoint{1.091293in}{2.103280in}}%
\pgfpathlineto{\pgfqpoint{0.864797in}{1.772344in}}%
\pgfpathlineto{\pgfqpoint{0.935572in}{1.589448in}}%
\pgfpathclose%
\pgfusepath{fill}%
\end{pgfscope}%
\begin{pgfscope}%
\pgfpathrectangle{\pgfqpoint{0.254231in}{0.147348in}}{\pgfqpoint{2.735294in}{2.735294in}}%
\pgfusepath{clip}%
\pgfsetbuttcap%
\pgfsetroundjoin%
\definecolor{currentfill}{rgb}{0.062760,0.240916,0.364410}%
\pgfsetfillcolor{currentfill}%
\pgfsetlinewidth{0.000000pt}%
\definecolor{currentstroke}{rgb}{0.000000,0.000000,0.000000}%
\pgfsetstrokecolor{currentstroke}%
\pgfsetdash{}{0pt}%
\pgfpathmoveto{\pgfqpoint{2.452887in}{1.772344in}}%
\pgfpathlineto{\pgfqpoint{2.226391in}{2.103280in}}%
\pgfpathlineto{\pgfqpoint{2.382112in}{1.589448in}}%
\pgfpathlineto{\pgfqpoint{2.452887in}{1.772344in}}%
\pgfpathclose%
\pgfusepath{fill}%
\end{pgfscope}%
\begin{pgfscope}%
\pgfpathrectangle{\pgfqpoint{0.254231in}{0.147348in}}{\pgfqpoint{2.735294in}{2.735294in}}%
\pgfusepath{clip}%
\pgfsetbuttcap%
\pgfsetroundjoin%
\definecolor{currentfill}{rgb}{0.043508,0.167016,0.252629}%
\pgfsetfillcolor{currentfill}%
\pgfsetlinewidth{0.000000pt}%
\definecolor{currentstroke}{rgb}{0.000000,0.000000,0.000000}%
\pgfsetstrokecolor{currentstroke}%
\pgfsetdash{}{0pt}%
\pgfpathmoveto{\pgfqpoint{1.092413in}{1.111775in}}%
\pgfpathlineto{\pgfqpoint{1.224246in}{0.985051in}}%
\pgfpathlineto{\pgfqpoint{1.256196in}{1.360106in}}%
\pgfpathlineto{\pgfqpoint{1.092413in}{1.111775in}}%
\pgfpathclose%
\pgfusepath{fill}%
\end{pgfscope}%
\begin{pgfscope}%
\pgfpathrectangle{\pgfqpoint{0.254231in}{0.147348in}}{\pgfqpoint{2.735294in}{2.735294in}}%
\pgfusepath{clip}%
\pgfsetbuttcap%
\pgfsetroundjoin%
\definecolor{currentfill}{rgb}{0.043508,0.167016,0.252629}%
\pgfsetfillcolor{currentfill}%
\pgfsetlinewidth{0.000000pt}%
\definecolor{currentstroke}{rgb}{0.000000,0.000000,0.000000}%
\pgfsetstrokecolor{currentstroke}%
\pgfsetdash{}{0pt}%
\pgfpathmoveto{\pgfqpoint{2.061488in}{1.360106in}}%
\pgfpathlineto{\pgfqpoint{2.093438in}{0.985051in}}%
\pgfpathlineto{\pgfqpoint{2.225271in}{1.111775in}}%
\pgfpathlineto{\pgfqpoint{2.061488in}{1.360106in}}%
\pgfpathclose%
\pgfusepath{fill}%
\end{pgfscope}%
\begin{pgfscope}%
\pgfpathrectangle{\pgfqpoint{0.254231in}{0.147348in}}{\pgfqpoint{2.735294in}{2.735294in}}%
\pgfusepath{clip}%
\pgfsetbuttcap%
\pgfsetroundjoin%
\definecolor{currentfill}{rgb}{0.050011,0.191979,0.290388}%
\pgfsetfillcolor{currentfill}%
\pgfsetlinewidth{0.000000pt}%
\definecolor{currentstroke}{rgb}{0.000000,0.000000,0.000000}%
\pgfsetstrokecolor{currentstroke}%
\pgfsetdash{}{0pt}%
\pgfpathmoveto{\pgfqpoint{0.935572in}{1.589448in}}%
\pgfpathlineto{\pgfqpoint{0.926707in}{1.171211in}}%
\pgfpathlineto{\pgfqpoint{1.132195in}{1.562087in}}%
\pgfpathlineto{\pgfqpoint{0.935572in}{1.589448in}}%
\pgfpathclose%
\pgfusepath{fill}%
\end{pgfscope}%
\begin{pgfscope}%
\pgfpathrectangle{\pgfqpoint{0.254231in}{0.147348in}}{\pgfqpoint{2.735294in}{2.735294in}}%
\pgfusepath{clip}%
\pgfsetbuttcap%
\pgfsetroundjoin%
\definecolor{currentfill}{rgb}{0.050011,0.191979,0.290388}%
\pgfsetfillcolor{currentfill}%
\pgfsetlinewidth{0.000000pt}%
\definecolor{currentstroke}{rgb}{0.000000,0.000000,0.000000}%
\pgfsetstrokecolor{currentstroke}%
\pgfsetdash{}{0pt}%
\pgfpathmoveto{\pgfqpoint{2.185489in}{1.562087in}}%
\pgfpathlineto{\pgfqpoint{2.390977in}{1.171211in}}%
\pgfpathlineto{\pgfqpoint{2.382112in}{1.589448in}}%
\pgfpathlineto{\pgfqpoint{2.185489in}{1.562087in}}%
\pgfpathclose%
\pgfusepath{fill}%
\end{pgfscope}%
\begin{pgfscope}%
\pgfpathrectangle{\pgfqpoint{0.254231in}{0.147348in}}{\pgfqpoint{2.735294in}{2.735294in}}%
\pgfusepath{clip}%
\pgfsetbuttcap%
\pgfsetroundjoin%
\definecolor{currentfill}{rgb}{0.049941,0.191710,0.289982}%
\pgfsetfillcolor{currentfill}%
\pgfsetlinewidth{0.000000pt}%
\definecolor{currentstroke}{rgb}{0.000000,0.000000,0.000000}%
\pgfsetstrokecolor{currentstroke}%
\pgfsetdash{}{0pt}%
\pgfpathmoveto{\pgfqpoint{2.020098in}{1.063020in}}%
\pgfpathlineto{\pgfqpoint{2.093438in}{0.985051in}}%
\pgfpathlineto{\pgfqpoint{2.061488in}{1.360106in}}%
\pgfpathlineto{\pgfqpoint{2.020098in}{1.063020in}}%
\pgfpathclose%
\pgfusepath{fill}%
\end{pgfscope}%
\begin{pgfscope}%
\pgfpathrectangle{\pgfqpoint{0.254231in}{0.147348in}}{\pgfqpoint{2.735294in}{2.735294in}}%
\pgfusepath{clip}%
\pgfsetbuttcap%
\pgfsetroundjoin%
\definecolor{currentfill}{rgb}{0.049941,0.191710,0.289982}%
\pgfsetfillcolor{currentfill}%
\pgfsetlinewidth{0.000000pt}%
\definecolor{currentstroke}{rgb}{0.000000,0.000000,0.000000}%
\pgfsetstrokecolor{currentstroke}%
\pgfsetdash{}{0pt}%
\pgfpathmoveto{\pgfqpoint{1.256196in}{1.360106in}}%
\pgfpathlineto{\pgfqpoint{1.224246in}{0.985051in}}%
\pgfpathlineto{\pgfqpoint{1.297586in}{1.063020in}}%
\pgfpathlineto{\pgfqpoint{1.256196in}{1.360106in}}%
\pgfpathclose%
\pgfusepath{fill}%
\end{pgfscope}%
\begin{pgfscope}%
\pgfpathrectangle{\pgfqpoint{0.254231in}{0.147348in}}{\pgfqpoint{2.735294in}{2.735294in}}%
\pgfusepath{clip}%
\pgfsetbuttcap%
\pgfsetroundjoin%
\definecolor{currentfill}{rgb}{0.078663,0.301965,0.456754}%
\pgfsetfillcolor{currentfill}%
\pgfsetlinewidth{0.000000pt}%
\definecolor{currentstroke}{rgb}{0.000000,0.000000,0.000000}%
\pgfsetstrokecolor{currentstroke}%
\pgfsetdash{}{0pt}%
\pgfpathmoveto{\pgfqpoint{1.534055in}{2.124855in}}%
\pgfpathlineto{\pgfqpoint{1.223147in}{2.253326in}}%
\pgfpathlineto{\pgfqpoint{1.296827in}{2.116981in}}%
\pgfpathlineto{\pgfqpoint{1.534055in}{2.124855in}}%
\pgfpathclose%
\pgfusepath{fill}%
\end{pgfscope}%
\begin{pgfscope}%
\pgfpathrectangle{\pgfqpoint{0.254231in}{0.147348in}}{\pgfqpoint{2.735294in}{2.735294in}}%
\pgfusepath{clip}%
\pgfsetbuttcap%
\pgfsetroundjoin%
\definecolor{currentfill}{rgb}{0.078663,0.301965,0.456754}%
\pgfsetfillcolor{currentfill}%
\pgfsetlinewidth{0.000000pt}%
\definecolor{currentstroke}{rgb}{0.000000,0.000000,0.000000}%
\pgfsetstrokecolor{currentstroke}%
\pgfsetdash{}{0pt}%
\pgfpathmoveto{\pgfqpoint{2.020857in}{2.116981in}}%
\pgfpathlineto{\pgfqpoint{2.094537in}{2.253326in}}%
\pgfpathlineto{\pgfqpoint{1.783629in}{2.124855in}}%
\pgfpathlineto{\pgfqpoint{2.020857in}{2.116981in}}%
\pgfpathclose%
\pgfusepath{fill}%
\end{pgfscope}%
\begin{pgfscope}%
\pgfpathrectangle{\pgfqpoint{0.254231in}{0.147348in}}{\pgfqpoint{2.735294in}{2.735294in}}%
\pgfusepath{clip}%
\pgfsetbuttcap%
\pgfsetroundjoin%
\definecolor{currentfill}{rgb}{0.064954,0.249341,0.377155}%
\pgfsetfillcolor{currentfill}%
\pgfsetlinewidth{0.000000pt}%
\definecolor{currentstroke}{rgb}{0.000000,0.000000,0.000000}%
\pgfsetstrokecolor{currentstroke}%
\pgfsetdash{}{0pt}%
\pgfpathmoveto{\pgfqpoint{1.132195in}{1.562087in}}%
\pgfpathlineto{\pgfqpoint{1.091293in}{2.103280in}}%
\pgfpathlineto{\pgfqpoint{0.935572in}{1.589448in}}%
\pgfpathlineto{\pgfqpoint{1.132195in}{1.562087in}}%
\pgfpathclose%
\pgfusepath{fill}%
\end{pgfscope}%
\begin{pgfscope}%
\pgfpathrectangle{\pgfqpoint{0.254231in}{0.147348in}}{\pgfqpoint{2.735294in}{2.735294in}}%
\pgfusepath{clip}%
\pgfsetbuttcap%
\pgfsetroundjoin%
\definecolor{currentfill}{rgb}{0.064954,0.249341,0.377155}%
\pgfsetfillcolor{currentfill}%
\pgfsetlinewidth{0.000000pt}%
\definecolor{currentstroke}{rgb}{0.000000,0.000000,0.000000}%
\pgfsetstrokecolor{currentstroke}%
\pgfsetdash{}{0pt}%
\pgfpathmoveto{\pgfqpoint{2.382112in}{1.589448in}}%
\pgfpathlineto{\pgfqpoint{2.226391in}{2.103280in}}%
\pgfpathlineto{\pgfqpoint{2.185489in}{1.562087in}}%
\pgfpathlineto{\pgfqpoint{2.382112in}{1.589448in}}%
\pgfpathclose%
\pgfusepath{fill}%
\end{pgfscope}%
\begin{pgfscope}%
\pgfpathrectangle{\pgfqpoint{0.254231in}{0.147348in}}{\pgfqpoint{2.735294in}{2.735294in}}%
\pgfusepath{clip}%
\pgfsetbuttcap%
\pgfsetroundjoin%
\definecolor{currentfill}{rgb}{0.042669,0.163794,0.247755}%
\pgfsetfillcolor{currentfill}%
\pgfsetlinewidth{0.000000pt}%
\definecolor{currentstroke}{rgb}{0.000000,0.000000,0.000000}%
\pgfsetstrokecolor{currentstroke}%
\pgfsetdash{}{0pt}%
\pgfpathmoveto{\pgfqpoint{1.432345in}{0.946905in}}%
\pgfpathlineto{\pgfqpoint{1.518932in}{1.337788in}}%
\pgfpathlineto{\pgfqpoint{1.297586in}{1.063020in}}%
\pgfpathlineto{\pgfqpoint{1.432345in}{0.946905in}}%
\pgfpathclose%
\pgfusepath{fill}%
\end{pgfscope}%
\begin{pgfscope}%
\pgfpathrectangle{\pgfqpoint{0.254231in}{0.147348in}}{\pgfqpoint{2.735294in}{2.735294in}}%
\pgfusepath{clip}%
\pgfsetbuttcap%
\pgfsetroundjoin%
\definecolor{currentfill}{rgb}{0.042669,0.163794,0.247755}%
\pgfsetfillcolor{currentfill}%
\pgfsetlinewidth{0.000000pt}%
\definecolor{currentstroke}{rgb}{0.000000,0.000000,0.000000}%
\pgfsetstrokecolor{currentstroke}%
\pgfsetdash{}{0pt}%
\pgfpathmoveto{\pgfqpoint{2.020098in}{1.063020in}}%
\pgfpathlineto{\pgfqpoint{1.798752in}{1.337788in}}%
\pgfpathlineto{\pgfqpoint{1.885339in}{0.946905in}}%
\pgfpathlineto{\pgfqpoint{2.020098in}{1.063020in}}%
\pgfpathclose%
\pgfusepath{fill}%
\end{pgfscope}%
\begin{pgfscope}%
\pgfpathrectangle{\pgfqpoint{0.254231in}{0.147348in}}{\pgfqpoint{2.735294in}{2.735294in}}%
\pgfusepath{clip}%
\pgfsetbuttcap%
\pgfsetroundjoin%
\definecolor{currentfill}{rgb}{0.068541,0.263111,0.397982}%
\pgfsetfillcolor{currentfill}%
\pgfsetlinewidth{0.000000pt}%
\definecolor{currentstroke}{rgb}{0.000000,0.000000,0.000000}%
\pgfsetstrokecolor{currentstroke}%
\pgfsetdash{}{0pt}%
\pgfpathmoveto{\pgfqpoint{1.296827in}{2.116981in}}%
\pgfpathlineto{\pgfqpoint{1.091293in}{2.103280in}}%
\pgfpathlineto{\pgfqpoint{1.395317in}{1.946079in}}%
\pgfpathlineto{\pgfqpoint{1.296827in}{2.116981in}}%
\pgfpathclose%
\pgfusepath{fill}%
\end{pgfscope}%
\begin{pgfscope}%
\pgfpathrectangle{\pgfqpoint{0.254231in}{0.147348in}}{\pgfqpoint{2.735294in}{2.735294in}}%
\pgfusepath{clip}%
\pgfsetbuttcap%
\pgfsetroundjoin%
\definecolor{currentfill}{rgb}{0.068541,0.263111,0.397982}%
\pgfsetfillcolor{currentfill}%
\pgfsetlinewidth{0.000000pt}%
\definecolor{currentstroke}{rgb}{0.000000,0.000000,0.000000}%
\pgfsetstrokecolor{currentstroke}%
\pgfsetdash{}{0pt}%
\pgfpathmoveto{\pgfqpoint{1.922367in}{1.946079in}}%
\pgfpathlineto{\pgfqpoint{2.226391in}{2.103280in}}%
\pgfpathlineto{\pgfqpoint{2.020857in}{2.116981in}}%
\pgfpathlineto{\pgfqpoint{1.922367in}{1.946079in}}%
\pgfpathclose%
\pgfusepath{fill}%
\end{pgfscope}%
\begin{pgfscope}%
\pgfpathrectangle{\pgfqpoint{0.254231in}{0.147348in}}{\pgfqpoint{2.735294in}{2.735294in}}%
\pgfusepath{clip}%
\pgfsetbuttcap%
\pgfsetroundjoin%
\definecolor{currentfill}{rgb}{0.047555,0.182548,0.276123}%
\pgfsetfillcolor{currentfill}%
\pgfsetlinewidth{0.000000pt}%
\definecolor{currentstroke}{rgb}{0.000000,0.000000,0.000000}%
\pgfsetstrokecolor{currentstroke}%
\pgfsetdash{}{0pt}%
\pgfpathmoveto{\pgfqpoint{1.432345in}{0.946905in}}%
\pgfpathlineto{\pgfqpoint{1.534325in}{1.035006in}}%
\pgfpathlineto{\pgfqpoint{1.518932in}{1.337788in}}%
\pgfpathlineto{\pgfqpoint{1.432345in}{0.946905in}}%
\pgfpathclose%
\pgfusepath{fill}%
\end{pgfscope}%
\begin{pgfscope}%
\pgfpathrectangle{\pgfqpoint{0.254231in}{0.147348in}}{\pgfqpoint{2.735294in}{2.735294in}}%
\pgfusepath{clip}%
\pgfsetbuttcap%
\pgfsetroundjoin%
\definecolor{currentfill}{rgb}{0.047555,0.182548,0.276123}%
\pgfsetfillcolor{currentfill}%
\pgfsetlinewidth{0.000000pt}%
\definecolor{currentstroke}{rgb}{0.000000,0.000000,0.000000}%
\pgfsetstrokecolor{currentstroke}%
\pgfsetdash{}{0pt}%
\pgfpathmoveto{\pgfqpoint{1.798752in}{1.337788in}}%
\pgfpathlineto{\pgfqpoint{1.783359in}{1.035006in}}%
\pgfpathlineto{\pgfqpoint{1.885339in}{0.946905in}}%
\pgfpathlineto{\pgfqpoint{1.798752in}{1.337788in}}%
\pgfpathclose%
\pgfusepath{fill}%
\end{pgfscope}%
\begin{pgfscope}%
\pgfpathrectangle{\pgfqpoint{0.254231in}{0.147348in}}{\pgfqpoint{2.735294in}{2.735294in}}%
\pgfusepath{clip}%
\pgfsetbuttcap%
\pgfsetroundjoin%
\definecolor{currentfill}{rgb}{0.046101,0.176968,0.267683}%
\pgfsetfillcolor{currentfill}%
\pgfsetlinewidth{0.000000pt}%
\definecolor{currentstroke}{rgb}{0.000000,0.000000,0.000000}%
\pgfsetstrokecolor{currentstroke}%
\pgfsetdash{}{0pt}%
\pgfpathmoveto{\pgfqpoint{1.658842in}{0.933095in}}%
\pgfpathlineto{\pgfqpoint{1.518932in}{1.337788in}}%
\pgfpathlineto{\pgfqpoint{1.534325in}{1.035006in}}%
\pgfpathlineto{\pgfqpoint{1.658842in}{0.933095in}}%
\pgfpathclose%
\pgfusepath{fill}%
\end{pgfscope}%
\begin{pgfscope}%
\pgfpathrectangle{\pgfqpoint{0.254231in}{0.147348in}}{\pgfqpoint{2.735294in}{2.735294in}}%
\pgfusepath{clip}%
\pgfsetbuttcap%
\pgfsetroundjoin%
\definecolor{currentfill}{rgb}{0.046101,0.176968,0.267683}%
\pgfsetfillcolor{currentfill}%
\pgfsetlinewidth{0.000000pt}%
\definecolor{currentstroke}{rgb}{0.000000,0.000000,0.000000}%
\pgfsetstrokecolor{currentstroke}%
\pgfsetdash{}{0pt}%
\pgfpathmoveto{\pgfqpoint{1.783359in}{1.035006in}}%
\pgfpathlineto{\pgfqpoint{1.798752in}{1.337788in}}%
\pgfpathlineto{\pgfqpoint{1.658842in}{0.933095in}}%
\pgfpathlineto{\pgfqpoint{1.783359in}{1.035006in}}%
\pgfpathclose%
\pgfusepath{fill}%
\end{pgfscope}%
\begin{pgfscope}%
\pgfpathrectangle{\pgfqpoint{0.254231in}{0.147348in}}{\pgfqpoint{2.735294in}{2.735294in}}%
\pgfusepath{clip}%
\pgfsetbuttcap%
\pgfsetroundjoin%
\definecolor{currentfill}{rgb}{0.051850,0.199036,0.301063}%
\pgfsetfillcolor{currentfill}%
\pgfsetlinewidth{0.000000pt}%
\definecolor{currentstroke}{rgb}{0.000000,0.000000,0.000000}%
\pgfsetstrokecolor{currentstroke}%
\pgfsetdash{}{0pt}%
\pgfpathmoveto{\pgfqpoint{1.092413in}{1.111775in}}%
\pgfpathlineto{\pgfqpoint{1.256196in}{1.360106in}}%
\pgfpathlineto{\pgfqpoint{1.132195in}{1.562087in}}%
\pgfpathlineto{\pgfqpoint{1.092413in}{1.111775in}}%
\pgfpathclose%
\pgfusepath{fill}%
\end{pgfscope}%
\begin{pgfscope}%
\pgfpathrectangle{\pgfqpoint{0.254231in}{0.147348in}}{\pgfqpoint{2.735294in}{2.735294in}}%
\pgfusepath{clip}%
\pgfsetbuttcap%
\pgfsetroundjoin%
\definecolor{currentfill}{rgb}{0.051850,0.199036,0.301063}%
\pgfsetfillcolor{currentfill}%
\pgfsetlinewidth{0.000000pt}%
\definecolor{currentstroke}{rgb}{0.000000,0.000000,0.000000}%
\pgfsetstrokecolor{currentstroke}%
\pgfsetdash{}{0pt}%
\pgfpathmoveto{\pgfqpoint{2.185489in}{1.562087in}}%
\pgfpathlineto{\pgfqpoint{2.061488in}{1.360106in}}%
\pgfpathlineto{\pgfqpoint{2.225271in}{1.111775in}}%
\pgfpathlineto{\pgfqpoint{2.185489in}{1.562087in}}%
\pgfpathclose%
\pgfusepath{fill}%
\end{pgfscope}%
\begin{pgfscope}%
\pgfpathrectangle{\pgfqpoint{0.254231in}{0.147348in}}{\pgfqpoint{2.735294in}{2.735294in}}%
\pgfusepath{clip}%
\pgfsetbuttcap%
\pgfsetroundjoin%
\definecolor{currentfill}{rgb}{0.064759,0.248590,0.376018}%
\pgfsetfillcolor{currentfill}%
\pgfsetlinewidth{0.000000pt}%
\definecolor{currentstroke}{rgb}{0.000000,0.000000,0.000000}%
\pgfsetstrokecolor{currentstroke}%
\pgfsetdash{}{0pt}%
\pgfpathmoveto{\pgfqpoint{2.226391in}{2.103280in}}%
\pgfpathlineto{\pgfqpoint{1.922367in}{1.946079in}}%
\pgfpathlineto{\pgfqpoint{2.185489in}{1.562087in}}%
\pgfpathlineto{\pgfqpoint{2.226391in}{2.103280in}}%
\pgfpathclose%
\pgfusepath{fill}%
\end{pgfscope}%
\begin{pgfscope}%
\pgfpathrectangle{\pgfqpoint{0.254231in}{0.147348in}}{\pgfqpoint{2.735294in}{2.735294in}}%
\pgfusepath{clip}%
\pgfsetbuttcap%
\pgfsetroundjoin%
\definecolor{currentfill}{rgb}{0.064759,0.248590,0.376018}%
\pgfsetfillcolor{currentfill}%
\pgfsetlinewidth{0.000000pt}%
\definecolor{currentstroke}{rgb}{0.000000,0.000000,0.000000}%
\pgfsetstrokecolor{currentstroke}%
\pgfsetdash{}{0pt}%
\pgfpathmoveto{\pgfqpoint{1.132195in}{1.562087in}}%
\pgfpathlineto{\pgfqpoint{1.395317in}{1.946079in}}%
\pgfpathlineto{\pgfqpoint{1.091293in}{2.103280in}}%
\pgfpathlineto{\pgfqpoint{1.132195in}{1.562087in}}%
\pgfpathclose%
\pgfusepath{fill}%
\end{pgfscope}%
\begin{pgfscope}%
\pgfpathrectangle{\pgfqpoint{0.254231in}{0.147348in}}{\pgfqpoint{2.735294in}{2.735294in}}%
\pgfusepath{clip}%
\pgfsetbuttcap%
\pgfsetroundjoin%
\definecolor{currentfill}{rgb}{0.071694,0.275212,0.416288}%
\pgfsetfillcolor{currentfill}%
\pgfsetlinewidth{0.000000pt}%
\definecolor{currentstroke}{rgb}{0.000000,0.000000,0.000000}%
\pgfsetstrokecolor{currentstroke}%
\pgfsetdash{}{0pt}%
\pgfpathmoveto{\pgfqpoint{1.395317in}{1.946079in}}%
\pgfpathlineto{\pgfqpoint{1.534055in}{2.124855in}}%
\pgfpathlineto{\pgfqpoint{1.296827in}{2.116981in}}%
\pgfpathlineto{\pgfqpoint{1.395317in}{1.946079in}}%
\pgfpathclose%
\pgfusepath{fill}%
\end{pgfscope}%
\begin{pgfscope}%
\pgfpathrectangle{\pgfqpoint{0.254231in}{0.147348in}}{\pgfqpoint{2.735294in}{2.735294in}}%
\pgfusepath{clip}%
\pgfsetbuttcap%
\pgfsetroundjoin%
\definecolor{currentfill}{rgb}{0.071694,0.275212,0.416288}%
\pgfsetfillcolor{currentfill}%
\pgfsetlinewidth{0.000000pt}%
\definecolor{currentstroke}{rgb}{0.000000,0.000000,0.000000}%
\pgfsetstrokecolor{currentstroke}%
\pgfsetdash{}{0pt}%
\pgfpathmoveto{\pgfqpoint{2.020857in}{2.116981in}}%
\pgfpathlineto{\pgfqpoint{1.783629in}{2.124855in}}%
\pgfpathlineto{\pgfqpoint{1.922367in}{1.946079in}}%
\pgfpathlineto{\pgfqpoint{2.020857in}{2.116981in}}%
\pgfpathclose%
\pgfusepath{fill}%
\end{pgfscope}%
\begin{pgfscope}%
\pgfpathrectangle{\pgfqpoint{0.254231in}{0.147348in}}{\pgfqpoint{2.735294in}{2.735294in}}%
\pgfusepath{clip}%
\pgfsetbuttcap%
\pgfsetroundjoin%
\definecolor{currentfill}{rgb}{0.071636,0.274990,0.415951}%
\pgfsetfillcolor{currentfill}%
\pgfsetlinewidth{0.000000pt}%
\definecolor{currentstroke}{rgb}{0.000000,0.000000,0.000000}%
\pgfsetstrokecolor{currentstroke}%
\pgfsetdash{}{0pt}%
\pgfpathmoveto{\pgfqpoint{1.658842in}{1.947266in}}%
\pgfpathlineto{\pgfqpoint{1.783629in}{2.124855in}}%
\pgfpathlineto{\pgfqpoint{1.534055in}{2.124855in}}%
\pgfpathlineto{\pgfqpoint{1.658842in}{1.947266in}}%
\pgfpathclose%
\pgfusepath{fill}%
\end{pgfscope}%
\begin{pgfscope}%
\pgfpathrectangle{\pgfqpoint{0.254231in}{0.147348in}}{\pgfqpoint{2.735294in}{2.735294in}}%
\pgfusepath{clip}%
\pgfsetbuttcap%
\pgfsetroundjoin%
\definecolor{currentfill}{rgb}{0.045820,0.175891,0.266053}%
\pgfsetfillcolor{currentfill}%
\pgfsetlinewidth{0.000000pt}%
\definecolor{currentstroke}{rgb}{0.000000,0.000000,0.000000}%
\pgfsetstrokecolor{currentstroke}%
\pgfsetdash{}{0pt}%
\pgfpathmoveto{\pgfqpoint{1.297586in}{1.063020in}}%
\pgfpathlineto{\pgfqpoint{1.378909in}{1.541647in}}%
\pgfpathlineto{\pgfqpoint{1.256196in}{1.360106in}}%
\pgfpathlineto{\pgfqpoint{1.297586in}{1.063020in}}%
\pgfpathclose%
\pgfusepath{fill}%
\end{pgfscope}%
\begin{pgfscope}%
\pgfpathrectangle{\pgfqpoint{0.254231in}{0.147348in}}{\pgfqpoint{2.735294in}{2.735294in}}%
\pgfusepath{clip}%
\pgfsetbuttcap%
\pgfsetroundjoin%
\definecolor{currentfill}{rgb}{0.045820,0.175891,0.266053}%
\pgfsetfillcolor{currentfill}%
\pgfsetlinewidth{0.000000pt}%
\definecolor{currentstroke}{rgb}{0.000000,0.000000,0.000000}%
\pgfsetstrokecolor{currentstroke}%
\pgfsetdash{}{0pt}%
\pgfpathmoveto{\pgfqpoint{2.061488in}{1.360106in}}%
\pgfpathlineto{\pgfqpoint{1.938775in}{1.541647in}}%
\pgfpathlineto{\pgfqpoint{2.020098in}{1.063020in}}%
\pgfpathlineto{\pgfqpoint{2.061488in}{1.360106in}}%
\pgfpathclose%
\pgfusepath{fill}%
\end{pgfscope}%
\begin{pgfscope}%
\pgfpathrectangle{\pgfqpoint{0.254231in}{0.147348in}}{\pgfqpoint{2.735294in}{2.735294in}}%
\pgfusepath{clip}%
\pgfsetbuttcap%
\pgfsetroundjoin%
\definecolor{currentfill}{rgb}{0.046814,0.179706,0.271825}%
\pgfsetfillcolor{currentfill}%
\pgfsetlinewidth{0.000000pt}%
\definecolor{currentstroke}{rgb}{0.000000,0.000000,0.000000}%
\pgfsetstrokecolor{currentstroke}%
\pgfsetdash{}{0pt}%
\pgfpathmoveto{\pgfqpoint{1.658842in}{1.533948in}}%
\pgfpathlineto{\pgfqpoint{1.658842in}{0.933095in}}%
\pgfpathlineto{\pgfqpoint{1.798752in}{1.337788in}}%
\pgfpathlineto{\pgfqpoint{1.658842in}{1.533948in}}%
\pgfpathclose%
\pgfusepath{fill}%
\end{pgfscope}%
\begin{pgfscope}%
\pgfpathrectangle{\pgfqpoint{0.254231in}{0.147348in}}{\pgfqpoint{2.735294in}{2.735294in}}%
\pgfusepath{clip}%
\pgfsetbuttcap%
\pgfsetroundjoin%
\definecolor{currentfill}{rgb}{0.046814,0.179706,0.271825}%
\pgfsetfillcolor{currentfill}%
\pgfsetlinewidth{0.000000pt}%
\definecolor{currentstroke}{rgb}{0.000000,0.000000,0.000000}%
\pgfsetstrokecolor{currentstroke}%
\pgfsetdash{}{0pt}%
\pgfpathmoveto{\pgfqpoint{1.518932in}{1.337788in}}%
\pgfpathlineto{\pgfqpoint{1.658842in}{0.933095in}}%
\pgfpathlineto{\pgfqpoint{1.658842in}{1.533948in}}%
\pgfpathlineto{\pgfqpoint{1.518932in}{1.337788in}}%
\pgfpathclose%
\pgfusepath{fill}%
\end{pgfscope}%
\begin{pgfscope}%
\pgfpathrectangle{\pgfqpoint{0.254231in}{0.147348in}}{\pgfqpoint{2.735294in}{2.735294in}}%
\pgfusepath{clip}%
\pgfsetbuttcap%
\pgfsetroundjoin%
\definecolor{currentfill}{rgb}{0.069261,0.265872,0.402159}%
\pgfsetfillcolor{currentfill}%
\pgfsetlinewidth{0.000000pt}%
\definecolor{currentstroke}{rgb}{0.000000,0.000000,0.000000}%
\pgfsetstrokecolor{currentstroke}%
\pgfsetdash{}{0pt}%
\pgfpathmoveto{\pgfqpoint{1.922367in}{1.946079in}}%
\pgfpathlineto{\pgfqpoint{1.783629in}{2.124855in}}%
\pgfpathlineto{\pgfqpoint{1.658842in}{1.947266in}}%
\pgfpathlineto{\pgfqpoint{1.922367in}{1.946079in}}%
\pgfpathclose%
\pgfusepath{fill}%
\end{pgfscope}%
\begin{pgfscope}%
\pgfpathrectangle{\pgfqpoint{0.254231in}{0.147348in}}{\pgfqpoint{2.735294in}{2.735294in}}%
\pgfusepath{clip}%
\pgfsetbuttcap%
\pgfsetroundjoin%
\definecolor{currentfill}{rgb}{0.069261,0.265872,0.402159}%
\pgfsetfillcolor{currentfill}%
\pgfsetlinewidth{0.000000pt}%
\definecolor{currentstroke}{rgb}{0.000000,0.000000,0.000000}%
\pgfsetstrokecolor{currentstroke}%
\pgfsetdash{}{0pt}%
\pgfpathmoveto{\pgfqpoint{1.658842in}{1.947266in}}%
\pgfpathlineto{\pgfqpoint{1.534055in}{2.124855in}}%
\pgfpathlineto{\pgfqpoint{1.395317in}{1.946079in}}%
\pgfpathlineto{\pgfqpoint{1.658842in}{1.947266in}}%
\pgfpathclose%
\pgfusepath{fill}%
\end{pgfscope}%
\begin{pgfscope}%
\pgfpathrectangle{\pgfqpoint{0.254231in}{0.147348in}}{\pgfqpoint{2.735294in}{2.735294in}}%
\pgfusepath{clip}%
\pgfsetbuttcap%
\pgfsetroundjoin%
\definecolor{currentfill}{rgb}{0.049465,0.189883,0.287218}%
\pgfsetfillcolor{currentfill}%
\pgfsetlinewidth{0.000000pt}%
\definecolor{currentstroke}{rgb}{0.000000,0.000000,0.000000}%
\pgfsetstrokecolor{currentstroke}%
\pgfsetdash{}{0pt}%
\pgfpathmoveto{\pgfqpoint{1.297586in}{1.063020in}}%
\pgfpathlineto{\pgfqpoint{1.518932in}{1.337788in}}%
\pgfpathlineto{\pgfqpoint{1.378909in}{1.541647in}}%
\pgfpathlineto{\pgfqpoint{1.297586in}{1.063020in}}%
\pgfpathclose%
\pgfusepath{fill}%
\end{pgfscope}%
\begin{pgfscope}%
\pgfpathrectangle{\pgfqpoint{0.254231in}{0.147348in}}{\pgfqpoint{2.735294in}{2.735294in}}%
\pgfusepath{clip}%
\pgfsetbuttcap%
\pgfsetroundjoin%
\definecolor{currentfill}{rgb}{0.049465,0.189883,0.287218}%
\pgfsetfillcolor{currentfill}%
\pgfsetlinewidth{0.000000pt}%
\definecolor{currentstroke}{rgb}{0.000000,0.000000,0.000000}%
\pgfsetstrokecolor{currentstroke}%
\pgfsetdash{}{0pt}%
\pgfpathmoveto{\pgfqpoint{1.938775in}{1.541647in}}%
\pgfpathlineto{\pgfqpoint{1.798752in}{1.337788in}}%
\pgfpathlineto{\pgfqpoint{2.020098in}{1.063020in}}%
\pgfpathlineto{\pgfqpoint{1.938775in}{1.541647in}}%
\pgfpathclose%
\pgfusepath{fill}%
\end{pgfscope}%
\begin{pgfscope}%
\pgfpathrectangle{\pgfqpoint{0.254231in}{0.147348in}}{\pgfqpoint{2.735294in}{2.735294in}}%
\pgfusepath{clip}%
\pgfsetbuttcap%
\pgfsetroundjoin%
\definecolor{currentfill}{rgb}{0.061576,0.236373,0.357539}%
\pgfsetfillcolor{currentfill}%
\pgfsetlinewidth{0.000000pt}%
\definecolor{currentstroke}{rgb}{0.000000,0.000000,0.000000}%
\pgfsetstrokecolor{currentstroke}%
\pgfsetdash{}{0pt}%
\pgfpathmoveto{\pgfqpoint{1.378909in}{1.541647in}}%
\pgfpathlineto{\pgfqpoint{1.395317in}{1.946079in}}%
\pgfpathlineto{\pgfqpoint{1.132195in}{1.562087in}}%
\pgfpathlineto{\pgfqpoint{1.378909in}{1.541647in}}%
\pgfpathclose%
\pgfusepath{fill}%
\end{pgfscope}%
\begin{pgfscope}%
\pgfpathrectangle{\pgfqpoint{0.254231in}{0.147348in}}{\pgfqpoint{2.735294in}{2.735294in}}%
\pgfusepath{clip}%
\pgfsetbuttcap%
\pgfsetroundjoin%
\definecolor{currentfill}{rgb}{0.061576,0.236373,0.357539}%
\pgfsetfillcolor{currentfill}%
\pgfsetlinewidth{0.000000pt}%
\definecolor{currentstroke}{rgb}{0.000000,0.000000,0.000000}%
\pgfsetstrokecolor{currentstroke}%
\pgfsetdash{}{0pt}%
\pgfpathmoveto{\pgfqpoint{2.185489in}{1.562087in}}%
\pgfpathlineto{\pgfqpoint{1.922367in}{1.946079in}}%
\pgfpathlineto{\pgfqpoint{1.938775in}{1.541647in}}%
\pgfpathlineto{\pgfqpoint{2.185489in}{1.562087in}}%
\pgfpathclose%
\pgfusepath{fill}%
\end{pgfscope}%
\begin{pgfscope}%
\pgfpathrectangle{\pgfqpoint{0.254231in}{0.147348in}}{\pgfqpoint{2.735294in}{2.735294in}}%
\pgfusepath{clip}%
\pgfsetbuttcap%
\pgfsetroundjoin%
\definecolor{currentfill}{rgb}{0.053541,0.205528,0.310883}%
\pgfsetfillcolor{currentfill}%
\pgfsetlinewidth{0.000000pt}%
\definecolor{currentstroke}{rgb}{0.000000,0.000000,0.000000}%
\pgfsetstrokecolor{currentstroke}%
\pgfsetdash{}{0pt}%
\pgfpathmoveto{\pgfqpoint{1.132195in}{1.562087in}}%
\pgfpathlineto{\pgfqpoint{1.256196in}{1.360106in}}%
\pgfpathlineto{\pgfqpoint{1.378909in}{1.541647in}}%
\pgfpathlineto{\pgfqpoint{1.132195in}{1.562087in}}%
\pgfpathclose%
\pgfusepath{fill}%
\end{pgfscope}%
\begin{pgfscope}%
\pgfpathrectangle{\pgfqpoint{0.254231in}{0.147348in}}{\pgfqpoint{2.735294in}{2.735294in}}%
\pgfusepath{clip}%
\pgfsetbuttcap%
\pgfsetroundjoin%
\definecolor{currentfill}{rgb}{0.053541,0.205528,0.310883}%
\pgfsetfillcolor{currentfill}%
\pgfsetlinewidth{0.000000pt}%
\definecolor{currentstroke}{rgb}{0.000000,0.000000,0.000000}%
\pgfsetstrokecolor{currentstroke}%
\pgfsetdash{}{0pt}%
\pgfpathmoveto{\pgfqpoint{1.938775in}{1.541647in}}%
\pgfpathlineto{\pgfqpoint{2.061488in}{1.360106in}}%
\pgfpathlineto{\pgfqpoint{2.185489in}{1.562087in}}%
\pgfpathlineto{\pgfqpoint{1.938775in}{1.541647in}}%
\pgfpathclose%
\pgfusepath{fill}%
\end{pgfscope}%
\begin{pgfscope}%
\pgfpathrectangle{\pgfqpoint{0.254231in}{0.147348in}}{\pgfqpoint{2.735294in}{2.735294in}}%
\pgfusepath{clip}%
\pgfsetbuttcap%
\pgfsetroundjoin%
\definecolor{currentfill}{rgb}{0.060634,0.232757,0.352069}%
\pgfsetfillcolor{currentfill}%
\pgfsetlinewidth{0.000000pt}%
\definecolor{currentstroke}{rgb}{0.000000,0.000000,0.000000}%
\pgfsetstrokecolor{currentstroke}%
\pgfsetdash{}{0pt}%
\pgfpathmoveto{\pgfqpoint{1.395317in}{1.946079in}}%
\pgfpathlineto{\pgfqpoint{1.378909in}{1.541647in}}%
\pgfpathlineto{\pgfqpoint{1.658842in}{1.947266in}}%
\pgfpathlineto{\pgfqpoint{1.395317in}{1.946079in}}%
\pgfpathclose%
\pgfusepath{fill}%
\end{pgfscope}%
\begin{pgfscope}%
\pgfpathrectangle{\pgfqpoint{0.254231in}{0.147348in}}{\pgfqpoint{2.735294in}{2.735294in}}%
\pgfusepath{clip}%
\pgfsetbuttcap%
\pgfsetroundjoin%
\definecolor{currentfill}{rgb}{0.060634,0.232757,0.352069}%
\pgfsetfillcolor{currentfill}%
\pgfsetlinewidth{0.000000pt}%
\definecolor{currentstroke}{rgb}{0.000000,0.000000,0.000000}%
\pgfsetstrokecolor{currentstroke}%
\pgfsetdash{}{0pt}%
\pgfpathmoveto{\pgfqpoint{1.658842in}{1.947266in}}%
\pgfpathlineto{\pgfqpoint{1.938775in}{1.541647in}}%
\pgfpathlineto{\pgfqpoint{1.922367in}{1.946079in}}%
\pgfpathlineto{\pgfqpoint{1.658842in}{1.947266in}}%
\pgfpathclose%
\pgfusepath{fill}%
\end{pgfscope}%
\begin{pgfscope}%
\pgfpathrectangle{\pgfqpoint{0.254231in}{0.147348in}}{\pgfqpoint{2.735294in}{2.735294in}}%
\pgfusepath{clip}%
\pgfsetbuttcap%
\pgfsetroundjoin%
\definecolor{currentfill}{rgb}{0.060773,0.233289,0.352874}%
\pgfsetfillcolor{currentfill}%
\pgfsetlinewidth{0.000000pt}%
\definecolor{currentstroke}{rgb}{0.000000,0.000000,0.000000}%
\pgfsetstrokecolor{currentstroke}%
\pgfsetdash{}{0pt}%
\pgfpathmoveto{\pgfqpoint{1.658842in}{1.533948in}}%
\pgfpathlineto{\pgfqpoint{1.658842in}{1.947266in}}%
\pgfpathlineto{\pgfqpoint{1.378909in}{1.541647in}}%
\pgfpathlineto{\pgfqpoint{1.658842in}{1.533948in}}%
\pgfpathclose%
\pgfusepath{fill}%
\end{pgfscope}%
\begin{pgfscope}%
\pgfpathrectangle{\pgfqpoint{0.254231in}{0.147348in}}{\pgfqpoint{2.735294in}{2.735294in}}%
\pgfusepath{clip}%
\pgfsetbuttcap%
\pgfsetroundjoin%
\definecolor{currentfill}{rgb}{0.060773,0.233289,0.352874}%
\pgfsetfillcolor{currentfill}%
\pgfsetlinewidth{0.000000pt}%
\definecolor{currentstroke}{rgb}{0.000000,0.000000,0.000000}%
\pgfsetstrokecolor{currentstroke}%
\pgfsetdash{}{0pt}%
\pgfpathmoveto{\pgfqpoint{1.938775in}{1.541647in}}%
\pgfpathlineto{\pgfqpoint{1.658842in}{1.947266in}}%
\pgfpathlineto{\pgfqpoint{1.658842in}{1.533948in}}%
\pgfpathlineto{\pgfqpoint{1.938775in}{1.541647in}}%
\pgfpathclose%
\pgfusepath{fill}%
\end{pgfscope}%
\begin{pgfscope}%
\pgfpathrectangle{\pgfqpoint{0.254231in}{0.147348in}}{\pgfqpoint{2.735294in}{2.735294in}}%
\pgfusepath{clip}%
\pgfsetbuttcap%
\pgfsetroundjoin%
\definecolor{currentfill}{rgb}{0.052607,0.201942,0.305459}%
\pgfsetfillcolor{currentfill}%
\pgfsetlinewidth{0.000000pt}%
\definecolor{currentstroke}{rgb}{0.000000,0.000000,0.000000}%
\pgfsetstrokecolor{currentstroke}%
\pgfsetdash{}{0pt}%
\pgfpathmoveto{\pgfqpoint{1.378909in}{1.541647in}}%
\pgfpathlineto{\pgfqpoint{1.518932in}{1.337788in}}%
\pgfpathlineto{\pgfqpoint{1.658842in}{1.533948in}}%
\pgfpathlineto{\pgfqpoint{1.378909in}{1.541647in}}%
\pgfpathclose%
\pgfusepath{fill}%
\end{pgfscope}%
\begin{pgfscope}%
\pgfpathrectangle{\pgfqpoint{0.254231in}{0.147348in}}{\pgfqpoint{2.735294in}{2.735294in}}%
\pgfusepath{clip}%
\pgfsetbuttcap%
\pgfsetroundjoin%
\definecolor{currentfill}{rgb}{0.052607,0.201942,0.305459}%
\pgfsetfillcolor{currentfill}%
\pgfsetlinewidth{0.000000pt}%
\definecolor{currentstroke}{rgb}{0.000000,0.000000,0.000000}%
\pgfsetstrokecolor{currentstroke}%
\pgfsetdash{}{0pt}%
\pgfpathmoveto{\pgfqpoint{1.658842in}{1.533948in}}%
\pgfpathlineto{\pgfqpoint{1.798752in}{1.337788in}}%
\pgfpathlineto{\pgfqpoint{1.938775in}{1.541647in}}%
\pgfpathlineto{\pgfqpoint{1.658842in}{1.533948in}}%
\pgfpathclose%
\pgfusepath{fill}%
\end{pgfscope}%
\begin{pgfscope}%
\pgfpathrectangle{\pgfqpoint{0.254231in}{0.147348in}}{\pgfqpoint{2.735294in}{2.735294in}}%
\pgfusepath{clip}%
\pgfsetbuttcap%
\pgfsetroundjoin%
\definecolor{currentfill}{rgb}{0.839216,0.152941,0.156863}%
\pgfsetfillcolor{currentfill}%
\pgfsetfillopacity{0.300000}%
\pgfsetlinewidth{1.003750pt}%
\definecolor{currentstroke}{rgb}{0.839216,0.152941,0.156863}%
\pgfsetstrokecolor{currentstroke}%
\pgfsetstrokeopacity{0.300000}%
\pgfsetdash{}{0pt}%
\pgfpathmoveto{\pgfqpoint{1.065369in}{0.944929in}}%
\pgfpathcurveto{\pgfqpoint{1.075457in}{0.944929in}}{\pgfqpoint{1.085132in}{0.948937in}}{\pgfqpoint{1.092265in}{0.956070in}}%
\pgfpathcurveto{\pgfqpoint{1.099398in}{0.963203in}}{\pgfqpoint{1.103406in}{0.972878in}}{\pgfqpoint{1.103406in}{0.982966in}}%
\pgfpathcurveto{\pgfqpoint{1.103406in}{0.993053in}}{\pgfqpoint{1.099398in}{1.002729in}}{\pgfqpoint{1.092265in}{1.009861in}}%
\pgfpathcurveto{\pgfqpoint{1.085132in}{1.016994in}}{\pgfqpoint{1.075457in}{1.021002in}}{\pgfqpoint{1.065369in}{1.021002in}}%
\pgfpathcurveto{\pgfqpoint{1.055282in}{1.021002in}}{\pgfqpoint{1.045607in}{1.016994in}}{\pgfqpoint{1.038474in}{1.009861in}}%
\pgfpathcurveto{\pgfqpoint{1.031341in}{1.002729in}}{\pgfqpoint{1.027333in}{0.993053in}}{\pgfqpoint{1.027333in}{0.982966in}}%
\pgfpathcurveto{\pgfqpoint{1.027333in}{0.972878in}}{\pgfqpoint{1.031341in}{0.963203in}}{\pgfqpoint{1.038474in}{0.956070in}}%
\pgfpathcurveto{\pgfqpoint{1.045607in}{0.948937in}}{\pgfqpoint{1.055282in}{0.944929in}}{\pgfqpoint{1.065369in}{0.944929in}}%
\pgfpathlineto{\pgfqpoint{1.065369in}{0.944929in}}%
\pgfpathclose%
\pgfusepath{stroke,fill}%
\end{pgfscope}%
\begin{pgfscope}%
\pgfpathrectangle{\pgfqpoint{0.254231in}{0.147348in}}{\pgfqpoint{2.735294in}{2.735294in}}%
\pgfusepath{clip}%
\pgfsetbuttcap%
\pgfsetroundjoin%
\definecolor{currentfill}{rgb}{0.839216,0.152941,0.156863}%
\pgfsetfillcolor{currentfill}%
\pgfsetfillopacity{0.383610}%
\pgfsetlinewidth{1.003750pt}%
\definecolor{currentstroke}{rgb}{0.839216,0.152941,0.156863}%
\pgfsetstrokecolor{currentstroke}%
\pgfsetstrokeopacity{0.383610}%
\pgfsetdash{}{0pt}%
\pgfpathmoveto{\pgfqpoint{1.045506in}{1.009596in}}%
\pgfpathcurveto{\pgfqpoint{1.055594in}{1.009596in}}{\pgfqpoint{1.065269in}{1.013604in}}{\pgfqpoint{1.072402in}{1.020737in}}%
\pgfpathcurveto{\pgfqpoint{1.079535in}{1.027870in}}{\pgfqpoint{1.083543in}{1.037545in}}{\pgfqpoint{1.083543in}{1.047633in}}%
\pgfpathcurveto{\pgfqpoint{1.083543in}{1.057720in}}{\pgfqpoint{1.079535in}{1.067396in}}{\pgfqpoint{1.072402in}{1.074528in}}%
\pgfpathcurveto{\pgfqpoint{1.065269in}{1.081661in}}{\pgfqpoint{1.055594in}{1.085669in}}{\pgfqpoint{1.045506in}{1.085669in}}%
\pgfpathcurveto{\pgfqpoint{1.035419in}{1.085669in}}{\pgfqpoint{1.025743in}{1.081661in}}{\pgfqpoint{1.018610in}{1.074528in}}%
\pgfpathcurveto{\pgfqpoint{1.011478in}{1.067396in}}{\pgfqpoint{1.007470in}{1.057720in}}{\pgfqpoint{1.007470in}{1.047633in}}%
\pgfpathcurveto{\pgfqpoint{1.007470in}{1.037545in}}{\pgfqpoint{1.011478in}{1.027870in}}{\pgfqpoint{1.018610in}{1.020737in}}%
\pgfpathcurveto{\pgfqpoint{1.025743in}{1.013604in}}{\pgfqpoint{1.035419in}{1.009596in}}{\pgfqpoint{1.045506in}{1.009596in}}%
\pgfpathlineto{\pgfqpoint{1.045506in}{1.009596in}}%
\pgfpathclose%
\pgfusepath{stroke,fill}%
\end{pgfscope}%
\begin{pgfscope}%
\pgfpathrectangle{\pgfqpoint{0.254231in}{0.147348in}}{\pgfqpoint{2.735294in}{2.735294in}}%
\pgfusepath{clip}%
\pgfsetbuttcap%
\pgfsetroundjoin%
\definecolor{currentfill}{rgb}{0.839216,0.152941,0.156863}%
\pgfsetfillcolor{currentfill}%
\pgfsetfillopacity{0.457533}%
\pgfsetlinewidth{1.003750pt}%
\definecolor{currentstroke}{rgb}{0.839216,0.152941,0.156863}%
\pgfsetstrokecolor{currentstroke}%
\pgfsetstrokeopacity{0.457533}%
\pgfsetdash{}{0pt}%
\pgfpathmoveto{\pgfqpoint{1.201522in}{0.962534in}}%
\pgfpathcurveto{\pgfqpoint{1.211609in}{0.962534in}}{\pgfqpoint{1.221285in}{0.966542in}}{\pgfqpoint{1.228417in}{0.973675in}}%
\pgfpathcurveto{\pgfqpoint{1.235550in}{0.980808in}}{\pgfqpoint{1.239558in}{0.990483in}}{\pgfqpoint{1.239558in}{1.000570in}}%
\pgfpathcurveto{\pgfqpoint{1.239558in}{1.010658in}}{\pgfqpoint{1.235550in}{1.020333in}}{\pgfqpoint{1.228417in}{1.027466in}}%
\pgfpathcurveto{\pgfqpoint{1.221285in}{1.034599in}}{\pgfqpoint{1.211609in}{1.038607in}}{\pgfqpoint{1.201522in}{1.038607in}}%
\pgfpathcurveto{\pgfqpoint{1.191434in}{1.038607in}}{\pgfqpoint{1.181759in}{1.034599in}}{\pgfqpoint{1.174626in}{1.027466in}}%
\pgfpathcurveto{\pgfqpoint{1.167493in}{1.020333in}}{\pgfqpoint{1.163485in}{1.010658in}}{\pgfqpoint{1.163485in}{1.000570in}}%
\pgfpathcurveto{\pgfqpoint{1.163485in}{0.990483in}}{\pgfqpoint{1.167493in}{0.980808in}}{\pgfqpoint{1.174626in}{0.973675in}}%
\pgfpathcurveto{\pgfqpoint{1.181759in}{0.966542in}}{\pgfqpoint{1.191434in}{0.962534in}}{\pgfqpoint{1.201522in}{0.962534in}}%
\pgfpathlineto{\pgfqpoint{1.201522in}{0.962534in}}%
\pgfpathclose%
\pgfusepath{stroke,fill}%
\end{pgfscope}%
\begin{pgfscope}%
\pgfpathrectangle{\pgfqpoint{0.254231in}{0.147348in}}{\pgfqpoint{2.735294in}{2.735294in}}%
\pgfusepath{clip}%
\pgfsetbuttcap%
\pgfsetroundjoin%
\definecolor{currentfill}{rgb}{0.839216,0.152941,0.156863}%
\pgfsetfillcolor{currentfill}%
\pgfsetfillopacity{0.492303}%
\pgfsetlinewidth{1.003750pt}%
\definecolor{currentstroke}{rgb}{0.839216,0.152941,0.156863}%
\pgfsetstrokecolor{currentstroke}%
\pgfsetstrokeopacity{0.492303}%
\pgfsetdash{}{0pt}%
\pgfpathmoveto{\pgfqpoint{0.878198in}{1.511653in}}%
\pgfpathcurveto{\pgfqpoint{0.888285in}{1.511653in}}{\pgfqpoint{0.897960in}{1.515661in}}{\pgfqpoint{0.905093in}{1.522794in}}%
\pgfpathcurveto{\pgfqpoint{0.912226in}{1.529927in}}{\pgfqpoint{0.916234in}{1.539602in}}{\pgfqpoint{0.916234in}{1.549689in}}%
\pgfpathcurveto{\pgfqpoint{0.916234in}{1.559777in}}{\pgfqpoint{0.912226in}{1.569452in}}{\pgfqpoint{0.905093in}{1.576585in}}%
\pgfpathcurveto{\pgfqpoint{0.897960in}{1.583718in}}{\pgfqpoint{0.888285in}{1.587726in}}{\pgfqpoint{0.878198in}{1.587726in}}%
\pgfpathcurveto{\pgfqpoint{0.868110in}{1.587726in}}{\pgfqpoint{0.858435in}{1.583718in}}{\pgfqpoint{0.851302in}{1.576585in}}%
\pgfpathcurveto{\pgfqpoint{0.844169in}{1.569452in}}{\pgfqpoint{0.840161in}{1.559777in}}{\pgfqpoint{0.840161in}{1.549689in}}%
\pgfpathcurveto{\pgfqpoint{0.840161in}{1.539602in}}{\pgfqpoint{0.844169in}{1.529927in}}{\pgfqpoint{0.851302in}{1.522794in}}%
\pgfpathcurveto{\pgfqpoint{0.858435in}{1.515661in}}{\pgfqpoint{0.868110in}{1.511653in}}{\pgfqpoint{0.878198in}{1.511653in}}%
\pgfpathlineto{\pgfqpoint{0.878198in}{1.511653in}}%
\pgfpathclose%
\pgfusepath{stroke,fill}%
\end{pgfscope}%
\begin{pgfscope}%
\pgfpathrectangle{\pgfqpoint{0.254231in}{0.147348in}}{\pgfqpoint{2.735294in}{2.735294in}}%
\pgfusepath{clip}%
\pgfsetbuttcap%
\pgfsetroundjoin%
\definecolor{currentfill}{rgb}{0.839216,0.152941,0.156863}%
\pgfsetfillcolor{currentfill}%
\pgfsetfillopacity{0.498590}%
\pgfsetlinewidth{1.003750pt}%
\definecolor{currentstroke}{rgb}{0.839216,0.152941,0.156863}%
\pgfsetstrokecolor{currentstroke}%
\pgfsetstrokeopacity{0.498590}%
\pgfsetdash{}{0pt}%
\pgfpathmoveto{\pgfqpoint{2.525473in}{1.621636in}}%
\pgfpathcurveto{\pgfqpoint{2.535560in}{1.621636in}}{\pgfqpoint{2.545236in}{1.625643in}}{\pgfqpoint{2.552368in}{1.632776in}}%
\pgfpathcurveto{\pgfqpoint{2.559501in}{1.639909in}}{\pgfqpoint{2.563509in}{1.649585in}}{\pgfqpoint{2.563509in}{1.659672in}}%
\pgfpathcurveto{\pgfqpoint{2.563509in}{1.669759in}}{\pgfqpoint{2.559501in}{1.679435in}}{\pgfqpoint{2.552368in}{1.686568in}}%
\pgfpathcurveto{\pgfqpoint{2.545236in}{1.693701in}}{\pgfqpoint{2.535560in}{1.697708in}}{\pgfqpoint{2.525473in}{1.697708in}}%
\pgfpathcurveto{\pgfqpoint{2.515385in}{1.697708in}}{\pgfqpoint{2.505710in}{1.693701in}}{\pgfqpoint{2.498577in}{1.686568in}}%
\pgfpathcurveto{\pgfqpoint{2.491444in}{1.679435in}}{\pgfqpoint{2.487436in}{1.669759in}}{\pgfqpoint{2.487436in}{1.659672in}}%
\pgfpathcurveto{\pgfqpoint{2.487436in}{1.649585in}}{\pgfqpoint{2.491444in}{1.639909in}}{\pgfqpoint{2.498577in}{1.632776in}}%
\pgfpathcurveto{\pgfqpoint{2.505710in}{1.625643in}}{\pgfqpoint{2.515385in}{1.621636in}}{\pgfqpoint{2.525473in}{1.621636in}}%
\pgfpathlineto{\pgfqpoint{2.525473in}{1.621636in}}%
\pgfpathclose%
\pgfusepath{stroke,fill}%
\end{pgfscope}%
\begin{pgfscope}%
\pgfpathrectangle{\pgfqpoint{0.254231in}{0.147348in}}{\pgfqpoint{2.735294in}{2.735294in}}%
\pgfusepath{clip}%
\pgfsetbuttcap%
\pgfsetroundjoin%
\definecolor{currentfill}{rgb}{0.839216,0.152941,0.156863}%
\pgfsetfillcolor{currentfill}%
\pgfsetfillopacity{0.612876}%
\pgfsetlinewidth{1.003750pt}%
\definecolor{currentstroke}{rgb}{0.839216,0.152941,0.156863}%
\pgfsetstrokecolor{currentstroke}%
\pgfsetstrokeopacity{0.612876}%
\pgfsetdash{}{0pt}%
\pgfpathmoveto{\pgfqpoint{1.213770in}{0.979547in}}%
\pgfpathcurveto{\pgfqpoint{1.223857in}{0.979547in}}{\pgfqpoint{1.233533in}{0.983554in}}{\pgfqpoint{1.240666in}{0.990687in}}%
\pgfpathcurveto{\pgfqpoint{1.247798in}{0.997820in}}{\pgfqpoint{1.251806in}{1.007495in}}{\pgfqpoint{1.251806in}{1.017583in}}%
\pgfpathcurveto{\pgfqpoint{1.251806in}{1.027670in}}{\pgfqpoint{1.247798in}{1.037346in}}{\pgfqpoint{1.240666in}{1.044479in}}%
\pgfpathcurveto{\pgfqpoint{1.233533in}{1.051611in}}{\pgfqpoint{1.223857in}{1.055619in}}{\pgfqpoint{1.213770in}{1.055619in}}%
\pgfpathcurveto{\pgfqpoint{1.203683in}{1.055619in}}{\pgfqpoint{1.194007in}{1.051611in}}{\pgfqpoint{1.186874in}{1.044479in}}%
\pgfpathcurveto{\pgfqpoint{1.179741in}{1.037346in}}{\pgfqpoint{1.175734in}{1.027670in}}{\pgfqpoint{1.175734in}{1.017583in}}%
\pgfpathcurveto{\pgfqpoint{1.175734in}{1.007495in}}{\pgfqpoint{1.179741in}{0.997820in}}{\pgfqpoint{1.186874in}{0.990687in}}%
\pgfpathcurveto{\pgfqpoint{1.194007in}{0.983554in}}{\pgfqpoint{1.203683in}{0.979547in}}{\pgfqpoint{1.213770in}{0.979547in}}%
\pgfpathlineto{\pgfqpoint{1.213770in}{0.979547in}}%
\pgfpathclose%
\pgfusepath{stroke,fill}%
\end{pgfscope}%
\begin{pgfscope}%
\pgfpathrectangle{\pgfqpoint{0.254231in}{0.147348in}}{\pgfqpoint{2.735294in}{2.735294in}}%
\pgfusepath{clip}%
\pgfsetbuttcap%
\pgfsetroundjoin%
\definecolor{currentfill}{rgb}{0.839216,0.152941,0.156863}%
\pgfsetfillcolor{currentfill}%
\pgfsetfillopacity{0.625674}%
\pgfsetlinewidth{1.003750pt}%
\definecolor{currentstroke}{rgb}{0.839216,0.152941,0.156863}%
\pgfsetstrokecolor{currentstroke}%
\pgfsetstrokeopacity{0.625674}%
\pgfsetdash{}{0pt}%
\pgfpathmoveto{\pgfqpoint{2.092670in}{2.164239in}}%
\pgfpathcurveto{\pgfqpoint{2.102758in}{2.164239in}}{\pgfqpoint{2.112433in}{2.168247in}}{\pgfqpoint{2.119566in}{2.175380in}}%
\pgfpathcurveto{\pgfqpoint{2.126699in}{2.182513in}}{\pgfqpoint{2.130707in}{2.192188in}}{\pgfqpoint{2.130707in}{2.202275in}}%
\pgfpathcurveto{\pgfqpoint{2.130707in}{2.212363in}}{\pgfqpoint{2.126699in}{2.222038in}}{\pgfqpoint{2.119566in}{2.229171in}}%
\pgfpathcurveto{\pgfqpoint{2.112433in}{2.236304in}}{\pgfqpoint{2.102758in}{2.240312in}}{\pgfqpoint{2.092670in}{2.240312in}}%
\pgfpathcurveto{\pgfqpoint{2.082583in}{2.240312in}}{\pgfqpoint{2.072907in}{2.236304in}}{\pgfqpoint{2.065775in}{2.229171in}}%
\pgfpathcurveto{\pgfqpoint{2.058642in}{2.222038in}}{\pgfqpoint{2.054634in}{2.212363in}}{\pgfqpoint{2.054634in}{2.202275in}}%
\pgfpathcurveto{\pgfqpoint{2.054634in}{2.192188in}}{\pgfqpoint{2.058642in}{2.182513in}}{\pgfqpoint{2.065775in}{2.175380in}}%
\pgfpathcurveto{\pgfqpoint{2.072907in}{2.168247in}}{\pgfqpoint{2.082583in}{2.164239in}}{\pgfqpoint{2.092670in}{2.164239in}}%
\pgfpathlineto{\pgfqpoint{2.092670in}{2.164239in}}%
\pgfpathclose%
\pgfusepath{stroke,fill}%
\end{pgfscope}%
\begin{pgfscope}%
\pgfpathrectangle{\pgfqpoint{0.254231in}{0.147348in}}{\pgfqpoint{2.735294in}{2.735294in}}%
\pgfusepath{clip}%
\pgfsetbuttcap%
\pgfsetroundjoin%
\definecolor{currentfill}{rgb}{0.839216,0.152941,0.156863}%
\pgfsetfillcolor{currentfill}%
\pgfsetfillopacity{0.631635}%
\pgfsetlinewidth{1.003750pt}%
\definecolor{currentstroke}{rgb}{0.839216,0.152941,0.156863}%
\pgfsetstrokecolor{currentstroke}%
\pgfsetstrokeopacity{0.631635}%
\pgfsetdash{}{0pt}%
\pgfpathmoveto{\pgfqpoint{2.083772in}{2.104757in}}%
\pgfpathcurveto{\pgfqpoint{2.093859in}{2.104757in}}{\pgfqpoint{2.103535in}{2.108765in}}{\pgfqpoint{2.110668in}{2.115898in}}%
\pgfpathcurveto{\pgfqpoint{2.117801in}{2.123030in}}{\pgfqpoint{2.121808in}{2.132706in}}{\pgfqpoint{2.121808in}{2.142793in}}%
\pgfpathcurveto{\pgfqpoint{2.121808in}{2.152881in}}{\pgfqpoint{2.117801in}{2.162556in}}{\pgfqpoint{2.110668in}{2.169689in}}%
\pgfpathcurveto{\pgfqpoint{2.103535in}{2.176822in}}{\pgfqpoint{2.093859in}{2.180830in}}{\pgfqpoint{2.083772in}{2.180830in}}%
\pgfpathcurveto{\pgfqpoint{2.073685in}{2.180830in}}{\pgfqpoint{2.064009in}{2.176822in}}{\pgfqpoint{2.056876in}{2.169689in}}%
\pgfpathcurveto{\pgfqpoint{2.049744in}{2.162556in}}{\pgfqpoint{2.045736in}{2.152881in}}{\pgfqpoint{2.045736in}{2.142793in}}%
\pgfpathcurveto{\pgfqpoint{2.045736in}{2.132706in}}{\pgfqpoint{2.049744in}{2.123030in}}{\pgfqpoint{2.056876in}{2.115898in}}%
\pgfpathcurveto{\pgfqpoint{2.064009in}{2.108765in}}{\pgfqpoint{2.073685in}{2.104757in}}{\pgfqpoint{2.083772in}{2.104757in}}%
\pgfpathlineto{\pgfqpoint{2.083772in}{2.104757in}}%
\pgfpathclose%
\pgfusepath{stroke,fill}%
\end{pgfscope}%
\begin{pgfscope}%
\pgfpathrectangle{\pgfqpoint{0.254231in}{0.147348in}}{\pgfqpoint{2.735294in}{2.735294in}}%
\pgfusepath{clip}%
\pgfsetbuttcap%
\pgfsetroundjoin%
\definecolor{currentfill}{rgb}{0.839216,0.152941,0.156863}%
\pgfsetfillcolor{currentfill}%
\pgfsetfillopacity{0.634032}%
\pgfsetlinewidth{1.003750pt}%
\definecolor{currentstroke}{rgb}{0.839216,0.152941,0.156863}%
\pgfsetstrokecolor{currentstroke}%
\pgfsetstrokeopacity{0.634032}%
\pgfsetdash{}{0pt}%
\pgfpathmoveto{\pgfqpoint{2.354767in}{1.533362in}}%
\pgfpathcurveto{\pgfqpoint{2.364854in}{1.533362in}}{\pgfqpoint{2.374530in}{1.537370in}}{\pgfqpoint{2.381663in}{1.544503in}}%
\pgfpathcurveto{\pgfqpoint{2.388796in}{1.551635in}}{\pgfqpoint{2.392803in}{1.561311in}}{\pgfqpoint{2.392803in}{1.571398in}}%
\pgfpathcurveto{\pgfqpoint{2.392803in}{1.581486in}}{\pgfqpoint{2.388796in}{1.591161in}}{\pgfqpoint{2.381663in}{1.598294in}}%
\pgfpathcurveto{\pgfqpoint{2.374530in}{1.605427in}}{\pgfqpoint{2.364854in}{1.609435in}}{\pgfqpoint{2.354767in}{1.609435in}}%
\pgfpathcurveto{\pgfqpoint{2.344680in}{1.609435in}}{\pgfqpoint{2.335004in}{1.605427in}}{\pgfqpoint{2.327871in}{1.598294in}}%
\pgfpathcurveto{\pgfqpoint{2.320739in}{1.591161in}}{\pgfqpoint{2.316731in}{1.581486in}}{\pgfqpoint{2.316731in}{1.571398in}}%
\pgfpathcurveto{\pgfqpoint{2.316731in}{1.561311in}}{\pgfqpoint{2.320739in}{1.551635in}}{\pgfqpoint{2.327871in}{1.544503in}}%
\pgfpathcurveto{\pgfqpoint{2.335004in}{1.537370in}}{\pgfqpoint{2.344680in}{1.533362in}}{\pgfqpoint{2.354767in}{1.533362in}}%
\pgfpathlineto{\pgfqpoint{2.354767in}{1.533362in}}%
\pgfpathclose%
\pgfusepath{stroke,fill}%
\end{pgfscope}%
\begin{pgfscope}%
\pgfpathrectangle{\pgfqpoint{0.254231in}{0.147348in}}{\pgfqpoint{2.735294in}{2.735294in}}%
\pgfusepath{clip}%
\pgfsetbuttcap%
\pgfsetroundjoin%
\definecolor{currentfill}{rgb}{0.839216,0.152941,0.156863}%
\pgfsetfillcolor{currentfill}%
\pgfsetfillopacity{0.652064}%
\pgfsetlinewidth{1.003750pt}%
\definecolor{currentstroke}{rgb}{0.839216,0.152941,0.156863}%
\pgfsetstrokecolor{currentstroke}%
\pgfsetstrokeopacity{0.652064}%
\pgfsetdash{}{0pt}%
\pgfpathmoveto{\pgfqpoint{1.468055in}{0.952320in}}%
\pgfpathcurveto{\pgfqpoint{1.478142in}{0.952320in}}{\pgfqpoint{1.487818in}{0.956328in}}{\pgfqpoint{1.494951in}{0.963461in}}%
\pgfpathcurveto{\pgfqpoint{1.502084in}{0.970593in}}{\pgfqpoint{1.506091in}{0.980269in}}{\pgfqpoint{1.506091in}{0.990356in}}%
\pgfpathcurveto{\pgfqpoint{1.506091in}{1.000444in}}{\pgfqpoint{1.502084in}{1.010119in}}{\pgfqpoint{1.494951in}{1.017252in}}%
\pgfpathcurveto{\pgfqpoint{1.487818in}{1.024385in}}{\pgfqpoint{1.478142in}{1.028393in}}{\pgfqpoint{1.468055in}{1.028393in}}%
\pgfpathcurveto{\pgfqpoint{1.457968in}{1.028393in}}{\pgfqpoint{1.448292in}{1.024385in}}{\pgfqpoint{1.441159in}{1.017252in}}%
\pgfpathcurveto{\pgfqpoint{1.434027in}{1.010119in}}{\pgfqpoint{1.430019in}{1.000444in}}{\pgfqpoint{1.430019in}{0.990356in}}%
\pgfpathcurveto{\pgfqpoint{1.430019in}{0.980269in}}{\pgfqpoint{1.434027in}{0.970593in}}{\pgfqpoint{1.441159in}{0.963461in}}%
\pgfpathcurveto{\pgfqpoint{1.448292in}{0.956328in}}{\pgfqpoint{1.457968in}{0.952320in}}{\pgfqpoint{1.468055in}{0.952320in}}%
\pgfpathlineto{\pgfqpoint{1.468055in}{0.952320in}}%
\pgfpathclose%
\pgfusepath{stroke,fill}%
\end{pgfscope}%
\begin{pgfscope}%
\pgfpathrectangle{\pgfqpoint{0.254231in}{0.147348in}}{\pgfqpoint{2.735294in}{2.735294in}}%
\pgfusepath{clip}%
\pgfsetbuttcap%
\pgfsetroundjoin%
\definecolor{currentfill}{rgb}{0.839216,0.152941,0.156863}%
\pgfsetfillcolor{currentfill}%
\pgfsetfillopacity{0.738747}%
\pgfsetlinewidth{1.003750pt}%
\definecolor{currentstroke}{rgb}{0.839216,0.152941,0.156863}%
\pgfsetstrokecolor{currentstroke}%
\pgfsetstrokeopacity{0.738747}%
\pgfsetdash{}{0pt}%
\pgfpathmoveto{\pgfqpoint{0.858170in}{1.507716in}}%
\pgfpathcurveto{\pgfqpoint{0.868258in}{1.507716in}}{\pgfqpoint{0.877933in}{1.511723in}}{\pgfqpoint{0.885066in}{1.518856in}}%
\pgfpathcurveto{\pgfqpoint{0.892199in}{1.525989in}}{\pgfqpoint{0.896207in}{1.535665in}}{\pgfqpoint{0.896207in}{1.545752in}}%
\pgfpathcurveto{\pgfqpoint{0.896207in}{1.555839in}}{\pgfqpoint{0.892199in}{1.565515in}}{\pgfqpoint{0.885066in}{1.572648in}}%
\pgfpathcurveto{\pgfqpoint{0.877933in}{1.579780in}}{\pgfqpoint{0.868258in}{1.583788in}}{\pgfqpoint{0.858170in}{1.583788in}}%
\pgfpathcurveto{\pgfqpoint{0.848083in}{1.583788in}}{\pgfqpoint{0.838408in}{1.579780in}}{\pgfqpoint{0.831275in}{1.572648in}}%
\pgfpathcurveto{\pgfqpoint{0.824142in}{1.565515in}}{\pgfqpoint{0.820134in}{1.555839in}}{\pgfqpoint{0.820134in}{1.545752in}}%
\pgfpathcurveto{\pgfqpoint{0.820134in}{1.535665in}}{\pgfqpoint{0.824142in}{1.525989in}}{\pgfqpoint{0.831275in}{1.518856in}}%
\pgfpathcurveto{\pgfqpoint{0.838408in}{1.511723in}}{\pgfqpoint{0.848083in}{1.507716in}}{\pgfqpoint{0.858170in}{1.507716in}}%
\pgfpathlineto{\pgfqpoint{0.858170in}{1.507716in}}%
\pgfpathclose%
\pgfusepath{stroke,fill}%
\end{pgfscope}%
\begin{pgfscope}%
\pgfpathrectangle{\pgfqpoint{0.254231in}{0.147348in}}{\pgfqpoint{2.735294in}{2.735294in}}%
\pgfusepath{clip}%
\pgfsetbuttcap%
\pgfsetroundjoin%
\definecolor{currentfill}{rgb}{0.839216,0.152941,0.156863}%
\pgfsetfillcolor{currentfill}%
\pgfsetfillopacity{0.791813}%
\pgfsetlinewidth{1.003750pt}%
\definecolor{currentstroke}{rgb}{0.839216,0.152941,0.156863}%
\pgfsetstrokecolor{currentstroke}%
\pgfsetstrokeopacity{0.791813}%
\pgfsetdash{}{0pt}%
\pgfpathmoveto{\pgfqpoint{1.351308in}{2.014183in}}%
\pgfpathcurveto{\pgfqpoint{1.361395in}{2.014183in}}{\pgfqpoint{1.371071in}{2.018191in}}{\pgfqpoint{1.378204in}{2.025323in}}%
\pgfpathcurveto{\pgfqpoint{1.385337in}{2.032456in}}{\pgfqpoint{1.389344in}{2.042132in}}{\pgfqpoint{1.389344in}{2.052219in}}%
\pgfpathcurveto{\pgfqpoint{1.389344in}{2.062307in}}{\pgfqpoint{1.385337in}{2.071982in}}{\pgfqpoint{1.378204in}{2.079115in}}%
\pgfpathcurveto{\pgfqpoint{1.371071in}{2.086248in}}{\pgfqpoint{1.361395in}{2.090255in}}{\pgfqpoint{1.351308in}{2.090255in}}%
\pgfpathcurveto{\pgfqpoint{1.341221in}{2.090255in}}{\pgfqpoint{1.331545in}{2.086248in}}{\pgfqpoint{1.324412in}{2.079115in}}%
\pgfpathcurveto{\pgfqpoint{1.317280in}{2.071982in}}{\pgfqpoint{1.313272in}{2.062307in}}{\pgfqpoint{1.313272in}{2.052219in}}%
\pgfpathcurveto{\pgfqpoint{1.313272in}{2.042132in}}{\pgfqpoint{1.317280in}{2.032456in}}{\pgfqpoint{1.324412in}{2.025323in}}%
\pgfpathcurveto{\pgfqpoint{1.331545in}{2.018191in}}{\pgfqpoint{1.341221in}{2.014183in}}{\pgfqpoint{1.351308in}{2.014183in}}%
\pgfpathlineto{\pgfqpoint{1.351308in}{2.014183in}}%
\pgfpathclose%
\pgfusepath{stroke,fill}%
\end{pgfscope}%
\begin{pgfscope}%
\pgfpathrectangle{\pgfqpoint{0.254231in}{0.147348in}}{\pgfqpoint{2.735294in}{2.735294in}}%
\pgfusepath{clip}%
\pgfsetbuttcap%
\pgfsetroundjoin%
\definecolor{currentfill}{rgb}{0.839216,0.152941,0.156863}%
\pgfsetfillcolor{currentfill}%
\pgfsetfillopacity{0.864233}%
\pgfsetlinewidth{1.003750pt}%
\definecolor{currentstroke}{rgb}{0.839216,0.152941,0.156863}%
\pgfsetstrokecolor{currentstroke}%
\pgfsetstrokeopacity{0.864233}%
\pgfsetdash{}{0pt}%
\pgfpathmoveto{\pgfqpoint{2.017635in}{1.926870in}}%
\pgfpathcurveto{\pgfqpoint{2.027722in}{1.926870in}}{\pgfqpoint{2.037397in}{1.930878in}}{\pgfqpoint{2.044530in}{1.938011in}}%
\pgfpathcurveto{\pgfqpoint{2.051663in}{1.945143in}}{\pgfqpoint{2.055671in}{1.954819in}}{\pgfqpoint{2.055671in}{1.964906in}}%
\pgfpathcurveto{\pgfqpoint{2.055671in}{1.974994in}}{\pgfqpoint{2.051663in}{1.984669in}}{\pgfqpoint{2.044530in}{1.991802in}}%
\pgfpathcurveto{\pgfqpoint{2.037397in}{1.998935in}}{\pgfqpoint{2.027722in}{2.002943in}}{\pgfqpoint{2.017635in}{2.002943in}}%
\pgfpathcurveto{\pgfqpoint{2.007547in}{2.002943in}}{\pgfqpoint{1.997872in}{1.998935in}}{\pgfqpoint{1.990739in}{1.991802in}}%
\pgfpathcurveto{\pgfqpoint{1.983606in}{1.984669in}}{\pgfqpoint{1.979598in}{1.974994in}}{\pgfqpoint{1.979598in}{1.964906in}}%
\pgfpathcurveto{\pgfqpoint{1.979598in}{1.954819in}}{\pgfqpoint{1.983606in}{1.945143in}}{\pgfqpoint{1.990739in}{1.938011in}}%
\pgfpathcurveto{\pgfqpoint{1.997872in}{1.930878in}}{\pgfqpoint{2.007547in}{1.926870in}}{\pgfqpoint{2.017635in}{1.926870in}}%
\pgfpathlineto{\pgfqpoint{2.017635in}{1.926870in}}%
\pgfpathclose%
\pgfusepath{stroke,fill}%
\end{pgfscope}%
\begin{pgfscope}%
\pgfpathrectangle{\pgfqpoint{0.254231in}{0.147348in}}{\pgfqpoint{2.735294in}{2.735294in}}%
\pgfusepath{clip}%
\pgfsetbuttcap%
\pgfsetroundjoin%
\definecolor{currentfill}{rgb}{0.839216,0.152941,0.156863}%
\pgfsetfillcolor{currentfill}%
\pgfsetfillopacity{0.929084}%
\pgfsetlinewidth{1.003750pt}%
\definecolor{currentstroke}{rgb}{0.839216,0.152941,0.156863}%
\pgfsetstrokecolor{currentstroke}%
\pgfsetstrokeopacity{0.929084}%
\pgfsetdash{}{0pt}%
\pgfpathmoveto{\pgfqpoint{2.077047in}{1.288920in}}%
\pgfpathcurveto{\pgfqpoint{2.087135in}{1.288920in}}{\pgfqpoint{2.096810in}{1.292928in}}{\pgfqpoint{2.103943in}{1.300060in}}%
\pgfpathcurveto{\pgfqpoint{2.111076in}{1.307193in}}{\pgfqpoint{2.115084in}{1.316869in}}{\pgfqpoint{2.115084in}{1.326956in}}%
\pgfpathcurveto{\pgfqpoint{2.115084in}{1.337043in}}{\pgfqpoint{2.111076in}{1.346719in}}{\pgfqpoint{2.103943in}{1.353852in}}%
\pgfpathcurveto{\pgfqpoint{2.096810in}{1.360985in}}{\pgfqpoint{2.087135in}{1.364992in}}{\pgfqpoint{2.077047in}{1.364992in}}%
\pgfpathcurveto{\pgfqpoint{2.066960in}{1.364992in}}{\pgfqpoint{2.057285in}{1.360985in}}{\pgfqpoint{2.050152in}{1.353852in}}%
\pgfpathcurveto{\pgfqpoint{2.043019in}{1.346719in}}{\pgfqpoint{2.039011in}{1.337043in}}{\pgfqpoint{2.039011in}{1.326956in}}%
\pgfpathcurveto{\pgfqpoint{2.039011in}{1.316869in}}{\pgfqpoint{2.043019in}{1.307193in}}{\pgfqpoint{2.050152in}{1.300060in}}%
\pgfpathcurveto{\pgfqpoint{2.057285in}{1.292928in}}{\pgfqpoint{2.066960in}{1.288920in}}{\pgfqpoint{2.077047in}{1.288920in}}%
\pgfpathlineto{\pgfqpoint{2.077047in}{1.288920in}}%
\pgfpathclose%
\pgfusepath{stroke,fill}%
\end{pgfscope}%
\begin{pgfscope}%
\pgfpathrectangle{\pgfqpoint{0.254231in}{0.147348in}}{\pgfqpoint{2.735294in}{2.735294in}}%
\pgfusepath{clip}%
\pgfsetbuttcap%
\pgfsetroundjoin%
\definecolor{currentfill}{rgb}{0.839216,0.152941,0.156863}%
\pgfsetfillcolor{currentfill}%
\pgfsetlinewidth{1.003750pt}%
\definecolor{currentstroke}{rgb}{0.839216,0.152941,0.156863}%
\pgfsetstrokecolor{currentstroke}%
\pgfsetdash{}{0pt}%
\pgfpathmoveto{\pgfqpoint{1.957013in}{1.506712in}}%
\pgfpathcurveto{\pgfqpoint{1.967100in}{1.506712in}}{\pgfqpoint{1.976776in}{1.510720in}}{\pgfqpoint{1.983908in}{1.517853in}}%
\pgfpathcurveto{\pgfqpoint{1.991041in}{1.524986in}}{\pgfqpoint{1.995049in}{1.534661in}}{\pgfqpoint{1.995049in}{1.544748in}}%
\pgfpathcurveto{\pgfqpoint{1.995049in}{1.554836in}}{\pgfqpoint{1.991041in}{1.564511in}}{\pgfqpoint{1.983908in}{1.571644in}}%
\pgfpathcurveto{\pgfqpoint{1.976776in}{1.578777in}}{\pgfqpoint{1.967100in}{1.582785in}}{\pgfqpoint{1.957013in}{1.582785in}}%
\pgfpathcurveto{\pgfqpoint{1.946925in}{1.582785in}}{\pgfqpoint{1.937250in}{1.578777in}}{\pgfqpoint{1.930117in}{1.571644in}}%
\pgfpathcurveto{\pgfqpoint{1.922984in}{1.564511in}}{\pgfqpoint{1.918976in}{1.554836in}}{\pgfqpoint{1.918976in}{1.544748in}}%
\pgfpathcurveto{\pgfqpoint{1.918976in}{1.534661in}}{\pgfqpoint{1.922984in}{1.524986in}}{\pgfqpoint{1.930117in}{1.517853in}}%
\pgfpathcurveto{\pgfqpoint{1.937250in}{1.510720in}}{\pgfqpoint{1.946925in}{1.506712in}}{\pgfqpoint{1.957013in}{1.506712in}}%
\pgfpathlineto{\pgfqpoint{1.957013in}{1.506712in}}%
\pgfpathclose%
\pgfusepath{stroke,fill}%
\end{pgfscope}%
\begin{pgfscope}%
\pgfpathrectangle{\pgfqpoint{0.254231in}{0.147348in}}{\pgfqpoint{2.735294in}{2.735294in}}%
\pgfusepath{clip}%
\pgfsetbuttcap%
\pgfsetroundjoin%
\definecolor{currentfill}{rgb}{0.071067,0.258424,0.071067}%
\pgfsetfillcolor{currentfill}%
\pgfsetlinewidth{0.000000pt}%
\definecolor{currentstroke}{rgb}{0.000000,0.000000,0.000000}%
\pgfsetstrokecolor{currentstroke}%
\pgfsetdash{}{0pt}%
\pgfpathmoveto{\pgfqpoint{0.697594in}{1.088921in}}%
\pgfpathlineto{\pgfqpoint{0.601174in}{1.228895in}}%
\pgfpathlineto{\pgfqpoint{0.601748in}{1.155548in}}%
\pgfpathlineto{\pgfqpoint{0.697594in}{1.088921in}}%
\pgfpathclose%
\pgfusepath{fill}%
\end{pgfscope}%
\begin{pgfscope}%
\pgfpathrectangle{\pgfqpoint{0.254231in}{0.147348in}}{\pgfqpoint{2.735294in}{2.735294in}}%
\pgfusepath{clip}%
\pgfsetbuttcap%
\pgfsetroundjoin%
\definecolor{currentfill}{rgb}{0.071067,0.258424,0.071067}%
\pgfsetfillcolor{currentfill}%
\pgfsetlinewidth{0.000000pt}%
\definecolor{currentstroke}{rgb}{0.000000,0.000000,0.000000}%
\pgfsetstrokecolor{currentstroke}%
\pgfsetdash{}{0pt}%
\pgfpathmoveto{\pgfqpoint{2.716510in}{1.228895in}}%
\pgfpathlineto{\pgfqpoint{2.620090in}{1.088921in}}%
\pgfpathlineto{\pgfqpoint{2.715936in}{1.155548in}}%
\pgfpathlineto{\pgfqpoint{2.716510in}{1.228895in}}%
\pgfpathclose%
\pgfusepath{fill}%
\end{pgfscope}%
\begin{pgfscope}%
\pgfpathrectangle{\pgfqpoint{0.254231in}{0.147348in}}{\pgfqpoint{2.735294in}{2.735294in}}%
\pgfusepath{clip}%
\pgfsetbuttcap%
\pgfsetroundjoin%
\definecolor{currentfill}{rgb}{0.128601,0.467641,0.128601}%
\pgfsetfillcolor{currentfill}%
\pgfsetlinewidth{0.000000pt}%
\definecolor{currentstroke}{rgb}{0.000000,0.000000,0.000000}%
\pgfsetstrokecolor{currentstroke}%
\pgfsetdash{}{0pt}%
\pgfpathmoveto{\pgfqpoint{1.562421in}{2.632943in}}%
\pgfpathlineto{\pgfqpoint{1.755263in}{2.632943in}}%
\pgfpathlineto{\pgfqpoint{1.658842in}{2.699366in}}%
\pgfpathlineto{\pgfqpoint{1.562421in}{2.632943in}}%
\pgfpathclose%
\pgfusepath{fill}%
\end{pgfscope}%
\begin{pgfscope}%
\pgfpathrectangle{\pgfqpoint{0.254231in}{0.147348in}}{\pgfqpoint{2.735294in}{2.735294in}}%
\pgfusepath{clip}%
\pgfsetbuttcap%
\pgfsetroundjoin%
\definecolor{currentfill}{rgb}{0.067488,0.245410,0.067488}%
\pgfsetfillcolor{currentfill}%
\pgfsetlinewidth{0.000000pt}%
\definecolor{currentstroke}{rgb}{0.000000,0.000000,0.000000}%
\pgfsetstrokecolor{currentstroke}%
\pgfsetdash{}{0pt}%
\pgfpathmoveto{\pgfqpoint{0.822114in}{1.018707in}}%
\pgfpathlineto{\pgfqpoint{0.707491in}{1.163589in}}%
\pgfpathlineto{\pgfqpoint{0.697594in}{1.088921in}}%
\pgfpathlineto{\pgfqpoint{0.822114in}{1.018707in}}%
\pgfpathclose%
\pgfusepath{fill}%
\end{pgfscope}%
\begin{pgfscope}%
\pgfpathrectangle{\pgfqpoint{0.254231in}{0.147348in}}{\pgfqpoint{2.735294in}{2.735294in}}%
\pgfusepath{clip}%
\pgfsetbuttcap%
\pgfsetroundjoin%
\definecolor{currentfill}{rgb}{0.067488,0.245410,0.067488}%
\pgfsetfillcolor{currentfill}%
\pgfsetlinewidth{0.000000pt}%
\definecolor{currentstroke}{rgb}{0.000000,0.000000,0.000000}%
\pgfsetstrokecolor{currentstroke}%
\pgfsetdash{}{0pt}%
\pgfpathmoveto{\pgfqpoint{2.620090in}{1.088921in}}%
\pgfpathlineto{\pgfqpoint{2.610193in}{1.163589in}}%
\pgfpathlineto{\pgfqpoint{2.495569in}{1.018707in}}%
\pgfpathlineto{\pgfqpoint{2.620090in}{1.088921in}}%
\pgfpathclose%
\pgfusepath{fill}%
\end{pgfscope}%
\begin{pgfscope}%
\pgfpathrectangle{\pgfqpoint{0.254231in}{0.147348in}}{\pgfqpoint{2.735294in}{2.735294in}}%
\pgfusepath{clip}%
\pgfsetbuttcap%
\pgfsetroundjoin%
\definecolor{currentfill}{rgb}{0.069492,0.252698,0.069492}%
\pgfsetfillcolor{currentfill}%
\pgfsetlinewidth{0.000000pt}%
\definecolor{currentstroke}{rgb}{0.000000,0.000000,0.000000}%
\pgfsetstrokecolor{currentstroke}%
\pgfsetdash{}{0pt}%
\pgfpathmoveto{\pgfqpoint{0.601174in}{1.228895in}}%
\pgfpathlineto{\pgfqpoint{0.697594in}{1.088921in}}%
\pgfpathlineto{\pgfqpoint{0.695386in}{1.589726in}}%
\pgfpathlineto{\pgfqpoint{0.601174in}{1.228895in}}%
\pgfpathclose%
\pgfusepath{fill}%
\end{pgfscope}%
\begin{pgfscope}%
\pgfpathrectangle{\pgfqpoint{0.254231in}{0.147348in}}{\pgfqpoint{2.735294in}{2.735294in}}%
\pgfusepath{clip}%
\pgfsetbuttcap%
\pgfsetroundjoin%
\definecolor{currentfill}{rgb}{0.069492,0.252698,0.069492}%
\pgfsetfillcolor{currentfill}%
\pgfsetlinewidth{0.000000pt}%
\definecolor{currentstroke}{rgb}{0.000000,0.000000,0.000000}%
\pgfsetstrokecolor{currentstroke}%
\pgfsetdash{}{0pt}%
\pgfpathmoveto{\pgfqpoint{2.716510in}{1.228895in}}%
\pgfpathlineto{\pgfqpoint{2.622298in}{1.589726in}}%
\pgfpathlineto{\pgfqpoint{2.620090in}{1.088921in}}%
\pgfpathlineto{\pgfqpoint{2.716510in}{1.228895in}}%
\pgfpathclose%
\pgfusepath{fill}%
\end{pgfscope}%
\begin{pgfscope}%
\pgfpathrectangle{\pgfqpoint{0.254231in}{0.147348in}}{\pgfqpoint{2.735294in}{2.735294in}}%
\pgfusepath{clip}%
\pgfsetbuttcap%
\pgfsetroundjoin%
\definecolor{currentfill}{rgb}{0.099716,0.362602,0.099716}%
\pgfsetfillcolor{currentfill}%
\pgfsetlinewidth{0.000000pt}%
\definecolor{currentstroke}{rgb}{0.000000,0.000000,0.000000}%
\pgfsetstrokecolor{currentstroke}%
\pgfsetdash{}{0pt}%
\pgfpathmoveto{\pgfqpoint{0.695386in}{1.589726in}}%
\pgfpathlineto{\pgfqpoint{0.697594in}{1.088921in}}%
\pgfpathlineto{\pgfqpoint{0.707491in}{1.163589in}}%
\pgfpathlineto{\pgfqpoint{0.695386in}{1.589726in}}%
\pgfpathclose%
\pgfusepath{fill}%
\end{pgfscope}%
\begin{pgfscope}%
\pgfpathrectangle{\pgfqpoint{0.254231in}{0.147348in}}{\pgfqpoint{2.735294in}{2.735294in}}%
\pgfusepath{clip}%
\pgfsetbuttcap%
\pgfsetroundjoin%
\definecolor{currentfill}{rgb}{0.099716,0.362602,0.099716}%
\pgfsetfillcolor{currentfill}%
\pgfsetlinewidth{0.000000pt}%
\definecolor{currentstroke}{rgb}{0.000000,0.000000,0.000000}%
\pgfsetstrokecolor{currentstroke}%
\pgfsetdash{}{0pt}%
\pgfpathmoveto{\pgfqpoint{2.610193in}{1.163589in}}%
\pgfpathlineto{\pgfqpoint{2.620090in}{1.088921in}}%
\pgfpathlineto{\pgfqpoint{2.622298in}{1.589726in}}%
\pgfpathlineto{\pgfqpoint{2.610193in}{1.163589in}}%
\pgfpathclose%
\pgfusepath{fill}%
\end{pgfscope}%
\begin{pgfscope}%
\pgfpathrectangle{\pgfqpoint{0.254231in}{0.147348in}}{\pgfqpoint{2.735294in}{2.735294in}}%
\pgfusepath{clip}%
\pgfsetbuttcap%
\pgfsetroundjoin%
\definecolor{currentfill}{rgb}{0.063840,0.232145,0.063840}%
\pgfsetfillcolor{currentfill}%
\pgfsetlinewidth{0.000000pt}%
\definecolor{currentstroke}{rgb}{0.000000,0.000000,0.000000}%
\pgfsetstrokecolor{currentstroke}%
\pgfsetdash{}{0pt}%
\pgfpathmoveto{\pgfqpoint{0.981201in}{0.949464in}}%
\pgfpathlineto{\pgfqpoint{0.848606in}{1.094945in}}%
\pgfpathlineto{\pgfqpoint{0.822114in}{1.018707in}}%
\pgfpathlineto{\pgfqpoint{0.981201in}{0.949464in}}%
\pgfpathclose%
\pgfusepath{fill}%
\end{pgfscope}%
\begin{pgfscope}%
\pgfpathrectangle{\pgfqpoint{0.254231in}{0.147348in}}{\pgfqpoint{2.735294in}{2.735294in}}%
\pgfusepath{clip}%
\pgfsetbuttcap%
\pgfsetroundjoin%
\definecolor{currentfill}{rgb}{0.063840,0.232145,0.063840}%
\pgfsetfillcolor{currentfill}%
\pgfsetlinewidth{0.000000pt}%
\definecolor{currentstroke}{rgb}{0.000000,0.000000,0.000000}%
\pgfsetstrokecolor{currentstroke}%
\pgfsetdash{}{0pt}%
\pgfpathmoveto{\pgfqpoint{2.495569in}{1.018707in}}%
\pgfpathlineto{\pgfqpoint{2.469078in}{1.094945in}}%
\pgfpathlineto{\pgfqpoint{2.336483in}{0.949464in}}%
\pgfpathlineto{\pgfqpoint{2.495569in}{1.018707in}}%
\pgfpathclose%
\pgfusepath{fill}%
\end{pgfscope}%
\begin{pgfscope}%
\pgfpathrectangle{\pgfqpoint{0.254231in}{0.147348in}}{\pgfqpoint{2.735294in}{2.735294in}}%
\pgfusepath{clip}%
\pgfsetbuttcap%
\pgfsetroundjoin%
\definecolor{currentfill}{rgb}{0.116321,0.422987,0.116321}%
\pgfsetfillcolor{currentfill}%
\pgfsetlinewidth{0.000000pt}%
\definecolor{currentstroke}{rgb}{0.000000,0.000000,0.000000}%
\pgfsetstrokecolor{currentstroke}%
\pgfsetdash{}{0pt}%
\pgfpathmoveto{\pgfqpoint{1.755263in}{2.632943in}}%
\pgfpathlineto{\pgfqpoint{1.562421in}{2.632943in}}%
\pgfpathlineto{\pgfqpoint{1.520550in}{2.150689in}}%
\pgfpathlineto{\pgfqpoint{1.755263in}{2.632943in}}%
\pgfpathclose%
\pgfusepath{fill}%
\end{pgfscope}%
\begin{pgfscope}%
\pgfpathrectangle{\pgfqpoint{0.254231in}{0.147348in}}{\pgfqpoint{2.735294in}{2.735294in}}%
\pgfusepath{clip}%
\pgfsetbuttcap%
\pgfsetroundjoin%
\definecolor{currentfill}{rgb}{0.067061,0.243857,0.067061}%
\pgfsetfillcolor{currentfill}%
\pgfsetlinewidth{0.000000pt}%
\definecolor{currentstroke}{rgb}{0.000000,0.000000,0.000000}%
\pgfsetstrokecolor{currentstroke}%
\pgfsetdash{}{0pt}%
\pgfpathmoveto{\pgfqpoint{0.707491in}{1.163589in}}%
\pgfpathlineto{\pgfqpoint{0.822114in}{1.018707in}}%
\pgfpathlineto{\pgfqpoint{0.857590in}{1.557460in}}%
\pgfpathlineto{\pgfqpoint{0.707491in}{1.163589in}}%
\pgfpathclose%
\pgfusepath{fill}%
\end{pgfscope}%
\begin{pgfscope}%
\pgfpathrectangle{\pgfqpoint{0.254231in}{0.147348in}}{\pgfqpoint{2.735294in}{2.735294in}}%
\pgfusepath{clip}%
\pgfsetbuttcap%
\pgfsetroundjoin%
\definecolor{currentfill}{rgb}{0.067061,0.243857,0.067061}%
\pgfsetfillcolor{currentfill}%
\pgfsetlinewidth{0.000000pt}%
\definecolor{currentstroke}{rgb}{0.000000,0.000000,0.000000}%
\pgfsetstrokecolor{currentstroke}%
\pgfsetdash{}{0pt}%
\pgfpathmoveto{\pgfqpoint{2.460094in}{1.557460in}}%
\pgfpathlineto{\pgfqpoint{2.495569in}{1.018707in}}%
\pgfpathlineto{\pgfqpoint{2.610193in}{1.163589in}}%
\pgfpathlineto{\pgfqpoint{2.460094in}{1.557460in}}%
\pgfpathclose%
\pgfusepath{fill}%
\end{pgfscope}%
\begin{pgfscope}%
\pgfpathrectangle{\pgfqpoint{0.254231in}{0.147348in}}{\pgfqpoint{2.735294in}{2.735294in}}%
\pgfusepath{clip}%
\pgfsetbuttcap%
\pgfsetroundjoin%
\definecolor{currentfill}{rgb}{0.095351,0.346729,0.095351}%
\pgfsetfillcolor{currentfill}%
\pgfsetlinewidth{0.000000pt}%
\definecolor{currentstroke}{rgb}{0.000000,0.000000,0.000000}%
\pgfsetstrokecolor{currentstroke}%
\pgfsetdash{}{0pt}%
\pgfpathmoveto{\pgfqpoint{0.857590in}{1.557460in}}%
\pgfpathlineto{\pgfqpoint{0.822114in}{1.018707in}}%
\pgfpathlineto{\pgfqpoint{0.848606in}{1.094945in}}%
\pgfpathlineto{\pgfqpoint{0.857590in}{1.557460in}}%
\pgfpathclose%
\pgfusepath{fill}%
\end{pgfscope}%
\begin{pgfscope}%
\pgfpathrectangle{\pgfqpoint{0.254231in}{0.147348in}}{\pgfqpoint{2.735294in}{2.735294in}}%
\pgfusepath{clip}%
\pgfsetbuttcap%
\pgfsetroundjoin%
\definecolor{currentfill}{rgb}{0.095351,0.346729,0.095351}%
\pgfsetfillcolor{currentfill}%
\pgfsetlinewidth{0.000000pt}%
\definecolor{currentstroke}{rgb}{0.000000,0.000000,0.000000}%
\pgfsetstrokecolor{currentstroke}%
\pgfsetdash{}{0pt}%
\pgfpathmoveto{\pgfqpoint{2.469078in}{1.094945in}}%
\pgfpathlineto{\pgfqpoint{2.495569in}{1.018707in}}%
\pgfpathlineto{\pgfqpoint{2.460094in}{1.557460in}}%
\pgfpathlineto{\pgfqpoint{2.469078in}{1.094945in}}%
\pgfpathclose%
\pgfusepath{fill}%
\end{pgfscope}%
\begin{pgfscope}%
\pgfpathrectangle{\pgfqpoint{0.254231in}{0.147348in}}{\pgfqpoint{2.735294in}{2.735294in}}%
\pgfusepath{clip}%
\pgfsetbuttcap%
\pgfsetroundjoin%
\definecolor{currentfill}{rgb}{0.060435,0.219763,0.060435}%
\pgfsetfillcolor{currentfill}%
\pgfsetlinewidth{0.000000pt}%
\definecolor{currentstroke}{rgb}{0.000000,0.000000,0.000000}%
\pgfsetstrokecolor{currentstroke}%
\pgfsetdash{}{0pt}%
\pgfpathmoveto{\pgfqpoint{2.286012in}{1.028737in}}%
\pgfpathlineto{\pgfqpoint{2.139898in}{0.888724in}}%
\pgfpathlineto{\pgfqpoint{2.336483in}{0.949464in}}%
\pgfpathlineto{\pgfqpoint{2.286012in}{1.028737in}}%
\pgfpathclose%
\pgfusepath{fill}%
\end{pgfscope}%
\begin{pgfscope}%
\pgfpathrectangle{\pgfqpoint{0.254231in}{0.147348in}}{\pgfqpoint{2.735294in}{2.735294in}}%
\pgfusepath{clip}%
\pgfsetbuttcap%
\pgfsetroundjoin%
\definecolor{currentfill}{rgb}{0.060435,0.219763,0.060435}%
\pgfsetfillcolor{currentfill}%
\pgfsetlinewidth{0.000000pt}%
\definecolor{currentstroke}{rgb}{0.000000,0.000000,0.000000}%
\pgfsetstrokecolor{currentstroke}%
\pgfsetdash{}{0pt}%
\pgfpathmoveto{\pgfqpoint{0.981201in}{0.949464in}}%
\pgfpathlineto{\pgfqpoint{1.177786in}{0.888724in}}%
\pgfpathlineto{\pgfqpoint{1.031671in}{1.028737in}}%
\pgfpathlineto{\pgfqpoint{0.981201in}{0.949464in}}%
\pgfpathclose%
\pgfusepath{fill}%
\end{pgfscope}%
\begin{pgfscope}%
\pgfpathrectangle{\pgfqpoint{0.254231in}{0.147348in}}{\pgfqpoint{2.735294in}{2.735294in}}%
\pgfusepath{clip}%
\pgfsetbuttcap%
\pgfsetroundjoin%
\definecolor{currentfill}{rgb}{0.074506,0.270932,0.074506}%
\pgfsetfillcolor{currentfill}%
\pgfsetlinewidth{0.000000pt}%
\definecolor{currentstroke}{rgb}{0.000000,0.000000,0.000000}%
\pgfsetstrokecolor{currentstroke}%
\pgfsetdash{}{0pt}%
\pgfpathmoveto{\pgfqpoint{0.695386in}{1.589726in}}%
\pgfpathlineto{\pgfqpoint{0.707491in}{1.163589in}}%
\pgfpathlineto{\pgfqpoint{0.857590in}{1.557460in}}%
\pgfpathlineto{\pgfqpoint{0.695386in}{1.589726in}}%
\pgfpathclose%
\pgfusepath{fill}%
\end{pgfscope}%
\begin{pgfscope}%
\pgfpathrectangle{\pgfqpoint{0.254231in}{0.147348in}}{\pgfqpoint{2.735294in}{2.735294in}}%
\pgfusepath{clip}%
\pgfsetbuttcap%
\pgfsetroundjoin%
\definecolor{currentfill}{rgb}{0.074506,0.270932,0.074506}%
\pgfsetfillcolor{currentfill}%
\pgfsetlinewidth{0.000000pt}%
\definecolor{currentstroke}{rgb}{0.000000,0.000000,0.000000}%
\pgfsetstrokecolor{currentstroke}%
\pgfsetdash{}{0pt}%
\pgfpathmoveto{\pgfqpoint{2.460094in}{1.557460in}}%
\pgfpathlineto{\pgfqpoint{2.610193in}{1.163589in}}%
\pgfpathlineto{\pgfqpoint{2.622298in}{1.589726in}}%
\pgfpathlineto{\pgfqpoint{2.460094in}{1.557460in}}%
\pgfpathclose%
\pgfusepath{fill}%
\end{pgfscope}%
\begin{pgfscope}%
\pgfpathrectangle{\pgfqpoint{0.254231in}{0.147348in}}{\pgfqpoint{2.735294in}{2.735294in}}%
\pgfusepath{clip}%
\pgfsetbuttcap%
\pgfsetroundjoin%
\definecolor{currentfill}{rgb}{0.116785,0.424671,0.116785}%
\pgfsetfillcolor{currentfill}%
\pgfsetlinewidth{0.000000pt}%
\definecolor{currentstroke}{rgb}{0.000000,0.000000,0.000000}%
\pgfsetstrokecolor{currentstroke}%
\pgfsetdash{}{0pt}%
\pgfpathmoveto{\pgfqpoint{2.141244in}{2.292690in}}%
\pgfpathlineto{\pgfqpoint{1.755263in}{2.632943in}}%
\pgfpathlineto{\pgfqpoint{1.797134in}{2.150689in}}%
\pgfpathlineto{\pgfqpoint{2.141244in}{2.292690in}}%
\pgfpathclose%
\pgfusepath{fill}%
\end{pgfscope}%
\begin{pgfscope}%
\pgfpathrectangle{\pgfqpoint{0.254231in}{0.147348in}}{\pgfqpoint{2.735294in}{2.735294in}}%
\pgfusepath{clip}%
\pgfsetbuttcap%
\pgfsetroundjoin%
\definecolor{currentfill}{rgb}{0.116785,0.424671,0.116785}%
\pgfsetfillcolor{currentfill}%
\pgfsetlinewidth{0.000000pt}%
\definecolor{currentstroke}{rgb}{0.000000,0.000000,0.000000}%
\pgfsetstrokecolor{currentstroke}%
\pgfsetdash{}{0pt}%
\pgfpathmoveto{\pgfqpoint{1.520550in}{2.150689in}}%
\pgfpathlineto{\pgfqpoint{1.562421in}{2.632943in}}%
\pgfpathlineto{\pgfqpoint{1.176440in}{2.292690in}}%
\pgfpathlineto{\pgfqpoint{1.520550in}{2.150689in}}%
\pgfpathclose%
\pgfusepath{fill}%
\end{pgfscope}%
\begin{pgfscope}%
\pgfpathrectangle{\pgfqpoint{0.254231in}{0.147348in}}{\pgfqpoint{2.735294in}{2.735294in}}%
\pgfusepath{clip}%
\pgfsetbuttcap%
\pgfsetroundjoin%
\definecolor{currentfill}{rgb}{0.064867,0.235879,0.064867}%
\pgfsetfillcolor{currentfill}%
\pgfsetlinewidth{0.000000pt}%
\definecolor{currentstroke}{rgb}{0.000000,0.000000,0.000000}%
\pgfsetstrokecolor{currentstroke}%
\pgfsetdash{}{0pt}%
\pgfpathmoveto{\pgfqpoint{0.848606in}{1.094945in}}%
\pgfpathlineto{\pgfqpoint{0.981201in}{0.949464in}}%
\pgfpathlineto{\pgfqpoint{1.075030in}{1.526914in}}%
\pgfpathlineto{\pgfqpoint{0.848606in}{1.094945in}}%
\pgfpathclose%
\pgfusepath{fill}%
\end{pgfscope}%
\begin{pgfscope}%
\pgfpathrectangle{\pgfqpoint{0.254231in}{0.147348in}}{\pgfqpoint{2.735294in}{2.735294in}}%
\pgfusepath{clip}%
\pgfsetbuttcap%
\pgfsetroundjoin%
\definecolor{currentfill}{rgb}{0.064867,0.235879,0.064867}%
\pgfsetfillcolor{currentfill}%
\pgfsetlinewidth{0.000000pt}%
\definecolor{currentstroke}{rgb}{0.000000,0.000000,0.000000}%
\pgfsetstrokecolor{currentstroke}%
\pgfsetdash{}{0pt}%
\pgfpathmoveto{\pgfqpoint{2.242654in}{1.526914in}}%
\pgfpathlineto{\pgfqpoint{2.336483in}{0.949464in}}%
\pgfpathlineto{\pgfqpoint{2.469078in}{1.094945in}}%
\pgfpathlineto{\pgfqpoint{2.242654in}{1.526914in}}%
\pgfpathclose%
\pgfusepath{fill}%
\end{pgfscope}%
\begin{pgfscope}%
\pgfpathrectangle{\pgfqpoint{0.254231in}{0.147348in}}{\pgfqpoint{2.735294in}{2.735294in}}%
\pgfusepath{clip}%
\pgfsetbuttcap%
\pgfsetroundjoin%
\definecolor{currentfill}{rgb}{0.057724,0.209904,0.057724}%
\pgfsetfillcolor{currentfill}%
\pgfsetlinewidth{0.000000pt}%
\definecolor{currentstroke}{rgb}{0.000000,0.000000,0.000000}%
\pgfsetstrokecolor{currentstroke}%
\pgfsetdash{}{0pt}%
\pgfpathmoveto{\pgfqpoint{1.258681in}{0.974376in}}%
\pgfpathlineto{\pgfqpoint{1.177786in}{0.888724in}}%
\pgfpathlineto{\pgfqpoint{1.408065in}{0.846223in}}%
\pgfpathlineto{\pgfqpoint{1.258681in}{0.974376in}}%
\pgfpathclose%
\pgfusepath{fill}%
\end{pgfscope}%
\begin{pgfscope}%
\pgfpathrectangle{\pgfqpoint{0.254231in}{0.147348in}}{\pgfqpoint{2.735294in}{2.735294in}}%
\pgfusepath{clip}%
\pgfsetbuttcap%
\pgfsetroundjoin%
\definecolor{currentfill}{rgb}{0.057724,0.209904,0.057724}%
\pgfsetfillcolor{currentfill}%
\pgfsetlinewidth{0.000000pt}%
\definecolor{currentstroke}{rgb}{0.000000,0.000000,0.000000}%
\pgfsetstrokecolor{currentstroke}%
\pgfsetdash{}{0pt}%
\pgfpathmoveto{\pgfqpoint{1.909619in}{0.846223in}}%
\pgfpathlineto{\pgfqpoint{2.139898in}{0.888724in}}%
\pgfpathlineto{\pgfqpoint{2.059003in}{0.974376in}}%
\pgfpathlineto{\pgfqpoint{1.909619in}{0.846223in}}%
\pgfpathclose%
\pgfusepath{fill}%
\end{pgfscope}%
\begin{pgfscope}%
\pgfpathrectangle{\pgfqpoint{0.254231in}{0.147348in}}{\pgfqpoint{2.735294in}{2.735294in}}%
\pgfusepath{clip}%
\pgfsetbuttcap%
\pgfsetroundjoin%
\definecolor{currentfill}{rgb}{0.086498,0.314539,0.086498}%
\pgfsetfillcolor{currentfill}%
\pgfsetlinewidth{0.000000pt}%
\definecolor{currentstroke}{rgb}{0.000000,0.000000,0.000000}%
\pgfsetstrokecolor{currentstroke}%
\pgfsetdash{}{0pt}%
\pgfpathmoveto{\pgfqpoint{0.857590in}{1.557460in}}%
\pgfpathlineto{\pgfqpoint{0.779523in}{1.760124in}}%
\pgfpathlineto{\pgfqpoint{0.695386in}{1.589726in}}%
\pgfpathlineto{\pgfqpoint{0.857590in}{1.557460in}}%
\pgfpathclose%
\pgfusepath{fill}%
\end{pgfscope}%
\begin{pgfscope}%
\pgfpathrectangle{\pgfqpoint{0.254231in}{0.147348in}}{\pgfqpoint{2.735294in}{2.735294in}}%
\pgfusepath{clip}%
\pgfsetbuttcap%
\pgfsetroundjoin%
\definecolor{currentfill}{rgb}{0.086498,0.314539,0.086498}%
\pgfsetfillcolor{currentfill}%
\pgfsetlinewidth{0.000000pt}%
\definecolor{currentstroke}{rgb}{0.000000,0.000000,0.000000}%
\pgfsetstrokecolor{currentstroke}%
\pgfsetdash{}{0pt}%
\pgfpathmoveto{\pgfqpoint{2.622298in}{1.589726in}}%
\pgfpathlineto{\pgfqpoint{2.538161in}{1.760124in}}%
\pgfpathlineto{\pgfqpoint{2.460094in}{1.557460in}}%
\pgfpathlineto{\pgfqpoint{2.622298in}{1.589726in}}%
\pgfpathclose%
\pgfusepath{fill}%
\end{pgfscope}%
\begin{pgfscope}%
\pgfpathrectangle{\pgfqpoint{0.254231in}{0.147348in}}{\pgfqpoint{2.735294in}{2.735294in}}%
\pgfusepath{clip}%
\pgfsetbuttcap%
\pgfsetroundjoin%
\definecolor{currentfill}{rgb}{0.090812,0.330224,0.090812}%
\pgfsetfillcolor{currentfill}%
\pgfsetlinewidth{0.000000pt}%
\definecolor{currentstroke}{rgb}{0.000000,0.000000,0.000000}%
\pgfsetstrokecolor{currentstroke}%
\pgfsetdash{}{0pt}%
\pgfpathmoveto{\pgfqpoint{1.075030in}{1.526914in}}%
\pgfpathlineto{\pgfqpoint{0.981201in}{0.949464in}}%
\pgfpathlineto{\pgfqpoint{1.031671in}{1.028737in}}%
\pgfpathlineto{\pgfqpoint{1.075030in}{1.526914in}}%
\pgfpathclose%
\pgfusepath{fill}%
\end{pgfscope}%
\begin{pgfscope}%
\pgfpathrectangle{\pgfqpoint{0.254231in}{0.147348in}}{\pgfqpoint{2.735294in}{2.735294in}}%
\pgfusepath{clip}%
\pgfsetbuttcap%
\pgfsetroundjoin%
\definecolor{currentfill}{rgb}{0.090812,0.330224,0.090812}%
\pgfsetfillcolor{currentfill}%
\pgfsetlinewidth{0.000000pt}%
\definecolor{currentstroke}{rgb}{0.000000,0.000000,0.000000}%
\pgfsetstrokecolor{currentstroke}%
\pgfsetdash{}{0pt}%
\pgfpathmoveto{\pgfqpoint{2.286012in}{1.028737in}}%
\pgfpathlineto{\pgfqpoint{2.336483in}{0.949464in}}%
\pgfpathlineto{\pgfqpoint{2.242654in}{1.526914in}}%
\pgfpathlineto{\pgfqpoint{2.286012in}{1.028737in}}%
\pgfpathclose%
\pgfusepath{fill}%
\end{pgfscope}%
\begin{pgfscope}%
\pgfpathrectangle{\pgfqpoint{0.254231in}{0.147348in}}{\pgfqpoint{2.735294in}{2.735294in}}%
\pgfusepath{clip}%
\pgfsetbuttcap%
\pgfsetroundjoin%
\definecolor{currentfill}{rgb}{0.116321,0.422987,0.116321}%
\pgfsetfillcolor{currentfill}%
\pgfsetlinewidth{0.000000pt}%
\definecolor{currentstroke}{rgb}{0.000000,0.000000,0.000000}%
\pgfsetstrokecolor{currentstroke}%
\pgfsetdash{}{0pt}%
\pgfpathmoveto{\pgfqpoint{1.520550in}{2.150689in}}%
\pgfpathlineto{\pgfqpoint{1.797134in}{2.150689in}}%
\pgfpathlineto{\pgfqpoint{1.755263in}{2.632943in}}%
\pgfpathlineto{\pgfqpoint{1.520550in}{2.150689in}}%
\pgfpathclose%
\pgfusepath{fill}%
\end{pgfscope}%
\begin{pgfscope}%
\pgfpathrectangle{\pgfqpoint{0.254231in}{0.147348in}}{\pgfqpoint{2.735294in}{2.735294in}}%
\pgfusepath{clip}%
\pgfsetbuttcap%
\pgfsetroundjoin%
\definecolor{currentfill}{rgb}{0.056200,0.204363,0.056200}%
\pgfsetfillcolor{currentfill}%
\pgfsetlinewidth{0.000000pt}%
\definecolor{currentstroke}{rgb}{0.000000,0.000000,0.000000}%
\pgfsetstrokecolor{currentstroke}%
\pgfsetdash{}{0pt}%
\pgfpathmoveto{\pgfqpoint{1.658842in}{0.830831in}}%
\pgfpathlineto{\pgfqpoint{1.520882in}{0.943120in}}%
\pgfpathlineto{\pgfqpoint{1.408065in}{0.846223in}}%
\pgfpathlineto{\pgfqpoint{1.658842in}{0.830831in}}%
\pgfpathclose%
\pgfusepath{fill}%
\end{pgfscope}%
\begin{pgfscope}%
\pgfpathrectangle{\pgfqpoint{0.254231in}{0.147348in}}{\pgfqpoint{2.735294in}{2.735294in}}%
\pgfusepath{clip}%
\pgfsetbuttcap%
\pgfsetroundjoin%
\definecolor{currentfill}{rgb}{0.056200,0.204363,0.056200}%
\pgfsetfillcolor{currentfill}%
\pgfsetlinewidth{0.000000pt}%
\definecolor{currentstroke}{rgb}{0.000000,0.000000,0.000000}%
\pgfsetstrokecolor{currentstroke}%
\pgfsetdash{}{0pt}%
\pgfpathmoveto{\pgfqpoint{1.909619in}{0.846223in}}%
\pgfpathlineto{\pgfqpoint{1.796802in}{0.943120in}}%
\pgfpathlineto{\pgfqpoint{1.658842in}{0.830831in}}%
\pgfpathlineto{\pgfqpoint{1.909619in}{0.846223in}}%
\pgfpathclose%
\pgfusepath{fill}%
\end{pgfscope}%
\begin{pgfscope}%
\pgfpathrectangle{\pgfqpoint{0.254231in}{0.147348in}}{\pgfqpoint{2.735294in}{2.735294in}}%
\pgfusepath{clip}%
\pgfsetbuttcap%
\pgfsetroundjoin%
\definecolor{currentfill}{rgb}{0.107070,0.389346,0.107070}%
\pgfsetfillcolor{currentfill}%
\pgfsetlinewidth{0.000000pt}%
\definecolor{currentstroke}{rgb}{0.000000,0.000000,0.000000}%
\pgfsetstrokecolor{currentstroke}%
\pgfsetdash{}{0pt}%
\pgfpathmoveto{\pgfqpoint{1.257750in}{2.141900in}}%
\pgfpathlineto{\pgfqpoint{1.176440in}{2.292690in}}%
\pgfpathlineto{\pgfqpoint{1.030299in}{2.126617in}}%
\pgfpathlineto{\pgfqpoint{1.257750in}{2.141900in}}%
\pgfpathclose%
\pgfusepath{fill}%
\end{pgfscope}%
\begin{pgfscope}%
\pgfpathrectangle{\pgfqpoint{0.254231in}{0.147348in}}{\pgfqpoint{2.735294in}{2.735294in}}%
\pgfusepath{clip}%
\pgfsetbuttcap%
\pgfsetroundjoin%
\definecolor{currentfill}{rgb}{0.107070,0.389346,0.107070}%
\pgfsetfillcolor{currentfill}%
\pgfsetlinewidth{0.000000pt}%
\definecolor{currentstroke}{rgb}{0.000000,0.000000,0.000000}%
\pgfsetstrokecolor{currentstroke}%
\pgfsetdash{}{0pt}%
\pgfpathmoveto{\pgfqpoint{2.287385in}{2.126617in}}%
\pgfpathlineto{\pgfqpoint{2.141244in}{2.292690in}}%
\pgfpathlineto{\pgfqpoint{2.059934in}{2.141900in}}%
\pgfpathlineto{\pgfqpoint{2.287385in}{2.126617in}}%
\pgfpathclose%
\pgfusepath{fill}%
\end{pgfscope}%
\begin{pgfscope}%
\pgfpathrectangle{\pgfqpoint{0.254231in}{0.147348in}}{\pgfqpoint{2.735294in}{2.735294in}}%
\pgfusepath{clip}%
\pgfsetbuttcap%
\pgfsetroundjoin%
\definecolor{currentfill}{rgb}{0.089078,0.323920,0.089078}%
\pgfsetfillcolor{currentfill}%
\pgfsetlinewidth{0.000000pt}%
\definecolor{currentstroke}{rgb}{0.000000,0.000000,0.000000}%
\pgfsetstrokecolor{currentstroke}%
\pgfsetdash{}{0pt}%
\pgfpathmoveto{\pgfqpoint{0.857590in}{1.557460in}}%
\pgfpathlineto{\pgfqpoint{1.030299in}{2.126617in}}%
\pgfpathlineto{\pgfqpoint{0.779523in}{1.760124in}}%
\pgfpathlineto{\pgfqpoint{0.857590in}{1.557460in}}%
\pgfpathclose%
\pgfusepath{fill}%
\end{pgfscope}%
\begin{pgfscope}%
\pgfpathrectangle{\pgfqpoint{0.254231in}{0.147348in}}{\pgfqpoint{2.735294in}{2.735294in}}%
\pgfusepath{clip}%
\pgfsetbuttcap%
\pgfsetroundjoin%
\definecolor{currentfill}{rgb}{0.089078,0.323920,0.089078}%
\pgfsetfillcolor{currentfill}%
\pgfsetlinewidth{0.000000pt}%
\definecolor{currentstroke}{rgb}{0.000000,0.000000,0.000000}%
\pgfsetstrokecolor{currentstroke}%
\pgfsetdash{}{0pt}%
\pgfpathmoveto{\pgfqpoint{2.538161in}{1.760124in}}%
\pgfpathlineto{\pgfqpoint{2.287385in}{2.126617in}}%
\pgfpathlineto{\pgfqpoint{2.460094in}{1.557460in}}%
\pgfpathlineto{\pgfqpoint{2.538161in}{1.760124in}}%
\pgfpathclose%
\pgfusepath{fill}%
\end{pgfscope}%
\begin{pgfscope}%
\pgfpathrectangle{\pgfqpoint{0.254231in}{0.147348in}}{\pgfqpoint{2.735294in}{2.735294in}}%
\pgfusepath{clip}%
\pgfsetbuttcap%
\pgfsetroundjoin%
\definecolor{currentfill}{rgb}{0.061754,0.224559,0.061754}%
\pgfsetfillcolor{currentfill}%
\pgfsetlinewidth{0.000000pt}%
\definecolor{currentstroke}{rgb}{0.000000,0.000000,0.000000}%
\pgfsetstrokecolor{currentstroke}%
\pgfsetdash{}{0pt}%
\pgfpathmoveto{\pgfqpoint{1.031671in}{1.028737in}}%
\pgfpathlineto{\pgfqpoint{1.177786in}{0.888724in}}%
\pgfpathlineto{\pgfqpoint{1.212441in}{1.302943in}}%
\pgfpathlineto{\pgfqpoint{1.031671in}{1.028737in}}%
\pgfpathclose%
\pgfusepath{fill}%
\end{pgfscope}%
\begin{pgfscope}%
\pgfpathrectangle{\pgfqpoint{0.254231in}{0.147348in}}{\pgfqpoint{2.735294in}{2.735294in}}%
\pgfusepath{clip}%
\pgfsetbuttcap%
\pgfsetroundjoin%
\definecolor{currentfill}{rgb}{0.061754,0.224559,0.061754}%
\pgfsetfillcolor{currentfill}%
\pgfsetlinewidth{0.000000pt}%
\definecolor{currentstroke}{rgb}{0.000000,0.000000,0.000000}%
\pgfsetstrokecolor{currentstroke}%
\pgfsetdash{}{0pt}%
\pgfpathmoveto{\pgfqpoint{2.105243in}{1.302943in}}%
\pgfpathlineto{\pgfqpoint{2.139898in}{0.888724in}}%
\pgfpathlineto{\pgfqpoint{2.286012in}{1.028737in}}%
\pgfpathlineto{\pgfqpoint{2.105243in}{1.302943in}}%
\pgfpathclose%
\pgfusepath{fill}%
\end{pgfscope}%
\begin{pgfscope}%
\pgfpathrectangle{\pgfqpoint{0.254231in}{0.147348in}}{\pgfqpoint{2.735294in}{2.735294in}}%
\pgfusepath{clip}%
\pgfsetbuttcap%
\pgfsetroundjoin%
\definecolor{currentfill}{rgb}{0.070984,0.258123,0.070984}%
\pgfsetfillcolor{currentfill}%
\pgfsetlinewidth{0.000000pt}%
\definecolor{currentstroke}{rgb}{0.000000,0.000000,0.000000}%
\pgfsetstrokecolor{currentstroke}%
\pgfsetdash{}{0pt}%
\pgfpathmoveto{\pgfqpoint{0.857590in}{1.557460in}}%
\pgfpathlineto{\pgfqpoint{0.848606in}{1.094945in}}%
\pgfpathlineto{\pgfqpoint{1.075030in}{1.526914in}}%
\pgfpathlineto{\pgfqpoint{0.857590in}{1.557460in}}%
\pgfpathclose%
\pgfusepath{fill}%
\end{pgfscope}%
\begin{pgfscope}%
\pgfpathrectangle{\pgfqpoint{0.254231in}{0.147348in}}{\pgfqpoint{2.735294in}{2.735294in}}%
\pgfusepath{clip}%
\pgfsetbuttcap%
\pgfsetroundjoin%
\definecolor{currentfill}{rgb}{0.070984,0.258123,0.070984}%
\pgfsetfillcolor{currentfill}%
\pgfsetlinewidth{0.000000pt}%
\definecolor{currentstroke}{rgb}{0.000000,0.000000,0.000000}%
\pgfsetstrokecolor{currentstroke}%
\pgfsetdash{}{0pt}%
\pgfpathmoveto{\pgfqpoint{2.242654in}{1.526914in}}%
\pgfpathlineto{\pgfqpoint{2.469078in}{1.094945in}}%
\pgfpathlineto{\pgfqpoint{2.460094in}{1.557460in}}%
\pgfpathlineto{\pgfqpoint{2.242654in}{1.526914in}}%
\pgfpathclose%
\pgfusepath{fill}%
\end{pgfscope}%
\begin{pgfscope}%
\pgfpathrectangle{\pgfqpoint{0.254231in}{0.147348in}}{\pgfqpoint{2.735294in}{2.735294in}}%
\pgfusepath{clip}%
\pgfsetbuttcap%
\pgfsetroundjoin%
\definecolor{currentfill}{rgb}{0.070885,0.257762,0.070885}%
\pgfsetfillcolor{currentfill}%
\pgfsetlinewidth{0.000000pt}%
\definecolor{currentstroke}{rgb}{0.000000,0.000000,0.000000}%
\pgfsetstrokecolor{currentstroke}%
\pgfsetdash{}{0pt}%
\pgfpathmoveto{\pgfqpoint{2.059003in}{0.974376in}}%
\pgfpathlineto{\pgfqpoint{2.139898in}{0.888724in}}%
\pgfpathlineto{\pgfqpoint{2.105243in}{1.302943in}}%
\pgfpathlineto{\pgfqpoint{2.059003in}{0.974376in}}%
\pgfpathclose%
\pgfusepath{fill}%
\end{pgfscope}%
\begin{pgfscope}%
\pgfpathrectangle{\pgfqpoint{0.254231in}{0.147348in}}{\pgfqpoint{2.735294in}{2.735294in}}%
\pgfusepath{clip}%
\pgfsetbuttcap%
\pgfsetroundjoin%
\definecolor{currentfill}{rgb}{0.070885,0.257762,0.070885}%
\pgfsetfillcolor{currentfill}%
\pgfsetlinewidth{0.000000pt}%
\definecolor{currentstroke}{rgb}{0.000000,0.000000,0.000000}%
\pgfsetstrokecolor{currentstroke}%
\pgfsetdash{}{0pt}%
\pgfpathmoveto{\pgfqpoint{1.212441in}{1.302943in}}%
\pgfpathlineto{\pgfqpoint{1.177786in}{0.888724in}}%
\pgfpathlineto{\pgfqpoint{1.258681in}{0.974376in}}%
\pgfpathlineto{\pgfqpoint{1.212441in}{1.302943in}}%
\pgfpathclose%
\pgfusepath{fill}%
\end{pgfscope}%
\begin{pgfscope}%
\pgfpathrectangle{\pgfqpoint{0.254231in}{0.147348in}}{\pgfqpoint{2.735294in}{2.735294in}}%
\pgfusepath{clip}%
\pgfsetbuttcap%
\pgfsetroundjoin%
\definecolor{currentfill}{rgb}{0.111651,0.406004,0.111651}%
\pgfsetfillcolor{currentfill}%
\pgfsetlinewidth{0.000000pt}%
\definecolor{currentstroke}{rgb}{0.000000,0.000000,0.000000}%
\pgfsetstrokecolor{currentstroke}%
\pgfsetdash{}{0pt}%
\pgfpathmoveto{\pgfqpoint{2.059934in}{2.141900in}}%
\pgfpathlineto{\pgfqpoint{2.141244in}{2.292690in}}%
\pgfpathlineto{\pgfqpoint{1.797134in}{2.150689in}}%
\pgfpathlineto{\pgfqpoint{2.059934in}{2.141900in}}%
\pgfpathclose%
\pgfusepath{fill}%
\end{pgfscope}%
\begin{pgfscope}%
\pgfpathrectangle{\pgfqpoint{0.254231in}{0.147348in}}{\pgfqpoint{2.735294in}{2.735294in}}%
\pgfusepath{clip}%
\pgfsetbuttcap%
\pgfsetroundjoin%
\definecolor{currentfill}{rgb}{0.111651,0.406004,0.111651}%
\pgfsetfillcolor{currentfill}%
\pgfsetlinewidth{0.000000pt}%
\definecolor{currentstroke}{rgb}{0.000000,0.000000,0.000000}%
\pgfsetstrokecolor{currentstroke}%
\pgfsetdash{}{0pt}%
\pgfpathmoveto{\pgfqpoint{1.520550in}{2.150689in}}%
\pgfpathlineto{\pgfqpoint{1.176440in}{2.292690in}}%
\pgfpathlineto{\pgfqpoint{1.257750in}{2.141900in}}%
\pgfpathlineto{\pgfqpoint{1.520550in}{2.150689in}}%
\pgfpathclose%
\pgfusepath{fill}%
\end{pgfscope}%
\begin{pgfscope}%
\pgfpathrectangle{\pgfqpoint{0.254231in}{0.147348in}}{\pgfqpoint{2.735294in}{2.735294in}}%
\pgfusepath{clip}%
\pgfsetbuttcap%
\pgfsetroundjoin%
\definecolor{currentfill}{rgb}{0.092193,0.335248,0.092193}%
\pgfsetfillcolor{currentfill}%
\pgfsetlinewidth{0.000000pt}%
\definecolor{currentstroke}{rgb}{0.000000,0.000000,0.000000}%
\pgfsetstrokecolor{currentstroke}%
\pgfsetdash{}{0pt}%
\pgfpathmoveto{\pgfqpoint{1.075030in}{1.526914in}}%
\pgfpathlineto{\pgfqpoint{1.030299in}{2.126617in}}%
\pgfpathlineto{\pgfqpoint{0.857590in}{1.557460in}}%
\pgfpathlineto{\pgfqpoint{1.075030in}{1.526914in}}%
\pgfpathclose%
\pgfusepath{fill}%
\end{pgfscope}%
\begin{pgfscope}%
\pgfpathrectangle{\pgfqpoint{0.254231in}{0.147348in}}{\pgfqpoint{2.735294in}{2.735294in}}%
\pgfusepath{clip}%
\pgfsetbuttcap%
\pgfsetroundjoin%
\definecolor{currentfill}{rgb}{0.092193,0.335248,0.092193}%
\pgfsetfillcolor{currentfill}%
\pgfsetlinewidth{0.000000pt}%
\definecolor{currentstroke}{rgb}{0.000000,0.000000,0.000000}%
\pgfsetstrokecolor{currentstroke}%
\pgfsetdash{}{0pt}%
\pgfpathmoveto{\pgfqpoint{2.460094in}{1.557460in}}%
\pgfpathlineto{\pgfqpoint{2.287385in}{2.126617in}}%
\pgfpathlineto{\pgfqpoint{2.242654in}{1.526914in}}%
\pgfpathlineto{\pgfqpoint{2.460094in}{1.557460in}}%
\pgfpathclose%
\pgfusepath{fill}%
\end{pgfscope}%
\begin{pgfscope}%
\pgfpathrectangle{\pgfqpoint{0.254231in}{0.147348in}}{\pgfqpoint{2.735294in}{2.735294in}}%
\pgfusepath{clip}%
\pgfsetbuttcap%
\pgfsetroundjoin%
\definecolor{currentfill}{rgb}{0.060562,0.220227,0.060562}%
\pgfsetfillcolor{currentfill}%
\pgfsetlinewidth{0.000000pt}%
\definecolor{currentstroke}{rgb}{0.000000,0.000000,0.000000}%
\pgfsetstrokecolor{currentstroke}%
\pgfsetdash{}{0pt}%
\pgfpathmoveto{\pgfqpoint{2.059003in}{0.974376in}}%
\pgfpathlineto{\pgfqpoint{1.814008in}{1.277996in}}%
\pgfpathlineto{\pgfqpoint{1.909619in}{0.846223in}}%
\pgfpathlineto{\pgfqpoint{2.059003in}{0.974376in}}%
\pgfpathclose%
\pgfusepath{fill}%
\end{pgfscope}%
\begin{pgfscope}%
\pgfpathrectangle{\pgfqpoint{0.254231in}{0.147348in}}{\pgfqpoint{2.735294in}{2.735294in}}%
\pgfusepath{clip}%
\pgfsetbuttcap%
\pgfsetroundjoin%
\definecolor{currentfill}{rgb}{0.060562,0.220227,0.060562}%
\pgfsetfillcolor{currentfill}%
\pgfsetlinewidth{0.000000pt}%
\definecolor{currentstroke}{rgb}{0.000000,0.000000,0.000000}%
\pgfsetstrokecolor{currentstroke}%
\pgfsetdash{}{0pt}%
\pgfpathmoveto{\pgfqpoint{1.408065in}{0.846223in}}%
\pgfpathlineto{\pgfqpoint{1.503676in}{1.277996in}}%
\pgfpathlineto{\pgfqpoint{1.258681in}{0.974376in}}%
\pgfpathlineto{\pgfqpoint{1.408065in}{0.846223in}}%
\pgfpathclose%
\pgfusepath{fill}%
\end{pgfscope}%
\begin{pgfscope}%
\pgfpathrectangle{\pgfqpoint{0.254231in}{0.147348in}}{\pgfqpoint{2.735294in}{2.735294in}}%
\pgfusepath{clip}%
\pgfsetbuttcap%
\pgfsetroundjoin%
\definecolor{currentfill}{rgb}{0.097285,0.353762,0.097285}%
\pgfsetfillcolor{currentfill}%
\pgfsetlinewidth{0.000000pt}%
\definecolor{currentstroke}{rgb}{0.000000,0.000000,0.000000}%
\pgfsetstrokecolor{currentstroke}%
\pgfsetdash{}{0pt}%
\pgfpathmoveto{\pgfqpoint{1.257750in}{2.141900in}}%
\pgfpathlineto{\pgfqpoint{1.030299in}{2.126617in}}%
\pgfpathlineto{\pgfqpoint{1.366689in}{1.952591in}}%
\pgfpathlineto{\pgfqpoint{1.257750in}{2.141900in}}%
\pgfpathclose%
\pgfusepath{fill}%
\end{pgfscope}%
\begin{pgfscope}%
\pgfpathrectangle{\pgfqpoint{0.254231in}{0.147348in}}{\pgfqpoint{2.735294in}{2.735294in}}%
\pgfusepath{clip}%
\pgfsetbuttcap%
\pgfsetroundjoin%
\definecolor{currentfill}{rgb}{0.097285,0.353762,0.097285}%
\pgfsetfillcolor{currentfill}%
\pgfsetlinewidth{0.000000pt}%
\definecolor{currentstroke}{rgb}{0.000000,0.000000,0.000000}%
\pgfsetstrokecolor{currentstroke}%
\pgfsetdash{}{0pt}%
\pgfpathmoveto{\pgfqpoint{1.950995in}{1.952591in}}%
\pgfpathlineto{\pgfqpoint{2.287385in}{2.126617in}}%
\pgfpathlineto{\pgfqpoint{2.059934in}{2.141900in}}%
\pgfpathlineto{\pgfqpoint{1.950995in}{1.952591in}}%
\pgfpathclose%
\pgfusepath{fill}%
\end{pgfscope}%
\begin{pgfscope}%
\pgfpathrectangle{\pgfqpoint{0.254231in}{0.147348in}}{\pgfqpoint{2.735294in}{2.735294in}}%
\pgfusepath{clip}%
\pgfsetbuttcap%
\pgfsetroundjoin%
\definecolor{currentfill}{rgb}{0.067497,0.245443,0.067497}%
\pgfsetfillcolor{currentfill}%
\pgfsetlinewidth{0.000000pt}%
\definecolor{currentstroke}{rgb}{0.000000,0.000000,0.000000}%
\pgfsetstrokecolor{currentstroke}%
\pgfsetdash{}{0pt}%
\pgfpathmoveto{\pgfqpoint{1.408065in}{0.846223in}}%
\pgfpathlineto{\pgfqpoint{1.520882in}{0.943120in}}%
\pgfpathlineto{\pgfqpoint{1.503676in}{1.277996in}}%
\pgfpathlineto{\pgfqpoint{1.408065in}{0.846223in}}%
\pgfpathclose%
\pgfusepath{fill}%
\end{pgfscope}%
\begin{pgfscope}%
\pgfpathrectangle{\pgfqpoint{0.254231in}{0.147348in}}{\pgfqpoint{2.735294in}{2.735294in}}%
\pgfusepath{clip}%
\pgfsetbuttcap%
\pgfsetroundjoin%
\definecolor{currentfill}{rgb}{0.067497,0.245443,0.067497}%
\pgfsetfillcolor{currentfill}%
\pgfsetlinewidth{0.000000pt}%
\definecolor{currentstroke}{rgb}{0.000000,0.000000,0.000000}%
\pgfsetstrokecolor{currentstroke}%
\pgfsetdash{}{0pt}%
\pgfpathmoveto{\pgfqpoint{1.814008in}{1.277996in}}%
\pgfpathlineto{\pgfqpoint{1.796802in}{0.943120in}}%
\pgfpathlineto{\pgfqpoint{1.909619in}{0.846223in}}%
\pgfpathlineto{\pgfqpoint{1.814008in}{1.277996in}}%
\pgfpathclose%
\pgfusepath{fill}%
\end{pgfscope}%
\begin{pgfscope}%
\pgfpathrectangle{\pgfqpoint{0.254231in}{0.147348in}}{\pgfqpoint{2.735294in}{2.735294in}}%
\pgfusepath{clip}%
\pgfsetbuttcap%
\pgfsetroundjoin%
\definecolor{currentfill}{rgb}{0.065434,0.237940,0.065434}%
\pgfsetfillcolor{currentfill}%
\pgfsetlinewidth{0.000000pt}%
\definecolor{currentstroke}{rgb}{0.000000,0.000000,0.000000}%
\pgfsetstrokecolor{currentstroke}%
\pgfsetdash{}{0pt}%
\pgfpathmoveto{\pgfqpoint{1.658842in}{0.830831in}}%
\pgfpathlineto{\pgfqpoint{1.503676in}{1.277996in}}%
\pgfpathlineto{\pgfqpoint{1.520882in}{0.943120in}}%
\pgfpathlineto{\pgfqpoint{1.658842in}{0.830831in}}%
\pgfpathclose%
\pgfusepath{fill}%
\end{pgfscope}%
\begin{pgfscope}%
\pgfpathrectangle{\pgfqpoint{0.254231in}{0.147348in}}{\pgfqpoint{2.735294in}{2.735294in}}%
\pgfusepath{clip}%
\pgfsetbuttcap%
\pgfsetroundjoin%
\definecolor{currentfill}{rgb}{0.065434,0.237940,0.065434}%
\pgfsetfillcolor{currentfill}%
\pgfsetlinewidth{0.000000pt}%
\definecolor{currentstroke}{rgb}{0.000000,0.000000,0.000000}%
\pgfsetstrokecolor{currentstroke}%
\pgfsetdash{}{0pt}%
\pgfpathmoveto{\pgfqpoint{1.796802in}{0.943120in}}%
\pgfpathlineto{\pgfqpoint{1.814008in}{1.277996in}}%
\pgfpathlineto{\pgfqpoint{1.658842in}{0.830831in}}%
\pgfpathlineto{\pgfqpoint{1.796802in}{0.943120in}}%
\pgfpathclose%
\pgfusepath{fill}%
\end{pgfscope}%
\begin{pgfscope}%
\pgfpathrectangle{\pgfqpoint{0.254231in}{0.147348in}}{\pgfqpoint{2.735294in}{2.735294in}}%
\pgfusepath{clip}%
\pgfsetbuttcap%
\pgfsetroundjoin%
\definecolor{currentfill}{rgb}{0.073593,0.267612,0.073593}%
\pgfsetfillcolor{currentfill}%
\pgfsetlinewidth{0.000000pt}%
\definecolor{currentstroke}{rgb}{0.000000,0.000000,0.000000}%
\pgfsetstrokecolor{currentstroke}%
\pgfsetdash{}{0pt}%
\pgfpathmoveto{\pgfqpoint{1.031671in}{1.028737in}}%
\pgfpathlineto{\pgfqpoint{1.212441in}{1.302943in}}%
\pgfpathlineto{\pgfqpoint{1.075030in}{1.526914in}}%
\pgfpathlineto{\pgfqpoint{1.031671in}{1.028737in}}%
\pgfpathclose%
\pgfusepath{fill}%
\end{pgfscope}%
\begin{pgfscope}%
\pgfpathrectangle{\pgfqpoint{0.254231in}{0.147348in}}{\pgfqpoint{2.735294in}{2.735294in}}%
\pgfusepath{clip}%
\pgfsetbuttcap%
\pgfsetroundjoin%
\definecolor{currentfill}{rgb}{0.073593,0.267612,0.073593}%
\pgfsetfillcolor{currentfill}%
\pgfsetlinewidth{0.000000pt}%
\definecolor{currentstroke}{rgb}{0.000000,0.000000,0.000000}%
\pgfsetstrokecolor{currentstroke}%
\pgfsetdash{}{0pt}%
\pgfpathmoveto{\pgfqpoint{2.242654in}{1.526914in}}%
\pgfpathlineto{\pgfqpoint{2.105243in}{1.302943in}}%
\pgfpathlineto{\pgfqpoint{2.286012in}{1.028737in}}%
\pgfpathlineto{\pgfqpoint{2.242654in}{1.526914in}}%
\pgfpathclose%
\pgfusepath{fill}%
\end{pgfscope}%
\begin{pgfscope}%
\pgfpathrectangle{\pgfqpoint{0.254231in}{0.147348in}}{\pgfqpoint{2.735294in}{2.735294in}}%
\pgfusepath{clip}%
\pgfsetbuttcap%
\pgfsetroundjoin%
\definecolor{currentfill}{rgb}{0.091915,0.334238,0.091915}%
\pgfsetfillcolor{currentfill}%
\pgfsetlinewidth{0.000000pt}%
\definecolor{currentstroke}{rgb}{0.000000,0.000000,0.000000}%
\pgfsetstrokecolor{currentstroke}%
\pgfsetdash{}{0pt}%
\pgfpathmoveto{\pgfqpoint{1.075030in}{1.526914in}}%
\pgfpathlineto{\pgfqpoint{1.366689in}{1.952591in}}%
\pgfpathlineto{\pgfqpoint{1.030299in}{2.126617in}}%
\pgfpathlineto{\pgfqpoint{1.075030in}{1.526914in}}%
\pgfpathclose%
\pgfusepath{fill}%
\end{pgfscope}%
\begin{pgfscope}%
\pgfpathrectangle{\pgfqpoint{0.254231in}{0.147348in}}{\pgfqpoint{2.735294in}{2.735294in}}%
\pgfusepath{clip}%
\pgfsetbuttcap%
\pgfsetroundjoin%
\definecolor{currentfill}{rgb}{0.091915,0.334238,0.091915}%
\pgfsetfillcolor{currentfill}%
\pgfsetlinewidth{0.000000pt}%
\definecolor{currentstroke}{rgb}{0.000000,0.000000,0.000000}%
\pgfsetstrokecolor{currentstroke}%
\pgfsetdash{}{0pt}%
\pgfpathmoveto{\pgfqpoint{2.287385in}{2.126617in}}%
\pgfpathlineto{\pgfqpoint{1.950995in}{1.952591in}}%
\pgfpathlineto{\pgfqpoint{2.242654in}{1.526914in}}%
\pgfpathlineto{\pgfqpoint{2.287385in}{2.126617in}}%
\pgfpathclose%
\pgfusepath{fill}%
\end{pgfscope}%
\begin{pgfscope}%
\pgfpathrectangle{\pgfqpoint{0.254231in}{0.147348in}}{\pgfqpoint{2.735294in}{2.735294in}}%
\pgfusepath{clip}%
\pgfsetbuttcap%
\pgfsetroundjoin%
\definecolor{currentfill}{rgb}{0.101759,0.370033,0.101759}%
\pgfsetfillcolor{currentfill}%
\pgfsetlinewidth{0.000000pt}%
\definecolor{currentstroke}{rgb}{0.000000,0.000000,0.000000}%
\pgfsetstrokecolor{currentstroke}%
\pgfsetdash{}{0pt}%
\pgfpathmoveto{\pgfqpoint{1.366689in}{1.952591in}}%
\pgfpathlineto{\pgfqpoint{1.520550in}{2.150689in}}%
\pgfpathlineto{\pgfqpoint{1.257750in}{2.141900in}}%
\pgfpathlineto{\pgfqpoint{1.366689in}{1.952591in}}%
\pgfpathclose%
\pgfusepath{fill}%
\end{pgfscope}%
\begin{pgfscope}%
\pgfpathrectangle{\pgfqpoint{0.254231in}{0.147348in}}{\pgfqpoint{2.735294in}{2.735294in}}%
\pgfusepath{clip}%
\pgfsetbuttcap%
\pgfsetroundjoin%
\definecolor{currentfill}{rgb}{0.101759,0.370033,0.101759}%
\pgfsetfillcolor{currentfill}%
\pgfsetlinewidth{0.000000pt}%
\definecolor{currentstroke}{rgb}{0.000000,0.000000,0.000000}%
\pgfsetstrokecolor{currentstroke}%
\pgfsetdash{}{0pt}%
\pgfpathmoveto{\pgfqpoint{2.059934in}{2.141900in}}%
\pgfpathlineto{\pgfqpoint{1.797134in}{2.150689in}}%
\pgfpathlineto{\pgfqpoint{1.950995in}{1.952591in}}%
\pgfpathlineto{\pgfqpoint{2.059934in}{2.141900in}}%
\pgfpathclose%
\pgfusepath{fill}%
\end{pgfscope}%
\begin{pgfscope}%
\pgfpathrectangle{\pgfqpoint{0.254231in}{0.147348in}}{\pgfqpoint{2.735294in}{2.735294in}}%
\pgfusepath{clip}%
\pgfsetbuttcap%
\pgfsetroundjoin%
\definecolor{currentfill}{rgb}{0.101677,0.369734,0.101677}%
\pgfsetfillcolor{currentfill}%
\pgfsetlinewidth{0.000000pt}%
\definecolor{currentstroke}{rgb}{0.000000,0.000000,0.000000}%
\pgfsetstrokecolor{currentstroke}%
\pgfsetdash{}{0pt}%
\pgfpathmoveto{\pgfqpoint{1.658842in}{1.953918in}}%
\pgfpathlineto{\pgfqpoint{1.797134in}{2.150689in}}%
\pgfpathlineto{\pgfqpoint{1.520550in}{2.150689in}}%
\pgfpathlineto{\pgfqpoint{1.658842in}{1.953918in}}%
\pgfpathclose%
\pgfusepath{fill}%
\end{pgfscope}%
\begin{pgfscope}%
\pgfpathrectangle{\pgfqpoint{0.254231in}{0.147348in}}{\pgfqpoint{2.735294in}{2.735294in}}%
\pgfusepath{clip}%
\pgfsetbuttcap%
\pgfsetroundjoin%
\definecolor{currentfill}{rgb}{0.065035,0.236492,0.065035}%
\pgfsetfillcolor{currentfill}%
\pgfsetlinewidth{0.000000pt}%
\definecolor{currentstroke}{rgb}{0.000000,0.000000,0.000000}%
\pgfsetstrokecolor{currentstroke}%
\pgfsetdash{}{0pt}%
\pgfpathmoveto{\pgfqpoint{1.258681in}{0.974376in}}%
\pgfpathlineto{\pgfqpoint{1.348371in}{1.504068in}}%
\pgfpathlineto{\pgfqpoint{1.212441in}{1.302943in}}%
\pgfpathlineto{\pgfqpoint{1.258681in}{0.974376in}}%
\pgfpathclose%
\pgfusepath{fill}%
\end{pgfscope}%
\begin{pgfscope}%
\pgfpathrectangle{\pgfqpoint{0.254231in}{0.147348in}}{\pgfqpoint{2.735294in}{2.735294in}}%
\pgfusepath{clip}%
\pgfsetbuttcap%
\pgfsetroundjoin%
\definecolor{currentfill}{rgb}{0.065035,0.236492,0.065035}%
\pgfsetfillcolor{currentfill}%
\pgfsetlinewidth{0.000000pt}%
\definecolor{currentstroke}{rgb}{0.000000,0.000000,0.000000}%
\pgfsetstrokecolor{currentstroke}%
\pgfsetdash{}{0pt}%
\pgfpathmoveto{\pgfqpoint{2.105243in}{1.302943in}}%
\pgfpathlineto{\pgfqpoint{1.969313in}{1.504068in}}%
\pgfpathlineto{\pgfqpoint{2.059003in}{0.974376in}}%
\pgfpathlineto{\pgfqpoint{2.105243in}{1.302943in}}%
\pgfpathclose%
\pgfusepath{fill}%
\end{pgfscope}%
\begin{pgfscope}%
\pgfpathrectangle{\pgfqpoint{0.254231in}{0.147348in}}{\pgfqpoint{2.735294in}{2.735294in}}%
\pgfusepath{clip}%
\pgfsetbuttcap%
\pgfsetroundjoin%
\definecolor{currentfill}{rgb}{0.066446,0.241622,0.066446}%
\pgfsetfillcolor{currentfill}%
\pgfsetlinewidth{0.000000pt}%
\definecolor{currentstroke}{rgb}{0.000000,0.000000,0.000000}%
\pgfsetstrokecolor{currentstroke}%
\pgfsetdash{}{0pt}%
\pgfpathmoveto{\pgfqpoint{1.503676in}{1.277996in}}%
\pgfpathlineto{\pgfqpoint{1.658842in}{0.830831in}}%
\pgfpathlineto{\pgfqpoint{1.658842in}{1.495457in}}%
\pgfpathlineto{\pgfqpoint{1.503676in}{1.277996in}}%
\pgfpathclose%
\pgfusepath{fill}%
\end{pgfscope}%
\begin{pgfscope}%
\pgfpathrectangle{\pgfqpoint{0.254231in}{0.147348in}}{\pgfqpoint{2.735294in}{2.735294in}}%
\pgfusepath{clip}%
\pgfsetbuttcap%
\pgfsetroundjoin%
\definecolor{currentfill}{rgb}{0.066446,0.241622,0.066446}%
\pgfsetfillcolor{currentfill}%
\pgfsetlinewidth{0.000000pt}%
\definecolor{currentstroke}{rgb}{0.000000,0.000000,0.000000}%
\pgfsetstrokecolor{currentstroke}%
\pgfsetdash{}{0pt}%
\pgfpathmoveto{\pgfqpoint{1.658842in}{1.495457in}}%
\pgfpathlineto{\pgfqpoint{1.658842in}{0.830831in}}%
\pgfpathlineto{\pgfqpoint{1.814008in}{1.277996in}}%
\pgfpathlineto{\pgfqpoint{1.658842in}{1.495457in}}%
\pgfpathclose%
\pgfusepath{fill}%
\end{pgfscope}%
\begin{pgfscope}%
\pgfpathrectangle{\pgfqpoint{0.254231in}{0.147348in}}{\pgfqpoint{2.735294in}{2.735294in}}%
\pgfusepath{clip}%
\pgfsetbuttcap%
\pgfsetroundjoin%
\definecolor{currentfill}{rgb}{0.098306,0.357475,0.098306}%
\pgfsetfillcolor{currentfill}%
\pgfsetlinewidth{0.000000pt}%
\definecolor{currentstroke}{rgb}{0.000000,0.000000,0.000000}%
\pgfsetstrokecolor{currentstroke}%
\pgfsetdash{}{0pt}%
\pgfpathmoveto{\pgfqpoint{1.658842in}{1.953918in}}%
\pgfpathlineto{\pgfqpoint{1.520550in}{2.150689in}}%
\pgfpathlineto{\pgfqpoint{1.366689in}{1.952591in}}%
\pgfpathlineto{\pgfqpoint{1.658842in}{1.953918in}}%
\pgfpathclose%
\pgfusepath{fill}%
\end{pgfscope}%
\begin{pgfscope}%
\pgfpathrectangle{\pgfqpoint{0.254231in}{0.147348in}}{\pgfqpoint{2.735294in}{2.735294in}}%
\pgfusepath{clip}%
\pgfsetbuttcap%
\pgfsetroundjoin%
\definecolor{currentfill}{rgb}{0.098306,0.357475,0.098306}%
\pgfsetfillcolor{currentfill}%
\pgfsetlinewidth{0.000000pt}%
\definecolor{currentstroke}{rgb}{0.000000,0.000000,0.000000}%
\pgfsetstrokecolor{currentstroke}%
\pgfsetdash{}{0pt}%
\pgfpathmoveto{\pgfqpoint{1.950995in}{1.952591in}}%
\pgfpathlineto{\pgfqpoint{1.797134in}{2.150689in}}%
\pgfpathlineto{\pgfqpoint{1.658842in}{1.953918in}}%
\pgfpathlineto{\pgfqpoint{1.950995in}{1.952591in}}%
\pgfpathclose%
\pgfusepath{fill}%
\end{pgfscope}%
\begin{pgfscope}%
\pgfpathrectangle{\pgfqpoint{0.254231in}{0.147348in}}{\pgfqpoint{2.735294in}{2.735294in}}%
\pgfusepath{clip}%
\pgfsetbuttcap%
\pgfsetroundjoin%
\definecolor{currentfill}{rgb}{0.070209,0.255305,0.070209}%
\pgfsetfillcolor{currentfill}%
\pgfsetlinewidth{0.000000pt}%
\definecolor{currentstroke}{rgb}{0.000000,0.000000,0.000000}%
\pgfsetstrokecolor{currentstroke}%
\pgfsetdash{}{0pt}%
\pgfpathmoveto{\pgfqpoint{1.258681in}{0.974376in}}%
\pgfpathlineto{\pgfqpoint{1.503676in}{1.277996in}}%
\pgfpathlineto{\pgfqpoint{1.348371in}{1.504068in}}%
\pgfpathlineto{\pgfqpoint{1.258681in}{0.974376in}}%
\pgfpathclose%
\pgfusepath{fill}%
\end{pgfscope}%
\begin{pgfscope}%
\pgfpathrectangle{\pgfqpoint{0.254231in}{0.147348in}}{\pgfqpoint{2.735294in}{2.735294in}}%
\pgfusepath{clip}%
\pgfsetbuttcap%
\pgfsetroundjoin%
\definecolor{currentfill}{rgb}{0.070209,0.255305,0.070209}%
\pgfsetfillcolor{currentfill}%
\pgfsetlinewidth{0.000000pt}%
\definecolor{currentstroke}{rgb}{0.000000,0.000000,0.000000}%
\pgfsetstrokecolor{currentstroke}%
\pgfsetdash{}{0pt}%
\pgfpathmoveto{\pgfqpoint{1.969313in}{1.504068in}}%
\pgfpathlineto{\pgfqpoint{1.814008in}{1.277996in}}%
\pgfpathlineto{\pgfqpoint{2.059003in}{0.974376in}}%
\pgfpathlineto{\pgfqpoint{1.969313in}{1.504068in}}%
\pgfpathclose%
\pgfusepath{fill}%
\end{pgfscope}%
\begin{pgfscope}%
\pgfpathrectangle{\pgfqpoint{0.254231in}{0.147348in}}{\pgfqpoint{2.735294in}{2.735294in}}%
\pgfusepath{clip}%
\pgfsetbuttcap%
\pgfsetroundjoin%
\definecolor{currentfill}{rgb}{0.087398,0.317812,0.087398}%
\pgfsetfillcolor{currentfill}%
\pgfsetlinewidth{0.000000pt}%
\definecolor{currentstroke}{rgb}{0.000000,0.000000,0.000000}%
\pgfsetstrokecolor{currentstroke}%
\pgfsetdash{}{0pt}%
\pgfpathmoveto{\pgfqpoint{2.242654in}{1.526914in}}%
\pgfpathlineto{\pgfqpoint{1.950995in}{1.952591in}}%
\pgfpathlineto{\pgfqpoint{1.969313in}{1.504068in}}%
\pgfpathlineto{\pgfqpoint{2.242654in}{1.526914in}}%
\pgfpathclose%
\pgfusepath{fill}%
\end{pgfscope}%
\begin{pgfscope}%
\pgfpathrectangle{\pgfqpoint{0.254231in}{0.147348in}}{\pgfqpoint{2.735294in}{2.735294in}}%
\pgfusepath{clip}%
\pgfsetbuttcap%
\pgfsetroundjoin%
\definecolor{currentfill}{rgb}{0.087398,0.317812,0.087398}%
\pgfsetfillcolor{currentfill}%
\pgfsetlinewidth{0.000000pt}%
\definecolor{currentstroke}{rgb}{0.000000,0.000000,0.000000}%
\pgfsetstrokecolor{currentstroke}%
\pgfsetdash{}{0pt}%
\pgfpathmoveto{\pgfqpoint{1.348371in}{1.504068in}}%
\pgfpathlineto{\pgfqpoint{1.366689in}{1.952591in}}%
\pgfpathlineto{\pgfqpoint{1.075030in}{1.526914in}}%
\pgfpathlineto{\pgfqpoint{1.348371in}{1.504068in}}%
\pgfpathclose%
\pgfusepath{fill}%
\end{pgfscope}%
\begin{pgfscope}%
\pgfpathrectangle{\pgfqpoint{0.254231in}{0.147348in}}{\pgfqpoint{2.735294in}{2.735294in}}%
\pgfusepath{clip}%
\pgfsetbuttcap%
\pgfsetroundjoin%
\definecolor{currentfill}{rgb}{0.075994,0.276341,0.075994}%
\pgfsetfillcolor{currentfill}%
\pgfsetlinewidth{0.000000pt}%
\definecolor{currentstroke}{rgb}{0.000000,0.000000,0.000000}%
\pgfsetstrokecolor{currentstroke}%
\pgfsetdash{}{0pt}%
\pgfpathmoveto{\pgfqpoint{1.075030in}{1.526914in}}%
\pgfpathlineto{\pgfqpoint{1.212441in}{1.302943in}}%
\pgfpathlineto{\pgfqpoint{1.348371in}{1.504068in}}%
\pgfpathlineto{\pgfqpoint{1.075030in}{1.526914in}}%
\pgfpathclose%
\pgfusepath{fill}%
\end{pgfscope}%
\begin{pgfscope}%
\pgfpathrectangle{\pgfqpoint{0.254231in}{0.147348in}}{\pgfqpoint{2.735294in}{2.735294in}}%
\pgfusepath{clip}%
\pgfsetbuttcap%
\pgfsetroundjoin%
\definecolor{currentfill}{rgb}{0.075994,0.276341,0.075994}%
\pgfsetfillcolor{currentfill}%
\pgfsetlinewidth{0.000000pt}%
\definecolor{currentstroke}{rgb}{0.000000,0.000000,0.000000}%
\pgfsetstrokecolor{currentstroke}%
\pgfsetdash{}{0pt}%
\pgfpathmoveto{\pgfqpoint{1.969313in}{1.504068in}}%
\pgfpathlineto{\pgfqpoint{2.105243in}{1.302943in}}%
\pgfpathlineto{\pgfqpoint{2.242654in}{1.526914in}}%
\pgfpathlineto{\pgfqpoint{1.969313in}{1.504068in}}%
\pgfpathclose%
\pgfusepath{fill}%
\end{pgfscope}%
\begin{pgfscope}%
\pgfpathrectangle{\pgfqpoint{0.254231in}{0.147348in}}{\pgfqpoint{2.735294in}{2.735294in}}%
\pgfusepath{clip}%
\pgfsetbuttcap%
\pgfsetroundjoin%
\definecolor{currentfill}{rgb}{0.086061,0.312950,0.086061}%
\pgfsetfillcolor{currentfill}%
\pgfsetlinewidth{0.000000pt}%
\definecolor{currentstroke}{rgb}{0.000000,0.000000,0.000000}%
\pgfsetstrokecolor{currentstroke}%
\pgfsetdash{}{0pt}%
\pgfpathmoveto{\pgfqpoint{1.366689in}{1.952591in}}%
\pgfpathlineto{\pgfqpoint{1.348371in}{1.504068in}}%
\pgfpathlineto{\pgfqpoint{1.658842in}{1.953918in}}%
\pgfpathlineto{\pgfqpoint{1.366689in}{1.952591in}}%
\pgfpathclose%
\pgfusepath{fill}%
\end{pgfscope}%
\begin{pgfscope}%
\pgfpathrectangle{\pgfqpoint{0.254231in}{0.147348in}}{\pgfqpoint{2.735294in}{2.735294in}}%
\pgfusepath{clip}%
\pgfsetbuttcap%
\pgfsetroundjoin%
\definecolor{currentfill}{rgb}{0.086061,0.312950,0.086061}%
\pgfsetfillcolor{currentfill}%
\pgfsetlinewidth{0.000000pt}%
\definecolor{currentstroke}{rgb}{0.000000,0.000000,0.000000}%
\pgfsetstrokecolor{currentstroke}%
\pgfsetdash{}{0pt}%
\pgfpathmoveto{\pgfqpoint{1.658842in}{1.953918in}}%
\pgfpathlineto{\pgfqpoint{1.969313in}{1.504068in}}%
\pgfpathlineto{\pgfqpoint{1.950995in}{1.952591in}}%
\pgfpathlineto{\pgfqpoint{1.658842in}{1.953918in}}%
\pgfpathclose%
\pgfusepath{fill}%
\end{pgfscope}%
\begin{pgfscope}%
\pgfpathrectangle{\pgfqpoint{0.254231in}{0.147348in}}{\pgfqpoint{2.735294in}{2.735294in}}%
\pgfusepath{clip}%
\pgfsetbuttcap%
\pgfsetroundjoin%
\definecolor{currentfill}{rgb}{0.086258,0.313666,0.086258}%
\pgfsetfillcolor{currentfill}%
\pgfsetlinewidth{0.000000pt}%
\definecolor{currentstroke}{rgb}{0.000000,0.000000,0.000000}%
\pgfsetstrokecolor{currentstroke}%
\pgfsetdash{}{0pt}%
\pgfpathmoveto{\pgfqpoint{1.658842in}{1.495457in}}%
\pgfpathlineto{\pgfqpoint{1.658842in}{1.953918in}}%
\pgfpathlineto{\pgfqpoint{1.348371in}{1.504068in}}%
\pgfpathlineto{\pgfqpoint{1.658842in}{1.495457in}}%
\pgfpathclose%
\pgfusepath{fill}%
\end{pgfscope}%
\begin{pgfscope}%
\pgfpathrectangle{\pgfqpoint{0.254231in}{0.147348in}}{\pgfqpoint{2.735294in}{2.735294in}}%
\pgfusepath{clip}%
\pgfsetbuttcap%
\pgfsetroundjoin%
\definecolor{currentfill}{rgb}{0.086258,0.313666,0.086258}%
\pgfsetfillcolor{currentfill}%
\pgfsetlinewidth{0.000000pt}%
\definecolor{currentstroke}{rgb}{0.000000,0.000000,0.000000}%
\pgfsetstrokecolor{currentstroke}%
\pgfsetdash{}{0pt}%
\pgfpathmoveto{\pgfqpoint{1.969313in}{1.504068in}}%
\pgfpathlineto{\pgfqpoint{1.658842in}{1.953918in}}%
\pgfpathlineto{\pgfqpoint{1.658842in}{1.495457in}}%
\pgfpathlineto{\pgfqpoint{1.969313in}{1.504068in}}%
\pgfpathclose%
\pgfusepath{fill}%
\end{pgfscope}%
\begin{pgfscope}%
\pgfpathrectangle{\pgfqpoint{0.254231in}{0.147348in}}{\pgfqpoint{2.735294in}{2.735294in}}%
\pgfusepath{clip}%
\pgfsetbuttcap%
\pgfsetroundjoin%
\definecolor{currentfill}{rgb}{0.074668,0.271519,0.074668}%
\pgfsetfillcolor{currentfill}%
\pgfsetlinewidth{0.000000pt}%
\definecolor{currentstroke}{rgb}{0.000000,0.000000,0.000000}%
\pgfsetstrokecolor{currentstroke}%
\pgfsetdash{}{0pt}%
\pgfpathmoveto{\pgfqpoint{1.348371in}{1.504068in}}%
\pgfpathlineto{\pgfqpoint{1.503676in}{1.277996in}}%
\pgfpathlineto{\pgfqpoint{1.658842in}{1.495457in}}%
\pgfpathlineto{\pgfqpoint{1.348371in}{1.504068in}}%
\pgfpathclose%
\pgfusepath{fill}%
\end{pgfscope}%
\begin{pgfscope}%
\pgfpathrectangle{\pgfqpoint{0.254231in}{0.147348in}}{\pgfqpoint{2.735294in}{2.735294in}}%
\pgfusepath{clip}%
\pgfsetbuttcap%
\pgfsetroundjoin%
\definecolor{currentfill}{rgb}{0.074668,0.271519,0.074668}%
\pgfsetfillcolor{currentfill}%
\pgfsetlinewidth{0.000000pt}%
\definecolor{currentstroke}{rgb}{0.000000,0.000000,0.000000}%
\pgfsetstrokecolor{currentstroke}%
\pgfsetdash{}{0pt}%
\pgfpathmoveto{\pgfqpoint{1.658842in}{1.495457in}}%
\pgfpathlineto{\pgfqpoint{1.814008in}{1.277996in}}%
\pgfpathlineto{\pgfqpoint{1.969313in}{1.504068in}}%
\pgfpathlineto{\pgfqpoint{1.658842in}{1.495457in}}%
\pgfpathclose%
\pgfusepath{fill}%
\end{pgfscope}%
\end{pgfpicture}%
\makeatother%
\endgroup%
}
  \caption{From left to right: An opaque Pareto front; a translucent Pareto front 
  showing the interior points above a sub-optimal front; and the sub-optimal front 
  hiding the interior points from a different angle.}
  \label{fig:3d-mga}
\end{figure}

\subsection{Farthest First Traversal}
Previous studies emphasized that a key benefit of \ac{mga} is
obtaining a set of near-optimal solutions that are maximally different in design
space \cite{decarolis_modelling_2016, yue_review_2018-1}. Instead of
constraining a linear program to guarantee a maximally-different solution set.
\ac{osier} implements a ``greedy'' search algorithm in decision space to find
solutions \cite{hochbaum_best_1985}. The algorithm is
\begin{enumerate}
  \item Calculate a distance matrix $\mathcal{D}$ containing the distances among
  all points. 
  \item Choose an initial point (either randomly or a specified point).
  \item Step towards the point that has the maximum distance from the current
  point that has not already been visited.
  \item Continue until the desired number of points has been reached.
\end{enumerate}

Figure \ref{fig:mga-fft} demonstrates \ac{mga} with
``farthest-first-traversal.'' The left plot in Figure \ref{fig:mga-fft} shows
the objective space for the same problem as in Figure \ref{fig:nd-mga}. The plot
on the right shows the corresponding design space. Both plots show the Pareto
front as red dots. The colored dots represent the points selected by the ``farthest-first-traversal''
algorithm. These points are connected by similarly colored arrows in the design space plot. The arrow
color corresponds to the color of the next selected point (i.e., the next farthest point).

\begin{figure}[H]
  \centering
  \resizebox{1\columnwidth}{!}{%% Creator: Matplotlib, PGF backend
%%
%% To include the figure in your LaTeX document, write
%%   \input{<filename>.pgf}
%%
%% Make sure the required packages are loaded in your preamble
%%   \usepackage{pgf}
%%
%% Also ensure that all the required font packages are loaded; for instance,
%% the lmodern package is sometimes necessary when using math font.
%%   \usepackage{lmodern}
%%
%% Figures using additional raster images can only be included by \input if
%% they are in the same directory as the main LaTeX file. For loading figures
%% from other directories you can use the `import` package
%%   \usepackage{import}
%%
%% and then include the figures with
%%   \import{<path to file>}{<filename>.pgf}
%%
%% Matplotlib used the following preamble
%%   \def\mathdefault#1{#1}
%%   \everymath=\expandafter{\the\everymath\displaystyle}
%%   \IfFileExists{scrextend.sty}{
%%     \usepackage[fontsize=10.000000pt]{scrextend}
%%   }{
%%     \renewcommand{\normalsize}{\fontsize{10.000000}{12.000000}\selectfont}
%%     \normalsize
%%   }
%%   
%%   \ifdefined\pdftexversion\else  % non-pdftex case.
%%     \usepackage{fontspec}
%%     \setmainfont{DejaVuSerif.ttf}[Path=\detokenize{/Users/samdotson/miniforge3/envs/2025-dotson-thesis/lib/python3.11/site-packages/matplotlib/mpl-data/fonts/ttf/}]
%%     \setsansfont{DejaVuSans.ttf}[Path=\detokenize{/Users/samdotson/miniforge3/envs/2025-dotson-thesis/lib/python3.11/site-packages/matplotlib/mpl-data/fonts/ttf/}]
%%     \setmonofont{DejaVuSansMono.ttf}[Path=\detokenize{/Users/samdotson/miniforge3/envs/2025-dotson-thesis/lib/python3.11/site-packages/matplotlib/mpl-data/fonts/ttf/}]
%%   \fi
%%   \makeatletter\@ifpackageloaded{underscore}{}{\usepackage[strings]{underscore}}\makeatother
%%
\begingroup%
\makeatletter%
\begin{pgfpicture}%
\pgfpathrectangle{\pgfpointorigin}{\pgfqpoint{14.000000in}{6.000000in}}%
\pgfusepath{use as bounding box, clip}%
\begin{pgfscope}%
\pgfsetbuttcap%
\pgfsetmiterjoin%
\definecolor{currentfill}{rgb}{1.000000,1.000000,1.000000}%
\pgfsetfillcolor{currentfill}%
\pgfsetlinewidth{0.000000pt}%
\definecolor{currentstroke}{rgb}{1.000000,1.000000,1.000000}%
\pgfsetstrokecolor{currentstroke}%
\pgfsetdash{}{0pt}%
\pgfpathmoveto{\pgfqpoint{0.000000in}{0.000000in}}%
\pgfpathlineto{\pgfqpoint{14.000000in}{0.000000in}}%
\pgfpathlineto{\pgfqpoint{14.000000in}{6.000000in}}%
\pgfpathlineto{\pgfqpoint{0.000000in}{6.000000in}}%
\pgfpathlineto{\pgfqpoint{0.000000in}{0.000000in}}%
\pgfpathclose%
\pgfusepath{fill}%
\end{pgfscope}%
\begin{pgfscope}%
\pgfsetbuttcap%
\pgfsetmiterjoin%
\definecolor{currentfill}{rgb}{1.000000,1.000000,1.000000}%
\pgfsetfillcolor{currentfill}%
\pgfsetlinewidth{0.000000pt}%
\definecolor{currentstroke}{rgb}{0.000000,0.000000,0.000000}%
\pgfsetstrokecolor{currentstroke}%
\pgfsetstrokeopacity{0.000000}%
\pgfsetdash{}{0pt}%
\pgfpathmoveto{\pgfqpoint{0.494722in}{0.437222in}}%
\pgfpathlineto{\pgfqpoint{6.770313in}{0.437222in}}%
\pgfpathlineto{\pgfqpoint{6.770313in}{5.596667in}}%
\pgfpathlineto{\pgfqpoint{0.494722in}{5.596667in}}%
\pgfpathlineto{\pgfqpoint{0.494722in}{0.437222in}}%
\pgfpathclose%
\pgfusepath{fill}%
\end{pgfscope}%
\begin{pgfscope}%
\pgfpathrectangle{\pgfqpoint{0.494722in}{0.437222in}}{\pgfqpoint{6.275590in}{5.159444in}}%
\pgfusepath{clip}%
\pgfsetbuttcap%
\pgfsetroundjoin%
\pgfsetlinewidth{1.003750pt}%
\definecolor{currentstroke}{rgb}{0.827451,0.827451,0.827451}%
\pgfsetstrokecolor{currentstroke}%
\pgfsetstrokeopacity{0.800000}%
\pgfsetdash{}{0pt}%
\pgfpathmoveto{\pgfqpoint{1.185063in}{2.437307in}}%
\pgfpathcurveto{\pgfqpoint{1.196113in}{2.437307in}}{\pgfqpoint{1.206712in}{2.441697in}}{\pgfqpoint{1.214526in}{2.449511in}}%
\pgfpathcurveto{\pgfqpoint{1.222340in}{2.457325in}}{\pgfqpoint{1.226730in}{2.467924in}}{\pgfqpoint{1.226730in}{2.478974in}}%
\pgfpathcurveto{\pgfqpoint{1.226730in}{2.490024in}}{\pgfqpoint{1.222340in}{2.500623in}}{\pgfqpoint{1.214526in}{2.508437in}}%
\pgfpathcurveto{\pgfqpoint{1.206712in}{2.516250in}}{\pgfqpoint{1.196113in}{2.520641in}}{\pgfqpoint{1.185063in}{2.520641in}}%
\pgfpathcurveto{\pgfqpoint{1.174013in}{2.520641in}}{\pgfqpoint{1.163414in}{2.516250in}}{\pgfqpoint{1.155600in}{2.508437in}}%
\pgfpathcurveto{\pgfqpoint{1.147787in}{2.500623in}}{\pgfqpoint{1.143397in}{2.490024in}}{\pgfqpoint{1.143397in}{2.478974in}}%
\pgfpathcurveto{\pgfqpoint{1.143397in}{2.467924in}}{\pgfqpoint{1.147787in}{2.457325in}}{\pgfqpoint{1.155600in}{2.449511in}}%
\pgfpathcurveto{\pgfqpoint{1.163414in}{2.441697in}}{\pgfqpoint{1.174013in}{2.437307in}}{\pgfqpoint{1.185063in}{2.437307in}}%
\pgfpathlineto{\pgfqpoint{1.185063in}{2.437307in}}%
\pgfpathclose%
\pgfusepath{stroke}%
\end{pgfscope}%
\begin{pgfscope}%
\pgfpathrectangle{\pgfqpoint{0.494722in}{0.437222in}}{\pgfqpoint{6.275590in}{5.159444in}}%
\pgfusepath{clip}%
\pgfsetbuttcap%
\pgfsetroundjoin%
\pgfsetlinewidth{1.003750pt}%
\definecolor{currentstroke}{rgb}{0.827451,0.827451,0.827451}%
\pgfsetstrokecolor{currentstroke}%
\pgfsetstrokeopacity{0.800000}%
\pgfsetdash{}{0pt}%
\pgfpathmoveto{\pgfqpoint{2.117156in}{1.399260in}}%
\pgfpathcurveto{\pgfqpoint{2.128206in}{1.399260in}}{\pgfqpoint{2.138805in}{1.403650in}}{\pgfqpoint{2.146618in}{1.411464in}}%
\pgfpathcurveto{\pgfqpoint{2.154432in}{1.419277in}}{\pgfqpoint{2.158822in}{1.429876in}}{\pgfqpoint{2.158822in}{1.440927in}}%
\pgfpathcurveto{\pgfqpoint{2.158822in}{1.451977in}}{\pgfqpoint{2.154432in}{1.462576in}}{\pgfqpoint{2.146618in}{1.470389in}}%
\pgfpathcurveto{\pgfqpoint{2.138805in}{1.478203in}}{\pgfqpoint{2.128206in}{1.482593in}}{\pgfqpoint{2.117156in}{1.482593in}}%
\pgfpathcurveto{\pgfqpoint{2.106106in}{1.482593in}}{\pgfqpoint{2.095506in}{1.478203in}}{\pgfqpoint{2.087693in}{1.470389in}}%
\pgfpathcurveto{\pgfqpoint{2.079879in}{1.462576in}}{\pgfqpoint{2.075489in}{1.451977in}}{\pgfqpoint{2.075489in}{1.440927in}}%
\pgfpathcurveto{\pgfqpoint{2.075489in}{1.429876in}}{\pgfqpoint{2.079879in}{1.419277in}}{\pgfqpoint{2.087693in}{1.411464in}}%
\pgfpathcurveto{\pgfqpoint{2.095506in}{1.403650in}}{\pgfqpoint{2.106106in}{1.399260in}}{\pgfqpoint{2.117156in}{1.399260in}}%
\pgfpathlineto{\pgfqpoint{2.117156in}{1.399260in}}%
\pgfpathclose%
\pgfusepath{stroke}%
\end{pgfscope}%
\begin{pgfscope}%
\pgfpathrectangle{\pgfqpoint{0.494722in}{0.437222in}}{\pgfqpoint{6.275590in}{5.159444in}}%
\pgfusepath{clip}%
\pgfsetbuttcap%
\pgfsetroundjoin%
\pgfsetlinewidth{1.003750pt}%
\definecolor{currentstroke}{rgb}{0.827451,0.827451,0.827451}%
\pgfsetstrokecolor{currentstroke}%
\pgfsetstrokeopacity{0.800000}%
\pgfsetdash{}{0pt}%
\pgfpathmoveto{\pgfqpoint{1.702429in}{1.852953in}}%
\pgfpathcurveto{\pgfqpoint{1.713479in}{1.852953in}}{\pgfqpoint{1.724078in}{1.857344in}}{\pgfqpoint{1.731891in}{1.865157in}}%
\pgfpathcurveto{\pgfqpoint{1.739705in}{1.872971in}}{\pgfqpoint{1.744095in}{1.883570in}}{\pgfqpoint{1.744095in}{1.894620in}}%
\pgfpathcurveto{\pgfqpoint{1.744095in}{1.905670in}}{\pgfqpoint{1.739705in}{1.916269in}}{\pgfqpoint{1.731891in}{1.924083in}}%
\pgfpathcurveto{\pgfqpoint{1.724078in}{1.931897in}}{\pgfqpoint{1.713479in}{1.936287in}}{\pgfqpoint{1.702429in}{1.936287in}}%
\pgfpathcurveto{\pgfqpoint{1.691379in}{1.936287in}}{\pgfqpoint{1.680780in}{1.931897in}}{\pgfqpoint{1.672966in}{1.924083in}}%
\pgfpathcurveto{\pgfqpoint{1.665152in}{1.916269in}}{\pgfqpoint{1.660762in}{1.905670in}}{\pgfqpoint{1.660762in}{1.894620in}}%
\pgfpathcurveto{\pgfqpoint{1.660762in}{1.883570in}}{\pgfqpoint{1.665152in}{1.872971in}}{\pgfqpoint{1.672966in}{1.865157in}}%
\pgfpathcurveto{\pgfqpoint{1.680780in}{1.857344in}}{\pgfqpoint{1.691379in}{1.852953in}}{\pgfqpoint{1.702429in}{1.852953in}}%
\pgfpathlineto{\pgfqpoint{1.702429in}{1.852953in}}%
\pgfpathclose%
\pgfusepath{stroke}%
\end{pgfscope}%
\begin{pgfscope}%
\pgfpathrectangle{\pgfqpoint{0.494722in}{0.437222in}}{\pgfqpoint{6.275590in}{5.159444in}}%
\pgfusepath{clip}%
\pgfsetbuttcap%
\pgfsetroundjoin%
\pgfsetlinewidth{1.003750pt}%
\definecolor{currentstroke}{rgb}{0.827451,0.827451,0.827451}%
\pgfsetstrokecolor{currentstroke}%
\pgfsetstrokeopacity{0.800000}%
\pgfsetdash{}{0pt}%
\pgfpathmoveto{\pgfqpoint{1.477105in}{1.953817in}}%
\pgfpathcurveto{\pgfqpoint{1.488155in}{1.953817in}}{\pgfqpoint{1.498754in}{1.958207in}}{\pgfqpoint{1.506568in}{1.966021in}}%
\pgfpathcurveto{\pgfqpoint{1.514381in}{1.973835in}}{\pgfqpoint{1.518772in}{1.984434in}}{\pgfqpoint{1.518772in}{1.995484in}}%
\pgfpathcurveto{\pgfqpoint{1.518772in}{2.006534in}}{\pgfqpoint{1.514381in}{2.017133in}}{\pgfqpoint{1.506568in}{2.024947in}}%
\pgfpathcurveto{\pgfqpoint{1.498754in}{2.032760in}}{\pgfqpoint{1.488155in}{2.037151in}}{\pgfqpoint{1.477105in}{2.037151in}}%
\pgfpathcurveto{\pgfqpoint{1.466055in}{2.037151in}}{\pgfqpoint{1.455456in}{2.032760in}}{\pgfqpoint{1.447642in}{2.024947in}}%
\pgfpathcurveto{\pgfqpoint{1.439828in}{2.017133in}}{\pgfqpoint{1.435438in}{2.006534in}}{\pgfqpoint{1.435438in}{1.995484in}}%
\pgfpathcurveto{\pgfqpoint{1.435438in}{1.984434in}}{\pgfqpoint{1.439828in}{1.973835in}}{\pgfqpoint{1.447642in}{1.966021in}}%
\pgfpathcurveto{\pgfqpoint{1.455456in}{1.958207in}}{\pgfqpoint{1.466055in}{1.953817in}}{\pgfqpoint{1.477105in}{1.953817in}}%
\pgfpathlineto{\pgfqpoint{1.477105in}{1.953817in}}%
\pgfpathclose%
\pgfusepath{stroke}%
\end{pgfscope}%
\begin{pgfscope}%
\pgfpathrectangle{\pgfqpoint{0.494722in}{0.437222in}}{\pgfqpoint{6.275590in}{5.159444in}}%
\pgfusepath{clip}%
\pgfsetbuttcap%
\pgfsetroundjoin%
\pgfsetlinewidth{1.003750pt}%
\definecolor{currentstroke}{rgb}{0.827451,0.827451,0.827451}%
\pgfsetstrokecolor{currentstroke}%
\pgfsetstrokeopacity{0.800000}%
\pgfsetdash{}{0pt}%
\pgfpathmoveto{\pgfqpoint{3.937358in}{0.655597in}}%
\pgfpathcurveto{\pgfqpoint{3.948408in}{0.655597in}}{\pgfqpoint{3.959007in}{0.659987in}}{\pgfqpoint{3.966821in}{0.667801in}}%
\pgfpathcurveto{\pgfqpoint{3.974634in}{0.675614in}}{\pgfqpoint{3.979025in}{0.686213in}}{\pgfqpoint{3.979025in}{0.697263in}}%
\pgfpathcurveto{\pgfqpoint{3.979025in}{0.708314in}}{\pgfqpoint{3.974634in}{0.718913in}}{\pgfqpoint{3.966821in}{0.726726in}}%
\pgfpathcurveto{\pgfqpoint{3.959007in}{0.734540in}}{\pgfqpoint{3.948408in}{0.738930in}}{\pgfqpoint{3.937358in}{0.738930in}}%
\pgfpathcurveto{\pgfqpoint{3.926308in}{0.738930in}}{\pgfqpoint{3.915709in}{0.734540in}}{\pgfqpoint{3.907895in}{0.726726in}}%
\pgfpathcurveto{\pgfqpoint{3.900082in}{0.718913in}}{\pgfqpoint{3.895691in}{0.708314in}}{\pgfqpoint{3.895691in}{0.697263in}}%
\pgfpathcurveto{\pgfqpoint{3.895691in}{0.686213in}}{\pgfqpoint{3.900082in}{0.675614in}}{\pgfqpoint{3.907895in}{0.667801in}}%
\pgfpathcurveto{\pgfqpoint{3.915709in}{0.659987in}}{\pgfqpoint{3.926308in}{0.655597in}}{\pgfqpoint{3.937358in}{0.655597in}}%
\pgfpathlineto{\pgfqpoint{3.937358in}{0.655597in}}%
\pgfpathclose%
\pgfusepath{stroke}%
\end{pgfscope}%
\begin{pgfscope}%
\pgfpathrectangle{\pgfqpoint{0.494722in}{0.437222in}}{\pgfqpoint{6.275590in}{5.159444in}}%
\pgfusepath{clip}%
\pgfsetbuttcap%
\pgfsetroundjoin%
\pgfsetlinewidth{1.003750pt}%
\definecolor{currentstroke}{rgb}{0.827451,0.827451,0.827451}%
\pgfsetstrokecolor{currentstroke}%
\pgfsetstrokeopacity{0.800000}%
\pgfsetdash{}{0pt}%
\pgfpathmoveto{\pgfqpoint{1.551682in}{1.872518in}}%
\pgfpathcurveto{\pgfqpoint{1.562732in}{1.872518in}}{\pgfqpoint{1.573331in}{1.876908in}}{\pgfqpoint{1.581144in}{1.884722in}}%
\pgfpathcurveto{\pgfqpoint{1.588958in}{1.892535in}}{\pgfqpoint{1.593348in}{1.903134in}}{\pgfqpoint{1.593348in}{1.914184in}}%
\pgfpathcurveto{\pgfqpoint{1.593348in}{1.925234in}}{\pgfqpoint{1.588958in}{1.935833in}}{\pgfqpoint{1.581144in}{1.943647in}}%
\pgfpathcurveto{\pgfqpoint{1.573331in}{1.951461in}}{\pgfqpoint{1.562732in}{1.955851in}}{\pgfqpoint{1.551682in}{1.955851in}}%
\pgfpathcurveto{\pgfqpoint{1.540632in}{1.955851in}}{\pgfqpoint{1.530033in}{1.951461in}}{\pgfqpoint{1.522219in}{1.943647in}}%
\pgfpathcurveto{\pgfqpoint{1.514405in}{1.935833in}}{\pgfqpoint{1.510015in}{1.925234in}}{\pgfqpoint{1.510015in}{1.914184in}}%
\pgfpathcurveto{\pgfqpoint{1.510015in}{1.903134in}}{\pgfqpoint{1.514405in}{1.892535in}}{\pgfqpoint{1.522219in}{1.884722in}}%
\pgfpathcurveto{\pgfqpoint{1.530033in}{1.876908in}}{\pgfqpoint{1.540632in}{1.872518in}}{\pgfqpoint{1.551682in}{1.872518in}}%
\pgfpathlineto{\pgfqpoint{1.551682in}{1.872518in}}%
\pgfpathclose%
\pgfusepath{stroke}%
\end{pgfscope}%
\begin{pgfscope}%
\pgfpathrectangle{\pgfqpoint{0.494722in}{0.437222in}}{\pgfqpoint{6.275590in}{5.159444in}}%
\pgfusepath{clip}%
\pgfsetbuttcap%
\pgfsetroundjoin%
\pgfsetlinewidth{1.003750pt}%
\definecolor{currentstroke}{rgb}{0.827451,0.827451,0.827451}%
\pgfsetstrokecolor{currentstroke}%
\pgfsetstrokeopacity{0.800000}%
\pgfsetdash{}{0pt}%
\pgfpathmoveto{\pgfqpoint{0.709087in}{3.392876in}}%
\pgfpathcurveto{\pgfqpoint{0.720137in}{3.392876in}}{\pgfqpoint{0.730736in}{3.397266in}}{\pgfqpoint{0.738550in}{3.405079in}}%
\pgfpathcurveto{\pgfqpoint{0.746363in}{3.412893in}}{\pgfqpoint{0.750753in}{3.423492in}}{\pgfqpoint{0.750753in}{3.434542in}}%
\pgfpathcurveto{\pgfqpoint{0.750753in}{3.445592in}}{\pgfqpoint{0.746363in}{3.456191in}}{\pgfqpoint{0.738550in}{3.464005in}}%
\pgfpathcurveto{\pgfqpoint{0.730736in}{3.471819in}}{\pgfqpoint{0.720137in}{3.476209in}}{\pgfqpoint{0.709087in}{3.476209in}}%
\pgfpathcurveto{\pgfqpoint{0.698037in}{3.476209in}}{\pgfqpoint{0.687438in}{3.471819in}}{\pgfqpoint{0.679624in}{3.464005in}}%
\pgfpathcurveto{\pgfqpoint{0.671810in}{3.456191in}}{\pgfqpoint{0.667420in}{3.445592in}}{\pgfqpoint{0.667420in}{3.434542in}}%
\pgfpathcurveto{\pgfqpoint{0.667420in}{3.423492in}}{\pgfqpoint{0.671810in}{3.412893in}}{\pgfqpoint{0.679624in}{3.405079in}}%
\pgfpathcurveto{\pgfqpoint{0.687438in}{3.397266in}}{\pgfqpoint{0.698037in}{3.392876in}}{\pgfqpoint{0.709087in}{3.392876in}}%
\pgfpathlineto{\pgfqpoint{0.709087in}{3.392876in}}%
\pgfpathclose%
\pgfusepath{stroke}%
\end{pgfscope}%
\begin{pgfscope}%
\pgfpathrectangle{\pgfqpoint{0.494722in}{0.437222in}}{\pgfqpoint{6.275590in}{5.159444in}}%
\pgfusepath{clip}%
\pgfsetbuttcap%
\pgfsetroundjoin%
\pgfsetlinewidth{1.003750pt}%
\definecolor{currentstroke}{rgb}{0.827451,0.827451,0.827451}%
\pgfsetstrokecolor{currentstroke}%
\pgfsetstrokeopacity{0.800000}%
\pgfsetdash{}{0pt}%
\pgfpathmoveto{\pgfqpoint{1.991938in}{1.503092in}}%
\pgfpathcurveto{\pgfqpoint{2.002988in}{1.503092in}}{\pgfqpoint{2.013587in}{1.507482in}}{\pgfqpoint{2.021400in}{1.515296in}}%
\pgfpathcurveto{\pgfqpoint{2.029214in}{1.523110in}}{\pgfqpoint{2.033604in}{1.533709in}}{\pgfqpoint{2.033604in}{1.544759in}}%
\pgfpathcurveto{\pgfqpoint{2.033604in}{1.555809in}}{\pgfqpoint{2.029214in}{1.566408in}}{\pgfqpoint{2.021400in}{1.574222in}}%
\pgfpathcurveto{\pgfqpoint{2.013587in}{1.582035in}}{\pgfqpoint{2.002988in}{1.586425in}}{\pgfqpoint{1.991938in}{1.586425in}}%
\pgfpathcurveto{\pgfqpoint{1.980887in}{1.586425in}}{\pgfqpoint{1.970288in}{1.582035in}}{\pgfqpoint{1.962475in}{1.574222in}}%
\pgfpathcurveto{\pgfqpoint{1.954661in}{1.566408in}}{\pgfqpoint{1.950271in}{1.555809in}}{\pgfqpoint{1.950271in}{1.544759in}}%
\pgfpathcurveto{\pgfqpoint{1.950271in}{1.533709in}}{\pgfqpoint{1.954661in}{1.523110in}}{\pgfqpoint{1.962475in}{1.515296in}}%
\pgfpathcurveto{\pgfqpoint{1.970288in}{1.507482in}}{\pgfqpoint{1.980887in}{1.503092in}}{\pgfqpoint{1.991938in}{1.503092in}}%
\pgfpathlineto{\pgfqpoint{1.991938in}{1.503092in}}%
\pgfpathclose%
\pgfusepath{stroke}%
\end{pgfscope}%
\begin{pgfscope}%
\pgfpathrectangle{\pgfqpoint{0.494722in}{0.437222in}}{\pgfqpoint{6.275590in}{5.159444in}}%
\pgfusepath{clip}%
\pgfsetbuttcap%
\pgfsetroundjoin%
\pgfsetlinewidth{1.003750pt}%
\definecolor{currentstroke}{rgb}{0.827451,0.827451,0.827451}%
\pgfsetstrokecolor{currentstroke}%
\pgfsetstrokeopacity{0.800000}%
\pgfsetdash{}{0pt}%
\pgfpathmoveto{\pgfqpoint{1.344163in}{2.102732in}}%
\pgfpathcurveto{\pgfqpoint{1.355213in}{2.102732in}}{\pgfqpoint{1.365812in}{2.107122in}}{\pgfqpoint{1.373626in}{2.114936in}}%
\pgfpathcurveto{\pgfqpoint{1.381439in}{2.122749in}}{\pgfqpoint{1.385830in}{2.133348in}}{\pgfqpoint{1.385830in}{2.144398in}}%
\pgfpathcurveto{\pgfqpoint{1.385830in}{2.155448in}}{\pgfqpoint{1.381439in}{2.166047in}}{\pgfqpoint{1.373626in}{2.173861in}}%
\pgfpathcurveto{\pgfqpoint{1.365812in}{2.181675in}}{\pgfqpoint{1.355213in}{2.186065in}}{\pgfqpoint{1.344163in}{2.186065in}}%
\pgfpathcurveto{\pgfqpoint{1.333113in}{2.186065in}}{\pgfqpoint{1.322514in}{2.181675in}}{\pgfqpoint{1.314700in}{2.173861in}}%
\pgfpathcurveto{\pgfqpoint{1.306887in}{2.166047in}}{\pgfqpoint{1.302496in}{2.155448in}}{\pgfqpoint{1.302496in}{2.144398in}}%
\pgfpathcurveto{\pgfqpoint{1.302496in}{2.133348in}}{\pgfqpoint{1.306887in}{2.122749in}}{\pgfqpoint{1.314700in}{2.114936in}}%
\pgfpathcurveto{\pgfqpoint{1.322514in}{2.107122in}}{\pgfqpoint{1.333113in}{2.102732in}}{\pgfqpoint{1.344163in}{2.102732in}}%
\pgfpathlineto{\pgfqpoint{1.344163in}{2.102732in}}%
\pgfpathclose%
\pgfusepath{stroke}%
\end{pgfscope}%
\begin{pgfscope}%
\pgfpathrectangle{\pgfqpoint{0.494722in}{0.437222in}}{\pgfqpoint{6.275590in}{5.159444in}}%
\pgfusepath{clip}%
\pgfsetbuttcap%
\pgfsetroundjoin%
\pgfsetlinewidth{1.003750pt}%
\definecolor{currentstroke}{rgb}{0.827451,0.827451,0.827451}%
\pgfsetstrokecolor{currentstroke}%
\pgfsetstrokeopacity{0.800000}%
\pgfsetdash{}{0pt}%
\pgfpathmoveto{\pgfqpoint{2.429002in}{1.219445in}}%
\pgfpathcurveto{\pgfqpoint{2.440052in}{1.219445in}}{\pgfqpoint{2.450651in}{1.223835in}}{\pgfqpoint{2.458465in}{1.231648in}}%
\pgfpathcurveto{\pgfqpoint{2.466278in}{1.239462in}}{\pgfqpoint{2.470669in}{1.250061in}}{\pgfqpoint{2.470669in}{1.261111in}}%
\pgfpathcurveto{\pgfqpoint{2.470669in}{1.272161in}}{\pgfqpoint{2.466278in}{1.282760in}}{\pgfqpoint{2.458465in}{1.290574in}}%
\pgfpathcurveto{\pgfqpoint{2.450651in}{1.298388in}}{\pgfqpoint{2.440052in}{1.302778in}}{\pgfqpoint{2.429002in}{1.302778in}}%
\pgfpathcurveto{\pgfqpoint{2.417952in}{1.302778in}}{\pgfqpoint{2.407353in}{1.298388in}}{\pgfqpoint{2.399539in}{1.290574in}}%
\pgfpathcurveto{\pgfqpoint{2.391726in}{1.282760in}}{\pgfqpoint{2.387335in}{1.272161in}}{\pgfqpoint{2.387335in}{1.261111in}}%
\pgfpathcurveto{\pgfqpoint{2.387335in}{1.250061in}}{\pgfqpoint{2.391726in}{1.239462in}}{\pgfqpoint{2.399539in}{1.231648in}}%
\pgfpathcurveto{\pgfqpoint{2.407353in}{1.223835in}}{\pgfqpoint{2.417952in}{1.219445in}}{\pgfqpoint{2.429002in}{1.219445in}}%
\pgfpathlineto{\pgfqpoint{2.429002in}{1.219445in}}%
\pgfpathclose%
\pgfusepath{stroke}%
\end{pgfscope}%
\begin{pgfscope}%
\pgfpathrectangle{\pgfqpoint{0.494722in}{0.437222in}}{\pgfqpoint{6.275590in}{5.159444in}}%
\pgfusepath{clip}%
\pgfsetbuttcap%
\pgfsetroundjoin%
\pgfsetlinewidth{1.003750pt}%
\definecolor{currentstroke}{rgb}{0.827451,0.827451,0.827451}%
\pgfsetstrokecolor{currentstroke}%
\pgfsetstrokeopacity{0.800000}%
\pgfsetdash{}{0pt}%
\pgfpathmoveto{\pgfqpoint{0.556707in}{3.885184in}}%
\pgfpathcurveto{\pgfqpoint{0.567757in}{3.885184in}}{\pgfqpoint{0.578356in}{3.889574in}}{\pgfqpoint{0.586170in}{3.897388in}}%
\pgfpathcurveto{\pgfqpoint{0.593983in}{3.905201in}}{\pgfqpoint{0.598374in}{3.915800in}}{\pgfqpoint{0.598374in}{3.926850in}}%
\pgfpathcurveto{\pgfqpoint{0.598374in}{3.937900in}}{\pgfqpoint{0.593983in}{3.948499in}}{\pgfqpoint{0.586170in}{3.956313in}}%
\pgfpathcurveto{\pgfqpoint{0.578356in}{3.964127in}}{\pgfqpoint{0.567757in}{3.968517in}}{\pgfqpoint{0.556707in}{3.968517in}}%
\pgfpathcurveto{\pgfqpoint{0.545657in}{3.968517in}}{\pgfqpoint{0.535058in}{3.964127in}}{\pgfqpoint{0.527244in}{3.956313in}}%
\pgfpathcurveto{\pgfqpoint{0.519430in}{3.948499in}}{\pgfqpoint{0.515040in}{3.937900in}}{\pgfqpoint{0.515040in}{3.926850in}}%
\pgfpathcurveto{\pgfqpoint{0.515040in}{3.915800in}}{\pgfqpoint{0.519430in}{3.905201in}}{\pgfqpoint{0.527244in}{3.897388in}}%
\pgfpathcurveto{\pgfqpoint{0.535058in}{3.889574in}}{\pgfqpoint{0.545657in}{3.885184in}}{\pgfqpoint{0.556707in}{3.885184in}}%
\pgfpathlineto{\pgfqpoint{0.556707in}{3.885184in}}%
\pgfpathclose%
\pgfusepath{stroke}%
\end{pgfscope}%
\begin{pgfscope}%
\pgfpathrectangle{\pgfqpoint{0.494722in}{0.437222in}}{\pgfqpoint{6.275590in}{5.159444in}}%
\pgfusepath{clip}%
\pgfsetbuttcap%
\pgfsetroundjoin%
\pgfsetlinewidth{1.003750pt}%
\definecolor{currentstroke}{rgb}{0.827451,0.827451,0.827451}%
\pgfsetstrokecolor{currentstroke}%
\pgfsetstrokeopacity{0.800000}%
\pgfsetdash{}{0pt}%
\pgfpathmoveto{\pgfqpoint{1.278200in}{2.184905in}}%
\pgfpathcurveto{\pgfqpoint{1.289250in}{2.184905in}}{\pgfqpoint{1.299849in}{2.189295in}}{\pgfqpoint{1.307663in}{2.197109in}}%
\pgfpathcurveto{\pgfqpoint{1.315477in}{2.204922in}}{\pgfqpoint{1.319867in}{2.215521in}}{\pgfqpoint{1.319867in}{2.226572in}}%
\pgfpathcurveto{\pgfqpoint{1.319867in}{2.237622in}}{\pgfqpoint{1.315477in}{2.248221in}}{\pgfqpoint{1.307663in}{2.256034in}}%
\pgfpathcurveto{\pgfqpoint{1.299849in}{2.263848in}}{\pgfqpoint{1.289250in}{2.268238in}}{\pgfqpoint{1.278200in}{2.268238in}}%
\pgfpathcurveto{\pgfqpoint{1.267150in}{2.268238in}}{\pgfqpoint{1.256551in}{2.263848in}}{\pgfqpoint{1.248738in}{2.256034in}}%
\pgfpathcurveto{\pgfqpoint{1.240924in}{2.248221in}}{\pgfqpoint{1.236534in}{2.237622in}}{\pgfqpoint{1.236534in}{2.226572in}}%
\pgfpathcurveto{\pgfqpoint{1.236534in}{2.215521in}}{\pgfqpoint{1.240924in}{2.204922in}}{\pgfqpoint{1.248738in}{2.197109in}}%
\pgfpathcurveto{\pgfqpoint{1.256551in}{2.189295in}}{\pgfqpoint{1.267150in}{2.184905in}}{\pgfqpoint{1.278200in}{2.184905in}}%
\pgfpathlineto{\pgfqpoint{1.278200in}{2.184905in}}%
\pgfpathclose%
\pgfusepath{stroke}%
\end{pgfscope}%
\begin{pgfscope}%
\pgfpathrectangle{\pgfqpoint{0.494722in}{0.437222in}}{\pgfqpoint{6.275590in}{5.159444in}}%
\pgfusepath{clip}%
\pgfsetbuttcap%
\pgfsetroundjoin%
\pgfsetlinewidth{1.003750pt}%
\definecolor{currentstroke}{rgb}{0.827451,0.827451,0.827451}%
\pgfsetstrokecolor{currentstroke}%
\pgfsetstrokeopacity{0.800000}%
\pgfsetdash{}{0pt}%
\pgfpathmoveto{\pgfqpoint{3.164762in}{0.836209in}}%
\pgfpathcurveto{\pgfqpoint{3.175812in}{0.836209in}}{\pgfqpoint{3.186411in}{0.840599in}}{\pgfqpoint{3.194224in}{0.848413in}}%
\pgfpathcurveto{\pgfqpoint{3.202038in}{0.856226in}}{\pgfqpoint{3.206428in}{0.866825in}}{\pgfqpoint{3.206428in}{0.877875in}}%
\pgfpathcurveto{\pgfqpoint{3.206428in}{0.888926in}}{\pgfqpoint{3.202038in}{0.899525in}}{\pgfqpoint{3.194224in}{0.907338in}}%
\pgfpathcurveto{\pgfqpoint{3.186411in}{0.915152in}}{\pgfqpoint{3.175812in}{0.919542in}}{\pgfqpoint{3.164762in}{0.919542in}}%
\pgfpathcurveto{\pgfqpoint{3.153711in}{0.919542in}}{\pgfqpoint{3.143112in}{0.915152in}}{\pgfqpoint{3.135299in}{0.907338in}}%
\pgfpathcurveto{\pgfqpoint{3.127485in}{0.899525in}}{\pgfqpoint{3.123095in}{0.888926in}}{\pgfqpoint{3.123095in}{0.877875in}}%
\pgfpathcurveto{\pgfqpoint{3.123095in}{0.866825in}}{\pgfqpoint{3.127485in}{0.856226in}}{\pgfqpoint{3.135299in}{0.848413in}}%
\pgfpathcurveto{\pgfqpoint{3.143112in}{0.840599in}}{\pgfqpoint{3.153711in}{0.836209in}}{\pgfqpoint{3.164762in}{0.836209in}}%
\pgfpathlineto{\pgfqpoint{3.164762in}{0.836209in}}%
\pgfpathclose%
\pgfusepath{stroke}%
\end{pgfscope}%
\begin{pgfscope}%
\pgfpathrectangle{\pgfqpoint{0.494722in}{0.437222in}}{\pgfqpoint{6.275590in}{5.159444in}}%
\pgfusepath{clip}%
\pgfsetbuttcap%
\pgfsetroundjoin%
\pgfsetlinewidth{1.003750pt}%
\definecolor{currentstroke}{rgb}{0.827451,0.827451,0.827451}%
\pgfsetstrokecolor{currentstroke}%
\pgfsetstrokeopacity{0.800000}%
\pgfsetdash{}{0pt}%
\pgfpathmoveto{\pgfqpoint{4.335578in}{0.505432in}}%
\pgfpathcurveto{\pgfqpoint{4.346628in}{0.505432in}}{\pgfqpoint{4.357227in}{0.509823in}}{\pgfqpoint{4.365041in}{0.517636in}}%
\pgfpathcurveto{\pgfqpoint{4.372855in}{0.525450in}}{\pgfqpoint{4.377245in}{0.536049in}}{\pgfqpoint{4.377245in}{0.547099in}}%
\pgfpathcurveto{\pgfqpoint{4.377245in}{0.558149in}}{\pgfqpoint{4.372855in}{0.568748in}}{\pgfqpoint{4.365041in}{0.576562in}}%
\pgfpathcurveto{\pgfqpoint{4.357227in}{0.584375in}}{\pgfqpoint{4.346628in}{0.588766in}}{\pgfqpoint{4.335578in}{0.588766in}}%
\pgfpathcurveto{\pgfqpoint{4.324528in}{0.588766in}}{\pgfqpoint{4.313929in}{0.584375in}}{\pgfqpoint{4.306115in}{0.576562in}}%
\pgfpathcurveto{\pgfqpoint{4.298302in}{0.568748in}}{\pgfqpoint{4.293911in}{0.558149in}}{\pgfqpoint{4.293911in}{0.547099in}}%
\pgfpathcurveto{\pgfqpoint{4.293911in}{0.536049in}}{\pgfqpoint{4.298302in}{0.525450in}}{\pgfqpoint{4.306115in}{0.517636in}}%
\pgfpathcurveto{\pgfqpoint{4.313929in}{0.509823in}}{\pgfqpoint{4.324528in}{0.505432in}}{\pgfqpoint{4.335578in}{0.505432in}}%
\pgfpathlineto{\pgfqpoint{4.335578in}{0.505432in}}%
\pgfpathclose%
\pgfusepath{stroke}%
\end{pgfscope}%
\begin{pgfscope}%
\pgfpathrectangle{\pgfqpoint{0.494722in}{0.437222in}}{\pgfqpoint{6.275590in}{5.159444in}}%
\pgfusepath{clip}%
\pgfsetbuttcap%
\pgfsetroundjoin%
\pgfsetlinewidth{1.003750pt}%
\definecolor{currentstroke}{rgb}{0.827451,0.827451,0.827451}%
\pgfsetstrokecolor{currentstroke}%
\pgfsetstrokeopacity{0.800000}%
\pgfsetdash{}{0pt}%
\pgfpathmoveto{\pgfqpoint{2.656452in}{1.108305in}}%
\pgfpathcurveto{\pgfqpoint{2.667502in}{1.108305in}}{\pgfqpoint{2.678102in}{1.112695in}}{\pgfqpoint{2.685915in}{1.120509in}}%
\pgfpathcurveto{\pgfqpoint{2.693729in}{1.128322in}}{\pgfqpoint{2.698119in}{1.138921in}}{\pgfqpoint{2.698119in}{1.149971in}}%
\pgfpathcurveto{\pgfqpoint{2.698119in}{1.161021in}}{\pgfqpoint{2.693729in}{1.171620in}}{\pgfqpoint{2.685915in}{1.179434in}}%
\pgfpathcurveto{\pgfqpoint{2.678102in}{1.187248in}}{\pgfqpoint{2.667502in}{1.191638in}}{\pgfqpoint{2.656452in}{1.191638in}}%
\pgfpathcurveto{\pgfqpoint{2.645402in}{1.191638in}}{\pgfqpoint{2.634803in}{1.187248in}}{\pgfqpoint{2.626990in}{1.179434in}}%
\pgfpathcurveto{\pgfqpoint{2.619176in}{1.171620in}}{\pgfqpoint{2.614786in}{1.161021in}}{\pgfqpoint{2.614786in}{1.149971in}}%
\pgfpathcurveto{\pgfqpoint{2.614786in}{1.138921in}}{\pgfqpoint{2.619176in}{1.128322in}}{\pgfqpoint{2.626990in}{1.120509in}}%
\pgfpathcurveto{\pgfqpoint{2.634803in}{1.112695in}}{\pgfqpoint{2.645402in}{1.108305in}}{\pgfqpoint{2.656452in}{1.108305in}}%
\pgfpathlineto{\pgfqpoint{2.656452in}{1.108305in}}%
\pgfpathclose%
\pgfusepath{stroke}%
\end{pgfscope}%
\begin{pgfscope}%
\pgfpathrectangle{\pgfqpoint{0.494722in}{0.437222in}}{\pgfqpoint{6.275590in}{5.159444in}}%
\pgfusepath{clip}%
\pgfsetbuttcap%
\pgfsetroundjoin%
\pgfsetlinewidth{1.003750pt}%
\definecolor{currentstroke}{rgb}{0.827451,0.827451,0.827451}%
\pgfsetstrokecolor{currentstroke}%
\pgfsetstrokeopacity{0.800000}%
\pgfsetdash{}{0pt}%
\pgfpathmoveto{\pgfqpoint{0.575099in}{3.779944in}}%
\pgfpathcurveto{\pgfqpoint{0.586149in}{3.779944in}}{\pgfqpoint{0.596748in}{3.784334in}}{\pgfqpoint{0.604561in}{3.792148in}}%
\pgfpathcurveto{\pgfqpoint{0.612375in}{3.799961in}}{\pgfqpoint{0.616765in}{3.810560in}}{\pgfqpoint{0.616765in}{3.821610in}}%
\pgfpathcurveto{\pgfqpoint{0.616765in}{3.832661in}}{\pgfqpoint{0.612375in}{3.843260in}}{\pgfqpoint{0.604561in}{3.851073in}}%
\pgfpathcurveto{\pgfqpoint{0.596748in}{3.858887in}}{\pgfqpoint{0.586149in}{3.863277in}}{\pgfqpoint{0.575099in}{3.863277in}}%
\pgfpathcurveto{\pgfqpoint{0.564048in}{3.863277in}}{\pgfqpoint{0.553449in}{3.858887in}}{\pgfqpoint{0.545636in}{3.851073in}}%
\pgfpathcurveto{\pgfqpoint{0.537822in}{3.843260in}}{\pgfqpoint{0.533432in}{3.832661in}}{\pgfqpoint{0.533432in}{3.821610in}}%
\pgfpathcurveto{\pgfqpoint{0.533432in}{3.810560in}}{\pgfqpoint{0.537822in}{3.799961in}}{\pgfqpoint{0.545636in}{3.792148in}}%
\pgfpathcurveto{\pgfqpoint{0.553449in}{3.784334in}}{\pgfqpoint{0.564048in}{3.779944in}}{\pgfqpoint{0.575099in}{3.779944in}}%
\pgfpathlineto{\pgfqpoint{0.575099in}{3.779944in}}%
\pgfpathclose%
\pgfusepath{stroke}%
\end{pgfscope}%
\begin{pgfscope}%
\pgfpathrectangle{\pgfqpoint{0.494722in}{0.437222in}}{\pgfqpoint{6.275590in}{5.159444in}}%
\pgfusepath{clip}%
\pgfsetbuttcap%
\pgfsetroundjoin%
\pgfsetlinewidth{1.003750pt}%
\definecolor{currentstroke}{rgb}{0.827451,0.827451,0.827451}%
\pgfsetstrokecolor{currentstroke}%
\pgfsetstrokeopacity{0.800000}%
\pgfsetdash{}{0pt}%
\pgfpathmoveto{\pgfqpoint{1.581391in}{1.870889in}}%
\pgfpathcurveto{\pgfqpoint{1.592441in}{1.870889in}}{\pgfqpoint{1.603040in}{1.875279in}}{\pgfqpoint{1.610854in}{1.883093in}}%
\pgfpathcurveto{\pgfqpoint{1.618667in}{1.890906in}}{\pgfqpoint{1.623058in}{1.901505in}}{\pgfqpoint{1.623058in}{1.912556in}}%
\pgfpathcurveto{\pgfqpoint{1.623058in}{1.923606in}}{\pgfqpoint{1.618667in}{1.934205in}}{\pgfqpoint{1.610854in}{1.942018in}}%
\pgfpathcurveto{\pgfqpoint{1.603040in}{1.949832in}}{\pgfqpoint{1.592441in}{1.954222in}}{\pgfqpoint{1.581391in}{1.954222in}}%
\pgfpathcurveto{\pgfqpoint{1.570341in}{1.954222in}}{\pgfqpoint{1.559742in}{1.949832in}}{\pgfqpoint{1.551928in}{1.942018in}}%
\pgfpathcurveto{\pgfqpoint{1.544115in}{1.934205in}}{\pgfqpoint{1.539724in}{1.923606in}}{\pgfqpoint{1.539724in}{1.912556in}}%
\pgfpathcurveto{\pgfqpoint{1.539724in}{1.901505in}}{\pgfqpoint{1.544115in}{1.890906in}}{\pgfqpoint{1.551928in}{1.883093in}}%
\pgfpathcurveto{\pgfqpoint{1.559742in}{1.875279in}}{\pgfqpoint{1.570341in}{1.870889in}}{\pgfqpoint{1.581391in}{1.870889in}}%
\pgfpathlineto{\pgfqpoint{1.581391in}{1.870889in}}%
\pgfpathclose%
\pgfusepath{stroke}%
\end{pgfscope}%
\begin{pgfscope}%
\pgfpathrectangle{\pgfqpoint{0.494722in}{0.437222in}}{\pgfqpoint{6.275590in}{5.159444in}}%
\pgfusepath{clip}%
\pgfsetbuttcap%
\pgfsetroundjoin%
\pgfsetlinewidth{1.003750pt}%
\definecolor{currentstroke}{rgb}{0.827451,0.827451,0.827451}%
\pgfsetstrokecolor{currentstroke}%
\pgfsetstrokeopacity{0.800000}%
\pgfsetdash{}{0pt}%
\pgfpathmoveto{\pgfqpoint{1.772500in}{1.679456in}}%
\pgfpathcurveto{\pgfqpoint{1.783550in}{1.679456in}}{\pgfqpoint{1.794149in}{1.683846in}}{\pgfqpoint{1.801962in}{1.691660in}}%
\pgfpathcurveto{\pgfqpoint{1.809776in}{1.699473in}}{\pgfqpoint{1.814166in}{1.710072in}}{\pgfqpoint{1.814166in}{1.721122in}}%
\pgfpathcurveto{\pgfqpoint{1.814166in}{1.732172in}}{\pgfqpoint{1.809776in}{1.742772in}}{\pgfqpoint{1.801962in}{1.750585in}}%
\pgfpathcurveto{\pgfqpoint{1.794149in}{1.758399in}}{\pgfqpoint{1.783550in}{1.762789in}}{\pgfqpoint{1.772500in}{1.762789in}}%
\pgfpathcurveto{\pgfqpoint{1.761450in}{1.762789in}}{\pgfqpoint{1.750850in}{1.758399in}}{\pgfqpoint{1.743037in}{1.750585in}}%
\pgfpathcurveto{\pgfqpoint{1.735223in}{1.742772in}}{\pgfqpoint{1.730833in}{1.732172in}}{\pgfqpoint{1.730833in}{1.721122in}}%
\pgfpathcurveto{\pgfqpoint{1.730833in}{1.710072in}}{\pgfqpoint{1.735223in}{1.699473in}}{\pgfqpoint{1.743037in}{1.691660in}}%
\pgfpathcurveto{\pgfqpoint{1.750850in}{1.683846in}}{\pgfqpoint{1.761450in}{1.679456in}}{\pgfqpoint{1.772500in}{1.679456in}}%
\pgfpathlineto{\pgfqpoint{1.772500in}{1.679456in}}%
\pgfpathclose%
\pgfusepath{stroke}%
\end{pgfscope}%
\begin{pgfscope}%
\pgfpathrectangle{\pgfqpoint{0.494722in}{0.437222in}}{\pgfqpoint{6.275590in}{5.159444in}}%
\pgfusepath{clip}%
\pgfsetbuttcap%
\pgfsetroundjoin%
\pgfsetlinewidth{1.003750pt}%
\definecolor{currentstroke}{rgb}{0.827451,0.827451,0.827451}%
\pgfsetstrokecolor{currentstroke}%
\pgfsetstrokeopacity{0.800000}%
\pgfsetdash{}{0pt}%
\pgfpathmoveto{\pgfqpoint{3.652979in}{0.754601in}}%
\pgfpathcurveto{\pgfqpoint{3.664029in}{0.754601in}}{\pgfqpoint{3.674628in}{0.758991in}}{\pgfqpoint{3.682442in}{0.766805in}}%
\pgfpathcurveto{\pgfqpoint{3.690255in}{0.774619in}}{\pgfqpoint{3.694645in}{0.785218in}}{\pgfqpoint{3.694645in}{0.796268in}}%
\pgfpathcurveto{\pgfqpoint{3.694645in}{0.807318in}}{\pgfqpoint{3.690255in}{0.817917in}}{\pgfqpoint{3.682442in}{0.825731in}}%
\pgfpathcurveto{\pgfqpoint{3.674628in}{0.833544in}}{\pgfqpoint{3.664029in}{0.837935in}}{\pgfqpoint{3.652979in}{0.837935in}}%
\pgfpathcurveto{\pgfqpoint{3.641929in}{0.837935in}}{\pgfqpoint{3.631330in}{0.833544in}}{\pgfqpoint{3.623516in}{0.825731in}}%
\pgfpathcurveto{\pgfqpoint{3.615702in}{0.817917in}}{\pgfqpoint{3.611312in}{0.807318in}}{\pgfqpoint{3.611312in}{0.796268in}}%
\pgfpathcurveto{\pgfqpoint{3.611312in}{0.785218in}}{\pgfqpoint{3.615702in}{0.774619in}}{\pgfqpoint{3.623516in}{0.766805in}}%
\pgfpathcurveto{\pgfqpoint{3.631330in}{0.758991in}}{\pgfqpoint{3.641929in}{0.754601in}}{\pgfqpoint{3.652979in}{0.754601in}}%
\pgfpathlineto{\pgfqpoint{3.652979in}{0.754601in}}%
\pgfpathclose%
\pgfusepath{stroke}%
\end{pgfscope}%
\begin{pgfscope}%
\pgfpathrectangle{\pgfqpoint{0.494722in}{0.437222in}}{\pgfqpoint{6.275590in}{5.159444in}}%
\pgfusepath{clip}%
\pgfsetbuttcap%
\pgfsetroundjoin%
\pgfsetlinewidth{1.003750pt}%
\definecolor{currentstroke}{rgb}{0.827451,0.827451,0.827451}%
\pgfsetstrokecolor{currentstroke}%
\pgfsetstrokeopacity{0.800000}%
\pgfsetdash{}{0pt}%
\pgfpathmoveto{\pgfqpoint{1.042231in}{2.681267in}}%
\pgfpathcurveto{\pgfqpoint{1.053281in}{2.681267in}}{\pgfqpoint{1.063880in}{2.685657in}}{\pgfqpoint{1.071694in}{2.693471in}}%
\pgfpathcurveto{\pgfqpoint{1.079507in}{2.701284in}}{\pgfqpoint{1.083898in}{2.711883in}}{\pgfqpoint{1.083898in}{2.722933in}}%
\pgfpathcurveto{\pgfqpoint{1.083898in}{2.733984in}}{\pgfqpoint{1.079507in}{2.744583in}}{\pgfqpoint{1.071694in}{2.752396in}}%
\pgfpathcurveto{\pgfqpoint{1.063880in}{2.760210in}}{\pgfqpoint{1.053281in}{2.764600in}}{\pgfqpoint{1.042231in}{2.764600in}}%
\pgfpathcurveto{\pgfqpoint{1.031181in}{2.764600in}}{\pgfqpoint{1.020582in}{2.760210in}}{\pgfqpoint{1.012768in}{2.752396in}}%
\pgfpathcurveto{\pgfqpoint{1.004954in}{2.744583in}}{\pgfqpoint{1.000564in}{2.733984in}}{\pgfqpoint{1.000564in}{2.722933in}}%
\pgfpathcurveto{\pgfqpoint{1.000564in}{2.711883in}}{\pgfqpoint{1.004954in}{2.701284in}}{\pgfqpoint{1.012768in}{2.693471in}}%
\pgfpathcurveto{\pgfqpoint{1.020582in}{2.685657in}}{\pgfqpoint{1.031181in}{2.681267in}}{\pgfqpoint{1.042231in}{2.681267in}}%
\pgfpathlineto{\pgfqpoint{1.042231in}{2.681267in}}%
\pgfpathclose%
\pgfusepath{stroke}%
\end{pgfscope}%
\begin{pgfscope}%
\pgfpathrectangle{\pgfqpoint{0.494722in}{0.437222in}}{\pgfqpoint{6.275590in}{5.159444in}}%
\pgfusepath{clip}%
\pgfsetbuttcap%
\pgfsetroundjoin%
\pgfsetlinewidth{1.003750pt}%
\definecolor{currentstroke}{rgb}{0.827451,0.827451,0.827451}%
\pgfsetstrokecolor{currentstroke}%
\pgfsetstrokeopacity{0.800000}%
\pgfsetdash{}{0pt}%
\pgfpathmoveto{\pgfqpoint{2.867283in}{0.963868in}}%
\pgfpathcurveto{\pgfqpoint{2.878333in}{0.963868in}}{\pgfqpoint{2.888932in}{0.968258in}}{\pgfqpoint{2.896746in}{0.976072in}}%
\pgfpathcurveto{\pgfqpoint{2.904560in}{0.983885in}}{\pgfqpoint{2.908950in}{0.994484in}}{\pgfqpoint{2.908950in}{1.005535in}}%
\pgfpathcurveto{\pgfqpoint{2.908950in}{1.016585in}}{\pgfqpoint{2.904560in}{1.027184in}}{\pgfqpoint{2.896746in}{1.034997in}}%
\pgfpathcurveto{\pgfqpoint{2.888932in}{1.042811in}}{\pgfqpoint{2.878333in}{1.047201in}}{\pgfqpoint{2.867283in}{1.047201in}}%
\pgfpathcurveto{\pgfqpoint{2.856233in}{1.047201in}}{\pgfqpoint{2.845634in}{1.042811in}}{\pgfqpoint{2.837820in}{1.034997in}}%
\pgfpathcurveto{\pgfqpoint{2.830007in}{1.027184in}}{\pgfqpoint{2.825617in}{1.016585in}}{\pgfqpoint{2.825617in}{1.005535in}}%
\pgfpathcurveto{\pgfqpoint{2.825617in}{0.994484in}}{\pgfqpoint{2.830007in}{0.983885in}}{\pgfqpoint{2.837820in}{0.976072in}}%
\pgfpathcurveto{\pgfqpoint{2.845634in}{0.968258in}}{\pgfqpoint{2.856233in}{0.963868in}}{\pgfqpoint{2.867283in}{0.963868in}}%
\pgfpathlineto{\pgfqpoint{2.867283in}{0.963868in}}%
\pgfpathclose%
\pgfusepath{stroke}%
\end{pgfscope}%
\begin{pgfscope}%
\pgfpathrectangle{\pgfqpoint{0.494722in}{0.437222in}}{\pgfqpoint{6.275590in}{5.159444in}}%
\pgfusepath{clip}%
\pgfsetbuttcap%
\pgfsetroundjoin%
\pgfsetlinewidth{1.003750pt}%
\definecolor{currentstroke}{rgb}{0.827451,0.827451,0.827451}%
\pgfsetstrokecolor{currentstroke}%
\pgfsetstrokeopacity{0.800000}%
\pgfsetdash{}{0pt}%
\pgfpathmoveto{\pgfqpoint{1.756385in}{1.772664in}}%
\pgfpathcurveto{\pgfqpoint{1.767436in}{1.772664in}}{\pgfqpoint{1.778035in}{1.777054in}}{\pgfqpoint{1.785848in}{1.784868in}}%
\pgfpathcurveto{\pgfqpoint{1.793662in}{1.792682in}}{\pgfqpoint{1.798052in}{1.803281in}}{\pgfqpoint{1.798052in}{1.814331in}}%
\pgfpathcurveto{\pgfqpoint{1.798052in}{1.825381in}}{\pgfqpoint{1.793662in}{1.835980in}}{\pgfqpoint{1.785848in}{1.843793in}}%
\pgfpathcurveto{\pgfqpoint{1.778035in}{1.851607in}}{\pgfqpoint{1.767436in}{1.855997in}}{\pgfqpoint{1.756385in}{1.855997in}}%
\pgfpathcurveto{\pgfqpoint{1.745335in}{1.855997in}}{\pgfqpoint{1.734736in}{1.851607in}}{\pgfqpoint{1.726923in}{1.843793in}}%
\pgfpathcurveto{\pgfqpoint{1.719109in}{1.835980in}}{\pgfqpoint{1.714719in}{1.825381in}}{\pgfqpoint{1.714719in}{1.814331in}}%
\pgfpathcurveto{\pgfqpoint{1.714719in}{1.803281in}}{\pgfqpoint{1.719109in}{1.792682in}}{\pgfqpoint{1.726923in}{1.784868in}}%
\pgfpathcurveto{\pgfqpoint{1.734736in}{1.777054in}}{\pgfqpoint{1.745335in}{1.772664in}}{\pgfqpoint{1.756385in}{1.772664in}}%
\pgfpathlineto{\pgfqpoint{1.756385in}{1.772664in}}%
\pgfpathclose%
\pgfusepath{stroke}%
\end{pgfscope}%
\begin{pgfscope}%
\pgfpathrectangle{\pgfqpoint{0.494722in}{0.437222in}}{\pgfqpoint{6.275590in}{5.159444in}}%
\pgfusepath{clip}%
\pgfsetbuttcap%
\pgfsetroundjoin%
\pgfsetlinewidth{1.003750pt}%
\definecolor{currentstroke}{rgb}{0.827451,0.827451,0.827451}%
\pgfsetstrokecolor{currentstroke}%
\pgfsetstrokeopacity{0.800000}%
\pgfsetdash{}{0pt}%
\pgfpathmoveto{\pgfqpoint{2.572850in}{1.116571in}}%
\pgfpathcurveto{\pgfqpoint{2.583900in}{1.116571in}}{\pgfqpoint{2.594499in}{1.120961in}}{\pgfqpoint{2.602313in}{1.128774in}}%
\pgfpathcurveto{\pgfqpoint{2.610126in}{1.136588in}}{\pgfqpoint{2.614517in}{1.147187in}}{\pgfqpoint{2.614517in}{1.158237in}}%
\pgfpathcurveto{\pgfqpoint{2.614517in}{1.169287in}}{\pgfqpoint{2.610126in}{1.179886in}}{\pgfqpoint{2.602313in}{1.187700in}}%
\pgfpathcurveto{\pgfqpoint{2.594499in}{1.195514in}}{\pgfqpoint{2.583900in}{1.199904in}}{\pgfqpoint{2.572850in}{1.199904in}}%
\pgfpathcurveto{\pgfqpoint{2.561800in}{1.199904in}}{\pgfqpoint{2.551201in}{1.195514in}}{\pgfqpoint{2.543387in}{1.187700in}}%
\pgfpathcurveto{\pgfqpoint{2.535574in}{1.179886in}}{\pgfqpoint{2.531183in}{1.169287in}}{\pgfqpoint{2.531183in}{1.158237in}}%
\pgfpathcurveto{\pgfqpoint{2.531183in}{1.147187in}}{\pgfqpoint{2.535574in}{1.136588in}}{\pgfqpoint{2.543387in}{1.128774in}}%
\pgfpathcurveto{\pgfqpoint{2.551201in}{1.120961in}}{\pgfqpoint{2.561800in}{1.116571in}}{\pgfqpoint{2.572850in}{1.116571in}}%
\pgfpathlineto{\pgfqpoint{2.572850in}{1.116571in}}%
\pgfpathclose%
\pgfusepath{stroke}%
\end{pgfscope}%
\begin{pgfscope}%
\pgfpathrectangle{\pgfqpoint{0.494722in}{0.437222in}}{\pgfqpoint{6.275590in}{5.159444in}}%
\pgfusepath{clip}%
\pgfsetbuttcap%
\pgfsetroundjoin%
\pgfsetlinewidth{1.003750pt}%
\definecolor{currentstroke}{rgb}{0.827451,0.827451,0.827451}%
\pgfsetstrokecolor{currentstroke}%
\pgfsetstrokeopacity{0.800000}%
\pgfsetdash{}{0pt}%
\pgfpathmoveto{\pgfqpoint{0.963748in}{2.873526in}}%
\pgfpathcurveto{\pgfqpoint{0.974798in}{2.873526in}}{\pgfqpoint{0.985398in}{2.877916in}}{\pgfqpoint{0.993211in}{2.885730in}}%
\pgfpathcurveto{\pgfqpoint{1.001025in}{2.893543in}}{\pgfqpoint{1.005415in}{2.904142in}}{\pgfqpoint{1.005415in}{2.915192in}}%
\pgfpathcurveto{\pgfqpoint{1.005415in}{2.926243in}}{\pgfqpoint{1.001025in}{2.936842in}}{\pgfqpoint{0.993211in}{2.944655in}}%
\pgfpathcurveto{\pgfqpoint{0.985398in}{2.952469in}}{\pgfqpoint{0.974798in}{2.956859in}}{\pgfqpoint{0.963748in}{2.956859in}}%
\pgfpathcurveto{\pgfqpoint{0.952698in}{2.956859in}}{\pgfqpoint{0.942099in}{2.952469in}}{\pgfqpoint{0.934286in}{2.944655in}}%
\pgfpathcurveto{\pgfqpoint{0.926472in}{2.936842in}}{\pgfqpoint{0.922082in}{2.926243in}}{\pgfqpoint{0.922082in}{2.915192in}}%
\pgfpathcurveto{\pgfqpoint{0.922082in}{2.904142in}}{\pgfqpoint{0.926472in}{2.893543in}}{\pgfqpoint{0.934286in}{2.885730in}}%
\pgfpathcurveto{\pgfqpoint{0.942099in}{2.877916in}}{\pgfqpoint{0.952698in}{2.873526in}}{\pgfqpoint{0.963748in}{2.873526in}}%
\pgfpathlineto{\pgfqpoint{0.963748in}{2.873526in}}%
\pgfpathclose%
\pgfusepath{stroke}%
\end{pgfscope}%
\begin{pgfscope}%
\pgfpathrectangle{\pgfqpoint{0.494722in}{0.437222in}}{\pgfqpoint{6.275590in}{5.159444in}}%
\pgfusepath{clip}%
\pgfsetbuttcap%
\pgfsetroundjoin%
\pgfsetlinewidth{1.003750pt}%
\definecolor{currentstroke}{rgb}{0.827451,0.827451,0.827451}%
\pgfsetstrokecolor{currentstroke}%
\pgfsetstrokeopacity{0.800000}%
\pgfsetdash{}{0pt}%
\pgfpathmoveto{\pgfqpoint{2.512961in}{1.165916in}}%
\pgfpathcurveto{\pgfqpoint{2.524011in}{1.165916in}}{\pgfqpoint{2.534610in}{1.170306in}}{\pgfqpoint{2.542424in}{1.178119in}}%
\pgfpathcurveto{\pgfqpoint{2.550237in}{1.185933in}}{\pgfqpoint{2.554627in}{1.196532in}}{\pgfqpoint{2.554627in}{1.207582in}}%
\pgfpathcurveto{\pgfqpoint{2.554627in}{1.218632in}}{\pgfqpoint{2.550237in}{1.229231in}}{\pgfqpoint{2.542424in}{1.237045in}}%
\pgfpathcurveto{\pgfqpoint{2.534610in}{1.244859in}}{\pgfqpoint{2.524011in}{1.249249in}}{\pgfqpoint{2.512961in}{1.249249in}}%
\pgfpathcurveto{\pgfqpoint{2.501911in}{1.249249in}}{\pgfqpoint{2.491312in}{1.244859in}}{\pgfqpoint{2.483498in}{1.237045in}}%
\pgfpathcurveto{\pgfqpoint{2.475684in}{1.229231in}}{\pgfqpoint{2.471294in}{1.218632in}}{\pgfqpoint{2.471294in}{1.207582in}}%
\pgfpathcurveto{\pgfqpoint{2.471294in}{1.196532in}}{\pgfqpoint{2.475684in}{1.185933in}}{\pgfqpoint{2.483498in}{1.178119in}}%
\pgfpathcurveto{\pgfqpoint{2.491312in}{1.170306in}}{\pgfqpoint{2.501911in}{1.165916in}}{\pgfqpoint{2.512961in}{1.165916in}}%
\pgfpathlineto{\pgfqpoint{2.512961in}{1.165916in}}%
\pgfpathclose%
\pgfusepath{stroke}%
\end{pgfscope}%
\begin{pgfscope}%
\pgfpathrectangle{\pgfqpoint{0.494722in}{0.437222in}}{\pgfqpoint{6.275590in}{5.159444in}}%
\pgfusepath{clip}%
\pgfsetbuttcap%
\pgfsetroundjoin%
\pgfsetlinewidth{1.003750pt}%
\definecolor{currentstroke}{rgb}{0.827451,0.827451,0.827451}%
\pgfsetstrokecolor{currentstroke}%
\pgfsetstrokeopacity{0.800000}%
\pgfsetdash{}{0pt}%
\pgfpathmoveto{\pgfqpoint{1.016820in}{2.832444in}}%
\pgfpathcurveto{\pgfqpoint{1.027870in}{2.832444in}}{\pgfqpoint{1.038469in}{2.836834in}}{\pgfqpoint{1.046283in}{2.844648in}}%
\pgfpathcurveto{\pgfqpoint{1.054096in}{2.852462in}}{\pgfqpoint{1.058487in}{2.863061in}}{\pgfqpoint{1.058487in}{2.874111in}}%
\pgfpathcurveto{\pgfqpoint{1.058487in}{2.885161in}}{\pgfqpoint{1.054096in}{2.895760in}}{\pgfqpoint{1.046283in}{2.903573in}}%
\pgfpathcurveto{\pgfqpoint{1.038469in}{2.911387in}}{\pgfqpoint{1.027870in}{2.915777in}}{\pgfqpoint{1.016820in}{2.915777in}}%
\pgfpathcurveto{\pgfqpoint{1.005770in}{2.915777in}}{\pgfqpoint{0.995171in}{2.911387in}}{\pgfqpoint{0.987357in}{2.903573in}}%
\pgfpathcurveto{\pgfqpoint{0.979544in}{2.895760in}}{\pgfqpoint{0.975153in}{2.885161in}}{\pgfqpoint{0.975153in}{2.874111in}}%
\pgfpathcurveto{\pgfqpoint{0.975153in}{2.863061in}}{\pgfqpoint{0.979544in}{2.852462in}}{\pgfqpoint{0.987357in}{2.844648in}}%
\pgfpathcurveto{\pgfqpoint{0.995171in}{2.836834in}}{\pgfqpoint{1.005770in}{2.832444in}}{\pgfqpoint{1.016820in}{2.832444in}}%
\pgfpathlineto{\pgfqpoint{1.016820in}{2.832444in}}%
\pgfpathclose%
\pgfusepath{stroke}%
\end{pgfscope}%
\begin{pgfscope}%
\pgfpathrectangle{\pgfqpoint{0.494722in}{0.437222in}}{\pgfqpoint{6.275590in}{5.159444in}}%
\pgfusepath{clip}%
\pgfsetbuttcap%
\pgfsetroundjoin%
\pgfsetlinewidth{1.003750pt}%
\definecolor{currentstroke}{rgb}{0.827451,0.827451,0.827451}%
\pgfsetstrokecolor{currentstroke}%
\pgfsetstrokeopacity{0.800000}%
\pgfsetdash{}{0pt}%
\pgfpathmoveto{\pgfqpoint{3.709744in}{0.686390in}}%
\pgfpathcurveto{\pgfqpoint{3.720794in}{0.686390in}}{\pgfqpoint{3.731393in}{0.690780in}}{\pgfqpoint{3.739207in}{0.698593in}}%
\pgfpathcurveto{\pgfqpoint{3.747021in}{0.706407in}}{\pgfqpoint{3.751411in}{0.717006in}}{\pgfqpoint{3.751411in}{0.728056in}}%
\pgfpathcurveto{\pgfqpoint{3.751411in}{0.739106in}}{\pgfqpoint{3.747021in}{0.749705in}}{\pgfqpoint{3.739207in}{0.757519in}}%
\pgfpathcurveto{\pgfqpoint{3.731393in}{0.765333in}}{\pgfqpoint{3.720794in}{0.769723in}}{\pgfqpoint{3.709744in}{0.769723in}}%
\pgfpathcurveto{\pgfqpoint{3.698694in}{0.769723in}}{\pgfqpoint{3.688095in}{0.765333in}}{\pgfqpoint{3.680281in}{0.757519in}}%
\pgfpathcurveto{\pgfqpoint{3.672468in}{0.749705in}}{\pgfqpoint{3.668077in}{0.739106in}}{\pgfqpoint{3.668077in}{0.728056in}}%
\pgfpathcurveto{\pgfqpoint{3.668077in}{0.717006in}}{\pgfqpoint{3.672468in}{0.706407in}}{\pgfqpoint{3.680281in}{0.698593in}}%
\pgfpathcurveto{\pgfqpoint{3.688095in}{0.690780in}}{\pgfqpoint{3.698694in}{0.686390in}}{\pgfqpoint{3.709744in}{0.686390in}}%
\pgfpathlineto{\pgfqpoint{3.709744in}{0.686390in}}%
\pgfpathclose%
\pgfusepath{stroke}%
\end{pgfscope}%
\begin{pgfscope}%
\pgfpathrectangle{\pgfqpoint{0.494722in}{0.437222in}}{\pgfqpoint{6.275590in}{5.159444in}}%
\pgfusepath{clip}%
\pgfsetbuttcap%
\pgfsetroundjoin%
\pgfsetlinewidth{1.003750pt}%
\definecolor{currentstroke}{rgb}{0.827451,0.827451,0.827451}%
\pgfsetstrokecolor{currentstroke}%
\pgfsetstrokeopacity{0.800000}%
\pgfsetdash{}{0pt}%
\pgfpathmoveto{\pgfqpoint{0.859512in}{3.088326in}}%
\pgfpathcurveto{\pgfqpoint{0.870562in}{3.088326in}}{\pgfqpoint{0.881161in}{3.092716in}}{\pgfqpoint{0.888974in}{3.100530in}}%
\pgfpathcurveto{\pgfqpoint{0.896788in}{3.108344in}}{\pgfqpoint{0.901178in}{3.118943in}}{\pgfqpoint{0.901178in}{3.129993in}}%
\pgfpathcurveto{\pgfqpoint{0.901178in}{3.141043in}}{\pgfqpoint{0.896788in}{3.151642in}}{\pgfqpoint{0.888974in}{3.159455in}}%
\pgfpathcurveto{\pgfqpoint{0.881161in}{3.167269in}}{\pgfqpoint{0.870562in}{3.171659in}}{\pgfqpoint{0.859512in}{3.171659in}}%
\pgfpathcurveto{\pgfqpoint{0.848462in}{3.171659in}}{\pgfqpoint{0.837863in}{3.167269in}}{\pgfqpoint{0.830049in}{3.159455in}}%
\pgfpathcurveto{\pgfqpoint{0.822235in}{3.151642in}}{\pgfqpoint{0.817845in}{3.141043in}}{\pgfqpoint{0.817845in}{3.129993in}}%
\pgfpathcurveto{\pgfqpoint{0.817845in}{3.118943in}}{\pgfqpoint{0.822235in}{3.108344in}}{\pgfqpoint{0.830049in}{3.100530in}}%
\pgfpathcurveto{\pgfqpoint{0.837863in}{3.092716in}}{\pgfqpoint{0.848462in}{3.088326in}}{\pgfqpoint{0.859512in}{3.088326in}}%
\pgfpathlineto{\pgfqpoint{0.859512in}{3.088326in}}%
\pgfpathclose%
\pgfusepath{stroke}%
\end{pgfscope}%
\begin{pgfscope}%
\pgfpathrectangle{\pgfqpoint{0.494722in}{0.437222in}}{\pgfqpoint{6.275590in}{5.159444in}}%
\pgfusepath{clip}%
\pgfsetbuttcap%
\pgfsetroundjoin%
\pgfsetlinewidth{1.003750pt}%
\definecolor{currentstroke}{rgb}{0.827451,0.827451,0.827451}%
\pgfsetstrokecolor{currentstroke}%
\pgfsetstrokeopacity{0.800000}%
\pgfsetdash{}{0pt}%
\pgfpathmoveto{\pgfqpoint{1.545871in}{1.889457in}}%
\pgfpathcurveto{\pgfqpoint{1.556921in}{1.889457in}}{\pgfqpoint{1.567520in}{1.893848in}}{\pgfqpoint{1.575334in}{1.901661in}}%
\pgfpathcurveto{\pgfqpoint{1.583148in}{1.909475in}}{\pgfqpoint{1.587538in}{1.920074in}}{\pgfqpoint{1.587538in}{1.931124in}}%
\pgfpathcurveto{\pgfqpoint{1.587538in}{1.942174in}}{\pgfqpoint{1.583148in}{1.952773in}}{\pgfqpoint{1.575334in}{1.960587in}}%
\pgfpathcurveto{\pgfqpoint{1.567520in}{1.968400in}}{\pgfqpoint{1.556921in}{1.972791in}}{\pgfqpoint{1.545871in}{1.972791in}}%
\pgfpathcurveto{\pgfqpoint{1.534821in}{1.972791in}}{\pgfqpoint{1.524222in}{1.968400in}}{\pgfqpoint{1.516408in}{1.960587in}}%
\pgfpathcurveto{\pgfqpoint{1.508595in}{1.952773in}}{\pgfqpoint{1.504204in}{1.942174in}}{\pgfqpoint{1.504204in}{1.931124in}}%
\pgfpathcurveto{\pgfqpoint{1.504204in}{1.920074in}}{\pgfqpoint{1.508595in}{1.909475in}}{\pgfqpoint{1.516408in}{1.901661in}}%
\pgfpathcurveto{\pgfqpoint{1.524222in}{1.893848in}}{\pgfqpoint{1.534821in}{1.889457in}}{\pgfqpoint{1.545871in}{1.889457in}}%
\pgfpathlineto{\pgfqpoint{1.545871in}{1.889457in}}%
\pgfpathclose%
\pgfusepath{stroke}%
\end{pgfscope}%
\begin{pgfscope}%
\pgfpathrectangle{\pgfqpoint{0.494722in}{0.437222in}}{\pgfqpoint{6.275590in}{5.159444in}}%
\pgfusepath{clip}%
\pgfsetbuttcap%
\pgfsetroundjoin%
\pgfsetlinewidth{1.003750pt}%
\definecolor{currentstroke}{rgb}{0.827451,0.827451,0.827451}%
\pgfsetstrokecolor{currentstroke}%
\pgfsetstrokeopacity{0.800000}%
\pgfsetdash{}{0pt}%
\pgfpathmoveto{\pgfqpoint{0.520093in}{4.158669in}}%
\pgfpathcurveto{\pgfqpoint{0.531143in}{4.158669in}}{\pgfqpoint{0.541742in}{4.163059in}}{\pgfqpoint{0.549556in}{4.170873in}}%
\pgfpathcurveto{\pgfqpoint{0.557369in}{4.178687in}}{\pgfqpoint{0.561759in}{4.189286in}}{\pgfqpoint{0.561759in}{4.200336in}}%
\pgfpathcurveto{\pgfqpoint{0.561759in}{4.211386in}}{\pgfqpoint{0.557369in}{4.221985in}}{\pgfqpoint{0.549556in}{4.229799in}}%
\pgfpathcurveto{\pgfqpoint{0.541742in}{4.237612in}}{\pgfqpoint{0.531143in}{4.242003in}}{\pgfqpoint{0.520093in}{4.242003in}}%
\pgfpathcurveto{\pgfqpoint{0.509043in}{4.242003in}}{\pgfqpoint{0.498444in}{4.237612in}}{\pgfqpoint{0.490630in}{4.229799in}}%
\pgfpathcurveto{\pgfqpoint{0.482816in}{4.221985in}}{\pgfqpoint{0.478426in}{4.211386in}}{\pgfqpoint{0.478426in}{4.200336in}}%
\pgfpathcurveto{\pgfqpoint{0.478426in}{4.189286in}}{\pgfqpoint{0.482816in}{4.178687in}}{\pgfqpoint{0.490630in}{4.170873in}}%
\pgfpathcurveto{\pgfqpoint{0.498444in}{4.163059in}}{\pgfqpoint{0.509043in}{4.158669in}}{\pgfqpoint{0.520093in}{4.158669in}}%
\pgfpathlineto{\pgfqpoint{0.520093in}{4.158669in}}%
\pgfpathclose%
\pgfusepath{stroke}%
\end{pgfscope}%
\begin{pgfscope}%
\pgfpathrectangle{\pgfqpoint{0.494722in}{0.437222in}}{\pgfqpoint{6.275590in}{5.159444in}}%
\pgfusepath{clip}%
\pgfsetbuttcap%
\pgfsetroundjoin%
\pgfsetlinewidth{1.003750pt}%
\definecolor{currentstroke}{rgb}{0.827451,0.827451,0.827451}%
\pgfsetstrokecolor{currentstroke}%
\pgfsetstrokeopacity{0.800000}%
\pgfsetdash{}{0pt}%
\pgfpathmoveto{\pgfqpoint{1.396743in}{2.042450in}}%
\pgfpathcurveto{\pgfqpoint{1.407793in}{2.042450in}}{\pgfqpoint{1.418392in}{2.046840in}}{\pgfqpoint{1.426206in}{2.054654in}}%
\pgfpathcurveto{\pgfqpoint{1.434019in}{2.062467in}}{\pgfqpoint{1.438409in}{2.073066in}}{\pgfqpoint{1.438409in}{2.084117in}}%
\pgfpathcurveto{\pgfqpoint{1.438409in}{2.095167in}}{\pgfqpoint{1.434019in}{2.105766in}}{\pgfqpoint{1.426206in}{2.113579in}}%
\pgfpathcurveto{\pgfqpoint{1.418392in}{2.121393in}}{\pgfqpoint{1.407793in}{2.125783in}}{\pgfqpoint{1.396743in}{2.125783in}}%
\pgfpathcurveto{\pgfqpoint{1.385693in}{2.125783in}}{\pgfqpoint{1.375094in}{2.121393in}}{\pgfqpoint{1.367280in}{2.113579in}}%
\pgfpathcurveto{\pgfqpoint{1.359466in}{2.105766in}}{\pgfqpoint{1.355076in}{2.095167in}}{\pgfqpoint{1.355076in}{2.084117in}}%
\pgfpathcurveto{\pgfqpoint{1.355076in}{2.073066in}}{\pgfqpoint{1.359466in}{2.062467in}}{\pgfqpoint{1.367280in}{2.054654in}}%
\pgfpathcurveto{\pgfqpoint{1.375094in}{2.046840in}}{\pgfqpoint{1.385693in}{2.042450in}}{\pgfqpoint{1.396743in}{2.042450in}}%
\pgfpathlineto{\pgfqpoint{1.396743in}{2.042450in}}%
\pgfpathclose%
\pgfusepath{stroke}%
\end{pgfscope}%
\begin{pgfscope}%
\pgfpathrectangle{\pgfqpoint{0.494722in}{0.437222in}}{\pgfqpoint{6.275590in}{5.159444in}}%
\pgfusepath{clip}%
\pgfsetbuttcap%
\pgfsetroundjoin%
\pgfsetlinewidth{1.003750pt}%
\definecolor{currentstroke}{rgb}{0.827451,0.827451,0.827451}%
\pgfsetstrokecolor{currentstroke}%
\pgfsetstrokeopacity{0.800000}%
\pgfsetdash{}{0pt}%
\pgfpathmoveto{\pgfqpoint{0.637564in}{3.549862in}}%
\pgfpathcurveto{\pgfqpoint{0.648614in}{3.549862in}}{\pgfqpoint{0.659214in}{3.554252in}}{\pgfqpoint{0.667027in}{3.562065in}}%
\pgfpathcurveto{\pgfqpoint{0.674841in}{3.569879in}}{\pgfqpoint{0.679231in}{3.580478in}}{\pgfqpoint{0.679231in}{3.591528in}}%
\pgfpathcurveto{\pgfqpoint{0.679231in}{3.602578in}}{\pgfqpoint{0.674841in}{3.613177in}}{\pgfqpoint{0.667027in}{3.620991in}}%
\pgfpathcurveto{\pgfqpoint{0.659214in}{3.628805in}}{\pgfqpoint{0.648614in}{3.633195in}}{\pgfqpoint{0.637564in}{3.633195in}}%
\pgfpathcurveto{\pgfqpoint{0.626514in}{3.633195in}}{\pgfqpoint{0.615915in}{3.628805in}}{\pgfqpoint{0.608102in}{3.620991in}}%
\pgfpathcurveto{\pgfqpoint{0.600288in}{3.613177in}}{\pgfqpoint{0.595898in}{3.602578in}}{\pgfqpoint{0.595898in}{3.591528in}}%
\pgfpathcurveto{\pgfqpoint{0.595898in}{3.580478in}}{\pgfqpoint{0.600288in}{3.569879in}}{\pgfqpoint{0.608102in}{3.562065in}}%
\pgfpathcurveto{\pgfqpoint{0.615915in}{3.554252in}}{\pgfqpoint{0.626514in}{3.549862in}}{\pgfqpoint{0.637564in}{3.549862in}}%
\pgfpathlineto{\pgfqpoint{0.637564in}{3.549862in}}%
\pgfpathclose%
\pgfusepath{stroke}%
\end{pgfscope}%
\begin{pgfscope}%
\pgfpathrectangle{\pgfqpoint{0.494722in}{0.437222in}}{\pgfqpoint{6.275590in}{5.159444in}}%
\pgfusepath{clip}%
\pgfsetbuttcap%
\pgfsetroundjoin%
\pgfsetlinewidth{1.003750pt}%
\definecolor{currentstroke}{rgb}{0.827451,0.827451,0.827451}%
\pgfsetstrokecolor{currentstroke}%
\pgfsetstrokeopacity{0.800000}%
\pgfsetdash{}{0pt}%
\pgfpathmoveto{\pgfqpoint{1.346241in}{2.100145in}}%
\pgfpathcurveto{\pgfqpoint{1.357291in}{2.100145in}}{\pgfqpoint{1.367890in}{2.104535in}}{\pgfqpoint{1.375703in}{2.112349in}}%
\pgfpathcurveto{\pgfqpoint{1.383517in}{2.120163in}}{\pgfqpoint{1.387907in}{2.130762in}}{\pgfqpoint{1.387907in}{2.141812in}}%
\pgfpathcurveto{\pgfqpoint{1.387907in}{2.152862in}}{\pgfqpoint{1.383517in}{2.163461in}}{\pgfqpoint{1.375703in}{2.171274in}}%
\pgfpathcurveto{\pgfqpoint{1.367890in}{2.179088in}}{\pgfqpoint{1.357291in}{2.183478in}}{\pgfqpoint{1.346241in}{2.183478in}}%
\pgfpathcurveto{\pgfqpoint{1.335191in}{2.183478in}}{\pgfqpoint{1.324592in}{2.179088in}}{\pgfqpoint{1.316778in}{2.171274in}}%
\pgfpathcurveto{\pgfqpoint{1.308964in}{2.163461in}}{\pgfqpoint{1.304574in}{2.152862in}}{\pgfqpoint{1.304574in}{2.141812in}}%
\pgfpathcurveto{\pgfqpoint{1.304574in}{2.130762in}}{\pgfqpoint{1.308964in}{2.120163in}}{\pgfqpoint{1.316778in}{2.112349in}}%
\pgfpathcurveto{\pgfqpoint{1.324592in}{2.104535in}}{\pgfqpoint{1.335191in}{2.100145in}}{\pgfqpoint{1.346241in}{2.100145in}}%
\pgfpathlineto{\pgfqpoint{1.346241in}{2.100145in}}%
\pgfpathclose%
\pgfusepath{stroke}%
\end{pgfscope}%
\begin{pgfscope}%
\pgfpathrectangle{\pgfqpoint{0.494722in}{0.437222in}}{\pgfqpoint{6.275590in}{5.159444in}}%
\pgfusepath{clip}%
\pgfsetbuttcap%
\pgfsetroundjoin%
\pgfsetlinewidth{1.003750pt}%
\definecolor{currentstroke}{rgb}{0.827451,0.827451,0.827451}%
\pgfsetstrokecolor{currentstroke}%
\pgfsetstrokeopacity{0.800000}%
\pgfsetdash{}{0pt}%
\pgfpathmoveto{\pgfqpoint{0.679488in}{3.431236in}}%
\pgfpathcurveto{\pgfqpoint{0.690538in}{3.431236in}}{\pgfqpoint{0.701137in}{3.435626in}}{\pgfqpoint{0.708951in}{3.443440in}}%
\pgfpathcurveto{\pgfqpoint{0.716765in}{3.451253in}}{\pgfqpoint{0.721155in}{3.461852in}}{\pgfqpoint{0.721155in}{3.472902in}}%
\pgfpathcurveto{\pgfqpoint{0.721155in}{3.483953in}}{\pgfqpoint{0.716765in}{3.494552in}}{\pgfqpoint{0.708951in}{3.502365in}}%
\pgfpathcurveto{\pgfqpoint{0.701137in}{3.510179in}}{\pgfqpoint{0.690538in}{3.514569in}}{\pgfqpoint{0.679488in}{3.514569in}}%
\pgfpathcurveto{\pgfqpoint{0.668438in}{3.514569in}}{\pgfqpoint{0.657839in}{3.510179in}}{\pgfqpoint{0.650026in}{3.502365in}}%
\pgfpathcurveto{\pgfqpoint{0.642212in}{3.494552in}}{\pgfqpoint{0.637822in}{3.483953in}}{\pgfqpoint{0.637822in}{3.472902in}}%
\pgfpathcurveto{\pgfqpoint{0.637822in}{3.461852in}}{\pgfqpoint{0.642212in}{3.451253in}}{\pgfqpoint{0.650026in}{3.443440in}}%
\pgfpathcurveto{\pgfqpoint{0.657839in}{3.435626in}}{\pgfqpoint{0.668438in}{3.431236in}}{\pgfqpoint{0.679488in}{3.431236in}}%
\pgfpathlineto{\pgfqpoint{0.679488in}{3.431236in}}%
\pgfpathclose%
\pgfusepath{stroke}%
\end{pgfscope}%
\begin{pgfscope}%
\pgfpathrectangle{\pgfqpoint{0.494722in}{0.437222in}}{\pgfqpoint{6.275590in}{5.159444in}}%
\pgfusepath{clip}%
\pgfsetbuttcap%
\pgfsetroundjoin%
\pgfsetlinewidth{1.003750pt}%
\definecolor{currentstroke}{rgb}{0.827451,0.827451,0.827451}%
\pgfsetstrokecolor{currentstroke}%
\pgfsetstrokeopacity{0.800000}%
\pgfsetdash{}{0pt}%
\pgfpathmoveto{\pgfqpoint{3.950409in}{0.627171in}}%
\pgfpathcurveto{\pgfqpoint{3.961459in}{0.627171in}}{\pgfqpoint{3.972058in}{0.631561in}}{\pgfqpoint{3.979872in}{0.639374in}}%
\pgfpathcurveto{\pgfqpoint{3.987685in}{0.647188in}}{\pgfqpoint{3.992076in}{0.657787in}}{\pgfqpoint{3.992076in}{0.668837in}}%
\pgfpathcurveto{\pgfqpoint{3.992076in}{0.679887in}}{\pgfqpoint{3.987685in}{0.690486in}}{\pgfqpoint{3.979872in}{0.698300in}}%
\pgfpathcurveto{\pgfqpoint{3.972058in}{0.706114in}}{\pgfqpoint{3.961459in}{0.710504in}}{\pgfqpoint{3.950409in}{0.710504in}}%
\pgfpathcurveto{\pgfqpoint{3.939359in}{0.710504in}}{\pgfqpoint{3.928760in}{0.706114in}}{\pgfqpoint{3.920946in}{0.698300in}}%
\pgfpathcurveto{\pgfqpoint{3.913133in}{0.690486in}}{\pgfqpoint{3.908742in}{0.679887in}}{\pgfqpoint{3.908742in}{0.668837in}}%
\pgfpathcurveto{\pgfqpoint{3.908742in}{0.657787in}}{\pgfqpoint{3.913133in}{0.647188in}}{\pgfqpoint{3.920946in}{0.639374in}}%
\pgfpathcurveto{\pgfqpoint{3.928760in}{0.631561in}}{\pgfqpoint{3.939359in}{0.627171in}}{\pgfqpoint{3.950409in}{0.627171in}}%
\pgfpathlineto{\pgfqpoint{3.950409in}{0.627171in}}%
\pgfpathclose%
\pgfusepath{stroke}%
\end{pgfscope}%
\begin{pgfscope}%
\pgfpathrectangle{\pgfqpoint{0.494722in}{0.437222in}}{\pgfqpoint{6.275590in}{5.159444in}}%
\pgfusepath{clip}%
\pgfsetbuttcap%
\pgfsetroundjoin%
\pgfsetlinewidth{1.003750pt}%
\definecolor{currentstroke}{rgb}{0.827451,0.827451,0.827451}%
\pgfsetstrokecolor{currentstroke}%
\pgfsetstrokeopacity{0.800000}%
\pgfsetdash{}{0pt}%
\pgfpathmoveto{\pgfqpoint{4.942939in}{0.453295in}}%
\pgfpathcurveto{\pgfqpoint{4.953989in}{0.453295in}}{\pgfqpoint{4.964589in}{0.457685in}}{\pgfqpoint{4.972402in}{0.465499in}}%
\pgfpathcurveto{\pgfqpoint{4.980216in}{0.473313in}}{\pgfqpoint{4.984606in}{0.483912in}}{\pgfqpoint{4.984606in}{0.494962in}}%
\pgfpathcurveto{\pgfqpoint{4.984606in}{0.506012in}}{\pgfqpoint{4.980216in}{0.516611in}}{\pgfqpoint{4.972402in}{0.524425in}}%
\pgfpathcurveto{\pgfqpoint{4.964589in}{0.532238in}}{\pgfqpoint{4.953989in}{0.536628in}}{\pgfqpoint{4.942939in}{0.536628in}}%
\pgfpathcurveto{\pgfqpoint{4.931889in}{0.536628in}}{\pgfqpoint{4.921290in}{0.532238in}}{\pgfqpoint{4.913477in}{0.524425in}}%
\pgfpathcurveto{\pgfqpoint{4.905663in}{0.516611in}}{\pgfqpoint{4.901273in}{0.506012in}}{\pgfqpoint{4.901273in}{0.494962in}}%
\pgfpathcurveto{\pgfqpoint{4.901273in}{0.483912in}}{\pgfqpoint{4.905663in}{0.473313in}}{\pgfqpoint{4.913477in}{0.465499in}}%
\pgfpathcurveto{\pgfqpoint{4.921290in}{0.457685in}}{\pgfqpoint{4.931889in}{0.453295in}}{\pgfqpoint{4.942939in}{0.453295in}}%
\pgfpathlineto{\pgfqpoint{4.942939in}{0.453295in}}%
\pgfpathclose%
\pgfusepath{stroke}%
\end{pgfscope}%
\begin{pgfscope}%
\pgfpathrectangle{\pgfqpoint{0.494722in}{0.437222in}}{\pgfqpoint{6.275590in}{5.159444in}}%
\pgfusepath{clip}%
\pgfsetbuttcap%
\pgfsetroundjoin%
\pgfsetlinewidth{1.003750pt}%
\definecolor{currentstroke}{rgb}{0.827451,0.827451,0.827451}%
\pgfsetstrokecolor{currentstroke}%
\pgfsetstrokeopacity{0.800000}%
\pgfsetdash{}{0pt}%
\pgfpathmoveto{\pgfqpoint{1.926347in}{1.574489in}}%
\pgfpathcurveto{\pgfqpoint{1.937397in}{1.574489in}}{\pgfqpoint{1.947996in}{1.578879in}}{\pgfqpoint{1.955809in}{1.586693in}}%
\pgfpathcurveto{\pgfqpoint{1.963623in}{1.594506in}}{\pgfqpoint{1.968013in}{1.605105in}}{\pgfqpoint{1.968013in}{1.616155in}}%
\pgfpathcurveto{\pgfqpoint{1.968013in}{1.627206in}}{\pgfqpoint{1.963623in}{1.637805in}}{\pgfqpoint{1.955809in}{1.645618in}}%
\pgfpathcurveto{\pgfqpoint{1.947996in}{1.653432in}}{\pgfqpoint{1.937397in}{1.657822in}}{\pgfqpoint{1.926347in}{1.657822in}}%
\pgfpathcurveto{\pgfqpoint{1.915296in}{1.657822in}}{\pgfqpoint{1.904697in}{1.653432in}}{\pgfqpoint{1.896884in}{1.645618in}}%
\pgfpathcurveto{\pgfqpoint{1.889070in}{1.637805in}}{\pgfqpoint{1.884680in}{1.627206in}}{\pgfqpoint{1.884680in}{1.616155in}}%
\pgfpathcurveto{\pgfqpoint{1.884680in}{1.605105in}}{\pgfqpoint{1.889070in}{1.594506in}}{\pgfqpoint{1.896884in}{1.586693in}}%
\pgfpathcurveto{\pgfqpoint{1.904697in}{1.578879in}}{\pgfqpoint{1.915296in}{1.574489in}}{\pgfqpoint{1.926347in}{1.574489in}}%
\pgfpathlineto{\pgfqpoint{1.926347in}{1.574489in}}%
\pgfpathclose%
\pgfusepath{stroke}%
\end{pgfscope}%
\begin{pgfscope}%
\pgfpathrectangle{\pgfqpoint{0.494722in}{0.437222in}}{\pgfqpoint{6.275590in}{5.159444in}}%
\pgfusepath{clip}%
\pgfsetbuttcap%
\pgfsetroundjoin%
\pgfsetlinewidth{1.003750pt}%
\definecolor{currentstroke}{rgb}{0.827451,0.827451,0.827451}%
\pgfsetstrokecolor{currentstroke}%
\pgfsetstrokeopacity{0.800000}%
\pgfsetdash{}{0pt}%
\pgfpathmoveto{\pgfqpoint{1.766183in}{1.687308in}}%
\pgfpathcurveto{\pgfqpoint{1.777233in}{1.687308in}}{\pgfqpoint{1.787832in}{1.691698in}}{\pgfqpoint{1.795646in}{1.699512in}}%
\pgfpathcurveto{\pgfqpoint{1.803459in}{1.707325in}}{\pgfqpoint{1.807850in}{1.717924in}}{\pgfqpoint{1.807850in}{1.728974in}}%
\pgfpathcurveto{\pgfqpoint{1.807850in}{1.740025in}}{\pgfqpoint{1.803459in}{1.750624in}}{\pgfqpoint{1.795646in}{1.758437in}}%
\pgfpathcurveto{\pgfqpoint{1.787832in}{1.766251in}}{\pgfqpoint{1.777233in}{1.770641in}}{\pgfqpoint{1.766183in}{1.770641in}}%
\pgfpathcurveto{\pgfqpoint{1.755133in}{1.770641in}}{\pgfqpoint{1.744534in}{1.766251in}}{\pgfqpoint{1.736720in}{1.758437in}}%
\pgfpathcurveto{\pgfqpoint{1.728907in}{1.750624in}}{\pgfqpoint{1.724516in}{1.740025in}}{\pgfqpoint{1.724516in}{1.728974in}}%
\pgfpathcurveto{\pgfqpoint{1.724516in}{1.717924in}}{\pgfqpoint{1.728907in}{1.707325in}}{\pgfqpoint{1.736720in}{1.699512in}}%
\pgfpathcurveto{\pgfqpoint{1.744534in}{1.691698in}}{\pgfqpoint{1.755133in}{1.687308in}}{\pgfqpoint{1.766183in}{1.687308in}}%
\pgfpathlineto{\pgfqpoint{1.766183in}{1.687308in}}%
\pgfpathclose%
\pgfusepath{stroke}%
\end{pgfscope}%
\begin{pgfscope}%
\pgfpathrectangle{\pgfqpoint{0.494722in}{0.437222in}}{\pgfqpoint{6.275590in}{5.159444in}}%
\pgfusepath{clip}%
\pgfsetbuttcap%
\pgfsetroundjoin%
\pgfsetlinewidth{1.003750pt}%
\definecolor{currentstroke}{rgb}{0.827451,0.827451,0.827451}%
\pgfsetstrokecolor{currentstroke}%
\pgfsetstrokeopacity{0.800000}%
\pgfsetdash{}{0pt}%
\pgfpathmoveto{\pgfqpoint{0.750952in}{3.177253in}}%
\pgfpathcurveto{\pgfqpoint{0.762002in}{3.177253in}}{\pgfqpoint{0.772601in}{3.181643in}}{\pgfqpoint{0.780414in}{3.189457in}}%
\pgfpathcurveto{\pgfqpoint{0.788228in}{3.197270in}}{\pgfqpoint{0.792618in}{3.207870in}}{\pgfqpoint{0.792618in}{3.218920in}}%
\pgfpathcurveto{\pgfqpoint{0.792618in}{3.229970in}}{\pgfqpoint{0.788228in}{3.240569in}}{\pgfqpoint{0.780414in}{3.248382in}}%
\pgfpathcurveto{\pgfqpoint{0.772601in}{3.256196in}}{\pgfqpoint{0.762002in}{3.260586in}}{\pgfqpoint{0.750952in}{3.260586in}}%
\pgfpathcurveto{\pgfqpoint{0.739901in}{3.260586in}}{\pgfqpoint{0.729302in}{3.256196in}}{\pgfqpoint{0.721489in}{3.248382in}}%
\pgfpathcurveto{\pgfqpoint{0.713675in}{3.240569in}}{\pgfqpoint{0.709285in}{3.229970in}}{\pgfqpoint{0.709285in}{3.218920in}}%
\pgfpathcurveto{\pgfqpoint{0.709285in}{3.207870in}}{\pgfqpoint{0.713675in}{3.197270in}}{\pgfqpoint{0.721489in}{3.189457in}}%
\pgfpathcurveto{\pgfqpoint{0.729302in}{3.181643in}}{\pgfqpoint{0.739901in}{3.177253in}}{\pgfqpoint{0.750952in}{3.177253in}}%
\pgfpathlineto{\pgfqpoint{0.750952in}{3.177253in}}%
\pgfpathclose%
\pgfusepath{stroke}%
\end{pgfscope}%
\begin{pgfscope}%
\pgfpathrectangle{\pgfqpoint{0.494722in}{0.437222in}}{\pgfqpoint{6.275590in}{5.159444in}}%
\pgfusepath{clip}%
\pgfsetbuttcap%
\pgfsetroundjoin%
\pgfsetlinewidth{1.003750pt}%
\definecolor{currentstroke}{rgb}{0.827451,0.827451,0.827451}%
\pgfsetstrokecolor{currentstroke}%
\pgfsetstrokeopacity{0.800000}%
\pgfsetdash{}{0pt}%
\pgfpathmoveto{\pgfqpoint{4.282315in}{0.548168in}}%
\pgfpathcurveto{\pgfqpoint{4.293365in}{0.548168in}}{\pgfqpoint{4.303964in}{0.552559in}}{\pgfqpoint{4.311778in}{0.560372in}}%
\pgfpathcurveto{\pgfqpoint{4.319591in}{0.568186in}}{\pgfqpoint{4.323982in}{0.578785in}}{\pgfqpoint{4.323982in}{0.589835in}}%
\pgfpathcurveto{\pgfqpoint{4.323982in}{0.600885in}}{\pgfqpoint{4.319591in}{0.611484in}}{\pgfqpoint{4.311778in}{0.619298in}}%
\pgfpathcurveto{\pgfqpoint{4.303964in}{0.627111in}}{\pgfqpoint{4.293365in}{0.631502in}}{\pgfqpoint{4.282315in}{0.631502in}}%
\pgfpathcurveto{\pgfqpoint{4.271265in}{0.631502in}}{\pgfqpoint{4.260666in}{0.627111in}}{\pgfqpoint{4.252852in}{0.619298in}}%
\pgfpathcurveto{\pgfqpoint{4.245039in}{0.611484in}}{\pgfqpoint{4.240648in}{0.600885in}}{\pgfqpoint{4.240648in}{0.589835in}}%
\pgfpathcurveto{\pgfqpoint{4.240648in}{0.578785in}}{\pgfqpoint{4.245039in}{0.568186in}}{\pgfqpoint{4.252852in}{0.560372in}}%
\pgfpathcurveto{\pgfqpoint{4.260666in}{0.552559in}}{\pgfqpoint{4.271265in}{0.548168in}}{\pgfqpoint{4.282315in}{0.548168in}}%
\pgfpathlineto{\pgfqpoint{4.282315in}{0.548168in}}%
\pgfpathclose%
\pgfusepath{stroke}%
\end{pgfscope}%
\begin{pgfscope}%
\pgfpathrectangle{\pgfqpoint{0.494722in}{0.437222in}}{\pgfqpoint{6.275590in}{5.159444in}}%
\pgfusepath{clip}%
\pgfsetbuttcap%
\pgfsetroundjoin%
\pgfsetlinewidth{1.003750pt}%
\definecolor{currentstroke}{rgb}{0.827451,0.827451,0.827451}%
\pgfsetstrokecolor{currentstroke}%
\pgfsetstrokeopacity{0.800000}%
\pgfsetdash{}{0pt}%
\pgfpathmoveto{\pgfqpoint{2.283147in}{1.470627in}}%
\pgfpathcurveto{\pgfqpoint{2.294197in}{1.470627in}}{\pgfqpoint{2.304796in}{1.475017in}}{\pgfqpoint{2.312610in}{1.482831in}}%
\pgfpathcurveto{\pgfqpoint{2.320423in}{1.490645in}}{\pgfqpoint{2.324813in}{1.501244in}}{\pgfqpoint{2.324813in}{1.512294in}}%
\pgfpathcurveto{\pgfqpoint{2.324813in}{1.523344in}}{\pgfqpoint{2.320423in}{1.533943in}}{\pgfqpoint{2.312610in}{1.541757in}}%
\pgfpathcurveto{\pgfqpoint{2.304796in}{1.549570in}}{\pgfqpoint{2.294197in}{1.553960in}}{\pgfqpoint{2.283147in}{1.553960in}}%
\pgfpathcurveto{\pgfqpoint{2.272097in}{1.553960in}}{\pgfqpoint{2.261498in}{1.549570in}}{\pgfqpoint{2.253684in}{1.541757in}}%
\pgfpathcurveto{\pgfqpoint{2.245870in}{1.533943in}}{\pgfqpoint{2.241480in}{1.523344in}}{\pgfqpoint{2.241480in}{1.512294in}}%
\pgfpathcurveto{\pgfqpoint{2.241480in}{1.501244in}}{\pgfqpoint{2.245870in}{1.490645in}}{\pgfqpoint{2.253684in}{1.482831in}}%
\pgfpathcurveto{\pgfqpoint{2.261498in}{1.475017in}}{\pgfqpoint{2.272097in}{1.470627in}}{\pgfqpoint{2.283147in}{1.470627in}}%
\pgfpathlineto{\pgfqpoint{2.283147in}{1.470627in}}%
\pgfpathclose%
\pgfusepath{stroke}%
\end{pgfscope}%
\begin{pgfscope}%
\pgfpathrectangle{\pgfqpoint{0.494722in}{0.437222in}}{\pgfqpoint{6.275590in}{5.159444in}}%
\pgfusepath{clip}%
\pgfsetbuttcap%
\pgfsetroundjoin%
\pgfsetlinewidth{1.003750pt}%
\definecolor{currentstroke}{rgb}{0.827451,0.827451,0.827451}%
\pgfsetstrokecolor{currentstroke}%
\pgfsetstrokeopacity{0.800000}%
\pgfsetdash{}{0pt}%
\pgfpathmoveto{\pgfqpoint{1.342899in}{2.272259in}}%
\pgfpathcurveto{\pgfqpoint{1.353949in}{2.272259in}}{\pgfqpoint{1.364548in}{2.276649in}}{\pgfqpoint{1.372362in}{2.284463in}}%
\pgfpathcurveto{\pgfqpoint{1.380175in}{2.292277in}}{\pgfqpoint{1.384565in}{2.302876in}}{\pgfqpoint{1.384565in}{2.313926in}}%
\pgfpathcurveto{\pgfqpoint{1.384565in}{2.324976in}}{\pgfqpoint{1.380175in}{2.335575in}}{\pgfqpoint{1.372362in}{2.343389in}}%
\pgfpathcurveto{\pgfqpoint{1.364548in}{2.351202in}}{\pgfqpoint{1.353949in}{2.355592in}}{\pgfqpoint{1.342899in}{2.355592in}}%
\pgfpathcurveto{\pgfqpoint{1.331849in}{2.355592in}}{\pgfqpoint{1.321250in}{2.351202in}}{\pgfqpoint{1.313436in}{2.343389in}}%
\pgfpathcurveto{\pgfqpoint{1.305622in}{2.335575in}}{\pgfqpoint{1.301232in}{2.324976in}}{\pgfqpoint{1.301232in}{2.313926in}}%
\pgfpathcurveto{\pgfqpoint{1.301232in}{2.302876in}}{\pgfqpoint{1.305622in}{2.292277in}}{\pgfqpoint{1.313436in}{2.284463in}}%
\pgfpathcurveto{\pgfqpoint{1.321250in}{2.276649in}}{\pgfqpoint{1.331849in}{2.272259in}}{\pgfqpoint{1.342899in}{2.272259in}}%
\pgfpathlineto{\pgfqpoint{1.342899in}{2.272259in}}%
\pgfpathclose%
\pgfusepath{stroke}%
\end{pgfscope}%
\begin{pgfscope}%
\pgfpathrectangle{\pgfqpoint{0.494722in}{0.437222in}}{\pgfqpoint{6.275590in}{5.159444in}}%
\pgfusepath{clip}%
\pgfsetbuttcap%
\pgfsetroundjoin%
\pgfsetlinewidth{1.003750pt}%
\definecolor{currentstroke}{rgb}{0.827451,0.827451,0.827451}%
\pgfsetstrokecolor{currentstroke}%
\pgfsetstrokeopacity{0.800000}%
\pgfsetdash{}{0pt}%
\pgfpathmoveto{\pgfqpoint{3.510856in}{0.913827in}}%
\pgfpathcurveto{\pgfqpoint{3.521906in}{0.913827in}}{\pgfqpoint{3.532505in}{0.918217in}}{\pgfqpoint{3.540319in}{0.926030in}}%
\pgfpathcurveto{\pgfqpoint{3.548133in}{0.933844in}}{\pgfqpoint{3.552523in}{0.944443in}}{\pgfqpoint{3.552523in}{0.955493in}}%
\pgfpathcurveto{\pgfqpoint{3.552523in}{0.966543in}}{\pgfqpoint{3.548133in}{0.977142in}}{\pgfqpoint{3.540319in}{0.984956in}}%
\pgfpathcurveto{\pgfqpoint{3.532505in}{0.992770in}}{\pgfqpoint{3.521906in}{0.997160in}}{\pgfqpoint{3.510856in}{0.997160in}}%
\pgfpathcurveto{\pgfqpoint{3.499806in}{0.997160in}}{\pgfqpoint{3.489207in}{0.992770in}}{\pgfqpoint{3.481394in}{0.984956in}}%
\pgfpathcurveto{\pgfqpoint{3.473580in}{0.977142in}}{\pgfqpoint{3.469190in}{0.966543in}}{\pgfqpoint{3.469190in}{0.955493in}}%
\pgfpathcurveto{\pgfqpoint{3.469190in}{0.944443in}}{\pgfqpoint{3.473580in}{0.933844in}}{\pgfqpoint{3.481394in}{0.926030in}}%
\pgfpathcurveto{\pgfqpoint{3.489207in}{0.918217in}}{\pgfqpoint{3.499806in}{0.913827in}}{\pgfqpoint{3.510856in}{0.913827in}}%
\pgfpathlineto{\pgfqpoint{3.510856in}{0.913827in}}%
\pgfpathclose%
\pgfusepath{stroke}%
\end{pgfscope}%
\begin{pgfscope}%
\pgfpathrectangle{\pgfqpoint{0.494722in}{0.437222in}}{\pgfqpoint{6.275590in}{5.159444in}}%
\pgfusepath{clip}%
\pgfsetbuttcap%
\pgfsetroundjoin%
\pgfsetlinewidth{1.003750pt}%
\definecolor{currentstroke}{rgb}{0.827451,0.827451,0.827451}%
\pgfsetstrokecolor{currentstroke}%
\pgfsetstrokeopacity{0.800000}%
\pgfsetdash{}{0pt}%
\pgfpathmoveto{\pgfqpoint{3.592349in}{0.908416in}}%
\pgfpathcurveto{\pgfqpoint{3.603399in}{0.908416in}}{\pgfqpoint{3.613998in}{0.912806in}}{\pgfqpoint{3.621812in}{0.920620in}}%
\pgfpathcurveto{\pgfqpoint{3.629625in}{0.928434in}}{\pgfqpoint{3.634015in}{0.939033in}}{\pgfqpoint{3.634015in}{0.950083in}}%
\pgfpathcurveto{\pgfqpoint{3.634015in}{0.961133in}}{\pgfqpoint{3.629625in}{0.971732in}}{\pgfqpoint{3.621812in}{0.979546in}}%
\pgfpathcurveto{\pgfqpoint{3.613998in}{0.987359in}}{\pgfqpoint{3.603399in}{0.991750in}}{\pgfqpoint{3.592349in}{0.991750in}}%
\pgfpathcurveto{\pgfqpoint{3.581299in}{0.991750in}}{\pgfqpoint{3.570700in}{0.987359in}}{\pgfqpoint{3.562886in}{0.979546in}}%
\pgfpathcurveto{\pgfqpoint{3.555072in}{0.971732in}}{\pgfqpoint{3.550682in}{0.961133in}}{\pgfqpoint{3.550682in}{0.950083in}}%
\pgfpathcurveto{\pgfqpoint{3.550682in}{0.939033in}}{\pgfqpoint{3.555072in}{0.928434in}}{\pgfqpoint{3.562886in}{0.920620in}}%
\pgfpathcurveto{\pgfqpoint{3.570700in}{0.912806in}}{\pgfqpoint{3.581299in}{0.908416in}}{\pgfqpoint{3.592349in}{0.908416in}}%
\pgfpathlineto{\pgfqpoint{3.592349in}{0.908416in}}%
\pgfpathclose%
\pgfusepath{stroke}%
\end{pgfscope}%
\begin{pgfscope}%
\pgfpathrectangle{\pgfqpoint{0.494722in}{0.437222in}}{\pgfqpoint{6.275590in}{5.159444in}}%
\pgfusepath{clip}%
\pgfsetbuttcap%
\pgfsetroundjoin%
\pgfsetlinewidth{1.003750pt}%
\definecolor{currentstroke}{rgb}{0.827451,0.827451,0.827451}%
\pgfsetstrokecolor{currentstroke}%
\pgfsetstrokeopacity{0.800000}%
\pgfsetdash{}{0pt}%
\pgfpathmoveto{\pgfqpoint{1.064415in}{2.774487in}}%
\pgfpathcurveto{\pgfqpoint{1.075465in}{2.774487in}}{\pgfqpoint{1.086064in}{2.778877in}}{\pgfqpoint{1.093878in}{2.786691in}}%
\pgfpathcurveto{\pgfqpoint{1.101691in}{2.794504in}}{\pgfqpoint{1.106082in}{2.805103in}}{\pgfqpoint{1.106082in}{2.816154in}}%
\pgfpathcurveto{\pgfqpoint{1.106082in}{2.827204in}}{\pgfqpoint{1.101691in}{2.837803in}}{\pgfqpoint{1.093878in}{2.845616in}}%
\pgfpathcurveto{\pgfqpoint{1.086064in}{2.853430in}}{\pgfqpoint{1.075465in}{2.857820in}}{\pgfqpoint{1.064415in}{2.857820in}}%
\pgfpathcurveto{\pgfqpoint{1.053365in}{2.857820in}}{\pgfqpoint{1.042766in}{2.853430in}}{\pgfqpoint{1.034952in}{2.845616in}}%
\pgfpathcurveto{\pgfqpoint{1.027139in}{2.837803in}}{\pgfqpoint{1.022748in}{2.827204in}}{\pgfqpoint{1.022748in}{2.816154in}}%
\pgfpathcurveto{\pgfqpoint{1.022748in}{2.805103in}}{\pgfqpoint{1.027139in}{2.794504in}}{\pgfqpoint{1.034952in}{2.786691in}}%
\pgfpathcurveto{\pgfqpoint{1.042766in}{2.778877in}}{\pgfqpoint{1.053365in}{2.774487in}}{\pgfqpoint{1.064415in}{2.774487in}}%
\pgfpathlineto{\pgfqpoint{1.064415in}{2.774487in}}%
\pgfpathclose%
\pgfusepath{stroke}%
\end{pgfscope}%
\begin{pgfscope}%
\pgfpathrectangle{\pgfqpoint{0.494722in}{0.437222in}}{\pgfqpoint{6.275590in}{5.159444in}}%
\pgfusepath{clip}%
\pgfsetbuttcap%
\pgfsetroundjoin%
\pgfsetlinewidth{1.003750pt}%
\definecolor{currentstroke}{rgb}{0.827451,0.827451,0.827451}%
\pgfsetstrokecolor{currentstroke}%
\pgfsetstrokeopacity{0.800000}%
\pgfsetdash{}{0pt}%
\pgfpathmoveto{\pgfqpoint{3.044108in}{1.012416in}}%
\pgfpathcurveto{\pgfqpoint{3.055158in}{1.012416in}}{\pgfqpoint{3.065757in}{1.016806in}}{\pgfqpoint{3.073571in}{1.024620in}}%
\pgfpathcurveto{\pgfqpoint{3.081385in}{1.032433in}}{\pgfqpoint{3.085775in}{1.043032in}}{\pgfqpoint{3.085775in}{1.054083in}}%
\pgfpathcurveto{\pgfqpoint{3.085775in}{1.065133in}}{\pgfqpoint{3.081385in}{1.075732in}}{\pgfqpoint{3.073571in}{1.083545in}}%
\pgfpathcurveto{\pgfqpoint{3.065757in}{1.091359in}}{\pgfqpoint{3.055158in}{1.095749in}}{\pgfqpoint{3.044108in}{1.095749in}}%
\pgfpathcurveto{\pgfqpoint{3.033058in}{1.095749in}}{\pgfqpoint{3.022459in}{1.091359in}}{\pgfqpoint{3.014645in}{1.083545in}}%
\pgfpathcurveto{\pgfqpoint{3.006832in}{1.075732in}}{\pgfqpoint{3.002441in}{1.065133in}}{\pgfqpoint{3.002441in}{1.054083in}}%
\pgfpathcurveto{\pgfqpoint{3.002441in}{1.043032in}}{\pgfqpoint{3.006832in}{1.032433in}}{\pgfqpoint{3.014645in}{1.024620in}}%
\pgfpathcurveto{\pgfqpoint{3.022459in}{1.016806in}}{\pgfqpoint{3.033058in}{1.012416in}}{\pgfqpoint{3.044108in}{1.012416in}}%
\pgfpathlineto{\pgfqpoint{3.044108in}{1.012416in}}%
\pgfpathclose%
\pgfusepath{stroke}%
\end{pgfscope}%
\begin{pgfscope}%
\pgfpathrectangle{\pgfqpoint{0.494722in}{0.437222in}}{\pgfqpoint{6.275590in}{5.159444in}}%
\pgfusepath{clip}%
\pgfsetbuttcap%
\pgfsetroundjoin%
\pgfsetlinewidth{1.003750pt}%
\definecolor{currentstroke}{rgb}{0.827451,0.827451,0.827451}%
\pgfsetstrokecolor{currentstroke}%
\pgfsetstrokeopacity{0.800000}%
\pgfsetdash{}{0pt}%
\pgfpathmoveto{\pgfqpoint{3.141060in}{0.995987in}}%
\pgfpathcurveto{\pgfqpoint{3.152111in}{0.995987in}}{\pgfqpoint{3.162710in}{1.000377in}}{\pgfqpoint{3.170523in}{1.008191in}}%
\pgfpathcurveto{\pgfqpoint{3.178337in}{1.016005in}}{\pgfqpoint{3.182727in}{1.026604in}}{\pgfqpoint{3.182727in}{1.037654in}}%
\pgfpathcurveto{\pgfqpoint{3.182727in}{1.048704in}}{\pgfqpoint{3.178337in}{1.059303in}}{\pgfqpoint{3.170523in}{1.067117in}}%
\pgfpathcurveto{\pgfqpoint{3.162710in}{1.074930in}}{\pgfqpoint{3.152111in}{1.079320in}}{\pgfqpoint{3.141060in}{1.079320in}}%
\pgfpathcurveto{\pgfqpoint{3.130010in}{1.079320in}}{\pgfqpoint{3.119411in}{1.074930in}}{\pgfqpoint{3.111598in}{1.067117in}}%
\pgfpathcurveto{\pgfqpoint{3.103784in}{1.059303in}}{\pgfqpoint{3.099394in}{1.048704in}}{\pgfqpoint{3.099394in}{1.037654in}}%
\pgfpathcurveto{\pgfqpoint{3.099394in}{1.026604in}}{\pgfqpoint{3.103784in}{1.016005in}}{\pgfqpoint{3.111598in}{1.008191in}}%
\pgfpathcurveto{\pgfqpoint{3.119411in}{1.000377in}}{\pgfqpoint{3.130010in}{0.995987in}}{\pgfqpoint{3.141060in}{0.995987in}}%
\pgfpathlineto{\pgfqpoint{3.141060in}{0.995987in}}%
\pgfpathclose%
\pgfusepath{stroke}%
\end{pgfscope}%
\begin{pgfscope}%
\pgfpathrectangle{\pgfqpoint{0.494722in}{0.437222in}}{\pgfqpoint{6.275590in}{5.159444in}}%
\pgfusepath{clip}%
\pgfsetbuttcap%
\pgfsetroundjoin%
\pgfsetlinewidth{1.003750pt}%
\definecolor{currentstroke}{rgb}{0.827451,0.827451,0.827451}%
\pgfsetstrokecolor{currentstroke}%
\pgfsetstrokeopacity{0.800000}%
\pgfsetdash{}{0pt}%
\pgfpathmoveto{\pgfqpoint{2.652266in}{1.295677in}}%
\pgfpathcurveto{\pgfqpoint{2.663316in}{1.295677in}}{\pgfqpoint{2.673915in}{1.300067in}}{\pgfqpoint{2.681729in}{1.307881in}}%
\pgfpathcurveto{\pgfqpoint{2.689543in}{1.315694in}}{\pgfqpoint{2.693933in}{1.326293in}}{\pgfqpoint{2.693933in}{1.337343in}}%
\pgfpathcurveto{\pgfqpoint{2.693933in}{1.348393in}}{\pgfqpoint{2.689543in}{1.358992in}}{\pgfqpoint{2.681729in}{1.366806in}}%
\pgfpathcurveto{\pgfqpoint{2.673915in}{1.374620in}}{\pgfqpoint{2.663316in}{1.379010in}}{\pgfqpoint{2.652266in}{1.379010in}}%
\pgfpathcurveto{\pgfqpoint{2.641216in}{1.379010in}}{\pgfqpoint{2.630617in}{1.374620in}}{\pgfqpoint{2.622803in}{1.366806in}}%
\pgfpathcurveto{\pgfqpoint{2.614990in}{1.358992in}}{\pgfqpoint{2.610600in}{1.348393in}}{\pgfqpoint{2.610600in}{1.337343in}}%
\pgfpathcurveto{\pgfqpoint{2.610600in}{1.326293in}}{\pgfqpoint{2.614990in}{1.315694in}}{\pgfqpoint{2.622803in}{1.307881in}}%
\pgfpathcurveto{\pgfqpoint{2.630617in}{1.300067in}}{\pgfqpoint{2.641216in}{1.295677in}}{\pgfqpoint{2.652266in}{1.295677in}}%
\pgfpathlineto{\pgfqpoint{2.652266in}{1.295677in}}%
\pgfpathclose%
\pgfusepath{stroke}%
\end{pgfscope}%
\begin{pgfscope}%
\pgfpathrectangle{\pgfqpoint{0.494722in}{0.437222in}}{\pgfqpoint{6.275590in}{5.159444in}}%
\pgfusepath{clip}%
\pgfsetbuttcap%
\pgfsetroundjoin%
\pgfsetlinewidth{1.003750pt}%
\definecolor{currentstroke}{rgb}{0.827451,0.827451,0.827451}%
\pgfsetstrokecolor{currentstroke}%
\pgfsetstrokeopacity{0.800000}%
\pgfsetdash{}{0pt}%
\pgfpathmoveto{\pgfqpoint{3.722236in}{0.790666in}}%
\pgfpathcurveto{\pgfqpoint{3.733286in}{0.790666in}}{\pgfqpoint{3.743885in}{0.795056in}}{\pgfqpoint{3.751699in}{0.802870in}}%
\pgfpathcurveto{\pgfqpoint{3.759513in}{0.810684in}}{\pgfqpoint{3.763903in}{0.821283in}}{\pgfqpoint{3.763903in}{0.832333in}}%
\pgfpathcurveto{\pgfqpoint{3.763903in}{0.843383in}}{\pgfqpoint{3.759513in}{0.853982in}}{\pgfqpoint{3.751699in}{0.861796in}}%
\pgfpathcurveto{\pgfqpoint{3.743885in}{0.869609in}}{\pgfqpoint{3.733286in}{0.873999in}}{\pgfqpoint{3.722236in}{0.873999in}}%
\pgfpathcurveto{\pgfqpoint{3.711186in}{0.873999in}}{\pgfqpoint{3.700587in}{0.869609in}}{\pgfqpoint{3.692773in}{0.861796in}}%
\pgfpathcurveto{\pgfqpoint{3.684960in}{0.853982in}}{\pgfqpoint{3.680570in}{0.843383in}}{\pgfqpoint{3.680570in}{0.832333in}}%
\pgfpathcurveto{\pgfqpoint{3.680570in}{0.821283in}}{\pgfqpoint{3.684960in}{0.810684in}}{\pgfqpoint{3.692773in}{0.802870in}}%
\pgfpathcurveto{\pgfqpoint{3.700587in}{0.795056in}}{\pgfqpoint{3.711186in}{0.790666in}}{\pgfqpoint{3.722236in}{0.790666in}}%
\pgfpathlineto{\pgfqpoint{3.722236in}{0.790666in}}%
\pgfpathclose%
\pgfusepath{stroke}%
\end{pgfscope}%
\begin{pgfscope}%
\pgfpathrectangle{\pgfqpoint{0.494722in}{0.437222in}}{\pgfqpoint{6.275590in}{5.159444in}}%
\pgfusepath{clip}%
\pgfsetbuttcap%
\pgfsetroundjoin%
\pgfsetlinewidth{1.003750pt}%
\definecolor{currentstroke}{rgb}{0.827451,0.827451,0.827451}%
\pgfsetstrokecolor{currentstroke}%
\pgfsetstrokeopacity{0.800000}%
\pgfsetdash{}{0pt}%
\pgfpathmoveto{\pgfqpoint{1.351308in}{2.103615in}}%
\pgfpathcurveto{\pgfqpoint{1.362358in}{2.103615in}}{\pgfqpoint{1.372957in}{2.108005in}}{\pgfqpoint{1.380771in}{2.115819in}}%
\pgfpathcurveto{\pgfqpoint{1.388585in}{2.123633in}}{\pgfqpoint{1.392975in}{2.134232in}}{\pgfqpoint{1.392975in}{2.145282in}}%
\pgfpathcurveto{\pgfqpoint{1.392975in}{2.156332in}}{\pgfqpoint{1.388585in}{2.166931in}}{\pgfqpoint{1.380771in}{2.174745in}}%
\pgfpathcurveto{\pgfqpoint{1.372957in}{2.182558in}}{\pgfqpoint{1.362358in}{2.186949in}}{\pgfqpoint{1.351308in}{2.186949in}}%
\pgfpathcurveto{\pgfqpoint{1.340258in}{2.186949in}}{\pgfqpoint{1.329659in}{2.182558in}}{\pgfqpoint{1.321845in}{2.174745in}}%
\pgfpathcurveto{\pgfqpoint{1.314032in}{2.166931in}}{\pgfqpoint{1.309641in}{2.156332in}}{\pgfqpoint{1.309641in}{2.145282in}}%
\pgfpathcurveto{\pgfqpoint{1.309641in}{2.134232in}}{\pgfqpoint{1.314032in}{2.123633in}}{\pgfqpoint{1.321845in}{2.115819in}}%
\pgfpathcurveto{\pgfqpoint{1.329659in}{2.108005in}}{\pgfqpoint{1.340258in}{2.103615in}}{\pgfqpoint{1.351308in}{2.103615in}}%
\pgfpathlineto{\pgfqpoint{1.351308in}{2.103615in}}%
\pgfpathclose%
\pgfusepath{stroke}%
\end{pgfscope}%
\begin{pgfscope}%
\pgfpathrectangle{\pgfqpoint{0.494722in}{0.437222in}}{\pgfqpoint{6.275590in}{5.159444in}}%
\pgfusepath{clip}%
\pgfsetbuttcap%
\pgfsetroundjoin%
\pgfsetlinewidth{1.003750pt}%
\definecolor{currentstroke}{rgb}{0.827451,0.827451,0.827451}%
\pgfsetstrokecolor{currentstroke}%
\pgfsetstrokeopacity{0.800000}%
\pgfsetdash{}{0pt}%
\pgfpathmoveto{\pgfqpoint{1.365286in}{2.103593in}}%
\pgfpathcurveto{\pgfqpoint{1.376336in}{2.103593in}}{\pgfqpoint{1.386936in}{2.107984in}}{\pgfqpoint{1.394749in}{2.115797in}}%
\pgfpathcurveto{\pgfqpoint{1.402563in}{2.123611in}}{\pgfqpoint{1.406953in}{2.134210in}}{\pgfqpoint{1.406953in}{2.145260in}}%
\pgfpathcurveto{\pgfqpoint{1.406953in}{2.156310in}}{\pgfqpoint{1.402563in}{2.166909in}}{\pgfqpoint{1.394749in}{2.174723in}}%
\pgfpathcurveto{\pgfqpoint{1.386936in}{2.182537in}}{\pgfqpoint{1.376336in}{2.186927in}}{\pgfqpoint{1.365286in}{2.186927in}}%
\pgfpathcurveto{\pgfqpoint{1.354236in}{2.186927in}}{\pgfqpoint{1.343637in}{2.182537in}}{\pgfqpoint{1.335824in}{2.174723in}}%
\pgfpathcurveto{\pgfqpoint{1.328010in}{2.166909in}}{\pgfqpoint{1.323620in}{2.156310in}}{\pgfqpoint{1.323620in}{2.145260in}}%
\pgfpathcurveto{\pgfqpoint{1.323620in}{2.134210in}}{\pgfqpoint{1.328010in}{2.123611in}}{\pgfqpoint{1.335824in}{2.115797in}}%
\pgfpathcurveto{\pgfqpoint{1.343637in}{2.107984in}}{\pgfqpoint{1.354236in}{2.103593in}}{\pgfqpoint{1.365286in}{2.103593in}}%
\pgfpathlineto{\pgfqpoint{1.365286in}{2.103593in}}%
\pgfpathclose%
\pgfusepath{stroke}%
\end{pgfscope}%
\begin{pgfscope}%
\pgfpathrectangle{\pgfqpoint{0.494722in}{0.437222in}}{\pgfqpoint{6.275590in}{5.159444in}}%
\pgfusepath{clip}%
\pgfsetbuttcap%
\pgfsetroundjoin%
\pgfsetlinewidth{1.003750pt}%
\definecolor{currentstroke}{rgb}{0.827451,0.827451,0.827451}%
\pgfsetstrokecolor{currentstroke}%
\pgfsetstrokeopacity{0.800000}%
\pgfsetdash{}{0pt}%
\pgfpathmoveto{\pgfqpoint{5.169286in}{0.532515in}}%
\pgfpathcurveto{\pgfqpoint{5.180336in}{0.532515in}}{\pgfqpoint{5.190935in}{0.536906in}}{\pgfqpoint{5.198749in}{0.544719in}}%
\pgfpathcurveto{\pgfqpoint{5.206563in}{0.552533in}}{\pgfqpoint{5.210953in}{0.563132in}}{\pgfqpoint{5.210953in}{0.574182in}}%
\pgfpathcurveto{\pgfqpoint{5.210953in}{0.585232in}}{\pgfqpoint{5.206563in}{0.595831in}}{\pgfqpoint{5.198749in}{0.603645in}}%
\pgfpathcurveto{\pgfqpoint{5.190935in}{0.611458in}}{\pgfqpoint{5.180336in}{0.615849in}}{\pgfqpoint{5.169286in}{0.615849in}}%
\pgfpathcurveto{\pgfqpoint{5.158236in}{0.615849in}}{\pgfqpoint{5.147637in}{0.611458in}}{\pgfqpoint{5.139823in}{0.603645in}}%
\pgfpathcurveto{\pgfqpoint{5.132010in}{0.595831in}}{\pgfqpoint{5.127619in}{0.585232in}}{\pgfqpoint{5.127619in}{0.574182in}}%
\pgfpathcurveto{\pgfqpoint{5.127619in}{0.563132in}}{\pgfqpoint{5.132010in}{0.552533in}}{\pgfqpoint{5.139823in}{0.544719in}}%
\pgfpathcurveto{\pgfqpoint{5.147637in}{0.536906in}}{\pgfqpoint{5.158236in}{0.532515in}}{\pgfqpoint{5.169286in}{0.532515in}}%
\pgfpathlineto{\pgfqpoint{5.169286in}{0.532515in}}%
\pgfpathclose%
\pgfusepath{stroke}%
\end{pgfscope}%
\begin{pgfscope}%
\pgfpathrectangle{\pgfqpoint{0.494722in}{0.437222in}}{\pgfqpoint{6.275590in}{5.159444in}}%
\pgfusepath{clip}%
\pgfsetbuttcap%
\pgfsetroundjoin%
\pgfsetlinewidth{1.003750pt}%
\definecolor{currentstroke}{rgb}{0.827451,0.827451,0.827451}%
\pgfsetstrokecolor{currentstroke}%
\pgfsetstrokeopacity{0.800000}%
\pgfsetdash{}{0pt}%
\pgfpathmoveto{\pgfqpoint{1.976414in}{1.606802in}}%
\pgfpathcurveto{\pgfqpoint{1.987464in}{1.606802in}}{\pgfqpoint{1.998063in}{1.611193in}}{\pgfqpoint{2.005877in}{1.619006in}}%
\pgfpathcurveto{\pgfqpoint{2.013691in}{1.626820in}}{\pgfqpoint{2.018081in}{1.637419in}}{\pgfqpoint{2.018081in}{1.648469in}}%
\pgfpathcurveto{\pgfqpoint{2.018081in}{1.659519in}}{\pgfqpoint{2.013691in}{1.670118in}}{\pgfqpoint{2.005877in}{1.677932in}}%
\pgfpathcurveto{\pgfqpoint{1.998063in}{1.685745in}}{\pgfqpoint{1.987464in}{1.690136in}}{\pgfqpoint{1.976414in}{1.690136in}}%
\pgfpathcurveto{\pgfqpoint{1.965364in}{1.690136in}}{\pgfqpoint{1.954765in}{1.685745in}}{\pgfqpoint{1.946952in}{1.677932in}}%
\pgfpathcurveto{\pgfqpoint{1.939138in}{1.670118in}}{\pgfqpoint{1.934748in}{1.659519in}}{\pgfqpoint{1.934748in}{1.648469in}}%
\pgfpathcurveto{\pgfqpoint{1.934748in}{1.637419in}}{\pgfqpoint{1.939138in}{1.626820in}}{\pgfqpoint{1.946952in}{1.619006in}}%
\pgfpathcurveto{\pgfqpoint{1.954765in}{1.611193in}}{\pgfqpoint{1.965364in}{1.606802in}}{\pgfqpoint{1.976414in}{1.606802in}}%
\pgfpathlineto{\pgfqpoint{1.976414in}{1.606802in}}%
\pgfpathclose%
\pgfusepath{stroke}%
\end{pgfscope}%
\begin{pgfscope}%
\pgfpathrectangle{\pgfqpoint{0.494722in}{0.437222in}}{\pgfqpoint{6.275590in}{5.159444in}}%
\pgfusepath{clip}%
\pgfsetbuttcap%
\pgfsetroundjoin%
\pgfsetlinewidth{1.003750pt}%
\definecolor{currentstroke}{rgb}{0.827451,0.827451,0.827451}%
\pgfsetstrokecolor{currentstroke}%
\pgfsetstrokeopacity{0.800000}%
\pgfsetdash{}{0pt}%
\pgfpathmoveto{\pgfqpoint{1.844439in}{1.882865in}}%
\pgfpathcurveto{\pgfqpoint{1.855489in}{1.882865in}}{\pgfqpoint{1.866088in}{1.887255in}}{\pgfqpoint{1.873902in}{1.895069in}}%
\pgfpathcurveto{\pgfqpoint{1.881716in}{1.902883in}}{\pgfqpoint{1.886106in}{1.913482in}}{\pgfqpoint{1.886106in}{1.924532in}}%
\pgfpathcurveto{\pgfqpoint{1.886106in}{1.935582in}}{\pgfqpoint{1.881716in}{1.946181in}}{\pgfqpoint{1.873902in}{1.953994in}}%
\pgfpathcurveto{\pgfqpoint{1.866088in}{1.961808in}}{\pgfqpoint{1.855489in}{1.966198in}}{\pgfqpoint{1.844439in}{1.966198in}}%
\pgfpathcurveto{\pgfqpoint{1.833389in}{1.966198in}}{\pgfqpoint{1.822790in}{1.961808in}}{\pgfqpoint{1.814977in}{1.953994in}}%
\pgfpathcurveto{\pgfqpoint{1.807163in}{1.946181in}}{\pgfqpoint{1.802773in}{1.935582in}}{\pgfqpoint{1.802773in}{1.924532in}}%
\pgfpathcurveto{\pgfqpoint{1.802773in}{1.913482in}}{\pgfqpoint{1.807163in}{1.902883in}}{\pgfqpoint{1.814977in}{1.895069in}}%
\pgfpathcurveto{\pgfqpoint{1.822790in}{1.887255in}}{\pgfqpoint{1.833389in}{1.882865in}}{\pgfqpoint{1.844439in}{1.882865in}}%
\pgfpathlineto{\pgfqpoint{1.844439in}{1.882865in}}%
\pgfpathclose%
\pgfusepath{stroke}%
\end{pgfscope}%
\begin{pgfscope}%
\pgfpathrectangle{\pgfqpoint{0.494722in}{0.437222in}}{\pgfqpoint{6.275590in}{5.159444in}}%
\pgfusepath{clip}%
\pgfsetbuttcap%
\pgfsetroundjoin%
\pgfsetlinewidth{1.003750pt}%
\definecolor{currentstroke}{rgb}{0.827451,0.827451,0.827451}%
\pgfsetstrokecolor{currentstroke}%
\pgfsetstrokeopacity{0.800000}%
\pgfsetdash{}{0pt}%
\pgfpathmoveto{\pgfqpoint{1.797146in}{2.098797in}}%
\pgfpathcurveto{\pgfqpoint{1.808196in}{2.098797in}}{\pgfqpoint{1.818795in}{2.103187in}}{\pgfqpoint{1.826609in}{2.111000in}}%
\pgfpathcurveto{\pgfqpoint{1.834422in}{2.118814in}}{\pgfqpoint{1.838813in}{2.129413in}}{\pgfqpoint{1.838813in}{2.140463in}}%
\pgfpathcurveto{\pgfqpoint{1.838813in}{2.151513in}}{\pgfqpoint{1.834422in}{2.162112in}}{\pgfqpoint{1.826609in}{2.169926in}}%
\pgfpathcurveto{\pgfqpoint{1.818795in}{2.177740in}}{\pgfqpoint{1.808196in}{2.182130in}}{\pgfqpoint{1.797146in}{2.182130in}}%
\pgfpathcurveto{\pgfqpoint{1.786096in}{2.182130in}}{\pgfqpoint{1.775497in}{2.177740in}}{\pgfqpoint{1.767683in}{2.169926in}}%
\pgfpathcurveto{\pgfqpoint{1.759870in}{2.162112in}}{\pgfqpoint{1.755479in}{2.151513in}}{\pgfqpoint{1.755479in}{2.140463in}}%
\pgfpathcurveto{\pgfqpoint{1.755479in}{2.129413in}}{\pgfqpoint{1.759870in}{2.118814in}}{\pgfqpoint{1.767683in}{2.111000in}}%
\pgfpathcurveto{\pgfqpoint{1.775497in}{2.103187in}}{\pgfqpoint{1.786096in}{2.098797in}}{\pgfqpoint{1.797146in}{2.098797in}}%
\pgfpathlineto{\pgfqpoint{1.797146in}{2.098797in}}%
\pgfpathclose%
\pgfusepath{stroke}%
\end{pgfscope}%
\begin{pgfscope}%
\pgfpathrectangle{\pgfqpoint{0.494722in}{0.437222in}}{\pgfqpoint{6.275590in}{5.159444in}}%
\pgfusepath{clip}%
\pgfsetbuttcap%
\pgfsetroundjoin%
\pgfsetlinewidth{1.003750pt}%
\definecolor{currentstroke}{rgb}{0.827451,0.827451,0.827451}%
\pgfsetstrokecolor{currentstroke}%
\pgfsetstrokeopacity{0.800000}%
\pgfsetdash{}{0pt}%
\pgfpathmoveto{\pgfqpoint{1.947777in}{1.728525in}}%
\pgfpathcurveto{\pgfqpoint{1.958827in}{1.728525in}}{\pgfqpoint{1.969426in}{1.732916in}}{\pgfqpoint{1.977239in}{1.740729in}}%
\pgfpathcurveto{\pgfqpoint{1.985053in}{1.748543in}}{\pgfqpoint{1.989443in}{1.759142in}}{\pgfqpoint{1.989443in}{1.770192in}}%
\pgfpathcurveto{\pgfqpoint{1.989443in}{1.781242in}}{\pgfqpoint{1.985053in}{1.791841in}}{\pgfqpoint{1.977239in}{1.799655in}}%
\pgfpathcurveto{\pgfqpoint{1.969426in}{1.807468in}}{\pgfqpoint{1.958827in}{1.811859in}}{\pgfqpoint{1.947777in}{1.811859in}}%
\pgfpathcurveto{\pgfqpoint{1.936726in}{1.811859in}}{\pgfqpoint{1.926127in}{1.807468in}}{\pgfqpoint{1.918314in}{1.799655in}}%
\pgfpathcurveto{\pgfqpoint{1.910500in}{1.791841in}}{\pgfqpoint{1.906110in}{1.781242in}}{\pgfqpoint{1.906110in}{1.770192in}}%
\pgfpathcurveto{\pgfqpoint{1.906110in}{1.759142in}}{\pgfqpoint{1.910500in}{1.748543in}}{\pgfqpoint{1.918314in}{1.740729in}}%
\pgfpathcurveto{\pgfqpoint{1.926127in}{1.732916in}}{\pgfqpoint{1.936726in}{1.728525in}}{\pgfqpoint{1.947777in}{1.728525in}}%
\pgfpathlineto{\pgfqpoint{1.947777in}{1.728525in}}%
\pgfpathclose%
\pgfusepath{stroke}%
\end{pgfscope}%
\begin{pgfscope}%
\pgfpathrectangle{\pgfqpoint{0.494722in}{0.437222in}}{\pgfqpoint{6.275590in}{5.159444in}}%
\pgfusepath{clip}%
\pgfsetbuttcap%
\pgfsetroundjoin%
\pgfsetlinewidth{1.003750pt}%
\definecolor{currentstroke}{rgb}{0.827451,0.827451,0.827451}%
\pgfsetstrokecolor{currentstroke}%
\pgfsetstrokeopacity{0.800000}%
\pgfsetdash{}{0pt}%
\pgfpathmoveto{\pgfqpoint{0.767639in}{3.316864in}}%
\pgfpathcurveto{\pgfqpoint{0.778690in}{3.316864in}}{\pgfqpoint{0.789289in}{3.321254in}}{\pgfqpoint{0.797102in}{3.329068in}}%
\pgfpathcurveto{\pgfqpoint{0.804916in}{3.336882in}}{\pgfqpoint{0.809306in}{3.347481in}}{\pgfqpoint{0.809306in}{3.358531in}}%
\pgfpathcurveto{\pgfqpoint{0.809306in}{3.369581in}}{\pgfqpoint{0.804916in}{3.380180in}}{\pgfqpoint{0.797102in}{3.387994in}}%
\pgfpathcurveto{\pgfqpoint{0.789289in}{3.395807in}}{\pgfqpoint{0.778690in}{3.400198in}}{\pgfqpoint{0.767639in}{3.400198in}}%
\pgfpathcurveto{\pgfqpoint{0.756589in}{3.400198in}}{\pgfqpoint{0.745990in}{3.395807in}}{\pgfqpoint{0.738177in}{3.387994in}}%
\pgfpathcurveto{\pgfqpoint{0.730363in}{3.380180in}}{\pgfqpoint{0.725973in}{3.369581in}}{\pgfqpoint{0.725973in}{3.358531in}}%
\pgfpathcurveto{\pgfqpoint{0.725973in}{3.347481in}}{\pgfqpoint{0.730363in}{3.336882in}}{\pgfqpoint{0.738177in}{3.329068in}}%
\pgfpathcurveto{\pgfqpoint{0.745990in}{3.321254in}}{\pgfqpoint{0.756589in}{3.316864in}}{\pgfqpoint{0.767639in}{3.316864in}}%
\pgfpathlineto{\pgfqpoint{0.767639in}{3.316864in}}%
\pgfpathclose%
\pgfusepath{stroke}%
\end{pgfscope}%
\begin{pgfscope}%
\pgfpathrectangle{\pgfqpoint{0.494722in}{0.437222in}}{\pgfqpoint{6.275590in}{5.159444in}}%
\pgfusepath{clip}%
\pgfsetbuttcap%
\pgfsetroundjoin%
\pgfsetlinewidth{1.003750pt}%
\definecolor{currentstroke}{rgb}{0.827451,0.827451,0.827451}%
\pgfsetstrokecolor{currentstroke}%
\pgfsetstrokeopacity{0.800000}%
\pgfsetdash{}{0pt}%
\pgfpathmoveto{\pgfqpoint{4.986066in}{0.665355in}}%
\pgfpathcurveto{\pgfqpoint{4.997116in}{0.665355in}}{\pgfqpoint{5.007715in}{0.669745in}}{\pgfqpoint{5.015529in}{0.677559in}}%
\pgfpathcurveto{\pgfqpoint{5.023342in}{0.685373in}}{\pgfqpoint{5.027732in}{0.695972in}}{\pgfqpoint{5.027732in}{0.707022in}}%
\pgfpathcurveto{\pgfqpoint{5.027732in}{0.718072in}}{\pgfqpoint{5.023342in}{0.728671in}}{\pgfqpoint{5.015529in}{0.736485in}}%
\pgfpathcurveto{\pgfqpoint{5.007715in}{0.744298in}}{\pgfqpoint{4.997116in}{0.748688in}}{\pgfqpoint{4.986066in}{0.748688in}}%
\pgfpathcurveto{\pgfqpoint{4.975016in}{0.748688in}}{\pgfqpoint{4.964417in}{0.744298in}}{\pgfqpoint{4.956603in}{0.736485in}}%
\pgfpathcurveto{\pgfqpoint{4.948789in}{0.728671in}}{\pgfqpoint{4.944399in}{0.718072in}}{\pgfqpoint{4.944399in}{0.707022in}}%
\pgfpathcurveto{\pgfqpoint{4.944399in}{0.695972in}}{\pgfqpoint{4.948789in}{0.685373in}}{\pgfqpoint{4.956603in}{0.677559in}}%
\pgfpathcurveto{\pgfqpoint{4.964417in}{0.669745in}}{\pgfqpoint{4.975016in}{0.665355in}}{\pgfqpoint{4.986066in}{0.665355in}}%
\pgfpathlineto{\pgfqpoint{4.986066in}{0.665355in}}%
\pgfpathclose%
\pgfusepath{stroke}%
\end{pgfscope}%
\begin{pgfscope}%
\pgfpathrectangle{\pgfqpoint{0.494722in}{0.437222in}}{\pgfqpoint{6.275590in}{5.159444in}}%
\pgfusepath{clip}%
\pgfsetbuttcap%
\pgfsetroundjoin%
\pgfsetlinewidth{1.003750pt}%
\definecolor{currentstroke}{rgb}{0.827451,0.827451,0.827451}%
\pgfsetstrokecolor{currentstroke}%
\pgfsetstrokeopacity{0.800000}%
\pgfsetdash{}{0pt}%
\pgfpathmoveto{\pgfqpoint{2.341197in}{1.587172in}}%
\pgfpathcurveto{\pgfqpoint{2.352247in}{1.587172in}}{\pgfqpoint{2.362846in}{1.591562in}}{\pgfqpoint{2.370660in}{1.599376in}}%
\pgfpathcurveto{\pgfqpoint{2.378473in}{1.607189in}}{\pgfqpoint{2.382863in}{1.617789in}}{\pgfqpoint{2.382863in}{1.628839in}}%
\pgfpathcurveto{\pgfqpoint{2.382863in}{1.639889in}}{\pgfqpoint{2.378473in}{1.650488in}}{\pgfqpoint{2.370660in}{1.658301in}}%
\pgfpathcurveto{\pgfqpoint{2.362846in}{1.666115in}}{\pgfqpoint{2.352247in}{1.670505in}}{\pgfqpoint{2.341197in}{1.670505in}}%
\pgfpathcurveto{\pgfqpoint{2.330147in}{1.670505in}}{\pgfqpoint{2.319548in}{1.666115in}}{\pgfqpoint{2.311734in}{1.658301in}}%
\pgfpathcurveto{\pgfqpoint{2.303920in}{1.650488in}}{\pgfqpoint{2.299530in}{1.639889in}}{\pgfqpoint{2.299530in}{1.628839in}}%
\pgfpathcurveto{\pgfqpoint{2.299530in}{1.617789in}}{\pgfqpoint{2.303920in}{1.607189in}}{\pgfqpoint{2.311734in}{1.599376in}}%
\pgfpathcurveto{\pgfqpoint{2.319548in}{1.591562in}}{\pgfqpoint{2.330147in}{1.587172in}}{\pgfqpoint{2.341197in}{1.587172in}}%
\pgfpathlineto{\pgfqpoint{2.341197in}{1.587172in}}%
\pgfpathclose%
\pgfusepath{stroke}%
\end{pgfscope}%
\begin{pgfscope}%
\pgfpathrectangle{\pgfqpoint{0.494722in}{0.437222in}}{\pgfqpoint{6.275590in}{5.159444in}}%
\pgfusepath{clip}%
\pgfsetbuttcap%
\pgfsetroundjoin%
\pgfsetlinewidth{1.003750pt}%
\definecolor{currentstroke}{rgb}{0.827451,0.827451,0.827451}%
\pgfsetstrokecolor{currentstroke}%
\pgfsetstrokeopacity{0.800000}%
\pgfsetdash{}{0pt}%
\pgfpathmoveto{\pgfqpoint{2.360105in}{1.495665in}}%
\pgfpathcurveto{\pgfqpoint{2.371155in}{1.495665in}}{\pgfqpoint{2.381754in}{1.500056in}}{\pgfqpoint{2.389568in}{1.507869in}}%
\pgfpathcurveto{\pgfqpoint{2.397382in}{1.515683in}}{\pgfqpoint{2.401772in}{1.526282in}}{\pgfqpoint{2.401772in}{1.537332in}}%
\pgfpathcurveto{\pgfqpoint{2.401772in}{1.548382in}}{\pgfqpoint{2.397382in}{1.558981in}}{\pgfqpoint{2.389568in}{1.566795in}}%
\pgfpathcurveto{\pgfqpoint{2.381754in}{1.574608in}}{\pgfqpoint{2.371155in}{1.578999in}}{\pgfqpoint{2.360105in}{1.578999in}}%
\pgfpathcurveto{\pgfqpoint{2.349055in}{1.578999in}}{\pgfqpoint{2.338456in}{1.574608in}}{\pgfqpoint{2.330643in}{1.566795in}}%
\pgfpathcurveto{\pgfqpoint{2.322829in}{1.558981in}}{\pgfqpoint{2.318439in}{1.548382in}}{\pgfqpoint{2.318439in}{1.537332in}}%
\pgfpathcurveto{\pgfqpoint{2.318439in}{1.526282in}}{\pgfqpoint{2.322829in}{1.515683in}}{\pgfqpoint{2.330643in}{1.507869in}}%
\pgfpathcurveto{\pgfqpoint{2.338456in}{1.500056in}}{\pgfqpoint{2.349055in}{1.495665in}}{\pgfqpoint{2.360105in}{1.495665in}}%
\pgfpathlineto{\pgfqpoint{2.360105in}{1.495665in}}%
\pgfpathclose%
\pgfusepath{stroke}%
\end{pgfscope}%
\begin{pgfscope}%
\pgfpathrectangle{\pgfqpoint{0.494722in}{0.437222in}}{\pgfqpoint{6.275590in}{5.159444in}}%
\pgfusepath{clip}%
\pgfsetbuttcap%
\pgfsetroundjoin%
\pgfsetlinewidth{1.003750pt}%
\definecolor{currentstroke}{rgb}{0.827451,0.827451,0.827451}%
\pgfsetstrokecolor{currentstroke}%
\pgfsetstrokeopacity{0.800000}%
\pgfsetdash{}{0pt}%
\pgfpathmoveto{\pgfqpoint{1.072122in}{3.062639in}}%
\pgfpathcurveto{\pgfqpoint{1.083172in}{3.062639in}}{\pgfqpoint{1.093771in}{3.067029in}}{\pgfqpoint{1.101584in}{3.074842in}}%
\pgfpathcurveto{\pgfqpoint{1.109398in}{3.082656in}}{\pgfqpoint{1.113788in}{3.093255in}}{\pgfqpoint{1.113788in}{3.104305in}}%
\pgfpathcurveto{\pgfqpoint{1.113788in}{3.115355in}}{\pgfqpoint{1.109398in}{3.125954in}}{\pgfqpoint{1.101584in}{3.133768in}}%
\pgfpathcurveto{\pgfqpoint{1.093771in}{3.141582in}}{\pgfqpoint{1.083172in}{3.145972in}}{\pgfqpoint{1.072122in}{3.145972in}}%
\pgfpathcurveto{\pgfqpoint{1.061071in}{3.145972in}}{\pgfqpoint{1.050472in}{3.141582in}}{\pgfqpoint{1.042659in}{3.133768in}}%
\pgfpathcurveto{\pgfqpoint{1.034845in}{3.125954in}}{\pgfqpoint{1.030455in}{3.115355in}}{\pgfqpoint{1.030455in}{3.104305in}}%
\pgfpathcurveto{\pgfqpoint{1.030455in}{3.093255in}}{\pgfqpoint{1.034845in}{3.082656in}}{\pgfqpoint{1.042659in}{3.074842in}}%
\pgfpathcurveto{\pgfqpoint{1.050472in}{3.067029in}}{\pgfqpoint{1.061071in}{3.062639in}}{\pgfqpoint{1.072122in}{3.062639in}}%
\pgfpathlineto{\pgfqpoint{1.072122in}{3.062639in}}%
\pgfpathclose%
\pgfusepath{stroke}%
\end{pgfscope}%
\begin{pgfscope}%
\pgfpathrectangle{\pgfqpoint{0.494722in}{0.437222in}}{\pgfqpoint{6.275590in}{5.159444in}}%
\pgfusepath{clip}%
\pgfsetbuttcap%
\pgfsetroundjoin%
\pgfsetlinewidth{1.003750pt}%
\definecolor{currentstroke}{rgb}{0.827451,0.827451,0.827451}%
\pgfsetstrokecolor{currentstroke}%
\pgfsetstrokeopacity{0.800000}%
\pgfsetdash{}{0pt}%
\pgfpathmoveto{\pgfqpoint{1.268981in}{2.957639in}}%
\pgfpathcurveto{\pgfqpoint{1.280031in}{2.957639in}}{\pgfqpoint{1.290630in}{2.962029in}}{\pgfqpoint{1.298443in}{2.969843in}}%
\pgfpathcurveto{\pgfqpoint{1.306257in}{2.977657in}}{\pgfqpoint{1.310647in}{2.988256in}}{\pgfqpoint{1.310647in}{2.999306in}}%
\pgfpathcurveto{\pgfqpoint{1.310647in}{3.010356in}}{\pgfqpoint{1.306257in}{3.020955in}}{\pgfqpoint{1.298443in}{3.028769in}}%
\pgfpathcurveto{\pgfqpoint{1.290630in}{3.036582in}}{\pgfqpoint{1.280031in}{3.040973in}}{\pgfqpoint{1.268981in}{3.040973in}}%
\pgfpathcurveto{\pgfqpoint{1.257931in}{3.040973in}}{\pgfqpoint{1.247332in}{3.036582in}}{\pgfqpoint{1.239518in}{3.028769in}}%
\pgfpathcurveto{\pgfqpoint{1.231704in}{3.020955in}}{\pgfqpoint{1.227314in}{3.010356in}}{\pgfqpoint{1.227314in}{2.999306in}}%
\pgfpathcurveto{\pgfqpoint{1.227314in}{2.988256in}}{\pgfqpoint{1.231704in}{2.977657in}}{\pgfqpoint{1.239518in}{2.969843in}}%
\pgfpathcurveto{\pgfqpoint{1.247332in}{2.962029in}}{\pgfqpoint{1.257931in}{2.957639in}}{\pgfqpoint{1.268981in}{2.957639in}}%
\pgfpathlineto{\pgfqpoint{1.268981in}{2.957639in}}%
\pgfpathclose%
\pgfusepath{stroke}%
\end{pgfscope}%
\begin{pgfscope}%
\pgfpathrectangle{\pgfqpoint{0.494722in}{0.437222in}}{\pgfqpoint{6.275590in}{5.159444in}}%
\pgfusepath{clip}%
\pgfsetbuttcap%
\pgfsetroundjoin%
\pgfsetlinewidth{1.003750pt}%
\definecolor{currentstroke}{rgb}{0.827451,0.827451,0.827451}%
\pgfsetstrokecolor{currentstroke}%
\pgfsetstrokeopacity{0.800000}%
\pgfsetdash{}{0pt}%
\pgfpathmoveto{\pgfqpoint{3.121016in}{1.150537in}}%
\pgfpathcurveto{\pgfqpoint{3.132066in}{1.150537in}}{\pgfqpoint{3.142665in}{1.154927in}}{\pgfqpoint{3.150478in}{1.162741in}}%
\pgfpathcurveto{\pgfqpoint{3.158292in}{1.170554in}}{\pgfqpoint{3.162682in}{1.181153in}}{\pgfqpoint{3.162682in}{1.192203in}}%
\pgfpathcurveto{\pgfqpoint{3.162682in}{1.203253in}}{\pgfqpoint{3.158292in}{1.213853in}}{\pgfqpoint{3.150478in}{1.221666in}}%
\pgfpathcurveto{\pgfqpoint{3.142665in}{1.229480in}}{\pgfqpoint{3.132066in}{1.233870in}}{\pgfqpoint{3.121016in}{1.233870in}}%
\pgfpathcurveto{\pgfqpoint{3.109965in}{1.233870in}}{\pgfqpoint{3.099366in}{1.229480in}}{\pgfqpoint{3.091553in}{1.221666in}}%
\pgfpathcurveto{\pgfqpoint{3.083739in}{1.213853in}}{\pgfqpoint{3.079349in}{1.203253in}}{\pgfqpoint{3.079349in}{1.192203in}}%
\pgfpathcurveto{\pgfqpoint{3.079349in}{1.181153in}}{\pgfqpoint{3.083739in}{1.170554in}}{\pgfqpoint{3.091553in}{1.162741in}}%
\pgfpathcurveto{\pgfqpoint{3.099366in}{1.154927in}}{\pgfqpoint{3.109965in}{1.150537in}}{\pgfqpoint{3.121016in}{1.150537in}}%
\pgfpathlineto{\pgfqpoint{3.121016in}{1.150537in}}%
\pgfpathclose%
\pgfusepath{stroke}%
\end{pgfscope}%
\begin{pgfscope}%
\pgfpathrectangle{\pgfqpoint{0.494722in}{0.437222in}}{\pgfqpoint{6.275590in}{5.159444in}}%
\pgfusepath{clip}%
\pgfsetbuttcap%
\pgfsetroundjoin%
\pgfsetlinewidth{1.003750pt}%
\definecolor{currentstroke}{rgb}{0.827451,0.827451,0.827451}%
\pgfsetstrokecolor{currentstroke}%
\pgfsetstrokeopacity{0.800000}%
\pgfsetdash{}{0pt}%
\pgfpathmoveto{\pgfqpoint{3.382793in}{1.084291in}}%
\pgfpathcurveto{\pgfqpoint{3.393844in}{1.084291in}}{\pgfqpoint{3.404443in}{1.088681in}}{\pgfqpoint{3.412256in}{1.096495in}}%
\pgfpathcurveto{\pgfqpoint{3.420070in}{1.104308in}}{\pgfqpoint{3.424460in}{1.114907in}}{\pgfqpoint{3.424460in}{1.125957in}}%
\pgfpathcurveto{\pgfqpoint{3.424460in}{1.137008in}}{\pgfqpoint{3.420070in}{1.147607in}}{\pgfqpoint{3.412256in}{1.155420in}}%
\pgfpathcurveto{\pgfqpoint{3.404443in}{1.163234in}}{\pgfqpoint{3.393844in}{1.167624in}}{\pgfqpoint{3.382793in}{1.167624in}}%
\pgfpathcurveto{\pgfqpoint{3.371743in}{1.167624in}}{\pgfqpoint{3.361144in}{1.163234in}}{\pgfqpoint{3.353331in}{1.155420in}}%
\pgfpathcurveto{\pgfqpoint{3.345517in}{1.147607in}}{\pgfqpoint{3.341127in}{1.137008in}}{\pgfqpoint{3.341127in}{1.125957in}}%
\pgfpathcurveto{\pgfqpoint{3.341127in}{1.114907in}}{\pgfqpoint{3.345517in}{1.104308in}}{\pgfqpoint{3.353331in}{1.096495in}}%
\pgfpathcurveto{\pgfqpoint{3.361144in}{1.088681in}}{\pgfqpoint{3.371743in}{1.084291in}}{\pgfqpoint{3.382793in}{1.084291in}}%
\pgfpathlineto{\pgfqpoint{3.382793in}{1.084291in}}%
\pgfpathclose%
\pgfusepath{stroke}%
\end{pgfscope}%
\begin{pgfscope}%
\pgfpathrectangle{\pgfqpoint{0.494722in}{0.437222in}}{\pgfqpoint{6.275590in}{5.159444in}}%
\pgfusepath{clip}%
\pgfsetbuttcap%
\pgfsetroundjoin%
\pgfsetlinewidth{1.003750pt}%
\definecolor{currentstroke}{rgb}{0.827451,0.827451,0.827451}%
\pgfsetstrokecolor{currentstroke}%
\pgfsetstrokeopacity{0.800000}%
\pgfsetdash{}{0pt}%
\pgfpathmoveto{\pgfqpoint{4.389282in}{0.813645in}}%
\pgfpathcurveto{\pgfqpoint{4.400332in}{0.813645in}}{\pgfqpoint{4.410931in}{0.818035in}}{\pgfqpoint{4.418744in}{0.825849in}}%
\pgfpathcurveto{\pgfqpoint{4.426558in}{0.833663in}}{\pgfqpoint{4.430948in}{0.844262in}}{\pgfqpoint{4.430948in}{0.855312in}}%
\pgfpathcurveto{\pgfqpoint{4.430948in}{0.866362in}}{\pgfqpoint{4.426558in}{0.876961in}}{\pgfqpoint{4.418744in}{0.884774in}}%
\pgfpathcurveto{\pgfqpoint{4.410931in}{0.892588in}}{\pgfqpoint{4.400332in}{0.896978in}}{\pgfqpoint{4.389282in}{0.896978in}}%
\pgfpathcurveto{\pgfqpoint{4.378231in}{0.896978in}}{\pgfqpoint{4.367632in}{0.892588in}}{\pgfqpoint{4.359819in}{0.884774in}}%
\pgfpathcurveto{\pgfqpoint{4.352005in}{0.876961in}}{\pgfqpoint{4.347615in}{0.866362in}}{\pgfqpoint{4.347615in}{0.855312in}}%
\pgfpathcurveto{\pgfqpoint{4.347615in}{0.844262in}}{\pgfqpoint{4.352005in}{0.833663in}}{\pgfqpoint{4.359819in}{0.825849in}}%
\pgfpathcurveto{\pgfqpoint{4.367632in}{0.818035in}}{\pgfqpoint{4.378231in}{0.813645in}}{\pgfqpoint{4.389282in}{0.813645in}}%
\pgfpathlineto{\pgfqpoint{4.389282in}{0.813645in}}%
\pgfpathclose%
\pgfusepath{stroke}%
\end{pgfscope}%
\begin{pgfscope}%
\pgfpathrectangle{\pgfqpoint{0.494722in}{0.437222in}}{\pgfqpoint{6.275590in}{5.159444in}}%
\pgfusepath{clip}%
\pgfsetbuttcap%
\pgfsetroundjoin%
\pgfsetlinewidth{1.003750pt}%
\definecolor{currentstroke}{rgb}{0.827451,0.827451,0.827451}%
\pgfsetstrokecolor{currentstroke}%
\pgfsetstrokeopacity{0.800000}%
\pgfsetdash{}{0pt}%
\pgfpathmoveto{\pgfqpoint{3.981806in}{0.841160in}}%
\pgfpathcurveto{\pgfqpoint{3.992856in}{0.841160in}}{\pgfqpoint{4.003455in}{0.845550in}}{\pgfqpoint{4.011268in}{0.853364in}}%
\pgfpathcurveto{\pgfqpoint{4.019082in}{0.861177in}}{\pgfqpoint{4.023472in}{0.871776in}}{\pgfqpoint{4.023472in}{0.882826in}}%
\pgfpathcurveto{\pgfqpoint{4.023472in}{0.893877in}}{\pgfqpoint{4.019082in}{0.904476in}}{\pgfqpoint{4.011268in}{0.912289in}}%
\pgfpathcurveto{\pgfqpoint{4.003455in}{0.920103in}}{\pgfqpoint{3.992856in}{0.924493in}}{\pgfqpoint{3.981806in}{0.924493in}}%
\pgfpathcurveto{\pgfqpoint{3.970756in}{0.924493in}}{\pgfqpoint{3.960157in}{0.920103in}}{\pgfqpoint{3.952343in}{0.912289in}}%
\pgfpathcurveto{\pgfqpoint{3.944529in}{0.904476in}}{\pgfqpoint{3.940139in}{0.893877in}}{\pgfqpoint{3.940139in}{0.882826in}}%
\pgfpathcurveto{\pgfqpoint{3.940139in}{0.871776in}}{\pgfqpoint{3.944529in}{0.861177in}}{\pgfqpoint{3.952343in}{0.853364in}}%
\pgfpathcurveto{\pgfqpoint{3.960157in}{0.845550in}}{\pgfqpoint{3.970756in}{0.841160in}}{\pgfqpoint{3.981806in}{0.841160in}}%
\pgfpathlineto{\pgfqpoint{3.981806in}{0.841160in}}%
\pgfpathclose%
\pgfusepath{stroke}%
\end{pgfscope}%
\begin{pgfscope}%
\pgfpathrectangle{\pgfqpoint{0.494722in}{0.437222in}}{\pgfqpoint{6.275590in}{5.159444in}}%
\pgfusepath{clip}%
\pgfsetbuttcap%
\pgfsetroundjoin%
\pgfsetlinewidth{1.003750pt}%
\definecolor{currentstroke}{rgb}{0.827451,0.827451,0.827451}%
\pgfsetstrokecolor{currentstroke}%
\pgfsetstrokeopacity{0.800000}%
\pgfsetdash{}{0pt}%
\pgfpathmoveto{\pgfqpoint{1.439787in}{2.163927in}}%
\pgfpathcurveto{\pgfqpoint{1.450837in}{2.163927in}}{\pgfqpoint{1.461436in}{2.168317in}}{\pgfqpoint{1.469250in}{2.176131in}}%
\pgfpathcurveto{\pgfqpoint{1.477063in}{2.183944in}}{\pgfqpoint{1.481454in}{2.194543in}}{\pgfqpoint{1.481454in}{2.205593in}}%
\pgfpathcurveto{\pgfqpoint{1.481454in}{2.216644in}}{\pgfqpoint{1.477063in}{2.227243in}}{\pgfqpoint{1.469250in}{2.235056in}}%
\pgfpathcurveto{\pgfqpoint{1.461436in}{2.242870in}}{\pgfqpoint{1.450837in}{2.247260in}}{\pgfqpoint{1.439787in}{2.247260in}}%
\pgfpathcurveto{\pgfqpoint{1.428737in}{2.247260in}}{\pgfqpoint{1.418138in}{2.242870in}}{\pgfqpoint{1.410324in}{2.235056in}}%
\pgfpathcurveto{\pgfqpoint{1.402511in}{2.227243in}}{\pgfqpoint{1.398120in}{2.216644in}}{\pgfqpoint{1.398120in}{2.205593in}}%
\pgfpathcurveto{\pgfqpoint{1.398120in}{2.194543in}}{\pgfqpoint{1.402511in}{2.183944in}}{\pgfqpoint{1.410324in}{2.176131in}}%
\pgfpathcurveto{\pgfqpoint{1.418138in}{2.168317in}}{\pgfqpoint{1.428737in}{2.163927in}}{\pgfqpoint{1.439787in}{2.163927in}}%
\pgfpathlineto{\pgfqpoint{1.439787in}{2.163927in}}%
\pgfpathclose%
\pgfusepath{stroke}%
\end{pgfscope}%
\begin{pgfscope}%
\pgfpathrectangle{\pgfqpoint{0.494722in}{0.437222in}}{\pgfqpoint{6.275590in}{5.159444in}}%
\pgfusepath{clip}%
\pgfsetbuttcap%
\pgfsetroundjoin%
\pgfsetlinewidth{1.003750pt}%
\definecolor{currentstroke}{rgb}{0.827451,0.827451,0.827451}%
\pgfsetstrokecolor{currentstroke}%
\pgfsetstrokeopacity{0.800000}%
\pgfsetdash{}{0pt}%
\pgfpathmoveto{\pgfqpoint{1.549288in}{2.110229in}}%
\pgfpathcurveto{\pgfqpoint{1.560338in}{2.110229in}}{\pgfqpoint{1.570937in}{2.114619in}}{\pgfqpoint{1.578750in}{2.122433in}}%
\pgfpathcurveto{\pgfqpoint{1.586564in}{2.130246in}}{\pgfqpoint{1.590954in}{2.140845in}}{\pgfqpoint{1.590954in}{2.151895in}}%
\pgfpathcurveto{\pgfqpoint{1.590954in}{2.162945in}}{\pgfqpoint{1.586564in}{2.173544in}}{\pgfqpoint{1.578750in}{2.181358in}}%
\pgfpathcurveto{\pgfqpoint{1.570937in}{2.189172in}}{\pgfqpoint{1.560338in}{2.193562in}}{\pgfqpoint{1.549288in}{2.193562in}}%
\pgfpathcurveto{\pgfqpoint{1.538237in}{2.193562in}}{\pgfqpoint{1.527638in}{2.189172in}}{\pgfqpoint{1.519825in}{2.181358in}}%
\pgfpathcurveto{\pgfqpoint{1.512011in}{2.173544in}}{\pgfqpoint{1.507621in}{2.162945in}}{\pgfqpoint{1.507621in}{2.151895in}}%
\pgfpathcurveto{\pgfqpoint{1.507621in}{2.140845in}}{\pgfqpoint{1.512011in}{2.130246in}}{\pgfqpoint{1.519825in}{2.122433in}}%
\pgfpathcurveto{\pgfqpoint{1.527638in}{2.114619in}}{\pgfqpoint{1.538237in}{2.110229in}}{\pgfqpoint{1.549288in}{2.110229in}}%
\pgfpathlineto{\pgfqpoint{1.549288in}{2.110229in}}%
\pgfpathclose%
\pgfusepath{stroke}%
\end{pgfscope}%
\begin{pgfscope}%
\pgfpathrectangle{\pgfqpoint{0.494722in}{0.437222in}}{\pgfqpoint{6.275590in}{5.159444in}}%
\pgfusepath{clip}%
\pgfsetbuttcap%
\pgfsetroundjoin%
\pgfsetlinewidth{1.003750pt}%
\definecolor{currentstroke}{rgb}{0.827451,0.827451,0.827451}%
\pgfsetstrokecolor{currentstroke}%
\pgfsetstrokeopacity{0.800000}%
\pgfsetdash{}{0pt}%
\pgfpathmoveto{\pgfqpoint{1.377176in}{2.324710in}}%
\pgfpathcurveto{\pgfqpoint{1.388227in}{2.324710in}}{\pgfqpoint{1.398826in}{2.329101in}}{\pgfqpoint{1.406639in}{2.336914in}}%
\pgfpathcurveto{\pgfqpoint{1.414453in}{2.344728in}}{\pgfqpoint{1.418843in}{2.355327in}}{\pgfqpoint{1.418843in}{2.366377in}}%
\pgfpathcurveto{\pgfqpoint{1.418843in}{2.377427in}}{\pgfqpoint{1.414453in}{2.388026in}}{\pgfqpoint{1.406639in}{2.395840in}}%
\pgfpathcurveto{\pgfqpoint{1.398826in}{2.403654in}}{\pgfqpoint{1.388227in}{2.408044in}}{\pgfqpoint{1.377176in}{2.408044in}}%
\pgfpathcurveto{\pgfqpoint{1.366126in}{2.408044in}}{\pgfqpoint{1.355527in}{2.403654in}}{\pgfqpoint{1.347714in}{2.395840in}}%
\pgfpathcurveto{\pgfqpoint{1.339900in}{2.388026in}}{\pgfqpoint{1.335510in}{2.377427in}}{\pgfqpoint{1.335510in}{2.366377in}}%
\pgfpathcurveto{\pgfqpoint{1.335510in}{2.355327in}}{\pgfqpoint{1.339900in}{2.344728in}}{\pgfqpoint{1.347714in}{2.336914in}}%
\pgfpathcurveto{\pgfqpoint{1.355527in}{2.329101in}}{\pgfqpoint{1.366126in}{2.324710in}}{\pgfqpoint{1.377176in}{2.324710in}}%
\pgfpathlineto{\pgfqpoint{1.377176in}{2.324710in}}%
\pgfpathclose%
\pgfusepath{stroke}%
\end{pgfscope}%
\begin{pgfscope}%
\pgfpathrectangle{\pgfqpoint{0.494722in}{0.437222in}}{\pgfqpoint{6.275590in}{5.159444in}}%
\pgfusepath{clip}%
\pgfsetbuttcap%
\pgfsetroundjoin%
\pgfsetlinewidth{1.003750pt}%
\definecolor{currentstroke}{rgb}{0.827451,0.827451,0.827451}%
\pgfsetstrokecolor{currentstroke}%
\pgfsetstrokeopacity{0.800000}%
\pgfsetdash{}{0pt}%
\pgfpathmoveto{\pgfqpoint{5.178838in}{0.549524in}}%
\pgfpathcurveto{\pgfqpoint{5.189888in}{0.549524in}}{\pgfqpoint{5.200487in}{0.553914in}}{\pgfqpoint{5.208301in}{0.561727in}}%
\pgfpathcurveto{\pgfqpoint{5.216114in}{0.569541in}}{\pgfqpoint{5.220505in}{0.580140in}}{\pgfqpoint{5.220505in}{0.591190in}}%
\pgfpathcurveto{\pgfqpoint{5.220505in}{0.602240in}}{\pgfqpoint{5.216114in}{0.612839in}}{\pgfqpoint{5.208301in}{0.620653in}}%
\pgfpathcurveto{\pgfqpoint{5.200487in}{0.628467in}}{\pgfqpoint{5.189888in}{0.632857in}}{\pgfqpoint{5.178838in}{0.632857in}}%
\pgfpathcurveto{\pgfqpoint{5.167788in}{0.632857in}}{\pgfqpoint{5.157189in}{0.628467in}}{\pgfqpoint{5.149375in}{0.620653in}}%
\pgfpathcurveto{\pgfqpoint{5.141562in}{0.612839in}}{\pgfqpoint{5.137171in}{0.602240in}}{\pgfqpoint{5.137171in}{0.591190in}}%
\pgfpathcurveto{\pgfqpoint{5.137171in}{0.580140in}}{\pgfqpoint{5.141562in}{0.569541in}}{\pgfqpoint{5.149375in}{0.561727in}}%
\pgfpathcurveto{\pgfqpoint{5.157189in}{0.553914in}}{\pgfqpoint{5.167788in}{0.549524in}}{\pgfqpoint{5.178838in}{0.549524in}}%
\pgfpathlineto{\pgfqpoint{5.178838in}{0.549524in}}%
\pgfpathclose%
\pgfusepath{stroke}%
\end{pgfscope}%
\begin{pgfscope}%
\pgfpathrectangle{\pgfqpoint{0.494722in}{0.437222in}}{\pgfqpoint{6.275590in}{5.159444in}}%
\pgfusepath{clip}%
\pgfsetbuttcap%
\pgfsetroundjoin%
\pgfsetlinewidth{1.003750pt}%
\definecolor{currentstroke}{rgb}{0.827451,0.827451,0.827451}%
\pgfsetstrokecolor{currentstroke}%
\pgfsetstrokeopacity{0.800000}%
\pgfsetdash{}{0pt}%
\pgfpathmoveto{\pgfqpoint{1.890283in}{1.888988in}}%
\pgfpathcurveto{\pgfqpoint{1.901334in}{1.888988in}}{\pgfqpoint{1.911933in}{1.893378in}}{\pgfqpoint{1.919746in}{1.901192in}}%
\pgfpathcurveto{\pgfqpoint{1.927560in}{1.909005in}}{\pgfqpoint{1.931950in}{1.919604in}}{\pgfqpoint{1.931950in}{1.930654in}}%
\pgfpathcurveto{\pgfqpoint{1.931950in}{1.941705in}}{\pgfqpoint{1.927560in}{1.952304in}}{\pgfqpoint{1.919746in}{1.960117in}}%
\pgfpathcurveto{\pgfqpoint{1.911933in}{1.967931in}}{\pgfqpoint{1.901334in}{1.972321in}}{\pgfqpoint{1.890283in}{1.972321in}}%
\pgfpathcurveto{\pgfqpoint{1.879233in}{1.972321in}}{\pgfqpoint{1.868634in}{1.967931in}}{\pgfqpoint{1.860821in}{1.960117in}}%
\pgfpathcurveto{\pgfqpoint{1.853007in}{1.952304in}}{\pgfqpoint{1.848617in}{1.941705in}}{\pgfqpoint{1.848617in}{1.930654in}}%
\pgfpathcurveto{\pgfqpoint{1.848617in}{1.919604in}}{\pgfqpoint{1.853007in}{1.909005in}}{\pgfqpoint{1.860821in}{1.901192in}}%
\pgfpathcurveto{\pgfqpoint{1.868634in}{1.893378in}}{\pgfqpoint{1.879233in}{1.888988in}}{\pgfqpoint{1.890283in}{1.888988in}}%
\pgfpathlineto{\pgfqpoint{1.890283in}{1.888988in}}%
\pgfpathclose%
\pgfusepath{stroke}%
\end{pgfscope}%
\begin{pgfscope}%
\pgfpathrectangle{\pgfqpoint{0.494722in}{0.437222in}}{\pgfqpoint{6.275590in}{5.159444in}}%
\pgfusepath{clip}%
\pgfsetbuttcap%
\pgfsetroundjoin%
\pgfsetlinewidth{1.003750pt}%
\definecolor{currentstroke}{rgb}{0.827451,0.827451,0.827451}%
\pgfsetstrokecolor{currentstroke}%
\pgfsetstrokeopacity{0.800000}%
\pgfsetdash{}{0pt}%
\pgfpathmoveto{\pgfqpoint{2.065113in}{1.810415in}}%
\pgfpathcurveto{\pgfqpoint{2.076164in}{1.810415in}}{\pgfqpoint{2.086763in}{1.814805in}}{\pgfqpoint{2.094576in}{1.822619in}}%
\pgfpathcurveto{\pgfqpoint{2.102390in}{1.830433in}}{\pgfqpoint{2.106780in}{1.841032in}}{\pgfqpoint{2.106780in}{1.852082in}}%
\pgfpathcurveto{\pgfqpoint{2.106780in}{1.863132in}}{\pgfqpoint{2.102390in}{1.873731in}}{\pgfqpoint{2.094576in}{1.881545in}}%
\pgfpathcurveto{\pgfqpoint{2.086763in}{1.889358in}}{\pgfqpoint{2.076164in}{1.893748in}}{\pgfqpoint{2.065113in}{1.893748in}}%
\pgfpathcurveto{\pgfqpoint{2.054063in}{1.893748in}}{\pgfqpoint{2.043464in}{1.889358in}}{\pgfqpoint{2.035651in}{1.881545in}}%
\pgfpathcurveto{\pgfqpoint{2.027837in}{1.873731in}}{\pgfqpoint{2.023447in}{1.863132in}}{\pgfqpoint{2.023447in}{1.852082in}}%
\pgfpathcurveto{\pgfqpoint{2.023447in}{1.841032in}}{\pgfqpoint{2.027837in}{1.830433in}}{\pgfqpoint{2.035651in}{1.822619in}}%
\pgfpathcurveto{\pgfqpoint{2.043464in}{1.814805in}}{\pgfqpoint{2.054063in}{1.810415in}}{\pgfqpoint{2.065113in}{1.810415in}}%
\pgfpathlineto{\pgfqpoint{2.065113in}{1.810415in}}%
\pgfpathclose%
\pgfusepath{stroke}%
\end{pgfscope}%
\begin{pgfscope}%
\pgfpathrectangle{\pgfqpoint{0.494722in}{0.437222in}}{\pgfqpoint{6.275590in}{5.159444in}}%
\pgfusepath{clip}%
\pgfsetbuttcap%
\pgfsetroundjoin%
\pgfsetlinewidth{1.003750pt}%
\definecolor{currentstroke}{rgb}{0.827451,0.827451,0.827451}%
\pgfsetstrokecolor{currentstroke}%
\pgfsetstrokeopacity{0.800000}%
\pgfsetdash{}{0pt}%
\pgfpathmoveto{\pgfqpoint{2.985789in}{1.750025in}}%
\pgfpathcurveto{\pgfqpoint{2.996839in}{1.750025in}}{\pgfqpoint{3.007438in}{1.754415in}}{\pgfqpoint{3.015252in}{1.762229in}}%
\pgfpathcurveto{\pgfqpoint{3.023065in}{1.770042in}}{\pgfqpoint{3.027456in}{1.780641in}}{\pgfqpoint{3.027456in}{1.791691in}}%
\pgfpathcurveto{\pgfqpoint{3.027456in}{1.802742in}}{\pgfqpoint{3.023065in}{1.813341in}}{\pgfqpoint{3.015252in}{1.821154in}}%
\pgfpathcurveto{\pgfqpoint{3.007438in}{1.828968in}}{\pgfqpoint{2.996839in}{1.833358in}}{\pgfqpoint{2.985789in}{1.833358in}}%
\pgfpathcurveto{\pgfqpoint{2.974739in}{1.833358in}}{\pgfqpoint{2.964140in}{1.828968in}}{\pgfqpoint{2.956326in}{1.821154in}}%
\pgfpathcurveto{\pgfqpoint{2.948513in}{1.813341in}}{\pgfqpoint{2.944122in}{1.802742in}}{\pgfqpoint{2.944122in}{1.791691in}}%
\pgfpathcurveto{\pgfqpoint{2.944122in}{1.780641in}}{\pgfqpoint{2.948513in}{1.770042in}}{\pgfqpoint{2.956326in}{1.762229in}}%
\pgfpathcurveto{\pgfqpoint{2.964140in}{1.754415in}}{\pgfqpoint{2.974739in}{1.750025in}}{\pgfqpoint{2.985789in}{1.750025in}}%
\pgfpathlineto{\pgfqpoint{2.985789in}{1.750025in}}%
\pgfpathclose%
\pgfusepath{stroke}%
\end{pgfscope}%
\begin{pgfscope}%
\pgfpathrectangle{\pgfqpoint{0.494722in}{0.437222in}}{\pgfqpoint{6.275590in}{5.159444in}}%
\pgfusepath{clip}%
\pgfsetbuttcap%
\pgfsetroundjoin%
\pgfsetlinewidth{1.003750pt}%
\definecolor{currentstroke}{rgb}{0.827451,0.827451,0.827451}%
\pgfsetstrokecolor{currentstroke}%
\pgfsetstrokeopacity{0.800000}%
\pgfsetdash{}{0pt}%
\pgfpathmoveto{\pgfqpoint{3.424608in}{1.141705in}}%
\pgfpathcurveto{\pgfqpoint{3.435658in}{1.141705in}}{\pgfqpoint{3.446257in}{1.146096in}}{\pgfqpoint{3.454071in}{1.153909in}}%
\pgfpathcurveto{\pgfqpoint{3.461885in}{1.161723in}}{\pgfqpoint{3.466275in}{1.172322in}}{\pgfqpoint{3.466275in}{1.183372in}}%
\pgfpathcurveto{\pgfqpoint{3.466275in}{1.194422in}}{\pgfqpoint{3.461885in}{1.205021in}}{\pgfqpoint{3.454071in}{1.212835in}}%
\pgfpathcurveto{\pgfqpoint{3.446257in}{1.220649in}}{\pgfqpoint{3.435658in}{1.225039in}}{\pgfqpoint{3.424608in}{1.225039in}}%
\pgfpathcurveto{\pgfqpoint{3.413558in}{1.225039in}}{\pgfqpoint{3.402959in}{1.220649in}}{\pgfqpoint{3.395145in}{1.212835in}}%
\pgfpathcurveto{\pgfqpoint{3.387332in}{1.205021in}}{\pgfqpoint{3.382941in}{1.194422in}}{\pgfqpoint{3.382941in}{1.183372in}}%
\pgfpathcurveto{\pgfqpoint{3.382941in}{1.172322in}}{\pgfqpoint{3.387332in}{1.161723in}}{\pgfqpoint{3.395145in}{1.153909in}}%
\pgfpathcurveto{\pgfqpoint{3.402959in}{1.146096in}}{\pgfqpoint{3.413558in}{1.141705in}}{\pgfqpoint{3.424608in}{1.141705in}}%
\pgfpathlineto{\pgfqpoint{3.424608in}{1.141705in}}%
\pgfpathclose%
\pgfusepath{stroke}%
\end{pgfscope}%
\begin{pgfscope}%
\pgfpathrectangle{\pgfqpoint{0.494722in}{0.437222in}}{\pgfqpoint{6.275590in}{5.159444in}}%
\pgfusepath{clip}%
\pgfsetbuttcap%
\pgfsetroundjoin%
\pgfsetlinewidth{1.003750pt}%
\definecolor{currentstroke}{rgb}{0.827451,0.827451,0.827451}%
\pgfsetstrokecolor{currentstroke}%
\pgfsetstrokeopacity{0.800000}%
\pgfsetdash{}{0pt}%
\pgfpathmoveto{\pgfqpoint{4.509124in}{0.892065in}}%
\pgfpathcurveto{\pgfqpoint{4.520174in}{0.892065in}}{\pgfqpoint{4.530773in}{0.896455in}}{\pgfqpoint{4.538586in}{0.904269in}}%
\pgfpathcurveto{\pgfqpoint{4.546400in}{0.912082in}}{\pgfqpoint{4.550790in}{0.922681in}}{\pgfqpoint{4.550790in}{0.933732in}}%
\pgfpathcurveto{\pgfqpoint{4.550790in}{0.944782in}}{\pgfqpoint{4.546400in}{0.955381in}}{\pgfqpoint{4.538586in}{0.963194in}}%
\pgfpathcurveto{\pgfqpoint{4.530773in}{0.971008in}}{\pgfqpoint{4.520174in}{0.975398in}}{\pgfqpoint{4.509124in}{0.975398in}}%
\pgfpathcurveto{\pgfqpoint{4.498073in}{0.975398in}}{\pgfqpoint{4.487474in}{0.971008in}}{\pgfqpoint{4.479661in}{0.963194in}}%
\pgfpathcurveto{\pgfqpoint{4.471847in}{0.955381in}}{\pgfqpoint{4.467457in}{0.944782in}}{\pgfqpoint{4.467457in}{0.933732in}}%
\pgfpathcurveto{\pgfqpoint{4.467457in}{0.922681in}}{\pgfqpoint{4.471847in}{0.912082in}}{\pgfqpoint{4.479661in}{0.904269in}}%
\pgfpathcurveto{\pgfqpoint{4.487474in}{0.896455in}}{\pgfqpoint{4.498073in}{0.892065in}}{\pgfqpoint{4.509124in}{0.892065in}}%
\pgfpathlineto{\pgfqpoint{4.509124in}{0.892065in}}%
\pgfpathclose%
\pgfusepath{stroke}%
\end{pgfscope}%
\begin{pgfscope}%
\pgfpathrectangle{\pgfqpoint{0.494722in}{0.437222in}}{\pgfqpoint{6.275590in}{5.159444in}}%
\pgfusepath{clip}%
\pgfsetbuttcap%
\pgfsetroundjoin%
\pgfsetlinewidth{1.003750pt}%
\definecolor{currentstroke}{rgb}{0.827451,0.827451,0.827451}%
\pgfsetstrokecolor{currentstroke}%
\pgfsetstrokeopacity{0.800000}%
\pgfsetdash{}{0pt}%
\pgfpathmoveto{\pgfqpoint{4.354384in}{0.940900in}}%
\pgfpathcurveto{\pgfqpoint{4.365435in}{0.940900in}}{\pgfqpoint{4.376034in}{0.945290in}}{\pgfqpoint{4.383847in}{0.953104in}}%
\pgfpathcurveto{\pgfqpoint{4.391661in}{0.960918in}}{\pgfqpoint{4.396051in}{0.971517in}}{\pgfqpoint{4.396051in}{0.982567in}}%
\pgfpathcurveto{\pgfqpoint{4.396051in}{0.993617in}}{\pgfqpoint{4.391661in}{1.004216in}}{\pgfqpoint{4.383847in}{1.012030in}}%
\pgfpathcurveto{\pgfqpoint{4.376034in}{1.019843in}}{\pgfqpoint{4.365435in}{1.024234in}}{\pgfqpoint{4.354384in}{1.024234in}}%
\pgfpathcurveto{\pgfqpoint{4.343334in}{1.024234in}}{\pgfqpoint{4.332735in}{1.019843in}}{\pgfqpoint{4.324922in}{1.012030in}}%
\pgfpathcurveto{\pgfqpoint{4.317108in}{1.004216in}}{\pgfqpoint{4.312718in}{0.993617in}}{\pgfqpoint{4.312718in}{0.982567in}}%
\pgfpathcurveto{\pgfqpoint{4.312718in}{0.971517in}}{\pgfqpoint{4.317108in}{0.960918in}}{\pgfqpoint{4.324922in}{0.953104in}}%
\pgfpathcurveto{\pgfqpoint{4.332735in}{0.945290in}}{\pgfqpoint{4.343334in}{0.940900in}}{\pgfqpoint{4.354384in}{0.940900in}}%
\pgfpathlineto{\pgfqpoint{4.354384in}{0.940900in}}%
\pgfpathclose%
\pgfusepath{stroke}%
\end{pgfscope}%
\begin{pgfscope}%
\pgfpathrectangle{\pgfqpoint{0.494722in}{0.437222in}}{\pgfqpoint{6.275590in}{5.159444in}}%
\pgfusepath{clip}%
\pgfsetbuttcap%
\pgfsetroundjoin%
\pgfsetlinewidth{1.003750pt}%
\definecolor{currentstroke}{rgb}{0.827451,0.827451,0.827451}%
\pgfsetstrokecolor{currentstroke}%
\pgfsetstrokeopacity{0.800000}%
\pgfsetdash{}{0pt}%
\pgfpathmoveto{\pgfqpoint{4.849115in}{0.855862in}}%
\pgfpathcurveto{\pgfqpoint{4.860165in}{0.855862in}}{\pgfqpoint{4.870764in}{0.860252in}}{\pgfqpoint{4.878578in}{0.868066in}}%
\pgfpathcurveto{\pgfqpoint{4.886391in}{0.875879in}}{\pgfqpoint{4.890781in}{0.886478in}}{\pgfqpoint{4.890781in}{0.897528in}}%
\pgfpathcurveto{\pgfqpoint{4.890781in}{0.908579in}}{\pgfqpoint{4.886391in}{0.919178in}}{\pgfqpoint{4.878578in}{0.926991in}}%
\pgfpathcurveto{\pgfqpoint{4.870764in}{0.934805in}}{\pgfqpoint{4.860165in}{0.939195in}}{\pgfqpoint{4.849115in}{0.939195in}}%
\pgfpathcurveto{\pgfqpoint{4.838065in}{0.939195in}}{\pgfqpoint{4.827466in}{0.934805in}}{\pgfqpoint{4.819652in}{0.926991in}}%
\pgfpathcurveto{\pgfqpoint{4.811838in}{0.919178in}}{\pgfqpoint{4.807448in}{0.908579in}}{\pgfqpoint{4.807448in}{0.897528in}}%
\pgfpathcurveto{\pgfqpoint{4.807448in}{0.886478in}}{\pgfqpoint{4.811838in}{0.875879in}}{\pgfqpoint{4.819652in}{0.868066in}}%
\pgfpathcurveto{\pgfqpoint{4.827466in}{0.860252in}}{\pgfqpoint{4.838065in}{0.855862in}}{\pgfqpoint{4.849115in}{0.855862in}}%
\pgfpathlineto{\pgfqpoint{4.849115in}{0.855862in}}%
\pgfpathclose%
\pgfusepath{stroke}%
\end{pgfscope}%
\begin{pgfscope}%
\pgfpathrectangle{\pgfqpoint{0.494722in}{0.437222in}}{\pgfqpoint{6.275590in}{5.159444in}}%
\pgfusepath{clip}%
\pgfsetbuttcap%
\pgfsetroundjoin%
\pgfsetlinewidth{1.003750pt}%
\definecolor{currentstroke}{rgb}{0.827451,0.827451,0.827451}%
\pgfsetstrokecolor{currentstroke}%
\pgfsetstrokeopacity{0.800000}%
\pgfsetdash{}{0pt}%
\pgfpathmoveto{\pgfqpoint{1.445308in}{2.448646in}}%
\pgfpathcurveto{\pgfqpoint{1.456358in}{2.448646in}}{\pgfqpoint{1.466957in}{2.453036in}}{\pgfqpoint{1.474771in}{2.460849in}}%
\pgfpathcurveto{\pgfqpoint{1.482584in}{2.468663in}}{\pgfqpoint{1.486975in}{2.479262in}}{\pgfqpoint{1.486975in}{2.490312in}}%
\pgfpathcurveto{\pgfqpoint{1.486975in}{2.501362in}}{\pgfqpoint{1.482584in}{2.511961in}}{\pgfqpoint{1.474771in}{2.519775in}}%
\pgfpathcurveto{\pgfqpoint{1.466957in}{2.527589in}}{\pgfqpoint{1.456358in}{2.531979in}}{\pgfqpoint{1.445308in}{2.531979in}}%
\pgfpathcurveto{\pgfqpoint{1.434258in}{2.531979in}}{\pgfqpoint{1.423659in}{2.527589in}}{\pgfqpoint{1.415845in}{2.519775in}}%
\pgfpathcurveto{\pgfqpoint{1.408032in}{2.511961in}}{\pgfqpoint{1.403641in}{2.501362in}}{\pgfqpoint{1.403641in}{2.490312in}}%
\pgfpathcurveto{\pgfqpoint{1.403641in}{2.479262in}}{\pgfqpoint{1.408032in}{2.468663in}}{\pgfqpoint{1.415845in}{2.460849in}}%
\pgfpathcurveto{\pgfqpoint{1.423659in}{2.453036in}}{\pgfqpoint{1.434258in}{2.448646in}}{\pgfqpoint{1.445308in}{2.448646in}}%
\pgfpathlineto{\pgfqpoint{1.445308in}{2.448646in}}%
\pgfpathclose%
\pgfusepath{stroke}%
\end{pgfscope}%
\begin{pgfscope}%
\pgfpathrectangle{\pgfqpoint{0.494722in}{0.437222in}}{\pgfqpoint{6.275590in}{5.159444in}}%
\pgfusepath{clip}%
\pgfsetbuttcap%
\pgfsetroundjoin%
\pgfsetlinewidth{1.003750pt}%
\definecolor{currentstroke}{rgb}{0.827451,0.827451,0.827451}%
\pgfsetstrokecolor{currentstroke}%
\pgfsetstrokeopacity{0.800000}%
\pgfsetdash{}{0pt}%
\pgfpathmoveto{\pgfqpoint{1.383380in}{2.686790in}}%
\pgfpathcurveto{\pgfqpoint{1.394430in}{2.686790in}}{\pgfqpoint{1.405029in}{2.691180in}}{\pgfqpoint{1.412843in}{2.698994in}}%
\pgfpathcurveto{\pgfqpoint{1.420656in}{2.706807in}}{\pgfqpoint{1.425046in}{2.717406in}}{\pgfqpoint{1.425046in}{2.728456in}}%
\pgfpathcurveto{\pgfqpoint{1.425046in}{2.739507in}}{\pgfqpoint{1.420656in}{2.750106in}}{\pgfqpoint{1.412843in}{2.757919in}}%
\pgfpathcurveto{\pgfqpoint{1.405029in}{2.765733in}}{\pgfqpoint{1.394430in}{2.770123in}}{\pgfqpoint{1.383380in}{2.770123in}}%
\pgfpathcurveto{\pgfqpoint{1.372330in}{2.770123in}}{\pgfqpoint{1.361731in}{2.765733in}}{\pgfqpoint{1.353917in}{2.757919in}}%
\pgfpathcurveto{\pgfqpoint{1.346103in}{2.750106in}}{\pgfqpoint{1.341713in}{2.739507in}}{\pgfqpoint{1.341713in}{2.728456in}}%
\pgfpathcurveto{\pgfqpoint{1.341713in}{2.717406in}}{\pgfqpoint{1.346103in}{2.706807in}}{\pgfqpoint{1.353917in}{2.698994in}}%
\pgfpathcurveto{\pgfqpoint{1.361731in}{2.691180in}}{\pgfqpoint{1.372330in}{2.686790in}}{\pgfqpoint{1.383380in}{2.686790in}}%
\pgfpathlineto{\pgfqpoint{1.383380in}{2.686790in}}%
\pgfpathclose%
\pgfusepath{stroke}%
\end{pgfscope}%
\begin{pgfscope}%
\pgfpathrectangle{\pgfqpoint{0.494722in}{0.437222in}}{\pgfqpoint{6.275590in}{5.159444in}}%
\pgfusepath{clip}%
\pgfsetbuttcap%
\pgfsetroundjoin%
\pgfsetlinewidth{1.003750pt}%
\definecolor{currentstroke}{rgb}{0.827451,0.827451,0.827451}%
\pgfsetstrokecolor{currentstroke}%
\pgfsetstrokeopacity{0.800000}%
\pgfsetdash{}{0pt}%
\pgfpathmoveto{\pgfqpoint{5.313102in}{0.623346in}}%
\pgfpathcurveto{\pgfqpoint{5.324152in}{0.623346in}}{\pgfqpoint{5.334751in}{0.627736in}}{\pgfqpoint{5.342565in}{0.635550in}}%
\pgfpathcurveto{\pgfqpoint{5.350378in}{0.643363in}}{\pgfqpoint{5.354769in}{0.653962in}}{\pgfqpoint{5.354769in}{0.665012in}}%
\pgfpathcurveto{\pgfqpoint{5.354769in}{0.676063in}}{\pgfqpoint{5.350378in}{0.686662in}}{\pgfqpoint{5.342565in}{0.694475in}}%
\pgfpathcurveto{\pgfqpoint{5.334751in}{0.702289in}}{\pgfqpoint{5.324152in}{0.706679in}}{\pgfqpoint{5.313102in}{0.706679in}}%
\pgfpathcurveto{\pgfqpoint{5.302052in}{0.706679in}}{\pgfqpoint{5.291453in}{0.702289in}}{\pgfqpoint{5.283639in}{0.694475in}}%
\pgfpathcurveto{\pgfqpoint{5.275825in}{0.686662in}}{\pgfqpoint{5.271435in}{0.676063in}}{\pgfqpoint{5.271435in}{0.665012in}}%
\pgfpathcurveto{\pgfqpoint{5.271435in}{0.653962in}}{\pgfqpoint{5.275825in}{0.643363in}}{\pgfqpoint{5.283639in}{0.635550in}}%
\pgfpathcurveto{\pgfqpoint{5.291453in}{0.627736in}}{\pgfqpoint{5.302052in}{0.623346in}}{\pgfqpoint{5.313102in}{0.623346in}}%
\pgfpathlineto{\pgfqpoint{5.313102in}{0.623346in}}%
\pgfpathclose%
\pgfusepath{stroke}%
\end{pgfscope}%
\begin{pgfscope}%
\pgfpathrectangle{\pgfqpoint{0.494722in}{0.437222in}}{\pgfqpoint{6.275590in}{5.159444in}}%
\pgfusepath{clip}%
\pgfsetbuttcap%
\pgfsetroundjoin%
\pgfsetlinewidth{1.003750pt}%
\definecolor{currentstroke}{rgb}{0.827451,0.827451,0.827451}%
\pgfsetstrokecolor{currentstroke}%
\pgfsetstrokeopacity{0.800000}%
\pgfsetdash{}{0pt}%
\pgfpathmoveto{\pgfqpoint{1.942213in}{2.414771in}}%
\pgfpathcurveto{\pgfqpoint{1.953263in}{2.414771in}}{\pgfqpoint{1.963862in}{2.419162in}}{\pgfqpoint{1.971676in}{2.426975in}}%
\pgfpathcurveto{\pgfqpoint{1.979489in}{2.434789in}}{\pgfqpoint{1.983880in}{2.445388in}}{\pgfqpoint{1.983880in}{2.456438in}}%
\pgfpathcurveto{\pgfqpoint{1.983880in}{2.467488in}}{\pgfqpoint{1.979489in}{2.478087in}}{\pgfqpoint{1.971676in}{2.485901in}}%
\pgfpathcurveto{\pgfqpoint{1.963862in}{2.493715in}}{\pgfqpoint{1.953263in}{2.498105in}}{\pgfqpoint{1.942213in}{2.498105in}}%
\pgfpathcurveto{\pgfqpoint{1.931163in}{2.498105in}}{\pgfqpoint{1.920564in}{2.493715in}}{\pgfqpoint{1.912750in}{2.485901in}}%
\pgfpathcurveto{\pgfqpoint{1.904937in}{2.478087in}}{\pgfqpoint{1.900546in}{2.467488in}}{\pgfqpoint{1.900546in}{2.456438in}}%
\pgfpathcurveto{\pgfqpoint{1.900546in}{2.445388in}}{\pgfqpoint{1.904937in}{2.434789in}}{\pgfqpoint{1.912750in}{2.426975in}}%
\pgfpathcurveto{\pgfqpoint{1.920564in}{2.419162in}}{\pgfqpoint{1.931163in}{2.414771in}}{\pgfqpoint{1.942213in}{2.414771in}}%
\pgfpathlineto{\pgfqpoint{1.942213in}{2.414771in}}%
\pgfpathclose%
\pgfusepath{stroke}%
\end{pgfscope}%
\begin{pgfscope}%
\pgfpathrectangle{\pgfqpoint{0.494722in}{0.437222in}}{\pgfqpoint{6.275590in}{5.159444in}}%
\pgfusepath{clip}%
\pgfsetbuttcap%
\pgfsetroundjoin%
\pgfsetlinewidth{1.003750pt}%
\definecolor{currentstroke}{rgb}{0.827451,0.827451,0.827451}%
\pgfsetstrokecolor{currentstroke}%
\pgfsetstrokeopacity{0.800000}%
\pgfsetdash{}{0pt}%
\pgfpathmoveto{\pgfqpoint{2.481396in}{2.133918in}}%
\pgfpathcurveto{\pgfqpoint{2.492446in}{2.133918in}}{\pgfqpoint{2.503045in}{2.138308in}}{\pgfqpoint{2.510859in}{2.146121in}}%
\pgfpathcurveto{\pgfqpoint{2.518672in}{2.153935in}}{\pgfqpoint{2.523063in}{2.164534in}}{\pgfqpoint{2.523063in}{2.175584in}}%
\pgfpathcurveto{\pgfqpoint{2.523063in}{2.186634in}}{\pgfqpoint{2.518672in}{2.197233in}}{\pgfqpoint{2.510859in}{2.205047in}}%
\pgfpathcurveto{\pgfqpoint{2.503045in}{2.212861in}}{\pgfqpoint{2.492446in}{2.217251in}}{\pgfqpoint{2.481396in}{2.217251in}}%
\pgfpathcurveto{\pgfqpoint{2.470346in}{2.217251in}}{\pgfqpoint{2.459747in}{2.212861in}}{\pgfqpoint{2.451933in}{2.205047in}}%
\pgfpathcurveto{\pgfqpoint{2.444120in}{2.197233in}}{\pgfqpoint{2.439729in}{2.186634in}}{\pgfqpoint{2.439729in}{2.175584in}}%
\pgfpathcurveto{\pgfqpoint{2.439729in}{2.164534in}}{\pgfqpoint{2.444120in}{2.153935in}}{\pgfqpoint{2.451933in}{2.146121in}}%
\pgfpathcurveto{\pgfqpoint{2.459747in}{2.138308in}}{\pgfqpoint{2.470346in}{2.133918in}}{\pgfqpoint{2.481396in}{2.133918in}}%
\pgfpathlineto{\pgfqpoint{2.481396in}{2.133918in}}%
\pgfpathclose%
\pgfusepath{stroke}%
\end{pgfscope}%
\begin{pgfscope}%
\pgfpathrectangle{\pgfqpoint{0.494722in}{0.437222in}}{\pgfqpoint{6.275590in}{5.159444in}}%
\pgfusepath{clip}%
\pgfsetbuttcap%
\pgfsetroundjoin%
\pgfsetlinewidth{1.003750pt}%
\definecolor{currentstroke}{rgb}{0.827451,0.827451,0.827451}%
\pgfsetstrokecolor{currentstroke}%
\pgfsetstrokeopacity{0.800000}%
\pgfsetdash{}{0pt}%
\pgfpathmoveto{\pgfqpoint{2.604544in}{2.123353in}}%
\pgfpathcurveto{\pgfqpoint{2.615594in}{2.123353in}}{\pgfqpoint{2.626193in}{2.127743in}}{\pgfqpoint{2.634007in}{2.135556in}}%
\pgfpathcurveto{\pgfqpoint{2.641821in}{2.143370in}}{\pgfqpoint{2.646211in}{2.153969in}}{\pgfqpoint{2.646211in}{2.165019in}}%
\pgfpathcurveto{\pgfqpoint{2.646211in}{2.176069in}}{\pgfqpoint{2.641821in}{2.186668in}}{\pgfqpoint{2.634007in}{2.194482in}}%
\pgfpathcurveto{\pgfqpoint{2.626193in}{2.202296in}}{\pgfqpoint{2.615594in}{2.206686in}}{\pgfqpoint{2.604544in}{2.206686in}}%
\pgfpathcurveto{\pgfqpoint{2.593494in}{2.206686in}}{\pgfqpoint{2.582895in}{2.202296in}}{\pgfqpoint{2.575081in}{2.194482in}}%
\pgfpathcurveto{\pgfqpoint{2.567268in}{2.186668in}}{\pgfqpoint{2.562877in}{2.176069in}}{\pgfqpoint{2.562877in}{2.165019in}}%
\pgfpathcurveto{\pgfqpoint{2.562877in}{2.153969in}}{\pgfqpoint{2.567268in}{2.143370in}}{\pgfqpoint{2.575081in}{2.135556in}}%
\pgfpathcurveto{\pgfqpoint{2.582895in}{2.127743in}}{\pgfqpoint{2.593494in}{2.123353in}}{\pgfqpoint{2.604544in}{2.123353in}}%
\pgfpathlineto{\pgfqpoint{2.604544in}{2.123353in}}%
\pgfpathclose%
\pgfusepath{stroke}%
\end{pgfscope}%
\begin{pgfscope}%
\pgfpathrectangle{\pgfqpoint{0.494722in}{0.437222in}}{\pgfqpoint{6.275590in}{5.159444in}}%
\pgfusepath{clip}%
\pgfsetbuttcap%
\pgfsetroundjoin%
\pgfsetlinewidth{1.003750pt}%
\definecolor{currentstroke}{rgb}{0.827451,0.827451,0.827451}%
\pgfsetstrokecolor{currentstroke}%
\pgfsetstrokeopacity{0.800000}%
\pgfsetdash{}{0pt}%
\pgfpathmoveto{\pgfqpoint{2.128187in}{2.291777in}}%
\pgfpathcurveto{\pgfqpoint{2.139237in}{2.291777in}}{\pgfqpoint{2.149836in}{2.296167in}}{\pgfqpoint{2.157650in}{2.303981in}}%
\pgfpathcurveto{\pgfqpoint{2.165463in}{2.311794in}}{\pgfqpoint{2.169853in}{2.322393in}}{\pgfqpoint{2.169853in}{2.333444in}}%
\pgfpathcurveto{\pgfqpoint{2.169853in}{2.344494in}}{\pgfqpoint{2.165463in}{2.355093in}}{\pgfqpoint{2.157650in}{2.362906in}}%
\pgfpathcurveto{\pgfqpoint{2.149836in}{2.370720in}}{\pgfqpoint{2.139237in}{2.375110in}}{\pgfqpoint{2.128187in}{2.375110in}}%
\pgfpathcurveto{\pgfqpoint{2.117137in}{2.375110in}}{\pgfqpoint{2.106538in}{2.370720in}}{\pgfqpoint{2.098724in}{2.362906in}}%
\pgfpathcurveto{\pgfqpoint{2.090910in}{2.355093in}}{\pgfqpoint{2.086520in}{2.344494in}}{\pgfqpoint{2.086520in}{2.333444in}}%
\pgfpathcurveto{\pgfqpoint{2.086520in}{2.322393in}}{\pgfqpoint{2.090910in}{2.311794in}}{\pgfqpoint{2.098724in}{2.303981in}}%
\pgfpathcurveto{\pgfqpoint{2.106538in}{2.296167in}}{\pgfqpoint{2.117137in}{2.291777in}}{\pgfqpoint{2.128187in}{2.291777in}}%
\pgfpathlineto{\pgfqpoint{2.128187in}{2.291777in}}%
\pgfpathclose%
\pgfusepath{stroke}%
\end{pgfscope}%
\begin{pgfscope}%
\pgfpathrectangle{\pgfqpoint{0.494722in}{0.437222in}}{\pgfqpoint{6.275590in}{5.159444in}}%
\pgfusepath{clip}%
\pgfsetbuttcap%
\pgfsetroundjoin%
\pgfsetlinewidth{1.003750pt}%
\definecolor{currentstroke}{rgb}{0.827451,0.827451,0.827451}%
\pgfsetstrokecolor{currentstroke}%
\pgfsetstrokeopacity{0.800000}%
\pgfsetdash{}{0pt}%
\pgfpathmoveto{\pgfqpoint{3.047639in}{2.040281in}}%
\pgfpathcurveto{\pgfqpoint{3.058689in}{2.040281in}}{\pgfqpoint{3.069288in}{2.044671in}}{\pgfqpoint{3.077102in}{2.052485in}}%
\pgfpathcurveto{\pgfqpoint{3.084915in}{2.060298in}}{\pgfqpoint{3.089305in}{2.070897in}}{\pgfqpoint{3.089305in}{2.081948in}}%
\pgfpathcurveto{\pgfqpoint{3.089305in}{2.092998in}}{\pgfqpoint{3.084915in}{2.103597in}}{\pgfqpoint{3.077102in}{2.111410in}}%
\pgfpathcurveto{\pgfqpoint{3.069288in}{2.119224in}}{\pgfqpoint{3.058689in}{2.123614in}}{\pgfqpoint{3.047639in}{2.123614in}}%
\pgfpathcurveto{\pgfqpoint{3.036589in}{2.123614in}}{\pgfqpoint{3.025990in}{2.119224in}}{\pgfqpoint{3.018176in}{2.111410in}}%
\pgfpathcurveto{\pgfqpoint{3.010362in}{2.103597in}}{\pgfqpoint{3.005972in}{2.092998in}}{\pgfqpoint{3.005972in}{2.081948in}}%
\pgfpathcurveto{\pgfqpoint{3.005972in}{2.070897in}}{\pgfqpoint{3.010362in}{2.060298in}}{\pgfqpoint{3.018176in}{2.052485in}}%
\pgfpathcurveto{\pgfqpoint{3.025990in}{2.044671in}}{\pgfqpoint{3.036589in}{2.040281in}}{\pgfqpoint{3.047639in}{2.040281in}}%
\pgfpathlineto{\pgfqpoint{3.047639in}{2.040281in}}%
\pgfpathclose%
\pgfusepath{stroke}%
\end{pgfscope}%
\begin{pgfscope}%
\pgfpathrectangle{\pgfqpoint{0.494722in}{0.437222in}}{\pgfqpoint{6.275590in}{5.159444in}}%
\pgfusepath{clip}%
\pgfsetbuttcap%
\pgfsetroundjoin%
\pgfsetlinewidth{1.003750pt}%
\definecolor{currentstroke}{rgb}{0.827451,0.827451,0.827451}%
\pgfsetstrokecolor{currentstroke}%
\pgfsetstrokeopacity{0.800000}%
\pgfsetdash{}{0pt}%
\pgfpathmoveto{\pgfqpoint{3.726147in}{1.321972in}}%
\pgfpathcurveto{\pgfqpoint{3.737197in}{1.321972in}}{\pgfqpoint{3.747796in}{1.326362in}}{\pgfqpoint{3.755610in}{1.334176in}}%
\pgfpathcurveto{\pgfqpoint{3.763423in}{1.341989in}}{\pgfqpoint{3.767814in}{1.352588in}}{\pgfqpoint{3.767814in}{1.363639in}}%
\pgfpathcurveto{\pgfqpoint{3.767814in}{1.374689in}}{\pgfqpoint{3.763423in}{1.385288in}}{\pgfqpoint{3.755610in}{1.393101in}}%
\pgfpathcurveto{\pgfqpoint{3.747796in}{1.400915in}}{\pgfqpoint{3.737197in}{1.405305in}}{\pgfqpoint{3.726147in}{1.405305in}}%
\pgfpathcurveto{\pgfqpoint{3.715097in}{1.405305in}}{\pgfqpoint{3.704498in}{1.400915in}}{\pgfqpoint{3.696684in}{1.393101in}}%
\pgfpathcurveto{\pgfqpoint{3.688871in}{1.385288in}}{\pgfqpoint{3.684480in}{1.374689in}}{\pgfqpoint{3.684480in}{1.363639in}}%
\pgfpathcurveto{\pgfqpoint{3.684480in}{1.352588in}}{\pgfqpoint{3.688871in}{1.341989in}}{\pgfqpoint{3.696684in}{1.334176in}}%
\pgfpathcurveto{\pgfqpoint{3.704498in}{1.326362in}}{\pgfqpoint{3.715097in}{1.321972in}}{\pgfqpoint{3.726147in}{1.321972in}}%
\pgfpathlineto{\pgfqpoint{3.726147in}{1.321972in}}%
\pgfpathclose%
\pgfusepath{stroke}%
\end{pgfscope}%
\begin{pgfscope}%
\pgfpathrectangle{\pgfqpoint{0.494722in}{0.437222in}}{\pgfqpoint{6.275590in}{5.159444in}}%
\pgfusepath{clip}%
\pgfsetbuttcap%
\pgfsetroundjoin%
\pgfsetlinewidth{1.003750pt}%
\definecolor{currentstroke}{rgb}{0.827451,0.827451,0.827451}%
\pgfsetstrokecolor{currentstroke}%
\pgfsetstrokeopacity{0.800000}%
\pgfsetdash{}{0pt}%
\pgfpathmoveto{\pgfqpoint{3.470954in}{1.654406in}}%
\pgfpathcurveto{\pgfqpoint{3.482004in}{1.654406in}}{\pgfqpoint{3.492603in}{1.658796in}}{\pgfqpoint{3.500417in}{1.666610in}}%
\pgfpathcurveto{\pgfqpoint{3.508231in}{1.674423in}}{\pgfqpoint{3.512621in}{1.685022in}}{\pgfqpoint{3.512621in}{1.696072in}}%
\pgfpathcurveto{\pgfqpoint{3.512621in}{1.707123in}}{\pgfqpoint{3.508231in}{1.717722in}}{\pgfqpoint{3.500417in}{1.725535in}}%
\pgfpathcurveto{\pgfqpoint{3.492603in}{1.733349in}}{\pgfqpoint{3.482004in}{1.737739in}}{\pgfqpoint{3.470954in}{1.737739in}}%
\pgfpathcurveto{\pgfqpoint{3.459904in}{1.737739in}}{\pgfqpoint{3.449305in}{1.733349in}}{\pgfqpoint{3.441491in}{1.725535in}}%
\pgfpathcurveto{\pgfqpoint{3.433678in}{1.717722in}}{\pgfqpoint{3.429288in}{1.707123in}}{\pgfqpoint{3.429288in}{1.696072in}}%
\pgfpathcurveto{\pgfqpoint{3.429288in}{1.685022in}}{\pgfqpoint{3.433678in}{1.674423in}}{\pgfqpoint{3.441491in}{1.666610in}}%
\pgfpathcurveto{\pgfqpoint{3.449305in}{1.658796in}}{\pgfqpoint{3.459904in}{1.654406in}}{\pgfqpoint{3.470954in}{1.654406in}}%
\pgfpathlineto{\pgfqpoint{3.470954in}{1.654406in}}%
\pgfpathclose%
\pgfusepath{stroke}%
\end{pgfscope}%
\begin{pgfscope}%
\pgfpathrectangle{\pgfqpoint{0.494722in}{0.437222in}}{\pgfqpoint{6.275590in}{5.159444in}}%
\pgfusepath{clip}%
\pgfsetbuttcap%
\pgfsetroundjoin%
\pgfsetlinewidth{1.003750pt}%
\definecolor{currentstroke}{rgb}{0.827451,0.827451,0.827451}%
\pgfsetstrokecolor{currentstroke}%
\pgfsetstrokeopacity{0.800000}%
\pgfsetdash{}{0pt}%
\pgfpathmoveto{\pgfqpoint{4.678377in}{1.049450in}}%
\pgfpathcurveto{\pgfqpoint{4.689427in}{1.049450in}}{\pgfqpoint{4.700026in}{1.053840in}}{\pgfqpoint{4.707840in}{1.061654in}}%
\pgfpathcurveto{\pgfqpoint{4.715653in}{1.069468in}}{\pgfqpoint{4.720043in}{1.080067in}}{\pgfqpoint{4.720043in}{1.091117in}}%
\pgfpathcurveto{\pgfqpoint{4.720043in}{1.102167in}}{\pgfqpoint{4.715653in}{1.112766in}}{\pgfqpoint{4.707840in}{1.120580in}}%
\pgfpathcurveto{\pgfqpoint{4.700026in}{1.128393in}}{\pgfqpoint{4.689427in}{1.132783in}}{\pgfqpoint{4.678377in}{1.132783in}}%
\pgfpathcurveto{\pgfqpoint{4.667327in}{1.132783in}}{\pgfqpoint{4.656728in}{1.128393in}}{\pgfqpoint{4.648914in}{1.120580in}}%
\pgfpathcurveto{\pgfqpoint{4.641100in}{1.112766in}}{\pgfqpoint{4.636710in}{1.102167in}}{\pgfqpoint{4.636710in}{1.091117in}}%
\pgfpathcurveto{\pgfqpoint{4.636710in}{1.080067in}}{\pgfqpoint{4.641100in}{1.069468in}}{\pgfqpoint{4.648914in}{1.061654in}}%
\pgfpathcurveto{\pgfqpoint{4.656728in}{1.053840in}}{\pgfqpoint{4.667327in}{1.049450in}}{\pgfqpoint{4.678377in}{1.049450in}}%
\pgfpathlineto{\pgfqpoint{4.678377in}{1.049450in}}%
\pgfpathclose%
\pgfusepath{stroke}%
\end{pgfscope}%
\begin{pgfscope}%
\pgfpathrectangle{\pgfqpoint{0.494722in}{0.437222in}}{\pgfqpoint{6.275590in}{5.159444in}}%
\pgfusepath{clip}%
\pgfsetbuttcap%
\pgfsetroundjoin%
\pgfsetlinewidth{1.003750pt}%
\definecolor{currentstroke}{rgb}{0.827451,0.827451,0.827451}%
\pgfsetstrokecolor{currentstroke}%
\pgfsetstrokeopacity{0.800000}%
\pgfsetdash{}{0pt}%
\pgfpathmoveto{\pgfqpoint{1.487681in}{2.861200in}}%
\pgfpathcurveto{\pgfqpoint{1.498731in}{2.861200in}}{\pgfqpoint{1.509330in}{2.865590in}}{\pgfqpoint{1.517144in}{2.873404in}}%
\pgfpathcurveto{\pgfqpoint{1.524957in}{2.881217in}}{\pgfqpoint{1.529348in}{2.891816in}}{\pgfqpoint{1.529348in}{2.902866in}}%
\pgfpathcurveto{\pgfqpoint{1.529348in}{2.913917in}}{\pgfqpoint{1.524957in}{2.924516in}}{\pgfqpoint{1.517144in}{2.932329in}}%
\pgfpathcurveto{\pgfqpoint{1.509330in}{2.940143in}}{\pgfqpoint{1.498731in}{2.944533in}}{\pgfqpoint{1.487681in}{2.944533in}}%
\pgfpathcurveto{\pgfqpoint{1.476631in}{2.944533in}}{\pgfqpoint{1.466032in}{2.940143in}}{\pgfqpoint{1.458218in}{2.932329in}}%
\pgfpathcurveto{\pgfqpoint{1.450405in}{2.924516in}}{\pgfqpoint{1.446014in}{2.913917in}}{\pgfqpoint{1.446014in}{2.902866in}}%
\pgfpathcurveto{\pgfqpoint{1.446014in}{2.891816in}}{\pgfqpoint{1.450405in}{2.881217in}}{\pgfqpoint{1.458218in}{2.873404in}}%
\pgfpathcurveto{\pgfqpoint{1.466032in}{2.865590in}}{\pgfqpoint{1.476631in}{2.861200in}}{\pgfqpoint{1.487681in}{2.861200in}}%
\pgfpathlineto{\pgfqpoint{1.487681in}{2.861200in}}%
\pgfpathclose%
\pgfusepath{stroke}%
\end{pgfscope}%
\begin{pgfscope}%
\pgfpathrectangle{\pgfqpoint{0.494722in}{0.437222in}}{\pgfqpoint{6.275590in}{5.159444in}}%
\pgfusepath{clip}%
\pgfsetbuttcap%
\pgfsetroundjoin%
\pgfsetlinewidth{1.003750pt}%
\definecolor{currentstroke}{rgb}{0.827451,0.827451,0.827451}%
\pgfsetstrokecolor{currentstroke}%
\pgfsetstrokeopacity{0.800000}%
\pgfsetdash{}{0pt}%
\pgfpathmoveto{\pgfqpoint{1.751055in}{2.745645in}}%
\pgfpathcurveto{\pgfqpoint{1.762106in}{2.745645in}}{\pgfqpoint{1.772705in}{2.750035in}}{\pgfqpoint{1.780518in}{2.757849in}}%
\pgfpathcurveto{\pgfqpoint{1.788332in}{2.765662in}}{\pgfqpoint{1.792722in}{2.776261in}}{\pgfqpoint{1.792722in}{2.787312in}}%
\pgfpathcurveto{\pgfqpoint{1.792722in}{2.798362in}}{\pgfqpoint{1.788332in}{2.808961in}}{\pgfqpoint{1.780518in}{2.816774in}}%
\pgfpathcurveto{\pgfqpoint{1.772705in}{2.824588in}}{\pgfqpoint{1.762106in}{2.828978in}}{\pgfqpoint{1.751055in}{2.828978in}}%
\pgfpathcurveto{\pgfqpoint{1.740005in}{2.828978in}}{\pgfqpoint{1.729406in}{2.824588in}}{\pgfqpoint{1.721593in}{2.816774in}}%
\pgfpathcurveto{\pgfqpoint{1.713779in}{2.808961in}}{\pgfqpoint{1.709389in}{2.798362in}}{\pgfqpoint{1.709389in}{2.787312in}}%
\pgfpathcurveto{\pgfqpoint{1.709389in}{2.776261in}}{\pgfqpoint{1.713779in}{2.765662in}}{\pgfqpoint{1.721593in}{2.757849in}}%
\pgfpathcurveto{\pgfqpoint{1.729406in}{2.750035in}}{\pgfqpoint{1.740005in}{2.745645in}}{\pgfqpoint{1.751055in}{2.745645in}}%
\pgfpathlineto{\pgfqpoint{1.751055in}{2.745645in}}%
\pgfpathclose%
\pgfusepath{stroke}%
\end{pgfscope}%
\begin{pgfscope}%
\pgfpathrectangle{\pgfqpoint{0.494722in}{0.437222in}}{\pgfqpoint{6.275590in}{5.159444in}}%
\pgfusepath{clip}%
\pgfsetbuttcap%
\pgfsetroundjoin%
\pgfsetlinewidth{1.003750pt}%
\definecolor{currentstroke}{rgb}{0.827451,0.827451,0.827451}%
\pgfsetstrokecolor{currentstroke}%
\pgfsetstrokeopacity{0.800000}%
\pgfsetdash{}{0pt}%
\pgfpathmoveto{\pgfqpoint{3.615792in}{1.718720in}}%
\pgfpathcurveto{\pgfqpoint{3.626843in}{1.718720in}}{\pgfqpoint{3.637442in}{1.723110in}}{\pgfqpoint{3.645255in}{1.730924in}}%
\pgfpathcurveto{\pgfqpoint{3.653069in}{1.738737in}}{\pgfqpoint{3.657459in}{1.749336in}}{\pgfqpoint{3.657459in}{1.760386in}}%
\pgfpathcurveto{\pgfqpoint{3.657459in}{1.771436in}}{\pgfqpoint{3.653069in}{1.782035in}}{\pgfqpoint{3.645255in}{1.789849in}}%
\pgfpathcurveto{\pgfqpoint{3.637442in}{1.797663in}}{\pgfqpoint{3.626843in}{1.802053in}}{\pgfqpoint{3.615792in}{1.802053in}}%
\pgfpathcurveto{\pgfqpoint{3.604742in}{1.802053in}}{\pgfqpoint{3.594143in}{1.797663in}}{\pgfqpoint{3.586330in}{1.789849in}}%
\pgfpathcurveto{\pgfqpoint{3.578516in}{1.782035in}}{\pgfqpoint{3.574126in}{1.771436in}}{\pgfqpoint{3.574126in}{1.760386in}}%
\pgfpathcurveto{\pgfqpoint{3.574126in}{1.749336in}}{\pgfqpoint{3.578516in}{1.738737in}}{\pgfqpoint{3.586330in}{1.730924in}}%
\pgfpathcurveto{\pgfqpoint{3.594143in}{1.723110in}}{\pgfqpoint{3.604742in}{1.718720in}}{\pgfqpoint{3.615792in}{1.718720in}}%
\pgfpathlineto{\pgfqpoint{3.615792in}{1.718720in}}%
\pgfpathclose%
\pgfusepath{stroke}%
\end{pgfscope}%
\begin{pgfscope}%
\pgfpathrectangle{\pgfqpoint{0.494722in}{0.437222in}}{\pgfqpoint{6.275590in}{5.159444in}}%
\pgfusepath{clip}%
\pgfsetbuttcap%
\pgfsetroundjoin%
\pgfsetlinewidth{1.003750pt}%
\definecolor{currentstroke}{rgb}{0.827451,0.827451,0.827451}%
\pgfsetstrokecolor{currentstroke}%
\pgfsetstrokeopacity{0.800000}%
\pgfsetdash{}{0pt}%
\pgfpathmoveto{\pgfqpoint{4.184126in}{1.944252in}}%
\pgfpathcurveto{\pgfqpoint{4.195176in}{1.944252in}}{\pgfqpoint{4.205775in}{1.948642in}}{\pgfqpoint{4.213589in}{1.956456in}}%
\pgfpathcurveto{\pgfqpoint{4.221403in}{1.964269in}}{\pgfqpoint{4.225793in}{1.974869in}}{\pgfqpoint{4.225793in}{1.985919in}}%
\pgfpathcurveto{\pgfqpoint{4.225793in}{1.996969in}}{\pgfqpoint{4.221403in}{2.007568in}}{\pgfqpoint{4.213589in}{2.015381in}}%
\pgfpathcurveto{\pgfqpoint{4.205775in}{2.023195in}}{\pgfqpoint{4.195176in}{2.027585in}}{\pgfqpoint{4.184126in}{2.027585in}}%
\pgfpathcurveto{\pgfqpoint{4.173076in}{2.027585in}}{\pgfqpoint{4.162477in}{2.023195in}}{\pgfqpoint{4.154664in}{2.015381in}}%
\pgfpathcurveto{\pgfqpoint{4.146850in}{2.007568in}}{\pgfqpoint{4.142460in}{1.996969in}}{\pgfqpoint{4.142460in}{1.985919in}}%
\pgfpathcurveto{\pgfqpoint{4.142460in}{1.974869in}}{\pgfqpoint{4.146850in}{1.964269in}}{\pgfqpoint{4.154664in}{1.956456in}}%
\pgfpathcurveto{\pgfqpoint{4.162477in}{1.948642in}}{\pgfqpoint{4.173076in}{1.944252in}}{\pgfqpoint{4.184126in}{1.944252in}}%
\pgfpathlineto{\pgfqpoint{4.184126in}{1.944252in}}%
\pgfpathclose%
\pgfusepath{stroke}%
\end{pgfscope}%
\begin{pgfscope}%
\pgfpathrectangle{\pgfqpoint{0.494722in}{0.437222in}}{\pgfqpoint{6.275590in}{5.159444in}}%
\pgfusepath{clip}%
\pgfsetbuttcap%
\pgfsetroundjoin%
\pgfsetlinewidth{1.003750pt}%
\definecolor{currentstroke}{rgb}{0.827451,0.827451,0.827451}%
\pgfsetstrokecolor{currentstroke}%
\pgfsetstrokeopacity{0.800000}%
\pgfsetdash{}{0pt}%
\pgfpathmoveto{\pgfqpoint{4.420634in}{1.860999in}}%
\pgfpathcurveto{\pgfqpoint{4.431685in}{1.860999in}}{\pgfqpoint{4.442284in}{1.865390in}}{\pgfqpoint{4.450097in}{1.873203in}}%
\pgfpathcurveto{\pgfqpoint{4.457911in}{1.881017in}}{\pgfqpoint{4.462301in}{1.891616in}}{\pgfqpoint{4.462301in}{1.902666in}}%
\pgfpathcurveto{\pgfqpoint{4.462301in}{1.913716in}}{\pgfqpoint{4.457911in}{1.924315in}}{\pgfqpoint{4.450097in}{1.932129in}}%
\pgfpathcurveto{\pgfqpoint{4.442284in}{1.939942in}}{\pgfqpoint{4.431685in}{1.944333in}}{\pgfqpoint{4.420634in}{1.944333in}}%
\pgfpathcurveto{\pgfqpoint{4.409584in}{1.944333in}}{\pgfqpoint{4.398985in}{1.939942in}}{\pgfqpoint{4.391172in}{1.932129in}}%
\pgfpathcurveto{\pgfqpoint{4.383358in}{1.924315in}}{\pgfqpoint{4.378968in}{1.913716in}}{\pgfqpoint{4.378968in}{1.902666in}}%
\pgfpathcurveto{\pgfqpoint{4.378968in}{1.891616in}}{\pgfqpoint{4.383358in}{1.881017in}}{\pgfqpoint{4.391172in}{1.873203in}}%
\pgfpathcurveto{\pgfqpoint{4.398985in}{1.865390in}}{\pgfqpoint{4.409584in}{1.860999in}}{\pgfqpoint{4.420634in}{1.860999in}}%
\pgfpathlineto{\pgfqpoint{4.420634in}{1.860999in}}%
\pgfpathclose%
\pgfusepath{stroke}%
\end{pgfscope}%
\begin{pgfscope}%
\pgfpathrectangle{\pgfqpoint{0.494722in}{0.437222in}}{\pgfqpoint{6.275590in}{5.159444in}}%
\pgfusepath{clip}%
\pgfsetbuttcap%
\pgfsetroundjoin%
\pgfsetlinewidth{1.003750pt}%
\definecolor{currentstroke}{rgb}{0.827451,0.827451,0.827451}%
\pgfsetstrokecolor{currentstroke}%
\pgfsetstrokeopacity{0.800000}%
\pgfsetdash{}{0pt}%
\pgfpathmoveto{\pgfqpoint{0.622498in}{3.849136in}}%
\pgfpathcurveto{\pgfqpoint{0.633548in}{3.849136in}}{\pgfqpoint{0.644148in}{3.853526in}}{\pgfqpoint{0.651961in}{3.861340in}}%
\pgfpathcurveto{\pgfqpoint{0.659775in}{3.869154in}}{\pgfqpoint{0.664165in}{3.879753in}}{\pgfqpoint{0.664165in}{3.890803in}}%
\pgfpathcurveto{\pgfqpoint{0.664165in}{3.901853in}}{\pgfqpoint{0.659775in}{3.912452in}}{\pgfqpoint{0.651961in}{3.920266in}}%
\pgfpathcurveto{\pgfqpoint{0.644148in}{3.928079in}}{\pgfqpoint{0.633548in}{3.932469in}}{\pgfqpoint{0.622498in}{3.932469in}}%
\pgfpathcurveto{\pgfqpoint{0.611448in}{3.932469in}}{\pgfqpoint{0.600849in}{3.928079in}}{\pgfqpoint{0.593036in}{3.920266in}}%
\pgfpathcurveto{\pgfqpoint{0.585222in}{3.912452in}}{\pgfqpoint{0.580832in}{3.901853in}}{\pgfqpoint{0.580832in}{3.890803in}}%
\pgfpathcurveto{\pgfqpoint{0.580832in}{3.879753in}}{\pgfqpoint{0.585222in}{3.869154in}}{\pgfqpoint{0.593036in}{3.861340in}}%
\pgfpathcurveto{\pgfqpoint{0.600849in}{3.853526in}}{\pgfqpoint{0.611448in}{3.849136in}}{\pgfqpoint{0.622498in}{3.849136in}}%
\pgfpathlineto{\pgfqpoint{0.622498in}{3.849136in}}%
\pgfpathclose%
\pgfusepath{stroke}%
\end{pgfscope}%
\begin{pgfscope}%
\pgfpathrectangle{\pgfqpoint{0.494722in}{0.437222in}}{\pgfqpoint{6.275590in}{5.159444in}}%
\pgfusepath{clip}%
\pgfsetbuttcap%
\pgfsetroundjoin%
\pgfsetlinewidth{1.003750pt}%
\definecolor{currentstroke}{rgb}{0.827451,0.827451,0.827451}%
\pgfsetstrokecolor{currentstroke}%
\pgfsetstrokeopacity{0.800000}%
\pgfsetdash{}{0pt}%
\pgfpathmoveto{\pgfqpoint{0.947159in}{3.147990in}}%
\pgfpathcurveto{\pgfqpoint{0.958209in}{3.147990in}}{\pgfqpoint{0.968808in}{3.152380in}}{\pgfqpoint{0.976622in}{3.160193in}}%
\pgfpathcurveto{\pgfqpoint{0.984435in}{3.168007in}}{\pgfqpoint{0.988826in}{3.178606in}}{\pgfqpoint{0.988826in}{3.189656in}}%
\pgfpathcurveto{\pgfqpoint{0.988826in}{3.200706in}}{\pgfqpoint{0.984435in}{3.211305in}}{\pgfqpoint{0.976622in}{3.219119in}}%
\pgfpathcurveto{\pgfqpoint{0.968808in}{3.226933in}}{\pgfqpoint{0.958209in}{3.231323in}}{\pgfqpoint{0.947159in}{3.231323in}}%
\pgfpathcurveto{\pgfqpoint{0.936109in}{3.231323in}}{\pgfqpoint{0.925510in}{3.226933in}}{\pgfqpoint{0.917696in}{3.219119in}}%
\pgfpathcurveto{\pgfqpoint{0.909883in}{3.211305in}}{\pgfqpoint{0.905492in}{3.200706in}}{\pgfqpoint{0.905492in}{3.189656in}}%
\pgfpathcurveto{\pgfqpoint{0.905492in}{3.178606in}}{\pgfqpoint{0.909883in}{3.168007in}}{\pgfqpoint{0.917696in}{3.160193in}}%
\pgfpathcurveto{\pgfqpoint{0.925510in}{3.152380in}}{\pgfqpoint{0.936109in}{3.147990in}}{\pgfqpoint{0.947159in}{3.147990in}}%
\pgfpathlineto{\pgfqpoint{0.947159in}{3.147990in}}%
\pgfpathclose%
\pgfusepath{stroke}%
\end{pgfscope}%
\begin{pgfscope}%
\pgfpathrectangle{\pgfqpoint{0.494722in}{0.437222in}}{\pgfqpoint{6.275590in}{5.159444in}}%
\pgfusepath{clip}%
\pgfsetbuttcap%
\pgfsetroundjoin%
\pgfsetlinewidth{1.003750pt}%
\definecolor{currentstroke}{rgb}{0.827451,0.827451,0.827451}%
\pgfsetstrokecolor{currentstroke}%
\pgfsetstrokeopacity{0.800000}%
\pgfsetdash{}{0pt}%
\pgfpathmoveto{\pgfqpoint{1.140274in}{2.899383in}}%
\pgfpathcurveto{\pgfqpoint{1.151324in}{2.899383in}}{\pgfqpoint{1.161923in}{2.903774in}}{\pgfqpoint{1.169737in}{2.911587in}}%
\pgfpathcurveto{\pgfqpoint{1.177550in}{2.919401in}}{\pgfqpoint{1.181941in}{2.930000in}}{\pgfqpoint{1.181941in}{2.941050in}}%
\pgfpathcurveto{\pgfqpoint{1.181941in}{2.952100in}}{\pgfqpoint{1.177550in}{2.962699in}}{\pgfqpoint{1.169737in}{2.970513in}}%
\pgfpathcurveto{\pgfqpoint{1.161923in}{2.978326in}}{\pgfqpoint{1.151324in}{2.982717in}}{\pgfqpoint{1.140274in}{2.982717in}}%
\pgfpathcurveto{\pgfqpoint{1.129224in}{2.982717in}}{\pgfqpoint{1.118625in}{2.978326in}}{\pgfqpoint{1.110811in}{2.970513in}}%
\pgfpathcurveto{\pgfqpoint{1.102998in}{2.962699in}}{\pgfqpoint{1.098607in}{2.952100in}}{\pgfqpoint{1.098607in}{2.941050in}}%
\pgfpathcurveto{\pgfqpoint{1.098607in}{2.930000in}}{\pgfqpoint{1.102998in}{2.919401in}}{\pgfqpoint{1.110811in}{2.911587in}}%
\pgfpathcurveto{\pgfqpoint{1.118625in}{2.903774in}}{\pgfqpoint{1.129224in}{2.899383in}}{\pgfqpoint{1.140274in}{2.899383in}}%
\pgfpathlineto{\pgfqpoint{1.140274in}{2.899383in}}%
\pgfpathclose%
\pgfusepath{stroke}%
\end{pgfscope}%
\begin{pgfscope}%
\pgfpathrectangle{\pgfqpoint{0.494722in}{0.437222in}}{\pgfqpoint{6.275590in}{5.159444in}}%
\pgfusepath{clip}%
\pgfsetbuttcap%
\pgfsetroundjoin%
\pgfsetlinewidth{1.003750pt}%
\definecolor{currentstroke}{rgb}{0.827451,0.827451,0.827451}%
\pgfsetstrokecolor{currentstroke}%
\pgfsetstrokeopacity{0.800000}%
\pgfsetdash{}{0pt}%
\pgfpathmoveto{\pgfqpoint{0.996198in}{3.150073in}}%
\pgfpathcurveto{\pgfqpoint{1.007248in}{3.150073in}}{\pgfqpoint{1.017847in}{3.154463in}}{\pgfqpoint{1.025661in}{3.162277in}}%
\pgfpathcurveto{\pgfqpoint{1.033475in}{3.170090in}}{\pgfqpoint{1.037865in}{3.180689in}}{\pgfqpoint{1.037865in}{3.191739in}}%
\pgfpathcurveto{\pgfqpoint{1.037865in}{3.202790in}}{\pgfqpoint{1.033475in}{3.213389in}}{\pgfqpoint{1.025661in}{3.221202in}}%
\pgfpathcurveto{\pgfqpoint{1.017847in}{3.229016in}}{\pgfqpoint{1.007248in}{3.233406in}}{\pgfqpoint{0.996198in}{3.233406in}}%
\pgfpathcurveto{\pgfqpoint{0.985148in}{3.233406in}}{\pgfqpoint{0.974549in}{3.229016in}}{\pgfqpoint{0.966735in}{3.221202in}}%
\pgfpathcurveto{\pgfqpoint{0.958922in}{3.213389in}}{\pgfqpoint{0.954532in}{3.202790in}}{\pgfqpoint{0.954532in}{3.191739in}}%
\pgfpathcurveto{\pgfqpoint{0.954532in}{3.180689in}}{\pgfqpoint{0.958922in}{3.170090in}}{\pgfqpoint{0.966735in}{3.162277in}}%
\pgfpathcurveto{\pgfqpoint{0.974549in}{3.154463in}}{\pgfqpoint{0.985148in}{3.150073in}}{\pgfqpoint{0.996198in}{3.150073in}}%
\pgfpathlineto{\pgfqpoint{0.996198in}{3.150073in}}%
\pgfpathclose%
\pgfusepath{stroke}%
\end{pgfscope}%
\begin{pgfscope}%
\pgfpathrectangle{\pgfqpoint{0.494722in}{0.437222in}}{\pgfqpoint{6.275590in}{5.159444in}}%
\pgfusepath{clip}%
\pgfsetbuttcap%
\pgfsetroundjoin%
\pgfsetlinewidth{1.003750pt}%
\definecolor{currentstroke}{rgb}{0.827451,0.827451,0.827451}%
\pgfsetstrokecolor{currentstroke}%
\pgfsetstrokeopacity{0.800000}%
\pgfsetdash{}{0pt}%
\pgfpathmoveto{\pgfqpoint{1.146637in}{2.932993in}}%
\pgfpathcurveto{\pgfqpoint{1.157687in}{2.932993in}}{\pgfqpoint{1.168286in}{2.937384in}}{\pgfqpoint{1.176100in}{2.945197in}}%
\pgfpathcurveto{\pgfqpoint{1.183913in}{2.953011in}}{\pgfqpoint{1.188304in}{2.963610in}}{\pgfqpoint{1.188304in}{2.974660in}}%
\pgfpathcurveto{\pgfqpoint{1.188304in}{2.985710in}}{\pgfqpoint{1.183913in}{2.996309in}}{\pgfqpoint{1.176100in}{3.004123in}}%
\pgfpathcurveto{\pgfqpoint{1.168286in}{3.011936in}}{\pgfqpoint{1.157687in}{3.016327in}}{\pgfqpoint{1.146637in}{3.016327in}}%
\pgfpathcurveto{\pgfqpoint{1.135587in}{3.016327in}}{\pgfqpoint{1.124988in}{3.011936in}}{\pgfqpoint{1.117174in}{3.004123in}}%
\pgfpathcurveto{\pgfqpoint{1.109361in}{2.996309in}}{\pgfqpoint{1.104970in}{2.985710in}}{\pgfqpoint{1.104970in}{2.974660in}}%
\pgfpathcurveto{\pgfqpoint{1.104970in}{2.963610in}}{\pgfqpoint{1.109361in}{2.953011in}}{\pgfqpoint{1.117174in}{2.945197in}}%
\pgfpathcurveto{\pgfqpoint{1.124988in}{2.937384in}}{\pgfqpoint{1.135587in}{2.932993in}}{\pgfqpoint{1.146637in}{2.932993in}}%
\pgfpathlineto{\pgfqpoint{1.146637in}{2.932993in}}%
\pgfpathclose%
\pgfusepath{stroke}%
\end{pgfscope}%
\begin{pgfscope}%
\pgfpathrectangle{\pgfqpoint{0.494722in}{0.437222in}}{\pgfqpoint{6.275590in}{5.159444in}}%
\pgfusepath{clip}%
\pgfsetbuttcap%
\pgfsetroundjoin%
\pgfsetlinewidth{1.003750pt}%
\definecolor{currentstroke}{rgb}{0.827451,0.827451,0.827451}%
\pgfsetstrokecolor{currentstroke}%
\pgfsetstrokeopacity{0.800000}%
\pgfsetdash{}{0pt}%
\pgfpathmoveto{\pgfqpoint{1.822265in}{2.259778in}}%
\pgfpathcurveto{\pgfqpoint{1.833315in}{2.259778in}}{\pgfqpoint{1.843914in}{2.264168in}}{\pgfqpoint{1.851728in}{2.271982in}}%
\pgfpathcurveto{\pgfqpoint{1.859542in}{2.279795in}}{\pgfqpoint{1.863932in}{2.290394in}}{\pgfqpoint{1.863932in}{2.301445in}}%
\pgfpathcurveto{\pgfqpoint{1.863932in}{2.312495in}}{\pgfqpoint{1.859542in}{2.323094in}}{\pgfqpoint{1.851728in}{2.330907in}}%
\pgfpathcurveto{\pgfqpoint{1.843914in}{2.338721in}}{\pgfqpoint{1.833315in}{2.343111in}}{\pgfqpoint{1.822265in}{2.343111in}}%
\pgfpathcurveto{\pgfqpoint{1.811215in}{2.343111in}}{\pgfqpoint{1.800616in}{2.338721in}}{\pgfqpoint{1.792802in}{2.330907in}}%
\pgfpathcurveto{\pgfqpoint{1.784989in}{2.323094in}}{\pgfqpoint{1.780598in}{2.312495in}}{\pgfqpoint{1.780598in}{2.301445in}}%
\pgfpathcurveto{\pgfqpoint{1.780598in}{2.290394in}}{\pgfqpoint{1.784989in}{2.279795in}}{\pgfqpoint{1.792802in}{2.271982in}}%
\pgfpathcurveto{\pgfqpoint{1.800616in}{2.264168in}}{\pgfqpoint{1.811215in}{2.259778in}}{\pgfqpoint{1.822265in}{2.259778in}}%
\pgfpathlineto{\pgfqpoint{1.822265in}{2.259778in}}%
\pgfpathclose%
\pgfusepath{stroke}%
\end{pgfscope}%
\begin{pgfscope}%
\pgfpathrectangle{\pgfqpoint{0.494722in}{0.437222in}}{\pgfqpoint{6.275590in}{5.159444in}}%
\pgfusepath{clip}%
\pgfsetbuttcap%
\pgfsetroundjoin%
\pgfsetlinewidth{1.003750pt}%
\definecolor{currentstroke}{rgb}{0.827451,0.827451,0.827451}%
\pgfsetstrokecolor{currentstroke}%
\pgfsetstrokeopacity{0.800000}%
\pgfsetdash{}{0pt}%
\pgfpathmoveto{\pgfqpoint{1.640440in}{2.774042in}}%
\pgfpathcurveto{\pgfqpoint{1.651491in}{2.774042in}}{\pgfqpoint{1.662090in}{2.778432in}}{\pgfqpoint{1.669903in}{2.786246in}}%
\pgfpathcurveto{\pgfqpoint{1.677717in}{2.794059in}}{\pgfqpoint{1.682107in}{2.804658in}}{\pgfqpoint{1.682107in}{2.815709in}}%
\pgfpathcurveto{\pgfqpoint{1.682107in}{2.826759in}}{\pgfqpoint{1.677717in}{2.837358in}}{\pgfqpoint{1.669903in}{2.845171in}}%
\pgfpathcurveto{\pgfqpoint{1.662090in}{2.852985in}}{\pgfqpoint{1.651491in}{2.857375in}}{\pgfqpoint{1.640440in}{2.857375in}}%
\pgfpathcurveto{\pgfqpoint{1.629390in}{2.857375in}}{\pgfqpoint{1.618791in}{2.852985in}}{\pgfqpoint{1.610978in}{2.845171in}}%
\pgfpathcurveto{\pgfqpoint{1.603164in}{2.837358in}}{\pgfqpoint{1.598774in}{2.826759in}}{\pgfqpoint{1.598774in}{2.815709in}}%
\pgfpathcurveto{\pgfqpoint{1.598774in}{2.804658in}}{\pgfqpoint{1.603164in}{2.794059in}}{\pgfqpoint{1.610978in}{2.786246in}}%
\pgfpathcurveto{\pgfqpoint{1.618791in}{2.778432in}}{\pgfqpoint{1.629390in}{2.774042in}}{\pgfqpoint{1.640440in}{2.774042in}}%
\pgfpathlineto{\pgfqpoint{1.640440in}{2.774042in}}%
\pgfpathclose%
\pgfusepath{stroke}%
\end{pgfscope}%
\begin{pgfscope}%
\pgfpathrectangle{\pgfqpoint{0.494722in}{0.437222in}}{\pgfqpoint{6.275590in}{5.159444in}}%
\pgfusepath{clip}%
\pgfsetbuttcap%
\pgfsetroundjoin%
\pgfsetlinewidth{1.003750pt}%
\definecolor{currentstroke}{rgb}{0.827451,0.827451,0.827451}%
\pgfsetstrokecolor{currentstroke}%
\pgfsetstrokeopacity{0.800000}%
\pgfsetdash{}{0pt}%
\pgfpathmoveto{\pgfqpoint{1.185063in}{2.437307in}}%
\pgfpathcurveto{\pgfqpoint{1.196113in}{2.437307in}}{\pgfqpoint{1.206712in}{2.441697in}}{\pgfqpoint{1.214526in}{2.449511in}}%
\pgfpathcurveto{\pgfqpoint{1.222340in}{2.457325in}}{\pgfqpoint{1.226730in}{2.467924in}}{\pgfqpoint{1.226730in}{2.478974in}}%
\pgfpathcurveto{\pgfqpoint{1.226730in}{2.490024in}}{\pgfqpoint{1.222340in}{2.500623in}}{\pgfqpoint{1.214526in}{2.508437in}}%
\pgfpathcurveto{\pgfqpoint{1.206712in}{2.516250in}}{\pgfqpoint{1.196113in}{2.520641in}}{\pgfqpoint{1.185063in}{2.520641in}}%
\pgfpathcurveto{\pgfqpoint{1.174013in}{2.520641in}}{\pgfqpoint{1.163414in}{2.516250in}}{\pgfqpoint{1.155600in}{2.508437in}}%
\pgfpathcurveto{\pgfqpoint{1.147787in}{2.500623in}}{\pgfqpoint{1.143397in}{2.490024in}}{\pgfqpoint{1.143397in}{2.478974in}}%
\pgfpathcurveto{\pgfqpoint{1.143397in}{2.467924in}}{\pgfqpoint{1.147787in}{2.457325in}}{\pgfqpoint{1.155600in}{2.449511in}}%
\pgfpathcurveto{\pgfqpoint{1.163414in}{2.441697in}}{\pgfqpoint{1.174013in}{2.437307in}}{\pgfqpoint{1.185063in}{2.437307in}}%
\pgfpathlineto{\pgfqpoint{1.185063in}{2.437307in}}%
\pgfpathclose%
\pgfusepath{stroke}%
\end{pgfscope}%
\begin{pgfscope}%
\pgfpathrectangle{\pgfqpoint{0.494722in}{0.437222in}}{\pgfqpoint{6.275590in}{5.159444in}}%
\pgfusepath{clip}%
\pgfsetbuttcap%
\pgfsetroundjoin%
\pgfsetlinewidth{1.003750pt}%
\definecolor{currentstroke}{rgb}{0.827451,0.827451,0.827451}%
\pgfsetstrokecolor{currentstroke}%
\pgfsetstrokeopacity{0.800000}%
\pgfsetdash{}{0pt}%
\pgfpathmoveto{\pgfqpoint{2.117156in}{1.399260in}}%
\pgfpathcurveto{\pgfqpoint{2.128206in}{1.399260in}}{\pgfqpoint{2.138805in}{1.403650in}}{\pgfqpoint{2.146618in}{1.411464in}}%
\pgfpathcurveto{\pgfqpoint{2.154432in}{1.419277in}}{\pgfqpoint{2.158822in}{1.429876in}}{\pgfqpoint{2.158822in}{1.440927in}}%
\pgfpathcurveto{\pgfqpoint{2.158822in}{1.451977in}}{\pgfqpoint{2.154432in}{1.462576in}}{\pgfqpoint{2.146618in}{1.470389in}}%
\pgfpathcurveto{\pgfqpoint{2.138805in}{1.478203in}}{\pgfqpoint{2.128206in}{1.482593in}}{\pgfqpoint{2.117156in}{1.482593in}}%
\pgfpathcurveto{\pgfqpoint{2.106106in}{1.482593in}}{\pgfqpoint{2.095506in}{1.478203in}}{\pgfqpoint{2.087693in}{1.470389in}}%
\pgfpathcurveto{\pgfqpoint{2.079879in}{1.462576in}}{\pgfqpoint{2.075489in}{1.451977in}}{\pgfqpoint{2.075489in}{1.440927in}}%
\pgfpathcurveto{\pgfqpoint{2.075489in}{1.429876in}}{\pgfqpoint{2.079879in}{1.419277in}}{\pgfqpoint{2.087693in}{1.411464in}}%
\pgfpathcurveto{\pgfqpoint{2.095506in}{1.403650in}}{\pgfqpoint{2.106106in}{1.399260in}}{\pgfqpoint{2.117156in}{1.399260in}}%
\pgfpathlineto{\pgfqpoint{2.117156in}{1.399260in}}%
\pgfpathclose%
\pgfusepath{stroke}%
\end{pgfscope}%
\begin{pgfscope}%
\pgfpathrectangle{\pgfqpoint{0.494722in}{0.437222in}}{\pgfqpoint{6.275590in}{5.159444in}}%
\pgfusepath{clip}%
\pgfsetbuttcap%
\pgfsetroundjoin%
\pgfsetlinewidth{1.003750pt}%
\definecolor{currentstroke}{rgb}{0.827451,0.827451,0.827451}%
\pgfsetstrokecolor{currentstroke}%
\pgfsetstrokeopacity{0.800000}%
\pgfsetdash{}{0pt}%
\pgfpathmoveto{\pgfqpoint{1.702429in}{1.852953in}}%
\pgfpathcurveto{\pgfqpoint{1.713479in}{1.852953in}}{\pgfqpoint{1.724078in}{1.857344in}}{\pgfqpoint{1.731891in}{1.865157in}}%
\pgfpathcurveto{\pgfqpoint{1.739705in}{1.872971in}}{\pgfqpoint{1.744095in}{1.883570in}}{\pgfqpoint{1.744095in}{1.894620in}}%
\pgfpathcurveto{\pgfqpoint{1.744095in}{1.905670in}}{\pgfqpoint{1.739705in}{1.916269in}}{\pgfqpoint{1.731891in}{1.924083in}}%
\pgfpathcurveto{\pgfqpoint{1.724078in}{1.931897in}}{\pgfqpoint{1.713479in}{1.936287in}}{\pgfqpoint{1.702429in}{1.936287in}}%
\pgfpathcurveto{\pgfqpoint{1.691379in}{1.936287in}}{\pgfqpoint{1.680780in}{1.931897in}}{\pgfqpoint{1.672966in}{1.924083in}}%
\pgfpathcurveto{\pgfqpoint{1.665152in}{1.916269in}}{\pgfqpoint{1.660762in}{1.905670in}}{\pgfqpoint{1.660762in}{1.894620in}}%
\pgfpathcurveto{\pgfqpoint{1.660762in}{1.883570in}}{\pgfqpoint{1.665152in}{1.872971in}}{\pgfqpoint{1.672966in}{1.865157in}}%
\pgfpathcurveto{\pgfqpoint{1.680780in}{1.857344in}}{\pgfqpoint{1.691379in}{1.852953in}}{\pgfqpoint{1.702429in}{1.852953in}}%
\pgfpathlineto{\pgfqpoint{1.702429in}{1.852953in}}%
\pgfpathclose%
\pgfusepath{stroke}%
\end{pgfscope}%
\begin{pgfscope}%
\pgfpathrectangle{\pgfqpoint{0.494722in}{0.437222in}}{\pgfqpoint{6.275590in}{5.159444in}}%
\pgfusepath{clip}%
\pgfsetbuttcap%
\pgfsetroundjoin%
\pgfsetlinewidth{1.003750pt}%
\definecolor{currentstroke}{rgb}{0.827451,0.827451,0.827451}%
\pgfsetstrokecolor{currentstroke}%
\pgfsetstrokeopacity{0.800000}%
\pgfsetdash{}{0pt}%
\pgfpathmoveto{\pgfqpoint{1.477105in}{1.953817in}}%
\pgfpathcurveto{\pgfqpoint{1.488155in}{1.953817in}}{\pgfqpoint{1.498754in}{1.958207in}}{\pgfqpoint{1.506568in}{1.966021in}}%
\pgfpathcurveto{\pgfqpoint{1.514381in}{1.973835in}}{\pgfqpoint{1.518772in}{1.984434in}}{\pgfqpoint{1.518772in}{1.995484in}}%
\pgfpathcurveto{\pgfqpoint{1.518772in}{2.006534in}}{\pgfqpoint{1.514381in}{2.017133in}}{\pgfqpoint{1.506568in}{2.024947in}}%
\pgfpathcurveto{\pgfqpoint{1.498754in}{2.032760in}}{\pgfqpoint{1.488155in}{2.037151in}}{\pgfqpoint{1.477105in}{2.037151in}}%
\pgfpathcurveto{\pgfqpoint{1.466055in}{2.037151in}}{\pgfqpoint{1.455456in}{2.032760in}}{\pgfqpoint{1.447642in}{2.024947in}}%
\pgfpathcurveto{\pgfqpoint{1.439828in}{2.017133in}}{\pgfqpoint{1.435438in}{2.006534in}}{\pgfqpoint{1.435438in}{1.995484in}}%
\pgfpathcurveto{\pgfqpoint{1.435438in}{1.984434in}}{\pgfqpoint{1.439828in}{1.973835in}}{\pgfqpoint{1.447642in}{1.966021in}}%
\pgfpathcurveto{\pgfqpoint{1.455456in}{1.958207in}}{\pgfqpoint{1.466055in}{1.953817in}}{\pgfqpoint{1.477105in}{1.953817in}}%
\pgfpathlineto{\pgfqpoint{1.477105in}{1.953817in}}%
\pgfpathclose%
\pgfusepath{stroke}%
\end{pgfscope}%
\begin{pgfscope}%
\pgfpathrectangle{\pgfqpoint{0.494722in}{0.437222in}}{\pgfqpoint{6.275590in}{5.159444in}}%
\pgfusepath{clip}%
\pgfsetbuttcap%
\pgfsetroundjoin%
\pgfsetlinewidth{1.003750pt}%
\definecolor{currentstroke}{rgb}{0.827451,0.827451,0.827451}%
\pgfsetstrokecolor{currentstroke}%
\pgfsetstrokeopacity{0.800000}%
\pgfsetdash{}{0pt}%
\pgfpathmoveto{\pgfqpoint{3.937358in}{0.655597in}}%
\pgfpathcurveto{\pgfqpoint{3.948408in}{0.655597in}}{\pgfqpoint{3.959007in}{0.659987in}}{\pgfqpoint{3.966821in}{0.667801in}}%
\pgfpathcurveto{\pgfqpoint{3.974634in}{0.675614in}}{\pgfqpoint{3.979025in}{0.686213in}}{\pgfqpoint{3.979025in}{0.697263in}}%
\pgfpathcurveto{\pgfqpoint{3.979025in}{0.708314in}}{\pgfqpoint{3.974634in}{0.718913in}}{\pgfqpoint{3.966821in}{0.726726in}}%
\pgfpathcurveto{\pgfqpoint{3.959007in}{0.734540in}}{\pgfqpoint{3.948408in}{0.738930in}}{\pgfqpoint{3.937358in}{0.738930in}}%
\pgfpathcurveto{\pgfqpoint{3.926308in}{0.738930in}}{\pgfqpoint{3.915709in}{0.734540in}}{\pgfqpoint{3.907895in}{0.726726in}}%
\pgfpathcurveto{\pgfqpoint{3.900082in}{0.718913in}}{\pgfqpoint{3.895691in}{0.708314in}}{\pgfqpoint{3.895691in}{0.697263in}}%
\pgfpathcurveto{\pgfqpoint{3.895691in}{0.686213in}}{\pgfqpoint{3.900082in}{0.675614in}}{\pgfqpoint{3.907895in}{0.667801in}}%
\pgfpathcurveto{\pgfqpoint{3.915709in}{0.659987in}}{\pgfqpoint{3.926308in}{0.655597in}}{\pgfqpoint{3.937358in}{0.655597in}}%
\pgfpathlineto{\pgfqpoint{3.937358in}{0.655597in}}%
\pgfpathclose%
\pgfusepath{stroke}%
\end{pgfscope}%
\begin{pgfscope}%
\pgfpathrectangle{\pgfqpoint{0.494722in}{0.437222in}}{\pgfqpoint{6.275590in}{5.159444in}}%
\pgfusepath{clip}%
\pgfsetbuttcap%
\pgfsetroundjoin%
\pgfsetlinewidth{1.003750pt}%
\definecolor{currentstroke}{rgb}{0.827451,0.827451,0.827451}%
\pgfsetstrokecolor{currentstroke}%
\pgfsetstrokeopacity{0.800000}%
\pgfsetdash{}{0pt}%
\pgfpathmoveto{\pgfqpoint{1.551682in}{1.872518in}}%
\pgfpathcurveto{\pgfqpoint{1.562732in}{1.872518in}}{\pgfqpoint{1.573331in}{1.876908in}}{\pgfqpoint{1.581144in}{1.884722in}}%
\pgfpathcurveto{\pgfqpoint{1.588958in}{1.892535in}}{\pgfqpoint{1.593348in}{1.903134in}}{\pgfqpoint{1.593348in}{1.914184in}}%
\pgfpathcurveto{\pgfqpoint{1.593348in}{1.925234in}}{\pgfqpoint{1.588958in}{1.935833in}}{\pgfqpoint{1.581144in}{1.943647in}}%
\pgfpathcurveto{\pgfqpoint{1.573331in}{1.951461in}}{\pgfqpoint{1.562732in}{1.955851in}}{\pgfqpoint{1.551682in}{1.955851in}}%
\pgfpathcurveto{\pgfqpoint{1.540632in}{1.955851in}}{\pgfqpoint{1.530033in}{1.951461in}}{\pgfqpoint{1.522219in}{1.943647in}}%
\pgfpathcurveto{\pgfqpoint{1.514405in}{1.935833in}}{\pgfqpoint{1.510015in}{1.925234in}}{\pgfqpoint{1.510015in}{1.914184in}}%
\pgfpathcurveto{\pgfqpoint{1.510015in}{1.903134in}}{\pgfqpoint{1.514405in}{1.892535in}}{\pgfqpoint{1.522219in}{1.884722in}}%
\pgfpathcurveto{\pgfqpoint{1.530033in}{1.876908in}}{\pgfqpoint{1.540632in}{1.872518in}}{\pgfqpoint{1.551682in}{1.872518in}}%
\pgfpathlineto{\pgfqpoint{1.551682in}{1.872518in}}%
\pgfpathclose%
\pgfusepath{stroke}%
\end{pgfscope}%
\begin{pgfscope}%
\pgfpathrectangle{\pgfqpoint{0.494722in}{0.437222in}}{\pgfqpoint{6.275590in}{5.159444in}}%
\pgfusepath{clip}%
\pgfsetbuttcap%
\pgfsetroundjoin%
\pgfsetlinewidth{1.003750pt}%
\definecolor{currentstroke}{rgb}{0.827451,0.827451,0.827451}%
\pgfsetstrokecolor{currentstroke}%
\pgfsetstrokeopacity{0.800000}%
\pgfsetdash{}{0pt}%
\pgfpathmoveto{\pgfqpoint{1.991938in}{1.503092in}}%
\pgfpathcurveto{\pgfqpoint{2.002988in}{1.503092in}}{\pgfqpoint{2.013587in}{1.507482in}}{\pgfqpoint{2.021400in}{1.515296in}}%
\pgfpathcurveto{\pgfqpoint{2.029214in}{1.523110in}}{\pgfqpoint{2.033604in}{1.533709in}}{\pgfqpoint{2.033604in}{1.544759in}}%
\pgfpathcurveto{\pgfqpoint{2.033604in}{1.555809in}}{\pgfqpoint{2.029214in}{1.566408in}}{\pgfqpoint{2.021400in}{1.574222in}}%
\pgfpathcurveto{\pgfqpoint{2.013587in}{1.582035in}}{\pgfqpoint{2.002988in}{1.586425in}}{\pgfqpoint{1.991938in}{1.586425in}}%
\pgfpathcurveto{\pgfqpoint{1.980887in}{1.586425in}}{\pgfqpoint{1.970288in}{1.582035in}}{\pgfqpoint{1.962475in}{1.574222in}}%
\pgfpathcurveto{\pgfqpoint{1.954661in}{1.566408in}}{\pgfqpoint{1.950271in}{1.555809in}}{\pgfqpoint{1.950271in}{1.544759in}}%
\pgfpathcurveto{\pgfqpoint{1.950271in}{1.533709in}}{\pgfqpoint{1.954661in}{1.523110in}}{\pgfqpoint{1.962475in}{1.515296in}}%
\pgfpathcurveto{\pgfqpoint{1.970288in}{1.507482in}}{\pgfqpoint{1.980887in}{1.503092in}}{\pgfqpoint{1.991938in}{1.503092in}}%
\pgfpathlineto{\pgfqpoint{1.991938in}{1.503092in}}%
\pgfpathclose%
\pgfusepath{stroke}%
\end{pgfscope}%
\begin{pgfscope}%
\pgfpathrectangle{\pgfqpoint{0.494722in}{0.437222in}}{\pgfqpoint{6.275590in}{5.159444in}}%
\pgfusepath{clip}%
\pgfsetbuttcap%
\pgfsetroundjoin%
\pgfsetlinewidth{1.003750pt}%
\definecolor{currentstroke}{rgb}{0.827451,0.827451,0.827451}%
\pgfsetstrokecolor{currentstroke}%
\pgfsetstrokeopacity{0.800000}%
\pgfsetdash{}{0pt}%
\pgfpathmoveto{\pgfqpoint{1.344163in}{2.102732in}}%
\pgfpathcurveto{\pgfqpoint{1.355213in}{2.102732in}}{\pgfqpoint{1.365812in}{2.107122in}}{\pgfqpoint{1.373626in}{2.114936in}}%
\pgfpathcurveto{\pgfqpoint{1.381439in}{2.122749in}}{\pgfqpoint{1.385830in}{2.133348in}}{\pgfqpoint{1.385830in}{2.144398in}}%
\pgfpathcurveto{\pgfqpoint{1.385830in}{2.155448in}}{\pgfqpoint{1.381439in}{2.166047in}}{\pgfqpoint{1.373626in}{2.173861in}}%
\pgfpathcurveto{\pgfqpoint{1.365812in}{2.181675in}}{\pgfqpoint{1.355213in}{2.186065in}}{\pgfqpoint{1.344163in}{2.186065in}}%
\pgfpathcurveto{\pgfqpoint{1.333113in}{2.186065in}}{\pgfqpoint{1.322514in}{2.181675in}}{\pgfqpoint{1.314700in}{2.173861in}}%
\pgfpathcurveto{\pgfqpoint{1.306887in}{2.166047in}}{\pgfqpoint{1.302496in}{2.155448in}}{\pgfqpoint{1.302496in}{2.144398in}}%
\pgfpathcurveto{\pgfqpoint{1.302496in}{2.133348in}}{\pgfqpoint{1.306887in}{2.122749in}}{\pgfqpoint{1.314700in}{2.114936in}}%
\pgfpathcurveto{\pgfqpoint{1.322514in}{2.107122in}}{\pgfqpoint{1.333113in}{2.102732in}}{\pgfqpoint{1.344163in}{2.102732in}}%
\pgfpathlineto{\pgfqpoint{1.344163in}{2.102732in}}%
\pgfpathclose%
\pgfusepath{stroke}%
\end{pgfscope}%
\begin{pgfscope}%
\pgfpathrectangle{\pgfqpoint{0.494722in}{0.437222in}}{\pgfqpoint{6.275590in}{5.159444in}}%
\pgfusepath{clip}%
\pgfsetbuttcap%
\pgfsetroundjoin%
\pgfsetlinewidth{1.003750pt}%
\definecolor{currentstroke}{rgb}{0.827451,0.827451,0.827451}%
\pgfsetstrokecolor{currentstroke}%
\pgfsetstrokeopacity{0.800000}%
\pgfsetdash{}{0pt}%
\pgfpathmoveto{\pgfqpoint{2.429002in}{1.219445in}}%
\pgfpathcurveto{\pgfqpoint{2.440052in}{1.219445in}}{\pgfqpoint{2.450651in}{1.223835in}}{\pgfqpoint{2.458465in}{1.231648in}}%
\pgfpathcurveto{\pgfqpoint{2.466278in}{1.239462in}}{\pgfqpoint{2.470669in}{1.250061in}}{\pgfqpoint{2.470669in}{1.261111in}}%
\pgfpathcurveto{\pgfqpoint{2.470669in}{1.272161in}}{\pgfqpoint{2.466278in}{1.282760in}}{\pgfqpoint{2.458465in}{1.290574in}}%
\pgfpathcurveto{\pgfqpoint{2.450651in}{1.298388in}}{\pgfqpoint{2.440052in}{1.302778in}}{\pgfqpoint{2.429002in}{1.302778in}}%
\pgfpathcurveto{\pgfqpoint{2.417952in}{1.302778in}}{\pgfqpoint{2.407353in}{1.298388in}}{\pgfqpoint{2.399539in}{1.290574in}}%
\pgfpathcurveto{\pgfqpoint{2.391726in}{1.282760in}}{\pgfqpoint{2.387335in}{1.272161in}}{\pgfqpoint{2.387335in}{1.261111in}}%
\pgfpathcurveto{\pgfqpoint{2.387335in}{1.250061in}}{\pgfqpoint{2.391726in}{1.239462in}}{\pgfqpoint{2.399539in}{1.231648in}}%
\pgfpathcurveto{\pgfqpoint{2.407353in}{1.223835in}}{\pgfqpoint{2.417952in}{1.219445in}}{\pgfqpoint{2.429002in}{1.219445in}}%
\pgfpathlineto{\pgfqpoint{2.429002in}{1.219445in}}%
\pgfpathclose%
\pgfusepath{stroke}%
\end{pgfscope}%
\begin{pgfscope}%
\pgfpathrectangle{\pgfqpoint{0.494722in}{0.437222in}}{\pgfqpoint{6.275590in}{5.159444in}}%
\pgfusepath{clip}%
\pgfsetbuttcap%
\pgfsetroundjoin%
\pgfsetlinewidth{1.003750pt}%
\definecolor{currentstroke}{rgb}{0.827451,0.827451,0.827451}%
\pgfsetstrokecolor{currentstroke}%
\pgfsetstrokeopacity{0.800000}%
\pgfsetdash{}{0pt}%
\pgfpathmoveto{\pgfqpoint{0.556707in}{3.885184in}}%
\pgfpathcurveto{\pgfqpoint{0.567757in}{3.885184in}}{\pgfqpoint{0.578356in}{3.889574in}}{\pgfqpoint{0.586170in}{3.897388in}}%
\pgfpathcurveto{\pgfqpoint{0.593983in}{3.905201in}}{\pgfqpoint{0.598374in}{3.915800in}}{\pgfqpoint{0.598374in}{3.926850in}}%
\pgfpathcurveto{\pgfqpoint{0.598374in}{3.937900in}}{\pgfqpoint{0.593983in}{3.948499in}}{\pgfqpoint{0.586170in}{3.956313in}}%
\pgfpathcurveto{\pgfqpoint{0.578356in}{3.964127in}}{\pgfqpoint{0.567757in}{3.968517in}}{\pgfqpoint{0.556707in}{3.968517in}}%
\pgfpathcurveto{\pgfqpoint{0.545657in}{3.968517in}}{\pgfqpoint{0.535058in}{3.964127in}}{\pgfqpoint{0.527244in}{3.956313in}}%
\pgfpathcurveto{\pgfqpoint{0.519430in}{3.948499in}}{\pgfqpoint{0.515040in}{3.937900in}}{\pgfqpoint{0.515040in}{3.926850in}}%
\pgfpathcurveto{\pgfqpoint{0.515040in}{3.915800in}}{\pgfqpoint{0.519430in}{3.905201in}}{\pgfqpoint{0.527244in}{3.897388in}}%
\pgfpathcurveto{\pgfqpoint{0.535058in}{3.889574in}}{\pgfqpoint{0.545657in}{3.885184in}}{\pgfqpoint{0.556707in}{3.885184in}}%
\pgfpathlineto{\pgfqpoint{0.556707in}{3.885184in}}%
\pgfpathclose%
\pgfusepath{stroke}%
\end{pgfscope}%
\begin{pgfscope}%
\pgfpathrectangle{\pgfqpoint{0.494722in}{0.437222in}}{\pgfqpoint{6.275590in}{5.159444in}}%
\pgfusepath{clip}%
\pgfsetbuttcap%
\pgfsetroundjoin%
\pgfsetlinewidth{1.003750pt}%
\definecolor{currentstroke}{rgb}{0.827451,0.827451,0.827451}%
\pgfsetstrokecolor{currentstroke}%
\pgfsetstrokeopacity{0.800000}%
\pgfsetdash{}{0pt}%
\pgfpathmoveto{\pgfqpoint{1.278200in}{2.184905in}}%
\pgfpathcurveto{\pgfqpoint{1.289250in}{2.184905in}}{\pgfqpoint{1.299849in}{2.189295in}}{\pgfqpoint{1.307663in}{2.197109in}}%
\pgfpathcurveto{\pgfqpoint{1.315477in}{2.204922in}}{\pgfqpoint{1.319867in}{2.215521in}}{\pgfqpoint{1.319867in}{2.226572in}}%
\pgfpathcurveto{\pgfqpoint{1.319867in}{2.237622in}}{\pgfqpoint{1.315477in}{2.248221in}}{\pgfqpoint{1.307663in}{2.256034in}}%
\pgfpathcurveto{\pgfqpoint{1.299849in}{2.263848in}}{\pgfqpoint{1.289250in}{2.268238in}}{\pgfqpoint{1.278200in}{2.268238in}}%
\pgfpathcurveto{\pgfqpoint{1.267150in}{2.268238in}}{\pgfqpoint{1.256551in}{2.263848in}}{\pgfqpoint{1.248738in}{2.256034in}}%
\pgfpathcurveto{\pgfqpoint{1.240924in}{2.248221in}}{\pgfqpoint{1.236534in}{2.237622in}}{\pgfqpoint{1.236534in}{2.226572in}}%
\pgfpathcurveto{\pgfqpoint{1.236534in}{2.215521in}}{\pgfqpoint{1.240924in}{2.204922in}}{\pgfqpoint{1.248738in}{2.197109in}}%
\pgfpathcurveto{\pgfqpoint{1.256551in}{2.189295in}}{\pgfqpoint{1.267150in}{2.184905in}}{\pgfqpoint{1.278200in}{2.184905in}}%
\pgfpathlineto{\pgfqpoint{1.278200in}{2.184905in}}%
\pgfpathclose%
\pgfusepath{stroke}%
\end{pgfscope}%
\begin{pgfscope}%
\pgfpathrectangle{\pgfqpoint{0.494722in}{0.437222in}}{\pgfqpoint{6.275590in}{5.159444in}}%
\pgfusepath{clip}%
\pgfsetbuttcap%
\pgfsetroundjoin%
\pgfsetlinewidth{1.003750pt}%
\definecolor{currentstroke}{rgb}{0.827451,0.827451,0.827451}%
\pgfsetstrokecolor{currentstroke}%
\pgfsetstrokeopacity{0.800000}%
\pgfsetdash{}{0pt}%
\pgfpathmoveto{\pgfqpoint{3.164762in}{0.836209in}}%
\pgfpathcurveto{\pgfqpoint{3.175812in}{0.836209in}}{\pgfqpoint{3.186411in}{0.840599in}}{\pgfqpoint{3.194224in}{0.848413in}}%
\pgfpathcurveto{\pgfqpoint{3.202038in}{0.856226in}}{\pgfqpoint{3.206428in}{0.866825in}}{\pgfqpoint{3.206428in}{0.877875in}}%
\pgfpathcurveto{\pgfqpoint{3.206428in}{0.888926in}}{\pgfqpoint{3.202038in}{0.899525in}}{\pgfqpoint{3.194224in}{0.907338in}}%
\pgfpathcurveto{\pgfqpoint{3.186411in}{0.915152in}}{\pgfqpoint{3.175812in}{0.919542in}}{\pgfqpoint{3.164762in}{0.919542in}}%
\pgfpathcurveto{\pgfqpoint{3.153711in}{0.919542in}}{\pgfqpoint{3.143112in}{0.915152in}}{\pgfqpoint{3.135299in}{0.907338in}}%
\pgfpathcurveto{\pgfqpoint{3.127485in}{0.899525in}}{\pgfqpoint{3.123095in}{0.888926in}}{\pgfqpoint{3.123095in}{0.877875in}}%
\pgfpathcurveto{\pgfqpoint{3.123095in}{0.866825in}}{\pgfqpoint{3.127485in}{0.856226in}}{\pgfqpoint{3.135299in}{0.848413in}}%
\pgfpathcurveto{\pgfqpoint{3.143112in}{0.840599in}}{\pgfqpoint{3.153711in}{0.836209in}}{\pgfqpoint{3.164762in}{0.836209in}}%
\pgfpathlineto{\pgfqpoint{3.164762in}{0.836209in}}%
\pgfpathclose%
\pgfusepath{stroke}%
\end{pgfscope}%
\begin{pgfscope}%
\pgfpathrectangle{\pgfqpoint{0.494722in}{0.437222in}}{\pgfqpoint{6.275590in}{5.159444in}}%
\pgfusepath{clip}%
\pgfsetbuttcap%
\pgfsetroundjoin%
\pgfsetlinewidth{1.003750pt}%
\definecolor{currentstroke}{rgb}{0.827451,0.827451,0.827451}%
\pgfsetstrokecolor{currentstroke}%
\pgfsetstrokeopacity{0.800000}%
\pgfsetdash{}{0pt}%
\pgfpathmoveto{\pgfqpoint{4.335578in}{0.505432in}}%
\pgfpathcurveto{\pgfqpoint{4.346628in}{0.505432in}}{\pgfqpoint{4.357227in}{0.509823in}}{\pgfqpoint{4.365041in}{0.517636in}}%
\pgfpathcurveto{\pgfqpoint{4.372855in}{0.525450in}}{\pgfqpoint{4.377245in}{0.536049in}}{\pgfqpoint{4.377245in}{0.547099in}}%
\pgfpathcurveto{\pgfqpoint{4.377245in}{0.558149in}}{\pgfqpoint{4.372855in}{0.568748in}}{\pgfqpoint{4.365041in}{0.576562in}}%
\pgfpathcurveto{\pgfqpoint{4.357227in}{0.584375in}}{\pgfqpoint{4.346628in}{0.588766in}}{\pgfqpoint{4.335578in}{0.588766in}}%
\pgfpathcurveto{\pgfqpoint{4.324528in}{0.588766in}}{\pgfqpoint{4.313929in}{0.584375in}}{\pgfqpoint{4.306115in}{0.576562in}}%
\pgfpathcurveto{\pgfqpoint{4.298302in}{0.568748in}}{\pgfqpoint{4.293911in}{0.558149in}}{\pgfqpoint{4.293911in}{0.547099in}}%
\pgfpathcurveto{\pgfqpoint{4.293911in}{0.536049in}}{\pgfqpoint{4.298302in}{0.525450in}}{\pgfqpoint{4.306115in}{0.517636in}}%
\pgfpathcurveto{\pgfqpoint{4.313929in}{0.509823in}}{\pgfqpoint{4.324528in}{0.505432in}}{\pgfqpoint{4.335578in}{0.505432in}}%
\pgfpathlineto{\pgfqpoint{4.335578in}{0.505432in}}%
\pgfpathclose%
\pgfusepath{stroke}%
\end{pgfscope}%
\begin{pgfscope}%
\pgfpathrectangle{\pgfqpoint{0.494722in}{0.437222in}}{\pgfqpoint{6.275590in}{5.159444in}}%
\pgfusepath{clip}%
\pgfsetbuttcap%
\pgfsetroundjoin%
\pgfsetlinewidth{1.003750pt}%
\definecolor{currentstroke}{rgb}{0.827451,0.827451,0.827451}%
\pgfsetstrokecolor{currentstroke}%
\pgfsetstrokeopacity{0.800000}%
\pgfsetdash{}{0pt}%
\pgfpathmoveto{\pgfqpoint{2.656452in}{1.108305in}}%
\pgfpathcurveto{\pgfqpoint{2.667502in}{1.108305in}}{\pgfqpoint{2.678102in}{1.112695in}}{\pgfqpoint{2.685915in}{1.120509in}}%
\pgfpathcurveto{\pgfqpoint{2.693729in}{1.128322in}}{\pgfqpoint{2.698119in}{1.138921in}}{\pgfqpoint{2.698119in}{1.149971in}}%
\pgfpathcurveto{\pgfqpoint{2.698119in}{1.161021in}}{\pgfqpoint{2.693729in}{1.171620in}}{\pgfqpoint{2.685915in}{1.179434in}}%
\pgfpathcurveto{\pgfqpoint{2.678102in}{1.187248in}}{\pgfqpoint{2.667502in}{1.191638in}}{\pgfqpoint{2.656452in}{1.191638in}}%
\pgfpathcurveto{\pgfqpoint{2.645402in}{1.191638in}}{\pgfqpoint{2.634803in}{1.187248in}}{\pgfqpoint{2.626990in}{1.179434in}}%
\pgfpathcurveto{\pgfqpoint{2.619176in}{1.171620in}}{\pgfqpoint{2.614786in}{1.161021in}}{\pgfqpoint{2.614786in}{1.149971in}}%
\pgfpathcurveto{\pgfqpoint{2.614786in}{1.138921in}}{\pgfqpoint{2.619176in}{1.128322in}}{\pgfqpoint{2.626990in}{1.120509in}}%
\pgfpathcurveto{\pgfqpoint{2.634803in}{1.112695in}}{\pgfqpoint{2.645402in}{1.108305in}}{\pgfqpoint{2.656452in}{1.108305in}}%
\pgfpathlineto{\pgfqpoint{2.656452in}{1.108305in}}%
\pgfpathclose%
\pgfusepath{stroke}%
\end{pgfscope}%
\begin{pgfscope}%
\pgfpathrectangle{\pgfqpoint{0.494722in}{0.437222in}}{\pgfqpoint{6.275590in}{5.159444in}}%
\pgfusepath{clip}%
\pgfsetbuttcap%
\pgfsetroundjoin%
\pgfsetlinewidth{1.003750pt}%
\definecolor{currentstroke}{rgb}{0.827451,0.827451,0.827451}%
\pgfsetstrokecolor{currentstroke}%
\pgfsetstrokeopacity{0.800000}%
\pgfsetdash{}{0pt}%
\pgfpathmoveto{\pgfqpoint{0.575099in}{3.779944in}}%
\pgfpathcurveto{\pgfqpoint{0.586149in}{3.779944in}}{\pgfqpoint{0.596748in}{3.784334in}}{\pgfqpoint{0.604561in}{3.792148in}}%
\pgfpathcurveto{\pgfqpoint{0.612375in}{3.799961in}}{\pgfqpoint{0.616765in}{3.810560in}}{\pgfqpoint{0.616765in}{3.821610in}}%
\pgfpathcurveto{\pgfqpoint{0.616765in}{3.832661in}}{\pgfqpoint{0.612375in}{3.843260in}}{\pgfqpoint{0.604561in}{3.851073in}}%
\pgfpathcurveto{\pgfqpoint{0.596748in}{3.858887in}}{\pgfqpoint{0.586149in}{3.863277in}}{\pgfqpoint{0.575099in}{3.863277in}}%
\pgfpathcurveto{\pgfqpoint{0.564048in}{3.863277in}}{\pgfqpoint{0.553449in}{3.858887in}}{\pgfqpoint{0.545636in}{3.851073in}}%
\pgfpathcurveto{\pgfqpoint{0.537822in}{3.843260in}}{\pgfqpoint{0.533432in}{3.832661in}}{\pgfqpoint{0.533432in}{3.821610in}}%
\pgfpathcurveto{\pgfqpoint{0.533432in}{3.810560in}}{\pgfqpoint{0.537822in}{3.799961in}}{\pgfqpoint{0.545636in}{3.792148in}}%
\pgfpathcurveto{\pgfqpoint{0.553449in}{3.784334in}}{\pgfqpoint{0.564048in}{3.779944in}}{\pgfqpoint{0.575099in}{3.779944in}}%
\pgfpathlineto{\pgfqpoint{0.575099in}{3.779944in}}%
\pgfpathclose%
\pgfusepath{stroke}%
\end{pgfscope}%
\begin{pgfscope}%
\pgfpathrectangle{\pgfqpoint{0.494722in}{0.437222in}}{\pgfqpoint{6.275590in}{5.159444in}}%
\pgfusepath{clip}%
\pgfsetbuttcap%
\pgfsetroundjoin%
\pgfsetlinewidth{1.003750pt}%
\definecolor{currentstroke}{rgb}{0.827451,0.827451,0.827451}%
\pgfsetstrokecolor{currentstroke}%
\pgfsetstrokeopacity{0.800000}%
\pgfsetdash{}{0pt}%
\pgfpathmoveto{\pgfqpoint{1.581391in}{1.870889in}}%
\pgfpathcurveto{\pgfqpoint{1.592441in}{1.870889in}}{\pgfqpoint{1.603040in}{1.875279in}}{\pgfqpoint{1.610854in}{1.883093in}}%
\pgfpathcurveto{\pgfqpoint{1.618667in}{1.890906in}}{\pgfqpoint{1.623058in}{1.901505in}}{\pgfqpoint{1.623058in}{1.912556in}}%
\pgfpathcurveto{\pgfqpoint{1.623058in}{1.923606in}}{\pgfqpoint{1.618667in}{1.934205in}}{\pgfqpoint{1.610854in}{1.942018in}}%
\pgfpathcurveto{\pgfqpoint{1.603040in}{1.949832in}}{\pgfqpoint{1.592441in}{1.954222in}}{\pgfqpoint{1.581391in}{1.954222in}}%
\pgfpathcurveto{\pgfqpoint{1.570341in}{1.954222in}}{\pgfqpoint{1.559742in}{1.949832in}}{\pgfqpoint{1.551928in}{1.942018in}}%
\pgfpathcurveto{\pgfqpoint{1.544115in}{1.934205in}}{\pgfqpoint{1.539724in}{1.923606in}}{\pgfqpoint{1.539724in}{1.912556in}}%
\pgfpathcurveto{\pgfqpoint{1.539724in}{1.901505in}}{\pgfqpoint{1.544115in}{1.890906in}}{\pgfqpoint{1.551928in}{1.883093in}}%
\pgfpathcurveto{\pgfqpoint{1.559742in}{1.875279in}}{\pgfqpoint{1.570341in}{1.870889in}}{\pgfqpoint{1.581391in}{1.870889in}}%
\pgfpathlineto{\pgfqpoint{1.581391in}{1.870889in}}%
\pgfpathclose%
\pgfusepath{stroke}%
\end{pgfscope}%
\begin{pgfscope}%
\pgfpathrectangle{\pgfqpoint{0.494722in}{0.437222in}}{\pgfqpoint{6.275590in}{5.159444in}}%
\pgfusepath{clip}%
\pgfsetbuttcap%
\pgfsetroundjoin%
\pgfsetlinewidth{1.003750pt}%
\definecolor{currentstroke}{rgb}{0.827451,0.827451,0.827451}%
\pgfsetstrokecolor{currentstroke}%
\pgfsetstrokeopacity{0.800000}%
\pgfsetdash{}{0pt}%
\pgfpathmoveto{\pgfqpoint{1.772500in}{1.679456in}}%
\pgfpathcurveto{\pgfqpoint{1.783550in}{1.679456in}}{\pgfqpoint{1.794149in}{1.683846in}}{\pgfqpoint{1.801962in}{1.691660in}}%
\pgfpathcurveto{\pgfqpoint{1.809776in}{1.699473in}}{\pgfqpoint{1.814166in}{1.710072in}}{\pgfqpoint{1.814166in}{1.721122in}}%
\pgfpathcurveto{\pgfqpoint{1.814166in}{1.732172in}}{\pgfqpoint{1.809776in}{1.742772in}}{\pgfqpoint{1.801962in}{1.750585in}}%
\pgfpathcurveto{\pgfqpoint{1.794149in}{1.758399in}}{\pgfqpoint{1.783550in}{1.762789in}}{\pgfqpoint{1.772500in}{1.762789in}}%
\pgfpathcurveto{\pgfqpoint{1.761450in}{1.762789in}}{\pgfqpoint{1.750850in}{1.758399in}}{\pgfqpoint{1.743037in}{1.750585in}}%
\pgfpathcurveto{\pgfqpoint{1.735223in}{1.742772in}}{\pgfqpoint{1.730833in}{1.732172in}}{\pgfqpoint{1.730833in}{1.721122in}}%
\pgfpathcurveto{\pgfqpoint{1.730833in}{1.710072in}}{\pgfqpoint{1.735223in}{1.699473in}}{\pgfqpoint{1.743037in}{1.691660in}}%
\pgfpathcurveto{\pgfqpoint{1.750850in}{1.683846in}}{\pgfqpoint{1.761450in}{1.679456in}}{\pgfqpoint{1.772500in}{1.679456in}}%
\pgfpathlineto{\pgfqpoint{1.772500in}{1.679456in}}%
\pgfpathclose%
\pgfusepath{stroke}%
\end{pgfscope}%
\begin{pgfscope}%
\pgfpathrectangle{\pgfqpoint{0.494722in}{0.437222in}}{\pgfqpoint{6.275590in}{5.159444in}}%
\pgfusepath{clip}%
\pgfsetbuttcap%
\pgfsetroundjoin%
\pgfsetlinewidth{1.003750pt}%
\definecolor{currentstroke}{rgb}{0.827451,0.827451,0.827451}%
\pgfsetstrokecolor{currentstroke}%
\pgfsetstrokeopacity{0.800000}%
\pgfsetdash{}{0pt}%
\pgfpathmoveto{\pgfqpoint{3.652979in}{0.754601in}}%
\pgfpathcurveto{\pgfqpoint{3.664029in}{0.754601in}}{\pgfqpoint{3.674628in}{0.758991in}}{\pgfqpoint{3.682442in}{0.766805in}}%
\pgfpathcurveto{\pgfqpoint{3.690255in}{0.774619in}}{\pgfqpoint{3.694645in}{0.785218in}}{\pgfqpoint{3.694645in}{0.796268in}}%
\pgfpathcurveto{\pgfqpoint{3.694645in}{0.807318in}}{\pgfqpoint{3.690255in}{0.817917in}}{\pgfqpoint{3.682442in}{0.825731in}}%
\pgfpathcurveto{\pgfqpoint{3.674628in}{0.833544in}}{\pgfqpoint{3.664029in}{0.837935in}}{\pgfqpoint{3.652979in}{0.837935in}}%
\pgfpathcurveto{\pgfqpoint{3.641929in}{0.837935in}}{\pgfqpoint{3.631330in}{0.833544in}}{\pgfqpoint{3.623516in}{0.825731in}}%
\pgfpathcurveto{\pgfqpoint{3.615702in}{0.817917in}}{\pgfqpoint{3.611312in}{0.807318in}}{\pgfqpoint{3.611312in}{0.796268in}}%
\pgfpathcurveto{\pgfqpoint{3.611312in}{0.785218in}}{\pgfqpoint{3.615702in}{0.774619in}}{\pgfqpoint{3.623516in}{0.766805in}}%
\pgfpathcurveto{\pgfqpoint{3.631330in}{0.758991in}}{\pgfqpoint{3.641929in}{0.754601in}}{\pgfqpoint{3.652979in}{0.754601in}}%
\pgfpathlineto{\pgfqpoint{3.652979in}{0.754601in}}%
\pgfpathclose%
\pgfusepath{stroke}%
\end{pgfscope}%
\begin{pgfscope}%
\pgfpathrectangle{\pgfqpoint{0.494722in}{0.437222in}}{\pgfqpoint{6.275590in}{5.159444in}}%
\pgfusepath{clip}%
\pgfsetbuttcap%
\pgfsetroundjoin%
\pgfsetlinewidth{1.003750pt}%
\definecolor{currentstroke}{rgb}{0.827451,0.827451,0.827451}%
\pgfsetstrokecolor{currentstroke}%
\pgfsetstrokeopacity{0.800000}%
\pgfsetdash{}{0pt}%
\pgfpathmoveto{\pgfqpoint{1.042231in}{2.681267in}}%
\pgfpathcurveto{\pgfqpoint{1.053281in}{2.681267in}}{\pgfqpoint{1.063880in}{2.685657in}}{\pgfqpoint{1.071694in}{2.693471in}}%
\pgfpathcurveto{\pgfqpoint{1.079507in}{2.701284in}}{\pgfqpoint{1.083898in}{2.711883in}}{\pgfqpoint{1.083898in}{2.722933in}}%
\pgfpathcurveto{\pgfqpoint{1.083898in}{2.733984in}}{\pgfqpoint{1.079507in}{2.744583in}}{\pgfqpoint{1.071694in}{2.752396in}}%
\pgfpathcurveto{\pgfqpoint{1.063880in}{2.760210in}}{\pgfqpoint{1.053281in}{2.764600in}}{\pgfqpoint{1.042231in}{2.764600in}}%
\pgfpathcurveto{\pgfqpoint{1.031181in}{2.764600in}}{\pgfqpoint{1.020582in}{2.760210in}}{\pgfqpoint{1.012768in}{2.752396in}}%
\pgfpathcurveto{\pgfqpoint{1.004954in}{2.744583in}}{\pgfqpoint{1.000564in}{2.733984in}}{\pgfqpoint{1.000564in}{2.722933in}}%
\pgfpathcurveto{\pgfqpoint{1.000564in}{2.711883in}}{\pgfqpoint{1.004954in}{2.701284in}}{\pgfqpoint{1.012768in}{2.693471in}}%
\pgfpathcurveto{\pgfqpoint{1.020582in}{2.685657in}}{\pgfqpoint{1.031181in}{2.681267in}}{\pgfqpoint{1.042231in}{2.681267in}}%
\pgfpathlineto{\pgfqpoint{1.042231in}{2.681267in}}%
\pgfpathclose%
\pgfusepath{stroke}%
\end{pgfscope}%
\begin{pgfscope}%
\pgfpathrectangle{\pgfqpoint{0.494722in}{0.437222in}}{\pgfqpoint{6.275590in}{5.159444in}}%
\pgfusepath{clip}%
\pgfsetbuttcap%
\pgfsetroundjoin%
\pgfsetlinewidth{1.003750pt}%
\definecolor{currentstroke}{rgb}{0.827451,0.827451,0.827451}%
\pgfsetstrokecolor{currentstroke}%
\pgfsetstrokeopacity{0.800000}%
\pgfsetdash{}{0pt}%
\pgfpathmoveto{\pgfqpoint{2.867283in}{0.963868in}}%
\pgfpathcurveto{\pgfqpoint{2.878333in}{0.963868in}}{\pgfqpoint{2.888932in}{0.968258in}}{\pgfqpoint{2.896746in}{0.976072in}}%
\pgfpathcurveto{\pgfqpoint{2.904560in}{0.983885in}}{\pgfqpoint{2.908950in}{0.994484in}}{\pgfqpoint{2.908950in}{1.005535in}}%
\pgfpathcurveto{\pgfqpoint{2.908950in}{1.016585in}}{\pgfqpoint{2.904560in}{1.027184in}}{\pgfqpoint{2.896746in}{1.034997in}}%
\pgfpathcurveto{\pgfqpoint{2.888932in}{1.042811in}}{\pgfqpoint{2.878333in}{1.047201in}}{\pgfqpoint{2.867283in}{1.047201in}}%
\pgfpathcurveto{\pgfqpoint{2.856233in}{1.047201in}}{\pgfqpoint{2.845634in}{1.042811in}}{\pgfqpoint{2.837820in}{1.034997in}}%
\pgfpathcurveto{\pgfqpoint{2.830007in}{1.027184in}}{\pgfqpoint{2.825617in}{1.016585in}}{\pgfqpoint{2.825617in}{1.005535in}}%
\pgfpathcurveto{\pgfqpoint{2.825617in}{0.994484in}}{\pgfqpoint{2.830007in}{0.983885in}}{\pgfqpoint{2.837820in}{0.976072in}}%
\pgfpathcurveto{\pgfqpoint{2.845634in}{0.968258in}}{\pgfqpoint{2.856233in}{0.963868in}}{\pgfqpoint{2.867283in}{0.963868in}}%
\pgfpathlineto{\pgfqpoint{2.867283in}{0.963868in}}%
\pgfpathclose%
\pgfusepath{stroke}%
\end{pgfscope}%
\begin{pgfscope}%
\pgfpathrectangle{\pgfqpoint{0.494722in}{0.437222in}}{\pgfqpoint{6.275590in}{5.159444in}}%
\pgfusepath{clip}%
\pgfsetbuttcap%
\pgfsetroundjoin%
\pgfsetlinewidth{1.003750pt}%
\definecolor{currentstroke}{rgb}{0.827451,0.827451,0.827451}%
\pgfsetstrokecolor{currentstroke}%
\pgfsetstrokeopacity{0.800000}%
\pgfsetdash{}{0pt}%
\pgfpathmoveto{\pgfqpoint{1.756385in}{1.772664in}}%
\pgfpathcurveto{\pgfqpoint{1.767436in}{1.772664in}}{\pgfqpoint{1.778035in}{1.777054in}}{\pgfqpoint{1.785848in}{1.784868in}}%
\pgfpathcurveto{\pgfqpoint{1.793662in}{1.792682in}}{\pgfqpoint{1.798052in}{1.803281in}}{\pgfqpoint{1.798052in}{1.814331in}}%
\pgfpathcurveto{\pgfqpoint{1.798052in}{1.825381in}}{\pgfqpoint{1.793662in}{1.835980in}}{\pgfqpoint{1.785848in}{1.843793in}}%
\pgfpathcurveto{\pgfqpoint{1.778035in}{1.851607in}}{\pgfqpoint{1.767436in}{1.855997in}}{\pgfqpoint{1.756385in}{1.855997in}}%
\pgfpathcurveto{\pgfqpoint{1.745335in}{1.855997in}}{\pgfqpoint{1.734736in}{1.851607in}}{\pgfqpoint{1.726923in}{1.843793in}}%
\pgfpathcurveto{\pgfqpoint{1.719109in}{1.835980in}}{\pgfqpoint{1.714719in}{1.825381in}}{\pgfqpoint{1.714719in}{1.814331in}}%
\pgfpathcurveto{\pgfqpoint{1.714719in}{1.803281in}}{\pgfqpoint{1.719109in}{1.792682in}}{\pgfqpoint{1.726923in}{1.784868in}}%
\pgfpathcurveto{\pgfqpoint{1.734736in}{1.777054in}}{\pgfqpoint{1.745335in}{1.772664in}}{\pgfqpoint{1.756385in}{1.772664in}}%
\pgfpathlineto{\pgfqpoint{1.756385in}{1.772664in}}%
\pgfpathclose%
\pgfusepath{stroke}%
\end{pgfscope}%
\begin{pgfscope}%
\pgfpathrectangle{\pgfqpoint{0.494722in}{0.437222in}}{\pgfqpoint{6.275590in}{5.159444in}}%
\pgfusepath{clip}%
\pgfsetbuttcap%
\pgfsetroundjoin%
\pgfsetlinewidth{1.003750pt}%
\definecolor{currentstroke}{rgb}{0.827451,0.827451,0.827451}%
\pgfsetstrokecolor{currentstroke}%
\pgfsetstrokeopacity{0.800000}%
\pgfsetdash{}{0pt}%
\pgfpathmoveto{\pgfqpoint{2.572850in}{1.116571in}}%
\pgfpathcurveto{\pgfqpoint{2.583900in}{1.116571in}}{\pgfqpoint{2.594499in}{1.120961in}}{\pgfqpoint{2.602313in}{1.128774in}}%
\pgfpathcurveto{\pgfqpoint{2.610126in}{1.136588in}}{\pgfqpoint{2.614517in}{1.147187in}}{\pgfqpoint{2.614517in}{1.158237in}}%
\pgfpathcurveto{\pgfqpoint{2.614517in}{1.169287in}}{\pgfqpoint{2.610126in}{1.179886in}}{\pgfqpoint{2.602313in}{1.187700in}}%
\pgfpathcurveto{\pgfqpoint{2.594499in}{1.195514in}}{\pgfqpoint{2.583900in}{1.199904in}}{\pgfqpoint{2.572850in}{1.199904in}}%
\pgfpathcurveto{\pgfqpoint{2.561800in}{1.199904in}}{\pgfqpoint{2.551201in}{1.195514in}}{\pgfqpoint{2.543387in}{1.187700in}}%
\pgfpathcurveto{\pgfqpoint{2.535574in}{1.179886in}}{\pgfqpoint{2.531183in}{1.169287in}}{\pgfqpoint{2.531183in}{1.158237in}}%
\pgfpathcurveto{\pgfqpoint{2.531183in}{1.147187in}}{\pgfqpoint{2.535574in}{1.136588in}}{\pgfqpoint{2.543387in}{1.128774in}}%
\pgfpathcurveto{\pgfqpoint{2.551201in}{1.120961in}}{\pgfqpoint{2.561800in}{1.116571in}}{\pgfqpoint{2.572850in}{1.116571in}}%
\pgfpathlineto{\pgfqpoint{2.572850in}{1.116571in}}%
\pgfpathclose%
\pgfusepath{stroke}%
\end{pgfscope}%
\begin{pgfscope}%
\pgfpathrectangle{\pgfqpoint{0.494722in}{0.437222in}}{\pgfqpoint{6.275590in}{5.159444in}}%
\pgfusepath{clip}%
\pgfsetbuttcap%
\pgfsetroundjoin%
\pgfsetlinewidth{1.003750pt}%
\definecolor{currentstroke}{rgb}{0.827451,0.827451,0.827451}%
\pgfsetstrokecolor{currentstroke}%
\pgfsetstrokeopacity{0.800000}%
\pgfsetdash{}{0pt}%
\pgfpathmoveto{\pgfqpoint{0.963748in}{2.873526in}}%
\pgfpathcurveto{\pgfqpoint{0.974798in}{2.873526in}}{\pgfqpoint{0.985398in}{2.877916in}}{\pgfqpoint{0.993211in}{2.885730in}}%
\pgfpathcurveto{\pgfqpoint{1.001025in}{2.893543in}}{\pgfqpoint{1.005415in}{2.904142in}}{\pgfqpoint{1.005415in}{2.915192in}}%
\pgfpathcurveto{\pgfqpoint{1.005415in}{2.926243in}}{\pgfqpoint{1.001025in}{2.936842in}}{\pgfqpoint{0.993211in}{2.944655in}}%
\pgfpathcurveto{\pgfqpoint{0.985398in}{2.952469in}}{\pgfqpoint{0.974798in}{2.956859in}}{\pgfqpoint{0.963748in}{2.956859in}}%
\pgfpathcurveto{\pgfqpoint{0.952698in}{2.956859in}}{\pgfqpoint{0.942099in}{2.952469in}}{\pgfqpoint{0.934286in}{2.944655in}}%
\pgfpathcurveto{\pgfqpoint{0.926472in}{2.936842in}}{\pgfqpoint{0.922082in}{2.926243in}}{\pgfqpoint{0.922082in}{2.915192in}}%
\pgfpathcurveto{\pgfqpoint{0.922082in}{2.904142in}}{\pgfqpoint{0.926472in}{2.893543in}}{\pgfqpoint{0.934286in}{2.885730in}}%
\pgfpathcurveto{\pgfqpoint{0.942099in}{2.877916in}}{\pgfqpoint{0.952698in}{2.873526in}}{\pgfqpoint{0.963748in}{2.873526in}}%
\pgfpathlineto{\pgfqpoint{0.963748in}{2.873526in}}%
\pgfpathclose%
\pgfusepath{stroke}%
\end{pgfscope}%
\begin{pgfscope}%
\pgfpathrectangle{\pgfqpoint{0.494722in}{0.437222in}}{\pgfqpoint{6.275590in}{5.159444in}}%
\pgfusepath{clip}%
\pgfsetbuttcap%
\pgfsetroundjoin%
\pgfsetlinewidth{1.003750pt}%
\definecolor{currentstroke}{rgb}{0.827451,0.827451,0.827451}%
\pgfsetstrokecolor{currentstroke}%
\pgfsetstrokeopacity{0.800000}%
\pgfsetdash{}{0pt}%
\pgfpathmoveto{\pgfqpoint{2.512961in}{1.165916in}}%
\pgfpathcurveto{\pgfqpoint{2.524011in}{1.165916in}}{\pgfqpoint{2.534610in}{1.170306in}}{\pgfqpoint{2.542424in}{1.178119in}}%
\pgfpathcurveto{\pgfqpoint{2.550237in}{1.185933in}}{\pgfqpoint{2.554627in}{1.196532in}}{\pgfqpoint{2.554627in}{1.207582in}}%
\pgfpathcurveto{\pgfqpoint{2.554627in}{1.218632in}}{\pgfqpoint{2.550237in}{1.229231in}}{\pgfqpoint{2.542424in}{1.237045in}}%
\pgfpathcurveto{\pgfqpoint{2.534610in}{1.244859in}}{\pgfqpoint{2.524011in}{1.249249in}}{\pgfqpoint{2.512961in}{1.249249in}}%
\pgfpathcurveto{\pgfqpoint{2.501911in}{1.249249in}}{\pgfqpoint{2.491312in}{1.244859in}}{\pgfqpoint{2.483498in}{1.237045in}}%
\pgfpathcurveto{\pgfqpoint{2.475684in}{1.229231in}}{\pgfqpoint{2.471294in}{1.218632in}}{\pgfqpoint{2.471294in}{1.207582in}}%
\pgfpathcurveto{\pgfqpoint{2.471294in}{1.196532in}}{\pgfqpoint{2.475684in}{1.185933in}}{\pgfqpoint{2.483498in}{1.178119in}}%
\pgfpathcurveto{\pgfqpoint{2.491312in}{1.170306in}}{\pgfqpoint{2.501911in}{1.165916in}}{\pgfqpoint{2.512961in}{1.165916in}}%
\pgfpathlineto{\pgfqpoint{2.512961in}{1.165916in}}%
\pgfpathclose%
\pgfusepath{stroke}%
\end{pgfscope}%
\begin{pgfscope}%
\pgfpathrectangle{\pgfqpoint{0.494722in}{0.437222in}}{\pgfqpoint{6.275590in}{5.159444in}}%
\pgfusepath{clip}%
\pgfsetbuttcap%
\pgfsetroundjoin%
\pgfsetlinewidth{1.003750pt}%
\definecolor{currentstroke}{rgb}{0.827451,0.827451,0.827451}%
\pgfsetstrokecolor{currentstroke}%
\pgfsetstrokeopacity{0.800000}%
\pgfsetdash{}{0pt}%
\pgfpathmoveto{\pgfqpoint{1.016820in}{2.832444in}}%
\pgfpathcurveto{\pgfqpoint{1.027870in}{2.832444in}}{\pgfqpoint{1.038469in}{2.836834in}}{\pgfqpoint{1.046283in}{2.844648in}}%
\pgfpathcurveto{\pgfqpoint{1.054096in}{2.852462in}}{\pgfqpoint{1.058487in}{2.863061in}}{\pgfqpoint{1.058487in}{2.874111in}}%
\pgfpathcurveto{\pgfqpoint{1.058487in}{2.885161in}}{\pgfqpoint{1.054096in}{2.895760in}}{\pgfqpoint{1.046283in}{2.903573in}}%
\pgfpathcurveto{\pgfqpoint{1.038469in}{2.911387in}}{\pgfqpoint{1.027870in}{2.915777in}}{\pgfqpoint{1.016820in}{2.915777in}}%
\pgfpathcurveto{\pgfqpoint{1.005770in}{2.915777in}}{\pgfqpoint{0.995171in}{2.911387in}}{\pgfqpoint{0.987357in}{2.903573in}}%
\pgfpathcurveto{\pgfqpoint{0.979544in}{2.895760in}}{\pgfqpoint{0.975153in}{2.885161in}}{\pgfqpoint{0.975153in}{2.874111in}}%
\pgfpathcurveto{\pgfqpoint{0.975153in}{2.863061in}}{\pgfqpoint{0.979544in}{2.852462in}}{\pgfqpoint{0.987357in}{2.844648in}}%
\pgfpathcurveto{\pgfqpoint{0.995171in}{2.836834in}}{\pgfqpoint{1.005770in}{2.832444in}}{\pgfqpoint{1.016820in}{2.832444in}}%
\pgfpathlineto{\pgfqpoint{1.016820in}{2.832444in}}%
\pgfpathclose%
\pgfusepath{stroke}%
\end{pgfscope}%
\begin{pgfscope}%
\pgfpathrectangle{\pgfqpoint{0.494722in}{0.437222in}}{\pgfqpoint{6.275590in}{5.159444in}}%
\pgfusepath{clip}%
\pgfsetbuttcap%
\pgfsetroundjoin%
\pgfsetlinewidth{1.003750pt}%
\definecolor{currentstroke}{rgb}{0.827451,0.827451,0.827451}%
\pgfsetstrokecolor{currentstroke}%
\pgfsetstrokeopacity{0.800000}%
\pgfsetdash{}{0pt}%
\pgfpathmoveto{\pgfqpoint{3.709744in}{0.686390in}}%
\pgfpathcurveto{\pgfqpoint{3.720794in}{0.686390in}}{\pgfqpoint{3.731393in}{0.690780in}}{\pgfqpoint{3.739207in}{0.698593in}}%
\pgfpathcurveto{\pgfqpoint{3.747021in}{0.706407in}}{\pgfqpoint{3.751411in}{0.717006in}}{\pgfqpoint{3.751411in}{0.728056in}}%
\pgfpathcurveto{\pgfqpoint{3.751411in}{0.739106in}}{\pgfqpoint{3.747021in}{0.749705in}}{\pgfqpoint{3.739207in}{0.757519in}}%
\pgfpathcurveto{\pgfqpoint{3.731393in}{0.765333in}}{\pgfqpoint{3.720794in}{0.769723in}}{\pgfqpoint{3.709744in}{0.769723in}}%
\pgfpathcurveto{\pgfqpoint{3.698694in}{0.769723in}}{\pgfqpoint{3.688095in}{0.765333in}}{\pgfqpoint{3.680281in}{0.757519in}}%
\pgfpathcurveto{\pgfqpoint{3.672468in}{0.749705in}}{\pgfqpoint{3.668077in}{0.739106in}}{\pgfqpoint{3.668077in}{0.728056in}}%
\pgfpathcurveto{\pgfqpoint{3.668077in}{0.717006in}}{\pgfqpoint{3.672468in}{0.706407in}}{\pgfqpoint{3.680281in}{0.698593in}}%
\pgfpathcurveto{\pgfqpoint{3.688095in}{0.690780in}}{\pgfqpoint{3.698694in}{0.686390in}}{\pgfqpoint{3.709744in}{0.686390in}}%
\pgfpathlineto{\pgfqpoint{3.709744in}{0.686390in}}%
\pgfpathclose%
\pgfusepath{stroke}%
\end{pgfscope}%
\begin{pgfscope}%
\pgfpathrectangle{\pgfqpoint{0.494722in}{0.437222in}}{\pgfqpoint{6.275590in}{5.159444in}}%
\pgfusepath{clip}%
\pgfsetbuttcap%
\pgfsetroundjoin%
\pgfsetlinewidth{1.003750pt}%
\definecolor{currentstroke}{rgb}{0.827451,0.827451,0.827451}%
\pgfsetstrokecolor{currentstroke}%
\pgfsetstrokeopacity{0.800000}%
\pgfsetdash{}{0pt}%
\pgfpathmoveto{\pgfqpoint{0.859512in}{3.088326in}}%
\pgfpathcurveto{\pgfqpoint{0.870562in}{3.088326in}}{\pgfqpoint{0.881161in}{3.092716in}}{\pgfqpoint{0.888974in}{3.100530in}}%
\pgfpathcurveto{\pgfqpoint{0.896788in}{3.108344in}}{\pgfqpoint{0.901178in}{3.118943in}}{\pgfqpoint{0.901178in}{3.129993in}}%
\pgfpathcurveto{\pgfqpoint{0.901178in}{3.141043in}}{\pgfqpoint{0.896788in}{3.151642in}}{\pgfqpoint{0.888974in}{3.159455in}}%
\pgfpathcurveto{\pgfqpoint{0.881161in}{3.167269in}}{\pgfqpoint{0.870562in}{3.171659in}}{\pgfqpoint{0.859512in}{3.171659in}}%
\pgfpathcurveto{\pgfqpoint{0.848462in}{3.171659in}}{\pgfqpoint{0.837863in}{3.167269in}}{\pgfqpoint{0.830049in}{3.159455in}}%
\pgfpathcurveto{\pgfqpoint{0.822235in}{3.151642in}}{\pgfqpoint{0.817845in}{3.141043in}}{\pgfqpoint{0.817845in}{3.129993in}}%
\pgfpathcurveto{\pgfqpoint{0.817845in}{3.118943in}}{\pgfqpoint{0.822235in}{3.108344in}}{\pgfqpoint{0.830049in}{3.100530in}}%
\pgfpathcurveto{\pgfqpoint{0.837863in}{3.092716in}}{\pgfqpoint{0.848462in}{3.088326in}}{\pgfqpoint{0.859512in}{3.088326in}}%
\pgfpathlineto{\pgfqpoint{0.859512in}{3.088326in}}%
\pgfpathclose%
\pgfusepath{stroke}%
\end{pgfscope}%
\begin{pgfscope}%
\pgfpathrectangle{\pgfqpoint{0.494722in}{0.437222in}}{\pgfqpoint{6.275590in}{5.159444in}}%
\pgfusepath{clip}%
\pgfsetbuttcap%
\pgfsetroundjoin%
\pgfsetlinewidth{1.003750pt}%
\definecolor{currentstroke}{rgb}{0.827451,0.827451,0.827451}%
\pgfsetstrokecolor{currentstroke}%
\pgfsetstrokeopacity{0.800000}%
\pgfsetdash{}{0pt}%
\pgfpathmoveto{\pgfqpoint{1.545871in}{1.889457in}}%
\pgfpathcurveto{\pgfqpoint{1.556921in}{1.889457in}}{\pgfqpoint{1.567520in}{1.893848in}}{\pgfqpoint{1.575334in}{1.901661in}}%
\pgfpathcurveto{\pgfqpoint{1.583148in}{1.909475in}}{\pgfqpoint{1.587538in}{1.920074in}}{\pgfqpoint{1.587538in}{1.931124in}}%
\pgfpathcurveto{\pgfqpoint{1.587538in}{1.942174in}}{\pgfqpoint{1.583148in}{1.952773in}}{\pgfqpoint{1.575334in}{1.960587in}}%
\pgfpathcurveto{\pgfqpoint{1.567520in}{1.968400in}}{\pgfqpoint{1.556921in}{1.972791in}}{\pgfqpoint{1.545871in}{1.972791in}}%
\pgfpathcurveto{\pgfqpoint{1.534821in}{1.972791in}}{\pgfqpoint{1.524222in}{1.968400in}}{\pgfqpoint{1.516408in}{1.960587in}}%
\pgfpathcurveto{\pgfqpoint{1.508595in}{1.952773in}}{\pgfqpoint{1.504204in}{1.942174in}}{\pgfqpoint{1.504204in}{1.931124in}}%
\pgfpathcurveto{\pgfqpoint{1.504204in}{1.920074in}}{\pgfqpoint{1.508595in}{1.909475in}}{\pgfqpoint{1.516408in}{1.901661in}}%
\pgfpathcurveto{\pgfqpoint{1.524222in}{1.893848in}}{\pgfqpoint{1.534821in}{1.889457in}}{\pgfqpoint{1.545871in}{1.889457in}}%
\pgfpathlineto{\pgfqpoint{1.545871in}{1.889457in}}%
\pgfpathclose%
\pgfusepath{stroke}%
\end{pgfscope}%
\begin{pgfscope}%
\pgfpathrectangle{\pgfqpoint{0.494722in}{0.437222in}}{\pgfqpoint{6.275590in}{5.159444in}}%
\pgfusepath{clip}%
\pgfsetbuttcap%
\pgfsetroundjoin%
\pgfsetlinewidth{1.003750pt}%
\definecolor{currentstroke}{rgb}{0.827451,0.827451,0.827451}%
\pgfsetstrokecolor{currentstroke}%
\pgfsetstrokeopacity{0.800000}%
\pgfsetdash{}{0pt}%
\pgfpathmoveto{\pgfqpoint{0.520093in}{4.158669in}}%
\pgfpathcurveto{\pgfqpoint{0.531143in}{4.158669in}}{\pgfqpoint{0.541742in}{4.163059in}}{\pgfqpoint{0.549556in}{4.170873in}}%
\pgfpathcurveto{\pgfqpoint{0.557369in}{4.178687in}}{\pgfqpoint{0.561759in}{4.189286in}}{\pgfqpoint{0.561759in}{4.200336in}}%
\pgfpathcurveto{\pgfqpoint{0.561759in}{4.211386in}}{\pgfqpoint{0.557369in}{4.221985in}}{\pgfqpoint{0.549556in}{4.229799in}}%
\pgfpathcurveto{\pgfqpoint{0.541742in}{4.237612in}}{\pgfqpoint{0.531143in}{4.242003in}}{\pgfqpoint{0.520093in}{4.242003in}}%
\pgfpathcurveto{\pgfqpoint{0.509043in}{4.242003in}}{\pgfqpoint{0.498444in}{4.237612in}}{\pgfqpoint{0.490630in}{4.229799in}}%
\pgfpathcurveto{\pgfqpoint{0.482816in}{4.221985in}}{\pgfqpoint{0.478426in}{4.211386in}}{\pgfqpoint{0.478426in}{4.200336in}}%
\pgfpathcurveto{\pgfqpoint{0.478426in}{4.189286in}}{\pgfqpoint{0.482816in}{4.178687in}}{\pgfqpoint{0.490630in}{4.170873in}}%
\pgfpathcurveto{\pgfqpoint{0.498444in}{4.163059in}}{\pgfqpoint{0.509043in}{4.158669in}}{\pgfqpoint{0.520093in}{4.158669in}}%
\pgfpathlineto{\pgfqpoint{0.520093in}{4.158669in}}%
\pgfpathclose%
\pgfusepath{stroke}%
\end{pgfscope}%
\begin{pgfscope}%
\pgfpathrectangle{\pgfqpoint{0.494722in}{0.437222in}}{\pgfqpoint{6.275590in}{5.159444in}}%
\pgfusepath{clip}%
\pgfsetbuttcap%
\pgfsetroundjoin%
\pgfsetlinewidth{1.003750pt}%
\definecolor{currentstroke}{rgb}{0.827451,0.827451,0.827451}%
\pgfsetstrokecolor{currentstroke}%
\pgfsetstrokeopacity{0.800000}%
\pgfsetdash{}{0pt}%
\pgfpathmoveto{\pgfqpoint{1.396743in}{2.042450in}}%
\pgfpathcurveto{\pgfqpoint{1.407793in}{2.042450in}}{\pgfqpoint{1.418392in}{2.046840in}}{\pgfqpoint{1.426206in}{2.054654in}}%
\pgfpathcurveto{\pgfqpoint{1.434019in}{2.062467in}}{\pgfqpoint{1.438409in}{2.073066in}}{\pgfqpoint{1.438409in}{2.084117in}}%
\pgfpathcurveto{\pgfqpoint{1.438409in}{2.095167in}}{\pgfqpoint{1.434019in}{2.105766in}}{\pgfqpoint{1.426206in}{2.113579in}}%
\pgfpathcurveto{\pgfqpoint{1.418392in}{2.121393in}}{\pgfqpoint{1.407793in}{2.125783in}}{\pgfqpoint{1.396743in}{2.125783in}}%
\pgfpathcurveto{\pgfqpoint{1.385693in}{2.125783in}}{\pgfqpoint{1.375094in}{2.121393in}}{\pgfqpoint{1.367280in}{2.113579in}}%
\pgfpathcurveto{\pgfqpoint{1.359466in}{2.105766in}}{\pgfqpoint{1.355076in}{2.095167in}}{\pgfqpoint{1.355076in}{2.084117in}}%
\pgfpathcurveto{\pgfqpoint{1.355076in}{2.073066in}}{\pgfqpoint{1.359466in}{2.062467in}}{\pgfqpoint{1.367280in}{2.054654in}}%
\pgfpathcurveto{\pgfqpoint{1.375094in}{2.046840in}}{\pgfqpoint{1.385693in}{2.042450in}}{\pgfqpoint{1.396743in}{2.042450in}}%
\pgfpathlineto{\pgfqpoint{1.396743in}{2.042450in}}%
\pgfpathclose%
\pgfusepath{stroke}%
\end{pgfscope}%
\begin{pgfscope}%
\pgfpathrectangle{\pgfqpoint{0.494722in}{0.437222in}}{\pgfqpoint{6.275590in}{5.159444in}}%
\pgfusepath{clip}%
\pgfsetbuttcap%
\pgfsetroundjoin%
\pgfsetlinewidth{1.003750pt}%
\definecolor{currentstroke}{rgb}{0.827451,0.827451,0.827451}%
\pgfsetstrokecolor{currentstroke}%
\pgfsetstrokeopacity{0.800000}%
\pgfsetdash{}{0pt}%
\pgfpathmoveto{\pgfqpoint{0.637564in}{3.549862in}}%
\pgfpathcurveto{\pgfqpoint{0.648614in}{3.549862in}}{\pgfqpoint{0.659214in}{3.554252in}}{\pgfqpoint{0.667027in}{3.562065in}}%
\pgfpathcurveto{\pgfqpoint{0.674841in}{3.569879in}}{\pgfqpoint{0.679231in}{3.580478in}}{\pgfqpoint{0.679231in}{3.591528in}}%
\pgfpathcurveto{\pgfqpoint{0.679231in}{3.602578in}}{\pgfqpoint{0.674841in}{3.613177in}}{\pgfqpoint{0.667027in}{3.620991in}}%
\pgfpathcurveto{\pgfqpoint{0.659214in}{3.628805in}}{\pgfqpoint{0.648614in}{3.633195in}}{\pgfqpoint{0.637564in}{3.633195in}}%
\pgfpathcurveto{\pgfqpoint{0.626514in}{3.633195in}}{\pgfqpoint{0.615915in}{3.628805in}}{\pgfqpoint{0.608102in}{3.620991in}}%
\pgfpathcurveto{\pgfqpoint{0.600288in}{3.613177in}}{\pgfqpoint{0.595898in}{3.602578in}}{\pgfqpoint{0.595898in}{3.591528in}}%
\pgfpathcurveto{\pgfqpoint{0.595898in}{3.580478in}}{\pgfqpoint{0.600288in}{3.569879in}}{\pgfqpoint{0.608102in}{3.562065in}}%
\pgfpathcurveto{\pgfqpoint{0.615915in}{3.554252in}}{\pgfqpoint{0.626514in}{3.549862in}}{\pgfqpoint{0.637564in}{3.549862in}}%
\pgfpathlineto{\pgfqpoint{0.637564in}{3.549862in}}%
\pgfpathclose%
\pgfusepath{stroke}%
\end{pgfscope}%
\begin{pgfscope}%
\pgfpathrectangle{\pgfqpoint{0.494722in}{0.437222in}}{\pgfqpoint{6.275590in}{5.159444in}}%
\pgfusepath{clip}%
\pgfsetbuttcap%
\pgfsetroundjoin%
\pgfsetlinewidth{1.003750pt}%
\definecolor{currentstroke}{rgb}{0.827451,0.827451,0.827451}%
\pgfsetstrokecolor{currentstroke}%
\pgfsetstrokeopacity{0.800000}%
\pgfsetdash{}{0pt}%
\pgfpathmoveto{\pgfqpoint{1.346241in}{2.100145in}}%
\pgfpathcurveto{\pgfqpoint{1.357291in}{2.100145in}}{\pgfqpoint{1.367890in}{2.104535in}}{\pgfqpoint{1.375703in}{2.112349in}}%
\pgfpathcurveto{\pgfqpoint{1.383517in}{2.120163in}}{\pgfqpoint{1.387907in}{2.130762in}}{\pgfqpoint{1.387907in}{2.141812in}}%
\pgfpathcurveto{\pgfqpoint{1.387907in}{2.152862in}}{\pgfqpoint{1.383517in}{2.163461in}}{\pgfqpoint{1.375703in}{2.171274in}}%
\pgfpathcurveto{\pgfqpoint{1.367890in}{2.179088in}}{\pgfqpoint{1.357291in}{2.183478in}}{\pgfqpoint{1.346241in}{2.183478in}}%
\pgfpathcurveto{\pgfqpoint{1.335191in}{2.183478in}}{\pgfqpoint{1.324592in}{2.179088in}}{\pgfqpoint{1.316778in}{2.171274in}}%
\pgfpathcurveto{\pgfqpoint{1.308964in}{2.163461in}}{\pgfqpoint{1.304574in}{2.152862in}}{\pgfqpoint{1.304574in}{2.141812in}}%
\pgfpathcurveto{\pgfqpoint{1.304574in}{2.130762in}}{\pgfqpoint{1.308964in}{2.120163in}}{\pgfqpoint{1.316778in}{2.112349in}}%
\pgfpathcurveto{\pgfqpoint{1.324592in}{2.104535in}}{\pgfqpoint{1.335191in}{2.100145in}}{\pgfqpoint{1.346241in}{2.100145in}}%
\pgfpathlineto{\pgfqpoint{1.346241in}{2.100145in}}%
\pgfpathclose%
\pgfusepath{stroke}%
\end{pgfscope}%
\begin{pgfscope}%
\pgfpathrectangle{\pgfqpoint{0.494722in}{0.437222in}}{\pgfqpoint{6.275590in}{5.159444in}}%
\pgfusepath{clip}%
\pgfsetbuttcap%
\pgfsetroundjoin%
\pgfsetlinewidth{1.003750pt}%
\definecolor{currentstroke}{rgb}{0.827451,0.827451,0.827451}%
\pgfsetstrokecolor{currentstroke}%
\pgfsetstrokeopacity{0.800000}%
\pgfsetdash{}{0pt}%
\pgfpathmoveto{\pgfqpoint{0.679488in}{3.431236in}}%
\pgfpathcurveto{\pgfqpoint{0.690538in}{3.431236in}}{\pgfqpoint{0.701137in}{3.435626in}}{\pgfqpoint{0.708951in}{3.443440in}}%
\pgfpathcurveto{\pgfqpoint{0.716765in}{3.451253in}}{\pgfqpoint{0.721155in}{3.461852in}}{\pgfqpoint{0.721155in}{3.472902in}}%
\pgfpathcurveto{\pgfqpoint{0.721155in}{3.483953in}}{\pgfqpoint{0.716765in}{3.494552in}}{\pgfqpoint{0.708951in}{3.502365in}}%
\pgfpathcurveto{\pgfqpoint{0.701137in}{3.510179in}}{\pgfqpoint{0.690538in}{3.514569in}}{\pgfqpoint{0.679488in}{3.514569in}}%
\pgfpathcurveto{\pgfqpoint{0.668438in}{3.514569in}}{\pgfqpoint{0.657839in}{3.510179in}}{\pgfqpoint{0.650026in}{3.502365in}}%
\pgfpathcurveto{\pgfqpoint{0.642212in}{3.494552in}}{\pgfqpoint{0.637822in}{3.483953in}}{\pgfqpoint{0.637822in}{3.472902in}}%
\pgfpathcurveto{\pgfqpoint{0.637822in}{3.461852in}}{\pgfqpoint{0.642212in}{3.451253in}}{\pgfqpoint{0.650026in}{3.443440in}}%
\pgfpathcurveto{\pgfqpoint{0.657839in}{3.435626in}}{\pgfqpoint{0.668438in}{3.431236in}}{\pgfqpoint{0.679488in}{3.431236in}}%
\pgfpathlineto{\pgfqpoint{0.679488in}{3.431236in}}%
\pgfpathclose%
\pgfusepath{stroke}%
\end{pgfscope}%
\begin{pgfscope}%
\pgfpathrectangle{\pgfqpoint{0.494722in}{0.437222in}}{\pgfqpoint{6.275590in}{5.159444in}}%
\pgfusepath{clip}%
\pgfsetbuttcap%
\pgfsetroundjoin%
\pgfsetlinewidth{1.003750pt}%
\definecolor{currentstroke}{rgb}{0.827451,0.827451,0.827451}%
\pgfsetstrokecolor{currentstroke}%
\pgfsetstrokeopacity{0.800000}%
\pgfsetdash{}{0pt}%
\pgfpathmoveto{\pgfqpoint{3.950409in}{0.627171in}}%
\pgfpathcurveto{\pgfqpoint{3.961459in}{0.627171in}}{\pgfqpoint{3.972058in}{0.631561in}}{\pgfqpoint{3.979872in}{0.639374in}}%
\pgfpathcurveto{\pgfqpoint{3.987685in}{0.647188in}}{\pgfqpoint{3.992076in}{0.657787in}}{\pgfqpoint{3.992076in}{0.668837in}}%
\pgfpathcurveto{\pgfqpoint{3.992076in}{0.679887in}}{\pgfqpoint{3.987685in}{0.690486in}}{\pgfqpoint{3.979872in}{0.698300in}}%
\pgfpathcurveto{\pgfqpoint{3.972058in}{0.706114in}}{\pgfqpoint{3.961459in}{0.710504in}}{\pgfqpoint{3.950409in}{0.710504in}}%
\pgfpathcurveto{\pgfqpoint{3.939359in}{0.710504in}}{\pgfqpoint{3.928760in}{0.706114in}}{\pgfqpoint{3.920946in}{0.698300in}}%
\pgfpathcurveto{\pgfqpoint{3.913133in}{0.690486in}}{\pgfqpoint{3.908742in}{0.679887in}}{\pgfqpoint{3.908742in}{0.668837in}}%
\pgfpathcurveto{\pgfqpoint{3.908742in}{0.657787in}}{\pgfqpoint{3.913133in}{0.647188in}}{\pgfqpoint{3.920946in}{0.639374in}}%
\pgfpathcurveto{\pgfqpoint{3.928760in}{0.631561in}}{\pgfqpoint{3.939359in}{0.627171in}}{\pgfqpoint{3.950409in}{0.627171in}}%
\pgfpathlineto{\pgfqpoint{3.950409in}{0.627171in}}%
\pgfpathclose%
\pgfusepath{stroke}%
\end{pgfscope}%
\begin{pgfscope}%
\pgfpathrectangle{\pgfqpoint{0.494722in}{0.437222in}}{\pgfqpoint{6.275590in}{5.159444in}}%
\pgfusepath{clip}%
\pgfsetbuttcap%
\pgfsetroundjoin%
\pgfsetlinewidth{1.003750pt}%
\definecolor{currentstroke}{rgb}{0.827451,0.827451,0.827451}%
\pgfsetstrokecolor{currentstroke}%
\pgfsetstrokeopacity{0.800000}%
\pgfsetdash{}{0pt}%
\pgfpathmoveto{\pgfqpoint{4.942939in}{0.453295in}}%
\pgfpathcurveto{\pgfqpoint{4.953989in}{0.453295in}}{\pgfqpoint{4.964589in}{0.457685in}}{\pgfqpoint{4.972402in}{0.465499in}}%
\pgfpathcurveto{\pgfqpoint{4.980216in}{0.473313in}}{\pgfqpoint{4.984606in}{0.483912in}}{\pgfqpoint{4.984606in}{0.494962in}}%
\pgfpathcurveto{\pgfqpoint{4.984606in}{0.506012in}}{\pgfqpoint{4.980216in}{0.516611in}}{\pgfqpoint{4.972402in}{0.524425in}}%
\pgfpathcurveto{\pgfqpoint{4.964589in}{0.532238in}}{\pgfqpoint{4.953989in}{0.536628in}}{\pgfqpoint{4.942939in}{0.536628in}}%
\pgfpathcurveto{\pgfqpoint{4.931889in}{0.536628in}}{\pgfqpoint{4.921290in}{0.532238in}}{\pgfqpoint{4.913477in}{0.524425in}}%
\pgfpathcurveto{\pgfqpoint{4.905663in}{0.516611in}}{\pgfqpoint{4.901273in}{0.506012in}}{\pgfqpoint{4.901273in}{0.494962in}}%
\pgfpathcurveto{\pgfqpoint{4.901273in}{0.483912in}}{\pgfqpoint{4.905663in}{0.473313in}}{\pgfqpoint{4.913477in}{0.465499in}}%
\pgfpathcurveto{\pgfqpoint{4.921290in}{0.457685in}}{\pgfqpoint{4.931889in}{0.453295in}}{\pgfqpoint{4.942939in}{0.453295in}}%
\pgfpathlineto{\pgfqpoint{4.942939in}{0.453295in}}%
\pgfpathclose%
\pgfusepath{stroke}%
\end{pgfscope}%
\begin{pgfscope}%
\pgfpathrectangle{\pgfqpoint{0.494722in}{0.437222in}}{\pgfqpoint{6.275590in}{5.159444in}}%
\pgfusepath{clip}%
\pgfsetbuttcap%
\pgfsetroundjoin%
\pgfsetlinewidth{1.003750pt}%
\definecolor{currentstroke}{rgb}{0.827451,0.827451,0.827451}%
\pgfsetstrokecolor{currentstroke}%
\pgfsetstrokeopacity{0.800000}%
\pgfsetdash{}{0pt}%
\pgfpathmoveto{\pgfqpoint{1.926347in}{1.574489in}}%
\pgfpathcurveto{\pgfqpoint{1.937397in}{1.574489in}}{\pgfqpoint{1.947996in}{1.578879in}}{\pgfqpoint{1.955809in}{1.586693in}}%
\pgfpathcurveto{\pgfqpoint{1.963623in}{1.594506in}}{\pgfqpoint{1.968013in}{1.605105in}}{\pgfqpoint{1.968013in}{1.616155in}}%
\pgfpathcurveto{\pgfqpoint{1.968013in}{1.627206in}}{\pgfqpoint{1.963623in}{1.637805in}}{\pgfqpoint{1.955809in}{1.645618in}}%
\pgfpathcurveto{\pgfqpoint{1.947996in}{1.653432in}}{\pgfqpoint{1.937397in}{1.657822in}}{\pgfqpoint{1.926347in}{1.657822in}}%
\pgfpathcurveto{\pgfqpoint{1.915296in}{1.657822in}}{\pgfqpoint{1.904697in}{1.653432in}}{\pgfqpoint{1.896884in}{1.645618in}}%
\pgfpathcurveto{\pgfqpoint{1.889070in}{1.637805in}}{\pgfqpoint{1.884680in}{1.627206in}}{\pgfqpoint{1.884680in}{1.616155in}}%
\pgfpathcurveto{\pgfqpoint{1.884680in}{1.605105in}}{\pgfqpoint{1.889070in}{1.594506in}}{\pgfqpoint{1.896884in}{1.586693in}}%
\pgfpathcurveto{\pgfqpoint{1.904697in}{1.578879in}}{\pgfqpoint{1.915296in}{1.574489in}}{\pgfqpoint{1.926347in}{1.574489in}}%
\pgfpathlineto{\pgfqpoint{1.926347in}{1.574489in}}%
\pgfpathclose%
\pgfusepath{stroke}%
\end{pgfscope}%
\begin{pgfscope}%
\pgfpathrectangle{\pgfqpoint{0.494722in}{0.437222in}}{\pgfqpoint{6.275590in}{5.159444in}}%
\pgfusepath{clip}%
\pgfsetbuttcap%
\pgfsetroundjoin%
\pgfsetlinewidth{1.003750pt}%
\definecolor{currentstroke}{rgb}{0.827451,0.827451,0.827451}%
\pgfsetstrokecolor{currentstroke}%
\pgfsetstrokeopacity{0.800000}%
\pgfsetdash{}{0pt}%
\pgfpathmoveto{\pgfqpoint{1.766183in}{1.687308in}}%
\pgfpathcurveto{\pgfqpoint{1.777233in}{1.687308in}}{\pgfqpoint{1.787832in}{1.691698in}}{\pgfqpoint{1.795646in}{1.699512in}}%
\pgfpathcurveto{\pgfqpoint{1.803459in}{1.707325in}}{\pgfqpoint{1.807850in}{1.717924in}}{\pgfqpoint{1.807850in}{1.728974in}}%
\pgfpathcurveto{\pgfqpoint{1.807850in}{1.740025in}}{\pgfqpoint{1.803459in}{1.750624in}}{\pgfqpoint{1.795646in}{1.758437in}}%
\pgfpathcurveto{\pgfqpoint{1.787832in}{1.766251in}}{\pgfqpoint{1.777233in}{1.770641in}}{\pgfqpoint{1.766183in}{1.770641in}}%
\pgfpathcurveto{\pgfqpoint{1.755133in}{1.770641in}}{\pgfqpoint{1.744534in}{1.766251in}}{\pgfqpoint{1.736720in}{1.758437in}}%
\pgfpathcurveto{\pgfqpoint{1.728907in}{1.750624in}}{\pgfqpoint{1.724516in}{1.740025in}}{\pgfqpoint{1.724516in}{1.728974in}}%
\pgfpathcurveto{\pgfqpoint{1.724516in}{1.717924in}}{\pgfqpoint{1.728907in}{1.707325in}}{\pgfqpoint{1.736720in}{1.699512in}}%
\pgfpathcurveto{\pgfqpoint{1.744534in}{1.691698in}}{\pgfqpoint{1.755133in}{1.687308in}}{\pgfqpoint{1.766183in}{1.687308in}}%
\pgfpathlineto{\pgfqpoint{1.766183in}{1.687308in}}%
\pgfpathclose%
\pgfusepath{stroke}%
\end{pgfscope}%
\begin{pgfscope}%
\pgfpathrectangle{\pgfqpoint{0.494722in}{0.437222in}}{\pgfqpoint{6.275590in}{5.159444in}}%
\pgfusepath{clip}%
\pgfsetbuttcap%
\pgfsetroundjoin%
\pgfsetlinewidth{1.003750pt}%
\definecolor{currentstroke}{rgb}{0.827451,0.827451,0.827451}%
\pgfsetstrokecolor{currentstroke}%
\pgfsetstrokeopacity{0.800000}%
\pgfsetdash{}{0pt}%
\pgfpathmoveto{\pgfqpoint{0.750952in}{3.177253in}}%
\pgfpathcurveto{\pgfqpoint{0.762002in}{3.177253in}}{\pgfqpoint{0.772601in}{3.181643in}}{\pgfqpoint{0.780414in}{3.189457in}}%
\pgfpathcurveto{\pgfqpoint{0.788228in}{3.197270in}}{\pgfqpoint{0.792618in}{3.207870in}}{\pgfqpoint{0.792618in}{3.218920in}}%
\pgfpathcurveto{\pgfqpoint{0.792618in}{3.229970in}}{\pgfqpoint{0.788228in}{3.240569in}}{\pgfqpoint{0.780414in}{3.248382in}}%
\pgfpathcurveto{\pgfqpoint{0.772601in}{3.256196in}}{\pgfqpoint{0.762002in}{3.260586in}}{\pgfqpoint{0.750952in}{3.260586in}}%
\pgfpathcurveto{\pgfqpoint{0.739901in}{3.260586in}}{\pgfqpoint{0.729302in}{3.256196in}}{\pgfqpoint{0.721489in}{3.248382in}}%
\pgfpathcurveto{\pgfqpoint{0.713675in}{3.240569in}}{\pgfqpoint{0.709285in}{3.229970in}}{\pgfqpoint{0.709285in}{3.218920in}}%
\pgfpathcurveto{\pgfqpoint{0.709285in}{3.207870in}}{\pgfqpoint{0.713675in}{3.197270in}}{\pgfqpoint{0.721489in}{3.189457in}}%
\pgfpathcurveto{\pgfqpoint{0.729302in}{3.181643in}}{\pgfqpoint{0.739901in}{3.177253in}}{\pgfqpoint{0.750952in}{3.177253in}}%
\pgfpathlineto{\pgfqpoint{0.750952in}{3.177253in}}%
\pgfpathclose%
\pgfusepath{stroke}%
\end{pgfscope}%
\begin{pgfscope}%
\pgfpathrectangle{\pgfqpoint{0.494722in}{0.437222in}}{\pgfqpoint{6.275590in}{5.159444in}}%
\pgfusepath{clip}%
\pgfsetbuttcap%
\pgfsetroundjoin%
\pgfsetlinewidth{1.003750pt}%
\definecolor{currentstroke}{rgb}{0.827451,0.827451,0.827451}%
\pgfsetstrokecolor{currentstroke}%
\pgfsetstrokeopacity{0.800000}%
\pgfsetdash{}{0pt}%
\pgfpathmoveto{\pgfqpoint{0.974123in}{2.860396in}}%
\pgfpathcurveto{\pgfqpoint{0.985174in}{2.860396in}}{\pgfqpoint{0.995773in}{2.864786in}}{\pgfqpoint{1.003586in}{2.872600in}}%
\pgfpathcurveto{\pgfqpoint{1.011400in}{2.880414in}}{\pgfqpoint{1.015790in}{2.891013in}}{\pgfqpoint{1.015790in}{2.902063in}}%
\pgfpathcurveto{\pgfqpoint{1.015790in}{2.913113in}}{\pgfqpoint{1.011400in}{2.923712in}}{\pgfqpoint{1.003586in}{2.931526in}}%
\pgfpathcurveto{\pgfqpoint{0.995773in}{2.939339in}}{\pgfqpoint{0.985174in}{2.943729in}}{\pgfqpoint{0.974123in}{2.943729in}}%
\pgfpathcurveto{\pgfqpoint{0.963073in}{2.943729in}}{\pgfqpoint{0.952474in}{2.939339in}}{\pgfqpoint{0.944661in}{2.931526in}}%
\pgfpathcurveto{\pgfqpoint{0.936847in}{2.923712in}}{\pgfqpoint{0.932457in}{2.913113in}}{\pgfqpoint{0.932457in}{2.902063in}}%
\pgfpathcurveto{\pgfqpoint{0.932457in}{2.891013in}}{\pgfqpoint{0.936847in}{2.880414in}}{\pgfqpoint{0.944661in}{2.872600in}}%
\pgfpathcurveto{\pgfqpoint{0.952474in}{2.864786in}}{\pgfqpoint{0.963073in}{2.860396in}}{\pgfqpoint{0.974123in}{2.860396in}}%
\pgfpathlineto{\pgfqpoint{0.974123in}{2.860396in}}%
\pgfpathclose%
\pgfusepath{stroke}%
\end{pgfscope}%
\begin{pgfscope}%
\pgfpathrectangle{\pgfqpoint{0.494722in}{0.437222in}}{\pgfqpoint{6.275590in}{5.159444in}}%
\pgfusepath{clip}%
\pgfsetbuttcap%
\pgfsetroundjoin%
\pgfsetlinewidth{1.003750pt}%
\definecolor{currentstroke}{rgb}{0.827451,0.827451,0.827451}%
\pgfsetstrokecolor{currentstroke}%
\pgfsetstrokeopacity{0.800000}%
\pgfsetdash{}{0pt}%
\pgfpathmoveto{\pgfqpoint{1.099988in}{2.633956in}}%
\pgfpathcurveto{\pgfqpoint{1.111038in}{2.633956in}}{\pgfqpoint{1.121637in}{2.638346in}}{\pgfqpoint{1.129451in}{2.646160in}}%
\pgfpathcurveto{\pgfqpoint{1.137265in}{2.653974in}}{\pgfqpoint{1.141655in}{2.664573in}}{\pgfqpoint{1.141655in}{2.675623in}}%
\pgfpathcurveto{\pgfqpoint{1.141655in}{2.686673in}}{\pgfqpoint{1.137265in}{2.697272in}}{\pgfqpoint{1.129451in}{2.705086in}}%
\pgfpathcurveto{\pgfqpoint{1.121637in}{2.712899in}}{\pgfqpoint{1.111038in}{2.717290in}}{\pgfqpoint{1.099988in}{2.717290in}}%
\pgfpathcurveto{\pgfqpoint{1.088938in}{2.717290in}}{\pgfqpoint{1.078339in}{2.712899in}}{\pgfqpoint{1.070525in}{2.705086in}}%
\pgfpathcurveto{\pgfqpoint{1.062712in}{2.697272in}}{\pgfqpoint{1.058322in}{2.686673in}}{\pgfqpoint{1.058322in}{2.675623in}}%
\pgfpathcurveto{\pgfqpoint{1.058322in}{2.664573in}}{\pgfqpoint{1.062712in}{2.653974in}}{\pgfqpoint{1.070525in}{2.646160in}}%
\pgfpathcurveto{\pgfqpoint{1.078339in}{2.638346in}}{\pgfqpoint{1.088938in}{2.633956in}}{\pgfqpoint{1.099988in}{2.633956in}}%
\pgfpathlineto{\pgfqpoint{1.099988in}{2.633956in}}%
\pgfpathclose%
\pgfusepath{stroke}%
\end{pgfscope}%
\begin{pgfscope}%
\pgfpathrectangle{\pgfqpoint{0.494722in}{0.437222in}}{\pgfqpoint{6.275590in}{5.159444in}}%
\pgfusepath{clip}%
\pgfsetbuttcap%
\pgfsetroundjoin%
\pgfsetlinewidth{1.003750pt}%
\definecolor{currentstroke}{rgb}{0.827451,0.827451,0.827451}%
\pgfsetstrokecolor{currentstroke}%
\pgfsetstrokeopacity{0.800000}%
\pgfsetdash{}{0pt}%
\pgfpathmoveto{\pgfqpoint{4.970960in}{0.441718in}}%
\pgfpathcurveto{\pgfqpoint{4.982010in}{0.441718in}}{\pgfqpoint{4.992609in}{0.446108in}}{\pgfqpoint{5.000423in}{0.453922in}}%
\pgfpathcurveto{\pgfqpoint{5.008236in}{0.461735in}}{\pgfqpoint{5.012626in}{0.472334in}}{\pgfqpoint{5.012626in}{0.483385in}}%
\pgfpathcurveto{\pgfqpoint{5.012626in}{0.494435in}}{\pgfqpoint{5.008236in}{0.505034in}}{\pgfqpoint{5.000423in}{0.512847in}}%
\pgfpathcurveto{\pgfqpoint{4.992609in}{0.520661in}}{\pgfqpoint{4.982010in}{0.525051in}}{\pgfqpoint{4.970960in}{0.525051in}}%
\pgfpathcurveto{\pgfqpoint{4.959910in}{0.525051in}}{\pgfqpoint{4.949311in}{0.520661in}}{\pgfqpoint{4.941497in}{0.512847in}}%
\pgfpathcurveto{\pgfqpoint{4.933683in}{0.505034in}}{\pgfqpoint{4.929293in}{0.494435in}}{\pgfqpoint{4.929293in}{0.483385in}}%
\pgfpathcurveto{\pgfqpoint{4.929293in}{0.472334in}}{\pgfqpoint{4.933683in}{0.461735in}}{\pgfqpoint{4.941497in}{0.453922in}}%
\pgfpathcurveto{\pgfqpoint{4.949311in}{0.446108in}}{\pgfqpoint{4.959910in}{0.441718in}}{\pgfqpoint{4.970960in}{0.441718in}}%
\pgfpathlineto{\pgfqpoint{4.970960in}{0.441718in}}%
\pgfpathclose%
\pgfusepath{stroke}%
\end{pgfscope}%
\begin{pgfscope}%
\pgfpathrectangle{\pgfqpoint{0.494722in}{0.437222in}}{\pgfqpoint{6.275590in}{5.159444in}}%
\pgfusepath{clip}%
\pgfsetbuttcap%
\pgfsetroundjoin%
\pgfsetlinewidth{1.003750pt}%
\definecolor{currentstroke}{rgb}{0.827451,0.827451,0.827451}%
\pgfsetstrokecolor{currentstroke}%
\pgfsetstrokeopacity{0.800000}%
\pgfsetdash{}{0pt}%
\pgfpathmoveto{\pgfqpoint{1.381781in}{2.056420in}}%
\pgfpathcurveto{\pgfqpoint{1.392831in}{2.056420in}}{\pgfqpoint{1.403430in}{2.060810in}}{\pgfqpoint{1.411244in}{2.068624in}}%
\pgfpathcurveto{\pgfqpoint{1.419057in}{2.076437in}}{\pgfqpoint{1.423448in}{2.087036in}}{\pgfqpoint{1.423448in}{2.098087in}}%
\pgfpathcurveto{\pgfqpoint{1.423448in}{2.109137in}}{\pgfqpoint{1.419057in}{2.119736in}}{\pgfqpoint{1.411244in}{2.127549in}}%
\pgfpathcurveto{\pgfqpoint{1.403430in}{2.135363in}}{\pgfqpoint{1.392831in}{2.139753in}}{\pgfqpoint{1.381781in}{2.139753in}}%
\pgfpathcurveto{\pgfqpoint{1.370731in}{2.139753in}}{\pgfqpoint{1.360132in}{2.135363in}}{\pgfqpoint{1.352318in}{2.127549in}}%
\pgfpathcurveto{\pgfqpoint{1.344505in}{2.119736in}}{\pgfqpoint{1.340114in}{2.109137in}}{\pgfqpoint{1.340114in}{2.098087in}}%
\pgfpathcurveto{\pgfqpoint{1.340114in}{2.087036in}}{\pgfqpoint{1.344505in}{2.076437in}}{\pgfqpoint{1.352318in}{2.068624in}}%
\pgfpathcurveto{\pgfqpoint{1.360132in}{2.060810in}}{\pgfqpoint{1.370731in}{2.056420in}}{\pgfqpoint{1.381781in}{2.056420in}}%
\pgfpathlineto{\pgfqpoint{1.381781in}{2.056420in}}%
\pgfpathclose%
\pgfusepath{stroke}%
\end{pgfscope}%
\begin{pgfscope}%
\pgfpathrectangle{\pgfqpoint{0.494722in}{0.437222in}}{\pgfqpoint{6.275590in}{5.159444in}}%
\pgfusepath{clip}%
\pgfsetbuttcap%
\pgfsetroundjoin%
\pgfsetlinewidth{1.003750pt}%
\definecolor{currentstroke}{rgb}{0.827451,0.827451,0.827451}%
\pgfsetstrokecolor{currentstroke}%
\pgfsetstrokeopacity{0.800000}%
\pgfsetdash{}{0pt}%
\pgfpathmoveto{\pgfqpoint{1.786945in}{1.657846in}}%
\pgfpathcurveto{\pgfqpoint{1.797995in}{1.657846in}}{\pgfqpoint{1.808594in}{1.662237in}}{\pgfqpoint{1.816407in}{1.670050in}}%
\pgfpathcurveto{\pgfqpoint{1.824221in}{1.677864in}}{\pgfqpoint{1.828611in}{1.688463in}}{\pgfqpoint{1.828611in}{1.699513in}}%
\pgfpathcurveto{\pgfqpoint{1.828611in}{1.710563in}}{\pgfqpoint{1.824221in}{1.721162in}}{\pgfqpoint{1.816407in}{1.728976in}}%
\pgfpathcurveto{\pgfqpoint{1.808594in}{1.736789in}}{\pgfqpoint{1.797995in}{1.741180in}}{\pgfqpoint{1.786945in}{1.741180in}}%
\pgfpathcurveto{\pgfqpoint{1.775894in}{1.741180in}}{\pgfqpoint{1.765295in}{1.736789in}}{\pgfqpoint{1.757482in}{1.728976in}}%
\pgfpathcurveto{\pgfqpoint{1.749668in}{1.721162in}}{\pgfqpoint{1.745278in}{1.710563in}}{\pgfqpoint{1.745278in}{1.699513in}}%
\pgfpathcurveto{\pgfqpoint{1.745278in}{1.688463in}}{\pgfqpoint{1.749668in}{1.677864in}}{\pgfqpoint{1.757482in}{1.670050in}}%
\pgfpathcurveto{\pgfqpoint{1.765295in}{1.662237in}}{\pgfqpoint{1.775894in}{1.657846in}}{\pgfqpoint{1.786945in}{1.657846in}}%
\pgfpathlineto{\pgfqpoint{1.786945in}{1.657846in}}%
\pgfpathclose%
\pgfusepath{stroke}%
\end{pgfscope}%
\begin{pgfscope}%
\pgfpathrectangle{\pgfqpoint{0.494722in}{0.437222in}}{\pgfqpoint{6.275590in}{5.159444in}}%
\pgfusepath{clip}%
\pgfsetbuttcap%
\pgfsetroundjoin%
\pgfsetlinewidth{1.003750pt}%
\definecolor{currentstroke}{rgb}{0.827451,0.827451,0.827451}%
\pgfsetstrokecolor{currentstroke}%
\pgfsetstrokeopacity{0.800000}%
\pgfsetdash{}{0pt}%
\pgfpathmoveto{\pgfqpoint{1.102768in}{2.568103in}}%
\pgfpathcurveto{\pgfqpoint{1.113818in}{2.568103in}}{\pgfqpoint{1.124417in}{2.572493in}}{\pgfqpoint{1.132231in}{2.580307in}}%
\pgfpathcurveto{\pgfqpoint{1.140045in}{2.588121in}}{\pgfqpoint{1.144435in}{2.598720in}}{\pgfqpoint{1.144435in}{2.609770in}}%
\pgfpathcurveto{\pgfqpoint{1.144435in}{2.620820in}}{\pgfqpoint{1.140045in}{2.631419in}}{\pgfqpoint{1.132231in}{2.639233in}}%
\pgfpathcurveto{\pgfqpoint{1.124417in}{2.647046in}}{\pgfqpoint{1.113818in}{2.651436in}}{\pgfqpoint{1.102768in}{2.651436in}}%
\pgfpathcurveto{\pgfqpoint{1.091718in}{2.651436in}}{\pgfqpoint{1.081119in}{2.647046in}}{\pgfqpoint{1.073306in}{2.639233in}}%
\pgfpathcurveto{\pgfqpoint{1.065492in}{2.631419in}}{\pgfqpoint{1.061102in}{2.620820in}}{\pgfqpoint{1.061102in}{2.609770in}}%
\pgfpathcurveto{\pgfqpoint{1.061102in}{2.598720in}}{\pgfqpoint{1.065492in}{2.588121in}}{\pgfqpoint{1.073306in}{2.580307in}}%
\pgfpathcurveto{\pgfqpoint{1.081119in}{2.572493in}}{\pgfqpoint{1.091718in}{2.568103in}}{\pgfqpoint{1.102768in}{2.568103in}}%
\pgfpathlineto{\pgfqpoint{1.102768in}{2.568103in}}%
\pgfpathclose%
\pgfusepath{stroke}%
\end{pgfscope}%
\begin{pgfscope}%
\pgfpathrectangle{\pgfqpoint{0.494722in}{0.437222in}}{\pgfqpoint{6.275590in}{5.159444in}}%
\pgfusepath{clip}%
\pgfsetbuttcap%
\pgfsetroundjoin%
\pgfsetlinewidth{1.003750pt}%
\definecolor{currentstroke}{rgb}{0.827451,0.827451,0.827451}%
\pgfsetstrokecolor{currentstroke}%
\pgfsetstrokeopacity{0.800000}%
\pgfsetdash{}{0pt}%
\pgfpathmoveto{\pgfqpoint{0.683182in}{3.327378in}}%
\pgfpathcurveto{\pgfqpoint{0.694232in}{3.327378in}}{\pgfqpoint{0.704831in}{3.331768in}}{\pgfqpoint{0.712645in}{3.339581in}}%
\pgfpathcurveto{\pgfqpoint{0.720459in}{3.347395in}}{\pgfqpoint{0.724849in}{3.357994in}}{\pgfqpoint{0.724849in}{3.369044in}}%
\pgfpathcurveto{\pgfqpoint{0.724849in}{3.380094in}}{\pgfqpoint{0.720459in}{3.390693in}}{\pgfqpoint{0.712645in}{3.398507in}}%
\pgfpathcurveto{\pgfqpoint{0.704831in}{3.406321in}}{\pgfqpoint{0.694232in}{3.410711in}}{\pgfqpoint{0.683182in}{3.410711in}}%
\pgfpathcurveto{\pgfqpoint{0.672132in}{3.410711in}}{\pgfqpoint{0.661533in}{3.406321in}}{\pgfqpoint{0.653720in}{3.398507in}}%
\pgfpathcurveto{\pgfqpoint{0.645906in}{3.390693in}}{\pgfqpoint{0.641516in}{3.380094in}}{\pgfqpoint{0.641516in}{3.369044in}}%
\pgfpathcurveto{\pgfqpoint{0.641516in}{3.357994in}}{\pgfqpoint{0.645906in}{3.347395in}}{\pgfqpoint{0.653720in}{3.339581in}}%
\pgfpathcurveto{\pgfqpoint{0.661533in}{3.331768in}}{\pgfqpoint{0.672132in}{3.327378in}}{\pgfqpoint{0.683182in}{3.327378in}}%
\pgfpathlineto{\pgfqpoint{0.683182in}{3.327378in}}%
\pgfpathclose%
\pgfusepath{stroke}%
\end{pgfscope}%
\begin{pgfscope}%
\pgfpathrectangle{\pgfqpoint{0.494722in}{0.437222in}}{\pgfqpoint{6.275590in}{5.159444in}}%
\pgfusepath{clip}%
\pgfsetbuttcap%
\pgfsetroundjoin%
\pgfsetlinewidth{1.003750pt}%
\definecolor{currentstroke}{rgb}{0.827451,0.827451,0.827451}%
\pgfsetstrokecolor{currentstroke}%
\pgfsetstrokeopacity{0.800000}%
\pgfsetdash{}{0pt}%
\pgfpathmoveto{\pgfqpoint{1.315544in}{2.144827in}}%
\pgfpathcurveto{\pgfqpoint{1.326594in}{2.144827in}}{\pgfqpoint{1.337193in}{2.149218in}}{\pgfqpoint{1.345006in}{2.157031in}}%
\pgfpathcurveto{\pgfqpoint{1.352820in}{2.164845in}}{\pgfqpoint{1.357210in}{2.175444in}}{\pgfqpoint{1.357210in}{2.186494in}}%
\pgfpathcurveto{\pgfqpoint{1.357210in}{2.197544in}}{\pgfqpoint{1.352820in}{2.208143in}}{\pgfqpoint{1.345006in}{2.215957in}}%
\pgfpathcurveto{\pgfqpoint{1.337193in}{2.223770in}}{\pgfqpoint{1.326594in}{2.228161in}}{\pgfqpoint{1.315544in}{2.228161in}}%
\pgfpathcurveto{\pgfqpoint{1.304494in}{2.228161in}}{\pgfqpoint{1.293894in}{2.223770in}}{\pgfqpoint{1.286081in}{2.215957in}}%
\pgfpathcurveto{\pgfqpoint{1.278267in}{2.208143in}}{\pgfqpoint{1.273877in}{2.197544in}}{\pgfqpoint{1.273877in}{2.186494in}}%
\pgfpathcurveto{\pgfqpoint{1.273877in}{2.175444in}}{\pgfqpoint{1.278267in}{2.164845in}}{\pgfqpoint{1.286081in}{2.157031in}}%
\pgfpathcurveto{\pgfqpoint{1.293894in}{2.149218in}}{\pgfqpoint{1.304494in}{2.144827in}}{\pgfqpoint{1.315544in}{2.144827in}}%
\pgfpathlineto{\pgfqpoint{1.315544in}{2.144827in}}%
\pgfpathclose%
\pgfusepath{stroke}%
\end{pgfscope}%
\begin{pgfscope}%
\pgfpathrectangle{\pgfqpoint{0.494722in}{0.437222in}}{\pgfqpoint{6.275590in}{5.159444in}}%
\pgfusepath{clip}%
\pgfsetbuttcap%
\pgfsetroundjoin%
\pgfsetlinewidth{1.003750pt}%
\definecolor{currentstroke}{rgb}{0.827451,0.827451,0.827451}%
\pgfsetstrokecolor{currentstroke}%
\pgfsetstrokeopacity{0.800000}%
\pgfsetdash{}{0pt}%
\pgfpathmoveto{\pgfqpoint{4.222858in}{0.545257in}}%
\pgfpathcurveto{\pgfqpoint{4.233908in}{0.545257in}}{\pgfqpoint{4.244507in}{0.549647in}}{\pgfqpoint{4.252321in}{0.557461in}}%
\pgfpathcurveto{\pgfqpoint{4.260134in}{0.565274in}}{\pgfqpoint{4.264525in}{0.575873in}}{\pgfqpoint{4.264525in}{0.586923in}}%
\pgfpathcurveto{\pgfqpoint{4.264525in}{0.597974in}}{\pgfqpoint{4.260134in}{0.608573in}}{\pgfqpoint{4.252321in}{0.616386in}}%
\pgfpathcurveto{\pgfqpoint{4.244507in}{0.624200in}}{\pgfqpoint{4.233908in}{0.628590in}}{\pgfqpoint{4.222858in}{0.628590in}}%
\pgfpathcurveto{\pgfqpoint{4.211808in}{0.628590in}}{\pgfqpoint{4.201209in}{0.624200in}}{\pgfqpoint{4.193395in}{0.616386in}}%
\pgfpathcurveto{\pgfqpoint{4.185581in}{0.608573in}}{\pgfqpoint{4.181191in}{0.597974in}}{\pgfqpoint{4.181191in}{0.586923in}}%
\pgfpathcurveto{\pgfqpoint{4.181191in}{0.575873in}}{\pgfqpoint{4.185581in}{0.565274in}}{\pgfqpoint{4.193395in}{0.557461in}}%
\pgfpathcurveto{\pgfqpoint{4.201209in}{0.549647in}}{\pgfqpoint{4.211808in}{0.545257in}}{\pgfqpoint{4.222858in}{0.545257in}}%
\pgfpathlineto{\pgfqpoint{4.222858in}{0.545257in}}%
\pgfpathclose%
\pgfusepath{stroke}%
\end{pgfscope}%
\begin{pgfscope}%
\pgfpathrectangle{\pgfqpoint{0.494722in}{0.437222in}}{\pgfqpoint{6.275590in}{5.159444in}}%
\pgfusepath{clip}%
\pgfsetbuttcap%
\pgfsetroundjoin%
\pgfsetlinewidth{1.003750pt}%
\definecolor{currentstroke}{rgb}{0.827451,0.827451,0.827451}%
\pgfsetstrokecolor{currentstroke}%
\pgfsetstrokeopacity{0.800000}%
\pgfsetdash{}{0pt}%
\pgfpathmoveto{\pgfqpoint{1.286891in}{2.175474in}}%
\pgfpathcurveto{\pgfqpoint{1.297941in}{2.175474in}}{\pgfqpoint{1.308540in}{2.179864in}}{\pgfqpoint{1.316354in}{2.187678in}}%
\pgfpathcurveto{\pgfqpoint{1.324167in}{2.195491in}}{\pgfqpoint{1.328558in}{2.206090in}}{\pgfqpoint{1.328558in}{2.217140in}}%
\pgfpathcurveto{\pgfqpoint{1.328558in}{2.228190in}}{\pgfqpoint{1.324167in}{2.238789in}}{\pgfqpoint{1.316354in}{2.246603in}}%
\pgfpathcurveto{\pgfqpoint{1.308540in}{2.254417in}}{\pgfqpoint{1.297941in}{2.258807in}}{\pgfqpoint{1.286891in}{2.258807in}}%
\pgfpathcurveto{\pgfqpoint{1.275841in}{2.258807in}}{\pgfqpoint{1.265242in}{2.254417in}}{\pgfqpoint{1.257428in}{2.246603in}}%
\pgfpathcurveto{\pgfqpoint{1.249614in}{2.238789in}}{\pgfqpoint{1.245224in}{2.228190in}}{\pgfqpoint{1.245224in}{2.217140in}}%
\pgfpathcurveto{\pgfqpoint{1.245224in}{2.206090in}}{\pgfqpoint{1.249614in}{2.195491in}}{\pgfqpoint{1.257428in}{2.187678in}}%
\pgfpathcurveto{\pgfqpoint{1.265242in}{2.179864in}}{\pgfqpoint{1.275841in}{2.175474in}}{\pgfqpoint{1.286891in}{2.175474in}}%
\pgfpathlineto{\pgfqpoint{1.286891in}{2.175474in}}%
\pgfpathclose%
\pgfusepath{stroke}%
\end{pgfscope}%
\begin{pgfscope}%
\pgfpathrectangle{\pgfqpoint{0.494722in}{0.437222in}}{\pgfqpoint{6.275590in}{5.159444in}}%
\pgfusepath{clip}%
\pgfsetbuttcap%
\pgfsetroundjoin%
\pgfsetlinewidth{1.003750pt}%
\definecolor{currentstroke}{rgb}{0.827451,0.827451,0.827451}%
\pgfsetstrokecolor{currentstroke}%
\pgfsetstrokeopacity{0.800000}%
\pgfsetdash{}{0pt}%
\pgfpathmoveto{\pgfqpoint{3.321971in}{0.791856in}}%
\pgfpathcurveto{\pgfqpoint{3.333021in}{0.791856in}}{\pgfqpoint{3.343620in}{0.796246in}}{\pgfqpoint{3.351433in}{0.804060in}}%
\pgfpathcurveto{\pgfqpoint{3.359247in}{0.811873in}}{\pgfqpoint{3.363637in}{0.822472in}}{\pgfqpoint{3.363637in}{0.833523in}}%
\pgfpathcurveto{\pgfqpoint{3.363637in}{0.844573in}}{\pgfqpoint{3.359247in}{0.855172in}}{\pgfqpoint{3.351433in}{0.862985in}}%
\pgfpathcurveto{\pgfqpoint{3.343620in}{0.870799in}}{\pgfqpoint{3.333021in}{0.875189in}}{\pgfqpoint{3.321971in}{0.875189in}}%
\pgfpathcurveto{\pgfqpoint{3.310921in}{0.875189in}}{\pgfqpoint{3.300321in}{0.870799in}}{\pgfqpoint{3.292508in}{0.862985in}}%
\pgfpathcurveto{\pgfqpoint{3.284694in}{0.855172in}}{\pgfqpoint{3.280304in}{0.844573in}}{\pgfqpoint{3.280304in}{0.833523in}}%
\pgfpathcurveto{\pgfqpoint{3.280304in}{0.822472in}}{\pgfqpoint{3.284694in}{0.811873in}}{\pgfqpoint{3.292508in}{0.804060in}}%
\pgfpathcurveto{\pgfqpoint{3.300321in}{0.796246in}}{\pgfqpoint{3.310921in}{0.791856in}}{\pgfqpoint{3.321971in}{0.791856in}}%
\pgfpathlineto{\pgfqpoint{3.321971in}{0.791856in}}%
\pgfpathclose%
\pgfusepath{stroke}%
\end{pgfscope}%
\begin{pgfscope}%
\pgfpathrectangle{\pgfqpoint{0.494722in}{0.437222in}}{\pgfqpoint{6.275590in}{5.159444in}}%
\pgfusepath{clip}%
\pgfsetbuttcap%
\pgfsetroundjoin%
\pgfsetlinewidth{1.003750pt}%
\definecolor{currentstroke}{rgb}{0.827451,0.827451,0.827451}%
\pgfsetstrokecolor{currentstroke}%
\pgfsetstrokeopacity{0.800000}%
\pgfsetdash{}{0pt}%
\pgfpathmoveto{\pgfqpoint{2.062312in}{1.444503in}}%
\pgfpathcurveto{\pgfqpoint{2.073362in}{1.444503in}}{\pgfqpoint{2.083961in}{1.448893in}}{\pgfqpoint{2.091775in}{1.456707in}}%
\pgfpathcurveto{\pgfqpoint{2.099588in}{1.464520in}}{\pgfqpoint{2.103979in}{1.475119in}}{\pgfqpoint{2.103979in}{1.486169in}}%
\pgfpathcurveto{\pgfqpoint{2.103979in}{1.497220in}}{\pgfqpoint{2.099588in}{1.507819in}}{\pgfqpoint{2.091775in}{1.515632in}}%
\pgfpathcurveto{\pgfqpoint{2.083961in}{1.523446in}}{\pgfqpoint{2.073362in}{1.527836in}}{\pgfqpoint{2.062312in}{1.527836in}}%
\pgfpathcurveto{\pgfqpoint{2.051262in}{1.527836in}}{\pgfqpoint{2.040663in}{1.523446in}}{\pgfqpoint{2.032849in}{1.515632in}}%
\pgfpathcurveto{\pgfqpoint{2.025036in}{1.507819in}}{\pgfqpoint{2.020645in}{1.497220in}}{\pgfqpoint{2.020645in}{1.486169in}}%
\pgfpathcurveto{\pgfqpoint{2.020645in}{1.475119in}}{\pgfqpoint{2.025036in}{1.464520in}}{\pgfqpoint{2.032849in}{1.456707in}}%
\pgfpathcurveto{\pgfqpoint{2.040663in}{1.448893in}}{\pgfqpoint{2.051262in}{1.444503in}}{\pgfqpoint{2.062312in}{1.444503in}}%
\pgfpathlineto{\pgfqpoint{2.062312in}{1.444503in}}%
\pgfpathclose%
\pgfusepath{stroke}%
\end{pgfscope}%
\begin{pgfscope}%
\pgfpathrectangle{\pgfqpoint{0.494722in}{0.437222in}}{\pgfqpoint{6.275590in}{5.159444in}}%
\pgfusepath{clip}%
\pgfsetbuttcap%
\pgfsetroundjoin%
\pgfsetlinewidth{1.003750pt}%
\definecolor{currentstroke}{rgb}{0.827451,0.827451,0.827451}%
\pgfsetstrokecolor{currentstroke}%
\pgfsetstrokeopacity{0.800000}%
\pgfsetdash{}{0pt}%
\pgfpathmoveto{\pgfqpoint{2.203827in}{1.337053in}}%
\pgfpathcurveto{\pgfqpoint{2.214877in}{1.337053in}}{\pgfqpoint{2.225476in}{1.341443in}}{\pgfqpoint{2.233290in}{1.349257in}}%
\pgfpathcurveto{\pgfqpoint{2.241103in}{1.357070in}}{\pgfqpoint{2.245494in}{1.367669in}}{\pgfqpoint{2.245494in}{1.378720in}}%
\pgfpathcurveto{\pgfqpoint{2.245494in}{1.389770in}}{\pgfqpoint{2.241103in}{1.400369in}}{\pgfqpoint{2.233290in}{1.408182in}}%
\pgfpathcurveto{\pgfqpoint{2.225476in}{1.415996in}}{\pgfqpoint{2.214877in}{1.420386in}}{\pgfqpoint{2.203827in}{1.420386in}}%
\pgfpathcurveto{\pgfqpoint{2.192777in}{1.420386in}}{\pgfqpoint{2.182178in}{1.415996in}}{\pgfqpoint{2.174364in}{1.408182in}}%
\pgfpathcurveto{\pgfqpoint{2.166550in}{1.400369in}}{\pgfqpoint{2.162160in}{1.389770in}}{\pgfqpoint{2.162160in}{1.378720in}}%
\pgfpathcurveto{\pgfqpoint{2.162160in}{1.367669in}}{\pgfqpoint{2.166550in}{1.357070in}}{\pgfqpoint{2.174364in}{1.349257in}}%
\pgfpathcurveto{\pgfqpoint{2.182178in}{1.341443in}}{\pgfqpoint{2.192777in}{1.337053in}}{\pgfqpoint{2.203827in}{1.337053in}}%
\pgfpathlineto{\pgfqpoint{2.203827in}{1.337053in}}%
\pgfpathclose%
\pgfusepath{stroke}%
\end{pgfscope}%
\begin{pgfscope}%
\pgfpathrectangle{\pgfqpoint{0.494722in}{0.437222in}}{\pgfqpoint{6.275590in}{5.159444in}}%
\pgfusepath{clip}%
\pgfsetbuttcap%
\pgfsetroundjoin%
\pgfsetlinewidth{1.003750pt}%
\definecolor{currentstroke}{rgb}{0.827451,0.827451,0.827451}%
\pgfsetstrokecolor{currentstroke}%
\pgfsetstrokeopacity{0.800000}%
\pgfsetdash{}{0pt}%
\pgfpathmoveto{\pgfqpoint{1.143945in}{2.535677in}}%
\pgfpathcurveto{\pgfqpoint{1.154995in}{2.535677in}}{\pgfqpoint{1.165594in}{2.540067in}}{\pgfqpoint{1.173408in}{2.547881in}}%
\pgfpathcurveto{\pgfqpoint{1.181222in}{2.555695in}}{\pgfqpoint{1.185612in}{2.566294in}}{\pgfqpoint{1.185612in}{2.577344in}}%
\pgfpathcurveto{\pgfqpoint{1.185612in}{2.588394in}}{\pgfqpoint{1.181222in}{2.598993in}}{\pgfqpoint{1.173408in}{2.606807in}}%
\pgfpathcurveto{\pgfqpoint{1.165594in}{2.614620in}}{\pgfqpoint{1.154995in}{2.619011in}}{\pgfqpoint{1.143945in}{2.619011in}}%
\pgfpathcurveto{\pgfqpoint{1.132895in}{2.619011in}}{\pgfqpoint{1.122296in}{2.614620in}}{\pgfqpoint{1.114482in}{2.606807in}}%
\pgfpathcurveto{\pgfqpoint{1.106669in}{2.598993in}}{\pgfqpoint{1.102279in}{2.588394in}}{\pgfqpoint{1.102279in}{2.577344in}}%
\pgfpathcurveto{\pgfqpoint{1.102279in}{2.566294in}}{\pgfqpoint{1.106669in}{2.555695in}}{\pgfqpoint{1.114482in}{2.547881in}}%
\pgfpathcurveto{\pgfqpoint{1.122296in}{2.540067in}}{\pgfqpoint{1.132895in}{2.535677in}}{\pgfqpoint{1.143945in}{2.535677in}}%
\pgfpathlineto{\pgfqpoint{1.143945in}{2.535677in}}%
\pgfpathclose%
\pgfusepath{stroke}%
\end{pgfscope}%
\begin{pgfscope}%
\pgfpathrectangle{\pgfqpoint{0.494722in}{0.437222in}}{\pgfqpoint{6.275590in}{5.159444in}}%
\pgfusepath{clip}%
\pgfsetbuttcap%
\pgfsetroundjoin%
\pgfsetlinewidth{1.003750pt}%
\definecolor{currentstroke}{rgb}{0.827451,0.827451,0.827451}%
\pgfsetstrokecolor{currentstroke}%
\pgfsetstrokeopacity{0.800000}%
\pgfsetdash{}{0pt}%
\pgfpathmoveto{\pgfqpoint{0.600098in}{3.692599in}}%
\pgfpathcurveto{\pgfqpoint{0.611148in}{3.692599in}}{\pgfqpoint{0.621747in}{3.696989in}}{\pgfqpoint{0.629561in}{3.704803in}}%
\pgfpathcurveto{\pgfqpoint{0.637375in}{3.712616in}}{\pgfqpoint{0.641765in}{3.723215in}}{\pgfqpoint{0.641765in}{3.734266in}}%
\pgfpathcurveto{\pgfqpoint{0.641765in}{3.745316in}}{\pgfqpoint{0.637375in}{3.755915in}}{\pgfqpoint{0.629561in}{3.763728in}}%
\pgfpathcurveto{\pgfqpoint{0.621747in}{3.771542in}}{\pgfqpoint{0.611148in}{3.775932in}}{\pgfqpoint{0.600098in}{3.775932in}}%
\pgfpathcurveto{\pgfqpoint{0.589048in}{3.775932in}}{\pgfqpoint{0.578449in}{3.771542in}}{\pgfqpoint{0.570636in}{3.763728in}}%
\pgfpathcurveto{\pgfqpoint{0.562822in}{3.755915in}}{\pgfqpoint{0.558432in}{3.745316in}}{\pgfqpoint{0.558432in}{3.734266in}}%
\pgfpathcurveto{\pgfqpoint{0.558432in}{3.723215in}}{\pgfqpoint{0.562822in}{3.712616in}}{\pgfqpoint{0.570636in}{3.704803in}}%
\pgfpathcurveto{\pgfqpoint{0.578449in}{3.696989in}}{\pgfqpoint{0.589048in}{3.692599in}}{\pgfqpoint{0.600098in}{3.692599in}}%
\pgfpathlineto{\pgfqpoint{0.600098in}{3.692599in}}%
\pgfpathclose%
\pgfusepath{stroke}%
\end{pgfscope}%
\begin{pgfscope}%
\pgfpathrectangle{\pgfqpoint{0.494722in}{0.437222in}}{\pgfqpoint{6.275590in}{5.159444in}}%
\pgfusepath{clip}%
\pgfsetbuttcap%
\pgfsetroundjoin%
\pgfsetlinewidth{1.003750pt}%
\definecolor{currentstroke}{rgb}{0.827451,0.827451,0.827451}%
\pgfsetstrokecolor{currentstroke}%
\pgfsetstrokeopacity{0.800000}%
\pgfsetdash{}{0pt}%
\pgfpathmoveto{\pgfqpoint{0.744905in}{3.205367in}}%
\pgfpathcurveto{\pgfqpoint{0.755955in}{3.205367in}}{\pgfqpoint{0.766554in}{3.209757in}}{\pgfqpoint{0.774368in}{3.217571in}}%
\pgfpathcurveto{\pgfqpoint{0.782181in}{3.225384in}}{\pgfqpoint{0.786571in}{3.235983in}}{\pgfqpoint{0.786571in}{3.247034in}}%
\pgfpathcurveto{\pgfqpoint{0.786571in}{3.258084in}}{\pgfqpoint{0.782181in}{3.268683in}}{\pgfqpoint{0.774368in}{3.276496in}}%
\pgfpathcurveto{\pgfqpoint{0.766554in}{3.284310in}}{\pgfqpoint{0.755955in}{3.288700in}}{\pgfqpoint{0.744905in}{3.288700in}}%
\pgfpathcurveto{\pgfqpoint{0.733855in}{3.288700in}}{\pgfqpoint{0.723256in}{3.284310in}}{\pgfqpoint{0.715442in}{3.276496in}}%
\pgfpathcurveto{\pgfqpoint{0.707628in}{3.268683in}}{\pgfqpoint{0.703238in}{3.258084in}}{\pgfqpoint{0.703238in}{3.247034in}}%
\pgfpathcurveto{\pgfqpoint{0.703238in}{3.235983in}}{\pgfqpoint{0.707628in}{3.225384in}}{\pgfqpoint{0.715442in}{3.217571in}}%
\pgfpathcurveto{\pgfqpoint{0.723256in}{3.209757in}}{\pgfqpoint{0.733855in}{3.205367in}}{\pgfqpoint{0.744905in}{3.205367in}}%
\pgfpathlineto{\pgfqpoint{0.744905in}{3.205367in}}%
\pgfpathclose%
\pgfusepath{stroke}%
\end{pgfscope}%
\begin{pgfscope}%
\pgfpathrectangle{\pgfqpoint{0.494722in}{0.437222in}}{\pgfqpoint{6.275590in}{5.159444in}}%
\pgfusepath{clip}%
\pgfsetbuttcap%
\pgfsetroundjoin%
\pgfsetlinewidth{1.003750pt}%
\definecolor{currentstroke}{rgb}{0.827451,0.827451,0.827451}%
\pgfsetstrokecolor{currentstroke}%
\pgfsetstrokeopacity{0.800000}%
\pgfsetdash{}{0pt}%
\pgfpathmoveto{\pgfqpoint{3.754586in}{0.666564in}}%
\pgfpathcurveto{\pgfqpoint{3.765636in}{0.666564in}}{\pgfqpoint{3.776235in}{0.670954in}}{\pgfqpoint{3.784048in}{0.678767in}}%
\pgfpathcurveto{\pgfqpoint{3.791862in}{0.686581in}}{\pgfqpoint{3.796252in}{0.697180in}}{\pgfqpoint{3.796252in}{0.708230in}}%
\pgfpathcurveto{\pgfqpoint{3.796252in}{0.719280in}}{\pgfqpoint{3.791862in}{0.729879in}}{\pgfqpoint{3.784048in}{0.737693in}}%
\pgfpathcurveto{\pgfqpoint{3.776235in}{0.745507in}}{\pgfqpoint{3.765636in}{0.749897in}}{\pgfqpoint{3.754586in}{0.749897in}}%
\pgfpathcurveto{\pgfqpoint{3.743536in}{0.749897in}}{\pgfqpoint{3.732937in}{0.745507in}}{\pgfqpoint{3.725123in}{0.737693in}}%
\pgfpathcurveto{\pgfqpoint{3.717309in}{0.729879in}}{\pgfqpoint{3.712919in}{0.719280in}}{\pgfqpoint{3.712919in}{0.708230in}}%
\pgfpathcurveto{\pgfqpoint{3.712919in}{0.697180in}}{\pgfqpoint{3.717309in}{0.686581in}}{\pgfqpoint{3.725123in}{0.678767in}}%
\pgfpathcurveto{\pgfqpoint{3.732937in}{0.670954in}}{\pgfqpoint{3.743536in}{0.666564in}}{\pgfqpoint{3.754586in}{0.666564in}}%
\pgfpathlineto{\pgfqpoint{3.754586in}{0.666564in}}%
\pgfpathclose%
\pgfusepath{stroke}%
\end{pgfscope}%
\begin{pgfscope}%
\pgfpathrectangle{\pgfqpoint{0.494722in}{0.437222in}}{\pgfqpoint{6.275590in}{5.159444in}}%
\pgfusepath{clip}%
\pgfsetbuttcap%
\pgfsetroundjoin%
\pgfsetlinewidth{1.003750pt}%
\definecolor{currentstroke}{rgb}{0.827451,0.827451,0.827451}%
\pgfsetstrokecolor{currentstroke}%
\pgfsetstrokeopacity{0.800000}%
\pgfsetdash{}{0pt}%
\pgfpathmoveto{\pgfqpoint{1.455153in}{2.010549in}}%
\pgfpathcurveto{\pgfqpoint{1.466203in}{2.010549in}}{\pgfqpoint{1.476802in}{2.014939in}}{\pgfqpoint{1.484615in}{2.022752in}}%
\pgfpathcurveto{\pgfqpoint{1.492429in}{2.030566in}}{\pgfqpoint{1.496819in}{2.041165in}}{\pgfqpoint{1.496819in}{2.052215in}}%
\pgfpathcurveto{\pgfqpoint{1.496819in}{2.063265in}}{\pgfqpoint{1.492429in}{2.073864in}}{\pgfqpoint{1.484615in}{2.081678in}}%
\pgfpathcurveto{\pgfqpoint{1.476802in}{2.089492in}}{\pgfqpoint{1.466203in}{2.093882in}}{\pgfqpoint{1.455153in}{2.093882in}}%
\pgfpathcurveto{\pgfqpoint{1.444103in}{2.093882in}}{\pgfqpoint{1.433504in}{2.089492in}}{\pgfqpoint{1.425690in}{2.081678in}}%
\pgfpathcurveto{\pgfqpoint{1.417876in}{2.073864in}}{\pgfqpoint{1.413486in}{2.063265in}}{\pgfqpoint{1.413486in}{2.052215in}}%
\pgfpathcurveto{\pgfqpoint{1.413486in}{2.041165in}}{\pgfqpoint{1.417876in}{2.030566in}}{\pgfqpoint{1.425690in}{2.022752in}}%
\pgfpathcurveto{\pgfqpoint{1.433504in}{2.014939in}}{\pgfqpoint{1.444103in}{2.010549in}}{\pgfqpoint{1.455153in}{2.010549in}}%
\pgfpathlineto{\pgfqpoint{1.455153in}{2.010549in}}%
\pgfpathclose%
\pgfusepath{stroke}%
\end{pgfscope}%
\begin{pgfscope}%
\pgfpathrectangle{\pgfqpoint{0.494722in}{0.437222in}}{\pgfqpoint{6.275590in}{5.159444in}}%
\pgfusepath{clip}%
\pgfsetbuttcap%
\pgfsetroundjoin%
\pgfsetlinewidth{1.003750pt}%
\definecolor{currentstroke}{rgb}{0.827451,0.827451,0.827451}%
\pgfsetstrokecolor{currentstroke}%
\pgfsetstrokeopacity{0.800000}%
\pgfsetdash{}{0pt}%
\pgfpathmoveto{\pgfqpoint{2.365416in}{1.256449in}}%
\pgfpathcurveto{\pgfqpoint{2.376466in}{1.256449in}}{\pgfqpoint{2.387065in}{1.260839in}}{\pgfqpoint{2.394879in}{1.268653in}}%
\pgfpathcurveto{\pgfqpoint{2.402693in}{1.276466in}}{\pgfqpoint{2.407083in}{1.287065in}}{\pgfqpoint{2.407083in}{1.298115in}}%
\pgfpathcurveto{\pgfqpoint{2.407083in}{1.309166in}}{\pgfqpoint{2.402693in}{1.319765in}}{\pgfqpoint{2.394879in}{1.327578in}}%
\pgfpathcurveto{\pgfqpoint{2.387065in}{1.335392in}}{\pgfqpoint{2.376466in}{1.339782in}}{\pgfqpoint{2.365416in}{1.339782in}}%
\pgfpathcurveto{\pgfqpoint{2.354366in}{1.339782in}}{\pgfqpoint{2.343767in}{1.335392in}}{\pgfqpoint{2.335953in}{1.327578in}}%
\pgfpathcurveto{\pgfqpoint{2.328140in}{1.319765in}}{\pgfqpoint{2.323750in}{1.309166in}}{\pgfqpoint{2.323750in}{1.298115in}}%
\pgfpathcurveto{\pgfqpoint{2.323750in}{1.287065in}}{\pgfqpoint{2.328140in}{1.276466in}}{\pgfqpoint{2.335953in}{1.268653in}}%
\pgfpathcurveto{\pgfqpoint{2.343767in}{1.260839in}}{\pgfqpoint{2.354366in}{1.256449in}}{\pgfqpoint{2.365416in}{1.256449in}}%
\pgfpathlineto{\pgfqpoint{2.365416in}{1.256449in}}%
\pgfpathclose%
\pgfusepath{stroke}%
\end{pgfscope}%
\begin{pgfscope}%
\pgfpathrectangle{\pgfqpoint{0.494722in}{0.437222in}}{\pgfqpoint{6.275590in}{5.159444in}}%
\pgfusepath{clip}%
\pgfsetbuttcap%
\pgfsetroundjoin%
\pgfsetlinewidth{1.003750pt}%
\definecolor{currentstroke}{rgb}{0.827451,0.827451,0.827451}%
\pgfsetstrokecolor{currentstroke}%
\pgfsetstrokeopacity{0.800000}%
\pgfsetdash{}{0pt}%
\pgfpathmoveto{\pgfqpoint{2.664982in}{1.104826in}}%
\pgfpathcurveto{\pgfqpoint{2.676032in}{1.104826in}}{\pgfqpoint{2.686631in}{1.109216in}}{\pgfqpoint{2.694445in}{1.117030in}}%
\pgfpathcurveto{\pgfqpoint{2.702258in}{1.124843in}}{\pgfqpoint{2.706648in}{1.135442in}}{\pgfqpoint{2.706648in}{1.146492in}}%
\pgfpathcurveto{\pgfqpoint{2.706648in}{1.157542in}}{\pgfqpoint{2.702258in}{1.168142in}}{\pgfqpoint{2.694445in}{1.175955in}}%
\pgfpathcurveto{\pgfqpoint{2.686631in}{1.183769in}}{\pgfqpoint{2.676032in}{1.188159in}}{\pgfqpoint{2.664982in}{1.188159in}}%
\pgfpathcurveto{\pgfqpoint{2.653932in}{1.188159in}}{\pgfqpoint{2.643333in}{1.183769in}}{\pgfqpoint{2.635519in}{1.175955in}}%
\pgfpathcurveto{\pgfqpoint{2.627705in}{1.168142in}}{\pgfqpoint{2.623315in}{1.157542in}}{\pgfqpoint{2.623315in}{1.146492in}}%
\pgfpathcurveto{\pgfqpoint{2.623315in}{1.135442in}}{\pgfqpoint{2.627705in}{1.124843in}}{\pgfqpoint{2.635519in}{1.117030in}}%
\pgfpathcurveto{\pgfqpoint{2.643333in}{1.109216in}}{\pgfqpoint{2.653932in}{1.104826in}}{\pgfqpoint{2.664982in}{1.104826in}}%
\pgfpathlineto{\pgfqpoint{2.664982in}{1.104826in}}%
\pgfpathclose%
\pgfusepath{stroke}%
\end{pgfscope}%
\begin{pgfscope}%
\pgfpathrectangle{\pgfqpoint{0.494722in}{0.437222in}}{\pgfqpoint{6.275590in}{5.159444in}}%
\pgfusepath{clip}%
\pgfsetbuttcap%
\pgfsetroundjoin%
\pgfsetlinewidth{1.003750pt}%
\definecolor{currentstroke}{rgb}{0.827451,0.827451,0.827451}%
\pgfsetstrokecolor{currentstroke}%
\pgfsetstrokeopacity{0.800000}%
\pgfsetdash{}{0pt}%
\pgfpathmoveto{\pgfqpoint{1.941684in}{1.555345in}}%
\pgfpathcurveto{\pgfqpoint{1.952734in}{1.555345in}}{\pgfqpoint{1.963333in}{1.559735in}}{\pgfqpoint{1.971146in}{1.567549in}}%
\pgfpathcurveto{\pgfqpoint{1.978960in}{1.575362in}}{\pgfqpoint{1.983350in}{1.585961in}}{\pgfqpoint{1.983350in}{1.597011in}}%
\pgfpathcurveto{\pgfqpoint{1.983350in}{1.608062in}}{\pgfqpoint{1.978960in}{1.618661in}}{\pgfqpoint{1.971146in}{1.626474in}}%
\pgfpathcurveto{\pgfqpoint{1.963333in}{1.634288in}}{\pgfqpoint{1.952734in}{1.638678in}}{\pgfqpoint{1.941684in}{1.638678in}}%
\pgfpathcurveto{\pgfqpoint{1.930633in}{1.638678in}}{\pgfqpoint{1.920034in}{1.634288in}}{\pgfqpoint{1.912221in}{1.626474in}}%
\pgfpathcurveto{\pgfqpoint{1.904407in}{1.618661in}}{\pgfqpoint{1.900017in}{1.608062in}}{\pgfqpoint{1.900017in}{1.597011in}}%
\pgfpathcurveto{\pgfqpoint{1.900017in}{1.585961in}}{\pgfqpoint{1.904407in}{1.575362in}}{\pgfqpoint{1.912221in}{1.567549in}}%
\pgfpathcurveto{\pgfqpoint{1.920034in}{1.559735in}}{\pgfqpoint{1.930633in}{1.555345in}}{\pgfqpoint{1.941684in}{1.555345in}}%
\pgfpathlineto{\pgfqpoint{1.941684in}{1.555345in}}%
\pgfpathclose%
\pgfusepath{stroke}%
\end{pgfscope}%
\begin{pgfscope}%
\pgfpathrectangle{\pgfqpoint{0.494722in}{0.437222in}}{\pgfqpoint{6.275590in}{5.159444in}}%
\pgfusepath{clip}%
\pgfsetbuttcap%
\pgfsetroundjoin%
\pgfsetlinewidth{1.003750pt}%
\definecolor{currentstroke}{rgb}{0.827451,0.827451,0.827451}%
\pgfsetstrokecolor{currentstroke}%
\pgfsetstrokeopacity{0.800000}%
\pgfsetdash{}{0pt}%
\pgfpathmoveto{\pgfqpoint{1.874789in}{1.638879in}}%
\pgfpathcurveto{\pgfqpoint{1.885839in}{1.638879in}}{\pgfqpoint{1.896438in}{1.643269in}}{\pgfqpoint{1.904252in}{1.651082in}}%
\pgfpathcurveto{\pgfqpoint{1.912066in}{1.658896in}}{\pgfqpoint{1.916456in}{1.669495in}}{\pgfqpoint{1.916456in}{1.680545in}}%
\pgfpathcurveto{\pgfqpoint{1.916456in}{1.691595in}}{\pgfqpoint{1.912066in}{1.702194in}}{\pgfqpoint{1.904252in}{1.710008in}}%
\pgfpathcurveto{\pgfqpoint{1.896438in}{1.717822in}}{\pgfqpoint{1.885839in}{1.722212in}}{\pgfqpoint{1.874789in}{1.722212in}}%
\pgfpathcurveto{\pgfqpoint{1.863739in}{1.722212in}}{\pgfqpoint{1.853140in}{1.717822in}}{\pgfqpoint{1.845326in}{1.710008in}}%
\pgfpathcurveto{\pgfqpoint{1.837513in}{1.702194in}}{\pgfqpoint{1.833123in}{1.691595in}}{\pgfqpoint{1.833123in}{1.680545in}}%
\pgfpathcurveto{\pgfqpoint{1.833123in}{1.669495in}}{\pgfqpoint{1.837513in}{1.658896in}}{\pgfqpoint{1.845326in}{1.651082in}}%
\pgfpathcurveto{\pgfqpoint{1.853140in}{1.643269in}}{\pgfqpoint{1.863739in}{1.638879in}}{\pgfqpoint{1.874789in}{1.638879in}}%
\pgfpathlineto{\pgfqpoint{1.874789in}{1.638879in}}%
\pgfpathclose%
\pgfusepath{stroke}%
\end{pgfscope}%
\begin{pgfscope}%
\pgfpathrectangle{\pgfqpoint{0.494722in}{0.437222in}}{\pgfqpoint{6.275590in}{5.159444in}}%
\pgfusepath{clip}%
\pgfsetbuttcap%
\pgfsetroundjoin%
\pgfsetlinewidth{1.003750pt}%
\definecolor{currentstroke}{rgb}{0.827451,0.827451,0.827451}%
\pgfsetstrokecolor{currentstroke}%
\pgfsetstrokeopacity{0.800000}%
\pgfsetdash{}{0pt}%
\pgfpathmoveto{\pgfqpoint{0.532177in}{4.076860in}}%
\pgfpathcurveto{\pgfqpoint{0.543227in}{4.076860in}}{\pgfqpoint{0.553826in}{4.081250in}}{\pgfqpoint{0.561639in}{4.089064in}}%
\pgfpathcurveto{\pgfqpoint{0.569453in}{4.096877in}}{\pgfqpoint{0.573843in}{4.107476in}}{\pgfqpoint{0.573843in}{4.118526in}}%
\pgfpathcurveto{\pgfqpoint{0.573843in}{4.129577in}}{\pgfqpoint{0.569453in}{4.140176in}}{\pgfqpoint{0.561639in}{4.147989in}}%
\pgfpathcurveto{\pgfqpoint{0.553826in}{4.155803in}}{\pgfqpoint{0.543227in}{4.160193in}}{\pgfqpoint{0.532177in}{4.160193in}}%
\pgfpathcurveto{\pgfqpoint{0.521127in}{4.160193in}}{\pgfqpoint{0.510528in}{4.155803in}}{\pgfqpoint{0.502714in}{4.147989in}}%
\pgfpathcurveto{\pgfqpoint{0.494900in}{4.140176in}}{\pgfqpoint{0.490510in}{4.129577in}}{\pgfqpoint{0.490510in}{4.118526in}}%
\pgfpathcurveto{\pgfqpoint{0.490510in}{4.107476in}}{\pgfqpoint{0.494900in}{4.096877in}}{\pgfqpoint{0.502714in}{4.089064in}}%
\pgfpathcurveto{\pgfqpoint{0.510528in}{4.081250in}}{\pgfqpoint{0.521127in}{4.076860in}}{\pgfqpoint{0.532177in}{4.076860in}}%
\pgfpathlineto{\pgfqpoint{0.532177in}{4.076860in}}%
\pgfpathclose%
\pgfusepath{stroke}%
\end{pgfscope}%
\begin{pgfscope}%
\pgfpathrectangle{\pgfqpoint{0.494722in}{0.437222in}}{\pgfqpoint{6.275590in}{5.159444in}}%
\pgfusepath{clip}%
\pgfsetbuttcap%
\pgfsetroundjoin%
\pgfsetlinewidth{1.003750pt}%
\definecolor{currentstroke}{rgb}{0.827451,0.827451,0.827451}%
\pgfsetstrokecolor{currentstroke}%
\pgfsetstrokeopacity{0.800000}%
\pgfsetdash{}{0pt}%
\pgfpathmoveto{\pgfqpoint{2.974947in}{0.945504in}}%
\pgfpathcurveto{\pgfqpoint{2.985997in}{0.945504in}}{\pgfqpoint{2.996596in}{0.949895in}}{\pgfqpoint{3.004410in}{0.957708in}}%
\pgfpathcurveto{\pgfqpoint{3.012223in}{0.965522in}}{\pgfqpoint{3.016614in}{0.976121in}}{\pgfqpoint{3.016614in}{0.987171in}}%
\pgfpathcurveto{\pgfqpoint{3.016614in}{0.998221in}}{\pgfqpoint{3.012223in}{1.008820in}}{\pgfqpoint{3.004410in}{1.016634in}}%
\pgfpathcurveto{\pgfqpoint{2.996596in}{1.024447in}}{\pgfqpoint{2.985997in}{1.028838in}}{\pgfqpoint{2.974947in}{1.028838in}}%
\pgfpathcurveto{\pgfqpoint{2.963897in}{1.028838in}}{\pgfqpoint{2.953298in}{1.024447in}}{\pgfqpoint{2.945484in}{1.016634in}}%
\pgfpathcurveto{\pgfqpoint{2.937670in}{1.008820in}}{\pgfqpoint{2.933280in}{0.998221in}}{\pgfqpoint{2.933280in}{0.987171in}}%
\pgfpathcurveto{\pgfqpoint{2.933280in}{0.976121in}}{\pgfqpoint{2.937670in}{0.965522in}}{\pgfqpoint{2.945484in}{0.957708in}}%
\pgfpathcurveto{\pgfqpoint{2.953298in}{0.949895in}}{\pgfqpoint{2.963897in}{0.945504in}}{\pgfqpoint{2.974947in}{0.945504in}}%
\pgfpathlineto{\pgfqpoint{2.974947in}{0.945504in}}%
\pgfpathclose%
\pgfusepath{stroke}%
\end{pgfscope}%
\begin{pgfscope}%
\pgfpathrectangle{\pgfqpoint{0.494722in}{0.437222in}}{\pgfqpoint{6.275590in}{5.159444in}}%
\pgfusepath{clip}%
\pgfsetbuttcap%
\pgfsetroundjoin%
\pgfsetlinewidth{1.003750pt}%
\definecolor{currentstroke}{rgb}{0.827451,0.827451,0.827451}%
\pgfsetstrokecolor{currentstroke}%
\pgfsetstrokeopacity{0.800000}%
\pgfsetdash{}{0pt}%
\pgfpathmoveto{\pgfqpoint{4.226627in}{0.519229in}}%
\pgfpathcurveto{\pgfqpoint{4.237677in}{0.519229in}}{\pgfqpoint{4.248276in}{0.523619in}}{\pgfqpoint{4.256090in}{0.531433in}}%
\pgfpathcurveto{\pgfqpoint{4.263903in}{0.539247in}}{\pgfqpoint{4.268294in}{0.549846in}}{\pgfqpoint{4.268294in}{0.560896in}}%
\pgfpathcurveto{\pgfqpoint{4.268294in}{0.571946in}}{\pgfqpoint{4.263903in}{0.582545in}}{\pgfqpoint{4.256090in}{0.590358in}}%
\pgfpathcurveto{\pgfqpoint{4.248276in}{0.598172in}}{\pgfqpoint{4.237677in}{0.602562in}}{\pgfqpoint{4.226627in}{0.602562in}}%
\pgfpathcurveto{\pgfqpoint{4.215577in}{0.602562in}}{\pgfqpoint{4.204978in}{0.598172in}}{\pgfqpoint{4.197164in}{0.590358in}}%
\pgfpathcurveto{\pgfqpoint{4.189350in}{0.582545in}}{\pgfqpoint{4.184960in}{0.571946in}}{\pgfqpoint{4.184960in}{0.560896in}}%
\pgfpathcurveto{\pgfqpoint{4.184960in}{0.549846in}}{\pgfqpoint{4.189350in}{0.539247in}}{\pgfqpoint{4.197164in}{0.531433in}}%
\pgfpathcurveto{\pgfqpoint{4.204978in}{0.523619in}}{\pgfqpoint{4.215577in}{0.519229in}}{\pgfqpoint{4.226627in}{0.519229in}}%
\pgfpathlineto{\pgfqpoint{4.226627in}{0.519229in}}%
\pgfpathclose%
\pgfusepath{stroke}%
\end{pgfscope}%
\begin{pgfscope}%
\pgfpathrectangle{\pgfqpoint{0.494722in}{0.437222in}}{\pgfqpoint{6.275590in}{5.159444in}}%
\pgfusepath{clip}%
\pgfsetbuttcap%
\pgfsetroundjoin%
\pgfsetlinewidth{1.003750pt}%
\definecolor{currentstroke}{rgb}{0.827451,0.827451,0.827451}%
\pgfsetstrokecolor{currentstroke}%
\pgfsetstrokeopacity{0.800000}%
\pgfsetdash{}{0pt}%
\pgfpathmoveto{\pgfqpoint{0.536273in}{3.987890in}}%
\pgfpathcurveto{\pgfqpoint{0.547324in}{3.987890in}}{\pgfqpoint{0.557923in}{3.992280in}}{\pgfqpoint{0.565736in}{4.000093in}}%
\pgfpathcurveto{\pgfqpoint{0.573550in}{4.007907in}}{\pgfqpoint{0.577940in}{4.018506in}}{\pgfqpoint{0.577940in}{4.029556in}}%
\pgfpathcurveto{\pgfqpoint{0.577940in}{4.040606in}}{\pgfqpoint{0.573550in}{4.051205in}}{\pgfqpoint{0.565736in}{4.059019in}}%
\pgfpathcurveto{\pgfqpoint{0.557923in}{4.066833in}}{\pgfqpoint{0.547324in}{4.071223in}}{\pgfqpoint{0.536273in}{4.071223in}}%
\pgfpathcurveto{\pgfqpoint{0.525223in}{4.071223in}}{\pgfqpoint{0.514624in}{4.066833in}}{\pgfqpoint{0.506811in}{4.059019in}}%
\pgfpathcurveto{\pgfqpoint{0.498997in}{4.051205in}}{\pgfqpoint{0.494607in}{4.040606in}}{\pgfqpoint{0.494607in}{4.029556in}}%
\pgfpathcurveto{\pgfqpoint{0.494607in}{4.018506in}}{\pgfqpoint{0.498997in}{4.007907in}}{\pgfqpoint{0.506811in}{4.000093in}}%
\pgfpathcurveto{\pgfqpoint{0.514624in}{3.992280in}}{\pgfqpoint{0.525223in}{3.987890in}}{\pgfqpoint{0.536273in}{3.987890in}}%
\pgfpathlineto{\pgfqpoint{0.536273in}{3.987890in}}%
\pgfpathclose%
\pgfusepath{stroke}%
\end{pgfscope}%
\begin{pgfscope}%
\pgfpathrectangle{\pgfqpoint{0.494722in}{0.437222in}}{\pgfqpoint{6.275590in}{5.159444in}}%
\pgfusepath{clip}%
\pgfsetbuttcap%
\pgfsetroundjoin%
\pgfsetlinewidth{1.003750pt}%
\definecolor{currentstroke}{rgb}{0.827451,0.827451,0.827451}%
\pgfsetstrokecolor{currentstroke}%
\pgfsetstrokeopacity{0.800000}%
\pgfsetdash{}{0pt}%
\pgfpathmoveto{\pgfqpoint{1.704904in}{1.849160in}}%
\pgfpathcurveto{\pgfqpoint{1.715954in}{1.849160in}}{\pgfqpoint{1.726553in}{1.853550in}}{\pgfqpoint{1.734367in}{1.861364in}}%
\pgfpathcurveto{\pgfqpoint{1.742180in}{1.869178in}}{\pgfqpoint{1.746571in}{1.879777in}}{\pgfqpoint{1.746571in}{1.890827in}}%
\pgfpathcurveto{\pgfqpoint{1.746571in}{1.901877in}}{\pgfqpoint{1.742180in}{1.912476in}}{\pgfqpoint{1.734367in}{1.920290in}}%
\pgfpathcurveto{\pgfqpoint{1.726553in}{1.928103in}}{\pgfqpoint{1.715954in}{1.932493in}}{\pgfqpoint{1.704904in}{1.932493in}}%
\pgfpathcurveto{\pgfqpoint{1.693854in}{1.932493in}}{\pgfqpoint{1.683255in}{1.928103in}}{\pgfqpoint{1.675441in}{1.920290in}}%
\pgfpathcurveto{\pgfqpoint{1.667628in}{1.912476in}}{\pgfqpoint{1.663237in}{1.901877in}}{\pgfqpoint{1.663237in}{1.890827in}}%
\pgfpathcurveto{\pgfqpoint{1.663237in}{1.879777in}}{\pgfqpoint{1.667628in}{1.869178in}}{\pgfqpoint{1.675441in}{1.861364in}}%
\pgfpathcurveto{\pgfqpoint{1.683255in}{1.853550in}}{\pgfqpoint{1.693854in}{1.849160in}}{\pgfqpoint{1.704904in}{1.849160in}}%
\pgfpathlineto{\pgfqpoint{1.704904in}{1.849160in}}%
\pgfpathclose%
\pgfusepath{stroke}%
\end{pgfscope}%
\begin{pgfscope}%
\pgfpathrectangle{\pgfqpoint{0.494722in}{0.437222in}}{\pgfqpoint{6.275590in}{5.159444in}}%
\pgfusepath{clip}%
\pgfsetbuttcap%
\pgfsetroundjoin%
\pgfsetlinewidth{1.003750pt}%
\definecolor{currentstroke}{rgb}{0.827451,0.827451,0.827451}%
\pgfsetstrokecolor{currentstroke}%
\pgfsetstrokeopacity{0.800000}%
\pgfsetdash{}{0pt}%
\pgfpathmoveto{\pgfqpoint{2.589198in}{1.110833in}}%
\pgfpathcurveto{\pgfqpoint{2.600248in}{1.110833in}}{\pgfqpoint{2.610847in}{1.115224in}}{\pgfqpoint{2.618661in}{1.123037in}}%
\pgfpathcurveto{\pgfqpoint{2.626474in}{1.130851in}}{\pgfqpoint{2.630865in}{1.141450in}}{\pgfqpoint{2.630865in}{1.152500in}}%
\pgfpathcurveto{\pgfqpoint{2.630865in}{1.163550in}}{\pgfqpoint{2.626474in}{1.174149in}}{\pgfqpoint{2.618661in}{1.181963in}}%
\pgfpathcurveto{\pgfqpoint{2.610847in}{1.189777in}}{\pgfqpoint{2.600248in}{1.194167in}}{\pgfqpoint{2.589198in}{1.194167in}}%
\pgfpathcurveto{\pgfqpoint{2.578148in}{1.194167in}}{\pgfqpoint{2.567549in}{1.189777in}}{\pgfqpoint{2.559735in}{1.181963in}}%
\pgfpathcurveto{\pgfqpoint{2.551922in}{1.174149in}}{\pgfqpoint{2.547531in}{1.163550in}}{\pgfqpoint{2.547531in}{1.152500in}}%
\pgfpathcurveto{\pgfqpoint{2.547531in}{1.141450in}}{\pgfqpoint{2.551922in}{1.130851in}}{\pgfqpoint{2.559735in}{1.123037in}}%
\pgfpathcurveto{\pgfqpoint{2.567549in}{1.115224in}}{\pgfqpoint{2.578148in}{1.110833in}}{\pgfqpoint{2.589198in}{1.110833in}}%
\pgfpathlineto{\pgfqpoint{2.589198in}{1.110833in}}%
\pgfpathclose%
\pgfusepath{stroke}%
\end{pgfscope}%
\begin{pgfscope}%
\pgfpathrectangle{\pgfqpoint{0.494722in}{0.437222in}}{\pgfqpoint{6.275590in}{5.159444in}}%
\pgfusepath{clip}%
\pgfsetbuttcap%
\pgfsetroundjoin%
\pgfsetlinewidth{1.003750pt}%
\definecolor{currentstroke}{rgb}{0.827451,0.827451,0.827451}%
\pgfsetstrokecolor{currentstroke}%
\pgfsetstrokeopacity{0.800000}%
\pgfsetdash{}{0pt}%
\pgfpathmoveto{\pgfqpoint{3.753908in}{0.666834in}}%
\pgfpathcurveto{\pgfqpoint{3.764958in}{0.666834in}}{\pgfqpoint{3.775557in}{0.671224in}}{\pgfqpoint{3.783371in}{0.679038in}}%
\pgfpathcurveto{\pgfqpoint{3.791185in}{0.686852in}}{\pgfqpoint{3.795575in}{0.697451in}}{\pgfqpoint{3.795575in}{0.708501in}}%
\pgfpathcurveto{\pgfqpoint{3.795575in}{0.719551in}}{\pgfqpoint{3.791185in}{0.730150in}}{\pgfqpoint{3.783371in}{0.737964in}}%
\pgfpathcurveto{\pgfqpoint{3.775557in}{0.745777in}}{\pgfqpoint{3.764958in}{0.750167in}}{\pgfqpoint{3.753908in}{0.750167in}}%
\pgfpathcurveto{\pgfqpoint{3.742858in}{0.750167in}}{\pgfqpoint{3.732259in}{0.745777in}}{\pgfqpoint{3.724445in}{0.737964in}}%
\pgfpathcurveto{\pgfqpoint{3.716632in}{0.730150in}}{\pgfqpoint{3.712242in}{0.719551in}}{\pgfqpoint{3.712242in}{0.708501in}}%
\pgfpathcurveto{\pgfqpoint{3.712242in}{0.697451in}}{\pgfqpoint{3.716632in}{0.686852in}}{\pgfqpoint{3.724445in}{0.679038in}}%
\pgfpathcurveto{\pgfqpoint{3.732259in}{0.671224in}}{\pgfqpoint{3.742858in}{0.666834in}}{\pgfqpoint{3.753908in}{0.666834in}}%
\pgfpathlineto{\pgfqpoint{3.753908in}{0.666834in}}%
\pgfpathclose%
\pgfusepath{stroke}%
\end{pgfscope}%
\begin{pgfscope}%
\pgfpathrectangle{\pgfqpoint{0.494722in}{0.437222in}}{\pgfqpoint{6.275590in}{5.159444in}}%
\pgfusepath{clip}%
\pgfsetbuttcap%
\pgfsetroundjoin%
\pgfsetlinewidth{1.003750pt}%
\definecolor{currentstroke}{rgb}{0.827451,0.827451,0.827451}%
\pgfsetstrokecolor{currentstroke}%
\pgfsetstrokeopacity{0.800000}%
\pgfsetdash{}{0pt}%
\pgfpathmoveto{\pgfqpoint{1.482549in}{1.997645in}}%
\pgfpathcurveto{\pgfqpoint{1.493599in}{1.997645in}}{\pgfqpoint{1.504198in}{2.002035in}}{\pgfqpoint{1.512012in}{2.009849in}}%
\pgfpathcurveto{\pgfqpoint{1.519825in}{2.017662in}}{\pgfqpoint{1.524216in}{2.028261in}}{\pgfqpoint{1.524216in}{2.039312in}}%
\pgfpathcurveto{\pgfqpoint{1.524216in}{2.050362in}}{\pgfqpoint{1.519825in}{2.060961in}}{\pgfqpoint{1.512012in}{2.068774in}}%
\pgfpathcurveto{\pgfqpoint{1.504198in}{2.076588in}}{\pgfqpoint{1.493599in}{2.080978in}}{\pgfqpoint{1.482549in}{2.080978in}}%
\pgfpathcurveto{\pgfqpoint{1.471499in}{2.080978in}}{\pgfqpoint{1.460900in}{2.076588in}}{\pgfqpoint{1.453086in}{2.068774in}}%
\pgfpathcurveto{\pgfqpoint{1.445273in}{2.060961in}}{\pgfqpoint{1.440882in}{2.050362in}}{\pgfqpoint{1.440882in}{2.039312in}}%
\pgfpathcurveto{\pgfqpoint{1.440882in}{2.028261in}}{\pgfqpoint{1.445273in}{2.017662in}}{\pgfqpoint{1.453086in}{2.009849in}}%
\pgfpathcurveto{\pgfqpoint{1.460900in}{2.002035in}}{\pgfqpoint{1.471499in}{1.997645in}}{\pgfqpoint{1.482549in}{1.997645in}}%
\pgfpathlineto{\pgfqpoint{1.482549in}{1.997645in}}%
\pgfpathclose%
\pgfusepath{stroke}%
\end{pgfscope}%
\begin{pgfscope}%
\pgfpathrectangle{\pgfqpoint{0.494722in}{0.437222in}}{\pgfqpoint{6.275590in}{5.159444in}}%
\pgfusepath{clip}%
\pgfsetbuttcap%
\pgfsetroundjoin%
\pgfsetlinewidth{1.003750pt}%
\definecolor{currentstroke}{rgb}{0.827451,0.827451,0.827451}%
\pgfsetstrokecolor{currentstroke}%
\pgfsetstrokeopacity{0.800000}%
\pgfsetdash{}{0pt}%
\pgfpathmoveto{\pgfqpoint{3.177676in}{0.931251in}}%
\pgfpathcurveto{\pgfqpoint{3.188726in}{0.931251in}}{\pgfqpoint{3.199325in}{0.935642in}}{\pgfqpoint{3.207139in}{0.943455in}}%
\pgfpathcurveto{\pgfqpoint{3.214952in}{0.951269in}}{\pgfqpoint{3.219342in}{0.961868in}}{\pgfqpoint{3.219342in}{0.972918in}}%
\pgfpathcurveto{\pgfqpoint{3.219342in}{0.983968in}}{\pgfqpoint{3.214952in}{0.994567in}}{\pgfqpoint{3.207139in}{1.002381in}}%
\pgfpathcurveto{\pgfqpoint{3.199325in}{1.010195in}}{\pgfqpoint{3.188726in}{1.014585in}}{\pgfqpoint{3.177676in}{1.014585in}}%
\pgfpathcurveto{\pgfqpoint{3.166626in}{1.014585in}}{\pgfqpoint{3.156027in}{1.010195in}}{\pgfqpoint{3.148213in}{1.002381in}}%
\pgfpathcurveto{\pgfqpoint{3.140399in}{0.994567in}}{\pgfqpoint{3.136009in}{0.983968in}}{\pgfqpoint{3.136009in}{0.972918in}}%
\pgfpathcurveto{\pgfqpoint{3.136009in}{0.961868in}}{\pgfqpoint{3.140399in}{0.951269in}}{\pgfqpoint{3.148213in}{0.943455in}}%
\pgfpathcurveto{\pgfqpoint{3.156027in}{0.935642in}}{\pgfqpoint{3.166626in}{0.931251in}}{\pgfqpoint{3.177676in}{0.931251in}}%
\pgfpathlineto{\pgfqpoint{3.177676in}{0.931251in}}%
\pgfpathclose%
\pgfusepath{stroke}%
\end{pgfscope}%
\begin{pgfscope}%
\pgfpathrectangle{\pgfqpoint{0.494722in}{0.437222in}}{\pgfqpoint{6.275590in}{5.159444in}}%
\pgfusepath{clip}%
\pgfsetbuttcap%
\pgfsetroundjoin%
\pgfsetlinewidth{1.003750pt}%
\definecolor{currentstroke}{rgb}{0.827451,0.827451,0.827451}%
\pgfsetstrokecolor{currentstroke}%
\pgfsetstrokeopacity{0.800000}%
\pgfsetdash{}{0pt}%
\pgfpathmoveto{\pgfqpoint{1.064415in}{2.774487in}}%
\pgfpathcurveto{\pgfqpoint{1.075465in}{2.774487in}}{\pgfqpoint{1.086064in}{2.778877in}}{\pgfqpoint{1.093878in}{2.786691in}}%
\pgfpathcurveto{\pgfqpoint{1.101691in}{2.794504in}}{\pgfqpoint{1.106082in}{2.805103in}}{\pgfqpoint{1.106082in}{2.816154in}}%
\pgfpathcurveto{\pgfqpoint{1.106082in}{2.827204in}}{\pgfqpoint{1.101691in}{2.837803in}}{\pgfqpoint{1.093878in}{2.845616in}}%
\pgfpathcurveto{\pgfqpoint{1.086064in}{2.853430in}}{\pgfqpoint{1.075465in}{2.857820in}}{\pgfqpoint{1.064415in}{2.857820in}}%
\pgfpathcurveto{\pgfqpoint{1.053365in}{2.857820in}}{\pgfqpoint{1.042766in}{2.853430in}}{\pgfqpoint{1.034952in}{2.845616in}}%
\pgfpathcurveto{\pgfqpoint{1.027139in}{2.837803in}}{\pgfqpoint{1.022748in}{2.827204in}}{\pgfqpoint{1.022748in}{2.816154in}}%
\pgfpathcurveto{\pgfqpoint{1.022748in}{2.805103in}}{\pgfqpoint{1.027139in}{2.794504in}}{\pgfqpoint{1.034952in}{2.786691in}}%
\pgfpathcurveto{\pgfqpoint{1.042766in}{2.778877in}}{\pgfqpoint{1.053365in}{2.774487in}}{\pgfqpoint{1.064415in}{2.774487in}}%
\pgfpathlineto{\pgfqpoint{1.064415in}{2.774487in}}%
\pgfpathclose%
\pgfusepath{stroke}%
\end{pgfscope}%
\begin{pgfscope}%
\pgfpathrectangle{\pgfqpoint{0.494722in}{0.437222in}}{\pgfqpoint{6.275590in}{5.159444in}}%
\pgfusepath{clip}%
\pgfsetbuttcap%
\pgfsetroundjoin%
\pgfsetlinewidth{1.003750pt}%
\definecolor{currentstroke}{rgb}{0.827451,0.827451,0.827451}%
\pgfsetstrokecolor{currentstroke}%
\pgfsetstrokeopacity{0.800000}%
\pgfsetdash{}{0pt}%
\pgfpathmoveto{\pgfqpoint{2.904991in}{0.980053in}}%
\pgfpathcurveto{\pgfqpoint{2.916041in}{0.980053in}}{\pgfqpoint{2.926640in}{0.984443in}}{\pgfqpoint{2.934454in}{0.992257in}}%
\pgfpathcurveto{\pgfqpoint{2.942267in}{1.000071in}}{\pgfqpoint{2.946658in}{1.010670in}}{\pgfqpoint{2.946658in}{1.021720in}}%
\pgfpathcurveto{\pgfqpoint{2.946658in}{1.032770in}}{\pgfqpoint{2.942267in}{1.043369in}}{\pgfqpoint{2.934454in}{1.051183in}}%
\pgfpathcurveto{\pgfqpoint{2.926640in}{1.058996in}}{\pgfqpoint{2.916041in}{1.063387in}}{\pgfqpoint{2.904991in}{1.063387in}}%
\pgfpathcurveto{\pgfqpoint{2.893941in}{1.063387in}}{\pgfqpoint{2.883342in}{1.058996in}}{\pgfqpoint{2.875528in}{1.051183in}}%
\pgfpathcurveto{\pgfqpoint{2.867715in}{1.043369in}}{\pgfqpoint{2.863324in}{1.032770in}}{\pgfqpoint{2.863324in}{1.021720in}}%
\pgfpathcurveto{\pgfqpoint{2.863324in}{1.010670in}}{\pgfqpoint{2.867715in}{1.000071in}}{\pgfqpoint{2.875528in}{0.992257in}}%
\pgfpathcurveto{\pgfqpoint{2.883342in}{0.984443in}}{\pgfqpoint{2.893941in}{0.980053in}}{\pgfqpoint{2.904991in}{0.980053in}}%
\pgfpathlineto{\pgfqpoint{2.904991in}{0.980053in}}%
\pgfpathclose%
\pgfusepath{stroke}%
\end{pgfscope}%
\begin{pgfscope}%
\pgfpathrectangle{\pgfqpoint{0.494722in}{0.437222in}}{\pgfqpoint{6.275590in}{5.159444in}}%
\pgfusepath{clip}%
\pgfsetbuttcap%
\pgfsetroundjoin%
\pgfsetlinewidth{1.003750pt}%
\definecolor{currentstroke}{rgb}{0.827451,0.827451,0.827451}%
\pgfsetstrokecolor{currentstroke}%
\pgfsetstrokeopacity{0.800000}%
\pgfsetdash{}{0pt}%
\pgfpathmoveto{\pgfqpoint{3.722236in}{0.790666in}}%
\pgfpathcurveto{\pgfqpoint{3.733286in}{0.790666in}}{\pgfqpoint{3.743885in}{0.795056in}}{\pgfqpoint{3.751699in}{0.802870in}}%
\pgfpathcurveto{\pgfqpoint{3.759513in}{0.810684in}}{\pgfqpoint{3.763903in}{0.821283in}}{\pgfqpoint{3.763903in}{0.832333in}}%
\pgfpathcurveto{\pgfqpoint{3.763903in}{0.843383in}}{\pgfqpoint{3.759513in}{0.853982in}}{\pgfqpoint{3.751699in}{0.861796in}}%
\pgfpathcurveto{\pgfqpoint{3.743885in}{0.869609in}}{\pgfqpoint{3.733286in}{0.873999in}}{\pgfqpoint{3.722236in}{0.873999in}}%
\pgfpathcurveto{\pgfqpoint{3.711186in}{0.873999in}}{\pgfqpoint{3.700587in}{0.869609in}}{\pgfqpoint{3.692773in}{0.861796in}}%
\pgfpathcurveto{\pgfqpoint{3.684960in}{0.853982in}}{\pgfqpoint{3.680570in}{0.843383in}}{\pgfqpoint{3.680570in}{0.832333in}}%
\pgfpathcurveto{\pgfqpoint{3.680570in}{0.821283in}}{\pgfqpoint{3.684960in}{0.810684in}}{\pgfqpoint{3.692773in}{0.802870in}}%
\pgfpathcurveto{\pgfqpoint{3.700587in}{0.795056in}}{\pgfqpoint{3.711186in}{0.790666in}}{\pgfqpoint{3.722236in}{0.790666in}}%
\pgfpathlineto{\pgfqpoint{3.722236in}{0.790666in}}%
\pgfpathclose%
\pgfusepath{stroke}%
\end{pgfscope}%
\begin{pgfscope}%
\pgfpathrectangle{\pgfqpoint{0.494722in}{0.437222in}}{\pgfqpoint{6.275590in}{5.159444in}}%
\pgfusepath{clip}%
\pgfsetbuttcap%
\pgfsetroundjoin%
\pgfsetlinewidth{1.003750pt}%
\definecolor{currentstroke}{rgb}{0.827451,0.827451,0.827451}%
\pgfsetstrokecolor{currentstroke}%
\pgfsetstrokeopacity{0.800000}%
\pgfsetdash{}{0pt}%
\pgfpathmoveto{\pgfqpoint{3.728055in}{0.692195in}}%
\pgfpathcurveto{\pgfqpoint{3.739105in}{0.692195in}}{\pgfqpoint{3.749704in}{0.696585in}}{\pgfqpoint{3.757518in}{0.704399in}}%
\pgfpathcurveto{\pgfqpoint{3.765332in}{0.712213in}}{\pgfqpoint{3.769722in}{0.722812in}}{\pgfqpoint{3.769722in}{0.733862in}}%
\pgfpathcurveto{\pgfqpoint{3.769722in}{0.744912in}}{\pgfqpoint{3.765332in}{0.755511in}}{\pgfqpoint{3.757518in}{0.763325in}}%
\pgfpathcurveto{\pgfqpoint{3.749704in}{0.771138in}}{\pgfqpoint{3.739105in}{0.775528in}}{\pgfqpoint{3.728055in}{0.775528in}}%
\pgfpathcurveto{\pgfqpoint{3.717005in}{0.775528in}}{\pgfqpoint{3.706406in}{0.771138in}}{\pgfqpoint{3.698593in}{0.763325in}}%
\pgfpathcurveto{\pgfqpoint{3.690779in}{0.755511in}}{\pgfqpoint{3.686389in}{0.744912in}}{\pgfqpoint{3.686389in}{0.733862in}}%
\pgfpathcurveto{\pgfqpoint{3.686389in}{0.722812in}}{\pgfqpoint{3.690779in}{0.712213in}}{\pgfqpoint{3.698593in}{0.704399in}}%
\pgfpathcurveto{\pgfqpoint{3.706406in}{0.696585in}}{\pgfqpoint{3.717005in}{0.692195in}}{\pgfqpoint{3.728055in}{0.692195in}}%
\pgfpathlineto{\pgfqpoint{3.728055in}{0.692195in}}%
\pgfpathclose%
\pgfusepath{stroke}%
\end{pgfscope}%
\begin{pgfscope}%
\pgfpathrectangle{\pgfqpoint{0.494722in}{0.437222in}}{\pgfqpoint{6.275590in}{5.159444in}}%
\pgfusepath{clip}%
\pgfsetbuttcap%
\pgfsetroundjoin%
\pgfsetlinewidth{1.003750pt}%
\definecolor{currentstroke}{rgb}{0.827451,0.827451,0.827451}%
\pgfsetstrokecolor{currentstroke}%
\pgfsetstrokeopacity{0.800000}%
\pgfsetdash{}{0pt}%
\pgfpathmoveto{\pgfqpoint{1.351308in}{2.103615in}}%
\pgfpathcurveto{\pgfqpoint{1.362358in}{2.103615in}}{\pgfqpoint{1.372957in}{2.108005in}}{\pgfqpoint{1.380771in}{2.115819in}}%
\pgfpathcurveto{\pgfqpoint{1.388585in}{2.123633in}}{\pgfqpoint{1.392975in}{2.134232in}}{\pgfqpoint{1.392975in}{2.145282in}}%
\pgfpathcurveto{\pgfqpoint{1.392975in}{2.156332in}}{\pgfqpoint{1.388585in}{2.166931in}}{\pgfqpoint{1.380771in}{2.174745in}}%
\pgfpathcurveto{\pgfqpoint{1.372957in}{2.182558in}}{\pgfqpoint{1.362358in}{2.186949in}}{\pgfqpoint{1.351308in}{2.186949in}}%
\pgfpathcurveto{\pgfqpoint{1.340258in}{2.186949in}}{\pgfqpoint{1.329659in}{2.182558in}}{\pgfqpoint{1.321845in}{2.174745in}}%
\pgfpathcurveto{\pgfqpoint{1.314032in}{2.166931in}}{\pgfqpoint{1.309641in}{2.156332in}}{\pgfqpoint{1.309641in}{2.145282in}}%
\pgfpathcurveto{\pgfqpoint{1.309641in}{2.134232in}}{\pgfqpoint{1.314032in}{2.123633in}}{\pgfqpoint{1.321845in}{2.115819in}}%
\pgfpathcurveto{\pgfqpoint{1.329659in}{2.108005in}}{\pgfqpoint{1.340258in}{2.103615in}}{\pgfqpoint{1.351308in}{2.103615in}}%
\pgfpathlineto{\pgfqpoint{1.351308in}{2.103615in}}%
\pgfpathclose%
\pgfusepath{stroke}%
\end{pgfscope}%
\begin{pgfscope}%
\pgfpathrectangle{\pgfqpoint{0.494722in}{0.437222in}}{\pgfqpoint{6.275590in}{5.159444in}}%
\pgfusepath{clip}%
\pgfsetbuttcap%
\pgfsetroundjoin%
\pgfsetlinewidth{1.003750pt}%
\definecolor{currentstroke}{rgb}{0.827451,0.827451,0.827451}%
\pgfsetstrokecolor{currentstroke}%
\pgfsetstrokeopacity{0.800000}%
\pgfsetdash{}{0pt}%
\pgfpathmoveto{\pgfqpoint{1.365286in}{2.103593in}}%
\pgfpathcurveto{\pgfqpoint{1.376336in}{2.103593in}}{\pgfqpoint{1.386936in}{2.107984in}}{\pgfqpoint{1.394749in}{2.115797in}}%
\pgfpathcurveto{\pgfqpoint{1.402563in}{2.123611in}}{\pgfqpoint{1.406953in}{2.134210in}}{\pgfqpoint{1.406953in}{2.145260in}}%
\pgfpathcurveto{\pgfqpoint{1.406953in}{2.156310in}}{\pgfqpoint{1.402563in}{2.166909in}}{\pgfqpoint{1.394749in}{2.174723in}}%
\pgfpathcurveto{\pgfqpoint{1.386936in}{2.182537in}}{\pgfqpoint{1.376336in}{2.186927in}}{\pgfqpoint{1.365286in}{2.186927in}}%
\pgfpathcurveto{\pgfqpoint{1.354236in}{2.186927in}}{\pgfqpoint{1.343637in}{2.182537in}}{\pgfqpoint{1.335824in}{2.174723in}}%
\pgfpathcurveto{\pgfqpoint{1.328010in}{2.166909in}}{\pgfqpoint{1.323620in}{2.156310in}}{\pgfqpoint{1.323620in}{2.145260in}}%
\pgfpathcurveto{\pgfqpoint{1.323620in}{2.134210in}}{\pgfqpoint{1.328010in}{2.123611in}}{\pgfqpoint{1.335824in}{2.115797in}}%
\pgfpathcurveto{\pgfqpoint{1.343637in}{2.107984in}}{\pgfqpoint{1.354236in}{2.103593in}}{\pgfqpoint{1.365286in}{2.103593in}}%
\pgfpathlineto{\pgfqpoint{1.365286in}{2.103593in}}%
\pgfpathclose%
\pgfusepath{stroke}%
\end{pgfscope}%
\begin{pgfscope}%
\pgfpathrectangle{\pgfqpoint{0.494722in}{0.437222in}}{\pgfqpoint{6.275590in}{5.159444in}}%
\pgfusepath{clip}%
\pgfsetbuttcap%
\pgfsetroundjoin%
\pgfsetlinewidth{1.003750pt}%
\definecolor{currentstroke}{rgb}{0.827451,0.827451,0.827451}%
\pgfsetstrokecolor{currentstroke}%
\pgfsetstrokeopacity{0.800000}%
\pgfsetdash{}{0pt}%
\pgfpathmoveto{\pgfqpoint{3.980070in}{0.633768in}}%
\pgfpathcurveto{\pgfqpoint{3.991120in}{0.633768in}}{\pgfqpoint{4.001719in}{0.638158in}}{\pgfqpoint{4.009533in}{0.645971in}}%
\pgfpathcurveto{\pgfqpoint{4.017347in}{0.653785in}}{\pgfqpoint{4.021737in}{0.664384in}}{\pgfqpoint{4.021737in}{0.675434in}}%
\pgfpathcurveto{\pgfqpoint{4.021737in}{0.686484in}}{\pgfqpoint{4.017347in}{0.697083in}}{\pgfqpoint{4.009533in}{0.704897in}}%
\pgfpathcurveto{\pgfqpoint{4.001719in}{0.712711in}}{\pgfqpoint{3.991120in}{0.717101in}}{\pgfqpoint{3.980070in}{0.717101in}}%
\pgfpathcurveto{\pgfqpoint{3.969020in}{0.717101in}}{\pgfqpoint{3.958421in}{0.712711in}}{\pgfqpoint{3.950608in}{0.704897in}}%
\pgfpathcurveto{\pgfqpoint{3.942794in}{0.697083in}}{\pgfqpoint{3.938404in}{0.686484in}}{\pgfqpoint{3.938404in}{0.675434in}}%
\pgfpathcurveto{\pgfqpoint{3.938404in}{0.664384in}}{\pgfqpoint{3.942794in}{0.653785in}}{\pgfqpoint{3.950608in}{0.645971in}}%
\pgfpathcurveto{\pgfqpoint{3.958421in}{0.638158in}}{\pgfqpoint{3.969020in}{0.633768in}}{\pgfqpoint{3.980070in}{0.633768in}}%
\pgfpathlineto{\pgfqpoint{3.980070in}{0.633768in}}%
\pgfpathclose%
\pgfusepath{stroke}%
\end{pgfscope}%
\begin{pgfscope}%
\pgfpathrectangle{\pgfqpoint{0.494722in}{0.437222in}}{\pgfqpoint{6.275590in}{5.159444in}}%
\pgfusepath{clip}%
\pgfsetbuttcap%
\pgfsetroundjoin%
\pgfsetlinewidth{1.003750pt}%
\definecolor{currentstroke}{rgb}{0.827451,0.827451,0.827451}%
\pgfsetstrokecolor{currentstroke}%
\pgfsetstrokeopacity{0.800000}%
\pgfsetdash{}{0pt}%
\pgfpathmoveto{\pgfqpoint{0.928797in}{3.189816in}}%
\pgfpathcurveto{\pgfqpoint{0.939847in}{3.189816in}}{\pgfqpoint{0.950446in}{3.194207in}}{\pgfqpoint{0.958260in}{3.202020in}}%
\pgfpathcurveto{\pgfqpoint{0.966073in}{3.209834in}}{\pgfqpoint{0.970464in}{3.220433in}}{\pgfqpoint{0.970464in}{3.231483in}}%
\pgfpathcurveto{\pgfqpoint{0.970464in}{3.242533in}}{\pgfqpoint{0.966073in}{3.253132in}}{\pgfqpoint{0.958260in}{3.260946in}}%
\pgfpathcurveto{\pgfqpoint{0.950446in}{3.268760in}}{\pgfqpoint{0.939847in}{3.273150in}}{\pgfqpoint{0.928797in}{3.273150in}}%
\pgfpathcurveto{\pgfqpoint{0.917747in}{3.273150in}}{\pgfqpoint{0.907148in}{3.268760in}}{\pgfqpoint{0.899334in}{3.260946in}}%
\pgfpathcurveto{\pgfqpoint{0.891521in}{3.253132in}}{\pgfqpoint{0.887130in}{3.242533in}}{\pgfqpoint{0.887130in}{3.231483in}}%
\pgfpathcurveto{\pgfqpoint{0.887130in}{3.220433in}}{\pgfqpoint{0.891521in}{3.209834in}}{\pgfqpoint{0.899334in}{3.202020in}}%
\pgfpathcurveto{\pgfqpoint{0.907148in}{3.194207in}}{\pgfqpoint{0.917747in}{3.189816in}}{\pgfqpoint{0.928797in}{3.189816in}}%
\pgfpathlineto{\pgfqpoint{0.928797in}{3.189816in}}%
\pgfpathclose%
\pgfusepath{stroke}%
\end{pgfscope}%
\begin{pgfscope}%
\pgfpathrectangle{\pgfqpoint{0.494722in}{0.437222in}}{\pgfqpoint{6.275590in}{5.159444in}}%
\pgfusepath{clip}%
\pgfsetbuttcap%
\pgfsetroundjoin%
\pgfsetlinewidth{1.003750pt}%
\definecolor{currentstroke}{rgb}{0.827451,0.827451,0.827451}%
\pgfsetstrokecolor{currentstroke}%
\pgfsetstrokeopacity{0.800000}%
\pgfsetdash{}{0pt}%
\pgfpathmoveto{\pgfqpoint{5.600353in}{0.494251in}}%
\pgfpathcurveto{\pgfqpoint{5.611403in}{0.494251in}}{\pgfqpoint{5.622002in}{0.498641in}}{\pgfqpoint{5.629816in}{0.506455in}}%
\pgfpathcurveto{\pgfqpoint{5.637629in}{0.514268in}}{\pgfqpoint{5.642020in}{0.524867in}}{\pgfqpoint{5.642020in}{0.535918in}}%
\pgfpathcurveto{\pgfqpoint{5.642020in}{0.546968in}}{\pgfqpoint{5.637629in}{0.557567in}}{\pgfqpoint{5.629816in}{0.565380in}}%
\pgfpathcurveto{\pgfqpoint{5.622002in}{0.573194in}}{\pgfqpoint{5.611403in}{0.577584in}}{\pgfqpoint{5.600353in}{0.577584in}}%
\pgfpathcurveto{\pgfqpoint{5.589303in}{0.577584in}}{\pgfqpoint{5.578704in}{0.573194in}}{\pgfqpoint{5.570890in}{0.565380in}}%
\pgfpathcurveto{\pgfqpoint{5.563077in}{0.557567in}}{\pgfqpoint{5.558686in}{0.546968in}}{\pgfqpoint{5.558686in}{0.535918in}}%
\pgfpathcurveto{\pgfqpoint{5.558686in}{0.524867in}}{\pgfqpoint{5.563077in}{0.514268in}}{\pgfqpoint{5.570890in}{0.506455in}}%
\pgfpathcurveto{\pgfqpoint{5.578704in}{0.498641in}}{\pgfqpoint{5.589303in}{0.494251in}}{\pgfqpoint{5.600353in}{0.494251in}}%
\pgfpathlineto{\pgfqpoint{5.600353in}{0.494251in}}%
\pgfpathclose%
\pgfusepath{stroke}%
\end{pgfscope}%
\begin{pgfscope}%
\pgfpathrectangle{\pgfqpoint{0.494722in}{0.437222in}}{\pgfqpoint{6.275590in}{5.159444in}}%
\pgfusepath{clip}%
\pgfsetbuttcap%
\pgfsetroundjoin%
\pgfsetlinewidth{1.003750pt}%
\definecolor{currentstroke}{rgb}{0.827451,0.827451,0.827451}%
\pgfsetstrokecolor{currentstroke}%
\pgfsetstrokeopacity{0.800000}%
\pgfsetdash{}{0pt}%
\pgfpathmoveto{\pgfqpoint{1.421539in}{2.095628in}}%
\pgfpathcurveto{\pgfqpoint{1.432589in}{2.095628in}}{\pgfqpoint{1.443188in}{2.100019in}}{\pgfqpoint{1.451002in}{2.107832in}}%
\pgfpathcurveto{\pgfqpoint{1.458816in}{2.115646in}}{\pgfqpoint{1.463206in}{2.126245in}}{\pgfqpoint{1.463206in}{2.137295in}}%
\pgfpathcurveto{\pgfqpoint{1.463206in}{2.148345in}}{\pgfqpoint{1.458816in}{2.158944in}}{\pgfqpoint{1.451002in}{2.166758in}}%
\pgfpathcurveto{\pgfqpoint{1.443188in}{2.174571in}}{\pgfqpoint{1.432589in}{2.178962in}}{\pgfqpoint{1.421539in}{2.178962in}}%
\pgfpathcurveto{\pgfqpoint{1.410489in}{2.178962in}}{\pgfqpoint{1.399890in}{2.174571in}}{\pgfqpoint{1.392077in}{2.166758in}}%
\pgfpathcurveto{\pgfqpoint{1.384263in}{2.158944in}}{\pgfqpoint{1.379873in}{2.148345in}}{\pgfqpoint{1.379873in}{2.137295in}}%
\pgfpathcurveto{\pgfqpoint{1.379873in}{2.126245in}}{\pgfqpoint{1.384263in}{2.115646in}}{\pgfqpoint{1.392077in}{2.107832in}}%
\pgfpathcurveto{\pgfqpoint{1.399890in}{2.100019in}}{\pgfqpoint{1.410489in}{2.095628in}}{\pgfqpoint{1.421539in}{2.095628in}}%
\pgfpathlineto{\pgfqpoint{1.421539in}{2.095628in}}%
\pgfpathclose%
\pgfusepath{stroke}%
\end{pgfscope}%
\begin{pgfscope}%
\pgfpathrectangle{\pgfqpoint{0.494722in}{0.437222in}}{\pgfqpoint{6.275590in}{5.159444in}}%
\pgfusepath{clip}%
\pgfsetbuttcap%
\pgfsetroundjoin%
\pgfsetlinewidth{1.003750pt}%
\definecolor{currentstroke}{rgb}{0.827451,0.827451,0.827451}%
\pgfsetstrokecolor{currentstroke}%
\pgfsetstrokeopacity{0.800000}%
\pgfsetdash{}{0pt}%
\pgfpathmoveto{\pgfqpoint{1.792434in}{1.664641in}}%
\pgfpathcurveto{\pgfqpoint{1.803485in}{1.664641in}}{\pgfqpoint{1.814084in}{1.669032in}}{\pgfqpoint{1.821897in}{1.676845in}}%
\pgfpathcurveto{\pgfqpoint{1.829711in}{1.684659in}}{\pgfqpoint{1.834101in}{1.695258in}}{\pgfqpoint{1.834101in}{1.706308in}}%
\pgfpathcurveto{\pgfqpoint{1.834101in}{1.717358in}}{\pgfqpoint{1.829711in}{1.727957in}}{\pgfqpoint{1.821897in}{1.735771in}}%
\pgfpathcurveto{\pgfqpoint{1.814084in}{1.743584in}}{\pgfqpoint{1.803485in}{1.747975in}}{\pgfqpoint{1.792434in}{1.747975in}}%
\pgfpathcurveto{\pgfqpoint{1.781384in}{1.747975in}}{\pgfqpoint{1.770785in}{1.743584in}}{\pgfqpoint{1.762972in}{1.735771in}}%
\pgfpathcurveto{\pgfqpoint{1.755158in}{1.727957in}}{\pgfqpoint{1.750768in}{1.717358in}}{\pgfqpoint{1.750768in}{1.706308in}}%
\pgfpathcurveto{\pgfqpoint{1.750768in}{1.695258in}}{\pgfqpoint{1.755158in}{1.684659in}}{\pgfqpoint{1.762972in}{1.676845in}}%
\pgfpathcurveto{\pgfqpoint{1.770785in}{1.669032in}}{\pgfqpoint{1.781384in}{1.664641in}}{\pgfqpoint{1.792434in}{1.664641in}}%
\pgfpathlineto{\pgfqpoint{1.792434in}{1.664641in}}%
\pgfpathclose%
\pgfusepath{stroke}%
\end{pgfscope}%
\begin{pgfscope}%
\pgfpathrectangle{\pgfqpoint{0.494722in}{0.437222in}}{\pgfqpoint{6.275590in}{5.159444in}}%
\pgfusepath{clip}%
\pgfsetbuttcap%
\pgfsetroundjoin%
\pgfsetlinewidth{1.003750pt}%
\definecolor{currentstroke}{rgb}{0.827451,0.827451,0.827451}%
\pgfsetstrokecolor{currentstroke}%
\pgfsetstrokeopacity{0.800000}%
\pgfsetdash{}{0pt}%
\pgfpathmoveto{\pgfqpoint{0.709087in}{3.392876in}}%
\pgfpathcurveto{\pgfqpoint{0.720137in}{3.392876in}}{\pgfqpoint{0.730736in}{3.397266in}}{\pgfqpoint{0.738550in}{3.405079in}}%
\pgfpathcurveto{\pgfqpoint{0.746363in}{3.412893in}}{\pgfqpoint{0.750753in}{3.423492in}}{\pgfqpoint{0.750753in}{3.434542in}}%
\pgfpathcurveto{\pgfqpoint{0.750753in}{3.445592in}}{\pgfqpoint{0.746363in}{3.456191in}}{\pgfqpoint{0.738550in}{3.464005in}}%
\pgfpathcurveto{\pgfqpoint{0.730736in}{3.471819in}}{\pgfqpoint{0.720137in}{3.476209in}}{\pgfqpoint{0.709087in}{3.476209in}}%
\pgfpathcurveto{\pgfqpoint{0.698037in}{3.476209in}}{\pgfqpoint{0.687438in}{3.471819in}}{\pgfqpoint{0.679624in}{3.464005in}}%
\pgfpathcurveto{\pgfqpoint{0.671810in}{3.456191in}}{\pgfqpoint{0.667420in}{3.445592in}}{\pgfqpoint{0.667420in}{3.434542in}}%
\pgfpathcurveto{\pgfqpoint{0.667420in}{3.423492in}}{\pgfqpoint{0.671810in}{3.412893in}}{\pgfqpoint{0.679624in}{3.405079in}}%
\pgfpathcurveto{\pgfqpoint{0.687438in}{3.397266in}}{\pgfqpoint{0.698037in}{3.392876in}}{\pgfqpoint{0.709087in}{3.392876in}}%
\pgfpathlineto{\pgfqpoint{0.709087in}{3.392876in}}%
\pgfpathclose%
\pgfusepath{stroke}%
\end{pgfscope}%
\begin{pgfscope}%
\pgfpathrectangle{\pgfqpoint{0.494722in}{0.437222in}}{\pgfqpoint{6.275590in}{5.159444in}}%
\pgfusepath{clip}%
\pgfsetbuttcap%
\pgfsetroundjoin%
\pgfsetlinewidth{1.003750pt}%
\definecolor{currentstroke}{rgb}{0.827451,0.827451,0.827451}%
\pgfsetstrokecolor{currentstroke}%
\pgfsetstrokeopacity{0.800000}%
\pgfsetdash{}{0pt}%
\pgfpathmoveto{\pgfqpoint{1.342899in}{2.272259in}}%
\pgfpathcurveto{\pgfqpoint{1.353949in}{2.272259in}}{\pgfqpoint{1.364548in}{2.276649in}}{\pgfqpoint{1.372362in}{2.284463in}}%
\pgfpathcurveto{\pgfqpoint{1.380175in}{2.292277in}}{\pgfqpoint{1.384565in}{2.302876in}}{\pgfqpoint{1.384565in}{2.313926in}}%
\pgfpathcurveto{\pgfqpoint{1.384565in}{2.324976in}}{\pgfqpoint{1.380175in}{2.335575in}}{\pgfqpoint{1.372362in}{2.343389in}}%
\pgfpathcurveto{\pgfqpoint{1.364548in}{2.351202in}}{\pgfqpoint{1.353949in}{2.355592in}}{\pgfqpoint{1.342899in}{2.355592in}}%
\pgfpathcurveto{\pgfqpoint{1.331849in}{2.355592in}}{\pgfqpoint{1.321250in}{2.351202in}}{\pgfqpoint{1.313436in}{2.343389in}}%
\pgfpathcurveto{\pgfqpoint{1.305622in}{2.335575in}}{\pgfqpoint{1.301232in}{2.324976in}}{\pgfqpoint{1.301232in}{2.313926in}}%
\pgfpathcurveto{\pgfqpoint{1.301232in}{2.302876in}}{\pgfqpoint{1.305622in}{2.292277in}}{\pgfqpoint{1.313436in}{2.284463in}}%
\pgfpathcurveto{\pgfqpoint{1.321250in}{2.276649in}}{\pgfqpoint{1.331849in}{2.272259in}}{\pgfqpoint{1.342899in}{2.272259in}}%
\pgfpathlineto{\pgfqpoint{1.342899in}{2.272259in}}%
\pgfpathclose%
\pgfusepath{stroke}%
\end{pgfscope}%
\begin{pgfscope}%
\pgfpathrectangle{\pgfqpoint{0.494722in}{0.437222in}}{\pgfqpoint{6.275590in}{5.159444in}}%
\pgfusepath{clip}%
\pgfsetbuttcap%
\pgfsetroundjoin%
\pgfsetlinewidth{1.003750pt}%
\definecolor{currentstroke}{rgb}{0.827451,0.827451,0.827451}%
\pgfsetstrokecolor{currentstroke}%
\pgfsetstrokeopacity{0.800000}%
\pgfsetdash{}{0pt}%
\pgfpathmoveto{\pgfqpoint{1.325678in}{2.290604in}}%
\pgfpathcurveto{\pgfqpoint{1.336728in}{2.290604in}}{\pgfqpoint{1.347327in}{2.294994in}}{\pgfqpoint{1.355141in}{2.302808in}}%
\pgfpathcurveto{\pgfqpoint{1.362955in}{2.310622in}}{\pgfqpoint{1.367345in}{2.321221in}}{\pgfqpoint{1.367345in}{2.332271in}}%
\pgfpathcurveto{\pgfqpoint{1.367345in}{2.343321in}}{\pgfqpoint{1.362955in}{2.353920in}}{\pgfqpoint{1.355141in}{2.361733in}}%
\pgfpathcurveto{\pgfqpoint{1.347327in}{2.369547in}}{\pgfqpoint{1.336728in}{2.373937in}}{\pgfqpoint{1.325678in}{2.373937in}}%
\pgfpathcurveto{\pgfqpoint{1.314628in}{2.373937in}}{\pgfqpoint{1.304029in}{2.369547in}}{\pgfqpoint{1.296215in}{2.361733in}}%
\pgfpathcurveto{\pgfqpoint{1.288402in}{2.353920in}}{\pgfqpoint{1.284012in}{2.343321in}}{\pgfqpoint{1.284012in}{2.332271in}}%
\pgfpathcurveto{\pgfqpoint{1.284012in}{2.321221in}}{\pgfqpoint{1.288402in}{2.310622in}}{\pgfqpoint{1.296215in}{2.302808in}}%
\pgfpathcurveto{\pgfqpoint{1.304029in}{2.294994in}}{\pgfqpoint{1.314628in}{2.290604in}}{\pgfqpoint{1.325678in}{2.290604in}}%
\pgfpathlineto{\pgfqpoint{1.325678in}{2.290604in}}%
\pgfpathclose%
\pgfusepath{stroke}%
\end{pgfscope}%
\begin{pgfscope}%
\pgfpathrectangle{\pgfqpoint{0.494722in}{0.437222in}}{\pgfqpoint{6.275590in}{5.159444in}}%
\pgfusepath{clip}%
\pgfsetbuttcap%
\pgfsetroundjoin%
\pgfsetlinewidth{1.003750pt}%
\definecolor{currentstroke}{rgb}{0.827451,0.827451,0.827451}%
\pgfsetstrokecolor{currentstroke}%
\pgfsetstrokeopacity{0.800000}%
\pgfsetdash{}{0pt}%
\pgfpathmoveto{\pgfqpoint{3.346796in}{0.826687in}}%
\pgfpathcurveto{\pgfqpoint{3.357846in}{0.826687in}}{\pgfqpoint{3.368445in}{0.831077in}}{\pgfqpoint{3.376259in}{0.838891in}}%
\pgfpathcurveto{\pgfqpoint{3.384072in}{0.846704in}}{\pgfqpoint{3.388463in}{0.857303in}}{\pgfqpoint{3.388463in}{0.868354in}}%
\pgfpathcurveto{\pgfqpoint{3.388463in}{0.879404in}}{\pgfqpoint{3.384072in}{0.890003in}}{\pgfqpoint{3.376259in}{0.897816in}}%
\pgfpathcurveto{\pgfqpoint{3.368445in}{0.905630in}}{\pgfqpoint{3.357846in}{0.910020in}}{\pgfqpoint{3.346796in}{0.910020in}}%
\pgfpathcurveto{\pgfqpoint{3.335746in}{0.910020in}}{\pgfqpoint{3.325147in}{0.905630in}}{\pgfqpoint{3.317333in}{0.897816in}}%
\pgfpathcurveto{\pgfqpoint{3.309520in}{0.890003in}}{\pgfqpoint{3.305129in}{0.879404in}}{\pgfqpoint{3.305129in}{0.868354in}}%
\pgfpathcurveto{\pgfqpoint{3.305129in}{0.857303in}}{\pgfqpoint{3.309520in}{0.846704in}}{\pgfqpoint{3.317333in}{0.838891in}}%
\pgfpathcurveto{\pgfqpoint{3.325147in}{0.831077in}}{\pgfqpoint{3.335746in}{0.826687in}}{\pgfqpoint{3.346796in}{0.826687in}}%
\pgfpathlineto{\pgfqpoint{3.346796in}{0.826687in}}%
\pgfpathclose%
\pgfusepath{stroke}%
\end{pgfscope}%
\begin{pgfscope}%
\pgfpathrectangle{\pgfqpoint{0.494722in}{0.437222in}}{\pgfqpoint{6.275590in}{5.159444in}}%
\pgfusepath{clip}%
\pgfsetbuttcap%
\pgfsetroundjoin%
\pgfsetlinewidth{1.003750pt}%
\definecolor{currentstroke}{rgb}{0.827451,0.827451,0.827451}%
\pgfsetstrokecolor{currentstroke}%
\pgfsetstrokeopacity{0.800000}%
\pgfsetdash{}{0pt}%
\pgfpathmoveto{\pgfqpoint{3.672192in}{0.824807in}}%
\pgfpathcurveto{\pgfqpoint{3.683242in}{0.824807in}}{\pgfqpoint{3.693841in}{0.829198in}}{\pgfqpoint{3.701655in}{0.837011in}}%
\pgfpathcurveto{\pgfqpoint{3.709469in}{0.844825in}}{\pgfqpoint{3.713859in}{0.855424in}}{\pgfqpoint{3.713859in}{0.866474in}}%
\pgfpathcurveto{\pgfqpoint{3.713859in}{0.877524in}}{\pgfqpoint{3.709469in}{0.888123in}}{\pgfqpoint{3.701655in}{0.895937in}}%
\pgfpathcurveto{\pgfqpoint{3.693841in}{0.903750in}}{\pgfqpoint{3.683242in}{0.908141in}}{\pgfqpoint{3.672192in}{0.908141in}}%
\pgfpathcurveto{\pgfqpoint{3.661142in}{0.908141in}}{\pgfqpoint{3.650543in}{0.903750in}}{\pgfqpoint{3.642729in}{0.895937in}}%
\pgfpathcurveto{\pgfqpoint{3.634916in}{0.888123in}}{\pgfqpoint{3.630525in}{0.877524in}}{\pgfqpoint{3.630525in}{0.866474in}}%
\pgfpathcurveto{\pgfqpoint{3.630525in}{0.855424in}}{\pgfqpoint{3.634916in}{0.844825in}}{\pgfqpoint{3.642729in}{0.837011in}}%
\pgfpathcurveto{\pgfqpoint{3.650543in}{0.829198in}}{\pgfqpoint{3.661142in}{0.824807in}}{\pgfqpoint{3.672192in}{0.824807in}}%
\pgfpathlineto{\pgfqpoint{3.672192in}{0.824807in}}%
\pgfpathclose%
\pgfusepath{stroke}%
\end{pgfscope}%
\begin{pgfscope}%
\pgfpathrectangle{\pgfqpoint{0.494722in}{0.437222in}}{\pgfqpoint{6.275590in}{5.159444in}}%
\pgfusepath{clip}%
\pgfsetbuttcap%
\pgfsetroundjoin%
\pgfsetlinewidth{1.003750pt}%
\definecolor{currentstroke}{rgb}{0.827451,0.827451,0.827451}%
\pgfsetstrokecolor{currentstroke}%
\pgfsetstrokeopacity{0.800000}%
\pgfsetdash{}{0pt}%
\pgfpathmoveto{\pgfqpoint{2.125169in}{1.462026in}}%
\pgfpathcurveto{\pgfqpoint{2.136219in}{1.462026in}}{\pgfqpoint{2.146818in}{1.466416in}}{\pgfqpoint{2.154631in}{1.474229in}}%
\pgfpathcurveto{\pgfqpoint{2.162445in}{1.482043in}}{\pgfqpoint{2.166835in}{1.492642in}}{\pgfqpoint{2.166835in}{1.503692in}}%
\pgfpathcurveto{\pgfqpoint{2.166835in}{1.514742in}}{\pgfqpoint{2.162445in}{1.525341in}}{\pgfqpoint{2.154631in}{1.533155in}}%
\pgfpathcurveto{\pgfqpoint{2.146818in}{1.540969in}}{\pgfqpoint{2.136219in}{1.545359in}}{\pgfqpoint{2.125169in}{1.545359in}}%
\pgfpathcurveto{\pgfqpoint{2.114118in}{1.545359in}}{\pgfqpoint{2.103519in}{1.540969in}}{\pgfqpoint{2.095706in}{1.533155in}}%
\pgfpathcurveto{\pgfqpoint{2.087892in}{1.525341in}}{\pgfqpoint{2.083502in}{1.514742in}}{\pgfqpoint{2.083502in}{1.503692in}}%
\pgfpathcurveto{\pgfqpoint{2.083502in}{1.492642in}}{\pgfqpoint{2.087892in}{1.482043in}}{\pgfqpoint{2.095706in}{1.474229in}}%
\pgfpathcurveto{\pgfqpoint{2.103519in}{1.466416in}}{\pgfqpoint{2.114118in}{1.462026in}}{\pgfqpoint{2.125169in}{1.462026in}}%
\pgfpathlineto{\pgfqpoint{2.125169in}{1.462026in}}%
\pgfpathclose%
\pgfusepath{stroke}%
\end{pgfscope}%
\begin{pgfscope}%
\pgfpathrectangle{\pgfqpoint{0.494722in}{0.437222in}}{\pgfqpoint{6.275590in}{5.159444in}}%
\pgfusepath{clip}%
\pgfsetbuttcap%
\pgfsetroundjoin%
\pgfsetlinewidth{1.003750pt}%
\definecolor{currentstroke}{rgb}{0.827451,0.827451,0.827451}%
\pgfsetstrokecolor{currentstroke}%
\pgfsetstrokeopacity{0.800000}%
\pgfsetdash{}{0pt}%
\pgfpathmoveto{\pgfqpoint{0.767639in}{3.316864in}}%
\pgfpathcurveto{\pgfqpoint{0.778690in}{3.316864in}}{\pgfqpoint{0.789289in}{3.321254in}}{\pgfqpoint{0.797102in}{3.329068in}}%
\pgfpathcurveto{\pgfqpoint{0.804916in}{3.336882in}}{\pgfqpoint{0.809306in}{3.347481in}}{\pgfqpoint{0.809306in}{3.358531in}}%
\pgfpathcurveto{\pgfqpoint{0.809306in}{3.369581in}}{\pgfqpoint{0.804916in}{3.380180in}}{\pgfqpoint{0.797102in}{3.387994in}}%
\pgfpathcurveto{\pgfqpoint{0.789289in}{3.395807in}}{\pgfqpoint{0.778690in}{3.400198in}}{\pgfqpoint{0.767639in}{3.400198in}}%
\pgfpathcurveto{\pgfqpoint{0.756589in}{3.400198in}}{\pgfqpoint{0.745990in}{3.395807in}}{\pgfqpoint{0.738177in}{3.387994in}}%
\pgfpathcurveto{\pgfqpoint{0.730363in}{3.380180in}}{\pgfqpoint{0.725973in}{3.369581in}}{\pgfqpoint{0.725973in}{3.358531in}}%
\pgfpathcurveto{\pgfqpoint{0.725973in}{3.347481in}}{\pgfqpoint{0.730363in}{3.336882in}}{\pgfqpoint{0.738177in}{3.329068in}}%
\pgfpathcurveto{\pgfqpoint{0.745990in}{3.321254in}}{\pgfqpoint{0.756589in}{3.316864in}}{\pgfqpoint{0.767639in}{3.316864in}}%
\pgfpathlineto{\pgfqpoint{0.767639in}{3.316864in}}%
\pgfpathclose%
\pgfusepath{stroke}%
\end{pgfscope}%
\begin{pgfscope}%
\pgfpathrectangle{\pgfqpoint{0.494722in}{0.437222in}}{\pgfqpoint{6.275590in}{5.159444in}}%
\pgfusepath{clip}%
\pgfsetbuttcap%
\pgfsetroundjoin%
\pgfsetlinewidth{1.003750pt}%
\definecolor{currentstroke}{rgb}{0.827451,0.827451,0.827451}%
\pgfsetstrokecolor{currentstroke}%
\pgfsetstrokeopacity{0.800000}%
\pgfsetdash{}{0pt}%
\pgfpathmoveto{\pgfqpoint{2.516236in}{1.449604in}}%
\pgfpathcurveto{\pgfqpoint{2.527286in}{1.449604in}}{\pgfqpoint{2.537885in}{1.453994in}}{\pgfqpoint{2.545699in}{1.461808in}}%
\pgfpathcurveto{\pgfqpoint{2.553513in}{1.469621in}}{\pgfqpoint{2.557903in}{1.480220in}}{\pgfqpoint{2.557903in}{1.491270in}}%
\pgfpathcurveto{\pgfqpoint{2.557903in}{1.502321in}}{\pgfqpoint{2.553513in}{1.512920in}}{\pgfqpoint{2.545699in}{1.520733in}}%
\pgfpathcurveto{\pgfqpoint{2.537885in}{1.528547in}}{\pgfqpoint{2.527286in}{1.532937in}}{\pgfqpoint{2.516236in}{1.532937in}}%
\pgfpathcurveto{\pgfqpoint{2.505186in}{1.532937in}}{\pgfqpoint{2.494587in}{1.528547in}}{\pgfqpoint{2.486773in}{1.520733in}}%
\pgfpathcurveto{\pgfqpoint{2.478960in}{1.512920in}}{\pgfqpoint{2.474569in}{1.502321in}}{\pgfqpoint{2.474569in}{1.491270in}}%
\pgfpathcurveto{\pgfqpoint{2.474569in}{1.480220in}}{\pgfqpoint{2.478960in}{1.469621in}}{\pgfqpoint{2.486773in}{1.461808in}}%
\pgfpathcurveto{\pgfqpoint{2.494587in}{1.453994in}}{\pgfqpoint{2.505186in}{1.449604in}}{\pgfqpoint{2.516236in}{1.449604in}}%
\pgfpathlineto{\pgfqpoint{2.516236in}{1.449604in}}%
\pgfpathclose%
\pgfusepath{stroke}%
\end{pgfscope}%
\begin{pgfscope}%
\pgfpathrectangle{\pgfqpoint{0.494722in}{0.437222in}}{\pgfqpoint{6.275590in}{5.159444in}}%
\pgfusepath{clip}%
\pgfsetbuttcap%
\pgfsetroundjoin%
\pgfsetlinewidth{1.003750pt}%
\definecolor{currentstroke}{rgb}{0.827451,0.827451,0.827451}%
\pgfsetstrokecolor{currentstroke}%
\pgfsetstrokeopacity{0.800000}%
\pgfsetdash{}{0pt}%
\pgfpathmoveto{\pgfqpoint{2.858741in}{1.109438in}}%
\pgfpathcurveto{\pgfqpoint{2.869791in}{1.109438in}}{\pgfqpoint{2.880390in}{1.113829in}}{\pgfqpoint{2.888204in}{1.121642in}}%
\pgfpathcurveto{\pgfqpoint{2.896018in}{1.129456in}}{\pgfqpoint{2.900408in}{1.140055in}}{\pgfqpoint{2.900408in}{1.151105in}}%
\pgfpathcurveto{\pgfqpoint{2.900408in}{1.162155in}}{\pgfqpoint{2.896018in}{1.172754in}}{\pgfqpoint{2.888204in}{1.180568in}}%
\pgfpathcurveto{\pgfqpoint{2.880390in}{1.188381in}}{\pgfqpoint{2.869791in}{1.192772in}}{\pgfqpoint{2.858741in}{1.192772in}}%
\pgfpathcurveto{\pgfqpoint{2.847691in}{1.192772in}}{\pgfqpoint{2.837092in}{1.188381in}}{\pgfqpoint{2.829278in}{1.180568in}}%
\pgfpathcurveto{\pgfqpoint{2.821465in}{1.172754in}}{\pgfqpoint{2.817075in}{1.162155in}}{\pgfqpoint{2.817075in}{1.151105in}}%
\pgfpathcurveto{\pgfqpoint{2.817075in}{1.140055in}}{\pgfqpoint{2.821465in}{1.129456in}}{\pgfqpoint{2.829278in}{1.121642in}}%
\pgfpathcurveto{\pgfqpoint{2.837092in}{1.113829in}}{\pgfqpoint{2.847691in}{1.109438in}}{\pgfqpoint{2.858741in}{1.109438in}}%
\pgfpathlineto{\pgfqpoint{2.858741in}{1.109438in}}%
\pgfpathclose%
\pgfusepath{stroke}%
\end{pgfscope}%
\begin{pgfscope}%
\pgfpathrectangle{\pgfqpoint{0.494722in}{0.437222in}}{\pgfqpoint{6.275590in}{5.159444in}}%
\pgfusepath{clip}%
\pgfsetbuttcap%
\pgfsetroundjoin%
\pgfsetlinewidth{1.003750pt}%
\definecolor{currentstroke}{rgb}{0.827451,0.827451,0.827451}%
\pgfsetstrokecolor{currentstroke}%
\pgfsetstrokeopacity{0.800000}%
\pgfsetdash{}{0pt}%
\pgfpathmoveto{\pgfqpoint{1.976414in}{1.606802in}}%
\pgfpathcurveto{\pgfqpoint{1.987464in}{1.606802in}}{\pgfqpoint{1.998063in}{1.611193in}}{\pgfqpoint{2.005877in}{1.619006in}}%
\pgfpathcurveto{\pgfqpoint{2.013691in}{1.626820in}}{\pgfqpoint{2.018081in}{1.637419in}}{\pgfqpoint{2.018081in}{1.648469in}}%
\pgfpathcurveto{\pgfqpoint{2.018081in}{1.659519in}}{\pgfqpoint{2.013691in}{1.670118in}}{\pgfqpoint{2.005877in}{1.677932in}}%
\pgfpathcurveto{\pgfqpoint{1.998063in}{1.685745in}}{\pgfqpoint{1.987464in}{1.690136in}}{\pgfqpoint{1.976414in}{1.690136in}}%
\pgfpathcurveto{\pgfqpoint{1.965364in}{1.690136in}}{\pgfqpoint{1.954765in}{1.685745in}}{\pgfqpoint{1.946952in}{1.677932in}}%
\pgfpathcurveto{\pgfqpoint{1.939138in}{1.670118in}}{\pgfqpoint{1.934748in}{1.659519in}}{\pgfqpoint{1.934748in}{1.648469in}}%
\pgfpathcurveto{\pgfqpoint{1.934748in}{1.637419in}}{\pgfqpoint{1.939138in}{1.626820in}}{\pgfqpoint{1.946952in}{1.619006in}}%
\pgfpathcurveto{\pgfqpoint{1.954765in}{1.611193in}}{\pgfqpoint{1.965364in}{1.606802in}}{\pgfqpoint{1.976414in}{1.606802in}}%
\pgfpathlineto{\pgfqpoint{1.976414in}{1.606802in}}%
\pgfpathclose%
\pgfusepath{stroke}%
\end{pgfscope}%
\begin{pgfscope}%
\pgfpathrectangle{\pgfqpoint{0.494722in}{0.437222in}}{\pgfqpoint{6.275590in}{5.159444in}}%
\pgfusepath{clip}%
\pgfsetbuttcap%
\pgfsetroundjoin%
\pgfsetlinewidth{1.003750pt}%
\definecolor{currentstroke}{rgb}{0.827451,0.827451,0.827451}%
\pgfsetstrokecolor{currentstroke}%
\pgfsetstrokeopacity{0.800000}%
\pgfsetdash{}{0pt}%
\pgfpathmoveto{\pgfqpoint{1.951363in}{1.620394in}}%
\pgfpathcurveto{\pgfqpoint{1.962413in}{1.620394in}}{\pgfqpoint{1.973012in}{1.624784in}}{\pgfqpoint{1.980826in}{1.632598in}}%
\pgfpathcurveto{\pgfqpoint{1.988640in}{1.640411in}}{\pgfqpoint{1.993030in}{1.651010in}}{\pgfqpoint{1.993030in}{1.662060in}}%
\pgfpathcurveto{\pgfqpoint{1.993030in}{1.673111in}}{\pgfqpoint{1.988640in}{1.683710in}}{\pgfqpoint{1.980826in}{1.691523in}}%
\pgfpathcurveto{\pgfqpoint{1.973012in}{1.699337in}}{\pgfqpoint{1.962413in}{1.703727in}}{\pgfqpoint{1.951363in}{1.703727in}}%
\pgfpathcurveto{\pgfqpoint{1.940313in}{1.703727in}}{\pgfqpoint{1.929714in}{1.699337in}}{\pgfqpoint{1.921900in}{1.691523in}}%
\pgfpathcurveto{\pgfqpoint{1.914087in}{1.683710in}}{\pgfqpoint{1.909696in}{1.673111in}}{\pgfqpoint{1.909696in}{1.662060in}}%
\pgfpathcurveto{\pgfqpoint{1.909696in}{1.651010in}}{\pgfqpoint{1.914087in}{1.640411in}}{\pgfqpoint{1.921900in}{1.632598in}}%
\pgfpathcurveto{\pgfqpoint{1.929714in}{1.624784in}}{\pgfqpoint{1.940313in}{1.620394in}}{\pgfqpoint{1.951363in}{1.620394in}}%
\pgfpathlineto{\pgfqpoint{1.951363in}{1.620394in}}%
\pgfpathclose%
\pgfusepath{stroke}%
\end{pgfscope}%
\begin{pgfscope}%
\pgfpathrectangle{\pgfqpoint{0.494722in}{0.437222in}}{\pgfqpoint{6.275590in}{5.159444in}}%
\pgfusepath{clip}%
\pgfsetbuttcap%
\pgfsetroundjoin%
\pgfsetlinewidth{1.003750pt}%
\definecolor{currentstroke}{rgb}{0.827451,0.827451,0.827451}%
\pgfsetstrokecolor{currentstroke}%
\pgfsetstrokeopacity{0.800000}%
\pgfsetdash{}{0pt}%
\pgfpathmoveto{\pgfqpoint{4.282315in}{0.548168in}}%
\pgfpathcurveto{\pgfqpoint{4.293365in}{0.548168in}}{\pgfqpoint{4.303964in}{0.552559in}}{\pgfqpoint{4.311778in}{0.560372in}}%
\pgfpathcurveto{\pgfqpoint{4.319591in}{0.568186in}}{\pgfqpoint{4.323982in}{0.578785in}}{\pgfqpoint{4.323982in}{0.589835in}}%
\pgfpathcurveto{\pgfqpoint{4.323982in}{0.600885in}}{\pgfqpoint{4.319591in}{0.611484in}}{\pgfqpoint{4.311778in}{0.619298in}}%
\pgfpathcurveto{\pgfqpoint{4.303964in}{0.627111in}}{\pgfqpoint{4.293365in}{0.631502in}}{\pgfqpoint{4.282315in}{0.631502in}}%
\pgfpathcurveto{\pgfqpoint{4.271265in}{0.631502in}}{\pgfqpoint{4.260666in}{0.627111in}}{\pgfqpoint{4.252852in}{0.619298in}}%
\pgfpathcurveto{\pgfqpoint{4.245039in}{0.611484in}}{\pgfqpoint{4.240648in}{0.600885in}}{\pgfqpoint{4.240648in}{0.589835in}}%
\pgfpathcurveto{\pgfqpoint{4.240648in}{0.578785in}}{\pgfqpoint{4.245039in}{0.568186in}}{\pgfqpoint{4.252852in}{0.560372in}}%
\pgfpathcurveto{\pgfqpoint{4.260666in}{0.552559in}}{\pgfqpoint{4.271265in}{0.548168in}}{\pgfqpoint{4.282315in}{0.548168in}}%
\pgfpathlineto{\pgfqpoint{4.282315in}{0.548168in}}%
\pgfpathclose%
\pgfusepath{stroke}%
\end{pgfscope}%
\begin{pgfscope}%
\pgfpathrectangle{\pgfqpoint{0.494722in}{0.437222in}}{\pgfqpoint{6.275590in}{5.159444in}}%
\pgfusepath{clip}%
\pgfsetbuttcap%
\pgfsetroundjoin%
\pgfsetlinewidth{1.003750pt}%
\definecolor{currentstroke}{rgb}{0.827451,0.827451,0.827451}%
\pgfsetstrokecolor{currentstroke}%
\pgfsetstrokeopacity{0.800000}%
\pgfsetdash{}{0pt}%
\pgfpathmoveto{\pgfqpoint{5.169286in}{0.532515in}}%
\pgfpathcurveto{\pgfqpoint{5.180336in}{0.532515in}}{\pgfqpoint{5.190935in}{0.536906in}}{\pgfqpoint{5.198749in}{0.544719in}}%
\pgfpathcurveto{\pgfqpoint{5.206563in}{0.552533in}}{\pgfqpoint{5.210953in}{0.563132in}}{\pgfqpoint{5.210953in}{0.574182in}}%
\pgfpathcurveto{\pgfqpoint{5.210953in}{0.585232in}}{\pgfqpoint{5.206563in}{0.595831in}}{\pgfqpoint{5.198749in}{0.603645in}}%
\pgfpathcurveto{\pgfqpoint{5.190935in}{0.611458in}}{\pgfqpoint{5.180336in}{0.615849in}}{\pgfqpoint{5.169286in}{0.615849in}}%
\pgfpathcurveto{\pgfqpoint{5.158236in}{0.615849in}}{\pgfqpoint{5.147637in}{0.611458in}}{\pgfqpoint{5.139823in}{0.603645in}}%
\pgfpathcurveto{\pgfqpoint{5.132010in}{0.595831in}}{\pgfqpoint{5.127619in}{0.585232in}}{\pgfqpoint{5.127619in}{0.574182in}}%
\pgfpathcurveto{\pgfqpoint{5.127619in}{0.563132in}}{\pgfqpoint{5.132010in}{0.552533in}}{\pgfqpoint{5.139823in}{0.544719in}}%
\pgfpathcurveto{\pgfqpoint{5.147637in}{0.536906in}}{\pgfqpoint{5.158236in}{0.532515in}}{\pgfqpoint{5.169286in}{0.532515in}}%
\pgfpathlineto{\pgfqpoint{5.169286in}{0.532515in}}%
\pgfpathclose%
\pgfusepath{stroke}%
\end{pgfscope}%
\begin{pgfscope}%
\pgfpathrectangle{\pgfqpoint{0.494722in}{0.437222in}}{\pgfqpoint{6.275590in}{5.159444in}}%
\pgfusepath{clip}%
\pgfsetbuttcap%
\pgfsetroundjoin%
\pgfsetlinewidth{1.003750pt}%
\definecolor{currentstroke}{rgb}{0.827451,0.827451,0.827451}%
\pgfsetstrokecolor{currentstroke}%
\pgfsetstrokeopacity{0.800000}%
\pgfsetdash{}{0pt}%
\pgfpathmoveto{\pgfqpoint{5.176884in}{0.528635in}}%
\pgfpathcurveto{\pgfqpoint{5.187934in}{0.528635in}}{\pgfqpoint{5.198533in}{0.533026in}}{\pgfqpoint{5.206347in}{0.540839in}}%
\pgfpathcurveto{\pgfqpoint{5.214160in}{0.548653in}}{\pgfqpoint{5.218551in}{0.559252in}}{\pgfqpoint{5.218551in}{0.570302in}}%
\pgfpathcurveto{\pgfqpoint{5.218551in}{0.581352in}}{\pgfqpoint{5.214160in}{0.591951in}}{\pgfqpoint{5.206347in}{0.599765in}}%
\pgfpathcurveto{\pgfqpoint{5.198533in}{0.607579in}}{\pgfqpoint{5.187934in}{0.611969in}}{\pgfqpoint{5.176884in}{0.611969in}}%
\pgfpathcurveto{\pgfqpoint{5.165834in}{0.611969in}}{\pgfqpoint{5.155235in}{0.607579in}}{\pgfqpoint{5.147421in}{0.599765in}}%
\pgfpathcurveto{\pgfqpoint{5.139608in}{0.591951in}}{\pgfqpoint{5.135217in}{0.581352in}}{\pgfqpoint{5.135217in}{0.570302in}}%
\pgfpathcurveto{\pgfqpoint{5.135217in}{0.559252in}}{\pgfqpoint{5.139608in}{0.548653in}}{\pgfqpoint{5.147421in}{0.540839in}}%
\pgfpathcurveto{\pgfqpoint{5.155235in}{0.533026in}}{\pgfqpoint{5.165834in}{0.528635in}}{\pgfqpoint{5.176884in}{0.528635in}}%
\pgfpathlineto{\pgfqpoint{5.176884in}{0.528635in}}%
\pgfpathclose%
\pgfusepath{stroke}%
\end{pgfscope}%
\begin{pgfscope}%
\pgfpathrectangle{\pgfqpoint{0.494722in}{0.437222in}}{\pgfqpoint{6.275590in}{5.159444in}}%
\pgfusepath{clip}%
\pgfsetbuttcap%
\pgfsetroundjoin%
\pgfsetlinewidth{1.003750pt}%
\definecolor{currentstroke}{rgb}{0.827451,0.827451,0.827451}%
\pgfsetstrokecolor{currentstroke}%
\pgfsetstrokeopacity{0.800000}%
\pgfsetdash{}{0pt}%
\pgfpathmoveto{\pgfqpoint{4.266828in}{0.548175in}}%
\pgfpathcurveto{\pgfqpoint{4.277878in}{0.548175in}}{\pgfqpoint{4.288477in}{0.552565in}}{\pgfqpoint{4.296291in}{0.560379in}}%
\pgfpathcurveto{\pgfqpoint{4.304104in}{0.568193in}}{\pgfqpoint{4.308495in}{0.578792in}}{\pgfqpoint{4.308495in}{0.589842in}}%
\pgfpathcurveto{\pgfqpoint{4.308495in}{0.600892in}}{\pgfqpoint{4.304104in}{0.611491in}}{\pgfqpoint{4.296291in}{0.619304in}}%
\pgfpathcurveto{\pgfqpoint{4.288477in}{0.627118in}}{\pgfqpoint{4.277878in}{0.631508in}}{\pgfqpoint{4.266828in}{0.631508in}}%
\pgfpathcurveto{\pgfqpoint{4.255778in}{0.631508in}}{\pgfqpoint{4.245179in}{0.627118in}}{\pgfqpoint{4.237365in}{0.619304in}}%
\pgfpathcurveto{\pgfqpoint{4.229552in}{0.611491in}}{\pgfqpoint{4.225161in}{0.600892in}}{\pgfqpoint{4.225161in}{0.589842in}}%
\pgfpathcurveto{\pgfqpoint{4.225161in}{0.578792in}}{\pgfqpoint{4.229552in}{0.568193in}}{\pgfqpoint{4.237365in}{0.560379in}}%
\pgfpathcurveto{\pgfqpoint{4.245179in}{0.552565in}}{\pgfqpoint{4.255778in}{0.548175in}}{\pgfqpoint{4.266828in}{0.548175in}}%
\pgfpathlineto{\pgfqpoint{4.266828in}{0.548175in}}%
\pgfpathclose%
\pgfusepath{stroke}%
\end{pgfscope}%
\begin{pgfscope}%
\pgfpathrectangle{\pgfqpoint{0.494722in}{0.437222in}}{\pgfqpoint{6.275590in}{5.159444in}}%
\pgfusepath{clip}%
\pgfsetbuttcap%
\pgfsetroundjoin%
\pgfsetlinewidth{1.003750pt}%
\definecolor{currentstroke}{rgb}{0.827451,0.827451,0.827451}%
\pgfsetstrokecolor{currentstroke}%
\pgfsetstrokeopacity{0.800000}%
\pgfsetdash{}{0pt}%
\pgfpathmoveto{\pgfqpoint{2.652266in}{1.295677in}}%
\pgfpathcurveto{\pgfqpoint{2.663316in}{1.295677in}}{\pgfqpoint{2.673915in}{1.300067in}}{\pgfqpoint{2.681729in}{1.307881in}}%
\pgfpathcurveto{\pgfqpoint{2.689543in}{1.315694in}}{\pgfqpoint{2.693933in}{1.326293in}}{\pgfqpoint{2.693933in}{1.337343in}}%
\pgfpathcurveto{\pgfqpoint{2.693933in}{1.348393in}}{\pgfqpoint{2.689543in}{1.358992in}}{\pgfqpoint{2.681729in}{1.366806in}}%
\pgfpathcurveto{\pgfqpoint{2.673915in}{1.374620in}}{\pgfqpoint{2.663316in}{1.379010in}}{\pgfqpoint{2.652266in}{1.379010in}}%
\pgfpathcurveto{\pgfqpoint{2.641216in}{1.379010in}}{\pgfqpoint{2.630617in}{1.374620in}}{\pgfqpoint{2.622803in}{1.366806in}}%
\pgfpathcurveto{\pgfqpoint{2.614990in}{1.358992in}}{\pgfqpoint{2.610600in}{1.348393in}}{\pgfqpoint{2.610600in}{1.337343in}}%
\pgfpathcurveto{\pgfqpoint{2.610600in}{1.326293in}}{\pgfqpoint{2.614990in}{1.315694in}}{\pgfqpoint{2.622803in}{1.307881in}}%
\pgfpathcurveto{\pgfqpoint{2.630617in}{1.300067in}}{\pgfqpoint{2.641216in}{1.295677in}}{\pgfqpoint{2.652266in}{1.295677in}}%
\pgfpathlineto{\pgfqpoint{2.652266in}{1.295677in}}%
\pgfpathclose%
\pgfusepath{stroke}%
\end{pgfscope}%
\begin{pgfscope}%
\pgfpathrectangle{\pgfqpoint{0.494722in}{0.437222in}}{\pgfqpoint{6.275590in}{5.159444in}}%
\pgfusepath{clip}%
\pgfsetbuttcap%
\pgfsetroundjoin%
\pgfsetlinewidth{1.003750pt}%
\definecolor{currentstroke}{rgb}{0.827451,0.827451,0.827451}%
\pgfsetstrokecolor{currentstroke}%
\pgfsetstrokeopacity{0.800000}%
\pgfsetdash{}{0pt}%
\pgfpathmoveto{\pgfqpoint{2.738066in}{1.171850in}}%
\pgfpathcurveto{\pgfqpoint{2.749116in}{1.171850in}}{\pgfqpoint{2.759715in}{1.176240in}}{\pgfqpoint{2.767528in}{1.184053in}}%
\pgfpathcurveto{\pgfqpoint{2.775342in}{1.191867in}}{\pgfqpoint{2.779732in}{1.202466in}}{\pgfqpoint{2.779732in}{1.213516in}}%
\pgfpathcurveto{\pgfqpoint{2.779732in}{1.224566in}}{\pgfqpoint{2.775342in}{1.235165in}}{\pgfqpoint{2.767528in}{1.242979in}}%
\pgfpathcurveto{\pgfqpoint{2.759715in}{1.250793in}}{\pgfqpoint{2.749116in}{1.255183in}}{\pgfqpoint{2.738066in}{1.255183in}}%
\pgfpathcurveto{\pgfqpoint{2.727016in}{1.255183in}}{\pgfqpoint{2.716417in}{1.250793in}}{\pgfqpoint{2.708603in}{1.242979in}}%
\pgfpathcurveto{\pgfqpoint{2.700789in}{1.235165in}}{\pgfqpoint{2.696399in}{1.224566in}}{\pgfqpoint{2.696399in}{1.213516in}}%
\pgfpathcurveto{\pgfqpoint{2.696399in}{1.202466in}}{\pgfqpoint{2.700789in}{1.191867in}}{\pgfqpoint{2.708603in}{1.184053in}}%
\pgfpathcurveto{\pgfqpoint{2.716417in}{1.176240in}}{\pgfqpoint{2.727016in}{1.171850in}}{\pgfqpoint{2.738066in}{1.171850in}}%
\pgfpathlineto{\pgfqpoint{2.738066in}{1.171850in}}%
\pgfpathclose%
\pgfusepath{stroke}%
\end{pgfscope}%
\begin{pgfscope}%
\pgfpathrectangle{\pgfqpoint{0.494722in}{0.437222in}}{\pgfqpoint{6.275590in}{5.159444in}}%
\pgfusepath{clip}%
\pgfsetbuttcap%
\pgfsetroundjoin%
\pgfsetlinewidth{1.003750pt}%
\definecolor{currentstroke}{rgb}{0.827451,0.827451,0.827451}%
\pgfsetstrokecolor{currentstroke}%
\pgfsetstrokeopacity{0.800000}%
\pgfsetdash{}{0pt}%
\pgfpathmoveto{\pgfqpoint{1.072122in}{3.062639in}}%
\pgfpathcurveto{\pgfqpoint{1.083172in}{3.062639in}}{\pgfqpoint{1.093771in}{3.067029in}}{\pgfqpoint{1.101584in}{3.074842in}}%
\pgfpathcurveto{\pgfqpoint{1.109398in}{3.082656in}}{\pgfqpoint{1.113788in}{3.093255in}}{\pgfqpoint{1.113788in}{3.104305in}}%
\pgfpathcurveto{\pgfqpoint{1.113788in}{3.115355in}}{\pgfqpoint{1.109398in}{3.125954in}}{\pgfqpoint{1.101584in}{3.133768in}}%
\pgfpathcurveto{\pgfqpoint{1.093771in}{3.141582in}}{\pgfqpoint{1.083172in}{3.145972in}}{\pgfqpoint{1.072122in}{3.145972in}}%
\pgfpathcurveto{\pgfqpoint{1.061071in}{3.145972in}}{\pgfqpoint{1.050472in}{3.141582in}}{\pgfqpoint{1.042659in}{3.133768in}}%
\pgfpathcurveto{\pgfqpoint{1.034845in}{3.125954in}}{\pgfqpoint{1.030455in}{3.115355in}}{\pgfqpoint{1.030455in}{3.104305in}}%
\pgfpathcurveto{\pgfqpoint{1.030455in}{3.093255in}}{\pgfqpoint{1.034845in}{3.082656in}}{\pgfqpoint{1.042659in}{3.074842in}}%
\pgfpathcurveto{\pgfqpoint{1.050472in}{3.067029in}}{\pgfqpoint{1.061071in}{3.062639in}}{\pgfqpoint{1.072122in}{3.062639in}}%
\pgfpathlineto{\pgfqpoint{1.072122in}{3.062639in}}%
\pgfpathclose%
\pgfusepath{stroke}%
\end{pgfscope}%
\begin{pgfscope}%
\pgfpathrectangle{\pgfqpoint{0.494722in}{0.437222in}}{\pgfqpoint{6.275590in}{5.159444in}}%
\pgfusepath{clip}%
\pgfsetbuttcap%
\pgfsetroundjoin%
\pgfsetlinewidth{1.003750pt}%
\definecolor{currentstroke}{rgb}{0.827451,0.827451,0.827451}%
\pgfsetstrokecolor{currentstroke}%
\pgfsetstrokeopacity{0.800000}%
\pgfsetdash{}{0pt}%
\pgfpathmoveto{\pgfqpoint{5.178838in}{0.549524in}}%
\pgfpathcurveto{\pgfqpoint{5.189888in}{0.549524in}}{\pgfqpoint{5.200487in}{0.553914in}}{\pgfqpoint{5.208301in}{0.561727in}}%
\pgfpathcurveto{\pgfqpoint{5.216114in}{0.569541in}}{\pgfqpoint{5.220505in}{0.580140in}}{\pgfqpoint{5.220505in}{0.591190in}}%
\pgfpathcurveto{\pgfqpoint{5.220505in}{0.602240in}}{\pgfqpoint{5.216114in}{0.612839in}}{\pgfqpoint{5.208301in}{0.620653in}}%
\pgfpathcurveto{\pgfqpoint{5.200487in}{0.628467in}}{\pgfqpoint{5.189888in}{0.632857in}}{\pgfqpoint{5.178838in}{0.632857in}}%
\pgfpathcurveto{\pgfqpoint{5.167788in}{0.632857in}}{\pgfqpoint{5.157189in}{0.628467in}}{\pgfqpoint{5.149375in}{0.620653in}}%
\pgfpathcurveto{\pgfqpoint{5.141562in}{0.612839in}}{\pgfqpoint{5.137171in}{0.602240in}}{\pgfqpoint{5.137171in}{0.591190in}}%
\pgfpathcurveto{\pgfqpoint{5.137171in}{0.580140in}}{\pgfqpoint{5.141562in}{0.569541in}}{\pgfqpoint{5.149375in}{0.561727in}}%
\pgfpathcurveto{\pgfqpoint{5.157189in}{0.553914in}}{\pgfqpoint{5.167788in}{0.549524in}}{\pgfqpoint{5.178838in}{0.549524in}}%
\pgfpathlineto{\pgfqpoint{5.178838in}{0.549524in}}%
\pgfpathclose%
\pgfusepath{stroke}%
\end{pgfscope}%
\begin{pgfscope}%
\pgfpathrectangle{\pgfqpoint{0.494722in}{0.437222in}}{\pgfqpoint{6.275590in}{5.159444in}}%
\pgfusepath{clip}%
\pgfsetbuttcap%
\pgfsetroundjoin%
\pgfsetlinewidth{1.003750pt}%
\definecolor{currentstroke}{rgb}{0.827451,0.827451,0.827451}%
\pgfsetstrokecolor{currentstroke}%
\pgfsetstrokeopacity{0.800000}%
\pgfsetdash{}{0pt}%
\pgfpathmoveto{\pgfqpoint{3.869729in}{0.718148in}}%
\pgfpathcurveto{\pgfqpoint{3.880779in}{0.718148in}}{\pgfqpoint{3.891378in}{0.722538in}}{\pgfqpoint{3.899192in}{0.730352in}}%
\pgfpathcurveto{\pgfqpoint{3.907005in}{0.738165in}}{\pgfqpoint{3.911396in}{0.748764in}}{\pgfqpoint{3.911396in}{0.759814in}}%
\pgfpathcurveto{\pgfqpoint{3.911396in}{0.770865in}}{\pgfqpoint{3.907005in}{0.781464in}}{\pgfqpoint{3.899192in}{0.789277in}}%
\pgfpathcurveto{\pgfqpoint{3.891378in}{0.797091in}}{\pgfqpoint{3.880779in}{0.801481in}}{\pgfqpoint{3.869729in}{0.801481in}}%
\pgfpathcurveto{\pgfqpoint{3.858679in}{0.801481in}}{\pgfqpoint{3.848080in}{0.797091in}}{\pgfqpoint{3.840266in}{0.789277in}}%
\pgfpathcurveto{\pgfqpoint{3.832452in}{0.781464in}}{\pgfqpoint{3.828062in}{0.770865in}}{\pgfqpoint{3.828062in}{0.759814in}}%
\pgfpathcurveto{\pgfqpoint{3.828062in}{0.748764in}}{\pgfqpoint{3.832452in}{0.738165in}}{\pgfqpoint{3.840266in}{0.730352in}}%
\pgfpathcurveto{\pgfqpoint{3.848080in}{0.722538in}}{\pgfqpoint{3.858679in}{0.718148in}}{\pgfqpoint{3.869729in}{0.718148in}}%
\pgfpathlineto{\pgfqpoint{3.869729in}{0.718148in}}%
\pgfpathclose%
\pgfusepath{stroke}%
\end{pgfscope}%
\begin{pgfscope}%
\pgfpathrectangle{\pgfqpoint{0.494722in}{0.437222in}}{\pgfqpoint{6.275590in}{5.159444in}}%
\pgfusepath{clip}%
\pgfsetbuttcap%
\pgfsetroundjoin%
\pgfsetlinewidth{1.003750pt}%
\definecolor{currentstroke}{rgb}{0.827451,0.827451,0.827451}%
\pgfsetstrokecolor{currentstroke}%
\pgfsetstrokeopacity{0.800000}%
\pgfsetdash{}{0pt}%
\pgfpathmoveto{\pgfqpoint{2.117156in}{1.399260in}}%
\pgfpathcurveto{\pgfqpoint{2.128206in}{1.399260in}}{\pgfqpoint{2.138805in}{1.403650in}}{\pgfqpoint{2.146618in}{1.411464in}}%
\pgfpathcurveto{\pgfqpoint{2.154432in}{1.419277in}}{\pgfqpoint{2.158822in}{1.429876in}}{\pgfqpoint{2.158822in}{1.440927in}}%
\pgfpathcurveto{\pgfqpoint{2.158822in}{1.451977in}}{\pgfqpoint{2.154432in}{1.462576in}}{\pgfqpoint{2.146618in}{1.470389in}}%
\pgfpathcurveto{\pgfqpoint{2.138805in}{1.478203in}}{\pgfqpoint{2.128206in}{1.482593in}}{\pgfqpoint{2.117156in}{1.482593in}}%
\pgfpathcurveto{\pgfqpoint{2.106106in}{1.482593in}}{\pgfqpoint{2.095506in}{1.478203in}}{\pgfqpoint{2.087693in}{1.470389in}}%
\pgfpathcurveto{\pgfqpoint{2.079879in}{1.462576in}}{\pgfqpoint{2.075489in}{1.451977in}}{\pgfqpoint{2.075489in}{1.440927in}}%
\pgfpathcurveto{\pgfqpoint{2.075489in}{1.429876in}}{\pgfqpoint{2.079879in}{1.419277in}}{\pgfqpoint{2.087693in}{1.411464in}}%
\pgfpathcurveto{\pgfqpoint{2.095506in}{1.403650in}}{\pgfqpoint{2.106106in}{1.399260in}}{\pgfqpoint{2.117156in}{1.399260in}}%
\pgfpathlineto{\pgfqpoint{2.117156in}{1.399260in}}%
\pgfpathclose%
\pgfusepath{stroke}%
\end{pgfscope}%
\begin{pgfscope}%
\pgfpathrectangle{\pgfqpoint{0.494722in}{0.437222in}}{\pgfqpoint{6.275590in}{5.159444in}}%
\pgfusepath{clip}%
\pgfsetbuttcap%
\pgfsetroundjoin%
\pgfsetlinewidth{1.003750pt}%
\definecolor{currentstroke}{rgb}{0.827451,0.827451,0.827451}%
\pgfsetstrokecolor{currentstroke}%
\pgfsetstrokeopacity{0.800000}%
\pgfsetdash{}{0pt}%
\pgfpathmoveto{\pgfqpoint{1.477105in}{1.953817in}}%
\pgfpathcurveto{\pgfqpoint{1.488155in}{1.953817in}}{\pgfqpoint{1.498754in}{1.958207in}}{\pgfqpoint{1.506568in}{1.966021in}}%
\pgfpathcurveto{\pgfqpoint{1.514381in}{1.973835in}}{\pgfqpoint{1.518772in}{1.984434in}}{\pgfqpoint{1.518772in}{1.995484in}}%
\pgfpathcurveto{\pgfqpoint{1.518772in}{2.006534in}}{\pgfqpoint{1.514381in}{2.017133in}}{\pgfqpoint{1.506568in}{2.024947in}}%
\pgfpathcurveto{\pgfqpoint{1.498754in}{2.032760in}}{\pgfqpoint{1.488155in}{2.037151in}}{\pgfqpoint{1.477105in}{2.037151in}}%
\pgfpathcurveto{\pgfqpoint{1.466055in}{2.037151in}}{\pgfqpoint{1.455456in}{2.032760in}}{\pgfqpoint{1.447642in}{2.024947in}}%
\pgfpathcurveto{\pgfqpoint{1.439828in}{2.017133in}}{\pgfqpoint{1.435438in}{2.006534in}}{\pgfqpoint{1.435438in}{1.995484in}}%
\pgfpathcurveto{\pgfqpoint{1.435438in}{1.984434in}}{\pgfqpoint{1.439828in}{1.973835in}}{\pgfqpoint{1.447642in}{1.966021in}}%
\pgfpathcurveto{\pgfqpoint{1.455456in}{1.958207in}}{\pgfqpoint{1.466055in}{1.953817in}}{\pgfqpoint{1.477105in}{1.953817in}}%
\pgfpathlineto{\pgfqpoint{1.477105in}{1.953817in}}%
\pgfpathclose%
\pgfusepath{stroke}%
\end{pgfscope}%
\begin{pgfscope}%
\pgfpathrectangle{\pgfqpoint{0.494722in}{0.437222in}}{\pgfqpoint{6.275590in}{5.159444in}}%
\pgfusepath{clip}%
\pgfsetbuttcap%
\pgfsetroundjoin%
\pgfsetlinewidth{1.003750pt}%
\definecolor{currentstroke}{rgb}{0.827451,0.827451,0.827451}%
\pgfsetstrokecolor{currentstroke}%
\pgfsetstrokeopacity{0.800000}%
\pgfsetdash{}{0pt}%
\pgfpathmoveto{\pgfqpoint{1.551682in}{1.872518in}}%
\pgfpathcurveto{\pgfqpoint{1.562732in}{1.872518in}}{\pgfqpoint{1.573331in}{1.876908in}}{\pgfqpoint{1.581144in}{1.884722in}}%
\pgfpathcurveto{\pgfqpoint{1.588958in}{1.892535in}}{\pgfqpoint{1.593348in}{1.903134in}}{\pgfqpoint{1.593348in}{1.914184in}}%
\pgfpathcurveto{\pgfqpoint{1.593348in}{1.925234in}}{\pgfqpoint{1.588958in}{1.935833in}}{\pgfqpoint{1.581144in}{1.943647in}}%
\pgfpathcurveto{\pgfqpoint{1.573331in}{1.951461in}}{\pgfqpoint{1.562732in}{1.955851in}}{\pgfqpoint{1.551682in}{1.955851in}}%
\pgfpathcurveto{\pgfqpoint{1.540632in}{1.955851in}}{\pgfqpoint{1.530033in}{1.951461in}}{\pgfqpoint{1.522219in}{1.943647in}}%
\pgfpathcurveto{\pgfqpoint{1.514405in}{1.935833in}}{\pgfqpoint{1.510015in}{1.925234in}}{\pgfqpoint{1.510015in}{1.914184in}}%
\pgfpathcurveto{\pgfqpoint{1.510015in}{1.903134in}}{\pgfqpoint{1.514405in}{1.892535in}}{\pgfqpoint{1.522219in}{1.884722in}}%
\pgfpathcurveto{\pgfqpoint{1.530033in}{1.876908in}}{\pgfqpoint{1.540632in}{1.872518in}}{\pgfqpoint{1.551682in}{1.872518in}}%
\pgfpathlineto{\pgfqpoint{1.551682in}{1.872518in}}%
\pgfpathclose%
\pgfusepath{stroke}%
\end{pgfscope}%
\begin{pgfscope}%
\pgfpathrectangle{\pgfqpoint{0.494722in}{0.437222in}}{\pgfqpoint{6.275590in}{5.159444in}}%
\pgfusepath{clip}%
\pgfsetbuttcap%
\pgfsetroundjoin%
\pgfsetlinewidth{1.003750pt}%
\definecolor{currentstroke}{rgb}{0.827451,0.827451,0.827451}%
\pgfsetstrokecolor{currentstroke}%
\pgfsetstrokeopacity{0.800000}%
\pgfsetdash{}{0pt}%
\pgfpathmoveto{\pgfqpoint{1.991938in}{1.503092in}}%
\pgfpathcurveto{\pgfqpoint{2.002988in}{1.503092in}}{\pgfqpoint{2.013587in}{1.507482in}}{\pgfqpoint{2.021400in}{1.515296in}}%
\pgfpathcurveto{\pgfqpoint{2.029214in}{1.523110in}}{\pgfqpoint{2.033604in}{1.533709in}}{\pgfqpoint{2.033604in}{1.544759in}}%
\pgfpathcurveto{\pgfqpoint{2.033604in}{1.555809in}}{\pgfqpoint{2.029214in}{1.566408in}}{\pgfqpoint{2.021400in}{1.574222in}}%
\pgfpathcurveto{\pgfqpoint{2.013587in}{1.582035in}}{\pgfqpoint{2.002988in}{1.586425in}}{\pgfqpoint{1.991938in}{1.586425in}}%
\pgfpathcurveto{\pgfqpoint{1.980887in}{1.586425in}}{\pgfqpoint{1.970288in}{1.582035in}}{\pgfqpoint{1.962475in}{1.574222in}}%
\pgfpathcurveto{\pgfqpoint{1.954661in}{1.566408in}}{\pgfqpoint{1.950271in}{1.555809in}}{\pgfqpoint{1.950271in}{1.544759in}}%
\pgfpathcurveto{\pgfqpoint{1.950271in}{1.533709in}}{\pgfqpoint{1.954661in}{1.523110in}}{\pgfqpoint{1.962475in}{1.515296in}}%
\pgfpathcurveto{\pgfqpoint{1.970288in}{1.507482in}}{\pgfqpoint{1.980887in}{1.503092in}}{\pgfqpoint{1.991938in}{1.503092in}}%
\pgfpathlineto{\pgfqpoint{1.991938in}{1.503092in}}%
\pgfpathclose%
\pgfusepath{stroke}%
\end{pgfscope}%
\begin{pgfscope}%
\pgfpathrectangle{\pgfqpoint{0.494722in}{0.437222in}}{\pgfqpoint{6.275590in}{5.159444in}}%
\pgfusepath{clip}%
\pgfsetbuttcap%
\pgfsetroundjoin%
\pgfsetlinewidth{1.003750pt}%
\definecolor{currentstroke}{rgb}{0.827451,0.827451,0.827451}%
\pgfsetstrokecolor{currentstroke}%
\pgfsetstrokeopacity{0.800000}%
\pgfsetdash{}{0pt}%
\pgfpathmoveto{\pgfqpoint{1.344163in}{2.102732in}}%
\pgfpathcurveto{\pgfqpoint{1.355213in}{2.102732in}}{\pgfqpoint{1.365812in}{2.107122in}}{\pgfqpoint{1.373626in}{2.114936in}}%
\pgfpathcurveto{\pgfqpoint{1.381439in}{2.122749in}}{\pgfqpoint{1.385830in}{2.133348in}}{\pgfqpoint{1.385830in}{2.144398in}}%
\pgfpathcurveto{\pgfqpoint{1.385830in}{2.155448in}}{\pgfqpoint{1.381439in}{2.166047in}}{\pgfqpoint{1.373626in}{2.173861in}}%
\pgfpathcurveto{\pgfqpoint{1.365812in}{2.181675in}}{\pgfqpoint{1.355213in}{2.186065in}}{\pgfqpoint{1.344163in}{2.186065in}}%
\pgfpathcurveto{\pgfqpoint{1.333113in}{2.186065in}}{\pgfqpoint{1.322514in}{2.181675in}}{\pgfqpoint{1.314700in}{2.173861in}}%
\pgfpathcurveto{\pgfqpoint{1.306887in}{2.166047in}}{\pgfqpoint{1.302496in}{2.155448in}}{\pgfqpoint{1.302496in}{2.144398in}}%
\pgfpathcurveto{\pgfqpoint{1.302496in}{2.133348in}}{\pgfqpoint{1.306887in}{2.122749in}}{\pgfqpoint{1.314700in}{2.114936in}}%
\pgfpathcurveto{\pgfqpoint{1.322514in}{2.107122in}}{\pgfqpoint{1.333113in}{2.102732in}}{\pgfqpoint{1.344163in}{2.102732in}}%
\pgfpathlineto{\pgfqpoint{1.344163in}{2.102732in}}%
\pgfpathclose%
\pgfusepath{stroke}%
\end{pgfscope}%
\begin{pgfscope}%
\pgfpathrectangle{\pgfqpoint{0.494722in}{0.437222in}}{\pgfqpoint{6.275590in}{5.159444in}}%
\pgfusepath{clip}%
\pgfsetbuttcap%
\pgfsetroundjoin%
\pgfsetlinewidth{1.003750pt}%
\definecolor{currentstroke}{rgb}{0.827451,0.827451,0.827451}%
\pgfsetstrokecolor{currentstroke}%
\pgfsetstrokeopacity{0.800000}%
\pgfsetdash{}{0pt}%
\pgfpathmoveto{\pgfqpoint{2.429002in}{1.219445in}}%
\pgfpathcurveto{\pgfqpoint{2.440052in}{1.219445in}}{\pgfqpoint{2.450651in}{1.223835in}}{\pgfqpoint{2.458465in}{1.231648in}}%
\pgfpathcurveto{\pgfqpoint{2.466278in}{1.239462in}}{\pgfqpoint{2.470669in}{1.250061in}}{\pgfqpoint{2.470669in}{1.261111in}}%
\pgfpathcurveto{\pgfqpoint{2.470669in}{1.272161in}}{\pgfqpoint{2.466278in}{1.282760in}}{\pgfqpoint{2.458465in}{1.290574in}}%
\pgfpathcurveto{\pgfqpoint{2.450651in}{1.298388in}}{\pgfqpoint{2.440052in}{1.302778in}}{\pgfqpoint{2.429002in}{1.302778in}}%
\pgfpathcurveto{\pgfqpoint{2.417952in}{1.302778in}}{\pgfqpoint{2.407353in}{1.298388in}}{\pgfqpoint{2.399539in}{1.290574in}}%
\pgfpathcurveto{\pgfqpoint{2.391726in}{1.282760in}}{\pgfqpoint{2.387335in}{1.272161in}}{\pgfqpoint{2.387335in}{1.261111in}}%
\pgfpathcurveto{\pgfqpoint{2.387335in}{1.250061in}}{\pgfqpoint{2.391726in}{1.239462in}}{\pgfqpoint{2.399539in}{1.231648in}}%
\pgfpathcurveto{\pgfqpoint{2.407353in}{1.223835in}}{\pgfqpoint{2.417952in}{1.219445in}}{\pgfqpoint{2.429002in}{1.219445in}}%
\pgfpathlineto{\pgfqpoint{2.429002in}{1.219445in}}%
\pgfpathclose%
\pgfusepath{stroke}%
\end{pgfscope}%
\begin{pgfscope}%
\pgfpathrectangle{\pgfqpoint{0.494722in}{0.437222in}}{\pgfqpoint{6.275590in}{5.159444in}}%
\pgfusepath{clip}%
\pgfsetbuttcap%
\pgfsetroundjoin%
\pgfsetlinewidth{1.003750pt}%
\definecolor{currentstroke}{rgb}{0.827451,0.827451,0.827451}%
\pgfsetstrokecolor{currentstroke}%
\pgfsetstrokeopacity{0.800000}%
\pgfsetdash{}{0pt}%
\pgfpathmoveto{\pgfqpoint{0.556707in}{3.885184in}}%
\pgfpathcurveto{\pgfqpoint{0.567757in}{3.885184in}}{\pgfqpoint{0.578356in}{3.889574in}}{\pgfqpoint{0.586170in}{3.897388in}}%
\pgfpathcurveto{\pgfqpoint{0.593983in}{3.905201in}}{\pgfqpoint{0.598374in}{3.915800in}}{\pgfqpoint{0.598374in}{3.926850in}}%
\pgfpathcurveto{\pgfqpoint{0.598374in}{3.937900in}}{\pgfqpoint{0.593983in}{3.948499in}}{\pgfqpoint{0.586170in}{3.956313in}}%
\pgfpathcurveto{\pgfqpoint{0.578356in}{3.964127in}}{\pgfqpoint{0.567757in}{3.968517in}}{\pgfqpoint{0.556707in}{3.968517in}}%
\pgfpathcurveto{\pgfqpoint{0.545657in}{3.968517in}}{\pgfqpoint{0.535058in}{3.964127in}}{\pgfqpoint{0.527244in}{3.956313in}}%
\pgfpathcurveto{\pgfqpoint{0.519430in}{3.948499in}}{\pgfqpoint{0.515040in}{3.937900in}}{\pgfqpoint{0.515040in}{3.926850in}}%
\pgfpathcurveto{\pgfqpoint{0.515040in}{3.915800in}}{\pgfqpoint{0.519430in}{3.905201in}}{\pgfqpoint{0.527244in}{3.897388in}}%
\pgfpathcurveto{\pgfqpoint{0.535058in}{3.889574in}}{\pgfqpoint{0.545657in}{3.885184in}}{\pgfqpoint{0.556707in}{3.885184in}}%
\pgfpathlineto{\pgfqpoint{0.556707in}{3.885184in}}%
\pgfpathclose%
\pgfusepath{stroke}%
\end{pgfscope}%
\begin{pgfscope}%
\pgfpathrectangle{\pgfqpoint{0.494722in}{0.437222in}}{\pgfqpoint{6.275590in}{5.159444in}}%
\pgfusepath{clip}%
\pgfsetbuttcap%
\pgfsetroundjoin%
\pgfsetlinewidth{1.003750pt}%
\definecolor{currentstroke}{rgb}{0.827451,0.827451,0.827451}%
\pgfsetstrokecolor{currentstroke}%
\pgfsetstrokeopacity{0.800000}%
\pgfsetdash{}{0pt}%
\pgfpathmoveto{\pgfqpoint{1.278200in}{2.184905in}}%
\pgfpathcurveto{\pgfqpoint{1.289250in}{2.184905in}}{\pgfqpoint{1.299849in}{2.189295in}}{\pgfqpoint{1.307663in}{2.197109in}}%
\pgfpathcurveto{\pgfqpoint{1.315477in}{2.204922in}}{\pgfqpoint{1.319867in}{2.215521in}}{\pgfqpoint{1.319867in}{2.226572in}}%
\pgfpathcurveto{\pgfqpoint{1.319867in}{2.237622in}}{\pgfqpoint{1.315477in}{2.248221in}}{\pgfqpoint{1.307663in}{2.256034in}}%
\pgfpathcurveto{\pgfqpoint{1.299849in}{2.263848in}}{\pgfqpoint{1.289250in}{2.268238in}}{\pgfqpoint{1.278200in}{2.268238in}}%
\pgfpathcurveto{\pgfqpoint{1.267150in}{2.268238in}}{\pgfqpoint{1.256551in}{2.263848in}}{\pgfqpoint{1.248738in}{2.256034in}}%
\pgfpathcurveto{\pgfqpoint{1.240924in}{2.248221in}}{\pgfqpoint{1.236534in}{2.237622in}}{\pgfqpoint{1.236534in}{2.226572in}}%
\pgfpathcurveto{\pgfqpoint{1.236534in}{2.215521in}}{\pgfqpoint{1.240924in}{2.204922in}}{\pgfqpoint{1.248738in}{2.197109in}}%
\pgfpathcurveto{\pgfqpoint{1.256551in}{2.189295in}}{\pgfqpoint{1.267150in}{2.184905in}}{\pgfqpoint{1.278200in}{2.184905in}}%
\pgfpathlineto{\pgfqpoint{1.278200in}{2.184905in}}%
\pgfpathclose%
\pgfusepath{stroke}%
\end{pgfscope}%
\begin{pgfscope}%
\pgfpathrectangle{\pgfqpoint{0.494722in}{0.437222in}}{\pgfqpoint{6.275590in}{5.159444in}}%
\pgfusepath{clip}%
\pgfsetbuttcap%
\pgfsetroundjoin%
\pgfsetlinewidth{1.003750pt}%
\definecolor{currentstroke}{rgb}{0.827451,0.827451,0.827451}%
\pgfsetstrokecolor{currentstroke}%
\pgfsetstrokeopacity{0.800000}%
\pgfsetdash{}{0pt}%
\pgfpathmoveto{\pgfqpoint{3.164762in}{0.836209in}}%
\pgfpathcurveto{\pgfqpoint{3.175812in}{0.836209in}}{\pgfqpoint{3.186411in}{0.840599in}}{\pgfqpoint{3.194224in}{0.848413in}}%
\pgfpathcurveto{\pgfqpoint{3.202038in}{0.856226in}}{\pgfqpoint{3.206428in}{0.866825in}}{\pgfqpoint{3.206428in}{0.877875in}}%
\pgfpathcurveto{\pgfqpoint{3.206428in}{0.888926in}}{\pgfqpoint{3.202038in}{0.899525in}}{\pgfqpoint{3.194224in}{0.907338in}}%
\pgfpathcurveto{\pgfqpoint{3.186411in}{0.915152in}}{\pgfqpoint{3.175812in}{0.919542in}}{\pgfqpoint{3.164762in}{0.919542in}}%
\pgfpathcurveto{\pgfqpoint{3.153711in}{0.919542in}}{\pgfqpoint{3.143112in}{0.915152in}}{\pgfqpoint{3.135299in}{0.907338in}}%
\pgfpathcurveto{\pgfqpoint{3.127485in}{0.899525in}}{\pgfqpoint{3.123095in}{0.888926in}}{\pgfqpoint{3.123095in}{0.877875in}}%
\pgfpathcurveto{\pgfqpoint{3.123095in}{0.866825in}}{\pgfqpoint{3.127485in}{0.856226in}}{\pgfqpoint{3.135299in}{0.848413in}}%
\pgfpathcurveto{\pgfqpoint{3.143112in}{0.840599in}}{\pgfqpoint{3.153711in}{0.836209in}}{\pgfqpoint{3.164762in}{0.836209in}}%
\pgfpathlineto{\pgfqpoint{3.164762in}{0.836209in}}%
\pgfpathclose%
\pgfusepath{stroke}%
\end{pgfscope}%
\begin{pgfscope}%
\pgfpathrectangle{\pgfqpoint{0.494722in}{0.437222in}}{\pgfqpoint{6.275590in}{5.159444in}}%
\pgfusepath{clip}%
\pgfsetbuttcap%
\pgfsetroundjoin%
\pgfsetlinewidth{1.003750pt}%
\definecolor{currentstroke}{rgb}{0.827451,0.827451,0.827451}%
\pgfsetstrokecolor{currentstroke}%
\pgfsetstrokeopacity{0.800000}%
\pgfsetdash{}{0pt}%
\pgfpathmoveto{\pgfqpoint{4.335578in}{0.505432in}}%
\pgfpathcurveto{\pgfqpoint{4.346628in}{0.505432in}}{\pgfqpoint{4.357227in}{0.509823in}}{\pgfqpoint{4.365041in}{0.517636in}}%
\pgfpathcurveto{\pgfqpoint{4.372855in}{0.525450in}}{\pgfqpoint{4.377245in}{0.536049in}}{\pgfqpoint{4.377245in}{0.547099in}}%
\pgfpathcurveto{\pgfqpoint{4.377245in}{0.558149in}}{\pgfqpoint{4.372855in}{0.568748in}}{\pgfqpoint{4.365041in}{0.576562in}}%
\pgfpathcurveto{\pgfqpoint{4.357227in}{0.584375in}}{\pgfqpoint{4.346628in}{0.588766in}}{\pgfqpoint{4.335578in}{0.588766in}}%
\pgfpathcurveto{\pgfqpoint{4.324528in}{0.588766in}}{\pgfqpoint{4.313929in}{0.584375in}}{\pgfqpoint{4.306115in}{0.576562in}}%
\pgfpathcurveto{\pgfqpoint{4.298302in}{0.568748in}}{\pgfqpoint{4.293911in}{0.558149in}}{\pgfqpoint{4.293911in}{0.547099in}}%
\pgfpathcurveto{\pgfqpoint{4.293911in}{0.536049in}}{\pgfqpoint{4.298302in}{0.525450in}}{\pgfqpoint{4.306115in}{0.517636in}}%
\pgfpathcurveto{\pgfqpoint{4.313929in}{0.509823in}}{\pgfqpoint{4.324528in}{0.505432in}}{\pgfqpoint{4.335578in}{0.505432in}}%
\pgfpathlineto{\pgfqpoint{4.335578in}{0.505432in}}%
\pgfpathclose%
\pgfusepath{stroke}%
\end{pgfscope}%
\begin{pgfscope}%
\pgfpathrectangle{\pgfqpoint{0.494722in}{0.437222in}}{\pgfqpoint{6.275590in}{5.159444in}}%
\pgfusepath{clip}%
\pgfsetbuttcap%
\pgfsetroundjoin%
\pgfsetlinewidth{1.003750pt}%
\definecolor{currentstroke}{rgb}{0.827451,0.827451,0.827451}%
\pgfsetstrokecolor{currentstroke}%
\pgfsetstrokeopacity{0.800000}%
\pgfsetdash{}{0pt}%
\pgfpathmoveto{\pgfqpoint{2.656452in}{1.108305in}}%
\pgfpathcurveto{\pgfqpoint{2.667502in}{1.108305in}}{\pgfqpoint{2.678102in}{1.112695in}}{\pgfqpoint{2.685915in}{1.120509in}}%
\pgfpathcurveto{\pgfqpoint{2.693729in}{1.128322in}}{\pgfqpoint{2.698119in}{1.138921in}}{\pgfqpoint{2.698119in}{1.149971in}}%
\pgfpathcurveto{\pgfqpoint{2.698119in}{1.161021in}}{\pgfqpoint{2.693729in}{1.171620in}}{\pgfqpoint{2.685915in}{1.179434in}}%
\pgfpathcurveto{\pgfqpoint{2.678102in}{1.187248in}}{\pgfqpoint{2.667502in}{1.191638in}}{\pgfqpoint{2.656452in}{1.191638in}}%
\pgfpathcurveto{\pgfqpoint{2.645402in}{1.191638in}}{\pgfqpoint{2.634803in}{1.187248in}}{\pgfqpoint{2.626990in}{1.179434in}}%
\pgfpathcurveto{\pgfqpoint{2.619176in}{1.171620in}}{\pgfqpoint{2.614786in}{1.161021in}}{\pgfqpoint{2.614786in}{1.149971in}}%
\pgfpathcurveto{\pgfqpoint{2.614786in}{1.138921in}}{\pgfqpoint{2.619176in}{1.128322in}}{\pgfqpoint{2.626990in}{1.120509in}}%
\pgfpathcurveto{\pgfqpoint{2.634803in}{1.112695in}}{\pgfqpoint{2.645402in}{1.108305in}}{\pgfqpoint{2.656452in}{1.108305in}}%
\pgfpathlineto{\pgfqpoint{2.656452in}{1.108305in}}%
\pgfpathclose%
\pgfusepath{stroke}%
\end{pgfscope}%
\begin{pgfscope}%
\pgfpathrectangle{\pgfqpoint{0.494722in}{0.437222in}}{\pgfqpoint{6.275590in}{5.159444in}}%
\pgfusepath{clip}%
\pgfsetbuttcap%
\pgfsetroundjoin%
\pgfsetlinewidth{1.003750pt}%
\definecolor{currentstroke}{rgb}{0.827451,0.827451,0.827451}%
\pgfsetstrokecolor{currentstroke}%
\pgfsetstrokeopacity{0.800000}%
\pgfsetdash{}{0pt}%
\pgfpathmoveto{\pgfqpoint{0.575099in}{3.779944in}}%
\pgfpathcurveto{\pgfqpoint{0.586149in}{3.779944in}}{\pgfqpoint{0.596748in}{3.784334in}}{\pgfqpoint{0.604561in}{3.792148in}}%
\pgfpathcurveto{\pgfqpoint{0.612375in}{3.799961in}}{\pgfqpoint{0.616765in}{3.810560in}}{\pgfqpoint{0.616765in}{3.821610in}}%
\pgfpathcurveto{\pgfqpoint{0.616765in}{3.832661in}}{\pgfqpoint{0.612375in}{3.843260in}}{\pgfqpoint{0.604561in}{3.851073in}}%
\pgfpathcurveto{\pgfqpoint{0.596748in}{3.858887in}}{\pgfqpoint{0.586149in}{3.863277in}}{\pgfqpoint{0.575099in}{3.863277in}}%
\pgfpathcurveto{\pgfqpoint{0.564048in}{3.863277in}}{\pgfqpoint{0.553449in}{3.858887in}}{\pgfqpoint{0.545636in}{3.851073in}}%
\pgfpathcurveto{\pgfqpoint{0.537822in}{3.843260in}}{\pgfqpoint{0.533432in}{3.832661in}}{\pgfqpoint{0.533432in}{3.821610in}}%
\pgfpathcurveto{\pgfqpoint{0.533432in}{3.810560in}}{\pgfqpoint{0.537822in}{3.799961in}}{\pgfqpoint{0.545636in}{3.792148in}}%
\pgfpathcurveto{\pgfqpoint{0.553449in}{3.784334in}}{\pgfqpoint{0.564048in}{3.779944in}}{\pgfqpoint{0.575099in}{3.779944in}}%
\pgfpathlineto{\pgfqpoint{0.575099in}{3.779944in}}%
\pgfpathclose%
\pgfusepath{stroke}%
\end{pgfscope}%
\begin{pgfscope}%
\pgfpathrectangle{\pgfqpoint{0.494722in}{0.437222in}}{\pgfqpoint{6.275590in}{5.159444in}}%
\pgfusepath{clip}%
\pgfsetbuttcap%
\pgfsetroundjoin%
\pgfsetlinewidth{1.003750pt}%
\definecolor{currentstroke}{rgb}{0.827451,0.827451,0.827451}%
\pgfsetstrokecolor{currentstroke}%
\pgfsetstrokeopacity{0.800000}%
\pgfsetdash{}{0pt}%
\pgfpathmoveto{\pgfqpoint{1.581391in}{1.870889in}}%
\pgfpathcurveto{\pgfqpoint{1.592441in}{1.870889in}}{\pgfqpoint{1.603040in}{1.875279in}}{\pgfqpoint{1.610854in}{1.883093in}}%
\pgfpathcurveto{\pgfqpoint{1.618667in}{1.890906in}}{\pgfqpoint{1.623058in}{1.901505in}}{\pgfqpoint{1.623058in}{1.912556in}}%
\pgfpathcurveto{\pgfqpoint{1.623058in}{1.923606in}}{\pgfqpoint{1.618667in}{1.934205in}}{\pgfqpoint{1.610854in}{1.942018in}}%
\pgfpathcurveto{\pgfqpoint{1.603040in}{1.949832in}}{\pgfqpoint{1.592441in}{1.954222in}}{\pgfqpoint{1.581391in}{1.954222in}}%
\pgfpathcurveto{\pgfqpoint{1.570341in}{1.954222in}}{\pgfqpoint{1.559742in}{1.949832in}}{\pgfqpoint{1.551928in}{1.942018in}}%
\pgfpathcurveto{\pgfqpoint{1.544115in}{1.934205in}}{\pgfqpoint{1.539724in}{1.923606in}}{\pgfqpoint{1.539724in}{1.912556in}}%
\pgfpathcurveto{\pgfqpoint{1.539724in}{1.901505in}}{\pgfqpoint{1.544115in}{1.890906in}}{\pgfqpoint{1.551928in}{1.883093in}}%
\pgfpathcurveto{\pgfqpoint{1.559742in}{1.875279in}}{\pgfqpoint{1.570341in}{1.870889in}}{\pgfqpoint{1.581391in}{1.870889in}}%
\pgfpathlineto{\pgfqpoint{1.581391in}{1.870889in}}%
\pgfpathclose%
\pgfusepath{stroke}%
\end{pgfscope}%
\begin{pgfscope}%
\pgfpathrectangle{\pgfqpoint{0.494722in}{0.437222in}}{\pgfqpoint{6.275590in}{5.159444in}}%
\pgfusepath{clip}%
\pgfsetbuttcap%
\pgfsetroundjoin%
\pgfsetlinewidth{1.003750pt}%
\definecolor{currentstroke}{rgb}{0.827451,0.827451,0.827451}%
\pgfsetstrokecolor{currentstroke}%
\pgfsetstrokeopacity{0.800000}%
\pgfsetdash{}{0pt}%
\pgfpathmoveto{\pgfqpoint{1.772500in}{1.679456in}}%
\pgfpathcurveto{\pgfqpoint{1.783550in}{1.679456in}}{\pgfqpoint{1.794149in}{1.683846in}}{\pgfqpoint{1.801962in}{1.691660in}}%
\pgfpathcurveto{\pgfqpoint{1.809776in}{1.699473in}}{\pgfqpoint{1.814166in}{1.710072in}}{\pgfqpoint{1.814166in}{1.721122in}}%
\pgfpathcurveto{\pgfqpoint{1.814166in}{1.732172in}}{\pgfqpoint{1.809776in}{1.742772in}}{\pgfqpoint{1.801962in}{1.750585in}}%
\pgfpathcurveto{\pgfqpoint{1.794149in}{1.758399in}}{\pgfqpoint{1.783550in}{1.762789in}}{\pgfqpoint{1.772500in}{1.762789in}}%
\pgfpathcurveto{\pgfqpoint{1.761450in}{1.762789in}}{\pgfqpoint{1.750850in}{1.758399in}}{\pgfqpoint{1.743037in}{1.750585in}}%
\pgfpathcurveto{\pgfqpoint{1.735223in}{1.742772in}}{\pgfqpoint{1.730833in}{1.732172in}}{\pgfqpoint{1.730833in}{1.721122in}}%
\pgfpathcurveto{\pgfqpoint{1.730833in}{1.710072in}}{\pgfqpoint{1.735223in}{1.699473in}}{\pgfqpoint{1.743037in}{1.691660in}}%
\pgfpathcurveto{\pgfqpoint{1.750850in}{1.683846in}}{\pgfqpoint{1.761450in}{1.679456in}}{\pgfqpoint{1.772500in}{1.679456in}}%
\pgfpathlineto{\pgfqpoint{1.772500in}{1.679456in}}%
\pgfpathclose%
\pgfusepath{stroke}%
\end{pgfscope}%
\begin{pgfscope}%
\pgfpathrectangle{\pgfqpoint{0.494722in}{0.437222in}}{\pgfqpoint{6.275590in}{5.159444in}}%
\pgfusepath{clip}%
\pgfsetbuttcap%
\pgfsetroundjoin%
\pgfsetlinewidth{1.003750pt}%
\definecolor{currentstroke}{rgb}{0.827451,0.827451,0.827451}%
\pgfsetstrokecolor{currentstroke}%
\pgfsetstrokeopacity{0.800000}%
\pgfsetdash{}{0pt}%
\pgfpathmoveto{\pgfqpoint{1.042231in}{2.681267in}}%
\pgfpathcurveto{\pgfqpoint{1.053281in}{2.681267in}}{\pgfqpoint{1.063880in}{2.685657in}}{\pgfqpoint{1.071694in}{2.693471in}}%
\pgfpathcurveto{\pgfqpoint{1.079507in}{2.701284in}}{\pgfqpoint{1.083898in}{2.711883in}}{\pgfqpoint{1.083898in}{2.722933in}}%
\pgfpathcurveto{\pgfqpoint{1.083898in}{2.733984in}}{\pgfqpoint{1.079507in}{2.744583in}}{\pgfqpoint{1.071694in}{2.752396in}}%
\pgfpathcurveto{\pgfqpoint{1.063880in}{2.760210in}}{\pgfqpoint{1.053281in}{2.764600in}}{\pgfqpoint{1.042231in}{2.764600in}}%
\pgfpathcurveto{\pgfqpoint{1.031181in}{2.764600in}}{\pgfqpoint{1.020582in}{2.760210in}}{\pgfqpoint{1.012768in}{2.752396in}}%
\pgfpathcurveto{\pgfqpoint{1.004954in}{2.744583in}}{\pgfqpoint{1.000564in}{2.733984in}}{\pgfqpoint{1.000564in}{2.722933in}}%
\pgfpathcurveto{\pgfqpoint{1.000564in}{2.711883in}}{\pgfqpoint{1.004954in}{2.701284in}}{\pgfqpoint{1.012768in}{2.693471in}}%
\pgfpathcurveto{\pgfqpoint{1.020582in}{2.685657in}}{\pgfqpoint{1.031181in}{2.681267in}}{\pgfqpoint{1.042231in}{2.681267in}}%
\pgfpathlineto{\pgfqpoint{1.042231in}{2.681267in}}%
\pgfpathclose%
\pgfusepath{stroke}%
\end{pgfscope}%
\begin{pgfscope}%
\pgfpathrectangle{\pgfqpoint{0.494722in}{0.437222in}}{\pgfqpoint{6.275590in}{5.159444in}}%
\pgfusepath{clip}%
\pgfsetbuttcap%
\pgfsetroundjoin%
\pgfsetlinewidth{1.003750pt}%
\definecolor{currentstroke}{rgb}{0.827451,0.827451,0.827451}%
\pgfsetstrokecolor{currentstroke}%
\pgfsetstrokeopacity{0.800000}%
\pgfsetdash{}{0pt}%
\pgfpathmoveto{\pgfqpoint{2.867283in}{0.963868in}}%
\pgfpathcurveto{\pgfqpoint{2.878333in}{0.963868in}}{\pgfqpoint{2.888932in}{0.968258in}}{\pgfqpoint{2.896746in}{0.976072in}}%
\pgfpathcurveto{\pgfqpoint{2.904560in}{0.983885in}}{\pgfqpoint{2.908950in}{0.994484in}}{\pgfqpoint{2.908950in}{1.005535in}}%
\pgfpathcurveto{\pgfqpoint{2.908950in}{1.016585in}}{\pgfqpoint{2.904560in}{1.027184in}}{\pgfqpoint{2.896746in}{1.034997in}}%
\pgfpathcurveto{\pgfqpoint{2.888932in}{1.042811in}}{\pgfqpoint{2.878333in}{1.047201in}}{\pgfqpoint{2.867283in}{1.047201in}}%
\pgfpathcurveto{\pgfqpoint{2.856233in}{1.047201in}}{\pgfqpoint{2.845634in}{1.042811in}}{\pgfqpoint{2.837820in}{1.034997in}}%
\pgfpathcurveto{\pgfqpoint{2.830007in}{1.027184in}}{\pgfqpoint{2.825617in}{1.016585in}}{\pgfqpoint{2.825617in}{1.005535in}}%
\pgfpathcurveto{\pgfqpoint{2.825617in}{0.994484in}}{\pgfqpoint{2.830007in}{0.983885in}}{\pgfqpoint{2.837820in}{0.976072in}}%
\pgfpathcurveto{\pgfqpoint{2.845634in}{0.968258in}}{\pgfqpoint{2.856233in}{0.963868in}}{\pgfqpoint{2.867283in}{0.963868in}}%
\pgfpathlineto{\pgfqpoint{2.867283in}{0.963868in}}%
\pgfpathclose%
\pgfusepath{stroke}%
\end{pgfscope}%
\begin{pgfscope}%
\pgfpathrectangle{\pgfqpoint{0.494722in}{0.437222in}}{\pgfqpoint{6.275590in}{5.159444in}}%
\pgfusepath{clip}%
\pgfsetbuttcap%
\pgfsetroundjoin%
\pgfsetlinewidth{1.003750pt}%
\definecolor{currentstroke}{rgb}{0.827451,0.827451,0.827451}%
\pgfsetstrokecolor{currentstroke}%
\pgfsetstrokeopacity{0.800000}%
\pgfsetdash{}{0pt}%
\pgfpathmoveto{\pgfqpoint{2.572850in}{1.116571in}}%
\pgfpathcurveto{\pgfqpoint{2.583900in}{1.116571in}}{\pgfqpoint{2.594499in}{1.120961in}}{\pgfqpoint{2.602313in}{1.128774in}}%
\pgfpathcurveto{\pgfqpoint{2.610126in}{1.136588in}}{\pgfqpoint{2.614517in}{1.147187in}}{\pgfqpoint{2.614517in}{1.158237in}}%
\pgfpathcurveto{\pgfqpoint{2.614517in}{1.169287in}}{\pgfqpoint{2.610126in}{1.179886in}}{\pgfqpoint{2.602313in}{1.187700in}}%
\pgfpathcurveto{\pgfqpoint{2.594499in}{1.195514in}}{\pgfqpoint{2.583900in}{1.199904in}}{\pgfqpoint{2.572850in}{1.199904in}}%
\pgfpathcurveto{\pgfqpoint{2.561800in}{1.199904in}}{\pgfqpoint{2.551201in}{1.195514in}}{\pgfqpoint{2.543387in}{1.187700in}}%
\pgfpathcurveto{\pgfqpoint{2.535574in}{1.179886in}}{\pgfqpoint{2.531183in}{1.169287in}}{\pgfqpoint{2.531183in}{1.158237in}}%
\pgfpathcurveto{\pgfqpoint{2.531183in}{1.147187in}}{\pgfqpoint{2.535574in}{1.136588in}}{\pgfqpoint{2.543387in}{1.128774in}}%
\pgfpathcurveto{\pgfqpoint{2.551201in}{1.120961in}}{\pgfqpoint{2.561800in}{1.116571in}}{\pgfqpoint{2.572850in}{1.116571in}}%
\pgfpathlineto{\pgfqpoint{2.572850in}{1.116571in}}%
\pgfpathclose%
\pgfusepath{stroke}%
\end{pgfscope}%
\begin{pgfscope}%
\pgfpathrectangle{\pgfqpoint{0.494722in}{0.437222in}}{\pgfqpoint{6.275590in}{5.159444in}}%
\pgfusepath{clip}%
\pgfsetbuttcap%
\pgfsetroundjoin%
\pgfsetlinewidth{1.003750pt}%
\definecolor{currentstroke}{rgb}{0.827451,0.827451,0.827451}%
\pgfsetstrokecolor{currentstroke}%
\pgfsetstrokeopacity{0.800000}%
\pgfsetdash{}{0pt}%
\pgfpathmoveto{\pgfqpoint{0.963748in}{2.873526in}}%
\pgfpathcurveto{\pgfqpoint{0.974798in}{2.873526in}}{\pgfqpoint{0.985398in}{2.877916in}}{\pgfqpoint{0.993211in}{2.885730in}}%
\pgfpathcurveto{\pgfqpoint{1.001025in}{2.893543in}}{\pgfqpoint{1.005415in}{2.904142in}}{\pgfqpoint{1.005415in}{2.915192in}}%
\pgfpathcurveto{\pgfqpoint{1.005415in}{2.926243in}}{\pgfqpoint{1.001025in}{2.936842in}}{\pgfqpoint{0.993211in}{2.944655in}}%
\pgfpathcurveto{\pgfqpoint{0.985398in}{2.952469in}}{\pgfqpoint{0.974798in}{2.956859in}}{\pgfqpoint{0.963748in}{2.956859in}}%
\pgfpathcurveto{\pgfqpoint{0.952698in}{2.956859in}}{\pgfqpoint{0.942099in}{2.952469in}}{\pgfqpoint{0.934286in}{2.944655in}}%
\pgfpathcurveto{\pgfqpoint{0.926472in}{2.936842in}}{\pgfqpoint{0.922082in}{2.926243in}}{\pgfqpoint{0.922082in}{2.915192in}}%
\pgfpathcurveto{\pgfqpoint{0.922082in}{2.904142in}}{\pgfqpoint{0.926472in}{2.893543in}}{\pgfqpoint{0.934286in}{2.885730in}}%
\pgfpathcurveto{\pgfqpoint{0.942099in}{2.877916in}}{\pgfqpoint{0.952698in}{2.873526in}}{\pgfqpoint{0.963748in}{2.873526in}}%
\pgfpathlineto{\pgfqpoint{0.963748in}{2.873526in}}%
\pgfpathclose%
\pgfusepath{stroke}%
\end{pgfscope}%
\begin{pgfscope}%
\pgfpathrectangle{\pgfqpoint{0.494722in}{0.437222in}}{\pgfqpoint{6.275590in}{5.159444in}}%
\pgfusepath{clip}%
\pgfsetbuttcap%
\pgfsetroundjoin%
\pgfsetlinewidth{1.003750pt}%
\definecolor{currentstroke}{rgb}{0.827451,0.827451,0.827451}%
\pgfsetstrokecolor{currentstroke}%
\pgfsetstrokeopacity{0.800000}%
\pgfsetdash{}{0pt}%
\pgfpathmoveto{\pgfqpoint{2.512961in}{1.165916in}}%
\pgfpathcurveto{\pgfqpoint{2.524011in}{1.165916in}}{\pgfqpoint{2.534610in}{1.170306in}}{\pgfqpoint{2.542424in}{1.178119in}}%
\pgfpathcurveto{\pgfqpoint{2.550237in}{1.185933in}}{\pgfqpoint{2.554627in}{1.196532in}}{\pgfqpoint{2.554627in}{1.207582in}}%
\pgfpathcurveto{\pgfqpoint{2.554627in}{1.218632in}}{\pgfqpoint{2.550237in}{1.229231in}}{\pgfqpoint{2.542424in}{1.237045in}}%
\pgfpathcurveto{\pgfqpoint{2.534610in}{1.244859in}}{\pgfqpoint{2.524011in}{1.249249in}}{\pgfqpoint{2.512961in}{1.249249in}}%
\pgfpathcurveto{\pgfqpoint{2.501911in}{1.249249in}}{\pgfqpoint{2.491312in}{1.244859in}}{\pgfqpoint{2.483498in}{1.237045in}}%
\pgfpathcurveto{\pgfqpoint{2.475684in}{1.229231in}}{\pgfqpoint{2.471294in}{1.218632in}}{\pgfqpoint{2.471294in}{1.207582in}}%
\pgfpathcurveto{\pgfqpoint{2.471294in}{1.196532in}}{\pgfqpoint{2.475684in}{1.185933in}}{\pgfqpoint{2.483498in}{1.178119in}}%
\pgfpathcurveto{\pgfqpoint{2.491312in}{1.170306in}}{\pgfqpoint{2.501911in}{1.165916in}}{\pgfqpoint{2.512961in}{1.165916in}}%
\pgfpathlineto{\pgfqpoint{2.512961in}{1.165916in}}%
\pgfpathclose%
\pgfusepath{stroke}%
\end{pgfscope}%
\begin{pgfscope}%
\pgfpathrectangle{\pgfqpoint{0.494722in}{0.437222in}}{\pgfqpoint{6.275590in}{5.159444in}}%
\pgfusepath{clip}%
\pgfsetbuttcap%
\pgfsetroundjoin%
\pgfsetlinewidth{1.003750pt}%
\definecolor{currentstroke}{rgb}{0.827451,0.827451,0.827451}%
\pgfsetstrokecolor{currentstroke}%
\pgfsetstrokeopacity{0.800000}%
\pgfsetdash{}{0pt}%
\pgfpathmoveto{\pgfqpoint{1.016820in}{2.832444in}}%
\pgfpathcurveto{\pgfqpoint{1.027870in}{2.832444in}}{\pgfqpoint{1.038469in}{2.836834in}}{\pgfqpoint{1.046283in}{2.844648in}}%
\pgfpathcurveto{\pgfqpoint{1.054096in}{2.852462in}}{\pgfqpoint{1.058487in}{2.863061in}}{\pgfqpoint{1.058487in}{2.874111in}}%
\pgfpathcurveto{\pgfqpoint{1.058487in}{2.885161in}}{\pgfqpoint{1.054096in}{2.895760in}}{\pgfqpoint{1.046283in}{2.903573in}}%
\pgfpathcurveto{\pgfqpoint{1.038469in}{2.911387in}}{\pgfqpoint{1.027870in}{2.915777in}}{\pgfqpoint{1.016820in}{2.915777in}}%
\pgfpathcurveto{\pgfqpoint{1.005770in}{2.915777in}}{\pgfqpoint{0.995171in}{2.911387in}}{\pgfqpoint{0.987357in}{2.903573in}}%
\pgfpathcurveto{\pgfqpoint{0.979544in}{2.895760in}}{\pgfqpoint{0.975153in}{2.885161in}}{\pgfqpoint{0.975153in}{2.874111in}}%
\pgfpathcurveto{\pgfqpoint{0.975153in}{2.863061in}}{\pgfqpoint{0.979544in}{2.852462in}}{\pgfqpoint{0.987357in}{2.844648in}}%
\pgfpathcurveto{\pgfqpoint{0.995171in}{2.836834in}}{\pgfqpoint{1.005770in}{2.832444in}}{\pgfqpoint{1.016820in}{2.832444in}}%
\pgfpathlineto{\pgfqpoint{1.016820in}{2.832444in}}%
\pgfpathclose%
\pgfusepath{stroke}%
\end{pgfscope}%
\begin{pgfscope}%
\pgfpathrectangle{\pgfqpoint{0.494722in}{0.437222in}}{\pgfqpoint{6.275590in}{5.159444in}}%
\pgfusepath{clip}%
\pgfsetbuttcap%
\pgfsetroundjoin%
\pgfsetlinewidth{1.003750pt}%
\definecolor{currentstroke}{rgb}{0.827451,0.827451,0.827451}%
\pgfsetstrokecolor{currentstroke}%
\pgfsetstrokeopacity{0.800000}%
\pgfsetdash{}{0pt}%
\pgfpathmoveto{\pgfqpoint{3.709744in}{0.686390in}}%
\pgfpathcurveto{\pgfqpoint{3.720794in}{0.686390in}}{\pgfqpoint{3.731393in}{0.690780in}}{\pgfqpoint{3.739207in}{0.698593in}}%
\pgfpathcurveto{\pgfqpoint{3.747021in}{0.706407in}}{\pgfqpoint{3.751411in}{0.717006in}}{\pgfqpoint{3.751411in}{0.728056in}}%
\pgfpathcurveto{\pgfqpoint{3.751411in}{0.739106in}}{\pgfqpoint{3.747021in}{0.749705in}}{\pgfqpoint{3.739207in}{0.757519in}}%
\pgfpathcurveto{\pgfqpoint{3.731393in}{0.765333in}}{\pgfqpoint{3.720794in}{0.769723in}}{\pgfqpoint{3.709744in}{0.769723in}}%
\pgfpathcurveto{\pgfqpoint{3.698694in}{0.769723in}}{\pgfqpoint{3.688095in}{0.765333in}}{\pgfqpoint{3.680281in}{0.757519in}}%
\pgfpathcurveto{\pgfqpoint{3.672468in}{0.749705in}}{\pgfqpoint{3.668077in}{0.739106in}}{\pgfqpoint{3.668077in}{0.728056in}}%
\pgfpathcurveto{\pgfqpoint{3.668077in}{0.717006in}}{\pgfqpoint{3.672468in}{0.706407in}}{\pgfqpoint{3.680281in}{0.698593in}}%
\pgfpathcurveto{\pgfqpoint{3.688095in}{0.690780in}}{\pgfqpoint{3.698694in}{0.686390in}}{\pgfqpoint{3.709744in}{0.686390in}}%
\pgfpathlineto{\pgfqpoint{3.709744in}{0.686390in}}%
\pgfpathclose%
\pgfusepath{stroke}%
\end{pgfscope}%
\begin{pgfscope}%
\pgfpathrectangle{\pgfqpoint{0.494722in}{0.437222in}}{\pgfqpoint{6.275590in}{5.159444in}}%
\pgfusepath{clip}%
\pgfsetbuttcap%
\pgfsetroundjoin%
\pgfsetlinewidth{1.003750pt}%
\definecolor{currentstroke}{rgb}{0.827451,0.827451,0.827451}%
\pgfsetstrokecolor{currentstroke}%
\pgfsetstrokeopacity{0.800000}%
\pgfsetdash{}{0pt}%
\pgfpathmoveto{\pgfqpoint{1.545871in}{1.889457in}}%
\pgfpathcurveto{\pgfqpoint{1.556921in}{1.889457in}}{\pgfqpoint{1.567520in}{1.893848in}}{\pgfqpoint{1.575334in}{1.901661in}}%
\pgfpathcurveto{\pgfqpoint{1.583148in}{1.909475in}}{\pgfqpoint{1.587538in}{1.920074in}}{\pgfqpoint{1.587538in}{1.931124in}}%
\pgfpathcurveto{\pgfqpoint{1.587538in}{1.942174in}}{\pgfqpoint{1.583148in}{1.952773in}}{\pgfqpoint{1.575334in}{1.960587in}}%
\pgfpathcurveto{\pgfqpoint{1.567520in}{1.968400in}}{\pgfqpoint{1.556921in}{1.972791in}}{\pgfqpoint{1.545871in}{1.972791in}}%
\pgfpathcurveto{\pgfqpoint{1.534821in}{1.972791in}}{\pgfqpoint{1.524222in}{1.968400in}}{\pgfqpoint{1.516408in}{1.960587in}}%
\pgfpathcurveto{\pgfqpoint{1.508595in}{1.952773in}}{\pgfqpoint{1.504204in}{1.942174in}}{\pgfqpoint{1.504204in}{1.931124in}}%
\pgfpathcurveto{\pgfqpoint{1.504204in}{1.920074in}}{\pgfqpoint{1.508595in}{1.909475in}}{\pgfqpoint{1.516408in}{1.901661in}}%
\pgfpathcurveto{\pgfqpoint{1.524222in}{1.893848in}}{\pgfqpoint{1.534821in}{1.889457in}}{\pgfqpoint{1.545871in}{1.889457in}}%
\pgfpathlineto{\pgfqpoint{1.545871in}{1.889457in}}%
\pgfpathclose%
\pgfusepath{stroke}%
\end{pgfscope}%
\begin{pgfscope}%
\pgfpathrectangle{\pgfqpoint{0.494722in}{0.437222in}}{\pgfqpoint{6.275590in}{5.159444in}}%
\pgfusepath{clip}%
\pgfsetbuttcap%
\pgfsetroundjoin%
\pgfsetlinewidth{1.003750pt}%
\definecolor{currentstroke}{rgb}{0.827451,0.827451,0.827451}%
\pgfsetstrokecolor{currentstroke}%
\pgfsetstrokeopacity{0.800000}%
\pgfsetdash{}{0pt}%
\pgfpathmoveto{\pgfqpoint{0.520093in}{4.158669in}}%
\pgfpathcurveto{\pgfqpoint{0.531143in}{4.158669in}}{\pgfqpoint{0.541742in}{4.163059in}}{\pgfqpoint{0.549556in}{4.170873in}}%
\pgfpathcurveto{\pgfqpoint{0.557369in}{4.178687in}}{\pgfqpoint{0.561759in}{4.189286in}}{\pgfqpoint{0.561759in}{4.200336in}}%
\pgfpathcurveto{\pgfqpoint{0.561759in}{4.211386in}}{\pgfqpoint{0.557369in}{4.221985in}}{\pgfqpoint{0.549556in}{4.229799in}}%
\pgfpathcurveto{\pgfqpoint{0.541742in}{4.237612in}}{\pgfqpoint{0.531143in}{4.242003in}}{\pgfqpoint{0.520093in}{4.242003in}}%
\pgfpathcurveto{\pgfqpoint{0.509043in}{4.242003in}}{\pgfqpoint{0.498444in}{4.237612in}}{\pgfqpoint{0.490630in}{4.229799in}}%
\pgfpathcurveto{\pgfqpoint{0.482816in}{4.221985in}}{\pgfqpoint{0.478426in}{4.211386in}}{\pgfqpoint{0.478426in}{4.200336in}}%
\pgfpathcurveto{\pgfqpoint{0.478426in}{4.189286in}}{\pgfqpoint{0.482816in}{4.178687in}}{\pgfqpoint{0.490630in}{4.170873in}}%
\pgfpathcurveto{\pgfqpoint{0.498444in}{4.163059in}}{\pgfqpoint{0.509043in}{4.158669in}}{\pgfqpoint{0.520093in}{4.158669in}}%
\pgfpathlineto{\pgfqpoint{0.520093in}{4.158669in}}%
\pgfpathclose%
\pgfusepath{stroke}%
\end{pgfscope}%
\begin{pgfscope}%
\pgfpathrectangle{\pgfqpoint{0.494722in}{0.437222in}}{\pgfqpoint{6.275590in}{5.159444in}}%
\pgfusepath{clip}%
\pgfsetbuttcap%
\pgfsetroundjoin%
\pgfsetlinewidth{1.003750pt}%
\definecolor{currentstroke}{rgb}{0.827451,0.827451,0.827451}%
\pgfsetstrokecolor{currentstroke}%
\pgfsetstrokeopacity{0.800000}%
\pgfsetdash{}{0pt}%
\pgfpathmoveto{\pgfqpoint{1.396743in}{2.042450in}}%
\pgfpathcurveto{\pgfqpoint{1.407793in}{2.042450in}}{\pgfqpoint{1.418392in}{2.046840in}}{\pgfqpoint{1.426206in}{2.054654in}}%
\pgfpathcurveto{\pgfqpoint{1.434019in}{2.062467in}}{\pgfqpoint{1.438409in}{2.073066in}}{\pgfqpoint{1.438409in}{2.084117in}}%
\pgfpathcurveto{\pgfqpoint{1.438409in}{2.095167in}}{\pgfqpoint{1.434019in}{2.105766in}}{\pgfqpoint{1.426206in}{2.113579in}}%
\pgfpathcurveto{\pgfqpoint{1.418392in}{2.121393in}}{\pgfqpoint{1.407793in}{2.125783in}}{\pgfqpoint{1.396743in}{2.125783in}}%
\pgfpathcurveto{\pgfqpoint{1.385693in}{2.125783in}}{\pgfqpoint{1.375094in}{2.121393in}}{\pgfqpoint{1.367280in}{2.113579in}}%
\pgfpathcurveto{\pgfqpoint{1.359466in}{2.105766in}}{\pgfqpoint{1.355076in}{2.095167in}}{\pgfqpoint{1.355076in}{2.084117in}}%
\pgfpathcurveto{\pgfqpoint{1.355076in}{2.073066in}}{\pgfqpoint{1.359466in}{2.062467in}}{\pgfqpoint{1.367280in}{2.054654in}}%
\pgfpathcurveto{\pgfqpoint{1.375094in}{2.046840in}}{\pgfqpoint{1.385693in}{2.042450in}}{\pgfqpoint{1.396743in}{2.042450in}}%
\pgfpathlineto{\pgfqpoint{1.396743in}{2.042450in}}%
\pgfpathclose%
\pgfusepath{stroke}%
\end{pgfscope}%
\begin{pgfscope}%
\pgfpathrectangle{\pgfqpoint{0.494722in}{0.437222in}}{\pgfqpoint{6.275590in}{5.159444in}}%
\pgfusepath{clip}%
\pgfsetbuttcap%
\pgfsetroundjoin%
\pgfsetlinewidth{1.003750pt}%
\definecolor{currentstroke}{rgb}{0.827451,0.827451,0.827451}%
\pgfsetstrokecolor{currentstroke}%
\pgfsetstrokeopacity{0.800000}%
\pgfsetdash{}{0pt}%
\pgfpathmoveto{\pgfqpoint{1.346241in}{2.100145in}}%
\pgfpathcurveto{\pgfqpoint{1.357291in}{2.100145in}}{\pgfqpoint{1.367890in}{2.104535in}}{\pgfqpoint{1.375703in}{2.112349in}}%
\pgfpathcurveto{\pgfqpoint{1.383517in}{2.120163in}}{\pgfqpoint{1.387907in}{2.130762in}}{\pgfqpoint{1.387907in}{2.141812in}}%
\pgfpathcurveto{\pgfqpoint{1.387907in}{2.152862in}}{\pgfqpoint{1.383517in}{2.163461in}}{\pgfqpoint{1.375703in}{2.171274in}}%
\pgfpathcurveto{\pgfqpoint{1.367890in}{2.179088in}}{\pgfqpoint{1.357291in}{2.183478in}}{\pgfqpoint{1.346241in}{2.183478in}}%
\pgfpathcurveto{\pgfqpoint{1.335191in}{2.183478in}}{\pgfqpoint{1.324592in}{2.179088in}}{\pgfqpoint{1.316778in}{2.171274in}}%
\pgfpathcurveto{\pgfqpoint{1.308964in}{2.163461in}}{\pgfqpoint{1.304574in}{2.152862in}}{\pgfqpoint{1.304574in}{2.141812in}}%
\pgfpathcurveto{\pgfqpoint{1.304574in}{2.130762in}}{\pgfqpoint{1.308964in}{2.120163in}}{\pgfqpoint{1.316778in}{2.112349in}}%
\pgfpathcurveto{\pgfqpoint{1.324592in}{2.104535in}}{\pgfqpoint{1.335191in}{2.100145in}}{\pgfqpoint{1.346241in}{2.100145in}}%
\pgfpathlineto{\pgfqpoint{1.346241in}{2.100145in}}%
\pgfpathclose%
\pgfusepath{stroke}%
\end{pgfscope}%
\begin{pgfscope}%
\pgfpathrectangle{\pgfqpoint{0.494722in}{0.437222in}}{\pgfqpoint{6.275590in}{5.159444in}}%
\pgfusepath{clip}%
\pgfsetbuttcap%
\pgfsetroundjoin%
\pgfsetlinewidth{1.003750pt}%
\definecolor{currentstroke}{rgb}{0.827451,0.827451,0.827451}%
\pgfsetstrokecolor{currentstroke}%
\pgfsetstrokeopacity{0.800000}%
\pgfsetdash{}{0pt}%
\pgfpathmoveto{\pgfqpoint{0.679488in}{3.431236in}}%
\pgfpathcurveto{\pgfqpoint{0.690538in}{3.431236in}}{\pgfqpoint{0.701137in}{3.435626in}}{\pgfqpoint{0.708951in}{3.443440in}}%
\pgfpathcurveto{\pgfqpoint{0.716765in}{3.451253in}}{\pgfqpoint{0.721155in}{3.461852in}}{\pgfqpoint{0.721155in}{3.472902in}}%
\pgfpathcurveto{\pgfqpoint{0.721155in}{3.483953in}}{\pgfqpoint{0.716765in}{3.494552in}}{\pgfqpoint{0.708951in}{3.502365in}}%
\pgfpathcurveto{\pgfqpoint{0.701137in}{3.510179in}}{\pgfqpoint{0.690538in}{3.514569in}}{\pgfqpoint{0.679488in}{3.514569in}}%
\pgfpathcurveto{\pgfqpoint{0.668438in}{3.514569in}}{\pgfqpoint{0.657839in}{3.510179in}}{\pgfqpoint{0.650026in}{3.502365in}}%
\pgfpathcurveto{\pgfqpoint{0.642212in}{3.494552in}}{\pgfqpoint{0.637822in}{3.483953in}}{\pgfqpoint{0.637822in}{3.472902in}}%
\pgfpathcurveto{\pgfqpoint{0.637822in}{3.461852in}}{\pgfqpoint{0.642212in}{3.451253in}}{\pgfqpoint{0.650026in}{3.443440in}}%
\pgfpathcurveto{\pgfqpoint{0.657839in}{3.435626in}}{\pgfqpoint{0.668438in}{3.431236in}}{\pgfqpoint{0.679488in}{3.431236in}}%
\pgfpathlineto{\pgfqpoint{0.679488in}{3.431236in}}%
\pgfpathclose%
\pgfusepath{stroke}%
\end{pgfscope}%
\begin{pgfscope}%
\pgfpathrectangle{\pgfqpoint{0.494722in}{0.437222in}}{\pgfqpoint{6.275590in}{5.159444in}}%
\pgfusepath{clip}%
\pgfsetbuttcap%
\pgfsetroundjoin%
\pgfsetlinewidth{1.003750pt}%
\definecolor{currentstroke}{rgb}{0.827451,0.827451,0.827451}%
\pgfsetstrokecolor{currentstroke}%
\pgfsetstrokeopacity{0.800000}%
\pgfsetdash{}{0pt}%
\pgfpathmoveto{\pgfqpoint{3.950409in}{0.627171in}}%
\pgfpathcurveto{\pgfqpoint{3.961459in}{0.627171in}}{\pgfqpoint{3.972058in}{0.631561in}}{\pgfqpoint{3.979872in}{0.639374in}}%
\pgfpathcurveto{\pgfqpoint{3.987685in}{0.647188in}}{\pgfqpoint{3.992076in}{0.657787in}}{\pgfqpoint{3.992076in}{0.668837in}}%
\pgfpathcurveto{\pgfqpoint{3.992076in}{0.679887in}}{\pgfqpoint{3.987685in}{0.690486in}}{\pgfqpoint{3.979872in}{0.698300in}}%
\pgfpathcurveto{\pgfqpoint{3.972058in}{0.706114in}}{\pgfqpoint{3.961459in}{0.710504in}}{\pgfqpoint{3.950409in}{0.710504in}}%
\pgfpathcurveto{\pgfqpoint{3.939359in}{0.710504in}}{\pgfqpoint{3.928760in}{0.706114in}}{\pgfqpoint{3.920946in}{0.698300in}}%
\pgfpathcurveto{\pgfqpoint{3.913133in}{0.690486in}}{\pgfqpoint{3.908742in}{0.679887in}}{\pgfqpoint{3.908742in}{0.668837in}}%
\pgfpathcurveto{\pgfqpoint{3.908742in}{0.657787in}}{\pgfqpoint{3.913133in}{0.647188in}}{\pgfqpoint{3.920946in}{0.639374in}}%
\pgfpathcurveto{\pgfqpoint{3.928760in}{0.631561in}}{\pgfqpoint{3.939359in}{0.627171in}}{\pgfqpoint{3.950409in}{0.627171in}}%
\pgfpathlineto{\pgfqpoint{3.950409in}{0.627171in}}%
\pgfpathclose%
\pgfusepath{stroke}%
\end{pgfscope}%
\begin{pgfscope}%
\pgfpathrectangle{\pgfqpoint{0.494722in}{0.437222in}}{\pgfqpoint{6.275590in}{5.159444in}}%
\pgfusepath{clip}%
\pgfsetbuttcap%
\pgfsetroundjoin%
\pgfsetlinewidth{1.003750pt}%
\definecolor{currentstroke}{rgb}{0.827451,0.827451,0.827451}%
\pgfsetstrokecolor{currentstroke}%
\pgfsetstrokeopacity{0.800000}%
\pgfsetdash{}{0pt}%
\pgfpathmoveto{\pgfqpoint{4.942939in}{0.453295in}}%
\pgfpathcurveto{\pgfqpoint{4.953989in}{0.453295in}}{\pgfqpoint{4.964589in}{0.457685in}}{\pgfqpoint{4.972402in}{0.465499in}}%
\pgfpathcurveto{\pgfqpoint{4.980216in}{0.473313in}}{\pgfqpoint{4.984606in}{0.483912in}}{\pgfqpoint{4.984606in}{0.494962in}}%
\pgfpathcurveto{\pgfqpoint{4.984606in}{0.506012in}}{\pgfqpoint{4.980216in}{0.516611in}}{\pgfqpoint{4.972402in}{0.524425in}}%
\pgfpathcurveto{\pgfqpoint{4.964589in}{0.532238in}}{\pgfqpoint{4.953989in}{0.536628in}}{\pgfqpoint{4.942939in}{0.536628in}}%
\pgfpathcurveto{\pgfqpoint{4.931889in}{0.536628in}}{\pgfqpoint{4.921290in}{0.532238in}}{\pgfqpoint{4.913477in}{0.524425in}}%
\pgfpathcurveto{\pgfqpoint{4.905663in}{0.516611in}}{\pgfqpoint{4.901273in}{0.506012in}}{\pgfqpoint{4.901273in}{0.494962in}}%
\pgfpathcurveto{\pgfqpoint{4.901273in}{0.483912in}}{\pgfqpoint{4.905663in}{0.473313in}}{\pgfqpoint{4.913477in}{0.465499in}}%
\pgfpathcurveto{\pgfqpoint{4.921290in}{0.457685in}}{\pgfqpoint{4.931889in}{0.453295in}}{\pgfqpoint{4.942939in}{0.453295in}}%
\pgfpathlineto{\pgfqpoint{4.942939in}{0.453295in}}%
\pgfpathclose%
\pgfusepath{stroke}%
\end{pgfscope}%
\begin{pgfscope}%
\pgfpathrectangle{\pgfqpoint{0.494722in}{0.437222in}}{\pgfqpoint{6.275590in}{5.159444in}}%
\pgfusepath{clip}%
\pgfsetbuttcap%
\pgfsetroundjoin%
\pgfsetlinewidth{1.003750pt}%
\definecolor{currentstroke}{rgb}{0.827451,0.827451,0.827451}%
\pgfsetstrokecolor{currentstroke}%
\pgfsetstrokeopacity{0.800000}%
\pgfsetdash{}{0pt}%
\pgfpathmoveto{\pgfqpoint{1.926347in}{1.574489in}}%
\pgfpathcurveto{\pgfqpoint{1.937397in}{1.574489in}}{\pgfqpoint{1.947996in}{1.578879in}}{\pgfqpoint{1.955809in}{1.586693in}}%
\pgfpathcurveto{\pgfqpoint{1.963623in}{1.594506in}}{\pgfqpoint{1.968013in}{1.605105in}}{\pgfqpoint{1.968013in}{1.616155in}}%
\pgfpathcurveto{\pgfqpoint{1.968013in}{1.627206in}}{\pgfqpoint{1.963623in}{1.637805in}}{\pgfqpoint{1.955809in}{1.645618in}}%
\pgfpathcurveto{\pgfqpoint{1.947996in}{1.653432in}}{\pgfqpoint{1.937397in}{1.657822in}}{\pgfqpoint{1.926347in}{1.657822in}}%
\pgfpathcurveto{\pgfqpoint{1.915296in}{1.657822in}}{\pgfqpoint{1.904697in}{1.653432in}}{\pgfqpoint{1.896884in}{1.645618in}}%
\pgfpathcurveto{\pgfqpoint{1.889070in}{1.637805in}}{\pgfqpoint{1.884680in}{1.627206in}}{\pgfqpoint{1.884680in}{1.616155in}}%
\pgfpathcurveto{\pgfqpoint{1.884680in}{1.605105in}}{\pgfqpoint{1.889070in}{1.594506in}}{\pgfqpoint{1.896884in}{1.586693in}}%
\pgfpathcurveto{\pgfqpoint{1.904697in}{1.578879in}}{\pgfqpoint{1.915296in}{1.574489in}}{\pgfqpoint{1.926347in}{1.574489in}}%
\pgfpathlineto{\pgfqpoint{1.926347in}{1.574489in}}%
\pgfpathclose%
\pgfusepath{stroke}%
\end{pgfscope}%
\begin{pgfscope}%
\pgfpathrectangle{\pgfqpoint{0.494722in}{0.437222in}}{\pgfqpoint{6.275590in}{5.159444in}}%
\pgfusepath{clip}%
\pgfsetbuttcap%
\pgfsetroundjoin%
\pgfsetlinewidth{1.003750pt}%
\definecolor{currentstroke}{rgb}{0.827451,0.827451,0.827451}%
\pgfsetstrokecolor{currentstroke}%
\pgfsetstrokeopacity{0.800000}%
\pgfsetdash{}{0pt}%
\pgfpathmoveto{\pgfqpoint{1.766183in}{1.687308in}}%
\pgfpathcurveto{\pgfqpoint{1.777233in}{1.687308in}}{\pgfqpoint{1.787832in}{1.691698in}}{\pgfqpoint{1.795646in}{1.699512in}}%
\pgfpathcurveto{\pgfqpoint{1.803459in}{1.707325in}}{\pgfqpoint{1.807850in}{1.717924in}}{\pgfqpoint{1.807850in}{1.728974in}}%
\pgfpathcurveto{\pgfqpoint{1.807850in}{1.740025in}}{\pgfqpoint{1.803459in}{1.750624in}}{\pgfqpoint{1.795646in}{1.758437in}}%
\pgfpathcurveto{\pgfqpoint{1.787832in}{1.766251in}}{\pgfqpoint{1.777233in}{1.770641in}}{\pgfqpoint{1.766183in}{1.770641in}}%
\pgfpathcurveto{\pgfqpoint{1.755133in}{1.770641in}}{\pgfqpoint{1.744534in}{1.766251in}}{\pgfqpoint{1.736720in}{1.758437in}}%
\pgfpathcurveto{\pgfqpoint{1.728907in}{1.750624in}}{\pgfqpoint{1.724516in}{1.740025in}}{\pgfqpoint{1.724516in}{1.728974in}}%
\pgfpathcurveto{\pgfqpoint{1.724516in}{1.717924in}}{\pgfqpoint{1.728907in}{1.707325in}}{\pgfqpoint{1.736720in}{1.699512in}}%
\pgfpathcurveto{\pgfqpoint{1.744534in}{1.691698in}}{\pgfqpoint{1.755133in}{1.687308in}}{\pgfqpoint{1.766183in}{1.687308in}}%
\pgfpathlineto{\pgfqpoint{1.766183in}{1.687308in}}%
\pgfpathclose%
\pgfusepath{stroke}%
\end{pgfscope}%
\begin{pgfscope}%
\pgfpathrectangle{\pgfqpoint{0.494722in}{0.437222in}}{\pgfqpoint{6.275590in}{5.159444in}}%
\pgfusepath{clip}%
\pgfsetbuttcap%
\pgfsetroundjoin%
\pgfsetlinewidth{1.003750pt}%
\definecolor{currentstroke}{rgb}{0.827451,0.827451,0.827451}%
\pgfsetstrokecolor{currentstroke}%
\pgfsetstrokeopacity{0.800000}%
\pgfsetdash{}{0pt}%
\pgfpathmoveto{\pgfqpoint{0.750952in}{3.177253in}}%
\pgfpathcurveto{\pgfqpoint{0.762002in}{3.177253in}}{\pgfqpoint{0.772601in}{3.181643in}}{\pgfqpoint{0.780414in}{3.189457in}}%
\pgfpathcurveto{\pgfqpoint{0.788228in}{3.197270in}}{\pgfqpoint{0.792618in}{3.207870in}}{\pgfqpoint{0.792618in}{3.218920in}}%
\pgfpathcurveto{\pgfqpoint{0.792618in}{3.229970in}}{\pgfqpoint{0.788228in}{3.240569in}}{\pgfqpoint{0.780414in}{3.248382in}}%
\pgfpathcurveto{\pgfqpoint{0.772601in}{3.256196in}}{\pgfqpoint{0.762002in}{3.260586in}}{\pgfqpoint{0.750952in}{3.260586in}}%
\pgfpathcurveto{\pgfqpoint{0.739901in}{3.260586in}}{\pgfqpoint{0.729302in}{3.256196in}}{\pgfqpoint{0.721489in}{3.248382in}}%
\pgfpathcurveto{\pgfqpoint{0.713675in}{3.240569in}}{\pgfqpoint{0.709285in}{3.229970in}}{\pgfqpoint{0.709285in}{3.218920in}}%
\pgfpathcurveto{\pgfqpoint{0.709285in}{3.207870in}}{\pgfqpoint{0.713675in}{3.197270in}}{\pgfqpoint{0.721489in}{3.189457in}}%
\pgfpathcurveto{\pgfqpoint{0.729302in}{3.181643in}}{\pgfqpoint{0.739901in}{3.177253in}}{\pgfqpoint{0.750952in}{3.177253in}}%
\pgfpathlineto{\pgfqpoint{0.750952in}{3.177253in}}%
\pgfpathclose%
\pgfusepath{stroke}%
\end{pgfscope}%
\begin{pgfscope}%
\pgfpathrectangle{\pgfqpoint{0.494722in}{0.437222in}}{\pgfqpoint{6.275590in}{5.159444in}}%
\pgfusepath{clip}%
\pgfsetbuttcap%
\pgfsetroundjoin%
\pgfsetlinewidth{1.003750pt}%
\definecolor{currentstroke}{rgb}{0.827451,0.827451,0.827451}%
\pgfsetstrokecolor{currentstroke}%
\pgfsetstrokeopacity{0.800000}%
\pgfsetdash{}{0pt}%
\pgfpathmoveto{\pgfqpoint{0.974123in}{2.860396in}}%
\pgfpathcurveto{\pgfqpoint{0.985174in}{2.860396in}}{\pgfqpoint{0.995773in}{2.864786in}}{\pgfqpoint{1.003586in}{2.872600in}}%
\pgfpathcurveto{\pgfqpoint{1.011400in}{2.880414in}}{\pgfqpoint{1.015790in}{2.891013in}}{\pgfqpoint{1.015790in}{2.902063in}}%
\pgfpathcurveto{\pgfqpoint{1.015790in}{2.913113in}}{\pgfqpoint{1.011400in}{2.923712in}}{\pgfqpoint{1.003586in}{2.931526in}}%
\pgfpathcurveto{\pgfqpoint{0.995773in}{2.939339in}}{\pgfqpoint{0.985174in}{2.943729in}}{\pgfqpoint{0.974123in}{2.943729in}}%
\pgfpathcurveto{\pgfqpoint{0.963073in}{2.943729in}}{\pgfqpoint{0.952474in}{2.939339in}}{\pgfqpoint{0.944661in}{2.931526in}}%
\pgfpathcurveto{\pgfqpoint{0.936847in}{2.923712in}}{\pgfqpoint{0.932457in}{2.913113in}}{\pgfqpoint{0.932457in}{2.902063in}}%
\pgfpathcurveto{\pgfqpoint{0.932457in}{2.891013in}}{\pgfqpoint{0.936847in}{2.880414in}}{\pgfqpoint{0.944661in}{2.872600in}}%
\pgfpathcurveto{\pgfqpoint{0.952474in}{2.864786in}}{\pgfqpoint{0.963073in}{2.860396in}}{\pgfqpoint{0.974123in}{2.860396in}}%
\pgfpathlineto{\pgfqpoint{0.974123in}{2.860396in}}%
\pgfpathclose%
\pgfusepath{stroke}%
\end{pgfscope}%
\begin{pgfscope}%
\pgfpathrectangle{\pgfqpoint{0.494722in}{0.437222in}}{\pgfqpoint{6.275590in}{5.159444in}}%
\pgfusepath{clip}%
\pgfsetbuttcap%
\pgfsetroundjoin%
\pgfsetlinewidth{1.003750pt}%
\definecolor{currentstroke}{rgb}{0.827451,0.827451,0.827451}%
\pgfsetstrokecolor{currentstroke}%
\pgfsetstrokeopacity{0.800000}%
\pgfsetdash{}{0pt}%
\pgfpathmoveto{\pgfqpoint{4.970960in}{0.441718in}}%
\pgfpathcurveto{\pgfqpoint{4.982010in}{0.441718in}}{\pgfqpoint{4.992609in}{0.446108in}}{\pgfqpoint{5.000423in}{0.453922in}}%
\pgfpathcurveto{\pgfqpoint{5.008236in}{0.461735in}}{\pgfqpoint{5.012626in}{0.472334in}}{\pgfqpoint{5.012626in}{0.483385in}}%
\pgfpathcurveto{\pgfqpoint{5.012626in}{0.494435in}}{\pgfqpoint{5.008236in}{0.505034in}}{\pgfqpoint{5.000423in}{0.512847in}}%
\pgfpathcurveto{\pgfqpoint{4.992609in}{0.520661in}}{\pgfqpoint{4.982010in}{0.525051in}}{\pgfqpoint{4.970960in}{0.525051in}}%
\pgfpathcurveto{\pgfqpoint{4.959910in}{0.525051in}}{\pgfqpoint{4.949311in}{0.520661in}}{\pgfqpoint{4.941497in}{0.512847in}}%
\pgfpathcurveto{\pgfqpoint{4.933683in}{0.505034in}}{\pgfqpoint{4.929293in}{0.494435in}}{\pgfqpoint{4.929293in}{0.483385in}}%
\pgfpathcurveto{\pgfqpoint{4.929293in}{0.472334in}}{\pgfqpoint{4.933683in}{0.461735in}}{\pgfqpoint{4.941497in}{0.453922in}}%
\pgfpathcurveto{\pgfqpoint{4.949311in}{0.446108in}}{\pgfqpoint{4.959910in}{0.441718in}}{\pgfqpoint{4.970960in}{0.441718in}}%
\pgfpathlineto{\pgfqpoint{4.970960in}{0.441718in}}%
\pgfpathclose%
\pgfusepath{stroke}%
\end{pgfscope}%
\begin{pgfscope}%
\pgfpathrectangle{\pgfqpoint{0.494722in}{0.437222in}}{\pgfqpoint{6.275590in}{5.159444in}}%
\pgfusepath{clip}%
\pgfsetbuttcap%
\pgfsetroundjoin%
\pgfsetlinewidth{1.003750pt}%
\definecolor{currentstroke}{rgb}{0.827451,0.827451,0.827451}%
\pgfsetstrokecolor{currentstroke}%
\pgfsetstrokeopacity{0.800000}%
\pgfsetdash{}{0pt}%
\pgfpathmoveto{\pgfqpoint{1.381781in}{2.056420in}}%
\pgfpathcurveto{\pgfqpoint{1.392831in}{2.056420in}}{\pgfqpoint{1.403430in}{2.060810in}}{\pgfqpoint{1.411244in}{2.068624in}}%
\pgfpathcurveto{\pgfqpoint{1.419057in}{2.076437in}}{\pgfqpoint{1.423448in}{2.087036in}}{\pgfqpoint{1.423448in}{2.098087in}}%
\pgfpathcurveto{\pgfqpoint{1.423448in}{2.109137in}}{\pgfqpoint{1.419057in}{2.119736in}}{\pgfqpoint{1.411244in}{2.127549in}}%
\pgfpathcurveto{\pgfqpoint{1.403430in}{2.135363in}}{\pgfqpoint{1.392831in}{2.139753in}}{\pgfqpoint{1.381781in}{2.139753in}}%
\pgfpathcurveto{\pgfqpoint{1.370731in}{2.139753in}}{\pgfqpoint{1.360132in}{2.135363in}}{\pgfqpoint{1.352318in}{2.127549in}}%
\pgfpathcurveto{\pgfqpoint{1.344505in}{2.119736in}}{\pgfqpoint{1.340114in}{2.109137in}}{\pgfqpoint{1.340114in}{2.098087in}}%
\pgfpathcurveto{\pgfqpoint{1.340114in}{2.087036in}}{\pgfqpoint{1.344505in}{2.076437in}}{\pgfqpoint{1.352318in}{2.068624in}}%
\pgfpathcurveto{\pgfqpoint{1.360132in}{2.060810in}}{\pgfqpoint{1.370731in}{2.056420in}}{\pgfqpoint{1.381781in}{2.056420in}}%
\pgfpathlineto{\pgfqpoint{1.381781in}{2.056420in}}%
\pgfpathclose%
\pgfusepath{stroke}%
\end{pgfscope}%
\begin{pgfscope}%
\pgfpathrectangle{\pgfqpoint{0.494722in}{0.437222in}}{\pgfqpoint{6.275590in}{5.159444in}}%
\pgfusepath{clip}%
\pgfsetbuttcap%
\pgfsetroundjoin%
\pgfsetlinewidth{1.003750pt}%
\definecolor{currentstroke}{rgb}{0.827451,0.827451,0.827451}%
\pgfsetstrokecolor{currentstroke}%
\pgfsetstrokeopacity{0.800000}%
\pgfsetdash{}{0pt}%
\pgfpathmoveto{\pgfqpoint{1.786945in}{1.657846in}}%
\pgfpathcurveto{\pgfqpoint{1.797995in}{1.657846in}}{\pgfqpoint{1.808594in}{1.662237in}}{\pgfqpoint{1.816407in}{1.670050in}}%
\pgfpathcurveto{\pgfqpoint{1.824221in}{1.677864in}}{\pgfqpoint{1.828611in}{1.688463in}}{\pgfqpoint{1.828611in}{1.699513in}}%
\pgfpathcurveto{\pgfqpoint{1.828611in}{1.710563in}}{\pgfqpoint{1.824221in}{1.721162in}}{\pgfqpoint{1.816407in}{1.728976in}}%
\pgfpathcurveto{\pgfqpoint{1.808594in}{1.736789in}}{\pgfqpoint{1.797995in}{1.741180in}}{\pgfqpoint{1.786945in}{1.741180in}}%
\pgfpathcurveto{\pgfqpoint{1.775894in}{1.741180in}}{\pgfqpoint{1.765295in}{1.736789in}}{\pgfqpoint{1.757482in}{1.728976in}}%
\pgfpathcurveto{\pgfqpoint{1.749668in}{1.721162in}}{\pgfqpoint{1.745278in}{1.710563in}}{\pgfqpoint{1.745278in}{1.699513in}}%
\pgfpathcurveto{\pgfqpoint{1.745278in}{1.688463in}}{\pgfqpoint{1.749668in}{1.677864in}}{\pgfqpoint{1.757482in}{1.670050in}}%
\pgfpathcurveto{\pgfqpoint{1.765295in}{1.662237in}}{\pgfqpoint{1.775894in}{1.657846in}}{\pgfqpoint{1.786945in}{1.657846in}}%
\pgfpathlineto{\pgfqpoint{1.786945in}{1.657846in}}%
\pgfpathclose%
\pgfusepath{stroke}%
\end{pgfscope}%
\begin{pgfscope}%
\pgfpathrectangle{\pgfqpoint{0.494722in}{0.437222in}}{\pgfqpoint{6.275590in}{5.159444in}}%
\pgfusepath{clip}%
\pgfsetbuttcap%
\pgfsetroundjoin%
\pgfsetlinewidth{1.003750pt}%
\definecolor{currentstroke}{rgb}{0.827451,0.827451,0.827451}%
\pgfsetstrokecolor{currentstroke}%
\pgfsetstrokeopacity{0.800000}%
\pgfsetdash{}{0pt}%
\pgfpathmoveto{\pgfqpoint{0.683182in}{3.327378in}}%
\pgfpathcurveto{\pgfqpoint{0.694232in}{3.327378in}}{\pgfqpoint{0.704831in}{3.331768in}}{\pgfqpoint{0.712645in}{3.339581in}}%
\pgfpathcurveto{\pgfqpoint{0.720459in}{3.347395in}}{\pgfqpoint{0.724849in}{3.357994in}}{\pgfqpoint{0.724849in}{3.369044in}}%
\pgfpathcurveto{\pgfqpoint{0.724849in}{3.380094in}}{\pgfqpoint{0.720459in}{3.390693in}}{\pgfqpoint{0.712645in}{3.398507in}}%
\pgfpathcurveto{\pgfqpoint{0.704831in}{3.406321in}}{\pgfqpoint{0.694232in}{3.410711in}}{\pgfqpoint{0.683182in}{3.410711in}}%
\pgfpathcurveto{\pgfqpoint{0.672132in}{3.410711in}}{\pgfqpoint{0.661533in}{3.406321in}}{\pgfqpoint{0.653720in}{3.398507in}}%
\pgfpathcurveto{\pgfqpoint{0.645906in}{3.390693in}}{\pgfqpoint{0.641516in}{3.380094in}}{\pgfqpoint{0.641516in}{3.369044in}}%
\pgfpathcurveto{\pgfqpoint{0.641516in}{3.357994in}}{\pgfqpoint{0.645906in}{3.347395in}}{\pgfqpoint{0.653720in}{3.339581in}}%
\pgfpathcurveto{\pgfqpoint{0.661533in}{3.331768in}}{\pgfqpoint{0.672132in}{3.327378in}}{\pgfqpoint{0.683182in}{3.327378in}}%
\pgfpathlineto{\pgfqpoint{0.683182in}{3.327378in}}%
\pgfpathclose%
\pgfusepath{stroke}%
\end{pgfscope}%
\begin{pgfscope}%
\pgfpathrectangle{\pgfqpoint{0.494722in}{0.437222in}}{\pgfqpoint{6.275590in}{5.159444in}}%
\pgfusepath{clip}%
\pgfsetbuttcap%
\pgfsetroundjoin%
\pgfsetlinewidth{1.003750pt}%
\definecolor{currentstroke}{rgb}{0.827451,0.827451,0.827451}%
\pgfsetstrokecolor{currentstroke}%
\pgfsetstrokeopacity{0.800000}%
\pgfsetdash{}{0pt}%
\pgfpathmoveto{\pgfqpoint{1.315544in}{2.144827in}}%
\pgfpathcurveto{\pgfqpoint{1.326594in}{2.144827in}}{\pgfqpoint{1.337193in}{2.149218in}}{\pgfqpoint{1.345006in}{2.157031in}}%
\pgfpathcurveto{\pgfqpoint{1.352820in}{2.164845in}}{\pgfqpoint{1.357210in}{2.175444in}}{\pgfqpoint{1.357210in}{2.186494in}}%
\pgfpathcurveto{\pgfqpoint{1.357210in}{2.197544in}}{\pgfqpoint{1.352820in}{2.208143in}}{\pgfqpoint{1.345006in}{2.215957in}}%
\pgfpathcurveto{\pgfqpoint{1.337193in}{2.223770in}}{\pgfqpoint{1.326594in}{2.228161in}}{\pgfqpoint{1.315544in}{2.228161in}}%
\pgfpathcurveto{\pgfqpoint{1.304494in}{2.228161in}}{\pgfqpoint{1.293894in}{2.223770in}}{\pgfqpoint{1.286081in}{2.215957in}}%
\pgfpathcurveto{\pgfqpoint{1.278267in}{2.208143in}}{\pgfqpoint{1.273877in}{2.197544in}}{\pgfqpoint{1.273877in}{2.186494in}}%
\pgfpathcurveto{\pgfqpoint{1.273877in}{2.175444in}}{\pgfqpoint{1.278267in}{2.164845in}}{\pgfqpoint{1.286081in}{2.157031in}}%
\pgfpathcurveto{\pgfqpoint{1.293894in}{2.149218in}}{\pgfqpoint{1.304494in}{2.144827in}}{\pgfqpoint{1.315544in}{2.144827in}}%
\pgfpathlineto{\pgfqpoint{1.315544in}{2.144827in}}%
\pgfpathclose%
\pgfusepath{stroke}%
\end{pgfscope}%
\begin{pgfscope}%
\pgfpathrectangle{\pgfqpoint{0.494722in}{0.437222in}}{\pgfqpoint{6.275590in}{5.159444in}}%
\pgfusepath{clip}%
\pgfsetbuttcap%
\pgfsetroundjoin%
\pgfsetlinewidth{1.003750pt}%
\definecolor{currentstroke}{rgb}{0.827451,0.827451,0.827451}%
\pgfsetstrokecolor{currentstroke}%
\pgfsetstrokeopacity{0.800000}%
\pgfsetdash{}{0pt}%
\pgfpathmoveto{\pgfqpoint{4.222858in}{0.545257in}}%
\pgfpathcurveto{\pgfqpoint{4.233908in}{0.545257in}}{\pgfqpoint{4.244507in}{0.549647in}}{\pgfqpoint{4.252321in}{0.557461in}}%
\pgfpathcurveto{\pgfqpoint{4.260134in}{0.565274in}}{\pgfqpoint{4.264525in}{0.575873in}}{\pgfqpoint{4.264525in}{0.586923in}}%
\pgfpathcurveto{\pgfqpoint{4.264525in}{0.597974in}}{\pgfqpoint{4.260134in}{0.608573in}}{\pgfqpoint{4.252321in}{0.616386in}}%
\pgfpathcurveto{\pgfqpoint{4.244507in}{0.624200in}}{\pgfqpoint{4.233908in}{0.628590in}}{\pgfqpoint{4.222858in}{0.628590in}}%
\pgfpathcurveto{\pgfqpoint{4.211808in}{0.628590in}}{\pgfqpoint{4.201209in}{0.624200in}}{\pgfqpoint{4.193395in}{0.616386in}}%
\pgfpathcurveto{\pgfqpoint{4.185581in}{0.608573in}}{\pgfqpoint{4.181191in}{0.597974in}}{\pgfqpoint{4.181191in}{0.586923in}}%
\pgfpathcurveto{\pgfqpoint{4.181191in}{0.575873in}}{\pgfqpoint{4.185581in}{0.565274in}}{\pgfqpoint{4.193395in}{0.557461in}}%
\pgfpathcurveto{\pgfqpoint{4.201209in}{0.549647in}}{\pgfqpoint{4.211808in}{0.545257in}}{\pgfqpoint{4.222858in}{0.545257in}}%
\pgfpathlineto{\pgfqpoint{4.222858in}{0.545257in}}%
\pgfpathclose%
\pgfusepath{stroke}%
\end{pgfscope}%
\begin{pgfscope}%
\pgfpathrectangle{\pgfqpoint{0.494722in}{0.437222in}}{\pgfqpoint{6.275590in}{5.159444in}}%
\pgfusepath{clip}%
\pgfsetbuttcap%
\pgfsetroundjoin%
\pgfsetlinewidth{1.003750pt}%
\definecolor{currentstroke}{rgb}{0.827451,0.827451,0.827451}%
\pgfsetstrokecolor{currentstroke}%
\pgfsetstrokeopacity{0.800000}%
\pgfsetdash{}{0pt}%
\pgfpathmoveto{\pgfqpoint{1.286891in}{2.175474in}}%
\pgfpathcurveto{\pgfqpoint{1.297941in}{2.175474in}}{\pgfqpoint{1.308540in}{2.179864in}}{\pgfqpoint{1.316354in}{2.187678in}}%
\pgfpathcurveto{\pgfqpoint{1.324167in}{2.195491in}}{\pgfqpoint{1.328558in}{2.206090in}}{\pgfqpoint{1.328558in}{2.217140in}}%
\pgfpathcurveto{\pgfqpoint{1.328558in}{2.228190in}}{\pgfqpoint{1.324167in}{2.238789in}}{\pgfqpoint{1.316354in}{2.246603in}}%
\pgfpathcurveto{\pgfqpoint{1.308540in}{2.254417in}}{\pgfqpoint{1.297941in}{2.258807in}}{\pgfqpoint{1.286891in}{2.258807in}}%
\pgfpathcurveto{\pgfqpoint{1.275841in}{2.258807in}}{\pgfqpoint{1.265242in}{2.254417in}}{\pgfqpoint{1.257428in}{2.246603in}}%
\pgfpathcurveto{\pgfqpoint{1.249614in}{2.238789in}}{\pgfqpoint{1.245224in}{2.228190in}}{\pgfqpoint{1.245224in}{2.217140in}}%
\pgfpathcurveto{\pgfqpoint{1.245224in}{2.206090in}}{\pgfqpoint{1.249614in}{2.195491in}}{\pgfqpoint{1.257428in}{2.187678in}}%
\pgfpathcurveto{\pgfqpoint{1.265242in}{2.179864in}}{\pgfqpoint{1.275841in}{2.175474in}}{\pgfqpoint{1.286891in}{2.175474in}}%
\pgfpathlineto{\pgfqpoint{1.286891in}{2.175474in}}%
\pgfpathclose%
\pgfusepath{stroke}%
\end{pgfscope}%
\begin{pgfscope}%
\pgfpathrectangle{\pgfqpoint{0.494722in}{0.437222in}}{\pgfqpoint{6.275590in}{5.159444in}}%
\pgfusepath{clip}%
\pgfsetbuttcap%
\pgfsetroundjoin%
\pgfsetlinewidth{1.003750pt}%
\definecolor{currentstroke}{rgb}{0.827451,0.827451,0.827451}%
\pgfsetstrokecolor{currentstroke}%
\pgfsetstrokeopacity{0.800000}%
\pgfsetdash{}{0pt}%
\pgfpathmoveto{\pgfqpoint{3.321971in}{0.791856in}}%
\pgfpathcurveto{\pgfqpoint{3.333021in}{0.791856in}}{\pgfqpoint{3.343620in}{0.796246in}}{\pgfqpoint{3.351433in}{0.804060in}}%
\pgfpathcurveto{\pgfqpoint{3.359247in}{0.811873in}}{\pgfqpoint{3.363637in}{0.822472in}}{\pgfqpoint{3.363637in}{0.833523in}}%
\pgfpathcurveto{\pgfqpoint{3.363637in}{0.844573in}}{\pgfqpoint{3.359247in}{0.855172in}}{\pgfqpoint{3.351433in}{0.862985in}}%
\pgfpathcurveto{\pgfqpoint{3.343620in}{0.870799in}}{\pgfqpoint{3.333021in}{0.875189in}}{\pgfqpoint{3.321971in}{0.875189in}}%
\pgfpathcurveto{\pgfqpoint{3.310921in}{0.875189in}}{\pgfqpoint{3.300321in}{0.870799in}}{\pgfqpoint{3.292508in}{0.862985in}}%
\pgfpathcurveto{\pgfqpoint{3.284694in}{0.855172in}}{\pgfqpoint{3.280304in}{0.844573in}}{\pgfqpoint{3.280304in}{0.833523in}}%
\pgfpathcurveto{\pgfqpoint{3.280304in}{0.822472in}}{\pgfqpoint{3.284694in}{0.811873in}}{\pgfqpoint{3.292508in}{0.804060in}}%
\pgfpathcurveto{\pgfqpoint{3.300321in}{0.796246in}}{\pgfqpoint{3.310921in}{0.791856in}}{\pgfqpoint{3.321971in}{0.791856in}}%
\pgfpathlineto{\pgfqpoint{3.321971in}{0.791856in}}%
\pgfpathclose%
\pgfusepath{stroke}%
\end{pgfscope}%
\begin{pgfscope}%
\pgfpathrectangle{\pgfqpoint{0.494722in}{0.437222in}}{\pgfqpoint{6.275590in}{5.159444in}}%
\pgfusepath{clip}%
\pgfsetbuttcap%
\pgfsetroundjoin%
\pgfsetlinewidth{1.003750pt}%
\definecolor{currentstroke}{rgb}{0.827451,0.827451,0.827451}%
\pgfsetstrokecolor{currentstroke}%
\pgfsetstrokeopacity{0.800000}%
\pgfsetdash{}{0pt}%
\pgfpathmoveto{\pgfqpoint{2.062312in}{1.444503in}}%
\pgfpathcurveto{\pgfqpoint{2.073362in}{1.444503in}}{\pgfqpoint{2.083961in}{1.448893in}}{\pgfqpoint{2.091775in}{1.456707in}}%
\pgfpathcurveto{\pgfqpoint{2.099588in}{1.464520in}}{\pgfqpoint{2.103979in}{1.475119in}}{\pgfqpoint{2.103979in}{1.486169in}}%
\pgfpathcurveto{\pgfqpoint{2.103979in}{1.497220in}}{\pgfqpoint{2.099588in}{1.507819in}}{\pgfqpoint{2.091775in}{1.515632in}}%
\pgfpathcurveto{\pgfqpoint{2.083961in}{1.523446in}}{\pgfqpoint{2.073362in}{1.527836in}}{\pgfqpoint{2.062312in}{1.527836in}}%
\pgfpathcurveto{\pgfqpoint{2.051262in}{1.527836in}}{\pgfqpoint{2.040663in}{1.523446in}}{\pgfqpoint{2.032849in}{1.515632in}}%
\pgfpathcurveto{\pgfqpoint{2.025036in}{1.507819in}}{\pgfqpoint{2.020645in}{1.497220in}}{\pgfqpoint{2.020645in}{1.486169in}}%
\pgfpathcurveto{\pgfqpoint{2.020645in}{1.475119in}}{\pgfqpoint{2.025036in}{1.464520in}}{\pgfqpoint{2.032849in}{1.456707in}}%
\pgfpathcurveto{\pgfqpoint{2.040663in}{1.448893in}}{\pgfqpoint{2.051262in}{1.444503in}}{\pgfqpoint{2.062312in}{1.444503in}}%
\pgfpathlineto{\pgfqpoint{2.062312in}{1.444503in}}%
\pgfpathclose%
\pgfusepath{stroke}%
\end{pgfscope}%
\begin{pgfscope}%
\pgfpathrectangle{\pgfqpoint{0.494722in}{0.437222in}}{\pgfqpoint{6.275590in}{5.159444in}}%
\pgfusepath{clip}%
\pgfsetbuttcap%
\pgfsetroundjoin%
\pgfsetlinewidth{1.003750pt}%
\definecolor{currentstroke}{rgb}{0.827451,0.827451,0.827451}%
\pgfsetstrokecolor{currentstroke}%
\pgfsetstrokeopacity{0.800000}%
\pgfsetdash{}{0pt}%
\pgfpathmoveto{\pgfqpoint{2.203827in}{1.337053in}}%
\pgfpathcurveto{\pgfqpoint{2.214877in}{1.337053in}}{\pgfqpoint{2.225476in}{1.341443in}}{\pgfqpoint{2.233290in}{1.349257in}}%
\pgfpathcurveto{\pgfqpoint{2.241103in}{1.357070in}}{\pgfqpoint{2.245494in}{1.367669in}}{\pgfqpoint{2.245494in}{1.378720in}}%
\pgfpathcurveto{\pgfqpoint{2.245494in}{1.389770in}}{\pgfqpoint{2.241103in}{1.400369in}}{\pgfqpoint{2.233290in}{1.408182in}}%
\pgfpathcurveto{\pgfqpoint{2.225476in}{1.415996in}}{\pgfqpoint{2.214877in}{1.420386in}}{\pgfqpoint{2.203827in}{1.420386in}}%
\pgfpathcurveto{\pgfqpoint{2.192777in}{1.420386in}}{\pgfqpoint{2.182178in}{1.415996in}}{\pgfqpoint{2.174364in}{1.408182in}}%
\pgfpathcurveto{\pgfqpoint{2.166550in}{1.400369in}}{\pgfqpoint{2.162160in}{1.389770in}}{\pgfqpoint{2.162160in}{1.378720in}}%
\pgfpathcurveto{\pgfqpoint{2.162160in}{1.367669in}}{\pgfqpoint{2.166550in}{1.357070in}}{\pgfqpoint{2.174364in}{1.349257in}}%
\pgfpathcurveto{\pgfqpoint{2.182178in}{1.341443in}}{\pgfqpoint{2.192777in}{1.337053in}}{\pgfqpoint{2.203827in}{1.337053in}}%
\pgfpathlineto{\pgfqpoint{2.203827in}{1.337053in}}%
\pgfpathclose%
\pgfusepath{stroke}%
\end{pgfscope}%
\begin{pgfscope}%
\pgfpathrectangle{\pgfqpoint{0.494722in}{0.437222in}}{\pgfqpoint{6.275590in}{5.159444in}}%
\pgfusepath{clip}%
\pgfsetbuttcap%
\pgfsetroundjoin%
\pgfsetlinewidth{1.003750pt}%
\definecolor{currentstroke}{rgb}{0.827451,0.827451,0.827451}%
\pgfsetstrokecolor{currentstroke}%
\pgfsetstrokeopacity{0.800000}%
\pgfsetdash{}{0pt}%
\pgfpathmoveto{\pgfqpoint{1.143945in}{2.535677in}}%
\pgfpathcurveto{\pgfqpoint{1.154995in}{2.535677in}}{\pgfqpoint{1.165594in}{2.540067in}}{\pgfqpoint{1.173408in}{2.547881in}}%
\pgfpathcurveto{\pgfqpoint{1.181222in}{2.555695in}}{\pgfqpoint{1.185612in}{2.566294in}}{\pgfqpoint{1.185612in}{2.577344in}}%
\pgfpathcurveto{\pgfqpoint{1.185612in}{2.588394in}}{\pgfqpoint{1.181222in}{2.598993in}}{\pgfqpoint{1.173408in}{2.606807in}}%
\pgfpathcurveto{\pgfqpoint{1.165594in}{2.614620in}}{\pgfqpoint{1.154995in}{2.619011in}}{\pgfqpoint{1.143945in}{2.619011in}}%
\pgfpathcurveto{\pgfqpoint{1.132895in}{2.619011in}}{\pgfqpoint{1.122296in}{2.614620in}}{\pgfqpoint{1.114482in}{2.606807in}}%
\pgfpathcurveto{\pgfqpoint{1.106669in}{2.598993in}}{\pgfqpoint{1.102279in}{2.588394in}}{\pgfqpoint{1.102279in}{2.577344in}}%
\pgfpathcurveto{\pgfqpoint{1.102279in}{2.566294in}}{\pgfqpoint{1.106669in}{2.555695in}}{\pgfqpoint{1.114482in}{2.547881in}}%
\pgfpathcurveto{\pgfqpoint{1.122296in}{2.540067in}}{\pgfqpoint{1.132895in}{2.535677in}}{\pgfqpoint{1.143945in}{2.535677in}}%
\pgfpathlineto{\pgfqpoint{1.143945in}{2.535677in}}%
\pgfpathclose%
\pgfusepath{stroke}%
\end{pgfscope}%
\begin{pgfscope}%
\pgfpathrectangle{\pgfqpoint{0.494722in}{0.437222in}}{\pgfqpoint{6.275590in}{5.159444in}}%
\pgfusepath{clip}%
\pgfsetbuttcap%
\pgfsetroundjoin%
\pgfsetlinewidth{1.003750pt}%
\definecolor{currentstroke}{rgb}{0.827451,0.827451,0.827451}%
\pgfsetstrokecolor{currentstroke}%
\pgfsetstrokeopacity{0.800000}%
\pgfsetdash{}{0pt}%
\pgfpathmoveto{\pgfqpoint{0.600098in}{3.692599in}}%
\pgfpathcurveto{\pgfqpoint{0.611148in}{3.692599in}}{\pgfqpoint{0.621747in}{3.696989in}}{\pgfqpoint{0.629561in}{3.704803in}}%
\pgfpathcurveto{\pgfqpoint{0.637375in}{3.712616in}}{\pgfqpoint{0.641765in}{3.723215in}}{\pgfqpoint{0.641765in}{3.734266in}}%
\pgfpathcurveto{\pgfqpoint{0.641765in}{3.745316in}}{\pgfqpoint{0.637375in}{3.755915in}}{\pgfqpoint{0.629561in}{3.763728in}}%
\pgfpathcurveto{\pgfqpoint{0.621747in}{3.771542in}}{\pgfqpoint{0.611148in}{3.775932in}}{\pgfqpoint{0.600098in}{3.775932in}}%
\pgfpathcurveto{\pgfqpoint{0.589048in}{3.775932in}}{\pgfqpoint{0.578449in}{3.771542in}}{\pgfqpoint{0.570636in}{3.763728in}}%
\pgfpathcurveto{\pgfqpoint{0.562822in}{3.755915in}}{\pgfqpoint{0.558432in}{3.745316in}}{\pgfqpoint{0.558432in}{3.734266in}}%
\pgfpathcurveto{\pgfqpoint{0.558432in}{3.723215in}}{\pgfqpoint{0.562822in}{3.712616in}}{\pgfqpoint{0.570636in}{3.704803in}}%
\pgfpathcurveto{\pgfqpoint{0.578449in}{3.696989in}}{\pgfqpoint{0.589048in}{3.692599in}}{\pgfqpoint{0.600098in}{3.692599in}}%
\pgfpathlineto{\pgfqpoint{0.600098in}{3.692599in}}%
\pgfpathclose%
\pgfusepath{stroke}%
\end{pgfscope}%
\begin{pgfscope}%
\pgfpathrectangle{\pgfqpoint{0.494722in}{0.437222in}}{\pgfqpoint{6.275590in}{5.159444in}}%
\pgfusepath{clip}%
\pgfsetbuttcap%
\pgfsetroundjoin%
\pgfsetlinewidth{1.003750pt}%
\definecolor{currentstroke}{rgb}{0.827451,0.827451,0.827451}%
\pgfsetstrokecolor{currentstroke}%
\pgfsetstrokeopacity{0.800000}%
\pgfsetdash{}{0pt}%
\pgfpathmoveto{\pgfqpoint{0.744905in}{3.205367in}}%
\pgfpathcurveto{\pgfqpoint{0.755955in}{3.205367in}}{\pgfqpoint{0.766554in}{3.209757in}}{\pgfqpoint{0.774368in}{3.217571in}}%
\pgfpathcurveto{\pgfqpoint{0.782181in}{3.225384in}}{\pgfqpoint{0.786571in}{3.235983in}}{\pgfqpoint{0.786571in}{3.247034in}}%
\pgfpathcurveto{\pgfqpoint{0.786571in}{3.258084in}}{\pgfqpoint{0.782181in}{3.268683in}}{\pgfqpoint{0.774368in}{3.276496in}}%
\pgfpathcurveto{\pgfqpoint{0.766554in}{3.284310in}}{\pgfqpoint{0.755955in}{3.288700in}}{\pgfqpoint{0.744905in}{3.288700in}}%
\pgfpathcurveto{\pgfqpoint{0.733855in}{3.288700in}}{\pgfqpoint{0.723256in}{3.284310in}}{\pgfqpoint{0.715442in}{3.276496in}}%
\pgfpathcurveto{\pgfqpoint{0.707628in}{3.268683in}}{\pgfqpoint{0.703238in}{3.258084in}}{\pgfqpoint{0.703238in}{3.247034in}}%
\pgfpathcurveto{\pgfqpoint{0.703238in}{3.235983in}}{\pgfqpoint{0.707628in}{3.225384in}}{\pgfqpoint{0.715442in}{3.217571in}}%
\pgfpathcurveto{\pgfqpoint{0.723256in}{3.209757in}}{\pgfqpoint{0.733855in}{3.205367in}}{\pgfqpoint{0.744905in}{3.205367in}}%
\pgfpathlineto{\pgfqpoint{0.744905in}{3.205367in}}%
\pgfpathclose%
\pgfusepath{stroke}%
\end{pgfscope}%
\begin{pgfscope}%
\pgfpathrectangle{\pgfqpoint{0.494722in}{0.437222in}}{\pgfqpoint{6.275590in}{5.159444in}}%
\pgfusepath{clip}%
\pgfsetbuttcap%
\pgfsetroundjoin%
\pgfsetlinewidth{1.003750pt}%
\definecolor{currentstroke}{rgb}{0.827451,0.827451,0.827451}%
\pgfsetstrokecolor{currentstroke}%
\pgfsetstrokeopacity{0.800000}%
\pgfsetdash{}{0pt}%
\pgfpathmoveto{\pgfqpoint{3.754586in}{0.666564in}}%
\pgfpathcurveto{\pgfqpoint{3.765636in}{0.666564in}}{\pgfqpoint{3.776235in}{0.670954in}}{\pgfqpoint{3.784048in}{0.678767in}}%
\pgfpathcurveto{\pgfqpoint{3.791862in}{0.686581in}}{\pgfqpoint{3.796252in}{0.697180in}}{\pgfqpoint{3.796252in}{0.708230in}}%
\pgfpathcurveto{\pgfqpoint{3.796252in}{0.719280in}}{\pgfqpoint{3.791862in}{0.729879in}}{\pgfqpoint{3.784048in}{0.737693in}}%
\pgfpathcurveto{\pgfqpoint{3.776235in}{0.745507in}}{\pgfqpoint{3.765636in}{0.749897in}}{\pgfqpoint{3.754586in}{0.749897in}}%
\pgfpathcurveto{\pgfqpoint{3.743536in}{0.749897in}}{\pgfqpoint{3.732937in}{0.745507in}}{\pgfqpoint{3.725123in}{0.737693in}}%
\pgfpathcurveto{\pgfqpoint{3.717309in}{0.729879in}}{\pgfqpoint{3.712919in}{0.719280in}}{\pgfqpoint{3.712919in}{0.708230in}}%
\pgfpathcurveto{\pgfqpoint{3.712919in}{0.697180in}}{\pgfqpoint{3.717309in}{0.686581in}}{\pgfqpoint{3.725123in}{0.678767in}}%
\pgfpathcurveto{\pgfqpoint{3.732937in}{0.670954in}}{\pgfqpoint{3.743536in}{0.666564in}}{\pgfqpoint{3.754586in}{0.666564in}}%
\pgfpathlineto{\pgfqpoint{3.754586in}{0.666564in}}%
\pgfpathclose%
\pgfusepath{stroke}%
\end{pgfscope}%
\begin{pgfscope}%
\pgfpathrectangle{\pgfqpoint{0.494722in}{0.437222in}}{\pgfqpoint{6.275590in}{5.159444in}}%
\pgfusepath{clip}%
\pgfsetbuttcap%
\pgfsetroundjoin%
\pgfsetlinewidth{1.003750pt}%
\definecolor{currentstroke}{rgb}{0.827451,0.827451,0.827451}%
\pgfsetstrokecolor{currentstroke}%
\pgfsetstrokeopacity{0.800000}%
\pgfsetdash{}{0pt}%
\pgfpathmoveto{\pgfqpoint{1.455153in}{2.010549in}}%
\pgfpathcurveto{\pgfqpoint{1.466203in}{2.010549in}}{\pgfqpoint{1.476802in}{2.014939in}}{\pgfqpoint{1.484615in}{2.022752in}}%
\pgfpathcurveto{\pgfqpoint{1.492429in}{2.030566in}}{\pgfqpoint{1.496819in}{2.041165in}}{\pgfqpoint{1.496819in}{2.052215in}}%
\pgfpathcurveto{\pgfqpoint{1.496819in}{2.063265in}}{\pgfqpoint{1.492429in}{2.073864in}}{\pgfqpoint{1.484615in}{2.081678in}}%
\pgfpathcurveto{\pgfqpoint{1.476802in}{2.089492in}}{\pgfqpoint{1.466203in}{2.093882in}}{\pgfqpoint{1.455153in}{2.093882in}}%
\pgfpathcurveto{\pgfqpoint{1.444103in}{2.093882in}}{\pgfqpoint{1.433504in}{2.089492in}}{\pgfqpoint{1.425690in}{2.081678in}}%
\pgfpathcurveto{\pgfqpoint{1.417876in}{2.073864in}}{\pgfqpoint{1.413486in}{2.063265in}}{\pgfqpoint{1.413486in}{2.052215in}}%
\pgfpathcurveto{\pgfqpoint{1.413486in}{2.041165in}}{\pgfqpoint{1.417876in}{2.030566in}}{\pgfqpoint{1.425690in}{2.022752in}}%
\pgfpathcurveto{\pgfqpoint{1.433504in}{2.014939in}}{\pgfqpoint{1.444103in}{2.010549in}}{\pgfqpoint{1.455153in}{2.010549in}}%
\pgfpathlineto{\pgfqpoint{1.455153in}{2.010549in}}%
\pgfpathclose%
\pgfusepath{stroke}%
\end{pgfscope}%
\begin{pgfscope}%
\pgfpathrectangle{\pgfqpoint{0.494722in}{0.437222in}}{\pgfqpoint{6.275590in}{5.159444in}}%
\pgfusepath{clip}%
\pgfsetbuttcap%
\pgfsetroundjoin%
\pgfsetlinewidth{1.003750pt}%
\definecolor{currentstroke}{rgb}{0.827451,0.827451,0.827451}%
\pgfsetstrokecolor{currentstroke}%
\pgfsetstrokeopacity{0.800000}%
\pgfsetdash{}{0pt}%
\pgfpathmoveto{\pgfqpoint{2.365416in}{1.256449in}}%
\pgfpathcurveto{\pgfqpoint{2.376466in}{1.256449in}}{\pgfqpoint{2.387065in}{1.260839in}}{\pgfqpoint{2.394879in}{1.268653in}}%
\pgfpathcurveto{\pgfqpoint{2.402693in}{1.276466in}}{\pgfqpoint{2.407083in}{1.287065in}}{\pgfqpoint{2.407083in}{1.298115in}}%
\pgfpathcurveto{\pgfqpoint{2.407083in}{1.309166in}}{\pgfqpoint{2.402693in}{1.319765in}}{\pgfqpoint{2.394879in}{1.327578in}}%
\pgfpathcurveto{\pgfqpoint{2.387065in}{1.335392in}}{\pgfqpoint{2.376466in}{1.339782in}}{\pgfqpoint{2.365416in}{1.339782in}}%
\pgfpathcurveto{\pgfqpoint{2.354366in}{1.339782in}}{\pgfqpoint{2.343767in}{1.335392in}}{\pgfqpoint{2.335953in}{1.327578in}}%
\pgfpathcurveto{\pgfqpoint{2.328140in}{1.319765in}}{\pgfqpoint{2.323750in}{1.309166in}}{\pgfqpoint{2.323750in}{1.298115in}}%
\pgfpathcurveto{\pgfqpoint{2.323750in}{1.287065in}}{\pgfqpoint{2.328140in}{1.276466in}}{\pgfqpoint{2.335953in}{1.268653in}}%
\pgfpathcurveto{\pgfqpoint{2.343767in}{1.260839in}}{\pgfqpoint{2.354366in}{1.256449in}}{\pgfqpoint{2.365416in}{1.256449in}}%
\pgfpathlineto{\pgfqpoint{2.365416in}{1.256449in}}%
\pgfpathclose%
\pgfusepath{stroke}%
\end{pgfscope}%
\begin{pgfscope}%
\pgfpathrectangle{\pgfqpoint{0.494722in}{0.437222in}}{\pgfqpoint{6.275590in}{5.159444in}}%
\pgfusepath{clip}%
\pgfsetbuttcap%
\pgfsetroundjoin%
\pgfsetlinewidth{1.003750pt}%
\definecolor{currentstroke}{rgb}{0.827451,0.827451,0.827451}%
\pgfsetstrokecolor{currentstroke}%
\pgfsetstrokeopacity{0.800000}%
\pgfsetdash{}{0pt}%
\pgfpathmoveto{\pgfqpoint{1.941684in}{1.555345in}}%
\pgfpathcurveto{\pgfqpoint{1.952734in}{1.555345in}}{\pgfqpoint{1.963333in}{1.559735in}}{\pgfqpoint{1.971146in}{1.567549in}}%
\pgfpathcurveto{\pgfqpoint{1.978960in}{1.575362in}}{\pgfqpoint{1.983350in}{1.585961in}}{\pgfqpoint{1.983350in}{1.597011in}}%
\pgfpathcurveto{\pgfqpoint{1.983350in}{1.608062in}}{\pgfqpoint{1.978960in}{1.618661in}}{\pgfqpoint{1.971146in}{1.626474in}}%
\pgfpathcurveto{\pgfqpoint{1.963333in}{1.634288in}}{\pgfqpoint{1.952734in}{1.638678in}}{\pgfqpoint{1.941684in}{1.638678in}}%
\pgfpathcurveto{\pgfqpoint{1.930633in}{1.638678in}}{\pgfqpoint{1.920034in}{1.634288in}}{\pgfqpoint{1.912221in}{1.626474in}}%
\pgfpathcurveto{\pgfqpoint{1.904407in}{1.618661in}}{\pgfqpoint{1.900017in}{1.608062in}}{\pgfqpoint{1.900017in}{1.597011in}}%
\pgfpathcurveto{\pgfqpoint{1.900017in}{1.585961in}}{\pgfqpoint{1.904407in}{1.575362in}}{\pgfqpoint{1.912221in}{1.567549in}}%
\pgfpathcurveto{\pgfqpoint{1.920034in}{1.559735in}}{\pgfqpoint{1.930633in}{1.555345in}}{\pgfqpoint{1.941684in}{1.555345in}}%
\pgfpathlineto{\pgfqpoint{1.941684in}{1.555345in}}%
\pgfpathclose%
\pgfusepath{stroke}%
\end{pgfscope}%
\begin{pgfscope}%
\pgfpathrectangle{\pgfqpoint{0.494722in}{0.437222in}}{\pgfqpoint{6.275590in}{5.159444in}}%
\pgfusepath{clip}%
\pgfsetbuttcap%
\pgfsetroundjoin%
\pgfsetlinewidth{1.003750pt}%
\definecolor{currentstroke}{rgb}{0.827451,0.827451,0.827451}%
\pgfsetstrokecolor{currentstroke}%
\pgfsetstrokeopacity{0.800000}%
\pgfsetdash{}{0pt}%
\pgfpathmoveto{\pgfqpoint{0.532177in}{4.076860in}}%
\pgfpathcurveto{\pgfqpoint{0.543227in}{4.076860in}}{\pgfqpoint{0.553826in}{4.081250in}}{\pgfqpoint{0.561639in}{4.089064in}}%
\pgfpathcurveto{\pgfqpoint{0.569453in}{4.096877in}}{\pgfqpoint{0.573843in}{4.107476in}}{\pgfqpoint{0.573843in}{4.118526in}}%
\pgfpathcurveto{\pgfqpoint{0.573843in}{4.129577in}}{\pgfqpoint{0.569453in}{4.140176in}}{\pgfqpoint{0.561639in}{4.147989in}}%
\pgfpathcurveto{\pgfqpoint{0.553826in}{4.155803in}}{\pgfqpoint{0.543227in}{4.160193in}}{\pgfqpoint{0.532177in}{4.160193in}}%
\pgfpathcurveto{\pgfqpoint{0.521127in}{4.160193in}}{\pgfqpoint{0.510528in}{4.155803in}}{\pgfqpoint{0.502714in}{4.147989in}}%
\pgfpathcurveto{\pgfqpoint{0.494900in}{4.140176in}}{\pgfqpoint{0.490510in}{4.129577in}}{\pgfqpoint{0.490510in}{4.118526in}}%
\pgfpathcurveto{\pgfqpoint{0.490510in}{4.107476in}}{\pgfqpoint{0.494900in}{4.096877in}}{\pgfqpoint{0.502714in}{4.089064in}}%
\pgfpathcurveto{\pgfqpoint{0.510528in}{4.081250in}}{\pgfqpoint{0.521127in}{4.076860in}}{\pgfqpoint{0.532177in}{4.076860in}}%
\pgfpathlineto{\pgfqpoint{0.532177in}{4.076860in}}%
\pgfpathclose%
\pgfusepath{stroke}%
\end{pgfscope}%
\begin{pgfscope}%
\pgfpathrectangle{\pgfqpoint{0.494722in}{0.437222in}}{\pgfqpoint{6.275590in}{5.159444in}}%
\pgfusepath{clip}%
\pgfsetbuttcap%
\pgfsetroundjoin%
\pgfsetlinewidth{1.003750pt}%
\definecolor{currentstroke}{rgb}{0.827451,0.827451,0.827451}%
\pgfsetstrokecolor{currentstroke}%
\pgfsetstrokeopacity{0.800000}%
\pgfsetdash{}{0pt}%
\pgfpathmoveto{\pgfqpoint{2.974947in}{0.945504in}}%
\pgfpathcurveto{\pgfqpoint{2.985997in}{0.945504in}}{\pgfqpoint{2.996596in}{0.949895in}}{\pgfqpoint{3.004410in}{0.957708in}}%
\pgfpathcurveto{\pgfqpoint{3.012223in}{0.965522in}}{\pgfqpoint{3.016614in}{0.976121in}}{\pgfqpoint{3.016614in}{0.987171in}}%
\pgfpathcurveto{\pgfqpoint{3.016614in}{0.998221in}}{\pgfqpoint{3.012223in}{1.008820in}}{\pgfqpoint{3.004410in}{1.016634in}}%
\pgfpathcurveto{\pgfqpoint{2.996596in}{1.024447in}}{\pgfqpoint{2.985997in}{1.028838in}}{\pgfqpoint{2.974947in}{1.028838in}}%
\pgfpathcurveto{\pgfqpoint{2.963897in}{1.028838in}}{\pgfqpoint{2.953298in}{1.024447in}}{\pgfqpoint{2.945484in}{1.016634in}}%
\pgfpathcurveto{\pgfqpoint{2.937670in}{1.008820in}}{\pgfqpoint{2.933280in}{0.998221in}}{\pgfqpoint{2.933280in}{0.987171in}}%
\pgfpathcurveto{\pgfqpoint{2.933280in}{0.976121in}}{\pgfqpoint{2.937670in}{0.965522in}}{\pgfqpoint{2.945484in}{0.957708in}}%
\pgfpathcurveto{\pgfqpoint{2.953298in}{0.949895in}}{\pgfqpoint{2.963897in}{0.945504in}}{\pgfqpoint{2.974947in}{0.945504in}}%
\pgfpathlineto{\pgfqpoint{2.974947in}{0.945504in}}%
\pgfpathclose%
\pgfusepath{stroke}%
\end{pgfscope}%
\begin{pgfscope}%
\pgfpathrectangle{\pgfqpoint{0.494722in}{0.437222in}}{\pgfqpoint{6.275590in}{5.159444in}}%
\pgfusepath{clip}%
\pgfsetbuttcap%
\pgfsetroundjoin%
\pgfsetlinewidth{1.003750pt}%
\definecolor{currentstroke}{rgb}{0.827451,0.827451,0.827451}%
\pgfsetstrokecolor{currentstroke}%
\pgfsetstrokeopacity{0.800000}%
\pgfsetdash{}{0pt}%
\pgfpathmoveto{\pgfqpoint{4.226627in}{0.519229in}}%
\pgfpathcurveto{\pgfqpoint{4.237677in}{0.519229in}}{\pgfqpoint{4.248276in}{0.523619in}}{\pgfqpoint{4.256090in}{0.531433in}}%
\pgfpathcurveto{\pgfqpoint{4.263903in}{0.539247in}}{\pgfqpoint{4.268294in}{0.549846in}}{\pgfqpoint{4.268294in}{0.560896in}}%
\pgfpathcurveto{\pgfqpoint{4.268294in}{0.571946in}}{\pgfqpoint{4.263903in}{0.582545in}}{\pgfqpoint{4.256090in}{0.590358in}}%
\pgfpathcurveto{\pgfqpoint{4.248276in}{0.598172in}}{\pgfqpoint{4.237677in}{0.602562in}}{\pgfqpoint{4.226627in}{0.602562in}}%
\pgfpathcurveto{\pgfqpoint{4.215577in}{0.602562in}}{\pgfqpoint{4.204978in}{0.598172in}}{\pgfqpoint{4.197164in}{0.590358in}}%
\pgfpathcurveto{\pgfqpoint{4.189350in}{0.582545in}}{\pgfqpoint{4.184960in}{0.571946in}}{\pgfqpoint{4.184960in}{0.560896in}}%
\pgfpathcurveto{\pgfqpoint{4.184960in}{0.549846in}}{\pgfqpoint{4.189350in}{0.539247in}}{\pgfqpoint{4.197164in}{0.531433in}}%
\pgfpathcurveto{\pgfqpoint{4.204978in}{0.523619in}}{\pgfqpoint{4.215577in}{0.519229in}}{\pgfqpoint{4.226627in}{0.519229in}}%
\pgfpathlineto{\pgfqpoint{4.226627in}{0.519229in}}%
\pgfpathclose%
\pgfusepath{stroke}%
\end{pgfscope}%
\begin{pgfscope}%
\pgfpathrectangle{\pgfqpoint{0.494722in}{0.437222in}}{\pgfqpoint{6.275590in}{5.159444in}}%
\pgfusepath{clip}%
\pgfsetbuttcap%
\pgfsetroundjoin%
\pgfsetlinewidth{1.003750pt}%
\definecolor{currentstroke}{rgb}{0.827451,0.827451,0.827451}%
\pgfsetstrokecolor{currentstroke}%
\pgfsetstrokeopacity{0.800000}%
\pgfsetdash{}{0pt}%
\pgfpathmoveto{\pgfqpoint{0.536273in}{3.987890in}}%
\pgfpathcurveto{\pgfqpoint{0.547324in}{3.987890in}}{\pgfqpoint{0.557923in}{3.992280in}}{\pgfqpoint{0.565736in}{4.000093in}}%
\pgfpathcurveto{\pgfqpoint{0.573550in}{4.007907in}}{\pgfqpoint{0.577940in}{4.018506in}}{\pgfqpoint{0.577940in}{4.029556in}}%
\pgfpathcurveto{\pgfqpoint{0.577940in}{4.040606in}}{\pgfqpoint{0.573550in}{4.051205in}}{\pgfqpoint{0.565736in}{4.059019in}}%
\pgfpathcurveto{\pgfqpoint{0.557923in}{4.066833in}}{\pgfqpoint{0.547324in}{4.071223in}}{\pgfqpoint{0.536273in}{4.071223in}}%
\pgfpathcurveto{\pgfqpoint{0.525223in}{4.071223in}}{\pgfqpoint{0.514624in}{4.066833in}}{\pgfqpoint{0.506811in}{4.059019in}}%
\pgfpathcurveto{\pgfqpoint{0.498997in}{4.051205in}}{\pgfqpoint{0.494607in}{4.040606in}}{\pgfqpoint{0.494607in}{4.029556in}}%
\pgfpathcurveto{\pgfqpoint{0.494607in}{4.018506in}}{\pgfqpoint{0.498997in}{4.007907in}}{\pgfqpoint{0.506811in}{4.000093in}}%
\pgfpathcurveto{\pgfqpoint{0.514624in}{3.992280in}}{\pgfqpoint{0.525223in}{3.987890in}}{\pgfqpoint{0.536273in}{3.987890in}}%
\pgfpathlineto{\pgfqpoint{0.536273in}{3.987890in}}%
\pgfpathclose%
\pgfusepath{stroke}%
\end{pgfscope}%
\begin{pgfscope}%
\pgfpathrectangle{\pgfqpoint{0.494722in}{0.437222in}}{\pgfqpoint{6.275590in}{5.159444in}}%
\pgfusepath{clip}%
\pgfsetbuttcap%
\pgfsetroundjoin%
\pgfsetlinewidth{1.003750pt}%
\definecolor{currentstroke}{rgb}{0.827451,0.827451,0.827451}%
\pgfsetstrokecolor{currentstroke}%
\pgfsetstrokeopacity{0.800000}%
\pgfsetdash{}{0pt}%
\pgfpathmoveto{\pgfqpoint{2.589198in}{1.110833in}}%
\pgfpathcurveto{\pgfqpoint{2.600248in}{1.110833in}}{\pgfqpoint{2.610847in}{1.115224in}}{\pgfqpoint{2.618661in}{1.123037in}}%
\pgfpathcurveto{\pgfqpoint{2.626474in}{1.130851in}}{\pgfqpoint{2.630865in}{1.141450in}}{\pgfqpoint{2.630865in}{1.152500in}}%
\pgfpathcurveto{\pgfqpoint{2.630865in}{1.163550in}}{\pgfqpoint{2.626474in}{1.174149in}}{\pgfqpoint{2.618661in}{1.181963in}}%
\pgfpathcurveto{\pgfqpoint{2.610847in}{1.189777in}}{\pgfqpoint{2.600248in}{1.194167in}}{\pgfqpoint{2.589198in}{1.194167in}}%
\pgfpathcurveto{\pgfqpoint{2.578148in}{1.194167in}}{\pgfqpoint{2.567549in}{1.189777in}}{\pgfqpoint{2.559735in}{1.181963in}}%
\pgfpathcurveto{\pgfqpoint{2.551922in}{1.174149in}}{\pgfqpoint{2.547531in}{1.163550in}}{\pgfqpoint{2.547531in}{1.152500in}}%
\pgfpathcurveto{\pgfqpoint{2.547531in}{1.141450in}}{\pgfqpoint{2.551922in}{1.130851in}}{\pgfqpoint{2.559735in}{1.123037in}}%
\pgfpathcurveto{\pgfqpoint{2.567549in}{1.115224in}}{\pgfqpoint{2.578148in}{1.110833in}}{\pgfqpoint{2.589198in}{1.110833in}}%
\pgfpathlineto{\pgfqpoint{2.589198in}{1.110833in}}%
\pgfpathclose%
\pgfusepath{stroke}%
\end{pgfscope}%
\begin{pgfscope}%
\pgfpathrectangle{\pgfqpoint{0.494722in}{0.437222in}}{\pgfqpoint{6.275590in}{5.159444in}}%
\pgfusepath{clip}%
\pgfsetbuttcap%
\pgfsetroundjoin%
\pgfsetlinewidth{1.003750pt}%
\definecolor{currentstroke}{rgb}{0.827451,0.827451,0.827451}%
\pgfsetstrokecolor{currentstroke}%
\pgfsetstrokeopacity{0.800000}%
\pgfsetdash{}{0pt}%
\pgfpathmoveto{\pgfqpoint{3.753908in}{0.666834in}}%
\pgfpathcurveto{\pgfqpoint{3.764958in}{0.666834in}}{\pgfqpoint{3.775557in}{0.671224in}}{\pgfqpoint{3.783371in}{0.679038in}}%
\pgfpathcurveto{\pgfqpoint{3.791185in}{0.686852in}}{\pgfqpoint{3.795575in}{0.697451in}}{\pgfqpoint{3.795575in}{0.708501in}}%
\pgfpathcurveto{\pgfqpoint{3.795575in}{0.719551in}}{\pgfqpoint{3.791185in}{0.730150in}}{\pgfqpoint{3.783371in}{0.737964in}}%
\pgfpathcurveto{\pgfqpoint{3.775557in}{0.745777in}}{\pgfqpoint{3.764958in}{0.750167in}}{\pgfqpoint{3.753908in}{0.750167in}}%
\pgfpathcurveto{\pgfqpoint{3.742858in}{0.750167in}}{\pgfqpoint{3.732259in}{0.745777in}}{\pgfqpoint{3.724445in}{0.737964in}}%
\pgfpathcurveto{\pgfqpoint{3.716632in}{0.730150in}}{\pgfqpoint{3.712242in}{0.719551in}}{\pgfqpoint{3.712242in}{0.708501in}}%
\pgfpathcurveto{\pgfqpoint{3.712242in}{0.697451in}}{\pgfqpoint{3.716632in}{0.686852in}}{\pgfqpoint{3.724445in}{0.679038in}}%
\pgfpathcurveto{\pgfqpoint{3.732259in}{0.671224in}}{\pgfqpoint{3.742858in}{0.666834in}}{\pgfqpoint{3.753908in}{0.666834in}}%
\pgfpathlineto{\pgfqpoint{3.753908in}{0.666834in}}%
\pgfpathclose%
\pgfusepath{stroke}%
\end{pgfscope}%
\begin{pgfscope}%
\pgfpathrectangle{\pgfqpoint{0.494722in}{0.437222in}}{\pgfqpoint{6.275590in}{5.159444in}}%
\pgfusepath{clip}%
\pgfsetbuttcap%
\pgfsetroundjoin%
\pgfsetlinewidth{1.003750pt}%
\definecolor{currentstroke}{rgb}{0.827451,0.827451,0.827451}%
\pgfsetstrokecolor{currentstroke}%
\pgfsetstrokeopacity{0.800000}%
\pgfsetdash{}{0pt}%
\pgfpathmoveto{\pgfqpoint{2.224297in}{1.325142in}}%
\pgfpathcurveto{\pgfqpoint{2.235347in}{1.325142in}}{\pgfqpoint{2.245946in}{1.329532in}}{\pgfqpoint{2.253760in}{1.337346in}}%
\pgfpathcurveto{\pgfqpoint{2.261573in}{1.345159in}}{\pgfqpoint{2.265964in}{1.355758in}}{\pgfqpoint{2.265964in}{1.366809in}}%
\pgfpathcurveto{\pgfqpoint{2.265964in}{1.377859in}}{\pgfqpoint{2.261573in}{1.388458in}}{\pgfqpoint{2.253760in}{1.396271in}}%
\pgfpathcurveto{\pgfqpoint{2.245946in}{1.404085in}}{\pgfqpoint{2.235347in}{1.408475in}}{\pgfqpoint{2.224297in}{1.408475in}}%
\pgfpathcurveto{\pgfqpoint{2.213247in}{1.408475in}}{\pgfqpoint{2.202648in}{1.404085in}}{\pgfqpoint{2.194834in}{1.396271in}}%
\pgfpathcurveto{\pgfqpoint{2.187020in}{1.388458in}}{\pgfqpoint{2.182630in}{1.377859in}}{\pgfqpoint{2.182630in}{1.366809in}}%
\pgfpathcurveto{\pgfqpoint{2.182630in}{1.355758in}}{\pgfqpoint{2.187020in}{1.345159in}}{\pgfqpoint{2.194834in}{1.337346in}}%
\pgfpathcurveto{\pgfqpoint{2.202648in}{1.329532in}}{\pgfqpoint{2.213247in}{1.325142in}}{\pgfqpoint{2.224297in}{1.325142in}}%
\pgfpathlineto{\pgfqpoint{2.224297in}{1.325142in}}%
\pgfpathclose%
\pgfusepath{stroke}%
\end{pgfscope}%
\begin{pgfscope}%
\pgfpathrectangle{\pgfqpoint{0.494722in}{0.437222in}}{\pgfqpoint{6.275590in}{5.159444in}}%
\pgfusepath{clip}%
\pgfsetbuttcap%
\pgfsetroundjoin%
\pgfsetlinewidth{1.003750pt}%
\definecolor{currentstroke}{rgb}{0.827451,0.827451,0.827451}%
\pgfsetstrokecolor{currentstroke}%
\pgfsetstrokeopacity{0.800000}%
\pgfsetdash{}{0pt}%
\pgfpathmoveto{\pgfqpoint{5.060352in}{0.429754in}}%
\pgfpathcurveto{\pgfqpoint{5.071402in}{0.429754in}}{\pgfqpoint{5.082001in}{0.434145in}}{\pgfqpoint{5.089815in}{0.441958in}}%
\pgfpathcurveto{\pgfqpoint{5.097629in}{0.449772in}}{\pgfqpoint{5.102019in}{0.460371in}}{\pgfqpoint{5.102019in}{0.471421in}}%
\pgfpathcurveto{\pgfqpoint{5.102019in}{0.482471in}}{\pgfqpoint{5.097629in}{0.493070in}}{\pgfqpoint{5.089815in}{0.500884in}}%
\pgfpathcurveto{\pgfqpoint{5.082001in}{0.508697in}}{\pgfqpoint{5.071402in}{0.513088in}}{\pgfqpoint{5.060352in}{0.513088in}}%
\pgfpathcurveto{\pgfqpoint{5.049302in}{0.513088in}}{\pgfqpoint{5.038703in}{0.508697in}}{\pgfqpoint{5.030890in}{0.500884in}}%
\pgfpathcurveto{\pgfqpoint{5.023076in}{0.493070in}}{\pgfqpoint{5.018686in}{0.482471in}}{\pgfqpoint{5.018686in}{0.471421in}}%
\pgfpathcurveto{\pgfqpoint{5.018686in}{0.460371in}}{\pgfqpoint{5.023076in}{0.449772in}}{\pgfqpoint{5.030890in}{0.441958in}}%
\pgfpathcurveto{\pgfqpoint{5.038703in}{0.434145in}}{\pgfqpoint{5.049302in}{0.429754in}}{\pgfqpoint{5.060352in}{0.429754in}}%
\pgfpathlineto{\pgfqpoint{5.060352in}{0.429754in}}%
\pgfpathclose%
\pgfusepath{stroke}%
\end{pgfscope}%
\begin{pgfscope}%
\pgfpathrectangle{\pgfqpoint{0.494722in}{0.437222in}}{\pgfqpoint{6.275590in}{5.159444in}}%
\pgfusepath{clip}%
\pgfsetbuttcap%
\pgfsetroundjoin%
\pgfsetlinewidth{1.003750pt}%
\definecolor{currentstroke}{rgb}{0.827451,0.827451,0.827451}%
\pgfsetstrokecolor{currentstroke}%
\pgfsetstrokeopacity{0.800000}%
\pgfsetdash{}{0pt}%
\pgfpathmoveto{\pgfqpoint{1.215821in}{2.264081in}}%
\pgfpathcurveto{\pgfqpoint{1.226871in}{2.264081in}}{\pgfqpoint{1.237470in}{2.268471in}}{\pgfqpoint{1.245283in}{2.276285in}}%
\pgfpathcurveto{\pgfqpoint{1.253097in}{2.284099in}}{\pgfqpoint{1.257487in}{2.294698in}}{\pgfqpoint{1.257487in}{2.305748in}}%
\pgfpathcurveto{\pgfqpoint{1.257487in}{2.316798in}}{\pgfqpoint{1.253097in}{2.327397in}}{\pgfqpoint{1.245283in}{2.335210in}}%
\pgfpathcurveto{\pgfqpoint{1.237470in}{2.343024in}}{\pgfqpoint{1.226871in}{2.347414in}}{\pgfqpoint{1.215821in}{2.347414in}}%
\pgfpathcurveto{\pgfqpoint{1.204771in}{2.347414in}}{\pgfqpoint{1.194172in}{2.343024in}}{\pgfqpoint{1.186358in}{2.335210in}}%
\pgfpathcurveto{\pgfqpoint{1.178544in}{2.327397in}}{\pgfqpoint{1.174154in}{2.316798in}}{\pgfqpoint{1.174154in}{2.305748in}}%
\pgfpathcurveto{\pgfqpoint{1.174154in}{2.294698in}}{\pgfqpoint{1.178544in}{2.284099in}}{\pgfqpoint{1.186358in}{2.276285in}}%
\pgfpathcurveto{\pgfqpoint{1.194172in}{2.268471in}}{\pgfqpoint{1.204771in}{2.264081in}}{\pgfqpoint{1.215821in}{2.264081in}}%
\pgfpathlineto{\pgfqpoint{1.215821in}{2.264081in}}%
\pgfpathclose%
\pgfusepath{stroke}%
\end{pgfscope}%
\begin{pgfscope}%
\pgfpathrectangle{\pgfqpoint{0.494722in}{0.437222in}}{\pgfqpoint{6.275590in}{5.159444in}}%
\pgfusepath{clip}%
\pgfsetbuttcap%
\pgfsetroundjoin%
\pgfsetlinewidth{1.003750pt}%
\definecolor{currentstroke}{rgb}{0.827451,0.827451,0.827451}%
\pgfsetstrokecolor{currentstroke}%
\pgfsetstrokeopacity{0.800000}%
\pgfsetdash{}{0pt}%
\pgfpathmoveto{\pgfqpoint{2.124515in}{1.392642in}}%
\pgfpathcurveto{\pgfqpoint{2.135565in}{1.392642in}}{\pgfqpoint{2.146164in}{1.397032in}}{\pgfqpoint{2.153977in}{1.404846in}}%
\pgfpathcurveto{\pgfqpoint{2.161791in}{1.412659in}}{\pgfqpoint{2.166181in}{1.423258in}}{\pgfqpoint{2.166181in}{1.434309in}}%
\pgfpathcurveto{\pgfqpoint{2.166181in}{1.445359in}}{\pgfqpoint{2.161791in}{1.455958in}}{\pgfqpoint{2.153977in}{1.463771in}}%
\pgfpathcurveto{\pgfqpoint{2.146164in}{1.471585in}}{\pgfqpoint{2.135565in}{1.475975in}}{\pgfqpoint{2.124515in}{1.475975in}}%
\pgfpathcurveto{\pgfqpoint{2.113465in}{1.475975in}}{\pgfqpoint{2.102865in}{1.471585in}}{\pgfqpoint{2.095052in}{1.463771in}}%
\pgfpathcurveto{\pgfqpoint{2.087238in}{1.455958in}}{\pgfqpoint{2.082848in}{1.445359in}}{\pgfqpoint{2.082848in}{1.434309in}}%
\pgfpathcurveto{\pgfqpoint{2.082848in}{1.423258in}}{\pgfqpoint{2.087238in}{1.412659in}}{\pgfqpoint{2.095052in}{1.404846in}}%
\pgfpathcurveto{\pgfqpoint{2.102865in}{1.397032in}}{\pgfqpoint{2.113465in}{1.392642in}}{\pgfqpoint{2.124515in}{1.392642in}}%
\pgfpathlineto{\pgfqpoint{2.124515in}{1.392642in}}%
\pgfpathclose%
\pgfusepath{stroke}%
\end{pgfscope}%
\begin{pgfscope}%
\pgfpathrectangle{\pgfqpoint{0.494722in}{0.437222in}}{\pgfqpoint{6.275590in}{5.159444in}}%
\pgfusepath{clip}%
\pgfsetbuttcap%
\pgfsetroundjoin%
\pgfsetlinewidth{1.003750pt}%
\definecolor{currentstroke}{rgb}{0.827451,0.827451,0.827451}%
\pgfsetstrokecolor{currentstroke}%
\pgfsetstrokeopacity{0.800000}%
\pgfsetdash{}{0pt}%
\pgfpathmoveto{\pgfqpoint{0.627307in}{3.514990in}}%
\pgfpathcurveto{\pgfqpoint{0.638357in}{3.514990in}}{\pgfqpoint{0.648956in}{3.519381in}}{\pgfqpoint{0.656770in}{3.527194in}}%
\pgfpathcurveto{\pgfqpoint{0.664583in}{3.535008in}}{\pgfqpoint{0.668974in}{3.545607in}}{\pgfqpoint{0.668974in}{3.556657in}}%
\pgfpathcurveto{\pgfqpoint{0.668974in}{3.567707in}}{\pgfqpoint{0.664583in}{3.578306in}}{\pgfqpoint{0.656770in}{3.586120in}}%
\pgfpathcurveto{\pgfqpoint{0.648956in}{3.593933in}}{\pgfqpoint{0.638357in}{3.598324in}}{\pgfqpoint{0.627307in}{3.598324in}}%
\pgfpathcurveto{\pgfqpoint{0.616257in}{3.598324in}}{\pgfqpoint{0.605658in}{3.593933in}}{\pgfqpoint{0.597844in}{3.586120in}}%
\pgfpathcurveto{\pgfqpoint{0.590030in}{3.578306in}}{\pgfqpoint{0.585640in}{3.567707in}}{\pgfqpoint{0.585640in}{3.556657in}}%
\pgfpathcurveto{\pgfqpoint{0.585640in}{3.545607in}}{\pgfqpoint{0.590030in}{3.535008in}}{\pgfqpoint{0.597844in}{3.527194in}}%
\pgfpathcurveto{\pgfqpoint{0.605658in}{3.519381in}}{\pgfqpoint{0.616257in}{3.514990in}}{\pgfqpoint{0.627307in}{3.514990in}}%
\pgfpathlineto{\pgfqpoint{0.627307in}{3.514990in}}%
\pgfpathclose%
\pgfusepath{stroke}%
\end{pgfscope}%
\begin{pgfscope}%
\pgfpathrectangle{\pgfqpoint{0.494722in}{0.437222in}}{\pgfqpoint{6.275590in}{5.159444in}}%
\pgfusepath{clip}%
\pgfsetbuttcap%
\pgfsetroundjoin%
\pgfsetlinewidth{1.003750pt}%
\definecolor{currentstroke}{rgb}{0.827451,0.827451,0.827451}%
\pgfsetstrokecolor{currentstroke}%
\pgfsetstrokeopacity{0.800000}%
\pgfsetdash{}{0pt}%
\pgfpathmoveto{\pgfqpoint{1.700731in}{1.766489in}}%
\pgfpathcurveto{\pgfqpoint{1.711781in}{1.766489in}}{\pgfqpoint{1.722380in}{1.770879in}}{\pgfqpoint{1.730194in}{1.778693in}}%
\pgfpathcurveto{\pgfqpoint{1.738007in}{1.786506in}}{\pgfqpoint{1.742398in}{1.797105in}}{\pgfqpoint{1.742398in}{1.808156in}}%
\pgfpathcurveto{\pgfqpoint{1.742398in}{1.819206in}}{\pgfqpoint{1.738007in}{1.829805in}}{\pgfqpoint{1.730194in}{1.837618in}}%
\pgfpathcurveto{\pgfqpoint{1.722380in}{1.845432in}}{\pgfqpoint{1.711781in}{1.849822in}}{\pgfqpoint{1.700731in}{1.849822in}}%
\pgfpathcurveto{\pgfqpoint{1.689681in}{1.849822in}}{\pgfqpoint{1.679082in}{1.845432in}}{\pgfqpoint{1.671268in}{1.837618in}}%
\pgfpathcurveto{\pgfqpoint{1.663455in}{1.829805in}}{\pgfqpoint{1.659064in}{1.819206in}}{\pgfqpoint{1.659064in}{1.808156in}}%
\pgfpathcurveto{\pgfqpoint{1.659064in}{1.797105in}}{\pgfqpoint{1.663455in}{1.786506in}}{\pgfqpoint{1.671268in}{1.778693in}}%
\pgfpathcurveto{\pgfqpoint{1.679082in}{1.770879in}}{\pgfqpoint{1.689681in}{1.766489in}}{\pgfqpoint{1.700731in}{1.766489in}}%
\pgfpathlineto{\pgfqpoint{1.700731in}{1.766489in}}%
\pgfpathclose%
\pgfusepath{stroke}%
\end{pgfscope}%
\begin{pgfscope}%
\pgfpathrectangle{\pgfqpoint{0.494722in}{0.437222in}}{\pgfqpoint{6.275590in}{5.159444in}}%
\pgfusepath{clip}%
\pgfsetbuttcap%
\pgfsetroundjoin%
\pgfsetlinewidth{1.003750pt}%
\definecolor{currentstroke}{rgb}{0.827451,0.827451,0.827451}%
\pgfsetstrokecolor{currentstroke}%
\pgfsetstrokeopacity{0.800000}%
\pgfsetdash{}{0pt}%
\pgfpathmoveto{\pgfqpoint{1.964907in}{1.517037in}}%
\pgfpathcurveto{\pgfqpoint{1.975957in}{1.517037in}}{\pgfqpoint{1.986556in}{1.521427in}}{\pgfqpoint{1.994370in}{1.529240in}}%
\pgfpathcurveto{\pgfqpoint{2.002183in}{1.537054in}}{\pgfqpoint{2.006573in}{1.547653in}}{\pgfqpoint{2.006573in}{1.558703in}}%
\pgfpathcurveto{\pgfqpoint{2.006573in}{1.569753in}}{\pgfqpoint{2.002183in}{1.580352in}}{\pgfqpoint{1.994370in}{1.588166in}}%
\pgfpathcurveto{\pgfqpoint{1.986556in}{1.595980in}}{\pgfqpoint{1.975957in}{1.600370in}}{\pgfqpoint{1.964907in}{1.600370in}}%
\pgfpathcurveto{\pgfqpoint{1.953857in}{1.600370in}}{\pgfqpoint{1.943258in}{1.595980in}}{\pgfqpoint{1.935444in}{1.588166in}}%
\pgfpathcurveto{\pgfqpoint{1.927630in}{1.580352in}}{\pgfqpoint{1.923240in}{1.569753in}}{\pgfqpoint{1.923240in}{1.558703in}}%
\pgfpathcurveto{\pgfqpoint{1.923240in}{1.547653in}}{\pgfqpoint{1.927630in}{1.537054in}}{\pgfqpoint{1.935444in}{1.529240in}}%
\pgfpathcurveto{\pgfqpoint{1.943258in}{1.521427in}}{\pgfqpoint{1.953857in}{1.517037in}}{\pgfqpoint{1.964907in}{1.517037in}}%
\pgfpathlineto{\pgfqpoint{1.964907in}{1.517037in}}%
\pgfpathclose%
\pgfusepath{stroke}%
\end{pgfscope}%
\begin{pgfscope}%
\pgfpathrectangle{\pgfqpoint{0.494722in}{0.437222in}}{\pgfqpoint{6.275590in}{5.159444in}}%
\pgfusepath{clip}%
\pgfsetbuttcap%
\pgfsetroundjoin%
\pgfsetlinewidth{1.003750pt}%
\definecolor{currentstroke}{rgb}{0.827451,0.827451,0.827451}%
\pgfsetstrokecolor{currentstroke}%
\pgfsetstrokeopacity{0.800000}%
\pgfsetdash{}{0pt}%
\pgfpathmoveto{\pgfqpoint{1.091684in}{2.547519in}}%
\pgfpathcurveto{\pgfqpoint{1.102734in}{2.547519in}}{\pgfqpoint{1.113333in}{2.551909in}}{\pgfqpoint{1.121147in}{2.559722in}}%
\pgfpathcurveto{\pgfqpoint{1.128960in}{2.567536in}}{\pgfqpoint{1.133351in}{2.578135in}}{\pgfqpoint{1.133351in}{2.589185in}}%
\pgfpathcurveto{\pgfqpoint{1.133351in}{2.600235in}}{\pgfqpoint{1.128960in}{2.610834in}}{\pgfqpoint{1.121147in}{2.618648in}}%
\pgfpathcurveto{\pgfqpoint{1.113333in}{2.626462in}}{\pgfqpoint{1.102734in}{2.630852in}}{\pgfqpoint{1.091684in}{2.630852in}}%
\pgfpathcurveto{\pgfqpoint{1.080634in}{2.630852in}}{\pgfqpoint{1.070035in}{2.626462in}}{\pgfqpoint{1.062221in}{2.618648in}}%
\pgfpathcurveto{\pgfqpoint{1.054407in}{2.610834in}}{\pgfqpoint{1.050017in}{2.600235in}}{\pgfqpoint{1.050017in}{2.589185in}}%
\pgfpathcurveto{\pgfqpoint{1.050017in}{2.578135in}}{\pgfqpoint{1.054407in}{2.567536in}}{\pgfqpoint{1.062221in}{2.559722in}}%
\pgfpathcurveto{\pgfqpoint{1.070035in}{2.551909in}}{\pgfqpoint{1.080634in}{2.547519in}}{\pgfqpoint{1.091684in}{2.547519in}}%
\pgfpathlineto{\pgfqpoint{1.091684in}{2.547519in}}%
\pgfpathclose%
\pgfusepath{stroke}%
\end{pgfscope}%
\begin{pgfscope}%
\pgfpathrectangle{\pgfqpoint{0.494722in}{0.437222in}}{\pgfqpoint{6.275590in}{5.159444in}}%
\pgfusepath{clip}%
\pgfsetbuttcap%
\pgfsetroundjoin%
\pgfsetlinewidth{1.003750pt}%
\definecolor{currentstroke}{rgb}{0.827451,0.827451,0.827451}%
\pgfsetstrokecolor{currentstroke}%
\pgfsetstrokeopacity{0.800000}%
\pgfsetdash{}{0pt}%
\pgfpathmoveto{\pgfqpoint{0.525220in}{4.120020in}}%
\pgfpathcurveto{\pgfqpoint{0.536270in}{4.120020in}}{\pgfqpoint{0.546869in}{4.124411in}}{\pgfqpoint{0.554683in}{4.132224in}}%
\pgfpathcurveto{\pgfqpoint{0.562496in}{4.140038in}}{\pgfqpoint{0.566887in}{4.150637in}}{\pgfqpoint{0.566887in}{4.161687in}}%
\pgfpathcurveto{\pgfqpoint{0.566887in}{4.172737in}}{\pgfqpoint{0.562496in}{4.183336in}}{\pgfqpoint{0.554683in}{4.191150in}}%
\pgfpathcurveto{\pgfqpoint{0.546869in}{4.198963in}}{\pgfqpoint{0.536270in}{4.203354in}}{\pgfqpoint{0.525220in}{4.203354in}}%
\pgfpathcurveto{\pgfqpoint{0.514170in}{4.203354in}}{\pgfqpoint{0.503571in}{4.198963in}}{\pgfqpoint{0.495757in}{4.191150in}}%
\pgfpathcurveto{\pgfqpoint{0.487944in}{4.183336in}}{\pgfqpoint{0.483553in}{4.172737in}}{\pgfqpoint{0.483553in}{4.161687in}}%
\pgfpathcurveto{\pgfqpoint{0.483553in}{4.150637in}}{\pgfqpoint{0.487944in}{4.140038in}}{\pgfqpoint{0.495757in}{4.132224in}}%
\pgfpathcurveto{\pgfqpoint{0.503571in}{4.124411in}}{\pgfqpoint{0.514170in}{4.120020in}}{\pgfqpoint{0.525220in}{4.120020in}}%
\pgfpathlineto{\pgfqpoint{0.525220in}{4.120020in}}%
\pgfpathclose%
\pgfusepath{stroke}%
\end{pgfscope}%
\begin{pgfscope}%
\pgfpathrectangle{\pgfqpoint{0.494722in}{0.437222in}}{\pgfqpoint{6.275590in}{5.159444in}}%
\pgfusepath{clip}%
\pgfsetbuttcap%
\pgfsetroundjoin%
\pgfsetlinewidth{1.003750pt}%
\definecolor{currentstroke}{rgb}{0.827451,0.827451,0.827451}%
\pgfsetstrokecolor{currentstroke}%
\pgfsetstrokeopacity{0.800000}%
\pgfsetdash{}{0pt}%
\pgfpathmoveto{\pgfqpoint{1.347485in}{2.096739in}}%
\pgfpathcurveto{\pgfqpoint{1.358535in}{2.096739in}}{\pgfqpoint{1.369135in}{2.101129in}}{\pgfqpoint{1.376948in}{2.108943in}}%
\pgfpathcurveto{\pgfqpoint{1.384762in}{2.116757in}}{\pgfqpoint{1.389152in}{2.127356in}}{\pgfqpoint{1.389152in}{2.138406in}}%
\pgfpathcurveto{\pgfqpoint{1.389152in}{2.149456in}}{\pgfqpoint{1.384762in}{2.160055in}}{\pgfqpoint{1.376948in}{2.167869in}}%
\pgfpathcurveto{\pgfqpoint{1.369135in}{2.175682in}}{\pgfqpoint{1.358535in}{2.180072in}}{\pgfqpoint{1.347485in}{2.180072in}}%
\pgfpathcurveto{\pgfqpoint{1.336435in}{2.180072in}}{\pgfqpoint{1.325836in}{2.175682in}}{\pgfqpoint{1.318023in}{2.167869in}}%
\pgfpathcurveto{\pgfqpoint{1.310209in}{2.160055in}}{\pgfqpoint{1.305819in}{2.149456in}}{\pgfqpoint{1.305819in}{2.138406in}}%
\pgfpathcurveto{\pgfqpoint{1.305819in}{2.127356in}}{\pgfqpoint{1.310209in}{2.116757in}}{\pgfqpoint{1.318023in}{2.108943in}}%
\pgfpathcurveto{\pgfqpoint{1.325836in}{2.101129in}}{\pgfqpoint{1.336435in}{2.096739in}}{\pgfqpoint{1.347485in}{2.096739in}}%
\pgfpathlineto{\pgfqpoint{1.347485in}{2.096739in}}%
\pgfpathclose%
\pgfusepath{stroke}%
\end{pgfscope}%
\begin{pgfscope}%
\pgfpathrectangle{\pgfqpoint{0.494722in}{0.437222in}}{\pgfqpoint{6.275590in}{5.159444in}}%
\pgfusepath{clip}%
\pgfsetbuttcap%
\pgfsetroundjoin%
\pgfsetlinewidth{1.003750pt}%
\definecolor{currentstroke}{rgb}{0.827451,0.827451,0.827451}%
\pgfsetstrokecolor{currentstroke}%
\pgfsetstrokeopacity{0.800000}%
\pgfsetdash{}{0pt}%
\pgfpathmoveto{\pgfqpoint{2.552339in}{1.147505in}}%
\pgfpathcurveto{\pgfqpoint{2.563390in}{1.147505in}}{\pgfqpoint{2.573989in}{1.151895in}}{\pgfqpoint{2.581802in}{1.159709in}}%
\pgfpathcurveto{\pgfqpoint{2.589616in}{1.167522in}}{\pgfqpoint{2.594006in}{1.178121in}}{\pgfqpoint{2.594006in}{1.189172in}}%
\pgfpathcurveto{\pgfqpoint{2.594006in}{1.200222in}}{\pgfqpoint{2.589616in}{1.210821in}}{\pgfqpoint{2.581802in}{1.218634in}}%
\pgfpathcurveto{\pgfqpoint{2.573989in}{1.226448in}}{\pgfqpoint{2.563390in}{1.230838in}}{\pgfqpoint{2.552339in}{1.230838in}}%
\pgfpathcurveto{\pgfqpoint{2.541289in}{1.230838in}}{\pgfqpoint{2.530690in}{1.226448in}}{\pgfqpoint{2.522877in}{1.218634in}}%
\pgfpathcurveto{\pgfqpoint{2.515063in}{1.210821in}}{\pgfqpoint{2.510673in}{1.200222in}}{\pgfqpoint{2.510673in}{1.189172in}}%
\pgfpathcurveto{\pgfqpoint{2.510673in}{1.178121in}}{\pgfqpoint{2.515063in}{1.167522in}}{\pgfqpoint{2.522877in}{1.159709in}}%
\pgfpathcurveto{\pgfqpoint{2.530690in}{1.151895in}}{\pgfqpoint{2.541289in}{1.147505in}}{\pgfqpoint{2.552339in}{1.147505in}}%
\pgfpathlineto{\pgfqpoint{2.552339in}{1.147505in}}%
\pgfpathclose%
\pgfusepath{stroke}%
\end{pgfscope}%
\begin{pgfscope}%
\pgfpathrectangle{\pgfqpoint{0.494722in}{0.437222in}}{\pgfqpoint{6.275590in}{5.159444in}}%
\pgfusepath{clip}%
\pgfsetbuttcap%
\pgfsetroundjoin%
\pgfsetlinewidth{1.003750pt}%
\definecolor{currentstroke}{rgb}{0.827451,0.827451,0.827451}%
\pgfsetstrokecolor{currentstroke}%
\pgfsetstrokeopacity{0.800000}%
\pgfsetdash{}{0pt}%
\pgfpathmoveto{\pgfqpoint{4.229480in}{0.518110in}}%
\pgfpathcurveto{\pgfqpoint{4.240530in}{0.518110in}}{\pgfqpoint{4.251129in}{0.522500in}}{\pgfqpoint{4.258942in}{0.530314in}}%
\pgfpathcurveto{\pgfqpoint{4.266756in}{0.538127in}}{\pgfqpoint{4.271146in}{0.548726in}}{\pgfqpoint{4.271146in}{0.559776in}}%
\pgfpathcurveto{\pgfqpoint{4.271146in}{0.570827in}}{\pgfqpoint{4.266756in}{0.581426in}}{\pgfqpoint{4.258942in}{0.589239in}}%
\pgfpathcurveto{\pgfqpoint{4.251129in}{0.597053in}}{\pgfqpoint{4.240530in}{0.601443in}}{\pgfqpoint{4.229480in}{0.601443in}}%
\pgfpathcurveto{\pgfqpoint{4.218429in}{0.601443in}}{\pgfqpoint{4.207830in}{0.597053in}}{\pgfqpoint{4.200017in}{0.589239in}}%
\pgfpathcurveto{\pgfqpoint{4.192203in}{0.581426in}}{\pgfqpoint{4.187813in}{0.570827in}}{\pgfqpoint{4.187813in}{0.559776in}}%
\pgfpathcurveto{\pgfqpoint{4.187813in}{0.548726in}}{\pgfqpoint{4.192203in}{0.538127in}}{\pgfqpoint{4.200017in}{0.530314in}}%
\pgfpathcurveto{\pgfqpoint{4.207830in}{0.522500in}}{\pgfqpoint{4.218429in}{0.518110in}}{\pgfqpoint{4.229480in}{0.518110in}}%
\pgfpathlineto{\pgfqpoint{4.229480in}{0.518110in}}%
\pgfpathclose%
\pgfusepath{stroke}%
\end{pgfscope}%
\begin{pgfscope}%
\pgfpathrectangle{\pgfqpoint{0.494722in}{0.437222in}}{\pgfqpoint{6.275590in}{5.159444in}}%
\pgfusepath{clip}%
\pgfsetbuttcap%
\pgfsetroundjoin%
\pgfsetlinewidth{1.003750pt}%
\definecolor{currentstroke}{rgb}{0.827451,0.827451,0.827451}%
\pgfsetstrokecolor{currentstroke}%
\pgfsetstrokeopacity{0.800000}%
\pgfsetdash{}{0pt}%
\pgfpathmoveto{\pgfqpoint{1.135214in}{2.539256in}}%
\pgfpathcurveto{\pgfqpoint{1.146264in}{2.539256in}}{\pgfqpoint{1.156863in}{2.543646in}}{\pgfqpoint{1.164677in}{2.551460in}}%
\pgfpathcurveto{\pgfqpoint{1.172491in}{2.559274in}}{\pgfqpoint{1.176881in}{2.569873in}}{\pgfqpoint{1.176881in}{2.580923in}}%
\pgfpathcurveto{\pgfqpoint{1.176881in}{2.591973in}}{\pgfqpoint{1.172491in}{2.602572in}}{\pgfqpoint{1.164677in}{2.610386in}}%
\pgfpathcurveto{\pgfqpoint{1.156863in}{2.618199in}}{\pgfqpoint{1.146264in}{2.622590in}}{\pgfqpoint{1.135214in}{2.622590in}}%
\pgfpathcurveto{\pgfqpoint{1.124164in}{2.622590in}}{\pgfqpoint{1.113565in}{2.618199in}}{\pgfqpoint{1.105752in}{2.610386in}}%
\pgfpathcurveto{\pgfqpoint{1.097938in}{2.602572in}}{\pgfqpoint{1.093548in}{2.591973in}}{\pgfqpoint{1.093548in}{2.580923in}}%
\pgfpathcurveto{\pgfqpoint{1.093548in}{2.569873in}}{\pgfqpoint{1.097938in}{2.559274in}}{\pgfqpoint{1.105752in}{2.551460in}}%
\pgfpathcurveto{\pgfqpoint{1.113565in}{2.543646in}}{\pgfqpoint{1.124164in}{2.539256in}}{\pgfqpoint{1.135214in}{2.539256in}}%
\pgfpathlineto{\pgfqpoint{1.135214in}{2.539256in}}%
\pgfpathclose%
\pgfusepath{stroke}%
\end{pgfscope}%
\begin{pgfscope}%
\pgfpathrectangle{\pgfqpoint{0.494722in}{0.437222in}}{\pgfqpoint{6.275590in}{5.159444in}}%
\pgfusepath{clip}%
\pgfsetbuttcap%
\pgfsetroundjoin%
\pgfsetlinewidth{1.003750pt}%
\definecolor{currentstroke}{rgb}{0.827451,0.827451,0.827451}%
\pgfsetstrokecolor{currentstroke}%
\pgfsetstrokeopacity{0.800000}%
\pgfsetdash{}{0pt}%
\pgfpathmoveto{\pgfqpoint{0.498024in}{4.460098in}}%
\pgfpathcurveto{\pgfqpoint{0.509074in}{4.460098in}}{\pgfqpoint{0.519673in}{4.464488in}}{\pgfqpoint{0.527487in}{4.472302in}}%
\pgfpathcurveto{\pgfqpoint{0.535301in}{4.480116in}}{\pgfqpoint{0.539691in}{4.490715in}}{\pgfqpoint{0.539691in}{4.501765in}}%
\pgfpathcurveto{\pgfqpoint{0.539691in}{4.512815in}}{\pgfqpoint{0.535301in}{4.523414in}}{\pgfqpoint{0.527487in}{4.531228in}}%
\pgfpathcurveto{\pgfqpoint{0.519673in}{4.539041in}}{\pgfqpoint{0.509074in}{4.543432in}}{\pgfqpoint{0.498024in}{4.543432in}}%
\pgfpathcurveto{\pgfqpoint{0.486974in}{4.543432in}}{\pgfqpoint{0.476375in}{4.539041in}}{\pgfqpoint{0.468561in}{4.531228in}}%
\pgfpathcurveto{\pgfqpoint{0.460748in}{4.523414in}}{\pgfqpoint{0.456358in}{4.512815in}}{\pgfqpoint{0.456358in}{4.501765in}}%
\pgfpathcurveto{\pgfqpoint{0.456358in}{4.490715in}}{\pgfqpoint{0.460748in}{4.480116in}}{\pgfqpoint{0.468561in}{4.472302in}}%
\pgfpathcurveto{\pgfqpoint{0.476375in}{4.464488in}}{\pgfqpoint{0.486974in}{4.460098in}}{\pgfqpoint{0.498024in}{4.460098in}}%
\pgfpathlineto{\pgfqpoint{0.498024in}{4.460098in}}%
\pgfpathclose%
\pgfusepath{stroke}%
\end{pgfscope}%
\begin{pgfscope}%
\pgfpathrectangle{\pgfqpoint{0.494722in}{0.437222in}}{\pgfqpoint{6.275590in}{5.159444in}}%
\pgfusepath{clip}%
\pgfsetbuttcap%
\pgfsetroundjoin%
\pgfsetlinewidth{1.003750pt}%
\definecolor{currentstroke}{rgb}{0.827451,0.827451,0.827451}%
\pgfsetstrokecolor{currentstroke}%
\pgfsetstrokeopacity{0.800000}%
\pgfsetdash{}{0pt}%
\pgfpathmoveto{\pgfqpoint{1.026519in}{2.788625in}}%
\pgfpathcurveto{\pgfqpoint{1.037569in}{2.788625in}}{\pgfqpoint{1.048168in}{2.793015in}}{\pgfqpoint{1.055982in}{2.800829in}}%
\pgfpathcurveto{\pgfqpoint{1.063796in}{2.808642in}}{\pgfqpoint{1.068186in}{2.819241in}}{\pgfqpoint{1.068186in}{2.830291in}}%
\pgfpathcurveto{\pgfqpoint{1.068186in}{2.841341in}}{\pgfqpoint{1.063796in}{2.851940in}}{\pgfqpoint{1.055982in}{2.859754in}}%
\pgfpathcurveto{\pgfqpoint{1.048168in}{2.867568in}}{\pgfqpoint{1.037569in}{2.871958in}}{\pgfqpoint{1.026519in}{2.871958in}}%
\pgfpathcurveto{\pgfqpoint{1.015469in}{2.871958in}}{\pgfqpoint{1.004870in}{2.867568in}}{\pgfqpoint{0.997056in}{2.859754in}}%
\pgfpathcurveto{\pgfqpoint{0.989243in}{2.851940in}}{\pgfqpoint{0.984853in}{2.841341in}}{\pgfqpoint{0.984853in}{2.830291in}}%
\pgfpathcurveto{\pgfqpoint{0.984853in}{2.819241in}}{\pgfqpoint{0.989243in}{2.808642in}}{\pgfqpoint{0.997056in}{2.800829in}}%
\pgfpathcurveto{\pgfqpoint{1.004870in}{2.793015in}}{\pgfqpoint{1.015469in}{2.788625in}}{\pgfqpoint{1.026519in}{2.788625in}}%
\pgfpathlineto{\pgfqpoint{1.026519in}{2.788625in}}%
\pgfpathclose%
\pgfusepath{stroke}%
\end{pgfscope}%
\begin{pgfscope}%
\pgfpathrectangle{\pgfqpoint{0.494722in}{0.437222in}}{\pgfqpoint{6.275590in}{5.159444in}}%
\pgfusepath{clip}%
\pgfsetbuttcap%
\pgfsetroundjoin%
\pgfsetlinewidth{1.003750pt}%
\definecolor{currentstroke}{rgb}{0.827451,0.827451,0.827451}%
\pgfsetstrokecolor{currentstroke}%
\pgfsetstrokeopacity{0.800000}%
\pgfsetdash{}{0pt}%
\pgfpathmoveto{\pgfqpoint{3.807534in}{0.638716in}}%
\pgfpathcurveto{\pgfqpoint{3.818584in}{0.638716in}}{\pgfqpoint{3.829183in}{0.643106in}}{\pgfqpoint{3.836997in}{0.650919in}}%
\pgfpathcurveto{\pgfqpoint{3.844810in}{0.658733in}}{\pgfqpoint{3.849201in}{0.669332in}}{\pgfqpoint{3.849201in}{0.680382in}}%
\pgfpathcurveto{\pgfqpoint{3.849201in}{0.691432in}}{\pgfqpoint{3.844810in}{0.702031in}}{\pgfqpoint{3.836997in}{0.709845in}}%
\pgfpathcurveto{\pgfqpoint{3.829183in}{0.717659in}}{\pgfqpoint{3.818584in}{0.722049in}}{\pgfqpoint{3.807534in}{0.722049in}}%
\pgfpathcurveto{\pgfqpoint{3.796484in}{0.722049in}}{\pgfqpoint{3.785885in}{0.717659in}}{\pgfqpoint{3.778071in}{0.709845in}}%
\pgfpathcurveto{\pgfqpoint{3.770257in}{0.702031in}}{\pgfqpoint{3.765867in}{0.691432in}}{\pgfqpoint{3.765867in}{0.680382in}}%
\pgfpathcurveto{\pgfqpoint{3.765867in}{0.669332in}}{\pgfqpoint{3.770257in}{0.658733in}}{\pgfqpoint{3.778071in}{0.650919in}}%
\pgfpathcurveto{\pgfqpoint{3.785885in}{0.643106in}}{\pgfqpoint{3.796484in}{0.638716in}}{\pgfqpoint{3.807534in}{0.638716in}}%
\pgfpathlineto{\pgfqpoint{3.807534in}{0.638716in}}%
\pgfpathclose%
\pgfusepath{stroke}%
\end{pgfscope}%
\begin{pgfscope}%
\pgfpathrectangle{\pgfqpoint{0.494722in}{0.437222in}}{\pgfqpoint{6.275590in}{5.159444in}}%
\pgfusepath{clip}%
\pgfsetbuttcap%
\pgfsetroundjoin%
\pgfsetlinewidth{1.003750pt}%
\definecolor{currentstroke}{rgb}{0.827451,0.827451,0.827451}%
\pgfsetstrokecolor{currentstroke}%
\pgfsetstrokeopacity{0.800000}%
\pgfsetdash{}{0pt}%
\pgfpathmoveto{\pgfqpoint{0.711186in}{3.233934in}}%
\pgfpathcurveto{\pgfqpoint{0.722236in}{3.233934in}}{\pgfqpoint{0.732835in}{3.238324in}}{\pgfqpoint{0.740648in}{3.246138in}}%
\pgfpathcurveto{\pgfqpoint{0.748462in}{3.253952in}}{\pgfqpoint{0.752852in}{3.264551in}}{\pgfqpoint{0.752852in}{3.275601in}}%
\pgfpathcurveto{\pgfqpoint{0.752852in}{3.286651in}}{\pgfqpoint{0.748462in}{3.297250in}}{\pgfqpoint{0.740648in}{3.305064in}}%
\pgfpathcurveto{\pgfqpoint{0.732835in}{3.312877in}}{\pgfqpoint{0.722236in}{3.317267in}}{\pgfqpoint{0.711186in}{3.317267in}}%
\pgfpathcurveto{\pgfqpoint{0.700135in}{3.317267in}}{\pgfqpoint{0.689536in}{3.312877in}}{\pgfqpoint{0.681723in}{3.305064in}}%
\pgfpathcurveto{\pgfqpoint{0.673909in}{3.297250in}}{\pgfqpoint{0.669519in}{3.286651in}}{\pgfqpoint{0.669519in}{3.275601in}}%
\pgfpathcurveto{\pgfqpoint{0.669519in}{3.264551in}}{\pgfqpoint{0.673909in}{3.253952in}}{\pgfqpoint{0.681723in}{3.246138in}}%
\pgfpathcurveto{\pgfqpoint{0.689536in}{3.238324in}}{\pgfqpoint{0.700135in}{3.233934in}}{\pgfqpoint{0.711186in}{3.233934in}}%
\pgfpathlineto{\pgfqpoint{0.711186in}{3.233934in}}%
\pgfpathclose%
\pgfusepath{stroke}%
\end{pgfscope}%
\begin{pgfscope}%
\pgfpathrectangle{\pgfqpoint{0.494722in}{0.437222in}}{\pgfqpoint{6.275590in}{5.159444in}}%
\pgfusepath{clip}%
\pgfsetbuttcap%
\pgfsetroundjoin%
\pgfsetlinewidth{1.003750pt}%
\definecolor{currentstroke}{rgb}{0.827451,0.827451,0.827451}%
\pgfsetstrokecolor{currentstroke}%
\pgfsetstrokeopacity{0.800000}%
\pgfsetdash{}{0pt}%
\pgfpathmoveto{\pgfqpoint{4.583581in}{0.484214in}}%
\pgfpathcurveto{\pgfqpoint{4.594631in}{0.484214in}}{\pgfqpoint{4.605230in}{0.488604in}}{\pgfqpoint{4.613044in}{0.496418in}}%
\pgfpathcurveto{\pgfqpoint{4.620858in}{0.504232in}}{\pgfqpoint{4.625248in}{0.514831in}}{\pgfqpoint{4.625248in}{0.525881in}}%
\pgfpathcurveto{\pgfqpoint{4.625248in}{0.536931in}}{\pgfqpoint{4.620858in}{0.547530in}}{\pgfqpoint{4.613044in}{0.555344in}}%
\pgfpathcurveto{\pgfqpoint{4.605230in}{0.563157in}}{\pgfqpoint{4.594631in}{0.567548in}}{\pgfqpoint{4.583581in}{0.567548in}}%
\pgfpathcurveto{\pgfqpoint{4.572531in}{0.567548in}}{\pgfqpoint{4.561932in}{0.563157in}}{\pgfqpoint{4.554118in}{0.555344in}}%
\pgfpathcurveto{\pgfqpoint{4.546305in}{0.547530in}}{\pgfqpoint{4.541915in}{0.536931in}}{\pgfqpoint{4.541915in}{0.525881in}}%
\pgfpathcurveto{\pgfqpoint{4.541915in}{0.514831in}}{\pgfqpoint{4.546305in}{0.504232in}}{\pgfqpoint{4.554118in}{0.496418in}}%
\pgfpathcurveto{\pgfqpoint{4.561932in}{0.488604in}}{\pgfqpoint{4.572531in}{0.484214in}}{\pgfqpoint{4.583581in}{0.484214in}}%
\pgfpathlineto{\pgfqpoint{4.583581in}{0.484214in}}%
\pgfpathclose%
\pgfusepath{stroke}%
\end{pgfscope}%
\begin{pgfscope}%
\pgfpathrectangle{\pgfqpoint{0.494722in}{0.437222in}}{\pgfqpoint{6.275590in}{5.159444in}}%
\pgfusepath{clip}%
\pgfsetbuttcap%
\pgfsetroundjoin%
\pgfsetlinewidth{1.003750pt}%
\definecolor{currentstroke}{rgb}{0.827451,0.827451,0.827451}%
\pgfsetstrokecolor{currentstroke}%
\pgfsetstrokeopacity{0.800000}%
\pgfsetdash{}{0pt}%
\pgfpathmoveto{\pgfqpoint{2.659012in}{1.065276in}}%
\pgfpathcurveto{\pgfqpoint{2.670062in}{1.065276in}}{\pgfqpoint{2.680661in}{1.069667in}}{\pgfqpoint{2.688475in}{1.077480in}}%
\pgfpathcurveto{\pgfqpoint{2.696289in}{1.085294in}}{\pgfqpoint{2.700679in}{1.095893in}}{\pgfqpoint{2.700679in}{1.106943in}}%
\pgfpathcurveto{\pgfqpoint{2.700679in}{1.117993in}}{\pgfqpoint{2.696289in}{1.128592in}}{\pgfqpoint{2.688475in}{1.136406in}}%
\pgfpathcurveto{\pgfqpoint{2.680661in}{1.144219in}}{\pgfqpoint{2.670062in}{1.148610in}}{\pgfqpoint{2.659012in}{1.148610in}}%
\pgfpathcurveto{\pgfqpoint{2.647962in}{1.148610in}}{\pgfqpoint{2.637363in}{1.144219in}}{\pgfqpoint{2.629550in}{1.136406in}}%
\pgfpathcurveto{\pgfqpoint{2.621736in}{1.128592in}}{\pgfqpoint{2.617346in}{1.117993in}}{\pgfqpoint{2.617346in}{1.106943in}}%
\pgfpathcurveto{\pgfqpoint{2.617346in}{1.095893in}}{\pgfqpoint{2.621736in}{1.085294in}}{\pgfqpoint{2.629550in}{1.077480in}}%
\pgfpathcurveto{\pgfqpoint{2.637363in}{1.069667in}}{\pgfqpoint{2.647962in}{1.065276in}}{\pgfqpoint{2.659012in}{1.065276in}}%
\pgfpathlineto{\pgfqpoint{2.659012in}{1.065276in}}%
\pgfpathclose%
\pgfusepath{stroke}%
\end{pgfscope}%
\begin{pgfscope}%
\pgfpathrectangle{\pgfqpoint{0.494722in}{0.437222in}}{\pgfqpoint{6.275590in}{5.159444in}}%
\pgfusepath{clip}%
\pgfsetbuttcap%
\pgfsetroundjoin%
\pgfsetlinewidth{1.003750pt}%
\definecolor{currentstroke}{rgb}{0.827451,0.827451,0.827451}%
\pgfsetstrokecolor{currentstroke}%
\pgfsetstrokeopacity{0.800000}%
\pgfsetdash{}{0pt}%
\pgfpathmoveto{\pgfqpoint{3.150746in}{0.887921in}}%
\pgfpathcurveto{\pgfqpoint{3.161796in}{0.887921in}}{\pgfqpoint{3.172395in}{0.892311in}}{\pgfqpoint{3.180209in}{0.900125in}}%
\pgfpathcurveto{\pgfqpoint{3.188023in}{0.907938in}}{\pgfqpoint{3.192413in}{0.918537in}}{\pgfqpoint{3.192413in}{0.929588in}}%
\pgfpathcurveto{\pgfqpoint{3.192413in}{0.940638in}}{\pgfqpoint{3.188023in}{0.951237in}}{\pgfqpoint{3.180209in}{0.959050in}}%
\pgfpathcurveto{\pgfqpoint{3.172395in}{0.966864in}}{\pgfqpoint{3.161796in}{0.971254in}}{\pgfqpoint{3.150746in}{0.971254in}}%
\pgfpathcurveto{\pgfqpoint{3.139696in}{0.971254in}}{\pgfqpoint{3.129097in}{0.966864in}}{\pgfqpoint{3.121283in}{0.959050in}}%
\pgfpathcurveto{\pgfqpoint{3.113470in}{0.951237in}}{\pgfqpoint{3.109079in}{0.940638in}}{\pgfqpoint{3.109079in}{0.929588in}}%
\pgfpathcurveto{\pgfqpoint{3.109079in}{0.918537in}}{\pgfqpoint{3.113470in}{0.907938in}}{\pgfqpoint{3.121283in}{0.900125in}}%
\pgfpathcurveto{\pgfqpoint{3.129097in}{0.892311in}}{\pgfqpoint{3.139696in}{0.887921in}}{\pgfqpoint{3.150746in}{0.887921in}}%
\pgfpathlineto{\pgfqpoint{3.150746in}{0.887921in}}%
\pgfpathclose%
\pgfusepath{stroke}%
\end{pgfscope}%
\begin{pgfscope}%
\pgfpathrectangle{\pgfqpoint{0.494722in}{0.437222in}}{\pgfqpoint{6.275590in}{5.159444in}}%
\pgfusepath{clip}%
\pgfsetbuttcap%
\pgfsetroundjoin%
\pgfsetlinewidth{1.003750pt}%
\definecolor{currentstroke}{rgb}{0.827451,0.827451,0.827451}%
\pgfsetstrokecolor{currentstroke}%
\pgfsetstrokeopacity{0.800000}%
\pgfsetdash{}{0pt}%
\pgfpathmoveto{\pgfqpoint{1.476546in}{1.953872in}}%
\pgfpathcurveto{\pgfqpoint{1.487596in}{1.953872in}}{\pgfqpoint{1.498195in}{1.958263in}}{\pgfqpoint{1.506008in}{1.966076in}}%
\pgfpathcurveto{\pgfqpoint{1.513822in}{1.973890in}}{\pgfqpoint{1.518212in}{1.984489in}}{\pgfqpoint{1.518212in}{1.995539in}}%
\pgfpathcurveto{\pgfqpoint{1.518212in}{2.006589in}}{\pgfqpoint{1.513822in}{2.017188in}}{\pgfqpoint{1.506008in}{2.025002in}}%
\pgfpathcurveto{\pgfqpoint{1.498195in}{2.032815in}}{\pgfqpoint{1.487596in}{2.037206in}}{\pgfqpoint{1.476546in}{2.037206in}}%
\pgfpathcurveto{\pgfqpoint{1.465496in}{2.037206in}}{\pgfqpoint{1.454897in}{2.032815in}}{\pgfqpoint{1.447083in}{2.025002in}}%
\pgfpathcurveto{\pgfqpoint{1.439269in}{2.017188in}}{\pgfqpoint{1.434879in}{2.006589in}}{\pgfqpoint{1.434879in}{1.995539in}}%
\pgfpathcurveto{\pgfqpoint{1.434879in}{1.984489in}}{\pgfqpoint{1.439269in}{1.973890in}}{\pgfqpoint{1.447083in}{1.966076in}}%
\pgfpathcurveto{\pgfqpoint{1.454897in}{1.958263in}}{\pgfqpoint{1.465496in}{1.953872in}}{\pgfqpoint{1.476546in}{1.953872in}}%
\pgfpathlineto{\pgfqpoint{1.476546in}{1.953872in}}%
\pgfpathclose%
\pgfusepath{stroke}%
\end{pgfscope}%
\begin{pgfscope}%
\pgfpathrectangle{\pgfqpoint{0.494722in}{0.437222in}}{\pgfqpoint{6.275590in}{5.159444in}}%
\pgfusepath{clip}%
\pgfsetbuttcap%
\pgfsetroundjoin%
\pgfsetlinewidth{1.003750pt}%
\definecolor{currentstroke}{rgb}{0.827451,0.827451,0.827451}%
\pgfsetstrokecolor{currentstroke}%
\pgfsetstrokeopacity{0.800000}%
\pgfsetdash{}{0pt}%
\pgfpathmoveto{\pgfqpoint{3.208707in}{0.820522in}}%
\pgfpathcurveto{\pgfqpoint{3.219757in}{0.820522in}}{\pgfqpoint{3.230356in}{0.824912in}}{\pgfqpoint{3.238170in}{0.832726in}}%
\pgfpathcurveto{\pgfqpoint{3.245983in}{0.840539in}}{\pgfqpoint{3.250373in}{0.851139in}}{\pgfqpoint{3.250373in}{0.862189in}}%
\pgfpathcurveto{\pgfqpoint{3.250373in}{0.873239in}}{\pgfqpoint{3.245983in}{0.883838in}}{\pgfqpoint{3.238170in}{0.891651in}}%
\pgfpathcurveto{\pgfqpoint{3.230356in}{0.899465in}}{\pgfqpoint{3.219757in}{0.903855in}}{\pgfqpoint{3.208707in}{0.903855in}}%
\pgfpathcurveto{\pgfqpoint{3.197657in}{0.903855in}}{\pgfqpoint{3.187058in}{0.899465in}}{\pgfqpoint{3.179244in}{0.891651in}}%
\pgfpathcurveto{\pgfqpoint{3.171430in}{0.883838in}}{\pgfqpoint{3.167040in}{0.873239in}}{\pgfqpoint{3.167040in}{0.862189in}}%
\pgfpathcurveto{\pgfqpoint{3.167040in}{0.851139in}}{\pgfqpoint{3.171430in}{0.840539in}}{\pgfqpoint{3.179244in}{0.832726in}}%
\pgfpathcurveto{\pgfqpoint{3.187058in}{0.824912in}}{\pgfqpoint{3.197657in}{0.820522in}}{\pgfqpoint{3.208707in}{0.820522in}}%
\pgfpathlineto{\pgfqpoint{3.208707in}{0.820522in}}%
\pgfpathclose%
\pgfusepath{stroke}%
\end{pgfscope}%
\begin{pgfscope}%
\pgfpathrectangle{\pgfqpoint{0.494722in}{0.437222in}}{\pgfqpoint{6.275590in}{5.159444in}}%
\pgfusepath{clip}%
\pgfsetbuttcap%
\pgfsetroundjoin%
\pgfsetlinewidth{1.003750pt}%
\definecolor{currentstroke}{rgb}{0.827451,0.827451,0.827451}%
\pgfsetstrokecolor{currentstroke}%
\pgfsetstrokeopacity{0.800000}%
\pgfsetdash{}{0pt}%
\pgfpathmoveto{\pgfqpoint{1.167993in}{2.373873in}}%
\pgfpathcurveto{\pgfqpoint{1.179043in}{2.373873in}}{\pgfqpoint{1.189642in}{2.378264in}}{\pgfqpoint{1.197455in}{2.386077in}}%
\pgfpathcurveto{\pgfqpoint{1.205269in}{2.393891in}}{\pgfqpoint{1.209659in}{2.404490in}}{\pgfqpoint{1.209659in}{2.415540in}}%
\pgfpathcurveto{\pgfqpoint{1.209659in}{2.426590in}}{\pgfqpoint{1.205269in}{2.437189in}}{\pgfqpoint{1.197455in}{2.445003in}}%
\pgfpathcurveto{\pgfqpoint{1.189642in}{2.452816in}}{\pgfqpoint{1.179043in}{2.457207in}}{\pgfqpoint{1.167993in}{2.457207in}}%
\pgfpathcurveto{\pgfqpoint{1.156943in}{2.457207in}}{\pgfqpoint{1.146343in}{2.452816in}}{\pgfqpoint{1.138530in}{2.445003in}}%
\pgfpathcurveto{\pgfqpoint{1.130716in}{2.437189in}}{\pgfqpoint{1.126326in}{2.426590in}}{\pgfqpoint{1.126326in}{2.415540in}}%
\pgfpathcurveto{\pgfqpoint{1.126326in}{2.404490in}}{\pgfqpoint{1.130716in}{2.393891in}}{\pgfqpoint{1.138530in}{2.386077in}}%
\pgfpathcurveto{\pgfqpoint{1.146343in}{2.378264in}}{\pgfqpoint{1.156943in}{2.373873in}}{\pgfqpoint{1.167993in}{2.373873in}}%
\pgfpathlineto{\pgfqpoint{1.167993in}{2.373873in}}%
\pgfpathclose%
\pgfusepath{stroke}%
\end{pgfscope}%
\begin{pgfscope}%
\pgfpathrectangle{\pgfqpoint{0.494722in}{0.437222in}}{\pgfqpoint{6.275590in}{5.159444in}}%
\pgfusepath{clip}%
\pgfsetbuttcap%
\pgfsetroundjoin%
\pgfsetlinewidth{1.003750pt}%
\definecolor{currentstroke}{rgb}{0.827451,0.827451,0.827451}%
\pgfsetstrokecolor{currentstroke}%
\pgfsetstrokeopacity{0.800000}%
\pgfsetdash{}{0pt}%
\pgfpathmoveto{\pgfqpoint{0.584355in}{3.716646in}}%
\pgfpathcurveto{\pgfqpoint{0.595406in}{3.716646in}}{\pgfqpoint{0.606005in}{3.721036in}}{\pgfqpoint{0.613818in}{3.728850in}}%
\pgfpathcurveto{\pgfqpoint{0.621632in}{3.736663in}}{\pgfqpoint{0.626022in}{3.747262in}}{\pgfqpoint{0.626022in}{3.758313in}}%
\pgfpathcurveto{\pgfqpoint{0.626022in}{3.769363in}}{\pgfqpoint{0.621632in}{3.779962in}}{\pgfqpoint{0.613818in}{3.787775in}}%
\pgfpathcurveto{\pgfqpoint{0.606005in}{3.795589in}}{\pgfqpoint{0.595406in}{3.799979in}}{\pgfqpoint{0.584355in}{3.799979in}}%
\pgfpathcurveto{\pgfqpoint{0.573305in}{3.799979in}}{\pgfqpoint{0.562706in}{3.795589in}}{\pgfqpoint{0.554893in}{3.787775in}}%
\pgfpathcurveto{\pgfqpoint{0.547079in}{3.779962in}}{\pgfqpoint{0.542689in}{3.769363in}}{\pgfqpoint{0.542689in}{3.758313in}}%
\pgfpathcurveto{\pgfqpoint{0.542689in}{3.747262in}}{\pgfqpoint{0.547079in}{3.736663in}}{\pgfqpoint{0.554893in}{3.728850in}}%
\pgfpathcurveto{\pgfqpoint{0.562706in}{3.721036in}}{\pgfqpoint{0.573305in}{3.716646in}}{\pgfqpoint{0.584355in}{3.716646in}}%
\pgfpathlineto{\pgfqpoint{0.584355in}{3.716646in}}%
\pgfpathclose%
\pgfusepath{stroke}%
\end{pgfscope}%
\begin{pgfscope}%
\pgfpathrectangle{\pgfqpoint{0.494722in}{0.437222in}}{\pgfqpoint{6.275590in}{5.159444in}}%
\pgfusepath{clip}%
\pgfsetbuttcap%
\pgfsetroundjoin%
\pgfsetlinewidth{1.003750pt}%
\definecolor{currentstroke}{rgb}{0.827451,0.827451,0.827451}%
\pgfsetstrokecolor{currentstroke}%
\pgfsetstrokeopacity{0.800000}%
\pgfsetdash{}{0pt}%
\pgfpathmoveto{\pgfqpoint{2.141447in}{1.382526in}}%
\pgfpathcurveto{\pgfqpoint{2.152497in}{1.382526in}}{\pgfqpoint{2.163096in}{1.386916in}}{\pgfqpoint{2.170910in}{1.394730in}}%
\pgfpathcurveto{\pgfqpoint{2.178723in}{1.402543in}}{\pgfqpoint{2.183114in}{1.413142in}}{\pgfqpoint{2.183114in}{1.424193in}}%
\pgfpathcurveto{\pgfqpoint{2.183114in}{1.435243in}}{\pgfqpoint{2.178723in}{1.445842in}}{\pgfqpoint{2.170910in}{1.453655in}}%
\pgfpathcurveto{\pgfqpoint{2.163096in}{1.461469in}}{\pgfqpoint{2.152497in}{1.465859in}}{\pgfqpoint{2.141447in}{1.465859in}}%
\pgfpathcurveto{\pgfqpoint{2.130397in}{1.465859in}}{\pgfqpoint{2.119798in}{1.461469in}}{\pgfqpoint{2.111984in}{1.453655in}}%
\pgfpathcurveto{\pgfqpoint{2.104170in}{1.445842in}}{\pgfqpoint{2.099780in}{1.435243in}}{\pgfqpoint{2.099780in}{1.424193in}}%
\pgfpathcurveto{\pgfqpoint{2.099780in}{1.413142in}}{\pgfqpoint{2.104170in}{1.402543in}}{\pgfqpoint{2.111984in}{1.394730in}}%
\pgfpathcurveto{\pgfqpoint{2.119798in}{1.386916in}}{\pgfqpoint{2.130397in}{1.382526in}}{\pgfqpoint{2.141447in}{1.382526in}}%
\pgfpathlineto{\pgfqpoint{2.141447in}{1.382526in}}%
\pgfpathclose%
\pgfusepath{stroke}%
\end{pgfscope}%
\begin{pgfscope}%
\pgfpathrectangle{\pgfqpoint{0.494722in}{0.437222in}}{\pgfqpoint{6.275590in}{5.159444in}}%
\pgfusepath{clip}%
\pgfsetbuttcap%
\pgfsetroundjoin%
\pgfsetlinewidth{1.003750pt}%
\definecolor{currentstroke}{rgb}{0.827451,0.827451,0.827451}%
\pgfsetstrokecolor{currentstroke}%
\pgfsetstrokeopacity{0.800000}%
\pgfsetdash{}{0pt}%
\pgfpathmoveto{\pgfqpoint{0.770298in}{3.077236in}}%
\pgfpathcurveto{\pgfqpoint{0.781349in}{3.077236in}}{\pgfqpoint{0.791948in}{3.081626in}}{\pgfqpoint{0.799761in}{3.089440in}}%
\pgfpathcurveto{\pgfqpoint{0.807575in}{3.097254in}}{\pgfqpoint{0.811965in}{3.107853in}}{\pgfqpoint{0.811965in}{3.118903in}}%
\pgfpathcurveto{\pgfqpoint{0.811965in}{3.129953in}}{\pgfqpoint{0.807575in}{3.140552in}}{\pgfqpoint{0.799761in}{3.148366in}}%
\pgfpathcurveto{\pgfqpoint{0.791948in}{3.156179in}}{\pgfqpoint{0.781349in}{3.160569in}}{\pgfqpoint{0.770298in}{3.160569in}}%
\pgfpathcurveto{\pgfqpoint{0.759248in}{3.160569in}}{\pgfqpoint{0.748649in}{3.156179in}}{\pgfqpoint{0.740836in}{3.148366in}}%
\pgfpathcurveto{\pgfqpoint{0.733022in}{3.140552in}}{\pgfqpoint{0.728632in}{3.129953in}}{\pgfqpoint{0.728632in}{3.118903in}}%
\pgfpathcurveto{\pgfqpoint{0.728632in}{3.107853in}}{\pgfqpoint{0.733022in}{3.097254in}}{\pgfqpoint{0.740836in}{3.089440in}}%
\pgfpathcurveto{\pgfqpoint{0.748649in}{3.081626in}}{\pgfqpoint{0.759248in}{3.077236in}}{\pgfqpoint{0.770298in}{3.077236in}}%
\pgfpathlineto{\pgfqpoint{0.770298in}{3.077236in}}%
\pgfpathclose%
\pgfusepath{stroke}%
\end{pgfscope}%
\begin{pgfscope}%
\pgfpathrectangle{\pgfqpoint{0.494722in}{0.437222in}}{\pgfqpoint{6.275590in}{5.159444in}}%
\pgfusepath{clip}%
\pgfsetbuttcap%
\pgfsetroundjoin%
\pgfsetlinewidth{1.003750pt}%
\definecolor{currentstroke}{rgb}{0.827451,0.827451,0.827451}%
\pgfsetstrokecolor{currentstroke}%
\pgfsetstrokeopacity{0.800000}%
\pgfsetdash{}{0pt}%
\pgfpathmoveto{\pgfqpoint{5.425914in}{0.407036in}}%
\pgfpathcurveto{\pgfqpoint{5.436964in}{0.407036in}}{\pgfqpoint{5.447563in}{0.411426in}}{\pgfqpoint{5.455377in}{0.419240in}}%
\pgfpathcurveto{\pgfqpoint{5.463191in}{0.427053in}}{\pgfqpoint{5.467581in}{0.437652in}}{\pgfqpoint{5.467581in}{0.448702in}}%
\pgfpathcurveto{\pgfqpoint{5.467581in}{0.459753in}}{\pgfqpoint{5.463191in}{0.470352in}}{\pgfqpoint{5.455377in}{0.478165in}}%
\pgfpathcurveto{\pgfqpoint{5.447563in}{0.485979in}}{\pgfqpoint{5.436964in}{0.490369in}}{\pgfqpoint{5.425914in}{0.490369in}}%
\pgfpathcurveto{\pgfqpoint{5.414864in}{0.490369in}}{\pgfqpoint{5.404265in}{0.485979in}}{\pgfqpoint{5.396451in}{0.478165in}}%
\pgfpathcurveto{\pgfqpoint{5.388638in}{0.470352in}}{\pgfqpoint{5.384247in}{0.459753in}}{\pgfqpoint{5.384247in}{0.448702in}}%
\pgfpathcurveto{\pgfqpoint{5.384247in}{0.437652in}}{\pgfqpoint{5.388638in}{0.427053in}}{\pgfqpoint{5.396451in}{0.419240in}}%
\pgfpathcurveto{\pgfqpoint{5.404265in}{0.411426in}}{\pgfqpoint{5.414864in}{0.407036in}}{\pgfqpoint{5.425914in}{0.407036in}}%
\pgfusepath{stroke}%
\end{pgfscope}%
\begin{pgfscope}%
\pgfpathrectangle{\pgfqpoint{0.494722in}{0.437222in}}{\pgfqpoint{6.275590in}{5.159444in}}%
\pgfusepath{clip}%
\pgfsetbuttcap%
\pgfsetroundjoin%
\pgfsetlinewidth{1.003750pt}%
\definecolor{currentstroke}{rgb}{0.827451,0.827451,0.827451}%
\pgfsetstrokecolor{currentstroke}%
\pgfsetstrokeopacity{0.800000}%
\pgfsetdash{}{0pt}%
\pgfpathmoveto{\pgfqpoint{4.897389in}{0.472749in}}%
\pgfpathcurveto{\pgfqpoint{4.908439in}{0.472749in}}{\pgfqpoint{4.919038in}{0.477140in}}{\pgfqpoint{4.926852in}{0.484953in}}%
\pgfpathcurveto{\pgfqpoint{4.934665in}{0.492767in}}{\pgfqpoint{4.939056in}{0.503366in}}{\pgfqpoint{4.939056in}{0.514416in}}%
\pgfpathcurveto{\pgfqpoint{4.939056in}{0.525466in}}{\pgfqpoint{4.934665in}{0.536065in}}{\pgfqpoint{4.926852in}{0.543879in}}%
\pgfpathcurveto{\pgfqpoint{4.919038in}{0.551693in}}{\pgfqpoint{4.908439in}{0.556083in}}{\pgfqpoint{4.897389in}{0.556083in}}%
\pgfpathcurveto{\pgfqpoint{4.886339in}{0.556083in}}{\pgfqpoint{4.875740in}{0.551693in}}{\pgfqpoint{4.867926in}{0.543879in}}%
\pgfpathcurveto{\pgfqpoint{4.860113in}{0.536065in}}{\pgfqpoint{4.855722in}{0.525466in}}{\pgfqpoint{4.855722in}{0.514416in}}%
\pgfpathcurveto{\pgfqpoint{4.855722in}{0.503366in}}{\pgfqpoint{4.860113in}{0.492767in}}{\pgfqpoint{4.867926in}{0.484953in}}%
\pgfpathcurveto{\pgfqpoint{4.875740in}{0.477140in}}{\pgfqpoint{4.886339in}{0.472749in}}{\pgfqpoint{4.897389in}{0.472749in}}%
\pgfpathlineto{\pgfqpoint{4.897389in}{0.472749in}}%
\pgfpathclose%
\pgfusepath{stroke}%
\end{pgfscope}%
\begin{pgfscope}%
\pgfpathrectangle{\pgfqpoint{0.494722in}{0.437222in}}{\pgfqpoint{6.275590in}{5.159444in}}%
\pgfusepath{clip}%
\pgfsetbuttcap%
\pgfsetroundjoin%
\pgfsetlinewidth{1.003750pt}%
\definecolor{currentstroke}{rgb}{0.827451,0.827451,0.827451}%
\pgfsetstrokecolor{currentstroke}%
\pgfsetstrokeopacity{0.800000}%
\pgfsetdash{}{0pt}%
\pgfpathmoveto{\pgfqpoint{2.156999in}{1.373559in}}%
\pgfpathcurveto{\pgfqpoint{2.168049in}{1.373559in}}{\pgfqpoint{2.178648in}{1.377949in}}{\pgfqpoint{2.186462in}{1.385763in}}%
\pgfpathcurveto{\pgfqpoint{2.194275in}{1.393576in}}{\pgfqpoint{2.198665in}{1.404175in}}{\pgfqpoint{2.198665in}{1.415226in}}%
\pgfpathcurveto{\pgfqpoint{2.198665in}{1.426276in}}{\pgfqpoint{2.194275in}{1.436875in}}{\pgfqpoint{2.186462in}{1.444688in}}%
\pgfpathcurveto{\pgfqpoint{2.178648in}{1.452502in}}{\pgfqpoint{2.168049in}{1.456892in}}{\pgfqpoint{2.156999in}{1.456892in}}%
\pgfpathcurveto{\pgfqpoint{2.145949in}{1.456892in}}{\pgfqpoint{2.135350in}{1.452502in}}{\pgfqpoint{2.127536in}{1.444688in}}%
\pgfpathcurveto{\pgfqpoint{2.119722in}{1.436875in}}{\pgfqpoint{2.115332in}{1.426276in}}{\pgfqpoint{2.115332in}{1.415226in}}%
\pgfpathcurveto{\pgfqpoint{2.115332in}{1.404175in}}{\pgfqpoint{2.119722in}{1.393576in}}{\pgfqpoint{2.127536in}{1.385763in}}%
\pgfpathcurveto{\pgfqpoint{2.135350in}{1.377949in}}{\pgfqpoint{2.145949in}{1.373559in}}{\pgfqpoint{2.156999in}{1.373559in}}%
\pgfpathlineto{\pgfqpoint{2.156999in}{1.373559in}}%
\pgfpathclose%
\pgfusepath{stroke}%
\end{pgfscope}%
\begin{pgfscope}%
\pgfpathrectangle{\pgfqpoint{0.494722in}{0.437222in}}{\pgfqpoint{6.275590in}{5.159444in}}%
\pgfusepath{clip}%
\pgfsetbuttcap%
\pgfsetroundjoin%
\pgfsetlinewidth{1.003750pt}%
\definecolor{currentstroke}{rgb}{0.827451,0.827451,0.827451}%
\pgfsetstrokecolor{currentstroke}%
\pgfsetstrokeopacity{0.800000}%
\pgfsetdash{}{0pt}%
\pgfpathmoveto{\pgfqpoint{0.710037in}{3.308361in}}%
\pgfpathcurveto{\pgfqpoint{0.721087in}{3.308361in}}{\pgfqpoint{0.731686in}{3.312751in}}{\pgfqpoint{0.739500in}{3.320565in}}%
\pgfpathcurveto{\pgfqpoint{0.747314in}{3.328378in}}{\pgfqpoint{0.751704in}{3.338977in}}{\pgfqpoint{0.751704in}{3.350028in}}%
\pgfpathcurveto{\pgfqpoint{0.751704in}{3.361078in}}{\pgfqpoint{0.747314in}{3.371677in}}{\pgfqpoint{0.739500in}{3.379490in}}%
\pgfpathcurveto{\pgfqpoint{0.731686in}{3.387304in}}{\pgfqpoint{0.721087in}{3.391694in}}{\pgfqpoint{0.710037in}{3.391694in}}%
\pgfpathcurveto{\pgfqpoint{0.698987in}{3.391694in}}{\pgfqpoint{0.688388in}{3.387304in}}{\pgfqpoint{0.680574in}{3.379490in}}%
\pgfpathcurveto{\pgfqpoint{0.672761in}{3.371677in}}{\pgfqpoint{0.668371in}{3.361078in}}{\pgfqpoint{0.668371in}{3.350028in}}%
\pgfpathcurveto{\pgfqpoint{0.668371in}{3.338977in}}{\pgfqpoint{0.672761in}{3.328378in}}{\pgfqpoint{0.680574in}{3.320565in}}%
\pgfpathcurveto{\pgfqpoint{0.688388in}{3.312751in}}{\pgfqpoint{0.698987in}{3.308361in}}{\pgfqpoint{0.710037in}{3.308361in}}%
\pgfpathlineto{\pgfqpoint{0.710037in}{3.308361in}}%
\pgfpathclose%
\pgfusepath{stroke}%
\end{pgfscope}%
\begin{pgfscope}%
\pgfpathrectangle{\pgfqpoint{0.494722in}{0.437222in}}{\pgfqpoint{6.275590in}{5.159444in}}%
\pgfusepath{clip}%
\pgfsetbuttcap%
\pgfsetroundjoin%
\pgfsetlinewidth{1.003750pt}%
\definecolor{currentstroke}{rgb}{0.827451,0.827451,0.827451}%
\pgfsetstrokecolor{currentstroke}%
\pgfsetstrokeopacity{0.800000}%
\pgfsetdash{}{0pt}%
\pgfpathmoveto{\pgfqpoint{1.852502in}{1.624574in}}%
\pgfpathcurveto{\pgfqpoint{1.863553in}{1.624574in}}{\pgfqpoint{1.874152in}{1.628964in}}{\pgfqpoint{1.881965in}{1.636778in}}%
\pgfpathcurveto{\pgfqpoint{1.889779in}{1.644591in}}{\pgfqpoint{1.894169in}{1.655190in}}{\pgfqpoint{1.894169in}{1.666240in}}%
\pgfpathcurveto{\pgfqpoint{1.894169in}{1.677291in}}{\pgfqpoint{1.889779in}{1.687890in}}{\pgfqpoint{1.881965in}{1.695703in}}%
\pgfpathcurveto{\pgfqpoint{1.874152in}{1.703517in}}{\pgfqpoint{1.863553in}{1.707907in}}{\pgfqpoint{1.852502in}{1.707907in}}%
\pgfpathcurveto{\pgfqpoint{1.841452in}{1.707907in}}{\pgfqpoint{1.830853in}{1.703517in}}{\pgfqpoint{1.823040in}{1.695703in}}%
\pgfpathcurveto{\pgfqpoint{1.815226in}{1.687890in}}{\pgfqpoint{1.810836in}{1.677291in}}{\pgfqpoint{1.810836in}{1.666240in}}%
\pgfpathcurveto{\pgfqpoint{1.810836in}{1.655190in}}{\pgfqpoint{1.815226in}{1.644591in}}{\pgfqpoint{1.823040in}{1.636778in}}%
\pgfpathcurveto{\pgfqpoint{1.830853in}{1.628964in}}{\pgfqpoint{1.841452in}{1.624574in}}{\pgfqpoint{1.852502in}{1.624574in}}%
\pgfpathlineto{\pgfqpoint{1.852502in}{1.624574in}}%
\pgfpathclose%
\pgfusepath{stroke}%
\end{pgfscope}%
\begin{pgfscope}%
\pgfpathrectangle{\pgfqpoint{0.494722in}{0.437222in}}{\pgfqpoint{6.275590in}{5.159444in}}%
\pgfusepath{clip}%
\pgfsetbuttcap%
\pgfsetroundjoin%
\pgfsetlinewidth{1.003750pt}%
\definecolor{currentstroke}{rgb}{0.827451,0.827451,0.827451}%
\pgfsetstrokecolor{currentstroke}%
\pgfsetstrokeopacity{0.800000}%
\pgfsetdash{}{0pt}%
\pgfpathmoveto{\pgfqpoint{0.862000in}{2.887614in}}%
\pgfpathcurveto{\pgfqpoint{0.873051in}{2.887614in}}{\pgfqpoint{0.883650in}{2.892005in}}{\pgfqpoint{0.891463in}{2.899818in}}%
\pgfpathcurveto{\pgfqpoint{0.899277in}{2.907632in}}{\pgfqpoint{0.903667in}{2.918231in}}{\pgfqpoint{0.903667in}{2.929281in}}%
\pgfpathcurveto{\pgfqpoint{0.903667in}{2.940331in}}{\pgfqpoint{0.899277in}{2.950930in}}{\pgfqpoint{0.891463in}{2.958744in}}%
\pgfpathcurveto{\pgfqpoint{0.883650in}{2.966557in}}{\pgfqpoint{0.873051in}{2.970948in}}{\pgfqpoint{0.862000in}{2.970948in}}%
\pgfpathcurveto{\pgfqpoint{0.850950in}{2.970948in}}{\pgfqpoint{0.840351in}{2.966557in}}{\pgfqpoint{0.832538in}{2.958744in}}%
\pgfpathcurveto{\pgfqpoint{0.824724in}{2.950930in}}{\pgfqpoint{0.820334in}{2.940331in}}{\pgfqpoint{0.820334in}{2.929281in}}%
\pgfpathcurveto{\pgfqpoint{0.820334in}{2.918231in}}{\pgfqpoint{0.824724in}{2.907632in}}{\pgfqpoint{0.832538in}{2.899818in}}%
\pgfpathcurveto{\pgfqpoint{0.840351in}{2.892005in}}{\pgfqpoint{0.850950in}{2.887614in}}{\pgfqpoint{0.862000in}{2.887614in}}%
\pgfpathlineto{\pgfqpoint{0.862000in}{2.887614in}}%
\pgfpathclose%
\pgfusepath{stroke}%
\end{pgfscope}%
\begin{pgfscope}%
\pgfpathrectangle{\pgfqpoint{0.494722in}{0.437222in}}{\pgfqpoint{6.275590in}{5.159444in}}%
\pgfusepath{clip}%
\pgfsetbuttcap%
\pgfsetroundjoin%
\pgfsetlinewidth{1.003750pt}%
\definecolor{currentstroke}{rgb}{0.827451,0.827451,0.827451}%
\pgfsetstrokecolor{currentstroke}%
\pgfsetstrokeopacity{0.800000}%
\pgfsetdash{}{0pt}%
\pgfpathmoveto{\pgfqpoint{3.526336in}{0.700005in}}%
\pgfpathcurveto{\pgfqpoint{3.537386in}{0.700005in}}{\pgfqpoint{3.547985in}{0.704396in}}{\pgfqpoint{3.555799in}{0.712209in}}%
\pgfpathcurveto{\pgfqpoint{3.563612in}{0.720023in}}{\pgfqpoint{3.568002in}{0.730622in}}{\pgfqpoint{3.568002in}{0.741672in}}%
\pgfpathcurveto{\pgfqpoint{3.568002in}{0.752722in}}{\pgfqpoint{3.563612in}{0.763321in}}{\pgfqpoint{3.555799in}{0.771135in}}%
\pgfpathcurveto{\pgfqpoint{3.547985in}{0.778948in}}{\pgfqpoint{3.537386in}{0.783339in}}{\pgfqpoint{3.526336in}{0.783339in}}%
\pgfpathcurveto{\pgfqpoint{3.515286in}{0.783339in}}{\pgfqpoint{3.504687in}{0.778948in}}{\pgfqpoint{3.496873in}{0.771135in}}%
\pgfpathcurveto{\pgfqpoint{3.489059in}{0.763321in}}{\pgfqpoint{3.484669in}{0.752722in}}{\pgfqpoint{3.484669in}{0.741672in}}%
\pgfpathcurveto{\pgfqpoint{3.484669in}{0.730622in}}{\pgfqpoint{3.489059in}{0.720023in}}{\pgfqpoint{3.496873in}{0.712209in}}%
\pgfpathcurveto{\pgfqpoint{3.504687in}{0.704396in}}{\pgfqpoint{3.515286in}{0.700005in}}{\pgfqpoint{3.526336in}{0.700005in}}%
\pgfpathlineto{\pgfqpoint{3.526336in}{0.700005in}}%
\pgfpathclose%
\pgfusepath{stroke}%
\end{pgfscope}%
\begin{pgfscope}%
\pgfpathrectangle{\pgfqpoint{0.494722in}{0.437222in}}{\pgfqpoint{6.275590in}{5.159444in}}%
\pgfusepath{clip}%
\pgfsetbuttcap%
\pgfsetroundjoin%
\pgfsetlinewidth{1.003750pt}%
\definecolor{currentstroke}{rgb}{0.827451,0.827451,0.827451}%
\pgfsetstrokecolor{currentstroke}%
\pgfsetstrokeopacity{0.800000}%
\pgfsetdash{}{0pt}%
\pgfpathmoveto{\pgfqpoint{1.464755in}{1.963622in}}%
\pgfpathcurveto{\pgfqpoint{1.475805in}{1.963622in}}{\pgfqpoint{1.486404in}{1.968013in}}{\pgfqpoint{1.494217in}{1.975826in}}%
\pgfpathcurveto{\pgfqpoint{1.502031in}{1.983640in}}{\pgfqpoint{1.506421in}{1.994239in}}{\pgfqpoint{1.506421in}{2.005289in}}%
\pgfpathcurveto{\pgfqpoint{1.506421in}{2.016339in}}{\pgfqpoint{1.502031in}{2.026938in}}{\pgfqpoint{1.494217in}{2.034752in}}%
\pgfpathcurveto{\pgfqpoint{1.486404in}{2.042566in}}{\pgfqpoint{1.475805in}{2.046956in}}{\pgfqpoint{1.464755in}{2.046956in}}%
\pgfpathcurveto{\pgfqpoint{1.453705in}{2.046956in}}{\pgfqpoint{1.443106in}{2.042566in}}{\pgfqpoint{1.435292in}{2.034752in}}%
\pgfpathcurveto{\pgfqpoint{1.427478in}{2.026938in}}{\pgfqpoint{1.423088in}{2.016339in}}{\pgfqpoint{1.423088in}{2.005289in}}%
\pgfpathcurveto{\pgfqpoint{1.423088in}{1.994239in}}{\pgfqpoint{1.427478in}{1.983640in}}{\pgfqpoint{1.435292in}{1.975826in}}%
\pgfpathcurveto{\pgfqpoint{1.443106in}{1.968013in}}{\pgfqpoint{1.453705in}{1.963622in}}{\pgfqpoint{1.464755in}{1.963622in}}%
\pgfpathlineto{\pgfqpoint{1.464755in}{1.963622in}}%
\pgfpathclose%
\pgfusepath{stroke}%
\end{pgfscope}%
\begin{pgfscope}%
\pgfpathrectangle{\pgfqpoint{0.494722in}{0.437222in}}{\pgfqpoint{6.275590in}{5.159444in}}%
\pgfusepath{clip}%
\pgfsetbuttcap%
\pgfsetroundjoin%
\pgfsetlinewidth{1.003750pt}%
\definecolor{currentstroke}{rgb}{0.827451,0.827451,0.827451}%
\pgfsetstrokecolor{currentstroke}%
\pgfsetstrokeopacity{0.800000}%
\pgfsetdash{}{0pt}%
\pgfpathmoveto{\pgfqpoint{0.638504in}{3.493003in}}%
\pgfpathcurveto{\pgfqpoint{0.649554in}{3.493003in}}{\pgfqpoint{0.660153in}{3.497394in}}{\pgfqpoint{0.667967in}{3.505207in}}%
\pgfpathcurveto{\pgfqpoint{0.675780in}{3.513021in}}{\pgfqpoint{0.680170in}{3.523620in}}{\pgfqpoint{0.680170in}{3.534670in}}%
\pgfpathcurveto{\pgfqpoint{0.680170in}{3.545720in}}{\pgfqpoint{0.675780in}{3.556319in}}{\pgfqpoint{0.667967in}{3.564133in}}%
\pgfpathcurveto{\pgfqpoint{0.660153in}{3.571946in}}{\pgfqpoint{0.649554in}{3.576337in}}{\pgfqpoint{0.638504in}{3.576337in}}%
\pgfpathcurveto{\pgfqpoint{0.627454in}{3.576337in}}{\pgfqpoint{0.616855in}{3.571946in}}{\pgfqpoint{0.609041in}{3.564133in}}%
\pgfpathcurveto{\pgfqpoint{0.601227in}{3.556319in}}{\pgfqpoint{0.596837in}{3.545720in}}{\pgfqpoint{0.596837in}{3.534670in}}%
\pgfpathcurveto{\pgfqpoint{0.596837in}{3.523620in}}{\pgfqpoint{0.601227in}{3.513021in}}{\pgfqpoint{0.609041in}{3.505207in}}%
\pgfpathcurveto{\pgfqpoint{0.616855in}{3.497394in}}{\pgfqpoint{0.627454in}{3.493003in}}{\pgfqpoint{0.638504in}{3.493003in}}%
\pgfpathlineto{\pgfqpoint{0.638504in}{3.493003in}}%
\pgfpathclose%
\pgfusepath{stroke}%
\end{pgfscope}%
\begin{pgfscope}%
\pgfpathrectangle{\pgfqpoint{0.494722in}{0.437222in}}{\pgfqpoint{6.275590in}{5.159444in}}%
\pgfusepath{clip}%
\pgfsetbuttcap%
\pgfsetroundjoin%
\pgfsetlinewidth{1.003750pt}%
\definecolor{currentstroke}{rgb}{0.827451,0.827451,0.827451}%
\pgfsetstrokecolor{currentstroke}%
\pgfsetstrokeopacity{0.800000}%
\pgfsetdash{}{0pt}%
\pgfpathmoveto{\pgfqpoint{3.460812in}{0.723768in}}%
\pgfpathcurveto{\pgfqpoint{3.471862in}{0.723768in}}{\pgfqpoint{3.482461in}{0.728158in}}{\pgfqpoint{3.490274in}{0.735972in}}%
\pgfpathcurveto{\pgfqpoint{3.498088in}{0.743785in}}{\pgfqpoint{3.502478in}{0.754384in}}{\pgfqpoint{3.502478in}{0.765434in}}%
\pgfpathcurveto{\pgfqpoint{3.502478in}{0.776484in}}{\pgfqpoint{3.498088in}{0.787083in}}{\pgfqpoint{3.490274in}{0.794897in}}%
\pgfpathcurveto{\pgfqpoint{3.482461in}{0.802711in}}{\pgfqpoint{3.471862in}{0.807101in}}{\pgfqpoint{3.460812in}{0.807101in}}%
\pgfpathcurveto{\pgfqpoint{3.449762in}{0.807101in}}{\pgfqpoint{3.439162in}{0.802711in}}{\pgfqpoint{3.431349in}{0.794897in}}%
\pgfpathcurveto{\pgfqpoint{3.423535in}{0.787083in}}{\pgfqpoint{3.419145in}{0.776484in}}{\pgfqpoint{3.419145in}{0.765434in}}%
\pgfpathcurveto{\pgfqpoint{3.419145in}{0.754384in}}{\pgfqpoint{3.423535in}{0.743785in}}{\pgfqpoint{3.431349in}{0.735972in}}%
\pgfpathcurveto{\pgfqpoint{3.439162in}{0.728158in}}{\pgfqpoint{3.449762in}{0.723768in}}{\pgfqpoint{3.460812in}{0.723768in}}%
\pgfpathlineto{\pgfqpoint{3.460812in}{0.723768in}}%
\pgfpathclose%
\pgfusepath{stroke}%
\end{pgfscope}%
\begin{pgfscope}%
\pgfpathrectangle{\pgfqpoint{0.494722in}{0.437222in}}{\pgfqpoint{6.275590in}{5.159444in}}%
\pgfusepath{clip}%
\pgfsetbuttcap%
\pgfsetroundjoin%
\pgfsetlinewidth{1.003750pt}%
\definecolor{currentstroke}{rgb}{0.827451,0.827451,0.827451}%
\pgfsetstrokecolor{currentstroke}%
\pgfsetstrokeopacity{0.800000}%
\pgfsetdash{}{0pt}%
\pgfpathmoveto{\pgfqpoint{0.630923in}{3.550009in}}%
\pgfpathcurveto{\pgfqpoint{0.641973in}{3.550009in}}{\pgfqpoint{0.652572in}{3.554399in}}{\pgfqpoint{0.660385in}{3.562213in}}%
\pgfpathcurveto{\pgfqpoint{0.668199in}{3.570026in}}{\pgfqpoint{0.672589in}{3.580625in}}{\pgfqpoint{0.672589in}{3.591675in}}%
\pgfpathcurveto{\pgfqpoint{0.672589in}{3.602726in}}{\pgfqpoint{0.668199in}{3.613325in}}{\pgfqpoint{0.660385in}{3.621138in}}%
\pgfpathcurveto{\pgfqpoint{0.652572in}{3.628952in}}{\pgfqpoint{0.641973in}{3.633342in}}{\pgfqpoint{0.630923in}{3.633342in}}%
\pgfpathcurveto{\pgfqpoint{0.619873in}{3.633342in}}{\pgfqpoint{0.609274in}{3.628952in}}{\pgfqpoint{0.601460in}{3.621138in}}%
\pgfpathcurveto{\pgfqpoint{0.593646in}{3.613325in}}{\pgfqpoint{0.589256in}{3.602726in}}{\pgfqpoint{0.589256in}{3.591675in}}%
\pgfpathcurveto{\pgfqpoint{0.589256in}{3.580625in}}{\pgfqpoint{0.593646in}{3.570026in}}{\pgfqpoint{0.601460in}{3.562213in}}%
\pgfpathcurveto{\pgfqpoint{0.609274in}{3.554399in}}{\pgfqpoint{0.619873in}{3.550009in}}{\pgfqpoint{0.630923in}{3.550009in}}%
\pgfpathlineto{\pgfqpoint{0.630923in}{3.550009in}}%
\pgfpathclose%
\pgfusepath{stroke}%
\end{pgfscope}%
\begin{pgfscope}%
\pgfpathrectangle{\pgfqpoint{0.494722in}{0.437222in}}{\pgfqpoint{6.275590in}{5.159444in}}%
\pgfusepath{clip}%
\pgfsetbuttcap%
\pgfsetroundjoin%
\pgfsetlinewidth{1.003750pt}%
\definecolor{currentstroke}{rgb}{0.827451,0.827451,0.827451}%
\pgfsetstrokecolor{currentstroke}%
\pgfsetstrokeopacity{0.800000}%
\pgfsetdash{}{0pt}%
\pgfpathmoveto{\pgfqpoint{5.510053in}{0.442055in}}%
\pgfpathcurveto{\pgfqpoint{5.521103in}{0.442055in}}{\pgfqpoint{5.531702in}{0.446445in}}{\pgfqpoint{5.539516in}{0.454259in}}%
\pgfpathcurveto{\pgfqpoint{5.547330in}{0.462073in}}{\pgfqpoint{5.551720in}{0.472672in}}{\pgfqpoint{5.551720in}{0.483722in}}%
\pgfpathcurveto{\pgfqpoint{5.551720in}{0.494772in}}{\pgfqpoint{5.547330in}{0.505371in}}{\pgfqpoint{5.539516in}{0.513185in}}%
\pgfpathcurveto{\pgfqpoint{5.531702in}{0.520998in}}{\pgfqpoint{5.521103in}{0.525388in}}{\pgfqpoint{5.510053in}{0.525388in}}%
\pgfpathcurveto{\pgfqpoint{5.499003in}{0.525388in}}{\pgfqpoint{5.488404in}{0.520998in}}{\pgfqpoint{5.480590in}{0.513185in}}%
\pgfpathcurveto{\pgfqpoint{5.472777in}{0.505371in}}{\pgfqpoint{5.468387in}{0.494772in}}{\pgfqpoint{5.468387in}{0.483722in}}%
\pgfpathcurveto{\pgfqpoint{5.468387in}{0.472672in}}{\pgfqpoint{5.472777in}{0.462073in}}{\pgfqpoint{5.480590in}{0.454259in}}%
\pgfpathcurveto{\pgfqpoint{5.488404in}{0.446445in}}{\pgfqpoint{5.499003in}{0.442055in}}{\pgfqpoint{5.510053in}{0.442055in}}%
\pgfpathlineto{\pgfqpoint{5.510053in}{0.442055in}}%
\pgfpathclose%
\pgfusepath{stroke}%
\end{pgfscope}%
\begin{pgfscope}%
\pgfpathrectangle{\pgfqpoint{0.494722in}{0.437222in}}{\pgfqpoint{6.275590in}{5.159444in}}%
\pgfusepath{clip}%
\pgfsetbuttcap%
\pgfsetroundjoin%
\pgfsetlinewidth{1.003750pt}%
\definecolor{currentstroke}{rgb}{0.827451,0.827451,0.827451}%
\pgfsetstrokecolor{currentstroke}%
\pgfsetstrokeopacity{0.800000}%
\pgfsetdash{}{0pt}%
\pgfpathmoveto{\pgfqpoint{4.773441in}{0.510022in}}%
\pgfpathcurveto{\pgfqpoint{4.784491in}{0.510022in}}{\pgfqpoint{4.795090in}{0.514412in}}{\pgfqpoint{4.802904in}{0.522226in}}%
\pgfpathcurveto{\pgfqpoint{4.810717in}{0.530040in}}{\pgfqpoint{4.815107in}{0.540639in}}{\pgfqpoint{4.815107in}{0.551689in}}%
\pgfpathcurveto{\pgfqpoint{4.815107in}{0.562739in}}{\pgfqpoint{4.810717in}{0.573338in}}{\pgfqpoint{4.802904in}{0.581152in}}%
\pgfpathcurveto{\pgfqpoint{4.795090in}{0.588965in}}{\pgfqpoint{4.784491in}{0.593355in}}{\pgfqpoint{4.773441in}{0.593355in}}%
\pgfpathcurveto{\pgfqpoint{4.762391in}{0.593355in}}{\pgfqpoint{4.751792in}{0.588965in}}{\pgfqpoint{4.743978in}{0.581152in}}%
\pgfpathcurveto{\pgfqpoint{4.736164in}{0.573338in}}{\pgfqpoint{4.731774in}{0.562739in}}{\pgfqpoint{4.731774in}{0.551689in}}%
\pgfpathcurveto{\pgfqpoint{4.731774in}{0.540639in}}{\pgfqpoint{4.736164in}{0.530040in}}{\pgfqpoint{4.743978in}{0.522226in}}%
\pgfpathcurveto{\pgfqpoint{4.751792in}{0.514412in}}{\pgfqpoint{4.762391in}{0.510022in}}{\pgfqpoint{4.773441in}{0.510022in}}%
\pgfpathlineto{\pgfqpoint{4.773441in}{0.510022in}}%
\pgfpathclose%
\pgfusepath{stroke}%
\end{pgfscope}%
\begin{pgfscope}%
\pgfpathrectangle{\pgfqpoint{0.494722in}{0.437222in}}{\pgfqpoint{6.275590in}{5.159444in}}%
\pgfusepath{clip}%
\pgfsetbuttcap%
\pgfsetroundjoin%
\pgfsetlinewidth{1.003750pt}%
\definecolor{currentstroke}{rgb}{0.827451,0.827451,0.827451}%
\pgfsetstrokecolor{currentstroke}%
\pgfsetstrokeopacity{0.800000}%
\pgfsetdash{}{0pt}%
\pgfpathmoveto{\pgfqpoint{2.331250in}{1.359105in}}%
\pgfpathcurveto{\pgfqpoint{2.342300in}{1.359105in}}{\pgfqpoint{2.352899in}{1.363495in}}{\pgfqpoint{2.360713in}{1.371309in}}%
\pgfpathcurveto{\pgfqpoint{2.368526in}{1.379122in}}{\pgfqpoint{2.372917in}{1.389721in}}{\pgfqpoint{2.372917in}{1.400771in}}%
\pgfpathcurveto{\pgfqpoint{2.372917in}{1.411822in}}{\pgfqpoint{2.368526in}{1.422421in}}{\pgfqpoint{2.360713in}{1.430234in}}%
\pgfpathcurveto{\pgfqpoint{2.352899in}{1.438048in}}{\pgfqpoint{2.342300in}{1.442438in}}{\pgfqpoint{2.331250in}{1.442438in}}%
\pgfpathcurveto{\pgfqpoint{2.320200in}{1.442438in}}{\pgfqpoint{2.309601in}{1.438048in}}{\pgfqpoint{2.301787in}{1.430234in}}%
\pgfpathcurveto{\pgfqpoint{2.293974in}{1.422421in}}{\pgfqpoint{2.289583in}{1.411822in}}{\pgfqpoint{2.289583in}{1.400771in}}%
\pgfpathcurveto{\pgfqpoint{2.289583in}{1.389721in}}{\pgfqpoint{2.293974in}{1.379122in}}{\pgfqpoint{2.301787in}{1.371309in}}%
\pgfpathcurveto{\pgfqpoint{2.309601in}{1.363495in}}{\pgfqpoint{2.320200in}{1.359105in}}{\pgfqpoint{2.331250in}{1.359105in}}%
\pgfpathlineto{\pgfqpoint{2.331250in}{1.359105in}}%
\pgfpathclose%
\pgfusepath{stroke}%
\end{pgfscope}%
\begin{pgfscope}%
\pgfpathrectangle{\pgfqpoint{0.494722in}{0.437222in}}{\pgfqpoint{6.275590in}{5.159444in}}%
\pgfusepath{clip}%
\pgfsetbuttcap%
\pgfsetroundjoin%
\pgfsetlinewidth{1.003750pt}%
\definecolor{currentstroke}{rgb}{0.827451,0.827451,0.827451}%
\pgfsetstrokecolor{currentstroke}%
\pgfsetstrokeopacity{0.800000}%
\pgfsetdash{}{0pt}%
\pgfpathmoveto{\pgfqpoint{2.648065in}{1.178148in}}%
\pgfpathcurveto{\pgfqpoint{2.659115in}{1.178148in}}{\pgfqpoint{2.669714in}{1.182539in}}{\pgfqpoint{2.677528in}{1.190352in}}%
\pgfpathcurveto{\pgfqpoint{2.685341in}{1.198166in}}{\pgfqpoint{2.689732in}{1.208765in}}{\pgfqpoint{2.689732in}{1.219815in}}%
\pgfpathcurveto{\pgfqpoint{2.689732in}{1.230865in}}{\pgfqpoint{2.685341in}{1.241464in}}{\pgfqpoint{2.677528in}{1.249278in}}%
\pgfpathcurveto{\pgfqpoint{2.669714in}{1.257091in}}{\pgfqpoint{2.659115in}{1.261482in}}{\pgfqpoint{2.648065in}{1.261482in}}%
\pgfpathcurveto{\pgfqpoint{2.637015in}{1.261482in}}{\pgfqpoint{2.626416in}{1.257091in}}{\pgfqpoint{2.618602in}{1.249278in}}%
\pgfpathcurveto{\pgfqpoint{2.610789in}{1.241464in}}{\pgfqpoint{2.606398in}{1.230865in}}{\pgfqpoint{2.606398in}{1.219815in}}%
\pgfpathcurveto{\pgfqpoint{2.606398in}{1.208765in}}{\pgfqpoint{2.610789in}{1.198166in}}{\pgfqpoint{2.618602in}{1.190352in}}%
\pgfpathcurveto{\pgfqpoint{2.626416in}{1.182539in}}{\pgfqpoint{2.637015in}{1.178148in}}{\pgfqpoint{2.648065in}{1.178148in}}%
\pgfpathlineto{\pgfqpoint{2.648065in}{1.178148in}}%
\pgfpathclose%
\pgfusepath{stroke}%
\end{pgfscope}%
\begin{pgfscope}%
\pgfpathrectangle{\pgfqpoint{0.494722in}{0.437222in}}{\pgfqpoint{6.275590in}{5.159444in}}%
\pgfusepath{clip}%
\pgfsetbuttcap%
\pgfsetroundjoin%
\pgfsetlinewidth{1.003750pt}%
\definecolor{currentstroke}{rgb}{0.827451,0.827451,0.827451}%
\pgfsetstrokecolor{currentstroke}%
\pgfsetstrokeopacity{0.800000}%
\pgfsetdash{}{0pt}%
\pgfpathmoveto{\pgfqpoint{1.185063in}{2.437307in}}%
\pgfpathcurveto{\pgfqpoint{1.196113in}{2.437307in}}{\pgfqpoint{1.206712in}{2.441697in}}{\pgfqpoint{1.214526in}{2.449511in}}%
\pgfpathcurveto{\pgfqpoint{1.222340in}{2.457325in}}{\pgfqpoint{1.226730in}{2.467924in}}{\pgfqpoint{1.226730in}{2.478974in}}%
\pgfpathcurveto{\pgfqpoint{1.226730in}{2.490024in}}{\pgfqpoint{1.222340in}{2.500623in}}{\pgfqpoint{1.214526in}{2.508437in}}%
\pgfpathcurveto{\pgfqpoint{1.206712in}{2.516250in}}{\pgfqpoint{1.196113in}{2.520641in}}{\pgfqpoint{1.185063in}{2.520641in}}%
\pgfpathcurveto{\pgfqpoint{1.174013in}{2.520641in}}{\pgfqpoint{1.163414in}{2.516250in}}{\pgfqpoint{1.155600in}{2.508437in}}%
\pgfpathcurveto{\pgfqpoint{1.147787in}{2.500623in}}{\pgfqpoint{1.143397in}{2.490024in}}{\pgfqpoint{1.143397in}{2.478974in}}%
\pgfpathcurveto{\pgfqpoint{1.143397in}{2.467924in}}{\pgfqpoint{1.147787in}{2.457325in}}{\pgfqpoint{1.155600in}{2.449511in}}%
\pgfpathcurveto{\pgfqpoint{1.163414in}{2.441697in}}{\pgfqpoint{1.174013in}{2.437307in}}{\pgfqpoint{1.185063in}{2.437307in}}%
\pgfpathlineto{\pgfqpoint{1.185063in}{2.437307in}}%
\pgfpathclose%
\pgfusepath{stroke}%
\end{pgfscope}%
\begin{pgfscope}%
\pgfpathrectangle{\pgfqpoint{0.494722in}{0.437222in}}{\pgfqpoint{6.275590in}{5.159444in}}%
\pgfusepath{clip}%
\pgfsetbuttcap%
\pgfsetroundjoin%
\pgfsetlinewidth{1.003750pt}%
\definecolor{currentstroke}{rgb}{0.827451,0.827451,0.827451}%
\pgfsetstrokecolor{currentstroke}%
\pgfsetstrokeopacity{0.800000}%
\pgfsetdash{}{0pt}%
\pgfpathmoveto{\pgfqpoint{0.767639in}{3.316864in}}%
\pgfpathcurveto{\pgfqpoint{0.778690in}{3.316864in}}{\pgfqpoint{0.789289in}{3.321254in}}{\pgfqpoint{0.797102in}{3.329068in}}%
\pgfpathcurveto{\pgfqpoint{0.804916in}{3.336882in}}{\pgfqpoint{0.809306in}{3.347481in}}{\pgfqpoint{0.809306in}{3.358531in}}%
\pgfpathcurveto{\pgfqpoint{0.809306in}{3.369581in}}{\pgfqpoint{0.804916in}{3.380180in}}{\pgfqpoint{0.797102in}{3.387994in}}%
\pgfpathcurveto{\pgfqpoint{0.789289in}{3.395807in}}{\pgfqpoint{0.778690in}{3.400198in}}{\pgfqpoint{0.767639in}{3.400198in}}%
\pgfpathcurveto{\pgfqpoint{0.756589in}{3.400198in}}{\pgfqpoint{0.745990in}{3.395807in}}{\pgfqpoint{0.738177in}{3.387994in}}%
\pgfpathcurveto{\pgfqpoint{0.730363in}{3.380180in}}{\pgfqpoint{0.725973in}{3.369581in}}{\pgfqpoint{0.725973in}{3.358531in}}%
\pgfpathcurveto{\pgfqpoint{0.725973in}{3.347481in}}{\pgfqpoint{0.730363in}{3.336882in}}{\pgfqpoint{0.738177in}{3.329068in}}%
\pgfpathcurveto{\pgfqpoint{0.745990in}{3.321254in}}{\pgfqpoint{0.756589in}{3.316864in}}{\pgfqpoint{0.767639in}{3.316864in}}%
\pgfpathlineto{\pgfqpoint{0.767639in}{3.316864in}}%
\pgfpathclose%
\pgfusepath{stroke}%
\end{pgfscope}%
\begin{pgfscope}%
\pgfpathrectangle{\pgfqpoint{0.494722in}{0.437222in}}{\pgfqpoint{6.275590in}{5.159444in}}%
\pgfusepath{clip}%
\pgfsetbuttcap%
\pgfsetroundjoin%
\pgfsetlinewidth{1.003750pt}%
\definecolor{currentstroke}{rgb}{0.827451,0.827451,0.827451}%
\pgfsetstrokecolor{currentstroke}%
\pgfsetstrokeopacity{0.800000}%
\pgfsetdash{}{0pt}%
\pgfpathmoveto{\pgfqpoint{2.904991in}{0.980053in}}%
\pgfpathcurveto{\pgfqpoint{2.916041in}{0.980053in}}{\pgfqpoint{2.926640in}{0.984443in}}{\pgfqpoint{2.934454in}{0.992257in}}%
\pgfpathcurveto{\pgfqpoint{2.942267in}{1.000071in}}{\pgfqpoint{2.946658in}{1.010670in}}{\pgfqpoint{2.946658in}{1.021720in}}%
\pgfpathcurveto{\pgfqpoint{2.946658in}{1.032770in}}{\pgfqpoint{2.942267in}{1.043369in}}{\pgfqpoint{2.934454in}{1.051183in}}%
\pgfpathcurveto{\pgfqpoint{2.926640in}{1.058996in}}{\pgfqpoint{2.916041in}{1.063387in}}{\pgfqpoint{2.904991in}{1.063387in}}%
\pgfpathcurveto{\pgfqpoint{2.893941in}{1.063387in}}{\pgfqpoint{2.883342in}{1.058996in}}{\pgfqpoint{2.875528in}{1.051183in}}%
\pgfpathcurveto{\pgfqpoint{2.867715in}{1.043369in}}{\pgfqpoint{2.863324in}{1.032770in}}{\pgfqpoint{2.863324in}{1.021720in}}%
\pgfpathcurveto{\pgfqpoint{2.863324in}{1.010670in}}{\pgfqpoint{2.867715in}{1.000071in}}{\pgfqpoint{2.875528in}{0.992257in}}%
\pgfpathcurveto{\pgfqpoint{2.883342in}{0.984443in}}{\pgfqpoint{2.893941in}{0.980053in}}{\pgfqpoint{2.904991in}{0.980053in}}%
\pgfpathlineto{\pgfqpoint{2.904991in}{0.980053in}}%
\pgfpathclose%
\pgfusepath{stroke}%
\end{pgfscope}%
\begin{pgfscope}%
\pgfpathrectangle{\pgfqpoint{0.494722in}{0.437222in}}{\pgfqpoint{6.275590in}{5.159444in}}%
\pgfusepath{clip}%
\pgfsetbuttcap%
\pgfsetroundjoin%
\pgfsetlinewidth{1.003750pt}%
\definecolor{currentstroke}{rgb}{0.827451,0.827451,0.827451}%
\pgfsetstrokecolor{currentstroke}%
\pgfsetstrokeopacity{0.800000}%
\pgfsetdash{}{0pt}%
\pgfpathmoveto{\pgfqpoint{4.282315in}{0.548168in}}%
\pgfpathcurveto{\pgfqpoint{4.293365in}{0.548168in}}{\pgfqpoint{4.303964in}{0.552559in}}{\pgfqpoint{4.311778in}{0.560372in}}%
\pgfpathcurveto{\pgfqpoint{4.319591in}{0.568186in}}{\pgfqpoint{4.323982in}{0.578785in}}{\pgfqpoint{4.323982in}{0.589835in}}%
\pgfpathcurveto{\pgfqpoint{4.323982in}{0.600885in}}{\pgfqpoint{4.319591in}{0.611484in}}{\pgfqpoint{4.311778in}{0.619298in}}%
\pgfpathcurveto{\pgfqpoint{4.303964in}{0.627111in}}{\pgfqpoint{4.293365in}{0.631502in}}{\pgfqpoint{4.282315in}{0.631502in}}%
\pgfpathcurveto{\pgfqpoint{4.271265in}{0.631502in}}{\pgfqpoint{4.260666in}{0.627111in}}{\pgfqpoint{4.252852in}{0.619298in}}%
\pgfpathcurveto{\pgfqpoint{4.245039in}{0.611484in}}{\pgfqpoint{4.240648in}{0.600885in}}{\pgfqpoint{4.240648in}{0.589835in}}%
\pgfpathcurveto{\pgfqpoint{4.240648in}{0.578785in}}{\pgfqpoint{4.245039in}{0.568186in}}{\pgfqpoint{4.252852in}{0.560372in}}%
\pgfpathcurveto{\pgfqpoint{4.260666in}{0.552559in}}{\pgfqpoint{4.271265in}{0.548168in}}{\pgfqpoint{4.282315in}{0.548168in}}%
\pgfpathlineto{\pgfqpoint{4.282315in}{0.548168in}}%
\pgfpathclose%
\pgfusepath{stroke}%
\end{pgfscope}%
\begin{pgfscope}%
\pgfpathrectangle{\pgfqpoint{0.494722in}{0.437222in}}{\pgfqpoint{6.275590in}{5.159444in}}%
\pgfusepath{clip}%
\pgfsetbuttcap%
\pgfsetroundjoin%
\pgfsetlinewidth{1.003750pt}%
\definecolor{currentstroke}{rgb}{0.827451,0.827451,0.827451}%
\pgfsetstrokecolor{currentstroke}%
\pgfsetstrokeopacity{0.800000}%
\pgfsetdash{}{0pt}%
\pgfpathmoveto{\pgfqpoint{0.498024in}{4.460098in}}%
\pgfpathcurveto{\pgfqpoint{0.509074in}{4.460098in}}{\pgfqpoint{0.519673in}{4.464488in}}{\pgfqpoint{0.527487in}{4.472302in}}%
\pgfpathcurveto{\pgfqpoint{0.535301in}{4.480116in}}{\pgfqpoint{0.539691in}{4.490715in}}{\pgfqpoint{0.539691in}{4.501765in}}%
\pgfpathcurveto{\pgfqpoint{0.539691in}{4.512815in}}{\pgfqpoint{0.535301in}{4.523414in}}{\pgfqpoint{0.527487in}{4.531228in}}%
\pgfpathcurveto{\pgfqpoint{0.519673in}{4.539041in}}{\pgfqpoint{0.509074in}{4.543432in}}{\pgfqpoint{0.498024in}{4.543432in}}%
\pgfpathcurveto{\pgfqpoint{0.486974in}{4.543432in}}{\pgfqpoint{0.476375in}{4.539041in}}{\pgfqpoint{0.468561in}{4.531228in}}%
\pgfpathcurveto{\pgfqpoint{0.460748in}{4.523414in}}{\pgfqpoint{0.456358in}{4.512815in}}{\pgfqpoint{0.456358in}{4.501765in}}%
\pgfpathcurveto{\pgfqpoint{0.456358in}{4.490715in}}{\pgfqpoint{0.460748in}{4.480116in}}{\pgfqpoint{0.468561in}{4.472302in}}%
\pgfpathcurveto{\pgfqpoint{0.476375in}{4.464488in}}{\pgfqpoint{0.486974in}{4.460098in}}{\pgfqpoint{0.498024in}{4.460098in}}%
\pgfpathlineto{\pgfqpoint{0.498024in}{4.460098in}}%
\pgfpathclose%
\pgfusepath{stroke}%
\end{pgfscope}%
\begin{pgfscope}%
\pgfpathrectangle{\pgfqpoint{0.494722in}{0.437222in}}{\pgfqpoint{6.275590in}{5.159444in}}%
\pgfusepath{clip}%
\pgfsetbuttcap%
\pgfsetroundjoin%
\pgfsetlinewidth{1.003750pt}%
\definecolor{currentstroke}{rgb}{0.827451,0.827451,0.827451}%
\pgfsetstrokecolor{currentstroke}%
\pgfsetstrokeopacity{0.800000}%
\pgfsetdash{}{0pt}%
\pgfpathmoveto{\pgfqpoint{5.425914in}{0.407036in}}%
\pgfpathcurveto{\pgfqpoint{5.436964in}{0.407036in}}{\pgfqpoint{5.447563in}{0.411426in}}{\pgfqpoint{5.455377in}{0.419240in}}%
\pgfpathcurveto{\pgfqpoint{5.463191in}{0.427053in}}{\pgfqpoint{5.467581in}{0.437652in}}{\pgfqpoint{5.467581in}{0.448702in}}%
\pgfpathcurveto{\pgfqpoint{5.467581in}{0.459753in}}{\pgfqpoint{5.463191in}{0.470352in}}{\pgfqpoint{5.455377in}{0.478165in}}%
\pgfpathcurveto{\pgfqpoint{5.447563in}{0.485979in}}{\pgfqpoint{5.436964in}{0.490369in}}{\pgfqpoint{5.425914in}{0.490369in}}%
\pgfpathcurveto{\pgfqpoint{5.414864in}{0.490369in}}{\pgfqpoint{5.404265in}{0.485979in}}{\pgfqpoint{5.396451in}{0.478165in}}%
\pgfpathcurveto{\pgfqpoint{5.388638in}{0.470352in}}{\pgfqpoint{5.384247in}{0.459753in}}{\pgfqpoint{5.384247in}{0.448702in}}%
\pgfpathcurveto{\pgfqpoint{5.384247in}{0.437652in}}{\pgfqpoint{5.388638in}{0.427053in}}{\pgfqpoint{5.396451in}{0.419240in}}%
\pgfpathcurveto{\pgfqpoint{5.404265in}{0.411426in}}{\pgfqpoint{5.414864in}{0.407036in}}{\pgfqpoint{5.425914in}{0.407036in}}%
\pgfusepath{stroke}%
\end{pgfscope}%
\begin{pgfscope}%
\pgfpathrectangle{\pgfqpoint{0.494722in}{0.437222in}}{\pgfqpoint{6.275590in}{5.159444in}}%
\pgfusepath{clip}%
\pgfsetbuttcap%
\pgfsetroundjoin%
\pgfsetlinewidth{1.003750pt}%
\definecolor{currentstroke}{rgb}{0.827451,0.827451,0.827451}%
\pgfsetstrokecolor{currentstroke}%
\pgfsetstrokeopacity{0.800000}%
\pgfsetdash{}{0pt}%
\pgfpathmoveto{\pgfqpoint{4.010167in}{0.576865in}}%
\pgfpathcurveto{\pgfqpoint{4.021217in}{0.576865in}}{\pgfqpoint{4.031816in}{0.581255in}}{\pgfqpoint{4.039630in}{0.589069in}}%
\pgfpathcurveto{\pgfqpoint{4.047443in}{0.596882in}}{\pgfqpoint{4.051834in}{0.607481in}}{\pgfqpoint{4.051834in}{0.618531in}}%
\pgfpathcurveto{\pgfqpoint{4.051834in}{0.629581in}}{\pgfqpoint{4.047443in}{0.640180in}}{\pgfqpoint{4.039630in}{0.647994in}}%
\pgfpathcurveto{\pgfqpoint{4.031816in}{0.655808in}}{\pgfqpoint{4.021217in}{0.660198in}}{\pgfqpoint{4.010167in}{0.660198in}}%
\pgfpathcurveto{\pgfqpoint{3.999117in}{0.660198in}}{\pgfqpoint{3.988518in}{0.655808in}}{\pgfqpoint{3.980704in}{0.647994in}}%
\pgfpathcurveto{\pgfqpoint{3.972890in}{0.640180in}}{\pgfqpoint{3.968500in}{0.629581in}}{\pgfqpoint{3.968500in}{0.618531in}}%
\pgfpathcurveto{\pgfqpoint{3.968500in}{0.607481in}}{\pgfqpoint{3.972890in}{0.596882in}}{\pgfqpoint{3.980704in}{0.589069in}}%
\pgfpathcurveto{\pgfqpoint{3.988518in}{0.581255in}}{\pgfqpoint{3.999117in}{0.576865in}}{\pgfqpoint{4.010167in}{0.576865in}}%
\pgfpathlineto{\pgfqpoint{4.010167in}{0.576865in}}%
\pgfpathclose%
\pgfusepath{stroke}%
\end{pgfscope}%
\begin{pgfscope}%
\pgfpathrectangle{\pgfqpoint{0.494722in}{0.437222in}}{\pgfqpoint{6.275590in}{5.159444in}}%
\pgfusepath{clip}%
\pgfsetbuttcap%
\pgfsetroundjoin%
\pgfsetlinewidth{1.003750pt}%
\definecolor{currentstroke}{rgb}{0.827451,0.827451,0.827451}%
\pgfsetstrokecolor{currentstroke}%
\pgfsetstrokeopacity{0.800000}%
\pgfsetdash{}{0pt}%
\pgfpathmoveto{\pgfqpoint{0.520093in}{4.158669in}}%
\pgfpathcurveto{\pgfqpoint{0.531143in}{4.158669in}}{\pgfqpoint{0.541742in}{4.163059in}}{\pgfqpoint{0.549556in}{4.170873in}}%
\pgfpathcurveto{\pgfqpoint{0.557369in}{4.178687in}}{\pgfqpoint{0.561759in}{4.189286in}}{\pgfqpoint{0.561759in}{4.200336in}}%
\pgfpathcurveto{\pgfqpoint{0.561759in}{4.211386in}}{\pgfqpoint{0.557369in}{4.221985in}}{\pgfqpoint{0.549556in}{4.229799in}}%
\pgfpathcurveto{\pgfqpoint{0.541742in}{4.237612in}}{\pgfqpoint{0.531143in}{4.242003in}}{\pgfqpoint{0.520093in}{4.242003in}}%
\pgfpathcurveto{\pgfqpoint{0.509043in}{4.242003in}}{\pgfqpoint{0.498444in}{4.237612in}}{\pgfqpoint{0.490630in}{4.229799in}}%
\pgfpathcurveto{\pgfqpoint{0.482816in}{4.221985in}}{\pgfqpoint{0.478426in}{4.211386in}}{\pgfqpoint{0.478426in}{4.200336in}}%
\pgfpathcurveto{\pgfqpoint{0.478426in}{4.189286in}}{\pgfqpoint{0.482816in}{4.178687in}}{\pgfqpoint{0.490630in}{4.170873in}}%
\pgfpathcurveto{\pgfqpoint{0.498444in}{4.163059in}}{\pgfqpoint{0.509043in}{4.158669in}}{\pgfqpoint{0.520093in}{4.158669in}}%
\pgfpathlineto{\pgfqpoint{0.520093in}{4.158669in}}%
\pgfpathclose%
\pgfusepath{stroke}%
\end{pgfscope}%
\begin{pgfscope}%
\pgfpathrectangle{\pgfqpoint{0.494722in}{0.437222in}}{\pgfqpoint{6.275590in}{5.159444in}}%
\pgfusepath{clip}%
\pgfsetbuttcap%
\pgfsetroundjoin%
\pgfsetlinewidth{1.003750pt}%
\definecolor{currentstroke}{rgb}{0.827451,0.827451,0.827451}%
\pgfsetstrokecolor{currentstroke}%
\pgfsetstrokeopacity{0.800000}%
\pgfsetdash{}{0pt}%
\pgfpathmoveto{\pgfqpoint{0.770298in}{3.077236in}}%
\pgfpathcurveto{\pgfqpoint{0.781349in}{3.077236in}}{\pgfqpoint{0.791948in}{3.081626in}}{\pgfqpoint{0.799761in}{3.089440in}}%
\pgfpathcurveto{\pgfqpoint{0.807575in}{3.097254in}}{\pgfqpoint{0.811965in}{3.107853in}}{\pgfqpoint{0.811965in}{3.118903in}}%
\pgfpathcurveto{\pgfqpoint{0.811965in}{3.129953in}}{\pgfqpoint{0.807575in}{3.140552in}}{\pgfqpoint{0.799761in}{3.148366in}}%
\pgfpathcurveto{\pgfqpoint{0.791948in}{3.156179in}}{\pgfqpoint{0.781349in}{3.160569in}}{\pgfqpoint{0.770298in}{3.160569in}}%
\pgfpathcurveto{\pgfqpoint{0.759248in}{3.160569in}}{\pgfqpoint{0.748649in}{3.156179in}}{\pgfqpoint{0.740836in}{3.148366in}}%
\pgfpathcurveto{\pgfqpoint{0.733022in}{3.140552in}}{\pgfqpoint{0.728632in}{3.129953in}}{\pgfqpoint{0.728632in}{3.118903in}}%
\pgfpathcurveto{\pgfqpoint{0.728632in}{3.107853in}}{\pgfqpoint{0.733022in}{3.097254in}}{\pgfqpoint{0.740836in}{3.089440in}}%
\pgfpathcurveto{\pgfqpoint{0.748649in}{3.081626in}}{\pgfqpoint{0.759248in}{3.077236in}}{\pgfqpoint{0.770298in}{3.077236in}}%
\pgfpathlineto{\pgfqpoint{0.770298in}{3.077236in}}%
\pgfpathclose%
\pgfusepath{stroke}%
\end{pgfscope}%
\begin{pgfscope}%
\pgfpathrectangle{\pgfqpoint{0.494722in}{0.437222in}}{\pgfqpoint{6.275590in}{5.159444in}}%
\pgfusepath{clip}%
\pgfsetbuttcap%
\pgfsetroundjoin%
\pgfsetlinewidth{1.003750pt}%
\definecolor{currentstroke}{rgb}{0.827451,0.827451,0.827451}%
\pgfsetstrokecolor{currentstroke}%
\pgfsetstrokeopacity{0.800000}%
\pgfsetdash{}{0pt}%
\pgfpathmoveto{\pgfqpoint{5.183443in}{0.422556in}}%
\pgfpathcurveto{\pgfqpoint{5.194493in}{0.422556in}}{\pgfqpoint{5.205092in}{0.426946in}}{\pgfqpoint{5.212906in}{0.434760in}}%
\pgfpathcurveto{\pgfqpoint{5.220719in}{0.442573in}}{\pgfqpoint{5.225109in}{0.453172in}}{\pgfqpoint{5.225109in}{0.464222in}}%
\pgfpathcurveto{\pgfqpoint{5.225109in}{0.475273in}}{\pgfqpoint{5.220719in}{0.485872in}}{\pgfqpoint{5.212906in}{0.493685in}}%
\pgfpathcurveto{\pgfqpoint{5.205092in}{0.501499in}}{\pgfqpoint{5.194493in}{0.505889in}}{\pgfqpoint{5.183443in}{0.505889in}}%
\pgfpathcurveto{\pgfqpoint{5.172393in}{0.505889in}}{\pgfqpoint{5.161794in}{0.501499in}}{\pgfqpoint{5.153980in}{0.493685in}}%
\pgfpathcurveto{\pgfqpoint{5.146166in}{0.485872in}}{\pgfqpoint{5.141776in}{0.475273in}}{\pgfqpoint{5.141776in}{0.464222in}}%
\pgfpathcurveto{\pgfqpoint{5.141776in}{0.453172in}}{\pgfqpoint{5.146166in}{0.442573in}}{\pgfqpoint{5.153980in}{0.434760in}}%
\pgfpathcurveto{\pgfqpoint{5.161794in}{0.426946in}}{\pgfqpoint{5.172393in}{0.422556in}}{\pgfqpoint{5.183443in}{0.422556in}}%
\pgfusepath{stroke}%
\end{pgfscope}%
\begin{pgfscope}%
\pgfpathrectangle{\pgfqpoint{0.494722in}{0.437222in}}{\pgfqpoint{6.275590in}{5.159444in}}%
\pgfusepath{clip}%
\pgfsetbuttcap%
\pgfsetroundjoin%
\pgfsetlinewidth{1.003750pt}%
\definecolor{currentstroke}{rgb}{0.827451,0.827451,0.827451}%
\pgfsetstrokecolor{currentstroke}%
\pgfsetstrokeopacity{0.800000}%
\pgfsetdash{}{0pt}%
\pgfpathmoveto{\pgfqpoint{4.702565in}{0.471432in}}%
\pgfpathcurveto{\pgfqpoint{4.713615in}{0.471432in}}{\pgfqpoint{4.724214in}{0.475823in}}{\pgfqpoint{4.732028in}{0.483636in}}%
\pgfpathcurveto{\pgfqpoint{4.739841in}{0.491450in}}{\pgfqpoint{4.744232in}{0.502049in}}{\pgfqpoint{4.744232in}{0.513099in}}%
\pgfpathcurveto{\pgfqpoint{4.744232in}{0.524149in}}{\pgfqpoint{4.739841in}{0.534748in}}{\pgfqpoint{4.732028in}{0.542562in}}%
\pgfpathcurveto{\pgfqpoint{4.724214in}{0.550376in}}{\pgfqpoint{4.713615in}{0.554766in}}{\pgfqpoint{4.702565in}{0.554766in}}%
\pgfpathcurveto{\pgfqpoint{4.691515in}{0.554766in}}{\pgfqpoint{4.680916in}{0.550376in}}{\pgfqpoint{4.673102in}{0.542562in}}%
\pgfpathcurveto{\pgfqpoint{4.665289in}{0.534748in}}{\pgfqpoint{4.660898in}{0.524149in}}{\pgfqpoint{4.660898in}{0.513099in}}%
\pgfpathcurveto{\pgfqpoint{4.660898in}{0.502049in}}{\pgfqpoint{4.665289in}{0.491450in}}{\pgfqpoint{4.673102in}{0.483636in}}%
\pgfpathcurveto{\pgfqpoint{4.680916in}{0.475823in}}{\pgfqpoint{4.691515in}{0.471432in}}{\pgfqpoint{4.702565in}{0.471432in}}%
\pgfpathlineto{\pgfqpoint{4.702565in}{0.471432in}}%
\pgfpathclose%
\pgfusepath{stroke}%
\end{pgfscope}%
\begin{pgfscope}%
\pgfpathrectangle{\pgfqpoint{0.494722in}{0.437222in}}{\pgfqpoint{6.275590in}{5.159444in}}%
\pgfusepath{clip}%
\pgfsetbuttcap%
\pgfsetroundjoin%
\pgfsetlinewidth{1.003750pt}%
\definecolor{currentstroke}{rgb}{0.827451,0.827451,0.827451}%
\pgfsetstrokecolor{currentstroke}%
\pgfsetstrokeopacity{0.800000}%
\pgfsetdash{}{0pt}%
\pgfpathmoveto{\pgfqpoint{0.862000in}{2.887614in}}%
\pgfpathcurveto{\pgfqpoint{0.873051in}{2.887614in}}{\pgfqpoint{0.883650in}{2.892005in}}{\pgfqpoint{0.891463in}{2.899818in}}%
\pgfpathcurveto{\pgfqpoint{0.899277in}{2.907632in}}{\pgfqpoint{0.903667in}{2.918231in}}{\pgfqpoint{0.903667in}{2.929281in}}%
\pgfpathcurveto{\pgfqpoint{0.903667in}{2.940331in}}{\pgfqpoint{0.899277in}{2.950930in}}{\pgfqpoint{0.891463in}{2.958744in}}%
\pgfpathcurveto{\pgfqpoint{0.883650in}{2.966557in}}{\pgfqpoint{0.873051in}{2.970948in}}{\pgfqpoint{0.862000in}{2.970948in}}%
\pgfpathcurveto{\pgfqpoint{0.850950in}{2.970948in}}{\pgfqpoint{0.840351in}{2.966557in}}{\pgfqpoint{0.832538in}{2.958744in}}%
\pgfpathcurveto{\pgfqpoint{0.824724in}{2.950930in}}{\pgfqpoint{0.820334in}{2.940331in}}{\pgfqpoint{0.820334in}{2.929281in}}%
\pgfpathcurveto{\pgfqpoint{0.820334in}{2.918231in}}{\pgfqpoint{0.824724in}{2.907632in}}{\pgfqpoint{0.832538in}{2.899818in}}%
\pgfpathcurveto{\pgfqpoint{0.840351in}{2.892005in}}{\pgfqpoint{0.850950in}{2.887614in}}{\pgfqpoint{0.862000in}{2.887614in}}%
\pgfpathlineto{\pgfqpoint{0.862000in}{2.887614in}}%
\pgfpathclose%
\pgfusepath{stroke}%
\end{pgfscope}%
\begin{pgfscope}%
\pgfpathrectangle{\pgfqpoint{0.494722in}{0.437222in}}{\pgfqpoint{6.275590in}{5.159444in}}%
\pgfusepath{clip}%
\pgfsetbuttcap%
\pgfsetroundjoin%
\pgfsetlinewidth{1.003750pt}%
\definecolor{currentstroke}{rgb}{0.827451,0.827451,0.827451}%
\pgfsetstrokecolor{currentstroke}%
\pgfsetstrokeopacity{0.800000}%
\pgfsetdash{}{0pt}%
\pgfpathmoveto{\pgfqpoint{1.006510in}{2.581216in}}%
\pgfpathcurveto{\pgfqpoint{1.017560in}{2.581216in}}{\pgfqpoint{1.028159in}{2.585606in}}{\pgfqpoint{1.035973in}{2.593420in}}%
\pgfpathcurveto{\pgfqpoint{1.043786in}{2.601234in}}{\pgfqpoint{1.048177in}{2.611833in}}{\pgfqpoint{1.048177in}{2.622883in}}%
\pgfpathcurveto{\pgfqpoint{1.048177in}{2.633933in}}{\pgfqpoint{1.043786in}{2.644532in}}{\pgfqpoint{1.035973in}{2.652345in}}%
\pgfpathcurveto{\pgfqpoint{1.028159in}{2.660159in}}{\pgfqpoint{1.017560in}{2.664549in}}{\pgfqpoint{1.006510in}{2.664549in}}%
\pgfpathcurveto{\pgfqpoint{0.995460in}{2.664549in}}{\pgfqpoint{0.984861in}{2.660159in}}{\pgfqpoint{0.977047in}{2.652345in}}%
\pgfpathcurveto{\pgfqpoint{0.969234in}{2.644532in}}{\pgfqpoint{0.964843in}{2.633933in}}{\pgfqpoint{0.964843in}{2.622883in}}%
\pgfpathcurveto{\pgfqpoint{0.964843in}{2.611833in}}{\pgfqpoint{0.969234in}{2.601234in}}{\pgfqpoint{0.977047in}{2.593420in}}%
\pgfpathcurveto{\pgfqpoint{0.984861in}{2.585606in}}{\pgfqpoint{0.995460in}{2.581216in}}{\pgfqpoint{1.006510in}{2.581216in}}%
\pgfpathlineto{\pgfqpoint{1.006510in}{2.581216in}}%
\pgfpathclose%
\pgfusepath{stroke}%
\end{pgfscope}%
\begin{pgfscope}%
\pgfpathrectangle{\pgfqpoint{0.494722in}{0.437222in}}{\pgfqpoint{6.275590in}{5.159444in}}%
\pgfusepath{clip}%
\pgfsetbuttcap%
\pgfsetroundjoin%
\pgfsetlinewidth{1.003750pt}%
\definecolor{currentstroke}{rgb}{0.827451,0.827451,0.827451}%
\pgfsetstrokecolor{currentstroke}%
\pgfsetstrokeopacity{0.800000}%
\pgfsetdash{}{0pt}%
\pgfpathmoveto{\pgfqpoint{1.700731in}{1.766489in}}%
\pgfpathcurveto{\pgfqpoint{1.711781in}{1.766489in}}{\pgfqpoint{1.722380in}{1.770879in}}{\pgfqpoint{1.730194in}{1.778693in}}%
\pgfpathcurveto{\pgfqpoint{1.738007in}{1.786506in}}{\pgfqpoint{1.742398in}{1.797105in}}{\pgfqpoint{1.742398in}{1.808156in}}%
\pgfpathcurveto{\pgfqpoint{1.742398in}{1.819206in}}{\pgfqpoint{1.738007in}{1.829805in}}{\pgfqpoint{1.730194in}{1.837618in}}%
\pgfpathcurveto{\pgfqpoint{1.722380in}{1.845432in}}{\pgfqpoint{1.711781in}{1.849822in}}{\pgfqpoint{1.700731in}{1.849822in}}%
\pgfpathcurveto{\pgfqpoint{1.689681in}{1.849822in}}{\pgfqpoint{1.679082in}{1.845432in}}{\pgfqpoint{1.671268in}{1.837618in}}%
\pgfpathcurveto{\pgfqpoint{1.663455in}{1.829805in}}{\pgfqpoint{1.659064in}{1.819206in}}{\pgfqpoint{1.659064in}{1.808156in}}%
\pgfpathcurveto{\pgfqpoint{1.659064in}{1.797105in}}{\pgfqpoint{1.663455in}{1.786506in}}{\pgfqpoint{1.671268in}{1.778693in}}%
\pgfpathcurveto{\pgfqpoint{1.679082in}{1.770879in}}{\pgfqpoint{1.689681in}{1.766489in}}{\pgfqpoint{1.700731in}{1.766489in}}%
\pgfpathlineto{\pgfqpoint{1.700731in}{1.766489in}}%
\pgfpathclose%
\pgfusepath{stroke}%
\end{pgfscope}%
\begin{pgfscope}%
\pgfpathrectangle{\pgfqpoint{0.494722in}{0.437222in}}{\pgfqpoint{6.275590in}{5.159444in}}%
\pgfusepath{clip}%
\pgfsetbuttcap%
\pgfsetroundjoin%
\pgfsetlinewidth{1.003750pt}%
\definecolor{currentstroke}{rgb}{0.827451,0.827451,0.827451}%
\pgfsetstrokecolor{currentstroke}%
\pgfsetstrokeopacity{0.800000}%
\pgfsetdash{}{0pt}%
\pgfpathmoveto{\pgfqpoint{0.944970in}{2.756986in}}%
\pgfpathcurveto{\pgfqpoint{0.956020in}{2.756986in}}{\pgfqpoint{0.966619in}{2.761376in}}{\pgfqpoint{0.974433in}{2.769190in}}%
\pgfpathcurveto{\pgfqpoint{0.982246in}{2.777004in}}{\pgfqpoint{0.986637in}{2.787603in}}{\pgfqpoint{0.986637in}{2.798653in}}%
\pgfpathcurveto{\pgfqpoint{0.986637in}{2.809703in}}{\pgfqpoint{0.982246in}{2.820302in}}{\pgfqpoint{0.974433in}{2.828116in}}%
\pgfpathcurveto{\pgfqpoint{0.966619in}{2.835929in}}{\pgfqpoint{0.956020in}{2.840319in}}{\pgfqpoint{0.944970in}{2.840319in}}%
\pgfpathcurveto{\pgfqpoint{0.933920in}{2.840319in}}{\pgfqpoint{0.923321in}{2.835929in}}{\pgfqpoint{0.915507in}{2.828116in}}%
\pgfpathcurveto{\pgfqpoint{0.907694in}{2.820302in}}{\pgfqpoint{0.903303in}{2.809703in}}{\pgfqpoint{0.903303in}{2.798653in}}%
\pgfpathcurveto{\pgfqpoint{0.903303in}{2.787603in}}{\pgfqpoint{0.907694in}{2.777004in}}{\pgfqpoint{0.915507in}{2.769190in}}%
\pgfpathcurveto{\pgfqpoint{0.923321in}{2.761376in}}{\pgfqpoint{0.933920in}{2.756986in}}{\pgfqpoint{0.944970in}{2.756986in}}%
\pgfpathlineto{\pgfqpoint{0.944970in}{2.756986in}}%
\pgfpathclose%
\pgfusepath{stroke}%
\end{pgfscope}%
\begin{pgfscope}%
\pgfpathrectangle{\pgfqpoint{0.494722in}{0.437222in}}{\pgfqpoint{6.275590in}{5.159444in}}%
\pgfusepath{clip}%
\pgfsetbuttcap%
\pgfsetroundjoin%
\pgfsetlinewidth{1.003750pt}%
\definecolor{currentstroke}{rgb}{0.827451,0.827451,0.827451}%
\pgfsetstrokecolor{currentstroke}%
\pgfsetstrokeopacity{0.800000}%
\pgfsetdash{}{0pt}%
\pgfpathmoveto{\pgfqpoint{4.942939in}{0.453295in}}%
\pgfpathcurveto{\pgfqpoint{4.953989in}{0.453295in}}{\pgfqpoint{4.964589in}{0.457685in}}{\pgfqpoint{4.972402in}{0.465499in}}%
\pgfpathcurveto{\pgfqpoint{4.980216in}{0.473313in}}{\pgfqpoint{4.984606in}{0.483912in}}{\pgfqpoint{4.984606in}{0.494962in}}%
\pgfpathcurveto{\pgfqpoint{4.984606in}{0.506012in}}{\pgfqpoint{4.980216in}{0.516611in}}{\pgfqpoint{4.972402in}{0.524425in}}%
\pgfpathcurveto{\pgfqpoint{4.964589in}{0.532238in}}{\pgfqpoint{4.953989in}{0.536628in}}{\pgfqpoint{4.942939in}{0.536628in}}%
\pgfpathcurveto{\pgfqpoint{4.931889in}{0.536628in}}{\pgfqpoint{4.921290in}{0.532238in}}{\pgfqpoint{4.913477in}{0.524425in}}%
\pgfpathcurveto{\pgfqpoint{4.905663in}{0.516611in}}{\pgfqpoint{4.901273in}{0.506012in}}{\pgfqpoint{4.901273in}{0.494962in}}%
\pgfpathcurveto{\pgfqpoint{4.901273in}{0.483912in}}{\pgfqpoint{4.905663in}{0.473313in}}{\pgfqpoint{4.913477in}{0.465499in}}%
\pgfpathcurveto{\pgfqpoint{4.921290in}{0.457685in}}{\pgfqpoint{4.931889in}{0.453295in}}{\pgfqpoint{4.942939in}{0.453295in}}%
\pgfpathlineto{\pgfqpoint{4.942939in}{0.453295in}}%
\pgfpathclose%
\pgfusepath{stroke}%
\end{pgfscope}%
\begin{pgfscope}%
\pgfpathrectangle{\pgfqpoint{0.494722in}{0.437222in}}{\pgfqpoint{6.275590in}{5.159444in}}%
\pgfusepath{clip}%
\pgfsetbuttcap%
\pgfsetroundjoin%
\pgfsetlinewidth{1.003750pt}%
\definecolor{currentstroke}{rgb}{0.827451,0.827451,0.827451}%
\pgfsetstrokecolor{currentstroke}%
\pgfsetstrokeopacity{0.800000}%
\pgfsetdash{}{0pt}%
\pgfpathmoveto{\pgfqpoint{0.556707in}{3.885184in}}%
\pgfpathcurveto{\pgfqpoint{0.567757in}{3.885184in}}{\pgfqpoint{0.578356in}{3.889574in}}{\pgfqpoint{0.586170in}{3.897388in}}%
\pgfpathcurveto{\pgfqpoint{0.593983in}{3.905201in}}{\pgfqpoint{0.598374in}{3.915800in}}{\pgfqpoint{0.598374in}{3.926850in}}%
\pgfpathcurveto{\pgfqpoint{0.598374in}{3.937900in}}{\pgfqpoint{0.593983in}{3.948499in}}{\pgfqpoint{0.586170in}{3.956313in}}%
\pgfpathcurveto{\pgfqpoint{0.578356in}{3.964127in}}{\pgfqpoint{0.567757in}{3.968517in}}{\pgfqpoint{0.556707in}{3.968517in}}%
\pgfpathcurveto{\pgfqpoint{0.545657in}{3.968517in}}{\pgfqpoint{0.535058in}{3.964127in}}{\pgfqpoint{0.527244in}{3.956313in}}%
\pgfpathcurveto{\pgfqpoint{0.519430in}{3.948499in}}{\pgfqpoint{0.515040in}{3.937900in}}{\pgfqpoint{0.515040in}{3.926850in}}%
\pgfpathcurveto{\pgfqpoint{0.515040in}{3.915800in}}{\pgfqpoint{0.519430in}{3.905201in}}{\pgfqpoint{0.527244in}{3.897388in}}%
\pgfpathcurveto{\pgfqpoint{0.535058in}{3.889574in}}{\pgfqpoint{0.545657in}{3.885184in}}{\pgfqpoint{0.556707in}{3.885184in}}%
\pgfpathlineto{\pgfqpoint{0.556707in}{3.885184in}}%
\pgfpathclose%
\pgfusepath{stroke}%
\end{pgfscope}%
\begin{pgfscope}%
\pgfpathrectangle{\pgfqpoint{0.494722in}{0.437222in}}{\pgfqpoint{6.275590in}{5.159444in}}%
\pgfusepath{clip}%
\pgfsetbuttcap%
\pgfsetroundjoin%
\pgfsetlinewidth{1.003750pt}%
\definecolor{currentstroke}{rgb}{0.827451,0.827451,0.827451}%
\pgfsetstrokecolor{currentstroke}%
\pgfsetstrokeopacity{0.800000}%
\pgfsetdash{}{0pt}%
\pgfpathmoveto{\pgfqpoint{4.222858in}{0.545257in}}%
\pgfpathcurveto{\pgfqpoint{4.233908in}{0.545257in}}{\pgfqpoint{4.244507in}{0.549647in}}{\pgfqpoint{4.252321in}{0.557461in}}%
\pgfpathcurveto{\pgfqpoint{4.260134in}{0.565274in}}{\pgfqpoint{4.264525in}{0.575873in}}{\pgfqpoint{4.264525in}{0.586923in}}%
\pgfpathcurveto{\pgfqpoint{4.264525in}{0.597974in}}{\pgfqpoint{4.260134in}{0.608573in}}{\pgfqpoint{4.252321in}{0.616386in}}%
\pgfpathcurveto{\pgfqpoint{4.244507in}{0.624200in}}{\pgfqpoint{4.233908in}{0.628590in}}{\pgfqpoint{4.222858in}{0.628590in}}%
\pgfpathcurveto{\pgfqpoint{4.211808in}{0.628590in}}{\pgfqpoint{4.201209in}{0.624200in}}{\pgfqpoint{4.193395in}{0.616386in}}%
\pgfpathcurveto{\pgfqpoint{4.185581in}{0.608573in}}{\pgfqpoint{4.181191in}{0.597974in}}{\pgfqpoint{4.181191in}{0.586923in}}%
\pgfpathcurveto{\pgfqpoint{4.181191in}{0.575873in}}{\pgfqpoint{4.185581in}{0.565274in}}{\pgfqpoint{4.193395in}{0.557461in}}%
\pgfpathcurveto{\pgfqpoint{4.201209in}{0.549647in}}{\pgfqpoint{4.211808in}{0.545257in}}{\pgfqpoint{4.222858in}{0.545257in}}%
\pgfpathlineto{\pgfqpoint{4.222858in}{0.545257in}}%
\pgfpathclose%
\pgfusepath{stroke}%
\end{pgfscope}%
\begin{pgfscope}%
\pgfpathrectangle{\pgfqpoint{0.494722in}{0.437222in}}{\pgfqpoint{6.275590in}{5.159444in}}%
\pgfusepath{clip}%
\pgfsetbuttcap%
\pgfsetroundjoin%
\pgfsetlinewidth{1.003750pt}%
\definecolor{currentstroke}{rgb}{0.827451,0.827451,0.827451}%
\pgfsetstrokecolor{currentstroke}%
\pgfsetstrokeopacity{0.800000}%
\pgfsetdash{}{0pt}%
\pgfpathmoveto{\pgfqpoint{4.499239in}{0.480344in}}%
\pgfpathcurveto{\pgfqpoint{4.510290in}{0.480344in}}{\pgfqpoint{4.520889in}{0.484734in}}{\pgfqpoint{4.528702in}{0.492548in}}%
\pgfpathcurveto{\pgfqpoint{4.536516in}{0.500361in}}{\pgfqpoint{4.540906in}{0.510960in}}{\pgfqpoint{4.540906in}{0.522011in}}%
\pgfpathcurveto{\pgfqpoint{4.540906in}{0.533061in}}{\pgfqpoint{4.536516in}{0.543660in}}{\pgfqpoint{4.528702in}{0.551473in}}%
\pgfpathcurveto{\pgfqpoint{4.520889in}{0.559287in}}{\pgfqpoint{4.510290in}{0.563677in}}{\pgfqpoint{4.499239in}{0.563677in}}%
\pgfpathcurveto{\pgfqpoint{4.488189in}{0.563677in}}{\pgfqpoint{4.477590in}{0.559287in}}{\pgfqpoint{4.469777in}{0.551473in}}%
\pgfpathcurveto{\pgfqpoint{4.461963in}{0.543660in}}{\pgfqpoint{4.457573in}{0.533061in}}{\pgfqpoint{4.457573in}{0.522011in}}%
\pgfpathcurveto{\pgfqpoint{4.457573in}{0.510960in}}{\pgfqpoint{4.461963in}{0.500361in}}{\pgfqpoint{4.469777in}{0.492548in}}%
\pgfpathcurveto{\pgfqpoint{4.477590in}{0.484734in}}{\pgfqpoint{4.488189in}{0.480344in}}{\pgfqpoint{4.499239in}{0.480344in}}%
\pgfpathlineto{\pgfqpoint{4.499239in}{0.480344in}}%
\pgfpathclose%
\pgfusepath{stroke}%
\end{pgfscope}%
\begin{pgfscope}%
\pgfpathrectangle{\pgfqpoint{0.494722in}{0.437222in}}{\pgfqpoint{6.275590in}{5.159444in}}%
\pgfusepath{clip}%
\pgfsetbuttcap%
\pgfsetroundjoin%
\pgfsetlinewidth{1.003750pt}%
\definecolor{currentstroke}{rgb}{0.827451,0.827451,0.827451}%
\pgfsetstrokecolor{currentstroke}%
\pgfsetstrokeopacity{0.800000}%
\pgfsetdash{}{0pt}%
\pgfpathmoveto{\pgfqpoint{2.765798in}{1.011077in}}%
\pgfpathcurveto{\pgfqpoint{2.776848in}{1.011077in}}{\pgfqpoint{2.787447in}{1.015467in}}{\pgfqpoint{2.795261in}{1.023280in}}%
\pgfpathcurveto{\pgfqpoint{2.803075in}{1.031094in}}{\pgfqpoint{2.807465in}{1.041693in}}{\pgfqpoint{2.807465in}{1.052743in}}%
\pgfpathcurveto{\pgfqpoint{2.807465in}{1.063793in}}{\pgfqpoint{2.803075in}{1.074392in}}{\pgfqpoint{2.795261in}{1.082206in}}%
\pgfpathcurveto{\pgfqpoint{2.787447in}{1.090020in}}{\pgfqpoint{2.776848in}{1.094410in}}{\pgfqpoint{2.765798in}{1.094410in}}%
\pgfpathcurveto{\pgfqpoint{2.754748in}{1.094410in}}{\pgfqpoint{2.744149in}{1.090020in}}{\pgfqpoint{2.736335in}{1.082206in}}%
\pgfpathcurveto{\pgfqpoint{2.728522in}{1.074392in}}{\pgfqpoint{2.724132in}{1.063793in}}{\pgfqpoint{2.724132in}{1.052743in}}%
\pgfpathcurveto{\pgfqpoint{2.724132in}{1.041693in}}{\pgfqpoint{2.728522in}{1.031094in}}{\pgfqpoint{2.736335in}{1.023280in}}%
\pgfpathcurveto{\pgfqpoint{2.744149in}{1.015467in}}{\pgfqpoint{2.754748in}{1.011077in}}{\pgfqpoint{2.765798in}{1.011077in}}%
\pgfpathlineto{\pgfqpoint{2.765798in}{1.011077in}}%
\pgfpathclose%
\pgfusepath{stroke}%
\end{pgfscope}%
\begin{pgfscope}%
\pgfpathrectangle{\pgfqpoint{0.494722in}{0.437222in}}{\pgfqpoint{6.275590in}{5.159444in}}%
\pgfusepath{clip}%
\pgfsetbuttcap%
\pgfsetroundjoin%
\pgfsetlinewidth{1.003750pt}%
\definecolor{currentstroke}{rgb}{0.827451,0.827451,0.827451}%
\pgfsetstrokecolor{currentstroke}%
\pgfsetstrokeopacity{0.800000}%
\pgfsetdash{}{0pt}%
\pgfpathmoveto{\pgfqpoint{3.832585in}{0.609011in}}%
\pgfpathcurveto{\pgfqpoint{3.843635in}{0.609011in}}{\pgfqpoint{3.854234in}{0.613402in}}{\pgfqpoint{3.862047in}{0.621215in}}%
\pgfpathcurveto{\pgfqpoint{3.869861in}{0.629029in}}{\pgfqpoint{3.874251in}{0.639628in}}{\pgfqpoint{3.874251in}{0.650678in}}%
\pgfpathcurveto{\pgfqpoint{3.874251in}{0.661728in}}{\pgfqpoint{3.869861in}{0.672327in}}{\pgfqpoint{3.862047in}{0.680141in}}%
\pgfpathcurveto{\pgfqpoint{3.854234in}{0.687954in}}{\pgfqpoint{3.843635in}{0.692345in}}{\pgfqpoint{3.832585in}{0.692345in}}%
\pgfpathcurveto{\pgfqpoint{3.821534in}{0.692345in}}{\pgfqpoint{3.810935in}{0.687954in}}{\pgfqpoint{3.803122in}{0.680141in}}%
\pgfpathcurveto{\pgfqpoint{3.795308in}{0.672327in}}{\pgfqpoint{3.790918in}{0.661728in}}{\pgfqpoint{3.790918in}{0.650678in}}%
\pgfpathcurveto{\pgfqpoint{3.790918in}{0.639628in}}{\pgfqpoint{3.795308in}{0.629029in}}{\pgfqpoint{3.803122in}{0.621215in}}%
\pgfpathcurveto{\pgfqpoint{3.810935in}{0.613402in}}{\pgfqpoint{3.821534in}{0.609011in}}{\pgfqpoint{3.832585in}{0.609011in}}%
\pgfpathlineto{\pgfqpoint{3.832585in}{0.609011in}}%
\pgfpathclose%
\pgfusepath{stroke}%
\end{pgfscope}%
\begin{pgfscope}%
\pgfpathrectangle{\pgfqpoint{0.494722in}{0.437222in}}{\pgfqpoint{6.275590in}{5.159444in}}%
\pgfusepath{clip}%
\pgfsetbuttcap%
\pgfsetroundjoin%
\pgfsetlinewidth{1.003750pt}%
\definecolor{currentstroke}{rgb}{0.827451,0.827451,0.827451}%
\pgfsetstrokecolor{currentstroke}%
\pgfsetstrokeopacity{0.800000}%
\pgfsetdash{}{0pt}%
\pgfpathmoveto{\pgfqpoint{1.006340in}{2.736690in}}%
\pgfpathcurveto{\pgfqpoint{1.017390in}{2.736690in}}{\pgfqpoint{1.027989in}{2.741080in}}{\pgfqpoint{1.035803in}{2.748894in}}%
\pgfpathcurveto{\pgfqpoint{1.043616in}{2.756708in}}{\pgfqpoint{1.048007in}{2.767307in}}{\pgfqpoint{1.048007in}{2.778357in}}%
\pgfpathcurveto{\pgfqpoint{1.048007in}{2.789407in}}{\pgfqpoint{1.043616in}{2.800006in}}{\pgfqpoint{1.035803in}{2.807819in}}%
\pgfpathcurveto{\pgfqpoint{1.027989in}{2.815633in}}{\pgfqpoint{1.017390in}{2.820023in}}{\pgfqpoint{1.006340in}{2.820023in}}%
\pgfpathcurveto{\pgfqpoint{0.995290in}{2.820023in}}{\pgfqpoint{0.984691in}{2.815633in}}{\pgfqpoint{0.976877in}{2.807819in}}%
\pgfpathcurveto{\pgfqpoint{0.969064in}{2.800006in}}{\pgfqpoint{0.964673in}{2.789407in}}{\pgfqpoint{0.964673in}{2.778357in}}%
\pgfpathcurveto{\pgfqpoint{0.964673in}{2.767307in}}{\pgfqpoint{0.969064in}{2.756708in}}{\pgfqpoint{0.976877in}{2.748894in}}%
\pgfpathcurveto{\pgfqpoint{0.984691in}{2.741080in}}{\pgfqpoint{0.995290in}{2.736690in}}{\pgfqpoint{1.006340in}{2.736690in}}%
\pgfpathlineto{\pgfqpoint{1.006340in}{2.736690in}}%
\pgfpathclose%
\pgfusepath{stroke}%
\end{pgfscope}%
\begin{pgfscope}%
\pgfpathrectangle{\pgfqpoint{0.494722in}{0.437222in}}{\pgfqpoint{6.275590in}{5.159444in}}%
\pgfusepath{clip}%
\pgfsetbuttcap%
\pgfsetroundjoin%
\pgfsetlinewidth{1.003750pt}%
\definecolor{currentstroke}{rgb}{0.827451,0.827451,0.827451}%
\pgfsetstrokecolor{currentstroke}%
\pgfsetstrokeopacity{0.800000}%
\pgfsetdash{}{0pt}%
\pgfpathmoveto{\pgfqpoint{3.709744in}{0.686390in}}%
\pgfpathcurveto{\pgfqpoint{3.720794in}{0.686390in}}{\pgfqpoint{3.731393in}{0.690780in}}{\pgfqpoint{3.739207in}{0.698593in}}%
\pgfpathcurveto{\pgfqpoint{3.747021in}{0.706407in}}{\pgfqpoint{3.751411in}{0.717006in}}{\pgfqpoint{3.751411in}{0.728056in}}%
\pgfpathcurveto{\pgfqpoint{3.751411in}{0.739106in}}{\pgfqpoint{3.747021in}{0.749705in}}{\pgfqpoint{3.739207in}{0.757519in}}%
\pgfpathcurveto{\pgfqpoint{3.731393in}{0.765333in}}{\pgfqpoint{3.720794in}{0.769723in}}{\pgfqpoint{3.709744in}{0.769723in}}%
\pgfpathcurveto{\pgfqpoint{3.698694in}{0.769723in}}{\pgfqpoint{3.688095in}{0.765333in}}{\pgfqpoint{3.680281in}{0.757519in}}%
\pgfpathcurveto{\pgfqpoint{3.672468in}{0.749705in}}{\pgfqpoint{3.668077in}{0.739106in}}{\pgfqpoint{3.668077in}{0.728056in}}%
\pgfpathcurveto{\pgfqpoint{3.668077in}{0.717006in}}{\pgfqpoint{3.672468in}{0.706407in}}{\pgfqpoint{3.680281in}{0.698593in}}%
\pgfpathcurveto{\pgfqpoint{3.688095in}{0.690780in}}{\pgfqpoint{3.698694in}{0.686390in}}{\pgfqpoint{3.709744in}{0.686390in}}%
\pgfpathlineto{\pgfqpoint{3.709744in}{0.686390in}}%
\pgfpathclose%
\pgfusepath{stroke}%
\end{pgfscope}%
\begin{pgfscope}%
\pgfpathrectangle{\pgfqpoint{0.494722in}{0.437222in}}{\pgfqpoint{6.275590in}{5.159444in}}%
\pgfusepath{clip}%
\pgfsetbuttcap%
\pgfsetroundjoin%
\pgfsetlinewidth{1.003750pt}%
\definecolor{currentstroke}{rgb}{0.827451,0.827451,0.827451}%
\pgfsetstrokecolor{currentstroke}%
\pgfsetstrokeopacity{0.800000}%
\pgfsetdash{}{0pt}%
\pgfpathmoveto{\pgfqpoint{3.022481in}{0.888267in}}%
\pgfpathcurveto{\pgfqpoint{3.033531in}{0.888267in}}{\pgfqpoint{3.044130in}{0.892657in}}{\pgfqpoint{3.051944in}{0.900471in}}%
\pgfpathcurveto{\pgfqpoint{3.059758in}{0.908285in}}{\pgfqpoint{3.064148in}{0.918884in}}{\pgfqpoint{3.064148in}{0.929934in}}%
\pgfpathcurveto{\pgfqpoint{3.064148in}{0.940984in}}{\pgfqpoint{3.059758in}{0.951583in}}{\pgfqpoint{3.051944in}{0.959397in}}%
\pgfpathcurveto{\pgfqpoint{3.044130in}{0.967210in}}{\pgfqpoint{3.033531in}{0.971600in}}{\pgfqpoint{3.022481in}{0.971600in}}%
\pgfpathcurveto{\pgfqpoint{3.011431in}{0.971600in}}{\pgfqpoint{3.000832in}{0.967210in}}{\pgfqpoint{2.993019in}{0.959397in}}%
\pgfpathcurveto{\pgfqpoint{2.985205in}{0.951583in}}{\pgfqpoint{2.980815in}{0.940984in}}{\pgfqpoint{2.980815in}{0.929934in}}%
\pgfpathcurveto{\pgfqpoint{2.980815in}{0.918884in}}{\pgfqpoint{2.985205in}{0.908285in}}{\pgfqpoint{2.993019in}{0.900471in}}%
\pgfpathcurveto{\pgfqpoint{3.000832in}{0.892657in}}{\pgfqpoint{3.011431in}{0.888267in}}{\pgfqpoint{3.022481in}{0.888267in}}%
\pgfpathlineto{\pgfqpoint{3.022481in}{0.888267in}}%
\pgfpathclose%
\pgfusepath{stroke}%
\end{pgfscope}%
\begin{pgfscope}%
\pgfpathrectangle{\pgfqpoint{0.494722in}{0.437222in}}{\pgfqpoint{6.275590in}{5.159444in}}%
\pgfusepath{clip}%
\pgfsetbuttcap%
\pgfsetroundjoin%
\pgfsetlinewidth{1.003750pt}%
\definecolor{currentstroke}{rgb}{0.827451,0.827451,0.827451}%
\pgfsetstrokecolor{currentstroke}%
\pgfsetstrokeopacity{0.800000}%
\pgfsetdash{}{0pt}%
\pgfpathmoveto{\pgfqpoint{1.113110in}{2.511719in}}%
\pgfpathcurveto{\pgfqpoint{1.124161in}{2.511719in}}{\pgfqpoint{1.134760in}{2.516109in}}{\pgfqpoint{1.142573in}{2.523923in}}%
\pgfpathcurveto{\pgfqpoint{1.150387in}{2.531736in}}{\pgfqpoint{1.154777in}{2.542335in}}{\pgfqpoint{1.154777in}{2.553385in}}%
\pgfpathcurveto{\pgfqpoint{1.154777in}{2.564435in}}{\pgfqpoint{1.150387in}{2.575034in}}{\pgfqpoint{1.142573in}{2.582848in}}%
\pgfpathcurveto{\pgfqpoint{1.134760in}{2.590662in}}{\pgfqpoint{1.124161in}{2.595052in}}{\pgfqpoint{1.113110in}{2.595052in}}%
\pgfpathcurveto{\pgfqpoint{1.102060in}{2.595052in}}{\pgfqpoint{1.091461in}{2.590662in}}{\pgfqpoint{1.083648in}{2.582848in}}%
\pgfpathcurveto{\pgfqpoint{1.075834in}{2.575034in}}{\pgfqpoint{1.071444in}{2.564435in}}{\pgfqpoint{1.071444in}{2.553385in}}%
\pgfpathcurveto{\pgfqpoint{1.071444in}{2.542335in}}{\pgfqpoint{1.075834in}{2.531736in}}{\pgfqpoint{1.083648in}{2.523923in}}%
\pgfpathcurveto{\pgfqpoint{1.091461in}{2.516109in}}{\pgfqpoint{1.102060in}{2.511719in}}{\pgfqpoint{1.113110in}{2.511719in}}%
\pgfpathlineto{\pgfqpoint{1.113110in}{2.511719in}}%
\pgfpathclose%
\pgfusepath{stroke}%
\end{pgfscope}%
\begin{pgfscope}%
\pgfpathrectangle{\pgfqpoint{0.494722in}{0.437222in}}{\pgfqpoint{6.275590in}{5.159444in}}%
\pgfusepath{clip}%
\pgfsetbuttcap%
\pgfsetroundjoin%
\pgfsetlinewidth{1.003750pt}%
\definecolor{currentstroke}{rgb}{0.827451,0.827451,0.827451}%
\pgfsetstrokecolor{currentstroke}%
\pgfsetstrokeopacity{0.800000}%
\pgfsetdash{}{0pt}%
\pgfpathmoveto{\pgfqpoint{2.429002in}{1.219445in}}%
\pgfpathcurveto{\pgfqpoint{2.440052in}{1.219445in}}{\pgfqpoint{2.450651in}{1.223835in}}{\pgfqpoint{2.458465in}{1.231648in}}%
\pgfpathcurveto{\pgfqpoint{2.466278in}{1.239462in}}{\pgfqpoint{2.470669in}{1.250061in}}{\pgfqpoint{2.470669in}{1.261111in}}%
\pgfpathcurveto{\pgfqpoint{2.470669in}{1.272161in}}{\pgfqpoint{2.466278in}{1.282760in}}{\pgfqpoint{2.458465in}{1.290574in}}%
\pgfpathcurveto{\pgfqpoint{2.450651in}{1.298388in}}{\pgfqpoint{2.440052in}{1.302778in}}{\pgfqpoint{2.429002in}{1.302778in}}%
\pgfpathcurveto{\pgfqpoint{2.417952in}{1.302778in}}{\pgfqpoint{2.407353in}{1.298388in}}{\pgfqpoint{2.399539in}{1.290574in}}%
\pgfpathcurveto{\pgfqpoint{2.391726in}{1.282760in}}{\pgfqpoint{2.387335in}{1.272161in}}{\pgfqpoint{2.387335in}{1.261111in}}%
\pgfpathcurveto{\pgfqpoint{2.387335in}{1.250061in}}{\pgfqpoint{2.391726in}{1.239462in}}{\pgfqpoint{2.399539in}{1.231648in}}%
\pgfpathcurveto{\pgfqpoint{2.407353in}{1.223835in}}{\pgfqpoint{2.417952in}{1.219445in}}{\pgfqpoint{2.429002in}{1.219445in}}%
\pgfpathlineto{\pgfqpoint{2.429002in}{1.219445in}}%
\pgfpathclose%
\pgfusepath{stroke}%
\end{pgfscope}%
\begin{pgfscope}%
\pgfpathrectangle{\pgfqpoint{0.494722in}{0.437222in}}{\pgfqpoint{6.275590in}{5.159444in}}%
\pgfusepath{clip}%
\pgfsetbuttcap%
\pgfsetroundjoin%
\pgfsetlinewidth{1.003750pt}%
\definecolor{currentstroke}{rgb}{0.827451,0.827451,0.827451}%
\pgfsetstrokecolor{currentstroke}%
\pgfsetstrokeopacity{0.800000}%
\pgfsetdash{}{0pt}%
\pgfpathmoveto{\pgfqpoint{1.623472in}{1.825964in}}%
\pgfpathcurveto{\pgfqpoint{1.634522in}{1.825964in}}{\pgfqpoint{1.645121in}{1.830354in}}{\pgfqpoint{1.652935in}{1.838167in}}%
\pgfpathcurveto{\pgfqpoint{1.660748in}{1.845981in}}{\pgfqpoint{1.665139in}{1.856580in}}{\pgfqpoint{1.665139in}{1.867630in}}%
\pgfpathcurveto{\pgfqpoint{1.665139in}{1.878680in}}{\pgfqpoint{1.660748in}{1.889279in}}{\pgfqpoint{1.652935in}{1.897093in}}%
\pgfpathcurveto{\pgfqpoint{1.645121in}{1.904907in}}{\pgfqpoint{1.634522in}{1.909297in}}{\pgfqpoint{1.623472in}{1.909297in}}%
\pgfpathcurveto{\pgfqpoint{1.612422in}{1.909297in}}{\pgfqpoint{1.601823in}{1.904907in}}{\pgfqpoint{1.594009in}{1.897093in}}%
\pgfpathcurveto{\pgfqpoint{1.586196in}{1.889279in}}{\pgfqpoint{1.581805in}{1.878680in}}{\pgfqpoint{1.581805in}{1.867630in}}%
\pgfpathcurveto{\pgfqpoint{1.581805in}{1.856580in}}{\pgfqpoint{1.586196in}{1.845981in}}{\pgfqpoint{1.594009in}{1.838167in}}%
\pgfpathcurveto{\pgfqpoint{1.601823in}{1.830354in}}{\pgfqpoint{1.612422in}{1.825964in}}{\pgfqpoint{1.623472in}{1.825964in}}%
\pgfpathlineto{\pgfqpoint{1.623472in}{1.825964in}}%
\pgfpathclose%
\pgfusepath{stroke}%
\end{pgfscope}%
\begin{pgfscope}%
\pgfpathrectangle{\pgfqpoint{0.494722in}{0.437222in}}{\pgfqpoint{6.275590in}{5.159444in}}%
\pgfusepath{clip}%
\pgfsetbuttcap%
\pgfsetroundjoin%
\pgfsetlinewidth{1.003750pt}%
\definecolor{currentstroke}{rgb}{0.827451,0.827451,0.827451}%
\pgfsetstrokecolor{currentstroke}%
\pgfsetstrokeopacity{0.800000}%
\pgfsetdash{}{0pt}%
\pgfpathmoveto{\pgfqpoint{0.679488in}{3.431236in}}%
\pgfpathcurveto{\pgfqpoint{0.690538in}{3.431236in}}{\pgfqpoint{0.701137in}{3.435626in}}{\pgfqpoint{0.708951in}{3.443440in}}%
\pgfpathcurveto{\pgfqpoint{0.716765in}{3.451253in}}{\pgfqpoint{0.721155in}{3.461852in}}{\pgfqpoint{0.721155in}{3.472902in}}%
\pgfpathcurveto{\pgfqpoint{0.721155in}{3.483953in}}{\pgfqpoint{0.716765in}{3.494552in}}{\pgfqpoint{0.708951in}{3.502365in}}%
\pgfpathcurveto{\pgfqpoint{0.701137in}{3.510179in}}{\pgfqpoint{0.690538in}{3.514569in}}{\pgfqpoint{0.679488in}{3.514569in}}%
\pgfpathcurveto{\pgfqpoint{0.668438in}{3.514569in}}{\pgfqpoint{0.657839in}{3.510179in}}{\pgfqpoint{0.650026in}{3.502365in}}%
\pgfpathcurveto{\pgfqpoint{0.642212in}{3.494552in}}{\pgfqpoint{0.637822in}{3.483953in}}{\pgfqpoint{0.637822in}{3.472902in}}%
\pgfpathcurveto{\pgfqpoint{0.637822in}{3.461852in}}{\pgfqpoint{0.642212in}{3.451253in}}{\pgfqpoint{0.650026in}{3.443440in}}%
\pgfpathcurveto{\pgfqpoint{0.657839in}{3.435626in}}{\pgfqpoint{0.668438in}{3.431236in}}{\pgfqpoint{0.679488in}{3.431236in}}%
\pgfpathlineto{\pgfqpoint{0.679488in}{3.431236in}}%
\pgfpathclose%
\pgfusepath{stroke}%
\end{pgfscope}%
\begin{pgfscope}%
\pgfpathrectangle{\pgfqpoint{0.494722in}{0.437222in}}{\pgfqpoint{6.275590in}{5.159444in}}%
\pgfusepath{clip}%
\pgfsetbuttcap%
\pgfsetroundjoin%
\pgfsetlinewidth{1.003750pt}%
\definecolor{currentstroke}{rgb}{0.827451,0.827451,0.827451}%
\pgfsetstrokecolor{currentstroke}%
\pgfsetstrokeopacity{0.800000}%
\pgfsetdash{}{0pt}%
\pgfpathmoveto{\pgfqpoint{0.870505in}{2.877763in}}%
\pgfpathcurveto{\pgfqpoint{0.881555in}{2.877763in}}{\pgfqpoint{0.892154in}{2.882154in}}{\pgfqpoint{0.899967in}{2.889967in}}%
\pgfpathcurveto{\pgfqpoint{0.907781in}{2.897781in}}{\pgfqpoint{0.912171in}{2.908380in}}{\pgfqpoint{0.912171in}{2.919430in}}%
\pgfpathcurveto{\pgfqpoint{0.912171in}{2.930480in}}{\pgfqpoint{0.907781in}{2.941079in}}{\pgfqpoint{0.899967in}{2.948893in}}%
\pgfpathcurveto{\pgfqpoint{0.892154in}{2.956706in}}{\pgfqpoint{0.881555in}{2.961097in}}{\pgfqpoint{0.870505in}{2.961097in}}%
\pgfpathcurveto{\pgfqpoint{0.859455in}{2.961097in}}{\pgfqpoint{0.848856in}{2.956706in}}{\pgfqpoint{0.841042in}{2.948893in}}%
\pgfpathcurveto{\pgfqpoint{0.833228in}{2.941079in}}{\pgfqpoint{0.828838in}{2.930480in}}{\pgfqpoint{0.828838in}{2.919430in}}%
\pgfpathcurveto{\pgfqpoint{0.828838in}{2.908380in}}{\pgfqpoint{0.833228in}{2.897781in}}{\pgfqpoint{0.841042in}{2.889967in}}%
\pgfpathcurveto{\pgfqpoint{0.848856in}{2.882154in}}{\pgfqpoint{0.859455in}{2.877763in}}{\pgfqpoint{0.870505in}{2.877763in}}%
\pgfpathlineto{\pgfqpoint{0.870505in}{2.877763in}}%
\pgfpathclose%
\pgfusepath{stroke}%
\end{pgfscope}%
\begin{pgfscope}%
\pgfpathrectangle{\pgfqpoint{0.494722in}{0.437222in}}{\pgfqpoint{6.275590in}{5.159444in}}%
\pgfusepath{clip}%
\pgfsetbuttcap%
\pgfsetroundjoin%
\pgfsetlinewidth{1.003750pt}%
\definecolor{currentstroke}{rgb}{0.827451,0.827451,0.827451}%
\pgfsetstrokecolor{currentstroke}%
\pgfsetstrokeopacity{0.800000}%
\pgfsetdash{}{0pt}%
\pgfpathmoveto{\pgfqpoint{1.852502in}{1.624574in}}%
\pgfpathcurveto{\pgfqpoint{1.863553in}{1.624574in}}{\pgfqpoint{1.874152in}{1.628964in}}{\pgfqpoint{1.881965in}{1.636778in}}%
\pgfpathcurveto{\pgfqpoint{1.889779in}{1.644591in}}{\pgfqpoint{1.894169in}{1.655190in}}{\pgfqpoint{1.894169in}{1.666240in}}%
\pgfpathcurveto{\pgfqpoint{1.894169in}{1.677291in}}{\pgfqpoint{1.889779in}{1.687890in}}{\pgfqpoint{1.881965in}{1.695703in}}%
\pgfpathcurveto{\pgfqpoint{1.874152in}{1.703517in}}{\pgfqpoint{1.863553in}{1.707907in}}{\pgfqpoint{1.852502in}{1.707907in}}%
\pgfpathcurveto{\pgfqpoint{1.841452in}{1.707907in}}{\pgfqpoint{1.830853in}{1.703517in}}{\pgfqpoint{1.823040in}{1.695703in}}%
\pgfpathcurveto{\pgfqpoint{1.815226in}{1.687890in}}{\pgfqpoint{1.810836in}{1.677291in}}{\pgfqpoint{1.810836in}{1.666240in}}%
\pgfpathcurveto{\pgfqpoint{1.810836in}{1.655190in}}{\pgfqpoint{1.815226in}{1.644591in}}{\pgfqpoint{1.823040in}{1.636778in}}%
\pgfpathcurveto{\pgfqpoint{1.830853in}{1.628964in}}{\pgfqpoint{1.841452in}{1.624574in}}{\pgfqpoint{1.852502in}{1.624574in}}%
\pgfpathlineto{\pgfqpoint{1.852502in}{1.624574in}}%
\pgfpathclose%
\pgfusepath{stroke}%
\end{pgfscope}%
\begin{pgfscope}%
\pgfpathrectangle{\pgfqpoint{0.494722in}{0.437222in}}{\pgfqpoint{6.275590in}{5.159444in}}%
\pgfusepath{clip}%
\pgfsetbuttcap%
\pgfsetroundjoin%
\pgfsetlinewidth{1.003750pt}%
\definecolor{currentstroke}{rgb}{0.827451,0.827451,0.827451}%
\pgfsetstrokecolor{currentstroke}%
\pgfsetstrokeopacity{0.800000}%
\pgfsetdash{}{0pt}%
\pgfpathmoveto{\pgfqpoint{2.867283in}{0.963868in}}%
\pgfpathcurveto{\pgfqpoint{2.878333in}{0.963868in}}{\pgfqpoint{2.888932in}{0.968258in}}{\pgfqpoint{2.896746in}{0.976072in}}%
\pgfpathcurveto{\pgfqpoint{2.904560in}{0.983885in}}{\pgfqpoint{2.908950in}{0.994484in}}{\pgfqpoint{2.908950in}{1.005535in}}%
\pgfpathcurveto{\pgfqpoint{2.908950in}{1.016585in}}{\pgfqpoint{2.904560in}{1.027184in}}{\pgfqpoint{2.896746in}{1.034997in}}%
\pgfpathcurveto{\pgfqpoint{2.888932in}{1.042811in}}{\pgfqpoint{2.878333in}{1.047201in}}{\pgfqpoint{2.867283in}{1.047201in}}%
\pgfpathcurveto{\pgfqpoint{2.856233in}{1.047201in}}{\pgfqpoint{2.845634in}{1.042811in}}{\pgfqpoint{2.837820in}{1.034997in}}%
\pgfpathcurveto{\pgfqpoint{2.830007in}{1.027184in}}{\pgfqpoint{2.825617in}{1.016585in}}{\pgfqpoint{2.825617in}{1.005535in}}%
\pgfpathcurveto{\pgfqpoint{2.825617in}{0.994484in}}{\pgfqpoint{2.830007in}{0.983885in}}{\pgfqpoint{2.837820in}{0.976072in}}%
\pgfpathcurveto{\pgfqpoint{2.845634in}{0.968258in}}{\pgfqpoint{2.856233in}{0.963868in}}{\pgfqpoint{2.867283in}{0.963868in}}%
\pgfpathlineto{\pgfqpoint{2.867283in}{0.963868in}}%
\pgfpathclose%
\pgfusepath{stroke}%
\end{pgfscope}%
\begin{pgfscope}%
\pgfpathrectangle{\pgfqpoint{0.494722in}{0.437222in}}{\pgfqpoint{6.275590in}{5.159444in}}%
\pgfusepath{clip}%
\pgfsetbuttcap%
\pgfsetroundjoin%
\pgfsetlinewidth{1.003750pt}%
\definecolor{currentstroke}{rgb}{0.827451,0.827451,0.827451}%
\pgfsetstrokecolor{currentstroke}%
\pgfsetstrokeopacity{0.800000}%
\pgfsetdash{}{0pt}%
\pgfpathmoveto{\pgfqpoint{5.060352in}{0.429754in}}%
\pgfpathcurveto{\pgfqpoint{5.071402in}{0.429754in}}{\pgfqpoint{5.082001in}{0.434145in}}{\pgfqpoint{5.089815in}{0.441958in}}%
\pgfpathcurveto{\pgfqpoint{5.097629in}{0.449772in}}{\pgfqpoint{5.102019in}{0.460371in}}{\pgfqpoint{5.102019in}{0.471421in}}%
\pgfpathcurveto{\pgfqpoint{5.102019in}{0.482471in}}{\pgfqpoint{5.097629in}{0.493070in}}{\pgfqpoint{5.089815in}{0.500884in}}%
\pgfpathcurveto{\pgfqpoint{5.082001in}{0.508697in}}{\pgfqpoint{5.071402in}{0.513088in}}{\pgfqpoint{5.060352in}{0.513088in}}%
\pgfpathcurveto{\pgfqpoint{5.049302in}{0.513088in}}{\pgfqpoint{5.038703in}{0.508697in}}{\pgfqpoint{5.030890in}{0.500884in}}%
\pgfpathcurveto{\pgfqpoint{5.023076in}{0.493070in}}{\pgfqpoint{5.018686in}{0.482471in}}{\pgfqpoint{5.018686in}{0.471421in}}%
\pgfpathcurveto{\pgfqpoint{5.018686in}{0.460371in}}{\pgfqpoint{5.023076in}{0.449772in}}{\pgfqpoint{5.030890in}{0.441958in}}%
\pgfpathcurveto{\pgfqpoint{5.038703in}{0.434145in}}{\pgfqpoint{5.049302in}{0.429754in}}{\pgfqpoint{5.060352in}{0.429754in}}%
\pgfpathlineto{\pgfqpoint{5.060352in}{0.429754in}}%
\pgfpathclose%
\pgfusepath{stroke}%
\end{pgfscope}%
\begin{pgfscope}%
\pgfpathrectangle{\pgfqpoint{0.494722in}{0.437222in}}{\pgfqpoint{6.275590in}{5.159444in}}%
\pgfusepath{clip}%
\pgfsetbuttcap%
\pgfsetroundjoin%
\pgfsetlinewidth{1.003750pt}%
\definecolor{currentstroke}{rgb}{0.827451,0.827451,0.827451}%
\pgfsetstrokecolor{currentstroke}%
\pgfsetstrokeopacity{0.800000}%
\pgfsetdash{}{0pt}%
\pgfpathmoveto{\pgfqpoint{0.575099in}{3.779944in}}%
\pgfpathcurveto{\pgfqpoint{0.586149in}{3.779944in}}{\pgfqpoint{0.596748in}{3.784334in}}{\pgfqpoint{0.604561in}{3.792148in}}%
\pgfpathcurveto{\pgfqpoint{0.612375in}{3.799961in}}{\pgfqpoint{0.616765in}{3.810560in}}{\pgfqpoint{0.616765in}{3.821610in}}%
\pgfpathcurveto{\pgfqpoint{0.616765in}{3.832661in}}{\pgfqpoint{0.612375in}{3.843260in}}{\pgfqpoint{0.604561in}{3.851073in}}%
\pgfpathcurveto{\pgfqpoint{0.596748in}{3.858887in}}{\pgfqpoint{0.586149in}{3.863277in}}{\pgfqpoint{0.575099in}{3.863277in}}%
\pgfpathcurveto{\pgfqpoint{0.564048in}{3.863277in}}{\pgfqpoint{0.553449in}{3.858887in}}{\pgfqpoint{0.545636in}{3.851073in}}%
\pgfpathcurveto{\pgfqpoint{0.537822in}{3.843260in}}{\pgfqpoint{0.533432in}{3.832661in}}{\pgfqpoint{0.533432in}{3.821610in}}%
\pgfpathcurveto{\pgfqpoint{0.533432in}{3.810560in}}{\pgfqpoint{0.537822in}{3.799961in}}{\pgfqpoint{0.545636in}{3.792148in}}%
\pgfpathcurveto{\pgfqpoint{0.553449in}{3.784334in}}{\pgfqpoint{0.564048in}{3.779944in}}{\pgfqpoint{0.575099in}{3.779944in}}%
\pgfpathlineto{\pgfqpoint{0.575099in}{3.779944in}}%
\pgfpathclose%
\pgfusepath{stroke}%
\end{pgfscope}%
\begin{pgfscope}%
\pgfpathrectangle{\pgfqpoint{0.494722in}{0.437222in}}{\pgfqpoint{6.275590in}{5.159444in}}%
\pgfusepath{clip}%
\pgfsetbuttcap%
\pgfsetroundjoin%
\pgfsetlinewidth{1.003750pt}%
\definecolor{currentstroke}{rgb}{0.827451,0.827451,0.827451}%
\pgfsetstrokecolor{currentstroke}%
\pgfsetstrokeopacity{0.800000}%
\pgfsetdash{}{0pt}%
\pgfpathmoveto{\pgfqpoint{1.157055in}{2.384514in}}%
\pgfpathcurveto{\pgfqpoint{1.168106in}{2.384514in}}{\pgfqpoint{1.178705in}{2.388904in}}{\pgfqpoint{1.186518in}{2.396718in}}%
\pgfpathcurveto{\pgfqpoint{1.194332in}{2.404531in}}{\pgfqpoint{1.198722in}{2.415131in}}{\pgfqpoint{1.198722in}{2.426181in}}%
\pgfpathcurveto{\pgfqpoint{1.198722in}{2.437231in}}{\pgfqpoint{1.194332in}{2.447830in}}{\pgfqpoint{1.186518in}{2.455643in}}%
\pgfpathcurveto{\pgfqpoint{1.178705in}{2.463457in}}{\pgfqpoint{1.168106in}{2.467847in}}{\pgfqpoint{1.157055in}{2.467847in}}%
\pgfpathcurveto{\pgfqpoint{1.146005in}{2.467847in}}{\pgfqpoint{1.135406in}{2.463457in}}{\pgfqpoint{1.127593in}{2.455643in}}%
\pgfpathcurveto{\pgfqpoint{1.119779in}{2.447830in}}{\pgfqpoint{1.115389in}{2.437231in}}{\pgfqpoint{1.115389in}{2.426181in}}%
\pgfpathcurveto{\pgfqpoint{1.115389in}{2.415131in}}{\pgfqpoint{1.119779in}{2.404531in}}{\pgfqpoint{1.127593in}{2.396718in}}%
\pgfpathcurveto{\pgfqpoint{1.135406in}{2.388904in}}{\pgfqpoint{1.146005in}{2.384514in}}{\pgfqpoint{1.157055in}{2.384514in}}%
\pgfpathlineto{\pgfqpoint{1.157055in}{2.384514in}}%
\pgfpathclose%
\pgfusepath{stroke}%
\end{pgfscope}%
\begin{pgfscope}%
\pgfpathrectangle{\pgfqpoint{0.494722in}{0.437222in}}{\pgfqpoint{6.275590in}{5.159444in}}%
\pgfusepath{clip}%
\pgfsetbuttcap%
\pgfsetroundjoin%
\pgfsetlinewidth{1.003750pt}%
\definecolor{currentstroke}{rgb}{0.827451,0.827451,0.827451}%
\pgfsetstrokecolor{currentstroke}%
\pgfsetstrokeopacity{0.800000}%
\pgfsetdash{}{0pt}%
\pgfpathmoveto{\pgfqpoint{3.399179in}{0.734229in}}%
\pgfpathcurveto{\pgfqpoint{3.410229in}{0.734229in}}{\pgfqpoint{3.420828in}{0.738620in}}{\pgfqpoint{3.428642in}{0.746433in}}%
\pgfpathcurveto{\pgfqpoint{3.436456in}{0.754247in}}{\pgfqpoint{3.440846in}{0.764846in}}{\pgfqpoint{3.440846in}{0.775896in}}%
\pgfpathcurveto{\pgfqpoint{3.440846in}{0.786946in}}{\pgfqpoint{3.436456in}{0.797545in}}{\pgfqpoint{3.428642in}{0.805359in}}%
\pgfpathcurveto{\pgfqpoint{3.420828in}{0.813172in}}{\pgfqpoint{3.410229in}{0.817563in}}{\pgfqpoint{3.399179in}{0.817563in}}%
\pgfpathcurveto{\pgfqpoint{3.388129in}{0.817563in}}{\pgfqpoint{3.377530in}{0.813172in}}{\pgfqpoint{3.369716in}{0.805359in}}%
\pgfpathcurveto{\pgfqpoint{3.361903in}{0.797545in}}{\pgfqpoint{3.357513in}{0.786946in}}{\pgfqpoint{3.357513in}{0.775896in}}%
\pgfpathcurveto{\pgfqpoint{3.357513in}{0.764846in}}{\pgfqpoint{3.361903in}{0.754247in}}{\pgfqpoint{3.369716in}{0.746433in}}%
\pgfpathcurveto{\pgfqpoint{3.377530in}{0.738620in}}{\pgfqpoint{3.388129in}{0.734229in}}{\pgfqpoint{3.399179in}{0.734229in}}%
\pgfpathlineto{\pgfqpoint{3.399179in}{0.734229in}}%
\pgfpathclose%
\pgfusepath{stroke}%
\end{pgfscope}%
\begin{pgfscope}%
\pgfpathrectangle{\pgfqpoint{0.494722in}{0.437222in}}{\pgfqpoint{6.275590in}{5.159444in}}%
\pgfusepath{clip}%
\pgfsetbuttcap%
\pgfsetroundjoin%
\pgfsetlinewidth{1.003750pt}%
\definecolor{currentstroke}{rgb}{0.827451,0.827451,0.827451}%
\pgfsetstrokecolor{currentstroke}%
\pgfsetstrokeopacity{0.800000}%
\pgfsetdash{}{0pt}%
\pgfpathmoveto{\pgfqpoint{0.683182in}{3.327378in}}%
\pgfpathcurveto{\pgfqpoint{0.694232in}{3.327378in}}{\pgfqpoint{0.704831in}{3.331768in}}{\pgfqpoint{0.712645in}{3.339581in}}%
\pgfpathcurveto{\pgfqpoint{0.720459in}{3.347395in}}{\pgfqpoint{0.724849in}{3.357994in}}{\pgfqpoint{0.724849in}{3.369044in}}%
\pgfpathcurveto{\pgfqpoint{0.724849in}{3.380094in}}{\pgfqpoint{0.720459in}{3.390693in}}{\pgfqpoint{0.712645in}{3.398507in}}%
\pgfpathcurveto{\pgfqpoint{0.704831in}{3.406321in}}{\pgfqpoint{0.694232in}{3.410711in}}{\pgfqpoint{0.683182in}{3.410711in}}%
\pgfpathcurveto{\pgfqpoint{0.672132in}{3.410711in}}{\pgfqpoint{0.661533in}{3.406321in}}{\pgfqpoint{0.653720in}{3.398507in}}%
\pgfpathcurveto{\pgfqpoint{0.645906in}{3.390693in}}{\pgfqpoint{0.641516in}{3.380094in}}{\pgfqpoint{0.641516in}{3.369044in}}%
\pgfpathcurveto{\pgfqpoint{0.641516in}{3.357994in}}{\pgfqpoint{0.645906in}{3.347395in}}{\pgfqpoint{0.653720in}{3.339581in}}%
\pgfpathcurveto{\pgfqpoint{0.661533in}{3.331768in}}{\pgfqpoint{0.672132in}{3.327378in}}{\pgfqpoint{0.683182in}{3.327378in}}%
\pgfpathlineto{\pgfqpoint{0.683182in}{3.327378in}}%
\pgfpathclose%
\pgfusepath{stroke}%
\end{pgfscope}%
\begin{pgfscope}%
\pgfpathrectangle{\pgfqpoint{0.494722in}{0.437222in}}{\pgfqpoint{6.275590in}{5.159444in}}%
\pgfusepath{clip}%
\pgfsetbuttcap%
\pgfsetroundjoin%
\pgfsetlinewidth{1.003750pt}%
\definecolor{currentstroke}{rgb}{0.827451,0.827451,0.827451}%
\pgfsetstrokecolor{currentstroke}%
\pgfsetstrokeopacity{0.800000}%
\pgfsetdash{}{0pt}%
\pgfpathmoveto{\pgfqpoint{2.308193in}{1.269247in}}%
\pgfpathcurveto{\pgfqpoint{2.319243in}{1.269247in}}{\pgfqpoint{2.329842in}{1.273637in}}{\pgfqpoint{2.337656in}{1.281451in}}%
\pgfpathcurveto{\pgfqpoint{2.345469in}{1.289264in}}{\pgfqpoint{2.349860in}{1.299864in}}{\pgfqpoint{2.349860in}{1.310914in}}%
\pgfpathcurveto{\pgfqpoint{2.349860in}{1.321964in}}{\pgfqpoint{2.345469in}{1.332563in}}{\pgfqpoint{2.337656in}{1.340376in}}%
\pgfpathcurveto{\pgfqpoint{2.329842in}{1.348190in}}{\pgfqpoint{2.319243in}{1.352580in}}{\pgfqpoint{2.308193in}{1.352580in}}%
\pgfpathcurveto{\pgfqpoint{2.297143in}{1.352580in}}{\pgfqpoint{2.286544in}{1.348190in}}{\pgfqpoint{2.278730in}{1.340376in}}%
\pgfpathcurveto{\pgfqpoint{2.270917in}{1.332563in}}{\pgfqpoint{2.266526in}{1.321964in}}{\pgfqpoint{2.266526in}{1.310914in}}%
\pgfpathcurveto{\pgfqpoint{2.266526in}{1.299864in}}{\pgfqpoint{2.270917in}{1.289264in}}{\pgfqpoint{2.278730in}{1.281451in}}%
\pgfpathcurveto{\pgfqpoint{2.286544in}{1.273637in}}{\pgfqpoint{2.297143in}{1.269247in}}{\pgfqpoint{2.308193in}{1.269247in}}%
\pgfpathlineto{\pgfqpoint{2.308193in}{1.269247in}}%
\pgfpathclose%
\pgfusepath{stroke}%
\end{pgfscope}%
\begin{pgfscope}%
\pgfpathrectangle{\pgfqpoint{0.494722in}{0.437222in}}{\pgfqpoint{6.275590in}{5.159444in}}%
\pgfusepath{clip}%
\pgfsetbuttcap%
\pgfsetroundjoin%
\pgfsetlinewidth{1.003750pt}%
\definecolor{currentstroke}{rgb}{0.827451,0.827451,0.827451}%
\pgfsetstrokecolor{currentstroke}%
\pgfsetstrokeopacity{0.800000}%
\pgfsetdash{}{0pt}%
\pgfpathmoveto{\pgfqpoint{4.586350in}{0.475898in}}%
\pgfpathcurveto{\pgfqpoint{4.597400in}{0.475898in}}{\pgfqpoint{4.607999in}{0.480288in}}{\pgfqpoint{4.615813in}{0.488101in}}%
\pgfpathcurveto{\pgfqpoint{4.623627in}{0.495915in}}{\pgfqpoint{4.628017in}{0.506514in}}{\pgfqpoint{4.628017in}{0.517564in}}%
\pgfpathcurveto{\pgfqpoint{4.628017in}{0.528614in}}{\pgfqpoint{4.623627in}{0.539213in}}{\pgfqpoint{4.615813in}{0.547027in}}%
\pgfpathcurveto{\pgfqpoint{4.607999in}{0.554841in}}{\pgfqpoint{4.597400in}{0.559231in}}{\pgfqpoint{4.586350in}{0.559231in}}%
\pgfpathcurveto{\pgfqpoint{4.575300in}{0.559231in}}{\pgfqpoint{4.564701in}{0.554841in}}{\pgfqpoint{4.556887in}{0.547027in}}%
\pgfpathcurveto{\pgfqpoint{4.549074in}{0.539213in}}{\pgfqpoint{4.544683in}{0.528614in}}{\pgfqpoint{4.544683in}{0.517564in}}%
\pgfpathcurveto{\pgfqpoint{4.544683in}{0.506514in}}{\pgfqpoint{4.549074in}{0.495915in}}{\pgfqpoint{4.556887in}{0.488101in}}%
\pgfpathcurveto{\pgfqpoint{4.564701in}{0.480288in}}{\pgfqpoint{4.575300in}{0.475898in}}{\pgfqpoint{4.586350in}{0.475898in}}%
\pgfpathlineto{\pgfqpoint{4.586350in}{0.475898in}}%
\pgfpathclose%
\pgfusepath{stroke}%
\end{pgfscope}%
\begin{pgfscope}%
\pgfpathrectangle{\pgfqpoint{0.494722in}{0.437222in}}{\pgfqpoint{6.275590in}{5.159444in}}%
\pgfusepath{clip}%
\pgfsetbuttcap%
\pgfsetroundjoin%
\pgfsetlinewidth{1.003750pt}%
\definecolor{currentstroke}{rgb}{0.827451,0.827451,0.827451}%
\pgfsetstrokecolor{currentstroke}%
\pgfsetstrokeopacity{0.800000}%
\pgfsetdash{}{0pt}%
\pgfpathmoveto{\pgfqpoint{3.526336in}{0.700005in}}%
\pgfpathcurveto{\pgfqpoint{3.537386in}{0.700005in}}{\pgfqpoint{3.547985in}{0.704396in}}{\pgfqpoint{3.555799in}{0.712209in}}%
\pgfpathcurveto{\pgfqpoint{3.563612in}{0.720023in}}{\pgfqpoint{3.568002in}{0.730622in}}{\pgfqpoint{3.568002in}{0.741672in}}%
\pgfpathcurveto{\pgfqpoint{3.568002in}{0.752722in}}{\pgfqpoint{3.563612in}{0.763321in}}{\pgfqpoint{3.555799in}{0.771135in}}%
\pgfpathcurveto{\pgfqpoint{3.547985in}{0.778948in}}{\pgfqpoint{3.537386in}{0.783339in}}{\pgfqpoint{3.526336in}{0.783339in}}%
\pgfpathcurveto{\pgfqpoint{3.515286in}{0.783339in}}{\pgfqpoint{3.504687in}{0.778948in}}{\pgfqpoint{3.496873in}{0.771135in}}%
\pgfpathcurveto{\pgfqpoint{3.489059in}{0.763321in}}{\pgfqpoint{3.484669in}{0.752722in}}{\pgfqpoint{3.484669in}{0.741672in}}%
\pgfpathcurveto{\pgfqpoint{3.484669in}{0.730622in}}{\pgfqpoint{3.489059in}{0.720023in}}{\pgfqpoint{3.496873in}{0.712209in}}%
\pgfpathcurveto{\pgfqpoint{3.504687in}{0.704396in}}{\pgfqpoint{3.515286in}{0.700005in}}{\pgfqpoint{3.526336in}{0.700005in}}%
\pgfpathlineto{\pgfqpoint{3.526336in}{0.700005in}}%
\pgfpathclose%
\pgfusepath{stroke}%
\end{pgfscope}%
\begin{pgfscope}%
\pgfpathrectangle{\pgfqpoint{0.494722in}{0.437222in}}{\pgfqpoint{6.275590in}{5.159444in}}%
\pgfusepath{clip}%
\pgfsetbuttcap%
\pgfsetroundjoin%
\pgfsetlinewidth{1.003750pt}%
\definecolor{currentstroke}{rgb}{0.827451,0.827451,0.827451}%
\pgfsetstrokecolor{currentstroke}%
\pgfsetstrokeopacity{0.800000}%
\pgfsetdash{}{0pt}%
\pgfpathmoveto{\pgfqpoint{2.512961in}{1.165916in}}%
\pgfpathcurveto{\pgfqpoint{2.524011in}{1.165916in}}{\pgfqpoint{2.534610in}{1.170306in}}{\pgfqpoint{2.542424in}{1.178119in}}%
\pgfpathcurveto{\pgfqpoint{2.550237in}{1.185933in}}{\pgfqpoint{2.554627in}{1.196532in}}{\pgfqpoint{2.554627in}{1.207582in}}%
\pgfpathcurveto{\pgfqpoint{2.554627in}{1.218632in}}{\pgfqpoint{2.550237in}{1.229231in}}{\pgfqpoint{2.542424in}{1.237045in}}%
\pgfpathcurveto{\pgfqpoint{2.534610in}{1.244859in}}{\pgfqpoint{2.524011in}{1.249249in}}{\pgfqpoint{2.512961in}{1.249249in}}%
\pgfpathcurveto{\pgfqpoint{2.501911in}{1.249249in}}{\pgfqpoint{2.491312in}{1.244859in}}{\pgfqpoint{2.483498in}{1.237045in}}%
\pgfpathcurveto{\pgfqpoint{2.475684in}{1.229231in}}{\pgfqpoint{2.471294in}{1.218632in}}{\pgfqpoint{2.471294in}{1.207582in}}%
\pgfpathcurveto{\pgfqpoint{2.471294in}{1.196532in}}{\pgfqpoint{2.475684in}{1.185933in}}{\pgfqpoint{2.483498in}{1.178119in}}%
\pgfpathcurveto{\pgfqpoint{2.491312in}{1.170306in}}{\pgfqpoint{2.501911in}{1.165916in}}{\pgfqpoint{2.512961in}{1.165916in}}%
\pgfpathlineto{\pgfqpoint{2.512961in}{1.165916in}}%
\pgfpathclose%
\pgfusepath{stroke}%
\end{pgfscope}%
\begin{pgfscope}%
\pgfpathrectangle{\pgfqpoint{0.494722in}{0.437222in}}{\pgfqpoint{6.275590in}{5.159444in}}%
\pgfusepath{clip}%
\pgfsetbuttcap%
\pgfsetroundjoin%
\pgfsetlinewidth{1.003750pt}%
\definecolor{currentstroke}{rgb}{0.827451,0.827451,0.827451}%
\pgfsetstrokecolor{currentstroke}%
\pgfsetstrokeopacity{0.800000}%
\pgfsetdash{}{0pt}%
\pgfpathmoveto{\pgfqpoint{0.536273in}{3.987890in}}%
\pgfpathcurveto{\pgfqpoint{0.547324in}{3.987890in}}{\pgfqpoint{0.557923in}{3.992280in}}{\pgfqpoint{0.565736in}{4.000093in}}%
\pgfpathcurveto{\pgfqpoint{0.573550in}{4.007907in}}{\pgfqpoint{0.577940in}{4.018506in}}{\pgfqpoint{0.577940in}{4.029556in}}%
\pgfpathcurveto{\pgfqpoint{0.577940in}{4.040606in}}{\pgfqpoint{0.573550in}{4.051205in}}{\pgfqpoint{0.565736in}{4.059019in}}%
\pgfpathcurveto{\pgfqpoint{0.557923in}{4.066833in}}{\pgfqpoint{0.547324in}{4.071223in}}{\pgfqpoint{0.536273in}{4.071223in}}%
\pgfpathcurveto{\pgfqpoint{0.525223in}{4.071223in}}{\pgfqpoint{0.514624in}{4.066833in}}{\pgfqpoint{0.506811in}{4.059019in}}%
\pgfpathcurveto{\pgfqpoint{0.498997in}{4.051205in}}{\pgfqpoint{0.494607in}{4.040606in}}{\pgfqpoint{0.494607in}{4.029556in}}%
\pgfpathcurveto{\pgfqpoint{0.494607in}{4.018506in}}{\pgfqpoint{0.498997in}{4.007907in}}{\pgfqpoint{0.506811in}{4.000093in}}%
\pgfpathcurveto{\pgfqpoint{0.514624in}{3.992280in}}{\pgfqpoint{0.525223in}{3.987890in}}{\pgfqpoint{0.536273in}{3.987890in}}%
\pgfpathlineto{\pgfqpoint{0.536273in}{3.987890in}}%
\pgfpathclose%
\pgfusepath{stroke}%
\end{pgfscope}%
\begin{pgfscope}%
\pgfpathrectangle{\pgfqpoint{0.494722in}{0.437222in}}{\pgfqpoint{6.275590in}{5.159444in}}%
\pgfusepath{clip}%
\pgfsetbuttcap%
\pgfsetroundjoin%
\pgfsetlinewidth{1.003750pt}%
\definecolor{currentstroke}{rgb}{0.827451,0.827451,0.827451}%
\pgfsetstrokecolor{currentstroke}%
\pgfsetstrokeopacity{0.800000}%
\pgfsetdash{}{0pt}%
\pgfpathmoveto{\pgfqpoint{3.121483in}{0.847531in}}%
\pgfpathcurveto{\pgfqpoint{3.132533in}{0.847531in}}{\pgfqpoint{3.143132in}{0.851921in}}{\pgfqpoint{3.150946in}{0.859735in}}%
\pgfpathcurveto{\pgfqpoint{3.158760in}{0.867548in}}{\pgfqpoint{3.163150in}{0.878147in}}{\pgfqpoint{3.163150in}{0.889198in}}%
\pgfpathcurveto{\pgfqpoint{3.163150in}{0.900248in}}{\pgfqpoint{3.158760in}{0.910847in}}{\pgfqpoint{3.150946in}{0.918660in}}%
\pgfpathcurveto{\pgfqpoint{3.143132in}{0.926474in}}{\pgfqpoint{3.132533in}{0.930864in}}{\pgfqpoint{3.121483in}{0.930864in}}%
\pgfpathcurveto{\pgfqpoint{3.110433in}{0.930864in}}{\pgfqpoint{3.099834in}{0.926474in}}{\pgfqpoint{3.092021in}{0.918660in}}%
\pgfpathcurveto{\pgfqpoint{3.084207in}{0.910847in}}{\pgfqpoint{3.079817in}{0.900248in}}{\pgfqpoint{3.079817in}{0.889198in}}%
\pgfpathcurveto{\pgfqpoint{3.079817in}{0.878147in}}{\pgfqpoint{3.084207in}{0.867548in}}{\pgfqpoint{3.092021in}{0.859735in}}%
\pgfpathcurveto{\pgfqpoint{3.099834in}{0.851921in}}{\pgfqpoint{3.110433in}{0.847531in}}{\pgfqpoint{3.121483in}{0.847531in}}%
\pgfpathlineto{\pgfqpoint{3.121483in}{0.847531in}}%
\pgfpathclose%
\pgfusepath{stroke}%
\end{pgfscope}%
\begin{pgfscope}%
\pgfpathrectangle{\pgfqpoint{0.494722in}{0.437222in}}{\pgfqpoint{6.275590in}{5.159444in}}%
\pgfusepath{clip}%
\pgfsetbuttcap%
\pgfsetroundjoin%
\pgfsetlinewidth{1.003750pt}%
\definecolor{currentstroke}{rgb}{0.827451,0.827451,0.827451}%
\pgfsetstrokecolor{currentstroke}%
\pgfsetstrokeopacity{0.800000}%
\pgfsetdash{}{0pt}%
\pgfpathmoveto{\pgfqpoint{3.321971in}{0.791856in}}%
\pgfpathcurveto{\pgfqpoint{3.333021in}{0.791856in}}{\pgfqpoint{3.343620in}{0.796246in}}{\pgfqpoint{3.351433in}{0.804060in}}%
\pgfpathcurveto{\pgfqpoint{3.359247in}{0.811873in}}{\pgfqpoint{3.363637in}{0.822472in}}{\pgfqpoint{3.363637in}{0.833523in}}%
\pgfpathcurveto{\pgfqpoint{3.363637in}{0.844573in}}{\pgfqpoint{3.359247in}{0.855172in}}{\pgfqpoint{3.351433in}{0.862985in}}%
\pgfpathcurveto{\pgfqpoint{3.343620in}{0.870799in}}{\pgfqpoint{3.333021in}{0.875189in}}{\pgfqpoint{3.321971in}{0.875189in}}%
\pgfpathcurveto{\pgfqpoint{3.310921in}{0.875189in}}{\pgfqpoint{3.300321in}{0.870799in}}{\pgfqpoint{3.292508in}{0.862985in}}%
\pgfpathcurveto{\pgfqpoint{3.284694in}{0.855172in}}{\pgfqpoint{3.280304in}{0.844573in}}{\pgfqpoint{3.280304in}{0.833523in}}%
\pgfpathcurveto{\pgfqpoint{3.280304in}{0.822472in}}{\pgfqpoint{3.284694in}{0.811873in}}{\pgfqpoint{3.292508in}{0.804060in}}%
\pgfpathcurveto{\pgfqpoint{3.300321in}{0.796246in}}{\pgfqpoint{3.310921in}{0.791856in}}{\pgfqpoint{3.321971in}{0.791856in}}%
\pgfpathlineto{\pgfqpoint{3.321971in}{0.791856in}}%
\pgfpathclose%
\pgfusepath{stroke}%
\end{pgfscope}%
\begin{pgfscope}%
\pgfpathrectangle{\pgfqpoint{0.494722in}{0.437222in}}{\pgfqpoint{6.275590in}{5.159444in}}%
\pgfusepath{clip}%
\pgfsetbuttcap%
\pgfsetroundjoin%
\pgfsetlinewidth{1.003750pt}%
\definecolor{currentstroke}{rgb}{0.827451,0.827451,0.827451}%
\pgfsetstrokecolor{currentstroke}%
\pgfsetstrokeopacity{0.800000}%
\pgfsetdash{}{0pt}%
\pgfpathmoveto{\pgfqpoint{4.336302in}{0.500526in}}%
\pgfpathcurveto{\pgfqpoint{4.347353in}{0.500526in}}{\pgfqpoint{4.357952in}{0.504917in}}{\pgfqpoint{4.365765in}{0.512730in}}%
\pgfpathcurveto{\pgfqpoint{4.373579in}{0.520544in}}{\pgfqpoint{4.377969in}{0.531143in}}{\pgfqpoint{4.377969in}{0.542193in}}%
\pgfpathcurveto{\pgfqpoint{4.377969in}{0.553243in}}{\pgfqpoint{4.373579in}{0.563842in}}{\pgfqpoint{4.365765in}{0.571656in}}%
\pgfpathcurveto{\pgfqpoint{4.357952in}{0.579469in}}{\pgfqpoint{4.347353in}{0.583860in}}{\pgfqpoint{4.336302in}{0.583860in}}%
\pgfpathcurveto{\pgfqpoint{4.325252in}{0.583860in}}{\pgfqpoint{4.314653in}{0.579469in}}{\pgfqpoint{4.306840in}{0.571656in}}%
\pgfpathcurveto{\pgfqpoint{4.299026in}{0.563842in}}{\pgfqpoint{4.294636in}{0.553243in}}{\pgfqpoint{4.294636in}{0.542193in}}%
\pgfpathcurveto{\pgfqpoint{4.294636in}{0.531143in}}{\pgfqpoint{4.299026in}{0.520544in}}{\pgfqpoint{4.306840in}{0.512730in}}%
\pgfpathcurveto{\pgfqpoint{4.314653in}{0.504917in}}{\pgfqpoint{4.325252in}{0.500526in}}{\pgfqpoint{4.336302in}{0.500526in}}%
\pgfpathlineto{\pgfqpoint{4.336302in}{0.500526in}}%
\pgfpathclose%
\pgfusepath{stroke}%
\end{pgfscope}%
\begin{pgfscope}%
\pgfpathrectangle{\pgfqpoint{0.494722in}{0.437222in}}{\pgfqpoint{6.275590in}{5.159444in}}%
\pgfusepath{clip}%
\pgfsetbuttcap%
\pgfsetroundjoin%
\pgfsetlinewidth{1.003750pt}%
\definecolor{currentstroke}{rgb}{0.827451,0.827451,0.827451}%
\pgfsetstrokecolor{currentstroke}%
\pgfsetstrokeopacity{0.800000}%
\pgfsetdash{}{0pt}%
\pgfpathmoveto{\pgfqpoint{1.091684in}{2.547519in}}%
\pgfpathcurveto{\pgfqpoint{1.102734in}{2.547519in}}{\pgfqpoint{1.113333in}{2.551909in}}{\pgfqpoint{1.121147in}{2.559722in}}%
\pgfpathcurveto{\pgfqpoint{1.128960in}{2.567536in}}{\pgfqpoint{1.133351in}{2.578135in}}{\pgfqpoint{1.133351in}{2.589185in}}%
\pgfpathcurveto{\pgfqpoint{1.133351in}{2.600235in}}{\pgfqpoint{1.128960in}{2.610834in}}{\pgfqpoint{1.121147in}{2.618648in}}%
\pgfpathcurveto{\pgfqpoint{1.113333in}{2.626462in}}{\pgfqpoint{1.102734in}{2.630852in}}{\pgfqpoint{1.091684in}{2.630852in}}%
\pgfpathcurveto{\pgfqpoint{1.080634in}{2.630852in}}{\pgfqpoint{1.070035in}{2.626462in}}{\pgfqpoint{1.062221in}{2.618648in}}%
\pgfpathcurveto{\pgfqpoint{1.054407in}{2.610834in}}{\pgfqpoint{1.050017in}{2.600235in}}{\pgfqpoint{1.050017in}{2.589185in}}%
\pgfpathcurveto{\pgfqpoint{1.050017in}{2.578135in}}{\pgfqpoint{1.054407in}{2.567536in}}{\pgfqpoint{1.062221in}{2.559722in}}%
\pgfpathcurveto{\pgfqpoint{1.070035in}{2.551909in}}{\pgfqpoint{1.080634in}{2.547519in}}{\pgfqpoint{1.091684in}{2.547519in}}%
\pgfpathlineto{\pgfqpoint{1.091684in}{2.547519in}}%
\pgfpathclose%
\pgfusepath{stroke}%
\end{pgfscope}%
\begin{pgfscope}%
\pgfpathrectangle{\pgfqpoint{0.494722in}{0.437222in}}{\pgfqpoint{6.275590in}{5.159444in}}%
\pgfusepath{clip}%
\pgfsetbuttcap%
\pgfsetroundjoin%
\pgfsetlinewidth{1.003750pt}%
\definecolor{currentstroke}{rgb}{0.827451,0.827451,0.827451}%
\pgfsetstrokecolor{currentstroke}%
\pgfsetstrokeopacity{0.800000}%
\pgfsetdash{}{0pt}%
\pgfpathmoveto{\pgfqpoint{1.215821in}{2.264081in}}%
\pgfpathcurveto{\pgfqpoint{1.226871in}{2.264081in}}{\pgfqpoint{1.237470in}{2.268471in}}{\pgfqpoint{1.245283in}{2.276285in}}%
\pgfpathcurveto{\pgfqpoint{1.253097in}{2.284099in}}{\pgfqpoint{1.257487in}{2.294698in}}{\pgfqpoint{1.257487in}{2.305748in}}%
\pgfpathcurveto{\pgfqpoint{1.257487in}{2.316798in}}{\pgfqpoint{1.253097in}{2.327397in}}{\pgfqpoint{1.245283in}{2.335210in}}%
\pgfpathcurveto{\pgfqpoint{1.237470in}{2.343024in}}{\pgfqpoint{1.226871in}{2.347414in}}{\pgfqpoint{1.215821in}{2.347414in}}%
\pgfpathcurveto{\pgfqpoint{1.204771in}{2.347414in}}{\pgfqpoint{1.194172in}{2.343024in}}{\pgfqpoint{1.186358in}{2.335210in}}%
\pgfpathcurveto{\pgfqpoint{1.178544in}{2.327397in}}{\pgfqpoint{1.174154in}{2.316798in}}{\pgfqpoint{1.174154in}{2.305748in}}%
\pgfpathcurveto{\pgfqpoint{1.174154in}{2.294698in}}{\pgfqpoint{1.178544in}{2.284099in}}{\pgfqpoint{1.186358in}{2.276285in}}%
\pgfpathcurveto{\pgfqpoint{1.194172in}{2.268471in}}{\pgfqpoint{1.204771in}{2.264081in}}{\pgfqpoint{1.215821in}{2.264081in}}%
\pgfpathlineto{\pgfqpoint{1.215821in}{2.264081in}}%
\pgfpathclose%
\pgfusepath{stroke}%
\end{pgfscope}%
\begin{pgfscope}%
\pgfpathrectangle{\pgfqpoint{0.494722in}{0.437222in}}{\pgfqpoint{6.275590in}{5.159444in}}%
\pgfusepath{clip}%
\pgfsetbuttcap%
\pgfsetroundjoin%
\pgfsetlinewidth{1.003750pt}%
\definecolor{currentstroke}{rgb}{0.827451,0.827451,0.827451}%
\pgfsetstrokecolor{currentstroke}%
\pgfsetstrokeopacity{0.800000}%
\pgfsetdash{}{0pt}%
\pgfpathmoveto{\pgfqpoint{2.224297in}{1.325142in}}%
\pgfpathcurveto{\pgfqpoint{2.235347in}{1.325142in}}{\pgfqpoint{2.245946in}{1.329532in}}{\pgfqpoint{2.253760in}{1.337346in}}%
\pgfpathcurveto{\pgfqpoint{2.261573in}{1.345159in}}{\pgfqpoint{2.265964in}{1.355758in}}{\pgfqpoint{2.265964in}{1.366809in}}%
\pgfpathcurveto{\pgfqpoint{2.265964in}{1.377859in}}{\pgfqpoint{2.261573in}{1.388458in}}{\pgfqpoint{2.253760in}{1.396271in}}%
\pgfpathcurveto{\pgfqpoint{2.245946in}{1.404085in}}{\pgfqpoint{2.235347in}{1.408475in}}{\pgfqpoint{2.224297in}{1.408475in}}%
\pgfpathcurveto{\pgfqpoint{2.213247in}{1.408475in}}{\pgfqpoint{2.202648in}{1.404085in}}{\pgfqpoint{2.194834in}{1.396271in}}%
\pgfpathcurveto{\pgfqpoint{2.187020in}{1.388458in}}{\pgfqpoint{2.182630in}{1.377859in}}{\pgfqpoint{2.182630in}{1.366809in}}%
\pgfpathcurveto{\pgfqpoint{2.182630in}{1.355758in}}{\pgfqpoint{2.187020in}{1.345159in}}{\pgfqpoint{2.194834in}{1.337346in}}%
\pgfpathcurveto{\pgfqpoint{2.202648in}{1.329532in}}{\pgfqpoint{2.213247in}{1.325142in}}{\pgfqpoint{2.224297in}{1.325142in}}%
\pgfpathlineto{\pgfqpoint{2.224297in}{1.325142in}}%
\pgfpathclose%
\pgfusepath{stroke}%
\end{pgfscope}%
\begin{pgfscope}%
\pgfpathrectangle{\pgfqpoint{0.494722in}{0.437222in}}{\pgfqpoint{6.275590in}{5.159444in}}%
\pgfusepath{clip}%
\pgfsetbuttcap%
\pgfsetroundjoin%
\pgfsetlinewidth{1.003750pt}%
\definecolor{currentstroke}{rgb}{0.827451,0.827451,0.827451}%
\pgfsetstrokecolor{currentstroke}%
\pgfsetstrokeopacity{0.800000}%
\pgfsetdash{}{0pt}%
\pgfpathmoveto{\pgfqpoint{1.171136in}{2.367910in}}%
\pgfpathcurveto{\pgfqpoint{1.182186in}{2.367910in}}{\pgfqpoint{1.192785in}{2.372300in}}{\pgfqpoint{1.200599in}{2.380114in}}%
\pgfpathcurveto{\pgfqpoint{1.208412in}{2.387927in}}{\pgfqpoint{1.212803in}{2.398526in}}{\pgfqpoint{1.212803in}{2.409576in}}%
\pgfpathcurveto{\pgfqpoint{1.212803in}{2.420626in}}{\pgfqpoint{1.208412in}{2.431225in}}{\pgfqpoint{1.200599in}{2.439039in}}%
\pgfpathcurveto{\pgfqpoint{1.192785in}{2.446853in}}{\pgfqpoint{1.182186in}{2.451243in}}{\pgfqpoint{1.171136in}{2.451243in}}%
\pgfpathcurveto{\pgfqpoint{1.160086in}{2.451243in}}{\pgfqpoint{1.149487in}{2.446853in}}{\pgfqpoint{1.141673in}{2.439039in}}%
\pgfpathcurveto{\pgfqpoint{1.133860in}{2.431225in}}{\pgfqpoint{1.129469in}{2.420626in}}{\pgfqpoint{1.129469in}{2.409576in}}%
\pgfpathcurveto{\pgfqpoint{1.129469in}{2.398526in}}{\pgfqpoint{1.133860in}{2.387927in}}{\pgfqpoint{1.141673in}{2.380114in}}%
\pgfpathcurveto{\pgfqpoint{1.149487in}{2.372300in}}{\pgfqpoint{1.160086in}{2.367910in}}{\pgfqpoint{1.171136in}{2.367910in}}%
\pgfpathlineto{\pgfqpoint{1.171136in}{2.367910in}}%
\pgfpathclose%
\pgfusepath{stroke}%
\end{pgfscope}%
\begin{pgfscope}%
\pgfpathrectangle{\pgfqpoint{0.494722in}{0.437222in}}{\pgfqpoint{6.275590in}{5.159444in}}%
\pgfusepath{clip}%
\pgfsetbuttcap%
\pgfsetroundjoin%
\pgfsetlinewidth{1.003750pt}%
\definecolor{currentstroke}{rgb}{0.827451,0.827451,0.827451}%
\pgfsetstrokecolor{currentstroke}%
\pgfsetstrokeopacity{0.800000}%
\pgfsetdash{}{0pt}%
\pgfpathmoveto{\pgfqpoint{0.750297in}{3.133310in}}%
\pgfpathcurveto{\pgfqpoint{0.761348in}{3.133310in}}{\pgfqpoint{0.771947in}{3.137700in}}{\pgfqpoint{0.779760in}{3.145514in}}%
\pgfpathcurveto{\pgfqpoint{0.787574in}{3.153328in}}{\pgfqpoint{0.791964in}{3.163927in}}{\pgfqpoint{0.791964in}{3.174977in}}%
\pgfpathcurveto{\pgfqpoint{0.791964in}{3.186027in}}{\pgfqpoint{0.787574in}{3.196626in}}{\pgfqpoint{0.779760in}{3.204439in}}%
\pgfpathcurveto{\pgfqpoint{0.771947in}{3.212253in}}{\pgfqpoint{0.761348in}{3.216643in}}{\pgfqpoint{0.750297in}{3.216643in}}%
\pgfpathcurveto{\pgfqpoint{0.739247in}{3.216643in}}{\pgfqpoint{0.728648in}{3.212253in}}{\pgfqpoint{0.720835in}{3.204439in}}%
\pgfpathcurveto{\pgfqpoint{0.713021in}{3.196626in}}{\pgfqpoint{0.708631in}{3.186027in}}{\pgfqpoint{0.708631in}{3.174977in}}%
\pgfpathcurveto{\pgfqpoint{0.708631in}{3.163927in}}{\pgfqpoint{0.713021in}{3.153328in}}{\pgfqpoint{0.720835in}{3.145514in}}%
\pgfpathcurveto{\pgfqpoint{0.728648in}{3.137700in}}{\pgfqpoint{0.739247in}{3.133310in}}{\pgfqpoint{0.750297in}{3.133310in}}%
\pgfpathlineto{\pgfqpoint{0.750297in}{3.133310in}}%
\pgfpathclose%
\pgfusepath{stroke}%
\end{pgfscope}%
\begin{pgfscope}%
\pgfpathrectangle{\pgfqpoint{0.494722in}{0.437222in}}{\pgfqpoint{6.275590in}{5.159444in}}%
\pgfusepath{clip}%
\pgfsetbuttcap%
\pgfsetroundjoin%
\pgfsetlinewidth{1.003750pt}%
\definecolor{currentstroke}{rgb}{0.827451,0.827451,0.827451}%
\pgfsetstrokecolor{currentstroke}%
\pgfsetstrokeopacity{0.800000}%
\pgfsetdash{}{0pt}%
\pgfpathmoveto{\pgfqpoint{1.766183in}{1.687308in}}%
\pgfpathcurveto{\pgfqpoint{1.777233in}{1.687308in}}{\pgfqpoint{1.787832in}{1.691698in}}{\pgfqpoint{1.795646in}{1.699512in}}%
\pgfpathcurveto{\pgfqpoint{1.803459in}{1.707325in}}{\pgfqpoint{1.807850in}{1.717924in}}{\pgfqpoint{1.807850in}{1.728974in}}%
\pgfpathcurveto{\pgfqpoint{1.807850in}{1.740025in}}{\pgfqpoint{1.803459in}{1.750624in}}{\pgfqpoint{1.795646in}{1.758437in}}%
\pgfpathcurveto{\pgfqpoint{1.787832in}{1.766251in}}{\pgfqpoint{1.777233in}{1.770641in}}{\pgfqpoint{1.766183in}{1.770641in}}%
\pgfpathcurveto{\pgfqpoint{1.755133in}{1.770641in}}{\pgfqpoint{1.744534in}{1.766251in}}{\pgfqpoint{1.736720in}{1.758437in}}%
\pgfpathcurveto{\pgfqpoint{1.728907in}{1.750624in}}{\pgfqpoint{1.724516in}{1.740025in}}{\pgfqpoint{1.724516in}{1.728974in}}%
\pgfpathcurveto{\pgfqpoint{1.724516in}{1.717924in}}{\pgfqpoint{1.728907in}{1.707325in}}{\pgfqpoint{1.736720in}{1.699512in}}%
\pgfpathcurveto{\pgfqpoint{1.744534in}{1.691698in}}{\pgfqpoint{1.755133in}{1.687308in}}{\pgfqpoint{1.766183in}{1.687308in}}%
\pgfpathlineto{\pgfqpoint{1.766183in}{1.687308in}}%
\pgfpathclose%
\pgfusepath{stroke}%
\end{pgfscope}%
\begin{pgfscope}%
\pgfpathrectangle{\pgfqpoint{0.494722in}{0.437222in}}{\pgfqpoint{6.275590in}{5.159444in}}%
\pgfusepath{clip}%
\pgfsetbuttcap%
\pgfsetroundjoin%
\pgfsetlinewidth{1.003750pt}%
\definecolor{currentstroke}{rgb}{0.827451,0.827451,0.827451}%
\pgfsetstrokecolor{currentstroke}%
\pgfsetstrokeopacity{0.800000}%
\pgfsetdash{}{0pt}%
\pgfpathmoveto{\pgfqpoint{1.926347in}{1.574489in}}%
\pgfpathcurveto{\pgfqpoint{1.937397in}{1.574489in}}{\pgfqpoint{1.947996in}{1.578879in}}{\pgfqpoint{1.955809in}{1.586693in}}%
\pgfpathcurveto{\pgfqpoint{1.963623in}{1.594506in}}{\pgfqpoint{1.968013in}{1.605105in}}{\pgfqpoint{1.968013in}{1.616155in}}%
\pgfpathcurveto{\pgfqpoint{1.968013in}{1.627206in}}{\pgfqpoint{1.963623in}{1.637805in}}{\pgfqpoint{1.955809in}{1.645618in}}%
\pgfpathcurveto{\pgfqpoint{1.947996in}{1.653432in}}{\pgfqpoint{1.937397in}{1.657822in}}{\pgfqpoint{1.926347in}{1.657822in}}%
\pgfpathcurveto{\pgfqpoint{1.915296in}{1.657822in}}{\pgfqpoint{1.904697in}{1.653432in}}{\pgfqpoint{1.896884in}{1.645618in}}%
\pgfpathcurveto{\pgfqpoint{1.889070in}{1.637805in}}{\pgfqpoint{1.884680in}{1.627206in}}{\pgfqpoint{1.884680in}{1.616155in}}%
\pgfpathcurveto{\pgfqpoint{1.884680in}{1.605105in}}{\pgfqpoint{1.889070in}{1.594506in}}{\pgfqpoint{1.896884in}{1.586693in}}%
\pgfpathcurveto{\pgfqpoint{1.904697in}{1.578879in}}{\pgfqpoint{1.915296in}{1.574489in}}{\pgfqpoint{1.926347in}{1.574489in}}%
\pgfpathlineto{\pgfqpoint{1.926347in}{1.574489in}}%
\pgfpathclose%
\pgfusepath{stroke}%
\end{pgfscope}%
\begin{pgfscope}%
\pgfpathrectangle{\pgfqpoint{0.494722in}{0.437222in}}{\pgfqpoint{6.275590in}{5.159444in}}%
\pgfusepath{clip}%
\pgfsetbuttcap%
\pgfsetroundjoin%
\pgfsetlinewidth{1.003750pt}%
\definecolor{currentstroke}{rgb}{0.827451,0.827451,0.827451}%
\pgfsetstrokecolor{currentstroke}%
\pgfsetstrokeopacity{0.800000}%
\pgfsetdash{}{0pt}%
\pgfpathmoveto{\pgfqpoint{2.705879in}{1.060745in}}%
\pgfpathcurveto{\pgfqpoint{2.716929in}{1.060745in}}{\pgfqpoint{2.727528in}{1.065135in}}{\pgfqpoint{2.735341in}{1.072948in}}%
\pgfpathcurveto{\pgfqpoint{2.743155in}{1.080762in}}{\pgfqpoint{2.747545in}{1.091361in}}{\pgfqpoint{2.747545in}{1.102411in}}%
\pgfpathcurveto{\pgfqpoint{2.747545in}{1.113461in}}{\pgfqpoint{2.743155in}{1.124060in}}{\pgfqpoint{2.735341in}{1.131874in}}%
\pgfpathcurveto{\pgfqpoint{2.727528in}{1.139688in}}{\pgfqpoint{2.716929in}{1.144078in}}{\pgfqpoint{2.705879in}{1.144078in}}%
\pgfpathcurveto{\pgfqpoint{2.694829in}{1.144078in}}{\pgfqpoint{2.684229in}{1.139688in}}{\pgfqpoint{2.676416in}{1.131874in}}%
\pgfpathcurveto{\pgfqpoint{2.668602in}{1.124060in}}{\pgfqpoint{2.664212in}{1.113461in}}{\pgfqpoint{2.664212in}{1.102411in}}%
\pgfpathcurveto{\pgfqpoint{2.664212in}{1.091361in}}{\pgfqpoint{2.668602in}{1.080762in}}{\pgfqpoint{2.676416in}{1.072948in}}%
\pgfpathcurveto{\pgfqpoint{2.684229in}{1.065135in}}{\pgfqpoint{2.694829in}{1.060745in}}{\pgfqpoint{2.705879in}{1.060745in}}%
\pgfpathlineto{\pgfqpoint{2.705879in}{1.060745in}}%
\pgfpathclose%
\pgfusepath{stroke}%
\end{pgfscope}%
\begin{pgfscope}%
\pgfpathrectangle{\pgfqpoint{0.494722in}{0.437222in}}{\pgfqpoint{6.275590in}{5.159444in}}%
\pgfusepath{clip}%
\pgfsetbuttcap%
\pgfsetroundjoin%
\pgfsetlinewidth{1.003750pt}%
\definecolor{currentstroke}{rgb}{0.827451,0.827451,0.827451}%
\pgfsetstrokecolor{currentstroke}%
\pgfsetstrokeopacity{0.800000}%
\pgfsetdash{}{0pt}%
\pgfpathmoveto{\pgfqpoint{0.530867in}{4.033508in}}%
\pgfpathcurveto{\pgfqpoint{0.541917in}{4.033508in}}{\pgfqpoint{0.552516in}{4.037898in}}{\pgfqpoint{0.560330in}{4.045712in}}%
\pgfpathcurveto{\pgfqpoint{0.568143in}{4.053525in}}{\pgfqpoint{0.572534in}{4.064124in}}{\pgfqpoint{0.572534in}{4.075174in}}%
\pgfpathcurveto{\pgfqpoint{0.572534in}{4.086225in}}{\pgfqpoint{0.568143in}{4.096824in}}{\pgfqpoint{0.560330in}{4.104637in}}%
\pgfpathcurveto{\pgfqpoint{0.552516in}{4.112451in}}{\pgfqpoint{0.541917in}{4.116841in}}{\pgfqpoint{0.530867in}{4.116841in}}%
\pgfpathcurveto{\pgfqpoint{0.519817in}{4.116841in}}{\pgfqpoint{0.509218in}{4.112451in}}{\pgfqpoint{0.501404in}{4.104637in}}%
\pgfpathcurveto{\pgfqpoint{0.493591in}{4.096824in}}{\pgfqpoint{0.489200in}{4.086225in}}{\pgfqpoint{0.489200in}{4.075174in}}%
\pgfpathcurveto{\pgfqpoint{0.489200in}{4.064124in}}{\pgfqpoint{0.493591in}{4.053525in}}{\pgfqpoint{0.501404in}{4.045712in}}%
\pgfpathcurveto{\pgfqpoint{0.509218in}{4.037898in}}{\pgfqpoint{0.519817in}{4.033508in}}{\pgfqpoint{0.530867in}{4.033508in}}%
\pgfpathlineto{\pgfqpoint{0.530867in}{4.033508in}}%
\pgfpathclose%
\pgfusepath{stroke}%
\end{pgfscope}%
\begin{pgfscope}%
\pgfpathrectangle{\pgfqpoint{0.494722in}{0.437222in}}{\pgfqpoint{6.275590in}{5.159444in}}%
\pgfusepath{clip}%
\pgfsetbuttcap%
\pgfsetroundjoin%
\pgfsetlinewidth{1.003750pt}%
\definecolor{currentstroke}{rgb}{0.827451,0.827451,0.827451}%
\pgfsetstrokecolor{currentstroke}%
\pgfsetstrokeopacity{0.800000}%
\pgfsetdash{}{0pt}%
\pgfpathmoveto{\pgfqpoint{2.974947in}{0.945504in}}%
\pgfpathcurveto{\pgfqpoint{2.985997in}{0.945504in}}{\pgfqpoint{2.996596in}{0.949895in}}{\pgfqpoint{3.004410in}{0.957708in}}%
\pgfpathcurveto{\pgfqpoint{3.012223in}{0.965522in}}{\pgfqpoint{3.016614in}{0.976121in}}{\pgfqpoint{3.016614in}{0.987171in}}%
\pgfpathcurveto{\pgfqpoint{3.016614in}{0.998221in}}{\pgfqpoint{3.012223in}{1.008820in}}{\pgfqpoint{3.004410in}{1.016634in}}%
\pgfpathcurveto{\pgfqpoint{2.996596in}{1.024447in}}{\pgfqpoint{2.985997in}{1.028838in}}{\pgfqpoint{2.974947in}{1.028838in}}%
\pgfpathcurveto{\pgfqpoint{2.963897in}{1.028838in}}{\pgfqpoint{2.953298in}{1.024447in}}{\pgfqpoint{2.945484in}{1.016634in}}%
\pgfpathcurveto{\pgfqpoint{2.937670in}{1.008820in}}{\pgfqpoint{2.933280in}{0.998221in}}{\pgfqpoint{2.933280in}{0.987171in}}%
\pgfpathcurveto{\pgfqpoint{2.933280in}{0.976121in}}{\pgfqpoint{2.937670in}{0.965522in}}{\pgfqpoint{2.945484in}{0.957708in}}%
\pgfpathcurveto{\pgfqpoint{2.953298in}{0.949895in}}{\pgfqpoint{2.963897in}{0.945504in}}{\pgfqpoint{2.974947in}{0.945504in}}%
\pgfpathlineto{\pgfqpoint{2.974947in}{0.945504in}}%
\pgfpathclose%
\pgfusepath{stroke}%
\end{pgfscope}%
\begin{pgfscope}%
\pgfpathrectangle{\pgfqpoint{0.494722in}{0.437222in}}{\pgfqpoint{6.275590in}{5.159444in}}%
\pgfusepath{clip}%
\pgfsetbuttcap%
\pgfsetroundjoin%
\pgfsetlinewidth{1.003750pt}%
\definecolor{currentstroke}{rgb}{0.827451,0.827451,0.827451}%
\pgfsetstrokecolor{currentstroke}%
\pgfsetstrokeopacity{0.800000}%
\pgfsetdash{}{0pt}%
\pgfpathmoveto{\pgfqpoint{0.619224in}{3.610864in}}%
\pgfpathcurveto{\pgfqpoint{0.630274in}{3.610864in}}{\pgfqpoint{0.640874in}{3.615254in}}{\pgfqpoint{0.648687in}{3.623068in}}%
\pgfpathcurveto{\pgfqpoint{0.656501in}{3.630881in}}{\pgfqpoint{0.660891in}{3.641480in}}{\pgfqpoint{0.660891in}{3.652530in}}%
\pgfpathcurveto{\pgfqpoint{0.660891in}{3.663580in}}{\pgfqpoint{0.656501in}{3.674179in}}{\pgfqpoint{0.648687in}{3.681993in}}%
\pgfpathcurveto{\pgfqpoint{0.640874in}{3.689807in}}{\pgfqpoint{0.630274in}{3.694197in}}{\pgfqpoint{0.619224in}{3.694197in}}%
\pgfpathcurveto{\pgfqpoint{0.608174in}{3.694197in}}{\pgfqpoint{0.597575in}{3.689807in}}{\pgfqpoint{0.589762in}{3.681993in}}%
\pgfpathcurveto{\pgfqpoint{0.581948in}{3.674179in}}{\pgfqpoint{0.577558in}{3.663580in}}{\pgfqpoint{0.577558in}{3.652530in}}%
\pgfpathcurveto{\pgfqpoint{0.577558in}{3.641480in}}{\pgfqpoint{0.581948in}{3.630881in}}{\pgfqpoint{0.589762in}{3.623068in}}%
\pgfpathcurveto{\pgfqpoint{0.597575in}{3.615254in}}{\pgfqpoint{0.608174in}{3.610864in}}{\pgfqpoint{0.619224in}{3.610864in}}%
\pgfpathlineto{\pgfqpoint{0.619224in}{3.610864in}}%
\pgfpathclose%
\pgfusepath{stroke}%
\end{pgfscope}%
\begin{pgfscope}%
\pgfpathrectangle{\pgfqpoint{0.494722in}{0.437222in}}{\pgfqpoint{6.275590in}{5.159444in}}%
\pgfusepath{clip}%
\pgfsetbuttcap%
\pgfsetroundjoin%
\pgfsetlinewidth{1.003750pt}%
\definecolor{currentstroke}{rgb}{0.827451,0.827451,0.827451}%
\pgfsetstrokecolor{currentstroke}%
\pgfsetstrokeopacity{0.800000}%
\pgfsetdash{}{0pt}%
\pgfpathmoveto{\pgfqpoint{1.278200in}{2.184905in}}%
\pgfpathcurveto{\pgfqpoint{1.289250in}{2.184905in}}{\pgfqpoint{1.299849in}{2.189295in}}{\pgfqpoint{1.307663in}{2.197109in}}%
\pgfpathcurveto{\pgfqpoint{1.315477in}{2.204922in}}{\pgfqpoint{1.319867in}{2.215521in}}{\pgfqpoint{1.319867in}{2.226572in}}%
\pgfpathcurveto{\pgfqpoint{1.319867in}{2.237622in}}{\pgfqpoint{1.315477in}{2.248221in}}{\pgfqpoint{1.307663in}{2.256034in}}%
\pgfpathcurveto{\pgfqpoint{1.299849in}{2.263848in}}{\pgfqpoint{1.289250in}{2.268238in}}{\pgfqpoint{1.278200in}{2.268238in}}%
\pgfpathcurveto{\pgfqpoint{1.267150in}{2.268238in}}{\pgfqpoint{1.256551in}{2.263848in}}{\pgfqpoint{1.248738in}{2.256034in}}%
\pgfpathcurveto{\pgfqpoint{1.240924in}{2.248221in}}{\pgfqpoint{1.236534in}{2.237622in}}{\pgfqpoint{1.236534in}{2.226572in}}%
\pgfpathcurveto{\pgfqpoint{1.236534in}{2.215521in}}{\pgfqpoint{1.240924in}{2.204922in}}{\pgfqpoint{1.248738in}{2.197109in}}%
\pgfpathcurveto{\pgfqpoint{1.256551in}{2.189295in}}{\pgfqpoint{1.267150in}{2.184905in}}{\pgfqpoint{1.278200in}{2.184905in}}%
\pgfpathlineto{\pgfqpoint{1.278200in}{2.184905in}}%
\pgfpathclose%
\pgfusepath{stroke}%
\end{pgfscope}%
\begin{pgfscope}%
\pgfpathrectangle{\pgfqpoint{0.494722in}{0.437222in}}{\pgfqpoint{6.275590in}{5.159444in}}%
\pgfusepath{clip}%
\pgfsetbuttcap%
\pgfsetroundjoin%
\pgfsetlinewidth{1.003750pt}%
\definecolor{currentstroke}{rgb}{0.827451,0.827451,0.827451}%
\pgfsetstrokecolor{currentstroke}%
\pgfsetstrokeopacity{0.800000}%
\pgfsetdash{}{0pt}%
\pgfpathmoveto{\pgfqpoint{1.444344in}{1.986059in}}%
\pgfpathcurveto{\pgfqpoint{1.455394in}{1.986059in}}{\pgfqpoint{1.465993in}{1.990450in}}{\pgfqpoint{1.473807in}{1.998263in}}%
\pgfpathcurveto{\pgfqpoint{1.481620in}{2.006077in}}{\pgfqpoint{1.486010in}{2.016676in}}{\pgfqpoint{1.486010in}{2.027726in}}%
\pgfpathcurveto{\pgfqpoint{1.486010in}{2.038776in}}{\pgfqpoint{1.481620in}{2.049375in}}{\pgfqpoint{1.473807in}{2.057189in}}%
\pgfpathcurveto{\pgfqpoint{1.465993in}{2.065002in}}{\pgfqpoint{1.455394in}{2.069393in}}{\pgfqpoint{1.444344in}{2.069393in}}%
\pgfpathcurveto{\pgfqpoint{1.433294in}{2.069393in}}{\pgfqpoint{1.422695in}{2.065002in}}{\pgfqpoint{1.414881in}{2.057189in}}%
\pgfpathcurveto{\pgfqpoint{1.407067in}{2.049375in}}{\pgfqpoint{1.402677in}{2.038776in}}{\pgfqpoint{1.402677in}{2.027726in}}%
\pgfpathcurveto{\pgfqpoint{1.402677in}{2.016676in}}{\pgfqpoint{1.407067in}{2.006077in}}{\pgfqpoint{1.414881in}{1.998263in}}%
\pgfpathcurveto{\pgfqpoint{1.422695in}{1.990450in}}{\pgfqpoint{1.433294in}{1.986059in}}{\pgfqpoint{1.444344in}{1.986059in}}%
\pgfpathlineto{\pgfqpoint{1.444344in}{1.986059in}}%
\pgfpathclose%
\pgfusepath{stroke}%
\end{pgfscope}%
\begin{pgfscope}%
\pgfpathrectangle{\pgfqpoint{0.494722in}{0.437222in}}{\pgfqpoint{6.275590in}{5.159444in}}%
\pgfusepath{clip}%
\pgfsetbuttcap%
\pgfsetroundjoin%
\pgfsetlinewidth{1.003750pt}%
\definecolor{currentstroke}{rgb}{0.827451,0.827451,0.827451}%
\pgfsetstrokecolor{currentstroke}%
\pgfsetstrokeopacity{0.800000}%
\pgfsetdash{}{0pt}%
\pgfpathmoveto{\pgfqpoint{0.525220in}{4.120020in}}%
\pgfpathcurveto{\pgfqpoint{0.536270in}{4.120020in}}{\pgfqpoint{0.546869in}{4.124411in}}{\pgfqpoint{0.554683in}{4.132224in}}%
\pgfpathcurveto{\pgfqpoint{0.562496in}{4.140038in}}{\pgfqpoint{0.566887in}{4.150637in}}{\pgfqpoint{0.566887in}{4.161687in}}%
\pgfpathcurveto{\pgfqpoint{0.566887in}{4.172737in}}{\pgfqpoint{0.562496in}{4.183336in}}{\pgfqpoint{0.554683in}{4.191150in}}%
\pgfpathcurveto{\pgfqpoint{0.546869in}{4.198963in}}{\pgfqpoint{0.536270in}{4.203354in}}{\pgfqpoint{0.525220in}{4.203354in}}%
\pgfpathcurveto{\pgfqpoint{0.514170in}{4.203354in}}{\pgfqpoint{0.503571in}{4.198963in}}{\pgfqpoint{0.495757in}{4.191150in}}%
\pgfpathcurveto{\pgfqpoint{0.487944in}{4.183336in}}{\pgfqpoint{0.483553in}{4.172737in}}{\pgfqpoint{0.483553in}{4.161687in}}%
\pgfpathcurveto{\pgfqpoint{0.483553in}{4.150637in}}{\pgfqpoint{0.487944in}{4.140038in}}{\pgfqpoint{0.495757in}{4.132224in}}%
\pgfpathcurveto{\pgfqpoint{0.503571in}{4.124411in}}{\pgfqpoint{0.514170in}{4.120020in}}{\pgfqpoint{0.525220in}{4.120020in}}%
\pgfpathlineto{\pgfqpoint{0.525220in}{4.120020in}}%
\pgfpathclose%
\pgfusepath{stroke}%
\end{pgfscope}%
\begin{pgfscope}%
\pgfpathrectangle{\pgfqpoint{0.494722in}{0.437222in}}{\pgfqpoint{6.275590in}{5.159444in}}%
\pgfusepath{clip}%
\pgfsetbuttcap%
\pgfsetroundjoin%
\pgfsetlinewidth{1.003750pt}%
\definecolor{currentstroke}{rgb}{0.827451,0.827451,0.827451}%
\pgfsetstrokecolor{currentstroke}%
\pgfsetstrokeopacity{0.800000}%
\pgfsetdash{}{0pt}%
\pgfpathmoveto{\pgfqpoint{0.744905in}{3.205367in}}%
\pgfpathcurveto{\pgfqpoint{0.755955in}{3.205367in}}{\pgfqpoint{0.766554in}{3.209757in}}{\pgfqpoint{0.774368in}{3.217571in}}%
\pgfpathcurveto{\pgfqpoint{0.782181in}{3.225384in}}{\pgfqpoint{0.786571in}{3.235983in}}{\pgfqpoint{0.786571in}{3.247034in}}%
\pgfpathcurveto{\pgfqpoint{0.786571in}{3.258084in}}{\pgfqpoint{0.782181in}{3.268683in}}{\pgfqpoint{0.774368in}{3.276496in}}%
\pgfpathcurveto{\pgfqpoint{0.766554in}{3.284310in}}{\pgfqpoint{0.755955in}{3.288700in}}{\pgfqpoint{0.744905in}{3.288700in}}%
\pgfpathcurveto{\pgfqpoint{0.733855in}{3.288700in}}{\pgfqpoint{0.723256in}{3.284310in}}{\pgfqpoint{0.715442in}{3.276496in}}%
\pgfpathcurveto{\pgfqpoint{0.707628in}{3.268683in}}{\pgfqpoint{0.703238in}{3.258084in}}{\pgfqpoint{0.703238in}{3.247034in}}%
\pgfpathcurveto{\pgfqpoint{0.703238in}{3.235983in}}{\pgfqpoint{0.707628in}{3.225384in}}{\pgfqpoint{0.715442in}{3.217571in}}%
\pgfpathcurveto{\pgfqpoint{0.723256in}{3.209757in}}{\pgfqpoint{0.733855in}{3.205367in}}{\pgfqpoint{0.744905in}{3.205367in}}%
\pgfpathlineto{\pgfqpoint{0.744905in}{3.205367in}}%
\pgfpathclose%
\pgfusepath{stroke}%
\end{pgfscope}%
\begin{pgfscope}%
\pgfpathrectangle{\pgfqpoint{0.494722in}{0.437222in}}{\pgfqpoint{6.275590in}{5.159444in}}%
\pgfusepath{clip}%
\pgfsetbuttcap%
\pgfsetroundjoin%
\pgfsetlinewidth{1.003750pt}%
\definecolor{currentstroke}{rgb}{0.827451,0.827451,0.827451}%
\pgfsetstrokecolor{currentstroke}%
\pgfsetstrokeopacity{0.800000}%
\pgfsetdash{}{0pt}%
\pgfpathmoveto{\pgfqpoint{1.222540in}{2.255340in}}%
\pgfpathcurveto{\pgfqpoint{1.233590in}{2.255340in}}{\pgfqpoint{1.244189in}{2.259731in}}{\pgfqpoint{1.252002in}{2.267544in}}%
\pgfpathcurveto{\pgfqpoint{1.259816in}{2.275358in}}{\pgfqpoint{1.264206in}{2.285957in}}{\pgfqpoint{1.264206in}{2.297007in}}%
\pgfpathcurveto{\pgfqpoint{1.264206in}{2.308057in}}{\pgfqpoint{1.259816in}{2.318656in}}{\pgfqpoint{1.252002in}{2.326470in}}%
\pgfpathcurveto{\pgfqpoint{1.244189in}{2.334283in}}{\pgfqpoint{1.233590in}{2.338674in}}{\pgfqpoint{1.222540in}{2.338674in}}%
\pgfpathcurveto{\pgfqpoint{1.211490in}{2.338674in}}{\pgfqpoint{1.200891in}{2.334283in}}{\pgfqpoint{1.193077in}{2.326470in}}%
\pgfpathcurveto{\pgfqpoint{1.185263in}{2.318656in}}{\pgfqpoint{1.180873in}{2.308057in}}{\pgfqpoint{1.180873in}{2.297007in}}%
\pgfpathcurveto{\pgfqpoint{1.180873in}{2.285957in}}{\pgfqpoint{1.185263in}{2.275358in}}{\pgfqpoint{1.193077in}{2.267544in}}%
\pgfpathcurveto{\pgfqpoint{1.200891in}{2.259731in}}{\pgfqpoint{1.211490in}{2.255340in}}{\pgfqpoint{1.222540in}{2.255340in}}%
\pgfpathlineto{\pgfqpoint{1.222540in}{2.255340in}}%
\pgfpathclose%
\pgfusepath{stroke}%
\end{pgfscope}%
\begin{pgfscope}%
\pgfpathrectangle{\pgfqpoint{0.494722in}{0.437222in}}{\pgfqpoint{6.275590in}{5.159444in}}%
\pgfusepath{clip}%
\pgfsetbuttcap%
\pgfsetroundjoin%
\pgfsetlinewidth{1.003750pt}%
\definecolor{currentstroke}{rgb}{0.827451,0.827451,0.827451}%
\pgfsetstrokecolor{currentstroke}%
\pgfsetstrokeopacity{0.800000}%
\pgfsetdash{}{0pt}%
\pgfpathmoveto{\pgfqpoint{2.062312in}{1.444503in}}%
\pgfpathcurveto{\pgfqpoint{2.073362in}{1.444503in}}{\pgfqpoint{2.083961in}{1.448893in}}{\pgfqpoint{2.091775in}{1.456707in}}%
\pgfpathcurveto{\pgfqpoint{2.099588in}{1.464520in}}{\pgfqpoint{2.103979in}{1.475119in}}{\pgfqpoint{2.103979in}{1.486169in}}%
\pgfpathcurveto{\pgfqpoint{2.103979in}{1.497220in}}{\pgfqpoint{2.099588in}{1.507819in}}{\pgfqpoint{2.091775in}{1.515632in}}%
\pgfpathcurveto{\pgfqpoint{2.083961in}{1.523446in}}{\pgfqpoint{2.073362in}{1.527836in}}{\pgfqpoint{2.062312in}{1.527836in}}%
\pgfpathcurveto{\pgfqpoint{2.051262in}{1.527836in}}{\pgfqpoint{2.040663in}{1.523446in}}{\pgfqpoint{2.032849in}{1.515632in}}%
\pgfpathcurveto{\pgfqpoint{2.025036in}{1.507819in}}{\pgfqpoint{2.020645in}{1.497220in}}{\pgfqpoint{2.020645in}{1.486169in}}%
\pgfpathcurveto{\pgfqpoint{2.020645in}{1.475119in}}{\pgfqpoint{2.025036in}{1.464520in}}{\pgfqpoint{2.032849in}{1.456707in}}%
\pgfpathcurveto{\pgfqpoint{2.040663in}{1.448893in}}{\pgfqpoint{2.051262in}{1.444503in}}{\pgfqpoint{2.062312in}{1.444503in}}%
\pgfpathlineto{\pgfqpoint{2.062312in}{1.444503in}}%
\pgfpathclose%
\pgfusepath{stroke}%
\end{pgfscope}%
\begin{pgfscope}%
\pgfpathrectangle{\pgfqpoint{0.494722in}{0.437222in}}{\pgfqpoint{6.275590in}{5.159444in}}%
\pgfusepath{clip}%
\pgfsetbuttcap%
\pgfsetroundjoin%
\pgfsetlinewidth{1.003750pt}%
\definecolor{currentstroke}{rgb}{0.827451,0.827451,0.827451}%
\pgfsetstrokecolor{currentstroke}%
\pgfsetstrokeopacity{0.800000}%
\pgfsetdash{}{0pt}%
\pgfpathmoveto{\pgfqpoint{0.638504in}{3.493003in}}%
\pgfpathcurveto{\pgfqpoint{0.649554in}{3.493003in}}{\pgfqpoint{0.660153in}{3.497394in}}{\pgfqpoint{0.667967in}{3.505207in}}%
\pgfpathcurveto{\pgfqpoint{0.675780in}{3.513021in}}{\pgfqpoint{0.680170in}{3.523620in}}{\pgfqpoint{0.680170in}{3.534670in}}%
\pgfpathcurveto{\pgfqpoint{0.680170in}{3.545720in}}{\pgfqpoint{0.675780in}{3.556319in}}{\pgfqpoint{0.667967in}{3.564133in}}%
\pgfpathcurveto{\pgfqpoint{0.660153in}{3.571946in}}{\pgfqpoint{0.649554in}{3.576337in}}{\pgfqpoint{0.638504in}{3.576337in}}%
\pgfpathcurveto{\pgfqpoint{0.627454in}{3.576337in}}{\pgfqpoint{0.616855in}{3.571946in}}{\pgfqpoint{0.609041in}{3.564133in}}%
\pgfpathcurveto{\pgfqpoint{0.601227in}{3.556319in}}{\pgfqpoint{0.596837in}{3.545720in}}{\pgfqpoint{0.596837in}{3.534670in}}%
\pgfpathcurveto{\pgfqpoint{0.596837in}{3.523620in}}{\pgfqpoint{0.601227in}{3.513021in}}{\pgfqpoint{0.609041in}{3.505207in}}%
\pgfpathcurveto{\pgfqpoint{0.616855in}{3.497394in}}{\pgfqpoint{0.627454in}{3.493003in}}{\pgfqpoint{0.638504in}{3.493003in}}%
\pgfpathlineto{\pgfqpoint{0.638504in}{3.493003in}}%
\pgfpathclose%
\pgfusepath{stroke}%
\end{pgfscope}%
\begin{pgfscope}%
\pgfpathrectangle{\pgfqpoint{0.494722in}{0.437222in}}{\pgfqpoint{6.275590in}{5.159444in}}%
\pgfusepath{clip}%
\pgfsetbuttcap%
\pgfsetroundjoin%
\pgfsetlinewidth{1.003750pt}%
\definecolor{currentstroke}{rgb}{0.827451,0.827451,0.827451}%
\pgfsetstrokecolor{currentstroke}%
\pgfsetstrokeopacity{0.800000}%
\pgfsetdash{}{0pt}%
\pgfpathmoveto{\pgfqpoint{2.659012in}{1.065276in}}%
\pgfpathcurveto{\pgfqpoint{2.670062in}{1.065276in}}{\pgfqpoint{2.680661in}{1.069667in}}{\pgfqpoint{2.688475in}{1.077480in}}%
\pgfpathcurveto{\pgfqpoint{2.696289in}{1.085294in}}{\pgfqpoint{2.700679in}{1.095893in}}{\pgfqpoint{2.700679in}{1.106943in}}%
\pgfpathcurveto{\pgfqpoint{2.700679in}{1.117993in}}{\pgfqpoint{2.696289in}{1.128592in}}{\pgfqpoint{2.688475in}{1.136406in}}%
\pgfpathcurveto{\pgfqpoint{2.680661in}{1.144219in}}{\pgfqpoint{2.670062in}{1.148610in}}{\pgfqpoint{2.659012in}{1.148610in}}%
\pgfpathcurveto{\pgfqpoint{2.647962in}{1.148610in}}{\pgfqpoint{2.637363in}{1.144219in}}{\pgfqpoint{2.629550in}{1.136406in}}%
\pgfpathcurveto{\pgfqpoint{2.621736in}{1.128592in}}{\pgfqpoint{2.617346in}{1.117993in}}{\pgfqpoint{2.617346in}{1.106943in}}%
\pgfpathcurveto{\pgfqpoint{2.617346in}{1.095893in}}{\pgfqpoint{2.621736in}{1.085294in}}{\pgfqpoint{2.629550in}{1.077480in}}%
\pgfpathcurveto{\pgfqpoint{2.637363in}{1.069667in}}{\pgfqpoint{2.647962in}{1.065276in}}{\pgfqpoint{2.659012in}{1.065276in}}%
\pgfpathlineto{\pgfqpoint{2.659012in}{1.065276in}}%
\pgfpathclose%
\pgfusepath{stroke}%
\end{pgfscope}%
\begin{pgfscope}%
\pgfpathrectangle{\pgfqpoint{0.494722in}{0.437222in}}{\pgfqpoint{6.275590in}{5.159444in}}%
\pgfusepath{clip}%
\pgfsetbuttcap%
\pgfsetroundjoin%
\pgfsetlinewidth{1.003750pt}%
\definecolor{currentstroke}{rgb}{0.827451,0.827451,0.827451}%
\pgfsetstrokecolor{currentstroke}%
\pgfsetstrokeopacity{0.800000}%
\pgfsetdash{}{0pt}%
\pgfpathmoveto{\pgfqpoint{1.396743in}{2.042450in}}%
\pgfpathcurveto{\pgfqpoint{1.407793in}{2.042450in}}{\pgfqpoint{1.418392in}{2.046840in}}{\pgfqpoint{1.426206in}{2.054654in}}%
\pgfpathcurveto{\pgfqpoint{1.434019in}{2.062467in}}{\pgfqpoint{1.438409in}{2.073066in}}{\pgfqpoint{1.438409in}{2.084117in}}%
\pgfpathcurveto{\pgfqpoint{1.438409in}{2.095167in}}{\pgfqpoint{1.434019in}{2.105766in}}{\pgfqpoint{1.426206in}{2.113579in}}%
\pgfpathcurveto{\pgfqpoint{1.418392in}{2.121393in}}{\pgfqpoint{1.407793in}{2.125783in}}{\pgfqpoint{1.396743in}{2.125783in}}%
\pgfpathcurveto{\pgfqpoint{1.385693in}{2.125783in}}{\pgfqpoint{1.375094in}{2.121393in}}{\pgfqpoint{1.367280in}{2.113579in}}%
\pgfpathcurveto{\pgfqpoint{1.359466in}{2.105766in}}{\pgfqpoint{1.355076in}{2.095167in}}{\pgfqpoint{1.355076in}{2.084117in}}%
\pgfpathcurveto{\pgfqpoint{1.355076in}{2.073066in}}{\pgfqpoint{1.359466in}{2.062467in}}{\pgfqpoint{1.367280in}{2.054654in}}%
\pgfpathcurveto{\pgfqpoint{1.375094in}{2.046840in}}{\pgfqpoint{1.385693in}{2.042450in}}{\pgfqpoint{1.396743in}{2.042450in}}%
\pgfpathlineto{\pgfqpoint{1.396743in}{2.042450in}}%
\pgfpathclose%
\pgfusepath{stroke}%
\end{pgfscope}%
\begin{pgfscope}%
\pgfpathrectangle{\pgfqpoint{0.494722in}{0.437222in}}{\pgfqpoint{6.275590in}{5.159444in}}%
\pgfusepath{clip}%
\pgfsetbuttcap%
\pgfsetroundjoin%
\pgfsetlinewidth{1.003750pt}%
\definecolor{currentstroke}{rgb}{0.827451,0.827451,0.827451}%
\pgfsetstrokecolor{currentstroke}%
\pgfsetstrokeopacity{0.800000}%
\pgfsetdash{}{0pt}%
\pgfpathmoveto{\pgfqpoint{1.507243in}{1.940589in}}%
\pgfpathcurveto{\pgfqpoint{1.518293in}{1.940589in}}{\pgfqpoint{1.528892in}{1.944980in}}{\pgfqpoint{1.536706in}{1.952793in}}%
\pgfpathcurveto{\pgfqpoint{1.544520in}{1.960607in}}{\pgfqpoint{1.548910in}{1.971206in}}{\pgfqpoint{1.548910in}{1.982256in}}%
\pgfpathcurveto{\pgfqpoint{1.548910in}{1.993306in}}{\pgfqpoint{1.544520in}{2.003905in}}{\pgfqpoint{1.536706in}{2.011719in}}%
\pgfpathcurveto{\pgfqpoint{1.528892in}{2.019532in}}{\pgfqpoint{1.518293in}{2.023923in}}{\pgfqpoint{1.507243in}{2.023923in}}%
\pgfpathcurveto{\pgfqpoint{1.496193in}{2.023923in}}{\pgfqpoint{1.485594in}{2.019532in}}{\pgfqpoint{1.477781in}{2.011719in}}%
\pgfpathcurveto{\pgfqpoint{1.469967in}{2.003905in}}{\pgfqpoint{1.465577in}{1.993306in}}{\pgfqpoint{1.465577in}{1.982256in}}%
\pgfpathcurveto{\pgfqpoint{1.465577in}{1.971206in}}{\pgfqpoint{1.469967in}{1.960607in}}{\pgfqpoint{1.477781in}{1.952793in}}%
\pgfpathcurveto{\pgfqpoint{1.485594in}{1.944980in}}{\pgfqpoint{1.496193in}{1.940589in}}{\pgfqpoint{1.507243in}{1.940589in}}%
\pgfpathlineto{\pgfqpoint{1.507243in}{1.940589in}}%
\pgfpathclose%
\pgfusepath{stroke}%
\end{pgfscope}%
\begin{pgfscope}%
\pgfpathrectangle{\pgfqpoint{0.494722in}{0.437222in}}{\pgfqpoint{6.275590in}{5.159444in}}%
\pgfusepath{clip}%
\pgfsetbuttcap%
\pgfsetroundjoin%
\pgfsetlinewidth{1.003750pt}%
\definecolor{currentstroke}{rgb}{0.827451,0.827451,0.827451}%
\pgfsetstrokecolor{currentstroke}%
\pgfsetstrokeopacity{0.800000}%
\pgfsetdash{}{0pt}%
\pgfpathmoveto{\pgfqpoint{1.786945in}{1.657846in}}%
\pgfpathcurveto{\pgfqpoint{1.797995in}{1.657846in}}{\pgfqpoint{1.808594in}{1.662237in}}{\pgfqpoint{1.816407in}{1.670050in}}%
\pgfpathcurveto{\pgfqpoint{1.824221in}{1.677864in}}{\pgfqpoint{1.828611in}{1.688463in}}{\pgfqpoint{1.828611in}{1.699513in}}%
\pgfpathcurveto{\pgfqpoint{1.828611in}{1.710563in}}{\pgfqpoint{1.824221in}{1.721162in}}{\pgfqpoint{1.816407in}{1.728976in}}%
\pgfpathcurveto{\pgfqpoint{1.808594in}{1.736789in}}{\pgfqpoint{1.797995in}{1.741180in}}{\pgfqpoint{1.786945in}{1.741180in}}%
\pgfpathcurveto{\pgfqpoint{1.775894in}{1.741180in}}{\pgfqpoint{1.765295in}{1.736789in}}{\pgfqpoint{1.757482in}{1.728976in}}%
\pgfpathcurveto{\pgfqpoint{1.749668in}{1.721162in}}{\pgfqpoint{1.745278in}{1.710563in}}{\pgfqpoint{1.745278in}{1.699513in}}%
\pgfpathcurveto{\pgfqpoint{1.745278in}{1.688463in}}{\pgfqpoint{1.749668in}{1.677864in}}{\pgfqpoint{1.757482in}{1.670050in}}%
\pgfpathcurveto{\pgfqpoint{1.765295in}{1.662237in}}{\pgfqpoint{1.775894in}{1.657846in}}{\pgfqpoint{1.786945in}{1.657846in}}%
\pgfpathlineto{\pgfqpoint{1.786945in}{1.657846in}}%
\pgfpathclose%
\pgfusepath{stroke}%
\end{pgfscope}%
\begin{pgfscope}%
\pgfpathrectangle{\pgfqpoint{0.494722in}{0.437222in}}{\pgfqpoint{6.275590in}{5.159444in}}%
\pgfusepath{clip}%
\pgfsetbuttcap%
\pgfsetroundjoin%
\pgfsetlinewidth{1.003750pt}%
\definecolor{currentstroke}{rgb}{0.827451,0.827451,0.827451}%
\pgfsetstrokecolor{currentstroke}%
\pgfsetstrokeopacity{0.800000}%
\pgfsetdash{}{0pt}%
\pgfpathmoveto{\pgfqpoint{4.970960in}{0.441718in}}%
\pgfpathcurveto{\pgfqpoint{4.982010in}{0.441718in}}{\pgfqpoint{4.992609in}{0.446108in}}{\pgfqpoint{5.000423in}{0.453922in}}%
\pgfpathcurveto{\pgfqpoint{5.008236in}{0.461735in}}{\pgfqpoint{5.012626in}{0.472334in}}{\pgfqpoint{5.012626in}{0.483385in}}%
\pgfpathcurveto{\pgfqpoint{5.012626in}{0.494435in}}{\pgfqpoint{5.008236in}{0.505034in}}{\pgfqpoint{5.000423in}{0.512847in}}%
\pgfpathcurveto{\pgfqpoint{4.992609in}{0.520661in}}{\pgfqpoint{4.982010in}{0.525051in}}{\pgfqpoint{4.970960in}{0.525051in}}%
\pgfpathcurveto{\pgfqpoint{4.959910in}{0.525051in}}{\pgfqpoint{4.949311in}{0.520661in}}{\pgfqpoint{4.941497in}{0.512847in}}%
\pgfpathcurveto{\pgfqpoint{4.933683in}{0.505034in}}{\pgfqpoint{4.929293in}{0.494435in}}{\pgfqpoint{4.929293in}{0.483385in}}%
\pgfpathcurveto{\pgfqpoint{4.929293in}{0.472334in}}{\pgfqpoint{4.933683in}{0.461735in}}{\pgfqpoint{4.941497in}{0.453922in}}%
\pgfpathcurveto{\pgfqpoint{4.949311in}{0.446108in}}{\pgfqpoint{4.959910in}{0.441718in}}{\pgfqpoint{4.970960in}{0.441718in}}%
\pgfpathlineto{\pgfqpoint{4.970960in}{0.441718in}}%
\pgfpathclose%
\pgfusepath{stroke}%
\end{pgfscope}%
\begin{pgfscope}%
\pgfpathrectangle{\pgfqpoint{0.494722in}{0.437222in}}{\pgfqpoint{6.275590in}{5.159444in}}%
\pgfusepath{clip}%
\pgfsetbuttcap%
\pgfsetroundjoin%
\pgfsetlinewidth{1.003750pt}%
\definecolor{currentstroke}{rgb}{0.827451,0.827451,0.827451}%
\pgfsetstrokecolor{currentstroke}%
\pgfsetstrokeopacity{0.800000}%
\pgfsetdash{}{0pt}%
\pgfpathmoveto{\pgfqpoint{0.700772in}{3.267894in}}%
\pgfpathcurveto{\pgfqpoint{0.711822in}{3.267894in}}{\pgfqpoint{0.722421in}{3.272284in}}{\pgfqpoint{0.730235in}{3.280098in}}%
\pgfpathcurveto{\pgfqpoint{0.738048in}{3.287911in}}{\pgfqpoint{0.742438in}{3.298510in}}{\pgfqpoint{0.742438in}{3.309561in}}%
\pgfpathcurveto{\pgfqpoint{0.742438in}{3.320611in}}{\pgfqpoint{0.738048in}{3.331210in}}{\pgfqpoint{0.730235in}{3.339023in}}%
\pgfpathcurveto{\pgfqpoint{0.722421in}{3.346837in}}{\pgfqpoint{0.711822in}{3.351227in}}{\pgfqpoint{0.700772in}{3.351227in}}%
\pgfpathcurveto{\pgfqpoint{0.689722in}{3.351227in}}{\pgfqpoint{0.679123in}{3.346837in}}{\pgfqpoint{0.671309in}{3.339023in}}%
\pgfpathcurveto{\pgfqpoint{0.663495in}{3.331210in}}{\pgfqpoint{0.659105in}{3.320611in}}{\pgfqpoint{0.659105in}{3.309561in}}%
\pgfpathcurveto{\pgfqpoint{0.659105in}{3.298510in}}{\pgfqpoint{0.663495in}{3.287911in}}{\pgfqpoint{0.671309in}{3.280098in}}%
\pgfpathcurveto{\pgfqpoint{0.679123in}{3.272284in}}{\pgfqpoint{0.689722in}{3.267894in}}{\pgfqpoint{0.700772in}{3.267894in}}%
\pgfpathlineto{\pgfqpoint{0.700772in}{3.267894in}}%
\pgfpathclose%
\pgfusepath{stroke}%
\end{pgfscope}%
\begin{pgfscope}%
\pgfpathrectangle{\pgfqpoint{0.494722in}{0.437222in}}{\pgfqpoint{6.275590in}{5.159444in}}%
\pgfusepath{clip}%
\pgfsetbuttcap%
\pgfsetroundjoin%
\pgfsetlinewidth{1.003750pt}%
\definecolor{currentstroke}{rgb}{0.827451,0.827451,0.827451}%
\pgfsetstrokecolor{currentstroke}%
\pgfsetstrokeopacity{0.800000}%
\pgfsetdash{}{0pt}%
\pgfpathmoveto{\pgfqpoint{3.460812in}{0.723768in}}%
\pgfpathcurveto{\pgfqpoint{3.471862in}{0.723768in}}{\pgfqpoint{3.482461in}{0.728158in}}{\pgfqpoint{3.490274in}{0.735972in}}%
\pgfpathcurveto{\pgfqpoint{3.498088in}{0.743785in}}{\pgfqpoint{3.502478in}{0.754384in}}{\pgfqpoint{3.502478in}{0.765434in}}%
\pgfpathcurveto{\pgfqpoint{3.502478in}{0.776484in}}{\pgfqpoint{3.498088in}{0.787083in}}{\pgfqpoint{3.490274in}{0.794897in}}%
\pgfpathcurveto{\pgfqpoint{3.482461in}{0.802711in}}{\pgfqpoint{3.471862in}{0.807101in}}{\pgfqpoint{3.460812in}{0.807101in}}%
\pgfpathcurveto{\pgfqpoint{3.449762in}{0.807101in}}{\pgfqpoint{3.439162in}{0.802711in}}{\pgfqpoint{3.431349in}{0.794897in}}%
\pgfpathcurveto{\pgfqpoint{3.423535in}{0.787083in}}{\pgfqpoint{3.419145in}{0.776484in}}{\pgfqpoint{3.419145in}{0.765434in}}%
\pgfpathcurveto{\pgfqpoint{3.419145in}{0.754384in}}{\pgfqpoint{3.423535in}{0.743785in}}{\pgfqpoint{3.431349in}{0.735972in}}%
\pgfpathcurveto{\pgfqpoint{3.439162in}{0.728158in}}{\pgfqpoint{3.449762in}{0.723768in}}{\pgfqpoint{3.460812in}{0.723768in}}%
\pgfpathlineto{\pgfqpoint{3.460812in}{0.723768in}}%
\pgfpathclose%
\pgfusepath{stroke}%
\end{pgfscope}%
\begin{pgfscope}%
\pgfpathrectangle{\pgfqpoint{0.494722in}{0.437222in}}{\pgfqpoint{6.275590in}{5.159444in}}%
\pgfusepath{clip}%
\pgfsetbuttcap%
\pgfsetroundjoin%
\pgfsetlinewidth{1.003750pt}%
\definecolor{currentstroke}{rgb}{0.827451,0.827451,0.827451}%
\pgfsetstrokecolor{currentstroke}%
\pgfsetstrokeopacity{0.800000}%
\pgfsetdash{}{0pt}%
\pgfpathmoveto{\pgfqpoint{2.032862in}{1.459393in}}%
\pgfpathcurveto{\pgfqpoint{2.043912in}{1.459393in}}{\pgfqpoint{2.054511in}{1.463783in}}{\pgfqpoint{2.062325in}{1.471597in}}%
\pgfpathcurveto{\pgfqpoint{2.070138in}{1.479411in}}{\pgfqpoint{2.074529in}{1.490010in}}{\pgfqpoint{2.074529in}{1.501060in}}%
\pgfpathcurveto{\pgfqpoint{2.074529in}{1.512110in}}{\pgfqpoint{2.070138in}{1.522709in}}{\pgfqpoint{2.062325in}{1.530523in}}%
\pgfpathcurveto{\pgfqpoint{2.054511in}{1.538336in}}{\pgfqpoint{2.043912in}{1.542726in}}{\pgfqpoint{2.032862in}{1.542726in}}%
\pgfpathcurveto{\pgfqpoint{2.021812in}{1.542726in}}{\pgfqpoint{2.011213in}{1.538336in}}{\pgfqpoint{2.003399in}{1.530523in}}%
\pgfpathcurveto{\pgfqpoint{1.995586in}{1.522709in}}{\pgfqpoint{1.991195in}{1.512110in}}{\pgfqpoint{1.991195in}{1.501060in}}%
\pgfpathcurveto{\pgfqpoint{1.991195in}{1.490010in}}{\pgfqpoint{1.995586in}{1.479411in}}{\pgfqpoint{2.003399in}{1.471597in}}%
\pgfpathcurveto{\pgfqpoint{2.011213in}{1.463783in}}{\pgfqpoint{2.021812in}{1.459393in}}{\pgfqpoint{2.032862in}{1.459393in}}%
\pgfpathlineto{\pgfqpoint{2.032862in}{1.459393in}}%
\pgfpathclose%
\pgfusepath{stroke}%
\end{pgfscope}%
\begin{pgfscope}%
\pgfpathrectangle{\pgfqpoint{0.494722in}{0.437222in}}{\pgfqpoint{6.275590in}{5.159444in}}%
\pgfusepath{clip}%
\pgfsetbuttcap%
\pgfsetroundjoin%
\pgfsetlinewidth{1.003750pt}%
\definecolor{currentstroke}{rgb}{0.827451,0.827451,0.827451}%
\pgfsetstrokecolor{currentstroke}%
\pgfsetstrokeopacity{0.800000}%
\pgfsetdash{}{0pt}%
\pgfpathmoveto{\pgfqpoint{1.315544in}{2.144827in}}%
\pgfpathcurveto{\pgfqpoint{1.326594in}{2.144827in}}{\pgfqpoint{1.337193in}{2.149218in}}{\pgfqpoint{1.345006in}{2.157031in}}%
\pgfpathcurveto{\pgfqpoint{1.352820in}{2.164845in}}{\pgfqpoint{1.357210in}{2.175444in}}{\pgfqpoint{1.357210in}{2.186494in}}%
\pgfpathcurveto{\pgfqpoint{1.357210in}{2.197544in}}{\pgfqpoint{1.352820in}{2.208143in}}{\pgfqpoint{1.345006in}{2.215957in}}%
\pgfpathcurveto{\pgfqpoint{1.337193in}{2.223770in}}{\pgfqpoint{1.326594in}{2.228161in}}{\pgfqpoint{1.315544in}{2.228161in}}%
\pgfpathcurveto{\pgfqpoint{1.304494in}{2.228161in}}{\pgfqpoint{1.293894in}{2.223770in}}{\pgfqpoint{1.286081in}{2.215957in}}%
\pgfpathcurveto{\pgfqpoint{1.278267in}{2.208143in}}{\pgfqpoint{1.273877in}{2.197544in}}{\pgfqpoint{1.273877in}{2.186494in}}%
\pgfpathcurveto{\pgfqpoint{1.273877in}{2.175444in}}{\pgfqpoint{1.278267in}{2.164845in}}{\pgfqpoint{1.286081in}{2.157031in}}%
\pgfpathcurveto{\pgfqpoint{1.293894in}{2.149218in}}{\pgfqpoint{1.304494in}{2.144827in}}{\pgfqpoint{1.315544in}{2.144827in}}%
\pgfpathlineto{\pgfqpoint{1.315544in}{2.144827in}}%
\pgfpathclose%
\pgfusepath{stroke}%
\end{pgfscope}%
\begin{pgfscope}%
\pgfpathrectangle{\pgfqpoint{0.494722in}{0.437222in}}{\pgfqpoint{6.275590in}{5.159444in}}%
\pgfusepath{clip}%
\pgfsetbuttcap%
\pgfsetroundjoin%
\pgfsetlinewidth{1.003750pt}%
\definecolor{currentstroke}{rgb}{0.827451,0.827451,0.827451}%
\pgfsetstrokecolor{currentstroke}%
\pgfsetstrokeopacity{0.800000}%
\pgfsetdash{}{0pt}%
\pgfpathmoveto{\pgfqpoint{2.940552in}{0.958089in}}%
\pgfpathcurveto{\pgfqpoint{2.951602in}{0.958089in}}{\pgfqpoint{2.962201in}{0.962479in}}{\pgfqpoint{2.970015in}{0.970293in}}%
\pgfpathcurveto{\pgfqpoint{2.977829in}{0.978106in}}{\pgfqpoint{2.982219in}{0.988705in}}{\pgfqpoint{2.982219in}{0.999756in}}%
\pgfpathcurveto{\pgfqpoint{2.982219in}{1.010806in}}{\pgfqpoint{2.977829in}{1.021405in}}{\pgfqpoint{2.970015in}{1.029218in}}%
\pgfpathcurveto{\pgfqpoint{2.962201in}{1.037032in}}{\pgfqpoint{2.951602in}{1.041422in}}{\pgfqpoint{2.940552in}{1.041422in}}%
\pgfpathcurveto{\pgfqpoint{2.929502in}{1.041422in}}{\pgfqpoint{2.918903in}{1.037032in}}{\pgfqpoint{2.911089in}{1.029218in}}%
\pgfpathcurveto{\pgfqpoint{2.903276in}{1.021405in}}{\pgfqpoint{2.898885in}{1.010806in}}{\pgfqpoint{2.898885in}{0.999756in}}%
\pgfpathcurveto{\pgfqpoint{2.898885in}{0.988705in}}{\pgfqpoint{2.903276in}{0.978106in}}{\pgfqpoint{2.911089in}{0.970293in}}%
\pgfpathcurveto{\pgfqpoint{2.918903in}{0.962479in}}{\pgfqpoint{2.929502in}{0.958089in}}{\pgfqpoint{2.940552in}{0.958089in}}%
\pgfpathlineto{\pgfqpoint{2.940552in}{0.958089in}}%
\pgfpathclose%
\pgfusepath{stroke}%
\end{pgfscope}%
\begin{pgfscope}%
\pgfpathrectangle{\pgfqpoint{0.494722in}{0.437222in}}{\pgfqpoint{6.275590in}{5.159444in}}%
\pgfusepath{clip}%
\pgfsetbuttcap%
\pgfsetroundjoin%
\pgfsetlinewidth{1.003750pt}%
\definecolor{currentstroke}{rgb}{0.827451,0.827451,0.827451}%
\pgfsetstrokecolor{currentstroke}%
\pgfsetstrokeopacity{0.800000}%
\pgfsetdash{}{0pt}%
\pgfpathmoveto{\pgfqpoint{0.584355in}{3.716646in}}%
\pgfpathcurveto{\pgfqpoint{0.595406in}{3.716646in}}{\pgfqpoint{0.606005in}{3.721036in}}{\pgfqpoint{0.613818in}{3.728850in}}%
\pgfpathcurveto{\pgfqpoint{0.621632in}{3.736663in}}{\pgfqpoint{0.626022in}{3.747262in}}{\pgfqpoint{0.626022in}{3.758313in}}%
\pgfpathcurveto{\pgfqpoint{0.626022in}{3.769363in}}{\pgfqpoint{0.621632in}{3.779962in}}{\pgfqpoint{0.613818in}{3.787775in}}%
\pgfpathcurveto{\pgfqpoint{0.606005in}{3.795589in}}{\pgfqpoint{0.595406in}{3.799979in}}{\pgfqpoint{0.584355in}{3.799979in}}%
\pgfpathcurveto{\pgfqpoint{0.573305in}{3.799979in}}{\pgfqpoint{0.562706in}{3.795589in}}{\pgfqpoint{0.554893in}{3.787775in}}%
\pgfpathcurveto{\pgfqpoint{0.547079in}{3.779962in}}{\pgfqpoint{0.542689in}{3.769363in}}{\pgfqpoint{0.542689in}{3.758313in}}%
\pgfpathcurveto{\pgfqpoint{0.542689in}{3.747262in}}{\pgfqpoint{0.547079in}{3.736663in}}{\pgfqpoint{0.554893in}{3.728850in}}%
\pgfpathcurveto{\pgfqpoint{0.562706in}{3.721036in}}{\pgfqpoint{0.573305in}{3.716646in}}{\pgfqpoint{0.584355in}{3.716646in}}%
\pgfpathlineto{\pgfqpoint{0.584355in}{3.716646in}}%
\pgfpathclose%
\pgfusepath{stroke}%
\end{pgfscope}%
\begin{pgfscope}%
\pgfpathrectangle{\pgfqpoint{0.494722in}{0.437222in}}{\pgfqpoint{6.275590in}{5.159444in}}%
\pgfusepath{clip}%
\pgfsetbuttcap%
\pgfsetroundjoin%
\pgfsetlinewidth{1.003750pt}%
\definecolor{currentstroke}{rgb}{0.827451,0.827451,0.827451}%
\pgfsetstrokecolor{currentstroke}%
\pgfsetstrokeopacity{0.800000}%
\pgfsetdash{}{0pt}%
\pgfpathmoveto{\pgfqpoint{3.208707in}{0.820522in}}%
\pgfpathcurveto{\pgfqpoint{3.219757in}{0.820522in}}{\pgfqpoint{3.230356in}{0.824912in}}{\pgfqpoint{3.238170in}{0.832726in}}%
\pgfpathcurveto{\pgfqpoint{3.245983in}{0.840539in}}{\pgfqpoint{3.250373in}{0.851139in}}{\pgfqpoint{3.250373in}{0.862189in}}%
\pgfpathcurveto{\pgfqpoint{3.250373in}{0.873239in}}{\pgfqpoint{3.245983in}{0.883838in}}{\pgfqpoint{3.238170in}{0.891651in}}%
\pgfpathcurveto{\pgfqpoint{3.230356in}{0.899465in}}{\pgfqpoint{3.219757in}{0.903855in}}{\pgfqpoint{3.208707in}{0.903855in}}%
\pgfpathcurveto{\pgfqpoint{3.197657in}{0.903855in}}{\pgfqpoint{3.187058in}{0.899465in}}{\pgfqpoint{3.179244in}{0.891651in}}%
\pgfpathcurveto{\pgfqpoint{3.171430in}{0.883838in}}{\pgfqpoint{3.167040in}{0.873239in}}{\pgfqpoint{3.167040in}{0.862189in}}%
\pgfpathcurveto{\pgfqpoint{3.167040in}{0.851139in}}{\pgfqpoint{3.171430in}{0.840539in}}{\pgfqpoint{3.179244in}{0.832726in}}%
\pgfpathcurveto{\pgfqpoint{3.187058in}{0.824912in}}{\pgfqpoint{3.197657in}{0.820522in}}{\pgfqpoint{3.208707in}{0.820522in}}%
\pgfpathlineto{\pgfqpoint{3.208707in}{0.820522in}}%
\pgfpathclose%
\pgfusepath{stroke}%
\end{pgfscope}%
\begin{pgfscope}%
\pgfpathrectangle{\pgfqpoint{0.494722in}{0.437222in}}{\pgfqpoint{6.275590in}{5.159444in}}%
\pgfusepath{clip}%
\pgfsetbuttcap%
\pgfsetroundjoin%
\pgfsetlinewidth{1.003750pt}%
\definecolor{currentstroke}{rgb}{0.827451,0.827451,0.827451}%
\pgfsetstrokecolor{currentstroke}%
\pgfsetstrokeopacity{0.800000}%
\pgfsetdash{}{0pt}%
\pgfpathmoveto{\pgfqpoint{1.581391in}{1.870889in}}%
\pgfpathcurveto{\pgfqpoint{1.592441in}{1.870889in}}{\pgfqpoint{1.603040in}{1.875279in}}{\pgfqpoint{1.610854in}{1.883093in}}%
\pgfpathcurveto{\pgfqpoint{1.618667in}{1.890906in}}{\pgfqpoint{1.623058in}{1.901505in}}{\pgfqpoint{1.623058in}{1.912556in}}%
\pgfpathcurveto{\pgfqpoint{1.623058in}{1.923606in}}{\pgfqpoint{1.618667in}{1.934205in}}{\pgfqpoint{1.610854in}{1.942018in}}%
\pgfpathcurveto{\pgfqpoint{1.603040in}{1.949832in}}{\pgfqpoint{1.592441in}{1.954222in}}{\pgfqpoint{1.581391in}{1.954222in}}%
\pgfpathcurveto{\pgfqpoint{1.570341in}{1.954222in}}{\pgfqpoint{1.559742in}{1.949832in}}{\pgfqpoint{1.551928in}{1.942018in}}%
\pgfpathcurveto{\pgfqpoint{1.544115in}{1.934205in}}{\pgfqpoint{1.539724in}{1.923606in}}{\pgfqpoint{1.539724in}{1.912556in}}%
\pgfpathcurveto{\pgfqpoint{1.539724in}{1.901505in}}{\pgfqpoint{1.544115in}{1.890906in}}{\pgfqpoint{1.551928in}{1.883093in}}%
\pgfpathcurveto{\pgfqpoint{1.559742in}{1.875279in}}{\pgfqpoint{1.570341in}{1.870889in}}{\pgfqpoint{1.581391in}{1.870889in}}%
\pgfpathlineto{\pgfqpoint{1.581391in}{1.870889in}}%
\pgfpathclose%
\pgfusepath{stroke}%
\end{pgfscope}%
\begin{pgfscope}%
\pgfpathrectangle{\pgfqpoint{0.494722in}{0.437222in}}{\pgfqpoint{6.275590in}{5.159444in}}%
\pgfusepath{clip}%
\pgfsetbuttcap%
\pgfsetroundjoin%
\pgfsetlinewidth{1.003750pt}%
\definecolor{currentstroke}{rgb}{0.827451,0.827451,0.827451}%
\pgfsetstrokecolor{currentstroke}%
\pgfsetstrokeopacity{0.800000}%
\pgfsetdash{}{0pt}%
\pgfpathmoveto{\pgfqpoint{4.335578in}{0.505432in}}%
\pgfpathcurveto{\pgfqpoint{4.346628in}{0.505432in}}{\pgfqpoint{4.357227in}{0.509823in}}{\pgfqpoint{4.365041in}{0.517636in}}%
\pgfpathcurveto{\pgfqpoint{4.372855in}{0.525450in}}{\pgfqpoint{4.377245in}{0.536049in}}{\pgfqpoint{4.377245in}{0.547099in}}%
\pgfpathcurveto{\pgfqpoint{4.377245in}{0.558149in}}{\pgfqpoint{4.372855in}{0.568748in}}{\pgfqpoint{4.365041in}{0.576562in}}%
\pgfpathcurveto{\pgfqpoint{4.357227in}{0.584375in}}{\pgfqpoint{4.346628in}{0.588766in}}{\pgfqpoint{4.335578in}{0.588766in}}%
\pgfpathcurveto{\pgfqpoint{4.324528in}{0.588766in}}{\pgfqpoint{4.313929in}{0.584375in}}{\pgfqpoint{4.306115in}{0.576562in}}%
\pgfpathcurveto{\pgfqpoint{4.298302in}{0.568748in}}{\pgfqpoint{4.293911in}{0.558149in}}{\pgfqpoint{4.293911in}{0.547099in}}%
\pgfpathcurveto{\pgfqpoint{4.293911in}{0.536049in}}{\pgfqpoint{4.298302in}{0.525450in}}{\pgfqpoint{4.306115in}{0.517636in}}%
\pgfpathcurveto{\pgfqpoint{4.313929in}{0.509823in}}{\pgfqpoint{4.324528in}{0.505432in}}{\pgfqpoint{4.335578in}{0.505432in}}%
\pgfpathlineto{\pgfqpoint{4.335578in}{0.505432in}}%
\pgfpathclose%
\pgfusepath{stroke}%
\end{pgfscope}%
\begin{pgfscope}%
\pgfpathrectangle{\pgfqpoint{0.494722in}{0.437222in}}{\pgfqpoint{6.275590in}{5.159444in}}%
\pgfusepath{clip}%
\pgfsetbuttcap%
\pgfsetroundjoin%
\pgfsetlinewidth{1.003750pt}%
\definecolor{currentstroke}{rgb}{0.827451,0.827451,0.827451}%
\pgfsetstrokecolor{currentstroke}%
\pgfsetstrokeopacity{0.800000}%
\pgfsetdash{}{0pt}%
\pgfpathmoveto{\pgfqpoint{1.545871in}{1.889457in}}%
\pgfpathcurveto{\pgfqpoint{1.556921in}{1.889457in}}{\pgfqpoint{1.567520in}{1.893848in}}{\pgfqpoint{1.575334in}{1.901661in}}%
\pgfpathcurveto{\pgfqpoint{1.583148in}{1.909475in}}{\pgfqpoint{1.587538in}{1.920074in}}{\pgfqpoint{1.587538in}{1.931124in}}%
\pgfpathcurveto{\pgfqpoint{1.587538in}{1.942174in}}{\pgfqpoint{1.583148in}{1.952773in}}{\pgfqpoint{1.575334in}{1.960587in}}%
\pgfpathcurveto{\pgfqpoint{1.567520in}{1.968400in}}{\pgfqpoint{1.556921in}{1.972791in}}{\pgfqpoint{1.545871in}{1.972791in}}%
\pgfpathcurveto{\pgfqpoint{1.534821in}{1.972791in}}{\pgfqpoint{1.524222in}{1.968400in}}{\pgfqpoint{1.516408in}{1.960587in}}%
\pgfpathcurveto{\pgfqpoint{1.508595in}{1.952773in}}{\pgfqpoint{1.504204in}{1.942174in}}{\pgfqpoint{1.504204in}{1.931124in}}%
\pgfpathcurveto{\pgfqpoint{1.504204in}{1.920074in}}{\pgfqpoint{1.508595in}{1.909475in}}{\pgfqpoint{1.516408in}{1.901661in}}%
\pgfpathcurveto{\pgfqpoint{1.524222in}{1.893848in}}{\pgfqpoint{1.534821in}{1.889457in}}{\pgfqpoint{1.545871in}{1.889457in}}%
\pgfpathlineto{\pgfqpoint{1.545871in}{1.889457in}}%
\pgfpathclose%
\pgfusepath{stroke}%
\end{pgfscope}%
\begin{pgfscope}%
\pgfpathrectangle{\pgfqpoint{0.494722in}{0.437222in}}{\pgfqpoint{6.275590in}{5.159444in}}%
\pgfusepath{clip}%
\pgfsetbuttcap%
\pgfsetroundjoin%
\pgfsetlinewidth{1.003750pt}%
\definecolor{currentstroke}{rgb}{0.827451,0.827451,0.827451}%
\pgfsetstrokecolor{currentstroke}%
\pgfsetstrokeopacity{0.800000}%
\pgfsetdash{}{0pt}%
\pgfpathmoveto{\pgfqpoint{4.229480in}{0.518110in}}%
\pgfpathcurveto{\pgfqpoint{4.240530in}{0.518110in}}{\pgfqpoint{4.251129in}{0.522500in}}{\pgfqpoint{4.258942in}{0.530314in}}%
\pgfpathcurveto{\pgfqpoint{4.266756in}{0.538127in}}{\pgfqpoint{4.271146in}{0.548726in}}{\pgfqpoint{4.271146in}{0.559776in}}%
\pgfpathcurveto{\pgfqpoint{4.271146in}{0.570827in}}{\pgfqpoint{4.266756in}{0.581426in}}{\pgfqpoint{4.258942in}{0.589239in}}%
\pgfpathcurveto{\pgfqpoint{4.251129in}{0.597053in}}{\pgfqpoint{4.240530in}{0.601443in}}{\pgfqpoint{4.229480in}{0.601443in}}%
\pgfpathcurveto{\pgfqpoint{4.218429in}{0.601443in}}{\pgfqpoint{4.207830in}{0.597053in}}{\pgfqpoint{4.200017in}{0.589239in}}%
\pgfpathcurveto{\pgfqpoint{4.192203in}{0.581426in}}{\pgfqpoint{4.187813in}{0.570827in}}{\pgfqpoint{4.187813in}{0.559776in}}%
\pgfpathcurveto{\pgfqpoint{4.187813in}{0.548726in}}{\pgfqpoint{4.192203in}{0.538127in}}{\pgfqpoint{4.200017in}{0.530314in}}%
\pgfpathcurveto{\pgfqpoint{4.207830in}{0.522500in}}{\pgfqpoint{4.218429in}{0.518110in}}{\pgfqpoint{4.229480in}{0.518110in}}%
\pgfpathlineto{\pgfqpoint{4.229480in}{0.518110in}}%
\pgfpathclose%
\pgfusepath{stroke}%
\end{pgfscope}%
\begin{pgfscope}%
\pgfpathrectangle{\pgfqpoint{0.494722in}{0.437222in}}{\pgfqpoint{6.275590in}{5.159444in}}%
\pgfusepath{clip}%
\pgfsetbuttcap%
\pgfsetroundjoin%
\pgfsetlinewidth{1.003750pt}%
\definecolor{currentstroke}{rgb}{0.827451,0.827451,0.827451}%
\pgfsetstrokecolor{currentstroke}%
\pgfsetstrokeopacity{0.800000}%
\pgfsetdash{}{0pt}%
\pgfpathmoveto{\pgfqpoint{2.117156in}{1.399260in}}%
\pgfpathcurveto{\pgfqpoint{2.128206in}{1.399260in}}{\pgfqpoint{2.138805in}{1.403650in}}{\pgfqpoint{2.146618in}{1.411464in}}%
\pgfpathcurveto{\pgfqpoint{2.154432in}{1.419277in}}{\pgfqpoint{2.158822in}{1.429876in}}{\pgfqpoint{2.158822in}{1.440927in}}%
\pgfpathcurveto{\pgfqpoint{2.158822in}{1.451977in}}{\pgfqpoint{2.154432in}{1.462576in}}{\pgfqpoint{2.146618in}{1.470389in}}%
\pgfpathcurveto{\pgfqpoint{2.138805in}{1.478203in}}{\pgfqpoint{2.128206in}{1.482593in}}{\pgfqpoint{2.117156in}{1.482593in}}%
\pgfpathcurveto{\pgfqpoint{2.106106in}{1.482593in}}{\pgfqpoint{2.095506in}{1.478203in}}{\pgfqpoint{2.087693in}{1.470389in}}%
\pgfpathcurveto{\pgfqpoint{2.079879in}{1.462576in}}{\pgfqpoint{2.075489in}{1.451977in}}{\pgfqpoint{2.075489in}{1.440927in}}%
\pgfpathcurveto{\pgfqpoint{2.075489in}{1.429876in}}{\pgfqpoint{2.079879in}{1.419277in}}{\pgfqpoint{2.087693in}{1.411464in}}%
\pgfpathcurveto{\pgfqpoint{2.095506in}{1.403650in}}{\pgfqpoint{2.106106in}{1.399260in}}{\pgfqpoint{2.117156in}{1.399260in}}%
\pgfpathlineto{\pgfqpoint{2.117156in}{1.399260in}}%
\pgfpathclose%
\pgfusepath{stroke}%
\end{pgfscope}%
\begin{pgfscope}%
\pgfpathrectangle{\pgfqpoint{0.494722in}{0.437222in}}{\pgfqpoint{6.275590in}{5.159444in}}%
\pgfusepath{clip}%
\pgfsetbuttcap%
\pgfsetroundjoin%
\pgfsetlinewidth{1.003750pt}%
\definecolor{currentstroke}{rgb}{0.827451,0.827451,0.827451}%
\pgfsetstrokecolor{currentstroke}%
\pgfsetstrokeopacity{0.800000}%
\pgfsetdash{}{0pt}%
\pgfpathmoveto{\pgfqpoint{2.600306in}{1.097647in}}%
\pgfpathcurveto{\pgfqpoint{2.611356in}{1.097647in}}{\pgfqpoint{2.621955in}{1.102037in}}{\pgfqpoint{2.629769in}{1.109851in}}%
\pgfpathcurveto{\pgfqpoint{2.637582in}{1.117665in}}{\pgfqpoint{2.641972in}{1.128264in}}{\pgfqpoint{2.641972in}{1.139314in}}%
\pgfpathcurveto{\pgfqpoint{2.641972in}{1.150364in}}{\pgfqpoint{2.637582in}{1.160963in}}{\pgfqpoint{2.629769in}{1.168777in}}%
\pgfpathcurveto{\pgfqpoint{2.621955in}{1.176590in}}{\pgfqpoint{2.611356in}{1.180981in}}{\pgfqpoint{2.600306in}{1.180981in}}%
\pgfpathcurveto{\pgfqpoint{2.589256in}{1.180981in}}{\pgfqpoint{2.578657in}{1.176590in}}{\pgfqpoint{2.570843in}{1.168777in}}%
\pgfpathcurveto{\pgfqpoint{2.563029in}{1.160963in}}{\pgfqpoint{2.558639in}{1.150364in}}{\pgfqpoint{2.558639in}{1.139314in}}%
\pgfpathcurveto{\pgfqpoint{2.558639in}{1.128264in}}{\pgfqpoint{2.563029in}{1.117665in}}{\pgfqpoint{2.570843in}{1.109851in}}%
\pgfpathcurveto{\pgfqpoint{2.578657in}{1.102037in}}{\pgfqpoint{2.589256in}{1.097647in}}{\pgfqpoint{2.600306in}{1.097647in}}%
\pgfpathlineto{\pgfqpoint{2.600306in}{1.097647in}}%
\pgfpathclose%
\pgfusepath{stroke}%
\end{pgfscope}%
\begin{pgfscope}%
\pgfpathrectangle{\pgfqpoint{0.494722in}{0.437222in}}{\pgfqpoint{6.275590in}{5.159444in}}%
\pgfusepath{clip}%
\pgfsetbuttcap%
\pgfsetroundjoin%
\pgfsetlinewidth{1.003750pt}%
\definecolor{currentstroke}{rgb}{0.827451,0.827451,0.827451}%
\pgfsetstrokecolor{currentstroke}%
\pgfsetstrokeopacity{0.800000}%
\pgfsetdash{}{0pt}%
\pgfpathmoveto{\pgfqpoint{0.625304in}{3.523383in}}%
\pgfpathcurveto{\pgfqpoint{0.636354in}{3.523383in}}{\pgfqpoint{0.646953in}{3.527774in}}{\pgfqpoint{0.654767in}{3.535587in}}%
\pgfpathcurveto{\pgfqpoint{0.662581in}{3.543401in}}{\pgfqpoint{0.666971in}{3.554000in}}{\pgfqpoint{0.666971in}{3.565050in}}%
\pgfpathcurveto{\pgfqpoint{0.666971in}{3.576100in}}{\pgfqpoint{0.662581in}{3.586699in}}{\pgfqpoint{0.654767in}{3.594513in}}%
\pgfpathcurveto{\pgfqpoint{0.646953in}{3.602326in}}{\pgfqpoint{0.636354in}{3.606717in}}{\pgfqpoint{0.625304in}{3.606717in}}%
\pgfpathcurveto{\pgfqpoint{0.614254in}{3.606717in}}{\pgfqpoint{0.603655in}{3.602326in}}{\pgfqpoint{0.595841in}{3.594513in}}%
\pgfpathcurveto{\pgfqpoint{0.588028in}{3.586699in}}{\pgfqpoint{0.583638in}{3.576100in}}{\pgfqpoint{0.583638in}{3.565050in}}%
\pgfpathcurveto{\pgfqpoint{0.583638in}{3.554000in}}{\pgfqpoint{0.588028in}{3.543401in}}{\pgfqpoint{0.595841in}{3.535587in}}%
\pgfpathcurveto{\pgfqpoint{0.603655in}{3.527774in}}{\pgfqpoint{0.614254in}{3.523383in}}{\pgfqpoint{0.625304in}{3.523383in}}%
\pgfpathlineto{\pgfqpoint{0.625304in}{3.523383in}}%
\pgfpathclose%
\pgfusepath{stroke}%
\end{pgfscope}%
\begin{pgfscope}%
\pgfpathrectangle{\pgfqpoint{0.494722in}{0.437222in}}{\pgfqpoint{6.275590in}{5.159444in}}%
\pgfusepath{clip}%
\pgfsetbuttcap%
\pgfsetroundjoin%
\pgfsetlinewidth{1.003750pt}%
\definecolor{currentstroke}{rgb}{0.827451,0.827451,0.827451}%
\pgfsetstrokecolor{currentstroke}%
\pgfsetstrokeopacity{0.800000}%
\pgfsetdash{}{0pt}%
\pgfpathmoveto{\pgfqpoint{0.711186in}{3.233934in}}%
\pgfpathcurveto{\pgfqpoint{0.722236in}{3.233934in}}{\pgfqpoint{0.732835in}{3.238324in}}{\pgfqpoint{0.740648in}{3.246138in}}%
\pgfpathcurveto{\pgfqpoint{0.748462in}{3.253952in}}{\pgfqpoint{0.752852in}{3.264551in}}{\pgfqpoint{0.752852in}{3.275601in}}%
\pgfpathcurveto{\pgfqpoint{0.752852in}{3.286651in}}{\pgfqpoint{0.748462in}{3.297250in}}{\pgfqpoint{0.740648in}{3.305064in}}%
\pgfpathcurveto{\pgfqpoint{0.732835in}{3.312877in}}{\pgfqpoint{0.722236in}{3.317267in}}{\pgfqpoint{0.711186in}{3.317267in}}%
\pgfpathcurveto{\pgfqpoint{0.700135in}{3.317267in}}{\pgfqpoint{0.689536in}{3.312877in}}{\pgfqpoint{0.681723in}{3.305064in}}%
\pgfpathcurveto{\pgfqpoint{0.673909in}{3.297250in}}{\pgfqpoint{0.669519in}{3.286651in}}{\pgfqpoint{0.669519in}{3.275601in}}%
\pgfpathcurveto{\pgfqpoint{0.669519in}{3.264551in}}{\pgfqpoint{0.673909in}{3.253952in}}{\pgfqpoint{0.681723in}{3.246138in}}%
\pgfpathcurveto{\pgfqpoint{0.689536in}{3.238324in}}{\pgfqpoint{0.700135in}{3.233934in}}{\pgfqpoint{0.711186in}{3.233934in}}%
\pgfpathlineto{\pgfqpoint{0.711186in}{3.233934in}}%
\pgfpathclose%
\pgfusepath{stroke}%
\end{pgfscope}%
\begin{pgfscope}%
\pgfpathrectangle{\pgfqpoint{0.494722in}{0.437222in}}{\pgfqpoint{6.275590in}{5.159444in}}%
\pgfusepath{clip}%
\pgfsetbuttcap%
\pgfsetroundjoin%
\pgfsetlinewidth{1.003750pt}%
\definecolor{currentstroke}{rgb}{0.827451,0.827451,0.827451}%
\pgfsetstrokecolor{currentstroke}%
\pgfsetstrokeopacity{0.800000}%
\pgfsetdash{}{0pt}%
\pgfpathmoveto{\pgfqpoint{3.164762in}{0.836209in}}%
\pgfpathcurveto{\pgfqpoint{3.175812in}{0.836209in}}{\pgfqpoint{3.186411in}{0.840599in}}{\pgfqpoint{3.194224in}{0.848413in}}%
\pgfpathcurveto{\pgfqpoint{3.202038in}{0.856226in}}{\pgfqpoint{3.206428in}{0.866825in}}{\pgfqpoint{3.206428in}{0.877875in}}%
\pgfpathcurveto{\pgfqpoint{3.206428in}{0.888926in}}{\pgfqpoint{3.202038in}{0.899525in}}{\pgfqpoint{3.194224in}{0.907338in}}%
\pgfpathcurveto{\pgfqpoint{3.186411in}{0.915152in}}{\pgfqpoint{3.175812in}{0.919542in}}{\pgfqpoint{3.164762in}{0.919542in}}%
\pgfpathcurveto{\pgfqpoint{3.153711in}{0.919542in}}{\pgfqpoint{3.143112in}{0.915152in}}{\pgfqpoint{3.135299in}{0.907338in}}%
\pgfpathcurveto{\pgfqpoint{3.127485in}{0.899525in}}{\pgfqpoint{3.123095in}{0.888926in}}{\pgfqpoint{3.123095in}{0.877875in}}%
\pgfpathcurveto{\pgfqpoint{3.123095in}{0.866825in}}{\pgfqpoint{3.127485in}{0.856226in}}{\pgfqpoint{3.135299in}{0.848413in}}%
\pgfpathcurveto{\pgfqpoint{3.143112in}{0.840599in}}{\pgfqpoint{3.153711in}{0.836209in}}{\pgfqpoint{3.164762in}{0.836209in}}%
\pgfpathlineto{\pgfqpoint{3.164762in}{0.836209in}}%
\pgfpathclose%
\pgfusepath{stroke}%
\end{pgfscope}%
\begin{pgfscope}%
\pgfpathrectangle{\pgfqpoint{0.494722in}{0.437222in}}{\pgfqpoint{6.275590in}{5.159444in}}%
\pgfusepath{clip}%
\pgfsetbuttcap%
\pgfsetroundjoin%
\pgfsetlinewidth{1.003750pt}%
\definecolor{currentstroke}{rgb}{0.827451,0.827451,0.827451}%
\pgfsetstrokecolor{currentstroke}%
\pgfsetstrokeopacity{0.800000}%
\pgfsetdash{}{0pt}%
\pgfpathmoveto{\pgfqpoint{2.552339in}{1.147505in}}%
\pgfpathcurveto{\pgfqpoint{2.563390in}{1.147505in}}{\pgfqpoint{2.573989in}{1.151895in}}{\pgfqpoint{2.581802in}{1.159709in}}%
\pgfpathcurveto{\pgfqpoint{2.589616in}{1.167522in}}{\pgfqpoint{2.594006in}{1.178121in}}{\pgfqpoint{2.594006in}{1.189172in}}%
\pgfpathcurveto{\pgfqpoint{2.594006in}{1.200222in}}{\pgfqpoint{2.589616in}{1.210821in}}{\pgfqpoint{2.581802in}{1.218634in}}%
\pgfpathcurveto{\pgfqpoint{2.573989in}{1.226448in}}{\pgfqpoint{2.563390in}{1.230838in}}{\pgfqpoint{2.552339in}{1.230838in}}%
\pgfpathcurveto{\pgfqpoint{2.541289in}{1.230838in}}{\pgfqpoint{2.530690in}{1.226448in}}{\pgfqpoint{2.522877in}{1.218634in}}%
\pgfpathcurveto{\pgfqpoint{2.515063in}{1.210821in}}{\pgfqpoint{2.510673in}{1.200222in}}{\pgfqpoint{2.510673in}{1.189172in}}%
\pgfpathcurveto{\pgfqpoint{2.510673in}{1.178121in}}{\pgfqpoint{2.515063in}{1.167522in}}{\pgfqpoint{2.522877in}{1.159709in}}%
\pgfpathcurveto{\pgfqpoint{2.530690in}{1.151895in}}{\pgfqpoint{2.541289in}{1.147505in}}{\pgfqpoint{2.552339in}{1.147505in}}%
\pgfpathlineto{\pgfqpoint{2.552339in}{1.147505in}}%
\pgfpathclose%
\pgfusepath{stroke}%
\end{pgfscope}%
\begin{pgfscope}%
\pgfpathrectangle{\pgfqpoint{0.494722in}{0.437222in}}{\pgfqpoint{6.275590in}{5.159444in}}%
\pgfusepath{clip}%
\pgfsetbuttcap%
\pgfsetroundjoin%
\pgfsetlinewidth{1.003750pt}%
\definecolor{currentstroke}{rgb}{0.827451,0.827451,0.827451}%
\pgfsetstrokecolor{currentstroke}%
\pgfsetstrokeopacity{0.800000}%
\pgfsetdash{}{0pt}%
\pgfpathmoveto{\pgfqpoint{2.365416in}{1.256449in}}%
\pgfpathcurveto{\pgfqpoint{2.376466in}{1.256449in}}{\pgfqpoint{2.387065in}{1.260839in}}{\pgfqpoint{2.394879in}{1.268653in}}%
\pgfpathcurveto{\pgfqpoint{2.402693in}{1.276466in}}{\pgfqpoint{2.407083in}{1.287065in}}{\pgfqpoint{2.407083in}{1.298115in}}%
\pgfpathcurveto{\pgfqpoint{2.407083in}{1.309166in}}{\pgfqpoint{2.402693in}{1.319765in}}{\pgfqpoint{2.394879in}{1.327578in}}%
\pgfpathcurveto{\pgfqpoint{2.387065in}{1.335392in}}{\pgfqpoint{2.376466in}{1.339782in}}{\pgfqpoint{2.365416in}{1.339782in}}%
\pgfpathcurveto{\pgfqpoint{2.354366in}{1.339782in}}{\pgfqpoint{2.343767in}{1.335392in}}{\pgfqpoint{2.335953in}{1.327578in}}%
\pgfpathcurveto{\pgfqpoint{2.328140in}{1.319765in}}{\pgfqpoint{2.323750in}{1.309166in}}{\pgfqpoint{2.323750in}{1.298115in}}%
\pgfpathcurveto{\pgfqpoint{2.323750in}{1.287065in}}{\pgfqpoint{2.328140in}{1.276466in}}{\pgfqpoint{2.335953in}{1.268653in}}%
\pgfpathcurveto{\pgfqpoint{2.343767in}{1.260839in}}{\pgfqpoint{2.354366in}{1.256449in}}{\pgfqpoint{2.365416in}{1.256449in}}%
\pgfpathlineto{\pgfqpoint{2.365416in}{1.256449in}}%
\pgfpathclose%
\pgfusepath{stroke}%
\end{pgfscope}%
\begin{pgfscope}%
\pgfpathrectangle{\pgfqpoint{0.494722in}{0.437222in}}{\pgfqpoint{6.275590in}{5.159444in}}%
\pgfusepath{clip}%
\pgfsetbuttcap%
\pgfsetroundjoin%
\pgfsetlinewidth{1.003750pt}%
\definecolor{currentstroke}{rgb}{0.827451,0.827451,0.827451}%
\pgfsetstrokecolor{currentstroke}%
\pgfsetstrokeopacity{0.800000}%
\pgfsetdash{}{0pt}%
\pgfpathmoveto{\pgfqpoint{0.608606in}{3.646796in}}%
\pgfpathcurveto{\pgfqpoint{0.619657in}{3.646796in}}{\pgfqpoint{0.630256in}{3.651186in}}{\pgfqpoint{0.638069in}{3.659000in}}%
\pgfpathcurveto{\pgfqpoint{0.645883in}{3.666813in}}{\pgfqpoint{0.650273in}{3.677412in}}{\pgfqpoint{0.650273in}{3.688462in}}%
\pgfpathcurveto{\pgfqpoint{0.650273in}{3.699512in}}{\pgfqpoint{0.645883in}{3.710112in}}{\pgfqpoint{0.638069in}{3.717925in}}%
\pgfpathcurveto{\pgfqpoint{0.630256in}{3.725739in}}{\pgfqpoint{0.619657in}{3.730129in}}{\pgfqpoint{0.608606in}{3.730129in}}%
\pgfpathcurveto{\pgfqpoint{0.597556in}{3.730129in}}{\pgfqpoint{0.586957in}{3.725739in}}{\pgfqpoint{0.579144in}{3.717925in}}%
\pgfpathcurveto{\pgfqpoint{0.571330in}{3.710112in}}{\pgfqpoint{0.566940in}{3.699512in}}{\pgfqpoint{0.566940in}{3.688462in}}%
\pgfpathcurveto{\pgfqpoint{0.566940in}{3.677412in}}{\pgfqpoint{0.571330in}{3.666813in}}{\pgfqpoint{0.579144in}{3.659000in}}%
\pgfpathcurveto{\pgfqpoint{0.586957in}{3.651186in}}{\pgfqpoint{0.597556in}{3.646796in}}{\pgfqpoint{0.608606in}{3.646796in}}%
\pgfpathlineto{\pgfqpoint{0.608606in}{3.646796in}}%
\pgfpathclose%
\pgfusepath{stroke}%
\end{pgfscope}%
\begin{pgfscope}%
\pgfpathrectangle{\pgfqpoint{0.494722in}{0.437222in}}{\pgfqpoint{6.275590in}{5.159444in}}%
\pgfusepath{clip}%
\pgfsetbuttcap%
\pgfsetroundjoin%
\pgfsetlinewidth{1.003750pt}%
\definecolor{currentstroke}{rgb}{0.827451,0.827451,0.827451}%
\pgfsetstrokecolor{currentstroke}%
\pgfsetstrokeopacity{0.800000}%
\pgfsetdash{}{0pt}%
\pgfpathmoveto{\pgfqpoint{1.381781in}{2.056420in}}%
\pgfpathcurveto{\pgfqpoint{1.392831in}{2.056420in}}{\pgfqpoint{1.403430in}{2.060810in}}{\pgfqpoint{1.411244in}{2.068624in}}%
\pgfpathcurveto{\pgfqpoint{1.419057in}{2.076437in}}{\pgfqpoint{1.423448in}{2.087036in}}{\pgfqpoint{1.423448in}{2.098087in}}%
\pgfpathcurveto{\pgfqpoint{1.423448in}{2.109137in}}{\pgfqpoint{1.419057in}{2.119736in}}{\pgfqpoint{1.411244in}{2.127549in}}%
\pgfpathcurveto{\pgfqpoint{1.403430in}{2.135363in}}{\pgfqpoint{1.392831in}{2.139753in}}{\pgfqpoint{1.381781in}{2.139753in}}%
\pgfpathcurveto{\pgfqpoint{1.370731in}{2.139753in}}{\pgfqpoint{1.360132in}{2.135363in}}{\pgfqpoint{1.352318in}{2.127549in}}%
\pgfpathcurveto{\pgfqpoint{1.344505in}{2.119736in}}{\pgfqpoint{1.340114in}{2.109137in}}{\pgfqpoint{1.340114in}{2.098087in}}%
\pgfpathcurveto{\pgfqpoint{1.340114in}{2.087036in}}{\pgfqpoint{1.344505in}{2.076437in}}{\pgfqpoint{1.352318in}{2.068624in}}%
\pgfpathcurveto{\pgfqpoint{1.360132in}{2.060810in}}{\pgfqpoint{1.370731in}{2.056420in}}{\pgfqpoint{1.381781in}{2.056420in}}%
\pgfpathlineto{\pgfqpoint{1.381781in}{2.056420in}}%
\pgfpathclose%
\pgfusepath{stroke}%
\end{pgfscope}%
\begin{pgfscope}%
\pgfpathrectangle{\pgfqpoint{0.494722in}{0.437222in}}{\pgfqpoint{6.275590in}{5.159444in}}%
\pgfusepath{clip}%
\pgfsetbuttcap%
\pgfsetroundjoin%
\pgfsetlinewidth{1.003750pt}%
\definecolor{currentstroke}{rgb}{0.827451,0.827451,0.827451}%
\pgfsetstrokecolor{currentstroke}%
\pgfsetstrokeopacity{0.800000}%
\pgfsetdash{}{0pt}%
\pgfpathmoveto{\pgfqpoint{1.964907in}{1.517037in}}%
\pgfpathcurveto{\pgfqpoint{1.975957in}{1.517037in}}{\pgfqpoint{1.986556in}{1.521427in}}{\pgfqpoint{1.994370in}{1.529240in}}%
\pgfpathcurveto{\pgfqpoint{2.002183in}{1.537054in}}{\pgfqpoint{2.006573in}{1.547653in}}{\pgfqpoint{2.006573in}{1.558703in}}%
\pgfpathcurveto{\pgfqpoint{2.006573in}{1.569753in}}{\pgfqpoint{2.002183in}{1.580352in}}{\pgfqpoint{1.994370in}{1.588166in}}%
\pgfpathcurveto{\pgfqpoint{1.986556in}{1.595980in}}{\pgfqpoint{1.975957in}{1.600370in}}{\pgfqpoint{1.964907in}{1.600370in}}%
\pgfpathcurveto{\pgfqpoint{1.953857in}{1.600370in}}{\pgfqpoint{1.943258in}{1.595980in}}{\pgfqpoint{1.935444in}{1.588166in}}%
\pgfpathcurveto{\pgfqpoint{1.927630in}{1.580352in}}{\pgfqpoint{1.923240in}{1.569753in}}{\pgfqpoint{1.923240in}{1.558703in}}%
\pgfpathcurveto{\pgfqpoint{1.923240in}{1.547653in}}{\pgfqpoint{1.927630in}{1.537054in}}{\pgfqpoint{1.935444in}{1.529240in}}%
\pgfpathcurveto{\pgfqpoint{1.943258in}{1.521427in}}{\pgfqpoint{1.953857in}{1.517037in}}{\pgfqpoint{1.964907in}{1.517037in}}%
\pgfpathlineto{\pgfqpoint{1.964907in}{1.517037in}}%
\pgfpathclose%
\pgfusepath{stroke}%
\end{pgfscope}%
\begin{pgfscope}%
\pgfpathrectangle{\pgfqpoint{0.494722in}{0.437222in}}{\pgfqpoint{6.275590in}{5.159444in}}%
\pgfusepath{clip}%
\pgfsetbuttcap%
\pgfsetroundjoin%
\pgfsetlinewidth{1.003750pt}%
\definecolor{currentstroke}{rgb}{0.827451,0.827451,0.827451}%
\pgfsetstrokecolor{currentstroke}%
\pgfsetstrokeopacity{0.800000}%
\pgfsetdash{}{0pt}%
\pgfpathmoveto{\pgfqpoint{1.941684in}{1.555345in}}%
\pgfpathcurveto{\pgfqpoint{1.952734in}{1.555345in}}{\pgfqpoint{1.963333in}{1.559735in}}{\pgfqpoint{1.971146in}{1.567549in}}%
\pgfpathcurveto{\pgfqpoint{1.978960in}{1.575362in}}{\pgfqpoint{1.983350in}{1.585961in}}{\pgfqpoint{1.983350in}{1.597011in}}%
\pgfpathcurveto{\pgfqpoint{1.983350in}{1.608062in}}{\pgfqpoint{1.978960in}{1.618661in}}{\pgfqpoint{1.971146in}{1.626474in}}%
\pgfpathcurveto{\pgfqpoint{1.963333in}{1.634288in}}{\pgfqpoint{1.952734in}{1.638678in}}{\pgfqpoint{1.941684in}{1.638678in}}%
\pgfpathcurveto{\pgfqpoint{1.930633in}{1.638678in}}{\pgfqpoint{1.920034in}{1.634288in}}{\pgfqpoint{1.912221in}{1.626474in}}%
\pgfpathcurveto{\pgfqpoint{1.904407in}{1.618661in}}{\pgfqpoint{1.900017in}{1.608062in}}{\pgfqpoint{1.900017in}{1.597011in}}%
\pgfpathcurveto{\pgfqpoint{1.900017in}{1.585961in}}{\pgfqpoint{1.904407in}{1.575362in}}{\pgfqpoint{1.912221in}{1.567549in}}%
\pgfpathcurveto{\pgfqpoint{1.920034in}{1.559735in}}{\pgfqpoint{1.930633in}{1.555345in}}{\pgfqpoint{1.941684in}{1.555345in}}%
\pgfpathlineto{\pgfqpoint{1.941684in}{1.555345in}}%
\pgfpathclose%
\pgfusepath{stroke}%
\end{pgfscope}%
\begin{pgfscope}%
\pgfpathrectangle{\pgfqpoint{0.494722in}{0.437222in}}{\pgfqpoint{6.275590in}{5.159444in}}%
\pgfusepath{clip}%
\pgfsetbuttcap%
\pgfsetroundjoin%
\pgfsetlinewidth{1.003750pt}%
\definecolor{currentstroke}{rgb}{0.827451,0.827451,0.827451}%
\pgfsetstrokecolor{currentstroke}%
\pgfsetstrokeopacity{0.800000}%
\pgfsetdash{}{0pt}%
\pgfpathmoveto{\pgfqpoint{2.203827in}{1.337053in}}%
\pgfpathcurveto{\pgfqpoint{2.214877in}{1.337053in}}{\pgfqpoint{2.225476in}{1.341443in}}{\pgfqpoint{2.233290in}{1.349257in}}%
\pgfpathcurveto{\pgfqpoint{2.241103in}{1.357070in}}{\pgfqpoint{2.245494in}{1.367669in}}{\pgfqpoint{2.245494in}{1.378720in}}%
\pgfpathcurveto{\pgfqpoint{2.245494in}{1.389770in}}{\pgfqpoint{2.241103in}{1.400369in}}{\pgfqpoint{2.233290in}{1.408182in}}%
\pgfpathcurveto{\pgfqpoint{2.225476in}{1.415996in}}{\pgfqpoint{2.214877in}{1.420386in}}{\pgfqpoint{2.203827in}{1.420386in}}%
\pgfpathcurveto{\pgfqpoint{2.192777in}{1.420386in}}{\pgfqpoint{2.182178in}{1.415996in}}{\pgfqpoint{2.174364in}{1.408182in}}%
\pgfpathcurveto{\pgfqpoint{2.166550in}{1.400369in}}{\pgfqpoint{2.162160in}{1.389770in}}{\pgfqpoint{2.162160in}{1.378720in}}%
\pgfpathcurveto{\pgfqpoint{2.162160in}{1.367669in}}{\pgfqpoint{2.166550in}{1.357070in}}{\pgfqpoint{2.174364in}{1.349257in}}%
\pgfpathcurveto{\pgfqpoint{2.182178in}{1.341443in}}{\pgfqpoint{2.192777in}{1.337053in}}{\pgfqpoint{2.203827in}{1.337053in}}%
\pgfpathlineto{\pgfqpoint{2.203827in}{1.337053in}}%
\pgfpathclose%
\pgfusepath{stroke}%
\end{pgfscope}%
\begin{pgfscope}%
\pgfpathrectangle{\pgfqpoint{0.494722in}{0.437222in}}{\pgfqpoint{6.275590in}{5.159444in}}%
\pgfusepath{clip}%
\pgfsetbuttcap%
\pgfsetroundjoin%
\pgfsetlinewidth{1.003750pt}%
\definecolor{currentstroke}{rgb}{0.827451,0.827451,0.827451}%
\pgfsetstrokecolor{currentstroke}%
\pgfsetstrokeopacity{0.800000}%
\pgfsetdash{}{0pt}%
\pgfpathmoveto{\pgfqpoint{3.250298in}{0.800705in}}%
\pgfpathcurveto{\pgfqpoint{3.261348in}{0.800705in}}{\pgfqpoint{3.271947in}{0.805096in}}{\pgfqpoint{3.279761in}{0.812909in}}%
\pgfpathcurveto{\pgfqpoint{3.287574in}{0.820723in}}{\pgfqpoint{3.291965in}{0.831322in}}{\pgfqpoint{3.291965in}{0.842372in}}%
\pgfpathcurveto{\pgfqpoint{3.291965in}{0.853422in}}{\pgfqpoint{3.287574in}{0.864021in}}{\pgfqpoint{3.279761in}{0.871835in}}%
\pgfpathcurveto{\pgfqpoint{3.271947in}{0.879649in}}{\pgfqpoint{3.261348in}{0.884039in}}{\pgfqpoint{3.250298in}{0.884039in}}%
\pgfpathcurveto{\pgfqpoint{3.239248in}{0.884039in}}{\pgfqpoint{3.228649in}{0.879649in}}{\pgfqpoint{3.220835in}{0.871835in}}%
\pgfpathcurveto{\pgfqpoint{3.213022in}{0.864021in}}{\pgfqpoint{3.208631in}{0.853422in}}{\pgfqpoint{3.208631in}{0.842372in}}%
\pgfpathcurveto{\pgfqpoint{3.208631in}{0.831322in}}{\pgfqpoint{3.213022in}{0.820723in}}{\pgfqpoint{3.220835in}{0.812909in}}%
\pgfpathcurveto{\pgfqpoint{3.228649in}{0.805096in}}{\pgfqpoint{3.239248in}{0.800705in}}{\pgfqpoint{3.250298in}{0.800705in}}%
\pgfpathlineto{\pgfqpoint{3.250298in}{0.800705in}}%
\pgfpathclose%
\pgfusepath{stroke}%
\end{pgfscope}%
\begin{pgfscope}%
\pgfpathrectangle{\pgfqpoint{0.494722in}{0.437222in}}{\pgfqpoint{6.275590in}{5.159444in}}%
\pgfusepath{clip}%
\pgfsetbuttcap%
\pgfsetroundjoin%
\pgfsetlinewidth{1.003750pt}%
\definecolor{currentstroke}{rgb}{0.827451,0.827451,0.827451}%
\pgfsetstrokecolor{currentstroke}%
\pgfsetstrokeopacity{0.800000}%
\pgfsetdash{}{0pt}%
\pgfpathmoveto{\pgfqpoint{1.993886in}{1.499770in}}%
\pgfpathcurveto{\pgfqpoint{2.004936in}{1.499770in}}{\pgfqpoint{2.015535in}{1.504160in}}{\pgfqpoint{2.023349in}{1.511974in}}%
\pgfpathcurveto{\pgfqpoint{2.031162in}{1.519788in}}{\pgfqpoint{2.035553in}{1.530387in}}{\pgfqpoint{2.035553in}{1.541437in}}%
\pgfpathcurveto{\pgfqpoint{2.035553in}{1.552487in}}{\pgfqpoint{2.031162in}{1.563086in}}{\pgfqpoint{2.023349in}{1.570900in}}%
\pgfpathcurveto{\pgfqpoint{2.015535in}{1.578713in}}{\pgfqpoint{2.004936in}{1.583103in}}{\pgfqpoint{1.993886in}{1.583103in}}%
\pgfpathcurveto{\pgfqpoint{1.982836in}{1.583103in}}{\pgfqpoint{1.972237in}{1.578713in}}{\pgfqpoint{1.964423in}{1.570900in}}%
\pgfpathcurveto{\pgfqpoint{1.956610in}{1.563086in}}{\pgfqpoint{1.952219in}{1.552487in}}{\pgfqpoint{1.952219in}{1.541437in}}%
\pgfpathcurveto{\pgfqpoint{1.952219in}{1.530387in}}{\pgfqpoint{1.956610in}{1.519788in}}{\pgfqpoint{1.964423in}{1.511974in}}%
\pgfpathcurveto{\pgfqpoint{1.972237in}{1.504160in}}{\pgfqpoint{1.982836in}{1.499770in}}{\pgfqpoint{1.993886in}{1.499770in}}%
\pgfpathlineto{\pgfqpoint{1.993886in}{1.499770in}}%
\pgfpathclose%
\pgfusepath{stroke}%
\end{pgfscope}%
\begin{pgfscope}%
\pgfpathrectangle{\pgfqpoint{0.494722in}{0.437222in}}{\pgfqpoint{6.275590in}{5.159444in}}%
\pgfusepath{clip}%
\pgfsetbuttcap%
\pgfsetroundjoin%
\pgfsetlinewidth{1.003750pt}%
\definecolor{currentstroke}{rgb}{0.827451,0.827451,0.827451}%
\pgfsetstrokecolor{currentstroke}%
\pgfsetstrokeopacity{0.800000}%
\pgfsetdash{}{0pt}%
\pgfpathmoveto{\pgfqpoint{2.183447in}{1.371651in}}%
\pgfpathcurveto{\pgfqpoint{2.194497in}{1.371651in}}{\pgfqpoint{2.205096in}{1.376041in}}{\pgfqpoint{2.212910in}{1.383855in}}%
\pgfpathcurveto{\pgfqpoint{2.220724in}{1.391668in}}{\pgfqpoint{2.225114in}{1.402267in}}{\pgfqpoint{2.225114in}{1.413318in}}%
\pgfpathcurveto{\pgfqpoint{2.225114in}{1.424368in}}{\pgfqpoint{2.220724in}{1.434967in}}{\pgfqpoint{2.212910in}{1.442780in}}%
\pgfpathcurveto{\pgfqpoint{2.205096in}{1.450594in}}{\pgfqpoint{2.194497in}{1.454984in}}{\pgfqpoint{2.183447in}{1.454984in}}%
\pgfpathcurveto{\pgfqpoint{2.172397in}{1.454984in}}{\pgfqpoint{2.161798in}{1.450594in}}{\pgfqpoint{2.153984in}{1.442780in}}%
\pgfpathcurveto{\pgfqpoint{2.146171in}{1.434967in}}{\pgfqpoint{2.141781in}{1.424368in}}{\pgfqpoint{2.141781in}{1.413318in}}%
\pgfpathcurveto{\pgfqpoint{2.141781in}{1.402267in}}{\pgfqpoint{2.146171in}{1.391668in}}{\pgfqpoint{2.153984in}{1.383855in}}%
\pgfpathcurveto{\pgfqpoint{2.161798in}{1.376041in}}{\pgfqpoint{2.172397in}{1.371651in}}{\pgfqpoint{2.183447in}{1.371651in}}%
\pgfpathlineto{\pgfqpoint{2.183447in}{1.371651in}}%
\pgfpathclose%
\pgfusepath{stroke}%
\end{pgfscope}%
\begin{pgfscope}%
\pgfpathrectangle{\pgfqpoint{0.494722in}{0.437222in}}{\pgfqpoint{6.275590in}{5.159444in}}%
\pgfusepath{clip}%
\pgfsetbuttcap%
\pgfsetroundjoin%
\pgfsetlinewidth{1.003750pt}%
\definecolor{currentstroke}{rgb}{0.827451,0.827451,0.827451}%
\pgfsetstrokecolor{currentstroke}%
\pgfsetstrokeopacity{0.800000}%
\pgfsetdash{}{0pt}%
\pgfpathmoveto{\pgfqpoint{1.347485in}{2.096739in}}%
\pgfpathcurveto{\pgfqpoint{1.358535in}{2.096739in}}{\pgfqpoint{1.369135in}{2.101129in}}{\pgfqpoint{1.376948in}{2.108943in}}%
\pgfpathcurveto{\pgfqpoint{1.384762in}{2.116757in}}{\pgfqpoint{1.389152in}{2.127356in}}{\pgfqpoint{1.389152in}{2.138406in}}%
\pgfpathcurveto{\pgfqpoint{1.389152in}{2.149456in}}{\pgfqpoint{1.384762in}{2.160055in}}{\pgfqpoint{1.376948in}{2.167869in}}%
\pgfpathcurveto{\pgfqpoint{1.369135in}{2.175682in}}{\pgfqpoint{1.358535in}{2.180072in}}{\pgfqpoint{1.347485in}{2.180072in}}%
\pgfpathcurveto{\pgfqpoint{1.336435in}{2.180072in}}{\pgfqpoint{1.325836in}{2.175682in}}{\pgfqpoint{1.318023in}{2.167869in}}%
\pgfpathcurveto{\pgfqpoint{1.310209in}{2.160055in}}{\pgfqpoint{1.305819in}{2.149456in}}{\pgfqpoint{1.305819in}{2.138406in}}%
\pgfpathcurveto{\pgfqpoint{1.305819in}{2.127356in}}{\pgfqpoint{1.310209in}{2.116757in}}{\pgfqpoint{1.318023in}{2.108943in}}%
\pgfpathcurveto{\pgfqpoint{1.325836in}{2.101129in}}{\pgfqpoint{1.336435in}{2.096739in}}{\pgfqpoint{1.347485in}{2.096739in}}%
\pgfpathlineto{\pgfqpoint{1.347485in}{2.096739in}}%
\pgfpathclose%
\pgfusepath{stroke}%
\end{pgfscope}%
\begin{pgfscope}%
\pgfpathrectangle{\pgfqpoint{0.494722in}{0.437222in}}{\pgfqpoint{6.275590in}{5.159444in}}%
\pgfusepath{clip}%
\pgfsetbuttcap%
\pgfsetroundjoin%
\pgfsetlinewidth{1.003750pt}%
\definecolor{currentstroke}{rgb}{0.827451,0.827451,0.827451}%
\pgfsetstrokecolor{currentstroke}%
\pgfsetstrokeopacity{0.800000}%
\pgfsetdash{}{0pt}%
\pgfpathmoveto{\pgfqpoint{3.805298in}{0.639733in}}%
\pgfpathcurveto{\pgfqpoint{3.816348in}{0.639733in}}{\pgfqpoint{3.826947in}{0.644124in}}{\pgfqpoint{3.834761in}{0.651937in}}%
\pgfpathcurveto{\pgfqpoint{3.842574in}{0.659751in}}{\pgfqpoint{3.846965in}{0.670350in}}{\pgfqpoint{3.846965in}{0.681400in}}%
\pgfpathcurveto{\pgfqpoint{3.846965in}{0.692450in}}{\pgfqpoint{3.842574in}{0.703049in}}{\pgfqpoint{3.834761in}{0.710863in}}%
\pgfpathcurveto{\pgfqpoint{3.826947in}{0.718676in}}{\pgfqpoint{3.816348in}{0.723067in}}{\pgfqpoint{3.805298in}{0.723067in}}%
\pgfpathcurveto{\pgfqpoint{3.794248in}{0.723067in}}{\pgfqpoint{3.783649in}{0.718676in}}{\pgfqpoint{3.775835in}{0.710863in}}%
\pgfpathcurveto{\pgfqpoint{3.768022in}{0.703049in}}{\pgfqpoint{3.763631in}{0.692450in}}{\pgfqpoint{3.763631in}{0.681400in}}%
\pgfpathcurveto{\pgfqpoint{3.763631in}{0.670350in}}{\pgfqpoint{3.768022in}{0.659751in}}{\pgfqpoint{3.775835in}{0.651937in}}%
\pgfpathcurveto{\pgfqpoint{3.783649in}{0.644124in}}{\pgfqpoint{3.794248in}{0.639733in}}{\pgfqpoint{3.805298in}{0.639733in}}%
\pgfpathlineto{\pgfqpoint{3.805298in}{0.639733in}}%
\pgfpathclose%
\pgfusepath{stroke}%
\end{pgfscope}%
\begin{pgfscope}%
\pgfpathrectangle{\pgfqpoint{0.494722in}{0.437222in}}{\pgfqpoint{6.275590in}{5.159444in}}%
\pgfusepath{clip}%
\pgfsetbuttcap%
\pgfsetroundjoin%
\pgfsetlinewidth{1.003750pt}%
\definecolor{currentstroke}{rgb}{0.827451,0.827451,0.827451}%
\pgfsetstrokecolor{currentstroke}%
\pgfsetstrokeopacity{0.800000}%
\pgfsetdash{}{0pt}%
\pgfpathmoveto{\pgfqpoint{1.286854in}{2.171124in}}%
\pgfpathcurveto{\pgfqpoint{1.297904in}{2.171124in}}{\pgfqpoint{1.308503in}{2.175514in}}{\pgfqpoint{1.316317in}{2.183328in}}%
\pgfpathcurveto{\pgfqpoint{1.324130in}{2.191141in}}{\pgfqpoint{1.328520in}{2.201740in}}{\pgfqpoint{1.328520in}{2.212790in}}%
\pgfpathcurveto{\pgfqpoint{1.328520in}{2.223841in}}{\pgfqpoint{1.324130in}{2.234440in}}{\pgfqpoint{1.316317in}{2.242253in}}%
\pgfpathcurveto{\pgfqpoint{1.308503in}{2.250067in}}{\pgfqpoint{1.297904in}{2.254457in}}{\pgfqpoint{1.286854in}{2.254457in}}%
\pgfpathcurveto{\pgfqpoint{1.275804in}{2.254457in}}{\pgfqpoint{1.265205in}{2.250067in}}{\pgfqpoint{1.257391in}{2.242253in}}%
\pgfpathcurveto{\pgfqpoint{1.249577in}{2.234440in}}{\pgfqpoint{1.245187in}{2.223841in}}{\pgfqpoint{1.245187in}{2.212790in}}%
\pgfpathcurveto{\pgfqpoint{1.245187in}{2.201740in}}{\pgfqpoint{1.249577in}{2.191141in}}{\pgfqpoint{1.257391in}{2.183328in}}%
\pgfpathcurveto{\pgfqpoint{1.265205in}{2.175514in}}{\pgfqpoint{1.275804in}{2.171124in}}{\pgfqpoint{1.286854in}{2.171124in}}%
\pgfpathlineto{\pgfqpoint{1.286854in}{2.171124in}}%
\pgfpathclose%
\pgfusepath{stroke}%
\end{pgfscope}%
\begin{pgfscope}%
\pgfpathrectangle{\pgfqpoint{0.494722in}{0.437222in}}{\pgfqpoint{6.275590in}{5.159444in}}%
\pgfusepath{clip}%
\pgfsetbuttcap%
\pgfsetroundjoin%
\pgfsetlinewidth{1.003750pt}%
\definecolor{currentstroke}{rgb}{0.827451,0.827451,0.827451}%
\pgfsetstrokecolor{currentstroke}%
\pgfsetstrokeopacity{0.800000}%
\pgfsetdash{}{0pt}%
\pgfpathmoveto{\pgfqpoint{3.754586in}{0.666564in}}%
\pgfpathcurveto{\pgfqpoint{3.765636in}{0.666564in}}{\pgfqpoint{3.776235in}{0.670954in}}{\pgfqpoint{3.784048in}{0.678767in}}%
\pgfpathcurveto{\pgfqpoint{3.791862in}{0.686581in}}{\pgfqpoint{3.796252in}{0.697180in}}{\pgfqpoint{3.796252in}{0.708230in}}%
\pgfpathcurveto{\pgfqpoint{3.796252in}{0.719280in}}{\pgfqpoint{3.791862in}{0.729879in}}{\pgfqpoint{3.784048in}{0.737693in}}%
\pgfpathcurveto{\pgfqpoint{3.776235in}{0.745507in}}{\pgfqpoint{3.765636in}{0.749897in}}{\pgfqpoint{3.754586in}{0.749897in}}%
\pgfpathcurveto{\pgfqpoint{3.743536in}{0.749897in}}{\pgfqpoint{3.732937in}{0.745507in}}{\pgfqpoint{3.725123in}{0.737693in}}%
\pgfpathcurveto{\pgfqpoint{3.717309in}{0.729879in}}{\pgfqpoint{3.712919in}{0.719280in}}{\pgfqpoint{3.712919in}{0.708230in}}%
\pgfpathcurveto{\pgfqpoint{3.712919in}{0.697180in}}{\pgfqpoint{3.717309in}{0.686581in}}{\pgfqpoint{3.725123in}{0.678767in}}%
\pgfpathcurveto{\pgfqpoint{3.732937in}{0.670954in}}{\pgfqpoint{3.743536in}{0.666564in}}{\pgfqpoint{3.754586in}{0.666564in}}%
\pgfpathlineto{\pgfqpoint{3.754586in}{0.666564in}}%
\pgfpathclose%
\pgfusepath{stroke}%
\end{pgfscope}%
\begin{pgfscope}%
\pgfpathrectangle{\pgfqpoint{0.494722in}{0.437222in}}{\pgfqpoint{6.275590in}{5.159444in}}%
\pgfusepath{clip}%
\pgfsetbuttcap%
\pgfsetroundjoin%
\pgfsetlinewidth{1.003750pt}%
\definecolor{currentstroke}{rgb}{0.827451,0.827451,0.827451}%
\pgfsetstrokecolor{currentstroke}%
\pgfsetstrokeopacity{0.800000}%
\pgfsetdash{}{0pt}%
\pgfpathmoveto{\pgfqpoint{3.275853in}{0.797691in}}%
\pgfpathcurveto{\pgfqpoint{3.286903in}{0.797691in}}{\pgfqpoint{3.297502in}{0.802081in}}{\pgfqpoint{3.305316in}{0.809895in}}%
\pgfpathcurveto{\pgfqpoint{3.313130in}{0.817709in}}{\pgfqpoint{3.317520in}{0.828308in}}{\pgfqpoint{3.317520in}{0.839358in}}%
\pgfpathcurveto{\pgfqpoint{3.317520in}{0.850408in}}{\pgfqpoint{3.313130in}{0.861007in}}{\pgfqpoint{3.305316in}{0.868821in}}%
\pgfpathcurveto{\pgfqpoint{3.297502in}{0.876634in}}{\pgfqpoint{3.286903in}{0.881025in}}{\pgfqpoint{3.275853in}{0.881025in}}%
\pgfpathcurveto{\pgfqpoint{3.264803in}{0.881025in}}{\pgfqpoint{3.254204in}{0.876634in}}{\pgfqpoint{3.246390in}{0.868821in}}%
\pgfpathcurveto{\pgfqpoint{3.238577in}{0.861007in}}{\pgfqpoint{3.234186in}{0.850408in}}{\pgfqpoint{3.234186in}{0.839358in}}%
\pgfpathcurveto{\pgfqpoint{3.234186in}{0.828308in}}{\pgfqpoint{3.238577in}{0.817709in}}{\pgfqpoint{3.246390in}{0.809895in}}%
\pgfpathcurveto{\pgfqpoint{3.254204in}{0.802081in}}{\pgfqpoint{3.264803in}{0.797691in}}{\pgfqpoint{3.275853in}{0.797691in}}%
\pgfpathlineto{\pgfqpoint{3.275853in}{0.797691in}}%
\pgfpathclose%
\pgfusepath{stroke}%
\end{pgfscope}%
\begin{pgfscope}%
\pgfpathrectangle{\pgfqpoint{0.494722in}{0.437222in}}{\pgfqpoint{6.275590in}{5.159444in}}%
\pgfusepath{clip}%
\pgfsetbuttcap%
\pgfsetroundjoin%
\pgfsetlinewidth{1.003750pt}%
\definecolor{currentstroke}{rgb}{0.827451,0.827451,0.827451}%
\pgfsetstrokecolor{currentstroke}%
\pgfsetstrokeopacity{0.800000}%
\pgfsetdash{}{0pt}%
\pgfpathmoveto{\pgfqpoint{2.572850in}{1.116571in}}%
\pgfpathcurveto{\pgfqpoint{2.583900in}{1.116571in}}{\pgfqpoint{2.594499in}{1.120961in}}{\pgfqpoint{2.602313in}{1.128774in}}%
\pgfpathcurveto{\pgfqpoint{2.610126in}{1.136588in}}{\pgfqpoint{2.614517in}{1.147187in}}{\pgfqpoint{2.614517in}{1.158237in}}%
\pgfpathcurveto{\pgfqpoint{2.614517in}{1.169287in}}{\pgfqpoint{2.610126in}{1.179886in}}{\pgfqpoint{2.602313in}{1.187700in}}%
\pgfpathcurveto{\pgfqpoint{2.594499in}{1.195514in}}{\pgfqpoint{2.583900in}{1.199904in}}{\pgfqpoint{2.572850in}{1.199904in}}%
\pgfpathcurveto{\pgfqpoint{2.561800in}{1.199904in}}{\pgfqpoint{2.551201in}{1.195514in}}{\pgfqpoint{2.543387in}{1.187700in}}%
\pgfpathcurveto{\pgfqpoint{2.535574in}{1.179886in}}{\pgfqpoint{2.531183in}{1.169287in}}{\pgfqpoint{2.531183in}{1.158237in}}%
\pgfpathcurveto{\pgfqpoint{2.531183in}{1.147187in}}{\pgfqpoint{2.535574in}{1.136588in}}{\pgfqpoint{2.543387in}{1.128774in}}%
\pgfpathcurveto{\pgfqpoint{2.551201in}{1.120961in}}{\pgfqpoint{2.561800in}{1.116571in}}{\pgfqpoint{2.572850in}{1.116571in}}%
\pgfpathlineto{\pgfqpoint{2.572850in}{1.116571in}}%
\pgfpathclose%
\pgfusepath{stroke}%
\end{pgfscope}%
\begin{pgfscope}%
\pgfpathrectangle{\pgfqpoint{0.494722in}{0.437222in}}{\pgfqpoint{6.275590in}{5.159444in}}%
\pgfusepath{clip}%
\pgfsetbuttcap%
\pgfsetroundjoin%
\pgfsetlinewidth{1.003750pt}%
\definecolor{currentstroke}{rgb}{0.827451,0.827451,0.827451}%
\pgfsetstrokecolor{currentstroke}%
\pgfsetstrokeopacity{0.800000}%
\pgfsetdash{}{0pt}%
\pgfpathmoveto{\pgfqpoint{3.503126in}{0.705979in}}%
\pgfpathcurveto{\pgfqpoint{3.514177in}{0.705979in}}{\pgfqpoint{3.524776in}{0.710369in}}{\pgfqpoint{3.532589in}{0.718183in}}%
\pgfpathcurveto{\pgfqpoint{3.540403in}{0.725996in}}{\pgfqpoint{3.544793in}{0.736595in}}{\pgfqpoint{3.544793in}{0.747645in}}%
\pgfpathcurveto{\pgfqpoint{3.544793in}{0.758696in}}{\pgfqpoint{3.540403in}{0.769295in}}{\pgfqpoint{3.532589in}{0.777108in}}%
\pgfpathcurveto{\pgfqpoint{3.524776in}{0.784922in}}{\pgfqpoint{3.514177in}{0.789312in}}{\pgfqpoint{3.503126in}{0.789312in}}%
\pgfpathcurveto{\pgfqpoint{3.492076in}{0.789312in}}{\pgfqpoint{3.481477in}{0.784922in}}{\pgfqpoint{3.473664in}{0.777108in}}%
\pgfpathcurveto{\pgfqpoint{3.465850in}{0.769295in}}{\pgfqpoint{3.461460in}{0.758696in}}{\pgfqpoint{3.461460in}{0.747645in}}%
\pgfpathcurveto{\pgfqpoint{3.461460in}{0.736595in}}{\pgfqpoint{3.465850in}{0.725996in}}{\pgfqpoint{3.473664in}{0.718183in}}%
\pgfpathcurveto{\pgfqpoint{3.481477in}{0.710369in}}{\pgfqpoint{3.492076in}{0.705979in}}{\pgfqpoint{3.503126in}{0.705979in}}%
\pgfpathlineto{\pgfqpoint{3.503126in}{0.705979in}}%
\pgfpathclose%
\pgfusepath{stroke}%
\end{pgfscope}%
\begin{pgfscope}%
\pgfpathrectangle{\pgfqpoint{0.494722in}{0.437222in}}{\pgfqpoint{6.275590in}{5.159444in}}%
\pgfusepath{clip}%
\pgfsetbuttcap%
\pgfsetroundjoin%
\pgfsetlinewidth{1.003750pt}%
\definecolor{currentstroke}{rgb}{0.827451,0.827451,0.827451}%
\pgfsetstrokecolor{currentstroke}%
\pgfsetstrokeopacity{0.800000}%
\pgfsetdash{}{0pt}%
\pgfpathmoveto{\pgfqpoint{1.340965in}{2.104570in}}%
\pgfpathcurveto{\pgfqpoint{1.352015in}{2.104570in}}{\pgfqpoint{1.362614in}{2.108960in}}{\pgfqpoint{1.370428in}{2.116774in}}%
\pgfpathcurveto{\pgfqpoint{1.378241in}{2.124587in}}{\pgfqpoint{1.382631in}{2.135186in}}{\pgfqpoint{1.382631in}{2.146237in}}%
\pgfpathcurveto{\pgfqpoint{1.382631in}{2.157287in}}{\pgfqpoint{1.378241in}{2.167886in}}{\pgfqpoint{1.370428in}{2.175699in}}%
\pgfpathcurveto{\pgfqpoint{1.362614in}{2.183513in}}{\pgfqpoint{1.352015in}{2.187903in}}{\pgfqpoint{1.340965in}{2.187903in}}%
\pgfpathcurveto{\pgfqpoint{1.329915in}{2.187903in}}{\pgfqpoint{1.319316in}{2.183513in}}{\pgfqpoint{1.311502in}{2.175699in}}%
\pgfpathcurveto{\pgfqpoint{1.303688in}{2.167886in}}{\pgfqpoint{1.299298in}{2.157287in}}{\pgfqpoint{1.299298in}{2.146237in}}%
\pgfpathcurveto{\pgfqpoint{1.299298in}{2.135186in}}{\pgfqpoint{1.303688in}{2.124587in}}{\pgfqpoint{1.311502in}{2.116774in}}%
\pgfpathcurveto{\pgfqpoint{1.319316in}{2.108960in}}{\pgfqpoint{1.329915in}{2.104570in}}{\pgfqpoint{1.340965in}{2.104570in}}%
\pgfpathlineto{\pgfqpoint{1.340965in}{2.104570in}}%
\pgfpathclose%
\pgfusepath{stroke}%
\end{pgfscope}%
\begin{pgfscope}%
\pgfpathrectangle{\pgfqpoint{0.494722in}{0.437222in}}{\pgfqpoint{6.275590in}{5.159444in}}%
\pgfusepath{clip}%
\pgfsetbuttcap%
\pgfsetroundjoin%
\pgfsetlinewidth{1.003750pt}%
\definecolor{currentstroke}{rgb}{0.827451,0.827451,0.827451}%
\pgfsetstrokecolor{currentstroke}%
\pgfsetstrokeopacity{0.800000}%
\pgfsetdash{}{0pt}%
\pgfpathmoveto{\pgfqpoint{2.358056in}{1.258815in}}%
\pgfpathcurveto{\pgfqpoint{2.369106in}{1.258815in}}{\pgfqpoint{2.379705in}{1.263205in}}{\pgfqpoint{2.387518in}{1.271019in}}%
\pgfpathcurveto{\pgfqpoint{2.395332in}{1.278832in}}{\pgfqpoint{2.399722in}{1.289431in}}{\pgfqpoint{2.399722in}{1.300481in}}%
\pgfpathcurveto{\pgfqpoint{2.399722in}{1.311532in}}{\pgfqpoint{2.395332in}{1.322131in}}{\pgfqpoint{2.387518in}{1.329944in}}%
\pgfpathcurveto{\pgfqpoint{2.379705in}{1.337758in}}{\pgfqpoint{2.369106in}{1.342148in}}{\pgfqpoint{2.358056in}{1.342148in}}%
\pgfpathcurveto{\pgfqpoint{2.347006in}{1.342148in}}{\pgfqpoint{2.336407in}{1.337758in}}{\pgfqpoint{2.328593in}{1.329944in}}%
\pgfpathcurveto{\pgfqpoint{2.320779in}{1.322131in}}{\pgfqpoint{2.316389in}{1.311532in}}{\pgfqpoint{2.316389in}{1.300481in}}%
\pgfpathcurveto{\pgfqpoint{2.316389in}{1.289431in}}{\pgfqpoint{2.320779in}{1.278832in}}{\pgfqpoint{2.328593in}{1.271019in}}%
\pgfpathcurveto{\pgfqpoint{2.336407in}{1.263205in}}{\pgfqpoint{2.347006in}{1.258815in}}{\pgfqpoint{2.358056in}{1.258815in}}%
\pgfpathlineto{\pgfqpoint{2.358056in}{1.258815in}}%
\pgfpathclose%
\pgfusepath{stroke}%
\end{pgfscope}%
\begin{pgfscope}%
\pgfpathrectangle{\pgfqpoint{0.494722in}{0.437222in}}{\pgfqpoint{6.275590in}{5.159444in}}%
\pgfusepath{clip}%
\pgfsetbuttcap%
\pgfsetroundjoin%
\pgfsetlinewidth{1.003750pt}%
\definecolor{currentstroke}{rgb}{0.827451,0.827451,0.827451}%
\pgfsetstrokecolor{currentstroke}%
\pgfsetstrokeopacity{0.800000}%
\pgfsetdash{}{0pt}%
\pgfpathmoveto{\pgfqpoint{1.464755in}{1.963622in}}%
\pgfpathcurveto{\pgfqpoint{1.475805in}{1.963622in}}{\pgfqpoint{1.486404in}{1.968013in}}{\pgfqpoint{1.494217in}{1.975826in}}%
\pgfpathcurveto{\pgfqpoint{1.502031in}{1.983640in}}{\pgfqpoint{1.506421in}{1.994239in}}{\pgfqpoint{1.506421in}{2.005289in}}%
\pgfpathcurveto{\pgfqpoint{1.506421in}{2.016339in}}{\pgfqpoint{1.502031in}{2.026938in}}{\pgfqpoint{1.494217in}{2.034752in}}%
\pgfpathcurveto{\pgfqpoint{1.486404in}{2.042566in}}{\pgfqpoint{1.475805in}{2.046956in}}{\pgfqpoint{1.464755in}{2.046956in}}%
\pgfpathcurveto{\pgfqpoint{1.453705in}{2.046956in}}{\pgfqpoint{1.443106in}{2.042566in}}{\pgfqpoint{1.435292in}{2.034752in}}%
\pgfpathcurveto{\pgfqpoint{1.427478in}{2.026938in}}{\pgfqpoint{1.423088in}{2.016339in}}{\pgfqpoint{1.423088in}{2.005289in}}%
\pgfpathcurveto{\pgfqpoint{1.423088in}{1.994239in}}{\pgfqpoint{1.427478in}{1.983640in}}{\pgfqpoint{1.435292in}{1.975826in}}%
\pgfpathcurveto{\pgfqpoint{1.443106in}{1.968013in}}{\pgfqpoint{1.453705in}{1.963622in}}{\pgfqpoint{1.464755in}{1.963622in}}%
\pgfpathlineto{\pgfqpoint{1.464755in}{1.963622in}}%
\pgfpathclose%
\pgfusepath{stroke}%
\end{pgfscope}%
\begin{pgfscope}%
\pgfpathrectangle{\pgfqpoint{0.494722in}{0.437222in}}{\pgfqpoint{6.275590in}{5.159444in}}%
\pgfusepath{clip}%
\pgfsetbuttcap%
\pgfsetroundjoin%
\pgfsetlinewidth{1.003750pt}%
\definecolor{currentstroke}{rgb}{0.827451,0.827451,0.827451}%
\pgfsetstrokecolor{currentstroke}%
\pgfsetstrokeopacity{0.800000}%
\pgfsetdash{}{0pt}%
\pgfpathmoveto{\pgfqpoint{3.753908in}{0.666834in}}%
\pgfpathcurveto{\pgfqpoint{3.764958in}{0.666834in}}{\pgfqpoint{3.775557in}{0.671224in}}{\pgfqpoint{3.783371in}{0.679038in}}%
\pgfpathcurveto{\pgfqpoint{3.791185in}{0.686852in}}{\pgfqpoint{3.795575in}{0.697451in}}{\pgfqpoint{3.795575in}{0.708501in}}%
\pgfpathcurveto{\pgfqpoint{3.795575in}{0.719551in}}{\pgfqpoint{3.791185in}{0.730150in}}{\pgfqpoint{3.783371in}{0.737964in}}%
\pgfpathcurveto{\pgfqpoint{3.775557in}{0.745777in}}{\pgfqpoint{3.764958in}{0.750167in}}{\pgfqpoint{3.753908in}{0.750167in}}%
\pgfpathcurveto{\pgfqpoint{3.742858in}{0.750167in}}{\pgfqpoint{3.732259in}{0.745777in}}{\pgfqpoint{3.724445in}{0.737964in}}%
\pgfpathcurveto{\pgfqpoint{3.716632in}{0.730150in}}{\pgfqpoint{3.712242in}{0.719551in}}{\pgfqpoint{3.712242in}{0.708501in}}%
\pgfpathcurveto{\pgfqpoint{3.712242in}{0.697451in}}{\pgfqpoint{3.716632in}{0.686852in}}{\pgfqpoint{3.724445in}{0.679038in}}%
\pgfpathcurveto{\pgfqpoint{3.732259in}{0.671224in}}{\pgfqpoint{3.742858in}{0.666834in}}{\pgfqpoint{3.753908in}{0.666834in}}%
\pgfpathlineto{\pgfqpoint{3.753908in}{0.666834in}}%
\pgfpathclose%
\pgfusepath{stroke}%
\end{pgfscope}%
\begin{pgfscope}%
\pgfpathrectangle{\pgfqpoint{0.494722in}{0.437222in}}{\pgfqpoint{6.275590in}{5.159444in}}%
\pgfusepath{clip}%
\pgfsetbuttcap%
\pgfsetroundjoin%
\pgfsetlinewidth{1.003750pt}%
\definecolor{currentstroke}{rgb}{0.827451,0.827451,0.827451}%
\pgfsetstrokecolor{currentstroke}%
\pgfsetstrokeopacity{0.800000}%
\pgfsetdash{}{0pt}%
\pgfpathmoveto{\pgfqpoint{0.498024in}{4.460098in}}%
\pgfpathcurveto{\pgfqpoint{0.509074in}{4.460098in}}{\pgfqpoint{0.519673in}{4.464488in}}{\pgfqpoint{0.527487in}{4.472302in}}%
\pgfpathcurveto{\pgfqpoint{0.535301in}{4.480116in}}{\pgfqpoint{0.539691in}{4.490715in}}{\pgfqpoint{0.539691in}{4.501765in}}%
\pgfpathcurveto{\pgfqpoint{0.539691in}{4.512815in}}{\pgfqpoint{0.535301in}{4.523414in}}{\pgfqpoint{0.527487in}{4.531228in}}%
\pgfpathcurveto{\pgfqpoint{0.519673in}{4.539041in}}{\pgfqpoint{0.509074in}{4.543432in}}{\pgfqpoint{0.498024in}{4.543432in}}%
\pgfpathcurveto{\pgfqpoint{0.486974in}{4.543432in}}{\pgfqpoint{0.476375in}{4.539041in}}{\pgfqpoint{0.468561in}{4.531228in}}%
\pgfpathcurveto{\pgfqpoint{0.460748in}{4.523414in}}{\pgfqpoint{0.456358in}{4.512815in}}{\pgfqpoint{0.456358in}{4.501765in}}%
\pgfpathcurveto{\pgfqpoint{0.456358in}{4.490715in}}{\pgfqpoint{0.460748in}{4.480116in}}{\pgfqpoint{0.468561in}{4.472302in}}%
\pgfpathcurveto{\pgfqpoint{0.476375in}{4.464488in}}{\pgfqpoint{0.486974in}{4.460098in}}{\pgfqpoint{0.498024in}{4.460098in}}%
\pgfpathlineto{\pgfqpoint{0.498024in}{4.460098in}}%
\pgfpathclose%
\pgfusepath{stroke}%
\end{pgfscope}%
\begin{pgfscope}%
\pgfpathrectangle{\pgfqpoint{0.494722in}{0.437222in}}{\pgfqpoint{6.275590in}{5.159444in}}%
\pgfusepath{clip}%
\pgfsetbuttcap%
\pgfsetroundjoin%
\pgfsetlinewidth{1.003750pt}%
\definecolor{currentstroke}{rgb}{0.827451,0.827451,0.827451}%
\pgfsetstrokecolor{currentstroke}%
\pgfsetstrokeopacity{0.800000}%
\pgfsetdash{}{0pt}%
\pgfpathmoveto{\pgfqpoint{5.744566in}{0.401389in}}%
\pgfpathcurveto{\pgfqpoint{5.755616in}{0.401389in}}{\pgfqpoint{5.766215in}{0.405779in}}{\pgfqpoint{5.774029in}{0.413593in}}%
\pgfpathcurveto{\pgfqpoint{5.781843in}{0.421407in}}{\pgfqpoint{5.786233in}{0.432006in}}{\pgfqpoint{5.786233in}{0.443056in}}%
\pgfpathcurveto{\pgfqpoint{5.786233in}{0.454106in}}{\pgfqpoint{5.781843in}{0.464705in}}{\pgfqpoint{5.774029in}{0.472518in}}%
\pgfpathcurveto{\pgfqpoint{5.766215in}{0.480332in}}{\pgfqpoint{5.755616in}{0.484722in}}{\pgfqpoint{5.744566in}{0.484722in}}%
\pgfpathcurveto{\pgfqpoint{5.733516in}{0.484722in}}{\pgfqpoint{5.722917in}{0.480332in}}{\pgfqpoint{5.715103in}{0.472518in}}%
\pgfpathcurveto{\pgfqpoint{5.707290in}{0.464705in}}{\pgfqpoint{5.702900in}{0.454106in}}{\pgfqpoint{5.702900in}{0.443056in}}%
\pgfpathcurveto{\pgfqpoint{5.702900in}{0.432006in}}{\pgfqpoint{5.707290in}{0.421407in}}{\pgfqpoint{5.715103in}{0.413593in}}%
\pgfpathcurveto{\pgfqpoint{5.722917in}{0.405779in}}{\pgfqpoint{5.733516in}{0.401389in}}{\pgfqpoint{5.744566in}{0.401389in}}%
\pgfusepath{stroke}%
\end{pgfscope}%
\begin{pgfscope}%
\pgfpathrectangle{\pgfqpoint{0.494722in}{0.437222in}}{\pgfqpoint{6.275590in}{5.159444in}}%
\pgfusepath{clip}%
\pgfsetbuttcap%
\pgfsetroundjoin%
\pgfsetlinewidth{1.003750pt}%
\definecolor{currentstroke}{rgb}{0.827451,0.827451,0.827451}%
\pgfsetstrokecolor{currentstroke}%
\pgfsetstrokeopacity{0.800000}%
\pgfsetdash{}{0pt}%
\pgfpathmoveto{\pgfqpoint{5.377588in}{0.405033in}}%
\pgfpathcurveto{\pgfqpoint{5.388638in}{0.405033in}}{\pgfqpoint{5.399237in}{0.409423in}}{\pgfqpoint{5.407050in}{0.417237in}}%
\pgfpathcurveto{\pgfqpoint{5.414864in}{0.425051in}}{\pgfqpoint{5.419254in}{0.435650in}}{\pgfqpoint{5.419254in}{0.446700in}}%
\pgfpathcurveto{\pgfqpoint{5.419254in}{0.457750in}}{\pgfqpoint{5.414864in}{0.468349in}}{\pgfqpoint{5.407050in}{0.476163in}}%
\pgfpathcurveto{\pgfqpoint{5.399237in}{0.483976in}}{\pgfqpoint{5.388638in}{0.488366in}}{\pgfqpoint{5.377588in}{0.488366in}}%
\pgfpathcurveto{\pgfqpoint{5.366537in}{0.488366in}}{\pgfqpoint{5.355938in}{0.483976in}}{\pgfqpoint{5.348125in}{0.476163in}}%
\pgfpathcurveto{\pgfqpoint{5.340311in}{0.468349in}}{\pgfqpoint{5.335921in}{0.457750in}}{\pgfqpoint{5.335921in}{0.446700in}}%
\pgfpathcurveto{\pgfqpoint{5.335921in}{0.435650in}}{\pgfqpoint{5.340311in}{0.425051in}}{\pgfqpoint{5.348125in}{0.417237in}}%
\pgfpathcurveto{\pgfqpoint{5.355938in}{0.409423in}}{\pgfqpoint{5.366537in}{0.405033in}}{\pgfqpoint{5.377588in}{0.405033in}}%
\pgfusepath{stroke}%
\end{pgfscope}%
\begin{pgfscope}%
\pgfpathrectangle{\pgfqpoint{0.494722in}{0.437222in}}{\pgfqpoint{6.275590in}{5.159444in}}%
\pgfusepath{clip}%
\pgfsetbuttcap%
\pgfsetroundjoin%
\pgfsetlinewidth{1.003750pt}%
\definecolor{currentstroke}{rgb}{0.827451,0.827451,0.827451}%
\pgfsetstrokecolor{currentstroke}%
\pgfsetstrokeopacity{0.800000}%
\pgfsetdash{}{0pt}%
\pgfpathmoveto{\pgfqpoint{4.702565in}{0.471432in}}%
\pgfpathcurveto{\pgfqpoint{4.713615in}{0.471432in}}{\pgfqpoint{4.724214in}{0.475823in}}{\pgfqpoint{4.732028in}{0.483636in}}%
\pgfpathcurveto{\pgfqpoint{4.739841in}{0.491450in}}{\pgfqpoint{4.744232in}{0.502049in}}{\pgfqpoint{4.744232in}{0.513099in}}%
\pgfpathcurveto{\pgfqpoint{4.744232in}{0.524149in}}{\pgfqpoint{4.739841in}{0.534748in}}{\pgfqpoint{4.732028in}{0.542562in}}%
\pgfpathcurveto{\pgfqpoint{4.724214in}{0.550376in}}{\pgfqpoint{4.713615in}{0.554766in}}{\pgfqpoint{4.702565in}{0.554766in}}%
\pgfpathcurveto{\pgfqpoint{4.691515in}{0.554766in}}{\pgfqpoint{4.680916in}{0.550376in}}{\pgfqpoint{4.673102in}{0.542562in}}%
\pgfpathcurveto{\pgfqpoint{4.665289in}{0.534748in}}{\pgfqpoint{4.660898in}{0.524149in}}{\pgfqpoint{4.660898in}{0.513099in}}%
\pgfpathcurveto{\pgfqpoint{4.660898in}{0.502049in}}{\pgfqpoint{4.665289in}{0.491450in}}{\pgfqpoint{4.673102in}{0.483636in}}%
\pgfpathcurveto{\pgfqpoint{4.680916in}{0.475823in}}{\pgfqpoint{4.691515in}{0.471432in}}{\pgfqpoint{4.702565in}{0.471432in}}%
\pgfpathlineto{\pgfqpoint{4.702565in}{0.471432in}}%
\pgfpathclose%
\pgfusepath{stroke}%
\end{pgfscope}%
\begin{pgfscope}%
\pgfpathrectangle{\pgfqpoint{0.494722in}{0.437222in}}{\pgfqpoint{6.275590in}{5.159444in}}%
\pgfusepath{clip}%
\pgfsetbuttcap%
\pgfsetroundjoin%
\pgfsetlinewidth{1.003750pt}%
\definecolor{currentstroke}{rgb}{0.827451,0.827451,0.827451}%
\pgfsetstrokecolor{currentstroke}%
\pgfsetstrokeopacity{0.800000}%
\pgfsetdash{}{0pt}%
\pgfpathmoveto{\pgfqpoint{0.501219in}{4.373826in}}%
\pgfpathcurveto{\pgfqpoint{0.512269in}{4.373826in}}{\pgfqpoint{0.522868in}{4.378216in}}{\pgfqpoint{0.530682in}{4.386030in}}%
\pgfpathcurveto{\pgfqpoint{0.538495in}{4.393844in}}{\pgfqpoint{0.542886in}{4.404443in}}{\pgfqpoint{0.542886in}{4.415493in}}%
\pgfpathcurveto{\pgfqpoint{0.542886in}{4.426543in}}{\pgfqpoint{0.538495in}{4.437142in}}{\pgfqpoint{0.530682in}{4.444955in}}%
\pgfpathcurveto{\pgfqpoint{0.522868in}{4.452769in}}{\pgfqpoint{0.512269in}{4.457159in}}{\pgfqpoint{0.501219in}{4.457159in}}%
\pgfpathcurveto{\pgfqpoint{0.490169in}{4.457159in}}{\pgfqpoint{0.479570in}{4.452769in}}{\pgfqpoint{0.471756in}{4.444955in}}%
\pgfpathcurveto{\pgfqpoint{0.463943in}{4.437142in}}{\pgfqpoint{0.459552in}{4.426543in}}{\pgfqpoint{0.459552in}{4.415493in}}%
\pgfpathcurveto{\pgfqpoint{0.459552in}{4.404443in}}{\pgfqpoint{0.463943in}{4.393844in}}{\pgfqpoint{0.471756in}{4.386030in}}%
\pgfpathcurveto{\pgfqpoint{0.479570in}{4.378216in}}{\pgfqpoint{0.490169in}{4.373826in}}{\pgfqpoint{0.501219in}{4.373826in}}%
\pgfpathlineto{\pgfqpoint{0.501219in}{4.373826in}}%
\pgfpathclose%
\pgfusepath{stroke}%
\end{pgfscope}%
\begin{pgfscope}%
\pgfpathrectangle{\pgfqpoint{0.494722in}{0.437222in}}{\pgfqpoint{6.275590in}{5.159444in}}%
\pgfusepath{clip}%
\pgfsetbuttcap%
\pgfsetroundjoin%
\pgfsetlinewidth{1.003750pt}%
\definecolor{currentstroke}{rgb}{0.827451,0.827451,0.827451}%
\pgfsetstrokecolor{currentstroke}%
\pgfsetstrokeopacity{0.800000}%
\pgfsetdash{}{0pt}%
\pgfpathmoveto{\pgfqpoint{3.832585in}{0.609011in}}%
\pgfpathcurveto{\pgfqpoint{3.843635in}{0.609011in}}{\pgfqpoint{3.854234in}{0.613402in}}{\pgfqpoint{3.862047in}{0.621215in}}%
\pgfpathcurveto{\pgfqpoint{3.869861in}{0.629029in}}{\pgfqpoint{3.874251in}{0.639628in}}{\pgfqpoint{3.874251in}{0.650678in}}%
\pgfpathcurveto{\pgfqpoint{3.874251in}{0.661728in}}{\pgfqpoint{3.869861in}{0.672327in}}{\pgfqpoint{3.862047in}{0.680141in}}%
\pgfpathcurveto{\pgfqpoint{3.854234in}{0.687954in}}{\pgfqpoint{3.843635in}{0.692345in}}{\pgfqpoint{3.832585in}{0.692345in}}%
\pgfpathcurveto{\pgfqpoint{3.821534in}{0.692345in}}{\pgfqpoint{3.810935in}{0.687954in}}{\pgfqpoint{3.803122in}{0.680141in}}%
\pgfpathcurveto{\pgfqpoint{3.795308in}{0.672327in}}{\pgfqpoint{3.790918in}{0.661728in}}{\pgfqpoint{3.790918in}{0.650678in}}%
\pgfpathcurveto{\pgfqpoint{3.790918in}{0.639628in}}{\pgfqpoint{3.795308in}{0.629029in}}{\pgfqpoint{3.803122in}{0.621215in}}%
\pgfpathcurveto{\pgfqpoint{3.810935in}{0.613402in}}{\pgfqpoint{3.821534in}{0.609011in}}{\pgfqpoint{3.832585in}{0.609011in}}%
\pgfpathlineto{\pgfqpoint{3.832585in}{0.609011in}}%
\pgfpathclose%
\pgfusepath{stroke}%
\end{pgfscope}%
\begin{pgfscope}%
\pgfpathrectangle{\pgfqpoint{0.494722in}{0.437222in}}{\pgfqpoint{6.275590in}{5.159444in}}%
\pgfusepath{clip}%
\pgfsetbuttcap%
\pgfsetroundjoin%
\pgfsetlinewidth{1.003750pt}%
\definecolor{currentstroke}{rgb}{0.827451,0.827451,0.827451}%
\pgfsetstrokecolor{currentstroke}%
\pgfsetstrokeopacity{0.800000}%
\pgfsetdash{}{0pt}%
\pgfpathmoveto{\pgfqpoint{5.183443in}{0.422556in}}%
\pgfpathcurveto{\pgfqpoint{5.194493in}{0.422556in}}{\pgfqpoint{5.205092in}{0.426946in}}{\pgfqpoint{5.212906in}{0.434760in}}%
\pgfpathcurveto{\pgfqpoint{5.220719in}{0.442573in}}{\pgfqpoint{5.225109in}{0.453172in}}{\pgfqpoint{5.225109in}{0.464222in}}%
\pgfpathcurveto{\pgfqpoint{5.225109in}{0.475273in}}{\pgfqpoint{5.220719in}{0.485872in}}{\pgfqpoint{5.212906in}{0.493685in}}%
\pgfpathcurveto{\pgfqpoint{5.205092in}{0.501499in}}{\pgfqpoint{5.194493in}{0.505889in}}{\pgfqpoint{5.183443in}{0.505889in}}%
\pgfpathcurveto{\pgfqpoint{5.172393in}{0.505889in}}{\pgfqpoint{5.161794in}{0.501499in}}{\pgfqpoint{5.153980in}{0.493685in}}%
\pgfpathcurveto{\pgfqpoint{5.146166in}{0.485872in}}{\pgfqpoint{5.141776in}{0.475273in}}{\pgfqpoint{5.141776in}{0.464222in}}%
\pgfpathcurveto{\pgfqpoint{5.141776in}{0.453172in}}{\pgfqpoint{5.146166in}{0.442573in}}{\pgfqpoint{5.153980in}{0.434760in}}%
\pgfpathcurveto{\pgfqpoint{5.161794in}{0.426946in}}{\pgfqpoint{5.172393in}{0.422556in}}{\pgfqpoint{5.183443in}{0.422556in}}%
\pgfusepath{stroke}%
\end{pgfscope}%
\begin{pgfscope}%
\pgfpathrectangle{\pgfqpoint{0.494722in}{0.437222in}}{\pgfqpoint{6.275590in}{5.159444in}}%
\pgfusepath{clip}%
\pgfsetbuttcap%
\pgfsetroundjoin%
\pgfsetlinewidth{1.003750pt}%
\definecolor{currentstroke}{rgb}{0.827451,0.827451,0.827451}%
\pgfsetstrokecolor{currentstroke}%
\pgfsetstrokeopacity{0.800000}%
\pgfsetdash{}{0pt}%
\pgfpathmoveto{\pgfqpoint{0.841077in}{2.944285in}}%
\pgfpathcurveto{\pgfqpoint{0.852127in}{2.944285in}}{\pgfqpoint{0.862726in}{2.948676in}}{\pgfqpoint{0.870540in}{2.956489in}}%
\pgfpathcurveto{\pgfqpoint{0.878353in}{2.964303in}}{\pgfqpoint{0.882744in}{2.974902in}}{\pgfqpoint{0.882744in}{2.985952in}}%
\pgfpathcurveto{\pgfqpoint{0.882744in}{2.997002in}}{\pgfqpoint{0.878353in}{3.007601in}}{\pgfqpoint{0.870540in}{3.015415in}}%
\pgfpathcurveto{\pgfqpoint{0.862726in}{3.023228in}}{\pgfqpoint{0.852127in}{3.027619in}}{\pgfqpoint{0.841077in}{3.027619in}}%
\pgfpathcurveto{\pgfqpoint{0.830027in}{3.027619in}}{\pgfqpoint{0.819428in}{3.023228in}}{\pgfqpoint{0.811614in}{3.015415in}}%
\pgfpathcurveto{\pgfqpoint{0.803801in}{3.007601in}}{\pgfqpoint{0.799410in}{2.997002in}}{\pgfqpoint{0.799410in}{2.985952in}}%
\pgfpathcurveto{\pgfqpoint{0.799410in}{2.974902in}}{\pgfqpoint{0.803801in}{2.964303in}}{\pgfqpoint{0.811614in}{2.956489in}}%
\pgfpathcurveto{\pgfqpoint{0.819428in}{2.948676in}}{\pgfqpoint{0.830027in}{2.944285in}}{\pgfqpoint{0.841077in}{2.944285in}}%
\pgfpathlineto{\pgfqpoint{0.841077in}{2.944285in}}%
\pgfpathclose%
\pgfusepath{stroke}%
\end{pgfscope}%
\begin{pgfscope}%
\pgfpathrectangle{\pgfqpoint{0.494722in}{0.437222in}}{\pgfqpoint{6.275590in}{5.159444in}}%
\pgfusepath{clip}%
\pgfsetbuttcap%
\pgfsetroundjoin%
\pgfsetlinewidth{1.003750pt}%
\definecolor{currentstroke}{rgb}{0.827451,0.827451,0.827451}%
\pgfsetstrokecolor{currentstroke}%
\pgfsetstrokeopacity{0.800000}%
\pgfsetdash{}{0pt}%
\pgfpathmoveto{\pgfqpoint{0.770298in}{3.077236in}}%
\pgfpathcurveto{\pgfqpoint{0.781349in}{3.077236in}}{\pgfqpoint{0.791948in}{3.081626in}}{\pgfqpoint{0.799761in}{3.089440in}}%
\pgfpathcurveto{\pgfqpoint{0.807575in}{3.097254in}}{\pgfqpoint{0.811965in}{3.107853in}}{\pgfqpoint{0.811965in}{3.118903in}}%
\pgfpathcurveto{\pgfqpoint{0.811965in}{3.129953in}}{\pgfqpoint{0.807575in}{3.140552in}}{\pgfqpoint{0.799761in}{3.148366in}}%
\pgfpathcurveto{\pgfqpoint{0.791948in}{3.156179in}}{\pgfqpoint{0.781349in}{3.160569in}}{\pgfqpoint{0.770298in}{3.160569in}}%
\pgfpathcurveto{\pgfqpoint{0.759248in}{3.160569in}}{\pgfqpoint{0.748649in}{3.156179in}}{\pgfqpoint{0.740836in}{3.148366in}}%
\pgfpathcurveto{\pgfqpoint{0.733022in}{3.140552in}}{\pgfqpoint{0.728632in}{3.129953in}}{\pgfqpoint{0.728632in}{3.118903in}}%
\pgfpathcurveto{\pgfqpoint{0.728632in}{3.107853in}}{\pgfqpoint{0.733022in}{3.097254in}}{\pgfqpoint{0.740836in}{3.089440in}}%
\pgfpathcurveto{\pgfqpoint{0.748649in}{3.081626in}}{\pgfqpoint{0.759248in}{3.077236in}}{\pgfqpoint{0.770298in}{3.077236in}}%
\pgfpathlineto{\pgfqpoint{0.770298in}{3.077236in}}%
\pgfpathclose%
\pgfusepath{stroke}%
\end{pgfscope}%
\begin{pgfscope}%
\pgfpathrectangle{\pgfqpoint{0.494722in}{0.437222in}}{\pgfqpoint{6.275590in}{5.159444in}}%
\pgfusepath{clip}%
\pgfsetbuttcap%
\pgfsetroundjoin%
\pgfsetlinewidth{1.003750pt}%
\definecolor{currentstroke}{rgb}{0.827451,0.827451,0.827451}%
\pgfsetstrokecolor{currentstroke}%
\pgfsetstrokeopacity{0.800000}%
\pgfsetdash{}{0pt}%
\pgfpathmoveto{\pgfqpoint{0.520093in}{4.158669in}}%
\pgfpathcurveto{\pgfqpoint{0.531143in}{4.158669in}}{\pgfqpoint{0.541742in}{4.163059in}}{\pgfqpoint{0.549556in}{4.170873in}}%
\pgfpathcurveto{\pgfqpoint{0.557369in}{4.178687in}}{\pgfqpoint{0.561759in}{4.189286in}}{\pgfqpoint{0.561759in}{4.200336in}}%
\pgfpathcurveto{\pgfqpoint{0.561759in}{4.211386in}}{\pgfqpoint{0.557369in}{4.221985in}}{\pgfqpoint{0.549556in}{4.229799in}}%
\pgfpathcurveto{\pgfqpoint{0.541742in}{4.237612in}}{\pgfqpoint{0.531143in}{4.242003in}}{\pgfqpoint{0.520093in}{4.242003in}}%
\pgfpathcurveto{\pgfqpoint{0.509043in}{4.242003in}}{\pgfqpoint{0.498444in}{4.237612in}}{\pgfqpoint{0.490630in}{4.229799in}}%
\pgfpathcurveto{\pgfqpoint{0.482816in}{4.221985in}}{\pgfqpoint{0.478426in}{4.211386in}}{\pgfqpoint{0.478426in}{4.200336in}}%
\pgfpathcurveto{\pgfqpoint{0.478426in}{4.189286in}}{\pgfqpoint{0.482816in}{4.178687in}}{\pgfqpoint{0.490630in}{4.170873in}}%
\pgfpathcurveto{\pgfqpoint{0.498444in}{4.163059in}}{\pgfqpoint{0.509043in}{4.158669in}}{\pgfqpoint{0.520093in}{4.158669in}}%
\pgfpathlineto{\pgfqpoint{0.520093in}{4.158669in}}%
\pgfpathclose%
\pgfusepath{stroke}%
\end{pgfscope}%
\begin{pgfscope}%
\pgfpathrectangle{\pgfqpoint{0.494722in}{0.437222in}}{\pgfqpoint{6.275590in}{5.159444in}}%
\pgfusepath{clip}%
\pgfsetbuttcap%
\pgfsetroundjoin%
\pgfsetlinewidth{1.003750pt}%
\definecolor{currentstroke}{rgb}{0.827451,0.827451,0.827451}%
\pgfsetstrokecolor{currentstroke}%
\pgfsetstrokeopacity{0.800000}%
\pgfsetdash{}{0pt}%
\pgfpathmoveto{\pgfqpoint{0.944970in}{2.756986in}}%
\pgfpathcurveto{\pgfqpoint{0.956020in}{2.756986in}}{\pgfqpoint{0.966619in}{2.761376in}}{\pgfqpoint{0.974433in}{2.769190in}}%
\pgfpathcurveto{\pgfqpoint{0.982246in}{2.777004in}}{\pgfqpoint{0.986637in}{2.787603in}}{\pgfqpoint{0.986637in}{2.798653in}}%
\pgfpathcurveto{\pgfqpoint{0.986637in}{2.809703in}}{\pgfqpoint{0.982246in}{2.820302in}}{\pgfqpoint{0.974433in}{2.828116in}}%
\pgfpathcurveto{\pgfqpoint{0.966619in}{2.835929in}}{\pgfqpoint{0.956020in}{2.840319in}}{\pgfqpoint{0.944970in}{2.840319in}}%
\pgfpathcurveto{\pgfqpoint{0.933920in}{2.840319in}}{\pgfqpoint{0.923321in}{2.835929in}}{\pgfqpoint{0.915507in}{2.828116in}}%
\pgfpathcurveto{\pgfqpoint{0.907694in}{2.820302in}}{\pgfqpoint{0.903303in}{2.809703in}}{\pgfqpoint{0.903303in}{2.798653in}}%
\pgfpathcurveto{\pgfqpoint{0.903303in}{2.787603in}}{\pgfqpoint{0.907694in}{2.777004in}}{\pgfqpoint{0.915507in}{2.769190in}}%
\pgfpathcurveto{\pgfqpoint{0.923321in}{2.761376in}}{\pgfqpoint{0.933920in}{2.756986in}}{\pgfqpoint{0.944970in}{2.756986in}}%
\pgfpathlineto{\pgfqpoint{0.944970in}{2.756986in}}%
\pgfpathclose%
\pgfusepath{stroke}%
\end{pgfscope}%
\begin{pgfscope}%
\pgfpathrectangle{\pgfqpoint{0.494722in}{0.437222in}}{\pgfqpoint{6.275590in}{5.159444in}}%
\pgfusepath{clip}%
\pgfsetbuttcap%
\pgfsetroundjoin%
\pgfsetlinewidth{1.003750pt}%
\definecolor{currentstroke}{rgb}{0.827451,0.827451,0.827451}%
\pgfsetstrokecolor{currentstroke}%
\pgfsetstrokeopacity{0.800000}%
\pgfsetdash{}{0pt}%
\pgfpathmoveto{\pgfqpoint{4.942939in}{0.453295in}}%
\pgfpathcurveto{\pgfqpoint{4.953989in}{0.453295in}}{\pgfqpoint{4.964589in}{0.457685in}}{\pgfqpoint{4.972402in}{0.465499in}}%
\pgfpathcurveto{\pgfqpoint{4.980216in}{0.473313in}}{\pgfqpoint{4.984606in}{0.483912in}}{\pgfqpoint{4.984606in}{0.494962in}}%
\pgfpathcurveto{\pgfqpoint{4.984606in}{0.506012in}}{\pgfqpoint{4.980216in}{0.516611in}}{\pgfqpoint{4.972402in}{0.524425in}}%
\pgfpathcurveto{\pgfqpoint{4.964589in}{0.532238in}}{\pgfqpoint{4.953989in}{0.536628in}}{\pgfqpoint{4.942939in}{0.536628in}}%
\pgfpathcurveto{\pgfqpoint{4.931889in}{0.536628in}}{\pgfqpoint{4.921290in}{0.532238in}}{\pgfqpoint{4.913477in}{0.524425in}}%
\pgfpathcurveto{\pgfqpoint{4.905663in}{0.516611in}}{\pgfqpoint{4.901273in}{0.506012in}}{\pgfqpoint{4.901273in}{0.494962in}}%
\pgfpathcurveto{\pgfqpoint{4.901273in}{0.483912in}}{\pgfqpoint{4.905663in}{0.473313in}}{\pgfqpoint{4.913477in}{0.465499in}}%
\pgfpathcurveto{\pgfqpoint{4.921290in}{0.457685in}}{\pgfqpoint{4.931889in}{0.453295in}}{\pgfqpoint{4.942939in}{0.453295in}}%
\pgfpathlineto{\pgfqpoint{4.942939in}{0.453295in}}%
\pgfpathclose%
\pgfusepath{stroke}%
\end{pgfscope}%
\begin{pgfscope}%
\pgfpathrectangle{\pgfqpoint{0.494722in}{0.437222in}}{\pgfqpoint{6.275590in}{5.159444in}}%
\pgfusepath{clip}%
\pgfsetbuttcap%
\pgfsetroundjoin%
\pgfsetlinewidth{1.003750pt}%
\definecolor{currentstroke}{rgb}{0.827451,0.827451,0.827451}%
\pgfsetstrokecolor{currentstroke}%
\pgfsetstrokeopacity{0.800000}%
\pgfsetdash{}{0pt}%
\pgfpathmoveto{\pgfqpoint{3.022481in}{0.888267in}}%
\pgfpathcurveto{\pgfqpoint{3.033531in}{0.888267in}}{\pgfqpoint{3.044130in}{0.892657in}}{\pgfqpoint{3.051944in}{0.900471in}}%
\pgfpathcurveto{\pgfqpoint{3.059758in}{0.908285in}}{\pgfqpoint{3.064148in}{0.918884in}}{\pgfqpoint{3.064148in}{0.929934in}}%
\pgfpathcurveto{\pgfqpoint{3.064148in}{0.940984in}}{\pgfqpoint{3.059758in}{0.951583in}}{\pgfqpoint{3.051944in}{0.959397in}}%
\pgfpathcurveto{\pgfqpoint{3.044130in}{0.967210in}}{\pgfqpoint{3.033531in}{0.971600in}}{\pgfqpoint{3.022481in}{0.971600in}}%
\pgfpathcurveto{\pgfqpoint{3.011431in}{0.971600in}}{\pgfqpoint{3.000832in}{0.967210in}}{\pgfqpoint{2.993019in}{0.959397in}}%
\pgfpathcurveto{\pgfqpoint{2.985205in}{0.951583in}}{\pgfqpoint{2.980815in}{0.940984in}}{\pgfqpoint{2.980815in}{0.929934in}}%
\pgfpathcurveto{\pgfqpoint{2.980815in}{0.918884in}}{\pgfqpoint{2.985205in}{0.908285in}}{\pgfqpoint{2.993019in}{0.900471in}}%
\pgfpathcurveto{\pgfqpoint{3.000832in}{0.892657in}}{\pgfqpoint{3.011431in}{0.888267in}}{\pgfqpoint{3.022481in}{0.888267in}}%
\pgfpathlineto{\pgfqpoint{3.022481in}{0.888267in}}%
\pgfpathclose%
\pgfusepath{stroke}%
\end{pgfscope}%
\begin{pgfscope}%
\pgfpathrectangle{\pgfqpoint{0.494722in}{0.437222in}}{\pgfqpoint{6.275590in}{5.159444in}}%
\pgfusepath{clip}%
\pgfsetbuttcap%
\pgfsetroundjoin%
\pgfsetlinewidth{1.003750pt}%
\definecolor{currentstroke}{rgb}{0.827451,0.827451,0.827451}%
\pgfsetstrokecolor{currentstroke}%
\pgfsetstrokeopacity{0.800000}%
\pgfsetdash{}{0pt}%
\pgfpathmoveto{\pgfqpoint{3.709744in}{0.686390in}}%
\pgfpathcurveto{\pgfqpoint{3.720794in}{0.686390in}}{\pgfqpoint{3.731393in}{0.690780in}}{\pgfqpoint{3.739207in}{0.698593in}}%
\pgfpathcurveto{\pgfqpoint{3.747021in}{0.706407in}}{\pgfqpoint{3.751411in}{0.717006in}}{\pgfqpoint{3.751411in}{0.728056in}}%
\pgfpathcurveto{\pgfqpoint{3.751411in}{0.739106in}}{\pgfqpoint{3.747021in}{0.749705in}}{\pgfqpoint{3.739207in}{0.757519in}}%
\pgfpathcurveto{\pgfqpoint{3.731393in}{0.765333in}}{\pgfqpoint{3.720794in}{0.769723in}}{\pgfqpoint{3.709744in}{0.769723in}}%
\pgfpathcurveto{\pgfqpoint{3.698694in}{0.769723in}}{\pgfqpoint{3.688095in}{0.765333in}}{\pgfqpoint{3.680281in}{0.757519in}}%
\pgfpathcurveto{\pgfqpoint{3.672468in}{0.749705in}}{\pgfqpoint{3.668077in}{0.739106in}}{\pgfqpoint{3.668077in}{0.728056in}}%
\pgfpathcurveto{\pgfqpoint{3.668077in}{0.717006in}}{\pgfqpoint{3.672468in}{0.706407in}}{\pgfqpoint{3.680281in}{0.698593in}}%
\pgfpathcurveto{\pgfqpoint{3.688095in}{0.690780in}}{\pgfqpoint{3.698694in}{0.686390in}}{\pgfqpoint{3.709744in}{0.686390in}}%
\pgfpathlineto{\pgfqpoint{3.709744in}{0.686390in}}%
\pgfpathclose%
\pgfusepath{stroke}%
\end{pgfscope}%
\begin{pgfscope}%
\pgfpathrectangle{\pgfqpoint{0.494722in}{0.437222in}}{\pgfqpoint{6.275590in}{5.159444in}}%
\pgfusepath{clip}%
\pgfsetbuttcap%
\pgfsetroundjoin%
\pgfsetlinewidth{1.003750pt}%
\definecolor{currentstroke}{rgb}{0.827451,0.827451,0.827451}%
\pgfsetstrokecolor{currentstroke}%
\pgfsetstrokeopacity{0.800000}%
\pgfsetdash{}{0pt}%
\pgfpathmoveto{\pgfqpoint{4.119757in}{0.538399in}}%
\pgfpathcurveto{\pgfqpoint{4.130807in}{0.538399in}}{\pgfqpoint{4.141406in}{0.542789in}}{\pgfqpoint{4.149220in}{0.550603in}}%
\pgfpathcurveto{\pgfqpoint{4.157034in}{0.558417in}}{\pgfqpoint{4.161424in}{0.569016in}}{\pgfqpoint{4.161424in}{0.580066in}}%
\pgfpathcurveto{\pgfqpoint{4.161424in}{0.591116in}}{\pgfqpoint{4.157034in}{0.601715in}}{\pgfqpoint{4.149220in}{0.609529in}}%
\pgfpathcurveto{\pgfqpoint{4.141406in}{0.617342in}}{\pgfqpoint{4.130807in}{0.621732in}}{\pgfqpoint{4.119757in}{0.621732in}}%
\pgfpathcurveto{\pgfqpoint{4.108707in}{0.621732in}}{\pgfqpoint{4.098108in}{0.617342in}}{\pgfqpoint{4.090294in}{0.609529in}}%
\pgfpathcurveto{\pgfqpoint{4.082481in}{0.601715in}}{\pgfqpoint{4.078091in}{0.591116in}}{\pgfqpoint{4.078091in}{0.580066in}}%
\pgfpathcurveto{\pgfqpoint{4.078091in}{0.569016in}}{\pgfqpoint{4.082481in}{0.558417in}}{\pgfqpoint{4.090294in}{0.550603in}}%
\pgfpathcurveto{\pgfqpoint{4.098108in}{0.542789in}}{\pgfqpoint{4.108707in}{0.538399in}}{\pgfqpoint{4.119757in}{0.538399in}}%
\pgfpathlineto{\pgfqpoint{4.119757in}{0.538399in}}%
\pgfpathclose%
\pgfusepath{stroke}%
\end{pgfscope}%
\begin{pgfscope}%
\pgfpathrectangle{\pgfqpoint{0.494722in}{0.437222in}}{\pgfqpoint{6.275590in}{5.159444in}}%
\pgfusepath{clip}%
\pgfsetbuttcap%
\pgfsetroundjoin%
\pgfsetlinewidth{1.003750pt}%
\definecolor{currentstroke}{rgb}{0.827451,0.827451,0.827451}%
\pgfsetstrokecolor{currentstroke}%
\pgfsetstrokeopacity{0.800000}%
\pgfsetdash{}{0pt}%
\pgfpathmoveto{\pgfqpoint{1.700731in}{1.766489in}}%
\pgfpathcurveto{\pgfqpoint{1.711781in}{1.766489in}}{\pgfqpoint{1.722380in}{1.770879in}}{\pgfqpoint{1.730194in}{1.778693in}}%
\pgfpathcurveto{\pgfqpoint{1.738007in}{1.786506in}}{\pgfqpoint{1.742398in}{1.797105in}}{\pgfqpoint{1.742398in}{1.808156in}}%
\pgfpathcurveto{\pgfqpoint{1.742398in}{1.819206in}}{\pgfqpoint{1.738007in}{1.829805in}}{\pgfqpoint{1.730194in}{1.837618in}}%
\pgfpathcurveto{\pgfqpoint{1.722380in}{1.845432in}}{\pgfqpoint{1.711781in}{1.849822in}}{\pgfqpoint{1.700731in}{1.849822in}}%
\pgfpathcurveto{\pgfqpoint{1.689681in}{1.849822in}}{\pgfqpoint{1.679082in}{1.845432in}}{\pgfqpoint{1.671268in}{1.837618in}}%
\pgfpathcurveto{\pgfqpoint{1.663455in}{1.829805in}}{\pgfqpoint{1.659064in}{1.819206in}}{\pgfqpoint{1.659064in}{1.808156in}}%
\pgfpathcurveto{\pgfqpoint{1.659064in}{1.797105in}}{\pgfqpoint{1.663455in}{1.786506in}}{\pgfqpoint{1.671268in}{1.778693in}}%
\pgfpathcurveto{\pgfqpoint{1.679082in}{1.770879in}}{\pgfqpoint{1.689681in}{1.766489in}}{\pgfqpoint{1.700731in}{1.766489in}}%
\pgfpathlineto{\pgfqpoint{1.700731in}{1.766489in}}%
\pgfpathclose%
\pgfusepath{stroke}%
\end{pgfscope}%
\begin{pgfscope}%
\pgfpathrectangle{\pgfqpoint{0.494722in}{0.437222in}}{\pgfqpoint{6.275590in}{5.159444in}}%
\pgfusepath{clip}%
\pgfsetbuttcap%
\pgfsetroundjoin%
\pgfsetlinewidth{1.003750pt}%
\definecolor{currentstroke}{rgb}{0.827451,0.827451,0.827451}%
\pgfsetstrokecolor{currentstroke}%
\pgfsetstrokeopacity{0.800000}%
\pgfsetdash{}{0pt}%
\pgfpathmoveto{\pgfqpoint{4.229480in}{0.518110in}}%
\pgfpathcurveto{\pgfqpoint{4.240530in}{0.518110in}}{\pgfqpoint{4.251129in}{0.522500in}}{\pgfqpoint{4.258942in}{0.530314in}}%
\pgfpathcurveto{\pgfqpoint{4.266756in}{0.538127in}}{\pgfqpoint{4.271146in}{0.548726in}}{\pgfqpoint{4.271146in}{0.559776in}}%
\pgfpathcurveto{\pgfqpoint{4.271146in}{0.570827in}}{\pgfqpoint{4.266756in}{0.581426in}}{\pgfqpoint{4.258942in}{0.589239in}}%
\pgfpathcurveto{\pgfqpoint{4.251129in}{0.597053in}}{\pgfqpoint{4.240530in}{0.601443in}}{\pgfqpoint{4.229480in}{0.601443in}}%
\pgfpathcurveto{\pgfqpoint{4.218429in}{0.601443in}}{\pgfqpoint{4.207830in}{0.597053in}}{\pgfqpoint{4.200017in}{0.589239in}}%
\pgfpathcurveto{\pgfqpoint{4.192203in}{0.581426in}}{\pgfqpoint{4.187813in}{0.570827in}}{\pgfqpoint{4.187813in}{0.559776in}}%
\pgfpathcurveto{\pgfqpoint{4.187813in}{0.548726in}}{\pgfqpoint{4.192203in}{0.538127in}}{\pgfqpoint{4.200017in}{0.530314in}}%
\pgfpathcurveto{\pgfqpoint{4.207830in}{0.522500in}}{\pgfqpoint{4.218429in}{0.518110in}}{\pgfqpoint{4.229480in}{0.518110in}}%
\pgfpathlineto{\pgfqpoint{4.229480in}{0.518110in}}%
\pgfpathclose%
\pgfusepath{stroke}%
\end{pgfscope}%
\begin{pgfscope}%
\pgfpathrectangle{\pgfqpoint{0.494722in}{0.437222in}}{\pgfqpoint{6.275590in}{5.159444in}}%
\pgfusepath{clip}%
\pgfsetbuttcap%
\pgfsetroundjoin%
\pgfsetlinewidth{1.003750pt}%
\definecolor{currentstroke}{rgb}{0.827451,0.827451,0.827451}%
\pgfsetstrokecolor{currentstroke}%
\pgfsetstrokeopacity{0.800000}%
\pgfsetdash{}{0pt}%
\pgfpathmoveto{\pgfqpoint{2.932655in}{0.929387in}}%
\pgfpathcurveto{\pgfqpoint{2.943705in}{0.929387in}}{\pgfqpoint{2.954304in}{0.933777in}}{\pgfqpoint{2.962118in}{0.941591in}}%
\pgfpathcurveto{\pgfqpoint{2.969932in}{0.949404in}}{\pgfqpoint{2.974322in}{0.960003in}}{\pgfqpoint{2.974322in}{0.971053in}}%
\pgfpathcurveto{\pgfqpoint{2.974322in}{0.982103in}}{\pgfqpoint{2.969932in}{0.992702in}}{\pgfqpoint{2.962118in}{1.000516in}}%
\pgfpathcurveto{\pgfqpoint{2.954304in}{1.008330in}}{\pgfqpoint{2.943705in}{1.012720in}}{\pgfqpoint{2.932655in}{1.012720in}}%
\pgfpathcurveto{\pgfqpoint{2.921605in}{1.012720in}}{\pgfqpoint{2.911006in}{1.008330in}}{\pgfqpoint{2.903193in}{1.000516in}}%
\pgfpathcurveto{\pgfqpoint{2.895379in}{0.992702in}}{\pgfqpoint{2.890989in}{0.982103in}}{\pgfqpoint{2.890989in}{0.971053in}}%
\pgfpathcurveto{\pgfqpoint{2.890989in}{0.960003in}}{\pgfqpoint{2.895379in}{0.949404in}}{\pgfqpoint{2.903193in}{0.941591in}}%
\pgfpathcurveto{\pgfqpoint{2.911006in}{0.933777in}}{\pgfqpoint{2.921605in}{0.929387in}}{\pgfqpoint{2.932655in}{0.929387in}}%
\pgfpathlineto{\pgfqpoint{2.932655in}{0.929387in}}%
\pgfpathclose%
\pgfusepath{stroke}%
\end{pgfscope}%
\begin{pgfscope}%
\pgfpathrectangle{\pgfqpoint{0.494722in}{0.437222in}}{\pgfqpoint{6.275590in}{5.159444in}}%
\pgfusepath{clip}%
\pgfsetbuttcap%
\pgfsetroundjoin%
\pgfsetlinewidth{1.003750pt}%
\definecolor{currentstroke}{rgb}{0.827451,0.827451,0.827451}%
\pgfsetstrokecolor{currentstroke}%
\pgfsetstrokeopacity{0.800000}%
\pgfsetdash{}{0pt}%
\pgfpathmoveto{\pgfqpoint{0.870505in}{2.877763in}}%
\pgfpathcurveto{\pgfqpoint{0.881555in}{2.877763in}}{\pgfqpoint{0.892154in}{2.882154in}}{\pgfqpoint{0.899967in}{2.889967in}}%
\pgfpathcurveto{\pgfqpoint{0.907781in}{2.897781in}}{\pgfqpoint{0.912171in}{2.908380in}}{\pgfqpoint{0.912171in}{2.919430in}}%
\pgfpathcurveto{\pgfqpoint{0.912171in}{2.930480in}}{\pgfqpoint{0.907781in}{2.941079in}}{\pgfqpoint{0.899967in}{2.948893in}}%
\pgfpathcurveto{\pgfqpoint{0.892154in}{2.956706in}}{\pgfqpoint{0.881555in}{2.961097in}}{\pgfqpoint{0.870505in}{2.961097in}}%
\pgfpathcurveto{\pgfqpoint{0.859455in}{2.961097in}}{\pgfqpoint{0.848856in}{2.956706in}}{\pgfqpoint{0.841042in}{2.948893in}}%
\pgfpathcurveto{\pgfqpoint{0.833228in}{2.941079in}}{\pgfqpoint{0.828838in}{2.930480in}}{\pgfqpoint{0.828838in}{2.919430in}}%
\pgfpathcurveto{\pgfqpoint{0.828838in}{2.908380in}}{\pgfqpoint{0.833228in}{2.897781in}}{\pgfqpoint{0.841042in}{2.889967in}}%
\pgfpathcurveto{\pgfqpoint{0.848856in}{2.882154in}}{\pgfqpoint{0.859455in}{2.877763in}}{\pgfqpoint{0.870505in}{2.877763in}}%
\pgfpathlineto{\pgfqpoint{0.870505in}{2.877763in}}%
\pgfpathclose%
\pgfusepath{stroke}%
\end{pgfscope}%
\begin{pgfscope}%
\pgfpathrectangle{\pgfqpoint{0.494722in}{0.437222in}}{\pgfqpoint{6.275590in}{5.159444in}}%
\pgfusepath{clip}%
\pgfsetbuttcap%
\pgfsetroundjoin%
\pgfsetlinewidth{1.003750pt}%
\definecolor{currentstroke}{rgb}{0.827451,0.827451,0.827451}%
\pgfsetstrokecolor{currentstroke}%
\pgfsetstrokeopacity{0.800000}%
\pgfsetdash{}{0pt}%
\pgfpathmoveto{\pgfqpoint{0.679488in}{3.431236in}}%
\pgfpathcurveto{\pgfqpoint{0.690538in}{3.431236in}}{\pgfqpoint{0.701137in}{3.435626in}}{\pgfqpoint{0.708951in}{3.443440in}}%
\pgfpathcurveto{\pgfqpoint{0.716765in}{3.451253in}}{\pgfqpoint{0.721155in}{3.461852in}}{\pgfqpoint{0.721155in}{3.472902in}}%
\pgfpathcurveto{\pgfqpoint{0.721155in}{3.483953in}}{\pgfqpoint{0.716765in}{3.494552in}}{\pgfqpoint{0.708951in}{3.502365in}}%
\pgfpathcurveto{\pgfqpoint{0.701137in}{3.510179in}}{\pgfqpoint{0.690538in}{3.514569in}}{\pgfqpoint{0.679488in}{3.514569in}}%
\pgfpathcurveto{\pgfqpoint{0.668438in}{3.514569in}}{\pgfqpoint{0.657839in}{3.510179in}}{\pgfqpoint{0.650026in}{3.502365in}}%
\pgfpathcurveto{\pgfqpoint{0.642212in}{3.494552in}}{\pgfqpoint{0.637822in}{3.483953in}}{\pgfqpoint{0.637822in}{3.472902in}}%
\pgfpathcurveto{\pgfqpoint{0.637822in}{3.461852in}}{\pgfqpoint{0.642212in}{3.451253in}}{\pgfqpoint{0.650026in}{3.443440in}}%
\pgfpathcurveto{\pgfqpoint{0.657839in}{3.435626in}}{\pgfqpoint{0.668438in}{3.431236in}}{\pgfqpoint{0.679488in}{3.431236in}}%
\pgfpathlineto{\pgfqpoint{0.679488in}{3.431236in}}%
\pgfpathclose%
\pgfusepath{stroke}%
\end{pgfscope}%
\begin{pgfscope}%
\pgfpathrectangle{\pgfqpoint{0.494722in}{0.437222in}}{\pgfqpoint{6.275590in}{5.159444in}}%
\pgfusepath{clip}%
\pgfsetbuttcap%
\pgfsetroundjoin%
\pgfsetlinewidth{1.003750pt}%
\definecolor{currentstroke}{rgb}{0.827451,0.827451,0.827451}%
\pgfsetstrokecolor{currentstroke}%
\pgfsetstrokeopacity{0.800000}%
\pgfsetdash{}{0pt}%
\pgfpathmoveto{\pgfqpoint{5.060352in}{0.429754in}}%
\pgfpathcurveto{\pgfqpoint{5.071402in}{0.429754in}}{\pgfqpoint{5.082001in}{0.434145in}}{\pgfqpoint{5.089815in}{0.441958in}}%
\pgfpathcurveto{\pgfqpoint{5.097629in}{0.449772in}}{\pgfqpoint{5.102019in}{0.460371in}}{\pgfqpoint{5.102019in}{0.471421in}}%
\pgfpathcurveto{\pgfqpoint{5.102019in}{0.482471in}}{\pgfqpoint{5.097629in}{0.493070in}}{\pgfqpoint{5.089815in}{0.500884in}}%
\pgfpathcurveto{\pgfqpoint{5.082001in}{0.508697in}}{\pgfqpoint{5.071402in}{0.513088in}}{\pgfqpoint{5.060352in}{0.513088in}}%
\pgfpathcurveto{\pgfqpoint{5.049302in}{0.513088in}}{\pgfqpoint{5.038703in}{0.508697in}}{\pgfqpoint{5.030890in}{0.500884in}}%
\pgfpathcurveto{\pgfqpoint{5.023076in}{0.493070in}}{\pgfqpoint{5.018686in}{0.482471in}}{\pgfqpoint{5.018686in}{0.471421in}}%
\pgfpathcurveto{\pgfqpoint{5.018686in}{0.460371in}}{\pgfqpoint{5.023076in}{0.449772in}}{\pgfqpoint{5.030890in}{0.441958in}}%
\pgfpathcurveto{\pgfqpoint{5.038703in}{0.434145in}}{\pgfqpoint{5.049302in}{0.429754in}}{\pgfqpoint{5.060352in}{0.429754in}}%
\pgfpathlineto{\pgfqpoint{5.060352in}{0.429754in}}%
\pgfpathclose%
\pgfusepath{stroke}%
\end{pgfscope}%
\begin{pgfscope}%
\pgfpathrectangle{\pgfqpoint{0.494722in}{0.437222in}}{\pgfqpoint{6.275590in}{5.159444in}}%
\pgfusepath{clip}%
\pgfsetbuttcap%
\pgfsetroundjoin%
\pgfsetlinewidth{1.003750pt}%
\definecolor{currentstroke}{rgb}{0.827451,0.827451,0.827451}%
\pgfsetstrokecolor{currentstroke}%
\pgfsetstrokeopacity{0.800000}%
\pgfsetdash{}{0pt}%
\pgfpathmoveto{\pgfqpoint{0.683182in}{3.327378in}}%
\pgfpathcurveto{\pgfqpoint{0.694232in}{3.327378in}}{\pgfqpoint{0.704831in}{3.331768in}}{\pgfqpoint{0.712645in}{3.339581in}}%
\pgfpathcurveto{\pgfqpoint{0.720459in}{3.347395in}}{\pgfqpoint{0.724849in}{3.357994in}}{\pgfqpoint{0.724849in}{3.369044in}}%
\pgfpathcurveto{\pgfqpoint{0.724849in}{3.380094in}}{\pgfqpoint{0.720459in}{3.390693in}}{\pgfqpoint{0.712645in}{3.398507in}}%
\pgfpathcurveto{\pgfqpoint{0.704831in}{3.406321in}}{\pgfqpoint{0.694232in}{3.410711in}}{\pgfqpoint{0.683182in}{3.410711in}}%
\pgfpathcurveto{\pgfqpoint{0.672132in}{3.410711in}}{\pgfqpoint{0.661533in}{3.406321in}}{\pgfqpoint{0.653720in}{3.398507in}}%
\pgfpathcurveto{\pgfqpoint{0.645906in}{3.390693in}}{\pgfqpoint{0.641516in}{3.380094in}}{\pgfqpoint{0.641516in}{3.369044in}}%
\pgfpathcurveto{\pgfqpoint{0.641516in}{3.357994in}}{\pgfqpoint{0.645906in}{3.347395in}}{\pgfqpoint{0.653720in}{3.339581in}}%
\pgfpathcurveto{\pgfqpoint{0.661533in}{3.331768in}}{\pgfqpoint{0.672132in}{3.327378in}}{\pgfqpoint{0.683182in}{3.327378in}}%
\pgfpathlineto{\pgfqpoint{0.683182in}{3.327378in}}%
\pgfpathclose%
\pgfusepath{stroke}%
\end{pgfscope}%
\begin{pgfscope}%
\pgfpathrectangle{\pgfqpoint{0.494722in}{0.437222in}}{\pgfqpoint{6.275590in}{5.159444in}}%
\pgfusepath{clip}%
\pgfsetbuttcap%
\pgfsetroundjoin%
\pgfsetlinewidth{1.003750pt}%
\definecolor{currentstroke}{rgb}{0.827451,0.827451,0.827451}%
\pgfsetstrokecolor{currentstroke}%
\pgfsetstrokeopacity{0.800000}%
\pgfsetdash{}{0pt}%
\pgfpathmoveto{\pgfqpoint{1.000236in}{2.738855in}}%
\pgfpathcurveto{\pgfqpoint{1.011286in}{2.738855in}}{\pgfqpoint{1.021885in}{2.743245in}}{\pgfqpoint{1.029699in}{2.751059in}}%
\pgfpathcurveto{\pgfqpoint{1.037513in}{2.758873in}}{\pgfqpoint{1.041903in}{2.769472in}}{\pgfqpoint{1.041903in}{2.780522in}}%
\pgfpathcurveto{\pgfqpoint{1.041903in}{2.791572in}}{\pgfqpoint{1.037513in}{2.802171in}}{\pgfqpoint{1.029699in}{2.809985in}}%
\pgfpathcurveto{\pgfqpoint{1.021885in}{2.817798in}}{\pgfqpoint{1.011286in}{2.822189in}}{\pgfqpoint{1.000236in}{2.822189in}}%
\pgfpathcurveto{\pgfqpoint{0.989186in}{2.822189in}}{\pgfqpoint{0.978587in}{2.817798in}}{\pgfqpoint{0.970773in}{2.809985in}}%
\pgfpathcurveto{\pgfqpoint{0.962960in}{2.802171in}}{\pgfqpoint{0.958570in}{2.791572in}}{\pgfqpoint{0.958570in}{2.780522in}}%
\pgfpathcurveto{\pgfqpoint{0.958570in}{2.769472in}}{\pgfqpoint{0.962960in}{2.758873in}}{\pgfqpoint{0.970773in}{2.751059in}}%
\pgfpathcurveto{\pgfqpoint{0.978587in}{2.743245in}}{\pgfqpoint{0.989186in}{2.738855in}}{\pgfqpoint{1.000236in}{2.738855in}}%
\pgfpathlineto{\pgfqpoint{1.000236in}{2.738855in}}%
\pgfpathclose%
\pgfusepath{stroke}%
\end{pgfscope}%
\begin{pgfscope}%
\pgfpathrectangle{\pgfqpoint{0.494722in}{0.437222in}}{\pgfqpoint{6.275590in}{5.159444in}}%
\pgfusepath{clip}%
\pgfsetbuttcap%
\pgfsetroundjoin%
\pgfsetlinewidth{1.003750pt}%
\definecolor{currentstroke}{rgb}{0.827451,0.827451,0.827451}%
\pgfsetstrokecolor{currentstroke}%
\pgfsetstrokeopacity{0.800000}%
\pgfsetdash{}{0pt}%
\pgfpathmoveto{\pgfqpoint{4.010167in}{0.576865in}}%
\pgfpathcurveto{\pgfqpoint{4.021217in}{0.576865in}}{\pgfqpoint{4.031816in}{0.581255in}}{\pgfqpoint{4.039630in}{0.589069in}}%
\pgfpathcurveto{\pgfqpoint{4.047443in}{0.596882in}}{\pgfqpoint{4.051834in}{0.607481in}}{\pgfqpoint{4.051834in}{0.618531in}}%
\pgfpathcurveto{\pgfqpoint{4.051834in}{0.629581in}}{\pgfqpoint{4.047443in}{0.640180in}}{\pgfqpoint{4.039630in}{0.647994in}}%
\pgfpathcurveto{\pgfqpoint{4.031816in}{0.655808in}}{\pgfqpoint{4.021217in}{0.660198in}}{\pgfqpoint{4.010167in}{0.660198in}}%
\pgfpathcurveto{\pgfqpoint{3.999117in}{0.660198in}}{\pgfqpoint{3.988518in}{0.655808in}}{\pgfqpoint{3.980704in}{0.647994in}}%
\pgfpathcurveto{\pgfqpoint{3.972890in}{0.640180in}}{\pgfqpoint{3.968500in}{0.629581in}}{\pgfqpoint{3.968500in}{0.618531in}}%
\pgfpathcurveto{\pgfqpoint{3.968500in}{0.607481in}}{\pgfqpoint{3.972890in}{0.596882in}}{\pgfqpoint{3.980704in}{0.589069in}}%
\pgfpathcurveto{\pgfqpoint{3.988518in}{0.581255in}}{\pgfqpoint{3.999117in}{0.576865in}}{\pgfqpoint{4.010167in}{0.576865in}}%
\pgfpathlineto{\pgfqpoint{4.010167in}{0.576865in}}%
\pgfpathclose%
\pgfusepath{stroke}%
\end{pgfscope}%
\begin{pgfscope}%
\pgfpathrectangle{\pgfqpoint{0.494722in}{0.437222in}}{\pgfqpoint{6.275590in}{5.159444in}}%
\pgfusepath{clip}%
\pgfsetbuttcap%
\pgfsetroundjoin%
\pgfsetlinewidth{1.003750pt}%
\definecolor{currentstroke}{rgb}{0.827451,0.827451,0.827451}%
\pgfsetstrokecolor{currentstroke}%
\pgfsetstrokeopacity{0.800000}%
\pgfsetdash{}{0pt}%
\pgfpathmoveto{\pgfqpoint{3.737625in}{0.637421in}}%
\pgfpathcurveto{\pgfqpoint{3.748675in}{0.637421in}}{\pgfqpoint{3.759274in}{0.641812in}}{\pgfqpoint{3.767088in}{0.649625in}}%
\pgfpathcurveto{\pgfqpoint{3.774901in}{0.657439in}}{\pgfqpoint{3.779292in}{0.668038in}}{\pgfqpoint{3.779292in}{0.679088in}}%
\pgfpathcurveto{\pgfqpoint{3.779292in}{0.690138in}}{\pgfqpoint{3.774901in}{0.700737in}}{\pgfqpoint{3.767088in}{0.708551in}}%
\pgfpathcurveto{\pgfqpoint{3.759274in}{0.716364in}}{\pgfqpoint{3.748675in}{0.720755in}}{\pgfqpoint{3.737625in}{0.720755in}}%
\pgfpathcurveto{\pgfqpoint{3.726575in}{0.720755in}}{\pgfqpoint{3.715976in}{0.716364in}}{\pgfqpoint{3.708162in}{0.708551in}}%
\pgfpathcurveto{\pgfqpoint{3.700349in}{0.700737in}}{\pgfqpoint{3.695958in}{0.690138in}}{\pgfqpoint{3.695958in}{0.679088in}}%
\pgfpathcurveto{\pgfqpoint{3.695958in}{0.668038in}}{\pgfqpoint{3.700349in}{0.657439in}}{\pgfqpoint{3.708162in}{0.649625in}}%
\pgfpathcurveto{\pgfqpoint{3.715976in}{0.641812in}}{\pgfqpoint{3.726575in}{0.637421in}}{\pgfqpoint{3.737625in}{0.637421in}}%
\pgfpathlineto{\pgfqpoint{3.737625in}{0.637421in}}%
\pgfpathclose%
\pgfusepath{stroke}%
\end{pgfscope}%
\begin{pgfscope}%
\pgfpathrectangle{\pgfqpoint{0.494722in}{0.437222in}}{\pgfqpoint{6.275590in}{5.159444in}}%
\pgfusepath{clip}%
\pgfsetbuttcap%
\pgfsetroundjoin%
\pgfsetlinewidth{1.003750pt}%
\definecolor{currentstroke}{rgb}{0.827451,0.827451,0.827451}%
\pgfsetstrokecolor{currentstroke}%
\pgfsetstrokeopacity{0.800000}%
\pgfsetdash{}{0pt}%
\pgfpathmoveto{\pgfqpoint{2.308193in}{1.269247in}}%
\pgfpathcurveto{\pgfqpoint{2.319243in}{1.269247in}}{\pgfqpoint{2.329842in}{1.273637in}}{\pgfqpoint{2.337656in}{1.281451in}}%
\pgfpathcurveto{\pgfqpoint{2.345469in}{1.289264in}}{\pgfqpoint{2.349860in}{1.299864in}}{\pgfqpoint{2.349860in}{1.310914in}}%
\pgfpathcurveto{\pgfqpoint{2.349860in}{1.321964in}}{\pgfqpoint{2.345469in}{1.332563in}}{\pgfqpoint{2.337656in}{1.340376in}}%
\pgfpathcurveto{\pgfqpoint{2.329842in}{1.348190in}}{\pgfqpoint{2.319243in}{1.352580in}}{\pgfqpoint{2.308193in}{1.352580in}}%
\pgfpathcurveto{\pgfqpoint{2.297143in}{1.352580in}}{\pgfqpoint{2.286544in}{1.348190in}}{\pgfqpoint{2.278730in}{1.340376in}}%
\pgfpathcurveto{\pgfqpoint{2.270917in}{1.332563in}}{\pgfqpoint{2.266526in}{1.321964in}}{\pgfqpoint{2.266526in}{1.310914in}}%
\pgfpathcurveto{\pgfqpoint{2.266526in}{1.299864in}}{\pgfqpoint{2.270917in}{1.289264in}}{\pgfqpoint{2.278730in}{1.281451in}}%
\pgfpathcurveto{\pgfqpoint{2.286544in}{1.273637in}}{\pgfqpoint{2.297143in}{1.269247in}}{\pgfqpoint{2.308193in}{1.269247in}}%
\pgfpathlineto{\pgfqpoint{2.308193in}{1.269247in}}%
\pgfpathclose%
\pgfusepath{stroke}%
\end{pgfscope}%
\begin{pgfscope}%
\pgfpathrectangle{\pgfqpoint{0.494722in}{0.437222in}}{\pgfqpoint{6.275590in}{5.159444in}}%
\pgfusepath{clip}%
\pgfsetbuttcap%
\pgfsetroundjoin%
\pgfsetlinewidth{1.003750pt}%
\definecolor{currentstroke}{rgb}{0.827451,0.827451,0.827451}%
\pgfsetstrokecolor{currentstroke}%
\pgfsetstrokeopacity{0.800000}%
\pgfsetdash{}{0pt}%
\pgfpathmoveto{\pgfqpoint{1.113110in}{2.511719in}}%
\pgfpathcurveto{\pgfqpoint{1.124161in}{2.511719in}}{\pgfqpoint{1.134760in}{2.516109in}}{\pgfqpoint{1.142573in}{2.523923in}}%
\pgfpathcurveto{\pgfqpoint{1.150387in}{2.531736in}}{\pgfqpoint{1.154777in}{2.542335in}}{\pgfqpoint{1.154777in}{2.553385in}}%
\pgfpathcurveto{\pgfqpoint{1.154777in}{2.564435in}}{\pgfqpoint{1.150387in}{2.575034in}}{\pgfqpoint{1.142573in}{2.582848in}}%
\pgfpathcurveto{\pgfqpoint{1.134760in}{2.590662in}}{\pgfqpoint{1.124161in}{2.595052in}}{\pgfqpoint{1.113110in}{2.595052in}}%
\pgfpathcurveto{\pgfqpoint{1.102060in}{2.595052in}}{\pgfqpoint{1.091461in}{2.590662in}}{\pgfqpoint{1.083648in}{2.582848in}}%
\pgfpathcurveto{\pgfqpoint{1.075834in}{2.575034in}}{\pgfqpoint{1.071444in}{2.564435in}}{\pgfqpoint{1.071444in}{2.553385in}}%
\pgfpathcurveto{\pgfqpoint{1.071444in}{2.542335in}}{\pgfqpoint{1.075834in}{2.531736in}}{\pgfqpoint{1.083648in}{2.523923in}}%
\pgfpathcurveto{\pgfqpoint{1.091461in}{2.516109in}}{\pgfqpoint{1.102060in}{2.511719in}}{\pgfqpoint{1.113110in}{2.511719in}}%
\pgfpathlineto{\pgfqpoint{1.113110in}{2.511719in}}%
\pgfpathclose%
\pgfusepath{stroke}%
\end{pgfscope}%
\begin{pgfscope}%
\pgfpathrectangle{\pgfqpoint{0.494722in}{0.437222in}}{\pgfqpoint{6.275590in}{5.159444in}}%
\pgfusepath{clip}%
\pgfsetbuttcap%
\pgfsetroundjoin%
\pgfsetlinewidth{1.003750pt}%
\definecolor{currentstroke}{rgb}{0.827451,0.827451,0.827451}%
\pgfsetstrokecolor{currentstroke}%
\pgfsetstrokeopacity{0.800000}%
\pgfsetdash{}{0pt}%
\pgfpathmoveto{\pgfqpoint{1.006303in}{2.625866in}}%
\pgfpathcurveto{\pgfqpoint{1.017353in}{2.625866in}}{\pgfqpoint{1.027952in}{2.630257in}}{\pgfqpoint{1.035766in}{2.638070in}}%
\pgfpathcurveto{\pgfqpoint{1.043580in}{2.645884in}}{\pgfqpoint{1.047970in}{2.656483in}}{\pgfqpoint{1.047970in}{2.667533in}}%
\pgfpathcurveto{\pgfqpoint{1.047970in}{2.678583in}}{\pgfqpoint{1.043580in}{2.689182in}}{\pgfqpoint{1.035766in}{2.696996in}}%
\pgfpathcurveto{\pgfqpoint{1.027952in}{2.704810in}}{\pgfqpoint{1.017353in}{2.709200in}}{\pgfqpoint{1.006303in}{2.709200in}}%
\pgfpathcurveto{\pgfqpoint{0.995253in}{2.709200in}}{\pgfqpoint{0.984654in}{2.704810in}}{\pgfqpoint{0.976840in}{2.696996in}}%
\pgfpathcurveto{\pgfqpoint{0.969027in}{2.689182in}}{\pgfqpoint{0.964636in}{2.678583in}}{\pgfqpoint{0.964636in}{2.667533in}}%
\pgfpathcurveto{\pgfqpoint{0.964636in}{2.656483in}}{\pgfqpoint{0.969027in}{2.645884in}}{\pgfqpoint{0.976840in}{2.638070in}}%
\pgfpathcurveto{\pgfqpoint{0.984654in}{2.630257in}}{\pgfqpoint{0.995253in}{2.625866in}}{\pgfqpoint{1.006303in}{2.625866in}}%
\pgfpathlineto{\pgfqpoint{1.006303in}{2.625866in}}%
\pgfpathclose%
\pgfusepath{stroke}%
\end{pgfscope}%
\begin{pgfscope}%
\pgfpathrectangle{\pgfqpoint{0.494722in}{0.437222in}}{\pgfqpoint{6.275590in}{5.159444in}}%
\pgfusepath{clip}%
\pgfsetbuttcap%
\pgfsetroundjoin%
\pgfsetlinewidth{1.003750pt}%
\definecolor{currentstroke}{rgb}{0.827451,0.827451,0.827451}%
\pgfsetstrokecolor{currentstroke}%
\pgfsetstrokeopacity{0.800000}%
\pgfsetdash{}{0pt}%
\pgfpathmoveto{\pgfqpoint{3.121483in}{0.847531in}}%
\pgfpathcurveto{\pgfqpoint{3.132533in}{0.847531in}}{\pgfqpoint{3.143132in}{0.851921in}}{\pgfqpoint{3.150946in}{0.859735in}}%
\pgfpathcurveto{\pgfqpoint{3.158760in}{0.867548in}}{\pgfqpoint{3.163150in}{0.878147in}}{\pgfqpoint{3.163150in}{0.889198in}}%
\pgfpathcurveto{\pgfqpoint{3.163150in}{0.900248in}}{\pgfqpoint{3.158760in}{0.910847in}}{\pgfqpoint{3.150946in}{0.918660in}}%
\pgfpathcurveto{\pgfqpoint{3.143132in}{0.926474in}}{\pgfqpoint{3.132533in}{0.930864in}}{\pgfqpoint{3.121483in}{0.930864in}}%
\pgfpathcurveto{\pgfqpoint{3.110433in}{0.930864in}}{\pgfqpoint{3.099834in}{0.926474in}}{\pgfqpoint{3.092021in}{0.918660in}}%
\pgfpathcurveto{\pgfqpoint{3.084207in}{0.910847in}}{\pgfqpoint{3.079817in}{0.900248in}}{\pgfqpoint{3.079817in}{0.889198in}}%
\pgfpathcurveto{\pgfqpoint{3.079817in}{0.878147in}}{\pgfqpoint{3.084207in}{0.867548in}}{\pgfqpoint{3.092021in}{0.859735in}}%
\pgfpathcurveto{\pgfqpoint{3.099834in}{0.851921in}}{\pgfqpoint{3.110433in}{0.847531in}}{\pgfqpoint{3.121483in}{0.847531in}}%
\pgfpathlineto{\pgfqpoint{3.121483in}{0.847531in}}%
\pgfpathclose%
\pgfusepath{stroke}%
\end{pgfscope}%
\begin{pgfscope}%
\pgfpathrectangle{\pgfqpoint{0.494722in}{0.437222in}}{\pgfqpoint{6.275590in}{5.159444in}}%
\pgfusepath{clip}%
\pgfsetbuttcap%
\pgfsetroundjoin%
\pgfsetlinewidth{1.003750pt}%
\definecolor{currentstroke}{rgb}{0.827451,0.827451,0.827451}%
\pgfsetstrokecolor{currentstroke}%
\pgfsetstrokeopacity{0.800000}%
\pgfsetdash{}{0pt}%
\pgfpathmoveto{\pgfqpoint{1.133504in}{2.415631in}}%
\pgfpathcurveto{\pgfqpoint{1.144554in}{2.415631in}}{\pgfqpoint{1.155153in}{2.420021in}}{\pgfqpoint{1.162967in}{2.427835in}}%
\pgfpathcurveto{\pgfqpoint{1.170780in}{2.435648in}}{\pgfqpoint{1.175170in}{2.446247in}}{\pgfqpoint{1.175170in}{2.457297in}}%
\pgfpathcurveto{\pgfqpoint{1.175170in}{2.468347in}}{\pgfqpoint{1.170780in}{2.478946in}}{\pgfqpoint{1.162967in}{2.486760in}}%
\pgfpathcurveto{\pgfqpoint{1.155153in}{2.494574in}}{\pgfqpoint{1.144554in}{2.498964in}}{\pgfqpoint{1.133504in}{2.498964in}}%
\pgfpathcurveto{\pgfqpoint{1.122454in}{2.498964in}}{\pgfqpoint{1.111855in}{2.494574in}}{\pgfqpoint{1.104041in}{2.486760in}}%
\pgfpathcurveto{\pgfqpoint{1.096227in}{2.478946in}}{\pgfqpoint{1.091837in}{2.468347in}}{\pgfqpoint{1.091837in}{2.457297in}}%
\pgfpathcurveto{\pgfqpoint{1.091837in}{2.446247in}}{\pgfqpoint{1.096227in}{2.435648in}}{\pgfqpoint{1.104041in}{2.427835in}}%
\pgfpathcurveto{\pgfqpoint{1.111855in}{2.420021in}}{\pgfqpoint{1.122454in}{2.415631in}}{\pgfqpoint{1.133504in}{2.415631in}}%
\pgfpathlineto{\pgfqpoint{1.133504in}{2.415631in}}%
\pgfpathclose%
\pgfusepath{stroke}%
\end{pgfscope}%
\begin{pgfscope}%
\pgfpathrectangle{\pgfqpoint{0.494722in}{0.437222in}}{\pgfqpoint{6.275590in}{5.159444in}}%
\pgfusepath{clip}%
\pgfsetbuttcap%
\pgfsetroundjoin%
\pgfsetlinewidth{1.003750pt}%
\definecolor{currentstroke}{rgb}{0.827451,0.827451,0.827451}%
\pgfsetstrokecolor{currentstroke}%
\pgfsetstrokeopacity{0.800000}%
\pgfsetdash{}{0pt}%
\pgfpathmoveto{\pgfqpoint{3.527306in}{0.699588in}}%
\pgfpathcurveto{\pgfqpoint{3.538356in}{0.699588in}}{\pgfqpoint{3.548955in}{0.703979in}}{\pgfqpoint{3.556769in}{0.711792in}}%
\pgfpathcurveto{\pgfqpoint{3.564583in}{0.719606in}}{\pgfqpoint{3.568973in}{0.730205in}}{\pgfqpoint{3.568973in}{0.741255in}}%
\pgfpathcurveto{\pgfqpoint{3.568973in}{0.752305in}}{\pgfqpoint{3.564583in}{0.762904in}}{\pgfqpoint{3.556769in}{0.770718in}}%
\pgfpathcurveto{\pgfqpoint{3.548955in}{0.778532in}}{\pgfqpoint{3.538356in}{0.782922in}}{\pgfqpoint{3.527306in}{0.782922in}}%
\pgfpathcurveto{\pgfqpoint{3.516256in}{0.782922in}}{\pgfqpoint{3.505657in}{0.778532in}}{\pgfqpoint{3.497843in}{0.770718in}}%
\pgfpathcurveto{\pgfqpoint{3.490030in}{0.762904in}}{\pgfqpoint{3.485640in}{0.752305in}}{\pgfqpoint{3.485640in}{0.741255in}}%
\pgfpathcurveto{\pgfqpoint{3.485640in}{0.730205in}}{\pgfqpoint{3.490030in}{0.719606in}}{\pgfqpoint{3.497843in}{0.711792in}}%
\pgfpathcurveto{\pgfqpoint{3.505657in}{0.703979in}}{\pgfqpoint{3.516256in}{0.699588in}}{\pgfqpoint{3.527306in}{0.699588in}}%
\pgfpathlineto{\pgfqpoint{3.527306in}{0.699588in}}%
\pgfpathclose%
\pgfusepath{stroke}%
\end{pgfscope}%
\begin{pgfscope}%
\pgfpathrectangle{\pgfqpoint{0.494722in}{0.437222in}}{\pgfqpoint{6.275590in}{5.159444in}}%
\pgfusepath{clip}%
\pgfsetbuttcap%
\pgfsetroundjoin%
\pgfsetlinewidth{1.003750pt}%
\definecolor{currentstroke}{rgb}{0.827451,0.827451,0.827451}%
\pgfsetstrokecolor{currentstroke}%
\pgfsetstrokeopacity{0.800000}%
\pgfsetdash{}{0pt}%
\pgfpathmoveto{\pgfqpoint{4.586350in}{0.475898in}}%
\pgfpathcurveto{\pgfqpoint{4.597400in}{0.475898in}}{\pgfqpoint{4.607999in}{0.480288in}}{\pgfqpoint{4.615813in}{0.488101in}}%
\pgfpathcurveto{\pgfqpoint{4.623627in}{0.495915in}}{\pgfqpoint{4.628017in}{0.506514in}}{\pgfqpoint{4.628017in}{0.517564in}}%
\pgfpathcurveto{\pgfqpoint{4.628017in}{0.528614in}}{\pgfqpoint{4.623627in}{0.539213in}}{\pgfqpoint{4.615813in}{0.547027in}}%
\pgfpathcurveto{\pgfqpoint{4.607999in}{0.554841in}}{\pgfqpoint{4.597400in}{0.559231in}}{\pgfqpoint{4.586350in}{0.559231in}}%
\pgfpathcurveto{\pgfqpoint{4.575300in}{0.559231in}}{\pgfqpoint{4.564701in}{0.554841in}}{\pgfqpoint{4.556887in}{0.547027in}}%
\pgfpathcurveto{\pgfqpoint{4.549074in}{0.539213in}}{\pgfqpoint{4.544683in}{0.528614in}}{\pgfqpoint{4.544683in}{0.517564in}}%
\pgfpathcurveto{\pgfqpoint{4.544683in}{0.506514in}}{\pgfqpoint{4.549074in}{0.495915in}}{\pgfqpoint{4.556887in}{0.488101in}}%
\pgfpathcurveto{\pgfqpoint{4.564701in}{0.480288in}}{\pgfqpoint{4.575300in}{0.475898in}}{\pgfqpoint{4.586350in}{0.475898in}}%
\pgfpathlineto{\pgfqpoint{4.586350in}{0.475898in}}%
\pgfpathclose%
\pgfusepath{stroke}%
\end{pgfscope}%
\begin{pgfscope}%
\pgfpathrectangle{\pgfqpoint{0.494722in}{0.437222in}}{\pgfqpoint{6.275590in}{5.159444in}}%
\pgfusepath{clip}%
\pgfsetbuttcap%
\pgfsetroundjoin%
\pgfsetlinewidth{1.003750pt}%
\definecolor{currentstroke}{rgb}{0.827451,0.827451,0.827451}%
\pgfsetstrokecolor{currentstroke}%
\pgfsetstrokeopacity{0.800000}%
\pgfsetdash{}{0pt}%
\pgfpathmoveto{\pgfqpoint{1.091684in}{2.547519in}}%
\pgfpathcurveto{\pgfqpoint{1.102734in}{2.547519in}}{\pgfqpoint{1.113333in}{2.551909in}}{\pgfqpoint{1.121147in}{2.559722in}}%
\pgfpathcurveto{\pgfqpoint{1.128960in}{2.567536in}}{\pgfqpoint{1.133351in}{2.578135in}}{\pgfqpoint{1.133351in}{2.589185in}}%
\pgfpathcurveto{\pgfqpoint{1.133351in}{2.600235in}}{\pgfqpoint{1.128960in}{2.610834in}}{\pgfqpoint{1.121147in}{2.618648in}}%
\pgfpathcurveto{\pgfqpoint{1.113333in}{2.626462in}}{\pgfqpoint{1.102734in}{2.630852in}}{\pgfqpoint{1.091684in}{2.630852in}}%
\pgfpathcurveto{\pgfqpoint{1.080634in}{2.630852in}}{\pgfqpoint{1.070035in}{2.626462in}}{\pgfqpoint{1.062221in}{2.618648in}}%
\pgfpathcurveto{\pgfqpoint{1.054407in}{2.610834in}}{\pgfqpoint{1.050017in}{2.600235in}}{\pgfqpoint{1.050017in}{2.589185in}}%
\pgfpathcurveto{\pgfqpoint{1.050017in}{2.578135in}}{\pgfqpoint{1.054407in}{2.567536in}}{\pgfqpoint{1.062221in}{2.559722in}}%
\pgfpathcurveto{\pgfqpoint{1.070035in}{2.551909in}}{\pgfqpoint{1.080634in}{2.547519in}}{\pgfqpoint{1.091684in}{2.547519in}}%
\pgfpathlineto{\pgfqpoint{1.091684in}{2.547519in}}%
\pgfpathclose%
\pgfusepath{stroke}%
\end{pgfscope}%
\begin{pgfscope}%
\pgfpathrectangle{\pgfqpoint{0.494722in}{0.437222in}}{\pgfqpoint{6.275590in}{5.159444in}}%
\pgfusepath{clip}%
\pgfsetbuttcap%
\pgfsetroundjoin%
\pgfsetlinewidth{1.003750pt}%
\definecolor{currentstroke}{rgb}{0.827451,0.827451,0.827451}%
\pgfsetstrokecolor{currentstroke}%
\pgfsetstrokeopacity{0.800000}%
\pgfsetdash{}{0pt}%
\pgfpathmoveto{\pgfqpoint{0.575099in}{3.779944in}}%
\pgfpathcurveto{\pgfqpoint{0.586149in}{3.779944in}}{\pgfqpoint{0.596748in}{3.784334in}}{\pgfqpoint{0.604561in}{3.792148in}}%
\pgfpathcurveto{\pgfqpoint{0.612375in}{3.799961in}}{\pgfqpoint{0.616765in}{3.810560in}}{\pgfqpoint{0.616765in}{3.821610in}}%
\pgfpathcurveto{\pgfqpoint{0.616765in}{3.832661in}}{\pgfqpoint{0.612375in}{3.843260in}}{\pgfqpoint{0.604561in}{3.851073in}}%
\pgfpathcurveto{\pgfqpoint{0.596748in}{3.858887in}}{\pgfqpoint{0.586149in}{3.863277in}}{\pgfqpoint{0.575099in}{3.863277in}}%
\pgfpathcurveto{\pgfqpoint{0.564048in}{3.863277in}}{\pgfqpoint{0.553449in}{3.858887in}}{\pgfqpoint{0.545636in}{3.851073in}}%
\pgfpathcurveto{\pgfqpoint{0.537822in}{3.843260in}}{\pgfqpoint{0.533432in}{3.832661in}}{\pgfqpoint{0.533432in}{3.821610in}}%
\pgfpathcurveto{\pgfqpoint{0.533432in}{3.810560in}}{\pgfqpoint{0.537822in}{3.799961in}}{\pgfqpoint{0.545636in}{3.792148in}}%
\pgfpathcurveto{\pgfqpoint{0.553449in}{3.784334in}}{\pgfqpoint{0.564048in}{3.779944in}}{\pgfqpoint{0.575099in}{3.779944in}}%
\pgfpathlineto{\pgfqpoint{0.575099in}{3.779944in}}%
\pgfpathclose%
\pgfusepath{stroke}%
\end{pgfscope}%
\begin{pgfscope}%
\pgfpathrectangle{\pgfqpoint{0.494722in}{0.437222in}}{\pgfqpoint{6.275590in}{5.159444in}}%
\pgfusepath{clip}%
\pgfsetbuttcap%
\pgfsetroundjoin%
\pgfsetlinewidth{1.003750pt}%
\definecolor{currentstroke}{rgb}{0.827451,0.827451,0.827451}%
\pgfsetstrokecolor{currentstroke}%
\pgfsetstrokeopacity{0.800000}%
\pgfsetdash{}{0pt}%
\pgfpathmoveto{\pgfqpoint{2.224297in}{1.325142in}}%
\pgfpathcurveto{\pgfqpoint{2.235347in}{1.325142in}}{\pgfqpoint{2.245946in}{1.329532in}}{\pgfqpoint{2.253760in}{1.337346in}}%
\pgfpathcurveto{\pgfqpoint{2.261573in}{1.345159in}}{\pgfqpoint{2.265964in}{1.355758in}}{\pgfqpoint{2.265964in}{1.366809in}}%
\pgfpathcurveto{\pgfqpoint{2.265964in}{1.377859in}}{\pgfqpoint{2.261573in}{1.388458in}}{\pgfqpoint{2.253760in}{1.396271in}}%
\pgfpathcurveto{\pgfqpoint{2.245946in}{1.404085in}}{\pgfqpoint{2.235347in}{1.408475in}}{\pgfqpoint{2.224297in}{1.408475in}}%
\pgfpathcurveto{\pgfqpoint{2.213247in}{1.408475in}}{\pgfqpoint{2.202648in}{1.404085in}}{\pgfqpoint{2.194834in}{1.396271in}}%
\pgfpathcurveto{\pgfqpoint{2.187020in}{1.388458in}}{\pgfqpoint{2.182630in}{1.377859in}}{\pgfqpoint{2.182630in}{1.366809in}}%
\pgfpathcurveto{\pgfqpoint{2.182630in}{1.355758in}}{\pgfqpoint{2.187020in}{1.345159in}}{\pgfqpoint{2.194834in}{1.337346in}}%
\pgfpathcurveto{\pgfqpoint{2.202648in}{1.329532in}}{\pgfqpoint{2.213247in}{1.325142in}}{\pgfqpoint{2.224297in}{1.325142in}}%
\pgfpathlineto{\pgfqpoint{2.224297in}{1.325142in}}%
\pgfpathclose%
\pgfusepath{stroke}%
\end{pgfscope}%
\begin{pgfscope}%
\pgfpathrectangle{\pgfqpoint{0.494722in}{0.437222in}}{\pgfqpoint{6.275590in}{5.159444in}}%
\pgfusepath{clip}%
\pgfsetbuttcap%
\pgfsetroundjoin%
\pgfsetlinewidth{1.003750pt}%
\definecolor{currentstroke}{rgb}{0.827451,0.827451,0.827451}%
\pgfsetstrokecolor{currentstroke}%
\pgfsetstrokeopacity{0.800000}%
\pgfsetdash{}{0pt}%
\pgfpathmoveto{\pgfqpoint{1.006510in}{2.581216in}}%
\pgfpathcurveto{\pgfqpoint{1.017560in}{2.581216in}}{\pgfqpoint{1.028159in}{2.585606in}}{\pgfqpoint{1.035973in}{2.593420in}}%
\pgfpathcurveto{\pgfqpoint{1.043786in}{2.601234in}}{\pgfqpoint{1.048177in}{2.611833in}}{\pgfqpoint{1.048177in}{2.622883in}}%
\pgfpathcurveto{\pgfqpoint{1.048177in}{2.633933in}}{\pgfqpoint{1.043786in}{2.644532in}}{\pgfqpoint{1.035973in}{2.652345in}}%
\pgfpathcurveto{\pgfqpoint{1.028159in}{2.660159in}}{\pgfqpoint{1.017560in}{2.664549in}}{\pgfqpoint{1.006510in}{2.664549in}}%
\pgfpathcurveto{\pgfqpoint{0.995460in}{2.664549in}}{\pgfqpoint{0.984861in}{2.660159in}}{\pgfqpoint{0.977047in}{2.652345in}}%
\pgfpathcurveto{\pgfqpoint{0.969234in}{2.644532in}}{\pgfqpoint{0.964843in}{2.633933in}}{\pgfqpoint{0.964843in}{2.622883in}}%
\pgfpathcurveto{\pgfqpoint{0.964843in}{2.611833in}}{\pgfqpoint{0.969234in}{2.601234in}}{\pgfqpoint{0.977047in}{2.593420in}}%
\pgfpathcurveto{\pgfqpoint{0.984861in}{2.585606in}}{\pgfqpoint{0.995460in}{2.581216in}}{\pgfqpoint{1.006510in}{2.581216in}}%
\pgfpathlineto{\pgfqpoint{1.006510in}{2.581216in}}%
\pgfpathclose%
\pgfusepath{stroke}%
\end{pgfscope}%
\begin{pgfscope}%
\pgfpathrectangle{\pgfqpoint{0.494722in}{0.437222in}}{\pgfqpoint{6.275590in}{5.159444in}}%
\pgfusepath{clip}%
\pgfsetbuttcap%
\pgfsetroundjoin%
\pgfsetlinewidth{1.003750pt}%
\definecolor{currentstroke}{rgb}{0.827451,0.827451,0.827451}%
\pgfsetstrokecolor{currentstroke}%
\pgfsetstrokeopacity{0.800000}%
\pgfsetdash{}{0pt}%
\pgfpathmoveto{\pgfqpoint{0.556707in}{3.885184in}}%
\pgfpathcurveto{\pgfqpoint{0.567757in}{3.885184in}}{\pgfqpoint{0.578356in}{3.889574in}}{\pgfqpoint{0.586170in}{3.897388in}}%
\pgfpathcurveto{\pgfqpoint{0.593983in}{3.905201in}}{\pgfqpoint{0.598374in}{3.915800in}}{\pgfqpoint{0.598374in}{3.926850in}}%
\pgfpathcurveto{\pgfqpoint{0.598374in}{3.937900in}}{\pgfqpoint{0.593983in}{3.948499in}}{\pgfqpoint{0.586170in}{3.956313in}}%
\pgfpathcurveto{\pgfqpoint{0.578356in}{3.964127in}}{\pgfqpoint{0.567757in}{3.968517in}}{\pgfqpoint{0.556707in}{3.968517in}}%
\pgfpathcurveto{\pgfqpoint{0.545657in}{3.968517in}}{\pgfqpoint{0.535058in}{3.964127in}}{\pgfqpoint{0.527244in}{3.956313in}}%
\pgfpathcurveto{\pgfqpoint{0.519430in}{3.948499in}}{\pgfqpoint{0.515040in}{3.937900in}}{\pgfqpoint{0.515040in}{3.926850in}}%
\pgfpathcurveto{\pgfqpoint{0.515040in}{3.915800in}}{\pgfqpoint{0.519430in}{3.905201in}}{\pgfqpoint{0.527244in}{3.897388in}}%
\pgfpathcurveto{\pgfqpoint{0.535058in}{3.889574in}}{\pgfqpoint{0.545657in}{3.885184in}}{\pgfqpoint{0.556707in}{3.885184in}}%
\pgfpathlineto{\pgfqpoint{0.556707in}{3.885184in}}%
\pgfpathclose%
\pgfusepath{stroke}%
\end{pgfscope}%
\begin{pgfscope}%
\pgfpathrectangle{\pgfqpoint{0.494722in}{0.437222in}}{\pgfqpoint{6.275590in}{5.159444in}}%
\pgfusepath{clip}%
\pgfsetbuttcap%
\pgfsetroundjoin%
\pgfsetlinewidth{1.003750pt}%
\definecolor{currentstroke}{rgb}{0.827451,0.827451,0.827451}%
\pgfsetstrokecolor{currentstroke}%
\pgfsetstrokeopacity{0.800000}%
\pgfsetdash{}{0pt}%
\pgfpathmoveto{\pgfqpoint{0.750297in}{3.133310in}}%
\pgfpathcurveto{\pgfqpoint{0.761348in}{3.133310in}}{\pgfqpoint{0.771947in}{3.137700in}}{\pgfqpoint{0.779760in}{3.145514in}}%
\pgfpathcurveto{\pgfqpoint{0.787574in}{3.153328in}}{\pgfqpoint{0.791964in}{3.163927in}}{\pgfqpoint{0.791964in}{3.174977in}}%
\pgfpathcurveto{\pgfqpoint{0.791964in}{3.186027in}}{\pgfqpoint{0.787574in}{3.196626in}}{\pgfqpoint{0.779760in}{3.204439in}}%
\pgfpathcurveto{\pgfqpoint{0.771947in}{3.212253in}}{\pgfqpoint{0.761348in}{3.216643in}}{\pgfqpoint{0.750297in}{3.216643in}}%
\pgfpathcurveto{\pgfqpoint{0.739247in}{3.216643in}}{\pgfqpoint{0.728648in}{3.212253in}}{\pgfqpoint{0.720835in}{3.204439in}}%
\pgfpathcurveto{\pgfqpoint{0.713021in}{3.196626in}}{\pgfqpoint{0.708631in}{3.186027in}}{\pgfqpoint{0.708631in}{3.174977in}}%
\pgfpathcurveto{\pgfqpoint{0.708631in}{3.163927in}}{\pgfqpoint{0.713021in}{3.153328in}}{\pgfqpoint{0.720835in}{3.145514in}}%
\pgfpathcurveto{\pgfqpoint{0.728648in}{3.137700in}}{\pgfqpoint{0.739247in}{3.133310in}}{\pgfqpoint{0.750297in}{3.133310in}}%
\pgfpathlineto{\pgfqpoint{0.750297in}{3.133310in}}%
\pgfpathclose%
\pgfusepath{stroke}%
\end{pgfscope}%
\begin{pgfscope}%
\pgfpathrectangle{\pgfqpoint{0.494722in}{0.437222in}}{\pgfqpoint{6.275590in}{5.159444in}}%
\pgfusepath{clip}%
\pgfsetbuttcap%
\pgfsetroundjoin%
\pgfsetlinewidth{1.003750pt}%
\definecolor{currentstroke}{rgb}{0.827451,0.827451,0.827451}%
\pgfsetstrokecolor{currentstroke}%
\pgfsetstrokeopacity{0.800000}%
\pgfsetdash{}{0pt}%
\pgfpathmoveto{\pgfqpoint{2.117156in}{1.399260in}}%
\pgfpathcurveto{\pgfqpoint{2.128206in}{1.399260in}}{\pgfqpoint{2.138805in}{1.403650in}}{\pgfqpoint{2.146618in}{1.411464in}}%
\pgfpathcurveto{\pgfqpoint{2.154432in}{1.419277in}}{\pgfqpoint{2.158822in}{1.429876in}}{\pgfqpoint{2.158822in}{1.440927in}}%
\pgfpathcurveto{\pgfqpoint{2.158822in}{1.451977in}}{\pgfqpoint{2.154432in}{1.462576in}}{\pgfqpoint{2.146618in}{1.470389in}}%
\pgfpathcurveto{\pgfqpoint{2.138805in}{1.478203in}}{\pgfqpoint{2.128206in}{1.482593in}}{\pgfqpoint{2.117156in}{1.482593in}}%
\pgfpathcurveto{\pgfqpoint{2.106106in}{1.482593in}}{\pgfqpoint{2.095506in}{1.478203in}}{\pgfqpoint{2.087693in}{1.470389in}}%
\pgfpathcurveto{\pgfqpoint{2.079879in}{1.462576in}}{\pgfqpoint{2.075489in}{1.451977in}}{\pgfqpoint{2.075489in}{1.440927in}}%
\pgfpathcurveto{\pgfqpoint{2.075489in}{1.429876in}}{\pgfqpoint{2.079879in}{1.419277in}}{\pgfqpoint{2.087693in}{1.411464in}}%
\pgfpathcurveto{\pgfqpoint{2.095506in}{1.403650in}}{\pgfqpoint{2.106106in}{1.399260in}}{\pgfqpoint{2.117156in}{1.399260in}}%
\pgfpathlineto{\pgfqpoint{2.117156in}{1.399260in}}%
\pgfpathclose%
\pgfusepath{stroke}%
\end{pgfscope}%
\begin{pgfscope}%
\pgfpathrectangle{\pgfqpoint{0.494722in}{0.437222in}}{\pgfqpoint{6.275590in}{5.159444in}}%
\pgfusepath{clip}%
\pgfsetbuttcap%
\pgfsetroundjoin%
\pgfsetlinewidth{1.003750pt}%
\definecolor{currentstroke}{rgb}{0.827451,0.827451,0.827451}%
\pgfsetstrokecolor{currentstroke}%
\pgfsetstrokeopacity{0.800000}%
\pgfsetdash{}{0pt}%
\pgfpathmoveto{\pgfqpoint{2.702598in}{1.043180in}}%
\pgfpathcurveto{\pgfqpoint{2.713648in}{1.043180in}}{\pgfqpoint{2.724247in}{1.047570in}}{\pgfqpoint{2.732061in}{1.055384in}}%
\pgfpathcurveto{\pgfqpoint{2.739874in}{1.063198in}}{\pgfqpoint{2.744265in}{1.073797in}}{\pgfqpoint{2.744265in}{1.084847in}}%
\pgfpathcurveto{\pgfqpoint{2.744265in}{1.095897in}}{\pgfqpoint{2.739874in}{1.106496in}}{\pgfqpoint{2.732061in}{1.114310in}}%
\pgfpathcurveto{\pgfqpoint{2.724247in}{1.122123in}}{\pgfqpoint{2.713648in}{1.126513in}}{\pgfqpoint{2.702598in}{1.126513in}}%
\pgfpathcurveto{\pgfqpoint{2.691548in}{1.126513in}}{\pgfqpoint{2.680949in}{1.122123in}}{\pgfqpoint{2.673135in}{1.114310in}}%
\pgfpathcurveto{\pgfqpoint{2.665321in}{1.106496in}}{\pgfqpoint{2.660931in}{1.095897in}}{\pgfqpoint{2.660931in}{1.084847in}}%
\pgfpathcurveto{\pgfqpoint{2.660931in}{1.073797in}}{\pgfqpoint{2.665321in}{1.063198in}}{\pgfqpoint{2.673135in}{1.055384in}}%
\pgfpathcurveto{\pgfqpoint{2.680949in}{1.047570in}}{\pgfqpoint{2.691548in}{1.043180in}}{\pgfqpoint{2.702598in}{1.043180in}}%
\pgfpathlineto{\pgfqpoint{2.702598in}{1.043180in}}%
\pgfpathclose%
\pgfusepath{stroke}%
\end{pgfscope}%
\begin{pgfscope}%
\pgfpathrectangle{\pgfqpoint{0.494722in}{0.437222in}}{\pgfqpoint{6.275590in}{5.159444in}}%
\pgfusepath{clip}%
\pgfsetbuttcap%
\pgfsetroundjoin%
\pgfsetlinewidth{1.003750pt}%
\definecolor{currentstroke}{rgb}{0.827451,0.827451,0.827451}%
\pgfsetstrokecolor{currentstroke}%
\pgfsetstrokeopacity{0.800000}%
\pgfsetdash{}{0pt}%
\pgfpathmoveto{\pgfqpoint{4.362526in}{0.500037in}}%
\pgfpathcurveto{\pgfqpoint{4.373576in}{0.500037in}}{\pgfqpoint{4.384175in}{0.504427in}}{\pgfqpoint{4.391989in}{0.512241in}}%
\pgfpathcurveto{\pgfqpoint{4.399802in}{0.520054in}}{\pgfqpoint{4.404193in}{0.530653in}}{\pgfqpoint{4.404193in}{0.541703in}}%
\pgfpathcurveto{\pgfqpoint{4.404193in}{0.552753in}}{\pgfqpoint{4.399802in}{0.563352in}}{\pgfqpoint{4.391989in}{0.571166in}}%
\pgfpathcurveto{\pgfqpoint{4.384175in}{0.578980in}}{\pgfqpoint{4.373576in}{0.583370in}}{\pgfqpoint{4.362526in}{0.583370in}}%
\pgfpathcurveto{\pgfqpoint{4.351476in}{0.583370in}}{\pgfqpoint{4.340877in}{0.578980in}}{\pgfqpoint{4.333063in}{0.571166in}}%
\pgfpathcurveto{\pgfqpoint{4.325250in}{0.563352in}}{\pgfqpoint{4.320859in}{0.552753in}}{\pgfqpoint{4.320859in}{0.541703in}}%
\pgfpathcurveto{\pgfqpoint{4.320859in}{0.530653in}}{\pgfqpoint{4.325250in}{0.520054in}}{\pgfqpoint{4.333063in}{0.512241in}}%
\pgfpathcurveto{\pgfqpoint{4.340877in}{0.504427in}}{\pgfqpoint{4.351476in}{0.500037in}}{\pgfqpoint{4.362526in}{0.500037in}}%
\pgfpathlineto{\pgfqpoint{4.362526in}{0.500037in}}%
\pgfpathclose%
\pgfusepath{stroke}%
\end{pgfscope}%
\begin{pgfscope}%
\pgfpathrectangle{\pgfqpoint{0.494722in}{0.437222in}}{\pgfqpoint{6.275590in}{5.159444in}}%
\pgfusepath{clip}%
\pgfsetbuttcap%
\pgfsetroundjoin%
\pgfsetlinewidth{1.003750pt}%
\definecolor{currentstroke}{rgb}{0.827451,0.827451,0.827451}%
\pgfsetstrokecolor{currentstroke}%
\pgfsetstrokeopacity{0.800000}%
\pgfsetdash{}{0pt}%
\pgfpathmoveto{\pgfqpoint{0.619224in}{3.610864in}}%
\pgfpathcurveto{\pgfqpoint{0.630274in}{3.610864in}}{\pgfqpoint{0.640874in}{3.615254in}}{\pgfqpoint{0.648687in}{3.623068in}}%
\pgfpathcurveto{\pgfqpoint{0.656501in}{3.630881in}}{\pgfqpoint{0.660891in}{3.641480in}}{\pgfqpoint{0.660891in}{3.652530in}}%
\pgfpathcurveto{\pgfqpoint{0.660891in}{3.663580in}}{\pgfqpoint{0.656501in}{3.674179in}}{\pgfqpoint{0.648687in}{3.681993in}}%
\pgfpathcurveto{\pgfqpoint{0.640874in}{3.689807in}}{\pgfqpoint{0.630274in}{3.694197in}}{\pgfqpoint{0.619224in}{3.694197in}}%
\pgfpathcurveto{\pgfqpoint{0.608174in}{3.694197in}}{\pgfqpoint{0.597575in}{3.689807in}}{\pgfqpoint{0.589762in}{3.681993in}}%
\pgfpathcurveto{\pgfqpoint{0.581948in}{3.674179in}}{\pgfqpoint{0.577558in}{3.663580in}}{\pgfqpoint{0.577558in}{3.652530in}}%
\pgfpathcurveto{\pgfqpoint{0.577558in}{3.641480in}}{\pgfqpoint{0.581948in}{3.630881in}}{\pgfqpoint{0.589762in}{3.623068in}}%
\pgfpathcurveto{\pgfqpoint{0.597575in}{3.615254in}}{\pgfqpoint{0.608174in}{3.610864in}}{\pgfqpoint{0.619224in}{3.610864in}}%
\pgfpathlineto{\pgfqpoint{0.619224in}{3.610864in}}%
\pgfpathclose%
\pgfusepath{stroke}%
\end{pgfscope}%
\begin{pgfscope}%
\pgfpathrectangle{\pgfqpoint{0.494722in}{0.437222in}}{\pgfqpoint{6.275590in}{5.159444in}}%
\pgfusepath{clip}%
\pgfsetbuttcap%
\pgfsetroundjoin%
\pgfsetlinewidth{1.003750pt}%
\definecolor{currentstroke}{rgb}{0.827451,0.827451,0.827451}%
\pgfsetstrokecolor{currentstroke}%
\pgfsetstrokeopacity{0.800000}%
\pgfsetdash{}{0pt}%
\pgfpathmoveto{\pgfqpoint{1.766183in}{1.687308in}}%
\pgfpathcurveto{\pgfqpoint{1.777233in}{1.687308in}}{\pgfqpoint{1.787832in}{1.691698in}}{\pgfqpoint{1.795646in}{1.699512in}}%
\pgfpathcurveto{\pgfqpoint{1.803459in}{1.707325in}}{\pgfqpoint{1.807850in}{1.717924in}}{\pgfqpoint{1.807850in}{1.728974in}}%
\pgfpathcurveto{\pgfqpoint{1.807850in}{1.740025in}}{\pgfqpoint{1.803459in}{1.750624in}}{\pgfqpoint{1.795646in}{1.758437in}}%
\pgfpathcurveto{\pgfqpoint{1.787832in}{1.766251in}}{\pgfqpoint{1.777233in}{1.770641in}}{\pgfqpoint{1.766183in}{1.770641in}}%
\pgfpathcurveto{\pgfqpoint{1.755133in}{1.770641in}}{\pgfqpoint{1.744534in}{1.766251in}}{\pgfqpoint{1.736720in}{1.758437in}}%
\pgfpathcurveto{\pgfqpoint{1.728907in}{1.750624in}}{\pgfqpoint{1.724516in}{1.740025in}}{\pgfqpoint{1.724516in}{1.728974in}}%
\pgfpathcurveto{\pgfqpoint{1.724516in}{1.717924in}}{\pgfqpoint{1.728907in}{1.707325in}}{\pgfqpoint{1.736720in}{1.699512in}}%
\pgfpathcurveto{\pgfqpoint{1.744534in}{1.691698in}}{\pgfqpoint{1.755133in}{1.687308in}}{\pgfqpoint{1.766183in}{1.687308in}}%
\pgfpathlineto{\pgfqpoint{1.766183in}{1.687308in}}%
\pgfpathclose%
\pgfusepath{stroke}%
\end{pgfscope}%
\begin{pgfscope}%
\pgfpathrectangle{\pgfqpoint{0.494722in}{0.437222in}}{\pgfqpoint{6.275590in}{5.159444in}}%
\pgfusepath{clip}%
\pgfsetbuttcap%
\pgfsetroundjoin%
\pgfsetlinewidth{1.003750pt}%
\definecolor{currentstroke}{rgb}{0.827451,0.827451,0.827451}%
\pgfsetstrokecolor{currentstroke}%
\pgfsetstrokeopacity{0.800000}%
\pgfsetdash{}{0pt}%
\pgfpathmoveto{\pgfqpoint{0.608606in}{3.646796in}}%
\pgfpathcurveto{\pgfqpoint{0.619657in}{3.646796in}}{\pgfqpoint{0.630256in}{3.651186in}}{\pgfqpoint{0.638069in}{3.659000in}}%
\pgfpathcurveto{\pgfqpoint{0.645883in}{3.666813in}}{\pgfqpoint{0.650273in}{3.677412in}}{\pgfqpoint{0.650273in}{3.688462in}}%
\pgfpathcurveto{\pgfqpoint{0.650273in}{3.699512in}}{\pgfqpoint{0.645883in}{3.710112in}}{\pgfqpoint{0.638069in}{3.717925in}}%
\pgfpathcurveto{\pgfqpoint{0.630256in}{3.725739in}}{\pgfqpoint{0.619657in}{3.730129in}}{\pgfqpoint{0.608606in}{3.730129in}}%
\pgfpathcurveto{\pgfqpoint{0.597556in}{3.730129in}}{\pgfqpoint{0.586957in}{3.725739in}}{\pgfqpoint{0.579144in}{3.717925in}}%
\pgfpathcurveto{\pgfqpoint{0.571330in}{3.710112in}}{\pgfqpoint{0.566940in}{3.699512in}}{\pgfqpoint{0.566940in}{3.688462in}}%
\pgfpathcurveto{\pgfqpoint{0.566940in}{3.677412in}}{\pgfqpoint{0.571330in}{3.666813in}}{\pgfqpoint{0.579144in}{3.659000in}}%
\pgfpathcurveto{\pgfqpoint{0.586957in}{3.651186in}}{\pgfqpoint{0.597556in}{3.646796in}}{\pgfqpoint{0.608606in}{3.646796in}}%
\pgfpathlineto{\pgfqpoint{0.608606in}{3.646796in}}%
\pgfpathclose%
\pgfusepath{stroke}%
\end{pgfscope}%
\begin{pgfscope}%
\pgfpathrectangle{\pgfqpoint{0.494722in}{0.437222in}}{\pgfqpoint{6.275590in}{5.159444in}}%
\pgfusepath{clip}%
\pgfsetbuttcap%
\pgfsetroundjoin%
\pgfsetlinewidth{1.003750pt}%
\definecolor{currentstroke}{rgb}{0.827451,0.827451,0.827451}%
\pgfsetstrokecolor{currentstroke}%
\pgfsetstrokeopacity{0.800000}%
\pgfsetdash{}{0pt}%
\pgfpathmoveto{\pgfqpoint{1.444344in}{1.986059in}}%
\pgfpathcurveto{\pgfqpoint{1.455394in}{1.986059in}}{\pgfqpoint{1.465993in}{1.990450in}}{\pgfqpoint{1.473807in}{1.998263in}}%
\pgfpathcurveto{\pgfqpoint{1.481620in}{2.006077in}}{\pgfqpoint{1.486010in}{2.016676in}}{\pgfqpoint{1.486010in}{2.027726in}}%
\pgfpathcurveto{\pgfqpoint{1.486010in}{2.038776in}}{\pgfqpoint{1.481620in}{2.049375in}}{\pgfqpoint{1.473807in}{2.057189in}}%
\pgfpathcurveto{\pgfqpoint{1.465993in}{2.065002in}}{\pgfqpoint{1.455394in}{2.069393in}}{\pgfqpoint{1.444344in}{2.069393in}}%
\pgfpathcurveto{\pgfqpoint{1.433294in}{2.069393in}}{\pgfqpoint{1.422695in}{2.065002in}}{\pgfqpoint{1.414881in}{2.057189in}}%
\pgfpathcurveto{\pgfqpoint{1.407067in}{2.049375in}}{\pgfqpoint{1.402677in}{2.038776in}}{\pgfqpoint{1.402677in}{2.027726in}}%
\pgfpathcurveto{\pgfqpoint{1.402677in}{2.016676in}}{\pgfqpoint{1.407067in}{2.006077in}}{\pgfqpoint{1.414881in}{1.998263in}}%
\pgfpathcurveto{\pgfqpoint{1.422695in}{1.990450in}}{\pgfqpoint{1.433294in}{1.986059in}}{\pgfqpoint{1.444344in}{1.986059in}}%
\pgfpathlineto{\pgfqpoint{1.444344in}{1.986059in}}%
\pgfpathclose%
\pgfusepath{stroke}%
\end{pgfscope}%
\begin{pgfscope}%
\pgfpathrectangle{\pgfqpoint{0.494722in}{0.437222in}}{\pgfqpoint{6.275590in}{5.159444in}}%
\pgfusepath{clip}%
\pgfsetbuttcap%
\pgfsetroundjoin%
\pgfsetlinewidth{1.003750pt}%
\definecolor{currentstroke}{rgb}{0.827451,0.827451,0.827451}%
\pgfsetstrokecolor{currentstroke}%
\pgfsetstrokeopacity{0.800000}%
\pgfsetdash{}{0pt}%
\pgfpathmoveto{\pgfqpoint{0.744475in}{3.199365in}}%
\pgfpathcurveto{\pgfqpoint{0.755525in}{3.199365in}}{\pgfqpoint{0.766124in}{3.203755in}}{\pgfqpoint{0.773938in}{3.211569in}}%
\pgfpathcurveto{\pgfqpoint{0.781751in}{3.219382in}}{\pgfqpoint{0.786142in}{3.229981in}}{\pgfqpoint{0.786142in}{3.241032in}}%
\pgfpathcurveto{\pgfqpoint{0.786142in}{3.252082in}}{\pgfqpoint{0.781751in}{3.262681in}}{\pgfqpoint{0.773938in}{3.270494in}}%
\pgfpathcurveto{\pgfqpoint{0.766124in}{3.278308in}}{\pgfqpoint{0.755525in}{3.282698in}}{\pgfqpoint{0.744475in}{3.282698in}}%
\pgfpathcurveto{\pgfqpoint{0.733425in}{3.282698in}}{\pgfqpoint{0.722826in}{3.278308in}}{\pgfqpoint{0.715012in}{3.270494in}}%
\pgfpathcurveto{\pgfqpoint{0.707199in}{3.262681in}}{\pgfqpoint{0.702808in}{3.252082in}}{\pgfqpoint{0.702808in}{3.241032in}}%
\pgfpathcurveto{\pgfqpoint{0.702808in}{3.229981in}}{\pgfqpoint{0.707199in}{3.219382in}}{\pgfqpoint{0.715012in}{3.211569in}}%
\pgfpathcurveto{\pgfqpoint{0.722826in}{3.203755in}}{\pgfqpoint{0.733425in}{3.199365in}}{\pgfqpoint{0.744475in}{3.199365in}}%
\pgfpathlineto{\pgfqpoint{0.744475in}{3.199365in}}%
\pgfpathclose%
\pgfusepath{stroke}%
\end{pgfscope}%
\begin{pgfscope}%
\pgfpathrectangle{\pgfqpoint{0.494722in}{0.437222in}}{\pgfqpoint{6.275590in}{5.159444in}}%
\pgfusepath{clip}%
\pgfsetbuttcap%
\pgfsetroundjoin%
\pgfsetlinewidth{1.003750pt}%
\definecolor{currentstroke}{rgb}{0.827451,0.827451,0.827451}%
\pgfsetstrokecolor{currentstroke}%
\pgfsetstrokeopacity{0.800000}%
\pgfsetdash{}{0pt}%
\pgfpathmoveto{\pgfqpoint{1.926347in}{1.574489in}}%
\pgfpathcurveto{\pgfqpoint{1.937397in}{1.574489in}}{\pgfqpoint{1.947996in}{1.578879in}}{\pgfqpoint{1.955809in}{1.586693in}}%
\pgfpathcurveto{\pgfqpoint{1.963623in}{1.594506in}}{\pgfqpoint{1.968013in}{1.605105in}}{\pgfqpoint{1.968013in}{1.616155in}}%
\pgfpathcurveto{\pgfqpoint{1.968013in}{1.627206in}}{\pgfqpoint{1.963623in}{1.637805in}}{\pgfqpoint{1.955809in}{1.645618in}}%
\pgfpathcurveto{\pgfqpoint{1.947996in}{1.653432in}}{\pgfqpoint{1.937397in}{1.657822in}}{\pgfqpoint{1.926347in}{1.657822in}}%
\pgfpathcurveto{\pgfqpoint{1.915296in}{1.657822in}}{\pgfqpoint{1.904697in}{1.653432in}}{\pgfqpoint{1.896884in}{1.645618in}}%
\pgfpathcurveto{\pgfqpoint{1.889070in}{1.637805in}}{\pgfqpoint{1.884680in}{1.627206in}}{\pgfqpoint{1.884680in}{1.616155in}}%
\pgfpathcurveto{\pgfqpoint{1.884680in}{1.605105in}}{\pgfqpoint{1.889070in}{1.594506in}}{\pgfqpoint{1.896884in}{1.586693in}}%
\pgfpathcurveto{\pgfqpoint{1.904697in}{1.578879in}}{\pgfqpoint{1.915296in}{1.574489in}}{\pgfqpoint{1.926347in}{1.574489in}}%
\pgfpathlineto{\pgfqpoint{1.926347in}{1.574489in}}%
\pgfpathclose%
\pgfusepath{stroke}%
\end{pgfscope}%
\begin{pgfscope}%
\pgfpathrectangle{\pgfqpoint{0.494722in}{0.437222in}}{\pgfqpoint{6.275590in}{5.159444in}}%
\pgfusepath{clip}%
\pgfsetbuttcap%
\pgfsetroundjoin%
\pgfsetlinewidth{1.003750pt}%
\definecolor{currentstroke}{rgb}{0.827451,0.827451,0.827451}%
\pgfsetstrokecolor{currentstroke}%
\pgfsetstrokeopacity{0.800000}%
\pgfsetdash{}{0pt}%
\pgfpathmoveto{\pgfqpoint{2.183447in}{1.371651in}}%
\pgfpathcurveto{\pgfqpoint{2.194497in}{1.371651in}}{\pgfqpoint{2.205096in}{1.376041in}}{\pgfqpoint{2.212910in}{1.383855in}}%
\pgfpathcurveto{\pgfqpoint{2.220724in}{1.391668in}}{\pgfqpoint{2.225114in}{1.402267in}}{\pgfqpoint{2.225114in}{1.413318in}}%
\pgfpathcurveto{\pgfqpoint{2.225114in}{1.424368in}}{\pgfqpoint{2.220724in}{1.434967in}}{\pgfqpoint{2.212910in}{1.442780in}}%
\pgfpathcurveto{\pgfqpoint{2.205096in}{1.450594in}}{\pgfqpoint{2.194497in}{1.454984in}}{\pgfqpoint{2.183447in}{1.454984in}}%
\pgfpathcurveto{\pgfqpoint{2.172397in}{1.454984in}}{\pgfqpoint{2.161798in}{1.450594in}}{\pgfqpoint{2.153984in}{1.442780in}}%
\pgfpathcurveto{\pgfqpoint{2.146171in}{1.434967in}}{\pgfqpoint{2.141781in}{1.424368in}}{\pgfqpoint{2.141781in}{1.413318in}}%
\pgfpathcurveto{\pgfqpoint{2.141781in}{1.402267in}}{\pgfqpoint{2.146171in}{1.391668in}}{\pgfqpoint{2.153984in}{1.383855in}}%
\pgfpathcurveto{\pgfqpoint{2.161798in}{1.376041in}}{\pgfqpoint{2.172397in}{1.371651in}}{\pgfqpoint{2.183447in}{1.371651in}}%
\pgfpathlineto{\pgfqpoint{2.183447in}{1.371651in}}%
\pgfpathclose%
\pgfusepath{stroke}%
\end{pgfscope}%
\begin{pgfscope}%
\pgfpathrectangle{\pgfqpoint{0.494722in}{0.437222in}}{\pgfqpoint{6.275590in}{5.159444in}}%
\pgfusepath{clip}%
\pgfsetbuttcap%
\pgfsetroundjoin%
\pgfsetlinewidth{1.003750pt}%
\definecolor{currentstroke}{rgb}{0.827451,0.827451,0.827451}%
\pgfsetstrokecolor{currentstroke}%
\pgfsetstrokeopacity{0.800000}%
\pgfsetdash{}{0pt}%
\pgfpathmoveto{\pgfqpoint{3.321971in}{0.791856in}}%
\pgfpathcurveto{\pgfqpoint{3.333021in}{0.791856in}}{\pgfqpoint{3.343620in}{0.796246in}}{\pgfqpoint{3.351433in}{0.804060in}}%
\pgfpathcurveto{\pgfqpoint{3.359247in}{0.811873in}}{\pgfqpoint{3.363637in}{0.822472in}}{\pgfqpoint{3.363637in}{0.833523in}}%
\pgfpathcurveto{\pgfqpoint{3.363637in}{0.844573in}}{\pgfqpoint{3.359247in}{0.855172in}}{\pgfqpoint{3.351433in}{0.862985in}}%
\pgfpathcurveto{\pgfqpoint{3.343620in}{0.870799in}}{\pgfqpoint{3.333021in}{0.875189in}}{\pgfqpoint{3.321971in}{0.875189in}}%
\pgfpathcurveto{\pgfqpoint{3.310921in}{0.875189in}}{\pgfqpoint{3.300321in}{0.870799in}}{\pgfqpoint{3.292508in}{0.862985in}}%
\pgfpathcurveto{\pgfqpoint{3.284694in}{0.855172in}}{\pgfqpoint{3.280304in}{0.844573in}}{\pgfqpoint{3.280304in}{0.833523in}}%
\pgfpathcurveto{\pgfqpoint{3.280304in}{0.822472in}}{\pgfqpoint{3.284694in}{0.811873in}}{\pgfqpoint{3.292508in}{0.804060in}}%
\pgfpathcurveto{\pgfqpoint{3.300321in}{0.796246in}}{\pgfqpoint{3.310921in}{0.791856in}}{\pgfqpoint{3.321971in}{0.791856in}}%
\pgfpathlineto{\pgfqpoint{3.321971in}{0.791856in}}%
\pgfpathclose%
\pgfusepath{stroke}%
\end{pgfscope}%
\begin{pgfscope}%
\pgfpathrectangle{\pgfqpoint{0.494722in}{0.437222in}}{\pgfqpoint{6.275590in}{5.159444in}}%
\pgfusepath{clip}%
\pgfsetbuttcap%
\pgfsetroundjoin%
\pgfsetlinewidth{1.003750pt}%
\definecolor{currentstroke}{rgb}{0.827451,0.827451,0.827451}%
\pgfsetstrokecolor{currentstroke}%
\pgfsetstrokeopacity{0.800000}%
\pgfsetdash{}{0pt}%
\pgfpathmoveto{\pgfqpoint{0.584355in}{3.716646in}}%
\pgfpathcurveto{\pgfqpoint{0.595406in}{3.716646in}}{\pgfqpoint{0.606005in}{3.721036in}}{\pgfqpoint{0.613818in}{3.728850in}}%
\pgfpathcurveto{\pgfqpoint{0.621632in}{3.736663in}}{\pgfqpoint{0.626022in}{3.747262in}}{\pgfqpoint{0.626022in}{3.758313in}}%
\pgfpathcurveto{\pgfqpoint{0.626022in}{3.769363in}}{\pgfqpoint{0.621632in}{3.779962in}}{\pgfqpoint{0.613818in}{3.787775in}}%
\pgfpathcurveto{\pgfqpoint{0.606005in}{3.795589in}}{\pgfqpoint{0.595406in}{3.799979in}}{\pgfqpoint{0.584355in}{3.799979in}}%
\pgfpathcurveto{\pgfqpoint{0.573305in}{3.799979in}}{\pgfqpoint{0.562706in}{3.795589in}}{\pgfqpoint{0.554893in}{3.787775in}}%
\pgfpathcurveto{\pgfqpoint{0.547079in}{3.779962in}}{\pgfqpoint{0.542689in}{3.769363in}}{\pgfqpoint{0.542689in}{3.758313in}}%
\pgfpathcurveto{\pgfqpoint{0.542689in}{3.747262in}}{\pgfqpoint{0.547079in}{3.736663in}}{\pgfqpoint{0.554893in}{3.728850in}}%
\pgfpathcurveto{\pgfqpoint{0.562706in}{3.721036in}}{\pgfqpoint{0.573305in}{3.716646in}}{\pgfqpoint{0.584355in}{3.716646in}}%
\pgfpathlineto{\pgfqpoint{0.584355in}{3.716646in}}%
\pgfpathclose%
\pgfusepath{stroke}%
\end{pgfscope}%
\begin{pgfscope}%
\pgfpathrectangle{\pgfqpoint{0.494722in}{0.437222in}}{\pgfqpoint{6.275590in}{5.159444in}}%
\pgfusepath{clip}%
\pgfsetbuttcap%
\pgfsetroundjoin%
\pgfsetlinewidth{1.003750pt}%
\definecolor{currentstroke}{rgb}{0.827451,0.827451,0.827451}%
\pgfsetstrokecolor{currentstroke}%
\pgfsetstrokeopacity{0.800000}%
\pgfsetdash{}{0pt}%
\pgfpathmoveto{\pgfqpoint{1.786945in}{1.657846in}}%
\pgfpathcurveto{\pgfqpoint{1.797995in}{1.657846in}}{\pgfqpoint{1.808594in}{1.662237in}}{\pgfqpoint{1.816407in}{1.670050in}}%
\pgfpathcurveto{\pgfqpoint{1.824221in}{1.677864in}}{\pgfqpoint{1.828611in}{1.688463in}}{\pgfqpoint{1.828611in}{1.699513in}}%
\pgfpathcurveto{\pgfqpoint{1.828611in}{1.710563in}}{\pgfqpoint{1.824221in}{1.721162in}}{\pgfqpoint{1.816407in}{1.728976in}}%
\pgfpathcurveto{\pgfqpoint{1.808594in}{1.736789in}}{\pgfqpoint{1.797995in}{1.741180in}}{\pgfqpoint{1.786945in}{1.741180in}}%
\pgfpathcurveto{\pgfqpoint{1.775894in}{1.741180in}}{\pgfqpoint{1.765295in}{1.736789in}}{\pgfqpoint{1.757482in}{1.728976in}}%
\pgfpathcurveto{\pgfqpoint{1.749668in}{1.721162in}}{\pgfqpoint{1.745278in}{1.710563in}}{\pgfqpoint{1.745278in}{1.699513in}}%
\pgfpathcurveto{\pgfqpoint{1.745278in}{1.688463in}}{\pgfqpoint{1.749668in}{1.677864in}}{\pgfqpoint{1.757482in}{1.670050in}}%
\pgfpathcurveto{\pgfqpoint{1.765295in}{1.662237in}}{\pgfqpoint{1.775894in}{1.657846in}}{\pgfqpoint{1.786945in}{1.657846in}}%
\pgfpathlineto{\pgfqpoint{1.786945in}{1.657846in}}%
\pgfpathclose%
\pgfusepath{stroke}%
\end{pgfscope}%
\begin{pgfscope}%
\pgfpathrectangle{\pgfqpoint{0.494722in}{0.437222in}}{\pgfqpoint{6.275590in}{5.159444in}}%
\pgfusepath{clip}%
\pgfsetbuttcap%
\pgfsetroundjoin%
\pgfsetlinewidth{1.003750pt}%
\definecolor{currentstroke}{rgb}{0.827451,0.827451,0.827451}%
\pgfsetstrokecolor{currentstroke}%
\pgfsetstrokeopacity{0.800000}%
\pgfsetdash{}{0pt}%
\pgfpathmoveto{\pgfqpoint{2.867283in}{0.963868in}}%
\pgfpathcurveto{\pgfqpoint{2.878333in}{0.963868in}}{\pgfqpoint{2.888932in}{0.968258in}}{\pgfqpoint{2.896746in}{0.976072in}}%
\pgfpathcurveto{\pgfqpoint{2.904560in}{0.983885in}}{\pgfqpoint{2.908950in}{0.994484in}}{\pgfqpoint{2.908950in}{1.005535in}}%
\pgfpathcurveto{\pgfqpoint{2.908950in}{1.016585in}}{\pgfqpoint{2.904560in}{1.027184in}}{\pgfqpoint{2.896746in}{1.034997in}}%
\pgfpathcurveto{\pgfqpoint{2.888932in}{1.042811in}}{\pgfqpoint{2.878333in}{1.047201in}}{\pgfqpoint{2.867283in}{1.047201in}}%
\pgfpathcurveto{\pgfqpoint{2.856233in}{1.047201in}}{\pgfqpoint{2.845634in}{1.042811in}}{\pgfqpoint{2.837820in}{1.034997in}}%
\pgfpathcurveto{\pgfqpoint{2.830007in}{1.027184in}}{\pgfqpoint{2.825617in}{1.016585in}}{\pgfqpoint{2.825617in}{1.005535in}}%
\pgfpathcurveto{\pgfqpoint{2.825617in}{0.994484in}}{\pgfqpoint{2.830007in}{0.983885in}}{\pgfqpoint{2.837820in}{0.976072in}}%
\pgfpathcurveto{\pgfqpoint{2.845634in}{0.968258in}}{\pgfqpoint{2.856233in}{0.963868in}}{\pgfqpoint{2.867283in}{0.963868in}}%
\pgfpathlineto{\pgfqpoint{2.867283in}{0.963868in}}%
\pgfpathclose%
\pgfusepath{stroke}%
\end{pgfscope}%
\begin{pgfscope}%
\pgfpathrectangle{\pgfqpoint{0.494722in}{0.437222in}}{\pgfqpoint{6.275590in}{5.159444in}}%
\pgfusepath{clip}%
\pgfsetbuttcap%
\pgfsetroundjoin%
\pgfsetlinewidth{1.003750pt}%
\definecolor{currentstroke}{rgb}{0.827451,0.827451,0.827451}%
\pgfsetstrokecolor{currentstroke}%
\pgfsetstrokeopacity{0.800000}%
\pgfsetdash{}{0pt}%
\pgfpathmoveto{\pgfqpoint{1.581391in}{1.870889in}}%
\pgfpathcurveto{\pgfqpoint{1.592441in}{1.870889in}}{\pgfqpoint{1.603040in}{1.875279in}}{\pgfqpoint{1.610854in}{1.883093in}}%
\pgfpathcurveto{\pgfqpoint{1.618667in}{1.890906in}}{\pgfqpoint{1.623058in}{1.901505in}}{\pgfqpoint{1.623058in}{1.912556in}}%
\pgfpathcurveto{\pgfqpoint{1.623058in}{1.923606in}}{\pgfqpoint{1.618667in}{1.934205in}}{\pgfqpoint{1.610854in}{1.942018in}}%
\pgfpathcurveto{\pgfqpoint{1.603040in}{1.949832in}}{\pgfqpoint{1.592441in}{1.954222in}}{\pgfqpoint{1.581391in}{1.954222in}}%
\pgfpathcurveto{\pgfqpoint{1.570341in}{1.954222in}}{\pgfqpoint{1.559742in}{1.949832in}}{\pgfqpoint{1.551928in}{1.942018in}}%
\pgfpathcurveto{\pgfqpoint{1.544115in}{1.934205in}}{\pgfqpoint{1.539724in}{1.923606in}}{\pgfqpoint{1.539724in}{1.912556in}}%
\pgfpathcurveto{\pgfqpoint{1.539724in}{1.901505in}}{\pgfqpoint{1.544115in}{1.890906in}}{\pgfqpoint{1.551928in}{1.883093in}}%
\pgfpathcurveto{\pgfqpoint{1.559742in}{1.875279in}}{\pgfqpoint{1.570341in}{1.870889in}}{\pgfqpoint{1.581391in}{1.870889in}}%
\pgfpathlineto{\pgfqpoint{1.581391in}{1.870889in}}%
\pgfpathclose%
\pgfusepath{stroke}%
\end{pgfscope}%
\begin{pgfscope}%
\pgfpathrectangle{\pgfqpoint{0.494722in}{0.437222in}}{\pgfqpoint{6.275590in}{5.159444in}}%
\pgfusepath{clip}%
\pgfsetbuttcap%
\pgfsetroundjoin%
\pgfsetlinewidth{1.003750pt}%
\definecolor{currentstroke}{rgb}{0.827451,0.827451,0.827451}%
\pgfsetstrokecolor{currentstroke}%
\pgfsetstrokeopacity{0.800000}%
\pgfsetdash{}{0pt}%
\pgfpathmoveto{\pgfqpoint{1.623472in}{1.825964in}}%
\pgfpathcurveto{\pgfqpoint{1.634522in}{1.825964in}}{\pgfqpoint{1.645121in}{1.830354in}}{\pgfqpoint{1.652935in}{1.838167in}}%
\pgfpathcurveto{\pgfqpoint{1.660748in}{1.845981in}}{\pgfqpoint{1.665139in}{1.856580in}}{\pgfqpoint{1.665139in}{1.867630in}}%
\pgfpathcurveto{\pgfqpoint{1.665139in}{1.878680in}}{\pgfqpoint{1.660748in}{1.889279in}}{\pgfqpoint{1.652935in}{1.897093in}}%
\pgfpathcurveto{\pgfqpoint{1.645121in}{1.904907in}}{\pgfqpoint{1.634522in}{1.909297in}}{\pgfqpoint{1.623472in}{1.909297in}}%
\pgfpathcurveto{\pgfqpoint{1.612422in}{1.909297in}}{\pgfqpoint{1.601823in}{1.904907in}}{\pgfqpoint{1.594009in}{1.897093in}}%
\pgfpathcurveto{\pgfqpoint{1.586196in}{1.889279in}}{\pgfqpoint{1.581805in}{1.878680in}}{\pgfqpoint{1.581805in}{1.867630in}}%
\pgfpathcurveto{\pgfqpoint{1.581805in}{1.856580in}}{\pgfqpoint{1.586196in}{1.845981in}}{\pgfqpoint{1.594009in}{1.838167in}}%
\pgfpathcurveto{\pgfqpoint{1.601823in}{1.830354in}}{\pgfqpoint{1.612422in}{1.825964in}}{\pgfqpoint{1.623472in}{1.825964in}}%
\pgfpathlineto{\pgfqpoint{1.623472in}{1.825964in}}%
\pgfpathclose%
\pgfusepath{stroke}%
\end{pgfscope}%
\begin{pgfscope}%
\pgfpathrectangle{\pgfqpoint{0.494722in}{0.437222in}}{\pgfqpoint{6.275590in}{5.159444in}}%
\pgfusepath{clip}%
\pgfsetbuttcap%
\pgfsetroundjoin%
\pgfsetlinewidth{1.003750pt}%
\definecolor{currentstroke}{rgb}{0.827451,0.827451,0.827451}%
\pgfsetstrokecolor{currentstroke}%
\pgfsetstrokeopacity{0.800000}%
\pgfsetdash{}{0pt}%
\pgfpathmoveto{\pgfqpoint{0.637358in}{3.486654in}}%
\pgfpathcurveto{\pgfqpoint{0.648408in}{3.486654in}}{\pgfqpoint{0.659007in}{3.491045in}}{\pgfqpoint{0.666820in}{3.498858in}}%
\pgfpathcurveto{\pgfqpoint{0.674634in}{3.506672in}}{\pgfqpoint{0.679024in}{3.517271in}}{\pgfqpoint{0.679024in}{3.528321in}}%
\pgfpathcurveto{\pgfqpoint{0.679024in}{3.539371in}}{\pgfqpoint{0.674634in}{3.549970in}}{\pgfqpoint{0.666820in}{3.557784in}}%
\pgfpathcurveto{\pgfqpoint{0.659007in}{3.565597in}}{\pgfqpoint{0.648408in}{3.569988in}}{\pgfqpoint{0.637358in}{3.569988in}}%
\pgfpathcurveto{\pgfqpoint{0.626307in}{3.569988in}}{\pgfqpoint{0.615708in}{3.565597in}}{\pgfqpoint{0.607895in}{3.557784in}}%
\pgfpathcurveto{\pgfqpoint{0.600081in}{3.549970in}}{\pgfqpoint{0.595691in}{3.539371in}}{\pgfqpoint{0.595691in}{3.528321in}}%
\pgfpathcurveto{\pgfqpoint{0.595691in}{3.517271in}}{\pgfqpoint{0.600081in}{3.506672in}}{\pgfqpoint{0.607895in}{3.498858in}}%
\pgfpathcurveto{\pgfqpoint{0.615708in}{3.491045in}}{\pgfqpoint{0.626307in}{3.486654in}}{\pgfqpoint{0.637358in}{3.486654in}}%
\pgfpathlineto{\pgfqpoint{0.637358in}{3.486654in}}%
\pgfpathclose%
\pgfusepath{stroke}%
\end{pgfscope}%
\begin{pgfscope}%
\pgfpathrectangle{\pgfqpoint{0.494722in}{0.437222in}}{\pgfqpoint{6.275590in}{5.159444in}}%
\pgfusepath{clip}%
\pgfsetbuttcap%
\pgfsetroundjoin%
\pgfsetlinewidth{1.003750pt}%
\definecolor{currentstroke}{rgb}{0.827451,0.827451,0.827451}%
\pgfsetstrokecolor{currentstroke}%
\pgfsetstrokeopacity{0.800000}%
\pgfsetdash{}{0pt}%
\pgfpathmoveto{\pgfqpoint{1.663088in}{1.812477in}}%
\pgfpathcurveto{\pgfqpoint{1.674138in}{1.812477in}}{\pgfqpoint{1.684737in}{1.816868in}}{\pgfqpoint{1.692551in}{1.824681in}}%
\pgfpathcurveto{\pgfqpoint{1.700364in}{1.832495in}}{\pgfqpoint{1.704755in}{1.843094in}}{\pgfqpoint{1.704755in}{1.854144in}}%
\pgfpathcurveto{\pgfqpoint{1.704755in}{1.865194in}}{\pgfqpoint{1.700364in}{1.875793in}}{\pgfqpoint{1.692551in}{1.883607in}}%
\pgfpathcurveto{\pgfqpoint{1.684737in}{1.891421in}}{\pgfqpoint{1.674138in}{1.895811in}}{\pgfqpoint{1.663088in}{1.895811in}}%
\pgfpathcurveto{\pgfqpoint{1.652038in}{1.895811in}}{\pgfqpoint{1.641439in}{1.891421in}}{\pgfqpoint{1.633625in}{1.883607in}}%
\pgfpathcurveto{\pgfqpoint{1.625812in}{1.875793in}}{\pgfqpoint{1.621421in}{1.865194in}}{\pgfqpoint{1.621421in}{1.854144in}}%
\pgfpathcurveto{\pgfqpoint{1.621421in}{1.843094in}}{\pgfqpoint{1.625812in}{1.832495in}}{\pgfqpoint{1.633625in}{1.824681in}}%
\pgfpathcurveto{\pgfqpoint{1.641439in}{1.816868in}}{\pgfqpoint{1.652038in}{1.812477in}}{\pgfqpoint{1.663088in}{1.812477in}}%
\pgfpathlineto{\pgfqpoint{1.663088in}{1.812477in}}%
\pgfpathclose%
\pgfusepath{stroke}%
\end{pgfscope}%
\begin{pgfscope}%
\pgfpathrectangle{\pgfqpoint{0.494722in}{0.437222in}}{\pgfqpoint{6.275590in}{5.159444in}}%
\pgfusepath{clip}%
\pgfsetbuttcap%
\pgfsetroundjoin%
\pgfsetlinewidth{1.003750pt}%
\definecolor{currentstroke}{rgb}{0.827451,0.827451,0.827451}%
\pgfsetstrokecolor{currentstroke}%
\pgfsetstrokeopacity{0.800000}%
\pgfsetdash{}{0pt}%
\pgfpathmoveto{\pgfqpoint{1.396743in}{2.042450in}}%
\pgfpathcurveto{\pgfqpoint{1.407793in}{2.042450in}}{\pgfqpoint{1.418392in}{2.046840in}}{\pgfqpoint{1.426206in}{2.054654in}}%
\pgfpathcurveto{\pgfqpoint{1.434019in}{2.062467in}}{\pgfqpoint{1.438409in}{2.073066in}}{\pgfqpoint{1.438409in}{2.084117in}}%
\pgfpathcurveto{\pgfqpoint{1.438409in}{2.095167in}}{\pgfqpoint{1.434019in}{2.105766in}}{\pgfqpoint{1.426206in}{2.113579in}}%
\pgfpathcurveto{\pgfqpoint{1.418392in}{2.121393in}}{\pgfqpoint{1.407793in}{2.125783in}}{\pgfqpoint{1.396743in}{2.125783in}}%
\pgfpathcurveto{\pgfqpoint{1.385693in}{2.125783in}}{\pgfqpoint{1.375094in}{2.121393in}}{\pgfqpoint{1.367280in}{2.113579in}}%
\pgfpathcurveto{\pgfqpoint{1.359466in}{2.105766in}}{\pgfqpoint{1.355076in}{2.095167in}}{\pgfqpoint{1.355076in}{2.084117in}}%
\pgfpathcurveto{\pgfqpoint{1.355076in}{2.073066in}}{\pgfqpoint{1.359466in}{2.062467in}}{\pgfqpoint{1.367280in}{2.054654in}}%
\pgfpathcurveto{\pgfqpoint{1.375094in}{2.046840in}}{\pgfqpoint{1.385693in}{2.042450in}}{\pgfqpoint{1.396743in}{2.042450in}}%
\pgfpathlineto{\pgfqpoint{1.396743in}{2.042450in}}%
\pgfpathclose%
\pgfusepath{stroke}%
\end{pgfscope}%
\begin{pgfscope}%
\pgfpathrectangle{\pgfqpoint{0.494722in}{0.437222in}}{\pgfqpoint{6.275590in}{5.159444in}}%
\pgfusepath{clip}%
\pgfsetbuttcap%
\pgfsetroundjoin%
\pgfsetlinewidth{1.003750pt}%
\definecolor{currentstroke}{rgb}{0.827451,0.827451,0.827451}%
\pgfsetstrokecolor{currentstroke}%
\pgfsetstrokeopacity{0.800000}%
\pgfsetdash{}{0pt}%
\pgfpathmoveto{\pgfqpoint{0.553084in}{3.907249in}}%
\pgfpathcurveto{\pgfqpoint{0.564134in}{3.907249in}}{\pgfqpoint{0.574733in}{3.911639in}}{\pgfqpoint{0.582547in}{3.919453in}}%
\pgfpathcurveto{\pgfqpoint{0.590360in}{3.927267in}}{\pgfqpoint{0.594750in}{3.937866in}}{\pgfqpoint{0.594750in}{3.948916in}}%
\pgfpathcurveto{\pgfqpoint{0.594750in}{3.959966in}}{\pgfqpoint{0.590360in}{3.970565in}}{\pgfqpoint{0.582547in}{3.978378in}}%
\pgfpathcurveto{\pgfqpoint{0.574733in}{3.986192in}}{\pgfqpoint{0.564134in}{3.990582in}}{\pgfqpoint{0.553084in}{3.990582in}}%
\pgfpathcurveto{\pgfqpoint{0.542034in}{3.990582in}}{\pgfqpoint{0.531435in}{3.986192in}}{\pgfqpoint{0.523621in}{3.978378in}}%
\pgfpathcurveto{\pgfqpoint{0.515807in}{3.970565in}}{\pgfqpoint{0.511417in}{3.959966in}}{\pgfqpoint{0.511417in}{3.948916in}}%
\pgfpathcurveto{\pgfqpoint{0.511417in}{3.937866in}}{\pgfqpoint{0.515807in}{3.927267in}}{\pgfqpoint{0.523621in}{3.919453in}}%
\pgfpathcurveto{\pgfqpoint{0.531435in}{3.911639in}}{\pgfqpoint{0.542034in}{3.907249in}}{\pgfqpoint{0.553084in}{3.907249in}}%
\pgfpathlineto{\pgfqpoint{0.553084in}{3.907249in}}%
\pgfpathclose%
\pgfusepath{stroke}%
\end{pgfscope}%
\begin{pgfscope}%
\pgfpathrectangle{\pgfqpoint{0.494722in}{0.437222in}}{\pgfqpoint{6.275590in}{5.159444in}}%
\pgfusepath{clip}%
\pgfsetbuttcap%
\pgfsetroundjoin%
\pgfsetlinewidth{1.003750pt}%
\definecolor{currentstroke}{rgb}{0.827451,0.827451,0.827451}%
\pgfsetstrokecolor{currentstroke}%
\pgfsetstrokeopacity{0.800000}%
\pgfsetdash{}{0pt}%
\pgfpathmoveto{\pgfqpoint{2.443221in}{1.205971in}}%
\pgfpathcurveto{\pgfqpoint{2.454271in}{1.205971in}}{\pgfqpoint{2.464870in}{1.210362in}}{\pgfqpoint{2.472684in}{1.218175in}}%
\pgfpathcurveto{\pgfqpoint{2.480498in}{1.225989in}}{\pgfqpoint{2.484888in}{1.236588in}}{\pgfqpoint{2.484888in}{1.247638in}}%
\pgfpathcurveto{\pgfqpoint{2.484888in}{1.258688in}}{\pgfqpoint{2.480498in}{1.269287in}}{\pgfqpoint{2.472684in}{1.277101in}}%
\pgfpathcurveto{\pgfqpoint{2.464870in}{1.284914in}}{\pgfqpoint{2.454271in}{1.289305in}}{\pgfqpoint{2.443221in}{1.289305in}}%
\pgfpathcurveto{\pgfqpoint{2.432171in}{1.289305in}}{\pgfqpoint{2.421572in}{1.284914in}}{\pgfqpoint{2.413759in}{1.277101in}}%
\pgfpathcurveto{\pgfqpoint{2.405945in}{1.269287in}}{\pgfqpoint{2.401555in}{1.258688in}}{\pgfqpoint{2.401555in}{1.247638in}}%
\pgfpathcurveto{\pgfqpoint{2.401555in}{1.236588in}}{\pgfqpoint{2.405945in}{1.225989in}}{\pgfqpoint{2.413759in}{1.218175in}}%
\pgfpathcurveto{\pgfqpoint{2.421572in}{1.210362in}}{\pgfqpoint{2.432171in}{1.205971in}}{\pgfqpoint{2.443221in}{1.205971in}}%
\pgfpathlineto{\pgfqpoint{2.443221in}{1.205971in}}%
\pgfpathclose%
\pgfusepath{stroke}%
\end{pgfscope}%
\begin{pgfscope}%
\pgfpathrectangle{\pgfqpoint{0.494722in}{0.437222in}}{\pgfqpoint{6.275590in}{5.159444in}}%
\pgfusepath{clip}%
\pgfsetbuttcap%
\pgfsetroundjoin%
\pgfsetlinewidth{1.003750pt}%
\definecolor{currentstroke}{rgb}{0.827451,0.827451,0.827451}%
\pgfsetstrokecolor{currentstroke}%
\pgfsetstrokeopacity{0.800000}%
\pgfsetdash{}{0pt}%
\pgfpathmoveto{\pgfqpoint{2.765798in}{1.011077in}}%
\pgfpathcurveto{\pgfqpoint{2.776848in}{1.011077in}}{\pgfqpoint{2.787447in}{1.015467in}}{\pgfqpoint{2.795261in}{1.023280in}}%
\pgfpathcurveto{\pgfqpoint{2.803075in}{1.031094in}}{\pgfqpoint{2.807465in}{1.041693in}}{\pgfqpoint{2.807465in}{1.052743in}}%
\pgfpathcurveto{\pgfqpoint{2.807465in}{1.063793in}}{\pgfqpoint{2.803075in}{1.074392in}}{\pgfqpoint{2.795261in}{1.082206in}}%
\pgfpathcurveto{\pgfqpoint{2.787447in}{1.090020in}}{\pgfqpoint{2.776848in}{1.094410in}}{\pgfqpoint{2.765798in}{1.094410in}}%
\pgfpathcurveto{\pgfqpoint{2.754748in}{1.094410in}}{\pgfqpoint{2.744149in}{1.090020in}}{\pgfqpoint{2.736335in}{1.082206in}}%
\pgfpathcurveto{\pgfqpoint{2.728522in}{1.074392in}}{\pgfqpoint{2.724132in}{1.063793in}}{\pgfqpoint{2.724132in}{1.052743in}}%
\pgfpathcurveto{\pgfqpoint{2.724132in}{1.041693in}}{\pgfqpoint{2.728522in}{1.031094in}}{\pgfqpoint{2.736335in}{1.023280in}}%
\pgfpathcurveto{\pgfqpoint{2.744149in}{1.015467in}}{\pgfqpoint{2.754748in}{1.011077in}}{\pgfqpoint{2.765798in}{1.011077in}}%
\pgfpathlineto{\pgfqpoint{2.765798in}{1.011077in}}%
\pgfpathclose%
\pgfusepath{stroke}%
\end{pgfscope}%
\begin{pgfscope}%
\pgfpathrectangle{\pgfqpoint{0.494722in}{0.437222in}}{\pgfqpoint{6.275590in}{5.159444in}}%
\pgfusepath{clip}%
\pgfsetbuttcap%
\pgfsetroundjoin%
\pgfsetlinewidth{1.003750pt}%
\definecolor{currentstroke}{rgb}{0.827451,0.827451,0.827451}%
\pgfsetstrokecolor{currentstroke}%
\pgfsetstrokeopacity{0.800000}%
\pgfsetdash{}{0pt}%
\pgfpathmoveto{\pgfqpoint{3.372979in}{0.744763in}}%
\pgfpathcurveto{\pgfqpoint{3.384029in}{0.744763in}}{\pgfqpoint{3.394628in}{0.749153in}}{\pgfqpoint{3.402442in}{0.756967in}}%
\pgfpathcurveto{\pgfqpoint{3.410255in}{0.764781in}}{\pgfqpoint{3.414646in}{0.775380in}}{\pgfqpoint{3.414646in}{0.786430in}}%
\pgfpathcurveto{\pgfqpoint{3.414646in}{0.797480in}}{\pgfqpoint{3.410255in}{0.808079in}}{\pgfqpoint{3.402442in}{0.815893in}}%
\pgfpathcurveto{\pgfqpoint{3.394628in}{0.823706in}}{\pgfqpoint{3.384029in}{0.828097in}}{\pgfqpoint{3.372979in}{0.828097in}}%
\pgfpathcurveto{\pgfqpoint{3.361929in}{0.828097in}}{\pgfqpoint{3.351330in}{0.823706in}}{\pgfqpoint{3.343516in}{0.815893in}}%
\pgfpathcurveto{\pgfqpoint{3.335703in}{0.808079in}}{\pgfqpoint{3.331312in}{0.797480in}}{\pgfqpoint{3.331312in}{0.786430in}}%
\pgfpathcurveto{\pgfqpoint{3.331312in}{0.775380in}}{\pgfqpoint{3.335703in}{0.764781in}}{\pgfqpoint{3.343516in}{0.756967in}}%
\pgfpathcurveto{\pgfqpoint{3.351330in}{0.749153in}}{\pgfqpoint{3.361929in}{0.744763in}}{\pgfqpoint{3.372979in}{0.744763in}}%
\pgfpathlineto{\pgfqpoint{3.372979in}{0.744763in}}%
\pgfpathclose%
\pgfusepath{stroke}%
\end{pgfscope}%
\begin{pgfscope}%
\pgfpathrectangle{\pgfqpoint{0.494722in}{0.437222in}}{\pgfqpoint{6.275590in}{5.159444in}}%
\pgfusepath{clip}%
\pgfsetbuttcap%
\pgfsetroundjoin%
\pgfsetlinewidth{1.003750pt}%
\definecolor{currentstroke}{rgb}{0.827451,0.827451,0.827451}%
\pgfsetstrokecolor{currentstroke}%
\pgfsetstrokeopacity{0.800000}%
\pgfsetdash{}{0pt}%
\pgfpathmoveto{\pgfqpoint{4.495815in}{0.487123in}}%
\pgfpathcurveto{\pgfqpoint{4.506865in}{0.487123in}}{\pgfqpoint{4.517464in}{0.491513in}}{\pgfqpoint{4.525277in}{0.499327in}}%
\pgfpathcurveto{\pgfqpoint{4.533091in}{0.507140in}}{\pgfqpoint{4.537481in}{0.517740in}}{\pgfqpoint{4.537481in}{0.528790in}}%
\pgfpathcurveto{\pgfqpoint{4.537481in}{0.539840in}}{\pgfqpoint{4.533091in}{0.550439in}}{\pgfqpoint{4.525277in}{0.558252in}}%
\pgfpathcurveto{\pgfqpoint{4.517464in}{0.566066in}}{\pgfqpoint{4.506865in}{0.570456in}}{\pgfqpoint{4.495815in}{0.570456in}}%
\pgfpathcurveto{\pgfqpoint{4.484765in}{0.570456in}}{\pgfqpoint{4.474166in}{0.566066in}}{\pgfqpoint{4.466352in}{0.558252in}}%
\pgfpathcurveto{\pgfqpoint{4.458538in}{0.550439in}}{\pgfqpoint{4.454148in}{0.539840in}}{\pgfqpoint{4.454148in}{0.528790in}}%
\pgfpathcurveto{\pgfqpoint{4.454148in}{0.517740in}}{\pgfqpoint{4.458538in}{0.507140in}}{\pgfqpoint{4.466352in}{0.499327in}}%
\pgfpathcurveto{\pgfqpoint{4.474166in}{0.491513in}}{\pgfqpoint{4.484765in}{0.487123in}}{\pgfqpoint{4.495815in}{0.487123in}}%
\pgfpathlineto{\pgfqpoint{4.495815in}{0.487123in}}%
\pgfpathclose%
\pgfusepath{stroke}%
\end{pgfscope}%
\begin{pgfscope}%
\pgfpathrectangle{\pgfqpoint{0.494722in}{0.437222in}}{\pgfqpoint{6.275590in}{5.159444in}}%
\pgfusepath{clip}%
\pgfsetbuttcap%
\pgfsetroundjoin%
\pgfsetlinewidth{1.003750pt}%
\definecolor{currentstroke}{rgb}{0.827451,0.827451,0.827451}%
\pgfsetstrokecolor{currentstroke}%
\pgfsetstrokeopacity{0.800000}%
\pgfsetdash{}{0pt}%
\pgfpathmoveto{\pgfqpoint{4.023146in}{0.574102in}}%
\pgfpathcurveto{\pgfqpoint{4.034196in}{0.574102in}}{\pgfqpoint{4.044795in}{0.578492in}}{\pgfqpoint{4.052609in}{0.586306in}}%
\pgfpathcurveto{\pgfqpoint{4.060422in}{0.594119in}}{\pgfqpoint{4.064812in}{0.604719in}}{\pgfqpoint{4.064812in}{0.615769in}}%
\pgfpathcurveto{\pgfqpoint{4.064812in}{0.626819in}}{\pgfqpoint{4.060422in}{0.637418in}}{\pgfqpoint{4.052609in}{0.645231in}}%
\pgfpathcurveto{\pgfqpoint{4.044795in}{0.653045in}}{\pgfqpoint{4.034196in}{0.657435in}}{\pgfqpoint{4.023146in}{0.657435in}}%
\pgfpathcurveto{\pgfqpoint{4.012096in}{0.657435in}}{\pgfqpoint{4.001497in}{0.653045in}}{\pgfqpoint{3.993683in}{0.645231in}}%
\pgfpathcurveto{\pgfqpoint{3.985869in}{0.637418in}}{\pgfqpoint{3.981479in}{0.626819in}}{\pgfqpoint{3.981479in}{0.615769in}}%
\pgfpathcurveto{\pgfqpoint{3.981479in}{0.604719in}}{\pgfqpoint{3.985869in}{0.594119in}}{\pgfqpoint{3.993683in}{0.586306in}}%
\pgfpathcurveto{\pgfqpoint{4.001497in}{0.578492in}}{\pgfqpoint{4.012096in}{0.574102in}}{\pgfqpoint{4.023146in}{0.574102in}}%
\pgfpathlineto{\pgfqpoint{4.023146in}{0.574102in}}%
\pgfpathclose%
\pgfusepath{stroke}%
\end{pgfscope}%
\begin{pgfscope}%
\pgfpathrectangle{\pgfqpoint{0.494722in}{0.437222in}}{\pgfqpoint{6.275590in}{5.159444in}}%
\pgfusepath{clip}%
\pgfsetbuttcap%
\pgfsetroundjoin%
\pgfsetlinewidth{1.003750pt}%
\definecolor{currentstroke}{rgb}{0.827451,0.827451,0.827451}%
\pgfsetstrokecolor{currentstroke}%
\pgfsetstrokeopacity{0.800000}%
\pgfsetdash{}{0pt}%
\pgfpathmoveto{\pgfqpoint{4.970960in}{0.441718in}}%
\pgfpathcurveto{\pgfqpoint{4.982010in}{0.441718in}}{\pgfqpoint{4.992609in}{0.446108in}}{\pgfqpoint{5.000423in}{0.453922in}}%
\pgfpathcurveto{\pgfqpoint{5.008236in}{0.461735in}}{\pgfqpoint{5.012626in}{0.472334in}}{\pgfqpoint{5.012626in}{0.483385in}}%
\pgfpathcurveto{\pgfqpoint{5.012626in}{0.494435in}}{\pgfqpoint{5.008236in}{0.505034in}}{\pgfqpoint{5.000423in}{0.512847in}}%
\pgfpathcurveto{\pgfqpoint{4.992609in}{0.520661in}}{\pgfqpoint{4.982010in}{0.525051in}}{\pgfqpoint{4.970960in}{0.525051in}}%
\pgfpathcurveto{\pgfqpoint{4.959910in}{0.525051in}}{\pgfqpoint{4.949311in}{0.520661in}}{\pgfqpoint{4.941497in}{0.512847in}}%
\pgfpathcurveto{\pgfqpoint{4.933683in}{0.505034in}}{\pgfqpoint{4.929293in}{0.494435in}}{\pgfqpoint{4.929293in}{0.483385in}}%
\pgfpathcurveto{\pgfqpoint{4.929293in}{0.472334in}}{\pgfqpoint{4.933683in}{0.461735in}}{\pgfqpoint{4.941497in}{0.453922in}}%
\pgfpathcurveto{\pgfqpoint{4.949311in}{0.446108in}}{\pgfqpoint{4.959910in}{0.441718in}}{\pgfqpoint{4.970960in}{0.441718in}}%
\pgfpathlineto{\pgfqpoint{4.970960in}{0.441718in}}%
\pgfpathclose%
\pgfusepath{stroke}%
\end{pgfscope}%
\begin{pgfscope}%
\pgfpathrectangle{\pgfqpoint{0.494722in}{0.437222in}}{\pgfqpoint{6.275590in}{5.159444in}}%
\pgfusepath{clip}%
\pgfsetbuttcap%
\pgfsetroundjoin%
\pgfsetlinewidth{1.003750pt}%
\definecolor{currentstroke}{rgb}{0.827451,0.827451,0.827451}%
\pgfsetstrokecolor{currentstroke}%
\pgfsetstrokeopacity{0.800000}%
\pgfsetdash{}{0pt}%
\pgfpathmoveto{\pgfqpoint{2.429002in}{1.219445in}}%
\pgfpathcurveto{\pgfqpoint{2.440052in}{1.219445in}}{\pgfqpoint{2.450651in}{1.223835in}}{\pgfqpoint{2.458465in}{1.231648in}}%
\pgfpathcurveto{\pgfqpoint{2.466278in}{1.239462in}}{\pgfqpoint{2.470669in}{1.250061in}}{\pgfqpoint{2.470669in}{1.261111in}}%
\pgfpathcurveto{\pgfqpoint{2.470669in}{1.272161in}}{\pgfqpoint{2.466278in}{1.282760in}}{\pgfqpoint{2.458465in}{1.290574in}}%
\pgfpathcurveto{\pgfqpoint{2.450651in}{1.298388in}}{\pgfqpoint{2.440052in}{1.302778in}}{\pgfqpoint{2.429002in}{1.302778in}}%
\pgfpathcurveto{\pgfqpoint{2.417952in}{1.302778in}}{\pgfqpoint{2.407353in}{1.298388in}}{\pgfqpoint{2.399539in}{1.290574in}}%
\pgfpathcurveto{\pgfqpoint{2.391726in}{1.282760in}}{\pgfqpoint{2.387335in}{1.272161in}}{\pgfqpoint{2.387335in}{1.261111in}}%
\pgfpathcurveto{\pgfqpoint{2.387335in}{1.250061in}}{\pgfqpoint{2.391726in}{1.239462in}}{\pgfqpoint{2.399539in}{1.231648in}}%
\pgfpathcurveto{\pgfqpoint{2.407353in}{1.223835in}}{\pgfqpoint{2.417952in}{1.219445in}}{\pgfqpoint{2.429002in}{1.219445in}}%
\pgfpathlineto{\pgfqpoint{2.429002in}{1.219445in}}%
\pgfpathclose%
\pgfusepath{stroke}%
\end{pgfscope}%
\begin{pgfscope}%
\pgfpathrectangle{\pgfqpoint{0.494722in}{0.437222in}}{\pgfqpoint{6.275590in}{5.159444in}}%
\pgfusepath{clip}%
\pgfsetbuttcap%
\pgfsetroundjoin%
\pgfsetlinewidth{1.003750pt}%
\definecolor{currentstroke}{rgb}{0.827451,0.827451,0.827451}%
\pgfsetstrokecolor{currentstroke}%
\pgfsetstrokeopacity{0.800000}%
\pgfsetdash{}{0pt}%
\pgfpathmoveto{\pgfqpoint{1.852529in}{1.617217in}}%
\pgfpathcurveto{\pgfqpoint{1.863579in}{1.617217in}}{\pgfqpoint{1.874178in}{1.621608in}}{\pgfqpoint{1.881991in}{1.629421in}}%
\pgfpathcurveto{\pgfqpoint{1.889805in}{1.637235in}}{\pgfqpoint{1.894195in}{1.647834in}}{\pgfqpoint{1.894195in}{1.658884in}}%
\pgfpathcurveto{\pgfqpoint{1.894195in}{1.669934in}}{\pgfqpoint{1.889805in}{1.680533in}}{\pgfqpoint{1.881991in}{1.688347in}}%
\pgfpathcurveto{\pgfqpoint{1.874178in}{1.696160in}}{\pgfqpoint{1.863579in}{1.700551in}}{\pgfqpoint{1.852529in}{1.700551in}}%
\pgfpathcurveto{\pgfqpoint{1.841479in}{1.700551in}}{\pgfqpoint{1.830880in}{1.696160in}}{\pgfqpoint{1.823066in}{1.688347in}}%
\pgfpathcurveto{\pgfqpoint{1.815252in}{1.680533in}}{\pgfqpoint{1.810862in}{1.669934in}}{\pgfqpoint{1.810862in}{1.658884in}}%
\pgfpathcurveto{\pgfqpoint{1.810862in}{1.647834in}}{\pgfqpoint{1.815252in}{1.637235in}}{\pgfqpoint{1.823066in}{1.629421in}}%
\pgfpathcurveto{\pgfqpoint{1.830880in}{1.621608in}}{\pgfqpoint{1.841479in}{1.617217in}}{\pgfqpoint{1.852529in}{1.617217in}}%
\pgfpathlineto{\pgfqpoint{1.852529in}{1.617217in}}%
\pgfpathclose%
\pgfusepath{stroke}%
\end{pgfscope}%
\begin{pgfscope}%
\pgfpathrectangle{\pgfqpoint{0.494722in}{0.437222in}}{\pgfqpoint{6.275590in}{5.159444in}}%
\pgfusepath{clip}%
\pgfsetbuttcap%
\pgfsetroundjoin%
\pgfsetlinewidth{1.003750pt}%
\definecolor{currentstroke}{rgb}{0.827451,0.827451,0.827451}%
\pgfsetstrokecolor{currentstroke}%
\pgfsetstrokeopacity{0.800000}%
\pgfsetdash{}{0pt}%
\pgfpathmoveto{\pgfqpoint{0.625304in}{3.523383in}}%
\pgfpathcurveto{\pgfqpoint{0.636354in}{3.523383in}}{\pgfqpoint{0.646953in}{3.527774in}}{\pgfqpoint{0.654767in}{3.535587in}}%
\pgfpathcurveto{\pgfqpoint{0.662581in}{3.543401in}}{\pgfqpoint{0.666971in}{3.554000in}}{\pgfqpoint{0.666971in}{3.565050in}}%
\pgfpathcurveto{\pgfqpoint{0.666971in}{3.576100in}}{\pgfqpoint{0.662581in}{3.586699in}}{\pgfqpoint{0.654767in}{3.594513in}}%
\pgfpathcurveto{\pgfqpoint{0.646953in}{3.602326in}}{\pgfqpoint{0.636354in}{3.606717in}}{\pgfqpoint{0.625304in}{3.606717in}}%
\pgfpathcurveto{\pgfqpoint{0.614254in}{3.606717in}}{\pgfqpoint{0.603655in}{3.602326in}}{\pgfqpoint{0.595841in}{3.594513in}}%
\pgfpathcurveto{\pgfqpoint{0.588028in}{3.586699in}}{\pgfqpoint{0.583638in}{3.576100in}}{\pgfqpoint{0.583638in}{3.565050in}}%
\pgfpathcurveto{\pgfqpoint{0.583638in}{3.554000in}}{\pgfqpoint{0.588028in}{3.543401in}}{\pgfqpoint{0.595841in}{3.535587in}}%
\pgfpathcurveto{\pgfqpoint{0.603655in}{3.527774in}}{\pgfqpoint{0.614254in}{3.523383in}}{\pgfqpoint{0.625304in}{3.523383in}}%
\pgfpathlineto{\pgfqpoint{0.625304in}{3.523383in}}%
\pgfpathclose%
\pgfusepath{stroke}%
\end{pgfscope}%
\begin{pgfscope}%
\pgfpathrectangle{\pgfqpoint{0.494722in}{0.437222in}}{\pgfqpoint{6.275590in}{5.159444in}}%
\pgfusepath{clip}%
\pgfsetbuttcap%
\pgfsetroundjoin%
\pgfsetlinewidth{1.003750pt}%
\definecolor{currentstroke}{rgb}{0.827451,0.827451,0.827451}%
\pgfsetstrokecolor{currentstroke}%
\pgfsetstrokeopacity{0.800000}%
\pgfsetdash{}{0pt}%
\pgfpathmoveto{\pgfqpoint{4.327440in}{0.503110in}}%
\pgfpathcurveto{\pgfqpoint{4.338490in}{0.503110in}}{\pgfqpoint{4.349089in}{0.507500in}}{\pgfqpoint{4.356903in}{0.515314in}}%
\pgfpathcurveto{\pgfqpoint{4.364716in}{0.523127in}}{\pgfqpoint{4.369107in}{0.533727in}}{\pgfqpoint{4.369107in}{0.544777in}}%
\pgfpathcurveto{\pgfqpoint{4.369107in}{0.555827in}}{\pgfqpoint{4.364716in}{0.566426in}}{\pgfqpoint{4.356903in}{0.574239in}}%
\pgfpathcurveto{\pgfqpoint{4.349089in}{0.582053in}}{\pgfqpoint{4.338490in}{0.586443in}}{\pgfqpoint{4.327440in}{0.586443in}}%
\pgfpathcurveto{\pgfqpoint{4.316390in}{0.586443in}}{\pgfqpoint{4.305791in}{0.582053in}}{\pgfqpoint{4.297977in}{0.574239in}}%
\pgfpathcurveto{\pgfqpoint{4.290164in}{0.566426in}}{\pgfqpoint{4.285773in}{0.555827in}}{\pgfqpoint{4.285773in}{0.544777in}}%
\pgfpathcurveto{\pgfqpoint{4.285773in}{0.533727in}}{\pgfqpoint{4.290164in}{0.523127in}}{\pgfqpoint{4.297977in}{0.515314in}}%
\pgfpathcurveto{\pgfqpoint{4.305791in}{0.507500in}}{\pgfqpoint{4.316390in}{0.503110in}}{\pgfqpoint{4.327440in}{0.503110in}}%
\pgfpathlineto{\pgfqpoint{4.327440in}{0.503110in}}%
\pgfpathclose%
\pgfusepath{stroke}%
\end{pgfscope}%
\begin{pgfscope}%
\pgfpathrectangle{\pgfqpoint{0.494722in}{0.437222in}}{\pgfqpoint{6.275590in}{5.159444in}}%
\pgfusepath{clip}%
\pgfsetbuttcap%
\pgfsetroundjoin%
\pgfsetlinewidth{1.003750pt}%
\definecolor{currentstroke}{rgb}{0.827451,0.827451,0.827451}%
\pgfsetstrokecolor{currentstroke}%
\pgfsetstrokeopacity{0.800000}%
\pgfsetdash{}{0pt}%
\pgfpathmoveto{\pgfqpoint{1.993886in}{1.499770in}}%
\pgfpathcurveto{\pgfqpoint{2.004936in}{1.499770in}}{\pgfqpoint{2.015535in}{1.504160in}}{\pgfqpoint{2.023349in}{1.511974in}}%
\pgfpathcurveto{\pgfqpoint{2.031162in}{1.519788in}}{\pgfqpoint{2.035553in}{1.530387in}}{\pgfqpoint{2.035553in}{1.541437in}}%
\pgfpathcurveto{\pgfqpoint{2.035553in}{1.552487in}}{\pgfqpoint{2.031162in}{1.563086in}}{\pgfqpoint{2.023349in}{1.570900in}}%
\pgfpathcurveto{\pgfqpoint{2.015535in}{1.578713in}}{\pgfqpoint{2.004936in}{1.583103in}}{\pgfqpoint{1.993886in}{1.583103in}}%
\pgfpathcurveto{\pgfqpoint{1.982836in}{1.583103in}}{\pgfqpoint{1.972237in}{1.578713in}}{\pgfqpoint{1.964423in}{1.570900in}}%
\pgfpathcurveto{\pgfqpoint{1.956610in}{1.563086in}}{\pgfqpoint{1.952219in}{1.552487in}}{\pgfqpoint{1.952219in}{1.541437in}}%
\pgfpathcurveto{\pgfqpoint{1.952219in}{1.530387in}}{\pgfqpoint{1.956610in}{1.519788in}}{\pgfqpoint{1.964423in}{1.511974in}}%
\pgfpathcurveto{\pgfqpoint{1.972237in}{1.504160in}}{\pgfqpoint{1.982836in}{1.499770in}}{\pgfqpoint{1.993886in}{1.499770in}}%
\pgfpathlineto{\pgfqpoint{1.993886in}{1.499770in}}%
\pgfpathclose%
\pgfusepath{stroke}%
\end{pgfscope}%
\begin{pgfscope}%
\pgfpathrectangle{\pgfqpoint{0.494722in}{0.437222in}}{\pgfqpoint{6.275590in}{5.159444in}}%
\pgfusepath{clip}%
\pgfsetbuttcap%
\pgfsetroundjoin%
\pgfsetlinewidth{1.003750pt}%
\definecolor{currentstroke}{rgb}{0.827451,0.827451,0.827451}%
\pgfsetstrokecolor{currentstroke}%
\pgfsetstrokeopacity{0.800000}%
\pgfsetdash{}{0pt}%
\pgfpathmoveto{\pgfqpoint{0.536273in}{3.987890in}}%
\pgfpathcurveto{\pgfqpoint{0.547324in}{3.987890in}}{\pgfqpoint{0.557923in}{3.992280in}}{\pgfqpoint{0.565736in}{4.000093in}}%
\pgfpathcurveto{\pgfqpoint{0.573550in}{4.007907in}}{\pgfqpoint{0.577940in}{4.018506in}}{\pgfqpoint{0.577940in}{4.029556in}}%
\pgfpathcurveto{\pgfqpoint{0.577940in}{4.040606in}}{\pgfqpoint{0.573550in}{4.051205in}}{\pgfqpoint{0.565736in}{4.059019in}}%
\pgfpathcurveto{\pgfqpoint{0.557923in}{4.066833in}}{\pgfqpoint{0.547324in}{4.071223in}}{\pgfqpoint{0.536273in}{4.071223in}}%
\pgfpathcurveto{\pgfqpoint{0.525223in}{4.071223in}}{\pgfqpoint{0.514624in}{4.066833in}}{\pgfqpoint{0.506811in}{4.059019in}}%
\pgfpathcurveto{\pgfqpoint{0.498997in}{4.051205in}}{\pgfqpoint{0.494607in}{4.040606in}}{\pgfqpoint{0.494607in}{4.029556in}}%
\pgfpathcurveto{\pgfqpoint{0.494607in}{4.018506in}}{\pgfqpoint{0.498997in}{4.007907in}}{\pgfqpoint{0.506811in}{4.000093in}}%
\pgfpathcurveto{\pgfqpoint{0.514624in}{3.992280in}}{\pgfqpoint{0.525223in}{3.987890in}}{\pgfqpoint{0.536273in}{3.987890in}}%
\pgfpathlineto{\pgfqpoint{0.536273in}{3.987890in}}%
\pgfpathclose%
\pgfusepath{stroke}%
\end{pgfscope}%
\begin{pgfscope}%
\pgfpathrectangle{\pgfqpoint{0.494722in}{0.437222in}}{\pgfqpoint{6.275590in}{5.159444in}}%
\pgfusepath{clip}%
\pgfsetbuttcap%
\pgfsetroundjoin%
\pgfsetlinewidth{1.003750pt}%
\definecolor{currentstroke}{rgb}{0.827451,0.827451,0.827451}%
\pgfsetstrokecolor{currentstroke}%
\pgfsetstrokeopacity{0.800000}%
\pgfsetdash{}{0pt}%
\pgfpathmoveto{\pgfqpoint{2.512961in}{1.165916in}}%
\pgfpathcurveto{\pgfqpoint{2.524011in}{1.165916in}}{\pgfqpoint{2.534610in}{1.170306in}}{\pgfqpoint{2.542424in}{1.178119in}}%
\pgfpathcurveto{\pgfqpoint{2.550237in}{1.185933in}}{\pgfqpoint{2.554627in}{1.196532in}}{\pgfqpoint{2.554627in}{1.207582in}}%
\pgfpathcurveto{\pgfqpoint{2.554627in}{1.218632in}}{\pgfqpoint{2.550237in}{1.229231in}}{\pgfqpoint{2.542424in}{1.237045in}}%
\pgfpathcurveto{\pgfqpoint{2.534610in}{1.244859in}}{\pgfqpoint{2.524011in}{1.249249in}}{\pgfqpoint{2.512961in}{1.249249in}}%
\pgfpathcurveto{\pgfqpoint{2.501911in}{1.249249in}}{\pgfqpoint{2.491312in}{1.244859in}}{\pgfqpoint{2.483498in}{1.237045in}}%
\pgfpathcurveto{\pgfqpoint{2.475684in}{1.229231in}}{\pgfqpoint{2.471294in}{1.218632in}}{\pgfqpoint{2.471294in}{1.207582in}}%
\pgfpathcurveto{\pgfqpoint{2.471294in}{1.196532in}}{\pgfqpoint{2.475684in}{1.185933in}}{\pgfqpoint{2.483498in}{1.178119in}}%
\pgfpathcurveto{\pgfqpoint{2.491312in}{1.170306in}}{\pgfqpoint{2.501911in}{1.165916in}}{\pgfqpoint{2.512961in}{1.165916in}}%
\pgfpathlineto{\pgfqpoint{2.512961in}{1.165916in}}%
\pgfpathclose%
\pgfusepath{stroke}%
\end{pgfscope}%
\begin{pgfscope}%
\pgfpathrectangle{\pgfqpoint{0.494722in}{0.437222in}}{\pgfqpoint{6.275590in}{5.159444in}}%
\pgfusepath{clip}%
\pgfsetbuttcap%
\pgfsetroundjoin%
\pgfsetlinewidth{1.003750pt}%
\definecolor{currentstroke}{rgb}{0.827451,0.827451,0.827451}%
\pgfsetstrokecolor{currentstroke}%
\pgfsetstrokeopacity{0.800000}%
\pgfsetdash{}{0pt}%
\pgfpathmoveto{\pgfqpoint{1.964907in}{1.517037in}}%
\pgfpathcurveto{\pgfqpoint{1.975957in}{1.517037in}}{\pgfqpoint{1.986556in}{1.521427in}}{\pgfqpoint{1.994370in}{1.529240in}}%
\pgfpathcurveto{\pgfqpoint{2.002183in}{1.537054in}}{\pgfqpoint{2.006573in}{1.547653in}}{\pgfqpoint{2.006573in}{1.558703in}}%
\pgfpathcurveto{\pgfqpoint{2.006573in}{1.569753in}}{\pgfqpoint{2.002183in}{1.580352in}}{\pgfqpoint{1.994370in}{1.588166in}}%
\pgfpathcurveto{\pgfqpoint{1.986556in}{1.595980in}}{\pgfqpoint{1.975957in}{1.600370in}}{\pgfqpoint{1.964907in}{1.600370in}}%
\pgfpathcurveto{\pgfqpoint{1.953857in}{1.600370in}}{\pgfqpoint{1.943258in}{1.595980in}}{\pgfqpoint{1.935444in}{1.588166in}}%
\pgfpathcurveto{\pgfqpoint{1.927630in}{1.580352in}}{\pgfqpoint{1.923240in}{1.569753in}}{\pgfqpoint{1.923240in}{1.558703in}}%
\pgfpathcurveto{\pgfqpoint{1.923240in}{1.547653in}}{\pgfqpoint{1.927630in}{1.537054in}}{\pgfqpoint{1.935444in}{1.529240in}}%
\pgfpathcurveto{\pgfqpoint{1.943258in}{1.521427in}}{\pgfqpoint{1.953857in}{1.517037in}}{\pgfqpoint{1.964907in}{1.517037in}}%
\pgfpathlineto{\pgfqpoint{1.964907in}{1.517037in}}%
\pgfpathclose%
\pgfusepath{stroke}%
\end{pgfscope}%
\begin{pgfscope}%
\pgfpathrectangle{\pgfqpoint{0.494722in}{0.437222in}}{\pgfqpoint{6.275590in}{5.159444in}}%
\pgfusepath{clip}%
\pgfsetbuttcap%
\pgfsetroundjoin%
\pgfsetlinewidth{1.003750pt}%
\definecolor{currentstroke}{rgb}{0.827451,0.827451,0.827451}%
\pgfsetstrokecolor{currentstroke}%
\pgfsetstrokeopacity{0.800000}%
\pgfsetdash{}{0pt}%
\pgfpathmoveto{\pgfqpoint{2.365416in}{1.256449in}}%
\pgfpathcurveto{\pgfqpoint{2.376466in}{1.256449in}}{\pgfqpoint{2.387065in}{1.260839in}}{\pgfqpoint{2.394879in}{1.268653in}}%
\pgfpathcurveto{\pgfqpoint{2.402693in}{1.276466in}}{\pgfqpoint{2.407083in}{1.287065in}}{\pgfqpoint{2.407083in}{1.298115in}}%
\pgfpathcurveto{\pgfqpoint{2.407083in}{1.309166in}}{\pgfqpoint{2.402693in}{1.319765in}}{\pgfqpoint{2.394879in}{1.327578in}}%
\pgfpathcurveto{\pgfqpoint{2.387065in}{1.335392in}}{\pgfqpoint{2.376466in}{1.339782in}}{\pgfqpoint{2.365416in}{1.339782in}}%
\pgfpathcurveto{\pgfqpoint{2.354366in}{1.339782in}}{\pgfqpoint{2.343767in}{1.335392in}}{\pgfqpoint{2.335953in}{1.327578in}}%
\pgfpathcurveto{\pgfqpoint{2.328140in}{1.319765in}}{\pgfqpoint{2.323750in}{1.309166in}}{\pgfqpoint{2.323750in}{1.298115in}}%
\pgfpathcurveto{\pgfqpoint{2.323750in}{1.287065in}}{\pgfqpoint{2.328140in}{1.276466in}}{\pgfqpoint{2.335953in}{1.268653in}}%
\pgfpathcurveto{\pgfqpoint{2.343767in}{1.260839in}}{\pgfqpoint{2.354366in}{1.256449in}}{\pgfqpoint{2.365416in}{1.256449in}}%
\pgfpathlineto{\pgfqpoint{2.365416in}{1.256449in}}%
\pgfpathclose%
\pgfusepath{stroke}%
\end{pgfscope}%
\begin{pgfscope}%
\pgfpathrectangle{\pgfqpoint{0.494722in}{0.437222in}}{\pgfqpoint{6.275590in}{5.159444in}}%
\pgfusepath{clip}%
\pgfsetbuttcap%
\pgfsetroundjoin%
\pgfsetlinewidth{1.003750pt}%
\definecolor{currentstroke}{rgb}{0.827451,0.827451,0.827451}%
\pgfsetstrokecolor{currentstroke}%
\pgfsetstrokeopacity{0.800000}%
\pgfsetdash{}{0pt}%
\pgfpathmoveto{\pgfqpoint{0.530867in}{4.033508in}}%
\pgfpathcurveto{\pgfqpoint{0.541917in}{4.033508in}}{\pgfqpoint{0.552516in}{4.037898in}}{\pgfqpoint{0.560330in}{4.045712in}}%
\pgfpathcurveto{\pgfqpoint{0.568143in}{4.053525in}}{\pgfqpoint{0.572534in}{4.064124in}}{\pgfqpoint{0.572534in}{4.075174in}}%
\pgfpathcurveto{\pgfqpoint{0.572534in}{4.086225in}}{\pgfqpoint{0.568143in}{4.096824in}}{\pgfqpoint{0.560330in}{4.104637in}}%
\pgfpathcurveto{\pgfqpoint{0.552516in}{4.112451in}}{\pgfqpoint{0.541917in}{4.116841in}}{\pgfqpoint{0.530867in}{4.116841in}}%
\pgfpathcurveto{\pgfqpoint{0.519817in}{4.116841in}}{\pgfqpoint{0.509218in}{4.112451in}}{\pgfqpoint{0.501404in}{4.104637in}}%
\pgfpathcurveto{\pgfqpoint{0.493591in}{4.096824in}}{\pgfqpoint{0.489200in}{4.086225in}}{\pgfqpoint{0.489200in}{4.075174in}}%
\pgfpathcurveto{\pgfqpoint{0.489200in}{4.064124in}}{\pgfqpoint{0.493591in}{4.053525in}}{\pgfqpoint{0.501404in}{4.045712in}}%
\pgfpathcurveto{\pgfqpoint{0.509218in}{4.037898in}}{\pgfqpoint{0.519817in}{4.033508in}}{\pgfqpoint{0.530867in}{4.033508in}}%
\pgfpathlineto{\pgfqpoint{0.530867in}{4.033508in}}%
\pgfpathclose%
\pgfusepath{stroke}%
\end{pgfscope}%
\begin{pgfscope}%
\pgfpathrectangle{\pgfqpoint{0.494722in}{0.437222in}}{\pgfqpoint{6.275590in}{5.159444in}}%
\pgfusepath{clip}%
\pgfsetbuttcap%
\pgfsetroundjoin%
\pgfsetlinewidth{1.003750pt}%
\definecolor{currentstroke}{rgb}{0.827451,0.827451,0.827451}%
\pgfsetstrokecolor{currentstroke}%
\pgfsetstrokeopacity{0.800000}%
\pgfsetdash{}{0pt}%
\pgfpathmoveto{\pgfqpoint{0.700772in}{3.267894in}}%
\pgfpathcurveto{\pgfqpoint{0.711822in}{3.267894in}}{\pgfqpoint{0.722421in}{3.272284in}}{\pgfqpoint{0.730235in}{3.280098in}}%
\pgfpathcurveto{\pgfqpoint{0.738048in}{3.287911in}}{\pgfqpoint{0.742438in}{3.298510in}}{\pgfqpoint{0.742438in}{3.309561in}}%
\pgfpathcurveto{\pgfqpoint{0.742438in}{3.320611in}}{\pgfqpoint{0.738048in}{3.331210in}}{\pgfqpoint{0.730235in}{3.339023in}}%
\pgfpathcurveto{\pgfqpoint{0.722421in}{3.346837in}}{\pgfqpoint{0.711822in}{3.351227in}}{\pgfqpoint{0.700772in}{3.351227in}}%
\pgfpathcurveto{\pgfqpoint{0.689722in}{3.351227in}}{\pgfqpoint{0.679123in}{3.346837in}}{\pgfqpoint{0.671309in}{3.339023in}}%
\pgfpathcurveto{\pgfqpoint{0.663495in}{3.331210in}}{\pgfqpoint{0.659105in}{3.320611in}}{\pgfqpoint{0.659105in}{3.309561in}}%
\pgfpathcurveto{\pgfqpoint{0.659105in}{3.298510in}}{\pgfqpoint{0.663495in}{3.287911in}}{\pgfqpoint{0.671309in}{3.280098in}}%
\pgfpathcurveto{\pgfqpoint{0.679123in}{3.272284in}}{\pgfqpoint{0.689722in}{3.267894in}}{\pgfqpoint{0.700772in}{3.267894in}}%
\pgfpathlineto{\pgfqpoint{0.700772in}{3.267894in}}%
\pgfpathclose%
\pgfusepath{stroke}%
\end{pgfscope}%
\begin{pgfscope}%
\pgfpathrectangle{\pgfqpoint{0.494722in}{0.437222in}}{\pgfqpoint{6.275590in}{5.159444in}}%
\pgfusepath{clip}%
\pgfsetbuttcap%
\pgfsetroundjoin%
\pgfsetlinewidth{1.003750pt}%
\definecolor{currentstroke}{rgb}{0.827451,0.827451,0.827451}%
\pgfsetstrokecolor{currentstroke}%
\pgfsetstrokeopacity{0.800000}%
\pgfsetdash{}{0pt}%
\pgfpathmoveto{\pgfqpoint{2.552339in}{1.147505in}}%
\pgfpathcurveto{\pgfqpoint{2.563390in}{1.147505in}}{\pgfqpoint{2.573989in}{1.151895in}}{\pgfqpoint{2.581802in}{1.159709in}}%
\pgfpathcurveto{\pgfqpoint{2.589616in}{1.167522in}}{\pgfqpoint{2.594006in}{1.178121in}}{\pgfqpoint{2.594006in}{1.189172in}}%
\pgfpathcurveto{\pgfqpoint{2.594006in}{1.200222in}}{\pgfqpoint{2.589616in}{1.210821in}}{\pgfqpoint{2.581802in}{1.218634in}}%
\pgfpathcurveto{\pgfqpoint{2.573989in}{1.226448in}}{\pgfqpoint{2.563390in}{1.230838in}}{\pgfqpoint{2.552339in}{1.230838in}}%
\pgfpathcurveto{\pgfqpoint{2.541289in}{1.230838in}}{\pgfqpoint{2.530690in}{1.226448in}}{\pgfqpoint{2.522877in}{1.218634in}}%
\pgfpathcurveto{\pgfqpoint{2.515063in}{1.210821in}}{\pgfqpoint{2.510673in}{1.200222in}}{\pgfqpoint{2.510673in}{1.189172in}}%
\pgfpathcurveto{\pgfqpoint{2.510673in}{1.178121in}}{\pgfqpoint{2.515063in}{1.167522in}}{\pgfqpoint{2.522877in}{1.159709in}}%
\pgfpathcurveto{\pgfqpoint{2.530690in}{1.151895in}}{\pgfqpoint{2.541289in}{1.147505in}}{\pgfqpoint{2.552339in}{1.147505in}}%
\pgfpathlineto{\pgfqpoint{2.552339in}{1.147505in}}%
\pgfpathclose%
\pgfusepath{stroke}%
\end{pgfscope}%
\begin{pgfscope}%
\pgfpathrectangle{\pgfqpoint{0.494722in}{0.437222in}}{\pgfqpoint{6.275590in}{5.159444in}}%
\pgfusepath{clip}%
\pgfsetbuttcap%
\pgfsetroundjoin%
\pgfsetlinewidth{1.003750pt}%
\definecolor{currentstroke}{rgb}{0.827451,0.827451,0.827451}%
\pgfsetstrokecolor{currentstroke}%
\pgfsetstrokeopacity{0.800000}%
\pgfsetdash{}{0pt}%
\pgfpathmoveto{\pgfqpoint{1.259002in}{2.253021in}}%
\pgfpathcurveto{\pgfqpoint{1.270052in}{2.253021in}}{\pgfqpoint{1.280651in}{2.257412in}}{\pgfqpoint{1.288465in}{2.265225in}}%
\pgfpathcurveto{\pgfqpoint{1.296279in}{2.273039in}}{\pgfqpoint{1.300669in}{2.283638in}}{\pgfqpoint{1.300669in}{2.294688in}}%
\pgfpathcurveto{\pgfqpoint{1.300669in}{2.305738in}}{\pgfqpoint{1.296279in}{2.316337in}}{\pgfqpoint{1.288465in}{2.324151in}}%
\pgfpathcurveto{\pgfqpoint{1.280651in}{2.331964in}}{\pgfqpoint{1.270052in}{2.336355in}}{\pgfqpoint{1.259002in}{2.336355in}}%
\pgfpathcurveto{\pgfqpoint{1.247952in}{2.336355in}}{\pgfqpoint{1.237353in}{2.331964in}}{\pgfqpoint{1.229540in}{2.324151in}}%
\pgfpathcurveto{\pgfqpoint{1.221726in}{2.316337in}}{\pgfqpoint{1.217336in}{2.305738in}}{\pgfqpoint{1.217336in}{2.294688in}}%
\pgfpathcurveto{\pgfqpoint{1.217336in}{2.283638in}}{\pgfqpoint{1.221726in}{2.273039in}}{\pgfqpoint{1.229540in}{2.265225in}}%
\pgfpathcurveto{\pgfqpoint{1.237353in}{2.257412in}}{\pgfqpoint{1.247952in}{2.253021in}}{\pgfqpoint{1.259002in}{2.253021in}}%
\pgfpathlineto{\pgfqpoint{1.259002in}{2.253021in}}%
\pgfpathclose%
\pgfusepath{stroke}%
\end{pgfscope}%
\begin{pgfscope}%
\pgfpathrectangle{\pgfqpoint{0.494722in}{0.437222in}}{\pgfqpoint{6.275590in}{5.159444in}}%
\pgfusepath{clip}%
\pgfsetbuttcap%
\pgfsetroundjoin%
\pgfsetlinewidth{1.003750pt}%
\definecolor{currentstroke}{rgb}{0.827451,0.827451,0.827451}%
\pgfsetstrokecolor{currentstroke}%
\pgfsetstrokeopacity{0.800000}%
\pgfsetdash{}{0pt}%
\pgfpathmoveto{\pgfqpoint{1.381781in}{2.056420in}}%
\pgfpathcurveto{\pgfqpoint{1.392831in}{2.056420in}}{\pgfqpoint{1.403430in}{2.060810in}}{\pgfqpoint{1.411244in}{2.068624in}}%
\pgfpathcurveto{\pgfqpoint{1.419057in}{2.076437in}}{\pgfqpoint{1.423448in}{2.087036in}}{\pgfqpoint{1.423448in}{2.098087in}}%
\pgfpathcurveto{\pgfqpoint{1.423448in}{2.109137in}}{\pgfqpoint{1.419057in}{2.119736in}}{\pgfqpoint{1.411244in}{2.127549in}}%
\pgfpathcurveto{\pgfqpoint{1.403430in}{2.135363in}}{\pgfqpoint{1.392831in}{2.139753in}}{\pgfqpoint{1.381781in}{2.139753in}}%
\pgfpathcurveto{\pgfqpoint{1.370731in}{2.139753in}}{\pgfqpoint{1.360132in}{2.135363in}}{\pgfqpoint{1.352318in}{2.127549in}}%
\pgfpathcurveto{\pgfqpoint{1.344505in}{2.119736in}}{\pgfqpoint{1.340114in}{2.109137in}}{\pgfqpoint{1.340114in}{2.098087in}}%
\pgfpathcurveto{\pgfqpoint{1.340114in}{2.087036in}}{\pgfqpoint{1.344505in}{2.076437in}}{\pgfqpoint{1.352318in}{2.068624in}}%
\pgfpathcurveto{\pgfqpoint{1.360132in}{2.060810in}}{\pgfqpoint{1.370731in}{2.056420in}}{\pgfqpoint{1.381781in}{2.056420in}}%
\pgfpathlineto{\pgfqpoint{1.381781in}{2.056420in}}%
\pgfpathclose%
\pgfusepath{stroke}%
\end{pgfscope}%
\begin{pgfscope}%
\pgfpathrectangle{\pgfqpoint{0.494722in}{0.437222in}}{\pgfqpoint{6.275590in}{5.159444in}}%
\pgfusepath{clip}%
\pgfsetbuttcap%
\pgfsetroundjoin%
\pgfsetlinewidth{1.003750pt}%
\definecolor{currentstroke}{rgb}{0.827451,0.827451,0.827451}%
\pgfsetstrokecolor{currentstroke}%
\pgfsetstrokeopacity{0.800000}%
\pgfsetdash{}{0pt}%
\pgfpathmoveto{\pgfqpoint{2.659012in}{1.065276in}}%
\pgfpathcurveto{\pgfqpoint{2.670062in}{1.065276in}}{\pgfqpoint{2.680661in}{1.069667in}}{\pgfqpoint{2.688475in}{1.077480in}}%
\pgfpathcurveto{\pgfqpoint{2.696289in}{1.085294in}}{\pgfqpoint{2.700679in}{1.095893in}}{\pgfqpoint{2.700679in}{1.106943in}}%
\pgfpathcurveto{\pgfqpoint{2.700679in}{1.117993in}}{\pgfqpoint{2.696289in}{1.128592in}}{\pgfqpoint{2.688475in}{1.136406in}}%
\pgfpathcurveto{\pgfqpoint{2.680661in}{1.144219in}}{\pgfqpoint{2.670062in}{1.148610in}}{\pgfqpoint{2.659012in}{1.148610in}}%
\pgfpathcurveto{\pgfqpoint{2.647962in}{1.148610in}}{\pgfqpoint{2.637363in}{1.144219in}}{\pgfqpoint{2.629550in}{1.136406in}}%
\pgfpathcurveto{\pgfqpoint{2.621736in}{1.128592in}}{\pgfqpoint{2.617346in}{1.117993in}}{\pgfqpoint{2.617346in}{1.106943in}}%
\pgfpathcurveto{\pgfqpoint{2.617346in}{1.095893in}}{\pgfqpoint{2.621736in}{1.085294in}}{\pgfqpoint{2.629550in}{1.077480in}}%
\pgfpathcurveto{\pgfqpoint{2.637363in}{1.069667in}}{\pgfqpoint{2.647962in}{1.065276in}}{\pgfqpoint{2.659012in}{1.065276in}}%
\pgfpathlineto{\pgfqpoint{2.659012in}{1.065276in}}%
\pgfpathclose%
\pgfusepath{stroke}%
\end{pgfscope}%
\begin{pgfscope}%
\pgfpathrectangle{\pgfqpoint{0.494722in}{0.437222in}}{\pgfqpoint{6.275590in}{5.159444in}}%
\pgfusepath{clip}%
\pgfsetbuttcap%
\pgfsetroundjoin%
\pgfsetlinewidth{1.003750pt}%
\definecolor{currentstroke}{rgb}{0.827451,0.827451,0.827451}%
\pgfsetstrokecolor{currentstroke}%
\pgfsetstrokeopacity{0.800000}%
\pgfsetdash{}{0pt}%
\pgfpathmoveto{\pgfqpoint{1.852502in}{1.624574in}}%
\pgfpathcurveto{\pgfqpoint{1.863553in}{1.624574in}}{\pgfqpoint{1.874152in}{1.628964in}}{\pgfqpoint{1.881965in}{1.636778in}}%
\pgfpathcurveto{\pgfqpoint{1.889779in}{1.644591in}}{\pgfqpoint{1.894169in}{1.655190in}}{\pgfqpoint{1.894169in}{1.666240in}}%
\pgfpathcurveto{\pgfqpoint{1.894169in}{1.677291in}}{\pgfqpoint{1.889779in}{1.687890in}}{\pgfqpoint{1.881965in}{1.695703in}}%
\pgfpathcurveto{\pgfqpoint{1.874152in}{1.703517in}}{\pgfqpoint{1.863553in}{1.707907in}}{\pgfqpoint{1.852502in}{1.707907in}}%
\pgfpathcurveto{\pgfqpoint{1.841452in}{1.707907in}}{\pgfqpoint{1.830853in}{1.703517in}}{\pgfqpoint{1.823040in}{1.695703in}}%
\pgfpathcurveto{\pgfqpoint{1.815226in}{1.687890in}}{\pgfqpoint{1.810836in}{1.677291in}}{\pgfqpoint{1.810836in}{1.666240in}}%
\pgfpathcurveto{\pgfqpoint{1.810836in}{1.655190in}}{\pgfqpoint{1.815226in}{1.644591in}}{\pgfqpoint{1.823040in}{1.636778in}}%
\pgfpathcurveto{\pgfqpoint{1.830853in}{1.628964in}}{\pgfqpoint{1.841452in}{1.624574in}}{\pgfqpoint{1.852502in}{1.624574in}}%
\pgfpathlineto{\pgfqpoint{1.852502in}{1.624574in}}%
\pgfpathclose%
\pgfusepath{stroke}%
\end{pgfscope}%
\begin{pgfscope}%
\pgfpathrectangle{\pgfqpoint{0.494722in}{0.437222in}}{\pgfqpoint{6.275590in}{5.159444in}}%
\pgfusepath{clip}%
\pgfsetbuttcap%
\pgfsetroundjoin%
\pgfsetlinewidth{1.003750pt}%
\definecolor{currentstroke}{rgb}{0.827451,0.827451,0.827451}%
\pgfsetstrokecolor{currentstroke}%
\pgfsetstrokeopacity{0.800000}%
\pgfsetdash{}{0pt}%
\pgfpathmoveto{\pgfqpoint{1.464755in}{1.963622in}}%
\pgfpathcurveto{\pgfqpoint{1.475805in}{1.963622in}}{\pgfqpoint{1.486404in}{1.968013in}}{\pgfqpoint{1.494217in}{1.975826in}}%
\pgfpathcurveto{\pgfqpoint{1.502031in}{1.983640in}}{\pgfqpoint{1.506421in}{1.994239in}}{\pgfqpoint{1.506421in}{2.005289in}}%
\pgfpathcurveto{\pgfqpoint{1.506421in}{2.016339in}}{\pgfqpoint{1.502031in}{2.026938in}}{\pgfqpoint{1.494217in}{2.034752in}}%
\pgfpathcurveto{\pgfqpoint{1.486404in}{2.042566in}}{\pgfqpoint{1.475805in}{2.046956in}}{\pgfqpoint{1.464755in}{2.046956in}}%
\pgfpathcurveto{\pgfqpoint{1.453705in}{2.046956in}}{\pgfqpoint{1.443106in}{2.042566in}}{\pgfqpoint{1.435292in}{2.034752in}}%
\pgfpathcurveto{\pgfqpoint{1.427478in}{2.026938in}}{\pgfqpoint{1.423088in}{2.016339in}}{\pgfqpoint{1.423088in}{2.005289in}}%
\pgfpathcurveto{\pgfqpoint{1.423088in}{1.994239in}}{\pgfqpoint{1.427478in}{1.983640in}}{\pgfqpoint{1.435292in}{1.975826in}}%
\pgfpathcurveto{\pgfqpoint{1.443106in}{1.968013in}}{\pgfqpoint{1.453705in}{1.963622in}}{\pgfqpoint{1.464755in}{1.963622in}}%
\pgfpathlineto{\pgfqpoint{1.464755in}{1.963622in}}%
\pgfpathclose%
\pgfusepath{stroke}%
\end{pgfscope}%
\begin{pgfscope}%
\pgfpathrectangle{\pgfqpoint{0.494722in}{0.437222in}}{\pgfqpoint{6.275590in}{5.159444in}}%
\pgfusepath{clip}%
\pgfsetbuttcap%
\pgfsetroundjoin%
\pgfsetlinewidth{1.003750pt}%
\definecolor{currentstroke}{rgb}{0.827451,0.827451,0.827451}%
\pgfsetstrokecolor{currentstroke}%
\pgfsetstrokeopacity{0.800000}%
\pgfsetdash{}{0pt}%
\pgfpathmoveto{\pgfqpoint{0.525453in}{4.078418in}}%
\pgfpathcurveto{\pgfqpoint{0.536503in}{4.078418in}}{\pgfqpoint{0.547102in}{4.082808in}}{\pgfqpoint{0.554915in}{4.090621in}}%
\pgfpathcurveto{\pgfqpoint{0.562729in}{4.098435in}}{\pgfqpoint{0.567119in}{4.109034in}}{\pgfqpoint{0.567119in}{4.120084in}}%
\pgfpathcurveto{\pgfqpoint{0.567119in}{4.131134in}}{\pgfqpoint{0.562729in}{4.141733in}}{\pgfqpoint{0.554915in}{4.149547in}}%
\pgfpathcurveto{\pgfqpoint{0.547102in}{4.157361in}}{\pgfqpoint{0.536503in}{4.161751in}}{\pgfqpoint{0.525453in}{4.161751in}}%
\pgfpathcurveto{\pgfqpoint{0.514403in}{4.161751in}}{\pgfqpoint{0.503804in}{4.157361in}}{\pgfqpoint{0.495990in}{4.149547in}}%
\pgfpathcurveto{\pgfqpoint{0.488176in}{4.141733in}}{\pgfqpoint{0.483786in}{4.131134in}}{\pgfqpoint{0.483786in}{4.120084in}}%
\pgfpathcurveto{\pgfqpoint{0.483786in}{4.109034in}}{\pgfqpoint{0.488176in}{4.098435in}}{\pgfqpoint{0.495990in}{4.090621in}}%
\pgfpathcurveto{\pgfqpoint{0.503804in}{4.082808in}}{\pgfqpoint{0.514403in}{4.078418in}}{\pgfqpoint{0.525453in}{4.078418in}}%
\pgfpathlineto{\pgfqpoint{0.525453in}{4.078418in}}%
\pgfpathclose%
\pgfusepath{stroke}%
\end{pgfscope}%
\begin{pgfscope}%
\pgfpathrectangle{\pgfqpoint{0.494722in}{0.437222in}}{\pgfqpoint{6.275590in}{5.159444in}}%
\pgfusepath{clip}%
\pgfsetbuttcap%
\pgfsetroundjoin%
\pgfsetlinewidth{1.003750pt}%
\definecolor{currentstroke}{rgb}{0.827451,0.827451,0.827451}%
\pgfsetstrokecolor{currentstroke}%
\pgfsetstrokeopacity{0.800000}%
\pgfsetdash{}{0pt}%
\pgfpathmoveto{\pgfqpoint{0.862000in}{2.887614in}}%
\pgfpathcurveto{\pgfqpoint{0.873051in}{2.887614in}}{\pgfqpoint{0.883650in}{2.892005in}}{\pgfqpoint{0.891463in}{2.899818in}}%
\pgfpathcurveto{\pgfqpoint{0.899277in}{2.907632in}}{\pgfqpoint{0.903667in}{2.918231in}}{\pgfqpoint{0.903667in}{2.929281in}}%
\pgfpathcurveto{\pgfqpoint{0.903667in}{2.940331in}}{\pgfqpoint{0.899277in}{2.950930in}}{\pgfqpoint{0.891463in}{2.958744in}}%
\pgfpathcurveto{\pgfqpoint{0.883650in}{2.966557in}}{\pgfqpoint{0.873051in}{2.970948in}}{\pgfqpoint{0.862000in}{2.970948in}}%
\pgfpathcurveto{\pgfqpoint{0.850950in}{2.970948in}}{\pgfqpoint{0.840351in}{2.966557in}}{\pgfqpoint{0.832538in}{2.958744in}}%
\pgfpathcurveto{\pgfqpoint{0.824724in}{2.950930in}}{\pgfqpoint{0.820334in}{2.940331in}}{\pgfqpoint{0.820334in}{2.929281in}}%
\pgfpathcurveto{\pgfqpoint{0.820334in}{2.918231in}}{\pgfqpoint{0.824724in}{2.907632in}}{\pgfqpoint{0.832538in}{2.899818in}}%
\pgfpathcurveto{\pgfqpoint{0.840351in}{2.892005in}}{\pgfqpoint{0.850950in}{2.887614in}}{\pgfqpoint{0.862000in}{2.887614in}}%
\pgfpathlineto{\pgfqpoint{0.862000in}{2.887614in}}%
\pgfpathclose%
\pgfusepath{stroke}%
\end{pgfscope}%
\begin{pgfscope}%
\pgfpathrectangle{\pgfqpoint{0.494722in}{0.437222in}}{\pgfqpoint{6.275590in}{5.159444in}}%
\pgfusepath{clip}%
\pgfsetbuttcap%
\pgfsetroundjoin%
\pgfsetlinewidth{1.003750pt}%
\definecolor{currentstroke}{rgb}{0.827451,0.827451,0.827451}%
\pgfsetstrokecolor{currentstroke}%
\pgfsetstrokeopacity{0.800000}%
\pgfsetdash{}{0pt}%
\pgfpathmoveto{\pgfqpoint{0.499605in}{4.416043in}}%
\pgfpathcurveto{\pgfqpoint{0.510655in}{4.416043in}}{\pgfqpoint{0.521254in}{4.420433in}}{\pgfqpoint{0.529067in}{4.428247in}}%
\pgfpathcurveto{\pgfqpoint{0.536881in}{4.436060in}}{\pgfqpoint{0.541271in}{4.446659in}}{\pgfqpoint{0.541271in}{4.457710in}}%
\pgfpathcurveto{\pgfqpoint{0.541271in}{4.468760in}}{\pgfqpoint{0.536881in}{4.479359in}}{\pgfqpoint{0.529067in}{4.487172in}}%
\pgfpathcurveto{\pgfqpoint{0.521254in}{4.494986in}}{\pgfqpoint{0.510655in}{4.499376in}}{\pgfqpoint{0.499605in}{4.499376in}}%
\pgfpathcurveto{\pgfqpoint{0.488554in}{4.499376in}}{\pgfqpoint{0.477955in}{4.494986in}}{\pgfqpoint{0.470142in}{4.487172in}}%
\pgfpathcurveto{\pgfqpoint{0.462328in}{4.479359in}}{\pgfqpoint{0.457938in}{4.468760in}}{\pgfqpoint{0.457938in}{4.457710in}}%
\pgfpathcurveto{\pgfqpoint{0.457938in}{4.446659in}}{\pgfqpoint{0.462328in}{4.436060in}}{\pgfqpoint{0.470142in}{4.428247in}}%
\pgfpathcurveto{\pgfqpoint{0.477955in}{4.420433in}}{\pgfqpoint{0.488554in}{4.416043in}}{\pgfqpoint{0.499605in}{4.416043in}}%
\pgfpathlineto{\pgfqpoint{0.499605in}{4.416043in}}%
\pgfpathclose%
\pgfusepath{stroke}%
\end{pgfscope}%
\begin{pgfscope}%
\pgfpathrectangle{\pgfqpoint{0.494722in}{0.437222in}}{\pgfqpoint{6.275590in}{5.159444in}}%
\pgfusepath{clip}%
\pgfsetbuttcap%
\pgfsetroundjoin%
\pgfsetlinewidth{1.003750pt}%
\definecolor{currentstroke}{rgb}{0.827451,0.827451,0.827451}%
\pgfsetstrokecolor{currentstroke}%
\pgfsetstrokeopacity{0.800000}%
\pgfsetdash{}{0pt}%
\pgfpathmoveto{\pgfqpoint{2.081578in}{1.430058in}}%
\pgfpathcurveto{\pgfqpoint{2.092628in}{1.430058in}}{\pgfqpoint{2.103227in}{1.434449in}}{\pgfqpoint{2.111040in}{1.442262in}}%
\pgfpathcurveto{\pgfqpoint{2.118854in}{1.450076in}}{\pgfqpoint{2.123244in}{1.460675in}}{\pgfqpoint{2.123244in}{1.471725in}}%
\pgfpathcurveto{\pgfqpoint{2.123244in}{1.482775in}}{\pgfqpoint{2.118854in}{1.493374in}}{\pgfqpoint{2.111040in}{1.501188in}}%
\pgfpathcurveto{\pgfqpoint{2.103227in}{1.509001in}}{\pgfqpoint{2.092628in}{1.513392in}}{\pgfqpoint{2.081578in}{1.513392in}}%
\pgfpathcurveto{\pgfqpoint{2.070527in}{1.513392in}}{\pgfqpoint{2.059928in}{1.509001in}}{\pgfqpoint{2.052115in}{1.501188in}}%
\pgfpathcurveto{\pgfqpoint{2.044301in}{1.493374in}}{\pgfqpoint{2.039911in}{1.482775in}}{\pgfqpoint{2.039911in}{1.471725in}}%
\pgfpathcurveto{\pgfqpoint{2.039911in}{1.460675in}}{\pgfqpoint{2.044301in}{1.450076in}}{\pgfqpoint{2.052115in}{1.442262in}}%
\pgfpathcurveto{\pgfqpoint{2.059928in}{1.434449in}}{\pgfqpoint{2.070527in}{1.430058in}}{\pgfqpoint{2.081578in}{1.430058in}}%
\pgfpathlineto{\pgfqpoint{2.081578in}{1.430058in}}%
\pgfpathclose%
\pgfusepath{stroke}%
\end{pgfscope}%
\begin{pgfscope}%
\pgfpathrectangle{\pgfqpoint{0.494722in}{0.437222in}}{\pgfqpoint{6.275590in}{5.159444in}}%
\pgfusepath{clip}%
\pgfsetbuttcap%
\pgfsetroundjoin%
\pgfsetlinewidth{1.003750pt}%
\definecolor{currentstroke}{rgb}{0.827451,0.827451,0.827451}%
\pgfsetstrokecolor{currentstroke}%
\pgfsetstrokeopacity{0.800000}%
\pgfsetdash{}{0pt}%
\pgfpathmoveto{\pgfqpoint{1.941684in}{1.555345in}}%
\pgfpathcurveto{\pgfqpoint{1.952734in}{1.555345in}}{\pgfqpoint{1.963333in}{1.559735in}}{\pgfqpoint{1.971146in}{1.567549in}}%
\pgfpathcurveto{\pgfqpoint{1.978960in}{1.575362in}}{\pgfqpoint{1.983350in}{1.585961in}}{\pgfqpoint{1.983350in}{1.597011in}}%
\pgfpathcurveto{\pgfqpoint{1.983350in}{1.608062in}}{\pgfqpoint{1.978960in}{1.618661in}}{\pgfqpoint{1.971146in}{1.626474in}}%
\pgfpathcurveto{\pgfqpoint{1.963333in}{1.634288in}}{\pgfqpoint{1.952734in}{1.638678in}}{\pgfqpoint{1.941684in}{1.638678in}}%
\pgfpathcurveto{\pgfqpoint{1.930633in}{1.638678in}}{\pgfqpoint{1.920034in}{1.634288in}}{\pgfqpoint{1.912221in}{1.626474in}}%
\pgfpathcurveto{\pgfqpoint{1.904407in}{1.618661in}}{\pgfqpoint{1.900017in}{1.608062in}}{\pgfqpoint{1.900017in}{1.597011in}}%
\pgfpathcurveto{\pgfqpoint{1.900017in}{1.585961in}}{\pgfqpoint{1.904407in}{1.575362in}}{\pgfqpoint{1.912221in}{1.567549in}}%
\pgfpathcurveto{\pgfqpoint{1.920034in}{1.559735in}}{\pgfqpoint{1.930633in}{1.555345in}}{\pgfqpoint{1.941684in}{1.555345in}}%
\pgfpathlineto{\pgfqpoint{1.941684in}{1.555345in}}%
\pgfpathclose%
\pgfusepath{stroke}%
\end{pgfscope}%
\begin{pgfscope}%
\pgfpathrectangle{\pgfqpoint{0.494722in}{0.437222in}}{\pgfqpoint{6.275590in}{5.159444in}}%
\pgfusepath{clip}%
\pgfsetbuttcap%
\pgfsetroundjoin%
\pgfsetlinewidth{1.003750pt}%
\definecolor{currentstroke}{rgb}{0.827451,0.827451,0.827451}%
\pgfsetstrokecolor{currentstroke}%
\pgfsetstrokeopacity{0.800000}%
\pgfsetdash{}{0pt}%
\pgfpathmoveto{\pgfqpoint{2.572850in}{1.116571in}}%
\pgfpathcurveto{\pgfqpoint{2.583900in}{1.116571in}}{\pgfqpoint{2.594499in}{1.120961in}}{\pgfqpoint{2.602313in}{1.128774in}}%
\pgfpathcurveto{\pgfqpoint{2.610126in}{1.136588in}}{\pgfqpoint{2.614517in}{1.147187in}}{\pgfqpoint{2.614517in}{1.158237in}}%
\pgfpathcurveto{\pgfqpoint{2.614517in}{1.169287in}}{\pgfqpoint{2.610126in}{1.179886in}}{\pgfqpoint{2.602313in}{1.187700in}}%
\pgfpathcurveto{\pgfqpoint{2.594499in}{1.195514in}}{\pgfqpoint{2.583900in}{1.199904in}}{\pgfqpoint{2.572850in}{1.199904in}}%
\pgfpathcurveto{\pgfqpoint{2.561800in}{1.199904in}}{\pgfqpoint{2.551201in}{1.195514in}}{\pgfqpoint{2.543387in}{1.187700in}}%
\pgfpathcurveto{\pgfqpoint{2.535574in}{1.179886in}}{\pgfqpoint{2.531183in}{1.169287in}}{\pgfqpoint{2.531183in}{1.158237in}}%
\pgfpathcurveto{\pgfqpoint{2.531183in}{1.147187in}}{\pgfqpoint{2.535574in}{1.136588in}}{\pgfqpoint{2.543387in}{1.128774in}}%
\pgfpathcurveto{\pgfqpoint{2.551201in}{1.120961in}}{\pgfqpoint{2.561800in}{1.116571in}}{\pgfqpoint{2.572850in}{1.116571in}}%
\pgfpathlineto{\pgfqpoint{2.572850in}{1.116571in}}%
\pgfpathclose%
\pgfusepath{stroke}%
\end{pgfscope}%
\begin{pgfscope}%
\pgfpathrectangle{\pgfqpoint{0.494722in}{0.437222in}}{\pgfqpoint{6.275590in}{5.159444in}}%
\pgfusepath{clip}%
\pgfsetbuttcap%
\pgfsetroundjoin%
\pgfsetlinewidth{1.003750pt}%
\definecolor{currentstroke}{rgb}{0.827451,0.827451,0.827451}%
\pgfsetstrokecolor{currentstroke}%
\pgfsetstrokeopacity{0.800000}%
\pgfsetdash{}{0pt}%
\pgfpathmoveto{\pgfqpoint{1.171136in}{2.367910in}}%
\pgfpathcurveto{\pgfqpoint{1.182186in}{2.367910in}}{\pgfqpoint{1.192785in}{2.372300in}}{\pgfqpoint{1.200599in}{2.380114in}}%
\pgfpathcurveto{\pgfqpoint{1.208412in}{2.387927in}}{\pgfqpoint{1.212803in}{2.398526in}}{\pgfqpoint{1.212803in}{2.409576in}}%
\pgfpathcurveto{\pgfqpoint{1.212803in}{2.420626in}}{\pgfqpoint{1.208412in}{2.431225in}}{\pgfqpoint{1.200599in}{2.439039in}}%
\pgfpathcurveto{\pgfqpoint{1.192785in}{2.446853in}}{\pgfqpoint{1.182186in}{2.451243in}}{\pgfqpoint{1.171136in}{2.451243in}}%
\pgfpathcurveto{\pgfqpoint{1.160086in}{2.451243in}}{\pgfqpoint{1.149487in}{2.446853in}}{\pgfqpoint{1.141673in}{2.439039in}}%
\pgfpathcurveto{\pgfqpoint{1.133860in}{2.431225in}}{\pgfqpoint{1.129469in}{2.420626in}}{\pgfqpoint{1.129469in}{2.409576in}}%
\pgfpathcurveto{\pgfqpoint{1.129469in}{2.398526in}}{\pgfqpoint{1.133860in}{2.387927in}}{\pgfqpoint{1.141673in}{2.380114in}}%
\pgfpathcurveto{\pgfqpoint{1.149487in}{2.372300in}}{\pgfqpoint{1.160086in}{2.367910in}}{\pgfqpoint{1.171136in}{2.367910in}}%
\pgfpathlineto{\pgfqpoint{1.171136in}{2.367910in}}%
\pgfpathclose%
\pgfusepath{stroke}%
\end{pgfscope}%
\begin{pgfscope}%
\pgfpathrectangle{\pgfqpoint{0.494722in}{0.437222in}}{\pgfqpoint{6.275590in}{5.159444in}}%
\pgfusepath{clip}%
\pgfsetbuttcap%
\pgfsetroundjoin%
\pgfsetlinewidth{1.003750pt}%
\definecolor{currentstroke}{rgb}{0.827451,0.827451,0.827451}%
\pgfsetstrokecolor{currentstroke}%
\pgfsetstrokeopacity{0.800000}%
\pgfsetdash{}{0pt}%
\pgfpathmoveto{\pgfqpoint{1.268327in}{2.198214in}}%
\pgfpathcurveto{\pgfqpoint{1.279377in}{2.198214in}}{\pgfqpoint{1.289976in}{2.202604in}}{\pgfqpoint{1.297790in}{2.210418in}}%
\pgfpathcurveto{\pgfqpoint{1.305603in}{2.218231in}}{\pgfqpoint{1.309994in}{2.228830in}}{\pgfqpoint{1.309994in}{2.239880in}}%
\pgfpathcurveto{\pgfqpoint{1.309994in}{2.250930in}}{\pgfqpoint{1.305603in}{2.261529in}}{\pgfqpoint{1.297790in}{2.269343in}}%
\pgfpathcurveto{\pgfqpoint{1.289976in}{2.277157in}}{\pgfqpoint{1.279377in}{2.281547in}}{\pgfqpoint{1.268327in}{2.281547in}}%
\pgfpathcurveto{\pgfqpoint{1.257277in}{2.281547in}}{\pgfqpoint{1.246678in}{2.277157in}}{\pgfqpoint{1.238864in}{2.269343in}}%
\pgfpathcurveto{\pgfqpoint{1.231051in}{2.261529in}}{\pgfqpoint{1.226660in}{2.250930in}}{\pgfqpoint{1.226660in}{2.239880in}}%
\pgfpathcurveto{\pgfqpoint{1.226660in}{2.228830in}}{\pgfqpoint{1.231051in}{2.218231in}}{\pgfqpoint{1.238864in}{2.210418in}}%
\pgfpathcurveto{\pgfqpoint{1.246678in}{2.202604in}}{\pgfqpoint{1.257277in}{2.198214in}}{\pgfqpoint{1.268327in}{2.198214in}}%
\pgfpathlineto{\pgfqpoint{1.268327in}{2.198214in}}%
\pgfpathclose%
\pgfusepath{stroke}%
\end{pgfscope}%
\begin{pgfscope}%
\pgfpathrectangle{\pgfqpoint{0.494722in}{0.437222in}}{\pgfqpoint{6.275590in}{5.159444in}}%
\pgfusepath{clip}%
\pgfsetbuttcap%
\pgfsetroundjoin%
\pgfsetlinewidth{1.003750pt}%
\definecolor{currentstroke}{rgb}{0.827451,0.827451,0.827451}%
\pgfsetstrokecolor{currentstroke}%
\pgfsetstrokeopacity{0.800000}%
\pgfsetdash{}{0pt}%
\pgfpathmoveto{\pgfqpoint{2.600306in}{1.097647in}}%
\pgfpathcurveto{\pgfqpoint{2.611356in}{1.097647in}}{\pgfqpoint{2.621955in}{1.102037in}}{\pgfqpoint{2.629769in}{1.109851in}}%
\pgfpathcurveto{\pgfqpoint{2.637582in}{1.117665in}}{\pgfqpoint{2.641972in}{1.128264in}}{\pgfqpoint{2.641972in}{1.139314in}}%
\pgfpathcurveto{\pgfqpoint{2.641972in}{1.150364in}}{\pgfqpoint{2.637582in}{1.160963in}}{\pgfqpoint{2.629769in}{1.168777in}}%
\pgfpathcurveto{\pgfqpoint{2.621955in}{1.176590in}}{\pgfqpoint{2.611356in}{1.180981in}}{\pgfqpoint{2.600306in}{1.180981in}}%
\pgfpathcurveto{\pgfqpoint{2.589256in}{1.180981in}}{\pgfqpoint{2.578657in}{1.176590in}}{\pgfqpoint{2.570843in}{1.168777in}}%
\pgfpathcurveto{\pgfqpoint{2.563029in}{1.160963in}}{\pgfqpoint{2.558639in}{1.150364in}}{\pgfqpoint{2.558639in}{1.139314in}}%
\pgfpathcurveto{\pgfqpoint{2.558639in}{1.128264in}}{\pgfqpoint{2.563029in}{1.117665in}}{\pgfqpoint{2.570843in}{1.109851in}}%
\pgfpathcurveto{\pgfqpoint{2.578657in}{1.102037in}}{\pgfqpoint{2.589256in}{1.097647in}}{\pgfqpoint{2.600306in}{1.097647in}}%
\pgfpathlineto{\pgfqpoint{2.600306in}{1.097647in}}%
\pgfpathclose%
\pgfusepath{stroke}%
\end{pgfscope}%
\begin{pgfscope}%
\pgfpathrectangle{\pgfqpoint{0.494722in}{0.437222in}}{\pgfqpoint{6.275590in}{5.159444in}}%
\pgfusepath{clip}%
\pgfsetbuttcap%
\pgfsetroundjoin%
\pgfsetlinewidth{1.003750pt}%
\definecolor{currentstroke}{rgb}{0.827451,0.827451,0.827451}%
\pgfsetstrokecolor{currentstroke}%
\pgfsetstrokeopacity{0.800000}%
\pgfsetdash{}{0pt}%
\pgfpathmoveto{\pgfqpoint{1.186536in}{2.315623in}}%
\pgfpathcurveto{\pgfqpoint{1.197586in}{2.315623in}}{\pgfqpoint{1.208185in}{2.320013in}}{\pgfqpoint{1.215999in}{2.327827in}}%
\pgfpathcurveto{\pgfqpoint{1.223812in}{2.335641in}}{\pgfqpoint{1.228203in}{2.346240in}}{\pgfqpoint{1.228203in}{2.357290in}}%
\pgfpathcurveto{\pgfqpoint{1.228203in}{2.368340in}}{\pgfqpoint{1.223812in}{2.378939in}}{\pgfqpoint{1.215999in}{2.386753in}}%
\pgfpathcurveto{\pgfqpoint{1.208185in}{2.394566in}}{\pgfqpoint{1.197586in}{2.398957in}}{\pgfqpoint{1.186536in}{2.398957in}}%
\pgfpathcurveto{\pgfqpoint{1.175486in}{2.398957in}}{\pgfqpoint{1.164887in}{2.394566in}}{\pgfqpoint{1.157073in}{2.386753in}}%
\pgfpathcurveto{\pgfqpoint{1.149259in}{2.378939in}}{\pgfqpoint{1.144869in}{2.368340in}}{\pgfqpoint{1.144869in}{2.357290in}}%
\pgfpathcurveto{\pgfqpoint{1.144869in}{2.346240in}}{\pgfqpoint{1.149259in}{2.335641in}}{\pgfqpoint{1.157073in}{2.327827in}}%
\pgfpathcurveto{\pgfqpoint{1.164887in}{2.320013in}}{\pgfqpoint{1.175486in}{2.315623in}}{\pgfqpoint{1.186536in}{2.315623in}}%
\pgfpathlineto{\pgfqpoint{1.186536in}{2.315623in}}%
\pgfpathclose%
\pgfusepath{stroke}%
\end{pgfscope}%
\begin{pgfscope}%
\pgfpathrectangle{\pgfqpoint{0.494722in}{0.437222in}}{\pgfqpoint{6.275590in}{5.159444in}}%
\pgfusepath{clip}%
\pgfsetbuttcap%
\pgfsetroundjoin%
\pgfsetlinewidth{1.003750pt}%
\definecolor{currentstroke}{rgb}{0.827451,0.827451,0.827451}%
\pgfsetstrokecolor{currentstroke}%
\pgfsetstrokeopacity{0.800000}%
\pgfsetdash{}{0pt}%
\pgfpathmoveto{\pgfqpoint{0.711186in}{3.233934in}}%
\pgfpathcurveto{\pgfqpoint{0.722236in}{3.233934in}}{\pgfqpoint{0.732835in}{3.238324in}}{\pgfqpoint{0.740648in}{3.246138in}}%
\pgfpathcurveto{\pgfqpoint{0.748462in}{3.253952in}}{\pgfqpoint{0.752852in}{3.264551in}}{\pgfqpoint{0.752852in}{3.275601in}}%
\pgfpathcurveto{\pgfqpoint{0.752852in}{3.286651in}}{\pgfqpoint{0.748462in}{3.297250in}}{\pgfqpoint{0.740648in}{3.305064in}}%
\pgfpathcurveto{\pgfqpoint{0.732835in}{3.312877in}}{\pgfqpoint{0.722236in}{3.317267in}}{\pgfqpoint{0.711186in}{3.317267in}}%
\pgfpathcurveto{\pgfqpoint{0.700135in}{3.317267in}}{\pgfqpoint{0.689536in}{3.312877in}}{\pgfqpoint{0.681723in}{3.305064in}}%
\pgfpathcurveto{\pgfqpoint{0.673909in}{3.297250in}}{\pgfqpoint{0.669519in}{3.286651in}}{\pgfqpoint{0.669519in}{3.275601in}}%
\pgfpathcurveto{\pgfqpoint{0.669519in}{3.264551in}}{\pgfqpoint{0.673909in}{3.253952in}}{\pgfqpoint{0.681723in}{3.246138in}}%
\pgfpathcurveto{\pgfqpoint{0.689536in}{3.238324in}}{\pgfqpoint{0.700135in}{3.233934in}}{\pgfqpoint{0.711186in}{3.233934in}}%
\pgfpathlineto{\pgfqpoint{0.711186in}{3.233934in}}%
\pgfpathclose%
\pgfusepath{stroke}%
\end{pgfscope}%
\begin{pgfscope}%
\pgfpathrectangle{\pgfqpoint{0.494722in}{0.437222in}}{\pgfqpoint{6.275590in}{5.159444in}}%
\pgfusepath{clip}%
\pgfsetbuttcap%
\pgfsetroundjoin%
\pgfsetlinewidth{1.003750pt}%
\definecolor{currentstroke}{rgb}{0.827451,0.827451,0.827451}%
\pgfsetstrokecolor{currentstroke}%
\pgfsetstrokeopacity{0.800000}%
\pgfsetdash{}{0pt}%
\pgfpathmoveto{\pgfqpoint{1.347485in}{2.096739in}}%
\pgfpathcurveto{\pgfqpoint{1.358535in}{2.096739in}}{\pgfqpoint{1.369135in}{2.101129in}}{\pgfqpoint{1.376948in}{2.108943in}}%
\pgfpathcurveto{\pgfqpoint{1.384762in}{2.116757in}}{\pgfqpoint{1.389152in}{2.127356in}}{\pgfqpoint{1.389152in}{2.138406in}}%
\pgfpathcurveto{\pgfqpoint{1.389152in}{2.149456in}}{\pgfqpoint{1.384762in}{2.160055in}}{\pgfqpoint{1.376948in}{2.167869in}}%
\pgfpathcurveto{\pgfqpoint{1.369135in}{2.175682in}}{\pgfqpoint{1.358535in}{2.180072in}}{\pgfqpoint{1.347485in}{2.180072in}}%
\pgfpathcurveto{\pgfqpoint{1.336435in}{2.180072in}}{\pgfqpoint{1.325836in}{2.175682in}}{\pgfqpoint{1.318023in}{2.167869in}}%
\pgfpathcurveto{\pgfqpoint{1.310209in}{2.160055in}}{\pgfqpoint{1.305819in}{2.149456in}}{\pgfqpoint{1.305819in}{2.138406in}}%
\pgfpathcurveto{\pgfqpoint{1.305819in}{2.127356in}}{\pgfqpoint{1.310209in}{2.116757in}}{\pgfqpoint{1.318023in}{2.108943in}}%
\pgfpathcurveto{\pgfqpoint{1.325836in}{2.101129in}}{\pgfqpoint{1.336435in}{2.096739in}}{\pgfqpoint{1.347485in}{2.096739in}}%
\pgfpathlineto{\pgfqpoint{1.347485in}{2.096739in}}%
\pgfpathclose%
\pgfusepath{stroke}%
\end{pgfscope}%
\begin{pgfscope}%
\pgfpathrectangle{\pgfqpoint{0.494722in}{0.437222in}}{\pgfqpoint{6.275590in}{5.159444in}}%
\pgfusepath{clip}%
\pgfsetbuttcap%
\pgfsetroundjoin%
\pgfsetlinewidth{1.003750pt}%
\definecolor{currentstroke}{rgb}{0.827451,0.827451,0.827451}%
\pgfsetstrokecolor{currentstroke}%
\pgfsetstrokeopacity{0.800000}%
\pgfsetdash{}{0pt}%
\pgfpathmoveto{\pgfqpoint{2.203827in}{1.337053in}}%
\pgfpathcurveto{\pgfqpoint{2.214877in}{1.337053in}}{\pgfqpoint{2.225476in}{1.341443in}}{\pgfqpoint{2.233290in}{1.349257in}}%
\pgfpathcurveto{\pgfqpoint{2.241103in}{1.357070in}}{\pgfqpoint{2.245494in}{1.367669in}}{\pgfqpoint{2.245494in}{1.378720in}}%
\pgfpathcurveto{\pgfqpoint{2.245494in}{1.389770in}}{\pgfqpoint{2.241103in}{1.400369in}}{\pgfqpoint{2.233290in}{1.408182in}}%
\pgfpathcurveto{\pgfqpoint{2.225476in}{1.415996in}}{\pgfqpoint{2.214877in}{1.420386in}}{\pgfqpoint{2.203827in}{1.420386in}}%
\pgfpathcurveto{\pgfqpoint{2.192777in}{1.420386in}}{\pgfqpoint{2.182178in}{1.415996in}}{\pgfqpoint{2.174364in}{1.408182in}}%
\pgfpathcurveto{\pgfqpoint{2.166550in}{1.400369in}}{\pgfqpoint{2.162160in}{1.389770in}}{\pgfqpoint{2.162160in}{1.378720in}}%
\pgfpathcurveto{\pgfqpoint{2.162160in}{1.367669in}}{\pgfqpoint{2.166550in}{1.357070in}}{\pgfqpoint{2.174364in}{1.349257in}}%
\pgfpathcurveto{\pgfqpoint{2.182178in}{1.341443in}}{\pgfqpoint{2.192777in}{1.337053in}}{\pgfqpoint{2.203827in}{1.337053in}}%
\pgfpathlineto{\pgfqpoint{2.203827in}{1.337053in}}%
\pgfpathclose%
\pgfusepath{stroke}%
\end{pgfscope}%
\begin{pgfscope}%
\pgfpathrectangle{\pgfqpoint{0.494722in}{0.437222in}}{\pgfqpoint{6.275590in}{5.159444in}}%
\pgfusepath{clip}%
\pgfsetbuttcap%
\pgfsetroundjoin%
\pgfsetlinewidth{1.003750pt}%
\definecolor{currentstroke}{rgb}{0.827451,0.827451,0.827451}%
\pgfsetstrokecolor{currentstroke}%
\pgfsetstrokeopacity{0.800000}%
\pgfsetdash{}{0pt}%
\pgfpathmoveto{\pgfqpoint{1.545871in}{1.889457in}}%
\pgfpathcurveto{\pgfqpoint{1.556921in}{1.889457in}}{\pgfqpoint{1.567520in}{1.893848in}}{\pgfqpoint{1.575334in}{1.901661in}}%
\pgfpathcurveto{\pgfqpoint{1.583148in}{1.909475in}}{\pgfqpoint{1.587538in}{1.920074in}}{\pgfqpoint{1.587538in}{1.931124in}}%
\pgfpathcurveto{\pgfqpoint{1.587538in}{1.942174in}}{\pgfqpoint{1.583148in}{1.952773in}}{\pgfqpoint{1.575334in}{1.960587in}}%
\pgfpathcurveto{\pgfqpoint{1.567520in}{1.968400in}}{\pgfqpoint{1.556921in}{1.972791in}}{\pgfqpoint{1.545871in}{1.972791in}}%
\pgfpathcurveto{\pgfqpoint{1.534821in}{1.972791in}}{\pgfqpoint{1.524222in}{1.968400in}}{\pgfqpoint{1.516408in}{1.960587in}}%
\pgfpathcurveto{\pgfqpoint{1.508595in}{1.952773in}}{\pgfqpoint{1.504204in}{1.942174in}}{\pgfqpoint{1.504204in}{1.931124in}}%
\pgfpathcurveto{\pgfqpoint{1.504204in}{1.920074in}}{\pgfqpoint{1.508595in}{1.909475in}}{\pgfqpoint{1.516408in}{1.901661in}}%
\pgfpathcurveto{\pgfqpoint{1.524222in}{1.893848in}}{\pgfqpoint{1.534821in}{1.889457in}}{\pgfqpoint{1.545871in}{1.889457in}}%
\pgfpathlineto{\pgfqpoint{1.545871in}{1.889457in}}%
\pgfpathclose%
\pgfusepath{stroke}%
\end{pgfscope}%
\begin{pgfscope}%
\pgfpathrectangle{\pgfqpoint{0.494722in}{0.437222in}}{\pgfqpoint{6.275590in}{5.159444in}}%
\pgfusepath{clip}%
\pgfsetbuttcap%
\pgfsetroundjoin%
\pgfsetlinewidth{1.003750pt}%
\definecolor{currentstroke}{rgb}{0.827451,0.827451,0.827451}%
\pgfsetstrokecolor{currentstroke}%
\pgfsetstrokeopacity{0.800000}%
\pgfsetdash{}{0pt}%
\pgfpathmoveto{\pgfqpoint{2.652245in}{1.095772in}}%
\pgfpathcurveto{\pgfqpoint{2.663295in}{1.095772in}}{\pgfqpoint{2.673894in}{1.100162in}}{\pgfqpoint{2.681708in}{1.107976in}}%
\pgfpathcurveto{\pgfqpoint{2.689522in}{1.115790in}}{\pgfqpoint{2.693912in}{1.126389in}}{\pgfqpoint{2.693912in}{1.137439in}}%
\pgfpathcurveto{\pgfqpoint{2.693912in}{1.148489in}}{\pgfqpoint{2.689522in}{1.159088in}}{\pgfqpoint{2.681708in}{1.166902in}}%
\pgfpathcurveto{\pgfqpoint{2.673894in}{1.174715in}}{\pgfqpoint{2.663295in}{1.179106in}}{\pgfqpoint{2.652245in}{1.179106in}}%
\pgfpathcurveto{\pgfqpoint{2.641195in}{1.179106in}}{\pgfqpoint{2.630596in}{1.174715in}}{\pgfqpoint{2.622782in}{1.166902in}}%
\pgfpathcurveto{\pgfqpoint{2.614969in}{1.159088in}}{\pgfqpoint{2.610578in}{1.148489in}}{\pgfqpoint{2.610578in}{1.137439in}}%
\pgfpathcurveto{\pgfqpoint{2.610578in}{1.126389in}}{\pgfqpoint{2.614969in}{1.115790in}}{\pgfqpoint{2.622782in}{1.107976in}}%
\pgfpathcurveto{\pgfqpoint{2.630596in}{1.100162in}}{\pgfqpoint{2.641195in}{1.095772in}}{\pgfqpoint{2.652245in}{1.095772in}}%
\pgfpathlineto{\pgfqpoint{2.652245in}{1.095772in}}%
\pgfpathclose%
\pgfusepath{stroke}%
\end{pgfscope}%
\begin{pgfscope}%
\pgfpathrectangle{\pgfqpoint{0.494722in}{0.437222in}}{\pgfqpoint{6.275590in}{5.159444in}}%
\pgfusepath{clip}%
\pgfsetbuttcap%
\pgfsetroundjoin%
\pgfsetlinewidth{1.003750pt}%
\definecolor{currentstroke}{rgb}{0.827451,0.827451,0.827451}%
\pgfsetstrokecolor{currentstroke}%
\pgfsetstrokeopacity{0.800000}%
\pgfsetdash{}{0pt}%
\pgfpathmoveto{\pgfqpoint{3.250298in}{0.800705in}}%
\pgfpathcurveto{\pgfqpoint{3.261348in}{0.800705in}}{\pgfqpoint{3.271947in}{0.805096in}}{\pgfqpoint{3.279761in}{0.812909in}}%
\pgfpathcurveto{\pgfqpoint{3.287574in}{0.820723in}}{\pgfqpoint{3.291965in}{0.831322in}}{\pgfqpoint{3.291965in}{0.842372in}}%
\pgfpathcurveto{\pgfqpoint{3.291965in}{0.853422in}}{\pgfqpoint{3.287574in}{0.864021in}}{\pgfqpoint{3.279761in}{0.871835in}}%
\pgfpathcurveto{\pgfqpoint{3.271947in}{0.879649in}}{\pgfqpoint{3.261348in}{0.884039in}}{\pgfqpoint{3.250298in}{0.884039in}}%
\pgfpathcurveto{\pgfqpoint{3.239248in}{0.884039in}}{\pgfqpoint{3.228649in}{0.879649in}}{\pgfqpoint{3.220835in}{0.871835in}}%
\pgfpathcurveto{\pgfqpoint{3.213022in}{0.864021in}}{\pgfqpoint{3.208631in}{0.853422in}}{\pgfqpoint{3.208631in}{0.842372in}}%
\pgfpathcurveto{\pgfqpoint{3.208631in}{0.831322in}}{\pgfqpoint{3.213022in}{0.820723in}}{\pgfqpoint{3.220835in}{0.812909in}}%
\pgfpathcurveto{\pgfqpoint{3.228649in}{0.805096in}}{\pgfqpoint{3.239248in}{0.800705in}}{\pgfqpoint{3.250298in}{0.800705in}}%
\pgfpathlineto{\pgfqpoint{3.250298in}{0.800705in}}%
\pgfpathclose%
\pgfusepath{stroke}%
\end{pgfscope}%
\begin{pgfscope}%
\pgfpathrectangle{\pgfqpoint{0.494722in}{0.437222in}}{\pgfqpoint{6.275590in}{5.159444in}}%
\pgfusepath{clip}%
\pgfsetbuttcap%
\pgfsetroundjoin%
\pgfsetlinewidth{1.003750pt}%
\definecolor{currentstroke}{rgb}{0.827451,0.827451,0.827451}%
\pgfsetstrokecolor{currentstroke}%
\pgfsetstrokeopacity{0.800000}%
\pgfsetdash{}{0pt}%
\pgfpathmoveto{\pgfqpoint{3.503126in}{0.705979in}}%
\pgfpathcurveto{\pgfqpoint{3.514177in}{0.705979in}}{\pgfqpoint{3.524776in}{0.710369in}}{\pgfqpoint{3.532589in}{0.718183in}}%
\pgfpathcurveto{\pgfqpoint{3.540403in}{0.725996in}}{\pgfqpoint{3.544793in}{0.736595in}}{\pgfqpoint{3.544793in}{0.747645in}}%
\pgfpathcurveto{\pgfqpoint{3.544793in}{0.758696in}}{\pgfqpoint{3.540403in}{0.769295in}}{\pgfqpoint{3.532589in}{0.777108in}}%
\pgfpathcurveto{\pgfqpoint{3.524776in}{0.784922in}}{\pgfqpoint{3.514177in}{0.789312in}}{\pgfqpoint{3.503126in}{0.789312in}}%
\pgfpathcurveto{\pgfqpoint{3.492076in}{0.789312in}}{\pgfqpoint{3.481477in}{0.784922in}}{\pgfqpoint{3.473664in}{0.777108in}}%
\pgfpathcurveto{\pgfqpoint{3.465850in}{0.769295in}}{\pgfqpoint{3.461460in}{0.758696in}}{\pgfqpoint{3.461460in}{0.747645in}}%
\pgfpathcurveto{\pgfqpoint{3.461460in}{0.736595in}}{\pgfqpoint{3.465850in}{0.725996in}}{\pgfqpoint{3.473664in}{0.718183in}}%
\pgfpathcurveto{\pgfqpoint{3.481477in}{0.710369in}}{\pgfqpoint{3.492076in}{0.705979in}}{\pgfqpoint{3.503126in}{0.705979in}}%
\pgfpathlineto{\pgfqpoint{3.503126in}{0.705979in}}%
\pgfpathclose%
\pgfusepath{stroke}%
\end{pgfscope}%
\begin{pgfscope}%
\pgfpathrectangle{\pgfqpoint{0.494722in}{0.437222in}}{\pgfqpoint{6.275590in}{5.159444in}}%
\pgfusepath{clip}%
\pgfsetbuttcap%
\pgfsetroundjoin%
\pgfsetlinewidth{1.003750pt}%
\definecolor{currentstroke}{rgb}{0.827451,0.827451,0.827451}%
\pgfsetstrokecolor{currentstroke}%
\pgfsetstrokeopacity{0.800000}%
\pgfsetdash{}{0pt}%
\pgfpathmoveto{\pgfqpoint{1.207472in}{2.314347in}}%
\pgfpathcurveto{\pgfqpoint{1.218522in}{2.314347in}}{\pgfqpoint{1.229121in}{2.318737in}}{\pgfqpoint{1.236935in}{2.326551in}}%
\pgfpathcurveto{\pgfqpoint{1.244749in}{2.334365in}}{\pgfqpoint{1.249139in}{2.344964in}}{\pgfqpoint{1.249139in}{2.356014in}}%
\pgfpathcurveto{\pgfqpoint{1.249139in}{2.367064in}}{\pgfqpoint{1.244749in}{2.377663in}}{\pgfqpoint{1.236935in}{2.385477in}}%
\pgfpathcurveto{\pgfqpoint{1.229121in}{2.393290in}}{\pgfqpoint{1.218522in}{2.397680in}}{\pgfqpoint{1.207472in}{2.397680in}}%
\pgfpathcurveto{\pgfqpoint{1.196422in}{2.397680in}}{\pgfqpoint{1.185823in}{2.393290in}}{\pgfqpoint{1.178010in}{2.385477in}}%
\pgfpathcurveto{\pgfqpoint{1.170196in}{2.377663in}}{\pgfqpoint{1.165806in}{2.367064in}}{\pgfqpoint{1.165806in}{2.356014in}}%
\pgfpathcurveto{\pgfqpoint{1.165806in}{2.344964in}}{\pgfqpoint{1.170196in}{2.334365in}}{\pgfqpoint{1.178010in}{2.326551in}}%
\pgfpathcurveto{\pgfqpoint{1.185823in}{2.318737in}}{\pgfqpoint{1.196422in}{2.314347in}}{\pgfqpoint{1.207472in}{2.314347in}}%
\pgfpathlineto{\pgfqpoint{1.207472in}{2.314347in}}%
\pgfpathclose%
\pgfusepath{stroke}%
\end{pgfscope}%
\begin{pgfscope}%
\pgfpathrectangle{\pgfqpoint{0.494722in}{0.437222in}}{\pgfqpoint{6.275590in}{5.159444in}}%
\pgfusepath{clip}%
\pgfsetbuttcap%
\pgfsetroundjoin%
\pgfsetlinewidth{1.003750pt}%
\definecolor{currentstroke}{rgb}{0.827451,0.827451,0.827451}%
\pgfsetstrokecolor{currentstroke}%
\pgfsetstrokeopacity{0.800000}%
\pgfsetdash{}{0pt}%
\pgfpathmoveto{\pgfqpoint{3.460812in}{0.723768in}}%
\pgfpathcurveto{\pgfqpoint{3.471862in}{0.723768in}}{\pgfqpoint{3.482461in}{0.728158in}}{\pgfqpoint{3.490274in}{0.735972in}}%
\pgfpathcurveto{\pgfqpoint{3.498088in}{0.743785in}}{\pgfqpoint{3.502478in}{0.754384in}}{\pgfqpoint{3.502478in}{0.765434in}}%
\pgfpathcurveto{\pgfqpoint{3.502478in}{0.776484in}}{\pgfqpoint{3.498088in}{0.787083in}}{\pgfqpoint{3.490274in}{0.794897in}}%
\pgfpathcurveto{\pgfqpoint{3.482461in}{0.802711in}}{\pgfqpoint{3.471862in}{0.807101in}}{\pgfqpoint{3.460812in}{0.807101in}}%
\pgfpathcurveto{\pgfqpoint{3.449762in}{0.807101in}}{\pgfqpoint{3.439162in}{0.802711in}}{\pgfqpoint{3.431349in}{0.794897in}}%
\pgfpathcurveto{\pgfqpoint{3.423535in}{0.787083in}}{\pgfqpoint{3.419145in}{0.776484in}}{\pgfqpoint{3.419145in}{0.765434in}}%
\pgfpathcurveto{\pgfqpoint{3.419145in}{0.754384in}}{\pgfqpoint{3.423535in}{0.743785in}}{\pgfqpoint{3.431349in}{0.735972in}}%
\pgfpathcurveto{\pgfqpoint{3.439162in}{0.728158in}}{\pgfqpoint{3.449762in}{0.723768in}}{\pgfqpoint{3.460812in}{0.723768in}}%
\pgfpathlineto{\pgfqpoint{3.460812in}{0.723768in}}%
\pgfpathclose%
\pgfusepath{stroke}%
\end{pgfscope}%
\begin{pgfscope}%
\pgfpathrectangle{\pgfqpoint{0.494722in}{0.437222in}}{\pgfqpoint{6.275590in}{5.159444in}}%
\pgfusepath{clip}%
\pgfsetbuttcap%
\pgfsetroundjoin%
\pgfsetlinewidth{1.003750pt}%
\definecolor{currentstroke}{rgb}{0.827451,0.827451,0.827451}%
\pgfsetstrokecolor{currentstroke}%
\pgfsetstrokeopacity{0.800000}%
\pgfsetdash{}{0pt}%
\pgfpathmoveto{\pgfqpoint{0.495484in}{4.544725in}}%
\pgfpathcurveto{\pgfqpoint{0.506534in}{4.544725in}}{\pgfqpoint{0.517133in}{4.549116in}}{\pgfqpoint{0.524947in}{4.556929in}}%
\pgfpathcurveto{\pgfqpoint{0.532761in}{4.564743in}}{\pgfqpoint{0.537151in}{4.575342in}}{\pgfqpoint{0.537151in}{4.586392in}}%
\pgfpathcurveto{\pgfqpoint{0.537151in}{4.597442in}}{\pgfqpoint{0.532761in}{4.608041in}}{\pgfqpoint{0.524947in}{4.615855in}}%
\pgfpathcurveto{\pgfqpoint{0.517133in}{4.623668in}}{\pgfqpoint{0.506534in}{4.628059in}}{\pgfqpoint{0.495484in}{4.628059in}}%
\pgfpathcurveto{\pgfqpoint{0.484434in}{4.628059in}}{\pgfqpoint{0.473835in}{4.623668in}}{\pgfqpoint{0.466021in}{4.615855in}}%
\pgfpathcurveto{\pgfqpoint{0.458208in}{4.608041in}}{\pgfqpoint{0.453817in}{4.597442in}}{\pgfqpoint{0.453817in}{4.586392in}}%
\pgfpathcurveto{\pgfqpoint{0.453817in}{4.575342in}}{\pgfqpoint{0.458208in}{4.564743in}}{\pgfqpoint{0.466021in}{4.556929in}}%
\pgfpathcurveto{\pgfqpoint{0.473835in}{4.549116in}}{\pgfqpoint{0.484434in}{4.544725in}}{\pgfqpoint{0.495484in}{4.544725in}}%
\pgfpathlineto{\pgfqpoint{0.495484in}{4.544725in}}%
\pgfpathclose%
\pgfusepath{stroke}%
\end{pgfscope}%
\begin{pgfscope}%
\pgfpathrectangle{\pgfqpoint{0.494722in}{0.437222in}}{\pgfqpoint{6.275590in}{5.159444in}}%
\pgfusepath{clip}%
\pgfsetbuttcap%
\pgfsetroundjoin%
\pgfsetlinewidth{1.003750pt}%
\definecolor{currentstroke}{rgb}{0.827451,0.827451,0.827451}%
\pgfsetstrokecolor{currentstroke}%
\pgfsetstrokeopacity{0.800000}%
\pgfsetdash{}{0pt}%
\pgfpathmoveto{\pgfqpoint{5.741672in}{0.399556in}}%
\pgfpathcurveto{\pgfqpoint{5.752722in}{0.399556in}}{\pgfqpoint{5.763321in}{0.403947in}}{\pgfqpoint{5.771134in}{0.411760in}}%
\pgfpathcurveto{\pgfqpoint{5.778948in}{0.419574in}}{\pgfqpoint{5.783338in}{0.430173in}}{\pgfqpoint{5.783338in}{0.441223in}}%
\pgfpathcurveto{\pgfqpoint{5.783338in}{0.452273in}}{\pgfqpoint{5.778948in}{0.462872in}}{\pgfqpoint{5.771134in}{0.470686in}}%
\pgfpathcurveto{\pgfqpoint{5.763321in}{0.478499in}}{\pgfqpoint{5.752722in}{0.482890in}}{\pgfqpoint{5.741672in}{0.482890in}}%
\pgfpathcurveto{\pgfqpoint{5.730622in}{0.482890in}}{\pgfqpoint{5.720023in}{0.478499in}}{\pgfqpoint{5.712209in}{0.470686in}}%
\pgfpathcurveto{\pgfqpoint{5.704395in}{0.462872in}}{\pgfqpoint{5.700005in}{0.452273in}}{\pgfqpoint{5.700005in}{0.441223in}}%
\pgfpathcurveto{\pgfqpoint{5.700005in}{0.430173in}}{\pgfqpoint{5.704395in}{0.419574in}}{\pgfqpoint{5.712209in}{0.411760in}}%
\pgfpathcurveto{\pgfqpoint{5.720023in}{0.403947in}}{\pgfqpoint{5.730622in}{0.399556in}}{\pgfqpoint{5.741672in}{0.399556in}}%
\pgfusepath{stroke}%
\end{pgfscope}%
\begin{pgfscope}%
\pgfpathrectangle{\pgfqpoint{0.494722in}{0.437222in}}{\pgfqpoint{6.275590in}{5.159444in}}%
\pgfusepath{clip}%
\pgfsetbuttcap%
\pgfsetroundjoin%
\pgfsetlinewidth{1.003750pt}%
\definecolor{currentstroke}{rgb}{0.827451,0.827451,0.827451}%
\pgfsetstrokecolor{currentstroke}%
\pgfsetstrokeopacity{0.800000}%
\pgfsetdash{}{0pt}%
\pgfpathmoveto{\pgfqpoint{4.702565in}{0.471432in}}%
\pgfpathcurveto{\pgfqpoint{4.713615in}{0.471432in}}{\pgfqpoint{4.724214in}{0.475823in}}{\pgfqpoint{4.732028in}{0.483636in}}%
\pgfpathcurveto{\pgfqpoint{4.739841in}{0.491450in}}{\pgfqpoint{4.744232in}{0.502049in}}{\pgfqpoint{4.744232in}{0.513099in}}%
\pgfpathcurveto{\pgfqpoint{4.744232in}{0.524149in}}{\pgfqpoint{4.739841in}{0.534748in}}{\pgfqpoint{4.732028in}{0.542562in}}%
\pgfpathcurveto{\pgfqpoint{4.724214in}{0.550376in}}{\pgfqpoint{4.713615in}{0.554766in}}{\pgfqpoint{4.702565in}{0.554766in}}%
\pgfpathcurveto{\pgfqpoint{4.691515in}{0.554766in}}{\pgfqpoint{4.680916in}{0.550376in}}{\pgfqpoint{4.673102in}{0.542562in}}%
\pgfpathcurveto{\pgfqpoint{4.665289in}{0.534748in}}{\pgfqpoint{4.660898in}{0.524149in}}{\pgfqpoint{4.660898in}{0.513099in}}%
\pgfpathcurveto{\pgfqpoint{4.660898in}{0.502049in}}{\pgfqpoint{4.665289in}{0.491450in}}{\pgfqpoint{4.673102in}{0.483636in}}%
\pgfpathcurveto{\pgfqpoint{4.680916in}{0.475823in}}{\pgfqpoint{4.691515in}{0.471432in}}{\pgfqpoint{4.702565in}{0.471432in}}%
\pgfpathlineto{\pgfqpoint{4.702565in}{0.471432in}}%
\pgfpathclose%
\pgfusepath{stroke}%
\end{pgfscope}%
\begin{pgfscope}%
\pgfpathrectangle{\pgfqpoint{0.494722in}{0.437222in}}{\pgfqpoint{6.275590in}{5.159444in}}%
\pgfusepath{clip}%
\pgfsetbuttcap%
\pgfsetroundjoin%
\pgfsetlinewidth{1.003750pt}%
\definecolor{currentstroke}{rgb}{0.827451,0.827451,0.827451}%
\pgfsetstrokecolor{currentstroke}%
\pgfsetstrokeopacity{0.800000}%
\pgfsetdash{}{0pt}%
\pgfpathmoveto{\pgfqpoint{5.560565in}{0.403701in}}%
\pgfpathcurveto{\pgfqpoint{5.571615in}{0.403701in}}{\pgfqpoint{5.582214in}{0.408092in}}{\pgfqpoint{5.590028in}{0.415905in}}%
\pgfpathcurveto{\pgfqpoint{5.597842in}{0.423719in}}{\pgfqpoint{5.602232in}{0.434318in}}{\pgfqpoint{5.602232in}{0.445368in}}%
\pgfpathcurveto{\pgfqpoint{5.602232in}{0.456418in}}{\pgfqpoint{5.597842in}{0.467017in}}{\pgfqpoint{5.590028in}{0.474831in}}%
\pgfpathcurveto{\pgfqpoint{5.582214in}{0.482644in}}{\pgfqpoint{5.571615in}{0.487035in}}{\pgfqpoint{5.560565in}{0.487035in}}%
\pgfpathcurveto{\pgfqpoint{5.549515in}{0.487035in}}{\pgfqpoint{5.538916in}{0.482644in}}{\pgfqpoint{5.531102in}{0.474831in}}%
\pgfpathcurveto{\pgfqpoint{5.523289in}{0.467017in}}{\pgfqpoint{5.518898in}{0.456418in}}{\pgfqpoint{5.518898in}{0.445368in}}%
\pgfpathcurveto{\pgfqpoint{5.518898in}{0.434318in}}{\pgfqpoint{5.523289in}{0.423719in}}{\pgfqpoint{5.531102in}{0.415905in}}%
\pgfpathcurveto{\pgfqpoint{5.538916in}{0.408092in}}{\pgfqpoint{5.549515in}{0.403701in}}{\pgfqpoint{5.560565in}{0.403701in}}%
\pgfusepath{stroke}%
\end{pgfscope}%
\begin{pgfscope}%
\pgfpathrectangle{\pgfqpoint{0.494722in}{0.437222in}}{\pgfqpoint{6.275590in}{5.159444in}}%
\pgfusepath{clip}%
\pgfsetbuttcap%
\pgfsetroundjoin%
\pgfsetlinewidth{1.003750pt}%
\definecolor{currentstroke}{rgb}{0.827451,0.827451,0.827451}%
\pgfsetstrokecolor{currentstroke}%
\pgfsetstrokeopacity{0.800000}%
\pgfsetdash{}{0pt}%
\pgfpathmoveto{\pgfqpoint{5.183443in}{0.422556in}}%
\pgfpathcurveto{\pgfqpoint{5.194493in}{0.422556in}}{\pgfqpoint{5.205092in}{0.426946in}}{\pgfqpoint{5.212906in}{0.434760in}}%
\pgfpathcurveto{\pgfqpoint{5.220719in}{0.442573in}}{\pgfqpoint{5.225109in}{0.453172in}}{\pgfqpoint{5.225109in}{0.464222in}}%
\pgfpathcurveto{\pgfqpoint{5.225109in}{0.475273in}}{\pgfqpoint{5.220719in}{0.485872in}}{\pgfqpoint{5.212906in}{0.493685in}}%
\pgfpathcurveto{\pgfqpoint{5.205092in}{0.501499in}}{\pgfqpoint{5.194493in}{0.505889in}}{\pgfqpoint{5.183443in}{0.505889in}}%
\pgfpathcurveto{\pgfqpoint{5.172393in}{0.505889in}}{\pgfqpoint{5.161794in}{0.501499in}}{\pgfqpoint{5.153980in}{0.493685in}}%
\pgfpathcurveto{\pgfqpoint{5.146166in}{0.485872in}}{\pgfqpoint{5.141776in}{0.475273in}}{\pgfqpoint{5.141776in}{0.464222in}}%
\pgfpathcurveto{\pgfqpoint{5.141776in}{0.453172in}}{\pgfqpoint{5.146166in}{0.442573in}}{\pgfqpoint{5.153980in}{0.434760in}}%
\pgfpathcurveto{\pgfqpoint{5.161794in}{0.426946in}}{\pgfqpoint{5.172393in}{0.422556in}}{\pgfqpoint{5.183443in}{0.422556in}}%
\pgfusepath{stroke}%
\end{pgfscope}%
\begin{pgfscope}%
\pgfpathrectangle{\pgfqpoint{0.494722in}{0.437222in}}{\pgfqpoint{6.275590in}{5.159444in}}%
\pgfusepath{clip}%
\pgfsetbuttcap%
\pgfsetroundjoin%
\pgfsetlinewidth{1.003750pt}%
\definecolor{currentstroke}{rgb}{0.827451,0.827451,0.827451}%
\pgfsetstrokecolor{currentstroke}%
\pgfsetstrokeopacity{0.800000}%
\pgfsetdash{}{0pt}%
\pgfpathmoveto{\pgfqpoint{0.501219in}{4.373826in}}%
\pgfpathcurveto{\pgfqpoint{0.512269in}{4.373826in}}{\pgfqpoint{0.522868in}{4.378216in}}{\pgfqpoint{0.530682in}{4.386030in}}%
\pgfpathcurveto{\pgfqpoint{0.538495in}{4.393844in}}{\pgfqpoint{0.542886in}{4.404443in}}{\pgfqpoint{0.542886in}{4.415493in}}%
\pgfpathcurveto{\pgfqpoint{0.542886in}{4.426543in}}{\pgfqpoint{0.538495in}{4.437142in}}{\pgfqpoint{0.530682in}{4.444955in}}%
\pgfpathcurveto{\pgfqpoint{0.522868in}{4.452769in}}{\pgfqpoint{0.512269in}{4.457159in}}{\pgfqpoint{0.501219in}{4.457159in}}%
\pgfpathcurveto{\pgfqpoint{0.490169in}{4.457159in}}{\pgfqpoint{0.479570in}{4.452769in}}{\pgfqpoint{0.471756in}{4.444955in}}%
\pgfpathcurveto{\pgfqpoint{0.463943in}{4.437142in}}{\pgfqpoint{0.459552in}{4.426543in}}{\pgfqpoint{0.459552in}{4.415493in}}%
\pgfpathcurveto{\pgfqpoint{0.459552in}{4.404443in}}{\pgfqpoint{0.463943in}{4.393844in}}{\pgfqpoint{0.471756in}{4.386030in}}%
\pgfpathcurveto{\pgfqpoint{0.479570in}{4.378216in}}{\pgfqpoint{0.490169in}{4.373826in}}{\pgfqpoint{0.501219in}{4.373826in}}%
\pgfpathlineto{\pgfqpoint{0.501219in}{4.373826in}}%
\pgfpathclose%
\pgfusepath{stroke}%
\end{pgfscope}%
\begin{pgfscope}%
\pgfpathrectangle{\pgfqpoint{0.494722in}{0.437222in}}{\pgfqpoint{6.275590in}{5.159444in}}%
\pgfusepath{clip}%
\pgfsetbuttcap%
\pgfsetroundjoin%
\pgfsetlinewidth{1.003750pt}%
\definecolor{currentstroke}{rgb}{0.827451,0.827451,0.827451}%
\pgfsetstrokecolor{currentstroke}%
\pgfsetstrokeopacity{0.800000}%
\pgfsetdash{}{0pt}%
\pgfpathmoveto{\pgfqpoint{4.942939in}{0.453295in}}%
\pgfpathcurveto{\pgfqpoint{4.953989in}{0.453295in}}{\pgfqpoint{4.964589in}{0.457685in}}{\pgfqpoint{4.972402in}{0.465499in}}%
\pgfpathcurveto{\pgfqpoint{4.980216in}{0.473313in}}{\pgfqpoint{4.984606in}{0.483912in}}{\pgfqpoint{4.984606in}{0.494962in}}%
\pgfpathcurveto{\pgfqpoint{4.984606in}{0.506012in}}{\pgfqpoint{4.980216in}{0.516611in}}{\pgfqpoint{4.972402in}{0.524425in}}%
\pgfpathcurveto{\pgfqpoint{4.964589in}{0.532238in}}{\pgfqpoint{4.953989in}{0.536628in}}{\pgfqpoint{4.942939in}{0.536628in}}%
\pgfpathcurveto{\pgfqpoint{4.931889in}{0.536628in}}{\pgfqpoint{4.921290in}{0.532238in}}{\pgfqpoint{4.913477in}{0.524425in}}%
\pgfpathcurveto{\pgfqpoint{4.905663in}{0.516611in}}{\pgfqpoint{4.901273in}{0.506012in}}{\pgfqpoint{4.901273in}{0.494962in}}%
\pgfpathcurveto{\pgfqpoint{4.901273in}{0.483912in}}{\pgfqpoint{4.905663in}{0.473313in}}{\pgfqpoint{4.913477in}{0.465499in}}%
\pgfpathcurveto{\pgfqpoint{4.921290in}{0.457685in}}{\pgfqpoint{4.931889in}{0.453295in}}{\pgfqpoint{4.942939in}{0.453295in}}%
\pgfpathlineto{\pgfqpoint{4.942939in}{0.453295in}}%
\pgfpathclose%
\pgfusepath{stroke}%
\end{pgfscope}%
\begin{pgfscope}%
\pgfpathrectangle{\pgfqpoint{0.494722in}{0.437222in}}{\pgfqpoint{6.275590in}{5.159444in}}%
\pgfusepath{clip}%
\pgfsetbuttcap%
\pgfsetroundjoin%
\pgfsetlinewidth{1.003750pt}%
\definecolor{currentstroke}{rgb}{0.827451,0.827451,0.827451}%
\pgfsetstrokecolor{currentstroke}%
\pgfsetstrokeopacity{0.800000}%
\pgfsetdash{}{0pt}%
\pgfpathmoveto{\pgfqpoint{0.520093in}{4.158669in}}%
\pgfpathcurveto{\pgfqpoint{0.531143in}{4.158669in}}{\pgfqpoint{0.541742in}{4.163059in}}{\pgfqpoint{0.549556in}{4.170873in}}%
\pgfpathcurveto{\pgfqpoint{0.557369in}{4.178687in}}{\pgfqpoint{0.561759in}{4.189286in}}{\pgfqpoint{0.561759in}{4.200336in}}%
\pgfpathcurveto{\pgfqpoint{0.561759in}{4.211386in}}{\pgfqpoint{0.557369in}{4.221985in}}{\pgfqpoint{0.549556in}{4.229799in}}%
\pgfpathcurveto{\pgfqpoint{0.541742in}{4.237612in}}{\pgfqpoint{0.531143in}{4.242003in}}{\pgfqpoint{0.520093in}{4.242003in}}%
\pgfpathcurveto{\pgfqpoint{0.509043in}{4.242003in}}{\pgfqpoint{0.498444in}{4.237612in}}{\pgfqpoint{0.490630in}{4.229799in}}%
\pgfpathcurveto{\pgfqpoint{0.482816in}{4.221985in}}{\pgfqpoint{0.478426in}{4.211386in}}{\pgfqpoint{0.478426in}{4.200336in}}%
\pgfpathcurveto{\pgfqpoint{0.478426in}{4.189286in}}{\pgfqpoint{0.482816in}{4.178687in}}{\pgfqpoint{0.490630in}{4.170873in}}%
\pgfpathcurveto{\pgfqpoint{0.498444in}{4.163059in}}{\pgfqpoint{0.509043in}{4.158669in}}{\pgfqpoint{0.520093in}{4.158669in}}%
\pgfpathlineto{\pgfqpoint{0.520093in}{4.158669in}}%
\pgfpathclose%
\pgfusepath{stroke}%
\end{pgfscope}%
\begin{pgfscope}%
\pgfpathrectangle{\pgfqpoint{0.494722in}{0.437222in}}{\pgfqpoint{6.275590in}{5.159444in}}%
\pgfusepath{clip}%
\pgfsetbuttcap%
\pgfsetroundjoin%
\pgfsetlinewidth{1.003750pt}%
\definecolor{currentstroke}{rgb}{0.827451,0.827451,0.827451}%
\pgfsetstrokecolor{currentstroke}%
\pgfsetstrokeopacity{0.800000}%
\pgfsetdash{}{0pt}%
\pgfpathmoveto{\pgfqpoint{3.709744in}{0.686390in}}%
\pgfpathcurveto{\pgfqpoint{3.720794in}{0.686390in}}{\pgfqpoint{3.731393in}{0.690780in}}{\pgfqpoint{3.739207in}{0.698593in}}%
\pgfpathcurveto{\pgfqpoint{3.747021in}{0.706407in}}{\pgfqpoint{3.751411in}{0.717006in}}{\pgfqpoint{3.751411in}{0.728056in}}%
\pgfpathcurveto{\pgfqpoint{3.751411in}{0.739106in}}{\pgfqpoint{3.747021in}{0.749705in}}{\pgfqpoint{3.739207in}{0.757519in}}%
\pgfpathcurveto{\pgfqpoint{3.731393in}{0.765333in}}{\pgfqpoint{3.720794in}{0.769723in}}{\pgfqpoint{3.709744in}{0.769723in}}%
\pgfpathcurveto{\pgfqpoint{3.698694in}{0.769723in}}{\pgfqpoint{3.688095in}{0.765333in}}{\pgfqpoint{3.680281in}{0.757519in}}%
\pgfpathcurveto{\pgfqpoint{3.672468in}{0.749705in}}{\pgfqpoint{3.668077in}{0.739106in}}{\pgfqpoint{3.668077in}{0.728056in}}%
\pgfpathcurveto{\pgfqpoint{3.668077in}{0.717006in}}{\pgfqpoint{3.672468in}{0.706407in}}{\pgfqpoint{3.680281in}{0.698593in}}%
\pgfpathcurveto{\pgfqpoint{3.688095in}{0.690780in}}{\pgfqpoint{3.698694in}{0.686390in}}{\pgfqpoint{3.709744in}{0.686390in}}%
\pgfpathlineto{\pgfqpoint{3.709744in}{0.686390in}}%
\pgfpathclose%
\pgfusepath{stroke}%
\end{pgfscope}%
\begin{pgfscope}%
\pgfpathrectangle{\pgfqpoint{0.494722in}{0.437222in}}{\pgfqpoint{6.275590in}{5.159444in}}%
\pgfusepath{clip}%
\pgfsetbuttcap%
\pgfsetroundjoin%
\pgfsetlinewidth{1.003750pt}%
\definecolor{currentstroke}{rgb}{0.827451,0.827451,0.827451}%
\pgfsetstrokecolor{currentstroke}%
\pgfsetstrokeopacity{0.800000}%
\pgfsetdash{}{0pt}%
\pgfpathmoveto{\pgfqpoint{2.765798in}{1.011077in}}%
\pgfpathcurveto{\pgfqpoint{2.776848in}{1.011077in}}{\pgfqpoint{2.787447in}{1.015467in}}{\pgfqpoint{2.795261in}{1.023280in}}%
\pgfpathcurveto{\pgfqpoint{2.803075in}{1.031094in}}{\pgfqpoint{2.807465in}{1.041693in}}{\pgfqpoint{2.807465in}{1.052743in}}%
\pgfpathcurveto{\pgfqpoint{2.807465in}{1.063793in}}{\pgfqpoint{2.803075in}{1.074392in}}{\pgfqpoint{2.795261in}{1.082206in}}%
\pgfpathcurveto{\pgfqpoint{2.787447in}{1.090020in}}{\pgfqpoint{2.776848in}{1.094410in}}{\pgfqpoint{2.765798in}{1.094410in}}%
\pgfpathcurveto{\pgfqpoint{2.754748in}{1.094410in}}{\pgfqpoint{2.744149in}{1.090020in}}{\pgfqpoint{2.736335in}{1.082206in}}%
\pgfpathcurveto{\pgfqpoint{2.728522in}{1.074392in}}{\pgfqpoint{2.724132in}{1.063793in}}{\pgfqpoint{2.724132in}{1.052743in}}%
\pgfpathcurveto{\pgfqpoint{2.724132in}{1.041693in}}{\pgfqpoint{2.728522in}{1.031094in}}{\pgfqpoint{2.736335in}{1.023280in}}%
\pgfpathcurveto{\pgfqpoint{2.744149in}{1.015467in}}{\pgfqpoint{2.754748in}{1.011077in}}{\pgfqpoint{2.765798in}{1.011077in}}%
\pgfpathlineto{\pgfqpoint{2.765798in}{1.011077in}}%
\pgfpathclose%
\pgfusepath{stroke}%
\end{pgfscope}%
\begin{pgfscope}%
\pgfpathrectangle{\pgfqpoint{0.494722in}{0.437222in}}{\pgfqpoint{6.275590in}{5.159444in}}%
\pgfusepath{clip}%
\pgfsetbuttcap%
\pgfsetroundjoin%
\pgfsetlinewidth{1.003750pt}%
\definecolor{currentstroke}{rgb}{0.827451,0.827451,0.827451}%
\pgfsetstrokecolor{currentstroke}%
\pgfsetstrokeopacity{0.800000}%
\pgfsetdash{}{0pt}%
\pgfpathmoveto{\pgfqpoint{3.250298in}{0.800705in}}%
\pgfpathcurveto{\pgfqpoint{3.261348in}{0.800705in}}{\pgfqpoint{3.271947in}{0.805096in}}{\pgfqpoint{3.279761in}{0.812909in}}%
\pgfpathcurveto{\pgfqpoint{3.287574in}{0.820723in}}{\pgfqpoint{3.291965in}{0.831322in}}{\pgfqpoint{3.291965in}{0.842372in}}%
\pgfpathcurveto{\pgfqpoint{3.291965in}{0.853422in}}{\pgfqpoint{3.287574in}{0.864021in}}{\pgfqpoint{3.279761in}{0.871835in}}%
\pgfpathcurveto{\pgfqpoint{3.271947in}{0.879649in}}{\pgfqpoint{3.261348in}{0.884039in}}{\pgfqpoint{3.250298in}{0.884039in}}%
\pgfpathcurveto{\pgfqpoint{3.239248in}{0.884039in}}{\pgfqpoint{3.228649in}{0.879649in}}{\pgfqpoint{3.220835in}{0.871835in}}%
\pgfpathcurveto{\pgfqpoint{3.213022in}{0.864021in}}{\pgfqpoint{3.208631in}{0.853422in}}{\pgfqpoint{3.208631in}{0.842372in}}%
\pgfpathcurveto{\pgfqpoint{3.208631in}{0.831322in}}{\pgfqpoint{3.213022in}{0.820723in}}{\pgfqpoint{3.220835in}{0.812909in}}%
\pgfpathcurveto{\pgfqpoint{3.228649in}{0.805096in}}{\pgfqpoint{3.239248in}{0.800705in}}{\pgfqpoint{3.250298in}{0.800705in}}%
\pgfpathlineto{\pgfqpoint{3.250298in}{0.800705in}}%
\pgfpathclose%
\pgfusepath{stroke}%
\end{pgfscope}%
\begin{pgfscope}%
\pgfpathrectangle{\pgfqpoint{0.494722in}{0.437222in}}{\pgfqpoint{6.275590in}{5.159444in}}%
\pgfusepath{clip}%
\pgfsetbuttcap%
\pgfsetroundjoin%
\pgfsetlinewidth{1.003750pt}%
\definecolor{currentstroke}{rgb}{0.827451,0.827451,0.827451}%
\pgfsetstrokecolor{currentstroke}%
\pgfsetstrokeopacity{0.800000}%
\pgfsetdash{}{0pt}%
\pgfpathmoveto{\pgfqpoint{0.575099in}{3.779944in}}%
\pgfpathcurveto{\pgfqpoint{0.586149in}{3.779944in}}{\pgfqpoint{0.596748in}{3.784334in}}{\pgfqpoint{0.604561in}{3.792148in}}%
\pgfpathcurveto{\pgfqpoint{0.612375in}{3.799961in}}{\pgfqpoint{0.616765in}{3.810560in}}{\pgfqpoint{0.616765in}{3.821610in}}%
\pgfpathcurveto{\pgfqpoint{0.616765in}{3.832661in}}{\pgfqpoint{0.612375in}{3.843260in}}{\pgfqpoint{0.604561in}{3.851073in}}%
\pgfpathcurveto{\pgfqpoint{0.596748in}{3.858887in}}{\pgfqpoint{0.586149in}{3.863277in}}{\pgfqpoint{0.575099in}{3.863277in}}%
\pgfpathcurveto{\pgfqpoint{0.564048in}{3.863277in}}{\pgfqpoint{0.553449in}{3.858887in}}{\pgfqpoint{0.545636in}{3.851073in}}%
\pgfpathcurveto{\pgfqpoint{0.537822in}{3.843260in}}{\pgfqpoint{0.533432in}{3.832661in}}{\pgfqpoint{0.533432in}{3.821610in}}%
\pgfpathcurveto{\pgfqpoint{0.533432in}{3.810560in}}{\pgfqpoint{0.537822in}{3.799961in}}{\pgfqpoint{0.545636in}{3.792148in}}%
\pgfpathcurveto{\pgfqpoint{0.553449in}{3.784334in}}{\pgfqpoint{0.564048in}{3.779944in}}{\pgfqpoint{0.575099in}{3.779944in}}%
\pgfpathlineto{\pgfqpoint{0.575099in}{3.779944in}}%
\pgfpathclose%
\pgfusepath{stroke}%
\end{pgfscope}%
\begin{pgfscope}%
\pgfpathrectangle{\pgfqpoint{0.494722in}{0.437222in}}{\pgfqpoint{6.275590in}{5.159444in}}%
\pgfusepath{clip}%
\pgfsetbuttcap%
\pgfsetroundjoin%
\pgfsetlinewidth{1.003750pt}%
\definecolor{currentstroke}{rgb}{0.827451,0.827451,0.827451}%
\pgfsetstrokecolor{currentstroke}%
\pgfsetstrokeopacity{0.800000}%
\pgfsetdash{}{0pt}%
\pgfpathmoveto{\pgfqpoint{1.133504in}{2.415631in}}%
\pgfpathcurveto{\pgfqpoint{1.144554in}{2.415631in}}{\pgfqpoint{1.155153in}{2.420021in}}{\pgfqpoint{1.162967in}{2.427835in}}%
\pgfpathcurveto{\pgfqpoint{1.170780in}{2.435648in}}{\pgfqpoint{1.175170in}{2.446247in}}{\pgfqpoint{1.175170in}{2.457297in}}%
\pgfpathcurveto{\pgfqpoint{1.175170in}{2.468347in}}{\pgfqpoint{1.170780in}{2.478946in}}{\pgfqpoint{1.162967in}{2.486760in}}%
\pgfpathcurveto{\pgfqpoint{1.155153in}{2.494574in}}{\pgfqpoint{1.144554in}{2.498964in}}{\pgfqpoint{1.133504in}{2.498964in}}%
\pgfpathcurveto{\pgfqpoint{1.122454in}{2.498964in}}{\pgfqpoint{1.111855in}{2.494574in}}{\pgfqpoint{1.104041in}{2.486760in}}%
\pgfpathcurveto{\pgfqpoint{1.096227in}{2.478946in}}{\pgfqpoint{1.091837in}{2.468347in}}{\pgfqpoint{1.091837in}{2.457297in}}%
\pgfpathcurveto{\pgfqpoint{1.091837in}{2.446247in}}{\pgfqpoint{1.096227in}{2.435648in}}{\pgfqpoint{1.104041in}{2.427835in}}%
\pgfpathcurveto{\pgfqpoint{1.111855in}{2.420021in}}{\pgfqpoint{1.122454in}{2.415631in}}{\pgfqpoint{1.133504in}{2.415631in}}%
\pgfpathlineto{\pgfqpoint{1.133504in}{2.415631in}}%
\pgfpathclose%
\pgfusepath{stroke}%
\end{pgfscope}%
\begin{pgfscope}%
\pgfpathrectangle{\pgfqpoint{0.494722in}{0.437222in}}{\pgfqpoint{6.275590in}{5.159444in}}%
\pgfusepath{clip}%
\pgfsetbuttcap%
\pgfsetroundjoin%
\pgfsetlinewidth{1.003750pt}%
\definecolor{currentstroke}{rgb}{0.827451,0.827451,0.827451}%
\pgfsetstrokecolor{currentstroke}%
\pgfsetstrokeopacity{0.800000}%
\pgfsetdash{}{0pt}%
\pgfpathmoveto{\pgfqpoint{1.545871in}{1.889457in}}%
\pgfpathcurveto{\pgfqpoint{1.556921in}{1.889457in}}{\pgfqpoint{1.567520in}{1.893848in}}{\pgfqpoint{1.575334in}{1.901661in}}%
\pgfpathcurveto{\pgfqpoint{1.583148in}{1.909475in}}{\pgfqpoint{1.587538in}{1.920074in}}{\pgfqpoint{1.587538in}{1.931124in}}%
\pgfpathcurveto{\pgfqpoint{1.587538in}{1.942174in}}{\pgfqpoint{1.583148in}{1.952773in}}{\pgfqpoint{1.575334in}{1.960587in}}%
\pgfpathcurveto{\pgfqpoint{1.567520in}{1.968400in}}{\pgfqpoint{1.556921in}{1.972791in}}{\pgfqpoint{1.545871in}{1.972791in}}%
\pgfpathcurveto{\pgfqpoint{1.534821in}{1.972791in}}{\pgfqpoint{1.524222in}{1.968400in}}{\pgfqpoint{1.516408in}{1.960587in}}%
\pgfpathcurveto{\pgfqpoint{1.508595in}{1.952773in}}{\pgfqpoint{1.504204in}{1.942174in}}{\pgfqpoint{1.504204in}{1.931124in}}%
\pgfpathcurveto{\pgfqpoint{1.504204in}{1.920074in}}{\pgfqpoint{1.508595in}{1.909475in}}{\pgfqpoint{1.516408in}{1.901661in}}%
\pgfpathcurveto{\pgfqpoint{1.524222in}{1.893848in}}{\pgfqpoint{1.534821in}{1.889457in}}{\pgfqpoint{1.545871in}{1.889457in}}%
\pgfpathlineto{\pgfqpoint{1.545871in}{1.889457in}}%
\pgfpathclose%
\pgfusepath{stroke}%
\end{pgfscope}%
\begin{pgfscope}%
\pgfpathrectangle{\pgfqpoint{0.494722in}{0.437222in}}{\pgfqpoint{6.275590in}{5.159444in}}%
\pgfusepath{clip}%
\pgfsetbuttcap%
\pgfsetroundjoin%
\pgfsetlinewidth{1.003750pt}%
\definecolor{currentstroke}{rgb}{0.827451,0.827451,0.827451}%
\pgfsetstrokecolor{currentstroke}%
\pgfsetstrokeopacity{0.800000}%
\pgfsetdash{}{0pt}%
\pgfpathmoveto{\pgfqpoint{3.527306in}{0.699588in}}%
\pgfpathcurveto{\pgfqpoint{3.538356in}{0.699588in}}{\pgfqpoint{3.548955in}{0.703979in}}{\pgfqpoint{3.556769in}{0.711792in}}%
\pgfpathcurveto{\pgfqpoint{3.564583in}{0.719606in}}{\pgfqpoint{3.568973in}{0.730205in}}{\pgfqpoint{3.568973in}{0.741255in}}%
\pgfpathcurveto{\pgfqpoint{3.568973in}{0.752305in}}{\pgfqpoint{3.564583in}{0.762904in}}{\pgfqpoint{3.556769in}{0.770718in}}%
\pgfpathcurveto{\pgfqpoint{3.548955in}{0.778532in}}{\pgfqpoint{3.538356in}{0.782922in}}{\pgfqpoint{3.527306in}{0.782922in}}%
\pgfpathcurveto{\pgfqpoint{3.516256in}{0.782922in}}{\pgfqpoint{3.505657in}{0.778532in}}{\pgfqpoint{3.497843in}{0.770718in}}%
\pgfpathcurveto{\pgfqpoint{3.490030in}{0.762904in}}{\pgfqpoint{3.485640in}{0.752305in}}{\pgfqpoint{3.485640in}{0.741255in}}%
\pgfpathcurveto{\pgfqpoint{3.485640in}{0.730205in}}{\pgfqpoint{3.490030in}{0.719606in}}{\pgfqpoint{3.497843in}{0.711792in}}%
\pgfpathcurveto{\pgfqpoint{3.505657in}{0.703979in}}{\pgfqpoint{3.516256in}{0.699588in}}{\pgfqpoint{3.527306in}{0.699588in}}%
\pgfpathlineto{\pgfqpoint{3.527306in}{0.699588in}}%
\pgfpathclose%
\pgfusepath{stroke}%
\end{pgfscope}%
\begin{pgfscope}%
\pgfpathrectangle{\pgfqpoint{0.494722in}{0.437222in}}{\pgfqpoint{6.275590in}{5.159444in}}%
\pgfusepath{clip}%
\pgfsetbuttcap%
\pgfsetroundjoin%
\pgfsetlinewidth{1.003750pt}%
\definecolor{currentstroke}{rgb}{0.827451,0.827451,0.827451}%
\pgfsetstrokecolor{currentstroke}%
\pgfsetstrokeopacity{0.800000}%
\pgfsetdash{}{0pt}%
\pgfpathmoveto{\pgfqpoint{2.308193in}{1.269247in}}%
\pgfpathcurveto{\pgfqpoint{2.319243in}{1.269247in}}{\pgfqpoint{2.329842in}{1.273637in}}{\pgfqpoint{2.337656in}{1.281451in}}%
\pgfpathcurveto{\pgfqpoint{2.345469in}{1.289264in}}{\pgfqpoint{2.349860in}{1.299864in}}{\pgfqpoint{2.349860in}{1.310914in}}%
\pgfpathcurveto{\pgfqpoint{2.349860in}{1.321964in}}{\pgfqpoint{2.345469in}{1.332563in}}{\pgfqpoint{2.337656in}{1.340376in}}%
\pgfpathcurveto{\pgfqpoint{2.329842in}{1.348190in}}{\pgfqpoint{2.319243in}{1.352580in}}{\pgfqpoint{2.308193in}{1.352580in}}%
\pgfpathcurveto{\pgfqpoint{2.297143in}{1.352580in}}{\pgfqpoint{2.286544in}{1.348190in}}{\pgfqpoint{2.278730in}{1.340376in}}%
\pgfpathcurveto{\pgfqpoint{2.270917in}{1.332563in}}{\pgfqpoint{2.266526in}{1.321964in}}{\pgfqpoint{2.266526in}{1.310914in}}%
\pgfpathcurveto{\pgfqpoint{2.266526in}{1.299864in}}{\pgfqpoint{2.270917in}{1.289264in}}{\pgfqpoint{2.278730in}{1.281451in}}%
\pgfpathcurveto{\pgfqpoint{2.286544in}{1.273637in}}{\pgfqpoint{2.297143in}{1.269247in}}{\pgfqpoint{2.308193in}{1.269247in}}%
\pgfpathlineto{\pgfqpoint{2.308193in}{1.269247in}}%
\pgfpathclose%
\pgfusepath{stroke}%
\end{pgfscope}%
\begin{pgfscope}%
\pgfpathrectangle{\pgfqpoint{0.494722in}{0.437222in}}{\pgfqpoint{6.275590in}{5.159444in}}%
\pgfusepath{clip}%
\pgfsetbuttcap%
\pgfsetroundjoin%
\pgfsetlinewidth{1.003750pt}%
\definecolor{currentstroke}{rgb}{0.827451,0.827451,0.827451}%
\pgfsetstrokecolor{currentstroke}%
\pgfsetstrokeopacity{0.800000}%
\pgfsetdash{}{0pt}%
\pgfpathmoveto{\pgfqpoint{0.903250in}{2.772216in}}%
\pgfpathcurveto{\pgfqpoint{0.914300in}{2.772216in}}{\pgfqpoint{0.924899in}{2.776607in}}{\pgfqpoint{0.932712in}{2.784420in}}%
\pgfpathcurveto{\pgfqpoint{0.940526in}{2.792234in}}{\pgfqpoint{0.944916in}{2.802833in}}{\pgfqpoint{0.944916in}{2.813883in}}%
\pgfpathcurveto{\pgfqpoint{0.944916in}{2.824933in}}{\pgfqpoint{0.940526in}{2.835532in}}{\pgfqpoint{0.932712in}{2.843346in}}%
\pgfpathcurveto{\pgfqpoint{0.924899in}{2.851159in}}{\pgfqpoint{0.914300in}{2.855550in}}{\pgfqpoint{0.903250in}{2.855550in}}%
\pgfpathcurveto{\pgfqpoint{0.892199in}{2.855550in}}{\pgfqpoint{0.881600in}{2.851159in}}{\pgfqpoint{0.873787in}{2.843346in}}%
\pgfpathcurveto{\pgfqpoint{0.865973in}{2.835532in}}{\pgfqpoint{0.861583in}{2.824933in}}{\pgfqpoint{0.861583in}{2.813883in}}%
\pgfpathcurveto{\pgfqpoint{0.861583in}{2.802833in}}{\pgfqpoint{0.865973in}{2.792234in}}{\pgfqpoint{0.873787in}{2.784420in}}%
\pgfpathcurveto{\pgfqpoint{0.881600in}{2.776607in}}{\pgfqpoint{0.892199in}{2.772216in}}{\pgfqpoint{0.903250in}{2.772216in}}%
\pgfpathlineto{\pgfqpoint{0.903250in}{2.772216in}}%
\pgfpathclose%
\pgfusepath{stroke}%
\end{pgfscope}%
\begin{pgfscope}%
\pgfpathrectangle{\pgfqpoint{0.494722in}{0.437222in}}{\pgfqpoint{6.275590in}{5.159444in}}%
\pgfusepath{clip}%
\pgfsetbuttcap%
\pgfsetroundjoin%
\pgfsetlinewidth{1.003750pt}%
\definecolor{currentstroke}{rgb}{0.827451,0.827451,0.827451}%
\pgfsetstrokecolor{currentstroke}%
\pgfsetstrokeopacity{0.800000}%
\pgfsetdash{}{0pt}%
\pgfpathmoveto{\pgfqpoint{1.000236in}{2.738855in}}%
\pgfpathcurveto{\pgfqpoint{1.011286in}{2.738855in}}{\pgfqpoint{1.021885in}{2.743245in}}{\pgfqpoint{1.029699in}{2.751059in}}%
\pgfpathcurveto{\pgfqpoint{1.037513in}{2.758873in}}{\pgfqpoint{1.041903in}{2.769472in}}{\pgfqpoint{1.041903in}{2.780522in}}%
\pgfpathcurveto{\pgfqpoint{1.041903in}{2.791572in}}{\pgfqpoint{1.037513in}{2.802171in}}{\pgfqpoint{1.029699in}{2.809985in}}%
\pgfpathcurveto{\pgfqpoint{1.021885in}{2.817798in}}{\pgfqpoint{1.011286in}{2.822189in}}{\pgfqpoint{1.000236in}{2.822189in}}%
\pgfpathcurveto{\pgfqpoint{0.989186in}{2.822189in}}{\pgfqpoint{0.978587in}{2.817798in}}{\pgfqpoint{0.970773in}{2.809985in}}%
\pgfpathcurveto{\pgfqpoint{0.962960in}{2.802171in}}{\pgfqpoint{0.958570in}{2.791572in}}{\pgfqpoint{0.958570in}{2.780522in}}%
\pgfpathcurveto{\pgfqpoint{0.958570in}{2.769472in}}{\pgfqpoint{0.962960in}{2.758873in}}{\pgfqpoint{0.970773in}{2.751059in}}%
\pgfpathcurveto{\pgfqpoint{0.978587in}{2.743245in}}{\pgfqpoint{0.989186in}{2.738855in}}{\pgfqpoint{1.000236in}{2.738855in}}%
\pgfpathlineto{\pgfqpoint{1.000236in}{2.738855in}}%
\pgfpathclose%
\pgfusepath{stroke}%
\end{pgfscope}%
\begin{pgfscope}%
\pgfpathrectangle{\pgfqpoint{0.494722in}{0.437222in}}{\pgfqpoint{6.275590in}{5.159444in}}%
\pgfusepath{clip}%
\pgfsetbuttcap%
\pgfsetroundjoin%
\pgfsetlinewidth{1.003750pt}%
\definecolor{currentstroke}{rgb}{0.827451,0.827451,0.827451}%
\pgfsetstrokecolor{currentstroke}%
\pgfsetstrokeopacity{0.800000}%
\pgfsetdash{}{0pt}%
\pgfpathmoveto{\pgfqpoint{3.372979in}{0.744763in}}%
\pgfpathcurveto{\pgfqpoint{3.384029in}{0.744763in}}{\pgfqpoint{3.394628in}{0.749153in}}{\pgfqpoint{3.402442in}{0.756967in}}%
\pgfpathcurveto{\pgfqpoint{3.410255in}{0.764781in}}{\pgfqpoint{3.414646in}{0.775380in}}{\pgfqpoint{3.414646in}{0.786430in}}%
\pgfpathcurveto{\pgfqpoint{3.414646in}{0.797480in}}{\pgfqpoint{3.410255in}{0.808079in}}{\pgfqpoint{3.402442in}{0.815893in}}%
\pgfpathcurveto{\pgfqpoint{3.394628in}{0.823706in}}{\pgfqpoint{3.384029in}{0.828097in}}{\pgfqpoint{3.372979in}{0.828097in}}%
\pgfpathcurveto{\pgfqpoint{3.361929in}{0.828097in}}{\pgfqpoint{3.351330in}{0.823706in}}{\pgfqpoint{3.343516in}{0.815893in}}%
\pgfpathcurveto{\pgfqpoint{3.335703in}{0.808079in}}{\pgfqpoint{3.331312in}{0.797480in}}{\pgfqpoint{3.331312in}{0.786430in}}%
\pgfpathcurveto{\pgfqpoint{3.331312in}{0.775380in}}{\pgfqpoint{3.335703in}{0.764781in}}{\pgfqpoint{3.343516in}{0.756967in}}%
\pgfpathcurveto{\pgfqpoint{3.351330in}{0.749153in}}{\pgfqpoint{3.361929in}{0.744763in}}{\pgfqpoint{3.372979in}{0.744763in}}%
\pgfpathlineto{\pgfqpoint{3.372979in}{0.744763in}}%
\pgfpathclose%
\pgfusepath{stroke}%
\end{pgfscope}%
\begin{pgfscope}%
\pgfpathrectangle{\pgfqpoint{0.494722in}{0.437222in}}{\pgfqpoint{6.275590in}{5.159444in}}%
\pgfusepath{clip}%
\pgfsetbuttcap%
\pgfsetroundjoin%
\pgfsetlinewidth{1.003750pt}%
\definecolor{currentstroke}{rgb}{0.827451,0.827451,0.827451}%
\pgfsetstrokecolor{currentstroke}%
\pgfsetstrokeopacity{0.800000}%
\pgfsetdash{}{0pt}%
\pgfpathmoveto{\pgfqpoint{3.901710in}{0.604104in}}%
\pgfpathcurveto{\pgfqpoint{3.912761in}{0.604104in}}{\pgfqpoint{3.923360in}{0.608495in}}{\pgfqpoint{3.931173in}{0.616308in}}%
\pgfpathcurveto{\pgfqpoint{3.938987in}{0.624122in}}{\pgfqpoint{3.943377in}{0.634721in}}{\pgfqpoint{3.943377in}{0.645771in}}%
\pgfpathcurveto{\pgfqpoint{3.943377in}{0.656821in}}{\pgfqpoint{3.938987in}{0.667420in}}{\pgfqpoint{3.931173in}{0.675234in}}%
\pgfpathcurveto{\pgfqpoint{3.923360in}{0.683047in}}{\pgfqpoint{3.912761in}{0.687438in}}{\pgfqpoint{3.901710in}{0.687438in}}%
\pgfpathcurveto{\pgfqpoint{3.890660in}{0.687438in}}{\pgfqpoint{3.880061in}{0.683047in}}{\pgfqpoint{3.872248in}{0.675234in}}%
\pgfpathcurveto{\pgfqpoint{3.864434in}{0.667420in}}{\pgfqpoint{3.860044in}{0.656821in}}{\pgfqpoint{3.860044in}{0.645771in}}%
\pgfpathcurveto{\pgfqpoint{3.860044in}{0.634721in}}{\pgfqpoint{3.864434in}{0.624122in}}{\pgfqpoint{3.872248in}{0.616308in}}%
\pgfpathcurveto{\pgfqpoint{3.880061in}{0.608495in}}{\pgfqpoint{3.890660in}{0.604104in}}{\pgfqpoint{3.901710in}{0.604104in}}%
\pgfpathlineto{\pgfqpoint{3.901710in}{0.604104in}}%
\pgfpathclose%
\pgfusepath{stroke}%
\end{pgfscope}%
\begin{pgfscope}%
\pgfpathrectangle{\pgfqpoint{0.494722in}{0.437222in}}{\pgfqpoint{6.275590in}{5.159444in}}%
\pgfusepath{clip}%
\pgfsetbuttcap%
\pgfsetroundjoin%
\pgfsetlinewidth{1.003750pt}%
\definecolor{currentstroke}{rgb}{0.827451,0.827451,0.827451}%
\pgfsetstrokecolor{currentstroke}%
\pgfsetstrokeopacity{0.800000}%
\pgfsetdash{}{0pt}%
\pgfpathmoveto{\pgfqpoint{5.377588in}{0.405033in}}%
\pgfpathcurveto{\pgfqpoint{5.388638in}{0.405033in}}{\pgfqpoint{5.399237in}{0.409423in}}{\pgfqpoint{5.407050in}{0.417237in}}%
\pgfpathcurveto{\pgfqpoint{5.414864in}{0.425051in}}{\pgfqpoint{5.419254in}{0.435650in}}{\pgfqpoint{5.419254in}{0.446700in}}%
\pgfpathcurveto{\pgfqpoint{5.419254in}{0.457750in}}{\pgfqpoint{5.414864in}{0.468349in}}{\pgfqpoint{5.407050in}{0.476163in}}%
\pgfpathcurveto{\pgfqpoint{5.399237in}{0.483976in}}{\pgfqpoint{5.388638in}{0.488366in}}{\pgfqpoint{5.377588in}{0.488366in}}%
\pgfpathcurveto{\pgfqpoint{5.366537in}{0.488366in}}{\pgfqpoint{5.355938in}{0.483976in}}{\pgfqpoint{5.348125in}{0.476163in}}%
\pgfpathcurveto{\pgfqpoint{5.340311in}{0.468349in}}{\pgfqpoint{5.335921in}{0.457750in}}{\pgfqpoint{5.335921in}{0.446700in}}%
\pgfpathcurveto{\pgfqpoint{5.335921in}{0.435650in}}{\pgfqpoint{5.340311in}{0.425051in}}{\pgfqpoint{5.348125in}{0.417237in}}%
\pgfpathcurveto{\pgfqpoint{5.355938in}{0.409423in}}{\pgfqpoint{5.366537in}{0.405033in}}{\pgfqpoint{5.377588in}{0.405033in}}%
\pgfusepath{stroke}%
\end{pgfscope}%
\begin{pgfscope}%
\pgfpathrectangle{\pgfqpoint{0.494722in}{0.437222in}}{\pgfqpoint{6.275590in}{5.159444in}}%
\pgfusepath{clip}%
\pgfsetbuttcap%
\pgfsetroundjoin%
\pgfsetlinewidth{1.003750pt}%
\definecolor{currentstroke}{rgb}{0.827451,0.827451,0.827451}%
\pgfsetstrokecolor{currentstroke}%
\pgfsetstrokeopacity{0.800000}%
\pgfsetdash{}{0pt}%
\pgfpathmoveto{\pgfqpoint{4.586350in}{0.475898in}}%
\pgfpathcurveto{\pgfqpoint{4.597400in}{0.475898in}}{\pgfqpoint{4.607999in}{0.480288in}}{\pgfqpoint{4.615813in}{0.488101in}}%
\pgfpathcurveto{\pgfqpoint{4.623627in}{0.495915in}}{\pgfqpoint{4.628017in}{0.506514in}}{\pgfqpoint{4.628017in}{0.517564in}}%
\pgfpathcurveto{\pgfqpoint{4.628017in}{0.528614in}}{\pgfqpoint{4.623627in}{0.539213in}}{\pgfqpoint{4.615813in}{0.547027in}}%
\pgfpathcurveto{\pgfqpoint{4.607999in}{0.554841in}}{\pgfqpoint{4.597400in}{0.559231in}}{\pgfqpoint{4.586350in}{0.559231in}}%
\pgfpathcurveto{\pgfqpoint{4.575300in}{0.559231in}}{\pgfqpoint{4.564701in}{0.554841in}}{\pgfqpoint{4.556887in}{0.547027in}}%
\pgfpathcurveto{\pgfqpoint{4.549074in}{0.539213in}}{\pgfqpoint{4.544683in}{0.528614in}}{\pgfqpoint{4.544683in}{0.517564in}}%
\pgfpathcurveto{\pgfqpoint{4.544683in}{0.506514in}}{\pgfqpoint{4.549074in}{0.495915in}}{\pgfqpoint{4.556887in}{0.488101in}}%
\pgfpathcurveto{\pgfqpoint{4.564701in}{0.480288in}}{\pgfqpoint{4.575300in}{0.475898in}}{\pgfqpoint{4.586350in}{0.475898in}}%
\pgfpathlineto{\pgfqpoint{4.586350in}{0.475898in}}%
\pgfpathclose%
\pgfusepath{stroke}%
\end{pgfscope}%
\begin{pgfscope}%
\pgfpathrectangle{\pgfqpoint{0.494722in}{0.437222in}}{\pgfqpoint{6.275590in}{5.159444in}}%
\pgfusepath{clip}%
\pgfsetbuttcap%
\pgfsetroundjoin%
\pgfsetlinewidth{1.003750pt}%
\definecolor{currentstroke}{rgb}{0.827451,0.827451,0.827451}%
\pgfsetstrokecolor{currentstroke}%
\pgfsetstrokeopacity{0.800000}%
\pgfsetdash{}{0pt}%
\pgfpathmoveto{\pgfqpoint{4.119757in}{0.538399in}}%
\pgfpathcurveto{\pgfqpoint{4.130807in}{0.538399in}}{\pgfqpoint{4.141406in}{0.542789in}}{\pgfqpoint{4.149220in}{0.550603in}}%
\pgfpathcurveto{\pgfqpoint{4.157034in}{0.558417in}}{\pgfqpoint{4.161424in}{0.569016in}}{\pgfqpoint{4.161424in}{0.580066in}}%
\pgfpathcurveto{\pgfqpoint{4.161424in}{0.591116in}}{\pgfqpoint{4.157034in}{0.601715in}}{\pgfqpoint{4.149220in}{0.609529in}}%
\pgfpathcurveto{\pgfqpoint{4.141406in}{0.617342in}}{\pgfqpoint{4.130807in}{0.621732in}}{\pgfqpoint{4.119757in}{0.621732in}}%
\pgfpathcurveto{\pgfqpoint{4.108707in}{0.621732in}}{\pgfqpoint{4.098108in}{0.617342in}}{\pgfqpoint{4.090294in}{0.609529in}}%
\pgfpathcurveto{\pgfqpoint{4.082481in}{0.601715in}}{\pgfqpoint{4.078091in}{0.591116in}}{\pgfqpoint{4.078091in}{0.580066in}}%
\pgfpathcurveto{\pgfqpoint{4.078091in}{0.569016in}}{\pgfqpoint{4.082481in}{0.558417in}}{\pgfqpoint{4.090294in}{0.550603in}}%
\pgfpathcurveto{\pgfqpoint{4.098108in}{0.542789in}}{\pgfqpoint{4.108707in}{0.538399in}}{\pgfqpoint{4.119757in}{0.538399in}}%
\pgfpathlineto{\pgfqpoint{4.119757in}{0.538399in}}%
\pgfpathclose%
\pgfusepath{stroke}%
\end{pgfscope}%
\begin{pgfscope}%
\pgfpathrectangle{\pgfqpoint{0.494722in}{0.437222in}}{\pgfqpoint{6.275590in}{5.159444in}}%
\pgfusepath{clip}%
\pgfsetbuttcap%
\pgfsetroundjoin%
\pgfsetlinewidth{1.003750pt}%
\definecolor{currentstroke}{rgb}{0.827451,0.827451,0.827451}%
\pgfsetstrokecolor{currentstroke}%
\pgfsetstrokeopacity{0.800000}%
\pgfsetdash{}{0pt}%
\pgfpathmoveto{\pgfqpoint{2.081578in}{1.430058in}}%
\pgfpathcurveto{\pgfqpoint{2.092628in}{1.430058in}}{\pgfqpoint{2.103227in}{1.434449in}}{\pgfqpoint{2.111040in}{1.442262in}}%
\pgfpathcurveto{\pgfqpoint{2.118854in}{1.450076in}}{\pgfqpoint{2.123244in}{1.460675in}}{\pgfqpoint{2.123244in}{1.471725in}}%
\pgfpathcurveto{\pgfqpoint{2.123244in}{1.482775in}}{\pgfqpoint{2.118854in}{1.493374in}}{\pgfqpoint{2.111040in}{1.501188in}}%
\pgfpathcurveto{\pgfqpoint{2.103227in}{1.509001in}}{\pgfqpoint{2.092628in}{1.513392in}}{\pgfqpoint{2.081578in}{1.513392in}}%
\pgfpathcurveto{\pgfqpoint{2.070527in}{1.513392in}}{\pgfqpoint{2.059928in}{1.509001in}}{\pgfqpoint{2.052115in}{1.501188in}}%
\pgfpathcurveto{\pgfqpoint{2.044301in}{1.493374in}}{\pgfqpoint{2.039911in}{1.482775in}}{\pgfqpoint{2.039911in}{1.471725in}}%
\pgfpathcurveto{\pgfqpoint{2.039911in}{1.460675in}}{\pgfqpoint{2.044301in}{1.450076in}}{\pgfqpoint{2.052115in}{1.442262in}}%
\pgfpathcurveto{\pgfqpoint{2.059928in}{1.434449in}}{\pgfqpoint{2.070527in}{1.430058in}}{\pgfqpoint{2.081578in}{1.430058in}}%
\pgfpathlineto{\pgfqpoint{2.081578in}{1.430058in}}%
\pgfpathclose%
\pgfusepath{stroke}%
\end{pgfscope}%
\begin{pgfscope}%
\pgfpathrectangle{\pgfqpoint{0.494722in}{0.437222in}}{\pgfqpoint{6.275590in}{5.159444in}}%
\pgfusepath{clip}%
\pgfsetbuttcap%
\pgfsetroundjoin%
\pgfsetlinewidth{1.003750pt}%
\definecolor{currentstroke}{rgb}{0.827451,0.827451,0.827451}%
\pgfsetstrokecolor{currentstroke}%
\pgfsetstrokeopacity{0.800000}%
\pgfsetdash{}{0pt}%
\pgfpathmoveto{\pgfqpoint{1.113110in}{2.511719in}}%
\pgfpathcurveto{\pgfqpoint{1.124161in}{2.511719in}}{\pgfqpoint{1.134760in}{2.516109in}}{\pgfqpoint{1.142573in}{2.523923in}}%
\pgfpathcurveto{\pgfqpoint{1.150387in}{2.531736in}}{\pgfqpoint{1.154777in}{2.542335in}}{\pgfqpoint{1.154777in}{2.553385in}}%
\pgfpathcurveto{\pgfqpoint{1.154777in}{2.564435in}}{\pgfqpoint{1.150387in}{2.575034in}}{\pgfqpoint{1.142573in}{2.582848in}}%
\pgfpathcurveto{\pgfqpoint{1.134760in}{2.590662in}}{\pgfqpoint{1.124161in}{2.595052in}}{\pgfqpoint{1.113110in}{2.595052in}}%
\pgfpathcurveto{\pgfqpoint{1.102060in}{2.595052in}}{\pgfqpoint{1.091461in}{2.590662in}}{\pgfqpoint{1.083648in}{2.582848in}}%
\pgfpathcurveto{\pgfqpoint{1.075834in}{2.575034in}}{\pgfqpoint{1.071444in}{2.564435in}}{\pgfqpoint{1.071444in}{2.553385in}}%
\pgfpathcurveto{\pgfqpoint{1.071444in}{2.542335in}}{\pgfqpoint{1.075834in}{2.531736in}}{\pgfqpoint{1.083648in}{2.523923in}}%
\pgfpathcurveto{\pgfqpoint{1.091461in}{2.516109in}}{\pgfqpoint{1.102060in}{2.511719in}}{\pgfqpoint{1.113110in}{2.511719in}}%
\pgfpathlineto{\pgfqpoint{1.113110in}{2.511719in}}%
\pgfpathclose%
\pgfusepath{stroke}%
\end{pgfscope}%
\begin{pgfscope}%
\pgfpathrectangle{\pgfqpoint{0.494722in}{0.437222in}}{\pgfqpoint{6.275590in}{5.159444in}}%
\pgfusepath{clip}%
\pgfsetbuttcap%
\pgfsetroundjoin%
\pgfsetlinewidth{1.003750pt}%
\definecolor{currentstroke}{rgb}{0.827451,0.827451,0.827451}%
\pgfsetstrokecolor{currentstroke}%
\pgfsetstrokeopacity{0.800000}%
\pgfsetdash{}{0pt}%
\pgfpathmoveto{\pgfqpoint{1.325845in}{2.126824in}}%
\pgfpathcurveto{\pgfqpoint{1.336895in}{2.126824in}}{\pgfqpoint{1.347494in}{2.131214in}}{\pgfqpoint{1.355308in}{2.139028in}}%
\pgfpathcurveto{\pgfqpoint{1.363121in}{2.146842in}}{\pgfqpoint{1.367512in}{2.157441in}}{\pgfqpoint{1.367512in}{2.168491in}}%
\pgfpathcurveto{\pgfqpoint{1.367512in}{2.179541in}}{\pgfqpoint{1.363121in}{2.190140in}}{\pgfqpoint{1.355308in}{2.197954in}}%
\pgfpathcurveto{\pgfqpoint{1.347494in}{2.205767in}}{\pgfqpoint{1.336895in}{2.210157in}}{\pgfqpoint{1.325845in}{2.210157in}}%
\pgfpathcurveto{\pgfqpoint{1.314795in}{2.210157in}}{\pgfqpoint{1.304196in}{2.205767in}}{\pgfqpoint{1.296382in}{2.197954in}}%
\pgfpathcurveto{\pgfqpoint{1.288569in}{2.190140in}}{\pgfqpoint{1.284178in}{2.179541in}}{\pgfqpoint{1.284178in}{2.168491in}}%
\pgfpathcurveto{\pgfqpoint{1.284178in}{2.157441in}}{\pgfqpoint{1.288569in}{2.146842in}}{\pgfqpoint{1.296382in}{2.139028in}}%
\pgfpathcurveto{\pgfqpoint{1.304196in}{2.131214in}}{\pgfqpoint{1.314795in}{2.126824in}}{\pgfqpoint{1.325845in}{2.126824in}}%
\pgfpathlineto{\pgfqpoint{1.325845in}{2.126824in}}%
\pgfpathclose%
\pgfusepath{stroke}%
\end{pgfscope}%
\begin{pgfscope}%
\pgfpathrectangle{\pgfqpoint{0.494722in}{0.437222in}}{\pgfqpoint{6.275590in}{5.159444in}}%
\pgfusepath{clip}%
\pgfsetbuttcap%
\pgfsetroundjoin%
\pgfsetlinewidth{1.003750pt}%
\definecolor{currentstroke}{rgb}{0.827451,0.827451,0.827451}%
\pgfsetstrokecolor{currentstroke}%
\pgfsetstrokeopacity{0.800000}%
\pgfsetdash{}{0pt}%
\pgfpathmoveto{\pgfqpoint{1.006303in}{2.625866in}}%
\pgfpathcurveto{\pgfqpoint{1.017353in}{2.625866in}}{\pgfqpoint{1.027952in}{2.630257in}}{\pgfqpoint{1.035766in}{2.638070in}}%
\pgfpathcurveto{\pgfqpoint{1.043580in}{2.645884in}}{\pgfqpoint{1.047970in}{2.656483in}}{\pgfqpoint{1.047970in}{2.667533in}}%
\pgfpathcurveto{\pgfqpoint{1.047970in}{2.678583in}}{\pgfqpoint{1.043580in}{2.689182in}}{\pgfqpoint{1.035766in}{2.696996in}}%
\pgfpathcurveto{\pgfqpoint{1.027952in}{2.704810in}}{\pgfqpoint{1.017353in}{2.709200in}}{\pgfqpoint{1.006303in}{2.709200in}}%
\pgfpathcurveto{\pgfqpoint{0.995253in}{2.709200in}}{\pgfqpoint{0.984654in}{2.704810in}}{\pgfqpoint{0.976840in}{2.696996in}}%
\pgfpathcurveto{\pgfqpoint{0.969027in}{2.689182in}}{\pgfqpoint{0.964636in}{2.678583in}}{\pgfqpoint{0.964636in}{2.667533in}}%
\pgfpathcurveto{\pgfqpoint{0.964636in}{2.656483in}}{\pgfqpoint{0.969027in}{2.645884in}}{\pgfqpoint{0.976840in}{2.638070in}}%
\pgfpathcurveto{\pgfqpoint{0.984654in}{2.630257in}}{\pgfqpoint{0.995253in}{2.625866in}}{\pgfqpoint{1.006303in}{2.625866in}}%
\pgfpathlineto{\pgfqpoint{1.006303in}{2.625866in}}%
\pgfpathclose%
\pgfusepath{stroke}%
\end{pgfscope}%
\begin{pgfscope}%
\pgfpathrectangle{\pgfqpoint{0.494722in}{0.437222in}}{\pgfqpoint{6.275590in}{5.159444in}}%
\pgfusepath{clip}%
\pgfsetbuttcap%
\pgfsetroundjoin%
\pgfsetlinewidth{1.003750pt}%
\definecolor{currentstroke}{rgb}{0.827451,0.827451,0.827451}%
\pgfsetstrokecolor{currentstroke}%
\pgfsetstrokeopacity{0.800000}%
\pgfsetdash{}{0pt}%
\pgfpathmoveto{\pgfqpoint{1.464755in}{1.963622in}}%
\pgfpathcurveto{\pgfqpoint{1.475805in}{1.963622in}}{\pgfqpoint{1.486404in}{1.968013in}}{\pgfqpoint{1.494217in}{1.975826in}}%
\pgfpathcurveto{\pgfqpoint{1.502031in}{1.983640in}}{\pgfqpoint{1.506421in}{1.994239in}}{\pgfqpoint{1.506421in}{2.005289in}}%
\pgfpathcurveto{\pgfqpoint{1.506421in}{2.016339in}}{\pgfqpoint{1.502031in}{2.026938in}}{\pgfqpoint{1.494217in}{2.034752in}}%
\pgfpathcurveto{\pgfqpoint{1.486404in}{2.042566in}}{\pgfqpoint{1.475805in}{2.046956in}}{\pgfqpoint{1.464755in}{2.046956in}}%
\pgfpathcurveto{\pgfqpoint{1.453705in}{2.046956in}}{\pgfqpoint{1.443106in}{2.042566in}}{\pgfqpoint{1.435292in}{2.034752in}}%
\pgfpathcurveto{\pgfqpoint{1.427478in}{2.026938in}}{\pgfqpoint{1.423088in}{2.016339in}}{\pgfqpoint{1.423088in}{2.005289in}}%
\pgfpathcurveto{\pgfqpoint{1.423088in}{1.994239in}}{\pgfqpoint{1.427478in}{1.983640in}}{\pgfqpoint{1.435292in}{1.975826in}}%
\pgfpathcurveto{\pgfqpoint{1.443106in}{1.968013in}}{\pgfqpoint{1.453705in}{1.963622in}}{\pgfqpoint{1.464755in}{1.963622in}}%
\pgfpathlineto{\pgfqpoint{1.464755in}{1.963622in}}%
\pgfpathclose%
\pgfusepath{stroke}%
\end{pgfscope}%
\begin{pgfscope}%
\pgfpathrectangle{\pgfqpoint{0.494722in}{0.437222in}}{\pgfqpoint{6.275590in}{5.159444in}}%
\pgfusepath{clip}%
\pgfsetbuttcap%
\pgfsetroundjoin%
\pgfsetlinewidth{1.003750pt}%
\definecolor{currentstroke}{rgb}{0.827451,0.827451,0.827451}%
\pgfsetstrokecolor{currentstroke}%
\pgfsetstrokeopacity{0.800000}%
\pgfsetdash{}{0pt}%
\pgfpathmoveto{\pgfqpoint{0.584355in}{3.716646in}}%
\pgfpathcurveto{\pgfqpoint{0.595406in}{3.716646in}}{\pgfqpoint{0.606005in}{3.721036in}}{\pgfqpoint{0.613818in}{3.728850in}}%
\pgfpathcurveto{\pgfqpoint{0.621632in}{3.736663in}}{\pgfqpoint{0.626022in}{3.747262in}}{\pgfqpoint{0.626022in}{3.758313in}}%
\pgfpathcurveto{\pgfqpoint{0.626022in}{3.769363in}}{\pgfqpoint{0.621632in}{3.779962in}}{\pgfqpoint{0.613818in}{3.787775in}}%
\pgfpathcurveto{\pgfqpoint{0.606005in}{3.795589in}}{\pgfqpoint{0.595406in}{3.799979in}}{\pgfqpoint{0.584355in}{3.799979in}}%
\pgfpathcurveto{\pgfqpoint{0.573305in}{3.799979in}}{\pgfqpoint{0.562706in}{3.795589in}}{\pgfqpoint{0.554893in}{3.787775in}}%
\pgfpathcurveto{\pgfqpoint{0.547079in}{3.779962in}}{\pgfqpoint{0.542689in}{3.769363in}}{\pgfqpoint{0.542689in}{3.758313in}}%
\pgfpathcurveto{\pgfqpoint{0.542689in}{3.747262in}}{\pgfqpoint{0.547079in}{3.736663in}}{\pgfqpoint{0.554893in}{3.728850in}}%
\pgfpathcurveto{\pgfqpoint{0.562706in}{3.721036in}}{\pgfqpoint{0.573305in}{3.716646in}}{\pgfqpoint{0.584355in}{3.716646in}}%
\pgfpathlineto{\pgfqpoint{0.584355in}{3.716646in}}%
\pgfpathclose%
\pgfusepath{stroke}%
\end{pgfscope}%
\begin{pgfscope}%
\pgfpathrectangle{\pgfqpoint{0.494722in}{0.437222in}}{\pgfqpoint{6.275590in}{5.159444in}}%
\pgfusepath{clip}%
\pgfsetbuttcap%
\pgfsetroundjoin%
\pgfsetlinewidth{1.003750pt}%
\definecolor{currentstroke}{rgb}{0.827451,0.827451,0.827451}%
\pgfsetstrokecolor{currentstroke}%
\pgfsetstrokeopacity{0.800000}%
\pgfsetdash{}{0pt}%
\pgfpathmoveto{\pgfqpoint{3.321971in}{0.791856in}}%
\pgfpathcurveto{\pgfqpoint{3.333021in}{0.791856in}}{\pgfqpoint{3.343620in}{0.796246in}}{\pgfqpoint{3.351433in}{0.804060in}}%
\pgfpathcurveto{\pgfqpoint{3.359247in}{0.811873in}}{\pgfqpoint{3.363637in}{0.822472in}}{\pgfqpoint{3.363637in}{0.833523in}}%
\pgfpathcurveto{\pgfqpoint{3.363637in}{0.844573in}}{\pgfqpoint{3.359247in}{0.855172in}}{\pgfqpoint{3.351433in}{0.862985in}}%
\pgfpathcurveto{\pgfqpoint{3.343620in}{0.870799in}}{\pgfqpoint{3.333021in}{0.875189in}}{\pgfqpoint{3.321971in}{0.875189in}}%
\pgfpathcurveto{\pgfqpoint{3.310921in}{0.875189in}}{\pgfqpoint{3.300321in}{0.870799in}}{\pgfqpoint{3.292508in}{0.862985in}}%
\pgfpathcurveto{\pgfqpoint{3.284694in}{0.855172in}}{\pgfqpoint{3.280304in}{0.844573in}}{\pgfqpoint{3.280304in}{0.833523in}}%
\pgfpathcurveto{\pgfqpoint{3.280304in}{0.822472in}}{\pgfqpoint{3.284694in}{0.811873in}}{\pgfqpoint{3.292508in}{0.804060in}}%
\pgfpathcurveto{\pgfqpoint{3.300321in}{0.796246in}}{\pgfqpoint{3.310921in}{0.791856in}}{\pgfqpoint{3.321971in}{0.791856in}}%
\pgfpathlineto{\pgfqpoint{3.321971in}{0.791856in}}%
\pgfpathclose%
\pgfusepath{stroke}%
\end{pgfscope}%
\begin{pgfscope}%
\pgfpathrectangle{\pgfqpoint{0.494722in}{0.437222in}}{\pgfqpoint{6.275590in}{5.159444in}}%
\pgfusepath{clip}%
\pgfsetbuttcap%
\pgfsetroundjoin%
\pgfsetlinewidth{1.003750pt}%
\definecolor{currentstroke}{rgb}{0.827451,0.827451,0.827451}%
\pgfsetstrokecolor{currentstroke}%
\pgfsetstrokeopacity{0.800000}%
\pgfsetdash{}{0pt}%
\pgfpathmoveto{\pgfqpoint{2.224297in}{1.325142in}}%
\pgfpathcurveto{\pgfqpoint{2.235347in}{1.325142in}}{\pgfqpoint{2.245946in}{1.329532in}}{\pgfqpoint{2.253760in}{1.337346in}}%
\pgfpathcurveto{\pgfqpoint{2.261573in}{1.345159in}}{\pgfqpoint{2.265964in}{1.355758in}}{\pgfqpoint{2.265964in}{1.366809in}}%
\pgfpathcurveto{\pgfqpoint{2.265964in}{1.377859in}}{\pgfqpoint{2.261573in}{1.388458in}}{\pgfqpoint{2.253760in}{1.396271in}}%
\pgfpathcurveto{\pgfqpoint{2.245946in}{1.404085in}}{\pgfqpoint{2.235347in}{1.408475in}}{\pgfqpoint{2.224297in}{1.408475in}}%
\pgfpathcurveto{\pgfqpoint{2.213247in}{1.408475in}}{\pgfqpoint{2.202648in}{1.404085in}}{\pgfqpoint{2.194834in}{1.396271in}}%
\pgfpathcurveto{\pgfqpoint{2.187020in}{1.388458in}}{\pgfqpoint{2.182630in}{1.377859in}}{\pgfqpoint{2.182630in}{1.366809in}}%
\pgfpathcurveto{\pgfqpoint{2.182630in}{1.355758in}}{\pgfqpoint{2.187020in}{1.345159in}}{\pgfqpoint{2.194834in}{1.337346in}}%
\pgfpathcurveto{\pgfqpoint{2.202648in}{1.329532in}}{\pgfqpoint{2.213247in}{1.325142in}}{\pgfqpoint{2.224297in}{1.325142in}}%
\pgfpathlineto{\pgfqpoint{2.224297in}{1.325142in}}%
\pgfpathclose%
\pgfusepath{stroke}%
\end{pgfscope}%
\begin{pgfscope}%
\pgfpathrectangle{\pgfqpoint{0.494722in}{0.437222in}}{\pgfqpoint{6.275590in}{5.159444in}}%
\pgfusepath{clip}%
\pgfsetbuttcap%
\pgfsetroundjoin%
\pgfsetlinewidth{1.003750pt}%
\definecolor{currentstroke}{rgb}{0.827451,0.827451,0.827451}%
\pgfsetstrokecolor{currentstroke}%
\pgfsetstrokeopacity{0.800000}%
\pgfsetdash{}{0pt}%
\pgfpathmoveto{\pgfqpoint{1.259002in}{2.253021in}}%
\pgfpathcurveto{\pgfqpoint{1.270052in}{2.253021in}}{\pgfqpoint{1.280651in}{2.257412in}}{\pgfqpoint{1.288465in}{2.265225in}}%
\pgfpathcurveto{\pgfqpoint{1.296279in}{2.273039in}}{\pgfqpoint{1.300669in}{2.283638in}}{\pgfqpoint{1.300669in}{2.294688in}}%
\pgfpathcurveto{\pgfqpoint{1.300669in}{2.305738in}}{\pgfqpoint{1.296279in}{2.316337in}}{\pgfqpoint{1.288465in}{2.324151in}}%
\pgfpathcurveto{\pgfqpoint{1.280651in}{2.331964in}}{\pgfqpoint{1.270052in}{2.336355in}}{\pgfqpoint{1.259002in}{2.336355in}}%
\pgfpathcurveto{\pgfqpoint{1.247952in}{2.336355in}}{\pgfqpoint{1.237353in}{2.331964in}}{\pgfqpoint{1.229540in}{2.324151in}}%
\pgfpathcurveto{\pgfqpoint{1.221726in}{2.316337in}}{\pgfqpoint{1.217336in}{2.305738in}}{\pgfqpoint{1.217336in}{2.294688in}}%
\pgfpathcurveto{\pgfqpoint{1.217336in}{2.283638in}}{\pgfqpoint{1.221726in}{2.273039in}}{\pgfqpoint{1.229540in}{2.265225in}}%
\pgfpathcurveto{\pgfqpoint{1.237353in}{2.257412in}}{\pgfqpoint{1.247952in}{2.253021in}}{\pgfqpoint{1.259002in}{2.253021in}}%
\pgfpathlineto{\pgfqpoint{1.259002in}{2.253021in}}%
\pgfpathclose%
\pgfusepath{stroke}%
\end{pgfscope}%
\begin{pgfscope}%
\pgfpathrectangle{\pgfqpoint{0.494722in}{0.437222in}}{\pgfqpoint{6.275590in}{5.159444in}}%
\pgfusepath{clip}%
\pgfsetbuttcap%
\pgfsetroundjoin%
\pgfsetlinewidth{1.003750pt}%
\definecolor{currentstroke}{rgb}{0.827451,0.827451,0.827451}%
\pgfsetstrokecolor{currentstroke}%
\pgfsetstrokeopacity{0.800000}%
\pgfsetdash{}{0pt}%
\pgfpathmoveto{\pgfqpoint{2.932655in}{0.929387in}}%
\pgfpathcurveto{\pgfqpoint{2.943705in}{0.929387in}}{\pgfqpoint{2.954304in}{0.933777in}}{\pgfqpoint{2.962118in}{0.941591in}}%
\pgfpathcurveto{\pgfqpoint{2.969932in}{0.949404in}}{\pgfqpoint{2.974322in}{0.960003in}}{\pgfqpoint{2.974322in}{0.971053in}}%
\pgfpathcurveto{\pgfqpoint{2.974322in}{0.982103in}}{\pgfqpoint{2.969932in}{0.992702in}}{\pgfqpoint{2.962118in}{1.000516in}}%
\pgfpathcurveto{\pgfqpoint{2.954304in}{1.008330in}}{\pgfqpoint{2.943705in}{1.012720in}}{\pgfqpoint{2.932655in}{1.012720in}}%
\pgfpathcurveto{\pgfqpoint{2.921605in}{1.012720in}}{\pgfqpoint{2.911006in}{1.008330in}}{\pgfqpoint{2.903193in}{1.000516in}}%
\pgfpathcurveto{\pgfqpoint{2.895379in}{0.992702in}}{\pgfqpoint{2.890989in}{0.982103in}}{\pgfqpoint{2.890989in}{0.971053in}}%
\pgfpathcurveto{\pgfqpoint{2.890989in}{0.960003in}}{\pgfqpoint{2.895379in}{0.949404in}}{\pgfqpoint{2.903193in}{0.941591in}}%
\pgfpathcurveto{\pgfqpoint{2.911006in}{0.933777in}}{\pgfqpoint{2.921605in}{0.929387in}}{\pgfqpoint{2.932655in}{0.929387in}}%
\pgfpathlineto{\pgfqpoint{2.932655in}{0.929387in}}%
\pgfpathclose%
\pgfusepath{stroke}%
\end{pgfscope}%
\begin{pgfscope}%
\pgfpathrectangle{\pgfqpoint{0.494722in}{0.437222in}}{\pgfqpoint{6.275590in}{5.159444in}}%
\pgfusepath{clip}%
\pgfsetbuttcap%
\pgfsetroundjoin%
\pgfsetlinewidth{1.003750pt}%
\definecolor{currentstroke}{rgb}{0.827451,0.827451,0.827451}%
\pgfsetstrokecolor{currentstroke}%
\pgfsetstrokeopacity{0.800000}%
\pgfsetdash{}{0pt}%
\pgfpathmoveto{\pgfqpoint{4.362526in}{0.500037in}}%
\pgfpathcurveto{\pgfqpoint{4.373576in}{0.500037in}}{\pgfqpoint{4.384175in}{0.504427in}}{\pgfqpoint{4.391989in}{0.512241in}}%
\pgfpathcurveto{\pgfqpoint{4.399802in}{0.520054in}}{\pgfqpoint{4.404193in}{0.530653in}}{\pgfqpoint{4.404193in}{0.541703in}}%
\pgfpathcurveto{\pgfqpoint{4.404193in}{0.552753in}}{\pgfqpoint{4.399802in}{0.563352in}}{\pgfqpoint{4.391989in}{0.571166in}}%
\pgfpathcurveto{\pgfqpoint{4.384175in}{0.578980in}}{\pgfqpoint{4.373576in}{0.583370in}}{\pgfqpoint{4.362526in}{0.583370in}}%
\pgfpathcurveto{\pgfqpoint{4.351476in}{0.583370in}}{\pgfqpoint{4.340877in}{0.578980in}}{\pgfqpoint{4.333063in}{0.571166in}}%
\pgfpathcurveto{\pgfqpoint{4.325250in}{0.563352in}}{\pgfqpoint{4.320859in}{0.552753in}}{\pgfqpoint{4.320859in}{0.541703in}}%
\pgfpathcurveto{\pgfqpoint{4.320859in}{0.530653in}}{\pgfqpoint{4.325250in}{0.520054in}}{\pgfqpoint{4.333063in}{0.512241in}}%
\pgfpathcurveto{\pgfqpoint{4.340877in}{0.504427in}}{\pgfqpoint{4.351476in}{0.500037in}}{\pgfqpoint{4.362526in}{0.500037in}}%
\pgfpathlineto{\pgfqpoint{4.362526in}{0.500037in}}%
\pgfpathclose%
\pgfusepath{stroke}%
\end{pgfscope}%
\begin{pgfscope}%
\pgfpathrectangle{\pgfqpoint{0.494722in}{0.437222in}}{\pgfqpoint{6.275590in}{5.159444in}}%
\pgfusepath{clip}%
\pgfsetbuttcap%
\pgfsetroundjoin%
\pgfsetlinewidth{1.003750pt}%
\definecolor{currentstroke}{rgb}{0.827451,0.827451,0.827451}%
\pgfsetstrokecolor{currentstroke}%
\pgfsetstrokeopacity{0.800000}%
\pgfsetdash{}{0pt}%
\pgfpathmoveto{\pgfqpoint{3.144732in}{0.844774in}}%
\pgfpathcurveto{\pgfqpoint{3.155782in}{0.844774in}}{\pgfqpoint{3.166381in}{0.849165in}}{\pgfqpoint{3.174195in}{0.856978in}}%
\pgfpathcurveto{\pgfqpoint{3.182009in}{0.864792in}}{\pgfqpoint{3.186399in}{0.875391in}}{\pgfqpoint{3.186399in}{0.886441in}}%
\pgfpathcurveto{\pgfqpoint{3.186399in}{0.897491in}}{\pgfqpoint{3.182009in}{0.908090in}}{\pgfqpoint{3.174195in}{0.915904in}}%
\pgfpathcurveto{\pgfqpoint{3.166381in}{0.923717in}}{\pgfqpoint{3.155782in}{0.928108in}}{\pgfqpoint{3.144732in}{0.928108in}}%
\pgfpathcurveto{\pgfqpoint{3.133682in}{0.928108in}}{\pgfqpoint{3.123083in}{0.923717in}}{\pgfqpoint{3.115269in}{0.915904in}}%
\pgfpathcurveto{\pgfqpoint{3.107456in}{0.908090in}}{\pgfqpoint{3.103066in}{0.897491in}}{\pgfqpoint{3.103066in}{0.886441in}}%
\pgfpathcurveto{\pgfqpoint{3.103066in}{0.875391in}}{\pgfqpoint{3.107456in}{0.864792in}}{\pgfqpoint{3.115269in}{0.856978in}}%
\pgfpathcurveto{\pgfqpoint{3.123083in}{0.849165in}}{\pgfqpoint{3.133682in}{0.844774in}}{\pgfqpoint{3.144732in}{0.844774in}}%
\pgfpathlineto{\pgfqpoint{3.144732in}{0.844774in}}%
\pgfpathclose%
\pgfusepath{stroke}%
\end{pgfscope}%
\begin{pgfscope}%
\pgfpathrectangle{\pgfqpoint{0.494722in}{0.437222in}}{\pgfqpoint{6.275590in}{5.159444in}}%
\pgfusepath{clip}%
\pgfsetbuttcap%
\pgfsetroundjoin%
\pgfsetlinewidth{1.003750pt}%
\definecolor{currentstroke}{rgb}{0.827451,0.827451,0.827451}%
\pgfsetstrokecolor{currentstroke}%
\pgfsetstrokeopacity{0.800000}%
\pgfsetdash{}{0pt}%
\pgfpathmoveto{\pgfqpoint{0.870505in}{2.877763in}}%
\pgfpathcurveto{\pgfqpoint{0.881555in}{2.877763in}}{\pgfqpoint{0.892154in}{2.882154in}}{\pgfqpoint{0.899967in}{2.889967in}}%
\pgfpathcurveto{\pgfqpoint{0.907781in}{2.897781in}}{\pgfqpoint{0.912171in}{2.908380in}}{\pgfqpoint{0.912171in}{2.919430in}}%
\pgfpathcurveto{\pgfqpoint{0.912171in}{2.930480in}}{\pgfqpoint{0.907781in}{2.941079in}}{\pgfqpoint{0.899967in}{2.948893in}}%
\pgfpathcurveto{\pgfqpoint{0.892154in}{2.956706in}}{\pgfqpoint{0.881555in}{2.961097in}}{\pgfqpoint{0.870505in}{2.961097in}}%
\pgfpathcurveto{\pgfqpoint{0.859455in}{2.961097in}}{\pgfqpoint{0.848856in}{2.956706in}}{\pgfqpoint{0.841042in}{2.948893in}}%
\pgfpathcurveto{\pgfqpoint{0.833228in}{2.941079in}}{\pgfqpoint{0.828838in}{2.930480in}}{\pgfqpoint{0.828838in}{2.919430in}}%
\pgfpathcurveto{\pgfqpoint{0.828838in}{2.908380in}}{\pgfqpoint{0.833228in}{2.897781in}}{\pgfqpoint{0.841042in}{2.889967in}}%
\pgfpathcurveto{\pgfqpoint{0.848856in}{2.882154in}}{\pgfqpoint{0.859455in}{2.877763in}}{\pgfqpoint{0.870505in}{2.877763in}}%
\pgfpathlineto{\pgfqpoint{0.870505in}{2.877763in}}%
\pgfpathclose%
\pgfusepath{stroke}%
\end{pgfscope}%
\begin{pgfscope}%
\pgfpathrectangle{\pgfqpoint{0.494722in}{0.437222in}}{\pgfqpoint{6.275590in}{5.159444in}}%
\pgfusepath{clip}%
\pgfsetbuttcap%
\pgfsetroundjoin%
\pgfsetlinewidth{1.003750pt}%
\definecolor{currentstroke}{rgb}{0.827451,0.827451,0.827451}%
\pgfsetstrokecolor{currentstroke}%
\pgfsetstrokeopacity{0.800000}%
\pgfsetdash{}{0pt}%
\pgfpathmoveto{\pgfqpoint{5.378653in}{0.404615in}}%
\pgfpathcurveto{\pgfqpoint{5.389703in}{0.404615in}}{\pgfqpoint{5.400302in}{0.409005in}}{\pgfqpoint{5.408116in}{0.416819in}}%
\pgfpathcurveto{\pgfqpoint{5.415930in}{0.424632in}}{\pgfqpoint{5.420320in}{0.435231in}}{\pgfqpoint{5.420320in}{0.446281in}}%
\pgfpathcurveto{\pgfqpoint{5.420320in}{0.457331in}}{\pgfqpoint{5.415930in}{0.467930in}}{\pgfqpoint{5.408116in}{0.475744in}}%
\pgfpathcurveto{\pgfqpoint{5.400302in}{0.483558in}}{\pgfqpoint{5.389703in}{0.487948in}}{\pgfqpoint{5.378653in}{0.487948in}}%
\pgfpathcurveto{\pgfqpoint{5.367603in}{0.487948in}}{\pgfqpoint{5.357004in}{0.483558in}}{\pgfqpoint{5.349190in}{0.475744in}}%
\pgfpathcurveto{\pgfqpoint{5.341377in}{0.467930in}}{\pgfqpoint{5.336987in}{0.457331in}}{\pgfqpoint{5.336987in}{0.446281in}}%
\pgfpathcurveto{\pgfqpoint{5.336987in}{0.435231in}}{\pgfqpoint{5.341377in}{0.424632in}}{\pgfqpoint{5.349190in}{0.416819in}}%
\pgfpathcurveto{\pgfqpoint{5.357004in}{0.409005in}}{\pgfqpoint{5.367603in}{0.404615in}}{\pgfqpoint{5.378653in}{0.404615in}}%
\pgfusepath{stroke}%
\end{pgfscope}%
\begin{pgfscope}%
\pgfpathrectangle{\pgfqpoint{0.494722in}{0.437222in}}{\pgfqpoint{6.275590in}{5.159444in}}%
\pgfusepath{clip}%
\pgfsetbuttcap%
\pgfsetroundjoin%
\pgfsetlinewidth{1.003750pt}%
\definecolor{currentstroke}{rgb}{0.827451,0.827451,0.827451}%
\pgfsetstrokecolor{currentstroke}%
\pgfsetstrokeopacity{0.800000}%
\pgfsetdash{}{0pt}%
\pgfpathmoveto{\pgfqpoint{1.171136in}{2.367910in}}%
\pgfpathcurveto{\pgfqpoint{1.182186in}{2.367910in}}{\pgfqpoint{1.192785in}{2.372300in}}{\pgfqpoint{1.200599in}{2.380114in}}%
\pgfpathcurveto{\pgfqpoint{1.208412in}{2.387927in}}{\pgfqpoint{1.212803in}{2.398526in}}{\pgfqpoint{1.212803in}{2.409576in}}%
\pgfpathcurveto{\pgfqpoint{1.212803in}{2.420626in}}{\pgfqpoint{1.208412in}{2.431225in}}{\pgfqpoint{1.200599in}{2.439039in}}%
\pgfpathcurveto{\pgfqpoint{1.192785in}{2.446853in}}{\pgfqpoint{1.182186in}{2.451243in}}{\pgfqpoint{1.171136in}{2.451243in}}%
\pgfpathcurveto{\pgfqpoint{1.160086in}{2.451243in}}{\pgfqpoint{1.149487in}{2.446853in}}{\pgfqpoint{1.141673in}{2.439039in}}%
\pgfpathcurveto{\pgfqpoint{1.133860in}{2.431225in}}{\pgfqpoint{1.129469in}{2.420626in}}{\pgfqpoint{1.129469in}{2.409576in}}%
\pgfpathcurveto{\pgfqpoint{1.129469in}{2.398526in}}{\pgfqpoint{1.133860in}{2.387927in}}{\pgfqpoint{1.141673in}{2.380114in}}%
\pgfpathcurveto{\pgfqpoint{1.149487in}{2.372300in}}{\pgfqpoint{1.160086in}{2.367910in}}{\pgfqpoint{1.171136in}{2.367910in}}%
\pgfpathlineto{\pgfqpoint{1.171136in}{2.367910in}}%
\pgfpathclose%
\pgfusepath{stroke}%
\end{pgfscope}%
\begin{pgfscope}%
\pgfpathrectangle{\pgfqpoint{0.494722in}{0.437222in}}{\pgfqpoint{6.275590in}{5.159444in}}%
\pgfusepath{clip}%
\pgfsetbuttcap%
\pgfsetroundjoin%
\pgfsetlinewidth{1.003750pt}%
\definecolor{currentstroke}{rgb}{0.827451,0.827451,0.827451}%
\pgfsetstrokecolor{currentstroke}%
\pgfsetstrokeopacity{0.800000}%
\pgfsetdash{}{0pt}%
\pgfpathmoveto{\pgfqpoint{3.460812in}{0.723768in}}%
\pgfpathcurveto{\pgfqpoint{3.471862in}{0.723768in}}{\pgfqpoint{3.482461in}{0.728158in}}{\pgfqpoint{3.490274in}{0.735972in}}%
\pgfpathcurveto{\pgfqpoint{3.498088in}{0.743785in}}{\pgfqpoint{3.502478in}{0.754384in}}{\pgfqpoint{3.502478in}{0.765434in}}%
\pgfpathcurveto{\pgfqpoint{3.502478in}{0.776484in}}{\pgfqpoint{3.498088in}{0.787083in}}{\pgfqpoint{3.490274in}{0.794897in}}%
\pgfpathcurveto{\pgfqpoint{3.482461in}{0.802711in}}{\pgfqpoint{3.471862in}{0.807101in}}{\pgfqpoint{3.460812in}{0.807101in}}%
\pgfpathcurveto{\pgfqpoint{3.449762in}{0.807101in}}{\pgfqpoint{3.439162in}{0.802711in}}{\pgfqpoint{3.431349in}{0.794897in}}%
\pgfpathcurveto{\pgfqpoint{3.423535in}{0.787083in}}{\pgfqpoint{3.419145in}{0.776484in}}{\pgfqpoint{3.419145in}{0.765434in}}%
\pgfpathcurveto{\pgfqpoint{3.419145in}{0.754384in}}{\pgfqpoint{3.423535in}{0.743785in}}{\pgfqpoint{3.431349in}{0.735972in}}%
\pgfpathcurveto{\pgfqpoint{3.439162in}{0.728158in}}{\pgfqpoint{3.449762in}{0.723768in}}{\pgfqpoint{3.460812in}{0.723768in}}%
\pgfpathlineto{\pgfqpoint{3.460812in}{0.723768in}}%
\pgfpathclose%
\pgfusepath{stroke}%
\end{pgfscope}%
\begin{pgfscope}%
\pgfpathrectangle{\pgfqpoint{0.494722in}{0.437222in}}{\pgfqpoint{6.275590in}{5.159444in}}%
\pgfusepath{clip}%
\pgfsetbuttcap%
\pgfsetroundjoin%
\pgfsetlinewidth{1.003750pt}%
\definecolor{currentstroke}{rgb}{0.827451,0.827451,0.827451}%
\pgfsetstrokecolor{currentstroke}%
\pgfsetstrokeopacity{0.800000}%
\pgfsetdash{}{0pt}%
\pgfpathmoveto{\pgfqpoint{2.016640in}{1.484877in}}%
\pgfpathcurveto{\pgfqpoint{2.027690in}{1.484877in}}{\pgfqpoint{2.038289in}{1.489267in}}{\pgfqpoint{2.046102in}{1.497081in}}%
\pgfpathcurveto{\pgfqpoint{2.053916in}{1.504894in}}{\pgfqpoint{2.058306in}{1.515493in}}{\pgfqpoint{2.058306in}{1.526544in}}%
\pgfpathcurveto{\pgfqpoint{2.058306in}{1.537594in}}{\pgfqpoint{2.053916in}{1.548193in}}{\pgfqpoint{2.046102in}{1.556006in}}%
\pgfpathcurveto{\pgfqpoint{2.038289in}{1.563820in}}{\pgfqpoint{2.027690in}{1.568210in}}{\pgfqpoint{2.016640in}{1.568210in}}%
\pgfpathcurveto{\pgfqpoint{2.005589in}{1.568210in}}{\pgfqpoint{1.994990in}{1.563820in}}{\pgfqpoint{1.987177in}{1.556006in}}%
\pgfpathcurveto{\pgfqpoint{1.979363in}{1.548193in}}{\pgfqpoint{1.974973in}{1.537594in}}{\pgfqpoint{1.974973in}{1.526544in}}%
\pgfpathcurveto{\pgfqpoint{1.974973in}{1.515493in}}{\pgfqpoint{1.979363in}{1.504894in}}{\pgfqpoint{1.987177in}{1.497081in}}%
\pgfpathcurveto{\pgfqpoint{1.994990in}{1.489267in}}{\pgfqpoint{2.005589in}{1.484877in}}{\pgfqpoint{2.016640in}{1.484877in}}%
\pgfpathlineto{\pgfqpoint{2.016640in}{1.484877in}}%
\pgfpathclose%
\pgfusepath{stroke}%
\end{pgfscope}%
\begin{pgfscope}%
\pgfpathrectangle{\pgfqpoint{0.494722in}{0.437222in}}{\pgfqpoint{6.275590in}{5.159444in}}%
\pgfusepath{clip}%
\pgfsetbuttcap%
\pgfsetroundjoin%
\pgfsetlinewidth{1.003750pt}%
\definecolor{currentstroke}{rgb}{0.827451,0.827451,0.827451}%
\pgfsetstrokecolor{currentstroke}%
\pgfsetstrokeopacity{0.800000}%
\pgfsetdash{}{0pt}%
\pgfpathmoveto{\pgfqpoint{2.117156in}{1.399260in}}%
\pgfpathcurveto{\pgfqpoint{2.128206in}{1.399260in}}{\pgfqpoint{2.138805in}{1.403650in}}{\pgfqpoint{2.146618in}{1.411464in}}%
\pgfpathcurveto{\pgfqpoint{2.154432in}{1.419277in}}{\pgfqpoint{2.158822in}{1.429876in}}{\pgfqpoint{2.158822in}{1.440927in}}%
\pgfpathcurveto{\pgfqpoint{2.158822in}{1.451977in}}{\pgfqpoint{2.154432in}{1.462576in}}{\pgfqpoint{2.146618in}{1.470389in}}%
\pgfpathcurveto{\pgfqpoint{2.138805in}{1.478203in}}{\pgfqpoint{2.128206in}{1.482593in}}{\pgfqpoint{2.117156in}{1.482593in}}%
\pgfpathcurveto{\pgfqpoint{2.106106in}{1.482593in}}{\pgfqpoint{2.095506in}{1.478203in}}{\pgfqpoint{2.087693in}{1.470389in}}%
\pgfpathcurveto{\pgfqpoint{2.079879in}{1.462576in}}{\pgfqpoint{2.075489in}{1.451977in}}{\pgfqpoint{2.075489in}{1.440927in}}%
\pgfpathcurveto{\pgfqpoint{2.075489in}{1.429876in}}{\pgfqpoint{2.079879in}{1.419277in}}{\pgfqpoint{2.087693in}{1.411464in}}%
\pgfpathcurveto{\pgfqpoint{2.095506in}{1.403650in}}{\pgfqpoint{2.106106in}{1.399260in}}{\pgfqpoint{2.117156in}{1.399260in}}%
\pgfpathlineto{\pgfqpoint{2.117156in}{1.399260in}}%
\pgfpathclose%
\pgfusepath{stroke}%
\end{pgfscope}%
\begin{pgfscope}%
\pgfpathrectangle{\pgfqpoint{0.494722in}{0.437222in}}{\pgfqpoint{6.275590in}{5.159444in}}%
\pgfusepath{clip}%
\pgfsetbuttcap%
\pgfsetroundjoin%
\pgfsetlinewidth{1.003750pt}%
\definecolor{currentstroke}{rgb}{0.827451,0.827451,0.827451}%
\pgfsetstrokecolor{currentstroke}%
\pgfsetstrokeopacity{0.800000}%
\pgfsetdash{}{0pt}%
\pgfpathmoveto{\pgfqpoint{2.702598in}{1.043180in}}%
\pgfpathcurveto{\pgfqpoint{2.713648in}{1.043180in}}{\pgfqpoint{2.724247in}{1.047570in}}{\pgfqpoint{2.732061in}{1.055384in}}%
\pgfpathcurveto{\pgfqpoint{2.739874in}{1.063198in}}{\pgfqpoint{2.744265in}{1.073797in}}{\pgfqpoint{2.744265in}{1.084847in}}%
\pgfpathcurveto{\pgfqpoint{2.744265in}{1.095897in}}{\pgfqpoint{2.739874in}{1.106496in}}{\pgfqpoint{2.732061in}{1.114310in}}%
\pgfpathcurveto{\pgfqpoint{2.724247in}{1.122123in}}{\pgfqpoint{2.713648in}{1.126513in}}{\pgfqpoint{2.702598in}{1.126513in}}%
\pgfpathcurveto{\pgfqpoint{2.691548in}{1.126513in}}{\pgfqpoint{2.680949in}{1.122123in}}{\pgfqpoint{2.673135in}{1.114310in}}%
\pgfpathcurveto{\pgfqpoint{2.665321in}{1.106496in}}{\pgfqpoint{2.660931in}{1.095897in}}{\pgfqpoint{2.660931in}{1.084847in}}%
\pgfpathcurveto{\pgfqpoint{2.660931in}{1.073797in}}{\pgfqpoint{2.665321in}{1.063198in}}{\pgfqpoint{2.673135in}{1.055384in}}%
\pgfpathcurveto{\pgfqpoint{2.680949in}{1.047570in}}{\pgfqpoint{2.691548in}{1.043180in}}{\pgfqpoint{2.702598in}{1.043180in}}%
\pgfpathlineto{\pgfqpoint{2.702598in}{1.043180in}}%
\pgfpathclose%
\pgfusepath{stroke}%
\end{pgfscope}%
\begin{pgfscope}%
\pgfpathrectangle{\pgfqpoint{0.494722in}{0.437222in}}{\pgfqpoint{6.275590in}{5.159444in}}%
\pgfusepath{clip}%
\pgfsetbuttcap%
\pgfsetroundjoin%
\pgfsetlinewidth{1.003750pt}%
\definecolor{currentstroke}{rgb}{0.827451,0.827451,0.827451}%
\pgfsetstrokecolor{currentstroke}%
\pgfsetstrokeopacity{0.800000}%
\pgfsetdash{}{0pt}%
\pgfpathmoveto{\pgfqpoint{0.625304in}{3.523383in}}%
\pgfpathcurveto{\pgfqpoint{0.636354in}{3.523383in}}{\pgfqpoint{0.646953in}{3.527774in}}{\pgfqpoint{0.654767in}{3.535587in}}%
\pgfpathcurveto{\pgfqpoint{0.662581in}{3.543401in}}{\pgfqpoint{0.666971in}{3.554000in}}{\pgfqpoint{0.666971in}{3.565050in}}%
\pgfpathcurveto{\pgfqpoint{0.666971in}{3.576100in}}{\pgfqpoint{0.662581in}{3.586699in}}{\pgfqpoint{0.654767in}{3.594513in}}%
\pgfpathcurveto{\pgfqpoint{0.646953in}{3.602326in}}{\pgfqpoint{0.636354in}{3.606717in}}{\pgfqpoint{0.625304in}{3.606717in}}%
\pgfpathcurveto{\pgfqpoint{0.614254in}{3.606717in}}{\pgfqpoint{0.603655in}{3.602326in}}{\pgfqpoint{0.595841in}{3.594513in}}%
\pgfpathcurveto{\pgfqpoint{0.588028in}{3.586699in}}{\pgfqpoint{0.583638in}{3.576100in}}{\pgfqpoint{0.583638in}{3.565050in}}%
\pgfpathcurveto{\pgfqpoint{0.583638in}{3.554000in}}{\pgfqpoint{0.588028in}{3.543401in}}{\pgfqpoint{0.595841in}{3.535587in}}%
\pgfpathcurveto{\pgfqpoint{0.603655in}{3.527774in}}{\pgfqpoint{0.614254in}{3.523383in}}{\pgfqpoint{0.625304in}{3.523383in}}%
\pgfpathlineto{\pgfqpoint{0.625304in}{3.523383in}}%
\pgfpathclose%
\pgfusepath{stroke}%
\end{pgfscope}%
\begin{pgfscope}%
\pgfpathrectangle{\pgfqpoint{0.494722in}{0.437222in}}{\pgfqpoint{6.275590in}{5.159444in}}%
\pgfusepath{clip}%
\pgfsetbuttcap%
\pgfsetroundjoin%
\pgfsetlinewidth{1.003750pt}%
\definecolor{currentstroke}{rgb}{0.827451,0.827451,0.827451}%
\pgfsetstrokecolor{currentstroke}%
\pgfsetstrokeopacity{0.800000}%
\pgfsetdash{}{0pt}%
\pgfpathmoveto{\pgfqpoint{3.737625in}{0.637421in}}%
\pgfpathcurveto{\pgfqpoint{3.748675in}{0.637421in}}{\pgfqpoint{3.759274in}{0.641812in}}{\pgfqpoint{3.767088in}{0.649625in}}%
\pgfpathcurveto{\pgfqpoint{3.774901in}{0.657439in}}{\pgfqpoint{3.779292in}{0.668038in}}{\pgfqpoint{3.779292in}{0.679088in}}%
\pgfpathcurveto{\pgfqpoint{3.779292in}{0.690138in}}{\pgfqpoint{3.774901in}{0.700737in}}{\pgfqpoint{3.767088in}{0.708551in}}%
\pgfpathcurveto{\pgfqpoint{3.759274in}{0.716364in}}{\pgfqpoint{3.748675in}{0.720755in}}{\pgfqpoint{3.737625in}{0.720755in}}%
\pgfpathcurveto{\pgfqpoint{3.726575in}{0.720755in}}{\pgfqpoint{3.715976in}{0.716364in}}{\pgfqpoint{3.708162in}{0.708551in}}%
\pgfpathcurveto{\pgfqpoint{3.700349in}{0.700737in}}{\pgfqpoint{3.695958in}{0.690138in}}{\pgfqpoint{3.695958in}{0.679088in}}%
\pgfpathcurveto{\pgfqpoint{3.695958in}{0.668038in}}{\pgfqpoint{3.700349in}{0.657439in}}{\pgfqpoint{3.708162in}{0.649625in}}%
\pgfpathcurveto{\pgfqpoint{3.715976in}{0.641812in}}{\pgfqpoint{3.726575in}{0.637421in}}{\pgfqpoint{3.737625in}{0.637421in}}%
\pgfpathlineto{\pgfqpoint{3.737625in}{0.637421in}}%
\pgfpathclose%
\pgfusepath{stroke}%
\end{pgfscope}%
\begin{pgfscope}%
\pgfpathrectangle{\pgfqpoint{0.494722in}{0.437222in}}{\pgfqpoint{6.275590in}{5.159444in}}%
\pgfusepath{clip}%
\pgfsetbuttcap%
\pgfsetroundjoin%
\pgfsetlinewidth{1.003750pt}%
\definecolor{currentstroke}{rgb}{0.827451,0.827451,0.827451}%
\pgfsetstrokecolor{currentstroke}%
\pgfsetstrokeopacity{0.800000}%
\pgfsetdash{}{0pt}%
\pgfpathmoveto{\pgfqpoint{1.926347in}{1.574489in}}%
\pgfpathcurveto{\pgfqpoint{1.937397in}{1.574489in}}{\pgfqpoint{1.947996in}{1.578879in}}{\pgfqpoint{1.955809in}{1.586693in}}%
\pgfpathcurveto{\pgfqpoint{1.963623in}{1.594506in}}{\pgfqpoint{1.968013in}{1.605105in}}{\pgfqpoint{1.968013in}{1.616155in}}%
\pgfpathcurveto{\pgfqpoint{1.968013in}{1.627206in}}{\pgfqpoint{1.963623in}{1.637805in}}{\pgfqpoint{1.955809in}{1.645618in}}%
\pgfpathcurveto{\pgfqpoint{1.947996in}{1.653432in}}{\pgfqpoint{1.937397in}{1.657822in}}{\pgfqpoint{1.926347in}{1.657822in}}%
\pgfpathcurveto{\pgfqpoint{1.915296in}{1.657822in}}{\pgfqpoint{1.904697in}{1.653432in}}{\pgfqpoint{1.896884in}{1.645618in}}%
\pgfpathcurveto{\pgfqpoint{1.889070in}{1.637805in}}{\pgfqpoint{1.884680in}{1.627206in}}{\pgfqpoint{1.884680in}{1.616155in}}%
\pgfpathcurveto{\pgfqpoint{1.884680in}{1.605105in}}{\pgfqpoint{1.889070in}{1.594506in}}{\pgfqpoint{1.896884in}{1.586693in}}%
\pgfpathcurveto{\pgfqpoint{1.904697in}{1.578879in}}{\pgfqpoint{1.915296in}{1.574489in}}{\pgfqpoint{1.926347in}{1.574489in}}%
\pgfpathlineto{\pgfqpoint{1.926347in}{1.574489in}}%
\pgfpathclose%
\pgfusepath{stroke}%
\end{pgfscope}%
\begin{pgfscope}%
\pgfpathrectangle{\pgfqpoint{0.494722in}{0.437222in}}{\pgfqpoint{6.275590in}{5.159444in}}%
\pgfusepath{clip}%
\pgfsetbuttcap%
\pgfsetroundjoin%
\pgfsetlinewidth{1.003750pt}%
\definecolor{currentstroke}{rgb}{0.827451,0.827451,0.827451}%
\pgfsetstrokecolor{currentstroke}%
\pgfsetstrokeopacity{0.800000}%
\pgfsetdash{}{0pt}%
\pgfpathmoveto{\pgfqpoint{0.525453in}{4.078418in}}%
\pgfpathcurveto{\pgfqpoint{0.536503in}{4.078418in}}{\pgfqpoint{0.547102in}{4.082808in}}{\pgfqpoint{0.554915in}{4.090621in}}%
\pgfpathcurveto{\pgfqpoint{0.562729in}{4.098435in}}{\pgfqpoint{0.567119in}{4.109034in}}{\pgfqpoint{0.567119in}{4.120084in}}%
\pgfpathcurveto{\pgfqpoint{0.567119in}{4.131134in}}{\pgfqpoint{0.562729in}{4.141733in}}{\pgfqpoint{0.554915in}{4.149547in}}%
\pgfpathcurveto{\pgfqpoint{0.547102in}{4.157361in}}{\pgfqpoint{0.536503in}{4.161751in}}{\pgfqpoint{0.525453in}{4.161751in}}%
\pgfpathcurveto{\pgfqpoint{0.514403in}{4.161751in}}{\pgfqpoint{0.503804in}{4.157361in}}{\pgfqpoint{0.495990in}{4.149547in}}%
\pgfpathcurveto{\pgfqpoint{0.488176in}{4.141733in}}{\pgfqpoint{0.483786in}{4.131134in}}{\pgfqpoint{0.483786in}{4.120084in}}%
\pgfpathcurveto{\pgfqpoint{0.483786in}{4.109034in}}{\pgfqpoint{0.488176in}{4.098435in}}{\pgfqpoint{0.495990in}{4.090621in}}%
\pgfpathcurveto{\pgfqpoint{0.503804in}{4.082808in}}{\pgfqpoint{0.514403in}{4.078418in}}{\pgfqpoint{0.525453in}{4.078418in}}%
\pgfpathlineto{\pgfqpoint{0.525453in}{4.078418in}}%
\pgfpathclose%
\pgfusepath{stroke}%
\end{pgfscope}%
\begin{pgfscope}%
\pgfpathrectangle{\pgfqpoint{0.494722in}{0.437222in}}{\pgfqpoint{6.275590in}{5.159444in}}%
\pgfusepath{clip}%
\pgfsetbuttcap%
\pgfsetroundjoin%
\pgfsetlinewidth{1.003750pt}%
\definecolor{currentstroke}{rgb}{0.827451,0.827451,0.827451}%
\pgfsetstrokecolor{currentstroke}%
\pgfsetstrokeopacity{0.800000}%
\pgfsetdash{}{0pt}%
\pgfpathmoveto{\pgfqpoint{0.556707in}{3.885184in}}%
\pgfpathcurveto{\pgfqpoint{0.567757in}{3.885184in}}{\pgfqpoint{0.578356in}{3.889574in}}{\pgfqpoint{0.586170in}{3.897388in}}%
\pgfpathcurveto{\pgfqpoint{0.593983in}{3.905201in}}{\pgfqpoint{0.598374in}{3.915800in}}{\pgfqpoint{0.598374in}{3.926850in}}%
\pgfpathcurveto{\pgfqpoint{0.598374in}{3.937900in}}{\pgfqpoint{0.593983in}{3.948499in}}{\pgfqpoint{0.586170in}{3.956313in}}%
\pgfpathcurveto{\pgfqpoint{0.578356in}{3.964127in}}{\pgfqpoint{0.567757in}{3.968517in}}{\pgfqpoint{0.556707in}{3.968517in}}%
\pgfpathcurveto{\pgfqpoint{0.545657in}{3.968517in}}{\pgfqpoint{0.535058in}{3.964127in}}{\pgfqpoint{0.527244in}{3.956313in}}%
\pgfpathcurveto{\pgfqpoint{0.519430in}{3.948499in}}{\pgfqpoint{0.515040in}{3.937900in}}{\pgfqpoint{0.515040in}{3.926850in}}%
\pgfpathcurveto{\pgfqpoint{0.515040in}{3.915800in}}{\pgfqpoint{0.519430in}{3.905201in}}{\pgfqpoint{0.527244in}{3.897388in}}%
\pgfpathcurveto{\pgfqpoint{0.535058in}{3.889574in}}{\pgfqpoint{0.545657in}{3.885184in}}{\pgfqpoint{0.556707in}{3.885184in}}%
\pgfpathlineto{\pgfqpoint{0.556707in}{3.885184in}}%
\pgfpathclose%
\pgfusepath{stroke}%
\end{pgfscope}%
\begin{pgfscope}%
\pgfpathrectangle{\pgfqpoint{0.494722in}{0.437222in}}{\pgfqpoint{6.275590in}{5.159444in}}%
\pgfusepath{clip}%
\pgfsetbuttcap%
\pgfsetroundjoin%
\pgfsetlinewidth{1.003750pt}%
\definecolor{currentstroke}{rgb}{0.827451,0.827451,0.827451}%
\pgfsetstrokecolor{currentstroke}%
\pgfsetstrokeopacity{0.800000}%
\pgfsetdash{}{0pt}%
\pgfpathmoveto{\pgfqpoint{0.498024in}{4.460098in}}%
\pgfpathcurveto{\pgfqpoint{0.509074in}{4.460098in}}{\pgfqpoint{0.519673in}{4.464488in}}{\pgfqpoint{0.527487in}{4.472302in}}%
\pgfpathcurveto{\pgfqpoint{0.535301in}{4.480116in}}{\pgfqpoint{0.539691in}{4.490715in}}{\pgfqpoint{0.539691in}{4.501765in}}%
\pgfpathcurveto{\pgfqpoint{0.539691in}{4.512815in}}{\pgfqpoint{0.535301in}{4.523414in}}{\pgfqpoint{0.527487in}{4.531228in}}%
\pgfpathcurveto{\pgfqpoint{0.519673in}{4.539041in}}{\pgfqpoint{0.509074in}{4.543432in}}{\pgfqpoint{0.498024in}{4.543432in}}%
\pgfpathcurveto{\pgfqpoint{0.486974in}{4.543432in}}{\pgfqpoint{0.476375in}{4.539041in}}{\pgfqpoint{0.468561in}{4.531228in}}%
\pgfpathcurveto{\pgfqpoint{0.460748in}{4.523414in}}{\pgfqpoint{0.456358in}{4.512815in}}{\pgfqpoint{0.456358in}{4.501765in}}%
\pgfpathcurveto{\pgfqpoint{0.456358in}{4.490715in}}{\pgfqpoint{0.460748in}{4.480116in}}{\pgfqpoint{0.468561in}{4.472302in}}%
\pgfpathcurveto{\pgfqpoint{0.476375in}{4.464488in}}{\pgfqpoint{0.486974in}{4.460098in}}{\pgfqpoint{0.498024in}{4.460098in}}%
\pgfpathlineto{\pgfqpoint{0.498024in}{4.460098in}}%
\pgfpathclose%
\pgfusepath{stroke}%
\end{pgfscope}%
\begin{pgfscope}%
\pgfpathrectangle{\pgfqpoint{0.494722in}{0.437222in}}{\pgfqpoint{6.275590in}{5.159444in}}%
\pgfusepath{clip}%
\pgfsetbuttcap%
\pgfsetroundjoin%
\pgfsetlinewidth{1.003750pt}%
\definecolor{currentstroke}{rgb}{0.827451,0.827451,0.827451}%
\pgfsetstrokecolor{currentstroke}%
\pgfsetstrokeopacity{0.800000}%
\pgfsetdash{}{0pt}%
\pgfpathmoveto{\pgfqpoint{0.734476in}{3.173741in}}%
\pgfpathcurveto{\pgfqpoint{0.745526in}{3.173741in}}{\pgfqpoint{0.756125in}{3.178131in}}{\pgfqpoint{0.763939in}{3.185945in}}%
\pgfpathcurveto{\pgfqpoint{0.771753in}{3.193758in}}{\pgfqpoint{0.776143in}{3.204357in}}{\pgfqpoint{0.776143in}{3.215408in}}%
\pgfpathcurveto{\pgfqpoint{0.776143in}{3.226458in}}{\pgfqpoint{0.771753in}{3.237057in}}{\pgfqpoint{0.763939in}{3.244870in}}%
\pgfpathcurveto{\pgfqpoint{0.756125in}{3.252684in}}{\pgfqpoint{0.745526in}{3.257074in}}{\pgfqpoint{0.734476in}{3.257074in}}%
\pgfpathcurveto{\pgfqpoint{0.723426in}{3.257074in}}{\pgfqpoint{0.712827in}{3.252684in}}{\pgfqpoint{0.705013in}{3.244870in}}%
\pgfpathcurveto{\pgfqpoint{0.697200in}{3.237057in}}{\pgfqpoint{0.692810in}{3.226458in}}{\pgfqpoint{0.692810in}{3.215408in}}%
\pgfpathcurveto{\pgfqpoint{0.692810in}{3.204357in}}{\pgfqpoint{0.697200in}{3.193758in}}{\pgfqpoint{0.705013in}{3.185945in}}%
\pgfpathcurveto{\pgfqpoint{0.712827in}{3.178131in}}{\pgfqpoint{0.723426in}{3.173741in}}{\pgfqpoint{0.734476in}{3.173741in}}%
\pgfpathlineto{\pgfqpoint{0.734476in}{3.173741in}}%
\pgfpathclose%
\pgfusepath{stroke}%
\end{pgfscope}%
\begin{pgfscope}%
\pgfpathrectangle{\pgfqpoint{0.494722in}{0.437222in}}{\pgfqpoint{6.275590in}{5.159444in}}%
\pgfusepath{clip}%
\pgfsetbuttcap%
\pgfsetroundjoin%
\pgfsetlinewidth{1.003750pt}%
\definecolor{currentstroke}{rgb}{0.827451,0.827451,0.827451}%
\pgfsetstrokecolor{currentstroke}%
\pgfsetstrokeopacity{0.800000}%
\pgfsetdash{}{0pt}%
\pgfpathmoveto{\pgfqpoint{1.786945in}{1.657846in}}%
\pgfpathcurveto{\pgfqpoint{1.797995in}{1.657846in}}{\pgfqpoint{1.808594in}{1.662237in}}{\pgfqpoint{1.816407in}{1.670050in}}%
\pgfpathcurveto{\pgfqpoint{1.824221in}{1.677864in}}{\pgfqpoint{1.828611in}{1.688463in}}{\pgfqpoint{1.828611in}{1.699513in}}%
\pgfpathcurveto{\pgfqpoint{1.828611in}{1.710563in}}{\pgfqpoint{1.824221in}{1.721162in}}{\pgfqpoint{1.816407in}{1.728976in}}%
\pgfpathcurveto{\pgfqpoint{1.808594in}{1.736789in}}{\pgfqpoint{1.797995in}{1.741180in}}{\pgfqpoint{1.786945in}{1.741180in}}%
\pgfpathcurveto{\pgfqpoint{1.775894in}{1.741180in}}{\pgfqpoint{1.765295in}{1.736789in}}{\pgfqpoint{1.757482in}{1.728976in}}%
\pgfpathcurveto{\pgfqpoint{1.749668in}{1.721162in}}{\pgfqpoint{1.745278in}{1.710563in}}{\pgfqpoint{1.745278in}{1.699513in}}%
\pgfpathcurveto{\pgfqpoint{1.745278in}{1.688463in}}{\pgfqpoint{1.749668in}{1.677864in}}{\pgfqpoint{1.757482in}{1.670050in}}%
\pgfpathcurveto{\pgfqpoint{1.765295in}{1.662237in}}{\pgfqpoint{1.775894in}{1.657846in}}{\pgfqpoint{1.786945in}{1.657846in}}%
\pgfpathlineto{\pgfqpoint{1.786945in}{1.657846in}}%
\pgfpathclose%
\pgfusepath{stroke}%
\end{pgfscope}%
\begin{pgfscope}%
\pgfpathrectangle{\pgfqpoint{0.494722in}{0.437222in}}{\pgfqpoint{6.275590in}{5.159444in}}%
\pgfusepath{clip}%
\pgfsetbuttcap%
\pgfsetroundjoin%
\pgfsetlinewidth{1.003750pt}%
\definecolor{currentstroke}{rgb}{0.827451,0.827451,0.827451}%
\pgfsetstrokecolor{currentstroke}%
\pgfsetstrokeopacity{0.800000}%
\pgfsetdash{}{0pt}%
\pgfpathmoveto{\pgfqpoint{1.268327in}{2.198214in}}%
\pgfpathcurveto{\pgfqpoint{1.279377in}{2.198214in}}{\pgfqpoint{1.289976in}{2.202604in}}{\pgfqpoint{1.297790in}{2.210418in}}%
\pgfpathcurveto{\pgfqpoint{1.305603in}{2.218231in}}{\pgfqpoint{1.309994in}{2.228830in}}{\pgfqpoint{1.309994in}{2.239880in}}%
\pgfpathcurveto{\pgfqpoint{1.309994in}{2.250930in}}{\pgfqpoint{1.305603in}{2.261529in}}{\pgfqpoint{1.297790in}{2.269343in}}%
\pgfpathcurveto{\pgfqpoint{1.289976in}{2.277157in}}{\pgfqpoint{1.279377in}{2.281547in}}{\pgfqpoint{1.268327in}{2.281547in}}%
\pgfpathcurveto{\pgfqpoint{1.257277in}{2.281547in}}{\pgfqpoint{1.246678in}{2.277157in}}{\pgfqpoint{1.238864in}{2.269343in}}%
\pgfpathcurveto{\pgfqpoint{1.231051in}{2.261529in}}{\pgfqpoint{1.226660in}{2.250930in}}{\pgfqpoint{1.226660in}{2.239880in}}%
\pgfpathcurveto{\pgfqpoint{1.226660in}{2.228830in}}{\pgfqpoint{1.231051in}{2.218231in}}{\pgfqpoint{1.238864in}{2.210418in}}%
\pgfpathcurveto{\pgfqpoint{1.246678in}{2.202604in}}{\pgfqpoint{1.257277in}{2.198214in}}{\pgfqpoint{1.268327in}{2.198214in}}%
\pgfpathlineto{\pgfqpoint{1.268327in}{2.198214in}}%
\pgfpathclose%
\pgfusepath{stroke}%
\end{pgfscope}%
\begin{pgfscope}%
\pgfpathrectangle{\pgfqpoint{0.494722in}{0.437222in}}{\pgfqpoint{6.275590in}{5.159444in}}%
\pgfusepath{clip}%
\pgfsetbuttcap%
\pgfsetroundjoin%
\pgfsetlinewidth{1.003750pt}%
\definecolor{currentstroke}{rgb}{0.827451,0.827451,0.827451}%
\pgfsetstrokecolor{currentstroke}%
\pgfsetstrokeopacity{0.800000}%
\pgfsetdash{}{0pt}%
\pgfpathmoveto{\pgfqpoint{2.183447in}{1.371651in}}%
\pgfpathcurveto{\pgfqpoint{2.194497in}{1.371651in}}{\pgfqpoint{2.205096in}{1.376041in}}{\pgfqpoint{2.212910in}{1.383855in}}%
\pgfpathcurveto{\pgfqpoint{2.220724in}{1.391668in}}{\pgfqpoint{2.225114in}{1.402267in}}{\pgfqpoint{2.225114in}{1.413318in}}%
\pgfpathcurveto{\pgfqpoint{2.225114in}{1.424368in}}{\pgfqpoint{2.220724in}{1.434967in}}{\pgfqpoint{2.212910in}{1.442780in}}%
\pgfpathcurveto{\pgfqpoint{2.205096in}{1.450594in}}{\pgfqpoint{2.194497in}{1.454984in}}{\pgfqpoint{2.183447in}{1.454984in}}%
\pgfpathcurveto{\pgfqpoint{2.172397in}{1.454984in}}{\pgfqpoint{2.161798in}{1.450594in}}{\pgfqpoint{2.153984in}{1.442780in}}%
\pgfpathcurveto{\pgfqpoint{2.146171in}{1.434967in}}{\pgfqpoint{2.141781in}{1.424368in}}{\pgfqpoint{2.141781in}{1.413318in}}%
\pgfpathcurveto{\pgfqpoint{2.141781in}{1.402267in}}{\pgfqpoint{2.146171in}{1.391668in}}{\pgfqpoint{2.153984in}{1.383855in}}%
\pgfpathcurveto{\pgfqpoint{2.161798in}{1.376041in}}{\pgfqpoint{2.172397in}{1.371651in}}{\pgfqpoint{2.183447in}{1.371651in}}%
\pgfpathlineto{\pgfqpoint{2.183447in}{1.371651in}}%
\pgfpathclose%
\pgfusepath{stroke}%
\end{pgfscope}%
\begin{pgfscope}%
\pgfpathrectangle{\pgfqpoint{0.494722in}{0.437222in}}{\pgfqpoint{6.275590in}{5.159444in}}%
\pgfusepath{clip}%
\pgfsetbuttcap%
\pgfsetroundjoin%
\pgfsetlinewidth{1.003750pt}%
\definecolor{currentstroke}{rgb}{0.827451,0.827451,0.827451}%
\pgfsetstrokecolor{currentstroke}%
\pgfsetstrokeopacity{0.800000}%
\pgfsetdash{}{0pt}%
\pgfpathmoveto{\pgfqpoint{2.856780in}{0.965441in}}%
\pgfpathcurveto{\pgfqpoint{2.867831in}{0.965441in}}{\pgfqpoint{2.878430in}{0.969831in}}{\pgfqpoint{2.886243in}{0.977645in}}%
\pgfpathcurveto{\pgfqpoint{2.894057in}{0.985458in}}{\pgfqpoint{2.898447in}{0.996058in}}{\pgfqpoint{2.898447in}{1.007108in}}%
\pgfpathcurveto{\pgfqpoint{2.898447in}{1.018158in}}{\pgfqpoint{2.894057in}{1.028757in}}{\pgfqpoint{2.886243in}{1.036570in}}%
\pgfpathcurveto{\pgfqpoint{2.878430in}{1.044384in}}{\pgfqpoint{2.867831in}{1.048774in}}{\pgfqpoint{2.856780in}{1.048774in}}%
\pgfpathcurveto{\pgfqpoint{2.845730in}{1.048774in}}{\pgfqpoint{2.835131in}{1.044384in}}{\pgfqpoint{2.827318in}{1.036570in}}%
\pgfpathcurveto{\pgfqpoint{2.819504in}{1.028757in}}{\pgfqpoint{2.815114in}{1.018158in}}{\pgfqpoint{2.815114in}{1.007108in}}%
\pgfpathcurveto{\pgfqpoint{2.815114in}{0.996058in}}{\pgfqpoint{2.819504in}{0.985458in}}{\pgfqpoint{2.827318in}{0.977645in}}%
\pgfpathcurveto{\pgfqpoint{2.835131in}{0.969831in}}{\pgfqpoint{2.845730in}{0.965441in}}{\pgfqpoint{2.856780in}{0.965441in}}%
\pgfpathlineto{\pgfqpoint{2.856780in}{0.965441in}}%
\pgfpathclose%
\pgfusepath{stroke}%
\end{pgfscope}%
\begin{pgfscope}%
\pgfpathrectangle{\pgfqpoint{0.494722in}{0.437222in}}{\pgfqpoint{6.275590in}{5.159444in}}%
\pgfusepath{clip}%
\pgfsetbuttcap%
\pgfsetroundjoin%
\pgfsetlinewidth{1.003750pt}%
\definecolor{currentstroke}{rgb}{0.827451,0.827451,0.827451}%
\pgfsetstrokecolor{currentstroke}%
\pgfsetstrokeopacity{0.800000}%
\pgfsetdash{}{0pt}%
\pgfpathmoveto{\pgfqpoint{4.495815in}{0.487123in}}%
\pgfpathcurveto{\pgfqpoint{4.506865in}{0.487123in}}{\pgfqpoint{4.517464in}{0.491513in}}{\pgfqpoint{4.525277in}{0.499327in}}%
\pgfpathcurveto{\pgfqpoint{4.533091in}{0.507140in}}{\pgfqpoint{4.537481in}{0.517740in}}{\pgfqpoint{4.537481in}{0.528790in}}%
\pgfpathcurveto{\pgfqpoint{4.537481in}{0.539840in}}{\pgfqpoint{4.533091in}{0.550439in}}{\pgfqpoint{4.525277in}{0.558252in}}%
\pgfpathcurveto{\pgfqpoint{4.517464in}{0.566066in}}{\pgfqpoint{4.506865in}{0.570456in}}{\pgfqpoint{4.495815in}{0.570456in}}%
\pgfpathcurveto{\pgfqpoint{4.484765in}{0.570456in}}{\pgfqpoint{4.474166in}{0.566066in}}{\pgfqpoint{4.466352in}{0.558252in}}%
\pgfpathcurveto{\pgfqpoint{4.458538in}{0.550439in}}{\pgfqpoint{4.454148in}{0.539840in}}{\pgfqpoint{4.454148in}{0.528790in}}%
\pgfpathcurveto{\pgfqpoint{4.454148in}{0.517740in}}{\pgfqpoint{4.458538in}{0.507140in}}{\pgfqpoint{4.466352in}{0.499327in}}%
\pgfpathcurveto{\pgfqpoint{4.474166in}{0.491513in}}{\pgfqpoint{4.484765in}{0.487123in}}{\pgfqpoint{4.495815in}{0.487123in}}%
\pgfpathlineto{\pgfqpoint{4.495815in}{0.487123in}}%
\pgfpathclose%
\pgfusepath{stroke}%
\end{pgfscope}%
\begin{pgfscope}%
\pgfpathrectangle{\pgfqpoint{0.494722in}{0.437222in}}{\pgfqpoint{6.275590in}{5.159444in}}%
\pgfusepath{clip}%
\pgfsetbuttcap%
\pgfsetroundjoin%
\pgfsetlinewidth{1.003750pt}%
\definecolor{currentstroke}{rgb}{0.827451,0.827451,0.827451}%
\pgfsetstrokecolor{currentstroke}%
\pgfsetstrokeopacity{0.800000}%
\pgfsetdash{}{0pt}%
\pgfpathmoveto{\pgfqpoint{4.010167in}{0.576865in}}%
\pgfpathcurveto{\pgfqpoint{4.021217in}{0.576865in}}{\pgfqpoint{4.031816in}{0.581255in}}{\pgfqpoint{4.039630in}{0.589069in}}%
\pgfpathcurveto{\pgfqpoint{4.047443in}{0.596882in}}{\pgfqpoint{4.051834in}{0.607481in}}{\pgfqpoint{4.051834in}{0.618531in}}%
\pgfpathcurveto{\pgfqpoint{4.051834in}{0.629581in}}{\pgfqpoint{4.047443in}{0.640180in}}{\pgfqpoint{4.039630in}{0.647994in}}%
\pgfpathcurveto{\pgfqpoint{4.031816in}{0.655808in}}{\pgfqpoint{4.021217in}{0.660198in}}{\pgfqpoint{4.010167in}{0.660198in}}%
\pgfpathcurveto{\pgfqpoint{3.999117in}{0.660198in}}{\pgfqpoint{3.988518in}{0.655808in}}{\pgfqpoint{3.980704in}{0.647994in}}%
\pgfpathcurveto{\pgfqpoint{3.972890in}{0.640180in}}{\pgfqpoint{3.968500in}{0.629581in}}{\pgfqpoint{3.968500in}{0.618531in}}%
\pgfpathcurveto{\pgfqpoint{3.968500in}{0.607481in}}{\pgfqpoint{3.972890in}{0.596882in}}{\pgfqpoint{3.980704in}{0.589069in}}%
\pgfpathcurveto{\pgfqpoint{3.988518in}{0.581255in}}{\pgfqpoint{3.999117in}{0.576865in}}{\pgfqpoint{4.010167in}{0.576865in}}%
\pgfpathlineto{\pgfqpoint{4.010167in}{0.576865in}}%
\pgfpathclose%
\pgfusepath{stroke}%
\end{pgfscope}%
\begin{pgfscope}%
\pgfpathrectangle{\pgfqpoint{0.494722in}{0.437222in}}{\pgfqpoint{6.275590in}{5.159444in}}%
\pgfusepath{clip}%
\pgfsetbuttcap%
\pgfsetroundjoin%
\pgfsetlinewidth{1.003750pt}%
\definecolor{currentstroke}{rgb}{0.827451,0.827451,0.827451}%
\pgfsetstrokecolor{currentstroke}%
\pgfsetstrokeopacity{0.800000}%
\pgfsetdash{}{0pt}%
\pgfpathmoveto{\pgfqpoint{4.023146in}{0.574102in}}%
\pgfpathcurveto{\pgfqpoint{4.034196in}{0.574102in}}{\pgfqpoint{4.044795in}{0.578492in}}{\pgfqpoint{4.052609in}{0.586306in}}%
\pgfpathcurveto{\pgfqpoint{4.060422in}{0.594119in}}{\pgfqpoint{4.064812in}{0.604719in}}{\pgfqpoint{4.064812in}{0.615769in}}%
\pgfpathcurveto{\pgfqpoint{4.064812in}{0.626819in}}{\pgfqpoint{4.060422in}{0.637418in}}{\pgfqpoint{4.052609in}{0.645231in}}%
\pgfpathcurveto{\pgfqpoint{4.044795in}{0.653045in}}{\pgfqpoint{4.034196in}{0.657435in}}{\pgfqpoint{4.023146in}{0.657435in}}%
\pgfpathcurveto{\pgfqpoint{4.012096in}{0.657435in}}{\pgfqpoint{4.001497in}{0.653045in}}{\pgfqpoint{3.993683in}{0.645231in}}%
\pgfpathcurveto{\pgfqpoint{3.985869in}{0.637418in}}{\pgfqpoint{3.981479in}{0.626819in}}{\pgfqpoint{3.981479in}{0.615769in}}%
\pgfpathcurveto{\pgfqpoint{3.981479in}{0.604719in}}{\pgfqpoint{3.985869in}{0.594119in}}{\pgfqpoint{3.993683in}{0.586306in}}%
\pgfpathcurveto{\pgfqpoint{4.001497in}{0.578492in}}{\pgfqpoint{4.012096in}{0.574102in}}{\pgfqpoint{4.023146in}{0.574102in}}%
\pgfpathlineto{\pgfqpoint{4.023146in}{0.574102in}}%
\pgfpathclose%
\pgfusepath{stroke}%
\end{pgfscope}%
\begin{pgfscope}%
\pgfpathrectangle{\pgfqpoint{0.494722in}{0.437222in}}{\pgfqpoint{6.275590in}{5.159444in}}%
\pgfusepath{clip}%
\pgfsetbuttcap%
\pgfsetroundjoin%
\pgfsetlinewidth{1.003750pt}%
\definecolor{currentstroke}{rgb}{0.827451,0.827451,0.827451}%
\pgfsetstrokecolor{currentstroke}%
\pgfsetstrokeopacity{0.800000}%
\pgfsetdash{}{0pt}%
\pgfpathmoveto{\pgfqpoint{0.750297in}{3.133310in}}%
\pgfpathcurveto{\pgfqpoint{0.761348in}{3.133310in}}{\pgfqpoint{0.771947in}{3.137700in}}{\pgfqpoint{0.779760in}{3.145514in}}%
\pgfpathcurveto{\pgfqpoint{0.787574in}{3.153328in}}{\pgfqpoint{0.791964in}{3.163927in}}{\pgfqpoint{0.791964in}{3.174977in}}%
\pgfpathcurveto{\pgfqpoint{0.791964in}{3.186027in}}{\pgfqpoint{0.787574in}{3.196626in}}{\pgfqpoint{0.779760in}{3.204439in}}%
\pgfpathcurveto{\pgfqpoint{0.771947in}{3.212253in}}{\pgfqpoint{0.761348in}{3.216643in}}{\pgfqpoint{0.750297in}{3.216643in}}%
\pgfpathcurveto{\pgfqpoint{0.739247in}{3.216643in}}{\pgfqpoint{0.728648in}{3.212253in}}{\pgfqpoint{0.720835in}{3.204439in}}%
\pgfpathcurveto{\pgfqpoint{0.713021in}{3.196626in}}{\pgfqpoint{0.708631in}{3.186027in}}{\pgfqpoint{0.708631in}{3.174977in}}%
\pgfpathcurveto{\pgfqpoint{0.708631in}{3.163927in}}{\pgfqpoint{0.713021in}{3.153328in}}{\pgfqpoint{0.720835in}{3.145514in}}%
\pgfpathcurveto{\pgfqpoint{0.728648in}{3.137700in}}{\pgfqpoint{0.739247in}{3.133310in}}{\pgfqpoint{0.750297in}{3.133310in}}%
\pgfpathlineto{\pgfqpoint{0.750297in}{3.133310in}}%
\pgfpathclose%
\pgfusepath{stroke}%
\end{pgfscope}%
\begin{pgfscope}%
\pgfpathrectangle{\pgfqpoint{0.494722in}{0.437222in}}{\pgfqpoint{6.275590in}{5.159444in}}%
\pgfusepath{clip}%
\pgfsetbuttcap%
\pgfsetroundjoin%
\pgfsetlinewidth{1.003750pt}%
\definecolor{currentstroke}{rgb}{0.827451,0.827451,0.827451}%
\pgfsetstrokecolor{currentstroke}%
\pgfsetstrokeopacity{0.800000}%
\pgfsetdash{}{0pt}%
\pgfpathmoveto{\pgfqpoint{1.581391in}{1.870889in}}%
\pgfpathcurveto{\pgfqpoint{1.592441in}{1.870889in}}{\pgfqpoint{1.603040in}{1.875279in}}{\pgfqpoint{1.610854in}{1.883093in}}%
\pgfpathcurveto{\pgfqpoint{1.618667in}{1.890906in}}{\pgfqpoint{1.623058in}{1.901505in}}{\pgfqpoint{1.623058in}{1.912556in}}%
\pgfpathcurveto{\pgfqpoint{1.623058in}{1.923606in}}{\pgfqpoint{1.618667in}{1.934205in}}{\pgfqpoint{1.610854in}{1.942018in}}%
\pgfpathcurveto{\pgfqpoint{1.603040in}{1.949832in}}{\pgfqpoint{1.592441in}{1.954222in}}{\pgfqpoint{1.581391in}{1.954222in}}%
\pgfpathcurveto{\pgfqpoint{1.570341in}{1.954222in}}{\pgfqpoint{1.559742in}{1.949832in}}{\pgfqpoint{1.551928in}{1.942018in}}%
\pgfpathcurveto{\pgfqpoint{1.544115in}{1.934205in}}{\pgfqpoint{1.539724in}{1.923606in}}{\pgfqpoint{1.539724in}{1.912556in}}%
\pgfpathcurveto{\pgfqpoint{1.539724in}{1.901505in}}{\pgfqpoint{1.544115in}{1.890906in}}{\pgfqpoint{1.551928in}{1.883093in}}%
\pgfpathcurveto{\pgfqpoint{1.559742in}{1.875279in}}{\pgfqpoint{1.570341in}{1.870889in}}{\pgfqpoint{1.581391in}{1.870889in}}%
\pgfpathlineto{\pgfqpoint{1.581391in}{1.870889in}}%
\pgfpathclose%
\pgfusepath{stroke}%
\end{pgfscope}%
\begin{pgfscope}%
\pgfpathrectangle{\pgfqpoint{0.494722in}{0.437222in}}{\pgfqpoint{6.275590in}{5.159444in}}%
\pgfusepath{clip}%
\pgfsetbuttcap%
\pgfsetroundjoin%
\pgfsetlinewidth{1.003750pt}%
\definecolor{currentstroke}{rgb}{0.827451,0.827451,0.827451}%
\pgfsetstrokecolor{currentstroke}%
\pgfsetstrokeopacity{0.800000}%
\pgfsetdash{}{0pt}%
\pgfpathmoveto{\pgfqpoint{3.121483in}{0.847531in}}%
\pgfpathcurveto{\pgfqpoint{3.132533in}{0.847531in}}{\pgfqpoint{3.143132in}{0.851921in}}{\pgfqpoint{3.150946in}{0.859735in}}%
\pgfpathcurveto{\pgfqpoint{3.158760in}{0.867548in}}{\pgfqpoint{3.163150in}{0.878147in}}{\pgfqpoint{3.163150in}{0.889198in}}%
\pgfpathcurveto{\pgfqpoint{3.163150in}{0.900248in}}{\pgfqpoint{3.158760in}{0.910847in}}{\pgfqpoint{3.150946in}{0.918660in}}%
\pgfpathcurveto{\pgfqpoint{3.143132in}{0.926474in}}{\pgfqpoint{3.132533in}{0.930864in}}{\pgfqpoint{3.121483in}{0.930864in}}%
\pgfpathcurveto{\pgfqpoint{3.110433in}{0.930864in}}{\pgfqpoint{3.099834in}{0.926474in}}{\pgfqpoint{3.092021in}{0.918660in}}%
\pgfpathcurveto{\pgfqpoint{3.084207in}{0.910847in}}{\pgfqpoint{3.079817in}{0.900248in}}{\pgfqpoint{3.079817in}{0.889198in}}%
\pgfpathcurveto{\pgfqpoint{3.079817in}{0.878147in}}{\pgfqpoint{3.084207in}{0.867548in}}{\pgfqpoint{3.092021in}{0.859735in}}%
\pgfpathcurveto{\pgfqpoint{3.099834in}{0.851921in}}{\pgfqpoint{3.110433in}{0.847531in}}{\pgfqpoint{3.121483in}{0.847531in}}%
\pgfpathlineto{\pgfqpoint{3.121483in}{0.847531in}}%
\pgfpathclose%
\pgfusepath{stroke}%
\end{pgfscope}%
\begin{pgfscope}%
\pgfpathrectangle{\pgfqpoint{0.494722in}{0.437222in}}{\pgfqpoint{6.275590in}{5.159444in}}%
\pgfusepath{clip}%
\pgfsetbuttcap%
\pgfsetroundjoin%
\pgfsetlinewidth{1.003750pt}%
\definecolor{currentstroke}{rgb}{0.827451,0.827451,0.827451}%
\pgfsetstrokecolor{currentstroke}%
\pgfsetstrokeopacity{0.800000}%
\pgfsetdash{}{0pt}%
\pgfpathmoveto{\pgfqpoint{4.327440in}{0.503110in}}%
\pgfpathcurveto{\pgfqpoint{4.338490in}{0.503110in}}{\pgfqpoint{4.349089in}{0.507500in}}{\pgfqpoint{4.356903in}{0.515314in}}%
\pgfpathcurveto{\pgfqpoint{4.364716in}{0.523127in}}{\pgfqpoint{4.369107in}{0.533727in}}{\pgfqpoint{4.369107in}{0.544777in}}%
\pgfpathcurveto{\pgfqpoint{4.369107in}{0.555827in}}{\pgfqpoint{4.364716in}{0.566426in}}{\pgfqpoint{4.356903in}{0.574239in}}%
\pgfpathcurveto{\pgfqpoint{4.349089in}{0.582053in}}{\pgfqpoint{4.338490in}{0.586443in}}{\pgfqpoint{4.327440in}{0.586443in}}%
\pgfpathcurveto{\pgfqpoint{4.316390in}{0.586443in}}{\pgfqpoint{4.305791in}{0.582053in}}{\pgfqpoint{4.297977in}{0.574239in}}%
\pgfpathcurveto{\pgfqpoint{4.290164in}{0.566426in}}{\pgfqpoint{4.285773in}{0.555827in}}{\pgfqpoint{4.285773in}{0.544777in}}%
\pgfpathcurveto{\pgfqpoint{4.285773in}{0.533727in}}{\pgfqpoint{4.290164in}{0.523127in}}{\pgfqpoint{4.297977in}{0.515314in}}%
\pgfpathcurveto{\pgfqpoint{4.305791in}{0.507500in}}{\pgfqpoint{4.316390in}{0.503110in}}{\pgfqpoint{4.327440in}{0.503110in}}%
\pgfpathlineto{\pgfqpoint{4.327440in}{0.503110in}}%
\pgfpathclose%
\pgfusepath{stroke}%
\end{pgfscope}%
\begin{pgfscope}%
\pgfpathrectangle{\pgfqpoint{0.494722in}{0.437222in}}{\pgfqpoint{6.275590in}{5.159444in}}%
\pgfusepath{clip}%
\pgfsetbuttcap%
\pgfsetroundjoin%
\pgfsetlinewidth{1.003750pt}%
\definecolor{currentstroke}{rgb}{0.827451,0.827451,0.827451}%
\pgfsetstrokecolor{currentstroke}%
\pgfsetstrokeopacity{0.800000}%
\pgfsetdash{}{0pt}%
\pgfpathmoveto{\pgfqpoint{0.619224in}{3.610864in}}%
\pgfpathcurveto{\pgfqpoint{0.630274in}{3.610864in}}{\pgfqpoint{0.640874in}{3.615254in}}{\pgfqpoint{0.648687in}{3.623068in}}%
\pgfpathcurveto{\pgfqpoint{0.656501in}{3.630881in}}{\pgfqpoint{0.660891in}{3.641480in}}{\pgfqpoint{0.660891in}{3.652530in}}%
\pgfpathcurveto{\pgfqpoint{0.660891in}{3.663580in}}{\pgfqpoint{0.656501in}{3.674179in}}{\pgfqpoint{0.648687in}{3.681993in}}%
\pgfpathcurveto{\pgfqpoint{0.640874in}{3.689807in}}{\pgfqpoint{0.630274in}{3.694197in}}{\pgfqpoint{0.619224in}{3.694197in}}%
\pgfpathcurveto{\pgfqpoint{0.608174in}{3.694197in}}{\pgfqpoint{0.597575in}{3.689807in}}{\pgfqpoint{0.589762in}{3.681993in}}%
\pgfpathcurveto{\pgfqpoint{0.581948in}{3.674179in}}{\pgfqpoint{0.577558in}{3.663580in}}{\pgfqpoint{0.577558in}{3.652530in}}%
\pgfpathcurveto{\pgfqpoint{0.577558in}{3.641480in}}{\pgfqpoint{0.581948in}{3.630881in}}{\pgfqpoint{0.589762in}{3.623068in}}%
\pgfpathcurveto{\pgfqpoint{0.597575in}{3.615254in}}{\pgfqpoint{0.608174in}{3.610864in}}{\pgfqpoint{0.619224in}{3.610864in}}%
\pgfpathlineto{\pgfqpoint{0.619224in}{3.610864in}}%
\pgfpathclose%
\pgfusepath{stroke}%
\end{pgfscope}%
\begin{pgfscope}%
\pgfpathrectangle{\pgfqpoint{0.494722in}{0.437222in}}{\pgfqpoint{6.275590in}{5.159444in}}%
\pgfusepath{clip}%
\pgfsetbuttcap%
\pgfsetroundjoin%
\pgfsetlinewidth{1.003750pt}%
\definecolor{currentstroke}{rgb}{0.827451,0.827451,0.827451}%
\pgfsetstrokecolor{currentstroke}%
\pgfsetstrokeopacity{0.800000}%
\pgfsetdash{}{0pt}%
\pgfpathmoveto{\pgfqpoint{1.623472in}{1.825964in}}%
\pgfpathcurveto{\pgfqpoint{1.634522in}{1.825964in}}{\pgfqpoint{1.645121in}{1.830354in}}{\pgfqpoint{1.652935in}{1.838167in}}%
\pgfpathcurveto{\pgfqpoint{1.660748in}{1.845981in}}{\pgfqpoint{1.665139in}{1.856580in}}{\pgfqpoint{1.665139in}{1.867630in}}%
\pgfpathcurveto{\pgfqpoint{1.665139in}{1.878680in}}{\pgfqpoint{1.660748in}{1.889279in}}{\pgfqpoint{1.652935in}{1.897093in}}%
\pgfpathcurveto{\pgfqpoint{1.645121in}{1.904907in}}{\pgfqpoint{1.634522in}{1.909297in}}{\pgfqpoint{1.623472in}{1.909297in}}%
\pgfpathcurveto{\pgfqpoint{1.612422in}{1.909297in}}{\pgfqpoint{1.601823in}{1.904907in}}{\pgfqpoint{1.594009in}{1.897093in}}%
\pgfpathcurveto{\pgfqpoint{1.586196in}{1.889279in}}{\pgfqpoint{1.581805in}{1.878680in}}{\pgfqpoint{1.581805in}{1.867630in}}%
\pgfpathcurveto{\pgfqpoint{1.581805in}{1.856580in}}{\pgfqpoint{1.586196in}{1.845981in}}{\pgfqpoint{1.594009in}{1.838167in}}%
\pgfpathcurveto{\pgfqpoint{1.601823in}{1.830354in}}{\pgfqpoint{1.612422in}{1.825964in}}{\pgfqpoint{1.623472in}{1.825964in}}%
\pgfpathlineto{\pgfqpoint{1.623472in}{1.825964in}}%
\pgfpathclose%
\pgfusepath{stroke}%
\end{pgfscope}%
\begin{pgfscope}%
\pgfpathrectangle{\pgfqpoint{0.494722in}{0.437222in}}{\pgfqpoint{6.275590in}{5.159444in}}%
\pgfusepath{clip}%
\pgfsetbuttcap%
\pgfsetroundjoin%
\pgfsetlinewidth{1.003750pt}%
\definecolor{currentstroke}{rgb}{0.827451,0.827451,0.827451}%
\pgfsetstrokecolor{currentstroke}%
\pgfsetstrokeopacity{0.800000}%
\pgfsetdash{}{0pt}%
\pgfpathmoveto{\pgfqpoint{2.365416in}{1.256449in}}%
\pgfpathcurveto{\pgfqpoint{2.376466in}{1.256449in}}{\pgfqpoint{2.387065in}{1.260839in}}{\pgfqpoint{2.394879in}{1.268653in}}%
\pgfpathcurveto{\pgfqpoint{2.402693in}{1.276466in}}{\pgfqpoint{2.407083in}{1.287065in}}{\pgfqpoint{2.407083in}{1.298115in}}%
\pgfpathcurveto{\pgfqpoint{2.407083in}{1.309166in}}{\pgfqpoint{2.402693in}{1.319765in}}{\pgfqpoint{2.394879in}{1.327578in}}%
\pgfpathcurveto{\pgfqpoint{2.387065in}{1.335392in}}{\pgfqpoint{2.376466in}{1.339782in}}{\pgfqpoint{2.365416in}{1.339782in}}%
\pgfpathcurveto{\pgfqpoint{2.354366in}{1.339782in}}{\pgfqpoint{2.343767in}{1.335392in}}{\pgfqpoint{2.335953in}{1.327578in}}%
\pgfpathcurveto{\pgfqpoint{2.328140in}{1.319765in}}{\pgfqpoint{2.323750in}{1.309166in}}{\pgfqpoint{2.323750in}{1.298115in}}%
\pgfpathcurveto{\pgfqpoint{2.323750in}{1.287065in}}{\pgfqpoint{2.328140in}{1.276466in}}{\pgfqpoint{2.335953in}{1.268653in}}%
\pgfpathcurveto{\pgfqpoint{2.343767in}{1.260839in}}{\pgfqpoint{2.354366in}{1.256449in}}{\pgfqpoint{2.365416in}{1.256449in}}%
\pgfpathlineto{\pgfqpoint{2.365416in}{1.256449in}}%
\pgfpathclose%
\pgfusepath{stroke}%
\end{pgfscope}%
\begin{pgfscope}%
\pgfpathrectangle{\pgfqpoint{0.494722in}{0.437222in}}{\pgfqpoint{6.275590in}{5.159444in}}%
\pgfusepath{clip}%
\pgfsetbuttcap%
\pgfsetroundjoin%
\pgfsetlinewidth{1.003750pt}%
\definecolor{currentstroke}{rgb}{0.827451,0.827451,0.827451}%
\pgfsetstrokecolor{currentstroke}%
\pgfsetstrokeopacity{0.800000}%
\pgfsetdash{}{0pt}%
\pgfpathmoveto{\pgfqpoint{0.711186in}{3.233934in}}%
\pgfpathcurveto{\pgfqpoint{0.722236in}{3.233934in}}{\pgfqpoint{0.732835in}{3.238324in}}{\pgfqpoint{0.740648in}{3.246138in}}%
\pgfpathcurveto{\pgfqpoint{0.748462in}{3.253952in}}{\pgfqpoint{0.752852in}{3.264551in}}{\pgfqpoint{0.752852in}{3.275601in}}%
\pgfpathcurveto{\pgfqpoint{0.752852in}{3.286651in}}{\pgfqpoint{0.748462in}{3.297250in}}{\pgfqpoint{0.740648in}{3.305064in}}%
\pgfpathcurveto{\pgfqpoint{0.732835in}{3.312877in}}{\pgfqpoint{0.722236in}{3.317267in}}{\pgfqpoint{0.711186in}{3.317267in}}%
\pgfpathcurveto{\pgfqpoint{0.700135in}{3.317267in}}{\pgfqpoint{0.689536in}{3.312877in}}{\pgfqpoint{0.681723in}{3.305064in}}%
\pgfpathcurveto{\pgfqpoint{0.673909in}{3.297250in}}{\pgfqpoint{0.669519in}{3.286651in}}{\pgfqpoint{0.669519in}{3.275601in}}%
\pgfpathcurveto{\pgfqpoint{0.669519in}{3.264551in}}{\pgfqpoint{0.673909in}{3.253952in}}{\pgfqpoint{0.681723in}{3.246138in}}%
\pgfpathcurveto{\pgfqpoint{0.689536in}{3.238324in}}{\pgfqpoint{0.700135in}{3.233934in}}{\pgfqpoint{0.711186in}{3.233934in}}%
\pgfpathlineto{\pgfqpoint{0.711186in}{3.233934in}}%
\pgfpathclose%
\pgfusepath{stroke}%
\end{pgfscope}%
\begin{pgfscope}%
\pgfpathrectangle{\pgfqpoint{0.494722in}{0.437222in}}{\pgfqpoint{6.275590in}{5.159444in}}%
\pgfusepath{clip}%
\pgfsetbuttcap%
\pgfsetroundjoin%
\pgfsetlinewidth{1.003750pt}%
\definecolor{currentstroke}{rgb}{0.827451,0.827451,0.827451}%
\pgfsetstrokecolor{currentstroke}%
\pgfsetstrokeopacity{0.800000}%
\pgfsetdash{}{0pt}%
\pgfpathmoveto{\pgfqpoint{1.663088in}{1.812477in}}%
\pgfpathcurveto{\pgfqpoint{1.674138in}{1.812477in}}{\pgfqpoint{1.684737in}{1.816868in}}{\pgfqpoint{1.692551in}{1.824681in}}%
\pgfpathcurveto{\pgfqpoint{1.700364in}{1.832495in}}{\pgfqpoint{1.704755in}{1.843094in}}{\pgfqpoint{1.704755in}{1.854144in}}%
\pgfpathcurveto{\pgfqpoint{1.704755in}{1.865194in}}{\pgfqpoint{1.700364in}{1.875793in}}{\pgfqpoint{1.692551in}{1.883607in}}%
\pgfpathcurveto{\pgfqpoint{1.684737in}{1.891421in}}{\pgfqpoint{1.674138in}{1.895811in}}{\pgfqpoint{1.663088in}{1.895811in}}%
\pgfpathcurveto{\pgfqpoint{1.652038in}{1.895811in}}{\pgfqpoint{1.641439in}{1.891421in}}{\pgfqpoint{1.633625in}{1.883607in}}%
\pgfpathcurveto{\pgfqpoint{1.625812in}{1.875793in}}{\pgfqpoint{1.621421in}{1.865194in}}{\pgfqpoint{1.621421in}{1.854144in}}%
\pgfpathcurveto{\pgfqpoint{1.621421in}{1.843094in}}{\pgfqpoint{1.625812in}{1.832495in}}{\pgfqpoint{1.633625in}{1.824681in}}%
\pgfpathcurveto{\pgfqpoint{1.641439in}{1.816868in}}{\pgfqpoint{1.652038in}{1.812477in}}{\pgfqpoint{1.663088in}{1.812477in}}%
\pgfpathlineto{\pgfqpoint{1.663088in}{1.812477in}}%
\pgfpathclose%
\pgfusepath{stroke}%
\end{pgfscope}%
\begin{pgfscope}%
\pgfpathrectangle{\pgfqpoint{0.494722in}{0.437222in}}{\pgfqpoint{6.275590in}{5.159444in}}%
\pgfusepath{clip}%
\pgfsetbuttcap%
\pgfsetroundjoin%
\pgfsetlinewidth{1.003750pt}%
\definecolor{currentstroke}{rgb}{0.827451,0.827451,0.827451}%
\pgfsetstrokecolor{currentstroke}%
\pgfsetstrokeopacity{0.800000}%
\pgfsetdash{}{0pt}%
\pgfpathmoveto{\pgfqpoint{0.536273in}{3.987890in}}%
\pgfpathcurveto{\pgfqpoint{0.547324in}{3.987890in}}{\pgfqpoint{0.557923in}{3.992280in}}{\pgfqpoint{0.565736in}{4.000093in}}%
\pgfpathcurveto{\pgfqpoint{0.573550in}{4.007907in}}{\pgfqpoint{0.577940in}{4.018506in}}{\pgfqpoint{0.577940in}{4.029556in}}%
\pgfpathcurveto{\pgfqpoint{0.577940in}{4.040606in}}{\pgfqpoint{0.573550in}{4.051205in}}{\pgfqpoint{0.565736in}{4.059019in}}%
\pgfpathcurveto{\pgfqpoint{0.557923in}{4.066833in}}{\pgfqpoint{0.547324in}{4.071223in}}{\pgfqpoint{0.536273in}{4.071223in}}%
\pgfpathcurveto{\pgfqpoint{0.525223in}{4.071223in}}{\pgfqpoint{0.514624in}{4.066833in}}{\pgfqpoint{0.506811in}{4.059019in}}%
\pgfpathcurveto{\pgfqpoint{0.498997in}{4.051205in}}{\pgfqpoint{0.494607in}{4.040606in}}{\pgfqpoint{0.494607in}{4.029556in}}%
\pgfpathcurveto{\pgfqpoint{0.494607in}{4.018506in}}{\pgfqpoint{0.498997in}{4.007907in}}{\pgfqpoint{0.506811in}{4.000093in}}%
\pgfpathcurveto{\pgfqpoint{0.514624in}{3.992280in}}{\pgfqpoint{0.525223in}{3.987890in}}{\pgfqpoint{0.536273in}{3.987890in}}%
\pgfpathlineto{\pgfqpoint{0.536273in}{3.987890in}}%
\pgfpathclose%
\pgfusepath{stroke}%
\end{pgfscope}%
\begin{pgfscope}%
\pgfpathrectangle{\pgfqpoint{0.494722in}{0.437222in}}{\pgfqpoint{6.275590in}{5.159444in}}%
\pgfusepath{clip}%
\pgfsetbuttcap%
\pgfsetroundjoin%
\pgfsetlinewidth{1.003750pt}%
\definecolor{currentstroke}{rgb}{0.827451,0.827451,0.827451}%
\pgfsetstrokecolor{currentstroke}%
\pgfsetstrokeopacity{0.800000}%
\pgfsetdash{}{0pt}%
\pgfpathmoveto{\pgfqpoint{1.207472in}{2.314347in}}%
\pgfpathcurveto{\pgfqpoint{1.218522in}{2.314347in}}{\pgfqpoint{1.229121in}{2.318737in}}{\pgfqpoint{1.236935in}{2.326551in}}%
\pgfpathcurveto{\pgfqpoint{1.244749in}{2.334365in}}{\pgfqpoint{1.249139in}{2.344964in}}{\pgfqpoint{1.249139in}{2.356014in}}%
\pgfpathcurveto{\pgfqpoint{1.249139in}{2.367064in}}{\pgfqpoint{1.244749in}{2.377663in}}{\pgfqpoint{1.236935in}{2.385477in}}%
\pgfpathcurveto{\pgfqpoint{1.229121in}{2.393290in}}{\pgfqpoint{1.218522in}{2.397680in}}{\pgfqpoint{1.207472in}{2.397680in}}%
\pgfpathcurveto{\pgfqpoint{1.196422in}{2.397680in}}{\pgfqpoint{1.185823in}{2.393290in}}{\pgfqpoint{1.178010in}{2.385477in}}%
\pgfpathcurveto{\pgfqpoint{1.170196in}{2.377663in}}{\pgfqpoint{1.165806in}{2.367064in}}{\pgfqpoint{1.165806in}{2.356014in}}%
\pgfpathcurveto{\pgfqpoint{1.165806in}{2.344964in}}{\pgfqpoint{1.170196in}{2.334365in}}{\pgfqpoint{1.178010in}{2.326551in}}%
\pgfpathcurveto{\pgfqpoint{1.185823in}{2.318737in}}{\pgfqpoint{1.196422in}{2.314347in}}{\pgfqpoint{1.207472in}{2.314347in}}%
\pgfpathlineto{\pgfqpoint{1.207472in}{2.314347in}}%
\pgfpathclose%
\pgfusepath{stroke}%
\end{pgfscope}%
\begin{pgfscope}%
\pgfpathrectangle{\pgfqpoint{0.494722in}{0.437222in}}{\pgfqpoint{6.275590in}{5.159444in}}%
\pgfusepath{clip}%
\pgfsetbuttcap%
\pgfsetroundjoin%
\pgfsetlinewidth{1.003750pt}%
\definecolor{currentstroke}{rgb}{0.827451,0.827451,0.827451}%
\pgfsetstrokecolor{currentstroke}%
\pgfsetstrokeopacity{0.800000}%
\pgfsetdash{}{0pt}%
\pgfpathmoveto{\pgfqpoint{1.396743in}{2.042450in}}%
\pgfpathcurveto{\pgfqpoint{1.407793in}{2.042450in}}{\pgfqpoint{1.418392in}{2.046840in}}{\pgfqpoint{1.426206in}{2.054654in}}%
\pgfpathcurveto{\pgfqpoint{1.434019in}{2.062467in}}{\pgfqpoint{1.438409in}{2.073066in}}{\pgfqpoint{1.438409in}{2.084117in}}%
\pgfpathcurveto{\pgfqpoint{1.438409in}{2.095167in}}{\pgfqpoint{1.434019in}{2.105766in}}{\pgfqpoint{1.426206in}{2.113579in}}%
\pgfpathcurveto{\pgfqpoint{1.418392in}{2.121393in}}{\pgfqpoint{1.407793in}{2.125783in}}{\pgfqpoint{1.396743in}{2.125783in}}%
\pgfpathcurveto{\pgfqpoint{1.385693in}{2.125783in}}{\pgfqpoint{1.375094in}{2.121393in}}{\pgfqpoint{1.367280in}{2.113579in}}%
\pgfpathcurveto{\pgfqpoint{1.359466in}{2.105766in}}{\pgfqpoint{1.355076in}{2.095167in}}{\pgfqpoint{1.355076in}{2.084117in}}%
\pgfpathcurveto{\pgfqpoint{1.355076in}{2.073066in}}{\pgfqpoint{1.359466in}{2.062467in}}{\pgfqpoint{1.367280in}{2.054654in}}%
\pgfpathcurveto{\pgfqpoint{1.375094in}{2.046840in}}{\pgfqpoint{1.385693in}{2.042450in}}{\pgfqpoint{1.396743in}{2.042450in}}%
\pgfpathlineto{\pgfqpoint{1.396743in}{2.042450in}}%
\pgfpathclose%
\pgfusepath{stroke}%
\end{pgfscope}%
\begin{pgfscope}%
\pgfpathrectangle{\pgfqpoint{0.494722in}{0.437222in}}{\pgfqpoint{6.275590in}{5.159444in}}%
\pgfusepath{clip}%
\pgfsetbuttcap%
\pgfsetroundjoin%
\pgfsetlinewidth{1.003750pt}%
\definecolor{currentstroke}{rgb}{0.827451,0.827451,0.827451}%
\pgfsetstrokecolor{currentstroke}%
\pgfsetstrokeopacity{0.800000}%
\pgfsetdash{}{0pt}%
\pgfpathmoveto{\pgfqpoint{3.035676in}{0.882602in}}%
\pgfpathcurveto{\pgfqpoint{3.046726in}{0.882602in}}{\pgfqpoint{3.057325in}{0.886992in}}{\pgfqpoint{3.065138in}{0.894806in}}%
\pgfpathcurveto{\pgfqpoint{3.072952in}{0.902619in}}{\pgfqpoint{3.077342in}{0.913218in}}{\pgfqpoint{3.077342in}{0.924268in}}%
\pgfpathcurveto{\pgfqpoint{3.077342in}{0.935319in}}{\pgfqpoint{3.072952in}{0.945918in}}{\pgfqpoint{3.065138in}{0.953731in}}%
\pgfpathcurveto{\pgfqpoint{3.057325in}{0.961545in}}{\pgfqpoint{3.046726in}{0.965935in}}{\pgfqpoint{3.035676in}{0.965935in}}%
\pgfpathcurveto{\pgfqpoint{3.024626in}{0.965935in}}{\pgfqpoint{3.014027in}{0.961545in}}{\pgfqpoint{3.006213in}{0.953731in}}%
\pgfpathcurveto{\pgfqpoint{2.998399in}{0.945918in}}{\pgfqpoint{2.994009in}{0.935319in}}{\pgfqpoint{2.994009in}{0.924268in}}%
\pgfpathcurveto{\pgfqpoint{2.994009in}{0.913218in}}{\pgfqpoint{2.998399in}{0.902619in}}{\pgfqpoint{3.006213in}{0.894806in}}%
\pgfpathcurveto{\pgfqpoint{3.014027in}{0.886992in}}{\pgfqpoint{3.024626in}{0.882602in}}{\pgfqpoint{3.035676in}{0.882602in}}%
\pgfpathlineto{\pgfqpoint{3.035676in}{0.882602in}}%
\pgfpathclose%
\pgfusepath{stroke}%
\end{pgfscope}%
\begin{pgfscope}%
\pgfpathrectangle{\pgfqpoint{0.494722in}{0.437222in}}{\pgfqpoint{6.275590in}{5.159444in}}%
\pgfusepath{clip}%
\pgfsetbuttcap%
\pgfsetroundjoin%
\pgfsetlinewidth{1.003750pt}%
\definecolor{currentstroke}{rgb}{0.827451,0.827451,0.827451}%
\pgfsetstrokecolor{currentstroke}%
\pgfsetstrokeopacity{0.800000}%
\pgfsetdash{}{0pt}%
\pgfpathmoveto{\pgfqpoint{0.770298in}{3.077236in}}%
\pgfpathcurveto{\pgfqpoint{0.781349in}{3.077236in}}{\pgfqpoint{0.791948in}{3.081626in}}{\pgfqpoint{0.799761in}{3.089440in}}%
\pgfpathcurveto{\pgfqpoint{0.807575in}{3.097254in}}{\pgfqpoint{0.811965in}{3.107853in}}{\pgfqpoint{0.811965in}{3.118903in}}%
\pgfpathcurveto{\pgfqpoint{0.811965in}{3.129953in}}{\pgfqpoint{0.807575in}{3.140552in}}{\pgfqpoint{0.799761in}{3.148366in}}%
\pgfpathcurveto{\pgfqpoint{0.791948in}{3.156179in}}{\pgfqpoint{0.781349in}{3.160569in}}{\pgfqpoint{0.770298in}{3.160569in}}%
\pgfpathcurveto{\pgfqpoint{0.759248in}{3.160569in}}{\pgfqpoint{0.748649in}{3.156179in}}{\pgfqpoint{0.740836in}{3.148366in}}%
\pgfpathcurveto{\pgfqpoint{0.733022in}{3.140552in}}{\pgfqpoint{0.728632in}{3.129953in}}{\pgfqpoint{0.728632in}{3.118903in}}%
\pgfpathcurveto{\pgfqpoint{0.728632in}{3.107853in}}{\pgfqpoint{0.733022in}{3.097254in}}{\pgfqpoint{0.740836in}{3.089440in}}%
\pgfpathcurveto{\pgfqpoint{0.748649in}{3.081626in}}{\pgfqpoint{0.759248in}{3.077236in}}{\pgfqpoint{0.770298in}{3.077236in}}%
\pgfpathlineto{\pgfqpoint{0.770298in}{3.077236in}}%
\pgfpathclose%
\pgfusepath{stroke}%
\end{pgfscope}%
\begin{pgfscope}%
\pgfpathrectangle{\pgfqpoint{0.494722in}{0.437222in}}{\pgfqpoint{6.275590in}{5.159444in}}%
\pgfusepath{clip}%
\pgfsetbuttcap%
\pgfsetroundjoin%
\pgfsetlinewidth{1.003750pt}%
\definecolor{currentstroke}{rgb}{0.827451,0.827451,0.827451}%
\pgfsetstrokecolor{currentstroke}%
\pgfsetstrokeopacity{0.800000}%
\pgfsetdash{}{0pt}%
\pgfpathmoveto{\pgfqpoint{0.700772in}{3.267894in}}%
\pgfpathcurveto{\pgfqpoint{0.711822in}{3.267894in}}{\pgfqpoint{0.722421in}{3.272284in}}{\pgfqpoint{0.730235in}{3.280098in}}%
\pgfpathcurveto{\pgfqpoint{0.738048in}{3.287911in}}{\pgfqpoint{0.742438in}{3.298510in}}{\pgfqpoint{0.742438in}{3.309561in}}%
\pgfpathcurveto{\pgfqpoint{0.742438in}{3.320611in}}{\pgfqpoint{0.738048in}{3.331210in}}{\pgfqpoint{0.730235in}{3.339023in}}%
\pgfpathcurveto{\pgfqpoint{0.722421in}{3.346837in}}{\pgfqpoint{0.711822in}{3.351227in}}{\pgfqpoint{0.700772in}{3.351227in}}%
\pgfpathcurveto{\pgfqpoint{0.689722in}{3.351227in}}{\pgfqpoint{0.679123in}{3.346837in}}{\pgfqpoint{0.671309in}{3.339023in}}%
\pgfpathcurveto{\pgfqpoint{0.663495in}{3.331210in}}{\pgfqpoint{0.659105in}{3.320611in}}{\pgfqpoint{0.659105in}{3.309561in}}%
\pgfpathcurveto{\pgfqpoint{0.659105in}{3.298510in}}{\pgfqpoint{0.663495in}{3.287911in}}{\pgfqpoint{0.671309in}{3.280098in}}%
\pgfpathcurveto{\pgfqpoint{0.679123in}{3.272284in}}{\pgfqpoint{0.689722in}{3.267894in}}{\pgfqpoint{0.700772in}{3.267894in}}%
\pgfpathlineto{\pgfqpoint{0.700772in}{3.267894in}}%
\pgfpathclose%
\pgfusepath{stroke}%
\end{pgfscope}%
\begin{pgfscope}%
\pgfpathrectangle{\pgfqpoint{0.494722in}{0.437222in}}{\pgfqpoint{6.275590in}{5.159444in}}%
\pgfusepath{clip}%
\pgfsetbuttcap%
\pgfsetroundjoin%
\pgfsetlinewidth{1.003750pt}%
\definecolor{currentstroke}{rgb}{0.827451,0.827451,0.827451}%
\pgfsetstrokecolor{currentstroke}%
\pgfsetstrokeopacity{0.800000}%
\pgfsetdash{}{0pt}%
\pgfpathmoveto{\pgfqpoint{5.060352in}{0.429754in}}%
\pgfpathcurveto{\pgfqpoint{5.071402in}{0.429754in}}{\pgfqpoint{5.082001in}{0.434145in}}{\pgfqpoint{5.089815in}{0.441958in}}%
\pgfpathcurveto{\pgfqpoint{5.097629in}{0.449772in}}{\pgfqpoint{5.102019in}{0.460371in}}{\pgfqpoint{5.102019in}{0.471421in}}%
\pgfpathcurveto{\pgfqpoint{5.102019in}{0.482471in}}{\pgfqpoint{5.097629in}{0.493070in}}{\pgfqpoint{5.089815in}{0.500884in}}%
\pgfpathcurveto{\pgfqpoint{5.082001in}{0.508697in}}{\pgfqpoint{5.071402in}{0.513088in}}{\pgfqpoint{5.060352in}{0.513088in}}%
\pgfpathcurveto{\pgfqpoint{5.049302in}{0.513088in}}{\pgfqpoint{5.038703in}{0.508697in}}{\pgfqpoint{5.030890in}{0.500884in}}%
\pgfpathcurveto{\pgfqpoint{5.023076in}{0.493070in}}{\pgfqpoint{5.018686in}{0.482471in}}{\pgfqpoint{5.018686in}{0.471421in}}%
\pgfpathcurveto{\pgfqpoint{5.018686in}{0.460371in}}{\pgfqpoint{5.023076in}{0.449772in}}{\pgfqpoint{5.030890in}{0.441958in}}%
\pgfpathcurveto{\pgfqpoint{5.038703in}{0.434145in}}{\pgfqpoint{5.049302in}{0.429754in}}{\pgfqpoint{5.060352in}{0.429754in}}%
\pgfpathlineto{\pgfqpoint{5.060352in}{0.429754in}}%
\pgfpathclose%
\pgfusepath{stroke}%
\end{pgfscope}%
\begin{pgfscope}%
\pgfpathrectangle{\pgfqpoint{0.494722in}{0.437222in}}{\pgfqpoint{6.275590in}{5.159444in}}%
\pgfusepath{clip}%
\pgfsetbuttcap%
\pgfsetroundjoin%
\pgfsetlinewidth{1.003750pt}%
\definecolor{currentstroke}{rgb}{0.827451,0.827451,0.827451}%
\pgfsetstrokecolor{currentstroke}%
\pgfsetstrokeopacity{0.800000}%
\pgfsetdash{}{0pt}%
\pgfpathmoveto{\pgfqpoint{1.347485in}{2.096739in}}%
\pgfpathcurveto{\pgfqpoint{1.358535in}{2.096739in}}{\pgfqpoint{1.369135in}{2.101129in}}{\pgfqpoint{1.376948in}{2.108943in}}%
\pgfpathcurveto{\pgfqpoint{1.384762in}{2.116757in}}{\pgfqpoint{1.389152in}{2.127356in}}{\pgfqpoint{1.389152in}{2.138406in}}%
\pgfpathcurveto{\pgfqpoint{1.389152in}{2.149456in}}{\pgfqpoint{1.384762in}{2.160055in}}{\pgfqpoint{1.376948in}{2.167869in}}%
\pgfpathcurveto{\pgfqpoint{1.369135in}{2.175682in}}{\pgfqpoint{1.358535in}{2.180072in}}{\pgfqpoint{1.347485in}{2.180072in}}%
\pgfpathcurveto{\pgfqpoint{1.336435in}{2.180072in}}{\pgfqpoint{1.325836in}{2.175682in}}{\pgfqpoint{1.318023in}{2.167869in}}%
\pgfpathcurveto{\pgfqpoint{1.310209in}{2.160055in}}{\pgfqpoint{1.305819in}{2.149456in}}{\pgfqpoint{1.305819in}{2.138406in}}%
\pgfpathcurveto{\pgfqpoint{1.305819in}{2.127356in}}{\pgfqpoint{1.310209in}{2.116757in}}{\pgfqpoint{1.318023in}{2.108943in}}%
\pgfpathcurveto{\pgfqpoint{1.325836in}{2.101129in}}{\pgfqpoint{1.336435in}{2.096739in}}{\pgfqpoint{1.347485in}{2.096739in}}%
\pgfpathlineto{\pgfqpoint{1.347485in}{2.096739in}}%
\pgfpathclose%
\pgfusepath{stroke}%
\end{pgfscope}%
\begin{pgfscope}%
\pgfpathrectangle{\pgfqpoint{0.494722in}{0.437222in}}{\pgfqpoint{6.275590in}{5.159444in}}%
\pgfusepath{clip}%
\pgfsetbuttcap%
\pgfsetroundjoin%
\pgfsetlinewidth{1.003750pt}%
\definecolor{currentstroke}{rgb}{0.827451,0.827451,0.827451}%
\pgfsetstrokecolor{currentstroke}%
\pgfsetstrokeopacity{0.800000}%
\pgfsetdash{}{0pt}%
\pgfpathmoveto{\pgfqpoint{2.659012in}{1.065276in}}%
\pgfpathcurveto{\pgfqpoint{2.670062in}{1.065276in}}{\pgfqpoint{2.680661in}{1.069667in}}{\pgfqpoint{2.688475in}{1.077480in}}%
\pgfpathcurveto{\pgfqpoint{2.696289in}{1.085294in}}{\pgfqpoint{2.700679in}{1.095893in}}{\pgfqpoint{2.700679in}{1.106943in}}%
\pgfpathcurveto{\pgfqpoint{2.700679in}{1.117993in}}{\pgfqpoint{2.696289in}{1.128592in}}{\pgfqpoint{2.688475in}{1.136406in}}%
\pgfpathcurveto{\pgfqpoint{2.680661in}{1.144219in}}{\pgfqpoint{2.670062in}{1.148610in}}{\pgfqpoint{2.659012in}{1.148610in}}%
\pgfpathcurveto{\pgfqpoint{2.647962in}{1.148610in}}{\pgfqpoint{2.637363in}{1.144219in}}{\pgfqpoint{2.629550in}{1.136406in}}%
\pgfpathcurveto{\pgfqpoint{2.621736in}{1.128592in}}{\pgfqpoint{2.617346in}{1.117993in}}{\pgfqpoint{2.617346in}{1.106943in}}%
\pgfpathcurveto{\pgfqpoint{2.617346in}{1.095893in}}{\pgfqpoint{2.621736in}{1.085294in}}{\pgfqpoint{2.629550in}{1.077480in}}%
\pgfpathcurveto{\pgfqpoint{2.637363in}{1.069667in}}{\pgfqpoint{2.647962in}{1.065276in}}{\pgfqpoint{2.659012in}{1.065276in}}%
\pgfpathlineto{\pgfqpoint{2.659012in}{1.065276in}}%
\pgfpathclose%
\pgfusepath{stroke}%
\end{pgfscope}%
\begin{pgfscope}%
\pgfpathrectangle{\pgfqpoint{0.494722in}{0.437222in}}{\pgfqpoint{6.275590in}{5.159444in}}%
\pgfusepath{clip}%
\pgfsetbuttcap%
\pgfsetroundjoin%
\pgfsetlinewidth{1.003750pt}%
\definecolor{currentstroke}{rgb}{0.827451,0.827451,0.827451}%
\pgfsetstrokecolor{currentstroke}%
\pgfsetstrokeopacity{0.800000}%
\pgfsetdash{}{0pt}%
\pgfpathmoveto{\pgfqpoint{2.990259in}{0.911738in}}%
\pgfpathcurveto{\pgfqpoint{3.001309in}{0.911738in}}{\pgfqpoint{3.011908in}{0.916128in}}{\pgfqpoint{3.019722in}{0.923942in}}%
\pgfpathcurveto{\pgfqpoint{3.027536in}{0.931755in}}{\pgfqpoint{3.031926in}{0.942354in}}{\pgfqpoint{3.031926in}{0.953404in}}%
\pgfpathcurveto{\pgfqpoint{3.031926in}{0.964455in}}{\pgfqpoint{3.027536in}{0.975054in}}{\pgfqpoint{3.019722in}{0.982867in}}%
\pgfpathcurveto{\pgfqpoint{3.011908in}{0.990681in}}{\pgfqpoint{3.001309in}{0.995071in}}{\pgfqpoint{2.990259in}{0.995071in}}%
\pgfpathcurveto{\pgfqpoint{2.979209in}{0.995071in}}{\pgfqpoint{2.968610in}{0.990681in}}{\pgfqpoint{2.960796in}{0.982867in}}%
\pgfpathcurveto{\pgfqpoint{2.952983in}{0.975054in}}{\pgfqpoint{2.948593in}{0.964455in}}{\pgfqpoint{2.948593in}{0.953404in}}%
\pgfpathcurveto{\pgfqpoint{2.948593in}{0.942354in}}{\pgfqpoint{2.952983in}{0.931755in}}{\pgfqpoint{2.960796in}{0.923942in}}%
\pgfpathcurveto{\pgfqpoint{2.968610in}{0.916128in}}{\pgfqpoint{2.979209in}{0.911738in}}{\pgfqpoint{2.990259in}{0.911738in}}%
\pgfpathlineto{\pgfqpoint{2.990259in}{0.911738in}}%
\pgfpathclose%
\pgfusepath{stroke}%
\end{pgfscope}%
\begin{pgfscope}%
\pgfpathrectangle{\pgfqpoint{0.494722in}{0.437222in}}{\pgfqpoint{6.275590in}{5.159444in}}%
\pgfusepath{clip}%
\pgfsetbuttcap%
\pgfsetroundjoin%
\pgfsetlinewidth{1.003750pt}%
\definecolor{currentstroke}{rgb}{0.827451,0.827451,0.827451}%
\pgfsetstrokecolor{currentstroke}%
\pgfsetstrokeopacity{0.800000}%
\pgfsetdash{}{0pt}%
\pgfpathmoveto{\pgfqpoint{1.091684in}{2.547519in}}%
\pgfpathcurveto{\pgfqpoint{1.102734in}{2.547519in}}{\pgfqpoint{1.113333in}{2.551909in}}{\pgfqpoint{1.121147in}{2.559722in}}%
\pgfpathcurveto{\pgfqpoint{1.128960in}{2.567536in}}{\pgfqpoint{1.133351in}{2.578135in}}{\pgfqpoint{1.133351in}{2.589185in}}%
\pgfpathcurveto{\pgfqpoint{1.133351in}{2.600235in}}{\pgfqpoint{1.128960in}{2.610834in}}{\pgfqpoint{1.121147in}{2.618648in}}%
\pgfpathcurveto{\pgfqpoint{1.113333in}{2.626462in}}{\pgfqpoint{1.102734in}{2.630852in}}{\pgfqpoint{1.091684in}{2.630852in}}%
\pgfpathcurveto{\pgfqpoint{1.080634in}{2.630852in}}{\pgfqpoint{1.070035in}{2.626462in}}{\pgfqpoint{1.062221in}{2.618648in}}%
\pgfpathcurveto{\pgfqpoint{1.054407in}{2.610834in}}{\pgfqpoint{1.050017in}{2.600235in}}{\pgfqpoint{1.050017in}{2.589185in}}%
\pgfpathcurveto{\pgfqpoint{1.050017in}{2.578135in}}{\pgfqpoint{1.054407in}{2.567536in}}{\pgfqpoint{1.062221in}{2.559722in}}%
\pgfpathcurveto{\pgfqpoint{1.070035in}{2.551909in}}{\pgfqpoint{1.080634in}{2.547519in}}{\pgfqpoint{1.091684in}{2.547519in}}%
\pgfpathlineto{\pgfqpoint{1.091684in}{2.547519in}}%
\pgfpathclose%
\pgfusepath{stroke}%
\end{pgfscope}%
\begin{pgfscope}%
\pgfpathrectangle{\pgfqpoint{0.494722in}{0.437222in}}{\pgfqpoint{6.275590in}{5.159444in}}%
\pgfusepath{clip}%
\pgfsetbuttcap%
\pgfsetroundjoin%
\pgfsetlinewidth{1.003750pt}%
\definecolor{currentstroke}{rgb}{0.827451,0.827451,0.827451}%
\pgfsetstrokecolor{currentstroke}%
\pgfsetstrokeopacity{0.800000}%
\pgfsetdash{}{0pt}%
\pgfpathmoveto{\pgfqpoint{0.944970in}{2.756986in}}%
\pgfpathcurveto{\pgfqpoint{0.956020in}{2.756986in}}{\pgfqpoint{0.966619in}{2.761376in}}{\pgfqpoint{0.974433in}{2.769190in}}%
\pgfpathcurveto{\pgfqpoint{0.982246in}{2.777004in}}{\pgfqpoint{0.986637in}{2.787603in}}{\pgfqpoint{0.986637in}{2.798653in}}%
\pgfpathcurveto{\pgfqpoint{0.986637in}{2.809703in}}{\pgfqpoint{0.982246in}{2.820302in}}{\pgfqpoint{0.974433in}{2.828116in}}%
\pgfpathcurveto{\pgfqpoint{0.966619in}{2.835929in}}{\pgfqpoint{0.956020in}{2.840319in}}{\pgfqpoint{0.944970in}{2.840319in}}%
\pgfpathcurveto{\pgfqpoint{0.933920in}{2.840319in}}{\pgfqpoint{0.923321in}{2.835929in}}{\pgfqpoint{0.915507in}{2.828116in}}%
\pgfpathcurveto{\pgfqpoint{0.907694in}{2.820302in}}{\pgfqpoint{0.903303in}{2.809703in}}{\pgfqpoint{0.903303in}{2.798653in}}%
\pgfpathcurveto{\pgfqpoint{0.903303in}{2.787603in}}{\pgfqpoint{0.907694in}{2.777004in}}{\pgfqpoint{0.915507in}{2.769190in}}%
\pgfpathcurveto{\pgfqpoint{0.923321in}{2.761376in}}{\pgfqpoint{0.933920in}{2.756986in}}{\pgfqpoint{0.944970in}{2.756986in}}%
\pgfpathlineto{\pgfqpoint{0.944970in}{2.756986in}}%
\pgfpathclose%
\pgfusepath{stroke}%
\end{pgfscope}%
\begin{pgfscope}%
\pgfpathrectangle{\pgfqpoint{0.494722in}{0.437222in}}{\pgfqpoint{6.275590in}{5.159444in}}%
\pgfusepath{clip}%
\pgfsetbuttcap%
\pgfsetroundjoin%
\pgfsetlinewidth{1.003750pt}%
\definecolor{currentstroke}{rgb}{0.827451,0.827451,0.827451}%
\pgfsetstrokecolor{currentstroke}%
\pgfsetstrokeopacity{0.800000}%
\pgfsetdash{}{0pt}%
\pgfpathmoveto{\pgfqpoint{1.444344in}{1.986059in}}%
\pgfpathcurveto{\pgfqpoint{1.455394in}{1.986059in}}{\pgfqpoint{1.465993in}{1.990450in}}{\pgfqpoint{1.473807in}{1.998263in}}%
\pgfpathcurveto{\pgfqpoint{1.481620in}{2.006077in}}{\pgfqpoint{1.486010in}{2.016676in}}{\pgfqpoint{1.486010in}{2.027726in}}%
\pgfpathcurveto{\pgfqpoint{1.486010in}{2.038776in}}{\pgfqpoint{1.481620in}{2.049375in}}{\pgfqpoint{1.473807in}{2.057189in}}%
\pgfpathcurveto{\pgfqpoint{1.465993in}{2.065002in}}{\pgfqpoint{1.455394in}{2.069393in}}{\pgfqpoint{1.444344in}{2.069393in}}%
\pgfpathcurveto{\pgfqpoint{1.433294in}{2.069393in}}{\pgfqpoint{1.422695in}{2.065002in}}{\pgfqpoint{1.414881in}{2.057189in}}%
\pgfpathcurveto{\pgfqpoint{1.407067in}{2.049375in}}{\pgfqpoint{1.402677in}{2.038776in}}{\pgfqpoint{1.402677in}{2.027726in}}%
\pgfpathcurveto{\pgfqpoint{1.402677in}{2.016676in}}{\pgfqpoint{1.407067in}{2.006077in}}{\pgfqpoint{1.414881in}{1.998263in}}%
\pgfpathcurveto{\pgfqpoint{1.422695in}{1.990450in}}{\pgfqpoint{1.433294in}{1.986059in}}{\pgfqpoint{1.444344in}{1.986059in}}%
\pgfpathlineto{\pgfqpoint{1.444344in}{1.986059in}}%
\pgfpathclose%
\pgfusepath{stroke}%
\end{pgfscope}%
\begin{pgfscope}%
\pgfpathrectangle{\pgfqpoint{0.494722in}{0.437222in}}{\pgfqpoint{6.275590in}{5.159444in}}%
\pgfusepath{clip}%
\pgfsetbuttcap%
\pgfsetroundjoin%
\pgfsetlinewidth{1.003750pt}%
\definecolor{currentstroke}{rgb}{0.827451,0.827451,0.827451}%
\pgfsetstrokecolor{currentstroke}%
\pgfsetstrokeopacity{0.800000}%
\pgfsetdash{}{0pt}%
\pgfpathmoveto{\pgfqpoint{1.852529in}{1.617217in}}%
\pgfpathcurveto{\pgfqpoint{1.863579in}{1.617217in}}{\pgfqpoint{1.874178in}{1.621608in}}{\pgfqpoint{1.881991in}{1.629421in}}%
\pgfpathcurveto{\pgfqpoint{1.889805in}{1.637235in}}{\pgfqpoint{1.894195in}{1.647834in}}{\pgfqpoint{1.894195in}{1.658884in}}%
\pgfpathcurveto{\pgfqpoint{1.894195in}{1.669934in}}{\pgfqpoint{1.889805in}{1.680533in}}{\pgfqpoint{1.881991in}{1.688347in}}%
\pgfpathcurveto{\pgfqpoint{1.874178in}{1.696160in}}{\pgfqpoint{1.863579in}{1.700551in}}{\pgfqpoint{1.852529in}{1.700551in}}%
\pgfpathcurveto{\pgfqpoint{1.841479in}{1.700551in}}{\pgfqpoint{1.830880in}{1.696160in}}{\pgfqpoint{1.823066in}{1.688347in}}%
\pgfpathcurveto{\pgfqpoint{1.815252in}{1.680533in}}{\pgfqpoint{1.810862in}{1.669934in}}{\pgfqpoint{1.810862in}{1.658884in}}%
\pgfpathcurveto{\pgfqpoint{1.810862in}{1.647834in}}{\pgfqpoint{1.815252in}{1.637235in}}{\pgfqpoint{1.823066in}{1.629421in}}%
\pgfpathcurveto{\pgfqpoint{1.830880in}{1.621608in}}{\pgfqpoint{1.841479in}{1.617217in}}{\pgfqpoint{1.852529in}{1.617217in}}%
\pgfpathlineto{\pgfqpoint{1.852529in}{1.617217in}}%
\pgfpathclose%
\pgfusepath{stroke}%
\end{pgfscope}%
\begin{pgfscope}%
\pgfpathrectangle{\pgfqpoint{0.494722in}{0.437222in}}{\pgfqpoint{6.275590in}{5.159444in}}%
\pgfusepath{clip}%
\pgfsetbuttcap%
\pgfsetroundjoin%
\pgfsetlinewidth{1.003750pt}%
\definecolor{currentstroke}{rgb}{0.827451,0.827451,0.827451}%
\pgfsetstrokecolor{currentstroke}%
\pgfsetstrokeopacity{0.800000}%
\pgfsetdash{}{0pt}%
\pgfpathmoveto{\pgfqpoint{0.680103in}{3.350578in}}%
\pgfpathcurveto{\pgfqpoint{0.691153in}{3.350578in}}{\pgfqpoint{0.701753in}{3.354968in}}{\pgfqpoint{0.709566in}{3.362782in}}%
\pgfpathcurveto{\pgfqpoint{0.717380in}{3.370596in}}{\pgfqpoint{0.721770in}{3.381195in}}{\pgfqpoint{0.721770in}{3.392245in}}%
\pgfpathcurveto{\pgfqpoint{0.721770in}{3.403295in}}{\pgfqpoint{0.717380in}{3.413894in}}{\pgfqpoint{0.709566in}{3.421707in}}%
\pgfpathcurveto{\pgfqpoint{0.701753in}{3.429521in}}{\pgfqpoint{0.691153in}{3.433911in}}{\pgfqpoint{0.680103in}{3.433911in}}%
\pgfpathcurveto{\pgfqpoint{0.669053in}{3.433911in}}{\pgfqpoint{0.658454in}{3.429521in}}{\pgfqpoint{0.650641in}{3.421707in}}%
\pgfpathcurveto{\pgfqpoint{0.642827in}{3.413894in}}{\pgfqpoint{0.638437in}{3.403295in}}{\pgfqpoint{0.638437in}{3.392245in}}%
\pgfpathcurveto{\pgfqpoint{0.638437in}{3.381195in}}{\pgfqpoint{0.642827in}{3.370596in}}{\pgfqpoint{0.650641in}{3.362782in}}%
\pgfpathcurveto{\pgfqpoint{0.658454in}{3.354968in}}{\pgfqpoint{0.669053in}{3.350578in}}{\pgfqpoint{0.680103in}{3.350578in}}%
\pgfpathlineto{\pgfqpoint{0.680103in}{3.350578in}}%
\pgfpathclose%
\pgfusepath{stroke}%
\end{pgfscope}%
\begin{pgfscope}%
\pgfpathrectangle{\pgfqpoint{0.494722in}{0.437222in}}{\pgfqpoint{6.275590in}{5.159444in}}%
\pgfusepath{clip}%
\pgfsetbuttcap%
\pgfsetroundjoin%
\pgfsetlinewidth{1.003750pt}%
\definecolor{currentstroke}{rgb}{0.827451,0.827451,0.827451}%
\pgfsetstrokecolor{currentstroke}%
\pgfsetstrokeopacity{0.800000}%
\pgfsetdash{}{0pt}%
\pgfpathmoveto{\pgfqpoint{0.553084in}{3.907249in}}%
\pgfpathcurveto{\pgfqpoint{0.564134in}{3.907249in}}{\pgfqpoint{0.574733in}{3.911639in}}{\pgfqpoint{0.582547in}{3.919453in}}%
\pgfpathcurveto{\pgfqpoint{0.590360in}{3.927267in}}{\pgfqpoint{0.594750in}{3.937866in}}{\pgfqpoint{0.594750in}{3.948916in}}%
\pgfpathcurveto{\pgfqpoint{0.594750in}{3.959966in}}{\pgfqpoint{0.590360in}{3.970565in}}{\pgfqpoint{0.582547in}{3.978378in}}%
\pgfpathcurveto{\pgfqpoint{0.574733in}{3.986192in}}{\pgfqpoint{0.564134in}{3.990582in}}{\pgfqpoint{0.553084in}{3.990582in}}%
\pgfpathcurveto{\pgfqpoint{0.542034in}{3.990582in}}{\pgfqpoint{0.531435in}{3.986192in}}{\pgfqpoint{0.523621in}{3.978378in}}%
\pgfpathcurveto{\pgfqpoint{0.515807in}{3.970565in}}{\pgfqpoint{0.511417in}{3.959966in}}{\pgfqpoint{0.511417in}{3.948916in}}%
\pgfpathcurveto{\pgfqpoint{0.511417in}{3.937866in}}{\pgfqpoint{0.515807in}{3.927267in}}{\pgfqpoint{0.523621in}{3.919453in}}%
\pgfpathcurveto{\pgfqpoint{0.531435in}{3.911639in}}{\pgfqpoint{0.542034in}{3.907249in}}{\pgfqpoint{0.553084in}{3.907249in}}%
\pgfpathlineto{\pgfqpoint{0.553084in}{3.907249in}}%
\pgfpathclose%
\pgfusepath{stroke}%
\end{pgfscope}%
\begin{pgfscope}%
\pgfpathrectangle{\pgfqpoint{0.494722in}{0.437222in}}{\pgfqpoint{6.275590in}{5.159444in}}%
\pgfusepath{clip}%
\pgfsetbuttcap%
\pgfsetroundjoin%
\pgfsetlinewidth{1.003750pt}%
\definecolor{currentstroke}{rgb}{0.827451,0.827451,0.827451}%
\pgfsetstrokecolor{currentstroke}%
\pgfsetstrokeopacity{0.800000}%
\pgfsetdash{}{0pt}%
\pgfpathmoveto{\pgfqpoint{4.229480in}{0.518110in}}%
\pgfpathcurveto{\pgfqpoint{4.240530in}{0.518110in}}{\pgfqpoint{4.251129in}{0.522500in}}{\pgfqpoint{4.258942in}{0.530314in}}%
\pgfpathcurveto{\pgfqpoint{4.266756in}{0.538127in}}{\pgfqpoint{4.271146in}{0.548726in}}{\pgfqpoint{4.271146in}{0.559776in}}%
\pgfpathcurveto{\pgfqpoint{4.271146in}{0.570827in}}{\pgfqpoint{4.266756in}{0.581426in}}{\pgfqpoint{4.258942in}{0.589239in}}%
\pgfpathcurveto{\pgfqpoint{4.251129in}{0.597053in}}{\pgfqpoint{4.240530in}{0.601443in}}{\pgfqpoint{4.229480in}{0.601443in}}%
\pgfpathcurveto{\pgfqpoint{4.218429in}{0.601443in}}{\pgfqpoint{4.207830in}{0.597053in}}{\pgfqpoint{4.200017in}{0.589239in}}%
\pgfpathcurveto{\pgfqpoint{4.192203in}{0.581426in}}{\pgfqpoint{4.187813in}{0.570827in}}{\pgfqpoint{4.187813in}{0.559776in}}%
\pgfpathcurveto{\pgfqpoint{4.187813in}{0.548726in}}{\pgfqpoint{4.192203in}{0.538127in}}{\pgfqpoint{4.200017in}{0.530314in}}%
\pgfpathcurveto{\pgfqpoint{4.207830in}{0.522500in}}{\pgfqpoint{4.218429in}{0.518110in}}{\pgfqpoint{4.229480in}{0.518110in}}%
\pgfpathlineto{\pgfqpoint{4.229480in}{0.518110in}}%
\pgfpathclose%
\pgfusepath{stroke}%
\end{pgfscope}%
\begin{pgfscope}%
\pgfpathrectangle{\pgfqpoint{0.494722in}{0.437222in}}{\pgfqpoint{6.275590in}{5.159444in}}%
\pgfusepath{clip}%
\pgfsetbuttcap%
\pgfsetroundjoin%
\pgfsetlinewidth{1.003750pt}%
\definecolor{currentstroke}{rgb}{0.827451,0.827451,0.827451}%
\pgfsetstrokecolor{currentstroke}%
\pgfsetstrokeopacity{0.800000}%
\pgfsetdash{}{0pt}%
\pgfpathmoveto{\pgfqpoint{4.970960in}{0.441718in}}%
\pgfpathcurveto{\pgfqpoint{4.982010in}{0.441718in}}{\pgfqpoint{4.992609in}{0.446108in}}{\pgfqpoint{5.000423in}{0.453922in}}%
\pgfpathcurveto{\pgfqpoint{5.008236in}{0.461735in}}{\pgfqpoint{5.012626in}{0.472334in}}{\pgfqpoint{5.012626in}{0.483385in}}%
\pgfpathcurveto{\pgfqpoint{5.012626in}{0.494435in}}{\pgfqpoint{5.008236in}{0.505034in}}{\pgfqpoint{5.000423in}{0.512847in}}%
\pgfpathcurveto{\pgfqpoint{4.992609in}{0.520661in}}{\pgfqpoint{4.982010in}{0.525051in}}{\pgfqpoint{4.970960in}{0.525051in}}%
\pgfpathcurveto{\pgfqpoint{4.959910in}{0.525051in}}{\pgfqpoint{4.949311in}{0.520661in}}{\pgfqpoint{4.941497in}{0.512847in}}%
\pgfpathcurveto{\pgfqpoint{4.933683in}{0.505034in}}{\pgfqpoint{4.929293in}{0.494435in}}{\pgfqpoint{4.929293in}{0.483385in}}%
\pgfpathcurveto{\pgfqpoint{4.929293in}{0.472334in}}{\pgfqpoint{4.933683in}{0.461735in}}{\pgfqpoint{4.941497in}{0.453922in}}%
\pgfpathcurveto{\pgfqpoint{4.949311in}{0.446108in}}{\pgfqpoint{4.959910in}{0.441718in}}{\pgfqpoint{4.970960in}{0.441718in}}%
\pgfpathlineto{\pgfqpoint{4.970960in}{0.441718in}}%
\pgfpathclose%
\pgfusepath{stroke}%
\end{pgfscope}%
\begin{pgfscope}%
\pgfpathrectangle{\pgfqpoint{0.494722in}{0.437222in}}{\pgfqpoint{6.275590in}{5.159444in}}%
\pgfusepath{clip}%
\pgfsetbuttcap%
\pgfsetroundjoin%
\pgfsetlinewidth{1.003750pt}%
\definecolor{currentstroke}{rgb}{0.827451,0.827451,0.827451}%
\pgfsetstrokecolor{currentstroke}%
\pgfsetstrokeopacity{0.800000}%
\pgfsetdash{}{0pt}%
\pgfpathmoveto{\pgfqpoint{3.800468in}{0.625522in}}%
\pgfpathcurveto{\pgfqpoint{3.811518in}{0.625522in}}{\pgfqpoint{3.822117in}{0.629912in}}{\pgfqpoint{3.829931in}{0.637726in}}%
\pgfpathcurveto{\pgfqpoint{3.837744in}{0.645540in}}{\pgfqpoint{3.842134in}{0.656139in}}{\pgfqpoint{3.842134in}{0.667189in}}%
\pgfpathcurveto{\pgfqpoint{3.842134in}{0.678239in}}{\pgfqpoint{3.837744in}{0.688838in}}{\pgfqpoint{3.829931in}{0.696652in}}%
\pgfpathcurveto{\pgfqpoint{3.822117in}{0.704465in}}{\pgfqpoint{3.811518in}{0.708856in}}{\pgfqpoint{3.800468in}{0.708856in}}%
\pgfpathcurveto{\pgfqpoint{3.789418in}{0.708856in}}{\pgfqpoint{3.778819in}{0.704465in}}{\pgfqpoint{3.771005in}{0.696652in}}%
\pgfpathcurveto{\pgfqpoint{3.763191in}{0.688838in}}{\pgfqpoint{3.758801in}{0.678239in}}{\pgfqpoint{3.758801in}{0.667189in}}%
\pgfpathcurveto{\pgfqpoint{3.758801in}{0.656139in}}{\pgfqpoint{3.763191in}{0.645540in}}{\pgfqpoint{3.771005in}{0.637726in}}%
\pgfpathcurveto{\pgfqpoint{3.778819in}{0.629912in}}{\pgfqpoint{3.789418in}{0.625522in}}{\pgfqpoint{3.800468in}{0.625522in}}%
\pgfpathlineto{\pgfqpoint{3.800468in}{0.625522in}}%
\pgfpathclose%
\pgfusepath{stroke}%
\end{pgfscope}%
\begin{pgfscope}%
\pgfpathrectangle{\pgfqpoint{0.494722in}{0.437222in}}{\pgfqpoint{6.275590in}{5.159444in}}%
\pgfusepath{clip}%
\pgfsetbuttcap%
\pgfsetroundjoin%
\pgfsetlinewidth{1.003750pt}%
\definecolor{currentstroke}{rgb}{0.827451,0.827451,0.827451}%
\pgfsetstrokecolor{currentstroke}%
\pgfsetstrokeopacity{0.800000}%
\pgfsetdash{}{0pt}%
\pgfpathmoveto{\pgfqpoint{0.841077in}{2.944285in}}%
\pgfpathcurveto{\pgfqpoint{0.852127in}{2.944285in}}{\pgfqpoint{0.862726in}{2.948676in}}{\pgfqpoint{0.870540in}{2.956489in}}%
\pgfpathcurveto{\pgfqpoint{0.878353in}{2.964303in}}{\pgfqpoint{0.882744in}{2.974902in}}{\pgfqpoint{0.882744in}{2.985952in}}%
\pgfpathcurveto{\pgfqpoint{0.882744in}{2.997002in}}{\pgfqpoint{0.878353in}{3.007601in}}{\pgfqpoint{0.870540in}{3.015415in}}%
\pgfpathcurveto{\pgfqpoint{0.862726in}{3.023228in}}{\pgfqpoint{0.852127in}{3.027619in}}{\pgfqpoint{0.841077in}{3.027619in}}%
\pgfpathcurveto{\pgfqpoint{0.830027in}{3.027619in}}{\pgfqpoint{0.819428in}{3.023228in}}{\pgfqpoint{0.811614in}{3.015415in}}%
\pgfpathcurveto{\pgfqpoint{0.803801in}{3.007601in}}{\pgfqpoint{0.799410in}{2.997002in}}{\pgfqpoint{0.799410in}{2.985952in}}%
\pgfpathcurveto{\pgfqpoint{0.799410in}{2.974902in}}{\pgfqpoint{0.803801in}{2.964303in}}{\pgfqpoint{0.811614in}{2.956489in}}%
\pgfpathcurveto{\pgfqpoint{0.819428in}{2.948676in}}{\pgfqpoint{0.830027in}{2.944285in}}{\pgfqpoint{0.841077in}{2.944285in}}%
\pgfpathlineto{\pgfqpoint{0.841077in}{2.944285in}}%
\pgfpathclose%
\pgfusepath{stroke}%
\end{pgfscope}%
\begin{pgfscope}%
\pgfpathrectangle{\pgfqpoint{0.494722in}{0.437222in}}{\pgfqpoint{6.275590in}{5.159444in}}%
\pgfusepath{clip}%
\pgfsetbuttcap%
\pgfsetroundjoin%
\pgfsetlinewidth{1.003750pt}%
\definecolor{currentstroke}{rgb}{0.827451,0.827451,0.827451}%
\pgfsetstrokecolor{currentstroke}%
\pgfsetstrokeopacity{0.800000}%
\pgfsetdash{}{0pt}%
\pgfpathmoveto{\pgfqpoint{0.608606in}{3.646796in}}%
\pgfpathcurveto{\pgfqpoint{0.619657in}{3.646796in}}{\pgfqpoint{0.630256in}{3.651186in}}{\pgfqpoint{0.638069in}{3.659000in}}%
\pgfpathcurveto{\pgfqpoint{0.645883in}{3.666813in}}{\pgfqpoint{0.650273in}{3.677412in}}{\pgfqpoint{0.650273in}{3.688462in}}%
\pgfpathcurveto{\pgfqpoint{0.650273in}{3.699512in}}{\pgfqpoint{0.645883in}{3.710112in}}{\pgfqpoint{0.638069in}{3.717925in}}%
\pgfpathcurveto{\pgfqpoint{0.630256in}{3.725739in}}{\pgfqpoint{0.619657in}{3.730129in}}{\pgfqpoint{0.608606in}{3.730129in}}%
\pgfpathcurveto{\pgfqpoint{0.597556in}{3.730129in}}{\pgfqpoint{0.586957in}{3.725739in}}{\pgfqpoint{0.579144in}{3.717925in}}%
\pgfpathcurveto{\pgfqpoint{0.571330in}{3.710112in}}{\pgfqpoint{0.566940in}{3.699512in}}{\pgfqpoint{0.566940in}{3.688462in}}%
\pgfpathcurveto{\pgfqpoint{0.566940in}{3.677412in}}{\pgfqpoint{0.571330in}{3.666813in}}{\pgfqpoint{0.579144in}{3.659000in}}%
\pgfpathcurveto{\pgfqpoint{0.586957in}{3.651186in}}{\pgfqpoint{0.597556in}{3.646796in}}{\pgfqpoint{0.608606in}{3.646796in}}%
\pgfpathlineto{\pgfqpoint{0.608606in}{3.646796in}}%
\pgfpathclose%
\pgfusepath{stroke}%
\end{pgfscope}%
\begin{pgfscope}%
\pgfpathrectangle{\pgfqpoint{0.494722in}{0.437222in}}{\pgfqpoint{6.275590in}{5.159444in}}%
\pgfusepath{clip}%
\pgfsetbuttcap%
\pgfsetroundjoin%
\pgfsetlinewidth{1.003750pt}%
\definecolor{currentstroke}{rgb}{0.827451,0.827451,0.827451}%
\pgfsetstrokecolor{currentstroke}%
\pgfsetstrokeopacity{0.800000}%
\pgfsetdash{}{0pt}%
\pgfpathmoveto{\pgfqpoint{0.829675in}{2.964741in}}%
\pgfpathcurveto{\pgfqpoint{0.840725in}{2.964741in}}{\pgfqpoint{0.851324in}{2.969131in}}{\pgfqpoint{0.859137in}{2.976945in}}%
\pgfpathcurveto{\pgfqpoint{0.866951in}{2.984758in}}{\pgfqpoint{0.871341in}{2.995357in}}{\pgfqpoint{0.871341in}{3.006407in}}%
\pgfpathcurveto{\pgfqpoint{0.871341in}{3.017457in}}{\pgfqpoint{0.866951in}{3.028056in}}{\pgfqpoint{0.859137in}{3.035870in}}%
\pgfpathcurveto{\pgfqpoint{0.851324in}{3.043684in}}{\pgfqpoint{0.840725in}{3.048074in}}{\pgfqpoint{0.829675in}{3.048074in}}%
\pgfpathcurveto{\pgfqpoint{0.818625in}{3.048074in}}{\pgfqpoint{0.808026in}{3.043684in}}{\pgfqpoint{0.800212in}{3.035870in}}%
\pgfpathcurveto{\pgfqpoint{0.792398in}{3.028056in}}{\pgfqpoint{0.788008in}{3.017457in}}{\pgfqpoint{0.788008in}{3.006407in}}%
\pgfpathcurveto{\pgfqpoint{0.788008in}{2.995357in}}{\pgfqpoint{0.792398in}{2.984758in}}{\pgfqpoint{0.800212in}{2.976945in}}%
\pgfpathcurveto{\pgfqpoint{0.808026in}{2.969131in}}{\pgfqpoint{0.818625in}{2.964741in}}{\pgfqpoint{0.829675in}{2.964741in}}%
\pgfpathlineto{\pgfqpoint{0.829675in}{2.964741in}}%
\pgfpathclose%
\pgfusepath{stroke}%
\end{pgfscope}%
\begin{pgfscope}%
\pgfpathrectangle{\pgfqpoint{0.494722in}{0.437222in}}{\pgfqpoint{6.275590in}{5.159444in}}%
\pgfusepath{clip}%
\pgfsetbuttcap%
\pgfsetroundjoin%
\pgfsetlinewidth{1.003750pt}%
\definecolor{currentstroke}{rgb}{0.827451,0.827451,0.827451}%
\pgfsetstrokecolor{currentstroke}%
\pgfsetstrokeopacity{0.800000}%
\pgfsetdash{}{0pt}%
\pgfpathmoveto{\pgfqpoint{3.832585in}{0.609011in}}%
\pgfpathcurveto{\pgfqpoint{3.843635in}{0.609011in}}{\pgfqpoint{3.854234in}{0.613402in}}{\pgfqpoint{3.862047in}{0.621215in}}%
\pgfpathcurveto{\pgfqpoint{3.869861in}{0.629029in}}{\pgfqpoint{3.874251in}{0.639628in}}{\pgfqpoint{3.874251in}{0.650678in}}%
\pgfpathcurveto{\pgfqpoint{3.874251in}{0.661728in}}{\pgfqpoint{3.869861in}{0.672327in}}{\pgfqpoint{3.862047in}{0.680141in}}%
\pgfpathcurveto{\pgfqpoint{3.854234in}{0.687954in}}{\pgfqpoint{3.843635in}{0.692345in}}{\pgfqpoint{3.832585in}{0.692345in}}%
\pgfpathcurveto{\pgfqpoint{3.821534in}{0.692345in}}{\pgfqpoint{3.810935in}{0.687954in}}{\pgfqpoint{3.803122in}{0.680141in}}%
\pgfpathcurveto{\pgfqpoint{3.795308in}{0.672327in}}{\pgfqpoint{3.790918in}{0.661728in}}{\pgfqpoint{3.790918in}{0.650678in}}%
\pgfpathcurveto{\pgfqpoint{3.790918in}{0.639628in}}{\pgfqpoint{3.795308in}{0.629029in}}{\pgfqpoint{3.803122in}{0.621215in}}%
\pgfpathcurveto{\pgfqpoint{3.810935in}{0.613402in}}{\pgfqpoint{3.821534in}{0.609011in}}{\pgfqpoint{3.832585in}{0.609011in}}%
\pgfpathlineto{\pgfqpoint{3.832585in}{0.609011in}}%
\pgfpathclose%
\pgfusepath{stroke}%
\end{pgfscope}%
\begin{pgfscope}%
\pgfpathrectangle{\pgfqpoint{0.494722in}{0.437222in}}{\pgfqpoint{6.275590in}{5.159444in}}%
\pgfusepath{clip}%
\pgfsetbuttcap%
\pgfsetroundjoin%
\pgfsetlinewidth{1.003750pt}%
\definecolor{currentstroke}{rgb}{0.827451,0.827451,0.827451}%
\pgfsetstrokecolor{currentstroke}%
\pgfsetstrokeopacity{0.800000}%
\pgfsetdash{}{0pt}%
\pgfpathmoveto{\pgfqpoint{1.055628in}{2.578187in}}%
\pgfpathcurveto{\pgfqpoint{1.066678in}{2.578187in}}{\pgfqpoint{1.077277in}{2.582578in}}{\pgfqpoint{1.085091in}{2.590391in}}%
\pgfpathcurveto{\pgfqpoint{1.092904in}{2.598205in}}{\pgfqpoint{1.097295in}{2.608804in}}{\pgfqpoint{1.097295in}{2.619854in}}%
\pgfpathcurveto{\pgfqpoint{1.097295in}{2.630904in}}{\pgfqpoint{1.092904in}{2.641503in}}{\pgfqpoint{1.085091in}{2.649317in}}%
\pgfpathcurveto{\pgfqpoint{1.077277in}{2.657130in}}{\pgfqpoint{1.066678in}{2.661521in}}{\pgfqpoint{1.055628in}{2.661521in}}%
\pgfpathcurveto{\pgfqpoint{1.044578in}{2.661521in}}{\pgfqpoint{1.033979in}{2.657130in}}{\pgfqpoint{1.026165in}{2.649317in}}%
\pgfpathcurveto{\pgfqpoint{1.018352in}{2.641503in}}{\pgfqpoint{1.013961in}{2.630904in}}{\pgfqpoint{1.013961in}{2.619854in}}%
\pgfpathcurveto{\pgfqpoint{1.013961in}{2.608804in}}{\pgfqpoint{1.018352in}{2.598205in}}{\pgfqpoint{1.026165in}{2.590391in}}%
\pgfpathcurveto{\pgfqpoint{1.033979in}{2.582578in}}{\pgfqpoint{1.044578in}{2.578187in}}{\pgfqpoint{1.055628in}{2.578187in}}%
\pgfpathlineto{\pgfqpoint{1.055628in}{2.578187in}}%
\pgfpathclose%
\pgfusepath{stroke}%
\end{pgfscope}%
\begin{pgfscope}%
\pgfpathrectangle{\pgfqpoint{0.494722in}{0.437222in}}{\pgfqpoint{6.275590in}{5.159444in}}%
\pgfusepath{clip}%
\pgfsetbuttcap%
\pgfsetroundjoin%
\pgfsetlinewidth{1.003750pt}%
\definecolor{currentstroke}{rgb}{0.827451,0.827451,0.827451}%
\pgfsetstrokecolor{currentstroke}%
\pgfsetstrokeopacity{0.800000}%
\pgfsetdash{}{0pt}%
\pgfpathmoveto{\pgfqpoint{0.683182in}{3.327378in}}%
\pgfpathcurveto{\pgfqpoint{0.694232in}{3.327378in}}{\pgfqpoint{0.704831in}{3.331768in}}{\pgfqpoint{0.712645in}{3.339581in}}%
\pgfpathcurveto{\pgfqpoint{0.720459in}{3.347395in}}{\pgfqpoint{0.724849in}{3.357994in}}{\pgfqpoint{0.724849in}{3.369044in}}%
\pgfpathcurveto{\pgfqpoint{0.724849in}{3.380094in}}{\pgfqpoint{0.720459in}{3.390693in}}{\pgfqpoint{0.712645in}{3.398507in}}%
\pgfpathcurveto{\pgfqpoint{0.704831in}{3.406321in}}{\pgfqpoint{0.694232in}{3.410711in}}{\pgfqpoint{0.683182in}{3.410711in}}%
\pgfpathcurveto{\pgfqpoint{0.672132in}{3.410711in}}{\pgfqpoint{0.661533in}{3.406321in}}{\pgfqpoint{0.653720in}{3.398507in}}%
\pgfpathcurveto{\pgfqpoint{0.645906in}{3.390693in}}{\pgfqpoint{0.641516in}{3.380094in}}{\pgfqpoint{0.641516in}{3.369044in}}%
\pgfpathcurveto{\pgfqpoint{0.641516in}{3.357994in}}{\pgfqpoint{0.645906in}{3.347395in}}{\pgfqpoint{0.653720in}{3.339581in}}%
\pgfpathcurveto{\pgfqpoint{0.661533in}{3.331768in}}{\pgfqpoint{0.672132in}{3.327378in}}{\pgfqpoint{0.683182in}{3.327378in}}%
\pgfpathlineto{\pgfqpoint{0.683182in}{3.327378in}}%
\pgfpathclose%
\pgfusepath{stroke}%
\end{pgfscope}%
\begin{pgfscope}%
\pgfpathrectangle{\pgfqpoint{0.494722in}{0.437222in}}{\pgfqpoint{6.275590in}{5.159444in}}%
\pgfusepath{clip}%
\pgfsetbuttcap%
\pgfsetroundjoin%
\pgfsetlinewidth{1.003750pt}%
\definecolor{currentstroke}{rgb}{0.827451,0.827451,0.827451}%
\pgfsetstrokecolor{currentstroke}%
\pgfsetstrokeopacity{0.800000}%
\pgfsetdash{}{0pt}%
\pgfpathmoveto{\pgfqpoint{1.743973in}{1.699418in}}%
\pgfpathcurveto{\pgfqpoint{1.755023in}{1.699418in}}{\pgfqpoint{1.765622in}{1.703809in}}{\pgfqpoint{1.773436in}{1.711622in}}%
\pgfpathcurveto{\pgfqpoint{1.781250in}{1.719436in}}{\pgfqpoint{1.785640in}{1.730035in}}{\pgfqpoint{1.785640in}{1.741085in}}%
\pgfpathcurveto{\pgfqpoint{1.785640in}{1.752135in}}{\pgfqpoint{1.781250in}{1.762734in}}{\pgfqpoint{1.773436in}{1.770548in}}%
\pgfpathcurveto{\pgfqpoint{1.765622in}{1.778361in}}{\pgfqpoint{1.755023in}{1.782752in}}{\pgfqpoint{1.743973in}{1.782752in}}%
\pgfpathcurveto{\pgfqpoint{1.732923in}{1.782752in}}{\pgfqpoint{1.722324in}{1.778361in}}{\pgfqpoint{1.714511in}{1.770548in}}%
\pgfpathcurveto{\pgfqpoint{1.706697in}{1.762734in}}{\pgfqpoint{1.702307in}{1.752135in}}{\pgfqpoint{1.702307in}{1.741085in}}%
\pgfpathcurveto{\pgfqpoint{1.702307in}{1.730035in}}{\pgfqpoint{1.706697in}{1.719436in}}{\pgfqpoint{1.714511in}{1.711622in}}%
\pgfpathcurveto{\pgfqpoint{1.722324in}{1.703809in}}{\pgfqpoint{1.732923in}{1.699418in}}{\pgfqpoint{1.743973in}{1.699418in}}%
\pgfpathlineto{\pgfqpoint{1.743973in}{1.699418in}}%
\pgfpathclose%
\pgfusepath{stroke}%
\end{pgfscope}%
\begin{pgfscope}%
\pgfpathrectangle{\pgfqpoint{0.494722in}{0.437222in}}{\pgfqpoint{6.275590in}{5.159444in}}%
\pgfusepath{clip}%
\pgfsetbuttcap%
\pgfsetroundjoin%
\pgfsetlinewidth{1.003750pt}%
\definecolor{currentstroke}{rgb}{0.827451,0.827451,0.827451}%
\pgfsetstrokecolor{currentstroke}%
\pgfsetstrokeopacity{0.800000}%
\pgfsetdash{}{0pt}%
\pgfpathmoveto{\pgfqpoint{1.964907in}{1.517037in}}%
\pgfpathcurveto{\pgfqpoint{1.975957in}{1.517037in}}{\pgfqpoint{1.986556in}{1.521427in}}{\pgfqpoint{1.994370in}{1.529240in}}%
\pgfpathcurveto{\pgfqpoint{2.002183in}{1.537054in}}{\pgfqpoint{2.006573in}{1.547653in}}{\pgfqpoint{2.006573in}{1.558703in}}%
\pgfpathcurveto{\pgfqpoint{2.006573in}{1.569753in}}{\pgfqpoint{2.002183in}{1.580352in}}{\pgfqpoint{1.994370in}{1.588166in}}%
\pgfpathcurveto{\pgfqpoint{1.986556in}{1.595980in}}{\pgfqpoint{1.975957in}{1.600370in}}{\pgfqpoint{1.964907in}{1.600370in}}%
\pgfpathcurveto{\pgfqpoint{1.953857in}{1.600370in}}{\pgfqpoint{1.943258in}{1.595980in}}{\pgfqpoint{1.935444in}{1.588166in}}%
\pgfpathcurveto{\pgfqpoint{1.927630in}{1.580352in}}{\pgfqpoint{1.923240in}{1.569753in}}{\pgfqpoint{1.923240in}{1.558703in}}%
\pgfpathcurveto{\pgfqpoint{1.923240in}{1.547653in}}{\pgfqpoint{1.927630in}{1.537054in}}{\pgfqpoint{1.935444in}{1.529240in}}%
\pgfpathcurveto{\pgfqpoint{1.943258in}{1.521427in}}{\pgfqpoint{1.953857in}{1.517037in}}{\pgfqpoint{1.964907in}{1.517037in}}%
\pgfpathlineto{\pgfqpoint{1.964907in}{1.517037in}}%
\pgfpathclose%
\pgfusepath{stroke}%
\end{pgfscope}%
\begin{pgfscope}%
\pgfpathrectangle{\pgfqpoint{0.494722in}{0.437222in}}{\pgfqpoint{6.275590in}{5.159444in}}%
\pgfusepath{clip}%
\pgfsetbuttcap%
\pgfsetroundjoin%
\pgfsetlinewidth{1.003750pt}%
\definecolor{currentstroke}{rgb}{0.827451,0.827451,0.827451}%
\pgfsetstrokecolor{currentstroke}%
\pgfsetstrokeopacity{0.800000}%
\pgfsetdash{}{0pt}%
\pgfpathmoveto{\pgfqpoint{2.552339in}{1.147505in}}%
\pgfpathcurveto{\pgfqpoint{2.563390in}{1.147505in}}{\pgfqpoint{2.573989in}{1.151895in}}{\pgfqpoint{2.581802in}{1.159709in}}%
\pgfpathcurveto{\pgfqpoint{2.589616in}{1.167522in}}{\pgfqpoint{2.594006in}{1.178121in}}{\pgfqpoint{2.594006in}{1.189172in}}%
\pgfpathcurveto{\pgfqpoint{2.594006in}{1.200222in}}{\pgfqpoint{2.589616in}{1.210821in}}{\pgfqpoint{2.581802in}{1.218634in}}%
\pgfpathcurveto{\pgfqpoint{2.573989in}{1.226448in}}{\pgfqpoint{2.563390in}{1.230838in}}{\pgfqpoint{2.552339in}{1.230838in}}%
\pgfpathcurveto{\pgfqpoint{2.541289in}{1.230838in}}{\pgfqpoint{2.530690in}{1.226448in}}{\pgfqpoint{2.522877in}{1.218634in}}%
\pgfpathcurveto{\pgfqpoint{2.515063in}{1.210821in}}{\pgfqpoint{2.510673in}{1.200222in}}{\pgfqpoint{2.510673in}{1.189172in}}%
\pgfpathcurveto{\pgfqpoint{2.510673in}{1.178121in}}{\pgfqpoint{2.515063in}{1.167522in}}{\pgfqpoint{2.522877in}{1.159709in}}%
\pgfpathcurveto{\pgfqpoint{2.530690in}{1.151895in}}{\pgfqpoint{2.541289in}{1.147505in}}{\pgfqpoint{2.552339in}{1.147505in}}%
\pgfpathlineto{\pgfqpoint{2.552339in}{1.147505in}}%
\pgfpathclose%
\pgfusepath{stroke}%
\end{pgfscope}%
\begin{pgfscope}%
\pgfpathrectangle{\pgfqpoint{0.494722in}{0.437222in}}{\pgfqpoint{6.275590in}{5.159444in}}%
\pgfusepath{clip}%
\pgfsetbuttcap%
\pgfsetroundjoin%
\pgfsetlinewidth{1.003750pt}%
\definecolor{currentstroke}{rgb}{0.827451,0.827451,0.827451}%
\pgfsetstrokecolor{currentstroke}%
\pgfsetstrokeopacity{0.800000}%
\pgfsetdash{}{0pt}%
\pgfpathmoveto{\pgfqpoint{2.867283in}{0.963868in}}%
\pgfpathcurveto{\pgfqpoint{2.878333in}{0.963868in}}{\pgfqpoint{2.888932in}{0.968258in}}{\pgfqpoint{2.896746in}{0.976072in}}%
\pgfpathcurveto{\pgfqpoint{2.904560in}{0.983885in}}{\pgfqpoint{2.908950in}{0.994484in}}{\pgfqpoint{2.908950in}{1.005535in}}%
\pgfpathcurveto{\pgfqpoint{2.908950in}{1.016585in}}{\pgfqpoint{2.904560in}{1.027184in}}{\pgfqpoint{2.896746in}{1.034997in}}%
\pgfpathcurveto{\pgfqpoint{2.888932in}{1.042811in}}{\pgfqpoint{2.878333in}{1.047201in}}{\pgfqpoint{2.867283in}{1.047201in}}%
\pgfpathcurveto{\pgfqpoint{2.856233in}{1.047201in}}{\pgfqpoint{2.845634in}{1.042811in}}{\pgfqpoint{2.837820in}{1.034997in}}%
\pgfpathcurveto{\pgfqpoint{2.830007in}{1.027184in}}{\pgfqpoint{2.825617in}{1.016585in}}{\pgfqpoint{2.825617in}{1.005535in}}%
\pgfpathcurveto{\pgfqpoint{2.825617in}{0.994484in}}{\pgfqpoint{2.830007in}{0.983885in}}{\pgfqpoint{2.837820in}{0.976072in}}%
\pgfpathcurveto{\pgfqpoint{2.845634in}{0.968258in}}{\pgfqpoint{2.856233in}{0.963868in}}{\pgfqpoint{2.867283in}{0.963868in}}%
\pgfpathlineto{\pgfqpoint{2.867283in}{0.963868in}}%
\pgfpathclose%
\pgfusepath{stroke}%
\end{pgfscope}%
\begin{pgfscope}%
\pgfpathrectangle{\pgfqpoint{0.494722in}{0.437222in}}{\pgfqpoint{6.275590in}{5.159444in}}%
\pgfusepath{clip}%
\pgfsetbuttcap%
\pgfsetroundjoin%
\pgfsetlinewidth{1.003750pt}%
\definecolor{currentstroke}{rgb}{0.827451,0.827451,0.827451}%
\pgfsetstrokecolor{currentstroke}%
\pgfsetstrokeopacity{0.800000}%
\pgfsetdash{}{0pt}%
\pgfpathmoveto{\pgfqpoint{1.700731in}{1.766489in}}%
\pgfpathcurveto{\pgfqpoint{1.711781in}{1.766489in}}{\pgfqpoint{1.722380in}{1.770879in}}{\pgfqpoint{1.730194in}{1.778693in}}%
\pgfpathcurveto{\pgfqpoint{1.738007in}{1.786506in}}{\pgfqpoint{1.742398in}{1.797105in}}{\pgfqpoint{1.742398in}{1.808156in}}%
\pgfpathcurveto{\pgfqpoint{1.742398in}{1.819206in}}{\pgfqpoint{1.738007in}{1.829805in}}{\pgfqpoint{1.730194in}{1.837618in}}%
\pgfpathcurveto{\pgfqpoint{1.722380in}{1.845432in}}{\pgfqpoint{1.711781in}{1.849822in}}{\pgfqpoint{1.700731in}{1.849822in}}%
\pgfpathcurveto{\pgfqpoint{1.689681in}{1.849822in}}{\pgfqpoint{1.679082in}{1.845432in}}{\pgfqpoint{1.671268in}{1.837618in}}%
\pgfpathcurveto{\pgfqpoint{1.663455in}{1.829805in}}{\pgfqpoint{1.659064in}{1.819206in}}{\pgfqpoint{1.659064in}{1.808156in}}%
\pgfpathcurveto{\pgfqpoint{1.659064in}{1.797105in}}{\pgfqpoint{1.663455in}{1.786506in}}{\pgfqpoint{1.671268in}{1.778693in}}%
\pgfpathcurveto{\pgfqpoint{1.679082in}{1.770879in}}{\pgfqpoint{1.689681in}{1.766489in}}{\pgfqpoint{1.700731in}{1.766489in}}%
\pgfpathlineto{\pgfqpoint{1.700731in}{1.766489in}}%
\pgfpathclose%
\pgfusepath{stroke}%
\end{pgfscope}%
\begin{pgfscope}%
\pgfpathrectangle{\pgfqpoint{0.494722in}{0.437222in}}{\pgfqpoint{6.275590in}{5.159444in}}%
\pgfusepath{clip}%
\pgfsetbuttcap%
\pgfsetroundjoin%
\pgfsetlinewidth{1.003750pt}%
\definecolor{currentstroke}{rgb}{0.827451,0.827451,0.827451}%
\pgfsetstrokecolor{currentstroke}%
\pgfsetstrokeopacity{0.800000}%
\pgfsetdash{}{0pt}%
\pgfpathmoveto{\pgfqpoint{2.491855in}{1.177667in}}%
\pgfpathcurveto{\pgfqpoint{2.502905in}{1.177667in}}{\pgfqpoint{2.513504in}{1.182057in}}{\pgfqpoint{2.521318in}{1.189871in}}%
\pgfpathcurveto{\pgfqpoint{2.529131in}{1.197684in}}{\pgfqpoint{2.533521in}{1.208283in}}{\pgfqpoint{2.533521in}{1.219333in}}%
\pgfpathcurveto{\pgfqpoint{2.533521in}{1.230384in}}{\pgfqpoint{2.529131in}{1.240983in}}{\pgfqpoint{2.521318in}{1.248796in}}%
\pgfpathcurveto{\pgfqpoint{2.513504in}{1.256610in}}{\pgfqpoint{2.502905in}{1.261000in}}{\pgfqpoint{2.491855in}{1.261000in}}%
\pgfpathcurveto{\pgfqpoint{2.480805in}{1.261000in}}{\pgfqpoint{2.470206in}{1.256610in}}{\pgfqpoint{2.462392in}{1.248796in}}%
\pgfpathcurveto{\pgfqpoint{2.454578in}{1.240983in}}{\pgfqpoint{2.450188in}{1.230384in}}{\pgfqpoint{2.450188in}{1.219333in}}%
\pgfpathcurveto{\pgfqpoint{2.450188in}{1.208283in}}{\pgfqpoint{2.454578in}{1.197684in}}{\pgfqpoint{2.462392in}{1.189871in}}%
\pgfpathcurveto{\pgfqpoint{2.470206in}{1.182057in}}{\pgfqpoint{2.480805in}{1.177667in}}{\pgfqpoint{2.491855in}{1.177667in}}%
\pgfpathlineto{\pgfqpoint{2.491855in}{1.177667in}}%
\pgfpathclose%
\pgfusepath{stroke}%
\end{pgfscope}%
\begin{pgfscope}%
\pgfpathrectangle{\pgfqpoint{0.494722in}{0.437222in}}{\pgfqpoint{6.275590in}{5.159444in}}%
\pgfusepath{clip}%
\pgfsetbuttcap%
\pgfsetroundjoin%
\pgfsetlinewidth{1.003750pt}%
\definecolor{currentstroke}{rgb}{0.827451,0.827451,0.827451}%
\pgfsetstrokecolor{currentstroke}%
\pgfsetstrokeopacity{0.800000}%
\pgfsetdash{}{0pt}%
\pgfpathmoveto{\pgfqpoint{0.679488in}{3.431236in}}%
\pgfpathcurveto{\pgfqpoint{0.690538in}{3.431236in}}{\pgfqpoint{0.701137in}{3.435626in}}{\pgfqpoint{0.708951in}{3.443440in}}%
\pgfpathcurveto{\pgfqpoint{0.716765in}{3.451253in}}{\pgfqpoint{0.721155in}{3.461852in}}{\pgfqpoint{0.721155in}{3.472902in}}%
\pgfpathcurveto{\pgfqpoint{0.721155in}{3.483953in}}{\pgfqpoint{0.716765in}{3.494552in}}{\pgfqpoint{0.708951in}{3.502365in}}%
\pgfpathcurveto{\pgfqpoint{0.701137in}{3.510179in}}{\pgfqpoint{0.690538in}{3.514569in}}{\pgfqpoint{0.679488in}{3.514569in}}%
\pgfpathcurveto{\pgfqpoint{0.668438in}{3.514569in}}{\pgfqpoint{0.657839in}{3.510179in}}{\pgfqpoint{0.650026in}{3.502365in}}%
\pgfpathcurveto{\pgfqpoint{0.642212in}{3.494552in}}{\pgfqpoint{0.637822in}{3.483953in}}{\pgfqpoint{0.637822in}{3.472902in}}%
\pgfpathcurveto{\pgfqpoint{0.637822in}{3.461852in}}{\pgfqpoint{0.642212in}{3.451253in}}{\pgfqpoint{0.650026in}{3.443440in}}%
\pgfpathcurveto{\pgfqpoint{0.657839in}{3.435626in}}{\pgfqpoint{0.668438in}{3.431236in}}{\pgfqpoint{0.679488in}{3.431236in}}%
\pgfpathlineto{\pgfqpoint{0.679488in}{3.431236in}}%
\pgfpathclose%
\pgfusepath{stroke}%
\end{pgfscope}%
\begin{pgfscope}%
\pgfpathrectangle{\pgfqpoint{0.494722in}{0.437222in}}{\pgfqpoint{6.275590in}{5.159444in}}%
\pgfusepath{clip}%
\pgfsetbuttcap%
\pgfsetroundjoin%
\pgfsetlinewidth{1.003750pt}%
\definecolor{currentstroke}{rgb}{0.827451,0.827451,0.827451}%
\pgfsetstrokecolor{currentstroke}%
\pgfsetstrokeopacity{0.800000}%
\pgfsetdash{}{0pt}%
\pgfpathmoveto{\pgfqpoint{0.805070in}{3.017769in}}%
\pgfpathcurveto{\pgfqpoint{0.816120in}{3.017769in}}{\pgfqpoint{0.826719in}{3.022159in}}{\pgfqpoint{0.834533in}{3.029973in}}%
\pgfpathcurveto{\pgfqpoint{0.842347in}{3.037786in}}{\pgfqpoint{0.846737in}{3.048385in}}{\pgfqpoint{0.846737in}{3.059436in}}%
\pgfpathcurveto{\pgfqpoint{0.846737in}{3.070486in}}{\pgfqpoint{0.842347in}{3.081085in}}{\pgfqpoint{0.834533in}{3.088898in}}%
\pgfpathcurveto{\pgfqpoint{0.826719in}{3.096712in}}{\pgfqpoint{0.816120in}{3.101102in}}{\pgfqpoint{0.805070in}{3.101102in}}%
\pgfpathcurveto{\pgfqpoint{0.794020in}{3.101102in}}{\pgfqpoint{0.783421in}{3.096712in}}{\pgfqpoint{0.775608in}{3.088898in}}%
\pgfpathcurveto{\pgfqpoint{0.767794in}{3.081085in}}{\pgfqpoint{0.763404in}{3.070486in}}{\pgfqpoint{0.763404in}{3.059436in}}%
\pgfpathcurveto{\pgfqpoint{0.763404in}{3.048385in}}{\pgfqpoint{0.767794in}{3.037786in}}{\pgfqpoint{0.775608in}{3.029973in}}%
\pgfpathcurveto{\pgfqpoint{0.783421in}{3.022159in}}{\pgfqpoint{0.794020in}{3.017769in}}{\pgfqpoint{0.805070in}{3.017769in}}%
\pgfpathlineto{\pgfqpoint{0.805070in}{3.017769in}}%
\pgfpathclose%
\pgfusepath{stroke}%
\end{pgfscope}%
\begin{pgfscope}%
\pgfpathrectangle{\pgfqpoint{0.494722in}{0.437222in}}{\pgfqpoint{6.275590in}{5.159444in}}%
\pgfusepath{clip}%
\pgfsetbuttcap%
\pgfsetroundjoin%
\pgfsetlinewidth{1.003750pt}%
\definecolor{currentstroke}{rgb}{0.827451,0.827451,0.827451}%
\pgfsetstrokecolor{currentstroke}%
\pgfsetstrokeopacity{0.800000}%
\pgfsetdash{}{0pt}%
\pgfpathmoveto{\pgfqpoint{1.381781in}{2.056420in}}%
\pgfpathcurveto{\pgfqpoint{1.392831in}{2.056420in}}{\pgfqpoint{1.403430in}{2.060810in}}{\pgfqpoint{1.411244in}{2.068624in}}%
\pgfpathcurveto{\pgfqpoint{1.419057in}{2.076437in}}{\pgfqpoint{1.423448in}{2.087036in}}{\pgfqpoint{1.423448in}{2.098087in}}%
\pgfpathcurveto{\pgfqpoint{1.423448in}{2.109137in}}{\pgfqpoint{1.419057in}{2.119736in}}{\pgfqpoint{1.411244in}{2.127549in}}%
\pgfpathcurveto{\pgfqpoint{1.403430in}{2.135363in}}{\pgfqpoint{1.392831in}{2.139753in}}{\pgfqpoint{1.381781in}{2.139753in}}%
\pgfpathcurveto{\pgfqpoint{1.370731in}{2.139753in}}{\pgfqpoint{1.360132in}{2.135363in}}{\pgfqpoint{1.352318in}{2.127549in}}%
\pgfpathcurveto{\pgfqpoint{1.344505in}{2.119736in}}{\pgfqpoint{1.340114in}{2.109137in}}{\pgfqpoint{1.340114in}{2.098087in}}%
\pgfpathcurveto{\pgfqpoint{1.340114in}{2.087036in}}{\pgfqpoint{1.344505in}{2.076437in}}{\pgfqpoint{1.352318in}{2.068624in}}%
\pgfpathcurveto{\pgfqpoint{1.360132in}{2.060810in}}{\pgfqpoint{1.370731in}{2.056420in}}{\pgfqpoint{1.381781in}{2.056420in}}%
\pgfpathlineto{\pgfqpoint{1.381781in}{2.056420in}}%
\pgfpathclose%
\pgfusepath{stroke}%
\end{pgfscope}%
\begin{pgfscope}%
\pgfpathrectangle{\pgfqpoint{0.494722in}{0.437222in}}{\pgfqpoint{6.275590in}{5.159444in}}%
\pgfusepath{clip}%
\pgfsetbuttcap%
\pgfsetroundjoin%
\pgfsetlinewidth{1.003750pt}%
\definecolor{currentstroke}{rgb}{0.827451,0.827451,0.827451}%
\pgfsetstrokecolor{currentstroke}%
\pgfsetstrokeopacity{0.800000}%
\pgfsetdash{}{0pt}%
\pgfpathmoveto{\pgfqpoint{4.176161in}{0.527713in}}%
\pgfpathcurveto{\pgfqpoint{4.187212in}{0.527713in}}{\pgfqpoint{4.197811in}{0.532104in}}{\pgfqpoint{4.205624in}{0.539917in}}%
\pgfpathcurveto{\pgfqpoint{4.213438in}{0.547731in}}{\pgfqpoint{4.217828in}{0.558330in}}{\pgfqpoint{4.217828in}{0.569380in}}%
\pgfpathcurveto{\pgfqpoint{4.217828in}{0.580430in}}{\pgfqpoint{4.213438in}{0.591029in}}{\pgfqpoint{4.205624in}{0.598843in}}%
\pgfpathcurveto{\pgfqpoint{4.197811in}{0.606656in}}{\pgfqpoint{4.187212in}{0.611047in}}{\pgfqpoint{4.176161in}{0.611047in}}%
\pgfpathcurveto{\pgfqpoint{4.165111in}{0.611047in}}{\pgfqpoint{4.154512in}{0.606656in}}{\pgfqpoint{4.146699in}{0.598843in}}%
\pgfpathcurveto{\pgfqpoint{4.138885in}{0.591029in}}{\pgfqpoint{4.134495in}{0.580430in}}{\pgfqpoint{4.134495in}{0.569380in}}%
\pgfpathcurveto{\pgfqpoint{4.134495in}{0.558330in}}{\pgfqpoint{4.138885in}{0.547731in}}{\pgfqpoint{4.146699in}{0.539917in}}%
\pgfpathcurveto{\pgfqpoint{4.154512in}{0.532104in}}{\pgfqpoint{4.165111in}{0.527713in}}{\pgfqpoint{4.176161in}{0.527713in}}%
\pgfpathlineto{\pgfqpoint{4.176161in}{0.527713in}}%
\pgfpathclose%
\pgfusepath{stroke}%
\end{pgfscope}%
\begin{pgfscope}%
\pgfpathrectangle{\pgfqpoint{0.494722in}{0.437222in}}{\pgfqpoint{6.275590in}{5.159444in}}%
\pgfusepath{clip}%
\pgfsetbuttcap%
\pgfsetroundjoin%
\pgfsetlinewidth{1.003750pt}%
\definecolor{currentstroke}{rgb}{0.827451,0.827451,0.827451}%
\pgfsetstrokecolor{currentstroke}%
\pgfsetstrokeopacity{0.800000}%
\pgfsetdash{}{0pt}%
\pgfpathmoveto{\pgfqpoint{0.495079in}{4.572987in}}%
\pgfpathcurveto{\pgfqpoint{0.506130in}{4.572987in}}{\pgfqpoint{0.516729in}{4.577378in}}{\pgfqpoint{0.524542in}{4.585191in}}%
\pgfpathcurveto{\pgfqpoint{0.532356in}{4.593005in}}{\pgfqpoint{0.536746in}{4.603604in}}{\pgfqpoint{0.536746in}{4.614654in}}%
\pgfpathcurveto{\pgfqpoint{0.536746in}{4.625704in}}{\pgfqpoint{0.532356in}{4.636303in}}{\pgfqpoint{0.524542in}{4.644117in}}%
\pgfpathcurveto{\pgfqpoint{0.516729in}{4.651930in}}{\pgfqpoint{0.506130in}{4.656321in}}{\pgfqpoint{0.495079in}{4.656321in}}%
\pgfpathcurveto{\pgfqpoint{0.484029in}{4.656321in}}{\pgfqpoint{0.473430in}{4.651930in}}{\pgfqpoint{0.465617in}{4.644117in}}%
\pgfpathcurveto{\pgfqpoint{0.457803in}{4.636303in}}{\pgfqpoint{0.453413in}{4.625704in}}{\pgfqpoint{0.453413in}{4.614654in}}%
\pgfpathcurveto{\pgfqpoint{0.453413in}{4.603604in}}{\pgfqpoint{0.457803in}{4.593005in}}{\pgfqpoint{0.465617in}{4.585191in}}%
\pgfpathcurveto{\pgfqpoint{0.473430in}{4.577378in}}{\pgfqpoint{0.484029in}{4.572987in}}{\pgfqpoint{0.495079in}{4.572987in}}%
\pgfpathlineto{\pgfqpoint{0.495079in}{4.572987in}}%
\pgfpathclose%
\pgfusepath{stroke}%
\end{pgfscope}%
\begin{pgfscope}%
\pgfpathrectangle{\pgfqpoint{0.494722in}{0.437222in}}{\pgfqpoint{6.275590in}{5.159444in}}%
\pgfusepath{clip}%
\pgfsetbuttcap%
\pgfsetroundjoin%
\pgfsetlinewidth{1.003750pt}%
\definecolor{currentstroke}{rgb}{0.827451,0.827451,0.827451}%
\pgfsetstrokecolor{currentstroke}%
\pgfsetstrokeopacity{0.800000}%
\pgfsetdash{}{0pt}%
\pgfpathmoveto{\pgfqpoint{5.741672in}{0.399556in}}%
\pgfpathcurveto{\pgfqpoint{5.752722in}{0.399556in}}{\pgfqpoint{5.763321in}{0.403947in}}{\pgfqpoint{5.771134in}{0.411760in}}%
\pgfpathcurveto{\pgfqpoint{5.778948in}{0.419574in}}{\pgfqpoint{5.783338in}{0.430173in}}{\pgfqpoint{5.783338in}{0.441223in}}%
\pgfpathcurveto{\pgfqpoint{5.783338in}{0.452273in}}{\pgfqpoint{5.778948in}{0.462872in}}{\pgfqpoint{5.771134in}{0.470686in}}%
\pgfpathcurveto{\pgfqpoint{5.763321in}{0.478499in}}{\pgfqpoint{5.752722in}{0.482890in}}{\pgfqpoint{5.741672in}{0.482890in}}%
\pgfpathcurveto{\pgfqpoint{5.730622in}{0.482890in}}{\pgfqpoint{5.720023in}{0.478499in}}{\pgfqpoint{5.712209in}{0.470686in}}%
\pgfpathcurveto{\pgfqpoint{5.704395in}{0.462872in}}{\pgfqpoint{5.700005in}{0.452273in}}{\pgfqpoint{5.700005in}{0.441223in}}%
\pgfpathcurveto{\pgfqpoint{5.700005in}{0.430173in}}{\pgfqpoint{5.704395in}{0.419574in}}{\pgfqpoint{5.712209in}{0.411760in}}%
\pgfpathcurveto{\pgfqpoint{5.720023in}{0.403947in}}{\pgfqpoint{5.730622in}{0.399556in}}{\pgfqpoint{5.741672in}{0.399556in}}%
\pgfusepath{stroke}%
\end{pgfscope}%
\begin{pgfscope}%
\pgfpathrectangle{\pgfqpoint{0.494722in}{0.437222in}}{\pgfqpoint{6.275590in}{5.159444in}}%
\pgfusepath{clip}%
\pgfsetbuttcap%
\pgfsetroundjoin%
\pgfsetlinewidth{1.003750pt}%
\definecolor{currentstroke}{rgb}{0.827451,0.827451,0.827451}%
\pgfsetstrokecolor{currentstroke}%
\pgfsetstrokeopacity{0.800000}%
\pgfsetdash{}{0pt}%
\pgfpathmoveto{\pgfqpoint{5.560565in}{0.403701in}}%
\pgfpathcurveto{\pgfqpoint{5.571615in}{0.403701in}}{\pgfqpoint{5.582214in}{0.408092in}}{\pgfqpoint{5.590028in}{0.415905in}}%
\pgfpathcurveto{\pgfqpoint{5.597842in}{0.423719in}}{\pgfqpoint{5.602232in}{0.434318in}}{\pgfqpoint{5.602232in}{0.445368in}}%
\pgfpathcurveto{\pgfqpoint{5.602232in}{0.456418in}}{\pgfqpoint{5.597842in}{0.467017in}}{\pgfqpoint{5.590028in}{0.474831in}}%
\pgfpathcurveto{\pgfqpoint{5.582214in}{0.482644in}}{\pgfqpoint{5.571615in}{0.487035in}}{\pgfqpoint{5.560565in}{0.487035in}}%
\pgfpathcurveto{\pgfqpoint{5.549515in}{0.487035in}}{\pgfqpoint{5.538916in}{0.482644in}}{\pgfqpoint{5.531102in}{0.474831in}}%
\pgfpathcurveto{\pgfqpoint{5.523289in}{0.467017in}}{\pgfqpoint{5.518898in}{0.456418in}}{\pgfqpoint{5.518898in}{0.445368in}}%
\pgfpathcurveto{\pgfqpoint{5.518898in}{0.434318in}}{\pgfqpoint{5.523289in}{0.423719in}}{\pgfqpoint{5.531102in}{0.415905in}}%
\pgfpathcurveto{\pgfqpoint{5.538916in}{0.408092in}}{\pgfqpoint{5.549515in}{0.403701in}}{\pgfqpoint{5.560565in}{0.403701in}}%
\pgfusepath{stroke}%
\end{pgfscope}%
\begin{pgfscope}%
\pgfpathrectangle{\pgfqpoint{0.494722in}{0.437222in}}{\pgfqpoint{6.275590in}{5.159444in}}%
\pgfusepath{clip}%
\pgfsetbuttcap%
\pgfsetroundjoin%
\pgfsetlinewidth{1.003750pt}%
\definecolor{currentstroke}{rgb}{0.827451,0.827451,0.827451}%
\pgfsetstrokecolor{currentstroke}%
\pgfsetstrokeopacity{0.800000}%
\pgfsetdash{}{0pt}%
\pgfpathmoveto{\pgfqpoint{5.183443in}{0.422556in}}%
\pgfpathcurveto{\pgfqpoint{5.194493in}{0.422556in}}{\pgfqpoint{5.205092in}{0.426946in}}{\pgfqpoint{5.212906in}{0.434760in}}%
\pgfpathcurveto{\pgfqpoint{5.220719in}{0.442573in}}{\pgfqpoint{5.225109in}{0.453172in}}{\pgfqpoint{5.225109in}{0.464222in}}%
\pgfpathcurveto{\pgfqpoint{5.225109in}{0.475273in}}{\pgfqpoint{5.220719in}{0.485872in}}{\pgfqpoint{5.212906in}{0.493685in}}%
\pgfpathcurveto{\pgfqpoint{5.205092in}{0.501499in}}{\pgfqpoint{5.194493in}{0.505889in}}{\pgfqpoint{5.183443in}{0.505889in}}%
\pgfpathcurveto{\pgfqpoint{5.172393in}{0.505889in}}{\pgfqpoint{5.161794in}{0.501499in}}{\pgfqpoint{5.153980in}{0.493685in}}%
\pgfpathcurveto{\pgfqpoint{5.146166in}{0.485872in}}{\pgfqpoint{5.141776in}{0.475273in}}{\pgfqpoint{5.141776in}{0.464222in}}%
\pgfpathcurveto{\pgfqpoint{5.141776in}{0.453172in}}{\pgfqpoint{5.146166in}{0.442573in}}{\pgfqpoint{5.153980in}{0.434760in}}%
\pgfpathcurveto{\pgfqpoint{5.161794in}{0.426946in}}{\pgfqpoint{5.172393in}{0.422556in}}{\pgfqpoint{5.183443in}{0.422556in}}%
\pgfusepath{stroke}%
\end{pgfscope}%
\begin{pgfscope}%
\pgfpathrectangle{\pgfqpoint{0.494722in}{0.437222in}}{\pgfqpoint{6.275590in}{5.159444in}}%
\pgfusepath{clip}%
\pgfsetbuttcap%
\pgfsetroundjoin%
\pgfsetlinewidth{1.003750pt}%
\definecolor{currentstroke}{rgb}{0.827451,0.827451,0.827451}%
\pgfsetstrokecolor{currentstroke}%
\pgfsetstrokeopacity{0.800000}%
\pgfsetdash{}{0pt}%
\pgfpathmoveto{\pgfqpoint{3.527306in}{0.699588in}}%
\pgfpathcurveto{\pgfqpoint{3.538356in}{0.699588in}}{\pgfqpoint{3.548955in}{0.703979in}}{\pgfqpoint{3.556769in}{0.711792in}}%
\pgfpathcurveto{\pgfqpoint{3.564583in}{0.719606in}}{\pgfqpoint{3.568973in}{0.730205in}}{\pgfqpoint{3.568973in}{0.741255in}}%
\pgfpathcurveto{\pgfqpoint{3.568973in}{0.752305in}}{\pgfqpoint{3.564583in}{0.762904in}}{\pgfqpoint{3.556769in}{0.770718in}}%
\pgfpathcurveto{\pgfqpoint{3.548955in}{0.778532in}}{\pgfqpoint{3.538356in}{0.782922in}}{\pgfqpoint{3.527306in}{0.782922in}}%
\pgfpathcurveto{\pgfqpoint{3.516256in}{0.782922in}}{\pgfqpoint{3.505657in}{0.778532in}}{\pgfqpoint{3.497843in}{0.770718in}}%
\pgfpathcurveto{\pgfqpoint{3.490030in}{0.762904in}}{\pgfqpoint{3.485640in}{0.752305in}}{\pgfqpoint{3.485640in}{0.741255in}}%
\pgfpathcurveto{\pgfqpoint{3.485640in}{0.730205in}}{\pgfqpoint{3.490030in}{0.719606in}}{\pgfqpoint{3.497843in}{0.711792in}}%
\pgfpathcurveto{\pgfqpoint{3.505657in}{0.703979in}}{\pgfqpoint{3.516256in}{0.699588in}}{\pgfqpoint{3.527306in}{0.699588in}}%
\pgfpathlineto{\pgfqpoint{3.527306in}{0.699588in}}%
\pgfpathclose%
\pgfusepath{stroke}%
\end{pgfscope}%
\begin{pgfscope}%
\pgfpathrectangle{\pgfqpoint{0.494722in}{0.437222in}}{\pgfqpoint{6.275590in}{5.159444in}}%
\pgfusepath{clip}%
\pgfsetbuttcap%
\pgfsetroundjoin%
\pgfsetlinewidth{1.003750pt}%
\definecolor{currentstroke}{rgb}{0.827451,0.827451,0.827451}%
\pgfsetstrokecolor{currentstroke}%
\pgfsetstrokeopacity{0.800000}%
\pgfsetdash{}{0pt}%
\pgfpathmoveto{\pgfqpoint{4.706867in}{0.451120in}}%
\pgfpathcurveto{\pgfqpoint{4.717917in}{0.451120in}}{\pgfqpoint{4.728516in}{0.455510in}}{\pgfqpoint{4.736329in}{0.463324in}}%
\pgfpathcurveto{\pgfqpoint{4.744143in}{0.471138in}}{\pgfqpoint{4.748533in}{0.481737in}}{\pgfqpoint{4.748533in}{0.492787in}}%
\pgfpathcurveto{\pgfqpoint{4.748533in}{0.503837in}}{\pgfqpoint{4.744143in}{0.514436in}}{\pgfqpoint{4.736329in}{0.522250in}}%
\pgfpathcurveto{\pgfqpoint{4.728516in}{0.530063in}}{\pgfqpoint{4.717917in}{0.534453in}}{\pgfqpoint{4.706867in}{0.534453in}}%
\pgfpathcurveto{\pgfqpoint{4.695817in}{0.534453in}}{\pgfqpoint{4.685218in}{0.530063in}}{\pgfqpoint{4.677404in}{0.522250in}}%
\pgfpathcurveto{\pgfqpoint{4.669590in}{0.514436in}}{\pgfqpoint{4.665200in}{0.503837in}}{\pgfqpoint{4.665200in}{0.492787in}}%
\pgfpathcurveto{\pgfqpoint{4.665200in}{0.481737in}}{\pgfqpoint{4.669590in}{0.471138in}}{\pgfqpoint{4.677404in}{0.463324in}}%
\pgfpathcurveto{\pgfqpoint{4.685218in}{0.455510in}}{\pgfqpoint{4.695817in}{0.451120in}}{\pgfqpoint{4.706867in}{0.451120in}}%
\pgfpathlineto{\pgfqpoint{4.706867in}{0.451120in}}%
\pgfpathclose%
\pgfusepath{stroke}%
\end{pgfscope}%
\begin{pgfscope}%
\pgfpathrectangle{\pgfqpoint{0.494722in}{0.437222in}}{\pgfqpoint{6.275590in}{5.159444in}}%
\pgfusepath{clip}%
\pgfsetbuttcap%
\pgfsetroundjoin%
\pgfsetlinewidth{1.003750pt}%
\definecolor{currentstroke}{rgb}{0.827451,0.827451,0.827451}%
\pgfsetstrokecolor{currentstroke}%
\pgfsetstrokeopacity{0.800000}%
\pgfsetdash{}{0pt}%
\pgfpathmoveto{\pgfqpoint{4.970960in}{0.441718in}}%
\pgfpathcurveto{\pgfqpoint{4.982010in}{0.441718in}}{\pgfqpoint{4.992609in}{0.446108in}}{\pgfqpoint{5.000423in}{0.453922in}}%
\pgfpathcurveto{\pgfqpoint{5.008236in}{0.461735in}}{\pgfqpoint{5.012626in}{0.472334in}}{\pgfqpoint{5.012626in}{0.483385in}}%
\pgfpathcurveto{\pgfqpoint{5.012626in}{0.494435in}}{\pgfqpoint{5.008236in}{0.505034in}}{\pgfqpoint{5.000423in}{0.512847in}}%
\pgfpathcurveto{\pgfqpoint{4.992609in}{0.520661in}}{\pgfqpoint{4.982010in}{0.525051in}}{\pgfqpoint{4.970960in}{0.525051in}}%
\pgfpathcurveto{\pgfqpoint{4.959910in}{0.525051in}}{\pgfqpoint{4.949311in}{0.520661in}}{\pgfqpoint{4.941497in}{0.512847in}}%
\pgfpathcurveto{\pgfqpoint{4.933683in}{0.505034in}}{\pgfqpoint{4.929293in}{0.494435in}}{\pgfqpoint{4.929293in}{0.483385in}}%
\pgfpathcurveto{\pgfqpoint{4.929293in}{0.472334in}}{\pgfqpoint{4.933683in}{0.461735in}}{\pgfqpoint{4.941497in}{0.453922in}}%
\pgfpathcurveto{\pgfqpoint{4.949311in}{0.446108in}}{\pgfqpoint{4.959910in}{0.441718in}}{\pgfqpoint{4.970960in}{0.441718in}}%
\pgfpathlineto{\pgfqpoint{4.970960in}{0.441718in}}%
\pgfpathclose%
\pgfusepath{stroke}%
\end{pgfscope}%
\begin{pgfscope}%
\pgfpathrectangle{\pgfqpoint{0.494722in}{0.437222in}}{\pgfqpoint{6.275590in}{5.159444in}}%
\pgfusepath{clip}%
\pgfsetbuttcap%
\pgfsetroundjoin%
\pgfsetlinewidth{1.003750pt}%
\definecolor{currentstroke}{rgb}{0.827451,0.827451,0.827451}%
\pgfsetstrokecolor{currentstroke}%
\pgfsetstrokeopacity{0.800000}%
\pgfsetdash{}{0pt}%
\pgfpathmoveto{\pgfqpoint{3.697856in}{0.643246in}}%
\pgfpathcurveto{\pgfqpoint{3.708907in}{0.643246in}}{\pgfqpoint{3.719506in}{0.647637in}}{\pgfqpoint{3.727319in}{0.655450in}}%
\pgfpathcurveto{\pgfqpoint{3.735133in}{0.663264in}}{\pgfqpoint{3.739523in}{0.673863in}}{\pgfqpoint{3.739523in}{0.684913in}}%
\pgfpathcurveto{\pgfqpoint{3.739523in}{0.695963in}}{\pgfqpoint{3.735133in}{0.706562in}}{\pgfqpoint{3.727319in}{0.714376in}}%
\pgfpathcurveto{\pgfqpoint{3.719506in}{0.722189in}}{\pgfqpoint{3.708907in}{0.726580in}}{\pgfqpoint{3.697856in}{0.726580in}}%
\pgfpathcurveto{\pgfqpoint{3.686806in}{0.726580in}}{\pgfqpoint{3.676207in}{0.722189in}}{\pgfqpoint{3.668394in}{0.714376in}}%
\pgfpathcurveto{\pgfqpoint{3.660580in}{0.706562in}}{\pgfqpoint{3.656190in}{0.695963in}}{\pgfqpoint{3.656190in}{0.684913in}}%
\pgfpathcurveto{\pgfqpoint{3.656190in}{0.673863in}}{\pgfqpoint{3.660580in}{0.663264in}}{\pgfqpoint{3.668394in}{0.655450in}}%
\pgfpathcurveto{\pgfqpoint{3.676207in}{0.647637in}}{\pgfqpoint{3.686806in}{0.643246in}}{\pgfqpoint{3.697856in}{0.643246in}}%
\pgfpathlineto{\pgfqpoint{3.697856in}{0.643246in}}%
\pgfpathclose%
\pgfusepath{stroke}%
\end{pgfscope}%
\begin{pgfscope}%
\pgfpathrectangle{\pgfqpoint{0.494722in}{0.437222in}}{\pgfqpoint{6.275590in}{5.159444in}}%
\pgfusepath{clip}%
\pgfsetbuttcap%
\pgfsetroundjoin%
\pgfsetlinewidth{1.003750pt}%
\definecolor{currentstroke}{rgb}{0.827451,0.827451,0.827451}%
\pgfsetstrokecolor{currentstroke}%
\pgfsetstrokeopacity{0.800000}%
\pgfsetdash{}{0pt}%
\pgfpathmoveto{\pgfqpoint{0.679488in}{3.431236in}}%
\pgfpathcurveto{\pgfqpoint{0.690538in}{3.431236in}}{\pgfqpoint{0.701137in}{3.435626in}}{\pgfqpoint{0.708951in}{3.443440in}}%
\pgfpathcurveto{\pgfqpoint{0.716765in}{3.451253in}}{\pgfqpoint{0.721155in}{3.461852in}}{\pgfqpoint{0.721155in}{3.472902in}}%
\pgfpathcurveto{\pgfqpoint{0.721155in}{3.483953in}}{\pgfqpoint{0.716765in}{3.494552in}}{\pgfqpoint{0.708951in}{3.502365in}}%
\pgfpathcurveto{\pgfqpoint{0.701137in}{3.510179in}}{\pgfqpoint{0.690538in}{3.514569in}}{\pgfqpoint{0.679488in}{3.514569in}}%
\pgfpathcurveto{\pgfqpoint{0.668438in}{3.514569in}}{\pgfqpoint{0.657839in}{3.510179in}}{\pgfqpoint{0.650026in}{3.502365in}}%
\pgfpathcurveto{\pgfqpoint{0.642212in}{3.494552in}}{\pgfqpoint{0.637822in}{3.483953in}}{\pgfqpoint{0.637822in}{3.472902in}}%
\pgfpathcurveto{\pgfqpoint{0.637822in}{3.461852in}}{\pgfqpoint{0.642212in}{3.451253in}}{\pgfqpoint{0.650026in}{3.443440in}}%
\pgfpathcurveto{\pgfqpoint{0.657839in}{3.435626in}}{\pgfqpoint{0.668438in}{3.431236in}}{\pgfqpoint{0.679488in}{3.431236in}}%
\pgfpathlineto{\pgfqpoint{0.679488in}{3.431236in}}%
\pgfpathclose%
\pgfusepath{stroke}%
\end{pgfscope}%
\begin{pgfscope}%
\pgfpathrectangle{\pgfqpoint{0.494722in}{0.437222in}}{\pgfqpoint{6.275590in}{5.159444in}}%
\pgfusepath{clip}%
\pgfsetbuttcap%
\pgfsetroundjoin%
\pgfsetlinewidth{1.003750pt}%
\definecolor{currentstroke}{rgb}{0.827451,0.827451,0.827451}%
\pgfsetstrokecolor{currentstroke}%
\pgfsetstrokeopacity{0.800000}%
\pgfsetdash{}{0pt}%
\pgfpathmoveto{\pgfqpoint{2.365416in}{1.256449in}}%
\pgfpathcurveto{\pgfqpoint{2.376466in}{1.256449in}}{\pgfqpoint{2.387065in}{1.260839in}}{\pgfqpoint{2.394879in}{1.268653in}}%
\pgfpathcurveto{\pgfqpoint{2.402693in}{1.276466in}}{\pgfqpoint{2.407083in}{1.287065in}}{\pgfqpoint{2.407083in}{1.298115in}}%
\pgfpathcurveto{\pgfqpoint{2.407083in}{1.309166in}}{\pgfqpoint{2.402693in}{1.319765in}}{\pgfqpoint{2.394879in}{1.327578in}}%
\pgfpathcurveto{\pgfqpoint{2.387065in}{1.335392in}}{\pgfqpoint{2.376466in}{1.339782in}}{\pgfqpoint{2.365416in}{1.339782in}}%
\pgfpathcurveto{\pgfqpoint{2.354366in}{1.339782in}}{\pgfqpoint{2.343767in}{1.335392in}}{\pgfqpoint{2.335953in}{1.327578in}}%
\pgfpathcurveto{\pgfqpoint{2.328140in}{1.319765in}}{\pgfqpoint{2.323750in}{1.309166in}}{\pgfqpoint{2.323750in}{1.298115in}}%
\pgfpathcurveto{\pgfqpoint{2.323750in}{1.287065in}}{\pgfqpoint{2.328140in}{1.276466in}}{\pgfqpoint{2.335953in}{1.268653in}}%
\pgfpathcurveto{\pgfqpoint{2.343767in}{1.260839in}}{\pgfqpoint{2.354366in}{1.256449in}}{\pgfqpoint{2.365416in}{1.256449in}}%
\pgfpathlineto{\pgfqpoint{2.365416in}{1.256449in}}%
\pgfpathclose%
\pgfusepath{stroke}%
\end{pgfscope}%
\begin{pgfscope}%
\pgfpathrectangle{\pgfqpoint{0.494722in}{0.437222in}}{\pgfqpoint{6.275590in}{5.159444in}}%
\pgfusepath{clip}%
\pgfsetbuttcap%
\pgfsetroundjoin%
\pgfsetlinewidth{1.003750pt}%
\definecolor{currentstroke}{rgb}{0.827451,0.827451,0.827451}%
\pgfsetstrokecolor{currentstroke}%
\pgfsetstrokeopacity{0.800000}%
\pgfsetdash{}{0pt}%
\pgfpathmoveto{\pgfqpoint{4.495815in}{0.487123in}}%
\pgfpathcurveto{\pgfqpoint{4.506865in}{0.487123in}}{\pgfqpoint{4.517464in}{0.491513in}}{\pgfqpoint{4.525277in}{0.499327in}}%
\pgfpathcurveto{\pgfqpoint{4.533091in}{0.507140in}}{\pgfqpoint{4.537481in}{0.517740in}}{\pgfqpoint{4.537481in}{0.528790in}}%
\pgfpathcurveto{\pgfqpoint{4.537481in}{0.539840in}}{\pgfqpoint{4.533091in}{0.550439in}}{\pgfqpoint{4.525277in}{0.558252in}}%
\pgfpathcurveto{\pgfqpoint{4.517464in}{0.566066in}}{\pgfqpoint{4.506865in}{0.570456in}}{\pgfqpoint{4.495815in}{0.570456in}}%
\pgfpathcurveto{\pgfqpoint{4.484765in}{0.570456in}}{\pgfqpoint{4.474166in}{0.566066in}}{\pgfqpoint{4.466352in}{0.558252in}}%
\pgfpathcurveto{\pgfqpoint{4.458538in}{0.550439in}}{\pgfqpoint{4.454148in}{0.539840in}}{\pgfqpoint{4.454148in}{0.528790in}}%
\pgfpathcurveto{\pgfqpoint{4.454148in}{0.517740in}}{\pgfqpoint{4.458538in}{0.507140in}}{\pgfqpoint{4.466352in}{0.499327in}}%
\pgfpathcurveto{\pgfqpoint{4.474166in}{0.491513in}}{\pgfqpoint{4.484765in}{0.487123in}}{\pgfqpoint{4.495815in}{0.487123in}}%
\pgfpathlineto{\pgfqpoint{4.495815in}{0.487123in}}%
\pgfpathclose%
\pgfusepath{stroke}%
\end{pgfscope}%
\begin{pgfscope}%
\pgfpathrectangle{\pgfqpoint{0.494722in}{0.437222in}}{\pgfqpoint{6.275590in}{5.159444in}}%
\pgfusepath{clip}%
\pgfsetbuttcap%
\pgfsetroundjoin%
\pgfsetlinewidth{1.003750pt}%
\definecolor{currentstroke}{rgb}{0.827451,0.827451,0.827451}%
\pgfsetstrokecolor{currentstroke}%
\pgfsetstrokeopacity{0.800000}%
\pgfsetdash{}{0pt}%
\pgfpathmoveto{\pgfqpoint{0.625304in}{3.523383in}}%
\pgfpathcurveto{\pgfqpoint{0.636354in}{3.523383in}}{\pgfqpoint{0.646953in}{3.527774in}}{\pgfqpoint{0.654767in}{3.535587in}}%
\pgfpathcurveto{\pgfqpoint{0.662581in}{3.543401in}}{\pgfqpoint{0.666971in}{3.554000in}}{\pgfqpoint{0.666971in}{3.565050in}}%
\pgfpathcurveto{\pgfqpoint{0.666971in}{3.576100in}}{\pgfqpoint{0.662581in}{3.586699in}}{\pgfqpoint{0.654767in}{3.594513in}}%
\pgfpathcurveto{\pgfqpoint{0.646953in}{3.602326in}}{\pgfqpoint{0.636354in}{3.606717in}}{\pgfqpoint{0.625304in}{3.606717in}}%
\pgfpathcurveto{\pgfqpoint{0.614254in}{3.606717in}}{\pgfqpoint{0.603655in}{3.602326in}}{\pgfqpoint{0.595841in}{3.594513in}}%
\pgfpathcurveto{\pgfqpoint{0.588028in}{3.586699in}}{\pgfqpoint{0.583638in}{3.576100in}}{\pgfqpoint{0.583638in}{3.565050in}}%
\pgfpathcurveto{\pgfqpoint{0.583638in}{3.554000in}}{\pgfqpoint{0.588028in}{3.543401in}}{\pgfqpoint{0.595841in}{3.535587in}}%
\pgfpathcurveto{\pgfqpoint{0.603655in}{3.527774in}}{\pgfqpoint{0.614254in}{3.523383in}}{\pgfqpoint{0.625304in}{3.523383in}}%
\pgfpathlineto{\pgfqpoint{0.625304in}{3.523383in}}%
\pgfpathclose%
\pgfusepath{stroke}%
\end{pgfscope}%
\begin{pgfscope}%
\pgfpathrectangle{\pgfqpoint{0.494722in}{0.437222in}}{\pgfqpoint{6.275590in}{5.159444in}}%
\pgfusepath{clip}%
\pgfsetbuttcap%
\pgfsetroundjoin%
\pgfsetlinewidth{1.003750pt}%
\definecolor{currentstroke}{rgb}{0.827451,0.827451,0.827451}%
\pgfsetstrokecolor{currentstroke}%
\pgfsetstrokeopacity{0.800000}%
\pgfsetdash{}{0pt}%
\pgfpathmoveto{\pgfqpoint{1.852529in}{1.617217in}}%
\pgfpathcurveto{\pgfqpoint{1.863579in}{1.617217in}}{\pgfqpoint{1.874178in}{1.621608in}}{\pgfqpoint{1.881991in}{1.629421in}}%
\pgfpathcurveto{\pgfqpoint{1.889805in}{1.637235in}}{\pgfqpoint{1.894195in}{1.647834in}}{\pgfqpoint{1.894195in}{1.658884in}}%
\pgfpathcurveto{\pgfqpoint{1.894195in}{1.669934in}}{\pgfqpoint{1.889805in}{1.680533in}}{\pgfqpoint{1.881991in}{1.688347in}}%
\pgfpathcurveto{\pgfqpoint{1.874178in}{1.696160in}}{\pgfqpoint{1.863579in}{1.700551in}}{\pgfqpoint{1.852529in}{1.700551in}}%
\pgfpathcurveto{\pgfqpoint{1.841479in}{1.700551in}}{\pgfqpoint{1.830880in}{1.696160in}}{\pgfqpoint{1.823066in}{1.688347in}}%
\pgfpathcurveto{\pgfqpoint{1.815252in}{1.680533in}}{\pgfqpoint{1.810862in}{1.669934in}}{\pgfqpoint{1.810862in}{1.658884in}}%
\pgfpathcurveto{\pgfqpoint{1.810862in}{1.647834in}}{\pgfqpoint{1.815252in}{1.637235in}}{\pgfqpoint{1.823066in}{1.629421in}}%
\pgfpathcurveto{\pgfqpoint{1.830880in}{1.621608in}}{\pgfqpoint{1.841479in}{1.617217in}}{\pgfqpoint{1.852529in}{1.617217in}}%
\pgfpathlineto{\pgfqpoint{1.852529in}{1.617217in}}%
\pgfpathclose%
\pgfusepath{stroke}%
\end{pgfscope}%
\begin{pgfscope}%
\pgfpathrectangle{\pgfqpoint{0.494722in}{0.437222in}}{\pgfqpoint{6.275590in}{5.159444in}}%
\pgfusepath{clip}%
\pgfsetbuttcap%
\pgfsetroundjoin%
\pgfsetlinewidth{1.003750pt}%
\definecolor{currentstroke}{rgb}{0.827451,0.827451,0.827451}%
\pgfsetstrokecolor{currentstroke}%
\pgfsetstrokeopacity{0.800000}%
\pgfsetdash{}{0pt}%
\pgfpathmoveto{\pgfqpoint{0.536273in}{3.987890in}}%
\pgfpathcurveto{\pgfqpoint{0.547324in}{3.987890in}}{\pgfqpoint{0.557923in}{3.992280in}}{\pgfqpoint{0.565736in}{4.000093in}}%
\pgfpathcurveto{\pgfqpoint{0.573550in}{4.007907in}}{\pgfqpoint{0.577940in}{4.018506in}}{\pgfqpoint{0.577940in}{4.029556in}}%
\pgfpathcurveto{\pgfqpoint{0.577940in}{4.040606in}}{\pgfqpoint{0.573550in}{4.051205in}}{\pgfqpoint{0.565736in}{4.059019in}}%
\pgfpathcurveto{\pgfqpoint{0.557923in}{4.066833in}}{\pgfqpoint{0.547324in}{4.071223in}}{\pgfqpoint{0.536273in}{4.071223in}}%
\pgfpathcurveto{\pgfqpoint{0.525223in}{4.071223in}}{\pgfqpoint{0.514624in}{4.066833in}}{\pgfqpoint{0.506811in}{4.059019in}}%
\pgfpathcurveto{\pgfqpoint{0.498997in}{4.051205in}}{\pgfqpoint{0.494607in}{4.040606in}}{\pgfqpoint{0.494607in}{4.029556in}}%
\pgfpathcurveto{\pgfqpoint{0.494607in}{4.018506in}}{\pgfqpoint{0.498997in}{4.007907in}}{\pgfqpoint{0.506811in}{4.000093in}}%
\pgfpathcurveto{\pgfqpoint{0.514624in}{3.992280in}}{\pgfqpoint{0.525223in}{3.987890in}}{\pgfqpoint{0.536273in}{3.987890in}}%
\pgfpathlineto{\pgfqpoint{0.536273in}{3.987890in}}%
\pgfpathclose%
\pgfusepath{stroke}%
\end{pgfscope}%
\begin{pgfscope}%
\pgfpathrectangle{\pgfqpoint{0.494722in}{0.437222in}}{\pgfqpoint{6.275590in}{5.159444in}}%
\pgfusepath{clip}%
\pgfsetbuttcap%
\pgfsetroundjoin%
\pgfsetlinewidth{1.003750pt}%
\definecolor{currentstroke}{rgb}{0.827451,0.827451,0.827451}%
\pgfsetstrokecolor{currentstroke}%
\pgfsetstrokeopacity{0.800000}%
\pgfsetdash{}{0pt}%
\pgfpathmoveto{\pgfqpoint{1.926347in}{1.574489in}}%
\pgfpathcurveto{\pgfqpoint{1.937397in}{1.574489in}}{\pgfqpoint{1.947996in}{1.578879in}}{\pgfqpoint{1.955809in}{1.586693in}}%
\pgfpathcurveto{\pgfqpoint{1.963623in}{1.594506in}}{\pgfqpoint{1.968013in}{1.605105in}}{\pgfqpoint{1.968013in}{1.616155in}}%
\pgfpathcurveto{\pgfqpoint{1.968013in}{1.627206in}}{\pgfqpoint{1.963623in}{1.637805in}}{\pgfqpoint{1.955809in}{1.645618in}}%
\pgfpathcurveto{\pgfqpoint{1.947996in}{1.653432in}}{\pgfqpoint{1.937397in}{1.657822in}}{\pgfqpoint{1.926347in}{1.657822in}}%
\pgfpathcurveto{\pgfqpoint{1.915296in}{1.657822in}}{\pgfqpoint{1.904697in}{1.653432in}}{\pgfqpoint{1.896884in}{1.645618in}}%
\pgfpathcurveto{\pgfqpoint{1.889070in}{1.637805in}}{\pgfqpoint{1.884680in}{1.627206in}}{\pgfqpoint{1.884680in}{1.616155in}}%
\pgfpathcurveto{\pgfqpoint{1.884680in}{1.605105in}}{\pgfqpoint{1.889070in}{1.594506in}}{\pgfqpoint{1.896884in}{1.586693in}}%
\pgfpathcurveto{\pgfqpoint{1.904697in}{1.578879in}}{\pgfqpoint{1.915296in}{1.574489in}}{\pgfqpoint{1.926347in}{1.574489in}}%
\pgfpathlineto{\pgfqpoint{1.926347in}{1.574489in}}%
\pgfpathclose%
\pgfusepath{stroke}%
\end{pgfscope}%
\begin{pgfscope}%
\pgfpathrectangle{\pgfqpoint{0.494722in}{0.437222in}}{\pgfqpoint{6.275590in}{5.159444in}}%
\pgfusepath{clip}%
\pgfsetbuttcap%
\pgfsetroundjoin%
\pgfsetlinewidth{1.003750pt}%
\definecolor{currentstroke}{rgb}{0.827451,0.827451,0.827451}%
\pgfsetstrokecolor{currentstroke}%
\pgfsetstrokeopacity{0.800000}%
\pgfsetdash{}{0pt}%
\pgfpathmoveto{\pgfqpoint{0.903250in}{2.772216in}}%
\pgfpathcurveto{\pgfqpoint{0.914300in}{2.772216in}}{\pgfqpoint{0.924899in}{2.776607in}}{\pgfqpoint{0.932712in}{2.784420in}}%
\pgfpathcurveto{\pgfqpoint{0.940526in}{2.792234in}}{\pgfqpoint{0.944916in}{2.802833in}}{\pgfqpoint{0.944916in}{2.813883in}}%
\pgfpathcurveto{\pgfqpoint{0.944916in}{2.824933in}}{\pgfqpoint{0.940526in}{2.835532in}}{\pgfqpoint{0.932712in}{2.843346in}}%
\pgfpathcurveto{\pgfqpoint{0.924899in}{2.851159in}}{\pgfqpoint{0.914300in}{2.855550in}}{\pgfqpoint{0.903250in}{2.855550in}}%
\pgfpathcurveto{\pgfqpoint{0.892199in}{2.855550in}}{\pgfqpoint{0.881600in}{2.851159in}}{\pgfqpoint{0.873787in}{2.843346in}}%
\pgfpathcurveto{\pgfqpoint{0.865973in}{2.835532in}}{\pgfqpoint{0.861583in}{2.824933in}}{\pgfqpoint{0.861583in}{2.813883in}}%
\pgfpathcurveto{\pgfqpoint{0.861583in}{2.802833in}}{\pgfqpoint{0.865973in}{2.792234in}}{\pgfqpoint{0.873787in}{2.784420in}}%
\pgfpathcurveto{\pgfqpoint{0.881600in}{2.776607in}}{\pgfqpoint{0.892199in}{2.772216in}}{\pgfqpoint{0.903250in}{2.772216in}}%
\pgfpathlineto{\pgfqpoint{0.903250in}{2.772216in}}%
\pgfpathclose%
\pgfusepath{stroke}%
\end{pgfscope}%
\begin{pgfscope}%
\pgfpathrectangle{\pgfqpoint{0.494722in}{0.437222in}}{\pgfqpoint{6.275590in}{5.159444in}}%
\pgfusepath{clip}%
\pgfsetbuttcap%
\pgfsetroundjoin%
\pgfsetlinewidth{1.003750pt}%
\definecolor{currentstroke}{rgb}{0.827451,0.827451,0.827451}%
\pgfsetstrokecolor{currentstroke}%
\pgfsetstrokeopacity{0.800000}%
\pgfsetdash{}{0pt}%
\pgfpathmoveto{\pgfqpoint{0.525453in}{4.078418in}}%
\pgfpathcurveto{\pgfqpoint{0.536503in}{4.078418in}}{\pgfqpoint{0.547102in}{4.082808in}}{\pgfqpoint{0.554915in}{4.090621in}}%
\pgfpathcurveto{\pgfqpoint{0.562729in}{4.098435in}}{\pgfqpoint{0.567119in}{4.109034in}}{\pgfqpoint{0.567119in}{4.120084in}}%
\pgfpathcurveto{\pgfqpoint{0.567119in}{4.131134in}}{\pgfqpoint{0.562729in}{4.141733in}}{\pgfqpoint{0.554915in}{4.149547in}}%
\pgfpathcurveto{\pgfqpoint{0.547102in}{4.157361in}}{\pgfqpoint{0.536503in}{4.161751in}}{\pgfqpoint{0.525453in}{4.161751in}}%
\pgfpathcurveto{\pgfqpoint{0.514403in}{4.161751in}}{\pgfqpoint{0.503804in}{4.157361in}}{\pgfqpoint{0.495990in}{4.149547in}}%
\pgfpathcurveto{\pgfqpoint{0.488176in}{4.141733in}}{\pgfqpoint{0.483786in}{4.131134in}}{\pgfqpoint{0.483786in}{4.120084in}}%
\pgfpathcurveto{\pgfqpoint{0.483786in}{4.109034in}}{\pgfqpoint{0.488176in}{4.098435in}}{\pgfqpoint{0.495990in}{4.090621in}}%
\pgfpathcurveto{\pgfqpoint{0.503804in}{4.082808in}}{\pgfqpoint{0.514403in}{4.078418in}}{\pgfqpoint{0.525453in}{4.078418in}}%
\pgfpathlineto{\pgfqpoint{0.525453in}{4.078418in}}%
\pgfpathclose%
\pgfusepath{stroke}%
\end{pgfscope}%
\begin{pgfscope}%
\pgfpathrectangle{\pgfqpoint{0.494722in}{0.437222in}}{\pgfqpoint{6.275590in}{5.159444in}}%
\pgfusepath{clip}%
\pgfsetbuttcap%
\pgfsetroundjoin%
\pgfsetlinewidth{1.003750pt}%
\definecolor{currentstroke}{rgb}{0.827451,0.827451,0.827451}%
\pgfsetstrokecolor{currentstroke}%
\pgfsetstrokeopacity{0.800000}%
\pgfsetdash{}{0pt}%
\pgfpathmoveto{\pgfqpoint{0.944388in}{2.750541in}}%
\pgfpathcurveto{\pgfqpoint{0.955438in}{2.750541in}}{\pgfqpoint{0.966037in}{2.754931in}}{\pgfqpoint{0.973850in}{2.762745in}}%
\pgfpathcurveto{\pgfqpoint{0.981664in}{2.770558in}}{\pgfqpoint{0.986054in}{2.781157in}}{\pgfqpoint{0.986054in}{2.792208in}}%
\pgfpathcurveto{\pgfqpoint{0.986054in}{2.803258in}}{\pgfqpoint{0.981664in}{2.813857in}}{\pgfqpoint{0.973850in}{2.821670in}}%
\pgfpathcurveto{\pgfqpoint{0.966037in}{2.829484in}}{\pgfqpoint{0.955438in}{2.833874in}}{\pgfqpoint{0.944388in}{2.833874in}}%
\pgfpathcurveto{\pgfqpoint{0.933337in}{2.833874in}}{\pgfqpoint{0.922738in}{2.829484in}}{\pgfqpoint{0.914925in}{2.821670in}}%
\pgfpathcurveto{\pgfqpoint{0.907111in}{2.813857in}}{\pgfqpoint{0.902721in}{2.803258in}}{\pgfqpoint{0.902721in}{2.792208in}}%
\pgfpathcurveto{\pgfqpoint{0.902721in}{2.781157in}}{\pgfqpoint{0.907111in}{2.770558in}}{\pgfqpoint{0.914925in}{2.762745in}}%
\pgfpathcurveto{\pgfqpoint{0.922738in}{2.754931in}}{\pgfqpoint{0.933337in}{2.750541in}}{\pgfqpoint{0.944388in}{2.750541in}}%
\pgfpathlineto{\pgfqpoint{0.944388in}{2.750541in}}%
\pgfpathclose%
\pgfusepath{stroke}%
\end{pgfscope}%
\begin{pgfscope}%
\pgfpathrectangle{\pgfqpoint{0.494722in}{0.437222in}}{\pgfqpoint{6.275590in}{5.159444in}}%
\pgfusepath{clip}%
\pgfsetbuttcap%
\pgfsetroundjoin%
\pgfsetlinewidth{1.003750pt}%
\definecolor{currentstroke}{rgb}{0.827451,0.827451,0.827451}%
\pgfsetstrokecolor{currentstroke}%
\pgfsetstrokeopacity{0.800000}%
\pgfsetdash{}{0pt}%
\pgfpathmoveto{\pgfqpoint{0.870505in}{2.877763in}}%
\pgfpathcurveto{\pgfqpoint{0.881555in}{2.877763in}}{\pgfqpoint{0.892154in}{2.882154in}}{\pgfqpoint{0.899967in}{2.889967in}}%
\pgfpathcurveto{\pgfqpoint{0.907781in}{2.897781in}}{\pgfqpoint{0.912171in}{2.908380in}}{\pgfqpoint{0.912171in}{2.919430in}}%
\pgfpathcurveto{\pgfqpoint{0.912171in}{2.930480in}}{\pgfqpoint{0.907781in}{2.941079in}}{\pgfqpoint{0.899967in}{2.948893in}}%
\pgfpathcurveto{\pgfqpoint{0.892154in}{2.956706in}}{\pgfqpoint{0.881555in}{2.961097in}}{\pgfqpoint{0.870505in}{2.961097in}}%
\pgfpathcurveto{\pgfqpoint{0.859455in}{2.961097in}}{\pgfqpoint{0.848856in}{2.956706in}}{\pgfqpoint{0.841042in}{2.948893in}}%
\pgfpathcurveto{\pgfqpoint{0.833228in}{2.941079in}}{\pgfqpoint{0.828838in}{2.930480in}}{\pgfqpoint{0.828838in}{2.919430in}}%
\pgfpathcurveto{\pgfqpoint{0.828838in}{2.908380in}}{\pgfqpoint{0.833228in}{2.897781in}}{\pgfqpoint{0.841042in}{2.889967in}}%
\pgfpathcurveto{\pgfqpoint{0.848856in}{2.882154in}}{\pgfqpoint{0.859455in}{2.877763in}}{\pgfqpoint{0.870505in}{2.877763in}}%
\pgfpathlineto{\pgfqpoint{0.870505in}{2.877763in}}%
\pgfpathclose%
\pgfusepath{stroke}%
\end{pgfscope}%
\begin{pgfscope}%
\pgfpathrectangle{\pgfqpoint{0.494722in}{0.437222in}}{\pgfqpoint{6.275590in}{5.159444in}}%
\pgfusepath{clip}%
\pgfsetbuttcap%
\pgfsetroundjoin%
\pgfsetlinewidth{1.003750pt}%
\definecolor{currentstroke}{rgb}{0.827451,0.827451,0.827451}%
\pgfsetstrokecolor{currentstroke}%
\pgfsetstrokeopacity{0.800000}%
\pgfsetdash{}{0pt}%
\pgfpathmoveto{\pgfqpoint{4.586350in}{0.475898in}}%
\pgfpathcurveto{\pgfqpoint{4.597400in}{0.475898in}}{\pgfqpoint{4.607999in}{0.480288in}}{\pgfqpoint{4.615813in}{0.488101in}}%
\pgfpathcurveto{\pgfqpoint{4.623627in}{0.495915in}}{\pgfqpoint{4.628017in}{0.506514in}}{\pgfqpoint{4.628017in}{0.517564in}}%
\pgfpathcurveto{\pgfqpoint{4.628017in}{0.528614in}}{\pgfqpoint{4.623627in}{0.539213in}}{\pgfqpoint{4.615813in}{0.547027in}}%
\pgfpathcurveto{\pgfqpoint{4.607999in}{0.554841in}}{\pgfqpoint{4.597400in}{0.559231in}}{\pgfqpoint{4.586350in}{0.559231in}}%
\pgfpathcurveto{\pgfqpoint{4.575300in}{0.559231in}}{\pgfqpoint{4.564701in}{0.554841in}}{\pgfqpoint{4.556887in}{0.547027in}}%
\pgfpathcurveto{\pgfqpoint{4.549074in}{0.539213in}}{\pgfqpoint{4.544683in}{0.528614in}}{\pgfqpoint{4.544683in}{0.517564in}}%
\pgfpathcurveto{\pgfqpoint{4.544683in}{0.506514in}}{\pgfqpoint{4.549074in}{0.495915in}}{\pgfqpoint{4.556887in}{0.488101in}}%
\pgfpathcurveto{\pgfqpoint{4.564701in}{0.480288in}}{\pgfqpoint{4.575300in}{0.475898in}}{\pgfqpoint{4.586350in}{0.475898in}}%
\pgfpathlineto{\pgfqpoint{4.586350in}{0.475898in}}%
\pgfpathclose%
\pgfusepath{stroke}%
\end{pgfscope}%
\begin{pgfscope}%
\pgfpathrectangle{\pgfqpoint{0.494722in}{0.437222in}}{\pgfqpoint{6.275590in}{5.159444in}}%
\pgfusepath{clip}%
\pgfsetbuttcap%
\pgfsetroundjoin%
\pgfsetlinewidth{1.003750pt}%
\definecolor{currentstroke}{rgb}{0.827451,0.827451,0.827451}%
\pgfsetstrokecolor{currentstroke}%
\pgfsetstrokeopacity{0.800000}%
\pgfsetdash{}{0pt}%
\pgfpathmoveto{\pgfqpoint{2.016640in}{1.484877in}}%
\pgfpathcurveto{\pgfqpoint{2.027690in}{1.484877in}}{\pgfqpoint{2.038289in}{1.489267in}}{\pgfqpoint{2.046102in}{1.497081in}}%
\pgfpathcurveto{\pgfqpoint{2.053916in}{1.504894in}}{\pgfqpoint{2.058306in}{1.515493in}}{\pgfqpoint{2.058306in}{1.526544in}}%
\pgfpathcurveto{\pgfqpoint{2.058306in}{1.537594in}}{\pgfqpoint{2.053916in}{1.548193in}}{\pgfqpoint{2.046102in}{1.556006in}}%
\pgfpathcurveto{\pgfqpoint{2.038289in}{1.563820in}}{\pgfqpoint{2.027690in}{1.568210in}}{\pgfqpoint{2.016640in}{1.568210in}}%
\pgfpathcurveto{\pgfqpoint{2.005589in}{1.568210in}}{\pgfqpoint{1.994990in}{1.563820in}}{\pgfqpoint{1.987177in}{1.556006in}}%
\pgfpathcurveto{\pgfqpoint{1.979363in}{1.548193in}}{\pgfqpoint{1.974973in}{1.537594in}}{\pgfqpoint{1.974973in}{1.526544in}}%
\pgfpathcurveto{\pgfqpoint{1.974973in}{1.515493in}}{\pgfqpoint{1.979363in}{1.504894in}}{\pgfqpoint{1.987177in}{1.497081in}}%
\pgfpathcurveto{\pgfqpoint{1.994990in}{1.489267in}}{\pgfqpoint{2.005589in}{1.484877in}}{\pgfqpoint{2.016640in}{1.484877in}}%
\pgfpathlineto{\pgfqpoint{2.016640in}{1.484877in}}%
\pgfpathclose%
\pgfusepath{stroke}%
\end{pgfscope}%
\begin{pgfscope}%
\pgfpathrectangle{\pgfqpoint{0.494722in}{0.437222in}}{\pgfqpoint{6.275590in}{5.159444in}}%
\pgfusepath{clip}%
\pgfsetbuttcap%
\pgfsetroundjoin%
\pgfsetlinewidth{1.003750pt}%
\definecolor{currentstroke}{rgb}{0.827451,0.827451,0.827451}%
\pgfsetstrokecolor{currentstroke}%
\pgfsetstrokeopacity{0.800000}%
\pgfsetdash{}{0pt}%
\pgfpathmoveto{\pgfqpoint{2.452460in}{1.181670in}}%
\pgfpathcurveto{\pgfqpoint{2.463510in}{1.181670in}}{\pgfqpoint{2.474109in}{1.186061in}}{\pgfqpoint{2.481923in}{1.193874in}}%
\pgfpathcurveto{\pgfqpoint{2.489737in}{1.201688in}}{\pgfqpoint{2.494127in}{1.212287in}}{\pgfqpoint{2.494127in}{1.223337in}}%
\pgfpathcurveto{\pgfqpoint{2.494127in}{1.234387in}}{\pgfqpoint{2.489737in}{1.244986in}}{\pgfqpoint{2.481923in}{1.252800in}}%
\pgfpathcurveto{\pgfqpoint{2.474109in}{1.260613in}}{\pgfqpoint{2.463510in}{1.265004in}}{\pgfqpoint{2.452460in}{1.265004in}}%
\pgfpathcurveto{\pgfqpoint{2.441410in}{1.265004in}}{\pgfqpoint{2.430811in}{1.260613in}}{\pgfqpoint{2.422998in}{1.252800in}}%
\pgfpathcurveto{\pgfqpoint{2.415184in}{1.244986in}}{\pgfqpoint{2.410794in}{1.234387in}}{\pgfqpoint{2.410794in}{1.223337in}}%
\pgfpathcurveto{\pgfqpoint{2.410794in}{1.212287in}}{\pgfqpoint{2.415184in}{1.201688in}}{\pgfqpoint{2.422998in}{1.193874in}}%
\pgfpathcurveto{\pgfqpoint{2.430811in}{1.186061in}}{\pgfqpoint{2.441410in}{1.181670in}}{\pgfqpoint{2.452460in}{1.181670in}}%
\pgfpathlineto{\pgfqpoint{2.452460in}{1.181670in}}%
\pgfpathclose%
\pgfusepath{stroke}%
\end{pgfscope}%
\begin{pgfscope}%
\pgfpathrectangle{\pgfqpoint{0.494722in}{0.437222in}}{\pgfqpoint{6.275590in}{5.159444in}}%
\pgfusepath{clip}%
\pgfsetbuttcap%
\pgfsetroundjoin%
\pgfsetlinewidth{1.003750pt}%
\definecolor{currentstroke}{rgb}{0.827451,0.827451,0.827451}%
\pgfsetstrokecolor{currentstroke}%
\pgfsetstrokeopacity{0.800000}%
\pgfsetdash{}{0pt}%
\pgfpathmoveto{\pgfqpoint{5.060352in}{0.429754in}}%
\pgfpathcurveto{\pgfqpoint{5.071402in}{0.429754in}}{\pgfqpoint{5.082001in}{0.434145in}}{\pgfqpoint{5.089815in}{0.441958in}}%
\pgfpathcurveto{\pgfqpoint{5.097629in}{0.449772in}}{\pgfqpoint{5.102019in}{0.460371in}}{\pgfqpoint{5.102019in}{0.471421in}}%
\pgfpathcurveto{\pgfqpoint{5.102019in}{0.482471in}}{\pgfqpoint{5.097629in}{0.493070in}}{\pgfqpoint{5.089815in}{0.500884in}}%
\pgfpathcurveto{\pgfqpoint{5.082001in}{0.508697in}}{\pgfqpoint{5.071402in}{0.513088in}}{\pgfqpoint{5.060352in}{0.513088in}}%
\pgfpathcurveto{\pgfqpoint{5.049302in}{0.513088in}}{\pgfqpoint{5.038703in}{0.508697in}}{\pgfqpoint{5.030890in}{0.500884in}}%
\pgfpathcurveto{\pgfqpoint{5.023076in}{0.493070in}}{\pgfqpoint{5.018686in}{0.482471in}}{\pgfqpoint{5.018686in}{0.471421in}}%
\pgfpathcurveto{\pgfqpoint{5.018686in}{0.460371in}}{\pgfqpoint{5.023076in}{0.449772in}}{\pgfqpoint{5.030890in}{0.441958in}}%
\pgfpathcurveto{\pgfqpoint{5.038703in}{0.434145in}}{\pgfqpoint{5.049302in}{0.429754in}}{\pgfqpoint{5.060352in}{0.429754in}}%
\pgfpathlineto{\pgfqpoint{5.060352in}{0.429754in}}%
\pgfpathclose%
\pgfusepath{stroke}%
\end{pgfscope}%
\begin{pgfscope}%
\pgfpathrectangle{\pgfqpoint{0.494722in}{0.437222in}}{\pgfqpoint{6.275590in}{5.159444in}}%
\pgfusepath{clip}%
\pgfsetbuttcap%
\pgfsetroundjoin%
\pgfsetlinewidth{1.003750pt}%
\definecolor{currentstroke}{rgb}{0.827451,0.827451,0.827451}%
\pgfsetstrokecolor{currentstroke}%
\pgfsetstrokeopacity{0.800000}%
\pgfsetdash{}{0pt}%
\pgfpathmoveto{\pgfqpoint{0.498024in}{4.460098in}}%
\pgfpathcurveto{\pgfqpoint{0.509074in}{4.460098in}}{\pgfqpoint{0.519673in}{4.464488in}}{\pgfqpoint{0.527487in}{4.472302in}}%
\pgfpathcurveto{\pgfqpoint{0.535301in}{4.480116in}}{\pgfqpoint{0.539691in}{4.490715in}}{\pgfqpoint{0.539691in}{4.501765in}}%
\pgfpathcurveto{\pgfqpoint{0.539691in}{4.512815in}}{\pgfqpoint{0.535301in}{4.523414in}}{\pgfqpoint{0.527487in}{4.531228in}}%
\pgfpathcurveto{\pgfqpoint{0.519673in}{4.539041in}}{\pgfqpoint{0.509074in}{4.543432in}}{\pgfqpoint{0.498024in}{4.543432in}}%
\pgfpathcurveto{\pgfqpoint{0.486974in}{4.543432in}}{\pgfqpoint{0.476375in}{4.539041in}}{\pgfqpoint{0.468561in}{4.531228in}}%
\pgfpathcurveto{\pgfqpoint{0.460748in}{4.523414in}}{\pgfqpoint{0.456358in}{4.512815in}}{\pgfqpoint{0.456358in}{4.501765in}}%
\pgfpathcurveto{\pgfqpoint{0.456358in}{4.490715in}}{\pgfqpoint{0.460748in}{4.480116in}}{\pgfqpoint{0.468561in}{4.472302in}}%
\pgfpathcurveto{\pgfqpoint{0.476375in}{4.464488in}}{\pgfqpoint{0.486974in}{4.460098in}}{\pgfqpoint{0.498024in}{4.460098in}}%
\pgfpathlineto{\pgfqpoint{0.498024in}{4.460098in}}%
\pgfpathclose%
\pgfusepath{stroke}%
\end{pgfscope}%
\begin{pgfscope}%
\pgfpathrectangle{\pgfqpoint{0.494722in}{0.437222in}}{\pgfqpoint{6.275590in}{5.159444in}}%
\pgfusepath{clip}%
\pgfsetbuttcap%
\pgfsetroundjoin%
\pgfsetlinewidth{1.003750pt}%
\definecolor{currentstroke}{rgb}{0.827451,0.827451,0.827451}%
\pgfsetstrokecolor{currentstroke}%
\pgfsetstrokeopacity{0.800000}%
\pgfsetdash{}{0pt}%
\pgfpathmoveto{\pgfqpoint{0.970880in}{2.643850in}}%
\pgfpathcurveto{\pgfqpoint{0.981930in}{2.643850in}}{\pgfqpoint{0.992529in}{2.648240in}}{\pgfqpoint{1.000343in}{2.656053in}}%
\pgfpathcurveto{\pgfqpoint{1.008156in}{2.663867in}}{\pgfqpoint{1.012547in}{2.674466in}}{\pgfqpoint{1.012547in}{2.685516in}}%
\pgfpathcurveto{\pgfqpoint{1.012547in}{2.696566in}}{\pgfqpoint{1.008156in}{2.707165in}}{\pgfqpoint{1.000343in}{2.714979in}}%
\pgfpathcurveto{\pgfqpoint{0.992529in}{2.722793in}}{\pgfqpoint{0.981930in}{2.727183in}}{\pgfqpoint{0.970880in}{2.727183in}}%
\pgfpathcurveto{\pgfqpoint{0.959830in}{2.727183in}}{\pgfqpoint{0.949231in}{2.722793in}}{\pgfqpoint{0.941417in}{2.714979in}}%
\pgfpathcurveto{\pgfqpoint{0.933604in}{2.707165in}}{\pgfqpoint{0.929213in}{2.696566in}}{\pgfqpoint{0.929213in}{2.685516in}}%
\pgfpathcurveto{\pgfqpoint{0.929213in}{2.674466in}}{\pgfqpoint{0.933604in}{2.663867in}}{\pgfqpoint{0.941417in}{2.656053in}}%
\pgfpathcurveto{\pgfqpoint{0.949231in}{2.648240in}}{\pgfqpoint{0.959830in}{2.643850in}}{\pgfqpoint{0.970880in}{2.643850in}}%
\pgfpathlineto{\pgfqpoint{0.970880in}{2.643850in}}%
\pgfpathclose%
\pgfusepath{stroke}%
\end{pgfscope}%
\begin{pgfscope}%
\pgfpathrectangle{\pgfqpoint{0.494722in}{0.437222in}}{\pgfqpoint{6.275590in}{5.159444in}}%
\pgfusepath{clip}%
\pgfsetbuttcap%
\pgfsetroundjoin%
\pgfsetlinewidth{1.003750pt}%
\definecolor{currentstroke}{rgb}{0.827451,0.827451,0.827451}%
\pgfsetstrokecolor{currentstroke}%
\pgfsetstrokeopacity{0.800000}%
\pgfsetdash{}{0pt}%
\pgfpathmoveto{\pgfqpoint{5.377588in}{0.405033in}}%
\pgfpathcurveto{\pgfqpoint{5.388638in}{0.405033in}}{\pgfqpoint{5.399237in}{0.409423in}}{\pgfqpoint{5.407050in}{0.417237in}}%
\pgfpathcurveto{\pgfqpoint{5.414864in}{0.425051in}}{\pgfqpoint{5.419254in}{0.435650in}}{\pgfqpoint{5.419254in}{0.446700in}}%
\pgfpathcurveto{\pgfqpoint{5.419254in}{0.457750in}}{\pgfqpoint{5.414864in}{0.468349in}}{\pgfqpoint{5.407050in}{0.476163in}}%
\pgfpathcurveto{\pgfqpoint{5.399237in}{0.483976in}}{\pgfqpoint{5.388638in}{0.488366in}}{\pgfqpoint{5.377588in}{0.488366in}}%
\pgfpathcurveto{\pgfqpoint{5.366537in}{0.488366in}}{\pgfqpoint{5.355938in}{0.483976in}}{\pgfqpoint{5.348125in}{0.476163in}}%
\pgfpathcurveto{\pgfqpoint{5.340311in}{0.468349in}}{\pgfqpoint{5.335921in}{0.457750in}}{\pgfqpoint{5.335921in}{0.446700in}}%
\pgfpathcurveto{\pgfqpoint{5.335921in}{0.435650in}}{\pgfqpoint{5.340311in}{0.425051in}}{\pgfqpoint{5.348125in}{0.417237in}}%
\pgfpathcurveto{\pgfqpoint{5.355938in}{0.409423in}}{\pgfqpoint{5.366537in}{0.405033in}}{\pgfqpoint{5.377588in}{0.405033in}}%
\pgfusepath{stroke}%
\end{pgfscope}%
\begin{pgfscope}%
\pgfpathrectangle{\pgfqpoint{0.494722in}{0.437222in}}{\pgfqpoint{6.275590in}{5.159444in}}%
\pgfusepath{clip}%
\pgfsetbuttcap%
\pgfsetroundjoin%
\pgfsetlinewidth{1.003750pt}%
\definecolor{currentstroke}{rgb}{0.827451,0.827451,0.827451}%
\pgfsetstrokecolor{currentstroke}%
\pgfsetstrokeopacity{0.800000}%
\pgfsetdash{}{0pt}%
\pgfpathmoveto{\pgfqpoint{1.786945in}{1.657846in}}%
\pgfpathcurveto{\pgfqpoint{1.797995in}{1.657846in}}{\pgfqpoint{1.808594in}{1.662237in}}{\pgfqpoint{1.816407in}{1.670050in}}%
\pgfpathcurveto{\pgfqpoint{1.824221in}{1.677864in}}{\pgfqpoint{1.828611in}{1.688463in}}{\pgfqpoint{1.828611in}{1.699513in}}%
\pgfpathcurveto{\pgfqpoint{1.828611in}{1.710563in}}{\pgfqpoint{1.824221in}{1.721162in}}{\pgfqpoint{1.816407in}{1.728976in}}%
\pgfpathcurveto{\pgfqpoint{1.808594in}{1.736789in}}{\pgfqpoint{1.797995in}{1.741180in}}{\pgfqpoint{1.786945in}{1.741180in}}%
\pgfpathcurveto{\pgfqpoint{1.775894in}{1.741180in}}{\pgfqpoint{1.765295in}{1.736789in}}{\pgfqpoint{1.757482in}{1.728976in}}%
\pgfpathcurveto{\pgfqpoint{1.749668in}{1.721162in}}{\pgfqpoint{1.745278in}{1.710563in}}{\pgfqpoint{1.745278in}{1.699513in}}%
\pgfpathcurveto{\pgfqpoint{1.745278in}{1.688463in}}{\pgfqpoint{1.749668in}{1.677864in}}{\pgfqpoint{1.757482in}{1.670050in}}%
\pgfpathcurveto{\pgfqpoint{1.765295in}{1.662237in}}{\pgfqpoint{1.775894in}{1.657846in}}{\pgfqpoint{1.786945in}{1.657846in}}%
\pgfpathlineto{\pgfqpoint{1.786945in}{1.657846in}}%
\pgfpathclose%
\pgfusepath{stroke}%
\end{pgfscope}%
\begin{pgfscope}%
\pgfpathrectangle{\pgfqpoint{0.494722in}{0.437222in}}{\pgfqpoint{6.275590in}{5.159444in}}%
\pgfusepath{clip}%
\pgfsetbuttcap%
\pgfsetroundjoin%
\pgfsetlinewidth{1.003750pt}%
\definecolor{currentstroke}{rgb}{0.827451,0.827451,0.827451}%
\pgfsetstrokecolor{currentstroke}%
\pgfsetstrokeopacity{0.800000}%
\pgfsetdash{}{0pt}%
\pgfpathmoveto{\pgfqpoint{3.460812in}{0.723768in}}%
\pgfpathcurveto{\pgfqpoint{3.471862in}{0.723768in}}{\pgfqpoint{3.482461in}{0.728158in}}{\pgfqpoint{3.490274in}{0.735972in}}%
\pgfpathcurveto{\pgfqpoint{3.498088in}{0.743785in}}{\pgfqpoint{3.502478in}{0.754384in}}{\pgfqpoint{3.502478in}{0.765434in}}%
\pgfpathcurveto{\pgfqpoint{3.502478in}{0.776484in}}{\pgfqpoint{3.498088in}{0.787083in}}{\pgfqpoint{3.490274in}{0.794897in}}%
\pgfpathcurveto{\pgfqpoint{3.482461in}{0.802711in}}{\pgfqpoint{3.471862in}{0.807101in}}{\pgfqpoint{3.460812in}{0.807101in}}%
\pgfpathcurveto{\pgfqpoint{3.449762in}{0.807101in}}{\pgfqpoint{3.439162in}{0.802711in}}{\pgfqpoint{3.431349in}{0.794897in}}%
\pgfpathcurveto{\pgfqpoint{3.423535in}{0.787083in}}{\pgfqpoint{3.419145in}{0.776484in}}{\pgfqpoint{3.419145in}{0.765434in}}%
\pgfpathcurveto{\pgfqpoint{3.419145in}{0.754384in}}{\pgfqpoint{3.423535in}{0.743785in}}{\pgfqpoint{3.431349in}{0.735972in}}%
\pgfpathcurveto{\pgfqpoint{3.439162in}{0.728158in}}{\pgfqpoint{3.449762in}{0.723768in}}{\pgfqpoint{3.460812in}{0.723768in}}%
\pgfpathlineto{\pgfqpoint{3.460812in}{0.723768in}}%
\pgfpathclose%
\pgfusepath{stroke}%
\end{pgfscope}%
\begin{pgfscope}%
\pgfpathrectangle{\pgfqpoint{0.494722in}{0.437222in}}{\pgfqpoint{6.275590in}{5.159444in}}%
\pgfusepath{clip}%
\pgfsetbuttcap%
\pgfsetroundjoin%
\pgfsetlinewidth{1.003750pt}%
\definecolor{currentstroke}{rgb}{0.827451,0.827451,0.827451}%
\pgfsetstrokecolor{currentstroke}%
\pgfsetstrokeopacity{0.800000}%
\pgfsetdash{}{0pt}%
\pgfpathmoveto{\pgfqpoint{1.254182in}{2.218059in}}%
\pgfpathcurveto{\pgfqpoint{1.265232in}{2.218059in}}{\pgfqpoint{1.275831in}{2.222449in}}{\pgfqpoint{1.283645in}{2.230263in}}%
\pgfpathcurveto{\pgfqpoint{1.291458in}{2.238076in}}{\pgfqpoint{1.295849in}{2.248675in}}{\pgfqpoint{1.295849in}{2.259725in}}%
\pgfpathcurveto{\pgfqpoint{1.295849in}{2.270775in}}{\pgfqpoint{1.291458in}{2.281375in}}{\pgfqpoint{1.283645in}{2.289188in}}%
\pgfpathcurveto{\pgfqpoint{1.275831in}{2.297002in}}{\pgfqpoint{1.265232in}{2.301392in}}{\pgfqpoint{1.254182in}{2.301392in}}%
\pgfpathcurveto{\pgfqpoint{1.243132in}{2.301392in}}{\pgfqpoint{1.232533in}{2.297002in}}{\pgfqpoint{1.224719in}{2.289188in}}%
\pgfpathcurveto{\pgfqpoint{1.216905in}{2.281375in}}{\pgfqpoint{1.212515in}{2.270775in}}{\pgfqpoint{1.212515in}{2.259725in}}%
\pgfpathcurveto{\pgfqpoint{1.212515in}{2.248675in}}{\pgfqpoint{1.216905in}{2.238076in}}{\pgfqpoint{1.224719in}{2.230263in}}%
\pgfpathcurveto{\pgfqpoint{1.232533in}{2.222449in}}{\pgfqpoint{1.243132in}{2.218059in}}{\pgfqpoint{1.254182in}{2.218059in}}%
\pgfpathlineto{\pgfqpoint{1.254182in}{2.218059in}}%
\pgfpathclose%
\pgfusepath{stroke}%
\end{pgfscope}%
\begin{pgfscope}%
\pgfpathrectangle{\pgfqpoint{0.494722in}{0.437222in}}{\pgfqpoint{6.275590in}{5.159444in}}%
\pgfusepath{clip}%
\pgfsetbuttcap%
\pgfsetroundjoin%
\pgfsetlinewidth{1.003750pt}%
\definecolor{currentstroke}{rgb}{0.827451,0.827451,0.827451}%
\pgfsetstrokecolor{currentstroke}%
\pgfsetstrokeopacity{0.800000}%
\pgfsetdash{}{0pt}%
\pgfpathmoveto{\pgfqpoint{1.325845in}{2.126824in}}%
\pgfpathcurveto{\pgfqpoint{1.336895in}{2.126824in}}{\pgfqpoint{1.347494in}{2.131214in}}{\pgfqpoint{1.355308in}{2.139028in}}%
\pgfpathcurveto{\pgfqpoint{1.363121in}{2.146842in}}{\pgfqpoint{1.367512in}{2.157441in}}{\pgfqpoint{1.367512in}{2.168491in}}%
\pgfpathcurveto{\pgfqpoint{1.367512in}{2.179541in}}{\pgfqpoint{1.363121in}{2.190140in}}{\pgfqpoint{1.355308in}{2.197954in}}%
\pgfpathcurveto{\pgfqpoint{1.347494in}{2.205767in}}{\pgfqpoint{1.336895in}{2.210157in}}{\pgfqpoint{1.325845in}{2.210157in}}%
\pgfpathcurveto{\pgfqpoint{1.314795in}{2.210157in}}{\pgfqpoint{1.304196in}{2.205767in}}{\pgfqpoint{1.296382in}{2.197954in}}%
\pgfpathcurveto{\pgfqpoint{1.288569in}{2.190140in}}{\pgfqpoint{1.284178in}{2.179541in}}{\pgfqpoint{1.284178in}{2.168491in}}%
\pgfpathcurveto{\pgfqpoint{1.284178in}{2.157441in}}{\pgfqpoint{1.288569in}{2.146842in}}{\pgfqpoint{1.296382in}{2.139028in}}%
\pgfpathcurveto{\pgfqpoint{1.304196in}{2.131214in}}{\pgfqpoint{1.314795in}{2.126824in}}{\pgfqpoint{1.325845in}{2.126824in}}%
\pgfpathlineto{\pgfqpoint{1.325845in}{2.126824in}}%
\pgfpathclose%
\pgfusepath{stroke}%
\end{pgfscope}%
\begin{pgfscope}%
\pgfpathrectangle{\pgfqpoint{0.494722in}{0.437222in}}{\pgfqpoint{6.275590in}{5.159444in}}%
\pgfusepath{clip}%
\pgfsetbuttcap%
\pgfsetroundjoin%
\pgfsetlinewidth{1.003750pt}%
\definecolor{currentstroke}{rgb}{0.827451,0.827451,0.827451}%
\pgfsetstrokecolor{currentstroke}%
\pgfsetstrokeopacity{0.800000}%
\pgfsetdash{}{0pt}%
\pgfpathmoveto{\pgfqpoint{1.964907in}{1.517037in}}%
\pgfpathcurveto{\pgfqpoint{1.975957in}{1.517037in}}{\pgfqpoint{1.986556in}{1.521427in}}{\pgfqpoint{1.994370in}{1.529240in}}%
\pgfpathcurveto{\pgfqpoint{2.002183in}{1.537054in}}{\pgfqpoint{2.006573in}{1.547653in}}{\pgfqpoint{2.006573in}{1.558703in}}%
\pgfpathcurveto{\pgfqpoint{2.006573in}{1.569753in}}{\pgfqpoint{2.002183in}{1.580352in}}{\pgfqpoint{1.994370in}{1.588166in}}%
\pgfpathcurveto{\pgfqpoint{1.986556in}{1.595980in}}{\pgfqpoint{1.975957in}{1.600370in}}{\pgfqpoint{1.964907in}{1.600370in}}%
\pgfpathcurveto{\pgfqpoint{1.953857in}{1.600370in}}{\pgfqpoint{1.943258in}{1.595980in}}{\pgfqpoint{1.935444in}{1.588166in}}%
\pgfpathcurveto{\pgfqpoint{1.927630in}{1.580352in}}{\pgfqpoint{1.923240in}{1.569753in}}{\pgfqpoint{1.923240in}{1.558703in}}%
\pgfpathcurveto{\pgfqpoint{1.923240in}{1.547653in}}{\pgfqpoint{1.927630in}{1.537054in}}{\pgfqpoint{1.935444in}{1.529240in}}%
\pgfpathcurveto{\pgfqpoint{1.943258in}{1.521427in}}{\pgfqpoint{1.953857in}{1.517037in}}{\pgfqpoint{1.964907in}{1.517037in}}%
\pgfpathlineto{\pgfqpoint{1.964907in}{1.517037in}}%
\pgfpathclose%
\pgfusepath{stroke}%
\end{pgfscope}%
\begin{pgfscope}%
\pgfpathrectangle{\pgfqpoint{0.494722in}{0.437222in}}{\pgfqpoint{6.275590in}{5.159444in}}%
\pgfusepath{clip}%
\pgfsetbuttcap%
\pgfsetroundjoin%
\pgfsetlinewidth{1.003750pt}%
\definecolor{currentstroke}{rgb}{0.827451,0.827451,0.827451}%
\pgfsetstrokecolor{currentstroke}%
\pgfsetstrokeopacity{0.800000}%
\pgfsetdash{}{0pt}%
\pgfpathmoveto{\pgfqpoint{1.207472in}{2.314347in}}%
\pgfpathcurveto{\pgfqpoint{1.218522in}{2.314347in}}{\pgfqpoint{1.229121in}{2.318737in}}{\pgfqpoint{1.236935in}{2.326551in}}%
\pgfpathcurveto{\pgfqpoint{1.244749in}{2.334365in}}{\pgfqpoint{1.249139in}{2.344964in}}{\pgfqpoint{1.249139in}{2.356014in}}%
\pgfpathcurveto{\pgfqpoint{1.249139in}{2.367064in}}{\pgfqpoint{1.244749in}{2.377663in}}{\pgfqpoint{1.236935in}{2.385477in}}%
\pgfpathcurveto{\pgfqpoint{1.229121in}{2.393290in}}{\pgfqpoint{1.218522in}{2.397680in}}{\pgfqpoint{1.207472in}{2.397680in}}%
\pgfpathcurveto{\pgfqpoint{1.196422in}{2.397680in}}{\pgfqpoint{1.185823in}{2.393290in}}{\pgfqpoint{1.178010in}{2.385477in}}%
\pgfpathcurveto{\pgfqpoint{1.170196in}{2.377663in}}{\pgfqpoint{1.165806in}{2.367064in}}{\pgfqpoint{1.165806in}{2.356014in}}%
\pgfpathcurveto{\pgfqpoint{1.165806in}{2.344964in}}{\pgfqpoint{1.170196in}{2.334365in}}{\pgfqpoint{1.178010in}{2.326551in}}%
\pgfpathcurveto{\pgfqpoint{1.185823in}{2.318737in}}{\pgfqpoint{1.196422in}{2.314347in}}{\pgfqpoint{1.207472in}{2.314347in}}%
\pgfpathlineto{\pgfqpoint{1.207472in}{2.314347in}}%
\pgfpathclose%
\pgfusepath{stroke}%
\end{pgfscope}%
\begin{pgfscope}%
\pgfpathrectangle{\pgfqpoint{0.494722in}{0.437222in}}{\pgfqpoint{6.275590in}{5.159444in}}%
\pgfusepath{clip}%
\pgfsetbuttcap%
\pgfsetroundjoin%
\pgfsetlinewidth{1.003750pt}%
\definecolor{currentstroke}{rgb}{0.827451,0.827451,0.827451}%
\pgfsetstrokecolor{currentstroke}%
\pgfsetstrokeopacity{0.800000}%
\pgfsetdash{}{0pt}%
\pgfpathmoveto{\pgfqpoint{2.183447in}{1.371651in}}%
\pgfpathcurveto{\pgfqpoint{2.194497in}{1.371651in}}{\pgfqpoint{2.205096in}{1.376041in}}{\pgfqpoint{2.212910in}{1.383855in}}%
\pgfpathcurveto{\pgfqpoint{2.220724in}{1.391668in}}{\pgfqpoint{2.225114in}{1.402267in}}{\pgfqpoint{2.225114in}{1.413318in}}%
\pgfpathcurveto{\pgfqpoint{2.225114in}{1.424368in}}{\pgfqpoint{2.220724in}{1.434967in}}{\pgfqpoint{2.212910in}{1.442780in}}%
\pgfpathcurveto{\pgfqpoint{2.205096in}{1.450594in}}{\pgfqpoint{2.194497in}{1.454984in}}{\pgfqpoint{2.183447in}{1.454984in}}%
\pgfpathcurveto{\pgfqpoint{2.172397in}{1.454984in}}{\pgfqpoint{2.161798in}{1.450594in}}{\pgfqpoint{2.153984in}{1.442780in}}%
\pgfpathcurveto{\pgfqpoint{2.146171in}{1.434967in}}{\pgfqpoint{2.141781in}{1.424368in}}{\pgfqpoint{2.141781in}{1.413318in}}%
\pgfpathcurveto{\pgfqpoint{2.141781in}{1.402267in}}{\pgfqpoint{2.146171in}{1.391668in}}{\pgfqpoint{2.153984in}{1.383855in}}%
\pgfpathcurveto{\pgfqpoint{2.161798in}{1.376041in}}{\pgfqpoint{2.172397in}{1.371651in}}{\pgfqpoint{2.183447in}{1.371651in}}%
\pgfpathlineto{\pgfqpoint{2.183447in}{1.371651in}}%
\pgfpathclose%
\pgfusepath{stroke}%
\end{pgfscope}%
\begin{pgfscope}%
\pgfpathrectangle{\pgfqpoint{0.494722in}{0.437222in}}{\pgfqpoint{6.275590in}{5.159444in}}%
\pgfusepath{clip}%
\pgfsetbuttcap%
\pgfsetroundjoin%
\pgfsetlinewidth{1.003750pt}%
\definecolor{currentstroke}{rgb}{0.827451,0.827451,0.827451}%
\pgfsetstrokecolor{currentstroke}%
\pgfsetstrokeopacity{0.800000}%
\pgfsetdash{}{0pt}%
\pgfpathmoveto{\pgfqpoint{0.770298in}{3.077236in}}%
\pgfpathcurveto{\pgfqpoint{0.781349in}{3.077236in}}{\pgfqpoint{0.791948in}{3.081626in}}{\pgfqpoint{0.799761in}{3.089440in}}%
\pgfpathcurveto{\pgfqpoint{0.807575in}{3.097254in}}{\pgfqpoint{0.811965in}{3.107853in}}{\pgfqpoint{0.811965in}{3.118903in}}%
\pgfpathcurveto{\pgfqpoint{0.811965in}{3.129953in}}{\pgfqpoint{0.807575in}{3.140552in}}{\pgfqpoint{0.799761in}{3.148366in}}%
\pgfpathcurveto{\pgfqpoint{0.791948in}{3.156179in}}{\pgfqpoint{0.781349in}{3.160569in}}{\pgfqpoint{0.770298in}{3.160569in}}%
\pgfpathcurveto{\pgfqpoint{0.759248in}{3.160569in}}{\pgfqpoint{0.748649in}{3.156179in}}{\pgfqpoint{0.740836in}{3.148366in}}%
\pgfpathcurveto{\pgfqpoint{0.733022in}{3.140552in}}{\pgfqpoint{0.728632in}{3.129953in}}{\pgfqpoint{0.728632in}{3.118903in}}%
\pgfpathcurveto{\pgfqpoint{0.728632in}{3.107853in}}{\pgfqpoint{0.733022in}{3.097254in}}{\pgfqpoint{0.740836in}{3.089440in}}%
\pgfpathcurveto{\pgfqpoint{0.748649in}{3.081626in}}{\pgfqpoint{0.759248in}{3.077236in}}{\pgfqpoint{0.770298in}{3.077236in}}%
\pgfpathlineto{\pgfqpoint{0.770298in}{3.077236in}}%
\pgfpathclose%
\pgfusepath{stroke}%
\end{pgfscope}%
\begin{pgfscope}%
\pgfpathrectangle{\pgfqpoint{0.494722in}{0.437222in}}{\pgfqpoint{6.275590in}{5.159444in}}%
\pgfusepath{clip}%
\pgfsetbuttcap%
\pgfsetroundjoin%
\pgfsetlinewidth{1.003750pt}%
\definecolor{currentstroke}{rgb}{0.827451,0.827451,0.827451}%
\pgfsetstrokecolor{currentstroke}%
\pgfsetstrokeopacity{0.800000}%
\pgfsetdash{}{0pt}%
\pgfpathmoveto{\pgfqpoint{0.608606in}{3.646796in}}%
\pgfpathcurveto{\pgfqpoint{0.619657in}{3.646796in}}{\pgfqpoint{0.630256in}{3.651186in}}{\pgfqpoint{0.638069in}{3.659000in}}%
\pgfpathcurveto{\pgfqpoint{0.645883in}{3.666813in}}{\pgfqpoint{0.650273in}{3.677412in}}{\pgfqpoint{0.650273in}{3.688462in}}%
\pgfpathcurveto{\pgfqpoint{0.650273in}{3.699512in}}{\pgfqpoint{0.645883in}{3.710112in}}{\pgfqpoint{0.638069in}{3.717925in}}%
\pgfpathcurveto{\pgfqpoint{0.630256in}{3.725739in}}{\pgfqpoint{0.619657in}{3.730129in}}{\pgfqpoint{0.608606in}{3.730129in}}%
\pgfpathcurveto{\pgfqpoint{0.597556in}{3.730129in}}{\pgfqpoint{0.586957in}{3.725739in}}{\pgfqpoint{0.579144in}{3.717925in}}%
\pgfpathcurveto{\pgfqpoint{0.571330in}{3.710112in}}{\pgfqpoint{0.566940in}{3.699512in}}{\pgfqpoint{0.566940in}{3.688462in}}%
\pgfpathcurveto{\pgfqpoint{0.566940in}{3.677412in}}{\pgfqpoint{0.571330in}{3.666813in}}{\pgfqpoint{0.579144in}{3.659000in}}%
\pgfpathcurveto{\pgfqpoint{0.586957in}{3.651186in}}{\pgfqpoint{0.597556in}{3.646796in}}{\pgfqpoint{0.608606in}{3.646796in}}%
\pgfpathlineto{\pgfqpoint{0.608606in}{3.646796in}}%
\pgfpathclose%
\pgfusepath{stroke}%
\end{pgfscope}%
\begin{pgfscope}%
\pgfpathrectangle{\pgfqpoint{0.494722in}{0.437222in}}{\pgfqpoint{6.275590in}{5.159444in}}%
\pgfusepath{clip}%
\pgfsetbuttcap%
\pgfsetroundjoin%
\pgfsetlinewidth{1.003750pt}%
\definecolor{currentstroke}{rgb}{0.827451,0.827451,0.827451}%
\pgfsetstrokecolor{currentstroke}%
\pgfsetstrokeopacity{0.800000}%
\pgfsetdash{}{0pt}%
\pgfpathmoveto{\pgfqpoint{1.268327in}{2.198214in}}%
\pgfpathcurveto{\pgfqpoint{1.279377in}{2.198214in}}{\pgfqpoint{1.289976in}{2.202604in}}{\pgfqpoint{1.297790in}{2.210418in}}%
\pgfpathcurveto{\pgfqpoint{1.305603in}{2.218231in}}{\pgfqpoint{1.309994in}{2.228830in}}{\pgfqpoint{1.309994in}{2.239880in}}%
\pgfpathcurveto{\pgfqpoint{1.309994in}{2.250930in}}{\pgfqpoint{1.305603in}{2.261529in}}{\pgfqpoint{1.297790in}{2.269343in}}%
\pgfpathcurveto{\pgfqpoint{1.289976in}{2.277157in}}{\pgfqpoint{1.279377in}{2.281547in}}{\pgfqpoint{1.268327in}{2.281547in}}%
\pgfpathcurveto{\pgfqpoint{1.257277in}{2.281547in}}{\pgfqpoint{1.246678in}{2.277157in}}{\pgfqpoint{1.238864in}{2.269343in}}%
\pgfpathcurveto{\pgfqpoint{1.231051in}{2.261529in}}{\pgfqpoint{1.226660in}{2.250930in}}{\pgfqpoint{1.226660in}{2.239880in}}%
\pgfpathcurveto{\pgfqpoint{1.226660in}{2.228830in}}{\pgfqpoint{1.231051in}{2.218231in}}{\pgfqpoint{1.238864in}{2.210418in}}%
\pgfpathcurveto{\pgfqpoint{1.246678in}{2.202604in}}{\pgfqpoint{1.257277in}{2.198214in}}{\pgfqpoint{1.268327in}{2.198214in}}%
\pgfpathlineto{\pgfqpoint{1.268327in}{2.198214in}}%
\pgfpathclose%
\pgfusepath{stroke}%
\end{pgfscope}%
\begin{pgfscope}%
\pgfpathrectangle{\pgfqpoint{0.494722in}{0.437222in}}{\pgfqpoint{6.275590in}{5.159444in}}%
\pgfusepath{clip}%
\pgfsetbuttcap%
\pgfsetroundjoin%
\pgfsetlinewidth{1.003750pt}%
\definecolor{currentstroke}{rgb}{0.827451,0.827451,0.827451}%
\pgfsetstrokecolor{currentstroke}%
\pgfsetstrokeopacity{0.800000}%
\pgfsetdash{}{0pt}%
\pgfpathmoveto{\pgfqpoint{5.378653in}{0.404615in}}%
\pgfpathcurveto{\pgfqpoint{5.389703in}{0.404615in}}{\pgfqpoint{5.400302in}{0.409005in}}{\pgfqpoint{5.408116in}{0.416819in}}%
\pgfpathcurveto{\pgfqpoint{5.415930in}{0.424632in}}{\pgfqpoint{5.420320in}{0.435231in}}{\pgfqpoint{5.420320in}{0.446281in}}%
\pgfpathcurveto{\pgfqpoint{5.420320in}{0.457331in}}{\pgfqpoint{5.415930in}{0.467930in}}{\pgfqpoint{5.408116in}{0.475744in}}%
\pgfpathcurveto{\pgfqpoint{5.400302in}{0.483558in}}{\pgfqpoint{5.389703in}{0.487948in}}{\pgfqpoint{5.378653in}{0.487948in}}%
\pgfpathcurveto{\pgfqpoint{5.367603in}{0.487948in}}{\pgfqpoint{5.357004in}{0.483558in}}{\pgfqpoint{5.349190in}{0.475744in}}%
\pgfpathcurveto{\pgfqpoint{5.341377in}{0.467930in}}{\pgfqpoint{5.336987in}{0.457331in}}{\pgfqpoint{5.336987in}{0.446281in}}%
\pgfpathcurveto{\pgfqpoint{5.336987in}{0.435231in}}{\pgfqpoint{5.341377in}{0.424632in}}{\pgfqpoint{5.349190in}{0.416819in}}%
\pgfpathcurveto{\pgfqpoint{5.357004in}{0.409005in}}{\pgfqpoint{5.367603in}{0.404615in}}{\pgfqpoint{5.378653in}{0.404615in}}%
\pgfusepath{stroke}%
\end{pgfscope}%
\begin{pgfscope}%
\pgfpathrectangle{\pgfqpoint{0.494722in}{0.437222in}}{\pgfqpoint{6.275590in}{5.159444in}}%
\pgfusepath{clip}%
\pgfsetbuttcap%
\pgfsetroundjoin%
\pgfsetlinewidth{1.003750pt}%
\definecolor{currentstroke}{rgb}{0.827451,0.827451,0.827451}%
\pgfsetstrokecolor{currentstroke}%
\pgfsetstrokeopacity{0.800000}%
\pgfsetdash{}{0pt}%
\pgfpathmoveto{\pgfqpoint{4.362526in}{0.500037in}}%
\pgfpathcurveto{\pgfqpoint{4.373576in}{0.500037in}}{\pgfqpoint{4.384175in}{0.504427in}}{\pgfqpoint{4.391989in}{0.512241in}}%
\pgfpathcurveto{\pgfqpoint{4.399802in}{0.520054in}}{\pgfqpoint{4.404193in}{0.530653in}}{\pgfqpoint{4.404193in}{0.541703in}}%
\pgfpathcurveto{\pgfqpoint{4.404193in}{0.552753in}}{\pgfqpoint{4.399802in}{0.563352in}}{\pgfqpoint{4.391989in}{0.571166in}}%
\pgfpathcurveto{\pgfqpoint{4.384175in}{0.578980in}}{\pgfqpoint{4.373576in}{0.583370in}}{\pgfqpoint{4.362526in}{0.583370in}}%
\pgfpathcurveto{\pgfqpoint{4.351476in}{0.583370in}}{\pgfqpoint{4.340877in}{0.578980in}}{\pgfqpoint{4.333063in}{0.571166in}}%
\pgfpathcurveto{\pgfqpoint{4.325250in}{0.563352in}}{\pgfqpoint{4.320859in}{0.552753in}}{\pgfqpoint{4.320859in}{0.541703in}}%
\pgfpathcurveto{\pgfqpoint{4.320859in}{0.530653in}}{\pgfqpoint{4.325250in}{0.520054in}}{\pgfqpoint{4.333063in}{0.512241in}}%
\pgfpathcurveto{\pgfqpoint{4.340877in}{0.504427in}}{\pgfqpoint{4.351476in}{0.500037in}}{\pgfqpoint{4.362526in}{0.500037in}}%
\pgfpathlineto{\pgfqpoint{4.362526in}{0.500037in}}%
\pgfpathclose%
\pgfusepath{stroke}%
\end{pgfscope}%
\begin{pgfscope}%
\pgfpathrectangle{\pgfqpoint{0.494722in}{0.437222in}}{\pgfqpoint{6.275590in}{5.159444in}}%
\pgfusepath{clip}%
\pgfsetbuttcap%
\pgfsetroundjoin%
\pgfsetlinewidth{1.003750pt}%
\definecolor{currentstroke}{rgb}{0.827451,0.827451,0.827451}%
\pgfsetstrokecolor{currentstroke}%
\pgfsetstrokeopacity{0.800000}%
\pgfsetdash{}{0pt}%
\pgfpathmoveto{\pgfqpoint{1.464755in}{1.963622in}}%
\pgfpathcurveto{\pgfqpoint{1.475805in}{1.963622in}}{\pgfqpoint{1.486404in}{1.968013in}}{\pgfqpoint{1.494217in}{1.975826in}}%
\pgfpathcurveto{\pgfqpoint{1.502031in}{1.983640in}}{\pgfqpoint{1.506421in}{1.994239in}}{\pgfqpoint{1.506421in}{2.005289in}}%
\pgfpathcurveto{\pgfqpoint{1.506421in}{2.016339in}}{\pgfqpoint{1.502031in}{2.026938in}}{\pgfqpoint{1.494217in}{2.034752in}}%
\pgfpathcurveto{\pgfqpoint{1.486404in}{2.042566in}}{\pgfqpoint{1.475805in}{2.046956in}}{\pgfqpoint{1.464755in}{2.046956in}}%
\pgfpathcurveto{\pgfqpoint{1.453705in}{2.046956in}}{\pgfqpoint{1.443106in}{2.042566in}}{\pgfqpoint{1.435292in}{2.034752in}}%
\pgfpathcurveto{\pgfqpoint{1.427478in}{2.026938in}}{\pgfqpoint{1.423088in}{2.016339in}}{\pgfqpoint{1.423088in}{2.005289in}}%
\pgfpathcurveto{\pgfqpoint{1.423088in}{1.994239in}}{\pgfqpoint{1.427478in}{1.983640in}}{\pgfqpoint{1.435292in}{1.975826in}}%
\pgfpathcurveto{\pgfqpoint{1.443106in}{1.968013in}}{\pgfqpoint{1.453705in}{1.963622in}}{\pgfqpoint{1.464755in}{1.963622in}}%
\pgfpathlineto{\pgfqpoint{1.464755in}{1.963622in}}%
\pgfpathclose%
\pgfusepath{stroke}%
\end{pgfscope}%
\begin{pgfscope}%
\pgfpathrectangle{\pgfqpoint{0.494722in}{0.437222in}}{\pgfqpoint{6.275590in}{5.159444in}}%
\pgfusepath{clip}%
\pgfsetbuttcap%
\pgfsetroundjoin%
\pgfsetlinewidth{1.003750pt}%
\definecolor{currentstroke}{rgb}{0.827451,0.827451,0.827451}%
\pgfsetstrokecolor{currentstroke}%
\pgfsetstrokeopacity{0.800000}%
\pgfsetdash{}{0pt}%
\pgfpathmoveto{\pgfqpoint{2.658320in}{1.113694in}}%
\pgfpathcurveto{\pgfqpoint{2.669370in}{1.113694in}}{\pgfqpoint{2.679969in}{1.118084in}}{\pgfqpoint{2.687783in}{1.125898in}}%
\pgfpathcurveto{\pgfqpoint{2.695596in}{1.133711in}}{\pgfqpoint{2.699987in}{1.144310in}}{\pgfqpoint{2.699987in}{1.155361in}}%
\pgfpathcurveto{\pgfqpoint{2.699987in}{1.166411in}}{\pgfqpoint{2.695596in}{1.177010in}}{\pgfqpoint{2.687783in}{1.184823in}}%
\pgfpathcurveto{\pgfqpoint{2.679969in}{1.192637in}}{\pgfqpoint{2.669370in}{1.197027in}}{\pgfqpoint{2.658320in}{1.197027in}}%
\pgfpathcurveto{\pgfqpoint{2.647270in}{1.197027in}}{\pgfqpoint{2.636671in}{1.192637in}}{\pgfqpoint{2.628857in}{1.184823in}}%
\pgfpathcurveto{\pgfqpoint{2.621044in}{1.177010in}}{\pgfqpoint{2.616653in}{1.166411in}}{\pgfqpoint{2.616653in}{1.155361in}}%
\pgfpathcurveto{\pgfqpoint{2.616653in}{1.144310in}}{\pgfqpoint{2.621044in}{1.133711in}}{\pgfqpoint{2.628857in}{1.125898in}}%
\pgfpathcurveto{\pgfqpoint{2.636671in}{1.118084in}}{\pgfqpoint{2.647270in}{1.113694in}}{\pgfqpoint{2.658320in}{1.113694in}}%
\pgfpathlineto{\pgfqpoint{2.658320in}{1.113694in}}%
\pgfpathclose%
\pgfusepath{stroke}%
\end{pgfscope}%
\begin{pgfscope}%
\pgfpathrectangle{\pgfqpoint{0.494722in}{0.437222in}}{\pgfqpoint{6.275590in}{5.159444in}}%
\pgfusepath{clip}%
\pgfsetbuttcap%
\pgfsetroundjoin%
\pgfsetlinewidth{1.003750pt}%
\definecolor{currentstroke}{rgb}{0.827451,0.827451,0.827451}%
\pgfsetstrokecolor{currentstroke}%
\pgfsetstrokeopacity{0.800000}%
\pgfsetdash{}{0pt}%
\pgfpathmoveto{\pgfqpoint{2.702598in}{1.043180in}}%
\pgfpathcurveto{\pgfqpoint{2.713648in}{1.043180in}}{\pgfqpoint{2.724247in}{1.047570in}}{\pgfqpoint{2.732061in}{1.055384in}}%
\pgfpathcurveto{\pgfqpoint{2.739874in}{1.063198in}}{\pgfqpoint{2.744265in}{1.073797in}}{\pgfqpoint{2.744265in}{1.084847in}}%
\pgfpathcurveto{\pgfqpoint{2.744265in}{1.095897in}}{\pgfqpoint{2.739874in}{1.106496in}}{\pgfqpoint{2.732061in}{1.114310in}}%
\pgfpathcurveto{\pgfqpoint{2.724247in}{1.122123in}}{\pgfqpoint{2.713648in}{1.126513in}}{\pgfqpoint{2.702598in}{1.126513in}}%
\pgfpathcurveto{\pgfqpoint{2.691548in}{1.126513in}}{\pgfqpoint{2.680949in}{1.122123in}}{\pgfqpoint{2.673135in}{1.114310in}}%
\pgfpathcurveto{\pgfqpoint{2.665321in}{1.106496in}}{\pgfqpoint{2.660931in}{1.095897in}}{\pgfqpoint{2.660931in}{1.084847in}}%
\pgfpathcurveto{\pgfqpoint{2.660931in}{1.073797in}}{\pgfqpoint{2.665321in}{1.063198in}}{\pgfqpoint{2.673135in}{1.055384in}}%
\pgfpathcurveto{\pgfqpoint{2.680949in}{1.047570in}}{\pgfqpoint{2.691548in}{1.043180in}}{\pgfqpoint{2.702598in}{1.043180in}}%
\pgfpathlineto{\pgfqpoint{2.702598in}{1.043180in}}%
\pgfpathclose%
\pgfusepath{stroke}%
\end{pgfscope}%
\begin{pgfscope}%
\pgfpathrectangle{\pgfqpoint{0.494722in}{0.437222in}}{\pgfqpoint{6.275590in}{5.159444in}}%
\pgfusepath{clip}%
\pgfsetbuttcap%
\pgfsetroundjoin%
\pgfsetlinewidth{1.003750pt}%
\definecolor{currentstroke}{rgb}{0.827451,0.827451,0.827451}%
\pgfsetstrokecolor{currentstroke}%
\pgfsetstrokeopacity{0.800000}%
\pgfsetdash{}{0pt}%
\pgfpathmoveto{\pgfqpoint{1.520178in}{1.905257in}}%
\pgfpathcurveto{\pgfqpoint{1.531228in}{1.905257in}}{\pgfqpoint{1.541827in}{1.909647in}}{\pgfqpoint{1.549641in}{1.917461in}}%
\pgfpathcurveto{\pgfqpoint{1.557455in}{1.925274in}}{\pgfqpoint{1.561845in}{1.935873in}}{\pgfqpoint{1.561845in}{1.946923in}}%
\pgfpathcurveto{\pgfqpoint{1.561845in}{1.957973in}}{\pgfqpoint{1.557455in}{1.968573in}}{\pgfqpoint{1.549641in}{1.976386in}}%
\pgfpathcurveto{\pgfqpoint{1.541827in}{1.984200in}}{\pgfqpoint{1.531228in}{1.988590in}}{\pgfqpoint{1.520178in}{1.988590in}}%
\pgfpathcurveto{\pgfqpoint{1.509128in}{1.988590in}}{\pgfqpoint{1.498529in}{1.984200in}}{\pgfqpoint{1.490715in}{1.976386in}}%
\pgfpathcurveto{\pgfqpoint{1.482902in}{1.968573in}}{\pgfqpoint{1.478512in}{1.957973in}}{\pgfqpoint{1.478512in}{1.946923in}}%
\pgfpathcurveto{\pgfqpoint{1.478512in}{1.935873in}}{\pgfqpoint{1.482902in}{1.925274in}}{\pgfqpoint{1.490715in}{1.917461in}}%
\pgfpathcurveto{\pgfqpoint{1.498529in}{1.909647in}}{\pgfqpoint{1.509128in}{1.905257in}}{\pgfqpoint{1.520178in}{1.905257in}}%
\pgfpathlineto{\pgfqpoint{1.520178in}{1.905257in}}%
\pgfpathclose%
\pgfusepath{stroke}%
\end{pgfscope}%
\begin{pgfscope}%
\pgfpathrectangle{\pgfqpoint{0.494722in}{0.437222in}}{\pgfqpoint{6.275590in}{5.159444in}}%
\pgfusepath{clip}%
\pgfsetbuttcap%
\pgfsetroundjoin%
\pgfsetlinewidth{1.003750pt}%
\definecolor{currentstroke}{rgb}{0.827451,0.827451,0.827451}%
\pgfsetstrokecolor{currentstroke}%
\pgfsetstrokeopacity{0.800000}%
\pgfsetdash{}{0pt}%
\pgfpathmoveto{\pgfqpoint{3.372979in}{0.744763in}}%
\pgfpathcurveto{\pgfqpoint{3.384029in}{0.744763in}}{\pgfqpoint{3.394628in}{0.749153in}}{\pgfqpoint{3.402442in}{0.756967in}}%
\pgfpathcurveto{\pgfqpoint{3.410255in}{0.764781in}}{\pgfqpoint{3.414646in}{0.775380in}}{\pgfqpoint{3.414646in}{0.786430in}}%
\pgfpathcurveto{\pgfqpoint{3.414646in}{0.797480in}}{\pgfqpoint{3.410255in}{0.808079in}}{\pgfqpoint{3.402442in}{0.815893in}}%
\pgfpathcurveto{\pgfqpoint{3.394628in}{0.823706in}}{\pgfqpoint{3.384029in}{0.828097in}}{\pgfqpoint{3.372979in}{0.828097in}}%
\pgfpathcurveto{\pgfqpoint{3.361929in}{0.828097in}}{\pgfqpoint{3.351330in}{0.823706in}}{\pgfqpoint{3.343516in}{0.815893in}}%
\pgfpathcurveto{\pgfqpoint{3.335703in}{0.808079in}}{\pgfqpoint{3.331312in}{0.797480in}}{\pgfqpoint{3.331312in}{0.786430in}}%
\pgfpathcurveto{\pgfqpoint{3.331312in}{0.775380in}}{\pgfqpoint{3.335703in}{0.764781in}}{\pgfqpoint{3.343516in}{0.756967in}}%
\pgfpathcurveto{\pgfqpoint{3.351330in}{0.749153in}}{\pgfqpoint{3.361929in}{0.744763in}}{\pgfqpoint{3.372979in}{0.744763in}}%
\pgfpathlineto{\pgfqpoint{3.372979in}{0.744763in}}%
\pgfpathclose%
\pgfusepath{stroke}%
\end{pgfscope}%
\begin{pgfscope}%
\pgfpathrectangle{\pgfqpoint{0.494722in}{0.437222in}}{\pgfqpoint{6.275590in}{5.159444in}}%
\pgfusepath{clip}%
\pgfsetbuttcap%
\pgfsetroundjoin%
\pgfsetlinewidth{1.003750pt}%
\definecolor{currentstroke}{rgb}{0.827451,0.827451,0.827451}%
\pgfsetstrokecolor{currentstroke}%
\pgfsetstrokeopacity{0.800000}%
\pgfsetdash{}{0pt}%
\pgfpathmoveto{\pgfqpoint{0.611323in}{3.579035in}}%
\pgfpathcurveto{\pgfqpoint{0.622373in}{3.579035in}}{\pgfqpoint{0.632972in}{3.583425in}}{\pgfqpoint{0.640786in}{3.591238in}}%
\pgfpathcurveto{\pgfqpoint{0.648599in}{3.599052in}}{\pgfqpoint{0.652989in}{3.609651in}}{\pgfqpoint{0.652989in}{3.620701in}}%
\pgfpathcurveto{\pgfqpoint{0.652989in}{3.631751in}}{\pgfqpoint{0.648599in}{3.642350in}}{\pgfqpoint{0.640786in}{3.650164in}}%
\pgfpathcurveto{\pgfqpoint{0.632972in}{3.657978in}}{\pgfqpoint{0.622373in}{3.662368in}}{\pgfqpoint{0.611323in}{3.662368in}}%
\pgfpathcurveto{\pgfqpoint{0.600273in}{3.662368in}}{\pgfqpoint{0.589674in}{3.657978in}}{\pgfqpoint{0.581860in}{3.650164in}}%
\pgfpathcurveto{\pgfqpoint{0.574046in}{3.642350in}}{\pgfqpoint{0.569656in}{3.631751in}}{\pgfqpoint{0.569656in}{3.620701in}}%
\pgfpathcurveto{\pgfqpoint{0.569656in}{3.609651in}}{\pgfqpoint{0.574046in}{3.599052in}}{\pgfqpoint{0.581860in}{3.591238in}}%
\pgfpathcurveto{\pgfqpoint{0.589674in}{3.583425in}}{\pgfqpoint{0.600273in}{3.579035in}}{\pgfqpoint{0.611323in}{3.579035in}}%
\pgfpathlineto{\pgfqpoint{0.611323in}{3.579035in}}%
\pgfpathclose%
\pgfusepath{stroke}%
\end{pgfscope}%
\begin{pgfscope}%
\pgfpathrectangle{\pgfqpoint{0.494722in}{0.437222in}}{\pgfqpoint{6.275590in}{5.159444in}}%
\pgfusepath{clip}%
\pgfsetbuttcap%
\pgfsetroundjoin%
\pgfsetlinewidth{1.003750pt}%
\definecolor{currentstroke}{rgb}{0.827451,0.827451,0.827451}%
\pgfsetstrokecolor{currentstroke}%
\pgfsetstrokeopacity{0.800000}%
\pgfsetdash{}{0pt}%
\pgfpathmoveto{\pgfqpoint{1.113110in}{2.511719in}}%
\pgfpathcurveto{\pgfqpoint{1.124161in}{2.511719in}}{\pgfqpoint{1.134760in}{2.516109in}}{\pgfqpoint{1.142573in}{2.523923in}}%
\pgfpathcurveto{\pgfqpoint{1.150387in}{2.531736in}}{\pgfqpoint{1.154777in}{2.542335in}}{\pgfqpoint{1.154777in}{2.553385in}}%
\pgfpathcurveto{\pgfqpoint{1.154777in}{2.564435in}}{\pgfqpoint{1.150387in}{2.575034in}}{\pgfqpoint{1.142573in}{2.582848in}}%
\pgfpathcurveto{\pgfqpoint{1.134760in}{2.590662in}}{\pgfqpoint{1.124161in}{2.595052in}}{\pgfqpoint{1.113110in}{2.595052in}}%
\pgfpathcurveto{\pgfqpoint{1.102060in}{2.595052in}}{\pgfqpoint{1.091461in}{2.590662in}}{\pgfqpoint{1.083648in}{2.582848in}}%
\pgfpathcurveto{\pgfqpoint{1.075834in}{2.575034in}}{\pgfqpoint{1.071444in}{2.564435in}}{\pgfqpoint{1.071444in}{2.553385in}}%
\pgfpathcurveto{\pgfqpoint{1.071444in}{2.542335in}}{\pgfqpoint{1.075834in}{2.531736in}}{\pgfqpoint{1.083648in}{2.523923in}}%
\pgfpathcurveto{\pgfqpoint{1.091461in}{2.516109in}}{\pgfqpoint{1.102060in}{2.511719in}}{\pgfqpoint{1.113110in}{2.511719in}}%
\pgfpathlineto{\pgfqpoint{1.113110in}{2.511719in}}%
\pgfpathclose%
\pgfusepath{stroke}%
\end{pgfscope}%
\begin{pgfscope}%
\pgfpathrectangle{\pgfqpoint{0.494722in}{0.437222in}}{\pgfqpoint{6.275590in}{5.159444in}}%
\pgfusepath{clip}%
\pgfsetbuttcap%
\pgfsetroundjoin%
\pgfsetlinewidth{1.003750pt}%
\definecolor{currentstroke}{rgb}{0.827451,0.827451,0.827451}%
\pgfsetstrokecolor{currentstroke}%
\pgfsetstrokeopacity{0.800000}%
\pgfsetdash{}{0pt}%
\pgfpathmoveto{\pgfqpoint{1.006303in}{2.625866in}}%
\pgfpathcurveto{\pgfqpoint{1.017353in}{2.625866in}}{\pgfqpoint{1.027952in}{2.630257in}}{\pgfqpoint{1.035766in}{2.638070in}}%
\pgfpathcurveto{\pgfqpoint{1.043580in}{2.645884in}}{\pgfqpoint{1.047970in}{2.656483in}}{\pgfqpoint{1.047970in}{2.667533in}}%
\pgfpathcurveto{\pgfqpoint{1.047970in}{2.678583in}}{\pgfqpoint{1.043580in}{2.689182in}}{\pgfqpoint{1.035766in}{2.696996in}}%
\pgfpathcurveto{\pgfqpoint{1.027952in}{2.704810in}}{\pgfqpoint{1.017353in}{2.709200in}}{\pgfqpoint{1.006303in}{2.709200in}}%
\pgfpathcurveto{\pgfqpoint{0.995253in}{2.709200in}}{\pgfqpoint{0.984654in}{2.704810in}}{\pgfqpoint{0.976840in}{2.696996in}}%
\pgfpathcurveto{\pgfqpoint{0.969027in}{2.689182in}}{\pgfqpoint{0.964636in}{2.678583in}}{\pgfqpoint{0.964636in}{2.667533in}}%
\pgfpathcurveto{\pgfqpoint{0.964636in}{2.656483in}}{\pgfqpoint{0.969027in}{2.645884in}}{\pgfqpoint{0.976840in}{2.638070in}}%
\pgfpathcurveto{\pgfqpoint{0.984654in}{2.630257in}}{\pgfqpoint{0.995253in}{2.625866in}}{\pgfqpoint{1.006303in}{2.625866in}}%
\pgfpathlineto{\pgfqpoint{1.006303in}{2.625866in}}%
\pgfpathclose%
\pgfusepath{stroke}%
\end{pgfscope}%
\begin{pgfscope}%
\pgfpathrectangle{\pgfqpoint{0.494722in}{0.437222in}}{\pgfqpoint{6.275590in}{5.159444in}}%
\pgfusepath{clip}%
\pgfsetbuttcap%
\pgfsetroundjoin%
\pgfsetlinewidth{1.003750pt}%
\definecolor{currentstroke}{rgb}{0.827451,0.827451,0.827451}%
\pgfsetstrokecolor{currentstroke}%
\pgfsetstrokeopacity{0.800000}%
\pgfsetdash{}{0pt}%
\pgfpathmoveto{\pgfqpoint{0.513968in}{4.235716in}}%
\pgfpathcurveto{\pgfqpoint{0.525018in}{4.235716in}}{\pgfqpoint{0.535617in}{4.240106in}}{\pgfqpoint{0.543431in}{4.247920in}}%
\pgfpathcurveto{\pgfqpoint{0.551244in}{4.255733in}}{\pgfqpoint{0.555635in}{4.266332in}}{\pgfqpoint{0.555635in}{4.277382in}}%
\pgfpathcurveto{\pgfqpoint{0.555635in}{4.288433in}}{\pgfqpoint{0.551244in}{4.299032in}}{\pgfqpoint{0.543431in}{4.306845in}}%
\pgfpathcurveto{\pgfqpoint{0.535617in}{4.314659in}}{\pgfqpoint{0.525018in}{4.319049in}}{\pgfqpoint{0.513968in}{4.319049in}}%
\pgfpathcurveto{\pgfqpoint{0.502918in}{4.319049in}}{\pgfqpoint{0.492319in}{4.314659in}}{\pgfqpoint{0.484505in}{4.306845in}}%
\pgfpathcurveto{\pgfqpoint{0.476691in}{4.299032in}}{\pgfqpoint{0.472301in}{4.288433in}}{\pgfqpoint{0.472301in}{4.277382in}}%
\pgfpathcurveto{\pgfqpoint{0.472301in}{4.266332in}}{\pgfqpoint{0.476691in}{4.255733in}}{\pgfqpoint{0.484505in}{4.247920in}}%
\pgfpathcurveto{\pgfqpoint{0.492319in}{4.240106in}}{\pgfqpoint{0.502918in}{4.235716in}}{\pgfqpoint{0.513968in}{4.235716in}}%
\pgfpathlineto{\pgfqpoint{0.513968in}{4.235716in}}%
\pgfpathclose%
\pgfusepath{stroke}%
\end{pgfscope}%
\begin{pgfscope}%
\pgfpathrectangle{\pgfqpoint{0.494722in}{0.437222in}}{\pgfqpoint{6.275590in}{5.159444in}}%
\pgfusepath{clip}%
\pgfsetbuttcap%
\pgfsetroundjoin%
\pgfsetlinewidth{1.003750pt}%
\definecolor{currentstroke}{rgb}{0.827451,0.827451,0.827451}%
\pgfsetstrokecolor{currentstroke}%
\pgfsetstrokeopacity{0.800000}%
\pgfsetdash{}{0pt}%
\pgfpathmoveto{\pgfqpoint{1.133504in}{2.415631in}}%
\pgfpathcurveto{\pgfqpoint{1.144554in}{2.415631in}}{\pgfqpoint{1.155153in}{2.420021in}}{\pgfqpoint{1.162967in}{2.427835in}}%
\pgfpathcurveto{\pgfqpoint{1.170780in}{2.435648in}}{\pgfqpoint{1.175170in}{2.446247in}}{\pgfqpoint{1.175170in}{2.457297in}}%
\pgfpathcurveto{\pgfqpoint{1.175170in}{2.468347in}}{\pgfqpoint{1.170780in}{2.478946in}}{\pgfqpoint{1.162967in}{2.486760in}}%
\pgfpathcurveto{\pgfqpoint{1.155153in}{2.494574in}}{\pgfqpoint{1.144554in}{2.498964in}}{\pgfqpoint{1.133504in}{2.498964in}}%
\pgfpathcurveto{\pgfqpoint{1.122454in}{2.498964in}}{\pgfqpoint{1.111855in}{2.494574in}}{\pgfqpoint{1.104041in}{2.486760in}}%
\pgfpathcurveto{\pgfqpoint{1.096227in}{2.478946in}}{\pgfqpoint{1.091837in}{2.468347in}}{\pgfqpoint{1.091837in}{2.457297in}}%
\pgfpathcurveto{\pgfqpoint{1.091837in}{2.446247in}}{\pgfqpoint{1.096227in}{2.435648in}}{\pgfqpoint{1.104041in}{2.427835in}}%
\pgfpathcurveto{\pgfqpoint{1.111855in}{2.420021in}}{\pgfqpoint{1.122454in}{2.415631in}}{\pgfqpoint{1.133504in}{2.415631in}}%
\pgfpathlineto{\pgfqpoint{1.133504in}{2.415631in}}%
\pgfpathclose%
\pgfusepath{stroke}%
\end{pgfscope}%
\begin{pgfscope}%
\pgfpathrectangle{\pgfqpoint{0.494722in}{0.437222in}}{\pgfqpoint{6.275590in}{5.159444in}}%
\pgfusepath{clip}%
\pgfsetbuttcap%
\pgfsetroundjoin%
\pgfsetlinewidth{1.003750pt}%
\definecolor{currentstroke}{rgb}{0.827451,0.827451,0.827451}%
\pgfsetstrokecolor{currentstroke}%
\pgfsetstrokeopacity{0.800000}%
\pgfsetdash{}{0pt}%
\pgfpathmoveto{\pgfqpoint{2.081578in}{1.430058in}}%
\pgfpathcurveto{\pgfqpoint{2.092628in}{1.430058in}}{\pgfqpoint{2.103227in}{1.434449in}}{\pgfqpoint{2.111040in}{1.442262in}}%
\pgfpathcurveto{\pgfqpoint{2.118854in}{1.450076in}}{\pgfqpoint{2.123244in}{1.460675in}}{\pgfqpoint{2.123244in}{1.471725in}}%
\pgfpathcurveto{\pgfqpoint{2.123244in}{1.482775in}}{\pgfqpoint{2.118854in}{1.493374in}}{\pgfqpoint{2.111040in}{1.501188in}}%
\pgfpathcurveto{\pgfqpoint{2.103227in}{1.509001in}}{\pgfqpoint{2.092628in}{1.513392in}}{\pgfqpoint{2.081578in}{1.513392in}}%
\pgfpathcurveto{\pgfqpoint{2.070527in}{1.513392in}}{\pgfqpoint{2.059928in}{1.509001in}}{\pgfqpoint{2.052115in}{1.501188in}}%
\pgfpathcurveto{\pgfqpoint{2.044301in}{1.493374in}}{\pgfqpoint{2.039911in}{1.482775in}}{\pgfqpoint{2.039911in}{1.471725in}}%
\pgfpathcurveto{\pgfqpoint{2.039911in}{1.460675in}}{\pgfqpoint{2.044301in}{1.450076in}}{\pgfqpoint{2.052115in}{1.442262in}}%
\pgfpathcurveto{\pgfqpoint{2.059928in}{1.434449in}}{\pgfqpoint{2.070527in}{1.430058in}}{\pgfqpoint{2.081578in}{1.430058in}}%
\pgfpathlineto{\pgfqpoint{2.081578in}{1.430058in}}%
\pgfpathclose%
\pgfusepath{stroke}%
\end{pgfscope}%
\begin{pgfscope}%
\pgfpathrectangle{\pgfqpoint{0.494722in}{0.437222in}}{\pgfqpoint{6.275590in}{5.159444in}}%
\pgfusepath{clip}%
\pgfsetbuttcap%
\pgfsetroundjoin%
\pgfsetlinewidth{1.003750pt}%
\definecolor{currentstroke}{rgb}{0.827451,0.827451,0.827451}%
\pgfsetstrokecolor{currentstroke}%
\pgfsetstrokeopacity{0.800000}%
\pgfsetdash{}{0pt}%
\pgfpathmoveto{\pgfqpoint{2.308193in}{1.269247in}}%
\pgfpathcurveto{\pgfqpoint{2.319243in}{1.269247in}}{\pgfqpoint{2.329842in}{1.273637in}}{\pgfqpoint{2.337656in}{1.281451in}}%
\pgfpathcurveto{\pgfqpoint{2.345469in}{1.289264in}}{\pgfqpoint{2.349860in}{1.299864in}}{\pgfqpoint{2.349860in}{1.310914in}}%
\pgfpathcurveto{\pgfqpoint{2.349860in}{1.321964in}}{\pgfqpoint{2.345469in}{1.332563in}}{\pgfqpoint{2.337656in}{1.340376in}}%
\pgfpathcurveto{\pgfqpoint{2.329842in}{1.348190in}}{\pgfqpoint{2.319243in}{1.352580in}}{\pgfqpoint{2.308193in}{1.352580in}}%
\pgfpathcurveto{\pgfqpoint{2.297143in}{1.352580in}}{\pgfqpoint{2.286544in}{1.348190in}}{\pgfqpoint{2.278730in}{1.340376in}}%
\pgfpathcurveto{\pgfqpoint{2.270917in}{1.332563in}}{\pgfqpoint{2.266526in}{1.321964in}}{\pgfqpoint{2.266526in}{1.310914in}}%
\pgfpathcurveto{\pgfqpoint{2.266526in}{1.299864in}}{\pgfqpoint{2.270917in}{1.289264in}}{\pgfqpoint{2.278730in}{1.281451in}}%
\pgfpathcurveto{\pgfqpoint{2.286544in}{1.273637in}}{\pgfqpoint{2.297143in}{1.269247in}}{\pgfqpoint{2.308193in}{1.269247in}}%
\pgfpathlineto{\pgfqpoint{2.308193in}{1.269247in}}%
\pgfpathclose%
\pgfusepath{stroke}%
\end{pgfscope}%
\begin{pgfscope}%
\pgfpathrectangle{\pgfqpoint{0.494722in}{0.437222in}}{\pgfqpoint{6.275590in}{5.159444in}}%
\pgfusepath{clip}%
\pgfsetbuttcap%
\pgfsetroundjoin%
\pgfsetlinewidth{1.003750pt}%
\definecolor{currentstroke}{rgb}{0.827451,0.827451,0.827451}%
\pgfsetstrokecolor{currentstroke}%
\pgfsetstrokeopacity{0.800000}%
\pgfsetdash{}{0pt}%
\pgfpathmoveto{\pgfqpoint{1.743973in}{1.699418in}}%
\pgfpathcurveto{\pgfqpoint{1.755023in}{1.699418in}}{\pgfqpoint{1.765622in}{1.703809in}}{\pgfqpoint{1.773436in}{1.711622in}}%
\pgfpathcurveto{\pgfqpoint{1.781250in}{1.719436in}}{\pgfqpoint{1.785640in}{1.730035in}}{\pgfqpoint{1.785640in}{1.741085in}}%
\pgfpathcurveto{\pgfqpoint{1.785640in}{1.752135in}}{\pgfqpoint{1.781250in}{1.762734in}}{\pgfqpoint{1.773436in}{1.770548in}}%
\pgfpathcurveto{\pgfqpoint{1.765622in}{1.778361in}}{\pgfqpoint{1.755023in}{1.782752in}}{\pgfqpoint{1.743973in}{1.782752in}}%
\pgfpathcurveto{\pgfqpoint{1.732923in}{1.782752in}}{\pgfqpoint{1.722324in}{1.778361in}}{\pgfqpoint{1.714511in}{1.770548in}}%
\pgfpathcurveto{\pgfqpoint{1.706697in}{1.762734in}}{\pgfqpoint{1.702307in}{1.752135in}}{\pgfqpoint{1.702307in}{1.741085in}}%
\pgfpathcurveto{\pgfqpoint{1.702307in}{1.730035in}}{\pgfqpoint{1.706697in}{1.719436in}}{\pgfqpoint{1.714511in}{1.711622in}}%
\pgfpathcurveto{\pgfqpoint{1.722324in}{1.703809in}}{\pgfqpoint{1.732923in}{1.699418in}}{\pgfqpoint{1.743973in}{1.699418in}}%
\pgfpathlineto{\pgfqpoint{1.743973in}{1.699418in}}%
\pgfpathclose%
\pgfusepath{stroke}%
\end{pgfscope}%
\begin{pgfscope}%
\pgfpathrectangle{\pgfqpoint{0.494722in}{0.437222in}}{\pgfqpoint{6.275590in}{5.159444in}}%
\pgfusepath{clip}%
\pgfsetbuttcap%
\pgfsetroundjoin%
\pgfsetlinewidth{1.003750pt}%
\definecolor{currentstroke}{rgb}{0.827451,0.827451,0.827451}%
\pgfsetstrokecolor{currentstroke}%
\pgfsetstrokeopacity{0.800000}%
\pgfsetdash{}{0pt}%
\pgfpathmoveto{\pgfqpoint{3.222912in}{0.810111in}}%
\pgfpathcurveto{\pgfqpoint{3.233962in}{0.810111in}}{\pgfqpoint{3.244561in}{0.814501in}}{\pgfqpoint{3.252375in}{0.822314in}}%
\pgfpathcurveto{\pgfqpoint{3.260189in}{0.830128in}}{\pgfqpoint{3.264579in}{0.840727in}}{\pgfqpoint{3.264579in}{0.851777in}}%
\pgfpathcurveto{\pgfqpoint{3.264579in}{0.862827in}}{\pgfqpoint{3.260189in}{0.873426in}}{\pgfqpoint{3.252375in}{0.881240in}}%
\pgfpathcurveto{\pgfqpoint{3.244561in}{0.889054in}}{\pgfqpoint{3.233962in}{0.893444in}}{\pgfqpoint{3.222912in}{0.893444in}}%
\pgfpathcurveto{\pgfqpoint{3.211862in}{0.893444in}}{\pgfqpoint{3.201263in}{0.889054in}}{\pgfqpoint{3.193449in}{0.881240in}}%
\pgfpathcurveto{\pgfqpoint{3.185636in}{0.873426in}}{\pgfqpoint{3.181246in}{0.862827in}}{\pgfqpoint{3.181246in}{0.851777in}}%
\pgfpathcurveto{\pgfqpoint{3.181246in}{0.840727in}}{\pgfqpoint{3.185636in}{0.830128in}}{\pgfqpoint{3.193449in}{0.822314in}}%
\pgfpathcurveto{\pgfqpoint{3.201263in}{0.814501in}}{\pgfqpoint{3.211862in}{0.810111in}}{\pgfqpoint{3.222912in}{0.810111in}}%
\pgfpathlineto{\pgfqpoint{3.222912in}{0.810111in}}%
\pgfpathclose%
\pgfusepath{stroke}%
\end{pgfscope}%
\begin{pgfscope}%
\pgfpathrectangle{\pgfqpoint{0.494722in}{0.437222in}}{\pgfqpoint{6.275590in}{5.159444in}}%
\pgfusepath{clip}%
\pgfsetbuttcap%
\pgfsetroundjoin%
\pgfsetlinewidth{1.003750pt}%
\definecolor{currentstroke}{rgb}{0.827451,0.827451,0.827451}%
\pgfsetstrokecolor{currentstroke}%
\pgfsetstrokeopacity{0.800000}%
\pgfsetdash{}{0pt}%
\pgfpathmoveto{\pgfqpoint{0.805070in}{3.017769in}}%
\pgfpathcurveto{\pgfqpoint{0.816120in}{3.017769in}}{\pgfqpoint{0.826719in}{3.022159in}}{\pgfqpoint{0.834533in}{3.029973in}}%
\pgfpathcurveto{\pgfqpoint{0.842347in}{3.037786in}}{\pgfqpoint{0.846737in}{3.048385in}}{\pgfqpoint{0.846737in}{3.059436in}}%
\pgfpathcurveto{\pgfqpoint{0.846737in}{3.070486in}}{\pgfqpoint{0.842347in}{3.081085in}}{\pgfqpoint{0.834533in}{3.088898in}}%
\pgfpathcurveto{\pgfqpoint{0.826719in}{3.096712in}}{\pgfqpoint{0.816120in}{3.101102in}}{\pgfqpoint{0.805070in}{3.101102in}}%
\pgfpathcurveto{\pgfqpoint{0.794020in}{3.101102in}}{\pgfqpoint{0.783421in}{3.096712in}}{\pgfqpoint{0.775608in}{3.088898in}}%
\pgfpathcurveto{\pgfqpoint{0.767794in}{3.081085in}}{\pgfqpoint{0.763404in}{3.070486in}}{\pgfqpoint{0.763404in}{3.059436in}}%
\pgfpathcurveto{\pgfqpoint{0.763404in}{3.048385in}}{\pgfqpoint{0.767794in}{3.037786in}}{\pgfqpoint{0.775608in}{3.029973in}}%
\pgfpathcurveto{\pgfqpoint{0.783421in}{3.022159in}}{\pgfqpoint{0.794020in}{3.017769in}}{\pgfqpoint{0.805070in}{3.017769in}}%
\pgfpathlineto{\pgfqpoint{0.805070in}{3.017769in}}%
\pgfpathclose%
\pgfusepath{stroke}%
\end{pgfscope}%
\begin{pgfscope}%
\pgfpathrectangle{\pgfqpoint{0.494722in}{0.437222in}}{\pgfqpoint{6.275590in}{5.159444in}}%
\pgfusepath{clip}%
\pgfsetbuttcap%
\pgfsetroundjoin%
\pgfsetlinewidth{1.003750pt}%
\definecolor{currentstroke}{rgb}{0.827451,0.827451,0.827451}%
\pgfsetstrokecolor{currentstroke}%
\pgfsetstrokeopacity{0.800000}%
\pgfsetdash{}{0pt}%
\pgfpathmoveto{\pgfqpoint{1.572883in}{1.857148in}}%
\pgfpathcurveto{\pgfqpoint{1.583933in}{1.857148in}}{\pgfqpoint{1.594532in}{1.861538in}}{\pgfqpoint{1.602345in}{1.869352in}}%
\pgfpathcurveto{\pgfqpoint{1.610159in}{1.877166in}}{\pgfqpoint{1.614549in}{1.887765in}}{\pgfqpoint{1.614549in}{1.898815in}}%
\pgfpathcurveto{\pgfqpoint{1.614549in}{1.909865in}}{\pgfqpoint{1.610159in}{1.920464in}}{\pgfqpoint{1.602345in}{1.928278in}}%
\pgfpathcurveto{\pgfqpoint{1.594532in}{1.936091in}}{\pgfqpoint{1.583933in}{1.940481in}}{\pgfqpoint{1.572883in}{1.940481in}}%
\pgfpathcurveto{\pgfqpoint{1.561832in}{1.940481in}}{\pgfqpoint{1.551233in}{1.936091in}}{\pgfqpoint{1.543420in}{1.928278in}}%
\pgfpathcurveto{\pgfqpoint{1.535606in}{1.920464in}}{\pgfqpoint{1.531216in}{1.909865in}}{\pgfqpoint{1.531216in}{1.898815in}}%
\pgfpathcurveto{\pgfqpoint{1.531216in}{1.887765in}}{\pgfqpoint{1.535606in}{1.877166in}}{\pgfqpoint{1.543420in}{1.869352in}}%
\pgfpathcurveto{\pgfqpoint{1.551233in}{1.861538in}}{\pgfqpoint{1.561832in}{1.857148in}}{\pgfqpoint{1.572883in}{1.857148in}}%
\pgfpathlineto{\pgfqpoint{1.572883in}{1.857148in}}%
\pgfpathclose%
\pgfusepath{stroke}%
\end{pgfscope}%
\begin{pgfscope}%
\pgfpathrectangle{\pgfqpoint{0.494722in}{0.437222in}}{\pgfqpoint{6.275590in}{5.159444in}}%
\pgfusepath{clip}%
\pgfsetbuttcap%
\pgfsetroundjoin%
\pgfsetlinewidth{1.003750pt}%
\definecolor{currentstroke}{rgb}{0.827451,0.827451,0.827451}%
\pgfsetstrokecolor{currentstroke}%
\pgfsetstrokeopacity{0.800000}%
\pgfsetdash{}{0pt}%
\pgfpathmoveto{\pgfqpoint{0.584355in}{3.716646in}}%
\pgfpathcurveto{\pgfqpoint{0.595406in}{3.716646in}}{\pgfqpoint{0.606005in}{3.721036in}}{\pgfqpoint{0.613818in}{3.728850in}}%
\pgfpathcurveto{\pgfqpoint{0.621632in}{3.736663in}}{\pgfqpoint{0.626022in}{3.747262in}}{\pgfqpoint{0.626022in}{3.758313in}}%
\pgfpathcurveto{\pgfqpoint{0.626022in}{3.769363in}}{\pgfqpoint{0.621632in}{3.779962in}}{\pgfqpoint{0.613818in}{3.787775in}}%
\pgfpathcurveto{\pgfqpoint{0.606005in}{3.795589in}}{\pgfqpoint{0.595406in}{3.799979in}}{\pgfqpoint{0.584355in}{3.799979in}}%
\pgfpathcurveto{\pgfqpoint{0.573305in}{3.799979in}}{\pgfqpoint{0.562706in}{3.795589in}}{\pgfqpoint{0.554893in}{3.787775in}}%
\pgfpathcurveto{\pgfqpoint{0.547079in}{3.779962in}}{\pgfqpoint{0.542689in}{3.769363in}}{\pgfqpoint{0.542689in}{3.758313in}}%
\pgfpathcurveto{\pgfqpoint{0.542689in}{3.747262in}}{\pgfqpoint{0.547079in}{3.736663in}}{\pgfqpoint{0.554893in}{3.728850in}}%
\pgfpathcurveto{\pgfqpoint{0.562706in}{3.721036in}}{\pgfqpoint{0.573305in}{3.716646in}}{\pgfqpoint{0.584355in}{3.716646in}}%
\pgfpathlineto{\pgfqpoint{0.584355in}{3.716646in}}%
\pgfpathclose%
\pgfusepath{stroke}%
\end{pgfscope}%
\begin{pgfscope}%
\pgfpathrectangle{\pgfqpoint{0.494722in}{0.437222in}}{\pgfqpoint{6.275590in}{5.159444in}}%
\pgfusepath{clip}%
\pgfsetbuttcap%
\pgfsetroundjoin%
\pgfsetlinewidth{1.003750pt}%
\definecolor{currentstroke}{rgb}{0.827451,0.827451,0.827451}%
\pgfsetstrokecolor{currentstroke}%
\pgfsetstrokeopacity{0.800000}%
\pgfsetdash{}{0pt}%
\pgfpathmoveto{\pgfqpoint{0.711186in}{3.233934in}}%
\pgfpathcurveto{\pgfqpoint{0.722236in}{3.233934in}}{\pgfqpoint{0.732835in}{3.238324in}}{\pgfqpoint{0.740648in}{3.246138in}}%
\pgfpathcurveto{\pgfqpoint{0.748462in}{3.253952in}}{\pgfqpoint{0.752852in}{3.264551in}}{\pgfqpoint{0.752852in}{3.275601in}}%
\pgfpathcurveto{\pgfqpoint{0.752852in}{3.286651in}}{\pgfqpoint{0.748462in}{3.297250in}}{\pgfqpoint{0.740648in}{3.305064in}}%
\pgfpathcurveto{\pgfqpoint{0.732835in}{3.312877in}}{\pgfqpoint{0.722236in}{3.317267in}}{\pgfqpoint{0.711186in}{3.317267in}}%
\pgfpathcurveto{\pgfqpoint{0.700135in}{3.317267in}}{\pgfqpoint{0.689536in}{3.312877in}}{\pgfqpoint{0.681723in}{3.305064in}}%
\pgfpathcurveto{\pgfqpoint{0.673909in}{3.297250in}}{\pgfqpoint{0.669519in}{3.286651in}}{\pgfqpoint{0.669519in}{3.275601in}}%
\pgfpathcurveto{\pgfqpoint{0.669519in}{3.264551in}}{\pgfqpoint{0.673909in}{3.253952in}}{\pgfqpoint{0.681723in}{3.246138in}}%
\pgfpathcurveto{\pgfqpoint{0.689536in}{3.238324in}}{\pgfqpoint{0.700135in}{3.233934in}}{\pgfqpoint{0.711186in}{3.233934in}}%
\pgfpathlineto{\pgfqpoint{0.711186in}{3.233934in}}%
\pgfpathclose%
\pgfusepath{stroke}%
\end{pgfscope}%
\begin{pgfscope}%
\pgfpathrectangle{\pgfqpoint{0.494722in}{0.437222in}}{\pgfqpoint{6.275590in}{5.159444in}}%
\pgfusepath{clip}%
\pgfsetbuttcap%
\pgfsetroundjoin%
\pgfsetlinewidth{1.003750pt}%
\definecolor{currentstroke}{rgb}{0.827451,0.827451,0.827451}%
\pgfsetstrokecolor{currentstroke}%
\pgfsetstrokeopacity{0.800000}%
\pgfsetdash{}{0pt}%
\pgfpathmoveto{\pgfqpoint{4.702565in}{0.471432in}}%
\pgfpathcurveto{\pgfqpoint{4.713615in}{0.471432in}}{\pgfqpoint{4.724214in}{0.475823in}}{\pgfqpoint{4.732028in}{0.483636in}}%
\pgfpathcurveto{\pgfqpoint{4.739841in}{0.491450in}}{\pgfqpoint{4.744232in}{0.502049in}}{\pgfqpoint{4.744232in}{0.513099in}}%
\pgfpathcurveto{\pgfqpoint{4.744232in}{0.524149in}}{\pgfqpoint{4.739841in}{0.534748in}}{\pgfqpoint{4.732028in}{0.542562in}}%
\pgfpathcurveto{\pgfqpoint{4.724214in}{0.550376in}}{\pgfqpoint{4.713615in}{0.554766in}}{\pgfqpoint{4.702565in}{0.554766in}}%
\pgfpathcurveto{\pgfqpoint{4.691515in}{0.554766in}}{\pgfqpoint{4.680916in}{0.550376in}}{\pgfqpoint{4.673102in}{0.542562in}}%
\pgfpathcurveto{\pgfqpoint{4.665289in}{0.534748in}}{\pgfqpoint{4.660898in}{0.524149in}}{\pgfqpoint{4.660898in}{0.513099in}}%
\pgfpathcurveto{\pgfqpoint{4.660898in}{0.502049in}}{\pgfqpoint{4.665289in}{0.491450in}}{\pgfqpoint{4.673102in}{0.483636in}}%
\pgfpathcurveto{\pgfqpoint{4.680916in}{0.475823in}}{\pgfqpoint{4.691515in}{0.471432in}}{\pgfqpoint{4.702565in}{0.471432in}}%
\pgfpathlineto{\pgfqpoint{4.702565in}{0.471432in}}%
\pgfpathclose%
\pgfusepath{stroke}%
\end{pgfscope}%
\begin{pgfscope}%
\pgfpathrectangle{\pgfqpoint{0.494722in}{0.437222in}}{\pgfqpoint{6.275590in}{5.159444in}}%
\pgfusepath{clip}%
\pgfsetbuttcap%
\pgfsetroundjoin%
\pgfsetlinewidth{1.003750pt}%
\definecolor{currentstroke}{rgb}{0.827451,0.827451,0.827451}%
\pgfsetstrokecolor{currentstroke}%
\pgfsetstrokeopacity{0.800000}%
\pgfsetdash{}{0pt}%
\pgfpathmoveto{\pgfqpoint{0.512694in}{4.284212in}}%
\pgfpathcurveto{\pgfqpoint{0.523745in}{4.284212in}}{\pgfqpoint{0.534344in}{4.288603in}}{\pgfqpoint{0.542157in}{4.296416in}}%
\pgfpathcurveto{\pgfqpoint{0.549971in}{4.304230in}}{\pgfqpoint{0.554361in}{4.314829in}}{\pgfqpoint{0.554361in}{4.325879in}}%
\pgfpathcurveto{\pgfqpoint{0.554361in}{4.336929in}}{\pgfqpoint{0.549971in}{4.347528in}}{\pgfqpoint{0.542157in}{4.355342in}}%
\pgfpathcurveto{\pgfqpoint{0.534344in}{4.363155in}}{\pgfqpoint{0.523745in}{4.367546in}}{\pgfqpoint{0.512694in}{4.367546in}}%
\pgfpathcurveto{\pgfqpoint{0.501644in}{4.367546in}}{\pgfqpoint{0.491045in}{4.363155in}}{\pgfqpoint{0.483232in}{4.355342in}}%
\pgfpathcurveto{\pgfqpoint{0.475418in}{4.347528in}}{\pgfqpoint{0.471028in}{4.336929in}}{\pgfqpoint{0.471028in}{4.325879in}}%
\pgfpathcurveto{\pgfqpoint{0.471028in}{4.314829in}}{\pgfqpoint{0.475418in}{4.304230in}}{\pgfqpoint{0.483232in}{4.296416in}}%
\pgfpathcurveto{\pgfqpoint{0.491045in}{4.288603in}}{\pgfqpoint{0.501644in}{4.284212in}}{\pgfqpoint{0.512694in}{4.284212in}}%
\pgfpathlineto{\pgfqpoint{0.512694in}{4.284212in}}%
\pgfpathclose%
\pgfusepath{stroke}%
\end{pgfscope}%
\begin{pgfscope}%
\pgfpathrectangle{\pgfqpoint{0.494722in}{0.437222in}}{\pgfqpoint{6.275590in}{5.159444in}}%
\pgfusepath{clip}%
\pgfsetbuttcap%
\pgfsetroundjoin%
\pgfsetlinewidth{1.003750pt}%
\definecolor{currentstroke}{rgb}{0.827451,0.827451,0.827451}%
\pgfsetstrokecolor{currentstroke}%
\pgfsetstrokeopacity{0.800000}%
\pgfsetdash{}{0pt}%
\pgfpathmoveto{\pgfqpoint{0.560756in}{3.841058in}}%
\pgfpathcurveto{\pgfqpoint{0.571806in}{3.841058in}}{\pgfqpoint{0.582405in}{3.845449in}}{\pgfqpoint{0.590219in}{3.853262in}}%
\pgfpathcurveto{\pgfqpoint{0.598032in}{3.861076in}}{\pgfqpoint{0.602423in}{3.871675in}}{\pgfqpoint{0.602423in}{3.882725in}}%
\pgfpathcurveto{\pgfqpoint{0.602423in}{3.893775in}}{\pgfqpoint{0.598032in}{3.904374in}}{\pgfqpoint{0.590219in}{3.912188in}}%
\pgfpathcurveto{\pgfqpoint{0.582405in}{3.920001in}}{\pgfqpoint{0.571806in}{3.924392in}}{\pgfqpoint{0.560756in}{3.924392in}}%
\pgfpathcurveto{\pgfqpoint{0.549706in}{3.924392in}}{\pgfqpoint{0.539107in}{3.920001in}}{\pgfqpoint{0.531293in}{3.912188in}}%
\pgfpathcurveto{\pgfqpoint{0.523479in}{3.904374in}}{\pgfqpoint{0.519089in}{3.893775in}}{\pgfqpoint{0.519089in}{3.882725in}}%
\pgfpathcurveto{\pgfqpoint{0.519089in}{3.871675in}}{\pgfqpoint{0.523479in}{3.861076in}}{\pgfqpoint{0.531293in}{3.853262in}}%
\pgfpathcurveto{\pgfqpoint{0.539107in}{3.845449in}}{\pgfqpoint{0.549706in}{3.841058in}}{\pgfqpoint{0.560756in}{3.841058in}}%
\pgfpathlineto{\pgfqpoint{0.560756in}{3.841058in}}%
\pgfpathclose%
\pgfusepath{stroke}%
\end{pgfscope}%
\begin{pgfscope}%
\pgfpathrectangle{\pgfqpoint{0.494722in}{0.437222in}}{\pgfqpoint{6.275590in}{5.159444in}}%
\pgfusepath{clip}%
\pgfsetbuttcap%
\pgfsetroundjoin%
\pgfsetlinewidth{1.003750pt}%
\definecolor{currentstroke}{rgb}{0.827451,0.827451,0.827451}%
\pgfsetstrokecolor{currentstroke}%
\pgfsetstrokeopacity{0.800000}%
\pgfsetdash{}{0pt}%
\pgfpathmoveto{\pgfqpoint{1.444344in}{1.986059in}}%
\pgfpathcurveto{\pgfqpoint{1.455394in}{1.986059in}}{\pgfqpoint{1.465993in}{1.990450in}}{\pgfqpoint{1.473807in}{1.998263in}}%
\pgfpathcurveto{\pgfqpoint{1.481620in}{2.006077in}}{\pgfqpoint{1.486010in}{2.016676in}}{\pgfqpoint{1.486010in}{2.027726in}}%
\pgfpathcurveto{\pgfqpoint{1.486010in}{2.038776in}}{\pgfqpoint{1.481620in}{2.049375in}}{\pgfqpoint{1.473807in}{2.057189in}}%
\pgfpathcurveto{\pgfqpoint{1.465993in}{2.065002in}}{\pgfqpoint{1.455394in}{2.069393in}}{\pgfqpoint{1.444344in}{2.069393in}}%
\pgfpathcurveto{\pgfqpoint{1.433294in}{2.069393in}}{\pgfqpoint{1.422695in}{2.065002in}}{\pgfqpoint{1.414881in}{2.057189in}}%
\pgfpathcurveto{\pgfqpoint{1.407067in}{2.049375in}}{\pgfqpoint{1.402677in}{2.038776in}}{\pgfqpoint{1.402677in}{2.027726in}}%
\pgfpathcurveto{\pgfqpoint{1.402677in}{2.016676in}}{\pgfqpoint{1.407067in}{2.006077in}}{\pgfqpoint{1.414881in}{1.998263in}}%
\pgfpathcurveto{\pgfqpoint{1.422695in}{1.990450in}}{\pgfqpoint{1.433294in}{1.986059in}}{\pgfqpoint{1.444344in}{1.986059in}}%
\pgfpathlineto{\pgfqpoint{1.444344in}{1.986059in}}%
\pgfpathclose%
\pgfusepath{stroke}%
\end{pgfscope}%
\begin{pgfscope}%
\pgfpathrectangle{\pgfqpoint{0.494722in}{0.437222in}}{\pgfqpoint{6.275590in}{5.159444in}}%
\pgfusepath{clip}%
\pgfsetbuttcap%
\pgfsetroundjoin%
\pgfsetlinewidth{1.003750pt}%
\definecolor{currentstroke}{rgb}{0.827451,0.827451,0.827451}%
\pgfsetstrokecolor{currentstroke}%
\pgfsetstrokeopacity{0.800000}%
\pgfsetdash{}{0pt}%
\pgfpathmoveto{\pgfqpoint{0.553084in}{3.907249in}}%
\pgfpathcurveto{\pgfqpoint{0.564134in}{3.907249in}}{\pgfqpoint{0.574733in}{3.911639in}}{\pgfqpoint{0.582547in}{3.919453in}}%
\pgfpathcurveto{\pgfqpoint{0.590360in}{3.927267in}}{\pgfqpoint{0.594750in}{3.937866in}}{\pgfqpoint{0.594750in}{3.948916in}}%
\pgfpathcurveto{\pgfqpoint{0.594750in}{3.959966in}}{\pgfqpoint{0.590360in}{3.970565in}}{\pgfqpoint{0.582547in}{3.978378in}}%
\pgfpathcurveto{\pgfqpoint{0.574733in}{3.986192in}}{\pgfqpoint{0.564134in}{3.990582in}}{\pgfqpoint{0.553084in}{3.990582in}}%
\pgfpathcurveto{\pgfqpoint{0.542034in}{3.990582in}}{\pgfqpoint{0.531435in}{3.986192in}}{\pgfqpoint{0.523621in}{3.978378in}}%
\pgfpathcurveto{\pgfqpoint{0.515807in}{3.970565in}}{\pgfqpoint{0.511417in}{3.959966in}}{\pgfqpoint{0.511417in}{3.948916in}}%
\pgfpathcurveto{\pgfqpoint{0.511417in}{3.937866in}}{\pgfqpoint{0.515807in}{3.927267in}}{\pgfqpoint{0.523621in}{3.919453in}}%
\pgfpathcurveto{\pgfqpoint{0.531435in}{3.911639in}}{\pgfqpoint{0.542034in}{3.907249in}}{\pgfqpoint{0.553084in}{3.907249in}}%
\pgfpathlineto{\pgfqpoint{0.553084in}{3.907249in}}%
\pgfpathclose%
\pgfusepath{stroke}%
\end{pgfscope}%
\begin{pgfscope}%
\pgfpathrectangle{\pgfqpoint{0.494722in}{0.437222in}}{\pgfqpoint{6.275590in}{5.159444in}}%
\pgfusepath{clip}%
\pgfsetbuttcap%
\pgfsetroundjoin%
\pgfsetlinewidth{1.003750pt}%
\definecolor{currentstroke}{rgb}{0.827451,0.827451,0.827451}%
\pgfsetstrokecolor{currentstroke}%
\pgfsetstrokeopacity{0.800000}%
\pgfsetdash{}{0pt}%
\pgfpathmoveto{\pgfqpoint{2.554679in}{1.122957in}}%
\pgfpathcurveto{\pgfqpoint{2.565729in}{1.122957in}}{\pgfqpoint{2.576328in}{1.127347in}}{\pgfqpoint{2.584141in}{1.135161in}}%
\pgfpathcurveto{\pgfqpoint{2.591955in}{1.142974in}}{\pgfqpoint{2.596345in}{1.153573in}}{\pgfqpoint{2.596345in}{1.164624in}}%
\pgfpathcurveto{\pgfqpoint{2.596345in}{1.175674in}}{\pgfqpoint{2.591955in}{1.186273in}}{\pgfqpoint{2.584141in}{1.194086in}}%
\pgfpathcurveto{\pgfqpoint{2.576328in}{1.201900in}}{\pgfqpoint{2.565729in}{1.206290in}}{\pgfqpoint{2.554679in}{1.206290in}}%
\pgfpathcurveto{\pgfqpoint{2.543628in}{1.206290in}}{\pgfqpoint{2.533029in}{1.201900in}}{\pgfqpoint{2.525216in}{1.194086in}}%
\pgfpathcurveto{\pgfqpoint{2.517402in}{1.186273in}}{\pgfqpoint{2.513012in}{1.175674in}}{\pgfqpoint{2.513012in}{1.164624in}}%
\pgfpathcurveto{\pgfqpoint{2.513012in}{1.153573in}}{\pgfqpoint{2.517402in}{1.142974in}}{\pgfqpoint{2.525216in}{1.135161in}}%
\pgfpathcurveto{\pgfqpoint{2.533029in}{1.127347in}}{\pgfqpoint{2.543628in}{1.122957in}}{\pgfqpoint{2.554679in}{1.122957in}}%
\pgfpathlineto{\pgfqpoint{2.554679in}{1.122957in}}%
\pgfpathclose%
\pgfusepath{stroke}%
\end{pgfscope}%
\begin{pgfscope}%
\pgfpathrectangle{\pgfqpoint{0.494722in}{0.437222in}}{\pgfqpoint{6.275590in}{5.159444in}}%
\pgfusepath{clip}%
\pgfsetbuttcap%
\pgfsetroundjoin%
\pgfsetlinewidth{1.003750pt}%
\definecolor{currentstroke}{rgb}{0.827451,0.827451,0.827451}%
\pgfsetstrokecolor{currentstroke}%
\pgfsetstrokeopacity{0.800000}%
\pgfsetdash{}{0pt}%
\pgfpathmoveto{\pgfqpoint{2.785306in}{1.003395in}}%
\pgfpathcurveto{\pgfqpoint{2.796356in}{1.003395in}}{\pgfqpoint{2.806955in}{1.007786in}}{\pgfqpoint{2.814769in}{1.015599in}}%
\pgfpathcurveto{\pgfqpoint{2.822582in}{1.023413in}}{\pgfqpoint{2.826973in}{1.034012in}}{\pgfqpoint{2.826973in}{1.045062in}}%
\pgfpathcurveto{\pgfqpoint{2.826973in}{1.056112in}}{\pgfqpoint{2.822582in}{1.066711in}}{\pgfqpoint{2.814769in}{1.074525in}}%
\pgfpathcurveto{\pgfqpoint{2.806955in}{1.082338in}}{\pgfqpoint{2.796356in}{1.086729in}}{\pgfqpoint{2.785306in}{1.086729in}}%
\pgfpathcurveto{\pgfqpoint{2.774256in}{1.086729in}}{\pgfqpoint{2.763657in}{1.082338in}}{\pgfqpoint{2.755843in}{1.074525in}}%
\pgfpathcurveto{\pgfqpoint{2.748030in}{1.066711in}}{\pgfqpoint{2.743639in}{1.056112in}}{\pgfqpoint{2.743639in}{1.045062in}}%
\pgfpathcurveto{\pgfqpoint{2.743639in}{1.034012in}}{\pgfqpoint{2.748030in}{1.023413in}}{\pgfqpoint{2.755843in}{1.015599in}}%
\pgfpathcurveto{\pgfqpoint{2.763657in}{1.007786in}}{\pgfqpoint{2.774256in}{1.003395in}}{\pgfqpoint{2.785306in}{1.003395in}}%
\pgfpathlineto{\pgfqpoint{2.785306in}{1.003395in}}%
\pgfpathclose%
\pgfusepath{stroke}%
\end{pgfscope}%
\begin{pgfscope}%
\pgfpathrectangle{\pgfqpoint{0.494722in}{0.437222in}}{\pgfqpoint{6.275590in}{5.159444in}}%
\pgfusepath{clip}%
\pgfsetbuttcap%
\pgfsetroundjoin%
\pgfsetlinewidth{1.003750pt}%
\definecolor{currentstroke}{rgb}{0.827451,0.827451,0.827451}%
\pgfsetstrokecolor{currentstroke}%
\pgfsetstrokeopacity{0.800000}%
\pgfsetdash{}{0pt}%
\pgfpathmoveto{\pgfqpoint{0.501219in}{4.373826in}}%
\pgfpathcurveto{\pgfqpoint{0.512269in}{4.373826in}}{\pgfqpoint{0.522868in}{4.378216in}}{\pgfqpoint{0.530682in}{4.386030in}}%
\pgfpathcurveto{\pgfqpoint{0.538495in}{4.393844in}}{\pgfqpoint{0.542886in}{4.404443in}}{\pgfqpoint{0.542886in}{4.415493in}}%
\pgfpathcurveto{\pgfqpoint{0.542886in}{4.426543in}}{\pgfqpoint{0.538495in}{4.437142in}}{\pgfqpoint{0.530682in}{4.444955in}}%
\pgfpathcurveto{\pgfqpoint{0.522868in}{4.452769in}}{\pgfqpoint{0.512269in}{4.457159in}}{\pgfqpoint{0.501219in}{4.457159in}}%
\pgfpathcurveto{\pgfqpoint{0.490169in}{4.457159in}}{\pgfqpoint{0.479570in}{4.452769in}}{\pgfqpoint{0.471756in}{4.444955in}}%
\pgfpathcurveto{\pgfqpoint{0.463943in}{4.437142in}}{\pgfqpoint{0.459552in}{4.426543in}}{\pgfqpoint{0.459552in}{4.415493in}}%
\pgfpathcurveto{\pgfqpoint{0.459552in}{4.404443in}}{\pgfqpoint{0.463943in}{4.393844in}}{\pgfqpoint{0.471756in}{4.386030in}}%
\pgfpathcurveto{\pgfqpoint{0.479570in}{4.378216in}}{\pgfqpoint{0.490169in}{4.373826in}}{\pgfqpoint{0.501219in}{4.373826in}}%
\pgfpathlineto{\pgfqpoint{0.501219in}{4.373826in}}%
\pgfpathclose%
\pgfusepath{stroke}%
\end{pgfscope}%
\begin{pgfscope}%
\pgfpathrectangle{\pgfqpoint{0.494722in}{0.437222in}}{\pgfqpoint{6.275590in}{5.159444in}}%
\pgfusepath{clip}%
\pgfsetbuttcap%
\pgfsetroundjoin%
\pgfsetlinewidth{1.003750pt}%
\definecolor{currentstroke}{rgb}{0.827451,0.827451,0.827451}%
\pgfsetstrokecolor{currentstroke}%
\pgfsetstrokeopacity{0.800000}%
\pgfsetdash{}{0pt}%
\pgfpathmoveto{\pgfqpoint{1.161602in}{2.366586in}}%
\pgfpathcurveto{\pgfqpoint{1.172652in}{2.366586in}}{\pgfqpoint{1.183251in}{2.370976in}}{\pgfqpoint{1.191065in}{2.378790in}}%
\pgfpathcurveto{\pgfqpoint{1.198879in}{2.386603in}}{\pgfqpoint{1.203269in}{2.397202in}}{\pgfqpoint{1.203269in}{2.408252in}}%
\pgfpathcurveto{\pgfqpoint{1.203269in}{2.419303in}}{\pgfqpoint{1.198879in}{2.429902in}}{\pgfqpoint{1.191065in}{2.437715in}}%
\pgfpathcurveto{\pgfqpoint{1.183251in}{2.445529in}}{\pgfqpoint{1.172652in}{2.449919in}}{\pgfqpoint{1.161602in}{2.449919in}}%
\pgfpathcurveto{\pgfqpoint{1.150552in}{2.449919in}}{\pgfqpoint{1.139953in}{2.445529in}}{\pgfqpoint{1.132139in}{2.437715in}}%
\pgfpathcurveto{\pgfqpoint{1.124326in}{2.429902in}}{\pgfqpoint{1.119935in}{2.419303in}}{\pgfqpoint{1.119935in}{2.408252in}}%
\pgfpathcurveto{\pgfqpoint{1.119935in}{2.397202in}}{\pgfqpoint{1.124326in}{2.386603in}}{\pgfqpoint{1.132139in}{2.378790in}}%
\pgfpathcurveto{\pgfqpoint{1.139953in}{2.370976in}}{\pgfqpoint{1.150552in}{2.366586in}}{\pgfqpoint{1.161602in}{2.366586in}}%
\pgfpathlineto{\pgfqpoint{1.161602in}{2.366586in}}%
\pgfpathclose%
\pgfusepath{stroke}%
\end{pgfscope}%
\begin{pgfscope}%
\pgfpathrectangle{\pgfqpoint{0.494722in}{0.437222in}}{\pgfqpoint{6.275590in}{5.159444in}}%
\pgfusepath{clip}%
\pgfsetbuttcap%
\pgfsetroundjoin%
\pgfsetlinewidth{1.003750pt}%
\definecolor{currentstroke}{rgb}{0.827451,0.827451,0.827451}%
\pgfsetstrokecolor{currentstroke}%
\pgfsetstrokeopacity{0.800000}%
\pgfsetdash{}{0pt}%
\pgfpathmoveto{\pgfqpoint{1.623472in}{1.825964in}}%
\pgfpathcurveto{\pgfqpoint{1.634522in}{1.825964in}}{\pgfqpoint{1.645121in}{1.830354in}}{\pgfqpoint{1.652935in}{1.838167in}}%
\pgfpathcurveto{\pgfqpoint{1.660748in}{1.845981in}}{\pgfqpoint{1.665139in}{1.856580in}}{\pgfqpoint{1.665139in}{1.867630in}}%
\pgfpathcurveto{\pgfqpoint{1.665139in}{1.878680in}}{\pgfqpoint{1.660748in}{1.889279in}}{\pgfqpoint{1.652935in}{1.897093in}}%
\pgfpathcurveto{\pgfqpoint{1.645121in}{1.904907in}}{\pgfqpoint{1.634522in}{1.909297in}}{\pgfqpoint{1.623472in}{1.909297in}}%
\pgfpathcurveto{\pgfqpoint{1.612422in}{1.909297in}}{\pgfqpoint{1.601823in}{1.904907in}}{\pgfqpoint{1.594009in}{1.897093in}}%
\pgfpathcurveto{\pgfqpoint{1.586196in}{1.889279in}}{\pgfqpoint{1.581805in}{1.878680in}}{\pgfqpoint{1.581805in}{1.867630in}}%
\pgfpathcurveto{\pgfqpoint{1.581805in}{1.856580in}}{\pgfqpoint{1.586196in}{1.845981in}}{\pgfqpoint{1.594009in}{1.838167in}}%
\pgfpathcurveto{\pgfqpoint{1.601823in}{1.830354in}}{\pgfqpoint{1.612422in}{1.825964in}}{\pgfqpoint{1.623472in}{1.825964in}}%
\pgfpathlineto{\pgfqpoint{1.623472in}{1.825964in}}%
\pgfpathclose%
\pgfusepath{stroke}%
\end{pgfscope}%
\begin{pgfscope}%
\pgfpathrectangle{\pgfqpoint{0.494722in}{0.437222in}}{\pgfqpoint{6.275590in}{5.159444in}}%
\pgfusepath{clip}%
\pgfsetbuttcap%
\pgfsetroundjoin%
\pgfsetlinewidth{1.003750pt}%
\definecolor{currentstroke}{rgb}{0.827451,0.827451,0.827451}%
\pgfsetstrokecolor{currentstroke}%
\pgfsetstrokeopacity{0.800000}%
\pgfsetdash{}{0pt}%
\pgfpathmoveto{\pgfqpoint{2.117156in}{1.399260in}}%
\pgfpathcurveto{\pgfqpoint{2.128206in}{1.399260in}}{\pgfqpoint{2.138805in}{1.403650in}}{\pgfqpoint{2.146618in}{1.411464in}}%
\pgfpathcurveto{\pgfqpoint{2.154432in}{1.419277in}}{\pgfqpoint{2.158822in}{1.429876in}}{\pgfqpoint{2.158822in}{1.440927in}}%
\pgfpathcurveto{\pgfqpoint{2.158822in}{1.451977in}}{\pgfqpoint{2.154432in}{1.462576in}}{\pgfqpoint{2.146618in}{1.470389in}}%
\pgfpathcurveto{\pgfqpoint{2.138805in}{1.478203in}}{\pgfqpoint{2.128206in}{1.482593in}}{\pgfqpoint{2.117156in}{1.482593in}}%
\pgfpathcurveto{\pgfqpoint{2.106106in}{1.482593in}}{\pgfqpoint{2.095506in}{1.478203in}}{\pgfqpoint{2.087693in}{1.470389in}}%
\pgfpathcurveto{\pgfqpoint{2.079879in}{1.462576in}}{\pgfqpoint{2.075489in}{1.451977in}}{\pgfqpoint{2.075489in}{1.440927in}}%
\pgfpathcurveto{\pgfqpoint{2.075489in}{1.429876in}}{\pgfqpoint{2.079879in}{1.419277in}}{\pgfqpoint{2.087693in}{1.411464in}}%
\pgfpathcurveto{\pgfqpoint{2.095506in}{1.403650in}}{\pgfqpoint{2.106106in}{1.399260in}}{\pgfqpoint{2.117156in}{1.399260in}}%
\pgfpathlineto{\pgfqpoint{2.117156in}{1.399260in}}%
\pgfpathclose%
\pgfusepath{stroke}%
\end{pgfscope}%
\begin{pgfscope}%
\pgfpathrectangle{\pgfqpoint{0.494722in}{0.437222in}}{\pgfqpoint{6.275590in}{5.159444in}}%
\pgfusepath{clip}%
\pgfsetbuttcap%
\pgfsetroundjoin%
\pgfsetlinewidth{1.003750pt}%
\definecolor{currentstroke}{rgb}{0.827451,0.827451,0.827451}%
\pgfsetstrokecolor{currentstroke}%
\pgfsetstrokeopacity{0.800000}%
\pgfsetdash{}{0pt}%
\pgfpathmoveto{\pgfqpoint{0.700772in}{3.267894in}}%
\pgfpathcurveto{\pgfqpoint{0.711822in}{3.267894in}}{\pgfqpoint{0.722421in}{3.272284in}}{\pgfqpoint{0.730235in}{3.280098in}}%
\pgfpathcurveto{\pgfqpoint{0.738048in}{3.287911in}}{\pgfqpoint{0.742438in}{3.298510in}}{\pgfqpoint{0.742438in}{3.309561in}}%
\pgfpathcurveto{\pgfqpoint{0.742438in}{3.320611in}}{\pgfqpoint{0.738048in}{3.331210in}}{\pgfqpoint{0.730235in}{3.339023in}}%
\pgfpathcurveto{\pgfqpoint{0.722421in}{3.346837in}}{\pgfqpoint{0.711822in}{3.351227in}}{\pgfqpoint{0.700772in}{3.351227in}}%
\pgfpathcurveto{\pgfqpoint{0.689722in}{3.351227in}}{\pgfqpoint{0.679123in}{3.346837in}}{\pgfqpoint{0.671309in}{3.339023in}}%
\pgfpathcurveto{\pgfqpoint{0.663495in}{3.331210in}}{\pgfqpoint{0.659105in}{3.320611in}}{\pgfqpoint{0.659105in}{3.309561in}}%
\pgfpathcurveto{\pgfqpoint{0.659105in}{3.298510in}}{\pgfqpoint{0.663495in}{3.287911in}}{\pgfqpoint{0.671309in}{3.280098in}}%
\pgfpathcurveto{\pgfqpoint{0.679123in}{3.272284in}}{\pgfqpoint{0.689722in}{3.267894in}}{\pgfqpoint{0.700772in}{3.267894in}}%
\pgfpathlineto{\pgfqpoint{0.700772in}{3.267894in}}%
\pgfpathclose%
\pgfusepath{stroke}%
\end{pgfscope}%
\begin{pgfscope}%
\pgfpathrectangle{\pgfqpoint{0.494722in}{0.437222in}}{\pgfqpoint{6.275590in}{5.159444in}}%
\pgfusepath{clip}%
\pgfsetbuttcap%
\pgfsetroundjoin%
\pgfsetlinewidth{1.003750pt}%
\definecolor{currentstroke}{rgb}{0.827451,0.827451,0.827451}%
\pgfsetstrokecolor{currentstroke}%
\pgfsetstrokeopacity{0.800000}%
\pgfsetdash{}{0pt}%
\pgfpathmoveto{\pgfqpoint{0.495484in}{4.544725in}}%
\pgfpathcurveto{\pgfqpoint{0.506534in}{4.544725in}}{\pgfqpoint{0.517133in}{4.549116in}}{\pgfqpoint{0.524947in}{4.556929in}}%
\pgfpathcurveto{\pgfqpoint{0.532761in}{4.564743in}}{\pgfqpoint{0.537151in}{4.575342in}}{\pgfqpoint{0.537151in}{4.586392in}}%
\pgfpathcurveto{\pgfqpoint{0.537151in}{4.597442in}}{\pgfqpoint{0.532761in}{4.608041in}}{\pgfqpoint{0.524947in}{4.615855in}}%
\pgfpathcurveto{\pgfqpoint{0.517133in}{4.623668in}}{\pgfqpoint{0.506534in}{4.628059in}}{\pgfqpoint{0.495484in}{4.628059in}}%
\pgfpathcurveto{\pgfqpoint{0.484434in}{4.628059in}}{\pgfqpoint{0.473835in}{4.623668in}}{\pgfqpoint{0.466021in}{4.615855in}}%
\pgfpathcurveto{\pgfqpoint{0.458208in}{4.608041in}}{\pgfqpoint{0.453817in}{4.597442in}}{\pgfqpoint{0.453817in}{4.586392in}}%
\pgfpathcurveto{\pgfqpoint{0.453817in}{4.575342in}}{\pgfqpoint{0.458208in}{4.564743in}}{\pgfqpoint{0.466021in}{4.556929in}}%
\pgfpathcurveto{\pgfqpoint{0.473835in}{4.549116in}}{\pgfqpoint{0.484434in}{4.544725in}}{\pgfqpoint{0.495484in}{4.544725in}}%
\pgfpathlineto{\pgfqpoint{0.495484in}{4.544725in}}%
\pgfpathclose%
\pgfusepath{stroke}%
\end{pgfscope}%
\begin{pgfscope}%
\pgfpathrectangle{\pgfqpoint{0.494722in}{0.437222in}}{\pgfqpoint{6.275590in}{5.159444in}}%
\pgfusepath{clip}%
\pgfsetbuttcap%
\pgfsetroundjoin%
\pgfsetlinewidth{1.003750pt}%
\definecolor{currentstroke}{rgb}{0.827451,0.827451,0.827451}%
\pgfsetstrokecolor{currentstroke}%
\pgfsetstrokeopacity{0.800000}%
\pgfsetdash{}{0pt}%
\pgfpathmoveto{\pgfqpoint{1.347485in}{2.096739in}}%
\pgfpathcurveto{\pgfqpoint{1.358535in}{2.096739in}}{\pgfqpoint{1.369135in}{2.101129in}}{\pgfqpoint{1.376948in}{2.108943in}}%
\pgfpathcurveto{\pgfqpoint{1.384762in}{2.116757in}}{\pgfqpoint{1.389152in}{2.127356in}}{\pgfqpoint{1.389152in}{2.138406in}}%
\pgfpathcurveto{\pgfqpoint{1.389152in}{2.149456in}}{\pgfqpoint{1.384762in}{2.160055in}}{\pgfqpoint{1.376948in}{2.167869in}}%
\pgfpathcurveto{\pgfqpoint{1.369135in}{2.175682in}}{\pgfqpoint{1.358535in}{2.180072in}}{\pgfqpoint{1.347485in}{2.180072in}}%
\pgfpathcurveto{\pgfqpoint{1.336435in}{2.180072in}}{\pgfqpoint{1.325836in}{2.175682in}}{\pgfqpoint{1.318023in}{2.167869in}}%
\pgfpathcurveto{\pgfqpoint{1.310209in}{2.160055in}}{\pgfqpoint{1.305819in}{2.149456in}}{\pgfqpoint{1.305819in}{2.138406in}}%
\pgfpathcurveto{\pgfqpoint{1.305819in}{2.127356in}}{\pgfqpoint{1.310209in}{2.116757in}}{\pgfqpoint{1.318023in}{2.108943in}}%
\pgfpathcurveto{\pgfqpoint{1.325836in}{2.101129in}}{\pgfqpoint{1.336435in}{2.096739in}}{\pgfqpoint{1.347485in}{2.096739in}}%
\pgfpathlineto{\pgfqpoint{1.347485in}{2.096739in}}%
\pgfpathclose%
\pgfusepath{stroke}%
\end{pgfscope}%
\begin{pgfscope}%
\pgfpathrectangle{\pgfqpoint{0.494722in}{0.437222in}}{\pgfqpoint{6.275590in}{5.159444in}}%
\pgfusepath{clip}%
\pgfsetbuttcap%
\pgfsetroundjoin%
\pgfsetlinewidth{1.003750pt}%
\definecolor{currentstroke}{rgb}{0.827451,0.827451,0.827451}%
\pgfsetstrokecolor{currentstroke}%
\pgfsetstrokeopacity{0.800000}%
\pgfsetdash{}{0pt}%
\pgfpathmoveto{\pgfqpoint{2.491855in}{1.177667in}}%
\pgfpathcurveto{\pgfqpoint{2.502905in}{1.177667in}}{\pgfqpoint{2.513504in}{1.182057in}}{\pgfqpoint{2.521318in}{1.189871in}}%
\pgfpathcurveto{\pgfqpoint{2.529131in}{1.197684in}}{\pgfqpoint{2.533521in}{1.208283in}}{\pgfqpoint{2.533521in}{1.219333in}}%
\pgfpathcurveto{\pgfqpoint{2.533521in}{1.230384in}}{\pgfqpoint{2.529131in}{1.240983in}}{\pgfqpoint{2.521318in}{1.248796in}}%
\pgfpathcurveto{\pgfqpoint{2.513504in}{1.256610in}}{\pgfqpoint{2.502905in}{1.261000in}}{\pgfqpoint{2.491855in}{1.261000in}}%
\pgfpathcurveto{\pgfqpoint{2.480805in}{1.261000in}}{\pgfqpoint{2.470206in}{1.256610in}}{\pgfqpoint{2.462392in}{1.248796in}}%
\pgfpathcurveto{\pgfqpoint{2.454578in}{1.240983in}}{\pgfqpoint{2.450188in}{1.230384in}}{\pgfqpoint{2.450188in}{1.219333in}}%
\pgfpathcurveto{\pgfqpoint{2.450188in}{1.208283in}}{\pgfqpoint{2.454578in}{1.197684in}}{\pgfqpoint{2.462392in}{1.189871in}}%
\pgfpathcurveto{\pgfqpoint{2.470206in}{1.182057in}}{\pgfqpoint{2.480805in}{1.177667in}}{\pgfqpoint{2.491855in}{1.177667in}}%
\pgfpathlineto{\pgfqpoint{2.491855in}{1.177667in}}%
\pgfpathclose%
\pgfusepath{stroke}%
\end{pgfscope}%
\begin{pgfscope}%
\pgfpathrectangle{\pgfqpoint{0.494722in}{0.437222in}}{\pgfqpoint{6.275590in}{5.159444in}}%
\pgfusepath{clip}%
\pgfsetbuttcap%
\pgfsetroundjoin%
\pgfsetlinewidth{1.003750pt}%
\definecolor{currentstroke}{rgb}{0.827451,0.827451,0.827451}%
\pgfsetstrokecolor{currentstroke}%
\pgfsetstrokeopacity{0.800000}%
\pgfsetdash{}{0pt}%
\pgfpathmoveto{\pgfqpoint{3.942377in}{0.586956in}}%
\pgfpathcurveto{\pgfqpoint{3.953427in}{0.586956in}}{\pgfqpoint{3.964026in}{0.591347in}}{\pgfqpoint{3.971840in}{0.599160in}}%
\pgfpathcurveto{\pgfqpoint{3.979653in}{0.606974in}}{\pgfqpoint{3.984044in}{0.617573in}}{\pgfqpoint{3.984044in}{0.628623in}}%
\pgfpathcurveto{\pgfqpoint{3.984044in}{0.639673in}}{\pgfqpoint{3.979653in}{0.650272in}}{\pgfqpoint{3.971840in}{0.658086in}}%
\pgfpathcurveto{\pgfqpoint{3.964026in}{0.665900in}}{\pgfqpoint{3.953427in}{0.670290in}}{\pgfqpoint{3.942377in}{0.670290in}}%
\pgfpathcurveto{\pgfqpoint{3.931327in}{0.670290in}}{\pgfqpoint{3.920728in}{0.665900in}}{\pgfqpoint{3.912914in}{0.658086in}}%
\pgfpathcurveto{\pgfqpoint{3.905101in}{0.650272in}}{\pgfqpoint{3.900710in}{0.639673in}}{\pgfqpoint{3.900710in}{0.628623in}}%
\pgfpathcurveto{\pgfqpoint{3.900710in}{0.617573in}}{\pgfqpoint{3.905101in}{0.606974in}}{\pgfqpoint{3.912914in}{0.599160in}}%
\pgfpathcurveto{\pgfqpoint{3.920728in}{0.591347in}}{\pgfqpoint{3.931327in}{0.586956in}}{\pgfqpoint{3.942377in}{0.586956in}}%
\pgfpathlineto{\pgfqpoint{3.942377in}{0.586956in}}%
\pgfpathclose%
\pgfusepath{stroke}%
\end{pgfscope}%
\begin{pgfscope}%
\pgfpathrectangle{\pgfqpoint{0.494722in}{0.437222in}}{\pgfqpoint{6.275590in}{5.159444in}}%
\pgfusepath{clip}%
\pgfsetbuttcap%
\pgfsetroundjoin%
\pgfsetlinewidth{1.003750pt}%
\definecolor{currentstroke}{rgb}{0.827451,0.827451,0.827451}%
\pgfsetstrokecolor{currentstroke}%
\pgfsetstrokeopacity{0.800000}%
\pgfsetdash{}{0pt}%
\pgfpathmoveto{\pgfqpoint{4.067519in}{0.560512in}}%
\pgfpathcurveto{\pgfqpoint{4.078569in}{0.560512in}}{\pgfqpoint{4.089168in}{0.564902in}}{\pgfqpoint{4.096982in}{0.572715in}}%
\pgfpathcurveto{\pgfqpoint{4.104796in}{0.580529in}}{\pgfqpoint{4.109186in}{0.591128in}}{\pgfqpoint{4.109186in}{0.602178in}}%
\pgfpathcurveto{\pgfqpoint{4.109186in}{0.613228in}}{\pgfqpoint{4.104796in}{0.623827in}}{\pgfqpoint{4.096982in}{0.631641in}}%
\pgfpathcurveto{\pgfqpoint{4.089168in}{0.639455in}}{\pgfqpoint{4.078569in}{0.643845in}}{\pgfqpoint{4.067519in}{0.643845in}}%
\pgfpathcurveto{\pgfqpoint{4.056469in}{0.643845in}}{\pgfqpoint{4.045870in}{0.639455in}}{\pgfqpoint{4.038056in}{0.631641in}}%
\pgfpathcurveto{\pgfqpoint{4.030243in}{0.623827in}}{\pgfqpoint{4.025853in}{0.613228in}}{\pgfqpoint{4.025853in}{0.602178in}}%
\pgfpathcurveto{\pgfqpoint{4.025853in}{0.591128in}}{\pgfqpoint{4.030243in}{0.580529in}}{\pgfqpoint{4.038056in}{0.572715in}}%
\pgfpathcurveto{\pgfqpoint{4.045870in}{0.564902in}}{\pgfqpoint{4.056469in}{0.560512in}}{\pgfqpoint{4.067519in}{0.560512in}}%
\pgfpathlineto{\pgfqpoint{4.067519in}{0.560512in}}%
\pgfpathclose%
\pgfusepath{stroke}%
\end{pgfscope}%
\begin{pgfscope}%
\pgfpathrectangle{\pgfqpoint{0.494722in}{0.437222in}}{\pgfqpoint{6.275590in}{5.159444in}}%
\pgfusepath{clip}%
\pgfsetbuttcap%
\pgfsetroundjoin%
\pgfsetlinewidth{1.003750pt}%
\definecolor{currentstroke}{rgb}{0.827451,0.827451,0.827451}%
\pgfsetstrokecolor{currentstroke}%
\pgfsetstrokeopacity{0.800000}%
\pgfsetdash{}{0pt}%
\pgfpathmoveto{\pgfqpoint{3.901710in}{0.604104in}}%
\pgfpathcurveto{\pgfqpoint{3.912761in}{0.604104in}}{\pgfqpoint{3.923360in}{0.608495in}}{\pgfqpoint{3.931173in}{0.616308in}}%
\pgfpathcurveto{\pgfqpoint{3.938987in}{0.624122in}}{\pgfqpoint{3.943377in}{0.634721in}}{\pgfqpoint{3.943377in}{0.645771in}}%
\pgfpathcurveto{\pgfqpoint{3.943377in}{0.656821in}}{\pgfqpoint{3.938987in}{0.667420in}}{\pgfqpoint{3.931173in}{0.675234in}}%
\pgfpathcurveto{\pgfqpoint{3.923360in}{0.683047in}}{\pgfqpoint{3.912761in}{0.687438in}}{\pgfqpoint{3.901710in}{0.687438in}}%
\pgfpathcurveto{\pgfqpoint{3.890660in}{0.687438in}}{\pgfqpoint{3.880061in}{0.683047in}}{\pgfqpoint{3.872248in}{0.675234in}}%
\pgfpathcurveto{\pgfqpoint{3.864434in}{0.667420in}}{\pgfqpoint{3.860044in}{0.656821in}}{\pgfqpoint{3.860044in}{0.645771in}}%
\pgfpathcurveto{\pgfqpoint{3.860044in}{0.634721in}}{\pgfqpoint{3.864434in}{0.624122in}}{\pgfqpoint{3.872248in}{0.616308in}}%
\pgfpathcurveto{\pgfqpoint{3.880061in}{0.608495in}}{\pgfqpoint{3.890660in}{0.604104in}}{\pgfqpoint{3.901710in}{0.604104in}}%
\pgfpathlineto{\pgfqpoint{3.901710in}{0.604104in}}%
\pgfpathclose%
\pgfusepath{stroke}%
\end{pgfscope}%
\begin{pgfscope}%
\pgfpathrectangle{\pgfqpoint{0.494722in}{0.437222in}}{\pgfqpoint{6.275590in}{5.159444in}}%
\pgfusepath{clip}%
\pgfsetbuttcap%
\pgfsetroundjoin%
\pgfsetlinewidth{1.003750pt}%
\definecolor{currentstroke}{rgb}{0.827451,0.827451,0.827451}%
\pgfsetstrokecolor{currentstroke}%
\pgfsetstrokeopacity{0.800000}%
\pgfsetdash{}{0pt}%
\pgfpathmoveto{\pgfqpoint{0.575099in}{3.779944in}}%
\pgfpathcurveto{\pgfqpoint{0.586149in}{3.779944in}}{\pgfqpoint{0.596748in}{3.784334in}}{\pgfqpoint{0.604561in}{3.792148in}}%
\pgfpathcurveto{\pgfqpoint{0.612375in}{3.799961in}}{\pgfqpoint{0.616765in}{3.810560in}}{\pgfqpoint{0.616765in}{3.821610in}}%
\pgfpathcurveto{\pgfqpoint{0.616765in}{3.832661in}}{\pgfqpoint{0.612375in}{3.843260in}}{\pgfqpoint{0.604561in}{3.851073in}}%
\pgfpathcurveto{\pgfqpoint{0.596748in}{3.858887in}}{\pgfqpoint{0.586149in}{3.863277in}}{\pgfqpoint{0.575099in}{3.863277in}}%
\pgfpathcurveto{\pgfqpoint{0.564048in}{3.863277in}}{\pgfqpoint{0.553449in}{3.858887in}}{\pgfqpoint{0.545636in}{3.851073in}}%
\pgfpathcurveto{\pgfqpoint{0.537822in}{3.843260in}}{\pgfqpoint{0.533432in}{3.832661in}}{\pgfqpoint{0.533432in}{3.821610in}}%
\pgfpathcurveto{\pgfqpoint{0.533432in}{3.810560in}}{\pgfqpoint{0.537822in}{3.799961in}}{\pgfqpoint{0.545636in}{3.792148in}}%
\pgfpathcurveto{\pgfqpoint{0.553449in}{3.784334in}}{\pgfqpoint{0.564048in}{3.779944in}}{\pgfqpoint{0.575099in}{3.779944in}}%
\pgfpathlineto{\pgfqpoint{0.575099in}{3.779944in}}%
\pgfpathclose%
\pgfusepath{stroke}%
\end{pgfscope}%
\begin{pgfscope}%
\pgfpathrectangle{\pgfqpoint{0.494722in}{0.437222in}}{\pgfqpoint{6.275590in}{5.159444in}}%
\pgfusepath{clip}%
\pgfsetbuttcap%
\pgfsetroundjoin%
\pgfsetlinewidth{1.003750pt}%
\definecolor{currentstroke}{rgb}{0.827451,0.827451,0.827451}%
\pgfsetstrokecolor{currentstroke}%
\pgfsetstrokeopacity{0.800000}%
\pgfsetdash{}{0pt}%
\pgfpathmoveto{\pgfqpoint{2.258440in}{1.300760in}}%
\pgfpathcurveto{\pgfqpoint{2.269490in}{1.300760in}}{\pgfqpoint{2.280089in}{1.305150in}}{\pgfqpoint{2.287902in}{1.312964in}}%
\pgfpathcurveto{\pgfqpoint{2.295716in}{1.320777in}}{\pgfqpoint{2.300106in}{1.331376in}}{\pgfqpoint{2.300106in}{1.342426in}}%
\pgfpathcurveto{\pgfqpoint{2.300106in}{1.353477in}}{\pgfqpoint{2.295716in}{1.364076in}}{\pgfqpoint{2.287902in}{1.371889in}}%
\pgfpathcurveto{\pgfqpoint{2.280089in}{1.379703in}}{\pgfqpoint{2.269490in}{1.384093in}}{\pgfqpoint{2.258440in}{1.384093in}}%
\pgfpathcurveto{\pgfqpoint{2.247389in}{1.384093in}}{\pgfqpoint{2.236790in}{1.379703in}}{\pgfqpoint{2.228977in}{1.371889in}}%
\pgfpathcurveto{\pgfqpoint{2.221163in}{1.364076in}}{\pgfqpoint{2.216773in}{1.353477in}}{\pgfqpoint{2.216773in}{1.342426in}}%
\pgfpathcurveto{\pgfqpoint{2.216773in}{1.331376in}}{\pgfqpoint{2.221163in}{1.320777in}}{\pgfqpoint{2.228977in}{1.312964in}}%
\pgfpathcurveto{\pgfqpoint{2.236790in}{1.305150in}}{\pgfqpoint{2.247389in}{1.300760in}}{\pgfqpoint{2.258440in}{1.300760in}}%
\pgfpathlineto{\pgfqpoint{2.258440in}{1.300760in}}%
\pgfpathclose%
\pgfusepath{stroke}%
\end{pgfscope}%
\begin{pgfscope}%
\pgfpathrectangle{\pgfqpoint{0.494722in}{0.437222in}}{\pgfqpoint{6.275590in}{5.159444in}}%
\pgfusepath{clip}%
\pgfsetbuttcap%
\pgfsetroundjoin%
\pgfsetlinewidth{1.003750pt}%
\definecolor{currentstroke}{rgb}{0.827451,0.827451,0.827451}%
\pgfsetstrokecolor{currentstroke}%
\pgfsetstrokeopacity{0.800000}%
\pgfsetdash{}{0pt}%
\pgfpathmoveto{\pgfqpoint{1.700731in}{1.766489in}}%
\pgfpathcurveto{\pgfqpoint{1.711781in}{1.766489in}}{\pgfqpoint{1.722380in}{1.770879in}}{\pgfqpoint{1.730194in}{1.778693in}}%
\pgfpathcurveto{\pgfqpoint{1.738007in}{1.786506in}}{\pgfqpoint{1.742398in}{1.797105in}}{\pgfqpoint{1.742398in}{1.808156in}}%
\pgfpathcurveto{\pgfqpoint{1.742398in}{1.819206in}}{\pgfqpoint{1.738007in}{1.829805in}}{\pgfqpoint{1.730194in}{1.837618in}}%
\pgfpathcurveto{\pgfqpoint{1.722380in}{1.845432in}}{\pgfqpoint{1.711781in}{1.849822in}}{\pgfqpoint{1.700731in}{1.849822in}}%
\pgfpathcurveto{\pgfqpoint{1.689681in}{1.849822in}}{\pgfqpoint{1.679082in}{1.845432in}}{\pgfqpoint{1.671268in}{1.837618in}}%
\pgfpathcurveto{\pgfqpoint{1.663455in}{1.829805in}}{\pgfqpoint{1.659064in}{1.819206in}}{\pgfqpoint{1.659064in}{1.808156in}}%
\pgfpathcurveto{\pgfqpoint{1.659064in}{1.797105in}}{\pgfqpoint{1.663455in}{1.786506in}}{\pgfqpoint{1.671268in}{1.778693in}}%
\pgfpathcurveto{\pgfqpoint{1.679082in}{1.770879in}}{\pgfqpoint{1.689681in}{1.766489in}}{\pgfqpoint{1.700731in}{1.766489in}}%
\pgfpathlineto{\pgfqpoint{1.700731in}{1.766489in}}%
\pgfpathclose%
\pgfusepath{stroke}%
\end{pgfscope}%
\begin{pgfscope}%
\pgfpathrectangle{\pgfqpoint{0.494722in}{0.437222in}}{\pgfqpoint{6.275590in}{5.159444in}}%
\pgfusepath{clip}%
\pgfsetbuttcap%
\pgfsetroundjoin%
\pgfsetlinewidth{1.003750pt}%
\definecolor{currentstroke}{rgb}{0.827451,0.827451,0.827451}%
\pgfsetstrokecolor{currentstroke}%
\pgfsetstrokeopacity{0.800000}%
\pgfsetdash{}{0pt}%
\pgfpathmoveto{\pgfqpoint{0.680103in}{3.350578in}}%
\pgfpathcurveto{\pgfqpoint{0.691153in}{3.350578in}}{\pgfqpoint{0.701753in}{3.354968in}}{\pgfqpoint{0.709566in}{3.362782in}}%
\pgfpathcurveto{\pgfqpoint{0.717380in}{3.370596in}}{\pgfqpoint{0.721770in}{3.381195in}}{\pgfqpoint{0.721770in}{3.392245in}}%
\pgfpathcurveto{\pgfqpoint{0.721770in}{3.403295in}}{\pgfqpoint{0.717380in}{3.413894in}}{\pgfqpoint{0.709566in}{3.421707in}}%
\pgfpathcurveto{\pgfqpoint{0.701753in}{3.429521in}}{\pgfqpoint{0.691153in}{3.433911in}}{\pgfqpoint{0.680103in}{3.433911in}}%
\pgfpathcurveto{\pgfqpoint{0.669053in}{3.433911in}}{\pgfqpoint{0.658454in}{3.429521in}}{\pgfqpoint{0.650641in}{3.421707in}}%
\pgfpathcurveto{\pgfqpoint{0.642827in}{3.413894in}}{\pgfqpoint{0.638437in}{3.403295in}}{\pgfqpoint{0.638437in}{3.392245in}}%
\pgfpathcurveto{\pgfqpoint{0.638437in}{3.381195in}}{\pgfqpoint{0.642827in}{3.370596in}}{\pgfqpoint{0.650641in}{3.362782in}}%
\pgfpathcurveto{\pgfqpoint{0.658454in}{3.354968in}}{\pgfqpoint{0.669053in}{3.350578in}}{\pgfqpoint{0.680103in}{3.350578in}}%
\pgfpathlineto{\pgfqpoint{0.680103in}{3.350578in}}%
\pgfpathclose%
\pgfusepath{stroke}%
\end{pgfscope}%
\begin{pgfscope}%
\pgfpathrectangle{\pgfqpoint{0.494722in}{0.437222in}}{\pgfqpoint{6.275590in}{5.159444in}}%
\pgfusepath{clip}%
\pgfsetbuttcap%
\pgfsetroundjoin%
\pgfsetlinewidth{1.003750pt}%
\definecolor{currentstroke}{rgb}{0.827451,0.827451,0.827451}%
\pgfsetstrokecolor{currentstroke}%
\pgfsetstrokeopacity{0.800000}%
\pgfsetdash{}{0pt}%
\pgfpathmoveto{\pgfqpoint{4.176161in}{0.527713in}}%
\pgfpathcurveto{\pgfqpoint{4.187212in}{0.527713in}}{\pgfqpoint{4.197811in}{0.532104in}}{\pgfqpoint{4.205624in}{0.539917in}}%
\pgfpathcurveto{\pgfqpoint{4.213438in}{0.547731in}}{\pgfqpoint{4.217828in}{0.558330in}}{\pgfqpoint{4.217828in}{0.569380in}}%
\pgfpathcurveto{\pgfqpoint{4.217828in}{0.580430in}}{\pgfqpoint{4.213438in}{0.591029in}}{\pgfqpoint{4.205624in}{0.598843in}}%
\pgfpathcurveto{\pgfqpoint{4.197811in}{0.606656in}}{\pgfqpoint{4.187212in}{0.611047in}}{\pgfqpoint{4.176161in}{0.611047in}}%
\pgfpathcurveto{\pgfqpoint{4.165111in}{0.611047in}}{\pgfqpoint{4.154512in}{0.606656in}}{\pgfqpoint{4.146699in}{0.598843in}}%
\pgfpathcurveto{\pgfqpoint{4.138885in}{0.591029in}}{\pgfqpoint{4.134495in}{0.580430in}}{\pgfqpoint{4.134495in}{0.569380in}}%
\pgfpathcurveto{\pgfqpoint{4.134495in}{0.558330in}}{\pgfqpoint{4.138885in}{0.547731in}}{\pgfqpoint{4.146699in}{0.539917in}}%
\pgfpathcurveto{\pgfqpoint{4.154512in}{0.532104in}}{\pgfqpoint{4.165111in}{0.527713in}}{\pgfqpoint{4.176161in}{0.527713in}}%
\pgfpathlineto{\pgfqpoint{4.176161in}{0.527713in}}%
\pgfpathclose%
\pgfusepath{stroke}%
\end{pgfscope}%
\begin{pgfscope}%
\pgfpathrectangle{\pgfqpoint{0.494722in}{0.437222in}}{\pgfqpoint{6.275590in}{5.159444in}}%
\pgfusepath{clip}%
\pgfsetbuttcap%
\pgfsetroundjoin%
\pgfsetlinewidth{1.003750pt}%
\definecolor{currentstroke}{rgb}{0.827451,0.827451,0.827451}%
\pgfsetstrokecolor{currentstroke}%
\pgfsetstrokeopacity{0.800000}%
\pgfsetdash{}{0pt}%
\pgfpathmoveto{\pgfqpoint{2.659012in}{1.065276in}}%
\pgfpathcurveto{\pgfqpoint{2.670062in}{1.065276in}}{\pgfqpoint{2.680661in}{1.069667in}}{\pgfqpoint{2.688475in}{1.077480in}}%
\pgfpathcurveto{\pgfqpoint{2.696289in}{1.085294in}}{\pgfqpoint{2.700679in}{1.095893in}}{\pgfqpoint{2.700679in}{1.106943in}}%
\pgfpathcurveto{\pgfqpoint{2.700679in}{1.117993in}}{\pgfqpoint{2.696289in}{1.128592in}}{\pgfqpoint{2.688475in}{1.136406in}}%
\pgfpathcurveto{\pgfqpoint{2.680661in}{1.144219in}}{\pgfqpoint{2.670062in}{1.148610in}}{\pgfqpoint{2.659012in}{1.148610in}}%
\pgfpathcurveto{\pgfqpoint{2.647962in}{1.148610in}}{\pgfqpoint{2.637363in}{1.144219in}}{\pgfqpoint{2.629550in}{1.136406in}}%
\pgfpathcurveto{\pgfqpoint{2.621736in}{1.128592in}}{\pgfqpoint{2.617346in}{1.117993in}}{\pgfqpoint{2.617346in}{1.106943in}}%
\pgfpathcurveto{\pgfqpoint{2.617346in}{1.095893in}}{\pgfqpoint{2.621736in}{1.085294in}}{\pgfqpoint{2.629550in}{1.077480in}}%
\pgfpathcurveto{\pgfqpoint{2.637363in}{1.069667in}}{\pgfqpoint{2.647962in}{1.065276in}}{\pgfqpoint{2.659012in}{1.065276in}}%
\pgfpathlineto{\pgfqpoint{2.659012in}{1.065276in}}%
\pgfpathclose%
\pgfusepath{stroke}%
\end{pgfscope}%
\begin{pgfscope}%
\pgfpathrectangle{\pgfqpoint{0.494722in}{0.437222in}}{\pgfqpoint{6.275590in}{5.159444in}}%
\pgfusepath{clip}%
\pgfsetbuttcap%
\pgfsetroundjoin%
\pgfsetlinewidth{1.003750pt}%
\definecolor{currentstroke}{rgb}{0.827451,0.827451,0.827451}%
\pgfsetstrokecolor{currentstroke}%
\pgfsetstrokeopacity{0.800000}%
\pgfsetdash{}{0pt}%
\pgfpathmoveto{\pgfqpoint{2.765798in}{1.011077in}}%
\pgfpathcurveto{\pgfqpoint{2.776848in}{1.011077in}}{\pgfqpoint{2.787447in}{1.015467in}}{\pgfqpoint{2.795261in}{1.023280in}}%
\pgfpathcurveto{\pgfqpoint{2.803075in}{1.031094in}}{\pgfqpoint{2.807465in}{1.041693in}}{\pgfqpoint{2.807465in}{1.052743in}}%
\pgfpathcurveto{\pgfqpoint{2.807465in}{1.063793in}}{\pgfqpoint{2.803075in}{1.074392in}}{\pgfqpoint{2.795261in}{1.082206in}}%
\pgfpathcurveto{\pgfqpoint{2.787447in}{1.090020in}}{\pgfqpoint{2.776848in}{1.094410in}}{\pgfqpoint{2.765798in}{1.094410in}}%
\pgfpathcurveto{\pgfqpoint{2.754748in}{1.094410in}}{\pgfqpoint{2.744149in}{1.090020in}}{\pgfqpoint{2.736335in}{1.082206in}}%
\pgfpathcurveto{\pgfqpoint{2.728522in}{1.074392in}}{\pgfqpoint{2.724132in}{1.063793in}}{\pgfqpoint{2.724132in}{1.052743in}}%
\pgfpathcurveto{\pgfqpoint{2.724132in}{1.041693in}}{\pgfqpoint{2.728522in}{1.031094in}}{\pgfqpoint{2.736335in}{1.023280in}}%
\pgfpathcurveto{\pgfqpoint{2.744149in}{1.015467in}}{\pgfqpoint{2.754748in}{1.011077in}}{\pgfqpoint{2.765798in}{1.011077in}}%
\pgfpathlineto{\pgfqpoint{2.765798in}{1.011077in}}%
\pgfpathclose%
\pgfusepath{stroke}%
\end{pgfscope}%
\begin{pgfscope}%
\pgfpathrectangle{\pgfqpoint{0.494722in}{0.437222in}}{\pgfqpoint{6.275590in}{5.159444in}}%
\pgfusepath{clip}%
\pgfsetbuttcap%
\pgfsetroundjoin%
\pgfsetlinewidth{1.003750pt}%
\definecolor{currentstroke}{rgb}{0.827451,0.827451,0.827451}%
\pgfsetstrokecolor{currentstroke}%
\pgfsetstrokeopacity{0.800000}%
\pgfsetdash{}{0pt}%
\pgfpathmoveto{\pgfqpoint{0.750297in}{3.133310in}}%
\pgfpathcurveto{\pgfqpoint{0.761348in}{3.133310in}}{\pgfqpoint{0.771947in}{3.137700in}}{\pgfqpoint{0.779760in}{3.145514in}}%
\pgfpathcurveto{\pgfqpoint{0.787574in}{3.153328in}}{\pgfqpoint{0.791964in}{3.163927in}}{\pgfqpoint{0.791964in}{3.174977in}}%
\pgfpathcurveto{\pgfqpoint{0.791964in}{3.186027in}}{\pgfqpoint{0.787574in}{3.196626in}}{\pgfqpoint{0.779760in}{3.204439in}}%
\pgfpathcurveto{\pgfqpoint{0.771947in}{3.212253in}}{\pgfqpoint{0.761348in}{3.216643in}}{\pgfqpoint{0.750297in}{3.216643in}}%
\pgfpathcurveto{\pgfqpoint{0.739247in}{3.216643in}}{\pgfqpoint{0.728648in}{3.212253in}}{\pgfqpoint{0.720835in}{3.204439in}}%
\pgfpathcurveto{\pgfqpoint{0.713021in}{3.196626in}}{\pgfqpoint{0.708631in}{3.186027in}}{\pgfqpoint{0.708631in}{3.174977in}}%
\pgfpathcurveto{\pgfqpoint{0.708631in}{3.163927in}}{\pgfqpoint{0.713021in}{3.153328in}}{\pgfqpoint{0.720835in}{3.145514in}}%
\pgfpathcurveto{\pgfqpoint{0.728648in}{3.137700in}}{\pgfqpoint{0.739247in}{3.133310in}}{\pgfqpoint{0.750297in}{3.133310in}}%
\pgfpathlineto{\pgfqpoint{0.750297in}{3.133310in}}%
\pgfpathclose%
\pgfusepath{stroke}%
\end{pgfscope}%
\begin{pgfscope}%
\pgfpathrectangle{\pgfqpoint{0.494722in}{0.437222in}}{\pgfqpoint{6.275590in}{5.159444in}}%
\pgfusepath{clip}%
\pgfsetbuttcap%
\pgfsetroundjoin%
\pgfsetlinewidth{1.003750pt}%
\definecolor{currentstroke}{rgb}{0.827451,0.827451,0.827451}%
\pgfsetstrokecolor{currentstroke}%
\pgfsetstrokeopacity{0.800000}%
\pgfsetdash{}{0pt}%
\pgfpathmoveto{\pgfqpoint{1.055628in}{2.578187in}}%
\pgfpathcurveto{\pgfqpoint{1.066678in}{2.578187in}}{\pgfqpoint{1.077277in}{2.582578in}}{\pgfqpoint{1.085091in}{2.590391in}}%
\pgfpathcurveto{\pgfqpoint{1.092904in}{2.598205in}}{\pgfqpoint{1.097295in}{2.608804in}}{\pgfqpoint{1.097295in}{2.619854in}}%
\pgfpathcurveto{\pgfqpoint{1.097295in}{2.630904in}}{\pgfqpoint{1.092904in}{2.641503in}}{\pgfqpoint{1.085091in}{2.649317in}}%
\pgfpathcurveto{\pgfqpoint{1.077277in}{2.657130in}}{\pgfqpoint{1.066678in}{2.661521in}}{\pgfqpoint{1.055628in}{2.661521in}}%
\pgfpathcurveto{\pgfqpoint{1.044578in}{2.661521in}}{\pgfqpoint{1.033979in}{2.657130in}}{\pgfqpoint{1.026165in}{2.649317in}}%
\pgfpathcurveto{\pgfqpoint{1.018352in}{2.641503in}}{\pgfqpoint{1.013961in}{2.630904in}}{\pgfqpoint{1.013961in}{2.619854in}}%
\pgfpathcurveto{\pgfqpoint{1.013961in}{2.608804in}}{\pgfqpoint{1.018352in}{2.598205in}}{\pgfqpoint{1.026165in}{2.590391in}}%
\pgfpathcurveto{\pgfqpoint{1.033979in}{2.582578in}}{\pgfqpoint{1.044578in}{2.578187in}}{\pgfqpoint{1.055628in}{2.578187in}}%
\pgfpathlineto{\pgfqpoint{1.055628in}{2.578187in}}%
\pgfpathclose%
\pgfusepath{stroke}%
\end{pgfscope}%
\begin{pgfscope}%
\pgfpathrectangle{\pgfqpoint{0.494722in}{0.437222in}}{\pgfqpoint{6.275590in}{5.159444in}}%
\pgfusepath{clip}%
\pgfsetbuttcap%
\pgfsetroundjoin%
\pgfsetlinewidth{1.003750pt}%
\definecolor{currentstroke}{rgb}{0.827451,0.827451,0.827451}%
\pgfsetstrokecolor{currentstroke}%
\pgfsetstrokeopacity{0.800000}%
\pgfsetdash{}{0pt}%
\pgfpathmoveto{\pgfqpoint{2.856780in}{0.965441in}}%
\pgfpathcurveto{\pgfqpoint{2.867831in}{0.965441in}}{\pgfqpoint{2.878430in}{0.969831in}}{\pgfqpoint{2.886243in}{0.977645in}}%
\pgfpathcurveto{\pgfqpoint{2.894057in}{0.985458in}}{\pgfqpoint{2.898447in}{0.996058in}}{\pgfqpoint{2.898447in}{1.007108in}}%
\pgfpathcurveto{\pgfqpoint{2.898447in}{1.018158in}}{\pgfqpoint{2.894057in}{1.028757in}}{\pgfqpoint{2.886243in}{1.036570in}}%
\pgfpathcurveto{\pgfqpoint{2.878430in}{1.044384in}}{\pgfqpoint{2.867831in}{1.048774in}}{\pgfqpoint{2.856780in}{1.048774in}}%
\pgfpathcurveto{\pgfqpoint{2.845730in}{1.048774in}}{\pgfqpoint{2.835131in}{1.044384in}}{\pgfqpoint{2.827318in}{1.036570in}}%
\pgfpathcurveto{\pgfqpoint{2.819504in}{1.028757in}}{\pgfqpoint{2.815114in}{1.018158in}}{\pgfqpoint{2.815114in}{1.007108in}}%
\pgfpathcurveto{\pgfqpoint{2.815114in}{0.996058in}}{\pgfqpoint{2.819504in}{0.985458in}}{\pgfqpoint{2.827318in}{0.977645in}}%
\pgfpathcurveto{\pgfqpoint{2.835131in}{0.969831in}}{\pgfqpoint{2.845730in}{0.965441in}}{\pgfqpoint{2.856780in}{0.965441in}}%
\pgfpathlineto{\pgfqpoint{2.856780in}{0.965441in}}%
\pgfpathclose%
\pgfusepath{stroke}%
\end{pgfscope}%
\begin{pgfscope}%
\pgfpathrectangle{\pgfqpoint{0.494722in}{0.437222in}}{\pgfqpoint{6.275590in}{5.159444in}}%
\pgfusepath{clip}%
\pgfsetbuttcap%
\pgfsetroundjoin%
\pgfsetlinewidth{1.003750pt}%
\definecolor{currentstroke}{rgb}{0.827451,0.827451,0.827451}%
\pgfsetstrokecolor{currentstroke}%
\pgfsetstrokeopacity{0.800000}%
\pgfsetdash{}{0pt}%
\pgfpathmoveto{\pgfqpoint{2.552339in}{1.147505in}}%
\pgfpathcurveto{\pgfqpoint{2.563390in}{1.147505in}}{\pgfqpoint{2.573989in}{1.151895in}}{\pgfqpoint{2.581802in}{1.159709in}}%
\pgfpathcurveto{\pgfqpoint{2.589616in}{1.167522in}}{\pgfqpoint{2.594006in}{1.178121in}}{\pgfqpoint{2.594006in}{1.189172in}}%
\pgfpathcurveto{\pgfqpoint{2.594006in}{1.200222in}}{\pgfqpoint{2.589616in}{1.210821in}}{\pgfqpoint{2.581802in}{1.218634in}}%
\pgfpathcurveto{\pgfqpoint{2.573989in}{1.226448in}}{\pgfqpoint{2.563390in}{1.230838in}}{\pgfqpoint{2.552339in}{1.230838in}}%
\pgfpathcurveto{\pgfqpoint{2.541289in}{1.230838in}}{\pgfqpoint{2.530690in}{1.226448in}}{\pgfqpoint{2.522877in}{1.218634in}}%
\pgfpathcurveto{\pgfqpoint{2.515063in}{1.210821in}}{\pgfqpoint{2.510673in}{1.200222in}}{\pgfqpoint{2.510673in}{1.189172in}}%
\pgfpathcurveto{\pgfqpoint{2.510673in}{1.178121in}}{\pgfqpoint{2.515063in}{1.167522in}}{\pgfqpoint{2.522877in}{1.159709in}}%
\pgfpathcurveto{\pgfqpoint{2.530690in}{1.151895in}}{\pgfqpoint{2.541289in}{1.147505in}}{\pgfqpoint{2.552339in}{1.147505in}}%
\pgfpathlineto{\pgfqpoint{2.552339in}{1.147505in}}%
\pgfpathclose%
\pgfusepath{stroke}%
\end{pgfscope}%
\begin{pgfscope}%
\pgfpathrectangle{\pgfqpoint{0.494722in}{0.437222in}}{\pgfqpoint{6.275590in}{5.159444in}}%
\pgfusepath{clip}%
\pgfsetbuttcap%
\pgfsetroundjoin%
\pgfsetlinewidth{1.003750pt}%
\definecolor{currentstroke}{rgb}{0.827451,0.827451,0.827451}%
\pgfsetstrokecolor{currentstroke}%
\pgfsetstrokeopacity{0.800000}%
\pgfsetdash{}{0pt}%
\pgfpathmoveto{\pgfqpoint{1.712263in}{1.726647in}}%
\pgfpathcurveto{\pgfqpoint{1.723313in}{1.726647in}}{\pgfqpoint{1.733912in}{1.731037in}}{\pgfqpoint{1.741726in}{1.738851in}}%
\pgfpathcurveto{\pgfqpoint{1.749539in}{1.746664in}}{\pgfqpoint{1.753929in}{1.757263in}}{\pgfqpoint{1.753929in}{1.768314in}}%
\pgfpathcurveto{\pgfqpoint{1.753929in}{1.779364in}}{\pgfqpoint{1.749539in}{1.789963in}}{\pgfqpoint{1.741726in}{1.797776in}}%
\pgfpathcurveto{\pgfqpoint{1.733912in}{1.805590in}}{\pgfqpoint{1.723313in}{1.809980in}}{\pgfqpoint{1.712263in}{1.809980in}}%
\pgfpathcurveto{\pgfqpoint{1.701213in}{1.809980in}}{\pgfqpoint{1.690614in}{1.805590in}}{\pgfqpoint{1.682800in}{1.797776in}}%
\pgfpathcurveto{\pgfqpoint{1.674986in}{1.789963in}}{\pgfqpoint{1.670596in}{1.779364in}}{\pgfqpoint{1.670596in}{1.768314in}}%
\pgfpathcurveto{\pgfqpoint{1.670596in}{1.757263in}}{\pgfqpoint{1.674986in}{1.746664in}}{\pgfqpoint{1.682800in}{1.738851in}}%
\pgfpathcurveto{\pgfqpoint{1.690614in}{1.731037in}}{\pgfqpoint{1.701213in}{1.726647in}}{\pgfqpoint{1.712263in}{1.726647in}}%
\pgfpathlineto{\pgfqpoint{1.712263in}{1.726647in}}%
\pgfpathclose%
\pgfusepath{stroke}%
\end{pgfscope}%
\begin{pgfscope}%
\pgfpathrectangle{\pgfqpoint{0.494722in}{0.437222in}}{\pgfqpoint{6.275590in}{5.159444in}}%
\pgfusepath{clip}%
\pgfsetbuttcap%
\pgfsetroundjoin%
\pgfsetlinewidth{1.003750pt}%
\definecolor{currentstroke}{rgb}{0.827451,0.827451,0.827451}%
\pgfsetstrokecolor{currentstroke}%
\pgfsetstrokeopacity{0.800000}%
\pgfsetdash{}{0pt}%
\pgfpathmoveto{\pgfqpoint{3.321971in}{0.791856in}}%
\pgfpathcurveto{\pgfqpoint{3.333021in}{0.791856in}}{\pgfqpoint{3.343620in}{0.796246in}}{\pgfqpoint{3.351433in}{0.804060in}}%
\pgfpathcurveto{\pgfqpoint{3.359247in}{0.811873in}}{\pgfqpoint{3.363637in}{0.822472in}}{\pgfqpoint{3.363637in}{0.833523in}}%
\pgfpathcurveto{\pgfqpoint{3.363637in}{0.844573in}}{\pgfqpoint{3.359247in}{0.855172in}}{\pgfqpoint{3.351433in}{0.862985in}}%
\pgfpathcurveto{\pgfqpoint{3.343620in}{0.870799in}}{\pgfqpoint{3.333021in}{0.875189in}}{\pgfqpoint{3.321971in}{0.875189in}}%
\pgfpathcurveto{\pgfqpoint{3.310921in}{0.875189in}}{\pgfqpoint{3.300321in}{0.870799in}}{\pgfqpoint{3.292508in}{0.862985in}}%
\pgfpathcurveto{\pgfqpoint{3.284694in}{0.855172in}}{\pgfqpoint{3.280304in}{0.844573in}}{\pgfqpoint{3.280304in}{0.833523in}}%
\pgfpathcurveto{\pgfqpoint{3.280304in}{0.822472in}}{\pgfqpoint{3.284694in}{0.811873in}}{\pgfqpoint{3.292508in}{0.804060in}}%
\pgfpathcurveto{\pgfqpoint{3.300321in}{0.796246in}}{\pgfqpoint{3.310921in}{0.791856in}}{\pgfqpoint{3.321971in}{0.791856in}}%
\pgfpathlineto{\pgfqpoint{3.321971in}{0.791856in}}%
\pgfpathclose%
\pgfusepath{stroke}%
\end{pgfscope}%
\begin{pgfscope}%
\pgfpathrectangle{\pgfqpoint{0.494722in}{0.437222in}}{\pgfqpoint{6.275590in}{5.159444in}}%
\pgfusepath{clip}%
\pgfsetbuttcap%
\pgfsetroundjoin%
\pgfsetlinewidth{1.003750pt}%
\definecolor{currentstroke}{rgb}{0.827451,0.827451,0.827451}%
\pgfsetstrokecolor{currentstroke}%
\pgfsetstrokeopacity{0.800000}%
\pgfsetdash{}{0pt}%
\pgfpathmoveto{\pgfqpoint{3.737625in}{0.637421in}}%
\pgfpathcurveto{\pgfqpoint{3.748675in}{0.637421in}}{\pgfqpoint{3.759274in}{0.641812in}}{\pgfqpoint{3.767088in}{0.649625in}}%
\pgfpathcurveto{\pgfqpoint{3.774901in}{0.657439in}}{\pgfqpoint{3.779292in}{0.668038in}}{\pgfqpoint{3.779292in}{0.679088in}}%
\pgfpathcurveto{\pgfqpoint{3.779292in}{0.690138in}}{\pgfqpoint{3.774901in}{0.700737in}}{\pgfqpoint{3.767088in}{0.708551in}}%
\pgfpathcurveto{\pgfqpoint{3.759274in}{0.716364in}}{\pgfqpoint{3.748675in}{0.720755in}}{\pgfqpoint{3.737625in}{0.720755in}}%
\pgfpathcurveto{\pgfqpoint{3.726575in}{0.720755in}}{\pgfqpoint{3.715976in}{0.716364in}}{\pgfqpoint{3.708162in}{0.708551in}}%
\pgfpathcurveto{\pgfqpoint{3.700349in}{0.700737in}}{\pgfqpoint{3.695958in}{0.690138in}}{\pgfqpoint{3.695958in}{0.679088in}}%
\pgfpathcurveto{\pgfqpoint{3.695958in}{0.668038in}}{\pgfqpoint{3.700349in}{0.657439in}}{\pgfqpoint{3.708162in}{0.649625in}}%
\pgfpathcurveto{\pgfqpoint{3.715976in}{0.641812in}}{\pgfqpoint{3.726575in}{0.637421in}}{\pgfqpoint{3.737625in}{0.637421in}}%
\pgfpathlineto{\pgfqpoint{3.737625in}{0.637421in}}%
\pgfpathclose%
\pgfusepath{stroke}%
\end{pgfscope}%
\begin{pgfscope}%
\pgfpathrectangle{\pgfqpoint{0.494722in}{0.437222in}}{\pgfqpoint{6.275590in}{5.159444in}}%
\pgfusepath{clip}%
\pgfsetbuttcap%
\pgfsetroundjoin%
\pgfsetlinewidth{1.003750pt}%
\definecolor{currentstroke}{rgb}{0.827451,0.827451,0.827451}%
\pgfsetstrokecolor{currentstroke}%
\pgfsetstrokeopacity{0.800000}%
\pgfsetdash{}{0pt}%
\pgfpathmoveto{\pgfqpoint{0.683182in}{3.327378in}}%
\pgfpathcurveto{\pgfqpoint{0.694232in}{3.327378in}}{\pgfqpoint{0.704831in}{3.331768in}}{\pgfqpoint{0.712645in}{3.339581in}}%
\pgfpathcurveto{\pgfqpoint{0.720459in}{3.347395in}}{\pgfqpoint{0.724849in}{3.357994in}}{\pgfqpoint{0.724849in}{3.369044in}}%
\pgfpathcurveto{\pgfqpoint{0.724849in}{3.380094in}}{\pgfqpoint{0.720459in}{3.390693in}}{\pgfqpoint{0.712645in}{3.398507in}}%
\pgfpathcurveto{\pgfqpoint{0.704831in}{3.406321in}}{\pgfqpoint{0.694232in}{3.410711in}}{\pgfqpoint{0.683182in}{3.410711in}}%
\pgfpathcurveto{\pgfqpoint{0.672132in}{3.410711in}}{\pgfqpoint{0.661533in}{3.406321in}}{\pgfqpoint{0.653720in}{3.398507in}}%
\pgfpathcurveto{\pgfqpoint{0.645906in}{3.390693in}}{\pgfqpoint{0.641516in}{3.380094in}}{\pgfqpoint{0.641516in}{3.369044in}}%
\pgfpathcurveto{\pgfqpoint{0.641516in}{3.357994in}}{\pgfqpoint{0.645906in}{3.347395in}}{\pgfqpoint{0.653720in}{3.339581in}}%
\pgfpathcurveto{\pgfqpoint{0.661533in}{3.331768in}}{\pgfqpoint{0.672132in}{3.327378in}}{\pgfqpoint{0.683182in}{3.327378in}}%
\pgfpathlineto{\pgfqpoint{0.683182in}{3.327378in}}%
\pgfpathclose%
\pgfusepath{stroke}%
\end{pgfscope}%
\begin{pgfscope}%
\pgfpathrectangle{\pgfqpoint{0.494722in}{0.437222in}}{\pgfqpoint{6.275590in}{5.159444in}}%
\pgfusepath{clip}%
\pgfsetbuttcap%
\pgfsetroundjoin%
\pgfsetlinewidth{1.003750pt}%
\definecolor{currentstroke}{rgb}{0.827451,0.827451,0.827451}%
\pgfsetstrokecolor{currentstroke}%
\pgfsetstrokeopacity{0.800000}%
\pgfsetdash{}{0pt}%
\pgfpathmoveto{\pgfqpoint{0.502837in}{4.347346in}}%
\pgfpathcurveto{\pgfqpoint{0.513887in}{4.347346in}}{\pgfqpoint{0.524486in}{4.351736in}}{\pgfqpoint{0.532299in}{4.359550in}}%
\pgfpathcurveto{\pgfqpoint{0.540113in}{4.367363in}}{\pgfqpoint{0.544503in}{4.377962in}}{\pgfqpoint{0.544503in}{4.389013in}}%
\pgfpathcurveto{\pgfqpoint{0.544503in}{4.400063in}}{\pgfqpoint{0.540113in}{4.410662in}}{\pgfqpoint{0.532299in}{4.418475in}}%
\pgfpathcurveto{\pgfqpoint{0.524486in}{4.426289in}}{\pgfqpoint{0.513887in}{4.430679in}}{\pgfqpoint{0.502837in}{4.430679in}}%
\pgfpathcurveto{\pgfqpoint{0.491787in}{4.430679in}}{\pgfqpoint{0.481188in}{4.426289in}}{\pgfqpoint{0.473374in}{4.418475in}}%
\pgfpathcurveto{\pgfqpoint{0.465560in}{4.410662in}}{\pgfqpoint{0.461170in}{4.400063in}}{\pgfqpoint{0.461170in}{4.389013in}}%
\pgfpathcurveto{\pgfqpoint{0.461170in}{4.377962in}}{\pgfqpoint{0.465560in}{4.367363in}}{\pgfqpoint{0.473374in}{4.359550in}}%
\pgfpathcurveto{\pgfqpoint{0.481188in}{4.351736in}}{\pgfqpoint{0.491787in}{4.347346in}}{\pgfqpoint{0.502837in}{4.347346in}}%
\pgfpathlineto{\pgfqpoint{0.502837in}{4.347346in}}%
\pgfpathclose%
\pgfusepath{stroke}%
\end{pgfscope}%
\begin{pgfscope}%
\pgfpathrectangle{\pgfqpoint{0.494722in}{0.437222in}}{\pgfqpoint{6.275590in}{5.159444in}}%
\pgfusepath{clip}%
\pgfsetbuttcap%
\pgfsetroundjoin%
\pgfsetlinewidth{1.003750pt}%
\definecolor{currentstroke}{rgb}{0.827451,0.827451,0.827451}%
\pgfsetstrokecolor{currentstroke}%
\pgfsetstrokeopacity{0.800000}%
\pgfsetdash{}{0pt}%
\pgfpathmoveto{\pgfqpoint{0.734476in}{3.173741in}}%
\pgfpathcurveto{\pgfqpoint{0.745526in}{3.173741in}}{\pgfqpoint{0.756125in}{3.178131in}}{\pgfqpoint{0.763939in}{3.185945in}}%
\pgfpathcurveto{\pgfqpoint{0.771753in}{3.193758in}}{\pgfqpoint{0.776143in}{3.204357in}}{\pgfqpoint{0.776143in}{3.215408in}}%
\pgfpathcurveto{\pgfqpoint{0.776143in}{3.226458in}}{\pgfqpoint{0.771753in}{3.237057in}}{\pgfqpoint{0.763939in}{3.244870in}}%
\pgfpathcurveto{\pgfqpoint{0.756125in}{3.252684in}}{\pgfqpoint{0.745526in}{3.257074in}}{\pgfqpoint{0.734476in}{3.257074in}}%
\pgfpathcurveto{\pgfqpoint{0.723426in}{3.257074in}}{\pgfqpoint{0.712827in}{3.252684in}}{\pgfqpoint{0.705013in}{3.244870in}}%
\pgfpathcurveto{\pgfqpoint{0.697200in}{3.237057in}}{\pgfqpoint{0.692810in}{3.226458in}}{\pgfqpoint{0.692810in}{3.215408in}}%
\pgfpathcurveto{\pgfqpoint{0.692810in}{3.204357in}}{\pgfqpoint{0.697200in}{3.193758in}}{\pgfqpoint{0.705013in}{3.185945in}}%
\pgfpathcurveto{\pgfqpoint{0.712827in}{3.178131in}}{\pgfqpoint{0.723426in}{3.173741in}}{\pgfqpoint{0.734476in}{3.173741in}}%
\pgfpathlineto{\pgfqpoint{0.734476in}{3.173741in}}%
\pgfpathclose%
\pgfusepath{stroke}%
\end{pgfscope}%
\begin{pgfscope}%
\pgfpathrectangle{\pgfqpoint{0.494722in}{0.437222in}}{\pgfqpoint{6.275590in}{5.159444in}}%
\pgfusepath{clip}%
\pgfsetbuttcap%
\pgfsetroundjoin%
\pgfsetlinewidth{1.003750pt}%
\definecolor{currentstroke}{rgb}{0.827451,0.827451,0.827451}%
\pgfsetstrokecolor{currentstroke}%
\pgfsetstrokeopacity{0.800000}%
\pgfsetdash{}{0pt}%
\pgfpathmoveto{\pgfqpoint{2.990259in}{0.911738in}}%
\pgfpathcurveto{\pgfqpoint{3.001309in}{0.911738in}}{\pgfqpoint{3.011908in}{0.916128in}}{\pgfqpoint{3.019722in}{0.923942in}}%
\pgfpathcurveto{\pgfqpoint{3.027536in}{0.931755in}}{\pgfqpoint{3.031926in}{0.942354in}}{\pgfqpoint{3.031926in}{0.953404in}}%
\pgfpathcurveto{\pgfqpoint{3.031926in}{0.964455in}}{\pgfqpoint{3.027536in}{0.975054in}}{\pgfqpoint{3.019722in}{0.982867in}}%
\pgfpathcurveto{\pgfqpoint{3.011908in}{0.990681in}}{\pgfqpoint{3.001309in}{0.995071in}}{\pgfqpoint{2.990259in}{0.995071in}}%
\pgfpathcurveto{\pgfqpoint{2.979209in}{0.995071in}}{\pgfqpoint{2.968610in}{0.990681in}}{\pgfqpoint{2.960796in}{0.982867in}}%
\pgfpathcurveto{\pgfqpoint{2.952983in}{0.975054in}}{\pgfqpoint{2.948593in}{0.964455in}}{\pgfqpoint{2.948593in}{0.953404in}}%
\pgfpathcurveto{\pgfqpoint{2.948593in}{0.942354in}}{\pgfqpoint{2.952983in}{0.931755in}}{\pgfqpoint{2.960796in}{0.923942in}}%
\pgfpathcurveto{\pgfqpoint{2.968610in}{0.916128in}}{\pgfqpoint{2.979209in}{0.911738in}}{\pgfqpoint{2.990259in}{0.911738in}}%
\pgfpathlineto{\pgfqpoint{2.990259in}{0.911738in}}%
\pgfpathclose%
\pgfusepath{stroke}%
\end{pgfscope}%
\begin{pgfscope}%
\pgfpathrectangle{\pgfqpoint{0.494722in}{0.437222in}}{\pgfqpoint{6.275590in}{5.159444in}}%
\pgfusepath{clip}%
\pgfsetbuttcap%
\pgfsetroundjoin%
\pgfsetlinewidth{1.003750pt}%
\definecolor{currentstroke}{rgb}{0.827451,0.827451,0.827451}%
\pgfsetstrokecolor{currentstroke}%
\pgfsetstrokeopacity{0.800000}%
\pgfsetdash{}{0pt}%
\pgfpathmoveto{\pgfqpoint{3.144939in}{0.844698in}}%
\pgfpathcurveto{\pgfqpoint{3.155989in}{0.844698in}}{\pgfqpoint{3.166588in}{0.849089in}}{\pgfqpoint{3.174402in}{0.856902in}}%
\pgfpathcurveto{\pgfqpoint{3.182216in}{0.864716in}}{\pgfqpoint{3.186606in}{0.875315in}}{\pgfqpoint{3.186606in}{0.886365in}}%
\pgfpathcurveto{\pgfqpoint{3.186606in}{0.897415in}}{\pgfqpoint{3.182216in}{0.908014in}}{\pgfqpoint{3.174402in}{0.915828in}}%
\pgfpathcurveto{\pgfqpoint{3.166588in}{0.923641in}}{\pgfqpoint{3.155989in}{0.928032in}}{\pgfqpoint{3.144939in}{0.928032in}}%
\pgfpathcurveto{\pgfqpoint{3.133889in}{0.928032in}}{\pgfqpoint{3.123290in}{0.923641in}}{\pgfqpoint{3.115477in}{0.915828in}}%
\pgfpathcurveto{\pgfqpoint{3.107663in}{0.908014in}}{\pgfqpoint{3.103273in}{0.897415in}}{\pgfqpoint{3.103273in}{0.886365in}}%
\pgfpathcurveto{\pgfqpoint{3.103273in}{0.875315in}}{\pgfqpoint{3.107663in}{0.864716in}}{\pgfqpoint{3.115477in}{0.856902in}}%
\pgfpathcurveto{\pgfqpoint{3.123290in}{0.849089in}}{\pgfqpoint{3.133889in}{0.844698in}}{\pgfqpoint{3.144939in}{0.844698in}}%
\pgfpathlineto{\pgfqpoint{3.144939in}{0.844698in}}%
\pgfpathclose%
\pgfusepath{stroke}%
\end{pgfscope}%
\begin{pgfscope}%
\pgfpathrectangle{\pgfqpoint{0.494722in}{0.437222in}}{\pgfqpoint{6.275590in}{5.159444in}}%
\pgfusepath{clip}%
\pgfsetbuttcap%
\pgfsetroundjoin%
\pgfsetlinewidth{1.003750pt}%
\definecolor{currentstroke}{rgb}{0.827451,0.827451,0.827451}%
\pgfsetstrokecolor{currentstroke}%
\pgfsetstrokeopacity{0.800000}%
\pgfsetdash{}{0pt}%
\pgfpathmoveto{\pgfqpoint{1.402819in}{2.033337in}}%
\pgfpathcurveto{\pgfqpoint{1.413869in}{2.033337in}}{\pgfqpoint{1.424468in}{2.037727in}}{\pgfqpoint{1.432282in}{2.045541in}}%
\pgfpathcurveto{\pgfqpoint{1.440096in}{2.053354in}}{\pgfqpoint{1.444486in}{2.063953in}}{\pgfqpoint{1.444486in}{2.075004in}}%
\pgfpathcurveto{\pgfqpoint{1.444486in}{2.086054in}}{\pgfqpoint{1.440096in}{2.096653in}}{\pgfqpoint{1.432282in}{2.104466in}}%
\pgfpathcurveto{\pgfqpoint{1.424468in}{2.112280in}}{\pgfqpoint{1.413869in}{2.116670in}}{\pgfqpoint{1.402819in}{2.116670in}}%
\pgfpathcurveto{\pgfqpoint{1.391769in}{2.116670in}}{\pgfqpoint{1.381170in}{2.112280in}}{\pgfqpoint{1.373357in}{2.104466in}}%
\pgfpathcurveto{\pgfqpoint{1.365543in}{2.096653in}}{\pgfqpoint{1.361153in}{2.086054in}}{\pgfqpoint{1.361153in}{2.075004in}}%
\pgfpathcurveto{\pgfqpoint{1.361153in}{2.063953in}}{\pgfqpoint{1.365543in}{2.053354in}}{\pgfqpoint{1.373357in}{2.045541in}}%
\pgfpathcurveto{\pgfqpoint{1.381170in}{2.037727in}}{\pgfqpoint{1.391769in}{2.033337in}}{\pgfqpoint{1.402819in}{2.033337in}}%
\pgfpathlineto{\pgfqpoint{1.402819in}{2.033337in}}%
\pgfpathclose%
\pgfusepath{stroke}%
\end{pgfscope}%
\begin{pgfscope}%
\pgfpathrectangle{\pgfqpoint{0.494722in}{0.437222in}}{\pgfqpoint{6.275590in}{5.159444in}}%
\pgfusepath{clip}%
\pgfsetbuttcap%
\pgfsetroundjoin%
\pgfsetlinewidth{1.003750pt}%
\definecolor{currentstroke}{rgb}{0.827451,0.827451,0.827451}%
\pgfsetstrokecolor{currentstroke}%
\pgfsetstrokeopacity{0.800000}%
\pgfsetdash{}{0pt}%
\pgfpathmoveto{\pgfqpoint{1.381781in}{2.056420in}}%
\pgfpathcurveto{\pgfqpoint{1.392831in}{2.056420in}}{\pgfqpoint{1.403430in}{2.060810in}}{\pgfqpoint{1.411244in}{2.068624in}}%
\pgfpathcurveto{\pgfqpoint{1.419057in}{2.076437in}}{\pgfqpoint{1.423448in}{2.087036in}}{\pgfqpoint{1.423448in}{2.098087in}}%
\pgfpathcurveto{\pgfqpoint{1.423448in}{2.109137in}}{\pgfqpoint{1.419057in}{2.119736in}}{\pgfqpoint{1.411244in}{2.127549in}}%
\pgfpathcurveto{\pgfqpoint{1.403430in}{2.135363in}}{\pgfqpoint{1.392831in}{2.139753in}}{\pgfqpoint{1.381781in}{2.139753in}}%
\pgfpathcurveto{\pgfqpoint{1.370731in}{2.139753in}}{\pgfqpoint{1.360132in}{2.135363in}}{\pgfqpoint{1.352318in}{2.127549in}}%
\pgfpathcurveto{\pgfqpoint{1.344505in}{2.119736in}}{\pgfqpoint{1.340114in}{2.109137in}}{\pgfqpoint{1.340114in}{2.098087in}}%
\pgfpathcurveto{\pgfqpoint{1.340114in}{2.087036in}}{\pgfqpoint{1.344505in}{2.076437in}}{\pgfqpoint{1.352318in}{2.068624in}}%
\pgfpathcurveto{\pgfqpoint{1.360132in}{2.060810in}}{\pgfqpoint{1.370731in}{2.056420in}}{\pgfqpoint{1.381781in}{2.056420in}}%
\pgfpathlineto{\pgfqpoint{1.381781in}{2.056420in}}%
\pgfpathclose%
\pgfusepath{stroke}%
\end{pgfscope}%
\begin{pgfscope}%
\pgfpathrectangle{\pgfqpoint{0.494722in}{0.437222in}}{\pgfqpoint{6.275590in}{5.159444in}}%
\pgfusepath{clip}%
\pgfsetbuttcap%
\pgfsetroundjoin%
\pgfsetlinewidth{1.003750pt}%
\definecolor{currentstroke}{rgb}{0.827451,0.827451,0.827451}%
\pgfsetstrokecolor{currentstroke}%
\pgfsetstrokeopacity{0.800000}%
\pgfsetdash{}{0pt}%
\pgfpathmoveto{\pgfqpoint{4.270074in}{0.511112in}}%
\pgfpathcurveto{\pgfqpoint{4.281124in}{0.511112in}}{\pgfqpoint{4.291723in}{0.515502in}}{\pgfqpoint{4.299537in}{0.523316in}}%
\pgfpathcurveto{\pgfqpoint{4.307351in}{0.531129in}}{\pgfqpoint{4.311741in}{0.541728in}}{\pgfqpoint{4.311741in}{0.552778in}}%
\pgfpathcurveto{\pgfqpoint{4.311741in}{0.563829in}}{\pgfqpoint{4.307351in}{0.574428in}}{\pgfqpoint{4.299537in}{0.582241in}}%
\pgfpathcurveto{\pgfqpoint{4.291723in}{0.590055in}}{\pgfqpoint{4.281124in}{0.594445in}}{\pgfqpoint{4.270074in}{0.594445in}}%
\pgfpathcurveto{\pgfqpoint{4.259024in}{0.594445in}}{\pgfqpoint{4.248425in}{0.590055in}}{\pgfqpoint{4.240611in}{0.582241in}}%
\pgfpathcurveto{\pgfqpoint{4.232798in}{0.574428in}}{\pgfqpoint{4.228407in}{0.563829in}}{\pgfqpoint{4.228407in}{0.552778in}}%
\pgfpathcurveto{\pgfqpoint{4.228407in}{0.541728in}}{\pgfqpoint{4.232798in}{0.531129in}}{\pgfqpoint{4.240611in}{0.523316in}}%
\pgfpathcurveto{\pgfqpoint{4.248425in}{0.515502in}}{\pgfqpoint{4.259024in}{0.511112in}}{\pgfqpoint{4.270074in}{0.511112in}}%
\pgfpathlineto{\pgfqpoint{4.270074in}{0.511112in}}%
\pgfpathclose%
\pgfusepath{stroke}%
\end{pgfscope}%
\begin{pgfscope}%
\pgfpathrectangle{\pgfqpoint{0.494722in}{0.437222in}}{\pgfqpoint{6.275590in}{5.159444in}}%
\pgfusepath{clip}%
\pgfsetbuttcap%
\pgfsetroundjoin%
\pgfsetlinewidth{1.003750pt}%
\definecolor{currentstroke}{rgb}{0.827451,0.827451,0.827451}%
\pgfsetstrokecolor{currentstroke}%
\pgfsetstrokeopacity{0.800000}%
\pgfsetdash{}{0pt}%
\pgfpathmoveto{\pgfqpoint{2.224297in}{1.325142in}}%
\pgfpathcurveto{\pgfqpoint{2.235347in}{1.325142in}}{\pgfqpoint{2.245946in}{1.329532in}}{\pgfqpoint{2.253760in}{1.337346in}}%
\pgfpathcurveto{\pgfqpoint{2.261573in}{1.345159in}}{\pgfqpoint{2.265964in}{1.355758in}}{\pgfqpoint{2.265964in}{1.366809in}}%
\pgfpathcurveto{\pgfqpoint{2.265964in}{1.377859in}}{\pgfqpoint{2.261573in}{1.388458in}}{\pgfqpoint{2.253760in}{1.396271in}}%
\pgfpathcurveto{\pgfqpoint{2.245946in}{1.404085in}}{\pgfqpoint{2.235347in}{1.408475in}}{\pgfqpoint{2.224297in}{1.408475in}}%
\pgfpathcurveto{\pgfqpoint{2.213247in}{1.408475in}}{\pgfqpoint{2.202648in}{1.404085in}}{\pgfqpoint{2.194834in}{1.396271in}}%
\pgfpathcurveto{\pgfqpoint{2.187020in}{1.388458in}}{\pgfqpoint{2.182630in}{1.377859in}}{\pgfqpoint{2.182630in}{1.366809in}}%
\pgfpathcurveto{\pgfqpoint{2.182630in}{1.355758in}}{\pgfqpoint{2.187020in}{1.345159in}}{\pgfqpoint{2.194834in}{1.337346in}}%
\pgfpathcurveto{\pgfqpoint{2.202648in}{1.329532in}}{\pgfqpoint{2.213247in}{1.325142in}}{\pgfqpoint{2.224297in}{1.325142in}}%
\pgfpathlineto{\pgfqpoint{2.224297in}{1.325142in}}%
\pgfpathclose%
\pgfusepath{stroke}%
\end{pgfscope}%
\begin{pgfscope}%
\pgfpathrectangle{\pgfqpoint{0.494722in}{0.437222in}}{\pgfqpoint{6.275590in}{5.159444in}}%
\pgfusepath{clip}%
\pgfsetbuttcap%
\pgfsetroundjoin%
\pgfsetlinewidth{1.003750pt}%
\definecolor{currentstroke}{rgb}{0.827451,0.827451,0.827451}%
\pgfsetstrokecolor{currentstroke}%
\pgfsetstrokeopacity{0.800000}%
\pgfsetdash{}{0pt}%
\pgfpathmoveto{\pgfqpoint{4.229480in}{0.518110in}}%
\pgfpathcurveto{\pgfqpoint{4.240530in}{0.518110in}}{\pgfqpoint{4.251129in}{0.522500in}}{\pgfqpoint{4.258942in}{0.530314in}}%
\pgfpathcurveto{\pgfqpoint{4.266756in}{0.538127in}}{\pgfqpoint{4.271146in}{0.548726in}}{\pgfqpoint{4.271146in}{0.559776in}}%
\pgfpathcurveto{\pgfqpoint{4.271146in}{0.570827in}}{\pgfqpoint{4.266756in}{0.581426in}}{\pgfqpoint{4.258942in}{0.589239in}}%
\pgfpathcurveto{\pgfqpoint{4.251129in}{0.597053in}}{\pgfqpoint{4.240530in}{0.601443in}}{\pgfqpoint{4.229480in}{0.601443in}}%
\pgfpathcurveto{\pgfqpoint{4.218429in}{0.601443in}}{\pgfqpoint{4.207830in}{0.597053in}}{\pgfqpoint{4.200017in}{0.589239in}}%
\pgfpathcurveto{\pgfqpoint{4.192203in}{0.581426in}}{\pgfqpoint{4.187813in}{0.570827in}}{\pgfqpoint{4.187813in}{0.559776in}}%
\pgfpathcurveto{\pgfqpoint{4.187813in}{0.548726in}}{\pgfqpoint{4.192203in}{0.538127in}}{\pgfqpoint{4.200017in}{0.530314in}}%
\pgfpathcurveto{\pgfqpoint{4.207830in}{0.522500in}}{\pgfqpoint{4.218429in}{0.518110in}}{\pgfqpoint{4.229480in}{0.518110in}}%
\pgfpathlineto{\pgfqpoint{4.229480in}{0.518110in}}%
\pgfpathclose%
\pgfusepath{stroke}%
\end{pgfscope}%
\begin{pgfscope}%
\pgfpathrectangle{\pgfqpoint{0.494722in}{0.437222in}}{\pgfqpoint{6.275590in}{5.159444in}}%
\pgfusepath{clip}%
\pgfsetbuttcap%
\pgfsetroundjoin%
\pgfsetlinewidth{1.003750pt}%
\definecolor{currentstroke}{rgb}{0.827451,0.827451,0.827451}%
\pgfsetstrokecolor{currentstroke}%
\pgfsetstrokeopacity{0.800000}%
\pgfsetdash{}{0pt}%
\pgfpathmoveto{\pgfqpoint{1.545871in}{1.889457in}}%
\pgfpathcurveto{\pgfqpoint{1.556921in}{1.889457in}}{\pgfqpoint{1.567520in}{1.893848in}}{\pgfqpoint{1.575334in}{1.901661in}}%
\pgfpathcurveto{\pgfqpoint{1.583148in}{1.909475in}}{\pgfqpoint{1.587538in}{1.920074in}}{\pgfqpoint{1.587538in}{1.931124in}}%
\pgfpathcurveto{\pgfqpoint{1.587538in}{1.942174in}}{\pgfqpoint{1.583148in}{1.952773in}}{\pgfqpoint{1.575334in}{1.960587in}}%
\pgfpathcurveto{\pgfqpoint{1.567520in}{1.968400in}}{\pgfqpoint{1.556921in}{1.972791in}}{\pgfqpoint{1.545871in}{1.972791in}}%
\pgfpathcurveto{\pgfqpoint{1.534821in}{1.972791in}}{\pgfqpoint{1.524222in}{1.968400in}}{\pgfqpoint{1.516408in}{1.960587in}}%
\pgfpathcurveto{\pgfqpoint{1.508595in}{1.952773in}}{\pgfqpoint{1.504204in}{1.942174in}}{\pgfqpoint{1.504204in}{1.931124in}}%
\pgfpathcurveto{\pgfqpoint{1.504204in}{1.920074in}}{\pgfqpoint{1.508595in}{1.909475in}}{\pgfqpoint{1.516408in}{1.901661in}}%
\pgfpathcurveto{\pgfqpoint{1.524222in}{1.893848in}}{\pgfqpoint{1.534821in}{1.889457in}}{\pgfqpoint{1.545871in}{1.889457in}}%
\pgfpathlineto{\pgfqpoint{1.545871in}{1.889457in}}%
\pgfpathclose%
\pgfusepath{stroke}%
\end{pgfscope}%
\begin{pgfscope}%
\pgfpathrectangle{\pgfqpoint{0.494722in}{0.437222in}}{\pgfqpoint{6.275590in}{5.159444in}}%
\pgfusepath{clip}%
\pgfsetbuttcap%
\pgfsetroundjoin%
\pgfsetlinewidth{1.003750pt}%
\definecolor{currentstroke}{rgb}{0.827451,0.827451,0.827451}%
\pgfsetstrokecolor{currentstroke}%
\pgfsetstrokeopacity{0.800000}%
\pgfsetdash{}{0pt}%
\pgfpathmoveto{\pgfqpoint{0.841205in}{2.904303in}}%
\pgfpathcurveto{\pgfqpoint{0.852255in}{2.904303in}}{\pgfqpoint{0.862854in}{2.908693in}}{\pgfqpoint{0.870668in}{2.916506in}}%
\pgfpathcurveto{\pgfqpoint{0.878481in}{2.924320in}}{\pgfqpoint{0.882872in}{2.934919in}}{\pgfqpoint{0.882872in}{2.945969in}}%
\pgfpathcurveto{\pgfqpoint{0.882872in}{2.957019in}}{\pgfqpoint{0.878481in}{2.967618in}}{\pgfqpoint{0.870668in}{2.975432in}}%
\pgfpathcurveto{\pgfqpoint{0.862854in}{2.983246in}}{\pgfqpoint{0.852255in}{2.987636in}}{\pgfqpoint{0.841205in}{2.987636in}}%
\pgfpathcurveto{\pgfqpoint{0.830155in}{2.987636in}}{\pgfqpoint{0.819556in}{2.983246in}}{\pgfqpoint{0.811742in}{2.975432in}}%
\pgfpathcurveto{\pgfqpoint{0.803928in}{2.967618in}}{\pgfqpoint{0.799538in}{2.957019in}}{\pgfqpoint{0.799538in}{2.945969in}}%
\pgfpathcurveto{\pgfqpoint{0.799538in}{2.934919in}}{\pgfqpoint{0.803928in}{2.924320in}}{\pgfqpoint{0.811742in}{2.916506in}}%
\pgfpathcurveto{\pgfqpoint{0.819556in}{2.908693in}}{\pgfqpoint{0.830155in}{2.904303in}}{\pgfqpoint{0.841205in}{2.904303in}}%
\pgfpathlineto{\pgfqpoint{0.841205in}{2.904303in}}%
\pgfpathclose%
\pgfusepath{stroke}%
\end{pgfscope}%
\begin{pgfscope}%
\pgfpathrectangle{\pgfqpoint{0.494722in}{0.437222in}}{\pgfqpoint{6.275590in}{5.159444in}}%
\pgfusepath{clip}%
\pgfsetbuttcap%
\pgfsetroundjoin%
\pgfsetlinewidth{1.003750pt}%
\definecolor{currentstroke}{rgb}{0.827451,0.827451,0.827451}%
\pgfsetstrokecolor{currentstroke}%
\pgfsetstrokeopacity{0.800000}%
\pgfsetdash{}{0pt}%
\pgfpathmoveto{\pgfqpoint{5.757662in}{0.395992in}}%
\pgfpathcurveto{\pgfqpoint{5.768712in}{0.395992in}}{\pgfqpoint{5.779311in}{0.400382in}}{\pgfqpoint{5.787125in}{0.408196in}}%
\pgfpathcurveto{\pgfqpoint{5.794939in}{0.416009in}}{\pgfqpoint{5.799329in}{0.426608in}}{\pgfqpoint{5.799329in}{0.437658in}}%
\pgfpathcurveto{\pgfqpoint{5.799329in}{0.448709in}}{\pgfqpoint{5.794939in}{0.459308in}}{\pgfqpoint{5.787125in}{0.467121in}}%
\pgfpathcurveto{\pgfqpoint{5.779311in}{0.474935in}}{\pgfqpoint{5.768712in}{0.479325in}}{\pgfqpoint{5.757662in}{0.479325in}}%
\pgfpathcurveto{\pgfqpoint{5.746612in}{0.479325in}}{\pgfqpoint{5.736013in}{0.474935in}}{\pgfqpoint{5.728199in}{0.467121in}}%
\pgfpathcurveto{\pgfqpoint{5.720386in}{0.459308in}}{\pgfqpoint{5.715996in}{0.448709in}}{\pgfqpoint{5.715996in}{0.437658in}}%
\pgfpathcurveto{\pgfqpoint{5.715996in}{0.426608in}}{\pgfqpoint{5.720386in}{0.416009in}}{\pgfqpoint{5.728199in}{0.408196in}}%
\pgfpathcurveto{\pgfqpoint{5.736013in}{0.400382in}}{\pgfqpoint{5.746612in}{0.395992in}}{\pgfqpoint{5.757662in}{0.395992in}}%
\pgfusepath{stroke}%
\end{pgfscope}%
\begin{pgfscope}%
\pgfpathrectangle{\pgfqpoint{0.494722in}{0.437222in}}{\pgfqpoint{6.275590in}{5.159444in}}%
\pgfusepath{clip}%
\pgfsetbuttcap%
\pgfsetroundjoin%
\pgfsetlinewidth{1.003750pt}%
\definecolor{currentstroke}{rgb}{0.827451,0.827451,0.827451}%
\pgfsetstrokecolor{currentstroke}%
\pgfsetstrokeopacity{0.800000}%
\pgfsetdash{}{0pt}%
\pgfpathmoveto{\pgfqpoint{0.495079in}{4.572987in}}%
\pgfpathcurveto{\pgfqpoint{0.506130in}{4.572987in}}{\pgfqpoint{0.516729in}{4.577378in}}{\pgfqpoint{0.524542in}{4.585191in}}%
\pgfpathcurveto{\pgfqpoint{0.532356in}{4.593005in}}{\pgfqpoint{0.536746in}{4.603604in}}{\pgfqpoint{0.536746in}{4.614654in}}%
\pgfpathcurveto{\pgfqpoint{0.536746in}{4.625704in}}{\pgfqpoint{0.532356in}{4.636303in}}{\pgfqpoint{0.524542in}{4.644117in}}%
\pgfpathcurveto{\pgfqpoint{0.516729in}{4.651930in}}{\pgfqpoint{0.506130in}{4.656321in}}{\pgfqpoint{0.495079in}{4.656321in}}%
\pgfpathcurveto{\pgfqpoint{0.484029in}{4.656321in}}{\pgfqpoint{0.473430in}{4.651930in}}{\pgfqpoint{0.465617in}{4.644117in}}%
\pgfpathcurveto{\pgfqpoint{0.457803in}{4.636303in}}{\pgfqpoint{0.453413in}{4.625704in}}{\pgfqpoint{0.453413in}{4.614654in}}%
\pgfpathcurveto{\pgfqpoint{0.453413in}{4.603604in}}{\pgfqpoint{0.457803in}{4.593005in}}{\pgfqpoint{0.465617in}{4.585191in}}%
\pgfpathcurveto{\pgfqpoint{0.473430in}{4.577378in}}{\pgfqpoint{0.484029in}{4.572987in}}{\pgfqpoint{0.495079in}{4.572987in}}%
\pgfpathlineto{\pgfqpoint{0.495079in}{4.572987in}}%
\pgfpathclose%
\pgfusepath{stroke}%
\end{pgfscope}%
\begin{pgfscope}%
\pgfpathrectangle{\pgfqpoint{0.494722in}{0.437222in}}{\pgfqpoint{6.275590in}{5.159444in}}%
\pgfusepath{clip}%
\pgfsetbuttcap%
\pgfsetroundjoin%
\pgfsetlinewidth{1.003750pt}%
\definecolor{currentstroke}{rgb}{0.827451,0.827451,0.827451}%
\pgfsetstrokecolor{currentstroke}%
\pgfsetstrokeopacity{0.800000}%
\pgfsetdash{}{0pt}%
\pgfpathmoveto{\pgfqpoint{2.990259in}{0.911738in}}%
\pgfpathcurveto{\pgfqpoint{3.001309in}{0.911738in}}{\pgfqpoint{3.011908in}{0.916128in}}{\pgfqpoint{3.019722in}{0.923942in}}%
\pgfpathcurveto{\pgfqpoint{3.027536in}{0.931755in}}{\pgfqpoint{3.031926in}{0.942354in}}{\pgfqpoint{3.031926in}{0.953404in}}%
\pgfpathcurveto{\pgfqpoint{3.031926in}{0.964455in}}{\pgfqpoint{3.027536in}{0.975054in}}{\pgfqpoint{3.019722in}{0.982867in}}%
\pgfpathcurveto{\pgfqpoint{3.011908in}{0.990681in}}{\pgfqpoint{3.001309in}{0.995071in}}{\pgfqpoint{2.990259in}{0.995071in}}%
\pgfpathcurveto{\pgfqpoint{2.979209in}{0.995071in}}{\pgfqpoint{2.968610in}{0.990681in}}{\pgfqpoint{2.960796in}{0.982867in}}%
\pgfpathcurveto{\pgfqpoint{2.952983in}{0.975054in}}{\pgfqpoint{2.948593in}{0.964455in}}{\pgfqpoint{2.948593in}{0.953404in}}%
\pgfpathcurveto{\pgfqpoint{2.948593in}{0.942354in}}{\pgfqpoint{2.952983in}{0.931755in}}{\pgfqpoint{2.960796in}{0.923942in}}%
\pgfpathcurveto{\pgfqpoint{2.968610in}{0.916128in}}{\pgfqpoint{2.979209in}{0.911738in}}{\pgfqpoint{2.990259in}{0.911738in}}%
\pgfpathlineto{\pgfqpoint{2.990259in}{0.911738in}}%
\pgfpathclose%
\pgfusepath{stroke}%
\end{pgfscope}%
\begin{pgfscope}%
\pgfpathrectangle{\pgfqpoint{0.494722in}{0.437222in}}{\pgfqpoint{6.275590in}{5.159444in}}%
\pgfusepath{clip}%
\pgfsetbuttcap%
\pgfsetroundjoin%
\pgfsetlinewidth{1.003750pt}%
\definecolor{currentstroke}{rgb}{0.827451,0.827451,0.827451}%
\pgfsetstrokecolor{currentstroke}%
\pgfsetstrokeopacity{0.800000}%
\pgfsetdash{}{0pt}%
\pgfpathmoveto{\pgfqpoint{4.847728in}{0.439866in}}%
\pgfpathcurveto{\pgfqpoint{4.858778in}{0.439866in}}{\pgfqpoint{4.869377in}{0.444256in}}{\pgfqpoint{4.877190in}{0.452070in}}%
\pgfpathcurveto{\pgfqpoint{4.885004in}{0.459884in}}{\pgfqpoint{4.889394in}{0.470483in}}{\pgfqpoint{4.889394in}{0.481533in}}%
\pgfpathcurveto{\pgfqpoint{4.889394in}{0.492583in}}{\pgfqpoint{4.885004in}{0.503182in}}{\pgfqpoint{4.877190in}{0.510995in}}%
\pgfpathcurveto{\pgfqpoint{4.869377in}{0.518809in}}{\pgfqpoint{4.858778in}{0.523199in}}{\pgfqpoint{4.847728in}{0.523199in}}%
\pgfpathcurveto{\pgfqpoint{4.836678in}{0.523199in}}{\pgfqpoint{4.826078in}{0.518809in}}{\pgfqpoint{4.818265in}{0.510995in}}%
\pgfpathcurveto{\pgfqpoint{4.810451in}{0.503182in}}{\pgfqpoint{4.806061in}{0.492583in}}{\pgfqpoint{4.806061in}{0.481533in}}%
\pgfpathcurveto{\pgfqpoint{4.806061in}{0.470483in}}{\pgfqpoint{4.810451in}{0.459884in}}{\pgfqpoint{4.818265in}{0.452070in}}%
\pgfpathcurveto{\pgfqpoint{4.826078in}{0.444256in}}{\pgfqpoint{4.836678in}{0.439866in}}{\pgfqpoint{4.847728in}{0.439866in}}%
\pgfpathlineto{\pgfqpoint{4.847728in}{0.439866in}}%
\pgfpathclose%
\pgfusepath{stroke}%
\end{pgfscope}%
\begin{pgfscope}%
\pgfpathrectangle{\pgfqpoint{0.494722in}{0.437222in}}{\pgfqpoint{6.275590in}{5.159444in}}%
\pgfusepath{clip}%
\pgfsetbuttcap%
\pgfsetroundjoin%
\pgfsetlinewidth{1.003750pt}%
\definecolor{currentstroke}{rgb}{0.827451,0.827451,0.827451}%
\pgfsetstrokecolor{currentstroke}%
\pgfsetstrokeopacity{0.800000}%
\pgfsetdash{}{0pt}%
\pgfpathmoveto{\pgfqpoint{5.060352in}{0.429754in}}%
\pgfpathcurveto{\pgfqpoint{5.071402in}{0.429754in}}{\pgfqpoint{5.082001in}{0.434145in}}{\pgfqpoint{5.089815in}{0.441958in}}%
\pgfpathcurveto{\pgfqpoint{5.097629in}{0.449772in}}{\pgfqpoint{5.102019in}{0.460371in}}{\pgfqpoint{5.102019in}{0.471421in}}%
\pgfpathcurveto{\pgfqpoint{5.102019in}{0.482471in}}{\pgfqpoint{5.097629in}{0.493070in}}{\pgfqpoint{5.089815in}{0.500884in}}%
\pgfpathcurveto{\pgfqpoint{5.082001in}{0.508697in}}{\pgfqpoint{5.071402in}{0.513088in}}{\pgfqpoint{5.060352in}{0.513088in}}%
\pgfpathcurveto{\pgfqpoint{5.049302in}{0.513088in}}{\pgfqpoint{5.038703in}{0.508697in}}{\pgfqpoint{5.030890in}{0.500884in}}%
\pgfpathcurveto{\pgfqpoint{5.023076in}{0.493070in}}{\pgfqpoint{5.018686in}{0.482471in}}{\pgfqpoint{5.018686in}{0.471421in}}%
\pgfpathcurveto{\pgfqpoint{5.018686in}{0.460371in}}{\pgfqpoint{5.023076in}{0.449772in}}{\pgfqpoint{5.030890in}{0.441958in}}%
\pgfpathcurveto{\pgfqpoint{5.038703in}{0.434145in}}{\pgfqpoint{5.049302in}{0.429754in}}{\pgfqpoint{5.060352in}{0.429754in}}%
\pgfpathlineto{\pgfqpoint{5.060352in}{0.429754in}}%
\pgfpathclose%
\pgfusepath{stroke}%
\end{pgfscope}%
\begin{pgfscope}%
\pgfpathrectangle{\pgfqpoint{0.494722in}{0.437222in}}{\pgfqpoint{6.275590in}{5.159444in}}%
\pgfusepath{clip}%
\pgfsetbuttcap%
\pgfsetroundjoin%
\pgfsetlinewidth{1.003750pt}%
\definecolor{currentstroke}{rgb}{0.827451,0.827451,0.827451}%
\pgfsetstrokecolor{currentstroke}%
\pgfsetstrokeopacity{0.800000}%
\pgfsetdash{}{0pt}%
\pgfpathmoveto{\pgfqpoint{3.527306in}{0.699588in}}%
\pgfpathcurveto{\pgfqpoint{3.538356in}{0.699588in}}{\pgfqpoint{3.548955in}{0.703979in}}{\pgfqpoint{3.556769in}{0.711792in}}%
\pgfpathcurveto{\pgfqpoint{3.564583in}{0.719606in}}{\pgfqpoint{3.568973in}{0.730205in}}{\pgfqpoint{3.568973in}{0.741255in}}%
\pgfpathcurveto{\pgfqpoint{3.568973in}{0.752305in}}{\pgfqpoint{3.564583in}{0.762904in}}{\pgfqpoint{3.556769in}{0.770718in}}%
\pgfpathcurveto{\pgfqpoint{3.548955in}{0.778532in}}{\pgfqpoint{3.538356in}{0.782922in}}{\pgfqpoint{3.527306in}{0.782922in}}%
\pgfpathcurveto{\pgfqpoint{3.516256in}{0.782922in}}{\pgfqpoint{3.505657in}{0.778532in}}{\pgfqpoint{3.497843in}{0.770718in}}%
\pgfpathcurveto{\pgfqpoint{3.490030in}{0.762904in}}{\pgfqpoint{3.485640in}{0.752305in}}{\pgfqpoint{3.485640in}{0.741255in}}%
\pgfpathcurveto{\pgfqpoint{3.485640in}{0.730205in}}{\pgfqpoint{3.490030in}{0.719606in}}{\pgfqpoint{3.497843in}{0.711792in}}%
\pgfpathcurveto{\pgfqpoint{3.505657in}{0.703979in}}{\pgfqpoint{3.516256in}{0.699588in}}{\pgfqpoint{3.527306in}{0.699588in}}%
\pgfpathlineto{\pgfqpoint{3.527306in}{0.699588in}}%
\pgfpathclose%
\pgfusepath{stroke}%
\end{pgfscope}%
\begin{pgfscope}%
\pgfpathrectangle{\pgfqpoint{0.494722in}{0.437222in}}{\pgfqpoint{6.275590in}{5.159444in}}%
\pgfusepath{clip}%
\pgfsetbuttcap%
\pgfsetroundjoin%
\pgfsetlinewidth{1.003750pt}%
\definecolor{currentstroke}{rgb}{0.827451,0.827451,0.827451}%
\pgfsetstrokecolor{currentstroke}%
\pgfsetstrokeopacity{0.800000}%
\pgfsetdash{}{0pt}%
\pgfpathmoveto{\pgfqpoint{0.525453in}{4.078418in}}%
\pgfpathcurveto{\pgfqpoint{0.536503in}{4.078418in}}{\pgfqpoint{0.547102in}{4.082808in}}{\pgfqpoint{0.554915in}{4.090621in}}%
\pgfpathcurveto{\pgfqpoint{0.562729in}{4.098435in}}{\pgfqpoint{0.567119in}{4.109034in}}{\pgfqpoint{0.567119in}{4.120084in}}%
\pgfpathcurveto{\pgfqpoint{0.567119in}{4.131134in}}{\pgfqpoint{0.562729in}{4.141733in}}{\pgfqpoint{0.554915in}{4.149547in}}%
\pgfpathcurveto{\pgfqpoint{0.547102in}{4.157361in}}{\pgfqpoint{0.536503in}{4.161751in}}{\pgfqpoint{0.525453in}{4.161751in}}%
\pgfpathcurveto{\pgfqpoint{0.514403in}{4.161751in}}{\pgfqpoint{0.503804in}{4.157361in}}{\pgfqpoint{0.495990in}{4.149547in}}%
\pgfpathcurveto{\pgfqpoint{0.488176in}{4.141733in}}{\pgfqpoint{0.483786in}{4.131134in}}{\pgfqpoint{0.483786in}{4.120084in}}%
\pgfpathcurveto{\pgfqpoint{0.483786in}{4.109034in}}{\pgfqpoint{0.488176in}{4.098435in}}{\pgfqpoint{0.495990in}{4.090621in}}%
\pgfpathcurveto{\pgfqpoint{0.503804in}{4.082808in}}{\pgfqpoint{0.514403in}{4.078418in}}{\pgfqpoint{0.525453in}{4.078418in}}%
\pgfpathlineto{\pgfqpoint{0.525453in}{4.078418in}}%
\pgfpathclose%
\pgfusepath{stroke}%
\end{pgfscope}%
\begin{pgfscope}%
\pgfpathrectangle{\pgfqpoint{0.494722in}{0.437222in}}{\pgfqpoint{6.275590in}{5.159444in}}%
\pgfusepath{clip}%
\pgfsetbuttcap%
\pgfsetroundjoin%
\pgfsetlinewidth{1.003750pt}%
\definecolor{currentstroke}{rgb}{0.827451,0.827451,0.827451}%
\pgfsetstrokecolor{currentstroke}%
\pgfsetstrokeopacity{0.800000}%
\pgfsetdash{}{0pt}%
\pgfpathmoveto{\pgfqpoint{2.856780in}{0.965441in}}%
\pgfpathcurveto{\pgfqpoint{2.867831in}{0.965441in}}{\pgfqpoint{2.878430in}{0.969831in}}{\pgfqpoint{2.886243in}{0.977645in}}%
\pgfpathcurveto{\pgfqpoint{2.894057in}{0.985458in}}{\pgfqpoint{2.898447in}{0.996058in}}{\pgfqpoint{2.898447in}{1.007108in}}%
\pgfpathcurveto{\pgfqpoint{2.898447in}{1.018158in}}{\pgfqpoint{2.894057in}{1.028757in}}{\pgfqpoint{2.886243in}{1.036570in}}%
\pgfpathcurveto{\pgfqpoint{2.878430in}{1.044384in}}{\pgfqpoint{2.867831in}{1.048774in}}{\pgfqpoint{2.856780in}{1.048774in}}%
\pgfpathcurveto{\pgfqpoint{2.845730in}{1.048774in}}{\pgfqpoint{2.835131in}{1.044384in}}{\pgfqpoint{2.827318in}{1.036570in}}%
\pgfpathcurveto{\pgfqpoint{2.819504in}{1.028757in}}{\pgfqpoint{2.815114in}{1.018158in}}{\pgfqpoint{2.815114in}{1.007108in}}%
\pgfpathcurveto{\pgfqpoint{2.815114in}{0.996058in}}{\pgfqpoint{2.819504in}{0.985458in}}{\pgfqpoint{2.827318in}{0.977645in}}%
\pgfpathcurveto{\pgfqpoint{2.835131in}{0.969831in}}{\pgfqpoint{2.845730in}{0.965441in}}{\pgfqpoint{2.856780in}{0.965441in}}%
\pgfpathlineto{\pgfqpoint{2.856780in}{0.965441in}}%
\pgfpathclose%
\pgfusepath{stroke}%
\end{pgfscope}%
\begin{pgfscope}%
\pgfpathrectangle{\pgfqpoint{0.494722in}{0.437222in}}{\pgfqpoint{6.275590in}{5.159444in}}%
\pgfusepath{clip}%
\pgfsetbuttcap%
\pgfsetroundjoin%
\pgfsetlinewidth{1.003750pt}%
\definecolor{currentstroke}{rgb}{0.827451,0.827451,0.827451}%
\pgfsetstrokecolor{currentstroke}%
\pgfsetstrokeopacity{0.800000}%
\pgfsetdash{}{0pt}%
\pgfpathmoveto{\pgfqpoint{4.067519in}{0.560512in}}%
\pgfpathcurveto{\pgfqpoint{4.078569in}{0.560512in}}{\pgfqpoint{4.089168in}{0.564902in}}{\pgfqpoint{4.096982in}{0.572715in}}%
\pgfpathcurveto{\pgfqpoint{4.104796in}{0.580529in}}{\pgfqpoint{4.109186in}{0.591128in}}{\pgfqpoint{4.109186in}{0.602178in}}%
\pgfpathcurveto{\pgfqpoint{4.109186in}{0.613228in}}{\pgfqpoint{4.104796in}{0.623827in}}{\pgfqpoint{4.096982in}{0.631641in}}%
\pgfpathcurveto{\pgfqpoint{4.089168in}{0.639455in}}{\pgfqpoint{4.078569in}{0.643845in}}{\pgfqpoint{4.067519in}{0.643845in}}%
\pgfpathcurveto{\pgfqpoint{4.056469in}{0.643845in}}{\pgfqpoint{4.045870in}{0.639455in}}{\pgfqpoint{4.038056in}{0.631641in}}%
\pgfpathcurveto{\pgfqpoint{4.030243in}{0.623827in}}{\pgfqpoint{4.025853in}{0.613228in}}{\pgfqpoint{4.025853in}{0.602178in}}%
\pgfpathcurveto{\pgfqpoint{4.025853in}{0.591128in}}{\pgfqpoint{4.030243in}{0.580529in}}{\pgfqpoint{4.038056in}{0.572715in}}%
\pgfpathcurveto{\pgfqpoint{4.045870in}{0.564902in}}{\pgfqpoint{4.056469in}{0.560512in}}{\pgfqpoint{4.067519in}{0.560512in}}%
\pgfpathlineto{\pgfqpoint{4.067519in}{0.560512in}}%
\pgfpathclose%
\pgfusepath{stroke}%
\end{pgfscope}%
\begin{pgfscope}%
\pgfpathrectangle{\pgfqpoint{0.494722in}{0.437222in}}{\pgfqpoint{6.275590in}{5.159444in}}%
\pgfusepath{clip}%
\pgfsetbuttcap%
\pgfsetroundjoin%
\pgfsetlinewidth{1.003750pt}%
\definecolor{currentstroke}{rgb}{0.827451,0.827451,0.827451}%
\pgfsetstrokecolor{currentstroke}%
\pgfsetstrokeopacity{0.800000}%
\pgfsetdash{}{0pt}%
\pgfpathmoveto{\pgfqpoint{3.697856in}{0.643246in}}%
\pgfpathcurveto{\pgfqpoint{3.708907in}{0.643246in}}{\pgfqpoint{3.719506in}{0.647637in}}{\pgfqpoint{3.727319in}{0.655450in}}%
\pgfpathcurveto{\pgfqpoint{3.735133in}{0.663264in}}{\pgfqpoint{3.739523in}{0.673863in}}{\pgfqpoint{3.739523in}{0.684913in}}%
\pgfpathcurveto{\pgfqpoint{3.739523in}{0.695963in}}{\pgfqpoint{3.735133in}{0.706562in}}{\pgfqpoint{3.727319in}{0.714376in}}%
\pgfpathcurveto{\pgfqpoint{3.719506in}{0.722189in}}{\pgfqpoint{3.708907in}{0.726580in}}{\pgfqpoint{3.697856in}{0.726580in}}%
\pgfpathcurveto{\pgfqpoint{3.686806in}{0.726580in}}{\pgfqpoint{3.676207in}{0.722189in}}{\pgfqpoint{3.668394in}{0.714376in}}%
\pgfpathcurveto{\pgfqpoint{3.660580in}{0.706562in}}{\pgfqpoint{3.656190in}{0.695963in}}{\pgfqpoint{3.656190in}{0.684913in}}%
\pgfpathcurveto{\pgfqpoint{3.656190in}{0.673863in}}{\pgfqpoint{3.660580in}{0.663264in}}{\pgfqpoint{3.668394in}{0.655450in}}%
\pgfpathcurveto{\pgfqpoint{3.676207in}{0.647637in}}{\pgfqpoint{3.686806in}{0.643246in}}{\pgfqpoint{3.697856in}{0.643246in}}%
\pgfpathlineto{\pgfqpoint{3.697856in}{0.643246in}}%
\pgfpathclose%
\pgfusepath{stroke}%
\end{pgfscope}%
\begin{pgfscope}%
\pgfpathrectangle{\pgfqpoint{0.494722in}{0.437222in}}{\pgfqpoint{6.275590in}{5.159444in}}%
\pgfusepath{clip}%
\pgfsetbuttcap%
\pgfsetroundjoin%
\pgfsetlinewidth{1.003750pt}%
\definecolor{currentstroke}{rgb}{0.827451,0.827451,0.827451}%
\pgfsetstrokecolor{currentstroke}%
\pgfsetstrokeopacity{0.800000}%
\pgfsetdash{}{0pt}%
\pgfpathmoveto{\pgfqpoint{1.113110in}{2.511719in}}%
\pgfpathcurveto{\pgfqpoint{1.124161in}{2.511719in}}{\pgfqpoint{1.134760in}{2.516109in}}{\pgfqpoint{1.142573in}{2.523923in}}%
\pgfpathcurveto{\pgfqpoint{1.150387in}{2.531736in}}{\pgfqpoint{1.154777in}{2.542335in}}{\pgfqpoint{1.154777in}{2.553385in}}%
\pgfpathcurveto{\pgfqpoint{1.154777in}{2.564435in}}{\pgfqpoint{1.150387in}{2.575034in}}{\pgfqpoint{1.142573in}{2.582848in}}%
\pgfpathcurveto{\pgfqpoint{1.134760in}{2.590662in}}{\pgfqpoint{1.124161in}{2.595052in}}{\pgfqpoint{1.113110in}{2.595052in}}%
\pgfpathcurveto{\pgfqpoint{1.102060in}{2.595052in}}{\pgfqpoint{1.091461in}{2.590662in}}{\pgfqpoint{1.083648in}{2.582848in}}%
\pgfpathcurveto{\pgfqpoint{1.075834in}{2.575034in}}{\pgfqpoint{1.071444in}{2.564435in}}{\pgfqpoint{1.071444in}{2.553385in}}%
\pgfpathcurveto{\pgfqpoint{1.071444in}{2.542335in}}{\pgfqpoint{1.075834in}{2.531736in}}{\pgfqpoint{1.083648in}{2.523923in}}%
\pgfpathcurveto{\pgfqpoint{1.091461in}{2.516109in}}{\pgfqpoint{1.102060in}{2.511719in}}{\pgfqpoint{1.113110in}{2.511719in}}%
\pgfpathlineto{\pgfqpoint{1.113110in}{2.511719in}}%
\pgfpathclose%
\pgfusepath{stroke}%
\end{pgfscope}%
\begin{pgfscope}%
\pgfpathrectangle{\pgfqpoint{0.494722in}{0.437222in}}{\pgfqpoint{6.275590in}{5.159444in}}%
\pgfusepath{clip}%
\pgfsetbuttcap%
\pgfsetroundjoin%
\pgfsetlinewidth{1.003750pt}%
\definecolor{currentstroke}{rgb}{0.827451,0.827451,0.827451}%
\pgfsetstrokecolor{currentstroke}%
\pgfsetstrokeopacity{0.800000}%
\pgfsetdash{}{0pt}%
\pgfpathmoveto{\pgfqpoint{0.513968in}{4.235716in}}%
\pgfpathcurveto{\pgfqpoint{0.525018in}{4.235716in}}{\pgfqpoint{0.535617in}{4.240106in}}{\pgfqpoint{0.543431in}{4.247920in}}%
\pgfpathcurveto{\pgfqpoint{0.551244in}{4.255733in}}{\pgfqpoint{0.555635in}{4.266332in}}{\pgfqpoint{0.555635in}{4.277382in}}%
\pgfpathcurveto{\pgfqpoint{0.555635in}{4.288433in}}{\pgfqpoint{0.551244in}{4.299032in}}{\pgfqpoint{0.543431in}{4.306845in}}%
\pgfpathcurveto{\pgfqpoint{0.535617in}{4.314659in}}{\pgfqpoint{0.525018in}{4.319049in}}{\pgfqpoint{0.513968in}{4.319049in}}%
\pgfpathcurveto{\pgfqpoint{0.502918in}{4.319049in}}{\pgfqpoint{0.492319in}{4.314659in}}{\pgfqpoint{0.484505in}{4.306845in}}%
\pgfpathcurveto{\pgfqpoint{0.476691in}{4.299032in}}{\pgfqpoint{0.472301in}{4.288433in}}{\pgfqpoint{0.472301in}{4.277382in}}%
\pgfpathcurveto{\pgfqpoint{0.472301in}{4.266332in}}{\pgfqpoint{0.476691in}{4.255733in}}{\pgfqpoint{0.484505in}{4.247920in}}%
\pgfpathcurveto{\pgfqpoint{0.492319in}{4.240106in}}{\pgfqpoint{0.502918in}{4.235716in}}{\pgfqpoint{0.513968in}{4.235716in}}%
\pgfpathlineto{\pgfqpoint{0.513968in}{4.235716in}}%
\pgfpathclose%
\pgfusepath{stroke}%
\end{pgfscope}%
\begin{pgfscope}%
\pgfpathrectangle{\pgfqpoint{0.494722in}{0.437222in}}{\pgfqpoint{6.275590in}{5.159444in}}%
\pgfusepath{clip}%
\pgfsetbuttcap%
\pgfsetroundjoin%
\pgfsetlinewidth{1.003750pt}%
\definecolor{currentstroke}{rgb}{0.827451,0.827451,0.827451}%
\pgfsetstrokecolor{currentstroke}%
\pgfsetstrokeopacity{0.800000}%
\pgfsetdash{}{0pt}%
\pgfpathmoveto{\pgfqpoint{4.362526in}{0.500037in}}%
\pgfpathcurveto{\pgfqpoint{4.373576in}{0.500037in}}{\pgfqpoint{4.384175in}{0.504427in}}{\pgfqpoint{4.391989in}{0.512241in}}%
\pgfpathcurveto{\pgfqpoint{4.399802in}{0.520054in}}{\pgfqpoint{4.404193in}{0.530653in}}{\pgfqpoint{4.404193in}{0.541703in}}%
\pgfpathcurveto{\pgfqpoint{4.404193in}{0.552753in}}{\pgfqpoint{4.399802in}{0.563352in}}{\pgfqpoint{4.391989in}{0.571166in}}%
\pgfpathcurveto{\pgfqpoint{4.384175in}{0.578980in}}{\pgfqpoint{4.373576in}{0.583370in}}{\pgfqpoint{4.362526in}{0.583370in}}%
\pgfpathcurveto{\pgfqpoint{4.351476in}{0.583370in}}{\pgfqpoint{4.340877in}{0.578980in}}{\pgfqpoint{4.333063in}{0.571166in}}%
\pgfpathcurveto{\pgfqpoint{4.325250in}{0.563352in}}{\pgfqpoint{4.320859in}{0.552753in}}{\pgfqpoint{4.320859in}{0.541703in}}%
\pgfpathcurveto{\pgfqpoint{4.320859in}{0.530653in}}{\pgfqpoint{4.325250in}{0.520054in}}{\pgfqpoint{4.333063in}{0.512241in}}%
\pgfpathcurveto{\pgfqpoint{4.340877in}{0.504427in}}{\pgfqpoint{4.351476in}{0.500037in}}{\pgfqpoint{4.362526in}{0.500037in}}%
\pgfpathlineto{\pgfqpoint{4.362526in}{0.500037in}}%
\pgfpathclose%
\pgfusepath{stroke}%
\end{pgfscope}%
\begin{pgfscope}%
\pgfpathrectangle{\pgfqpoint{0.494722in}{0.437222in}}{\pgfqpoint{6.275590in}{5.159444in}}%
\pgfusepath{clip}%
\pgfsetbuttcap%
\pgfsetroundjoin%
\pgfsetlinewidth{1.003750pt}%
\definecolor{currentstroke}{rgb}{0.827451,0.827451,0.827451}%
\pgfsetstrokecolor{currentstroke}%
\pgfsetstrokeopacity{0.800000}%
\pgfsetdash{}{0pt}%
\pgfpathmoveto{\pgfqpoint{1.852529in}{1.617217in}}%
\pgfpathcurveto{\pgfqpoint{1.863579in}{1.617217in}}{\pgfqpoint{1.874178in}{1.621608in}}{\pgfqpoint{1.881991in}{1.629421in}}%
\pgfpathcurveto{\pgfqpoint{1.889805in}{1.637235in}}{\pgfqpoint{1.894195in}{1.647834in}}{\pgfqpoint{1.894195in}{1.658884in}}%
\pgfpathcurveto{\pgfqpoint{1.894195in}{1.669934in}}{\pgfqpoint{1.889805in}{1.680533in}}{\pgfqpoint{1.881991in}{1.688347in}}%
\pgfpathcurveto{\pgfqpoint{1.874178in}{1.696160in}}{\pgfqpoint{1.863579in}{1.700551in}}{\pgfqpoint{1.852529in}{1.700551in}}%
\pgfpathcurveto{\pgfqpoint{1.841479in}{1.700551in}}{\pgfqpoint{1.830880in}{1.696160in}}{\pgfqpoint{1.823066in}{1.688347in}}%
\pgfpathcurveto{\pgfqpoint{1.815252in}{1.680533in}}{\pgfqpoint{1.810862in}{1.669934in}}{\pgfqpoint{1.810862in}{1.658884in}}%
\pgfpathcurveto{\pgfqpoint{1.810862in}{1.647834in}}{\pgfqpoint{1.815252in}{1.637235in}}{\pgfqpoint{1.823066in}{1.629421in}}%
\pgfpathcurveto{\pgfqpoint{1.830880in}{1.621608in}}{\pgfqpoint{1.841479in}{1.617217in}}{\pgfqpoint{1.852529in}{1.617217in}}%
\pgfpathlineto{\pgfqpoint{1.852529in}{1.617217in}}%
\pgfpathclose%
\pgfusepath{stroke}%
\end{pgfscope}%
\begin{pgfscope}%
\pgfpathrectangle{\pgfqpoint{0.494722in}{0.437222in}}{\pgfqpoint{6.275590in}{5.159444in}}%
\pgfusepath{clip}%
\pgfsetbuttcap%
\pgfsetroundjoin%
\pgfsetlinewidth{1.003750pt}%
\definecolor{currentstroke}{rgb}{0.827451,0.827451,0.827451}%
\pgfsetstrokecolor{currentstroke}%
\pgfsetstrokeopacity{0.800000}%
\pgfsetdash{}{0pt}%
\pgfpathmoveto{\pgfqpoint{0.536273in}{3.987890in}}%
\pgfpathcurveto{\pgfqpoint{0.547324in}{3.987890in}}{\pgfqpoint{0.557923in}{3.992280in}}{\pgfqpoint{0.565736in}{4.000093in}}%
\pgfpathcurveto{\pgfqpoint{0.573550in}{4.007907in}}{\pgfqpoint{0.577940in}{4.018506in}}{\pgfqpoint{0.577940in}{4.029556in}}%
\pgfpathcurveto{\pgfqpoint{0.577940in}{4.040606in}}{\pgfqpoint{0.573550in}{4.051205in}}{\pgfqpoint{0.565736in}{4.059019in}}%
\pgfpathcurveto{\pgfqpoint{0.557923in}{4.066833in}}{\pgfqpoint{0.547324in}{4.071223in}}{\pgfqpoint{0.536273in}{4.071223in}}%
\pgfpathcurveto{\pgfqpoint{0.525223in}{4.071223in}}{\pgfqpoint{0.514624in}{4.066833in}}{\pgfqpoint{0.506811in}{4.059019in}}%
\pgfpathcurveto{\pgfqpoint{0.498997in}{4.051205in}}{\pgfqpoint{0.494607in}{4.040606in}}{\pgfqpoint{0.494607in}{4.029556in}}%
\pgfpathcurveto{\pgfqpoint{0.494607in}{4.018506in}}{\pgfqpoint{0.498997in}{4.007907in}}{\pgfqpoint{0.506811in}{4.000093in}}%
\pgfpathcurveto{\pgfqpoint{0.514624in}{3.992280in}}{\pgfqpoint{0.525223in}{3.987890in}}{\pgfqpoint{0.536273in}{3.987890in}}%
\pgfpathlineto{\pgfqpoint{0.536273in}{3.987890in}}%
\pgfpathclose%
\pgfusepath{stroke}%
\end{pgfscope}%
\begin{pgfscope}%
\pgfpathrectangle{\pgfqpoint{0.494722in}{0.437222in}}{\pgfqpoint{6.275590in}{5.159444in}}%
\pgfusepath{clip}%
\pgfsetbuttcap%
\pgfsetroundjoin%
\pgfsetlinewidth{1.003750pt}%
\definecolor{currentstroke}{rgb}{0.827451,0.827451,0.827451}%
\pgfsetstrokecolor{currentstroke}%
\pgfsetstrokeopacity{0.800000}%
\pgfsetdash{}{0pt}%
\pgfpathmoveto{\pgfqpoint{0.841205in}{2.904303in}}%
\pgfpathcurveto{\pgfqpoint{0.852255in}{2.904303in}}{\pgfqpoint{0.862854in}{2.908693in}}{\pgfqpoint{0.870668in}{2.916506in}}%
\pgfpathcurveto{\pgfqpoint{0.878481in}{2.924320in}}{\pgfqpoint{0.882872in}{2.934919in}}{\pgfqpoint{0.882872in}{2.945969in}}%
\pgfpathcurveto{\pgfqpoint{0.882872in}{2.957019in}}{\pgfqpoint{0.878481in}{2.967618in}}{\pgfqpoint{0.870668in}{2.975432in}}%
\pgfpathcurveto{\pgfqpoint{0.862854in}{2.983246in}}{\pgfqpoint{0.852255in}{2.987636in}}{\pgfqpoint{0.841205in}{2.987636in}}%
\pgfpathcurveto{\pgfqpoint{0.830155in}{2.987636in}}{\pgfqpoint{0.819556in}{2.983246in}}{\pgfqpoint{0.811742in}{2.975432in}}%
\pgfpathcurveto{\pgfqpoint{0.803928in}{2.967618in}}{\pgfqpoint{0.799538in}{2.957019in}}{\pgfqpoint{0.799538in}{2.945969in}}%
\pgfpathcurveto{\pgfqpoint{0.799538in}{2.934919in}}{\pgfqpoint{0.803928in}{2.924320in}}{\pgfqpoint{0.811742in}{2.916506in}}%
\pgfpathcurveto{\pgfqpoint{0.819556in}{2.908693in}}{\pgfqpoint{0.830155in}{2.904303in}}{\pgfqpoint{0.841205in}{2.904303in}}%
\pgfpathlineto{\pgfqpoint{0.841205in}{2.904303in}}%
\pgfpathclose%
\pgfusepath{stroke}%
\end{pgfscope}%
\begin{pgfscope}%
\pgfpathrectangle{\pgfqpoint{0.494722in}{0.437222in}}{\pgfqpoint{6.275590in}{5.159444in}}%
\pgfusepath{clip}%
\pgfsetbuttcap%
\pgfsetroundjoin%
\pgfsetlinewidth{1.003750pt}%
\definecolor{currentstroke}{rgb}{0.827451,0.827451,0.827451}%
\pgfsetstrokecolor{currentstroke}%
\pgfsetstrokeopacity{0.800000}%
\pgfsetdash{}{0pt}%
\pgfpathmoveto{\pgfqpoint{1.926347in}{1.574489in}}%
\pgfpathcurveto{\pgfqpoint{1.937397in}{1.574489in}}{\pgfqpoint{1.947996in}{1.578879in}}{\pgfqpoint{1.955809in}{1.586693in}}%
\pgfpathcurveto{\pgfqpoint{1.963623in}{1.594506in}}{\pgfqpoint{1.968013in}{1.605105in}}{\pgfqpoint{1.968013in}{1.616155in}}%
\pgfpathcurveto{\pgfqpoint{1.968013in}{1.627206in}}{\pgfqpoint{1.963623in}{1.637805in}}{\pgfqpoint{1.955809in}{1.645618in}}%
\pgfpathcurveto{\pgfqpoint{1.947996in}{1.653432in}}{\pgfqpoint{1.937397in}{1.657822in}}{\pgfqpoint{1.926347in}{1.657822in}}%
\pgfpathcurveto{\pgfqpoint{1.915296in}{1.657822in}}{\pgfqpoint{1.904697in}{1.653432in}}{\pgfqpoint{1.896884in}{1.645618in}}%
\pgfpathcurveto{\pgfqpoint{1.889070in}{1.637805in}}{\pgfqpoint{1.884680in}{1.627206in}}{\pgfqpoint{1.884680in}{1.616155in}}%
\pgfpathcurveto{\pgfqpoint{1.884680in}{1.605105in}}{\pgfqpoint{1.889070in}{1.594506in}}{\pgfqpoint{1.896884in}{1.586693in}}%
\pgfpathcurveto{\pgfqpoint{1.904697in}{1.578879in}}{\pgfqpoint{1.915296in}{1.574489in}}{\pgfqpoint{1.926347in}{1.574489in}}%
\pgfpathlineto{\pgfqpoint{1.926347in}{1.574489in}}%
\pgfpathclose%
\pgfusepath{stroke}%
\end{pgfscope}%
\begin{pgfscope}%
\pgfpathrectangle{\pgfqpoint{0.494722in}{0.437222in}}{\pgfqpoint{6.275590in}{5.159444in}}%
\pgfusepath{clip}%
\pgfsetbuttcap%
\pgfsetroundjoin%
\pgfsetlinewidth{1.003750pt}%
\definecolor{currentstroke}{rgb}{0.827451,0.827451,0.827451}%
\pgfsetstrokecolor{currentstroke}%
\pgfsetstrokeopacity{0.800000}%
\pgfsetdash{}{0pt}%
\pgfpathmoveto{\pgfqpoint{3.295644in}{0.786846in}}%
\pgfpathcurveto{\pgfqpoint{3.306694in}{0.786846in}}{\pgfqpoint{3.317293in}{0.791236in}}{\pgfqpoint{3.325106in}{0.799050in}}%
\pgfpathcurveto{\pgfqpoint{3.332920in}{0.806864in}}{\pgfqpoint{3.337310in}{0.817463in}}{\pgfqpoint{3.337310in}{0.828513in}}%
\pgfpathcurveto{\pgfqpoint{3.337310in}{0.839563in}}{\pgfqpoint{3.332920in}{0.850162in}}{\pgfqpoint{3.325106in}{0.857976in}}%
\pgfpathcurveto{\pgfqpoint{3.317293in}{0.865789in}}{\pgfqpoint{3.306694in}{0.870179in}}{\pgfqpoint{3.295644in}{0.870179in}}%
\pgfpathcurveto{\pgfqpoint{3.284593in}{0.870179in}}{\pgfqpoint{3.273994in}{0.865789in}}{\pgfqpoint{3.266181in}{0.857976in}}%
\pgfpathcurveto{\pgfqpoint{3.258367in}{0.850162in}}{\pgfqpoint{3.253977in}{0.839563in}}{\pgfqpoint{3.253977in}{0.828513in}}%
\pgfpathcurveto{\pgfqpoint{3.253977in}{0.817463in}}{\pgfqpoint{3.258367in}{0.806864in}}{\pgfqpoint{3.266181in}{0.799050in}}%
\pgfpathcurveto{\pgfqpoint{3.273994in}{0.791236in}}{\pgfqpoint{3.284593in}{0.786846in}}{\pgfqpoint{3.295644in}{0.786846in}}%
\pgfpathlineto{\pgfqpoint{3.295644in}{0.786846in}}%
\pgfpathclose%
\pgfusepath{stroke}%
\end{pgfscope}%
\begin{pgfscope}%
\pgfpathrectangle{\pgfqpoint{0.494722in}{0.437222in}}{\pgfqpoint{6.275590in}{5.159444in}}%
\pgfusepath{clip}%
\pgfsetbuttcap%
\pgfsetroundjoin%
\pgfsetlinewidth{1.003750pt}%
\definecolor{currentstroke}{rgb}{0.827451,0.827451,0.827451}%
\pgfsetstrokecolor{currentstroke}%
\pgfsetstrokeopacity{0.800000}%
\pgfsetdash{}{0pt}%
\pgfpathmoveto{\pgfqpoint{3.123069in}{0.847718in}}%
\pgfpathcurveto{\pgfqpoint{3.134119in}{0.847718in}}{\pgfqpoint{3.144718in}{0.852109in}}{\pgfqpoint{3.152532in}{0.859922in}}%
\pgfpathcurveto{\pgfqpoint{3.160345in}{0.867736in}}{\pgfqpoint{3.164735in}{0.878335in}}{\pgfqpoint{3.164735in}{0.889385in}}%
\pgfpathcurveto{\pgfqpoint{3.164735in}{0.900435in}}{\pgfqpoint{3.160345in}{0.911034in}}{\pgfqpoint{3.152532in}{0.918848in}}%
\pgfpathcurveto{\pgfqpoint{3.144718in}{0.926661in}}{\pgfqpoint{3.134119in}{0.931052in}}{\pgfqpoint{3.123069in}{0.931052in}}%
\pgfpathcurveto{\pgfqpoint{3.112019in}{0.931052in}}{\pgfqpoint{3.101420in}{0.926661in}}{\pgfqpoint{3.093606in}{0.918848in}}%
\pgfpathcurveto{\pgfqpoint{3.085792in}{0.911034in}}{\pgfqpoint{3.081402in}{0.900435in}}{\pgfqpoint{3.081402in}{0.889385in}}%
\pgfpathcurveto{\pgfqpoint{3.081402in}{0.878335in}}{\pgfqpoint{3.085792in}{0.867736in}}{\pgfqpoint{3.093606in}{0.859922in}}%
\pgfpathcurveto{\pgfqpoint{3.101420in}{0.852109in}}{\pgfqpoint{3.112019in}{0.847718in}}{\pgfqpoint{3.123069in}{0.847718in}}%
\pgfpathlineto{\pgfqpoint{3.123069in}{0.847718in}}%
\pgfpathclose%
\pgfusepath{stroke}%
\end{pgfscope}%
\begin{pgfscope}%
\pgfpathrectangle{\pgfqpoint{0.494722in}{0.437222in}}{\pgfqpoint{6.275590in}{5.159444in}}%
\pgfusepath{clip}%
\pgfsetbuttcap%
\pgfsetroundjoin%
\pgfsetlinewidth{1.003750pt}%
\definecolor{currentstroke}{rgb}{0.827451,0.827451,0.827451}%
\pgfsetstrokecolor{currentstroke}%
\pgfsetstrokeopacity{0.800000}%
\pgfsetdash{}{0pt}%
\pgfpathmoveto{\pgfqpoint{3.222912in}{0.810111in}}%
\pgfpathcurveto{\pgfqpoint{3.233962in}{0.810111in}}{\pgfqpoint{3.244561in}{0.814501in}}{\pgfqpoint{3.252375in}{0.822314in}}%
\pgfpathcurveto{\pgfqpoint{3.260189in}{0.830128in}}{\pgfqpoint{3.264579in}{0.840727in}}{\pgfqpoint{3.264579in}{0.851777in}}%
\pgfpathcurveto{\pgfqpoint{3.264579in}{0.862827in}}{\pgfqpoint{3.260189in}{0.873426in}}{\pgfqpoint{3.252375in}{0.881240in}}%
\pgfpathcurveto{\pgfqpoint{3.244561in}{0.889054in}}{\pgfqpoint{3.233962in}{0.893444in}}{\pgfqpoint{3.222912in}{0.893444in}}%
\pgfpathcurveto{\pgfqpoint{3.211862in}{0.893444in}}{\pgfqpoint{3.201263in}{0.889054in}}{\pgfqpoint{3.193449in}{0.881240in}}%
\pgfpathcurveto{\pgfqpoint{3.185636in}{0.873426in}}{\pgfqpoint{3.181246in}{0.862827in}}{\pgfqpoint{3.181246in}{0.851777in}}%
\pgfpathcurveto{\pgfqpoint{3.181246in}{0.840727in}}{\pgfqpoint{3.185636in}{0.830128in}}{\pgfqpoint{3.193449in}{0.822314in}}%
\pgfpathcurveto{\pgfqpoint{3.201263in}{0.814501in}}{\pgfqpoint{3.211862in}{0.810111in}}{\pgfqpoint{3.222912in}{0.810111in}}%
\pgfpathlineto{\pgfqpoint{3.222912in}{0.810111in}}%
\pgfpathclose%
\pgfusepath{stroke}%
\end{pgfscope}%
\begin{pgfscope}%
\pgfpathrectangle{\pgfqpoint{0.494722in}{0.437222in}}{\pgfqpoint{6.275590in}{5.159444in}}%
\pgfusepath{clip}%
\pgfsetbuttcap%
\pgfsetroundjoin%
\pgfsetlinewidth{1.003750pt}%
\definecolor{currentstroke}{rgb}{0.827451,0.827451,0.827451}%
\pgfsetstrokecolor{currentstroke}%
\pgfsetstrokeopacity{0.800000}%
\pgfsetdash{}{0pt}%
\pgfpathmoveto{\pgfqpoint{0.944388in}{2.750541in}}%
\pgfpathcurveto{\pgfqpoint{0.955438in}{2.750541in}}{\pgfqpoint{0.966037in}{2.754931in}}{\pgfqpoint{0.973850in}{2.762745in}}%
\pgfpathcurveto{\pgfqpoint{0.981664in}{2.770558in}}{\pgfqpoint{0.986054in}{2.781157in}}{\pgfqpoint{0.986054in}{2.792208in}}%
\pgfpathcurveto{\pgfqpoint{0.986054in}{2.803258in}}{\pgfqpoint{0.981664in}{2.813857in}}{\pgfqpoint{0.973850in}{2.821670in}}%
\pgfpathcurveto{\pgfqpoint{0.966037in}{2.829484in}}{\pgfqpoint{0.955438in}{2.833874in}}{\pgfqpoint{0.944388in}{2.833874in}}%
\pgfpathcurveto{\pgfqpoint{0.933337in}{2.833874in}}{\pgfqpoint{0.922738in}{2.829484in}}{\pgfqpoint{0.914925in}{2.821670in}}%
\pgfpathcurveto{\pgfqpoint{0.907111in}{2.813857in}}{\pgfqpoint{0.902721in}{2.803258in}}{\pgfqpoint{0.902721in}{2.792208in}}%
\pgfpathcurveto{\pgfqpoint{0.902721in}{2.781157in}}{\pgfqpoint{0.907111in}{2.770558in}}{\pgfqpoint{0.914925in}{2.762745in}}%
\pgfpathcurveto{\pgfqpoint{0.922738in}{2.754931in}}{\pgfqpoint{0.933337in}{2.750541in}}{\pgfqpoint{0.944388in}{2.750541in}}%
\pgfpathlineto{\pgfqpoint{0.944388in}{2.750541in}}%
\pgfpathclose%
\pgfusepath{stroke}%
\end{pgfscope}%
\begin{pgfscope}%
\pgfpathrectangle{\pgfqpoint{0.494722in}{0.437222in}}{\pgfqpoint{6.275590in}{5.159444in}}%
\pgfusepath{clip}%
\pgfsetbuttcap%
\pgfsetroundjoin%
\pgfsetlinewidth{1.003750pt}%
\definecolor{currentstroke}{rgb}{0.827451,0.827451,0.827451}%
\pgfsetstrokecolor{currentstroke}%
\pgfsetstrokeopacity{0.800000}%
\pgfsetdash{}{0pt}%
\pgfpathmoveto{\pgfqpoint{1.055628in}{2.578187in}}%
\pgfpathcurveto{\pgfqpoint{1.066678in}{2.578187in}}{\pgfqpoint{1.077277in}{2.582578in}}{\pgfqpoint{1.085091in}{2.590391in}}%
\pgfpathcurveto{\pgfqpoint{1.092904in}{2.598205in}}{\pgfqpoint{1.097295in}{2.608804in}}{\pgfqpoint{1.097295in}{2.619854in}}%
\pgfpathcurveto{\pgfqpoint{1.097295in}{2.630904in}}{\pgfqpoint{1.092904in}{2.641503in}}{\pgfqpoint{1.085091in}{2.649317in}}%
\pgfpathcurveto{\pgfqpoint{1.077277in}{2.657130in}}{\pgfqpoint{1.066678in}{2.661521in}}{\pgfqpoint{1.055628in}{2.661521in}}%
\pgfpathcurveto{\pgfqpoint{1.044578in}{2.661521in}}{\pgfqpoint{1.033979in}{2.657130in}}{\pgfqpoint{1.026165in}{2.649317in}}%
\pgfpathcurveto{\pgfqpoint{1.018352in}{2.641503in}}{\pgfqpoint{1.013961in}{2.630904in}}{\pgfqpoint{1.013961in}{2.619854in}}%
\pgfpathcurveto{\pgfqpoint{1.013961in}{2.608804in}}{\pgfqpoint{1.018352in}{2.598205in}}{\pgfqpoint{1.026165in}{2.590391in}}%
\pgfpathcurveto{\pgfqpoint{1.033979in}{2.582578in}}{\pgfqpoint{1.044578in}{2.578187in}}{\pgfqpoint{1.055628in}{2.578187in}}%
\pgfpathlineto{\pgfqpoint{1.055628in}{2.578187in}}%
\pgfpathclose%
\pgfusepath{stroke}%
\end{pgfscope}%
\begin{pgfscope}%
\pgfpathrectangle{\pgfqpoint{0.494722in}{0.437222in}}{\pgfqpoint{6.275590in}{5.159444in}}%
\pgfusepath{clip}%
\pgfsetbuttcap%
\pgfsetroundjoin%
\pgfsetlinewidth{1.003750pt}%
\definecolor{currentstroke}{rgb}{0.827451,0.827451,0.827451}%
\pgfsetstrokecolor{currentstroke}%
\pgfsetstrokeopacity{0.800000}%
\pgfsetdash{}{0pt}%
\pgfpathmoveto{\pgfqpoint{0.805070in}{3.017769in}}%
\pgfpathcurveto{\pgfqpoint{0.816120in}{3.017769in}}{\pgfqpoint{0.826719in}{3.022159in}}{\pgfqpoint{0.834533in}{3.029973in}}%
\pgfpathcurveto{\pgfqpoint{0.842347in}{3.037786in}}{\pgfqpoint{0.846737in}{3.048385in}}{\pgfqpoint{0.846737in}{3.059436in}}%
\pgfpathcurveto{\pgfqpoint{0.846737in}{3.070486in}}{\pgfqpoint{0.842347in}{3.081085in}}{\pgfqpoint{0.834533in}{3.088898in}}%
\pgfpathcurveto{\pgfqpoint{0.826719in}{3.096712in}}{\pgfqpoint{0.816120in}{3.101102in}}{\pgfqpoint{0.805070in}{3.101102in}}%
\pgfpathcurveto{\pgfqpoint{0.794020in}{3.101102in}}{\pgfqpoint{0.783421in}{3.096712in}}{\pgfqpoint{0.775608in}{3.088898in}}%
\pgfpathcurveto{\pgfqpoint{0.767794in}{3.081085in}}{\pgfqpoint{0.763404in}{3.070486in}}{\pgfqpoint{0.763404in}{3.059436in}}%
\pgfpathcurveto{\pgfqpoint{0.763404in}{3.048385in}}{\pgfqpoint{0.767794in}{3.037786in}}{\pgfqpoint{0.775608in}{3.029973in}}%
\pgfpathcurveto{\pgfqpoint{0.783421in}{3.022159in}}{\pgfqpoint{0.794020in}{3.017769in}}{\pgfqpoint{0.805070in}{3.017769in}}%
\pgfpathlineto{\pgfqpoint{0.805070in}{3.017769in}}%
\pgfpathclose%
\pgfusepath{stroke}%
\end{pgfscope}%
\begin{pgfscope}%
\pgfpathrectangle{\pgfqpoint{0.494722in}{0.437222in}}{\pgfqpoint{6.275590in}{5.159444in}}%
\pgfusepath{clip}%
\pgfsetbuttcap%
\pgfsetroundjoin%
\pgfsetlinewidth{1.003750pt}%
\definecolor{currentstroke}{rgb}{0.827451,0.827451,0.827451}%
\pgfsetstrokecolor{currentstroke}%
\pgfsetstrokeopacity{0.800000}%
\pgfsetdash{}{0pt}%
\pgfpathmoveto{\pgfqpoint{0.498024in}{4.460098in}}%
\pgfpathcurveto{\pgfqpoint{0.509074in}{4.460098in}}{\pgfqpoint{0.519673in}{4.464488in}}{\pgfqpoint{0.527487in}{4.472302in}}%
\pgfpathcurveto{\pgfqpoint{0.535301in}{4.480116in}}{\pgfqpoint{0.539691in}{4.490715in}}{\pgfqpoint{0.539691in}{4.501765in}}%
\pgfpathcurveto{\pgfqpoint{0.539691in}{4.512815in}}{\pgfqpoint{0.535301in}{4.523414in}}{\pgfqpoint{0.527487in}{4.531228in}}%
\pgfpathcurveto{\pgfqpoint{0.519673in}{4.539041in}}{\pgfqpoint{0.509074in}{4.543432in}}{\pgfqpoint{0.498024in}{4.543432in}}%
\pgfpathcurveto{\pgfqpoint{0.486974in}{4.543432in}}{\pgfqpoint{0.476375in}{4.539041in}}{\pgfqpoint{0.468561in}{4.531228in}}%
\pgfpathcurveto{\pgfqpoint{0.460748in}{4.523414in}}{\pgfqpoint{0.456358in}{4.512815in}}{\pgfqpoint{0.456358in}{4.501765in}}%
\pgfpathcurveto{\pgfqpoint{0.456358in}{4.490715in}}{\pgfqpoint{0.460748in}{4.480116in}}{\pgfqpoint{0.468561in}{4.472302in}}%
\pgfpathcurveto{\pgfqpoint{0.476375in}{4.464488in}}{\pgfqpoint{0.486974in}{4.460098in}}{\pgfqpoint{0.498024in}{4.460098in}}%
\pgfpathlineto{\pgfqpoint{0.498024in}{4.460098in}}%
\pgfpathclose%
\pgfusepath{stroke}%
\end{pgfscope}%
\begin{pgfscope}%
\pgfpathrectangle{\pgfqpoint{0.494722in}{0.437222in}}{\pgfqpoint{6.275590in}{5.159444in}}%
\pgfusepath{clip}%
\pgfsetbuttcap%
\pgfsetroundjoin%
\pgfsetlinewidth{1.003750pt}%
\definecolor{currentstroke}{rgb}{0.827451,0.827451,0.827451}%
\pgfsetstrokecolor{currentstroke}%
\pgfsetstrokeopacity{0.800000}%
\pgfsetdash{}{0pt}%
\pgfpathmoveto{\pgfqpoint{0.970880in}{2.643850in}}%
\pgfpathcurveto{\pgfqpoint{0.981930in}{2.643850in}}{\pgfqpoint{0.992529in}{2.648240in}}{\pgfqpoint{1.000343in}{2.656053in}}%
\pgfpathcurveto{\pgfqpoint{1.008156in}{2.663867in}}{\pgfqpoint{1.012547in}{2.674466in}}{\pgfqpoint{1.012547in}{2.685516in}}%
\pgfpathcurveto{\pgfqpoint{1.012547in}{2.696566in}}{\pgfqpoint{1.008156in}{2.707165in}}{\pgfqpoint{1.000343in}{2.714979in}}%
\pgfpathcurveto{\pgfqpoint{0.992529in}{2.722793in}}{\pgfqpoint{0.981930in}{2.727183in}}{\pgfqpoint{0.970880in}{2.727183in}}%
\pgfpathcurveto{\pgfqpoint{0.959830in}{2.727183in}}{\pgfqpoint{0.949231in}{2.722793in}}{\pgfqpoint{0.941417in}{2.714979in}}%
\pgfpathcurveto{\pgfqpoint{0.933604in}{2.707165in}}{\pgfqpoint{0.929213in}{2.696566in}}{\pgfqpoint{0.929213in}{2.685516in}}%
\pgfpathcurveto{\pgfqpoint{0.929213in}{2.674466in}}{\pgfqpoint{0.933604in}{2.663867in}}{\pgfqpoint{0.941417in}{2.656053in}}%
\pgfpathcurveto{\pgfqpoint{0.949231in}{2.648240in}}{\pgfqpoint{0.959830in}{2.643850in}}{\pgfqpoint{0.970880in}{2.643850in}}%
\pgfpathlineto{\pgfqpoint{0.970880in}{2.643850in}}%
\pgfpathclose%
\pgfusepath{stroke}%
\end{pgfscope}%
\begin{pgfscope}%
\pgfpathrectangle{\pgfqpoint{0.494722in}{0.437222in}}{\pgfqpoint{6.275590in}{5.159444in}}%
\pgfusepath{clip}%
\pgfsetbuttcap%
\pgfsetroundjoin%
\pgfsetlinewidth{1.003750pt}%
\definecolor{currentstroke}{rgb}{0.827451,0.827451,0.827451}%
\pgfsetstrokecolor{currentstroke}%
\pgfsetstrokeopacity{0.800000}%
\pgfsetdash{}{0pt}%
\pgfpathmoveto{\pgfqpoint{4.586350in}{0.475898in}}%
\pgfpathcurveto{\pgfqpoint{4.597400in}{0.475898in}}{\pgfqpoint{4.607999in}{0.480288in}}{\pgfqpoint{4.615813in}{0.488101in}}%
\pgfpathcurveto{\pgfqpoint{4.623627in}{0.495915in}}{\pgfqpoint{4.628017in}{0.506514in}}{\pgfqpoint{4.628017in}{0.517564in}}%
\pgfpathcurveto{\pgfqpoint{4.628017in}{0.528614in}}{\pgfqpoint{4.623627in}{0.539213in}}{\pgfqpoint{4.615813in}{0.547027in}}%
\pgfpathcurveto{\pgfqpoint{4.607999in}{0.554841in}}{\pgfqpoint{4.597400in}{0.559231in}}{\pgfqpoint{4.586350in}{0.559231in}}%
\pgfpathcurveto{\pgfqpoint{4.575300in}{0.559231in}}{\pgfqpoint{4.564701in}{0.554841in}}{\pgfqpoint{4.556887in}{0.547027in}}%
\pgfpathcurveto{\pgfqpoint{4.549074in}{0.539213in}}{\pgfqpoint{4.544683in}{0.528614in}}{\pgfqpoint{4.544683in}{0.517564in}}%
\pgfpathcurveto{\pgfqpoint{4.544683in}{0.506514in}}{\pgfqpoint{4.549074in}{0.495915in}}{\pgfqpoint{4.556887in}{0.488101in}}%
\pgfpathcurveto{\pgfqpoint{4.564701in}{0.480288in}}{\pgfqpoint{4.575300in}{0.475898in}}{\pgfqpoint{4.586350in}{0.475898in}}%
\pgfpathlineto{\pgfqpoint{4.586350in}{0.475898in}}%
\pgfpathclose%
\pgfusepath{stroke}%
\end{pgfscope}%
\begin{pgfscope}%
\pgfpathrectangle{\pgfqpoint{0.494722in}{0.437222in}}{\pgfqpoint{6.275590in}{5.159444in}}%
\pgfusepath{clip}%
\pgfsetbuttcap%
\pgfsetroundjoin%
\pgfsetlinewidth{1.003750pt}%
\definecolor{currentstroke}{rgb}{0.827451,0.827451,0.827451}%
\pgfsetstrokecolor{currentstroke}%
\pgfsetstrokeopacity{0.800000}%
\pgfsetdash{}{0pt}%
\pgfpathmoveto{\pgfqpoint{4.176161in}{0.527713in}}%
\pgfpathcurveto{\pgfqpoint{4.187212in}{0.527713in}}{\pgfqpoint{4.197811in}{0.532104in}}{\pgfqpoint{4.205624in}{0.539917in}}%
\pgfpathcurveto{\pgfqpoint{4.213438in}{0.547731in}}{\pgfqpoint{4.217828in}{0.558330in}}{\pgfqpoint{4.217828in}{0.569380in}}%
\pgfpathcurveto{\pgfqpoint{4.217828in}{0.580430in}}{\pgfqpoint{4.213438in}{0.591029in}}{\pgfqpoint{4.205624in}{0.598843in}}%
\pgfpathcurveto{\pgfqpoint{4.197811in}{0.606656in}}{\pgfqpoint{4.187212in}{0.611047in}}{\pgfqpoint{4.176161in}{0.611047in}}%
\pgfpathcurveto{\pgfqpoint{4.165111in}{0.611047in}}{\pgfqpoint{4.154512in}{0.606656in}}{\pgfqpoint{4.146699in}{0.598843in}}%
\pgfpathcurveto{\pgfqpoint{4.138885in}{0.591029in}}{\pgfqpoint{4.134495in}{0.580430in}}{\pgfqpoint{4.134495in}{0.569380in}}%
\pgfpathcurveto{\pgfqpoint{4.134495in}{0.558330in}}{\pgfqpoint{4.138885in}{0.547731in}}{\pgfqpoint{4.146699in}{0.539917in}}%
\pgfpathcurveto{\pgfqpoint{4.154512in}{0.532104in}}{\pgfqpoint{4.165111in}{0.527713in}}{\pgfqpoint{4.176161in}{0.527713in}}%
\pgfpathlineto{\pgfqpoint{4.176161in}{0.527713in}}%
\pgfpathclose%
\pgfusepath{stroke}%
\end{pgfscope}%
\begin{pgfscope}%
\pgfpathrectangle{\pgfqpoint{0.494722in}{0.437222in}}{\pgfqpoint{6.275590in}{5.159444in}}%
\pgfusepath{clip}%
\pgfsetbuttcap%
\pgfsetroundjoin%
\pgfsetlinewidth{1.003750pt}%
\definecolor{currentstroke}{rgb}{0.827451,0.827451,0.827451}%
\pgfsetstrokecolor{currentstroke}%
\pgfsetstrokeopacity{0.800000}%
\pgfsetdash{}{0pt}%
\pgfpathmoveto{\pgfqpoint{1.671516in}{1.760843in}}%
\pgfpathcurveto{\pgfqpoint{1.682566in}{1.760843in}}{\pgfqpoint{1.693165in}{1.765233in}}{\pgfqpoint{1.700979in}{1.773046in}}%
\pgfpathcurveto{\pgfqpoint{1.708792in}{1.780860in}}{\pgfqpoint{1.713183in}{1.791459in}}{\pgfqpoint{1.713183in}{1.802509in}}%
\pgfpathcurveto{\pgfqpoint{1.713183in}{1.813559in}}{\pgfqpoint{1.708792in}{1.824158in}}{\pgfqpoint{1.700979in}{1.831972in}}%
\pgfpathcurveto{\pgfqpoint{1.693165in}{1.839786in}}{\pgfqpoint{1.682566in}{1.844176in}}{\pgfqpoint{1.671516in}{1.844176in}}%
\pgfpathcurveto{\pgfqpoint{1.660466in}{1.844176in}}{\pgfqpoint{1.649867in}{1.839786in}}{\pgfqpoint{1.642053in}{1.831972in}}%
\pgfpathcurveto{\pgfqpoint{1.634240in}{1.824158in}}{\pgfqpoint{1.629849in}{1.813559in}}{\pgfqpoint{1.629849in}{1.802509in}}%
\pgfpathcurveto{\pgfqpoint{1.629849in}{1.791459in}}{\pgfqpoint{1.634240in}{1.780860in}}{\pgfqpoint{1.642053in}{1.773046in}}%
\pgfpathcurveto{\pgfqpoint{1.649867in}{1.765233in}}{\pgfqpoint{1.660466in}{1.760843in}}{\pgfqpoint{1.671516in}{1.760843in}}%
\pgfpathlineto{\pgfqpoint{1.671516in}{1.760843in}}%
\pgfpathclose%
\pgfusepath{stroke}%
\end{pgfscope}%
\begin{pgfscope}%
\pgfpathrectangle{\pgfqpoint{0.494722in}{0.437222in}}{\pgfqpoint{6.275590in}{5.159444in}}%
\pgfusepath{clip}%
\pgfsetbuttcap%
\pgfsetroundjoin%
\pgfsetlinewidth{1.003750pt}%
\definecolor{currentstroke}{rgb}{0.827451,0.827451,0.827451}%
\pgfsetstrokecolor{currentstroke}%
\pgfsetstrokeopacity{0.800000}%
\pgfsetdash{}{0pt}%
\pgfpathmoveto{\pgfqpoint{3.901710in}{0.604104in}}%
\pgfpathcurveto{\pgfqpoint{3.912761in}{0.604104in}}{\pgfqpoint{3.923360in}{0.608495in}}{\pgfqpoint{3.931173in}{0.616308in}}%
\pgfpathcurveto{\pgfqpoint{3.938987in}{0.624122in}}{\pgfqpoint{3.943377in}{0.634721in}}{\pgfqpoint{3.943377in}{0.645771in}}%
\pgfpathcurveto{\pgfqpoint{3.943377in}{0.656821in}}{\pgfqpoint{3.938987in}{0.667420in}}{\pgfqpoint{3.931173in}{0.675234in}}%
\pgfpathcurveto{\pgfqpoint{3.923360in}{0.683047in}}{\pgfqpoint{3.912761in}{0.687438in}}{\pgfqpoint{3.901710in}{0.687438in}}%
\pgfpathcurveto{\pgfqpoint{3.890660in}{0.687438in}}{\pgfqpoint{3.880061in}{0.683047in}}{\pgfqpoint{3.872248in}{0.675234in}}%
\pgfpathcurveto{\pgfqpoint{3.864434in}{0.667420in}}{\pgfqpoint{3.860044in}{0.656821in}}{\pgfqpoint{3.860044in}{0.645771in}}%
\pgfpathcurveto{\pgfqpoint{3.860044in}{0.634721in}}{\pgfqpoint{3.864434in}{0.624122in}}{\pgfqpoint{3.872248in}{0.616308in}}%
\pgfpathcurveto{\pgfqpoint{3.880061in}{0.608495in}}{\pgfqpoint{3.890660in}{0.604104in}}{\pgfqpoint{3.901710in}{0.604104in}}%
\pgfpathlineto{\pgfqpoint{3.901710in}{0.604104in}}%
\pgfpathclose%
\pgfusepath{stroke}%
\end{pgfscope}%
\begin{pgfscope}%
\pgfpathrectangle{\pgfqpoint{0.494722in}{0.437222in}}{\pgfqpoint{6.275590in}{5.159444in}}%
\pgfusepath{clip}%
\pgfsetbuttcap%
\pgfsetroundjoin%
\pgfsetlinewidth{1.003750pt}%
\definecolor{currentstroke}{rgb}{0.827451,0.827451,0.827451}%
\pgfsetstrokecolor{currentstroke}%
\pgfsetstrokeopacity{0.800000}%
\pgfsetdash{}{0pt}%
\pgfpathmoveto{\pgfqpoint{3.739550in}{0.626568in}}%
\pgfpathcurveto{\pgfqpoint{3.750600in}{0.626568in}}{\pgfqpoint{3.761199in}{0.630958in}}{\pgfqpoint{3.769013in}{0.638771in}}%
\pgfpathcurveto{\pgfqpoint{3.776827in}{0.646585in}}{\pgfqpoint{3.781217in}{0.657184in}}{\pgfqpoint{3.781217in}{0.668234in}}%
\pgfpathcurveto{\pgfqpoint{3.781217in}{0.679284in}}{\pgfqpoint{3.776827in}{0.689883in}}{\pgfqpoint{3.769013in}{0.697697in}}%
\pgfpathcurveto{\pgfqpoint{3.761199in}{0.705511in}}{\pgfqpoint{3.750600in}{0.709901in}}{\pgfqpoint{3.739550in}{0.709901in}}%
\pgfpathcurveto{\pgfqpoint{3.728500in}{0.709901in}}{\pgfqpoint{3.717901in}{0.705511in}}{\pgfqpoint{3.710087in}{0.697697in}}%
\pgfpathcurveto{\pgfqpoint{3.702274in}{0.689883in}}{\pgfqpoint{3.697884in}{0.679284in}}{\pgfqpoint{3.697884in}{0.668234in}}%
\pgfpathcurveto{\pgfqpoint{3.697884in}{0.657184in}}{\pgfqpoint{3.702274in}{0.646585in}}{\pgfqpoint{3.710087in}{0.638771in}}%
\pgfpathcurveto{\pgfqpoint{3.717901in}{0.630958in}}{\pgfqpoint{3.728500in}{0.626568in}}{\pgfqpoint{3.739550in}{0.626568in}}%
\pgfpathlineto{\pgfqpoint{3.739550in}{0.626568in}}%
\pgfpathclose%
\pgfusepath{stroke}%
\end{pgfscope}%
\begin{pgfscope}%
\pgfpathrectangle{\pgfqpoint{0.494722in}{0.437222in}}{\pgfqpoint{6.275590in}{5.159444in}}%
\pgfusepath{clip}%
\pgfsetbuttcap%
\pgfsetroundjoin%
\pgfsetlinewidth{1.003750pt}%
\definecolor{currentstroke}{rgb}{0.827451,0.827451,0.827451}%
\pgfsetstrokecolor{currentstroke}%
\pgfsetstrokeopacity{0.800000}%
\pgfsetdash{}{0pt}%
\pgfpathmoveto{\pgfqpoint{5.258388in}{0.413707in}}%
\pgfpathcurveto{\pgfqpoint{5.269438in}{0.413707in}}{\pgfqpoint{5.280037in}{0.418098in}}{\pgfqpoint{5.287850in}{0.425911in}}%
\pgfpathcurveto{\pgfqpoint{5.295664in}{0.433725in}}{\pgfqpoint{5.300054in}{0.444324in}}{\pgfqpoint{5.300054in}{0.455374in}}%
\pgfpathcurveto{\pgfqpoint{5.300054in}{0.466424in}}{\pgfqpoint{5.295664in}{0.477023in}}{\pgfqpoint{5.287850in}{0.484837in}}%
\pgfpathcurveto{\pgfqpoint{5.280037in}{0.492650in}}{\pgfqpoint{5.269438in}{0.497041in}}{\pgfqpoint{5.258388in}{0.497041in}}%
\pgfpathcurveto{\pgfqpoint{5.247337in}{0.497041in}}{\pgfqpoint{5.236738in}{0.492650in}}{\pgfqpoint{5.228925in}{0.484837in}}%
\pgfpathcurveto{\pgfqpoint{5.221111in}{0.477023in}}{\pgfqpoint{5.216721in}{0.466424in}}{\pgfqpoint{5.216721in}{0.455374in}}%
\pgfpathcurveto{\pgfqpoint{5.216721in}{0.444324in}}{\pgfqpoint{5.221111in}{0.433725in}}{\pgfqpoint{5.228925in}{0.425911in}}%
\pgfpathcurveto{\pgfqpoint{5.236738in}{0.418098in}}{\pgfqpoint{5.247337in}{0.413707in}}{\pgfqpoint{5.258388in}{0.413707in}}%
\pgfusepath{stroke}%
\end{pgfscope}%
\begin{pgfscope}%
\pgfpathrectangle{\pgfqpoint{0.494722in}{0.437222in}}{\pgfqpoint{6.275590in}{5.159444in}}%
\pgfusepath{clip}%
\pgfsetbuttcap%
\pgfsetroundjoin%
\pgfsetlinewidth{1.003750pt}%
\definecolor{currentstroke}{rgb}{0.827451,0.827451,0.827451}%
\pgfsetstrokecolor{currentstroke}%
\pgfsetstrokeopacity{0.800000}%
\pgfsetdash{}{0pt}%
\pgfpathmoveto{\pgfqpoint{0.625304in}{3.523383in}}%
\pgfpathcurveto{\pgfqpoint{0.636354in}{3.523383in}}{\pgfqpoint{0.646953in}{3.527774in}}{\pgfqpoint{0.654767in}{3.535587in}}%
\pgfpathcurveto{\pgfqpoint{0.662581in}{3.543401in}}{\pgfqpoint{0.666971in}{3.554000in}}{\pgfqpoint{0.666971in}{3.565050in}}%
\pgfpathcurveto{\pgfqpoint{0.666971in}{3.576100in}}{\pgfqpoint{0.662581in}{3.586699in}}{\pgfqpoint{0.654767in}{3.594513in}}%
\pgfpathcurveto{\pgfqpoint{0.646953in}{3.602326in}}{\pgfqpoint{0.636354in}{3.606717in}}{\pgfqpoint{0.625304in}{3.606717in}}%
\pgfpathcurveto{\pgfqpoint{0.614254in}{3.606717in}}{\pgfqpoint{0.603655in}{3.602326in}}{\pgfqpoint{0.595841in}{3.594513in}}%
\pgfpathcurveto{\pgfqpoint{0.588028in}{3.586699in}}{\pgfqpoint{0.583638in}{3.576100in}}{\pgfqpoint{0.583638in}{3.565050in}}%
\pgfpathcurveto{\pgfqpoint{0.583638in}{3.554000in}}{\pgfqpoint{0.588028in}{3.543401in}}{\pgfqpoint{0.595841in}{3.535587in}}%
\pgfpathcurveto{\pgfqpoint{0.603655in}{3.527774in}}{\pgfqpoint{0.614254in}{3.523383in}}{\pgfqpoint{0.625304in}{3.523383in}}%
\pgfpathlineto{\pgfqpoint{0.625304in}{3.523383in}}%
\pgfpathclose%
\pgfusepath{stroke}%
\end{pgfscope}%
\begin{pgfscope}%
\pgfpathrectangle{\pgfqpoint{0.494722in}{0.437222in}}{\pgfqpoint{6.275590in}{5.159444in}}%
\pgfusepath{clip}%
\pgfsetbuttcap%
\pgfsetroundjoin%
\pgfsetlinewidth{1.003750pt}%
\definecolor{currentstroke}{rgb}{0.827451,0.827451,0.827451}%
\pgfsetstrokecolor{currentstroke}%
\pgfsetstrokeopacity{0.800000}%
\pgfsetdash{}{0pt}%
\pgfpathmoveto{\pgfqpoint{0.584355in}{3.716646in}}%
\pgfpathcurveto{\pgfqpoint{0.595406in}{3.716646in}}{\pgfqpoint{0.606005in}{3.721036in}}{\pgfqpoint{0.613818in}{3.728850in}}%
\pgfpathcurveto{\pgfqpoint{0.621632in}{3.736663in}}{\pgfqpoint{0.626022in}{3.747262in}}{\pgfqpoint{0.626022in}{3.758313in}}%
\pgfpathcurveto{\pgfqpoint{0.626022in}{3.769363in}}{\pgfqpoint{0.621632in}{3.779962in}}{\pgfqpoint{0.613818in}{3.787775in}}%
\pgfpathcurveto{\pgfqpoint{0.606005in}{3.795589in}}{\pgfqpoint{0.595406in}{3.799979in}}{\pgfqpoint{0.584355in}{3.799979in}}%
\pgfpathcurveto{\pgfqpoint{0.573305in}{3.799979in}}{\pgfqpoint{0.562706in}{3.795589in}}{\pgfqpoint{0.554893in}{3.787775in}}%
\pgfpathcurveto{\pgfqpoint{0.547079in}{3.779962in}}{\pgfqpoint{0.542689in}{3.769363in}}{\pgfqpoint{0.542689in}{3.758313in}}%
\pgfpathcurveto{\pgfqpoint{0.542689in}{3.747262in}}{\pgfqpoint{0.547079in}{3.736663in}}{\pgfqpoint{0.554893in}{3.728850in}}%
\pgfpathcurveto{\pgfqpoint{0.562706in}{3.721036in}}{\pgfqpoint{0.573305in}{3.716646in}}{\pgfqpoint{0.584355in}{3.716646in}}%
\pgfpathlineto{\pgfqpoint{0.584355in}{3.716646in}}%
\pgfpathclose%
\pgfusepath{stroke}%
\end{pgfscope}%
\begin{pgfscope}%
\pgfpathrectangle{\pgfqpoint{0.494722in}{0.437222in}}{\pgfqpoint{6.275590in}{5.159444in}}%
\pgfusepath{clip}%
\pgfsetbuttcap%
\pgfsetroundjoin%
\pgfsetlinewidth{1.003750pt}%
\definecolor{currentstroke}{rgb}{0.827451,0.827451,0.827451}%
\pgfsetstrokecolor{currentstroke}%
\pgfsetstrokeopacity{0.800000}%
\pgfsetdash{}{0pt}%
\pgfpathmoveto{\pgfqpoint{0.608606in}{3.646796in}}%
\pgfpathcurveto{\pgfqpoint{0.619657in}{3.646796in}}{\pgfqpoint{0.630256in}{3.651186in}}{\pgfqpoint{0.638069in}{3.659000in}}%
\pgfpathcurveto{\pgfqpoint{0.645883in}{3.666813in}}{\pgfqpoint{0.650273in}{3.677412in}}{\pgfqpoint{0.650273in}{3.688462in}}%
\pgfpathcurveto{\pgfqpoint{0.650273in}{3.699512in}}{\pgfqpoint{0.645883in}{3.710112in}}{\pgfqpoint{0.638069in}{3.717925in}}%
\pgfpathcurveto{\pgfqpoint{0.630256in}{3.725739in}}{\pgfqpoint{0.619657in}{3.730129in}}{\pgfqpoint{0.608606in}{3.730129in}}%
\pgfpathcurveto{\pgfqpoint{0.597556in}{3.730129in}}{\pgfqpoint{0.586957in}{3.725739in}}{\pgfqpoint{0.579144in}{3.717925in}}%
\pgfpathcurveto{\pgfqpoint{0.571330in}{3.710112in}}{\pgfqpoint{0.566940in}{3.699512in}}{\pgfqpoint{0.566940in}{3.688462in}}%
\pgfpathcurveto{\pgfqpoint{0.566940in}{3.677412in}}{\pgfqpoint{0.571330in}{3.666813in}}{\pgfqpoint{0.579144in}{3.659000in}}%
\pgfpathcurveto{\pgfqpoint{0.586957in}{3.651186in}}{\pgfqpoint{0.597556in}{3.646796in}}{\pgfqpoint{0.608606in}{3.646796in}}%
\pgfpathlineto{\pgfqpoint{0.608606in}{3.646796in}}%
\pgfpathclose%
\pgfusepath{stroke}%
\end{pgfscope}%
\begin{pgfscope}%
\pgfpathrectangle{\pgfqpoint{0.494722in}{0.437222in}}{\pgfqpoint{6.275590in}{5.159444in}}%
\pgfusepath{clip}%
\pgfsetbuttcap%
\pgfsetroundjoin%
\pgfsetlinewidth{1.003750pt}%
\definecolor{currentstroke}{rgb}{0.827451,0.827451,0.827451}%
\pgfsetstrokecolor{currentstroke}%
\pgfsetstrokeopacity{0.800000}%
\pgfsetdash{}{0pt}%
\pgfpathmoveto{\pgfqpoint{3.942377in}{0.586956in}}%
\pgfpathcurveto{\pgfqpoint{3.953427in}{0.586956in}}{\pgfqpoint{3.964026in}{0.591347in}}{\pgfqpoint{3.971840in}{0.599160in}}%
\pgfpathcurveto{\pgfqpoint{3.979653in}{0.606974in}}{\pgfqpoint{3.984044in}{0.617573in}}{\pgfqpoint{3.984044in}{0.628623in}}%
\pgfpathcurveto{\pgfqpoint{3.984044in}{0.639673in}}{\pgfqpoint{3.979653in}{0.650272in}}{\pgfqpoint{3.971840in}{0.658086in}}%
\pgfpathcurveto{\pgfqpoint{3.964026in}{0.665900in}}{\pgfqpoint{3.953427in}{0.670290in}}{\pgfqpoint{3.942377in}{0.670290in}}%
\pgfpathcurveto{\pgfqpoint{3.931327in}{0.670290in}}{\pgfqpoint{3.920728in}{0.665900in}}{\pgfqpoint{3.912914in}{0.658086in}}%
\pgfpathcurveto{\pgfqpoint{3.905101in}{0.650272in}}{\pgfqpoint{3.900710in}{0.639673in}}{\pgfqpoint{3.900710in}{0.628623in}}%
\pgfpathcurveto{\pgfqpoint{3.900710in}{0.617573in}}{\pgfqpoint{3.905101in}{0.606974in}}{\pgfqpoint{3.912914in}{0.599160in}}%
\pgfpathcurveto{\pgfqpoint{3.920728in}{0.591347in}}{\pgfqpoint{3.931327in}{0.586956in}}{\pgfqpoint{3.942377in}{0.586956in}}%
\pgfpathlineto{\pgfqpoint{3.942377in}{0.586956in}}%
\pgfpathclose%
\pgfusepath{stroke}%
\end{pgfscope}%
\begin{pgfscope}%
\pgfpathrectangle{\pgfqpoint{0.494722in}{0.437222in}}{\pgfqpoint{6.275590in}{5.159444in}}%
\pgfusepath{clip}%
\pgfsetbuttcap%
\pgfsetroundjoin%
\pgfsetlinewidth{1.003750pt}%
\definecolor{currentstroke}{rgb}{0.827451,0.827451,0.827451}%
\pgfsetstrokecolor{currentstroke}%
\pgfsetstrokeopacity{0.800000}%
\pgfsetdash{}{0pt}%
\pgfpathmoveto{\pgfqpoint{3.460812in}{0.723768in}}%
\pgfpathcurveto{\pgfqpoint{3.471862in}{0.723768in}}{\pgfqpoint{3.482461in}{0.728158in}}{\pgfqpoint{3.490274in}{0.735972in}}%
\pgfpathcurveto{\pgfqpoint{3.498088in}{0.743785in}}{\pgfqpoint{3.502478in}{0.754384in}}{\pgfqpoint{3.502478in}{0.765434in}}%
\pgfpathcurveto{\pgfqpoint{3.502478in}{0.776484in}}{\pgfqpoint{3.498088in}{0.787083in}}{\pgfqpoint{3.490274in}{0.794897in}}%
\pgfpathcurveto{\pgfqpoint{3.482461in}{0.802711in}}{\pgfqpoint{3.471862in}{0.807101in}}{\pgfqpoint{3.460812in}{0.807101in}}%
\pgfpathcurveto{\pgfqpoint{3.449762in}{0.807101in}}{\pgfqpoint{3.439162in}{0.802711in}}{\pgfqpoint{3.431349in}{0.794897in}}%
\pgfpathcurveto{\pgfqpoint{3.423535in}{0.787083in}}{\pgfqpoint{3.419145in}{0.776484in}}{\pgfqpoint{3.419145in}{0.765434in}}%
\pgfpathcurveto{\pgfqpoint{3.419145in}{0.754384in}}{\pgfqpoint{3.423535in}{0.743785in}}{\pgfqpoint{3.431349in}{0.735972in}}%
\pgfpathcurveto{\pgfqpoint{3.439162in}{0.728158in}}{\pgfqpoint{3.449762in}{0.723768in}}{\pgfqpoint{3.460812in}{0.723768in}}%
\pgfpathlineto{\pgfqpoint{3.460812in}{0.723768in}}%
\pgfpathclose%
\pgfusepath{stroke}%
\end{pgfscope}%
\begin{pgfscope}%
\pgfpathrectangle{\pgfqpoint{0.494722in}{0.437222in}}{\pgfqpoint{6.275590in}{5.159444in}}%
\pgfusepath{clip}%
\pgfsetbuttcap%
\pgfsetroundjoin%
\pgfsetlinewidth{1.003750pt}%
\definecolor{currentstroke}{rgb}{0.827451,0.827451,0.827451}%
\pgfsetstrokecolor{currentstroke}%
\pgfsetstrokeopacity{0.800000}%
\pgfsetdash{}{0pt}%
\pgfpathmoveto{\pgfqpoint{1.133504in}{2.415631in}}%
\pgfpathcurveto{\pgfqpoint{1.144554in}{2.415631in}}{\pgfqpoint{1.155153in}{2.420021in}}{\pgfqpoint{1.162967in}{2.427835in}}%
\pgfpathcurveto{\pgfqpoint{1.170780in}{2.435648in}}{\pgfqpoint{1.175170in}{2.446247in}}{\pgfqpoint{1.175170in}{2.457297in}}%
\pgfpathcurveto{\pgfqpoint{1.175170in}{2.468347in}}{\pgfqpoint{1.170780in}{2.478946in}}{\pgfqpoint{1.162967in}{2.486760in}}%
\pgfpathcurveto{\pgfqpoint{1.155153in}{2.494574in}}{\pgfqpoint{1.144554in}{2.498964in}}{\pgfqpoint{1.133504in}{2.498964in}}%
\pgfpathcurveto{\pgfqpoint{1.122454in}{2.498964in}}{\pgfqpoint{1.111855in}{2.494574in}}{\pgfqpoint{1.104041in}{2.486760in}}%
\pgfpathcurveto{\pgfqpoint{1.096227in}{2.478946in}}{\pgfqpoint{1.091837in}{2.468347in}}{\pgfqpoint{1.091837in}{2.457297in}}%
\pgfpathcurveto{\pgfqpoint{1.091837in}{2.446247in}}{\pgfqpoint{1.096227in}{2.435648in}}{\pgfqpoint{1.104041in}{2.427835in}}%
\pgfpathcurveto{\pgfqpoint{1.111855in}{2.420021in}}{\pgfqpoint{1.122454in}{2.415631in}}{\pgfqpoint{1.133504in}{2.415631in}}%
\pgfpathlineto{\pgfqpoint{1.133504in}{2.415631in}}%
\pgfpathclose%
\pgfusepath{stroke}%
\end{pgfscope}%
\begin{pgfscope}%
\pgfpathrectangle{\pgfqpoint{0.494722in}{0.437222in}}{\pgfqpoint{6.275590in}{5.159444in}}%
\pgfusepath{clip}%
\pgfsetbuttcap%
\pgfsetroundjoin%
\pgfsetlinewidth{1.003750pt}%
\definecolor{currentstroke}{rgb}{0.827451,0.827451,0.827451}%
\pgfsetstrokecolor{currentstroke}%
\pgfsetstrokeopacity{0.800000}%
\pgfsetdash{}{0pt}%
\pgfpathmoveto{\pgfqpoint{5.560565in}{0.403701in}}%
\pgfpathcurveto{\pgfqpoint{5.571615in}{0.403701in}}{\pgfqpoint{5.582214in}{0.408092in}}{\pgfqpoint{5.590028in}{0.415905in}}%
\pgfpathcurveto{\pgfqpoint{5.597842in}{0.423719in}}{\pgfqpoint{5.602232in}{0.434318in}}{\pgfqpoint{5.602232in}{0.445368in}}%
\pgfpathcurveto{\pgfqpoint{5.602232in}{0.456418in}}{\pgfqpoint{5.597842in}{0.467017in}}{\pgfqpoint{5.590028in}{0.474831in}}%
\pgfpathcurveto{\pgfqpoint{5.582214in}{0.482644in}}{\pgfqpoint{5.571615in}{0.487035in}}{\pgfqpoint{5.560565in}{0.487035in}}%
\pgfpathcurveto{\pgfqpoint{5.549515in}{0.487035in}}{\pgfqpoint{5.538916in}{0.482644in}}{\pgfqpoint{5.531102in}{0.474831in}}%
\pgfpathcurveto{\pgfqpoint{5.523289in}{0.467017in}}{\pgfqpoint{5.518898in}{0.456418in}}{\pgfqpoint{5.518898in}{0.445368in}}%
\pgfpathcurveto{\pgfqpoint{5.518898in}{0.434318in}}{\pgfqpoint{5.523289in}{0.423719in}}{\pgfqpoint{5.531102in}{0.415905in}}%
\pgfpathcurveto{\pgfqpoint{5.538916in}{0.408092in}}{\pgfqpoint{5.549515in}{0.403701in}}{\pgfqpoint{5.560565in}{0.403701in}}%
\pgfusepath{stroke}%
\end{pgfscope}%
\begin{pgfscope}%
\pgfpathrectangle{\pgfqpoint{0.494722in}{0.437222in}}{\pgfqpoint{6.275590in}{5.159444in}}%
\pgfusepath{clip}%
\pgfsetbuttcap%
\pgfsetroundjoin%
\pgfsetlinewidth{1.003750pt}%
\definecolor{currentstroke}{rgb}{0.827451,0.827451,0.827451}%
\pgfsetstrokecolor{currentstroke}%
\pgfsetstrokeopacity{0.800000}%
\pgfsetdash{}{0pt}%
\pgfpathmoveto{\pgfqpoint{4.706867in}{0.451120in}}%
\pgfpathcurveto{\pgfqpoint{4.717917in}{0.451120in}}{\pgfqpoint{4.728516in}{0.455510in}}{\pgfqpoint{4.736329in}{0.463324in}}%
\pgfpathcurveto{\pgfqpoint{4.744143in}{0.471138in}}{\pgfqpoint{4.748533in}{0.481737in}}{\pgfqpoint{4.748533in}{0.492787in}}%
\pgfpathcurveto{\pgfqpoint{4.748533in}{0.503837in}}{\pgfqpoint{4.744143in}{0.514436in}}{\pgfqpoint{4.736329in}{0.522250in}}%
\pgfpathcurveto{\pgfqpoint{4.728516in}{0.530063in}}{\pgfqpoint{4.717917in}{0.534453in}}{\pgfqpoint{4.706867in}{0.534453in}}%
\pgfpathcurveto{\pgfqpoint{4.695817in}{0.534453in}}{\pgfqpoint{4.685218in}{0.530063in}}{\pgfqpoint{4.677404in}{0.522250in}}%
\pgfpathcurveto{\pgfqpoint{4.669590in}{0.514436in}}{\pgfqpoint{4.665200in}{0.503837in}}{\pgfqpoint{4.665200in}{0.492787in}}%
\pgfpathcurveto{\pgfqpoint{4.665200in}{0.481737in}}{\pgfqpoint{4.669590in}{0.471138in}}{\pgfqpoint{4.677404in}{0.463324in}}%
\pgfpathcurveto{\pgfqpoint{4.685218in}{0.455510in}}{\pgfqpoint{4.695817in}{0.451120in}}{\pgfqpoint{4.706867in}{0.451120in}}%
\pgfpathlineto{\pgfqpoint{4.706867in}{0.451120in}}%
\pgfpathclose%
\pgfusepath{stroke}%
\end{pgfscope}%
\begin{pgfscope}%
\pgfpathrectangle{\pgfqpoint{0.494722in}{0.437222in}}{\pgfqpoint{6.275590in}{5.159444in}}%
\pgfusepath{clip}%
\pgfsetbuttcap%
\pgfsetroundjoin%
\pgfsetlinewidth{1.003750pt}%
\definecolor{currentstroke}{rgb}{0.827451,0.827451,0.827451}%
\pgfsetstrokecolor{currentstroke}%
\pgfsetstrokeopacity{0.800000}%
\pgfsetdash{}{0pt}%
\pgfpathmoveto{\pgfqpoint{0.903250in}{2.772216in}}%
\pgfpathcurveto{\pgfqpoint{0.914300in}{2.772216in}}{\pgfqpoint{0.924899in}{2.776607in}}{\pgfqpoint{0.932712in}{2.784420in}}%
\pgfpathcurveto{\pgfqpoint{0.940526in}{2.792234in}}{\pgfqpoint{0.944916in}{2.802833in}}{\pgfqpoint{0.944916in}{2.813883in}}%
\pgfpathcurveto{\pgfqpoint{0.944916in}{2.824933in}}{\pgfqpoint{0.940526in}{2.835532in}}{\pgfqpoint{0.932712in}{2.843346in}}%
\pgfpathcurveto{\pgfqpoint{0.924899in}{2.851159in}}{\pgfqpoint{0.914300in}{2.855550in}}{\pgfqpoint{0.903250in}{2.855550in}}%
\pgfpathcurveto{\pgfqpoint{0.892199in}{2.855550in}}{\pgfqpoint{0.881600in}{2.851159in}}{\pgfqpoint{0.873787in}{2.843346in}}%
\pgfpathcurveto{\pgfqpoint{0.865973in}{2.835532in}}{\pgfqpoint{0.861583in}{2.824933in}}{\pgfqpoint{0.861583in}{2.813883in}}%
\pgfpathcurveto{\pgfqpoint{0.861583in}{2.802833in}}{\pgfqpoint{0.865973in}{2.792234in}}{\pgfqpoint{0.873787in}{2.784420in}}%
\pgfpathcurveto{\pgfqpoint{0.881600in}{2.776607in}}{\pgfqpoint{0.892199in}{2.772216in}}{\pgfqpoint{0.903250in}{2.772216in}}%
\pgfpathlineto{\pgfqpoint{0.903250in}{2.772216in}}%
\pgfpathclose%
\pgfusepath{stroke}%
\end{pgfscope}%
\begin{pgfscope}%
\pgfpathrectangle{\pgfqpoint{0.494722in}{0.437222in}}{\pgfqpoint{6.275590in}{5.159444in}}%
\pgfusepath{clip}%
\pgfsetbuttcap%
\pgfsetroundjoin%
\pgfsetlinewidth{1.003750pt}%
\definecolor{currentstroke}{rgb}{0.827451,0.827451,0.827451}%
\pgfsetstrokecolor{currentstroke}%
\pgfsetstrokeopacity{0.800000}%
\pgfsetdash{}{0pt}%
\pgfpathmoveto{\pgfqpoint{5.378653in}{0.404615in}}%
\pgfpathcurveto{\pgfqpoint{5.389703in}{0.404615in}}{\pgfqpoint{5.400302in}{0.409005in}}{\pgfqpoint{5.408116in}{0.416819in}}%
\pgfpathcurveto{\pgfqpoint{5.415930in}{0.424632in}}{\pgfqpoint{5.420320in}{0.435231in}}{\pgfqpoint{5.420320in}{0.446281in}}%
\pgfpathcurveto{\pgfqpoint{5.420320in}{0.457331in}}{\pgfqpoint{5.415930in}{0.467930in}}{\pgfqpoint{5.408116in}{0.475744in}}%
\pgfpathcurveto{\pgfqpoint{5.400302in}{0.483558in}}{\pgfqpoint{5.389703in}{0.487948in}}{\pgfqpoint{5.378653in}{0.487948in}}%
\pgfpathcurveto{\pgfqpoint{5.367603in}{0.487948in}}{\pgfqpoint{5.357004in}{0.483558in}}{\pgfqpoint{5.349190in}{0.475744in}}%
\pgfpathcurveto{\pgfqpoint{5.341377in}{0.467930in}}{\pgfqpoint{5.336987in}{0.457331in}}{\pgfqpoint{5.336987in}{0.446281in}}%
\pgfpathcurveto{\pgfqpoint{5.336987in}{0.435231in}}{\pgfqpoint{5.341377in}{0.424632in}}{\pgfqpoint{5.349190in}{0.416819in}}%
\pgfpathcurveto{\pgfqpoint{5.357004in}{0.409005in}}{\pgfqpoint{5.367603in}{0.404615in}}{\pgfqpoint{5.378653in}{0.404615in}}%
\pgfusepath{stroke}%
\end{pgfscope}%
\begin{pgfscope}%
\pgfpathrectangle{\pgfqpoint{0.494722in}{0.437222in}}{\pgfqpoint{6.275590in}{5.159444in}}%
\pgfusepath{clip}%
\pgfsetbuttcap%
\pgfsetroundjoin%
\pgfsetlinewidth{1.003750pt}%
\definecolor{currentstroke}{rgb}{0.827451,0.827451,0.827451}%
\pgfsetstrokecolor{currentstroke}%
\pgfsetstrokeopacity{0.800000}%
\pgfsetdash{}{0pt}%
\pgfpathmoveto{\pgfqpoint{2.387393in}{1.221780in}}%
\pgfpathcurveto{\pgfqpoint{2.398443in}{1.221780in}}{\pgfqpoint{2.409042in}{1.226170in}}{\pgfqpoint{2.416855in}{1.233984in}}%
\pgfpathcurveto{\pgfqpoint{2.424669in}{1.241798in}}{\pgfqpoint{2.429059in}{1.252397in}}{\pgfqpoint{2.429059in}{1.263447in}}%
\pgfpathcurveto{\pgfqpoint{2.429059in}{1.274497in}}{\pgfqpoint{2.424669in}{1.285096in}}{\pgfqpoint{2.416855in}{1.292909in}}%
\pgfpathcurveto{\pgfqpoint{2.409042in}{1.300723in}}{\pgfqpoint{2.398443in}{1.305113in}}{\pgfqpoint{2.387393in}{1.305113in}}%
\pgfpathcurveto{\pgfqpoint{2.376343in}{1.305113in}}{\pgfqpoint{2.365743in}{1.300723in}}{\pgfqpoint{2.357930in}{1.292909in}}%
\pgfpathcurveto{\pgfqpoint{2.350116in}{1.285096in}}{\pgfqpoint{2.345726in}{1.274497in}}{\pgfqpoint{2.345726in}{1.263447in}}%
\pgfpathcurveto{\pgfqpoint{2.345726in}{1.252397in}}{\pgfqpoint{2.350116in}{1.241798in}}{\pgfqpoint{2.357930in}{1.233984in}}%
\pgfpathcurveto{\pgfqpoint{2.365743in}{1.226170in}}{\pgfqpoint{2.376343in}{1.221780in}}{\pgfqpoint{2.387393in}{1.221780in}}%
\pgfpathlineto{\pgfqpoint{2.387393in}{1.221780in}}%
\pgfpathclose%
\pgfusepath{stroke}%
\end{pgfscope}%
\begin{pgfscope}%
\pgfpathrectangle{\pgfqpoint{0.494722in}{0.437222in}}{\pgfqpoint{6.275590in}{5.159444in}}%
\pgfusepath{clip}%
\pgfsetbuttcap%
\pgfsetroundjoin%
\pgfsetlinewidth{1.003750pt}%
\definecolor{currentstroke}{rgb}{0.827451,0.827451,0.827451}%
\pgfsetstrokecolor{currentstroke}%
\pgfsetstrokeopacity{0.800000}%
\pgfsetdash{}{0pt}%
\pgfpathmoveto{\pgfqpoint{1.786945in}{1.657846in}}%
\pgfpathcurveto{\pgfqpoint{1.797995in}{1.657846in}}{\pgfqpoint{1.808594in}{1.662237in}}{\pgfqpoint{1.816407in}{1.670050in}}%
\pgfpathcurveto{\pgfqpoint{1.824221in}{1.677864in}}{\pgfqpoint{1.828611in}{1.688463in}}{\pgfqpoint{1.828611in}{1.699513in}}%
\pgfpathcurveto{\pgfqpoint{1.828611in}{1.710563in}}{\pgfqpoint{1.824221in}{1.721162in}}{\pgfqpoint{1.816407in}{1.728976in}}%
\pgfpathcurveto{\pgfqpoint{1.808594in}{1.736789in}}{\pgfqpoint{1.797995in}{1.741180in}}{\pgfqpoint{1.786945in}{1.741180in}}%
\pgfpathcurveto{\pgfqpoint{1.775894in}{1.741180in}}{\pgfqpoint{1.765295in}{1.736789in}}{\pgfqpoint{1.757482in}{1.728976in}}%
\pgfpathcurveto{\pgfqpoint{1.749668in}{1.721162in}}{\pgfqpoint{1.745278in}{1.710563in}}{\pgfqpoint{1.745278in}{1.699513in}}%
\pgfpathcurveto{\pgfqpoint{1.745278in}{1.688463in}}{\pgfqpoint{1.749668in}{1.677864in}}{\pgfqpoint{1.757482in}{1.670050in}}%
\pgfpathcurveto{\pgfqpoint{1.765295in}{1.662237in}}{\pgfqpoint{1.775894in}{1.657846in}}{\pgfqpoint{1.786945in}{1.657846in}}%
\pgfpathlineto{\pgfqpoint{1.786945in}{1.657846in}}%
\pgfpathclose%
\pgfusepath{stroke}%
\end{pgfscope}%
\begin{pgfscope}%
\pgfpathrectangle{\pgfqpoint{0.494722in}{0.437222in}}{\pgfqpoint{6.275590in}{5.159444in}}%
\pgfusepath{clip}%
\pgfsetbuttcap%
\pgfsetroundjoin%
\pgfsetlinewidth{1.003750pt}%
\definecolor{currentstroke}{rgb}{0.827451,0.827451,0.827451}%
\pgfsetstrokecolor{currentstroke}%
\pgfsetstrokeopacity{0.800000}%
\pgfsetdash{}{0pt}%
\pgfpathmoveto{\pgfqpoint{0.575099in}{3.779944in}}%
\pgfpathcurveto{\pgfqpoint{0.586149in}{3.779944in}}{\pgfqpoint{0.596748in}{3.784334in}}{\pgfqpoint{0.604561in}{3.792148in}}%
\pgfpathcurveto{\pgfqpoint{0.612375in}{3.799961in}}{\pgfqpoint{0.616765in}{3.810560in}}{\pgfqpoint{0.616765in}{3.821610in}}%
\pgfpathcurveto{\pgfqpoint{0.616765in}{3.832661in}}{\pgfqpoint{0.612375in}{3.843260in}}{\pgfqpoint{0.604561in}{3.851073in}}%
\pgfpathcurveto{\pgfqpoint{0.596748in}{3.858887in}}{\pgfqpoint{0.586149in}{3.863277in}}{\pgfqpoint{0.575099in}{3.863277in}}%
\pgfpathcurveto{\pgfqpoint{0.564048in}{3.863277in}}{\pgfqpoint{0.553449in}{3.858887in}}{\pgfqpoint{0.545636in}{3.851073in}}%
\pgfpathcurveto{\pgfqpoint{0.537822in}{3.843260in}}{\pgfqpoint{0.533432in}{3.832661in}}{\pgfqpoint{0.533432in}{3.821610in}}%
\pgfpathcurveto{\pgfqpoint{0.533432in}{3.810560in}}{\pgfqpoint{0.537822in}{3.799961in}}{\pgfqpoint{0.545636in}{3.792148in}}%
\pgfpathcurveto{\pgfqpoint{0.553449in}{3.784334in}}{\pgfqpoint{0.564048in}{3.779944in}}{\pgfqpoint{0.575099in}{3.779944in}}%
\pgfpathlineto{\pgfqpoint{0.575099in}{3.779944in}}%
\pgfpathclose%
\pgfusepath{stroke}%
\end{pgfscope}%
\begin{pgfscope}%
\pgfpathrectangle{\pgfqpoint{0.494722in}{0.437222in}}{\pgfqpoint{6.275590in}{5.159444in}}%
\pgfusepath{clip}%
\pgfsetbuttcap%
\pgfsetroundjoin%
\pgfsetlinewidth{1.003750pt}%
\definecolor{currentstroke}{rgb}{0.827451,0.827451,0.827451}%
\pgfsetstrokecolor{currentstroke}%
\pgfsetstrokeopacity{0.800000}%
\pgfsetdash{}{0pt}%
\pgfpathmoveto{\pgfqpoint{0.560756in}{3.841058in}}%
\pgfpathcurveto{\pgfqpoint{0.571806in}{3.841058in}}{\pgfqpoint{0.582405in}{3.845449in}}{\pgfqpoint{0.590219in}{3.853262in}}%
\pgfpathcurveto{\pgfqpoint{0.598032in}{3.861076in}}{\pgfqpoint{0.602423in}{3.871675in}}{\pgfqpoint{0.602423in}{3.882725in}}%
\pgfpathcurveto{\pgfqpoint{0.602423in}{3.893775in}}{\pgfqpoint{0.598032in}{3.904374in}}{\pgfqpoint{0.590219in}{3.912188in}}%
\pgfpathcurveto{\pgfqpoint{0.582405in}{3.920001in}}{\pgfqpoint{0.571806in}{3.924392in}}{\pgfqpoint{0.560756in}{3.924392in}}%
\pgfpathcurveto{\pgfqpoint{0.549706in}{3.924392in}}{\pgfqpoint{0.539107in}{3.920001in}}{\pgfqpoint{0.531293in}{3.912188in}}%
\pgfpathcurveto{\pgfqpoint{0.523479in}{3.904374in}}{\pgfqpoint{0.519089in}{3.893775in}}{\pgfqpoint{0.519089in}{3.882725in}}%
\pgfpathcurveto{\pgfqpoint{0.519089in}{3.871675in}}{\pgfqpoint{0.523479in}{3.861076in}}{\pgfqpoint{0.531293in}{3.853262in}}%
\pgfpathcurveto{\pgfqpoint{0.539107in}{3.845449in}}{\pgfqpoint{0.549706in}{3.841058in}}{\pgfqpoint{0.560756in}{3.841058in}}%
\pgfpathlineto{\pgfqpoint{0.560756in}{3.841058in}}%
\pgfpathclose%
\pgfusepath{stroke}%
\end{pgfscope}%
\begin{pgfscope}%
\pgfpathrectangle{\pgfqpoint{0.494722in}{0.437222in}}{\pgfqpoint{6.275590in}{5.159444in}}%
\pgfusepath{clip}%
\pgfsetbuttcap%
\pgfsetroundjoin%
\pgfsetlinewidth{1.003750pt}%
\definecolor{currentstroke}{rgb}{0.827451,0.827451,0.827451}%
\pgfsetstrokecolor{currentstroke}%
\pgfsetstrokeopacity{0.800000}%
\pgfsetdash{}{0pt}%
\pgfpathmoveto{\pgfqpoint{5.677571in}{0.399663in}}%
\pgfpathcurveto{\pgfqpoint{5.688621in}{0.399663in}}{\pgfqpoint{5.699220in}{0.404053in}}{\pgfqpoint{5.707034in}{0.411866in}}%
\pgfpathcurveto{\pgfqpoint{5.714847in}{0.419680in}}{\pgfqpoint{5.719238in}{0.430279in}}{\pgfqpoint{5.719238in}{0.441329in}}%
\pgfpathcurveto{\pgfqpoint{5.719238in}{0.452379in}}{\pgfqpoint{5.714847in}{0.462978in}}{\pgfqpoint{5.707034in}{0.470792in}}%
\pgfpathcurveto{\pgfqpoint{5.699220in}{0.478606in}}{\pgfqpoint{5.688621in}{0.482996in}}{\pgfqpoint{5.677571in}{0.482996in}}%
\pgfpathcurveto{\pgfqpoint{5.666521in}{0.482996in}}{\pgfqpoint{5.655922in}{0.478606in}}{\pgfqpoint{5.648108in}{0.470792in}}%
\pgfpathcurveto{\pgfqpoint{5.640294in}{0.462978in}}{\pgfqpoint{5.635904in}{0.452379in}}{\pgfqpoint{5.635904in}{0.441329in}}%
\pgfpathcurveto{\pgfqpoint{5.635904in}{0.430279in}}{\pgfqpoint{5.640294in}{0.419680in}}{\pgfqpoint{5.648108in}{0.411866in}}%
\pgfpathcurveto{\pgfqpoint{5.655922in}{0.404053in}}{\pgfqpoint{5.666521in}{0.399663in}}{\pgfqpoint{5.677571in}{0.399663in}}%
\pgfusepath{stroke}%
\end{pgfscope}%
\begin{pgfscope}%
\pgfpathrectangle{\pgfqpoint{0.494722in}{0.437222in}}{\pgfqpoint{6.275590in}{5.159444in}}%
\pgfusepath{clip}%
\pgfsetbuttcap%
\pgfsetroundjoin%
\pgfsetlinewidth{1.003750pt}%
\definecolor{currentstroke}{rgb}{0.827451,0.827451,0.827451}%
\pgfsetstrokecolor{currentstroke}%
\pgfsetstrokeopacity{0.800000}%
\pgfsetdash{}{0pt}%
\pgfpathmoveto{\pgfqpoint{1.464755in}{1.963622in}}%
\pgfpathcurveto{\pgfqpoint{1.475805in}{1.963622in}}{\pgfqpoint{1.486404in}{1.968013in}}{\pgfqpoint{1.494217in}{1.975826in}}%
\pgfpathcurveto{\pgfqpoint{1.502031in}{1.983640in}}{\pgfqpoint{1.506421in}{1.994239in}}{\pgfqpoint{1.506421in}{2.005289in}}%
\pgfpathcurveto{\pgfqpoint{1.506421in}{2.016339in}}{\pgfqpoint{1.502031in}{2.026938in}}{\pgfqpoint{1.494217in}{2.034752in}}%
\pgfpathcurveto{\pgfqpoint{1.486404in}{2.042566in}}{\pgfqpoint{1.475805in}{2.046956in}}{\pgfqpoint{1.464755in}{2.046956in}}%
\pgfpathcurveto{\pgfqpoint{1.453705in}{2.046956in}}{\pgfqpoint{1.443106in}{2.042566in}}{\pgfqpoint{1.435292in}{2.034752in}}%
\pgfpathcurveto{\pgfqpoint{1.427478in}{2.026938in}}{\pgfqpoint{1.423088in}{2.016339in}}{\pgfqpoint{1.423088in}{2.005289in}}%
\pgfpathcurveto{\pgfqpoint{1.423088in}{1.994239in}}{\pgfqpoint{1.427478in}{1.983640in}}{\pgfqpoint{1.435292in}{1.975826in}}%
\pgfpathcurveto{\pgfqpoint{1.443106in}{1.968013in}}{\pgfqpoint{1.453705in}{1.963622in}}{\pgfqpoint{1.464755in}{1.963622in}}%
\pgfpathlineto{\pgfqpoint{1.464755in}{1.963622in}}%
\pgfpathclose%
\pgfusepath{stroke}%
\end{pgfscope}%
\begin{pgfscope}%
\pgfpathrectangle{\pgfqpoint{0.494722in}{0.437222in}}{\pgfqpoint{6.275590in}{5.159444in}}%
\pgfusepath{clip}%
\pgfsetbuttcap%
\pgfsetroundjoin%
\pgfsetlinewidth{1.003750pt}%
\definecolor{currentstroke}{rgb}{0.827451,0.827451,0.827451}%
\pgfsetstrokecolor{currentstroke}%
\pgfsetstrokeopacity{0.800000}%
\pgfsetdash{}{0pt}%
\pgfpathmoveto{\pgfqpoint{2.117156in}{1.399260in}}%
\pgfpathcurveto{\pgfqpoint{2.128206in}{1.399260in}}{\pgfqpoint{2.138805in}{1.403650in}}{\pgfqpoint{2.146618in}{1.411464in}}%
\pgfpathcurveto{\pgfqpoint{2.154432in}{1.419277in}}{\pgfqpoint{2.158822in}{1.429876in}}{\pgfqpoint{2.158822in}{1.440927in}}%
\pgfpathcurveto{\pgfqpoint{2.158822in}{1.451977in}}{\pgfqpoint{2.154432in}{1.462576in}}{\pgfqpoint{2.146618in}{1.470389in}}%
\pgfpathcurveto{\pgfqpoint{2.138805in}{1.478203in}}{\pgfqpoint{2.128206in}{1.482593in}}{\pgfqpoint{2.117156in}{1.482593in}}%
\pgfpathcurveto{\pgfqpoint{2.106106in}{1.482593in}}{\pgfqpoint{2.095506in}{1.478203in}}{\pgfqpoint{2.087693in}{1.470389in}}%
\pgfpathcurveto{\pgfqpoint{2.079879in}{1.462576in}}{\pgfqpoint{2.075489in}{1.451977in}}{\pgfqpoint{2.075489in}{1.440927in}}%
\pgfpathcurveto{\pgfqpoint{2.075489in}{1.429876in}}{\pgfqpoint{2.079879in}{1.419277in}}{\pgfqpoint{2.087693in}{1.411464in}}%
\pgfpathcurveto{\pgfqpoint{2.095506in}{1.403650in}}{\pgfqpoint{2.106106in}{1.399260in}}{\pgfqpoint{2.117156in}{1.399260in}}%
\pgfpathlineto{\pgfqpoint{2.117156in}{1.399260in}}%
\pgfpathclose%
\pgfusepath{stroke}%
\end{pgfscope}%
\begin{pgfscope}%
\pgfpathrectangle{\pgfqpoint{0.494722in}{0.437222in}}{\pgfqpoint{6.275590in}{5.159444in}}%
\pgfusepath{clip}%
\pgfsetbuttcap%
\pgfsetroundjoin%
\pgfsetlinewidth{1.003750pt}%
\definecolor{currentstroke}{rgb}{0.827451,0.827451,0.827451}%
\pgfsetstrokecolor{currentstroke}%
\pgfsetstrokeopacity{0.800000}%
\pgfsetdash{}{0pt}%
\pgfpathmoveto{\pgfqpoint{1.520178in}{1.905257in}}%
\pgfpathcurveto{\pgfqpoint{1.531228in}{1.905257in}}{\pgfqpoint{1.541827in}{1.909647in}}{\pgfqpoint{1.549641in}{1.917461in}}%
\pgfpathcurveto{\pgfqpoint{1.557455in}{1.925274in}}{\pgfqpoint{1.561845in}{1.935873in}}{\pgfqpoint{1.561845in}{1.946923in}}%
\pgfpathcurveto{\pgfqpoint{1.561845in}{1.957973in}}{\pgfqpoint{1.557455in}{1.968573in}}{\pgfqpoint{1.549641in}{1.976386in}}%
\pgfpathcurveto{\pgfqpoint{1.541827in}{1.984200in}}{\pgfqpoint{1.531228in}{1.988590in}}{\pgfqpoint{1.520178in}{1.988590in}}%
\pgfpathcurveto{\pgfqpoint{1.509128in}{1.988590in}}{\pgfqpoint{1.498529in}{1.984200in}}{\pgfqpoint{1.490715in}{1.976386in}}%
\pgfpathcurveto{\pgfqpoint{1.482902in}{1.968573in}}{\pgfqpoint{1.478512in}{1.957973in}}{\pgfqpoint{1.478512in}{1.946923in}}%
\pgfpathcurveto{\pgfqpoint{1.478512in}{1.935873in}}{\pgfqpoint{1.482902in}{1.925274in}}{\pgfqpoint{1.490715in}{1.917461in}}%
\pgfpathcurveto{\pgfqpoint{1.498529in}{1.909647in}}{\pgfqpoint{1.509128in}{1.905257in}}{\pgfqpoint{1.520178in}{1.905257in}}%
\pgfpathlineto{\pgfqpoint{1.520178in}{1.905257in}}%
\pgfpathclose%
\pgfusepath{stroke}%
\end{pgfscope}%
\begin{pgfscope}%
\pgfpathrectangle{\pgfqpoint{0.494722in}{0.437222in}}{\pgfqpoint{6.275590in}{5.159444in}}%
\pgfusepath{clip}%
\pgfsetbuttcap%
\pgfsetroundjoin%
\pgfsetlinewidth{1.003750pt}%
\definecolor{currentstroke}{rgb}{0.827451,0.827451,0.827451}%
\pgfsetstrokecolor{currentstroke}%
\pgfsetstrokeopacity{0.800000}%
\pgfsetdash{}{0pt}%
\pgfpathmoveto{\pgfqpoint{0.611323in}{3.579035in}}%
\pgfpathcurveto{\pgfqpoint{0.622373in}{3.579035in}}{\pgfqpoint{0.632972in}{3.583425in}}{\pgfqpoint{0.640786in}{3.591238in}}%
\pgfpathcurveto{\pgfqpoint{0.648599in}{3.599052in}}{\pgfqpoint{0.652989in}{3.609651in}}{\pgfqpoint{0.652989in}{3.620701in}}%
\pgfpathcurveto{\pgfqpoint{0.652989in}{3.631751in}}{\pgfqpoint{0.648599in}{3.642350in}}{\pgfqpoint{0.640786in}{3.650164in}}%
\pgfpathcurveto{\pgfqpoint{0.632972in}{3.657978in}}{\pgfqpoint{0.622373in}{3.662368in}}{\pgfqpoint{0.611323in}{3.662368in}}%
\pgfpathcurveto{\pgfqpoint{0.600273in}{3.662368in}}{\pgfqpoint{0.589674in}{3.657978in}}{\pgfqpoint{0.581860in}{3.650164in}}%
\pgfpathcurveto{\pgfqpoint{0.574046in}{3.642350in}}{\pgfqpoint{0.569656in}{3.631751in}}{\pgfqpoint{0.569656in}{3.620701in}}%
\pgfpathcurveto{\pgfqpoint{0.569656in}{3.609651in}}{\pgfqpoint{0.574046in}{3.599052in}}{\pgfqpoint{0.581860in}{3.591238in}}%
\pgfpathcurveto{\pgfqpoint{0.589674in}{3.583425in}}{\pgfqpoint{0.600273in}{3.579035in}}{\pgfqpoint{0.611323in}{3.579035in}}%
\pgfpathlineto{\pgfqpoint{0.611323in}{3.579035in}}%
\pgfpathclose%
\pgfusepath{stroke}%
\end{pgfscope}%
\begin{pgfscope}%
\pgfpathrectangle{\pgfqpoint{0.494722in}{0.437222in}}{\pgfqpoint{6.275590in}{5.159444in}}%
\pgfusepath{clip}%
\pgfsetbuttcap%
\pgfsetroundjoin%
\pgfsetlinewidth{1.003750pt}%
\definecolor{currentstroke}{rgb}{0.827451,0.827451,0.827451}%
\pgfsetstrokecolor{currentstroke}%
\pgfsetstrokeopacity{0.800000}%
\pgfsetdash{}{0pt}%
\pgfpathmoveto{\pgfqpoint{0.645918in}{3.445519in}}%
\pgfpathcurveto{\pgfqpoint{0.656968in}{3.445519in}}{\pgfqpoint{0.667567in}{3.449909in}}{\pgfqpoint{0.675381in}{3.457723in}}%
\pgfpathcurveto{\pgfqpoint{0.683195in}{3.465536in}}{\pgfqpoint{0.687585in}{3.476135in}}{\pgfqpoint{0.687585in}{3.487185in}}%
\pgfpathcurveto{\pgfqpoint{0.687585in}{3.498235in}}{\pgfqpoint{0.683195in}{3.508835in}}{\pgfqpoint{0.675381in}{3.516648in}}%
\pgfpathcurveto{\pgfqpoint{0.667567in}{3.524462in}}{\pgfqpoint{0.656968in}{3.528852in}}{\pgfqpoint{0.645918in}{3.528852in}}%
\pgfpathcurveto{\pgfqpoint{0.634868in}{3.528852in}}{\pgfqpoint{0.624269in}{3.524462in}}{\pgfqpoint{0.616455in}{3.516648in}}%
\pgfpathcurveto{\pgfqpoint{0.608642in}{3.508835in}}{\pgfqpoint{0.604251in}{3.498235in}}{\pgfqpoint{0.604251in}{3.487185in}}%
\pgfpathcurveto{\pgfqpoint{0.604251in}{3.476135in}}{\pgfqpoint{0.608642in}{3.465536in}}{\pgfqpoint{0.616455in}{3.457723in}}%
\pgfpathcurveto{\pgfqpoint{0.624269in}{3.449909in}}{\pgfqpoint{0.634868in}{3.445519in}}{\pgfqpoint{0.645918in}{3.445519in}}%
\pgfpathlineto{\pgfqpoint{0.645918in}{3.445519in}}%
\pgfpathclose%
\pgfusepath{stroke}%
\end{pgfscope}%
\begin{pgfscope}%
\pgfpathrectangle{\pgfqpoint{0.494722in}{0.437222in}}{\pgfqpoint{6.275590in}{5.159444in}}%
\pgfusepath{clip}%
\pgfsetbuttcap%
\pgfsetroundjoin%
\pgfsetlinewidth{1.003750pt}%
\definecolor{currentstroke}{rgb}{0.827451,0.827451,0.827451}%
\pgfsetstrokecolor{currentstroke}%
\pgfsetstrokeopacity{0.800000}%
\pgfsetdash{}{0pt}%
\pgfpathmoveto{\pgfqpoint{2.452460in}{1.181670in}}%
\pgfpathcurveto{\pgfqpoint{2.463510in}{1.181670in}}{\pgfqpoint{2.474109in}{1.186061in}}{\pgfqpoint{2.481923in}{1.193874in}}%
\pgfpathcurveto{\pgfqpoint{2.489737in}{1.201688in}}{\pgfqpoint{2.494127in}{1.212287in}}{\pgfqpoint{2.494127in}{1.223337in}}%
\pgfpathcurveto{\pgfqpoint{2.494127in}{1.234387in}}{\pgfqpoint{2.489737in}{1.244986in}}{\pgfqpoint{2.481923in}{1.252800in}}%
\pgfpathcurveto{\pgfqpoint{2.474109in}{1.260613in}}{\pgfqpoint{2.463510in}{1.265004in}}{\pgfqpoint{2.452460in}{1.265004in}}%
\pgfpathcurveto{\pgfqpoint{2.441410in}{1.265004in}}{\pgfqpoint{2.430811in}{1.260613in}}{\pgfqpoint{2.422998in}{1.252800in}}%
\pgfpathcurveto{\pgfqpoint{2.415184in}{1.244986in}}{\pgfqpoint{2.410794in}{1.234387in}}{\pgfqpoint{2.410794in}{1.223337in}}%
\pgfpathcurveto{\pgfqpoint{2.410794in}{1.212287in}}{\pgfqpoint{2.415184in}{1.201688in}}{\pgfqpoint{2.422998in}{1.193874in}}%
\pgfpathcurveto{\pgfqpoint{2.430811in}{1.186061in}}{\pgfqpoint{2.441410in}{1.181670in}}{\pgfqpoint{2.452460in}{1.181670in}}%
\pgfpathlineto{\pgfqpoint{2.452460in}{1.181670in}}%
\pgfpathclose%
\pgfusepath{stroke}%
\end{pgfscope}%
\begin{pgfscope}%
\pgfpathrectangle{\pgfqpoint{0.494722in}{0.437222in}}{\pgfqpoint{6.275590in}{5.159444in}}%
\pgfusepath{clip}%
\pgfsetbuttcap%
\pgfsetroundjoin%
\pgfsetlinewidth{1.003750pt}%
\definecolor{currentstroke}{rgb}{0.827451,0.827451,0.827451}%
\pgfsetstrokecolor{currentstroke}%
\pgfsetstrokeopacity{0.800000}%
\pgfsetdash{}{0pt}%
\pgfpathmoveto{\pgfqpoint{0.734476in}{3.173741in}}%
\pgfpathcurveto{\pgfqpoint{0.745526in}{3.173741in}}{\pgfqpoint{0.756125in}{3.178131in}}{\pgfqpoint{0.763939in}{3.185945in}}%
\pgfpathcurveto{\pgfqpoint{0.771753in}{3.193758in}}{\pgfqpoint{0.776143in}{3.204357in}}{\pgfqpoint{0.776143in}{3.215408in}}%
\pgfpathcurveto{\pgfqpoint{0.776143in}{3.226458in}}{\pgfqpoint{0.771753in}{3.237057in}}{\pgfqpoint{0.763939in}{3.244870in}}%
\pgfpathcurveto{\pgfqpoint{0.756125in}{3.252684in}}{\pgfqpoint{0.745526in}{3.257074in}}{\pgfqpoint{0.734476in}{3.257074in}}%
\pgfpathcurveto{\pgfqpoint{0.723426in}{3.257074in}}{\pgfqpoint{0.712827in}{3.252684in}}{\pgfqpoint{0.705013in}{3.244870in}}%
\pgfpathcurveto{\pgfqpoint{0.697200in}{3.237057in}}{\pgfqpoint{0.692810in}{3.226458in}}{\pgfqpoint{0.692810in}{3.215408in}}%
\pgfpathcurveto{\pgfqpoint{0.692810in}{3.204357in}}{\pgfqpoint{0.697200in}{3.193758in}}{\pgfqpoint{0.705013in}{3.185945in}}%
\pgfpathcurveto{\pgfqpoint{0.712827in}{3.178131in}}{\pgfqpoint{0.723426in}{3.173741in}}{\pgfqpoint{0.734476in}{3.173741in}}%
\pgfpathlineto{\pgfqpoint{0.734476in}{3.173741in}}%
\pgfpathclose%
\pgfusepath{stroke}%
\end{pgfscope}%
\begin{pgfscope}%
\pgfpathrectangle{\pgfqpoint{0.494722in}{0.437222in}}{\pgfqpoint{6.275590in}{5.159444in}}%
\pgfusepath{clip}%
\pgfsetbuttcap%
\pgfsetroundjoin%
\pgfsetlinewidth{1.003750pt}%
\definecolor{currentstroke}{rgb}{0.827451,0.827451,0.827451}%
\pgfsetstrokecolor{currentstroke}%
\pgfsetstrokeopacity{0.800000}%
\pgfsetdash{}{0pt}%
\pgfpathmoveto{\pgfqpoint{0.872365in}{2.839983in}}%
\pgfpathcurveto{\pgfqpoint{0.883415in}{2.839983in}}{\pgfqpoint{0.894014in}{2.844373in}}{\pgfqpoint{0.901828in}{2.852187in}}%
\pgfpathcurveto{\pgfqpoint{0.909642in}{2.860000in}}{\pgfqpoint{0.914032in}{2.870599in}}{\pgfqpoint{0.914032in}{2.881650in}}%
\pgfpathcurveto{\pgfqpoint{0.914032in}{2.892700in}}{\pgfqpoint{0.909642in}{2.903299in}}{\pgfqpoint{0.901828in}{2.911112in}}%
\pgfpathcurveto{\pgfqpoint{0.894014in}{2.918926in}}{\pgfqpoint{0.883415in}{2.923316in}}{\pgfqpoint{0.872365in}{2.923316in}}%
\pgfpathcurveto{\pgfqpoint{0.861315in}{2.923316in}}{\pgfqpoint{0.850716in}{2.918926in}}{\pgfqpoint{0.842902in}{2.911112in}}%
\pgfpathcurveto{\pgfqpoint{0.835089in}{2.903299in}}{\pgfqpoint{0.830699in}{2.892700in}}{\pgfqpoint{0.830699in}{2.881650in}}%
\pgfpathcurveto{\pgfqpoint{0.830699in}{2.870599in}}{\pgfqpoint{0.835089in}{2.860000in}}{\pgfqpoint{0.842902in}{2.852187in}}%
\pgfpathcurveto{\pgfqpoint{0.850716in}{2.844373in}}{\pgfqpoint{0.861315in}{2.839983in}}{\pgfqpoint{0.872365in}{2.839983in}}%
\pgfpathlineto{\pgfqpoint{0.872365in}{2.839983in}}%
\pgfpathclose%
\pgfusepath{stroke}%
\end{pgfscope}%
\begin{pgfscope}%
\pgfpathrectangle{\pgfqpoint{0.494722in}{0.437222in}}{\pgfqpoint{6.275590in}{5.159444in}}%
\pgfusepath{clip}%
\pgfsetbuttcap%
\pgfsetroundjoin%
\pgfsetlinewidth{1.003750pt}%
\definecolor{currentstroke}{rgb}{0.827451,0.827451,0.827451}%
\pgfsetstrokecolor{currentstroke}%
\pgfsetstrokeopacity{0.800000}%
\pgfsetdash{}{0pt}%
\pgfpathmoveto{\pgfqpoint{1.964907in}{1.517037in}}%
\pgfpathcurveto{\pgfqpoint{1.975957in}{1.517037in}}{\pgfqpoint{1.986556in}{1.521427in}}{\pgfqpoint{1.994370in}{1.529240in}}%
\pgfpathcurveto{\pgfqpoint{2.002183in}{1.537054in}}{\pgfqpoint{2.006573in}{1.547653in}}{\pgfqpoint{2.006573in}{1.558703in}}%
\pgfpathcurveto{\pgfqpoint{2.006573in}{1.569753in}}{\pgfqpoint{2.002183in}{1.580352in}}{\pgfqpoint{1.994370in}{1.588166in}}%
\pgfpathcurveto{\pgfqpoint{1.986556in}{1.595980in}}{\pgfqpoint{1.975957in}{1.600370in}}{\pgfqpoint{1.964907in}{1.600370in}}%
\pgfpathcurveto{\pgfqpoint{1.953857in}{1.600370in}}{\pgfqpoint{1.943258in}{1.595980in}}{\pgfqpoint{1.935444in}{1.588166in}}%
\pgfpathcurveto{\pgfqpoint{1.927630in}{1.580352in}}{\pgfqpoint{1.923240in}{1.569753in}}{\pgfqpoint{1.923240in}{1.558703in}}%
\pgfpathcurveto{\pgfqpoint{1.923240in}{1.547653in}}{\pgfqpoint{1.927630in}{1.537054in}}{\pgfqpoint{1.935444in}{1.529240in}}%
\pgfpathcurveto{\pgfqpoint{1.943258in}{1.521427in}}{\pgfqpoint{1.953857in}{1.517037in}}{\pgfqpoint{1.964907in}{1.517037in}}%
\pgfpathlineto{\pgfqpoint{1.964907in}{1.517037in}}%
\pgfpathclose%
\pgfusepath{stroke}%
\end{pgfscope}%
\begin{pgfscope}%
\pgfpathrectangle{\pgfqpoint{0.494722in}{0.437222in}}{\pgfqpoint{6.275590in}{5.159444in}}%
\pgfusepath{clip}%
\pgfsetbuttcap%
\pgfsetroundjoin%
\pgfsetlinewidth{1.003750pt}%
\definecolor{currentstroke}{rgb}{0.827451,0.827451,0.827451}%
\pgfsetstrokecolor{currentstroke}%
\pgfsetstrokeopacity{0.800000}%
\pgfsetdash{}{0pt}%
\pgfpathmoveto{\pgfqpoint{1.623472in}{1.825964in}}%
\pgfpathcurveto{\pgfqpoint{1.634522in}{1.825964in}}{\pgfqpoint{1.645121in}{1.830354in}}{\pgfqpoint{1.652935in}{1.838167in}}%
\pgfpathcurveto{\pgfqpoint{1.660748in}{1.845981in}}{\pgfqpoint{1.665139in}{1.856580in}}{\pgfqpoint{1.665139in}{1.867630in}}%
\pgfpathcurveto{\pgfqpoint{1.665139in}{1.878680in}}{\pgfqpoint{1.660748in}{1.889279in}}{\pgfqpoint{1.652935in}{1.897093in}}%
\pgfpathcurveto{\pgfqpoint{1.645121in}{1.904907in}}{\pgfqpoint{1.634522in}{1.909297in}}{\pgfqpoint{1.623472in}{1.909297in}}%
\pgfpathcurveto{\pgfqpoint{1.612422in}{1.909297in}}{\pgfqpoint{1.601823in}{1.904907in}}{\pgfqpoint{1.594009in}{1.897093in}}%
\pgfpathcurveto{\pgfqpoint{1.586196in}{1.889279in}}{\pgfqpoint{1.581805in}{1.878680in}}{\pgfqpoint{1.581805in}{1.867630in}}%
\pgfpathcurveto{\pgfqpoint{1.581805in}{1.856580in}}{\pgfqpoint{1.586196in}{1.845981in}}{\pgfqpoint{1.594009in}{1.838167in}}%
\pgfpathcurveto{\pgfqpoint{1.601823in}{1.830354in}}{\pgfqpoint{1.612422in}{1.825964in}}{\pgfqpoint{1.623472in}{1.825964in}}%
\pgfpathlineto{\pgfqpoint{1.623472in}{1.825964in}}%
\pgfpathclose%
\pgfusepath{stroke}%
\end{pgfscope}%
\begin{pgfscope}%
\pgfpathrectangle{\pgfqpoint{0.494722in}{0.437222in}}{\pgfqpoint{6.275590in}{5.159444in}}%
\pgfusepath{clip}%
\pgfsetbuttcap%
\pgfsetroundjoin%
\pgfsetlinewidth{1.003750pt}%
\definecolor{currentstroke}{rgb}{0.827451,0.827451,0.827451}%
\pgfsetstrokecolor{currentstroke}%
\pgfsetstrokeopacity{0.800000}%
\pgfsetdash{}{0pt}%
\pgfpathmoveto{\pgfqpoint{1.198501in}{2.304477in}}%
\pgfpathcurveto{\pgfqpoint{1.209551in}{2.304477in}}{\pgfqpoint{1.220150in}{2.308867in}}{\pgfqpoint{1.227963in}{2.316681in}}%
\pgfpathcurveto{\pgfqpoint{1.235777in}{2.324495in}}{\pgfqpoint{1.240167in}{2.335094in}}{\pgfqpoint{1.240167in}{2.346144in}}%
\pgfpathcurveto{\pgfqpoint{1.240167in}{2.357194in}}{\pgfqpoint{1.235777in}{2.367793in}}{\pgfqpoint{1.227963in}{2.375607in}}%
\pgfpathcurveto{\pgfqpoint{1.220150in}{2.383420in}}{\pgfqpoint{1.209551in}{2.387811in}}{\pgfqpoint{1.198501in}{2.387811in}}%
\pgfpathcurveto{\pgfqpoint{1.187451in}{2.387811in}}{\pgfqpoint{1.176852in}{2.383420in}}{\pgfqpoint{1.169038in}{2.375607in}}%
\pgfpathcurveto{\pgfqpoint{1.161224in}{2.367793in}}{\pgfqpoint{1.156834in}{2.357194in}}{\pgfqpoint{1.156834in}{2.346144in}}%
\pgfpathcurveto{\pgfqpoint{1.156834in}{2.335094in}}{\pgfqpoint{1.161224in}{2.324495in}}{\pgfqpoint{1.169038in}{2.316681in}}%
\pgfpathcurveto{\pgfqpoint{1.176852in}{2.308867in}}{\pgfqpoint{1.187451in}{2.304477in}}{\pgfqpoint{1.198501in}{2.304477in}}%
\pgfpathlineto{\pgfqpoint{1.198501in}{2.304477in}}%
\pgfpathclose%
\pgfusepath{stroke}%
\end{pgfscope}%
\begin{pgfscope}%
\pgfpathrectangle{\pgfqpoint{0.494722in}{0.437222in}}{\pgfqpoint{6.275590in}{5.159444in}}%
\pgfusepath{clip}%
\pgfsetbuttcap%
\pgfsetroundjoin%
\pgfsetlinewidth{1.003750pt}%
\definecolor{currentstroke}{rgb}{0.827451,0.827451,0.827451}%
\pgfsetstrokecolor{currentstroke}%
\pgfsetstrokeopacity{0.800000}%
\pgfsetdash{}{0pt}%
\pgfpathmoveto{\pgfqpoint{4.495815in}{0.487123in}}%
\pgfpathcurveto{\pgfqpoint{4.506865in}{0.487123in}}{\pgfqpoint{4.517464in}{0.491513in}}{\pgfqpoint{4.525277in}{0.499327in}}%
\pgfpathcurveto{\pgfqpoint{4.533091in}{0.507140in}}{\pgfqpoint{4.537481in}{0.517740in}}{\pgfqpoint{4.537481in}{0.528790in}}%
\pgfpathcurveto{\pgfqpoint{4.537481in}{0.539840in}}{\pgfqpoint{4.533091in}{0.550439in}}{\pgfqpoint{4.525277in}{0.558252in}}%
\pgfpathcurveto{\pgfqpoint{4.517464in}{0.566066in}}{\pgfqpoint{4.506865in}{0.570456in}}{\pgfqpoint{4.495815in}{0.570456in}}%
\pgfpathcurveto{\pgfqpoint{4.484765in}{0.570456in}}{\pgfqpoint{4.474166in}{0.566066in}}{\pgfqpoint{4.466352in}{0.558252in}}%
\pgfpathcurveto{\pgfqpoint{4.458538in}{0.550439in}}{\pgfqpoint{4.454148in}{0.539840in}}{\pgfqpoint{4.454148in}{0.528790in}}%
\pgfpathcurveto{\pgfqpoint{4.454148in}{0.517740in}}{\pgfqpoint{4.458538in}{0.507140in}}{\pgfqpoint{4.466352in}{0.499327in}}%
\pgfpathcurveto{\pgfqpoint{4.474166in}{0.491513in}}{\pgfqpoint{4.484765in}{0.487123in}}{\pgfqpoint{4.495815in}{0.487123in}}%
\pgfpathlineto{\pgfqpoint{4.495815in}{0.487123in}}%
\pgfpathclose%
\pgfusepath{stroke}%
\end{pgfscope}%
\begin{pgfscope}%
\pgfpathrectangle{\pgfqpoint{0.494722in}{0.437222in}}{\pgfqpoint{6.275590in}{5.159444in}}%
\pgfusepath{clip}%
\pgfsetbuttcap%
\pgfsetroundjoin%
\pgfsetlinewidth{1.003750pt}%
\definecolor{currentstroke}{rgb}{0.827451,0.827451,0.827451}%
\pgfsetstrokecolor{currentstroke}%
\pgfsetstrokeopacity{0.800000}%
\pgfsetdash{}{0pt}%
\pgfpathmoveto{\pgfqpoint{0.750297in}{3.133310in}}%
\pgfpathcurveto{\pgfqpoint{0.761348in}{3.133310in}}{\pgfqpoint{0.771947in}{3.137700in}}{\pgfqpoint{0.779760in}{3.145514in}}%
\pgfpathcurveto{\pgfqpoint{0.787574in}{3.153328in}}{\pgfqpoint{0.791964in}{3.163927in}}{\pgfqpoint{0.791964in}{3.174977in}}%
\pgfpathcurveto{\pgfqpoint{0.791964in}{3.186027in}}{\pgfqpoint{0.787574in}{3.196626in}}{\pgfqpoint{0.779760in}{3.204439in}}%
\pgfpathcurveto{\pgfqpoint{0.771947in}{3.212253in}}{\pgfqpoint{0.761348in}{3.216643in}}{\pgfqpoint{0.750297in}{3.216643in}}%
\pgfpathcurveto{\pgfqpoint{0.739247in}{3.216643in}}{\pgfqpoint{0.728648in}{3.212253in}}{\pgfqpoint{0.720835in}{3.204439in}}%
\pgfpathcurveto{\pgfqpoint{0.713021in}{3.196626in}}{\pgfqpoint{0.708631in}{3.186027in}}{\pgfqpoint{0.708631in}{3.174977in}}%
\pgfpathcurveto{\pgfqpoint{0.708631in}{3.163927in}}{\pgfqpoint{0.713021in}{3.153328in}}{\pgfqpoint{0.720835in}{3.145514in}}%
\pgfpathcurveto{\pgfqpoint{0.728648in}{3.137700in}}{\pgfqpoint{0.739247in}{3.133310in}}{\pgfqpoint{0.750297in}{3.133310in}}%
\pgfpathlineto{\pgfqpoint{0.750297in}{3.133310in}}%
\pgfpathclose%
\pgfusepath{stroke}%
\end{pgfscope}%
\begin{pgfscope}%
\pgfpathrectangle{\pgfqpoint{0.494722in}{0.437222in}}{\pgfqpoint{6.275590in}{5.159444in}}%
\pgfusepath{clip}%
\pgfsetbuttcap%
\pgfsetroundjoin%
\pgfsetlinewidth{1.003750pt}%
\definecolor{currentstroke}{rgb}{0.827451,0.827451,0.827451}%
\pgfsetstrokecolor{currentstroke}%
\pgfsetstrokeopacity{0.800000}%
\pgfsetdash{}{0pt}%
\pgfpathmoveto{\pgfqpoint{0.770298in}{3.077236in}}%
\pgfpathcurveto{\pgfqpoint{0.781349in}{3.077236in}}{\pgfqpoint{0.791948in}{3.081626in}}{\pgfqpoint{0.799761in}{3.089440in}}%
\pgfpathcurveto{\pgfqpoint{0.807575in}{3.097254in}}{\pgfqpoint{0.811965in}{3.107853in}}{\pgfqpoint{0.811965in}{3.118903in}}%
\pgfpathcurveto{\pgfqpoint{0.811965in}{3.129953in}}{\pgfqpoint{0.807575in}{3.140552in}}{\pgfqpoint{0.799761in}{3.148366in}}%
\pgfpathcurveto{\pgfqpoint{0.791948in}{3.156179in}}{\pgfqpoint{0.781349in}{3.160569in}}{\pgfqpoint{0.770298in}{3.160569in}}%
\pgfpathcurveto{\pgfqpoint{0.759248in}{3.160569in}}{\pgfqpoint{0.748649in}{3.156179in}}{\pgfqpoint{0.740836in}{3.148366in}}%
\pgfpathcurveto{\pgfqpoint{0.733022in}{3.140552in}}{\pgfqpoint{0.728632in}{3.129953in}}{\pgfqpoint{0.728632in}{3.118903in}}%
\pgfpathcurveto{\pgfqpoint{0.728632in}{3.107853in}}{\pgfqpoint{0.733022in}{3.097254in}}{\pgfqpoint{0.740836in}{3.089440in}}%
\pgfpathcurveto{\pgfqpoint{0.748649in}{3.081626in}}{\pgfqpoint{0.759248in}{3.077236in}}{\pgfqpoint{0.770298in}{3.077236in}}%
\pgfpathlineto{\pgfqpoint{0.770298in}{3.077236in}}%
\pgfpathclose%
\pgfusepath{stroke}%
\end{pgfscope}%
\begin{pgfscope}%
\pgfpathrectangle{\pgfqpoint{0.494722in}{0.437222in}}{\pgfqpoint{6.275590in}{5.159444in}}%
\pgfusepath{clip}%
\pgfsetbuttcap%
\pgfsetroundjoin%
\pgfsetlinewidth{1.003750pt}%
\definecolor{currentstroke}{rgb}{0.827451,0.827451,0.827451}%
\pgfsetstrokecolor{currentstroke}%
\pgfsetstrokeopacity{0.800000}%
\pgfsetdash{}{0pt}%
\pgfpathmoveto{\pgfqpoint{4.270074in}{0.511112in}}%
\pgfpathcurveto{\pgfqpoint{4.281124in}{0.511112in}}{\pgfqpoint{4.291723in}{0.515502in}}{\pgfqpoint{4.299537in}{0.523316in}}%
\pgfpathcurveto{\pgfqpoint{4.307351in}{0.531129in}}{\pgfqpoint{4.311741in}{0.541728in}}{\pgfqpoint{4.311741in}{0.552778in}}%
\pgfpathcurveto{\pgfqpoint{4.311741in}{0.563829in}}{\pgfqpoint{4.307351in}{0.574428in}}{\pgfqpoint{4.299537in}{0.582241in}}%
\pgfpathcurveto{\pgfqpoint{4.291723in}{0.590055in}}{\pgfqpoint{4.281124in}{0.594445in}}{\pgfqpoint{4.270074in}{0.594445in}}%
\pgfpathcurveto{\pgfqpoint{4.259024in}{0.594445in}}{\pgfqpoint{4.248425in}{0.590055in}}{\pgfqpoint{4.240611in}{0.582241in}}%
\pgfpathcurveto{\pgfqpoint{4.232798in}{0.574428in}}{\pgfqpoint{4.228407in}{0.563829in}}{\pgfqpoint{4.228407in}{0.552778in}}%
\pgfpathcurveto{\pgfqpoint{4.228407in}{0.541728in}}{\pgfqpoint{4.232798in}{0.531129in}}{\pgfqpoint{4.240611in}{0.523316in}}%
\pgfpathcurveto{\pgfqpoint{4.248425in}{0.515502in}}{\pgfqpoint{4.259024in}{0.511112in}}{\pgfqpoint{4.270074in}{0.511112in}}%
\pgfpathlineto{\pgfqpoint{4.270074in}{0.511112in}}%
\pgfpathclose%
\pgfusepath{stroke}%
\end{pgfscope}%
\begin{pgfscope}%
\pgfpathrectangle{\pgfqpoint{0.494722in}{0.437222in}}{\pgfqpoint{6.275590in}{5.159444in}}%
\pgfusepath{clip}%
\pgfsetbuttcap%
\pgfsetroundjoin%
\pgfsetlinewidth{1.003750pt}%
\definecolor{currentstroke}{rgb}{0.827451,0.827451,0.827451}%
\pgfsetstrokecolor{currentstroke}%
\pgfsetstrokeopacity{0.800000}%
\pgfsetdash{}{0pt}%
\pgfpathmoveto{\pgfqpoint{2.258440in}{1.300760in}}%
\pgfpathcurveto{\pgfqpoint{2.269490in}{1.300760in}}{\pgfqpoint{2.280089in}{1.305150in}}{\pgfqpoint{2.287902in}{1.312964in}}%
\pgfpathcurveto{\pgfqpoint{2.295716in}{1.320777in}}{\pgfqpoint{2.300106in}{1.331376in}}{\pgfqpoint{2.300106in}{1.342426in}}%
\pgfpathcurveto{\pgfqpoint{2.300106in}{1.353477in}}{\pgfqpoint{2.295716in}{1.364076in}}{\pgfqpoint{2.287902in}{1.371889in}}%
\pgfpathcurveto{\pgfqpoint{2.280089in}{1.379703in}}{\pgfqpoint{2.269490in}{1.384093in}}{\pgfqpoint{2.258440in}{1.384093in}}%
\pgfpathcurveto{\pgfqpoint{2.247389in}{1.384093in}}{\pgfqpoint{2.236790in}{1.379703in}}{\pgfqpoint{2.228977in}{1.371889in}}%
\pgfpathcurveto{\pgfqpoint{2.221163in}{1.364076in}}{\pgfqpoint{2.216773in}{1.353477in}}{\pgfqpoint{2.216773in}{1.342426in}}%
\pgfpathcurveto{\pgfqpoint{2.216773in}{1.331376in}}{\pgfqpoint{2.221163in}{1.320777in}}{\pgfqpoint{2.228977in}{1.312964in}}%
\pgfpathcurveto{\pgfqpoint{2.236790in}{1.305150in}}{\pgfqpoint{2.247389in}{1.300760in}}{\pgfqpoint{2.258440in}{1.300760in}}%
\pgfpathlineto{\pgfqpoint{2.258440in}{1.300760in}}%
\pgfpathclose%
\pgfusepath{stroke}%
\end{pgfscope}%
\begin{pgfscope}%
\pgfpathrectangle{\pgfqpoint{0.494722in}{0.437222in}}{\pgfqpoint{6.275590in}{5.159444in}}%
\pgfusepath{clip}%
\pgfsetbuttcap%
\pgfsetroundjoin%
\pgfsetlinewidth{1.003750pt}%
\definecolor{currentstroke}{rgb}{0.827451,0.827451,0.827451}%
\pgfsetstrokecolor{currentstroke}%
\pgfsetstrokeopacity{0.800000}%
\pgfsetdash{}{0pt}%
\pgfpathmoveto{\pgfqpoint{0.679488in}{3.431236in}}%
\pgfpathcurveto{\pgfqpoint{0.690538in}{3.431236in}}{\pgfqpoint{0.701137in}{3.435626in}}{\pgfqpoint{0.708951in}{3.443440in}}%
\pgfpathcurveto{\pgfqpoint{0.716765in}{3.451253in}}{\pgfqpoint{0.721155in}{3.461852in}}{\pgfqpoint{0.721155in}{3.472902in}}%
\pgfpathcurveto{\pgfqpoint{0.721155in}{3.483953in}}{\pgfqpoint{0.716765in}{3.494552in}}{\pgfqpoint{0.708951in}{3.502365in}}%
\pgfpathcurveto{\pgfqpoint{0.701137in}{3.510179in}}{\pgfqpoint{0.690538in}{3.514569in}}{\pgfqpoint{0.679488in}{3.514569in}}%
\pgfpathcurveto{\pgfqpoint{0.668438in}{3.514569in}}{\pgfqpoint{0.657839in}{3.510179in}}{\pgfqpoint{0.650026in}{3.502365in}}%
\pgfpathcurveto{\pgfqpoint{0.642212in}{3.494552in}}{\pgfqpoint{0.637822in}{3.483953in}}{\pgfqpoint{0.637822in}{3.472902in}}%
\pgfpathcurveto{\pgfqpoint{0.637822in}{3.461852in}}{\pgfqpoint{0.642212in}{3.451253in}}{\pgfqpoint{0.650026in}{3.443440in}}%
\pgfpathcurveto{\pgfqpoint{0.657839in}{3.435626in}}{\pgfqpoint{0.668438in}{3.431236in}}{\pgfqpoint{0.679488in}{3.431236in}}%
\pgfpathlineto{\pgfqpoint{0.679488in}{3.431236in}}%
\pgfpathclose%
\pgfusepath{stroke}%
\end{pgfscope}%
\begin{pgfscope}%
\pgfpathrectangle{\pgfqpoint{0.494722in}{0.437222in}}{\pgfqpoint{6.275590in}{5.159444in}}%
\pgfusepath{clip}%
\pgfsetbuttcap%
\pgfsetroundjoin%
\pgfsetlinewidth{1.003750pt}%
\definecolor{currentstroke}{rgb}{0.827451,0.827451,0.827451}%
\pgfsetstrokecolor{currentstroke}%
\pgfsetstrokeopacity{0.800000}%
\pgfsetdash{}{0pt}%
\pgfpathmoveto{\pgfqpoint{0.711186in}{3.233934in}}%
\pgfpathcurveto{\pgfqpoint{0.722236in}{3.233934in}}{\pgfqpoint{0.732835in}{3.238324in}}{\pgfqpoint{0.740648in}{3.246138in}}%
\pgfpathcurveto{\pgfqpoint{0.748462in}{3.253952in}}{\pgfqpoint{0.752852in}{3.264551in}}{\pgfqpoint{0.752852in}{3.275601in}}%
\pgfpathcurveto{\pgfqpoint{0.752852in}{3.286651in}}{\pgfqpoint{0.748462in}{3.297250in}}{\pgfqpoint{0.740648in}{3.305064in}}%
\pgfpathcurveto{\pgfqpoint{0.732835in}{3.312877in}}{\pgfqpoint{0.722236in}{3.317267in}}{\pgfqpoint{0.711186in}{3.317267in}}%
\pgfpathcurveto{\pgfqpoint{0.700135in}{3.317267in}}{\pgfqpoint{0.689536in}{3.312877in}}{\pgfqpoint{0.681723in}{3.305064in}}%
\pgfpathcurveto{\pgfqpoint{0.673909in}{3.297250in}}{\pgfqpoint{0.669519in}{3.286651in}}{\pgfqpoint{0.669519in}{3.275601in}}%
\pgfpathcurveto{\pgfqpoint{0.669519in}{3.264551in}}{\pgfqpoint{0.673909in}{3.253952in}}{\pgfqpoint{0.681723in}{3.246138in}}%
\pgfpathcurveto{\pgfqpoint{0.689536in}{3.238324in}}{\pgfqpoint{0.700135in}{3.233934in}}{\pgfqpoint{0.711186in}{3.233934in}}%
\pgfpathlineto{\pgfqpoint{0.711186in}{3.233934in}}%
\pgfpathclose%
\pgfusepath{stroke}%
\end{pgfscope}%
\begin{pgfscope}%
\pgfpathrectangle{\pgfqpoint{0.494722in}{0.437222in}}{\pgfqpoint{6.275590in}{5.159444in}}%
\pgfusepath{clip}%
\pgfsetbuttcap%
\pgfsetroundjoin%
\pgfsetlinewidth{1.003750pt}%
\definecolor{currentstroke}{rgb}{0.827451,0.827451,0.827451}%
\pgfsetstrokecolor{currentstroke}%
\pgfsetstrokeopacity{0.800000}%
\pgfsetdash{}{0pt}%
\pgfpathmoveto{\pgfqpoint{4.702565in}{0.471432in}}%
\pgfpathcurveto{\pgfqpoint{4.713615in}{0.471432in}}{\pgfqpoint{4.724214in}{0.475823in}}{\pgfqpoint{4.732028in}{0.483636in}}%
\pgfpathcurveto{\pgfqpoint{4.739841in}{0.491450in}}{\pgfqpoint{4.744232in}{0.502049in}}{\pgfqpoint{4.744232in}{0.513099in}}%
\pgfpathcurveto{\pgfqpoint{4.744232in}{0.524149in}}{\pgfqpoint{4.739841in}{0.534748in}}{\pgfqpoint{4.732028in}{0.542562in}}%
\pgfpathcurveto{\pgfqpoint{4.724214in}{0.550376in}}{\pgfqpoint{4.713615in}{0.554766in}}{\pgfqpoint{4.702565in}{0.554766in}}%
\pgfpathcurveto{\pgfqpoint{4.691515in}{0.554766in}}{\pgfqpoint{4.680916in}{0.550376in}}{\pgfqpoint{4.673102in}{0.542562in}}%
\pgfpathcurveto{\pgfqpoint{4.665289in}{0.534748in}}{\pgfqpoint{4.660898in}{0.524149in}}{\pgfqpoint{4.660898in}{0.513099in}}%
\pgfpathcurveto{\pgfqpoint{4.660898in}{0.502049in}}{\pgfqpoint{4.665289in}{0.491450in}}{\pgfqpoint{4.673102in}{0.483636in}}%
\pgfpathcurveto{\pgfqpoint{4.680916in}{0.475823in}}{\pgfqpoint{4.691515in}{0.471432in}}{\pgfqpoint{4.702565in}{0.471432in}}%
\pgfpathlineto{\pgfqpoint{4.702565in}{0.471432in}}%
\pgfpathclose%
\pgfusepath{stroke}%
\end{pgfscope}%
\begin{pgfscope}%
\pgfpathrectangle{\pgfqpoint{0.494722in}{0.437222in}}{\pgfqpoint{6.275590in}{5.159444in}}%
\pgfusepath{clip}%
\pgfsetbuttcap%
\pgfsetroundjoin%
\pgfsetlinewidth{1.003750pt}%
\definecolor{currentstroke}{rgb}{0.827451,0.827451,0.827451}%
\pgfsetstrokecolor{currentstroke}%
\pgfsetstrokeopacity{0.800000}%
\pgfsetdash{}{0pt}%
\pgfpathmoveto{\pgfqpoint{1.402819in}{2.033337in}}%
\pgfpathcurveto{\pgfqpoint{1.413869in}{2.033337in}}{\pgfqpoint{1.424468in}{2.037727in}}{\pgfqpoint{1.432282in}{2.045541in}}%
\pgfpathcurveto{\pgfqpoint{1.440096in}{2.053354in}}{\pgfqpoint{1.444486in}{2.063953in}}{\pgfqpoint{1.444486in}{2.075004in}}%
\pgfpathcurveto{\pgfqpoint{1.444486in}{2.086054in}}{\pgfqpoint{1.440096in}{2.096653in}}{\pgfqpoint{1.432282in}{2.104466in}}%
\pgfpathcurveto{\pgfqpoint{1.424468in}{2.112280in}}{\pgfqpoint{1.413869in}{2.116670in}}{\pgfqpoint{1.402819in}{2.116670in}}%
\pgfpathcurveto{\pgfqpoint{1.391769in}{2.116670in}}{\pgfqpoint{1.381170in}{2.112280in}}{\pgfqpoint{1.373357in}{2.104466in}}%
\pgfpathcurveto{\pgfqpoint{1.365543in}{2.096653in}}{\pgfqpoint{1.361153in}{2.086054in}}{\pgfqpoint{1.361153in}{2.075004in}}%
\pgfpathcurveto{\pgfqpoint{1.361153in}{2.063953in}}{\pgfqpoint{1.365543in}{2.053354in}}{\pgfqpoint{1.373357in}{2.045541in}}%
\pgfpathcurveto{\pgfqpoint{1.381170in}{2.037727in}}{\pgfqpoint{1.391769in}{2.033337in}}{\pgfqpoint{1.402819in}{2.033337in}}%
\pgfpathlineto{\pgfqpoint{1.402819in}{2.033337in}}%
\pgfpathclose%
\pgfusepath{stroke}%
\end{pgfscope}%
\begin{pgfscope}%
\pgfpathrectangle{\pgfqpoint{0.494722in}{0.437222in}}{\pgfqpoint{6.275590in}{5.159444in}}%
\pgfusepath{clip}%
\pgfsetbuttcap%
\pgfsetroundjoin%
\pgfsetlinewidth{1.003750pt}%
\definecolor{currentstroke}{rgb}{0.827451,0.827451,0.827451}%
\pgfsetstrokecolor{currentstroke}%
\pgfsetstrokeopacity{0.800000}%
\pgfsetdash{}{0pt}%
\pgfpathmoveto{\pgfqpoint{1.712263in}{1.726647in}}%
\pgfpathcurveto{\pgfqpoint{1.723313in}{1.726647in}}{\pgfqpoint{1.733912in}{1.731037in}}{\pgfqpoint{1.741726in}{1.738851in}}%
\pgfpathcurveto{\pgfqpoint{1.749539in}{1.746664in}}{\pgfqpoint{1.753929in}{1.757263in}}{\pgfqpoint{1.753929in}{1.768314in}}%
\pgfpathcurveto{\pgfqpoint{1.753929in}{1.779364in}}{\pgfqpoint{1.749539in}{1.789963in}}{\pgfqpoint{1.741726in}{1.797776in}}%
\pgfpathcurveto{\pgfqpoint{1.733912in}{1.805590in}}{\pgfqpoint{1.723313in}{1.809980in}}{\pgfqpoint{1.712263in}{1.809980in}}%
\pgfpathcurveto{\pgfqpoint{1.701213in}{1.809980in}}{\pgfqpoint{1.690614in}{1.805590in}}{\pgfqpoint{1.682800in}{1.797776in}}%
\pgfpathcurveto{\pgfqpoint{1.674986in}{1.789963in}}{\pgfqpoint{1.670596in}{1.779364in}}{\pgfqpoint{1.670596in}{1.768314in}}%
\pgfpathcurveto{\pgfqpoint{1.670596in}{1.757263in}}{\pgfqpoint{1.674986in}{1.746664in}}{\pgfqpoint{1.682800in}{1.738851in}}%
\pgfpathcurveto{\pgfqpoint{1.690614in}{1.731037in}}{\pgfqpoint{1.701213in}{1.726647in}}{\pgfqpoint{1.712263in}{1.726647in}}%
\pgfpathlineto{\pgfqpoint{1.712263in}{1.726647in}}%
\pgfpathclose%
\pgfusepath{stroke}%
\end{pgfscope}%
\begin{pgfscope}%
\pgfpathrectangle{\pgfqpoint{0.494722in}{0.437222in}}{\pgfqpoint{6.275590in}{5.159444in}}%
\pgfusepath{clip}%
\pgfsetbuttcap%
\pgfsetroundjoin%
\pgfsetlinewidth{1.003750pt}%
\definecolor{currentstroke}{rgb}{0.827451,0.827451,0.827451}%
\pgfsetstrokecolor{currentstroke}%
\pgfsetstrokeopacity{0.800000}%
\pgfsetdash{}{0pt}%
\pgfpathmoveto{\pgfqpoint{1.444344in}{1.986059in}}%
\pgfpathcurveto{\pgfqpoint{1.455394in}{1.986059in}}{\pgfqpoint{1.465993in}{1.990450in}}{\pgfqpoint{1.473807in}{1.998263in}}%
\pgfpathcurveto{\pgfqpoint{1.481620in}{2.006077in}}{\pgfqpoint{1.486010in}{2.016676in}}{\pgfqpoint{1.486010in}{2.027726in}}%
\pgfpathcurveto{\pgfqpoint{1.486010in}{2.038776in}}{\pgfqpoint{1.481620in}{2.049375in}}{\pgfqpoint{1.473807in}{2.057189in}}%
\pgfpathcurveto{\pgfqpoint{1.465993in}{2.065002in}}{\pgfqpoint{1.455394in}{2.069393in}}{\pgfqpoint{1.444344in}{2.069393in}}%
\pgfpathcurveto{\pgfqpoint{1.433294in}{2.069393in}}{\pgfqpoint{1.422695in}{2.065002in}}{\pgfqpoint{1.414881in}{2.057189in}}%
\pgfpathcurveto{\pgfqpoint{1.407067in}{2.049375in}}{\pgfqpoint{1.402677in}{2.038776in}}{\pgfqpoint{1.402677in}{2.027726in}}%
\pgfpathcurveto{\pgfqpoint{1.402677in}{2.016676in}}{\pgfqpoint{1.407067in}{2.006077in}}{\pgfqpoint{1.414881in}{1.998263in}}%
\pgfpathcurveto{\pgfqpoint{1.422695in}{1.990450in}}{\pgfqpoint{1.433294in}{1.986059in}}{\pgfqpoint{1.444344in}{1.986059in}}%
\pgfpathlineto{\pgfqpoint{1.444344in}{1.986059in}}%
\pgfpathclose%
\pgfusepath{stroke}%
\end{pgfscope}%
\begin{pgfscope}%
\pgfpathrectangle{\pgfqpoint{0.494722in}{0.437222in}}{\pgfqpoint{6.275590in}{5.159444in}}%
\pgfusepath{clip}%
\pgfsetbuttcap%
\pgfsetroundjoin%
\pgfsetlinewidth{1.003750pt}%
\definecolor{currentstroke}{rgb}{0.827451,0.827451,0.827451}%
\pgfsetstrokecolor{currentstroke}%
\pgfsetstrokeopacity{0.800000}%
\pgfsetdash{}{0pt}%
\pgfpathmoveto{\pgfqpoint{1.161602in}{2.366586in}}%
\pgfpathcurveto{\pgfqpoint{1.172652in}{2.366586in}}{\pgfqpoint{1.183251in}{2.370976in}}{\pgfqpoint{1.191065in}{2.378790in}}%
\pgfpathcurveto{\pgfqpoint{1.198879in}{2.386603in}}{\pgfqpoint{1.203269in}{2.397202in}}{\pgfqpoint{1.203269in}{2.408252in}}%
\pgfpathcurveto{\pgfqpoint{1.203269in}{2.419303in}}{\pgfqpoint{1.198879in}{2.429902in}}{\pgfqpoint{1.191065in}{2.437715in}}%
\pgfpathcurveto{\pgfqpoint{1.183251in}{2.445529in}}{\pgfqpoint{1.172652in}{2.449919in}}{\pgfqpoint{1.161602in}{2.449919in}}%
\pgfpathcurveto{\pgfqpoint{1.150552in}{2.449919in}}{\pgfqpoint{1.139953in}{2.445529in}}{\pgfqpoint{1.132139in}{2.437715in}}%
\pgfpathcurveto{\pgfqpoint{1.124326in}{2.429902in}}{\pgfqpoint{1.119935in}{2.419303in}}{\pgfqpoint{1.119935in}{2.408252in}}%
\pgfpathcurveto{\pgfqpoint{1.119935in}{2.397202in}}{\pgfqpoint{1.124326in}{2.386603in}}{\pgfqpoint{1.132139in}{2.378790in}}%
\pgfpathcurveto{\pgfqpoint{1.139953in}{2.370976in}}{\pgfqpoint{1.150552in}{2.366586in}}{\pgfqpoint{1.161602in}{2.366586in}}%
\pgfpathlineto{\pgfqpoint{1.161602in}{2.366586in}}%
\pgfpathclose%
\pgfusepath{stroke}%
\end{pgfscope}%
\begin{pgfscope}%
\pgfpathrectangle{\pgfqpoint{0.494722in}{0.437222in}}{\pgfqpoint{6.275590in}{5.159444in}}%
\pgfusepath{clip}%
\pgfsetbuttcap%
\pgfsetroundjoin%
\pgfsetlinewidth{1.003750pt}%
\definecolor{currentstroke}{rgb}{0.827451,0.827451,0.827451}%
\pgfsetstrokecolor{currentstroke}%
\pgfsetstrokeopacity{0.800000}%
\pgfsetdash{}{0pt}%
\pgfpathmoveto{\pgfqpoint{2.785306in}{1.003395in}}%
\pgfpathcurveto{\pgfqpoint{2.796356in}{1.003395in}}{\pgfqpoint{2.806955in}{1.007786in}}{\pgfqpoint{2.814769in}{1.015599in}}%
\pgfpathcurveto{\pgfqpoint{2.822582in}{1.023413in}}{\pgfqpoint{2.826973in}{1.034012in}}{\pgfqpoint{2.826973in}{1.045062in}}%
\pgfpathcurveto{\pgfqpoint{2.826973in}{1.056112in}}{\pgfqpoint{2.822582in}{1.066711in}}{\pgfqpoint{2.814769in}{1.074525in}}%
\pgfpathcurveto{\pgfqpoint{2.806955in}{1.082338in}}{\pgfqpoint{2.796356in}{1.086729in}}{\pgfqpoint{2.785306in}{1.086729in}}%
\pgfpathcurveto{\pgfqpoint{2.774256in}{1.086729in}}{\pgfqpoint{2.763657in}{1.082338in}}{\pgfqpoint{2.755843in}{1.074525in}}%
\pgfpathcurveto{\pgfqpoint{2.748030in}{1.066711in}}{\pgfqpoint{2.743639in}{1.056112in}}{\pgfqpoint{2.743639in}{1.045062in}}%
\pgfpathcurveto{\pgfqpoint{2.743639in}{1.034012in}}{\pgfqpoint{2.748030in}{1.023413in}}{\pgfqpoint{2.755843in}{1.015599in}}%
\pgfpathcurveto{\pgfqpoint{2.763657in}{1.007786in}}{\pgfqpoint{2.774256in}{1.003395in}}{\pgfqpoint{2.785306in}{1.003395in}}%
\pgfpathlineto{\pgfqpoint{2.785306in}{1.003395in}}%
\pgfpathclose%
\pgfusepath{stroke}%
\end{pgfscope}%
\begin{pgfscope}%
\pgfpathrectangle{\pgfqpoint{0.494722in}{0.437222in}}{\pgfqpoint{6.275590in}{5.159444in}}%
\pgfusepath{clip}%
\pgfsetbuttcap%
\pgfsetroundjoin%
\pgfsetlinewidth{1.003750pt}%
\definecolor{currentstroke}{rgb}{0.827451,0.827451,0.827451}%
\pgfsetstrokecolor{currentstroke}%
\pgfsetstrokeopacity{0.800000}%
\pgfsetdash{}{0pt}%
\pgfpathmoveto{\pgfqpoint{2.081578in}{1.430058in}}%
\pgfpathcurveto{\pgfqpoint{2.092628in}{1.430058in}}{\pgfqpoint{2.103227in}{1.434449in}}{\pgfqpoint{2.111040in}{1.442262in}}%
\pgfpathcurveto{\pgfqpoint{2.118854in}{1.450076in}}{\pgfqpoint{2.123244in}{1.460675in}}{\pgfqpoint{2.123244in}{1.471725in}}%
\pgfpathcurveto{\pgfqpoint{2.123244in}{1.482775in}}{\pgfqpoint{2.118854in}{1.493374in}}{\pgfqpoint{2.111040in}{1.501188in}}%
\pgfpathcurveto{\pgfqpoint{2.103227in}{1.509001in}}{\pgfqpoint{2.092628in}{1.513392in}}{\pgfqpoint{2.081578in}{1.513392in}}%
\pgfpathcurveto{\pgfqpoint{2.070527in}{1.513392in}}{\pgfqpoint{2.059928in}{1.509001in}}{\pgfqpoint{2.052115in}{1.501188in}}%
\pgfpathcurveto{\pgfqpoint{2.044301in}{1.493374in}}{\pgfqpoint{2.039911in}{1.482775in}}{\pgfqpoint{2.039911in}{1.471725in}}%
\pgfpathcurveto{\pgfqpoint{2.039911in}{1.460675in}}{\pgfqpoint{2.044301in}{1.450076in}}{\pgfqpoint{2.052115in}{1.442262in}}%
\pgfpathcurveto{\pgfqpoint{2.059928in}{1.434449in}}{\pgfqpoint{2.070527in}{1.430058in}}{\pgfqpoint{2.081578in}{1.430058in}}%
\pgfpathlineto{\pgfqpoint{2.081578in}{1.430058in}}%
\pgfpathclose%
\pgfusepath{stroke}%
\end{pgfscope}%
\begin{pgfscope}%
\pgfpathrectangle{\pgfqpoint{0.494722in}{0.437222in}}{\pgfqpoint{6.275590in}{5.159444in}}%
\pgfusepath{clip}%
\pgfsetbuttcap%
\pgfsetroundjoin%
\pgfsetlinewidth{1.003750pt}%
\definecolor{currentstroke}{rgb}{0.827451,0.827451,0.827451}%
\pgfsetstrokecolor{currentstroke}%
\pgfsetstrokeopacity{0.800000}%
\pgfsetdash{}{0pt}%
\pgfpathmoveto{\pgfqpoint{3.133116in}{0.840916in}}%
\pgfpathcurveto{\pgfqpoint{3.144166in}{0.840916in}}{\pgfqpoint{3.154765in}{0.845307in}}{\pgfqpoint{3.162579in}{0.853120in}}%
\pgfpathcurveto{\pgfqpoint{3.170393in}{0.860934in}}{\pgfqpoint{3.174783in}{0.871533in}}{\pgfqpoint{3.174783in}{0.882583in}}%
\pgfpathcurveto{\pgfqpoint{3.174783in}{0.893633in}}{\pgfqpoint{3.170393in}{0.904232in}}{\pgfqpoint{3.162579in}{0.912046in}}%
\pgfpathcurveto{\pgfqpoint{3.154765in}{0.919859in}}{\pgfqpoint{3.144166in}{0.924250in}}{\pgfqpoint{3.133116in}{0.924250in}}%
\pgfpathcurveto{\pgfqpoint{3.122066in}{0.924250in}}{\pgfqpoint{3.111467in}{0.919859in}}{\pgfqpoint{3.103653in}{0.912046in}}%
\pgfpathcurveto{\pgfqpoint{3.095840in}{0.904232in}}{\pgfqpoint{3.091450in}{0.893633in}}{\pgfqpoint{3.091450in}{0.882583in}}%
\pgfpathcurveto{\pgfqpoint{3.091450in}{0.871533in}}{\pgfqpoint{3.095840in}{0.860934in}}{\pgfqpoint{3.103653in}{0.853120in}}%
\pgfpathcurveto{\pgfqpoint{3.111467in}{0.845307in}}{\pgfqpoint{3.122066in}{0.840916in}}{\pgfqpoint{3.133116in}{0.840916in}}%
\pgfpathlineto{\pgfqpoint{3.133116in}{0.840916in}}%
\pgfpathclose%
\pgfusepath{stroke}%
\end{pgfscope}%
\begin{pgfscope}%
\pgfpathrectangle{\pgfqpoint{0.494722in}{0.437222in}}{\pgfqpoint{6.275590in}{5.159444in}}%
\pgfusepath{clip}%
\pgfsetbuttcap%
\pgfsetroundjoin%
\pgfsetlinewidth{1.003750pt}%
\definecolor{currentstroke}{rgb}{0.827451,0.827451,0.827451}%
\pgfsetstrokecolor{currentstroke}%
\pgfsetstrokeopacity{0.800000}%
\pgfsetdash{}{0pt}%
\pgfpathmoveto{\pgfqpoint{0.501219in}{4.373826in}}%
\pgfpathcurveto{\pgfqpoint{0.512269in}{4.373826in}}{\pgfqpoint{0.522868in}{4.378216in}}{\pgfqpoint{0.530682in}{4.386030in}}%
\pgfpathcurveto{\pgfqpoint{0.538495in}{4.393844in}}{\pgfqpoint{0.542886in}{4.404443in}}{\pgfqpoint{0.542886in}{4.415493in}}%
\pgfpathcurveto{\pgfqpoint{0.542886in}{4.426543in}}{\pgfqpoint{0.538495in}{4.437142in}}{\pgfqpoint{0.530682in}{4.444955in}}%
\pgfpathcurveto{\pgfqpoint{0.522868in}{4.452769in}}{\pgfqpoint{0.512269in}{4.457159in}}{\pgfqpoint{0.501219in}{4.457159in}}%
\pgfpathcurveto{\pgfqpoint{0.490169in}{4.457159in}}{\pgfqpoint{0.479570in}{4.452769in}}{\pgfqpoint{0.471756in}{4.444955in}}%
\pgfpathcurveto{\pgfqpoint{0.463943in}{4.437142in}}{\pgfqpoint{0.459552in}{4.426543in}}{\pgfqpoint{0.459552in}{4.415493in}}%
\pgfpathcurveto{\pgfqpoint{0.459552in}{4.404443in}}{\pgfqpoint{0.463943in}{4.393844in}}{\pgfqpoint{0.471756in}{4.386030in}}%
\pgfpathcurveto{\pgfqpoint{0.479570in}{4.378216in}}{\pgfqpoint{0.490169in}{4.373826in}}{\pgfqpoint{0.501219in}{4.373826in}}%
\pgfpathlineto{\pgfqpoint{0.501219in}{4.373826in}}%
\pgfpathclose%
\pgfusepath{stroke}%
\end{pgfscope}%
\begin{pgfscope}%
\pgfpathrectangle{\pgfqpoint{0.494722in}{0.437222in}}{\pgfqpoint{6.275590in}{5.159444in}}%
\pgfusepath{clip}%
\pgfsetbuttcap%
\pgfsetroundjoin%
\pgfsetlinewidth{1.003750pt}%
\definecolor{currentstroke}{rgb}{0.827451,0.827451,0.827451}%
\pgfsetstrokecolor{currentstroke}%
\pgfsetstrokeopacity{0.800000}%
\pgfsetdash{}{0pt}%
\pgfpathmoveto{\pgfqpoint{0.495484in}{4.544725in}}%
\pgfpathcurveto{\pgfqpoint{0.506534in}{4.544725in}}{\pgfqpoint{0.517133in}{4.549116in}}{\pgfqpoint{0.524947in}{4.556929in}}%
\pgfpathcurveto{\pgfqpoint{0.532761in}{4.564743in}}{\pgfqpoint{0.537151in}{4.575342in}}{\pgfqpoint{0.537151in}{4.586392in}}%
\pgfpathcurveto{\pgfqpoint{0.537151in}{4.597442in}}{\pgfqpoint{0.532761in}{4.608041in}}{\pgfqpoint{0.524947in}{4.615855in}}%
\pgfpathcurveto{\pgfqpoint{0.517133in}{4.623668in}}{\pgfqpoint{0.506534in}{4.628059in}}{\pgfqpoint{0.495484in}{4.628059in}}%
\pgfpathcurveto{\pgfqpoint{0.484434in}{4.628059in}}{\pgfqpoint{0.473835in}{4.623668in}}{\pgfqpoint{0.466021in}{4.615855in}}%
\pgfpathcurveto{\pgfqpoint{0.458208in}{4.608041in}}{\pgfqpoint{0.453817in}{4.597442in}}{\pgfqpoint{0.453817in}{4.586392in}}%
\pgfpathcurveto{\pgfqpoint{0.453817in}{4.575342in}}{\pgfqpoint{0.458208in}{4.564743in}}{\pgfqpoint{0.466021in}{4.556929in}}%
\pgfpathcurveto{\pgfqpoint{0.473835in}{4.549116in}}{\pgfqpoint{0.484434in}{4.544725in}}{\pgfqpoint{0.495484in}{4.544725in}}%
\pgfpathlineto{\pgfqpoint{0.495484in}{4.544725in}}%
\pgfpathclose%
\pgfusepath{stroke}%
\end{pgfscope}%
\begin{pgfscope}%
\pgfpathrectangle{\pgfqpoint{0.494722in}{0.437222in}}{\pgfqpoint{6.275590in}{5.159444in}}%
\pgfusepath{clip}%
\pgfsetbuttcap%
\pgfsetroundjoin%
\pgfsetlinewidth{1.003750pt}%
\definecolor{currentstroke}{rgb}{0.827451,0.827451,0.827451}%
\pgfsetstrokecolor{currentstroke}%
\pgfsetstrokeopacity{0.800000}%
\pgfsetdash{}{0pt}%
\pgfpathmoveto{\pgfqpoint{3.372979in}{0.744763in}}%
\pgfpathcurveto{\pgfqpoint{3.384029in}{0.744763in}}{\pgfqpoint{3.394628in}{0.749153in}}{\pgfqpoint{3.402442in}{0.756967in}}%
\pgfpathcurveto{\pgfqpoint{3.410255in}{0.764781in}}{\pgfqpoint{3.414646in}{0.775380in}}{\pgfqpoint{3.414646in}{0.786430in}}%
\pgfpathcurveto{\pgfqpoint{3.414646in}{0.797480in}}{\pgfqpoint{3.410255in}{0.808079in}}{\pgfqpoint{3.402442in}{0.815893in}}%
\pgfpathcurveto{\pgfqpoint{3.394628in}{0.823706in}}{\pgfqpoint{3.384029in}{0.828097in}}{\pgfqpoint{3.372979in}{0.828097in}}%
\pgfpathcurveto{\pgfqpoint{3.361929in}{0.828097in}}{\pgfqpoint{3.351330in}{0.823706in}}{\pgfqpoint{3.343516in}{0.815893in}}%
\pgfpathcurveto{\pgfqpoint{3.335703in}{0.808079in}}{\pgfqpoint{3.331312in}{0.797480in}}{\pgfqpoint{3.331312in}{0.786430in}}%
\pgfpathcurveto{\pgfqpoint{3.331312in}{0.775380in}}{\pgfqpoint{3.335703in}{0.764781in}}{\pgfqpoint{3.343516in}{0.756967in}}%
\pgfpathcurveto{\pgfqpoint{3.351330in}{0.749153in}}{\pgfqpoint{3.361929in}{0.744763in}}{\pgfqpoint{3.372979in}{0.744763in}}%
\pgfpathlineto{\pgfqpoint{3.372979in}{0.744763in}}%
\pgfpathclose%
\pgfusepath{stroke}%
\end{pgfscope}%
\begin{pgfscope}%
\pgfpathrectangle{\pgfqpoint{0.494722in}{0.437222in}}{\pgfqpoint{6.275590in}{5.159444in}}%
\pgfusepath{clip}%
\pgfsetbuttcap%
\pgfsetroundjoin%
\pgfsetlinewidth{1.003750pt}%
\definecolor{currentstroke}{rgb}{0.827451,0.827451,0.827451}%
\pgfsetstrokecolor{currentstroke}%
\pgfsetstrokeopacity{0.800000}%
\pgfsetdash{}{0pt}%
\pgfpathmoveto{\pgfqpoint{2.490073in}{1.171385in}}%
\pgfpathcurveto{\pgfqpoint{2.501123in}{1.171385in}}{\pgfqpoint{2.511722in}{1.175776in}}{\pgfqpoint{2.519535in}{1.183589in}}%
\pgfpathcurveto{\pgfqpoint{2.527349in}{1.191403in}}{\pgfqpoint{2.531739in}{1.202002in}}{\pgfqpoint{2.531739in}{1.213052in}}%
\pgfpathcurveto{\pgfqpoint{2.531739in}{1.224102in}}{\pgfqpoint{2.527349in}{1.234701in}}{\pgfqpoint{2.519535in}{1.242515in}}%
\pgfpathcurveto{\pgfqpoint{2.511722in}{1.250328in}}{\pgfqpoint{2.501123in}{1.254719in}}{\pgfqpoint{2.490073in}{1.254719in}}%
\pgfpathcurveto{\pgfqpoint{2.479023in}{1.254719in}}{\pgfqpoint{2.468423in}{1.250328in}}{\pgfqpoint{2.460610in}{1.242515in}}%
\pgfpathcurveto{\pgfqpoint{2.452796in}{1.234701in}}{\pgfqpoint{2.448406in}{1.224102in}}{\pgfqpoint{2.448406in}{1.213052in}}%
\pgfpathcurveto{\pgfqpoint{2.448406in}{1.202002in}}{\pgfqpoint{2.452796in}{1.191403in}}{\pgfqpoint{2.460610in}{1.183589in}}%
\pgfpathcurveto{\pgfqpoint{2.468423in}{1.175776in}}{\pgfqpoint{2.479023in}{1.171385in}}{\pgfqpoint{2.490073in}{1.171385in}}%
\pgfpathlineto{\pgfqpoint{2.490073in}{1.171385in}}%
\pgfpathclose%
\pgfusepath{stroke}%
\end{pgfscope}%
\begin{pgfscope}%
\pgfpathrectangle{\pgfqpoint{0.494722in}{0.437222in}}{\pgfqpoint{6.275590in}{5.159444in}}%
\pgfusepath{clip}%
\pgfsetbuttcap%
\pgfsetroundjoin%
\pgfsetlinewidth{1.003750pt}%
\definecolor{currentstroke}{rgb}{0.827451,0.827451,0.827451}%
\pgfsetstrokecolor{currentstroke}%
\pgfsetstrokeopacity{0.800000}%
\pgfsetdash{}{0pt}%
\pgfpathmoveto{\pgfqpoint{0.700772in}{3.267894in}}%
\pgfpathcurveto{\pgfqpoint{0.711822in}{3.267894in}}{\pgfqpoint{0.722421in}{3.272284in}}{\pgfqpoint{0.730235in}{3.280098in}}%
\pgfpathcurveto{\pgfqpoint{0.738048in}{3.287911in}}{\pgfqpoint{0.742438in}{3.298510in}}{\pgfqpoint{0.742438in}{3.309561in}}%
\pgfpathcurveto{\pgfqpoint{0.742438in}{3.320611in}}{\pgfqpoint{0.738048in}{3.331210in}}{\pgfqpoint{0.730235in}{3.339023in}}%
\pgfpathcurveto{\pgfqpoint{0.722421in}{3.346837in}}{\pgfqpoint{0.711822in}{3.351227in}}{\pgfqpoint{0.700772in}{3.351227in}}%
\pgfpathcurveto{\pgfqpoint{0.689722in}{3.351227in}}{\pgfqpoint{0.679123in}{3.346837in}}{\pgfqpoint{0.671309in}{3.339023in}}%
\pgfpathcurveto{\pgfqpoint{0.663495in}{3.331210in}}{\pgfqpoint{0.659105in}{3.320611in}}{\pgfqpoint{0.659105in}{3.309561in}}%
\pgfpathcurveto{\pgfqpoint{0.659105in}{3.298510in}}{\pgfqpoint{0.663495in}{3.287911in}}{\pgfqpoint{0.671309in}{3.280098in}}%
\pgfpathcurveto{\pgfqpoint{0.679123in}{3.272284in}}{\pgfqpoint{0.689722in}{3.267894in}}{\pgfqpoint{0.700772in}{3.267894in}}%
\pgfpathlineto{\pgfqpoint{0.700772in}{3.267894in}}%
\pgfpathclose%
\pgfusepath{stroke}%
\end{pgfscope}%
\begin{pgfscope}%
\pgfpathrectangle{\pgfqpoint{0.494722in}{0.437222in}}{\pgfqpoint{6.275590in}{5.159444in}}%
\pgfusepath{clip}%
\pgfsetbuttcap%
\pgfsetroundjoin%
\pgfsetlinewidth{1.003750pt}%
\definecolor{currentstroke}{rgb}{0.827451,0.827451,0.827451}%
\pgfsetstrokecolor{currentstroke}%
\pgfsetstrokeopacity{0.800000}%
\pgfsetdash{}{0pt}%
\pgfpathmoveto{\pgfqpoint{2.658320in}{1.113694in}}%
\pgfpathcurveto{\pgfqpoint{2.669370in}{1.113694in}}{\pgfqpoint{2.679969in}{1.118084in}}{\pgfqpoint{2.687783in}{1.125898in}}%
\pgfpathcurveto{\pgfqpoint{2.695596in}{1.133711in}}{\pgfqpoint{2.699987in}{1.144310in}}{\pgfqpoint{2.699987in}{1.155361in}}%
\pgfpathcurveto{\pgfqpoint{2.699987in}{1.166411in}}{\pgfqpoint{2.695596in}{1.177010in}}{\pgfqpoint{2.687783in}{1.184823in}}%
\pgfpathcurveto{\pgfqpoint{2.679969in}{1.192637in}}{\pgfqpoint{2.669370in}{1.197027in}}{\pgfqpoint{2.658320in}{1.197027in}}%
\pgfpathcurveto{\pgfqpoint{2.647270in}{1.197027in}}{\pgfqpoint{2.636671in}{1.192637in}}{\pgfqpoint{2.628857in}{1.184823in}}%
\pgfpathcurveto{\pgfqpoint{2.621044in}{1.177010in}}{\pgfqpoint{2.616653in}{1.166411in}}{\pgfqpoint{2.616653in}{1.155361in}}%
\pgfpathcurveto{\pgfqpoint{2.616653in}{1.144310in}}{\pgfqpoint{2.621044in}{1.133711in}}{\pgfqpoint{2.628857in}{1.125898in}}%
\pgfpathcurveto{\pgfqpoint{2.636671in}{1.118084in}}{\pgfqpoint{2.647270in}{1.113694in}}{\pgfqpoint{2.658320in}{1.113694in}}%
\pgfpathlineto{\pgfqpoint{2.658320in}{1.113694in}}%
\pgfpathclose%
\pgfusepath{stroke}%
\end{pgfscope}%
\begin{pgfscope}%
\pgfpathrectangle{\pgfqpoint{0.494722in}{0.437222in}}{\pgfqpoint{6.275590in}{5.159444in}}%
\pgfusepath{clip}%
\pgfsetbuttcap%
\pgfsetroundjoin%
\pgfsetlinewidth{1.003750pt}%
\definecolor{currentstroke}{rgb}{0.827451,0.827451,0.827451}%
\pgfsetstrokecolor{currentstroke}%
\pgfsetstrokeopacity{0.800000}%
\pgfsetdash{}{0pt}%
\pgfpathmoveto{\pgfqpoint{0.680103in}{3.350578in}}%
\pgfpathcurveto{\pgfqpoint{0.691153in}{3.350578in}}{\pgfqpoint{0.701753in}{3.354968in}}{\pgfqpoint{0.709566in}{3.362782in}}%
\pgfpathcurveto{\pgfqpoint{0.717380in}{3.370596in}}{\pgfqpoint{0.721770in}{3.381195in}}{\pgfqpoint{0.721770in}{3.392245in}}%
\pgfpathcurveto{\pgfqpoint{0.721770in}{3.403295in}}{\pgfqpoint{0.717380in}{3.413894in}}{\pgfqpoint{0.709566in}{3.421707in}}%
\pgfpathcurveto{\pgfqpoint{0.701753in}{3.429521in}}{\pgfqpoint{0.691153in}{3.433911in}}{\pgfqpoint{0.680103in}{3.433911in}}%
\pgfpathcurveto{\pgfqpoint{0.669053in}{3.433911in}}{\pgfqpoint{0.658454in}{3.429521in}}{\pgfqpoint{0.650641in}{3.421707in}}%
\pgfpathcurveto{\pgfqpoint{0.642827in}{3.413894in}}{\pgfqpoint{0.638437in}{3.403295in}}{\pgfqpoint{0.638437in}{3.392245in}}%
\pgfpathcurveto{\pgfqpoint{0.638437in}{3.381195in}}{\pgfqpoint{0.642827in}{3.370596in}}{\pgfqpoint{0.650641in}{3.362782in}}%
\pgfpathcurveto{\pgfqpoint{0.658454in}{3.354968in}}{\pgfqpoint{0.669053in}{3.350578in}}{\pgfqpoint{0.680103in}{3.350578in}}%
\pgfpathlineto{\pgfqpoint{0.680103in}{3.350578in}}%
\pgfpathclose%
\pgfusepath{stroke}%
\end{pgfscope}%
\begin{pgfscope}%
\pgfpathrectangle{\pgfqpoint{0.494722in}{0.437222in}}{\pgfqpoint{6.275590in}{5.159444in}}%
\pgfusepath{clip}%
\pgfsetbuttcap%
\pgfsetroundjoin%
\pgfsetlinewidth{1.003750pt}%
\definecolor{currentstroke}{rgb}{0.827451,0.827451,0.827451}%
\pgfsetstrokecolor{currentstroke}%
\pgfsetstrokeopacity{0.800000}%
\pgfsetdash{}{0pt}%
\pgfpathmoveto{\pgfqpoint{5.183443in}{0.422556in}}%
\pgfpathcurveto{\pgfqpoint{5.194493in}{0.422556in}}{\pgfqpoint{5.205092in}{0.426946in}}{\pgfqpoint{5.212906in}{0.434760in}}%
\pgfpathcurveto{\pgfqpoint{5.220719in}{0.442573in}}{\pgfqpoint{5.225109in}{0.453172in}}{\pgfqpoint{5.225109in}{0.464222in}}%
\pgfpathcurveto{\pgfqpoint{5.225109in}{0.475273in}}{\pgfqpoint{5.220719in}{0.485872in}}{\pgfqpoint{5.212906in}{0.493685in}}%
\pgfpathcurveto{\pgfqpoint{5.205092in}{0.501499in}}{\pgfqpoint{5.194493in}{0.505889in}}{\pgfqpoint{5.183443in}{0.505889in}}%
\pgfpathcurveto{\pgfqpoint{5.172393in}{0.505889in}}{\pgfqpoint{5.161794in}{0.501499in}}{\pgfqpoint{5.153980in}{0.493685in}}%
\pgfpathcurveto{\pgfqpoint{5.146166in}{0.485872in}}{\pgfqpoint{5.141776in}{0.475273in}}{\pgfqpoint{5.141776in}{0.464222in}}%
\pgfpathcurveto{\pgfqpoint{5.141776in}{0.453172in}}{\pgfqpoint{5.146166in}{0.442573in}}{\pgfqpoint{5.153980in}{0.434760in}}%
\pgfpathcurveto{\pgfqpoint{5.161794in}{0.426946in}}{\pgfqpoint{5.172393in}{0.422556in}}{\pgfqpoint{5.183443in}{0.422556in}}%
\pgfusepath{stroke}%
\end{pgfscope}%
\begin{pgfscope}%
\pgfpathrectangle{\pgfqpoint{0.494722in}{0.437222in}}{\pgfqpoint{6.275590in}{5.159444in}}%
\pgfusepath{clip}%
\pgfsetbuttcap%
\pgfsetroundjoin%
\pgfsetlinewidth{1.003750pt}%
\definecolor{currentstroke}{rgb}{0.827451,0.827451,0.827451}%
\pgfsetstrokecolor{currentstroke}%
\pgfsetstrokeopacity{0.800000}%
\pgfsetdash{}{0pt}%
\pgfpathmoveto{\pgfqpoint{1.381781in}{2.056420in}}%
\pgfpathcurveto{\pgfqpoint{1.392831in}{2.056420in}}{\pgfqpoint{1.403430in}{2.060810in}}{\pgfqpoint{1.411244in}{2.068624in}}%
\pgfpathcurveto{\pgfqpoint{1.419057in}{2.076437in}}{\pgfqpoint{1.423448in}{2.087036in}}{\pgfqpoint{1.423448in}{2.098087in}}%
\pgfpathcurveto{\pgfqpoint{1.423448in}{2.109137in}}{\pgfqpoint{1.419057in}{2.119736in}}{\pgfqpoint{1.411244in}{2.127549in}}%
\pgfpathcurveto{\pgfqpoint{1.403430in}{2.135363in}}{\pgfqpoint{1.392831in}{2.139753in}}{\pgfqpoint{1.381781in}{2.139753in}}%
\pgfpathcurveto{\pgfqpoint{1.370731in}{2.139753in}}{\pgfqpoint{1.360132in}{2.135363in}}{\pgfqpoint{1.352318in}{2.127549in}}%
\pgfpathcurveto{\pgfqpoint{1.344505in}{2.119736in}}{\pgfqpoint{1.340114in}{2.109137in}}{\pgfqpoint{1.340114in}{2.098087in}}%
\pgfpathcurveto{\pgfqpoint{1.340114in}{2.087036in}}{\pgfqpoint{1.344505in}{2.076437in}}{\pgfqpoint{1.352318in}{2.068624in}}%
\pgfpathcurveto{\pgfqpoint{1.360132in}{2.060810in}}{\pgfqpoint{1.370731in}{2.056420in}}{\pgfqpoint{1.381781in}{2.056420in}}%
\pgfpathlineto{\pgfqpoint{1.381781in}{2.056420in}}%
\pgfpathclose%
\pgfusepath{stroke}%
\end{pgfscope}%
\begin{pgfscope}%
\pgfpathrectangle{\pgfqpoint{0.494722in}{0.437222in}}{\pgfqpoint{6.275590in}{5.159444in}}%
\pgfusepath{clip}%
\pgfsetbuttcap%
\pgfsetroundjoin%
\pgfsetlinewidth{1.003750pt}%
\definecolor{currentstroke}{rgb}{0.827451,0.827451,0.827451}%
\pgfsetstrokecolor{currentstroke}%
\pgfsetstrokeopacity{0.800000}%
\pgfsetdash{}{0pt}%
\pgfpathmoveto{\pgfqpoint{2.308193in}{1.269247in}}%
\pgfpathcurveto{\pgfqpoint{2.319243in}{1.269247in}}{\pgfqpoint{2.329842in}{1.273637in}}{\pgfqpoint{2.337656in}{1.281451in}}%
\pgfpathcurveto{\pgfqpoint{2.345469in}{1.289264in}}{\pgfqpoint{2.349860in}{1.299864in}}{\pgfqpoint{2.349860in}{1.310914in}}%
\pgfpathcurveto{\pgfqpoint{2.349860in}{1.321964in}}{\pgfqpoint{2.345469in}{1.332563in}}{\pgfqpoint{2.337656in}{1.340376in}}%
\pgfpathcurveto{\pgfqpoint{2.329842in}{1.348190in}}{\pgfqpoint{2.319243in}{1.352580in}}{\pgfqpoint{2.308193in}{1.352580in}}%
\pgfpathcurveto{\pgfqpoint{2.297143in}{1.352580in}}{\pgfqpoint{2.286544in}{1.348190in}}{\pgfqpoint{2.278730in}{1.340376in}}%
\pgfpathcurveto{\pgfqpoint{2.270917in}{1.332563in}}{\pgfqpoint{2.266526in}{1.321964in}}{\pgfqpoint{2.266526in}{1.310914in}}%
\pgfpathcurveto{\pgfqpoint{2.266526in}{1.299864in}}{\pgfqpoint{2.270917in}{1.289264in}}{\pgfqpoint{2.278730in}{1.281451in}}%
\pgfpathcurveto{\pgfqpoint{2.286544in}{1.273637in}}{\pgfqpoint{2.297143in}{1.269247in}}{\pgfqpoint{2.308193in}{1.269247in}}%
\pgfpathlineto{\pgfqpoint{2.308193in}{1.269247in}}%
\pgfpathclose%
\pgfusepath{stroke}%
\end{pgfscope}%
\begin{pgfscope}%
\pgfpathrectangle{\pgfqpoint{0.494722in}{0.437222in}}{\pgfqpoint{6.275590in}{5.159444in}}%
\pgfusepath{clip}%
\pgfsetbuttcap%
\pgfsetroundjoin%
\pgfsetlinewidth{1.003750pt}%
\definecolor{currentstroke}{rgb}{0.827451,0.827451,0.827451}%
\pgfsetstrokecolor{currentstroke}%
\pgfsetstrokeopacity{0.800000}%
\pgfsetdash{}{0pt}%
\pgfpathmoveto{\pgfqpoint{5.373868in}{0.408915in}}%
\pgfpathcurveto{\pgfqpoint{5.384918in}{0.408915in}}{\pgfqpoint{5.395517in}{0.413306in}}{\pgfqpoint{5.403330in}{0.421119in}}%
\pgfpathcurveto{\pgfqpoint{5.411144in}{0.428933in}}{\pgfqpoint{5.415534in}{0.439532in}}{\pgfqpoint{5.415534in}{0.450582in}}%
\pgfpathcurveto{\pgfqpoint{5.415534in}{0.461632in}}{\pgfqpoint{5.411144in}{0.472231in}}{\pgfqpoint{5.403330in}{0.480045in}}%
\pgfpathcurveto{\pgfqpoint{5.395517in}{0.487858in}}{\pgfqpoint{5.384918in}{0.492249in}}{\pgfqpoint{5.373868in}{0.492249in}}%
\pgfpathcurveto{\pgfqpoint{5.362817in}{0.492249in}}{\pgfqpoint{5.352218in}{0.487858in}}{\pgfqpoint{5.344405in}{0.480045in}}%
\pgfpathcurveto{\pgfqpoint{5.336591in}{0.472231in}}{\pgfqpoint{5.332201in}{0.461632in}}{\pgfqpoint{5.332201in}{0.450582in}}%
\pgfpathcurveto{\pgfqpoint{5.332201in}{0.439532in}}{\pgfqpoint{5.336591in}{0.428933in}}{\pgfqpoint{5.344405in}{0.421119in}}%
\pgfpathcurveto{\pgfqpoint{5.352218in}{0.413306in}}{\pgfqpoint{5.362817in}{0.408915in}}{\pgfqpoint{5.373868in}{0.408915in}}%
\pgfusepath{stroke}%
\end{pgfscope}%
\begin{pgfscope}%
\pgfpathrectangle{\pgfqpoint{0.494722in}{0.437222in}}{\pgfqpoint{6.275590in}{5.159444in}}%
\pgfusepath{clip}%
\pgfsetbuttcap%
\pgfsetroundjoin%
\pgfsetlinewidth{1.003750pt}%
\definecolor{currentstroke}{rgb}{0.827451,0.827451,0.827451}%
\pgfsetstrokecolor{currentstroke}%
\pgfsetstrokeopacity{0.800000}%
\pgfsetdash{}{0pt}%
\pgfpathmoveto{\pgfqpoint{2.552339in}{1.147505in}}%
\pgfpathcurveto{\pgfqpoint{2.563390in}{1.147505in}}{\pgfqpoint{2.573989in}{1.151895in}}{\pgfqpoint{2.581802in}{1.159709in}}%
\pgfpathcurveto{\pgfqpoint{2.589616in}{1.167522in}}{\pgfqpoint{2.594006in}{1.178121in}}{\pgfqpoint{2.594006in}{1.189172in}}%
\pgfpathcurveto{\pgfqpoint{2.594006in}{1.200222in}}{\pgfqpoint{2.589616in}{1.210821in}}{\pgfqpoint{2.581802in}{1.218634in}}%
\pgfpathcurveto{\pgfqpoint{2.573989in}{1.226448in}}{\pgfqpoint{2.563390in}{1.230838in}}{\pgfqpoint{2.552339in}{1.230838in}}%
\pgfpathcurveto{\pgfqpoint{2.541289in}{1.230838in}}{\pgfqpoint{2.530690in}{1.226448in}}{\pgfqpoint{2.522877in}{1.218634in}}%
\pgfpathcurveto{\pgfqpoint{2.515063in}{1.210821in}}{\pgfqpoint{2.510673in}{1.200222in}}{\pgfqpoint{2.510673in}{1.189172in}}%
\pgfpathcurveto{\pgfqpoint{2.510673in}{1.178121in}}{\pgfqpoint{2.515063in}{1.167522in}}{\pgfqpoint{2.522877in}{1.159709in}}%
\pgfpathcurveto{\pgfqpoint{2.530690in}{1.151895in}}{\pgfqpoint{2.541289in}{1.147505in}}{\pgfqpoint{2.552339in}{1.147505in}}%
\pgfpathlineto{\pgfqpoint{2.552339in}{1.147505in}}%
\pgfpathclose%
\pgfusepath{stroke}%
\end{pgfscope}%
\begin{pgfscope}%
\pgfpathrectangle{\pgfqpoint{0.494722in}{0.437222in}}{\pgfqpoint{6.275590in}{5.159444in}}%
\pgfusepath{clip}%
\pgfsetbuttcap%
\pgfsetroundjoin%
\pgfsetlinewidth{1.003750pt}%
\definecolor{currentstroke}{rgb}{0.827451,0.827451,0.827451}%
\pgfsetstrokecolor{currentstroke}%
\pgfsetstrokeopacity{0.800000}%
\pgfsetdash{}{0pt}%
\pgfpathmoveto{\pgfqpoint{0.682684in}{3.324054in}}%
\pgfpathcurveto{\pgfqpoint{0.693734in}{3.324054in}}{\pgfqpoint{0.704333in}{3.328444in}}{\pgfqpoint{0.712146in}{3.336257in}}%
\pgfpathcurveto{\pgfqpoint{0.719960in}{3.344071in}}{\pgfqpoint{0.724350in}{3.354670in}}{\pgfqpoint{0.724350in}{3.365720in}}%
\pgfpathcurveto{\pgfqpoint{0.724350in}{3.376770in}}{\pgfqpoint{0.719960in}{3.387369in}}{\pgfqpoint{0.712146in}{3.395183in}}%
\pgfpathcurveto{\pgfqpoint{0.704333in}{3.402997in}}{\pgfqpoint{0.693734in}{3.407387in}}{\pgfqpoint{0.682684in}{3.407387in}}%
\pgfpathcurveto{\pgfqpoint{0.671634in}{3.407387in}}{\pgfqpoint{0.661035in}{3.402997in}}{\pgfqpoint{0.653221in}{3.395183in}}%
\pgfpathcurveto{\pgfqpoint{0.645407in}{3.387369in}}{\pgfqpoint{0.641017in}{3.376770in}}{\pgfqpoint{0.641017in}{3.365720in}}%
\pgfpathcurveto{\pgfqpoint{0.641017in}{3.354670in}}{\pgfqpoint{0.645407in}{3.344071in}}{\pgfqpoint{0.653221in}{3.336257in}}%
\pgfpathcurveto{\pgfqpoint{0.661035in}{3.328444in}}{\pgfqpoint{0.671634in}{3.324054in}}{\pgfqpoint{0.682684in}{3.324054in}}%
\pgfpathlineto{\pgfqpoint{0.682684in}{3.324054in}}%
\pgfpathclose%
\pgfusepath{stroke}%
\end{pgfscope}%
\begin{pgfscope}%
\pgfpathrectangle{\pgfqpoint{0.494722in}{0.437222in}}{\pgfqpoint{6.275590in}{5.159444in}}%
\pgfusepath{clip}%
\pgfsetbuttcap%
\pgfsetroundjoin%
\pgfsetlinewidth{1.003750pt}%
\definecolor{currentstroke}{rgb}{0.827451,0.827451,0.827451}%
\pgfsetstrokecolor{currentstroke}%
\pgfsetstrokeopacity{0.800000}%
\pgfsetdash{}{0pt}%
\pgfpathmoveto{\pgfqpoint{5.564107in}{0.399714in}}%
\pgfpathcurveto{\pgfqpoint{5.575157in}{0.399714in}}{\pgfqpoint{5.585756in}{0.404105in}}{\pgfqpoint{5.593570in}{0.411918in}}%
\pgfpathcurveto{\pgfqpoint{5.601383in}{0.419732in}}{\pgfqpoint{5.605774in}{0.430331in}}{\pgfqpoint{5.605774in}{0.441381in}}%
\pgfpathcurveto{\pgfqpoint{5.605774in}{0.452431in}}{\pgfqpoint{5.601383in}{0.463030in}}{\pgfqpoint{5.593570in}{0.470844in}}%
\pgfpathcurveto{\pgfqpoint{5.585756in}{0.478658in}}{\pgfqpoint{5.575157in}{0.483048in}}{\pgfqpoint{5.564107in}{0.483048in}}%
\pgfpathcurveto{\pgfqpoint{5.553057in}{0.483048in}}{\pgfqpoint{5.542458in}{0.478658in}}{\pgfqpoint{5.534644in}{0.470844in}}%
\pgfpathcurveto{\pgfqpoint{5.526830in}{0.463030in}}{\pgfqpoint{5.522440in}{0.452431in}}{\pgfqpoint{5.522440in}{0.441381in}}%
\pgfpathcurveto{\pgfqpoint{5.522440in}{0.430331in}}{\pgfqpoint{5.526830in}{0.419732in}}{\pgfqpoint{5.534644in}{0.411918in}}%
\pgfpathcurveto{\pgfqpoint{5.542458in}{0.404105in}}{\pgfqpoint{5.553057in}{0.399714in}}{\pgfqpoint{5.564107in}{0.399714in}}%
\pgfusepath{stroke}%
\end{pgfscope}%
\begin{pgfscope}%
\pgfpathrectangle{\pgfqpoint{0.494722in}{0.437222in}}{\pgfqpoint{6.275590in}{5.159444in}}%
\pgfusepath{clip}%
\pgfsetbuttcap%
\pgfsetroundjoin%
\pgfsetlinewidth{1.003750pt}%
\definecolor{currentstroke}{rgb}{0.827451,0.827451,0.827451}%
\pgfsetstrokecolor{currentstroke}%
\pgfsetstrokeopacity{0.800000}%
\pgfsetdash{}{0pt}%
\pgfpathmoveto{\pgfqpoint{1.308632in}{2.175867in}}%
\pgfpathcurveto{\pgfqpoint{1.319682in}{2.175867in}}{\pgfqpoint{1.330282in}{2.180258in}}{\pgfqpoint{1.338095in}{2.188071in}}%
\pgfpathcurveto{\pgfqpoint{1.345909in}{2.195885in}}{\pgfqpoint{1.350299in}{2.206484in}}{\pgfqpoint{1.350299in}{2.217534in}}%
\pgfpathcurveto{\pgfqpoint{1.350299in}{2.228584in}}{\pgfqpoint{1.345909in}{2.239183in}}{\pgfqpoint{1.338095in}{2.246997in}}%
\pgfpathcurveto{\pgfqpoint{1.330282in}{2.254810in}}{\pgfqpoint{1.319682in}{2.259201in}}{\pgfqpoint{1.308632in}{2.259201in}}%
\pgfpathcurveto{\pgfqpoint{1.297582in}{2.259201in}}{\pgfqpoint{1.286983in}{2.254810in}}{\pgfqpoint{1.279170in}{2.246997in}}%
\pgfpathcurveto{\pgfqpoint{1.271356in}{2.239183in}}{\pgfqpoint{1.266966in}{2.228584in}}{\pgfqpoint{1.266966in}{2.217534in}}%
\pgfpathcurveto{\pgfqpoint{1.266966in}{2.206484in}}{\pgfqpoint{1.271356in}{2.195885in}}{\pgfqpoint{1.279170in}{2.188071in}}%
\pgfpathcurveto{\pgfqpoint{1.286983in}{2.180258in}}{\pgfqpoint{1.297582in}{2.175867in}}{\pgfqpoint{1.308632in}{2.175867in}}%
\pgfpathlineto{\pgfqpoint{1.308632in}{2.175867in}}%
\pgfpathclose%
\pgfusepath{stroke}%
\end{pgfscope}%
\begin{pgfscope}%
\pgfpathrectangle{\pgfqpoint{0.494722in}{0.437222in}}{\pgfqpoint{6.275590in}{5.159444in}}%
\pgfusepath{clip}%
\pgfsetbuttcap%
\pgfsetroundjoin%
\pgfsetlinewidth{1.003750pt}%
\definecolor{currentstroke}{rgb}{0.827451,0.827451,0.827451}%
\pgfsetstrokecolor{currentstroke}%
\pgfsetstrokeopacity{0.800000}%
\pgfsetdash{}{0pt}%
\pgfpathmoveto{\pgfqpoint{0.553084in}{3.907249in}}%
\pgfpathcurveto{\pgfqpoint{0.564134in}{3.907249in}}{\pgfqpoint{0.574733in}{3.911639in}}{\pgfqpoint{0.582547in}{3.919453in}}%
\pgfpathcurveto{\pgfqpoint{0.590360in}{3.927267in}}{\pgfqpoint{0.594750in}{3.937866in}}{\pgfqpoint{0.594750in}{3.948916in}}%
\pgfpathcurveto{\pgfqpoint{0.594750in}{3.959966in}}{\pgfqpoint{0.590360in}{3.970565in}}{\pgfqpoint{0.582547in}{3.978378in}}%
\pgfpathcurveto{\pgfqpoint{0.574733in}{3.986192in}}{\pgfqpoint{0.564134in}{3.990582in}}{\pgfqpoint{0.553084in}{3.990582in}}%
\pgfpathcurveto{\pgfqpoint{0.542034in}{3.990582in}}{\pgfqpoint{0.531435in}{3.986192in}}{\pgfqpoint{0.523621in}{3.978378in}}%
\pgfpathcurveto{\pgfqpoint{0.515807in}{3.970565in}}{\pgfqpoint{0.511417in}{3.959966in}}{\pgfqpoint{0.511417in}{3.948916in}}%
\pgfpathcurveto{\pgfqpoint{0.511417in}{3.937866in}}{\pgfqpoint{0.515807in}{3.927267in}}{\pgfqpoint{0.523621in}{3.919453in}}%
\pgfpathcurveto{\pgfqpoint{0.531435in}{3.911639in}}{\pgfqpoint{0.542034in}{3.907249in}}{\pgfqpoint{0.553084in}{3.907249in}}%
\pgfpathlineto{\pgfqpoint{0.553084in}{3.907249in}}%
\pgfpathclose%
\pgfusepath{stroke}%
\end{pgfscope}%
\begin{pgfscope}%
\pgfpathrectangle{\pgfqpoint{0.494722in}{0.437222in}}{\pgfqpoint{6.275590in}{5.159444in}}%
\pgfusepath{clip}%
\pgfsetbuttcap%
\pgfsetroundjoin%
\pgfsetlinewidth{1.003750pt}%
\definecolor{currentstroke}{rgb}{0.827451,0.827451,0.827451}%
\pgfsetstrokecolor{currentstroke}%
\pgfsetstrokeopacity{0.800000}%
\pgfsetdash{}{0pt}%
\pgfpathmoveto{\pgfqpoint{2.224297in}{1.325142in}}%
\pgfpathcurveto{\pgfqpoint{2.235347in}{1.325142in}}{\pgfqpoint{2.245946in}{1.329532in}}{\pgfqpoint{2.253760in}{1.337346in}}%
\pgfpathcurveto{\pgfqpoint{2.261573in}{1.345159in}}{\pgfqpoint{2.265964in}{1.355758in}}{\pgfqpoint{2.265964in}{1.366809in}}%
\pgfpathcurveto{\pgfqpoint{2.265964in}{1.377859in}}{\pgfqpoint{2.261573in}{1.388458in}}{\pgfqpoint{2.253760in}{1.396271in}}%
\pgfpathcurveto{\pgfqpoint{2.245946in}{1.404085in}}{\pgfqpoint{2.235347in}{1.408475in}}{\pgfqpoint{2.224297in}{1.408475in}}%
\pgfpathcurveto{\pgfqpoint{2.213247in}{1.408475in}}{\pgfqpoint{2.202648in}{1.404085in}}{\pgfqpoint{2.194834in}{1.396271in}}%
\pgfpathcurveto{\pgfqpoint{2.187020in}{1.388458in}}{\pgfqpoint{2.182630in}{1.377859in}}{\pgfqpoint{2.182630in}{1.366809in}}%
\pgfpathcurveto{\pgfqpoint{2.182630in}{1.355758in}}{\pgfqpoint{2.187020in}{1.345159in}}{\pgfqpoint{2.194834in}{1.337346in}}%
\pgfpathcurveto{\pgfqpoint{2.202648in}{1.329532in}}{\pgfqpoint{2.213247in}{1.325142in}}{\pgfqpoint{2.224297in}{1.325142in}}%
\pgfpathlineto{\pgfqpoint{2.224297in}{1.325142in}}%
\pgfpathclose%
\pgfusepath{stroke}%
\end{pgfscope}%
\begin{pgfscope}%
\pgfpathrectangle{\pgfqpoint{0.494722in}{0.437222in}}{\pgfqpoint{6.275590in}{5.159444in}}%
\pgfusepath{clip}%
\pgfsetbuttcap%
\pgfsetroundjoin%
\pgfsetlinewidth{1.003750pt}%
\definecolor{currentstroke}{rgb}{0.827451,0.827451,0.827451}%
\pgfsetstrokecolor{currentstroke}%
\pgfsetstrokeopacity{0.800000}%
\pgfsetdash{}{0pt}%
\pgfpathmoveto{\pgfqpoint{1.325845in}{2.126824in}}%
\pgfpathcurveto{\pgfqpoint{1.336895in}{2.126824in}}{\pgfqpoint{1.347494in}{2.131214in}}{\pgfqpoint{1.355308in}{2.139028in}}%
\pgfpathcurveto{\pgfqpoint{1.363121in}{2.146842in}}{\pgfqpoint{1.367512in}{2.157441in}}{\pgfqpoint{1.367512in}{2.168491in}}%
\pgfpathcurveto{\pgfqpoint{1.367512in}{2.179541in}}{\pgfqpoint{1.363121in}{2.190140in}}{\pgfqpoint{1.355308in}{2.197954in}}%
\pgfpathcurveto{\pgfqpoint{1.347494in}{2.205767in}}{\pgfqpoint{1.336895in}{2.210157in}}{\pgfqpoint{1.325845in}{2.210157in}}%
\pgfpathcurveto{\pgfqpoint{1.314795in}{2.210157in}}{\pgfqpoint{1.304196in}{2.205767in}}{\pgfqpoint{1.296382in}{2.197954in}}%
\pgfpathcurveto{\pgfqpoint{1.288569in}{2.190140in}}{\pgfqpoint{1.284178in}{2.179541in}}{\pgfqpoint{1.284178in}{2.168491in}}%
\pgfpathcurveto{\pgfqpoint{1.284178in}{2.157441in}}{\pgfqpoint{1.288569in}{2.146842in}}{\pgfqpoint{1.296382in}{2.139028in}}%
\pgfpathcurveto{\pgfqpoint{1.304196in}{2.131214in}}{\pgfqpoint{1.314795in}{2.126824in}}{\pgfqpoint{1.325845in}{2.126824in}}%
\pgfpathlineto{\pgfqpoint{1.325845in}{2.126824in}}%
\pgfpathclose%
\pgfusepath{stroke}%
\end{pgfscope}%
\begin{pgfscope}%
\pgfpathrectangle{\pgfqpoint{0.494722in}{0.437222in}}{\pgfqpoint{6.275590in}{5.159444in}}%
\pgfusepath{clip}%
\pgfsetbuttcap%
\pgfsetroundjoin%
\pgfsetlinewidth{1.003750pt}%
\definecolor{currentstroke}{rgb}{0.827451,0.827451,0.827451}%
\pgfsetstrokecolor{currentstroke}%
\pgfsetstrokeopacity{0.800000}%
\pgfsetdash{}{0pt}%
\pgfpathmoveto{\pgfqpoint{2.579575in}{1.114207in}}%
\pgfpathcurveto{\pgfqpoint{2.590625in}{1.114207in}}{\pgfqpoint{2.601224in}{1.118597in}}{\pgfqpoint{2.609038in}{1.126411in}}%
\pgfpathcurveto{\pgfqpoint{2.616851in}{1.134224in}}{\pgfqpoint{2.621242in}{1.144823in}}{\pgfqpoint{2.621242in}{1.155874in}}%
\pgfpathcurveto{\pgfqpoint{2.621242in}{1.166924in}}{\pgfqpoint{2.616851in}{1.177523in}}{\pgfqpoint{2.609038in}{1.185336in}}%
\pgfpathcurveto{\pgfqpoint{2.601224in}{1.193150in}}{\pgfqpoint{2.590625in}{1.197540in}}{\pgfqpoint{2.579575in}{1.197540in}}%
\pgfpathcurveto{\pgfqpoint{2.568525in}{1.197540in}}{\pgfqpoint{2.557926in}{1.193150in}}{\pgfqpoint{2.550112in}{1.185336in}}%
\pgfpathcurveto{\pgfqpoint{2.542298in}{1.177523in}}{\pgfqpoint{2.537908in}{1.166924in}}{\pgfqpoint{2.537908in}{1.155874in}}%
\pgfpathcurveto{\pgfqpoint{2.537908in}{1.144823in}}{\pgfqpoint{2.542298in}{1.134224in}}{\pgfqpoint{2.550112in}{1.126411in}}%
\pgfpathcurveto{\pgfqpoint{2.557926in}{1.118597in}}{\pgfqpoint{2.568525in}{1.114207in}}{\pgfqpoint{2.579575in}{1.114207in}}%
\pgfpathlineto{\pgfqpoint{2.579575in}{1.114207in}}%
\pgfpathclose%
\pgfusepath{stroke}%
\end{pgfscope}%
\begin{pgfscope}%
\pgfpathrectangle{\pgfqpoint{0.494722in}{0.437222in}}{\pgfqpoint{6.275590in}{5.159444in}}%
\pgfusepath{clip}%
\pgfsetbuttcap%
\pgfsetroundjoin%
\pgfsetlinewidth{1.003750pt}%
\definecolor{currentstroke}{rgb}{0.827451,0.827451,0.827451}%
\pgfsetstrokecolor{currentstroke}%
\pgfsetstrokeopacity{0.800000}%
\pgfsetdash{}{0pt}%
\pgfpathmoveto{\pgfqpoint{4.229480in}{0.518110in}}%
\pgfpathcurveto{\pgfqpoint{4.240530in}{0.518110in}}{\pgfqpoint{4.251129in}{0.522500in}}{\pgfqpoint{4.258942in}{0.530314in}}%
\pgfpathcurveto{\pgfqpoint{4.266756in}{0.538127in}}{\pgfqpoint{4.271146in}{0.548726in}}{\pgfqpoint{4.271146in}{0.559776in}}%
\pgfpathcurveto{\pgfqpoint{4.271146in}{0.570827in}}{\pgfqpoint{4.266756in}{0.581426in}}{\pgfqpoint{4.258942in}{0.589239in}}%
\pgfpathcurveto{\pgfqpoint{4.251129in}{0.597053in}}{\pgfqpoint{4.240530in}{0.601443in}}{\pgfqpoint{4.229480in}{0.601443in}}%
\pgfpathcurveto{\pgfqpoint{4.218429in}{0.601443in}}{\pgfqpoint{4.207830in}{0.597053in}}{\pgfqpoint{4.200017in}{0.589239in}}%
\pgfpathcurveto{\pgfqpoint{4.192203in}{0.581426in}}{\pgfqpoint{4.187813in}{0.570827in}}{\pgfqpoint{4.187813in}{0.559776in}}%
\pgfpathcurveto{\pgfqpoint{4.187813in}{0.548726in}}{\pgfqpoint{4.192203in}{0.538127in}}{\pgfqpoint{4.200017in}{0.530314in}}%
\pgfpathcurveto{\pgfqpoint{4.207830in}{0.522500in}}{\pgfqpoint{4.218429in}{0.518110in}}{\pgfqpoint{4.229480in}{0.518110in}}%
\pgfpathlineto{\pgfqpoint{4.229480in}{0.518110in}}%
\pgfpathclose%
\pgfusepath{stroke}%
\end{pgfscope}%
\begin{pgfscope}%
\pgfpathrectangle{\pgfqpoint{0.494722in}{0.437222in}}{\pgfqpoint{6.275590in}{5.159444in}}%
\pgfusepath{clip}%
\pgfsetbuttcap%
\pgfsetroundjoin%
\pgfsetlinewidth{1.003750pt}%
\definecolor{currentstroke}{rgb}{0.827451,0.827451,0.827451}%
\pgfsetstrokecolor{currentstroke}%
\pgfsetstrokeopacity{0.800000}%
\pgfsetdash{}{0pt}%
\pgfpathmoveto{\pgfqpoint{3.381654in}{0.741657in}}%
\pgfpathcurveto{\pgfqpoint{3.392704in}{0.741657in}}{\pgfqpoint{3.403303in}{0.746047in}}{\pgfqpoint{3.411116in}{0.753861in}}%
\pgfpathcurveto{\pgfqpoint{3.418930in}{0.761674in}}{\pgfqpoint{3.423320in}{0.772273in}}{\pgfqpoint{3.423320in}{0.783324in}}%
\pgfpathcurveto{\pgfqpoint{3.423320in}{0.794374in}}{\pgfqpoint{3.418930in}{0.804973in}}{\pgfqpoint{3.411116in}{0.812786in}}%
\pgfpathcurveto{\pgfqpoint{3.403303in}{0.820600in}}{\pgfqpoint{3.392704in}{0.824990in}}{\pgfqpoint{3.381654in}{0.824990in}}%
\pgfpathcurveto{\pgfqpoint{3.370604in}{0.824990in}}{\pgfqpoint{3.360005in}{0.820600in}}{\pgfqpoint{3.352191in}{0.812786in}}%
\pgfpathcurveto{\pgfqpoint{3.344377in}{0.804973in}}{\pgfqpoint{3.339987in}{0.794374in}}{\pgfqpoint{3.339987in}{0.783324in}}%
\pgfpathcurveto{\pgfqpoint{3.339987in}{0.772273in}}{\pgfqpoint{3.344377in}{0.761674in}}{\pgfqpoint{3.352191in}{0.753861in}}%
\pgfpathcurveto{\pgfqpoint{3.360005in}{0.746047in}}{\pgfqpoint{3.370604in}{0.741657in}}{\pgfqpoint{3.381654in}{0.741657in}}%
\pgfpathlineto{\pgfqpoint{3.381654in}{0.741657in}}%
\pgfpathclose%
\pgfusepath{stroke}%
\end{pgfscope}%
\begin{pgfscope}%
\pgfpathrectangle{\pgfqpoint{0.494722in}{0.437222in}}{\pgfqpoint{6.275590in}{5.159444in}}%
\pgfusepath{clip}%
\pgfsetbuttcap%
\pgfsetroundjoin%
\pgfsetlinewidth{1.003750pt}%
\definecolor{currentstroke}{rgb}{0.827451,0.827451,0.827451}%
\pgfsetstrokecolor{currentstroke}%
\pgfsetstrokeopacity{0.800000}%
\pgfsetdash{}{0pt}%
\pgfpathmoveto{\pgfqpoint{2.183447in}{1.371651in}}%
\pgfpathcurveto{\pgfqpoint{2.194497in}{1.371651in}}{\pgfqpoint{2.205096in}{1.376041in}}{\pgfqpoint{2.212910in}{1.383855in}}%
\pgfpathcurveto{\pgfqpoint{2.220724in}{1.391668in}}{\pgfqpoint{2.225114in}{1.402267in}}{\pgfqpoint{2.225114in}{1.413318in}}%
\pgfpathcurveto{\pgfqpoint{2.225114in}{1.424368in}}{\pgfqpoint{2.220724in}{1.434967in}}{\pgfqpoint{2.212910in}{1.442780in}}%
\pgfpathcurveto{\pgfqpoint{2.205096in}{1.450594in}}{\pgfqpoint{2.194497in}{1.454984in}}{\pgfqpoint{2.183447in}{1.454984in}}%
\pgfpathcurveto{\pgfqpoint{2.172397in}{1.454984in}}{\pgfqpoint{2.161798in}{1.450594in}}{\pgfqpoint{2.153984in}{1.442780in}}%
\pgfpathcurveto{\pgfqpoint{2.146171in}{1.434967in}}{\pgfqpoint{2.141781in}{1.424368in}}{\pgfqpoint{2.141781in}{1.413318in}}%
\pgfpathcurveto{\pgfqpoint{2.141781in}{1.402267in}}{\pgfqpoint{2.146171in}{1.391668in}}{\pgfqpoint{2.153984in}{1.383855in}}%
\pgfpathcurveto{\pgfqpoint{2.161798in}{1.376041in}}{\pgfqpoint{2.172397in}{1.371651in}}{\pgfqpoint{2.183447in}{1.371651in}}%
\pgfpathlineto{\pgfqpoint{2.183447in}{1.371651in}}%
\pgfpathclose%
\pgfusepath{stroke}%
\end{pgfscope}%
\begin{pgfscope}%
\pgfpathrectangle{\pgfqpoint{0.494722in}{0.437222in}}{\pgfqpoint{6.275590in}{5.159444in}}%
\pgfusepath{clip}%
\pgfsetbuttcap%
\pgfsetroundjoin%
\pgfsetlinewidth{1.003750pt}%
\definecolor{currentstroke}{rgb}{0.827451,0.827451,0.827451}%
\pgfsetstrokecolor{currentstroke}%
\pgfsetstrokeopacity{0.800000}%
\pgfsetdash{}{0pt}%
\pgfpathmoveto{\pgfqpoint{0.870505in}{2.877763in}}%
\pgfpathcurveto{\pgfqpoint{0.881555in}{2.877763in}}{\pgfqpoint{0.892154in}{2.882154in}}{\pgfqpoint{0.899967in}{2.889967in}}%
\pgfpathcurveto{\pgfqpoint{0.907781in}{2.897781in}}{\pgfqpoint{0.912171in}{2.908380in}}{\pgfqpoint{0.912171in}{2.919430in}}%
\pgfpathcurveto{\pgfqpoint{0.912171in}{2.930480in}}{\pgfqpoint{0.907781in}{2.941079in}}{\pgfqpoint{0.899967in}{2.948893in}}%
\pgfpathcurveto{\pgfqpoint{0.892154in}{2.956706in}}{\pgfqpoint{0.881555in}{2.961097in}}{\pgfqpoint{0.870505in}{2.961097in}}%
\pgfpathcurveto{\pgfqpoint{0.859455in}{2.961097in}}{\pgfqpoint{0.848856in}{2.956706in}}{\pgfqpoint{0.841042in}{2.948893in}}%
\pgfpathcurveto{\pgfqpoint{0.833228in}{2.941079in}}{\pgfqpoint{0.828838in}{2.930480in}}{\pgfqpoint{0.828838in}{2.919430in}}%
\pgfpathcurveto{\pgfqpoint{0.828838in}{2.908380in}}{\pgfqpoint{0.833228in}{2.897781in}}{\pgfqpoint{0.841042in}{2.889967in}}%
\pgfpathcurveto{\pgfqpoint{0.848856in}{2.882154in}}{\pgfqpoint{0.859455in}{2.877763in}}{\pgfqpoint{0.870505in}{2.877763in}}%
\pgfpathlineto{\pgfqpoint{0.870505in}{2.877763in}}%
\pgfpathclose%
\pgfusepath{stroke}%
\end{pgfscope}%
\begin{pgfscope}%
\pgfpathrectangle{\pgfqpoint{0.494722in}{0.437222in}}{\pgfqpoint{6.275590in}{5.159444in}}%
\pgfusepath{clip}%
\pgfsetbuttcap%
\pgfsetroundjoin%
\pgfsetlinewidth{1.003750pt}%
\definecolor{currentstroke}{rgb}{0.827451,0.827451,0.827451}%
\pgfsetstrokecolor{currentstroke}%
\pgfsetstrokeopacity{0.800000}%
\pgfsetdash{}{0pt}%
\pgfpathmoveto{\pgfqpoint{0.784854in}{3.036924in}}%
\pgfpathcurveto{\pgfqpoint{0.795904in}{3.036924in}}{\pgfqpoint{0.806503in}{3.041314in}}{\pgfqpoint{0.814317in}{3.049128in}}%
\pgfpathcurveto{\pgfqpoint{0.822130in}{3.056941in}}{\pgfqpoint{0.826520in}{3.067540in}}{\pgfqpoint{0.826520in}{3.078590in}}%
\pgfpathcurveto{\pgfqpoint{0.826520in}{3.089640in}}{\pgfqpoint{0.822130in}{3.100239in}}{\pgfqpoint{0.814317in}{3.108053in}}%
\pgfpathcurveto{\pgfqpoint{0.806503in}{3.115867in}}{\pgfqpoint{0.795904in}{3.120257in}}{\pgfqpoint{0.784854in}{3.120257in}}%
\pgfpathcurveto{\pgfqpoint{0.773804in}{3.120257in}}{\pgfqpoint{0.763205in}{3.115867in}}{\pgfqpoint{0.755391in}{3.108053in}}%
\pgfpathcurveto{\pgfqpoint{0.747577in}{3.100239in}}{\pgfqpoint{0.743187in}{3.089640in}}{\pgfqpoint{0.743187in}{3.078590in}}%
\pgfpathcurveto{\pgfqpoint{0.743187in}{3.067540in}}{\pgfqpoint{0.747577in}{3.056941in}}{\pgfqpoint{0.755391in}{3.049128in}}%
\pgfpathcurveto{\pgfqpoint{0.763205in}{3.041314in}}{\pgfqpoint{0.773804in}{3.036924in}}{\pgfqpoint{0.784854in}{3.036924in}}%
\pgfpathlineto{\pgfqpoint{0.784854in}{3.036924in}}%
\pgfpathclose%
\pgfusepath{stroke}%
\end{pgfscope}%
\begin{pgfscope}%
\pgfpathrectangle{\pgfqpoint{0.494722in}{0.437222in}}{\pgfqpoint{6.275590in}{5.159444in}}%
\pgfusepath{clip}%
\pgfsetbuttcap%
\pgfsetroundjoin%
\pgfsetlinewidth{1.003750pt}%
\definecolor{currentstroke}{rgb}{0.827451,0.827451,0.827451}%
\pgfsetstrokecolor{currentstroke}%
\pgfsetstrokeopacity{0.800000}%
\pgfsetdash{}{0pt}%
\pgfpathmoveto{\pgfqpoint{1.013007in}{2.610425in}}%
\pgfpathcurveto{\pgfqpoint{1.024057in}{2.610425in}}{\pgfqpoint{1.034656in}{2.614815in}}{\pgfqpoint{1.042469in}{2.622629in}}%
\pgfpathcurveto{\pgfqpoint{1.050283in}{2.630443in}}{\pgfqpoint{1.054673in}{2.641042in}}{\pgfqpoint{1.054673in}{2.652092in}}%
\pgfpathcurveto{\pgfqpoint{1.054673in}{2.663142in}}{\pgfqpoint{1.050283in}{2.673741in}}{\pgfqpoint{1.042469in}{2.681555in}}%
\pgfpathcurveto{\pgfqpoint{1.034656in}{2.689368in}}{\pgfqpoint{1.024057in}{2.693758in}}{\pgfqpoint{1.013007in}{2.693758in}}%
\pgfpathcurveto{\pgfqpoint{1.001956in}{2.693758in}}{\pgfqpoint{0.991357in}{2.689368in}}{\pgfqpoint{0.983544in}{2.681555in}}%
\pgfpathcurveto{\pgfqpoint{0.975730in}{2.673741in}}{\pgfqpoint{0.971340in}{2.663142in}}{\pgfqpoint{0.971340in}{2.652092in}}%
\pgfpathcurveto{\pgfqpoint{0.971340in}{2.641042in}}{\pgfqpoint{0.975730in}{2.630443in}}{\pgfqpoint{0.983544in}{2.622629in}}%
\pgfpathcurveto{\pgfqpoint{0.991357in}{2.614815in}}{\pgfqpoint{1.001956in}{2.610425in}}{\pgfqpoint{1.013007in}{2.610425in}}%
\pgfpathlineto{\pgfqpoint{1.013007in}{2.610425in}}%
\pgfpathclose%
\pgfusepath{stroke}%
\end{pgfscope}%
\begin{pgfscope}%
\pgfpathrectangle{\pgfqpoint{0.494722in}{0.437222in}}{\pgfqpoint{6.275590in}{5.159444in}}%
\pgfusepath{clip}%
\pgfsetbuttcap%
\pgfsetroundjoin%
\pgfsetlinewidth{1.003750pt}%
\definecolor{currentstroke}{rgb}{0.827451,0.827451,0.827451}%
\pgfsetstrokecolor{currentstroke}%
\pgfsetstrokeopacity{0.800000}%
\pgfsetdash{}{0pt}%
\pgfpathmoveto{\pgfqpoint{1.347485in}{2.096739in}}%
\pgfpathcurveto{\pgfqpoint{1.358535in}{2.096739in}}{\pgfqpoint{1.369135in}{2.101129in}}{\pgfqpoint{1.376948in}{2.108943in}}%
\pgfpathcurveto{\pgfqpoint{1.384762in}{2.116757in}}{\pgfqpoint{1.389152in}{2.127356in}}{\pgfqpoint{1.389152in}{2.138406in}}%
\pgfpathcurveto{\pgfqpoint{1.389152in}{2.149456in}}{\pgfqpoint{1.384762in}{2.160055in}}{\pgfqpoint{1.376948in}{2.167869in}}%
\pgfpathcurveto{\pgfqpoint{1.369135in}{2.175682in}}{\pgfqpoint{1.358535in}{2.180072in}}{\pgfqpoint{1.347485in}{2.180072in}}%
\pgfpathcurveto{\pgfqpoint{1.336435in}{2.180072in}}{\pgfqpoint{1.325836in}{2.175682in}}{\pgfqpoint{1.318023in}{2.167869in}}%
\pgfpathcurveto{\pgfqpoint{1.310209in}{2.160055in}}{\pgfqpoint{1.305819in}{2.149456in}}{\pgfqpoint{1.305819in}{2.138406in}}%
\pgfpathcurveto{\pgfqpoint{1.305819in}{2.127356in}}{\pgfqpoint{1.310209in}{2.116757in}}{\pgfqpoint{1.318023in}{2.108943in}}%
\pgfpathcurveto{\pgfqpoint{1.325836in}{2.101129in}}{\pgfqpoint{1.336435in}{2.096739in}}{\pgfqpoint{1.347485in}{2.096739in}}%
\pgfpathlineto{\pgfqpoint{1.347485in}{2.096739in}}%
\pgfpathclose%
\pgfusepath{stroke}%
\end{pgfscope}%
\begin{pgfscope}%
\pgfpathrectangle{\pgfqpoint{0.494722in}{0.437222in}}{\pgfqpoint{6.275590in}{5.159444in}}%
\pgfusepath{clip}%
\pgfsetbuttcap%
\pgfsetroundjoin%
\pgfsetlinewidth{1.003750pt}%
\definecolor{currentstroke}{rgb}{0.827451,0.827451,0.827451}%
\pgfsetstrokecolor{currentstroke}%
\pgfsetstrokeopacity{0.800000}%
\pgfsetdash{}{0pt}%
\pgfpathmoveto{\pgfqpoint{1.227841in}{2.257546in}}%
\pgfpathcurveto{\pgfqpoint{1.238891in}{2.257546in}}{\pgfqpoint{1.249490in}{2.261936in}}{\pgfqpoint{1.257304in}{2.269750in}}%
\pgfpathcurveto{\pgfqpoint{1.265118in}{2.277564in}}{\pgfqpoint{1.269508in}{2.288163in}}{\pgfqpoint{1.269508in}{2.299213in}}%
\pgfpathcurveto{\pgfqpoint{1.269508in}{2.310263in}}{\pgfqpoint{1.265118in}{2.320862in}}{\pgfqpoint{1.257304in}{2.328675in}}%
\pgfpathcurveto{\pgfqpoint{1.249490in}{2.336489in}}{\pgfqpoint{1.238891in}{2.340879in}}{\pgfqpoint{1.227841in}{2.340879in}}%
\pgfpathcurveto{\pgfqpoint{1.216791in}{2.340879in}}{\pgfqpoint{1.206192in}{2.336489in}}{\pgfqpoint{1.198379in}{2.328675in}}%
\pgfpathcurveto{\pgfqpoint{1.190565in}{2.320862in}}{\pgfqpoint{1.186175in}{2.310263in}}{\pgfqpoint{1.186175in}{2.299213in}}%
\pgfpathcurveto{\pgfqpoint{1.186175in}{2.288163in}}{\pgfqpoint{1.190565in}{2.277564in}}{\pgfqpoint{1.198379in}{2.269750in}}%
\pgfpathcurveto{\pgfqpoint{1.206192in}{2.261936in}}{\pgfqpoint{1.216791in}{2.257546in}}{\pgfqpoint{1.227841in}{2.257546in}}%
\pgfpathlineto{\pgfqpoint{1.227841in}{2.257546in}}%
\pgfpathclose%
\pgfusepath{stroke}%
\end{pgfscope}%
\begin{pgfscope}%
\pgfpathrectangle{\pgfqpoint{0.494722in}{0.437222in}}{\pgfqpoint{6.275590in}{5.159444in}}%
\pgfusepath{clip}%
\pgfsetbuttcap%
\pgfsetroundjoin%
\pgfsetlinewidth{1.003750pt}%
\definecolor{currentstroke}{rgb}{0.827451,0.827451,0.827451}%
\pgfsetstrokecolor{currentstroke}%
\pgfsetstrokeopacity{0.800000}%
\pgfsetdash{}{0pt}%
\pgfpathmoveto{\pgfqpoint{1.545871in}{1.889457in}}%
\pgfpathcurveto{\pgfqpoint{1.556921in}{1.889457in}}{\pgfqpoint{1.567520in}{1.893848in}}{\pgfqpoint{1.575334in}{1.901661in}}%
\pgfpathcurveto{\pgfqpoint{1.583148in}{1.909475in}}{\pgfqpoint{1.587538in}{1.920074in}}{\pgfqpoint{1.587538in}{1.931124in}}%
\pgfpathcurveto{\pgfqpoint{1.587538in}{1.942174in}}{\pgfqpoint{1.583148in}{1.952773in}}{\pgfqpoint{1.575334in}{1.960587in}}%
\pgfpathcurveto{\pgfqpoint{1.567520in}{1.968400in}}{\pgfqpoint{1.556921in}{1.972791in}}{\pgfqpoint{1.545871in}{1.972791in}}%
\pgfpathcurveto{\pgfqpoint{1.534821in}{1.972791in}}{\pgfqpoint{1.524222in}{1.968400in}}{\pgfqpoint{1.516408in}{1.960587in}}%
\pgfpathcurveto{\pgfqpoint{1.508595in}{1.952773in}}{\pgfqpoint{1.504204in}{1.942174in}}{\pgfqpoint{1.504204in}{1.931124in}}%
\pgfpathcurveto{\pgfqpoint{1.504204in}{1.920074in}}{\pgfqpoint{1.508595in}{1.909475in}}{\pgfqpoint{1.516408in}{1.901661in}}%
\pgfpathcurveto{\pgfqpoint{1.524222in}{1.893848in}}{\pgfqpoint{1.534821in}{1.889457in}}{\pgfqpoint{1.545871in}{1.889457in}}%
\pgfpathlineto{\pgfqpoint{1.545871in}{1.889457in}}%
\pgfpathclose%
\pgfusepath{stroke}%
\end{pgfscope}%
\begin{pgfscope}%
\pgfpathrectangle{\pgfqpoint{0.494722in}{0.437222in}}{\pgfqpoint{6.275590in}{5.159444in}}%
\pgfusepath{clip}%
\pgfsetbuttcap%
\pgfsetroundjoin%
\pgfsetlinewidth{1.003750pt}%
\definecolor{currentstroke}{rgb}{0.827451,0.827451,0.827451}%
\pgfsetstrokecolor{currentstroke}%
\pgfsetstrokeopacity{0.800000}%
\pgfsetdash{}{0pt}%
\pgfpathmoveto{\pgfqpoint{0.495079in}{4.572987in}}%
\pgfpathcurveto{\pgfqpoint{0.506130in}{4.572987in}}{\pgfqpoint{0.516729in}{4.577378in}}{\pgfqpoint{0.524542in}{4.585191in}}%
\pgfpathcurveto{\pgfqpoint{0.532356in}{4.593005in}}{\pgfqpoint{0.536746in}{4.603604in}}{\pgfqpoint{0.536746in}{4.614654in}}%
\pgfpathcurveto{\pgfqpoint{0.536746in}{4.625704in}}{\pgfqpoint{0.532356in}{4.636303in}}{\pgfqpoint{0.524542in}{4.644117in}}%
\pgfpathcurveto{\pgfqpoint{0.516729in}{4.651930in}}{\pgfqpoint{0.506130in}{4.656321in}}{\pgfqpoint{0.495079in}{4.656321in}}%
\pgfpathcurveto{\pgfqpoint{0.484029in}{4.656321in}}{\pgfqpoint{0.473430in}{4.651930in}}{\pgfqpoint{0.465617in}{4.644117in}}%
\pgfpathcurveto{\pgfqpoint{0.457803in}{4.636303in}}{\pgfqpoint{0.453413in}{4.625704in}}{\pgfqpoint{0.453413in}{4.614654in}}%
\pgfpathcurveto{\pgfqpoint{0.453413in}{4.603604in}}{\pgfqpoint{0.457803in}{4.593005in}}{\pgfqpoint{0.465617in}{4.585191in}}%
\pgfpathcurveto{\pgfqpoint{0.473430in}{4.577378in}}{\pgfqpoint{0.484029in}{4.572987in}}{\pgfqpoint{0.495079in}{4.572987in}}%
\pgfpathlineto{\pgfqpoint{0.495079in}{4.572987in}}%
\pgfpathclose%
\pgfusepath{stroke}%
\end{pgfscope}%
\begin{pgfscope}%
\pgfpathrectangle{\pgfqpoint{0.494722in}{0.437222in}}{\pgfqpoint{6.275590in}{5.159444in}}%
\pgfusepath{clip}%
\pgfsetbuttcap%
\pgfsetroundjoin%
\pgfsetlinewidth{1.003750pt}%
\definecolor{currentstroke}{rgb}{0.827451,0.827451,0.827451}%
\pgfsetstrokecolor{currentstroke}%
\pgfsetstrokeopacity{0.800000}%
\pgfsetdash{}{0pt}%
\pgfpathmoveto{\pgfqpoint{5.757662in}{0.395992in}}%
\pgfpathcurveto{\pgfqpoint{5.768712in}{0.395992in}}{\pgfqpoint{5.779311in}{0.400382in}}{\pgfqpoint{5.787125in}{0.408196in}}%
\pgfpathcurveto{\pgfqpoint{5.794939in}{0.416009in}}{\pgfqpoint{5.799329in}{0.426608in}}{\pgfqpoint{5.799329in}{0.437658in}}%
\pgfpathcurveto{\pgfqpoint{5.799329in}{0.448709in}}{\pgfqpoint{5.794939in}{0.459308in}}{\pgfqpoint{5.787125in}{0.467121in}}%
\pgfpathcurveto{\pgfqpoint{5.779311in}{0.474935in}}{\pgfqpoint{5.768712in}{0.479325in}}{\pgfqpoint{5.757662in}{0.479325in}}%
\pgfpathcurveto{\pgfqpoint{5.746612in}{0.479325in}}{\pgfqpoint{5.736013in}{0.474935in}}{\pgfqpoint{5.728199in}{0.467121in}}%
\pgfpathcurveto{\pgfqpoint{5.720386in}{0.459308in}}{\pgfqpoint{5.715996in}{0.448709in}}{\pgfqpoint{5.715996in}{0.437658in}}%
\pgfpathcurveto{\pgfqpoint{5.715996in}{0.426608in}}{\pgfqpoint{5.720386in}{0.416009in}}{\pgfqpoint{5.728199in}{0.408196in}}%
\pgfpathcurveto{\pgfqpoint{5.736013in}{0.400382in}}{\pgfqpoint{5.746612in}{0.395992in}}{\pgfqpoint{5.757662in}{0.395992in}}%
\pgfusepath{stroke}%
\end{pgfscope}%
\begin{pgfscope}%
\pgfpathrectangle{\pgfqpoint{0.494722in}{0.437222in}}{\pgfqpoint{6.275590in}{5.159444in}}%
\pgfusepath{clip}%
\pgfsetbuttcap%
\pgfsetroundjoin%
\pgfsetlinewidth{1.003750pt}%
\definecolor{currentstroke}{rgb}{0.827451,0.827451,0.827451}%
\pgfsetstrokecolor{currentstroke}%
\pgfsetstrokeopacity{0.800000}%
\pgfsetdash{}{0pt}%
\pgfpathmoveto{\pgfqpoint{0.501219in}{4.373826in}}%
\pgfpathcurveto{\pgfqpoint{0.512269in}{4.373826in}}{\pgfqpoint{0.522868in}{4.378216in}}{\pgfqpoint{0.530682in}{4.386030in}}%
\pgfpathcurveto{\pgfqpoint{0.538495in}{4.393844in}}{\pgfqpoint{0.542886in}{4.404443in}}{\pgfqpoint{0.542886in}{4.415493in}}%
\pgfpathcurveto{\pgfqpoint{0.542886in}{4.426543in}}{\pgfqpoint{0.538495in}{4.437142in}}{\pgfqpoint{0.530682in}{4.444955in}}%
\pgfpathcurveto{\pgfqpoint{0.522868in}{4.452769in}}{\pgfqpoint{0.512269in}{4.457159in}}{\pgfqpoint{0.501219in}{4.457159in}}%
\pgfpathcurveto{\pgfqpoint{0.490169in}{4.457159in}}{\pgfqpoint{0.479570in}{4.452769in}}{\pgfqpoint{0.471756in}{4.444955in}}%
\pgfpathcurveto{\pgfqpoint{0.463943in}{4.437142in}}{\pgfqpoint{0.459552in}{4.426543in}}{\pgfqpoint{0.459552in}{4.415493in}}%
\pgfpathcurveto{\pgfqpoint{0.459552in}{4.404443in}}{\pgfqpoint{0.463943in}{4.393844in}}{\pgfqpoint{0.471756in}{4.386030in}}%
\pgfpathcurveto{\pgfqpoint{0.479570in}{4.378216in}}{\pgfqpoint{0.490169in}{4.373826in}}{\pgfqpoint{0.501219in}{4.373826in}}%
\pgfpathlineto{\pgfqpoint{0.501219in}{4.373826in}}%
\pgfpathclose%
\pgfusepath{stroke}%
\end{pgfscope}%
\begin{pgfscope}%
\pgfpathrectangle{\pgfqpoint{0.494722in}{0.437222in}}{\pgfqpoint{6.275590in}{5.159444in}}%
\pgfusepath{clip}%
\pgfsetbuttcap%
\pgfsetroundjoin%
\pgfsetlinewidth{1.003750pt}%
\definecolor{currentstroke}{rgb}{0.827451,0.827451,0.827451}%
\pgfsetstrokecolor{currentstroke}%
\pgfsetstrokeopacity{0.800000}%
\pgfsetdash{}{0pt}%
\pgfpathmoveto{\pgfqpoint{2.990259in}{0.911738in}}%
\pgfpathcurveto{\pgfqpoint{3.001309in}{0.911738in}}{\pgfqpoint{3.011908in}{0.916128in}}{\pgfqpoint{3.019722in}{0.923942in}}%
\pgfpathcurveto{\pgfqpoint{3.027536in}{0.931755in}}{\pgfqpoint{3.031926in}{0.942354in}}{\pgfqpoint{3.031926in}{0.953404in}}%
\pgfpathcurveto{\pgfqpoint{3.031926in}{0.964455in}}{\pgfqpoint{3.027536in}{0.975054in}}{\pgfqpoint{3.019722in}{0.982867in}}%
\pgfpathcurveto{\pgfqpoint{3.011908in}{0.990681in}}{\pgfqpoint{3.001309in}{0.995071in}}{\pgfqpoint{2.990259in}{0.995071in}}%
\pgfpathcurveto{\pgfqpoint{2.979209in}{0.995071in}}{\pgfqpoint{2.968610in}{0.990681in}}{\pgfqpoint{2.960796in}{0.982867in}}%
\pgfpathcurveto{\pgfqpoint{2.952983in}{0.975054in}}{\pgfqpoint{2.948593in}{0.964455in}}{\pgfqpoint{2.948593in}{0.953404in}}%
\pgfpathcurveto{\pgfqpoint{2.948593in}{0.942354in}}{\pgfqpoint{2.952983in}{0.931755in}}{\pgfqpoint{2.960796in}{0.923942in}}%
\pgfpathcurveto{\pgfqpoint{2.968610in}{0.916128in}}{\pgfqpoint{2.979209in}{0.911738in}}{\pgfqpoint{2.990259in}{0.911738in}}%
\pgfpathlineto{\pgfqpoint{2.990259in}{0.911738in}}%
\pgfpathclose%
\pgfusepath{stroke}%
\end{pgfscope}%
\begin{pgfscope}%
\pgfpathrectangle{\pgfqpoint{0.494722in}{0.437222in}}{\pgfqpoint{6.275590in}{5.159444in}}%
\pgfusepath{clip}%
\pgfsetbuttcap%
\pgfsetroundjoin%
\pgfsetlinewidth{1.003750pt}%
\definecolor{currentstroke}{rgb}{0.827451,0.827451,0.827451}%
\pgfsetstrokecolor{currentstroke}%
\pgfsetstrokeopacity{0.800000}%
\pgfsetdash{}{0pt}%
\pgfpathmoveto{\pgfqpoint{1.963941in}{1.515375in}}%
\pgfpathcurveto{\pgfqpoint{1.974991in}{1.515375in}}{\pgfqpoint{1.985590in}{1.519765in}}{\pgfqpoint{1.993404in}{1.527579in}}%
\pgfpathcurveto{\pgfqpoint{2.001218in}{1.535392in}}{\pgfqpoint{2.005608in}{1.545991in}}{\pgfqpoint{2.005608in}{1.557042in}}%
\pgfpathcurveto{\pgfqpoint{2.005608in}{1.568092in}}{\pgfqpoint{2.001218in}{1.578691in}}{\pgfqpoint{1.993404in}{1.586504in}}%
\pgfpathcurveto{\pgfqpoint{1.985590in}{1.594318in}}{\pgfqpoint{1.974991in}{1.598708in}}{\pgfqpoint{1.963941in}{1.598708in}}%
\pgfpathcurveto{\pgfqpoint{1.952891in}{1.598708in}}{\pgfqpoint{1.942292in}{1.594318in}}{\pgfqpoint{1.934478in}{1.586504in}}%
\pgfpathcurveto{\pgfqpoint{1.926665in}{1.578691in}}{\pgfqpoint{1.922274in}{1.568092in}}{\pgfqpoint{1.922274in}{1.557042in}}%
\pgfpathcurveto{\pgfqpoint{1.922274in}{1.545991in}}{\pgfqpoint{1.926665in}{1.535392in}}{\pgfqpoint{1.934478in}{1.527579in}}%
\pgfpathcurveto{\pgfqpoint{1.942292in}{1.519765in}}{\pgfqpoint{1.952891in}{1.515375in}}{\pgfqpoint{1.963941in}{1.515375in}}%
\pgfpathlineto{\pgfqpoint{1.963941in}{1.515375in}}%
\pgfpathclose%
\pgfusepath{stroke}%
\end{pgfscope}%
\begin{pgfscope}%
\pgfpathrectangle{\pgfqpoint{0.494722in}{0.437222in}}{\pgfqpoint{6.275590in}{5.159444in}}%
\pgfusepath{clip}%
\pgfsetbuttcap%
\pgfsetroundjoin%
\pgfsetlinewidth{1.003750pt}%
\definecolor{currentstroke}{rgb}{0.827451,0.827451,0.827451}%
\pgfsetstrokecolor{currentstroke}%
\pgfsetstrokeopacity{0.800000}%
\pgfsetdash{}{0pt}%
\pgfpathmoveto{\pgfqpoint{4.067519in}{0.560512in}}%
\pgfpathcurveto{\pgfqpoint{4.078569in}{0.560512in}}{\pgfqpoint{4.089168in}{0.564902in}}{\pgfqpoint{4.096982in}{0.572715in}}%
\pgfpathcurveto{\pgfqpoint{4.104796in}{0.580529in}}{\pgfqpoint{4.109186in}{0.591128in}}{\pgfqpoint{4.109186in}{0.602178in}}%
\pgfpathcurveto{\pgfqpoint{4.109186in}{0.613228in}}{\pgfqpoint{4.104796in}{0.623827in}}{\pgfqpoint{4.096982in}{0.631641in}}%
\pgfpathcurveto{\pgfqpoint{4.089168in}{0.639455in}}{\pgfqpoint{4.078569in}{0.643845in}}{\pgfqpoint{4.067519in}{0.643845in}}%
\pgfpathcurveto{\pgfqpoint{4.056469in}{0.643845in}}{\pgfqpoint{4.045870in}{0.639455in}}{\pgfqpoint{4.038056in}{0.631641in}}%
\pgfpathcurveto{\pgfqpoint{4.030243in}{0.623827in}}{\pgfqpoint{4.025853in}{0.613228in}}{\pgfqpoint{4.025853in}{0.602178in}}%
\pgfpathcurveto{\pgfqpoint{4.025853in}{0.591128in}}{\pgfqpoint{4.030243in}{0.580529in}}{\pgfqpoint{4.038056in}{0.572715in}}%
\pgfpathcurveto{\pgfqpoint{4.045870in}{0.564902in}}{\pgfqpoint{4.056469in}{0.560512in}}{\pgfqpoint{4.067519in}{0.560512in}}%
\pgfpathlineto{\pgfqpoint{4.067519in}{0.560512in}}%
\pgfpathclose%
\pgfusepath{stroke}%
\end{pgfscope}%
\begin{pgfscope}%
\pgfpathrectangle{\pgfqpoint{0.494722in}{0.437222in}}{\pgfqpoint{6.275590in}{5.159444in}}%
\pgfusepath{clip}%
\pgfsetbuttcap%
\pgfsetroundjoin%
\pgfsetlinewidth{1.003750pt}%
\definecolor{currentstroke}{rgb}{0.827451,0.827451,0.827451}%
\pgfsetstrokecolor{currentstroke}%
\pgfsetstrokeopacity{0.800000}%
\pgfsetdash{}{0pt}%
\pgfpathmoveto{\pgfqpoint{1.786945in}{1.657846in}}%
\pgfpathcurveto{\pgfqpoint{1.797995in}{1.657846in}}{\pgfqpoint{1.808594in}{1.662237in}}{\pgfqpoint{1.816407in}{1.670050in}}%
\pgfpathcurveto{\pgfqpoint{1.824221in}{1.677864in}}{\pgfqpoint{1.828611in}{1.688463in}}{\pgfqpoint{1.828611in}{1.699513in}}%
\pgfpathcurveto{\pgfqpoint{1.828611in}{1.710563in}}{\pgfqpoint{1.824221in}{1.721162in}}{\pgfqpoint{1.816407in}{1.728976in}}%
\pgfpathcurveto{\pgfqpoint{1.808594in}{1.736789in}}{\pgfqpoint{1.797995in}{1.741180in}}{\pgfqpoint{1.786945in}{1.741180in}}%
\pgfpathcurveto{\pgfqpoint{1.775894in}{1.741180in}}{\pgfqpoint{1.765295in}{1.736789in}}{\pgfqpoint{1.757482in}{1.728976in}}%
\pgfpathcurveto{\pgfqpoint{1.749668in}{1.721162in}}{\pgfqpoint{1.745278in}{1.710563in}}{\pgfqpoint{1.745278in}{1.699513in}}%
\pgfpathcurveto{\pgfqpoint{1.745278in}{1.688463in}}{\pgfqpoint{1.749668in}{1.677864in}}{\pgfqpoint{1.757482in}{1.670050in}}%
\pgfpathcurveto{\pgfqpoint{1.765295in}{1.662237in}}{\pgfqpoint{1.775894in}{1.657846in}}{\pgfqpoint{1.786945in}{1.657846in}}%
\pgfpathlineto{\pgfqpoint{1.786945in}{1.657846in}}%
\pgfpathclose%
\pgfusepath{stroke}%
\end{pgfscope}%
\begin{pgfscope}%
\pgfpathrectangle{\pgfqpoint{0.494722in}{0.437222in}}{\pgfqpoint{6.275590in}{5.159444in}}%
\pgfusepath{clip}%
\pgfsetbuttcap%
\pgfsetroundjoin%
\pgfsetlinewidth{1.003750pt}%
\definecolor{currentstroke}{rgb}{0.827451,0.827451,0.827451}%
\pgfsetstrokecolor{currentstroke}%
\pgfsetstrokeopacity{0.800000}%
\pgfsetdash{}{0pt}%
\pgfpathmoveto{\pgfqpoint{5.185000in}{0.412713in}}%
\pgfpathcurveto{\pgfqpoint{5.196050in}{0.412713in}}{\pgfqpoint{5.206649in}{0.417103in}}{\pgfqpoint{5.214463in}{0.424917in}}%
\pgfpathcurveto{\pgfqpoint{5.222276in}{0.432730in}}{\pgfqpoint{5.226667in}{0.443330in}}{\pgfqpoint{5.226667in}{0.454380in}}%
\pgfpathcurveto{\pgfqpoint{5.226667in}{0.465430in}}{\pgfqpoint{5.222276in}{0.476029in}}{\pgfqpoint{5.214463in}{0.483842in}}%
\pgfpathcurveto{\pgfqpoint{5.206649in}{0.491656in}}{\pgfqpoint{5.196050in}{0.496046in}}{\pgfqpoint{5.185000in}{0.496046in}}%
\pgfpathcurveto{\pgfqpoint{5.173950in}{0.496046in}}{\pgfqpoint{5.163351in}{0.491656in}}{\pgfqpoint{5.155537in}{0.483842in}}%
\pgfpathcurveto{\pgfqpoint{5.147724in}{0.476029in}}{\pgfqpoint{5.143333in}{0.465430in}}{\pgfqpoint{5.143333in}{0.454380in}}%
\pgfpathcurveto{\pgfqpoint{5.143333in}{0.443330in}}{\pgfqpoint{5.147724in}{0.432730in}}{\pgfqpoint{5.155537in}{0.424917in}}%
\pgfpathcurveto{\pgfqpoint{5.163351in}{0.417103in}}{\pgfqpoint{5.173950in}{0.412713in}}{\pgfqpoint{5.185000in}{0.412713in}}%
\pgfusepath{stroke}%
\end{pgfscope}%
\begin{pgfscope}%
\pgfpathrectangle{\pgfqpoint{0.494722in}{0.437222in}}{\pgfqpoint{6.275590in}{5.159444in}}%
\pgfusepath{clip}%
\pgfsetbuttcap%
\pgfsetroundjoin%
\pgfsetlinewidth{1.003750pt}%
\definecolor{currentstroke}{rgb}{0.827451,0.827451,0.827451}%
\pgfsetstrokecolor{currentstroke}%
\pgfsetstrokeopacity{0.800000}%
\pgfsetdash{}{0pt}%
\pgfpathmoveto{\pgfqpoint{4.847728in}{0.439866in}}%
\pgfpathcurveto{\pgfqpoint{4.858778in}{0.439866in}}{\pgfqpoint{4.869377in}{0.444256in}}{\pgfqpoint{4.877190in}{0.452070in}}%
\pgfpathcurveto{\pgfqpoint{4.885004in}{0.459884in}}{\pgfqpoint{4.889394in}{0.470483in}}{\pgfqpoint{4.889394in}{0.481533in}}%
\pgfpathcurveto{\pgfqpoint{4.889394in}{0.492583in}}{\pgfqpoint{4.885004in}{0.503182in}}{\pgfqpoint{4.877190in}{0.510995in}}%
\pgfpathcurveto{\pgfqpoint{4.869377in}{0.518809in}}{\pgfqpoint{4.858778in}{0.523199in}}{\pgfqpoint{4.847728in}{0.523199in}}%
\pgfpathcurveto{\pgfqpoint{4.836678in}{0.523199in}}{\pgfqpoint{4.826078in}{0.518809in}}{\pgfqpoint{4.818265in}{0.510995in}}%
\pgfpathcurveto{\pgfqpoint{4.810451in}{0.503182in}}{\pgfqpoint{4.806061in}{0.492583in}}{\pgfqpoint{4.806061in}{0.481533in}}%
\pgfpathcurveto{\pgfqpoint{4.806061in}{0.470483in}}{\pgfqpoint{4.810451in}{0.459884in}}{\pgfqpoint{4.818265in}{0.452070in}}%
\pgfpathcurveto{\pgfqpoint{4.826078in}{0.444256in}}{\pgfqpoint{4.836678in}{0.439866in}}{\pgfqpoint{4.847728in}{0.439866in}}%
\pgfpathlineto{\pgfqpoint{4.847728in}{0.439866in}}%
\pgfpathclose%
\pgfusepath{stroke}%
\end{pgfscope}%
\begin{pgfscope}%
\pgfpathrectangle{\pgfqpoint{0.494722in}{0.437222in}}{\pgfqpoint{6.275590in}{5.159444in}}%
\pgfusepath{clip}%
\pgfsetbuttcap%
\pgfsetroundjoin%
\pgfsetlinewidth{1.003750pt}%
\definecolor{currentstroke}{rgb}{0.827451,0.827451,0.827451}%
\pgfsetstrokecolor{currentstroke}%
\pgfsetstrokeopacity{0.800000}%
\pgfsetdash{}{0pt}%
\pgfpathmoveto{\pgfqpoint{2.747698in}{1.054484in}}%
\pgfpathcurveto{\pgfqpoint{2.758748in}{1.054484in}}{\pgfqpoint{2.769347in}{1.058875in}}{\pgfqpoint{2.777161in}{1.066688in}}%
\pgfpathcurveto{\pgfqpoint{2.784974in}{1.074502in}}{\pgfqpoint{2.789365in}{1.085101in}}{\pgfqpoint{2.789365in}{1.096151in}}%
\pgfpathcurveto{\pgfqpoint{2.789365in}{1.107201in}}{\pgfqpoint{2.784974in}{1.117800in}}{\pgfqpoint{2.777161in}{1.125614in}}%
\pgfpathcurveto{\pgfqpoint{2.769347in}{1.133427in}}{\pgfqpoint{2.758748in}{1.137818in}}{\pgfqpoint{2.747698in}{1.137818in}}%
\pgfpathcurveto{\pgfqpoint{2.736648in}{1.137818in}}{\pgfqpoint{2.726049in}{1.133427in}}{\pgfqpoint{2.718235in}{1.125614in}}%
\pgfpathcurveto{\pgfqpoint{2.710422in}{1.117800in}}{\pgfqpoint{2.706031in}{1.107201in}}{\pgfqpoint{2.706031in}{1.096151in}}%
\pgfpathcurveto{\pgfqpoint{2.706031in}{1.085101in}}{\pgfqpoint{2.710422in}{1.074502in}}{\pgfqpoint{2.718235in}{1.066688in}}%
\pgfpathcurveto{\pgfqpoint{2.726049in}{1.058875in}}{\pgfqpoint{2.736648in}{1.054484in}}{\pgfqpoint{2.747698in}{1.054484in}}%
\pgfpathlineto{\pgfqpoint{2.747698in}{1.054484in}}%
\pgfpathclose%
\pgfusepath{stroke}%
\end{pgfscope}%
\begin{pgfscope}%
\pgfpathrectangle{\pgfqpoint{0.494722in}{0.437222in}}{\pgfqpoint{6.275590in}{5.159444in}}%
\pgfusepath{clip}%
\pgfsetbuttcap%
\pgfsetroundjoin%
\pgfsetlinewidth{1.003750pt}%
\definecolor{currentstroke}{rgb}{0.827451,0.827451,0.827451}%
\pgfsetstrokecolor{currentstroke}%
\pgfsetstrokeopacity{0.800000}%
\pgfsetdash{}{0pt}%
\pgfpathmoveto{\pgfqpoint{3.697856in}{0.643246in}}%
\pgfpathcurveto{\pgfqpoint{3.708907in}{0.643246in}}{\pgfqpoint{3.719506in}{0.647637in}}{\pgfqpoint{3.727319in}{0.655450in}}%
\pgfpathcurveto{\pgfqpoint{3.735133in}{0.663264in}}{\pgfqpoint{3.739523in}{0.673863in}}{\pgfqpoint{3.739523in}{0.684913in}}%
\pgfpathcurveto{\pgfqpoint{3.739523in}{0.695963in}}{\pgfqpoint{3.735133in}{0.706562in}}{\pgfqpoint{3.727319in}{0.714376in}}%
\pgfpathcurveto{\pgfqpoint{3.719506in}{0.722189in}}{\pgfqpoint{3.708907in}{0.726580in}}{\pgfqpoint{3.697856in}{0.726580in}}%
\pgfpathcurveto{\pgfqpoint{3.686806in}{0.726580in}}{\pgfqpoint{3.676207in}{0.722189in}}{\pgfqpoint{3.668394in}{0.714376in}}%
\pgfpathcurveto{\pgfqpoint{3.660580in}{0.706562in}}{\pgfqpoint{3.656190in}{0.695963in}}{\pgfqpoint{3.656190in}{0.684913in}}%
\pgfpathcurveto{\pgfqpoint{3.656190in}{0.673863in}}{\pgfqpoint{3.660580in}{0.663264in}}{\pgfqpoint{3.668394in}{0.655450in}}%
\pgfpathcurveto{\pgfqpoint{3.676207in}{0.647637in}}{\pgfqpoint{3.686806in}{0.643246in}}{\pgfqpoint{3.697856in}{0.643246in}}%
\pgfpathlineto{\pgfqpoint{3.697856in}{0.643246in}}%
\pgfpathclose%
\pgfusepath{stroke}%
\end{pgfscope}%
\begin{pgfscope}%
\pgfpathrectangle{\pgfqpoint{0.494722in}{0.437222in}}{\pgfqpoint{6.275590in}{5.159444in}}%
\pgfusepath{clip}%
\pgfsetbuttcap%
\pgfsetroundjoin%
\pgfsetlinewidth{1.003750pt}%
\definecolor{currentstroke}{rgb}{0.827451,0.827451,0.827451}%
\pgfsetstrokecolor{currentstroke}%
\pgfsetstrokeopacity{0.800000}%
\pgfsetdash{}{0pt}%
\pgfpathmoveto{\pgfqpoint{4.561262in}{0.468696in}}%
\pgfpathcurveto{\pgfqpoint{4.572312in}{0.468696in}}{\pgfqpoint{4.582911in}{0.473086in}}{\pgfqpoint{4.590724in}{0.480900in}}%
\pgfpathcurveto{\pgfqpoint{4.598538in}{0.488713in}}{\pgfqpoint{4.602928in}{0.499312in}}{\pgfqpoint{4.602928in}{0.510362in}}%
\pgfpathcurveto{\pgfqpoint{4.602928in}{0.521413in}}{\pgfqpoint{4.598538in}{0.532012in}}{\pgfqpoint{4.590724in}{0.539825in}}%
\pgfpathcurveto{\pgfqpoint{4.582911in}{0.547639in}}{\pgfqpoint{4.572312in}{0.552029in}}{\pgfqpoint{4.561262in}{0.552029in}}%
\pgfpathcurveto{\pgfqpoint{4.550212in}{0.552029in}}{\pgfqpoint{4.539613in}{0.547639in}}{\pgfqpoint{4.531799in}{0.539825in}}%
\pgfpathcurveto{\pgfqpoint{4.523985in}{0.532012in}}{\pgfqpoint{4.519595in}{0.521413in}}{\pgfqpoint{4.519595in}{0.510362in}}%
\pgfpathcurveto{\pgfqpoint{4.519595in}{0.499312in}}{\pgfqpoint{4.523985in}{0.488713in}}{\pgfqpoint{4.531799in}{0.480900in}}%
\pgfpathcurveto{\pgfqpoint{4.539613in}{0.473086in}}{\pgfqpoint{4.550212in}{0.468696in}}{\pgfqpoint{4.561262in}{0.468696in}}%
\pgfpathlineto{\pgfqpoint{4.561262in}{0.468696in}}%
\pgfpathclose%
\pgfusepath{stroke}%
\end{pgfscope}%
\begin{pgfscope}%
\pgfpathrectangle{\pgfqpoint{0.494722in}{0.437222in}}{\pgfqpoint{6.275590in}{5.159444in}}%
\pgfusepath{clip}%
\pgfsetbuttcap%
\pgfsetroundjoin%
\pgfsetlinewidth{1.003750pt}%
\definecolor{currentstroke}{rgb}{0.827451,0.827451,0.827451}%
\pgfsetstrokecolor{currentstroke}%
\pgfsetstrokeopacity{0.800000}%
\pgfsetdash{}{0pt}%
\pgfpathmoveto{\pgfqpoint{2.856780in}{0.965441in}}%
\pgfpathcurveto{\pgfqpoint{2.867831in}{0.965441in}}{\pgfqpoint{2.878430in}{0.969831in}}{\pgfqpoint{2.886243in}{0.977645in}}%
\pgfpathcurveto{\pgfqpoint{2.894057in}{0.985458in}}{\pgfqpoint{2.898447in}{0.996058in}}{\pgfqpoint{2.898447in}{1.007108in}}%
\pgfpathcurveto{\pgfqpoint{2.898447in}{1.018158in}}{\pgfqpoint{2.894057in}{1.028757in}}{\pgfqpoint{2.886243in}{1.036570in}}%
\pgfpathcurveto{\pgfqpoint{2.878430in}{1.044384in}}{\pgfqpoint{2.867831in}{1.048774in}}{\pgfqpoint{2.856780in}{1.048774in}}%
\pgfpathcurveto{\pgfqpoint{2.845730in}{1.048774in}}{\pgfqpoint{2.835131in}{1.044384in}}{\pgfqpoint{2.827318in}{1.036570in}}%
\pgfpathcurveto{\pgfqpoint{2.819504in}{1.028757in}}{\pgfqpoint{2.815114in}{1.018158in}}{\pgfqpoint{2.815114in}{1.007108in}}%
\pgfpathcurveto{\pgfqpoint{2.815114in}{0.996058in}}{\pgfqpoint{2.819504in}{0.985458in}}{\pgfqpoint{2.827318in}{0.977645in}}%
\pgfpathcurveto{\pgfqpoint{2.835131in}{0.969831in}}{\pgfqpoint{2.845730in}{0.965441in}}{\pgfqpoint{2.856780in}{0.965441in}}%
\pgfpathlineto{\pgfqpoint{2.856780in}{0.965441in}}%
\pgfpathclose%
\pgfusepath{stroke}%
\end{pgfscope}%
\begin{pgfscope}%
\pgfpathrectangle{\pgfqpoint{0.494722in}{0.437222in}}{\pgfqpoint{6.275590in}{5.159444in}}%
\pgfusepath{clip}%
\pgfsetbuttcap%
\pgfsetroundjoin%
\pgfsetlinewidth{1.003750pt}%
\definecolor{currentstroke}{rgb}{0.827451,0.827451,0.827451}%
\pgfsetstrokecolor{currentstroke}%
\pgfsetstrokeopacity{0.800000}%
\pgfsetdash{}{0pt}%
\pgfpathmoveto{\pgfqpoint{2.387393in}{1.221780in}}%
\pgfpathcurveto{\pgfqpoint{2.398443in}{1.221780in}}{\pgfqpoint{2.409042in}{1.226170in}}{\pgfqpoint{2.416855in}{1.233984in}}%
\pgfpathcurveto{\pgfqpoint{2.424669in}{1.241798in}}{\pgfqpoint{2.429059in}{1.252397in}}{\pgfqpoint{2.429059in}{1.263447in}}%
\pgfpathcurveto{\pgfqpoint{2.429059in}{1.274497in}}{\pgfqpoint{2.424669in}{1.285096in}}{\pgfqpoint{2.416855in}{1.292909in}}%
\pgfpathcurveto{\pgfqpoint{2.409042in}{1.300723in}}{\pgfqpoint{2.398443in}{1.305113in}}{\pgfqpoint{2.387393in}{1.305113in}}%
\pgfpathcurveto{\pgfqpoint{2.376343in}{1.305113in}}{\pgfqpoint{2.365743in}{1.300723in}}{\pgfqpoint{2.357930in}{1.292909in}}%
\pgfpathcurveto{\pgfqpoint{2.350116in}{1.285096in}}{\pgfqpoint{2.345726in}{1.274497in}}{\pgfqpoint{2.345726in}{1.263447in}}%
\pgfpathcurveto{\pgfqpoint{2.345726in}{1.252397in}}{\pgfqpoint{2.350116in}{1.241798in}}{\pgfqpoint{2.357930in}{1.233984in}}%
\pgfpathcurveto{\pgfqpoint{2.365743in}{1.226170in}}{\pgfqpoint{2.376343in}{1.221780in}}{\pgfqpoint{2.387393in}{1.221780in}}%
\pgfpathlineto{\pgfqpoint{2.387393in}{1.221780in}}%
\pgfpathclose%
\pgfusepath{stroke}%
\end{pgfscope}%
\begin{pgfscope}%
\pgfpathrectangle{\pgfqpoint{0.494722in}{0.437222in}}{\pgfqpoint{6.275590in}{5.159444in}}%
\pgfusepath{clip}%
\pgfsetbuttcap%
\pgfsetroundjoin%
\pgfsetlinewidth{1.003750pt}%
\definecolor{currentstroke}{rgb}{0.827451,0.827451,0.827451}%
\pgfsetstrokecolor{currentstroke}%
\pgfsetstrokeopacity{0.800000}%
\pgfsetdash{}{0pt}%
\pgfpathmoveto{\pgfqpoint{1.712263in}{1.726647in}}%
\pgfpathcurveto{\pgfqpoint{1.723313in}{1.726647in}}{\pgfqpoint{1.733912in}{1.731037in}}{\pgfqpoint{1.741726in}{1.738851in}}%
\pgfpathcurveto{\pgfqpoint{1.749539in}{1.746664in}}{\pgfqpoint{1.753929in}{1.757263in}}{\pgfqpoint{1.753929in}{1.768314in}}%
\pgfpathcurveto{\pgfqpoint{1.753929in}{1.779364in}}{\pgfqpoint{1.749539in}{1.789963in}}{\pgfqpoint{1.741726in}{1.797776in}}%
\pgfpathcurveto{\pgfqpoint{1.733912in}{1.805590in}}{\pgfqpoint{1.723313in}{1.809980in}}{\pgfqpoint{1.712263in}{1.809980in}}%
\pgfpathcurveto{\pgfqpoint{1.701213in}{1.809980in}}{\pgfqpoint{1.690614in}{1.805590in}}{\pgfqpoint{1.682800in}{1.797776in}}%
\pgfpathcurveto{\pgfqpoint{1.674986in}{1.789963in}}{\pgfqpoint{1.670596in}{1.779364in}}{\pgfqpoint{1.670596in}{1.768314in}}%
\pgfpathcurveto{\pgfqpoint{1.670596in}{1.757263in}}{\pgfqpoint{1.674986in}{1.746664in}}{\pgfqpoint{1.682800in}{1.738851in}}%
\pgfpathcurveto{\pgfqpoint{1.690614in}{1.731037in}}{\pgfqpoint{1.701213in}{1.726647in}}{\pgfqpoint{1.712263in}{1.726647in}}%
\pgfpathlineto{\pgfqpoint{1.712263in}{1.726647in}}%
\pgfpathclose%
\pgfusepath{stroke}%
\end{pgfscope}%
\begin{pgfscope}%
\pgfpathrectangle{\pgfqpoint{0.494722in}{0.437222in}}{\pgfqpoint{6.275590in}{5.159444in}}%
\pgfusepath{clip}%
\pgfsetbuttcap%
\pgfsetroundjoin%
\pgfsetlinewidth{1.003750pt}%
\definecolor{currentstroke}{rgb}{0.827451,0.827451,0.827451}%
\pgfsetstrokecolor{currentstroke}%
\pgfsetstrokeopacity{0.800000}%
\pgfsetdash{}{0pt}%
\pgfpathmoveto{\pgfqpoint{0.970880in}{2.643850in}}%
\pgfpathcurveto{\pgfqpoint{0.981930in}{2.643850in}}{\pgfqpoint{0.992529in}{2.648240in}}{\pgfqpoint{1.000343in}{2.656053in}}%
\pgfpathcurveto{\pgfqpoint{1.008156in}{2.663867in}}{\pgfqpoint{1.012547in}{2.674466in}}{\pgfqpoint{1.012547in}{2.685516in}}%
\pgfpathcurveto{\pgfqpoint{1.012547in}{2.696566in}}{\pgfqpoint{1.008156in}{2.707165in}}{\pgfqpoint{1.000343in}{2.714979in}}%
\pgfpathcurveto{\pgfqpoint{0.992529in}{2.722793in}}{\pgfqpoint{0.981930in}{2.727183in}}{\pgfqpoint{0.970880in}{2.727183in}}%
\pgfpathcurveto{\pgfqpoint{0.959830in}{2.727183in}}{\pgfqpoint{0.949231in}{2.722793in}}{\pgfqpoint{0.941417in}{2.714979in}}%
\pgfpathcurveto{\pgfqpoint{0.933604in}{2.707165in}}{\pgfqpoint{0.929213in}{2.696566in}}{\pgfqpoint{0.929213in}{2.685516in}}%
\pgfpathcurveto{\pgfqpoint{0.929213in}{2.674466in}}{\pgfqpoint{0.933604in}{2.663867in}}{\pgfqpoint{0.941417in}{2.656053in}}%
\pgfpathcurveto{\pgfqpoint{0.949231in}{2.648240in}}{\pgfqpoint{0.959830in}{2.643850in}}{\pgfqpoint{0.970880in}{2.643850in}}%
\pgfpathlineto{\pgfqpoint{0.970880in}{2.643850in}}%
\pgfpathclose%
\pgfusepath{stroke}%
\end{pgfscope}%
\begin{pgfscope}%
\pgfpathrectangle{\pgfqpoint{0.494722in}{0.437222in}}{\pgfqpoint{6.275590in}{5.159444in}}%
\pgfusepath{clip}%
\pgfsetbuttcap%
\pgfsetroundjoin%
\pgfsetlinewidth{1.003750pt}%
\definecolor{currentstroke}{rgb}{0.827451,0.827451,0.827451}%
\pgfsetstrokecolor{currentstroke}%
\pgfsetstrokeopacity{0.800000}%
\pgfsetdash{}{0pt}%
\pgfpathmoveto{\pgfqpoint{1.852529in}{1.617217in}}%
\pgfpathcurveto{\pgfqpoint{1.863579in}{1.617217in}}{\pgfqpoint{1.874178in}{1.621608in}}{\pgfqpoint{1.881991in}{1.629421in}}%
\pgfpathcurveto{\pgfqpoint{1.889805in}{1.637235in}}{\pgfqpoint{1.894195in}{1.647834in}}{\pgfqpoint{1.894195in}{1.658884in}}%
\pgfpathcurveto{\pgfqpoint{1.894195in}{1.669934in}}{\pgfqpoint{1.889805in}{1.680533in}}{\pgfqpoint{1.881991in}{1.688347in}}%
\pgfpathcurveto{\pgfqpoint{1.874178in}{1.696160in}}{\pgfqpoint{1.863579in}{1.700551in}}{\pgfqpoint{1.852529in}{1.700551in}}%
\pgfpathcurveto{\pgfqpoint{1.841479in}{1.700551in}}{\pgfqpoint{1.830880in}{1.696160in}}{\pgfqpoint{1.823066in}{1.688347in}}%
\pgfpathcurveto{\pgfqpoint{1.815252in}{1.680533in}}{\pgfqpoint{1.810862in}{1.669934in}}{\pgfqpoint{1.810862in}{1.658884in}}%
\pgfpathcurveto{\pgfqpoint{1.810862in}{1.647834in}}{\pgfqpoint{1.815252in}{1.637235in}}{\pgfqpoint{1.823066in}{1.629421in}}%
\pgfpathcurveto{\pgfqpoint{1.830880in}{1.621608in}}{\pgfqpoint{1.841479in}{1.617217in}}{\pgfqpoint{1.852529in}{1.617217in}}%
\pgfpathlineto{\pgfqpoint{1.852529in}{1.617217in}}%
\pgfpathclose%
\pgfusepath{stroke}%
\end{pgfscope}%
\begin{pgfscope}%
\pgfpathrectangle{\pgfqpoint{0.494722in}{0.437222in}}{\pgfqpoint{6.275590in}{5.159444in}}%
\pgfusepath{clip}%
\pgfsetbuttcap%
\pgfsetroundjoin%
\pgfsetlinewidth{1.003750pt}%
\definecolor{currentstroke}{rgb}{0.827451,0.827451,0.827451}%
\pgfsetstrokecolor{currentstroke}%
\pgfsetstrokeopacity{0.800000}%
\pgfsetdash{}{0pt}%
\pgfpathmoveto{\pgfqpoint{4.706867in}{0.451120in}}%
\pgfpathcurveto{\pgfqpoint{4.717917in}{0.451120in}}{\pgfqpoint{4.728516in}{0.455510in}}{\pgfqpoint{4.736329in}{0.463324in}}%
\pgfpathcurveto{\pgfqpoint{4.744143in}{0.471138in}}{\pgfqpoint{4.748533in}{0.481737in}}{\pgfqpoint{4.748533in}{0.492787in}}%
\pgfpathcurveto{\pgfqpoint{4.748533in}{0.503837in}}{\pgfqpoint{4.744143in}{0.514436in}}{\pgfqpoint{4.736329in}{0.522250in}}%
\pgfpathcurveto{\pgfqpoint{4.728516in}{0.530063in}}{\pgfqpoint{4.717917in}{0.534453in}}{\pgfqpoint{4.706867in}{0.534453in}}%
\pgfpathcurveto{\pgfqpoint{4.695817in}{0.534453in}}{\pgfqpoint{4.685218in}{0.530063in}}{\pgfqpoint{4.677404in}{0.522250in}}%
\pgfpathcurveto{\pgfqpoint{4.669590in}{0.514436in}}{\pgfqpoint{4.665200in}{0.503837in}}{\pgfqpoint{4.665200in}{0.492787in}}%
\pgfpathcurveto{\pgfqpoint{4.665200in}{0.481737in}}{\pgfqpoint{4.669590in}{0.471138in}}{\pgfqpoint{4.677404in}{0.463324in}}%
\pgfpathcurveto{\pgfqpoint{4.685218in}{0.455510in}}{\pgfqpoint{4.695817in}{0.451120in}}{\pgfqpoint{4.706867in}{0.451120in}}%
\pgfpathlineto{\pgfqpoint{4.706867in}{0.451120in}}%
\pgfpathclose%
\pgfusepath{stroke}%
\end{pgfscope}%
\begin{pgfscope}%
\pgfpathrectangle{\pgfqpoint{0.494722in}{0.437222in}}{\pgfqpoint{6.275590in}{5.159444in}}%
\pgfusepath{clip}%
\pgfsetbuttcap%
\pgfsetroundjoin%
\pgfsetlinewidth{1.003750pt}%
\definecolor{currentstroke}{rgb}{0.827451,0.827451,0.827451}%
\pgfsetstrokecolor{currentstroke}%
\pgfsetstrokeopacity{0.800000}%
\pgfsetdash{}{0pt}%
\pgfpathmoveto{\pgfqpoint{2.658320in}{1.113694in}}%
\pgfpathcurveto{\pgfqpoint{2.669370in}{1.113694in}}{\pgfqpoint{2.679969in}{1.118084in}}{\pgfqpoint{2.687783in}{1.125898in}}%
\pgfpathcurveto{\pgfqpoint{2.695596in}{1.133711in}}{\pgfqpoint{2.699987in}{1.144310in}}{\pgfqpoint{2.699987in}{1.155361in}}%
\pgfpathcurveto{\pgfqpoint{2.699987in}{1.166411in}}{\pgfqpoint{2.695596in}{1.177010in}}{\pgfqpoint{2.687783in}{1.184823in}}%
\pgfpathcurveto{\pgfqpoint{2.679969in}{1.192637in}}{\pgfqpoint{2.669370in}{1.197027in}}{\pgfqpoint{2.658320in}{1.197027in}}%
\pgfpathcurveto{\pgfqpoint{2.647270in}{1.197027in}}{\pgfqpoint{2.636671in}{1.192637in}}{\pgfqpoint{2.628857in}{1.184823in}}%
\pgfpathcurveto{\pgfqpoint{2.621044in}{1.177010in}}{\pgfqpoint{2.616653in}{1.166411in}}{\pgfqpoint{2.616653in}{1.155361in}}%
\pgfpathcurveto{\pgfqpoint{2.616653in}{1.144310in}}{\pgfqpoint{2.621044in}{1.133711in}}{\pgfqpoint{2.628857in}{1.125898in}}%
\pgfpathcurveto{\pgfqpoint{2.636671in}{1.118084in}}{\pgfqpoint{2.647270in}{1.113694in}}{\pgfqpoint{2.658320in}{1.113694in}}%
\pgfpathlineto{\pgfqpoint{2.658320in}{1.113694in}}%
\pgfpathclose%
\pgfusepath{stroke}%
\end{pgfscope}%
\begin{pgfscope}%
\pgfpathrectangle{\pgfqpoint{0.494722in}{0.437222in}}{\pgfqpoint{6.275590in}{5.159444in}}%
\pgfusepath{clip}%
\pgfsetbuttcap%
\pgfsetroundjoin%
\pgfsetlinewidth{1.003750pt}%
\definecolor{currentstroke}{rgb}{0.827451,0.827451,0.827451}%
\pgfsetstrokecolor{currentstroke}%
\pgfsetstrokeopacity{0.800000}%
\pgfsetdash{}{0pt}%
\pgfpathmoveto{\pgfqpoint{3.558031in}{0.690537in}}%
\pgfpathcurveto{\pgfqpoint{3.569081in}{0.690537in}}{\pgfqpoint{3.579680in}{0.694927in}}{\pgfqpoint{3.587494in}{0.702741in}}%
\pgfpathcurveto{\pgfqpoint{3.595308in}{0.710554in}}{\pgfqpoint{3.599698in}{0.721153in}}{\pgfqpoint{3.599698in}{0.732204in}}%
\pgfpathcurveto{\pgfqpoint{3.599698in}{0.743254in}}{\pgfqpoint{3.595308in}{0.753853in}}{\pgfqpoint{3.587494in}{0.761666in}}%
\pgfpathcurveto{\pgfqpoint{3.579680in}{0.769480in}}{\pgfqpoint{3.569081in}{0.773870in}}{\pgfqpoint{3.558031in}{0.773870in}}%
\pgfpathcurveto{\pgfqpoint{3.546981in}{0.773870in}}{\pgfqpoint{3.536382in}{0.769480in}}{\pgfqpoint{3.528568in}{0.761666in}}%
\pgfpathcurveto{\pgfqpoint{3.520755in}{0.753853in}}{\pgfqpoint{3.516364in}{0.743254in}}{\pgfqpoint{3.516364in}{0.732204in}}%
\pgfpathcurveto{\pgfqpoint{3.516364in}{0.721153in}}{\pgfqpoint{3.520755in}{0.710554in}}{\pgfqpoint{3.528568in}{0.702741in}}%
\pgfpathcurveto{\pgfqpoint{3.536382in}{0.694927in}}{\pgfqpoint{3.546981in}{0.690537in}}{\pgfqpoint{3.558031in}{0.690537in}}%
\pgfpathlineto{\pgfqpoint{3.558031in}{0.690537in}}%
\pgfpathclose%
\pgfusepath{stroke}%
\end{pgfscope}%
\begin{pgfscope}%
\pgfpathrectangle{\pgfqpoint{0.494722in}{0.437222in}}{\pgfqpoint{6.275590in}{5.159444in}}%
\pgfusepath{clip}%
\pgfsetbuttcap%
\pgfsetroundjoin%
\pgfsetlinewidth{1.003750pt}%
\definecolor{currentstroke}{rgb}{0.827451,0.827451,0.827451}%
\pgfsetstrokecolor{currentstroke}%
\pgfsetstrokeopacity{0.800000}%
\pgfsetdash{}{0pt}%
\pgfpathmoveto{\pgfqpoint{1.926347in}{1.574489in}}%
\pgfpathcurveto{\pgfqpoint{1.937397in}{1.574489in}}{\pgfqpoint{1.947996in}{1.578879in}}{\pgfqpoint{1.955809in}{1.586693in}}%
\pgfpathcurveto{\pgfqpoint{1.963623in}{1.594506in}}{\pgfqpoint{1.968013in}{1.605105in}}{\pgfqpoint{1.968013in}{1.616155in}}%
\pgfpathcurveto{\pgfqpoint{1.968013in}{1.627206in}}{\pgfqpoint{1.963623in}{1.637805in}}{\pgfqpoint{1.955809in}{1.645618in}}%
\pgfpathcurveto{\pgfqpoint{1.947996in}{1.653432in}}{\pgfqpoint{1.937397in}{1.657822in}}{\pgfqpoint{1.926347in}{1.657822in}}%
\pgfpathcurveto{\pgfqpoint{1.915296in}{1.657822in}}{\pgfqpoint{1.904697in}{1.653432in}}{\pgfqpoint{1.896884in}{1.645618in}}%
\pgfpathcurveto{\pgfqpoint{1.889070in}{1.637805in}}{\pgfqpoint{1.884680in}{1.627206in}}{\pgfqpoint{1.884680in}{1.616155in}}%
\pgfpathcurveto{\pgfqpoint{1.884680in}{1.605105in}}{\pgfqpoint{1.889070in}{1.594506in}}{\pgfqpoint{1.896884in}{1.586693in}}%
\pgfpathcurveto{\pgfqpoint{1.904697in}{1.578879in}}{\pgfqpoint{1.915296in}{1.574489in}}{\pgfqpoint{1.926347in}{1.574489in}}%
\pgfpathlineto{\pgfqpoint{1.926347in}{1.574489in}}%
\pgfpathclose%
\pgfusepath{stroke}%
\end{pgfscope}%
\begin{pgfscope}%
\pgfpathrectangle{\pgfqpoint{0.494722in}{0.437222in}}{\pgfqpoint{6.275590in}{5.159444in}}%
\pgfusepath{clip}%
\pgfsetbuttcap%
\pgfsetroundjoin%
\pgfsetlinewidth{1.003750pt}%
\definecolor{currentstroke}{rgb}{0.827451,0.827451,0.827451}%
\pgfsetstrokecolor{currentstroke}%
\pgfsetstrokeopacity{0.800000}%
\pgfsetdash{}{0pt}%
\pgfpathmoveto{\pgfqpoint{0.512966in}{4.206085in}}%
\pgfpathcurveto{\pgfqpoint{0.524016in}{4.206085in}}{\pgfqpoint{0.534615in}{4.210475in}}{\pgfqpoint{0.542429in}{4.218289in}}%
\pgfpathcurveto{\pgfqpoint{0.550243in}{4.226102in}}{\pgfqpoint{0.554633in}{4.236701in}}{\pgfqpoint{0.554633in}{4.247752in}}%
\pgfpathcurveto{\pgfqpoint{0.554633in}{4.258802in}}{\pgfqpoint{0.550243in}{4.269401in}}{\pgfqpoint{0.542429in}{4.277214in}}%
\pgfpathcurveto{\pgfqpoint{0.534615in}{4.285028in}}{\pgfqpoint{0.524016in}{4.289418in}}{\pgfqpoint{0.512966in}{4.289418in}}%
\pgfpathcurveto{\pgfqpoint{0.501916in}{4.289418in}}{\pgfqpoint{0.491317in}{4.285028in}}{\pgfqpoint{0.483503in}{4.277214in}}%
\pgfpathcurveto{\pgfqpoint{0.475690in}{4.269401in}}{\pgfqpoint{0.471299in}{4.258802in}}{\pgfqpoint{0.471299in}{4.247752in}}%
\pgfpathcurveto{\pgfqpoint{0.471299in}{4.236701in}}{\pgfqpoint{0.475690in}{4.226102in}}{\pgfqpoint{0.483503in}{4.218289in}}%
\pgfpathcurveto{\pgfqpoint{0.491317in}{4.210475in}}{\pgfqpoint{0.501916in}{4.206085in}}{\pgfqpoint{0.512966in}{4.206085in}}%
\pgfpathlineto{\pgfqpoint{0.512966in}{4.206085in}}%
\pgfpathclose%
\pgfusepath{stroke}%
\end{pgfscope}%
\begin{pgfscope}%
\pgfpathrectangle{\pgfqpoint{0.494722in}{0.437222in}}{\pgfqpoint{6.275590in}{5.159444in}}%
\pgfusepath{clip}%
\pgfsetbuttcap%
\pgfsetroundjoin%
\pgfsetlinewidth{1.003750pt}%
\definecolor{currentstroke}{rgb}{0.827451,0.827451,0.827451}%
\pgfsetstrokecolor{currentstroke}%
\pgfsetstrokeopacity{0.800000}%
\pgfsetdash{}{0pt}%
\pgfpathmoveto{\pgfqpoint{1.133504in}{2.415631in}}%
\pgfpathcurveto{\pgfqpoint{1.144554in}{2.415631in}}{\pgfqpoint{1.155153in}{2.420021in}}{\pgfqpoint{1.162967in}{2.427835in}}%
\pgfpathcurveto{\pgfqpoint{1.170780in}{2.435648in}}{\pgfqpoint{1.175170in}{2.446247in}}{\pgfqpoint{1.175170in}{2.457297in}}%
\pgfpathcurveto{\pgfqpoint{1.175170in}{2.468347in}}{\pgfqpoint{1.170780in}{2.478946in}}{\pgfqpoint{1.162967in}{2.486760in}}%
\pgfpathcurveto{\pgfqpoint{1.155153in}{2.494574in}}{\pgfqpoint{1.144554in}{2.498964in}}{\pgfqpoint{1.133504in}{2.498964in}}%
\pgfpathcurveto{\pgfqpoint{1.122454in}{2.498964in}}{\pgfqpoint{1.111855in}{2.494574in}}{\pgfqpoint{1.104041in}{2.486760in}}%
\pgfpathcurveto{\pgfqpoint{1.096227in}{2.478946in}}{\pgfqpoint{1.091837in}{2.468347in}}{\pgfqpoint{1.091837in}{2.457297in}}%
\pgfpathcurveto{\pgfqpoint{1.091837in}{2.446247in}}{\pgfqpoint{1.096227in}{2.435648in}}{\pgfqpoint{1.104041in}{2.427835in}}%
\pgfpathcurveto{\pgfqpoint{1.111855in}{2.420021in}}{\pgfqpoint{1.122454in}{2.415631in}}{\pgfqpoint{1.133504in}{2.415631in}}%
\pgfpathlineto{\pgfqpoint{1.133504in}{2.415631in}}%
\pgfpathclose%
\pgfusepath{stroke}%
\end{pgfscope}%
\begin{pgfscope}%
\pgfpathrectangle{\pgfqpoint{0.494722in}{0.437222in}}{\pgfqpoint{6.275590in}{5.159444in}}%
\pgfusepath{clip}%
\pgfsetbuttcap%
\pgfsetroundjoin%
\pgfsetlinewidth{1.003750pt}%
\definecolor{currentstroke}{rgb}{0.827451,0.827451,0.827451}%
\pgfsetstrokecolor{currentstroke}%
\pgfsetstrokeopacity{0.800000}%
\pgfsetdash{}{0pt}%
\pgfpathmoveto{\pgfqpoint{0.944388in}{2.750541in}}%
\pgfpathcurveto{\pgfqpoint{0.955438in}{2.750541in}}{\pgfqpoint{0.966037in}{2.754931in}}{\pgfqpoint{0.973850in}{2.762745in}}%
\pgfpathcurveto{\pgfqpoint{0.981664in}{2.770558in}}{\pgfqpoint{0.986054in}{2.781157in}}{\pgfqpoint{0.986054in}{2.792208in}}%
\pgfpathcurveto{\pgfqpoint{0.986054in}{2.803258in}}{\pgfqpoint{0.981664in}{2.813857in}}{\pgfqpoint{0.973850in}{2.821670in}}%
\pgfpathcurveto{\pgfqpoint{0.966037in}{2.829484in}}{\pgfqpoint{0.955438in}{2.833874in}}{\pgfqpoint{0.944388in}{2.833874in}}%
\pgfpathcurveto{\pgfqpoint{0.933337in}{2.833874in}}{\pgfqpoint{0.922738in}{2.829484in}}{\pgfqpoint{0.914925in}{2.821670in}}%
\pgfpathcurveto{\pgfqpoint{0.907111in}{2.813857in}}{\pgfqpoint{0.902721in}{2.803258in}}{\pgfqpoint{0.902721in}{2.792208in}}%
\pgfpathcurveto{\pgfqpoint{0.902721in}{2.781157in}}{\pgfqpoint{0.907111in}{2.770558in}}{\pgfqpoint{0.914925in}{2.762745in}}%
\pgfpathcurveto{\pgfqpoint{0.922738in}{2.754931in}}{\pgfqpoint{0.933337in}{2.750541in}}{\pgfqpoint{0.944388in}{2.750541in}}%
\pgfpathlineto{\pgfqpoint{0.944388in}{2.750541in}}%
\pgfpathclose%
\pgfusepath{stroke}%
\end{pgfscope}%
\begin{pgfscope}%
\pgfpathrectangle{\pgfqpoint{0.494722in}{0.437222in}}{\pgfqpoint{6.275590in}{5.159444in}}%
\pgfusepath{clip}%
\pgfsetbuttcap%
\pgfsetroundjoin%
\pgfsetlinewidth{1.003750pt}%
\definecolor{currentstroke}{rgb}{0.827451,0.827451,0.827451}%
\pgfsetstrokecolor{currentstroke}%
\pgfsetstrokeopacity{0.800000}%
\pgfsetdash{}{0pt}%
\pgfpathmoveto{\pgfqpoint{2.308193in}{1.269247in}}%
\pgfpathcurveto{\pgfqpoint{2.319243in}{1.269247in}}{\pgfqpoint{2.329842in}{1.273637in}}{\pgfqpoint{2.337656in}{1.281451in}}%
\pgfpathcurveto{\pgfqpoint{2.345469in}{1.289264in}}{\pgfqpoint{2.349860in}{1.299864in}}{\pgfqpoint{2.349860in}{1.310914in}}%
\pgfpathcurveto{\pgfqpoint{2.349860in}{1.321964in}}{\pgfqpoint{2.345469in}{1.332563in}}{\pgfqpoint{2.337656in}{1.340376in}}%
\pgfpathcurveto{\pgfqpoint{2.329842in}{1.348190in}}{\pgfqpoint{2.319243in}{1.352580in}}{\pgfqpoint{2.308193in}{1.352580in}}%
\pgfpathcurveto{\pgfqpoint{2.297143in}{1.352580in}}{\pgfqpoint{2.286544in}{1.348190in}}{\pgfqpoint{2.278730in}{1.340376in}}%
\pgfpathcurveto{\pgfqpoint{2.270917in}{1.332563in}}{\pgfqpoint{2.266526in}{1.321964in}}{\pgfqpoint{2.266526in}{1.310914in}}%
\pgfpathcurveto{\pgfqpoint{2.266526in}{1.299864in}}{\pgfqpoint{2.270917in}{1.289264in}}{\pgfqpoint{2.278730in}{1.281451in}}%
\pgfpathcurveto{\pgfqpoint{2.286544in}{1.273637in}}{\pgfqpoint{2.297143in}{1.269247in}}{\pgfqpoint{2.308193in}{1.269247in}}%
\pgfpathlineto{\pgfqpoint{2.308193in}{1.269247in}}%
\pgfpathclose%
\pgfusepath{stroke}%
\end{pgfscope}%
\begin{pgfscope}%
\pgfpathrectangle{\pgfqpoint{0.494722in}{0.437222in}}{\pgfqpoint{6.275590in}{5.159444in}}%
\pgfusepath{clip}%
\pgfsetbuttcap%
\pgfsetroundjoin%
\pgfsetlinewidth{1.003750pt}%
\definecolor{currentstroke}{rgb}{0.827451,0.827451,0.827451}%
\pgfsetstrokecolor{currentstroke}%
\pgfsetstrokeopacity{0.800000}%
\pgfsetdash{}{0pt}%
\pgfpathmoveto{\pgfqpoint{5.032325in}{0.429761in}}%
\pgfpathcurveto{\pgfqpoint{5.043375in}{0.429761in}}{\pgfqpoint{5.053974in}{0.434151in}}{\pgfqpoint{5.061788in}{0.441965in}}%
\pgfpathcurveto{\pgfqpoint{5.069601in}{0.449778in}}{\pgfqpoint{5.073992in}{0.460377in}}{\pgfqpoint{5.073992in}{0.471427in}}%
\pgfpathcurveto{\pgfqpoint{5.073992in}{0.482477in}}{\pgfqpoint{5.069601in}{0.493076in}}{\pgfqpoint{5.061788in}{0.500890in}}%
\pgfpathcurveto{\pgfqpoint{5.053974in}{0.508704in}}{\pgfqpoint{5.043375in}{0.513094in}}{\pgfqpoint{5.032325in}{0.513094in}}%
\pgfpathcurveto{\pgfqpoint{5.021275in}{0.513094in}}{\pgfqpoint{5.010676in}{0.508704in}}{\pgfqpoint{5.002862in}{0.500890in}}%
\pgfpathcurveto{\pgfqpoint{4.995049in}{0.493076in}}{\pgfqpoint{4.990658in}{0.482477in}}{\pgfqpoint{4.990658in}{0.471427in}}%
\pgfpathcurveto{\pgfqpoint{4.990658in}{0.460377in}}{\pgfqpoint{4.995049in}{0.449778in}}{\pgfqpoint{5.002862in}{0.441965in}}%
\pgfpathcurveto{\pgfqpoint{5.010676in}{0.434151in}}{\pgfqpoint{5.021275in}{0.429761in}}{\pgfqpoint{5.032325in}{0.429761in}}%
\pgfpathlineto{\pgfqpoint{5.032325in}{0.429761in}}%
\pgfpathclose%
\pgfusepath{stroke}%
\end{pgfscope}%
\begin{pgfscope}%
\pgfpathrectangle{\pgfqpoint{0.494722in}{0.437222in}}{\pgfqpoint{6.275590in}{5.159444in}}%
\pgfusepath{clip}%
\pgfsetbuttcap%
\pgfsetroundjoin%
\pgfsetlinewidth{1.003750pt}%
\definecolor{currentstroke}{rgb}{0.827451,0.827451,0.827451}%
\pgfsetstrokecolor{currentstroke}%
\pgfsetstrokeopacity{0.800000}%
\pgfsetdash{}{0pt}%
\pgfpathmoveto{\pgfqpoint{2.059504in}{1.452198in}}%
\pgfpathcurveto{\pgfqpoint{2.070554in}{1.452198in}}{\pgfqpoint{2.081154in}{1.456588in}}{\pgfqpoint{2.088967in}{1.464402in}}%
\pgfpathcurveto{\pgfqpoint{2.096781in}{1.472216in}}{\pgfqpoint{2.101171in}{1.482815in}}{\pgfqpoint{2.101171in}{1.493865in}}%
\pgfpathcurveto{\pgfqpoint{2.101171in}{1.504915in}}{\pgfqpoint{2.096781in}{1.515514in}}{\pgfqpoint{2.088967in}{1.523328in}}%
\pgfpathcurveto{\pgfqpoint{2.081154in}{1.531141in}}{\pgfqpoint{2.070554in}{1.535531in}}{\pgfqpoint{2.059504in}{1.535531in}}%
\pgfpathcurveto{\pgfqpoint{2.048454in}{1.535531in}}{\pgfqpoint{2.037855in}{1.531141in}}{\pgfqpoint{2.030042in}{1.523328in}}%
\pgfpathcurveto{\pgfqpoint{2.022228in}{1.515514in}}{\pgfqpoint{2.017838in}{1.504915in}}{\pgfqpoint{2.017838in}{1.493865in}}%
\pgfpathcurveto{\pgfqpoint{2.017838in}{1.482815in}}{\pgfqpoint{2.022228in}{1.472216in}}{\pgfqpoint{2.030042in}{1.464402in}}%
\pgfpathcurveto{\pgfqpoint{2.037855in}{1.456588in}}{\pgfqpoint{2.048454in}{1.452198in}}{\pgfqpoint{2.059504in}{1.452198in}}%
\pgfpathlineto{\pgfqpoint{2.059504in}{1.452198in}}%
\pgfpathclose%
\pgfusepath{stroke}%
\end{pgfscope}%
\begin{pgfscope}%
\pgfpathrectangle{\pgfqpoint{0.494722in}{0.437222in}}{\pgfqpoint{6.275590in}{5.159444in}}%
\pgfusepath{clip}%
\pgfsetbuttcap%
\pgfsetroundjoin%
\pgfsetlinewidth{1.003750pt}%
\definecolor{currentstroke}{rgb}{0.827451,0.827451,0.827451}%
\pgfsetstrokecolor{currentstroke}%
\pgfsetstrokeopacity{0.800000}%
\pgfsetdash{}{0pt}%
\pgfpathmoveto{\pgfqpoint{1.308632in}{2.175867in}}%
\pgfpathcurveto{\pgfqpoint{1.319682in}{2.175867in}}{\pgfqpoint{1.330282in}{2.180258in}}{\pgfqpoint{1.338095in}{2.188071in}}%
\pgfpathcurveto{\pgfqpoint{1.345909in}{2.195885in}}{\pgfqpoint{1.350299in}{2.206484in}}{\pgfqpoint{1.350299in}{2.217534in}}%
\pgfpathcurveto{\pgfqpoint{1.350299in}{2.228584in}}{\pgfqpoint{1.345909in}{2.239183in}}{\pgfqpoint{1.338095in}{2.246997in}}%
\pgfpathcurveto{\pgfqpoint{1.330282in}{2.254810in}}{\pgfqpoint{1.319682in}{2.259201in}}{\pgfqpoint{1.308632in}{2.259201in}}%
\pgfpathcurveto{\pgfqpoint{1.297582in}{2.259201in}}{\pgfqpoint{1.286983in}{2.254810in}}{\pgfqpoint{1.279170in}{2.246997in}}%
\pgfpathcurveto{\pgfqpoint{1.271356in}{2.239183in}}{\pgfqpoint{1.266966in}{2.228584in}}{\pgfqpoint{1.266966in}{2.217534in}}%
\pgfpathcurveto{\pgfqpoint{1.266966in}{2.206484in}}{\pgfqpoint{1.271356in}{2.195885in}}{\pgfqpoint{1.279170in}{2.188071in}}%
\pgfpathcurveto{\pgfqpoint{1.286983in}{2.180258in}}{\pgfqpoint{1.297582in}{2.175867in}}{\pgfqpoint{1.308632in}{2.175867in}}%
\pgfpathlineto{\pgfqpoint{1.308632in}{2.175867in}}%
\pgfpathclose%
\pgfusepath{stroke}%
\end{pgfscope}%
\begin{pgfscope}%
\pgfpathrectangle{\pgfqpoint{0.494722in}{0.437222in}}{\pgfqpoint{6.275590in}{5.159444in}}%
\pgfusepath{clip}%
\pgfsetbuttcap%
\pgfsetroundjoin%
\pgfsetlinewidth{1.003750pt}%
\definecolor{currentstroke}{rgb}{0.827451,0.827451,0.827451}%
\pgfsetstrokecolor{currentstroke}%
\pgfsetstrokeopacity{0.800000}%
\pgfsetdash{}{0pt}%
\pgfpathmoveto{\pgfqpoint{1.038730in}{2.546051in}}%
\pgfpathcurveto{\pgfqpoint{1.049780in}{2.546051in}}{\pgfqpoint{1.060379in}{2.550441in}}{\pgfqpoint{1.068193in}{2.558254in}}%
\pgfpathcurveto{\pgfqpoint{1.076007in}{2.566068in}}{\pgfqpoint{1.080397in}{2.576667in}}{\pgfqpoint{1.080397in}{2.587717in}}%
\pgfpathcurveto{\pgfqpoint{1.080397in}{2.598767in}}{\pgfqpoint{1.076007in}{2.609366in}}{\pgfqpoint{1.068193in}{2.617180in}}%
\pgfpathcurveto{\pgfqpoint{1.060379in}{2.624994in}}{\pgfqpoint{1.049780in}{2.629384in}}{\pgfqpoint{1.038730in}{2.629384in}}%
\pgfpathcurveto{\pgfqpoint{1.027680in}{2.629384in}}{\pgfqpoint{1.017081in}{2.624994in}}{\pgfqpoint{1.009268in}{2.617180in}}%
\pgfpathcurveto{\pgfqpoint{1.001454in}{2.609366in}}{\pgfqpoint{0.997064in}{2.598767in}}{\pgfqpoint{0.997064in}{2.587717in}}%
\pgfpathcurveto{\pgfqpoint{0.997064in}{2.576667in}}{\pgfqpoint{1.001454in}{2.566068in}}{\pgfqpoint{1.009268in}{2.558254in}}%
\pgfpathcurveto{\pgfqpoint{1.017081in}{2.550441in}}{\pgfqpoint{1.027680in}{2.546051in}}{\pgfqpoint{1.038730in}{2.546051in}}%
\pgfpathlineto{\pgfqpoint{1.038730in}{2.546051in}}%
\pgfpathclose%
\pgfusepath{stroke}%
\end{pgfscope}%
\begin{pgfscope}%
\pgfpathrectangle{\pgfqpoint{0.494722in}{0.437222in}}{\pgfqpoint{6.275590in}{5.159444in}}%
\pgfusepath{clip}%
\pgfsetbuttcap%
\pgfsetroundjoin%
\pgfsetlinewidth{1.003750pt}%
\definecolor{currentstroke}{rgb}{0.827451,0.827451,0.827451}%
\pgfsetstrokecolor{currentstroke}%
\pgfsetstrokeopacity{0.800000}%
\pgfsetdash{}{0pt}%
\pgfpathmoveto{\pgfqpoint{4.176161in}{0.527713in}}%
\pgfpathcurveto{\pgfqpoint{4.187212in}{0.527713in}}{\pgfqpoint{4.197811in}{0.532104in}}{\pgfqpoint{4.205624in}{0.539917in}}%
\pgfpathcurveto{\pgfqpoint{4.213438in}{0.547731in}}{\pgfqpoint{4.217828in}{0.558330in}}{\pgfqpoint{4.217828in}{0.569380in}}%
\pgfpathcurveto{\pgfqpoint{4.217828in}{0.580430in}}{\pgfqpoint{4.213438in}{0.591029in}}{\pgfqpoint{4.205624in}{0.598843in}}%
\pgfpathcurveto{\pgfqpoint{4.197811in}{0.606656in}}{\pgfqpoint{4.187212in}{0.611047in}}{\pgfqpoint{4.176161in}{0.611047in}}%
\pgfpathcurveto{\pgfqpoint{4.165111in}{0.611047in}}{\pgfqpoint{4.154512in}{0.606656in}}{\pgfqpoint{4.146699in}{0.598843in}}%
\pgfpathcurveto{\pgfqpoint{4.138885in}{0.591029in}}{\pgfqpoint{4.134495in}{0.580430in}}{\pgfqpoint{4.134495in}{0.569380in}}%
\pgfpathcurveto{\pgfqpoint{4.134495in}{0.558330in}}{\pgfqpoint{4.138885in}{0.547731in}}{\pgfqpoint{4.146699in}{0.539917in}}%
\pgfpathcurveto{\pgfqpoint{4.154512in}{0.532104in}}{\pgfqpoint{4.165111in}{0.527713in}}{\pgfqpoint{4.176161in}{0.527713in}}%
\pgfpathlineto{\pgfqpoint{4.176161in}{0.527713in}}%
\pgfpathclose%
\pgfusepath{stroke}%
\end{pgfscope}%
\begin{pgfscope}%
\pgfpathrectangle{\pgfqpoint{0.494722in}{0.437222in}}{\pgfqpoint{6.275590in}{5.159444in}}%
\pgfusepath{clip}%
\pgfsetbuttcap%
\pgfsetroundjoin%
\pgfsetlinewidth{1.003750pt}%
\definecolor{currentstroke}{rgb}{0.827451,0.827451,0.827451}%
\pgfsetstrokecolor{currentstroke}%
\pgfsetstrokeopacity{0.800000}%
\pgfsetdash{}{0pt}%
\pgfpathmoveto{\pgfqpoint{1.671516in}{1.760843in}}%
\pgfpathcurveto{\pgfqpoint{1.682566in}{1.760843in}}{\pgfqpoint{1.693165in}{1.765233in}}{\pgfqpoint{1.700979in}{1.773046in}}%
\pgfpathcurveto{\pgfqpoint{1.708792in}{1.780860in}}{\pgfqpoint{1.713183in}{1.791459in}}{\pgfqpoint{1.713183in}{1.802509in}}%
\pgfpathcurveto{\pgfqpoint{1.713183in}{1.813559in}}{\pgfqpoint{1.708792in}{1.824158in}}{\pgfqpoint{1.700979in}{1.831972in}}%
\pgfpathcurveto{\pgfqpoint{1.693165in}{1.839786in}}{\pgfqpoint{1.682566in}{1.844176in}}{\pgfqpoint{1.671516in}{1.844176in}}%
\pgfpathcurveto{\pgfqpoint{1.660466in}{1.844176in}}{\pgfqpoint{1.649867in}{1.839786in}}{\pgfqpoint{1.642053in}{1.831972in}}%
\pgfpathcurveto{\pgfqpoint{1.634240in}{1.824158in}}{\pgfqpoint{1.629849in}{1.813559in}}{\pgfqpoint{1.629849in}{1.802509in}}%
\pgfpathcurveto{\pgfqpoint{1.629849in}{1.791459in}}{\pgfqpoint{1.634240in}{1.780860in}}{\pgfqpoint{1.642053in}{1.773046in}}%
\pgfpathcurveto{\pgfqpoint{1.649867in}{1.765233in}}{\pgfqpoint{1.660466in}{1.760843in}}{\pgfqpoint{1.671516in}{1.760843in}}%
\pgfpathlineto{\pgfqpoint{1.671516in}{1.760843in}}%
\pgfpathclose%
\pgfusepath{stroke}%
\end{pgfscope}%
\begin{pgfscope}%
\pgfpathrectangle{\pgfqpoint{0.494722in}{0.437222in}}{\pgfqpoint{6.275590in}{5.159444in}}%
\pgfusepath{clip}%
\pgfsetbuttcap%
\pgfsetroundjoin%
\pgfsetlinewidth{1.003750pt}%
\definecolor{currentstroke}{rgb}{0.827451,0.827451,0.827451}%
\pgfsetstrokecolor{currentstroke}%
\pgfsetstrokeopacity{0.800000}%
\pgfsetdash{}{0pt}%
\pgfpathmoveto{\pgfqpoint{3.055005in}{0.879521in}}%
\pgfpathcurveto{\pgfqpoint{3.066055in}{0.879521in}}{\pgfqpoint{3.076654in}{0.883911in}}{\pgfqpoint{3.084467in}{0.891725in}}%
\pgfpathcurveto{\pgfqpoint{3.092281in}{0.899538in}}{\pgfqpoint{3.096671in}{0.910137in}}{\pgfqpoint{3.096671in}{0.921187in}}%
\pgfpathcurveto{\pgfqpoint{3.096671in}{0.932237in}}{\pgfqpoint{3.092281in}{0.942836in}}{\pgfqpoint{3.084467in}{0.950650in}}%
\pgfpathcurveto{\pgfqpoint{3.076654in}{0.958464in}}{\pgfqpoint{3.066055in}{0.962854in}}{\pgfqpoint{3.055005in}{0.962854in}}%
\pgfpathcurveto{\pgfqpoint{3.043955in}{0.962854in}}{\pgfqpoint{3.033356in}{0.958464in}}{\pgfqpoint{3.025542in}{0.950650in}}%
\pgfpathcurveto{\pgfqpoint{3.017728in}{0.942836in}}{\pgfqpoint{3.013338in}{0.932237in}}{\pgfqpoint{3.013338in}{0.921187in}}%
\pgfpathcurveto{\pgfqpoint{3.013338in}{0.910137in}}{\pgfqpoint{3.017728in}{0.899538in}}{\pgfqpoint{3.025542in}{0.891725in}}%
\pgfpathcurveto{\pgfqpoint{3.033356in}{0.883911in}}{\pgfqpoint{3.043955in}{0.879521in}}{\pgfqpoint{3.055005in}{0.879521in}}%
\pgfpathlineto{\pgfqpoint{3.055005in}{0.879521in}}%
\pgfpathclose%
\pgfusepath{stroke}%
\end{pgfscope}%
\begin{pgfscope}%
\pgfpathrectangle{\pgfqpoint{0.494722in}{0.437222in}}{\pgfqpoint{6.275590in}{5.159444in}}%
\pgfusepath{clip}%
\pgfsetbuttcap%
\pgfsetroundjoin%
\pgfsetlinewidth{1.003750pt}%
\definecolor{currentstroke}{rgb}{0.827451,0.827451,0.827451}%
\pgfsetstrokecolor{currentstroke}%
\pgfsetstrokeopacity{0.800000}%
\pgfsetdash{}{0pt}%
\pgfpathmoveto{\pgfqpoint{1.161602in}{2.366586in}}%
\pgfpathcurveto{\pgfqpoint{1.172652in}{2.366586in}}{\pgfqpoint{1.183251in}{2.370976in}}{\pgfqpoint{1.191065in}{2.378790in}}%
\pgfpathcurveto{\pgfqpoint{1.198879in}{2.386603in}}{\pgfqpoint{1.203269in}{2.397202in}}{\pgfqpoint{1.203269in}{2.408252in}}%
\pgfpathcurveto{\pgfqpoint{1.203269in}{2.419303in}}{\pgfqpoint{1.198879in}{2.429902in}}{\pgfqpoint{1.191065in}{2.437715in}}%
\pgfpathcurveto{\pgfqpoint{1.183251in}{2.445529in}}{\pgfqpoint{1.172652in}{2.449919in}}{\pgfqpoint{1.161602in}{2.449919in}}%
\pgfpathcurveto{\pgfqpoint{1.150552in}{2.449919in}}{\pgfqpoint{1.139953in}{2.445529in}}{\pgfqpoint{1.132139in}{2.437715in}}%
\pgfpathcurveto{\pgfqpoint{1.124326in}{2.429902in}}{\pgfqpoint{1.119935in}{2.419303in}}{\pgfqpoint{1.119935in}{2.408252in}}%
\pgfpathcurveto{\pgfqpoint{1.119935in}{2.397202in}}{\pgfqpoint{1.124326in}{2.386603in}}{\pgfqpoint{1.132139in}{2.378790in}}%
\pgfpathcurveto{\pgfqpoint{1.139953in}{2.370976in}}{\pgfqpoint{1.150552in}{2.366586in}}{\pgfqpoint{1.161602in}{2.366586in}}%
\pgfpathlineto{\pgfqpoint{1.161602in}{2.366586in}}%
\pgfpathclose%
\pgfusepath{stroke}%
\end{pgfscope}%
\begin{pgfscope}%
\pgfpathrectangle{\pgfqpoint{0.494722in}{0.437222in}}{\pgfqpoint{6.275590in}{5.159444in}}%
\pgfusepath{clip}%
\pgfsetbuttcap%
\pgfsetroundjoin%
\pgfsetlinewidth{1.003750pt}%
\definecolor{currentstroke}{rgb}{0.827451,0.827451,0.827451}%
\pgfsetstrokecolor{currentstroke}%
\pgfsetstrokeopacity{0.800000}%
\pgfsetdash{}{0pt}%
\pgfpathmoveto{\pgfqpoint{5.373868in}{0.408915in}}%
\pgfpathcurveto{\pgfqpoint{5.384918in}{0.408915in}}{\pgfqpoint{5.395517in}{0.413306in}}{\pgfqpoint{5.403330in}{0.421119in}}%
\pgfpathcurveto{\pgfqpoint{5.411144in}{0.428933in}}{\pgfqpoint{5.415534in}{0.439532in}}{\pgfqpoint{5.415534in}{0.450582in}}%
\pgfpathcurveto{\pgfqpoint{5.415534in}{0.461632in}}{\pgfqpoint{5.411144in}{0.472231in}}{\pgfqpoint{5.403330in}{0.480045in}}%
\pgfpathcurveto{\pgfqpoint{5.395517in}{0.487858in}}{\pgfqpoint{5.384918in}{0.492249in}}{\pgfqpoint{5.373868in}{0.492249in}}%
\pgfpathcurveto{\pgfqpoint{5.362817in}{0.492249in}}{\pgfqpoint{5.352218in}{0.487858in}}{\pgfqpoint{5.344405in}{0.480045in}}%
\pgfpathcurveto{\pgfqpoint{5.336591in}{0.472231in}}{\pgfqpoint{5.332201in}{0.461632in}}{\pgfqpoint{5.332201in}{0.450582in}}%
\pgfpathcurveto{\pgfqpoint{5.332201in}{0.439532in}}{\pgfqpoint{5.336591in}{0.428933in}}{\pgfqpoint{5.344405in}{0.421119in}}%
\pgfpathcurveto{\pgfqpoint{5.352218in}{0.413306in}}{\pgfqpoint{5.362817in}{0.408915in}}{\pgfqpoint{5.373868in}{0.408915in}}%
\pgfusepath{stroke}%
\end{pgfscope}%
\begin{pgfscope}%
\pgfpathrectangle{\pgfqpoint{0.494722in}{0.437222in}}{\pgfqpoint{6.275590in}{5.159444in}}%
\pgfusepath{clip}%
\pgfsetbuttcap%
\pgfsetroundjoin%
\pgfsetlinewidth{1.003750pt}%
\definecolor{currentstroke}{rgb}{0.827451,0.827451,0.827451}%
\pgfsetstrokecolor{currentstroke}%
\pgfsetstrokeopacity{0.800000}%
\pgfsetdash{}{0pt}%
\pgfpathmoveto{\pgfqpoint{1.198501in}{2.304477in}}%
\pgfpathcurveto{\pgfqpoint{1.209551in}{2.304477in}}{\pgfqpoint{1.220150in}{2.308867in}}{\pgfqpoint{1.227963in}{2.316681in}}%
\pgfpathcurveto{\pgfqpoint{1.235777in}{2.324495in}}{\pgfqpoint{1.240167in}{2.335094in}}{\pgfqpoint{1.240167in}{2.346144in}}%
\pgfpathcurveto{\pgfqpoint{1.240167in}{2.357194in}}{\pgfqpoint{1.235777in}{2.367793in}}{\pgfqpoint{1.227963in}{2.375607in}}%
\pgfpathcurveto{\pgfqpoint{1.220150in}{2.383420in}}{\pgfqpoint{1.209551in}{2.387811in}}{\pgfqpoint{1.198501in}{2.387811in}}%
\pgfpathcurveto{\pgfqpoint{1.187451in}{2.387811in}}{\pgfqpoint{1.176852in}{2.383420in}}{\pgfqpoint{1.169038in}{2.375607in}}%
\pgfpathcurveto{\pgfqpoint{1.161224in}{2.367793in}}{\pgfqpoint{1.156834in}{2.357194in}}{\pgfqpoint{1.156834in}{2.346144in}}%
\pgfpathcurveto{\pgfqpoint{1.156834in}{2.335094in}}{\pgfqpoint{1.161224in}{2.324495in}}{\pgfqpoint{1.169038in}{2.316681in}}%
\pgfpathcurveto{\pgfqpoint{1.176852in}{2.308867in}}{\pgfqpoint{1.187451in}{2.304477in}}{\pgfqpoint{1.198501in}{2.304477in}}%
\pgfpathlineto{\pgfqpoint{1.198501in}{2.304477in}}%
\pgfpathclose%
\pgfusepath{stroke}%
\end{pgfscope}%
\begin{pgfscope}%
\pgfpathrectangle{\pgfqpoint{0.494722in}{0.437222in}}{\pgfqpoint{6.275590in}{5.159444in}}%
\pgfusepath{clip}%
\pgfsetbuttcap%
\pgfsetroundjoin%
\pgfsetlinewidth{1.003750pt}%
\definecolor{currentstroke}{rgb}{0.827451,0.827451,0.827451}%
\pgfsetstrokecolor{currentstroke}%
\pgfsetstrokeopacity{0.800000}%
\pgfsetdash{}{0pt}%
\pgfpathmoveto{\pgfqpoint{0.625304in}{3.523383in}}%
\pgfpathcurveto{\pgfqpoint{0.636354in}{3.523383in}}{\pgfqpoint{0.646953in}{3.527774in}}{\pgfqpoint{0.654767in}{3.535587in}}%
\pgfpathcurveto{\pgfqpoint{0.662581in}{3.543401in}}{\pgfqpoint{0.666971in}{3.554000in}}{\pgfqpoint{0.666971in}{3.565050in}}%
\pgfpathcurveto{\pgfqpoint{0.666971in}{3.576100in}}{\pgfqpoint{0.662581in}{3.586699in}}{\pgfqpoint{0.654767in}{3.594513in}}%
\pgfpathcurveto{\pgfqpoint{0.646953in}{3.602326in}}{\pgfqpoint{0.636354in}{3.606717in}}{\pgfqpoint{0.625304in}{3.606717in}}%
\pgfpathcurveto{\pgfqpoint{0.614254in}{3.606717in}}{\pgfqpoint{0.603655in}{3.602326in}}{\pgfqpoint{0.595841in}{3.594513in}}%
\pgfpathcurveto{\pgfqpoint{0.588028in}{3.586699in}}{\pgfqpoint{0.583638in}{3.576100in}}{\pgfqpoint{0.583638in}{3.565050in}}%
\pgfpathcurveto{\pgfqpoint{0.583638in}{3.554000in}}{\pgfqpoint{0.588028in}{3.543401in}}{\pgfqpoint{0.595841in}{3.535587in}}%
\pgfpathcurveto{\pgfqpoint{0.603655in}{3.527774in}}{\pgfqpoint{0.614254in}{3.523383in}}{\pgfqpoint{0.625304in}{3.523383in}}%
\pgfpathlineto{\pgfqpoint{0.625304in}{3.523383in}}%
\pgfpathclose%
\pgfusepath{stroke}%
\end{pgfscope}%
\begin{pgfscope}%
\pgfpathrectangle{\pgfqpoint{0.494722in}{0.437222in}}{\pgfqpoint{6.275590in}{5.159444in}}%
\pgfusepath{clip}%
\pgfsetbuttcap%
\pgfsetroundjoin%
\pgfsetlinewidth{1.003750pt}%
\definecolor{currentstroke}{rgb}{0.827451,0.827451,0.827451}%
\pgfsetstrokecolor{currentstroke}%
\pgfsetstrokeopacity{0.800000}%
\pgfsetdash{}{0pt}%
\pgfpathmoveto{\pgfqpoint{2.183447in}{1.371651in}}%
\pgfpathcurveto{\pgfqpoint{2.194497in}{1.371651in}}{\pgfqpoint{2.205096in}{1.376041in}}{\pgfqpoint{2.212910in}{1.383855in}}%
\pgfpathcurveto{\pgfqpoint{2.220724in}{1.391668in}}{\pgfqpoint{2.225114in}{1.402267in}}{\pgfqpoint{2.225114in}{1.413318in}}%
\pgfpathcurveto{\pgfqpoint{2.225114in}{1.424368in}}{\pgfqpoint{2.220724in}{1.434967in}}{\pgfqpoint{2.212910in}{1.442780in}}%
\pgfpathcurveto{\pgfqpoint{2.205096in}{1.450594in}}{\pgfqpoint{2.194497in}{1.454984in}}{\pgfqpoint{2.183447in}{1.454984in}}%
\pgfpathcurveto{\pgfqpoint{2.172397in}{1.454984in}}{\pgfqpoint{2.161798in}{1.450594in}}{\pgfqpoint{2.153984in}{1.442780in}}%
\pgfpathcurveto{\pgfqpoint{2.146171in}{1.434967in}}{\pgfqpoint{2.141781in}{1.424368in}}{\pgfqpoint{2.141781in}{1.413318in}}%
\pgfpathcurveto{\pgfqpoint{2.141781in}{1.402267in}}{\pgfqpoint{2.146171in}{1.391668in}}{\pgfqpoint{2.153984in}{1.383855in}}%
\pgfpathcurveto{\pgfqpoint{2.161798in}{1.376041in}}{\pgfqpoint{2.172397in}{1.371651in}}{\pgfqpoint{2.183447in}{1.371651in}}%
\pgfpathlineto{\pgfqpoint{2.183447in}{1.371651in}}%
\pgfpathclose%
\pgfusepath{stroke}%
\end{pgfscope}%
\begin{pgfscope}%
\pgfpathrectangle{\pgfqpoint{0.494722in}{0.437222in}}{\pgfqpoint{6.275590in}{5.159444in}}%
\pgfusepath{clip}%
\pgfsetbuttcap%
\pgfsetroundjoin%
\pgfsetlinewidth{1.003750pt}%
\definecolor{currentstroke}{rgb}{0.827451,0.827451,0.827451}%
\pgfsetstrokecolor{currentstroke}%
\pgfsetstrokeopacity{0.800000}%
\pgfsetdash{}{0pt}%
\pgfpathmoveto{\pgfqpoint{5.677571in}{0.399663in}}%
\pgfpathcurveto{\pgfqpoint{5.688621in}{0.399663in}}{\pgfqpoint{5.699220in}{0.404053in}}{\pgfqpoint{5.707034in}{0.411866in}}%
\pgfpathcurveto{\pgfqpoint{5.714847in}{0.419680in}}{\pgfqpoint{5.719238in}{0.430279in}}{\pgfqpoint{5.719238in}{0.441329in}}%
\pgfpathcurveto{\pgfqpoint{5.719238in}{0.452379in}}{\pgfqpoint{5.714847in}{0.462978in}}{\pgfqpoint{5.707034in}{0.470792in}}%
\pgfpathcurveto{\pgfqpoint{5.699220in}{0.478606in}}{\pgfqpoint{5.688621in}{0.482996in}}{\pgfqpoint{5.677571in}{0.482996in}}%
\pgfpathcurveto{\pgfqpoint{5.666521in}{0.482996in}}{\pgfqpoint{5.655922in}{0.478606in}}{\pgfqpoint{5.648108in}{0.470792in}}%
\pgfpathcurveto{\pgfqpoint{5.640294in}{0.462978in}}{\pgfqpoint{5.635904in}{0.452379in}}{\pgfqpoint{5.635904in}{0.441329in}}%
\pgfpathcurveto{\pgfqpoint{5.635904in}{0.430279in}}{\pgfqpoint{5.640294in}{0.419680in}}{\pgfqpoint{5.648108in}{0.411866in}}%
\pgfpathcurveto{\pgfqpoint{5.655922in}{0.404053in}}{\pgfqpoint{5.666521in}{0.399663in}}{\pgfqpoint{5.677571in}{0.399663in}}%
\pgfusepath{stroke}%
\end{pgfscope}%
\begin{pgfscope}%
\pgfpathrectangle{\pgfqpoint{0.494722in}{0.437222in}}{\pgfqpoint{6.275590in}{5.159444in}}%
\pgfusepath{clip}%
\pgfsetbuttcap%
\pgfsetroundjoin%
\pgfsetlinewidth{1.003750pt}%
\definecolor{currentstroke}{rgb}{0.827451,0.827451,0.827451}%
\pgfsetstrokecolor{currentstroke}%
\pgfsetstrokeopacity{0.800000}%
\pgfsetdash{}{0pt}%
\pgfpathmoveto{\pgfqpoint{3.820722in}{0.625629in}}%
\pgfpathcurveto{\pgfqpoint{3.831772in}{0.625629in}}{\pgfqpoint{3.842371in}{0.630019in}}{\pgfqpoint{3.850185in}{0.637833in}}%
\pgfpathcurveto{\pgfqpoint{3.857998in}{0.645646in}}{\pgfqpoint{3.862389in}{0.656245in}}{\pgfqpoint{3.862389in}{0.667295in}}%
\pgfpathcurveto{\pgfqpoint{3.862389in}{0.678345in}}{\pgfqpoint{3.857998in}{0.688944in}}{\pgfqpoint{3.850185in}{0.696758in}}%
\pgfpathcurveto{\pgfqpoint{3.842371in}{0.704572in}}{\pgfqpoint{3.831772in}{0.708962in}}{\pgfqpoint{3.820722in}{0.708962in}}%
\pgfpathcurveto{\pgfqpoint{3.809672in}{0.708962in}}{\pgfqpoint{3.799073in}{0.704572in}}{\pgfqpoint{3.791259in}{0.696758in}}%
\pgfpathcurveto{\pgfqpoint{3.783445in}{0.688944in}}{\pgfqpoint{3.779055in}{0.678345in}}{\pgfqpoint{3.779055in}{0.667295in}}%
\pgfpathcurveto{\pgfqpoint{3.779055in}{0.656245in}}{\pgfqpoint{3.783445in}{0.645646in}}{\pgfqpoint{3.791259in}{0.637833in}}%
\pgfpathcurveto{\pgfqpoint{3.799073in}{0.630019in}}{\pgfqpoint{3.809672in}{0.625629in}}{\pgfqpoint{3.820722in}{0.625629in}}%
\pgfpathlineto{\pgfqpoint{3.820722in}{0.625629in}}%
\pgfpathclose%
\pgfusepath{stroke}%
\end{pgfscope}%
\begin{pgfscope}%
\pgfpathrectangle{\pgfqpoint{0.494722in}{0.437222in}}{\pgfqpoint{6.275590in}{5.159444in}}%
\pgfusepath{clip}%
\pgfsetbuttcap%
\pgfsetroundjoin%
\pgfsetlinewidth{1.003750pt}%
\definecolor{currentstroke}{rgb}{0.827451,0.827451,0.827451}%
\pgfsetstrokecolor{currentstroke}%
\pgfsetstrokeopacity{0.800000}%
\pgfsetdash{}{0pt}%
\pgfpathmoveto{\pgfqpoint{5.544527in}{0.401839in}}%
\pgfpathcurveto{\pgfqpoint{5.555577in}{0.401839in}}{\pgfqpoint{5.566176in}{0.406230in}}{\pgfqpoint{5.573990in}{0.414043in}}%
\pgfpathcurveto{\pgfqpoint{5.581803in}{0.421857in}}{\pgfqpoint{5.586193in}{0.432456in}}{\pgfqpoint{5.586193in}{0.443506in}}%
\pgfpathcurveto{\pgfqpoint{5.586193in}{0.454556in}}{\pgfqpoint{5.581803in}{0.465155in}}{\pgfqpoint{5.573990in}{0.472969in}}%
\pgfpathcurveto{\pgfqpoint{5.566176in}{0.480782in}}{\pgfqpoint{5.555577in}{0.485173in}}{\pgfqpoint{5.544527in}{0.485173in}}%
\pgfpathcurveto{\pgfqpoint{5.533477in}{0.485173in}}{\pgfqpoint{5.522878in}{0.480782in}}{\pgfqpoint{5.515064in}{0.472969in}}%
\pgfpathcurveto{\pgfqpoint{5.507250in}{0.465155in}}{\pgfqpoint{5.502860in}{0.454556in}}{\pgfqpoint{5.502860in}{0.443506in}}%
\pgfpathcurveto{\pgfqpoint{5.502860in}{0.432456in}}{\pgfqpoint{5.507250in}{0.421857in}}{\pgfqpoint{5.515064in}{0.414043in}}%
\pgfpathcurveto{\pgfqpoint{5.522878in}{0.406230in}}{\pgfqpoint{5.533477in}{0.401839in}}{\pgfqpoint{5.544527in}{0.401839in}}%
\pgfusepath{stroke}%
\end{pgfscope}%
\begin{pgfscope}%
\pgfpathrectangle{\pgfqpoint{0.494722in}{0.437222in}}{\pgfqpoint{6.275590in}{5.159444in}}%
\pgfusepath{clip}%
\pgfsetbuttcap%
\pgfsetroundjoin%
\pgfsetlinewidth{1.003750pt}%
\definecolor{currentstroke}{rgb}{0.827451,0.827451,0.827451}%
\pgfsetstrokecolor{currentstroke}%
\pgfsetstrokeopacity{0.800000}%
\pgfsetdash{}{0pt}%
\pgfpathmoveto{\pgfqpoint{1.013007in}{2.610425in}}%
\pgfpathcurveto{\pgfqpoint{1.024057in}{2.610425in}}{\pgfqpoint{1.034656in}{2.614815in}}{\pgfqpoint{1.042469in}{2.622629in}}%
\pgfpathcurveto{\pgfqpoint{1.050283in}{2.630443in}}{\pgfqpoint{1.054673in}{2.641042in}}{\pgfqpoint{1.054673in}{2.652092in}}%
\pgfpathcurveto{\pgfqpoint{1.054673in}{2.663142in}}{\pgfqpoint{1.050283in}{2.673741in}}{\pgfqpoint{1.042469in}{2.681555in}}%
\pgfpathcurveto{\pgfqpoint{1.034656in}{2.689368in}}{\pgfqpoint{1.024057in}{2.693758in}}{\pgfqpoint{1.013007in}{2.693758in}}%
\pgfpathcurveto{\pgfqpoint{1.001956in}{2.693758in}}{\pgfqpoint{0.991357in}{2.689368in}}{\pgfqpoint{0.983544in}{2.681555in}}%
\pgfpathcurveto{\pgfqpoint{0.975730in}{2.673741in}}{\pgfqpoint{0.971340in}{2.663142in}}{\pgfqpoint{0.971340in}{2.652092in}}%
\pgfpathcurveto{\pgfqpoint{0.971340in}{2.641042in}}{\pgfqpoint{0.975730in}{2.630443in}}{\pgfqpoint{0.983544in}{2.622629in}}%
\pgfpathcurveto{\pgfqpoint{0.991357in}{2.614815in}}{\pgfqpoint{1.001956in}{2.610425in}}{\pgfqpoint{1.013007in}{2.610425in}}%
\pgfpathlineto{\pgfqpoint{1.013007in}{2.610425in}}%
\pgfpathclose%
\pgfusepath{stroke}%
\end{pgfscope}%
\begin{pgfscope}%
\pgfpathrectangle{\pgfqpoint{0.494722in}{0.437222in}}{\pgfqpoint{6.275590in}{5.159444in}}%
\pgfusepath{clip}%
\pgfsetbuttcap%
\pgfsetroundjoin%
\pgfsetlinewidth{1.003750pt}%
\definecolor{currentstroke}{rgb}{0.827451,0.827451,0.827451}%
\pgfsetstrokecolor{currentstroke}%
\pgfsetstrokeopacity{0.800000}%
\pgfsetdash{}{0pt}%
\pgfpathmoveto{\pgfqpoint{4.495815in}{0.487123in}}%
\pgfpathcurveto{\pgfqpoint{4.506865in}{0.487123in}}{\pgfqpoint{4.517464in}{0.491513in}}{\pgfqpoint{4.525277in}{0.499327in}}%
\pgfpathcurveto{\pgfqpoint{4.533091in}{0.507140in}}{\pgfqpoint{4.537481in}{0.517740in}}{\pgfqpoint{4.537481in}{0.528790in}}%
\pgfpathcurveto{\pgfqpoint{4.537481in}{0.539840in}}{\pgfqpoint{4.533091in}{0.550439in}}{\pgfqpoint{4.525277in}{0.558252in}}%
\pgfpathcurveto{\pgfqpoint{4.517464in}{0.566066in}}{\pgfqpoint{4.506865in}{0.570456in}}{\pgfqpoint{4.495815in}{0.570456in}}%
\pgfpathcurveto{\pgfqpoint{4.484765in}{0.570456in}}{\pgfqpoint{4.474166in}{0.566066in}}{\pgfqpoint{4.466352in}{0.558252in}}%
\pgfpathcurveto{\pgfqpoint{4.458538in}{0.550439in}}{\pgfqpoint{4.454148in}{0.539840in}}{\pgfqpoint{4.454148in}{0.528790in}}%
\pgfpathcurveto{\pgfqpoint{4.454148in}{0.517740in}}{\pgfqpoint{4.458538in}{0.507140in}}{\pgfqpoint{4.466352in}{0.499327in}}%
\pgfpathcurveto{\pgfqpoint{4.474166in}{0.491513in}}{\pgfqpoint{4.484765in}{0.487123in}}{\pgfqpoint{4.495815in}{0.487123in}}%
\pgfpathlineto{\pgfqpoint{4.495815in}{0.487123in}}%
\pgfpathclose%
\pgfusepath{stroke}%
\end{pgfscope}%
\begin{pgfscope}%
\pgfpathrectangle{\pgfqpoint{0.494722in}{0.437222in}}{\pgfqpoint{6.275590in}{5.159444in}}%
\pgfusepath{clip}%
\pgfsetbuttcap%
\pgfsetroundjoin%
\pgfsetlinewidth{1.003750pt}%
\definecolor{currentstroke}{rgb}{0.827451,0.827451,0.827451}%
\pgfsetstrokecolor{currentstroke}%
\pgfsetstrokeopacity{0.800000}%
\pgfsetdash{}{0pt}%
\pgfpathmoveto{\pgfqpoint{1.242947in}{2.239381in}}%
\pgfpathcurveto{\pgfqpoint{1.253997in}{2.239381in}}{\pgfqpoint{1.264596in}{2.243771in}}{\pgfqpoint{1.272410in}{2.251585in}}%
\pgfpathcurveto{\pgfqpoint{1.280224in}{2.259398in}}{\pgfqpoint{1.284614in}{2.269997in}}{\pgfqpoint{1.284614in}{2.281048in}}%
\pgfpathcurveto{\pgfqpoint{1.284614in}{2.292098in}}{\pgfqpoint{1.280224in}{2.302697in}}{\pgfqpoint{1.272410in}{2.310510in}}%
\pgfpathcurveto{\pgfqpoint{1.264596in}{2.318324in}}{\pgfqpoint{1.253997in}{2.322714in}}{\pgfqpoint{1.242947in}{2.322714in}}%
\pgfpathcurveto{\pgfqpoint{1.231897in}{2.322714in}}{\pgfqpoint{1.221298in}{2.318324in}}{\pgfqpoint{1.213484in}{2.310510in}}%
\pgfpathcurveto{\pgfqpoint{1.205671in}{2.302697in}}{\pgfqpoint{1.201280in}{2.292098in}}{\pgfqpoint{1.201280in}{2.281048in}}%
\pgfpathcurveto{\pgfqpoint{1.201280in}{2.269997in}}{\pgfqpoint{1.205671in}{2.259398in}}{\pgfqpoint{1.213484in}{2.251585in}}%
\pgfpathcurveto{\pgfqpoint{1.221298in}{2.243771in}}{\pgfqpoint{1.231897in}{2.239381in}}{\pgfqpoint{1.242947in}{2.239381in}}%
\pgfpathlineto{\pgfqpoint{1.242947in}{2.239381in}}%
\pgfpathclose%
\pgfusepath{stroke}%
\end{pgfscope}%
\begin{pgfscope}%
\pgfpathrectangle{\pgfqpoint{0.494722in}{0.437222in}}{\pgfqpoint{6.275590in}{5.159444in}}%
\pgfusepath{clip}%
\pgfsetbuttcap%
\pgfsetroundjoin%
\pgfsetlinewidth{1.003750pt}%
\definecolor{currentstroke}{rgb}{0.827451,0.827451,0.827451}%
\pgfsetstrokecolor{currentstroke}%
\pgfsetstrokeopacity{0.800000}%
\pgfsetdash{}{0pt}%
\pgfpathmoveto{\pgfqpoint{0.584355in}{3.716646in}}%
\pgfpathcurveto{\pgfqpoint{0.595406in}{3.716646in}}{\pgfqpoint{0.606005in}{3.721036in}}{\pgfqpoint{0.613818in}{3.728850in}}%
\pgfpathcurveto{\pgfqpoint{0.621632in}{3.736663in}}{\pgfqpoint{0.626022in}{3.747262in}}{\pgfqpoint{0.626022in}{3.758313in}}%
\pgfpathcurveto{\pgfqpoint{0.626022in}{3.769363in}}{\pgfqpoint{0.621632in}{3.779962in}}{\pgfqpoint{0.613818in}{3.787775in}}%
\pgfpathcurveto{\pgfqpoint{0.606005in}{3.795589in}}{\pgfqpoint{0.595406in}{3.799979in}}{\pgfqpoint{0.584355in}{3.799979in}}%
\pgfpathcurveto{\pgfqpoint{0.573305in}{3.799979in}}{\pgfqpoint{0.562706in}{3.795589in}}{\pgfqpoint{0.554893in}{3.787775in}}%
\pgfpathcurveto{\pgfqpoint{0.547079in}{3.779962in}}{\pgfqpoint{0.542689in}{3.769363in}}{\pgfqpoint{0.542689in}{3.758313in}}%
\pgfpathcurveto{\pgfqpoint{0.542689in}{3.747262in}}{\pgfqpoint{0.547079in}{3.736663in}}{\pgfqpoint{0.554893in}{3.728850in}}%
\pgfpathcurveto{\pgfqpoint{0.562706in}{3.721036in}}{\pgfqpoint{0.573305in}{3.716646in}}{\pgfqpoint{0.584355in}{3.716646in}}%
\pgfpathlineto{\pgfqpoint{0.584355in}{3.716646in}}%
\pgfpathclose%
\pgfusepath{stroke}%
\end{pgfscope}%
\begin{pgfscope}%
\pgfpathrectangle{\pgfqpoint{0.494722in}{0.437222in}}{\pgfqpoint{6.275590in}{5.159444in}}%
\pgfusepath{clip}%
\pgfsetbuttcap%
\pgfsetroundjoin%
\pgfsetlinewidth{1.003750pt}%
\definecolor{currentstroke}{rgb}{0.827451,0.827451,0.827451}%
\pgfsetstrokecolor{currentstroke}%
\pgfsetstrokeopacity{0.800000}%
\pgfsetdash{}{0pt}%
\pgfpathmoveto{\pgfqpoint{5.378653in}{0.404615in}}%
\pgfpathcurveto{\pgfqpoint{5.389703in}{0.404615in}}{\pgfqpoint{5.400302in}{0.409005in}}{\pgfqpoint{5.408116in}{0.416819in}}%
\pgfpathcurveto{\pgfqpoint{5.415930in}{0.424632in}}{\pgfqpoint{5.420320in}{0.435231in}}{\pgfqpoint{5.420320in}{0.446281in}}%
\pgfpathcurveto{\pgfqpoint{5.420320in}{0.457331in}}{\pgfqpoint{5.415930in}{0.467930in}}{\pgfqpoint{5.408116in}{0.475744in}}%
\pgfpathcurveto{\pgfqpoint{5.400302in}{0.483558in}}{\pgfqpoint{5.389703in}{0.487948in}}{\pgfqpoint{5.378653in}{0.487948in}}%
\pgfpathcurveto{\pgfqpoint{5.367603in}{0.487948in}}{\pgfqpoint{5.357004in}{0.483558in}}{\pgfqpoint{5.349190in}{0.475744in}}%
\pgfpathcurveto{\pgfqpoint{5.341377in}{0.467930in}}{\pgfqpoint{5.336987in}{0.457331in}}{\pgfqpoint{5.336987in}{0.446281in}}%
\pgfpathcurveto{\pgfqpoint{5.336987in}{0.435231in}}{\pgfqpoint{5.341377in}{0.424632in}}{\pgfqpoint{5.349190in}{0.416819in}}%
\pgfpathcurveto{\pgfqpoint{5.357004in}{0.409005in}}{\pgfqpoint{5.367603in}{0.404615in}}{\pgfqpoint{5.378653in}{0.404615in}}%
\pgfusepath{stroke}%
\end{pgfscope}%
\begin{pgfscope}%
\pgfpathrectangle{\pgfqpoint{0.494722in}{0.437222in}}{\pgfqpoint{6.275590in}{5.159444in}}%
\pgfusepath{clip}%
\pgfsetbuttcap%
\pgfsetroundjoin%
\pgfsetlinewidth{1.003750pt}%
\definecolor{currentstroke}{rgb}{0.827451,0.827451,0.827451}%
\pgfsetstrokecolor{currentstroke}%
\pgfsetstrokeopacity{0.800000}%
\pgfsetdash{}{0pt}%
\pgfpathmoveto{\pgfqpoint{2.117156in}{1.399260in}}%
\pgfpathcurveto{\pgfqpoint{2.128206in}{1.399260in}}{\pgfqpoint{2.138805in}{1.403650in}}{\pgfqpoint{2.146618in}{1.411464in}}%
\pgfpathcurveto{\pgfqpoint{2.154432in}{1.419277in}}{\pgfqpoint{2.158822in}{1.429876in}}{\pgfqpoint{2.158822in}{1.440927in}}%
\pgfpathcurveto{\pgfqpoint{2.158822in}{1.451977in}}{\pgfqpoint{2.154432in}{1.462576in}}{\pgfqpoint{2.146618in}{1.470389in}}%
\pgfpathcurveto{\pgfqpoint{2.138805in}{1.478203in}}{\pgfqpoint{2.128206in}{1.482593in}}{\pgfqpoint{2.117156in}{1.482593in}}%
\pgfpathcurveto{\pgfqpoint{2.106106in}{1.482593in}}{\pgfqpoint{2.095506in}{1.478203in}}{\pgfqpoint{2.087693in}{1.470389in}}%
\pgfpathcurveto{\pgfqpoint{2.079879in}{1.462576in}}{\pgfqpoint{2.075489in}{1.451977in}}{\pgfqpoint{2.075489in}{1.440927in}}%
\pgfpathcurveto{\pgfqpoint{2.075489in}{1.429876in}}{\pgfqpoint{2.079879in}{1.419277in}}{\pgfqpoint{2.087693in}{1.411464in}}%
\pgfpathcurveto{\pgfqpoint{2.095506in}{1.403650in}}{\pgfqpoint{2.106106in}{1.399260in}}{\pgfqpoint{2.117156in}{1.399260in}}%
\pgfpathlineto{\pgfqpoint{2.117156in}{1.399260in}}%
\pgfpathclose%
\pgfusepath{stroke}%
\end{pgfscope}%
\begin{pgfscope}%
\pgfpathrectangle{\pgfqpoint{0.494722in}{0.437222in}}{\pgfqpoint{6.275590in}{5.159444in}}%
\pgfusepath{clip}%
\pgfsetbuttcap%
\pgfsetroundjoin%
\pgfsetlinewidth{1.003750pt}%
\definecolor{currentstroke}{rgb}{0.827451,0.827451,0.827451}%
\pgfsetstrokecolor{currentstroke}%
\pgfsetstrokeopacity{0.800000}%
\pgfsetdash{}{0pt}%
\pgfpathmoveto{\pgfqpoint{0.575099in}{3.779944in}}%
\pgfpathcurveto{\pgfqpoint{0.586149in}{3.779944in}}{\pgfqpoint{0.596748in}{3.784334in}}{\pgfqpoint{0.604561in}{3.792148in}}%
\pgfpathcurveto{\pgfqpoint{0.612375in}{3.799961in}}{\pgfqpoint{0.616765in}{3.810560in}}{\pgfqpoint{0.616765in}{3.821610in}}%
\pgfpathcurveto{\pgfqpoint{0.616765in}{3.832661in}}{\pgfqpoint{0.612375in}{3.843260in}}{\pgfqpoint{0.604561in}{3.851073in}}%
\pgfpathcurveto{\pgfqpoint{0.596748in}{3.858887in}}{\pgfqpoint{0.586149in}{3.863277in}}{\pgfqpoint{0.575099in}{3.863277in}}%
\pgfpathcurveto{\pgfqpoint{0.564048in}{3.863277in}}{\pgfqpoint{0.553449in}{3.858887in}}{\pgfqpoint{0.545636in}{3.851073in}}%
\pgfpathcurveto{\pgfqpoint{0.537822in}{3.843260in}}{\pgfqpoint{0.533432in}{3.832661in}}{\pgfqpoint{0.533432in}{3.821610in}}%
\pgfpathcurveto{\pgfqpoint{0.533432in}{3.810560in}}{\pgfqpoint{0.537822in}{3.799961in}}{\pgfqpoint{0.545636in}{3.792148in}}%
\pgfpathcurveto{\pgfqpoint{0.553449in}{3.784334in}}{\pgfqpoint{0.564048in}{3.779944in}}{\pgfqpoint{0.575099in}{3.779944in}}%
\pgfpathlineto{\pgfqpoint{0.575099in}{3.779944in}}%
\pgfpathclose%
\pgfusepath{stroke}%
\end{pgfscope}%
\begin{pgfscope}%
\pgfpathrectangle{\pgfqpoint{0.494722in}{0.437222in}}{\pgfqpoint{6.275590in}{5.159444in}}%
\pgfusepath{clip}%
\pgfsetbuttcap%
\pgfsetroundjoin%
\pgfsetlinewidth{1.003750pt}%
\definecolor{currentstroke}{rgb}{0.827451,0.827451,0.827451}%
\pgfsetstrokecolor{currentstroke}%
\pgfsetstrokeopacity{0.800000}%
\pgfsetdash{}{0pt}%
\pgfpathmoveto{\pgfqpoint{0.611323in}{3.579035in}}%
\pgfpathcurveto{\pgfqpoint{0.622373in}{3.579035in}}{\pgfqpoint{0.632972in}{3.583425in}}{\pgfqpoint{0.640786in}{3.591238in}}%
\pgfpathcurveto{\pgfqpoint{0.648599in}{3.599052in}}{\pgfqpoint{0.652989in}{3.609651in}}{\pgfqpoint{0.652989in}{3.620701in}}%
\pgfpathcurveto{\pgfqpoint{0.652989in}{3.631751in}}{\pgfqpoint{0.648599in}{3.642350in}}{\pgfqpoint{0.640786in}{3.650164in}}%
\pgfpathcurveto{\pgfqpoint{0.632972in}{3.657978in}}{\pgfqpoint{0.622373in}{3.662368in}}{\pgfqpoint{0.611323in}{3.662368in}}%
\pgfpathcurveto{\pgfqpoint{0.600273in}{3.662368in}}{\pgfqpoint{0.589674in}{3.657978in}}{\pgfqpoint{0.581860in}{3.650164in}}%
\pgfpathcurveto{\pgfqpoint{0.574046in}{3.642350in}}{\pgfqpoint{0.569656in}{3.631751in}}{\pgfqpoint{0.569656in}{3.620701in}}%
\pgfpathcurveto{\pgfqpoint{0.569656in}{3.609651in}}{\pgfqpoint{0.574046in}{3.599052in}}{\pgfqpoint{0.581860in}{3.591238in}}%
\pgfpathcurveto{\pgfqpoint{0.589674in}{3.583425in}}{\pgfqpoint{0.600273in}{3.579035in}}{\pgfqpoint{0.611323in}{3.579035in}}%
\pgfpathlineto{\pgfqpoint{0.611323in}{3.579035in}}%
\pgfpathclose%
\pgfusepath{stroke}%
\end{pgfscope}%
\begin{pgfscope}%
\pgfpathrectangle{\pgfqpoint{0.494722in}{0.437222in}}{\pgfqpoint{6.275590in}{5.159444in}}%
\pgfusepath{clip}%
\pgfsetbuttcap%
\pgfsetroundjoin%
\pgfsetlinewidth{1.003750pt}%
\definecolor{currentstroke}{rgb}{0.827451,0.827451,0.827451}%
\pgfsetstrokecolor{currentstroke}%
\pgfsetstrokeopacity{0.800000}%
\pgfsetdash{}{0pt}%
\pgfpathmoveto{\pgfqpoint{3.295644in}{0.786846in}}%
\pgfpathcurveto{\pgfqpoint{3.306694in}{0.786846in}}{\pgfqpoint{3.317293in}{0.791236in}}{\pgfqpoint{3.325106in}{0.799050in}}%
\pgfpathcurveto{\pgfqpoint{3.332920in}{0.806864in}}{\pgfqpoint{3.337310in}{0.817463in}}{\pgfqpoint{3.337310in}{0.828513in}}%
\pgfpathcurveto{\pgfqpoint{3.337310in}{0.839563in}}{\pgfqpoint{3.332920in}{0.850162in}}{\pgfqpoint{3.325106in}{0.857976in}}%
\pgfpathcurveto{\pgfqpoint{3.317293in}{0.865789in}}{\pgfqpoint{3.306694in}{0.870179in}}{\pgfqpoint{3.295644in}{0.870179in}}%
\pgfpathcurveto{\pgfqpoint{3.284593in}{0.870179in}}{\pgfqpoint{3.273994in}{0.865789in}}{\pgfqpoint{3.266181in}{0.857976in}}%
\pgfpathcurveto{\pgfqpoint{3.258367in}{0.850162in}}{\pgfqpoint{3.253977in}{0.839563in}}{\pgfqpoint{3.253977in}{0.828513in}}%
\pgfpathcurveto{\pgfqpoint{3.253977in}{0.817463in}}{\pgfqpoint{3.258367in}{0.806864in}}{\pgfqpoint{3.266181in}{0.799050in}}%
\pgfpathcurveto{\pgfqpoint{3.273994in}{0.791236in}}{\pgfqpoint{3.284593in}{0.786846in}}{\pgfqpoint{3.295644in}{0.786846in}}%
\pgfpathlineto{\pgfqpoint{3.295644in}{0.786846in}}%
\pgfpathclose%
\pgfusepath{stroke}%
\end{pgfscope}%
\begin{pgfscope}%
\pgfpathrectangle{\pgfqpoint{0.494722in}{0.437222in}}{\pgfqpoint{6.275590in}{5.159444in}}%
\pgfusepath{clip}%
\pgfsetbuttcap%
\pgfsetroundjoin%
\pgfsetlinewidth{1.003750pt}%
\definecolor{currentstroke}{rgb}{0.827451,0.827451,0.827451}%
\pgfsetstrokecolor{currentstroke}%
\pgfsetstrokeopacity{0.800000}%
\pgfsetdash{}{0pt}%
\pgfpathmoveto{\pgfqpoint{0.645918in}{3.445519in}}%
\pgfpathcurveto{\pgfqpoint{0.656968in}{3.445519in}}{\pgfqpoint{0.667567in}{3.449909in}}{\pgfqpoint{0.675381in}{3.457723in}}%
\pgfpathcurveto{\pgfqpoint{0.683195in}{3.465536in}}{\pgfqpoint{0.687585in}{3.476135in}}{\pgfqpoint{0.687585in}{3.487185in}}%
\pgfpathcurveto{\pgfqpoint{0.687585in}{3.498235in}}{\pgfqpoint{0.683195in}{3.508835in}}{\pgfqpoint{0.675381in}{3.516648in}}%
\pgfpathcurveto{\pgfqpoint{0.667567in}{3.524462in}}{\pgfqpoint{0.656968in}{3.528852in}}{\pgfqpoint{0.645918in}{3.528852in}}%
\pgfpathcurveto{\pgfqpoint{0.634868in}{3.528852in}}{\pgfqpoint{0.624269in}{3.524462in}}{\pgfqpoint{0.616455in}{3.516648in}}%
\pgfpathcurveto{\pgfqpoint{0.608642in}{3.508835in}}{\pgfqpoint{0.604251in}{3.498235in}}{\pgfqpoint{0.604251in}{3.487185in}}%
\pgfpathcurveto{\pgfqpoint{0.604251in}{3.476135in}}{\pgfqpoint{0.608642in}{3.465536in}}{\pgfqpoint{0.616455in}{3.457723in}}%
\pgfpathcurveto{\pgfqpoint{0.624269in}{3.449909in}}{\pgfqpoint{0.634868in}{3.445519in}}{\pgfqpoint{0.645918in}{3.445519in}}%
\pgfpathlineto{\pgfqpoint{0.645918in}{3.445519in}}%
\pgfpathclose%
\pgfusepath{stroke}%
\end{pgfscope}%
\begin{pgfscope}%
\pgfpathrectangle{\pgfqpoint{0.494722in}{0.437222in}}{\pgfqpoint{6.275590in}{5.159444in}}%
\pgfusepath{clip}%
\pgfsetbuttcap%
\pgfsetroundjoin%
\pgfsetlinewidth{1.003750pt}%
\definecolor{currentstroke}{rgb}{0.827451,0.827451,0.827451}%
\pgfsetstrokecolor{currentstroke}%
\pgfsetstrokeopacity{0.800000}%
\pgfsetdash{}{0pt}%
\pgfpathmoveto{\pgfqpoint{2.452460in}{1.181670in}}%
\pgfpathcurveto{\pgfqpoint{2.463510in}{1.181670in}}{\pgfqpoint{2.474109in}{1.186061in}}{\pgfqpoint{2.481923in}{1.193874in}}%
\pgfpathcurveto{\pgfqpoint{2.489737in}{1.201688in}}{\pgfqpoint{2.494127in}{1.212287in}}{\pgfqpoint{2.494127in}{1.223337in}}%
\pgfpathcurveto{\pgfqpoint{2.494127in}{1.234387in}}{\pgfqpoint{2.489737in}{1.244986in}}{\pgfqpoint{2.481923in}{1.252800in}}%
\pgfpathcurveto{\pgfqpoint{2.474109in}{1.260613in}}{\pgfqpoint{2.463510in}{1.265004in}}{\pgfqpoint{2.452460in}{1.265004in}}%
\pgfpathcurveto{\pgfqpoint{2.441410in}{1.265004in}}{\pgfqpoint{2.430811in}{1.260613in}}{\pgfqpoint{2.422998in}{1.252800in}}%
\pgfpathcurveto{\pgfqpoint{2.415184in}{1.244986in}}{\pgfqpoint{2.410794in}{1.234387in}}{\pgfqpoint{2.410794in}{1.223337in}}%
\pgfpathcurveto{\pgfqpoint{2.410794in}{1.212287in}}{\pgfqpoint{2.415184in}{1.201688in}}{\pgfqpoint{2.422998in}{1.193874in}}%
\pgfpathcurveto{\pgfqpoint{2.430811in}{1.186061in}}{\pgfqpoint{2.441410in}{1.181670in}}{\pgfqpoint{2.452460in}{1.181670in}}%
\pgfpathlineto{\pgfqpoint{2.452460in}{1.181670in}}%
\pgfpathclose%
\pgfusepath{stroke}%
\end{pgfscope}%
\begin{pgfscope}%
\pgfpathrectangle{\pgfqpoint{0.494722in}{0.437222in}}{\pgfqpoint{6.275590in}{5.159444in}}%
\pgfusepath{clip}%
\pgfsetbuttcap%
\pgfsetroundjoin%
\pgfsetlinewidth{1.003750pt}%
\definecolor{currentstroke}{rgb}{0.827451,0.827451,0.827451}%
\pgfsetstrokecolor{currentstroke}%
\pgfsetstrokeopacity{0.800000}%
\pgfsetdash{}{0pt}%
\pgfpathmoveto{\pgfqpoint{1.623472in}{1.825964in}}%
\pgfpathcurveto{\pgfqpoint{1.634522in}{1.825964in}}{\pgfqpoint{1.645121in}{1.830354in}}{\pgfqpoint{1.652935in}{1.838167in}}%
\pgfpathcurveto{\pgfqpoint{1.660748in}{1.845981in}}{\pgfqpoint{1.665139in}{1.856580in}}{\pgfqpoint{1.665139in}{1.867630in}}%
\pgfpathcurveto{\pgfqpoint{1.665139in}{1.878680in}}{\pgfqpoint{1.660748in}{1.889279in}}{\pgfqpoint{1.652935in}{1.897093in}}%
\pgfpathcurveto{\pgfqpoint{1.645121in}{1.904907in}}{\pgfqpoint{1.634522in}{1.909297in}}{\pgfqpoint{1.623472in}{1.909297in}}%
\pgfpathcurveto{\pgfqpoint{1.612422in}{1.909297in}}{\pgfqpoint{1.601823in}{1.904907in}}{\pgfqpoint{1.594009in}{1.897093in}}%
\pgfpathcurveto{\pgfqpoint{1.586196in}{1.889279in}}{\pgfqpoint{1.581805in}{1.878680in}}{\pgfqpoint{1.581805in}{1.867630in}}%
\pgfpathcurveto{\pgfqpoint{1.581805in}{1.856580in}}{\pgfqpoint{1.586196in}{1.845981in}}{\pgfqpoint{1.594009in}{1.838167in}}%
\pgfpathcurveto{\pgfqpoint{1.601823in}{1.830354in}}{\pgfqpoint{1.612422in}{1.825964in}}{\pgfqpoint{1.623472in}{1.825964in}}%
\pgfpathlineto{\pgfqpoint{1.623472in}{1.825964in}}%
\pgfpathclose%
\pgfusepath{stroke}%
\end{pgfscope}%
\begin{pgfscope}%
\pgfpathrectangle{\pgfqpoint{0.494722in}{0.437222in}}{\pgfqpoint{6.275590in}{5.159444in}}%
\pgfusepath{clip}%
\pgfsetbuttcap%
\pgfsetroundjoin%
\pgfsetlinewidth{1.003750pt}%
\definecolor{currentstroke}{rgb}{0.827451,0.827451,0.827451}%
\pgfsetstrokecolor{currentstroke}%
\pgfsetstrokeopacity{0.800000}%
\pgfsetdash{}{0pt}%
\pgfpathmoveto{\pgfqpoint{2.224297in}{1.325142in}}%
\pgfpathcurveto{\pgfqpoint{2.235347in}{1.325142in}}{\pgfqpoint{2.245946in}{1.329532in}}{\pgfqpoint{2.253760in}{1.337346in}}%
\pgfpathcurveto{\pgfqpoint{2.261573in}{1.345159in}}{\pgfqpoint{2.265964in}{1.355758in}}{\pgfqpoint{2.265964in}{1.366809in}}%
\pgfpathcurveto{\pgfqpoint{2.265964in}{1.377859in}}{\pgfqpoint{2.261573in}{1.388458in}}{\pgfqpoint{2.253760in}{1.396271in}}%
\pgfpathcurveto{\pgfqpoint{2.245946in}{1.404085in}}{\pgfqpoint{2.235347in}{1.408475in}}{\pgfqpoint{2.224297in}{1.408475in}}%
\pgfpathcurveto{\pgfqpoint{2.213247in}{1.408475in}}{\pgfqpoint{2.202648in}{1.404085in}}{\pgfqpoint{2.194834in}{1.396271in}}%
\pgfpathcurveto{\pgfqpoint{2.187020in}{1.388458in}}{\pgfqpoint{2.182630in}{1.377859in}}{\pgfqpoint{2.182630in}{1.366809in}}%
\pgfpathcurveto{\pgfqpoint{2.182630in}{1.355758in}}{\pgfqpoint{2.187020in}{1.345159in}}{\pgfqpoint{2.194834in}{1.337346in}}%
\pgfpathcurveto{\pgfqpoint{2.202648in}{1.329532in}}{\pgfqpoint{2.213247in}{1.325142in}}{\pgfqpoint{2.224297in}{1.325142in}}%
\pgfpathlineto{\pgfqpoint{2.224297in}{1.325142in}}%
\pgfpathclose%
\pgfusepath{stroke}%
\end{pgfscope}%
\begin{pgfscope}%
\pgfpathrectangle{\pgfqpoint{0.494722in}{0.437222in}}{\pgfqpoint{6.275590in}{5.159444in}}%
\pgfusepath{clip}%
\pgfsetbuttcap%
\pgfsetroundjoin%
\pgfsetlinewidth{1.003750pt}%
\definecolor{currentstroke}{rgb}{0.827451,0.827451,0.827451}%
\pgfsetstrokecolor{currentstroke}%
\pgfsetstrokeopacity{0.800000}%
\pgfsetdash{}{0pt}%
\pgfpathmoveto{\pgfqpoint{0.553084in}{3.907249in}}%
\pgfpathcurveto{\pgfqpoint{0.564134in}{3.907249in}}{\pgfqpoint{0.574733in}{3.911639in}}{\pgfqpoint{0.582547in}{3.919453in}}%
\pgfpathcurveto{\pgfqpoint{0.590360in}{3.927267in}}{\pgfqpoint{0.594750in}{3.937866in}}{\pgfqpoint{0.594750in}{3.948916in}}%
\pgfpathcurveto{\pgfqpoint{0.594750in}{3.959966in}}{\pgfqpoint{0.590360in}{3.970565in}}{\pgfqpoint{0.582547in}{3.978378in}}%
\pgfpathcurveto{\pgfqpoint{0.574733in}{3.986192in}}{\pgfqpoint{0.564134in}{3.990582in}}{\pgfqpoint{0.553084in}{3.990582in}}%
\pgfpathcurveto{\pgfqpoint{0.542034in}{3.990582in}}{\pgfqpoint{0.531435in}{3.986192in}}{\pgfqpoint{0.523621in}{3.978378in}}%
\pgfpathcurveto{\pgfqpoint{0.515807in}{3.970565in}}{\pgfqpoint{0.511417in}{3.959966in}}{\pgfqpoint{0.511417in}{3.948916in}}%
\pgfpathcurveto{\pgfqpoint{0.511417in}{3.937866in}}{\pgfqpoint{0.515807in}{3.927267in}}{\pgfqpoint{0.523621in}{3.919453in}}%
\pgfpathcurveto{\pgfqpoint{0.531435in}{3.911639in}}{\pgfqpoint{0.542034in}{3.907249in}}{\pgfqpoint{0.553084in}{3.907249in}}%
\pgfpathlineto{\pgfqpoint{0.553084in}{3.907249in}}%
\pgfpathclose%
\pgfusepath{stroke}%
\end{pgfscope}%
\begin{pgfscope}%
\pgfpathrectangle{\pgfqpoint{0.494722in}{0.437222in}}{\pgfqpoint{6.275590in}{5.159444in}}%
\pgfusepath{clip}%
\pgfsetbuttcap%
\pgfsetroundjoin%
\pgfsetlinewidth{1.003750pt}%
\definecolor{currentstroke}{rgb}{0.827451,0.827451,0.827451}%
\pgfsetstrokecolor{currentstroke}%
\pgfsetstrokeopacity{0.800000}%
\pgfsetdash{}{0pt}%
\pgfpathmoveto{\pgfqpoint{3.460812in}{0.723768in}}%
\pgfpathcurveto{\pgfqpoint{3.471862in}{0.723768in}}{\pgfqpoint{3.482461in}{0.728158in}}{\pgfqpoint{3.490274in}{0.735972in}}%
\pgfpathcurveto{\pgfqpoint{3.498088in}{0.743785in}}{\pgfqpoint{3.502478in}{0.754384in}}{\pgfqpoint{3.502478in}{0.765434in}}%
\pgfpathcurveto{\pgfqpoint{3.502478in}{0.776484in}}{\pgfqpoint{3.498088in}{0.787083in}}{\pgfqpoint{3.490274in}{0.794897in}}%
\pgfpathcurveto{\pgfqpoint{3.482461in}{0.802711in}}{\pgfqpoint{3.471862in}{0.807101in}}{\pgfqpoint{3.460812in}{0.807101in}}%
\pgfpathcurveto{\pgfqpoint{3.449762in}{0.807101in}}{\pgfqpoint{3.439162in}{0.802711in}}{\pgfqpoint{3.431349in}{0.794897in}}%
\pgfpathcurveto{\pgfqpoint{3.423535in}{0.787083in}}{\pgfqpoint{3.419145in}{0.776484in}}{\pgfqpoint{3.419145in}{0.765434in}}%
\pgfpathcurveto{\pgfqpoint{3.419145in}{0.754384in}}{\pgfqpoint{3.423535in}{0.743785in}}{\pgfqpoint{3.431349in}{0.735972in}}%
\pgfpathcurveto{\pgfqpoint{3.439162in}{0.728158in}}{\pgfqpoint{3.449762in}{0.723768in}}{\pgfqpoint{3.460812in}{0.723768in}}%
\pgfpathlineto{\pgfqpoint{3.460812in}{0.723768in}}%
\pgfpathclose%
\pgfusepath{stroke}%
\end{pgfscope}%
\begin{pgfscope}%
\pgfpathrectangle{\pgfqpoint{0.494722in}{0.437222in}}{\pgfqpoint{6.275590in}{5.159444in}}%
\pgfusepath{clip}%
\pgfsetbuttcap%
\pgfsetroundjoin%
\pgfsetlinewidth{1.003750pt}%
\definecolor{currentstroke}{rgb}{0.827451,0.827451,0.827451}%
\pgfsetstrokecolor{currentstroke}%
\pgfsetstrokeopacity{0.800000}%
\pgfsetdash{}{0pt}%
\pgfpathmoveto{\pgfqpoint{2.552339in}{1.147505in}}%
\pgfpathcurveto{\pgfqpoint{2.563390in}{1.147505in}}{\pgfqpoint{2.573989in}{1.151895in}}{\pgfqpoint{2.581802in}{1.159709in}}%
\pgfpathcurveto{\pgfqpoint{2.589616in}{1.167522in}}{\pgfqpoint{2.594006in}{1.178121in}}{\pgfqpoint{2.594006in}{1.189172in}}%
\pgfpathcurveto{\pgfqpoint{2.594006in}{1.200222in}}{\pgfqpoint{2.589616in}{1.210821in}}{\pgfqpoint{2.581802in}{1.218634in}}%
\pgfpathcurveto{\pgfqpoint{2.573989in}{1.226448in}}{\pgfqpoint{2.563390in}{1.230838in}}{\pgfqpoint{2.552339in}{1.230838in}}%
\pgfpathcurveto{\pgfqpoint{2.541289in}{1.230838in}}{\pgfqpoint{2.530690in}{1.226448in}}{\pgfqpoint{2.522877in}{1.218634in}}%
\pgfpathcurveto{\pgfqpoint{2.515063in}{1.210821in}}{\pgfqpoint{2.510673in}{1.200222in}}{\pgfqpoint{2.510673in}{1.189172in}}%
\pgfpathcurveto{\pgfqpoint{2.510673in}{1.178121in}}{\pgfqpoint{2.515063in}{1.167522in}}{\pgfqpoint{2.522877in}{1.159709in}}%
\pgfpathcurveto{\pgfqpoint{2.530690in}{1.151895in}}{\pgfqpoint{2.541289in}{1.147505in}}{\pgfqpoint{2.552339in}{1.147505in}}%
\pgfpathlineto{\pgfqpoint{2.552339in}{1.147505in}}%
\pgfpathclose%
\pgfusepath{stroke}%
\end{pgfscope}%
\begin{pgfscope}%
\pgfpathrectangle{\pgfqpoint{0.494722in}{0.437222in}}{\pgfqpoint{6.275590in}{5.159444in}}%
\pgfusepath{clip}%
\pgfsetbuttcap%
\pgfsetroundjoin%
\pgfsetlinewidth{1.003750pt}%
\definecolor{currentstroke}{rgb}{0.827451,0.827451,0.827451}%
\pgfsetstrokecolor{currentstroke}%
\pgfsetstrokeopacity{0.800000}%
\pgfsetdash{}{0pt}%
\pgfpathmoveto{\pgfqpoint{3.889511in}{0.591465in}}%
\pgfpathcurveto{\pgfqpoint{3.900562in}{0.591465in}}{\pgfqpoint{3.911161in}{0.595856in}}{\pgfqpoint{3.918974in}{0.603669in}}%
\pgfpathcurveto{\pgfqpoint{3.926788in}{0.611483in}}{\pgfqpoint{3.931178in}{0.622082in}}{\pgfqpoint{3.931178in}{0.633132in}}%
\pgfpathcurveto{\pgfqpoint{3.931178in}{0.644182in}}{\pgfqpoint{3.926788in}{0.654781in}}{\pgfqpoint{3.918974in}{0.662595in}}%
\pgfpathcurveto{\pgfqpoint{3.911161in}{0.670408in}}{\pgfqpoint{3.900562in}{0.674799in}}{\pgfqpoint{3.889511in}{0.674799in}}%
\pgfpathcurveto{\pgfqpoint{3.878461in}{0.674799in}}{\pgfqpoint{3.867862in}{0.670408in}}{\pgfqpoint{3.860049in}{0.662595in}}%
\pgfpathcurveto{\pgfqpoint{3.852235in}{0.654781in}}{\pgfqpoint{3.847845in}{0.644182in}}{\pgfqpoint{3.847845in}{0.633132in}}%
\pgfpathcurveto{\pgfqpoint{3.847845in}{0.622082in}}{\pgfqpoint{3.852235in}{0.611483in}}{\pgfqpoint{3.860049in}{0.603669in}}%
\pgfpathcurveto{\pgfqpoint{3.867862in}{0.595856in}}{\pgfqpoint{3.878461in}{0.591465in}}{\pgfqpoint{3.889511in}{0.591465in}}%
\pgfpathlineto{\pgfqpoint{3.889511in}{0.591465in}}%
\pgfpathclose%
\pgfusepath{stroke}%
\end{pgfscope}%
\begin{pgfscope}%
\pgfpathrectangle{\pgfqpoint{0.494722in}{0.437222in}}{\pgfqpoint{6.275590in}{5.159444in}}%
\pgfusepath{clip}%
\pgfsetbuttcap%
\pgfsetroundjoin%
\pgfsetlinewidth{1.003750pt}%
\definecolor{currentstroke}{rgb}{0.827451,0.827451,0.827451}%
\pgfsetstrokecolor{currentstroke}%
\pgfsetstrokeopacity{0.800000}%
\pgfsetdash{}{0pt}%
\pgfpathmoveto{\pgfqpoint{0.734476in}{3.173741in}}%
\pgfpathcurveto{\pgfqpoint{0.745526in}{3.173741in}}{\pgfqpoint{0.756125in}{3.178131in}}{\pgfqpoint{0.763939in}{3.185945in}}%
\pgfpathcurveto{\pgfqpoint{0.771753in}{3.193758in}}{\pgfqpoint{0.776143in}{3.204357in}}{\pgfqpoint{0.776143in}{3.215408in}}%
\pgfpathcurveto{\pgfqpoint{0.776143in}{3.226458in}}{\pgfqpoint{0.771753in}{3.237057in}}{\pgfqpoint{0.763939in}{3.244870in}}%
\pgfpathcurveto{\pgfqpoint{0.756125in}{3.252684in}}{\pgfqpoint{0.745526in}{3.257074in}}{\pgfqpoint{0.734476in}{3.257074in}}%
\pgfpathcurveto{\pgfqpoint{0.723426in}{3.257074in}}{\pgfqpoint{0.712827in}{3.252684in}}{\pgfqpoint{0.705013in}{3.244870in}}%
\pgfpathcurveto{\pgfqpoint{0.697200in}{3.237057in}}{\pgfqpoint{0.692810in}{3.226458in}}{\pgfqpoint{0.692810in}{3.215408in}}%
\pgfpathcurveto{\pgfqpoint{0.692810in}{3.204357in}}{\pgfqpoint{0.697200in}{3.193758in}}{\pgfqpoint{0.705013in}{3.185945in}}%
\pgfpathcurveto{\pgfqpoint{0.712827in}{3.178131in}}{\pgfqpoint{0.723426in}{3.173741in}}{\pgfqpoint{0.734476in}{3.173741in}}%
\pgfpathlineto{\pgfqpoint{0.734476in}{3.173741in}}%
\pgfpathclose%
\pgfusepath{stroke}%
\end{pgfscope}%
\begin{pgfscope}%
\pgfpathrectangle{\pgfqpoint{0.494722in}{0.437222in}}{\pgfqpoint{6.275590in}{5.159444in}}%
\pgfusepath{clip}%
\pgfsetbuttcap%
\pgfsetroundjoin%
\pgfsetlinewidth{1.003750pt}%
\definecolor{currentstroke}{rgb}{0.827451,0.827451,0.827451}%
\pgfsetstrokecolor{currentstroke}%
\pgfsetstrokeopacity{0.800000}%
\pgfsetdash{}{0pt}%
\pgfpathmoveto{\pgfqpoint{4.270074in}{0.511112in}}%
\pgfpathcurveto{\pgfqpoint{4.281124in}{0.511112in}}{\pgfqpoint{4.291723in}{0.515502in}}{\pgfqpoint{4.299537in}{0.523316in}}%
\pgfpathcurveto{\pgfqpoint{4.307351in}{0.531129in}}{\pgfqpoint{4.311741in}{0.541728in}}{\pgfqpoint{4.311741in}{0.552778in}}%
\pgfpathcurveto{\pgfqpoint{4.311741in}{0.563829in}}{\pgfqpoint{4.307351in}{0.574428in}}{\pgfqpoint{4.299537in}{0.582241in}}%
\pgfpathcurveto{\pgfqpoint{4.291723in}{0.590055in}}{\pgfqpoint{4.281124in}{0.594445in}}{\pgfqpoint{4.270074in}{0.594445in}}%
\pgfpathcurveto{\pgfqpoint{4.259024in}{0.594445in}}{\pgfqpoint{4.248425in}{0.590055in}}{\pgfqpoint{4.240611in}{0.582241in}}%
\pgfpathcurveto{\pgfqpoint{4.232798in}{0.574428in}}{\pgfqpoint{4.228407in}{0.563829in}}{\pgfqpoint{4.228407in}{0.552778in}}%
\pgfpathcurveto{\pgfqpoint{4.228407in}{0.541728in}}{\pgfqpoint{4.232798in}{0.531129in}}{\pgfqpoint{4.240611in}{0.523316in}}%
\pgfpathcurveto{\pgfqpoint{4.248425in}{0.515502in}}{\pgfqpoint{4.259024in}{0.511112in}}{\pgfqpoint{4.270074in}{0.511112in}}%
\pgfpathlineto{\pgfqpoint{4.270074in}{0.511112in}}%
\pgfpathclose%
\pgfusepath{stroke}%
\end{pgfscope}%
\begin{pgfscope}%
\pgfpathrectangle{\pgfqpoint{0.494722in}{0.437222in}}{\pgfqpoint{6.275590in}{5.159444in}}%
\pgfusepath{clip}%
\pgfsetbuttcap%
\pgfsetroundjoin%
\pgfsetlinewidth{1.003750pt}%
\definecolor{currentstroke}{rgb}{0.827451,0.827451,0.827451}%
\pgfsetstrokecolor{currentstroke}%
\pgfsetstrokeopacity{0.800000}%
\pgfsetdash{}{0pt}%
\pgfpathmoveto{\pgfqpoint{0.808638in}{2.983312in}}%
\pgfpathcurveto{\pgfqpoint{0.819689in}{2.983312in}}{\pgfqpoint{0.830288in}{2.987702in}}{\pgfqpoint{0.838101in}{2.995515in}}%
\pgfpathcurveto{\pgfqpoint{0.845915in}{3.003329in}}{\pgfqpoint{0.850305in}{3.013928in}}{\pgfqpoint{0.850305in}{3.024978in}}%
\pgfpathcurveto{\pgfqpoint{0.850305in}{3.036028in}}{\pgfqpoint{0.845915in}{3.046627in}}{\pgfqpoint{0.838101in}{3.054441in}}%
\pgfpathcurveto{\pgfqpoint{0.830288in}{3.062255in}}{\pgfqpoint{0.819689in}{3.066645in}}{\pgfqpoint{0.808638in}{3.066645in}}%
\pgfpathcurveto{\pgfqpoint{0.797588in}{3.066645in}}{\pgfqpoint{0.786989in}{3.062255in}}{\pgfqpoint{0.779176in}{3.054441in}}%
\pgfpathcurveto{\pgfqpoint{0.771362in}{3.046627in}}{\pgfqpoint{0.766972in}{3.036028in}}{\pgfqpoint{0.766972in}{3.024978in}}%
\pgfpathcurveto{\pgfqpoint{0.766972in}{3.013928in}}{\pgfqpoint{0.771362in}{3.003329in}}{\pgfqpoint{0.779176in}{2.995515in}}%
\pgfpathcurveto{\pgfqpoint{0.786989in}{2.987702in}}{\pgfqpoint{0.797588in}{2.983312in}}{\pgfqpoint{0.808638in}{2.983312in}}%
\pgfpathlineto{\pgfqpoint{0.808638in}{2.983312in}}%
\pgfpathclose%
\pgfusepath{stroke}%
\end{pgfscope}%
\begin{pgfscope}%
\pgfpathrectangle{\pgfqpoint{0.494722in}{0.437222in}}{\pgfqpoint{6.275590in}{5.159444in}}%
\pgfusepath{clip}%
\pgfsetbuttcap%
\pgfsetroundjoin%
\pgfsetlinewidth{1.003750pt}%
\definecolor{currentstroke}{rgb}{0.827451,0.827451,0.827451}%
\pgfsetstrokecolor{currentstroke}%
\pgfsetstrokeopacity{0.800000}%
\pgfsetdash{}{0pt}%
\pgfpathmoveto{\pgfqpoint{4.362526in}{0.500037in}}%
\pgfpathcurveto{\pgfqpoint{4.373576in}{0.500037in}}{\pgfqpoint{4.384175in}{0.504427in}}{\pgfqpoint{4.391989in}{0.512241in}}%
\pgfpathcurveto{\pgfqpoint{4.399802in}{0.520054in}}{\pgfqpoint{4.404193in}{0.530653in}}{\pgfqpoint{4.404193in}{0.541703in}}%
\pgfpathcurveto{\pgfqpoint{4.404193in}{0.552753in}}{\pgfqpoint{4.399802in}{0.563352in}}{\pgfqpoint{4.391989in}{0.571166in}}%
\pgfpathcurveto{\pgfqpoint{4.384175in}{0.578980in}}{\pgfqpoint{4.373576in}{0.583370in}}{\pgfqpoint{4.362526in}{0.583370in}}%
\pgfpathcurveto{\pgfqpoint{4.351476in}{0.583370in}}{\pgfqpoint{4.340877in}{0.578980in}}{\pgfqpoint{4.333063in}{0.571166in}}%
\pgfpathcurveto{\pgfqpoint{4.325250in}{0.563352in}}{\pgfqpoint{4.320859in}{0.552753in}}{\pgfqpoint{4.320859in}{0.541703in}}%
\pgfpathcurveto{\pgfqpoint{4.320859in}{0.530653in}}{\pgfqpoint{4.325250in}{0.520054in}}{\pgfqpoint{4.333063in}{0.512241in}}%
\pgfpathcurveto{\pgfqpoint{4.340877in}{0.504427in}}{\pgfqpoint{4.351476in}{0.500037in}}{\pgfqpoint{4.362526in}{0.500037in}}%
\pgfpathlineto{\pgfqpoint{4.362526in}{0.500037in}}%
\pgfpathclose%
\pgfusepath{stroke}%
\end{pgfscope}%
\begin{pgfscope}%
\pgfpathrectangle{\pgfqpoint{0.494722in}{0.437222in}}{\pgfqpoint{6.275590in}{5.159444in}}%
\pgfusepath{clip}%
\pgfsetbuttcap%
\pgfsetroundjoin%
\pgfsetlinewidth{1.003750pt}%
\definecolor{currentstroke}{rgb}{0.827451,0.827451,0.827451}%
\pgfsetstrokecolor{currentstroke}%
\pgfsetstrokeopacity{0.800000}%
\pgfsetdash{}{0pt}%
\pgfpathmoveto{\pgfqpoint{4.774246in}{0.444010in}}%
\pgfpathcurveto{\pgfqpoint{4.785296in}{0.444010in}}{\pgfqpoint{4.795895in}{0.448400in}}{\pgfqpoint{4.803708in}{0.456214in}}%
\pgfpathcurveto{\pgfqpoint{4.811522in}{0.464028in}}{\pgfqpoint{4.815912in}{0.474627in}}{\pgfqpoint{4.815912in}{0.485677in}}%
\pgfpathcurveto{\pgfqpoint{4.815912in}{0.496727in}}{\pgfqpoint{4.811522in}{0.507326in}}{\pgfqpoint{4.803708in}{0.515140in}}%
\pgfpathcurveto{\pgfqpoint{4.795895in}{0.522953in}}{\pgfqpoint{4.785296in}{0.527343in}}{\pgfqpoint{4.774246in}{0.527343in}}%
\pgfpathcurveto{\pgfqpoint{4.763195in}{0.527343in}}{\pgfqpoint{4.752596in}{0.522953in}}{\pgfqpoint{4.744783in}{0.515140in}}%
\pgfpathcurveto{\pgfqpoint{4.736969in}{0.507326in}}{\pgfqpoint{4.732579in}{0.496727in}}{\pgfqpoint{4.732579in}{0.485677in}}%
\pgfpathcurveto{\pgfqpoint{4.732579in}{0.474627in}}{\pgfqpoint{4.736969in}{0.464028in}}{\pgfqpoint{4.744783in}{0.456214in}}%
\pgfpathcurveto{\pgfqpoint{4.752596in}{0.448400in}}{\pgfqpoint{4.763195in}{0.444010in}}{\pgfqpoint{4.774246in}{0.444010in}}%
\pgfpathlineto{\pgfqpoint{4.774246in}{0.444010in}}%
\pgfpathclose%
\pgfusepath{stroke}%
\end{pgfscope}%
\begin{pgfscope}%
\pgfpathrectangle{\pgfqpoint{0.494722in}{0.437222in}}{\pgfqpoint{6.275590in}{5.159444in}}%
\pgfusepath{clip}%
\pgfsetbuttcap%
\pgfsetroundjoin%
\pgfsetlinewidth{1.003750pt}%
\definecolor{currentstroke}{rgb}{0.827451,0.827451,0.827451}%
\pgfsetstrokecolor{currentstroke}%
\pgfsetstrokeopacity{0.800000}%
\pgfsetdash{}{0pt}%
\pgfpathmoveto{\pgfqpoint{2.258440in}{1.300760in}}%
\pgfpathcurveto{\pgfqpoint{2.269490in}{1.300760in}}{\pgfqpoint{2.280089in}{1.305150in}}{\pgfqpoint{2.287902in}{1.312964in}}%
\pgfpathcurveto{\pgfqpoint{2.295716in}{1.320777in}}{\pgfqpoint{2.300106in}{1.331376in}}{\pgfqpoint{2.300106in}{1.342426in}}%
\pgfpathcurveto{\pgfqpoint{2.300106in}{1.353477in}}{\pgfqpoint{2.295716in}{1.364076in}}{\pgfqpoint{2.287902in}{1.371889in}}%
\pgfpathcurveto{\pgfqpoint{2.280089in}{1.379703in}}{\pgfqpoint{2.269490in}{1.384093in}}{\pgfqpoint{2.258440in}{1.384093in}}%
\pgfpathcurveto{\pgfqpoint{2.247389in}{1.384093in}}{\pgfqpoint{2.236790in}{1.379703in}}{\pgfqpoint{2.228977in}{1.371889in}}%
\pgfpathcurveto{\pgfqpoint{2.221163in}{1.364076in}}{\pgfqpoint{2.216773in}{1.353477in}}{\pgfqpoint{2.216773in}{1.342426in}}%
\pgfpathcurveto{\pgfqpoint{2.216773in}{1.331376in}}{\pgfqpoint{2.221163in}{1.320777in}}{\pgfqpoint{2.228977in}{1.312964in}}%
\pgfpathcurveto{\pgfqpoint{2.236790in}{1.305150in}}{\pgfqpoint{2.247389in}{1.300760in}}{\pgfqpoint{2.258440in}{1.300760in}}%
\pgfpathlineto{\pgfqpoint{2.258440in}{1.300760in}}%
\pgfpathclose%
\pgfusepath{stroke}%
\end{pgfscope}%
\begin{pgfscope}%
\pgfpathrectangle{\pgfqpoint{0.494722in}{0.437222in}}{\pgfqpoint{6.275590in}{5.159444in}}%
\pgfusepath{clip}%
\pgfsetbuttcap%
\pgfsetroundjoin%
\pgfsetlinewidth{1.003750pt}%
\definecolor{currentstroke}{rgb}{0.827451,0.827451,0.827451}%
\pgfsetstrokecolor{currentstroke}%
\pgfsetstrokeopacity{0.800000}%
\pgfsetdash{}{0pt}%
\pgfpathmoveto{\pgfqpoint{0.655608in}{3.415313in}}%
\pgfpathcurveto{\pgfqpoint{0.666658in}{3.415313in}}{\pgfqpoint{0.677257in}{3.419703in}}{\pgfqpoint{0.685071in}{3.427517in}}%
\pgfpathcurveto{\pgfqpoint{0.692885in}{3.435330in}}{\pgfqpoint{0.697275in}{3.445929in}}{\pgfqpoint{0.697275in}{3.456980in}}%
\pgfpathcurveto{\pgfqpoint{0.697275in}{3.468030in}}{\pgfqpoint{0.692885in}{3.478629in}}{\pgfqpoint{0.685071in}{3.486442in}}%
\pgfpathcurveto{\pgfqpoint{0.677257in}{3.494256in}}{\pgfqpoint{0.666658in}{3.498646in}}{\pgfqpoint{0.655608in}{3.498646in}}%
\pgfpathcurveto{\pgfqpoint{0.644558in}{3.498646in}}{\pgfqpoint{0.633959in}{3.494256in}}{\pgfqpoint{0.626145in}{3.486442in}}%
\pgfpathcurveto{\pgfqpoint{0.618332in}{3.478629in}}{\pgfqpoint{0.613941in}{3.468030in}}{\pgfqpoint{0.613941in}{3.456980in}}%
\pgfpathcurveto{\pgfqpoint{0.613941in}{3.445929in}}{\pgfqpoint{0.618332in}{3.435330in}}{\pgfqpoint{0.626145in}{3.427517in}}%
\pgfpathcurveto{\pgfqpoint{0.633959in}{3.419703in}}{\pgfqpoint{0.644558in}{3.415313in}}{\pgfqpoint{0.655608in}{3.415313in}}%
\pgfpathlineto{\pgfqpoint{0.655608in}{3.415313in}}%
\pgfpathclose%
\pgfusepath{stroke}%
\end{pgfscope}%
\begin{pgfscope}%
\pgfpathrectangle{\pgfqpoint{0.494722in}{0.437222in}}{\pgfqpoint{6.275590in}{5.159444in}}%
\pgfusepath{clip}%
\pgfsetbuttcap%
\pgfsetroundjoin%
\pgfsetlinewidth{1.003750pt}%
\definecolor{currentstroke}{rgb}{0.827451,0.827451,0.827451}%
\pgfsetstrokecolor{currentstroke}%
\pgfsetstrokeopacity{0.800000}%
\pgfsetdash{}{0pt}%
\pgfpathmoveto{\pgfqpoint{0.536273in}{3.987890in}}%
\pgfpathcurveto{\pgfqpoint{0.547324in}{3.987890in}}{\pgfqpoint{0.557923in}{3.992280in}}{\pgfqpoint{0.565736in}{4.000093in}}%
\pgfpathcurveto{\pgfqpoint{0.573550in}{4.007907in}}{\pgfqpoint{0.577940in}{4.018506in}}{\pgfqpoint{0.577940in}{4.029556in}}%
\pgfpathcurveto{\pgfqpoint{0.577940in}{4.040606in}}{\pgfqpoint{0.573550in}{4.051205in}}{\pgfqpoint{0.565736in}{4.059019in}}%
\pgfpathcurveto{\pgfqpoint{0.557923in}{4.066833in}}{\pgfqpoint{0.547324in}{4.071223in}}{\pgfqpoint{0.536273in}{4.071223in}}%
\pgfpathcurveto{\pgfqpoint{0.525223in}{4.071223in}}{\pgfqpoint{0.514624in}{4.066833in}}{\pgfqpoint{0.506811in}{4.059019in}}%
\pgfpathcurveto{\pgfqpoint{0.498997in}{4.051205in}}{\pgfqpoint{0.494607in}{4.040606in}}{\pgfqpoint{0.494607in}{4.029556in}}%
\pgfpathcurveto{\pgfqpoint{0.494607in}{4.018506in}}{\pgfqpoint{0.498997in}{4.007907in}}{\pgfqpoint{0.506811in}{4.000093in}}%
\pgfpathcurveto{\pgfqpoint{0.514624in}{3.992280in}}{\pgfqpoint{0.525223in}{3.987890in}}{\pgfqpoint{0.536273in}{3.987890in}}%
\pgfpathlineto{\pgfqpoint{0.536273in}{3.987890in}}%
\pgfpathclose%
\pgfusepath{stroke}%
\end{pgfscope}%
\begin{pgfscope}%
\pgfpathrectangle{\pgfqpoint{0.494722in}{0.437222in}}{\pgfqpoint{6.275590in}{5.159444in}}%
\pgfusepath{clip}%
\pgfsetbuttcap%
\pgfsetroundjoin%
\pgfsetlinewidth{1.003750pt}%
\definecolor{currentstroke}{rgb}{0.827451,0.827451,0.827451}%
\pgfsetstrokecolor{currentstroke}%
\pgfsetstrokeopacity{0.800000}%
\pgfsetdash{}{0pt}%
\pgfpathmoveto{\pgfqpoint{4.399590in}{0.490376in}}%
\pgfpathcurveto{\pgfqpoint{4.410640in}{0.490376in}}{\pgfqpoint{4.421239in}{0.494766in}}{\pgfqpoint{4.429053in}{0.502580in}}%
\pgfpathcurveto{\pgfqpoint{4.436866in}{0.510393in}}{\pgfqpoint{4.441257in}{0.520992in}}{\pgfqpoint{4.441257in}{0.532042in}}%
\pgfpathcurveto{\pgfqpoint{4.441257in}{0.543093in}}{\pgfqpoint{4.436866in}{0.553692in}}{\pgfqpoint{4.429053in}{0.561505in}}%
\pgfpathcurveto{\pgfqpoint{4.421239in}{0.569319in}}{\pgfqpoint{4.410640in}{0.573709in}}{\pgfqpoint{4.399590in}{0.573709in}}%
\pgfpathcurveto{\pgfqpoint{4.388540in}{0.573709in}}{\pgfqpoint{4.377941in}{0.569319in}}{\pgfqpoint{4.370127in}{0.561505in}}%
\pgfpathcurveto{\pgfqpoint{4.362314in}{0.553692in}}{\pgfqpoint{4.357923in}{0.543093in}}{\pgfqpoint{4.357923in}{0.532042in}}%
\pgfpathcurveto{\pgfqpoint{4.357923in}{0.520992in}}{\pgfqpoint{4.362314in}{0.510393in}}{\pgfqpoint{4.370127in}{0.502580in}}%
\pgfpathcurveto{\pgfqpoint{4.377941in}{0.494766in}}{\pgfqpoint{4.388540in}{0.490376in}}{\pgfqpoint{4.399590in}{0.490376in}}%
\pgfpathlineto{\pgfqpoint{4.399590in}{0.490376in}}%
\pgfpathclose%
\pgfusepath{stroke}%
\end{pgfscope}%
\begin{pgfscope}%
\pgfpathrectangle{\pgfqpoint{0.494722in}{0.437222in}}{\pgfqpoint{6.275590in}{5.159444in}}%
\pgfusepath{clip}%
\pgfsetbuttcap%
\pgfsetroundjoin%
\pgfsetlinewidth{1.003750pt}%
\definecolor{currentstroke}{rgb}{0.827451,0.827451,0.827451}%
\pgfsetstrokecolor{currentstroke}%
\pgfsetstrokeopacity{0.800000}%
\pgfsetdash{}{0pt}%
\pgfpathmoveto{\pgfqpoint{3.949353in}{0.575507in}}%
\pgfpathcurveto{\pgfqpoint{3.960403in}{0.575507in}}{\pgfqpoint{3.971002in}{0.579897in}}{\pgfqpoint{3.978815in}{0.587711in}}%
\pgfpathcurveto{\pgfqpoint{3.986629in}{0.595525in}}{\pgfqpoint{3.991019in}{0.606124in}}{\pgfqpoint{3.991019in}{0.617174in}}%
\pgfpathcurveto{\pgfqpoint{3.991019in}{0.628224in}}{\pgfqpoint{3.986629in}{0.638823in}}{\pgfqpoint{3.978815in}{0.646637in}}%
\pgfpathcurveto{\pgfqpoint{3.971002in}{0.654450in}}{\pgfqpoint{3.960403in}{0.658841in}}{\pgfqpoint{3.949353in}{0.658841in}}%
\pgfpathcurveto{\pgfqpoint{3.938303in}{0.658841in}}{\pgfqpoint{3.927704in}{0.654450in}}{\pgfqpoint{3.919890in}{0.646637in}}%
\pgfpathcurveto{\pgfqpoint{3.912076in}{0.638823in}}{\pgfqpoint{3.907686in}{0.628224in}}{\pgfqpoint{3.907686in}{0.617174in}}%
\pgfpathcurveto{\pgfqpoint{3.907686in}{0.606124in}}{\pgfqpoint{3.912076in}{0.595525in}}{\pgfqpoint{3.919890in}{0.587711in}}%
\pgfpathcurveto{\pgfqpoint{3.927704in}{0.579897in}}{\pgfqpoint{3.938303in}{0.575507in}}{\pgfqpoint{3.949353in}{0.575507in}}%
\pgfpathlineto{\pgfqpoint{3.949353in}{0.575507in}}%
\pgfpathclose%
\pgfusepath{stroke}%
\end{pgfscope}%
\begin{pgfscope}%
\pgfpathrectangle{\pgfqpoint{0.494722in}{0.437222in}}{\pgfqpoint{6.275590in}{5.159444in}}%
\pgfusepath{clip}%
\pgfsetbuttcap%
\pgfsetroundjoin%
\pgfsetlinewidth{1.003750pt}%
\definecolor{currentstroke}{rgb}{0.827451,0.827451,0.827451}%
\pgfsetstrokecolor{currentstroke}%
\pgfsetstrokeopacity{0.800000}%
\pgfsetdash{}{0pt}%
\pgfpathmoveto{\pgfqpoint{2.785306in}{1.003395in}}%
\pgfpathcurveto{\pgfqpoint{2.796356in}{1.003395in}}{\pgfqpoint{2.806955in}{1.007786in}}{\pgfqpoint{2.814769in}{1.015599in}}%
\pgfpathcurveto{\pgfqpoint{2.822582in}{1.023413in}}{\pgfqpoint{2.826973in}{1.034012in}}{\pgfqpoint{2.826973in}{1.045062in}}%
\pgfpathcurveto{\pgfqpoint{2.826973in}{1.056112in}}{\pgfqpoint{2.822582in}{1.066711in}}{\pgfqpoint{2.814769in}{1.074525in}}%
\pgfpathcurveto{\pgfqpoint{2.806955in}{1.082338in}}{\pgfqpoint{2.796356in}{1.086729in}}{\pgfqpoint{2.785306in}{1.086729in}}%
\pgfpathcurveto{\pgfqpoint{2.774256in}{1.086729in}}{\pgfqpoint{2.763657in}{1.082338in}}{\pgfqpoint{2.755843in}{1.074525in}}%
\pgfpathcurveto{\pgfqpoint{2.748030in}{1.066711in}}{\pgfqpoint{2.743639in}{1.056112in}}{\pgfqpoint{2.743639in}{1.045062in}}%
\pgfpathcurveto{\pgfqpoint{2.743639in}{1.034012in}}{\pgfqpoint{2.748030in}{1.023413in}}{\pgfqpoint{2.755843in}{1.015599in}}%
\pgfpathcurveto{\pgfqpoint{2.763657in}{1.007786in}}{\pgfqpoint{2.774256in}{1.003395in}}{\pgfqpoint{2.785306in}{1.003395in}}%
\pgfpathlineto{\pgfqpoint{2.785306in}{1.003395in}}%
\pgfpathclose%
\pgfusepath{stroke}%
\end{pgfscope}%
\begin{pgfscope}%
\pgfpathrectangle{\pgfqpoint{0.494722in}{0.437222in}}{\pgfqpoint{6.275590in}{5.159444in}}%
\pgfusepath{clip}%
\pgfsetbuttcap%
\pgfsetroundjoin%
\pgfsetlinewidth{1.003750pt}%
\definecolor{currentstroke}{rgb}{0.827451,0.827451,0.827451}%
\pgfsetstrokecolor{currentstroke}%
\pgfsetstrokeopacity{0.800000}%
\pgfsetdash{}{0pt}%
\pgfpathmoveto{\pgfqpoint{2.579575in}{1.114207in}}%
\pgfpathcurveto{\pgfqpoint{2.590625in}{1.114207in}}{\pgfqpoint{2.601224in}{1.118597in}}{\pgfqpoint{2.609038in}{1.126411in}}%
\pgfpathcurveto{\pgfqpoint{2.616851in}{1.134224in}}{\pgfqpoint{2.621242in}{1.144823in}}{\pgfqpoint{2.621242in}{1.155874in}}%
\pgfpathcurveto{\pgfqpoint{2.621242in}{1.166924in}}{\pgfqpoint{2.616851in}{1.177523in}}{\pgfqpoint{2.609038in}{1.185336in}}%
\pgfpathcurveto{\pgfqpoint{2.601224in}{1.193150in}}{\pgfqpoint{2.590625in}{1.197540in}}{\pgfqpoint{2.579575in}{1.197540in}}%
\pgfpathcurveto{\pgfqpoint{2.568525in}{1.197540in}}{\pgfqpoint{2.557926in}{1.193150in}}{\pgfqpoint{2.550112in}{1.185336in}}%
\pgfpathcurveto{\pgfqpoint{2.542298in}{1.177523in}}{\pgfqpoint{2.537908in}{1.166924in}}{\pgfqpoint{2.537908in}{1.155874in}}%
\pgfpathcurveto{\pgfqpoint{2.537908in}{1.144823in}}{\pgfqpoint{2.542298in}{1.134224in}}{\pgfqpoint{2.550112in}{1.126411in}}%
\pgfpathcurveto{\pgfqpoint{2.557926in}{1.118597in}}{\pgfqpoint{2.568525in}{1.114207in}}{\pgfqpoint{2.579575in}{1.114207in}}%
\pgfpathlineto{\pgfqpoint{2.579575in}{1.114207in}}%
\pgfpathclose%
\pgfusepath{stroke}%
\end{pgfscope}%
\begin{pgfscope}%
\pgfpathrectangle{\pgfqpoint{0.494722in}{0.437222in}}{\pgfqpoint{6.275590in}{5.159444in}}%
\pgfusepath{clip}%
\pgfsetbuttcap%
\pgfsetroundjoin%
\pgfsetlinewidth{1.003750pt}%
\definecolor{currentstroke}{rgb}{0.827451,0.827451,0.827451}%
\pgfsetstrokecolor{currentstroke}%
\pgfsetstrokeopacity{0.800000}%
\pgfsetdash{}{0pt}%
\pgfpathmoveto{\pgfqpoint{5.111741in}{0.421566in}}%
\pgfpathcurveto{\pgfqpoint{5.122791in}{0.421566in}}{\pgfqpoint{5.133390in}{0.425957in}}{\pgfqpoint{5.141204in}{0.433770in}}%
\pgfpathcurveto{\pgfqpoint{5.149018in}{0.441584in}}{\pgfqpoint{5.153408in}{0.452183in}}{\pgfqpoint{5.153408in}{0.463233in}}%
\pgfpathcurveto{\pgfqpoint{5.153408in}{0.474283in}}{\pgfqpoint{5.149018in}{0.484882in}}{\pgfqpoint{5.141204in}{0.492696in}}%
\pgfpathcurveto{\pgfqpoint{5.133390in}{0.500509in}}{\pgfqpoint{5.122791in}{0.504900in}}{\pgfqpoint{5.111741in}{0.504900in}}%
\pgfpathcurveto{\pgfqpoint{5.100691in}{0.504900in}}{\pgfqpoint{5.090092in}{0.500509in}}{\pgfqpoint{5.082278in}{0.492696in}}%
\pgfpathcurveto{\pgfqpoint{5.074465in}{0.484882in}}{\pgfqpoint{5.070074in}{0.474283in}}{\pgfqpoint{5.070074in}{0.463233in}}%
\pgfpathcurveto{\pgfqpoint{5.070074in}{0.452183in}}{\pgfqpoint{5.074465in}{0.441584in}}{\pgfqpoint{5.082278in}{0.433770in}}%
\pgfpathcurveto{\pgfqpoint{5.090092in}{0.425957in}}{\pgfqpoint{5.100691in}{0.421566in}}{\pgfqpoint{5.111741in}{0.421566in}}%
\pgfusepath{stroke}%
\end{pgfscope}%
\begin{pgfscope}%
\pgfpathrectangle{\pgfqpoint{0.494722in}{0.437222in}}{\pgfqpoint{6.275590in}{5.159444in}}%
\pgfusepath{clip}%
\pgfsetbuttcap%
\pgfsetroundjoin%
\pgfsetlinewidth{1.003750pt}%
\definecolor{currentstroke}{rgb}{0.827451,0.827451,0.827451}%
\pgfsetstrokecolor{currentstroke}%
\pgfsetstrokeopacity{0.800000}%
\pgfsetdash{}{0pt}%
\pgfpathmoveto{\pgfqpoint{0.523223in}{4.112992in}}%
\pgfpathcurveto{\pgfqpoint{0.534273in}{4.112992in}}{\pgfqpoint{0.544872in}{4.117382in}}{\pgfqpoint{0.552686in}{4.125196in}}%
\pgfpathcurveto{\pgfqpoint{0.560499in}{4.133010in}}{\pgfqpoint{0.564889in}{4.143609in}}{\pgfqpoint{0.564889in}{4.154659in}}%
\pgfpathcurveto{\pgfqpoint{0.564889in}{4.165709in}}{\pgfqpoint{0.560499in}{4.176308in}}{\pgfqpoint{0.552686in}{4.184121in}}%
\pgfpathcurveto{\pgfqpoint{0.544872in}{4.191935in}}{\pgfqpoint{0.534273in}{4.196325in}}{\pgfqpoint{0.523223in}{4.196325in}}%
\pgfpathcurveto{\pgfqpoint{0.512173in}{4.196325in}}{\pgfqpoint{0.501574in}{4.191935in}}{\pgfqpoint{0.493760in}{4.184121in}}%
\pgfpathcurveto{\pgfqpoint{0.485946in}{4.176308in}}{\pgfqpoint{0.481556in}{4.165709in}}{\pgfqpoint{0.481556in}{4.154659in}}%
\pgfpathcurveto{\pgfqpoint{0.481556in}{4.143609in}}{\pgfqpoint{0.485946in}{4.133010in}}{\pgfqpoint{0.493760in}{4.125196in}}%
\pgfpathcurveto{\pgfqpoint{0.501574in}{4.117382in}}{\pgfqpoint{0.512173in}{4.112992in}}{\pgfqpoint{0.523223in}{4.112992in}}%
\pgfpathlineto{\pgfqpoint{0.523223in}{4.112992in}}%
\pgfpathclose%
\pgfusepath{stroke}%
\end{pgfscope}%
\begin{pgfscope}%
\pgfpathrectangle{\pgfqpoint{0.494722in}{0.437222in}}{\pgfqpoint{6.275590in}{5.159444in}}%
\pgfusepath{clip}%
\pgfsetbuttcap%
\pgfsetroundjoin%
\pgfsetlinewidth{1.003750pt}%
\definecolor{currentstroke}{rgb}{0.827451,0.827451,0.827451}%
\pgfsetstrokecolor{currentstroke}%
\pgfsetstrokeopacity{0.800000}%
\pgfsetdash{}{0pt}%
\pgfpathmoveto{\pgfqpoint{0.495484in}{4.544725in}}%
\pgfpathcurveto{\pgfqpoint{0.506534in}{4.544725in}}{\pgfqpoint{0.517133in}{4.549116in}}{\pgfqpoint{0.524947in}{4.556929in}}%
\pgfpathcurveto{\pgfqpoint{0.532761in}{4.564743in}}{\pgfqpoint{0.537151in}{4.575342in}}{\pgfqpoint{0.537151in}{4.586392in}}%
\pgfpathcurveto{\pgfqpoint{0.537151in}{4.597442in}}{\pgfqpoint{0.532761in}{4.608041in}}{\pgfqpoint{0.524947in}{4.615855in}}%
\pgfpathcurveto{\pgfqpoint{0.517133in}{4.623668in}}{\pgfqpoint{0.506534in}{4.628059in}}{\pgfqpoint{0.495484in}{4.628059in}}%
\pgfpathcurveto{\pgfqpoint{0.484434in}{4.628059in}}{\pgfqpoint{0.473835in}{4.623668in}}{\pgfqpoint{0.466021in}{4.615855in}}%
\pgfpathcurveto{\pgfqpoint{0.458208in}{4.608041in}}{\pgfqpoint{0.453817in}{4.597442in}}{\pgfqpoint{0.453817in}{4.586392in}}%
\pgfpathcurveto{\pgfqpoint{0.453817in}{4.575342in}}{\pgfqpoint{0.458208in}{4.564743in}}{\pgfqpoint{0.466021in}{4.556929in}}%
\pgfpathcurveto{\pgfqpoint{0.473835in}{4.549116in}}{\pgfqpoint{0.484434in}{4.544725in}}{\pgfqpoint{0.495484in}{4.544725in}}%
\pgfpathlineto{\pgfqpoint{0.495484in}{4.544725in}}%
\pgfpathclose%
\pgfusepath{stroke}%
\end{pgfscope}%
\begin{pgfscope}%
\pgfpathrectangle{\pgfqpoint{0.494722in}{0.437222in}}{\pgfqpoint{6.275590in}{5.159444in}}%
\pgfusepath{clip}%
\pgfsetbuttcap%
\pgfsetroundjoin%
\pgfsetlinewidth{1.003750pt}%
\definecolor{currentstroke}{rgb}{0.827451,0.827451,0.827451}%
\pgfsetstrokecolor{currentstroke}%
\pgfsetstrokeopacity{0.800000}%
\pgfsetdash{}{0pt}%
\pgfpathmoveto{\pgfqpoint{3.739550in}{0.626568in}}%
\pgfpathcurveto{\pgfqpoint{3.750600in}{0.626568in}}{\pgfqpoint{3.761199in}{0.630958in}}{\pgfqpoint{3.769013in}{0.638771in}}%
\pgfpathcurveto{\pgfqpoint{3.776827in}{0.646585in}}{\pgfqpoint{3.781217in}{0.657184in}}{\pgfqpoint{3.781217in}{0.668234in}}%
\pgfpathcurveto{\pgfqpoint{3.781217in}{0.679284in}}{\pgfqpoint{3.776827in}{0.689883in}}{\pgfqpoint{3.769013in}{0.697697in}}%
\pgfpathcurveto{\pgfqpoint{3.761199in}{0.705511in}}{\pgfqpoint{3.750600in}{0.709901in}}{\pgfqpoint{3.739550in}{0.709901in}}%
\pgfpathcurveto{\pgfqpoint{3.728500in}{0.709901in}}{\pgfqpoint{3.717901in}{0.705511in}}{\pgfqpoint{3.710087in}{0.697697in}}%
\pgfpathcurveto{\pgfqpoint{3.702274in}{0.689883in}}{\pgfqpoint{3.697884in}{0.679284in}}{\pgfqpoint{3.697884in}{0.668234in}}%
\pgfpathcurveto{\pgfqpoint{3.697884in}{0.657184in}}{\pgfqpoint{3.702274in}{0.646585in}}{\pgfqpoint{3.710087in}{0.638771in}}%
\pgfpathcurveto{\pgfqpoint{3.717901in}{0.630958in}}{\pgfqpoint{3.728500in}{0.626568in}}{\pgfqpoint{3.739550in}{0.626568in}}%
\pgfpathlineto{\pgfqpoint{3.739550in}{0.626568in}}%
\pgfpathclose%
\pgfusepath{stroke}%
\end{pgfscope}%
\begin{pgfscope}%
\pgfpathrectangle{\pgfqpoint{0.494722in}{0.437222in}}{\pgfqpoint{6.275590in}{5.159444in}}%
\pgfusepath{clip}%
\pgfsetbuttcap%
\pgfsetroundjoin%
\pgfsetlinewidth{1.003750pt}%
\definecolor{currentstroke}{rgb}{0.827451,0.827451,0.827451}%
\pgfsetstrokecolor{currentstroke}%
\pgfsetstrokeopacity{0.800000}%
\pgfsetdash{}{0pt}%
\pgfpathmoveto{\pgfqpoint{1.347485in}{2.096739in}}%
\pgfpathcurveto{\pgfqpoint{1.358535in}{2.096739in}}{\pgfqpoint{1.369135in}{2.101129in}}{\pgfqpoint{1.376948in}{2.108943in}}%
\pgfpathcurveto{\pgfqpoint{1.384762in}{2.116757in}}{\pgfqpoint{1.389152in}{2.127356in}}{\pgfqpoint{1.389152in}{2.138406in}}%
\pgfpathcurveto{\pgfqpoint{1.389152in}{2.149456in}}{\pgfqpoint{1.384762in}{2.160055in}}{\pgfqpoint{1.376948in}{2.167869in}}%
\pgfpathcurveto{\pgfqpoint{1.369135in}{2.175682in}}{\pgfqpoint{1.358535in}{2.180072in}}{\pgfqpoint{1.347485in}{2.180072in}}%
\pgfpathcurveto{\pgfqpoint{1.336435in}{2.180072in}}{\pgfqpoint{1.325836in}{2.175682in}}{\pgfqpoint{1.318023in}{2.167869in}}%
\pgfpathcurveto{\pgfqpoint{1.310209in}{2.160055in}}{\pgfqpoint{1.305819in}{2.149456in}}{\pgfqpoint{1.305819in}{2.138406in}}%
\pgfpathcurveto{\pgfqpoint{1.305819in}{2.127356in}}{\pgfqpoint{1.310209in}{2.116757in}}{\pgfqpoint{1.318023in}{2.108943in}}%
\pgfpathcurveto{\pgfqpoint{1.325836in}{2.101129in}}{\pgfqpoint{1.336435in}{2.096739in}}{\pgfqpoint{1.347485in}{2.096739in}}%
\pgfpathlineto{\pgfqpoint{1.347485in}{2.096739in}}%
\pgfpathclose%
\pgfusepath{stroke}%
\end{pgfscope}%
\begin{pgfscope}%
\pgfpathrectangle{\pgfqpoint{0.494722in}{0.437222in}}{\pgfqpoint{6.275590in}{5.159444in}}%
\pgfusepath{clip}%
\pgfsetbuttcap%
\pgfsetroundjoin%
\pgfsetlinewidth{1.003750pt}%
\definecolor{currentstroke}{rgb}{0.827451,0.827451,0.827451}%
\pgfsetstrokecolor{currentstroke}%
\pgfsetstrokeopacity{0.800000}%
\pgfsetdash{}{0pt}%
\pgfpathmoveto{\pgfqpoint{1.113110in}{2.511719in}}%
\pgfpathcurveto{\pgfqpoint{1.124161in}{2.511719in}}{\pgfqpoint{1.134760in}{2.516109in}}{\pgfqpoint{1.142573in}{2.523923in}}%
\pgfpathcurveto{\pgfqpoint{1.150387in}{2.531736in}}{\pgfqpoint{1.154777in}{2.542335in}}{\pgfqpoint{1.154777in}{2.553385in}}%
\pgfpathcurveto{\pgfqpoint{1.154777in}{2.564435in}}{\pgfqpoint{1.150387in}{2.575034in}}{\pgfqpoint{1.142573in}{2.582848in}}%
\pgfpathcurveto{\pgfqpoint{1.134760in}{2.590662in}}{\pgfqpoint{1.124161in}{2.595052in}}{\pgfqpoint{1.113110in}{2.595052in}}%
\pgfpathcurveto{\pgfqpoint{1.102060in}{2.595052in}}{\pgfqpoint{1.091461in}{2.590662in}}{\pgfqpoint{1.083648in}{2.582848in}}%
\pgfpathcurveto{\pgfqpoint{1.075834in}{2.575034in}}{\pgfqpoint{1.071444in}{2.564435in}}{\pgfqpoint{1.071444in}{2.553385in}}%
\pgfpathcurveto{\pgfqpoint{1.071444in}{2.542335in}}{\pgfqpoint{1.075834in}{2.531736in}}{\pgfqpoint{1.083648in}{2.523923in}}%
\pgfpathcurveto{\pgfqpoint{1.091461in}{2.516109in}}{\pgfqpoint{1.102060in}{2.511719in}}{\pgfqpoint{1.113110in}{2.511719in}}%
\pgfpathlineto{\pgfqpoint{1.113110in}{2.511719in}}%
\pgfpathclose%
\pgfusepath{stroke}%
\end{pgfscope}%
\begin{pgfscope}%
\pgfpathrectangle{\pgfqpoint{0.494722in}{0.437222in}}{\pgfqpoint{6.275590in}{5.159444in}}%
\pgfusepath{clip}%
\pgfsetbuttcap%
\pgfsetroundjoin%
\pgfsetlinewidth{1.003750pt}%
\definecolor{currentstroke}{rgb}{0.827451,0.827451,0.827451}%
\pgfsetstrokecolor{currentstroke}%
\pgfsetstrokeopacity{0.800000}%
\pgfsetdash{}{0pt}%
\pgfpathmoveto{\pgfqpoint{2.490073in}{1.171385in}}%
\pgfpathcurveto{\pgfqpoint{2.501123in}{1.171385in}}{\pgfqpoint{2.511722in}{1.175776in}}{\pgfqpoint{2.519535in}{1.183589in}}%
\pgfpathcurveto{\pgfqpoint{2.527349in}{1.191403in}}{\pgfqpoint{2.531739in}{1.202002in}}{\pgfqpoint{2.531739in}{1.213052in}}%
\pgfpathcurveto{\pgfqpoint{2.531739in}{1.224102in}}{\pgfqpoint{2.527349in}{1.234701in}}{\pgfqpoint{2.519535in}{1.242515in}}%
\pgfpathcurveto{\pgfqpoint{2.511722in}{1.250328in}}{\pgfqpoint{2.501123in}{1.254719in}}{\pgfqpoint{2.490073in}{1.254719in}}%
\pgfpathcurveto{\pgfqpoint{2.479023in}{1.254719in}}{\pgfqpoint{2.468423in}{1.250328in}}{\pgfqpoint{2.460610in}{1.242515in}}%
\pgfpathcurveto{\pgfqpoint{2.452796in}{1.234701in}}{\pgfqpoint{2.448406in}{1.224102in}}{\pgfqpoint{2.448406in}{1.213052in}}%
\pgfpathcurveto{\pgfqpoint{2.448406in}{1.202002in}}{\pgfqpoint{2.452796in}{1.191403in}}{\pgfqpoint{2.460610in}{1.183589in}}%
\pgfpathcurveto{\pgfqpoint{2.468423in}{1.175776in}}{\pgfqpoint{2.479023in}{1.171385in}}{\pgfqpoint{2.490073in}{1.171385in}}%
\pgfpathlineto{\pgfqpoint{2.490073in}{1.171385in}}%
\pgfpathclose%
\pgfusepath{stroke}%
\end{pgfscope}%
\begin{pgfscope}%
\pgfpathrectangle{\pgfqpoint{0.494722in}{0.437222in}}{\pgfqpoint{6.275590in}{5.159444in}}%
\pgfusepath{clip}%
\pgfsetbuttcap%
\pgfsetroundjoin%
\pgfsetlinewidth{1.003750pt}%
\definecolor{currentstroke}{rgb}{0.827451,0.827451,0.827451}%
\pgfsetstrokecolor{currentstroke}%
\pgfsetstrokeopacity{0.800000}%
\pgfsetdash{}{0pt}%
\pgfpathmoveto{\pgfqpoint{0.680103in}{3.350578in}}%
\pgfpathcurveto{\pgfqpoint{0.691153in}{3.350578in}}{\pgfqpoint{0.701753in}{3.354968in}}{\pgfqpoint{0.709566in}{3.362782in}}%
\pgfpathcurveto{\pgfqpoint{0.717380in}{3.370596in}}{\pgfqpoint{0.721770in}{3.381195in}}{\pgfqpoint{0.721770in}{3.392245in}}%
\pgfpathcurveto{\pgfqpoint{0.721770in}{3.403295in}}{\pgfqpoint{0.717380in}{3.413894in}}{\pgfqpoint{0.709566in}{3.421707in}}%
\pgfpathcurveto{\pgfqpoint{0.701753in}{3.429521in}}{\pgfqpoint{0.691153in}{3.433911in}}{\pgfqpoint{0.680103in}{3.433911in}}%
\pgfpathcurveto{\pgfqpoint{0.669053in}{3.433911in}}{\pgfqpoint{0.658454in}{3.429521in}}{\pgfqpoint{0.650641in}{3.421707in}}%
\pgfpathcurveto{\pgfqpoint{0.642827in}{3.413894in}}{\pgfqpoint{0.638437in}{3.403295in}}{\pgfqpoint{0.638437in}{3.392245in}}%
\pgfpathcurveto{\pgfqpoint{0.638437in}{3.381195in}}{\pgfqpoint{0.642827in}{3.370596in}}{\pgfqpoint{0.650641in}{3.362782in}}%
\pgfpathcurveto{\pgfqpoint{0.658454in}{3.354968in}}{\pgfqpoint{0.669053in}{3.350578in}}{\pgfqpoint{0.680103in}{3.350578in}}%
\pgfpathlineto{\pgfqpoint{0.680103in}{3.350578in}}%
\pgfpathclose%
\pgfusepath{stroke}%
\end{pgfscope}%
\begin{pgfscope}%
\pgfpathrectangle{\pgfqpoint{0.494722in}{0.437222in}}{\pgfqpoint{6.275590in}{5.159444in}}%
\pgfusepath{clip}%
\pgfsetbuttcap%
\pgfsetroundjoin%
\pgfsetlinewidth{1.003750pt}%
\definecolor{currentstroke}{rgb}{0.827451,0.827451,0.827451}%
\pgfsetstrokecolor{currentstroke}%
\pgfsetstrokeopacity{0.800000}%
\pgfsetdash{}{0pt}%
\pgfpathmoveto{\pgfqpoint{3.527306in}{0.699588in}}%
\pgfpathcurveto{\pgfqpoint{3.538356in}{0.699588in}}{\pgfqpoint{3.548955in}{0.703979in}}{\pgfqpoint{3.556769in}{0.711792in}}%
\pgfpathcurveto{\pgfqpoint{3.564583in}{0.719606in}}{\pgfqpoint{3.568973in}{0.730205in}}{\pgfqpoint{3.568973in}{0.741255in}}%
\pgfpathcurveto{\pgfqpoint{3.568973in}{0.752305in}}{\pgfqpoint{3.564583in}{0.762904in}}{\pgfqpoint{3.556769in}{0.770718in}}%
\pgfpathcurveto{\pgfqpoint{3.548955in}{0.778532in}}{\pgfqpoint{3.538356in}{0.782922in}}{\pgfqpoint{3.527306in}{0.782922in}}%
\pgfpathcurveto{\pgfqpoint{3.516256in}{0.782922in}}{\pgfqpoint{3.505657in}{0.778532in}}{\pgfqpoint{3.497843in}{0.770718in}}%
\pgfpathcurveto{\pgfqpoint{3.490030in}{0.762904in}}{\pgfqpoint{3.485640in}{0.752305in}}{\pgfqpoint{3.485640in}{0.741255in}}%
\pgfpathcurveto{\pgfqpoint{3.485640in}{0.730205in}}{\pgfqpoint{3.490030in}{0.719606in}}{\pgfqpoint{3.497843in}{0.711792in}}%
\pgfpathcurveto{\pgfqpoint{3.505657in}{0.703979in}}{\pgfqpoint{3.516256in}{0.699588in}}{\pgfqpoint{3.527306in}{0.699588in}}%
\pgfpathlineto{\pgfqpoint{3.527306in}{0.699588in}}%
\pgfpathclose%
\pgfusepath{stroke}%
\end{pgfscope}%
\begin{pgfscope}%
\pgfpathrectangle{\pgfqpoint{0.494722in}{0.437222in}}{\pgfqpoint{6.275590in}{5.159444in}}%
\pgfusepath{clip}%
\pgfsetbuttcap%
\pgfsetroundjoin%
\pgfsetlinewidth{1.003750pt}%
\definecolor{currentstroke}{rgb}{0.827451,0.827451,0.827451}%
\pgfsetstrokecolor{currentstroke}%
\pgfsetstrokeopacity{0.800000}%
\pgfsetdash{}{0pt}%
\pgfpathmoveto{\pgfqpoint{1.325845in}{2.126824in}}%
\pgfpathcurveto{\pgfqpoint{1.336895in}{2.126824in}}{\pgfqpoint{1.347494in}{2.131214in}}{\pgfqpoint{1.355308in}{2.139028in}}%
\pgfpathcurveto{\pgfqpoint{1.363121in}{2.146842in}}{\pgfqpoint{1.367512in}{2.157441in}}{\pgfqpoint{1.367512in}{2.168491in}}%
\pgfpathcurveto{\pgfqpoint{1.367512in}{2.179541in}}{\pgfqpoint{1.363121in}{2.190140in}}{\pgfqpoint{1.355308in}{2.197954in}}%
\pgfpathcurveto{\pgfqpoint{1.347494in}{2.205767in}}{\pgfqpoint{1.336895in}{2.210157in}}{\pgfqpoint{1.325845in}{2.210157in}}%
\pgfpathcurveto{\pgfqpoint{1.314795in}{2.210157in}}{\pgfqpoint{1.304196in}{2.205767in}}{\pgfqpoint{1.296382in}{2.197954in}}%
\pgfpathcurveto{\pgfqpoint{1.288569in}{2.190140in}}{\pgfqpoint{1.284178in}{2.179541in}}{\pgfqpoint{1.284178in}{2.168491in}}%
\pgfpathcurveto{\pgfqpoint{1.284178in}{2.157441in}}{\pgfqpoint{1.288569in}{2.146842in}}{\pgfqpoint{1.296382in}{2.139028in}}%
\pgfpathcurveto{\pgfqpoint{1.304196in}{2.131214in}}{\pgfqpoint{1.314795in}{2.126824in}}{\pgfqpoint{1.325845in}{2.126824in}}%
\pgfpathlineto{\pgfqpoint{1.325845in}{2.126824in}}%
\pgfpathclose%
\pgfusepath{stroke}%
\end{pgfscope}%
\begin{pgfscope}%
\pgfpathrectangle{\pgfqpoint{0.494722in}{0.437222in}}{\pgfqpoint{6.275590in}{5.159444in}}%
\pgfusepath{clip}%
\pgfsetbuttcap%
\pgfsetroundjoin%
\pgfsetlinewidth{1.003750pt}%
\definecolor{currentstroke}{rgb}{0.827451,0.827451,0.827451}%
\pgfsetstrokecolor{currentstroke}%
\pgfsetstrokeopacity{0.800000}%
\pgfsetdash{}{0pt}%
\pgfpathmoveto{\pgfqpoint{5.564107in}{0.399714in}}%
\pgfpathcurveto{\pgfqpoint{5.575157in}{0.399714in}}{\pgfqpoint{5.585756in}{0.404105in}}{\pgfqpoint{5.593570in}{0.411918in}}%
\pgfpathcurveto{\pgfqpoint{5.601383in}{0.419732in}}{\pgfqpoint{5.605774in}{0.430331in}}{\pgfqpoint{5.605774in}{0.441381in}}%
\pgfpathcurveto{\pgfqpoint{5.605774in}{0.452431in}}{\pgfqpoint{5.601383in}{0.463030in}}{\pgfqpoint{5.593570in}{0.470844in}}%
\pgfpathcurveto{\pgfqpoint{5.585756in}{0.478658in}}{\pgfqpoint{5.575157in}{0.483048in}}{\pgfqpoint{5.564107in}{0.483048in}}%
\pgfpathcurveto{\pgfqpoint{5.553057in}{0.483048in}}{\pgfqpoint{5.542458in}{0.478658in}}{\pgfqpoint{5.534644in}{0.470844in}}%
\pgfpathcurveto{\pgfqpoint{5.526830in}{0.463030in}}{\pgfqpoint{5.522440in}{0.452431in}}{\pgfqpoint{5.522440in}{0.441381in}}%
\pgfpathcurveto{\pgfqpoint{5.522440in}{0.430331in}}{\pgfqpoint{5.526830in}{0.419732in}}{\pgfqpoint{5.534644in}{0.411918in}}%
\pgfpathcurveto{\pgfqpoint{5.542458in}{0.404105in}}{\pgfqpoint{5.553057in}{0.399714in}}{\pgfqpoint{5.564107in}{0.399714in}}%
\pgfusepath{stroke}%
\end{pgfscope}%
\begin{pgfscope}%
\pgfpathrectangle{\pgfqpoint{0.494722in}{0.437222in}}{\pgfqpoint{6.275590in}{5.159444in}}%
\pgfusepath{clip}%
\pgfsetbuttcap%
\pgfsetroundjoin%
\pgfsetlinewidth{1.003750pt}%
\definecolor{currentstroke}{rgb}{0.827451,0.827451,0.827451}%
\pgfsetstrokecolor{currentstroke}%
\pgfsetstrokeopacity{0.800000}%
\pgfsetdash{}{0pt}%
\pgfpathmoveto{\pgfqpoint{1.381781in}{2.056420in}}%
\pgfpathcurveto{\pgfqpoint{1.392831in}{2.056420in}}{\pgfqpoint{1.403430in}{2.060810in}}{\pgfqpoint{1.411244in}{2.068624in}}%
\pgfpathcurveto{\pgfqpoint{1.419057in}{2.076437in}}{\pgfqpoint{1.423448in}{2.087036in}}{\pgfqpoint{1.423448in}{2.098087in}}%
\pgfpathcurveto{\pgfqpoint{1.423448in}{2.109137in}}{\pgfqpoint{1.419057in}{2.119736in}}{\pgfqpoint{1.411244in}{2.127549in}}%
\pgfpathcurveto{\pgfqpoint{1.403430in}{2.135363in}}{\pgfqpoint{1.392831in}{2.139753in}}{\pgfqpoint{1.381781in}{2.139753in}}%
\pgfpathcurveto{\pgfqpoint{1.370731in}{2.139753in}}{\pgfqpoint{1.360132in}{2.135363in}}{\pgfqpoint{1.352318in}{2.127549in}}%
\pgfpathcurveto{\pgfqpoint{1.344505in}{2.119736in}}{\pgfqpoint{1.340114in}{2.109137in}}{\pgfqpoint{1.340114in}{2.098087in}}%
\pgfpathcurveto{\pgfqpoint{1.340114in}{2.087036in}}{\pgfqpoint{1.344505in}{2.076437in}}{\pgfqpoint{1.352318in}{2.068624in}}%
\pgfpathcurveto{\pgfqpoint{1.360132in}{2.060810in}}{\pgfqpoint{1.370731in}{2.056420in}}{\pgfqpoint{1.381781in}{2.056420in}}%
\pgfpathlineto{\pgfqpoint{1.381781in}{2.056420in}}%
\pgfpathclose%
\pgfusepath{stroke}%
\end{pgfscope}%
\begin{pgfscope}%
\pgfpathrectangle{\pgfqpoint{0.494722in}{0.437222in}}{\pgfqpoint{6.275590in}{5.159444in}}%
\pgfusepath{clip}%
\pgfsetbuttcap%
\pgfsetroundjoin%
\pgfsetlinewidth{1.003750pt}%
\definecolor{currentstroke}{rgb}{0.827451,0.827451,0.827451}%
\pgfsetstrokecolor{currentstroke}%
\pgfsetstrokeopacity{0.800000}%
\pgfsetdash{}{0pt}%
\pgfpathmoveto{\pgfqpoint{0.834443in}{2.917471in}}%
\pgfpathcurveto{\pgfqpoint{0.845494in}{2.917471in}}{\pgfqpoint{0.856093in}{2.921861in}}{\pgfqpoint{0.863906in}{2.929674in}}%
\pgfpathcurveto{\pgfqpoint{0.871720in}{2.937488in}}{\pgfqpoint{0.876110in}{2.948087in}}{\pgfqpoint{0.876110in}{2.959137in}}%
\pgfpathcurveto{\pgfqpoint{0.876110in}{2.970187in}}{\pgfqpoint{0.871720in}{2.980786in}}{\pgfqpoint{0.863906in}{2.988600in}}%
\pgfpathcurveto{\pgfqpoint{0.856093in}{2.996414in}}{\pgfqpoint{0.845494in}{3.000804in}}{\pgfqpoint{0.834443in}{3.000804in}}%
\pgfpathcurveto{\pgfqpoint{0.823393in}{3.000804in}}{\pgfqpoint{0.812794in}{2.996414in}}{\pgfqpoint{0.804981in}{2.988600in}}%
\pgfpathcurveto{\pgfqpoint{0.797167in}{2.980786in}}{\pgfqpoint{0.792777in}{2.970187in}}{\pgfqpoint{0.792777in}{2.959137in}}%
\pgfpathcurveto{\pgfqpoint{0.792777in}{2.948087in}}{\pgfqpoint{0.797167in}{2.937488in}}{\pgfqpoint{0.804981in}{2.929674in}}%
\pgfpathcurveto{\pgfqpoint{0.812794in}{2.921861in}}{\pgfqpoint{0.823393in}{2.917471in}}{\pgfqpoint{0.834443in}{2.917471in}}%
\pgfpathlineto{\pgfqpoint{0.834443in}{2.917471in}}%
\pgfpathclose%
\pgfusepath{stroke}%
\end{pgfscope}%
\begin{pgfscope}%
\pgfpathrectangle{\pgfqpoint{0.494722in}{0.437222in}}{\pgfqpoint{6.275590in}{5.159444in}}%
\pgfusepath{clip}%
\pgfsetbuttcap%
\pgfsetroundjoin%
\pgfsetlinewidth{1.003750pt}%
\definecolor{currentstroke}{rgb}{0.827451,0.827451,0.827451}%
\pgfsetstrokecolor{currentstroke}%
\pgfsetstrokeopacity{0.800000}%
\pgfsetdash{}{0pt}%
\pgfpathmoveto{\pgfqpoint{3.369630in}{0.746512in}}%
\pgfpathcurveto{\pgfqpoint{3.380680in}{0.746512in}}{\pgfqpoint{3.391279in}{0.750903in}}{\pgfqpoint{3.399093in}{0.758716in}}%
\pgfpathcurveto{\pgfqpoint{3.406907in}{0.766530in}}{\pgfqpoint{3.411297in}{0.777129in}}{\pgfqpoint{3.411297in}{0.788179in}}%
\pgfpathcurveto{\pgfqpoint{3.411297in}{0.799229in}}{\pgfqpoint{3.406907in}{0.809828in}}{\pgfqpoint{3.399093in}{0.817642in}}%
\pgfpathcurveto{\pgfqpoint{3.391279in}{0.825455in}}{\pgfqpoint{3.380680in}{0.829846in}}{\pgfqpoint{3.369630in}{0.829846in}}%
\pgfpathcurveto{\pgfqpoint{3.358580in}{0.829846in}}{\pgfqpoint{3.347981in}{0.825455in}}{\pgfqpoint{3.340167in}{0.817642in}}%
\pgfpathcurveto{\pgfqpoint{3.332354in}{0.809828in}}{\pgfqpoint{3.327963in}{0.799229in}}{\pgfqpoint{3.327963in}{0.788179in}}%
\pgfpathcurveto{\pgfqpoint{3.327963in}{0.777129in}}{\pgfqpoint{3.332354in}{0.766530in}}{\pgfqpoint{3.340167in}{0.758716in}}%
\pgfpathcurveto{\pgfqpoint{3.347981in}{0.750903in}}{\pgfqpoint{3.358580in}{0.746512in}}{\pgfqpoint{3.369630in}{0.746512in}}%
\pgfpathlineto{\pgfqpoint{3.369630in}{0.746512in}}%
\pgfpathclose%
\pgfusepath{stroke}%
\end{pgfscope}%
\begin{pgfscope}%
\pgfpathrectangle{\pgfqpoint{0.494722in}{0.437222in}}{\pgfqpoint{6.275590in}{5.159444in}}%
\pgfusepath{clip}%
\pgfsetbuttcap%
\pgfsetroundjoin%
\pgfsetlinewidth{1.003750pt}%
\definecolor{currentstroke}{rgb}{0.827451,0.827451,0.827451}%
\pgfsetstrokecolor{currentstroke}%
\pgfsetstrokeopacity{0.800000}%
\pgfsetdash{}{0pt}%
\pgfpathmoveto{\pgfqpoint{0.517395in}{4.197711in}}%
\pgfpathcurveto{\pgfqpoint{0.528445in}{4.197711in}}{\pgfqpoint{0.539044in}{4.202101in}}{\pgfqpoint{0.546858in}{4.209915in}}%
\pgfpathcurveto{\pgfqpoint{0.554671in}{4.217728in}}{\pgfqpoint{0.559061in}{4.228327in}}{\pgfqpoint{0.559061in}{4.239378in}}%
\pgfpathcurveto{\pgfqpoint{0.559061in}{4.250428in}}{\pgfqpoint{0.554671in}{4.261027in}}{\pgfqpoint{0.546858in}{4.268840in}}%
\pgfpathcurveto{\pgfqpoint{0.539044in}{4.276654in}}{\pgfqpoint{0.528445in}{4.281044in}}{\pgfqpoint{0.517395in}{4.281044in}}%
\pgfpathcurveto{\pgfqpoint{0.506345in}{4.281044in}}{\pgfqpoint{0.495746in}{4.276654in}}{\pgfqpoint{0.487932in}{4.268840in}}%
\pgfpathcurveto{\pgfqpoint{0.480118in}{4.261027in}}{\pgfqpoint{0.475728in}{4.250428in}}{\pgfqpoint{0.475728in}{4.239378in}}%
\pgfpathcurveto{\pgfqpoint{0.475728in}{4.228327in}}{\pgfqpoint{0.480118in}{4.217728in}}{\pgfqpoint{0.487932in}{4.209915in}}%
\pgfpathcurveto{\pgfqpoint{0.495746in}{4.202101in}}{\pgfqpoint{0.506345in}{4.197711in}}{\pgfqpoint{0.517395in}{4.197711in}}%
\pgfpathlineto{\pgfqpoint{0.517395in}{4.197711in}}%
\pgfpathclose%
\pgfusepath{stroke}%
\end{pgfscope}%
\begin{pgfscope}%
\pgfpathrectangle{\pgfqpoint{0.494722in}{0.437222in}}{\pgfqpoint{6.275590in}{5.159444in}}%
\pgfusepath{clip}%
\pgfsetbuttcap%
\pgfsetroundjoin%
\pgfsetlinewidth{1.003750pt}%
\definecolor{currentstroke}{rgb}{0.827451,0.827451,0.827451}%
\pgfsetstrokecolor{currentstroke}%
\pgfsetstrokeopacity{0.800000}%
\pgfsetdash{}{0pt}%
\pgfpathmoveto{\pgfqpoint{3.123069in}{0.847718in}}%
\pgfpathcurveto{\pgfqpoint{3.134119in}{0.847718in}}{\pgfqpoint{3.144718in}{0.852109in}}{\pgfqpoint{3.152532in}{0.859922in}}%
\pgfpathcurveto{\pgfqpoint{3.160345in}{0.867736in}}{\pgfqpoint{3.164735in}{0.878335in}}{\pgfqpoint{3.164735in}{0.889385in}}%
\pgfpathcurveto{\pgfqpoint{3.164735in}{0.900435in}}{\pgfqpoint{3.160345in}{0.911034in}}{\pgfqpoint{3.152532in}{0.918848in}}%
\pgfpathcurveto{\pgfqpoint{3.144718in}{0.926661in}}{\pgfqpoint{3.134119in}{0.931052in}}{\pgfqpoint{3.123069in}{0.931052in}}%
\pgfpathcurveto{\pgfqpoint{3.112019in}{0.931052in}}{\pgfqpoint{3.101420in}{0.926661in}}{\pgfqpoint{3.093606in}{0.918848in}}%
\pgfpathcurveto{\pgfqpoint{3.085792in}{0.911034in}}{\pgfqpoint{3.081402in}{0.900435in}}{\pgfqpoint{3.081402in}{0.889385in}}%
\pgfpathcurveto{\pgfqpoint{3.081402in}{0.878335in}}{\pgfqpoint{3.085792in}{0.867736in}}{\pgfqpoint{3.093606in}{0.859922in}}%
\pgfpathcurveto{\pgfqpoint{3.101420in}{0.852109in}}{\pgfqpoint{3.112019in}{0.847718in}}{\pgfqpoint{3.123069in}{0.847718in}}%
\pgfpathlineto{\pgfqpoint{3.123069in}{0.847718in}}%
\pgfpathclose%
\pgfusepath{stroke}%
\end{pgfscope}%
\begin{pgfscope}%
\pgfpathrectangle{\pgfqpoint{0.494722in}{0.437222in}}{\pgfqpoint{6.275590in}{5.159444in}}%
\pgfusepath{clip}%
\pgfsetbuttcap%
\pgfsetroundjoin%
\pgfsetlinewidth{1.003750pt}%
\definecolor{currentstroke}{rgb}{0.827451,0.827451,0.827451}%
\pgfsetstrokecolor{currentstroke}%
\pgfsetstrokeopacity{0.800000}%
\pgfsetdash{}{0pt}%
\pgfpathmoveto{\pgfqpoint{1.227841in}{2.257546in}}%
\pgfpathcurveto{\pgfqpoint{1.238891in}{2.257546in}}{\pgfqpoint{1.249490in}{2.261936in}}{\pgfqpoint{1.257304in}{2.269750in}}%
\pgfpathcurveto{\pgfqpoint{1.265118in}{2.277564in}}{\pgfqpoint{1.269508in}{2.288163in}}{\pgfqpoint{1.269508in}{2.299213in}}%
\pgfpathcurveto{\pgfqpoint{1.269508in}{2.310263in}}{\pgfqpoint{1.265118in}{2.320862in}}{\pgfqpoint{1.257304in}{2.328675in}}%
\pgfpathcurveto{\pgfqpoint{1.249490in}{2.336489in}}{\pgfqpoint{1.238891in}{2.340879in}}{\pgfqpoint{1.227841in}{2.340879in}}%
\pgfpathcurveto{\pgfqpoint{1.216791in}{2.340879in}}{\pgfqpoint{1.206192in}{2.336489in}}{\pgfqpoint{1.198379in}{2.328675in}}%
\pgfpathcurveto{\pgfqpoint{1.190565in}{2.320862in}}{\pgfqpoint{1.186175in}{2.310263in}}{\pgfqpoint{1.186175in}{2.299213in}}%
\pgfpathcurveto{\pgfqpoint{1.186175in}{2.288163in}}{\pgfqpoint{1.190565in}{2.277564in}}{\pgfqpoint{1.198379in}{2.269750in}}%
\pgfpathcurveto{\pgfqpoint{1.206192in}{2.261936in}}{\pgfqpoint{1.216791in}{2.257546in}}{\pgfqpoint{1.227841in}{2.257546in}}%
\pgfpathlineto{\pgfqpoint{1.227841in}{2.257546in}}%
\pgfpathclose%
\pgfusepath{stroke}%
\end{pgfscope}%
\begin{pgfscope}%
\pgfpathrectangle{\pgfqpoint{0.494722in}{0.437222in}}{\pgfqpoint{6.275590in}{5.159444in}}%
\pgfusepath{clip}%
\pgfsetbuttcap%
\pgfsetroundjoin%
\pgfsetlinewidth{1.003750pt}%
\definecolor{currentstroke}{rgb}{0.827451,0.827451,0.827451}%
\pgfsetstrokecolor{currentstroke}%
\pgfsetstrokeopacity{0.800000}%
\pgfsetdash{}{0pt}%
\pgfpathmoveto{\pgfqpoint{1.596571in}{1.832310in}}%
\pgfpathcurveto{\pgfqpoint{1.607621in}{1.832310in}}{\pgfqpoint{1.618220in}{1.836700in}}{\pgfqpoint{1.626034in}{1.844514in}}%
\pgfpathcurveto{\pgfqpoint{1.633847in}{1.852327in}}{\pgfqpoint{1.638238in}{1.862926in}}{\pgfqpoint{1.638238in}{1.873976in}}%
\pgfpathcurveto{\pgfqpoint{1.638238in}{1.885027in}}{\pgfqpoint{1.633847in}{1.895626in}}{\pgfqpoint{1.626034in}{1.903439in}}%
\pgfpathcurveto{\pgfqpoint{1.618220in}{1.911253in}}{\pgfqpoint{1.607621in}{1.915643in}}{\pgfqpoint{1.596571in}{1.915643in}}%
\pgfpathcurveto{\pgfqpoint{1.585521in}{1.915643in}}{\pgfqpoint{1.574922in}{1.911253in}}{\pgfqpoint{1.567108in}{1.903439in}}%
\pgfpathcurveto{\pgfqpoint{1.559295in}{1.895626in}}{\pgfqpoint{1.554904in}{1.885027in}}{\pgfqpoint{1.554904in}{1.873976in}}%
\pgfpathcurveto{\pgfqpoint{1.554904in}{1.862926in}}{\pgfqpoint{1.559295in}{1.852327in}}{\pgfqpoint{1.567108in}{1.844514in}}%
\pgfpathcurveto{\pgfqpoint{1.574922in}{1.836700in}}{\pgfqpoint{1.585521in}{1.832310in}}{\pgfqpoint{1.596571in}{1.832310in}}%
\pgfpathlineto{\pgfqpoint{1.596571in}{1.832310in}}%
\pgfpathclose%
\pgfusepath{stroke}%
\end{pgfscope}%
\begin{pgfscope}%
\pgfpathrectangle{\pgfqpoint{0.494722in}{0.437222in}}{\pgfqpoint{6.275590in}{5.159444in}}%
\pgfusepath{clip}%
\pgfsetbuttcap%
\pgfsetroundjoin%
\pgfsetlinewidth{1.003750pt}%
\definecolor{currentstroke}{rgb}{0.827451,0.827451,0.827451}%
\pgfsetstrokecolor{currentstroke}%
\pgfsetstrokeopacity{0.800000}%
\pgfsetdash{}{0pt}%
\pgfpathmoveto{\pgfqpoint{0.527771in}{4.064004in}}%
\pgfpathcurveto{\pgfqpoint{0.538821in}{4.064004in}}{\pgfqpoint{0.549420in}{4.068394in}}{\pgfqpoint{0.557234in}{4.076208in}}%
\pgfpathcurveto{\pgfqpoint{0.565047in}{4.084022in}}{\pgfqpoint{0.569438in}{4.094621in}}{\pgfqpoint{0.569438in}{4.105671in}}%
\pgfpathcurveto{\pgfqpoint{0.569438in}{4.116721in}}{\pgfqpoint{0.565047in}{4.127320in}}{\pgfqpoint{0.557234in}{4.135134in}}%
\pgfpathcurveto{\pgfqpoint{0.549420in}{4.142947in}}{\pgfqpoint{0.538821in}{4.147338in}}{\pgfqpoint{0.527771in}{4.147338in}}%
\pgfpathcurveto{\pgfqpoint{0.516721in}{4.147338in}}{\pgfqpoint{0.506122in}{4.142947in}}{\pgfqpoint{0.498308in}{4.135134in}}%
\pgfpathcurveto{\pgfqpoint{0.490494in}{4.127320in}}{\pgfqpoint{0.486104in}{4.116721in}}{\pgfqpoint{0.486104in}{4.105671in}}%
\pgfpathcurveto{\pgfqpoint{0.486104in}{4.094621in}}{\pgfqpoint{0.490494in}{4.084022in}}{\pgfqpoint{0.498308in}{4.076208in}}%
\pgfpathcurveto{\pgfqpoint{0.506122in}{4.068394in}}{\pgfqpoint{0.516721in}{4.064004in}}{\pgfqpoint{0.527771in}{4.064004in}}%
\pgfpathlineto{\pgfqpoint{0.527771in}{4.064004in}}%
\pgfpathclose%
\pgfusepath{stroke}%
\end{pgfscope}%
\begin{pgfscope}%
\pgfpathrectangle{\pgfqpoint{0.494722in}{0.437222in}}{\pgfqpoint{6.275590in}{5.159444in}}%
\pgfusepath{clip}%
\pgfsetbuttcap%
\pgfsetroundjoin%
\pgfsetlinewidth{1.003750pt}%
\definecolor{currentstroke}{rgb}{0.827451,0.827451,0.827451}%
\pgfsetstrokecolor{currentstroke}%
\pgfsetstrokeopacity{0.800000}%
\pgfsetdash{}{0pt}%
\pgfpathmoveto{\pgfqpoint{2.081578in}{1.430058in}}%
\pgfpathcurveto{\pgfqpoint{2.092628in}{1.430058in}}{\pgfqpoint{2.103227in}{1.434449in}}{\pgfqpoint{2.111040in}{1.442262in}}%
\pgfpathcurveto{\pgfqpoint{2.118854in}{1.450076in}}{\pgfqpoint{2.123244in}{1.460675in}}{\pgfqpoint{2.123244in}{1.471725in}}%
\pgfpathcurveto{\pgfqpoint{2.123244in}{1.482775in}}{\pgfqpoint{2.118854in}{1.493374in}}{\pgfqpoint{2.111040in}{1.501188in}}%
\pgfpathcurveto{\pgfqpoint{2.103227in}{1.509001in}}{\pgfqpoint{2.092628in}{1.513392in}}{\pgfqpoint{2.081578in}{1.513392in}}%
\pgfpathcurveto{\pgfqpoint{2.070527in}{1.513392in}}{\pgfqpoint{2.059928in}{1.509001in}}{\pgfqpoint{2.052115in}{1.501188in}}%
\pgfpathcurveto{\pgfqpoint{2.044301in}{1.493374in}}{\pgfqpoint{2.039911in}{1.482775in}}{\pgfqpoint{2.039911in}{1.471725in}}%
\pgfpathcurveto{\pgfqpoint{2.039911in}{1.460675in}}{\pgfqpoint{2.044301in}{1.450076in}}{\pgfqpoint{2.052115in}{1.442262in}}%
\pgfpathcurveto{\pgfqpoint{2.059928in}{1.434449in}}{\pgfqpoint{2.070527in}{1.430058in}}{\pgfqpoint{2.081578in}{1.430058in}}%
\pgfpathlineto{\pgfqpoint{2.081578in}{1.430058in}}%
\pgfpathclose%
\pgfusepath{stroke}%
\end{pgfscope}%
\begin{pgfscope}%
\pgfpathrectangle{\pgfqpoint{0.494722in}{0.437222in}}{\pgfqpoint{6.275590in}{5.159444in}}%
\pgfusepath{clip}%
\pgfsetbuttcap%
\pgfsetroundjoin%
\pgfsetlinewidth{1.003750pt}%
\definecolor{currentstroke}{rgb}{0.827451,0.827451,0.827451}%
\pgfsetstrokecolor{currentstroke}%
\pgfsetstrokeopacity{0.800000}%
\pgfsetdash{}{0pt}%
\pgfpathmoveto{\pgfqpoint{0.711186in}{3.233934in}}%
\pgfpathcurveto{\pgfqpoint{0.722236in}{3.233934in}}{\pgfqpoint{0.732835in}{3.238324in}}{\pgfqpoint{0.740648in}{3.246138in}}%
\pgfpathcurveto{\pgfqpoint{0.748462in}{3.253952in}}{\pgfqpoint{0.752852in}{3.264551in}}{\pgfqpoint{0.752852in}{3.275601in}}%
\pgfpathcurveto{\pgfqpoint{0.752852in}{3.286651in}}{\pgfqpoint{0.748462in}{3.297250in}}{\pgfqpoint{0.740648in}{3.305064in}}%
\pgfpathcurveto{\pgfqpoint{0.732835in}{3.312877in}}{\pgfqpoint{0.722236in}{3.317267in}}{\pgfqpoint{0.711186in}{3.317267in}}%
\pgfpathcurveto{\pgfqpoint{0.700135in}{3.317267in}}{\pgfqpoint{0.689536in}{3.312877in}}{\pgfqpoint{0.681723in}{3.305064in}}%
\pgfpathcurveto{\pgfqpoint{0.673909in}{3.297250in}}{\pgfqpoint{0.669519in}{3.286651in}}{\pgfqpoint{0.669519in}{3.275601in}}%
\pgfpathcurveto{\pgfqpoint{0.669519in}{3.264551in}}{\pgfqpoint{0.673909in}{3.253952in}}{\pgfqpoint{0.681723in}{3.246138in}}%
\pgfpathcurveto{\pgfqpoint{0.689536in}{3.238324in}}{\pgfqpoint{0.700135in}{3.233934in}}{\pgfqpoint{0.711186in}{3.233934in}}%
\pgfpathlineto{\pgfqpoint{0.711186in}{3.233934in}}%
\pgfpathclose%
\pgfusepath{stroke}%
\end{pgfscope}%
\begin{pgfscope}%
\pgfpathrectangle{\pgfqpoint{0.494722in}{0.437222in}}{\pgfqpoint{6.275590in}{5.159444in}}%
\pgfusepath{clip}%
\pgfsetbuttcap%
\pgfsetroundjoin%
\pgfsetlinewidth{1.003750pt}%
\definecolor{currentstroke}{rgb}{0.827451,0.827451,0.827451}%
\pgfsetstrokecolor{currentstroke}%
\pgfsetstrokeopacity{0.800000}%
\pgfsetdash{}{0pt}%
\pgfpathmoveto{\pgfqpoint{1.444344in}{1.986059in}}%
\pgfpathcurveto{\pgfqpoint{1.455394in}{1.986059in}}{\pgfqpoint{1.465993in}{1.990450in}}{\pgfqpoint{1.473807in}{1.998263in}}%
\pgfpathcurveto{\pgfqpoint{1.481620in}{2.006077in}}{\pgfqpoint{1.486010in}{2.016676in}}{\pgfqpoint{1.486010in}{2.027726in}}%
\pgfpathcurveto{\pgfqpoint{1.486010in}{2.038776in}}{\pgfqpoint{1.481620in}{2.049375in}}{\pgfqpoint{1.473807in}{2.057189in}}%
\pgfpathcurveto{\pgfqpoint{1.465993in}{2.065002in}}{\pgfqpoint{1.455394in}{2.069393in}}{\pgfqpoint{1.444344in}{2.069393in}}%
\pgfpathcurveto{\pgfqpoint{1.433294in}{2.069393in}}{\pgfqpoint{1.422695in}{2.065002in}}{\pgfqpoint{1.414881in}{2.057189in}}%
\pgfpathcurveto{\pgfqpoint{1.407067in}{2.049375in}}{\pgfqpoint{1.402677in}{2.038776in}}{\pgfqpoint{1.402677in}{2.027726in}}%
\pgfpathcurveto{\pgfqpoint{1.402677in}{2.016676in}}{\pgfqpoint{1.407067in}{2.006077in}}{\pgfqpoint{1.414881in}{1.998263in}}%
\pgfpathcurveto{\pgfqpoint{1.422695in}{1.990450in}}{\pgfqpoint{1.433294in}{1.986059in}}{\pgfqpoint{1.444344in}{1.986059in}}%
\pgfpathlineto{\pgfqpoint{1.444344in}{1.986059in}}%
\pgfpathclose%
\pgfusepath{stroke}%
\end{pgfscope}%
\begin{pgfscope}%
\pgfpathrectangle{\pgfqpoint{0.494722in}{0.437222in}}{\pgfqpoint{6.275590in}{5.159444in}}%
\pgfusepath{clip}%
\pgfsetbuttcap%
\pgfsetroundjoin%
\pgfsetlinewidth{1.003750pt}%
\definecolor{currentstroke}{rgb}{0.827451,0.827451,0.827451}%
\pgfsetstrokecolor{currentstroke}%
\pgfsetstrokeopacity{0.800000}%
\pgfsetdash{}{0pt}%
\pgfpathmoveto{\pgfqpoint{0.552392in}{3.964585in}}%
\pgfpathcurveto{\pgfqpoint{0.563443in}{3.964585in}}{\pgfqpoint{0.574042in}{3.968976in}}{\pgfqpoint{0.581855in}{3.976789in}}%
\pgfpathcurveto{\pgfqpoint{0.589669in}{3.984603in}}{\pgfqpoint{0.594059in}{3.995202in}}{\pgfqpoint{0.594059in}{4.006252in}}%
\pgfpathcurveto{\pgfqpoint{0.594059in}{4.017302in}}{\pgfqpoint{0.589669in}{4.027901in}}{\pgfqpoint{0.581855in}{4.035715in}}%
\pgfpathcurveto{\pgfqpoint{0.574042in}{4.043529in}}{\pgfqpoint{0.563443in}{4.047919in}}{\pgfqpoint{0.552392in}{4.047919in}}%
\pgfpathcurveto{\pgfqpoint{0.541342in}{4.047919in}}{\pgfqpoint{0.530743in}{4.043529in}}{\pgfqpoint{0.522930in}{4.035715in}}%
\pgfpathcurveto{\pgfqpoint{0.515116in}{4.027901in}}{\pgfqpoint{0.510726in}{4.017302in}}{\pgfqpoint{0.510726in}{4.006252in}}%
\pgfpathcurveto{\pgfqpoint{0.510726in}{3.995202in}}{\pgfqpoint{0.515116in}{3.984603in}}{\pgfqpoint{0.522930in}{3.976789in}}%
\pgfpathcurveto{\pgfqpoint{0.530743in}{3.968976in}}{\pgfqpoint{0.541342in}{3.964585in}}{\pgfqpoint{0.552392in}{3.964585in}}%
\pgfpathlineto{\pgfqpoint{0.552392in}{3.964585in}}%
\pgfpathclose%
\pgfusepath{stroke}%
\end{pgfscope}%
\begin{pgfscope}%
\pgfpathrectangle{\pgfqpoint{0.494722in}{0.437222in}}{\pgfqpoint{6.275590in}{5.159444in}}%
\pgfusepath{clip}%
\pgfsetbuttcap%
\pgfsetroundjoin%
\pgfsetlinewidth{1.003750pt}%
\definecolor{currentstroke}{rgb}{0.827451,0.827451,0.827451}%
\pgfsetstrokecolor{currentstroke}%
\pgfsetstrokeopacity{0.800000}%
\pgfsetdash{}{0pt}%
\pgfpathmoveto{\pgfqpoint{1.112658in}{2.512839in}}%
\pgfpathcurveto{\pgfqpoint{1.123708in}{2.512839in}}{\pgfqpoint{1.134307in}{2.517229in}}{\pgfqpoint{1.142121in}{2.525043in}}%
\pgfpathcurveto{\pgfqpoint{1.149935in}{2.532856in}}{\pgfqpoint{1.154325in}{2.543455in}}{\pgfqpoint{1.154325in}{2.554505in}}%
\pgfpathcurveto{\pgfqpoint{1.154325in}{2.565556in}}{\pgfqpoint{1.149935in}{2.576155in}}{\pgfqpoint{1.142121in}{2.583968in}}%
\pgfpathcurveto{\pgfqpoint{1.134307in}{2.591782in}}{\pgfqpoint{1.123708in}{2.596172in}}{\pgfqpoint{1.112658in}{2.596172in}}%
\pgfpathcurveto{\pgfqpoint{1.101608in}{2.596172in}}{\pgfqpoint{1.091009in}{2.591782in}}{\pgfqpoint{1.083195in}{2.583968in}}%
\pgfpathcurveto{\pgfqpoint{1.075382in}{2.576155in}}{\pgfqpoint{1.070992in}{2.565556in}}{\pgfqpoint{1.070992in}{2.554505in}}%
\pgfpathcurveto{\pgfqpoint{1.070992in}{2.543455in}}{\pgfqpoint{1.075382in}{2.532856in}}{\pgfqpoint{1.083195in}{2.525043in}}%
\pgfpathcurveto{\pgfqpoint{1.091009in}{2.517229in}}{\pgfqpoint{1.101608in}{2.512839in}}{\pgfqpoint{1.112658in}{2.512839in}}%
\pgfpathlineto{\pgfqpoint{1.112658in}{2.512839in}}%
\pgfpathclose%
\pgfusepath{stroke}%
\end{pgfscope}%
\begin{pgfscope}%
\pgfpathrectangle{\pgfqpoint{0.494722in}{0.437222in}}{\pgfqpoint{6.275590in}{5.159444in}}%
\pgfusepath{clip}%
\pgfsetbuttcap%
\pgfsetroundjoin%
\pgfsetlinewidth{1.003750pt}%
\definecolor{currentstroke}{rgb}{0.827451,0.827451,0.827451}%
\pgfsetstrokecolor{currentstroke}%
\pgfsetstrokeopacity{0.800000}%
\pgfsetdash{}{0pt}%
\pgfpathmoveto{\pgfqpoint{1.554134in}{1.871128in}}%
\pgfpathcurveto{\pgfqpoint{1.565185in}{1.871128in}}{\pgfqpoint{1.575784in}{1.875518in}}{\pgfqpoint{1.583597in}{1.883332in}}%
\pgfpathcurveto{\pgfqpoint{1.591411in}{1.891146in}}{\pgfqpoint{1.595801in}{1.901745in}}{\pgfqpoint{1.595801in}{1.912795in}}%
\pgfpathcurveto{\pgfqpoint{1.595801in}{1.923845in}}{\pgfqpoint{1.591411in}{1.934444in}}{\pgfqpoint{1.583597in}{1.942258in}}%
\pgfpathcurveto{\pgfqpoint{1.575784in}{1.950071in}}{\pgfqpoint{1.565185in}{1.954462in}}{\pgfqpoint{1.554134in}{1.954462in}}%
\pgfpathcurveto{\pgfqpoint{1.543084in}{1.954462in}}{\pgfqpoint{1.532485in}{1.950071in}}{\pgfqpoint{1.524672in}{1.942258in}}%
\pgfpathcurveto{\pgfqpoint{1.516858in}{1.934444in}}{\pgfqpoint{1.512468in}{1.923845in}}{\pgfqpoint{1.512468in}{1.912795in}}%
\pgfpathcurveto{\pgfqpoint{1.512468in}{1.901745in}}{\pgfqpoint{1.516858in}{1.891146in}}{\pgfqpoint{1.524672in}{1.883332in}}%
\pgfpathcurveto{\pgfqpoint{1.532485in}{1.875518in}}{\pgfqpoint{1.543084in}{1.871128in}}{\pgfqpoint{1.554134in}{1.871128in}}%
\pgfpathlineto{\pgfqpoint{1.554134in}{1.871128in}}%
\pgfpathclose%
\pgfusepath{stroke}%
\end{pgfscope}%
\begin{pgfscope}%
\pgfpathrectangle{\pgfqpoint{0.494722in}{0.437222in}}{\pgfqpoint{6.275590in}{5.159444in}}%
\pgfusepath{clip}%
\pgfsetbuttcap%
\pgfsetroundjoin%
\pgfsetlinewidth{1.003750pt}%
\definecolor{currentstroke}{rgb}{0.827451,0.827451,0.827451}%
\pgfsetstrokecolor{currentstroke}%
\pgfsetstrokeopacity{0.800000}%
\pgfsetdash{}{0pt}%
\pgfpathmoveto{\pgfqpoint{4.229480in}{0.518110in}}%
\pgfpathcurveto{\pgfqpoint{4.240530in}{0.518110in}}{\pgfqpoint{4.251129in}{0.522500in}}{\pgfqpoint{4.258942in}{0.530314in}}%
\pgfpathcurveto{\pgfqpoint{4.266756in}{0.538127in}}{\pgfqpoint{4.271146in}{0.548726in}}{\pgfqpoint{4.271146in}{0.559776in}}%
\pgfpathcurveto{\pgfqpoint{4.271146in}{0.570827in}}{\pgfqpoint{4.266756in}{0.581426in}}{\pgfqpoint{4.258942in}{0.589239in}}%
\pgfpathcurveto{\pgfqpoint{4.251129in}{0.597053in}}{\pgfqpoint{4.240530in}{0.601443in}}{\pgfqpoint{4.229480in}{0.601443in}}%
\pgfpathcurveto{\pgfqpoint{4.218429in}{0.601443in}}{\pgfqpoint{4.207830in}{0.597053in}}{\pgfqpoint{4.200017in}{0.589239in}}%
\pgfpathcurveto{\pgfqpoint{4.192203in}{0.581426in}}{\pgfqpoint{4.187813in}{0.570827in}}{\pgfqpoint{4.187813in}{0.559776in}}%
\pgfpathcurveto{\pgfqpoint{4.187813in}{0.548726in}}{\pgfqpoint{4.192203in}{0.538127in}}{\pgfqpoint{4.200017in}{0.530314in}}%
\pgfpathcurveto{\pgfqpoint{4.207830in}{0.522500in}}{\pgfqpoint{4.218429in}{0.518110in}}{\pgfqpoint{4.229480in}{0.518110in}}%
\pgfpathlineto{\pgfqpoint{4.229480in}{0.518110in}}%
\pgfpathclose%
\pgfusepath{stroke}%
\end{pgfscope}%
\begin{pgfscope}%
\pgfpathrectangle{\pgfqpoint{0.494722in}{0.437222in}}{\pgfqpoint{6.275590in}{5.159444in}}%
\pgfusepath{clip}%
\pgfsetbuttcap%
\pgfsetroundjoin%
\pgfsetlinewidth{1.003750pt}%
\definecolor{currentstroke}{rgb}{0.827451,0.827451,0.827451}%
\pgfsetstrokecolor{currentstroke}%
\pgfsetstrokeopacity{0.800000}%
\pgfsetdash{}{0pt}%
\pgfpathmoveto{\pgfqpoint{1.464755in}{1.963622in}}%
\pgfpathcurveto{\pgfqpoint{1.475805in}{1.963622in}}{\pgfqpoint{1.486404in}{1.968013in}}{\pgfqpoint{1.494217in}{1.975826in}}%
\pgfpathcurveto{\pgfqpoint{1.502031in}{1.983640in}}{\pgfqpoint{1.506421in}{1.994239in}}{\pgfqpoint{1.506421in}{2.005289in}}%
\pgfpathcurveto{\pgfqpoint{1.506421in}{2.016339in}}{\pgfqpoint{1.502031in}{2.026938in}}{\pgfqpoint{1.494217in}{2.034752in}}%
\pgfpathcurveto{\pgfqpoint{1.486404in}{2.042566in}}{\pgfqpoint{1.475805in}{2.046956in}}{\pgfqpoint{1.464755in}{2.046956in}}%
\pgfpathcurveto{\pgfqpoint{1.453705in}{2.046956in}}{\pgfqpoint{1.443106in}{2.042566in}}{\pgfqpoint{1.435292in}{2.034752in}}%
\pgfpathcurveto{\pgfqpoint{1.427478in}{2.026938in}}{\pgfqpoint{1.423088in}{2.016339in}}{\pgfqpoint{1.423088in}{2.005289in}}%
\pgfpathcurveto{\pgfqpoint{1.423088in}{1.994239in}}{\pgfqpoint{1.427478in}{1.983640in}}{\pgfqpoint{1.435292in}{1.975826in}}%
\pgfpathcurveto{\pgfqpoint{1.443106in}{1.968013in}}{\pgfqpoint{1.453705in}{1.963622in}}{\pgfqpoint{1.464755in}{1.963622in}}%
\pgfpathlineto{\pgfqpoint{1.464755in}{1.963622in}}%
\pgfpathclose%
\pgfusepath{stroke}%
\end{pgfscope}%
\begin{pgfscope}%
\pgfpathrectangle{\pgfqpoint{0.494722in}{0.437222in}}{\pgfqpoint{6.275590in}{5.159444in}}%
\pgfusepath{clip}%
\pgfsetbuttcap%
\pgfsetroundjoin%
\pgfsetlinewidth{1.003750pt}%
\definecolor{currentstroke}{rgb}{0.827451,0.827451,0.827451}%
\pgfsetstrokecolor{currentstroke}%
\pgfsetstrokeopacity{0.800000}%
\pgfsetdash{}{0pt}%
\pgfpathmoveto{\pgfqpoint{0.595178in}{3.651665in}}%
\pgfpathcurveto{\pgfqpoint{0.606228in}{3.651665in}}{\pgfqpoint{0.616827in}{3.656056in}}{\pgfqpoint{0.624640in}{3.663869in}}%
\pgfpathcurveto{\pgfqpoint{0.632454in}{3.671683in}}{\pgfqpoint{0.636844in}{3.682282in}}{\pgfqpoint{0.636844in}{3.693332in}}%
\pgfpathcurveto{\pgfqpoint{0.636844in}{3.704382in}}{\pgfqpoint{0.632454in}{3.714981in}}{\pgfqpoint{0.624640in}{3.722795in}}%
\pgfpathcurveto{\pgfqpoint{0.616827in}{3.730609in}}{\pgfqpoint{0.606228in}{3.734999in}}{\pgfqpoint{0.595178in}{3.734999in}}%
\pgfpathcurveto{\pgfqpoint{0.584128in}{3.734999in}}{\pgfqpoint{0.573529in}{3.730609in}}{\pgfqpoint{0.565715in}{3.722795in}}%
\pgfpathcurveto{\pgfqpoint{0.557901in}{3.714981in}}{\pgfqpoint{0.553511in}{3.704382in}}{\pgfqpoint{0.553511in}{3.693332in}}%
\pgfpathcurveto{\pgfqpoint{0.553511in}{3.682282in}}{\pgfqpoint{0.557901in}{3.671683in}}{\pgfqpoint{0.565715in}{3.663869in}}%
\pgfpathcurveto{\pgfqpoint{0.573529in}{3.656056in}}{\pgfqpoint{0.584128in}{3.651665in}}{\pgfqpoint{0.595178in}{3.651665in}}%
\pgfpathlineto{\pgfqpoint{0.595178in}{3.651665in}}%
\pgfpathclose%
\pgfusepath{stroke}%
\end{pgfscope}%
\begin{pgfscope}%
\pgfpathrectangle{\pgfqpoint{0.494722in}{0.437222in}}{\pgfqpoint{6.275590in}{5.159444in}}%
\pgfusepath{clip}%
\pgfsetbuttcap%
\pgfsetroundjoin%
\pgfsetlinewidth{1.003750pt}%
\definecolor{currentstroke}{rgb}{0.827451,0.827451,0.827451}%
\pgfsetstrokecolor{currentstroke}%
\pgfsetstrokeopacity{0.800000}%
\pgfsetdash{}{0pt}%
\pgfpathmoveto{\pgfqpoint{0.870505in}{2.877763in}}%
\pgfpathcurveto{\pgfqpoint{0.881555in}{2.877763in}}{\pgfqpoint{0.892154in}{2.882154in}}{\pgfqpoint{0.899967in}{2.889967in}}%
\pgfpathcurveto{\pgfqpoint{0.907781in}{2.897781in}}{\pgfqpoint{0.912171in}{2.908380in}}{\pgfqpoint{0.912171in}{2.919430in}}%
\pgfpathcurveto{\pgfqpoint{0.912171in}{2.930480in}}{\pgfqpoint{0.907781in}{2.941079in}}{\pgfqpoint{0.899967in}{2.948893in}}%
\pgfpathcurveto{\pgfqpoint{0.892154in}{2.956706in}}{\pgfqpoint{0.881555in}{2.961097in}}{\pgfqpoint{0.870505in}{2.961097in}}%
\pgfpathcurveto{\pgfqpoint{0.859455in}{2.961097in}}{\pgfqpoint{0.848856in}{2.956706in}}{\pgfqpoint{0.841042in}{2.948893in}}%
\pgfpathcurveto{\pgfqpoint{0.833228in}{2.941079in}}{\pgfqpoint{0.828838in}{2.930480in}}{\pgfqpoint{0.828838in}{2.919430in}}%
\pgfpathcurveto{\pgfqpoint{0.828838in}{2.908380in}}{\pgfqpoint{0.833228in}{2.897781in}}{\pgfqpoint{0.841042in}{2.889967in}}%
\pgfpathcurveto{\pgfqpoint{0.848856in}{2.882154in}}{\pgfqpoint{0.859455in}{2.877763in}}{\pgfqpoint{0.870505in}{2.877763in}}%
\pgfpathlineto{\pgfqpoint{0.870505in}{2.877763in}}%
\pgfpathclose%
\pgfusepath{stroke}%
\end{pgfscope}%
\begin{pgfscope}%
\pgfpathrectangle{\pgfqpoint{0.494722in}{0.437222in}}{\pgfqpoint{6.275590in}{5.159444in}}%
\pgfusepath{clip}%
\pgfsetbuttcap%
\pgfsetroundjoin%
\pgfsetlinewidth{1.003750pt}%
\definecolor{currentstroke}{rgb}{0.827451,0.827451,0.827451}%
\pgfsetstrokecolor{currentstroke}%
\pgfsetstrokeopacity{0.800000}%
\pgfsetdash{}{0pt}%
\pgfpathmoveto{\pgfqpoint{0.498838in}{4.458588in}}%
\pgfpathcurveto{\pgfqpoint{0.509888in}{4.458588in}}{\pgfqpoint{0.520487in}{4.462978in}}{\pgfqpoint{0.528301in}{4.470792in}}%
\pgfpathcurveto{\pgfqpoint{0.536115in}{4.478605in}}{\pgfqpoint{0.540505in}{4.489204in}}{\pgfqpoint{0.540505in}{4.500255in}}%
\pgfpathcurveto{\pgfqpoint{0.540505in}{4.511305in}}{\pgfqpoint{0.536115in}{4.521904in}}{\pgfqpoint{0.528301in}{4.529717in}}%
\pgfpathcurveto{\pgfqpoint{0.520487in}{4.537531in}}{\pgfqpoint{0.509888in}{4.541921in}}{\pgfqpoint{0.498838in}{4.541921in}}%
\pgfpathcurveto{\pgfqpoint{0.487788in}{4.541921in}}{\pgfqpoint{0.477189in}{4.537531in}}{\pgfqpoint{0.469375in}{4.529717in}}%
\pgfpathcurveto{\pgfqpoint{0.461562in}{4.521904in}}{\pgfqpoint{0.457172in}{4.511305in}}{\pgfqpoint{0.457172in}{4.500255in}}%
\pgfpathcurveto{\pgfqpoint{0.457172in}{4.489204in}}{\pgfqpoint{0.461562in}{4.478605in}}{\pgfqpoint{0.469375in}{4.470792in}}%
\pgfpathcurveto{\pgfqpoint{0.477189in}{4.462978in}}{\pgfqpoint{0.487788in}{4.458588in}}{\pgfqpoint{0.498838in}{4.458588in}}%
\pgfpathlineto{\pgfqpoint{0.498838in}{4.458588in}}%
\pgfpathclose%
\pgfusepath{stroke}%
\end{pgfscope}%
\begin{pgfscope}%
\pgfpathrectangle{\pgfqpoint{0.494722in}{0.437222in}}{\pgfqpoint{6.275590in}{5.159444in}}%
\pgfusepath{clip}%
\pgfsetbuttcap%
\pgfsetroundjoin%
\pgfsetlinewidth{1.003750pt}%
\definecolor{currentstroke}{rgb}{0.827451,0.827451,0.827451}%
\pgfsetstrokecolor{currentstroke}%
\pgfsetstrokeopacity{0.800000}%
\pgfsetdash{}{0pt}%
\pgfpathmoveto{\pgfqpoint{0.498024in}{4.460098in}}%
\pgfpathcurveto{\pgfqpoint{0.509074in}{4.460098in}}{\pgfqpoint{0.519673in}{4.464488in}}{\pgfqpoint{0.527487in}{4.472302in}}%
\pgfpathcurveto{\pgfqpoint{0.535301in}{4.480116in}}{\pgfqpoint{0.539691in}{4.490715in}}{\pgfqpoint{0.539691in}{4.501765in}}%
\pgfpathcurveto{\pgfqpoint{0.539691in}{4.512815in}}{\pgfqpoint{0.535301in}{4.523414in}}{\pgfqpoint{0.527487in}{4.531228in}}%
\pgfpathcurveto{\pgfqpoint{0.519673in}{4.539041in}}{\pgfqpoint{0.509074in}{4.543432in}}{\pgfqpoint{0.498024in}{4.543432in}}%
\pgfpathcurveto{\pgfqpoint{0.486974in}{4.543432in}}{\pgfqpoint{0.476375in}{4.539041in}}{\pgfqpoint{0.468561in}{4.531228in}}%
\pgfpathcurveto{\pgfqpoint{0.460748in}{4.523414in}}{\pgfqpoint{0.456358in}{4.512815in}}{\pgfqpoint{0.456358in}{4.501765in}}%
\pgfpathcurveto{\pgfqpoint{0.456358in}{4.490715in}}{\pgfqpoint{0.460748in}{4.480116in}}{\pgfqpoint{0.468561in}{4.472302in}}%
\pgfpathcurveto{\pgfqpoint{0.476375in}{4.464488in}}{\pgfqpoint{0.486974in}{4.460098in}}{\pgfqpoint{0.498024in}{4.460098in}}%
\pgfpathlineto{\pgfqpoint{0.498024in}{4.460098in}}%
\pgfpathclose%
\pgfusepath{stroke}%
\end{pgfscope}%
\begin{pgfscope}%
\pgfpathrectangle{\pgfqpoint{0.494722in}{0.437222in}}{\pgfqpoint{6.275590in}{5.159444in}}%
\pgfusepath{clip}%
\pgfsetbuttcap%
\pgfsetroundjoin%
\pgfsetlinewidth{1.003750pt}%
\definecolor{currentstroke}{rgb}{0.827451,0.827451,0.827451}%
\pgfsetstrokecolor{currentstroke}%
\pgfsetstrokeopacity{0.800000}%
\pgfsetdash{}{0pt}%
\pgfpathmoveto{\pgfqpoint{5.757662in}{0.395992in}}%
\pgfpathcurveto{\pgfqpoint{5.768712in}{0.395992in}}{\pgfqpoint{5.779311in}{0.400382in}}{\pgfqpoint{5.787125in}{0.408196in}}%
\pgfpathcurveto{\pgfqpoint{5.794939in}{0.416009in}}{\pgfqpoint{5.799329in}{0.426608in}}{\pgfqpoint{5.799329in}{0.437658in}}%
\pgfpathcurveto{\pgfqpoint{5.799329in}{0.448709in}}{\pgfqpoint{5.794939in}{0.459308in}}{\pgfqpoint{5.787125in}{0.467121in}}%
\pgfpathcurveto{\pgfqpoint{5.779311in}{0.474935in}}{\pgfqpoint{5.768712in}{0.479325in}}{\pgfqpoint{5.757662in}{0.479325in}}%
\pgfpathcurveto{\pgfqpoint{5.746612in}{0.479325in}}{\pgfqpoint{5.736013in}{0.474935in}}{\pgfqpoint{5.728199in}{0.467121in}}%
\pgfpathcurveto{\pgfqpoint{5.720386in}{0.459308in}}{\pgfqpoint{5.715996in}{0.448709in}}{\pgfqpoint{5.715996in}{0.437658in}}%
\pgfpathcurveto{\pgfqpoint{5.715996in}{0.426608in}}{\pgfqpoint{5.720386in}{0.416009in}}{\pgfqpoint{5.728199in}{0.408196in}}%
\pgfpathcurveto{\pgfqpoint{5.736013in}{0.400382in}}{\pgfqpoint{5.746612in}{0.395992in}}{\pgfqpoint{5.757662in}{0.395992in}}%
\pgfusepath{stroke}%
\end{pgfscope}%
\begin{pgfscope}%
\pgfpathrectangle{\pgfqpoint{0.494722in}{0.437222in}}{\pgfqpoint{6.275590in}{5.159444in}}%
\pgfusepath{clip}%
\pgfsetbuttcap%
\pgfsetroundjoin%
\pgfsetlinewidth{1.003750pt}%
\definecolor{currentstroke}{rgb}{0.827451,0.827451,0.827451}%
\pgfsetstrokecolor{currentstroke}%
\pgfsetstrokeopacity{0.800000}%
\pgfsetdash{}{0pt}%
\pgfpathmoveto{\pgfqpoint{0.494942in}{4.585023in}}%
\pgfpathcurveto{\pgfqpoint{0.505992in}{4.585023in}}{\pgfqpoint{0.516591in}{4.589413in}}{\pgfqpoint{0.524405in}{4.597227in}}%
\pgfpathcurveto{\pgfqpoint{0.532219in}{4.605040in}}{\pgfqpoint{0.536609in}{4.615639in}}{\pgfqpoint{0.536609in}{4.626689in}}%
\pgfpathcurveto{\pgfqpoint{0.536609in}{4.637740in}}{\pgfqpoint{0.532219in}{4.648339in}}{\pgfqpoint{0.524405in}{4.656152in}}%
\pgfpathcurveto{\pgfqpoint{0.516591in}{4.663966in}}{\pgfqpoint{0.505992in}{4.668356in}}{\pgfqpoint{0.494942in}{4.668356in}}%
\pgfpathcurveto{\pgfqpoint{0.483892in}{4.668356in}}{\pgfqpoint{0.473293in}{4.663966in}}{\pgfqpoint{0.465479in}{4.656152in}}%
\pgfpathcurveto{\pgfqpoint{0.457666in}{4.648339in}}{\pgfqpoint{0.453275in}{4.637740in}}{\pgfqpoint{0.453275in}{4.626689in}}%
\pgfpathcurveto{\pgfqpoint{0.453275in}{4.615639in}}{\pgfqpoint{0.457666in}{4.605040in}}{\pgfqpoint{0.465479in}{4.597227in}}%
\pgfpathcurveto{\pgfqpoint{0.473293in}{4.589413in}}{\pgfqpoint{0.483892in}{4.585023in}}{\pgfqpoint{0.494942in}{4.585023in}}%
\pgfpathlineto{\pgfqpoint{0.494942in}{4.585023in}}%
\pgfpathclose%
\pgfusepath{stroke}%
\end{pgfscope}%
\begin{pgfscope}%
\pgfpathrectangle{\pgfqpoint{0.494722in}{0.437222in}}{\pgfqpoint{6.275590in}{5.159444in}}%
\pgfusepath{clip}%
\pgfsetbuttcap%
\pgfsetroundjoin%
\pgfsetlinewidth{1.003750pt}%
\definecolor{currentstroke}{rgb}{0.827451,0.827451,0.827451}%
\pgfsetstrokecolor{currentstroke}%
\pgfsetstrokeopacity{0.800000}%
\pgfsetdash{}{0pt}%
\pgfpathmoveto{\pgfqpoint{0.734476in}{3.173741in}}%
\pgfpathcurveto{\pgfqpoint{0.745526in}{3.173741in}}{\pgfqpoint{0.756125in}{3.178131in}}{\pgfqpoint{0.763939in}{3.185945in}}%
\pgfpathcurveto{\pgfqpoint{0.771753in}{3.193758in}}{\pgfqpoint{0.776143in}{3.204357in}}{\pgfqpoint{0.776143in}{3.215408in}}%
\pgfpathcurveto{\pgfqpoint{0.776143in}{3.226458in}}{\pgfqpoint{0.771753in}{3.237057in}}{\pgfqpoint{0.763939in}{3.244870in}}%
\pgfpathcurveto{\pgfqpoint{0.756125in}{3.252684in}}{\pgfqpoint{0.745526in}{3.257074in}}{\pgfqpoint{0.734476in}{3.257074in}}%
\pgfpathcurveto{\pgfqpoint{0.723426in}{3.257074in}}{\pgfqpoint{0.712827in}{3.252684in}}{\pgfqpoint{0.705013in}{3.244870in}}%
\pgfpathcurveto{\pgfqpoint{0.697200in}{3.237057in}}{\pgfqpoint{0.692810in}{3.226458in}}{\pgfqpoint{0.692810in}{3.215408in}}%
\pgfpathcurveto{\pgfqpoint{0.692810in}{3.204357in}}{\pgfqpoint{0.697200in}{3.193758in}}{\pgfqpoint{0.705013in}{3.185945in}}%
\pgfpathcurveto{\pgfqpoint{0.712827in}{3.178131in}}{\pgfqpoint{0.723426in}{3.173741in}}{\pgfqpoint{0.734476in}{3.173741in}}%
\pgfpathlineto{\pgfqpoint{0.734476in}{3.173741in}}%
\pgfpathclose%
\pgfusepath{stroke}%
\end{pgfscope}%
\begin{pgfscope}%
\pgfpathrectangle{\pgfqpoint{0.494722in}{0.437222in}}{\pgfqpoint{6.275590in}{5.159444in}}%
\pgfusepath{clip}%
\pgfsetbuttcap%
\pgfsetroundjoin%
\pgfsetlinewidth{1.003750pt}%
\definecolor{currentstroke}{rgb}{0.827451,0.827451,0.827451}%
\pgfsetstrokecolor{currentstroke}%
\pgfsetstrokeopacity{0.800000}%
\pgfsetdash{}{0pt}%
\pgfpathmoveto{\pgfqpoint{0.801098in}{3.001018in}}%
\pgfpathcurveto{\pgfqpoint{0.812149in}{3.001018in}}{\pgfqpoint{0.822748in}{3.005408in}}{\pgfqpoint{0.830561in}{3.013222in}}%
\pgfpathcurveto{\pgfqpoint{0.838375in}{3.021035in}}{\pgfqpoint{0.842765in}{3.031634in}}{\pgfqpoint{0.842765in}{3.042685in}}%
\pgfpathcurveto{\pgfqpoint{0.842765in}{3.053735in}}{\pgfqpoint{0.838375in}{3.064334in}}{\pgfqpoint{0.830561in}{3.072147in}}%
\pgfpathcurveto{\pgfqpoint{0.822748in}{3.079961in}}{\pgfqpoint{0.812149in}{3.084351in}}{\pgfqpoint{0.801098in}{3.084351in}}%
\pgfpathcurveto{\pgfqpoint{0.790048in}{3.084351in}}{\pgfqpoint{0.779449in}{3.079961in}}{\pgfqpoint{0.771636in}{3.072147in}}%
\pgfpathcurveto{\pgfqpoint{0.763822in}{3.064334in}}{\pgfqpoint{0.759432in}{3.053735in}}{\pgfqpoint{0.759432in}{3.042685in}}%
\pgfpathcurveto{\pgfqpoint{0.759432in}{3.031634in}}{\pgfqpoint{0.763822in}{3.021035in}}{\pgfqpoint{0.771636in}{3.013222in}}%
\pgfpathcurveto{\pgfqpoint{0.779449in}{3.005408in}}{\pgfqpoint{0.790048in}{3.001018in}}{\pgfqpoint{0.801098in}{3.001018in}}%
\pgfpathlineto{\pgfqpoint{0.801098in}{3.001018in}}%
\pgfpathclose%
\pgfusepath{stroke}%
\end{pgfscope}%
\begin{pgfscope}%
\pgfpathrectangle{\pgfqpoint{0.494722in}{0.437222in}}{\pgfqpoint{6.275590in}{5.159444in}}%
\pgfusepath{clip}%
\pgfsetbuttcap%
\pgfsetroundjoin%
\pgfsetlinewidth{1.003750pt}%
\definecolor{currentstroke}{rgb}{0.827451,0.827451,0.827451}%
\pgfsetstrokecolor{currentstroke}%
\pgfsetstrokeopacity{0.800000}%
\pgfsetdash{}{0pt}%
\pgfpathmoveto{\pgfqpoint{3.123069in}{0.847718in}}%
\pgfpathcurveto{\pgfqpoint{3.134119in}{0.847718in}}{\pgfqpoint{3.144718in}{0.852109in}}{\pgfqpoint{3.152532in}{0.859922in}}%
\pgfpathcurveto{\pgfqpoint{3.160345in}{0.867736in}}{\pgfqpoint{3.164735in}{0.878335in}}{\pgfqpoint{3.164735in}{0.889385in}}%
\pgfpathcurveto{\pgfqpoint{3.164735in}{0.900435in}}{\pgfqpoint{3.160345in}{0.911034in}}{\pgfqpoint{3.152532in}{0.918848in}}%
\pgfpathcurveto{\pgfqpoint{3.144718in}{0.926661in}}{\pgfqpoint{3.134119in}{0.931052in}}{\pgfqpoint{3.123069in}{0.931052in}}%
\pgfpathcurveto{\pgfqpoint{3.112019in}{0.931052in}}{\pgfqpoint{3.101420in}{0.926661in}}{\pgfqpoint{3.093606in}{0.918848in}}%
\pgfpathcurveto{\pgfqpoint{3.085792in}{0.911034in}}{\pgfqpoint{3.081402in}{0.900435in}}{\pgfqpoint{3.081402in}{0.889385in}}%
\pgfpathcurveto{\pgfqpoint{3.081402in}{0.878335in}}{\pgfqpoint{3.085792in}{0.867736in}}{\pgfqpoint{3.093606in}{0.859922in}}%
\pgfpathcurveto{\pgfqpoint{3.101420in}{0.852109in}}{\pgfqpoint{3.112019in}{0.847718in}}{\pgfqpoint{3.123069in}{0.847718in}}%
\pgfpathlineto{\pgfqpoint{3.123069in}{0.847718in}}%
\pgfpathclose%
\pgfusepath{stroke}%
\end{pgfscope}%
\begin{pgfscope}%
\pgfpathrectangle{\pgfqpoint{0.494722in}{0.437222in}}{\pgfqpoint{6.275590in}{5.159444in}}%
\pgfusepath{clip}%
\pgfsetbuttcap%
\pgfsetroundjoin%
\pgfsetlinewidth{1.003750pt}%
\definecolor{currentstroke}{rgb}{0.827451,0.827451,0.827451}%
\pgfsetstrokecolor{currentstroke}%
\pgfsetstrokeopacity{0.800000}%
\pgfsetdash{}{0pt}%
\pgfpathmoveto{\pgfqpoint{0.680103in}{3.350578in}}%
\pgfpathcurveto{\pgfqpoint{0.691153in}{3.350578in}}{\pgfqpoint{0.701753in}{3.354968in}}{\pgfqpoint{0.709566in}{3.362782in}}%
\pgfpathcurveto{\pgfqpoint{0.717380in}{3.370596in}}{\pgfqpoint{0.721770in}{3.381195in}}{\pgfqpoint{0.721770in}{3.392245in}}%
\pgfpathcurveto{\pgfqpoint{0.721770in}{3.403295in}}{\pgfqpoint{0.717380in}{3.413894in}}{\pgfqpoint{0.709566in}{3.421707in}}%
\pgfpathcurveto{\pgfqpoint{0.701753in}{3.429521in}}{\pgfqpoint{0.691153in}{3.433911in}}{\pgfqpoint{0.680103in}{3.433911in}}%
\pgfpathcurveto{\pgfqpoint{0.669053in}{3.433911in}}{\pgfqpoint{0.658454in}{3.429521in}}{\pgfqpoint{0.650641in}{3.421707in}}%
\pgfpathcurveto{\pgfqpoint{0.642827in}{3.413894in}}{\pgfqpoint{0.638437in}{3.403295in}}{\pgfqpoint{0.638437in}{3.392245in}}%
\pgfpathcurveto{\pgfqpoint{0.638437in}{3.381195in}}{\pgfqpoint{0.642827in}{3.370596in}}{\pgfqpoint{0.650641in}{3.362782in}}%
\pgfpathcurveto{\pgfqpoint{0.658454in}{3.354968in}}{\pgfqpoint{0.669053in}{3.350578in}}{\pgfqpoint{0.680103in}{3.350578in}}%
\pgfpathlineto{\pgfqpoint{0.680103in}{3.350578in}}%
\pgfpathclose%
\pgfusepath{stroke}%
\end{pgfscope}%
\begin{pgfscope}%
\pgfpathrectangle{\pgfqpoint{0.494722in}{0.437222in}}{\pgfqpoint{6.275590in}{5.159444in}}%
\pgfusepath{clip}%
\pgfsetbuttcap%
\pgfsetroundjoin%
\pgfsetlinewidth{1.003750pt}%
\definecolor{currentstroke}{rgb}{0.827451,0.827451,0.827451}%
\pgfsetstrokecolor{currentstroke}%
\pgfsetstrokeopacity{0.800000}%
\pgfsetdash{}{0pt}%
\pgfpathmoveto{\pgfqpoint{0.711186in}{3.233934in}}%
\pgfpathcurveto{\pgfqpoint{0.722236in}{3.233934in}}{\pgfqpoint{0.732835in}{3.238324in}}{\pgfqpoint{0.740648in}{3.246138in}}%
\pgfpathcurveto{\pgfqpoint{0.748462in}{3.253952in}}{\pgfqpoint{0.752852in}{3.264551in}}{\pgfqpoint{0.752852in}{3.275601in}}%
\pgfpathcurveto{\pgfqpoint{0.752852in}{3.286651in}}{\pgfqpoint{0.748462in}{3.297250in}}{\pgfqpoint{0.740648in}{3.305064in}}%
\pgfpathcurveto{\pgfqpoint{0.732835in}{3.312877in}}{\pgfqpoint{0.722236in}{3.317267in}}{\pgfqpoint{0.711186in}{3.317267in}}%
\pgfpathcurveto{\pgfqpoint{0.700135in}{3.317267in}}{\pgfqpoint{0.689536in}{3.312877in}}{\pgfqpoint{0.681723in}{3.305064in}}%
\pgfpathcurveto{\pgfqpoint{0.673909in}{3.297250in}}{\pgfqpoint{0.669519in}{3.286651in}}{\pgfqpoint{0.669519in}{3.275601in}}%
\pgfpathcurveto{\pgfqpoint{0.669519in}{3.264551in}}{\pgfqpoint{0.673909in}{3.253952in}}{\pgfqpoint{0.681723in}{3.246138in}}%
\pgfpathcurveto{\pgfqpoint{0.689536in}{3.238324in}}{\pgfqpoint{0.700135in}{3.233934in}}{\pgfqpoint{0.711186in}{3.233934in}}%
\pgfpathlineto{\pgfqpoint{0.711186in}{3.233934in}}%
\pgfpathclose%
\pgfusepath{stroke}%
\end{pgfscope}%
\begin{pgfscope}%
\pgfpathrectangle{\pgfqpoint{0.494722in}{0.437222in}}{\pgfqpoint{6.275590in}{5.159444in}}%
\pgfusepath{clip}%
\pgfsetbuttcap%
\pgfsetroundjoin%
\pgfsetlinewidth{1.003750pt}%
\definecolor{currentstroke}{rgb}{0.827451,0.827451,0.827451}%
\pgfsetstrokecolor{currentstroke}%
\pgfsetstrokeopacity{0.800000}%
\pgfsetdash{}{0pt}%
\pgfpathmoveto{\pgfqpoint{4.847728in}{0.439866in}}%
\pgfpathcurveto{\pgfqpoint{4.858778in}{0.439866in}}{\pgfqpoint{4.869377in}{0.444256in}}{\pgfqpoint{4.877190in}{0.452070in}}%
\pgfpathcurveto{\pgfqpoint{4.885004in}{0.459884in}}{\pgfqpoint{4.889394in}{0.470483in}}{\pgfqpoint{4.889394in}{0.481533in}}%
\pgfpathcurveto{\pgfqpoint{4.889394in}{0.492583in}}{\pgfqpoint{4.885004in}{0.503182in}}{\pgfqpoint{4.877190in}{0.510995in}}%
\pgfpathcurveto{\pgfqpoint{4.869377in}{0.518809in}}{\pgfqpoint{4.858778in}{0.523199in}}{\pgfqpoint{4.847728in}{0.523199in}}%
\pgfpathcurveto{\pgfqpoint{4.836678in}{0.523199in}}{\pgfqpoint{4.826078in}{0.518809in}}{\pgfqpoint{4.818265in}{0.510995in}}%
\pgfpathcurveto{\pgfqpoint{4.810451in}{0.503182in}}{\pgfqpoint{4.806061in}{0.492583in}}{\pgfqpoint{4.806061in}{0.481533in}}%
\pgfpathcurveto{\pgfqpoint{4.806061in}{0.470483in}}{\pgfqpoint{4.810451in}{0.459884in}}{\pgfqpoint{4.818265in}{0.452070in}}%
\pgfpathcurveto{\pgfqpoint{4.826078in}{0.444256in}}{\pgfqpoint{4.836678in}{0.439866in}}{\pgfqpoint{4.847728in}{0.439866in}}%
\pgfpathlineto{\pgfqpoint{4.847728in}{0.439866in}}%
\pgfpathclose%
\pgfusepath{stroke}%
\end{pgfscope}%
\begin{pgfscope}%
\pgfpathrectangle{\pgfqpoint{0.494722in}{0.437222in}}{\pgfqpoint{6.275590in}{5.159444in}}%
\pgfusepath{clip}%
\pgfsetbuttcap%
\pgfsetroundjoin%
\pgfsetlinewidth{1.003750pt}%
\definecolor{currentstroke}{rgb}{0.827451,0.827451,0.827451}%
\pgfsetstrokecolor{currentstroke}%
\pgfsetstrokeopacity{0.800000}%
\pgfsetdash{}{0pt}%
\pgfpathmoveto{\pgfqpoint{0.575099in}{3.779944in}}%
\pgfpathcurveto{\pgfqpoint{0.586149in}{3.779944in}}{\pgfqpoint{0.596748in}{3.784334in}}{\pgfqpoint{0.604561in}{3.792148in}}%
\pgfpathcurveto{\pgfqpoint{0.612375in}{3.799961in}}{\pgfqpoint{0.616765in}{3.810560in}}{\pgfqpoint{0.616765in}{3.821610in}}%
\pgfpathcurveto{\pgfqpoint{0.616765in}{3.832661in}}{\pgfqpoint{0.612375in}{3.843260in}}{\pgfqpoint{0.604561in}{3.851073in}}%
\pgfpathcurveto{\pgfqpoint{0.596748in}{3.858887in}}{\pgfqpoint{0.586149in}{3.863277in}}{\pgfqpoint{0.575099in}{3.863277in}}%
\pgfpathcurveto{\pgfqpoint{0.564048in}{3.863277in}}{\pgfqpoint{0.553449in}{3.858887in}}{\pgfqpoint{0.545636in}{3.851073in}}%
\pgfpathcurveto{\pgfqpoint{0.537822in}{3.843260in}}{\pgfqpoint{0.533432in}{3.832661in}}{\pgfqpoint{0.533432in}{3.821610in}}%
\pgfpathcurveto{\pgfqpoint{0.533432in}{3.810560in}}{\pgfqpoint{0.537822in}{3.799961in}}{\pgfqpoint{0.545636in}{3.792148in}}%
\pgfpathcurveto{\pgfqpoint{0.553449in}{3.784334in}}{\pgfqpoint{0.564048in}{3.779944in}}{\pgfqpoint{0.575099in}{3.779944in}}%
\pgfpathlineto{\pgfqpoint{0.575099in}{3.779944in}}%
\pgfpathclose%
\pgfusepath{stroke}%
\end{pgfscope}%
\begin{pgfscope}%
\pgfpathrectangle{\pgfqpoint{0.494722in}{0.437222in}}{\pgfqpoint{6.275590in}{5.159444in}}%
\pgfusepath{clip}%
\pgfsetbuttcap%
\pgfsetroundjoin%
\pgfsetlinewidth{1.003750pt}%
\definecolor{currentstroke}{rgb}{0.827451,0.827451,0.827451}%
\pgfsetstrokecolor{currentstroke}%
\pgfsetstrokeopacity{0.800000}%
\pgfsetdash{}{0pt}%
\pgfpathmoveto{\pgfqpoint{2.856780in}{0.965441in}}%
\pgfpathcurveto{\pgfqpoint{2.867831in}{0.965441in}}{\pgfqpoint{2.878430in}{0.969831in}}{\pgfqpoint{2.886243in}{0.977645in}}%
\pgfpathcurveto{\pgfqpoint{2.894057in}{0.985458in}}{\pgfqpoint{2.898447in}{0.996058in}}{\pgfqpoint{2.898447in}{1.007108in}}%
\pgfpathcurveto{\pgfqpoint{2.898447in}{1.018158in}}{\pgfqpoint{2.894057in}{1.028757in}}{\pgfqpoint{2.886243in}{1.036570in}}%
\pgfpathcurveto{\pgfqpoint{2.878430in}{1.044384in}}{\pgfqpoint{2.867831in}{1.048774in}}{\pgfqpoint{2.856780in}{1.048774in}}%
\pgfpathcurveto{\pgfqpoint{2.845730in}{1.048774in}}{\pgfqpoint{2.835131in}{1.044384in}}{\pgfqpoint{2.827318in}{1.036570in}}%
\pgfpathcurveto{\pgfqpoint{2.819504in}{1.028757in}}{\pgfqpoint{2.815114in}{1.018158in}}{\pgfqpoint{2.815114in}{1.007108in}}%
\pgfpathcurveto{\pgfqpoint{2.815114in}{0.996058in}}{\pgfqpoint{2.819504in}{0.985458in}}{\pgfqpoint{2.827318in}{0.977645in}}%
\pgfpathcurveto{\pgfqpoint{2.835131in}{0.969831in}}{\pgfqpoint{2.845730in}{0.965441in}}{\pgfqpoint{2.856780in}{0.965441in}}%
\pgfpathlineto{\pgfqpoint{2.856780in}{0.965441in}}%
\pgfpathclose%
\pgfusepath{stroke}%
\end{pgfscope}%
\begin{pgfscope}%
\pgfpathrectangle{\pgfqpoint{0.494722in}{0.437222in}}{\pgfqpoint{6.275590in}{5.159444in}}%
\pgfusepath{clip}%
\pgfsetbuttcap%
\pgfsetroundjoin%
\pgfsetlinewidth{1.003750pt}%
\definecolor{currentstroke}{rgb}{0.827451,0.827451,0.827451}%
\pgfsetstrokecolor{currentstroke}%
\pgfsetstrokeopacity{0.800000}%
\pgfsetdash{}{0pt}%
\pgfpathmoveto{\pgfqpoint{3.558031in}{0.690537in}}%
\pgfpathcurveto{\pgfqpoint{3.569081in}{0.690537in}}{\pgfqpoint{3.579680in}{0.694927in}}{\pgfqpoint{3.587494in}{0.702741in}}%
\pgfpathcurveto{\pgfqpoint{3.595308in}{0.710554in}}{\pgfqpoint{3.599698in}{0.721153in}}{\pgfqpoint{3.599698in}{0.732204in}}%
\pgfpathcurveto{\pgfqpoint{3.599698in}{0.743254in}}{\pgfqpoint{3.595308in}{0.753853in}}{\pgfqpoint{3.587494in}{0.761666in}}%
\pgfpathcurveto{\pgfqpoint{3.579680in}{0.769480in}}{\pgfqpoint{3.569081in}{0.773870in}}{\pgfqpoint{3.558031in}{0.773870in}}%
\pgfpathcurveto{\pgfqpoint{3.546981in}{0.773870in}}{\pgfqpoint{3.536382in}{0.769480in}}{\pgfqpoint{3.528568in}{0.761666in}}%
\pgfpathcurveto{\pgfqpoint{3.520755in}{0.753853in}}{\pgfqpoint{3.516364in}{0.743254in}}{\pgfqpoint{3.516364in}{0.732204in}}%
\pgfpathcurveto{\pgfqpoint{3.516364in}{0.721153in}}{\pgfqpoint{3.520755in}{0.710554in}}{\pgfqpoint{3.528568in}{0.702741in}}%
\pgfpathcurveto{\pgfqpoint{3.536382in}{0.694927in}}{\pgfqpoint{3.546981in}{0.690537in}}{\pgfqpoint{3.558031in}{0.690537in}}%
\pgfpathlineto{\pgfqpoint{3.558031in}{0.690537in}}%
\pgfpathclose%
\pgfusepath{stroke}%
\end{pgfscope}%
\begin{pgfscope}%
\pgfpathrectangle{\pgfqpoint{0.494722in}{0.437222in}}{\pgfqpoint{6.275590in}{5.159444in}}%
\pgfusepath{clip}%
\pgfsetbuttcap%
\pgfsetroundjoin%
\pgfsetlinewidth{1.003750pt}%
\definecolor{currentstroke}{rgb}{0.827451,0.827451,0.827451}%
\pgfsetstrokecolor{currentstroke}%
\pgfsetstrokeopacity{0.800000}%
\pgfsetdash{}{0pt}%
\pgfpathmoveto{\pgfqpoint{3.697856in}{0.643246in}}%
\pgfpathcurveto{\pgfqpoint{3.708907in}{0.643246in}}{\pgfqpoint{3.719506in}{0.647637in}}{\pgfqpoint{3.727319in}{0.655450in}}%
\pgfpathcurveto{\pgfqpoint{3.735133in}{0.663264in}}{\pgfqpoint{3.739523in}{0.673863in}}{\pgfqpoint{3.739523in}{0.684913in}}%
\pgfpathcurveto{\pgfqpoint{3.739523in}{0.695963in}}{\pgfqpoint{3.735133in}{0.706562in}}{\pgfqpoint{3.727319in}{0.714376in}}%
\pgfpathcurveto{\pgfqpoint{3.719506in}{0.722189in}}{\pgfqpoint{3.708907in}{0.726580in}}{\pgfqpoint{3.697856in}{0.726580in}}%
\pgfpathcurveto{\pgfqpoint{3.686806in}{0.726580in}}{\pgfqpoint{3.676207in}{0.722189in}}{\pgfqpoint{3.668394in}{0.714376in}}%
\pgfpathcurveto{\pgfqpoint{3.660580in}{0.706562in}}{\pgfqpoint{3.656190in}{0.695963in}}{\pgfqpoint{3.656190in}{0.684913in}}%
\pgfpathcurveto{\pgfqpoint{3.656190in}{0.673863in}}{\pgfqpoint{3.660580in}{0.663264in}}{\pgfqpoint{3.668394in}{0.655450in}}%
\pgfpathcurveto{\pgfqpoint{3.676207in}{0.647637in}}{\pgfqpoint{3.686806in}{0.643246in}}{\pgfqpoint{3.697856in}{0.643246in}}%
\pgfpathlineto{\pgfqpoint{3.697856in}{0.643246in}}%
\pgfpathclose%
\pgfusepath{stroke}%
\end{pgfscope}%
\begin{pgfscope}%
\pgfpathrectangle{\pgfqpoint{0.494722in}{0.437222in}}{\pgfqpoint{6.275590in}{5.159444in}}%
\pgfusepath{clip}%
\pgfsetbuttcap%
\pgfsetroundjoin%
\pgfsetlinewidth{1.003750pt}%
\definecolor{currentstroke}{rgb}{0.827451,0.827451,0.827451}%
\pgfsetstrokecolor{currentstroke}%
\pgfsetstrokeopacity{0.800000}%
\pgfsetdash{}{0pt}%
\pgfpathmoveto{\pgfqpoint{1.554134in}{1.871128in}}%
\pgfpathcurveto{\pgfqpoint{1.565185in}{1.871128in}}{\pgfqpoint{1.575784in}{1.875518in}}{\pgfqpoint{1.583597in}{1.883332in}}%
\pgfpathcurveto{\pgfqpoint{1.591411in}{1.891146in}}{\pgfqpoint{1.595801in}{1.901745in}}{\pgfqpoint{1.595801in}{1.912795in}}%
\pgfpathcurveto{\pgfqpoint{1.595801in}{1.923845in}}{\pgfqpoint{1.591411in}{1.934444in}}{\pgfqpoint{1.583597in}{1.942258in}}%
\pgfpathcurveto{\pgfqpoint{1.575784in}{1.950071in}}{\pgfqpoint{1.565185in}{1.954462in}}{\pgfqpoint{1.554134in}{1.954462in}}%
\pgfpathcurveto{\pgfqpoint{1.543084in}{1.954462in}}{\pgfqpoint{1.532485in}{1.950071in}}{\pgfqpoint{1.524672in}{1.942258in}}%
\pgfpathcurveto{\pgfqpoint{1.516858in}{1.934444in}}{\pgfqpoint{1.512468in}{1.923845in}}{\pgfqpoint{1.512468in}{1.912795in}}%
\pgfpathcurveto{\pgfqpoint{1.512468in}{1.901745in}}{\pgfqpoint{1.516858in}{1.891146in}}{\pgfqpoint{1.524672in}{1.883332in}}%
\pgfpathcurveto{\pgfqpoint{1.532485in}{1.875518in}}{\pgfqpoint{1.543084in}{1.871128in}}{\pgfqpoint{1.554134in}{1.871128in}}%
\pgfpathlineto{\pgfqpoint{1.554134in}{1.871128in}}%
\pgfpathclose%
\pgfusepath{stroke}%
\end{pgfscope}%
\begin{pgfscope}%
\pgfpathrectangle{\pgfqpoint{0.494722in}{0.437222in}}{\pgfqpoint{6.275590in}{5.159444in}}%
\pgfusepath{clip}%
\pgfsetbuttcap%
\pgfsetroundjoin%
\pgfsetlinewidth{1.003750pt}%
\definecolor{currentstroke}{rgb}{0.827451,0.827451,0.827451}%
\pgfsetstrokecolor{currentstroke}%
\pgfsetstrokeopacity{0.800000}%
\pgfsetdash{}{0pt}%
\pgfpathmoveto{\pgfqpoint{3.889511in}{0.591465in}}%
\pgfpathcurveto{\pgfqpoint{3.900562in}{0.591465in}}{\pgfqpoint{3.911161in}{0.595856in}}{\pgfqpoint{3.918974in}{0.603669in}}%
\pgfpathcurveto{\pgfqpoint{3.926788in}{0.611483in}}{\pgfqpoint{3.931178in}{0.622082in}}{\pgfqpoint{3.931178in}{0.633132in}}%
\pgfpathcurveto{\pgfqpoint{3.931178in}{0.644182in}}{\pgfqpoint{3.926788in}{0.654781in}}{\pgfqpoint{3.918974in}{0.662595in}}%
\pgfpathcurveto{\pgfqpoint{3.911161in}{0.670408in}}{\pgfqpoint{3.900562in}{0.674799in}}{\pgfqpoint{3.889511in}{0.674799in}}%
\pgfpathcurveto{\pgfqpoint{3.878461in}{0.674799in}}{\pgfqpoint{3.867862in}{0.670408in}}{\pgfqpoint{3.860049in}{0.662595in}}%
\pgfpathcurveto{\pgfqpoint{3.852235in}{0.654781in}}{\pgfqpoint{3.847845in}{0.644182in}}{\pgfqpoint{3.847845in}{0.633132in}}%
\pgfpathcurveto{\pgfqpoint{3.847845in}{0.622082in}}{\pgfqpoint{3.852235in}{0.611483in}}{\pgfqpoint{3.860049in}{0.603669in}}%
\pgfpathcurveto{\pgfqpoint{3.867862in}{0.595856in}}{\pgfqpoint{3.878461in}{0.591465in}}{\pgfqpoint{3.889511in}{0.591465in}}%
\pgfpathlineto{\pgfqpoint{3.889511in}{0.591465in}}%
\pgfpathclose%
\pgfusepath{stroke}%
\end{pgfscope}%
\begin{pgfscope}%
\pgfpathrectangle{\pgfqpoint{0.494722in}{0.437222in}}{\pgfqpoint{6.275590in}{5.159444in}}%
\pgfusepath{clip}%
\pgfsetbuttcap%
\pgfsetroundjoin%
\pgfsetlinewidth{1.003750pt}%
\definecolor{currentstroke}{rgb}{0.827451,0.827451,0.827451}%
\pgfsetstrokecolor{currentstroke}%
\pgfsetstrokeopacity{0.800000}%
\pgfsetdash{}{0pt}%
\pgfpathmoveto{\pgfqpoint{1.786945in}{1.657846in}}%
\pgfpathcurveto{\pgfqpoint{1.797995in}{1.657846in}}{\pgfqpoint{1.808594in}{1.662237in}}{\pgfqpoint{1.816407in}{1.670050in}}%
\pgfpathcurveto{\pgfqpoint{1.824221in}{1.677864in}}{\pgfqpoint{1.828611in}{1.688463in}}{\pgfqpoint{1.828611in}{1.699513in}}%
\pgfpathcurveto{\pgfqpoint{1.828611in}{1.710563in}}{\pgfqpoint{1.824221in}{1.721162in}}{\pgfqpoint{1.816407in}{1.728976in}}%
\pgfpathcurveto{\pgfqpoint{1.808594in}{1.736789in}}{\pgfqpoint{1.797995in}{1.741180in}}{\pgfqpoint{1.786945in}{1.741180in}}%
\pgfpathcurveto{\pgfqpoint{1.775894in}{1.741180in}}{\pgfqpoint{1.765295in}{1.736789in}}{\pgfqpoint{1.757482in}{1.728976in}}%
\pgfpathcurveto{\pgfqpoint{1.749668in}{1.721162in}}{\pgfqpoint{1.745278in}{1.710563in}}{\pgfqpoint{1.745278in}{1.699513in}}%
\pgfpathcurveto{\pgfqpoint{1.745278in}{1.688463in}}{\pgfqpoint{1.749668in}{1.677864in}}{\pgfqpoint{1.757482in}{1.670050in}}%
\pgfpathcurveto{\pgfqpoint{1.765295in}{1.662237in}}{\pgfqpoint{1.775894in}{1.657846in}}{\pgfqpoint{1.786945in}{1.657846in}}%
\pgfpathlineto{\pgfqpoint{1.786945in}{1.657846in}}%
\pgfpathclose%
\pgfusepath{stroke}%
\end{pgfscope}%
\begin{pgfscope}%
\pgfpathrectangle{\pgfqpoint{0.494722in}{0.437222in}}{\pgfqpoint{6.275590in}{5.159444in}}%
\pgfusepath{clip}%
\pgfsetbuttcap%
\pgfsetroundjoin%
\pgfsetlinewidth{1.003750pt}%
\definecolor{currentstroke}{rgb}{0.827451,0.827451,0.827451}%
\pgfsetstrokecolor{currentstroke}%
\pgfsetstrokeopacity{0.800000}%
\pgfsetdash{}{0pt}%
\pgfpathmoveto{\pgfqpoint{3.250537in}{0.808630in}}%
\pgfpathcurveto{\pgfqpoint{3.261587in}{0.808630in}}{\pgfqpoint{3.272186in}{0.813020in}}{\pgfqpoint{3.280000in}{0.820834in}}%
\pgfpathcurveto{\pgfqpoint{3.287813in}{0.828647in}}{\pgfqpoint{3.292203in}{0.839246in}}{\pgfqpoint{3.292203in}{0.850297in}}%
\pgfpathcurveto{\pgfqpoint{3.292203in}{0.861347in}}{\pgfqpoint{3.287813in}{0.871946in}}{\pgfqpoint{3.280000in}{0.879759in}}%
\pgfpathcurveto{\pgfqpoint{3.272186in}{0.887573in}}{\pgfqpoint{3.261587in}{0.891963in}}{\pgfqpoint{3.250537in}{0.891963in}}%
\pgfpathcurveto{\pgfqpoint{3.239487in}{0.891963in}}{\pgfqpoint{3.228888in}{0.887573in}}{\pgfqpoint{3.221074in}{0.879759in}}%
\pgfpathcurveto{\pgfqpoint{3.213260in}{0.871946in}}{\pgfqpoint{3.208870in}{0.861347in}}{\pgfqpoint{3.208870in}{0.850297in}}%
\pgfpathcurveto{\pgfqpoint{3.208870in}{0.839246in}}{\pgfqpoint{3.213260in}{0.828647in}}{\pgfqpoint{3.221074in}{0.820834in}}%
\pgfpathcurveto{\pgfqpoint{3.228888in}{0.813020in}}{\pgfqpoint{3.239487in}{0.808630in}}{\pgfqpoint{3.250537in}{0.808630in}}%
\pgfpathlineto{\pgfqpoint{3.250537in}{0.808630in}}%
\pgfpathclose%
\pgfusepath{stroke}%
\end{pgfscope}%
\begin{pgfscope}%
\pgfpathrectangle{\pgfqpoint{0.494722in}{0.437222in}}{\pgfqpoint{6.275590in}{5.159444in}}%
\pgfusepath{clip}%
\pgfsetbuttcap%
\pgfsetroundjoin%
\pgfsetlinewidth{1.003750pt}%
\definecolor{currentstroke}{rgb}{0.827451,0.827451,0.827451}%
\pgfsetstrokecolor{currentstroke}%
\pgfsetstrokeopacity{0.800000}%
\pgfsetdash{}{0pt}%
\pgfpathmoveto{\pgfqpoint{5.032325in}{0.429761in}}%
\pgfpathcurveto{\pgfqpoint{5.043375in}{0.429761in}}{\pgfqpoint{5.053974in}{0.434151in}}{\pgfqpoint{5.061788in}{0.441965in}}%
\pgfpathcurveto{\pgfqpoint{5.069601in}{0.449778in}}{\pgfqpoint{5.073992in}{0.460377in}}{\pgfqpoint{5.073992in}{0.471427in}}%
\pgfpathcurveto{\pgfqpoint{5.073992in}{0.482477in}}{\pgfqpoint{5.069601in}{0.493076in}}{\pgfqpoint{5.061788in}{0.500890in}}%
\pgfpathcurveto{\pgfqpoint{5.053974in}{0.508704in}}{\pgfqpoint{5.043375in}{0.513094in}}{\pgfqpoint{5.032325in}{0.513094in}}%
\pgfpathcurveto{\pgfqpoint{5.021275in}{0.513094in}}{\pgfqpoint{5.010676in}{0.508704in}}{\pgfqpoint{5.002862in}{0.500890in}}%
\pgfpathcurveto{\pgfqpoint{4.995049in}{0.493076in}}{\pgfqpoint{4.990658in}{0.482477in}}{\pgfqpoint{4.990658in}{0.471427in}}%
\pgfpathcurveto{\pgfqpoint{4.990658in}{0.460377in}}{\pgfqpoint{4.995049in}{0.449778in}}{\pgfqpoint{5.002862in}{0.441965in}}%
\pgfpathcurveto{\pgfqpoint{5.010676in}{0.434151in}}{\pgfqpoint{5.021275in}{0.429761in}}{\pgfqpoint{5.032325in}{0.429761in}}%
\pgfpathlineto{\pgfqpoint{5.032325in}{0.429761in}}%
\pgfpathclose%
\pgfusepath{stroke}%
\end{pgfscope}%
\begin{pgfscope}%
\pgfpathrectangle{\pgfqpoint{0.494722in}{0.437222in}}{\pgfqpoint{6.275590in}{5.159444in}}%
\pgfusepath{clip}%
\pgfsetbuttcap%
\pgfsetroundjoin%
\pgfsetlinewidth{1.003750pt}%
\definecolor{currentstroke}{rgb}{0.827451,0.827451,0.827451}%
\pgfsetstrokecolor{currentstroke}%
\pgfsetstrokeopacity{0.800000}%
\pgfsetdash{}{0pt}%
\pgfpathmoveto{\pgfqpoint{1.464755in}{1.963622in}}%
\pgfpathcurveto{\pgfqpoint{1.475805in}{1.963622in}}{\pgfqpoint{1.486404in}{1.968013in}}{\pgfqpoint{1.494217in}{1.975826in}}%
\pgfpathcurveto{\pgfqpoint{1.502031in}{1.983640in}}{\pgfqpoint{1.506421in}{1.994239in}}{\pgfqpoint{1.506421in}{2.005289in}}%
\pgfpathcurveto{\pgfqpoint{1.506421in}{2.016339in}}{\pgfqpoint{1.502031in}{2.026938in}}{\pgfqpoint{1.494217in}{2.034752in}}%
\pgfpathcurveto{\pgfqpoint{1.486404in}{2.042566in}}{\pgfqpoint{1.475805in}{2.046956in}}{\pgfqpoint{1.464755in}{2.046956in}}%
\pgfpathcurveto{\pgfqpoint{1.453705in}{2.046956in}}{\pgfqpoint{1.443106in}{2.042566in}}{\pgfqpoint{1.435292in}{2.034752in}}%
\pgfpathcurveto{\pgfqpoint{1.427478in}{2.026938in}}{\pgfqpoint{1.423088in}{2.016339in}}{\pgfqpoint{1.423088in}{2.005289in}}%
\pgfpathcurveto{\pgfqpoint{1.423088in}{1.994239in}}{\pgfqpoint{1.427478in}{1.983640in}}{\pgfqpoint{1.435292in}{1.975826in}}%
\pgfpathcurveto{\pgfqpoint{1.443106in}{1.968013in}}{\pgfqpoint{1.453705in}{1.963622in}}{\pgfqpoint{1.464755in}{1.963622in}}%
\pgfpathlineto{\pgfqpoint{1.464755in}{1.963622in}}%
\pgfpathclose%
\pgfusepath{stroke}%
\end{pgfscope}%
\begin{pgfscope}%
\pgfpathrectangle{\pgfqpoint{0.494722in}{0.437222in}}{\pgfqpoint{6.275590in}{5.159444in}}%
\pgfusepath{clip}%
\pgfsetbuttcap%
\pgfsetroundjoin%
\pgfsetlinewidth{1.003750pt}%
\definecolor{currentstroke}{rgb}{0.827451,0.827451,0.827451}%
\pgfsetstrokecolor{currentstroke}%
\pgfsetstrokeopacity{0.800000}%
\pgfsetdash{}{0pt}%
\pgfpathmoveto{\pgfqpoint{0.970880in}{2.643850in}}%
\pgfpathcurveto{\pgfqpoint{0.981930in}{2.643850in}}{\pgfqpoint{0.992529in}{2.648240in}}{\pgfqpoint{1.000343in}{2.656053in}}%
\pgfpathcurveto{\pgfqpoint{1.008156in}{2.663867in}}{\pgfqpoint{1.012547in}{2.674466in}}{\pgfqpoint{1.012547in}{2.685516in}}%
\pgfpathcurveto{\pgfqpoint{1.012547in}{2.696566in}}{\pgfqpoint{1.008156in}{2.707165in}}{\pgfqpoint{1.000343in}{2.714979in}}%
\pgfpathcurveto{\pgfqpoint{0.992529in}{2.722793in}}{\pgfqpoint{0.981930in}{2.727183in}}{\pgfqpoint{0.970880in}{2.727183in}}%
\pgfpathcurveto{\pgfqpoint{0.959830in}{2.727183in}}{\pgfqpoint{0.949231in}{2.722793in}}{\pgfqpoint{0.941417in}{2.714979in}}%
\pgfpathcurveto{\pgfqpoint{0.933604in}{2.707165in}}{\pgfqpoint{0.929213in}{2.696566in}}{\pgfqpoint{0.929213in}{2.685516in}}%
\pgfpathcurveto{\pgfqpoint{0.929213in}{2.674466in}}{\pgfqpoint{0.933604in}{2.663867in}}{\pgfqpoint{0.941417in}{2.656053in}}%
\pgfpathcurveto{\pgfqpoint{0.949231in}{2.648240in}}{\pgfqpoint{0.959830in}{2.643850in}}{\pgfqpoint{0.970880in}{2.643850in}}%
\pgfpathlineto{\pgfqpoint{0.970880in}{2.643850in}}%
\pgfpathclose%
\pgfusepath{stroke}%
\end{pgfscope}%
\begin{pgfscope}%
\pgfpathrectangle{\pgfqpoint{0.494722in}{0.437222in}}{\pgfqpoint{6.275590in}{5.159444in}}%
\pgfusepath{clip}%
\pgfsetbuttcap%
\pgfsetroundjoin%
\pgfsetlinewidth{1.003750pt}%
\definecolor{currentstroke}{rgb}{0.827451,0.827451,0.827451}%
\pgfsetstrokecolor{currentstroke}%
\pgfsetstrokeopacity{0.800000}%
\pgfsetdash{}{0pt}%
\pgfpathmoveto{\pgfqpoint{0.512966in}{4.206085in}}%
\pgfpathcurveto{\pgfqpoint{0.524016in}{4.206085in}}{\pgfqpoint{0.534615in}{4.210475in}}{\pgfqpoint{0.542429in}{4.218289in}}%
\pgfpathcurveto{\pgfqpoint{0.550243in}{4.226102in}}{\pgfqpoint{0.554633in}{4.236701in}}{\pgfqpoint{0.554633in}{4.247752in}}%
\pgfpathcurveto{\pgfqpoint{0.554633in}{4.258802in}}{\pgfqpoint{0.550243in}{4.269401in}}{\pgfqpoint{0.542429in}{4.277214in}}%
\pgfpathcurveto{\pgfqpoint{0.534615in}{4.285028in}}{\pgfqpoint{0.524016in}{4.289418in}}{\pgfqpoint{0.512966in}{4.289418in}}%
\pgfpathcurveto{\pgfqpoint{0.501916in}{4.289418in}}{\pgfqpoint{0.491317in}{4.285028in}}{\pgfqpoint{0.483503in}{4.277214in}}%
\pgfpathcurveto{\pgfqpoint{0.475690in}{4.269401in}}{\pgfqpoint{0.471299in}{4.258802in}}{\pgfqpoint{0.471299in}{4.247752in}}%
\pgfpathcurveto{\pgfqpoint{0.471299in}{4.236701in}}{\pgfqpoint{0.475690in}{4.226102in}}{\pgfqpoint{0.483503in}{4.218289in}}%
\pgfpathcurveto{\pgfqpoint{0.491317in}{4.210475in}}{\pgfqpoint{0.501916in}{4.206085in}}{\pgfqpoint{0.512966in}{4.206085in}}%
\pgfpathlineto{\pgfqpoint{0.512966in}{4.206085in}}%
\pgfpathclose%
\pgfusepath{stroke}%
\end{pgfscope}%
\begin{pgfscope}%
\pgfpathrectangle{\pgfqpoint{0.494722in}{0.437222in}}{\pgfqpoint{6.275590in}{5.159444in}}%
\pgfusepath{clip}%
\pgfsetbuttcap%
\pgfsetroundjoin%
\pgfsetlinewidth{1.003750pt}%
\definecolor{currentstroke}{rgb}{0.827451,0.827451,0.827451}%
\pgfsetstrokecolor{currentstroke}%
\pgfsetstrokeopacity{0.800000}%
\pgfsetdash{}{0pt}%
\pgfpathmoveto{\pgfqpoint{1.038730in}{2.546051in}}%
\pgfpathcurveto{\pgfqpoint{1.049780in}{2.546051in}}{\pgfqpoint{1.060379in}{2.550441in}}{\pgfqpoint{1.068193in}{2.558254in}}%
\pgfpathcurveto{\pgfqpoint{1.076007in}{2.566068in}}{\pgfqpoint{1.080397in}{2.576667in}}{\pgfqpoint{1.080397in}{2.587717in}}%
\pgfpathcurveto{\pgfqpoint{1.080397in}{2.598767in}}{\pgfqpoint{1.076007in}{2.609366in}}{\pgfqpoint{1.068193in}{2.617180in}}%
\pgfpathcurveto{\pgfqpoint{1.060379in}{2.624994in}}{\pgfqpoint{1.049780in}{2.629384in}}{\pgfqpoint{1.038730in}{2.629384in}}%
\pgfpathcurveto{\pgfqpoint{1.027680in}{2.629384in}}{\pgfqpoint{1.017081in}{2.624994in}}{\pgfqpoint{1.009268in}{2.617180in}}%
\pgfpathcurveto{\pgfqpoint{1.001454in}{2.609366in}}{\pgfqpoint{0.997064in}{2.598767in}}{\pgfqpoint{0.997064in}{2.587717in}}%
\pgfpathcurveto{\pgfqpoint{0.997064in}{2.576667in}}{\pgfqpoint{1.001454in}{2.566068in}}{\pgfqpoint{1.009268in}{2.558254in}}%
\pgfpathcurveto{\pgfqpoint{1.017081in}{2.550441in}}{\pgfqpoint{1.027680in}{2.546051in}}{\pgfqpoint{1.038730in}{2.546051in}}%
\pgfpathlineto{\pgfqpoint{1.038730in}{2.546051in}}%
\pgfpathclose%
\pgfusepath{stroke}%
\end{pgfscope}%
\begin{pgfscope}%
\pgfpathrectangle{\pgfqpoint{0.494722in}{0.437222in}}{\pgfqpoint{6.275590in}{5.159444in}}%
\pgfusepath{clip}%
\pgfsetbuttcap%
\pgfsetroundjoin%
\pgfsetlinewidth{1.003750pt}%
\definecolor{currentstroke}{rgb}{0.827451,0.827451,0.827451}%
\pgfsetstrokecolor{currentstroke}%
\pgfsetstrokeopacity{0.800000}%
\pgfsetdash{}{0pt}%
\pgfpathmoveto{\pgfqpoint{2.629632in}{1.109341in}}%
\pgfpathcurveto{\pgfqpoint{2.640682in}{1.109341in}}{\pgfqpoint{2.651281in}{1.113731in}}{\pgfqpoint{2.659094in}{1.121545in}}%
\pgfpathcurveto{\pgfqpoint{2.666908in}{1.129358in}}{\pgfqpoint{2.671298in}{1.139957in}}{\pgfqpoint{2.671298in}{1.151007in}}%
\pgfpathcurveto{\pgfqpoint{2.671298in}{1.162057in}}{\pgfqpoint{2.666908in}{1.172656in}}{\pgfqpoint{2.659094in}{1.180470in}}%
\pgfpathcurveto{\pgfqpoint{2.651281in}{1.188284in}}{\pgfqpoint{2.640682in}{1.192674in}}{\pgfqpoint{2.629632in}{1.192674in}}%
\pgfpathcurveto{\pgfqpoint{2.618582in}{1.192674in}}{\pgfqpoint{2.607983in}{1.188284in}}{\pgfqpoint{2.600169in}{1.180470in}}%
\pgfpathcurveto{\pgfqpoint{2.592355in}{1.172656in}}{\pgfqpoint{2.587965in}{1.162057in}}{\pgfqpoint{2.587965in}{1.151007in}}%
\pgfpathcurveto{\pgfqpoint{2.587965in}{1.139957in}}{\pgfqpoint{2.592355in}{1.129358in}}{\pgfqpoint{2.600169in}{1.121545in}}%
\pgfpathcurveto{\pgfqpoint{2.607983in}{1.113731in}}{\pgfqpoint{2.618582in}{1.109341in}}{\pgfqpoint{2.629632in}{1.109341in}}%
\pgfpathlineto{\pgfqpoint{2.629632in}{1.109341in}}%
\pgfpathclose%
\pgfusepath{stroke}%
\end{pgfscope}%
\begin{pgfscope}%
\pgfpathrectangle{\pgfqpoint{0.494722in}{0.437222in}}{\pgfqpoint{6.275590in}{5.159444in}}%
\pgfusepath{clip}%
\pgfsetbuttcap%
\pgfsetroundjoin%
\pgfsetlinewidth{1.003750pt}%
\definecolor{currentstroke}{rgb}{0.827451,0.827451,0.827451}%
\pgfsetstrokecolor{currentstroke}%
\pgfsetstrokeopacity{0.800000}%
\pgfsetdash{}{0pt}%
\pgfpathmoveto{\pgfqpoint{0.882503in}{2.817015in}}%
\pgfpathcurveto{\pgfqpoint{0.893553in}{2.817015in}}{\pgfqpoint{0.904152in}{2.821405in}}{\pgfqpoint{0.911966in}{2.829219in}}%
\pgfpathcurveto{\pgfqpoint{0.919779in}{2.837033in}}{\pgfqpoint{0.924170in}{2.847632in}}{\pgfqpoint{0.924170in}{2.858682in}}%
\pgfpathcurveto{\pgfqpoint{0.924170in}{2.869732in}}{\pgfqpoint{0.919779in}{2.880331in}}{\pgfqpoint{0.911966in}{2.888145in}}%
\pgfpathcurveto{\pgfqpoint{0.904152in}{2.895958in}}{\pgfqpoint{0.893553in}{2.900348in}}{\pgfqpoint{0.882503in}{2.900348in}}%
\pgfpathcurveto{\pgfqpoint{0.871453in}{2.900348in}}{\pgfqpoint{0.860854in}{2.895958in}}{\pgfqpoint{0.853040in}{2.888145in}}%
\pgfpathcurveto{\pgfqpoint{0.845226in}{2.880331in}}{\pgfqpoint{0.840836in}{2.869732in}}{\pgfqpoint{0.840836in}{2.858682in}}%
\pgfpathcurveto{\pgfqpoint{0.840836in}{2.847632in}}{\pgfqpoint{0.845226in}{2.837033in}}{\pgfqpoint{0.853040in}{2.829219in}}%
\pgfpathcurveto{\pgfqpoint{0.860854in}{2.821405in}}{\pgfqpoint{0.871453in}{2.817015in}}{\pgfqpoint{0.882503in}{2.817015in}}%
\pgfpathlineto{\pgfqpoint{0.882503in}{2.817015in}}%
\pgfpathclose%
\pgfusepath{stroke}%
\end{pgfscope}%
\begin{pgfscope}%
\pgfpathrectangle{\pgfqpoint{0.494722in}{0.437222in}}{\pgfqpoint{6.275590in}{5.159444in}}%
\pgfusepath{clip}%
\pgfsetbuttcap%
\pgfsetroundjoin%
\pgfsetlinewidth{1.003750pt}%
\definecolor{currentstroke}{rgb}{0.827451,0.827451,0.827451}%
\pgfsetstrokecolor{currentstroke}%
\pgfsetstrokeopacity{0.800000}%
\pgfsetdash{}{0pt}%
\pgfpathmoveto{\pgfqpoint{0.501219in}{4.373826in}}%
\pgfpathcurveto{\pgfqpoint{0.512269in}{4.373826in}}{\pgfqpoint{0.522868in}{4.378216in}}{\pgfqpoint{0.530682in}{4.386030in}}%
\pgfpathcurveto{\pgfqpoint{0.538495in}{4.393844in}}{\pgfqpoint{0.542886in}{4.404443in}}{\pgfqpoint{0.542886in}{4.415493in}}%
\pgfpathcurveto{\pgfqpoint{0.542886in}{4.426543in}}{\pgfqpoint{0.538495in}{4.437142in}}{\pgfqpoint{0.530682in}{4.444955in}}%
\pgfpathcurveto{\pgfqpoint{0.522868in}{4.452769in}}{\pgfqpoint{0.512269in}{4.457159in}}{\pgfqpoint{0.501219in}{4.457159in}}%
\pgfpathcurveto{\pgfqpoint{0.490169in}{4.457159in}}{\pgfqpoint{0.479570in}{4.452769in}}{\pgfqpoint{0.471756in}{4.444955in}}%
\pgfpathcurveto{\pgfqpoint{0.463943in}{4.437142in}}{\pgfqpoint{0.459552in}{4.426543in}}{\pgfqpoint{0.459552in}{4.415493in}}%
\pgfpathcurveto{\pgfqpoint{0.459552in}{4.404443in}}{\pgfqpoint{0.463943in}{4.393844in}}{\pgfqpoint{0.471756in}{4.386030in}}%
\pgfpathcurveto{\pgfqpoint{0.479570in}{4.378216in}}{\pgfqpoint{0.490169in}{4.373826in}}{\pgfqpoint{0.501219in}{4.373826in}}%
\pgfpathlineto{\pgfqpoint{0.501219in}{4.373826in}}%
\pgfpathclose%
\pgfusepath{stroke}%
\end{pgfscope}%
\begin{pgfscope}%
\pgfpathrectangle{\pgfqpoint{0.494722in}{0.437222in}}{\pgfqpoint{6.275590in}{5.159444in}}%
\pgfusepath{clip}%
\pgfsetbuttcap%
\pgfsetroundjoin%
\pgfsetlinewidth{1.003750pt}%
\definecolor{currentstroke}{rgb}{0.827451,0.827451,0.827451}%
\pgfsetstrokecolor{currentstroke}%
\pgfsetstrokeopacity{0.800000}%
\pgfsetdash{}{0pt}%
\pgfpathmoveto{\pgfqpoint{3.949353in}{0.575507in}}%
\pgfpathcurveto{\pgfqpoint{3.960403in}{0.575507in}}{\pgfqpoint{3.971002in}{0.579897in}}{\pgfqpoint{3.978815in}{0.587711in}}%
\pgfpathcurveto{\pgfqpoint{3.986629in}{0.595525in}}{\pgfqpoint{3.991019in}{0.606124in}}{\pgfqpoint{3.991019in}{0.617174in}}%
\pgfpathcurveto{\pgfqpoint{3.991019in}{0.628224in}}{\pgfqpoint{3.986629in}{0.638823in}}{\pgfqpoint{3.978815in}{0.646637in}}%
\pgfpathcurveto{\pgfqpoint{3.971002in}{0.654450in}}{\pgfqpoint{3.960403in}{0.658841in}}{\pgfqpoint{3.949353in}{0.658841in}}%
\pgfpathcurveto{\pgfqpoint{3.938303in}{0.658841in}}{\pgfqpoint{3.927704in}{0.654450in}}{\pgfqpoint{3.919890in}{0.646637in}}%
\pgfpathcurveto{\pgfqpoint{3.912076in}{0.638823in}}{\pgfqpoint{3.907686in}{0.628224in}}{\pgfqpoint{3.907686in}{0.617174in}}%
\pgfpathcurveto{\pgfqpoint{3.907686in}{0.606124in}}{\pgfqpoint{3.912076in}{0.595525in}}{\pgfqpoint{3.919890in}{0.587711in}}%
\pgfpathcurveto{\pgfqpoint{3.927704in}{0.579897in}}{\pgfqpoint{3.938303in}{0.575507in}}{\pgfqpoint{3.949353in}{0.575507in}}%
\pgfpathlineto{\pgfqpoint{3.949353in}{0.575507in}}%
\pgfpathclose%
\pgfusepath{stroke}%
\end{pgfscope}%
\begin{pgfscope}%
\pgfpathrectangle{\pgfqpoint{0.494722in}{0.437222in}}{\pgfqpoint{6.275590in}{5.159444in}}%
\pgfusepath{clip}%
\pgfsetbuttcap%
\pgfsetroundjoin%
\pgfsetlinewidth{1.003750pt}%
\definecolor{currentstroke}{rgb}{0.827451,0.827451,0.827451}%
\pgfsetstrokecolor{currentstroke}%
\pgfsetstrokeopacity{0.800000}%
\pgfsetdash{}{0pt}%
\pgfpathmoveto{\pgfqpoint{4.561262in}{0.468696in}}%
\pgfpathcurveto{\pgfqpoint{4.572312in}{0.468696in}}{\pgfqpoint{4.582911in}{0.473086in}}{\pgfqpoint{4.590724in}{0.480900in}}%
\pgfpathcurveto{\pgfqpoint{4.598538in}{0.488713in}}{\pgfqpoint{4.602928in}{0.499312in}}{\pgfqpoint{4.602928in}{0.510362in}}%
\pgfpathcurveto{\pgfqpoint{4.602928in}{0.521413in}}{\pgfqpoint{4.598538in}{0.532012in}}{\pgfqpoint{4.590724in}{0.539825in}}%
\pgfpathcurveto{\pgfqpoint{4.582911in}{0.547639in}}{\pgfqpoint{4.572312in}{0.552029in}}{\pgfqpoint{4.561262in}{0.552029in}}%
\pgfpathcurveto{\pgfqpoint{4.550212in}{0.552029in}}{\pgfqpoint{4.539613in}{0.547639in}}{\pgfqpoint{4.531799in}{0.539825in}}%
\pgfpathcurveto{\pgfqpoint{4.523985in}{0.532012in}}{\pgfqpoint{4.519595in}{0.521413in}}{\pgfqpoint{4.519595in}{0.510362in}}%
\pgfpathcurveto{\pgfqpoint{4.519595in}{0.499312in}}{\pgfqpoint{4.523985in}{0.488713in}}{\pgfqpoint{4.531799in}{0.480900in}}%
\pgfpathcurveto{\pgfqpoint{4.539613in}{0.473086in}}{\pgfqpoint{4.550212in}{0.468696in}}{\pgfqpoint{4.561262in}{0.468696in}}%
\pgfpathlineto{\pgfqpoint{4.561262in}{0.468696in}}%
\pgfpathclose%
\pgfusepath{stroke}%
\end{pgfscope}%
\begin{pgfscope}%
\pgfpathrectangle{\pgfqpoint{0.494722in}{0.437222in}}{\pgfqpoint{6.275590in}{5.159444in}}%
\pgfusepath{clip}%
\pgfsetbuttcap%
\pgfsetroundjoin%
\pgfsetlinewidth{1.003750pt}%
\definecolor{currentstroke}{rgb}{0.827451,0.827451,0.827451}%
\pgfsetstrokecolor{currentstroke}%
\pgfsetstrokeopacity{0.800000}%
\pgfsetdash{}{0pt}%
\pgfpathmoveto{\pgfqpoint{2.747698in}{1.054484in}}%
\pgfpathcurveto{\pgfqpoint{2.758748in}{1.054484in}}{\pgfqpoint{2.769347in}{1.058875in}}{\pgfqpoint{2.777161in}{1.066688in}}%
\pgfpathcurveto{\pgfqpoint{2.784974in}{1.074502in}}{\pgfqpoint{2.789365in}{1.085101in}}{\pgfqpoint{2.789365in}{1.096151in}}%
\pgfpathcurveto{\pgfqpoint{2.789365in}{1.107201in}}{\pgfqpoint{2.784974in}{1.117800in}}{\pgfqpoint{2.777161in}{1.125614in}}%
\pgfpathcurveto{\pgfqpoint{2.769347in}{1.133427in}}{\pgfqpoint{2.758748in}{1.137818in}}{\pgfqpoint{2.747698in}{1.137818in}}%
\pgfpathcurveto{\pgfqpoint{2.736648in}{1.137818in}}{\pgfqpoint{2.726049in}{1.133427in}}{\pgfqpoint{2.718235in}{1.125614in}}%
\pgfpathcurveto{\pgfqpoint{2.710422in}{1.117800in}}{\pgfqpoint{2.706031in}{1.107201in}}{\pgfqpoint{2.706031in}{1.096151in}}%
\pgfpathcurveto{\pgfqpoint{2.706031in}{1.085101in}}{\pgfqpoint{2.710422in}{1.074502in}}{\pgfqpoint{2.718235in}{1.066688in}}%
\pgfpathcurveto{\pgfqpoint{2.726049in}{1.058875in}}{\pgfqpoint{2.736648in}{1.054484in}}{\pgfqpoint{2.747698in}{1.054484in}}%
\pgfpathlineto{\pgfqpoint{2.747698in}{1.054484in}}%
\pgfpathclose%
\pgfusepath{stroke}%
\end{pgfscope}%
\begin{pgfscope}%
\pgfpathrectangle{\pgfqpoint{0.494722in}{0.437222in}}{\pgfqpoint{6.275590in}{5.159444in}}%
\pgfusepath{clip}%
\pgfsetbuttcap%
\pgfsetroundjoin%
\pgfsetlinewidth{1.003750pt}%
\definecolor{currentstroke}{rgb}{0.827451,0.827451,0.827451}%
\pgfsetstrokecolor{currentstroke}%
\pgfsetstrokeopacity{0.800000}%
\pgfsetdash{}{0pt}%
\pgfpathmoveto{\pgfqpoint{1.198501in}{2.304477in}}%
\pgfpathcurveto{\pgfqpoint{1.209551in}{2.304477in}}{\pgfqpoint{1.220150in}{2.308867in}}{\pgfqpoint{1.227963in}{2.316681in}}%
\pgfpathcurveto{\pgfqpoint{1.235777in}{2.324495in}}{\pgfqpoint{1.240167in}{2.335094in}}{\pgfqpoint{1.240167in}{2.346144in}}%
\pgfpathcurveto{\pgfqpoint{1.240167in}{2.357194in}}{\pgfqpoint{1.235777in}{2.367793in}}{\pgfqpoint{1.227963in}{2.375607in}}%
\pgfpathcurveto{\pgfqpoint{1.220150in}{2.383420in}}{\pgfqpoint{1.209551in}{2.387811in}}{\pgfqpoint{1.198501in}{2.387811in}}%
\pgfpathcurveto{\pgfqpoint{1.187451in}{2.387811in}}{\pgfqpoint{1.176852in}{2.383420in}}{\pgfqpoint{1.169038in}{2.375607in}}%
\pgfpathcurveto{\pgfqpoint{1.161224in}{2.367793in}}{\pgfqpoint{1.156834in}{2.357194in}}{\pgfqpoint{1.156834in}{2.346144in}}%
\pgfpathcurveto{\pgfqpoint{1.156834in}{2.335094in}}{\pgfqpoint{1.161224in}{2.324495in}}{\pgfqpoint{1.169038in}{2.316681in}}%
\pgfpathcurveto{\pgfqpoint{1.176852in}{2.308867in}}{\pgfqpoint{1.187451in}{2.304477in}}{\pgfqpoint{1.198501in}{2.304477in}}%
\pgfpathlineto{\pgfqpoint{1.198501in}{2.304477in}}%
\pgfpathclose%
\pgfusepath{stroke}%
\end{pgfscope}%
\begin{pgfscope}%
\pgfpathrectangle{\pgfqpoint{0.494722in}{0.437222in}}{\pgfqpoint{6.275590in}{5.159444in}}%
\pgfusepath{clip}%
\pgfsetbuttcap%
\pgfsetroundjoin%
\pgfsetlinewidth{1.003750pt}%
\definecolor{currentstroke}{rgb}{0.827451,0.827451,0.827451}%
\pgfsetstrokecolor{currentstroke}%
\pgfsetstrokeopacity{0.800000}%
\pgfsetdash{}{0pt}%
\pgfpathmoveto{\pgfqpoint{2.785306in}{1.003395in}}%
\pgfpathcurveto{\pgfqpoint{2.796356in}{1.003395in}}{\pgfqpoint{2.806955in}{1.007786in}}{\pgfqpoint{2.814769in}{1.015599in}}%
\pgfpathcurveto{\pgfqpoint{2.822582in}{1.023413in}}{\pgfqpoint{2.826973in}{1.034012in}}{\pgfqpoint{2.826973in}{1.045062in}}%
\pgfpathcurveto{\pgfqpoint{2.826973in}{1.056112in}}{\pgfqpoint{2.822582in}{1.066711in}}{\pgfqpoint{2.814769in}{1.074525in}}%
\pgfpathcurveto{\pgfqpoint{2.806955in}{1.082338in}}{\pgfqpoint{2.796356in}{1.086729in}}{\pgfqpoint{2.785306in}{1.086729in}}%
\pgfpathcurveto{\pgfqpoint{2.774256in}{1.086729in}}{\pgfqpoint{2.763657in}{1.082338in}}{\pgfqpoint{2.755843in}{1.074525in}}%
\pgfpathcurveto{\pgfqpoint{2.748030in}{1.066711in}}{\pgfqpoint{2.743639in}{1.056112in}}{\pgfqpoint{2.743639in}{1.045062in}}%
\pgfpathcurveto{\pgfqpoint{2.743639in}{1.034012in}}{\pgfqpoint{2.748030in}{1.023413in}}{\pgfqpoint{2.755843in}{1.015599in}}%
\pgfpathcurveto{\pgfqpoint{2.763657in}{1.007786in}}{\pgfqpoint{2.774256in}{1.003395in}}{\pgfqpoint{2.785306in}{1.003395in}}%
\pgfpathlineto{\pgfqpoint{2.785306in}{1.003395in}}%
\pgfpathclose%
\pgfusepath{stroke}%
\end{pgfscope}%
\begin{pgfscope}%
\pgfpathrectangle{\pgfqpoint{0.494722in}{0.437222in}}{\pgfqpoint{6.275590in}{5.159444in}}%
\pgfusepath{clip}%
\pgfsetbuttcap%
\pgfsetroundjoin%
\pgfsetlinewidth{1.003750pt}%
\definecolor{currentstroke}{rgb}{0.827451,0.827451,0.827451}%
\pgfsetstrokecolor{currentstroke}%
\pgfsetstrokeopacity{0.800000}%
\pgfsetdash{}{0pt}%
\pgfpathmoveto{\pgfqpoint{0.553084in}{3.907249in}}%
\pgfpathcurveto{\pgfqpoint{0.564134in}{3.907249in}}{\pgfqpoint{0.574733in}{3.911639in}}{\pgfqpoint{0.582547in}{3.919453in}}%
\pgfpathcurveto{\pgfqpoint{0.590360in}{3.927267in}}{\pgfqpoint{0.594750in}{3.937866in}}{\pgfqpoint{0.594750in}{3.948916in}}%
\pgfpathcurveto{\pgfqpoint{0.594750in}{3.959966in}}{\pgfqpoint{0.590360in}{3.970565in}}{\pgfqpoint{0.582547in}{3.978378in}}%
\pgfpathcurveto{\pgfqpoint{0.574733in}{3.986192in}}{\pgfqpoint{0.564134in}{3.990582in}}{\pgfqpoint{0.553084in}{3.990582in}}%
\pgfpathcurveto{\pgfqpoint{0.542034in}{3.990582in}}{\pgfqpoint{0.531435in}{3.986192in}}{\pgfqpoint{0.523621in}{3.978378in}}%
\pgfpathcurveto{\pgfqpoint{0.515807in}{3.970565in}}{\pgfqpoint{0.511417in}{3.959966in}}{\pgfqpoint{0.511417in}{3.948916in}}%
\pgfpathcurveto{\pgfqpoint{0.511417in}{3.937866in}}{\pgfqpoint{0.515807in}{3.927267in}}{\pgfqpoint{0.523621in}{3.919453in}}%
\pgfpathcurveto{\pgfqpoint{0.531435in}{3.911639in}}{\pgfqpoint{0.542034in}{3.907249in}}{\pgfqpoint{0.553084in}{3.907249in}}%
\pgfpathlineto{\pgfqpoint{0.553084in}{3.907249in}}%
\pgfpathclose%
\pgfusepath{stroke}%
\end{pgfscope}%
\begin{pgfscope}%
\pgfpathrectangle{\pgfqpoint{0.494722in}{0.437222in}}{\pgfqpoint{6.275590in}{5.159444in}}%
\pgfusepath{clip}%
\pgfsetbuttcap%
\pgfsetroundjoin%
\pgfsetlinewidth{1.003750pt}%
\definecolor{currentstroke}{rgb}{0.827451,0.827451,0.827451}%
\pgfsetstrokecolor{currentstroke}%
\pgfsetstrokeopacity{0.800000}%
\pgfsetdash{}{0pt}%
\pgfpathmoveto{\pgfqpoint{2.946434in}{0.922859in}}%
\pgfpathcurveto{\pgfqpoint{2.957484in}{0.922859in}}{\pgfqpoint{2.968083in}{0.927249in}}{\pgfqpoint{2.975897in}{0.935063in}}%
\pgfpathcurveto{\pgfqpoint{2.983711in}{0.942876in}}{\pgfqpoint{2.988101in}{0.953475in}}{\pgfqpoint{2.988101in}{0.964525in}}%
\pgfpathcurveto{\pgfqpoint{2.988101in}{0.975576in}}{\pgfqpoint{2.983711in}{0.986175in}}{\pgfqpoint{2.975897in}{0.993988in}}%
\pgfpathcurveto{\pgfqpoint{2.968083in}{1.001802in}}{\pgfqpoint{2.957484in}{1.006192in}}{\pgfqpoint{2.946434in}{1.006192in}}%
\pgfpathcurveto{\pgfqpoint{2.935384in}{1.006192in}}{\pgfqpoint{2.924785in}{1.001802in}}{\pgfqpoint{2.916971in}{0.993988in}}%
\pgfpathcurveto{\pgfqpoint{2.909158in}{0.986175in}}{\pgfqpoint{2.904768in}{0.975576in}}{\pgfqpoint{2.904768in}{0.964525in}}%
\pgfpathcurveto{\pgfqpoint{2.904768in}{0.953475in}}{\pgfqpoint{2.909158in}{0.942876in}}{\pgfqpoint{2.916971in}{0.935063in}}%
\pgfpathcurveto{\pgfqpoint{2.924785in}{0.927249in}}{\pgfqpoint{2.935384in}{0.922859in}}{\pgfqpoint{2.946434in}{0.922859in}}%
\pgfpathlineto{\pgfqpoint{2.946434in}{0.922859in}}%
\pgfpathclose%
\pgfusepath{stroke}%
\end{pgfscope}%
\begin{pgfscope}%
\pgfpathrectangle{\pgfqpoint{0.494722in}{0.437222in}}{\pgfqpoint{6.275590in}{5.159444in}}%
\pgfusepath{clip}%
\pgfsetbuttcap%
\pgfsetroundjoin%
\pgfsetlinewidth{1.003750pt}%
\definecolor{currentstroke}{rgb}{0.827451,0.827451,0.827451}%
\pgfsetstrokecolor{currentstroke}%
\pgfsetstrokeopacity{0.800000}%
\pgfsetdash{}{0pt}%
\pgfpathmoveto{\pgfqpoint{3.295644in}{0.786846in}}%
\pgfpathcurveto{\pgfqpoint{3.306694in}{0.786846in}}{\pgfqpoint{3.317293in}{0.791236in}}{\pgfqpoint{3.325106in}{0.799050in}}%
\pgfpathcurveto{\pgfqpoint{3.332920in}{0.806864in}}{\pgfqpoint{3.337310in}{0.817463in}}{\pgfqpoint{3.337310in}{0.828513in}}%
\pgfpathcurveto{\pgfqpoint{3.337310in}{0.839563in}}{\pgfqpoint{3.332920in}{0.850162in}}{\pgfqpoint{3.325106in}{0.857976in}}%
\pgfpathcurveto{\pgfqpoint{3.317293in}{0.865789in}}{\pgfqpoint{3.306694in}{0.870179in}}{\pgfqpoint{3.295644in}{0.870179in}}%
\pgfpathcurveto{\pgfqpoint{3.284593in}{0.870179in}}{\pgfqpoint{3.273994in}{0.865789in}}{\pgfqpoint{3.266181in}{0.857976in}}%
\pgfpathcurveto{\pgfqpoint{3.258367in}{0.850162in}}{\pgfqpoint{3.253977in}{0.839563in}}{\pgfqpoint{3.253977in}{0.828513in}}%
\pgfpathcurveto{\pgfqpoint{3.253977in}{0.817463in}}{\pgfqpoint{3.258367in}{0.806864in}}{\pgfqpoint{3.266181in}{0.799050in}}%
\pgfpathcurveto{\pgfqpoint{3.273994in}{0.791236in}}{\pgfqpoint{3.284593in}{0.786846in}}{\pgfqpoint{3.295644in}{0.786846in}}%
\pgfpathlineto{\pgfqpoint{3.295644in}{0.786846in}}%
\pgfpathclose%
\pgfusepath{stroke}%
\end{pgfscope}%
\begin{pgfscope}%
\pgfpathrectangle{\pgfqpoint{0.494722in}{0.437222in}}{\pgfqpoint{6.275590in}{5.159444in}}%
\pgfusepath{clip}%
\pgfsetbuttcap%
\pgfsetroundjoin%
\pgfsetlinewidth{1.003750pt}%
\definecolor{currentstroke}{rgb}{0.827451,0.827451,0.827451}%
\pgfsetstrokecolor{currentstroke}%
\pgfsetstrokeopacity{0.800000}%
\pgfsetdash{}{0pt}%
\pgfpathmoveto{\pgfqpoint{1.671516in}{1.760843in}}%
\pgfpathcurveto{\pgfqpoint{1.682566in}{1.760843in}}{\pgfqpoint{1.693165in}{1.765233in}}{\pgfqpoint{1.700979in}{1.773046in}}%
\pgfpathcurveto{\pgfqpoint{1.708792in}{1.780860in}}{\pgfqpoint{1.713183in}{1.791459in}}{\pgfqpoint{1.713183in}{1.802509in}}%
\pgfpathcurveto{\pgfqpoint{1.713183in}{1.813559in}}{\pgfqpoint{1.708792in}{1.824158in}}{\pgfqpoint{1.700979in}{1.831972in}}%
\pgfpathcurveto{\pgfqpoint{1.693165in}{1.839786in}}{\pgfqpoint{1.682566in}{1.844176in}}{\pgfqpoint{1.671516in}{1.844176in}}%
\pgfpathcurveto{\pgfqpoint{1.660466in}{1.844176in}}{\pgfqpoint{1.649867in}{1.839786in}}{\pgfqpoint{1.642053in}{1.831972in}}%
\pgfpathcurveto{\pgfqpoint{1.634240in}{1.824158in}}{\pgfqpoint{1.629849in}{1.813559in}}{\pgfqpoint{1.629849in}{1.802509in}}%
\pgfpathcurveto{\pgfqpoint{1.629849in}{1.791459in}}{\pgfqpoint{1.634240in}{1.780860in}}{\pgfqpoint{1.642053in}{1.773046in}}%
\pgfpathcurveto{\pgfqpoint{1.649867in}{1.765233in}}{\pgfqpoint{1.660466in}{1.760843in}}{\pgfqpoint{1.671516in}{1.760843in}}%
\pgfpathlineto{\pgfqpoint{1.671516in}{1.760843in}}%
\pgfpathclose%
\pgfusepath{stroke}%
\end{pgfscope}%
\begin{pgfscope}%
\pgfpathrectangle{\pgfqpoint{0.494722in}{0.437222in}}{\pgfqpoint{6.275590in}{5.159444in}}%
\pgfusepath{clip}%
\pgfsetbuttcap%
\pgfsetroundjoin%
\pgfsetlinewidth{1.003750pt}%
\definecolor{currentstroke}{rgb}{0.827451,0.827451,0.827451}%
\pgfsetstrokecolor{currentstroke}%
\pgfsetstrokeopacity{0.800000}%
\pgfsetdash{}{0pt}%
\pgfpathmoveto{\pgfqpoint{0.834443in}{2.917471in}}%
\pgfpathcurveto{\pgfqpoint{0.845494in}{2.917471in}}{\pgfqpoint{0.856093in}{2.921861in}}{\pgfqpoint{0.863906in}{2.929674in}}%
\pgfpathcurveto{\pgfqpoint{0.871720in}{2.937488in}}{\pgfqpoint{0.876110in}{2.948087in}}{\pgfqpoint{0.876110in}{2.959137in}}%
\pgfpathcurveto{\pgfqpoint{0.876110in}{2.970187in}}{\pgfqpoint{0.871720in}{2.980786in}}{\pgfqpoint{0.863906in}{2.988600in}}%
\pgfpathcurveto{\pgfqpoint{0.856093in}{2.996414in}}{\pgfqpoint{0.845494in}{3.000804in}}{\pgfqpoint{0.834443in}{3.000804in}}%
\pgfpathcurveto{\pgfqpoint{0.823393in}{3.000804in}}{\pgfqpoint{0.812794in}{2.996414in}}{\pgfqpoint{0.804981in}{2.988600in}}%
\pgfpathcurveto{\pgfqpoint{0.797167in}{2.980786in}}{\pgfqpoint{0.792777in}{2.970187in}}{\pgfqpoint{0.792777in}{2.959137in}}%
\pgfpathcurveto{\pgfqpoint{0.792777in}{2.948087in}}{\pgfqpoint{0.797167in}{2.937488in}}{\pgfqpoint{0.804981in}{2.929674in}}%
\pgfpathcurveto{\pgfqpoint{0.812794in}{2.921861in}}{\pgfqpoint{0.823393in}{2.917471in}}{\pgfqpoint{0.834443in}{2.917471in}}%
\pgfpathlineto{\pgfqpoint{0.834443in}{2.917471in}}%
\pgfpathclose%
\pgfusepath{stroke}%
\end{pgfscope}%
\begin{pgfscope}%
\pgfpathrectangle{\pgfqpoint{0.494722in}{0.437222in}}{\pgfqpoint{6.275590in}{5.159444in}}%
\pgfusepath{clip}%
\pgfsetbuttcap%
\pgfsetroundjoin%
\pgfsetlinewidth{1.003750pt}%
\definecolor{currentstroke}{rgb}{0.827451,0.827451,0.827451}%
\pgfsetstrokecolor{currentstroke}%
\pgfsetstrokeopacity{0.800000}%
\pgfsetdash{}{0pt}%
\pgfpathmoveto{\pgfqpoint{3.753443in}{0.622825in}}%
\pgfpathcurveto{\pgfqpoint{3.764493in}{0.622825in}}{\pgfqpoint{3.775092in}{0.627215in}}{\pgfqpoint{3.782905in}{0.635029in}}%
\pgfpathcurveto{\pgfqpoint{3.790719in}{0.642842in}}{\pgfqpoint{3.795109in}{0.653441in}}{\pgfqpoint{3.795109in}{0.664491in}}%
\pgfpathcurveto{\pgfqpoint{3.795109in}{0.675542in}}{\pgfqpoint{3.790719in}{0.686141in}}{\pgfqpoint{3.782905in}{0.693954in}}%
\pgfpathcurveto{\pgfqpoint{3.775092in}{0.701768in}}{\pgfqpoint{3.764493in}{0.706158in}}{\pgfqpoint{3.753443in}{0.706158in}}%
\pgfpathcurveto{\pgfqpoint{3.742393in}{0.706158in}}{\pgfqpoint{3.731794in}{0.701768in}}{\pgfqpoint{3.723980in}{0.693954in}}%
\pgfpathcurveto{\pgfqpoint{3.716166in}{0.686141in}}{\pgfqpoint{3.711776in}{0.675542in}}{\pgfqpoint{3.711776in}{0.664491in}}%
\pgfpathcurveto{\pgfqpoint{3.711776in}{0.653441in}}{\pgfqpoint{3.716166in}{0.642842in}}{\pgfqpoint{3.723980in}{0.635029in}}%
\pgfpathcurveto{\pgfqpoint{3.731794in}{0.627215in}}{\pgfqpoint{3.742393in}{0.622825in}}{\pgfqpoint{3.753443in}{0.622825in}}%
\pgfpathlineto{\pgfqpoint{3.753443in}{0.622825in}}%
\pgfpathclose%
\pgfusepath{stroke}%
\end{pgfscope}%
\begin{pgfscope}%
\pgfpathrectangle{\pgfqpoint{0.494722in}{0.437222in}}{\pgfqpoint{6.275590in}{5.159444in}}%
\pgfusepath{clip}%
\pgfsetbuttcap%
\pgfsetroundjoin%
\pgfsetlinewidth{1.003750pt}%
\definecolor{currentstroke}{rgb}{0.827451,0.827451,0.827451}%
\pgfsetstrokecolor{currentstroke}%
\pgfsetstrokeopacity{0.800000}%
\pgfsetdash{}{0pt}%
\pgfpathmoveto{\pgfqpoint{5.544527in}{0.401839in}}%
\pgfpathcurveto{\pgfqpoint{5.555577in}{0.401839in}}{\pgfqpoint{5.566176in}{0.406230in}}{\pgfqpoint{5.573990in}{0.414043in}}%
\pgfpathcurveto{\pgfqpoint{5.581803in}{0.421857in}}{\pgfqpoint{5.586193in}{0.432456in}}{\pgfqpoint{5.586193in}{0.443506in}}%
\pgfpathcurveto{\pgfqpoint{5.586193in}{0.454556in}}{\pgfqpoint{5.581803in}{0.465155in}}{\pgfqpoint{5.573990in}{0.472969in}}%
\pgfpathcurveto{\pgfqpoint{5.566176in}{0.480782in}}{\pgfqpoint{5.555577in}{0.485173in}}{\pgfqpoint{5.544527in}{0.485173in}}%
\pgfpathcurveto{\pgfqpoint{5.533477in}{0.485173in}}{\pgfqpoint{5.522878in}{0.480782in}}{\pgfqpoint{5.515064in}{0.472969in}}%
\pgfpathcurveto{\pgfqpoint{5.507250in}{0.465155in}}{\pgfqpoint{5.502860in}{0.454556in}}{\pgfqpoint{5.502860in}{0.443506in}}%
\pgfpathcurveto{\pgfqpoint{5.502860in}{0.432456in}}{\pgfqpoint{5.507250in}{0.421857in}}{\pgfqpoint{5.515064in}{0.414043in}}%
\pgfpathcurveto{\pgfqpoint{5.522878in}{0.406230in}}{\pgfqpoint{5.533477in}{0.401839in}}{\pgfqpoint{5.544527in}{0.401839in}}%
\pgfusepath{stroke}%
\end{pgfscope}%
\begin{pgfscope}%
\pgfpathrectangle{\pgfqpoint{0.494722in}{0.437222in}}{\pgfqpoint{6.275590in}{5.159444in}}%
\pgfusepath{clip}%
\pgfsetbuttcap%
\pgfsetroundjoin%
\pgfsetlinewidth{1.003750pt}%
\definecolor{currentstroke}{rgb}{0.827451,0.827451,0.827451}%
\pgfsetstrokecolor{currentstroke}%
\pgfsetstrokeopacity{0.800000}%
\pgfsetdash{}{0pt}%
\pgfpathmoveto{\pgfqpoint{0.994863in}{2.601032in}}%
\pgfpathcurveto{\pgfqpoint{1.005913in}{2.601032in}}{\pgfqpoint{1.016513in}{2.605422in}}{\pgfqpoint{1.024326in}{2.613236in}}%
\pgfpathcurveto{\pgfqpoint{1.032140in}{2.621050in}}{\pgfqpoint{1.036530in}{2.631649in}}{\pgfqpoint{1.036530in}{2.642699in}}%
\pgfpathcurveto{\pgfqpoint{1.036530in}{2.653749in}}{\pgfqpoint{1.032140in}{2.664348in}}{\pgfqpoint{1.024326in}{2.672162in}}%
\pgfpathcurveto{\pgfqpoint{1.016513in}{2.679975in}}{\pgfqpoint{1.005913in}{2.684365in}}{\pgfqpoint{0.994863in}{2.684365in}}%
\pgfpathcurveto{\pgfqpoint{0.983813in}{2.684365in}}{\pgfqpoint{0.973214in}{2.679975in}}{\pgfqpoint{0.965401in}{2.672162in}}%
\pgfpathcurveto{\pgfqpoint{0.957587in}{2.664348in}}{\pgfqpoint{0.953197in}{2.653749in}}{\pgfqpoint{0.953197in}{2.642699in}}%
\pgfpathcurveto{\pgfqpoint{0.953197in}{2.631649in}}{\pgfqpoint{0.957587in}{2.621050in}}{\pgfqpoint{0.965401in}{2.613236in}}%
\pgfpathcurveto{\pgfqpoint{0.973214in}{2.605422in}}{\pgfqpoint{0.983813in}{2.601032in}}{\pgfqpoint{0.994863in}{2.601032in}}%
\pgfpathlineto{\pgfqpoint{0.994863in}{2.601032in}}%
\pgfpathclose%
\pgfusepath{stroke}%
\end{pgfscope}%
\begin{pgfscope}%
\pgfpathrectangle{\pgfqpoint{0.494722in}{0.437222in}}{\pgfqpoint{6.275590in}{5.159444in}}%
\pgfusepath{clip}%
\pgfsetbuttcap%
\pgfsetroundjoin%
\pgfsetlinewidth{1.003750pt}%
\definecolor{currentstroke}{rgb}{0.827451,0.827451,0.827451}%
\pgfsetstrokecolor{currentstroke}%
\pgfsetstrokeopacity{0.800000}%
\pgfsetdash{}{0pt}%
\pgfpathmoveto{\pgfqpoint{1.381781in}{2.056420in}}%
\pgfpathcurveto{\pgfqpoint{1.392831in}{2.056420in}}{\pgfqpoint{1.403430in}{2.060810in}}{\pgfqpoint{1.411244in}{2.068624in}}%
\pgfpathcurveto{\pgfqpoint{1.419057in}{2.076437in}}{\pgfqpoint{1.423448in}{2.087036in}}{\pgfqpoint{1.423448in}{2.098087in}}%
\pgfpathcurveto{\pgfqpoint{1.423448in}{2.109137in}}{\pgfqpoint{1.419057in}{2.119736in}}{\pgfqpoint{1.411244in}{2.127549in}}%
\pgfpathcurveto{\pgfqpoint{1.403430in}{2.135363in}}{\pgfqpoint{1.392831in}{2.139753in}}{\pgfqpoint{1.381781in}{2.139753in}}%
\pgfpathcurveto{\pgfqpoint{1.370731in}{2.139753in}}{\pgfqpoint{1.360132in}{2.135363in}}{\pgfqpoint{1.352318in}{2.127549in}}%
\pgfpathcurveto{\pgfqpoint{1.344505in}{2.119736in}}{\pgfqpoint{1.340114in}{2.109137in}}{\pgfqpoint{1.340114in}{2.098087in}}%
\pgfpathcurveto{\pgfqpoint{1.340114in}{2.087036in}}{\pgfqpoint{1.344505in}{2.076437in}}{\pgfqpoint{1.352318in}{2.068624in}}%
\pgfpathcurveto{\pgfqpoint{1.360132in}{2.060810in}}{\pgfqpoint{1.370731in}{2.056420in}}{\pgfqpoint{1.381781in}{2.056420in}}%
\pgfpathlineto{\pgfqpoint{1.381781in}{2.056420in}}%
\pgfpathclose%
\pgfusepath{stroke}%
\end{pgfscope}%
\begin{pgfscope}%
\pgfpathrectangle{\pgfqpoint{0.494722in}{0.437222in}}{\pgfqpoint{6.275590in}{5.159444in}}%
\pgfusepath{clip}%
\pgfsetbuttcap%
\pgfsetroundjoin%
\pgfsetlinewidth{1.003750pt}%
\definecolor{currentstroke}{rgb}{0.827451,0.827451,0.827451}%
\pgfsetstrokecolor{currentstroke}%
\pgfsetstrokeopacity{0.800000}%
\pgfsetdash{}{0pt}%
\pgfpathmoveto{\pgfqpoint{1.161602in}{2.366586in}}%
\pgfpathcurveto{\pgfqpoint{1.172652in}{2.366586in}}{\pgfqpoint{1.183251in}{2.370976in}}{\pgfqpoint{1.191065in}{2.378790in}}%
\pgfpathcurveto{\pgfqpoint{1.198879in}{2.386603in}}{\pgfqpoint{1.203269in}{2.397202in}}{\pgfqpoint{1.203269in}{2.408252in}}%
\pgfpathcurveto{\pgfqpoint{1.203269in}{2.419303in}}{\pgfqpoint{1.198879in}{2.429902in}}{\pgfqpoint{1.191065in}{2.437715in}}%
\pgfpathcurveto{\pgfqpoint{1.183251in}{2.445529in}}{\pgfqpoint{1.172652in}{2.449919in}}{\pgfqpoint{1.161602in}{2.449919in}}%
\pgfpathcurveto{\pgfqpoint{1.150552in}{2.449919in}}{\pgfqpoint{1.139953in}{2.445529in}}{\pgfqpoint{1.132139in}{2.437715in}}%
\pgfpathcurveto{\pgfqpoint{1.124326in}{2.429902in}}{\pgfqpoint{1.119935in}{2.419303in}}{\pgfqpoint{1.119935in}{2.408252in}}%
\pgfpathcurveto{\pgfqpoint{1.119935in}{2.397202in}}{\pgfqpoint{1.124326in}{2.386603in}}{\pgfqpoint{1.132139in}{2.378790in}}%
\pgfpathcurveto{\pgfqpoint{1.139953in}{2.370976in}}{\pgfqpoint{1.150552in}{2.366586in}}{\pgfqpoint{1.161602in}{2.366586in}}%
\pgfpathlineto{\pgfqpoint{1.161602in}{2.366586in}}%
\pgfpathclose%
\pgfusepath{stroke}%
\end{pgfscope}%
\begin{pgfscope}%
\pgfpathrectangle{\pgfqpoint{0.494722in}{0.437222in}}{\pgfqpoint{6.275590in}{5.159444in}}%
\pgfusepath{clip}%
\pgfsetbuttcap%
\pgfsetroundjoin%
\pgfsetlinewidth{1.003750pt}%
\definecolor{currentstroke}{rgb}{0.827451,0.827451,0.827451}%
\pgfsetstrokecolor{currentstroke}%
\pgfsetstrokeopacity{0.800000}%
\pgfsetdash{}{0pt}%
\pgfpathmoveto{\pgfqpoint{2.175955in}{1.360063in}}%
\pgfpathcurveto{\pgfqpoint{2.187005in}{1.360063in}}{\pgfqpoint{2.197604in}{1.364454in}}{\pgfqpoint{2.205418in}{1.372267in}}%
\pgfpathcurveto{\pgfqpoint{2.213231in}{1.380081in}}{\pgfqpoint{2.217621in}{1.390680in}}{\pgfqpoint{2.217621in}{1.401730in}}%
\pgfpathcurveto{\pgfqpoint{2.217621in}{1.412780in}}{\pgfqpoint{2.213231in}{1.423379in}}{\pgfqpoint{2.205418in}{1.431193in}}%
\pgfpathcurveto{\pgfqpoint{2.197604in}{1.439006in}}{\pgfqpoint{2.187005in}{1.443397in}}{\pgfqpoint{2.175955in}{1.443397in}}%
\pgfpathcurveto{\pgfqpoint{2.164905in}{1.443397in}}{\pgfqpoint{2.154306in}{1.439006in}}{\pgfqpoint{2.146492in}{1.431193in}}%
\pgfpathcurveto{\pgfqpoint{2.138678in}{1.423379in}}{\pgfqpoint{2.134288in}{1.412780in}}{\pgfqpoint{2.134288in}{1.401730in}}%
\pgfpathcurveto{\pgfqpoint{2.134288in}{1.390680in}}{\pgfqpoint{2.138678in}{1.380081in}}{\pgfqpoint{2.146492in}{1.372267in}}%
\pgfpathcurveto{\pgfqpoint{2.154306in}{1.364454in}}{\pgfqpoint{2.164905in}{1.360063in}}{\pgfqpoint{2.175955in}{1.360063in}}%
\pgfpathlineto{\pgfqpoint{2.175955in}{1.360063in}}%
\pgfpathclose%
\pgfusepath{stroke}%
\end{pgfscope}%
\begin{pgfscope}%
\pgfpathrectangle{\pgfqpoint{0.494722in}{0.437222in}}{\pgfqpoint{6.275590in}{5.159444in}}%
\pgfusepath{clip}%
\pgfsetbuttcap%
\pgfsetroundjoin%
\pgfsetlinewidth{1.003750pt}%
\definecolor{currentstroke}{rgb}{0.827451,0.827451,0.827451}%
\pgfsetstrokecolor{currentstroke}%
\pgfsetstrokeopacity{0.800000}%
\pgfsetdash{}{0pt}%
\pgfpathmoveto{\pgfqpoint{0.584355in}{3.716646in}}%
\pgfpathcurveto{\pgfqpoint{0.595406in}{3.716646in}}{\pgfqpoint{0.606005in}{3.721036in}}{\pgfqpoint{0.613818in}{3.728850in}}%
\pgfpathcurveto{\pgfqpoint{0.621632in}{3.736663in}}{\pgfqpoint{0.626022in}{3.747262in}}{\pgfqpoint{0.626022in}{3.758313in}}%
\pgfpathcurveto{\pgfqpoint{0.626022in}{3.769363in}}{\pgfqpoint{0.621632in}{3.779962in}}{\pgfqpoint{0.613818in}{3.787775in}}%
\pgfpathcurveto{\pgfqpoint{0.606005in}{3.795589in}}{\pgfqpoint{0.595406in}{3.799979in}}{\pgfqpoint{0.584355in}{3.799979in}}%
\pgfpathcurveto{\pgfqpoint{0.573305in}{3.799979in}}{\pgfqpoint{0.562706in}{3.795589in}}{\pgfqpoint{0.554893in}{3.787775in}}%
\pgfpathcurveto{\pgfqpoint{0.547079in}{3.779962in}}{\pgfqpoint{0.542689in}{3.769363in}}{\pgfqpoint{0.542689in}{3.758313in}}%
\pgfpathcurveto{\pgfqpoint{0.542689in}{3.747262in}}{\pgfqpoint{0.547079in}{3.736663in}}{\pgfqpoint{0.554893in}{3.728850in}}%
\pgfpathcurveto{\pgfqpoint{0.562706in}{3.721036in}}{\pgfqpoint{0.573305in}{3.716646in}}{\pgfqpoint{0.584355in}{3.716646in}}%
\pgfpathlineto{\pgfqpoint{0.584355in}{3.716646in}}%
\pgfpathclose%
\pgfusepath{stroke}%
\end{pgfscope}%
\begin{pgfscope}%
\pgfpathrectangle{\pgfqpoint{0.494722in}{0.437222in}}{\pgfqpoint{6.275590in}{5.159444in}}%
\pgfusepath{clip}%
\pgfsetbuttcap%
\pgfsetroundjoin%
\pgfsetlinewidth{1.003750pt}%
\definecolor{currentstroke}{rgb}{0.827451,0.827451,0.827451}%
\pgfsetstrokecolor{currentstroke}%
\pgfsetstrokeopacity{0.800000}%
\pgfsetdash{}{0pt}%
\pgfpathmoveto{\pgfqpoint{5.217979in}{0.410951in}}%
\pgfpathcurveto{\pgfqpoint{5.229029in}{0.410951in}}{\pgfqpoint{5.239628in}{0.415341in}}{\pgfqpoint{5.247441in}{0.423155in}}%
\pgfpathcurveto{\pgfqpoint{5.255255in}{0.430969in}}{\pgfqpoint{5.259645in}{0.441568in}}{\pgfqpoint{5.259645in}{0.452618in}}%
\pgfpathcurveto{\pgfqpoint{5.259645in}{0.463668in}}{\pgfqpoint{5.255255in}{0.474267in}}{\pgfqpoint{5.247441in}{0.482081in}}%
\pgfpathcurveto{\pgfqpoint{5.239628in}{0.489894in}}{\pgfqpoint{5.229029in}{0.494284in}}{\pgfqpoint{5.217979in}{0.494284in}}%
\pgfpathcurveto{\pgfqpoint{5.206929in}{0.494284in}}{\pgfqpoint{5.196329in}{0.489894in}}{\pgfqpoint{5.188516in}{0.482081in}}%
\pgfpathcurveto{\pgfqpoint{5.180702in}{0.474267in}}{\pgfqpoint{5.176312in}{0.463668in}}{\pgfqpoint{5.176312in}{0.452618in}}%
\pgfpathcurveto{\pgfqpoint{5.176312in}{0.441568in}}{\pgfqpoint{5.180702in}{0.430969in}}{\pgfqpoint{5.188516in}{0.423155in}}%
\pgfpathcurveto{\pgfqpoint{5.196329in}{0.415341in}}{\pgfqpoint{5.206929in}{0.410951in}}{\pgfqpoint{5.217979in}{0.410951in}}%
\pgfusepath{stroke}%
\end{pgfscope}%
\begin{pgfscope}%
\pgfpathrectangle{\pgfqpoint{0.494722in}{0.437222in}}{\pgfqpoint{6.275590in}{5.159444in}}%
\pgfusepath{clip}%
\pgfsetbuttcap%
\pgfsetroundjoin%
\pgfsetlinewidth{1.003750pt}%
\definecolor{currentstroke}{rgb}{0.827451,0.827451,0.827451}%
\pgfsetstrokecolor{currentstroke}%
\pgfsetstrokeopacity{0.800000}%
\pgfsetdash{}{0pt}%
\pgfpathmoveto{\pgfqpoint{2.308193in}{1.269247in}}%
\pgfpathcurveto{\pgfqpoint{2.319243in}{1.269247in}}{\pgfqpoint{2.329842in}{1.273637in}}{\pgfqpoint{2.337656in}{1.281451in}}%
\pgfpathcurveto{\pgfqpoint{2.345469in}{1.289264in}}{\pgfqpoint{2.349860in}{1.299864in}}{\pgfqpoint{2.349860in}{1.310914in}}%
\pgfpathcurveto{\pgfqpoint{2.349860in}{1.321964in}}{\pgfqpoint{2.345469in}{1.332563in}}{\pgfqpoint{2.337656in}{1.340376in}}%
\pgfpathcurveto{\pgfqpoint{2.329842in}{1.348190in}}{\pgfqpoint{2.319243in}{1.352580in}}{\pgfqpoint{2.308193in}{1.352580in}}%
\pgfpathcurveto{\pgfqpoint{2.297143in}{1.352580in}}{\pgfqpoint{2.286544in}{1.348190in}}{\pgfqpoint{2.278730in}{1.340376in}}%
\pgfpathcurveto{\pgfqpoint{2.270917in}{1.332563in}}{\pgfqpoint{2.266526in}{1.321964in}}{\pgfqpoint{2.266526in}{1.310914in}}%
\pgfpathcurveto{\pgfqpoint{2.266526in}{1.299864in}}{\pgfqpoint{2.270917in}{1.289264in}}{\pgfqpoint{2.278730in}{1.281451in}}%
\pgfpathcurveto{\pgfqpoint{2.286544in}{1.273637in}}{\pgfqpoint{2.297143in}{1.269247in}}{\pgfqpoint{2.308193in}{1.269247in}}%
\pgfpathlineto{\pgfqpoint{2.308193in}{1.269247in}}%
\pgfpathclose%
\pgfusepath{stroke}%
\end{pgfscope}%
\begin{pgfscope}%
\pgfpathrectangle{\pgfqpoint{0.494722in}{0.437222in}}{\pgfqpoint{6.275590in}{5.159444in}}%
\pgfusepath{clip}%
\pgfsetbuttcap%
\pgfsetroundjoin%
\pgfsetlinewidth{1.003750pt}%
\definecolor{currentstroke}{rgb}{0.827451,0.827451,0.827451}%
\pgfsetstrokecolor{currentstroke}%
\pgfsetstrokeopacity{0.800000}%
\pgfsetdash{}{0pt}%
\pgfpathmoveto{\pgfqpoint{5.378653in}{0.404615in}}%
\pgfpathcurveto{\pgfqpoint{5.389703in}{0.404615in}}{\pgfqpoint{5.400302in}{0.409005in}}{\pgfqpoint{5.408116in}{0.416819in}}%
\pgfpathcurveto{\pgfqpoint{5.415930in}{0.424632in}}{\pgfqpoint{5.420320in}{0.435231in}}{\pgfqpoint{5.420320in}{0.446281in}}%
\pgfpathcurveto{\pgfqpoint{5.420320in}{0.457331in}}{\pgfqpoint{5.415930in}{0.467930in}}{\pgfqpoint{5.408116in}{0.475744in}}%
\pgfpathcurveto{\pgfqpoint{5.400302in}{0.483558in}}{\pgfqpoint{5.389703in}{0.487948in}}{\pgfqpoint{5.378653in}{0.487948in}}%
\pgfpathcurveto{\pgfqpoint{5.367603in}{0.487948in}}{\pgfqpoint{5.357004in}{0.483558in}}{\pgfqpoint{5.349190in}{0.475744in}}%
\pgfpathcurveto{\pgfqpoint{5.341377in}{0.467930in}}{\pgfqpoint{5.336987in}{0.457331in}}{\pgfqpoint{5.336987in}{0.446281in}}%
\pgfpathcurveto{\pgfqpoint{5.336987in}{0.435231in}}{\pgfqpoint{5.341377in}{0.424632in}}{\pgfqpoint{5.349190in}{0.416819in}}%
\pgfpathcurveto{\pgfqpoint{5.357004in}{0.409005in}}{\pgfqpoint{5.367603in}{0.404615in}}{\pgfqpoint{5.378653in}{0.404615in}}%
\pgfusepath{stroke}%
\end{pgfscope}%
\begin{pgfscope}%
\pgfpathrectangle{\pgfqpoint{0.494722in}{0.437222in}}{\pgfqpoint{6.275590in}{5.159444in}}%
\pgfusepath{clip}%
\pgfsetbuttcap%
\pgfsetroundjoin%
\pgfsetlinewidth{1.003750pt}%
\definecolor{currentstroke}{rgb}{0.827451,0.827451,0.827451}%
\pgfsetstrokecolor{currentstroke}%
\pgfsetstrokeopacity{0.800000}%
\pgfsetdash{}{0pt}%
\pgfpathmoveto{\pgfqpoint{1.963941in}{1.515375in}}%
\pgfpathcurveto{\pgfqpoint{1.974991in}{1.515375in}}{\pgfqpoint{1.985590in}{1.519765in}}{\pgfqpoint{1.993404in}{1.527579in}}%
\pgfpathcurveto{\pgfqpoint{2.001218in}{1.535392in}}{\pgfqpoint{2.005608in}{1.545991in}}{\pgfqpoint{2.005608in}{1.557042in}}%
\pgfpathcurveto{\pgfqpoint{2.005608in}{1.568092in}}{\pgfqpoint{2.001218in}{1.578691in}}{\pgfqpoint{1.993404in}{1.586504in}}%
\pgfpathcurveto{\pgfqpoint{1.985590in}{1.594318in}}{\pgfqpoint{1.974991in}{1.598708in}}{\pgfqpoint{1.963941in}{1.598708in}}%
\pgfpathcurveto{\pgfqpoint{1.952891in}{1.598708in}}{\pgfqpoint{1.942292in}{1.594318in}}{\pgfqpoint{1.934478in}{1.586504in}}%
\pgfpathcurveto{\pgfqpoint{1.926665in}{1.578691in}}{\pgfqpoint{1.922274in}{1.568092in}}{\pgfqpoint{1.922274in}{1.557042in}}%
\pgfpathcurveto{\pgfqpoint{1.922274in}{1.545991in}}{\pgfqpoint{1.926665in}{1.535392in}}{\pgfqpoint{1.934478in}{1.527579in}}%
\pgfpathcurveto{\pgfqpoint{1.942292in}{1.519765in}}{\pgfqpoint{1.952891in}{1.515375in}}{\pgfqpoint{1.963941in}{1.515375in}}%
\pgfpathlineto{\pgfqpoint{1.963941in}{1.515375in}}%
\pgfpathclose%
\pgfusepath{stroke}%
\end{pgfscope}%
\begin{pgfscope}%
\pgfpathrectangle{\pgfqpoint{0.494722in}{0.437222in}}{\pgfqpoint{6.275590in}{5.159444in}}%
\pgfusepath{clip}%
\pgfsetbuttcap%
\pgfsetroundjoin%
\pgfsetlinewidth{1.003750pt}%
\definecolor{currentstroke}{rgb}{0.827451,0.827451,0.827451}%
\pgfsetstrokecolor{currentstroke}%
\pgfsetstrokeopacity{0.800000}%
\pgfsetdash{}{0pt}%
\pgfpathmoveto{\pgfqpoint{4.067519in}{0.560512in}}%
\pgfpathcurveto{\pgfqpoint{4.078569in}{0.560512in}}{\pgfqpoint{4.089168in}{0.564902in}}{\pgfqpoint{4.096982in}{0.572715in}}%
\pgfpathcurveto{\pgfqpoint{4.104796in}{0.580529in}}{\pgfqpoint{4.109186in}{0.591128in}}{\pgfqpoint{4.109186in}{0.602178in}}%
\pgfpathcurveto{\pgfqpoint{4.109186in}{0.613228in}}{\pgfqpoint{4.104796in}{0.623827in}}{\pgfqpoint{4.096982in}{0.631641in}}%
\pgfpathcurveto{\pgfqpoint{4.089168in}{0.639455in}}{\pgfqpoint{4.078569in}{0.643845in}}{\pgfqpoint{4.067519in}{0.643845in}}%
\pgfpathcurveto{\pgfqpoint{4.056469in}{0.643845in}}{\pgfqpoint{4.045870in}{0.639455in}}{\pgfqpoint{4.038056in}{0.631641in}}%
\pgfpathcurveto{\pgfqpoint{4.030243in}{0.623827in}}{\pgfqpoint{4.025853in}{0.613228in}}{\pgfqpoint{4.025853in}{0.602178in}}%
\pgfpathcurveto{\pgfqpoint{4.025853in}{0.591128in}}{\pgfqpoint{4.030243in}{0.580529in}}{\pgfqpoint{4.038056in}{0.572715in}}%
\pgfpathcurveto{\pgfqpoint{4.045870in}{0.564902in}}{\pgfqpoint{4.056469in}{0.560512in}}{\pgfqpoint{4.067519in}{0.560512in}}%
\pgfpathlineto{\pgfqpoint{4.067519in}{0.560512in}}%
\pgfpathclose%
\pgfusepath{stroke}%
\end{pgfscope}%
\begin{pgfscope}%
\pgfpathrectangle{\pgfqpoint{0.494722in}{0.437222in}}{\pgfqpoint{6.275590in}{5.159444in}}%
\pgfusepath{clip}%
\pgfsetbuttcap%
\pgfsetroundjoin%
\pgfsetlinewidth{1.003750pt}%
\definecolor{currentstroke}{rgb}{0.827451,0.827451,0.827451}%
\pgfsetstrokecolor{currentstroke}%
\pgfsetstrokeopacity{0.800000}%
\pgfsetdash{}{0pt}%
\pgfpathmoveto{\pgfqpoint{0.870505in}{2.877763in}}%
\pgfpathcurveto{\pgfqpoint{0.881555in}{2.877763in}}{\pgfqpoint{0.892154in}{2.882154in}}{\pgfqpoint{0.899967in}{2.889967in}}%
\pgfpathcurveto{\pgfqpoint{0.907781in}{2.897781in}}{\pgfqpoint{0.912171in}{2.908380in}}{\pgfqpoint{0.912171in}{2.919430in}}%
\pgfpathcurveto{\pgfqpoint{0.912171in}{2.930480in}}{\pgfqpoint{0.907781in}{2.941079in}}{\pgfqpoint{0.899967in}{2.948893in}}%
\pgfpathcurveto{\pgfqpoint{0.892154in}{2.956706in}}{\pgfqpoint{0.881555in}{2.961097in}}{\pgfqpoint{0.870505in}{2.961097in}}%
\pgfpathcurveto{\pgfqpoint{0.859455in}{2.961097in}}{\pgfqpoint{0.848856in}{2.956706in}}{\pgfqpoint{0.841042in}{2.948893in}}%
\pgfpathcurveto{\pgfqpoint{0.833228in}{2.941079in}}{\pgfqpoint{0.828838in}{2.930480in}}{\pgfqpoint{0.828838in}{2.919430in}}%
\pgfpathcurveto{\pgfqpoint{0.828838in}{2.908380in}}{\pgfqpoint{0.833228in}{2.897781in}}{\pgfqpoint{0.841042in}{2.889967in}}%
\pgfpathcurveto{\pgfqpoint{0.848856in}{2.882154in}}{\pgfqpoint{0.859455in}{2.877763in}}{\pgfqpoint{0.870505in}{2.877763in}}%
\pgfpathlineto{\pgfqpoint{0.870505in}{2.877763in}}%
\pgfpathclose%
\pgfusepath{stroke}%
\end{pgfscope}%
\begin{pgfscope}%
\pgfpathrectangle{\pgfqpoint{0.494722in}{0.437222in}}{\pgfqpoint{6.275590in}{5.159444in}}%
\pgfusepath{clip}%
\pgfsetbuttcap%
\pgfsetroundjoin%
\pgfsetlinewidth{1.003750pt}%
\definecolor{currentstroke}{rgb}{0.827451,0.827451,0.827451}%
\pgfsetstrokecolor{currentstroke}%
\pgfsetstrokeopacity{0.800000}%
\pgfsetdash{}{0pt}%
\pgfpathmoveto{\pgfqpoint{5.070546in}{0.420375in}}%
\pgfpathcurveto{\pgfqpoint{5.081596in}{0.420375in}}{\pgfqpoint{5.092195in}{0.424765in}}{\pgfqpoint{5.100009in}{0.432578in}}%
\pgfpathcurveto{\pgfqpoint{5.107822in}{0.440392in}}{\pgfqpoint{5.112213in}{0.450991in}}{\pgfqpoint{5.112213in}{0.462041in}}%
\pgfpathcurveto{\pgfqpoint{5.112213in}{0.473091in}}{\pgfqpoint{5.107822in}{0.483690in}}{\pgfqpoint{5.100009in}{0.491504in}}%
\pgfpathcurveto{\pgfqpoint{5.092195in}{0.499318in}}{\pgfqpoint{5.081596in}{0.503708in}}{\pgfqpoint{5.070546in}{0.503708in}}%
\pgfpathcurveto{\pgfqpoint{5.059496in}{0.503708in}}{\pgfqpoint{5.048897in}{0.499318in}}{\pgfqpoint{5.041083in}{0.491504in}}%
\pgfpathcurveto{\pgfqpoint{5.033270in}{0.483690in}}{\pgfqpoint{5.028879in}{0.473091in}}{\pgfqpoint{5.028879in}{0.462041in}}%
\pgfpathcurveto{\pgfqpoint{5.028879in}{0.450991in}}{\pgfqpoint{5.033270in}{0.440392in}}{\pgfqpoint{5.041083in}{0.432578in}}%
\pgfpathcurveto{\pgfqpoint{5.048897in}{0.424765in}}{\pgfqpoint{5.059496in}{0.420375in}}{\pgfqpoint{5.070546in}{0.420375in}}%
\pgfusepath{stroke}%
\end{pgfscope}%
\begin{pgfscope}%
\pgfpathrectangle{\pgfqpoint{0.494722in}{0.437222in}}{\pgfqpoint{6.275590in}{5.159444in}}%
\pgfusepath{clip}%
\pgfsetbuttcap%
\pgfsetroundjoin%
\pgfsetlinewidth{1.003750pt}%
\definecolor{currentstroke}{rgb}{0.827451,0.827451,0.827451}%
\pgfsetstrokecolor{currentstroke}%
\pgfsetstrokeopacity{0.800000}%
\pgfsetdash{}{0pt}%
\pgfpathmoveto{\pgfqpoint{1.926347in}{1.574489in}}%
\pgfpathcurveto{\pgfqpoint{1.937397in}{1.574489in}}{\pgfqpoint{1.947996in}{1.578879in}}{\pgfqpoint{1.955809in}{1.586693in}}%
\pgfpathcurveto{\pgfqpoint{1.963623in}{1.594506in}}{\pgfqpoint{1.968013in}{1.605105in}}{\pgfqpoint{1.968013in}{1.616155in}}%
\pgfpathcurveto{\pgfqpoint{1.968013in}{1.627206in}}{\pgfqpoint{1.963623in}{1.637805in}}{\pgfqpoint{1.955809in}{1.645618in}}%
\pgfpathcurveto{\pgfqpoint{1.947996in}{1.653432in}}{\pgfqpoint{1.937397in}{1.657822in}}{\pgfqpoint{1.926347in}{1.657822in}}%
\pgfpathcurveto{\pgfqpoint{1.915296in}{1.657822in}}{\pgfqpoint{1.904697in}{1.653432in}}{\pgfqpoint{1.896884in}{1.645618in}}%
\pgfpathcurveto{\pgfqpoint{1.889070in}{1.637805in}}{\pgfqpoint{1.884680in}{1.627206in}}{\pgfqpoint{1.884680in}{1.616155in}}%
\pgfpathcurveto{\pgfqpoint{1.884680in}{1.605105in}}{\pgfqpoint{1.889070in}{1.594506in}}{\pgfqpoint{1.896884in}{1.586693in}}%
\pgfpathcurveto{\pgfqpoint{1.904697in}{1.578879in}}{\pgfqpoint{1.915296in}{1.574489in}}{\pgfqpoint{1.926347in}{1.574489in}}%
\pgfpathlineto{\pgfqpoint{1.926347in}{1.574489in}}%
\pgfpathclose%
\pgfusepath{stroke}%
\end{pgfscope}%
\begin{pgfscope}%
\pgfpathrectangle{\pgfqpoint{0.494722in}{0.437222in}}{\pgfqpoint{6.275590in}{5.159444in}}%
\pgfusepath{clip}%
\pgfsetbuttcap%
\pgfsetroundjoin%
\pgfsetlinewidth{1.003750pt}%
\definecolor{currentstroke}{rgb}{0.827451,0.827451,0.827451}%
\pgfsetstrokecolor{currentstroke}%
\pgfsetstrokeopacity{0.800000}%
\pgfsetdash{}{0pt}%
\pgfpathmoveto{\pgfqpoint{5.373868in}{0.408915in}}%
\pgfpathcurveto{\pgfqpoint{5.384918in}{0.408915in}}{\pgfqpoint{5.395517in}{0.413306in}}{\pgfqpoint{5.403330in}{0.421119in}}%
\pgfpathcurveto{\pgfqpoint{5.411144in}{0.428933in}}{\pgfqpoint{5.415534in}{0.439532in}}{\pgfqpoint{5.415534in}{0.450582in}}%
\pgfpathcurveto{\pgfqpoint{5.415534in}{0.461632in}}{\pgfqpoint{5.411144in}{0.472231in}}{\pgfqpoint{5.403330in}{0.480045in}}%
\pgfpathcurveto{\pgfqpoint{5.395517in}{0.487858in}}{\pgfqpoint{5.384918in}{0.492249in}}{\pgfqpoint{5.373868in}{0.492249in}}%
\pgfpathcurveto{\pgfqpoint{5.362817in}{0.492249in}}{\pgfqpoint{5.352218in}{0.487858in}}{\pgfqpoint{5.344405in}{0.480045in}}%
\pgfpathcurveto{\pgfqpoint{5.336591in}{0.472231in}}{\pgfqpoint{5.332201in}{0.461632in}}{\pgfqpoint{5.332201in}{0.450582in}}%
\pgfpathcurveto{\pgfqpoint{5.332201in}{0.439532in}}{\pgfqpoint{5.336591in}{0.428933in}}{\pgfqpoint{5.344405in}{0.421119in}}%
\pgfpathcurveto{\pgfqpoint{5.352218in}{0.413306in}}{\pgfqpoint{5.362817in}{0.408915in}}{\pgfqpoint{5.373868in}{0.408915in}}%
\pgfusepath{stroke}%
\end{pgfscope}%
\begin{pgfscope}%
\pgfpathrectangle{\pgfqpoint{0.494722in}{0.437222in}}{\pgfqpoint{6.275590in}{5.159444in}}%
\pgfusepath{clip}%
\pgfsetbuttcap%
\pgfsetroundjoin%
\pgfsetlinewidth{1.003750pt}%
\definecolor{currentstroke}{rgb}{0.827451,0.827451,0.827451}%
\pgfsetstrokecolor{currentstroke}%
\pgfsetstrokeopacity{0.800000}%
\pgfsetdash{}{0pt}%
\pgfpathmoveto{\pgfqpoint{4.686306in}{0.459933in}}%
\pgfpathcurveto{\pgfqpoint{4.697356in}{0.459933in}}{\pgfqpoint{4.707955in}{0.464323in}}{\pgfqpoint{4.715768in}{0.472137in}}%
\pgfpathcurveto{\pgfqpoint{4.723582in}{0.479950in}}{\pgfqpoint{4.727972in}{0.490549in}}{\pgfqpoint{4.727972in}{0.501599in}}%
\pgfpathcurveto{\pgfqpoint{4.727972in}{0.512649in}}{\pgfqpoint{4.723582in}{0.523248in}}{\pgfqpoint{4.715768in}{0.531062in}}%
\pgfpathcurveto{\pgfqpoint{4.707955in}{0.538876in}}{\pgfqpoint{4.697356in}{0.543266in}}{\pgfqpoint{4.686306in}{0.543266in}}%
\pgfpathcurveto{\pgfqpoint{4.675256in}{0.543266in}}{\pgfqpoint{4.664656in}{0.538876in}}{\pgfqpoint{4.656843in}{0.531062in}}%
\pgfpathcurveto{\pgfqpoint{4.649029in}{0.523248in}}{\pgfqpoint{4.644639in}{0.512649in}}{\pgfqpoint{4.644639in}{0.501599in}}%
\pgfpathcurveto{\pgfqpoint{4.644639in}{0.490549in}}{\pgfqpoint{4.649029in}{0.479950in}}{\pgfqpoint{4.656843in}{0.472137in}}%
\pgfpathcurveto{\pgfqpoint{4.664656in}{0.464323in}}{\pgfqpoint{4.675256in}{0.459933in}}{\pgfqpoint{4.686306in}{0.459933in}}%
\pgfpathlineto{\pgfqpoint{4.686306in}{0.459933in}}%
\pgfpathclose%
\pgfusepath{stroke}%
\end{pgfscope}%
\begin{pgfscope}%
\pgfpathrectangle{\pgfqpoint{0.494722in}{0.437222in}}{\pgfqpoint{6.275590in}{5.159444in}}%
\pgfusepath{clip}%
\pgfsetbuttcap%
\pgfsetroundjoin%
\pgfsetlinewidth{1.003750pt}%
\definecolor{currentstroke}{rgb}{0.827451,0.827451,0.827451}%
\pgfsetstrokecolor{currentstroke}%
\pgfsetstrokeopacity{0.800000}%
\pgfsetdash{}{0pt}%
\pgfpathmoveto{\pgfqpoint{0.944388in}{2.750541in}}%
\pgfpathcurveto{\pgfqpoint{0.955438in}{2.750541in}}{\pgfqpoint{0.966037in}{2.754931in}}{\pgfqpoint{0.973850in}{2.762745in}}%
\pgfpathcurveto{\pgfqpoint{0.981664in}{2.770558in}}{\pgfqpoint{0.986054in}{2.781157in}}{\pgfqpoint{0.986054in}{2.792208in}}%
\pgfpathcurveto{\pgfqpoint{0.986054in}{2.803258in}}{\pgfqpoint{0.981664in}{2.813857in}}{\pgfqpoint{0.973850in}{2.821670in}}%
\pgfpathcurveto{\pgfqpoint{0.966037in}{2.829484in}}{\pgfqpoint{0.955438in}{2.833874in}}{\pgfqpoint{0.944388in}{2.833874in}}%
\pgfpathcurveto{\pgfqpoint{0.933337in}{2.833874in}}{\pgfqpoint{0.922738in}{2.829484in}}{\pgfqpoint{0.914925in}{2.821670in}}%
\pgfpathcurveto{\pgfqpoint{0.907111in}{2.813857in}}{\pgfqpoint{0.902721in}{2.803258in}}{\pgfqpoint{0.902721in}{2.792208in}}%
\pgfpathcurveto{\pgfqpoint{0.902721in}{2.781157in}}{\pgfqpoint{0.907111in}{2.770558in}}{\pgfqpoint{0.914925in}{2.762745in}}%
\pgfpathcurveto{\pgfqpoint{0.922738in}{2.754931in}}{\pgfqpoint{0.933337in}{2.750541in}}{\pgfqpoint{0.944388in}{2.750541in}}%
\pgfpathlineto{\pgfqpoint{0.944388in}{2.750541in}}%
\pgfpathclose%
\pgfusepath{stroke}%
\end{pgfscope}%
\begin{pgfscope}%
\pgfpathrectangle{\pgfqpoint{0.494722in}{0.437222in}}{\pgfqpoint{6.275590in}{5.159444in}}%
\pgfusepath{clip}%
\pgfsetbuttcap%
\pgfsetroundjoin%
\pgfsetlinewidth{1.003750pt}%
\definecolor{currentstroke}{rgb}{0.827451,0.827451,0.827451}%
\pgfsetstrokecolor{currentstroke}%
\pgfsetstrokeopacity{0.800000}%
\pgfsetdash{}{0pt}%
\pgfpathmoveto{\pgfqpoint{3.460812in}{0.723768in}}%
\pgfpathcurveto{\pgfqpoint{3.471862in}{0.723768in}}{\pgfqpoint{3.482461in}{0.728158in}}{\pgfqpoint{3.490274in}{0.735972in}}%
\pgfpathcurveto{\pgfqpoint{3.498088in}{0.743785in}}{\pgfqpoint{3.502478in}{0.754384in}}{\pgfqpoint{3.502478in}{0.765434in}}%
\pgfpathcurveto{\pgfqpoint{3.502478in}{0.776484in}}{\pgfqpoint{3.498088in}{0.787083in}}{\pgfqpoint{3.490274in}{0.794897in}}%
\pgfpathcurveto{\pgfqpoint{3.482461in}{0.802711in}}{\pgfqpoint{3.471862in}{0.807101in}}{\pgfqpoint{3.460812in}{0.807101in}}%
\pgfpathcurveto{\pgfqpoint{3.449762in}{0.807101in}}{\pgfqpoint{3.439162in}{0.802711in}}{\pgfqpoint{3.431349in}{0.794897in}}%
\pgfpathcurveto{\pgfqpoint{3.423535in}{0.787083in}}{\pgfqpoint{3.419145in}{0.776484in}}{\pgfqpoint{3.419145in}{0.765434in}}%
\pgfpathcurveto{\pgfqpoint{3.419145in}{0.754384in}}{\pgfqpoint{3.423535in}{0.743785in}}{\pgfqpoint{3.431349in}{0.735972in}}%
\pgfpathcurveto{\pgfqpoint{3.439162in}{0.728158in}}{\pgfqpoint{3.449762in}{0.723768in}}{\pgfqpoint{3.460812in}{0.723768in}}%
\pgfpathlineto{\pgfqpoint{3.460812in}{0.723768in}}%
\pgfpathclose%
\pgfusepath{stroke}%
\end{pgfscope}%
\begin{pgfscope}%
\pgfpathrectangle{\pgfqpoint{0.494722in}{0.437222in}}{\pgfqpoint{6.275590in}{5.159444in}}%
\pgfusepath{clip}%
\pgfsetbuttcap%
\pgfsetroundjoin%
\pgfsetlinewidth{1.003750pt}%
\definecolor{currentstroke}{rgb}{0.827451,0.827451,0.827451}%
\pgfsetstrokecolor{currentstroke}%
\pgfsetstrokeopacity{0.800000}%
\pgfsetdash{}{0pt}%
\pgfpathmoveto{\pgfqpoint{1.444344in}{1.986059in}}%
\pgfpathcurveto{\pgfqpoint{1.455394in}{1.986059in}}{\pgfqpoint{1.465993in}{1.990450in}}{\pgfqpoint{1.473807in}{1.998263in}}%
\pgfpathcurveto{\pgfqpoint{1.481620in}{2.006077in}}{\pgfqpoint{1.486010in}{2.016676in}}{\pgfqpoint{1.486010in}{2.027726in}}%
\pgfpathcurveto{\pgfqpoint{1.486010in}{2.038776in}}{\pgfqpoint{1.481620in}{2.049375in}}{\pgfqpoint{1.473807in}{2.057189in}}%
\pgfpathcurveto{\pgfqpoint{1.465993in}{2.065002in}}{\pgfqpoint{1.455394in}{2.069393in}}{\pgfqpoint{1.444344in}{2.069393in}}%
\pgfpathcurveto{\pgfqpoint{1.433294in}{2.069393in}}{\pgfqpoint{1.422695in}{2.065002in}}{\pgfqpoint{1.414881in}{2.057189in}}%
\pgfpathcurveto{\pgfqpoint{1.407067in}{2.049375in}}{\pgfqpoint{1.402677in}{2.038776in}}{\pgfqpoint{1.402677in}{2.027726in}}%
\pgfpathcurveto{\pgfqpoint{1.402677in}{2.016676in}}{\pgfqpoint{1.407067in}{2.006077in}}{\pgfqpoint{1.414881in}{1.998263in}}%
\pgfpathcurveto{\pgfqpoint{1.422695in}{1.990450in}}{\pgfqpoint{1.433294in}{1.986059in}}{\pgfqpoint{1.444344in}{1.986059in}}%
\pgfpathlineto{\pgfqpoint{1.444344in}{1.986059in}}%
\pgfpathclose%
\pgfusepath{stroke}%
\end{pgfscope}%
\begin{pgfscope}%
\pgfpathrectangle{\pgfqpoint{0.494722in}{0.437222in}}{\pgfqpoint{6.275590in}{5.159444in}}%
\pgfusepath{clip}%
\pgfsetbuttcap%
\pgfsetroundjoin%
\pgfsetlinewidth{1.003750pt}%
\definecolor{currentstroke}{rgb}{0.827451,0.827451,0.827451}%
\pgfsetstrokecolor{currentstroke}%
\pgfsetstrokeopacity{0.800000}%
\pgfsetdash{}{0pt}%
\pgfpathmoveto{\pgfqpoint{0.523223in}{4.112992in}}%
\pgfpathcurveto{\pgfqpoint{0.534273in}{4.112992in}}{\pgfqpoint{0.544872in}{4.117382in}}{\pgfqpoint{0.552686in}{4.125196in}}%
\pgfpathcurveto{\pgfqpoint{0.560499in}{4.133010in}}{\pgfqpoint{0.564889in}{4.143609in}}{\pgfqpoint{0.564889in}{4.154659in}}%
\pgfpathcurveto{\pgfqpoint{0.564889in}{4.165709in}}{\pgfqpoint{0.560499in}{4.176308in}}{\pgfqpoint{0.552686in}{4.184121in}}%
\pgfpathcurveto{\pgfqpoint{0.544872in}{4.191935in}}{\pgfqpoint{0.534273in}{4.196325in}}{\pgfqpoint{0.523223in}{4.196325in}}%
\pgfpathcurveto{\pgfqpoint{0.512173in}{4.196325in}}{\pgfqpoint{0.501574in}{4.191935in}}{\pgfqpoint{0.493760in}{4.184121in}}%
\pgfpathcurveto{\pgfqpoint{0.485946in}{4.176308in}}{\pgfqpoint{0.481556in}{4.165709in}}{\pgfqpoint{0.481556in}{4.154659in}}%
\pgfpathcurveto{\pgfqpoint{0.481556in}{4.143609in}}{\pgfqpoint{0.485946in}{4.133010in}}{\pgfqpoint{0.493760in}{4.125196in}}%
\pgfpathcurveto{\pgfqpoint{0.501574in}{4.117382in}}{\pgfqpoint{0.512173in}{4.112992in}}{\pgfqpoint{0.523223in}{4.112992in}}%
\pgfpathlineto{\pgfqpoint{0.523223in}{4.112992in}}%
\pgfpathclose%
\pgfusepath{stroke}%
\end{pgfscope}%
\begin{pgfscope}%
\pgfpathrectangle{\pgfqpoint{0.494722in}{0.437222in}}{\pgfqpoint{6.275590in}{5.159444in}}%
\pgfusepath{clip}%
\pgfsetbuttcap%
\pgfsetroundjoin%
\pgfsetlinewidth{1.003750pt}%
\definecolor{currentstroke}{rgb}{0.827451,0.827451,0.827451}%
\pgfsetstrokecolor{currentstroke}%
\pgfsetstrokeopacity{0.800000}%
\pgfsetdash{}{0pt}%
\pgfpathmoveto{\pgfqpoint{1.133504in}{2.415631in}}%
\pgfpathcurveto{\pgfqpoint{1.144554in}{2.415631in}}{\pgfqpoint{1.155153in}{2.420021in}}{\pgfqpoint{1.162967in}{2.427835in}}%
\pgfpathcurveto{\pgfqpoint{1.170780in}{2.435648in}}{\pgfqpoint{1.175170in}{2.446247in}}{\pgfqpoint{1.175170in}{2.457297in}}%
\pgfpathcurveto{\pgfqpoint{1.175170in}{2.468347in}}{\pgfqpoint{1.170780in}{2.478946in}}{\pgfqpoint{1.162967in}{2.486760in}}%
\pgfpathcurveto{\pgfqpoint{1.155153in}{2.494574in}}{\pgfqpoint{1.144554in}{2.498964in}}{\pgfqpoint{1.133504in}{2.498964in}}%
\pgfpathcurveto{\pgfqpoint{1.122454in}{2.498964in}}{\pgfqpoint{1.111855in}{2.494574in}}{\pgfqpoint{1.104041in}{2.486760in}}%
\pgfpathcurveto{\pgfqpoint{1.096227in}{2.478946in}}{\pgfqpoint{1.091837in}{2.468347in}}{\pgfqpoint{1.091837in}{2.457297in}}%
\pgfpathcurveto{\pgfqpoint{1.091837in}{2.446247in}}{\pgfqpoint{1.096227in}{2.435648in}}{\pgfqpoint{1.104041in}{2.427835in}}%
\pgfpathcurveto{\pgfqpoint{1.111855in}{2.420021in}}{\pgfqpoint{1.122454in}{2.415631in}}{\pgfqpoint{1.133504in}{2.415631in}}%
\pgfpathlineto{\pgfqpoint{1.133504in}{2.415631in}}%
\pgfpathclose%
\pgfusepath{stroke}%
\end{pgfscope}%
\begin{pgfscope}%
\pgfpathrectangle{\pgfqpoint{0.494722in}{0.437222in}}{\pgfqpoint{6.275590in}{5.159444in}}%
\pgfusepath{clip}%
\pgfsetbuttcap%
\pgfsetroundjoin%
\pgfsetlinewidth{1.003750pt}%
\definecolor{currentstroke}{rgb}{0.827451,0.827451,0.827451}%
\pgfsetstrokecolor{currentstroke}%
\pgfsetstrokeopacity{0.800000}%
\pgfsetdash{}{0pt}%
\pgfpathmoveto{\pgfqpoint{5.618006in}{0.398631in}}%
\pgfpathcurveto{\pgfqpoint{5.629056in}{0.398631in}}{\pgfqpoint{5.639655in}{0.403021in}}{\pgfqpoint{5.647468in}{0.410835in}}%
\pgfpathcurveto{\pgfqpoint{5.655282in}{0.418649in}}{\pgfqpoint{5.659672in}{0.429248in}}{\pgfqpoint{5.659672in}{0.440298in}}%
\pgfpathcurveto{\pgfqpoint{5.659672in}{0.451348in}}{\pgfqpoint{5.655282in}{0.461947in}}{\pgfqpoint{5.647468in}{0.469761in}}%
\pgfpathcurveto{\pgfqpoint{5.639655in}{0.477574in}}{\pgfqpoint{5.629056in}{0.481964in}}{\pgfqpoint{5.618006in}{0.481964in}}%
\pgfpathcurveto{\pgfqpoint{5.606956in}{0.481964in}}{\pgfqpoint{5.596357in}{0.477574in}}{\pgfqpoint{5.588543in}{0.469761in}}%
\pgfpathcurveto{\pgfqpoint{5.580729in}{0.461947in}}{\pgfqpoint{5.576339in}{0.451348in}}{\pgfqpoint{5.576339in}{0.440298in}}%
\pgfpathcurveto{\pgfqpoint{5.576339in}{0.429248in}}{\pgfqpoint{5.580729in}{0.418649in}}{\pgfqpoint{5.588543in}{0.410835in}}%
\pgfpathcurveto{\pgfqpoint{5.596357in}{0.403021in}}{\pgfqpoint{5.606956in}{0.398631in}}{\pgfqpoint{5.618006in}{0.398631in}}%
\pgfusepath{stroke}%
\end{pgfscope}%
\begin{pgfscope}%
\pgfpathrectangle{\pgfqpoint{0.494722in}{0.437222in}}{\pgfqpoint{6.275590in}{5.159444in}}%
\pgfusepath{clip}%
\pgfsetbuttcap%
\pgfsetroundjoin%
\pgfsetlinewidth{1.003750pt}%
\definecolor{currentstroke}{rgb}{0.827451,0.827451,0.827451}%
\pgfsetstrokecolor{currentstroke}%
\pgfsetstrokeopacity{0.800000}%
\pgfsetdash{}{0pt}%
\pgfpathmoveto{\pgfqpoint{4.270074in}{0.511112in}}%
\pgfpathcurveto{\pgfqpoint{4.281124in}{0.511112in}}{\pgfqpoint{4.291723in}{0.515502in}}{\pgfqpoint{4.299537in}{0.523316in}}%
\pgfpathcurveto{\pgfqpoint{4.307351in}{0.531129in}}{\pgfqpoint{4.311741in}{0.541728in}}{\pgfqpoint{4.311741in}{0.552778in}}%
\pgfpathcurveto{\pgfqpoint{4.311741in}{0.563829in}}{\pgfqpoint{4.307351in}{0.574428in}}{\pgfqpoint{4.299537in}{0.582241in}}%
\pgfpathcurveto{\pgfqpoint{4.291723in}{0.590055in}}{\pgfqpoint{4.281124in}{0.594445in}}{\pgfqpoint{4.270074in}{0.594445in}}%
\pgfpathcurveto{\pgfqpoint{4.259024in}{0.594445in}}{\pgfqpoint{4.248425in}{0.590055in}}{\pgfqpoint{4.240611in}{0.582241in}}%
\pgfpathcurveto{\pgfqpoint{4.232798in}{0.574428in}}{\pgfqpoint{4.228407in}{0.563829in}}{\pgfqpoint{4.228407in}{0.552778in}}%
\pgfpathcurveto{\pgfqpoint{4.228407in}{0.541728in}}{\pgfqpoint{4.232798in}{0.531129in}}{\pgfqpoint{4.240611in}{0.523316in}}%
\pgfpathcurveto{\pgfqpoint{4.248425in}{0.515502in}}{\pgfqpoint{4.259024in}{0.511112in}}{\pgfqpoint{4.270074in}{0.511112in}}%
\pgfpathlineto{\pgfqpoint{4.270074in}{0.511112in}}%
\pgfpathclose%
\pgfusepath{stroke}%
\end{pgfscope}%
\begin{pgfscope}%
\pgfpathrectangle{\pgfqpoint{0.494722in}{0.437222in}}{\pgfqpoint{6.275590in}{5.159444in}}%
\pgfusepath{clip}%
\pgfsetbuttcap%
\pgfsetroundjoin%
\pgfsetlinewidth{1.003750pt}%
\definecolor{currentstroke}{rgb}{0.827451,0.827451,0.827451}%
\pgfsetstrokecolor{currentstroke}%
\pgfsetstrokeopacity{0.800000}%
\pgfsetdash{}{0pt}%
\pgfpathmoveto{\pgfqpoint{3.369630in}{0.746512in}}%
\pgfpathcurveto{\pgfqpoint{3.380680in}{0.746512in}}{\pgfqpoint{3.391279in}{0.750903in}}{\pgfqpoint{3.399093in}{0.758716in}}%
\pgfpathcurveto{\pgfqpoint{3.406907in}{0.766530in}}{\pgfqpoint{3.411297in}{0.777129in}}{\pgfqpoint{3.411297in}{0.788179in}}%
\pgfpathcurveto{\pgfqpoint{3.411297in}{0.799229in}}{\pgfqpoint{3.406907in}{0.809828in}}{\pgfqpoint{3.399093in}{0.817642in}}%
\pgfpathcurveto{\pgfqpoint{3.391279in}{0.825455in}}{\pgfqpoint{3.380680in}{0.829846in}}{\pgfqpoint{3.369630in}{0.829846in}}%
\pgfpathcurveto{\pgfqpoint{3.358580in}{0.829846in}}{\pgfqpoint{3.347981in}{0.825455in}}{\pgfqpoint{3.340167in}{0.817642in}}%
\pgfpathcurveto{\pgfqpoint{3.332354in}{0.809828in}}{\pgfqpoint{3.327963in}{0.799229in}}{\pgfqpoint{3.327963in}{0.788179in}}%
\pgfpathcurveto{\pgfqpoint{3.327963in}{0.777129in}}{\pgfqpoint{3.332354in}{0.766530in}}{\pgfqpoint{3.340167in}{0.758716in}}%
\pgfpathcurveto{\pgfqpoint{3.347981in}{0.750903in}}{\pgfqpoint{3.358580in}{0.746512in}}{\pgfqpoint{3.369630in}{0.746512in}}%
\pgfpathlineto{\pgfqpoint{3.369630in}{0.746512in}}%
\pgfpathclose%
\pgfusepath{stroke}%
\end{pgfscope}%
\begin{pgfscope}%
\pgfpathrectangle{\pgfqpoint{0.494722in}{0.437222in}}{\pgfqpoint{6.275590in}{5.159444in}}%
\pgfusepath{clip}%
\pgfsetbuttcap%
\pgfsetroundjoin%
\pgfsetlinewidth{1.003750pt}%
\definecolor{currentstroke}{rgb}{0.827451,0.827451,0.827451}%
\pgfsetstrokecolor{currentstroke}%
\pgfsetstrokeopacity{0.800000}%
\pgfsetdash{}{0pt}%
\pgfpathmoveto{\pgfqpoint{0.611323in}{3.579035in}}%
\pgfpathcurveto{\pgfqpoint{0.622373in}{3.579035in}}{\pgfqpoint{0.632972in}{3.583425in}}{\pgfqpoint{0.640786in}{3.591238in}}%
\pgfpathcurveto{\pgfqpoint{0.648599in}{3.599052in}}{\pgfqpoint{0.652989in}{3.609651in}}{\pgfqpoint{0.652989in}{3.620701in}}%
\pgfpathcurveto{\pgfqpoint{0.652989in}{3.631751in}}{\pgfqpoint{0.648599in}{3.642350in}}{\pgfqpoint{0.640786in}{3.650164in}}%
\pgfpathcurveto{\pgfqpoint{0.632972in}{3.657978in}}{\pgfqpoint{0.622373in}{3.662368in}}{\pgfqpoint{0.611323in}{3.662368in}}%
\pgfpathcurveto{\pgfqpoint{0.600273in}{3.662368in}}{\pgfqpoint{0.589674in}{3.657978in}}{\pgfqpoint{0.581860in}{3.650164in}}%
\pgfpathcurveto{\pgfqpoint{0.574046in}{3.642350in}}{\pgfqpoint{0.569656in}{3.631751in}}{\pgfqpoint{0.569656in}{3.620701in}}%
\pgfpathcurveto{\pgfqpoint{0.569656in}{3.609651in}}{\pgfqpoint{0.574046in}{3.599052in}}{\pgfqpoint{0.581860in}{3.591238in}}%
\pgfpathcurveto{\pgfqpoint{0.589674in}{3.583425in}}{\pgfqpoint{0.600273in}{3.579035in}}{\pgfqpoint{0.611323in}{3.579035in}}%
\pgfpathlineto{\pgfqpoint{0.611323in}{3.579035in}}%
\pgfpathclose%
\pgfusepath{stroke}%
\end{pgfscope}%
\begin{pgfscope}%
\pgfpathrectangle{\pgfqpoint{0.494722in}{0.437222in}}{\pgfqpoint{6.275590in}{5.159444in}}%
\pgfusepath{clip}%
\pgfsetbuttcap%
\pgfsetroundjoin%
\pgfsetlinewidth{1.003750pt}%
\definecolor{currentstroke}{rgb}{0.827451,0.827451,0.827451}%
\pgfsetstrokecolor{currentstroke}%
\pgfsetstrokeopacity{0.800000}%
\pgfsetdash{}{0pt}%
\pgfpathmoveto{\pgfqpoint{4.774246in}{0.444010in}}%
\pgfpathcurveto{\pgfqpoint{4.785296in}{0.444010in}}{\pgfqpoint{4.795895in}{0.448400in}}{\pgfqpoint{4.803708in}{0.456214in}}%
\pgfpathcurveto{\pgfqpoint{4.811522in}{0.464028in}}{\pgfqpoint{4.815912in}{0.474627in}}{\pgfqpoint{4.815912in}{0.485677in}}%
\pgfpathcurveto{\pgfqpoint{4.815912in}{0.496727in}}{\pgfqpoint{4.811522in}{0.507326in}}{\pgfqpoint{4.803708in}{0.515140in}}%
\pgfpathcurveto{\pgfqpoint{4.795895in}{0.522953in}}{\pgfqpoint{4.785296in}{0.527343in}}{\pgfqpoint{4.774246in}{0.527343in}}%
\pgfpathcurveto{\pgfqpoint{4.763195in}{0.527343in}}{\pgfqpoint{4.752596in}{0.522953in}}{\pgfqpoint{4.744783in}{0.515140in}}%
\pgfpathcurveto{\pgfqpoint{4.736969in}{0.507326in}}{\pgfqpoint{4.732579in}{0.496727in}}{\pgfqpoint{4.732579in}{0.485677in}}%
\pgfpathcurveto{\pgfqpoint{4.732579in}{0.474627in}}{\pgfqpoint{4.736969in}{0.464028in}}{\pgfqpoint{4.744783in}{0.456214in}}%
\pgfpathcurveto{\pgfqpoint{4.752596in}{0.448400in}}{\pgfqpoint{4.763195in}{0.444010in}}{\pgfqpoint{4.774246in}{0.444010in}}%
\pgfpathlineto{\pgfqpoint{4.774246in}{0.444010in}}%
\pgfpathclose%
\pgfusepath{stroke}%
\end{pgfscope}%
\begin{pgfscope}%
\pgfpathrectangle{\pgfqpoint{0.494722in}{0.437222in}}{\pgfqpoint{6.275590in}{5.159444in}}%
\pgfusepath{clip}%
\pgfsetbuttcap%
\pgfsetroundjoin%
\pgfsetlinewidth{1.003750pt}%
\definecolor{currentstroke}{rgb}{0.827451,0.827451,0.827451}%
\pgfsetstrokecolor{currentstroke}%
\pgfsetstrokeopacity{0.800000}%
\pgfsetdash{}{0pt}%
\pgfpathmoveto{\pgfqpoint{1.721741in}{1.714097in}}%
\pgfpathcurveto{\pgfqpoint{1.732791in}{1.714097in}}{\pgfqpoint{1.743390in}{1.718487in}}{\pgfqpoint{1.751204in}{1.726301in}}%
\pgfpathcurveto{\pgfqpoint{1.759018in}{1.734114in}}{\pgfqpoint{1.763408in}{1.744713in}}{\pgfqpoint{1.763408in}{1.755764in}}%
\pgfpathcurveto{\pgfqpoint{1.763408in}{1.766814in}}{\pgfqpoint{1.759018in}{1.777413in}}{\pgfqpoint{1.751204in}{1.785226in}}%
\pgfpathcurveto{\pgfqpoint{1.743390in}{1.793040in}}{\pgfqpoint{1.732791in}{1.797430in}}{\pgfqpoint{1.721741in}{1.797430in}}%
\pgfpathcurveto{\pgfqpoint{1.710691in}{1.797430in}}{\pgfqpoint{1.700092in}{1.793040in}}{\pgfqpoint{1.692279in}{1.785226in}}%
\pgfpathcurveto{\pgfqpoint{1.684465in}{1.777413in}}{\pgfqpoint{1.680075in}{1.766814in}}{\pgfqpoint{1.680075in}{1.755764in}}%
\pgfpathcurveto{\pgfqpoint{1.680075in}{1.744713in}}{\pgfqpoint{1.684465in}{1.734114in}}{\pgfqpoint{1.692279in}{1.726301in}}%
\pgfpathcurveto{\pgfqpoint{1.700092in}{1.718487in}}{\pgfqpoint{1.710691in}{1.714097in}}{\pgfqpoint{1.721741in}{1.714097in}}%
\pgfpathlineto{\pgfqpoint{1.721741in}{1.714097in}}%
\pgfpathclose%
\pgfusepath{stroke}%
\end{pgfscope}%
\begin{pgfscope}%
\pgfpathrectangle{\pgfqpoint{0.494722in}{0.437222in}}{\pgfqpoint{6.275590in}{5.159444in}}%
\pgfusepath{clip}%
\pgfsetbuttcap%
\pgfsetroundjoin%
\pgfsetlinewidth{1.003750pt}%
\definecolor{currentstroke}{rgb}{0.827451,0.827451,0.827451}%
\pgfsetstrokecolor{currentstroke}%
\pgfsetstrokeopacity{0.800000}%
\pgfsetdash{}{0pt}%
\pgfpathmoveto{\pgfqpoint{0.655608in}{3.415313in}}%
\pgfpathcurveto{\pgfqpoint{0.666658in}{3.415313in}}{\pgfqpoint{0.677257in}{3.419703in}}{\pgfqpoint{0.685071in}{3.427517in}}%
\pgfpathcurveto{\pgfqpoint{0.692885in}{3.435330in}}{\pgfqpoint{0.697275in}{3.445929in}}{\pgfqpoint{0.697275in}{3.456980in}}%
\pgfpathcurveto{\pgfqpoint{0.697275in}{3.468030in}}{\pgfqpoint{0.692885in}{3.478629in}}{\pgfqpoint{0.685071in}{3.486442in}}%
\pgfpathcurveto{\pgfqpoint{0.677257in}{3.494256in}}{\pgfqpoint{0.666658in}{3.498646in}}{\pgfqpoint{0.655608in}{3.498646in}}%
\pgfpathcurveto{\pgfqpoint{0.644558in}{3.498646in}}{\pgfqpoint{0.633959in}{3.494256in}}{\pgfqpoint{0.626145in}{3.486442in}}%
\pgfpathcurveto{\pgfqpoint{0.618332in}{3.478629in}}{\pgfqpoint{0.613941in}{3.468030in}}{\pgfqpoint{0.613941in}{3.456980in}}%
\pgfpathcurveto{\pgfqpoint{0.613941in}{3.445929in}}{\pgfqpoint{0.618332in}{3.435330in}}{\pgfqpoint{0.626145in}{3.427517in}}%
\pgfpathcurveto{\pgfqpoint{0.633959in}{3.419703in}}{\pgfqpoint{0.644558in}{3.415313in}}{\pgfqpoint{0.655608in}{3.415313in}}%
\pgfpathlineto{\pgfqpoint{0.655608in}{3.415313in}}%
\pgfpathclose%
\pgfusepath{stroke}%
\end{pgfscope}%
\begin{pgfscope}%
\pgfpathrectangle{\pgfqpoint{0.494722in}{0.437222in}}{\pgfqpoint{6.275590in}{5.159444in}}%
\pgfusepath{clip}%
\pgfsetbuttcap%
\pgfsetroundjoin%
\pgfsetlinewidth{1.003750pt}%
\definecolor{currentstroke}{rgb}{0.827451,0.827451,0.827451}%
\pgfsetstrokecolor{currentstroke}%
\pgfsetstrokeopacity{0.800000}%
\pgfsetdash{}{0pt}%
\pgfpathmoveto{\pgfqpoint{2.983913in}{0.907714in}}%
\pgfpathcurveto{\pgfqpoint{2.994963in}{0.907714in}}{\pgfqpoint{3.005562in}{0.912104in}}{\pgfqpoint{3.013375in}{0.919918in}}%
\pgfpathcurveto{\pgfqpoint{3.021189in}{0.927731in}}{\pgfqpoint{3.025579in}{0.938330in}}{\pgfqpoint{3.025579in}{0.949380in}}%
\pgfpathcurveto{\pgfqpoint{3.025579in}{0.960431in}}{\pgfqpoint{3.021189in}{0.971030in}}{\pgfqpoint{3.013375in}{0.978843in}}%
\pgfpathcurveto{\pgfqpoint{3.005562in}{0.986657in}}{\pgfqpoint{2.994963in}{0.991047in}}{\pgfqpoint{2.983913in}{0.991047in}}%
\pgfpathcurveto{\pgfqpoint{2.972862in}{0.991047in}}{\pgfqpoint{2.962263in}{0.986657in}}{\pgfqpoint{2.954450in}{0.978843in}}%
\pgfpathcurveto{\pgfqpoint{2.946636in}{0.971030in}}{\pgfqpoint{2.942246in}{0.960431in}}{\pgfqpoint{2.942246in}{0.949380in}}%
\pgfpathcurveto{\pgfqpoint{2.942246in}{0.938330in}}{\pgfqpoint{2.946636in}{0.927731in}}{\pgfqpoint{2.954450in}{0.919918in}}%
\pgfpathcurveto{\pgfqpoint{2.962263in}{0.912104in}}{\pgfqpoint{2.972862in}{0.907714in}}{\pgfqpoint{2.983913in}{0.907714in}}%
\pgfpathlineto{\pgfqpoint{2.983913in}{0.907714in}}%
\pgfpathclose%
\pgfusepath{stroke}%
\end{pgfscope}%
\begin{pgfscope}%
\pgfpathrectangle{\pgfqpoint{0.494722in}{0.437222in}}{\pgfqpoint{6.275590in}{5.159444in}}%
\pgfusepath{clip}%
\pgfsetbuttcap%
\pgfsetroundjoin%
\pgfsetlinewidth{1.003750pt}%
\definecolor{currentstroke}{rgb}{0.827451,0.827451,0.827451}%
\pgfsetstrokecolor{currentstroke}%
\pgfsetstrokeopacity{0.800000}%
\pgfsetdash{}{0pt}%
\pgfpathmoveto{\pgfqpoint{2.025354in}{1.472312in}}%
\pgfpathcurveto{\pgfqpoint{2.036404in}{1.472312in}}{\pgfqpoint{2.047003in}{1.476702in}}{\pgfqpoint{2.054816in}{1.484515in}}%
\pgfpathcurveto{\pgfqpoint{2.062630in}{1.492329in}}{\pgfqpoint{2.067020in}{1.502928in}}{\pgfqpoint{2.067020in}{1.513978in}}%
\pgfpathcurveto{\pgfqpoint{2.067020in}{1.525028in}}{\pgfqpoint{2.062630in}{1.535627in}}{\pgfqpoint{2.054816in}{1.543441in}}%
\pgfpathcurveto{\pgfqpoint{2.047003in}{1.551255in}}{\pgfqpoint{2.036404in}{1.555645in}}{\pgfqpoint{2.025354in}{1.555645in}}%
\pgfpathcurveto{\pgfqpoint{2.014304in}{1.555645in}}{\pgfqpoint{2.003704in}{1.551255in}}{\pgfqpoint{1.995891in}{1.543441in}}%
\pgfpathcurveto{\pgfqpoint{1.988077in}{1.535627in}}{\pgfqpoint{1.983687in}{1.525028in}}{\pgfqpoint{1.983687in}{1.513978in}}%
\pgfpathcurveto{\pgfqpoint{1.983687in}{1.502928in}}{\pgfqpoint{1.988077in}{1.492329in}}{\pgfqpoint{1.995891in}{1.484515in}}%
\pgfpathcurveto{\pgfqpoint{2.003704in}{1.476702in}}{\pgfqpoint{2.014304in}{1.472312in}}{\pgfqpoint{2.025354in}{1.472312in}}%
\pgfpathlineto{\pgfqpoint{2.025354in}{1.472312in}}%
\pgfpathclose%
\pgfusepath{stroke}%
\end{pgfscope}%
\begin{pgfscope}%
\pgfpathrectangle{\pgfqpoint{0.494722in}{0.437222in}}{\pgfqpoint{6.275590in}{5.159444in}}%
\pgfusepath{clip}%
\pgfsetbuttcap%
\pgfsetroundjoin%
\pgfsetlinewidth{1.003750pt}%
\definecolor{currentstroke}{rgb}{0.827451,0.827451,0.827451}%
\pgfsetstrokecolor{currentstroke}%
\pgfsetstrokeopacity{0.800000}%
\pgfsetdash{}{0pt}%
\pgfpathmoveto{\pgfqpoint{0.942220in}{2.753176in}}%
\pgfpathcurveto{\pgfqpoint{0.953270in}{2.753176in}}{\pgfqpoint{0.963869in}{2.757567in}}{\pgfqpoint{0.971683in}{2.765380in}}%
\pgfpathcurveto{\pgfqpoint{0.979497in}{2.773194in}}{\pgfqpoint{0.983887in}{2.783793in}}{\pgfqpoint{0.983887in}{2.794843in}}%
\pgfpathcurveto{\pgfqpoint{0.983887in}{2.805893in}}{\pgfqpoint{0.979497in}{2.816492in}}{\pgfqpoint{0.971683in}{2.824306in}}%
\pgfpathcurveto{\pgfqpoint{0.963869in}{2.832119in}}{\pgfqpoint{0.953270in}{2.836510in}}{\pgfqpoint{0.942220in}{2.836510in}}%
\pgfpathcurveto{\pgfqpoint{0.931170in}{2.836510in}}{\pgfqpoint{0.920571in}{2.832119in}}{\pgfqpoint{0.912757in}{2.824306in}}%
\pgfpathcurveto{\pgfqpoint{0.904944in}{2.816492in}}{\pgfqpoint{0.900554in}{2.805893in}}{\pgfqpoint{0.900554in}{2.794843in}}%
\pgfpathcurveto{\pgfqpoint{0.900554in}{2.783793in}}{\pgfqpoint{0.904944in}{2.773194in}}{\pgfqpoint{0.912757in}{2.765380in}}%
\pgfpathcurveto{\pgfqpoint{0.920571in}{2.757567in}}{\pgfqpoint{0.931170in}{2.753176in}}{\pgfqpoint{0.942220in}{2.753176in}}%
\pgfpathlineto{\pgfqpoint{0.942220in}{2.753176in}}%
\pgfpathclose%
\pgfusepath{stroke}%
\end{pgfscope}%
\begin{pgfscope}%
\pgfpathrectangle{\pgfqpoint{0.494722in}{0.437222in}}{\pgfqpoint{6.275590in}{5.159444in}}%
\pgfusepath{clip}%
\pgfsetbuttcap%
\pgfsetroundjoin%
\pgfsetlinewidth{1.003750pt}%
\definecolor{currentstroke}{rgb}{0.827451,0.827451,0.827451}%
\pgfsetstrokecolor{currentstroke}%
\pgfsetstrokeopacity{0.800000}%
\pgfsetdash{}{0pt}%
\pgfpathmoveto{\pgfqpoint{0.495484in}{4.544725in}}%
\pgfpathcurveto{\pgfqpoint{0.506534in}{4.544725in}}{\pgfqpoint{0.517133in}{4.549116in}}{\pgfqpoint{0.524947in}{4.556929in}}%
\pgfpathcurveto{\pgfqpoint{0.532761in}{4.564743in}}{\pgfqpoint{0.537151in}{4.575342in}}{\pgfqpoint{0.537151in}{4.586392in}}%
\pgfpathcurveto{\pgfqpoint{0.537151in}{4.597442in}}{\pgfqpoint{0.532761in}{4.608041in}}{\pgfqpoint{0.524947in}{4.615855in}}%
\pgfpathcurveto{\pgfqpoint{0.517133in}{4.623668in}}{\pgfqpoint{0.506534in}{4.628059in}}{\pgfqpoint{0.495484in}{4.628059in}}%
\pgfpathcurveto{\pgfqpoint{0.484434in}{4.628059in}}{\pgfqpoint{0.473835in}{4.623668in}}{\pgfqpoint{0.466021in}{4.615855in}}%
\pgfpathcurveto{\pgfqpoint{0.458208in}{4.608041in}}{\pgfqpoint{0.453817in}{4.597442in}}{\pgfqpoint{0.453817in}{4.586392in}}%
\pgfpathcurveto{\pgfqpoint{0.453817in}{4.575342in}}{\pgfqpoint{0.458208in}{4.564743in}}{\pgfqpoint{0.466021in}{4.556929in}}%
\pgfpathcurveto{\pgfqpoint{0.473835in}{4.549116in}}{\pgfqpoint{0.484434in}{4.544725in}}{\pgfqpoint{0.495484in}{4.544725in}}%
\pgfpathlineto{\pgfqpoint{0.495484in}{4.544725in}}%
\pgfpathclose%
\pgfusepath{stroke}%
\end{pgfscope}%
\begin{pgfscope}%
\pgfpathrectangle{\pgfqpoint{0.494722in}{0.437222in}}{\pgfqpoint{6.275590in}{5.159444in}}%
\pgfusepath{clip}%
\pgfsetbuttcap%
\pgfsetroundjoin%
\pgfsetlinewidth{1.003750pt}%
\definecolor{currentstroke}{rgb}{0.827451,0.827451,0.827451}%
\pgfsetstrokecolor{currentstroke}%
\pgfsetstrokeopacity{0.800000}%
\pgfsetdash{}{0pt}%
\pgfpathmoveto{\pgfqpoint{1.242947in}{2.239381in}}%
\pgfpathcurveto{\pgfqpoint{1.253997in}{2.239381in}}{\pgfqpoint{1.264596in}{2.243771in}}{\pgfqpoint{1.272410in}{2.251585in}}%
\pgfpathcurveto{\pgfqpoint{1.280224in}{2.259398in}}{\pgfqpoint{1.284614in}{2.269997in}}{\pgfqpoint{1.284614in}{2.281048in}}%
\pgfpathcurveto{\pgfqpoint{1.284614in}{2.292098in}}{\pgfqpoint{1.280224in}{2.302697in}}{\pgfqpoint{1.272410in}{2.310510in}}%
\pgfpathcurveto{\pgfqpoint{1.264596in}{2.318324in}}{\pgfqpoint{1.253997in}{2.322714in}}{\pgfqpoint{1.242947in}{2.322714in}}%
\pgfpathcurveto{\pgfqpoint{1.231897in}{2.322714in}}{\pgfqpoint{1.221298in}{2.318324in}}{\pgfqpoint{1.213484in}{2.310510in}}%
\pgfpathcurveto{\pgfqpoint{1.205671in}{2.302697in}}{\pgfqpoint{1.201280in}{2.292098in}}{\pgfqpoint{1.201280in}{2.281048in}}%
\pgfpathcurveto{\pgfqpoint{1.201280in}{2.269997in}}{\pgfqpoint{1.205671in}{2.259398in}}{\pgfqpoint{1.213484in}{2.251585in}}%
\pgfpathcurveto{\pgfqpoint{1.221298in}{2.243771in}}{\pgfqpoint{1.231897in}{2.239381in}}{\pgfqpoint{1.242947in}{2.239381in}}%
\pgfpathlineto{\pgfqpoint{1.242947in}{2.239381in}}%
\pgfpathclose%
\pgfusepath{stroke}%
\end{pgfscope}%
\begin{pgfscope}%
\pgfpathrectangle{\pgfqpoint{0.494722in}{0.437222in}}{\pgfqpoint{6.275590in}{5.159444in}}%
\pgfusepath{clip}%
\pgfsetbuttcap%
\pgfsetroundjoin%
\pgfsetlinewidth{1.003750pt}%
\definecolor{currentstroke}{rgb}{0.827451,0.827451,0.827451}%
\pgfsetstrokecolor{currentstroke}%
\pgfsetstrokeopacity{0.800000}%
\pgfsetdash{}{0pt}%
\pgfpathmoveto{\pgfqpoint{1.113110in}{2.511719in}}%
\pgfpathcurveto{\pgfqpoint{1.124161in}{2.511719in}}{\pgfqpoint{1.134760in}{2.516109in}}{\pgfqpoint{1.142573in}{2.523923in}}%
\pgfpathcurveto{\pgfqpoint{1.150387in}{2.531736in}}{\pgfqpoint{1.154777in}{2.542335in}}{\pgfqpoint{1.154777in}{2.553385in}}%
\pgfpathcurveto{\pgfqpoint{1.154777in}{2.564435in}}{\pgfqpoint{1.150387in}{2.575034in}}{\pgfqpoint{1.142573in}{2.582848in}}%
\pgfpathcurveto{\pgfqpoint{1.134760in}{2.590662in}}{\pgfqpoint{1.124161in}{2.595052in}}{\pgfqpoint{1.113110in}{2.595052in}}%
\pgfpathcurveto{\pgfqpoint{1.102060in}{2.595052in}}{\pgfqpoint{1.091461in}{2.590662in}}{\pgfqpoint{1.083648in}{2.582848in}}%
\pgfpathcurveto{\pgfqpoint{1.075834in}{2.575034in}}{\pgfqpoint{1.071444in}{2.564435in}}{\pgfqpoint{1.071444in}{2.553385in}}%
\pgfpathcurveto{\pgfqpoint{1.071444in}{2.542335in}}{\pgfqpoint{1.075834in}{2.531736in}}{\pgfqpoint{1.083648in}{2.523923in}}%
\pgfpathcurveto{\pgfqpoint{1.091461in}{2.516109in}}{\pgfqpoint{1.102060in}{2.511719in}}{\pgfqpoint{1.113110in}{2.511719in}}%
\pgfpathlineto{\pgfqpoint{1.113110in}{2.511719in}}%
\pgfpathclose%
\pgfusepath{stroke}%
\end{pgfscope}%
\begin{pgfscope}%
\pgfpathrectangle{\pgfqpoint{0.494722in}{0.437222in}}{\pgfqpoint{6.275590in}{5.159444in}}%
\pgfusepath{clip}%
\pgfsetbuttcap%
\pgfsetroundjoin%
\pgfsetlinewidth{1.003750pt}%
\definecolor{currentstroke}{rgb}{0.827451,0.827451,0.827451}%
\pgfsetstrokecolor{currentstroke}%
\pgfsetstrokeopacity{0.800000}%
\pgfsetdash{}{0pt}%
\pgfpathmoveto{\pgfqpoint{0.536273in}{3.987890in}}%
\pgfpathcurveto{\pgfqpoint{0.547324in}{3.987890in}}{\pgfqpoint{0.557923in}{3.992280in}}{\pgfqpoint{0.565736in}{4.000093in}}%
\pgfpathcurveto{\pgfqpoint{0.573550in}{4.007907in}}{\pgfqpoint{0.577940in}{4.018506in}}{\pgfqpoint{0.577940in}{4.029556in}}%
\pgfpathcurveto{\pgfqpoint{0.577940in}{4.040606in}}{\pgfqpoint{0.573550in}{4.051205in}}{\pgfqpoint{0.565736in}{4.059019in}}%
\pgfpathcurveto{\pgfqpoint{0.557923in}{4.066833in}}{\pgfqpoint{0.547324in}{4.071223in}}{\pgfqpoint{0.536273in}{4.071223in}}%
\pgfpathcurveto{\pgfqpoint{0.525223in}{4.071223in}}{\pgfqpoint{0.514624in}{4.066833in}}{\pgfqpoint{0.506811in}{4.059019in}}%
\pgfpathcurveto{\pgfqpoint{0.498997in}{4.051205in}}{\pgfqpoint{0.494607in}{4.040606in}}{\pgfqpoint{0.494607in}{4.029556in}}%
\pgfpathcurveto{\pgfqpoint{0.494607in}{4.018506in}}{\pgfqpoint{0.498997in}{4.007907in}}{\pgfqpoint{0.506811in}{4.000093in}}%
\pgfpathcurveto{\pgfqpoint{0.514624in}{3.992280in}}{\pgfqpoint{0.525223in}{3.987890in}}{\pgfqpoint{0.536273in}{3.987890in}}%
\pgfpathlineto{\pgfqpoint{0.536273in}{3.987890in}}%
\pgfpathclose%
\pgfusepath{stroke}%
\end{pgfscope}%
\begin{pgfscope}%
\pgfpathrectangle{\pgfqpoint{0.494722in}{0.437222in}}{\pgfqpoint{6.275590in}{5.159444in}}%
\pgfusepath{clip}%
\pgfsetbuttcap%
\pgfsetroundjoin%
\pgfsetlinewidth{1.003750pt}%
\definecolor{currentstroke}{rgb}{0.827451,0.827451,0.827451}%
\pgfsetstrokecolor{currentstroke}%
\pgfsetstrokeopacity{0.800000}%
\pgfsetdash{}{0pt}%
\pgfpathmoveto{\pgfqpoint{2.490073in}{1.171385in}}%
\pgfpathcurveto{\pgfqpoint{2.501123in}{1.171385in}}{\pgfqpoint{2.511722in}{1.175776in}}{\pgfqpoint{2.519535in}{1.183589in}}%
\pgfpathcurveto{\pgfqpoint{2.527349in}{1.191403in}}{\pgfqpoint{2.531739in}{1.202002in}}{\pgfqpoint{2.531739in}{1.213052in}}%
\pgfpathcurveto{\pgfqpoint{2.531739in}{1.224102in}}{\pgfqpoint{2.527349in}{1.234701in}}{\pgfqpoint{2.519535in}{1.242515in}}%
\pgfpathcurveto{\pgfqpoint{2.511722in}{1.250328in}}{\pgfqpoint{2.501123in}{1.254719in}}{\pgfqpoint{2.490073in}{1.254719in}}%
\pgfpathcurveto{\pgfqpoint{2.479023in}{1.254719in}}{\pgfqpoint{2.468423in}{1.250328in}}{\pgfqpoint{2.460610in}{1.242515in}}%
\pgfpathcurveto{\pgfqpoint{2.452796in}{1.234701in}}{\pgfqpoint{2.448406in}{1.224102in}}{\pgfqpoint{2.448406in}{1.213052in}}%
\pgfpathcurveto{\pgfqpoint{2.448406in}{1.202002in}}{\pgfqpoint{2.452796in}{1.191403in}}{\pgfqpoint{2.460610in}{1.183589in}}%
\pgfpathcurveto{\pgfqpoint{2.468423in}{1.175776in}}{\pgfqpoint{2.479023in}{1.171385in}}{\pgfqpoint{2.490073in}{1.171385in}}%
\pgfpathlineto{\pgfqpoint{2.490073in}{1.171385in}}%
\pgfpathclose%
\pgfusepath{stroke}%
\end{pgfscope}%
\begin{pgfscope}%
\pgfpathrectangle{\pgfqpoint{0.494722in}{0.437222in}}{\pgfqpoint{6.275590in}{5.159444in}}%
\pgfusepath{clip}%
\pgfsetbuttcap%
\pgfsetroundjoin%
\pgfsetlinewidth{1.003750pt}%
\definecolor{currentstroke}{rgb}{0.827451,0.827451,0.827451}%
\pgfsetstrokecolor{currentstroke}%
\pgfsetstrokeopacity{0.800000}%
\pgfsetdash{}{0pt}%
\pgfpathmoveto{\pgfqpoint{4.176161in}{0.527713in}}%
\pgfpathcurveto{\pgfqpoint{4.187212in}{0.527713in}}{\pgfqpoint{4.197811in}{0.532104in}}{\pgfqpoint{4.205624in}{0.539917in}}%
\pgfpathcurveto{\pgfqpoint{4.213438in}{0.547731in}}{\pgfqpoint{4.217828in}{0.558330in}}{\pgfqpoint{4.217828in}{0.569380in}}%
\pgfpathcurveto{\pgfqpoint{4.217828in}{0.580430in}}{\pgfqpoint{4.213438in}{0.591029in}}{\pgfqpoint{4.205624in}{0.598843in}}%
\pgfpathcurveto{\pgfqpoint{4.197811in}{0.606656in}}{\pgfqpoint{4.187212in}{0.611047in}}{\pgfqpoint{4.176161in}{0.611047in}}%
\pgfpathcurveto{\pgfqpoint{4.165111in}{0.611047in}}{\pgfqpoint{4.154512in}{0.606656in}}{\pgfqpoint{4.146699in}{0.598843in}}%
\pgfpathcurveto{\pgfqpoint{4.138885in}{0.591029in}}{\pgfqpoint{4.134495in}{0.580430in}}{\pgfqpoint{4.134495in}{0.569380in}}%
\pgfpathcurveto{\pgfqpoint{4.134495in}{0.558330in}}{\pgfqpoint{4.138885in}{0.547731in}}{\pgfqpoint{4.146699in}{0.539917in}}%
\pgfpathcurveto{\pgfqpoint{4.154512in}{0.532104in}}{\pgfqpoint{4.165111in}{0.527713in}}{\pgfqpoint{4.176161in}{0.527713in}}%
\pgfpathlineto{\pgfqpoint{4.176161in}{0.527713in}}%
\pgfpathclose%
\pgfusepath{stroke}%
\end{pgfscope}%
\begin{pgfscope}%
\pgfpathrectangle{\pgfqpoint{0.494722in}{0.437222in}}{\pgfqpoint{6.275590in}{5.159444in}}%
\pgfusepath{clip}%
\pgfsetbuttcap%
\pgfsetroundjoin%
\pgfsetlinewidth{1.003750pt}%
\definecolor{currentstroke}{rgb}{0.827451,0.827451,0.827451}%
\pgfsetstrokecolor{currentstroke}%
\pgfsetstrokeopacity{0.800000}%
\pgfsetdash{}{0pt}%
\pgfpathmoveto{\pgfqpoint{3.503642in}{0.697297in}}%
\pgfpathcurveto{\pgfqpoint{3.514692in}{0.697297in}}{\pgfqpoint{3.525291in}{0.701688in}}{\pgfqpoint{3.533105in}{0.709501in}}%
\pgfpathcurveto{\pgfqpoint{3.540918in}{0.717315in}}{\pgfqpoint{3.545308in}{0.727914in}}{\pgfqpoint{3.545308in}{0.738964in}}%
\pgfpathcurveto{\pgfqpoint{3.545308in}{0.750014in}}{\pgfqpoint{3.540918in}{0.760613in}}{\pgfqpoint{3.533105in}{0.768427in}}%
\pgfpathcurveto{\pgfqpoint{3.525291in}{0.776240in}}{\pgfqpoint{3.514692in}{0.780631in}}{\pgfqpoint{3.503642in}{0.780631in}}%
\pgfpathcurveto{\pgfqpoint{3.492592in}{0.780631in}}{\pgfqpoint{3.481993in}{0.776240in}}{\pgfqpoint{3.474179in}{0.768427in}}%
\pgfpathcurveto{\pgfqpoint{3.466365in}{0.760613in}}{\pgfqpoint{3.461975in}{0.750014in}}{\pgfqpoint{3.461975in}{0.738964in}}%
\pgfpathcurveto{\pgfqpoint{3.461975in}{0.727914in}}{\pgfqpoint{3.466365in}{0.717315in}}{\pgfqpoint{3.474179in}{0.709501in}}%
\pgfpathcurveto{\pgfqpoint{3.481993in}{0.701688in}}{\pgfqpoint{3.492592in}{0.697297in}}{\pgfqpoint{3.503642in}{0.697297in}}%
\pgfpathlineto{\pgfqpoint{3.503642in}{0.697297in}}%
\pgfpathclose%
\pgfusepath{stroke}%
\end{pgfscope}%
\begin{pgfscope}%
\pgfpathrectangle{\pgfqpoint{0.494722in}{0.437222in}}{\pgfqpoint{6.275590in}{5.159444in}}%
\pgfusepath{clip}%
\pgfsetbuttcap%
\pgfsetroundjoin%
\pgfsetlinewidth{1.003750pt}%
\definecolor{currentstroke}{rgb}{0.827451,0.827451,0.827451}%
\pgfsetstrokecolor{currentstroke}%
\pgfsetstrokeopacity{0.800000}%
\pgfsetdash{}{0pt}%
\pgfpathmoveto{\pgfqpoint{2.365669in}{1.242580in}}%
\pgfpathcurveto{\pgfqpoint{2.376719in}{1.242580in}}{\pgfqpoint{2.387318in}{1.246970in}}{\pgfqpoint{2.395131in}{1.254784in}}%
\pgfpathcurveto{\pgfqpoint{2.402945in}{1.262598in}}{\pgfqpoint{2.407335in}{1.273197in}}{\pgfqpoint{2.407335in}{1.284247in}}%
\pgfpathcurveto{\pgfqpoint{2.407335in}{1.295297in}}{\pgfqpoint{2.402945in}{1.305896in}}{\pgfqpoint{2.395131in}{1.313710in}}%
\pgfpathcurveto{\pgfqpoint{2.387318in}{1.321523in}}{\pgfqpoint{2.376719in}{1.325913in}}{\pgfqpoint{2.365669in}{1.325913in}}%
\pgfpathcurveto{\pgfqpoint{2.354618in}{1.325913in}}{\pgfqpoint{2.344019in}{1.321523in}}{\pgfqpoint{2.336206in}{1.313710in}}%
\pgfpathcurveto{\pgfqpoint{2.328392in}{1.305896in}}{\pgfqpoint{2.324002in}{1.295297in}}{\pgfqpoint{2.324002in}{1.284247in}}%
\pgfpathcurveto{\pgfqpoint{2.324002in}{1.273197in}}{\pgfqpoint{2.328392in}{1.262598in}}{\pgfqpoint{2.336206in}{1.254784in}}%
\pgfpathcurveto{\pgfqpoint{2.344019in}{1.246970in}}{\pgfqpoint{2.354618in}{1.242580in}}{\pgfqpoint{2.365669in}{1.242580in}}%
\pgfpathlineto{\pgfqpoint{2.365669in}{1.242580in}}%
\pgfpathclose%
\pgfusepath{stroke}%
\end{pgfscope}%
\begin{pgfscope}%
\pgfpathrectangle{\pgfqpoint{0.494722in}{0.437222in}}{\pgfqpoint{6.275590in}{5.159444in}}%
\pgfusepath{clip}%
\pgfsetbuttcap%
\pgfsetroundjoin%
\pgfsetlinewidth{1.003750pt}%
\definecolor{currentstroke}{rgb}{0.827451,0.827451,0.827451}%
\pgfsetstrokecolor{currentstroke}%
\pgfsetstrokeopacity{0.800000}%
\pgfsetdash{}{0pt}%
\pgfpathmoveto{\pgfqpoint{2.224297in}{1.325142in}}%
\pgfpathcurveto{\pgfqpoint{2.235347in}{1.325142in}}{\pgfqpoint{2.245946in}{1.329532in}}{\pgfqpoint{2.253760in}{1.337346in}}%
\pgfpathcurveto{\pgfqpoint{2.261573in}{1.345159in}}{\pgfqpoint{2.265964in}{1.355758in}}{\pgfqpoint{2.265964in}{1.366809in}}%
\pgfpathcurveto{\pgfqpoint{2.265964in}{1.377859in}}{\pgfqpoint{2.261573in}{1.388458in}}{\pgfqpoint{2.253760in}{1.396271in}}%
\pgfpathcurveto{\pgfqpoint{2.245946in}{1.404085in}}{\pgfqpoint{2.235347in}{1.408475in}}{\pgfqpoint{2.224297in}{1.408475in}}%
\pgfpathcurveto{\pgfqpoint{2.213247in}{1.408475in}}{\pgfqpoint{2.202648in}{1.404085in}}{\pgfqpoint{2.194834in}{1.396271in}}%
\pgfpathcurveto{\pgfqpoint{2.187020in}{1.388458in}}{\pgfqpoint{2.182630in}{1.377859in}}{\pgfqpoint{2.182630in}{1.366809in}}%
\pgfpathcurveto{\pgfqpoint{2.182630in}{1.355758in}}{\pgfqpoint{2.187020in}{1.345159in}}{\pgfqpoint{2.194834in}{1.337346in}}%
\pgfpathcurveto{\pgfqpoint{2.202648in}{1.329532in}}{\pgfqpoint{2.213247in}{1.325142in}}{\pgfqpoint{2.224297in}{1.325142in}}%
\pgfpathlineto{\pgfqpoint{2.224297in}{1.325142in}}%
\pgfpathclose%
\pgfusepath{stroke}%
\end{pgfscope}%
\begin{pgfscope}%
\pgfpathrectangle{\pgfqpoint{0.494722in}{0.437222in}}{\pgfqpoint{6.275590in}{5.159444in}}%
\pgfusepath{clip}%
\pgfsetbuttcap%
\pgfsetroundjoin%
\pgfsetlinewidth{1.003750pt}%
\definecolor{currentstroke}{rgb}{0.827451,0.827451,0.827451}%
\pgfsetstrokecolor{currentstroke}%
\pgfsetstrokeopacity{0.800000}%
\pgfsetdash{}{0pt}%
\pgfpathmoveto{\pgfqpoint{0.808638in}{2.983312in}}%
\pgfpathcurveto{\pgfqpoint{0.819689in}{2.983312in}}{\pgfqpoint{0.830288in}{2.987702in}}{\pgfqpoint{0.838101in}{2.995515in}}%
\pgfpathcurveto{\pgfqpoint{0.845915in}{3.003329in}}{\pgfqpoint{0.850305in}{3.013928in}}{\pgfqpoint{0.850305in}{3.024978in}}%
\pgfpathcurveto{\pgfqpoint{0.850305in}{3.036028in}}{\pgfqpoint{0.845915in}{3.046627in}}{\pgfqpoint{0.838101in}{3.054441in}}%
\pgfpathcurveto{\pgfqpoint{0.830288in}{3.062255in}}{\pgfqpoint{0.819689in}{3.066645in}}{\pgfqpoint{0.808638in}{3.066645in}}%
\pgfpathcurveto{\pgfqpoint{0.797588in}{3.066645in}}{\pgfqpoint{0.786989in}{3.062255in}}{\pgfqpoint{0.779176in}{3.054441in}}%
\pgfpathcurveto{\pgfqpoint{0.771362in}{3.046627in}}{\pgfqpoint{0.766972in}{3.036028in}}{\pgfqpoint{0.766972in}{3.024978in}}%
\pgfpathcurveto{\pgfqpoint{0.766972in}{3.013928in}}{\pgfqpoint{0.771362in}{3.003329in}}{\pgfqpoint{0.779176in}{2.995515in}}%
\pgfpathcurveto{\pgfqpoint{0.786989in}{2.987702in}}{\pgfqpoint{0.797588in}{2.983312in}}{\pgfqpoint{0.808638in}{2.983312in}}%
\pgfpathlineto{\pgfqpoint{0.808638in}{2.983312in}}%
\pgfpathclose%
\pgfusepath{stroke}%
\end{pgfscope}%
\begin{pgfscope}%
\pgfpathrectangle{\pgfqpoint{0.494722in}{0.437222in}}{\pgfqpoint{6.275590in}{5.159444in}}%
\pgfusepath{clip}%
\pgfsetbuttcap%
\pgfsetroundjoin%
\pgfsetlinewidth{1.003750pt}%
\definecolor{currentstroke}{rgb}{0.827451,0.827451,0.827451}%
\pgfsetstrokecolor{currentstroke}%
\pgfsetstrokeopacity{0.800000}%
\pgfsetdash{}{0pt}%
\pgfpathmoveto{\pgfqpoint{1.325845in}{2.126824in}}%
\pgfpathcurveto{\pgfqpoint{1.336895in}{2.126824in}}{\pgfqpoint{1.347494in}{2.131214in}}{\pgfqpoint{1.355308in}{2.139028in}}%
\pgfpathcurveto{\pgfqpoint{1.363121in}{2.146842in}}{\pgfqpoint{1.367512in}{2.157441in}}{\pgfqpoint{1.367512in}{2.168491in}}%
\pgfpathcurveto{\pgfqpoint{1.367512in}{2.179541in}}{\pgfqpoint{1.363121in}{2.190140in}}{\pgfqpoint{1.355308in}{2.197954in}}%
\pgfpathcurveto{\pgfqpoint{1.347494in}{2.205767in}}{\pgfqpoint{1.336895in}{2.210157in}}{\pgfqpoint{1.325845in}{2.210157in}}%
\pgfpathcurveto{\pgfqpoint{1.314795in}{2.210157in}}{\pgfqpoint{1.304196in}{2.205767in}}{\pgfqpoint{1.296382in}{2.197954in}}%
\pgfpathcurveto{\pgfqpoint{1.288569in}{2.190140in}}{\pgfqpoint{1.284178in}{2.179541in}}{\pgfqpoint{1.284178in}{2.168491in}}%
\pgfpathcurveto{\pgfqpoint{1.284178in}{2.157441in}}{\pgfqpoint{1.288569in}{2.146842in}}{\pgfqpoint{1.296382in}{2.139028in}}%
\pgfpathcurveto{\pgfqpoint{1.304196in}{2.131214in}}{\pgfqpoint{1.314795in}{2.126824in}}{\pgfqpoint{1.325845in}{2.126824in}}%
\pgfpathlineto{\pgfqpoint{1.325845in}{2.126824in}}%
\pgfpathclose%
\pgfusepath{stroke}%
\end{pgfscope}%
\begin{pgfscope}%
\pgfpathrectangle{\pgfqpoint{0.494722in}{0.437222in}}{\pgfqpoint{6.275590in}{5.159444in}}%
\pgfusepath{clip}%
\pgfsetbuttcap%
\pgfsetroundjoin%
\pgfsetlinewidth{1.003750pt}%
\definecolor{currentstroke}{rgb}{0.827451,0.827451,0.827451}%
\pgfsetstrokecolor{currentstroke}%
\pgfsetstrokeopacity{0.800000}%
\pgfsetdash{}{0pt}%
\pgfpathmoveto{\pgfqpoint{5.185000in}{0.412713in}}%
\pgfpathcurveto{\pgfqpoint{5.196050in}{0.412713in}}{\pgfqpoint{5.206649in}{0.417103in}}{\pgfqpoint{5.214463in}{0.424917in}}%
\pgfpathcurveto{\pgfqpoint{5.222276in}{0.432730in}}{\pgfqpoint{5.226667in}{0.443330in}}{\pgfqpoint{5.226667in}{0.454380in}}%
\pgfpathcurveto{\pgfqpoint{5.226667in}{0.465430in}}{\pgfqpoint{5.222276in}{0.476029in}}{\pgfqpoint{5.214463in}{0.483842in}}%
\pgfpathcurveto{\pgfqpoint{5.206649in}{0.491656in}}{\pgfqpoint{5.196050in}{0.496046in}}{\pgfqpoint{5.185000in}{0.496046in}}%
\pgfpathcurveto{\pgfqpoint{5.173950in}{0.496046in}}{\pgfqpoint{5.163351in}{0.491656in}}{\pgfqpoint{5.155537in}{0.483842in}}%
\pgfpathcurveto{\pgfqpoint{5.147724in}{0.476029in}}{\pgfqpoint{5.143333in}{0.465430in}}{\pgfqpoint{5.143333in}{0.454380in}}%
\pgfpathcurveto{\pgfqpoint{5.143333in}{0.443330in}}{\pgfqpoint{5.147724in}{0.432730in}}{\pgfqpoint{5.155537in}{0.424917in}}%
\pgfpathcurveto{\pgfqpoint{5.163351in}{0.417103in}}{\pgfqpoint{5.173950in}{0.412713in}}{\pgfqpoint{5.185000in}{0.412713in}}%
\pgfusepath{stroke}%
\end{pgfscope}%
\begin{pgfscope}%
\pgfpathrectangle{\pgfqpoint{0.494722in}{0.437222in}}{\pgfqpoint{6.275590in}{5.159444in}}%
\pgfusepath{clip}%
\pgfsetbuttcap%
\pgfsetroundjoin%
\pgfsetlinewidth{1.003750pt}%
\definecolor{currentstroke}{rgb}{0.827451,0.827451,0.827451}%
\pgfsetstrokecolor{currentstroke}%
\pgfsetstrokeopacity{0.800000}%
\pgfsetdash{}{0pt}%
\pgfpathmoveto{\pgfqpoint{0.645918in}{3.445519in}}%
\pgfpathcurveto{\pgfqpoint{0.656968in}{3.445519in}}{\pgfqpoint{0.667567in}{3.449909in}}{\pgfqpoint{0.675381in}{3.457723in}}%
\pgfpathcurveto{\pgfqpoint{0.683195in}{3.465536in}}{\pgfqpoint{0.687585in}{3.476135in}}{\pgfqpoint{0.687585in}{3.487185in}}%
\pgfpathcurveto{\pgfqpoint{0.687585in}{3.498235in}}{\pgfqpoint{0.683195in}{3.508835in}}{\pgfqpoint{0.675381in}{3.516648in}}%
\pgfpathcurveto{\pgfqpoint{0.667567in}{3.524462in}}{\pgfqpoint{0.656968in}{3.528852in}}{\pgfqpoint{0.645918in}{3.528852in}}%
\pgfpathcurveto{\pgfqpoint{0.634868in}{3.528852in}}{\pgfqpoint{0.624269in}{3.524462in}}{\pgfqpoint{0.616455in}{3.516648in}}%
\pgfpathcurveto{\pgfqpoint{0.608642in}{3.508835in}}{\pgfqpoint{0.604251in}{3.498235in}}{\pgfqpoint{0.604251in}{3.487185in}}%
\pgfpathcurveto{\pgfqpoint{0.604251in}{3.476135in}}{\pgfqpoint{0.608642in}{3.465536in}}{\pgfqpoint{0.616455in}{3.457723in}}%
\pgfpathcurveto{\pgfqpoint{0.624269in}{3.449909in}}{\pgfqpoint{0.634868in}{3.445519in}}{\pgfqpoint{0.645918in}{3.445519in}}%
\pgfpathlineto{\pgfqpoint{0.645918in}{3.445519in}}%
\pgfpathclose%
\pgfusepath{stroke}%
\end{pgfscope}%
\begin{pgfscope}%
\pgfpathrectangle{\pgfqpoint{0.494722in}{0.437222in}}{\pgfqpoint{6.275590in}{5.159444in}}%
\pgfusepath{clip}%
\pgfsetbuttcap%
\pgfsetroundjoin%
\pgfsetlinewidth{1.003750pt}%
\definecolor{currentstroke}{rgb}{0.827451,0.827451,0.827451}%
\pgfsetstrokecolor{currentstroke}%
\pgfsetstrokeopacity{0.800000}%
\pgfsetdash{}{0pt}%
\pgfpathmoveto{\pgfqpoint{4.362526in}{0.500037in}}%
\pgfpathcurveto{\pgfqpoint{4.373576in}{0.500037in}}{\pgfqpoint{4.384175in}{0.504427in}}{\pgfqpoint{4.391989in}{0.512241in}}%
\pgfpathcurveto{\pgfqpoint{4.399802in}{0.520054in}}{\pgfqpoint{4.404193in}{0.530653in}}{\pgfqpoint{4.404193in}{0.541703in}}%
\pgfpathcurveto{\pgfqpoint{4.404193in}{0.552753in}}{\pgfqpoint{4.399802in}{0.563352in}}{\pgfqpoint{4.391989in}{0.571166in}}%
\pgfpathcurveto{\pgfqpoint{4.384175in}{0.578980in}}{\pgfqpoint{4.373576in}{0.583370in}}{\pgfqpoint{4.362526in}{0.583370in}}%
\pgfpathcurveto{\pgfqpoint{4.351476in}{0.583370in}}{\pgfqpoint{4.340877in}{0.578980in}}{\pgfqpoint{4.333063in}{0.571166in}}%
\pgfpathcurveto{\pgfqpoint{4.325250in}{0.563352in}}{\pgfqpoint{4.320859in}{0.552753in}}{\pgfqpoint{4.320859in}{0.541703in}}%
\pgfpathcurveto{\pgfqpoint{4.320859in}{0.530653in}}{\pgfqpoint{4.325250in}{0.520054in}}{\pgfqpoint{4.333063in}{0.512241in}}%
\pgfpathcurveto{\pgfqpoint{4.340877in}{0.504427in}}{\pgfqpoint{4.351476in}{0.500037in}}{\pgfqpoint{4.362526in}{0.500037in}}%
\pgfpathlineto{\pgfqpoint{4.362526in}{0.500037in}}%
\pgfpathclose%
\pgfusepath{stroke}%
\end{pgfscope}%
\begin{pgfscope}%
\pgfpathrectangle{\pgfqpoint{0.494722in}{0.437222in}}{\pgfqpoint{6.275590in}{5.159444in}}%
\pgfusepath{clip}%
\pgfsetbuttcap%
\pgfsetroundjoin%
\pgfsetlinewidth{1.003750pt}%
\definecolor{currentstroke}{rgb}{0.827451,0.827451,0.827451}%
\pgfsetstrokecolor{currentstroke}%
\pgfsetstrokeopacity{0.800000}%
\pgfsetdash{}{0pt}%
\pgfpathmoveto{\pgfqpoint{1.632351in}{1.818842in}}%
\pgfpathcurveto{\pgfqpoint{1.643401in}{1.818842in}}{\pgfqpoint{1.654000in}{1.823232in}}{\pgfqpoint{1.661814in}{1.831046in}}%
\pgfpathcurveto{\pgfqpoint{1.669627in}{1.838860in}}{\pgfqpoint{1.674018in}{1.849459in}}{\pgfqpoint{1.674018in}{1.860509in}}%
\pgfpathcurveto{\pgfqpoint{1.674018in}{1.871559in}}{\pgfqpoint{1.669627in}{1.882158in}}{\pgfqpoint{1.661814in}{1.889972in}}%
\pgfpathcurveto{\pgfqpoint{1.654000in}{1.897785in}}{\pgfqpoint{1.643401in}{1.902176in}}{\pgfqpoint{1.632351in}{1.902176in}}%
\pgfpathcurveto{\pgfqpoint{1.621301in}{1.902176in}}{\pgfqpoint{1.610702in}{1.897785in}}{\pgfqpoint{1.602888in}{1.889972in}}%
\pgfpathcurveto{\pgfqpoint{1.595075in}{1.882158in}}{\pgfqpoint{1.590684in}{1.871559in}}{\pgfqpoint{1.590684in}{1.860509in}}%
\pgfpathcurveto{\pgfqpoint{1.590684in}{1.849459in}}{\pgfqpoint{1.595075in}{1.838860in}}{\pgfqpoint{1.602888in}{1.831046in}}%
\pgfpathcurveto{\pgfqpoint{1.610702in}{1.823232in}}{\pgfqpoint{1.621301in}{1.818842in}}{\pgfqpoint{1.632351in}{1.818842in}}%
\pgfpathlineto{\pgfqpoint{1.632351in}{1.818842in}}%
\pgfpathclose%
\pgfusepath{stroke}%
\end{pgfscope}%
\begin{pgfscope}%
\pgfpathrectangle{\pgfqpoint{0.494722in}{0.437222in}}{\pgfqpoint{6.275590in}{5.159444in}}%
\pgfusepath{clip}%
\pgfsetbuttcap%
\pgfsetroundjoin%
\pgfsetlinewidth{1.003750pt}%
\definecolor{currentstroke}{rgb}{0.827451,0.827451,0.827451}%
\pgfsetstrokecolor{currentstroke}%
\pgfsetstrokeopacity{0.800000}%
\pgfsetdash{}{0pt}%
\pgfpathmoveto{\pgfqpoint{0.595178in}{3.651665in}}%
\pgfpathcurveto{\pgfqpoint{0.606228in}{3.651665in}}{\pgfqpoint{0.616827in}{3.656056in}}{\pgfqpoint{0.624640in}{3.663869in}}%
\pgfpathcurveto{\pgfqpoint{0.632454in}{3.671683in}}{\pgfqpoint{0.636844in}{3.682282in}}{\pgfqpoint{0.636844in}{3.693332in}}%
\pgfpathcurveto{\pgfqpoint{0.636844in}{3.704382in}}{\pgfqpoint{0.632454in}{3.714981in}}{\pgfqpoint{0.624640in}{3.722795in}}%
\pgfpathcurveto{\pgfqpoint{0.616827in}{3.730609in}}{\pgfqpoint{0.606228in}{3.734999in}}{\pgfqpoint{0.595178in}{3.734999in}}%
\pgfpathcurveto{\pgfqpoint{0.584128in}{3.734999in}}{\pgfqpoint{0.573529in}{3.730609in}}{\pgfqpoint{0.565715in}{3.722795in}}%
\pgfpathcurveto{\pgfqpoint{0.557901in}{3.714981in}}{\pgfqpoint{0.553511in}{3.704382in}}{\pgfqpoint{0.553511in}{3.693332in}}%
\pgfpathcurveto{\pgfqpoint{0.553511in}{3.682282in}}{\pgfqpoint{0.557901in}{3.671683in}}{\pgfqpoint{0.565715in}{3.663869in}}%
\pgfpathcurveto{\pgfqpoint{0.573529in}{3.656056in}}{\pgfqpoint{0.584128in}{3.651665in}}{\pgfqpoint{0.595178in}{3.651665in}}%
\pgfpathlineto{\pgfqpoint{0.595178in}{3.651665in}}%
\pgfpathclose%
\pgfusepath{stroke}%
\end{pgfscope}%
\begin{pgfscope}%
\pgfpathrectangle{\pgfqpoint{0.494722in}{0.437222in}}{\pgfqpoint{6.275590in}{5.159444in}}%
\pgfusepath{clip}%
\pgfsetbuttcap%
\pgfsetroundjoin%
\pgfsetlinewidth{1.003750pt}%
\definecolor{currentstroke}{rgb}{0.827451,0.827451,0.827451}%
\pgfsetstrokecolor{currentstroke}%
\pgfsetstrokeopacity{0.800000}%
\pgfsetdash{}{0pt}%
\pgfpathmoveto{\pgfqpoint{1.854816in}{1.614653in}}%
\pgfpathcurveto{\pgfqpoint{1.865866in}{1.614653in}}{\pgfqpoint{1.876465in}{1.619043in}}{\pgfqpoint{1.884279in}{1.626857in}}%
\pgfpathcurveto{\pgfqpoint{1.892093in}{1.634670in}}{\pgfqpoint{1.896483in}{1.645269in}}{\pgfqpoint{1.896483in}{1.656319in}}%
\pgfpathcurveto{\pgfqpoint{1.896483in}{1.667370in}}{\pgfqpoint{1.892093in}{1.677969in}}{\pgfqpoint{1.884279in}{1.685782in}}%
\pgfpathcurveto{\pgfqpoint{1.876465in}{1.693596in}}{\pgfqpoint{1.865866in}{1.697986in}}{\pgfqpoint{1.854816in}{1.697986in}}%
\pgfpathcurveto{\pgfqpoint{1.843766in}{1.697986in}}{\pgfqpoint{1.833167in}{1.693596in}}{\pgfqpoint{1.825353in}{1.685782in}}%
\pgfpathcurveto{\pgfqpoint{1.817540in}{1.677969in}}{\pgfqpoint{1.813149in}{1.667370in}}{\pgfqpoint{1.813149in}{1.656319in}}%
\pgfpathcurveto{\pgfqpoint{1.813149in}{1.645269in}}{\pgfqpoint{1.817540in}{1.634670in}}{\pgfqpoint{1.825353in}{1.626857in}}%
\pgfpathcurveto{\pgfqpoint{1.833167in}{1.619043in}}{\pgfqpoint{1.843766in}{1.614653in}}{\pgfqpoint{1.854816in}{1.614653in}}%
\pgfpathlineto{\pgfqpoint{1.854816in}{1.614653in}}%
\pgfpathclose%
\pgfusepath{stroke}%
\end{pgfscope}%
\begin{pgfscope}%
\pgfpathrectangle{\pgfqpoint{0.494722in}{0.437222in}}{\pgfqpoint{6.275590in}{5.159444in}}%
\pgfusepath{clip}%
\pgfsetbuttcap%
\pgfsetroundjoin%
\pgfsetlinewidth{1.003750pt}%
\definecolor{currentstroke}{rgb}{0.827451,0.827451,0.827451}%
\pgfsetstrokecolor{currentstroke}%
\pgfsetstrokeopacity{0.800000}%
\pgfsetdash{}{0pt}%
\pgfpathmoveto{\pgfqpoint{0.517395in}{4.197711in}}%
\pgfpathcurveto{\pgfqpoint{0.528445in}{4.197711in}}{\pgfqpoint{0.539044in}{4.202101in}}{\pgfqpoint{0.546858in}{4.209915in}}%
\pgfpathcurveto{\pgfqpoint{0.554671in}{4.217728in}}{\pgfqpoint{0.559061in}{4.228327in}}{\pgfqpoint{0.559061in}{4.239378in}}%
\pgfpathcurveto{\pgfqpoint{0.559061in}{4.250428in}}{\pgfqpoint{0.554671in}{4.261027in}}{\pgfqpoint{0.546858in}{4.268840in}}%
\pgfpathcurveto{\pgfqpoint{0.539044in}{4.276654in}}{\pgfqpoint{0.528445in}{4.281044in}}{\pgfqpoint{0.517395in}{4.281044in}}%
\pgfpathcurveto{\pgfqpoint{0.506345in}{4.281044in}}{\pgfqpoint{0.495746in}{4.276654in}}{\pgfqpoint{0.487932in}{4.268840in}}%
\pgfpathcurveto{\pgfqpoint{0.480118in}{4.261027in}}{\pgfqpoint{0.475728in}{4.250428in}}{\pgfqpoint{0.475728in}{4.239378in}}%
\pgfpathcurveto{\pgfqpoint{0.475728in}{4.228327in}}{\pgfqpoint{0.480118in}{4.217728in}}{\pgfqpoint{0.487932in}{4.209915in}}%
\pgfpathcurveto{\pgfqpoint{0.495746in}{4.202101in}}{\pgfqpoint{0.506345in}{4.197711in}}{\pgfqpoint{0.517395in}{4.197711in}}%
\pgfpathlineto{\pgfqpoint{0.517395in}{4.197711in}}%
\pgfpathclose%
\pgfusepath{stroke}%
\end{pgfscope}%
\begin{pgfscope}%
\pgfpathrectangle{\pgfqpoint{0.494722in}{0.437222in}}{\pgfqpoint{6.275590in}{5.159444in}}%
\pgfusepath{clip}%
\pgfsetbuttcap%
\pgfsetroundjoin%
\pgfsetlinewidth{1.003750pt}%
\definecolor{currentstroke}{rgb}{0.827451,0.827451,0.827451}%
\pgfsetstrokecolor{currentstroke}%
\pgfsetstrokeopacity{0.800000}%
\pgfsetdash{}{0pt}%
\pgfpathmoveto{\pgfqpoint{1.227841in}{2.257546in}}%
\pgfpathcurveto{\pgfqpoint{1.238891in}{2.257546in}}{\pgfqpoint{1.249490in}{2.261936in}}{\pgfqpoint{1.257304in}{2.269750in}}%
\pgfpathcurveto{\pgfqpoint{1.265118in}{2.277564in}}{\pgfqpoint{1.269508in}{2.288163in}}{\pgfqpoint{1.269508in}{2.299213in}}%
\pgfpathcurveto{\pgfqpoint{1.269508in}{2.310263in}}{\pgfqpoint{1.265118in}{2.320862in}}{\pgfqpoint{1.257304in}{2.328675in}}%
\pgfpathcurveto{\pgfqpoint{1.249490in}{2.336489in}}{\pgfqpoint{1.238891in}{2.340879in}}{\pgfqpoint{1.227841in}{2.340879in}}%
\pgfpathcurveto{\pgfqpoint{1.216791in}{2.340879in}}{\pgfqpoint{1.206192in}{2.336489in}}{\pgfqpoint{1.198379in}{2.328675in}}%
\pgfpathcurveto{\pgfqpoint{1.190565in}{2.320862in}}{\pgfqpoint{1.186175in}{2.310263in}}{\pgfqpoint{1.186175in}{2.299213in}}%
\pgfpathcurveto{\pgfqpoint{1.186175in}{2.288163in}}{\pgfqpoint{1.190565in}{2.277564in}}{\pgfqpoint{1.198379in}{2.269750in}}%
\pgfpathcurveto{\pgfqpoint{1.206192in}{2.261936in}}{\pgfqpoint{1.216791in}{2.257546in}}{\pgfqpoint{1.227841in}{2.257546in}}%
\pgfpathlineto{\pgfqpoint{1.227841in}{2.257546in}}%
\pgfpathclose%
\pgfusepath{stroke}%
\end{pgfscope}%
\begin{pgfscope}%
\pgfpathrectangle{\pgfqpoint{0.494722in}{0.437222in}}{\pgfqpoint{6.275590in}{5.159444in}}%
\pgfusepath{clip}%
\pgfsetbuttcap%
\pgfsetroundjoin%
\pgfsetlinewidth{1.003750pt}%
\definecolor{currentstroke}{rgb}{0.827451,0.827451,0.827451}%
\pgfsetstrokecolor{currentstroke}%
\pgfsetstrokeopacity{0.800000}%
\pgfsetdash{}{0pt}%
\pgfpathmoveto{\pgfqpoint{4.399590in}{0.490376in}}%
\pgfpathcurveto{\pgfqpoint{4.410640in}{0.490376in}}{\pgfqpoint{4.421239in}{0.494766in}}{\pgfqpoint{4.429053in}{0.502580in}}%
\pgfpathcurveto{\pgfqpoint{4.436866in}{0.510393in}}{\pgfqpoint{4.441257in}{0.520992in}}{\pgfqpoint{4.441257in}{0.532042in}}%
\pgfpathcurveto{\pgfqpoint{4.441257in}{0.543093in}}{\pgfqpoint{4.436866in}{0.553692in}}{\pgfqpoint{4.429053in}{0.561505in}}%
\pgfpathcurveto{\pgfqpoint{4.421239in}{0.569319in}}{\pgfqpoint{4.410640in}{0.573709in}}{\pgfqpoint{4.399590in}{0.573709in}}%
\pgfpathcurveto{\pgfqpoint{4.388540in}{0.573709in}}{\pgfqpoint{4.377941in}{0.569319in}}{\pgfqpoint{4.370127in}{0.561505in}}%
\pgfpathcurveto{\pgfqpoint{4.362314in}{0.553692in}}{\pgfqpoint{4.357923in}{0.543093in}}{\pgfqpoint{4.357923in}{0.532042in}}%
\pgfpathcurveto{\pgfqpoint{4.357923in}{0.520992in}}{\pgfqpoint{4.362314in}{0.510393in}}{\pgfqpoint{4.370127in}{0.502580in}}%
\pgfpathcurveto{\pgfqpoint{4.377941in}{0.494766in}}{\pgfqpoint{4.388540in}{0.490376in}}{\pgfqpoint{4.399590in}{0.490376in}}%
\pgfpathlineto{\pgfqpoint{4.399590in}{0.490376in}}%
\pgfpathclose%
\pgfusepath{stroke}%
\end{pgfscope}%
\begin{pgfscope}%
\pgfpathrectangle{\pgfqpoint{0.494722in}{0.437222in}}{\pgfqpoint{6.275590in}{5.159444in}}%
\pgfusepath{clip}%
\pgfsetbuttcap%
\pgfsetroundjoin%
\pgfsetlinewidth{1.003750pt}%
\definecolor{currentstroke}{rgb}{0.827451,0.827451,0.827451}%
\pgfsetstrokecolor{currentstroke}%
\pgfsetstrokeopacity{0.800000}%
\pgfsetdash{}{0pt}%
\pgfpathmoveto{\pgfqpoint{0.527771in}{4.064004in}}%
\pgfpathcurveto{\pgfqpoint{0.538821in}{4.064004in}}{\pgfqpoint{0.549420in}{4.068394in}}{\pgfqpoint{0.557234in}{4.076208in}}%
\pgfpathcurveto{\pgfqpoint{0.565047in}{4.084022in}}{\pgfqpoint{0.569438in}{4.094621in}}{\pgfqpoint{0.569438in}{4.105671in}}%
\pgfpathcurveto{\pgfqpoint{0.569438in}{4.116721in}}{\pgfqpoint{0.565047in}{4.127320in}}{\pgfqpoint{0.557234in}{4.135134in}}%
\pgfpathcurveto{\pgfqpoint{0.549420in}{4.142947in}}{\pgfqpoint{0.538821in}{4.147338in}}{\pgfqpoint{0.527771in}{4.147338in}}%
\pgfpathcurveto{\pgfqpoint{0.516721in}{4.147338in}}{\pgfqpoint{0.506122in}{4.142947in}}{\pgfqpoint{0.498308in}{4.135134in}}%
\pgfpathcurveto{\pgfqpoint{0.490494in}{4.127320in}}{\pgfqpoint{0.486104in}{4.116721in}}{\pgfqpoint{0.486104in}{4.105671in}}%
\pgfpathcurveto{\pgfqpoint{0.486104in}{4.094621in}}{\pgfqpoint{0.490494in}{4.084022in}}{\pgfqpoint{0.498308in}{4.076208in}}%
\pgfpathcurveto{\pgfqpoint{0.506122in}{4.068394in}}{\pgfqpoint{0.516721in}{4.064004in}}{\pgfqpoint{0.527771in}{4.064004in}}%
\pgfpathlineto{\pgfqpoint{0.527771in}{4.064004in}}%
\pgfpathclose%
\pgfusepath{stroke}%
\end{pgfscope}%
\begin{pgfscope}%
\pgfpathrectangle{\pgfqpoint{0.494722in}{0.437222in}}{\pgfqpoint{6.275590in}{5.159444in}}%
\pgfusepath{clip}%
\pgfsetbuttcap%
\pgfsetroundjoin%
\pgfsetlinewidth{1.003750pt}%
\definecolor{currentstroke}{rgb}{0.827451,0.827451,0.827451}%
\pgfsetstrokecolor{currentstroke}%
\pgfsetstrokeopacity{0.800000}%
\pgfsetdash{}{0pt}%
\pgfpathmoveto{\pgfqpoint{1.852529in}{1.617217in}}%
\pgfpathcurveto{\pgfqpoint{1.863579in}{1.617217in}}{\pgfqpoint{1.874178in}{1.621608in}}{\pgfqpoint{1.881991in}{1.629421in}}%
\pgfpathcurveto{\pgfqpoint{1.889805in}{1.637235in}}{\pgfqpoint{1.894195in}{1.647834in}}{\pgfqpoint{1.894195in}{1.658884in}}%
\pgfpathcurveto{\pgfqpoint{1.894195in}{1.669934in}}{\pgfqpoint{1.889805in}{1.680533in}}{\pgfqpoint{1.881991in}{1.688347in}}%
\pgfpathcurveto{\pgfqpoint{1.874178in}{1.696160in}}{\pgfqpoint{1.863579in}{1.700551in}}{\pgfqpoint{1.852529in}{1.700551in}}%
\pgfpathcurveto{\pgfqpoint{1.841479in}{1.700551in}}{\pgfqpoint{1.830880in}{1.696160in}}{\pgfqpoint{1.823066in}{1.688347in}}%
\pgfpathcurveto{\pgfqpoint{1.815252in}{1.680533in}}{\pgfqpoint{1.810862in}{1.669934in}}{\pgfqpoint{1.810862in}{1.658884in}}%
\pgfpathcurveto{\pgfqpoint{1.810862in}{1.647834in}}{\pgfqpoint{1.815252in}{1.637235in}}{\pgfqpoint{1.823066in}{1.629421in}}%
\pgfpathcurveto{\pgfqpoint{1.830880in}{1.621608in}}{\pgfqpoint{1.841479in}{1.617217in}}{\pgfqpoint{1.852529in}{1.617217in}}%
\pgfpathlineto{\pgfqpoint{1.852529in}{1.617217in}}%
\pgfpathclose%
\pgfusepath{stroke}%
\end{pgfscope}%
\begin{pgfscope}%
\pgfpathrectangle{\pgfqpoint{0.494722in}{0.437222in}}{\pgfqpoint{6.275590in}{5.159444in}}%
\pgfusepath{clip}%
\pgfsetbuttcap%
\pgfsetroundjoin%
\pgfsetlinewidth{1.003750pt}%
\definecolor{currentstroke}{rgb}{0.827451,0.827451,0.827451}%
\pgfsetstrokecolor{currentstroke}%
\pgfsetstrokeopacity{0.800000}%
\pgfsetdash{}{0pt}%
\pgfpathmoveto{\pgfqpoint{3.396083in}{0.743072in}}%
\pgfpathcurveto{\pgfqpoint{3.407133in}{0.743072in}}{\pgfqpoint{3.417732in}{0.747462in}}{\pgfqpoint{3.425546in}{0.755276in}}%
\pgfpathcurveto{\pgfqpoint{3.433360in}{0.763089in}}{\pgfqpoint{3.437750in}{0.773688in}}{\pgfqpoint{3.437750in}{0.784739in}}%
\pgfpathcurveto{\pgfqpoint{3.437750in}{0.795789in}}{\pgfqpoint{3.433360in}{0.806388in}}{\pgfqpoint{3.425546in}{0.814201in}}%
\pgfpathcurveto{\pgfqpoint{3.417732in}{0.822015in}}{\pgfqpoint{3.407133in}{0.826405in}}{\pgfqpoint{3.396083in}{0.826405in}}%
\pgfpathcurveto{\pgfqpoint{3.385033in}{0.826405in}}{\pgfqpoint{3.374434in}{0.822015in}}{\pgfqpoint{3.366621in}{0.814201in}}%
\pgfpathcurveto{\pgfqpoint{3.358807in}{0.806388in}}{\pgfqpoint{3.354417in}{0.795789in}}{\pgfqpoint{3.354417in}{0.784739in}}%
\pgfpathcurveto{\pgfqpoint{3.354417in}{0.773688in}}{\pgfqpoint{3.358807in}{0.763089in}}{\pgfqpoint{3.366621in}{0.755276in}}%
\pgfpathcurveto{\pgfqpoint{3.374434in}{0.747462in}}{\pgfqpoint{3.385033in}{0.743072in}}{\pgfqpoint{3.396083in}{0.743072in}}%
\pgfpathlineto{\pgfqpoint{3.396083in}{0.743072in}}%
\pgfpathclose%
\pgfusepath{stroke}%
\end{pgfscope}%
\begin{pgfscope}%
\pgfpathrectangle{\pgfqpoint{0.494722in}{0.437222in}}{\pgfqpoint{6.275590in}{5.159444in}}%
\pgfusepath{clip}%
\pgfsetbuttcap%
\pgfsetroundjoin%
\pgfsetlinewidth{1.003750pt}%
\definecolor{currentstroke}{rgb}{0.827451,0.827451,0.827451}%
\pgfsetstrokecolor{currentstroke}%
\pgfsetstrokeopacity{0.800000}%
\pgfsetdash{}{0pt}%
\pgfpathmoveto{\pgfqpoint{3.055005in}{0.879521in}}%
\pgfpathcurveto{\pgfqpoint{3.066055in}{0.879521in}}{\pgfqpoint{3.076654in}{0.883911in}}{\pgfqpoint{3.084467in}{0.891725in}}%
\pgfpathcurveto{\pgfqpoint{3.092281in}{0.899538in}}{\pgfqpoint{3.096671in}{0.910137in}}{\pgfqpoint{3.096671in}{0.921187in}}%
\pgfpathcurveto{\pgfqpoint{3.096671in}{0.932237in}}{\pgfqpoint{3.092281in}{0.942836in}}{\pgfqpoint{3.084467in}{0.950650in}}%
\pgfpathcurveto{\pgfqpoint{3.076654in}{0.958464in}}{\pgfqpoint{3.066055in}{0.962854in}}{\pgfqpoint{3.055005in}{0.962854in}}%
\pgfpathcurveto{\pgfqpoint{3.043955in}{0.962854in}}{\pgfqpoint{3.033356in}{0.958464in}}{\pgfqpoint{3.025542in}{0.950650in}}%
\pgfpathcurveto{\pgfqpoint{3.017728in}{0.942836in}}{\pgfqpoint{3.013338in}{0.932237in}}{\pgfqpoint{3.013338in}{0.921187in}}%
\pgfpathcurveto{\pgfqpoint{3.013338in}{0.910137in}}{\pgfqpoint{3.017728in}{0.899538in}}{\pgfqpoint{3.025542in}{0.891725in}}%
\pgfpathcurveto{\pgfqpoint{3.033356in}{0.883911in}}{\pgfqpoint{3.043955in}{0.879521in}}{\pgfqpoint{3.055005in}{0.879521in}}%
\pgfpathlineto{\pgfqpoint{3.055005in}{0.879521in}}%
\pgfpathclose%
\pgfusepath{stroke}%
\end{pgfscope}%
\begin{pgfscope}%
\pgfpathrectangle{\pgfqpoint{0.494722in}{0.437222in}}{\pgfqpoint{6.275590in}{5.159444in}}%
\pgfusepath{clip}%
\pgfsetbuttcap%
\pgfsetroundjoin%
\pgfsetlinewidth{1.003750pt}%
\definecolor{currentstroke}{rgb}{0.827451,0.827451,0.827451}%
\pgfsetstrokecolor{currentstroke}%
\pgfsetstrokeopacity{0.800000}%
\pgfsetdash{}{0pt}%
\pgfpathmoveto{\pgfqpoint{1.293361in}{2.194969in}}%
\pgfpathcurveto{\pgfqpoint{1.304411in}{2.194969in}}{\pgfqpoint{1.315010in}{2.199360in}}{\pgfqpoint{1.322824in}{2.207173in}}%
\pgfpathcurveto{\pgfqpoint{1.330638in}{2.214987in}}{\pgfqpoint{1.335028in}{2.225586in}}{\pgfqpoint{1.335028in}{2.236636in}}%
\pgfpathcurveto{\pgfqpoint{1.335028in}{2.247686in}}{\pgfqpoint{1.330638in}{2.258285in}}{\pgfqpoint{1.322824in}{2.266099in}}%
\pgfpathcurveto{\pgfqpoint{1.315010in}{2.273912in}}{\pgfqpoint{1.304411in}{2.278303in}}{\pgfqpoint{1.293361in}{2.278303in}}%
\pgfpathcurveto{\pgfqpoint{1.282311in}{2.278303in}}{\pgfqpoint{1.271712in}{2.273912in}}{\pgfqpoint{1.263898in}{2.266099in}}%
\pgfpathcurveto{\pgfqpoint{1.256085in}{2.258285in}}{\pgfqpoint{1.251695in}{2.247686in}}{\pgfqpoint{1.251695in}{2.236636in}}%
\pgfpathcurveto{\pgfqpoint{1.251695in}{2.225586in}}{\pgfqpoint{1.256085in}{2.214987in}}{\pgfqpoint{1.263898in}{2.207173in}}%
\pgfpathcurveto{\pgfqpoint{1.271712in}{2.199360in}}{\pgfqpoint{1.282311in}{2.194969in}}{\pgfqpoint{1.293361in}{2.194969in}}%
\pgfpathlineto{\pgfqpoint{1.293361in}{2.194969in}}%
\pgfpathclose%
\pgfusepath{stroke}%
\end{pgfscope}%
\begin{pgfscope}%
\pgfpathrectangle{\pgfqpoint{0.494722in}{0.437222in}}{\pgfqpoint{6.275590in}{5.159444in}}%
\pgfusepath{clip}%
\pgfsetbuttcap%
\pgfsetroundjoin%
\pgfsetlinewidth{1.003750pt}%
\definecolor{currentstroke}{rgb}{0.827451,0.827451,0.827451}%
\pgfsetstrokecolor{currentstroke}%
\pgfsetstrokeopacity{0.800000}%
\pgfsetdash{}{0pt}%
\pgfpathmoveto{\pgfqpoint{0.632762in}{3.503002in}}%
\pgfpathcurveto{\pgfqpoint{0.643812in}{3.503002in}}{\pgfqpoint{0.654411in}{3.507392in}}{\pgfqpoint{0.662224in}{3.515206in}}%
\pgfpathcurveto{\pgfqpoint{0.670038in}{3.523019in}}{\pgfqpoint{0.674428in}{3.533618in}}{\pgfqpoint{0.674428in}{3.544668in}}%
\pgfpathcurveto{\pgfqpoint{0.674428in}{3.555719in}}{\pgfqpoint{0.670038in}{3.566318in}}{\pgfqpoint{0.662224in}{3.574131in}}%
\pgfpathcurveto{\pgfqpoint{0.654411in}{3.581945in}}{\pgfqpoint{0.643812in}{3.586335in}}{\pgfqpoint{0.632762in}{3.586335in}}%
\pgfpathcurveto{\pgfqpoint{0.621712in}{3.586335in}}{\pgfqpoint{0.611113in}{3.581945in}}{\pgfqpoint{0.603299in}{3.574131in}}%
\pgfpathcurveto{\pgfqpoint{0.595485in}{3.566318in}}{\pgfqpoint{0.591095in}{3.555719in}}{\pgfqpoint{0.591095in}{3.544668in}}%
\pgfpathcurveto{\pgfqpoint{0.591095in}{3.533618in}}{\pgfqpoint{0.595485in}{3.523019in}}{\pgfqpoint{0.603299in}{3.515206in}}%
\pgfpathcurveto{\pgfqpoint{0.611113in}{3.507392in}}{\pgfqpoint{0.621712in}{3.503002in}}{\pgfqpoint{0.632762in}{3.503002in}}%
\pgfpathlineto{\pgfqpoint{0.632762in}{3.503002in}}%
\pgfpathclose%
\pgfusepath{stroke}%
\end{pgfscope}%
\begin{pgfscope}%
\pgfpathrectangle{\pgfqpoint{0.494722in}{0.437222in}}{\pgfqpoint{6.275590in}{5.159444in}}%
\pgfusepath{clip}%
\pgfsetbuttcap%
\pgfsetroundjoin%
\pgfsetlinewidth{1.003750pt}%
\definecolor{currentstroke}{rgb}{0.827451,0.827451,0.827451}%
\pgfsetstrokecolor{currentstroke}%
\pgfsetstrokeopacity{0.800000}%
\pgfsetdash{}{0pt}%
\pgfpathmoveto{\pgfqpoint{1.112658in}{2.512839in}}%
\pgfpathcurveto{\pgfqpoint{1.123708in}{2.512839in}}{\pgfqpoint{1.134307in}{2.517229in}}{\pgfqpoint{1.142121in}{2.525043in}}%
\pgfpathcurveto{\pgfqpoint{1.149935in}{2.532856in}}{\pgfqpoint{1.154325in}{2.543455in}}{\pgfqpoint{1.154325in}{2.554505in}}%
\pgfpathcurveto{\pgfqpoint{1.154325in}{2.565556in}}{\pgfqpoint{1.149935in}{2.576155in}}{\pgfqpoint{1.142121in}{2.583968in}}%
\pgfpathcurveto{\pgfqpoint{1.134307in}{2.591782in}}{\pgfqpoint{1.123708in}{2.596172in}}{\pgfqpoint{1.112658in}{2.596172in}}%
\pgfpathcurveto{\pgfqpoint{1.101608in}{2.596172in}}{\pgfqpoint{1.091009in}{2.591782in}}{\pgfqpoint{1.083195in}{2.583968in}}%
\pgfpathcurveto{\pgfqpoint{1.075382in}{2.576155in}}{\pgfqpoint{1.070992in}{2.565556in}}{\pgfqpoint{1.070992in}{2.554505in}}%
\pgfpathcurveto{\pgfqpoint{1.070992in}{2.543455in}}{\pgfqpoint{1.075382in}{2.532856in}}{\pgfqpoint{1.083195in}{2.525043in}}%
\pgfpathcurveto{\pgfqpoint{1.091009in}{2.517229in}}{\pgfqpoint{1.101608in}{2.512839in}}{\pgfqpoint{1.112658in}{2.512839in}}%
\pgfpathlineto{\pgfqpoint{1.112658in}{2.512839in}}%
\pgfpathclose%
\pgfusepath{stroke}%
\end{pgfscope}%
\begin{pgfscope}%
\pgfpathrectangle{\pgfqpoint{0.494722in}{0.437222in}}{\pgfqpoint{6.275590in}{5.159444in}}%
\pgfusepath{clip}%
\pgfsetbuttcap%
\pgfsetroundjoin%
\pgfsetlinewidth{1.003750pt}%
\definecolor{currentstroke}{rgb}{0.827451,0.827451,0.827451}%
\pgfsetstrokecolor{currentstroke}%
\pgfsetstrokeopacity{0.800000}%
\pgfsetdash{}{0pt}%
\pgfpathmoveto{\pgfqpoint{0.625304in}{3.523383in}}%
\pgfpathcurveto{\pgfqpoint{0.636354in}{3.523383in}}{\pgfqpoint{0.646953in}{3.527774in}}{\pgfqpoint{0.654767in}{3.535587in}}%
\pgfpathcurveto{\pgfqpoint{0.662581in}{3.543401in}}{\pgfqpoint{0.666971in}{3.554000in}}{\pgfqpoint{0.666971in}{3.565050in}}%
\pgfpathcurveto{\pgfqpoint{0.666971in}{3.576100in}}{\pgfqpoint{0.662581in}{3.586699in}}{\pgfqpoint{0.654767in}{3.594513in}}%
\pgfpathcurveto{\pgfqpoint{0.646953in}{3.602326in}}{\pgfqpoint{0.636354in}{3.606717in}}{\pgfqpoint{0.625304in}{3.606717in}}%
\pgfpathcurveto{\pgfqpoint{0.614254in}{3.606717in}}{\pgfqpoint{0.603655in}{3.602326in}}{\pgfqpoint{0.595841in}{3.594513in}}%
\pgfpathcurveto{\pgfqpoint{0.588028in}{3.586699in}}{\pgfqpoint{0.583638in}{3.576100in}}{\pgfqpoint{0.583638in}{3.565050in}}%
\pgfpathcurveto{\pgfqpoint{0.583638in}{3.554000in}}{\pgfqpoint{0.588028in}{3.543401in}}{\pgfqpoint{0.595841in}{3.535587in}}%
\pgfpathcurveto{\pgfqpoint{0.603655in}{3.527774in}}{\pgfqpoint{0.614254in}{3.523383in}}{\pgfqpoint{0.625304in}{3.523383in}}%
\pgfpathlineto{\pgfqpoint{0.625304in}{3.523383in}}%
\pgfpathclose%
\pgfusepath{stroke}%
\end{pgfscope}%
\begin{pgfscope}%
\pgfpathrectangle{\pgfqpoint{0.494722in}{0.437222in}}{\pgfqpoint{6.275590in}{5.159444in}}%
\pgfusepath{clip}%
\pgfsetbuttcap%
\pgfsetroundjoin%
\pgfsetlinewidth{1.003750pt}%
\definecolor{currentstroke}{rgb}{0.827451,0.827451,0.827451}%
\pgfsetstrokecolor{currentstroke}%
\pgfsetstrokeopacity{0.800000}%
\pgfsetdash{}{0pt}%
\pgfpathmoveto{\pgfqpoint{4.229480in}{0.518110in}}%
\pgfpathcurveto{\pgfqpoint{4.240530in}{0.518110in}}{\pgfqpoint{4.251129in}{0.522500in}}{\pgfqpoint{4.258942in}{0.530314in}}%
\pgfpathcurveto{\pgfqpoint{4.266756in}{0.538127in}}{\pgfqpoint{4.271146in}{0.548726in}}{\pgfqpoint{4.271146in}{0.559776in}}%
\pgfpathcurveto{\pgfqpoint{4.271146in}{0.570827in}}{\pgfqpoint{4.266756in}{0.581426in}}{\pgfqpoint{4.258942in}{0.589239in}}%
\pgfpathcurveto{\pgfqpoint{4.251129in}{0.597053in}}{\pgfqpoint{4.240530in}{0.601443in}}{\pgfqpoint{4.229480in}{0.601443in}}%
\pgfpathcurveto{\pgfqpoint{4.218429in}{0.601443in}}{\pgfqpoint{4.207830in}{0.597053in}}{\pgfqpoint{4.200017in}{0.589239in}}%
\pgfpathcurveto{\pgfqpoint{4.192203in}{0.581426in}}{\pgfqpoint{4.187813in}{0.570827in}}{\pgfqpoint{4.187813in}{0.559776in}}%
\pgfpathcurveto{\pgfqpoint{4.187813in}{0.548726in}}{\pgfqpoint{4.192203in}{0.538127in}}{\pgfqpoint{4.200017in}{0.530314in}}%
\pgfpathcurveto{\pgfqpoint{4.207830in}{0.522500in}}{\pgfqpoint{4.218429in}{0.518110in}}{\pgfqpoint{4.229480in}{0.518110in}}%
\pgfpathlineto{\pgfqpoint{4.229480in}{0.518110in}}%
\pgfpathclose%
\pgfusepath{stroke}%
\end{pgfscope}%
\begin{pgfscope}%
\pgfpathrectangle{\pgfqpoint{0.494722in}{0.437222in}}{\pgfqpoint{6.275590in}{5.159444in}}%
\pgfusepath{clip}%
\pgfsetbuttcap%
\pgfsetroundjoin%
\pgfsetlinewidth{1.003750pt}%
\definecolor{currentstroke}{rgb}{0.827451,0.827451,0.827451}%
\pgfsetstrokecolor{currentstroke}%
\pgfsetstrokeopacity{0.800000}%
\pgfsetdash{}{0pt}%
\pgfpathmoveto{\pgfqpoint{4.495815in}{0.487123in}}%
\pgfpathcurveto{\pgfqpoint{4.506865in}{0.487123in}}{\pgfqpoint{4.517464in}{0.491513in}}{\pgfqpoint{4.525277in}{0.499327in}}%
\pgfpathcurveto{\pgfqpoint{4.533091in}{0.507140in}}{\pgfqpoint{4.537481in}{0.517740in}}{\pgfqpoint{4.537481in}{0.528790in}}%
\pgfpathcurveto{\pgfqpoint{4.537481in}{0.539840in}}{\pgfqpoint{4.533091in}{0.550439in}}{\pgfqpoint{4.525277in}{0.558252in}}%
\pgfpathcurveto{\pgfqpoint{4.517464in}{0.566066in}}{\pgfqpoint{4.506865in}{0.570456in}}{\pgfqpoint{4.495815in}{0.570456in}}%
\pgfpathcurveto{\pgfqpoint{4.484765in}{0.570456in}}{\pgfqpoint{4.474166in}{0.566066in}}{\pgfqpoint{4.466352in}{0.558252in}}%
\pgfpathcurveto{\pgfqpoint{4.458538in}{0.550439in}}{\pgfqpoint{4.454148in}{0.539840in}}{\pgfqpoint{4.454148in}{0.528790in}}%
\pgfpathcurveto{\pgfqpoint{4.454148in}{0.517740in}}{\pgfqpoint{4.458538in}{0.507140in}}{\pgfqpoint{4.466352in}{0.499327in}}%
\pgfpathcurveto{\pgfqpoint{4.474166in}{0.491513in}}{\pgfqpoint{4.484765in}{0.487123in}}{\pgfqpoint{4.495815in}{0.487123in}}%
\pgfpathlineto{\pgfqpoint{4.495815in}{0.487123in}}%
\pgfpathclose%
\pgfusepath{stroke}%
\end{pgfscope}%
\begin{pgfscope}%
\pgfpathrectangle{\pgfqpoint{0.494722in}{0.437222in}}{\pgfqpoint{6.275590in}{5.159444in}}%
\pgfusepath{clip}%
\pgfsetbuttcap%
\pgfsetroundjoin%
\pgfsetlinewidth{1.003750pt}%
\definecolor{currentstroke}{rgb}{0.827451,0.827451,0.827451}%
\pgfsetstrokecolor{currentstroke}%
\pgfsetstrokeopacity{0.800000}%
\pgfsetdash{}{0pt}%
\pgfpathmoveto{\pgfqpoint{1.930688in}{1.539098in}}%
\pgfpathcurveto{\pgfqpoint{1.941738in}{1.539098in}}{\pgfqpoint{1.952337in}{1.543488in}}{\pgfqpoint{1.960151in}{1.551302in}}%
\pgfpathcurveto{\pgfqpoint{1.967965in}{1.559115in}}{\pgfqpoint{1.972355in}{1.569714in}}{\pgfqpoint{1.972355in}{1.580765in}}%
\pgfpathcurveto{\pgfqpoint{1.972355in}{1.591815in}}{\pgfqpoint{1.967965in}{1.602414in}}{\pgfqpoint{1.960151in}{1.610227in}}%
\pgfpathcurveto{\pgfqpoint{1.952337in}{1.618041in}}{\pgfqpoint{1.941738in}{1.622431in}}{\pgfqpoint{1.930688in}{1.622431in}}%
\pgfpathcurveto{\pgfqpoint{1.919638in}{1.622431in}}{\pgfqpoint{1.909039in}{1.618041in}}{\pgfqpoint{1.901225in}{1.610227in}}%
\pgfpathcurveto{\pgfqpoint{1.893412in}{1.602414in}}{\pgfqpoint{1.889021in}{1.591815in}}{\pgfqpoint{1.889021in}{1.580765in}}%
\pgfpathcurveto{\pgfqpoint{1.889021in}{1.569714in}}{\pgfqpoint{1.893412in}{1.559115in}}{\pgfqpoint{1.901225in}{1.551302in}}%
\pgfpathcurveto{\pgfqpoint{1.909039in}{1.543488in}}{\pgfqpoint{1.919638in}{1.539098in}}{\pgfqpoint{1.930688in}{1.539098in}}%
\pgfpathlineto{\pgfqpoint{1.930688in}{1.539098in}}%
\pgfpathclose%
\pgfusepath{stroke}%
\end{pgfscope}%
\begin{pgfscope}%
\pgfpathrectangle{\pgfqpoint{0.494722in}{0.437222in}}{\pgfqpoint{6.275590in}{5.159444in}}%
\pgfusepath{clip}%
\pgfsetbuttcap%
\pgfsetroundjoin%
\pgfsetlinewidth{1.003750pt}%
\definecolor{currentstroke}{rgb}{0.827451,0.827451,0.827451}%
\pgfsetstrokecolor{currentstroke}%
\pgfsetstrokeopacity{0.800000}%
\pgfsetdash{}{0pt}%
\pgfpathmoveto{\pgfqpoint{0.498024in}{4.460098in}}%
\pgfpathcurveto{\pgfqpoint{0.509074in}{4.460098in}}{\pgfqpoint{0.519673in}{4.464488in}}{\pgfqpoint{0.527487in}{4.472302in}}%
\pgfpathcurveto{\pgfqpoint{0.535301in}{4.480116in}}{\pgfqpoint{0.539691in}{4.490715in}}{\pgfqpoint{0.539691in}{4.501765in}}%
\pgfpathcurveto{\pgfqpoint{0.539691in}{4.512815in}}{\pgfqpoint{0.535301in}{4.523414in}}{\pgfqpoint{0.527487in}{4.531228in}}%
\pgfpathcurveto{\pgfqpoint{0.519673in}{4.539041in}}{\pgfqpoint{0.509074in}{4.543432in}}{\pgfqpoint{0.498024in}{4.543432in}}%
\pgfpathcurveto{\pgfqpoint{0.486974in}{4.543432in}}{\pgfqpoint{0.476375in}{4.539041in}}{\pgfqpoint{0.468561in}{4.531228in}}%
\pgfpathcurveto{\pgfqpoint{0.460748in}{4.523414in}}{\pgfqpoint{0.456358in}{4.512815in}}{\pgfqpoint{0.456358in}{4.501765in}}%
\pgfpathcurveto{\pgfqpoint{0.456358in}{4.490715in}}{\pgfqpoint{0.460748in}{4.480116in}}{\pgfqpoint{0.468561in}{4.472302in}}%
\pgfpathcurveto{\pgfqpoint{0.476375in}{4.464488in}}{\pgfqpoint{0.486974in}{4.460098in}}{\pgfqpoint{0.498024in}{4.460098in}}%
\pgfpathlineto{\pgfqpoint{0.498024in}{4.460098in}}%
\pgfpathclose%
\pgfusepath{stroke}%
\end{pgfscope}%
\begin{pgfscope}%
\pgfpathrectangle{\pgfqpoint{0.494722in}{0.437222in}}{\pgfqpoint{6.275590in}{5.159444in}}%
\pgfusepath{clip}%
\pgfsetbuttcap%
\pgfsetroundjoin%
\pgfsetlinewidth{1.003750pt}%
\definecolor{currentstroke}{rgb}{0.827451,0.827451,0.827451}%
\pgfsetstrokecolor{currentstroke}%
\pgfsetstrokeopacity{0.800000}%
\pgfsetdash{}{0pt}%
\pgfpathmoveto{\pgfqpoint{0.498838in}{4.458588in}}%
\pgfpathcurveto{\pgfqpoint{0.509888in}{4.458588in}}{\pgfqpoint{0.520487in}{4.462978in}}{\pgfqpoint{0.528301in}{4.470792in}}%
\pgfpathcurveto{\pgfqpoint{0.536115in}{4.478605in}}{\pgfqpoint{0.540505in}{4.489204in}}{\pgfqpoint{0.540505in}{4.500255in}}%
\pgfpathcurveto{\pgfqpoint{0.540505in}{4.511305in}}{\pgfqpoint{0.536115in}{4.521904in}}{\pgfqpoint{0.528301in}{4.529717in}}%
\pgfpathcurveto{\pgfqpoint{0.520487in}{4.537531in}}{\pgfqpoint{0.509888in}{4.541921in}}{\pgfqpoint{0.498838in}{4.541921in}}%
\pgfpathcurveto{\pgfqpoint{0.487788in}{4.541921in}}{\pgfqpoint{0.477189in}{4.537531in}}{\pgfqpoint{0.469375in}{4.529717in}}%
\pgfpathcurveto{\pgfqpoint{0.461562in}{4.521904in}}{\pgfqpoint{0.457172in}{4.511305in}}{\pgfqpoint{0.457172in}{4.500255in}}%
\pgfpathcurveto{\pgfqpoint{0.457172in}{4.489204in}}{\pgfqpoint{0.461562in}{4.478605in}}{\pgfqpoint{0.469375in}{4.470792in}}%
\pgfpathcurveto{\pgfqpoint{0.477189in}{4.462978in}}{\pgfqpoint{0.487788in}{4.458588in}}{\pgfqpoint{0.498838in}{4.458588in}}%
\pgfpathlineto{\pgfqpoint{0.498838in}{4.458588in}}%
\pgfpathclose%
\pgfusepath{stroke}%
\end{pgfscope}%
\begin{pgfscope}%
\pgfpathrectangle{\pgfqpoint{0.494722in}{0.437222in}}{\pgfqpoint{6.275590in}{5.159444in}}%
\pgfusepath{clip}%
\pgfsetbuttcap%
\pgfsetroundjoin%
\pgfsetlinewidth{1.003750pt}%
\definecolor{currentstroke}{rgb}{0.827451,0.827451,0.827451}%
\pgfsetstrokecolor{currentstroke}%
\pgfsetstrokeopacity{0.800000}%
\pgfsetdash{}{0pt}%
\pgfpathmoveto{\pgfqpoint{4.118796in}{0.541521in}}%
\pgfpathcurveto{\pgfqpoint{4.129846in}{0.541521in}}{\pgfqpoint{4.140445in}{0.545912in}}{\pgfqpoint{4.148259in}{0.553725in}}%
\pgfpathcurveto{\pgfqpoint{4.156072in}{0.561539in}}{\pgfqpoint{4.160463in}{0.572138in}}{\pgfqpoint{4.160463in}{0.583188in}}%
\pgfpathcurveto{\pgfqpoint{4.160463in}{0.594238in}}{\pgfqpoint{4.156072in}{0.604837in}}{\pgfqpoint{4.148259in}{0.612651in}}%
\pgfpathcurveto{\pgfqpoint{4.140445in}{0.620464in}}{\pgfqpoint{4.129846in}{0.624855in}}{\pgfqpoint{4.118796in}{0.624855in}}%
\pgfpathcurveto{\pgfqpoint{4.107746in}{0.624855in}}{\pgfqpoint{4.097147in}{0.620464in}}{\pgfqpoint{4.089333in}{0.612651in}}%
\pgfpathcurveto{\pgfqpoint{4.081520in}{0.604837in}}{\pgfqpoint{4.077129in}{0.594238in}}{\pgfqpoint{4.077129in}{0.583188in}}%
\pgfpathcurveto{\pgfqpoint{4.077129in}{0.572138in}}{\pgfqpoint{4.081520in}{0.561539in}}{\pgfqpoint{4.089333in}{0.553725in}}%
\pgfpathcurveto{\pgfqpoint{4.097147in}{0.545912in}}{\pgfqpoint{4.107746in}{0.541521in}}{\pgfqpoint{4.118796in}{0.541521in}}%
\pgfpathlineto{\pgfqpoint{4.118796in}{0.541521in}}%
\pgfpathclose%
\pgfusepath{stroke}%
\end{pgfscope}%
\begin{pgfscope}%
\pgfpathrectangle{\pgfqpoint{0.494722in}{0.437222in}}{\pgfqpoint{6.275590in}{5.159444in}}%
\pgfusepath{clip}%
\pgfsetbuttcap%
\pgfsetroundjoin%
\pgfsetlinewidth{1.003750pt}%
\definecolor{currentstroke}{rgb}{1.000000,0.000000,0.000000}%
\pgfsetstrokecolor{currentstroke}%
\pgfsetdash{}{0pt}%
\pgfpathmoveto{\pgfqpoint{0.494722in}{4.633671in}}%
\pgfpathcurveto{\pgfqpoint{0.505772in}{4.633671in}}{\pgfqpoint{0.516371in}{4.638061in}}{\pgfqpoint{0.524185in}{4.645875in}}%
\pgfpathcurveto{\pgfqpoint{0.531999in}{4.653688in}}{\pgfqpoint{0.536389in}{4.664287in}}{\pgfqpoint{0.536389in}{4.675337in}}%
\pgfpathcurveto{\pgfqpoint{0.536389in}{4.686387in}}{\pgfqpoint{0.531999in}{4.696986in}}{\pgfqpoint{0.524185in}{4.704800in}}%
\pgfpathcurveto{\pgfqpoint{0.516371in}{4.712614in}}{\pgfqpoint{0.505772in}{4.717004in}}{\pgfqpoint{0.494722in}{4.717004in}}%
\pgfpathcurveto{\pgfqpoint{0.483672in}{4.717004in}}{\pgfqpoint{0.473073in}{4.712614in}}{\pgfqpoint{0.465259in}{4.704800in}}%
\pgfpathcurveto{\pgfqpoint{0.457446in}{4.696986in}}{\pgfqpoint{0.453056in}{4.686387in}}{\pgfqpoint{0.453056in}{4.675337in}}%
\pgfpathcurveto{\pgfqpoint{0.453056in}{4.664287in}}{\pgfqpoint{0.457446in}{4.653688in}}{\pgfqpoint{0.465259in}{4.645875in}}%
\pgfpathcurveto{\pgfqpoint{0.473073in}{4.638061in}}{\pgfqpoint{0.483672in}{4.633671in}}{\pgfqpoint{0.494722in}{4.633671in}}%
\pgfpathlineto{\pgfqpoint{0.494722in}{4.633671in}}%
\pgfpathclose%
\pgfusepath{stroke}%
\end{pgfscope}%
\begin{pgfscope}%
\pgfpathrectangle{\pgfqpoint{0.494722in}{0.437222in}}{\pgfqpoint{6.275590in}{5.159444in}}%
\pgfusepath{clip}%
\pgfsetbuttcap%
\pgfsetroundjoin%
\pgfsetlinewidth{1.003750pt}%
\definecolor{currentstroke}{rgb}{1.000000,0.000000,0.000000}%
\pgfsetstrokecolor{currentstroke}%
\pgfsetdash{}{0pt}%
\pgfpathmoveto{\pgfqpoint{5.828974in}{0.395556in}}%
\pgfpathcurveto{\pgfqpoint{5.840024in}{0.395556in}}{\pgfqpoint{5.850623in}{0.399946in}}{\pgfqpoint{5.858437in}{0.407759in}}%
\pgfpathcurveto{\pgfqpoint{5.866250in}{0.415573in}}{\pgfqpoint{5.870641in}{0.426172in}}{\pgfqpoint{5.870641in}{0.437222in}}%
\pgfpathcurveto{\pgfqpoint{5.870641in}{0.448272in}}{\pgfqpoint{5.866250in}{0.458871in}}{\pgfqpoint{5.858437in}{0.466685in}}%
\pgfpathcurveto{\pgfqpoint{5.850623in}{0.474499in}}{\pgfqpoint{5.840024in}{0.478889in}}{\pgfqpoint{5.828974in}{0.478889in}}%
\pgfpathcurveto{\pgfqpoint{5.817924in}{0.478889in}}{\pgfqpoint{5.807325in}{0.474499in}}{\pgfqpoint{5.799511in}{0.466685in}}%
\pgfpathcurveto{\pgfqpoint{5.791698in}{0.458871in}}{\pgfqpoint{5.787307in}{0.448272in}}{\pgfqpoint{5.787307in}{0.437222in}}%
\pgfpathcurveto{\pgfqpoint{5.787307in}{0.426172in}}{\pgfqpoint{5.791698in}{0.415573in}}{\pgfqpoint{5.799511in}{0.407759in}}%
\pgfpathcurveto{\pgfqpoint{5.807325in}{0.399946in}}{\pgfqpoint{5.817924in}{0.395556in}}{\pgfqpoint{5.828974in}{0.395556in}}%
\pgfusepath{stroke}%
\end{pgfscope}%
\begin{pgfscope}%
\pgfpathrectangle{\pgfqpoint{0.494722in}{0.437222in}}{\pgfqpoint{6.275590in}{5.159444in}}%
\pgfusepath{clip}%
\pgfsetbuttcap%
\pgfsetroundjoin%
\pgfsetlinewidth{1.003750pt}%
\definecolor{currentstroke}{rgb}{1.000000,0.000000,0.000000}%
\pgfsetstrokecolor{currentstroke}%
\pgfsetdash{}{0pt}%
\pgfpathmoveto{\pgfqpoint{2.798304in}{1.011370in}}%
\pgfpathcurveto{\pgfqpoint{2.809354in}{1.011370in}}{\pgfqpoint{2.819954in}{1.015760in}}{\pgfqpoint{2.827767in}{1.023574in}}%
\pgfpathcurveto{\pgfqpoint{2.835581in}{1.031387in}}{\pgfqpoint{2.839971in}{1.041986in}}{\pgfqpoint{2.839971in}{1.053036in}}%
\pgfpathcurveto{\pgfqpoint{2.839971in}{1.064087in}}{\pgfqpoint{2.835581in}{1.074686in}}{\pgfqpoint{2.827767in}{1.082499in}}%
\pgfpathcurveto{\pgfqpoint{2.819954in}{1.090313in}}{\pgfqpoint{2.809354in}{1.094703in}}{\pgfqpoint{2.798304in}{1.094703in}}%
\pgfpathcurveto{\pgfqpoint{2.787254in}{1.094703in}}{\pgfqpoint{2.776655in}{1.090313in}}{\pgfqpoint{2.768842in}{1.082499in}}%
\pgfpathcurveto{\pgfqpoint{2.761028in}{1.074686in}}{\pgfqpoint{2.756638in}{1.064087in}}{\pgfqpoint{2.756638in}{1.053036in}}%
\pgfpathcurveto{\pgfqpoint{2.756638in}{1.041986in}}{\pgfqpoint{2.761028in}{1.031387in}}{\pgfqpoint{2.768842in}{1.023574in}}%
\pgfpathcurveto{\pgfqpoint{2.776655in}{1.015760in}}{\pgfqpoint{2.787254in}{1.011370in}}{\pgfqpoint{2.798304in}{1.011370in}}%
\pgfpathlineto{\pgfqpoint{2.798304in}{1.011370in}}%
\pgfpathclose%
\pgfusepath{stroke}%
\end{pgfscope}%
\begin{pgfscope}%
\pgfpathrectangle{\pgfqpoint{0.494722in}{0.437222in}}{\pgfqpoint{6.275590in}{5.159444in}}%
\pgfusepath{clip}%
\pgfsetbuttcap%
\pgfsetroundjoin%
\pgfsetlinewidth{1.003750pt}%
\definecolor{currentstroke}{rgb}{1.000000,0.000000,0.000000}%
\pgfsetstrokecolor{currentstroke}%
\pgfsetdash{}{0pt}%
\pgfpathmoveto{\pgfqpoint{0.522614in}{4.105234in}}%
\pgfpathcurveto{\pgfqpoint{0.533665in}{4.105234in}}{\pgfqpoint{0.544264in}{4.109625in}}{\pgfqpoint{0.552077in}{4.117438in}}%
\pgfpathcurveto{\pgfqpoint{0.559891in}{4.125252in}}{\pgfqpoint{0.564281in}{4.135851in}}{\pgfqpoint{0.564281in}{4.146901in}}%
\pgfpathcurveto{\pgfqpoint{0.564281in}{4.157951in}}{\pgfqpoint{0.559891in}{4.168550in}}{\pgfqpoint{0.552077in}{4.176364in}}%
\pgfpathcurveto{\pgfqpoint{0.544264in}{4.184178in}}{\pgfqpoint{0.533665in}{4.188568in}}{\pgfqpoint{0.522614in}{4.188568in}}%
\pgfpathcurveto{\pgfqpoint{0.511564in}{4.188568in}}{\pgfqpoint{0.500965in}{4.184178in}}{\pgfqpoint{0.493152in}{4.176364in}}%
\pgfpathcurveto{\pgfqpoint{0.485338in}{4.168550in}}{\pgfqpoint{0.480948in}{4.157951in}}{\pgfqpoint{0.480948in}{4.146901in}}%
\pgfpathcurveto{\pgfqpoint{0.480948in}{4.135851in}}{\pgfqpoint{0.485338in}{4.125252in}}{\pgfqpoint{0.493152in}{4.117438in}}%
\pgfpathcurveto{\pgfqpoint{0.500965in}{4.109625in}}{\pgfqpoint{0.511564in}{4.105234in}}{\pgfqpoint{0.522614in}{4.105234in}}%
\pgfpathlineto{\pgfqpoint{0.522614in}{4.105234in}}%
\pgfpathclose%
\pgfusepath{stroke}%
\end{pgfscope}%
\begin{pgfscope}%
\pgfpathrectangle{\pgfqpoint{0.494722in}{0.437222in}}{\pgfqpoint{6.275590in}{5.159444in}}%
\pgfusepath{clip}%
\pgfsetbuttcap%
\pgfsetroundjoin%
\pgfsetlinewidth{1.003750pt}%
\definecolor{currentstroke}{rgb}{1.000000,0.000000,0.000000}%
\pgfsetstrokecolor{currentstroke}%
\pgfsetdash{}{0pt}%
\pgfpathmoveto{\pgfqpoint{4.802076in}{0.441235in}}%
\pgfpathcurveto{\pgfqpoint{4.813126in}{0.441235in}}{\pgfqpoint{4.823725in}{0.445625in}}{\pgfqpoint{4.831539in}{0.453439in}}%
\pgfpathcurveto{\pgfqpoint{4.839353in}{0.461252in}}{\pgfqpoint{4.843743in}{0.471852in}}{\pgfqpoint{4.843743in}{0.482902in}}%
\pgfpathcurveto{\pgfqpoint{4.843743in}{0.493952in}}{\pgfqpoint{4.839353in}{0.504551in}}{\pgfqpoint{4.831539in}{0.512364in}}%
\pgfpathcurveto{\pgfqpoint{4.823725in}{0.520178in}}{\pgfqpoint{4.813126in}{0.524568in}}{\pgfqpoint{4.802076in}{0.524568in}}%
\pgfpathcurveto{\pgfqpoint{4.791026in}{0.524568in}}{\pgfqpoint{4.780427in}{0.520178in}}{\pgfqpoint{4.772613in}{0.512364in}}%
\pgfpathcurveto{\pgfqpoint{4.764800in}{0.504551in}}{\pgfqpoint{4.760409in}{0.493952in}}{\pgfqpoint{4.760409in}{0.482902in}}%
\pgfpathcurveto{\pgfqpoint{4.760409in}{0.471852in}}{\pgfqpoint{4.764800in}{0.461252in}}{\pgfqpoint{4.772613in}{0.453439in}}%
\pgfpathcurveto{\pgfqpoint{4.780427in}{0.445625in}}{\pgfqpoint{4.791026in}{0.441235in}}{\pgfqpoint{4.802076in}{0.441235in}}%
\pgfpathlineto{\pgfqpoint{4.802076in}{0.441235in}}%
\pgfpathclose%
\pgfusepath{stroke}%
\end{pgfscope}%
\begin{pgfscope}%
\pgfpathrectangle{\pgfqpoint{0.494722in}{0.437222in}}{\pgfqpoint{6.275590in}{5.159444in}}%
\pgfusepath{clip}%
\pgfsetbuttcap%
\pgfsetroundjoin%
\pgfsetlinewidth{1.003750pt}%
\definecolor{currentstroke}{rgb}{1.000000,0.000000,0.000000}%
\pgfsetstrokecolor{currentstroke}%
\pgfsetdash{}{0pt}%
\pgfpathmoveto{\pgfqpoint{0.510746in}{4.232917in}}%
\pgfpathcurveto{\pgfqpoint{0.521796in}{4.232917in}}{\pgfqpoint{0.532395in}{4.237307in}}{\pgfqpoint{0.540209in}{4.245121in}}%
\pgfpathcurveto{\pgfqpoint{0.548022in}{4.252934in}}{\pgfqpoint{0.552413in}{4.263533in}}{\pgfqpoint{0.552413in}{4.274584in}}%
\pgfpathcurveto{\pgfqpoint{0.552413in}{4.285634in}}{\pgfqpoint{0.548022in}{4.296233in}}{\pgfqpoint{0.540209in}{4.304046in}}%
\pgfpathcurveto{\pgfqpoint{0.532395in}{4.311860in}}{\pgfqpoint{0.521796in}{4.316250in}}{\pgfqpoint{0.510746in}{4.316250in}}%
\pgfpathcurveto{\pgfqpoint{0.499696in}{4.316250in}}{\pgfqpoint{0.489097in}{4.311860in}}{\pgfqpoint{0.481283in}{4.304046in}}%
\pgfpathcurveto{\pgfqpoint{0.473470in}{4.296233in}}{\pgfqpoint{0.469079in}{4.285634in}}{\pgfqpoint{0.469079in}{4.274584in}}%
\pgfpathcurveto{\pgfqpoint{0.469079in}{4.263533in}}{\pgfqpoint{0.473470in}{4.252934in}}{\pgfqpoint{0.481283in}{4.245121in}}%
\pgfpathcurveto{\pgfqpoint{0.489097in}{4.237307in}}{\pgfqpoint{0.499696in}{4.232917in}}{\pgfqpoint{0.510746in}{4.232917in}}%
\pgfpathlineto{\pgfqpoint{0.510746in}{4.232917in}}%
\pgfpathclose%
\pgfusepath{stroke}%
\end{pgfscope}%
\begin{pgfscope}%
\pgfpathrectangle{\pgfqpoint{0.494722in}{0.437222in}}{\pgfqpoint{6.275590in}{5.159444in}}%
\pgfusepath{clip}%
\pgfsetbuttcap%
\pgfsetroundjoin%
\pgfsetlinewidth{1.003750pt}%
\definecolor{currentstroke}{rgb}{1.000000,0.000000,0.000000}%
\pgfsetstrokecolor{currentstroke}%
\pgfsetdash{}{0pt}%
\pgfpathmoveto{\pgfqpoint{2.692712in}{1.052222in}}%
\pgfpathcurveto{\pgfqpoint{2.703762in}{1.052222in}}{\pgfqpoint{2.714361in}{1.056612in}}{\pgfqpoint{2.722175in}{1.064425in}}%
\pgfpathcurveto{\pgfqpoint{2.729989in}{1.072239in}}{\pgfqpoint{2.734379in}{1.082838in}}{\pgfqpoint{2.734379in}{1.093888in}}%
\pgfpathcurveto{\pgfqpoint{2.734379in}{1.104938in}}{\pgfqpoint{2.729989in}{1.115537in}}{\pgfqpoint{2.722175in}{1.123351in}}%
\pgfpathcurveto{\pgfqpoint{2.714361in}{1.131165in}}{\pgfqpoint{2.703762in}{1.135555in}}{\pgfqpoint{2.692712in}{1.135555in}}%
\pgfpathcurveto{\pgfqpoint{2.681662in}{1.135555in}}{\pgfqpoint{2.671063in}{1.131165in}}{\pgfqpoint{2.663250in}{1.123351in}}%
\pgfpathcurveto{\pgfqpoint{2.655436in}{1.115537in}}{\pgfqpoint{2.651046in}{1.104938in}}{\pgfqpoint{2.651046in}{1.093888in}}%
\pgfpathcurveto{\pgfqpoint{2.651046in}{1.082838in}}{\pgfqpoint{2.655436in}{1.072239in}}{\pgfqpoint{2.663250in}{1.064425in}}%
\pgfpathcurveto{\pgfqpoint{2.671063in}{1.056612in}}{\pgfqpoint{2.681662in}{1.052222in}}{\pgfqpoint{2.692712in}{1.052222in}}%
\pgfpathlineto{\pgfqpoint{2.692712in}{1.052222in}}%
\pgfpathclose%
\pgfusepath{stroke}%
\end{pgfscope}%
\begin{pgfscope}%
\pgfpathrectangle{\pgfqpoint{0.494722in}{0.437222in}}{\pgfqpoint{6.275590in}{5.159444in}}%
\pgfusepath{clip}%
\pgfsetbuttcap%
\pgfsetroundjoin%
\pgfsetlinewidth{1.003750pt}%
\definecolor{currentstroke}{rgb}{1.000000,0.000000,0.000000}%
\pgfsetstrokecolor{currentstroke}%
\pgfsetdash{}{0pt}%
\pgfpathmoveto{\pgfqpoint{5.202487in}{0.411553in}}%
\pgfpathcurveto{\pgfqpoint{5.213537in}{0.411553in}}{\pgfqpoint{5.224136in}{0.415943in}}{\pgfqpoint{5.231949in}{0.423757in}}%
\pgfpathcurveto{\pgfqpoint{5.239763in}{0.431571in}}{\pgfqpoint{5.244153in}{0.442170in}}{\pgfqpoint{5.244153in}{0.453220in}}%
\pgfpathcurveto{\pgfqpoint{5.244153in}{0.464270in}}{\pgfqpoint{5.239763in}{0.474869in}}{\pgfqpoint{5.231949in}{0.482683in}}%
\pgfpathcurveto{\pgfqpoint{5.224136in}{0.490496in}}{\pgfqpoint{5.213537in}{0.494887in}}{\pgfqpoint{5.202487in}{0.494887in}}%
\pgfpathcurveto{\pgfqpoint{5.191436in}{0.494887in}}{\pgfqpoint{5.180837in}{0.490496in}}{\pgfqpoint{5.173024in}{0.482683in}}%
\pgfpathcurveto{\pgfqpoint{5.165210in}{0.474869in}}{\pgfqpoint{5.160820in}{0.464270in}}{\pgfqpoint{5.160820in}{0.453220in}}%
\pgfpathcurveto{\pgfqpoint{5.160820in}{0.442170in}}{\pgfqpoint{5.165210in}{0.431571in}}{\pgfqpoint{5.173024in}{0.423757in}}%
\pgfpathcurveto{\pgfqpoint{5.180837in}{0.415943in}}{\pgfqpoint{5.191436in}{0.411553in}}{\pgfqpoint{5.202487in}{0.411553in}}%
\pgfusepath{stroke}%
\end{pgfscope}%
\begin{pgfscope}%
\pgfpathrectangle{\pgfqpoint{0.494722in}{0.437222in}}{\pgfqpoint{6.275590in}{5.159444in}}%
\pgfusepath{clip}%
\pgfsetbuttcap%
\pgfsetroundjoin%
\pgfsetlinewidth{1.003750pt}%
\definecolor{currentstroke}{rgb}{1.000000,0.000000,0.000000}%
\pgfsetstrokecolor{currentstroke}%
\pgfsetdash{}{0pt}%
\pgfpathmoveto{\pgfqpoint{5.449066in}{0.406903in}}%
\pgfpathcurveto{\pgfqpoint{5.460116in}{0.406903in}}{\pgfqpoint{5.470715in}{0.411294in}}{\pgfqpoint{5.478529in}{0.419107in}}%
\pgfpathcurveto{\pgfqpoint{5.486342in}{0.426921in}}{\pgfqpoint{5.490733in}{0.437520in}}{\pgfqpoint{5.490733in}{0.448570in}}%
\pgfpathcurveto{\pgfqpoint{5.490733in}{0.459620in}}{\pgfqpoint{5.486342in}{0.470219in}}{\pgfqpoint{5.478529in}{0.478033in}}%
\pgfpathcurveto{\pgfqpoint{5.470715in}{0.485846in}}{\pgfqpoint{5.460116in}{0.490237in}}{\pgfqpoint{5.449066in}{0.490237in}}%
\pgfpathcurveto{\pgfqpoint{5.438016in}{0.490237in}}{\pgfqpoint{5.427417in}{0.485846in}}{\pgfqpoint{5.419603in}{0.478033in}}%
\pgfpathcurveto{\pgfqpoint{5.411790in}{0.470219in}}{\pgfqpoint{5.407399in}{0.459620in}}{\pgfqpoint{5.407399in}{0.448570in}}%
\pgfpathcurveto{\pgfqpoint{5.407399in}{0.437520in}}{\pgfqpoint{5.411790in}{0.426921in}}{\pgfqpoint{5.419603in}{0.419107in}}%
\pgfpathcurveto{\pgfqpoint{5.427417in}{0.411294in}}{\pgfqpoint{5.438016in}{0.406903in}}{\pgfqpoint{5.449066in}{0.406903in}}%
\pgfusepath{stroke}%
\end{pgfscope}%
\begin{pgfscope}%
\pgfpathrectangle{\pgfqpoint{0.494722in}{0.437222in}}{\pgfqpoint{6.275590in}{5.159444in}}%
\pgfusepath{clip}%
\pgfsetbuttcap%
\pgfsetroundjoin%
\pgfsetlinewidth{1.003750pt}%
\definecolor{currentstroke}{rgb}{1.000000,0.000000,0.000000}%
\pgfsetstrokecolor{currentstroke}%
\pgfsetdash{}{0pt}%
\pgfpathmoveto{\pgfqpoint{2.916039in}{0.941905in}}%
\pgfpathcurveto{\pgfqpoint{2.927089in}{0.941905in}}{\pgfqpoint{2.937688in}{0.946296in}}{\pgfqpoint{2.945502in}{0.954109in}}%
\pgfpathcurveto{\pgfqpoint{2.953315in}{0.961923in}}{\pgfqpoint{2.957706in}{0.972522in}}{\pgfqpoint{2.957706in}{0.983572in}}%
\pgfpathcurveto{\pgfqpoint{2.957706in}{0.994622in}}{\pgfqpoint{2.953315in}{1.005221in}}{\pgfqpoint{2.945502in}{1.013035in}}%
\pgfpathcurveto{\pgfqpoint{2.937688in}{1.020848in}}{\pgfqpoint{2.927089in}{1.025239in}}{\pgfqpoint{2.916039in}{1.025239in}}%
\pgfpathcurveto{\pgfqpoint{2.904989in}{1.025239in}}{\pgfqpoint{2.894390in}{1.020848in}}{\pgfqpoint{2.886576in}{1.013035in}}%
\pgfpathcurveto{\pgfqpoint{2.878763in}{1.005221in}}{\pgfqpoint{2.874372in}{0.994622in}}{\pgfqpoint{2.874372in}{0.983572in}}%
\pgfpathcurveto{\pgfqpoint{2.874372in}{0.972522in}}{\pgfqpoint{2.878763in}{0.961923in}}{\pgfqpoint{2.886576in}{0.954109in}}%
\pgfpathcurveto{\pgfqpoint{2.894390in}{0.946296in}}{\pgfqpoint{2.904989in}{0.941905in}}{\pgfqpoint{2.916039in}{0.941905in}}%
\pgfpathlineto{\pgfqpoint{2.916039in}{0.941905in}}%
\pgfpathclose%
\pgfusepath{stroke}%
\end{pgfscope}%
\begin{pgfscope}%
\pgfpathrectangle{\pgfqpoint{0.494722in}{0.437222in}}{\pgfqpoint{6.275590in}{5.159444in}}%
\pgfusepath{clip}%
\pgfsetbuttcap%
\pgfsetroundjoin%
\pgfsetlinewidth{1.003750pt}%
\definecolor{currentstroke}{rgb}{1.000000,0.000000,0.000000}%
\pgfsetstrokecolor{currentstroke}%
\pgfsetdash{}{0pt}%
\pgfpathmoveto{\pgfqpoint{0.607567in}{3.600655in}}%
\pgfpathcurveto{\pgfqpoint{0.618617in}{3.600655in}}{\pgfqpoint{0.629216in}{3.605045in}}{\pgfqpoint{0.637030in}{3.612859in}}%
\pgfpathcurveto{\pgfqpoint{0.644843in}{3.620672in}}{\pgfqpoint{0.649234in}{3.631271in}}{\pgfqpoint{0.649234in}{3.642322in}}%
\pgfpathcurveto{\pgfqpoint{0.649234in}{3.653372in}}{\pgfqpoint{0.644843in}{3.663971in}}{\pgfqpoint{0.637030in}{3.671784in}}%
\pgfpathcurveto{\pgfqpoint{0.629216in}{3.679598in}}{\pgfqpoint{0.618617in}{3.683988in}}{\pgfqpoint{0.607567in}{3.683988in}}%
\pgfpathcurveto{\pgfqpoint{0.596517in}{3.683988in}}{\pgfqpoint{0.585918in}{3.679598in}}{\pgfqpoint{0.578104in}{3.671784in}}%
\pgfpathcurveto{\pgfqpoint{0.570291in}{3.663971in}}{\pgfqpoint{0.565900in}{3.653372in}}{\pgfqpoint{0.565900in}{3.642322in}}%
\pgfpathcurveto{\pgfqpoint{0.565900in}{3.631271in}}{\pgfqpoint{0.570291in}{3.620672in}}{\pgfqpoint{0.578104in}{3.612859in}}%
\pgfpathcurveto{\pgfqpoint{0.585918in}{3.605045in}}{\pgfqpoint{0.596517in}{3.600655in}}{\pgfqpoint{0.607567in}{3.600655in}}%
\pgfpathlineto{\pgfqpoint{0.607567in}{3.600655in}}%
\pgfpathclose%
\pgfusepath{stroke}%
\end{pgfscope}%
\begin{pgfscope}%
\pgfpathrectangle{\pgfqpoint{0.494722in}{0.437222in}}{\pgfqpoint{6.275590in}{5.159444in}}%
\pgfusepath{clip}%
\pgfsetbuttcap%
\pgfsetroundjoin%
\pgfsetlinewidth{1.003750pt}%
\definecolor{currentstroke}{rgb}{1.000000,0.000000,0.000000}%
\pgfsetstrokecolor{currentstroke}%
\pgfsetdash{}{0pt}%
\pgfpathmoveto{\pgfqpoint{4.287083in}{0.508051in}}%
\pgfpathcurveto{\pgfqpoint{4.298133in}{0.508051in}}{\pgfqpoint{4.308732in}{0.512441in}}{\pgfqpoint{4.316546in}{0.520255in}}%
\pgfpathcurveto{\pgfqpoint{4.324360in}{0.528068in}}{\pgfqpoint{4.328750in}{0.538667in}}{\pgfqpoint{4.328750in}{0.549717in}}%
\pgfpathcurveto{\pgfqpoint{4.328750in}{0.560768in}}{\pgfqpoint{4.324360in}{0.571367in}}{\pgfqpoint{4.316546in}{0.579180in}}%
\pgfpathcurveto{\pgfqpoint{4.308732in}{0.586994in}}{\pgfqpoint{4.298133in}{0.591384in}}{\pgfqpoint{4.287083in}{0.591384in}}%
\pgfpathcurveto{\pgfqpoint{4.276033in}{0.591384in}}{\pgfqpoint{4.265434in}{0.586994in}}{\pgfqpoint{4.257620in}{0.579180in}}%
\pgfpathcurveto{\pgfqpoint{4.249807in}{0.571367in}}{\pgfqpoint{4.245416in}{0.560768in}}{\pgfqpoint{4.245416in}{0.549717in}}%
\pgfpathcurveto{\pgfqpoint{4.245416in}{0.538667in}}{\pgfqpoint{4.249807in}{0.528068in}}{\pgfqpoint{4.257620in}{0.520255in}}%
\pgfpathcurveto{\pgfqpoint{4.265434in}{0.512441in}}{\pgfqpoint{4.276033in}{0.508051in}}{\pgfqpoint{4.287083in}{0.508051in}}%
\pgfpathlineto{\pgfqpoint{4.287083in}{0.508051in}}%
\pgfpathclose%
\pgfusepath{stroke}%
\end{pgfscope}%
\begin{pgfscope}%
\pgfpathrectangle{\pgfqpoint{0.494722in}{0.437222in}}{\pgfqpoint{6.275590in}{5.159444in}}%
\pgfusepath{clip}%
\pgfsetbuttcap%
\pgfsetroundjoin%
\pgfsetlinewidth{1.003750pt}%
\definecolor{currentstroke}{rgb}{1.000000,0.000000,0.000000}%
\pgfsetstrokecolor{currentstroke}%
\pgfsetdash{}{0pt}%
\pgfpathmoveto{\pgfqpoint{0.916190in}{2.754470in}}%
\pgfpathcurveto{\pgfqpoint{0.927241in}{2.754470in}}{\pgfqpoint{0.937840in}{2.758860in}}{\pgfqpoint{0.945653in}{2.766674in}}%
\pgfpathcurveto{\pgfqpoint{0.953467in}{2.774488in}}{\pgfqpoint{0.957857in}{2.785087in}}{\pgfqpoint{0.957857in}{2.796137in}}%
\pgfpathcurveto{\pgfqpoint{0.957857in}{2.807187in}}{\pgfqpoint{0.953467in}{2.817786in}}{\pgfqpoint{0.945653in}{2.825600in}}%
\pgfpathcurveto{\pgfqpoint{0.937840in}{2.833413in}}{\pgfqpoint{0.927241in}{2.837804in}}{\pgfqpoint{0.916190in}{2.837804in}}%
\pgfpathcurveto{\pgfqpoint{0.905140in}{2.837804in}}{\pgfqpoint{0.894541in}{2.833413in}}{\pgfqpoint{0.886728in}{2.825600in}}%
\pgfpathcurveto{\pgfqpoint{0.878914in}{2.817786in}}{\pgfqpoint{0.874524in}{2.807187in}}{\pgfqpoint{0.874524in}{2.796137in}}%
\pgfpathcurveto{\pgfqpoint{0.874524in}{2.785087in}}{\pgfqpoint{0.878914in}{2.774488in}}{\pgfqpoint{0.886728in}{2.766674in}}%
\pgfpathcurveto{\pgfqpoint{0.894541in}{2.758860in}}{\pgfqpoint{0.905140in}{2.754470in}}{\pgfqpoint{0.916190in}{2.754470in}}%
\pgfpathlineto{\pgfqpoint{0.916190in}{2.754470in}}%
\pgfpathclose%
\pgfusepath{stroke}%
\end{pgfscope}%
\begin{pgfscope}%
\pgfpathrectangle{\pgfqpoint{0.494722in}{0.437222in}}{\pgfqpoint{6.275590in}{5.159444in}}%
\pgfusepath{clip}%
\pgfsetbuttcap%
\pgfsetroundjoin%
\pgfsetlinewidth{1.003750pt}%
\definecolor{currentstroke}{rgb}{1.000000,0.000000,0.000000}%
\pgfsetstrokecolor{currentstroke}%
\pgfsetdash{}{0pt}%
\pgfpathmoveto{\pgfqpoint{3.401226in}{0.733700in}}%
\pgfpathcurveto{\pgfqpoint{3.412276in}{0.733700in}}{\pgfqpoint{3.422875in}{0.738090in}}{\pgfqpoint{3.430689in}{0.745904in}}%
\pgfpathcurveto{\pgfqpoint{3.438502in}{0.753718in}}{\pgfqpoint{3.442893in}{0.764317in}}{\pgfqpoint{3.442893in}{0.775367in}}%
\pgfpathcurveto{\pgfqpoint{3.442893in}{0.786417in}}{\pgfqpoint{3.438502in}{0.797016in}}{\pgfqpoint{3.430689in}{0.804830in}}%
\pgfpathcurveto{\pgfqpoint{3.422875in}{0.812643in}}{\pgfqpoint{3.412276in}{0.817034in}}{\pgfqpoint{3.401226in}{0.817034in}}%
\pgfpathcurveto{\pgfqpoint{3.390176in}{0.817034in}}{\pgfqpoint{3.379577in}{0.812643in}}{\pgfqpoint{3.371763in}{0.804830in}}%
\pgfpathcurveto{\pgfqpoint{3.363949in}{0.797016in}}{\pgfqpoint{3.359559in}{0.786417in}}{\pgfqpoint{3.359559in}{0.775367in}}%
\pgfpathcurveto{\pgfqpoint{3.359559in}{0.764317in}}{\pgfqpoint{3.363949in}{0.753718in}}{\pgfqpoint{3.371763in}{0.745904in}}%
\pgfpathcurveto{\pgfqpoint{3.379577in}{0.738090in}}{\pgfqpoint{3.390176in}{0.733700in}}{\pgfqpoint{3.401226in}{0.733700in}}%
\pgfpathlineto{\pgfqpoint{3.401226in}{0.733700in}}%
\pgfpathclose%
\pgfusepath{stroke}%
\end{pgfscope}%
\begin{pgfscope}%
\pgfpathrectangle{\pgfqpoint{0.494722in}{0.437222in}}{\pgfqpoint{6.275590in}{5.159444in}}%
\pgfusepath{clip}%
\pgfsetbuttcap%
\pgfsetroundjoin%
\pgfsetlinewidth{1.003750pt}%
\definecolor{currentstroke}{rgb}{1.000000,0.000000,0.000000}%
\pgfsetstrokecolor{currentstroke}%
\pgfsetdash{}{0pt}%
\pgfpathmoveto{\pgfqpoint{2.027218in}{1.469417in}}%
\pgfpathcurveto{\pgfqpoint{2.038268in}{1.469417in}}{\pgfqpoint{2.048867in}{1.473807in}}{\pgfqpoint{2.056680in}{1.481621in}}%
\pgfpathcurveto{\pgfqpoint{2.064494in}{1.489434in}}{\pgfqpoint{2.068884in}{1.500033in}}{\pgfqpoint{2.068884in}{1.511083in}}%
\pgfpathcurveto{\pgfqpoint{2.068884in}{1.522134in}}{\pgfqpoint{2.064494in}{1.532733in}}{\pgfqpoint{2.056680in}{1.540546in}}%
\pgfpathcurveto{\pgfqpoint{2.048867in}{1.548360in}}{\pgfqpoint{2.038268in}{1.552750in}}{\pgfqpoint{2.027218in}{1.552750in}}%
\pgfpathcurveto{\pgfqpoint{2.016167in}{1.552750in}}{\pgfqpoint{2.005568in}{1.548360in}}{\pgfqpoint{1.997755in}{1.540546in}}%
\pgfpathcurveto{\pgfqpoint{1.989941in}{1.532733in}}{\pgfqpoint{1.985551in}{1.522134in}}{\pgfqpoint{1.985551in}{1.511083in}}%
\pgfpathcurveto{\pgfqpoint{1.985551in}{1.500033in}}{\pgfqpoint{1.989941in}{1.489434in}}{\pgfqpoint{1.997755in}{1.481621in}}%
\pgfpathcurveto{\pgfqpoint{2.005568in}{1.473807in}}{\pgfqpoint{2.016167in}{1.469417in}}{\pgfqpoint{2.027218in}{1.469417in}}%
\pgfpathlineto{\pgfqpoint{2.027218in}{1.469417in}}%
\pgfpathclose%
\pgfusepath{stroke}%
\end{pgfscope}%
\begin{pgfscope}%
\pgfpathrectangle{\pgfqpoint{0.494722in}{0.437222in}}{\pgfqpoint{6.275590in}{5.159444in}}%
\pgfusepath{clip}%
\pgfsetbuttcap%
\pgfsetroundjoin%
\pgfsetlinewidth{1.003750pt}%
\definecolor{currentstroke}{rgb}{1.000000,0.000000,0.000000}%
\pgfsetstrokecolor{currentstroke}%
\pgfsetdash{}{0pt}%
\pgfpathmoveto{\pgfqpoint{0.636078in}{3.479929in}}%
\pgfpathcurveto{\pgfqpoint{0.647128in}{3.479929in}}{\pgfqpoint{0.657727in}{3.484320in}}{\pgfqpoint{0.665541in}{3.492133in}}%
\pgfpathcurveto{\pgfqpoint{0.673354in}{3.499947in}}{\pgfqpoint{0.677745in}{3.510546in}}{\pgfqpoint{0.677745in}{3.521596in}}%
\pgfpathcurveto{\pgfqpoint{0.677745in}{3.532646in}}{\pgfqpoint{0.673354in}{3.543245in}}{\pgfqpoint{0.665541in}{3.551059in}}%
\pgfpathcurveto{\pgfqpoint{0.657727in}{3.558872in}}{\pgfqpoint{0.647128in}{3.563263in}}{\pgfqpoint{0.636078in}{3.563263in}}%
\pgfpathcurveto{\pgfqpoint{0.625028in}{3.563263in}}{\pgfqpoint{0.614429in}{3.558872in}}{\pgfqpoint{0.606615in}{3.551059in}}%
\pgfpathcurveto{\pgfqpoint{0.598802in}{3.543245in}}{\pgfqpoint{0.594411in}{3.532646in}}{\pgfqpoint{0.594411in}{3.521596in}}%
\pgfpathcurveto{\pgfqpoint{0.594411in}{3.510546in}}{\pgfqpoint{0.598802in}{3.499947in}}{\pgfqpoint{0.606615in}{3.492133in}}%
\pgfpathcurveto{\pgfqpoint{0.614429in}{3.484320in}}{\pgfqpoint{0.625028in}{3.479929in}}{\pgfqpoint{0.636078in}{3.479929in}}%
\pgfpathlineto{\pgfqpoint{0.636078in}{3.479929in}}%
\pgfpathclose%
\pgfusepath{stroke}%
\end{pgfscope}%
\begin{pgfscope}%
\pgfpathrectangle{\pgfqpoint{0.494722in}{0.437222in}}{\pgfqpoint{6.275590in}{5.159444in}}%
\pgfusepath{clip}%
\pgfsetbuttcap%
\pgfsetroundjoin%
\pgfsetlinewidth{1.003750pt}%
\definecolor{currentstroke}{rgb}{1.000000,0.000000,0.000000}%
\pgfsetstrokecolor{currentstroke}%
\pgfsetdash{}{0pt}%
\pgfpathmoveto{\pgfqpoint{1.505547in}{1.921986in}}%
\pgfpathcurveto{\pgfqpoint{1.516597in}{1.921986in}}{\pgfqpoint{1.527196in}{1.926377in}}{\pgfqpoint{1.535009in}{1.934190in}}%
\pgfpathcurveto{\pgfqpoint{1.542823in}{1.942004in}}{\pgfqpoint{1.547213in}{1.952603in}}{\pgfqpoint{1.547213in}{1.963653in}}%
\pgfpathcurveto{\pgfqpoint{1.547213in}{1.974703in}}{\pgfqpoint{1.542823in}{1.985302in}}{\pgfqpoint{1.535009in}{1.993116in}}%
\pgfpathcurveto{\pgfqpoint{1.527196in}{2.000930in}}{\pgfqpoint{1.516597in}{2.005320in}}{\pgfqpoint{1.505547in}{2.005320in}}%
\pgfpathcurveto{\pgfqpoint{1.494496in}{2.005320in}}{\pgfqpoint{1.483897in}{2.000930in}}{\pgfqpoint{1.476084in}{1.993116in}}%
\pgfpathcurveto{\pgfqpoint{1.468270in}{1.985302in}}{\pgfqpoint{1.463880in}{1.974703in}}{\pgfqpoint{1.463880in}{1.963653in}}%
\pgfpathcurveto{\pgfqpoint{1.463880in}{1.952603in}}{\pgfqpoint{1.468270in}{1.942004in}}{\pgfqpoint{1.476084in}{1.934190in}}%
\pgfpathcurveto{\pgfqpoint{1.483897in}{1.926377in}}{\pgfqpoint{1.494496in}{1.921986in}}{\pgfqpoint{1.505547in}{1.921986in}}%
\pgfpathlineto{\pgfqpoint{1.505547in}{1.921986in}}%
\pgfpathclose%
\pgfusepath{stroke}%
\end{pgfscope}%
\begin{pgfscope}%
\pgfpathrectangle{\pgfqpoint{0.494722in}{0.437222in}}{\pgfqpoint{6.275590in}{5.159444in}}%
\pgfusepath{clip}%
\pgfsetbuttcap%
\pgfsetroundjoin%
\pgfsetlinewidth{1.003750pt}%
\definecolor{currentstroke}{rgb}{1.000000,0.000000,0.000000}%
\pgfsetstrokecolor{currentstroke}%
\pgfsetdash{}{0pt}%
\pgfpathmoveto{\pgfqpoint{4.707854in}{0.450986in}}%
\pgfpathcurveto{\pgfqpoint{4.718905in}{0.450986in}}{\pgfqpoint{4.729504in}{0.455377in}}{\pgfqpoint{4.737317in}{0.463190in}}%
\pgfpathcurveto{\pgfqpoint{4.745131in}{0.471004in}}{\pgfqpoint{4.749521in}{0.481603in}}{\pgfqpoint{4.749521in}{0.492653in}}%
\pgfpathcurveto{\pgfqpoint{4.749521in}{0.503703in}}{\pgfqpoint{4.745131in}{0.514302in}}{\pgfqpoint{4.737317in}{0.522116in}}%
\pgfpathcurveto{\pgfqpoint{4.729504in}{0.529930in}}{\pgfqpoint{4.718905in}{0.534320in}}{\pgfqpoint{4.707854in}{0.534320in}}%
\pgfpathcurveto{\pgfqpoint{4.696804in}{0.534320in}}{\pgfqpoint{4.686205in}{0.529930in}}{\pgfqpoint{4.678392in}{0.522116in}}%
\pgfpathcurveto{\pgfqpoint{4.670578in}{0.514302in}}{\pgfqpoint{4.666188in}{0.503703in}}{\pgfqpoint{4.666188in}{0.492653in}}%
\pgfpathcurveto{\pgfqpoint{4.666188in}{0.481603in}}{\pgfqpoint{4.670578in}{0.471004in}}{\pgfqpoint{4.678392in}{0.463190in}}%
\pgfpathcurveto{\pgfqpoint{4.686205in}{0.455377in}}{\pgfqpoint{4.696804in}{0.450986in}}{\pgfqpoint{4.707854in}{0.450986in}}%
\pgfpathlineto{\pgfqpoint{4.707854in}{0.450986in}}%
\pgfpathclose%
\pgfusepath{stroke}%
\end{pgfscope}%
\begin{pgfscope}%
\pgfpathrectangle{\pgfqpoint{0.494722in}{0.437222in}}{\pgfqpoint{6.275590in}{5.159444in}}%
\pgfusepath{clip}%
\pgfsetbuttcap%
\pgfsetroundjoin%
\pgfsetlinewidth{1.003750pt}%
\definecolor{currentstroke}{rgb}{1.000000,0.000000,0.000000}%
\pgfsetstrokecolor{currentstroke}%
\pgfsetdash{}{0pt}%
\pgfpathmoveto{\pgfqpoint{2.399984in}{1.213482in}}%
\pgfpathcurveto{\pgfqpoint{2.411034in}{1.213482in}}{\pgfqpoint{2.421633in}{1.217872in}}{\pgfqpoint{2.429447in}{1.225686in}}%
\pgfpathcurveto{\pgfqpoint{2.437260in}{1.233499in}}{\pgfqpoint{2.441650in}{1.244099in}}{\pgfqpoint{2.441650in}{1.255149in}}%
\pgfpathcurveto{\pgfqpoint{2.441650in}{1.266199in}}{\pgfqpoint{2.437260in}{1.276798in}}{\pgfqpoint{2.429447in}{1.284611in}}%
\pgfpathcurveto{\pgfqpoint{2.421633in}{1.292425in}}{\pgfqpoint{2.411034in}{1.296815in}}{\pgfqpoint{2.399984in}{1.296815in}}%
\pgfpathcurveto{\pgfqpoint{2.388934in}{1.296815in}}{\pgfqpoint{2.378335in}{1.292425in}}{\pgfqpoint{2.370521in}{1.284611in}}%
\pgfpathcurveto{\pgfqpoint{2.362707in}{1.276798in}}{\pgfqpoint{2.358317in}{1.266199in}}{\pgfqpoint{2.358317in}{1.255149in}}%
\pgfpathcurveto{\pgfqpoint{2.358317in}{1.244099in}}{\pgfqpoint{2.362707in}{1.233499in}}{\pgfqpoint{2.370521in}{1.225686in}}%
\pgfpathcurveto{\pgfqpoint{2.378335in}{1.217872in}}{\pgfqpoint{2.388934in}{1.213482in}}{\pgfqpoint{2.399984in}{1.213482in}}%
\pgfpathlineto{\pgfqpoint{2.399984in}{1.213482in}}%
\pgfpathclose%
\pgfusepath{stroke}%
\end{pgfscope}%
\begin{pgfscope}%
\pgfpathrectangle{\pgfqpoint{0.494722in}{0.437222in}}{\pgfqpoint{6.275590in}{5.159444in}}%
\pgfusepath{clip}%
\pgfsetbuttcap%
\pgfsetroundjoin%
\pgfsetlinewidth{1.003750pt}%
\definecolor{currentstroke}{rgb}{1.000000,0.000000,0.000000}%
\pgfsetstrokecolor{currentstroke}%
\pgfsetdash{}{0pt}%
\pgfpathmoveto{\pgfqpoint{5.605037in}{0.397536in}}%
\pgfpathcurveto{\pgfqpoint{5.616087in}{0.397536in}}{\pgfqpoint{5.626686in}{0.401926in}}{\pgfqpoint{5.634500in}{0.409740in}}%
\pgfpathcurveto{\pgfqpoint{5.642313in}{0.417553in}}{\pgfqpoint{5.646704in}{0.428152in}}{\pgfqpoint{5.646704in}{0.439202in}}%
\pgfpathcurveto{\pgfqpoint{5.646704in}{0.450252in}}{\pgfqpoint{5.642313in}{0.460852in}}{\pgfqpoint{5.634500in}{0.468665in}}%
\pgfpathcurveto{\pgfqpoint{5.626686in}{0.476479in}}{\pgfqpoint{5.616087in}{0.480869in}}{\pgfqpoint{5.605037in}{0.480869in}}%
\pgfpathcurveto{\pgfqpoint{5.593987in}{0.480869in}}{\pgfqpoint{5.583388in}{0.476479in}}{\pgfqpoint{5.575574in}{0.468665in}}%
\pgfpathcurveto{\pgfqpoint{5.567761in}{0.460852in}}{\pgfqpoint{5.563370in}{0.450252in}}{\pgfqpoint{5.563370in}{0.439202in}}%
\pgfpathcurveto{\pgfqpoint{5.563370in}{0.428152in}}{\pgfqpoint{5.567761in}{0.417553in}}{\pgfqpoint{5.575574in}{0.409740in}}%
\pgfpathcurveto{\pgfqpoint{5.583388in}{0.401926in}}{\pgfqpoint{5.593987in}{0.397536in}}{\pgfqpoint{5.605037in}{0.397536in}}%
\pgfusepath{stroke}%
\end{pgfscope}%
\begin{pgfscope}%
\pgfpathrectangle{\pgfqpoint{0.494722in}{0.437222in}}{\pgfqpoint{6.275590in}{5.159444in}}%
\pgfusepath{clip}%
\pgfsetbuttcap%
\pgfsetroundjoin%
\pgfsetlinewidth{1.003750pt}%
\definecolor{currentstroke}{rgb}{1.000000,0.000000,0.000000}%
\pgfsetstrokecolor{currentstroke}%
\pgfsetdash{}{0pt}%
\pgfpathmoveto{\pgfqpoint{0.533036in}{4.013681in}}%
\pgfpathcurveto{\pgfqpoint{0.544086in}{4.013681in}}{\pgfqpoint{0.554685in}{4.018071in}}{\pgfqpoint{0.562499in}{4.025885in}}%
\pgfpathcurveto{\pgfqpoint{0.570312in}{4.033699in}}{\pgfqpoint{0.574702in}{4.044298in}}{\pgfqpoint{0.574702in}{4.055348in}}%
\pgfpathcurveto{\pgfqpoint{0.574702in}{4.066398in}}{\pgfqpoint{0.570312in}{4.076997in}}{\pgfqpoint{0.562499in}{4.084811in}}%
\pgfpathcurveto{\pgfqpoint{0.554685in}{4.092624in}}{\pgfqpoint{0.544086in}{4.097015in}}{\pgfqpoint{0.533036in}{4.097015in}}%
\pgfpathcurveto{\pgfqpoint{0.521986in}{4.097015in}}{\pgfqpoint{0.511387in}{4.092624in}}{\pgfqpoint{0.503573in}{4.084811in}}%
\pgfpathcurveto{\pgfqpoint{0.495759in}{4.076997in}}{\pgfqpoint{0.491369in}{4.066398in}}{\pgfqpoint{0.491369in}{4.055348in}}%
\pgfpathcurveto{\pgfqpoint{0.491369in}{4.044298in}}{\pgfqpoint{0.495759in}{4.033699in}}{\pgfqpoint{0.503573in}{4.025885in}}%
\pgfpathcurveto{\pgfqpoint{0.511387in}{4.018071in}}{\pgfqpoint{0.521986in}{4.013681in}}{\pgfqpoint{0.533036in}{4.013681in}}%
\pgfpathlineto{\pgfqpoint{0.533036in}{4.013681in}}%
\pgfpathclose%
\pgfusepath{stroke}%
\end{pgfscope}%
\begin{pgfscope}%
\pgfpathrectangle{\pgfqpoint{0.494722in}{0.437222in}}{\pgfqpoint{6.275590in}{5.159444in}}%
\pgfusepath{clip}%
\pgfsetbuttcap%
\pgfsetroundjoin%
\pgfsetlinewidth{1.003750pt}%
\definecolor{currentstroke}{rgb}{1.000000,0.000000,0.000000}%
\pgfsetstrokecolor{currentstroke}%
\pgfsetdash{}{0pt}%
\pgfpathmoveto{\pgfqpoint{2.075561in}{1.430327in}}%
\pgfpathcurveto{\pgfqpoint{2.086612in}{1.430327in}}{\pgfqpoint{2.097211in}{1.434717in}}{\pgfqpoint{2.105024in}{1.442531in}}%
\pgfpathcurveto{\pgfqpoint{2.112838in}{1.450345in}}{\pgfqpoint{2.117228in}{1.460944in}}{\pgfqpoint{2.117228in}{1.471994in}}%
\pgfpathcurveto{\pgfqpoint{2.117228in}{1.483044in}}{\pgfqpoint{2.112838in}{1.493643in}}{\pgfqpoint{2.105024in}{1.501456in}}%
\pgfpathcurveto{\pgfqpoint{2.097211in}{1.509270in}}{\pgfqpoint{2.086612in}{1.513660in}}{\pgfqpoint{2.075561in}{1.513660in}}%
\pgfpathcurveto{\pgfqpoint{2.064511in}{1.513660in}}{\pgfqpoint{2.053912in}{1.509270in}}{\pgfqpoint{2.046099in}{1.501456in}}%
\pgfpathcurveto{\pgfqpoint{2.038285in}{1.493643in}}{\pgfqpoint{2.033895in}{1.483044in}}{\pgfqpoint{2.033895in}{1.471994in}}%
\pgfpathcurveto{\pgfqpoint{2.033895in}{1.460944in}}{\pgfqpoint{2.038285in}{1.450345in}}{\pgfqpoint{2.046099in}{1.442531in}}%
\pgfpathcurveto{\pgfqpoint{2.053912in}{1.434717in}}{\pgfqpoint{2.064511in}{1.430327in}}{\pgfqpoint{2.075561in}{1.430327in}}%
\pgfpathlineto{\pgfqpoint{2.075561in}{1.430327in}}%
\pgfpathclose%
\pgfusepath{stroke}%
\end{pgfscope}%
\begin{pgfscope}%
\pgfpathrectangle{\pgfqpoint{0.494722in}{0.437222in}}{\pgfqpoint{6.275590in}{5.159444in}}%
\pgfusepath{clip}%
\pgfsetbuttcap%
\pgfsetroundjoin%
\pgfsetlinewidth{1.003750pt}%
\definecolor{currentstroke}{rgb}{1.000000,0.000000,0.000000}%
\pgfsetstrokecolor{currentstroke}%
\pgfsetdash{}{0pt}%
\pgfpathmoveto{\pgfqpoint{2.500315in}{1.199451in}}%
\pgfpathcurveto{\pgfqpoint{2.511366in}{1.199451in}}{\pgfqpoint{2.521965in}{1.203841in}}{\pgfqpoint{2.529778in}{1.211655in}}%
\pgfpathcurveto{\pgfqpoint{2.537592in}{1.219468in}}{\pgfqpoint{2.541982in}{1.230067in}}{\pgfqpoint{2.541982in}{1.241117in}}%
\pgfpathcurveto{\pgfqpoint{2.541982in}{1.252168in}}{\pgfqpoint{2.537592in}{1.262767in}}{\pgfqpoint{2.529778in}{1.270580in}}%
\pgfpathcurveto{\pgfqpoint{2.521965in}{1.278394in}}{\pgfqpoint{2.511366in}{1.282784in}}{\pgfqpoint{2.500315in}{1.282784in}}%
\pgfpathcurveto{\pgfqpoint{2.489265in}{1.282784in}}{\pgfqpoint{2.478666in}{1.278394in}}{\pgfqpoint{2.470853in}{1.270580in}}%
\pgfpathcurveto{\pgfqpoint{2.463039in}{1.262767in}}{\pgfqpoint{2.458649in}{1.252168in}}{\pgfqpoint{2.458649in}{1.241117in}}%
\pgfpathcurveto{\pgfqpoint{2.458649in}{1.230067in}}{\pgfqpoint{2.463039in}{1.219468in}}{\pgfqpoint{2.470853in}{1.211655in}}%
\pgfpathcurveto{\pgfqpoint{2.478666in}{1.203841in}}{\pgfqpoint{2.489265in}{1.199451in}}{\pgfqpoint{2.500315in}{1.199451in}}%
\pgfpathlineto{\pgfqpoint{2.500315in}{1.199451in}}%
\pgfpathclose%
\pgfusepath{stroke}%
\end{pgfscope}%
\begin{pgfscope}%
\pgfpathrectangle{\pgfqpoint{0.494722in}{0.437222in}}{\pgfqpoint{6.275590in}{5.159444in}}%
\pgfusepath{clip}%
\pgfsetbuttcap%
\pgfsetroundjoin%
\pgfsetlinewidth{1.003750pt}%
\definecolor{currentstroke}{rgb}{1.000000,0.000000,0.000000}%
\pgfsetstrokecolor{currentstroke}%
\pgfsetdash{}{0pt}%
\pgfpathmoveto{\pgfqpoint{3.666165in}{0.658105in}}%
\pgfpathcurveto{\pgfqpoint{3.677215in}{0.658105in}}{\pgfqpoint{3.687814in}{0.662495in}}{\pgfqpoint{3.695627in}{0.670308in}}%
\pgfpathcurveto{\pgfqpoint{3.703441in}{0.678122in}}{\pgfqpoint{3.707831in}{0.688721in}}{\pgfqpoint{3.707831in}{0.699771in}}%
\pgfpathcurveto{\pgfqpoint{3.707831in}{0.710821in}}{\pgfqpoint{3.703441in}{0.721420in}}{\pgfqpoint{3.695627in}{0.729234in}}%
\pgfpathcurveto{\pgfqpoint{3.687814in}{0.737048in}}{\pgfqpoint{3.677215in}{0.741438in}}{\pgfqpoint{3.666165in}{0.741438in}}%
\pgfpathcurveto{\pgfqpoint{3.655114in}{0.741438in}}{\pgfqpoint{3.644515in}{0.737048in}}{\pgfqpoint{3.636702in}{0.729234in}}%
\pgfpathcurveto{\pgfqpoint{3.628888in}{0.721420in}}{\pgfqpoint{3.624498in}{0.710821in}}{\pgfqpoint{3.624498in}{0.699771in}}%
\pgfpathcurveto{\pgfqpoint{3.624498in}{0.688721in}}{\pgfqpoint{3.628888in}{0.678122in}}{\pgfqpoint{3.636702in}{0.670308in}}%
\pgfpathcurveto{\pgfqpoint{3.644515in}{0.662495in}}{\pgfqpoint{3.655114in}{0.658105in}}{\pgfqpoint{3.666165in}{0.658105in}}%
\pgfpathlineto{\pgfqpoint{3.666165in}{0.658105in}}%
\pgfpathclose%
\pgfusepath{stroke}%
\end{pgfscope}%
\begin{pgfscope}%
\pgfpathrectangle{\pgfqpoint{0.494722in}{0.437222in}}{\pgfqpoint{6.275590in}{5.159444in}}%
\pgfusepath{clip}%
\pgfsetbuttcap%
\pgfsetroundjoin%
\pgfsetlinewidth{1.003750pt}%
\definecolor{currentstroke}{rgb}{1.000000,0.000000,0.000000}%
\pgfsetstrokecolor{currentstroke}%
\pgfsetdash{}{0pt}%
\pgfpathmoveto{\pgfqpoint{0.504517in}{4.320214in}}%
\pgfpathcurveto{\pgfqpoint{0.515567in}{4.320214in}}{\pgfqpoint{0.526166in}{4.324605in}}{\pgfqpoint{0.533979in}{4.332418in}}%
\pgfpathcurveto{\pgfqpoint{0.541793in}{4.340232in}}{\pgfqpoint{0.546183in}{4.350831in}}{\pgfqpoint{0.546183in}{4.361881in}}%
\pgfpathcurveto{\pgfqpoint{0.546183in}{4.372931in}}{\pgfqpoint{0.541793in}{4.383530in}}{\pgfqpoint{0.533979in}{4.391344in}}%
\pgfpathcurveto{\pgfqpoint{0.526166in}{4.399157in}}{\pgfqpoint{0.515567in}{4.403548in}}{\pgfqpoint{0.504517in}{4.403548in}}%
\pgfpathcurveto{\pgfqpoint{0.493466in}{4.403548in}}{\pgfqpoint{0.482867in}{4.399157in}}{\pgfqpoint{0.475054in}{4.391344in}}%
\pgfpathcurveto{\pgfqpoint{0.467240in}{4.383530in}}{\pgfqpoint{0.462850in}{4.372931in}}{\pgfqpoint{0.462850in}{4.361881in}}%
\pgfpathcurveto{\pgfqpoint{0.462850in}{4.350831in}}{\pgfqpoint{0.467240in}{4.340232in}}{\pgfqpoint{0.475054in}{4.332418in}}%
\pgfpathcurveto{\pgfqpoint{0.482867in}{4.324605in}}{\pgfqpoint{0.493466in}{4.320214in}}{\pgfqpoint{0.504517in}{4.320214in}}%
\pgfpathlineto{\pgfqpoint{0.504517in}{4.320214in}}%
\pgfpathclose%
\pgfusepath{stroke}%
\end{pgfscope}%
\begin{pgfscope}%
\pgfpathrectangle{\pgfqpoint{0.494722in}{0.437222in}}{\pgfqpoint{6.275590in}{5.159444in}}%
\pgfusepath{clip}%
\pgfsetbuttcap%
\pgfsetroundjoin%
\pgfsetlinewidth{1.003750pt}%
\definecolor{currentstroke}{rgb}{1.000000,0.000000,0.000000}%
\pgfsetstrokecolor{currentstroke}%
\pgfsetdash{}{0pt}%
\pgfpathmoveto{\pgfqpoint{1.951907in}{1.519541in}}%
\pgfpathcurveto{\pgfqpoint{1.962958in}{1.519541in}}{\pgfqpoint{1.973557in}{1.523931in}}{\pgfqpoint{1.981370in}{1.531745in}}%
\pgfpathcurveto{\pgfqpoint{1.989184in}{1.539558in}}{\pgfqpoint{1.993574in}{1.550157in}}{\pgfqpoint{1.993574in}{1.561208in}}%
\pgfpathcurveto{\pgfqpoint{1.993574in}{1.572258in}}{\pgfqpoint{1.989184in}{1.582857in}}{\pgfqpoint{1.981370in}{1.590670in}}%
\pgfpathcurveto{\pgfqpoint{1.973557in}{1.598484in}}{\pgfqpoint{1.962958in}{1.602874in}}{\pgfqpoint{1.951907in}{1.602874in}}%
\pgfpathcurveto{\pgfqpoint{1.940857in}{1.602874in}}{\pgfqpoint{1.930258in}{1.598484in}}{\pgfqpoint{1.922445in}{1.590670in}}%
\pgfpathcurveto{\pgfqpoint{1.914631in}{1.582857in}}{\pgfqpoint{1.910241in}{1.572258in}}{\pgfqpoint{1.910241in}{1.561208in}}%
\pgfpathcurveto{\pgfqpoint{1.910241in}{1.550157in}}{\pgfqpoint{1.914631in}{1.539558in}}{\pgfqpoint{1.922445in}{1.531745in}}%
\pgfpathcurveto{\pgfqpoint{1.930258in}{1.523931in}}{\pgfqpoint{1.940857in}{1.519541in}}{\pgfqpoint{1.951907in}{1.519541in}}%
\pgfpathlineto{\pgfqpoint{1.951907in}{1.519541in}}%
\pgfpathclose%
\pgfusepath{stroke}%
\end{pgfscope}%
\begin{pgfscope}%
\pgfpathrectangle{\pgfqpoint{0.494722in}{0.437222in}}{\pgfqpoint{6.275590in}{5.159444in}}%
\pgfusepath{clip}%
\pgfsetbuttcap%
\pgfsetroundjoin%
\pgfsetlinewidth{1.003750pt}%
\definecolor{currentstroke}{rgb}{1.000000,0.000000,0.000000}%
\pgfsetstrokecolor{currentstroke}%
\pgfsetdash{}{0pt}%
\pgfpathmoveto{\pgfqpoint{0.962953in}{2.660049in}}%
\pgfpathcurveto{\pgfqpoint{0.974003in}{2.660049in}}{\pgfqpoint{0.984602in}{2.664440in}}{\pgfqpoint{0.992416in}{2.672253in}}%
\pgfpathcurveto{\pgfqpoint{1.000229in}{2.680067in}}{\pgfqpoint{1.004619in}{2.690666in}}{\pgfqpoint{1.004619in}{2.701716in}}%
\pgfpathcurveto{\pgfqpoint{1.004619in}{2.712766in}}{\pgfqpoint{1.000229in}{2.723365in}}{\pgfqpoint{0.992416in}{2.731179in}}%
\pgfpathcurveto{\pgfqpoint{0.984602in}{2.738993in}}{\pgfqpoint{0.974003in}{2.743383in}}{\pgfqpoint{0.962953in}{2.743383in}}%
\pgfpathcurveto{\pgfqpoint{0.951903in}{2.743383in}}{\pgfqpoint{0.941304in}{2.738993in}}{\pgfqpoint{0.933490in}{2.731179in}}%
\pgfpathcurveto{\pgfqpoint{0.925676in}{2.723365in}}{\pgfqpoint{0.921286in}{2.712766in}}{\pgfqpoint{0.921286in}{2.701716in}}%
\pgfpathcurveto{\pgfqpoint{0.921286in}{2.690666in}}{\pgfqpoint{0.925676in}{2.680067in}}{\pgfqpoint{0.933490in}{2.672253in}}%
\pgfpathcurveto{\pgfqpoint{0.941304in}{2.664440in}}{\pgfqpoint{0.951903in}{2.660049in}}{\pgfqpoint{0.962953in}{2.660049in}}%
\pgfpathlineto{\pgfqpoint{0.962953in}{2.660049in}}%
\pgfpathclose%
\pgfusepath{stroke}%
\end{pgfscope}%
\begin{pgfscope}%
\pgfpathrectangle{\pgfqpoint{0.494722in}{0.437222in}}{\pgfqpoint{6.275590in}{5.159444in}}%
\pgfusepath{clip}%
\pgfsetbuttcap%
\pgfsetroundjoin%
\pgfsetlinewidth{1.003750pt}%
\definecolor{currentstroke}{rgb}{1.000000,0.000000,0.000000}%
\pgfsetstrokecolor{currentstroke}%
\pgfsetdash{}{0pt}%
\pgfpathmoveto{\pgfqpoint{4.142636in}{0.533790in}}%
\pgfpathcurveto{\pgfqpoint{4.153686in}{0.533790in}}{\pgfqpoint{4.164285in}{0.538180in}}{\pgfqpoint{4.172099in}{0.545994in}}%
\pgfpathcurveto{\pgfqpoint{4.179912in}{0.553807in}}{\pgfqpoint{4.184302in}{0.564406in}}{\pgfqpoint{4.184302in}{0.575456in}}%
\pgfpathcurveto{\pgfqpoint{4.184302in}{0.586506in}}{\pgfqpoint{4.179912in}{0.597105in}}{\pgfqpoint{4.172099in}{0.604919in}}%
\pgfpathcurveto{\pgfqpoint{4.164285in}{0.612733in}}{\pgfqpoint{4.153686in}{0.617123in}}{\pgfqpoint{4.142636in}{0.617123in}}%
\pgfpathcurveto{\pgfqpoint{4.131586in}{0.617123in}}{\pgfqpoint{4.120987in}{0.612733in}}{\pgfqpoint{4.113173in}{0.604919in}}%
\pgfpathcurveto{\pgfqpoint{4.105359in}{0.597105in}}{\pgfqpoint{4.100969in}{0.586506in}}{\pgfqpoint{4.100969in}{0.575456in}}%
\pgfpathcurveto{\pgfqpoint{4.100969in}{0.564406in}}{\pgfqpoint{4.105359in}{0.553807in}}{\pgfqpoint{4.113173in}{0.545994in}}%
\pgfpathcurveto{\pgfqpoint{4.120987in}{0.538180in}}{\pgfqpoint{4.131586in}{0.533790in}}{\pgfqpoint{4.142636in}{0.533790in}}%
\pgfpathlineto{\pgfqpoint{4.142636in}{0.533790in}}%
\pgfpathclose%
\pgfusepath{stroke}%
\end{pgfscope}%
\begin{pgfscope}%
\pgfpathrectangle{\pgfqpoint{0.494722in}{0.437222in}}{\pgfqpoint{6.275590in}{5.159444in}}%
\pgfusepath{clip}%
\pgfsetbuttcap%
\pgfsetroundjoin%
\pgfsetlinewidth{1.003750pt}%
\definecolor{currentstroke}{rgb}{1.000000,0.000000,0.000000}%
\pgfsetstrokecolor{currentstroke}%
\pgfsetdash{}{0pt}%
\pgfpathmoveto{\pgfqpoint{2.621373in}{1.089987in}}%
\pgfpathcurveto{\pgfqpoint{2.632423in}{1.089987in}}{\pgfqpoint{2.643022in}{1.094377in}}{\pgfqpoint{2.650836in}{1.102191in}}%
\pgfpathcurveto{\pgfqpoint{2.658650in}{1.110004in}}{\pgfqpoint{2.663040in}{1.120603in}}{\pgfqpoint{2.663040in}{1.131654in}}%
\pgfpathcurveto{\pgfqpoint{2.663040in}{1.142704in}}{\pgfqpoint{2.658650in}{1.153303in}}{\pgfqpoint{2.650836in}{1.161116in}}%
\pgfpathcurveto{\pgfqpoint{2.643022in}{1.168930in}}{\pgfqpoint{2.632423in}{1.173320in}}{\pgfqpoint{2.621373in}{1.173320in}}%
\pgfpathcurveto{\pgfqpoint{2.610323in}{1.173320in}}{\pgfqpoint{2.599724in}{1.168930in}}{\pgfqpoint{2.591911in}{1.161116in}}%
\pgfpathcurveto{\pgfqpoint{2.584097in}{1.153303in}}{\pgfqpoint{2.579707in}{1.142704in}}{\pgfqpoint{2.579707in}{1.131654in}}%
\pgfpathcurveto{\pgfqpoint{2.579707in}{1.120603in}}{\pgfqpoint{2.584097in}{1.110004in}}{\pgfqpoint{2.591911in}{1.102191in}}%
\pgfpathcurveto{\pgfqpoint{2.599724in}{1.094377in}}{\pgfqpoint{2.610323in}{1.089987in}}{\pgfqpoint{2.621373in}{1.089987in}}%
\pgfpathlineto{\pgfqpoint{2.621373in}{1.089987in}}%
\pgfpathclose%
\pgfusepath{stroke}%
\end{pgfscope}%
\begin{pgfscope}%
\pgfpathrectangle{\pgfqpoint{0.494722in}{0.437222in}}{\pgfqpoint{6.275590in}{5.159444in}}%
\pgfusepath{clip}%
\pgfsetbuttcap%
\pgfsetroundjoin%
\pgfsetlinewidth{1.003750pt}%
\definecolor{currentstroke}{rgb}{1.000000,0.000000,0.000000}%
\pgfsetstrokecolor{currentstroke}%
\pgfsetdash{}{0pt}%
\pgfpathmoveto{\pgfqpoint{0.725094in}{3.190590in}}%
\pgfpathcurveto{\pgfqpoint{0.736144in}{3.190590in}}{\pgfqpoint{0.746743in}{3.194980in}}{\pgfqpoint{0.754557in}{3.202793in}}%
\pgfpathcurveto{\pgfqpoint{0.762371in}{3.210607in}}{\pgfqpoint{0.766761in}{3.221206in}}{\pgfqpoint{0.766761in}{3.232256in}}%
\pgfpathcurveto{\pgfqpoint{0.766761in}{3.243306in}}{\pgfqpoint{0.762371in}{3.253905in}}{\pgfqpoint{0.754557in}{3.261719in}}%
\pgfpathcurveto{\pgfqpoint{0.746743in}{3.269533in}}{\pgfqpoint{0.736144in}{3.273923in}}{\pgfqpoint{0.725094in}{3.273923in}}%
\pgfpathcurveto{\pgfqpoint{0.714044in}{3.273923in}}{\pgfqpoint{0.703445in}{3.269533in}}{\pgfqpoint{0.695631in}{3.261719in}}%
\pgfpathcurveto{\pgfqpoint{0.687818in}{3.253905in}}{\pgfqpoint{0.683428in}{3.243306in}}{\pgfqpoint{0.683428in}{3.232256in}}%
\pgfpathcurveto{\pgfqpoint{0.683428in}{3.221206in}}{\pgfqpoint{0.687818in}{3.210607in}}{\pgfqpoint{0.695631in}{3.202793in}}%
\pgfpathcurveto{\pgfqpoint{0.703445in}{3.194980in}}{\pgfqpoint{0.714044in}{3.190590in}}{\pgfqpoint{0.725094in}{3.190590in}}%
\pgfpathlineto{\pgfqpoint{0.725094in}{3.190590in}}%
\pgfpathclose%
\pgfusepath{stroke}%
\end{pgfscope}%
\begin{pgfscope}%
\pgfpathrectangle{\pgfqpoint{0.494722in}{0.437222in}}{\pgfqpoint{6.275590in}{5.159444in}}%
\pgfusepath{clip}%
\pgfsetbuttcap%
\pgfsetroundjoin%
\pgfsetlinewidth{1.003750pt}%
\definecolor{currentstroke}{rgb}{1.000000,0.000000,0.000000}%
\pgfsetstrokecolor{currentstroke}%
\pgfsetdash{}{0pt}%
\pgfpathmoveto{\pgfqpoint{1.916673in}{1.547726in}}%
\pgfpathcurveto{\pgfqpoint{1.927723in}{1.547726in}}{\pgfqpoint{1.938322in}{1.552116in}}{\pgfqpoint{1.946136in}{1.559930in}}%
\pgfpathcurveto{\pgfqpoint{1.953950in}{1.567743in}}{\pgfqpoint{1.958340in}{1.578342in}}{\pgfqpoint{1.958340in}{1.589392in}}%
\pgfpathcurveto{\pgfqpoint{1.958340in}{1.600443in}}{\pgfqpoint{1.953950in}{1.611042in}}{\pgfqpoint{1.946136in}{1.618855in}}%
\pgfpathcurveto{\pgfqpoint{1.938322in}{1.626669in}}{\pgfqpoint{1.927723in}{1.631059in}}{\pgfqpoint{1.916673in}{1.631059in}}%
\pgfpathcurveto{\pgfqpoint{1.905623in}{1.631059in}}{\pgfqpoint{1.895024in}{1.626669in}}{\pgfqpoint{1.887210in}{1.618855in}}%
\pgfpathcurveto{\pgfqpoint{1.879397in}{1.611042in}}{\pgfqpoint{1.875007in}{1.600443in}}{\pgfqpoint{1.875007in}{1.589392in}}%
\pgfpathcurveto{\pgfqpoint{1.875007in}{1.578342in}}{\pgfqpoint{1.879397in}{1.567743in}}{\pgfqpoint{1.887210in}{1.559930in}}%
\pgfpathcurveto{\pgfqpoint{1.895024in}{1.552116in}}{\pgfqpoint{1.905623in}{1.547726in}}{\pgfqpoint{1.916673in}{1.547726in}}%
\pgfpathlineto{\pgfqpoint{1.916673in}{1.547726in}}%
\pgfpathclose%
\pgfusepath{stroke}%
\end{pgfscope}%
\begin{pgfscope}%
\pgfpathrectangle{\pgfqpoint{0.494722in}{0.437222in}}{\pgfqpoint{6.275590in}{5.159444in}}%
\pgfusepath{clip}%
\pgfsetbuttcap%
\pgfsetroundjoin%
\pgfsetlinewidth{1.003750pt}%
\definecolor{currentstroke}{rgb}{1.000000,0.000000,0.000000}%
\pgfsetstrokecolor{currentstroke}%
\pgfsetdash{}{0pt}%
\pgfpathmoveto{\pgfqpoint{4.011628in}{0.563560in}}%
\pgfpathcurveto{\pgfqpoint{4.022678in}{0.563560in}}{\pgfqpoint{4.033277in}{0.567950in}}{\pgfqpoint{4.041091in}{0.575764in}}%
\pgfpathcurveto{\pgfqpoint{4.048904in}{0.583577in}}{\pgfqpoint{4.053295in}{0.594176in}}{\pgfqpoint{4.053295in}{0.605226in}}%
\pgfpathcurveto{\pgfqpoint{4.053295in}{0.616277in}}{\pgfqpoint{4.048904in}{0.626876in}}{\pgfqpoint{4.041091in}{0.634689in}}%
\pgfpathcurveto{\pgfqpoint{4.033277in}{0.642503in}}{\pgfqpoint{4.022678in}{0.646893in}}{\pgfqpoint{4.011628in}{0.646893in}}%
\pgfpathcurveto{\pgfqpoint{4.000578in}{0.646893in}}{\pgfqpoint{3.989979in}{0.642503in}}{\pgfqpoint{3.982165in}{0.634689in}}%
\pgfpathcurveto{\pgfqpoint{3.974351in}{0.626876in}}{\pgfqpoint{3.969961in}{0.616277in}}{\pgfqpoint{3.969961in}{0.605226in}}%
\pgfpathcurveto{\pgfqpoint{3.969961in}{0.594176in}}{\pgfqpoint{3.974351in}{0.583577in}}{\pgfqpoint{3.982165in}{0.575764in}}%
\pgfpathcurveto{\pgfqpoint{3.989979in}{0.567950in}}{\pgfqpoint{4.000578in}{0.563560in}}{\pgfqpoint{4.011628in}{0.563560in}}%
\pgfpathlineto{\pgfqpoint{4.011628in}{0.563560in}}%
\pgfpathclose%
\pgfusepath{stroke}%
\end{pgfscope}%
\begin{pgfscope}%
\pgfpathrectangle{\pgfqpoint{0.494722in}{0.437222in}}{\pgfqpoint{6.275590in}{5.159444in}}%
\pgfusepath{clip}%
\pgfsetbuttcap%
\pgfsetroundjoin%
\pgfsetlinewidth{1.003750pt}%
\definecolor{currentstroke}{rgb}{1.000000,0.000000,0.000000}%
\pgfsetstrokecolor{currentstroke}%
\pgfsetdash{}{0pt}%
\pgfpathmoveto{\pgfqpoint{4.969140in}{0.426755in}}%
\pgfpathcurveto{\pgfqpoint{4.980190in}{0.426755in}}{\pgfqpoint{4.990789in}{0.431145in}}{\pgfqpoint{4.998602in}{0.438959in}}%
\pgfpathcurveto{\pgfqpoint{5.006416in}{0.446773in}}{\pgfqpoint{5.010806in}{0.457372in}}{\pgfqpoint{5.010806in}{0.468422in}}%
\pgfpathcurveto{\pgfqpoint{5.010806in}{0.479472in}}{\pgfqpoint{5.006416in}{0.490071in}}{\pgfqpoint{4.998602in}{0.497885in}}%
\pgfpathcurveto{\pgfqpoint{4.990789in}{0.505698in}}{\pgfqpoint{4.980190in}{0.510088in}}{\pgfqpoint{4.969140in}{0.510088in}}%
\pgfpathcurveto{\pgfqpoint{4.958089in}{0.510088in}}{\pgfqpoint{4.947490in}{0.505698in}}{\pgfqpoint{4.939677in}{0.497885in}}%
\pgfpathcurveto{\pgfqpoint{4.931863in}{0.490071in}}{\pgfqpoint{4.927473in}{0.479472in}}{\pgfqpoint{4.927473in}{0.468422in}}%
\pgfpathcurveto{\pgfqpoint{4.927473in}{0.457372in}}{\pgfqpoint{4.931863in}{0.446773in}}{\pgfqpoint{4.939677in}{0.438959in}}%
\pgfpathcurveto{\pgfqpoint{4.947490in}{0.431145in}}{\pgfqpoint{4.958089in}{0.426755in}}{\pgfqpoint{4.969140in}{0.426755in}}%
\pgfusepath{stroke}%
\end{pgfscope}%
\begin{pgfscope}%
\pgfpathrectangle{\pgfqpoint{0.494722in}{0.437222in}}{\pgfqpoint{6.275590in}{5.159444in}}%
\pgfusepath{clip}%
\pgfsetbuttcap%
\pgfsetroundjoin%
\pgfsetlinewidth{1.003750pt}%
\definecolor{currentstroke}{rgb}{1.000000,0.000000,0.000000}%
\pgfsetstrokecolor{currentstroke}%
\pgfsetdash{}{0pt}%
\pgfpathmoveto{\pgfqpoint{1.440962in}{1.989842in}}%
\pgfpathcurveto{\pgfqpoint{1.452012in}{1.989842in}}{\pgfqpoint{1.462611in}{1.994233in}}{\pgfqpoint{1.470425in}{2.002046in}}%
\pgfpathcurveto{\pgfqpoint{1.478238in}{2.009860in}}{\pgfqpoint{1.482629in}{2.020459in}}{\pgfqpoint{1.482629in}{2.031509in}}%
\pgfpathcurveto{\pgfqpoint{1.482629in}{2.042559in}}{\pgfqpoint{1.478238in}{2.053158in}}{\pgfqpoint{1.470425in}{2.060972in}}%
\pgfpathcurveto{\pgfqpoint{1.462611in}{2.068785in}}{\pgfqpoint{1.452012in}{2.073176in}}{\pgfqpoint{1.440962in}{2.073176in}}%
\pgfpathcurveto{\pgfqpoint{1.429912in}{2.073176in}}{\pgfqpoint{1.419313in}{2.068785in}}{\pgfqpoint{1.411499in}{2.060972in}}%
\pgfpathcurveto{\pgfqpoint{1.403686in}{2.053158in}}{\pgfqpoint{1.399295in}{2.042559in}}{\pgfqpoint{1.399295in}{2.031509in}}%
\pgfpathcurveto{\pgfqpoint{1.399295in}{2.020459in}}{\pgfqpoint{1.403686in}{2.009860in}}{\pgfqpoint{1.411499in}{2.002046in}}%
\pgfpathcurveto{\pgfqpoint{1.419313in}{1.994233in}}{\pgfqpoint{1.429912in}{1.989842in}}{\pgfqpoint{1.440962in}{1.989842in}}%
\pgfpathlineto{\pgfqpoint{1.440962in}{1.989842in}}%
\pgfpathclose%
\pgfusepath{stroke}%
\end{pgfscope}%
\begin{pgfscope}%
\pgfpathrectangle{\pgfqpoint{0.494722in}{0.437222in}}{\pgfqpoint{6.275590in}{5.159444in}}%
\pgfusepath{clip}%
\pgfsetbuttcap%
\pgfsetroundjoin%
\pgfsetlinewidth{1.003750pt}%
\definecolor{currentstroke}{rgb}{1.000000,0.000000,0.000000}%
\pgfsetstrokecolor{currentstroke}%
\pgfsetdash{}{0pt}%
\pgfpathmoveto{\pgfqpoint{1.260297in}{2.209171in}}%
\pgfpathcurveto{\pgfqpoint{1.271347in}{2.209171in}}{\pgfqpoint{1.281946in}{2.213561in}}{\pgfqpoint{1.289759in}{2.221375in}}%
\pgfpathcurveto{\pgfqpoint{1.297573in}{2.229189in}}{\pgfqpoint{1.301963in}{2.239788in}}{\pgfqpoint{1.301963in}{2.250838in}}%
\pgfpathcurveto{\pgfqpoint{1.301963in}{2.261888in}}{\pgfqpoint{1.297573in}{2.272487in}}{\pgfqpoint{1.289759in}{2.280300in}}%
\pgfpathcurveto{\pgfqpoint{1.281946in}{2.288114in}}{\pgfqpoint{1.271347in}{2.292504in}}{\pgfqpoint{1.260297in}{2.292504in}}%
\pgfpathcurveto{\pgfqpoint{1.249246in}{2.292504in}}{\pgfqpoint{1.238647in}{2.288114in}}{\pgfqpoint{1.230834in}{2.280300in}}%
\pgfpathcurveto{\pgfqpoint{1.223020in}{2.272487in}}{\pgfqpoint{1.218630in}{2.261888in}}{\pgfqpoint{1.218630in}{2.250838in}}%
\pgfpathcurveto{\pgfqpoint{1.218630in}{2.239788in}}{\pgfqpoint{1.223020in}{2.229189in}}{\pgfqpoint{1.230834in}{2.221375in}}%
\pgfpathcurveto{\pgfqpoint{1.238647in}{2.213561in}}{\pgfqpoint{1.249246in}{2.209171in}}{\pgfqpoint{1.260297in}{2.209171in}}%
\pgfpathlineto{\pgfqpoint{1.260297in}{2.209171in}}%
\pgfpathclose%
\pgfusepath{stroke}%
\end{pgfscope}%
\begin{pgfscope}%
\pgfpathrectangle{\pgfqpoint{0.494722in}{0.437222in}}{\pgfqpoint{6.275590in}{5.159444in}}%
\pgfusepath{clip}%
\pgfsetbuttcap%
\pgfsetroundjoin%
\pgfsetlinewidth{1.003750pt}%
\definecolor{currentstroke}{rgb}{1.000000,0.000000,0.000000}%
\pgfsetstrokecolor{currentstroke}%
\pgfsetdash{}{0pt}%
\pgfpathmoveto{\pgfqpoint{2.135789in}{1.383417in}}%
\pgfpathcurveto{\pgfqpoint{2.146839in}{1.383417in}}{\pgfqpoint{2.157438in}{1.387807in}}{\pgfqpoint{2.165252in}{1.395621in}}%
\pgfpathcurveto{\pgfqpoint{2.173065in}{1.403434in}}{\pgfqpoint{2.177456in}{1.414033in}}{\pgfqpoint{2.177456in}{1.425083in}}%
\pgfpathcurveto{\pgfqpoint{2.177456in}{1.436134in}}{\pgfqpoint{2.173065in}{1.446733in}}{\pgfqpoint{2.165252in}{1.454546in}}%
\pgfpathcurveto{\pgfqpoint{2.157438in}{1.462360in}}{\pgfqpoint{2.146839in}{1.466750in}}{\pgfqpoint{2.135789in}{1.466750in}}%
\pgfpathcurveto{\pgfqpoint{2.124739in}{1.466750in}}{\pgfqpoint{2.114140in}{1.462360in}}{\pgfqpoint{2.106326in}{1.454546in}}%
\pgfpathcurveto{\pgfqpoint{2.098513in}{1.446733in}}{\pgfqpoint{2.094122in}{1.436134in}}{\pgfqpoint{2.094122in}{1.425083in}}%
\pgfpathcurveto{\pgfqpoint{2.094122in}{1.414033in}}{\pgfqpoint{2.098513in}{1.403434in}}{\pgfqpoint{2.106326in}{1.395621in}}%
\pgfpathcurveto{\pgfqpoint{2.114140in}{1.387807in}}{\pgfqpoint{2.124739in}{1.383417in}}{\pgfqpoint{2.135789in}{1.383417in}}%
\pgfpathlineto{\pgfqpoint{2.135789in}{1.383417in}}%
\pgfpathclose%
\pgfusepath{stroke}%
\end{pgfscope}%
\begin{pgfscope}%
\pgfpathrectangle{\pgfqpoint{0.494722in}{0.437222in}}{\pgfqpoint{6.275590in}{5.159444in}}%
\pgfusepath{clip}%
\pgfsetbuttcap%
\pgfsetroundjoin%
\pgfsetlinewidth{1.003750pt}%
\definecolor{currentstroke}{rgb}{1.000000,0.000000,0.000000}%
\pgfsetstrokecolor{currentstroke}%
\pgfsetdash{}{0pt}%
\pgfpathmoveto{\pgfqpoint{0.703892in}{3.253417in}}%
\pgfpathcurveto{\pgfqpoint{0.714942in}{3.253417in}}{\pgfqpoint{0.725541in}{3.257807in}}{\pgfqpoint{0.733355in}{3.265621in}}%
\pgfpathcurveto{\pgfqpoint{0.741169in}{3.273435in}}{\pgfqpoint{0.745559in}{3.284034in}}{\pgfqpoint{0.745559in}{3.295084in}}%
\pgfpathcurveto{\pgfqpoint{0.745559in}{3.306134in}}{\pgfqpoint{0.741169in}{3.316733in}}{\pgfqpoint{0.733355in}{3.324546in}}%
\pgfpathcurveto{\pgfqpoint{0.725541in}{3.332360in}}{\pgfqpoint{0.714942in}{3.336750in}}{\pgfqpoint{0.703892in}{3.336750in}}%
\pgfpathcurveto{\pgfqpoint{0.692842in}{3.336750in}}{\pgfqpoint{0.682243in}{3.332360in}}{\pgfqpoint{0.674429in}{3.324546in}}%
\pgfpathcurveto{\pgfqpoint{0.666616in}{3.316733in}}{\pgfqpoint{0.662225in}{3.306134in}}{\pgfqpoint{0.662225in}{3.295084in}}%
\pgfpathcurveto{\pgfqpoint{0.662225in}{3.284034in}}{\pgfqpoint{0.666616in}{3.273435in}}{\pgfqpoint{0.674429in}{3.265621in}}%
\pgfpathcurveto{\pgfqpoint{0.682243in}{3.257807in}}{\pgfqpoint{0.692842in}{3.253417in}}{\pgfqpoint{0.703892in}{3.253417in}}%
\pgfpathlineto{\pgfqpoint{0.703892in}{3.253417in}}%
\pgfpathclose%
\pgfusepath{stroke}%
\end{pgfscope}%
\begin{pgfscope}%
\pgfpathrectangle{\pgfqpoint{0.494722in}{0.437222in}}{\pgfqpoint{6.275590in}{5.159444in}}%
\pgfusepath{clip}%
\pgfsetbuttcap%
\pgfsetroundjoin%
\pgfsetlinewidth{1.003750pt}%
\definecolor{currentstroke}{rgb}{1.000000,0.000000,0.000000}%
\pgfsetstrokecolor{currentstroke}%
\pgfsetdash{}{0pt}%
\pgfpathmoveto{\pgfqpoint{1.287037in}{2.183876in}}%
\pgfpathcurveto{\pgfqpoint{1.298087in}{2.183876in}}{\pgfqpoint{1.308687in}{2.188266in}}{\pgfqpoint{1.316500in}{2.196079in}}%
\pgfpathcurveto{\pgfqpoint{1.324314in}{2.203893in}}{\pgfqpoint{1.328704in}{2.214492in}}{\pgfqpoint{1.328704in}{2.225542in}}%
\pgfpathcurveto{\pgfqpoint{1.328704in}{2.236592in}}{\pgfqpoint{1.324314in}{2.247191in}}{\pgfqpoint{1.316500in}{2.255005in}}%
\pgfpathcurveto{\pgfqpoint{1.308687in}{2.262819in}}{\pgfqpoint{1.298087in}{2.267209in}}{\pgfqpoint{1.287037in}{2.267209in}}%
\pgfpathcurveto{\pgfqpoint{1.275987in}{2.267209in}}{\pgfqpoint{1.265388in}{2.262819in}}{\pgfqpoint{1.257575in}{2.255005in}}%
\pgfpathcurveto{\pgfqpoint{1.249761in}{2.247191in}}{\pgfqpoint{1.245371in}{2.236592in}}{\pgfqpoint{1.245371in}{2.225542in}}%
\pgfpathcurveto{\pgfqpoint{1.245371in}{2.214492in}}{\pgfqpoint{1.249761in}{2.203893in}}{\pgfqpoint{1.257575in}{2.196079in}}%
\pgfpathcurveto{\pgfqpoint{1.265388in}{2.188266in}}{\pgfqpoint{1.275987in}{2.183876in}}{\pgfqpoint{1.287037in}{2.183876in}}%
\pgfpathlineto{\pgfqpoint{1.287037in}{2.183876in}}%
\pgfpathclose%
\pgfusepath{stroke}%
\end{pgfscope}%
\begin{pgfscope}%
\pgfpathrectangle{\pgfqpoint{0.494722in}{0.437222in}}{\pgfqpoint{6.275590in}{5.159444in}}%
\pgfusepath{clip}%
\pgfsetbuttcap%
\pgfsetroundjoin%
\pgfsetlinewidth{1.003750pt}%
\definecolor{currentstroke}{rgb}{1.000000,0.000000,0.000000}%
\pgfsetstrokecolor{currentstroke}%
\pgfsetdash{}{0pt}%
\pgfpathmoveto{\pgfqpoint{4.173350in}{0.528017in}}%
\pgfpathcurveto{\pgfqpoint{4.184400in}{0.528017in}}{\pgfqpoint{4.194999in}{0.532408in}}{\pgfqpoint{4.202813in}{0.540221in}}%
\pgfpathcurveto{\pgfqpoint{4.210626in}{0.548035in}}{\pgfqpoint{4.215017in}{0.558634in}}{\pgfqpoint{4.215017in}{0.569684in}}%
\pgfpathcurveto{\pgfqpoint{4.215017in}{0.580734in}}{\pgfqpoint{4.210626in}{0.591333in}}{\pgfqpoint{4.202813in}{0.599147in}}%
\pgfpathcurveto{\pgfqpoint{4.194999in}{0.606960in}}{\pgfqpoint{4.184400in}{0.611351in}}{\pgfqpoint{4.173350in}{0.611351in}}%
\pgfpathcurveto{\pgfqpoint{4.162300in}{0.611351in}}{\pgfqpoint{4.151701in}{0.606960in}}{\pgfqpoint{4.143887in}{0.599147in}}%
\pgfpathcurveto{\pgfqpoint{4.136074in}{0.591333in}}{\pgfqpoint{4.131683in}{0.580734in}}{\pgfqpoint{4.131683in}{0.569684in}}%
\pgfpathcurveto{\pgfqpoint{4.131683in}{0.558634in}}{\pgfqpoint{4.136074in}{0.548035in}}{\pgfqpoint{4.143887in}{0.540221in}}%
\pgfpathcurveto{\pgfqpoint{4.151701in}{0.532408in}}{\pgfqpoint{4.162300in}{0.528017in}}{\pgfqpoint{4.173350in}{0.528017in}}%
\pgfpathlineto{\pgfqpoint{4.173350in}{0.528017in}}%
\pgfpathclose%
\pgfusepath{stroke}%
\end{pgfscope}%
\begin{pgfscope}%
\pgfpathrectangle{\pgfqpoint{0.494722in}{0.437222in}}{\pgfqpoint{6.275590in}{5.159444in}}%
\pgfusepath{clip}%
\pgfsetbuttcap%
\pgfsetroundjoin%
\pgfsetlinewidth{1.003750pt}%
\definecolor{currentstroke}{rgb}{1.000000,0.000000,0.000000}%
\pgfsetstrokecolor{currentstroke}%
\pgfsetdash{}{0pt}%
\pgfpathmoveto{\pgfqpoint{3.193164in}{0.816080in}}%
\pgfpathcurveto{\pgfqpoint{3.204214in}{0.816080in}}{\pgfqpoint{3.214813in}{0.820470in}}{\pgfqpoint{3.222627in}{0.828284in}}%
\pgfpathcurveto{\pgfqpoint{3.230440in}{0.836097in}}{\pgfqpoint{3.234831in}{0.846696in}}{\pgfqpoint{3.234831in}{0.857746in}}%
\pgfpathcurveto{\pgfqpoint{3.234831in}{0.868797in}}{\pgfqpoint{3.230440in}{0.879396in}}{\pgfqpoint{3.222627in}{0.887209in}}%
\pgfpathcurveto{\pgfqpoint{3.214813in}{0.895023in}}{\pgfqpoint{3.204214in}{0.899413in}}{\pgfqpoint{3.193164in}{0.899413in}}%
\pgfpathcurveto{\pgfqpoint{3.182114in}{0.899413in}}{\pgfqpoint{3.171515in}{0.895023in}}{\pgfqpoint{3.163701in}{0.887209in}}%
\pgfpathcurveto{\pgfqpoint{3.155888in}{0.879396in}}{\pgfqpoint{3.151497in}{0.868797in}}{\pgfqpoint{3.151497in}{0.857746in}}%
\pgfpathcurveto{\pgfqpoint{3.151497in}{0.846696in}}{\pgfqpoint{3.155888in}{0.836097in}}{\pgfqpoint{3.163701in}{0.828284in}}%
\pgfpathcurveto{\pgfqpoint{3.171515in}{0.820470in}}{\pgfqpoint{3.182114in}{0.816080in}}{\pgfqpoint{3.193164in}{0.816080in}}%
\pgfpathlineto{\pgfqpoint{3.193164in}{0.816080in}}%
\pgfpathclose%
\pgfusepath{stroke}%
\end{pgfscope}%
\begin{pgfscope}%
\pgfpathrectangle{\pgfqpoint{0.494722in}{0.437222in}}{\pgfqpoint{6.275590in}{5.159444in}}%
\pgfusepath{clip}%
\pgfsetbuttcap%
\pgfsetroundjoin%
\pgfsetlinewidth{1.003750pt}%
\definecolor{currentstroke}{rgb}{1.000000,0.000000,0.000000}%
\pgfsetstrokecolor{currentstroke}%
\pgfsetdash{}{0pt}%
\pgfpathmoveto{\pgfqpoint{3.785365in}{0.613950in}}%
\pgfpathcurveto{\pgfqpoint{3.796415in}{0.613950in}}{\pgfqpoint{3.807014in}{0.618341in}}{\pgfqpoint{3.814828in}{0.626154in}}%
\pgfpathcurveto{\pgfqpoint{3.822641in}{0.633968in}}{\pgfqpoint{3.827032in}{0.644567in}}{\pgfqpoint{3.827032in}{0.655617in}}%
\pgfpathcurveto{\pgfqpoint{3.827032in}{0.666667in}}{\pgfqpoint{3.822641in}{0.677266in}}{\pgfqpoint{3.814828in}{0.685080in}}%
\pgfpathcurveto{\pgfqpoint{3.807014in}{0.692893in}}{\pgfqpoint{3.796415in}{0.697284in}}{\pgfqpoint{3.785365in}{0.697284in}}%
\pgfpathcurveto{\pgfqpoint{3.774315in}{0.697284in}}{\pgfqpoint{3.763716in}{0.692893in}}{\pgfqpoint{3.755902in}{0.685080in}}%
\pgfpathcurveto{\pgfqpoint{3.748088in}{0.677266in}}{\pgfqpoint{3.743698in}{0.666667in}}{\pgfqpoint{3.743698in}{0.655617in}}%
\pgfpathcurveto{\pgfqpoint{3.743698in}{0.644567in}}{\pgfqpoint{3.748088in}{0.633968in}}{\pgfqpoint{3.755902in}{0.626154in}}%
\pgfpathcurveto{\pgfqpoint{3.763716in}{0.618341in}}{\pgfqpoint{3.774315in}{0.613950in}}{\pgfqpoint{3.785365in}{0.613950in}}%
\pgfpathlineto{\pgfqpoint{3.785365in}{0.613950in}}%
\pgfpathclose%
\pgfusepath{stroke}%
\end{pgfscope}%
\begin{pgfscope}%
\pgfpathrectangle{\pgfqpoint{0.494722in}{0.437222in}}{\pgfqpoint{6.275590in}{5.159444in}}%
\pgfusepath{clip}%
\pgfsetbuttcap%
\pgfsetroundjoin%
\pgfsetlinewidth{1.003750pt}%
\definecolor{currentstroke}{rgb}{1.000000,0.000000,0.000000}%
\pgfsetstrokecolor{currentstroke}%
\pgfsetdash{}{0pt}%
\pgfpathmoveto{\pgfqpoint{3.296089in}{0.773000in}}%
\pgfpathcurveto{\pgfqpoint{3.307139in}{0.773000in}}{\pgfqpoint{3.317738in}{0.777390in}}{\pgfqpoint{3.325552in}{0.785204in}}%
\pgfpathcurveto{\pgfqpoint{3.333366in}{0.793017in}}{\pgfqpoint{3.337756in}{0.803616in}}{\pgfqpoint{3.337756in}{0.814666in}}%
\pgfpathcurveto{\pgfqpoint{3.337756in}{0.825717in}}{\pgfqpoint{3.333366in}{0.836316in}}{\pgfqpoint{3.325552in}{0.844129in}}%
\pgfpathcurveto{\pgfqpoint{3.317738in}{0.851943in}}{\pgfqpoint{3.307139in}{0.856333in}}{\pgfqpoint{3.296089in}{0.856333in}}%
\pgfpathcurveto{\pgfqpoint{3.285039in}{0.856333in}}{\pgfqpoint{3.274440in}{0.851943in}}{\pgfqpoint{3.266627in}{0.844129in}}%
\pgfpathcurveto{\pgfqpoint{3.258813in}{0.836316in}}{\pgfqpoint{3.254423in}{0.825717in}}{\pgfqpoint{3.254423in}{0.814666in}}%
\pgfpathcurveto{\pgfqpoint{3.254423in}{0.803616in}}{\pgfqpoint{3.258813in}{0.793017in}}{\pgfqpoint{3.266627in}{0.785204in}}%
\pgfpathcurveto{\pgfqpoint{3.274440in}{0.777390in}}{\pgfqpoint{3.285039in}{0.773000in}}{\pgfqpoint{3.296089in}{0.773000in}}%
\pgfpathlineto{\pgfqpoint{3.296089in}{0.773000in}}%
\pgfpathclose%
\pgfusepath{stroke}%
\end{pgfscope}%
\begin{pgfscope}%
\pgfpathrectangle{\pgfqpoint{0.494722in}{0.437222in}}{\pgfqpoint{6.275590in}{5.159444in}}%
\pgfusepath{clip}%
\pgfsetbuttcap%
\pgfsetroundjoin%
\pgfsetlinewidth{1.003750pt}%
\definecolor{currentstroke}{rgb}{1.000000,0.000000,0.000000}%
\pgfsetstrokecolor{currentstroke}%
\pgfsetdash{}{0pt}%
\pgfpathmoveto{\pgfqpoint{1.784328in}{1.669178in}}%
\pgfpathcurveto{\pgfqpoint{1.795378in}{1.669178in}}{\pgfqpoint{1.805977in}{1.673568in}}{\pgfqpoint{1.813790in}{1.681382in}}%
\pgfpathcurveto{\pgfqpoint{1.821604in}{1.689195in}}{\pgfqpoint{1.825994in}{1.699794in}}{\pgfqpoint{1.825994in}{1.710845in}}%
\pgfpathcurveto{\pgfqpoint{1.825994in}{1.721895in}}{\pgfqpoint{1.821604in}{1.732494in}}{\pgfqpoint{1.813790in}{1.740307in}}%
\pgfpathcurveto{\pgfqpoint{1.805977in}{1.748121in}}{\pgfqpoint{1.795378in}{1.752511in}}{\pgfqpoint{1.784328in}{1.752511in}}%
\pgfpathcurveto{\pgfqpoint{1.773277in}{1.752511in}}{\pgfqpoint{1.762678in}{1.748121in}}{\pgfqpoint{1.754865in}{1.740307in}}%
\pgfpathcurveto{\pgfqpoint{1.747051in}{1.732494in}}{\pgfqpoint{1.742661in}{1.721895in}}{\pgfqpoint{1.742661in}{1.710845in}}%
\pgfpathcurveto{\pgfqpoint{1.742661in}{1.699794in}}{\pgfqpoint{1.747051in}{1.689195in}}{\pgfqpoint{1.754865in}{1.681382in}}%
\pgfpathcurveto{\pgfqpoint{1.762678in}{1.673568in}}{\pgfqpoint{1.773277in}{1.669178in}}{\pgfqpoint{1.784328in}{1.669178in}}%
\pgfpathlineto{\pgfqpoint{1.784328in}{1.669178in}}%
\pgfpathclose%
\pgfusepath{stroke}%
\end{pgfscope}%
\begin{pgfscope}%
\pgfpathrectangle{\pgfqpoint{0.494722in}{0.437222in}}{\pgfqpoint{6.275590in}{5.159444in}}%
\pgfusepath{clip}%
\pgfsetbuttcap%
\pgfsetroundjoin%
\pgfsetlinewidth{1.003750pt}%
\definecolor{currentstroke}{rgb}{1.000000,0.000000,0.000000}%
\pgfsetstrokecolor{currentstroke}%
\pgfsetdash{}{0pt}%
\pgfpathmoveto{\pgfqpoint{1.845276in}{1.604397in}}%
\pgfpathcurveto{\pgfqpoint{1.856326in}{1.604397in}}{\pgfqpoint{1.866925in}{1.608788in}}{\pgfqpoint{1.874739in}{1.616601in}}%
\pgfpathcurveto{\pgfqpoint{1.882552in}{1.624415in}}{\pgfqpoint{1.886943in}{1.635014in}}{\pgfqpoint{1.886943in}{1.646064in}}%
\pgfpathcurveto{\pgfqpoint{1.886943in}{1.657114in}}{\pgfqpoint{1.882552in}{1.667713in}}{\pgfqpoint{1.874739in}{1.675527in}}%
\pgfpathcurveto{\pgfqpoint{1.866925in}{1.683340in}}{\pgfqpoint{1.856326in}{1.687731in}}{\pgfqpoint{1.845276in}{1.687731in}}%
\pgfpathcurveto{\pgfqpoint{1.834226in}{1.687731in}}{\pgfqpoint{1.823627in}{1.683340in}}{\pgfqpoint{1.815813in}{1.675527in}}%
\pgfpathcurveto{\pgfqpoint{1.808000in}{1.667713in}}{\pgfqpoint{1.803609in}{1.657114in}}{\pgfqpoint{1.803609in}{1.646064in}}%
\pgfpathcurveto{\pgfqpoint{1.803609in}{1.635014in}}{\pgfqpoint{1.808000in}{1.624415in}}{\pgfqpoint{1.815813in}{1.616601in}}%
\pgfpathcurveto{\pgfqpoint{1.823627in}{1.608788in}}{\pgfqpoint{1.834226in}{1.604397in}}{\pgfqpoint{1.845276in}{1.604397in}}%
\pgfpathlineto{\pgfqpoint{1.845276in}{1.604397in}}%
\pgfpathclose%
\pgfusepath{stroke}%
\end{pgfscope}%
\begin{pgfscope}%
\pgfpathrectangle{\pgfqpoint{0.494722in}{0.437222in}}{\pgfqpoint{6.275590in}{5.159444in}}%
\pgfusepath{clip}%
\pgfsetbuttcap%
\pgfsetroundjoin%
\pgfsetlinewidth{1.003750pt}%
\definecolor{currentstroke}{rgb}{1.000000,0.000000,0.000000}%
\pgfsetstrokecolor{currentstroke}%
\pgfsetdash{}{0pt}%
\pgfpathmoveto{\pgfqpoint{5.464316in}{0.400891in}}%
\pgfpathcurveto{\pgfqpoint{5.475366in}{0.400891in}}{\pgfqpoint{5.485965in}{0.405281in}}{\pgfqpoint{5.493779in}{0.413095in}}%
\pgfpathcurveto{\pgfqpoint{5.501592in}{0.420908in}}{\pgfqpoint{5.505983in}{0.431507in}}{\pgfqpoint{5.505983in}{0.442558in}}%
\pgfpathcurveto{\pgfqpoint{5.505983in}{0.453608in}}{\pgfqpoint{5.501592in}{0.464207in}}{\pgfqpoint{5.493779in}{0.472020in}}%
\pgfpathcurveto{\pgfqpoint{5.485965in}{0.479834in}}{\pgfqpoint{5.475366in}{0.484224in}}{\pgfqpoint{5.464316in}{0.484224in}}%
\pgfpathcurveto{\pgfqpoint{5.453266in}{0.484224in}}{\pgfqpoint{5.442667in}{0.479834in}}{\pgfqpoint{5.434853in}{0.472020in}}%
\pgfpathcurveto{\pgfqpoint{5.427039in}{0.464207in}}{\pgfqpoint{5.422649in}{0.453608in}}{\pgfqpoint{5.422649in}{0.442558in}}%
\pgfpathcurveto{\pgfqpoint{5.422649in}{0.431507in}}{\pgfqpoint{5.427039in}{0.420908in}}{\pgfqpoint{5.434853in}{0.413095in}}%
\pgfpathcurveto{\pgfqpoint{5.442667in}{0.405281in}}{\pgfqpoint{5.453266in}{0.400891in}}{\pgfqpoint{5.464316in}{0.400891in}}%
\pgfusepath{stroke}%
\end{pgfscope}%
\begin{pgfscope}%
\pgfpathrectangle{\pgfqpoint{0.494722in}{0.437222in}}{\pgfqpoint{6.275590in}{5.159444in}}%
\pgfusepath{clip}%
\pgfsetbuttcap%
\pgfsetroundjoin%
\pgfsetlinewidth{1.003750pt}%
\definecolor{currentstroke}{rgb}{1.000000,0.000000,0.000000}%
\pgfsetstrokecolor{currentstroke}%
\pgfsetdash{}{0pt}%
\pgfpathmoveto{\pgfqpoint{3.903859in}{0.586053in}}%
\pgfpathcurveto{\pgfqpoint{3.914909in}{0.586053in}}{\pgfqpoint{3.925508in}{0.590443in}}{\pgfqpoint{3.933321in}{0.598257in}}%
\pgfpathcurveto{\pgfqpoint{3.941135in}{0.606070in}}{\pgfqpoint{3.945525in}{0.616669in}}{\pgfqpoint{3.945525in}{0.627719in}}%
\pgfpathcurveto{\pgfqpoint{3.945525in}{0.638770in}}{\pgfqpoint{3.941135in}{0.649369in}}{\pgfqpoint{3.933321in}{0.657182in}}%
\pgfpathcurveto{\pgfqpoint{3.925508in}{0.664996in}}{\pgfqpoint{3.914909in}{0.669386in}}{\pgfqpoint{3.903859in}{0.669386in}}%
\pgfpathcurveto{\pgfqpoint{3.892808in}{0.669386in}}{\pgfqpoint{3.882209in}{0.664996in}}{\pgfqpoint{3.874396in}{0.657182in}}%
\pgfpathcurveto{\pgfqpoint{3.866582in}{0.649369in}}{\pgfqpoint{3.862192in}{0.638770in}}{\pgfqpoint{3.862192in}{0.627719in}}%
\pgfpathcurveto{\pgfqpoint{3.862192in}{0.616669in}}{\pgfqpoint{3.866582in}{0.606070in}}{\pgfqpoint{3.874396in}{0.598257in}}%
\pgfpathcurveto{\pgfqpoint{3.882209in}{0.590443in}}{\pgfqpoint{3.892808in}{0.586053in}}{\pgfqpoint{3.903859in}{0.586053in}}%
\pgfpathlineto{\pgfqpoint{3.903859in}{0.586053in}}%
\pgfpathclose%
\pgfusepath{stroke}%
\end{pgfscope}%
\begin{pgfscope}%
\pgfpathrectangle{\pgfqpoint{0.494722in}{0.437222in}}{\pgfqpoint{6.275590in}{5.159444in}}%
\pgfusepath{clip}%
\pgfsetbuttcap%
\pgfsetroundjoin%
\pgfsetlinewidth{1.003750pt}%
\definecolor{currentstroke}{rgb}{1.000000,0.000000,0.000000}%
\pgfsetstrokecolor{currentstroke}%
\pgfsetdash{}{0pt}%
\pgfpathmoveto{\pgfqpoint{4.612074in}{0.465245in}}%
\pgfpathcurveto{\pgfqpoint{4.623125in}{0.465245in}}{\pgfqpoint{4.633724in}{0.469635in}}{\pgfqpoint{4.641537in}{0.477448in}}%
\pgfpathcurveto{\pgfqpoint{4.649351in}{0.485262in}}{\pgfqpoint{4.653741in}{0.495861in}}{\pgfqpoint{4.653741in}{0.506911in}}%
\pgfpathcurveto{\pgfqpoint{4.653741in}{0.517961in}}{\pgfqpoint{4.649351in}{0.528560in}}{\pgfqpoint{4.641537in}{0.536374in}}%
\pgfpathcurveto{\pgfqpoint{4.633724in}{0.544188in}}{\pgfqpoint{4.623125in}{0.548578in}}{\pgfqpoint{4.612074in}{0.548578in}}%
\pgfpathcurveto{\pgfqpoint{4.601024in}{0.548578in}}{\pgfqpoint{4.590425in}{0.544188in}}{\pgfqpoint{4.582612in}{0.536374in}}%
\pgfpathcurveto{\pgfqpoint{4.574798in}{0.528560in}}{\pgfqpoint{4.570408in}{0.517961in}}{\pgfqpoint{4.570408in}{0.506911in}}%
\pgfpathcurveto{\pgfqpoint{4.570408in}{0.495861in}}{\pgfqpoint{4.574798in}{0.485262in}}{\pgfqpoint{4.582612in}{0.477448in}}%
\pgfpathcurveto{\pgfqpoint{4.590425in}{0.469635in}}{\pgfqpoint{4.601024in}{0.465245in}}{\pgfqpoint{4.612074in}{0.465245in}}%
\pgfpathlineto{\pgfqpoint{4.612074in}{0.465245in}}%
\pgfpathclose%
\pgfusepath{stroke}%
\end{pgfscope}%
\begin{pgfscope}%
\pgfpathrectangle{\pgfqpoint{0.494722in}{0.437222in}}{\pgfqpoint{6.275590in}{5.159444in}}%
\pgfusepath{clip}%
\pgfsetbuttcap%
\pgfsetroundjoin%
\pgfsetlinewidth{1.003750pt}%
\definecolor{currentstroke}{rgb}{1.000000,0.000000,0.000000}%
\pgfsetstrokecolor{currentstroke}%
\pgfsetdash{}{0pt}%
\pgfpathmoveto{\pgfqpoint{0.746970in}{3.134702in}}%
\pgfpathcurveto{\pgfqpoint{0.758020in}{3.134702in}}{\pgfqpoint{0.768619in}{3.139092in}}{\pgfqpoint{0.776432in}{3.146906in}}%
\pgfpathcurveto{\pgfqpoint{0.784246in}{3.154720in}}{\pgfqpoint{0.788636in}{3.165319in}}{\pgfqpoint{0.788636in}{3.176369in}}%
\pgfpathcurveto{\pgfqpoint{0.788636in}{3.187419in}}{\pgfqpoint{0.784246in}{3.198018in}}{\pgfqpoint{0.776432in}{3.205832in}}%
\pgfpathcurveto{\pgfqpoint{0.768619in}{3.213645in}}{\pgfqpoint{0.758020in}{3.218035in}}{\pgfqpoint{0.746970in}{3.218035in}}%
\pgfpathcurveto{\pgfqpoint{0.735920in}{3.218035in}}{\pgfqpoint{0.725321in}{3.213645in}}{\pgfqpoint{0.717507in}{3.205832in}}%
\pgfpathcurveto{\pgfqpoint{0.709693in}{3.198018in}}{\pgfqpoint{0.705303in}{3.187419in}}{\pgfqpoint{0.705303in}{3.176369in}}%
\pgfpathcurveto{\pgfqpoint{0.705303in}{3.165319in}}{\pgfqpoint{0.709693in}{3.154720in}}{\pgfqpoint{0.717507in}{3.146906in}}%
\pgfpathcurveto{\pgfqpoint{0.725321in}{3.139092in}}{\pgfqpoint{0.735920in}{3.134702in}}{\pgfqpoint{0.746970in}{3.134702in}}%
\pgfpathlineto{\pgfqpoint{0.746970in}{3.134702in}}%
\pgfpathclose%
\pgfusepath{stroke}%
\end{pgfscope}%
\begin{pgfscope}%
\pgfpathrectangle{\pgfqpoint{0.494722in}{0.437222in}}{\pgfqpoint{6.275590in}{5.159444in}}%
\pgfusepath{clip}%
\pgfsetbuttcap%
\pgfsetroundjoin%
\pgfsetlinewidth{1.003750pt}%
\definecolor{currentstroke}{rgb}{1.000000,0.000000,0.000000}%
\pgfsetstrokecolor{currentstroke}%
\pgfsetdash{}{0pt}%
\pgfpathmoveto{\pgfqpoint{1.133257in}{2.380107in}}%
\pgfpathcurveto{\pgfqpoint{1.144307in}{2.380107in}}{\pgfqpoint{1.154906in}{2.384497in}}{\pgfqpoint{1.162720in}{2.392311in}}%
\pgfpathcurveto{\pgfqpoint{1.170533in}{2.400125in}}{\pgfqpoint{1.174924in}{2.410724in}}{\pgfqpoint{1.174924in}{2.421774in}}%
\pgfpathcurveto{\pgfqpoint{1.174924in}{2.432824in}}{\pgfqpoint{1.170533in}{2.443423in}}{\pgfqpoint{1.162720in}{2.451237in}}%
\pgfpathcurveto{\pgfqpoint{1.154906in}{2.459050in}}{\pgfqpoint{1.144307in}{2.463441in}}{\pgfqpoint{1.133257in}{2.463441in}}%
\pgfpathcurveto{\pgfqpoint{1.122207in}{2.463441in}}{\pgfqpoint{1.111608in}{2.459050in}}{\pgfqpoint{1.103794in}{2.451237in}}%
\pgfpathcurveto{\pgfqpoint{1.095981in}{2.443423in}}{\pgfqpoint{1.091590in}{2.432824in}}{\pgfqpoint{1.091590in}{2.421774in}}%
\pgfpathcurveto{\pgfqpoint{1.091590in}{2.410724in}}{\pgfqpoint{1.095981in}{2.400125in}}{\pgfqpoint{1.103794in}{2.392311in}}%
\pgfpathcurveto{\pgfqpoint{1.111608in}{2.384497in}}{\pgfqpoint{1.122207in}{2.380107in}}{\pgfqpoint{1.133257in}{2.380107in}}%
\pgfpathlineto{\pgfqpoint{1.133257in}{2.380107in}}%
\pgfpathclose%
\pgfusepath{stroke}%
\end{pgfscope}%
\begin{pgfscope}%
\pgfpathrectangle{\pgfqpoint{0.494722in}{0.437222in}}{\pgfqpoint{6.275590in}{5.159444in}}%
\pgfusepath{clip}%
\pgfsetbuttcap%
\pgfsetroundjoin%
\pgfsetlinewidth{1.003750pt}%
\definecolor{currentstroke}{rgb}{1.000000,0.000000,0.000000}%
\pgfsetstrokecolor{currentstroke}%
\pgfsetdash{}{0pt}%
\pgfpathmoveto{\pgfqpoint{1.426177in}{2.005975in}}%
\pgfpathcurveto{\pgfqpoint{1.437227in}{2.005975in}}{\pgfqpoint{1.447827in}{2.010366in}}{\pgfqpoint{1.455640in}{2.018179in}}%
\pgfpathcurveto{\pgfqpoint{1.463454in}{2.025993in}}{\pgfqpoint{1.467844in}{2.036592in}}{\pgfqpoint{1.467844in}{2.047642in}}%
\pgfpathcurveto{\pgfqpoint{1.467844in}{2.058692in}}{\pgfqpoint{1.463454in}{2.069291in}}{\pgfqpoint{1.455640in}{2.077105in}}%
\pgfpathcurveto{\pgfqpoint{1.447827in}{2.084918in}}{\pgfqpoint{1.437227in}{2.089309in}}{\pgfqpoint{1.426177in}{2.089309in}}%
\pgfpathcurveto{\pgfqpoint{1.415127in}{2.089309in}}{\pgfqpoint{1.404528in}{2.084918in}}{\pgfqpoint{1.396715in}{2.077105in}}%
\pgfpathcurveto{\pgfqpoint{1.388901in}{2.069291in}}{\pgfqpoint{1.384511in}{2.058692in}}{\pgfqpoint{1.384511in}{2.047642in}}%
\pgfpathcurveto{\pgfqpoint{1.384511in}{2.036592in}}{\pgfqpoint{1.388901in}{2.025993in}}{\pgfqpoint{1.396715in}{2.018179in}}%
\pgfpathcurveto{\pgfqpoint{1.404528in}{2.010366in}}{\pgfqpoint{1.415127in}{2.005975in}}{\pgfqpoint{1.426177in}{2.005975in}}%
\pgfpathlineto{\pgfqpoint{1.426177in}{2.005975in}}%
\pgfpathclose%
\pgfusepath{stroke}%
\end{pgfscope}%
\begin{pgfscope}%
\pgfpathrectangle{\pgfqpoint{0.494722in}{0.437222in}}{\pgfqpoint{6.275590in}{5.159444in}}%
\pgfusepath{clip}%
\pgfsetbuttcap%
\pgfsetroundjoin%
\pgfsetlinewidth{1.003750pt}%
\definecolor{currentstroke}{rgb}{1.000000,0.000000,0.000000}%
\pgfsetstrokecolor{currentstroke}%
\pgfsetdash{}{0pt}%
\pgfpathmoveto{\pgfqpoint{0.675241in}{3.343653in}}%
\pgfpathcurveto{\pgfqpoint{0.686291in}{3.343653in}}{\pgfqpoint{0.696890in}{3.348043in}}{\pgfqpoint{0.704704in}{3.355857in}}%
\pgfpathcurveto{\pgfqpoint{0.712517in}{3.363670in}}{\pgfqpoint{0.716908in}{3.374269in}}{\pgfqpoint{0.716908in}{3.385320in}}%
\pgfpathcurveto{\pgfqpoint{0.716908in}{3.396370in}}{\pgfqpoint{0.712517in}{3.406969in}}{\pgfqpoint{0.704704in}{3.414782in}}%
\pgfpathcurveto{\pgfqpoint{0.696890in}{3.422596in}}{\pgfqpoint{0.686291in}{3.426986in}}{\pgfqpoint{0.675241in}{3.426986in}}%
\pgfpathcurveto{\pgfqpoint{0.664191in}{3.426986in}}{\pgfqpoint{0.653592in}{3.422596in}}{\pgfqpoint{0.645778in}{3.414782in}}%
\pgfpathcurveto{\pgfqpoint{0.637965in}{3.406969in}}{\pgfqpoint{0.633574in}{3.396370in}}{\pgfqpoint{0.633574in}{3.385320in}}%
\pgfpathcurveto{\pgfqpoint{0.633574in}{3.374269in}}{\pgfqpoint{0.637965in}{3.363670in}}{\pgfqpoint{0.645778in}{3.355857in}}%
\pgfpathcurveto{\pgfqpoint{0.653592in}{3.348043in}}{\pgfqpoint{0.664191in}{3.343653in}}{\pgfqpoint{0.675241in}{3.343653in}}%
\pgfpathlineto{\pgfqpoint{0.675241in}{3.343653in}}%
\pgfpathclose%
\pgfusepath{stroke}%
\end{pgfscope}%
\begin{pgfscope}%
\pgfpathrectangle{\pgfqpoint{0.494722in}{0.437222in}}{\pgfqpoint{6.275590in}{5.159444in}}%
\pgfusepath{clip}%
\pgfsetbuttcap%
\pgfsetroundjoin%
\pgfsetlinewidth{1.003750pt}%
\definecolor{currentstroke}{rgb}{1.000000,0.000000,0.000000}%
\pgfsetstrokecolor{currentstroke}%
\pgfsetdash{}{0pt}%
\pgfpathmoveto{\pgfqpoint{3.307899in}{0.768490in}}%
\pgfpathcurveto{\pgfqpoint{3.318949in}{0.768490in}}{\pgfqpoint{3.329548in}{0.772880in}}{\pgfqpoint{3.337362in}{0.780694in}}%
\pgfpathcurveto{\pgfqpoint{3.345176in}{0.788507in}}{\pgfqpoint{3.349566in}{0.799106in}}{\pgfqpoint{3.349566in}{0.810157in}}%
\pgfpathcurveto{\pgfqpoint{3.349566in}{0.821207in}}{\pgfqpoint{3.345176in}{0.831806in}}{\pgfqpoint{3.337362in}{0.839619in}}%
\pgfpathcurveto{\pgfqpoint{3.329548in}{0.847433in}}{\pgfqpoint{3.318949in}{0.851823in}}{\pgfqpoint{3.307899in}{0.851823in}}%
\pgfpathcurveto{\pgfqpoint{3.296849in}{0.851823in}}{\pgfqpoint{3.286250in}{0.847433in}}{\pgfqpoint{3.278436in}{0.839619in}}%
\pgfpathcurveto{\pgfqpoint{3.270623in}{0.831806in}}{\pgfqpoint{3.266232in}{0.821207in}}{\pgfqpoint{3.266232in}{0.810157in}}%
\pgfpathcurveto{\pgfqpoint{3.266232in}{0.799106in}}{\pgfqpoint{3.270623in}{0.788507in}}{\pgfqpoint{3.278436in}{0.780694in}}%
\pgfpathcurveto{\pgfqpoint{3.286250in}{0.772880in}}{\pgfqpoint{3.296849in}{0.768490in}}{\pgfqpoint{3.307899in}{0.768490in}}%
\pgfpathlineto{\pgfqpoint{3.307899in}{0.768490in}}%
\pgfpathclose%
\pgfusepath{stroke}%
\end{pgfscope}%
\begin{pgfscope}%
\pgfpathrectangle{\pgfqpoint{0.494722in}{0.437222in}}{\pgfqpoint{6.275590in}{5.159444in}}%
\pgfusepath{clip}%
\pgfsetbuttcap%
\pgfsetroundjoin%
\pgfsetlinewidth{1.003750pt}%
\definecolor{currentstroke}{rgb}{1.000000,0.000000,0.000000}%
\pgfsetstrokecolor{currentstroke}%
\pgfsetdash{}{0pt}%
\pgfpathmoveto{\pgfqpoint{3.785365in}{0.613950in}}%
\pgfpathcurveto{\pgfqpoint{3.796415in}{0.613950in}}{\pgfqpoint{3.807014in}{0.618340in}}{\pgfqpoint{3.814828in}{0.626154in}}%
\pgfpathcurveto{\pgfqpoint{3.822641in}{0.633968in}}{\pgfqpoint{3.827032in}{0.644567in}}{\pgfqpoint{3.827032in}{0.655617in}}%
\pgfpathcurveto{\pgfqpoint{3.827032in}{0.666667in}}{\pgfqpoint{3.822641in}{0.677266in}}{\pgfqpoint{3.814828in}{0.685080in}}%
\pgfpathcurveto{\pgfqpoint{3.807014in}{0.692893in}}{\pgfqpoint{3.796415in}{0.697284in}}{\pgfqpoint{3.785365in}{0.697284in}}%
\pgfpathcurveto{\pgfqpoint{3.774315in}{0.697284in}}{\pgfqpoint{3.763716in}{0.692893in}}{\pgfqpoint{3.755902in}{0.685080in}}%
\pgfpathcurveto{\pgfqpoint{3.748089in}{0.677266in}}{\pgfqpoint{3.743698in}{0.666667in}}{\pgfqpoint{3.743698in}{0.655617in}}%
\pgfpathcurveto{\pgfqpoint{3.743698in}{0.644567in}}{\pgfqpoint{3.748089in}{0.633968in}}{\pgfqpoint{3.755902in}{0.626154in}}%
\pgfpathcurveto{\pgfqpoint{3.763716in}{0.618340in}}{\pgfqpoint{3.774315in}{0.613950in}}{\pgfqpoint{3.785365in}{0.613950in}}%
\pgfpathlineto{\pgfqpoint{3.785365in}{0.613950in}}%
\pgfpathclose%
\pgfusepath{stroke}%
\end{pgfscope}%
\begin{pgfscope}%
\pgfpathrectangle{\pgfqpoint{0.494722in}{0.437222in}}{\pgfqpoint{6.275590in}{5.159444in}}%
\pgfusepath{clip}%
\pgfsetbuttcap%
\pgfsetroundjoin%
\pgfsetlinewidth{1.003750pt}%
\definecolor{currentstroke}{rgb}{1.000000,0.000000,0.000000}%
\pgfsetstrokecolor{currentstroke}%
\pgfsetdash{}{0pt}%
\pgfpathmoveto{\pgfqpoint{0.761493in}{3.093125in}}%
\pgfpathcurveto{\pgfqpoint{0.772544in}{3.093125in}}{\pgfqpoint{0.783143in}{3.097516in}}{\pgfqpoint{0.790956in}{3.105329in}}%
\pgfpathcurveto{\pgfqpoint{0.798770in}{3.113143in}}{\pgfqpoint{0.803160in}{3.123742in}}{\pgfqpoint{0.803160in}{3.134792in}}%
\pgfpathcurveto{\pgfqpoint{0.803160in}{3.145842in}}{\pgfqpoint{0.798770in}{3.156441in}}{\pgfqpoint{0.790956in}{3.164255in}}%
\pgfpathcurveto{\pgfqpoint{0.783143in}{3.172068in}}{\pgfqpoint{0.772544in}{3.176459in}}{\pgfqpoint{0.761493in}{3.176459in}}%
\pgfpathcurveto{\pgfqpoint{0.750443in}{3.176459in}}{\pgfqpoint{0.739844in}{3.172068in}}{\pgfqpoint{0.732031in}{3.164255in}}%
\pgfpathcurveto{\pgfqpoint{0.724217in}{3.156441in}}{\pgfqpoint{0.719827in}{3.145842in}}{\pgfqpoint{0.719827in}{3.134792in}}%
\pgfpathcurveto{\pgfqpoint{0.719827in}{3.123742in}}{\pgfqpoint{0.724217in}{3.113143in}}{\pgfqpoint{0.732031in}{3.105329in}}%
\pgfpathcurveto{\pgfqpoint{0.739844in}{3.097516in}}{\pgfqpoint{0.750443in}{3.093125in}}{\pgfqpoint{0.761493in}{3.093125in}}%
\pgfpathlineto{\pgfqpoint{0.761493in}{3.093125in}}%
\pgfpathclose%
\pgfusepath{stroke}%
\end{pgfscope}%
\begin{pgfscope}%
\pgfpathrectangle{\pgfqpoint{0.494722in}{0.437222in}}{\pgfqpoint{6.275590in}{5.159444in}}%
\pgfusepath{clip}%
\pgfsetbuttcap%
\pgfsetroundjoin%
\pgfsetlinewidth{1.003750pt}%
\definecolor{currentstroke}{rgb}{1.000000,0.000000,0.000000}%
\pgfsetstrokecolor{currentstroke}%
\pgfsetdash{}{0pt}%
\pgfpathmoveto{\pgfqpoint{0.494841in}{4.608385in}}%
\pgfpathcurveto{\pgfqpoint{0.505891in}{4.608385in}}{\pgfqpoint{0.516490in}{4.612775in}}{\pgfqpoint{0.524304in}{4.620589in}}%
\pgfpathcurveto{\pgfqpoint{0.532117in}{4.628403in}}{\pgfqpoint{0.536508in}{4.639002in}}{\pgfqpoint{0.536508in}{4.650052in}}%
\pgfpathcurveto{\pgfqpoint{0.536508in}{4.661102in}}{\pgfqpoint{0.532117in}{4.671701in}}{\pgfqpoint{0.524304in}{4.679515in}}%
\pgfpathcurveto{\pgfqpoint{0.516490in}{4.687328in}}{\pgfqpoint{0.505891in}{4.691718in}}{\pgfqpoint{0.494841in}{4.691718in}}%
\pgfpathcurveto{\pgfqpoint{0.483791in}{4.691718in}}{\pgfqpoint{0.473192in}{4.687328in}}{\pgfqpoint{0.465378in}{4.679515in}}%
\pgfpathcurveto{\pgfqpoint{0.457565in}{4.671701in}}{\pgfqpoint{0.453174in}{4.661102in}}{\pgfqpoint{0.453174in}{4.650052in}}%
\pgfpathcurveto{\pgfqpoint{0.453174in}{4.639002in}}{\pgfqpoint{0.457565in}{4.628403in}}{\pgfqpoint{0.465378in}{4.620589in}}%
\pgfpathcurveto{\pgfqpoint{0.473192in}{4.612775in}}{\pgfqpoint{0.483791in}{4.608385in}}{\pgfqpoint{0.494841in}{4.608385in}}%
\pgfpathlineto{\pgfqpoint{0.494841in}{4.608385in}}%
\pgfpathclose%
\pgfusepath{stroke}%
\end{pgfscope}%
\begin{pgfscope}%
\pgfpathrectangle{\pgfqpoint{0.494722in}{0.437222in}}{\pgfqpoint{6.275590in}{5.159444in}}%
\pgfusepath{clip}%
\pgfsetbuttcap%
\pgfsetroundjoin%
\pgfsetlinewidth{1.003750pt}%
\definecolor{currentstroke}{rgb}{1.000000,0.000000,0.000000}%
\pgfsetstrokecolor{currentstroke}%
\pgfsetdash{}{0pt}%
\pgfpathmoveto{\pgfqpoint{1.714350in}{1.724510in}}%
\pgfpathcurveto{\pgfqpoint{1.725400in}{1.724510in}}{\pgfqpoint{1.735999in}{1.728901in}}{\pgfqpoint{1.743813in}{1.736714in}}%
\pgfpathcurveto{\pgfqpoint{1.751627in}{1.744528in}}{\pgfqpoint{1.756017in}{1.755127in}}{\pgfqpoint{1.756017in}{1.766177in}}%
\pgfpathcurveto{\pgfqpoint{1.756017in}{1.777227in}}{\pgfqpoint{1.751627in}{1.787826in}}{\pgfqpoint{1.743813in}{1.795640in}}%
\pgfpathcurveto{\pgfqpoint{1.735999in}{1.803453in}}{\pgfqpoint{1.725400in}{1.807844in}}{\pgfqpoint{1.714350in}{1.807844in}}%
\pgfpathcurveto{\pgfqpoint{1.703300in}{1.807844in}}{\pgfqpoint{1.692701in}{1.803453in}}{\pgfqpoint{1.684887in}{1.795640in}}%
\pgfpathcurveto{\pgfqpoint{1.677074in}{1.787826in}}{\pgfqpoint{1.672683in}{1.777227in}}{\pgfqpoint{1.672683in}{1.766177in}}%
\pgfpathcurveto{\pgfqpoint{1.672683in}{1.755127in}}{\pgfqpoint{1.677074in}{1.744528in}}{\pgfqpoint{1.684887in}{1.736714in}}%
\pgfpathcurveto{\pgfqpoint{1.692701in}{1.728901in}}{\pgfqpoint{1.703300in}{1.724510in}}{\pgfqpoint{1.714350in}{1.724510in}}%
\pgfpathlineto{\pgfqpoint{1.714350in}{1.724510in}}%
\pgfpathclose%
\pgfusepath{stroke}%
\end{pgfscope}%
\begin{pgfscope}%
\pgfpathrectangle{\pgfqpoint{0.494722in}{0.437222in}}{\pgfqpoint{6.275590in}{5.159444in}}%
\pgfusepath{clip}%
\pgfsetbuttcap%
\pgfsetroundjoin%
\pgfsetlinewidth{1.003750pt}%
\definecolor{currentstroke}{rgb}{1.000000,0.000000,0.000000}%
\pgfsetstrokecolor{currentstroke}%
\pgfsetdash{}{0pt}%
\pgfpathmoveto{\pgfqpoint{0.568080in}{3.809796in}}%
\pgfpathcurveto{\pgfqpoint{0.579130in}{3.809796in}}{\pgfqpoint{0.589729in}{3.814186in}}{\pgfqpoint{0.597543in}{3.822000in}}%
\pgfpathcurveto{\pgfqpoint{0.605356in}{3.829813in}}{\pgfqpoint{0.609746in}{3.840412in}}{\pgfqpoint{0.609746in}{3.851462in}}%
\pgfpathcurveto{\pgfqpoint{0.609746in}{3.862512in}}{\pgfqpoint{0.605356in}{3.873111in}}{\pgfqpoint{0.597543in}{3.880925in}}%
\pgfpathcurveto{\pgfqpoint{0.589729in}{3.888739in}}{\pgfqpoint{0.579130in}{3.893129in}}{\pgfqpoint{0.568080in}{3.893129in}}%
\pgfpathcurveto{\pgfqpoint{0.557030in}{3.893129in}}{\pgfqpoint{0.546431in}{3.888739in}}{\pgfqpoint{0.538617in}{3.880925in}}%
\pgfpathcurveto{\pgfqpoint{0.530803in}{3.873111in}}{\pgfqpoint{0.526413in}{3.862512in}}{\pgfqpoint{0.526413in}{3.851462in}}%
\pgfpathcurveto{\pgfqpoint{0.526413in}{3.840412in}}{\pgfqpoint{0.530803in}{3.829813in}}{\pgfqpoint{0.538617in}{3.822000in}}%
\pgfpathcurveto{\pgfqpoint{0.546431in}{3.814186in}}{\pgfqpoint{0.557030in}{3.809796in}}{\pgfqpoint{0.568080in}{3.809796in}}%
\pgfpathlineto{\pgfqpoint{0.568080in}{3.809796in}}%
\pgfpathclose%
\pgfusepath{stroke}%
\end{pgfscope}%
\begin{pgfscope}%
\pgfpathrectangle{\pgfqpoint{0.494722in}{0.437222in}}{\pgfqpoint{6.275590in}{5.159444in}}%
\pgfusepath{clip}%
\pgfsetbuttcap%
\pgfsetroundjoin%
\pgfsetlinewidth{1.003750pt}%
\definecolor{currentstroke}{rgb}{1.000000,0.000000,0.000000}%
\pgfsetstrokecolor{currentstroke}%
\pgfsetdash{}{0pt}%
\pgfpathmoveto{\pgfqpoint{1.211302in}{2.269892in}}%
\pgfpathcurveto{\pgfqpoint{1.222352in}{2.269892in}}{\pgfqpoint{1.232951in}{2.274283in}}{\pgfqpoint{1.240764in}{2.282096in}}%
\pgfpathcurveto{\pgfqpoint{1.248578in}{2.289910in}}{\pgfqpoint{1.252968in}{2.300509in}}{\pgfqpoint{1.252968in}{2.311559in}}%
\pgfpathcurveto{\pgfqpoint{1.252968in}{2.322609in}}{\pgfqpoint{1.248578in}{2.333208in}}{\pgfqpoint{1.240764in}{2.341022in}}%
\pgfpathcurveto{\pgfqpoint{1.232951in}{2.348836in}}{\pgfqpoint{1.222352in}{2.353226in}}{\pgfqpoint{1.211302in}{2.353226in}}%
\pgfpathcurveto{\pgfqpoint{1.200251in}{2.353226in}}{\pgfqpoint{1.189652in}{2.348836in}}{\pgfqpoint{1.181839in}{2.341022in}}%
\pgfpathcurveto{\pgfqpoint{1.174025in}{2.333208in}}{\pgfqpoint{1.169635in}{2.322609in}}{\pgfqpoint{1.169635in}{2.311559in}}%
\pgfpathcurveto{\pgfqpoint{1.169635in}{2.300509in}}{\pgfqpoint{1.174025in}{2.289910in}}{\pgfqpoint{1.181839in}{2.282096in}}%
\pgfpathcurveto{\pgfqpoint{1.189652in}{2.274283in}}{\pgfqpoint{1.200251in}{2.269892in}}{\pgfqpoint{1.211302in}{2.269892in}}%
\pgfpathlineto{\pgfqpoint{1.211302in}{2.269892in}}%
\pgfpathclose%
\pgfusepath{stroke}%
\end{pgfscope}%
\begin{pgfscope}%
\pgfpathrectangle{\pgfqpoint{0.494722in}{0.437222in}}{\pgfqpoint{6.275590in}{5.159444in}}%
\pgfusepath{clip}%
\pgfsetbuttcap%
\pgfsetroundjoin%
\pgfsetlinewidth{1.003750pt}%
\definecolor{currentstroke}{rgb}{1.000000,0.000000,0.000000}%
\pgfsetstrokecolor{currentstroke}%
\pgfsetdash{}{0pt}%
\pgfpathmoveto{\pgfqpoint{1.384216in}{2.068620in}}%
\pgfpathcurveto{\pgfqpoint{1.395266in}{2.068620in}}{\pgfqpoint{1.405865in}{2.073010in}}{\pgfqpoint{1.413678in}{2.080824in}}%
\pgfpathcurveto{\pgfqpoint{1.421492in}{2.088638in}}{\pgfqpoint{1.425882in}{2.099237in}}{\pgfqpoint{1.425882in}{2.110287in}}%
\pgfpathcurveto{\pgfqpoint{1.425882in}{2.121337in}}{\pgfqpoint{1.421492in}{2.131936in}}{\pgfqpoint{1.413678in}{2.139749in}}%
\pgfpathcurveto{\pgfqpoint{1.405865in}{2.147563in}}{\pgfqpoint{1.395266in}{2.151953in}}{\pgfqpoint{1.384216in}{2.151953in}}%
\pgfpathcurveto{\pgfqpoint{1.373166in}{2.151953in}}{\pgfqpoint{1.362566in}{2.147563in}}{\pgfqpoint{1.354753in}{2.139749in}}%
\pgfpathcurveto{\pgfqpoint{1.346939in}{2.131936in}}{\pgfqpoint{1.342549in}{2.121337in}}{\pgfqpoint{1.342549in}{2.110287in}}%
\pgfpathcurveto{\pgfqpoint{1.342549in}{2.099237in}}{\pgfqpoint{1.346939in}{2.088638in}}{\pgfqpoint{1.354753in}{2.080824in}}%
\pgfpathcurveto{\pgfqpoint{1.362566in}{2.073010in}}{\pgfqpoint{1.373166in}{2.068620in}}{\pgfqpoint{1.384216in}{2.068620in}}%
\pgfpathlineto{\pgfqpoint{1.384216in}{2.068620in}}%
\pgfpathclose%
\pgfusepath{stroke}%
\end{pgfscope}%
\begin{pgfscope}%
\pgfpathrectangle{\pgfqpoint{0.494722in}{0.437222in}}{\pgfqpoint{6.275590in}{5.159444in}}%
\pgfusepath{clip}%
\pgfsetbuttcap%
\pgfsetroundjoin%
\pgfsetlinewidth{1.003750pt}%
\definecolor{currentstroke}{rgb}{1.000000,0.000000,0.000000}%
\pgfsetstrokecolor{currentstroke}%
\pgfsetdash{}{0pt}%
\pgfpathmoveto{\pgfqpoint{1.178005in}{2.332742in}}%
\pgfpathcurveto{\pgfqpoint{1.189056in}{2.332742in}}{\pgfqpoint{1.199655in}{2.337133in}}{\pgfqpoint{1.207468in}{2.344946in}}%
\pgfpathcurveto{\pgfqpoint{1.215282in}{2.352760in}}{\pgfqpoint{1.219672in}{2.363359in}}{\pgfqpoint{1.219672in}{2.374409in}}%
\pgfpathcurveto{\pgfqpoint{1.219672in}{2.385459in}}{\pgfqpoint{1.215282in}{2.396058in}}{\pgfqpoint{1.207468in}{2.403872in}}%
\pgfpathcurveto{\pgfqpoint{1.199655in}{2.411685in}}{\pgfqpoint{1.189056in}{2.416076in}}{\pgfqpoint{1.178005in}{2.416076in}}%
\pgfpathcurveto{\pgfqpoint{1.166955in}{2.416076in}}{\pgfqpoint{1.156356in}{2.411685in}}{\pgfqpoint{1.148543in}{2.403872in}}%
\pgfpathcurveto{\pgfqpoint{1.140729in}{2.396058in}}{\pgfqpoint{1.136339in}{2.385459in}}{\pgfqpoint{1.136339in}{2.374409in}}%
\pgfpathcurveto{\pgfqpoint{1.136339in}{2.363359in}}{\pgfqpoint{1.140729in}{2.352760in}}{\pgfqpoint{1.148543in}{2.344946in}}%
\pgfpathcurveto{\pgfqpoint{1.156356in}{2.337133in}}{\pgfqpoint{1.166955in}{2.332742in}}{\pgfqpoint{1.178005in}{2.332742in}}%
\pgfpathlineto{\pgfqpoint{1.178005in}{2.332742in}}%
\pgfpathclose%
\pgfusepath{stroke}%
\end{pgfscope}%
\begin{pgfscope}%
\pgfpathrectangle{\pgfqpoint{0.494722in}{0.437222in}}{\pgfqpoint{6.275590in}{5.159444in}}%
\pgfusepath{clip}%
\pgfsetbuttcap%
\pgfsetroundjoin%
\pgfsetlinewidth{1.003750pt}%
\definecolor{currentstroke}{rgb}{1.000000,0.000000,0.000000}%
\pgfsetstrokecolor{currentstroke}%
\pgfsetdash{}{0pt}%
\pgfpathmoveto{\pgfqpoint{1.333008in}{2.124247in}}%
\pgfpathcurveto{\pgfqpoint{1.344058in}{2.124247in}}{\pgfqpoint{1.354657in}{2.128637in}}{\pgfqpoint{1.362471in}{2.136451in}}%
\pgfpathcurveto{\pgfqpoint{1.370284in}{2.144265in}}{\pgfqpoint{1.374675in}{2.154864in}}{\pgfqpoint{1.374675in}{2.165914in}}%
\pgfpathcurveto{\pgfqpoint{1.374675in}{2.176964in}}{\pgfqpoint{1.370284in}{2.187563in}}{\pgfqpoint{1.362471in}{2.195376in}}%
\pgfpathcurveto{\pgfqpoint{1.354657in}{2.203190in}}{\pgfqpoint{1.344058in}{2.207580in}}{\pgfqpoint{1.333008in}{2.207580in}}%
\pgfpathcurveto{\pgfqpoint{1.321958in}{2.207580in}}{\pgfqpoint{1.311359in}{2.203190in}}{\pgfqpoint{1.303545in}{2.195376in}}%
\pgfpathcurveto{\pgfqpoint{1.295731in}{2.187563in}}{\pgfqpoint{1.291341in}{2.176964in}}{\pgfqpoint{1.291341in}{2.165914in}}%
\pgfpathcurveto{\pgfqpoint{1.291341in}{2.154864in}}{\pgfqpoint{1.295731in}{2.144265in}}{\pgfqpoint{1.303545in}{2.136451in}}%
\pgfpathcurveto{\pgfqpoint{1.311359in}{2.128637in}}{\pgfqpoint{1.321958in}{2.124247in}}{\pgfqpoint{1.333008in}{2.124247in}}%
\pgfpathlineto{\pgfqpoint{1.333008in}{2.124247in}}%
\pgfpathclose%
\pgfusepath{stroke}%
\end{pgfscope}%
\begin{pgfscope}%
\pgfpathrectangle{\pgfqpoint{0.494722in}{0.437222in}}{\pgfqpoint{6.275590in}{5.159444in}}%
\pgfusepath{clip}%
\pgfsetbuttcap%
\pgfsetroundjoin%
\pgfsetlinewidth{1.003750pt}%
\definecolor{currentstroke}{rgb}{1.000000,0.000000,0.000000}%
\pgfsetstrokecolor{currentstroke}%
\pgfsetdash{}{0pt}%
\pgfpathmoveto{\pgfqpoint{1.867916in}{1.601903in}}%
\pgfpathcurveto{\pgfqpoint{1.878966in}{1.601903in}}{\pgfqpoint{1.889565in}{1.606293in}}{\pgfqpoint{1.897379in}{1.614107in}}%
\pgfpathcurveto{\pgfqpoint{1.905192in}{1.621921in}}{\pgfqpoint{1.909583in}{1.632520in}}{\pgfqpoint{1.909583in}{1.643570in}}%
\pgfpathcurveto{\pgfqpoint{1.909583in}{1.654620in}}{\pgfqpoint{1.905192in}{1.665219in}}{\pgfqpoint{1.897379in}{1.673033in}}%
\pgfpathcurveto{\pgfqpoint{1.889565in}{1.680846in}}{\pgfqpoint{1.878966in}{1.685237in}}{\pgfqpoint{1.867916in}{1.685237in}}%
\pgfpathcurveto{\pgfqpoint{1.856866in}{1.685237in}}{\pgfqpoint{1.846267in}{1.680846in}}{\pgfqpoint{1.838453in}{1.673033in}}%
\pgfpathcurveto{\pgfqpoint{1.830640in}{1.665219in}}{\pgfqpoint{1.826249in}{1.654620in}}{\pgfqpoint{1.826249in}{1.643570in}}%
\pgfpathcurveto{\pgfqpoint{1.826249in}{1.632520in}}{\pgfqpoint{1.830640in}{1.621921in}}{\pgfqpoint{1.838453in}{1.614107in}}%
\pgfpathcurveto{\pgfqpoint{1.846267in}{1.606293in}}{\pgfqpoint{1.856866in}{1.601903in}}{\pgfqpoint{1.867916in}{1.601903in}}%
\pgfpathlineto{\pgfqpoint{1.867916in}{1.601903in}}%
\pgfpathclose%
\pgfusepath{stroke}%
\end{pgfscope}%
\begin{pgfscope}%
\pgfpathrectangle{\pgfqpoint{0.494722in}{0.437222in}}{\pgfqpoint{6.275590in}{5.159444in}}%
\pgfusepath{clip}%
\pgfsetbuttcap%
\pgfsetroundjoin%
\pgfsetlinewidth{1.003750pt}%
\definecolor{currentstroke}{rgb}{1.000000,0.000000,0.000000}%
\pgfsetstrokecolor{currentstroke}%
\pgfsetdash{}{0pt}%
\pgfpathmoveto{\pgfqpoint{0.689837in}{3.306388in}}%
\pgfpathcurveto{\pgfqpoint{0.700887in}{3.306388in}}{\pgfqpoint{0.711486in}{3.310778in}}{\pgfqpoint{0.719300in}{3.318592in}}%
\pgfpathcurveto{\pgfqpoint{0.727113in}{3.326405in}}{\pgfqpoint{0.731504in}{3.337004in}}{\pgfqpoint{0.731504in}{3.348055in}}%
\pgfpathcurveto{\pgfqpoint{0.731504in}{3.359105in}}{\pgfqpoint{0.727113in}{3.369704in}}{\pgfqpoint{0.719300in}{3.377517in}}%
\pgfpathcurveto{\pgfqpoint{0.711486in}{3.385331in}}{\pgfqpoint{0.700887in}{3.389721in}}{\pgfqpoint{0.689837in}{3.389721in}}%
\pgfpathcurveto{\pgfqpoint{0.678787in}{3.389721in}}{\pgfqpoint{0.668188in}{3.385331in}}{\pgfqpoint{0.660374in}{3.377517in}}%
\pgfpathcurveto{\pgfqpoint{0.652560in}{3.369704in}}{\pgfqpoint{0.648170in}{3.359105in}}{\pgfqpoint{0.648170in}{3.348055in}}%
\pgfpathcurveto{\pgfqpoint{0.648170in}{3.337004in}}{\pgfqpoint{0.652560in}{3.326405in}}{\pgfqpoint{0.660374in}{3.318592in}}%
\pgfpathcurveto{\pgfqpoint{0.668188in}{3.310778in}}{\pgfqpoint{0.678787in}{3.306388in}}{\pgfqpoint{0.689837in}{3.306388in}}%
\pgfpathlineto{\pgfqpoint{0.689837in}{3.306388in}}%
\pgfpathclose%
\pgfusepath{stroke}%
\end{pgfscope}%
\begin{pgfscope}%
\pgfpathrectangle{\pgfqpoint{0.494722in}{0.437222in}}{\pgfqpoint{6.275590in}{5.159444in}}%
\pgfusepath{clip}%
\pgfsetbuttcap%
\pgfsetroundjoin%
\pgfsetlinewidth{1.003750pt}%
\definecolor{currentstroke}{rgb}{1.000000,0.000000,0.000000}%
\pgfsetstrokecolor{currentstroke}%
\pgfsetdash{}{0pt}%
\pgfpathmoveto{\pgfqpoint{0.549348in}{3.900862in}}%
\pgfpathcurveto{\pgfqpoint{0.560399in}{3.900862in}}{\pgfqpoint{0.570998in}{3.905252in}}{\pgfqpoint{0.578811in}{3.913066in}}%
\pgfpathcurveto{\pgfqpoint{0.586625in}{3.920879in}}{\pgfqpoint{0.591015in}{3.931478in}}{\pgfqpoint{0.591015in}{3.942529in}}%
\pgfpathcurveto{\pgfqpoint{0.591015in}{3.953579in}}{\pgfqpoint{0.586625in}{3.964178in}}{\pgfqpoint{0.578811in}{3.971991in}}%
\pgfpathcurveto{\pgfqpoint{0.570998in}{3.979805in}}{\pgfqpoint{0.560399in}{3.984195in}}{\pgfqpoint{0.549348in}{3.984195in}}%
\pgfpathcurveto{\pgfqpoint{0.538298in}{3.984195in}}{\pgfqpoint{0.527699in}{3.979805in}}{\pgfqpoint{0.519886in}{3.971991in}}%
\pgfpathcurveto{\pgfqpoint{0.512072in}{3.964178in}}{\pgfqpoint{0.507682in}{3.953579in}}{\pgfqpoint{0.507682in}{3.942529in}}%
\pgfpathcurveto{\pgfqpoint{0.507682in}{3.931478in}}{\pgfqpoint{0.512072in}{3.920879in}}{\pgfqpoint{0.519886in}{3.913066in}}%
\pgfpathcurveto{\pgfqpoint{0.527699in}{3.905252in}}{\pgfqpoint{0.538298in}{3.900862in}}{\pgfqpoint{0.549348in}{3.900862in}}%
\pgfpathlineto{\pgfqpoint{0.549348in}{3.900862in}}%
\pgfpathclose%
\pgfusepath{stroke}%
\end{pgfscope}%
\begin{pgfscope}%
\pgfpathrectangle{\pgfqpoint{0.494722in}{0.437222in}}{\pgfqpoint{6.275590in}{5.159444in}}%
\pgfusepath{clip}%
\pgfsetbuttcap%
\pgfsetroundjoin%
\pgfsetlinewidth{1.003750pt}%
\definecolor{currentstroke}{rgb}{1.000000,0.000000,0.000000}%
\pgfsetstrokecolor{currentstroke}%
\pgfsetdash{}{0pt}%
\pgfpathmoveto{\pgfqpoint{1.771573in}{1.681022in}}%
\pgfpathcurveto{\pgfqpoint{1.782623in}{1.681022in}}{\pgfqpoint{1.793222in}{1.685412in}}{\pgfqpoint{1.801036in}{1.693226in}}%
\pgfpathcurveto{\pgfqpoint{1.808849in}{1.701040in}}{\pgfqpoint{1.813240in}{1.711639in}}{\pgfqpoint{1.813240in}{1.722689in}}%
\pgfpathcurveto{\pgfqpoint{1.813240in}{1.733739in}}{\pgfqpoint{1.808849in}{1.744338in}}{\pgfqpoint{1.801036in}{1.752151in}}%
\pgfpathcurveto{\pgfqpoint{1.793222in}{1.759965in}}{\pgfqpoint{1.782623in}{1.764355in}}{\pgfqpoint{1.771573in}{1.764355in}}%
\pgfpathcurveto{\pgfqpoint{1.760523in}{1.764355in}}{\pgfqpoint{1.749924in}{1.759965in}}{\pgfqpoint{1.742110in}{1.752151in}}%
\pgfpathcurveto{\pgfqpoint{1.734297in}{1.744338in}}{\pgfqpoint{1.729906in}{1.733739in}}{\pgfqpoint{1.729906in}{1.722689in}}%
\pgfpathcurveto{\pgfqpoint{1.729906in}{1.711639in}}{\pgfqpoint{1.734297in}{1.701040in}}{\pgfqpoint{1.742110in}{1.693226in}}%
\pgfpathcurveto{\pgfqpoint{1.749924in}{1.685412in}}{\pgfqpoint{1.760523in}{1.681022in}}{\pgfqpoint{1.771573in}{1.681022in}}%
\pgfpathlineto{\pgfqpoint{1.771573in}{1.681022in}}%
\pgfpathclose%
\pgfusepath{stroke}%
\end{pgfscope}%
\begin{pgfscope}%
\pgfpathrectangle{\pgfqpoint{0.494722in}{0.437222in}}{\pgfqpoint{6.275590in}{5.159444in}}%
\pgfusepath{clip}%
\pgfsetbuttcap%
\pgfsetroundjoin%
\pgfsetlinewidth{1.003750pt}%
\definecolor{currentstroke}{rgb}{1.000000,0.000000,0.000000}%
\pgfsetstrokecolor{currentstroke}%
\pgfsetdash{}{0pt}%
\pgfpathmoveto{\pgfqpoint{0.811435in}{2.968497in}}%
\pgfpathcurveto{\pgfqpoint{0.822485in}{2.968497in}}{\pgfqpoint{0.833084in}{2.972888in}}{\pgfqpoint{0.840897in}{2.980701in}}%
\pgfpathcurveto{\pgfqpoint{0.848711in}{2.988515in}}{\pgfqpoint{0.853101in}{2.999114in}}{\pgfqpoint{0.853101in}{3.010164in}}%
\pgfpathcurveto{\pgfqpoint{0.853101in}{3.021214in}}{\pgfqpoint{0.848711in}{3.031813in}}{\pgfqpoint{0.840897in}{3.039627in}}%
\pgfpathcurveto{\pgfqpoint{0.833084in}{3.047441in}}{\pgfqpoint{0.822485in}{3.051831in}}{\pgfqpoint{0.811435in}{3.051831in}}%
\pgfpathcurveto{\pgfqpoint{0.800385in}{3.051831in}}{\pgfqpoint{0.789786in}{3.047441in}}{\pgfqpoint{0.781972in}{3.039627in}}%
\pgfpathcurveto{\pgfqpoint{0.774158in}{3.031813in}}{\pgfqpoint{0.769768in}{3.021214in}}{\pgfqpoint{0.769768in}{3.010164in}}%
\pgfpathcurveto{\pgfqpoint{0.769768in}{2.999114in}}{\pgfqpoint{0.774158in}{2.988515in}}{\pgfqpoint{0.781972in}{2.980701in}}%
\pgfpathcurveto{\pgfqpoint{0.789786in}{2.972888in}}{\pgfqpoint{0.800385in}{2.968497in}}{\pgfqpoint{0.811435in}{2.968497in}}%
\pgfpathlineto{\pgfqpoint{0.811435in}{2.968497in}}%
\pgfpathclose%
\pgfusepath{stroke}%
\end{pgfscope}%
\begin{pgfscope}%
\pgfpathrectangle{\pgfqpoint{0.494722in}{0.437222in}}{\pgfqpoint{6.275590in}{5.159444in}}%
\pgfusepath{clip}%
\pgfsetbuttcap%
\pgfsetroundjoin%
\pgfsetlinewidth{1.003750pt}%
\definecolor{currentstroke}{rgb}{1.000000,0.000000,0.000000}%
\pgfsetstrokecolor{currentstroke}%
\pgfsetdash{}{0pt}%
\pgfpathmoveto{\pgfqpoint{1.017588in}{2.562243in}}%
\pgfpathcurveto{\pgfqpoint{1.028638in}{2.562243in}}{\pgfqpoint{1.039237in}{2.566633in}}{\pgfqpoint{1.047050in}{2.574447in}}%
\pgfpathcurveto{\pgfqpoint{1.054864in}{2.582261in}}{\pgfqpoint{1.059254in}{2.592860in}}{\pgfqpoint{1.059254in}{2.603910in}}%
\pgfpathcurveto{\pgfqpoint{1.059254in}{2.614960in}}{\pgfqpoint{1.054864in}{2.625559in}}{\pgfqpoint{1.047050in}{2.633372in}}%
\pgfpathcurveto{\pgfqpoint{1.039237in}{2.641186in}}{\pgfqpoint{1.028638in}{2.645576in}}{\pgfqpoint{1.017588in}{2.645576in}}%
\pgfpathcurveto{\pgfqpoint{1.006537in}{2.645576in}}{\pgfqpoint{0.995938in}{2.641186in}}{\pgfqpoint{0.988125in}{2.633372in}}%
\pgfpathcurveto{\pgfqpoint{0.980311in}{2.625559in}}{\pgfqpoint{0.975921in}{2.614960in}}{\pgfqpoint{0.975921in}{2.603910in}}%
\pgfpathcurveto{\pgfqpoint{0.975921in}{2.592860in}}{\pgfqpoint{0.980311in}{2.582261in}}{\pgfqpoint{0.988125in}{2.574447in}}%
\pgfpathcurveto{\pgfqpoint{0.995938in}{2.566633in}}{\pgfqpoint{1.006537in}{2.562243in}}{\pgfqpoint{1.017588in}{2.562243in}}%
\pgfpathlineto{\pgfqpoint{1.017588in}{2.562243in}}%
\pgfpathclose%
\pgfusepath{stroke}%
\end{pgfscope}%
\begin{pgfscope}%
\pgfpathrectangle{\pgfqpoint{0.494722in}{0.437222in}}{\pgfqpoint{6.275590in}{5.159444in}}%
\pgfusepath{clip}%
\pgfsetbuttcap%
\pgfsetroundjoin%
\pgfsetlinewidth{1.003750pt}%
\definecolor{currentstroke}{rgb}{1.000000,0.000000,0.000000}%
\pgfsetstrokecolor{currentstroke}%
\pgfsetdash{}{0pt}%
\pgfpathmoveto{\pgfqpoint{3.636012in}{0.666302in}}%
\pgfpathcurveto{\pgfqpoint{3.647062in}{0.666302in}}{\pgfqpoint{3.657661in}{0.670692in}}{\pgfqpoint{3.665475in}{0.678506in}}%
\pgfpathcurveto{\pgfqpoint{3.673288in}{0.686319in}}{\pgfqpoint{3.677679in}{0.696918in}}{\pgfqpoint{3.677679in}{0.707968in}}%
\pgfpathcurveto{\pgfqpoint{3.677679in}{0.719019in}}{\pgfqpoint{3.673288in}{0.729618in}}{\pgfqpoint{3.665475in}{0.737431in}}%
\pgfpathcurveto{\pgfqpoint{3.657661in}{0.745245in}}{\pgfqpoint{3.647062in}{0.749635in}}{\pgfqpoint{3.636012in}{0.749635in}}%
\pgfpathcurveto{\pgfqpoint{3.624962in}{0.749635in}}{\pgfqpoint{3.614363in}{0.745245in}}{\pgfqpoint{3.606549in}{0.737431in}}%
\pgfpathcurveto{\pgfqpoint{3.598735in}{0.729618in}}{\pgfqpoint{3.594345in}{0.719019in}}{\pgfqpoint{3.594345in}{0.707968in}}%
\pgfpathcurveto{\pgfqpoint{3.594345in}{0.696918in}}{\pgfqpoint{3.598735in}{0.686319in}}{\pgfqpoint{3.606549in}{0.678506in}}%
\pgfpathcurveto{\pgfqpoint{3.614363in}{0.670692in}}{\pgfqpoint{3.624962in}{0.666302in}}{\pgfqpoint{3.636012in}{0.666302in}}%
\pgfpathlineto{\pgfqpoint{3.636012in}{0.666302in}}%
\pgfpathclose%
\pgfusepath{stroke}%
\end{pgfscope}%
\begin{pgfscope}%
\pgfpathrectangle{\pgfqpoint{0.494722in}{0.437222in}}{\pgfqpoint{6.275590in}{5.159444in}}%
\pgfusepath{clip}%
\pgfsetbuttcap%
\pgfsetroundjoin%
\pgfsetlinewidth{1.003750pt}%
\definecolor{currentstroke}{rgb}{1.000000,0.000000,0.000000}%
\pgfsetstrokecolor{currentstroke}%
\pgfsetdash{}{0pt}%
\pgfpathmoveto{\pgfqpoint{0.496024in}{4.518762in}}%
\pgfpathcurveto{\pgfqpoint{0.507074in}{4.518762in}}{\pgfqpoint{0.517673in}{4.523152in}}{\pgfqpoint{0.525486in}{4.530966in}}%
\pgfpathcurveto{\pgfqpoint{0.533300in}{4.538780in}}{\pgfqpoint{0.537690in}{4.549379in}}{\pgfqpoint{0.537690in}{4.560429in}}%
\pgfpathcurveto{\pgfqpoint{0.537690in}{4.571479in}}{\pgfqpoint{0.533300in}{4.582078in}}{\pgfqpoint{0.525486in}{4.589892in}}%
\pgfpathcurveto{\pgfqpoint{0.517673in}{4.597705in}}{\pgfqpoint{0.507074in}{4.602096in}}{\pgfqpoint{0.496024in}{4.602096in}}%
\pgfpathcurveto{\pgfqpoint{0.484974in}{4.602096in}}{\pgfqpoint{0.474374in}{4.597705in}}{\pgfqpoint{0.466561in}{4.589892in}}%
\pgfpathcurveto{\pgfqpoint{0.458747in}{4.582078in}}{\pgfqpoint{0.454357in}{4.571479in}}{\pgfqpoint{0.454357in}{4.560429in}}%
\pgfpathcurveto{\pgfqpoint{0.454357in}{4.549379in}}{\pgfqpoint{0.458747in}{4.538780in}}{\pgfqpoint{0.466561in}{4.530966in}}%
\pgfpathcurveto{\pgfqpoint{0.474374in}{4.523152in}}{\pgfqpoint{0.484974in}{4.518762in}}{\pgfqpoint{0.496024in}{4.518762in}}%
\pgfpathlineto{\pgfqpoint{0.496024in}{4.518762in}}%
\pgfpathclose%
\pgfusepath{stroke}%
\end{pgfscope}%
\begin{pgfscope}%
\pgfpathrectangle{\pgfqpoint{0.494722in}{0.437222in}}{\pgfqpoint{6.275590in}{5.159444in}}%
\pgfusepath{clip}%
\pgfsetbuttcap%
\pgfsetroundjoin%
\pgfsetlinewidth{1.003750pt}%
\definecolor{currentstroke}{rgb}{1.000000,0.000000,0.000000}%
\pgfsetstrokecolor{currentstroke}%
\pgfsetdash{}{0pt}%
\pgfpathmoveto{\pgfqpoint{3.557607in}{0.679710in}}%
\pgfpathcurveto{\pgfqpoint{3.568657in}{0.679710in}}{\pgfqpoint{3.579256in}{0.684101in}}{\pgfqpoint{3.587070in}{0.691914in}}%
\pgfpathcurveto{\pgfqpoint{3.594883in}{0.699728in}}{\pgfqpoint{3.599274in}{0.710327in}}{\pgfqpoint{3.599274in}{0.721377in}}%
\pgfpathcurveto{\pgfqpoint{3.599274in}{0.732427in}}{\pgfqpoint{3.594883in}{0.743026in}}{\pgfqpoint{3.587070in}{0.750840in}}%
\pgfpathcurveto{\pgfqpoint{3.579256in}{0.758653in}}{\pgfqpoint{3.568657in}{0.763044in}}{\pgfqpoint{3.557607in}{0.763044in}}%
\pgfpathcurveto{\pgfqpoint{3.546557in}{0.763044in}}{\pgfqpoint{3.535958in}{0.758653in}}{\pgfqpoint{3.528144in}{0.750840in}}%
\pgfpathcurveto{\pgfqpoint{3.520330in}{0.743026in}}{\pgfqpoint{3.515940in}{0.732427in}}{\pgfqpoint{3.515940in}{0.721377in}}%
\pgfpathcurveto{\pgfqpoint{3.515940in}{0.710327in}}{\pgfqpoint{3.520330in}{0.699728in}}{\pgfqpoint{3.528144in}{0.691914in}}%
\pgfpathcurveto{\pgfqpoint{3.535958in}{0.684101in}}{\pgfqpoint{3.546557in}{0.679710in}}{\pgfqpoint{3.557607in}{0.679710in}}%
\pgfpathlineto{\pgfqpoint{3.557607in}{0.679710in}}%
\pgfpathclose%
\pgfusepath{stroke}%
\end{pgfscope}%
\begin{pgfscope}%
\pgfpathrectangle{\pgfqpoint{0.494722in}{0.437222in}}{\pgfqpoint{6.275590in}{5.159444in}}%
\pgfusepath{clip}%
\pgfsetbuttcap%
\pgfsetroundjoin%
\pgfsetlinewidth{1.003750pt}%
\definecolor{currentstroke}{rgb}{1.000000,0.000000,0.000000}%
\pgfsetstrokecolor{currentstroke}%
\pgfsetdash{}{0pt}%
\pgfpathmoveto{\pgfqpoint{0.564033in}{3.841490in}}%
\pgfpathcurveto{\pgfqpoint{0.575083in}{3.841490in}}{\pgfqpoint{0.585682in}{3.845881in}}{\pgfqpoint{0.593496in}{3.853694in}}%
\pgfpathcurveto{\pgfqpoint{0.601310in}{3.861508in}}{\pgfqpoint{0.605700in}{3.872107in}}{\pgfqpoint{0.605700in}{3.883157in}}%
\pgfpathcurveto{\pgfqpoint{0.605700in}{3.894207in}}{\pgfqpoint{0.601310in}{3.904806in}}{\pgfqpoint{0.593496in}{3.912620in}}%
\pgfpathcurveto{\pgfqpoint{0.585682in}{3.920433in}}{\pgfqpoint{0.575083in}{3.924824in}}{\pgfqpoint{0.564033in}{3.924824in}}%
\pgfpathcurveto{\pgfqpoint{0.552983in}{3.924824in}}{\pgfqpoint{0.542384in}{3.920433in}}{\pgfqpoint{0.534570in}{3.912620in}}%
\pgfpathcurveto{\pgfqpoint{0.526757in}{3.904806in}}{\pgfqpoint{0.522366in}{3.894207in}}{\pgfqpoint{0.522366in}{3.883157in}}%
\pgfpathcurveto{\pgfqpoint{0.522366in}{3.872107in}}{\pgfqpoint{0.526757in}{3.861508in}}{\pgfqpoint{0.534570in}{3.853694in}}%
\pgfpathcurveto{\pgfqpoint{0.542384in}{3.845881in}}{\pgfqpoint{0.552983in}{3.841490in}}{\pgfqpoint{0.564033in}{3.841490in}}%
\pgfpathlineto{\pgfqpoint{0.564033in}{3.841490in}}%
\pgfpathclose%
\pgfusepath{stroke}%
\end{pgfscope}%
\begin{pgfscope}%
\pgfpathrectangle{\pgfqpoint{0.494722in}{0.437222in}}{\pgfqpoint{6.275590in}{5.159444in}}%
\pgfusepath{clip}%
\pgfsetbuttcap%
\pgfsetroundjoin%
\pgfsetlinewidth{1.003750pt}%
\definecolor{currentstroke}{rgb}{1.000000,0.000000,0.000000}%
\pgfsetstrokecolor{currentstroke}%
\pgfsetdash{}{0pt}%
\pgfpathmoveto{\pgfqpoint{2.341847in}{1.248275in}}%
\pgfpathcurveto{\pgfqpoint{2.352897in}{1.248275in}}{\pgfqpoint{2.363496in}{1.252665in}}{\pgfqpoint{2.371309in}{1.260479in}}%
\pgfpathcurveto{\pgfqpoint{2.379123in}{1.268292in}}{\pgfqpoint{2.383513in}{1.278891in}}{\pgfqpoint{2.383513in}{1.289942in}}%
\pgfpathcurveto{\pgfqpoint{2.383513in}{1.300992in}}{\pgfqpoint{2.379123in}{1.311591in}}{\pgfqpoint{2.371309in}{1.319404in}}%
\pgfpathcurveto{\pgfqpoint{2.363496in}{1.327218in}}{\pgfqpoint{2.352897in}{1.331608in}}{\pgfqpoint{2.341847in}{1.331608in}}%
\pgfpathcurveto{\pgfqpoint{2.330797in}{1.331608in}}{\pgfqpoint{2.320198in}{1.327218in}}{\pgfqpoint{2.312384in}{1.319404in}}%
\pgfpathcurveto{\pgfqpoint{2.304570in}{1.311591in}}{\pgfqpoint{2.300180in}{1.300992in}}{\pgfqpoint{2.300180in}{1.289942in}}%
\pgfpathcurveto{\pgfqpoint{2.300180in}{1.278891in}}{\pgfqpoint{2.304570in}{1.268292in}}{\pgfqpoint{2.312384in}{1.260479in}}%
\pgfpathcurveto{\pgfqpoint{2.320198in}{1.252665in}}{\pgfqpoint{2.330797in}{1.248275in}}{\pgfqpoint{2.341847in}{1.248275in}}%
\pgfpathlineto{\pgfqpoint{2.341847in}{1.248275in}}%
\pgfpathclose%
\pgfusepath{stroke}%
\end{pgfscope}%
\begin{pgfscope}%
\pgfpathrectangle{\pgfqpoint{0.494722in}{0.437222in}}{\pgfqpoint{6.275590in}{5.159444in}}%
\pgfusepath{clip}%
\pgfsetbuttcap%
\pgfsetroundjoin%
\pgfsetlinewidth{1.003750pt}%
\definecolor{currentstroke}{rgb}{1.000000,0.000000,0.000000}%
\pgfsetstrokecolor{currentstroke}%
\pgfsetdash{}{0pt}%
\pgfpathmoveto{\pgfqpoint{4.382334in}{0.496352in}}%
\pgfpathcurveto{\pgfqpoint{4.393384in}{0.496352in}}{\pgfqpoint{4.403983in}{0.500742in}}{\pgfqpoint{4.411796in}{0.508556in}}%
\pgfpathcurveto{\pgfqpoint{4.419610in}{0.516370in}}{\pgfqpoint{4.424000in}{0.526969in}}{\pgfqpoint{4.424000in}{0.538019in}}%
\pgfpathcurveto{\pgfqpoint{4.424000in}{0.549069in}}{\pgfqpoint{4.419610in}{0.559668in}}{\pgfqpoint{4.411796in}{0.567482in}}%
\pgfpathcurveto{\pgfqpoint{4.403983in}{0.575295in}}{\pgfqpoint{4.393384in}{0.579686in}}{\pgfqpoint{4.382334in}{0.579686in}}%
\pgfpathcurveto{\pgfqpoint{4.371284in}{0.579686in}}{\pgfqpoint{4.360684in}{0.575295in}}{\pgfqpoint{4.352871in}{0.567482in}}%
\pgfpathcurveto{\pgfqpoint{4.345057in}{0.559668in}}{\pgfqpoint{4.340667in}{0.549069in}}{\pgfqpoint{4.340667in}{0.538019in}}%
\pgfpathcurveto{\pgfqpoint{4.340667in}{0.526969in}}{\pgfqpoint{4.345057in}{0.516370in}}{\pgfqpoint{4.352871in}{0.508556in}}%
\pgfpathcurveto{\pgfqpoint{4.360684in}{0.500742in}}{\pgfqpoint{4.371284in}{0.496352in}}{\pgfqpoint{4.382334in}{0.496352in}}%
\pgfpathlineto{\pgfqpoint{4.382334in}{0.496352in}}%
\pgfpathclose%
\pgfusepath{stroke}%
\end{pgfscope}%
\begin{pgfscope}%
\pgfpathrectangle{\pgfqpoint{0.494722in}{0.437222in}}{\pgfqpoint{6.275590in}{5.159444in}}%
\pgfusepath{clip}%
\pgfsetbuttcap%
\pgfsetroundjoin%
\pgfsetlinewidth{1.003750pt}%
\definecolor{currentstroke}{rgb}{1.000000,0.000000,0.000000}%
\pgfsetstrokecolor{currentstroke}%
\pgfsetdash{}{0pt}%
\pgfpathmoveto{\pgfqpoint{4.468121in}{0.486014in}}%
\pgfpathcurveto{\pgfqpoint{4.479171in}{0.486014in}}{\pgfqpoint{4.489770in}{0.490404in}}{\pgfqpoint{4.497584in}{0.498218in}}%
\pgfpathcurveto{\pgfqpoint{4.505398in}{0.506032in}}{\pgfqpoint{4.509788in}{0.516631in}}{\pgfqpoint{4.509788in}{0.527681in}}%
\pgfpathcurveto{\pgfqpoint{4.509788in}{0.538731in}}{\pgfqpoint{4.505398in}{0.549330in}}{\pgfqpoint{4.497584in}{0.557144in}}%
\pgfpathcurveto{\pgfqpoint{4.489770in}{0.564957in}}{\pgfqpoint{4.479171in}{0.569347in}}{\pgfqpoint{4.468121in}{0.569347in}}%
\pgfpathcurveto{\pgfqpoint{4.457071in}{0.569347in}}{\pgfqpoint{4.446472in}{0.564957in}}{\pgfqpoint{4.438658in}{0.557144in}}%
\pgfpathcurveto{\pgfqpoint{4.430845in}{0.549330in}}{\pgfqpoint{4.426454in}{0.538731in}}{\pgfqpoint{4.426454in}{0.527681in}}%
\pgfpathcurveto{\pgfqpoint{4.426454in}{0.516631in}}{\pgfqpoint{4.430845in}{0.506032in}}{\pgfqpoint{4.438658in}{0.498218in}}%
\pgfpathcurveto{\pgfqpoint{4.446472in}{0.490404in}}{\pgfqpoint{4.457071in}{0.486014in}}{\pgfqpoint{4.468121in}{0.486014in}}%
\pgfpathlineto{\pgfqpoint{4.468121in}{0.486014in}}%
\pgfpathclose%
\pgfusepath{stroke}%
\end{pgfscope}%
\begin{pgfscope}%
\pgfpathrectangle{\pgfqpoint{0.494722in}{0.437222in}}{\pgfqpoint{6.275590in}{5.159444in}}%
\pgfusepath{clip}%
\pgfsetbuttcap%
\pgfsetroundjoin%
\pgfsetlinewidth{1.003750pt}%
\definecolor{currentstroke}{rgb}{1.000000,0.000000,0.000000}%
\pgfsetstrokecolor{currentstroke}%
\pgfsetdash{}{0pt}%
\pgfpathmoveto{\pgfqpoint{2.963276in}{0.921795in}}%
\pgfpathcurveto{\pgfqpoint{2.974326in}{0.921795in}}{\pgfqpoint{2.984925in}{0.926186in}}{\pgfqpoint{2.992738in}{0.933999in}}%
\pgfpathcurveto{\pgfqpoint{3.000552in}{0.941813in}}{\pgfqpoint{3.004942in}{0.952412in}}{\pgfqpoint{3.004942in}{0.963462in}}%
\pgfpathcurveto{\pgfqpoint{3.004942in}{0.974512in}}{\pgfqpoint{3.000552in}{0.985111in}}{\pgfqpoint{2.992738in}{0.992925in}}%
\pgfpathcurveto{\pgfqpoint{2.984925in}{1.000739in}}{\pgfqpoint{2.974326in}{1.005129in}}{\pgfqpoint{2.963276in}{1.005129in}}%
\pgfpathcurveto{\pgfqpoint{2.952226in}{1.005129in}}{\pgfqpoint{2.941626in}{1.000739in}}{\pgfqpoint{2.933813in}{0.992925in}}%
\pgfpathcurveto{\pgfqpoint{2.925999in}{0.985111in}}{\pgfqpoint{2.921609in}{0.974512in}}{\pgfqpoint{2.921609in}{0.963462in}}%
\pgfpathcurveto{\pgfqpoint{2.921609in}{0.952412in}}{\pgfqpoint{2.925999in}{0.941813in}}{\pgfqpoint{2.933813in}{0.933999in}}%
\pgfpathcurveto{\pgfqpoint{2.941626in}{0.926186in}}{\pgfqpoint{2.952226in}{0.921795in}}{\pgfqpoint{2.963276in}{0.921795in}}%
\pgfpathlineto{\pgfqpoint{2.963276in}{0.921795in}}%
\pgfpathclose%
\pgfusepath{stroke}%
\end{pgfscope}%
\begin{pgfscope}%
\pgfpathrectangle{\pgfqpoint{0.494722in}{0.437222in}}{\pgfqpoint{6.275590in}{5.159444in}}%
\pgfusepath{clip}%
\pgfsetbuttcap%
\pgfsetroundjoin%
\pgfsetlinewidth{1.003750pt}%
\definecolor{currentstroke}{rgb}{1.000000,0.000000,0.000000}%
\pgfsetstrokecolor{currentstroke}%
\pgfsetdash{}{0pt}%
\pgfpathmoveto{\pgfqpoint{0.597177in}{3.646870in}}%
\pgfpathcurveto{\pgfqpoint{0.608227in}{3.646870in}}{\pgfqpoint{0.618826in}{3.651260in}}{\pgfqpoint{0.626639in}{3.659074in}}%
\pgfpathcurveto{\pgfqpoint{0.634453in}{3.666887in}}{\pgfqpoint{0.638843in}{3.677486in}}{\pgfqpoint{0.638843in}{3.688537in}}%
\pgfpathcurveto{\pgfqpoint{0.638843in}{3.699587in}}{\pgfqpoint{0.634453in}{3.710186in}}{\pgfqpoint{0.626639in}{3.717999in}}%
\pgfpathcurveto{\pgfqpoint{0.618826in}{3.725813in}}{\pgfqpoint{0.608227in}{3.730203in}}{\pgfqpoint{0.597177in}{3.730203in}}%
\pgfpathcurveto{\pgfqpoint{0.586126in}{3.730203in}}{\pgfqpoint{0.575527in}{3.725813in}}{\pgfqpoint{0.567714in}{3.717999in}}%
\pgfpathcurveto{\pgfqpoint{0.559900in}{3.710186in}}{\pgfqpoint{0.555510in}{3.699587in}}{\pgfqpoint{0.555510in}{3.688537in}}%
\pgfpathcurveto{\pgfqpoint{0.555510in}{3.677486in}}{\pgfqpoint{0.559900in}{3.666887in}}{\pgfqpoint{0.567714in}{3.659074in}}%
\pgfpathcurveto{\pgfqpoint{0.575527in}{3.651260in}}{\pgfqpoint{0.586126in}{3.646870in}}{\pgfqpoint{0.597177in}{3.646870in}}%
\pgfpathlineto{\pgfqpoint{0.597177in}{3.646870in}}%
\pgfpathclose%
\pgfusepath{stroke}%
\end{pgfscope}%
\begin{pgfscope}%
\pgfpathrectangle{\pgfqpoint{0.494722in}{0.437222in}}{\pgfqpoint{6.275590in}{5.159444in}}%
\pgfusepath{clip}%
\pgfsetbuttcap%
\pgfsetroundjoin%
\pgfsetlinewidth{1.003750pt}%
\definecolor{currentstroke}{rgb}{1.000000,0.000000,0.000000}%
\pgfsetstrokecolor{currentstroke}%
\pgfsetdash{}{0pt}%
\pgfpathmoveto{\pgfqpoint{0.907688in}{2.810129in}}%
\pgfpathcurveto{\pgfqpoint{0.918738in}{2.810129in}}{\pgfqpoint{0.929337in}{2.814519in}}{\pgfqpoint{0.937151in}{2.822333in}}%
\pgfpathcurveto{\pgfqpoint{0.944964in}{2.830146in}}{\pgfqpoint{0.949354in}{2.840745in}}{\pgfqpoint{0.949354in}{2.851795in}}%
\pgfpathcurveto{\pgfqpoint{0.949354in}{2.862845in}}{\pgfqpoint{0.944964in}{2.873445in}}{\pgfqpoint{0.937151in}{2.881258in}}%
\pgfpathcurveto{\pgfqpoint{0.929337in}{2.889072in}}{\pgfqpoint{0.918738in}{2.893462in}}{\pgfqpoint{0.907688in}{2.893462in}}%
\pgfpathcurveto{\pgfqpoint{0.896638in}{2.893462in}}{\pgfqpoint{0.886039in}{2.889072in}}{\pgfqpoint{0.878225in}{2.881258in}}%
\pgfpathcurveto{\pgfqpoint{0.870411in}{2.873445in}}{\pgfqpoint{0.866021in}{2.862845in}}{\pgfqpoint{0.866021in}{2.851795in}}%
\pgfpathcurveto{\pgfqpoint{0.866021in}{2.840745in}}{\pgfqpoint{0.870411in}{2.830146in}}{\pgfqpoint{0.878225in}{2.822333in}}%
\pgfpathcurveto{\pgfqpoint{0.886039in}{2.814519in}}{\pgfqpoint{0.896638in}{2.810129in}}{\pgfqpoint{0.907688in}{2.810129in}}%
\pgfpathlineto{\pgfqpoint{0.907688in}{2.810129in}}%
\pgfpathclose%
\pgfusepath{stroke}%
\end{pgfscope}%
\begin{pgfscope}%
\pgfpathrectangle{\pgfqpoint{0.494722in}{0.437222in}}{\pgfqpoint{6.275590in}{5.159444in}}%
\pgfusepath{clip}%
\pgfsetbuttcap%
\pgfsetroundjoin%
\pgfsetlinewidth{1.003750pt}%
\definecolor{currentstroke}{rgb}{1.000000,0.000000,0.000000}%
\pgfsetstrokecolor{currentstroke}%
\pgfsetdash{}{0pt}%
\pgfpathmoveto{\pgfqpoint{2.523666in}{1.139615in}}%
\pgfpathcurveto{\pgfqpoint{2.534716in}{1.139615in}}{\pgfqpoint{2.545315in}{1.144006in}}{\pgfqpoint{2.553129in}{1.151819in}}%
\pgfpathcurveto{\pgfqpoint{2.560943in}{1.159633in}}{\pgfqpoint{2.565333in}{1.170232in}}{\pgfqpoint{2.565333in}{1.181282in}}%
\pgfpathcurveto{\pgfqpoint{2.565333in}{1.192332in}}{\pgfqpoint{2.560943in}{1.202931in}}{\pgfqpoint{2.553129in}{1.210745in}}%
\pgfpathcurveto{\pgfqpoint{2.545315in}{1.218558in}}{\pgfqpoint{2.534716in}{1.222949in}}{\pgfqpoint{2.523666in}{1.222949in}}%
\pgfpathcurveto{\pgfqpoint{2.512616in}{1.222949in}}{\pgfqpoint{2.502017in}{1.218558in}}{\pgfqpoint{2.494204in}{1.210745in}}%
\pgfpathcurveto{\pgfqpoint{2.486390in}{1.202931in}}{\pgfqpoint{2.482000in}{1.192332in}}{\pgfqpoint{2.482000in}{1.181282in}}%
\pgfpathcurveto{\pgfqpoint{2.482000in}{1.170232in}}{\pgfqpoint{2.486390in}{1.159633in}}{\pgfqpoint{2.494204in}{1.151819in}}%
\pgfpathcurveto{\pgfqpoint{2.502017in}{1.144006in}}{\pgfqpoint{2.512616in}{1.139615in}}{\pgfqpoint{2.523666in}{1.139615in}}%
\pgfpathlineto{\pgfqpoint{2.523666in}{1.139615in}}%
\pgfpathclose%
\pgfusepath{stroke}%
\end{pgfscope}%
\begin{pgfscope}%
\pgfpathrectangle{\pgfqpoint{0.494722in}{0.437222in}}{\pgfqpoint{6.275590in}{5.159444in}}%
\pgfusepath{clip}%
\pgfsetbuttcap%
\pgfsetroundjoin%
\pgfsetlinewidth{1.003750pt}%
\definecolor{currentstroke}{rgb}{1.000000,0.000000,0.000000}%
\pgfsetstrokecolor{currentstroke}%
\pgfsetdash{}{0pt}%
\pgfpathmoveto{\pgfqpoint{0.496362in}{4.502806in}}%
\pgfpathcurveto{\pgfqpoint{0.507412in}{4.502806in}}{\pgfqpoint{0.518011in}{4.507196in}}{\pgfqpoint{0.525824in}{4.515010in}}%
\pgfpathcurveto{\pgfqpoint{0.533638in}{4.522823in}}{\pgfqpoint{0.538028in}{4.533422in}}{\pgfqpoint{0.538028in}{4.544472in}}%
\pgfpathcurveto{\pgfqpoint{0.538028in}{4.555522in}}{\pgfqpoint{0.533638in}{4.566121in}}{\pgfqpoint{0.525824in}{4.573935in}}%
\pgfpathcurveto{\pgfqpoint{0.518011in}{4.581749in}}{\pgfqpoint{0.507412in}{4.586139in}}{\pgfqpoint{0.496362in}{4.586139in}}%
\pgfpathcurveto{\pgfqpoint{0.485311in}{4.586139in}}{\pgfqpoint{0.474712in}{4.581749in}}{\pgfqpoint{0.466899in}{4.573935in}}%
\pgfpathcurveto{\pgfqpoint{0.459085in}{4.566121in}}{\pgfqpoint{0.454695in}{4.555522in}}{\pgfqpoint{0.454695in}{4.544472in}}%
\pgfpathcurveto{\pgfqpoint{0.454695in}{4.533422in}}{\pgfqpoint{0.459085in}{4.522823in}}{\pgfqpoint{0.466899in}{4.515010in}}%
\pgfpathcurveto{\pgfqpoint{0.474712in}{4.507196in}}{\pgfqpoint{0.485311in}{4.502806in}}{\pgfqpoint{0.496362in}{4.502806in}}%
\pgfpathlineto{\pgfqpoint{0.496362in}{4.502806in}}%
\pgfpathclose%
\pgfusepath{stroke}%
\end{pgfscope}%
\begin{pgfscope}%
\pgfpathrectangle{\pgfqpoint{0.494722in}{0.437222in}}{\pgfqpoint{6.275590in}{5.159444in}}%
\pgfusepath{clip}%
\pgfsetbuttcap%
\pgfsetroundjoin%
\pgfsetlinewidth{1.003750pt}%
\definecolor{currentstroke}{rgb}{1.000000,0.000000,0.000000}%
\pgfsetstrokecolor{currentstroke}%
\pgfsetdash{}{0pt}%
\pgfpathmoveto{\pgfqpoint{1.543548in}{1.885906in}}%
\pgfpathcurveto{\pgfqpoint{1.554598in}{1.885906in}}{\pgfqpoint{1.565197in}{1.890296in}}{\pgfqpoint{1.573011in}{1.898110in}}%
\pgfpathcurveto{\pgfqpoint{1.580824in}{1.905924in}}{\pgfqpoint{1.585215in}{1.916523in}}{\pgfqpoint{1.585215in}{1.927573in}}%
\pgfpathcurveto{\pgfqpoint{1.585215in}{1.938623in}}{\pgfqpoint{1.580824in}{1.949222in}}{\pgfqpoint{1.573011in}{1.957035in}}%
\pgfpathcurveto{\pgfqpoint{1.565197in}{1.964849in}}{\pgfqpoint{1.554598in}{1.969239in}}{\pgfqpoint{1.543548in}{1.969239in}}%
\pgfpathcurveto{\pgfqpoint{1.532498in}{1.969239in}}{\pgfqpoint{1.521899in}{1.964849in}}{\pgfqpoint{1.514085in}{1.957035in}}%
\pgfpathcurveto{\pgfqpoint{1.506272in}{1.949222in}}{\pgfqpoint{1.501881in}{1.938623in}}{\pgfqpoint{1.501881in}{1.927573in}}%
\pgfpathcurveto{\pgfqpoint{1.501881in}{1.916523in}}{\pgfqpoint{1.506272in}{1.905924in}}{\pgfqpoint{1.514085in}{1.898110in}}%
\pgfpathcurveto{\pgfqpoint{1.521899in}{1.890296in}}{\pgfqpoint{1.532498in}{1.885906in}}{\pgfqpoint{1.543548in}{1.885906in}}%
\pgfpathlineto{\pgfqpoint{1.543548in}{1.885906in}}%
\pgfpathclose%
\pgfusepath{stroke}%
\end{pgfscope}%
\begin{pgfscope}%
\pgfpathrectangle{\pgfqpoint{0.494722in}{0.437222in}}{\pgfqpoint{6.275590in}{5.159444in}}%
\pgfusepath{clip}%
\pgfsetbuttcap%
\pgfsetroundjoin%
\pgfsetlinewidth{1.003750pt}%
\definecolor{currentstroke}{rgb}{1.000000,0.000000,0.000000}%
\pgfsetstrokecolor{currentstroke}%
\pgfsetdash{}{0pt}%
\pgfpathmoveto{\pgfqpoint{0.502047in}{4.394858in}}%
\pgfpathcurveto{\pgfqpoint{0.513097in}{4.394858in}}{\pgfqpoint{0.523696in}{4.399248in}}{\pgfqpoint{0.531510in}{4.407062in}}%
\pgfpathcurveto{\pgfqpoint{0.539323in}{4.414875in}}{\pgfqpoint{0.543714in}{4.425474in}}{\pgfqpoint{0.543714in}{4.436524in}}%
\pgfpathcurveto{\pgfqpoint{0.543714in}{4.447575in}}{\pgfqpoint{0.539323in}{4.458174in}}{\pgfqpoint{0.531510in}{4.465987in}}%
\pgfpathcurveto{\pgfqpoint{0.523696in}{4.473801in}}{\pgfqpoint{0.513097in}{4.478191in}}{\pgfqpoint{0.502047in}{4.478191in}}%
\pgfpathcurveto{\pgfqpoint{0.490997in}{4.478191in}}{\pgfqpoint{0.480398in}{4.473801in}}{\pgfqpoint{0.472584in}{4.465987in}}%
\pgfpathcurveto{\pgfqpoint{0.464770in}{4.458174in}}{\pgfqpoint{0.460380in}{4.447575in}}{\pgfqpoint{0.460380in}{4.436524in}}%
\pgfpathcurveto{\pgfqpoint{0.460380in}{4.425474in}}{\pgfqpoint{0.464770in}{4.414875in}}{\pgfqpoint{0.472584in}{4.407062in}}%
\pgfpathcurveto{\pgfqpoint{0.480398in}{4.399248in}}{\pgfqpoint{0.490997in}{4.394858in}}{\pgfqpoint{0.502047in}{4.394858in}}%
\pgfpathlineto{\pgfqpoint{0.502047in}{4.394858in}}%
\pgfpathclose%
\pgfusepath{stroke}%
\end{pgfscope}%
\begin{pgfscope}%
\pgfpathrectangle{\pgfqpoint{0.494722in}{0.437222in}}{\pgfqpoint{6.275590in}{5.159444in}}%
\pgfusepath{clip}%
\pgfsetbuttcap%
\pgfsetroundjoin%
\pgfsetlinewidth{1.003750pt}%
\definecolor{currentstroke}{rgb}{1.000000,0.000000,0.000000}%
\pgfsetstrokecolor{currentstroke}%
\pgfsetdash{}{0pt}%
\pgfpathmoveto{\pgfqpoint{2.207767in}{1.360044in}}%
\pgfpathcurveto{\pgfqpoint{2.218817in}{1.360044in}}{\pgfqpoint{2.229416in}{1.364435in}}{\pgfqpoint{2.237230in}{1.372248in}}%
\pgfpathcurveto{\pgfqpoint{2.245044in}{1.380062in}}{\pgfqpoint{2.249434in}{1.390661in}}{\pgfqpoint{2.249434in}{1.401711in}}%
\pgfpathcurveto{\pgfqpoint{2.249434in}{1.412761in}}{\pgfqpoint{2.245044in}{1.423360in}}{\pgfqpoint{2.237230in}{1.431174in}}%
\pgfpathcurveto{\pgfqpoint{2.229416in}{1.438987in}}{\pgfqpoint{2.218817in}{1.443378in}}{\pgfqpoint{2.207767in}{1.443378in}}%
\pgfpathcurveto{\pgfqpoint{2.196717in}{1.443378in}}{\pgfqpoint{2.186118in}{1.438987in}}{\pgfqpoint{2.178304in}{1.431174in}}%
\pgfpathcurveto{\pgfqpoint{2.170491in}{1.423360in}}{\pgfqpoint{2.166101in}{1.412761in}}{\pgfqpoint{2.166101in}{1.401711in}}%
\pgfpathcurveto{\pgfqpoint{2.166101in}{1.390661in}}{\pgfqpoint{2.170491in}{1.380062in}}{\pgfqpoint{2.178304in}{1.372248in}}%
\pgfpathcurveto{\pgfqpoint{2.186118in}{1.364435in}}{\pgfqpoint{2.196717in}{1.360044in}}{\pgfqpoint{2.207767in}{1.360044in}}%
\pgfpathlineto{\pgfqpoint{2.207767in}{1.360044in}}%
\pgfpathclose%
\pgfusepath{stroke}%
\end{pgfscope}%
\begin{pgfscope}%
\pgfpathrectangle{\pgfqpoint{0.494722in}{0.437222in}}{\pgfqpoint{6.275590in}{5.159444in}}%
\pgfusepath{clip}%
\pgfsetbuttcap%
\pgfsetroundjoin%
\pgfsetlinewidth{1.003750pt}%
\definecolor{currentstroke}{rgb}{1.000000,0.000000,0.000000}%
\pgfsetstrokecolor{currentstroke}%
\pgfsetdash{}{0pt}%
\pgfpathmoveto{\pgfqpoint{0.877909in}{2.824945in}}%
\pgfpathcurveto{\pgfqpoint{0.888959in}{2.824945in}}{\pgfqpoint{0.899558in}{2.829335in}}{\pgfqpoint{0.907371in}{2.837149in}}%
\pgfpathcurveto{\pgfqpoint{0.915185in}{2.844963in}}{\pgfqpoint{0.919575in}{2.855562in}}{\pgfqpoint{0.919575in}{2.866612in}}%
\pgfpathcurveto{\pgfqpoint{0.919575in}{2.877662in}}{\pgfqpoint{0.915185in}{2.888261in}}{\pgfqpoint{0.907371in}{2.896075in}}%
\pgfpathcurveto{\pgfqpoint{0.899558in}{2.903888in}}{\pgfqpoint{0.888959in}{2.908278in}}{\pgfqpoint{0.877909in}{2.908278in}}%
\pgfpathcurveto{\pgfqpoint{0.866858in}{2.908278in}}{\pgfqpoint{0.856259in}{2.903888in}}{\pgfqpoint{0.848446in}{2.896075in}}%
\pgfpathcurveto{\pgfqpoint{0.840632in}{2.888261in}}{\pgfqpoint{0.836242in}{2.877662in}}{\pgfqpoint{0.836242in}{2.866612in}}%
\pgfpathcurveto{\pgfqpoint{0.836242in}{2.855562in}}{\pgfqpoint{0.840632in}{2.844963in}}{\pgfqpoint{0.848446in}{2.837149in}}%
\pgfpathcurveto{\pgfqpoint{0.856259in}{2.829335in}}{\pgfqpoint{0.866858in}{2.824945in}}{\pgfqpoint{0.877909in}{2.824945in}}%
\pgfpathlineto{\pgfqpoint{0.877909in}{2.824945in}}%
\pgfpathclose%
\pgfusepath{stroke}%
\end{pgfscope}%
\begin{pgfscope}%
\pgfpathrectangle{\pgfqpoint{0.494722in}{0.437222in}}{\pgfqpoint{6.275590in}{5.159444in}}%
\pgfusepath{clip}%
\pgfsetbuttcap%
\pgfsetroundjoin%
\pgfsetlinewidth{1.003750pt}%
\definecolor{currentstroke}{rgb}{1.000000,0.000000,0.000000}%
\pgfsetstrokecolor{currentstroke}%
\pgfsetdash{}{0pt}%
\pgfpathmoveto{\pgfqpoint{3.994751in}{0.563988in}}%
\pgfpathcurveto{\pgfqpoint{4.005801in}{0.563988in}}{\pgfqpoint{4.016400in}{0.568378in}}{\pgfqpoint{4.024214in}{0.576191in}}%
\pgfpathcurveto{\pgfqpoint{4.032027in}{0.584005in}}{\pgfqpoint{4.036418in}{0.594604in}}{\pgfqpoint{4.036418in}{0.605654in}}%
\pgfpathcurveto{\pgfqpoint{4.036418in}{0.616704in}}{\pgfqpoint{4.032027in}{0.627303in}}{\pgfqpoint{4.024214in}{0.635117in}}%
\pgfpathcurveto{\pgfqpoint{4.016400in}{0.642931in}}{\pgfqpoint{4.005801in}{0.647321in}}{\pgfqpoint{3.994751in}{0.647321in}}%
\pgfpathcurveto{\pgfqpoint{3.983701in}{0.647321in}}{\pgfqpoint{3.973102in}{0.642931in}}{\pgfqpoint{3.965288in}{0.635117in}}%
\pgfpathcurveto{\pgfqpoint{3.957475in}{0.627303in}}{\pgfqpoint{3.953084in}{0.616704in}}{\pgfqpoint{3.953084in}{0.605654in}}%
\pgfpathcurveto{\pgfqpoint{3.953084in}{0.594604in}}{\pgfqpoint{3.957475in}{0.584005in}}{\pgfqpoint{3.965288in}{0.576191in}}%
\pgfpathcurveto{\pgfqpoint{3.973102in}{0.568378in}}{\pgfqpoint{3.983701in}{0.563988in}}{\pgfqpoint{3.994751in}{0.563988in}}%
\pgfpathlineto{\pgfqpoint{3.994751in}{0.563988in}}%
\pgfpathclose%
\pgfusepath{stroke}%
\end{pgfscope}%
\begin{pgfscope}%
\pgfpathrectangle{\pgfqpoint{0.494722in}{0.437222in}}{\pgfqpoint{6.275590in}{5.159444in}}%
\pgfusepath{clip}%
\pgfsetbuttcap%
\pgfsetroundjoin%
\pgfsetlinewidth{1.003750pt}%
\definecolor{currentstroke}{rgb}{1.000000,0.000000,0.000000}%
\pgfsetstrokecolor{currentstroke}%
\pgfsetdash{}{0pt}%
\pgfpathmoveto{\pgfqpoint{1.681197in}{1.759539in}}%
\pgfpathcurveto{\pgfqpoint{1.692247in}{1.759539in}}{\pgfqpoint{1.702846in}{1.763929in}}{\pgfqpoint{1.710659in}{1.771743in}}%
\pgfpathcurveto{\pgfqpoint{1.718473in}{1.779556in}}{\pgfqpoint{1.722863in}{1.790155in}}{\pgfqpoint{1.722863in}{1.801205in}}%
\pgfpathcurveto{\pgfqpoint{1.722863in}{1.812256in}}{\pgfqpoint{1.718473in}{1.822855in}}{\pgfqpoint{1.710659in}{1.830668in}}%
\pgfpathcurveto{\pgfqpoint{1.702846in}{1.838482in}}{\pgfqpoint{1.692247in}{1.842872in}}{\pgfqpoint{1.681197in}{1.842872in}}%
\pgfpathcurveto{\pgfqpoint{1.670146in}{1.842872in}}{\pgfqpoint{1.659547in}{1.838482in}}{\pgfqpoint{1.651734in}{1.830668in}}%
\pgfpathcurveto{\pgfqpoint{1.643920in}{1.822855in}}{\pgfqpoint{1.639530in}{1.812256in}}{\pgfqpoint{1.639530in}{1.801205in}}%
\pgfpathcurveto{\pgfqpoint{1.639530in}{1.790155in}}{\pgfqpoint{1.643920in}{1.779556in}}{\pgfqpoint{1.651734in}{1.771743in}}%
\pgfpathcurveto{\pgfqpoint{1.659547in}{1.763929in}}{\pgfqpoint{1.670146in}{1.759539in}}{\pgfqpoint{1.681197in}{1.759539in}}%
\pgfpathlineto{\pgfqpoint{1.681197in}{1.759539in}}%
\pgfpathclose%
\pgfusepath{stroke}%
\end{pgfscope}%
\begin{pgfscope}%
\pgfpathrectangle{\pgfqpoint{0.494722in}{0.437222in}}{\pgfqpoint{6.275590in}{5.159444in}}%
\pgfusepath{clip}%
\pgfsetbuttcap%
\pgfsetroundjoin%
\pgfsetlinewidth{1.003750pt}%
\definecolor{currentstroke}{rgb}{1.000000,0.000000,0.000000}%
\pgfsetstrokecolor{currentstroke}%
\pgfsetdash{}{0pt}%
\pgfpathmoveto{\pgfqpoint{1.062563in}{2.508760in}}%
\pgfpathcurveto{\pgfqpoint{1.073614in}{2.508760in}}{\pgfqpoint{1.084213in}{2.513150in}}{\pgfqpoint{1.092026in}{2.520964in}}%
\pgfpathcurveto{\pgfqpoint{1.099840in}{2.528778in}}{\pgfqpoint{1.104230in}{2.539377in}}{\pgfqpoint{1.104230in}{2.550427in}}%
\pgfpathcurveto{\pgfqpoint{1.104230in}{2.561477in}}{\pgfqpoint{1.099840in}{2.572076in}}{\pgfqpoint{1.092026in}{2.579890in}}%
\pgfpathcurveto{\pgfqpoint{1.084213in}{2.587703in}}{\pgfqpoint{1.073614in}{2.592093in}}{\pgfqpoint{1.062563in}{2.592093in}}%
\pgfpathcurveto{\pgfqpoint{1.051513in}{2.592093in}}{\pgfqpoint{1.040914in}{2.587703in}}{\pgfqpoint{1.033101in}{2.579890in}}%
\pgfpathcurveto{\pgfqpoint{1.025287in}{2.572076in}}{\pgfqpoint{1.020897in}{2.561477in}}{\pgfqpoint{1.020897in}{2.550427in}}%
\pgfpathcurveto{\pgfqpoint{1.020897in}{2.539377in}}{\pgfqpoint{1.025287in}{2.528778in}}{\pgfqpoint{1.033101in}{2.520964in}}%
\pgfpathcurveto{\pgfqpoint{1.040914in}{2.513150in}}{\pgfqpoint{1.051513in}{2.508760in}}{\pgfqpoint{1.062563in}{2.508760in}}%
\pgfpathlineto{\pgfqpoint{1.062563in}{2.508760in}}%
\pgfpathclose%
\pgfusepath{stroke}%
\end{pgfscope}%
\begin{pgfscope}%
\pgfpathrectangle{\pgfqpoint{0.494722in}{0.437222in}}{\pgfqpoint{6.275590in}{5.159444in}}%
\pgfusepath{clip}%
\pgfsetbuttcap%
\pgfsetroundjoin%
\pgfsetlinewidth{1.003750pt}%
\definecolor{currentstroke}{rgb}{1.000000,0.000000,0.000000}%
\pgfsetstrokecolor{currentstroke}%
\pgfsetdash{}{0pt}%
\pgfpathmoveto{\pgfqpoint{5.163605in}{0.413881in}}%
\pgfpathcurveto{\pgfqpoint{5.174655in}{0.413881in}}{\pgfqpoint{5.185254in}{0.418271in}}{\pgfqpoint{5.193068in}{0.426085in}}%
\pgfpathcurveto{\pgfqpoint{5.200881in}{0.433898in}}{\pgfqpoint{5.205271in}{0.444497in}}{\pgfqpoint{5.205271in}{0.455547in}}%
\pgfpathcurveto{\pgfqpoint{5.205271in}{0.466598in}}{\pgfqpoint{5.200881in}{0.477197in}}{\pgfqpoint{5.193068in}{0.485010in}}%
\pgfpathcurveto{\pgfqpoint{5.185254in}{0.492824in}}{\pgfqpoint{5.174655in}{0.497214in}}{\pgfqpoint{5.163605in}{0.497214in}}%
\pgfpathcurveto{\pgfqpoint{5.152555in}{0.497214in}}{\pgfqpoint{5.141956in}{0.492824in}}{\pgfqpoint{5.134142in}{0.485010in}}%
\pgfpathcurveto{\pgfqpoint{5.126328in}{0.477197in}}{\pgfqpoint{5.121938in}{0.466598in}}{\pgfqpoint{5.121938in}{0.455547in}}%
\pgfpathcurveto{\pgfqpoint{5.121938in}{0.444497in}}{\pgfqpoint{5.126328in}{0.433898in}}{\pgfqpoint{5.134142in}{0.426085in}}%
\pgfpathcurveto{\pgfqpoint{5.141956in}{0.418271in}}{\pgfqpoint{5.152555in}{0.413881in}}{\pgfqpoint{5.163605in}{0.413881in}}%
\pgfusepath{stroke}%
\end{pgfscope}%
\begin{pgfscope}%
\pgfpathrectangle{\pgfqpoint{0.494722in}{0.437222in}}{\pgfqpoint{6.275590in}{5.159444in}}%
\pgfusepath{clip}%
\pgfsetbuttcap%
\pgfsetroundjoin%
\pgfsetlinewidth{1.003750pt}%
\definecolor{currentstroke}{rgb}{1.000000,0.000000,0.000000}%
\pgfsetstrokecolor{currentstroke}%
\pgfsetdash{}{0pt}%
\pgfpathmoveto{\pgfqpoint{0.857135in}{2.867759in}}%
\pgfpathcurveto{\pgfqpoint{0.868185in}{2.867759in}}{\pgfqpoint{0.878784in}{2.872149in}}{\pgfqpoint{0.886597in}{2.879963in}}%
\pgfpathcurveto{\pgfqpoint{0.894411in}{2.887776in}}{\pgfqpoint{0.898801in}{2.898375in}}{\pgfqpoint{0.898801in}{2.909425in}}%
\pgfpathcurveto{\pgfqpoint{0.898801in}{2.920476in}}{\pgfqpoint{0.894411in}{2.931075in}}{\pgfqpoint{0.886597in}{2.938888in}}%
\pgfpathcurveto{\pgfqpoint{0.878784in}{2.946702in}}{\pgfqpoint{0.868185in}{2.951092in}}{\pgfqpoint{0.857135in}{2.951092in}}%
\pgfpathcurveto{\pgfqpoint{0.846084in}{2.951092in}}{\pgfqpoint{0.835485in}{2.946702in}}{\pgfqpoint{0.827672in}{2.938888in}}%
\pgfpathcurveto{\pgfqpoint{0.819858in}{2.931075in}}{\pgfqpoint{0.815468in}{2.920476in}}{\pgfqpoint{0.815468in}{2.909425in}}%
\pgfpathcurveto{\pgfqpoint{0.815468in}{2.898375in}}{\pgfqpoint{0.819858in}{2.887776in}}{\pgfqpoint{0.827672in}{2.879963in}}%
\pgfpathcurveto{\pgfqpoint{0.835485in}{2.872149in}}{\pgfqpoint{0.846084in}{2.867759in}}{\pgfqpoint{0.857135in}{2.867759in}}%
\pgfpathlineto{\pgfqpoint{0.857135in}{2.867759in}}%
\pgfpathclose%
\pgfusepath{stroke}%
\end{pgfscope}%
\begin{pgfscope}%
\pgfpathrectangle{\pgfqpoint{0.494722in}{0.437222in}}{\pgfqpoint{6.275590in}{5.159444in}}%
\pgfusepath{clip}%
\pgfsetbuttcap%
\pgfsetroundjoin%
\pgfsetlinewidth{1.003750pt}%
\definecolor{currentstroke}{rgb}{1.000000,0.000000,0.000000}%
\pgfsetstrokecolor{currentstroke}%
\pgfsetdash{}{0pt}%
\pgfpathmoveto{\pgfqpoint{0.590197in}{3.688465in}}%
\pgfpathcurveto{\pgfqpoint{0.601248in}{3.688465in}}{\pgfqpoint{0.611847in}{3.692855in}}{\pgfqpoint{0.619660in}{3.700668in}}%
\pgfpathcurveto{\pgfqpoint{0.627474in}{3.708482in}}{\pgfqpoint{0.631864in}{3.719081in}}{\pgfqpoint{0.631864in}{3.730131in}}%
\pgfpathcurveto{\pgfqpoint{0.631864in}{3.741181in}}{\pgfqpoint{0.627474in}{3.751780in}}{\pgfqpoint{0.619660in}{3.759594in}}%
\pgfpathcurveto{\pgfqpoint{0.611847in}{3.767408in}}{\pgfqpoint{0.601248in}{3.771798in}}{\pgfqpoint{0.590197in}{3.771798in}}%
\pgfpathcurveto{\pgfqpoint{0.579147in}{3.771798in}}{\pgfqpoint{0.568548in}{3.767408in}}{\pgfqpoint{0.560735in}{3.759594in}}%
\pgfpathcurveto{\pgfqpoint{0.552921in}{3.751780in}}{\pgfqpoint{0.548531in}{3.741181in}}{\pgfqpoint{0.548531in}{3.730131in}}%
\pgfpathcurveto{\pgfqpoint{0.548531in}{3.719081in}}{\pgfqpoint{0.552921in}{3.708482in}}{\pgfqpoint{0.560735in}{3.700668in}}%
\pgfpathcurveto{\pgfqpoint{0.568548in}{3.692855in}}{\pgfqpoint{0.579147in}{3.688465in}}{\pgfqpoint{0.590197in}{3.688465in}}%
\pgfpathlineto{\pgfqpoint{0.590197in}{3.688465in}}%
\pgfpathclose%
\pgfusepath{stroke}%
\end{pgfscope}%
\begin{pgfscope}%
\pgfpathrectangle{\pgfqpoint{0.494722in}{0.437222in}}{\pgfqpoint{6.275590in}{5.159444in}}%
\pgfusepath{clip}%
\pgfsetbuttcap%
\pgfsetroundjoin%
\pgfsetlinewidth{1.003750pt}%
\definecolor{currentstroke}{rgb}{1.000000,0.000000,0.000000}%
\pgfsetstrokecolor{currentstroke}%
\pgfsetdash{}{0pt}%
\pgfpathmoveto{\pgfqpoint{3.029347in}{0.892004in}}%
\pgfpathcurveto{\pgfqpoint{3.040398in}{0.892004in}}{\pgfqpoint{3.050997in}{0.896395in}}{\pgfqpoint{3.058810in}{0.904208in}}%
\pgfpathcurveto{\pgfqpoint{3.066624in}{0.912022in}}{\pgfqpoint{3.071014in}{0.922621in}}{\pgfqpoint{3.071014in}{0.933671in}}%
\pgfpathcurveto{\pgfqpoint{3.071014in}{0.944721in}}{\pgfqpoint{3.066624in}{0.955320in}}{\pgfqpoint{3.058810in}{0.963134in}}%
\pgfpathcurveto{\pgfqpoint{3.050997in}{0.970947in}}{\pgfqpoint{3.040398in}{0.975338in}}{\pgfqpoint{3.029347in}{0.975338in}}%
\pgfpathcurveto{\pgfqpoint{3.018297in}{0.975338in}}{\pgfqpoint{3.007698in}{0.970947in}}{\pgfqpoint{2.999885in}{0.963134in}}%
\pgfpathcurveto{\pgfqpoint{2.992071in}{0.955320in}}{\pgfqpoint{2.987681in}{0.944721in}}{\pgfqpoint{2.987681in}{0.933671in}}%
\pgfpathcurveto{\pgfqpoint{2.987681in}{0.922621in}}{\pgfqpoint{2.992071in}{0.912022in}}{\pgfqpoint{2.999885in}{0.904208in}}%
\pgfpathcurveto{\pgfqpoint{3.007698in}{0.896395in}}{\pgfqpoint{3.018297in}{0.892004in}}{\pgfqpoint{3.029347in}{0.892004in}}%
\pgfpathlineto{\pgfqpoint{3.029347in}{0.892004in}}%
\pgfpathclose%
\pgfusepath{stroke}%
\end{pgfscope}%
\begin{pgfscope}%
\pgfpathrectangle{\pgfqpoint{0.494722in}{0.437222in}}{\pgfqpoint{6.275590in}{5.159444in}}%
\pgfusepath{clip}%
\pgfsetbuttcap%
\pgfsetroundjoin%
\pgfsetlinewidth{1.003750pt}%
\definecolor{currentstroke}{rgb}{1.000000,0.000000,0.000000}%
\pgfsetstrokecolor{currentstroke}%
\pgfsetdash{}{0pt}%
\pgfpathmoveto{\pgfqpoint{3.474694in}{0.707558in}}%
\pgfpathcurveto{\pgfqpoint{3.485744in}{0.707558in}}{\pgfqpoint{3.496343in}{0.711948in}}{\pgfqpoint{3.504157in}{0.719762in}}%
\pgfpathcurveto{\pgfqpoint{3.511971in}{0.727576in}}{\pgfqpoint{3.516361in}{0.738175in}}{\pgfqpoint{3.516361in}{0.749225in}}%
\pgfpathcurveto{\pgfqpoint{3.516361in}{0.760275in}}{\pgfqpoint{3.511971in}{0.770874in}}{\pgfqpoint{3.504157in}{0.778688in}}%
\pgfpathcurveto{\pgfqpoint{3.496343in}{0.786501in}}{\pgfqpoint{3.485744in}{0.790892in}}{\pgfqpoint{3.474694in}{0.790892in}}%
\pgfpathcurveto{\pgfqpoint{3.463644in}{0.790892in}}{\pgfqpoint{3.453045in}{0.786501in}}{\pgfqpoint{3.445232in}{0.778688in}}%
\pgfpathcurveto{\pgfqpoint{3.437418in}{0.770874in}}{\pgfqpoint{3.433028in}{0.760275in}}{\pgfqpoint{3.433028in}{0.749225in}}%
\pgfpathcurveto{\pgfqpoint{3.433028in}{0.738175in}}{\pgfqpoint{3.437418in}{0.727576in}}{\pgfqpoint{3.445232in}{0.719762in}}%
\pgfpathcurveto{\pgfqpoint{3.453045in}{0.711948in}}{\pgfqpoint{3.463644in}{0.707558in}}{\pgfqpoint{3.474694in}{0.707558in}}%
\pgfpathlineto{\pgfqpoint{3.474694in}{0.707558in}}%
\pgfpathclose%
\pgfusepath{stroke}%
\end{pgfscope}%
\begin{pgfscope}%
\pgfpathrectangle{\pgfqpoint{0.494722in}{0.437222in}}{\pgfqpoint{6.275590in}{5.159444in}}%
\pgfusepath{clip}%
\pgfsetbuttcap%
\pgfsetroundjoin%
\pgfsetlinewidth{1.003750pt}%
\definecolor{currentstroke}{rgb}{1.000000,0.000000,0.000000}%
\pgfsetstrokecolor{currentstroke}%
\pgfsetdash{}{0pt}%
\pgfpathmoveto{\pgfqpoint{3.041403in}{0.887250in}}%
\pgfpathcurveto{\pgfqpoint{3.052453in}{0.887250in}}{\pgfqpoint{3.063052in}{0.891641in}}{\pgfqpoint{3.070866in}{0.899454in}}%
\pgfpathcurveto{\pgfqpoint{3.078679in}{0.907268in}}{\pgfqpoint{3.083070in}{0.917867in}}{\pgfqpoint{3.083070in}{0.928917in}}%
\pgfpathcurveto{\pgfqpoint{3.083070in}{0.939967in}}{\pgfqpoint{3.078679in}{0.950566in}}{\pgfqpoint{3.070866in}{0.958380in}}%
\pgfpathcurveto{\pgfqpoint{3.063052in}{0.966194in}}{\pgfqpoint{3.052453in}{0.970584in}}{\pgfqpoint{3.041403in}{0.970584in}}%
\pgfpathcurveto{\pgfqpoint{3.030353in}{0.970584in}}{\pgfqpoint{3.019754in}{0.966194in}}{\pgfqpoint{3.011940in}{0.958380in}}%
\pgfpathcurveto{\pgfqpoint{3.004127in}{0.950566in}}{\pgfqpoint{2.999736in}{0.939967in}}{\pgfqpoint{2.999736in}{0.928917in}}%
\pgfpathcurveto{\pgfqpoint{2.999736in}{0.917867in}}{\pgfqpoint{3.004127in}{0.907268in}}{\pgfqpoint{3.011940in}{0.899454in}}%
\pgfpathcurveto{\pgfqpoint{3.019754in}{0.891641in}}{\pgfqpoint{3.030353in}{0.887250in}}{\pgfqpoint{3.041403in}{0.887250in}}%
\pgfpathlineto{\pgfqpoint{3.041403in}{0.887250in}}%
\pgfpathclose%
\pgfusepath{stroke}%
\end{pgfscope}%
\begin{pgfscope}%
\pgfpathrectangle{\pgfqpoint{0.494722in}{0.437222in}}{\pgfqpoint{6.275590in}{5.159444in}}%
\pgfusepath{clip}%
\pgfsetbuttcap%
\pgfsetroundjoin%
\pgfsetlinewidth{1.003750pt}%
\definecolor{currentstroke}{rgb}{1.000000,0.000000,0.000000}%
\pgfsetstrokecolor{currentstroke}%
\pgfsetdash{}{0pt}%
\pgfpathmoveto{\pgfqpoint{1.083042in}{2.456415in}}%
\pgfpathcurveto{\pgfqpoint{1.094092in}{2.456415in}}{\pgfqpoint{1.104691in}{2.460806in}}{\pgfqpoint{1.112505in}{2.468619in}}%
\pgfpathcurveto{\pgfqpoint{1.120318in}{2.476433in}}{\pgfqpoint{1.124709in}{2.487032in}}{\pgfqpoint{1.124709in}{2.498082in}}%
\pgfpathcurveto{\pgfqpoint{1.124709in}{2.509132in}}{\pgfqpoint{1.120318in}{2.519731in}}{\pgfqpoint{1.112505in}{2.527545in}}%
\pgfpathcurveto{\pgfqpoint{1.104691in}{2.535358in}}{\pgfqpoint{1.094092in}{2.539749in}}{\pgfqpoint{1.083042in}{2.539749in}}%
\pgfpathcurveto{\pgfqpoint{1.071992in}{2.539749in}}{\pgfqpoint{1.061393in}{2.535358in}}{\pgfqpoint{1.053579in}{2.527545in}}%
\pgfpathcurveto{\pgfqpoint{1.045765in}{2.519731in}}{\pgfqpoint{1.041375in}{2.509132in}}{\pgfqpoint{1.041375in}{2.498082in}}%
\pgfpathcurveto{\pgfqpoint{1.041375in}{2.487032in}}{\pgfqpoint{1.045765in}{2.476433in}}{\pgfqpoint{1.053579in}{2.468619in}}%
\pgfpathcurveto{\pgfqpoint{1.061393in}{2.460806in}}{\pgfqpoint{1.071992in}{2.456415in}}{\pgfqpoint{1.083042in}{2.456415in}}%
\pgfpathlineto{\pgfqpoint{1.083042in}{2.456415in}}%
\pgfpathclose%
\pgfusepath{stroke}%
\end{pgfscope}%
\begin{pgfscope}%
\pgfpathrectangle{\pgfqpoint{0.494722in}{0.437222in}}{\pgfqpoint{6.275590in}{5.159444in}}%
\pgfusepath{clip}%
\pgfsetbuttcap%
\pgfsetroundjoin%
\pgfsetlinewidth{1.003750pt}%
\definecolor{currentstroke}{rgb}{1.000000,0.000000,0.000000}%
\pgfsetstrokecolor{currentstroke}%
\pgfsetdash{}{0pt}%
\pgfpathmoveto{\pgfqpoint{3.912836in}{0.582522in}}%
\pgfpathcurveto{\pgfqpoint{3.923886in}{0.582522in}}{\pgfqpoint{3.934485in}{0.586912in}}{\pgfqpoint{3.942299in}{0.594726in}}%
\pgfpathcurveto{\pgfqpoint{3.950112in}{0.602540in}}{\pgfqpoint{3.954503in}{0.613139in}}{\pgfqpoint{3.954503in}{0.624189in}}%
\pgfpathcurveto{\pgfqpoint{3.954503in}{0.635239in}}{\pgfqpoint{3.950112in}{0.645838in}}{\pgfqpoint{3.942299in}{0.653651in}}%
\pgfpathcurveto{\pgfqpoint{3.934485in}{0.661465in}}{\pgfqpoint{3.923886in}{0.665855in}}{\pgfqpoint{3.912836in}{0.665855in}}%
\pgfpathcurveto{\pgfqpoint{3.901786in}{0.665855in}}{\pgfqpoint{3.891187in}{0.661465in}}{\pgfqpoint{3.883373in}{0.653651in}}%
\pgfpathcurveto{\pgfqpoint{3.875560in}{0.645838in}}{\pgfqpoint{3.871169in}{0.635239in}}{\pgfqpoint{3.871169in}{0.624189in}}%
\pgfpathcurveto{\pgfqpoint{3.871169in}{0.613139in}}{\pgfqpoint{3.875560in}{0.602540in}}{\pgfqpoint{3.883373in}{0.594726in}}%
\pgfpathcurveto{\pgfqpoint{3.891187in}{0.586912in}}{\pgfqpoint{3.901786in}{0.582522in}}{\pgfqpoint{3.912836in}{0.582522in}}%
\pgfpathlineto{\pgfqpoint{3.912836in}{0.582522in}}%
\pgfpathclose%
\pgfusepath{stroke}%
\end{pgfscope}%
\begin{pgfscope}%
\pgfpathrectangle{\pgfqpoint{0.494722in}{0.437222in}}{\pgfqpoint{6.275590in}{5.159444in}}%
\pgfusepath{clip}%
\pgfsetbuttcap%
\pgfsetroundjoin%
\pgfsetlinewidth{1.003750pt}%
\definecolor{currentstroke}{rgb}{1.000000,0.000000,0.000000}%
\pgfsetstrokecolor{currentstroke}%
\pgfsetdash{}{0pt}%
\pgfpathmoveto{\pgfqpoint{3.180574in}{0.821221in}}%
\pgfpathcurveto{\pgfqpoint{3.191624in}{0.821221in}}{\pgfqpoint{3.202223in}{0.825611in}}{\pgfqpoint{3.210037in}{0.833425in}}%
\pgfpathcurveto{\pgfqpoint{3.217850in}{0.841239in}}{\pgfqpoint{3.222241in}{0.851838in}}{\pgfqpoint{3.222241in}{0.862888in}}%
\pgfpathcurveto{\pgfqpoint{3.222241in}{0.873938in}}{\pgfqpoint{3.217850in}{0.884537in}}{\pgfqpoint{3.210037in}{0.892351in}}%
\pgfpathcurveto{\pgfqpoint{3.202223in}{0.900164in}}{\pgfqpoint{3.191624in}{0.904554in}}{\pgfqpoint{3.180574in}{0.904554in}}%
\pgfpathcurveto{\pgfqpoint{3.169524in}{0.904554in}}{\pgfqpoint{3.158925in}{0.900164in}}{\pgfqpoint{3.151111in}{0.892351in}}%
\pgfpathcurveto{\pgfqpoint{3.143298in}{0.884537in}}{\pgfqpoint{3.138907in}{0.873938in}}{\pgfqpoint{3.138907in}{0.862888in}}%
\pgfpathcurveto{\pgfqpoint{3.138907in}{0.851838in}}{\pgfqpoint{3.143298in}{0.841239in}}{\pgfqpoint{3.151111in}{0.833425in}}%
\pgfpathcurveto{\pgfqpoint{3.158925in}{0.825611in}}{\pgfqpoint{3.169524in}{0.821221in}}{\pgfqpoint{3.180574in}{0.821221in}}%
\pgfpathlineto{\pgfqpoint{3.180574in}{0.821221in}}%
\pgfpathclose%
\pgfusepath{stroke}%
\end{pgfscope}%
\begin{pgfscope}%
\pgfpathrectangle{\pgfqpoint{0.494722in}{0.437222in}}{\pgfqpoint{6.275590in}{5.159444in}}%
\pgfusepath{clip}%
\pgfsetbuttcap%
\pgfsetroundjoin%
\pgfsetlinewidth{1.003750pt}%
\definecolor{currentstroke}{rgb}{1.000000,0.000000,0.000000}%
\pgfsetstrokecolor{currentstroke}%
\pgfsetdash{}{0pt}%
\pgfpathmoveto{\pgfqpoint{0.785689in}{3.031130in}}%
\pgfpathcurveto{\pgfqpoint{0.796739in}{3.031130in}}{\pgfqpoint{0.807338in}{3.035520in}}{\pgfqpoint{0.815151in}{3.043334in}}%
\pgfpathcurveto{\pgfqpoint{0.822965in}{3.051147in}}{\pgfqpoint{0.827355in}{3.061746in}}{\pgfqpoint{0.827355in}{3.072796in}}%
\pgfpathcurveto{\pgfqpoint{0.827355in}{3.083846in}}{\pgfqpoint{0.822965in}{3.094445in}}{\pgfqpoint{0.815151in}{3.102259in}}%
\pgfpathcurveto{\pgfqpoint{0.807338in}{3.110073in}}{\pgfqpoint{0.796739in}{3.114463in}}{\pgfqpoint{0.785689in}{3.114463in}}%
\pgfpathcurveto{\pgfqpoint{0.774639in}{3.114463in}}{\pgfqpoint{0.764040in}{3.110073in}}{\pgfqpoint{0.756226in}{3.102259in}}%
\pgfpathcurveto{\pgfqpoint{0.748412in}{3.094445in}}{\pgfqpoint{0.744022in}{3.083846in}}{\pgfqpoint{0.744022in}{3.072796in}}%
\pgfpathcurveto{\pgfqpoint{0.744022in}{3.061746in}}{\pgfqpoint{0.748412in}{3.051147in}}{\pgfqpoint{0.756226in}{3.043334in}}%
\pgfpathcurveto{\pgfqpoint{0.764040in}{3.035520in}}{\pgfqpoint{0.774639in}{3.031130in}}{\pgfqpoint{0.785689in}{3.031130in}}%
\pgfpathlineto{\pgfqpoint{0.785689in}{3.031130in}}%
\pgfpathclose%
\pgfusepath{stroke}%
\end{pgfscope}%
\begin{pgfscope}%
\pgfpathrectangle{\pgfqpoint{0.494722in}{0.437222in}}{\pgfqpoint{6.275590in}{5.159444in}}%
\pgfusepath{clip}%
\pgfsetbuttcap%
\pgfsetroundjoin%
\pgfsetlinewidth{1.003750pt}%
\definecolor{currentstroke}{rgb}{1.000000,0.000000,0.000000}%
\pgfsetstrokecolor{currentstroke}%
\pgfsetdash{}{0pt}%
\pgfpathmoveto{\pgfqpoint{0.820238in}{2.949093in}}%
\pgfpathcurveto{\pgfqpoint{0.831288in}{2.949093in}}{\pgfqpoint{0.841887in}{2.953483in}}{\pgfqpoint{0.849701in}{2.961297in}}%
\pgfpathcurveto{\pgfqpoint{0.857514in}{2.969110in}}{\pgfqpoint{0.861905in}{2.979709in}}{\pgfqpoint{0.861905in}{2.990760in}}%
\pgfpathcurveto{\pgfqpoint{0.861905in}{3.001810in}}{\pgfqpoint{0.857514in}{3.012409in}}{\pgfqpoint{0.849701in}{3.020222in}}%
\pgfpathcurveto{\pgfqpoint{0.841887in}{3.028036in}}{\pgfqpoint{0.831288in}{3.032426in}}{\pgfqpoint{0.820238in}{3.032426in}}%
\pgfpathcurveto{\pgfqpoint{0.809188in}{3.032426in}}{\pgfqpoint{0.798589in}{3.028036in}}{\pgfqpoint{0.790775in}{3.020222in}}%
\pgfpathcurveto{\pgfqpoint{0.782961in}{3.012409in}}{\pgfqpoint{0.778571in}{3.001810in}}{\pgfqpoint{0.778571in}{2.990760in}}%
\pgfpathcurveto{\pgfqpoint{0.778571in}{2.979709in}}{\pgfqpoint{0.782961in}{2.969110in}}{\pgfqpoint{0.790775in}{2.961297in}}%
\pgfpathcurveto{\pgfqpoint{0.798589in}{2.953483in}}{\pgfqpoint{0.809188in}{2.949093in}}{\pgfqpoint{0.820238in}{2.949093in}}%
\pgfpathlineto{\pgfqpoint{0.820238in}{2.949093in}}%
\pgfpathclose%
\pgfusepath{stroke}%
\end{pgfscope}%
\begin{pgfscope}%
\pgfpathrectangle{\pgfqpoint{0.494722in}{0.437222in}}{\pgfqpoint{6.275590in}{5.159444in}}%
\pgfusepath{clip}%
\pgfsetbuttcap%
\pgfsetroundjoin%
\pgfsetlinewidth{1.003750pt}%
\definecolor{currentstroke}{rgb}{1.000000,0.000000,0.000000}%
\pgfsetstrokecolor{currentstroke}%
\pgfsetdash{}{0pt}%
\pgfpathmoveto{\pgfqpoint{3.114896in}{0.847735in}}%
\pgfpathcurveto{\pgfqpoint{3.125946in}{0.847735in}}{\pgfqpoint{3.136545in}{0.852125in}}{\pgfqpoint{3.144359in}{0.859939in}}%
\pgfpathcurveto{\pgfqpoint{3.152173in}{0.867752in}}{\pgfqpoint{3.156563in}{0.878351in}}{\pgfqpoint{3.156563in}{0.889401in}}%
\pgfpathcurveto{\pgfqpoint{3.156563in}{0.900451in}}{\pgfqpoint{3.152173in}{0.911051in}}{\pgfqpoint{3.144359in}{0.918864in}}%
\pgfpathcurveto{\pgfqpoint{3.136545in}{0.926678in}}{\pgfqpoint{3.125946in}{0.931068in}}{\pgfqpoint{3.114896in}{0.931068in}}%
\pgfpathcurveto{\pgfqpoint{3.103846in}{0.931068in}}{\pgfqpoint{3.093247in}{0.926678in}}{\pgfqpoint{3.085433in}{0.918864in}}%
\pgfpathcurveto{\pgfqpoint{3.077620in}{0.911051in}}{\pgfqpoint{3.073229in}{0.900451in}}{\pgfqpoint{3.073229in}{0.889401in}}%
\pgfpathcurveto{\pgfqpoint{3.073229in}{0.878351in}}{\pgfqpoint{3.077620in}{0.867752in}}{\pgfqpoint{3.085433in}{0.859939in}}%
\pgfpathcurveto{\pgfqpoint{3.093247in}{0.852125in}}{\pgfqpoint{3.103846in}{0.847735in}}{\pgfqpoint{3.114896in}{0.847735in}}%
\pgfpathlineto{\pgfqpoint{3.114896in}{0.847735in}}%
\pgfpathclose%
\pgfusepath{stroke}%
\end{pgfscope}%
\begin{pgfscope}%
\pgfpathrectangle{\pgfqpoint{0.494722in}{0.437222in}}{\pgfqpoint{6.275590in}{5.159444in}}%
\pgfusepath{clip}%
\pgfsetbuttcap%
\pgfsetroundjoin%
\pgfsetlinewidth{1.003750pt}%
\definecolor{currentstroke}{rgb}{1.000000,0.000000,0.000000}%
\pgfsetstrokecolor{currentstroke}%
\pgfsetdash{}{0pt}%
\pgfpathmoveto{\pgfqpoint{0.659479in}{3.405105in}}%
\pgfpathcurveto{\pgfqpoint{0.670530in}{3.405105in}}{\pgfqpoint{0.681129in}{3.409496in}}{\pgfqpoint{0.688942in}{3.417309in}}%
\pgfpathcurveto{\pgfqpoint{0.696756in}{3.425123in}}{\pgfqpoint{0.701146in}{3.435722in}}{\pgfqpoint{0.701146in}{3.446772in}}%
\pgfpathcurveto{\pgfqpoint{0.701146in}{3.457822in}}{\pgfqpoint{0.696756in}{3.468421in}}{\pgfqpoint{0.688942in}{3.476235in}}%
\pgfpathcurveto{\pgfqpoint{0.681129in}{3.484048in}}{\pgfqpoint{0.670530in}{3.488439in}}{\pgfqpoint{0.659479in}{3.488439in}}%
\pgfpathcurveto{\pgfqpoint{0.648429in}{3.488439in}}{\pgfqpoint{0.637830in}{3.484048in}}{\pgfqpoint{0.630017in}{3.476235in}}%
\pgfpathcurveto{\pgfqpoint{0.622203in}{3.468421in}}{\pgfqpoint{0.617813in}{3.457822in}}{\pgfqpoint{0.617813in}{3.446772in}}%
\pgfpathcurveto{\pgfqpoint{0.617813in}{3.435722in}}{\pgfqpoint{0.622203in}{3.425123in}}{\pgfqpoint{0.630017in}{3.417309in}}%
\pgfpathcurveto{\pgfqpoint{0.637830in}{3.409496in}}{\pgfqpoint{0.648429in}{3.405105in}}{\pgfqpoint{0.659479in}{3.405105in}}%
\pgfpathlineto{\pgfqpoint{0.659479in}{3.405105in}}%
\pgfpathclose%
\pgfusepath{stroke}%
\end{pgfscope}%
\begin{pgfscope}%
\pgfpathrectangle{\pgfqpoint{0.494722in}{0.437222in}}{\pgfqpoint{6.275590in}{5.159444in}}%
\pgfusepath{clip}%
\pgfsetbuttcap%
\pgfsetroundjoin%
\pgfsetlinewidth{1.003750pt}%
\definecolor{currentstroke}{rgb}{1.000000,0.000000,0.000000}%
\pgfsetstrokecolor{currentstroke}%
\pgfsetdash{}{0pt}%
\pgfpathmoveto{\pgfqpoint{0.499724in}{4.422023in}}%
\pgfpathcurveto{\pgfqpoint{0.510775in}{4.422023in}}{\pgfqpoint{0.521374in}{4.426413in}}{\pgfqpoint{0.529187in}{4.434227in}}%
\pgfpathcurveto{\pgfqpoint{0.537001in}{4.442041in}}{\pgfqpoint{0.541391in}{4.452640in}}{\pgfqpoint{0.541391in}{4.463690in}}%
\pgfpathcurveto{\pgfqpoint{0.541391in}{4.474740in}}{\pgfqpoint{0.537001in}{4.485339in}}{\pgfqpoint{0.529187in}{4.493152in}}%
\pgfpathcurveto{\pgfqpoint{0.521374in}{4.500966in}}{\pgfqpoint{0.510775in}{4.505356in}}{\pgfqpoint{0.499724in}{4.505356in}}%
\pgfpathcurveto{\pgfqpoint{0.488674in}{4.505356in}}{\pgfqpoint{0.478075in}{4.500966in}}{\pgfqpoint{0.470262in}{4.493152in}}%
\pgfpathcurveto{\pgfqpoint{0.462448in}{4.485339in}}{\pgfqpoint{0.458058in}{4.474740in}}{\pgfqpoint{0.458058in}{4.463690in}}%
\pgfpathcurveto{\pgfqpoint{0.458058in}{4.452640in}}{\pgfqpoint{0.462448in}{4.442041in}}{\pgfqpoint{0.470262in}{4.434227in}}%
\pgfpathcurveto{\pgfqpoint{0.478075in}{4.426413in}}{\pgfqpoint{0.488674in}{4.422023in}}{\pgfqpoint{0.499724in}{4.422023in}}%
\pgfpathlineto{\pgfqpoint{0.499724in}{4.422023in}}%
\pgfpathclose%
\pgfusepath{stroke}%
\end{pgfscope}%
\begin{pgfscope}%
\pgfpathrectangle{\pgfqpoint{0.494722in}{0.437222in}}{\pgfqpoint{6.275590in}{5.159444in}}%
\pgfusepath{clip}%
\pgfsetbuttcap%
\pgfsetroundjoin%
\definecolor{currentfill}{rgb}{0.121569,0.466667,0.705882}%
\pgfsetfillcolor{currentfill}%
\pgfsetlinewidth{1.003750pt}%
\definecolor{currentstroke}{rgb}{0.121569,0.466667,0.705882}%
\pgfsetstrokecolor{currentstroke}%
\pgfsetdash{}{0pt}%
\pgfpathmoveto{\pgfqpoint{0.729588in}{3.177806in}}%
\pgfpathcurveto{\pgfqpoint{0.746061in}{3.177806in}}{\pgfqpoint{0.761861in}{3.184350in}}{\pgfqpoint{0.773509in}{3.195998in}}%
\pgfpathcurveto{\pgfqpoint{0.785157in}{3.207646in}}{\pgfqpoint{0.791701in}{3.223446in}}{\pgfqpoint{0.791701in}{3.239919in}}%
\pgfpathcurveto{\pgfqpoint{0.791701in}{3.256391in}}{\pgfqpoint{0.785157in}{3.272192in}}{\pgfqpoint{0.773509in}{3.283839in}}%
\pgfpathcurveto{\pgfqpoint{0.761861in}{3.295487in}}{\pgfqpoint{0.746061in}{3.302032in}}{\pgfqpoint{0.729588in}{3.302032in}}%
\pgfpathcurveto{\pgfqpoint{0.713116in}{3.302032in}}{\pgfqpoint{0.697316in}{3.295487in}}{\pgfqpoint{0.685668in}{3.283839in}}%
\pgfpathcurveto{\pgfqpoint{0.674020in}{3.272192in}}{\pgfqpoint{0.667475in}{3.256391in}}{\pgfqpoint{0.667475in}{3.239919in}}%
\pgfpathcurveto{\pgfqpoint{0.667475in}{3.223446in}}{\pgfqpoint{0.674020in}{3.207646in}}{\pgfqpoint{0.685668in}{3.195998in}}%
\pgfpathcurveto{\pgfqpoint{0.697316in}{3.184350in}}{\pgfqpoint{0.713116in}{3.177806in}}{\pgfqpoint{0.729588in}{3.177806in}}%
\pgfpathlineto{\pgfqpoint{0.729588in}{3.177806in}}%
\pgfpathclose%
\pgfusepath{stroke,fill}%
\end{pgfscope}%
\begin{pgfscope}%
\pgfpathrectangle{\pgfqpoint{0.494722in}{0.437222in}}{\pgfqpoint{6.275590in}{5.159444in}}%
\pgfusepath{clip}%
\pgfsetbuttcap%
\pgfsetroundjoin%
\definecolor{currentfill}{rgb}{0.172549,0.627451,0.172549}%
\pgfsetfillcolor{currentfill}%
\pgfsetlinewidth{1.003750pt}%
\definecolor{currentstroke}{rgb}{0.172549,0.627451,0.172549}%
\pgfsetstrokecolor{currentstroke}%
\pgfsetdash{}{0pt}%
\pgfpathmoveto{\pgfqpoint{5.744566in}{0.380943in}}%
\pgfpathcurveto{\pgfqpoint{5.761039in}{0.380943in}}{\pgfqpoint{5.776839in}{0.387487in}}{\pgfqpoint{5.788487in}{0.399135in}}%
\pgfpathcurveto{\pgfqpoint{5.800135in}{0.410783in}}{\pgfqpoint{5.806679in}{0.426583in}}{\pgfqpoint{5.806679in}{0.443056in}}%
\pgfpathcurveto{\pgfqpoint{5.806679in}{0.459528in}}{\pgfqpoint{5.800135in}{0.475328in}}{\pgfqpoint{5.788487in}{0.486976in}}%
\pgfpathcurveto{\pgfqpoint{5.776839in}{0.498624in}}{\pgfqpoint{5.761039in}{0.505169in}}{\pgfqpoint{5.744566in}{0.505169in}}%
\pgfpathcurveto{\pgfqpoint{5.728094in}{0.505169in}}{\pgfqpoint{5.712294in}{0.498624in}}{\pgfqpoint{5.700646in}{0.486976in}}%
\pgfpathcurveto{\pgfqpoint{5.688998in}{0.475328in}}{\pgfqpoint{5.682453in}{0.459528in}}{\pgfqpoint{5.682453in}{0.443056in}}%
\pgfpathcurveto{\pgfqpoint{5.682453in}{0.426583in}}{\pgfqpoint{5.688998in}{0.410783in}}{\pgfqpoint{5.700646in}{0.399135in}}%
\pgfpathcurveto{\pgfqpoint{5.712294in}{0.387487in}}{\pgfqpoint{5.728094in}{0.380943in}}{\pgfqpoint{5.744566in}{0.380943in}}%
\pgfpathlineto{\pgfqpoint{5.744566in}{0.380943in}}%
\pgfpathclose%
\pgfusepath{stroke,fill}%
\end{pgfscope}%
\begin{pgfscope}%
\pgfpathrectangle{\pgfqpoint{0.494722in}{0.437222in}}{\pgfqpoint{6.275590in}{5.159444in}}%
\pgfusepath{clip}%
\pgfsetbuttcap%
\pgfsetroundjoin%
\definecolor{currentfill}{rgb}{0.549020,0.337255,0.294118}%
\pgfsetfillcolor{currentfill}%
\pgfsetlinewidth{1.003750pt}%
\definecolor{currentstroke}{rgb}{0.549020,0.337255,0.294118}%
\pgfsetstrokecolor{currentstroke}%
\pgfsetdash{}{0pt}%
\pgfpathmoveto{\pgfqpoint{0.496144in}{4.520290in}}%
\pgfpathcurveto{\pgfqpoint{0.512617in}{4.520290in}}{\pgfqpoint{0.528417in}{4.526834in}}{\pgfqpoint{0.540065in}{4.538482in}}%
\pgfpathcurveto{\pgfqpoint{0.551713in}{4.550130in}}{\pgfqpoint{0.558257in}{4.565930in}}{\pgfqpoint{0.558257in}{4.582403in}}%
\pgfpathcurveto{\pgfqpoint{0.558257in}{4.598875in}}{\pgfqpoint{0.551713in}{4.614675in}}{\pgfqpoint{0.540065in}{4.626323in}}%
\pgfpathcurveto{\pgfqpoint{0.528417in}{4.637971in}}{\pgfqpoint{0.512617in}{4.644516in}}{\pgfqpoint{0.496144in}{4.644516in}}%
\pgfpathcurveto{\pgfqpoint{0.479672in}{4.644516in}}{\pgfqpoint{0.463872in}{4.637971in}}{\pgfqpoint{0.452224in}{4.626323in}}%
\pgfpathcurveto{\pgfqpoint{0.440576in}{4.614675in}}{\pgfqpoint{0.434031in}{4.598875in}}{\pgfqpoint{0.434031in}{4.582403in}}%
\pgfpathcurveto{\pgfqpoint{0.434031in}{4.565930in}}{\pgfqpoint{0.440576in}{4.550130in}}{\pgfqpoint{0.452224in}{4.538482in}}%
\pgfpathcurveto{\pgfqpoint{0.463872in}{4.526834in}}{\pgfqpoint{0.479672in}{4.520290in}}{\pgfqpoint{0.496144in}{4.520290in}}%
\pgfpathlineto{\pgfqpoint{0.496144in}{4.520290in}}%
\pgfpathclose%
\pgfusepath{stroke,fill}%
\end{pgfscope}%
\begin{pgfscope}%
\pgfpathrectangle{\pgfqpoint{0.494722in}{0.437222in}}{\pgfqpoint{6.275590in}{5.159444in}}%
\pgfusepath{clip}%
\pgfsetbuttcap%
\pgfsetroundjoin%
\definecolor{currentfill}{rgb}{0.498039,0.498039,0.498039}%
\pgfsetfillcolor{currentfill}%
\pgfsetlinewidth{1.003750pt}%
\definecolor{currentstroke}{rgb}{0.498039,0.498039,0.498039}%
\pgfsetstrokecolor{currentstroke}%
\pgfsetdash{}{0pt}%
\pgfpathmoveto{\pgfqpoint{5.600353in}{0.473805in}}%
\pgfpathcurveto{\pgfqpoint{5.616826in}{0.473805in}}{\pgfqpoint{5.632626in}{0.480349in}}{\pgfqpoint{5.644273in}{0.491997in}}%
\pgfpathcurveto{\pgfqpoint{5.655921in}{0.503645in}}{\pgfqpoint{5.662466in}{0.519445in}}{\pgfqpoint{5.662466in}{0.535918in}}%
\pgfpathcurveto{\pgfqpoint{5.662466in}{0.552390in}}{\pgfqpoint{5.655921in}{0.568190in}}{\pgfqpoint{5.644273in}{0.579838in}}%
\pgfpathcurveto{\pgfqpoint{5.632626in}{0.591486in}}{\pgfqpoint{5.616826in}{0.598031in}}{\pgfqpoint{5.600353in}{0.598031in}}%
\pgfpathcurveto{\pgfqpoint{5.583880in}{0.598031in}}{\pgfqpoint{5.568080in}{0.591486in}}{\pgfqpoint{5.556432in}{0.579838in}}%
\pgfpathcurveto{\pgfqpoint{5.544785in}{0.568190in}}{\pgfqpoint{5.538240in}{0.552390in}}{\pgfqpoint{5.538240in}{0.535918in}}%
\pgfpathcurveto{\pgfqpoint{5.538240in}{0.519445in}}{\pgfqpoint{5.544785in}{0.503645in}}{\pgfqpoint{5.556432in}{0.491997in}}%
\pgfpathcurveto{\pgfqpoint{5.568080in}{0.480349in}}{\pgfqpoint{5.583880in}{0.473805in}}{\pgfqpoint{5.600353in}{0.473805in}}%
\pgfpathlineto{\pgfqpoint{5.600353in}{0.473805in}}%
\pgfpathclose%
\pgfusepath{stroke,fill}%
\end{pgfscope}%
\begin{pgfscope}%
\pgfpathrectangle{\pgfqpoint{0.494722in}{0.437222in}}{\pgfqpoint{6.275590in}{5.159444in}}%
\pgfusepath{clip}%
\pgfsetbuttcap%
\pgfsetroundjoin%
\definecolor{currentfill}{rgb}{0.090196,0.745098,0.811765}%
\pgfsetfillcolor{currentfill}%
\pgfsetlinewidth{1.003750pt}%
\definecolor{currentstroke}{rgb}{0.090196,0.745098,0.811765}%
\pgfsetstrokecolor{currentstroke}%
\pgfsetdash{}{0pt}%
\pgfpathmoveto{\pgfqpoint{1.146637in}{2.912547in}}%
\pgfpathcurveto{\pgfqpoint{1.163110in}{2.912547in}}{\pgfqpoint{1.178910in}{2.919092in}}{\pgfqpoint{1.190558in}{2.930739in}}%
\pgfpathcurveto{\pgfqpoint{1.202205in}{2.942387in}}{\pgfqpoint{1.208750in}{2.958187in}}{\pgfqpoint{1.208750in}{2.974660in}}%
\pgfpathcurveto{\pgfqpoint{1.208750in}{2.991133in}}{\pgfqpoint{1.202205in}{3.006933in}}{\pgfqpoint{1.190558in}{3.018581in}}%
\pgfpathcurveto{\pgfqpoint{1.178910in}{3.030228in}}{\pgfqpoint{1.163110in}{3.036773in}}{\pgfqpoint{1.146637in}{3.036773in}}%
\pgfpathcurveto{\pgfqpoint{1.130165in}{3.036773in}}{\pgfqpoint{1.114364in}{3.030228in}}{\pgfqpoint{1.102717in}{3.018581in}}%
\pgfpathcurveto{\pgfqpoint{1.091069in}{3.006933in}}{\pgfqpoint{1.084524in}{2.991133in}}{\pgfqpoint{1.084524in}{2.974660in}}%
\pgfpathcurveto{\pgfqpoint{1.084524in}{2.958187in}}{\pgfqpoint{1.091069in}{2.942387in}}{\pgfqpoint{1.102717in}{2.930739in}}%
\pgfpathcurveto{\pgfqpoint{1.114364in}{2.919092in}}{\pgfqpoint{1.130165in}{2.912547in}}{\pgfqpoint{1.146637in}{2.912547in}}%
\pgfpathlineto{\pgfqpoint{1.146637in}{2.912547in}}%
\pgfpathclose%
\pgfusepath{stroke,fill}%
\end{pgfscope}%
\begin{pgfscope}%
\pgfpathrectangle{\pgfqpoint{0.494722in}{0.437222in}}{\pgfqpoint{6.275590in}{5.159444in}}%
\pgfusepath{clip}%
\pgfsetbuttcap%
\pgfsetmiterjoin%
\definecolor{currentfill}{rgb}{0.827451,0.827451,0.827451}%
\pgfsetfillcolor{currentfill}%
\pgfsetfillopacity{0.500000}%
\pgfsetlinewidth{0.000000pt}%
\definecolor{currentstroke}{rgb}{0.000000,0.000000,0.000000}%
\pgfsetstrokecolor{currentstroke}%
\pgfsetstrokeopacity{0.500000}%
\pgfsetdash{}{0pt}%
\pgfpathmoveto{\pgfqpoint{0.494722in}{4.675337in}}%
\pgfpathlineto{\pgfqpoint{0.495523in}{4.582744in}}%
\pgfpathlineto{\pgfqpoint{0.497924in}{4.491090in}}%
\pgfpathlineto{\pgfqpoint{0.501926in}{4.400377in}}%
\pgfpathlineto{\pgfqpoint{0.507528in}{4.310603in}}%
\pgfpathlineto{\pgfqpoint{0.514732in}{4.221770in}}%
\pgfpathlineto{\pgfqpoint{0.523536in}{4.133877in}}%
\pgfpathlineto{\pgfqpoint{0.533941in}{4.046923in}}%
\pgfpathlineto{\pgfqpoint{0.545946in}{3.960910in}}%
\pgfpathlineto{\pgfqpoint{0.559553in}{3.875837in}}%
\pgfpathlineto{\pgfqpoint{0.574760in}{3.791703in}}%
\pgfpathlineto{\pgfqpoint{0.591568in}{3.708510in}}%
\pgfpathlineto{\pgfqpoint{0.609976in}{3.626257in}}%
\pgfpathlineto{\pgfqpoint{0.629986in}{3.544944in}}%
\pgfpathlineto{\pgfqpoint{0.651596in}{3.464571in}}%
\pgfpathlineto{\pgfqpoint{0.674807in}{3.385138in}}%
\pgfpathlineto{\pgfqpoint{0.699619in}{3.306645in}}%
\pgfpathlineto{\pgfqpoint{0.726031in}{3.229092in}}%
\pgfpathlineto{\pgfqpoint{0.754044in}{3.152479in}}%
\pgfpathlineto{\pgfqpoint{0.783658in}{3.076806in}}%
\pgfpathlineto{\pgfqpoint{0.814873in}{3.002073in}}%
\pgfpathlineto{\pgfqpoint{0.847688in}{2.928280in}}%
\pgfpathlineto{\pgfqpoint{0.882104in}{2.855427in}}%
\pgfpathlineto{\pgfqpoint{0.918121in}{2.783515in}}%
\pgfpathlineto{\pgfqpoint{0.955739in}{2.712542in}}%
\pgfpathlineto{\pgfqpoint{0.994957in}{2.642509in}}%
\pgfpathlineto{\pgfqpoint{1.035777in}{2.573417in}}%
\pgfpathlineto{\pgfqpoint{1.078197in}{2.505264in}}%
\pgfpathlineto{\pgfqpoint{1.122217in}{2.438051in}}%
\pgfpathlineto{\pgfqpoint{1.167839in}{2.371779in}}%
\pgfpathlineto{\pgfqpoint{1.215061in}{2.306446in}}%
\pgfpathlineto{\pgfqpoint{1.263884in}{2.242054in}}%
\pgfpathlineto{\pgfqpoint{1.314308in}{2.178601in}}%
\pgfpathlineto{\pgfqpoint{1.366332in}{2.116089in}}%
\pgfpathlineto{\pgfqpoint{1.419957in}{2.054517in}}%
\pgfpathlineto{\pgfqpoint{1.475183in}{1.993884in}}%
\pgfpathlineto{\pgfqpoint{1.532010in}{1.934192in}}%
\pgfpathlineto{\pgfqpoint{1.590437in}{1.875440in}}%
\pgfpathlineto{\pgfqpoint{1.650466in}{1.817628in}}%
\pgfpathlineto{\pgfqpoint{1.712095in}{1.760755in}}%
\pgfpathlineto{\pgfqpoint{1.775324in}{1.704823in}}%
\pgfpathlineto{\pgfqpoint{1.840155in}{1.649831in}}%
\pgfpathlineto{\pgfqpoint{1.906586in}{1.595779in}}%
\pgfpathlineto{\pgfqpoint{1.974618in}{1.542667in}}%
\pgfpathlineto{\pgfqpoint{2.044251in}{1.490495in}}%
\pgfpathlineto{\pgfqpoint{2.115484in}{1.439263in}}%
\pgfpathlineto{\pgfqpoint{2.188318in}{1.388971in}}%
\pgfpathlineto{\pgfqpoint{2.262753in}{1.339619in}}%
\pgfpathlineto{\pgfqpoint{2.338789in}{1.291207in}}%
\pgfpathlineto{\pgfqpoint{2.416426in}{1.243736in}}%
\pgfpathlineto{\pgfqpoint{2.495663in}{1.197204in}}%
\pgfpathlineto{\pgfqpoint{2.576501in}{1.151612in}}%
\pgfpathlineto{\pgfqpoint{2.658940in}{1.106960in}}%
\pgfpathlineto{\pgfqpoint{2.742979in}{1.063249in}}%
\pgfpathlineto{\pgfqpoint{2.828619in}{1.020477in}}%
\pgfpathlineto{\pgfqpoint{2.915860in}{0.978645in}}%
\pgfpathlineto{\pgfqpoint{3.004702in}{0.937754in}}%
\pgfpathlineto{\pgfqpoint{3.095145in}{0.897802in}}%
\pgfpathlineto{\pgfqpoint{3.187188in}{0.858791in}}%
\pgfpathlineto{\pgfqpoint{3.280832in}{0.820719in}}%
\pgfpathlineto{\pgfqpoint{3.347407in}{0.794671in}}%
\pgfpathlineto{\pgfqpoint{3.395830in}{0.776575in}}%
\pgfpathlineto{\pgfqpoint{3.445053in}{0.758950in}}%
\pgfpathlineto{\pgfqpoint{3.495077in}{0.741794in}}%
\pgfpathlineto{\pgfqpoint{3.545901in}{0.725108in}}%
\pgfpathlineto{\pgfqpoint{3.597525in}{0.708893in}}%
\pgfpathlineto{\pgfqpoint{3.649950in}{0.693147in}}%
\pgfpathlineto{\pgfqpoint{3.703175in}{0.677871in}}%
\pgfpathlineto{\pgfqpoint{3.757200in}{0.663066in}}%
\pgfpathlineto{\pgfqpoint{3.812026in}{0.648730in}}%
\pgfpathlineto{\pgfqpoint{3.867652in}{0.634865in}}%
\pgfpathlineto{\pgfqpoint{3.924078in}{0.621469in}}%
\pgfpathlineto{\pgfqpoint{3.981305in}{0.608544in}}%
\pgfpathlineto{\pgfqpoint{4.039333in}{0.596088in}}%
\pgfpathlineto{\pgfqpoint{4.098160in}{0.584103in}}%
\pgfpathlineto{\pgfqpoint{4.157788in}{0.572587in}}%
\pgfpathlineto{\pgfqpoint{4.218217in}{0.561542in}}%
\pgfpathlineto{\pgfqpoint{4.279445in}{0.550967in}}%
\pgfpathlineto{\pgfqpoint{4.341475in}{0.540861in}}%
\pgfpathlineto{\pgfqpoint{4.404304in}{0.531226in}}%
\pgfpathlineto{\pgfqpoint{4.467934in}{0.522060in}}%
\pgfpathlineto{\pgfqpoint{4.532364in}{0.513365in}}%
\pgfpathlineto{\pgfqpoint{4.597595in}{0.505140in}}%
\pgfpathlineto{\pgfqpoint{4.663626in}{0.497385in}}%
\pgfpathlineto{\pgfqpoint{4.730458in}{0.490099in}}%
\pgfpathlineto{\pgfqpoint{4.798089in}{0.483284in}}%
\pgfpathlineto{\pgfqpoint{4.866521in}{0.476939in}}%
\pgfpathlineto{\pgfqpoint{4.935754in}{0.471064in}}%
\pgfpathlineto{\pgfqpoint{5.005787in}{0.465658in}}%
\pgfpathlineto{\pgfqpoint{5.076620in}{0.460723in}}%
\pgfpathlineto{\pgfqpoint{5.148254in}{0.456258in}}%
\pgfpathlineto{\pgfqpoint{5.220688in}{0.452263in}}%
\pgfpathlineto{\pgfqpoint{5.293922in}{0.448738in}}%
\pgfpathlineto{\pgfqpoint{5.367957in}{0.445683in}}%
\pgfpathlineto{\pgfqpoint{5.442792in}{0.443097in}}%
\pgfpathlineto{\pgfqpoint{5.518428in}{0.440982in}}%
\pgfpathlineto{\pgfqpoint{5.594864in}{0.439337in}}%
\pgfpathlineto{\pgfqpoint{5.672100in}{0.438162in}}%
\pgfpathlineto{\pgfqpoint{5.750137in}{0.437457in}}%
\pgfpathlineto{\pgfqpoint{5.828974in}{0.437222in}}%
\pgfpathlineto{\pgfqpoint{6.895824in}{0.510929in}}%
\pgfpathlineto{\pgfqpoint{6.801220in}{0.511211in}}%
\pgfpathlineto{\pgfqpoint{6.707576in}{0.512057in}}%
\pgfpathlineto{\pgfqpoint{6.614892in}{0.513467in}}%
\pgfpathlineto{\pgfqpoint{6.523169in}{0.515441in}}%
\pgfpathlineto{\pgfqpoint{6.432406in}{0.517979in}}%
\pgfpathlineto{\pgfqpoint{6.342604in}{0.521081in}}%
\pgfpathlineto{\pgfqpoint{6.253762in}{0.524747in}}%
\pgfpathlineto{\pgfqpoint{6.165881in}{0.528977in}}%
\pgfpathlineto{\pgfqpoint{6.078960in}{0.533771in}}%
\pgfpathlineto{\pgfqpoint{5.993000in}{0.539130in}}%
\pgfpathlineto{\pgfqpoint{5.908000in}{0.545052in}}%
\pgfpathlineto{\pgfqpoint{5.823960in}{0.551538in}}%
\pgfpathlineto{\pgfqpoint{5.740881in}{0.558588in}}%
\pgfpathlineto{\pgfqpoint{5.658763in}{0.566203in}}%
\pgfpathlineto{\pgfqpoint{5.577605in}{0.574381in}}%
\pgfpathlineto{\pgfqpoint{5.497407in}{0.583123in}}%
\pgfpathlineto{\pgfqpoint{5.418170in}{0.592430in}}%
\pgfpathlineto{\pgfqpoint{5.339893in}{0.602300in}}%
\pgfpathlineto{\pgfqpoint{5.262576in}{0.612734in}}%
\pgfpathlineto{\pgfqpoint{5.186221in}{0.623733in}}%
\pgfpathlineto{\pgfqpoint{5.110825in}{0.635295in}}%
\pgfpathlineto{\pgfqpoint{5.036390in}{0.647422in}}%
\pgfpathlineto{\pgfqpoint{4.962916in}{0.660112in}}%
\pgfpathlineto{\pgfqpoint{4.890402in}{0.673367in}}%
\pgfpathlineto{\pgfqpoint{4.818848in}{0.687185in}}%
\pgfpathlineto{\pgfqpoint{4.748255in}{0.701568in}}%
\pgfpathlineto{\pgfqpoint{4.678622in}{0.716514in}}%
\pgfpathlineto{\pgfqpoint{4.609950in}{0.732025in}}%
\pgfpathlineto{\pgfqpoint{4.542238in}{0.748100in}}%
\pgfpathlineto{\pgfqpoint{4.475486in}{0.764738in}}%
\pgfpathlineto{\pgfqpoint{4.409696in}{0.781941in}}%
\pgfpathlineto{\pgfqpoint{4.344865in}{0.799708in}}%
\pgfpathlineto{\pgfqpoint{4.280995in}{0.818038in}}%
\pgfpathlineto{\pgfqpoint{4.218085in}{0.836933in}}%
\pgfpathlineto{\pgfqpoint{4.156136in}{0.856392in}}%
\pgfpathlineto{\pgfqpoint{4.095148in}{0.876415in}}%
\pgfpathlineto{\pgfqpoint{4.035119in}{0.897001in}}%
\pgfpathlineto{\pgfqpoint{3.976052in}{0.918152in}}%
\pgfpathlineto{\pgfqpoint{3.917944in}{0.939867in}}%
\pgfpathlineto{\pgfqpoint{3.838054in}{0.971125in}}%
\pgfpathlineto{\pgfqpoint{3.725681in}{1.016811in}}%
\pgfpathlineto{\pgfqpoint{3.615229in}{1.063625in}}%
\pgfpathlineto{\pgfqpoint{3.506698in}{1.111567in}}%
\pgfpathlineto{\pgfqpoint{3.400088in}{1.160636in}}%
\pgfpathlineto{\pgfqpoint{3.295399in}{1.210834in}}%
\pgfpathlineto{\pgfqpoint{3.192631in}{1.262160in}}%
\pgfpathlineto{\pgfqpoint{3.091783in}{1.314614in}}%
\pgfpathlineto{\pgfqpoint{2.992857in}{1.368196in}}%
\pgfpathlineto{\pgfqpoint{2.895851in}{1.422906in}}%
\pgfpathlineto{\pgfqpoint{2.800766in}{1.478745in}}%
\pgfpathlineto{\pgfqpoint{2.707603in}{1.535711in}}%
\pgfpathlineto{\pgfqpoint{2.616360in}{1.593805in}}%
\pgfpathlineto{\pgfqpoint{2.527038in}{1.653027in}}%
\pgfpathlineto{\pgfqpoint{2.439637in}{1.713378in}}%
\pgfpathlineto{\pgfqpoint{2.354156in}{1.774856in}}%
\pgfpathlineto{\pgfqpoint{2.270597in}{1.837462in}}%
\pgfpathlineto{\pgfqpoint{2.188959in}{1.901197in}}%
\pgfpathlineto{\pgfqpoint{2.109241in}{1.966059in}}%
\pgfpathlineto{\pgfqpoint{2.031445in}{2.032050in}}%
\pgfpathlineto{\pgfqpoint{1.955569in}{2.099168in}}%
\pgfpathlineto{\pgfqpoint{1.881614in}{2.167415in}}%
\pgfpathlineto{\pgfqpoint{1.809580in}{2.236790in}}%
\pgfpathlineto{\pgfqpoint{1.739467in}{2.307292in}}%
\pgfpathlineto{\pgfqpoint{1.671275in}{2.378923in}}%
\pgfpathlineto{\pgfqpoint{1.605004in}{2.451682in}}%
\pgfpathlineto{\pgfqpoint{1.540654in}{2.525569in}}%
\pgfpathlineto{\pgfqpoint{1.478225in}{2.600584in}}%
\pgfpathlineto{\pgfqpoint{1.417716in}{2.676727in}}%
\pgfpathlineto{\pgfqpoint{1.359129in}{2.753998in}}%
\pgfpathlineto{\pgfqpoint{1.302462in}{2.832397in}}%
\pgfpathlineto{\pgfqpoint{1.247716in}{2.911924in}}%
\pgfpathlineto{\pgfqpoint{1.194891in}{2.992579in}}%
\pgfpathlineto{\pgfqpoint{1.143987in}{3.074362in}}%
\pgfpathlineto{\pgfqpoint{1.095004in}{3.157273in}}%
\pgfpathlineto{\pgfqpoint{1.047942in}{3.241312in}}%
\pgfpathlineto{\pgfqpoint{1.002801in}{3.326480in}}%
\pgfpathlineto{\pgfqpoint{0.959581in}{3.412775in}}%
\pgfpathlineto{\pgfqpoint{0.918281in}{3.500198in}}%
\pgfpathlineto{\pgfqpoint{0.878903in}{3.588750in}}%
\pgfpathlineto{\pgfqpoint{0.841445in}{3.678429in}}%
\pgfpathlineto{\pgfqpoint{0.805909in}{3.769237in}}%
\pgfpathlineto{\pgfqpoint{0.772293in}{3.861172in}}%
\pgfpathlineto{\pgfqpoint{0.740598in}{3.954236in}}%
\pgfpathlineto{\pgfqpoint{0.710824in}{4.048427in}}%
\pgfpathlineto{\pgfqpoint{0.682971in}{4.143747in}}%
\pgfpathlineto{\pgfqpoint{0.657039in}{4.240195in}}%
\pgfpathlineto{\pgfqpoint{0.633027in}{4.337770in}}%
\pgfpathlineto{\pgfqpoint{0.610937in}{4.436474in}}%
\pgfpathlineto{\pgfqpoint{0.590767in}{4.536306in}}%
\pgfpathlineto{\pgfqpoint{0.572519in}{4.637266in}}%
\pgfpathlineto{\pgfqpoint{0.556191in}{4.739354in}}%
\pgfpathlineto{\pgfqpoint{0.541784in}{4.842570in}}%
\pgfpathlineto{\pgfqpoint{0.529298in}{4.946914in}}%
\pgfpathlineto{\pgfqpoint{0.518734in}{5.052386in}}%
\pgfpathlineto{\pgfqpoint{0.510089in}{5.158986in}}%
\pgfpathlineto{\pgfqpoint{0.503366in}{5.266714in}}%
\pgfpathlineto{\pgfqpoint{0.498564in}{5.375570in}}%
\pgfpathlineto{\pgfqpoint{0.495683in}{5.485554in}}%
\pgfpathlineto{\pgfqpoint{0.494722in}{5.596667in}}%
\pgfpathlineto{\pgfqpoint{0.494722in}{4.675337in}}%
\pgfpathclose%
\pgfusepath{fill}%
\end{pgfscope}%
\begin{pgfscope}%
\pgfsetbuttcap%
\pgfsetroundjoin%
\definecolor{currentfill}{rgb}{0.000000,0.000000,0.000000}%
\pgfsetfillcolor{currentfill}%
\pgfsetlinewidth{0.803000pt}%
\definecolor{currentstroke}{rgb}{0.000000,0.000000,0.000000}%
\pgfsetstrokecolor{currentstroke}%
\pgfsetdash{}{0pt}%
\pgfsys@defobject{currentmarker}{\pgfqpoint{0.000000in}{-0.048611in}}{\pgfqpoint{0.000000in}{0.000000in}}{%
\pgfpathmoveto{\pgfqpoint{0.000000in}{0.000000in}}%
\pgfpathlineto{\pgfqpoint{0.000000in}{-0.048611in}}%
\pgfusepath{stroke,fill}%
}%
\begin{pgfscope}%
\pgfsys@transformshift{0.494722in}{0.437222in}%
\pgfsys@useobject{currentmarker}{}%
\end{pgfscope}%
\end{pgfscope}%
\begin{pgfscope}%
\definecolor{textcolor}{rgb}{0.000000,0.000000,0.000000}%
\pgfsetstrokecolor{textcolor}%
\pgfsetfillcolor{textcolor}%
\pgftext[x=0.494722in,y=0.340000in,,top]{\color{textcolor}{\sffamily\fontsize{14.000000}{16.800000}\selectfont\catcode`\^=\active\def^{\ifmmode\sp\else\^{}\fi}\catcode`\%=\active\def%{\%}0}}%
\end{pgfscope}%
\begin{pgfscope}%
\pgfsetbuttcap%
\pgfsetroundjoin%
\definecolor{currentfill}{rgb}{0.000000,0.000000,0.000000}%
\pgfsetfillcolor{currentfill}%
\pgfsetlinewidth{0.803000pt}%
\definecolor{currentstroke}{rgb}{0.000000,0.000000,0.000000}%
\pgfsetstrokecolor{currentstroke}%
\pgfsetdash{}{0pt}%
\pgfsys@defobject{currentmarker}{\pgfqpoint{0.000000in}{-0.048611in}}{\pgfqpoint{0.000000in}{0.000000in}}{%
\pgfpathmoveto{\pgfqpoint{0.000000in}{0.000000in}}%
\pgfpathlineto{\pgfqpoint{0.000000in}{-0.048611in}}%
\pgfusepath{stroke,fill}%
}%
\begin{pgfscope}%
\pgfsys@transformshift{1.279171in}{0.437222in}%
\pgfsys@useobject{currentmarker}{}%
\end{pgfscope}%
\end{pgfscope}%
\begin{pgfscope}%
\definecolor{textcolor}{rgb}{0.000000,0.000000,0.000000}%
\pgfsetstrokecolor{textcolor}%
\pgfsetfillcolor{textcolor}%
\pgftext[x=1.279171in,y=0.340000in,,top]{\color{textcolor}{\sffamily\fontsize{14.000000}{16.800000}\selectfont\catcode`\^=\active\def^{\ifmmode\sp\else\^{}\fi}\catcode`\%=\active\def%{\%}20}}%
\end{pgfscope}%
\begin{pgfscope}%
\pgfsetbuttcap%
\pgfsetroundjoin%
\definecolor{currentfill}{rgb}{0.000000,0.000000,0.000000}%
\pgfsetfillcolor{currentfill}%
\pgfsetlinewidth{0.803000pt}%
\definecolor{currentstroke}{rgb}{0.000000,0.000000,0.000000}%
\pgfsetstrokecolor{currentstroke}%
\pgfsetdash{}{0pt}%
\pgfsys@defobject{currentmarker}{\pgfqpoint{0.000000in}{-0.048611in}}{\pgfqpoint{0.000000in}{0.000000in}}{%
\pgfpathmoveto{\pgfqpoint{0.000000in}{0.000000in}}%
\pgfpathlineto{\pgfqpoint{0.000000in}{-0.048611in}}%
\pgfusepath{stroke,fill}%
}%
\begin{pgfscope}%
\pgfsys@transformshift{2.063620in}{0.437222in}%
\pgfsys@useobject{currentmarker}{}%
\end{pgfscope}%
\end{pgfscope}%
\begin{pgfscope}%
\definecolor{textcolor}{rgb}{0.000000,0.000000,0.000000}%
\pgfsetstrokecolor{textcolor}%
\pgfsetfillcolor{textcolor}%
\pgftext[x=2.063620in,y=0.340000in,,top]{\color{textcolor}{\sffamily\fontsize{14.000000}{16.800000}\selectfont\catcode`\^=\active\def^{\ifmmode\sp\else\^{}\fi}\catcode`\%=\active\def%{\%}40}}%
\end{pgfscope}%
\begin{pgfscope}%
\pgfsetbuttcap%
\pgfsetroundjoin%
\definecolor{currentfill}{rgb}{0.000000,0.000000,0.000000}%
\pgfsetfillcolor{currentfill}%
\pgfsetlinewidth{0.803000pt}%
\definecolor{currentstroke}{rgb}{0.000000,0.000000,0.000000}%
\pgfsetstrokecolor{currentstroke}%
\pgfsetdash{}{0pt}%
\pgfsys@defobject{currentmarker}{\pgfqpoint{0.000000in}{-0.048611in}}{\pgfqpoint{0.000000in}{0.000000in}}{%
\pgfpathmoveto{\pgfqpoint{0.000000in}{0.000000in}}%
\pgfpathlineto{\pgfqpoint{0.000000in}{-0.048611in}}%
\pgfusepath{stroke,fill}%
}%
\begin{pgfscope}%
\pgfsys@transformshift{2.848069in}{0.437222in}%
\pgfsys@useobject{currentmarker}{}%
\end{pgfscope}%
\end{pgfscope}%
\begin{pgfscope}%
\definecolor{textcolor}{rgb}{0.000000,0.000000,0.000000}%
\pgfsetstrokecolor{textcolor}%
\pgfsetfillcolor{textcolor}%
\pgftext[x=2.848069in,y=0.340000in,,top]{\color{textcolor}{\sffamily\fontsize{14.000000}{16.800000}\selectfont\catcode`\^=\active\def^{\ifmmode\sp\else\^{}\fi}\catcode`\%=\active\def%{\%}60}}%
\end{pgfscope}%
\begin{pgfscope}%
\pgfsetbuttcap%
\pgfsetroundjoin%
\definecolor{currentfill}{rgb}{0.000000,0.000000,0.000000}%
\pgfsetfillcolor{currentfill}%
\pgfsetlinewidth{0.803000pt}%
\definecolor{currentstroke}{rgb}{0.000000,0.000000,0.000000}%
\pgfsetstrokecolor{currentstroke}%
\pgfsetdash{}{0pt}%
\pgfsys@defobject{currentmarker}{\pgfqpoint{0.000000in}{-0.048611in}}{\pgfqpoint{0.000000in}{0.000000in}}{%
\pgfpathmoveto{\pgfqpoint{0.000000in}{0.000000in}}%
\pgfpathlineto{\pgfqpoint{0.000000in}{-0.048611in}}%
\pgfusepath{stroke,fill}%
}%
\begin{pgfscope}%
\pgfsys@transformshift{3.632517in}{0.437222in}%
\pgfsys@useobject{currentmarker}{}%
\end{pgfscope}%
\end{pgfscope}%
\begin{pgfscope}%
\definecolor{textcolor}{rgb}{0.000000,0.000000,0.000000}%
\pgfsetstrokecolor{textcolor}%
\pgfsetfillcolor{textcolor}%
\pgftext[x=3.632517in,y=0.340000in,,top]{\color{textcolor}{\sffamily\fontsize{14.000000}{16.800000}\selectfont\catcode`\^=\active\def^{\ifmmode\sp\else\^{}\fi}\catcode`\%=\active\def%{\%}80}}%
\end{pgfscope}%
\begin{pgfscope}%
\pgfsetbuttcap%
\pgfsetroundjoin%
\definecolor{currentfill}{rgb}{0.000000,0.000000,0.000000}%
\pgfsetfillcolor{currentfill}%
\pgfsetlinewidth{0.803000pt}%
\definecolor{currentstroke}{rgb}{0.000000,0.000000,0.000000}%
\pgfsetstrokecolor{currentstroke}%
\pgfsetdash{}{0pt}%
\pgfsys@defobject{currentmarker}{\pgfqpoint{0.000000in}{-0.048611in}}{\pgfqpoint{0.000000in}{0.000000in}}{%
\pgfpathmoveto{\pgfqpoint{0.000000in}{0.000000in}}%
\pgfpathlineto{\pgfqpoint{0.000000in}{-0.048611in}}%
\pgfusepath{stroke,fill}%
}%
\begin{pgfscope}%
\pgfsys@transformshift{4.416966in}{0.437222in}%
\pgfsys@useobject{currentmarker}{}%
\end{pgfscope}%
\end{pgfscope}%
\begin{pgfscope}%
\definecolor{textcolor}{rgb}{0.000000,0.000000,0.000000}%
\pgfsetstrokecolor{textcolor}%
\pgfsetfillcolor{textcolor}%
\pgftext[x=4.416966in,y=0.340000in,,top]{\color{textcolor}{\sffamily\fontsize{14.000000}{16.800000}\selectfont\catcode`\^=\active\def^{\ifmmode\sp\else\^{}\fi}\catcode`\%=\active\def%{\%}100}}%
\end{pgfscope}%
\begin{pgfscope}%
\pgfsetbuttcap%
\pgfsetroundjoin%
\definecolor{currentfill}{rgb}{0.000000,0.000000,0.000000}%
\pgfsetfillcolor{currentfill}%
\pgfsetlinewidth{0.803000pt}%
\definecolor{currentstroke}{rgb}{0.000000,0.000000,0.000000}%
\pgfsetstrokecolor{currentstroke}%
\pgfsetdash{}{0pt}%
\pgfsys@defobject{currentmarker}{\pgfqpoint{0.000000in}{-0.048611in}}{\pgfqpoint{0.000000in}{0.000000in}}{%
\pgfpathmoveto{\pgfqpoint{0.000000in}{0.000000in}}%
\pgfpathlineto{\pgfqpoint{0.000000in}{-0.048611in}}%
\pgfusepath{stroke,fill}%
}%
\begin{pgfscope}%
\pgfsys@transformshift{5.201415in}{0.437222in}%
\pgfsys@useobject{currentmarker}{}%
\end{pgfscope}%
\end{pgfscope}%
\begin{pgfscope}%
\definecolor{textcolor}{rgb}{0.000000,0.000000,0.000000}%
\pgfsetstrokecolor{textcolor}%
\pgfsetfillcolor{textcolor}%
\pgftext[x=5.201415in,y=0.340000in,,top]{\color{textcolor}{\sffamily\fontsize{14.000000}{16.800000}\selectfont\catcode`\^=\active\def^{\ifmmode\sp\else\^{}\fi}\catcode`\%=\active\def%{\%}120}}%
\end{pgfscope}%
\begin{pgfscope}%
\pgfsetbuttcap%
\pgfsetroundjoin%
\definecolor{currentfill}{rgb}{0.000000,0.000000,0.000000}%
\pgfsetfillcolor{currentfill}%
\pgfsetlinewidth{0.803000pt}%
\definecolor{currentstroke}{rgb}{0.000000,0.000000,0.000000}%
\pgfsetstrokecolor{currentstroke}%
\pgfsetdash{}{0pt}%
\pgfsys@defobject{currentmarker}{\pgfqpoint{0.000000in}{-0.048611in}}{\pgfqpoint{0.000000in}{0.000000in}}{%
\pgfpathmoveto{\pgfqpoint{0.000000in}{0.000000in}}%
\pgfpathlineto{\pgfqpoint{0.000000in}{-0.048611in}}%
\pgfusepath{stroke,fill}%
}%
\begin{pgfscope}%
\pgfsys@transformshift{5.985864in}{0.437222in}%
\pgfsys@useobject{currentmarker}{}%
\end{pgfscope}%
\end{pgfscope}%
\begin{pgfscope}%
\definecolor{textcolor}{rgb}{0.000000,0.000000,0.000000}%
\pgfsetstrokecolor{textcolor}%
\pgfsetfillcolor{textcolor}%
\pgftext[x=5.985864in,y=0.340000in,,top]{\color{textcolor}{\sffamily\fontsize{14.000000}{16.800000}\selectfont\catcode`\^=\active\def^{\ifmmode\sp\else\^{}\fi}\catcode`\%=\active\def%{\%}140}}%
\end{pgfscope}%
\begin{pgfscope}%
\pgfsetbuttcap%
\pgfsetroundjoin%
\definecolor{currentfill}{rgb}{0.000000,0.000000,0.000000}%
\pgfsetfillcolor{currentfill}%
\pgfsetlinewidth{0.803000pt}%
\definecolor{currentstroke}{rgb}{0.000000,0.000000,0.000000}%
\pgfsetstrokecolor{currentstroke}%
\pgfsetdash{}{0pt}%
\pgfsys@defobject{currentmarker}{\pgfqpoint{0.000000in}{-0.048611in}}{\pgfqpoint{0.000000in}{0.000000in}}{%
\pgfpathmoveto{\pgfqpoint{0.000000in}{0.000000in}}%
\pgfpathlineto{\pgfqpoint{0.000000in}{-0.048611in}}%
\pgfusepath{stroke,fill}%
}%
\begin{pgfscope}%
\pgfsys@transformshift{6.770313in}{0.437222in}%
\pgfsys@useobject{currentmarker}{}%
\end{pgfscope}%
\end{pgfscope}%
\begin{pgfscope}%
\definecolor{textcolor}{rgb}{0.000000,0.000000,0.000000}%
\pgfsetstrokecolor{textcolor}%
\pgfsetfillcolor{textcolor}%
\pgftext[x=6.770313in,y=0.340000in,,top]{\color{textcolor}{\sffamily\fontsize{14.000000}{16.800000}\selectfont\catcode`\^=\active\def^{\ifmmode\sp\else\^{}\fi}\catcode`\%=\active\def%{\%}160}}%
\end{pgfscope}%
\begin{pgfscope}%
\pgfsetbuttcap%
\pgfsetroundjoin%
\definecolor{currentfill}{rgb}{0.000000,0.000000,0.000000}%
\pgfsetfillcolor{currentfill}%
\pgfsetlinewidth{0.803000pt}%
\definecolor{currentstroke}{rgb}{0.000000,0.000000,0.000000}%
\pgfsetstrokecolor{currentstroke}%
\pgfsetdash{}{0pt}%
\pgfsys@defobject{currentmarker}{\pgfqpoint{-0.048611in}{0.000000in}}{\pgfqpoint{-0.000000in}{0.000000in}}{%
\pgfpathmoveto{\pgfqpoint{-0.000000in}{0.000000in}}%
\pgfpathlineto{\pgfqpoint{-0.048611in}{0.000000in}}%
\pgfusepath{stroke,fill}%
}%
\begin{pgfscope}%
\pgfsys@transformshift{0.494722in}{0.990020in}%
\pgfsys@useobject{currentmarker}{}%
\end{pgfscope}%
\end{pgfscope}%
\begin{pgfscope}%
\definecolor{textcolor}{rgb}{0.000000,0.000000,0.000000}%
\pgfsetstrokecolor{textcolor}%
\pgfsetfillcolor{textcolor}%
\pgftext[x=0.150077in, y=0.916154in, left, base]{\color{textcolor}{\sffamily\fontsize{14.000000}{16.800000}\selectfont\catcode`\^=\active\def^{\ifmmode\sp\else\^{}\fi}\catcode`\%=\active\def%{\%}10}}%
\end{pgfscope}%
\begin{pgfscope}%
\pgfsetbuttcap%
\pgfsetroundjoin%
\definecolor{currentfill}{rgb}{0.000000,0.000000,0.000000}%
\pgfsetfillcolor{currentfill}%
\pgfsetlinewidth{0.803000pt}%
\definecolor{currentstroke}{rgb}{0.000000,0.000000,0.000000}%
\pgfsetstrokecolor{currentstroke}%
\pgfsetdash{}{0pt}%
\pgfsys@defobject{currentmarker}{\pgfqpoint{-0.048611in}{0.000000in}}{\pgfqpoint{-0.000000in}{0.000000in}}{%
\pgfpathmoveto{\pgfqpoint{-0.000000in}{0.000000in}}%
\pgfpathlineto{\pgfqpoint{-0.048611in}{0.000000in}}%
\pgfusepath{stroke,fill}%
}%
\begin{pgfscope}%
\pgfsys@transformshift{0.494722in}{1.911349in}%
\pgfsys@useobject{currentmarker}{}%
\end{pgfscope}%
\end{pgfscope}%
\begin{pgfscope}%
\definecolor{textcolor}{rgb}{0.000000,0.000000,0.000000}%
\pgfsetstrokecolor{textcolor}%
\pgfsetfillcolor{textcolor}%
\pgftext[x=0.150077in, y=1.837483in, left, base]{\color{textcolor}{\sffamily\fontsize{14.000000}{16.800000}\selectfont\catcode`\^=\active\def^{\ifmmode\sp\else\^{}\fi}\catcode`\%=\active\def%{\%}20}}%
\end{pgfscope}%
\begin{pgfscope}%
\pgfsetbuttcap%
\pgfsetroundjoin%
\definecolor{currentfill}{rgb}{0.000000,0.000000,0.000000}%
\pgfsetfillcolor{currentfill}%
\pgfsetlinewidth{0.803000pt}%
\definecolor{currentstroke}{rgb}{0.000000,0.000000,0.000000}%
\pgfsetstrokecolor{currentstroke}%
\pgfsetdash{}{0pt}%
\pgfsys@defobject{currentmarker}{\pgfqpoint{-0.048611in}{0.000000in}}{\pgfqpoint{-0.000000in}{0.000000in}}{%
\pgfpathmoveto{\pgfqpoint{-0.000000in}{0.000000in}}%
\pgfpathlineto{\pgfqpoint{-0.048611in}{0.000000in}}%
\pgfusepath{stroke,fill}%
}%
\begin{pgfscope}%
\pgfsys@transformshift{0.494722in}{2.832679in}%
\pgfsys@useobject{currentmarker}{}%
\end{pgfscope}%
\end{pgfscope}%
\begin{pgfscope}%
\definecolor{textcolor}{rgb}{0.000000,0.000000,0.000000}%
\pgfsetstrokecolor{textcolor}%
\pgfsetfillcolor{textcolor}%
\pgftext[x=0.150077in, y=2.758812in, left, base]{\color{textcolor}{\sffamily\fontsize{14.000000}{16.800000}\selectfont\catcode`\^=\active\def^{\ifmmode\sp\else\^{}\fi}\catcode`\%=\active\def%{\%}30}}%
\end{pgfscope}%
\begin{pgfscope}%
\pgfsetbuttcap%
\pgfsetroundjoin%
\definecolor{currentfill}{rgb}{0.000000,0.000000,0.000000}%
\pgfsetfillcolor{currentfill}%
\pgfsetlinewidth{0.803000pt}%
\definecolor{currentstroke}{rgb}{0.000000,0.000000,0.000000}%
\pgfsetstrokecolor{currentstroke}%
\pgfsetdash{}{0pt}%
\pgfsys@defobject{currentmarker}{\pgfqpoint{-0.048611in}{0.000000in}}{\pgfqpoint{-0.000000in}{0.000000in}}{%
\pgfpathmoveto{\pgfqpoint{-0.000000in}{0.000000in}}%
\pgfpathlineto{\pgfqpoint{-0.048611in}{0.000000in}}%
\pgfusepath{stroke,fill}%
}%
\begin{pgfscope}%
\pgfsys@transformshift{0.494722in}{3.754008in}%
\pgfsys@useobject{currentmarker}{}%
\end{pgfscope}%
\end{pgfscope}%
\begin{pgfscope}%
\definecolor{textcolor}{rgb}{0.000000,0.000000,0.000000}%
\pgfsetstrokecolor{textcolor}%
\pgfsetfillcolor{textcolor}%
\pgftext[x=0.150077in, y=3.680142in, left, base]{\color{textcolor}{\sffamily\fontsize{14.000000}{16.800000}\selectfont\catcode`\^=\active\def^{\ifmmode\sp\else\^{}\fi}\catcode`\%=\active\def%{\%}40}}%
\end{pgfscope}%
\begin{pgfscope}%
\pgfsetbuttcap%
\pgfsetroundjoin%
\definecolor{currentfill}{rgb}{0.000000,0.000000,0.000000}%
\pgfsetfillcolor{currentfill}%
\pgfsetlinewidth{0.803000pt}%
\definecolor{currentstroke}{rgb}{0.000000,0.000000,0.000000}%
\pgfsetstrokecolor{currentstroke}%
\pgfsetdash{}{0pt}%
\pgfsys@defobject{currentmarker}{\pgfqpoint{-0.048611in}{0.000000in}}{\pgfqpoint{-0.000000in}{0.000000in}}{%
\pgfpathmoveto{\pgfqpoint{-0.000000in}{0.000000in}}%
\pgfpathlineto{\pgfqpoint{-0.048611in}{0.000000in}}%
\pgfusepath{stroke,fill}%
}%
\begin{pgfscope}%
\pgfsys@transformshift{0.494722in}{4.675337in}%
\pgfsys@useobject{currentmarker}{}%
\end{pgfscope}%
\end{pgfscope}%
\begin{pgfscope}%
\definecolor{textcolor}{rgb}{0.000000,0.000000,0.000000}%
\pgfsetstrokecolor{textcolor}%
\pgfsetfillcolor{textcolor}%
\pgftext[x=0.150077in, y=4.601471in, left, base]{\color{textcolor}{\sffamily\fontsize{14.000000}{16.800000}\selectfont\catcode`\^=\active\def^{\ifmmode\sp\else\^{}\fi}\catcode`\%=\active\def%{\%}50}}%
\end{pgfscope}%
\begin{pgfscope}%
\pgfsetbuttcap%
\pgfsetroundjoin%
\definecolor{currentfill}{rgb}{0.000000,0.000000,0.000000}%
\pgfsetfillcolor{currentfill}%
\pgfsetlinewidth{0.803000pt}%
\definecolor{currentstroke}{rgb}{0.000000,0.000000,0.000000}%
\pgfsetstrokecolor{currentstroke}%
\pgfsetdash{}{0pt}%
\pgfsys@defobject{currentmarker}{\pgfqpoint{-0.048611in}{0.000000in}}{\pgfqpoint{-0.000000in}{0.000000in}}{%
\pgfpathmoveto{\pgfqpoint{-0.000000in}{0.000000in}}%
\pgfpathlineto{\pgfqpoint{-0.048611in}{0.000000in}}%
\pgfusepath{stroke,fill}%
}%
\begin{pgfscope}%
\pgfsys@transformshift{0.494722in}{5.596667in}%
\pgfsys@useobject{currentmarker}{}%
\end{pgfscope}%
\end{pgfscope}%
\begin{pgfscope}%
\definecolor{textcolor}{rgb}{0.000000,0.000000,0.000000}%
\pgfsetstrokecolor{textcolor}%
\pgfsetfillcolor{textcolor}%
\pgftext[x=0.150077in, y=5.522801in, left, base]{\color{textcolor}{\sffamily\fontsize{14.000000}{16.800000}\selectfont\catcode`\^=\active\def^{\ifmmode\sp\else\^{}\fi}\catcode`\%=\active\def%{\%}60}}%
\end{pgfscope}%
\begin{pgfscope}%
\pgfsetrectcap%
\pgfsetmiterjoin%
\pgfsetlinewidth{0.803000pt}%
\definecolor{currentstroke}{rgb}{0.000000,0.000000,0.000000}%
\pgfsetstrokecolor{currentstroke}%
\pgfsetdash{}{0pt}%
\pgfpathmoveto{\pgfqpoint{0.494722in}{0.437222in}}%
\pgfpathlineto{\pgfqpoint{0.494722in}{5.596667in}}%
\pgfusepath{stroke}%
\end{pgfscope}%
\begin{pgfscope}%
\pgfsetrectcap%
\pgfsetmiterjoin%
\pgfsetlinewidth{0.803000pt}%
\definecolor{currentstroke}{rgb}{0.000000,0.000000,0.000000}%
\pgfsetstrokecolor{currentstroke}%
\pgfsetdash{}{0pt}%
\pgfpathmoveto{\pgfqpoint{6.770313in}{0.437222in}}%
\pgfpathlineto{\pgfqpoint{6.770313in}{5.596667in}}%
\pgfusepath{stroke}%
\end{pgfscope}%
\begin{pgfscope}%
\pgfsetrectcap%
\pgfsetmiterjoin%
\pgfsetlinewidth{0.803000pt}%
\definecolor{currentstroke}{rgb}{0.000000,0.000000,0.000000}%
\pgfsetstrokecolor{currentstroke}%
\pgfsetdash{}{0pt}%
\pgfpathmoveto{\pgfqpoint{0.494722in}{0.437222in}}%
\pgfpathlineto{\pgfqpoint{6.770313in}{0.437222in}}%
\pgfusepath{stroke}%
\end{pgfscope}%
\begin{pgfscope}%
\pgfsetrectcap%
\pgfsetmiterjoin%
\pgfsetlinewidth{0.803000pt}%
\definecolor{currentstroke}{rgb}{0.000000,0.000000,0.000000}%
\pgfsetstrokecolor{currentstroke}%
\pgfsetdash{}{0pt}%
\pgfpathmoveto{\pgfqpoint{0.494722in}{5.596667in}}%
\pgfpathlineto{\pgfqpoint{6.770313in}{5.596667in}}%
\pgfusepath{stroke}%
\end{pgfscope}%
\begin{pgfscope}%
\definecolor{textcolor}{rgb}{0.000000,0.000000,0.000000}%
\pgfsetstrokecolor{textcolor}%
\pgfsetfillcolor{textcolor}%
\pgftext[x=3.632517in,y=5.680000in,,base]{\color{textcolor}{\sffamily\fontsize{16.000000}{19.200000}\selectfont\catcode`\^=\active\def^{\ifmmode\sp\else\^{}\fi}\catcode`\%=\active\def%{\%}Objective Space}}%
\end{pgfscope}%
\begin{pgfscope}%
\pgfsetbuttcap%
\pgfsetmiterjoin%
\definecolor{currentfill}{rgb}{1.000000,1.000000,1.000000}%
\pgfsetfillcolor{currentfill}%
\pgfsetfillopacity{0.800000}%
\pgfsetlinewidth{1.003750pt}%
\definecolor{currentstroke}{rgb}{0.800000,0.800000,0.800000}%
\pgfsetstrokecolor{currentstroke}%
\pgfsetstrokeopacity{0.800000}%
\pgfsetdash{}{0pt}%
\pgfpathmoveto{\pgfqpoint{4.118994in}{4.299510in}}%
\pgfpathlineto{\pgfqpoint{6.634201in}{4.299510in}}%
\pgfpathquadraticcurveto{\pgfqpoint{6.673090in}{4.299510in}}{\pgfqpoint{6.673090in}{4.338399in}}%
\pgfpathlineto{\pgfqpoint{6.673090in}{5.460556in}}%
\pgfpathquadraticcurveto{\pgfqpoint{6.673090in}{5.499444in}}{\pgfqpoint{6.634201in}{5.499444in}}%
\pgfpathlineto{\pgfqpoint{4.118994in}{5.499444in}}%
\pgfpathquadraticcurveto{\pgfqpoint{4.080105in}{5.499444in}}{\pgfqpoint{4.080105in}{5.460556in}}%
\pgfpathlineto{\pgfqpoint{4.080105in}{4.338399in}}%
\pgfpathquadraticcurveto{\pgfqpoint{4.080105in}{4.299510in}}{\pgfqpoint{4.118994in}{4.299510in}}%
\pgfpathlineto{\pgfqpoint{4.118994in}{4.299510in}}%
\pgfpathclose%
\pgfusepath{stroke,fill}%
\end{pgfscope}%
\begin{pgfscope}%
\pgfsetbuttcap%
\pgfsetroundjoin%
\pgfsetlinewidth{1.003750pt}%
\definecolor{currentstroke}{rgb}{0.827451,0.827451,0.827451}%
\pgfsetstrokecolor{currentstroke}%
\pgfsetstrokeopacity{0.800000}%
\pgfsetdash{}{0pt}%
\pgfpathmoveto{\pgfqpoint{4.352328in}{5.283309in}}%
\pgfpathcurveto{\pgfqpoint{4.363378in}{5.283309in}}{\pgfqpoint{4.373977in}{5.287700in}}{\pgfqpoint{4.381790in}{5.295513in}}%
\pgfpathcurveto{\pgfqpoint{4.389604in}{5.303327in}}{\pgfqpoint{4.393994in}{5.313926in}}{\pgfqpoint{4.393994in}{5.324976in}}%
\pgfpathcurveto{\pgfqpoint{4.393994in}{5.336026in}}{\pgfqpoint{4.389604in}{5.346625in}}{\pgfqpoint{4.381790in}{5.354439in}}%
\pgfpathcurveto{\pgfqpoint{4.373977in}{5.362253in}}{\pgfqpoint{4.363378in}{5.366643in}}{\pgfqpoint{4.352328in}{5.366643in}}%
\pgfpathcurveto{\pgfqpoint{4.341277in}{5.366643in}}{\pgfqpoint{4.330678in}{5.362253in}}{\pgfqpoint{4.322865in}{5.354439in}}%
\pgfpathcurveto{\pgfqpoint{4.315051in}{5.346625in}}{\pgfqpoint{4.310661in}{5.336026in}}{\pgfqpoint{4.310661in}{5.324976in}}%
\pgfpathcurveto{\pgfqpoint{4.310661in}{5.313926in}}{\pgfqpoint{4.315051in}{5.303327in}}{\pgfqpoint{4.322865in}{5.295513in}}%
\pgfpathcurveto{\pgfqpoint{4.330678in}{5.287700in}}{\pgfqpoint{4.341277in}{5.283309in}}{\pgfqpoint{4.352328in}{5.283309in}}%
\pgfpathlineto{\pgfqpoint{4.352328in}{5.283309in}}%
\pgfpathclose%
\pgfusepath{stroke}%
\end{pgfscope}%
\begin{pgfscope}%
\definecolor{textcolor}{rgb}{0.000000,0.000000,0.000000}%
\pgfsetstrokecolor{textcolor}%
\pgfsetfillcolor{textcolor}%
\pgftext[x=4.702327in,y=5.273934in,left,base]{\color{textcolor}{\sffamily\fontsize{14.000000}{16.800000}\selectfont\catcode`\^=\active\def^{\ifmmode\sp\else\^{}\fi}\catcode`\%=\active\def%{\%}tested points}}%
\end{pgfscope}%
\begin{pgfscope}%
\pgfsetbuttcap%
\pgfsetroundjoin%
\pgfsetlinewidth{1.003750pt}%
\definecolor{currentstroke}{rgb}{1.000000,0.000000,0.000000}%
\pgfsetstrokecolor{currentstroke}%
\pgfsetdash{}{0pt}%
\pgfpathmoveto{\pgfqpoint{4.352328in}{4.997909in}}%
\pgfpathcurveto{\pgfqpoint{4.363378in}{4.997909in}}{\pgfqpoint{4.373977in}{5.002299in}}{\pgfqpoint{4.381790in}{5.010113in}}%
\pgfpathcurveto{\pgfqpoint{4.389604in}{5.017927in}}{\pgfqpoint{4.393994in}{5.028526in}}{\pgfqpoint{4.393994in}{5.039576in}}%
\pgfpathcurveto{\pgfqpoint{4.393994in}{5.050626in}}{\pgfqpoint{4.389604in}{5.061225in}}{\pgfqpoint{4.381790in}{5.069039in}}%
\pgfpathcurveto{\pgfqpoint{4.373977in}{5.076852in}}{\pgfqpoint{4.363378in}{5.081243in}}{\pgfqpoint{4.352328in}{5.081243in}}%
\pgfpathcurveto{\pgfqpoint{4.341277in}{5.081243in}}{\pgfqpoint{4.330678in}{5.076852in}}{\pgfqpoint{4.322865in}{5.069039in}}%
\pgfpathcurveto{\pgfqpoint{4.315051in}{5.061225in}}{\pgfqpoint{4.310661in}{5.050626in}}{\pgfqpoint{4.310661in}{5.039576in}}%
\pgfpathcurveto{\pgfqpoint{4.310661in}{5.028526in}}{\pgfqpoint{4.315051in}{5.017927in}}{\pgfqpoint{4.322865in}{5.010113in}}%
\pgfpathcurveto{\pgfqpoint{4.330678in}{5.002299in}}{\pgfqpoint{4.341277in}{4.997909in}}{\pgfqpoint{4.352328in}{4.997909in}}%
\pgfpathlineto{\pgfqpoint{4.352328in}{4.997909in}}%
\pgfpathclose%
\pgfusepath{stroke}%
\end{pgfscope}%
\begin{pgfscope}%
\definecolor{textcolor}{rgb}{0.000000,0.000000,0.000000}%
\pgfsetstrokecolor{textcolor}%
\pgfsetfillcolor{textcolor}%
\pgftext[x=4.702327in,y=4.988534in,left,base]{\color{textcolor}{\sffamily\fontsize{14.000000}{16.800000}\selectfont\catcode`\^=\active\def^{\ifmmode\sp\else\^{}\fi}\catcode`\%=\active\def%{\%}optimal points}}%
\end{pgfscope}%
\begin{pgfscope}%
\pgfsetbuttcap%
\pgfsetroundjoin%
\definecolor{currentfill}{rgb}{0.121569,0.466667,0.705882}%
\pgfsetfillcolor{currentfill}%
\pgfsetlinewidth{1.003750pt}%
\definecolor{currentstroke}{rgb}{0.121569,0.466667,0.705882}%
\pgfsetstrokecolor{currentstroke}%
\pgfsetdash{}{0pt}%
\pgfpathmoveto{\pgfqpoint{4.352328in}{4.692063in}}%
\pgfpathcurveto{\pgfqpoint{4.368800in}{4.692063in}}{\pgfqpoint{4.384600in}{4.698607in}}{\pgfqpoint{4.396248in}{4.710255in}}%
\pgfpathcurveto{\pgfqpoint{4.407896in}{4.721903in}}{\pgfqpoint{4.414440in}{4.737703in}}{\pgfqpoint{4.414440in}{4.754176in}}%
\pgfpathcurveto{\pgfqpoint{4.414440in}{4.770648in}}{\pgfqpoint{4.407896in}{4.786448in}}{\pgfqpoint{4.396248in}{4.798096in}}%
\pgfpathcurveto{\pgfqpoint{4.384600in}{4.809744in}}{\pgfqpoint{4.368800in}{4.816289in}}{\pgfqpoint{4.352328in}{4.816289in}}%
\pgfpathcurveto{\pgfqpoint{4.335855in}{4.816289in}}{\pgfqpoint{4.320055in}{4.809744in}}{\pgfqpoint{4.308407in}{4.798096in}}%
\pgfpathcurveto{\pgfqpoint{4.296759in}{4.786448in}}{\pgfqpoint{4.290215in}{4.770648in}}{\pgfqpoint{4.290215in}{4.754176in}}%
\pgfpathcurveto{\pgfqpoint{4.290215in}{4.737703in}}{\pgfqpoint{4.296759in}{4.721903in}}{\pgfqpoint{4.308407in}{4.710255in}}%
\pgfpathcurveto{\pgfqpoint{4.320055in}{4.698607in}}{\pgfqpoint{4.335855in}{4.692063in}}{\pgfqpoint{4.352328in}{4.692063in}}%
\pgfpathlineto{\pgfqpoint{4.352328in}{4.692063in}}%
\pgfpathclose%
\pgfusepath{stroke,fill}%
\end{pgfscope}%
\begin{pgfscope}%
\definecolor{textcolor}{rgb}{0.000000,0.000000,0.000000}%
\pgfsetstrokecolor{textcolor}%
\pgfsetfillcolor{textcolor}%
\pgftext[x=4.702327in,y=4.703134in,left,base]{\color{textcolor}{\sffamily\fontsize{14.000000}{16.800000}\selectfont\catcode`\^=\active\def^{\ifmmode\sp\else\^{}\fi}\catcode`\%=\active\def%{\%}selected points}}%
\end{pgfscope}%
\begin{pgfscope}%
\pgfsetbuttcap%
\pgfsetmiterjoin%
\definecolor{currentfill}{rgb}{0.827451,0.827451,0.827451}%
\pgfsetfillcolor{currentfill}%
\pgfsetfillopacity{0.500000}%
\pgfsetlinewidth{0.000000pt}%
\definecolor{currentstroke}{rgb}{0.000000,0.000000,0.000000}%
\pgfsetstrokecolor{currentstroke}%
\pgfsetstrokeopacity{0.500000}%
\pgfsetdash{}{0pt}%
\pgfpathmoveto{\pgfqpoint{4.157883in}{4.417734in}}%
\pgfpathlineto{\pgfqpoint{4.546772in}{4.417734in}}%
\pgfpathlineto{\pgfqpoint{4.546772in}{4.553845in}}%
\pgfpathlineto{\pgfqpoint{4.157883in}{4.553845in}}%
\pgfpathlineto{\pgfqpoint{4.157883in}{4.417734in}}%
\pgfpathclose%
\pgfusepath{fill}%
\end{pgfscope}%
\begin{pgfscope}%
\definecolor{textcolor}{rgb}{0.000000,0.000000,0.000000}%
\pgfsetstrokecolor{textcolor}%
\pgfsetfillcolor{textcolor}%
\pgftext[x=4.702327in,y=4.417734in,left,base]{\color{textcolor}{\sffamily\fontsize{14.000000}{16.800000}\selectfont\catcode`\^=\active\def^{\ifmmode\sp\else\^{}\fi}\catcode`\%=\active\def%{\%}Near-optimal space}}%
\end{pgfscope}%
\begin{pgfscope}%
\pgfsetbuttcap%
\pgfsetmiterjoin%
\definecolor{currentfill}{rgb}{1.000000,1.000000,1.000000}%
\pgfsetfillcolor{currentfill}%
\pgfsetlinewidth{0.000000pt}%
\definecolor{currentstroke}{rgb}{0.000000,0.000000,0.000000}%
\pgfsetstrokecolor{currentstroke}%
\pgfsetstrokeopacity{0.000000}%
\pgfsetdash{}{0pt}%
\pgfpathmoveto{\pgfqpoint{7.512535in}{0.437222in}}%
\pgfpathlineto{\pgfqpoint{13.788125in}{0.437222in}}%
\pgfpathlineto{\pgfqpoint{13.788125in}{5.596667in}}%
\pgfpathlineto{\pgfqpoint{7.512535in}{5.596667in}}%
\pgfpathlineto{\pgfqpoint{7.512535in}{0.437222in}}%
\pgfpathclose%
\pgfusepath{fill}%
\end{pgfscope}%
\begin{pgfscope}%
\pgfpathrectangle{\pgfqpoint{7.512535in}{0.437222in}}{\pgfqpoint{6.275590in}{5.159444in}}%
\pgfusepath{clip}%
\pgfsetbuttcap%
\pgfsetroundjoin%
\pgfsetlinewidth{1.003750pt}%
\definecolor{currentstroke}{rgb}{0.827451,0.827451,0.827451}%
\pgfsetstrokecolor{currentstroke}%
\pgfsetstrokeopacity{0.800000}%
\pgfsetdash{}{0pt}%
\pgfpathmoveto{\pgfqpoint{8.714432in}{3.605291in}}%
\pgfpathcurveto{\pgfqpoint{8.725482in}{3.605291in}}{\pgfqpoint{8.736081in}{3.609681in}}{\pgfqpoint{8.743895in}{3.617495in}}%
\pgfpathcurveto{\pgfqpoint{8.751709in}{3.625309in}}{\pgfqpoint{8.756099in}{3.635908in}}{\pgfqpoint{8.756099in}{3.646958in}}%
\pgfpathcurveto{\pgfqpoint{8.756099in}{3.658008in}}{\pgfqpoint{8.751709in}{3.668607in}}{\pgfqpoint{8.743895in}{3.676421in}}%
\pgfpathcurveto{\pgfqpoint{8.736081in}{3.684234in}}{\pgfqpoint{8.725482in}{3.688625in}}{\pgfqpoint{8.714432in}{3.688625in}}%
\pgfpathcurveto{\pgfqpoint{8.703382in}{3.688625in}}{\pgfqpoint{8.692783in}{3.684234in}}{\pgfqpoint{8.684970in}{3.676421in}}%
\pgfpathcurveto{\pgfqpoint{8.677156in}{3.668607in}}{\pgfqpoint{8.672766in}{3.658008in}}{\pgfqpoint{8.672766in}{3.646958in}}%
\pgfpathcurveto{\pgfqpoint{8.672766in}{3.635908in}}{\pgfqpoint{8.677156in}{3.625309in}}{\pgfqpoint{8.684970in}{3.617495in}}%
\pgfpathcurveto{\pgfqpoint{8.692783in}{3.609681in}}{\pgfqpoint{8.703382in}{3.605291in}}{\pgfqpoint{8.714432in}{3.605291in}}%
\pgfpathlineto{\pgfqpoint{8.714432in}{3.605291in}}%
\pgfpathclose%
\pgfusepath{stroke}%
\end{pgfscope}%
\begin{pgfscope}%
\pgfpathrectangle{\pgfqpoint{7.512535in}{0.437222in}}{\pgfqpoint{6.275590in}{5.159444in}}%
\pgfusepath{clip}%
\pgfsetbuttcap%
\pgfsetroundjoin%
\pgfsetlinewidth{1.003750pt}%
\definecolor{currentstroke}{rgb}{0.827451,0.827451,0.827451}%
\pgfsetstrokecolor{currentstroke}%
\pgfsetstrokeopacity{0.800000}%
\pgfsetdash{}{0pt}%
\pgfpathmoveto{\pgfqpoint{10.259535in}{4.447570in}}%
\pgfpathcurveto{\pgfqpoint{10.270585in}{4.447570in}}{\pgfqpoint{10.281184in}{4.451960in}}{\pgfqpoint{10.288997in}{4.459773in}}%
\pgfpathcurveto{\pgfqpoint{10.296811in}{4.467587in}}{\pgfqpoint{10.301201in}{4.478186in}}{\pgfqpoint{10.301201in}{4.489236in}}%
\pgfpathcurveto{\pgfqpoint{10.301201in}{4.500286in}}{\pgfqpoint{10.296811in}{4.510885in}}{\pgfqpoint{10.288997in}{4.518699in}}%
\pgfpathcurveto{\pgfqpoint{10.281184in}{4.526513in}}{\pgfqpoint{10.270585in}{4.530903in}}{\pgfqpoint{10.259535in}{4.530903in}}%
\pgfpathcurveto{\pgfqpoint{10.248485in}{4.530903in}}{\pgfqpoint{10.237886in}{4.526513in}}{\pgfqpoint{10.230072in}{4.518699in}}%
\pgfpathcurveto{\pgfqpoint{10.222258in}{4.510885in}}{\pgfqpoint{10.217868in}{4.500286in}}{\pgfqpoint{10.217868in}{4.489236in}}%
\pgfpathcurveto{\pgfqpoint{10.217868in}{4.478186in}}{\pgfqpoint{10.222258in}{4.467587in}}{\pgfqpoint{10.230072in}{4.459773in}}%
\pgfpathcurveto{\pgfqpoint{10.237886in}{4.451960in}}{\pgfqpoint{10.248485in}{4.447570in}}{\pgfqpoint{10.259535in}{4.447570in}}%
\pgfpathlineto{\pgfqpoint{10.259535in}{4.447570in}}%
\pgfpathclose%
\pgfusepath{stroke}%
\end{pgfscope}%
\begin{pgfscope}%
\pgfpathrectangle{\pgfqpoint{7.512535in}{0.437222in}}{\pgfqpoint{6.275590in}{5.159444in}}%
\pgfusepath{clip}%
\pgfsetbuttcap%
\pgfsetroundjoin%
\pgfsetlinewidth{1.003750pt}%
\definecolor{currentstroke}{rgb}{0.827451,0.827451,0.827451}%
\pgfsetstrokecolor{currentstroke}%
\pgfsetstrokeopacity{0.800000}%
\pgfsetdash{}{0pt}%
\pgfpathmoveto{\pgfqpoint{9.247511in}{4.532771in}}%
\pgfpathcurveto{\pgfqpoint{9.258561in}{4.532771in}}{\pgfqpoint{9.269160in}{4.537161in}}{\pgfqpoint{9.276974in}{4.544974in}}%
\pgfpathcurveto{\pgfqpoint{9.284788in}{4.552788in}}{\pgfqpoint{9.289178in}{4.563387in}}{\pgfqpoint{9.289178in}{4.574437in}}%
\pgfpathcurveto{\pgfqpoint{9.289178in}{4.585487in}}{\pgfqpoint{9.284788in}{4.596086in}}{\pgfqpoint{9.276974in}{4.603900in}}%
\pgfpathcurveto{\pgfqpoint{9.269160in}{4.611714in}}{\pgfqpoint{9.258561in}{4.616104in}}{\pgfqpoint{9.247511in}{4.616104in}}%
\pgfpathcurveto{\pgfqpoint{9.236461in}{4.616104in}}{\pgfqpoint{9.225862in}{4.611714in}}{\pgfqpoint{9.218048in}{4.603900in}}%
\pgfpathcurveto{\pgfqpoint{9.210235in}{4.596086in}}{\pgfqpoint{9.205844in}{4.585487in}}{\pgfqpoint{9.205844in}{4.574437in}}%
\pgfpathcurveto{\pgfqpoint{9.205844in}{4.563387in}}{\pgfqpoint{9.210235in}{4.552788in}}{\pgfqpoint{9.218048in}{4.544974in}}%
\pgfpathcurveto{\pgfqpoint{9.225862in}{4.537161in}}{\pgfqpoint{9.236461in}{4.532771in}}{\pgfqpoint{9.247511in}{4.532771in}}%
\pgfpathlineto{\pgfqpoint{9.247511in}{4.532771in}}%
\pgfpathclose%
\pgfusepath{stroke}%
\end{pgfscope}%
\begin{pgfscope}%
\pgfpathrectangle{\pgfqpoint{7.512535in}{0.437222in}}{\pgfqpoint{6.275590in}{5.159444in}}%
\pgfusepath{clip}%
\pgfsetbuttcap%
\pgfsetroundjoin%
\pgfsetlinewidth{1.003750pt}%
\definecolor{currentstroke}{rgb}{0.827451,0.827451,0.827451}%
\pgfsetstrokecolor{currentstroke}%
\pgfsetstrokeopacity{0.800000}%
\pgfsetdash{}{0pt}%
\pgfpathmoveto{\pgfqpoint{9.836077in}{3.291016in}}%
\pgfpathcurveto{\pgfqpoint{9.847127in}{3.291016in}}{\pgfqpoint{9.857726in}{3.295406in}}{\pgfqpoint{9.865540in}{3.303220in}}%
\pgfpathcurveto{\pgfqpoint{9.873353in}{3.311034in}}{\pgfqpoint{9.877743in}{3.321633in}}{\pgfqpoint{9.877743in}{3.332683in}}%
\pgfpathcurveto{\pgfqpoint{9.877743in}{3.343733in}}{\pgfqpoint{9.873353in}{3.354332in}}{\pgfqpoint{9.865540in}{3.362146in}}%
\pgfpathcurveto{\pgfqpoint{9.857726in}{3.369959in}}{\pgfqpoint{9.847127in}{3.374349in}}{\pgfqpoint{9.836077in}{3.374349in}}%
\pgfpathcurveto{\pgfqpoint{9.825027in}{3.374349in}}{\pgfqpoint{9.814428in}{3.369959in}}{\pgfqpoint{9.806614in}{3.362146in}}%
\pgfpathcurveto{\pgfqpoint{9.798800in}{3.354332in}}{\pgfqpoint{9.794410in}{3.343733in}}{\pgfqpoint{9.794410in}{3.332683in}}%
\pgfpathcurveto{\pgfqpoint{9.794410in}{3.321633in}}{\pgfqpoint{9.798800in}{3.311034in}}{\pgfqpoint{9.806614in}{3.303220in}}%
\pgfpathcurveto{\pgfqpoint{9.814428in}{3.295406in}}{\pgfqpoint{9.825027in}{3.291016in}}{\pgfqpoint{9.836077in}{3.291016in}}%
\pgfpathlineto{\pgfqpoint{9.836077in}{3.291016in}}%
\pgfpathclose%
\pgfusepath{stroke}%
\end{pgfscope}%
\begin{pgfscope}%
\pgfpathrectangle{\pgfqpoint{7.512535in}{0.437222in}}{\pgfqpoint{6.275590in}{5.159444in}}%
\pgfusepath{clip}%
\pgfsetbuttcap%
\pgfsetroundjoin%
\pgfsetlinewidth{1.003750pt}%
\definecolor{currentstroke}{rgb}{0.827451,0.827451,0.827451}%
\pgfsetstrokecolor{currentstroke}%
\pgfsetstrokeopacity{0.800000}%
\pgfsetdash{}{0pt}%
\pgfpathmoveto{\pgfqpoint{12.362478in}{4.949494in}}%
\pgfpathcurveto{\pgfqpoint{12.373528in}{4.949494in}}{\pgfqpoint{12.384127in}{4.953884in}}{\pgfqpoint{12.391941in}{4.961698in}}%
\pgfpathcurveto{\pgfqpoint{12.399754in}{4.969511in}}{\pgfqpoint{12.404145in}{4.980110in}}{\pgfqpoint{12.404145in}{4.991160in}}%
\pgfpathcurveto{\pgfqpoint{12.404145in}{5.002211in}}{\pgfqpoint{12.399754in}{5.012810in}}{\pgfqpoint{12.391941in}{5.020623in}}%
\pgfpathcurveto{\pgfqpoint{12.384127in}{5.028437in}}{\pgfqpoint{12.373528in}{5.032827in}}{\pgfqpoint{12.362478in}{5.032827in}}%
\pgfpathcurveto{\pgfqpoint{12.351428in}{5.032827in}}{\pgfqpoint{12.340829in}{5.028437in}}{\pgfqpoint{12.333015in}{5.020623in}}%
\pgfpathcurveto{\pgfqpoint{12.325202in}{5.012810in}}{\pgfqpoint{12.320811in}{5.002211in}}{\pgfqpoint{12.320811in}{4.991160in}}%
\pgfpathcurveto{\pgfqpoint{12.320811in}{4.980110in}}{\pgfqpoint{12.325202in}{4.969511in}}{\pgfqpoint{12.333015in}{4.961698in}}%
\pgfpathcurveto{\pgfqpoint{12.340829in}{4.953884in}}{\pgfqpoint{12.351428in}{4.949494in}}{\pgfqpoint{12.362478in}{4.949494in}}%
\pgfpathlineto{\pgfqpoint{12.362478in}{4.949494in}}%
\pgfpathclose%
\pgfusepath{stroke}%
\end{pgfscope}%
\begin{pgfscope}%
\pgfpathrectangle{\pgfqpoint{7.512535in}{0.437222in}}{\pgfqpoint{6.275590in}{5.159444in}}%
\pgfusepath{clip}%
\pgfsetbuttcap%
\pgfsetroundjoin%
\pgfsetlinewidth{1.003750pt}%
\definecolor{currentstroke}{rgb}{0.827451,0.827451,0.827451}%
\pgfsetstrokecolor{currentstroke}%
\pgfsetstrokeopacity{0.800000}%
\pgfsetdash{}{0pt}%
\pgfpathmoveto{\pgfqpoint{9.802410in}{3.570658in}}%
\pgfpathcurveto{\pgfqpoint{9.813460in}{3.570658in}}{\pgfqpoint{9.824059in}{3.575048in}}{\pgfqpoint{9.831872in}{3.582862in}}%
\pgfpathcurveto{\pgfqpoint{9.839686in}{3.590675in}}{\pgfqpoint{9.844076in}{3.601274in}}{\pgfqpoint{9.844076in}{3.612325in}}%
\pgfpathcurveto{\pgfqpoint{9.844076in}{3.623375in}}{\pgfqpoint{9.839686in}{3.633974in}}{\pgfqpoint{9.831872in}{3.641787in}}%
\pgfpathcurveto{\pgfqpoint{9.824059in}{3.649601in}}{\pgfqpoint{9.813460in}{3.653991in}}{\pgfqpoint{9.802410in}{3.653991in}}%
\pgfpathcurveto{\pgfqpoint{9.791360in}{3.653991in}}{\pgfqpoint{9.780760in}{3.649601in}}{\pgfqpoint{9.772947in}{3.641787in}}%
\pgfpathcurveto{\pgfqpoint{9.765133in}{3.633974in}}{\pgfqpoint{9.760743in}{3.623375in}}{\pgfqpoint{9.760743in}{3.612325in}}%
\pgfpathcurveto{\pgfqpoint{9.760743in}{3.601274in}}{\pgfqpoint{9.765133in}{3.590675in}}{\pgfqpoint{9.772947in}{3.582862in}}%
\pgfpathcurveto{\pgfqpoint{9.780760in}{3.575048in}}{\pgfqpoint{9.791360in}{3.570658in}}{\pgfqpoint{9.802410in}{3.570658in}}%
\pgfpathlineto{\pgfqpoint{9.802410in}{3.570658in}}%
\pgfpathclose%
\pgfusepath{stroke}%
\end{pgfscope}%
\begin{pgfscope}%
\pgfpathrectangle{\pgfqpoint{7.512535in}{0.437222in}}{\pgfqpoint{6.275590in}{5.159444in}}%
\pgfusepath{clip}%
\pgfsetbuttcap%
\pgfsetroundjoin%
\pgfsetlinewidth{1.003750pt}%
\definecolor{currentstroke}{rgb}{0.827451,0.827451,0.827451}%
\pgfsetstrokecolor{currentstroke}%
\pgfsetstrokeopacity{0.800000}%
\pgfsetdash{}{0pt}%
\pgfpathmoveto{\pgfqpoint{7.985597in}{2.298483in}}%
\pgfpathcurveto{\pgfqpoint{7.996647in}{2.298483in}}{\pgfqpoint{8.007246in}{2.302873in}}{\pgfqpoint{8.015059in}{2.310686in}}%
\pgfpathcurveto{\pgfqpoint{8.022873in}{2.318500in}}{\pgfqpoint{8.027263in}{2.329099in}}{\pgfqpoint{8.027263in}{2.340149in}}%
\pgfpathcurveto{\pgfqpoint{8.027263in}{2.351199in}}{\pgfqpoint{8.022873in}{2.361798in}}{\pgfqpoint{8.015059in}{2.369612in}}%
\pgfpathcurveto{\pgfqpoint{8.007246in}{2.377426in}}{\pgfqpoint{7.996647in}{2.381816in}}{\pgfqpoint{7.985597in}{2.381816in}}%
\pgfpathcurveto{\pgfqpoint{7.974546in}{2.381816in}}{\pgfqpoint{7.963947in}{2.377426in}}{\pgfqpoint{7.956134in}{2.369612in}}%
\pgfpathcurveto{\pgfqpoint{7.948320in}{2.361798in}}{\pgfqpoint{7.943930in}{2.351199in}}{\pgfqpoint{7.943930in}{2.340149in}}%
\pgfpathcurveto{\pgfqpoint{7.943930in}{2.329099in}}{\pgfqpoint{7.948320in}{2.318500in}}{\pgfqpoint{7.956134in}{2.310686in}}%
\pgfpathcurveto{\pgfqpoint{7.963947in}{2.302873in}}{\pgfqpoint{7.974546in}{2.298483in}}{\pgfqpoint{7.985597in}{2.298483in}}%
\pgfpathlineto{\pgfqpoint{7.985597in}{2.298483in}}%
\pgfpathclose%
\pgfusepath{stroke}%
\end{pgfscope}%
\begin{pgfscope}%
\pgfpathrectangle{\pgfqpoint{7.512535in}{0.437222in}}{\pgfqpoint{6.275590in}{5.159444in}}%
\pgfusepath{clip}%
\pgfsetbuttcap%
\pgfsetroundjoin%
\pgfsetlinewidth{1.003750pt}%
\definecolor{currentstroke}{rgb}{0.827451,0.827451,0.827451}%
\pgfsetstrokecolor{currentstroke}%
\pgfsetstrokeopacity{0.800000}%
\pgfsetdash{}{0pt}%
\pgfpathmoveto{\pgfqpoint{10.005215in}{4.464965in}}%
\pgfpathcurveto{\pgfqpoint{10.016265in}{4.464965in}}{\pgfqpoint{10.026865in}{4.469355in}}{\pgfqpoint{10.034678in}{4.477169in}}%
\pgfpathcurveto{\pgfqpoint{10.042492in}{4.484982in}}{\pgfqpoint{10.046882in}{4.495581in}}{\pgfqpoint{10.046882in}{4.506632in}}%
\pgfpathcurveto{\pgfqpoint{10.046882in}{4.517682in}}{\pgfqpoint{10.042492in}{4.528281in}}{\pgfqpoint{10.034678in}{4.536094in}}%
\pgfpathcurveto{\pgfqpoint{10.026865in}{4.543908in}}{\pgfqpoint{10.016265in}{4.548298in}}{\pgfqpoint{10.005215in}{4.548298in}}%
\pgfpathcurveto{\pgfqpoint{9.994165in}{4.548298in}}{\pgfqpoint{9.983566in}{4.543908in}}{\pgfqpoint{9.975753in}{4.536094in}}%
\pgfpathcurveto{\pgfqpoint{9.967939in}{4.528281in}}{\pgfqpoint{9.963549in}{4.517682in}}{\pgfqpoint{9.963549in}{4.506632in}}%
\pgfpathcurveto{\pgfqpoint{9.963549in}{4.495581in}}{\pgfqpoint{9.967939in}{4.484982in}}{\pgfqpoint{9.975753in}{4.477169in}}%
\pgfpathcurveto{\pgfqpoint{9.983566in}{4.469355in}}{\pgfqpoint{9.994165in}{4.464965in}}{\pgfqpoint{10.005215in}{4.464965in}}%
\pgfpathlineto{\pgfqpoint{10.005215in}{4.464965in}}%
\pgfpathclose%
\pgfusepath{stroke}%
\end{pgfscope}%
\begin{pgfscope}%
\pgfpathrectangle{\pgfqpoint{7.512535in}{0.437222in}}{\pgfqpoint{6.275590in}{5.159444in}}%
\pgfusepath{clip}%
\pgfsetbuttcap%
\pgfsetroundjoin%
\pgfsetlinewidth{1.003750pt}%
\definecolor{currentstroke}{rgb}{0.827451,0.827451,0.827451}%
\pgfsetstrokecolor{currentstroke}%
\pgfsetstrokeopacity{0.800000}%
\pgfsetdash{}{0pt}%
\pgfpathmoveto{\pgfqpoint{9.500868in}{3.326629in}}%
\pgfpathcurveto{\pgfqpoint{9.511919in}{3.326629in}}{\pgfqpoint{9.522518in}{3.331019in}}{\pgfqpoint{9.530331in}{3.338833in}}%
\pgfpathcurveto{\pgfqpoint{9.538145in}{3.346646in}}{\pgfqpoint{9.542535in}{3.357246in}}{\pgfqpoint{9.542535in}{3.368296in}}%
\pgfpathcurveto{\pgfqpoint{9.542535in}{3.379346in}}{\pgfqpoint{9.538145in}{3.389945in}}{\pgfqpoint{9.530331in}{3.397758in}}%
\pgfpathcurveto{\pgfqpoint{9.522518in}{3.405572in}}{\pgfqpoint{9.511919in}{3.409962in}}{\pgfqpoint{9.500868in}{3.409962in}}%
\pgfpathcurveto{\pgfqpoint{9.489818in}{3.409962in}}{\pgfqpoint{9.479219in}{3.405572in}}{\pgfqpoint{9.471406in}{3.397758in}}%
\pgfpathcurveto{\pgfqpoint{9.463592in}{3.389945in}}{\pgfqpoint{9.459202in}{3.379346in}}{\pgfqpoint{9.459202in}{3.368296in}}%
\pgfpathcurveto{\pgfqpoint{9.459202in}{3.357246in}}{\pgfqpoint{9.463592in}{3.346646in}}{\pgfqpoint{9.471406in}{3.338833in}}%
\pgfpathcurveto{\pgfqpoint{9.479219in}{3.331019in}}{\pgfqpoint{9.489818in}{3.326629in}}{\pgfqpoint{9.500868in}{3.326629in}}%
\pgfpathlineto{\pgfqpoint{9.500868in}{3.326629in}}%
\pgfpathclose%
\pgfusepath{stroke}%
\end{pgfscope}%
\begin{pgfscope}%
\pgfpathrectangle{\pgfqpoint{7.512535in}{0.437222in}}{\pgfqpoint{6.275590in}{5.159444in}}%
\pgfusepath{clip}%
\pgfsetbuttcap%
\pgfsetroundjoin%
\pgfsetlinewidth{1.003750pt}%
\definecolor{currentstroke}{rgb}{0.827451,0.827451,0.827451}%
\pgfsetstrokecolor{currentstroke}%
\pgfsetstrokeopacity{0.800000}%
\pgfsetdash{}{0pt}%
\pgfpathmoveto{\pgfqpoint{10.287229in}{5.087121in}}%
\pgfpathcurveto{\pgfqpoint{10.298279in}{5.087121in}}{\pgfqpoint{10.308878in}{5.091511in}}{\pgfqpoint{10.316692in}{5.099325in}}%
\pgfpathcurveto{\pgfqpoint{10.324505in}{5.107138in}}{\pgfqpoint{10.328896in}{5.117737in}}{\pgfqpoint{10.328896in}{5.128787in}}%
\pgfpathcurveto{\pgfqpoint{10.328896in}{5.139838in}}{\pgfqpoint{10.324505in}{5.150437in}}{\pgfqpoint{10.316692in}{5.158250in}}%
\pgfpathcurveto{\pgfqpoint{10.308878in}{5.166064in}}{\pgfqpoint{10.298279in}{5.170454in}}{\pgfqpoint{10.287229in}{5.170454in}}%
\pgfpathcurveto{\pgfqpoint{10.276179in}{5.170454in}}{\pgfqpoint{10.265580in}{5.166064in}}{\pgfqpoint{10.257766in}{5.158250in}}%
\pgfpathcurveto{\pgfqpoint{10.249953in}{5.150437in}}{\pgfqpoint{10.245562in}{5.139838in}}{\pgfqpoint{10.245562in}{5.128787in}}%
\pgfpathcurveto{\pgfqpoint{10.245562in}{5.117737in}}{\pgfqpoint{10.249953in}{5.107138in}}{\pgfqpoint{10.257766in}{5.099325in}}%
\pgfpathcurveto{\pgfqpoint{10.265580in}{5.091511in}}{\pgfqpoint{10.276179in}{5.087121in}}{\pgfqpoint{10.287229in}{5.087121in}}%
\pgfpathlineto{\pgfqpoint{10.287229in}{5.087121in}}%
\pgfpathclose%
\pgfusepath{stroke}%
\end{pgfscope}%
\begin{pgfscope}%
\pgfpathrectangle{\pgfqpoint{7.512535in}{0.437222in}}{\pgfqpoint{6.275590in}{5.159444in}}%
\pgfusepath{clip}%
\pgfsetbuttcap%
\pgfsetroundjoin%
\pgfsetlinewidth{1.003750pt}%
\definecolor{currentstroke}{rgb}{0.827451,0.827451,0.827451}%
\pgfsetstrokecolor{currentstroke}%
\pgfsetstrokeopacity{0.800000}%
\pgfsetdash{}{0pt}%
\pgfpathmoveto{\pgfqpoint{7.887873in}{1.346375in}}%
\pgfpathcurveto{\pgfqpoint{7.898923in}{1.346375in}}{\pgfqpoint{7.909522in}{1.350765in}}{\pgfqpoint{7.917336in}{1.358578in}}%
\pgfpathcurveto{\pgfqpoint{7.925149in}{1.366392in}}{\pgfqpoint{7.929540in}{1.376991in}}{\pgfqpoint{7.929540in}{1.388041in}}%
\pgfpathcurveto{\pgfqpoint{7.929540in}{1.399091in}}{\pgfqpoint{7.925149in}{1.409690in}}{\pgfqpoint{7.917336in}{1.417504in}}%
\pgfpathcurveto{\pgfqpoint{7.909522in}{1.425318in}}{\pgfqpoint{7.898923in}{1.429708in}}{\pgfqpoint{7.887873in}{1.429708in}}%
\pgfpathcurveto{\pgfqpoint{7.876823in}{1.429708in}}{\pgfqpoint{7.866224in}{1.425318in}}{\pgfqpoint{7.858410in}{1.417504in}}%
\pgfpathcurveto{\pgfqpoint{7.850596in}{1.409690in}}{\pgfqpoint{7.846206in}{1.399091in}}{\pgfqpoint{7.846206in}{1.388041in}}%
\pgfpathcurveto{\pgfqpoint{7.846206in}{1.376991in}}{\pgfqpoint{7.850596in}{1.366392in}}{\pgfqpoint{7.858410in}{1.358578in}}%
\pgfpathcurveto{\pgfqpoint{7.866224in}{1.350765in}}{\pgfqpoint{7.876823in}{1.346375in}}{\pgfqpoint{7.887873in}{1.346375in}}%
\pgfpathlineto{\pgfqpoint{7.887873in}{1.346375in}}%
\pgfpathclose%
\pgfusepath{stroke}%
\end{pgfscope}%
\begin{pgfscope}%
\pgfpathrectangle{\pgfqpoint{7.512535in}{0.437222in}}{\pgfqpoint{6.275590in}{5.159444in}}%
\pgfusepath{clip}%
\pgfsetbuttcap%
\pgfsetroundjoin%
\pgfsetlinewidth{1.003750pt}%
\definecolor{currentstroke}{rgb}{0.827451,0.827451,0.827451}%
\pgfsetstrokecolor{currentstroke}%
\pgfsetstrokeopacity{0.800000}%
\pgfsetdash{}{0pt}%
\pgfpathmoveto{\pgfqpoint{9.581399in}{2.990585in}}%
\pgfpathcurveto{\pgfqpoint{9.592449in}{2.990585in}}{\pgfqpoint{9.603048in}{2.994975in}}{\pgfqpoint{9.610862in}{3.002789in}}%
\pgfpathcurveto{\pgfqpoint{9.618675in}{3.010602in}}{\pgfqpoint{9.623065in}{3.021202in}}{\pgfqpoint{9.623065in}{3.032252in}}%
\pgfpathcurveto{\pgfqpoint{9.623065in}{3.043302in}}{\pgfqpoint{9.618675in}{3.053901in}}{\pgfqpoint{9.610862in}{3.061714in}}%
\pgfpathcurveto{\pgfqpoint{9.603048in}{3.069528in}}{\pgfqpoint{9.592449in}{3.073918in}}{\pgfqpoint{9.581399in}{3.073918in}}%
\pgfpathcurveto{\pgfqpoint{9.570349in}{3.073918in}}{\pgfqpoint{9.559750in}{3.069528in}}{\pgfqpoint{9.551936in}{3.061714in}}%
\pgfpathcurveto{\pgfqpoint{9.544122in}{3.053901in}}{\pgfqpoint{9.539732in}{3.043302in}}{\pgfqpoint{9.539732in}{3.032252in}}%
\pgfpathcurveto{\pgfqpoint{9.539732in}{3.021202in}}{\pgfqpoint{9.544122in}{3.010602in}}{\pgfqpoint{9.551936in}{3.002789in}}%
\pgfpathcurveto{\pgfqpoint{9.559750in}{2.994975in}}{\pgfqpoint{9.570349in}{2.990585in}}{\pgfqpoint{9.581399in}{2.990585in}}%
\pgfpathlineto{\pgfqpoint{9.581399in}{2.990585in}}%
\pgfpathclose%
\pgfusepath{stroke}%
\end{pgfscope}%
\begin{pgfscope}%
\pgfpathrectangle{\pgfqpoint{7.512535in}{0.437222in}}{\pgfqpoint{6.275590in}{5.159444in}}%
\pgfusepath{clip}%
\pgfsetbuttcap%
\pgfsetroundjoin%
\pgfsetlinewidth{1.003750pt}%
\definecolor{currentstroke}{rgb}{0.827451,0.827451,0.827451}%
\pgfsetstrokecolor{currentstroke}%
\pgfsetstrokeopacity{0.800000}%
\pgfsetdash{}{0pt}%
\pgfpathmoveto{\pgfqpoint{11.387346in}{5.101603in}}%
\pgfpathcurveto{\pgfqpoint{11.398396in}{5.101603in}}{\pgfqpoint{11.408995in}{5.105993in}}{\pgfqpoint{11.416809in}{5.113807in}}%
\pgfpathcurveto{\pgfqpoint{11.424622in}{5.121620in}}{\pgfqpoint{11.429013in}{5.132219in}}{\pgfqpoint{11.429013in}{5.143270in}}%
\pgfpathcurveto{\pgfqpoint{11.429013in}{5.154320in}}{\pgfqpoint{11.424622in}{5.164919in}}{\pgfqpoint{11.416809in}{5.172732in}}%
\pgfpathcurveto{\pgfqpoint{11.408995in}{5.180546in}}{\pgfqpoint{11.398396in}{5.184936in}}{\pgfqpoint{11.387346in}{5.184936in}}%
\pgfpathcurveto{\pgfqpoint{11.376296in}{5.184936in}}{\pgfqpoint{11.365697in}{5.180546in}}{\pgfqpoint{11.357883in}{5.172732in}}%
\pgfpathcurveto{\pgfqpoint{11.350070in}{5.164919in}}{\pgfqpoint{11.345679in}{5.154320in}}{\pgfqpoint{11.345679in}{5.143270in}}%
\pgfpathcurveto{\pgfqpoint{11.345679in}{5.132219in}}{\pgfqpoint{11.350070in}{5.121620in}}{\pgfqpoint{11.357883in}{5.113807in}}%
\pgfpathcurveto{\pgfqpoint{11.365697in}{5.105993in}}{\pgfqpoint{11.376296in}{5.101603in}}{\pgfqpoint{11.387346in}{5.101603in}}%
\pgfpathlineto{\pgfqpoint{11.387346in}{5.101603in}}%
\pgfpathclose%
\pgfusepath{stroke}%
\end{pgfscope}%
\begin{pgfscope}%
\pgfpathrectangle{\pgfqpoint{7.512535in}{0.437222in}}{\pgfqpoint{6.275590in}{5.159444in}}%
\pgfusepath{clip}%
\pgfsetbuttcap%
\pgfsetroundjoin%
\pgfsetlinewidth{1.003750pt}%
\definecolor{currentstroke}{rgb}{0.827451,0.827451,0.827451}%
\pgfsetstrokecolor{currentstroke}%
\pgfsetstrokeopacity{0.800000}%
\pgfsetdash{}{0pt}%
\pgfpathmoveto{\pgfqpoint{12.473540in}{5.514145in}}%
\pgfpathcurveto{\pgfqpoint{12.484590in}{5.514145in}}{\pgfqpoint{12.495189in}{5.518535in}}{\pgfqpoint{12.503003in}{5.526349in}}%
\pgfpathcurveto{\pgfqpoint{12.510817in}{5.534162in}}{\pgfqpoint{12.515207in}{5.544761in}}{\pgfqpoint{12.515207in}{5.555811in}}%
\pgfpathcurveto{\pgfqpoint{12.515207in}{5.566862in}}{\pgfqpoint{12.510817in}{5.577461in}}{\pgfqpoint{12.503003in}{5.585274in}}%
\pgfpathcurveto{\pgfqpoint{12.495189in}{5.593088in}}{\pgfqpoint{12.484590in}{5.597478in}}{\pgfqpoint{12.473540in}{5.597478in}}%
\pgfpathcurveto{\pgfqpoint{12.462490in}{5.597478in}}{\pgfqpoint{12.451891in}{5.593088in}}{\pgfqpoint{12.444078in}{5.585274in}}%
\pgfpathcurveto{\pgfqpoint{12.436264in}{5.577461in}}{\pgfqpoint{12.431874in}{5.566862in}}{\pgfqpoint{12.431874in}{5.555811in}}%
\pgfpathcurveto{\pgfqpoint{12.431874in}{5.544761in}}{\pgfqpoint{12.436264in}{5.534162in}}{\pgfqpoint{12.444078in}{5.526349in}}%
\pgfpathcurveto{\pgfqpoint{12.451891in}{5.518535in}}{\pgfqpoint{12.462490in}{5.514145in}}{\pgfqpoint{12.473540in}{5.514145in}}%
\pgfpathlineto{\pgfqpoint{12.473540in}{5.514145in}}%
\pgfpathclose%
\pgfusepath{stroke}%
\end{pgfscope}%
\begin{pgfscope}%
\pgfpathrectangle{\pgfqpoint{7.512535in}{0.437222in}}{\pgfqpoint{6.275590in}{5.159444in}}%
\pgfusepath{clip}%
\pgfsetbuttcap%
\pgfsetroundjoin%
\pgfsetlinewidth{1.003750pt}%
\definecolor{currentstroke}{rgb}{0.827451,0.827451,0.827451}%
\pgfsetstrokecolor{currentstroke}%
\pgfsetstrokeopacity{0.800000}%
\pgfsetdash{}{0pt}%
\pgfpathmoveto{\pgfqpoint{10.346895in}{5.462154in}}%
\pgfpathcurveto{\pgfqpoint{10.357945in}{5.462154in}}{\pgfqpoint{10.368544in}{5.466545in}}{\pgfqpoint{10.376358in}{5.474358in}}%
\pgfpathcurveto{\pgfqpoint{10.384171in}{5.482172in}}{\pgfqpoint{10.388562in}{5.492771in}}{\pgfqpoint{10.388562in}{5.503821in}}%
\pgfpathcurveto{\pgfqpoint{10.388562in}{5.514871in}}{\pgfqpoint{10.384171in}{5.525470in}}{\pgfqpoint{10.376358in}{5.533284in}}%
\pgfpathcurveto{\pgfqpoint{10.368544in}{5.541098in}}{\pgfqpoint{10.357945in}{5.545488in}}{\pgfqpoint{10.346895in}{5.545488in}}%
\pgfpathcurveto{\pgfqpoint{10.335845in}{5.545488in}}{\pgfqpoint{10.325246in}{5.541098in}}{\pgfqpoint{10.317432in}{5.533284in}}%
\pgfpathcurveto{\pgfqpoint{10.309619in}{5.525470in}}{\pgfqpoint{10.305228in}{5.514871in}}{\pgfqpoint{10.305228in}{5.503821in}}%
\pgfpathcurveto{\pgfqpoint{10.305228in}{5.492771in}}{\pgfqpoint{10.309619in}{5.482172in}}{\pgfqpoint{10.317432in}{5.474358in}}%
\pgfpathcurveto{\pgfqpoint{10.325246in}{5.466545in}}{\pgfqpoint{10.335845in}{5.462154in}}{\pgfqpoint{10.346895in}{5.462154in}}%
\pgfpathlineto{\pgfqpoint{10.346895in}{5.462154in}}%
\pgfpathclose%
\pgfusepath{stroke}%
\end{pgfscope}%
\begin{pgfscope}%
\pgfpathrectangle{\pgfqpoint{7.512535in}{0.437222in}}{\pgfqpoint{6.275590in}{5.159444in}}%
\pgfusepath{clip}%
\pgfsetbuttcap%
\pgfsetroundjoin%
\pgfsetlinewidth{1.003750pt}%
\definecolor{currentstroke}{rgb}{0.827451,0.827451,0.827451}%
\pgfsetstrokecolor{currentstroke}%
\pgfsetstrokeopacity{0.800000}%
\pgfsetdash{}{0pt}%
\pgfpathmoveto{\pgfqpoint{8.290348in}{1.011495in}}%
\pgfpathcurveto{\pgfqpoint{8.301398in}{1.011495in}}{\pgfqpoint{8.311997in}{1.015885in}}{\pgfqpoint{8.319811in}{1.023699in}}%
\pgfpathcurveto{\pgfqpoint{8.327625in}{1.031512in}}{\pgfqpoint{8.332015in}{1.042111in}}{\pgfqpoint{8.332015in}{1.053161in}}%
\pgfpathcurveto{\pgfqpoint{8.332015in}{1.064211in}}{\pgfqpoint{8.327625in}{1.074810in}}{\pgfqpoint{8.319811in}{1.082624in}}%
\pgfpathcurveto{\pgfqpoint{8.311997in}{1.090438in}}{\pgfqpoint{8.301398in}{1.094828in}}{\pgfqpoint{8.290348in}{1.094828in}}%
\pgfpathcurveto{\pgfqpoint{8.279298in}{1.094828in}}{\pgfqpoint{8.268699in}{1.090438in}}{\pgfqpoint{8.260886in}{1.082624in}}%
\pgfpathcurveto{\pgfqpoint{8.253072in}{1.074810in}}{\pgfqpoint{8.248682in}{1.064211in}}{\pgfqpoint{8.248682in}{1.053161in}}%
\pgfpathcurveto{\pgfqpoint{8.248682in}{1.042111in}}{\pgfqpoint{8.253072in}{1.031512in}}{\pgfqpoint{8.260886in}{1.023699in}}%
\pgfpathcurveto{\pgfqpoint{8.268699in}{1.015885in}}{\pgfqpoint{8.279298in}{1.011495in}}{\pgfqpoint{8.290348in}{1.011495in}}%
\pgfpathlineto{\pgfqpoint{8.290348in}{1.011495in}}%
\pgfpathclose%
\pgfusepath{stroke}%
\end{pgfscope}%
\begin{pgfscope}%
\pgfpathrectangle{\pgfqpoint{7.512535in}{0.437222in}}{\pgfqpoint{6.275590in}{5.159444in}}%
\pgfusepath{clip}%
\pgfsetbuttcap%
\pgfsetroundjoin%
\pgfsetlinewidth{1.003750pt}%
\definecolor{currentstroke}{rgb}{0.827451,0.827451,0.827451}%
\pgfsetstrokecolor{currentstroke}%
\pgfsetstrokeopacity{0.800000}%
\pgfsetdash{}{0pt}%
\pgfpathmoveto{\pgfqpoint{10.127909in}{3.160252in}}%
\pgfpathcurveto{\pgfqpoint{10.138959in}{3.160252in}}{\pgfqpoint{10.149558in}{3.164643in}}{\pgfqpoint{10.157372in}{3.172456in}}%
\pgfpathcurveto{\pgfqpoint{10.165186in}{3.180270in}}{\pgfqpoint{10.169576in}{3.190869in}}{\pgfqpoint{10.169576in}{3.201919in}}%
\pgfpathcurveto{\pgfqpoint{10.169576in}{3.212969in}}{\pgfqpoint{10.165186in}{3.223568in}}{\pgfqpoint{10.157372in}{3.231382in}}%
\pgfpathcurveto{\pgfqpoint{10.149558in}{3.239196in}}{\pgfqpoint{10.138959in}{3.243586in}}{\pgfqpoint{10.127909in}{3.243586in}}%
\pgfpathcurveto{\pgfqpoint{10.116859in}{3.243586in}}{\pgfqpoint{10.106260in}{3.239196in}}{\pgfqpoint{10.098446in}{3.231382in}}%
\pgfpathcurveto{\pgfqpoint{10.090633in}{3.223568in}}{\pgfqpoint{10.086243in}{3.212969in}}{\pgfqpoint{10.086243in}{3.201919in}}%
\pgfpathcurveto{\pgfqpoint{10.086243in}{3.190869in}}{\pgfqpoint{10.090633in}{3.180270in}}{\pgfqpoint{10.098446in}{3.172456in}}%
\pgfpathcurveto{\pgfqpoint{10.106260in}{3.164643in}}{\pgfqpoint{10.116859in}{3.160252in}}{\pgfqpoint{10.127909in}{3.160252in}}%
\pgfpathlineto{\pgfqpoint{10.127909in}{3.160252in}}%
\pgfpathclose%
\pgfusepath{stroke}%
\end{pgfscope}%
\begin{pgfscope}%
\pgfpathrectangle{\pgfqpoint{7.512535in}{0.437222in}}{\pgfqpoint{6.275590in}{5.159444in}}%
\pgfusepath{clip}%
\pgfsetbuttcap%
\pgfsetroundjoin%
\pgfsetlinewidth{1.003750pt}%
\definecolor{currentstroke}{rgb}{0.827451,0.827451,0.827451}%
\pgfsetstrokecolor{currentstroke}%
\pgfsetstrokeopacity{0.800000}%
\pgfsetdash{}{0pt}%
\pgfpathmoveto{\pgfqpoint{10.254297in}{3.553904in}}%
\pgfpathcurveto{\pgfqpoint{10.265347in}{3.553904in}}{\pgfqpoint{10.275946in}{3.558295in}}{\pgfqpoint{10.283760in}{3.566108in}}%
\pgfpathcurveto{\pgfqpoint{10.291574in}{3.573922in}}{\pgfqpoint{10.295964in}{3.584521in}}{\pgfqpoint{10.295964in}{3.595571in}}%
\pgfpathcurveto{\pgfqpoint{10.295964in}{3.606621in}}{\pgfqpoint{10.291574in}{3.617220in}}{\pgfqpoint{10.283760in}{3.625034in}}%
\pgfpathcurveto{\pgfqpoint{10.275946in}{3.632847in}}{\pgfqpoint{10.265347in}{3.637238in}}{\pgfqpoint{10.254297in}{3.637238in}}%
\pgfpathcurveto{\pgfqpoint{10.243247in}{3.637238in}}{\pgfqpoint{10.232648in}{3.632847in}}{\pgfqpoint{10.224834in}{3.625034in}}%
\pgfpathcurveto{\pgfqpoint{10.217021in}{3.617220in}}{\pgfqpoint{10.212631in}{3.606621in}}{\pgfqpoint{10.212631in}{3.595571in}}%
\pgfpathcurveto{\pgfqpoint{10.212631in}{3.584521in}}{\pgfqpoint{10.217021in}{3.573922in}}{\pgfqpoint{10.224834in}{3.566108in}}%
\pgfpathcurveto{\pgfqpoint{10.232648in}{3.558295in}}{\pgfqpoint{10.243247in}{3.553904in}}{\pgfqpoint{10.254297in}{3.553904in}}%
\pgfpathlineto{\pgfqpoint{10.254297in}{3.553904in}}%
\pgfpathclose%
\pgfusepath{stroke}%
\end{pgfscope}%
\begin{pgfscope}%
\pgfpathrectangle{\pgfqpoint{7.512535in}{0.437222in}}{\pgfqpoint{6.275590in}{5.159444in}}%
\pgfusepath{clip}%
\pgfsetbuttcap%
\pgfsetroundjoin%
\pgfsetlinewidth{1.003750pt}%
\definecolor{currentstroke}{rgb}{0.827451,0.827451,0.827451}%
\pgfsetstrokecolor{currentstroke}%
\pgfsetstrokeopacity{0.800000}%
\pgfsetdash{}{0pt}%
\pgfpathmoveto{\pgfqpoint{12.194523in}{4.683090in}}%
\pgfpathcurveto{\pgfqpoint{12.205573in}{4.683090in}}{\pgfqpoint{12.216172in}{4.687480in}}{\pgfqpoint{12.223986in}{4.695294in}}%
\pgfpathcurveto{\pgfqpoint{12.231799in}{4.703107in}}{\pgfqpoint{12.236190in}{4.713707in}}{\pgfqpoint{12.236190in}{4.724757in}}%
\pgfpathcurveto{\pgfqpoint{12.236190in}{4.735807in}}{\pgfqpoint{12.231799in}{4.746406in}}{\pgfqpoint{12.223986in}{4.754219in}}%
\pgfpathcurveto{\pgfqpoint{12.216172in}{4.762033in}}{\pgfqpoint{12.205573in}{4.766423in}}{\pgfqpoint{12.194523in}{4.766423in}}%
\pgfpathcurveto{\pgfqpoint{12.183473in}{4.766423in}}{\pgfqpoint{12.172874in}{4.762033in}}{\pgfqpoint{12.165060in}{4.754219in}}%
\pgfpathcurveto{\pgfqpoint{12.157247in}{4.746406in}}{\pgfqpoint{12.152856in}{4.735807in}}{\pgfqpoint{12.152856in}{4.724757in}}%
\pgfpathcurveto{\pgfqpoint{12.152856in}{4.713707in}}{\pgfqpoint{12.157247in}{4.703107in}}{\pgfqpoint{12.165060in}{4.695294in}}%
\pgfpathcurveto{\pgfqpoint{12.172874in}{4.687480in}}{\pgfqpoint{12.183473in}{4.683090in}}{\pgfqpoint{12.194523in}{4.683090in}}%
\pgfpathlineto{\pgfqpoint{12.194523in}{4.683090in}}%
\pgfpathclose%
\pgfusepath{stroke}%
\end{pgfscope}%
\begin{pgfscope}%
\pgfpathrectangle{\pgfqpoint{7.512535in}{0.437222in}}{\pgfqpoint{6.275590in}{5.159444in}}%
\pgfusepath{clip}%
\pgfsetbuttcap%
\pgfsetroundjoin%
\pgfsetlinewidth{1.003750pt}%
\definecolor{currentstroke}{rgb}{0.827451,0.827451,0.827451}%
\pgfsetstrokecolor{currentstroke}%
\pgfsetstrokeopacity{0.800000}%
\pgfsetdash{}{0pt}%
\pgfpathmoveto{\pgfqpoint{8.470019in}{3.328233in}}%
\pgfpathcurveto{\pgfqpoint{8.481069in}{3.328233in}}{\pgfqpoint{8.491668in}{3.332624in}}{\pgfqpoint{8.499482in}{3.340437in}}%
\pgfpathcurveto{\pgfqpoint{8.507295in}{3.348251in}}{\pgfqpoint{8.511686in}{3.358850in}}{\pgfqpoint{8.511686in}{3.369900in}}%
\pgfpathcurveto{\pgfqpoint{8.511686in}{3.380950in}}{\pgfqpoint{8.507295in}{3.391549in}}{\pgfqpoint{8.499482in}{3.399363in}}%
\pgfpathcurveto{\pgfqpoint{8.491668in}{3.407176in}}{\pgfqpoint{8.481069in}{3.411567in}}{\pgfqpoint{8.470019in}{3.411567in}}%
\pgfpathcurveto{\pgfqpoint{8.458969in}{3.411567in}}{\pgfqpoint{8.448370in}{3.407176in}}{\pgfqpoint{8.440556in}{3.399363in}}%
\pgfpathcurveto{\pgfqpoint{8.432743in}{3.391549in}}{\pgfqpoint{8.428352in}{3.380950in}}{\pgfqpoint{8.428352in}{3.369900in}}%
\pgfpathcurveto{\pgfqpoint{8.428352in}{3.358850in}}{\pgfqpoint{8.432743in}{3.348251in}}{\pgfqpoint{8.440556in}{3.340437in}}%
\pgfpathcurveto{\pgfqpoint{8.448370in}{3.332624in}}{\pgfqpoint{8.458969in}{3.328233in}}{\pgfqpoint{8.470019in}{3.328233in}}%
\pgfpathlineto{\pgfqpoint{8.470019in}{3.328233in}}%
\pgfpathclose%
\pgfusepath{stroke}%
\end{pgfscope}%
\begin{pgfscope}%
\pgfpathrectangle{\pgfqpoint{7.512535in}{0.437222in}}{\pgfqpoint{6.275590in}{5.159444in}}%
\pgfusepath{clip}%
\pgfsetbuttcap%
\pgfsetroundjoin%
\pgfsetlinewidth{1.003750pt}%
\definecolor{currentstroke}{rgb}{0.827451,0.827451,0.827451}%
\pgfsetstrokecolor{currentstroke}%
\pgfsetstrokeopacity{0.800000}%
\pgfsetdash{}{0pt}%
\pgfpathmoveto{\pgfqpoint{10.827454in}{5.304399in}}%
\pgfpathcurveto{\pgfqpoint{10.838504in}{5.304399in}}{\pgfqpoint{10.849103in}{5.308790in}}{\pgfqpoint{10.856917in}{5.316603in}}%
\pgfpathcurveto{\pgfqpoint{10.864730in}{5.324417in}}{\pgfqpoint{10.869121in}{5.335016in}}{\pgfqpoint{10.869121in}{5.346066in}}%
\pgfpathcurveto{\pgfqpoint{10.869121in}{5.357116in}}{\pgfqpoint{10.864730in}{5.367715in}}{\pgfqpoint{10.856917in}{5.375529in}}%
\pgfpathcurveto{\pgfqpoint{10.849103in}{5.383342in}}{\pgfqpoint{10.838504in}{5.387733in}}{\pgfqpoint{10.827454in}{5.387733in}}%
\pgfpathcurveto{\pgfqpoint{10.816404in}{5.387733in}}{\pgfqpoint{10.805805in}{5.383342in}}{\pgfqpoint{10.797991in}{5.375529in}}%
\pgfpathcurveto{\pgfqpoint{10.790177in}{5.367715in}}{\pgfqpoint{10.785787in}{5.357116in}}{\pgfqpoint{10.785787in}{5.346066in}}%
\pgfpathcurveto{\pgfqpoint{10.785787in}{5.335016in}}{\pgfqpoint{10.790177in}{5.324417in}}{\pgfqpoint{10.797991in}{5.316603in}}%
\pgfpathcurveto{\pgfqpoint{10.805805in}{5.308790in}}{\pgfqpoint{10.816404in}{5.304399in}}{\pgfqpoint{10.827454in}{5.304399in}}%
\pgfpathlineto{\pgfqpoint{10.827454in}{5.304399in}}%
\pgfpathclose%
\pgfusepath{stroke}%
\end{pgfscope}%
\begin{pgfscope}%
\pgfpathrectangle{\pgfqpoint{7.512535in}{0.437222in}}{\pgfqpoint{6.275590in}{5.159444in}}%
\pgfusepath{clip}%
\pgfsetbuttcap%
\pgfsetroundjoin%
\pgfsetlinewidth{1.003750pt}%
\definecolor{currentstroke}{rgb}{0.827451,0.827451,0.827451}%
\pgfsetstrokecolor{currentstroke}%
\pgfsetstrokeopacity{0.800000}%
\pgfsetdash{}{0pt}%
\pgfpathmoveto{\pgfqpoint{10.527072in}{2.988484in}}%
\pgfpathcurveto{\pgfqpoint{10.538122in}{2.988484in}}{\pgfqpoint{10.548721in}{2.992874in}}{\pgfqpoint{10.556535in}{3.000688in}}%
\pgfpathcurveto{\pgfqpoint{10.564349in}{3.008501in}}{\pgfqpoint{10.568739in}{3.019100in}}{\pgfqpoint{10.568739in}{3.030150in}}%
\pgfpathcurveto{\pgfqpoint{10.568739in}{3.041200in}}{\pgfqpoint{10.564349in}{3.051800in}}{\pgfqpoint{10.556535in}{3.059613in}}%
\pgfpathcurveto{\pgfqpoint{10.548721in}{3.067427in}}{\pgfqpoint{10.538122in}{3.071817in}}{\pgfqpoint{10.527072in}{3.071817in}}%
\pgfpathcurveto{\pgfqpoint{10.516022in}{3.071817in}}{\pgfqpoint{10.505423in}{3.067427in}}{\pgfqpoint{10.497609in}{3.059613in}}%
\pgfpathcurveto{\pgfqpoint{10.489796in}{3.051800in}}{\pgfqpoint{10.485405in}{3.041200in}}{\pgfqpoint{10.485405in}{3.030150in}}%
\pgfpathcurveto{\pgfqpoint{10.485405in}{3.019100in}}{\pgfqpoint{10.489796in}{3.008501in}}{\pgfqpoint{10.497609in}{3.000688in}}%
\pgfpathcurveto{\pgfqpoint{10.505423in}{2.992874in}}{\pgfqpoint{10.516022in}{2.988484in}}{\pgfqpoint{10.527072in}{2.988484in}}%
\pgfpathlineto{\pgfqpoint{10.527072in}{2.988484in}}%
\pgfpathclose%
\pgfusepath{stroke}%
\end{pgfscope}%
\begin{pgfscope}%
\pgfpathrectangle{\pgfqpoint{7.512535in}{0.437222in}}{\pgfqpoint{6.275590in}{5.159444in}}%
\pgfusepath{clip}%
\pgfsetbuttcap%
\pgfsetroundjoin%
\pgfsetlinewidth{1.003750pt}%
\definecolor{currentstroke}{rgb}{0.827451,0.827451,0.827451}%
\pgfsetstrokecolor{currentstroke}%
\pgfsetstrokeopacity{0.800000}%
\pgfsetdash{}{0pt}%
\pgfpathmoveto{\pgfqpoint{10.881762in}{4.622183in}}%
\pgfpathcurveto{\pgfqpoint{10.892812in}{4.622183in}}{\pgfqpoint{10.903411in}{4.626573in}}{\pgfqpoint{10.911225in}{4.634387in}}%
\pgfpathcurveto{\pgfqpoint{10.919039in}{4.642201in}}{\pgfqpoint{10.923429in}{4.652800in}}{\pgfqpoint{10.923429in}{4.663850in}}%
\pgfpathcurveto{\pgfqpoint{10.923429in}{4.674900in}}{\pgfqpoint{10.919039in}{4.685499in}}{\pgfqpoint{10.911225in}{4.693313in}}%
\pgfpathcurveto{\pgfqpoint{10.903411in}{4.701126in}}{\pgfqpoint{10.892812in}{4.705516in}}{\pgfqpoint{10.881762in}{4.705516in}}%
\pgfpathcurveto{\pgfqpoint{10.870712in}{4.705516in}}{\pgfqpoint{10.860113in}{4.701126in}}{\pgfqpoint{10.852300in}{4.693313in}}%
\pgfpathcurveto{\pgfqpoint{10.844486in}{4.685499in}}{\pgfqpoint{10.840096in}{4.674900in}}{\pgfqpoint{10.840096in}{4.663850in}}%
\pgfpathcurveto{\pgfqpoint{10.840096in}{4.652800in}}{\pgfqpoint{10.844486in}{4.642201in}}{\pgfqpoint{10.852300in}{4.634387in}}%
\pgfpathcurveto{\pgfqpoint{10.860113in}{4.626573in}}{\pgfqpoint{10.870712in}{4.622183in}}{\pgfqpoint{10.881762in}{4.622183in}}%
\pgfpathlineto{\pgfqpoint{10.881762in}{4.622183in}}%
\pgfpathclose%
\pgfusepath{stroke}%
\end{pgfscope}%
\begin{pgfscope}%
\pgfpathrectangle{\pgfqpoint{7.512535in}{0.437222in}}{\pgfqpoint{6.275590in}{5.159444in}}%
\pgfusepath{clip}%
\pgfsetbuttcap%
\pgfsetroundjoin%
\pgfsetlinewidth{1.003750pt}%
\definecolor{currentstroke}{rgb}{0.827451,0.827451,0.827451}%
\pgfsetstrokecolor{currentstroke}%
\pgfsetstrokeopacity{0.800000}%
\pgfsetdash{}{0pt}%
\pgfpathmoveto{\pgfqpoint{9.556541in}{1.394520in}}%
\pgfpathcurveto{\pgfqpoint{9.567591in}{1.394520in}}{\pgfqpoint{9.578190in}{1.398911in}}{\pgfqpoint{9.586004in}{1.406724in}}%
\pgfpathcurveto{\pgfqpoint{9.593817in}{1.414538in}}{\pgfqpoint{9.598208in}{1.425137in}}{\pgfqpoint{9.598208in}{1.436187in}}%
\pgfpathcurveto{\pgfqpoint{9.598208in}{1.447237in}}{\pgfqpoint{9.593817in}{1.457836in}}{\pgfqpoint{9.586004in}{1.465650in}}%
\pgfpathcurveto{\pgfqpoint{9.578190in}{1.473464in}}{\pgfqpoint{9.567591in}{1.477854in}}{\pgfqpoint{9.556541in}{1.477854in}}%
\pgfpathcurveto{\pgfqpoint{9.545491in}{1.477854in}}{\pgfqpoint{9.534892in}{1.473464in}}{\pgfqpoint{9.527078in}{1.465650in}}%
\pgfpathcurveto{\pgfqpoint{9.519265in}{1.457836in}}{\pgfqpoint{9.514874in}{1.447237in}}{\pgfqpoint{9.514874in}{1.436187in}}%
\pgfpathcurveto{\pgfqpoint{9.514874in}{1.425137in}}{\pgfqpoint{9.519265in}{1.414538in}}{\pgfqpoint{9.527078in}{1.406724in}}%
\pgfpathcurveto{\pgfqpoint{9.534892in}{1.398911in}}{\pgfqpoint{9.545491in}{1.394520in}}{\pgfqpoint{9.556541in}{1.394520in}}%
\pgfpathlineto{\pgfqpoint{9.556541in}{1.394520in}}%
\pgfpathclose%
\pgfusepath{stroke}%
\end{pgfscope}%
\begin{pgfscope}%
\pgfpathrectangle{\pgfqpoint{7.512535in}{0.437222in}}{\pgfqpoint{6.275590in}{5.159444in}}%
\pgfusepath{clip}%
\pgfsetbuttcap%
\pgfsetroundjoin%
\pgfsetlinewidth{1.003750pt}%
\definecolor{currentstroke}{rgb}{0.827451,0.827451,0.827451}%
\pgfsetstrokecolor{currentstroke}%
\pgfsetstrokeopacity{0.800000}%
\pgfsetdash{}{0pt}%
\pgfpathmoveto{\pgfqpoint{10.385573in}{5.144321in}}%
\pgfpathcurveto{\pgfqpoint{10.396623in}{5.144321in}}{\pgfqpoint{10.407222in}{5.148712in}}{\pgfqpoint{10.415036in}{5.156525in}}%
\pgfpathcurveto{\pgfqpoint{10.422849in}{5.164339in}}{\pgfqpoint{10.427240in}{5.174938in}}{\pgfqpoint{10.427240in}{5.185988in}}%
\pgfpathcurveto{\pgfqpoint{10.427240in}{5.197038in}}{\pgfqpoint{10.422849in}{5.207637in}}{\pgfqpoint{10.415036in}{5.215451in}}%
\pgfpathcurveto{\pgfqpoint{10.407222in}{5.223265in}}{\pgfqpoint{10.396623in}{5.227655in}}{\pgfqpoint{10.385573in}{5.227655in}}%
\pgfpathcurveto{\pgfqpoint{10.374523in}{5.227655in}}{\pgfqpoint{10.363924in}{5.223265in}}{\pgfqpoint{10.356110in}{5.215451in}}%
\pgfpathcurveto{\pgfqpoint{10.348297in}{5.207637in}}{\pgfqpoint{10.343906in}{5.197038in}}{\pgfqpoint{10.343906in}{5.185988in}}%
\pgfpathcurveto{\pgfqpoint{10.343906in}{5.174938in}}{\pgfqpoint{10.348297in}{5.164339in}}{\pgfqpoint{10.356110in}{5.156525in}}%
\pgfpathcurveto{\pgfqpoint{10.363924in}{5.148712in}}{\pgfqpoint{10.374523in}{5.144321in}}{\pgfqpoint{10.385573in}{5.144321in}}%
\pgfpathlineto{\pgfqpoint{10.385573in}{5.144321in}}%
\pgfpathclose%
\pgfusepath{stroke}%
\end{pgfscope}%
\begin{pgfscope}%
\pgfpathrectangle{\pgfqpoint{7.512535in}{0.437222in}}{\pgfqpoint{6.275590in}{5.159444in}}%
\pgfusepath{clip}%
\pgfsetbuttcap%
\pgfsetroundjoin%
\pgfsetlinewidth{1.003750pt}%
\definecolor{currentstroke}{rgb}{0.827451,0.827451,0.827451}%
\pgfsetstrokecolor{currentstroke}%
\pgfsetstrokeopacity{0.800000}%
\pgfsetdash{}{0pt}%
\pgfpathmoveto{\pgfqpoint{9.696495in}{1.337612in}}%
\pgfpathcurveto{\pgfqpoint{9.707545in}{1.337612in}}{\pgfqpoint{9.718144in}{1.342002in}}{\pgfqpoint{9.725958in}{1.349816in}}%
\pgfpathcurveto{\pgfqpoint{9.733772in}{1.357629in}}{\pgfqpoint{9.738162in}{1.368229in}}{\pgfqpoint{9.738162in}{1.379279in}}%
\pgfpathcurveto{\pgfqpoint{9.738162in}{1.390329in}}{\pgfqpoint{9.733772in}{1.400928in}}{\pgfqpoint{9.725958in}{1.408741in}}%
\pgfpathcurveto{\pgfqpoint{9.718144in}{1.416555in}}{\pgfqpoint{9.707545in}{1.420945in}}{\pgfqpoint{9.696495in}{1.420945in}}%
\pgfpathcurveto{\pgfqpoint{9.685445in}{1.420945in}}{\pgfqpoint{9.674846in}{1.416555in}}{\pgfqpoint{9.667033in}{1.408741in}}%
\pgfpathcurveto{\pgfqpoint{9.659219in}{1.400928in}}{\pgfqpoint{9.654829in}{1.390329in}}{\pgfqpoint{9.654829in}{1.379279in}}%
\pgfpathcurveto{\pgfqpoint{9.654829in}{1.368229in}}{\pgfqpoint{9.659219in}{1.357629in}}{\pgfqpoint{9.667033in}{1.349816in}}%
\pgfpathcurveto{\pgfqpoint{9.674846in}{1.342002in}}{\pgfqpoint{9.685445in}{1.337612in}}{\pgfqpoint{9.696495in}{1.337612in}}%
\pgfpathlineto{\pgfqpoint{9.696495in}{1.337612in}}%
\pgfpathclose%
\pgfusepath{stroke}%
\end{pgfscope}%
\begin{pgfscope}%
\pgfpathrectangle{\pgfqpoint{7.512535in}{0.437222in}}{\pgfqpoint{6.275590in}{5.159444in}}%
\pgfusepath{clip}%
\pgfsetbuttcap%
\pgfsetroundjoin%
\pgfsetlinewidth{1.003750pt}%
\definecolor{currentstroke}{rgb}{0.827451,0.827451,0.827451}%
\pgfsetstrokecolor{currentstroke}%
\pgfsetstrokeopacity{0.800000}%
\pgfsetdash{}{0pt}%
\pgfpathmoveto{\pgfqpoint{12.072745in}{5.039509in}}%
\pgfpathcurveto{\pgfqpoint{12.083796in}{5.039509in}}{\pgfqpoint{12.094395in}{5.043899in}}{\pgfqpoint{12.102208in}{5.051713in}}%
\pgfpathcurveto{\pgfqpoint{12.110022in}{5.059526in}}{\pgfqpoint{12.114412in}{5.070125in}}{\pgfqpoint{12.114412in}{5.081175in}}%
\pgfpathcurveto{\pgfqpoint{12.114412in}{5.092226in}}{\pgfqpoint{12.110022in}{5.102825in}}{\pgfqpoint{12.102208in}{5.110638in}}%
\pgfpathcurveto{\pgfqpoint{12.094395in}{5.118452in}}{\pgfqpoint{12.083796in}{5.122842in}}{\pgfqpoint{12.072745in}{5.122842in}}%
\pgfpathcurveto{\pgfqpoint{12.061695in}{5.122842in}}{\pgfqpoint{12.051096in}{5.118452in}}{\pgfqpoint{12.043283in}{5.110638in}}%
\pgfpathcurveto{\pgfqpoint{12.035469in}{5.102825in}}{\pgfqpoint{12.031079in}{5.092226in}}{\pgfqpoint{12.031079in}{5.081175in}}%
\pgfpathcurveto{\pgfqpoint{12.031079in}{5.070125in}}{\pgfqpoint{12.035469in}{5.059526in}}{\pgfqpoint{12.043283in}{5.051713in}}%
\pgfpathcurveto{\pgfqpoint{12.051096in}{5.043899in}}{\pgfqpoint{12.061695in}{5.039509in}}{\pgfqpoint{12.072745in}{5.039509in}}%
\pgfpathlineto{\pgfqpoint{12.072745in}{5.039509in}}%
\pgfpathclose%
\pgfusepath{stroke}%
\end{pgfscope}%
\begin{pgfscope}%
\pgfpathrectangle{\pgfqpoint{7.512535in}{0.437222in}}{\pgfqpoint{6.275590in}{5.159444in}}%
\pgfusepath{clip}%
\pgfsetbuttcap%
\pgfsetroundjoin%
\pgfsetlinewidth{1.003750pt}%
\definecolor{currentstroke}{rgb}{0.827451,0.827451,0.827451}%
\pgfsetstrokecolor{currentstroke}%
\pgfsetstrokeopacity{0.800000}%
\pgfsetdash{}{0pt}%
\pgfpathmoveto{\pgfqpoint{9.339517in}{1.176671in}}%
\pgfpathcurveto{\pgfqpoint{9.350568in}{1.176671in}}{\pgfqpoint{9.361167in}{1.181061in}}{\pgfqpoint{9.368980in}{1.188875in}}%
\pgfpathcurveto{\pgfqpoint{9.376794in}{1.196689in}}{\pgfqpoint{9.381184in}{1.207288in}}{\pgfqpoint{9.381184in}{1.218338in}}%
\pgfpathcurveto{\pgfqpoint{9.381184in}{1.229388in}}{\pgfqpoint{9.376794in}{1.239987in}}{\pgfqpoint{9.368980in}{1.247800in}}%
\pgfpathcurveto{\pgfqpoint{9.361167in}{1.255614in}}{\pgfqpoint{9.350568in}{1.260004in}}{\pgfqpoint{9.339517in}{1.260004in}}%
\pgfpathcurveto{\pgfqpoint{9.328467in}{1.260004in}}{\pgfqpoint{9.317868in}{1.255614in}}{\pgfqpoint{9.310055in}{1.247800in}}%
\pgfpathcurveto{\pgfqpoint{9.302241in}{1.239987in}}{\pgfqpoint{9.297851in}{1.229388in}}{\pgfqpoint{9.297851in}{1.218338in}}%
\pgfpathcurveto{\pgfqpoint{9.297851in}{1.207288in}}{\pgfqpoint{9.302241in}{1.196689in}}{\pgfqpoint{9.310055in}{1.188875in}}%
\pgfpathcurveto{\pgfqpoint{9.317868in}{1.181061in}}{\pgfqpoint{9.328467in}{1.176671in}}{\pgfqpoint{9.339517in}{1.176671in}}%
\pgfpathlineto{\pgfqpoint{9.339517in}{1.176671in}}%
\pgfpathclose%
\pgfusepath{stroke}%
\end{pgfscope}%
\begin{pgfscope}%
\pgfpathrectangle{\pgfqpoint{7.512535in}{0.437222in}}{\pgfqpoint{6.275590in}{5.159444in}}%
\pgfusepath{clip}%
\pgfsetbuttcap%
\pgfsetroundjoin%
\pgfsetlinewidth{1.003750pt}%
\definecolor{currentstroke}{rgb}{0.827451,0.827451,0.827451}%
\pgfsetstrokecolor{currentstroke}%
\pgfsetstrokeopacity{0.800000}%
\pgfsetdash{}{0pt}%
\pgfpathmoveto{\pgfqpoint{9.615954in}{3.788157in}}%
\pgfpathcurveto{\pgfqpoint{9.627004in}{3.788157in}}{\pgfqpoint{9.637603in}{3.792548in}}{\pgfqpoint{9.645416in}{3.800361in}}%
\pgfpathcurveto{\pgfqpoint{9.653230in}{3.808175in}}{\pgfqpoint{9.657620in}{3.818774in}}{\pgfqpoint{9.657620in}{3.829824in}}%
\pgfpathcurveto{\pgfqpoint{9.657620in}{3.840874in}}{\pgfqpoint{9.653230in}{3.851473in}}{\pgfqpoint{9.645416in}{3.859287in}}%
\pgfpathcurveto{\pgfqpoint{9.637603in}{3.867100in}}{\pgfqpoint{9.627004in}{3.871491in}}{\pgfqpoint{9.615954in}{3.871491in}}%
\pgfpathcurveto{\pgfqpoint{9.604903in}{3.871491in}}{\pgfqpoint{9.594304in}{3.867100in}}{\pgfqpoint{9.586491in}{3.859287in}}%
\pgfpathcurveto{\pgfqpoint{9.578677in}{3.851473in}}{\pgfqpoint{9.574287in}{3.840874in}}{\pgfqpoint{9.574287in}{3.829824in}}%
\pgfpathcurveto{\pgfqpoint{9.574287in}{3.818774in}}{\pgfqpoint{9.578677in}{3.808175in}}{\pgfqpoint{9.586491in}{3.800361in}}%
\pgfpathcurveto{\pgfqpoint{9.594304in}{3.792548in}}{\pgfqpoint{9.604903in}{3.788157in}}{\pgfqpoint{9.615954in}{3.788157in}}%
\pgfpathlineto{\pgfqpoint{9.615954in}{3.788157in}}%
\pgfpathclose%
\pgfusepath{stroke}%
\end{pgfscope}%
\begin{pgfscope}%
\pgfpathrectangle{\pgfqpoint{7.512535in}{0.437222in}}{\pgfqpoint{6.275590in}{5.159444in}}%
\pgfusepath{clip}%
\pgfsetbuttcap%
\pgfsetroundjoin%
\pgfsetlinewidth{1.003750pt}%
\definecolor{currentstroke}{rgb}{0.827451,0.827451,0.827451}%
\pgfsetstrokecolor{currentstroke}%
\pgfsetstrokeopacity{0.800000}%
\pgfsetdash{}{0pt}%
\pgfpathmoveto{\pgfqpoint{7.972802in}{0.679358in}}%
\pgfpathcurveto{\pgfqpoint{7.983853in}{0.679358in}}{\pgfqpoint{7.994452in}{0.683748in}}{\pgfqpoint{8.002265in}{0.691562in}}%
\pgfpathcurveto{\pgfqpoint{8.010079in}{0.699376in}}{\pgfqpoint{8.014469in}{0.709975in}}{\pgfqpoint{8.014469in}{0.721025in}}%
\pgfpathcurveto{\pgfqpoint{8.014469in}{0.732075in}}{\pgfqpoint{8.010079in}{0.742674in}}{\pgfqpoint{8.002265in}{0.750487in}}%
\pgfpathcurveto{\pgfqpoint{7.994452in}{0.758301in}}{\pgfqpoint{7.983853in}{0.762691in}}{\pgfqpoint{7.972802in}{0.762691in}}%
\pgfpathcurveto{\pgfqpoint{7.961752in}{0.762691in}}{\pgfqpoint{7.951153in}{0.758301in}}{\pgfqpoint{7.943340in}{0.750487in}}%
\pgfpathcurveto{\pgfqpoint{7.935526in}{0.742674in}}{\pgfqpoint{7.931136in}{0.732075in}}{\pgfqpoint{7.931136in}{0.721025in}}%
\pgfpathcurveto{\pgfqpoint{7.931136in}{0.709975in}}{\pgfqpoint{7.935526in}{0.699376in}}{\pgfqpoint{7.943340in}{0.691562in}}%
\pgfpathcurveto{\pgfqpoint{7.951153in}{0.683748in}}{\pgfqpoint{7.961752in}{0.679358in}}{\pgfqpoint{7.972802in}{0.679358in}}%
\pgfpathlineto{\pgfqpoint{7.972802in}{0.679358in}}%
\pgfpathclose%
\pgfusepath{stroke}%
\end{pgfscope}%
\begin{pgfscope}%
\pgfpathrectangle{\pgfqpoint{7.512535in}{0.437222in}}{\pgfqpoint{6.275590in}{5.159444in}}%
\pgfusepath{clip}%
\pgfsetbuttcap%
\pgfsetroundjoin%
\pgfsetlinewidth{1.003750pt}%
\definecolor{currentstroke}{rgb}{0.827451,0.827451,0.827451}%
\pgfsetstrokecolor{currentstroke}%
\pgfsetstrokeopacity{0.800000}%
\pgfsetdash{}{0pt}%
\pgfpathmoveto{\pgfqpoint{9.540773in}{3.442114in}}%
\pgfpathcurveto{\pgfqpoint{9.551823in}{3.442114in}}{\pgfqpoint{9.562422in}{3.446504in}}{\pgfqpoint{9.570236in}{3.454318in}}%
\pgfpathcurveto{\pgfqpoint{9.578049in}{3.462131in}}{\pgfqpoint{9.582440in}{3.472730in}}{\pgfqpoint{9.582440in}{3.483780in}}%
\pgfpathcurveto{\pgfqpoint{9.582440in}{3.494830in}}{\pgfqpoint{9.578049in}{3.505429in}}{\pgfqpoint{9.570236in}{3.513243in}}%
\pgfpathcurveto{\pgfqpoint{9.562422in}{3.521057in}}{\pgfqpoint{9.551823in}{3.525447in}}{\pgfqpoint{9.540773in}{3.525447in}}%
\pgfpathcurveto{\pgfqpoint{9.529723in}{3.525447in}}{\pgfqpoint{9.519124in}{3.521057in}}{\pgfqpoint{9.511310in}{3.513243in}}%
\pgfpathcurveto{\pgfqpoint{9.503497in}{3.505429in}}{\pgfqpoint{9.499106in}{3.494830in}}{\pgfqpoint{9.499106in}{3.483780in}}%
\pgfpathcurveto{\pgfqpoint{9.499106in}{3.472730in}}{\pgfqpoint{9.503497in}{3.462131in}}{\pgfqpoint{9.511310in}{3.454318in}}%
\pgfpathcurveto{\pgfqpoint{9.519124in}{3.446504in}}{\pgfqpoint{9.529723in}{3.442114in}}{\pgfqpoint{9.540773in}{3.442114in}}%
\pgfpathlineto{\pgfqpoint{9.540773in}{3.442114in}}%
\pgfpathclose%
\pgfusepath{stroke}%
\end{pgfscope}%
\begin{pgfscope}%
\pgfpathrectangle{\pgfqpoint{7.512535in}{0.437222in}}{\pgfqpoint{6.275590in}{5.159444in}}%
\pgfusepath{clip}%
\pgfsetbuttcap%
\pgfsetroundjoin%
\pgfsetlinewidth{1.003750pt}%
\definecolor{currentstroke}{rgb}{0.827451,0.827451,0.827451}%
\pgfsetstrokecolor{currentstroke}%
\pgfsetstrokeopacity{0.800000}%
\pgfsetdash{}{0pt}%
\pgfpathmoveto{\pgfqpoint{8.598634in}{1.087031in}}%
\pgfpathcurveto{\pgfqpoint{8.609684in}{1.087031in}}{\pgfqpoint{8.620283in}{1.091421in}}{\pgfqpoint{8.628097in}{1.099234in}}%
\pgfpathcurveto{\pgfqpoint{8.635910in}{1.107048in}}{\pgfqpoint{8.640300in}{1.117647in}}{\pgfqpoint{8.640300in}{1.128697in}}%
\pgfpathcurveto{\pgfqpoint{8.640300in}{1.139747in}}{\pgfqpoint{8.635910in}{1.150346in}}{\pgfqpoint{8.628097in}{1.158160in}}%
\pgfpathcurveto{\pgfqpoint{8.620283in}{1.165974in}}{\pgfqpoint{8.609684in}{1.170364in}}{\pgfqpoint{8.598634in}{1.170364in}}%
\pgfpathcurveto{\pgfqpoint{8.587584in}{1.170364in}}{\pgfqpoint{8.576985in}{1.165974in}}{\pgfqpoint{8.569171in}{1.158160in}}%
\pgfpathcurveto{\pgfqpoint{8.561357in}{1.150346in}}{\pgfqpoint{8.556967in}{1.139747in}}{\pgfqpoint{8.556967in}{1.128697in}}%
\pgfpathcurveto{\pgfqpoint{8.556967in}{1.117647in}}{\pgfqpoint{8.561357in}{1.107048in}}{\pgfqpoint{8.569171in}{1.099234in}}%
\pgfpathcurveto{\pgfqpoint{8.576985in}{1.091421in}}{\pgfqpoint{8.587584in}{1.087031in}}{\pgfqpoint{8.598634in}{1.087031in}}%
\pgfpathlineto{\pgfqpoint{8.598634in}{1.087031in}}%
\pgfpathclose%
\pgfusepath{stroke}%
\end{pgfscope}%
\begin{pgfscope}%
\pgfpathrectangle{\pgfqpoint{7.512535in}{0.437222in}}{\pgfqpoint{6.275590in}{5.159444in}}%
\pgfusepath{clip}%
\pgfsetbuttcap%
\pgfsetroundjoin%
\pgfsetlinewidth{1.003750pt}%
\definecolor{currentstroke}{rgb}{0.827451,0.827451,0.827451}%
\pgfsetstrokecolor{currentstroke}%
\pgfsetstrokeopacity{0.800000}%
\pgfsetdash{}{0pt}%
\pgfpathmoveto{\pgfqpoint{9.505328in}{3.327623in}}%
\pgfpathcurveto{\pgfqpoint{9.516379in}{3.327623in}}{\pgfqpoint{9.526978in}{3.332014in}}{\pgfqpoint{9.534791in}{3.339827in}}%
\pgfpathcurveto{\pgfqpoint{9.542605in}{3.347641in}}{\pgfqpoint{9.546995in}{3.358240in}}{\pgfqpoint{9.546995in}{3.369290in}}%
\pgfpathcurveto{\pgfqpoint{9.546995in}{3.380340in}}{\pgfqpoint{9.542605in}{3.390939in}}{\pgfqpoint{9.534791in}{3.398753in}}%
\pgfpathcurveto{\pgfqpoint{9.526978in}{3.406567in}}{\pgfqpoint{9.516379in}{3.410957in}}{\pgfqpoint{9.505328in}{3.410957in}}%
\pgfpathcurveto{\pgfqpoint{9.494278in}{3.410957in}}{\pgfqpoint{9.483679in}{3.406567in}}{\pgfqpoint{9.475866in}{3.398753in}}%
\pgfpathcurveto{\pgfqpoint{9.468052in}{3.390939in}}{\pgfqpoint{9.463662in}{3.380340in}}{\pgfqpoint{9.463662in}{3.369290in}}%
\pgfpathcurveto{\pgfqpoint{9.463662in}{3.358240in}}{\pgfqpoint{9.468052in}{3.347641in}}{\pgfqpoint{9.475866in}{3.339827in}}%
\pgfpathcurveto{\pgfqpoint{9.483679in}{3.332014in}}{\pgfqpoint{9.494278in}{3.327623in}}{\pgfqpoint{9.505328in}{3.327623in}}%
\pgfpathlineto{\pgfqpoint{9.505328in}{3.327623in}}%
\pgfpathclose%
\pgfusepath{stroke}%
\end{pgfscope}%
\begin{pgfscope}%
\pgfpathrectangle{\pgfqpoint{7.512535in}{0.437222in}}{\pgfqpoint{6.275590in}{5.159444in}}%
\pgfusepath{clip}%
\pgfsetbuttcap%
\pgfsetroundjoin%
\pgfsetlinewidth{1.003750pt}%
\definecolor{currentstroke}{rgb}{0.827451,0.827451,0.827451}%
\pgfsetstrokecolor{currentstroke}%
\pgfsetstrokeopacity{0.800000}%
\pgfsetdash{}{0pt}%
\pgfpathmoveto{\pgfqpoint{8.044541in}{2.113663in}}%
\pgfpathcurveto{\pgfqpoint{8.055591in}{2.113663in}}{\pgfqpoint{8.066190in}{2.118054in}}{\pgfqpoint{8.074003in}{2.125867in}}%
\pgfpathcurveto{\pgfqpoint{8.081817in}{2.133681in}}{\pgfqpoint{8.086207in}{2.144280in}}{\pgfqpoint{8.086207in}{2.155330in}}%
\pgfpathcurveto{\pgfqpoint{8.086207in}{2.166380in}}{\pgfqpoint{8.081817in}{2.176979in}}{\pgfqpoint{8.074003in}{2.184793in}}%
\pgfpathcurveto{\pgfqpoint{8.066190in}{2.192606in}}{\pgfqpoint{8.055591in}{2.196997in}}{\pgfqpoint{8.044541in}{2.196997in}}%
\pgfpathcurveto{\pgfqpoint{8.033491in}{2.196997in}}{\pgfqpoint{8.022891in}{2.192606in}}{\pgfqpoint{8.015078in}{2.184793in}}%
\pgfpathcurveto{\pgfqpoint{8.007264in}{2.176979in}}{\pgfqpoint{8.002874in}{2.166380in}}{\pgfqpoint{8.002874in}{2.155330in}}%
\pgfpathcurveto{\pgfqpoint{8.002874in}{2.144280in}}{\pgfqpoint{8.007264in}{2.133681in}}{\pgfqpoint{8.015078in}{2.125867in}}%
\pgfpathcurveto{\pgfqpoint{8.022891in}{2.118054in}}{\pgfqpoint{8.033491in}{2.113663in}}{\pgfqpoint{8.044541in}{2.113663in}}%
\pgfpathlineto{\pgfqpoint{8.044541in}{2.113663in}}%
\pgfpathclose%
\pgfusepath{stroke}%
\end{pgfscope}%
\begin{pgfscope}%
\pgfpathrectangle{\pgfqpoint{7.512535in}{0.437222in}}{\pgfqpoint{6.275590in}{5.159444in}}%
\pgfusepath{clip}%
\pgfsetbuttcap%
\pgfsetroundjoin%
\pgfsetlinewidth{1.003750pt}%
\definecolor{currentstroke}{rgb}{0.827451,0.827451,0.827451}%
\pgfsetstrokecolor{currentstroke}%
\pgfsetstrokeopacity{0.800000}%
\pgfsetdash{}{0pt}%
\pgfpathmoveto{\pgfqpoint{12.292049in}{5.113368in}}%
\pgfpathcurveto{\pgfqpoint{12.303099in}{5.113368in}}{\pgfqpoint{12.313698in}{5.117758in}}{\pgfqpoint{12.321512in}{5.125572in}}%
\pgfpathcurveto{\pgfqpoint{12.329325in}{5.133385in}}{\pgfqpoint{12.333716in}{5.143984in}}{\pgfqpoint{12.333716in}{5.155034in}}%
\pgfpathcurveto{\pgfqpoint{12.333716in}{5.166084in}}{\pgfqpoint{12.329325in}{5.176683in}}{\pgfqpoint{12.321512in}{5.184497in}}%
\pgfpathcurveto{\pgfqpoint{12.313698in}{5.192311in}}{\pgfqpoint{12.303099in}{5.196701in}}{\pgfqpoint{12.292049in}{5.196701in}}%
\pgfpathcurveto{\pgfqpoint{12.280999in}{5.196701in}}{\pgfqpoint{12.270400in}{5.192311in}}{\pgfqpoint{12.262586in}{5.184497in}}%
\pgfpathcurveto{\pgfqpoint{12.254772in}{5.176683in}}{\pgfqpoint{12.250382in}{5.166084in}}{\pgfqpoint{12.250382in}{5.155034in}}%
\pgfpathcurveto{\pgfqpoint{12.250382in}{5.143984in}}{\pgfqpoint{12.254772in}{5.133385in}}{\pgfqpoint{12.262586in}{5.125572in}}%
\pgfpathcurveto{\pgfqpoint{12.270400in}{5.117758in}}{\pgfqpoint{12.280999in}{5.113368in}}{\pgfqpoint{12.292049in}{5.113368in}}%
\pgfpathlineto{\pgfqpoint{12.292049in}{5.113368in}}%
\pgfpathclose%
\pgfusepath{stroke}%
\end{pgfscope}%
\begin{pgfscope}%
\pgfpathrectangle{\pgfqpoint{7.512535in}{0.437222in}}{\pgfqpoint{6.275590in}{5.159444in}}%
\pgfusepath{clip}%
\pgfsetbuttcap%
\pgfsetroundjoin%
\pgfsetlinewidth{1.003750pt}%
\definecolor{currentstroke}{rgb}{0.827451,0.827451,0.827451}%
\pgfsetstrokecolor{currentstroke}%
\pgfsetstrokeopacity{0.800000}%
\pgfsetdash{}{0pt}%
\pgfpathmoveto{\pgfqpoint{13.101307in}{5.417061in}}%
\pgfpathcurveto{\pgfqpoint{13.112357in}{5.417061in}}{\pgfqpoint{13.122956in}{5.421451in}}{\pgfqpoint{13.130770in}{5.429265in}}%
\pgfpathcurveto{\pgfqpoint{13.138583in}{5.437079in}}{\pgfqpoint{13.142973in}{5.447678in}}{\pgfqpoint{13.142973in}{5.458728in}}%
\pgfpathcurveto{\pgfqpoint{13.142973in}{5.469778in}}{\pgfqpoint{13.138583in}{5.480377in}}{\pgfqpoint{13.130770in}{5.488191in}}%
\pgfpathcurveto{\pgfqpoint{13.122956in}{5.496004in}}{\pgfqpoint{13.112357in}{5.500395in}}{\pgfqpoint{13.101307in}{5.500395in}}%
\pgfpathcurveto{\pgfqpoint{13.090257in}{5.500395in}}{\pgfqpoint{13.079658in}{5.496004in}}{\pgfqpoint{13.071844in}{5.488191in}}%
\pgfpathcurveto{\pgfqpoint{13.064030in}{5.480377in}}{\pgfqpoint{13.059640in}{5.469778in}}{\pgfqpoint{13.059640in}{5.458728in}}%
\pgfpathcurveto{\pgfqpoint{13.059640in}{5.447678in}}{\pgfqpoint{13.064030in}{5.437079in}}{\pgfqpoint{13.071844in}{5.429265in}}%
\pgfpathcurveto{\pgfqpoint{13.079658in}{5.421451in}}{\pgfqpoint{13.090257in}{5.417061in}}{\pgfqpoint{13.101307in}{5.417061in}}%
\pgfpathlineto{\pgfqpoint{13.101307in}{5.417061in}}%
\pgfpathclose%
\pgfusepath{stroke}%
\end{pgfscope}%
\begin{pgfscope}%
\pgfpathrectangle{\pgfqpoint{7.512535in}{0.437222in}}{\pgfqpoint{6.275590in}{5.159444in}}%
\pgfusepath{clip}%
\pgfsetbuttcap%
\pgfsetroundjoin%
\pgfsetlinewidth{1.003750pt}%
\definecolor{currentstroke}{rgb}{0.827451,0.827451,0.827451}%
\pgfsetstrokecolor{currentstroke}%
\pgfsetstrokeopacity{0.800000}%
\pgfsetdash{}{0pt}%
\pgfpathmoveto{\pgfqpoint{9.807010in}{4.531380in}}%
\pgfpathcurveto{\pgfqpoint{9.818060in}{4.531380in}}{\pgfqpoint{9.828659in}{4.535770in}}{\pgfqpoint{9.836473in}{4.543584in}}%
\pgfpathcurveto{\pgfqpoint{9.844287in}{4.551397in}}{\pgfqpoint{9.848677in}{4.561996in}}{\pgfqpoint{9.848677in}{4.573046in}}%
\pgfpathcurveto{\pgfqpoint{9.848677in}{4.584096in}}{\pgfqpoint{9.844287in}{4.594695in}}{\pgfqpoint{9.836473in}{4.602509in}}%
\pgfpathcurveto{\pgfqpoint{9.828659in}{4.610323in}}{\pgfqpoint{9.818060in}{4.614713in}}{\pgfqpoint{9.807010in}{4.614713in}}%
\pgfpathcurveto{\pgfqpoint{9.795960in}{4.614713in}}{\pgfqpoint{9.785361in}{4.610323in}}{\pgfqpoint{9.777548in}{4.602509in}}%
\pgfpathcurveto{\pgfqpoint{9.769734in}{4.594695in}}{\pgfqpoint{9.765344in}{4.584096in}}{\pgfqpoint{9.765344in}{4.573046in}}%
\pgfpathcurveto{\pgfqpoint{9.765344in}{4.561996in}}{\pgfqpoint{9.769734in}{4.551397in}}{\pgfqpoint{9.777548in}{4.543584in}}%
\pgfpathcurveto{\pgfqpoint{9.785361in}{4.535770in}}{\pgfqpoint{9.795960in}{4.531380in}}{\pgfqpoint{9.807010in}{4.531380in}}%
\pgfpathlineto{\pgfqpoint{9.807010in}{4.531380in}}%
\pgfpathclose%
\pgfusepath{stroke}%
\end{pgfscope}%
\begin{pgfscope}%
\pgfpathrectangle{\pgfqpoint{7.512535in}{0.437222in}}{\pgfqpoint{6.275590in}{5.159444in}}%
\pgfusepath{clip}%
\pgfsetbuttcap%
\pgfsetroundjoin%
\pgfsetlinewidth{1.003750pt}%
\definecolor{currentstroke}{rgb}{0.827451,0.827451,0.827451}%
\pgfsetstrokecolor{currentstroke}%
\pgfsetstrokeopacity{0.800000}%
\pgfsetdash{}{0pt}%
\pgfpathmoveto{\pgfqpoint{9.792079in}{4.165674in}}%
\pgfpathcurveto{\pgfqpoint{9.803129in}{4.165674in}}{\pgfqpoint{9.813728in}{4.170064in}}{\pgfqpoint{9.821541in}{4.177878in}}%
\pgfpathcurveto{\pgfqpoint{9.829355in}{4.185692in}}{\pgfqpoint{9.833745in}{4.196291in}}{\pgfqpoint{9.833745in}{4.207341in}}%
\pgfpathcurveto{\pgfqpoint{9.833745in}{4.218391in}}{\pgfqpoint{9.829355in}{4.228990in}}{\pgfqpoint{9.821541in}{4.236803in}}%
\pgfpathcurveto{\pgfqpoint{9.813728in}{4.244617in}}{\pgfqpoint{9.803129in}{4.249007in}}{\pgfqpoint{9.792079in}{4.249007in}}%
\pgfpathcurveto{\pgfqpoint{9.781029in}{4.249007in}}{\pgfqpoint{9.770429in}{4.244617in}}{\pgfqpoint{9.762616in}{4.236803in}}%
\pgfpathcurveto{\pgfqpoint{9.754802in}{4.228990in}}{\pgfqpoint{9.750412in}{4.218391in}}{\pgfqpoint{9.750412in}{4.207341in}}%
\pgfpathcurveto{\pgfqpoint{9.750412in}{4.196291in}}{\pgfqpoint{9.754802in}{4.185692in}}{\pgfqpoint{9.762616in}{4.177878in}}%
\pgfpathcurveto{\pgfqpoint{9.770429in}{4.170064in}}{\pgfqpoint{9.781029in}{4.165674in}}{\pgfqpoint{9.792079in}{4.165674in}}%
\pgfpathlineto{\pgfqpoint{9.792079in}{4.165674in}}%
\pgfpathclose%
\pgfusepath{stroke}%
\end{pgfscope}%
\begin{pgfscope}%
\pgfpathrectangle{\pgfqpoint{7.512535in}{0.437222in}}{\pgfqpoint{6.275590in}{5.159444in}}%
\pgfusepath{clip}%
\pgfsetbuttcap%
\pgfsetroundjoin%
\pgfsetlinewidth{1.003750pt}%
\definecolor{currentstroke}{rgb}{0.827451,0.827451,0.827451}%
\pgfsetstrokecolor{currentstroke}%
\pgfsetstrokeopacity{0.800000}%
\pgfsetdash{}{0pt}%
\pgfpathmoveto{\pgfqpoint{8.309719in}{2.302744in}}%
\pgfpathcurveto{\pgfqpoint{8.320769in}{2.302744in}}{\pgfqpoint{8.331368in}{2.307134in}}{\pgfqpoint{8.339182in}{2.314947in}}%
\pgfpathcurveto{\pgfqpoint{8.346996in}{2.322761in}}{\pgfqpoint{8.351386in}{2.333360in}}{\pgfqpoint{8.351386in}{2.344410in}}%
\pgfpathcurveto{\pgfqpoint{8.351386in}{2.355460in}}{\pgfqpoint{8.346996in}{2.366059in}}{\pgfqpoint{8.339182in}{2.373873in}}%
\pgfpathcurveto{\pgfqpoint{8.331368in}{2.381687in}}{\pgfqpoint{8.320769in}{2.386077in}}{\pgfqpoint{8.309719in}{2.386077in}}%
\pgfpathcurveto{\pgfqpoint{8.298669in}{2.386077in}}{\pgfqpoint{8.288070in}{2.381687in}}{\pgfqpoint{8.280256in}{2.373873in}}%
\pgfpathcurveto{\pgfqpoint{8.272443in}{2.366059in}}{\pgfqpoint{8.268052in}{2.355460in}}{\pgfqpoint{8.268052in}{2.344410in}}%
\pgfpathcurveto{\pgfqpoint{8.268052in}{2.333360in}}{\pgfqpoint{8.272443in}{2.322761in}}{\pgfqpoint{8.280256in}{2.314947in}}%
\pgfpathcurveto{\pgfqpoint{8.288070in}{2.307134in}}{\pgfqpoint{8.298669in}{2.302744in}}{\pgfqpoint{8.309719in}{2.302744in}}%
\pgfpathlineto{\pgfqpoint{8.309719in}{2.302744in}}%
\pgfpathclose%
\pgfusepath{stroke}%
\end{pgfscope}%
\begin{pgfscope}%
\pgfpathrectangle{\pgfqpoint{7.512535in}{0.437222in}}{\pgfqpoint{6.275590in}{5.159444in}}%
\pgfusepath{clip}%
\pgfsetbuttcap%
\pgfsetroundjoin%
\pgfsetlinewidth{1.003750pt}%
\definecolor{currentstroke}{rgb}{0.827451,0.827451,0.827451}%
\pgfsetstrokecolor{currentstroke}%
\pgfsetstrokeopacity{0.800000}%
\pgfsetdash{}{0pt}%
\pgfpathmoveto{\pgfqpoint{12.547906in}{5.274085in}}%
\pgfpathcurveto{\pgfqpoint{12.558956in}{5.274085in}}{\pgfqpoint{12.569556in}{5.278475in}}{\pgfqpoint{12.577369in}{5.286289in}}%
\pgfpathcurveto{\pgfqpoint{12.585183in}{5.294102in}}{\pgfqpoint{12.589573in}{5.304701in}}{\pgfqpoint{12.589573in}{5.315752in}}%
\pgfpathcurveto{\pgfqpoint{12.589573in}{5.326802in}}{\pgfqpoint{12.585183in}{5.337401in}}{\pgfqpoint{12.577369in}{5.345214in}}%
\pgfpathcurveto{\pgfqpoint{12.569556in}{5.353028in}}{\pgfqpoint{12.558956in}{5.357418in}}{\pgfqpoint{12.547906in}{5.357418in}}%
\pgfpathcurveto{\pgfqpoint{12.536856in}{5.357418in}}{\pgfqpoint{12.526257in}{5.353028in}}{\pgfqpoint{12.518444in}{5.345214in}}%
\pgfpathcurveto{\pgfqpoint{12.510630in}{5.337401in}}{\pgfqpoint{12.506240in}{5.326802in}}{\pgfqpoint{12.506240in}{5.315752in}}%
\pgfpathcurveto{\pgfqpoint{12.506240in}{5.304701in}}{\pgfqpoint{12.510630in}{5.294102in}}{\pgfqpoint{12.518444in}{5.286289in}}%
\pgfpathcurveto{\pgfqpoint{12.526257in}{5.278475in}}{\pgfqpoint{12.536856in}{5.274085in}}{\pgfqpoint{12.547906in}{5.274085in}}%
\pgfpathlineto{\pgfqpoint{12.547906in}{5.274085in}}%
\pgfpathclose%
\pgfusepath{stroke}%
\end{pgfscope}%
\begin{pgfscope}%
\pgfpathrectangle{\pgfqpoint{7.512535in}{0.437222in}}{\pgfqpoint{6.275590in}{5.159444in}}%
\pgfusepath{clip}%
\pgfsetbuttcap%
\pgfsetroundjoin%
\pgfsetlinewidth{1.003750pt}%
\definecolor{currentstroke}{rgb}{0.827451,0.827451,0.827451}%
\pgfsetstrokecolor{currentstroke}%
\pgfsetstrokeopacity{0.800000}%
\pgfsetdash{}{0pt}%
\pgfpathmoveto{\pgfqpoint{11.244156in}{3.147140in}}%
\pgfpathcurveto{\pgfqpoint{11.255206in}{3.147140in}}{\pgfqpoint{11.265805in}{3.151530in}}{\pgfqpoint{11.273619in}{3.159344in}}%
\pgfpathcurveto{\pgfqpoint{11.281433in}{3.167157in}}{\pgfqpoint{11.285823in}{3.177756in}}{\pgfqpoint{11.285823in}{3.188806in}}%
\pgfpathcurveto{\pgfqpoint{11.285823in}{3.199857in}}{\pgfqpoint{11.281433in}{3.210456in}}{\pgfqpoint{11.273619in}{3.218269in}}%
\pgfpathcurveto{\pgfqpoint{11.265805in}{3.226083in}}{\pgfqpoint{11.255206in}{3.230473in}}{\pgfqpoint{11.244156in}{3.230473in}}%
\pgfpathcurveto{\pgfqpoint{11.233106in}{3.230473in}}{\pgfqpoint{11.222507in}{3.226083in}}{\pgfqpoint{11.214693in}{3.218269in}}%
\pgfpathcurveto{\pgfqpoint{11.206880in}{3.210456in}}{\pgfqpoint{11.202490in}{3.199857in}}{\pgfqpoint{11.202490in}{3.188806in}}%
\pgfpathcurveto{\pgfqpoint{11.202490in}{3.177756in}}{\pgfqpoint{11.206880in}{3.167157in}}{\pgfqpoint{11.214693in}{3.159344in}}%
\pgfpathcurveto{\pgfqpoint{11.222507in}{3.151530in}}{\pgfqpoint{11.233106in}{3.147140in}}{\pgfqpoint{11.244156in}{3.147140in}}%
\pgfpathlineto{\pgfqpoint{11.244156in}{3.147140in}}%
\pgfpathclose%
\pgfusepath{stroke}%
\end{pgfscope}%
\begin{pgfscope}%
\pgfpathrectangle{\pgfqpoint{7.512535in}{0.437222in}}{\pgfqpoint{6.275590in}{5.159444in}}%
\pgfusepath{clip}%
\pgfsetbuttcap%
\pgfsetroundjoin%
\pgfsetlinewidth{1.003750pt}%
\definecolor{currentstroke}{rgb}{0.827451,0.827451,0.827451}%
\pgfsetstrokecolor{currentstroke}%
\pgfsetstrokeopacity{0.800000}%
\pgfsetdash{}{0pt}%
\pgfpathmoveto{\pgfqpoint{8.778671in}{3.998377in}}%
\pgfpathcurveto{\pgfqpoint{8.789721in}{3.998377in}}{\pgfqpoint{8.800320in}{4.002767in}}{\pgfqpoint{8.808134in}{4.010581in}}%
\pgfpathcurveto{\pgfqpoint{8.815947in}{4.018394in}}{\pgfqpoint{8.820337in}{4.028993in}}{\pgfqpoint{8.820337in}{4.040044in}}%
\pgfpathcurveto{\pgfqpoint{8.820337in}{4.051094in}}{\pgfqpoint{8.815947in}{4.061693in}}{\pgfqpoint{8.808134in}{4.069506in}}%
\pgfpathcurveto{\pgfqpoint{8.800320in}{4.077320in}}{\pgfqpoint{8.789721in}{4.081710in}}{\pgfqpoint{8.778671in}{4.081710in}}%
\pgfpathcurveto{\pgfqpoint{8.767621in}{4.081710in}}{\pgfqpoint{8.757022in}{4.077320in}}{\pgfqpoint{8.749208in}{4.069506in}}%
\pgfpathcurveto{\pgfqpoint{8.741394in}{4.061693in}}{\pgfqpoint{8.737004in}{4.051094in}}{\pgfqpoint{8.737004in}{4.040044in}}%
\pgfpathcurveto{\pgfqpoint{8.737004in}{4.028993in}}{\pgfqpoint{8.741394in}{4.018394in}}{\pgfqpoint{8.749208in}{4.010581in}}%
\pgfpathcurveto{\pgfqpoint{8.757022in}{4.002767in}}{\pgfqpoint{8.767621in}{3.998377in}}{\pgfqpoint{8.778671in}{3.998377in}}%
\pgfpathlineto{\pgfqpoint{8.778671in}{3.998377in}}%
\pgfpathclose%
\pgfusepath{stroke}%
\end{pgfscope}%
\begin{pgfscope}%
\pgfpathrectangle{\pgfqpoint{7.512535in}{0.437222in}}{\pgfqpoint{6.275590in}{5.159444in}}%
\pgfusepath{clip}%
\pgfsetbuttcap%
\pgfsetroundjoin%
\pgfsetlinewidth{1.003750pt}%
\definecolor{currentstroke}{rgb}{0.827451,0.827451,0.827451}%
\pgfsetstrokecolor{currentstroke}%
\pgfsetstrokeopacity{0.800000}%
\pgfsetdash{}{0pt}%
\pgfpathmoveto{\pgfqpoint{12.326644in}{4.049037in}}%
\pgfpathcurveto{\pgfqpoint{12.337695in}{4.049037in}}{\pgfqpoint{12.348294in}{4.053428in}}{\pgfqpoint{12.356107in}{4.061241in}}%
\pgfpathcurveto{\pgfqpoint{12.363921in}{4.069055in}}{\pgfqpoint{12.368311in}{4.079654in}}{\pgfqpoint{12.368311in}{4.090704in}}%
\pgfpathcurveto{\pgfqpoint{12.368311in}{4.101754in}}{\pgfqpoint{12.363921in}{4.112353in}}{\pgfqpoint{12.356107in}{4.120167in}}%
\pgfpathcurveto{\pgfqpoint{12.348294in}{4.127980in}}{\pgfqpoint{12.337695in}{4.132371in}}{\pgfqpoint{12.326644in}{4.132371in}}%
\pgfpathcurveto{\pgfqpoint{12.315594in}{4.132371in}}{\pgfqpoint{12.304995in}{4.127980in}}{\pgfqpoint{12.297182in}{4.120167in}}%
\pgfpathcurveto{\pgfqpoint{12.289368in}{4.112353in}}{\pgfqpoint{12.284978in}{4.101754in}}{\pgfqpoint{12.284978in}{4.090704in}}%
\pgfpathcurveto{\pgfqpoint{12.284978in}{4.079654in}}{\pgfqpoint{12.289368in}{4.069055in}}{\pgfqpoint{12.297182in}{4.061241in}}%
\pgfpathcurveto{\pgfqpoint{12.304995in}{4.053428in}}{\pgfqpoint{12.315594in}{4.049037in}}{\pgfqpoint{12.326644in}{4.049037in}}%
\pgfpathlineto{\pgfqpoint{12.326644in}{4.049037in}}%
\pgfpathclose%
\pgfusepath{stroke}%
\end{pgfscope}%
\begin{pgfscope}%
\pgfpathrectangle{\pgfqpoint{7.512535in}{0.437222in}}{\pgfqpoint{6.275590in}{5.159444in}}%
\pgfusepath{clip}%
\pgfsetbuttcap%
\pgfsetroundjoin%
\pgfsetlinewidth{1.003750pt}%
\definecolor{currentstroke}{rgb}{0.827451,0.827451,0.827451}%
\pgfsetstrokecolor{currentstroke}%
\pgfsetstrokeopacity{0.800000}%
\pgfsetdash{}{0pt}%
\pgfpathmoveto{\pgfqpoint{12.424663in}{4.014160in}}%
\pgfpathcurveto{\pgfqpoint{12.435713in}{4.014160in}}{\pgfqpoint{12.446312in}{4.018550in}}{\pgfqpoint{12.454125in}{4.026364in}}%
\pgfpathcurveto{\pgfqpoint{12.461939in}{4.034177in}}{\pgfqpoint{12.466329in}{4.044776in}}{\pgfqpoint{12.466329in}{4.055826in}}%
\pgfpathcurveto{\pgfqpoint{12.466329in}{4.066876in}}{\pgfqpoint{12.461939in}{4.077475in}}{\pgfqpoint{12.454125in}{4.085289in}}%
\pgfpathcurveto{\pgfqpoint{12.446312in}{4.093103in}}{\pgfqpoint{12.435713in}{4.097493in}}{\pgfqpoint{12.424663in}{4.097493in}}%
\pgfpathcurveto{\pgfqpoint{12.413612in}{4.097493in}}{\pgfqpoint{12.403013in}{4.093103in}}{\pgfqpoint{12.395200in}{4.085289in}}%
\pgfpathcurveto{\pgfqpoint{12.387386in}{4.077475in}}{\pgfqpoint{12.382996in}{4.066876in}}{\pgfqpoint{12.382996in}{4.055826in}}%
\pgfpathcurveto{\pgfqpoint{12.382996in}{4.044776in}}{\pgfqpoint{12.387386in}{4.034177in}}{\pgfqpoint{12.395200in}{4.026364in}}%
\pgfpathcurveto{\pgfqpoint{12.403013in}{4.018550in}}{\pgfqpoint{12.413612in}{4.014160in}}{\pgfqpoint{12.424663in}{4.014160in}}%
\pgfpathlineto{\pgfqpoint{12.424663in}{4.014160in}}%
\pgfpathclose%
\pgfusepath{stroke}%
\end{pgfscope}%
\begin{pgfscope}%
\pgfpathrectangle{\pgfqpoint{7.512535in}{0.437222in}}{\pgfqpoint{6.275590in}{5.159444in}}%
\pgfusepath{clip}%
\pgfsetbuttcap%
\pgfsetroundjoin%
\pgfsetlinewidth{1.003750pt}%
\definecolor{currentstroke}{rgb}{0.827451,0.827451,0.827451}%
\pgfsetstrokecolor{currentstroke}%
\pgfsetstrokeopacity{0.800000}%
\pgfsetdash{}{0pt}%
\pgfpathmoveto{\pgfqpoint{8.214738in}{3.528338in}}%
\pgfpathcurveto{\pgfqpoint{8.225788in}{3.528338in}}{\pgfqpoint{8.236387in}{3.532728in}}{\pgfqpoint{8.244200in}{3.540542in}}%
\pgfpathcurveto{\pgfqpoint{8.252014in}{3.548355in}}{\pgfqpoint{8.256404in}{3.558954in}}{\pgfqpoint{8.256404in}{3.570004in}}%
\pgfpathcurveto{\pgfqpoint{8.256404in}{3.581055in}}{\pgfqpoint{8.252014in}{3.591654in}}{\pgfqpoint{8.244200in}{3.599467in}}%
\pgfpathcurveto{\pgfqpoint{8.236387in}{3.607281in}}{\pgfqpoint{8.225788in}{3.611671in}}{\pgfqpoint{8.214738in}{3.611671in}}%
\pgfpathcurveto{\pgfqpoint{8.203687in}{3.611671in}}{\pgfqpoint{8.193088in}{3.607281in}}{\pgfqpoint{8.185275in}{3.599467in}}%
\pgfpathcurveto{\pgfqpoint{8.177461in}{3.591654in}}{\pgfqpoint{8.173071in}{3.581055in}}{\pgfqpoint{8.173071in}{3.570004in}}%
\pgfpathcurveto{\pgfqpoint{8.173071in}{3.558954in}}{\pgfqpoint{8.177461in}{3.548355in}}{\pgfqpoint{8.185275in}{3.540542in}}%
\pgfpathcurveto{\pgfqpoint{8.193088in}{3.532728in}}{\pgfqpoint{8.203687in}{3.528338in}}{\pgfqpoint{8.214738in}{3.528338in}}%
\pgfpathlineto{\pgfqpoint{8.214738in}{3.528338in}}%
\pgfpathclose%
\pgfusepath{stroke}%
\end{pgfscope}%
\begin{pgfscope}%
\pgfpathrectangle{\pgfqpoint{7.512535in}{0.437222in}}{\pgfqpoint{6.275590in}{5.159444in}}%
\pgfusepath{clip}%
\pgfsetbuttcap%
\pgfsetroundjoin%
\pgfsetlinewidth{1.003750pt}%
\definecolor{currentstroke}{rgb}{0.827451,0.827451,0.827451}%
\pgfsetstrokecolor{currentstroke}%
\pgfsetstrokeopacity{0.800000}%
\pgfsetdash{}{0pt}%
\pgfpathmoveto{\pgfqpoint{11.801668in}{4.072702in}}%
\pgfpathcurveto{\pgfqpoint{11.812718in}{4.072702in}}{\pgfqpoint{11.823317in}{4.077092in}}{\pgfqpoint{11.831131in}{4.084906in}}%
\pgfpathcurveto{\pgfqpoint{11.838944in}{4.092720in}}{\pgfqpoint{11.843335in}{4.103319in}}{\pgfqpoint{11.843335in}{4.114369in}}%
\pgfpathcurveto{\pgfqpoint{11.843335in}{4.125419in}}{\pgfqpoint{11.838944in}{4.136018in}}{\pgfqpoint{11.831131in}{4.143832in}}%
\pgfpathcurveto{\pgfqpoint{11.823317in}{4.151645in}}{\pgfqpoint{11.812718in}{4.156035in}}{\pgfqpoint{11.801668in}{4.156035in}}%
\pgfpathcurveto{\pgfqpoint{11.790618in}{4.156035in}}{\pgfqpoint{11.780019in}{4.151645in}}{\pgfqpoint{11.772205in}{4.143832in}}%
\pgfpathcurveto{\pgfqpoint{11.764392in}{4.136018in}}{\pgfqpoint{11.760001in}{4.125419in}}{\pgfqpoint{11.760001in}{4.114369in}}%
\pgfpathcurveto{\pgfqpoint{11.760001in}{4.103319in}}{\pgfqpoint{11.764392in}{4.092720in}}{\pgfqpoint{11.772205in}{4.084906in}}%
\pgfpathcurveto{\pgfqpoint{11.780019in}{4.077092in}}{\pgfqpoint{11.790618in}{4.072702in}}{\pgfqpoint{11.801668in}{4.072702in}}%
\pgfpathlineto{\pgfqpoint{11.801668in}{4.072702in}}%
\pgfpathclose%
\pgfusepath{stroke}%
\end{pgfscope}%
\begin{pgfscope}%
\pgfpathrectangle{\pgfqpoint{7.512535in}{0.437222in}}{\pgfqpoint{6.275590in}{5.159444in}}%
\pgfusepath{clip}%
\pgfsetbuttcap%
\pgfsetroundjoin%
\pgfsetlinewidth{1.003750pt}%
\definecolor{currentstroke}{rgb}{0.827451,0.827451,0.827451}%
\pgfsetstrokecolor{currentstroke}%
\pgfsetstrokeopacity{0.800000}%
\pgfsetdash{}{0pt}%
\pgfpathmoveto{\pgfqpoint{11.932188in}{4.030804in}}%
\pgfpathcurveto{\pgfqpoint{11.943238in}{4.030804in}}{\pgfqpoint{11.953837in}{4.035194in}}{\pgfqpoint{11.961651in}{4.043008in}}%
\pgfpathcurveto{\pgfqpoint{11.969464in}{4.050821in}}{\pgfqpoint{11.973855in}{4.061420in}}{\pgfqpoint{11.973855in}{4.072470in}}%
\pgfpathcurveto{\pgfqpoint{11.973855in}{4.083520in}}{\pgfqpoint{11.969464in}{4.094119in}}{\pgfqpoint{11.961651in}{4.101933in}}%
\pgfpathcurveto{\pgfqpoint{11.953837in}{4.109747in}}{\pgfqpoint{11.943238in}{4.114137in}}{\pgfqpoint{11.932188in}{4.114137in}}%
\pgfpathcurveto{\pgfqpoint{11.921138in}{4.114137in}}{\pgfqpoint{11.910539in}{4.109747in}}{\pgfqpoint{11.902725in}{4.101933in}}%
\pgfpathcurveto{\pgfqpoint{11.894912in}{4.094119in}}{\pgfqpoint{11.890521in}{4.083520in}}{\pgfqpoint{11.890521in}{4.072470in}}%
\pgfpathcurveto{\pgfqpoint{11.890521in}{4.061420in}}{\pgfqpoint{11.894912in}{4.050821in}}{\pgfqpoint{11.902725in}{4.043008in}}%
\pgfpathcurveto{\pgfqpoint{11.910539in}{4.035194in}}{\pgfqpoint{11.921138in}{4.030804in}}{\pgfqpoint{11.932188in}{4.030804in}}%
\pgfpathlineto{\pgfqpoint{11.932188in}{4.030804in}}%
\pgfpathclose%
\pgfusepath{stroke}%
\end{pgfscope}%
\begin{pgfscope}%
\pgfpathrectangle{\pgfqpoint{7.512535in}{0.437222in}}{\pgfqpoint{6.275590in}{5.159444in}}%
\pgfusepath{clip}%
\pgfsetbuttcap%
\pgfsetroundjoin%
\pgfsetlinewidth{1.003750pt}%
\definecolor{currentstroke}{rgb}{0.827451,0.827451,0.827451}%
\pgfsetstrokecolor{currentstroke}%
\pgfsetstrokeopacity{0.800000}%
\pgfsetdash{}{0pt}%
\pgfpathmoveto{\pgfqpoint{11.645781in}{3.328017in}}%
\pgfpathcurveto{\pgfqpoint{11.656832in}{3.328017in}}{\pgfqpoint{11.667431in}{3.332407in}}{\pgfqpoint{11.675244in}{3.340220in}}%
\pgfpathcurveto{\pgfqpoint{11.683058in}{3.348034in}}{\pgfqpoint{11.687448in}{3.358633in}}{\pgfqpoint{11.687448in}{3.369683in}}%
\pgfpathcurveto{\pgfqpoint{11.687448in}{3.380733in}}{\pgfqpoint{11.683058in}{3.391332in}}{\pgfqpoint{11.675244in}{3.399146in}}%
\pgfpathcurveto{\pgfqpoint{11.667431in}{3.406960in}}{\pgfqpoint{11.656832in}{3.411350in}}{\pgfqpoint{11.645781in}{3.411350in}}%
\pgfpathcurveto{\pgfqpoint{11.634731in}{3.411350in}}{\pgfqpoint{11.624132in}{3.406960in}}{\pgfqpoint{11.616319in}{3.399146in}}%
\pgfpathcurveto{\pgfqpoint{11.608505in}{3.391332in}}{\pgfqpoint{11.604115in}{3.380733in}}{\pgfqpoint{11.604115in}{3.369683in}}%
\pgfpathcurveto{\pgfqpoint{11.604115in}{3.358633in}}{\pgfqpoint{11.608505in}{3.348034in}}{\pgfqpoint{11.616319in}{3.340220in}}%
\pgfpathcurveto{\pgfqpoint{11.624132in}{3.332407in}}{\pgfqpoint{11.634731in}{3.328017in}}{\pgfqpoint{11.645781in}{3.328017in}}%
\pgfpathlineto{\pgfqpoint{11.645781in}{3.328017in}}%
\pgfpathclose%
\pgfusepath{stroke}%
\end{pgfscope}%
\begin{pgfscope}%
\pgfpathrectangle{\pgfqpoint{7.512535in}{0.437222in}}{\pgfqpoint{6.275590in}{5.159444in}}%
\pgfusepath{clip}%
\pgfsetbuttcap%
\pgfsetroundjoin%
\pgfsetlinewidth{1.003750pt}%
\definecolor{currentstroke}{rgb}{0.827451,0.827451,0.827451}%
\pgfsetstrokecolor{currentstroke}%
\pgfsetstrokeopacity{0.800000}%
\pgfsetdash{}{0pt}%
\pgfpathmoveto{\pgfqpoint{12.383527in}{4.432707in}}%
\pgfpathcurveto{\pgfqpoint{12.394577in}{4.432707in}}{\pgfqpoint{12.405176in}{4.437098in}}{\pgfqpoint{12.412990in}{4.444911in}}%
\pgfpathcurveto{\pgfqpoint{12.420803in}{4.452725in}}{\pgfqpoint{12.425194in}{4.463324in}}{\pgfqpoint{12.425194in}{4.474374in}}%
\pgfpathcurveto{\pgfqpoint{12.425194in}{4.485424in}}{\pgfqpoint{12.420803in}{4.496023in}}{\pgfqpoint{12.412990in}{4.503837in}}%
\pgfpathcurveto{\pgfqpoint{12.405176in}{4.511650in}}{\pgfqpoint{12.394577in}{4.516041in}}{\pgfqpoint{12.383527in}{4.516041in}}%
\pgfpathcurveto{\pgfqpoint{12.372477in}{4.516041in}}{\pgfqpoint{12.361878in}{4.511650in}}{\pgfqpoint{12.354064in}{4.503837in}}%
\pgfpathcurveto{\pgfqpoint{12.346251in}{4.496023in}}{\pgfqpoint{12.341860in}{4.485424in}}{\pgfqpoint{12.341860in}{4.474374in}}%
\pgfpathcurveto{\pgfqpoint{12.341860in}{4.463324in}}{\pgfqpoint{12.346251in}{4.452725in}}{\pgfqpoint{12.354064in}{4.444911in}}%
\pgfpathcurveto{\pgfqpoint{12.361878in}{4.437098in}}{\pgfqpoint{12.372477in}{4.432707in}}{\pgfqpoint{12.383527in}{4.432707in}}%
\pgfpathlineto{\pgfqpoint{12.383527in}{4.432707in}}%
\pgfpathclose%
\pgfusepath{stroke}%
\end{pgfscope}%
\begin{pgfscope}%
\pgfpathrectangle{\pgfqpoint{7.512535in}{0.437222in}}{\pgfqpoint{6.275590in}{5.159444in}}%
\pgfusepath{clip}%
\pgfsetbuttcap%
\pgfsetroundjoin%
\pgfsetlinewidth{1.003750pt}%
\definecolor{currentstroke}{rgb}{0.827451,0.827451,0.827451}%
\pgfsetstrokecolor{currentstroke}%
\pgfsetstrokeopacity{0.800000}%
\pgfsetdash{}{0pt}%
\pgfpathmoveto{\pgfqpoint{9.758049in}{2.980412in}}%
\pgfpathcurveto{\pgfqpoint{9.769099in}{2.980412in}}{\pgfqpoint{9.779698in}{2.984802in}}{\pgfqpoint{9.787512in}{2.992616in}}%
\pgfpathcurveto{\pgfqpoint{9.795326in}{3.000430in}}{\pgfqpoint{9.799716in}{3.011029in}}{\pgfqpoint{9.799716in}{3.022079in}}%
\pgfpathcurveto{\pgfqpoint{9.799716in}{3.033129in}}{\pgfqpoint{9.795326in}{3.043728in}}{\pgfqpoint{9.787512in}{3.051542in}}%
\pgfpathcurveto{\pgfqpoint{9.779698in}{3.059355in}}{\pgfqpoint{9.769099in}{3.063745in}}{\pgfqpoint{9.758049in}{3.063745in}}%
\pgfpathcurveto{\pgfqpoint{9.746999in}{3.063745in}}{\pgfqpoint{9.736400in}{3.059355in}}{\pgfqpoint{9.728587in}{3.051542in}}%
\pgfpathcurveto{\pgfqpoint{9.720773in}{3.043728in}}{\pgfqpoint{9.716383in}{3.033129in}}{\pgfqpoint{9.716383in}{3.022079in}}%
\pgfpathcurveto{\pgfqpoint{9.716383in}{3.011029in}}{\pgfqpoint{9.720773in}{3.000430in}}{\pgfqpoint{9.728587in}{2.992616in}}%
\pgfpathcurveto{\pgfqpoint{9.736400in}{2.984802in}}{\pgfqpoint{9.746999in}{2.980412in}}{\pgfqpoint{9.758049in}{2.980412in}}%
\pgfpathlineto{\pgfqpoint{9.758049in}{2.980412in}}%
\pgfpathclose%
\pgfusepath{stroke}%
\end{pgfscope}%
\begin{pgfscope}%
\pgfpathrectangle{\pgfqpoint{7.512535in}{0.437222in}}{\pgfqpoint{6.275590in}{5.159444in}}%
\pgfusepath{clip}%
\pgfsetbuttcap%
\pgfsetroundjoin%
\pgfsetlinewidth{1.003750pt}%
\definecolor{currentstroke}{rgb}{0.827451,0.827451,0.827451}%
\pgfsetstrokecolor{currentstroke}%
\pgfsetstrokeopacity{0.800000}%
\pgfsetdash{}{0pt}%
\pgfpathmoveto{\pgfqpoint{9.302653in}{3.619779in}}%
\pgfpathcurveto{\pgfqpoint{9.313703in}{3.619779in}}{\pgfqpoint{9.324302in}{3.624169in}}{\pgfqpoint{9.332116in}{3.631982in}}%
\pgfpathcurveto{\pgfqpoint{9.339929in}{3.639796in}}{\pgfqpoint{9.344320in}{3.650395in}}{\pgfqpoint{9.344320in}{3.661445in}}%
\pgfpathcurveto{\pgfqpoint{9.344320in}{3.672495in}}{\pgfqpoint{9.339929in}{3.683094in}}{\pgfqpoint{9.332116in}{3.690908in}}%
\pgfpathcurveto{\pgfqpoint{9.324302in}{3.698722in}}{\pgfqpoint{9.313703in}{3.703112in}}{\pgfqpoint{9.302653in}{3.703112in}}%
\pgfpathcurveto{\pgfqpoint{9.291603in}{3.703112in}}{\pgfqpoint{9.281004in}{3.698722in}}{\pgfqpoint{9.273190in}{3.690908in}}%
\pgfpathcurveto{\pgfqpoint{9.265376in}{3.683094in}}{\pgfqpoint{9.260986in}{3.672495in}}{\pgfqpoint{9.260986in}{3.661445in}}%
\pgfpathcurveto{\pgfqpoint{9.260986in}{3.650395in}}{\pgfqpoint{9.265376in}{3.639796in}}{\pgfqpoint{9.273190in}{3.631982in}}%
\pgfpathcurveto{\pgfqpoint{9.281004in}{3.624169in}}{\pgfqpoint{9.291603in}{3.619779in}}{\pgfqpoint{9.302653in}{3.619779in}}%
\pgfpathlineto{\pgfqpoint{9.302653in}{3.619779in}}%
\pgfpathclose%
\pgfusepath{stroke}%
\end{pgfscope}%
\begin{pgfscope}%
\pgfpathrectangle{\pgfqpoint{7.512535in}{0.437222in}}{\pgfqpoint{6.275590in}{5.159444in}}%
\pgfusepath{clip}%
\pgfsetbuttcap%
\pgfsetroundjoin%
\pgfsetlinewidth{1.003750pt}%
\definecolor{currentstroke}{rgb}{0.827451,0.827451,0.827451}%
\pgfsetstrokecolor{currentstroke}%
\pgfsetstrokeopacity{0.800000}%
\pgfsetdash{}{0pt}%
\pgfpathmoveto{\pgfqpoint{13.488123in}{4.987272in}}%
\pgfpathcurveto{\pgfqpoint{13.499173in}{4.987272in}}{\pgfqpoint{13.509772in}{4.991662in}}{\pgfqpoint{13.517586in}{4.999476in}}%
\pgfpathcurveto{\pgfqpoint{13.525399in}{5.007289in}}{\pgfqpoint{13.529790in}{5.017888in}}{\pgfqpoint{13.529790in}{5.028938in}}%
\pgfpathcurveto{\pgfqpoint{13.529790in}{5.039988in}}{\pgfqpoint{13.525399in}{5.050588in}}{\pgfqpoint{13.517586in}{5.058401in}}%
\pgfpathcurveto{\pgfqpoint{13.509772in}{5.066215in}}{\pgfqpoint{13.499173in}{5.070605in}}{\pgfqpoint{13.488123in}{5.070605in}}%
\pgfpathcurveto{\pgfqpoint{13.477073in}{5.070605in}}{\pgfqpoint{13.466474in}{5.066215in}}{\pgfqpoint{13.458660in}{5.058401in}}%
\pgfpathcurveto{\pgfqpoint{13.450847in}{5.050588in}}{\pgfqpoint{13.446456in}{5.039988in}}{\pgfqpoint{13.446456in}{5.028938in}}%
\pgfpathcurveto{\pgfqpoint{13.446456in}{5.017888in}}{\pgfqpoint{13.450847in}{5.007289in}}{\pgfqpoint{13.458660in}{4.999476in}}%
\pgfpathcurveto{\pgfqpoint{13.466474in}{4.991662in}}{\pgfqpoint{13.477073in}{4.987272in}}{\pgfqpoint{13.488123in}{4.987272in}}%
\pgfpathlineto{\pgfqpoint{13.488123in}{4.987272in}}%
\pgfpathclose%
\pgfusepath{stroke}%
\end{pgfscope}%
\begin{pgfscope}%
\pgfpathrectangle{\pgfqpoint{7.512535in}{0.437222in}}{\pgfqpoint{6.275590in}{5.159444in}}%
\pgfusepath{clip}%
\pgfsetbuttcap%
\pgfsetroundjoin%
\pgfsetlinewidth{1.003750pt}%
\definecolor{currentstroke}{rgb}{0.827451,0.827451,0.827451}%
\pgfsetstrokecolor{currentstroke}%
\pgfsetstrokeopacity{0.800000}%
\pgfsetdash{}{0pt}%
\pgfpathmoveto{\pgfqpoint{9.540145in}{4.891615in}}%
\pgfpathcurveto{\pgfqpoint{9.551195in}{4.891615in}}{\pgfqpoint{9.561794in}{4.896006in}}{\pgfqpoint{9.569607in}{4.903819in}}%
\pgfpathcurveto{\pgfqpoint{9.577421in}{4.911633in}}{\pgfqpoint{9.581811in}{4.922232in}}{\pgfqpoint{9.581811in}{4.933282in}}%
\pgfpathcurveto{\pgfqpoint{9.581811in}{4.944332in}}{\pgfqpoint{9.577421in}{4.954931in}}{\pgfqpoint{9.569607in}{4.962745in}}%
\pgfpathcurveto{\pgfqpoint{9.561794in}{4.970558in}}{\pgfqpoint{9.551195in}{4.974949in}}{\pgfqpoint{9.540145in}{4.974949in}}%
\pgfpathcurveto{\pgfqpoint{9.529095in}{4.974949in}}{\pgfqpoint{9.518495in}{4.970558in}}{\pgfqpoint{9.510682in}{4.962745in}}%
\pgfpathcurveto{\pgfqpoint{9.502868in}{4.954931in}}{\pgfqpoint{9.498478in}{4.944332in}}{\pgfqpoint{9.498478in}{4.933282in}}%
\pgfpathcurveto{\pgfqpoint{9.498478in}{4.922232in}}{\pgfqpoint{9.502868in}{4.911633in}}{\pgfqpoint{9.510682in}{4.903819in}}%
\pgfpathcurveto{\pgfqpoint{9.518495in}{4.896006in}}{\pgfqpoint{9.529095in}{4.891615in}}{\pgfqpoint{9.540145in}{4.891615in}}%
\pgfpathlineto{\pgfqpoint{9.540145in}{4.891615in}}%
\pgfpathclose%
\pgfusepath{stroke}%
\end{pgfscope}%
\begin{pgfscope}%
\pgfpathrectangle{\pgfqpoint{7.512535in}{0.437222in}}{\pgfqpoint{6.275590in}{5.159444in}}%
\pgfusepath{clip}%
\pgfsetbuttcap%
\pgfsetroundjoin%
\pgfsetlinewidth{1.003750pt}%
\definecolor{currentstroke}{rgb}{0.827451,0.827451,0.827451}%
\pgfsetstrokecolor{currentstroke}%
\pgfsetstrokeopacity{0.800000}%
\pgfsetdash{}{0pt}%
\pgfpathmoveto{\pgfqpoint{8.951017in}{5.038876in}}%
\pgfpathcurveto{\pgfqpoint{8.962067in}{5.038876in}}{\pgfqpoint{8.972666in}{5.043266in}}{\pgfqpoint{8.980479in}{5.051080in}}%
\pgfpathcurveto{\pgfqpoint{8.988293in}{5.058893in}}{\pgfqpoint{8.992683in}{5.069493in}}{\pgfqpoint{8.992683in}{5.080543in}}%
\pgfpathcurveto{\pgfqpoint{8.992683in}{5.091593in}}{\pgfqpoint{8.988293in}{5.102192in}}{\pgfqpoint{8.980479in}{5.110005in}}%
\pgfpathcurveto{\pgfqpoint{8.972666in}{5.117819in}}{\pgfqpoint{8.962067in}{5.122209in}}{\pgfqpoint{8.951017in}{5.122209in}}%
\pgfpathcurveto{\pgfqpoint{8.939966in}{5.122209in}}{\pgfqpoint{8.929367in}{5.117819in}}{\pgfqpoint{8.921554in}{5.110005in}}%
\pgfpathcurveto{\pgfqpoint{8.913740in}{5.102192in}}{\pgfqpoint{8.909350in}{5.091593in}}{\pgfqpoint{8.909350in}{5.080543in}}%
\pgfpathcurveto{\pgfqpoint{8.909350in}{5.069493in}}{\pgfqpoint{8.913740in}{5.058893in}}{\pgfqpoint{8.921554in}{5.051080in}}%
\pgfpathcurveto{\pgfqpoint{8.929367in}{5.043266in}}{\pgfqpoint{8.939966in}{5.038876in}}{\pgfqpoint{8.951017in}{5.038876in}}%
\pgfpathlineto{\pgfqpoint{8.951017in}{5.038876in}}%
\pgfpathclose%
\pgfusepath{stroke}%
\end{pgfscope}%
\begin{pgfscope}%
\pgfpathrectangle{\pgfqpoint{7.512535in}{0.437222in}}{\pgfqpoint{6.275590in}{5.159444in}}%
\pgfusepath{clip}%
\pgfsetbuttcap%
\pgfsetroundjoin%
\pgfsetlinewidth{1.003750pt}%
\definecolor{currentstroke}{rgb}{0.827451,0.827451,0.827451}%
\pgfsetstrokecolor{currentstroke}%
\pgfsetstrokeopacity{0.800000}%
\pgfsetdash{}{0pt}%
\pgfpathmoveto{\pgfqpoint{10.979603in}{1.804307in}}%
\pgfpathcurveto{\pgfqpoint{10.990654in}{1.804307in}}{\pgfqpoint{11.001253in}{1.808697in}}{\pgfqpoint{11.009066in}{1.816511in}}%
\pgfpathcurveto{\pgfqpoint{11.016880in}{1.824325in}}{\pgfqpoint{11.021270in}{1.834924in}}{\pgfqpoint{11.021270in}{1.845974in}}%
\pgfpathcurveto{\pgfqpoint{11.021270in}{1.857024in}}{\pgfqpoint{11.016880in}{1.867623in}}{\pgfqpoint{11.009066in}{1.875437in}}%
\pgfpathcurveto{\pgfqpoint{11.001253in}{1.883250in}}{\pgfqpoint{10.990654in}{1.887640in}}{\pgfqpoint{10.979603in}{1.887640in}}%
\pgfpathcurveto{\pgfqpoint{10.968553in}{1.887640in}}{\pgfqpoint{10.957954in}{1.883250in}}{\pgfqpoint{10.950141in}{1.875437in}}%
\pgfpathcurveto{\pgfqpoint{10.942327in}{1.867623in}}{\pgfqpoint{10.937937in}{1.857024in}}{\pgfqpoint{10.937937in}{1.845974in}}%
\pgfpathcurveto{\pgfqpoint{10.937937in}{1.834924in}}{\pgfqpoint{10.942327in}{1.824325in}}{\pgfqpoint{10.950141in}{1.816511in}}%
\pgfpathcurveto{\pgfqpoint{10.957954in}{1.808697in}}{\pgfqpoint{10.968553in}{1.804307in}}{\pgfqpoint{10.979603in}{1.804307in}}%
\pgfpathlineto{\pgfqpoint{10.979603in}{1.804307in}}%
\pgfpathclose%
\pgfusepath{stroke}%
\end{pgfscope}%
\begin{pgfscope}%
\pgfpathrectangle{\pgfqpoint{7.512535in}{0.437222in}}{\pgfqpoint{6.275590in}{5.159444in}}%
\pgfusepath{clip}%
\pgfsetbuttcap%
\pgfsetroundjoin%
\pgfsetlinewidth{1.003750pt}%
\definecolor{currentstroke}{rgb}{0.827451,0.827451,0.827451}%
\pgfsetstrokecolor{currentstroke}%
\pgfsetstrokeopacity{0.800000}%
\pgfsetdash{}{0pt}%
\pgfpathmoveto{\pgfqpoint{10.933498in}{2.723776in}}%
\pgfpathcurveto{\pgfqpoint{10.944548in}{2.723776in}}{\pgfqpoint{10.955147in}{2.728166in}}{\pgfqpoint{10.962961in}{2.735980in}}%
\pgfpathcurveto{\pgfqpoint{10.970774in}{2.743793in}}{\pgfqpoint{10.975165in}{2.754392in}}{\pgfqpoint{10.975165in}{2.765442in}}%
\pgfpathcurveto{\pgfqpoint{10.975165in}{2.776493in}}{\pgfqpoint{10.970774in}{2.787092in}}{\pgfqpoint{10.962961in}{2.794905in}}%
\pgfpathcurveto{\pgfqpoint{10.955147in}{2.802719in}}{\pgfqpoint{10.944548in}{2.807109in}}{\pgfqpoint{10.933498in}{2.807109in}}%
\pgfpathcurveto{\pgfqpoint{10.922448in}{2.807109in}}{\pgfqpoint{10.911849in}{2.802719in}}{\pgfqpoint{10.904035in}{2.794905in}}%
\pgfpathcurveto{\pgfqpoint{10.896222in}{2.787092in}}{\pgfqpoint{10.891831in}{2.776493in}}{\pgfqpoint{10.891831in}{2.765442in}}%
\pgfpathcurveto{\pgfqpoint{10.891831in}{2.754392in}}{\pgfqpoint{10.896222in}{2.743793in}}{\pgfqpoint{10.904035in}{2.735980in}}%
\pgfpathcurveto{\pgfqpoint{10.911849in}{2.728166in}}{\pgfqpoint{10.922448in}{2.723776in}}{\pgfqpoint{10.933498in}{2.723776in}}%
\pgfpathlineto{\pgfqpoint{10.933498in}{2.723776in}}%
\pgfpathclose%
\pgfusepath{stroke}%
\end{pgfscope}%
\begin{pgfscope}%
\pgfpathrectangle{\pgfqpoint{7.512535in}{0.437222in}}{\pgfqpoint{6.275590in}{5.159444in}}%
\pgfusepath{clip}%
\pgfsetbuttcap%
\pgfsetroundjoin%
\pgfsetlinewidth{1.003750pt}%
\definecolor{currentstroke}{rgb}{0.827451,0.827451,0.827451}%
\pgfsetstrokecolor{currentstroke}%
\pgfsetstrokeopacity{0.800000}%
\pgfsetdash{}{0pt}%
\pgfpathmoveto{\pgfqpoint{9.116910in}{0.954383in}}%
\pgfpathcurveto{\pgfqpoint{9.127960in}{0.954383in}}{\pgfqpoint{9.138559in}{0.958773in}}{\pgfqpoint{9.146373in}{0.966587in}}%
\pgfpathcurveto{\pgfqpoint{9.154186in}{0.974400in}}{\pgfqpoint{9.158577in}{0.984999in}}{\pgfqpoint{9.158577in}{0.996049in}}%
\pgfpathcurveto{\pgfqpoint{9.158577in}{1.007099in}}{\pgfqpoint{9.154186in}{1.017699in}}{\pgfqpoint{9.146373in}{1.025512in}}%
\pgfpathcurveto{\pgfqpoint{9.138559in}{1.033326in}}{\pgfqpoint{9.127960in}{1.037716in}}{\pgfqpoint{9.116910in}{1.037716in}}%
\pgfpathcurveto{\pgfqpoint{9.105860in}{1.037716in}}{\pgfqpoint{9.095261in}{1.033326in}}{\pgfqpoint{9.087447in}{1.025512in}}%
\pgfpathcurveto{\pgfqpoint{9.079634in}{1.017699in}}{\pgfqpoint{9.075243in}{1.007099in}}{\pgfqpoint{9.075243in}{0.996049in}}%
\pgfpathcurveto{\pgfqpoint{9.075243in}{0.984999in}}{\pgfqpoint{9.079634in}{0.974400in}}{\pgfqpoint{9.087447in}{0.966587in}}%
\pgfpathcurveto{\pgfqpoint{9.095261in}{0.958773in}}{\pgfqpoint{9.105860in}{0.954383in}}{\pgfqpoint{9.116910in}{0.954383in}}%
\pgfpathlineto{\pgfqpoint{9.116910in}{0.954383in}}%
\pgfpathclose%
\pgfusepath{stroke}%
\end{pgfscope}%
\begin{pgfscope}%
\pgfpathrectangle{\pgfqpoint{7.512535in}{0.437222in}}{\pgfqpoint{6.275590in}{5.159444in}}%
\pgfusepath{clip}%
\pgfsetbuttcap%
\pgfsetroundjoin%
\pgfsetlinewidth{1.003750pt}%
\definecolor{currentstroke}{rgb}{0.827451,0.827451,0.827451}%
\pgfsetstrokecolor{currentstroke}%
\pgfsetstrokeopacity{0.800000}%
\pgfsetdash{}{0pt}%
\pgfpathmoveto{\pgfqpoint{13.529582in}{4.481651in}}%
\pgfpathcurveto{\pgfqpoint{13.540632in}{4.481651in}}{\pgfqpoint{13.551231in}{4.486041in}}{\pgfqpoint{13.559045in}{4.493854in}}%
\pgfpathcurveto{\pgfqpoint{13.566858in}{4.501668in}}{\pgfqpoint{13.571248in}{4.512267in}}{\pgfqpoint{13.571248in}{4.523317in}}%
\pgfpathcurveto{\pgfqpoint{13.571248in}{4.534367in}}{\pgfqpoint{13.566858in}{4.544966in}}{\pgfqpoint{13.559045in}{4.552780in}}%
\pgfpathcurveto{\pgfqpoint{13.551231in}{4.560594in}}{\pgfqpoint{13.540632in}{4.564984in}}{\pgfqpoint{13.529582in}{4.564984in}}%
\pgfpathcurveto{\pgfqpoint{13.518532in}{4.564984in}}{\pgfqpoint{13.507933in}{4.560594in}}{\pgfqpoint{13.500119in}{4.552780in}}%
\pgfpathcurveto{\pgfqpoint{13.492305in}{4.544966in}}{\pgfqpoint{13.487915in}{4.534367in}}{\pgfqpoint{13.487915in}{4.523317in}}%
\pgfpathcurveto{\pgfqpoint{13.487915in}{4.512267in}}{\pgfqpoint{13.492305in}{4.501668in}}{\pgfqpoint{13.500119in}{4.493854in}}%
\pgfpathcurveto{\pgfqpoint{13.507933in}{4.486041in}}{\pgfqpoint{13.518532in}{4.481651in}}{\pgfqpoint{13.529582in}{4.481651in}}%
\pgfpathlineto{\pgfqpoint{13.529582in}{4.481651in}}%
\pgfpathclose%
\pgfusepath{stroke}%
\end{pgfscope}%
\begin{pgfscope}%
\pgfpathrectangle{\pgfqpoint{7.512535in}{0.437222in}}{\pgfqpoint{6.275590in}{5.159444in}}%
\pgfusepath{clip}%
\pgfsetbuttcap%
\pgfsetroundjoin%
\pgfsetlinewidth{1.003750pt}%
\definecolor{currentstroke}{rgb}{0.827451,0.827451,0.827451}%
\pgfsetstrokecolor{currentstroke}%
\pgfsetstrokeopacity{0.800000}%
\pgfsetdash{}{0pt}%
\pgfpathmoveto{\pgfqpoint{11.489538in}{2.656991in}}%
\pgfpathcurveto{\pgfqpoint{11.500588in}{2.656991in}}{\pgfqpoint{11.511187in}{2.661381in}}{\pgfqpoint{11.519001in}{2.669195in}}%
\pgfpathcurveto{\pgfqpoint{11.526814in}{2.677009in}}{\pgfqpoint{11.531205in}{2.687608in}}{\pgfqpoint{11.531205in}{2.698658in}}%
\pgfpathcurveto{\pgfqpoint{11.531205in}{2.709708in}}{\pgfqpoint{11.526814in}{2.720307in}}{\pgfqpoint{11.519001in}{2.728120in}}%
\pgfpathcurveto{\pgfqpoint{11.511187in}{2.735934in}}{\pgfqpoint{11.500588in}{2.740324in}}{\pgfqpoint{11.489538in}{2.740324in}}%
\pgfpathcurveto{\pgfqpoint{11.478488in}{2.740324in}}{\pgfqpoint{11.467889in}{2.735934in}}{\pgfqpoint{11.460075in}{2.728120in}}%
\pgfpathcurveto{\pgfqpoint{11.452262in}{2.720307in}}{\pgfqpoint{11.447871in}{2.709708in}}{\pgfqpoint{11.447871in}{2.698658in}}%
\pgfpathcurveto{\pgfqpoint{11.447871in}{2.687608in}}{\pgfqpoint{11.452262in}{2.677009in}}{\pgfqpoint{11.460075in}{2.669195in}}%
\pgfpathcurveto{\pgfqpoint{11.467889in}{2.661381in}}{\pgfqpoint{11.478488in}{2.656991in}}{\pgfqpoint{11.489538in}{2.656991in}}%
\pgfpathlineto{\pgfqpoint{11.489538in}{2.656991in}}%
\pgfpathclose%
\pgfusepath{stroke}%
\end{pgfscope}%
\begin{pgfscope}%
\pgfpathrectangle{\pgfqpoint{7.512535in}{0.437222in}}{\pgfqpoint{6.275590in}{5.159444in}}%
\pgfusepath{clip}%
\pgfsetbuttcap%
\pgfsetroundjoin%
\pgfsetlinewidth{1.003750pt}%
\definecolor{currentstroke}{rgb}{0.827451,0.827451,0.827451}%
\pgfsetstrokecolor{currentstroke}%
\pgfsetstrokeopacity{0.800000}%
\pgfsetdash{}{0pt}%
\pgfpathmoveto{\pgfqpoint{11.420825in}{2.942684in}}%
\pgfpathcurveto{\pgfqpoint{11.431876in}{2.942684in}}{\pgfqpoint{11.442475in}{2.947074in}}{\pgfqpoint{11.450288in}{2.954888in}}%
\pgfpathcurveto{\pgfqpoint{11.458102in}{2.962701in}}{\pgfqpoint{11.462492in}{2.973300in}}{\pgfqpoint{11.462492in}{2.984350in}}%
\pgfpathcurveto{\pgfqpoint{11.462492in}{2.995401in}}{\pgfqpoint{11.458102in}{3.006000in}}{\pgfqpoint{11.450288in}{3.013813in}}%
\pgfpathcurveto{\pgfqpoint{11.442475in}{3.021627in}}{\pgfqpoint{11.431876in}{3.026017in}}{\pgfqpoint{11.420825in}{3.026017in}}%
\pgfpathcurveto{\pgfqpoint{11.409775in}{3.026017in}}{\pgfqpoint{11.399176in}{3.021627in}}{\pgfqpoint{11.391363in}{3.013813in}}%
\pgfpathcurveto{\pgfqpoint{11.383549in}{3.006000in}}{\pgfqpoint{11.379159in}{2.995401in}}{\pgfqpoint{11.379159in}{2.984350in}}%
\pgfpathcurveto{\pgfqpoint{11.379159in}{2.973300in}}{\pgfqpoint{11.383549in}{2.962701in}}{\pgfqpoint{11.391363in}{2.954888in}}%
\pgfpathcurveto{\pgfqpoint{11.399176in}{2.947074in}}{\pgfqpoint{11.409775in}{2.942684in}}{\pgfqpoint{11.420825in}{2.942684in}}%
\pgfpathlineto{\pgfqpoint{11.420825in}{2.942684in}}%
\pgfpathclose%
\pgfusepath{stroke}%
\end{pgfscope}%
\begin{pgfscope}%
\pgfpathrectangle{\pgfqpoint{7.512535in}{0.437222in}}{\pgfqpoint{6.275590in}{5.159444in}}%
\pgfusepath{clip}%
\pgfsetbuttcap%
\pgfsetroundjoin%
\pgfsetlinewidth{1.003750pt}%
\definecolor{currentstroke}{rgb}{0.827451,0.827451,0.827451}%
\pgfsetstrokecolor{currentstroke}%
\pgfsetstrokeopacity{0.800000}%
\pgfsetdash{}{0pt}%
\pgfpathmoveto{\pgfqpoint{9.911802in}{0.673514in}}%
\pgfpathcurveto{\pgfqpoint{9.922853in}{0.673514in}}{\pgfqpoint{9.933452in}{0.677904in}}{\pgfqpoint{9.941265in}{0.685718in}}%
\pgfpathcurveto{\pgfqpoint{9.949079in}{0.693531in}}{\pgfqpoint{9.953469in}{0.704130in}}{\pgfqpoint{9.953469in}{0.715181in}}%
\pgfpathcurveto{\pgfqpoint{9.953469in}{0.726231in}}{\pgfqpoint{9.949079in}{0.736830in}}{\pgfqpoint{9.941265in}{0.744643in}}%
\pgfpathcurveto{\pgfqpoint{9.933452in}{0.752457in}}{\pgfqpoint{9.922853in}{0.756847in}}{\pgfqpoint{9.911802in}{0.756847in}}%
\pgfpathcurveto{\pgfqpoint{9.900752in}{0.756847in}}{\pgfqpoint{9.890153in}{0.752457in}}{\pgfqpoint{9.882340in}{0.744643in}}%
\pgfpathcurveto{\pgfqpoint{9.874526in}{0.736830in}}{\pgfqpoint{9.870136in}{0.726231in}}{\pgfqpoint{9.870136in}{0.715181in}}%
\pgfpathcurveto{\pgfqpoint{9.870136in}{0.704130in}}{\pgfqpoint{9.874526in}{0.693531in}}{\pgfqpoint{9.882340in}{0.685718in}}%
\pgfpathcurveto{\pgfqpoint{9.890153in}{0.677904in}}{\pgfqpoint{9.900752in}{0.673514in}}{\pgfqpoint{9.911802in}{0.673514in}}%
\pgfpathlineto{\pgfqpoint{9.911802in}{0.673514in}}%
\pgfpathclose%
\pgfusepath{stroke}%
\end{pgfscope}%
\begin{pgfscope}%
\pgfpathrectangle{\pgfqpoint{7.512535in}{0.437222in}}{\pgfqpoint{6.275590in}{5.159444in}}%
\pgfusepath{clip}%
\pgfsetbuttcap%
\pgfsetroundjoin%
\pgfsetlinewidth{1.003750pt}%
\definecolor{currentstroke}{rgb}{0.827451,0.827451,0.827451}%
\pgfsetstrokecolor{currentstroke}%
\pgfsetstrokeopacity{0.800000}%
\pgfsetdash{}{0pt}%
\pgfpathmoveto{\pgfqpoint{10.298326in}{0.555678in}}%
\pgfpathcurveto{\pgfqpoint{10.309376in}{0.555678in}}{\pgfqpoint{10.319975in}{0.560068in}}{\pgfqpoint{10.327788in}{0.567882in}}%
\pgfpathcurveto{\pgfqpoint{10.335602in}{0.575696in}}{\pgfqpoint{10.339992in}{0.586295in}}{\pgfqpoint{10.339992in}{0.597345in}}%
\pgfpathcurveto{\pgfqpoint{10.339992in}{0.608395in}}{\pgfqpoint{10.335602in}{0.618994in}}{\pgfqpoint{10.327788in}{0.626808in}}%
\pgfpathcurveto{\pgfqpoint{10.319975in}{0.634621in}}{\pgfqpoint{10.309376in}{0.639012in}}{\pgfqpoint{10.298326in}{0.639012in}}%
\pgfpathcurveto{\pgfqpoint{10.287275in}{0.639012in}}{\pgfqpoint{10.276676in}{0.634621in}}{\pgfqpoint{10.268863in}{0.626808in}}%
\pgfpathcurveto{\pgfqpoint{10.261049in}{0.618994in}}{\pgfqpoint{10.256659in}{0.608395in}}{\pgfqpoint{10.256659in}{0.597345in}}%
\pgfpathcurveto{\pgfqpoint{10.256659in}{0.586295in}}{\pgfqpoint{10.261049in}{0.575696in}}{\pgfqpoint{10.268863in}{0.567882in}}%
\pgfpathcurveto{\pgfqpoint{10.276676in}{0.560068in}}{\pgfqpoint{10.287275in}{0.555678in}}{\pgfqpoint{10.298326in}{0.555678in}}%
\pgfpathlineto{\pgfqpoint{10.298326in}{0.555678in}}%
\pgfpathclose%
\pgfusepath{stroke}%
\end{pgfscope}%
\begin{pgfscope}%
\pgfpathrectangle{\pgfqpoint{7.512535in}{0.437222in}}{\pgfqpoint{6.275590in}{5.159444in}}%
\pgfusepath{clip}%
\pgfsetbuttcap%
\pgfsetroundjoin%
\pgfsetlinewidth{1.003750pt}%
\definecolor{currentstroke}{rgb}{0.827451,0.827451,0.827451}%
\pgfsetstrokecolor{currentstroke}%
\pgfsetstrokeopacity{0.800000}%
\pgfsetdash{}{0pt}%
\pgfpathmoveto{\pgfqpoint{12.147203in}{3.425716in}}%
\pgfpathcurveto{\pgfqpoint{12.158253in}{3.425716in}}{\pgfqpoint{12.168852in}{3.430106in}}{\pgfqpoint{12.176666in}{3.437920in}}%
\pgfpathcurveto{\pgfqpoint{12.184479in}{3.445734in}}{\pgfqpoint{12.188869in}{3.456333in}}{\pgfqpoint{12.188869in}{3.467383in}}%
\pgfpathcurveto{\pgfqpoint{12.188869in}{3.478433in}}{\pgfqpoint{12.184479in}{3.489032in}}{\pgfqpoint{12.176666in}{3.496845in}}%
\pgfpathcurveto{\pgfqpoint{12.168852in}{3.504659in}}{\pgfqpoint{12.158253in}{3.509049in}}{\pgfqpoint{12.147203in}{3.509049in}}%
\pgfpathcurveto{\pgfqpoint{12.136153in}{3.509049in}}{\pgfqpoint{12.125554in}{3.504659in}}{\pgfqpoint{12.117740in}{3.496845in}}%
\pgfpathcurveto{\pgfqpoint{12.109926in}{3.489032in}}{\pgfqpoint{12.105536in}{3.478433in}}{\pgfqpoint{12.105536in}{3.467383in}}%
\pgfpathcurveto{\pgfqpoint{12.105536in}{3.456333in}}{\pgfqpoint{12.109926in}{3.445734in}}{\pgfqpoint{12.117740in}{3.437920in}}%
\pgfpathcurveto{\pgfqpoint{12.125554in}{3.430106in}}{\pgfqpoint{12.136153in}{3.425716in}}{\pgfqpoint{12.147203in}{3.425716in}}%
\pgfpathlineto{\pgfqpoint{12.147203in}{3.425716in}}%
\pgfpathclose%
\pgfusepath{stroke}%
\end{pgfscope}%
\begin{pgfscope}%
\pgfpathrectangle{\pgfqpoint{7.512535in}{0.437222in}}{\pgfqpoint{6.275590in}{5.159444in}}%
\pgfusepath{clip}%
\pgfsetbuttcap%
\pgfsetroundjoin%
\pgfsetlinewidth{1.003750pt}%
\definecolor{currentstroke}{rgb}{0.827451,0.827451,0.827451}%
\pgfsetstrokecolor{currentstroke}%
\pgfsetstrokeopacity{0.800000}%
\pgfsetdash{}{0pt}%
\pgfpathmoveto{\pgfqpoint{12.402247in}{3.486861in}}%
\pgfpathcurveto{\pgfqpoint{12.413298in}{3.486861in}}{\pgfqpoint{12.423897in}{3.491252in}}{\pgfqpoint{12.431710in}{3.499065in}}%
\pgfpathcurveto{\pgfqpoint{12.439524in}{3.506879in}}{\pgfqpoint{12.443914in}{3.517478in}}{\pgfqpoint{12.443914in}{3.528528in}}%
\pgfpathcurveto{\pgfqpoint{12.443914in}{3.539578in}}{\pgfqpoint{12.439524in}{3.550177in}}{\pgfqpoint{12.431710in}{3.557991in}}%
\pgfpathcurveto{\pgfqpoint{12.423897in}{3.565804in}}{\pgfqpoint{12.413298in}{3.570195in}}{\pgfqpoint{12.402247in}{3.570195in}}%
\pgfpathcurveto{\pgfqpoint{12.391197in}{3.570195in}}{\pgfqpoint{12.380598in}{3.565804in}}{\pgfqpoint{12.372785in}{3.557991in}}%
\pgfpathcurveto{\pgfqpoint{12.364971in}{3.550177in}}{\pgfqpoint{12.360581in}{3.539578in}}{\pgfqpoint{12.360581in}{3.528528in}}%
\pgfpathcurveto{\pgfqpoint{12.360581in}{3.517478in}}{\pgfqpoint{12.364971in}{3.506879in}}{\pgfqpoint{12.372785in}{3.499065in}}%
\pgfpathcurveto{\pgfqpoint{12.380598in}{3.491252in}}{\pgfqpoint{12.391197in}{3.486861in}}{\pgfqpoint{12.402247in}{3.486861in}}%
\pgfpathlineto{\pgfqpoint{12.402247in}{3.486861in}}%
\pgfpathclose%
\pgfusepath{stroke}%
\end{pgfscope}%
\begin{pgfscope}%
\pgfpathrectangle{\pgfqpoint{7.512535in}{0.437222in}}{\pgfqpoint{6.275590in}{5.159444in}}%
\pgfusepath{clip}%
\pgfsetbuttcap%
\pgfsetroundjoin%
\pgfsetlinewidth{1.003750pt}%
\definecolor{currentstroke}{rgb}{0.827451,0.827451,0.827451}%
\pgfsetstrokecolor{currentstroke}%
\pgfsetstrokeopacity{0.800000}%
\pgfsetdash{}{0pt}%
\pgfpathmoveto{\pgfqpoint{13.171837in}{4.040848in}}%
\pgfpathcurveto{\pgfqpoint{13.182887in}{4.040848in}}{\pgfqpoint{13.193486in}{4.045239in}}{\pgfqpoint{13.201300in}{4.053052in}}%
\pgfpathcurveto{\pgfqpoint{13.209114in}{4.060866in}}{\pgfqpoint{13.213504in}{4.071465in}}{\pgfqpoint{13.213504in}{4.082515in}}%
\pgfpathcurveto{\pgfqpoint{13.213504in}{4.093565in}}{\pgfqpoint{13.209114in}{4.104164in}}{\pgfqpoint{13.201300in}{4.111978in}}%
\pgfpathcurveto{\pgfqpoint{13.193486in}{4.119791in}}{\pgfqpoint{13.182887in}{4.124182in}}{\pgfqpoint{13.171837in}{4.124182in}}%
\pgfpathcurveto{\pgfqpoint{13.160787in}{4.124182in}}{\pgfqpoint{13.150188in}{4.119791in}}{\pgfqpoint{13.142374in}{4.111978in}}%
\pgfpathcurveto{\pgfqpoint{13.134561in}{4.104164in}}{\pgfqpoint{13.130171in}{4.093565in}}{\pgfqpoint{13.130171in}{4.082515in}}%
\pgfpathcurveto{\pgfqpoint{13.130171in}{4.071465in}}{\pgfqpoint{13.134561in}{4.060866in}}{\pgfqpoint{13.142374in}{4.053052in}}%
\pgfpathcurveto{\pgfqpoint{13.150188in}{4.045239in}}{\pgfqpoint{13.160787in}{4.040848in}}{\pgfqpoint{13.171837in}{4.040848in}}%
\pgfpathlineto{\pgfqpoint{13.171837in}{4.040848in}}%
\pgfpathclose%
\pgfusepath{stroke}%
\end{pgfscope}%
\begin{pgfscope}%
\pgfpathrectangle{\pgfqpoint{7.512535in}{0.437222in}}{\pgfqpoint{6.275590in}{5.159444in}}%
\pgfusepath{clip}%
\pgfsetbuttcap%
\pgfsetroundjoin%
\pgfsetlinewidth{1.003750pt}%
\definecolor{currentstroke}{rgb}{0.827451,0.827451,0.827451}%
\pgfsetstrokecolor{currentstroke}%
\pgfsetstrokeopacity{0.800000}%
\pgfsetdash{}{0pt}%
\pgfpathmoveto{\pgfqpoint{12.787405in}{4.069580in}}%
\pgfpathcurveto{\pgfqpoint{12.798455in}{4.069580in}}{\pgfqpoint{12.809054in}{4.073971in}}{\pgfqpoint{12.816868in}{4.081784in}}%
\pgfpathcurveto{\pgfqpoint{12.824681in}{4.089598in}}{\pgfqpoint{12.829072in}{4.100197in}}{\pgfqpoint{12.829072in}{4.111247in}}%
\pgfpathcurveto{\pgfqpoint{12.829072in}{4.122297in}}{\pgfqpoint{12.824681in}{4.132896in}}{\pgfqpoint{12.816868in}{4.140710in}}%
\pgfpathcurveto{\pgfqpoint{12.809054in}{4.148523in}}{\pgfqpoint{12.798455in}{4.152914in}}{\pgfqpoint{12.787405in}{4.152914in}}%
\pgfpathcurveto{\pgfqpoint{12.776355in}{4.152914in}}{\pgfqpoint{12.765756in}{4.148523in}}{\pgfqpoint{12.757942in}{4.140710in}}%
\pgfpathcurveto{\pgfqpoint{12.750129in}{4.132896in}}{\pgfqpoint{12.745738in}{4.122297in}}{\pgfqpoint{12.745738in}{4.111247in}}%
\pgfpathcurveto{\pgfqpoint{12.745738in}{4.100197in}}{\pgfqpoint{12.750129in}{4.089598in}}{\pgfqpoint{12.757942in}{4.081784in}}%
\pgfpathcurveto{\pgfqpoint{12.765756in}{4.073971in}}{\pgfqpoint{12.776355in}{4.069580in}}{\pgfqpoint{12.787405in}{4.069580in}}%
\pgfpathlineto{\pgfqpoint{12.787405in}{4.069580in}}%
\pgfpathclose%
\pgfusepath{stroke}%
\end{pgfscope}%
\begin{pgfscope}%
\pgfpathrectangle{\pgfqpoint{7.512535in}{0.437222in}}{\pgfqpoint{6.275590in}{5.159444in}}%
\pgfusepath{clip}%
\pgfsetbuttcap%
\pgfsetroundjoin%
\pgfsetlinewidth{1.003750pt}%
\definecolor{currentstroke}{rgb}{0.827451,0.827451,0.827451}%
\pgfsetstrokecolor{currentstroke}%
\pgfsetstrokeopacity{0.800000}%
\pgfsetdash{}{0pt}%
\pgfpathmoveto{\pgfqpoint{8.865766in}{4.187464in}}%
\pgfpathcurveto{\pgfqpoint{8.876816in}{4.187464in}}{\pgfqpoint{8.887415in}{4.191855in}}{\pgfqpoint{8.895229in}{4.199668in}}%
\pgfpathcurveto{\pgfqpoint{8.903042in}{4.207482in}}{\pgfqpoint{8.907432in}{4.218081in}}{\pgfqpoint{8.907432in}{4.229131in}}%
\pgfpathcurveto{\pgfqpoint{8.907432in}{4.240181in}}{\pgfqpoint{8.903042in}{4.250780in}}{\pgfqpoint{8.895229in}{4.258594in}}%
\pgfpathcurveto{\pgfqpoint{8.887415in}{4.266407in}}{\pgfqpoint{8.876816in}{4.270798in}}{\pgfqpoint{8.865766in}{4.270798in}}%
\pgfpathcurveto{\pgfqpoint{8.854716in}{4.270798in}}{\pgfqpoint{8.844117in}{4.266407in}}{\pgfqpoint{8.836303in}{4.258594in}}%
\pgfpathcurveto{\pgfqpoint{8.828489in}{4.250780in}}{\pgfqpoint{8.824099in}{4.240181in}}{\pgfqpoint{8.824099in}{4.229131in}}%
\pgfpathcurveto{\pgfqpoint{8.824099in}{4.218081in}}{\pgfqpoint{8.828489in}{4.207482in}}{\pgfqpoint{8.836303in}{4.199668in}}%
\pgfpathcurveto{\pgfqpoint{8.844117in}{4.191855in}}{\pgfqpoint{8.854716in}{4.187464in}}{\pgfqpoint{8.865766in}{4.187464in}}%
\pgfpathlineto{\pgfqpoint{8.865766in}{4.187464in}}%
\pgfpathclose%
\pgfusepath{stroke}%
\end{pgfscope}%
\begin{pgfscope}%
\pgfpathrectangle{\pgfqpoint{7.512535in}{0.437222in}}{\pgfqpoint{6.275590in}{5.159444in}}%
\pgfusepath{clip}%
\pgfsetbuttcap%
\pgfsetroundjoin%
\pgfsetlinewidth{1.003750pt}%
\definecolor{currentstroke}{rgb}{0.827451,0.827451,0.827451}%
\pgfsetstrokecolor{currentstroke}%
\pgfsetstrokeopacity{0.800000}%
\pgfsetdash{}{0pt}%
\pgfpathmoveto{\pgfqpoint{10.496996in}{2.172556in}}%
\pgfpathcurveto{\pgfqpoint{10.508046in}{2.172556in}}{\pgfqpoint{10.518645in}{2.176946in}}{\pgfqpoint{10.526458in}{2.184760in}}%
\pgfpathcurveto{\pgfqpoint{10.534272in}{2.192573in}}{\pgfqpoint{10.538662in}{2.203172in}}{\pgfqpoint{10.538662in}{2.214222in}}%
\pgfpathcurveto{\pgfqpoint{10.538662in}{2.225273in}}{\pgfqpoint{10.534272in}{2.235872in}}{\pgfqpoint{10.526458in}{2.243685in}}%
\pgfpathcurveto{\pgfqpoint{10.518645in}{2.251499in}}{\pgfqpoint{10.508046in}{2.255889in}}{\pgfqpoint{10.496996in}{2.255889in}}%
\pgfpathcurveto{\pgfqpoint{10.485946in}{2.255889in}}{\pgfqpoint{10.475347in}{2.251499in}}{\pgfqpoint{10.467533in}{2.243685in}}%
\pgfpathcurveto{\pgfqpoint{10.459719in}{2.235872in}}{\pgfqpoint{10.455329in}{2.225273in}}{\pgfqpoint{10.455329in}{2.214222in}}%
\pgfpathcurveto{\pgfqpoint{10.455329in}{2.203172in}}{\pgfqpoint{10.459719in}{2.192573in}}{\pgfqpoint{10.467533in}{2.184760in}}%
\pgfpathcurveto{\pgfqpoint{10.475347in}{2.176946in}}{\pgfqpoint{10.485946in}{2.172556in}}{\pgfqpoint{10.496996in}{2.172556in}}%
\pgfpathlineto{\pgfqpoint{10.496996in}{2.172556in}}%
\pgfpathclose%
\pgfusepath{stroke}%
\end{pgfscope}%
\begin{pgfscope}%
\pgfpathrectangle{\pgfqpoint{7.512535in}{0.437222in}}{\pgfqpoint{6.275590in}{5.159444in}}%
\pgfusepath{clip}%
\pgfsetbuttcap%
\pgfsetroundjoin%
\pgfsetlinewidth{1.003750pt}%
\definecolor{currentstroke}{rgb}{0.827451,0.827451,0.827451}%
\pgfsetstrokecolor{currentstroke}%
\pgfsetstrokeopacity{0.800000}%
\pgfsetdash{}{0pt}%
\pgfpathmoveto{\pgfqpoint{8.588580in}{4.198512in}}%
\pgfpathcurveto{\pgfqpoint{8.599631in}{4.198512in}}{\pgfqpoint{8.610230in}{4.202903in}}{\pgfqpoint{8.618043in}{4.210716in}}%
\pgfpathcurveto{\pgfqpoint{8.625857in}{4.218530in}}{\pgfqpoint{8.630247in}{4.229129in}}{\pgfqpoint{8.630247in}{4.240179in}}%
\pgfpathcurveto{\pgfqpoint{8.630247in}{4.251229in}}{\pgfqpoint{8.625857in}{4.261828in}}{\pgfqpoint{8.618043in}{4.269642in}}%
\pgfpathcurveto{\pgfqpoint{8.610230in}{4.277455in}}{\pgfqpoint{8.599631in}{4.281846in}}{\pgfqpoint{8.588580in}{4.281846in}}%
\pgfpathcurveto{\pgfqpoint{8.577530in}{4.281846in}}{\pgfqpoint{8.566931in}{4.277455in}}{\pgfqpoint{8.559118in}{4.269642in}}%
\pgfpathcurveto{\pgfqpoint{8.551304in}{4.261828in}}{\pgfqpoint{8.546914in}{4.251229in}}{\pgfqpoint{8.546914in}{4.240179in}}%
\pgfpathcurveto{\pgfqpoint{8.546914in}{4.229129in}}{\pgfqpoint{8.551304in}{4.218530in}}{\pgfqpoint{8.559118in}{4.210716in}}%
\pgfpathcurveto{\pgfqpoint{8.566931in}{4.202903in}}{\pgfqpoint{8.577530in}{4.198512in}}{\pgfqpoint{8.588580in}{4.198512in}}%
\pgfpathlineto{\pgfqpoint{8.588580in}{4.198512in}}%
\pgfpathclose%
\pgfusepath{stroke}%
\end{pgfscope}%
\begin{pgfscope}%
\pgfpathrectangle{\pgfqpoint{7.512535in}{0.437222in}}{\pgfqpoint{6.275590in}{5.159444in}}%
\pgfusepath{clip}%
\pgfsetbuttcap%
\pgfsetroundjoin%
\pgfsetlinewidth{1.003750pt}%
\definecolor{currentstroke}{rgb}{0.827451,0.827451,0.827451}%
\pgfsetstrokecolor{currentstroke}%
\pgfsetstrokeopacity{0.800000}%
\pgfsetdash{}{0pt}%
\pgfpathmoveto{\pgfqpoint{13.525425in}{4.914881in}}%
\pgfpathcurveto{\pgfqpoint{13.536475in}{4.914881in}}{\pgfqpoint{13.547074in}{4.919272in}}{\pgfqpoint{13.554888in}{4.927085in}}%
\pgfpathcurveto{\pgfqpoint{13.562701in}{4.934899in}}{\pgfqpoint{13.567091in}{4.945498in}}{\pgfqpoint{13.567091in}{4.956548in}}%
\pgfpathcurveto{\pgfqpoint{13.567091in}{4.967598in}}{\pgfqpoint{13.562701in}{4.978197in}}{\pgfqpoint{13.554888in}{4.986011in}}%
\pgfpathcurveto{\pgfqpoint{13.547074in}{4.993824in}}{\pgfqpoint{13.536475in}{4.998215in}}{\pgfqpoint{13.525425in}{4.998215in}}%
\pgfpathcurveto{\pgfqpoint{13.514375in}{4.998215in}}{\pgfqpoint{13.503776in}{4.993824in}}{\pgfqpoint{13.495962in}{4.986011in}}%
\pgfpathcurveto{\pgfqpoint{13.488148in}{4.978197in}}{\pgfqpoint{13.483758in}{4.967598in}}{\pgfqpoint{13.483758in}{4.956548in}}%
\pgfpathcurveto{\pgfqpoint{13.483758in}{4.945498in}}{\pgfqpoint{13.488148in}{4.934899in}}{\pgfqpoint{13.495962in}{4.927085in}}%
\pgfpathcurveto{\pgfqpoint{13.503776in}{4.919272in}}{\pgfqpoint{13.514375in}{4.914881in}}{\pgfqpoint{13.525425in}{4.914881in}}%
\pgfpathlineto{\pgfqpoint{13.525425in}{4.914881in}}%
\pgfpathclose%
\pgfusepath{stroke}%
\end{pgfscope}%
\begin{pgfscope}%
\pgfpathrectangle{\pgfqpoint{7.512535in}{0.437222in}}{\pgfqpoint{6.275590in}{5.159444in}}%
\pgfusepath{clip}%
\pgfsetbuttcap%
\pgfsetroundjoin%
\pgfsetlinewidth{1.003750pt}%
\definecolor{currentstroke}{rgb}{0.827451,0.827451,0.827451}%
\pgfsetstrokecolor{currentstroke}%
\pgfsetstrokeopacity{0.800000}%
\pgfsetdash{}{0pt}%
\pgfpathmoveto{\pgfqpoint{8.885584in}{5.167359in}}%
\pgfpathcurveto{\pgfqpoint{8.896634in}{5.167359in}}{\pgfqpoint{8.907233in}{5.171749in}}{\pgfqpoint{8.915047in}{5.179563in}}%
\pgfpathcurveto{\pgfqpoint{8.922861in}{5.187376in}}{\pgfqpoint{8.927251in}{5.197975in}}{\pgfqpoint{8.927251in}{5.209025in}}%
\pgfpathcurveto{\pgfqpoint{8.927251in}{5.220075in}}{\pgfqpoint{8.922861in}{5.230675in}}{\pgfqpoint{8.915047in}{5.238488in}}%
\pgfpathcurveto{\pgfqpoint{8.907233in}{5.246302in}}{\pgfqpoint{8.896634in}{5.250692in}}{\pgfqpoint{8.885584in}{5.250692in}}%
\pgfpathcurveto{\pgfqpoint{8.874534in}{5.250692in}}{\pgfqpoint{8.863935in}{5.246302in}}{\pgfqpoint{8.856122in}{5.238488in}}%
\pgfpathcurveto{\pgfqpoint{8.848308in}{5.230675in}}{\pgfqpoint{8.843918in}{5.220075in}}{\pgfqpoint{8.843918in}{5.209025in}}%
\pgfpathcurveto{\pgfqpoint{8.843918in}{5.197975in}}{\pgfqpoint{8.848308in}{5.187376in}}{\pgfqpoint{8.856122in}{5.179563in}}%
\pgfpathcurveto{\pgfqpoint{8.863935in}{5.171749in}}{\pgfqpoint{8.874534in}{5.167359in}}{\pgfqpoint{8.885584in}{5.167359in}}%
\pgfpathlineto{\pgfqpoint{8.885584in}{5.167359in}}%
\pgfpathclose%
\pgfusepath{stroke}%
\end{pgfscope}%
\begin{pgfscope}%
\pgfpathrectangle{\pgfqpoint{7.512535in}{0.437222in}}{\pgfqpoint{6.275590in}{5.159444in}}%
\pgfusepath{clip}%
\pgfsetbuttcap%
\pgfsetroundjoin%
\pgfsetlinewidth{1.003750pt}%
\definecolor{currentstroke}{rgb}{0.827451,0.827451,0.827451}%
\pgfsetstrokecolor{currentstroke}%
\pgfsetstrokeopacity{0.800000}%
\pgfsetdash{}{0pt}%
\pgfpathmoveto{\pgfqpoint{11.236541in}{2.284297in}}%
\pgfpathcurveto{\pgfqpoint{11.247591in}{2.284297in}}{\pgfqpoint{11.258190in}{2.288687in}}{\pgfqpoint{11.266004in}{2.296500in}}%
\pgfpathcurveto{\pgfqpoint{11.273817in}{2.304314in}}{\pgfqpoint{11.278208in}{2.314913in}}{\pgfqpoint{11.278208in}{2.325963in}}%
\pgfpathcurveto{\pgfqpoint{11.278208in}{2.337013in}}{\pgfqpoint{11.273817in}{2.347612in}}{\pgfqpoint{11.266004in}{2.355426in}}%
\pgfpathcurveto{\pgfqpoint{11.258190in}{2.363240in}}{\pgfqpoint{11.247591in}{2.367630in}}{\pgfqpoint{11.236541in}{2.367630in}}%
\pgfpathcurveto{\pgfqpoint{11.225491in}{2.367630in}}{\pgfqpoint{11.214892in}{2.363240in}}{\pgfqpoint{11.207078in}{2.355426in}}%
\pgfpathcurveto{\pgfqpoint{11.199265in}{2.347612in}}{\pgfqpoint{11.194874in}{2.337013in}}{\pgfqpoint{11.194874in}{2.325963in}}%
\pgfpathcurveto{\pgfqpoint{11.194874in}{2.314913in}}{\pgfqpoint{11.199265in}{2.304314in}}{\pgfqpoint{11.207078in}{2.296500in}}%
\pgfpathcurveto{\pgfqpoint{11.214892in}{2.288687in}}{\pgfqpoint{11.225491in}{2.284297in}}{\pgfqpoint{11.236541in}{2.284297in}}%
\pgfpathlineto{\pgfqpoint{11.236541in}{2.284297in}}%
\pgfpathclose%
\pgfusepath{stroke}%
\end{pgfscope}%
\begin{pgfscope}%
\pgfpathrectangle{\pgfqpoint{7.512535in}{0.437222in}}{\pgfqpoint{6.275590in}{5.159444in}}%
\pgfusepath{clip}%
\pgfsetbuttcap%
\pgfsetroundjoin%
\pgfsetlinewidth{1.003750pt}%
\definecolor{currentstroke}{rgb}{0.827451,0.827451,0.827451}%
\pgfsetstrokecolor{currentstroke}%
\pgfsetstrokeopacity{0.800000}%
\pgfsetdash{}{0pt}%
\pgfpathmoveto{\pgfqpoint{12.407343in}{1.801982in}}%
\pgfpathcurveto{\pgfqpoint{12.418393in}{1.801982in}}{\pgfqpoint{12.428992in}{1.806372in}}{\pgfqpoint{12.436806in}{1.814186in}}%
\pgfpathcurveto{\pgfqpoint{12.444620in}{1.821999in}}{\pgfqpoint{12.449010in}{1.832598in}}{\pgfqpoint{12.449010in}{1.843649in}}%
\pgfpathcurveto{\pgfqpoint{12.449010in}{1.854699in}}{\pgfqpoint{12.444620in}{1.865298in}}{\pgfqpoint{12.436806in}{1.873111in}}%
\pgfpathcurveto{\pgfqpoint{12.428992in}{1.880925in}}{\pgfqpoint{12.418393in}{1.885315in}}{\pgfqpoint{12.407343in}{1.885315in}}%
\pgfpathcurveto{\pgfqpoint{12.396293in}{1.885315in}}{\pgfqpoint{12.385694in}{1.880925in}}{\pgfqpoint{12.377881in}{1.873111in}}%
\pgfpathcurveto{\pgfqpoint{12.370067in}{1.865298in}}{\pgfqpoint{12.365677in}{1.854699in}}{\pgfqpoint{12.365677in}{1.843649in}}%
\pgfpathcurveto{\pgfqpoint{12.365677in}{1.832598in}}{\pgfqpoint{12.370067in}{1.821999in}}{\pgfqpoint{12.377881in}{1.814186in}}%
\pgfpathcurveto{\pgfqpoint{12.385694in}{1.806372in}}{\pgfqpoint{12.396293in}{1.801982in}}{\pgfqpoint{12.407343in}{1.801982in}}%
\pgfpathlineto{\pgfqpoint{12.407343in}{1.801982in}}%
\pgfpathclose%
\pgfusepath{stroke}%
\end{pgfscope}%
\begin{pgfscope}%
\pgfpathrectangle{\pgfqpoint{7.512535in}{0.437222in}}{\pgfqpoint{6.275590in}{5.159444in}}%
\pgfusepath{clip}%
\pgfsetbuttcap%
\pgfsetroundjoin%
\pgfsetlinewidth{1.003750pt}%
\definecolor{currentstroke}{rgb}{0.827451,0.827451,0.827451}%
\pgfsetstrokecolor{currentstroke}%
\pgfsetstrokeopacity{0.800000}%
\pgfsetdash{}{0pt}%
\pgfpathmoveto{\pgfqpoint{12.513347in}{3.273292in}}%
\pgfpathcurveto{\pgfqpoint{12.524397in}{3.273292in}}{\pgfqpoint{12.534996in}{3.277682in}}{\pgfqpoint{12.542809in}{3.285495in}}%
\pgfpathcurveto{\pgfqpoint{12.550623in}{3.293309in}}{\pgfqpoint{12.555013in}{3.303908in}}{\pgfqpoint{12.555013in}{3.314958in}}%
\pgfpathcurveto{\pgfqpoint{12.555013in}{3.326008in}}{\pgfqpoint{12.550623in}{3.336607in}}{\pgfqpoint{12.542809in}{3.344421in}}%
\pgfpathcurveto{\pgfqpoint{12.534996in}{3.352235in}}{\pgfqpoint{12.524397in}{3.356625in}}{\pgfqpoint{12.513347in}{3.356625in}}%
\pgfpathcurveto{\pgfqpoint{12.502297in}{3.356625in}}{\pgfqpoint{12.491698in}{3.352235in}}{\pgfqpoint{12.483884in}{3.344421in}}%
\pgfpathcurveto{\pgfqpoint{12.476070in}{3.336607in}}{\pgfqpoint{12.471680in}{3.326008in}}{\pgfqpoint{12.471680in}{3.314958in}}%
\pgfpathcurveto{\pgfqpoint{12.471680in}{3.303908in}}{\pgfqpoint{12.476070in}{3.293309in}}{\pgfqpoint{12.483884in}{3.285495in}}%
\pgfpathcurveto{\pgfqpoint{12.491698in}{3.277682in}}{\pgfqpoint{12.502297in}{3.273292in}}{\pgfqpoint{12.513347in}{3.273292in}}%
\pgfpathlineto{\pgfqpoint{12.513347in}{3.273292in}}%
\pgfpathclose%
\pgfusepath{stroke}%
\end{pgfscope}%
\begin{pgfscope}%
\pgfpathrectangle{\pgfqpoint{7.512535in}{0.437222in}}{\pgfqpoint{6.275590in}{5.159444in}}%
\pgfusepath{clip}%
\pgfsetbuttcap%
\pgfsetroundjoin%
\pgfsetlinewidth{1.003750pt}%
\definecolor{currentstroke}{rgb}{0.827451,0.827451,0.827451}%
\pgfsetstrokecolor{currentstroke}%
\pgfsetstrokeopacity{0.800000}%
\pgfsetdash{}{0pt}%
\pgfpathmoveto{\pgfqpoint{13.368540in}{3.756305in}}%
\pgfpathcurveto{\pgfqpoint{13.379590in}{3.756305in}}{\pgfqpoint{13.390189in}{3.760695in}}{\pgfqpoint{13.398002in}{3.768509in}}%
\pgfpathcurveto{\pgfqpoint{13.405816in}{3.776322in}}{\pgfqpoint{13.410206in}{3.786921in}}{\pgfqpoint{13.410206in}{3.797972in}}%
\pgfpathcurveto{\pgfqpoint{13.410206in}{3.809022in}}{\pgfqpoint{13.405816in}{3.819621in}}{\pgfqpoint{13.398002in}{3.827434in}}%
\pgfpathcurveto{\pgfqpoint{13.390189in}{3.835248in}}{\pgfqpoint{13.379590in}{3.839638in}}{\pgfqpoint{13.368540in}{3.839638in}}%
\pgfpathcurveto{\pgfqpoint{13.357490in}{3.839638in}}{\pgfqpoint{13.346891in}{3.835248in}}{\pgfqpoint{13.339077in}{3.827434in}}%
\pgfpathcurveto{\pgfqpoint{13.331263in}{3.819621in}}{\pgfqpoint{13.326873in}{3.809022in}}{\pgfqpoint{13.326873in}{3.797972in}}%
\pgfpathcurveto{\pgfqpoint{13.326873in}{3.786921in}}{\pgfqpoint{13.331263in}{3.776322in}}{\pgfqpoint{13.339077in}{3.768509in}}%
\pgfpathcurveto{\pgfqpoint{13.346891in}{3.760695in}}{\pgfqpoint{13.357490in}{3.756305in}}{\pgfqpoint{13.368540in}{3.756305in}}%
\pgfpathlineto{\pgfqpoint{13.368540in}{3.756305in}}%
\pgfpathclose%
\pgfusepath{stroke}%
\end{pgfscope}%
\begin{pgfscope}%
\pgfpathrectangle{\pgfqpoint{7.512535in}{0.437222in}}{\pgfqpoint{6.275590in}{5.159444in}}%
\pgfusepath{clip}%
\pgfsetbuttcap%
\pgfsetroundjoin%
\pgfsetlinewidth{1.003750pt}%
\definecolor{currentstroke}{rgb}{0.827451,0.827451,0.827451}%
\pgfsetstrokecolor{currentstroke}%
\pgfsetstrokeopacity{0.800000}%
\pgfsetdash{}{0pt}%
\pgfpathmoveto{\pgfqpoint{13.274770in}{3.624009in}}%
\pgfpathcurveto{\pgfqpoint{13.285820in}{3.624009in}}{\pgfqpoint{13.296419in}{3.628399in}}{\pgfqpoint{13.304232in}{3.636213in}}%
\pgfpathcurveto{\pgfqpoint{13.312046in}{3.644027in}}{\pgfqpoint{13.316436in}{3.654626in}}{\pgfqpoint{13.316436in}{3.665676in}}%
\pgfpathcurveto{\pgfqpoint{13.316436in}{3.676726in}}{\pgfqpoint{13.312046in}{3.687325in}}{\pgfqpoint{13.304232in}{3.695139in}}%
\pgfpathcurveto{\pgfqpoint{13.296419in}{3.702952in}}{\pgfqpoint{13.285820in}{3.707342in}}{\pgfqpoint{13.274770in}{3.707342in}}%
\pgfpathcurveto{\pgfqpoint{13.263720in}{3.707342in}}{\pgfqpoint{13.253120in}{3.702952in}}{\pgfqpoint{13.245307in}{3.695139in}}%
\pgfpathcurveto{\pgfqpoint{13.237493in}{3.687325in}}{\pgfqpoint{13.233103in}{3.676726in}}{\pgfqpoint{13.233103in}{3.665676in}}%
\pgfpathcurveto{\pgfqpoint{13.233103in}{3.654626in}}{\pgfqpoint{13.237493in}{3.644027in}}{\pgfqpoint{13.245307in}{3.636213in}}%
\pgfpathcurveto{\pgfqpoint{13.253120in}{3.628399in}}{\pgfqpoint{13.263720in}{3.624009in}}{\pgfqpoint{13.274770in}{3.624009in}}%
\pgfpathlineto{\pgfqpoint{13.274770in}{3.624009in}}%
\pgfpathclose%
\pgfusepath{stroke}%
\end{pgfscope}%
\begin{pgfscope}%
\pgfpathrectangle{\pgfqpoint{7.512535in}{0.437222in}}{\pgfqpoint{6.275590in}{5.159444in}}%
\pgfusepath{clip}%
\pgfsetbuttcap%
\pgfsetroundjoin%
\pgfsetlinewidth{1.003750pt}%
\definecolor{currentstroke}{rgb}{0.827451,0.827451,0.827451}%
\pgfsetstrokecolor{currentstroke}%
\pgfsetstrokeopacity{0.800000}%
\pgfsetdash{}{0pt}%
\pgfpathmoveto{\pgfqpoint{13.628067in}{3.840965in}}%
\pgfpathcurveto{\pgfqpoint{13.639117in}{3.840965in}}{\pgfqpoint{13.649716in}{3.845355in}}{\pgfqpoint{13.657530in}{3.853169in}}%
\pgfpathcurveto{\pgfqpoint{13.665343in}{3.860983in}}{\pgfqpoint{13.669734in}{3.871582in}}{\pgfqpoint{13.669734in}{3.882632in}}%
\pgfpathcurveto{\pgfqpoint{13.669734in}{3.893682in}}{\pgfqpoint{13.665343in}{3.904281in}}{\pgfqpoint{13.657530in}{3.912094in}}%
\pgfpathcurveto{\pgfqpoint{13.649716in}{3.919908in}}{\pgfqpoint{13.639117in}{3.924298in}}{\pgfqpoint{13.628067in}{3.924298in}}%
\pgfpathcurveto{\pgfqpoint{13.617017in}{3.924298in}}{\pgfqpoint{13.606418in}{3.919908in}}{\pgfqpoint{13.598604in}{3.912094in}}%
\pgfpathcurveto{\pgfqpoint{13.590791in}{3.904281in}}{\pgfqpoint{13.586400in}{3.893682in}}{\pgfqpoint{13.586400in}{3.882632in}}%
\pgfpathcurveto{\pgfqpoint{13.586400in}{3.871582in}}{\pgfqpoint{13.590791in}{3.860983in}}{\pgfqpoint{13.598604in}{3.853169in}}%
\pgfpathcurveto{\pgfqpoint{13.606418in}{3.845355in}}{\pgfqpoint{13.617017in}{3.840965in}}{\pgfqpoint{13.628067in}{3.840965in}}%
\pgfpathlineto{\pgfqpoint{13.628067in}{3.840965in}}%
\pgfpathclose%
\pgfusepath{stroke}%
\end{pgfscope}%
\begin{pgfscope}%
\pgfpathrectangle{\pgfqpoint{7.512535in}{0.437222in}}{\pgfqpoint{6.275590in}{5.159444in}}%
\pgfusepath{clip}%
\pgfsetbuttcap%
\pgfsetroundjoin%
\pgfsetlinewidth{1.003750pt}%
\definecolor{currentstroke}{rgb}{0.827451,0.827451,0.827451}%
\pgfsetstrokecolor{currentstroke}%
\pgfsetstrokeopacity{0.800000}%
\pgfsetdash{}{0pt}%
\pgfpathmoveto{\pgfqpoint{10.512855in}{1.405131in}}%
\pgfpathcurveto{\pgfqpoint{10.523906in}{1.405131in}}{\pgfqpoint{10.534505in}{1.409521in}}{\pgfqpoint{10.542318in}{1.417335in}}%
\pgfpathcurveto{\pgfqpoint{10.550132in}{1.425148in}}{\pgfqpoint{10.554522in}{1.435747in}}{\pgfqpoint{10.554522in}{1.446797in}}%
\pgfpathcurveto{\pgfqpoint{10.554522in}{1.457848in}}{\pgfqpoint{10.550132in}{1.468447in}}{\pgfqpoint{10.542318in}{1.476260in}}%
\pgfpathcurveto{\pgfqpoint{10.534505in}{1.484074in}}{\pgfqpoint{10.523906in}{1.488464in}}{\pgfqpoint{10.512855in}{1.488464in}}%
\pgfpathcurveto{\pgfqpoint{10.501805in}{1.488464in}}{\pgfqpoint{10.491206in}{1.484074in}}{\pgfqpoint{10.483393in}{1.476260in}}%
\pgfpathcurveto{\pgfqpoint{10.475579in}{1.468447in}}{\pgfqpoint{10.471189in}{1.457848in}}{\pgfqpoint{10.471189in}{1.446797in}}%
\pgfpathcurveto{\pgfqpoint{10.471189in}{1.435747in}}{\pgfqpoint{10.475579in}{1.425148in}}{\pgfqpoint{10.483393in}{1.417335in}}%
\pgfpathcurveto{\pgfqpoint{10.491206in}{1.409521in}}{\pgfqpoint{10.501805in}{1.405131in}}{\pgfqpoint{10.512855in}{1.405131in}}%
\pgfpathlineto{\pgfqpoint{10.512855in}{1.405131in}}%
\pgfpathclose%
\pgfusepath{stroke}%
\end{pgfscope}%
\begin{pgfscope}%
\pgfpathrectangle{\pgfqpoint{7.512535in}{0.437222in}}{\pgfqpoint{6.275590in}{5.159444in}}%
\pgfusepath{clip}%
\pgfsetbuttcap%
\pgfsetroundjoin%
\pgfsetlinewidth{1.003750pt}%
\definecolor{currentstroke}{rgb}{0.827451,0.827451,0.827451}%
\pgfsetstrokecolor{currentstroke}%
\pgfsetstrokeopacity{0.800000}%
\pgfsetdash{}{0pt}%
\pgfpathmoveto{\pgfqpoint{10.472307in}{0.948270in}}%
\pgfpathcurveto{\pgfqpoint{10.483357in}{0.948270in}}{\pgfqpoint{10.493956in}{0.952661in}}{\pgfqpoint{10.501770in}{0.960474in}}%
\pgfpathcurveto{\pgfqpoint{10.509584in}{0.968288in}}{\pgfqpoint{10.513974in}{0.978887in}}{\pgfqpoint{10.513974in}{0.989937in}}%
\pgfpathcurveto{\pgfqpoint{10.513974in}{1.000987in}}{\pgfqpoint{10.509584in}{1.011586in}}{\pgfqpoint{10.501770in}{1.019400in}}%
\pgfpathcurveto{\pgfqpoint{10.493956in}{1.027213in}}{\pgfqpoint{10.483357in}{1.031604in}}{\pgfqpoint{10.472307in}{1.031604in}}%
\pgfpathcurveto{\pgfqpoint{10.461257in}{1.031604in}}{\pgfqpoint{10.450658in}{1.027213in}}{\pgfqpoint{10.442844in}{1.019400in}}%
\pgfpathcurveto{\pgfqpoint{10.435031in}{1.011586in}}{\pgfqpoint{10.430641in}{1.000987in}}{\pgfqpoint{10.430641in}{0.989937in}}%
\pgfpathcurveto{\pgfqpoint{10.430641in}{0.978887in}}{\pgfqpoint{10.435031in}{0.968288in}}{\pgfqpoint{10.442844in}{0.960474in}}%
\pgfpathcurveto{\pgfqpoint{10.450658in}{0.952661in}}{\pgfqpoint{10.461257in}{0.948270in}}{\pgfqpoint{10.472307in}{0.948270in}}%
\pgfpathlineto{\pgfqpoint{10.472307in}{0.948270in}}%
\pgfpathclose%
\pgfusepath{stroke}%
\end{pgfscope}%
\begin{pgfscope}%
\pgfpathrectangle{\pgfqpoint{7.512535in}{0.437222in}}{\pgfqpoint{6.275590in}{5.159444in}}%
\pgfusepath{clip}%
\pgfsetbuttcap%
\pgfsetroundjoin%
\pgfsetlinewidth{1.003750pt}%
\definecolor{currentstroke}{rgb}{0.827451,0.827451,0.827451}%
\pgfsetstrokecolor{currentstroke}%
\pgfsetstrokeopacity{0.800000}%
\pgfsetdash{}{0pt}%
\pgfpathmoveto{\pgfqpoint{13.747756in}{4.619611in}}%
\pgfpathcurveto{\pgfqpoint{13.758807in}{4.619611in}}{\pgfqpoint{13.769406in}{4.624001in}}{\pgfqpoint{13.777219in}{4.631815in}}%
\pgfpathcurveto{\pgfqpoint{13.785033in}{4.639628in}}{\pgfqpoint{13.789423in}{4.650227in}}{\pgfqpoint{13.789423in}{4.661277in}}%
\pgfpathcurveto{\pgfqpoint{13.789423in}{4.672328in}}{\pgfqpoint{13.785033in}{4.682927in}}{\pgfqpoint{13.777219in}{4.690740in}}%
\pgfpathcurveto{\pgfqpoint{13.769406in}{4.698554in}}{\pgfqpoint{13.758807in}{4.702944in}}{\pgfqpoint{13.747756in}{4.702944in}}%
\pgfpathcurveto{\pgfqpoint{13.736706in}{4.702944in}}{\pgfqpoint{13.726107in}{4.698554in}}{\pgfqpoint{13.718294in}{4.690740in}}%
\pgfpathcurveto{\pgfqpoint{13.710480in}{4.682927in}}{\pgfqpoint{13.706090in}{4.672328in}}{\pgfqpoint{13.706090in}{4.661277in}}%
\pgfpathcurveto{\pgfqpoint{13.706090in}{4.650227in}}{\pgfqpoint{13.710480in}{4.639628in}}{\pgfqpoint{13.718294in}{4.631815in}}%
\pgfpathcurveto{\pgfqpoint{13.726107in}{4.624001in}}{\pgfqpoint{13.736706in}{4.619611in}}{\pgfqpoint{13.747756in}{4.619611in}}%
\pgfpathlineto{\pgfqpoint{13.747756in}{4.619611in}}%
\pgfpathclose%
\pgfusepath{stroke}%
\end{pgfscope}%
\begin{pgfscope}%
\pgfpathrectangle{\pgfqpoint{7.512535in}{0.437222in}}{\pgfqpoint{6.275590in}{5.159444in}}%
\pgfusepath{clip}%
\pgfsetbuttcap%
\pgfsetroundjoin%
\pgfsetlinewidth{1.003750pt}%
\definecolor{currentstroke}{rgb}{0.827451,0.827451,0.827451}%
\pgfsetstrokecolor{currentstroke}%
\pgfsetstrokeopacity{0.800000}%
\pgfsetdash{}{0pt}%
\pgfpathmoveto{\pgfqpoint{11.305087in}{0.927517in}}%
\pgfpathcurveto{\pgfqpoint{11.316137in}{0.927517in}}{\pgfqpoint{11.326736in}{0.931907in}}{\pgfqpoint{11.334550in}{0.939721in}}%
\pgfpathcurveto{\pgfqpoint{11.342364in}{0.947535in}}{\pgfqpoint{11.346754in}{0.958134in}}{\pgfqpoint{11.346754in}{0.969184in}}%
\pgfpathcurveto{\pgfqpoint{11.346754in}{0.980234in}}{\pgfqpoint{11.342364in}{0.990833in}}{\pgfqpoint{11.334550in}{0.998647in}}%
\pgfpathcurveto{\pgfqpoint{11.326736in}{1.006460in}}{\pgfqpoint{11.316137in}{1.010851in}}{\pgfqpoint{11.305087in}{1.010851in}}%
\pgfpathcurveto{\pgfqpoint{11.294037in}{1.010851in}}{\pgfqpoint{11.283438in}{1.006460in}}{\pgfqpoint{11.275625in}{0.998647in}}%
\pgfpathcurveto{\pgfqpoint{11.267811in}{0.990833in}}{\pgfqpoint{11.263421in}{0.980234in}}{\pgfqpoint{11.263421in}{0.969184in}}%
\pgfpathcurveto{\pgfqpoint{11.263421in}{0.958134in}}{\pgfqpoint{11.267811in}{0.947535in}}{\pgfqpoint{11.275625in}{0.939721in}}%
\pgfpathcurveto{\pgfqpoint{11.283438in}{0.931907in}}{\pgfqpoint{11.294037in}{0.927517in}}{\pgfqpoint{11.305087in}{0.927517in}}%
\pgfpathlineto{\pgfqpoint{11.305087in}{0.927517in}}%
\pgfpathclose%
\pgfusepath{stroke}%
\end{pgfscope}%
\begin{pgfscope}%
\pgfpathrectangle{\pgfqpoint{7.512535in}{0.437222in}}{\pgfqpoint{6.275590in}{5.159444in}}%
\pgfusepath{clip}%
\pgfsetbuttcap%
\pgfsetroundjoin%
\pgfsetlinewidth{1.003750pt}%
\definecolor{currentstroke}{rgb}{0.827451,0.827451,0.827451}%
\pgfsetstrokecolor{currentstroke}%
\pgfsetstrokeopacity{0.800000}%
\pgfsetdash{}{0pt}%
\pgfpathmoveto{\pgfqpoint{11.943086in}{1.168614in}}%
\pgfpathcurveto{\pgfqpoint{11.954136in}{1.168614in}}{\pgfqpoint{11.964735in}{1.173005in}}{\pgfqpoint{11.972549in}{1.180818in}}%
\pgfpathcurveto{\pgfqpoint{11.980363in}{1.188632in}}{\pgfqpoint{11.984753in}{1.199231in}}{\pgfqpoint{11.984753in}{1.210281in}}%
\pgfpathcurveto{\pgfqpoint{11.984753in}{1.221331in}}{\pgfqpoint{11.980363in}{1.231930in}}{\pgfqpoint{11.972549in}{1.239744in}}%
\pgfpathcurveto{\pgfqpoint{11.964735in}{1.247557in}}{\pgfqpoint{11.954136in}{1.251948in}}{\pgfqpoint{11.943086in}{1.251948in}}%
\pgfpathcurveto{\pgfqpoint{11.932036in}{1.251948in}}{\pgfqpoint{11.921437in}{1.247557in}}{\pgfqpoint{11.913623in}{1.239744in}}%
\pgfpathcurveto{\pgfqpoint{11.905810in}{1.231930in}}{\pgfqpoint{11.901420in}{1.221331in}}{\pgfqpoint{11.901420in}{1.210281in}}%
\pgfpathcurveto{\pgfqpoint{11.901420in}{1.199231in}}{\pgfqpoint{11.905810in}{1.188632in}}{\pgfqpoint{11.913623in}{1.180818in}}%
\pgfpathcurveto{\pgfqpoint{11.921437in}{1.173005in}}{\pgfqpoint{11.932036in}{1.168614in}}{\pgfqpoint{11.943086in}{1.168614in}}%
\pgfpathlineto{\pgfqpoint{11.943086in}{1.168614in}}%
\pgfpathclose%
\pgfusepath{stroke}%
\end{pgfscope}%
\begin{pgfscope}%
\pgfpathrectangle{\pgfqpoint{7.512535in}{0.437222in}}{\pgfqpoint{6.275590in}{5.159444in}}%
\pgfusepath{clip}%
\pgfsetbuttcap%
\pgfsetroundjoin%
\pgfsetlinewidth{1.003750pt}%
\definecolor{currentstroke}{rgb}{0.827451,0.827451,0.827451}%
\pgfsetstrokecolor{currentstroke}%
\pgfsetstrokeopacity{0.800000}%
\pgfsetdash{}{0pt}%
\pgfpathmoveto{\pgfqpoint{12.083855in}{1.130443in}}%
\pgfpathcurveto{\pgfqpoint{12.094905in}{1.130443in}}{\pgfqpoint{12.105504in}{1.134833in}}{\pgfqpoint{12.113317in}{1.142647in}}%
\pgfpathcurveto{\pgfqpoint{12.121131in}{1.150460in}}{\pgfqpoint{12.125521in}{1.161059in}}{\pgfqpoint{12.125521in}{1.172110in}}%
\pgfpathcurveto{\pgfqpoint{12.125521in}{1.183160in}}{\pgfqpoint{12.121131in}{1.193759in}}{\pgfqpoint{12.113317in}{1.201572in}}%
\pgfpathcurveto{\pgfqpoint{12.105504in}{1.209386in}}{\pgfqpoint{12.094905in}{1.213776in}}{\pgfqpoint{12.083855in}{1.213776in}}%
\pgfpathcurveto{\pgfqpoint{12.072805in}{1.213776in}}{\pgfqpoint{12.062205in}{1.209386in}}{\pgfqpoint{12.054392in}{1.201572in}}%
\pgfpathcurveto{\pgfqpoint{12.046578in}{1.193759in}}{\pgfqpoint{12.042188in}{1.183160in}}{\pgfqpoint{12.042188in}{1.172110in}}%
\pgfpathcurveto{\pgfqpoint{12.042188in}{1.161059in}}{\pgfqpoint{12.046578in}{1.150460in}}{\pgfqpoint{12.054392in}{1.142647in}}%
\pgfpathcurveto{\pgfqpoint{12.062205in}{1.134833in}}{\pgfqpoint{12.072805in}{1.130443in}}{\pgfqpoint{12.083855in}{1.130443in}}%
\pgfpathlineto{\pgfqpoint{12.083855in}{1.130443in}}%
\pgfpathclose%
\pgfusepath{stroke}%
\end{pgfscope}%
\begin{pgfscope}%
\pgfpathrectangle{\pgfqpoint{7.512535in}{0.437222in}}{\pgfqpoint{6.275590in}{5.159444in}}%
\pgfusepath{clip}%
\pgfsetbuttcap%
\pgfsetroundjoin%
\pgfsetlinewidth{1.003750pt}%
\definecolor{currentstroke}{rgb}{0.827451,0.827451,0.827451}%
\pgfsetstrokecolor{currentstroke}%
\pgfsetstrokeopacity{0.800000}%
\pgfsetdash{}{0pt}%
\pgfpathmoveto{\pgfqpoint{11.534438in}{1.046705in}}%
\pgfpathcurveto{\pgfqpoint{11.545488in}{1.046705in}}{\pgfqpoint{11.556087in}{1.051095in}}{\pgfqpoint{11.563901in}{1.058909in}}%
\pgfpathcurveto{\pgfqpoint{11.571714in}{1.066722in}}{\pgfqpoint{11.576104in}{1.077321in}}{\pgfqpoint{11.576104in}{1.088372in}}%
\pgfpathcurveto{\pgfqpoint{11.576104in}{1.099422in}}{\pgfqpoint{11.571714in}{1.110021in}}{\pgfqpoint{11.563901in}{1.117834in}}%
\pgfpathcurveto{\pgfqpoint{11.556087in}{1.125648in}}{\pgfqpoint{11.545488in}{1.130038in}}{\pgfqpoint{11.534438in}{1.130038in}}%
\pgfpathcurveto{\pgfqpoint{11.523388in}{1.130038in}}{\pgfqpoint{11.512789in}{1.125648in}}{\pgfqpoint{11.504975in}{1.117834in}}%
\pgfpathcurveto{\pgfqpoint{11.497161in}{1.110021in}}{\pgfqpoint{11.492771in}{1.099422in}}{\pgfqpoint{11.492771in}{1.088372in}}%
\pgfpathcurveto{\pgfqpoint{11.492771in}{1.077321in}}{\pgfqpoint{11.497161in}{1.066722in}}{\pgfqpoint{11.504975in}{1.058909in}}%
\pgfpathcurveto{\pgfqpoint{11.512789in}{1.051095in}}{\pgfqpoint{11.523388in}{1.046705in}}{\pgfqpoint{11.534438in}{1.046705in}}%
\pgfpathlineto{\pgfqpoint{11.534438in}{1.046705in}}%
\pgfpathclose%
\pgfusepath{stroke}%
\end{pgfscope}%
\begin{pgfscope}%
\pgfpathrectangle{\pgfqpoint{7.512535in}{0.437222in}}{\pgfqpoint{6.275590in}{5.159444in}}%
\pgfusepath{clip}%
\pgfsetbuttcap%
\pgfsetroundjoin%
\pgfsetlinewidth{1.003750pt}%
\definecolor{currentstroke}{rgb}{0.827451,0.827451,0.827451}%
\pgfsetstrokecolor{currentstroke}%
\pgfsetstrokeopacity{0.800000}%
\pgfsetdash{}{0pt}%
\pgfpathmoveto{\pgfqpoint{12.546485in}{1.137312in}}%
\pgfpathcurveto{\pgfqpoint{12.557535in}{1.137312in}}{\pgfqpoint{12.568134in}{1.141703in}}{\pgfqpoint{12.575947in}{1.149516in}}%
\pgfpathcurveto{\pgfqpoint{12.583761in}{1.157330in}}{\pgfqpoint{12.588151in}{1.167929in}}{\pgfqpoint{12.588151in}{1.178979in}}%
\pgfpathcurveto{\pgfqpoint{12.588151in}{1.190029in}}{\pgfqpoint{12.583761in}{1.200628in}}{\pgfqpoint{12.575947in}{1.208442in}}%
\pgfpathcurveto{\pgfqpoint{12.568134in}{1.216256in}}{\pgfqpoint{12.557535in}{1.220646in}}{\pgfqpoint{12.546485in}{1.220646in}}%
\pgfpathcurveto{\pgfqpoint{12.535435in}{1.220646in}}{\pgfqpoint{12.524836in}{1.216256in}}{\pgfqpoint{12.517022in}{1.208442in}}%
\pgfpathcurveto{\pgfqpoint{12.509208in}{1.200628in}}{\pgfqpoint{12.504818in}{1.190029in}}{\pgfqpoint{12.504818in}{1.178979in}}%
\pgfpathcurveto{\pgfqpoint{12.504818in}{1.167929in}}{\pgfqpoint{12.509208in}{1.157330in}}{\pgfqpoint{12.517022in}{1.149516in}}%
\pgfpathcurveto{\pgfqpoint{12.524836in}{1.141703in}}{\pgfqpoint{12.535435in}{1.137312in}}{\pgfqpoint{12.546485in}{1.137312in}}%
\pgfpathlineto{\pgfqpoint{12.546485in}{1.137312in}}%
\pgfpathclose%
\pgfusepath{stroke}%
\end{pgfscope}%
\begin{pgfscope}%
\pgfpathrectangle{\pgfqpoint{7.512535in}{0.437222in}}{\pgfqpoint{6.275590in}{5.159444in}}%
\pgfusepath{clip}%
\pgfsetbuttcap%
\pgfsetroundjoin%
\pgfsetlinewidth{1.003750pt}%
\definecolor{currentstroke}{rgb}{0.827451,0.827451,0.827451}%
\pgfsetstrokecolor{currentstroke}%
\pgfsetstrokeopacity{0.800000}%
\pgfsetdash{}{0pt}%
\pgfpathmoveto{\pgfqpoint{12.966822in}{2.645968in}}%
\pgfpathcurveto{\pgfqpoint{12.977872in}{2.645968in}}{\pgfqpoint{12.988471in}{2.650358in}}{\pgfqpoint{12.996285in}{2.658172in}}%
\pgfpathcurveto{\pgfqpoint{13.004099in}{2.665986in}}{\pgfqpoint{13.008489in}{2.676585in}}{\pgfqpoint{13.008489in}{2.687635in}}%
\pgfpathcurveto{\pgfqpoint{13.008489in}{2.698685in}}{\pgfqpoint{13.004099in}{2.709284in}}{\pgfqpoint{12.996285in}{2.717098in}}%
\pgfpathcurveto{\pgfqpoint{12.988471in}{2.724911in}}{\pgfqpoint{12.977872in}{2.729301in}}{\pgfqpoint{12.966822in}{2.729301in}}%
\pgfpathcurveto{\pgfqpoint{12.955772in}{2.729301in}}{\pgfqpoint{12.945173in}{2.724911in}}{\pgfqpoint{12.937359in}{2.717098in}}%
\pgfpathcurveto{\pgfqpoint{12.929546in}{2.709284in}}{\pgfqpoint{12.925155in}{2.698685in}}{\pgfqpoint{12.925155in}{2.687635in}}%
\pgfpathcurveto{\pgfqpoint{12.925155in}{2.676585in}}{\pgfqpoint{12.929546in}{2.665986in}}{\pgfqpoint{12.937359in}{2.658172in}}%
\pgfpathcurveto{\pgfqpoint{12.945173in}{2.650358in}}{\pgfqpoint{12.955772in}{2.645968in}}{\pgfqpoint{12.966822in}{2.645968in}}%
\pgfpathlineto{\pgfqpoint{12.966822in}{2.645968in}}%
\pgfpathclose%
\pgfusepath{stroke}%
\end{pgfscope}%
\begin{pgfscope}%
\pgfpathrectangle{\pgfqpoint{7.512535in}{0.437222in}}{\pgfqpoint{6.275590in}{5.159444in}}%
\pgfusepath{clip}%
\pgfsetbuttcap%
\pgfsetroundjoin%
\pgfsetlinewidth{1.003750pt}%
\definecolor{currentstroke}{rgb}{0.827451,0.827451,0.827451}%
\pgfsetstrokecolor{currentstroke}%
\pgfsetstrokeopacity{0.800000}%
\pgfsetdash{}{0pt}%
\pgfpathmoveto{\pgfqpoint{12.871252in}{1.876638in}}%
\pgfpathcurveto{\pgfqpoint{12.882302in}{1.876638in}}{\pgfqpoint{12.892901in}{1.881028in}}{\pgfqpoint{12.900715in}{1.888842in}}%
\pgfpathcurveto{\pgfqpoint{12.908529in}{1.896656in}}{\pgfqpoint{12.912919in}{1.907255in}}{\pgfqpoint{12.912919in}{1.918305in}}%
\pgfpathcurveto{\pgfqpoint{12.912919in}{1.929355in}}{\pgfqpoint{12.908529in}{1.939954in}}{\pgfqpoint{12.900715in}{1.947768in}}%
\pgfpathcurveto{\pgfqpoint{12.892901in}{1.955581in}}{\pgfqpoint{12.882302in}{1.959972in}}{\pgfqpoint{12.871252in}{1.959972in}}%
\pgfpathcurveto{\pgfqpoint{12.860202in}{1.959972in}}{\pgfqpoint{12.849603in}{1.955581in}}{\pgfqpoint{12.841789in}{1.947768in}}%
\pgfpathcurveto{\pgfqpoint{12.833976in}{1.939954in}}{\pgfqpoint{12.829586in}{1.929355in}}{\pgfqpoint{12.829586in}{1.918305in}}%
\pgfpathcurveto{\pgfqpoint{12.829586in}{1.907255in}}{\pgfqpoint{12.833976in}{1.896656in}}{\pgfqpoint{12.841789in}{1.888842in}}%
\pgfpathcurveto{\pgfqpoint{12.849603in}{1.881028in}}{\pgfqpoint{12.860202in}{1.876638in}}{\pgfqpoint{12.871252in}{1.876638in}}%
\pgfpathlineto{\pgfqpoint{12.871252in}{1.876638in}}%
\pgfpathclose%
\pgfusepath{stroke}%
\end{pgfscope}%
\begin{pgfscope}%
\pgfpathrectangle{\pgfqpoint{7.512535in}{0.437222in}}{\pgfqpoint{6.275590in}{5.159444in}}%
\pgfusepath{clip}%
\pgfsetbuttcap%
\pgfsetroundjoin%
\pgfsetlinewidth{1.003750pt}%
\definecolor{currentstroke}{rgb}{0.827451,0.827451,0.827451}%
\pgfsetstrokecolor{currentstroke}%
\pgfsetstrokeopacity{0.800000}%
\pgfsetdash{}{0pt}%
\pgfpathmoveto{\pgfqpoint{13.645302in}{3.268823in}}%
\pgfpathcurveto{\pgfqpoint{13.656352in}{3.268823in}}{\pgfqpoint{13.666951in}{3.273213in}}{\pgfqpoint{13.674764in}{3.281027in}}%
\pgfpathcurveto{\pgfqpoint{13.682578in}{3.288840in}}{\pgfqpoint{13.686968in}{3.299439in}}{\pgfqpoint{13.686968in}{3.310489in}}%
\pgfpathcurveto{\pgfqpoint{13.686968in}{3.321540in}}{\pgfqpoint{13.682578in}{3.332139in}}{\pgfqpoint{13.674764in}{3.339952in}}%
\pgfpathcurveto{\pgfqpoint{13.666951in}{3.347766in}}{\pgfqpoint{13.656352in}{3.352156in}}{\pgfqpoint{13.645302in}{3.352156in}}%
\pgfpathcurveto{\pgfqpoint{13.634251in}{3.352156in}}{\pgfqpoint{13.623652in}{3.347766in}}{\pgfqpoint{13.615839in}{3.339952in}}%
\pgfpathcurveto{\pgfqpoint{13.608025in}{3.332139in}}{\pgfqpoint{13.603635in}{3.321540in}}{\pgfqpoint{13.603635in}{3.310489in}}%
\pgfpathcurveto{\pgfqpoint{13.603635in}{3.299439in}}{\pgfqpoint{13.608025in}{3.288840in}}{\pgfqpoint{13.615839in}{3.281027in}}%
\pgfpathcurveto{\pgfqpoint{13.623652in}{3.273213in}}{\pgfqpoint{13.634251in}{3.268823in}}{\pgfqpoint{13.645302in}{3.268823in}}%
\pgfpathlineto{\pgfqpoint{13.645302in}{3.268823in}}%
\pgfpathclose%
\pgfusepath{stroke}%
\end{pgfscope}%
\begin{pgfscope}%
\pgfpathrectangle{\pgfqpoint{7.512535in}{0.437222in}}{\pgfqpoint{6.275590in}{5.159444in}}%
\pgfusepath{clip}%
\pgfsetbuttcap%
\pgfsetroundjoin%
\pgfsetlinewidth{1.003750pt}%
\definecolor{currentstroke}{rgb}{0.827451,0.827451,0.827451}%
\pgfsetstrokecolor{currentstroke}%
\pgfsetstrokeopacity{0.800000}%
\pgfsetdash{}{0pt}%
\pgfpathmoveto{\pgfqpoint{10.669679in}{0.466593in}}%
\pgfpathcurveto{\pgfqpoint{10.680729in}{0.466593in}}{\pgfqpoint{10.691328in}{0.470983in}}{\pgfqpoint{10.699141in}{0.478797in}}%
\pgfpathcurveto{\pgfqpoint{10.706955in}{0.486611in}}{\pgfqpoint{10.711345in}{0.497210in}}{\pgfqpoint{10.711345in}{0.508260in}}%
\pgfpathcurveto{\pgfqpoint{10.711345in}{0.519310in}}{\pgfqpoint{10.706955in}{0.529909in}}{\pgfqpoint{10.699141in}{0.537723in}}%
\pgfpathcurveto{\pgfqpoint{10.691328in}{0.545536in}}{\pgfqpoint{10.680729in}{0.549926in}}{\pgfqpoint{10.669679in}{0.549926in}}%
\pgfpathcurveto{\pgfqpoint{10.658628in}{0.549926in}}{\pgfqpoint{10.648029in}{0.545536in}}{\pgfqpoint{10.640216in}{0.537723in}}%
\pgfpathcurveto{\pgfqpoint{10.632402in}{0.529909in}}{\pgfqpoint{10.628012in}{0.519310in}}{\pgfqpoint{10.628012in}{0.508260in}}%
\pgfpathcurveto{\pgfqpoint{10.628012in}{0.497210in}}{\pgfqpoint{10.632402in}{0.486611in}}{\pgfqpoint{10.640216in}{0.478797in}}%
\pgfpathcurveto{\pgfqpoint{10.648029in}{0.470983in}}{\pgfqpoint{10.658628in}{0.466593in}}{\pgfqpoint{10.669679in}{0.466593in}}%
\pgfpathlineto{\pgfqpoint{10.669679in}{0.466593in}}%
\pgfpathclose%
\pgfusepath{stroke}%
\end{pgfscope}%
\begin{pgfscope}%
\pgfpathrectangle{\pgfqpoint{7.512535in}{0.437222in}}{\pgfqpoint{6.275590in}{5.159444in}}%
\pgfusepath{clip}%
\pgfsetbuttcap%
\pgfsetroundjoin%
\pgfsetlinewidth{1.003750pt}%
\definecolor{currentstroke}{rgb}{0.827451,0.827451,0.827451}%
\pgfsetstrokecolor{currentstroke}%
\pgfsetstrokeopacity{0.800000}%
\pgfsetdash{}{0pt}%
\pgfpathmoveto{\pgfqpoint{11.064171in}{0.430454in}}%
\pgfpathcurveto{\pgfqpoint{11.075222in}{0.430454in}}{\pgfqpoint{11.085821in}{0.434845in}}{\pgfqpoint{11.093634in}{0.442658in}}%
\pgfpathcurveto{\pgfqpoint{11.101448in}{0.450472in}}{\pgfqpoint{11.105838in}{0.461071in}}{\pgfqpoint{11.105838in}{0.472121in}}%
\pgfpathcurveto{\pgfqpoint{11.105838in}{0.483171in}}{\pgfqpoint{11.101448in}{0.493770in}}{\pgfqpoint{11.093634in}{0.501584in}}%
\pgfpathcurveto{\pgfqpoint{11.085821in}{0.509397in}}{\pgfqpoint{11.075222in}{0.513788in}}{\pgfqpoint{11.064171in}{0.513788in}}%
\pgfpathcurveto{\pgfqpoint{11.053121in}{0.513788in}}{\pgfqpoint{11.042522in}{0.509397in}}{\pgfqpoint{11.034709in}{0.501584in}}%
\pgfpathcurveto{\pgfqpoint{11.026895in}{0.493770in}}{\pgfqpoint{11.022505in}{0.483171in}}{\pgfqpoint{11.022505in}{0.472121in}}%
\pgfpathcurveto{\pgfqpoint{11.022505in}{0.461071in}}{\pgfqpoint{11.026895in}{0.450472in}}{\pgfqpoint{11.034709in}{0.442658in}}%
\pgfpathcurveto{\pgfqpoint{11.042522in}{0.434845in}}{\pgfqpoint{11.053121in}{0.430454in}}{\pgfqpoint{11.064171in}{0.430454in}}%
\pgfpathlineto{\pgfqpoint{11.064171in}{0.430454in}}%
\pgfpathclose%
\pgfusepath{stroke}%
\end{pgfscope}%
\begin{pgfscope}%
\pgfpathrectangle{\pgfqpoint{7.512535in}{0.437222in}}{\pgfqpoint{6.275590in}{5.159444in}}%
\pgfusepath{clip}%
\pgfsetbuttcap%
\pgfsetroundjoin%
\pgfsetlinewidth{1.003750pt}%
\definecolor{currentstroke}{rgb}{0.827451,0.827451,0.827451}%
\pgfsetstrokecolor{currentstroke}%
\pgfsetstrokeopacity{0.800000}%
\pgfsetdash{}{0pt}%
\pgfpathmoveto{\pgfqpoint{13.029213in}{1.698913in}}%
\pgfpathcurveto{\pgfqpoint{13.040263in}{1.698913in}}{\pgfqpoint{13.050862in}{1.703303in}}{\pgfqpoint{13.058675in}{1.711116in}}%
\pgfpathcurveto{\pgfqpoint{13.066489in}{1.718930in}}{\pgfqpoint{13.070879in}{1.729529in}}{\pgfqpoint{13.070879in}{1.740579in}}%
\pgfpathcurveto{\pgfqpoint{13.070879in}{1.751629in}}{\pgfqpoint{13.066489in}{1.762228in}}{\pgfqpoint{13.058675in}{1.770042in}}%
\pgfpathcurveto{\pgfqpoint{13.050862in}{1.777856in}}{\pgfqpoint{13.040263in}{1.782246in}}{\pgfqpoint{13.029213in}{1.782246in}}%
\pgfpathcurveto{\pgfqpoint{13.018162in}{1.782246in}}{\pgfqpoint{13.007563in}{1.777856in}}{\pgfqpoint{12.999750in}{1.770042in}}%
\pgfpathcurveto{\pgfqpoint{12.991936in}{1.762228in}}{\pgfqpoint{12.987546in}{1.751629in}}{\pgfqpoint{12.987546in}{1.740579in}}%
\pgfpathcurveto{\pgfqpoint{12.987546in}{1.729529in}}{\pgfqpoint{12.991936in}{1.718930in}}{\pgfqpoint{12.999750in}{1.711116in}}%
\pgfpathcurveto{\pgfqpoint{13.007563in}{1.703303in}}{\pgfqpoint{13.018162in}{1.698913in}}{\pgfqpoint{13.029213in}{1.698913in}}%
\pgfpathlineto{\pgfqpoint{13.029213in}{1.698913in}}%
\pgfpathclose%
\pgfusepath{stroke}%
\end{pgfscope}%
\begin{pgfscope}%
\pgfpathrectangle{\pgfqpoint{7.512535in}{0.437222in}}{\pgfqpoint{6.275590in}{5.159444in}}%
\pgfusepath{clip}%
\pgfsetbuttcap%
\pgfsetroundjoin%
\pgfsetlinewidth{1.003750pt}%
\definecolor{currentstroke}{rgb}{0.827451,0.827451,0.827451}%
\pgfsetstrokecolor{currentstroke}%
\pgfsetstrokeopacity{0.800000}%
\pgfsetdash{}{0pt}%
\pgfpathmoveto{\pgfqpoint{13.573729in}{1.154803in}}%
\pgfpathcurveto{\pgfqpoint{13.584779in}{1.154803in}}{\pgfqpoint{13.595378in}{1.159194in}}{\pgfqpoint{13.603192in}{1.167007in}}%
\pgfpathcurveto{\pgfqpoint{13.611005in}{1.174821in}}{\pgfqpoint{13.615396in}{1.185420in}}{\pgfqpoint{13.615396in}{1.196470in}}%
\pgfpathcurveto{\pgfqpoint{13.615396in}{1.207520in}}{\pgfqpoint{13.611005in}{1.218119in}}{\pgfqpoint{13.603192in}{1.225933in}}%
\pgfpathcurveto{\pgfqpoint{13.595378in}{1.233746in}}{\pgfqpoint{13.584779in}{1.238137in}}{\pgfqpoint{13.573729in}{1.238137in}}%
\pgfpathcurveto{\pgfqpoint{13.562679in}{1.238137in}}{\pgfqpoint{13.552080in}{1.233746in}}{\pgfqpoint{13.544266in}{1.225933in}}%
\pgfpathcurveto{\pgfqpoint{13.536453in}{1.218119in}}{\pgfqpoint{13.532062in}{1.207520in}}{\pgfqpoint{13.532062in}{1.196470in}}%
\pgfpathcurveto{\pgfqpoint{13.532062in}{1.185420in}}{\pgfqpoint{13.536453in}{1.174821in}}{\pgfqpoint{13.544266in}{1.167007in}}%
\pgfpathcurveto{\pgfqpoint{13.552080in}{1.159194in}}{\pgfqpoint{13.562679in}{1.154803in}}{\pgfqpoint{13.573729in}{1.154803in}}%
\pgfpathlineto{\pgfqpoint{13.573729in}{1.154803in}}%
\pgfpathclose%
\pgfusepath{stroke}%
\end{pgfscope}%
\begin{pgfscope}%
\pgfpathrectangle{\pgfqpoint{7.512535in}{0.437222in}}{\pgfqpoint{6.275590in}{5.159444in}}%
\pgfusepath{clip}%
\pgfsetbuttcap%
\pgfsetroundjoin%
\pgfsetlinewidth{1.003750pt}%
\definecolor{currentstroke}{rgb}{0.827451,0.827451,0.827451}%
\pgfsetstrokecolor{currentstroke}%
\pgfsetstrokeopacity{0.800000}%
\pgfsetdash{}{0pt}%
\pgfpathmoveto{\pgfqpoint{13.754575in}{1.321665in}}%
\pgfpathcurveto{\pgfqpoint{13.765625in}{1.321665in}}{\pgfqpoint{13.776224in}{1.326055in}}{\pgfqpoint{13.784037in}{1.333869in}}%
\pgfpathcurveto{\pgfqpoint{13.791851in}{1.341682in}}{\pgfqpoint{13.796241in}{1.352281in}}{\pgfqpoint{13.796241in}{1.363332in}}%
\pgfpathcurveto{\pgfqpoint{13.796241in}{1.374382in}}{\pgfqpoint{13.791851in}{1.384981in}}{\pgfqpoint{13.784037in}{1.392794in}}%
\pgfpathcurveto{\pgfqpoint{13.776224in}{1.400608in}}{\pgfqpoint{13.765625in}{1.404998in}}{\pgfqpoint{13.754575in}{1.404998in}}%
\pgfpathcurveto{\pgfqpoint{13.743524in}{1.404998in}}{\pgfqpoint{13.732925in}{1.400608in}}{\pgfqpoint{13.725112in}{1.392794in}}%
\pgfpathcurveto{\pgfqpoint{13.717298in}{1.384981in}}{\pgfqpoint{13.712908in}{1.374382in}}{\pgfqpoint{13.712908in}{1.363332in}}%
\pgfpathcurveto{\pgfqpoint{13.712908in}{1.352281in}}{\pgfqpoint{13.717298in}{1.341682in}}{\pgfqpoint{13.725112in}{1.333869in}}%
\pgfpathcurveto{\pgfqpoint{13.732925in}{1.326055in}}{\pgfqpoint{13.743524in}{1.321665in}}{\pgfqpoint{13.754575in}{1.321665in}}%
\pgfpathlineto{\pgfqpoint{13.754575in}{1.321665in}}%
\pgfpathclose%
\pgfusepath{stroke}%
\end{pgfscope}%
\begin{pgfscope}%
\pgfpathrectangle{\pgfqpoint{7.512535in}{0.437222in}}{\pgfqpoint{6.275590in}{5.159444in}}%
\pgfusepath{clip}%
\pgfsetbuttcap%
\pgfsetroundjoin%
\pgfsetlinewidth{1.003750pt}%
\definecolor{currentstroke}{rgb}{0.827451,0.827451,0.827451}%
\pgfsetstrokecolor{currentstroke}%
\pgfsetstrokeopacity{0.800000}%
\pgfsetdash{}{0pt}%
\pgfpathmoveto{\pgfqpoint{7.551497in}{1.946700in}}%
\pgfpathcurveto{\pgfqpoint{7.562547in}{1.946700in}}{\pgfqpoint{7.573146in}{1.951090in}}{\pgfqpoint{7.580960in}{1.958904in}}%
\pgfpathcurveto{\pgfqpoint{7.588773in}{1.966717in}}{\pgfqpoint{7.593163in}{1.977316in}}{\pgfqpoint{7.593163in}{1.988366in}}%
\pgfpathcurveto{\pgfqpoint{7.593163in}{1.999417in}}{\pgfqpoint{7.588773in}{2.010016in}}{\pgfqpoint{7.580960in}{2.017829in}}%
\pgfpathcurveto{\pgfqpoint{7.573146in}{2.025643in}}{\pgfqpoint{7.562547in}{2.030033in}}{\pgfqpoint{7.551497in}{2.030033in}}%
\pgfpathcurveto{\pgfqpoint{7.540447in}{2.030033in}}{\pgfqpoint{7.529848in}{2.025643in}}{\pgfqpoint{7.522034in}{2.017829in}}%
\pgfpathcurveto{\pgfqpoint{7.514220in}{2.010016in}}{\pgfqpoint{7.509830in}{1.999417in}}{\pgfqpoint{7.509830in}{1.988366in}}%
\pgfpathcurveto{\pgfqpoint{7.509830in}{1.977316in}}{\pgfqpoint{7.514220in}{1.966717in}}{\pgfqpoint{7.522034in}{1.958904in}}%
\pgfpathcurveto{\pgfqpoint{7.529848in}{1.951090in}}{\pgfqpoint{7.540447in}{1.946700in}}{\pgfqpoint{7.551497in}{1.946700in}}%
\pgfpathlineto{\pgfqpoint{7.551497in}{1.946700in}}%
\pgfpathclose%
\pgfusepath{stroke}%
\end{pgfscope}%
\begin{pgfscope}%
\pgfpathrectangle{\pgfqpoint{7.512535in}{0.437222in}}{\pgfqpoint{6.275590in}{5.159444in}}%
\pgfusepath{clip}%
\pgfsetbuttcap%
\pgfsetroundjoin%
\pgfsetlinewidth{1.003750pt}%
\definecolor{currentstroke}{rgb}{0.827451,0.827451,0.827451}%
\pgfsetstrokecolor{currentstroke}%
\pgfsetstrokeopacity{0.800000}%
\pgfsetdash{}{0pt}%
\pgfpathmoveto{\pgfqpoint{7.784419in}{3.292238in}}%
\pgfpathcurveto{\pgfqpoint{7.795469in}{3.292238in}}{\pgfqpoint{7.806068in}{3.296629in}}{\pgfqpoint{7.813882in}{3.304442in}}%
\pgfpathcurveto{\pgfqpoint{7.821695in}{3.312256in}}{\pgfqpoint{7.826085in}{3.322855in}}{\pgfqpoint{7.826085in}{3.333905in}}%
\pgfpathcurveto{\pgfqpoint{7.826085in}{3.344955in}}{\pgfqpoint{7.821695in}{3.355554in}}{\pgfqpoint{7.813882in}{3.363368in}}%
\pgfpathcurveto{\pgfqpoint{7.806068in}{3.371181in}}{\pgfqpoint{7.795469in}{3.375572in}}{\pgfqpoint{7.784419in}{3.375572in}}%
\pgfpathcurveto{\pgfqpoint{7.773369in}{3.375572in}}{\pgfqpoint{7.762770in}{3.371181in}}{\pgfqpoint{7.754956in}{3.363368in}}%
\pgfpathcurveto{\pgfqpoint{7.747142in}{3.355554in}}{\pgfqpoint{7.742752in}{3.344955in}}{\pgfqpoint{7.742752in}{3.333905in}}%
\pgfpathcurveto{\pgfqpoint{7.742752in}{3.322855in}}{\pgfqpoint{7.747142in}{3.312256in}}{\pgfqpoint{7.754956in}{3.304442in}}%
\pgfpathcurveto{\pgfqpoint{7.762770in}{3.296629in}}{\pgfqpoint{7.773369in}{3.292238in}}{\pgfqpoint{7.784419in}{3.292238in}}%
\pgfpathlineto{\pgfqpoint{7.784419in}{3.292238in}}%
\pgfpathclose%
\pgfusepath{stroke}%
\end{pgfscope}%
\begin{pgfscope}%
\pgfpathrectangle{\pgfqpoint{7.512535in}{0.437222in}}{\pgfqpoint{6.275590in}{5.159444in}}%
\pgfusepath{clip}%
\pgfsetbuttcap%
\pgfsetroundjoin%
\pgfsetlinewidth{1.003750pt}%
\definecolor{currentstroke}{rgb}{0.827451,0.827451,0.827451}%
\pgfsetstrokecolor{currentstroke}%
\pgfsetstrokeopacity{0.800000}%
\pgfsetdash{}{0pt}%
\pgfpathmoveto{\pgfqpoint{7.870969in}{3.849400in}}%
\pgfpathcurveto{\pgfqpoint{7.882019in}{3.849400in}}{\pgfqpoint{7.892618in}{3.853790in}}{\pgfqpoint{7.900432in}{3.861604in}}%
\pgfpathcurveto{\pgfqpoint{7.908245in}{3.869417in}}{\pgfqpoint{7.912636in}{3.880016in}}{\pgfqpoint{7.912636in}{3.891067in}}%
\pgfpathcurveto{\pgfqpoint{7.912636in}{3.902117in}}{\pgfqpoint{7.908245in}{3.912716in}}{\pgfqpoint{7.900432in}{3.920529in}}%
\pgfpathcurveto{\pgfqpoint{7.892618in}{3.928343in}}{\pgfqpoint{7.882019in}{3.932733in}}{\pgfqpoint{7.870969in}{3.932733in}}%
\pgfpathcurveto{\pgfqpoint{7.859919in}{3.932733in}}{\pgfqpoint{7.849320in}{3.928343in}}{\pgfqpoint{7.841506in}{3.920529in}}%
\pgfpathcurveto{\pgfqpoint{7.833693in}{3.912716in}}{\pgfqpoint{7.829302in}{3.902117in}}{\pgfqpoint{7.829302in}{3.891067in}}%
\pgfpathcurveto{\pgfqpoint{7.829302in}{3.880016in}}{\pgfqpoint{7.833693in}{3.869417in}}{\pgfqpoint{7.841506in}{3.861604in}}%
\pgfpathcurveto{\pgfqpoint{7.849320in}{3.853790in}}{\pgfqpoint{7.859919in}{3.849400in}}{\pgfqpoint{7.870969in}{3.849400in}}%
\pgfpathlineto{\pgfqpoint{7.870969in}{3.849400in}}%
\pgfpathclose%
\pgfusepath{stroke}%
\end{pgfscope}%
\begin{pgfscope}%
\pgfpathrectangle{\pgfqpoint{7.512535in}{0.437222in}}{\pgfqpoint{6.275590in}{5.159444in}}%
\pgfusepath{clip}%
\pgfsetbuttcap%
\pgfsetroundjoin%
\pgfsetlinewidth{1.003750pt}%
\definecolor{currentstroke}{rgb}{0.827451,0.827451,0.827451}%
\pgfsetstrokecolor{currentstroke}%
\pgfsetstrokeopacity{0.800000}%
\pgfsetdash{}{0pt}%
\pgfpathmoveto{\pgfqpoint{7.698587in}{3.459716in}}%
\pgfpathcurveto{\pgfqpoint{7.709637in}{3.459716in}}{\pgfqpoint{7.720236in}{3.464106in}}{\pgfqpoint{7.728050in}{3.471920in}}%
\pgfpathcurveto{\pgfqpoint{7.735864in}{3.479734in}}{\pgfqpoint{7.740254in}{3.490333in}}{\pgfqpoint{7.740254in}{3.501383in}}%
\pgfpathcurveto{\pgfqpoint{7.740254in}{3.512433in}}{\pgfqpoint{7.735864in}{3.523032in}}{\pgfqpoint{7.728050in}{3.530845in}}%
\pgfpathcurveto{\pgfqpoint{7.720236in}{3.538659in}}{\pgfqpoint{7.709637in}{3.543049in}}{\pgfqpoint{7.698587in}{3.543049in}}%
\pgfpathcurveto{\pgfqpoint{7.687537in}{3.543049in}}{\pgfqpoint{7.676938in}{3.538659in}}{\pgfqpoint{7.669124in}{3.530845in}}%
\pgfpathcurveto{\pgfqpoint{7.661311in}{3.523032in}}{\pgfqpoint{7.656920in}{3.512433in}}{\pgfqpoint{7.656920in}{3.501383in}}%
\pgfpathcurveto{\pgfqpoint{7.656920in}{3.490333in}}{\pgfqpoint{7.661311in}{3.479734in}}{\pgfqpoint{7.669124in}{3.471920in}}%
\pgfpathcurveto{\pgfqpoint{7.676938in}{3.464106in}}{\pgfqpoint{7.687537in}{3.459716in}}{\pgfqpoint{7.698587in}{3.459716in}}%
\pgfpathlineto{\pgfqpoint{7.698587in}{3.459716in}}%
\pgfpathclose%
\pgfusepath{stroke}%
\end{pgfscope}%
\begin{pgfscope}%
\pgfpathrectangle{\pgfqpoint{7.512535in}{0.437222in}}{\pgfqpoint{6.275590in}{5.159444in}}%
\pgfusepath{clip}%
\pgfsetbuttcap%
\pgfsetroundjoin%
\pgfsetlinewidth{1.003750pt}%
\definecolor{currentstroke}{rgb}{0.827451,0.827451,0.827451}%
\pgfsetstrokecolor{currentstroke}%
\pgfsetstrokeopacity{0.800000}%
\pgfsetdash{}{0pt}%
\pgfpathmoveto{\pgfqpoint{7.809717in}{3.877566in}}%
\pgfpathcurveto{\pgfqpoint{7.820767in}{3.877566in}}{\pgfqpoint{7.831366in}{3.881957in}}{\pgfqpoint{7.839180in}{3.889770in}}%
\pgfpathcurveto{\pgfqpoint{7.846993in}{3.897584in}}{\pgfqpoint{7.851384in}{3.908183in}}{\pgfqpoint{7.851384in}{3.919233in}}%
\pgfpathcurveto{\pgfqpoint{7.851384in}{3.930283in}}{\pgfqpoint{7.846993in}{3.940882in}}{\pgfqpoint{7.839180in}{3.948696in}}%
\pgfpathcurveto{\pgfqpoint{7.831366in}{3.956509in}}{\pgfqpoint{7.820767in}{3.960900in}}{\pgfqpoint{7.809717in}{3.960900in}}%
\pgfpathcurveto{\pgfqpoint{7.798667in}{3.960900in}}{\pgfqpoint{7.788068in}{3.956509in}}{\pgfqpoint{7.780254in}{3.948696in}}%
\pgfpathcurveto{\pgfqpoint{7.772441in}{3.940882in}}{\pgfqpoint{7.768050in}{3.930283in}}{\pgfqpoint{7.768050in}{3.919233in}}%
\pgfpathcurveto{\pgfqpoint{7.768050in}{3.908183in}}{\pgfqpoint{7.772441in}{3.897584in}}{\pgfqpoint{7.780254in}{3.889770in}}%
\pgfpathcurveto{\pgfqpoint{7.788068in}{3.881957in}}{\pgfqpoint{7.798667in}{3.877566in}}{\pgfqpoint{7.809717in}{3.877566in}}%
\pgfpathlineto{\pgfqpoint{7.809717in}{3.877566in}}%
\pgfpathclose%
\pgfusepath{stroke}%
\end{pgfscope}%
\begin{pgfscope}%
\pgfpathrectangle{\pgfqpoint{7.512535in}{0.437222in}}{\pgfqpoint{6.275590in}{5.159444in}}%
\pgfusepath{clip}%
\pgfsetbuttcap%
\pgfsetroundjoin%
\pgfsetlinewidth{1.003750pt}%
\definecolor{currentstroke}{rgb}{0.827451,0.827451,0.827451}%
\pgfsetstrokecolor{currentstroke}%
\pgfsetstrokeopacity{0.800000}%
\pgfsetdash{}{0pt}%
\pgfpathmoveto{\pgfqpoint{8.228364in}{5.301206in}}%
\pgfpathcurveto{\pgfqpoint{8.239414in}{5.301206in}}{\pgfqpoint{8.250013in}{5.305596in}}{\pgfqpoint{8.257827in}{5.313410in}}%
\pgfpathcurveto{\pgfqpoint{8.265641in}{5.321224in}}{\pgfqpoint{8.270031in}{5.331823in}}{\pgfqpoint{8.270031in}{5.342873in}}%
\pgfpathcurveto{\pgfqpoint{8.270031in}{5.353923in}}{\pgfqpoint{8.265641in}{5.364522in}}{\pgfqpoint{8.257827in}{5.372336in}}%
\pgfpathcurveto{\pgfqpoint{8.250013in}{5.380149in}}{\pgfqpoint{8.239414in}{5.384539in}}{\pgfqpoint{8.228364in}{5.384539in}}%
\pgfpathcurveto{\pgfqpoint{8.217314in}{5.384539in}}{\pgfqpoint{8.206715in}{5.380149in}}{\pgfqpoint{8.198902in}{5.372336in}}%
\pgfpathcurveto{\pgfqpoint{8.191088in}{5.364522in}}{\pgfqpoint{8.186698in}{5.353923in}}{\pgfqpoint{8.186698in}{5.342873in}}%
\pgfpathcurveto{\pgfqpoint{8.186698in}{5.331823in}}{\pgfqpoint{8.191088in}{5.321224in}}{\pgfqpoint{8.198902in}{5.313410in}}%
\pgfpathcurveto{\pgfqpoint{8.206715in}{5.305596in}}{\pgfqpoint{8.217314in}{5.301206in}}{\pgfqpoint{8.228364in}{5.301206in}}%
\pgfpathlineto{\pgfqpoint{8.228364in}{5.301206in}}%
\pgfpathclose%
\pgfusepath{stroke}%
\end{pgfscope}%
\begin{pgfscope}%
\pgfpathrectangle{\pgfqpoint{7.512535in}{0.437222in}}{\pgfqpoint{6.275590in}{5.159444in}}%
\pgfusepath{clip}%
\pgfsetbuttcap%
\pgfsetroundjoin%
\pgfsetlinewidth{1.003750pt}%
\definecolor{currentstroke}{rgb}{0.827451,0.827451,0.827451}%
\pgfsetstrokecolor{currentstroke}%
\pgfsetstrokeopacity{0.800000}%
\pgfsetdash{}{0pt}%
\pgfpathmoveto{\pgfqpoint{7.571203in}{5.042402in}}%
\pgfpathcurveto{\pgfqpoint{7.582253in}{5.042402in}}{\pgfqpoint{7.592852in}{5.046792in}}{\pgfqpoint{7.600665in}{5.054606in}}%
\pgfpathcurveto{\pgfqpoint{7.608479in}{5.062420in}}{\pgfqpoint{7.612869in}{5.073019in}}{\pgfqpoint{7.612869in}{5.084069in}}%
\pgfpathcurveto{\pgfqpoint{7.612869in}{5.095119in}}{\pgfqpoint{7.608479in}{5.105718in}}{\pgfqpoint{7.600665in}{5.113532in}}%
\pgfpathcurveto{\pgfqpoint{7.592852in}{5.121345in}}{\pgfqpoint{7.582253in}{5.125735in}}{\pgfqpoint{7.571203in}{5.125735in}}%
\pgfpathcurveto{\pgfqpoint{7.560152in}{5.125735in}}{\pgfqpoint{7.549553in}{5.121345in}}{\pgfqpoint{7.541740in}{5.113532in}}%
\pgfpathcurveto{\pgfqpoint{7.533926in}{5.105718in}}{\pgfqpoint{7.529536in}{5.095119in}}{\pgfqpoint{7.529536in}{5.084069in}}%
\pgfpathcurveto{\pgfqpoint{7.529536in}{5.073019in}}{\pgfqpoint{7.533926in}{5.062420in}}{\pgfqpoint{7.541740in}{5.054606in}}%
\pgfpathcurveto{\pgfqpoint{7.549553in}{5.046792in}}{\pgfqpoint{7.560152in}{5.042402in}}{\pgfqpoint{7.571203in}{5.042402in}}%
\pgfpathlineto{\pgfqpoint{7.571203in}{5.042402in}}%
\pgfpathclose%
\pgfusepath{stroke}%
\end{pgfscope}%
\begin{pgfscope}%
\pgfpathrectangle{\pgfqpoint{7.512535in}{0.437222in}}{\pgfqpoint{6.275590in}{5.159444in}}%
\pgfusepath{clip}%
\pgfsetbuttcap%
\pgfsetroundjoin%
\pgfsetlinewidth{1.003750pt}%
\definecolor{currentstroke}{rgb}{0.827451,0.827451,0.827451}%
\pgfsetstrokecolor{currentstroke}%
\pgfsetstrokeopacity{0.800000}%
\pgfsetdash{}{0pt}%
\pgfpathmoveto{\pgfqpoint{8.714432in}{3.605291in}}%
\pgfpathcurveto{\pgfqpoint{8.725482in}{3.605291in}}{\pgfqpoint{8.736081in}{3.609681in}}{\pgfqpoint{8.743895in}{3.617495in}}%
\pgfpathcurveto{\pgfqpoint{8.751709in}{3.625309in}}{\pgfqpoint{8.756099in}{3.635908in}}{\pgfqpoint{8.756099in}{3.646958in}}%
\pgfpathcurveto{\pgfqpoint{8.756099in}{3.658008in}}{\pgfqpoint{8.751709in}{3.668607in}}{\pgfqpoint{8.743895in}{3.676421in}}%
\pgfpathcurveto{\pgfqpoint{8.736081in}{3.684234in}}{\pgfqpoint{8.725482in}{3.688625in}}{\pgfqpoint{8.714432in}{3.688625in}}%
\pgfpathcurveto{\pgfqpoint{8.703382in}{3.688625in}}{\pgfqpoint{8.692783in}{3.684234in}}{\pgfqpoint{8.684970in}{3.676421in}}%
\pgfpathcurveto{\pgfqpoint{8.677156in}{3.668607in}}{\pgfqpoint{8.672766in}{3.658008in}}{\pgfqpoint{8.672766in}{3.646958in}}%
\pgfpathcurveto{\pgfqpoint{8.672766in}{3.635908in}}{\pgfqpoint{8.677156in}{3.625309in}}{\pgfqpoint{8.684970in}{3.617495in}}%
\pgfpathcurveto{\pgfqpoint{8.692783in}{3.609681in}}{\pgfqpoint{8.703382in}{3.605291in}}{\pgfqpoint{8.714432in}{3.605291in}}%
\pgfpathlineto{\pgfqpoint{8.714432in}{3.605291in}}%
\pgfpathclose%
\pgfusepath{stroke}%
\end{pgfscope}%
\begin{pgfscope}%
\pgfpathrectangle{\pgfqpoint{7.512535in}{0.437222in}}{\pgfqpoint{6.275590in}{5.159444in}}%
\pgfusepath{clip}%
\pgfsetbuttcap%
\pgfsetroundjoin%
\pgfsetlinewidth{1.003750pt}%
\definecolor{currentstroke}{rgb}{0.827451,0.827451,0.827451}%
\pgfsetstrokecolor{currentstroke}%
\pgfsetstrokeopacity{0.800000}%
\pgfsetdash{}{0pt}%
\pgfpathmoveto{\pgfqpoint{10.259535in}{4.447570in}}%
\pgfpathcurveto{\pgfqpoint{10.270585in}{4.447570in}}{\pgfqpoint{10.281184in}{4.451960in}}{\pgfqpoint{10.288997in}{4.459773in}}%
\pgfpathcurveto{\pgfqpoint{10.296811in}{4.467587in}}{\pgfqpoint{10.301201in}{4.478186in}}{\pgfqpoint{10.301201in}{4.489236in}}%
\pgfpathcurveto{\pgfqpoint{10.301201in}{4.500286in}}{\pgfqpoint{10.296811in}{4.510885in}}{\pgfqpoint{10.288997in}{4.518699in}}%
\pgfpathcurveto{\pgfqpoint{10.281184in}{4.526513in}}{\pgfqpoint{10.270585in}{4.530903in}}{\pgfqpoint{10.259535in}{4.530903in}}%
\pgfpathcurveto{\pgfqpoint{10.248485in}{4.530903in}}{\pgfqpoint{10.237886in}{4.526513in}}{\pgfqpoint{10.230072in}{4.518699in}}%
\pgfpathcurveto{\pgfqpoint{10.222258in}{4.510885in}}{\pgfqpoint{10.217868in}{4.500286in}}{\pgfqpoint{10.217868in}{4.489236in}}%
\pgfpathcurveto{\pgfqpoint{10.217868in}{4.478186in}}{\pgfqpoint{10.222258in}{4.467587in}}{\pgfqpoint{10.230072in}{4.459773in}}%
\pgfpathcurveto{\pgfqpoint{10.237886in}{4.451960in}}{\pgfqpoint{10.248485in}{4.447570in}}{\pgfqpoint{10.259535in}{4.447570in}}%
\pgfpathlineto{\pgfqpoint{10.259535in}{4.447570in}}%
\pgfpathclose%
\pgfusepath{stroke}%
\end{pgfscope}%
\begin{pgfscope}%
\pgfpathrectangle{\pgfqpoint{7.512535in}{0.437222in}}{\pgfqpoint{6.275590in}{5.159444in}}%
\pgfusepath{clip}%
\pgfsetbuttcap%
\pgfsetroundjoin%
\pgfsetlinewidth{1.003750pt}%
\definecolor{currentstroke}{rgb}{0.827451,0.827451,0.827451}%
\pgfsetstrokecolor{currentstroke}%
\pgfsetstrokeopacity{0.800000}%
\pgfsetdash{}{0pt}%
\pgfpathmoveto{\pgfqpoint{9.247511in}{4.532771in}}%
\pgfpathcurveto{\pgfqpoint{9.258561in}{4.532771in}}{\pgfqpoint{9.269160in}{4.537161in}}{\pgfqpoint{9.276974in}{4.544974in}}%
\pgfpathcurveto{\pgfqpoint{9.284788in}{4.552788in}}{\pgfqpoint{9.289178in}{4.563387in}}{\pgfqpoint{9.289178in}{4.574437in}}%
\pgfpathcurveto{\pgfqpoint{9.289178in}{4.585487in}}{\pgfqpoint{9.284788in}{4.596086in}}{\pgfqpoint{9.276974in}{4.603900in}}%
\pgfpathcurveto{\pgfqpoint{9.269160in}{4.611714in}}{\pgfqpoint{9.258561in}{4.616104in}}{\pgfqpoint{9.247511in}{4.616104in}}%
\pgfpathcurveto{\pgfqpoint{9.236461in}{4.616104in}}{\pgfqpoint{9.225862in}{4.611714in}}{\pgfqpoint{9.218048in}{4.603900in}}%
\pgfpathcurveto{\pgfqpoint{9.210235in}{4.596086in}}{\pgfqpoint{9.205844in}{4.585487in}}{\pgfqpoint{9.205844in}{4.574437in}}%
\pgfpathcurveto{\pgfqpoint{9.205844in}{4.563387in}}{\pgfqpoint{9.210235in}{4.552788in}}{\pgfqpoint{9.218048in}{4.544974in}}%
\pgfpathcurveto{\pgfqpoint{9.225862in}{4.537161in}}{\pgfqpoint{9.236461in}{4.532771in}}{\pgfqpoint{9.247511in}{4.532771in}}%
\pgfpathlineto{\pgfqpoint{9.247511in}{4.532771in}}%
\pgfpathclose%
\pgfusepath{stroke}%
\end{pgfscope}%
\begin{pgfscope}%
\pgfpathrectangle{\pgfqpoint{7.512535in}{0.437222in}}{\pgfqpoint{6.275590in}{5.159444in}}%
\pgfusepath{clip}%
\pgfsetbuttcap%
\pgfsetroundjoin%
\pgfsetlinewidth{1.003750pt}%
\definecolor{currentstroke}{rgb}{0.827451,0.827451,0.827451}%
\pgfsetstrokecolor{currentstroke}%
\pgfsetstrokeopacity{0.800000}%
\pgfsetdash{}{0pt}%
\pgfpathmoveto{\pgfqpoint{9.836077in}{3.291016in}}%
\pgfpathcurveto{\pgfqpoint{9.847127in}{3.291016in}}{\pgfqpoint{9.857726in}{3.295406in}}{\pgfqpoint{9.865540in}{3.303220in}}%
\pgfpathcurveto{\pgfqpoint{9.873353in}{3.311034in}}{\pgfqpoint{9.877743in}{3.321633in}}{\pgfqpoint{9.877743in}{3.332683in}}%
\pgfpathcurveto{\pgfqpoint{9.877743in}{3.343733in}}{\pgfqpoint{9.873353in}{3.354332in}}{\pgfqpoint{9.865540in}{3.362146in}}%
\pgfpathcurveto{\pgfqpoint{9.857726in}{3.369959in}}{\pgfqpoint{9.847127in}{3.374349in}}{\pgfqpoint{9.836077in}{3.374349in}}%
\pgfpathcurveto{\pgfqpoint{9.825027in}{3.374349in}}{\pgfqpoint{9.814428in}{3.369959in}}{\pgfqpoint{9.806614in}{3.362146in}}%
\pgfpathcurveto{\pgfqpoint{9.798800in}{3.354332in}}{\pgfqpoint{9.794410in}{3.343733in}}{\pgfqpoint{9.794410in}{3.332683in}}%
\pgfpathcurveto{\pgfqpoint{9.794410in}{3.321633in}}{\pgfqpoint{9.798800in}{3.311034in}}{\pgfqpoint{9.806614in}{3.303220in}}%
\pgfpathcurveto{\pgfqpoint{9.814428in}{3.295406in}}{\pgfqpoint{9.825027in}{3.291016in}}{\pgfqpoint{9.836077in}{3.291016in}}%
\pgfpathlineto{\pgfqpoint{9.836077in}{3.291016in}}%
\pgfpathclose%
\pgfusepath{stroke}%
\end{pgfscope}%
\begin{pgfscope}%
\pgfpathrectangle{\pgfqpoint{7.512535in}{0.437222in}}{\pgfqpoint{6.275590in}{5.159444in}}%
\pgfusepath{clip}%
\pgfsetbuttcap%
\pgfsetroundjoin%
\pgfsetlinewidth{1.003750pt}%
\definecolor{currentstroke}{rgb}{0.827451,0.827451,0.827451}%
\pgfsetstrokecolor{currentstroke}%
\pgfsetstrokeopacity{0.800000}%
\pgfsetdash{}{0pt}%
\pgfpathmoveto{\pgfqpoint{12.362478in}{4.949494in}}%
\pgfpathcurveto{\pgfqpoint{12.373528in}{4.949494in}}{\pgfqpoint{12.384127in}{4.953884in}}{\pgfqpoint{12.391941in}{4.961698in}}%
\pgfpathcurveto{\pgfqpoint{12.399754in}{4.969511in}}{\pgfqpoint{12.404145in}{4.980110in}}{\pgfqpoint{12.404145in}{4.991160in}}%
\pgfpathcurveto{\pgfqpoint{12.404145in}{5.002211in}}{\pgfqpoint{12.399754in}{5.012810in}}{\pgfqpoint{12.391941in}{5.020623in}}%
\pgfpathcurveto{\pgfqpoint{12.384127in}{5.028437in}}{\pgfqpoint{12.373528in}{5.032827in}}{\pgfqpoint{12.362478in}{5.032827in}}%
\pgfpathcurveto{\pgfqpoint{12.351428in}{5.032827in}}{\pgfqpoint{12.340829in}{5.028437in}}{\pgfqpoint{12.333015in}{5.020623in}}%
\pgfpathcurveto{\pgfqpoint{12.325202in}{5.012810in}}{\pgfqpoint{12.320811in}{5.002211in}}{\pgfqpoint{12.320811in}{4.991160in}}%
\pgfpathcurveto{\pgfqpoint{12.320811in}{4.980110in}}{\pgfqpoint{12.325202in}{4.969511in}}{\pgfqpoint{12.333015in}{4.961698in}}%
\pgfpathcurveto{\pgfqpoint{12.340829in}{4.953884in}}{\pgfqpoint{12.351428in}{4.949494in}}{\pgfqpoint{12.362478in}{4.949494in}}%
\pgfpathlineto{\pgfqpoint{12.362478in}{4.949494in}}%
\pgfpathclose%
\pgfusepath{stroke}%
\end{pgfscope}%
\begin{pgfscope}%
\pgfpathrectangle{\pgfqpoint{7.512535in}{0.437222in}}{\pgfqpoint{6.275590in}{5.159444in}}%
\pgfusepath{clip}%
\pgfsetbuttcap%
\pgfsetroundjoin%
\pgfsetlinewidth{1.003750pt}%
\definecolor{currentstroke}{rgb}{0.827451,0.827451,0.827451}%
\pgfsetstrokecolor{currentstroke}%
\pgfsetstrokeopacity{0.800000}%
\pgfsetdash{}{0pt}%
\pgfpathmoveto{\pgfqpoint{9.802410in}{3.570658in}}%
\pgfpathcurveto{\pgfqpoint{9.813460in}{3.570658in}}{\pgfqpoint{9.824059in}{3.575048in}}{\pgfqpoint{9.831872in}{3.582862in}}%
\pgfpathcurveto{\pgfqpoint{9.839686in}{3.590675in}}{\pgfqpoint{9.844076in}{3.601274in}}{\pgfqpoint{9.844076in}{3.612325in}}%
\pgfpathcurveto{\pgfqpoint{9.844076in}{3.623375in}}{\pgfqpoint{9.839686in}{3.633974in}}{\pgfqpoint{9.831872in}{3.641787in}}%
\pgfpathcurveto{\pgfqpoint{9.824059in}{3.649601in}}{\pgfqpoint{9.813460in}{3.653991in}}{\pgfqpoint{9.802410in}{3.653991in}}%
\pgfpathcurveto{\pgfqpoint{9.791360in}{3.653991in}}{\pgfqpoint{9.780760in}{3.649601in}}{\pgfqpoint{9.772947in}{3.641787in}}%
\pgfpathcurveto{\pgfqpoint{9.765133in}{3.633974in}}{\pgfqpoint{9.760743in}{3.623375in}}{\pgfqpoint{9.760743in}{3.612325in}}%
\pgfpathcurveto{\pgfqpoint{9.760743in}{3.601274in}}{\pgfqpoint{9.765133in}{3.590675in}}{\pgfqpoint{9.772947in}{3.582862in}}%
\pgfpathcurveto{\pgfqpoint{9.780760in}{3.575048in}}{\pgfqpoint{9.791360in}{3.570658in}}{\pgfqpoint{9.802410in}{3.570658in}}%
\pgfpathlineto{\pgfqpoint{9.802410in}{3.570658in}}%
\pgfpathclose%
\pgfusepath{stroke}%
\end{pgfscope}%
\begin{pgfscope}%
\pgfpathrectangle{\pgfqpoint{7.512535in}{0.437222in}}{\pgfqpoint{6.275590in}{5.159444in}}%
\pgfusepath{clip}%
\pgfsetbuttcap%
\pgfsetroundjoin%
\pgfsetlinewidth{1.003750pt}%
\definecolor{currentstroke}{rgb}{0.827451,0.827451,0.827451}%
\pgfsetstrokecolor{currentstroke}%
\pgfsetstrokeopacity{0.800000}%
\pgfsetdash{}{0pt}%
\pgfpathmoveto{\pgfqpoint{10.005215in}{4.464965in}}%
\pgfpathcurveto{\pgfqpoint{10.016265in}{4.464965in}}{\pgfqpoint{10.026865in}{4.469355in}}{\pgfqpoint{10.034678in}{4.477169in}}%
\pgfpathcurveto{\pgfqpoint{10.042492in}{4.484982in}}{\pgfqpoint{10.046882in}{4.495581in}}{\pgfqpoint{10.046882in}{4.506632in}}%
\pgfpathcurveto{\pgfqpoint{10.046882in}{4.517682in}}{\pgfqpoint{10.042492in}{4.528281in}}{\pgfqpoint{10.034678in}{4.536094in}}%
\pgfpathcurveto{\pgfqpoint{10.026865in}{4.543908in}}{\pgfqpoint{10.016265in}{4.548298in}}{\pgfqpoint{10.005215in}{4.548298in}}%
\pgfpathcurveto{\pgfqpoint{9.994165in}{4.548298in}}{\pgfqpoint{9.983566in}{4.543908in}}{\pgfqpoint{9.975753in}{4.536094in}}%
\pgfpathcurveto{\pgfqpoint{9.967939in}{4.528281in}}{\pgfqpoint{9.963549in}{4.517682in}}{\pgfqpoint{9.963549in}{4.506632in}}%
\pgfpathcurveto{\pgfqpoint{9.963549in}{4.495581in}}{\pgfqpoint{9.967939in}{4.484982in}}{\pgfqpoint{9.975753in}{4.477169in}}%
\pgfpathcurveto{\pgfqpoint{9.983566in}{4.469355in}}{\pgfqpoint{9.994165in}{4.464965in}}{\pgfqpoint{10.005215in}{4.464965in}}%
\pgfpathlineto{\pgfqpoint{10.005215in}{4.464965in}}%
\pgfpathclose%
\pgfusepath{stroke}%
\end{pgfscope}%
\begin{pgfscope}%
\pgfpathrectangle{\pgfqpoint{7.512535in}{0.437222in}}{\pgfqpoint{6.275590in}{5.159444in}}%
\pgfusepath{clip}%
\pgfsetbuttcap%
\pgfsetroundjoin%
\pgfsetlinewidth{1.003750pt}%
\definecolor{currentstroke}{rgb}{0.827451,0.827451,0.827451}%
\pgfsetstrokecolor{currentstroke}%
\pgfsetstrokeopacity{0.800000}%
\pgfsetdash{}{0pt}%
\pgfpathmoveto{\pgfqpoint{9.500868in}{3.326629in}}%
\pgfpathcurveto{\pgfqpoint{9.511919in}{3.326629in}}{\pgfqpoint{9.522518in}{3.331019in}}{\pgfqpoint{9.530331in}{3.338833in}}%
\pgfpathcurveto{\pgfqpoint{9.538145in}{3.346646in}}{\pgfqpoint{9.542535in}{3.357246in}}{\pgfqpoint{9.542535in}{3.368296in}}%
\pgfpathcurveto{\pgfqpoint{9.542535in}{3.379346in}}{\pgfqpoint{9.538145in}{3.389945in}}{\pgfqpoint{9.530331in}{3.397758in}}%
\pgfpathcurveto{\pgfqpoint{9.522518in}{3.405572in}}{\pgfqpoint{9.511919in}{3.409962in}}{\pgfqpoint{9.500868in}{3.409962in}}%
\pgfpathcurveto{\pgfqpoint{9.489818in}{3.409962in}}{\pgfqpoint{9.479219in}{3.405572in}}{\pgfqpoint{9.471406in}{3.397758in}}%
\pgfpathcurveto{\pgfqpoint{9.463592in}{3.389945in}}{\pgfqpoint{9.459202in}{3.379346in}}{\pgfqpoint{9.459202in}{3.368296in}}%
\pgfpathcurveto{\pgfqpoint{9.459202in}{3.357246in}}{\pgfqpoint{9.463592in}{3.346646in}}{\pgfqpoint{9.471406in}{3.338833in}}%
\pgfpathcurveto{\pgfqpoint{9.479219in}{3.331019in}}{\pgfqpoint{9.489818in}{3.326629in}}{\pgfqpoint{9.500868in}{3.326629in}}%
\pgfpathlineto{\pgfqpoint{9.500868in}{3.326629in}}%
\pgfpathclose%
\pgfusepath{stroke}%
\end{pgfscope}%
\begin{pgfscope}%
\pgfpathrectangle{\pgfqpoint{7.512535in}{0.437222in}}{\pgfqpoint{6.275590in}{5.159444in}}%
\pgfusepath{clip}%
\pgfsetbuttcap%
\pgfsetroundjoin%
\pgfsetlinewidth{1.003750pt}%
\definecolor{currentstroke}{rgb}{0.827451,0.827451,0.827451}%
\pgfsetstrokecolor{currentstroke}%
\pgfsetstrokeopacity{0.800000}%
\pgfsetdash{}{0pt}%
\pgfpathmoveto{\pgfqpoint{10.287229in}{5.087121in}}%
\pgfpathcurveto{\pgfqpoint{10.298279in}{5.087121in}}{\pgfqpoint{10.308878in}{5.091511in}}{\pgfqpoint{10.316692in}{5.099325in}}%
\pgfpathcurveto{\pgfqpoint{10.324505in}{5.107138in}}{\pgfqpoint{10.328896in}{5.117737in}}{\pgfqpoint{10.328896in}{5.128787in}}%
\pgfpathcurveto{\pgfqpoint{10.328896in}{5.139838in}}{\pgfqpoint{10.324505in}{5.150437in}}{\pgfqpoint{10.316692in}{5.158250in}}%
\pgfpathcurveto{\pgfqpoint{10.308878in}{5.166064in}}{\pgfqpoint{10.298279in}{5.170454in}}{\pgfqpoint{10.287229in}{5.170454in}}%
\pgfpathcurveto{\pgfqpoint{10.276179in}{5.170454in}}{\pgfqpoint{10.265580in}{5.166064in}}{\pgfqpoint{10.257766in}{5.158250in}}%
\pgfpathcurveto{\pgfqpoint{10.249953in}{5.150437in}}{\pgfqpoint{10.245562in}{5.139838in}}{\pgfqpoint{10.245562in}{5.128787in}}%
\pgfpathcurveto{\pgfqpoint{10.245562in}{5.117737in}}{\pgfqpoint{10.249953in}{5.107138in}}{\pgfqpoint{10.257766in}{5.099325in}}%
\pgfpathcurveto{\pgfqpoint{10.265580in}{5.091511in}}{\pgfqpoint{10.276179in}{5.087121in}}{\pgfqpoint{10.287229in}{5.087121in}}%
\pgfpathlineto{\pgfqpoint{10.287229in}{5.087121in}}%
\pgfpathclose%
\pgfusepath{stroke}%
\end{pgfscope}%
\begin{pgfscope}%
\pgfpathrectangle{\pgfqpoint{7.512535in}{0.437222in}}{\pgfqpoint{6.275590in}{5.159444in}}%
\pgfusepath{clip}%
\pgfsetbuttcap%
\pgfsetroundjoin%
\pgfsetlinewidth{1.003750pt}%
\definecolor{currentstroke}{rgb}{0.827451,0.827451,0.827451}%
\pgfsetstrokecolor{currentstroke}%
\pgfsetstrokeopacity{0.800000}%
\pgfsetdash{}{0pt}%
\pgfpathmoveto{\pgfqpoint{7.887873in}{1.346375in}}%
\pgfpathcurveto{\pgfqpoint{7.898923in}{1.346375in}}{\pgfqpoint{7.909522in}{1.350765in}}{\pgfqpoint{7.917336in}{1.358578in}}%
\pgfpathcurveto{\pgfqpoint{7.925149in}{1.366392in}}{\pgfqpoint{7.929540in}{1.376991in}}{\pgfqpoint{7.929540in}{1.388041in}}%
\pgfpathcurveto{\pgfqpoint{7.929540in}{1.399091in}}{\pgfqpoint{7.925149in}{1.409690in}}{\pgfqpoint{7.917336in}{1.417504in}}%
\pgfpathcurveto{\pgfqpoint{7.909522in}{1.425318in}}{\pgfqpoint{7.898923in}{1.429708in}}{\pgfqpoint{7.887873in}{1.429708in}}%
\pgfpathcurveto{\pgfqpoint{7.876823in}{1.429708in}}{\pgfqpoint{7.866224in}{1.425318in}}{\pgfqpoint{7.858410in}{1.417504in}}%
\pgfpathcurveto{\pgfqpoint{7.850596in}{1.409690in}}{\pgfqpoint{7.846206in}{1.399091in}}{\pgfqpoint{7.846206in}{1.388041in}}%
\pgfpathcurveto{\pgfqpoint{7.846206in}{1.376991in}}{\pgfqpoint{7.850596in}{1.366392in}}{\pgfqpoint{7.858410in}{1.358578in}}%
\pgfpathcurveto{\pgfqpoint{7.866224in}{1.350765in}}{\pgfqpoint{7.876823in}{1.346375in}}{\pgfqpoint{7.887873in}{1.346375in}}%
\pgfpathlineto{\pgfqpoint{7.887873in}{1.346375in}}%
\pgfpathclose%
\pgfusepath{stroke}%
\end{pgfscope}%
\begin{pgfscope}%
\pgfpathrectangle{\pgfqpoint{7.512535in}{0.437222in}}{\pgfqpoint{6.275590in}{5.159444in}}%
\pgfusepath{clip}%
\pgfsetbuttcap%
\pgfsetroundjoin%
\pgfsetlinewidth{1.003750pt}%
\definecolor{currentstroke}{rgb}{0.827451,0.827451,0.827451}%
\pgfsetstrokecolor{currentstroke}%
\pgfsetstrokeopacity{0.800000}%
\pgfsetdash{}{0pt}%
\pgfpathmoveto{\pgfqpoint{9.581399in}{2.990585in}}%
\pgfpathcurveto{\pgfqpoint{9.592449in}{2.990585in}}{\pgfqpoint{9.603048in}{2.994975in}}{\pgfqpoint{9.610862in}{3.002789in}}%
\pgfpathcurveto{\pgfqpoint{9.618675in}{3.010602in}}{\pgfqpoint{9.623065in}{3.021202in}}{\pgfqpoint{9.623065in}{3.032252in}}%
\pgfpathcurveto{\pgfqpoint{9.623065in}{3.043302in}}{\pgfqpoint{9.618675in}{3.053901in}}{\pgfqpoint{9.610862in}{3.061714in}}%
\pgfpathcurveto{\pgfqpoint{9.603048in}{3.069528in}}{\pgfqpoint{9.592449in}{3.073918in}}{\pgfqpoint{9.581399in}{3.073918in}}%
\pgfpathcurveto{\pgfqpoint{9.570349in}{3.073918in}}{\pgfqpoint{9.559750in}{3.069528in}}{\pgfqpoint{9.551936in}{3.061714in}}%
\pgfpathcurveto{\pgfqpoint{9.544122in}{3.053901in}}{\pgfqpoint{9.539732in}{3.043302in}}{\pgfqpoint{9.539732in}{3.032252in}}%
\pgfpathcurveto{\pgfqpoint{9.539732in}{3.021202in}}{\pgfqpoint{9.544122in}{3.010602in}}{\pgfqpoint{9.551936in}{3.002789in}}%
\pgfpathcurveto{\pgfqpoint{9.559750in}{2.994975in}}{\pgfqpoint{9.570349in}{2.990585in}}{\pgfqpoint{9.581399in}{2.990585in}}%
\pgfpathlineto{\pgfqpoint{9.581399in}{2.990585in}}%
\pgfpathclose%
\pgfusepath{stroke}%
\end{pgfscope}%
\begin{pgfscope}%
\pgfpathrectangle{\pgfqpoint{7.512535in}{0.437222in}}{\pgfqpoint{6.275590in}{5.159444in}}%
\pgfusepath{clip}%
\pgfsetbuttcap%
\pgfsetroundjoin%
\pgfsetlinewidth{1.003750pt}%
\definecolor{currentstroke}{rgb}{0.827451,0.827451,0.827451}%
\pgfsetstrokecolor{currentstroke}%
\pgfsetstrokeopacity{0.800000}%
\pgfsetdash{}{0pt}%
\pgfpathmoveto{\pgfqpoint{11.387346in}{5.101603in}}%
\pgfpathcurveto{\pgfqpoint{11.398396in}{5.101603in}}{\pgfqpoint{11.408995in}{5.105993in}}{\pgfqpoint{11.416809in}{5.113807in}}%
\pgfpathcurveto{\pgfqpoint{11.424622in}{5.121620in}}{\pgfqpoint{11.429013in}{5.132219in}}{\pgfqpoint{11.429013in}{5.143270in}}%
\pgfpathcurveto{\pgfqpoint{11.429013in}{5.154320in}}{\pgfqpoint{11.424622in}{5.164919in}}{\pgfqpoint{11.416809in}{5.172732in}}%
\pgfpathcurveto{\pgfqpoint{11.408995in}{5.180546in}}{\pgfqpoint{11.398396in}{5.184936in}}{\pgfqpoint{11.387346in}{5.184936in}}%
\pgfpathcurveto{\pgfqpoint{11.376296in}{5.184936in}}{\pgfqpoint{11.365697in}{5.180546in}}{\pgfqpoint{11.357883in}{5.172732in}}%
\pgfpathcurveto{\pgfqpoint{11.350070in}{5.164919in}}{\pgfqpoint{11.345679in}{5.154320in}}{\pgfqpoint{11.345679in}{5.143270in}}%
\pgfpathcurveto{\pgfqpoint{11.345679in}{5.132219in}}{\pgfqpoint{11.350070in}{5.121620in}}{\pgfqpoint{11.357883in}{5.113807in}}%
\pgfpathcurveto{\pgfqpoint{11.365697in}{5.105993in}}{\pgfqpoint{11.376296in}{5.101603in}}{\pgfqpoint{11.387346in}{5.101603in}}%
\pgfpathlineto{\pgfqpoint{11.387346in}{5.101603in}}%
\pgfpathclose%
\pgfusepath{stroke}%
\end{pgfscope}%
\begin{pgfscope}%
\pgfpathrectangle{\pgfqpoint{7.512535in}{0.437222in}}{\pgfqpoint{6.275590in}{5.159444in}}%
\pgfusepath{clip}%
\pgfsetbuttcap%
\pgfsetroundjoin%
\pgfsetlinewidth{1.003750pt}%
\definecolor{currentstroke}{rgb}{0.827451,0.827451,0.827451}%
\pgfsetstrokecolor{currentstroke}%
\pgfsetstrokeopacity{0.800000}%
\pgfsetdash{}{0pt}%
\pgfpathmoveto{\pgfqpoint{12.473540in}{5.514145in}}%
\pgfpathcurveto{\pgfqpoint{12.484590in}{5.514145in}}{\pgfqpoint{12.495189in}{5.518535in}}{\pgfqpoint{12.503003in}{5.526349in}}%
\pgfpathcurveto{\pgfqpoint{12.510817in}{5.534162in}}{\pgfqpoint{12.515207in}{5.544761in}}{\pgfqpoint{12.515207in}{5.555811in}}%
\pgfpathcurveto{\pgfqpoint{12.515207in}{5.566862in}}{\pgfqpoint{12.510817in}{5.577461in}}{\pgfqpoint{12.503003in}{5.585274in}}%
\pgfpathcurveto{\pgfqpoint{12.495189in}{5.593088in}}{\pgfqpoint{12.484590in}{5.597478in}}{\pgfqpoint{12.473540in}{5.597478in}}%
\pgfpathcurveto{\pgfqpoint{12.462490in}{5.597478in}}{\pgfqpoint{12.451891in}{5.593088in}}{\pgfqpoint{12.444078in}{5.585274in}}%
\pgfpathcurveto{\pgfqpoint{12.436264in}{5.577461in}}{\pgfqpoint{12.431874in}{5.566862in}}{\pgfqpoint{12.431874in}{5.555811in}}%
\pgfpathcurveto{\pgfqpoint{12.431874in}{5.544761in}}{\pgfqpoint{12.436264in}{5.534162in}}{\pgfqpoint{12.444078in}{5.526349in}}%
\pgfpathcurveto{\pgfqpoint{12.451891in}{5.518535in}}{\pgfqpoint{12.462490in}{5.514145in}}{\pgfqpoint{12.473540in}{5.514145in}}%
\pgfpathlineto{\pgfqpoint{12.473540in}{5.514145in}}%
\pgfpathclose%
\pgfusepath{stroke}%
\end{pgfscope}%
\begin{pgfscope}%
\pgfpathrectangle{\pgfqpoint{7.512535in}{0.437222in}}{\pgfqpoint{6.275590in}{5.159444in}}%
\pgfusepath{clip}%
\pgfsetbuttcap%
\pgfsetroundjoin%
\pgfsetlinewidth{1.003750pt}%
\definecolor{currentstroke}{rgb}{0.827451,0.827451,0.827451}%
\pgfsetstrokecolor{currentstroke}%
\pgfsetstrokeopacity{0.800000}%
\pgfsetdash{}{0pt}%
\pgfpathmoveto{\pgfqpoint{10.346895in}{5.462154in}}%
\pgfpathcurveto{\pgfqpoint{10.357945in}{5.462154in}}{\pgfqpoint{10.368544in}{5.466545in}}{\pgfqpoint{10.376358in}{5.474358in}}%
\pgfpathcurveto{\pgfqpoint{10.384171in}{5.482172in}}{\pgfqpoint{10.388562in}{5.492771in}}{\pgfqpoint{10.388562in}{5.503821in}}%
\pgfpathcurveto{\pgfqpoint{10.388562in}{5.514871in}}{\pgfqpoint{10.384171in}{5.525470in}}{\pgfqpoint{10.376358in}{5.533284in}}%
\pgfpathcurveto{\pgfqpoint{10.368544in}{5.541098in}}{\pgfqpoint{10.357945in}{5.545488in}}{\pgfqpoint{10.346895in}{5.545488in}}%
\pgfpathcurveto{\pgfqpoint{10.335845in}{5.545488in}}{\pgfqpoint{10.325246in}{5.541098in}}{\pgfqpoint{10.317432in}{5.533284in}}%
\pgfpathcurveto{\pgfqpoint{10.309619in}{5.525470in}}{\pgfqpoint{10.305228in}{5.514871in}}{\pgfqpoint{10.305228in}{5.503821in}}%
\pgfpathcurveto{\pgfqpoint{10.305228in}{5.492771in}}{\pgfqpoint{10.309619in}{5.482172in}}{\pgfqpoint{10.317432in}{5.474358in}}%
\pgfpathcurveto{\pgfqpoint{10.325246in}{5.466545in}}{\pgfqpoint{10.335845in}{5.462154in}}{\pgfqpoint{10.346895in}{5.462154in}}%
\pgfpathlineto{\pgfqpoint{10.346895in}{5.462154in}}%
\pgfpathclose%
\pgfusepath{stroke}%
\end{pgfscope}%
\begin{pgfscope}%
\pgfpathrectangle{\pgfqpoint{7.512535in}{0.437222in}}{\pgfqpoint{6.275590in}{5.159444in}}%
\pgfusepath{clip}%
\pgfsetbuttcap%
\pgfsetroundjoin%
\pgfsetlinewidth{1.003750pt}%
\definecolor{currentstroke}{rgb}{0.827451,0.827451,0.827451}%
\pgfsetstrokecolor{currentstroke}%
\pgfsetstrokeopacity{0.800000}%
\pgfsetdash{}{0pt}%
\pgfpathmoveto{\pgfqpoint{8.290348in}{1.011495in}}%
\pgfpathcurveto{\pgfqpoint{8.301398in}{1.011495in}}{\pgfqpoint{8.311997in}{1.015885in}}{\pgfqpoint{8.319811in}{1.023699in}}%
\pgfpathcurveto{\pgfqpoint{8.327625in}{1.031512in}}{\pgfqpoint{8.332015in}{1.042111in}}{\pgfqpoint{8.332015in}{1.053161in}}%
\pgfpathcurveto{\pgfqpoint{8.332015in}{1.064211in}}{\pgfqpoint{8.327625in}{1.074810in}}{\pgfqpoint{8.319811in}{1.082624in}}%
\pgfpathcurveto{\pgfqpoint{8.311997in}{1.090438in}}{\pgfqpoint{8.301398in}{1.094828in}}{\pgfqpoint{8.290348in}{1.094828in}}%
\pgfpathcurveto{\pgfqpoint{8.279298in}{1.094828in}}{\pgfqpoint{8.268699in}{1.090438in}}{\pgfqpoint{8.260886in}{1.082624in}}%
\pgfpathcurveto{\pgfqpoint{8.253072in}{1.074810in}}{\pgfqpoint{8.248682in}{1.064211in}}{\pgfqpoint{8.248682in}{1.053161in}}%
\pgfpathcurveto{\pgfqpoint{8.248682in}{1.042111in}}{\pgfqpoint{8.253072in}{1.031512in}}{\pgfqpoint{8.260886in}{1.023699in}}%
\pgfpathcurveto{\pgfqpoint{8.268699in}{1.015885in}}{\pgfqpoint{8.279298in}{1.011495in}}{\pgfqpoint{8.290348in}{1.011495in}}%
\pgfpathlineto{\pgfqpoint{8.290348in}{1.011495in}}%
\pgfpathclose%
\pgfusepath{stroke}%
\end{pgfscope}%
\begin{pgfscope}%
\pgfpathrectangle{\pgfqpoint{7.512535in}{0.437222in}}{\pgfqpoint{6.275590in}{5.159444in}}%
\pgfusepath{clip}%
\pgfsetbuttcap%
\pgfsetroundjoin%
\pgfsetlinewidth{1.003750pt}%
\definecolor{currentstroke}{rgb}{0.827451,0.827451,0.827451}%
\pgfsetstrokecolor{currentstroke}%
\pgfsetstrokeopacity{0.800000}%
\pgfsetdash{}{0pt}%
\pgfpathmoveto{\pgfqpoint{10.127909in}{3.160252in}}%
\pgfpathcurveto{\pgfqpoint{10.138959in}{3.160252in}}{\pgfqpoint{10.149558in}{3.164643in}}{\pgfqpoint{10.157372in}{3.172456in}}%
\pgfpathcurveto{\pgfqpoint{10.165186in}{3.180270in}}{\pgfqpoint{10.169576in}{3.190869in}}{\pgfqpoint{10.169576in}{3.201919in}}%
\pgfpathcurveto{\pgfqpoint{10.169576in}{3.212969in}}{\pgfqpoint{10.165186in}{3.223568in}}{\pgfqpoint{10.157372in}{3.231382in}}%
\pgfpathcurveto{\pgfqpoint{10.149558in}{3.239196in}}{\pgfqpoint{10.138959in}{3.243586in}}{\pgfqpoint{10.127909in}{3.243586in}}%
\pgfpathcurveto{\pgfqpoint{10.116859in}{3.243586in}}{\pgfqpoint{10.106260in}{3.239196in}}{\pgfqpoint{10.098446in}{3.231382in}}%
\pgfpathcurveto{\pgfqpoint{10.090633in}{3.223568in}}{\pgfqpoint{10.086243in}{3.212969in}}{\pgfqpoint{10.086243in}{3.201919in}}%
\pgfpathcurveto{\pgfqpoint{10.086243in}{3.190869in}}{\pgfqpoint{10.090633in}{3.180270in}}{\pgfqpoint{10.098446in}{3.172456in}}%
\pgfpathcurveto{\pgfqpoint{10.106260in}{3.164643in}}{\pgfqpoint{10.116859in}{3.160252in}}{\pgfqpoint{10.127909in}{3.160252in}}%
\pgfpathlineto{\pgfqpoint{10.127909in}{3.160252in}}%
\pgfpathclose%
\pgfusepath{stroke}%
\end{pgfscope}%
\begin{pgfscope}%
\pgfpathrectangle{\pgfqpoint{7.512535in}{0.437222in}}{\pgfqpoint{6.275590in}{5.159444in}}%
\pgfusepath{clip}%
\pgfsetbuttcap%
\pgfsetroundjoin%
\pgfsetlinewidth{1.003750pt}%
\definecolor{currentstroke}{rgb}{0.827451,0.827451,0.827451}%
\pgfsetstrokecolor{currentstroke}%
\pgfsetstrokeopacity{0.800000}%
\pgfsetdash{}{0pt}%
\pgfpathmoveto{\pgfqpoint{10.254297in}{3.553904in}}%
\pgfpathcurveto{\pgfqpoint{10.265347in}{3.553904in}}{\pgfqpoint{10.275946in}{3.558295in}}{\pgfqpoint{10.283760in}{3.566108in}}%
\pgfpathcurveto{\pgfqpoint{10.291574in}{3.573922in}}{\pgfqpoint{10.295964in}{3.584521in}}{\pgfqpoint{10.295964in}{3.595571in}}%
\pgfpathcurveto{\pgfqpoint{10.295964in}{3.606621in}}{\pgfqpoint{10.291574in}{3.617220in}}{\pgfqpoint{10.283760in}{3.625034in}}%
\pgfpathcurveto{\pgfqpoint{10.275946in}{3.632847in}}{\pgfqpoint{10.265347in}{3.637238in}}{\pgfqpoint{10.254297in}{3.637238in}}%
\pgfpathcurveto{\pgfqpoint{10.243247in}{3.637238in}}{\pgfqpoint{10.232648in}{3.632847in}}{\pgfqpoint{10.224834in}{3.625034in}}%
\pgfpathcurveto{\pgfqpoint{10.217021in}{3.617220in}}{\pgfqpoint{10.212631in}{3.606621in}}{\pgfqpoint{10.212631in}{3.595571in}}%
\pgfpathcurveto{\pgfqpoint{10.212631in}{3.584521in}}{\pgfqpoint{10.217021in}{3.573922in}}{\pgfqpoint{10.224834in}{3.566108in}}%
\pgfpathcurveto{\pgfqpoint{10.232648in}{3.558295in}}{\pgfqpoint{10.243247in}{3.553904in}}{\pgfqpoint{10.254297in}{3.553904in}}%
\pgfpathlineto{\pgfqpoint{10.254297in}{3.553904in}}%
\pgfpathclose%
\pgfusepath{stroke}%
\end{pgfscope}%
\begin{pgfscope}%
\pgfpathrectangle{\pgfqpoint{7.512535in}{0.437222in}}{\pgfqpoint{6.275590in}{5.159444in}}%
\pgfusepath{clip}%
\pgfsetbuttcap%
\pgfsetroundjoin%
\pgfsetlinewidth{1.003750pt}%
\definecolor{currentstroke}{rgb}{0.827451,0.827451,0.827451}%
\pgfsetstrokecolor{currentstroke}%
\pgfsetstrokeopacity{0.800000}%
\pgfsetdash{}{0pt}%
\pgfpathmoveto{\pgfqpoint{12.194523in}{4.683090in}}%
\pgfpathcurveto{\pgfqpoint{12.205573in}{4.683090in}}{\pgfqpoint{12.216172in}{4.687480in}}{\pgfqpoint{12.223986in}{4.695294in}}%
\pgfpathcurveto{\pgfqpoint{12.231799in}{4.703107in}}{\pgfqpoint{12.236190in}{4.713707in}}{\pgfqpoint{12.236190in}{4.724757in}}%
\pgfpathcurveto{\pgfqpoint{12.236190in}{4.735807in}}{\pgfqpoint{12.231799in}{4.746406in}}{\pgfqpoint{12.223986in}{4.754219in}}%
\pgfpathcurveto{\pgfqpoint{12.216172in}{4.762033in}}{\pgfqpoint{12.205573in}{4.766423in}}{\pgfqpoint{12.194523in}{4.766423in}}%
\pgfpathcurveto{\pgfqpoint{12.183473in}{4.766423in}}{\pgfqpoint{12.172874in}{4.762033in}}{\pgfqpoint{12.165060in}{4.754219in}}%
\pgfpathcurveto{\pgfqpoint{12.157247in}{4.746406in}}{\pgfqpoint{12.152856in}{4.735807in}}{\pgfqpoint{12.152856in}{4.724757in}}%
\pgfpathcurveto{\pgfqpoint{12.152856in}{4.713707in}}{\pgfqpoint{12.157247in}{4.703107in}}{\pgfqpoint{12.165060in}{4.695294in}}%
\pgfpathcurveto{\pgfqpoint{12.172874in}{4.687480in}}{\pgfqpoint{12.183473in}{4.683090in}}{\pgfqpoint{12.194523in}{4.683090in}}%
\pgfpathlineto{\pgfqpoint{12.194523in}{4.683090in}}%
\pgfpathclose%
\pgfusepath{stroke}%
\end{pgfscope}%
\begin{pgfscope}%
\pgfpathrectangle{\pgfqpoint{7.512535in}{0.437222in}}{\pgfqpoint{6.275590in}{5.159444in}}%
\pgfusepath{clip}%
\pgfsetbuttcap%
\pgfsetroundjoin%
\pgfsetlinewidth{1.003750pt}%
\definecolor{currentstroke}{rgb}{0.827451,0.827451,0.827451}%
\pgfsetstrokecolor{currentstroke}%
\pgfsetstrokeopacity{0.800000}%
\pgfsetdash{}{0pt}%
\pgfpathmoveto{\pgfqpoint{8.470019in}{3.328233in}}%
\pgfpathcurveto{\pgfqpoint{8.481069in}{3.328233in}}{\pgfqpoint{8.491668in}{3.332624in}}{\pgfqpoint{8.499482in}{3.340437in}}%
\pgfpathcurveto{\pgfqpoint{8.507295in}{3.348251in}}{\pgfqpoint{8.511686in}{3.358850in}}{\pgfqpoint{8.511686in}{3.369900in}}%
\pgfpathcurveto{\pgfqpoint{8.511686in}{3.380950in}}{\pgfqpoint{8.507295in}{3.391549in}}{\pgfqpoint{8.499482in}{3.399363in}}%
\pgfpathcurveto{\pgfqpoint{8.491668in}{3.407176in}}{\pgfqpoint{8.481069in}{3.411567in}}{\pgfqpoint{8.470019in}{3.411567in}}%
\pgfpathcurveto{\pgfqpoint{8.458969in}{3.411567in}}{\pgfqpoint{8.448370in}{3.407176in}}{\pgfqpoint{8.440556in}{3.399363in}}%
\pgfpathcurveto{\pgfqpoint{8.432743in}{3.391549in}}{\pgfqpoint{8.428352in}{3.380950in}}{\pgfqpoint{8.428352in}{3.369900in}}%
\pgfpathcurveto{\pgfqpoint{8.428352in}{3.358850in}}{\pgfqpoint{8.432743in}{3.348251in}}{\pgfqpoint{8.440556in}{3.340437in}}%
\pgfpathcurveto{\pgfqpoint{8.448370in}{3.332624in}}{\pgfqpoint{8.458969in}{3.328233in}}{\pgfqpoint{8.470019in}{3.328233in}}%
\pgfpathlineto{\pgfqpoint{8.470019in}{3.328233in}}%
\pgfpathclose%
\pgfusepath{stroke}%
\end{pgfscope}%
\begin{pgfscope}%
\pgfpathrectangle{\pgfqpoint{7.512535in}{0.437222in}}{\pgfqpoint{6.275590in}{5.159444in}}%
\pgfusepath{clip}%
\pgfsetbuttcap%
\pgfsetroundjoin%
\pgfsetlinewidth{1.003750pt}%
\definecolor{currentstroke}{rgb}{0.827451,0.827451,0.827451}%
\pgfsetstrokecolor{currentstroke}%
\pgfsetstrokeopacity{0.800000}%
\pgfsetdash{}{0pt}%
\pgfpathmoveto{\pgfqpoint{10.827454in}{5.304399in}}%
\pgfpathcurveto{\pgfqpoint{10.838504in}{5.304399in}}{\pgfqpoint{10.849103in}{5.308790in}}{\pgfqpoint{10.856917in}{5.316603in}}%
\pgfpathcurveto{\pgfqpoint{10.864730in}{5.324417in}}{\pgfqpoint{10.869121in}{5.335016in}}{\pgfqpoint{10.869121in}{5.346066in}}%
\pgfpathcurveto{\pgfqpoint{10.869121in}{5.357116in}}{\pgfqpoint{10.864730in}{5.367715in}}{\pgfqpoint{10.856917in}{5.375529in}}%
\pgfpathcurveto{\pgfqpoint{10.849103in}{5.383342in}}{\pgfqpoint{10.838504in}{5.387733in}}{\pgfqpoint{10.827454in}{5.387733in}}%
\pgfpathcurveto{\pgfqpoint{10.816404in}{5.387733in}}{\pgfqpoint{10.805805in}{5.383342in}}{\pgfqpoint{10.797991in}{5.375529in}}%
\pgfpathcurveto{\pgfqpoint{10.790177in}{5.367715in}}{\pgfqpoint{10.785787in}{5.357116in}}{\pgfqpoint{10.785787in}{5.346066in}}%
\pgfpathcurveto{\pgfqpoint{10.785787in}{5.335016in}}{\pgfqpoint{10.790177in}{5.324417in}}{\pgfqpoint{10.797991in}{5.316603in}}%
\pgfpathcurveto{\pgfqpoint{10.805805in}{5.308790in}}{\pgfqpoint{10.816404in}{5.304399in}}{\pgfqpoint{10.827454in}{5.304399in}}%
\pgfpathlineto{\pgfqpoint{10.827454in}{5.304399in}}%
\pgfpathclose%
\pgfusepath{stroke}%
\end{pgfscope}%
\begin{pgfscope}%
\pgfpathrectangle{\pgfqpoint{7.512535in}{0.437222in}}{\pgfqpoint{6.275590in}{5.159444in}}%
\pgfusepath{clip}%
\pgfsetbuttcap%
\pgfsetroundjoin%
\pgfsetlinewidth{1.003750pt}%
\definecolor{currentstroke}{rgb}{0.827451,0.827451,0.827451}%
\pgfsetstrokecolor{currentstroke}%
\pgfsetstrokeopacity{0.800000}%
\pgfsetdash{}{0pt}%
\pgfpathmoveto{\pgfqpoint{10.527072in}{2.988484in}}%
\pgfpathcurveto{\pgfqpoint{10.538122in}{2.988484in}}{\pgfqpoint{10.548721in}{2.992874in}}{\pgfqpoint{10.556535in}{3.000688in}}%
\pgfpathcurveto{\pgfqpoint{10.564349in}{3.008501in}}{\pgfqpoint{10.568739in}{3.019100in}}{\pgfqpoint{10.568739in}{3.030150in}}%
\pgfpathcurveto{\pgfqpoint{10.568739in}{3.041200in}}{\pgfqpoint{10.564349in}{3.051800in}}{\pgfqpoint{10.556535in}{3.059613in}}%
\pgfpathcurveto{\pgfqpoint{10.548721in}{3.067427in}}{\pgfqpoint{10.538122in}{3.071817in}}{\pgfqpoint{10.527072in}{3.071817in}}%
\pgfpathcurveto{\pgfqpoint{10.516022in}{3.071817in}}{\pgfqpoint{10.505423in}{3.067427in}}{\pgfqpoint{10.497609in}{3.059613in}}%
\pgfpathcurveto{\pgfqpoint{10.489796in}{3.051800in}}{\pgfqpoint{10.485405in}{3.041200in}}{\pgfqpoint{10.485405in}{3.030150in}}%
\pgfpathcurveto{\pgfqpoint{10.485405in}{3.019100in}}{\pgfqpoint{10.489796in}{3.008501in}}{\pgfqpoint{10.497609in}{3.000688in}}%
\pgfpathcurveto{\pgfqpoint{10.505423in}{2.992874in}}{\pgfqpoint{10.516022in}{2.988484in}}{\pgfqpoint{10.527072in}{2.988484in}}%
\pgfpathlineto{\pgfqpoint{10.527072in}{2.988484in}}%
\pgfpathclose%
\pgfusepath{stroke}%
\end{pgfscope}%
\begin{pgfscope}%
\pgfpathrectangle{\pgfqpoint{7.512535in}{0.437222in}}{\pgfqpoint{6.275590in}{5.159444in}}%
\pgfusepath{clip}%
\pgfsetbuttcap%
\pgfsetroundjoin%
\pgfsetlinewidth{1.003750pt}%
\definecolor{currentstroke}{rgb}{0.827451,0.827451,0.827451}%
\pgfsetstrokecolor{currentstroke}%
\pgfsetstrokeopacity{0.800000}%
\pgfsetdash{}{0pt}%
\pgfpathmoveto{\pgfqpoint{10.881762in}{4.622183in}}%
\pgfpathcurveto{\pgfqpoint{10.892812in}{4.622183in}}{\pgfqpoint{10.903411in}{4.626573in}}{\pgfqpoint{10.911225in}{4.634387in}}%
\pgfpathcurveto{\pgfqpoint{10.919039in}{4.642201in}}{\pgfqpoint{10.923429in}{4.652800in}}{\pgfqpoint{10.923429in}{4.663850in}}%
\pgfpathcurveto{\pgfqpoint{10.923429in}{4.674900in}}{\pgfqpoint{10.919039in}{4.685499in}}{\pgfqpoint{10.911225in}{4.693313in}}%
\pgfpathcurveto{\pgfqpoint{10.903411in}{4.701126in}}{\pgfqpoint{10.892812in}{4.705516in}}{\pgfqpoint{10.881762in}{4.705516in}}%
\pgfpathcurveto{\pgfqpoint{10.870712in}{4.705516in}}{\pgfqpoint{10.860113in}{4.701126in}}{\pgfqpoint{10.852300in}{4.693313in}}%
\pgfpathcurveto{\pgfqpoint{10.844486in}{4.685499in}}{\pgfqpoint{10.840096in}{4.674900in}}{\pgfqpoint{10.840096in}{4.663850in}}%
\pgfpathcurveto{\pgfqpoint{10.840096in}{4.652800in}}{\pgfqpoint{10.844486in}{4.642201in}}{\pgfqpoint{10.852300in}{4.634387in}}%
\pgfpathcurveto{\pgfqpoint{10.860113in}{4.626573in}}{\pgfqpoint{10.870712in}{4.622183in}}{\pgfqpoint{10.881762in}{4.622183in}}%
\pgfpathlineto{\pgfqpoint{10.881762in}{4.622183in}}%
\pgfpathclose%
\pgfusepath{stroke}%
\end{pgfscope}%
\begin{pgfscope}%
\pgfpathrectangle{\pgfqpoint{7.512535in}{0.437222in}}{\pgfqpoint{6.275590in}{5.159444in}}%
\pgfusepath{clip}%
\pgfsetbuttcap%
\pgfsetroundjoin%
\pgfsetlinewidth{1.003750pt}%
\definecolor{currentstroke}{rgb}{0.827451,0.827451,0.827451}%
\pgfsetstrokecolor{currentstroke}%
\pgfsetstrokeopacity{0.800000}%
\pgfsetdash{}{0pt}%
\pgfpathmoveto{\pgfqpoint{9.556541in}{1.394520in}}%
\pgfpathcurveto{\pgfqpoint{9.567591in}{1.394520in}}{\pgfqpoint{9.578190in}{1.398911in}}{\pgfqpoint{9.586004in}{1.406724in}}%
\pgfpathcurveto{\pgfqpoint{9.593817in}{1.414538in}}{\pgfqpoint{9.598208in}{1.425137in}}{\pgfqpoint{9.598208in}{1.436187in}}%
\pgfpathcurveto{\pgfqpoint{9.598208in}{1.447237in}}{\pgfqpoint{9.593817in}{1.457836in}}{\pgfqpoint{9.586004in}{1.465650in}}%
\pgfpathcurveto{\pgfqpoint{9.578190in}{1.473464in}}{\pgfqpoint{9.567591in}{1.477854in}}{\pgfqpoint{9.556541in}{1.477854in}}%
\pgfpathcurveto{\pgfqpoint{9.545491in}{1.477854in}}{\pgfqpoint{9.534892in}{1.473464in}}{\pgfqpoint{9.527078in}{1.465650in}}%
\pgfpathcurveto{\pgfqpoint{9.519265in}{1.457836in}}{\pgfqpoint{9.514874in}{1.447237in}}{\pgfqpoint{9.514874in}{1.436187in}}%
\pgfpathcurveto{\pgfqpoint{9.514874in}{1.425137in}}{\pgfqpoint{9.519265in}{1.414538in}}{\pgfqpoint{9.527078in}{1.406724in}}%
\pgfpathcurveto{\pgfqpoint{9.534892in}{1.398911in}}{\pgfqpoint{9.545491in}{1.394520in}}{\pgfqpoint{9.556541in}{1.394520in}}%
\pgfpathlineto{\pgfqpoint{9.556541in}{1.394520in}}%
\pgfpathclose%
\pgfusepath{stroke}%
\end{pgfscope}%
\begin{pgfscope}%
\pgfpathrectangle{\pgfqpoint{7.512535in}{0.437222in}}{\pgfqpoint{6.275590in}{5.159444in}}%
\pgfusepath{clip}%
\pgfsetbuttcap%
\pgfsetroundjoin%
\pgfsetlinewidth{1.003750pt}%
\definecolor{currentstroke}{rgb}{0.827451,0.827451,0.827451}%
\pgfsetstrokecolor{currentstroke}%
\pgfsetstrokeopacity{0.800000}%
\pgfsetdash{}{0pt}%
\pgfpathmoveto{\pgfqpoint{10.385573in}{5.144321in}}%
\pgfpathcurveto{\pgfqpoint{10.396623in}{5.144321in}}{\pgfqpoint{10.407222in}{5.148712in}}{\pgfqpoint{10.415036in}{5.156525in}}%
\pgfpathcurveto{\pgfqpoint{10.422849in}{5.164339in}}{\pgfqpoint{10.427240in}{5.174938in}}{\pgfqpoint{10.427240in}{5.185988in}}%
\pgfpathcurveto{\pgfqpoint{10.427240in}{5.197038in}}{\pgfqpoint{10.422849in}{5.207637in}}{\pgfqpoint{10.415036in}{5.215451in}}%
\pgfpathcurveto{\pgfqpoint{10.407222in}{5.223265in}}{\pgfqpoint{10.396623in}{5.227655in}}{\pgfqpoint{10.385573in}{5.227655in}}%
\pgfpathcurveto{\pgfqpoint{10.374523in}{5.227655in}}{\pgfqpoint{10.363924in}{5.223265in}}{\pgfqpoint{10.356110in}{5.215451in}}%
\pgfpathcurveto{\pgfqpoint{10.348297in}{5.207637in}}{\pgfqpoint{10.343906in}{5.197038in}}{\pgfqpoint{10.343906in}{5.185988in}}%
\pgfpathcurveto{\pgfqpoint{10.343906in}{5.174938in}}{\pgfqpoint{10.348297in}{5.164339in}}{\pgfqpoint{10.356110in}{5.156525in}}%
\pgfpathcurveto{\pgfqpoint{10.363924in}{5.148712in}}{\pgfqpoint{10.374523in}{5.144321in}}{\pgfqpoint{10.385573in}{5.144321in}}%
\pgfpathlineto{\pgfqpoint{10.385573in}{5.144321in}}%
\pgfpathclose%
\pgfusepath{stroke}%
\end{pgfscope}%
\begin{pgfscope}%
\pgfpathrectangle{\pgfqpoint{7.512535in}{0.437222in}}{\pgfqpoint{6.275590in}{5.159444in}}%
\pgfusepath{clip}%
\pgfsetbuttcap%
\pgfsetroundjoin%
\pgfsetlinewidth{1.003750pt}%
\definecolor{currentstroke}{rgb}{0.827451,0.827451,0.827451}%
\pgfsetstrokecolor{currentstroke}%
\pgfsetstrokeopacity{0.800000}%
\pgfsetdash{}{0pt}%
\pgfpathmoveto{\pgfqpoint{9.696495in}{1.337612in}}%
\pgfpathcurveto{\pgfqpoint{9.707545in}{1.337612in}}{\pgfqpoint{9.718144in}{1.342002in}}{\pgfqpoint{9.725958in}{1.349816in}}%
\pgfpathcurveto{\pgfqpoint{9.733772in}{1.357629in}}{\pgfqpoint{9.738162in}{1.368229in}}{\pgfqpoint{9.738162in}{1.379279in}}%
\pgfpathcurveto{\pgfqpoint{9.738162in}{1.390329in}}{\pgfqpoint{9.733772in}{1.400928in}}{\pgfqpoint{9.725958in}{1.408741in}}%
\pgfpathcurveto{\pgfqpoint{9.718144in}{1.416555in}}{\pgfqpoint{9.707545in}{1.420945in}}{\pgfqpoint{9.696495in}{1.420945in}}%
\pgfpathcurveto{\pgfqpoint{9.685445in}{1.420945in}}{\pgfqpoint{9.674846in}{1.416555in}}{\pgfqpoint{9.667033in}{1.408741in}}%
\pgfpathcurveto{\pgfqpoint{9.659219in}{1.400928in}}{\pgfqpoint{9.654829in}{1.390329in}}{\pgfqpoint{9.654829in}{1.379279in}}%
\pgfpathcurveto{\pgfqpoint{9.654829in}{1.368229in}}{\pgfqpoint{9.659219in}{1.357629in}}{\pgfqpoint{9.667033in}{1.349816in}}%
\pgfpathcurveto{\pgfqpoint{9.674846in}{1.342002in}}{\pgfqpoint{9.685445in}{1.337612in}}{\pgfqpoint{9.696495in}{1.337612in}}%
\pgfpathlineto{\pgfqpoint{9.696495in}{1.337612in}}%
\pgfpathclose%
\pgfusepath{stroke}%
\end{pgfscope}%
\begin{pgfscope}%
\pgfpathrectangle{\pgfqpoint{7.512535in}{0.437222in}}{\pgfqpoint{6.275590in}{5.159444in}}%
\pgfusepath{clip}%
\pgfsetbuttcap%
\pgfsetroundjoin%
\pgfsetlinewidth{1.003750pt}%
\definecolor{currentstroke}{rgb}{0.827451,0.827451,0.827451}%
\pgfsetstrokecolor{currentstroke}%
\pgfsetstrokeopacity{0.800000}%
\pgfsetdash{}{0pt}%
\pgfpathmoveto{\pgfqpoint{12.072745in}{5.039509in}}%
\pgfpathcurveto{\pgfqpoint{12.083796in}{5.039509in}}{\pgfqpoint{12.094395in}{5.043899in}}{\pgfqpoint{12.102208in}{5.051713in}}%
\pgfpathcurveto{\pgfqpoint{12.110022in}{5.059526in}}{\pgfqpoint{12.114412in}{5.070125in}}{\pgfqpoint{12.114412in}{5.081175in}}%
\pgfpathcurveto{\pgfqpoint{12.114412in}{5.092226in}}{\pgfqpoint{12.110022in}{5.102825in}}{\pgfqpoint{12.102208in}{5.110638in}}%
\pgfpathcurveto{\pgfqpoint{12.094395in}{5.118452in}}{\pgfqpoint{12.083796in}{5.122842in}}{\pgfqpoint{12.072745in}{5.122842in}}%
\pgfpathcurveto{\pgfqpoint{12.061695in}{5.122842in}}{\pgfqpoint{12.051096in}{5.118452in}}{\pgfqpoint{12.043283in}{5.110638in}}%
\pgfpathcurveto{\pgfqpoint{12.035469in}{5.102825in}}{\pgfqpoint{12.031079in}{5.092226in}}{\pgfqpoint{12.031079in}{5.081175in}}%
\pgfpathcurveto{\pgfqpoint{12.031079in}{5.070125in}}{\pgfqpoint{12.035469in}{5.059526in}}{\pgfqpoint{12.043283in}{5.051713in}}%
\pgfpathcurveto{\pgfqpoint{12.051096in}{5.043899in}}{\pgfqpoint{12.061695in}{5.039509in}}{\pgfqpoint{12.072745in}{5.039509in}}%
\pgfpathlineto{\pgfqpoint{12.072745in}{5.039509in}}%
\pgfpathclose%
\pgfusepath{stroke}%
\end{pgfscope}%
\begin{pgfscope}%
\pgfpathrectangle{\pgfqpoint{7.512535in}{0.437222in}}{\pgfqpoint{6.275590in}{5.159444in}}%
\pgfusepath{clip}%
\pgfsetbuttcap%
\pgfsetroundjoin%
\pgfsetlinewidth{1.003750pt}%
\definecolor{currentstroke}{rgb}{0.827451,0.827451,0.827451}%
\pgfsetstrokecolor{currentstroke}%
\pgfsetstrokeopacity{0.800000}%
\pgfsetdash{}{0pt}%
\pgfpathmoveto{\pgfqpoint{9.339517in}{1.176671in}}%
\pgfpathcurveto{\pgfqpoint{9.350568in}{1.176671in}}{\pgfqpoint{9.361167in}{1.181061in}}{\pgfqpoint{9.368980in}{1.188875in}}%
\pgfpathcurveto{\pgfqpoint{9.376794in}{1.196689in}}{\pgfqpoint{9.381184in}{1.207288in}}{\pgfqpoint{9.381184in}{1.218338in}}%
\pgfpathcurveto{\pgfqpoint{9.381184in}{1.229388in}}{\pgfqpoint{9.376794in}{1.239987in}}{\pgfqpoint{9.368980in}{1.247800in}}%
\pgfpathcurveto{\pgfqpoint{9.361167in}{1.255614in}}{\pgfqpoint{9.350568in}{1.260004in}}{\pgfqpoint{9.339517in}{1.260004in}}%
\pgfpathcurveto{\pgfqpoint{9.328467in}{1.260004in}}{\pgfqpoint{9.317868in}{1.255614in}}{\pgfqpoint{9.310055in}{1.247800in}}%
\pgfpathcurveto{\pgfqpoint{9.302241in}{1.239987in}}{\pgfqpoint{9.297851in}{1.229388in}}{\pgfqpoint{9.297851in}{1.218338in}}%
\pgfpathcurveto{\pgfqpoint{9.297851in}{1.207288in}}{\pgfqpoint{9.302241in}{1.196689in}}{\pgfqpoint{9.310055in}{1.188875in}}%
\pgfpathcurveto{\pgfqpoint{9.317868in}{1.181061in}}{\pgfqpoint{9.328467in}{1.176671in}}{\pgfqpoint{9.339517in}{1.176671in}}%
\pgfpathlineto{\pgfqpoint{9.339517in}{1.176671in}}%
\pgfpathclose%
\pgfusepath{stroke}%
\end{pgfscope}%
\begin{pgfscope}%
\pgfpathrectangle{\pgfqpoint{7.512535in}{0.437222in}}{\pgfqpoint{6.275590in}{5.159444in}}%
\pgfusepath{clip}%
\pgfsetbuttcap%
\pgfsetroundjoin%
\pgfsetlinewidth{1.003750pt}%
\definecolor{currentstroke}{rgb}{0.827451,0.827451,0.827451}%
\pgfsetstrokecolor{currentstroke}%
\pgfsetstrokeopacity{0.800000}%
\pgfsetdash{}{0pt}%
\pgfpathmoveto{\pgfqpoint{9.615954in}{3.788157in}}%
\pgfpathcurveto{\pgfqpoint{9.627004in}{3.788157in}}{\pgfqpoint{9.637603in}{3.792548in}}{\pgfqpoint{9.645416in}{3.800361in}}%
\pgfpathcurveto{\pgfqpoint{9.653230in}{3.808175in}}{\pgfqpoint{9.657620in}{3.818774in}}{\pgfqpoint{9.657620in}{3.829824in}}%
\pgfpathcurveto{\pgfqpoint{9.657620in}{3.840874in}}{\pgfqpoint{9.653230in}{3.851473in}}{\pgfqpoint{9.645416in}{3.859287in}}%
\pgfpathcurveto{\pgfqpoint{9.637603in}{3.867100in}}{\pgfqpoint{9.627004in}{3.871491in}}{\pgfqpoint{9.615954in}{3.871491in}}%
\pgfpathcurveto{\pgfqpoint{9.604903in}{3.871491in}}{\pgfqpoint{9.594304in}{3.867100in}}{\pgfqpoint{9.586491in}{3.859287in}}%
\pgfpathcurveto{\pgfqpoint{9.578677in}{3.851473in}}{\pgfqpoint{9.574287in}{3.840874in}}{\pgfqpoint{9.574287in}{3.829824in}}%
\pgfpathcurveto{\pgfqpoint{9.574287in}{3.818774in}}{\pgfqpoint{9.578677in}{3.808175in}}{\pgfqpoint{9.586491in}{3.800361in}}%
\pgfpathcurveto{\pgfqpoint{9.594304in}{3.792548in}}{\pgfqpoint{9.604903in}{3.788157in}}{\pgfqpoint{9.615954in}{3.788157in}}%
\pgfpathlineto{\pgfqpoint{9.615954in}{3.788157in}}%
\pgfpathclose%
\pgfusepath{stroke}%
\end{pgfscope}%
\begin{pgfscope}%
\pgfpathrectangle{\pgfqpoint{7.512535in}{0.437222in}}{\pgfqpoint{6.275590in}{5.159444in}}%
\pgfusepath{clip}%
\pgfsetbuttcap%
\pgfsetroundjoin%
\pgfsetlinewidth{1.003750pt}%
\definecolor{currentstroke}{rgb}{0.827451,0.827451,0.827451}%
\pgfsetstrokecolor{currentstroke}%
\pgfsetstrokeopacity{0.800000}%
\pgfsetdash{}{0pt}%
\pgfpathmoveto{\pgfqpoint{7.972802in}{0.679358in}}%
\pgfpathcurveto{\pgfqpoint{7.983853in}{0.679358in}}{\pgfqpoint{7.994452in}{0.683748in}}{\pgfqpoint{8.002265in}{0.691562in}}%
\pgfpathcurveto{\pgfqpoint{8.010079in}{0.699376in}}{\pgfqpoint{8.014469in}{0.709975in}}{\pgfqpoint{8.014469in}{0.721025in}}%
\pgfpathcurveto{\pgfqpoint{8.014469in}{0.732075in}}{\pgfqpoint{8.010079in}{0.742674in}}{\pgfqpoint{8.002265in}{0.750487in}}%
\pgfpathcurveto{\pgfqpoint{7.994452in}{0.758301in}}{\pgfqpoint{7.983853in}{0.762691in}}{\pgfqpoint{7.972802in}{0.762691in}}%
\pgfpathcurveto{\pgfqpoint{7.961752in}{0.762691in}}{\pgfqpoint{7.951153in}{0.758301in}}{\pgfqpoint{7.943340in}{0.750487in}}%
\pgfpathcurveto{\pgfqpoint{7.935526in}{0.742674in}}{\pgfqpoint{7.931136in}{0.732075in}}{\pgfqpoint{7.931136in}{0.721025in}}%
\pgfpathcurveto{\pgfqpoint{7.931136in}{0.709975in}}{\pgfqpoint{7.935526in}{0.699376in}}{\pgfqpoint{7.943340in}{0.691562in}}%
\pgfpathcurveto{\pgfqpoint{7.951153in}{0.683748in}}{\pgfqpoint{7.961752in}{0.679358in}}{\pgfqpoint{7.972802in}{0.679358in}}%
\pgfpathlineto{\pgfqpoint{7.972802in}{0.679358in}}%
\pgfpathclose%
\pgfusepath{stroke}%
\end{pgfscope}%
\begin{pgfscope}%
\pgfpathrectangle{\pgfqpoint{7.512535in}{0.437222in}}{\pgfqpoint{6.275590in}{5.159444in}}%
\pgfusepath{clip}%
\pgfsetbuttcap%
\pgfsetroundjoin%
\pgfsetlinewidth{1.003750pt}%
\definecolor{currentstroke}{rgb}{0.827451,0.827451,0.827451}%
\pgfsetstrokecolor{currentstroke}%
\pgfsetstrokeopacity{0.800000}%
\pgfsetdash{}{0pt}%
\pgfpathmoveto{\pgfqpoint{9.540773in}{3.442114in}}%
\pgfpathcurveto{\pgfqpoint{9.551823in}{3.442114in}}{\pgfqpoint{9.562422in}{3.446504in}}{\pgfqpoint{9.570236in}{3.454318in}}%
\pgfpathcurveto{\pgfqpoint{9.578049in}{3.462131in}}{\pgfqpoint{9.582440in}{3.472730in}}{\pgfqpoint{9.582440in}{3.483780in}}%
\pgfpathcurveto{\pgfqpoint{9.582440in}{3.494830in}}{\pgfqpoint{9.578049in}{3.505429in}}{\pgfqpoint{9.570236in}{3.513243in}}%
\pgfpathcurveto{\pgfqpoint{9.562422in}{3.521057in}}{\pgfqpoint{9.551823in}{3.525447in}}{\pgfqpoint{9.540773in}{3.525447in}}%
\pgfpathcurveto{\pgfqpoint{9.529723in}{3.525447in}}{\pgfqpoint{9.519124in}{3.521057in}}{\pgfqpoint{9.511310in}{3.513243in}}%
\pgfpathcurveto{\pgfqpoint{9.503497in}{3.505429in}}{\pgfqpoint{9.499106in}{3.494830in}}{\pgfqpoint{9.499106in}{3.483780in}}%
\pgfpathcurveto{\pgfqpoint{9.499106in}{3.472730in}}{\pgfqpoint{9.503497in}{3.462131in}}{\pgfqpoint{9.511310in}{3.454318in}}%
\pgfpathcurveto{\pgfqpoint{9.519124in}{3.446504in}}{\pgfqpoint{9.529723in}{3.442114in}}{\pgfqpoint{9.540773in}{3.442114in}}%
\pgfpathlineto{\pgfqpoint{9.540773in}{3.442114in}}%
\pgfpathclose%
\pgfusepath{stroke}%
\end{pgfscope}%
\begin{pgfscope}%
\pgfpathrectangle{\pgfqpoint{7.512535in}{0.437222in}}{\pgfqpoint{6.275590in}{5.159444in}}%
\pgfusepath{clip}%
\pgfsetbuttcap%
\pgfsetroundjoin%
\pgfsetlinewidth{1.003750pt}%
\definecolor{currentstroke}{rgb}{0.827451,0.827451,0.827451}%
\pgfsetstrokecolor{currentstroke}%
\pgfsetstrokeopacity{0.800000}%
\pgfsetdash{}{0pt}%
\pgfpathmoveto{\pgfqpoint{8.598634in}{1.087031in}}%
\pgfpathcurveto{\pgfqpoint{8.609684in}{1.087031in}}{\pgfqpoint{8.620283in}{1.091421in}}{\pgfqpoint{8.628097in}{1.099234in}}%
\pgfpathcurveto{\pgfqpoint{8.635910in}{1.107048in}}{\pgfqpoint{8.640300in}{1.117647in}}{\pgfqpoint{8.640300in}{1.128697in}}%
\pgfpathcurveto{\pgfqpoint{8.640300in}{1.139747in}}{\pgfqpoint{8.635910in}{1.150346in}}{\pgfqpoint{8.628097in}{1.158160in}}%
\pgfpathcurveto{\pgfqpoint{8.620283in}{1.165974in}}{\pgfqpoint{8.609684in}{1.170364in}}{\pgfqpoint{8.598634in}{1.170364in}}%
\pgfpathcurveto{\pgfqpoint{8.587584in}{1.170364in}}{\pgfqpoint{8.576985in}{1.165974in}}{\pgfqpoint{8.569171in}{1.158160in}}%
\pgfpathcurveto{\pgfqpoint{8.561357in}{1.150346in}}{\pgfqpoint{8.556967in}{1.139747in}}{\pgfqpoint{8.556967in}{1.128697in}}%
\pgfpathcurveto{\pgfqpoint{8.556967in}{1.117647in}}{\pgfqpoint{8.561357in}{1.107048in}}{\pgfqpoint{8.569171in}{1.099234in}}%
\pgfpathcurveto{\pgfqpoint{8.576985in}{1.091421in}}{\pgfqpoint{8.587584in}{1.087031in}}{\pgfqpoint{8.598634in}{1.087031in}}%
\pgfpathlineto{\pgfqpoint{8.598634in}{1.087031in}}%
\pgfpathclose%
\pgfusepath{stroke}%
\end{pgfscope}%
\begin{pgfscope}%
\pgfpathrectangle{\pgfqpoint{7.512535in}{0.437222in}}{\pgfqpoint{6.275590in}{5.159444in}}%
\pgfusepath{clip}%
\pgfsetbuttcap%
\pgfsetroundjoin%
\pgfsetlinewidth{1.003750pt}%
\definecolor{currentstroke}{rgb}{0.827451,0.827451,0.827451}%
\pgfsetstrokecolor{currentstroke}%
\pgfsetstrokeopacity{0.800000}%
\pgfsetdash{}{0pt}%
\pgfpathmoveto{\pgfqpoint{9.505328in}{3.327623in}}%
\pgfpathcurveto{\pgfqpoint{9.516379in}{3.327623in}}{\pgfqpoint{9.526978in}{3.332014in}}{\pgfqpoint{9.534791in}{3.339827in}}%
\pgfpathcurveto{\pgfqpoint{9.542605in}{3.347641in}}{\pgfqpoint{9.546995in}{3.358240in}}{\pgfqpoint{9.546995in}{3.369290in}}%
\pgfpathcurveto{\pgfqpoint{9.546995in}{3.380340in}}{\pgfqpoint{9.542605in}{3.390939in}}{\pgfqpoint{9.534791in}{3.398753in}}%
\pgfpathcurveto{\pgfqpoint{9.526978in}{3.406567in}}{\pgfqpoint{9.516379in}{3.410957in}}{\pgfqpoint{9.505328in}{3.410957in}}%
\pgfpathcurveto{\pgfqpoint{9.494278in}{3.410957in}}{\pgfqpoint{9.483679in}{3.406567in}}{\pgfqpoint{9.475866in}{3.398753in}}%
\pgfpathcurveto{\pgfqpoint{9.468052in}{3.390939in}}{\pgfqpoint{9.463662in}{3.380340in}}{\pgfqpoint{9.463662in}{3.369290in}}%
\pgfpathcurveto{\pgfqpoint{9.463662in}{3.358240in}}{\pgfqpoint{9.468052in}{3.347641in}}{\pgfqpoint{9.475866in}{3.339827in}}%
\pgfpathcurveto{\pgfqpoint{9.483679in}{3.332014in}}{\pgfqpoint{9.494278in}{3.327623in}}{\pgfqpoint{9.505328in}{3.327623in}}%
\pgfpathlineto{\pgfqpoint{9.505328in}{3.327623in}}%
\pgfpathclose%
\pgfusepath{stroke}%
\end{pgfscope}%
\begin{pgfscope}%
\pgfpathrectangle{\pgfqpoint{7.512535in}{0.437222in}}{\pgfqpoint{6.275590in}{5.159444in}}%
\pgfusepath{clip}%
\pgfsetbuttcap%
\pgfsetroundjoin%
\pgfsetlinewidth{1.003750pt}%
\definecolor{currentstroke}{rgb}{0.827451,0.827451,0.827451}%
\pgfsetstrokecolor{currentstroke}%
\pgfsetstrokeopacity{0.800000}%
\pgfsetdash{}{0pt}%
\pgfpathmoveto{\pgfqpoint{8.044541in}{2.113663in}}%
\pgfpathcurveto{\pgfqpoint{8.055591in}{2.113663in}}{\pgfqpoint{8.066190in}{2.118054in}}{\pgfqpoint{8.074003in}{2.125867in}}%
\pgfpathcurveto{\pgfqpoint{8.081817in}{2.133681in}}{\pgfqpoint{8.086207in}{2.144280in}}{\pgfqpoint{8.086207in}{2.155330in}}%
\pgfpathcurveto{\pgfqpoint{8.086207in}{2.166380in}}{\pgfqpoint{8.081817in}{2.176979in}}{\pgfqpoint{8.074003in}{2.184793in}}%
\pgfpathcurveto{\pgfqpoint{8.066190in}{2.192606in}}{\pgfqpoint{8.055591in}{2.196997in}}{\pgfqpoint{8.044541in}{2.196997in}}%
\pgfpathcurveto{\pgfqpoint{8.033491in}{2.196997in}}{\pgfqpoint{8.022891in}{2.192606in}}{\pgfqpoint{8.015078in}{2.184793in}}%
\pgfpathcurveto{\pgfqpoint{8.007264in}{2.176979in}}{\pgfqpoint{8.002874in}{2.166380in}}{\pgfqpoint{8.002874in}{2.155330in}}%
\pgfpathcurveto{\pgfqpoint{8.002874in}{2.144280in}}{\pgfqpoint{8.007264in}{2.133681in}}{\pgfqpoint{8.015078in}{2.125867in}}%
\pgfpathcurveto{\pgfqpoint{8.022891in}{2.118054in}}{\pgfqpoint{8.033491in}{2.113663in}}{\pgfqpoint{8.044541in}{2.113663in}}%
\pgfpathlineto{\pgfqpoint{8.044541in}{2.113663in}}%
\pgfpathclose%
\pgfusepath{stroke}%
\end{pgfscope}%
\begin{pgfscope}%
\pgfpathrectangle{\pgfqpoint{7.512535in}{0.437222in}}{\pgfqpoint{6.275590in}{5.159444in}}%
\pgfusepath{clip}%
\pgfsetbuttcap%
\pgfsetroundjoin%
\pgfsetlinewidth{1.003750pt}%
\definecolor{currentstroke}{rgb}{0.827451,0.827451,0.827451}%
\pgfsetstrokecolor{currentstroke}%
\pgfsetstrokeopacity{0.800000}%
\pgfsetdash{}{0pt}%
\pgfpathmoveto{\pgfqpoint{12.292049in}{5.113368in}}%
\pgfpathcurveto{\pgfqpoint{12.303099in}{5.113368in}}{\pgfqpoint{12.313698in}{5.117758in}}{\pgfqpoint{12.321512in}{5.125572in}}%
\pgfpathcurveto{\pgfqpoint{12.329325in}{5.133385in}}{\pgfqpoint{12.333716in}{5.143984in}}{\pgfqpoint{12.333716in}{5.155034in}}%
\pgfpathcurveto{\pgfqpoint{12.333716in}{5.166084in}}{\pgfqpoint{12.329325in}{5.176683in}}{\pgfqpoint{12.321512in}{5.184497in}}%
\pgfpathcurveto{\pgfqpoint{12.313698in}{5.192311in}}{\pgfqpoint{12.303099in}{5.196701in}}{\pgfqpoint{12.292049in}{5.196701in}}%
\pgfpathcurveto{\pgfqpoint{12.280999in}{5.196701in}}{\pgfqpoint{12.270400in}{5.192311in}}{\pgfqpoint{12.262586in}{5.184497in}}%
\pgfpathcurveto{\pgfqpoint{12.254772in}{5.176683in}}{\pgfqpoint{12.250382in}{5.166084in}}{\pgfqpoint{12.250382in}{5.155034in}}%
\pgfpathcurveto{\pgfqpoint{12.250382in}{5.143984in}}{\pgfqpoint{12.254772in}{5.133385in}}{\pgfqpoint{12.262586in}{5.125572in}}%
\pgfpathcurveto{\pgfqpoint{12.270400in}{5.117758in}}{\pgfqpoint{12.280999in}{5.113368in}}{\pgfqpoint{12.292049in}{5.113368in}}%
\pgfpathlineto{\pgfqpoint{12.292049in}{5.113368in}}%
\pgfpathclose%
\pgfusepath{stroke}%
\end{pgfscope}%
\begin{pgfscope}%
\pgfpathrectangle{\pgfqpoint{7.512535in}{0.437222in}}{\pgfqpoint{6.275590in}{5.159444in}}%
\pgfusepath{clip}%
\pgfsetbuttcap%
\pgfsetroundjoin%
\pgfsetlinewidth{1.003750pt}%
\definecolor{currentstroke}{rgb}{0.827451,0.827451,0.827451}%
\pgfsetstrokecolor{currentstroke}%
\pgfsetstrokeopacity{0.800000}%
\pgfsetdash{}{0pt}%
\pgfpathmoveto{\pgfqpoint{13.101307in}{5.417061in}}%
\pgfpathcurveto{\pgfqpoint{13.112357in}{5.417061in}}{\pgfqpoint{13.122956in}{5.421451in}}{\pgfqpoint{13.130770in}{5.429265in}}%
\pgfpathcurveto{\pgfqpoint{13.138583in}{5.437079in}}{\pgfqpoint{13.142973in}{5.447678in}}{\pgfqpoint{13.142973in}{5.458728in}}%
\pgfpathcurveto{\pgfqpoint{13.142973in}{5.469778in}}{\pgfqpoint{13.138583in}{5.480377in}}{\pgfqpoint{13.130770in}{5.488191in}}%
\pgfpathcurveto{\pgfqpoint{13.122956in}{5.496004in}}{\pgfqpoint{13.112357in}{5.500395in}}{\pgfqpoint{13.101307in}{5.500395in}}%
\pgfpathcurveto{\pgfqpoint{13.090257in}{5.500395in}}{\pgfqpoint{13.079658in}{5.496004in}}{\pgfqpoint{13.071844in}{5.488191in}}%
\pgfpathcurveto{\pgfqpoint{13.064030in}{5.480377in}}{\pgfqpoint{13.059640in}{5.469778in}}{\pgfqpoint{13.059640in}{5.458728in}}%
\pgfpathcurveto{\pgfqpoint{13.059640in}{5.447678in}}{\pgfqpoint{13.064030in}{5.437079in}}{\pgfqpoint{13.071844in}{5.429265in}}%
\pgfpathcurveto{\pgfqpoint{13.079658in}{5.421451in}}{\pgfqpoint{13.090257in}{5.417061in}}{\pgfqpoint{13.101307in}{5.417061in}}%
\pgfpathlineto{\pgfqpoint{13.101307in}{5.417061in}}%
\pgfpathclose%
\pgfusepath{stroke}%
\end{pgfscope}%
\begin{pgfscope}%
\pgfpathrectangle{\pgfqpoint{7.512535in}{0.437222in}}{\pgfqpoint{6.275590in}{5.159444in}}%
\pgfusepath{clip}%
\pgfsetbuttcap%
\pgfsetroundjoin%
\pgfsetlinewidth{1.003750pt}%
\definecolor{currentstroke}{rgb}{0.827451,0.827451,0.827451}%
\pgfsetstrokecolor{currentstroke}%
\pgfsetstrokeopacity{0.800000}%
\pgfsetdash{}{0pt}%
\pgfpathmoveto{\pgfqpoint{9.807010in}{4.531380in}}%
\pgfpathcurveto{\pgfqpoint{9.818060in}{4.531380in}}{\pgfqpoint{9.828659in}{4.535770in}}{\pgfqpoint{9.836473in}{4.543584in}}%
\pgfpathcurveto{\pgfqpoint{9.844287in}{4.551397in}}{\pgfqpoint{9.848677in}{4.561996in}}{\pgfqpoint{9.848677in}{4.573046in}}%
\pgfpathcurveto{\pgfqpoint{9.848677in}{4.584096in}}{\pgfqpoint{9.844287in}{4.594695in}}{\pgfqpoint{9.836473in}{4.602509in}}%
\pgfpathcurveto{\pgfqpoint{9.828659in}{4.610323in}}{\pgfqpoint{9.818060in}{4.614713in}}{\pgfqpoint{9.807010in}{4.614713in}}%
\pgfpathcurveto{\pgfqpoint{9.795960in}{4.614713in}}{\pgfqpoint{9.785361in}{4.610323in}}{\pgfqpoint{9.777548in}{4.602509in}}%
\pgfpathcurveto{\pgfqpoint{9.769734in}{4.594695in}}{\pgfqpoint{9.765344in}{4.584096in}}{\pgfqpoint{9.765344in}{4.573046in}}%
\pgfpathcurveto{\pgfqpoint{9.765344in}{4.561996in}}{\pgfqpoint{9.769734in}{4.551397in}}{\pgfqpoint{9.777548in}{4.543584in}}%
\pgfpathcurveto{\pgfqpoint{9.785361in}{4.535770in}}{\pgfqpoint{9.795960in}{4.531380in}}{\pgfqpoint{9.807010in}{4.531380in}}%
\pgfpathlineto{\pgfqpoint{9.807010in}{4.531380in}}%
\pgfpathclose%
\pgfusepath{stroke}%
\end{pgfscope}%
\begin{pgfscope}%
\pgfpathrectangle{\pgfqpoint{7.512535in}{0.437222in}}{\pgfqpoint{6.275590in}{5.159444in}}%
\pgfusepath{clip}%
\pgfsetbuttcap%
\pgfsetroundjoin%
\pgfsetlinewidth{1.003750pt}%
\definecolor{currentstroke}{rgb}{0.827451,0.827451,0.827451}%
\pgfsetstrokecolor{currentstroke}%
\pgfsetstrokeopacity{0.800000}%
\pgfsetdash{}{0pt}%
\pgfpathmoveto{\pgfqpoint{9.792079in}{4.165674in}}%
\pgfpathcurveto{\pgfqpoint{9.803129in}{4.165674in}}{\pgfqpoint{9.813728in}{4.170064in}}{\pgfqpoint{9.821541in}{4.177878in}}%
\pgfpathcurveto{\pgfqpoint{9.829355in}{4.185692in}}{\pgfqpoint{9.833745in}{4.196291in}}{\pgfqpoint{9.833745in}{4.207341in}}%
\pgfpathcurveto{\pgfqpoint{9.833745in}{4.218391in}}{\pgfqpoint{9.829355in}{4.228990in}}{\pgfqpoint{9.821541in}{4.236803in}}%
\pgfpathcurveto{\pgfqpoint{9.813728in}{4.244617in}}{\pgfqpoint{9.803129in}{4.249007in}}{\pgfqpoint{9.792079in}{4.249007in}}%
\pgfpathcurveto{\pgfqpoint{9.781029in}{4.249007in}}{\pgfqpoint{9.770429in}{4.244617in}}{\pgfqpoint{9.762616in}{4.236803in}}%
\pgfpathcurveto{\pgfqpoint{9.754802in}{4.228990in}}{\pgfqpoint{9.750412in}{4.218391in}}{\pgfqpoint{9.750412in}{4.207341in}}%
\pgfpathcurveto{\pgfqpoint{9.750412in}{4.196291in}}{\pgfqpoint{9.754802in}{4.185692in}}{\pgfqpoint{9.762616in}{4.177878in}}%
\pgfpathcurveto{\pgfqpoint{9.770429in}{4.170064in}}{\pgfqpoint{9.781029in}{4.165674in}}{\pgfqpoint{9.792079in}{4.165674in}}%
\pgfpathlineto{\pgfqpoint{9.792079in}{4.165674in}}%
\pgfpathclose%
\pgfusepath{stroke}%
\end{pgfscope}%
\begin{pgfscope}%
\pgfpathrectangle{\pgfqpoint{7.512535in}{0.437222in}}{\pgfqpoint{6.275590in}{5.159444in}}%
\pgfusepath{clip}%
\pgfsetbuttcap%
\pgfsetroundjoin%
\pgfsetlinewidth{1.003750pt}%
\definecolor{currentstroke}{rgb}{0.827451,0.827451,0.827451}%
\pgfsetstrokecolor{currentstroke}%
\pgfsetstrokeopacity{0.800000}%
\pgfsetdash{}{0pt}%
\pgfpathmoveto{\pgfqpoint{8.309719in}{2.302744in}}%
\pgfpathcurveto{\pgfqpoint{8.320769in}{2.302744in}}{\pgfqpoint{8.331368in}{2.307134in}}{\pgfqpoint{8.339182in}{2.314947in}}%
\pgfpathcurveto{\pgfqpoint{8.346996in}{2.322761in}}{\pgfqpoint{8.351386in}{2.333360in}}{\pgfqpoint{8.351386in}{2.344410in}}%
\pgfpathcurveto{\pgfqpoint{8.351386in}{2.355460in}}{\pgfqpoint{8.346996in}{2.366059in}}{\pgfqpoint{8.339182in}{2.373873in}}%
\pgfpathcurveto{\pgfqpoint{8.331368in}{2.381687in}}{\pgfqpoint{8.320769in}{2.386077in}}{\pgfqpoint{8.309719in}{2.386077in}}%
\pgfpathcurveto{\pgfqpoint{8.298669in}{2.386077in}}{\pgfqpoint{8.288070in}{2.381687in}}{\pgfqpoint{8.280256in}{2.373873in}}%
\pgfpathcurveto{\pgfqpoint{8.272443in}{2.366059in}}{\pgfqpoint{8.268052in}{2.355460in}}{\pgfqpoint{8.268052in}{2.344410in}}%
\pgfpathcurveto{\pgfqpoint{8.268052in}{2.333360in}}{\pgfqpoint{8.272443in}{2.322761in}}{\pgfqpoint{8.280256in}{2.314947in}}%
\pgfpathcurveto{\pgfqpoint{8.288070in}{2.307134in}}{\pgfqpoint{8.298669in}{2.302744in}}{\pgfqpoint{8.309719in}{2.302744in}}%
\pgfpathlineto{\pgfqpoint{8.309719in}{2.302744in}}%
\pgfpathclose%
\pgfusepath{stroke}%
\end{pgfscope}%
\begin{pgfscope}%
\pgfpathrectangle{\pgfqpoint{7.512535in}{0.437222in}}{\pgfqpoint{6.275590in}{5.159444in}}%
\pgfusepath{clip}%
\pgfsetbuttcap%
\pgfsetroundjoin%
\pgfsetlinewidth{1.003750pt}%
\definecolor{currentstroke}{rgb}{0.827451,0.827451,0.827451}%
\pgfsetstrokecolor{currentstroke}%
\pgfsetstrokeopacity{0.800000}%
\pgfsetdash{}{0pt}%
\pgfpathmoveto{\pgfqpoint{9.581399in}{1.396341in}}%
\pgfpathcurveto{\pgfqpoint{9.592449in}{1.396341in}}{\pgfqpoint{9.603048in}{1.400731in}}{\pgfqpoint{9.610862in}{1.408545in}}%
\pgfpathcurveto{\pgfqpoint{9.618675in}{1.416359in}}{\pgfqpoint{9.623065in}{1.426958in}}{\pgfqpoint{9.623065in}{1.438008in}}%
\pgfpathcurveto{\pgfqpoint{9.623065in}{1.449058in}}{\pgfqpoint{9.618675in}{1.459657in}}{\pgfqpoint{9.610862in}{1.467470in}}%
\pgfpathcurveto{\pgfqpoint{9.603048in}{1.475284in}}{\pgfqpoint{9.592449in}{1.479674in}}{\pgfqpoint{9.581399in}{1.479674in}}%
\pgfpathcurveto{\pgfqpoint{9.570349in}{1.479674in}}{\pgfqpoint{9.559750in}{1.475284in}}{\pgfqpoint{9.551936in}{1.467470in}}%
\pgfpathcurveto{\pgfqpoint{9.544122in}{1.459657in}}{\pgfqpoint{9.539732in}{1.449058in}}{\pgfqpoint{9.539732in}{1.438008in}}%
\pgfpathcurveto{\pgfqpoint{9.539732in}{1.426958in}}{\pgfqpoint{9.544122in}{1.416359in}}{\pgfqpoint{9.551936in}{1.408545in}}%
\pgfpathcurveto{\pgfqpoint{9.559750in}{1.400731in}}{\pgfqpoint{9.570349in}{1.396341in}}{\pgfqpoint{9.581399in}{1.396341in}}%
\pgfpathlineto{\pgfqpoint{9.581399in}{1.396341in}}%
\pgfpathclose%
\pgfusepath{stroke}%
\end{pgfscope}%
\begin{pgfscope}%
\pgfpathrectangle{\pgfqpoint{7.512535in}{0.437222in}}{\pgfqpoint{6.275590in}{5.159444in}}%
\pgfusepath{clip}%
\pgfsetbuttcap%
\pgfsetroundjoin%
\pgfsetlinewidth{1.003750pt}%
\definecolor{currentstroke}{rgb}{0.827451,0.827451,0.827451}%
\pgfsetstrokecolor{currentstroke}%
\pgfsetstrokeopacity{0.800000}%
\pgfsetdash{}{0pt}%
\pgfpathmoveto{\pgfqpoint{8.434640in}{3.528338in}}%
\pgfpathcurveto{\pgfqpoint{8.445690in}{3.528338in}}{\pgfqpoint{8.456289in}{3.532728in}}{\pgfqpoint{8.464103in}{3.540542in}}%
\pgfpathcurveto{\pgfqpoint{8.471916in}{3.548355in}}{\pgfqpoint{8.476306in}{3.558954in}}{\pgfqpoint{8.476306in}{3.570004in}}%
\pgfpathcurveto{\pgfqpoint{8.476306in}{3.581055in}}{\pgfqpoint{8.471916in}{3.591654in}}{\pgfqpoint{8.464103in}{3.599467in}}%
\pgfpathcurveto{\pgfqpoint{8.456289in}{3.607281in}}{\pgfqpoint{8.445690in}{3.611671in}}{\pgfqpoint{8.434640in}{3.611671in}}%
\pgfpathcurveto{\pgfqpoint{8.423590in}{3.611671in}}{\pgfqpoint{8.412991in}{3.607281in}}{\pgfqpoint{8.405177in}{3.599467in}}%
\pgfpathcurveto{\pgfqpoint{8.397363in}{3.591654in}}{\pgfqpoint{8.392973in}{3.581055in}}{\pgfqpoint{8.392973in}{3.570004in}}%
\pgfpathcurveto{\pgfqpoint{8.392973in}{3.558954in}}{\pgfqpoint{8.397363in}{3.548355in}}{\pgfqpoint{8.405177in}{3.540542in}}%
\pgfpathcurveto{\pgfqpoint{8.412991in}{3.532728in}}{\pgfqpoint{8.423590in}{3.528338in}}{\pgfqpoint{8.434640in}{3.528338in}}%
\pgfpathlineto{\pgfqpoint{8.434640in}{3.528338in}}%
\pgfpathclose%
\pgfusepath{stroke}%
\end{pgfscope}%
\begin{pgfscope}%
\pgfpathrectangle{\pgfqpoint{7.512535in}{0.437222in}}{\pgfqpoint{6.275590in}{5.159444in}}%
\pgfusepath{clip}%
\pgfsetbuttcap%
\pgfsetroundjoin%
\pgfsetlinewidth{1.003750pt}%
\definecolor{currentstroke}{rgb}{0.827451,0.827451,0.827451}%
\pgfsetstrokecolor{currentstroke}%
\pgfsetstrokeopacity{0.800000}%
\pgfsetdash{}{0pt}%
\pgfpathmoveto{\pgfqpoint{13.101307in}{5.469388in}}%
\pgfpathcurveto{\pgfqpoint{13.112357in}{5.469388in}}{\pgfqpoint{13.122956in}{5.473778in}}{\pgfqpoint{13.130770in}{5.481592in}}%
\pgfpathcurveto{\pgfqpoint{13.138583in}{5.489405in}}{\pgfqpoint{13.142973in}{5.500004in}}{\pgfqpoint{13.142973in}{5.511054in}}%
\pgfpathcurveto{\pgfqpoint{13.142973in}{5.522105in}}{\pgfqpoint{13.138583in}{5.532704in}}{\pgfqpoint{13.130770in}{5.540517in}}%
\pgfpathcurveto{\pgfqpoint{13.122956in}{5.548331in}}{\pgfqpoint{13.112357in}{5.552721in}}{\pgfqpoint{13.101307in}{5.552721in}}%
\pgfpathcurveto{\pgfqpoint{13.090257in}{5.552721in}}{\pgfqpoint{13.079658in}{5.548331in}}{\pgfqpoint{13.071844in}{5.540517in}}%
\pgfpathcurveto{\pgfqpoint{13.064030in}{5.532704in}}{\pgfqpoint{13.059640in}{5.522105in}}{\pgfqpoint{13.059640in}{5.511054in}}%
\pgfpathcurveto{\pgfqpoint{13.059640in}{5.500004in}}{\pgfqpoint{13.064030in}{5.489405in}}{\pgfqpoint{13.071844in}{5.481592in}}%
\pgfpathcurveto{\pgfqpoint{13.079658in}{5.473778in}}{\pgfqpoint{13.090257in}{5.469388in}}{\pgfqpoint{13.101307in}{5.469388in}}%
\pgfpathlineto{\pgfqpoint{13.101307in}{5.469388in}}%
\pgfpathclose%
\pgfusepath{stroke}%
\end{pgfscope}%
\begin{pgfscope}%
\pgfpathrectangle{\pgfqpoint{7.512535in}{0.437222in}}{\pgfqpoint{6.275590in}{5.159444in}}%
\pgfusepath{clip}%
\pgfsetbuttcap%
\pgfsetroundjoin%
\pgfsetlinewidth{1.003750pt}%
\definecolor{currentstroke}{rgb}{0.827451,0.827451,0.827451}%
\pgfsetstrokecolor{currentstroke}%
\pgfsetstrokeopacity{0.800000}%
\pgfsetdash{}{0pt}%
\pgfpathmoveto{\pgfqpoint{9.630920in}{3.276112in}}%
\pgfpathcurveto{\pgfqpoint{9.641970in}{3.276112in}}{\pgfqpoint{9.652569in}{3.280502in}}{\pgfqpoint{9.660383in}{3.288316in}}%
\pgfpathcurveto{\pgfqpoint{9.668196in}{3.296130in}}{\pgfqpoint{9.672587in}{3.306729in}}{\pgfqpoint{9.672587in}{3.317779in}}%
\pgfpathcurveto{\pgfqpoint{9.672587in}{3.328829in}}{\pgfqpoint{9.668196in}{3.339428in}}{\pgfqpoint{9.660383in}{3.347241in}}%
\pgfpathcurveto{\pgfqpoint{9.652569in}{3.355055in}}{\pgfqpoint{9.641970in}{3.359445in}}{\pgfqpoint{9.630920in}{3.359445in}}%
\pgfpathcurveto{\pgfqpoint{9.619870in}{3.359445in}}{\pgfqpoint{9.609271in}{3.355055in}}{\pgfqpoint{9.601457in}{3.347241in}}%
\pgfpathcurveto{\pgfqpoint{9.593644in}{3.339428in}}{\pgfqpoint{9.589253in}{3.328829in}}{\pgfqpoint{9.589253in}{3.317779in}}%
\pgfpathcurveto{\pgfqpoint{9.589253in}{3.306729in}}{\pgfqpoint{9.593644in}{3.296130in}}{\pgfqpoint{9.601457in}{3.288316in}}%
\pgfpathcurveto{\pgfqpoint{9.609271in}{3.280502in}}{\pgfqpoint{9.619870in}{3.276112in}}{\pgfqpoint{9.630920in}{3.276112in}}%
\pgfpathlineto{\pgfqpoint{9.630920in}{3.276112in}}%
\pgfpathclose%
\pgfusepath{stroke}%
\end{pgfscope}%
\begin{pgfscope}%
\pgfpathrectangle{\pgfqpoint{7.512535in}{0.437222in}}{\pgfqpoint{6.275590in}{5.159444in}}%
\pgfusepath{clip}%
\pgfsetbuttcap%
\pgfsetroundjoin%
\pgfsetlinewidth{1.003750pt}%
\definecolor{currentstroke}{rgb}{0.827451,0.827451,0.827451}%
\pgfsetstrokecolor{currentstroke}%
\pgfsetstrokeopacity{0.800000}%
\pgfsetdash{}{0pt}%
\pgfpathmoveto{\pgfqpoint{9.928499in}{4.056499in}}%
\pgfpathcurveto{\pgfqpoint{9.939549in}{4.056499in}}{\pgfqpoint{9.950148in}{4.060889in}}{\pgfqpoint{9.957962in}{4.068703in}}%
\pgfpathcurveto{\pgfqpoint{9.965775in}{4.076516in}}{\pgfqpoint{9.970165in}{4.087115in}}{\pgfqpoint{9.970165in}{4.098166in}}%
\pgfpathcurveto{\pgfqpoint{9.970165in}{4.109216in}}{\pgfqpoint{9.965775in}{4.119815in}}{\pgfqpoint{9.957962in}{4.127628in}}%
\pgfpathcurveto{\pgfqpoint{9.950148in}{4.135442in}}{\pgfqpoint{9.939549in}{4.139832in}}{\pgfqpoint{9.928499in}{4.139832in}}%
\pgfpathcurveto{\pgfqpoint{9.917449in}{4.139832in}}{\pgfqpoint{9.906850in}{4.135442in}}{\pgfqpoint{9.899036in}{4.127628in}}%
\pgfpathcurveto{\pgfqpoint{9.891222in}{4.119815in}}{\pgfqpoint{9.886832in}{4.109216in}}{\pgfqpoint{9.886832in}{4.098166in}}%
\pgfpathcurveto{\pgfqpoint{9.886832in}{4.087115in}}{\pgfqpoint{9.891222in}{4.076516in}}{\pgfqpoint{9.899036in}{4.068703in}}%
\pgfpathcurveto{\pgfqpoint{9.906850in}{4.060889in}}{\pgfqpoint{9.917449in}{4.056499in}}{\pgfqpoint{9.928499in}{4.056499in}}%
\pgfpathlineto{\pgfqpoint{9.928499in}{4.056499in}}%
\pgfpathclose%
\pgfusepath{stroke}%
\end{pgfscope}%
\begin{pgfscope}%
\pgfpathrectangle{\pgfqpoint{7.512535in}{0.437222in}}{\pgfqpoint{6.275590in}{5.159444in}}%
\pgfusepath{clip}%
\pgfsetbuttcap%
\pgfsetroundjoin%
\pgfsetlinewidth{1.003750pt}%
\definecolor{currentstroke}{rgb}{0.827451,0.827451,0.827451}%
\pgfsetstrokecolor{currentstroke}%
\pgfsetstrokeopacity{0.800000}%
\pgfsetdash{}{0pt}%
\pgfpathmoveto{\pgfqpoint{8.588580in}{3.443375in}}%
\pgfpathcurveto{\pgfqpoint{8.599631in}{3.443375in}}{\pgfqpoint{8.610230in}{3.447765in}}{\pgfqpoint{8.618043in}{3.455579in}}%
\pgfpathcurveto{\pgfqpoint{8.625857in}{3.463393in}}{\pgfqpoint{8.630247in}{3.473992in}}{\pgfqpoint{8.630247in}{3.485042in}}%
\pgfpathcurveto{\pgfqpoint{8.630247in}{3.496092in}}{\pgfqpoint{8.625857in}{3.506691in}}{\pgfqpoint{8.618043in}{3.514505in}}%
\pgfpathcurveto{\pgfqpoint{8.610230in}{3.522318in}}{\pgfqpoint{8.599631in}{3.526708in}}{\pgfqpoint{8.588580in}{3.526708in}}%
\pgfpathcurveto{\pgfqpoint{8.577530in}{3.526708in}}{\pgfqpoint{8.566931in}{3.522318in}}{\pgfqpoint{8.559118in}{3.514505in}}%
\pgfpathcurveto{\pgfqpoint{8.551304in}{3.506691in}}{\pgfqpoint{8.546914in}{3.496092in}}{\pgfqpoint{8.546914in}{3.485042in}}%
\pgfpathcurveto{\pgfqpoint{8.546914in}{3.473992in}}{\pgfqpoint{8.551304in}{3.463393in}}{\pgfqpoint{8.559118in}{3.455579in}}%
\pgfpathcurveto{\pgfqpoint{8.566931in}{3.447765in}}{\pgfqpoint{8.577530in}{3.443375in}}{\pgfqpoint{8.588580in}{3.443375in}}%
\pgfpathlineto{\pgfqpoint{8.588580in}{3.443375in}}%
\pgfpathclose%
\pgfusepath{stroke}%
\end{pgfscope}%
\begin{pgfscope}%
\pgfpathrectangle{\pgfqpoint{7.512535in}{0.437222in}}{\pgfqpoint{6.275590in}{5.159444in}}%
\pgfusepath{clip}%
\pgfsetbuttcap%
\pgfsetroundjoin%
\pgfsetlinewidth{1.003750pt}%
\definecolor{currentstroke}{rgb}{0.827451,0.827451,0.827451}%
\pgfsetstrokecolor{currentstroke}%
\pgfsetstrokeopacity{0.800000}%
\pgfsetdash{}{0pt}%
\pgfpathmoveto{\pgfqpoint{8.598634in}{1.552340in}}%
\pgfpathcurveto{\pgfqpoint{8.609684in}{1.552340in}}{\pgfqpoint{8.620283in}{1.556730in}}{\pgfqpoint{8.628097in}{1.564544in}}%
\pgfpathcurveto{\pgfqpoint{8.635910in}{1.572358in}}{\pgfqpoint{8.640300in}{1.582957in}}{\pgfqpoint{8.640300in}{1.594007in}}%
\pgfpathcurveto{\pgfqpoint{8.640300in}{1.605057in}}{\pgfqpoint{8.635910in}{1.615656in}}{\pgfqpoint{8.628097in}{1.623470in}}%
\pgfpathcurveto{\pgfqpoint{8.620283in}{1.631283in}}{\pgfqpoint{8.609684in}{1.635673in}}{\pgfqpoint{8.598634in}{1.635673in}}%
\pgfpathcurveto{\pgfqpoint{8.587584in}{1.635673in}}{\pgfqpoint{8.576985in}{1.631283in}}{\pgfqpoint{8.569171in}{1.623470in}}%
\pgfpathcurveto{\pgfqpoint{8.561357in}{1.615656in}}{\pgfqpoint{8.556967in}{1.605057in}}{\pgfqpoint{8.556967in}{1.594007in}}%
\pgfpathcurveto{\pgfqpoint{8.556967in}{1.582957in}}{\pgfqpoint{8.561357in}{1.572358in}}{\pgfqpoint{8.569171in}{1.564544in}}%
\pgfpathcurveto{\pgfqpoint{8.576985in}{1.556730in}}{\pgfqpoint{8.587584in}{1.552340in}}{\pgfqpoint{8.598634in}{1.552340in}}%
\pgfpathlineto{\pgfqpoint{8.598634in}{1.552340in}}%
\pgfpathclose%
\pgfusepath{stroke}%
\end{pgfscope}%
\begin{pgfscope}%
\pgfpathrectangle{\pgfqpoint{7.512535in}{0.437222in}}{\pgfqpoint{6.275590in}{5.159444in}}%
\pgfusepath{clip}%
\pgfsetbuttcap%
\pgfsetroundjoin%
\pgfsetlinewidth{1.003750pt}%
\definecolor{currentstroke}{rgb}{0.827451,0.827451,0.827451}%
\pgfsetstrokecolor{currentstroke}%
\pgfsetstrokeopacity{0.800000}%
\pgfsetdash{}{0pt}%
\pgfpathmoveto{\pgfqpoint{9.697330in}{2.947479in}}%
\pgfpathcurveto{\pgfqpoint{9.708380in}{2.947479in}}{\pgfqpoint{9.718979in}{2.951869in}}{\pgfqpoint{9.726792in}{2.959683in}}%
\pgfpathcurveto{\pgfqpoint{9.734606in}{2.967496in}}{\pgfqpoint{9.738996in}{2.978095in}}{\pgfqpoint{9.738996in}{2.989145in}}%
\pgfpathcurveto{\pgfqpoint{9.738996in}{3.000195in}}{\pgfqpoint{9.734606in}{3.010794in}}{\pgfqpoint{9.726792in}{3.018608in}}%
\pgfpathcurveto{\pgfqpoint{9.718979in}{3.026422in}}{\pgfqpoint{9.708380in}{3.030812in}}{\pgfqpoint{9.697330in}{3.030812in}}%
\pgfpathcurveto{\pgfqpoint{9.686279in}{3.030812in}}{\pgfqpoint{9.675680in}{3.026422in}}{\pgfqpoint{9.667867in}{3.018608in}}%
\pgfpathcurveto{\pgfqpoint{9.660053in}{3.010794in}}{\pgfqpoint{9.655663in}{3.000195in}}{\pgfqpoint{9.655663in}{2.989145in}}%
\pgfpathcurveto{\pgfqpoint{9.655663in}{2.978095in}}{\pgfqpoint{9.660053in}{2.967496in}}{\pgfqpoint{9.667867in}{2.959683in}}%
\pgfpathcurveto{\pgfqpoint{9.675680in}{2.951869in}}{\pgfqpoint{9.686279in}{2.947479in}}{\pgfqpoint{9.697330in}{2.947479in}}%
\pgfpathlineto{\pgfqpoint{9.697330in}{2.947479in}}%
\pgfpathclose%
\pgfusepath{stroke}%
\end{pgfscope}%
\begin{pgfscope}%
\pgfpathrectangle{\pgfqpoint{7.512535in}{0.437222in}}{\pgfqpoint{6.275590in}{5.159444in}}%
\pgfusepath{clip}%
\pgfsetbuttcap%
\pgfsetroundjoin%
\pgfsetlinewidth{1.003750pt}%
\definecolor{currentstroke}{rgb}{0.827451,0.827451,0.827451}%
\pgfsetstrokecolor{currentstroke}%
\pgfsetstrokeopacity{0.800000}%
\pgfsetdash{}{0pt}%
\pgfpathmoveto{\pgfqpoint{12.446113in}{5.353825in}}%
\pgfpathcurveto{\pgfqpoint{12.457163in}{5.353825in}}{\pgfqpoint{12.467762in}{5.358215in}}{\pgfqpoint{12.475576in}{5.366029in}}%
\pgfpathcurveto{\pgfqpoint{12.483389in}{5.373843in}}{\pgfqpoint{12.487779in}{5.384442in}}{\pgfqpoint{12.487779in}{5.395492in}}%
\pgfpathcurveto{\pgfqpoint{12.487779in}{5.406542in}}{\pgfqpoint{12.483389in}{5.417141in}}{\pgfqpoint{12.475576in}{5.424955in}}%
\pgfpathcurveto{\pgfqpoint{12.467762in}{5.432768in}}{\pgfqpoint{12.457163in}{5.437158in}}{\pgfqpoint{12.446113in}{5.437158in}}%
\pgfpathcurveto{\pgfqpoint{12.435063in}{5.437158in}}{\pgfqpoint{12.424464in}{5.432768in}}{\pgfqpoint{12.416650in}{5.424955in}}%
\pgfpathcurveto{\pgfqpoint{12.408836in}{5.417141in}}{\pgfqpoint{12.404446in}{5.406542in}}{\pgfqpoint{12.404446in}{5.395492in}}%
\pgfpathcurveto{\pgfqpoint{12.404446in}{5.384442in}}{\pgfqpoint{12.408836in}{5.373843in}}{\pgfqpoint{12.416650in}{5.366029in}}%
\pgfpathcurveto{\pgfqpoint{12.424464in}{5.358215in}}{\pgfqpoint{12.435063in}{5.353825in}}{\pgfqpoint{12.446113in}{5.353825in}}%
\pgfpathlineto{\pgfqpoint{12.446113in}{5.353825in}}%
\pgfpathclose%
\pgfusepath{stroke}%
\end{pgfscope}%
\begin{pgfscope}%
\pgfpathrectangle{\pgfqpoint{7.512535in}{0.437222in}}{\pgfqpoint{6.275590in}{5.159444in}}%
\pgfusepath{clip}%
\pgfsetbuttcap%
\pgfsetroundjoin%
\pgfsetlinewidth{1.003750pt}%
\definecolor{currentstroke}{rgb}{0.827451,0.827451,0.827451}%
\pgfsetstrokecolor{currentstroke}%
\pgfsetstrokeopacity{0.800000}%
\pgfsetdash{}{0pt}%
\pgfpathmoveto{\pgfqpoint{9.394437in}{3.273900in}}%
\pgfpathcurveto{\pgfqpoint{9.405487in}{3.273900in}}{\pgfqpoint{9.416086in}{3.278290in}}{\pgfqpoint{9.423899in}{3.286104in}}%
\pgfpathcurveto{\pgfqpoint{9.431713in}{3.293917in}}{\pgfqpoint{9.436103in}{3.304516in}}{\pgfqpoint{9.436103in}{3.315567in}}%
\pgfpathcurveto{\pgfqpoint{9.436103in}{3.326617in}}{\pgfqpoint{9.431713in}{3.337216in}}{\pgfqpoint{9.423899in}{3.345029in}}%
\pgfpathcurveto{\pgfqpoint{9.416086in}{3.352843in}}{\pgfqpoint{9.405487in}{3.357233in}}{\pgfqpoint{9.394437in}{3.357233in}}%
\pgfpathcurveto{\pgfqpoint{9.383386in}{3.357233in}}{\pgfqpoint{9.372787in}{3.352843in}}{\pgfqpoint{9.364974in}{3.345029in}}%
\pgfpathcurveto{\pgfqpoint{9.357160in}{3.337216in}}{\pgfqpoint{9.352770in}{3.326617in}}{\pgfqpoint{9.352770in}{3.315567in}}%
\pgfpathcurveto{\pgfqpoint{9.352770in}{3.304516in}}{\pgfqpoint{9.357160in}{3.293917in}}{\pgfqpoint{9.364974in}{3.286104in}}%
\pgfpathcurveto{\pgfqpoint{9.372787in}{3.278290in}}{\pgfqpoint{9.383386in}{3.273900in}}{\pgfqpoint{9.394437in}{3.273900in}}%
\pgfpathlineto{\pgfqpoint{9.394437in}{3.273900in}}%
\pgfpathclose%
\pgfusepath{stroke}%
\end{pgfscope}%
\begin{pgfscope}%
\pgfpathrectangle{\pgfqpoint{7.512535in}{0.437222in}}{\pgfqpoint{6.275590in}{5.159444in}}%
\pgfusepath{clip}%
\pgfsetbuttcap%
\pgfsetroundjoin%
\pgfsetlinewidth{1.003750pt}%
\definecolor{currentstroke}{rgb}{0.827451,0.827451,0.827451}%
\pgfsetstrokecolor{currentstroke}%
\pgfsetstrokeopacity{0.800000}%
\pgfsetdash{}{0pt}%
\pgfpathmoveto{\pgfqpoint{11.645781in}{5.002607in}}%
\pgfpathcurveto{\pgfqpoint{11.656832in}{5.002607in}}{\pgfqpoint{11.667431in}{5.006998in}}{\pgfqpoint{11.675244in}{5.014811in}}%
\pgfpathcurveto{\pgfqpoint{11.683058in}{5.022625in}}{\pgfqpoint{11.687448in}{5.033224in}}{\pgfqpoint{11.687448in}{5.044274in}}%
\pgfpathcurveto{\pgfqpoint{11.687448in}{5.055324in}}{\pgfqpoint{11.683058in}{5.065923in}}{\pgfqpoint{11.675244in}{5.073737in}}%
\pgfpathcurveto{\pgfqpoint{11.667431in}{5.081550in}}{\pgfqpoint{11.656832in}{5.085941in}}{\pgfqpoint{11.645781in}{5.085941in}}%
\pgfpathcurveto{\pgfqpoint{11.634731in}{5.085941in}}{\pgfqpoint{11.624132in}{5.081550in}}{\pgfqpoint{11.616319in}{5.073737in}}%
\pgfpathcurveto{\pgfqpoint{11.608505in}{5.065923in}}{\pgfqpoint{11.604115in}{5.055324in}}{\pgfqpoint{11.604115in}{5.044274in}}%
\pgfpathcurveto{\pgfqpoint{11.604115in}{5.033224in}}{\pgfqpoint{11.608505in}{5.022625in}}{\pgfqpoint{11.616319in}{5.014811in}}%
\pgfpathcurveto{\pgfqpoint{11.624132in}{5.006998in}}{\pgfqpoint{11.634731in}{5.002607in}}{\pgfqpoint{11.645781in}{5.002607in}}%
\pgfpathlineto{\pgfqpoint{11.645781in}{5.002607in}}%
\pgfpathclose%
\pgfusepath{stroke}%
\end{pgfscope}%
\begin{pgfscope}%
\pgfpathrectangle{\pgfqpoint{7.512535in}{0.437222in}}{\pgfqpoint{6.275590in}{5.159444in}}%
\pgfusepath{clip}%
\pgfsetbuttcap%
\pgfsetroundjoin%
\pgfsetlinewidth{1.003750pt}%
\definecolor{currentstroke}{rgb}{0.827451,0.827451,0.827451}%
\pgfsetstrokecolor{currentstroke}%
\pgfsetstrokeopacity{0.800000}%
\pgfsetdash{}{0pt}%
\pgfpathmoveto{\pgfqpoint{10.127909in}{4.483356in}}%
\pgfpathcurveto{\pgfqpoint{10.138959in}{4.483356in}}{\pgfqpoint{10.149558in}{4.487746in}}{\pgfqpoint{10.157372in}{4.495560in}}%
\pgfpathcurveto{\pgfqpoint{10.165186in}{4.503373in}}{\pgfqpoint{10.169576in}{4.513972in}}{\pgfqpoint{10.169576in}{4.525022in}}%
\pgfpathcurveto{\pgfqpoint{10.169576in}{4.536072in}}{\pgfqpoint{10.165186in}{4.546671in}}{\pgfqpoint{10.157372in}{4.554485in}}%
\pgfpathcurveto{\pgfqpoint{10.149558in}{4.562299in}}{\pgfqpoint{10.138959in}{4.566689in}}{\pgfqpoint{10.127909in}{4.566689in}}%
\pgfpathcurveto{\pgfqpoint{10.116859in}{4.566689in}}{\pgfqpoint{10.106260in}{4.562299in}}{\pgfqpoint{10.098446in}{4.554485in}}%
\pgfpathcurveto{\pgfqpoint{10.090633in}{4.546671in}}{\pgfqpoint{10.086243in}{4.536072in}}{\pgfqpoint{10.086243in}{4.525022in}}%
\pgfpathcurveto{\pgfqpoint{10.086243in}{4.513972in}}{\pgfqpoint{10.090633in}{4.503373in}}{\pgfqpoint{10.098446in}{4.495560in}}%
\pgfpathcurveto{\pgfqpoint{10.106260in}{4.487746in}}{\pgfqpoint{10.116859in}{4.483356in}}{\pgfqpoint{10.127909in}{4.483356in}}%
\pgfpathlineto{\pgfqpoint{10.127909in}{4.483356in}}%
\pgfpathclose%
\pgfusepath{stroke}%
\end{pgfscope}%
\begin{pgfscope}%
\pgfpathrectangle{\pgfqpoint{7.512535in}{0.437222in}}{\pgfqpoint{6.275590in}{5.159444in}}%
\pgfusepath{clip}%
\pgfsetbuttcap%
\pgfsetroundjoin%
\pgfsetlinewidth{1.003750pt}%
\definecolor{currentstroke}{rgb}{0.827451,0.827451,0.827451}%
\pgfsetstrokecolor{currentstroke}%
\pgfsetstrokeopacity{0.800000}%
\pgfsetdash{}{0pt}%
\pgfpathmoveto{\pgfqpoint{10.413616in}{4.447570in}}%
\pgfpathcurveto{\pgfqpoint{10.424666in}{4.447570in}}{\pgfqpoint{10.435265in}{4.451960in}}{\pgfqpoint{10.443078in}{4.459773in}}%
\pgfpathcurveto{\pgfqpoint{10.450892in}{4.467587in}}{\pgfqpoint{10.455282in}{4.478186in}}{\pgfqpoint{10.455282in}{4.489236in}}%
\pgfpathcurveto{\pgfqpoint{10.455282in}{4.500286in}}{\pgfqpoint{10.450892in}{4.510885in}}{\pgfqpoint{10.443078in}{4.518699in}}%
\pgfpathcurveto{\pgfqpoint{10.435265in}{4.526513in}}{\pgfqpoint{10.424666in}{4.530903in}}{\pgfqpoint{10.413616in}{4.530903in}}%
\pgfpathcurveto{\pgfqpoint{10.402565in}{4.530903in}}{\pgfqpoint{10.391966in}{4.526513in}}{\pgfqpoint{10.384153in}{4.518699in}}%
\pgfpathcurveto{\pgfqpoint{10.376339in}{4.510885in}}{\pgfqpoint{10.371949in}{4.500286in}}{\pgfqpoint{10.371949in}{4.489236in}}%
\pgfpathcurveto{\pgfqpoint{10.371949in}{4.478186in}}{\pgfqpoint{10.376339in}{4.467587in}}{\pgfqpoint{10.384153in}{4.459773in}}%
\pgfpathcurveto{\pgfqpoint{10.391966in}{4.451960in}}{\pgfqpoint{10.402565in}{4.447570in}}{\pgfqpoint{10.413616in}{4.447570in}}%
\pgfpathlineto{\pgfqpoint{10.413616in}{4.447570in}}%
\pgfpathclose%
\pgfusepath{stroke}%
\end{pgfscope}%
\begin{pgfscope}%
\pgfpathrectangle{\pgfqpoint{7.512535in}{0.437222in}}{\pgfqpoint{6.275590in}{5.159444in}}%
\pgfusepath{clip}%
\pgfsetbuttcap%
\pgfsetroundjoin%
\pgfsetlinewidth{1.003750pt}%
\definecolor{currentstroke}{rgb}{0.827451,0.827451,0.827451}%
\pgfsetstrokecolor{currentstroke}%
\pgfsetstrokeopacity{0.800000}%
\pgfsetdash{}{0pt}%
\pgfpathmoveto{\pgfqpoint{9.839703in}{1.834702in}}%
\pgfpathcurveto{\pgfqpoint{9.850753in}{1.834702in}}{\pgfqpoint{9.861352in}{1.839092in}}{\pgfqpoint{9.869165in}{1.846906in}}%
\pgfpathcurveto{\pgfqpoint{9.876979in}{1.854719in}}{\pgfqpoint{9.881369in}{1.865318in}}{\pgfqpoint{9.881369in}{1.876369in}}%
\pgfpathcurveto{\pgfqpoint{9.881369in}{1.887419in}}{\pgfqpoint{9.876979in}{1.898018in}}{\pgfqpoint{9.869165in}{1.905831in}}%
\pgfpathcurveto{\pgfqpoint{9.861352in}{1.913645in}}{\pgfqpoint{9.850753in}{1.918035in}}{\pgfqpoint{9.839703in}{1.918035in}}%
\pgfpathcurveto{\pgfqpoint{9.828652in}{1.918035in}}{\pgfqpoint{9.818053in}{1.913645in}}{\pgfqpoint{9.810240in}{1.905831in}}%
\pgfpathcurveto{\pgfqpoint{9.802426in}{1.898018in}}{\pgfqpoint{9.798036in}{1.887419in}}{\pgfqpoint{9.798036in}{1.876369in}}%
\pgfpathcurveto{\pgfqpoint{9.798036in}{1.865318in}}{\pgfqpoint{9.802426in}{1.854719in}}{\pgfqpoint{9.810240in}{1.846906in}}%
\pgfpathcurveto{\pgfqpoint{9.818053in}{1.839092in}}{\pgfqpoint{9.828652in}{1.834702in}}{\pgfqpoint{9.839703in}{1.834702in}}%
\pgfpathlineto{\pgfqpoint{9.839703in}{1.834702in}}%
\pgfpathclose%
\pgfusepath{stroke}%
\end{pgfscope}%
\begin{pgfscope}%
\pgfpathrectangle{\pgfqpoint{7.512535in}{0.437222in}}{\pgfqpoint{6.275590in}{5.159444in}}%
\pgfusepath{clip}%
\pgfsetbuttcap%
\pgfsetroundjoin%
\pgfsetlinewidth{1.003750pt}%
\definecolor{currentstroke}{rgb}{0.827451,0.827451,0.827451}%
\pgfsetstrokecolor{currentstroke}%
\pgfsetstrokeopacity{0.800000}%
\pgfsetdash{}{0pt}%
\pgfpathmoveto{\pgfqpoint{7.946023in}{1.673751in}}%
\pgfpathcurveto{\pgfqpoint{7.957073in}{1.673751in}}{\pgfqpoint{7.967672in}{1.678142in}}{\pgfqpoint{7.975486in}{1.685955in}}%
\pgfpathcurveto{\pgfqpoint{7.983300in}{1.693769in}}{\pgfqpoint{7.987690in}{1.704368in}}{\pgfqpoint{7.987690in}{1.715418in}}%
\pgfpathcurveto{\pgfqpoint{7.987690in}{1.726468in}}{\pgfqpoint{7.983300in}{1.737067in}}{\pgfqpoint{7.975486in}{1.744881in}}%
\pgfpathcurveto{\pgfqpoint{7.967672in}{1.752694in}}{\pgfqpoint{7.957073in}{1.757085in}}{\pgfqpoint{7.946023in}{1.757085in}}%
\pgfpathcurveto{\pgfqpoint{7.934973in}{1.757085in}}{\pgfqpoint{7.924374in}{1.752694in}}{\pgfqpoint{7.916560in}{1.744881in}}%
\pgfpathcurveto{\pgfqpoint{7.908747in}{1.737067in}}{\pgfqpoint{7.904357in}{1.726468in}}{\pgfqpoint{7.904357in}{1.715418in}}%
\pgfpathcurveto{\pgfqpoint{7.904357in}{1.704368in}}{\pgfqpoint{7.908747in}{1.693769in}}{\pgfqpoint{7.916560in}{1.685955in}}%
\pgfpathcurveto{\pgfqpoint{7.924374in}{1.678142in}}{\pgfqpoint{7.934973in}{1.673751in}}{\pgfqpoint{7.946023in}{1.673751in}}%
\pgfpathlineto{\pgfqpoint{7.946023in}{1.673751in}}%
\pgfpathclose%
\pgfusepath{stroke}%
\end{pgfscope}%
\begin{pgfscope}%
\pgfpathrectangle{\pgfqpoint{7.512535in}{0.437222in}}{\pgfqpoint{6.275590in}{5.159444in}}%
\pgfusepath{clip}%
\pgfsetbuttcap%
\pgfsetroundjoin%
\pgfsetlinewidth{1.003750pt}%
\definecolor{currentstroke}{rgb}{0.827451,0.827451,0.827451}%
\pgfsetstrokecolor{currentstroke}%
\pgfsetstrokeopacity{0.800000}%
\pgfsetdash{}{0pt}%
\pgfpathmoveto{\pgfqpoint{8.261759in}{2.309353in}}%
\pgfpathcurveto{\pgfqpoint{8.272809in}{2.309353in}}{\pgfqpoint{8.283408in}{2.313744in}}{\pgfqpoint{8.291221in}{2.321557in}}%
\pgfpathcurveto{\pgfqpoint{8.299035in}{2.329371in}}{\pgfqpoint{8.303425in}{2.339970in}}{\pgfqpoint{8.303425in}{2.351020in}}%
\pgfpathcurveto{\pgfqpoint{8.303425in}{2.362070in}}{\pgfqpoint{8.299035in}{2.372669in}}{\pgfqpoint{8.291221in}{2.380483in}}%
\pgfpathcurveto{\pgfqpoint{8.283408in}{2.388296in}}{\pgfqpoint{8.272809in}{2.392687in}}{\pgfqpoint{8.261759in}{2.392687in}}%
\pgfpathcurveto{\pgfqpoint{8.250708in}{2.392687in}}{\pgfqpoint{8.240109in}{2.388296in}}{\pgfqpoint{8.232296in}{2.380483in}}%
\pgfpathcurveto{\pgfqpoint{8.224482in}{2.372669in}}{\pgfqpoint{8.220092in}{2.362070in}}{\pgfqpoint{8.220092in}{2.351020in}}%
\pgfpathcurveto{\pgfqpoint{8.220092in}{2.339970in}}{\pgfqpoint{8.224482in}{2.329371in}}{\pgfqpoint{8.232296in}{2.321557in}}%
\pgfpathcurveto{\pgfqpoint{8.240109in}{2.313744in}}{\pgfqpoint{8.250708in}{2.309353in}}{\pgfqpoint{8.261759in}{2.309353in}}%
\pgfpathlineto{\pgfqpoint{8.261759in}{2.309353in}}%
\pgfpathclose%
\pgfusepath{stroke}%
\end{pgfscope}%
\begin{pgfscope}%
\pgfpathrectangle{\pgfqpoint{7.512535in}{0.437222in}}{\pgfqpoint{6.275590in}{5.159444in}}%
\pgfusepath{clip}%
\pgfsetbuttcap%
\pgfsetroundjoin%
\pgfsetlinewidth{1.003750pt}%
\definecolor{currentstroke}{rgb}{0.827451,0.827451,0.827451}%
\pgfsetstrokecolor{currentstroke}%
\pgfsetstrokeopacity{0.800000}%
\pgfsetdash{}{0pt}%
\pgfpathmoveto{\pgfqpoint{12.081725in}{5.113368in}}%
\pgfpathcurveto{\pgfqpoint{12.092776in}{5.113368in}}{\pgfqpoint{12.103375in}{5.117758in}}{\pgfqpoint{12.111188in}{5.125572in}}%
\pgfpathcurveto{\pgfqpoint{12.119002in}{5.133385in}}{\pgfqpoint{12.123392in}{5.143984in}}{\pgfqpoint{12.123392in}{5.155034in}}%
\pgfpathcurveto{\pgfqpoint{12.123392in}{5.166084in}}{\pgfqpoint{12.119002in}{5.176683in}}{\pgfqpoint{12.111188in}{5.184497in}}%
\pgfpathcurveto{\pgfqpoint{12.103375in}{5.192311in}}{\pgfqpoint{12.092776in}{5.196701in}}{\pgfqpoint{12.081725in}{5.196701in}}%
\pgfpathcurveto{\pgfqpoint{12.070675in}{5.196701in}}{\pgfqpoint{12.060076in}{5.192311in}}{\pgfqpoint{12.052263in}{5.184497in}}%
\pgfpathcurveto{\pgfqpoint{12.044449in}{5.176683in}}{\pgfqpoint{12.040059in}{5.166084in}}{\pgfqpoint{12.040059in}{5.155034in}}%
\pgfpathcurveto{\pgfqpoint{12.040059in}{5.143984in}}{\pgfqpoint{12.044449in}{5.133385in}}{\pgfqpoint{12.052263in}{5.125572in}}%
\pgfpathcurveto{\pgfqpoint{12.060076in}{5.117758in}}{\pgfqpoint{12.070675in}{5.113368in}}{\pgfqpoint{12.081725in}{5.113368in}}%
\pgfpathlineto{\pgfqpoint{12.081725in}{5.113368in}}%
\pgfpathclose%
\pgfusepath{stroke}%
\end{pgfscope}%
\begin{pgfscope}%
\pgfpathrectangle{\pgfqpoint{7.512535in}{0.437222in}}{\pgfqpoint{6.275590in}{5.159444in}}%
\pgfusepath{clip}%
\pgfsetbuttcap%
\pgfsetroundjoin%
\pgfsetlinewidth{1.003750pt}%
\definecolor{currentstroke}{rgb}{0.827451,0.827451,0.827451}%
\pgfsetstrokecolor{currentstroke}%
\pgfsetstrokeopacity{0.800000}%
\pgfsetdash{}{0pt}%
\pgfpathmoveto{\pgfqpoint{10.013603in}{2.917799in}}%
\pgfpathcurveto{\pgfqpoint{10.024653in}{2.917799in}}{\pgfqpoint{10.035252in}{2.922190in}}{\pgfqpoint{10.043066in}{2.930003in}}%
\pgfpathcurveto{\pgfqpoint{10.050880in}{2.937817in}}{\pgfqpoint{10.055270in}{2.948416in}}{\pgfqpoint{10.055270in}{2.959466in}}%
\pgfpathcurveto{\pgfqpoint{10.055270in}{2.970516in}}{\pgfqpoint{10.050880in}{2.981115in}}{\pgfqpoint{10.043066in}{2.988929in}}%
\pgfpathcurveto{\pgfqpoint{10.035252in}{2.996743in}}{\pgfqpoint{10.024653in}{3.001133in}}{\pgfqpoint{10.013603in}{3.001133in}}%
\pgfpathcurveto{\pgfqpoint{10.002553in}{3.001133in}}{\pgfqpoint{9.991954in}{2.996743in}}{\pgfqpoint{9.984141in}{2.988929in}}%
\pgfpathcurveto{\pgfqpoint{9.976327in}{2.981115in}}{\pgfqpoint{9.971937in}{2.970516in}}{\pgfqpoint{9.971937in}{2.959466in}}%
\pgfpathcurveto{\pgfqpoint{9.971937in}{2.948416in}}{\pgfqpoint{9.976327in}{2.937817in}}{\pgfqpoint{9.984141in}{2.930003in}}%
\pgfpathcurveto{\pgfqpoint{9.991954in}{2.922190in}}{\pgfqpoint{10.002553in}{2.917799in}}{\pgfqpoint{10.013603in}{2.917799in}}%
\pgfpathlineto{\pgfqpoint{10.013603in}{2.917799in}}%
\pgfpathclose%
\pgfusepath{stroke}%
\end{pgfscope}%
\begin{pgfscope}%
\pgfpathrectangle{\pgfqpoint{7.512535in}{0.437222in}}{\pgfqpoint{6.275590in}{5.159444in}}%
\pgfusepath{clip}%
\pgfsetbuttcap%
\pgfsetroundjoin%
\pgfsetlinewidth{1.003750pt}%
\definecolor{currentstroke}{rgb}{0.827451,0.827451,0.827451}%
\pgfsetstrokecolor{currentstroke}%
\pgfsetstrokeopacity{0.800000}%
\pgfsetdash{}{0pt}%
\pgfpathmoveto{\pgfqpoint{10.873384in}{4.145170in}}%
\pgfpathcurveto{\pgfqpoint{10.884434in}{4.145170in}}{\pgfqpoint{10.895033in}{4.149560in}}{\pgfqpoint{10.902847in}{4.157374in}}%
\pgfpathcurveto{\pgfqpoint{10.910660in}{4.165188in}}{\pgfqpoint{10.915050in}{4.175787in}}{\pgfqpoint{10.915050in}{4.186837in}}%
\pgfpathcurveto{\pgfqpoint{10.915050in}{4.197887in}}{\pgfqpoint{10.910660in}{4.208486in}}{\pgfqpoint{10.902847in}{4.216300in}}%
\pgfpathcurveto{\pgfqpoint{10.895033in}{4.224113in}}{\pgfqpoint{10.884434in}{4.228504in}}{\pgfqpoint{10.873384in}{4.228504in}}%
\pgfpathcurveto{\pgfqpoint{10.862334in}{4.228504in}}{\pgfqpoint{10.851735in}{4.224113in}}{\pgfqpoint{10.843921in}{4.216300in}}%
\pgfpathcurveto{\pgfqpoint{10.836107in}{4.208486in}}{\pgfqpoint{10.831717in}{4.197887in}}{\pgfqpoint{10.831717in}{4.186837in}}%
\pgfpathcurveto{\pgfqpoint{10.831717in}{4.175787in}}{\pgfqpoint{10.836107in}{4.165188in}}{\pgfqpoint{10.843921in}{4.157374in}}%
\pgfpathcurveto{\pgfqpoint{10.851735in}{4.149560in}}{\pgfqpoint{10.862334in}{4.145170in}}{\pgfqpoint{10.873384in}{4.145170in}}%
\pgfpathlineto{\pgfqpoint{10.873384in}{4.145170in}}%
\pgfpathclose%
\pgfusepath{stroke}%
\end{pgfscope}%
\begin{pgfscope}%
\pgfpathrectangle{\pgfqpoint{7.512535in}{0.437222in}}{\pgfqpoint{6.275590in}{5.159444in}}%
\pgfusepath{clip}%
\pgfsetbuttcap%
\pgfsetroundjoin%
\pgfsetlinewidth{1.003750pt}%
\definecolor{currentstroke}{rgb}{0.827451,0.827451,0.827451}%
\pgfsetstrokecolor{currentstroke}%
\pgfsetstrokeopacity{0.800000}%
\pgfsetdash{}{0pt}%
\pgfpathmoveto{\pgfqpoint{10.346895in}{5.477998in}}%
\pgfpathcurveto{\pgfqpoint{10.357945in}{5.477998in}}{\pgfqpoint{10.368544in}{5.482389in}}{\pgfqpoint{10.376358in}{5.490202in}}%
\pgfpathcurveto{\pgfqpoint{10.384171in}{5.498016in}}{\pgfqpoint{10.388562in}{5.508615in}}{\pgfqpoint{10.388562in}{5.519665in}}%
\pgfpathcurveto{\pgfqpoint{10.388562in}{5.530715in}}{\pgfqpoint{10.384171in}{5.541314in}}{\pgfqpoint{10.376358in}{5.549128in}}%
\pgfpathcurveto{\pgfqpoint{10.368544in}{5.556941in}}{\pgfqpoint{10.357945in}{5.561332in}}{\pgfqpoint{10.346895in}{5.561332in}}%
\pgfpathcurveto{\pgfqpoint{10.335845in}{5.561332in}}{\pgfqpoint{10.325246in}{5.556941in}}{\pgfqpoint{10.317432in}{5.549128in}}%
\pgfpathcurveto{\pgfqpoint{10.309619in}{5.541314in}}{\pgfqpoint{10.305228in}{5.530715in}}{\pgfqpoint{10.305228in}{5.519665in}}%
\pgfpathcurveto{\pgfqpoint{10.305228in}{5.508615in}}{\pgfqpoint{10.309619in}{5.498016in}}{\pgfqpoint{10.317432in}{5.490202in}}%
\pgfpathcurveto{\pgfqpoint{10.325246in}{5.482389in}}{\pgfqpoint{10.335845in}{5.477998in}}{\pgfqpoint{10.346895in}{5.477998in}}%
\pgfpathlineto{\pgfqpoint{10.346895in}{5.477998in}}%
\pgfpathclose%
\pgfusepath{stroke}%
\end{pgfscope}%
\begin{pgfscope}%
\pgfpathrectangle{\pgfqpoint{7.512535in}{0.437222in}}{\pgfqpoint{6.275590in}{5.159444in}}%
\pgfusepath{clip}%
\pgfsetbuttcap%
\pgfsetroundjoin%
\pgfsetlinewidth{1.003750pt}%
\definecolor{currentstroke}{rgb}{0.827451,0.827451,0.827451}%
\pgfsetstrokecolor{currentstroke}%
\pgfsetstrokeopacity{0.800000}%
\pgfsetdash{}{0pt}%
\pgfpathmoveto{\pgfqpoint{9.862917in}{4.507322in}}%
\pgfpathcurveto{\pgfqpoint{9.873967in}{4.507322in}}{\pgfqpoint{9.884566in}{4.511712in}}{\pgfqpoint{9.892380in}{4.519525in}}%
\pgfpathcurveto{\pgfqpoint{9.900194in}{4.527339in}}{\pgfqpoint{9.904584in}{4.537938in}}{\pgfqpoint{9.904584in}{4.548988in}}%
\pgfpathcurveto{\pgfqpoint{9.904584in}{4.560038in}}{\pgfqpoint{9.900194in}{4.570637in}}{\pgfqpoint{9.892380in}{4.578451in}}%
\pgfpathcurveto{\pgfqpoint{9.884566in}{4.586265in}}{\pgfqpoint{9.873967in}{4.590655in}}{\pgfqpoint{9.862917in}{4.590655in}}%
\pgfpathcurveto{\pgfqpoint{9.851867in}{4.590655in}}{\pgfqpoint{9.841268in}{4.586265in}}{\pgfqpoint{9.833454in}{4.578451in}}%
\pgfpathcurveto{\pgfqpoint{9.825641in}{4.570637in}}{\pgfqpoint{9.821250in}{4.560038in}}{\pgfqpoint{9.821250in}{4.548988in}}%
\pgfpathcurveto{\pgfqpoint{9.821250in}{4.537938in}}{\pgfqpoint{9.825641in}{4.527339in}}{\pgfqpoint{9.833454in}{4.519525in}}%
\pgfpathcurveto{\pgfqpoint{9.841268in}{4.511712in}}{\pgfqpoint{9.851867in}{4.507322in}}{\pgfqpoint{9.862917in}{4.507322in}}%
\pgfpathlineto{\pgfqpoint{9.862917in}{4.507322in}}%
\pgfpathclose%
\pgfusepath{stroke}%
\end{pgfscope}%
\begin{pgfscope}%
\pgfpathrectangle{\pgfqpoint{7.512535in}{0.437222in}}{\pgfqpoint{6.275590in}{5.159444in}}%
\pgfusepath{clip}%
\pgfsetbuttcap%
\pgfsetroundjoin%
\pgfsetlinewidth{1.003750pt}%
\definecolor{currentstroke}{rgb}{0.827451,0.827451,0.827451}%
\pgfsetstrokecolor{currentstroke}%
\pgfsetstrokeopacity{0.800000}%
\pgfsetdash{}{0pt}%
\pgfpathmoveto{\pgfqpoint{9.645576in}{4.575873in}}%
\pgfpathcurveto{\pgfqpoint{9.656626in}{4.575873in}}{\pgfqpoint{9.667225in}{4.580263in}}{\pgfqpoint{9.675039in}{4.588076in}}%
\pgfpathcurveto{\pgfqpoint{9.682852in}{4.595890in}}{\pgfqpoint{9.687243in}{4.606489in}}{\pgfqpoint{9.687243in}{4.617539in}}%
\pgfpathcurveto{\pgfqpoint{9.687243in}{4.628589in}}{\pgfqpoint{9.682852in}{4.639188in}}{\pgfqpoint{9.675039in}{4.647002in}}%
\pgfpathcurveto{\pgfqpoint{9.667225in}{4.654816in}}{\pgfqpoint{9.656626in}{4.659206in}}{\pgfqpoint{9.645576in}{4.659206in}}%
\pgfpathcurveto{\pgfqpoint{9.634526in}{4.659206in}}{\pgfqpoint{9.623927in}{4.654816in}}{\pgfqpoint{9.616113in}{4.647002in}}%
\pgfpathcurveto{\pgfqpoint{9.608300in}{4.639188in}}{\pgfqpoint{9.603909in}{4.628589in}}{\pgfqpoint{9.603909in}{4.617539in}}%
\pgfpathcurveto{\pgfqpoint{9.603909in}{4.606489in}}{\pgfqpoint{9.608300in}{4.595890in}}{\pgfqpoint{9.616113in}{4.588076in}}%
\pgfpathcurveto{\pgfqpoint{9.623927in}{4.580263in}}{\pgfqpoint{9.634526in}{4.575873in}}{\pgfqpoint{9.645576in}{4.575873in}}%
\pgfpathlineto{\pgfqpoint{9.645576in}{4.575873in}}%
\pgfpathclose%
\pgfusepath{stroke}%
\end{pgfscope}%
\begin{pgfscope}%
\pgfpathrectangle{\pgfqpoint{7.512535in}{0.437222in}}{\pgfqpoint{6.275590in}{5.159444in}}%
\pgfusepath{clip}%
\pgfsetbuttcap%
\pgfsetroundjoin%
\pgfsetlinewidth{1.003750pt}%
\definecolor{currentstroke}{rgb}{0.827451,0.827451,0.827451}%
\pgfsetstrokecolor{currentstroke}%
\pgfsetstrokeopacity{0.800000}%
\pgfsetdash{}{0pt}%
\pgfpathmoveto{\pgfqpoint{8.087355in}{0.688351in}}%
\pgfpathcurveto{\pgfqpoint{8.098405in}{0.688351in}}{\pgfqpoint{8.109004in}{0.692741in}}{\pgfqpoint{8.116818in}{0.700555in}}%
\pgfpathcurveto{\pgfqpoint{8.124631in}{0.708368in}}{\pgfqpoint{8.129022in}{0.718967in}}{\pgfqpoint{8.129022in}{0.730017in}}%
\pgfpathcurveto{\pgfqpoint{8.129022in}{0.741068in}}{\pgfqpoint{8.124631in}{0.751667in}}{\pgfqpoint{8.116818in}{0.759480in}}%
\pgfpathcurveto{\pgfqpoint{8.109004in}{0.767294in}}{\pgfqpoint{8.098405in}{0.771684in}}{\pgfqpoint{8.087355in}{0.771684in}}%
\pgfpathcurveto{\pgfqpoint{8.076305in}{0.771684in}}{\pgfqpoint{8.065706in}{0.767294in}}{\pgfqpoint{8.057892in}{0.759480in}}%
\pgfpathcurveto{\pgfqpoint{8.050079in}{0.751667in}}{\pgfqpoint{8.045688in}{0.741068in}}{\pgfqpoint{8.045688in}{0.730017in}}%
\pgfpathcurveto{\pgfqpoint{8.045688in}{0.718967in}}{\pgfqpoint{8.050079in}{0.708368in}}{\pgfqpoint{8.057892in}{0.700555in}}%
\pgfpathcurveto{\pgfqpoint{8.065706in}{0.692741in}}{\pgfqpoint{8.076305in}{0.688351in}}{\pgfqpoint{8.087355in}{0.688351in}}%
\pgfpathlineto{\pgfqpoint{8.087355in}{0.688351in}}%
\pgfpathclose%
\pgfusepath{stroke}%
\end{pgfscope}%
\begin{pgfscope}%
\pgfpathrectangle{\pgfqpoint{7.512535in}{0.437222in}}{\pgfqpoint{6.275590in}{5.159444in}}%
\pgfusepath{clip}%
\pgfsetbuttcap%
\pgfsetroundjoin%
\pgfsetlinewidth{1.003750pt}%
\definecolor{currentstroke}{rgb}{0.827451,0.827451,0.827451}%
\pgfsetstrokecolor{currentstroke}%
\pgfsetstrokeopacity{0.800000}%
\pgfsetdash{}{0pt}%
\pgfpathmoveto{\pgfqpoint{11.431549in}{4.628942in}}%
\pgfpathcurveto{\pgfqpoint{11.442599in}{4.628942in}}{\pgfqpoint{11.453198in}{4.633333in}}{\pgfqpoint{11.461012in}{4.641146in}}%
\pgfpathcurveto{\pgfqpoint{11.468825in}{4.648960in}}{\pgfqpoint{11.473215in}{4.659559in}}{\pgfqpoint{11.473215in}{4.670609in}}%
\pgfpathcurveto{\pgfqpoint{11.473215in}{4.681659in}}{\pgfqpoint{11.468825in}{4.692258in}}{\pgfqpoint{11.461012in}{4.700072in}}%
\pgfpathcurveto{\pgfqpoint{11.453198in}{4.707885in}}{\pgfqpoint{11.442599in}{4.712276in}}{\pgfqpoint{11.431549in}{4.712276in}}%
\pgfpathcurveto{\pgfqpoint{11.420499in}{4.712276in}}{\pgfqpoint{11.409900in}{4.707885in}}{\pgfqpoint{11.402086in}{4.700072in}}%
\pgfpathcurveto{\pgfqpoint{11.394272in}{4.692258in}}{\pgfqpoint{11.389882in}{4.681659in}}{\pgfqpoint{11.389882in}{4.670609in}}%
\pgfpathcurveto{\pgfqpoint{11.389882in}{4.659559in}}{\pgfqpoint{11.394272in}{4.648960in}}{\pgfqpoint{11.402086in}{4.641146in}}%
\pgfpathcurveto{\pgfqpoint{11.409900in}{4.633333in}}{\pgfqpoint{11.420499in}{4.628942in}}{\pgfqpoint{11.431549in}{4.628942in}}%
\pgfpathlineto{\pgfqpoint{11.431549in}{4.628942in}}%
\pgfpathclose%
\pgfusepath{stroke}%
\end{pgfscope}%
\begin{pgfscope}%
\pgfpathrectangle{\pgfqpoint{7.512535in}{0.437222in}}{\pgfqpoint{6.275590in}{5.159444in}}%
\pgfusepath{clip}%
\pgfsetbuttcap%
\pgfsetroundjoin%
\pgfsetlinewidth{1.003750pt}%
\definecolor{currentstroke}{rgb}{0.827451,0.827451,0.827451}%
\pgfsetstrokecolor{currentstroke}%
\pgfsetstrokeopacity{0.800000}%
\pgfsetdash{}{0pt}%
\pgfpathmoveto{\pgfqpoint{12.342968in}{5.547874in}}%
\pgfpathcurveto{\pgfqpoint{12.354019in}{5.547874in}}{\pgfqpoint{12.364618in}{5.552265in}}{\pgfqpoint{12.372431in}{5.560078in}}%
\pgfpathcurveto{\pgfqpoint{12.380245in}{5.567892in}}{\pgfqpoint{12.384635in}{5.578491in}}{\pgfqpoint{12.384635in}{5.589541in}}%
\pgfpathcurveto{\pgfqpoint{12.384635in}{5.600591in}}{\pgfqpoint{12.380245in}{5.611190in}}{\pgfqpoint{12.372431in}{5.619004in}}%
\pgfpathcurveto{\pgfqpoint{12.364618in}{5.626817in}}{\pgfqpoint{12.354019in}{5.631208in}}{\pgfqpoint{12.342968in}{5.631208in}}%
\pgfpathcurveto{\pgfqpoint{12.331918in}{5.631208in}}{\pgfqpoint{12.321319in}{5.626817in}}{\pgfqpoint{12.313506in}{5.619004in}}%
\pgfpathcurveto{\pgfqpoint{12.305692in}{5.611190in}}{\pgfqpoint{12.301302in}{5.600591in}}{\pgfqpoint{12.301302in}{5.589541in}}%
\pgfpathcurveto{\pgfqpoint{12.301302in}{5.578491in}}{\pgfqpoint{12.305692in}{5.567892in}}{\pgfqpoint{12.313506in}{5.560078in}}%
\pgfpathcurveto{\pgfqpoint{12.321319in}{5.552265in}}{\pgfqpoint{12.331918in}{5.547874in}}{\pgfqpoint{12.342968in}{5.547874in}}%
\pgfpathlineto{\pgfqpoint{12.342968in}{5.547874in}}%
\pgfpathclose%
\pgfusepath{stroke}%
\end{pgfscope}%
\begin{pgfscope}%
\pgfpathrectangle{\pgfqpoint{7.512535in}{0.437222in}}{\pgfqpoint{6.275590in}{5.159444in}}%
\pgfusepath{clip}%
\pgfsetbuttcap%
\pgfsetroundjoin%
\pgfsetlinewidth{1.003750pt}%
\definecolor{currentstroke}{rgb}{0.827451,0.827451,0.827451}%
\pgfsetstrokecolor{currentstroke}%
\pgfsetstrokeopacity{0.800000}%
\pgfsetdash{}{0pt}%
\pgfpathmoveto{\pgfqpoint{7.983939in}{1.000624in}}%
\pgfpathcurveto{\pgfqpoint{7.994989in}{1.000624in}}{\pgfqpoint{8.005588in}{1.005015in}}{\pgfqpoint{8.013401in}{1.012828in}}%
\pgfpathcurveto{\pgfqpoint{8.021215in}{1.020642in}}{\pgfqpoint{8.025605in}{1.031241in}}{\pgfqpoint{8.025605in}{1.042291in}}%
\pgfpathcurveto{\pgfqpoint{8.025605in}{1.053341in}}{\pgfqpoint{8.021215in}{1.063940in}}{\pgfqpoint{8.013401in}{1.071754in}}%
\pgfpathcurveto{\pgfqpoint{8.005588in}{1.079568in}}{\pgfqpoint{7.994989in}{1.083958in}}{\pgfqpoint{7.983939in}{1.083958in}}%
\pgfpathcurveto{\pgfqpoint{7.972888in}{1.083958in}}{\pgfqpoint{7.962289in}{1.079568in}}{\pgfqpoint{7.954476in}{1.071754in}}%
\pgfpathcurveto{\pgfqpoint{7.946662in}{1.063940in}}{\pgfqpoint{7.942272in}{1.053341in}}{\pgfqpoint{7.942272in}{1.042291in}}%
\pgfpathcurveto{\pgfqpoint{7.942272in}{1.031241in}}{\pgfqpoint{7.946662in}{1.020642in}}{\pgfqpoint{7.954476in}{1.012828in}}%
\pgfpathcurveto{\pgfqpoint{7.962289in}{1.005015in}}{\pgfqpoint{7.972888in}{1.000624in}}{\pgfqpoint{7.983939in}{1.000624in}}%
\pgfpathlineto{\pgfqpoint{7.983939in}{1.000624in}}%
\pgfpathclose%
\pgfusepath{stroke}%
\end{pgfscope}%
\begin{pgfscope}%
\pgfpathrectangle{\pgfqpoint{7.512535in}{0.437222in}}{\pgfqpoint{6.275590in}{5.159444in}}%
\pgfusepath{clip}%
\pgfsetbuttcap%
\pgfsetroundjoin%
\pgfsetlinewidth{1.003750pt}%
\definecolor{currentstroke}{rgb}{0.827451,0.827451,0.827451}%
\pgfsetstrokecolor{currentstroke}%
\pgfsetstrokeopacity{0.800000}%
\pgfsetdash{}{0pt}%
\pgfpathmoveto{\pgfqpoint{9.254659in}{4.532771in}}%
\pgfpathcurveto{\pgfqpoint{9.265709in}{4.532771in}}{\pgfqpoint{9.276308in}{4.537161in}}{\pgfqpoint{9.284122in}{4.544974in}}%
\pgfpathcurveto{\pgfqpoint{9.291935in}{4.552788in}}{\pgfqpoint{9.296326in}{4.563387in}}{\pgfqpoint{9.296326in}{4.574437in}}%
\pgfpathcurveto{\pgfqpoint{9.296326in}{4.585487in}}{\pgfqpoint{9.291935in}{4.596086in}}{\pgfqpoint{9.284122in}{4.603900in}}%
\pgfpathcurveto{\pgfqpoint{9.276308in}{4.611714in}}{\pgfqpoint{9.265709in}{4.616104in}}{\pgfqpoint{9.254659in}{4.616104in}}%
\pgfpathcurveto{\pgfqpoint{9.243609in}{4.616104in}}{\pgfqpoint{9.233010in}{4.611714in}}{\pgfqpoint{9.225196in}{4.603900in}}%
\pgfpathcurveto{\pgfqpoint{9.217383in}{4.596086in}}{\pgfqpoint{9.212992in}{4.585487in}}{\pgfqpoint{9.212992in}{4.574437in}}%
\pgfpathcurveto{\pgfqpoint{9.212992in}{4.563387in}}{\pgfqpoint{9.217383in}{4.552788in}}{\pgfqpoint{9.225196in}{4.544974in}}%
\pgfpathcurveto{\pgfqpoint{9.233010in}{4.537161in}}{\pgfqpoint{9.243609in}{4.532771in}}{\pgfqpoint{9.254659in}{4.532771in}}%
\pgfpathlineto{\pgfqpoint{9.254659in}{4.532771in}}%
\pgfpathclose%
\pgfusepath{stroke}%
\end{pgfscope}%
\begin{pgfscope}%
\pgfpathrectangle{\pgfqpoint{7.512535in}{0.437222in}}{\pgfqpoint{6.275590in}{5.159444in}}%
\pgfusepath{clip}%
\pgfsetbuttcap%
\pgfsetroundjoin%
\pgfsetlinewidth{1.003750pt}%
\definecolor{currentstroke}{rgb}{0.827451,0.827451,0.827451}%
\pgfsetstrokecolor{currentstroke}%
\pgfsetstrokeopacity{0.800000}%
\pgfsetdash{}{0pt}%
\pgfpathmoveto{\pgfqpoint{10.563346in}{5.087121in}}%
\pgfpathcurveto{\pgfqpoint{10.574396in}{5.087121in}}{\pgfqpoint{10.584995in}{5.091511in}}{\pgfqpoint{10.592808in}{5.099325in}}%
\pgfpathcurveto{\pgfqpoint{10.600622in}{5.107138in}}{\pgfqpoint{10.605012in}{5.117737in}}{\pgfqpoint{10.605012in}{5.128787in}}%
\pgfpathcurveto{\pgfqpoint{10.605012in}{5.139838in}}{\pgfqpoint{10.600622in}{5.150437in}}{\pgfqpoint{10.592808in}{5.158250in}}%
\pgfpathcurveto{\pgfqpoint{10.584995in}{5.166064in}}{\pgfqpoint{10.574396in}{5.170454in}}{\pgfqpoint{10.563346in}{5.170454in}}%
\pgfpathcurveto{\pgfqpoint{10.552295in}{5.170454in}}{\pgfqpoint{10.541696in}{5.166064in}}{\pgfqpoint{10.533883in}{5.158250in}}%
\pgfpathcurveto{\pgfqpoint{10.526069in}{5.150437in}}{\pgfqpoint{10.521679in}{5.139838in}}{\pgfqpoint{10.521679in}{5.128787in}}%
\pgfpathcurveto{\pgfqpoint{10.521679in}{5.117737in}}{\pgfqpoint{10.526069in}{5.107138in}}{\pgfqpoint{10.533883in}{5.099325in}}%
\pgfpathcurveto{\pgfqpoint{10.541696in}{5.091511in}}{\pgfqpoint{10.552295in}{5.087121in}}{\pgfqpoint{10.563346in}{5.087121in}}%
\pgfpathlineto{\pgfqpoint{10.563346in}{5.087121in}}%
\pgfpathclose%
\pgfusepath{stroke}%
\end{pgfscope}%
\begin{pgfscope}%
\pgfpathrectangle{\pgfqpoint{7.512535in}{0.437222in}}{\pgfqpoint{6.275590in}{5.159444in}}%
\pgfusepath{clip}%
\pgfsetbuttcap%
\pgfsetroundjoin%
\pgfsetlinewidth{1.003750pt}%
\definecolor{currentstroke}{rgb}{0.827451,0.827451,0.827451}%
\pgfsetstrokecolor{currentstroke}%
\pgfsetstrokeopacity{0.800000}%
\pgfsetdash{}{0pt}%
\pgfpathmoveto{\pgfqpoint{12.081489in}{5.112444in}}%
\pgfpathcurveto{\pgfqpoint{12.092539in}{5.112444in}}{\pgfqpoint{12.103138in}{5.116834in}}{\pgfqpoint{12.110952in}{5.124648in}}%
\pgfpathcurveto{\pgfqpoint{12.118765in}{5.132461in}}{\pgfqpoint{12.123156in}{5.143061in}}{\pgfqpoint{12.123156in}{5.154111in}}%
\pgfpathcurveto{\pgfqpoint{12.123156in}{5.165161in}}{\pgfqpoint{12.118765in}{5.175760in}}{\pgfqpoint{12.110952in}{5.183573in}}%
\pgfpathcurveto{\pgfqpoint{12.103138in}{5.191387in}}{\pgfqpoint{12.092539in}{5.195777in}}{\pgfqpoint{12.081489in}{5.195777in}}%
\pgfpathcurveto{\pgfqpoint{12.070439in}{5.195777in}}{\pgfqpoint{12.059840in}{5.191387in}}{\pgfqpoint{12.052026in}{5.183573in}}%
\pgfpathcurveto{\pgfqpoint{12.044212in}{5.175760in}}{\pgfqpoint{12.039822in}{5.165161in}}{\pgfqpoint{12.039822in}{5.154111in}}%
\pgfpathcurveto{\pgfqpoint{12.039822in}{5.143061in}}{\pgfqpoint{12.044212in}{5.132461in}}{\pgfqpoint{12.052026in}{5.124648in}}%
\pgfpathcurveto{\pgfqpoint{12.059840in}{5.116834in}}{\pgfqpoint{12.070439in}{5.112444in}}{\pgfqpoint{12.081489in}{5.112444in}}%
\pgfpathlineto{\pgfqpoint{12.081489in}{5.112444in}}%
\pgfpathclose%
\pgfusepath{stroke}%
\end{pgfscope}%
\begin{pgfscope}%
\pgfpathrectangle{\pgfqpoint{7.512535in}{0.437222in}}{\pgfqpoint{6.275590in}{5.159444in}}%
\pgfusepath{clip}%
\pgfsetbuttcap%
\pgfsetroundjoin%
\pgfsetlinewidth{1.003750pt}%
\definecolor{currentstroke}{rgb}{0.827451,0.827451,0.827451}%
\pgfsetstrokecolor{currentstroke}%
\pgfsetstrokeopacity{0.800000}%
\pgfsetdash{}{0pt}%
\pgfpathmoveto{\pgfqpoint{9.302653in}{3.946092in}}%
\pgfpathcurveto{\pgfqpoint{9.313703in}{3.946092in}}{\pgfqpoint{9.324302in}{3.950482in}}{\pgfqpoint{9.332116in}{3.958296in}}%
\pgfpathcurveto{\pgfqpoint{9.339929in}{3.966109in}}{\pgfqpoint{9.344320in}{3.976708in}}{\pgfqpoint{9.344320in}{3.987758in}}%
\pgfpathcurveto{\pgfqpoint{9.344320in}{3.998809in}}{\pgfqpoint{9.339929in}{4.009408in}}{\pgfqpoint{9.332116in}{4.017221in}}%
\pgfpathcurveto{\pgfqpoint{9.324302in}{4.025035in}}{\pgfqpoint{9.313703in}{4.029425in}}{\pgfqpoint{9.302653in}{4.029425in}}%
\pgfpathcurveto{\pgfqpoint{9.291603in}{4.029425in}}{\pgfqpoint{9.281004in}{4.025035in}}{\pgfqpoint{9.273190in}{4.017221in}}%
\pgfpathcurveto{\pgfqpoint{9.265376in}{4.009408in}}{\pgfqpoint{9.260986in}{3.998809in}}{\pgfqpoint{9.260986in}{3.987758in}}%
\pgfpathcurveto{\pgfqpoint{9.260986in}{3.976708in}}{\pgfqpoint{9.265376in}{3.966109in}}{\pgfqpoint{9.273190in}{3.958296in}}%
\pgfpathcurveto{\pgfqpoint{9.281004in}{3.950482in}}{\pgfqpoint{9.291603in}{3.946092in}}{\pgfqpoint{9.302653in}{3.946092in}}%
\pgfpathlineto{\pgfqpoint{9.302653in}{3.946092in}}%
\pgfpathclose%
\pgfusepath{stroke}%
\end{pgfscope}%
\begin{pgfscope}%
\pgfpathrectangle{\pgfqpoint{7.512535in}{0.437222in}}{\pgfqpoint{6.275590in}{5.159444in}}%
\pgfusepath{clip}%
\pgfsetbuttcap%
\pgfsetroundjoin%
\pgfsetlinewidth{1.003750pt}%
\definecolor{currentstroke}{rgb}{0.827451,0.827451,0.827451}%
\pgfsetstrokecolor{currentstroke}%
\pgfsetstrokeopacity{0.800000}%
\pgfsetdash{}{0pt}%
\pgfpathmoveto{\pgfqpoint{11.845972in}{4.309917in}}%
\pgfpathcurveto{\pgfqpoint{11.857023in}{4.309917in}}{\pgfqpoint{11.867622in}{4.314308in}}{\pgfqpoint{11.875435in}{4.322121in}}%
\pgfpathcurveto{\pgfqpoint{11.883249in}{4.329935in}}{\pgfqpoint{11.887639in}{4.340534in}}{\pgfqpoint{11.887639in}{4.351584in}}%
\pgfpathcurveto{\pgfqpoint{11.887639in}{4.362634in}}{\pgfqpoint{11.883249in}{4.373233in}}{\pgfqpoint{11.875435in}{4.381047in}}%
\pgfpathcurveto{\pgfqpoint{11.867622in}{4.388860in}}{\pgfqpoint{11.857023in}{4.393251in}}{\pgfqpoint{11.845972in}{4.393251in}}%
\pgfpathcurveto{\pgfqpoint{11.834922in}{4.393251in}}{\pgfqpoint{11.824323in}{4.388860in}}{\pgfqpoint{11.816510in}{4.381047in}}%
\pgfpathcurveto{\pgfqpoint{11.808696in}{4.373233in}}{\pgfqpoint{11.804306in}{4.362634in}}{\pgfqpoint{11.804306in}{4.351584in}}%
\pgfpathcurveto{\pgfqpoint{11.804306in}{4.340534in}}{\pgfqpoint{11.808696in}{4.329935in}}{\pgfqpoint{11.816510in}{4.322121in}}%
\pgfpathcurveto{\pgfqpoint{11.824323in}{4.314308in}}{\pgfqpoint{11.834922in}{4.309917in}}{\pgfqpoint{11.845972in}{4.309917in}}%
\pgfpathlineto{\pgfqpoint{11.845972in}{4.309917in}}%
\pgfpathclose%
\pgfusepath{stroke}%
\end{pgfscope}%
\begin{pgfscope}%
\pgfpathrectangle{\pgfqpoint{7.512535in}{0.437222in}}{\pgfqpoint{6.275590in}{5.159444in}}%
\pgfusepath{clip}%
\pgfsetbuttcap%
\pgfsetroundjoin%
\pgfsetlinewidth{1.003750pt}%
\definecolor{currentstroke}{rgb}{0.827451,0.827451,0.827451}%
\pgfsetstrokecolor{currentstroke}%
\pgfsetstrokeopacity{0.800000}%
\pgfsetdash{}{0pt}%
\pgfpathmoveto{\pgfqpoint{8.214738in}{3.528338in}}%
\pgfpathcurveto{\pgfqpoint{8.225788in}{3.528338in}}{\pgfqpoint{8.236387in}{3.532728in}}{\pgfqpoint{8.244200in}{3.540542in}}%
\pgfpathcurveto{\pgfqpoint{8.252014in}{3.548355in}}{\pgfqpoint{8.256404in}{3.558954in}}{\pgfqpoint{8.256404in}{3.570004in}}%
\pgfpathcurveto{\pgfqpoint{8.256404in}{3.581055in}}{\pgfqpoint{8.252014in}{3.591654in}}{\pgfqpoint{8.244200in}{3.599467in}}%
\pgfpathcurveto{\pgfqpoint{8.236387in}{3.607281in}}{\pgfqpoint{8.225788in}{3.611671in}}{\pgfqpoint{8.214738in}{3.611671in}}%
\pgfpathcurveto{\pgfqpoint{8.203687in}{3.611671in}}{\pgfqpoint{8.193088in}{3.607281in}}{\pgfqpoint{8.185275in}{3.599467in}}%
\pgfpathcurveto{\pgfqpoint{8.177461in}{3.591654in}}{\pgfqpoint{8.173071in}{3.581055in}}{\pgfqpoint{8.173071in}{3.570004in}}%
\pgfpathcurveto{\pgfqpoint{8.173071in}{3.558954in}}{\pgfqpoint{8.177461in}{3.548355in}}{\pgfqpoint{8.185275in}{3.540542in}}%
\pgfpathcurveto{\pgfqpoint{8.193088in}{3.532728in}}{\pgfqpoint{8.203687in}{3.528338in}}{\pgfqpoint{8.214738in}{3.528338in}}%
\pgfpathlineto{\pgfqpoint{8.214738in}{3.528338in}}%
\pgfpathclose%
\pgfusepath{stroke}%
\end{pgfscope}%
\begin{pgfscope}%
\pgfpathrectangle{\pgfqpoint{7.512535in}{0.437222in}}{\pgfqpoint{6.275590in}{5.159444in}}%
\pgfusepath{clip}%
\pgfsetbuttcap%
\pgfsetroundjoin%
\pgfsetlinewidth{1.003750pt}%
\definecolor{currentstroke}{rgb}{0.827451,0.827451,0.827451}%
\pgfsetstrokecolor{currentstroke}%
\pgfsetstrokeopacity{0.800000}%
\pgfsetdash{}{0pt}%
\pgfpathmoveto{\pgfqpoint{11.388601in}{4.546614in}}%
\pgfpathcurveto{\pgfqpoint{11.399651in}{4.546614in}}{\pgfqpoint{11.410250in}{4.551005in}}{\pgfqpoint{11.418064in}{4.558818in}}%
\pgfpathcurveto{\pgfqpoint{11.425878in}{4.566632in}}{\pgfqpoint{11.430268in}{4.577231in}}{\pgfqpoint{11.430268in}{4.588281in}}%
\pgfpathcurveto{\pgfqpoint{11.430268in}{4.599331in}}{\pgfqpoint{11.425878in}{4.609930in}}{\pgfqpoint{11.418064in}{4.617744in}}%
\pgfpathcurveto{\pgfqpoint{11.410250in}{4.625558in}}{\pgfqpoint{11.399651in}{4.629948in}}{\pgfqpoint{11.388601in}{4.629948in}}%
\pgfpathcurveto{\pgfqpoint{11.377551in}{4.629948in}}{\pgfqpoint{11.366952in}{4.625558in}}{\pgfqpoint{11.359139in}{4.617744in}}%
\pgfpathcurveto{\pgfqpoint{11.351325in}{4.609930in}}{\pgfqpoint{11.346935in}{4.599331in}}{\pgfqpoint{11.346935in}{4.588281in}}%
\pgfpathcurveto{\pgfqpoint{11.346935in}{4.577231in}}{\pgfqpoint{11.351325in}{4.566632in}}{\pgfqpoint{11.359139in}{4.558818in}}%
\pgfpathcurveto{\pgfqpoint{11.366952in}{4.551005in}}{\pgfqpoint{11.377551in}{4.546614in}}{\pgfqpoint{11.388601in}{4.546614in}}%
\pgfpathlineto{\pgfqpoint{11.388601in}{4.546614in}}%
\pgfpathclose%
\pgfusepath{stroke}%
\end{pgfscope}%
\begin{pgfscope}%
\pgfpathrectangle{\pgfqpoint{7.512535in}{0.437222in}}{\pgfqpoint{6.275590in}{5.159444in}}%
\pgfusepath{clip}%
\pgfsetbuttcap%
\pgfsetroundjoin%
\pgfsetlinewidth{1.003750pt}%
\definecolor{currentstroke}{rgb}{0.827451,0.827451,0.827451}%
\pgfsetstrokecolor{currentstroke}%
\pgfsetstrokeopacity{0.800000}%
\pgfsetdash{}{0pt}%
\pgfpathmoveto{\pgfqpoint{12.383527in}{4.432707in}}%
\pgfpathcurveto{\pgfqpoint{12.394577in}{4.432707in}}{\pgfqpoint{12.405176in}{4.437098in}}{\pgfqpoint{12.412990in}{4.444911in}}%
\pgfpathcurveto{\pgfqpoint{12.420803in}{4.452725in}}{\pgfqpoint{12.425194in}{4.463324in}}{\pgfqpoint{12.425194in}{4.474374in}}%
\pgfpathcurveto{\pgfqpoint{12.425194in}{4.485424in}}{\pgfqpoint{12.420803in}{4.496023in}}{\pgfqpoint{12.412990in}{4.503837in}}%
\pgfpathcurveto{\pgfqpoint{12.405176in}{4.511650in}}{\pgfqpoint{12.394577in}{4.516041in}}{\pgfqpoint{12.383527in}{4.516041in}}%
\pgfpathcurveto{\pgfqpoint{12.372477in}{4.516041in}}{\pgfqpoint{12.361878in}{4.511650in}}{\pgfqpoint{12.354064in}{4.503837in}}%
\pgfpathcurveto{\pgfqpoint{12.346251in}{4.496023in}}{\pgfqpoint{12.341860in}{4.485424in}}{\pgfqpoint{12.341860in}{4.474374in}}%
\pgfpathcurveto{\pgfqpoint{12.341860in}{4.463324in}}{\pgfqpoint{12.346251in}{4.452725in}}{\pgfqpoint{12.354064in}{4.444911in}}%
\pgfpathcurveto{\pgfqpoint{12.361878in}{4.437098in}}{\pgfqpoint{12.372477in}{4.432707in}}{\pgfqpoint{12.383527in}{4.432707in}}%
\pgfpathlineto{\pgfqpoint{12.383527in}{4.432707in}}%
\pgfpathclose%
\pgfusepath{stroke}%
\end{pgfscope}%
\begin{pgfscope}%
\pgfpathrectangle{\pgfqpoint{7.512535in}{0.437222in}}{\pgfqpoint{6.275590in}{5.159444in}}%
\pgfusepath{clip}%
\pgfsetbuttcap%
\pgfsetroundjoin%
\pgfsetlinewidth{1.003750pt}%
\definecolor{currentstroke}{rgb}{0.827451,0.827451,0.827451}%
\pgfsetstrokecolor{currentstroke}%
\pgfsetstrokeopacity{0.800000}%
\pgfsetdash{}{0pt}%
\pgfpathmoveto{\pgfqpoint{12.129315in}{4.971231in}}%
\pgfpathcurveto{\pgfqpoint{12.140365in}{4.971231in}}{\pgfqpoint{12.150964in}{4.975621in}}{\pgfqpoint{12.158778in}{4.983435in}}%
\pgfpathcurveto{\pgfqpoint{12.166591in}{4.991248in}}{\pgfqpoint{12.170981in}{5.001847in}}{\pgfqpoint{12.170981in}{5.012897in}}%
\pgfpathcurveto{\pgfqpoint{12.170981in}{5.023948in}}{\pgfqpoint{12.166591in}{5.034547in}}{\pgfqpoint{12.158778in}{5.042360in}}%
\pgfpathcurveto{\pgfqpoint{12.150964in}{5.050174in}}{\pgfqpoint{12.140365in}{5.054564in}}{\pgfqpoint{12.129315in}{5.054564in}}%
\pgfpathcurveto{\pgfqpoint{12.118265in}{5.054564in}}{\pgfqpoint{12.107666in}{5.050174in}}{\pgfqpoint{12.099852in}{5.042360in}}%
\pgfpathcurveto{\pgfqpoint{12.092038in}{5.034547in}}{\pgfqpoint{12.087648in}{5.023948in}}{\pgfqpoint{12.087648in}{5.012897in}}%
\pgfpathcurveto{\pgfqpoint{12.087648in}{5.001847in}}{\pgfqpoint{12.092038in}{4.991248in}}{\pgfqpoint{12.099852in}{4.983435in}}%
\pgfpathcurveto{\pgfqpoint{12.107666in}{4.975621in}}{\pgfqpoint{12.118265in}{4.971231in}}{\pgfqpoint{12.129315in}{4.971231in}}%
\pgfpathlineto{\pgfqpoint{12.129315in}{4.971231in}}%
\pgfpathclose%
\pgfusepath{stroke}%
\end{pgfscope}%
\begin{pgfscope}%
\pgfpathrectangle{\pgfqpoint{7.512535in}{0.437222in}}{\pgfqpoint{6.275590in}{5.159444in}}%
\pgfusepath{clip}%
\pgfsetbuttcap%
\pgfsetroundjoin%
\pgfsetlinewidth{1.003750pt}%
\definecolor{currentstroke}{rgb}{0.827451,0.827451,0.827451}%
\pgfsetstrokecolor{currentstroke}%
\pgfsetstrokeopacity{0.800000}%
\pgfsetdash{}{0pt}%
\pgfpathmoveto{\pgfqpoint{9.758049in}{2.980412in}}%
\pgfpathcurveto{\pgfqpoint{9.769099in}{2.980412in}}{\pgfqpoint{9.779698in}{2.984802in}}{\pgfqpoint{9.787512in}{2.992616in}}%
\pgfpathcurveto{\pgfqpoint{9.795326in}{3.000430in}}{\pgfqpoint{9.799716in}{3.011029in}}{\pgfqpoint{9.799716in}{3.022079in}}%
\pgfpathcurveto{\pgfqpoint{9.799716in}{3.033129in}}{\pgfqpoint{9.795326in}{3.043728in}}{\pgfqpoint{9.787512in}{3.051542in}}%
\pgfpathcurveto{\pgfqpoint{9.779698in}{3.059355in}}{\pgfqpoint{9.769099in}{3.063745in}}{\pgfqpoint{9.758049in}{3.063745in}}%
\pgfpathcurveto{\pgfqpoint{9.746999in}{3.063745in}}{\pgfqpoint{9.736400in}{3.059355in}}{\pgfqpoint{9.728587in}{3.051542in}}%
\pgfpathcurveto{\pgfqpoint{9.720773in}{3.043728in}}{\pgfqpoint{9.716383in}{3.033129in}}{\pgfqpoint{9.716383in}{3.022079in}}%
\pgfpathcurveto{\pgfqpoint{9.716383in}{3.011029in}}{\pgfqpoint{9.720773in}{3.000430in}}{\pgfqpoint{9.728587in}{2.992616in}}%
\pgfpathcurveto{\pgfqpoint{9.736400in}{2.984802in}}{\pgfqpoint{9.746999in}{2.980412in}}{\pgfqpoint{9.758049in}{2.980412in}}%
\pgfpathlineto{\pgfqpoint{9.758049in}{2.980412in}}%
\pgfpathclose%
\pgfusepath{stroke}%
\end{pgfscope}%
\begin{pgfscope}%
\pgfpathrectangle{\pgfqpoint{7.512535in}{0.437222in}}{\pgfqpoint{6.275590in}{5.159444in}}%
\pgfusepath{clip}%
\pgfsetbuttcap%
\pgfsetroundjoin%
\pgfsetlinewidth{1.003750pt}%
\definecolor{currentstroke}{rgb}{0.827451,0.827451,0.827451}%
\pgfsetstrokecolor{currentstroke}%
\pgfsetstrokeopacity{0.800000}%
\pgfsetdash{}{0pt}%
\pgfpathmoveto{\pgfqpoint{9.302653in}{3.619779in}}%
\pgfpathcurveto{\pgfqpoint{9.313703in}{3.619779in}}{\pgfqpoint{9.324302in}{3.624169in}}{\pgfqpoint{9.332116in}{3.631982in}}%
\pgfpathcurveto{\pgfqpoint{9.339929in}{3.639796in}}{\pgfqpoint{9.344320in}{3.650395in}}{\pgfqpoint{9.344320in}{3.661445in}}%
\pgfpathcurveto{\pgfqpoint{9.344320in}{3.672495in}}{\pgfqpoint{9.339929in}{3.683094in}}{\pgfqpoint{9.332116in}{3.690908in}}%
\pgfpathcurveto{\pgfqpoint{9.324302in}{3.698722in}}{\pgfqpoint{9.313703in}{3.703112in}}{\pgfqpoint{9.302653in}{3.703112in}}%
\pgfpathcurveto{\pgfqpoint{9.291603in}{3.703112in}}{\pgfqpoint{9.281004in}{3.698722in}}{\pgfqpoint{9.273190in}{3.690908in}}%
\pgfpathcurveto{\pgfqpoint{9.265376in}{3.683094in}}{\pgfqpoint{9.260986in}{3.672495in}}{\pgfqpoint{9.260986in}{3.661445in}}%
\pgfpathcurveto{\pgfqpoint{9.260986in}{3.650395in}}{\pgfqpoint{9.265376in}{3.639796in}}{\pgfqpoint{9.273190in}{3.631982in}}%
\pgfpathcurveto{\pgfqpoint{9.281004in}{3.624169in}}{\pgfqpoint{9.291603in}{3.619779in}}{\pgfqpoint{9.302653in}{3.619779in}}%
\pgfpathlineto{\pgfqpoint{9.302653in}{3.619779in}}%
\pgfpathclose%
\pgfusepath{stroke}%
\end{pgfscope}%
\begin{pgfscope}%
\pgfpathrectangle{\pgfqpoint{7.512535in}{0.437222in}}{\pgfqpoint{6.275590in}{5.159444in}}%
\pgfusepath{clip}%
\pgfsetbuttcap%
\pgfsetroundjoin%
\pgfsetlinewidth{1.003750pt}%
\definecolor{currentstroke}{rgb}{0.827451,0.827451,0.827451}%
\pgfsetstrokecolor{currentstroke}%
\pgfsetstrokeopacity{0.800000}%
\pgfsetdash{}{0pt}%
\pgfpathmoveto{\pgfqpoint{12.362478in}{5.037063in}}%
\pgfpathcurveto{\pgfqpoint{12.373528in}{5.037063in}}{\pgfqpoint{12.384127in}{5.041453in}}{\pgfqpoint{12.391941in}{5.049267in}}%
\pgfpathcurveto{\pgfqpoint{12.399754in}{5.057080in}}{\pgfqpoint{12.404145in}{5.067679in}}{\pgfqpoint{12.404145in}{5.078729in}}%
\pgfpathcurveto{\pgfqpoint{12.404145in}{5.089779in}}{\pgfqpoint{12.399754in}{5.100378in}}{\pgfqpoint{12.391941in}{5.108192in}}%
\pgfpathcurveto{\pgfqpoint{12.384127in}{5.116006in}}{\pgfqpoint{12.373528in}{5.120396in}}{\pgfqpoint{12.362478in}{5.120396in}}%
\pgfpathcurveto{\pgfqpoint{12.351428in}{5.120396in}}{\pgfqpoint{12.340829in}{5.116006in}}{\pgfqpoint{12.333015in}{5.108192in}}%
\pgfpathcurveto{\pgfqpoint{12.325202in}{5.100378in}}{\pgfqpoint{12.320811in}{5.089779in}}{\pgfqpoint{12.320811in}{5.078729in}}%
\pgfpathcurveto{\pgfqpoint{12.320811in}{5.067679in}}{\pgfqpoint{12.325202in}{5.057080in}}{\pgfqpoint{12.333015in}{5.049267in}}%
\pgfpathcurveto{\pgfqpoint{12.340829in}{5.041453in}}{\pgfqpoint{12.351428in}{5.037063in}}{\pgfqpoint{12.362478in}{5.037063in}}%
\pgfpathlineto{\pgfqpoint{12.362478in}{5.037063in}}%
\pgfpathclose%
\pgfusepath{stroke}%
\end{pgfscope}%
\begin{pgfscope}%
\pgfpathrectangle{\pgfqpoint{7.512535in}{0.437222in}}{\pgfqpoint{6.275590in}{5.159444in}}%
\pgfusepath{clip}%
\pgfsetbuttcap%
\pgfsetroundjoin%
\pgfsetlinewidth{1.003750pt}%
\definecolor{currentstroke}{rgb}{0.827451,0.827451,0.827451}%
\pgfsetstrokecolor{currentstroke}%
\pgfsetstrokeopacity{0.800000}%
\pgfsetdash{}{0pt}%
\pgfpathmoveto{\pgfqpoint{9.586356in}{0.724944in}}%
\pgfpathcurveto{\pgfqpoint{9.597406in}{0.724944in}}{\pgfqpoint{9.608005in}{0.729334in}}{\pgfqpoint{9.615819in}{0.737148in}}%
\pgfpathcurveto{\pgfqpoint{9.623632in}{0.744962in}}{\pgfqpoint{9.628022in}{0.755561in}}{\pgfqpoint{9.628022in}{0.766611in}}%
\pgfpathcurveto{\pgfqpoint{9.628022in}{0.777661in}}{\pgfqpoint{9.623632in}{0.788260in}}{\pgfqpoint{9.615819in}{0.796074in}}%
\pgfpathcurveto{\pgfqpoint{9.608005in}{0.803887in}}{\pgfqpoint{9.597406in}{0.808277in}}{\pgfqpoint{9.586356in}{0.808277in}}%
\pgfpathcurveto{\pgfqpoint{9.575306in}{0.808277in}}{\pgfqpoint{9.564707in}{0.803887in}}{\pgfqpoint{9.556893in}{0.796074in}}%
\pgfpathcurveto{\pgfqpoint{9.549079in}{0.788260in}}{\pgfqpoint{9.544689in}{0.777661in}}{\pgfqpoint{9.544689in}{0.766611in}}%
\pgfpathcurveto{\pgfqpoint{9.544689in}{0.755561in}}{\pgfqpoint{9.549079in}{0.744962in}}{\pgfqpoint{9.556893in}{0.737148in}}%
\pgfpathcurveto{\pgfqpoint{9.564707in}{0.729334in}}{\pgfqpoint{9.575306in}{0.724944in}}{\pgfqpoint{9.586356in}{0.724944in}}%
\pgfpathlineto{\pgfqpoint{9.586356in}{0.724944in}}%
\pgfpathclose%
\pgfusepath{stroke}%
\end{pgfscope}%
\begin{pgfscope}%
\pgfpathrectangle{\pgfqpoint{7.512535in}{0.437222in}}{\pgfqpoint{6.275590in}{5.159444in}}%
\pgfusepath{clip}%
\pgfsetbuttcap%
\pgfsetroundjoin%
\pgfsetlinewidth{1.003750pt}%
\definecolor{currentstroke}{rgb}{0.827451,0.827451,0.827451}%
\pgfsetstrokecolor{currentstroke}%
\pgfsetstrokeopacity{0.800000}%
\pgfsetdash{}{0pt}%
\pgfpathmoveto{\pgfqpoint{13.786989in}{5.121713in}}%
\pgfpathcurveto{\pgfqpoint{13.798039in}{5.121713in}}{\pgfqpoint{13.808638in}{5.126103in}}{\pgfqpoint{13.816452in}{5.133917in}}%
\pgfpathcurveto{\pgfqpoint{13.824265in}{5.141730in}}{\pgfqpoint{13.828656in}{5.152329in}}{\pgfqpoint{13.828656in}{5.163379in}}%
\pgfpathcurveto{\pgfqpoint{13.828656in}{5.174429in}}{\pgfqpoint{13.824265in}{5.185028in}}{\pgfqpoint{13.816452in}{5.192842in}}%
\pgfpathcurveto{\pgfqpoint{13.808638in}{5.200656in}}{\pgfqpoint{13.798039in}{5.205046in}}{\pgfqpoint{13.786989in}{5.205046in}}%
\pgfpathcurveto{\pgfqpoint{13.775939in}{5.205046in}}{\pgfqpoint{13.765340in}{5.200656in}}{\pgfqpoint{13.757526in}{5.192842in}}%
\pgfpathcurveto{\pgfqpoint{13.749713in}{5.185028in}}{\pgfqpoint{13.745322in}{5.174429in}}{\pgfqpoint{13.745322in}{5.163379in}}%
\pgfpathcurveto{\pgfqpoint{13.745322in}{5.152329in}}{\pgfqpoint{13.749713in}{5.141730in}}{\pgfqpoint{13.757526in}{5.133917in}}%
\pgfpathcurveto{\pgfqpoint{13.765340in}{5.126103in}}{\pgfqpoint{13.775939in}{5.121713in}}{\pgfqpoint{13.786989in}{5.121713in}}%
\pgfpathlineto{\pgfqpoint{13.786989in}{5.121713in}}%
\pgfpathclose%
\pgfusepath{stroke}%
\end{pgfscope}%
\begin{pgfscope}%
\pgfpathrectangle{\pgfqpoint{7.512535in}{0.437222in}}{\pgfqpoint{6.275590in}{5.159444in}}%
\pgfusepath{clip}%
\pgfsetbuttcap%
\pgfsetroundjoin%
\pgfsetlinewidth{1.003750pt}%
\definecolor{currentstroke}{rgb}{0.827451,0.827451,0.827451}%
\pgfsetstrokecolor{currentstroke}%
\pgfsetstrokeopacity{0.800000}%
\pgfsetdash{}{0pt}%
\pgfpathmoveto{\pgfqpoint{9.116910in}{3.950824in}}%
\pgfpathcurveto{\pgfqpoint{9.127960in}{3.950824in}}{\pgfqpoint{9.138559in}{3.955214in}}{\pgfqpoint{9.146373in}{3.963027in}}%
\pgfpathcurveto{\pgfqpoint{9.154186in}{3.970841in}}{\pgfqpoint{9.158577in}{3.981440in}}{\pgfqpoint{9.158577in}{3.992490in}}%
\pgfpathcurveto{\pgfqpoint{9.158577in}{4.003540in}}{\pgfqpoint{9.154186in}{4.014139in}}{\pgfqpoint{9.146373in}{4.021953in}}%
\pgfpathcurveto{\pgfqpoint{9.138559in}{4.029767in}}{\pgfqpoint{9.127960in}{4.034157in}}{\pgfqpoint{9.116910in}{4.034157in}}%
\pgfpathcurveto{\pgfqpoint{9.105860in}{4.034157in}}{\pgfqpoint{9.095261in}{4.029767in}}{\pgfqpoint{9.087447in}{4.021953in}}%
\pgfpathcurveto{\pgfqpoint{9.079634in}{4.014139in}}{\pgfqpoint{9.075243in}{4.003540in}}{\pgfqpoint{9.075243in}{3.992490in}}%
\pgfpathcurveto{\pgfqpoint{9.075243in}{3.981440in}}{\pgfqpoint{9.079634in}{3.970841in}}{\pgfqpoint{9.087447in}{3.963027in}}%
\pgfpathcurveto{\pgfqpoint{9.095261in}{3.955214in}}{\pgfqpoint{9.105860in}{3.950824in}}{\pgfqpoint{9.116910in}{3.950824in}}%
\pgfpathlineto{\pgfqpoint{9.116910in}{3.950824in}}%
\pgfpathclose%
\pgfusepath{stroke}%
\end{pgfscope}%
\begin{pgfscope}%
\pgfpathrectangle{\pgfqpoint{7.512535in}{0.437222in}}{\pgfqpoint{6.275590in}{5.159444in}}%
\pgfusepath{clip}%
\pgfsetbuttcap%
\pgfsetroundjoin%
\pgfsetlinewidth{1.003750pt}%
\definecolor{currentstroke}{rgb}{0.827451,0.827451,0.827451}%
\pgfsetstrokecolor{currentstroke}%
\pgfsetstrokeopacity{0.800000}%
\pgfsetdash{}{0pt}%
\pgfpathmoveto{\pgfqpoint{10.291211in}{3.552829in}}%
\pgfpathcurveto{\pgfqpoint{10.302261in}{3.552829in}}{\pgfqpoint{10.312860in}{3.557219in}}{\pgfqpoint{10.320674in}{3.565033in}}%
\pgfpathcurveto{\pgfqpoint{10.328488in}{3.572846in}}{\pgfqpoint{10.332878in}{3.583446in}}{\pgfqpoint{10.332878in}{3.594496in}}%
\pgfpathcurveto{\pgfqpoint{10.332878in}{3.605546in}}{\pgfqpoint{10.328488in}{3.616145in}}{\pgfqpoint{10.320674in}{3.623958in}}%
\pgfpathcurveto{\pgfqpoint{10.312860in}{3.631772in}}{\pgfqpoint{10.302261in}{3.636162in}}{\pgfqpoint{10.291211in}{3.636162in}}%
\pgfpathcurveto{\pgfqpoint{10.280161in}{3.636162in}}{\pgfqpoint{10.269562in}{3.631772in}}{\pgfqpoint{10.261748in}{3.623958in}}%
\pgfpathcurveto{\pgfqpoint{10.253935in}{3.616145in}}{\pgfqpoint{10.249545in}{3.605546in}}{\pgfqpoint{10.249545in}{3.594496in}}%
\pgfpathcurveto{\pgfqpoint{10.249545in}{3.583446in}}{\pgfqpoint{10.253935in}{3.572846in}}{\pgfqpoint{10.261748in}{3.565033in}}%
\pgfpathcurveto{\pgfqpoint{10.269562in}{3.557219in}}{\pgfqpoint{10.280161in}{3.552829in}}{\pgfqpoint{10.291211in}{3.552829in}}%
\pgfpathlineto{\pgfqpoint{10.291211in}{3.552829in}}%
\pgfpathclose%
\pgfusepath{stroke}%
\end{pgfscope}%
\begin{pgfscope}%
\pgfpathrectangle{\pgfqpoint{7.512535in}{0.437222in}}{\pgfqpoint{6.275590in}{5.159444in}}%
\pgfusepath{clip}%
\pgfsetbuttcap%
\pgfsetroundjoin%
\pgfsetlinewidth{1.003750pt}%
\definecolor{currentstroke}{rgb}{0.827451,0.827451,0.827451}%
\pgfsetstrokecolor{currentstroke}%
\pgfsetstrokeopacity{0.800000}%
\pgfsetdash{}{0pt}%
\pgfpathmoveto{\pgfqpoint{7.985597in}{2.298483in}}%
\pgfpathcurveto{\pgfqpoint{7.996647in}{2.298483in}}{\pgfqpoint{8.007246in}{2.302873in}}{\pgfqpoint{8.015059in}{2.310686in}}%
\pgfpathcurveto{\pgfqpoint{8.022873in}{2.318500in}}{\pgfqpoint{8.027263in}{2.329099in}}{\pgfqpoint{8.027263in}{2.340149in}}%
\pgfpathcurveto{\pgfqpoint{8.027263in}{2.351199in}}{\pgfqpoint{8.022873in}{2.361798in}}{\pgfqpoint{8.015059in}{2.369612in}}%
\pgfpathcurveto{\pgfqpoint{8.007246in}{2.377426in}}{\pgfqpoint{7.996647in}{2.381816in}}{\pgfqpoint{7.985597in}{2.381816in}}%
\pgfpathcurveto{\pgfqpoint{7.974546in}{2.381816in}}{\pgfqpoint{7.963947in}{2.377426in}}{\pgfqpoint{7.956134in}{2.369612in}}%
\pgfpathcurveto{\pgfqpoint{7.948320in}{2.361798in}}{\pgfqpoint{7.943930in}{2.351199in}}{\pgfqpoint{7.943930in}{2.340149in}}%
\pgfpathcurveto{\pgfqpoint{7.943930in}{2.329099in}}{\pgfqpoint{7.948320in}{2.318500in}}{\pgfqpoint{7.956134in}{2.310686in}}%
\pgfpathcurveto{\pgfqpoint{7.963947in}{2.302873in}}{\pgfqpoint{7.974546in}{2.298483in}}{\pgfqpoint{7.985597in}{2.298483in}}%
\pgfpathlineto{\pgfqpoint{7.985597in}{2.298483in}}%
\pgfpathclose%
\pgfusepath{stroke}%
\end{pgfscope}%
\begin{pgfscope}%
\pgfpathrectangle{\pgfqpoint{7.512535in}{0.437222in}}{\pgfqpoint{6.275590in}{5.159444in}}%
\pgfusepath{clip}%
\pgfsetbuttcap%
\pgfsetroundjoin%
\pgfsetlinewidth{1.003750pt}%
\definecolor{currentstroke}{rgb}{0.827451,0.827451,0.827451}%
\pgfsetstrokecolor{currentstroke}%
\pgfsetstrokeopacity{0.800000}%
\pgfsetdash{}{0pt}%
\pgfpathmoveto{\pgfqpoint{8.778671in}{3.998377in}}%
\pgfpathcurveto{\pgfqpoint{8.789721in}{3.998377in}}{\pgfqpoint{8.800320in}{4.002767in}}{\pgfqpoint{8.808134in}{4.010581in}}%
\pgfpathcurveto{\pgfqpoint{8.815947in}{4.018394in}}{\pgfqpoint{8.820337in}{4.028993in}}{\pgfqpoint{8.820337in}{4.040044in}}%
\pgfpathcurveto{\pgfqpoint{8.820337in}{4.051094in}}{\pgfqpoint{8.815947in}{4.061693in}}{\pgfqpoint{8.808134in}{4.069506in}}%
\pgfpathcurveto{\pgfqpoint{8.800320in}{4.077320in}}{\pgfqpoint{8.789721in}{4.081710in}}{\pgfqpoint{8.778671in}{4.081710in}}%
\pgfpathcurveto{\pgfqpoint{8.767621in}{4.081710in}}{\pgfqpoint{8.757022in}{4.077320in}}{\pgfqpoint{8.749208in}{4.069506in}}%
\pgfpathcurveto{\pgfqpoint{8.741394in}{4.061693in}}{\pgfqpoint{8.737004in}{4.051094in}}{\pgfqpoint{8.737004in}{4.040044in}}%
\pgfpathcurveto{\pgfqpoint{8.737004in}{4.028993in}}{\pgfqpoint{8.741394in}{4.018394in}}{\pgfqpoint{8.749208in}{4.010581in}}%
\pgfpathcurveto{\pgfqpoint{8.757022in}{4.002767in}}{\pgfqpoint{8.767621in}{3.998377in}}{\pgfqpoint{8.778671in}{3.998377in}}%
\pgfpathlineto{\pgfqpoint{8.778671in}{3.998377in}}%
\pgfpathclose%
\pgfusepath{stroke}%
\end{pgfscope}%
\begin{pgfscope}%
\pgfpathrectangle{\pgfqpoint{7.512535in}{0.437222in}}{\pgfqpoint{6.275590in}{5.159444in}}%
\pgfusepath{clip}%
\pgfsetbuttcap%
\pgfsetroundjoin%
\pgfsetlinewidth{1.003750pt}%
\definecolor{currentstroke}{rgb}{0.827451,0.827451,0.827451}%
\pgfsetstrokecolor{currentstroke}%
\pgfsetstrokeopacity{0.800000}%
\pgfsetdash{}{0pt}%
\pgfpathmoveto{\pgfqpoint{8.767644in}{3.960366in}}%
\pgfpathcurveto{\pgfqpoint{8.778694in}{3.960366in}}{\pgfqpoint{8.789293in}{3.964756in}}{\pgfqpoint{8.797107in}{3.972570in}}%
\pgfpathcurveto{\pgfqpoint{8.804920in}{3.980383in}}{\pgfqpoint{8.809311in}{3.990982in}}{\pgfqpoint{8.809311in}{4.002032in}}%
\pgfpathcurveto{\pgfqpoint{8.809311in}{4.013082in}}{\pgfqpoint{8.804920in}{4.023681in}}{\pgfqpoint{8.797107in}{4.031495in}}%
\pgfpathcurveto{\pgfqpoint{8.789293in}{4.039309in}}{\pgfqpoint{8.778694in}{4.043699in}}{\pgfqpoint{8.767644in}{4.043699in}}%
\pgfpathcurveto{\pgfqpoint{8.756594in}{4.043699in}}{\pgfqpoint{8.745995in}{4.039309in}}{\pgfqpoint{8.738181in}{4.031495in}}%
\pgfpathcurveto{\pgfqpoint{8.730367in}{4.023681in}}{\pgfqpoint{8.725977in}{4.013082in}}{\pgfqpoint{8.725977in}{4.002032in}}%
\pgfpathcurveto{\pgfqpoint{8.725977in}{3.990982in}}{\pgfqpoint{8.730367in}{3.980383in}}{\pgfqpoint{8.738181in}{3.972570in}}%
\pgfpathcurveto{\pgfqpoint{8.745995in}{3.964756in}}{\pgfqpoint{8.756594in}{3.960366in}}{\pgfqpoint{8.767644in}{3.960366in}}%
\pgfpathlineto{\pgfqpoint{8.767644in}{3.960366in}}%
\pgfpathclose%
\pgfusepath{stroke}%
\end{pgfscope}%
\begin{pgfscope}%
\pgfpathrectangle{\pgfqpoint{7.512535in}{0.437222in}}{\pgfqpoint{6.275590in}{5.159444in}}%
\pgfusepath{clip}%
\pgfsetbuttcap%
\pgfsetroundjoin%
\pgfsetlinewidth{1.003750pt}%
\definecolor{currentstroke}{rgb}{0.827451,0.827451,0.827451}%
\pgfsetstrokecolor{currentstroke}%
\pgfsetstrokeopacity{0.800000}%
\pgfsetdash{}{0pt}%
\pgfpathmoveto{\pgfqpoint{11.860040in}{4.671218in}}%
\pgfpathcurveto{\pgfqpoint{11.871090in}{4.671218in}}{\pgfqpoint{11.881689in}{4.675608in}}{\pgfqpoint{11.889502in}{4.683421in}}%
\pgfpathcurveto{\pgfqpoint{11.897316in}{4.691235in}}{\pgfqpoint{11.901706in}{4.701834in}}{\pgfqpoint{11.901706in}{4.712884in}}%
\pgfpathcurveto{\pgfqpoint{11.901706in}{4.723934in}}{\pgfqpoint{11.897316in}{4.734533in}}{\pgfqpoint{11.889502in}{4.742347in}}%
\pgfpathcurveto{\pgfqpoint{11.881689in}{4.750161in}}{\pgfqpoint{11.871090in}{4.754551in}}{\pgfqpoint{11.860040in}{4.754551in}}%
\pgfpathcurveto{\pgfqpoint{11.848989in}{4.754551in}}{\pgfqpoint{11.838390in}{4.750161in}}{\pgfqpoint{11.830577in}{4.742347in}}%
\pgfpathcurveto{\pgfqpoint{11.822763in}{4.734533in}}{\pgfqpoint{11.818373in}{4.723934in}}{\pgfqpoint{11.818373in}{4.712884in}}%
\pgfpathcurveto{\pgfqpoint{11.818373in}{4.701834in}}{\pgfqpoint{11.822763in}{4.691235in}}{\pgfqpoint{11.830577in}{4.683421in}}%
\pgfpathcurveto{\pgfqpoint{11.838390in}{4.675608in}}{\pgfqpoint{11.848989in}{4.671218in}}{\pgfqpoint{11.860040in}{4.671218in}}%
\pgfpathlineto{\pgfqpoint{11.860040in}{4.671218in}}%
\pgfpathclose%
\pgfusepath{stroke}%
\end{pgfscope}%
\begin{pgfscope}%
\pgfpathrectangle{\pgfqpoint{7.512535in}{0.437222in}}{\pgfqpoint{6.275590in}{5.159444in}}%
\pgfusepath{clip}%
\pgfsetbuttcap%
\pgfsetroundjoin%
\pgfsetlinewidth{1.003750pt}%
\definecolor{currentstroke}{rgb}{0.827451,0.827451,0.827451}%
\pgfsetstrokecolor{currentstroke}%
\pgfsetstrokeopacity{0.800000}%
\pgfsetdash{}{0pt}%
\pgfpathmoveto{\pgfqpoint{12.383527in}{4.314119in}}%
\pgfpathcurveto{\pgfqpoint{12.394577in}{4.314119in}}{\pgfqpoint{12.405176in}{4.318509in}}{\pgfqpoint{12.412990in}{4.326323in}}%
\pgfpathcurveto{\pgfqpoint{12.420803in}{4.334136in}}{\pgfqpoint{12.425194in}{4.344735in}}{\pgfqpoint{12.425194in}{4.355786in}}%
\pgfpathcurveto{\pgfqpoint{12.425194in}{4.366836in}}{\pgfqpoint{12.420803in}{4.377435in}}{\pgfqpoint{12.412990in}{4.385248in}}%
\pgfpathcurveto{\pgfqpoint{12.405176in}{4.393062in}}{\pgfqpoint{12.394577in}{4.397452in}}{\pgfqpoint{12.383527in}{4.397452in}}%
\pgfpathcurveto{\pgfqpoint{12.372477in}{4.397452in}}{\pgfqpoint{12.361878in}{4.393062in}}{\pgfqpoint{12.354064in}{4.385248in}}%
\pgfpathcurveto{\pgfqpoint{12.346251in}{4.377435in}}{\pgfqpoint{12.341860in}{4.366836in}}{\pgfqpoint{12.341860in}{4.355786in}}%
\pgfpathcurveto{\pgfqpoint{12.341860in}{4.344735in}}{\pgfqpoint{12.346251in}{4.334136in}}{\pgfqpoint{12.354064in}{4.326323in}}%
\pgfpathcurveto{\pgfqpoint{12.361878in}{4.318509in}}{\pgfqpoint{12.372477in}{4.314119in}}{\pgfqpoint{12.383527in}{4.314119in}}%
\pgfpathlineto{\pgfqpoint{12.383527in}{4.314119in}}%
\pgfpathclose%
\pgfusepath{stroke}%
\end{pgfscope}%
\begin{pgfscope}%
\pgfpathrectangle{\pgfqpoint{7.512535in}{0.437222in}}{\pgfqpoint{6.275590in}{5.159444in}}%
\pgfusepath{clip}%
\pgfsetbuttcap%
\pgfsetroundjoin%
\pgfsetlinewidth{1.003750pt}%
\definecolor{currentstroke}{rgb}{0.827451,0.827451,0.827451}%
\pgfsetstrokecolor{currentstroke}%
\pgfsetstrokeopacity{0.800000}%
\pgfsetdash{}{0pt}%
\pgfpathmoveto{\pgfqpoint{10.849745in}{3.530460in}}%
\pgfpathcurveto{\pgfqpoint{10.860795in}{3.530460in}}{\pgfqpoint{10.871394in}{3.534851in}}{\pgfqpoint{10.879207in}{3.542664in}}%
\pgfpathcurveto{\pgfqpoint{10.887021in}{3.550478in}}{\pgfqpoint{10.891411in}{3.561077in}}{\pgfqpoint{10.891411in}{3.572127in}}%
\pgfpathcurveto{\pgfqpoint{10.891411in}{3.583177in}}{\pgfqpoint{10.887021in}{3.593776in}}{\pgfqpoint{10.879207in}{3.601590in}}%
\pgfpathcurveto{\pgfqpoint{10.871394in}{3.609403in}}{\pgfqpoint{10.860795in}{3.613794in}}{\pgfqpoint{10.849745in}{3.613794in}}%
\pgfpathcurveto{\pgfqpoint{10.838695in}{3.613794in}}{\pgfqpoint{10.828096in}{3.609403in}}{\pgfqpoint{10.820282in}{3.601590in}}%
\pgfpathcurveto{\pgfqpoint{10.812468in}{3.593776in}}{\pgfqpoint{10.808078in}{3.583177in}}{\pgfqpoint{10.808078in}{3.572127in}}%
\pgfpathcurveto{\pgfqpoint{10.808078in}{3.561077in}}{\pgfqpoint{10.812468in}{3.550478in}}{\pgfqpoint{10.820282in}{3.542664in}}%
\pgfpathcurveto{\pgfqpoint{10.828096in}{3.534851in}}{\pgfqpoint{10.838695in}{3.530460in}}{\pgfqpoint{10.849745in}{3.530460in}}%
\pgfpathlineto{\pgfqpoint{10.849745in}{3.530460in}}%
\pgfpathclose%
\pgfusepath{stroke}%
\end{pgfscope}%
\begin{pgfscope}%
\pgfpathrectangle{\pgfqpoint{7.512535in}{0.437222in}}{\pgfqpoint{6.275590in}{5.159444in}}%
\pgfusepath{clip}%
\pgfsetbuttcap%
\pgfsetroundjoin%
\pgfsetlinewidth{1.003750pt}%
\definecolor{currentstroke}{rgb}{0.827451,0.827451,0.827451}%
\pgfsetstrokecolor{currentstroke}%
\pgfsetstrokeopacity{0.800000}%
\pgfsetdash{}{0pt}%
\pgfpathmoveto{\pgfqpoint{9.116910in}{0.954383in}}%
\pgfpathcurveto{\pgfqpoint{9.127960in}{0.954383in}}{\pgfqpoint{9.138559in}{0.958773in}}{\pgfqpoint{9.146373in}{0.966587in}}%
\pgfpathcurveto{\pgfqpoint{9.154186in}{0.974400in}}{\pgfqpoint{9.158577in}{0.984999in}}{\pgfqpoint{9.158577in}{0.996049in}}%
\pgfpathcurveto{\pgfqpoint{9.158577in}{1.007099in}}{\pgfqpoint{9.154186in}{1.017699in}}{\pgfqpoint{9.146373in}{1.025512in}}%
\pgfpathcurveto{\pgfqpoint{9.138559in}{1.033326in}}{\pgfqpoint{9.127960in}{1.037716in}}{\pgfqpoint{9.116910in}{1.037716in}}%
\pgfpathcurveto{\pgfqpoint{9.105860in}{1.037716in}}{\pgfqpoint{9.095261in}{1.033326in}}{\pgfqpoint{9.087447in}{1.025512in}}%
\pgfpathcurveto{\pgfqpoint{9.079634in}{1.017699in}}{\pgfqpoint{9.075243in}{1.007099in}}{\pgfqpoint{9.075243in}{0.996049in}}%
\pgfpathcurveto{\pgfqpoint{9.075243in}{0.984999in}}{\pgfqpoint{9.079634in}{0.974400in}}{\pgfqpoint{9.087447in}{0.966587in}}%
\pgfpathcurveto{\pgfqpoint{9.095261in}{0.958773in}}{\pgfqpoint{9.105860in}{0.954383in}}{\pgfqpoint{9.116910in}{0.954383in}}%
\pgfpathlineto{\pgfqpoint{9.116910in}{0.954383in}}%
\pgfpathclose%
\pgfusepath{stroke}%
\end{pgfscope}%
\begin{pgfscope}%
\pgfpathrectangle{\pgfqpoint{7.512535in}{0.437222in}}{\pgfqpoint{6.275590in}{5.159444in}}%
\pgfusepath{clip}%
\pgfsetbuttcap%
\pgfsetroundjoin%
\pgfsetlinewidth{1.003750pt}%
\definecolor{currentstroke}{rgb}{0.827451,0.827451,0.827451}%
\pgfsetstrokecolor{currentstroke}%
\pgfsetstrokeopacity{0.800000}%
\pgfsetdash{}{0pt}%
\pgfpathmoveto{\pgfqpoint{11.623063in}{2.922700in}}%
\pgfpathcurveto{\pgfqpoint{11.634113in}{2.922700in}}{\pgfqpoint{11.644712in}{2.927091in}}{\pgfqpoint{11.652526in}{2.934904in}}%
\pgfpathcurveto{\pgfqpoint{11.660340in}{2.942718in}}{\pgfqpoint{11.664730in}{2.953317in}}{\pgfqpoint{11.664730in}{2.964367in}}%
\pgfpathcurveto{\pgfqpoint{11.664730in}{2.975417in}}{\pgfqpoint{11.660340in}{2.986016in}}{\pgfqpoint{11.652526in}{2.993830in}}%
\pgfpathcurveto{\pgfqpoint{11.644712in}{3.001643in}}{\pgfqpoint{11.634113in}{3.006034in}}{\pgfqpoint{11.623063in}{3.006034in}}%
\pgfpathcurveto{\pgfqpoint{11.612013in}{3.006034in}}{\pgfqpoint{11.601414in}{3.001643in}}{\pgfqpoint{11.593600in}{2.993830in}}%
\pgfpathcurveto{\pgfqpoint{11.585787in}{2.986016in}}{\pgfqpoint{11.581396in}{2.975417in}}{\pgfqpoint{11.581396in}{2.964367in}}%
\pgfpathcurveto{\pgfqpoint{11.581396in}{2.953317in}}{\pgfqpoint{11.585787in}{2.942718in}}{\pgfqpoint{11.593600in}{2.934904in}}%
\pgfpathcurveto{\pgfqpoint{11.601414in}{2.927091in}}{\pgfqpoint{11.612013in}{2.922700in}}{\pgfqpoint{11.623063in}{2.922700in}}%
\pgfpathlineto{\pgfqpoint{11.623063in}{2.922700in}}%
\pgfpathclose%
\pgfusepath{stroke}%
\end{pgfscope}%
\begin{pgfscope}%
\pgfpathrectangle{\pgfqpoint{7.512535in}{0.437222in}}{\pgfqpoint{6.275590in}{5.159444in}}%
\pgfusepath{clip}%
\pgfsetbuttcap%
\pgfsetroundjoin%
\pgfsetlinewidth{1.003750pt}%
\definecolor{currentstroke}{rgb}{0.827451,0.827451,0.827451}%
\pgfsetstrokecolor{currentstroke}%
\pgfsetstrokeopacity{0.800000}%
\pgfsetdash{}{0pt}%
\pgfpathmoveto{\pgfqpoint{11.678390in}{3.857316in}}%
\pgfpathcurveto{\pgfqpoint{11.689440in}{3.857316in}}{\pgfqpoint{11.700039in}{3.861707in}}{\pgfqpoint{11.707853in}{3.869520in}}%
\pgfpathcurveto{\pgfqpoint{11.715666in}{3.877334in}}{\pgfqpoint{11.720057in}{3.887933in}}{\pgfqpoint{11.720057in}{3.898983in}}%
\pgfpathcurveto{\pgfqpoint{11.720057in}{3.910033in}}{\pgfqpoint{11.715666in}{3.920632in}}{\pgfqpoint{11.707853in}{3.928446in}}%
\pgfpathcurveto{\pgfqpoint{11.700039in}{3.936259in}}{\pgfqpoint{11.689440in}{3.940650in}}{\pgfqpoint{11.678390in}{3.940650in}}%
\pgfpathcurveto{\pgfqpoint{11.667340in}{3.940650in}}{\pgfqpoint{11.656741in}{3.936259in}}{\pgfqpoint{11.648927in}{3.928446in}}%
\pgfpathcurveto{\pgfqpoint{11.641114in}{3.920632in}}{\pgfqpoint{11.636723in}{3.910033in}}{\pgfqpoint{11.636723in}{3.898983in}}%
\pgfpathcurveto{\pgfqpoint{11.636723in}{3.887933in}}{\pgfqpoint{11.641114in}{3.877334in}}{\pgfqpoint{11.648927in}{3.869520in}}%
\pgfpathcurveto{\pgfqpoint{11.656741in}{3.861707in}}{\pgfqpoint{11.667340in}{3.857316in}}{\pgfqpoint{11.678390in}{3.857316in}}%
\pgfpathlineto{\pgfqpoint{11.678390in}{3.857316in}}%
\pgfpathclose%
\pgfusepath{stroke}%
\end{pgfscope}%
\begin{pgfscope}%
\pgfpathrectangle{\pgfqpoint{7.512535in}{0.437222in}}{\pgfqpoint{6.275590in}{5.159444in}}%
\pgfusepath{clip}%
\pgfsetbuttcap%
\pgfsetroundjoin%
\pgfsetlinewidth{1.003750pt}%
\definecolor{currentstroke}{rgb}{0.827451,0.827451,0.827451}%
\pgfsetstrokecolor{currentstroke}%
\pgfsetstrokeopacity{0.800000}%
\pgfsetdash{}{0pt}%
\pgfpathmoveto{\pgfqpoint{9.540145in}{4.891615in}}%
\pgfpathcurveto{\pgfqpoint{9.551195in}{4.891615in}}{\pgfqpoint{9.561794in}{4.896006in}}{\pgfqpoint{9.569607in}{4.903819in}}%
\pgfpathcurveto{\pgfqpoint{9.577421in}{4.911633in}}{\pgfqpoint{9.581811in}{4.922232in}}{\pgfqpoint{9.581811in}{4.933282in}}%
\pgfpathcurveto{\pgfqpoint{9.581811in}{4.944332in}}{\pgfqpoint{9.577421in}{4.954931in}}{\pgfqpoint{9.569607in}{4.962745in}}%
\pgfpathcurveto{\pgfqpoint{9.561794in}{4.970558in}}{\pgfqpoint{9.551195in}{4.974949in}}{\pgfqpoint{9.540145in}{4.974949in}}%
\pgfpathcurveto{\pgfqpoint{9.529095in}{4.974949in}}{\pgfqpoint{9.518495in}{4.970558in}}{\pgfqpoint{9.510682in}{4.962745in}}%
\pgfpathcurveto{\pgfqpoint{9.502868in}{4.954931in}}{\pgfqpoint{9.498478in}{4.944332in}}{\pgfqpoint{9.498478in}{4.933282in}}%
\pgfpathcurveto{\pgfqpoint{9.498478in}{4.922232in}}{\pgfqpoint{9.502868in}{4.911633in}}{\pgfqpoint{9.510682in}{4.903819in}}%
\pgfpathcurveto{\pgfqpoint{9.518495in}{4.896006in}}{\pgfqpoint{9.529095in}{4.891615in}}{\pgfqpoint{9.540145in}{4.891615in}}%
\pgfpathlineto{\pgfqpoint{9.540145in}{4.891615in}}%
\pgfpathclose%
\pgfusepath{stroke}%
\end{pgfscope}%
\begin{pgfscope}%
\pgfpathrectangle{\pgfqpoint{7.512535in}{0.437222in}}{\pgfqpoint{6.275590in}{5.159444in}}%
\pgfusepath{clip}%
\pgfsetbuttcap%
\pgfsetroundjoin%
\pgfsetlinewidth{1.003750pt}%
\definecolor{currentstroke}{rgb}{0.827451,0.827451,0.827451}%
\pgfsetstrokecolor{currentstroke}%
\pgfsetstrokeopacity{0.800000}%
\pgfsetdash{}{0pt}%
\pgfpathmoveto{\pgfqpoint{9.540145in}{4.838784in}}%
\pgfpathcurveto{\pgfqpoint{9.551195in}{4.838784in}}{\pgfqpoint{9.561794in}{4.843174in}}{\pgfqpoint{9.569607in}{4.850988in}}%
\pgfpathcurveto{\pgfqpoint{9.577421in}{4.858801in}}{\pgfqpoint{9.581811in}{4.869400in}}{\pgfqpoint{9.581811in}{4.880450in}}%
\pgfpathcurveto{\pgfqpoint{9.581811in}{4.891501in}}{\pgfqpoint{9.577421in}{4.902100in}}{\pgfqpoint{9.569607in}{4.909913in}}%
\pgfpathcurveto{\pgfqpoint{9.561794in}{4.917727in}}{\pgfqpoint{9.551195in}{4.922117in}}{\pgfqpoint{9.540145in}{4.922117in}}%
\pgfpathcurveto{\pgfqpoint{9.529095in}{4.922117in}}{\pgfqpoint{9.518495in}{4.917727in}}{\pgfqpoint{9.510682in}{4.909913in}}%
\pgfpathcurveto{\pgfqpoint{9.502868in}{4.902100in}}{\pgfqpoint{9.498478in}{4.891501in}}{\pgfqpoint{9.498478in}{4.880450in}}%
\pgfpathcurveto{\pgfqpoint{9.498478in}{4.869400in}}{\pgfqpoint{9.502868in}{4.858801in}}{\pgfqpoint{9.510682in}{4.850988in}}%
\pgfpathcurveto{\pgfqpoint{9.518495in}{4.843174in}}{\pgfqpoint{9.529095in}{4.838784in}}{\pgfqpoint{9.540145in}{4.838784in}}%
\pgfpathlineto{\pgfqpoint{9.540145in}{4.838784in}}%
\pgfpathclose%
\pgfusepath{stroke}%
\end{pgfscope}%
\begin{pgfscope}%
\pgfpathrectangle{\pgfqpoint{7.512535in}{0.437222in}}{\pgfqpoint{6.275590in}{5.159444in}}%
\pgfusepath{clip}%
\pgfsetbuttcap%
\pgfsetroundjoin%
\pgfsetlinewidth{1.003750pt}%
\definecolor{currentstroke}{rgb}{0.827451,0.827451,0.827451}%
\pgfsetstrokecolor{currentstroke}%
\pgfsetstrokeopacity{0.800000}%
\pgfsetdash{}{0pt}%
\pgfpathmoveto{\pgfqpoint{12.547906in}{5.274085in}}%
\pgfpathcurveto{\pgfqpoint{12.558956in}{5.274085in}}{\pgfqpoint{12.569556in}{5.278475in}}{\pgfqpoint{12.577369in}{5.286289in}}%
\pgfpathcurveto{\pgfqpoint{12.585183in}{5.294102in}}{\pgfqpoint{12.589573in}{5.304701in}}{\pgfqpoint{12.589573in}{5.315752in}}%
\pgfpathcurveto{\pgfqpoint{12.589573in}{5.326802in}}{\pgfqpoint{12.585183in}{5.337401in}}{\pgfqpoint{12.577369in}{5.345214in}}%
\pgfpathcurveto{\pgfqpoint{12.569556in}{5.353028in}}{\pgfqpoint{12.558956in}{5.357418in}}{\pgfqpoint{12.547906in}{5.357418in}}%
\pgfpathcurveto{\pgfqpoint{12.536856in}{5.357418in}}{\pgfqpoint{12.526257in}{5.353028in}}{\pgfqpoint{12.518444in}{5.345214in}}%
\pgfpathcurveto{\pgfqpoint{12.510630in}{5.337401in}}{\pgfqpoint{12.506240in}{5.326802in}}{\pgfqpoint{12.506240in}{5.315752in}}%
\pgfpathcurveto{\pgfqpoint{12.506240in}{5.304701in}}{\pgfqpoint{12.510630in}{5.294102in}}{\pgfqpoint{12.518444in}{5.286289in}}%
\pgfpathcurveto{\pgfqpoint{12.526257in}{5.278475in}}{\pgfqpoint{12.536856in}{5.274085in}}{\pgfqpoint{12.547906in}{5.274085in}}%
\pgfpathlineto{\pgfqpoint{12.547906in}{5.274085in}}%
\pgfpathclose%
\pgfusepath{stroke}%
\end{pgfscope}%
\begin{pgfscope}%
\pgfpathrectangle{\pgfqpoint{7.512535in}{0.437222in}}{\pgfqpoint{6.275590in}{5.159444in}}%
\pgfusepath{clip}%
\pgfsetbuttcap%
\pgfsetroundjoin%
\pgfsetlinewidth{1.003750pt}%
\definecolor{currentstroke}{rgb}{0.827451,0.827451,0.827451}%
\pgfsetstrokecolor{currentstroke}%
\pgfsetstrokeopacity{0.800000}%
\pgfsetdash{}{0pt}%
\pgfpathmoveto{\pgfqpoint{13.488123in}{4.987272in}}%
\pgfpathcurveto{\pgfqpoint{13.499173in}{4.987272in}}{\pgfqpoint{13.509772in}{4.991662in}}{\pgfqpoint{13.517586in}{4.999476in}}%
\pgfpathcurveto{\pgfqpoint{13.525399in}{5.007289in}}{\pgfqpoint{13.529790in}{5.017888in}}{\pgfqpoint{13.529790in}{5.028938in}}%
\pgfpathcurveto{\pgfqpoint{13.529790in}{5.039988in}}{\pgfqpoint{13.525399in}{5.050588in}}{\pgfqpoint{13.517586in}{5.058401in}}%
\pgfpathcurveto{\pgfqpoint{13.509772in}{5.066215in}}{\pgfqpoint{13.499173in}{5.070605in}}{\pgfqpoint{13.488123in}{5.070605in}}%
\pgfpathcurveto{\pgfqpoint{13.477073in}{5.070605in}}{\pgfqpoint{13.466474in}{5.066215in}}{\pgfqpoint{13.458660in}{5.058401in}}%
\pgfpathcurveto{\pgfqpoint{13.450847in}{5.050588in}}{\pgfqpoint{13.446456in}{5.039988in}}{\pgfqpoint{13.446456in}{5.028938in}}%
\pgfpathcurveto{\pgfqpoint{13.446456in}{5.017888in}}{\pgfqpoint{13.450847in}{5.007289in}}{\pgfqpoint{13.458660in}{4.999476in}}%
\pgfpathcurveto{\pgfqpoint{13.466474in}{4.991662in}}{\pgfqpoint{13.477073in}{4.987272in}}{\pgfqpoint{13.488123in}{4.987272in}}%
\pgfpathlineto{\pgfqpoint{13.488123in}{4.987272in}}%
\pgfpathclose%
\pgfusepath{stroke}%
\end{pgfscope}%
\begin{pgfscope}%
\pgfpathrectangle{\pgfqpoint{7.512535in}{0.437222in}}{\pgfqpoint{6.275590in}{5.159444in}}%
\pgfusepath{clip}%
\pgfsetbuttcap%
\pgfsetroundjoin%
\pgfsetlinewidth{1.003750pt}%
\definecolor{currentstroke}{rgb}{0.827451,0.827451,0.827451}%
\pgfsetstrokecolor{currentstroke}%
\pgfsetstrokeopacity{0.800000}%
\pgfsetdash{}{0pt}%
\pgfpathmoveto{\pgfqpoint{13.488123in}{5.002843in}}%
\pgfpathcurveto{\pgfqpoint{13.499173in}{5.002843in}}{\pgfqpoint{13.509772in}{5.007233in}}{\pgfqpoint{13.517586in}{5.015047in}}%
\pgfpathcurveto{\pgfqpoint{13.525399in}{5.022860in}}{\pgfqpoint{13.529790in}{5.033459in}}{\pgfqpoint{13.529790in}{5.044510in}}%
\pgfpathcurveto{\pgfqpoint{13.529790in}{5.055560in}}{\pgfqpoint{13.525399in}{5.066159in}}{\pgfqpoint{13.517586in}{5.073972in}}%
\pgfpathcurveto{\pgfqpoint{13.509772in}{5.081786in}}{\pgfqpoint{13.499173in}{5.086176in}}{\pgfqpoint{13.488123in}{5.086176in}}%
\pgfpathcurveto{\pgfqpoint{13.477073in}{5.086176in}}{\pgfqpoint{13.466474in}{5.081786in}}{\pgfqpoint{13.458660in}{5.073972in}}%
\pgfpathcurveto{\pgfqpoint{13.450847in}{5.066159in}}{\pgfqpoint{13.446456in}{5.055560in}}{\pgfqpoint{13.446456in}{5.044510in}}%
\pgfpathcurveto{\pgfqpoint{13.446456in}{5.033459in}}{\pgfqpoint{13.450847in}{5.022860in}}{\pgfqpoint{13.458660in}{5.015047in}}%
\pgfpathcurveto{\pgfqpoint{13.466474in}{5.007233in}}{\pgfqpoint{13.477073in}{5.002843in}}{\pgfqpoint{13.488123in}{5.002843in}}%
\pgfpathlineto{\pgfqpoint{13.488123in}{5.002843in}}%
\pgfpathclose%
\pgfusepath{stroke}%
\end{pgfscope}%
\begin{pgfscope}%
\pgfpathrectangle{\pgfqpoint{7.512535in}{0.437222in}}{\pgfqpoint{6.275590in}{5.159444in}}%
\pgfusepath{clip}%
\pgfsetbuttcap%
\pgfsetroundjoin%
\pgfsetlinewidth{1.003750pt}%
\definecolor{currentstroke}{rgb}{0.827451,0.827451,0.827451}%
\pgfsetstrokecolor{currentstroke}%
\pgfsetstrokeopacity{0.800000}%
\pgfsetdash{}{0pt}%
\pgfpathmoveto{\pgfqpoint{12.524904in}{5.288614in}}%
\pgfpathcurveto{\pgfqpoint{12.535954in}{5.288614in}}{\pgfqpoint{12.546553in}{5.293004in}}{\pgfqpoint{12.554367in}{5.300818in}}%
\pgfpathcurveto{\pgfqpoint{12.562181in}{5.308632in}}{\pgfqpoint{12.566571in}{5.319231in}}{\pgfqpoint{12.566571in}{5.330281in}}%
\pgfpathcurveto{\pgfqpoint{12.566571in}{5.341331in}}{\pgfqpoint{12.562181in}{5.351930in}}{\pgfqpoint{12.554367in}{5.359744in}}%
\pgfpathcurveto{\pgfqpoint{12.546553in}{5.367557in}}{\pgfqpoint{12.535954in}{5.371947in}}{\pgfqpoint{12.524904in}{5.371947in}}%
\pgfpathcurveto{\pgfqpoint{12.513854in}{5.371947in}}{\pgfqpoint{12.503255in}{5.367557in}}{\pgfqpoint{12.495441in}{5.359744in}}%
\pgfpathcurveto{\pgfqpoint{12.487628in}{5.351930in}}{\pgfqpoint{12.483237in}{5.341331in}}{\pgfqpoint{12.483237in}{5.330281in}}%
\pgfpathcurveto{\pgfqpoint{12.483237in}{5.319231in}}{\pgfqpoint{12.487628in}{5.308632in}}{\pgfqpoint{12.495441in}{5.300818in}}%
\pgfpathcurveto{\pgfqpoint{12.503255in}{5.293004in}}{\pgfqpoint{12.513854in}{5.288614in}}{\pgfqpoint{12.524904in}{5.288614in}}%
\pgfpathlineto{\pgfqpoint{12.524904in}{5.288614in}}%
\pgfpathclose%
\pgfusepath{stroke}%
\end{pgfscope}%
\begin{pgfscope}%
\pgfpathrectangle{\pgfqpoint{7.512535in}{0.437222in}}{\pgfqpoint{6.275590in}{5.159444in}}%
\pgfusepath{clip}%
\pgfsetbuttcap%
\pgfsetroundjoin%
\pgfsetlinewidth{1.003750pt}%
\definecolor{currentstroke}{rgb}{0.827451,0.827451,0.827451}%
\pgfsetstrokecolor{currentstroke}%
\pgfsetstrokeopacity{0.800000}%
\pgfsetdash{}{0pt}%
\pgfpathmoveto{\pgfqpoint{11.645781in}{3.328017in}}%
\pgfpathcurveto{\pgfqpoint{11.656832in}{3.328017in}}{\pgfqpoint{11.667431in}{3.332407in}}{\pgfqpoint{11.675244in}{3.340220in}}%
\pgfpathcurveto{\pgfqpoint{11.683058in}{3.348034in}}{\pgfqpoint{11.687448in}{3.358633in}}{\pgfqpoint{11.687448in}{3.369683in}}%
\pgfpathcurveto{\pgfqpoint{11.687448in}{3.380733in}}{\pgfqpoint{11.683058in}{3.391332in}}{\pgfqpoint{11.675244in}{3.399146in}}%
\pgfpathcurveto{\pgfqpoint{11.667431in}{3.406960in}}{\pgfqpoint{11.656832in}{3.411350in}}{\pgfqpoint{11.645781in}{3.411350in}}%
\pgfpathcurveto{\pgfqpoint{11.634731in}{3.411350in}}{\pgfqpoint{11.624132in}{3.406960in}}{\pgfqpoint{11.616319in}{3.399146in}}%
\pgfpathcurveto{\pgfqpoint{11.608505in}{3.391332in}}{\pgfqpoint{11.604115in}{3.380733in}}{\pgfqpoint{11.604115in}{3.369683in}}%
\pgfpathcurveto{\pgfqpoint{11.604115in}{3.358633in}}{\pgfqpoint{11.608505in}{3.348034in}}{\pgfqpoint{11.616319in}{3.340220in}}%
\pgfpathcurveto{\pgfqpoint{11.624132in}{3.332407in}}{\pgfqpoint{11.634731in}{3.328017in}}{\pgfqpoint{11.645781in}{3.328017in}}%
\pgfpathlineto{\pgfqpoint{11.645781in}{3.328017in}}%
\pgfpathclose%
\pgfusepath{stroke}%
\end{pgfscope}%
\begin{pgfscope}%
\pgfpathrectangle{\pgfqpoint{7.512535in}{0.437222in}}{\pgfqpoint{6.275590in}{5.159444in}}%
\pgfusepath{clip}%
\pgfsetbuttcap%
\pgfsetroundjoin%
\pgfsetlinewidth{1.003750pt}%
\definecolor{currentstroke}{rgb}{0.827451,0.827451,0.827451}%
\pgfsetstrokecolor{currentstroke}%
\pgfsetstrokeopacity{0.800000}%
\pgfsetdash{}{0pt}%
\pgfpathmoveto{\pgfqpoint{11.583598in}{3.738419in}}%
\pgfpathcurveto{\pgfqpoint{11.594648in}{3.738419in}}{\pgfqpoint{11.605248in}{3.742809in}}{\pgfqpoint{11.613061in}{3.750623in}}%
\pgfpathcurveto{\pgfqpoint{11.620875in}{3.758437in}}{\pgfqpoint{11.625265in}{3.769036in}}{\pgfqpoint{11.625265in}{3.780086in}}%
\pgfpathcurveto{\pgfqpoint{11.625265in}{3.791136in}}{\pgfqpoint{11.620875in}{3.801735in}}{\pgfqpoint{11.613061in}{3.809548in}}%
\pgfpathcurveto{\pgfqpoint{11.605248in}{3.817362in}}{\pgfqpoint{11.594648in}{3.821752in}}{\pgfqpoint{11.583598in}{3.821752in}}%
\pgfpathcurveto{\pgfqpoint{11.572548in}{3.821752in}}{\pgfqpoint{11.561949in}{3.817362in}}{\pgfqpoint{11.554136in}{3.809548in}}%
\pgfpathcurveto{\pgfqpoint{11.546322in}{3.801735in}}{\pgfqpoint{11.541932in}{3.791136in}}{\pgfqpoint{11.541932in}{3.780086in}}%
\pgfpathcurveto{\pgfqpoint{11.541932in}{3.769036in}}{\pgfqpoint{11.546322in}{3.758437in}}{\pgfqpoint{11.554136in}{3.750623in}}%
\pgfpathcurveto{\pgfqpoint{11.561949in}{3.742809in}}{\pgfqpoint{11.572548in}{3.738419in}}{\pgfqpoint{11.583598in}{3.738419in}}%
\pgfpathlineto{\pgfqpoint{11.583598in}{3.738419in}}%
\pgfpathclose%
\pgfusepath{stroke}%
\end{pgfscope}%
\begin{pgfscope}%
\pgfpathrectangle{\pgfqpoint{7.512535in}{0.437222in}}{\pgfqpoint{6.275590in}{5.159444in}}%
\pgfusepath{clip}%
\pgfsetbuttcap%
\pgfsetroundjoin%
\pgfsetlinewidth{1.003750pt}%
\definecolor{currentstroke}{rgb}{0.827451,0.827451,0.827451}%
\pgfsetstrokecolor{currentstroke}%
\pgfsetstrokeopacity{0.800000}%
\pgfsetdash{}{0pt}%
\pgfpathmoveto{\pgfqpoint{9.911802in}{0.673514in}}%
\pgfpathcurveto{\pgfqpoint{9.922853in}{0.673514in}}{\pgfqpoint{9.933452in}{0.677904in}}{\pgfqpoint{9.941265in}{0.685718in}}%
\pgfpathcurveto{\pgfqpoint{9.949079in}{0.693531in}}{\pgfqpoint{9.953469in}{0.704130in}}{\pgfqpoint{9.953469in}{0.715181in}}%
\pgfpathcurveto{\pgfqpoint{9.953469in}{0.726231in}}{\pgfqpoint{9.949079in}{0.736830in}}{\pgfqpoint{9.941265in}{0.744643in}}%
\pgfpathcurveto{\pgfqpoint{9.933452in}{0.752457in}}{\pgfqpoint{9.922853in}{0.756847in}}{\pgfqpoint{9.911802in}{0.756847in}}%
\pgfpathcurveto{\pgfqpoint{9.900752in}{0.756847in}}{\pgfqpoint{9.890153in}{0.752457in}}{\pgfqpoint{9.882340in}{0.744643in}}%
\pgfpathcurveto{\pgfqpoint{9.874526in}{0.736830in}}{\pgfqpoint{9.870136in}{0.726231in}}{\pgfqpoint{9.870136in}{0.715181in}}%
\pgfpathcurveto{\pgfqpoint{9.870136in}{0.704130in}}{\pgfqpoint{9.874526in}{0.693531in}}{\pgfqpoint{9.882340in}{0.685718in}}%
\pgfpathcurveto{\pgfqpoint{9.890153in}{0.677904in}}{\pgfqpoint{9.900752in}{0.673514in}}{\pgfqpoint{9.911802in}{0.673514in}}%
\pgfpathlineto{\pgfqpoint{9.911802in}{0.673514in}}%
\pgfpathclose%
\pgfusepath{stroke}%
\end{pgfscope}%
\begin{pgfscope}%
\pgfpathrectangle{\pgfqpoint{7.512535in}{0.437222in}}{\pgfqpoint{6.275590in}{5.159444in}}%
\pgfusepath{clip}%
\pgfsetbuttcap%
\pgfsetroundjoin%
\pgfsetlinewidth{1.003750pt}%
\definecolor{currentstroke}{rgb}{0.827451,0.827451,0.827451}%
\pgfsetstrokecolor{currentstroke}%
\pgfsetstrokeopacity{0.800000}%
\pgfsetdash{}{0pt}%
\pgfpathmoveto{\pgfqpoint{13.525425in}{4.914881in}}%
\pgfpathcurveto{\pgfqpoint{13.536475in}{4.914881in}}{\pgfqpoint{13.547074in}{4.919272in}}{\pgfqpoint{13.554888in}{4.927085in}}%
\pgfpathcurveto{\pgfqpoint{13.562701in}{4.934899in}}{\pgfqpoint{13.567091in}{4.945498in}}{\pgfqpoint{13.567091in}{4.956548in}}%
\pgfpathcurveto{\pgfqpoint{13.567091in}{4.967598in}}{\pgfqpoint{13.562701in}{4.978197in}}{\pgfqpoint{13.554888in}{4.986011in}}%
\pgfpathcurveto{\pgfqpoint{13.547074in}{4.993824in}}{\pgfqpoint{13.536475in}{4.998215in}}{\pgfqpoint{13.525425in}{4.998215in}}%
\pgfpathcurveto{\pgfqpoint{13.514375in}{4.998215in}}{\pgfqpoint{13.503776in}{4.993824in}}{\pgfqpoint{13.495962in}{4.986011in}}%
\pgfpathcurveto{\pgfqpoint{13.488148in}{4.978197in}}{\pgfqpoint{13.483758in}{4.967598in}}{\pgfqpoint{13.483758in}{4.956548in}}%
\pgfpathcurveto{\pgfqpoint{13.483758in}{4.945498in}}{\pgfqpoint{13.488148in}{4.934899in}}{\pgfqpoint{13.495962in}{4.927085in}}%
\pgfpathcurveto{\pgfqpoint{13.503776in}{4.919272in}}{\pgfqpoint{13.514375in}{4.914881in}}{\pgfqpoint{13.525425in}{4.914881in}}%
\pgfpathlineto{\pgfqpoint{13.525425in}{4.914881in}}%
\pgfpathclose%
\pgfusepath{stroke}%
\end{pgfscope}%
\begin{pgfscope}%
\pgfpathrectangle{\pgfqpoint{7.512535in}{0.437222in}}{\pgfqpoint{6.275590in}{5.159444in}}%
\pgfusepath{clip}%
\pgfsetbuttcap%
\pgfsetroundjoin%
\pgfsetlinewidth{1.003750pt}%
\definecolor{currentstroke}{rgb}{0.827451,0.827451,0.827451}%
\pgfsetstrokecolor{currentstroke}%
\pgfsetstrokeopacity{0.800000}%
\pgfsetdash{}{0pt}%
\pgfpathmoveto{\pgfqpoint{12.428187in}{4.668560in}}%
\pgfpathcurveto{\pgfqpoint{12.439237in}{4.668560in}}{\pgfqpoint{12.449836in}{4.672950in}}{\pgfqpoint{12.457650in}{4.680764in}}%
\pgfpathcurveto{\pgfqpoint{12.465463in}{4.688578in}}{\pgfqpoint{12.469854in}{4.699177in}}{\pgfqpoint{12.469854in}{4.710227in}}%
\pgfpathcurveto{\pgfqpoint{12.469854in}{4.721277in}}{\pgfqpoint{12.465463in}{4.731876in}}{\pgfqpoint{12.457650in}{4.739690in}}%
\pgfpathcurveto{\pgfqpoint{12.449836in}{4.747503in}}{\pgfqpoint{12.439237in}{4.751893in}}{\pgfqpoint{12.428187in}{4.751893in}}%
\pgfpathcurveto{\pgfqpoint{12.417137in}{4.751893in}}{\pgfqpoint{12.406538in}{4.747503in}}{\pgfqpoint{12.398724in}{4.739690in}}%
\pgfpathcurveto{\pgfqpoint{12.390911in}{4.731876in}}{\pgfqpoint{12.386520in}{4.721277in}}{\pgfqpoint{12.386520in}{4.710227in}}%
\pgfpathcurveto{\pgfqpoint{12.386520in}{4.699177in}}{\pgfqpoint{12.390911in}{4.688578in}}{\pgfqpoint{12.398724in}{4.680764in}}%
\pgfpathcurveto{\pgfqpoint{12.406538in}{4.672950in}}{\pgfqpoint{12.417137in}{4.668560in}}{\pgfqpoint{12.428187in}{4.668560in}}%
\pgfpathlineto{\pgfqpoint{12.428187in}{4.668560in}}%
\pgfpathclose%
\pgfusepath{stroke}%
\end{pgfscope}%
\begin{pgfscope}%
\pgfpathrectangle{\pgfqpoint{7.512535in}{0.437222in}}{\pgfqpoint{6.275590in}{5.159444in}}%
\pgfusepath{clip}%
\pgfsetbuttcap%
\pgfsetroundjoin%
\pgfsetlinewidth{1.003750pt}%
\definecolor{currentstroke}{rgb}{0.827451,0.827451,0.827451}%
\pgfsetstrokecolor{currentstroke}%
\pgfsetstrokeopacity{0.800000}%
\pgfsetdash{}{0pt}%
\pgfpathmoveto{\pgfqpoint{10.259535in}{4.447570in}}%
\pgfpathcurveto{\pgfqpoint{10.270585in}{4.447570in}}{\pgfqpoint{10.281184in}{4.451960in}}{\pgfqpoint{10.288997in}{4.459773in}}%
\pgfpathcurveto{\pgfqpoint{10.296811in}{4.467587in}}{\pgfqpoint{10.301201in}{4.478186in}}{\pgfqpoint{10.301201in}{4.489236in}}%
\pgfpathcurveto{\pgfqpoint{10.301201in}{4.500286in}}{\pgfqpoint{10.296811in}{4.510885in}}{\pgfqpoint{10.288997in}{4.518699in}}%
\pgfpathcurveto{\pgfqpoint{10.281184in}{4.526513in}}{\pgfqpoint{10.270585in}{4.530903in}}{\pgfqpoint{10.259535in}{4.530903in}}%
\pgfpathcurveto{\pgfqpoint{10.248485in}{4.530903in}}{\pgfqpoint{10.237886in}{4.526513in}}{\pgfqpoint{10.230072in}{4.518699in}}%
\pgfpathcurveto{\pgfqpoint{10.222258in}{4.510885in}}{\pgfqpoint{10.217868in}{4.500286in}}{\pgfqpoint{10.217868in}{4.489236in}}%
\pgfpathcurveto{\pgfqpoint{10.217868in}{4.478186in}}{\pgfqpoint{10.222258in}{4.467587in}}{\pgfqpoint{10.230072in}{4.459773in}}%
\pgfpathcurveto{\pgfqpoint{10.237886in}{4.451960in}}{\pgfqpoint{10.248485in}{4.447570in}}{\pgfqpoint{10.259535in}{4.447570in}}%
\pgfpathlineto{\pgfqpoint{10.259535in}{4.447570in}}%
\pgfpathclose%
\pgfusepath{stroke}%
\end{pgfscope}%
\begin{pgfscope}%
\pgfpathrectangle{\pgfqpoint{7.512535in}{0.437222in}}{\pgfqpoint{6.275590in}{5.159444in}}%
\pgfusepath{clip}%
\pgfsetbuttcap%
\pgfsetroundjoin%
\pgfsetlinewidth{1.003750pt}%
\definecolor{currentstroke}{rgb}{0.827451,0.827451,0.827451}%
\pgfsetstrokecolor{currentstroke}%
\pgfsetstrokeopacity{0.800000}%
\pgfsetdash{}{0pt}%
\pgfpathmoveto{\pgfqpoint{9.836077in}{3.291016in}}%
\pgfpathcurveto{\pgfqpoint{9.847127in}{3.291016in}}{\pgfqpoint{9.857726in}{3.295406in}}{\pgfqpoint{9.865540in}{3.303220in}}%
\pgfpathcurveto{\pgfqpoint{9.873353in}{3.311034in}}{\pgfqpoint{9.877743in}{3.321633in}}{\pgfqpoint{9.877743in}{3.332683in}}%
\pgfpathcurveto{\pgfqpoint{9.877743in}{3.343733in}}{\pgfqpoint{9.873353in}{3.354332in}}{\pgfqpoint{9.865540in}{3.362146in}}%
\pgfpathcurveto{\pgfqpoint{9.857726in}{3.369959in}}{\pgfqpoint{9.847127in}{3.374349in}}{\pgfqpoint{9.836077in}{3.374349in}}%
\pgfpathcurveto{\pgfqpoint{9.825027in}{3.374349in}}{\pgfqpoint{9.814428in}{3.369959in}}{\pgfqpoint{9.806614in}{3.362146in}}%
\pgfpathcurveto{\pgfqpoint{9.798800in}{3.354332in}}{\pgfqpoint{9.794410in}{3.343733in}}{\pgfqpoint{9.794410in}{3.332683in}}%
\pgfpathcurveto{\pgfqpoint{9.794410in}{3.321633in}}{\pgfqpoint{9.798800in}{3.311034in}}{\pgfqpoint{9.806614in}{3.303220in}}%
\pgfpathcurveto{\pgfqpoint{9.814428in}{3.295406in}}{\pgfqpoint{9.825027in}{3.291016in}}{\pgfqpoint{9.836077in}{3.291016in}}%
\pgfpathlineto{\pgfqpoint{9.836077in}{3.291016in}}%
\pgfpathclose%
\pgfusepath{stroke}%
\end{pgfscope}%
\begin{pgfscope}%
\pgfpathrectangle{\pgfqpoint{7.512535in}{0.437222in}}{\pgfqpoint{6.275590in}{5.159444in}}%
\pgfusepath{clip}%
\pgfsetbuttcap%
\pgfsetroundjoin%
\pgfsetlinewidth{1.003750pt}%
\definecolor{currentstroke}{rgb}{0.827451,0.827451,0.827451}%
\pgfsetstrokecolor{currentstroke}%
\pgfsetstrokeopacity{0.800000}%
\pgfsetdash{}{0pt}%
\pgfpathmoveto{\pgfqpoint{9.802410in}{3.570658in}}%
\pgfpathcurveto{\pgfqpoint{9.813460in}{3.570658in}}{\pgfqpoint{9.824059in}{3.575048in}}{\pgfqpoint{9.831872in}{3.582862in}}%
\pgfpathcurveto{\pgfqpoint{9.839686in}{3.590675in}}{\pgfqpoint{9.844076in}{3.601274in}}{\pgfqpoint{9.844076in}{3.612325in}}%
\pgfpathcurveto{\pgfqpoint{9.844076in}{3.623375in}}{\pgfqpoint{9.839686in}{3.633974in}}{\pgfqpoint{9.831872in}{3.641787in}}%
\pgfpathcurveto{\pgfqpoint{9.824059in}{3.649601in}}{\pgfqpoint{9.813460in}{3.653991in}}{\pgfqpoint{9.802410in}{3.653991in}}%
\pgfpathcurveto{\pgfqpoint{9.791360in}{3.653991in}}{\pgfqpoint{9.780760in}{3.649601in}}{\pgfqpoint{9.772947in}{3.641787in}}%
\pgfpathcurveto{\pgfqpoint{9.765133in}{3.633974in}}{\pgfqpoint{9.760743in}{3.623375in}}{\pgfqpoint{9.760743in}{3.612325in}}%
\pgfpathcurveto{\pgfqpoint{9.760743in}{3.601274in}}{\pgfqpoint{9.765133in}{3.590675in}}{\pgfqpoint{9.772947in}{3.582862in}}%
\pgfpathcurveto{\pgfqpoint{9.780760in}{3.575048in}}{\pgfqpoint{9.791360in}{3.570658in}}{\pgfqpoint{9.802410in}{3.570658in}}%
\pgfpathlineto{\pgfqpoint{9.802410in}{3.570658in}}%
\pgfpathclose%
\pgfusepath{stroke}%
\end{pgfscope}%
\begin{pgfscope}%
\pgfpathrectangle{\pgfqpoint{7.512535in}{0.437222in}}{\pgfqpoint{6.275590in}{5.159444in}}%
\pgfusepath{clip}%
\pgfsetbuttcap%
\pgfsetroundjoin%
\pgfsetlinewidth{1.003750pt}%
\definecolor{currentstroke}{rgb}{0.827451,0.827451,0.827451}%
\pgfsetstrokecolor{currentstroke}%
\pgfsetstrokeopacity{0.800000}%
\pgfsetdash{}{0pt}%
\pgfpathmoveto{\pgfqpoint{10.005215in}{4.464965in}}%
\pgfpathcurveto{\pgfqpoint{10.016265in}{4.464965in}}{\pgfqpoint{10.026865in}{4.469355in}}{\pgfqpoint{10.034678in}{4.477169in}}%
\pgfpathcurveto{\pgfqpoint{10.042492in}{4.484982in}}{\pgfqpoint{10.046882in}{4.495581in}}{\pgfqpoint{10.046882in}{4.506632in}}%
\pgfpathcurveto{\pgfqpoint{10.046882in}{4.517682in}}{\pgfqpoint{10.042492in}{4.528281in}}{\pgfqpoint{10.034678in}{4.536094in}}%
\pgfpathcurveto{\pgfqpoint{10.026865in}{4.543908in}}{\pgfqpoint{10.016265in}{4.548298in}}{\pgfqpoint{10.005215in}{4.548298in}}%
\pgfpathcurveto{\pgfqpoint{9.994165in}{4.548298in}}{\pgfqpoint{9.983566in}{4.543908in}}{\pgfqpoint{9.975753in}{4.536094in}}%
\pgfpathcurveto{\pgfqpoint{9.967939in}{4.528281in}}{\pgfqpoint{9.963549in}{4.517682in}}{\pgfqpoint{9.963549in}{4.506632in}}%
\pgfpathcurveto{\pgfqpoint{9.963549in}{4.495581in}}{\pgfqpoint{9.967939in}{4.484982in}}{\pgfqpoint{9.975753in}{4.477169in}}%
\pgfpathcurveto{\pgfqpoint{9.983566in}{4.469355in}}{\pgfqpoint{9.994165in}{4.464965in}}{\pgfqpoint{10.005215in}{4.464965in}}%
\pgfpathlineto{\pgfqpoint{10.005215in}{4.464965in}}%
\pgfpathclose%
\pgfusepath{stroke}%
\end{pgfscope}%
\begin{pgfscope}%
\pgfpathrectangle{\pgfqpoint{7.512535in}{0.437222in}}{\pgfqpoint{6.275590in}{5.159444in}}%
\pgfusepath{clip}%
\pgfsetbuttcap%
\pgfsetroundjoin%
\pgfsetlinewidth{1.003750pt}%
\definecolor{currentstroke}{rgb}{0.827451,0.827451,0.827451}%
\pgfsetstrokecolor{currentstroke}%
\pgfsetstrokeopacity{0.800000}%
\pgfsetdash{}{0pt}%
\pgfpathmoveto{\pgfqpoint{9.500868in}{3.326629in}}%
\pgfpathcurveto{\pgfqpoint{9.511919in}{3.326629in}}{\pgfqpoint{9.522518in}{3.331019in}}{\pgfqpoint{9.530331in}{3.338833in}}%
\pgfpathcurveto{\pgfqpoint{9.538145in}{3.346646in}}{\pgfqpoint{9.542535in}{3.357246in}}{\pgfqpoint{9.542535in}{3.368296in}}%
\pgfpathcurveto{\pgfqpoint{9.542535in}{3.379346in}}{\pgfqpoint{9.538145in}{3.389945in}}{\pgfqpoint{9.530331in}{3.397758in}}%
\pgfpathcurveto{\pgfqpoint{9.522518in}{3.405572in}}{\pgfqpoint{9.511919in}{3.409962in}}{\pgfqpoint{9.500868in}{3.409962in}}%
\pgfpathcurveto{\pgfqpoint{9.489818in}{3.409962in}}{\pgfqpoint{9.479219in}{3.405572in}}{\pgfqpoint{9.471406in}{3.397758in}}%
\pgfpathcurveto{\pgfqpoint{9.463592in}{3.389945in}}{\pgfqpoint{9.459202in}{3.379346in}}{\pgfqpoint{9.459202in}{3.368296in}}%
\pgfpathcurveto{\pgfqpoint{9.459202in}{3.357246in}}{\pgfqpoint{9.463592in}{3.346646in}}{\pgfqpoint{9.471406in}{3.338833in}}%
\pgfpathcurveto{\pgfqpoint{9.479219in}{3.331019in}}{\pgfqpoint{9.489818in}{3.326629in}}{\pgfqpoint{9.500868in}{3.326629in}}%
\pgfpathlineto{\pgfqpoint{9.500868in}{3.326629in}}%
\pgfpathclose%
\pgfusepath{stroke}%
\end{pgfscope}%
\begin{pgfscope}%
\pgfpathrectangle{\pgfqpoint{7.512535in}{0.437222in}}{\pgfqpoint{6.275590in}{5.159444in}}%
\pgfusepath{clip}%
\pgfsetbuttcap%
\pgfsetroundjoin%
\pgfsetlinewidth{1.003750pt}%
\definecolor{currentstroke}{rgb}{0.827451,0.827451,0.827451}%
\pgfsetstrokecolor{currentstroke}%
\pgfsetstrokeopacity{0.800000}%
\pgfsetdash{}{0pt}%
\pgfpathmoveto{\pgfqpoint{10.287229in}{5.087121in}}%
\pgfpathcurveto{\pgfqpoint{10.298279in}{5.087121in}}{\pgfqpoint{10.308878in}{5.091511in}}{\pgfqpoint{10.316692in}{5.099325in}}%
\pgfpathcurveto{\pgfqpoint{10.324505in}{5.107138in}}{\pgfqpoint{10.328896in}{5.117737in}}{\pgfqpoint{10.328896in}{5.128787in}}%
\pgfpathcurveto{\pgfqpoint{10.328896in}{5.139838in}}{\pgfqpoint{10.324505in}{5.150437in}}{\pgfqpoint{10.316692in}{5.158250in}}%
\pgfpathcurveto{\pgfqpoint{10.308878in}{5.166064in}}{\pgfqpoint{10.298279in}{5.170454in}}{\pgfqpoint{10.287229in}{5.170454in}}%
\pgfpathcurveto{\pgfqpoint{10.276179in}{5.170454in}}{\pgfqpoint{10.265580in}{5.166064in}}{\pgfqpoint{10.257766in}{5.158250in}}%
\pgfpathcurveto{\pgfqpoint{10.249953in}{5.150437in}}{\pgfqpoint{10.245562in}{5.139838in}}{\pgfqpoint{10.245562in}{5.128787in}}%
\pgfpathcurveto{\pgfqpoint{10.245562in}{5.117737in}}{\pgfqpoint{10.249953in}{5.107138in}}{\pgfqpoint{10.257766in}{5.099325in}}%
\pgfpathcurveto{\pgfqpoint{10.265580in}{5.091511in}}{\pgfqpoint{10.276179in}{5.087121in}}{\pgfqpoint{10.287229in}{5.087121in}}%
\pgfpathlineto{\pgfqpoint{10.287229in}{5.087121in}}%
\pgfpathclose%
\pgfusepath{stroke}%
\end{pgfscope}%
\begin{pgfscope}%
\pgfpathrectangle{\pgfqpoint{7.512535in}{0.437222in}}{\pgfqpoint{6.275590in}{5.159444in}}%
\pgfusepath{clip}%
\pgfsetbuttcap%
\pgfsetroundjoin%
\pgfsetlinewidth{1.003750pt}%
\definecolor{currentstroke}{rgb}{0.827451,0.827451,0.827451}%
\pgfsetstrokecolor{currentstroke}%
\pgfsetstrokeopacity{0.800000}%
\pgfsetdash{}{0pt}%
\pgfpathmoveto{\pgfqpoint{7.887873in}{1.346375in}}%
\pgfpathcurveto{\pgfqpoint{7.898923in}{1.346375in}}{\pgfqpoint{7.909522in}{1.350765in}}{\pgfqpoint{7.917336in}{1.358578in}}%
\pgfpathcurveto{\pgfqpoint{7.925149in}{1.366392in}}{\pgfqpoint{7.929540in}{1.376991in}}{\pgfqpoint{7.929540in}{1.388041in}}%
\pgfpathcurveto{\pgfqpoint{7.929540in}{1.399091in}}{\pgfqpoint{7.925149in}{1.409690in}}{\pgfqpoint{7.917336in}{1.417504in}}%
\pgfpathcurveto{\pgfqpoint{7.909522in}{1.425318in}}{\pgfqpoint{7.898923in}{1.429708in}}{\pgfqpoint{7.887873in}{1.429708in}}%
\pgfpathcurveto{\pgfqpoint{7.876823in}{1.429708in}}{\pgfqpoint{7.866224in}{1.425318in}}{\pgfqpoint{7.858410in}{1.417504in}}%
\pgfpathcurveto{\pgfqpoint{7.850596in}{1.409690in}}{\pgfqpoint{7.846206in}{1.399091in}}{\pgfqpoint{7.846206in}{1.388041in}}%
\pgfpathcurveto{\pgfqpoint{7.846206in}{1.376991in}}{\pgfqpoint{7.850596in}{1.366392in}}{\pgfqpoint{7.858410in}{1.358578in}}%
\pgfpathcurveto{\pgfqpoint{7.866224in}{1.350765in}}{\pgfqpoint{7.876823in}{1.346375in}}{\pgfqpoint{7.887873in}{1.346375in}}%
\pgfpathlineto{\pgfqpoint{7.887873in}{1.346375in}}%
\pgfpathclose%
\pgfusepath{stroke}%
\end{pgfscope}%
\begin{pgfscope}%
\pgfpathrectangle{\pgfqpoint{7.512535in}{0.437222in}}{\pgfqpoint{6.275590in}{5.159444in}}%
\pgfusepath{clip}%
\pgfsetbuttcap%
\pgfsetroundjoin%
\pgfsetlinewidth{1.003750pt}%
\definecolor{currentstroke}{rgb}{0.827451,0.827451,0.827451}%
\pgfsetstrokecolor{currentstroke}%
\pgfsetstrokeopacity{0.800000}%
\pgfsetdash{}{0pt}%
\pgfpathmoveto{\pgfqpoint{9.581399in}{2.990585in}}%
\pgfpathcurveto{\pgfqpoint{9.592449in}{2.990585in}}{\pgfqpoint{9.603048in}{2.994975in}}{\pgfqpoint{9.610862in}{3.002789in}}%
\pgfpathcurveto{\pgfqpoint{9.618675in}{3.010602in}}{\pgfqpoint{9.623065in}{3.021202in}}{\pgfqpoint{9.623065in}{3.032252in}}%
\pgfpathcurveto{\pgfqpoint{9.623065in}{3.043302in}}{\pgfqpoint{9.618675in}{3.053901in}}{\pgfqpoint{9.610862in}{3.061714in}}%
\pgfpathcurveto{\pgfqpoint{9.603048in}{3.069528in}}{\pgfqpoint{9.592449in}{3.073918in}}{\pgfqpoint{9.581399in}{3.073918in}}%
\pgfpathcurveto{\pgfqpoint{9.570349in}{3.073918in}}{\pgfqpoint{9.559750in}{3.069528in}}{\pgfqpoint{9.551936in}{3.061714in}}%
\pgfpathcurveto{\pgfqpoint{9.544122in}{3.053901in}}{\pgfqpoint{9.539732in}{3.043302in}}{\pgfqpoint{9.539732in}{3.032252in}}%
\pgfpathcurveto{\pgfqpoint{9.539732in}{3.021202in}}{\pgfqpoint{9.544122in}{3.010602in}}{\pgfqpoint{9.551936in}{3.002789in}}%
\pgfpathcurveto{\pgfqpoint{9.559750in}{2.994975in}}{\pgfqpoint{9.570349in}{2.990585in}}{\pgfqpoint{9.581399in}{2.990585in}}%
\pgfpathlineto{\pgfqpoint{9.581399in}{2.990585in}}%
\pgfpathclose%
\pgfusepath{stroke}%
\end{pgfscope}%
\begin{pgfscope}%
\pgfpathrectangle{\pgfqpoint{7.512535in}{0.437222in}}{\pgfqpoint{6.275590in}{5.159444in}}%
\pgfusepath{clip}%
\pgfsetbuttcap%
\pgfsetroundjoin%
\pgfsetlinewidth{1.003750pt}%
\definecolor{currentstroke}{rgb}{0.827451,0.827451,0.827451}%
\pgfsetstrokecolor{currentstroke}%
\pgfsetstrokeopacity{0.800000}%
\pgfsetdash{}{0pt}%
\pgfpathmoveto{\pgfqpoint{11.387346in}{5.101603in}}%
\pgfpathcurveto{\pgfqpoint{11.398396in}{5.101603in}}{\pgfqpoint{11.408995in}{5.105993in}}{\pgfqpoint{11.416809in}{5.113807in}}%
\pgfpathcurveto{\pgfqpoint{11.424622in}{5.121620in}}{\pgfqpoint{11.429013in}{5.132219in}}{\pgfqpoint{11.429013in}{5.143270in}}%
\pgfpathcurveto{\pgfqpoint{11.429013in}{5.154320in}}{\pgfqpoint{11.424622in}{5.164919in}}{\pgfqpoint{11.416809in}{5.172732in}}%
\pgfpathcurveto{\pgfqpoint{11.408995in}{5.180546in}}{\pgfqpoint{11.398396in}{5.184936in}}{\pgfqpoint{11.387346in}{5.184936in}}%
\pgfpathcurveto{\pgfqpoint{11.376296in}{5.184936in}}{\pgfqpoint{11.365697in}{5.180546in}}{\pgfqpoint{11.357883in}{5.172732in}}%
\pgfpathcurveto{\pgfqpoint{11.350070in}{5.164919in}}{\pgfqpoint{11.345679in}{5.154320in}}{\pgfqpoint{11.345679in}{5.143270in}}%
\pgfpathcurveto{\pgfqpoint{11.345679in}{5.132219in}}{\pgfqpoint{11.350070in}{5.121620in}}{\pgfqpoint{11.357883in}{5.113807in}}%
\pgfpathcurveto{\pgfqpoint{11.365697in}{5.105993in}}{\pgfqpoint{11.376296in}{5.101603in}}{\pgfqpoint{11.387346in}{5.101603in}}%
\pgfpathlineto{\pgfqpoint{11.387346in}{5.101603in}}%
\pgfpathclose%
\pgfusepath{stroke}%
\end{pgfscope}%
\begin{pgfscope}%
\pgfpathrectangle{\pgfqpoint{7.512535in}{0.437222in}}{\pgfqpoint{6.275590in}{5.159444in}}%
\pgfusepath{clip}%
\pgfsetbuttcap%
\pgfsetroundjoin%
\pgfsetlinewidth{1.003750pt}%
\definecolor{currentstroke}{rgb}{0.827451,0.827451,0.827451}%
\pgfsetstrokecolor{currentstroke}%
\pgfsetstrokeopacity{0.800000}%
\pgfsetdash{}{0pt}%
\pgfpathmoveto{\pgfqpoint{12.473540in}{5.514145in}}%
\pgfpathcurveto{\pgfqpoint{12.484590in}{5.514145in}}{\pgfqpoint{12.495189in}{5.518535in}}{\pgfqpoint{12.503003in}{5.526349in}}%
\pgfpathcurveto{\pgfqpoint{12.510817in}{5.534162in}}{\pgfqpoint{12.515207in}{5.544761in}}{\pgfqpoint{12.515207in}{5.555811in}}%
\pgfpathcurveto{\pgfqpoint{12.515207in}{5.566862in}}{\pgfqpoint{12.510817in}{5.577461in}}{\pgfqpoint{12.503003in}{5.585274in}}%
\pgfpathcurveto{\pgfqpoint{12.495189in}{5.593088in}}{\pgfqpoint{12.484590in}{5.597478in}}{\pgfqpoint{12.473540in}{5.597478in}}%
\pgfpathcurveto{\pgfqpoint{12.462490in}{5.597478in}}{\pgfqpoint{12.451891in}{5.593088in}}{\pgfqpoint{12.444078in}{5.585274in}}%
\pgfpathcurveto{\pgfqpoint{12.436264in}{5.577461in}}{\pgfqpoint{12.431874in}{5.566862in}}{\pgfqpoint{12.431874in}{5.555811in}}%
\pgfpathcurveto{\pgfqpoint{12.431874in}{5.544761in}}{\pgfqpoint{12.436264in}{5.534162in}}{\pgfqpoint{12.444078in}{5.526349in}}%
\pgfpathcurveto{\pgfqpoint{12.451891in}{5.518535in}}{\pgfqpoint{12.462490in}{5.514145in}}{\pgfqpoint{12.473540in}{5.514145in}}%
\pgfpathlineto{\pgfqpoint{12.473540in}{5.514145in}}%
\pgfpathclose%
\pgfusepath{stroke}%
\end{pgfscope}%
\begin{pgfscope}%
\pgfpathrectangle{\pgfqpoint{7.512535in}{0.437222in}}{\pgfqpoint{6.275590in}{5.159444in}}%
\pgfusepath{clip}%
\pgfsetbuttcap%
\pgfsetroundjoin%
\pgfsetlinewidth{1.003750pt}%
\definecolor{currentstroke}{rgb}{0.827451,0.827451,0.827451}%
\pgfsetstrokecolor{currentstroke}%
\pgfsetstrokeopacity{0.800000}%
\pgfsetdash{}{0pt}%
\pgfpathmoveto{\pgfqpoint{10.346895in}{5.462154in}}%
\pgfpathcurveto{\pgfqpoint{10.357945in}{5.462154in}}{\pgfqpoint{10.368544in}{5.466545in}}{\pgfqpoint{10.376358in}{5.474358in}}%
\pgfpathcurveto{\pgfqpoint{10.384171in}{5.482172in}}{\pgfqpoint{10.388562in}{5.492771in}}{\pgfqpoint{10.388562in}{5.503821in}}%
\pgfpathcurveto{\pgfqpoint{10.388562in}{5.514871in}}{\pgfqpoint{10.384171in}{5.525470in}}{\pgfqpoint{10.376358in}{5.533284in}}%
\pgfpathcurveto{\pgfqpoint{10.368544in}{5.541098in}}{\pgfqpoint{10.357945in}{5.545488in}}{\pgfqpoint{10.346895in}{5.545488in}}%
\pgfpathcurveto{\pgfqpoint{10.335845in}{5.545488in}}{\pgfqpoint{10.325246in}{5.541098in}}{\pgfqpoint{10.317432in}{5.533284in}}%
\pgfpathcurveto{\pgfqpoint{10.309619in}{5.525470in}}{\pgfqpoint{10.305228in}{5.514871in}}{\pgfqpoint{10.305228in}{5.503821in}}%
\pgfpathcurveto{\pgfqpoint{10.305228in}{5.492771in}}{\pgfqpoint{10.309619in}{5.482172in}}{\pgfqpoint{10.317432in}{5.474358in}}%
\pgfpathcurveto{\pgfqpoint{10.325246in}{5.466545in}}{\pgfqpoint{10.335845in}{5.462154in}}{\pgfqpoint{10.346895in}{5.462154in}}%
\pgfpathlineto{\pgfqpoint{10.346895in}{5.462154in}}%
\pgfpathclose%
\pgfusepath{stroke}%
\end{pgfscope}%
\begin{pgfscope}%
\pgfpathrectangle{\pgfqpoint{7.512535in}{0.437222in}}{\pgfqpoint{6.275590in}{5.159444in}}%
\pgfusepath{clip}%
\pgfsetbuttcap%
\pgfsetroundjoin%
\pgfsetlinewidth{1.003750pt}%
\definecolor{currentstroke}{rgb}{0.827451,0.827451,0.827451}%
\pgfsetstrokecolor{currentstroke}%
\pgfsetstrokeopacity{0.800000}%
\pgfsetdash{}{0pt}%
\pgfpathmoveto{\pgfqpoint{8.290348in}{1.011495in}}%
\pgfpathcurveto{\pgfqpoint{8.301398in}{1.011495in}}{\pgfqpoint{8.311997in}{1.015885in}}{\pgfqpoint{8.319811in}{1.023699in}}%
\pgfpathcurveto{\pgfqpoint{8.327625in}{1.031512in}}{\pgfqpoint{8.332015in}{1.042111in}}{\pgfqpoint{8.332015in}{1.053161in}}%
\pgfpathcurveto{\pgfqpoint{8.332015in}{1.064211in}}{\pgfqpoint{8.327625in}{1.074810in}}{\pgfqpoint{8.319811in}{1.082624in}}%
\pgfpathcurveto{\pgfqpoint{8.311997in}{1.090438in}}{\pgfqpoint{8.301398in}{1.094828in}}{\pgfqpoint{8.290348in}{1.094828in}}%
\pgfpathcurveto{\pgfqpoint{8.279298in}{1.094828in}}{\pgfqpoint{8.268699in}{1.090438in}}{\pgfqpoint{8.260886in}{1.082624in}}%
\pgfpathcurveto{\pgfqpoint{8.253072in}{1.074810in}}{\pgfqpoint{8.248682in}{1.064211in}}{\pgfqpoint{8.248682in}{1.053161in}}%
\pgfpathcurveto{\pgfqpoint{8.248682in}{1.042111in}}{\pgfqpoint{8.253072in}{1.031512in}}{\pgfqpoint{8.260886in}{1.023699in}}%
\pgfpathcurveto{\pgfqpoint{8.268699in}{1.015885in}}{\pgfqpoint{8.279298in}{1.011495in}}{\pgfqpoint{8.290348in}{1.011495in}}%
\pgfpathlineto{\pgfqpoint{8.290348in}{1.011495in}}%
\pgfpathclose%
\pgfusepath{stroke}%
\end{pgfscope}%
\begin{pgfscope}%
\pgfpathrectangle{\pgfqpoint{7.512535in}{0.437222in}}{\pgfqpoint{6.275590in}{5.159444in}}%
\pgfusepath{clip}%
\pgfsetbuttcap%
\pgfsetroundjoin%
\pgfsetlinewidth{1.003750pt}%
\definecolor{currentstroke}{rgb}{0.827451,0.827451,0.827451}%
\pgfsetstrokecolor{currentstroke}%
\pgfsetstrokeopacity{0.800000}%
\pgfsetdash{}{0pt}%
\pgfpathmoveto{\pgfqpoint{10.127909in}{3.160252in}}%
\pgfpathcurveto{\pgfqpoint{10.138959in}{3.160252in}}{\pgfqpoint{10.149558in}{3.164643in}}{\pgfqpoint{10.157372in}{3.172456in}}%
\pgfpathcurveto{\pgfqpoint{10.165186in}{3.180270in}}{\pgfqpoint{10.169576in}{3.190869in}}{\pgfqpoint{10.169576in}{3.201919in}}%
\pgfpathcurveto{\pgfqpoint{10.169576in}{3.212969in}}{\pgfqpoint{10.165186in}{3.223568in}}{\pgfqpoint{10.157372in}{3.231382in}}%
\pgfpathcurveto{\pgfqpoint{10.149558in}{3.239196in}}{\pgfqpoint{10.138959in}{3.243586in}}{\pgfqpoint{10.127909in}{3.243586in}}%
\pgfpathcurveto{\pgfqpoint{10.116859in}{3.243586in}}{\pgfqpoint{10.106260in}{3.239196in}}{\pgfqpoint{10.098446in}{3.231382in}}%
\pgfpathcurveto{\pgfqpoint{10.090633in}{3.223568in}}{\pgfqpoint{10.086243in}{3.212969in}}{\pgfqpoint{10.086243in}{3.201919in}}%
\pgfpathcurveto{\pgfqpoint{10.086243in}{3.190869in}}{\pgfqpoint{10.090633in}{3.180270in}}{\pgfqpoint{10.098446in}{3.172456in}}%
\pgfpathcurveto{\pgfqpoint{10.106260in}{3.164643in}}{\pgfqpoint{10.116859in}{3.160252in}}{\pgfqpoint{10.127909in}{3.160252in}}%
\pgfpathlineto{\pgfqpoint{10.127909in}{3.160252in}}%
\pgfpathclose%
\pgfusepath{stroke}%
\end{pgfscope}%
\begin{pgfscope}%
\pgfpathrectangle{\pgfqpoint{7.512535in}{0.437222in}}{\pgfqpoint{6.275590in}{5.159444in}}%
\pgfusepath{clip}%
\pgfsetbuttcap%
\pgfsetroundjoin%
\pgfsetlinewidth{1.003750pt}%
\definecolor{currentstroke}{rgb}{0.827451,0.827451,0.827451}%
\pgfsetstrokecolor{currentstroke}%
\pgfsetstrokeopacity{0.800000}%
\pgfsetdash{}{0pt}%
\pgfpathmoveto{\pgfqpoint{10.254297in}{3.553904in}}%
\pgfpathcurveto{\pgfqpoint{10.265347in}{3.553904in}}{\pgfqpoint{10.275946in}{3.558295in}}{\pgfqpoint{10.283760in}{3.566108in}}%
\pgfpathcurveto{\pgfqpoint{10.291574in}{3.573922in}}{\pgfqpoint{10.295964in}{3.584521in}}{\pgfqpoint{10.295964in}{3.595571in}}%
\pgfpathcurveto{\pgfqpoint{10.295964in}{3.606621in}}{\pgfqpoint{10.291574in}{3.617220in}}{\pgfqpoint{10.283760in}{3.625034in}}%
\pgfpathcurveto{\pgfqpoint{10.275946in}{3.632847in}}{\pgfqpoint{10.265347in}{3.637238in}}{\pgfqpoint{10.254297in}{3.637238in}}%
\pgfpathcurveto{\pgfqpoint{10.243247in}{3.637238in}}{\pgfqpoint{10.232648in}{3.632847in}}{\pgfqpoint{10.224834in}{3.625034in}}%
\pgfpathcurveto{\pgfqpoint{10.217021in}{3.617220in}}{\pgfqpoint{10.212631in}{3.606621in}}{\pgfqpoint{10.212631in}{3.595571in}}%
\pgfpathcurveto{\pgfqpoint{10.212631in}{3.584521in}}{\pgfqpoint{10.217021in}{3.573922in}}{\pgfqpoint{10.224834in}{3.566108in}}%
\pgfpathcurveto{\pgfqpoint{10.232648in}{3.558295in}}{\pgfqpoint{10.243247in}{3.553904in}}{\pgfqpoint{10.254297in}{3.553904in}}%
\pgfpathlineto{\pgfqpoint{10.254297in}{3.553904in}}%
\pgfpathclose%
\pgfusepath{stroke}%
\end{pgfscope}%
\begin{pgfscope}%
\pgfpathrectangle{\pgfqpoint{7.512535in}{0.437222in}}{\pgfqpoint{6.275590in}{5.159444in}}%
\pgfusepath{clip}%
\pgfsetbuttcap%
\pgfsetroundjoin%
\pgfsetlinewidth{1.003750pt}%
\definecolor{currentstroke}{rgb}{0.827451,0.827451,0.827451}%
\pgfsetstrokecolor{currentstroke}%
\pgfsetstrokeopacity{0.800000}%
\pgfsetdash{}{0pt}%
\pgfpathmoveto{\pgfqpoint{8.470019in}{3.328233in}}%
\pgfpathcurveto{\pgfqpoint{8.481069in}{3.328233in}}{\pgfqpoint{8.491668in}{3.332624in}}{\pgfqpoint{8.499482in}{3.340437in}}%
\pgfpathcurveto{\pgfqpoint{8.507295in}{3.348251in}}{\pgfqpoint{8.511686in}{3.358850in}}{\pgfqpoint{8.511686in}{3.369900in}}%
\pgfpathcurveto{\pgfqpoint{8.511686in}{3.380950in}}{\pgfqpoint{8.507295in}{3.391549in}}{\pgfqpoint{8.499482in}{3.399363in}}%
\pgfpathcurveto{\pgfqpoint{8.491668in}{3.407176in}}{\pgfqpoint{8.481069in}{3.411567in}}{\pgfqpoint{8.470019in}{3.411567in}}%
\pgfpathcurveto{\pgfqpoint{8.458969in}{3.411567in}}{\pgfqpoint{8.448370in}{3.407176in}}{\pgfqpoint{8.440556in}{3.399363in}}%
\pgfpathcurveto{\pgfqpoint{8.432743in}{3.391549in}}{\pgfqpoint{8.428352in}{3.380950in}}{\pgfqpoint{8.428352in}{3.369900in}}%
\pgfpathcurveto{\pgfqpoint{8.428352in}{3.358850in}}{\pgfqpoint{8.432743in}{3.348251in}}{\pgfqpoint{8.440556in}{3.340437in}}%
\pgfpathcurveto{\pgfqpoint{8.448370in}{3.332624in}}{\pgfqpoint{8.458969in}{3.328233in}}{\pgfqpoint{8.470019in}{3.328233in}}%
\pgfpathlineto{\pgfqpoint{8.470019in}{3.328233in}}%
\pgfpathclose%
\pgfusepath{stroke}%
\end{pgfscope}%
\begin{pgfscope}%
\pgfpathrectangle{\pgfqpoint{7.512535in}{0.437222in}}{\pgfqpoint{6.275590in}{5.159444in}}%
\pgfusepath{clip}%
\pgfsetbuttcap%
\pgfsetroundjoin%
\pgfsetlinewidth{1.003750pt}%
\definecolor{currentstroke}{rgb}{0.827451,0.827451,0.827451}%
\pgfsetstrokecolor{currentstroke}%
\pgfsetstrokeopacity{0.800000}%
\pgfsetdash{}{0pt}%
\pgfpathmoveto{\pgfqpoint{10.827454in}{5.304399in}}%
\pgfpathcurveto{\pgfqpoint{10.838504in}{5.304399in}}{\pgfqpoint{10.849103in}{5.308790in}}{\pgfqpoint{10.856917in}{5.316603in}}%
\pgfpathcurveto{\pgfqpoint{10.864730in}{5.324417in}}{\pgfqpoint{10.869121in}{5.335016in}}{\pgfqpoint{10.869121in}{5.346066in}}%
\pgfpathcurveto{\pgfqpoint{10.869121in}{5.357116in}}{\pgfqpoint{10.864730in}{5.367715in}}{\pgfqpoint{10.856917in}{5.375529in}}%
\pgfpathcurveto{\pgfqpoint{10.849103in}{5.383342in}}{\pgfqpoint{10.838504in}{5.387733in}}{\pgfqpoint{10.827454in}{5.387733in}}%
\pgfpathcurveto{\pgfqpoint{10.816404in}{5.387733in}}{\pgfqpoint{10.805805in}{5.383342in}}{\pgfqpoint{10.797991in}{5.375529in}}%
\pgfpathcurveto{\pgfqpoint{10.790177in}{5.367715in}}{\pgfqpoint{10.785787in}{5.357116in}}{\pgfqpoint{10.785787in}{5.346066in}}%
\pgfpathcurveto{\pgfqpoint{10.785787in}{5.335016in}}{\pgfqpoint{10.790177in}{5.324417in}}{\pgfqpoint{10.797991in}{5.316603in}}%
\pgfpathcurveto{\pgfqpoint{10.805805in}{5.308790in}}{\pgfqpoint{10.816404in}{5.304399in}}{\pgfqpoint{10.827454in}{5.304399in}}%
\pgfpathlineto{\pgfqpoint{10.827454in}{5.304399in}}%
\pgfpathclose%
\pgfusepath{stroke}%
\end{pgfscope}%
\begin{pgfscope}%
\pgfpathrectangle{\pgfqpoint{7.512535in}{0.437222in}}{\pgfqpoint{6.275590in}{5.159444in}}%
\pgfusepath{clip}%
\pgfsetbuttcap%
\pgfsetroundjoin%
\pgfsetlinewidth{1.003750pt}%
\definecolor{currentstroke}{rgb}{0.827451,0.827451,0.827451}%
\pgfsetstrokecolor{currentstroke}%
\pgfsetstrokeopacity{0.800000}%
\pgfsetdash{}{0pt}%
\pgfpathmoveto{\pgfqpoint{10.881762in}{4.622183in}}%
\pgfpathcurveto{\pgfqpoint{10.892812in}{4.622183in}}{\pgfqpoint{10.903411in}{4.626573in}}{\pgfqpoint{10.911225in}{4.634387in}}%
\pgfpathcurveto{\pgfqpoint{10.919039in}{4.642201in}}{\pgfqpoint{10.923429in}{4.652800in}}{\pgfqpoint{10.923429in}{4.663850in}}%
\pgfpathcurveto{\pgfqpoint{10.923429in}{4.674900in}}{\pgfqpoint{10.919039in}{4.685499in}}{\pgfqpoint{10.911225in}{4.693313in}}%
\pgfpathcurveto{\pgfqpoint{10.903411in}{4.701126in}}{\pgfqpoint{10.892812in}{4.705516in}}{\pgfqpoint{10.881762in}{4.705516in}}%
\pgfpathcurveto{\pgfqpoint{10.870712in}{4.705516in}}{\pgfqpoint{10.860113in}{4.701126in}}{\pgfqpoint{10.852300in}{4.693313in}}%
\pgfpathcurveto{\pgfqpoint{10.844486in}{4.685499in}}{\pgfqpoint{10.840096in}{4.674900in}}{\pgfqpoint{10.840096in}{4.663850in}}%
\pgfpathcurveto{\pgfqpoint{10.840096in}{4.652800in}}{\pgfqpoint{10.844486in}{4.642201in}}{\pgfqpoint{10.852300in}{4.634387in}}%
\pgfpathcurveto{\pgfqpoint{10.860113in}{4.626573in}}{\pgfqpoint{10.870712in}{4.622183in}}{\pgfqpoint{10.881762in}{4.622183in}}%
\pgfpathlineto{\pgfqpoint{10.881762in}{4.622183in}}%
\pgfpathclose%
\pgfusepath{stroke}%
\end{pgfscope}%
\begin{pgfscope}%
\pgfpathrectangle{\pgfqpoint{7.512535in}{0.437222in}}{\pgfqpoint{6.275590in}{5.159444in}}%
\pgfusepath{clip}%
\pgfsetbuttcap%
\pgfsetroundjoin%
\pgfsetlinewidth{1.003750pt}%
\definecolor{currentstroke}{rgb}{0.827451,0.827451,0.827451}%
\pgfsetstrokecolor{currentstroke}%
\pgfsetstrokeopacity{0.800000}%
\pgfsetdash{}{0pt}%
\pgfpathmoveto{\pgfqpoint{9.556541in}{1.394520in}}%
\pgfpathcurveto{\pgfqpoint{9.567591in}{1.394520in}}{\pgfqpoint{9.578190in}{1.398911in}}{\pgfqpoint{9.586004in}{1.406724in}}%
\pgfpathcurveto{\pgfqpoint{9.593817in}{1.414538in}}{\pgfqpoint{9.598208in}{1.425137in}}{\pgfqpoint{9.598208in}{1.436187in}}%
\pgfpathcurveto{\pgfqpoint{9.598208in}{1.447237in}}{\pgfqpoint{9.593817in}{1.457836in}}{\pgfqpoint{9.586004in}{1.465650in}}%
\pgfpathcurveto{\pgfqpoint{9.578190in}{1.473464in}}{\pgfqpoint{9.567591in}{1.477854in}}{\pgfqpoint{9.556541in}{1.477854in}}%
\pgfpathcurveto{\pgfqpoint{9.545491in}{1.477854in}}{\pgfqpoint{9.534892in}{1.473464in}}{\pgfqpoint{9.527078in}{1.465650in}}%
\pgfpathcurveto{\pgfqpoint{9.519265in}{1.457836in}}{\pgfqpoint{9.514874in}{1.447237in}}{\pgfqpoint{9.514874in}{1.436187in}}%
\pgfpathcurveto{\pgfqpoint{9.514874in}{1.425137in}}{\pgfqpoint{9.519265in}{1.414538in}}{\pgfqpoint{9.527078in}{1.406724in}}%
\pgfpathcurveto{\pgfqpoint{9.534892in}{1.398911in}}{\pgfqpoint{9.545491in}{1.394520in}}{\pgfqpoint{9.556541in}{1.394520in}}%
\pgfpathlineto{\pgfqpoint{9.556541in}{1.394520in}}%
\pgfpathclose%
\pgfusepath{stroke}%
\end{pgfscope}%
\begin{pgfscope}%
\pgfpathrectangle{\pgfqpoint{7.512535in}{0.437222in}}{\pgfqpoint{6.275590in}{5.159444in}}%
\pgfusepath{clip}%
\pgfsetbuttcap%
\pgfsetroundjoin%
\pgfsetlinewidth{1.003750pt}%
\definecolor{currentstroke}{rgb}{0.827451,0.827451,0.827451}%
\pgfsetstrokecolor{currentstroke}%
\pgfsetstrokeopacity{0.800000}%
\pgfsetdash{}{0pt}%
\pgfpathmoveto{\pgfqpoint{10.385573in}{5.144321in}}%
\pgfpathcurveto{\pgfqpoint{10.396623in}{5.144321in}}{\pgfqpoint{10.407222in}{5.148712in}}{\pgfqpoint{10.415036in}{5.156525in}}%
\pgfpathcurveto{\pgfqpoint{10.422849in}{5.164339in}}{\pgfqpoint{10.427240in}{5.174938in}}{\pgfqpoint{10.427240in}{5.185988in}}%
\pgfpathcurveto{\pgfqpoint{10.427240in}{5.197038in}}{\pgfqpoint{10.422849in}{5.207637in}}{\pgfqpoint{10.415036in}{5.215451in}}%
\pgfpathcurveto{\pgfqpoint{10.407222in}{5.223265in}}{\pgfqpoint{10.396623in}{5.227655in}}{\pgfqpoint{10.385573in}{5.227655in}}%
\pgfpathcurveto{\pgfqpoint{10.374523in}{5.227655in}}{\pgfqpoint{10.363924in}{5.223265in}}{\pgfqpoint{10.356110in}{5.215451in}}%
\pgfpathcurveto{\pgfqpoint{10.348297in}{5.207637in}}{\pgfqpoint{10.343906in}{5.197038in}}{\pgfqpoint{10.343906in}{5.185988in}}%
\pgfpathcurveto{\pgfqpoint{10.343906in}{5.174938in}}{\pgfqpoint{10.348297in}{5.164339in}}{\pgfqpoint{10.356110in}{5.156525in}}%
\pgfpathcurveto{\pgfqpoint{10.363924in}{5.148712in}}{\pgfqpoint{10.374523in}{5.144321in}}{\pgfqpoint{10.385573in}{5.144321in}}%
\pgfpathlineto{\pgfqpoint{10.385573in}{5.144321in}}%
\pgfpathclose%
\pgfusepath{stroke}%
\end{pgfscope}%
\begin{pgfscope}%
\pgfpathrectangle{\pgfqpoint{7.512535in}{0.437222in}}{\pgfqpoint{6.275590in}{5.159444in}}%
\pgfusepath{clip}%
\pgfsetbuttcap%
\pgfsetroundjoin%
\pgfsetlinewidth{1.003750pt}%
\definecolor{currentstroke}{rgb}{0.827451,0.827451,0.827451}%
\pgfsetstrokecolor{currentstroke}%
\pgfsetstrokeopacity{0.800000}%
\pgfsetdash{}{0pt}%
\pgfpathmoveto{\pgfqpoint{9.696495in}{1.337612in}}%
\pgfpathcurveto{\pgfqpoint{9.707545in}{1.337612in}}{\pgfqpoint{9.718144in}{1.342002in}}{\pgfqpoint{9.725958in}{1.349816in}}%
\pgfpathcurveto{\pgfqpoint{9.733772in}{1.357629in}}{\pgfqpoint{9.738162in}{1.368229in}}{\pgfqpoint{9.738162in}{1.379279in}}%
\pgfpathcurveto{\pgfqpoint{9.738162in}{1.390329in}}{\pgfqpoint{9.733772in}{1.400928in}}{\pgfqpoint{9.725958in}{1.408741in}}%
\pgfpathcurveto{\pgfqpoint{9.718144in}{1.416555in}}{\pgfqpoint{9.707545in}{1.420945in}}{\pgfqpoint{9.696495in}{1.420945in}}%
\pgfpathcurveto{\pgfqpoint{9.685445in}{1.420945in}}{\pgfqpoint{9.674846in}{1.416555in}}{\pgfqpoint{9.667033in}{1.408741in}}%
\pgfpathcurveto{\pgfqpoint{9.659219in}{1.400928in}}{\pgfqpoint{9.654829in}{1.390329in}}{\pgfqpoint{9.654829in}{1.379279in}}%
\pgfpathcurveto{\pgfqpoint{9.654829in}{1.368229in}}{\pgfqpoint{9.659219in}{1.357629in}}{\pgfqpoint{9.667033in}{1.349816in}}%
\pgfpathcurveto{\pgfqpoint{9.674846in}{1.342002in}}{\pgfqpoint{9.685445in}{1.337612in}}{\pgfqpoint{9.696495in}{1.337612in}}%
\pgfpathlineto{\pgfqpoint{9.696495in}{1.337612in}}%
\pgfpathclose%
\pgfusepath{stroke}%
\end{pgfscope}%
\begin{pgfscope}%
\pgfpathrectangle{\pgfqpoint{7.512535in}{0.437222in}}{\pgfqpoint{6.275590in}{5.159444in}}%
\pgfusepath{clip}%
\pgfsetbuttcap%
\pgfsetroundjoin%
\pgfsetlinewidth{1.003750pt}%
\definecolor{currentstroke}{rgb}{0.827451,0.827451,0.827451}%
\pgfsetstrokecolor{currentstroke}%
\pgfsetstrokeopacity{0.800000}%
\pgfsetdash{}{0pt}%
\pgfpathmoveto{\pgfqpoint{12.072745in}{5.039509in}}%
\pgfpathcurveto{\pgfqpoint{12.083796in}{5.039509in}}{\pgfqpoint{12.094395in}{5.043899in}}{\pgfqpoint{12.102208in}{5.051713in}}%
\pgfpathcurveto{\pgfqpoint{12.110022in}{5.059526in}}{\pgfqpoint{12.114412in}{5.070125in}}{\pgfqpoint{12.114412in}{5.081175in}}%
\pgfpathcurveto{\pgfqpoint{12.114412in}{5.092226in}}{\pgfqpoint{12.110022in}{5.102825in}}{\pgfqpoint{12.102208in}{5.110638in}}%
\pgfpathcurveto{\pgfqpoint{12.094395in}{5.118452in}}{\pgfqpoint{12.083796in}{5.122842in}}{\pgfqpoint{12.072745in}{5.122842in}}%
\pgfpathcurveto{\pgfqpoint{12.061695in}{5.122842in}}{\pgfqpoint{12.051096in}{5.118452in}}{\pgfqpoint{12.043283in}{5.110638in}}%
\pgfpathcurveto{\pgfqpoint{12.035469in}{5.102825in}}{\pgfqpoint{12.031079in}{5.092226in}}{\pgfqpoint{12.031079in}{5.081175in}}%
\pgfpathcurveto{\pgfqpoint{12.031079in}{5.070125in}}{\pgfqpoint{12.035469in}{5.059526in}}{\pgfqpoint{12.043283in}{5.051713in}}%
\pgfpathcurveto{\pgfqpoint{12.051096in}{5.043899in}}{\pgfqpoint{12.061695in}{5.039509in}}{\pgfqpoint{12.072745in}{5.039509in}}%
\pgfpathlineto{\pgfqpoint{12.072745in}{5.039509in}}%
\pgfpathclose%
\pgfusepath{stroke}%
\end{pgfscope}%
\begin{pgfscope}%
\pgfpathrectangle{\pgfqpoint{7.512535in}{0.437222in}}{\pgfqpoint{6.275590in}{5.159444in}}%
\pgfusepath{clip}%
\pgfsetbuttcap%
\pgfsetroundjoin%
\pgfsetlinewidth{1.003750pt}%
\definecolor{currentstroke}{rgb}{0.827451,0.827451,0.827451}%
\pgfsetstrokecolor{currentstroke}%
\pgfsetstrokeopacity{0.800000}%
\pgfsetdash{}{0pt}%
\pgfpathmoveto{\pgfqpoint{9.615954in}{3.788157in}}%
\pgfpathcurveto{\pgfqpoint{9.627004in}{3.788157in}}{\pgfqpoint{9.637603in}{3.792548in}}{\pgfqpoint{9.645416in}{3.800361in}}%
\pgfpathcurveto{\pgfqpoint{9.653230in}{3.808175in}}{\pgfqpoint{9.657620in}{3.818774in}}{\pgfqpoint{9.657620in}{3.829824in}}%
\pgfpathcurveto{\pgfqpoint{9.657620in}{3.840874in}}{\pgfqpoint{9.653230in}{3.851473in}}{\pgfqpoint{9.645416in}{3.859287in}}%
\pgfpathcurveto{\pgfqpoint{9.637603in}{3.867100in}}{\pgfqpoint{9.627004in}{3.871491in}}{\pgfqpoint{9.615954in}{3.871491in}}%
\pgfpathcurveto{\pgfqpoint{9.604903in}{3.871491in}}{\pgfqpoint{9.594304in}{3.867100in}}{\pgfqpoint{9.586491in}{3.859287in}}%
\pgfpathcurveto{\pgfqpoint{9.578677in}{3.851473in}}{\pgfqpoint{9.574287in}{3.840874in}}{\pgfqpoint{9.574287in}{3.829824in}}%
\pgfpathcurveto{\pgfqpoint{9.574287in}{3.818774in}}{\pgfqpoint{9.578677in}{3.808175in}}{\pgfqpoint{9.586491in}{3.800361in}}%
\pgfpathcurveto{\pgfqpoint{9.594304in}{3.792548in}}{\pgfqpoint{9.604903in}{3.788157in}}{\pgfqpoint{9.615954in}{3.788157in}}%
\pgfpathlineto{\pgfqpoint{9.615954in}{3.788157in}}%
\pgfpathclose%
\pgfusepath{stroke}%
\end{pgfscope}%
\begin{pgfscope}%
\pgfpathrectangle{\pgfqpoint{7.512535in}{0.437222in}}{\pgfqpoint{6.275590in}{5.159444in}}%
\pgfusepath{clip}%
\pgfsetbuttcap%
\pgfsetroundjoin%
\pgfsetlinewidth{1.003750pt}%
\definecolor{currentstroke}{rgb}{0.827451,0.827451,0.827451}%
\pgfsetstrokecolor{currentstroke}%
\pgfsetstrokeopacity{0.800000}%
\pgfsetdash{}{0pt}%
\pgfpathmoveto{\pgfqpoint{7.972802in}{0.679358in}}%
\pgfpathcurveto{\pgfqpoint{7.983853in}{0.679358in}}{\pgfqpoint{7.994452in}{0.683748in}}{\pgfqpoint{8.002265in}{0.691562in}}%
\pgfpathcurveto{\pgfqpoint{8.010079in}{0.699376in}}{\pgfqpoint{8.014469in}{0.709975in}}{\pgfqpoint{8.014469in}{0.721025in}}%
\pgfpathcurveto{\pgfqpoint{8.014469in}{0.732075in}}{\pgfqpoint{8.010079in}{0.742674in}}{\pgfqpoint{8.002265in}{0.750487in}}%
\pgfpathcurveto{\pgfqpoint{7.994452in}{0.758301in}}{\pgfqpoint{7.983853in}{0.762691in}}{\pgfqpoint{7.972802in}{0.762691in}}%
\pgfpathcurveto{\pgfqpoint{7.961752in}{0.762691in}}{\pgfqpoint{7.951153in}{0.758301in}}{\pgfqpoint{7.943340in}{0.750487in}}%
\pgfpathcurveto{\pgfqpoint{7.935526in}{0.742674in}}{\pgfqpoint{7.931136in}{0.732075in}}{\pgfqpoint{7.931136in}{0.721025in}}%
\pgfpathcurveto{\pgfqpoint{7.931136in}{0.709975in}}{\pgfqpoint{7.935526in}{0.699376in}}{\pgfqpoint{7.943340in}{0.691562in}}%
\pgfpathcurveto{\pgfqpoint{7.951153in}{0.683748in}}{\pgfqpoint{7.961752in}{0.679358in}}{\pgfqpoint{7.972802in}{0.679358in}}%
\pgfpathlineto{\pgfqpoint{7.972802in}{0.679358in}}%
\pgfpathclose%
\pgfusepath{stroke}%
\end{pgfscope}%
\begin{pgfscope}%
\pgfpathrectangle{\pgfqpoint{7.512535in}{0.437222in}}{\pgfqpoint{6.275590in}{5.159444in}}%
\pgfusepath{clip}%
\pgfsetbuttcap%
\pgfsetroundjoin%
\pgfsetlinewidth{1.003750pt}%
\definecolor{currentstroke}{rgb}{0.827451,0.827451,0.827451}%
\pgfsetstrokecolor{currentstroke}%
\pgfsetstrokeopacity{0.800000}%
\pgfsetdash{}{0pt}%
\pgfpathmoveto{\pgfqpoint{9.540773in}{3.442114in}}%
\pgfpathcurveto{\pgfqpoint{9.551823in}{3.442114in}}{\pgfqpoint{9.562422in}{3.446504in}}{\pgfqpoint{9.570236in}{3.454318in}}%
\pgfpathcurveto{\pgfqpoint{9.578049in}{3.462131in}}{\pgfqpoint{9.582440in}{3.472730in}}{\pgfqpoint{9.582440in}{3.483780in}}%
\pgfpathcurveto{\pgfqpoint{9.582440in}{3.494830in}}{\pgfqpoint{9.578049in}{3.505429in}}{\pgfqpoint{9.570236in}{3.513243in}}%
\pgfpathcurveto{\pgfqpoint{9.562422in}{3.521057in}}{\pgfqpoint{9.551823in}{3.525447in}}{\pgfqpoint{9.540773in}{3.525447in}}%
\pgfpathcurveto{\pgfqpoint{9.529723in}{3.525447in}}{\pgfqpoint{9.519124in}{3.521057in}}{\pgfqpoint{9.511310in}{3.513243in}}%
\pgfpathcurveto{\pgfqpoint{9.503497in}{3.505429in}}{\pgfqpoint{9.499106in}{3.494830in}}{\pgfqpoint{9.499106in}{3.483780in}}%
\pgfpathcurveto{\pgfqpoint{9.499106in}{3.472730in}}{\pgfqpoint{9.503497in}{3.462131in}}{\pgfqpoint{9.511310in}{3.454318in}}%
\pgfpathcurveto{\pgfqpoint{9.519124in}{3.446504in}}{\pgfqpoint{9.529723in}{3.442114in}}{\pgfqpoint{9.540773in}{3.442114in}}%
\pgfpathlineto{\pgfqpoint{9.540773in}{3.442114in}}%
\pgfpathclose%
\pgfusepath{stroke}%
\end{pgfscope}%
\begin{pgfscope}%
\pgfpathrectangle{\pgfqpoint{7.512535in}{0.437222in}}{\pgfqpoint{6.275590in}{5.159444in}}%
\pgfusepath{clip}%
\pgfsetbuttcap%
\pgfsetroundjoin%
\pgfsetlinewidth{1.003750pt}%
\definecolor{currentstroke}{rgb}{0.827451,0.827451,0.827451}%
\pgfsetstrokecolor{currentstroke}%
\pgfsetstrokeopacity{0.800000}%
\pgfsetdash{}{0pt}%
\pgfpathmoveto{\pgfqpoint{9.505328in}{3.327623in}}%
\pgfpathcurveto{\pgfqpoint{9.516379in}{3.327623in}}{\pgfqpoint{9.526978in}{3.332014in}}{\pgfqpoint{9.534791in}{3.339827in}}%
\pgfpathcurveto{\pgfqpoint{9.542605in}{3.347641in}}{\pgfqpoint{9.546995in}{3.358240in}}{\pgfqpoint{9.546995in}{3.369290in}}%
\pgfpathcurveto{\pgfqpoint{9.546995in}{3.380340in}}{\pgfqpoint{9.542605in}{3.390939in}}{\pgfqpoint{9.534791in}{3.398753in}}%
\pgfpathcurveto{\pgfqpoint{9.526978in}{3.406567in}}{\pgfqpoint{9.516379in}{3.410957in}}{\pgfqpoint{9.505328in}{3.410957in}}%
\pgfpathcurveto{\pgfqpoint{9.494278in}{3.410957in}}{\pgfqpoint{9.483679in}{3.406567in}}{\pgfqpoint{9.475866in}{3.398753in}}%
\pgfpathcurveto{\pgfqpoint{9.468052in}{3.390939in}}{\pgfqpoint{9.463662in}{3.380340in}}{\pgfqpoint{9.463662in}{3.369290in}}%
\pgfpathcurveto{\pgfqpoint{9.463662in}{3.358240in}}{\pgfqpoint{9.468052in}{3.347641in}}{\pgfqpoint{9.475866in}{3.339827in}}%
\pgfpathcurveto{\pgfqpoint{9.483679in}{3.332014in}}{\pgfqpoint{9.494278in}{3.327623in}}{\pgfqpoint{9.505328in}{3.327623in}}%
\pgfpathlineto{\pgfqpoint{9.505328in}{3.327623in}}%
\pgfpathclose%
\pgfusepath{stroke}%
\end{pgfscope}%
\begin{pgfscope}%
\pgfpathrectangle{\pgfqpoint{7.512535in}{0.437222in}}{\pgfqpoint{6.275590in}{5.159444in}}%
\pgfusepath{clip}%
\pgfsetbuttcap%
\pgfsetroundjoin%
\pgfsetlinewidth{1.003750pt}%
\definecolor{currentstroke}{rgb}{0.827451,0.827451,0.827451}%
\pgfsetstrokecolor{currentstroke}%
\pgfsetstrokeopacity{0.800000}%
\pgfsetdash{}{0pt}%
\pgfpathmoveto{\pgfqpoint{8.044541in}{2.113663in}}%
\pgfpathcurveto{\pgfqpoint{8.055591in}{2.113663in}}{\pgfqpoint{8.066190in}{2.118054in}}{\pgfqpoint{8.074003in}{2.125867in}}%
\pgfpathcurveto{\pgfqpoint{8.081817in}{2.133681in}}{\pgfqpoint{8.086207in}{2.144280in}}{\pgfqpoint{8.086207in}{2.155330in}}%
\pgfpathcurveto{\pgfqpoint{8.086207in}{2.166380in}}{\pgfqpoint{8.081817in}{2.176979in}}{\pgfqpoint{8.074003in}{2.184793in}}%
\pgfpathcurveto{\pgfqpoint{8.066190in}{2.192606in}}{\pgfqpoint{8.055591in}{2.196997in}}{\pgfqpoint{8.044541in}{2.196997in}}%
\pgfpathcurveto{\pgfqpoint{8.033491in}{2.196997in}}{\pgfqpoint{8.022891in}{2.192606in}}{\pgfqpoint{8.015078in}{2.184793in}}%
\pgfpathcurveto{\pgfqpoint{8.007264in}{2.176979in}}{\pgfqpoint{8.002874in}{2.166380in}}{\pgfqpoint{8.002874in}{2.155330in}}%
\pgfpathcurveto{\pgfqpoint{8.002874in}{2.144280in}}{\pgfqpoint{8.007264in}{2.133681in}}{\pgfqpoint{8.015078in}{2.125867in}}%
\pgfpathcurveto{\pgfqpoint{8.022891in}{2.118054in}}{\pgfqpoint{8.033491in}{2.113663in}}{\pgfqpoint{8.044541in}{2.113663in}}%
\pgfpathlineto{\pgfqpoint{8.044541in}{2.113663in}}%
\pgfpathclose%
\pgfusepath{stroke}%
\end{pgfscope}%
\begin{pgfscope}%
\pgfpathrectangle{\pgfqpoint{7.512535in}{0.437222in}}{\pgfqpoint{6.275590in}{5.159444in}}%
\pgfusepath{clip}%
\pgfsetbuttcap%
\pgfsetroundjoin%
\pgfsetlinewidth{1.003750pt}%
\definecolor{currentstroke}{rgb}{0.827451,0.827451,0.827451}%
\pgfsetstrokecolor{currentstroke}%
\pgfsetstrokeopacity{0.800000}%
\pgfsetdash{}{0pt}%
\pgfpathmoveto{\pgfqpoint{12.292049in}{5.113368in}}%
\pgfpathcurveto{\pgfqpoint{12.303099in}{5.113368in}}{\pgfqpoint{12.313698in}{5.117758in}}{\pgfqpoint{12.321512in}{5.125572in}}%
\pgfpathcurveto{\pgfqpoint{12.329325in}{5.133385in}}{\pgfqpoint{12.333716in}{5.143984in}}{\pgfqpoint{12.333716in}{5.155034in}}%
\pgfpathcurveto{\pgfqpoint{12.333716in}{5.166084in}}{\pgfqpoint{12.329325in}{5.176683in}}{\pgfqpoint{12.321512in}{5.184497in}}%
\pgfpathcurveto{\pgfqpoint{12.313698in}{5.192311in}}{\pgfqpoint{12.303099in}{5.196701in}}{\pgfqpoint{12.292049in}{5.196701in}}%
\pgfpathcurveto{\pgfqpoint{12.280999in}{5.196701in}}{\pgfqpoint{12.270400in}{5.192311in}}{\pgfqpoint{12.262586in}{5.184497in}}%
\pgfpathcurveto{\pgfqpoint{12.254772in}{5.176683in}}{\pgfqpoint{12.250382in}{5.166084in}}{\pgfqpoint{12.250382in}{5.155034in}}%
\pgfpathcurveto{\pgfqpoint{12.250382in}{5.143984in}}{\pgfqpoint{12.254772in}{5.133385in}}{\pgfqpoint{12.262586in}{5.125572in}}%
\pgfpathcurveto{\pgfqpoint{12.270400in}{5.117758in}}{\pgfqpoint{12.280999in}{5.113368in}}{\pgfqpoint{12.292049in}{5.113368in}}%
\pgfpathlineto{\pgfqpoint{12.292049in}{5.113368in}}%
\pgfpathclose%
\pgfusepath{stroke}%
\end{pgfscope}%
\begin{pgfscope}%
\pgfpathrectangle{\pgfqpoint{7.512535in}{0.437222in}}{\pgfqpoint{6.275590in}{5.159444in}}%
\pgfusepath{clip}%
\pgfsetbuttcap%
\pgfsetroundjoin%
\pgfsetlinewidth{1.003750pt}%
\definecolor{currentstroke}{rgb}{0.827451,0.827451,0.827451}%
\pgfsetstrokecolor{currentstroke}%
\pgfsetstrokeopacity{0.800000}%
\pgfsetdash{}{0pt}%
\pgfpathmoveto{\pgfqpoint{13.101307in}{5.417061in}}%
\pgfpathcurveto{\pgfqpoint{13.112357in}{5.417061in}}{\pgfqpoint{13.122956in}{5.421451in}}{\pgfqpoint{13.130770in}{5.429265in}}%
\pgfpathcurveto{\pgfqpoint{13.138583in}{5.437079in}}{\pgfqpoint{13.142973in}{5.447678in}}{\pgfqpoint{13.142973in}{5.458728in}}%
\pgfpathcurveto{\pgfqpoint{13.142973in}{5.469778in}}{\pgfqpoint{13.138583in}{5.480377in}}{\pgfqpoint{13.130770in}{5.488191in}}%
\pgfpathcurveto{\pgfqpoint{13.122956in}{5.496004in}}{\pgfqpoint{13.112357in}{5.500395in}}{\pgfqpoint{13.101307in}{5.500395in}}%
\pgfpathcurveto{\pgfqpoint{13.090257in}{5.500395in}}{\pgfqpoint{13.079658in}{5.496004in}}{\pgfqpoint{13.071844in}{5.488191in}}%
\pgfpathcurveto{\pgfqpoint{13.064030in}{5.480377in}}{\pgfqpoint{13.059640in}{5.469778in}}{\pgfqpoint{13.059640in}{5.458728in}}%
\pgfpathcurveto{\pgfqpoint{13.059640in}{5.447678in}}{\pgfqpoint{13.064030in}{5.437079in}}{\pgfqpoint{13.071844in}{5.429265in}}%
\pgfpathcurveto{\pgfqpoint{13.079658in}{5.421451in}}{\pgfqpoint{13.090257in}{5.417061in}}{\pgfqpoint{13.101307in}{5.417061in}}%
\pgfpathlineto{\pgfqpoint{13.101307in}{5.417061in}}%
\pgfpathclose%
\pgfusepath{stroke}%
\end{pgfscope}%
\begin{pgfscope}%
\pgfpathrectangle{\pgfqpoint{7.512535in}{0.437222in}}{\pgfqpoint{6.275590in}{5.159444in}}%
\pgfusepath{clip}%
\pgfsetbuttcap%
\pgfsetroundjoin%
\pgfsetlinewidth{1.003750pt}%
\definecolor{currentstroke}{rgb}{0.827451,0.827451,0.827451}%
\pgfsetstrokecolor{currentstroke}%
\pgfsetstrokeopacity{0.800000}%
\pgfsetdash{}{0pt}%
\pgfpathmoveto{\pgfqpoint{9.807010in}{4.531380in}}%
\pgfpathcurveto{\pgfqpoint{9.818060in}{4.531380in}}{\pgfqpoint{9.828659in}{4.535770in}}{\pgfqpoint{9.836473in}{4.543584in}}%
\pgfpathcurveto{\pgfqpoint{9.844287in}{4.551397in}}{\pgfqpoint{9.848677in}{4.561996in}}{\pgfqpoint{9.848677in}{4.573046in}}%
\pgfpathcurveto{\pgfqpoint{9.848677in}{4.584096in}}{\pgfqpoint{9.844287in}{4.594695in}}{\pgfqpoint{9.836473in}{4.602509in}}%
\pgfpathcurveto{\pgfqpoint{9.828659in}{4.610323in}}{\pgfqpoint{9.818060in}{4.614713in}}{\pgfqpoint{9.807010in}{4.614713in}}%
\pgfpathcurveto{\pgfqpoint{9.795960in}{4.614713in}}{\pgfqpoint{9.785361in}{4.610323in}}{\pgfqpoint{9.777548in}{4.602509in}}%
\pgfpathcurveto{\pgfqpoint{9.769734in}{4.594695in}}{\pgfqpoint{9.765344in}{4.584096in}}{\pgfqpoint{9.765344in}{4.573046in}}%
\pgfpathcurveto{\pgfqpoint{9.765344in}{4.561996in}}{\pgfqpoint{9.769734in}{4.551397in}}{\pgfqpoint{9.777548in}{4.543584in}}%
\pgfpathcurveto{\pgfqpoint{9.785361in}{4.535770in}}{\pgfqpoint{9.795960in}{4.531380in}}{\pgfqpoint{9.807010in}{4.531380in}}%
\pgfpathlineto{\pgfqpoint{9.807010in}{4.531380in}}%
\pgfpathclose%
\pgfusepath{stroke}%
\end{pgfscope}%
\begin{pgfscope}%
\pgfpathrectangle{\pgfqpoint{7.512535in}{0.437222in}}{\pgfqpoint{6.275590in}{5.159444in}}%
\pgfusepath{clip}%
\pgfsetbuttcap%
\pgfsetroundjoin%
\pgfsetlinewidth{1.003750pt}%
\definecolor{currentstroke}{rgb}{0.827451,0.827451,0.827451}%
\pgfsetstrokecolor{currentstroke}%
\pgfsetstrokeopacity{0.800000}%
\pgfsetdash{}{0pt}%
\pgfpathmoveto{\pgfqpoint{9.792079in}{4.165674in}}%
\pgfpathcurveto{\pgfqpoint{9.803129in}{4.165674in}}{\pgfqpoint{9.813728in}{4.170064in}}{\pgfqpoint{9.821541in}{4.177878in}}%
\pgfpathcurveto{\pgfqpoint{9.829355in}{4.185692in}}{\pgfqpoint{9.833745in}{4.196291in}}{\pgfqpoint{9.833745in}{4.207341in}}%
\pgfpathcurveto{\pgfqpoint{9.833745in}{4.218391in}}{\pgfqpoint{9.829355in}{4.228990in}}{\pgfqpoint{9.821541in}{4.236803in}}%
\pgfpathcurveto{\pgfqpoint{9.813728in}{4.244617in}}{\pgfqpoint{9.803129in}{4.249007in}}{\pgfqpoint{9.792079in}{4.249007in}}%
\pgfpathcurveto{\pgfqpoint{9.781029in}{4.249007in}}{\pgfqpoint{9.770429in}{4.244617in}}{\pgfqpoint{9.762616in}{4.236803in}}%
\pgfpathcurveto{\pgfqpoint{9.754802in}{4.228990in}}{\pgfqpoint{9.750412in}{4.218391in}}{\pgfqpoint{9.750412in}{4.207341in}}%
\pgfpathcurveto{\pgfqpoint{9.750412in}{4.196291in}}{\pgfqpoint{9.754802in}{4.185692in}}{\pgfqpoint{9.762616in}{4.177878in}}%
\pgfpathcurveto{\pgfqpoint{9.770429in}{4.170064in}}{\pgfqpoint{9.781029in}{4.165674in}}{\pgfqpoint{9.792079in}{4.165674in}}%
\pgfpathlineto{\pgfqpoint{9.792079in}{4.165674in}}%
\pgfpathclose%
\pgfusepath{stroke}%
\end{pgfscope}%
\begin{pgfscope}%
\pgfpathrectangle{\pgfqpoint{7.512535in}{0.437222in}}{\pgfqpoint{6.275590in}{5.159444in}}%
\pgfusepath{clip}%
\pgfsetbuttcap%
\pgfsetroundjoin%
\pgfsetlinewidth{1.003750pt}%
\definecolor{currentstroke}{rgb}{0.827451,0.827451,0.827451}%
\pgfsetstrokecolor{currentstroke}%
\pgfsetstrokeopacity{0.800000}%
\pgfsetdash{}{0pt}%
\pgfpathmoveto{\pgfqpoint{8.309719in}{2.302744in}}%
\pgfpathcurveto{\pgfqpoint{8.320769in}{2.302744in}}{\pgfqpoint{8.331368in}{2.307134in}}{\pgfqpoint{8.339182in}{2.314947in}}%
\pgfpathcurveto{\pgfqpoint{8.346996in}{2.322761in}}{\pgfqpoint{8.351386in}{2.333360in}}{\pgfqpoint{8.351386in}{2.344410in}}%
\pgfpathcurveto{\pgfqpoint{8.351386in}{2.355460in}}{\pgfqpoint{8.346996in}{2.366059in}}{\pgfqpoint{8.339182in}{2.373873in}}%
\pgfpathcurveto{\pgfqpoint{8.331368in}{2.381687in}}{\pgfqpoint{8.320769in}{2.386077in}}{\pgfqpoint{8.309719in}{2.386077in}}%
\pgfpathcurveto{\pgfqpoint{8.298669in}{2.386077in}}{\pgfqpoint{8.288070in}{2.381687in}}{\pgfqpoint{8.280256in}{2.373873in}}%
\pgfpathcurveto{\pgfqpoint{8.272443in}{2.366059in}}{\pgfqpoint{8.268052in}{2.355460in}}{\pgfqpoint{8.268052in}{2.344410in}}%
\pgfpathcurveto{\pgfqpoint{8.268052in}{2.333360in}}{\pgfqpoint{8.272443in}{2.322761in}}{\pgfqpoint{8.280256in}{2.314947in}}%
\pgfpathcurveto{\pgfqpoint{8.288070in}{2.307134in}}{\pgfqpoint{8.298669in}{2.302744in}}{\pgfqpoint{8.309719in}{2.302744in}}%
\pgfpathlineto{\pgfqpoint{8.309719in}{2.302744in}}%
\pgfpathclose%
\pgfusepath{stroke}%
\end{pgfscope}%
\begin{pgfscope}%
\pgfpathrectangle{\pgfqpoint{7.512535in}{0.437222in}}{\pgfqpoint{6.275590in}{5.159444in}}%
\pgfusepath{clip}%
\pgfsetbuttcap%
\pgfsetroundjoin%
\pgfsetlinewidth{1.003750pt}%
\definecolor{currentstroke}{rgb}{0.827451,0.827451,0.827451}%
\pgfsetstrokecolor{currentstroke}%
\pgfsetstrokeopacity{0.800000}%
\pgfsetdash{}{0pt}%
\pgfpathmoveto{\pgfqpoint{9.581399in}{1.396341in}}%
\pgfpathcurveto{\pgfqpoint{9.592449in}{1.396341in}}{\pgfqpoint{9.603048in}{1.400731in}}{\pgfqpoint{9.610862in}{1.408545in}}%
\pgfpathcurveto{\pgfqpoint{9.618675in}{1.416359in}}{\pgfqpoint{9.623065in}{1.426958in}}{\pgfqpoint{9.623065in}{1.438008in}}%
\pgfpathcurveto{\pgfqpoint{9.623065in}{1.449058in}}{\pgfqpoint{9.618675in}{1.459657in}}{\pgfqpoint{9.610862in}{1.467470in}}%
\pgfpathcurveto{\pgfqpoint{9.603048in}{1.475284in}}{\pgfqpoint{9.592449in}{1.479674in}}{\pgfqpoint{9.581399in}{1.479674in}}%
\pgfpathcurveto{\pgfqpoint{9.570349in}{1.479674in}}{\pgfqpoint{9.559750in}{1.475284in}}{\pgfqpoint{9.551936in}{1.467470in}}%
\pgfpathcurveto{\pgfqpoint{9.544122in}{1.459657in}}{\pgfqpoint{9.539732in}{1.449058in}}{\pgfqpoint{9.539732in}{1.438008in}}%
\pgfpathcurveto{\pgfqpoint{9.539732in}{1.426958in}}{\pgfqpoint{9.544122in}{1.416359in}}{\pgfqpoint{9.551936in}{1.408545in}}%
\pgfpathcurveto{\pgfqpoint{9.559750in}{1.400731in}}{\pgfqpoint{9.570349in}{1.396341in}}{\pgfqpoint{9.581399in}{1.396341in}}%
\pgfpathlineto{\pgfqpoint{9.581399in}{1.396341in}}%
\pgfpathclose%
\pgfusepath{stroke}%
\end{pgfscope}%
\begin{pgfscope}%
\pgfpathrectangle{\pgfqpoint{7.512535in}{0.437222in}}{\pgfqpoint{6.275590in}{5.159444in}}%
\pgfusepath{clip}%
\pgfsetbuttcap%
\pgfsetroundjoin%
\pgfsetlinewidth{1.003750pt}%
\definecolor{currentstroke}{rgb}{0.827451,0.827451,0.827451}%
\pgfsetstrokecolor{currentstroke}%
\pgfsetstrokeopacity{0.800000}%
\pgfsetdash{}{0pt}%
\pgfpathmoveto{\pgfqpoint{13.101307in}{5.469388in}}%
\pgfpathcurveto{\pgfqpoint{13.112357in}{5.469388in}}{\pgfqpoint{13.122956in}{5.473778in}}{\pgfqpoint{13.130770in}{5.481592in}}%
\pgfpathcurveto{\pgfqpoint{13.138583in}{5.489405in}}{\pgfqpoint{13.142973in}{5.500004in}}{\pgfqpoint{13.142973in}{5.511054in}}%
\pgfpathcurveto{\pgfqpoint{13.142973in}{5.522105in}}{\pgfqpoint{13.138583in}{5.532704in}}{\pgfqpoint{13.130770in}{5.540517in}}%
\pgfpathcurveto{\pgfqpoint{13.122956in}{5.548331in}}{\pgfqpoint{13.112357in}{5.552721in}}{\pgfqpoint{13.101307in}{5.552721in}}%
\pgfpathcurveto{\pgfqpoint{13.090257in}{5.552721in}}{\pgfqpoint{13.079658in}{5.548331in}}{\pgfqpoint{13.071844in}{5.540517in}}%
\pgfpathcurveto{\pgfqpoint{13.064030in}{5.532704in}}{\pgfqpoint{13.059640in}{5.522105in}}{\pgfqpoint{13.059640in}{5.511054in}}%
\pgfpathcurveto{\pgfqpoint{13.059640in}{5.500004in}}{\pgfqpoint{13.064030in}{5.489405in}}{\pgfqpoint{13.071844in}{5.481592in}}%
\pgfpathcurveto{\pgfqpoint{13.079658in}{5.473778in}}{\pgfqpoint{13.090257in}{5.469388in}}{\pgfqpoint{13.101307in}{5.469388in}}%
\pgfpathlineto{\pgfqpoint{13.101307in}{5.469388in}}%
\pgfpathclose%
\pgfusepath{stroke}%
\end{pgfscope}%
\begin{pgfscope}%
\pgfpathrectangle{\pgfqpoint{7.512535in}{0.437222in}}{\pgfqpoint{6.275590in}{5.159444in}}%
\pgfusepath{clip}%
\pgfsetbuttcap%
\pgfsetroundjoin%
\pgfsetlinewidth{1.003750pt}%
\definecolor{currentstroke}{rgb}{0.827451,0.827451,0.827451}%
\pgfsetstrokecolor{currentstroke}%
\pgfsetstrokeopacity{0.800000}%
\pgfsetdash{}{0pt}%
\pgfpathmoveto{\pgfqpoint{9.630920in}{3.276112in}}%
\pgfpathcurveto{\pgfqpoint{9.641970in}{3.276112in}}{\pgfqpoint{9.652569in}{3.280502in}}{\pgfqpoint{9.660383in}{3.288316in}}%
\pgfpathcurveto{\pgfqpoint{9.668196in}{3.296130in}}{\pgfqpoint{9.672587in}{3.306729in}}{\pgfqpoint{9.672587in}{3.317779in}}%
\pgfpathcurveto{\pgfqpoint{9.672587in}{3.328829in}}{\pgfqpoint{9.668196in}{3.339428in}}{\pgfqpoint{9.660383in}{3.347241in}}%
\pgfpathcurveto{\pgfqpoint{9.652569in}{3.355055in}}{\pgfqpoint{9.641970in}{3.359445in}}{\pgfqpoint{9.630920in}{3.359445in}}%
\pgfpathcurveto{\pgfqpoint{9.619870in}{3.359445in}}{\pgfqpoint{9.609271in}{3.355055in}}{\pgfqpoint{9.601457in}{3.347241in}}%
\pgfpathcurveto{\pgfqpoint{9.593644in}{3.339428in}}{\pgfqpoint{9.589253in}{3.328829in}}{\pgfqpoint{9.589253in}{3.317779in}}%
\pgfpathcurveto{\pgfqpoint{9.589253in}{3.306729in}}{\pgfqpoint{9.593644in}{3.296130in}}{\pgfqpoint{9.601457in}{3.288316in}}%
\pgfpathcurveto{\pgfqpoint{9.609271in}{3.280502in}}{\pgfqpoint{9.619870in}{3.276112in}}{\pgfqpoint{9.630920in}{3.276112in}}%
\pgfpathlineto{\pgfqpoint{9.630920in}{3.276112in}}%
\pgfpathclose%
\pgfusepath{stroke}%
\end{pgfscope}%
\begin{pgfscope}%
\pgfpathrectangle{\pgfqpoint{7.512535in}{0.437222in}}{\pgfqpoint{6.275590in}{5.159444in}}%
\pgfusepath{clip}%
\pgfsetbuttcap%
\pgfsetroundjoin%
\pgfsetlinewidth{1.003750pt}%
\definecolor{currentstroke}{rgb}{0.827451,0.827451,0.827451}%
\pgfsetstrokecolor{currentstroke}%
\pgfsetstrokeopacity{0.800000}%
\pgfsetdash{}{0pt}%
\pgfpathmoveto{\pgfqpoint{9.928499in}{4.056499in}}%
\pgfpathcurveto{\pgfqpoint{9.939549in}{4.056499in}}{\pgfqpoint{9.950148in}{4.060889in}}{\pgfqpoint{9.957962in}{4.068703in}}%
\pgfpathcurveto{\pgfqpoint{9.965775in}{4.076516in}}{\pgfqpoint{9.970165in}{4.087115in}}{\pgfqpoint{9.970165in}{4.098166in}}%
\pgfpathcurveto{\pgfqpoint{9.970165in}{4.109216in}}{\pgfqpoint{9.965775in}{4.119815in}}{\pgfqpoint{9.957962in}{4.127628in}}%
\pgfpathcurveto{\pgfqpoint{9.950148in}{4.135442in}}{\pgfqpoint{9.939549in}{4.139832in}}{\pgfqpoint{9.928499in}{4.139832in}}%
\pgfpathcurveto{\pgfqpoint{9.917449in}{4.139832in}}{\pgfqpoint{9.906850in}{4.135442in}}{\pgfqpoint{9.899036in}{4.127628in}}%
\pgfpathcurveto{\pgfqpoint{9.891222in}{4.119815in}}{\pgfqpoint{9.886832in}{4.109216in}}{\pgfqpoint{9.886832in}{4.098166in}}%
\pgfpathcurveto{\pgfqpoint{9.886832in}{4.087115in}}{\pgfqpoint{9.891222in}{4.076516in}}{\pgfqpoint{9.899036in}{4.068703in}}%
\pgfpathcurveto{\pgfqpoint{9.906850in}{4.060889in}}{\pgfqpoint{9.917449in}{4.056499in}}{\pgfqpoint{9.928499in}{4.056499in}}%
\pgfpathlineto{\pgfqpoint{9.928499in}{4.056499in}}%
\pgfpathclose%
\pgfusepath{stroke}%
\end{pgfscope}%
\begin{pgfscope}%
\pgfpathrectangle{\pgfqpoint{7.512535in}{0.437222in}}{\pgfqpoint{6.275590in}{5.159444in}}%
\pgfusepath{clip}%
\pgfsetbuttcap%
\pgfsetroundjoin%
\pgfsetlinewidth{1.003750pt}%
\definecolor{currentstroke}{rgb}{0.827451,0.827451,0.827451}%
\pgfsetstrokecolor{currentstroke}%
\pgfsetstrokeopacity{0.800000}%
\pgfsetdash{}{0pt}%
\pgfpathmoveto{\pgfqpoint{8.598634in}{1.552340in}}%
\pgfpathcurveto{\pgfqpoint{8.609684in}{1.552340in}}{\pgfqpoint{8.620283in}{1.556730in}}{\pgfqpoint{8.628097in}{1.564544in}}%
\pgfpathcurveto{\pgfqpoint{8.635910in}{1.572358in}}{\pgfqpoint{8.640300in}{1.582957in}}{\pgfqpoint{8.640300in}{1.594007in}}%
\pgfpathcurveto{\pgfqpoint{8.640300in}{1.605057in}}{\pgfqpoint{8.635910in}{1.615656in}}{\pgfqpoint{8.628097in}{1.623470in}}%
\pgfpathcurveto{\pgfqpoint{8.620283in}{1.631283in}}{\pgfqpoint{8.609684in}{1.635673in}}{\pgfqpoint{8.598634in}{1.635673in}}%
\pgfpathcurveto{\pgfqpoint{8.587584in}{1.635673in}}{\pgfqpoint{8.576985in}{1.631283in}}{\pgfqpoint{8.569171in}{1.623470in}}%
\pgfpathcurveto{\pgfqpoint{8.561357in}{1.615656in}}{\pgfqpoint{8.556967in}{1.605057in}}{\pgfqpoint{8.556967in}{1.594007in}}%
\pgfpathcurveto{\pgfqpoint{8.556967in}{1.582957in}}{\pgfqpoint{8.561357in}{1.572358in}}{\pgfqpoint{8.569171in}{1.564544in}}%
\pgfpathcurveto{\pgfqpoint{8.576985in}{1.556730in}}{\pgfqpoint{8.587584in}{1.552340in}}{\pgfqpoint{8.598634in}{1.552340in}}%
\pgfpathlineto{\pgfqpoint{8.598634in}{1.552340in}}%
\pgfpathclose%
\pgfusepath{stroke}%
\end{pgfscope}%
\begin{pgfscope}%
\pgfpathrectangle{\pgfqpoint{7.512535in}{0.437222in}}{\pgfqpoint{6.275590in}{5.159444in}}%
\pgfusepath{clip}%
\pgfsetbuttcap%
\pgfsetroundjoin%
\pgfsetlinewidth{1.003750pt}%
\definecolor{currentstroke}{rgb}{0.827451,0.827451,0.827451}%
\pgfsetstrokecolor{currentstroke}%
\pgfsetstrokeopacity{0.800000}%
\pgfsetdash{}{0pt}%
\pgfpathmoveto{\pgfqpoint{9.697330in}{2.947479in}}%
\pgfpathcurveto{\pgfqpoint{9.708380in}{2.947479in}}{\pgfqpoint{9.718979in}{2.951869in}}{\pgfqpoint{9.726792in}{2.959683in}}%
\pgfpathcurveto{\pgfqpoint{9.734606in}{2.967496in}}{\pgfqpoint{9.738996in}{2.978095in}}{\pgfqpoint{9.738996in}{2.989145in}}%
\pgfpathcurveto{\pgfqpoint{9.738996in}{3.000195in}}{\pgfqpoint{9.734606in}{3.010794in}}{\pgfqpoint{9.726792in}{3.018608in}}%
\pgfpathcurveto{\pgfqpoint{9.718979in}{3.026422in}}{\pgfqpoint{9.708380in}{3.030812in}}{\pgfqpoint{9.697330in}{3.030812in}}%
\pgfpathcurveto{\pgfqpoint{9.686279in}{3.030812in}}{\pgfqpoint{9.675680in}{3.026422in}}{\pgfqpoint{9.667867in}{3.018608in}}%
\pgfpathcurveto{\pgfqpoint{9.660053in}{3.010794in}}{\pgfqpoint{9.655663in}{3.000195in}}{\pgfqpoint{9.655663in}{2.989145in}}%
\pgfpathcurveto{\pgfqpoint{9.655663in}{2.978095in}}{\pgfqpoint{9.660053in}{2.967496in}}{\pgfqpoint{9.667867in}{2.959683in}}%
\pgfpathcurveto{\pgfqpoint{9.675680in}{2.951869in}}{\pgfqpoint{9.686279in}{2.947479in}}{\pgfqpoint{9.697330in}{2.947479in}}%
\pgfpathlineto{\pgfqpoint{9.697330in}{2.947479in}}%
\pgfpathclose%
\pgfusepath{stroke}%
\end{pgfscope}%
\begin{pgfscope}%
\pgfpathrectangle{\pgfqpoint{7.512535in}{0.437222in}}{\pgfqpoint{6.275590in}{5.159444in}}%
\pgfusepath{clip}%
\pgfsetbuttcap%
\pgfsetroundjoin%
\pgfsetlinewidth{1.003750pt}%
\definecolor{currentstroke}{rgb}{0.827451,0.827451,0.827451}%
\pgfsetstrokecolor{currentstroke}%
\pgfsetstrokeopacity{0.800000}%
\pgfsetdash{}{0pt}%
\pgfpathmoveto{\pgfqpoint{12.446113in}{5.353825in}}%
\pgfpathcurveto{\pgfqpoint{12.457163in}{5.353825in}}{\pgfqpoint{12.467762in}{5.358215in}}{\pgfqpoint{12.475576in}{5.366029in}}%
\pgfpathcurveto{\pgfqpoint{12.483389in}{5.373843in}}{\pgfqpoint{12.487779in}{5.384442in}}{\pgfqpoint{12.487779in}{5.395492in}}%
\pgfpathcurveto{\pgfqpoint{12.487779in}{5.406542in}}{\pgfqpoint{12.483389in}{5.417141in}}{\pgfqpoint{12.475576in}{5.424955in}}%
\pgfpathcurveto{\pgfqpoint{12.467762in}{5.432768in}}{\pgfqpoint{12.457163in}{5.437158in}}{\pgfqpoint{12.446113in}{5.437158in}}%
\pgfpathcurveto{\pgfqpoint{12.435063in}{5.437158in}}{\pgfqpoint{12.424464in}{5.432768in}}{\pgfqpoint{12.416650in}{5.424955in}}%
\pgfpathcurveto{\pgfqpoint{12.408836in}{5.417141in}}{\pgfqpoint{12.404446in}{5.406542in}}{\pgfqpoint{12.404446in}{5.395492in}}%
\pgfpathcurveto{\pgfqpoint{12.404446in}{5.384442in}}{\pgfqpoint{12.408836in}{5.373843in}}{\pgfqpoint{12.416650in}{5.366029in}}%
\pgfpathcurveto{\pgfqpoint{12.424464in}{5.358215in}}{\pgfqpoint{12.435063in}{5.353825in}}{\pgfqpoint{12.446113in}{5.353825in}}%
\pgfpathlineto{\pgfqpoint{12.446113in}{5.353825in}}%
\pgfpathclose%
\pgfusepath{stroke}%
\end{pgfscope}%
\begin{pgfscope}%
\pgfpathrectangle{\pgfqpoint{7.512535in}{0.437222in}}{\pgfqpoint{6.275590in}{5.159444in}}%
\pgfusepath{clip}%
\pgfsetbuttcap%
\pgfsetroundjoin%
\pgfsetlinewidth{1.003750pt}%
\definecolor{currentstroke}{rgb}{0.827451,0.827451,0.827451}%
\pgfsetstrokecolor{currentstroke}%
\pgfsetstrokeopacity{0.800000}%
\pgfsetdash{}{0pt}%
\pgfpathmoveto{\pgfqpoint{9.394437in}{3.273900in}}%
\pgfpathcurveto{\pgfqpoint{9.405487in}{3.273900in}}{\pgfqpoint{9.416086in}{3.278290in}}{\pgfqpoint{9.423899in}{3.286104in}}%
\pgfpathcurveto{\pgfqpoint{9.431713in}{3.293917in}}{\pgfqpoint{9.436103in}{3.304516in}}{\pgfqpoint{9.436103in}{3.315567in}}%
\pgfpathcurveto{\pgfqpoint{9.436103in}{3.326617in}}{\pgfqpoint{9.431713in}{3.337216in}}{\pgfqpoint{9.423899in}{3.345029in}}%
\pgfpathcurveto{\pgfqpoint{9.416086in}{3.352843in}}{\pgfqpoint{9.405487in}{3.357233in}}{\pgfqpoint{9.394437in}{3.357233in}}%
\pgfpathcurveto{\pgfqpoint{9.383386in}{3.357233in}}{\pgfqpoint{9.372787in}{3.352843in}}{\pgfqpoint{9.364974in}{3.345029in}}%
\pgfpathcurveto{\pgfqpoint{9.357160in}{3.337216in}}{\pgfqpoint{9.352770in}{3.326617in}}{\pgfqpoint{9.352770in}{3.315567in}}%
\pgfpathcurveto{\pgfqpoint{9.352770in}{3.304516in}}{\pgfqpoint{9.357160in}{3.293917in}}{\pgfqpoint{9.364974in}{3.286104in}}%
\pgfpathcurveto{\pgfqpoint{9.372787in}{3.278290in}}{\pgfqpoint{9.383386in}{3.273900in}}{\pgfqpoint{9.394437in}{3.273900in}}%
\pgfpathlineto{\pgfqpoint{9.394437in}{3.273900in}}%
\pgfpathclose%
\pgfusepath{stroke}%
\end{pgfscope}%
\begin{pgfscope}%
\pgfpathrectangle{\pgfqpoint{7.512535in}{0.437222in}}{\pgfqpoint{6.275590in}{5.159444in}}%
\pgfusepath{clip}%
\pgfsetbuttcap%
\pgfsetroundjoin%
\pgfsetlinewidth{1.003750pt}%
\definecolor{currentstroke}{rgb}{0.827451,0.827451,0.827451}%
\pgfsetstrokecolor{currentstroke}%
\pgfsetstrokeopacity{0.800000}%
\pgfsetdash{}{0pt}%
\pgfpathmoveto{\pgfqpoint{11.645781in}{5.002607in}}%
\pgfpathcurveto{\pgfqpoint{11.656832in}{5.002607in}}{\pgfqpoint{11.667431in}{5.006998in}}{\pgfqpoint{11.675244in}{5.014811in}}%
\pgfpathcurveto{\pgfqpoint{11.683058in}{5.022625in}}{\pgfqpoint{11.687448in}{5.033224in}}{\pgfqpoint{11.687448in}{5.044274in}}%
\pgfpathcurveto{\pgfqpoint{11.687448in}{5.055324in}}{\pgfqpoint{11.683058in}{5.065923in}}{\pgfqpoint{11.675244in}{5.073737in}}%
\pgfpathcurveto{\pgfqpoint{11.667431in}{5.081550in}}{\pgfqpoint{11.656832in}{5.085941in}}{\pgfqpoint{11.645781in}{5.085941in}}%
\pgfpathcurveto{\pgfqpoint{11.634731in}{5.085941in}}{\pgfqpoint{11.624132in}{5.081550in}}{\pgfqpoint{11.616319in}{5.073737in}}%
\pgfpathcurveto{\pgfqpoint{11.608505in}{5.065923in}}{\pgfqpoint{11.604115in}{5.055324in}}{\pgfqpoint{11.604115in}{5.044274in}}%
\pgfpathcurveto{\pgfqpoint{11.604115in}{5.033224in}}{\pgfqpoint{11.608505in}{5.022625in}}{\pgfqpoint{11.616319in}{5.014811in}}%
\pgfpathcurveto{\pgfqpoint{11.624132in}{5.006998in}}{\pgfqpoint{11.634731in}{5.002607in}}{\pgfqpoint{11.645781in}{5.002607in}}%
\pgfpathlineto{\pgfqpoint{11.645781in}{5.002607in}}%
\pgfpathclose%
\pgfusepath{stroke}%
\end{pgfscope}%
\begin{pgfscope}%
\pgfpathrectangle{\pgfqpoint{7.512535in}{0.437222in}}{\pgfqpoint{6.275590in}{5.159444in}}%
\pgfusepath{clip}%
\pgfsetbuttcap%
\pgfsetroundjoin%
\pgfsetlinewidth{1.003750pt}%
\definecolor{currentstroke}{rgb}{0.827451,0.827451,0.827451}%
\pgfsetstrokecolor{currentstroke}%
\pgfsetstrokeopacity{0.800000}%
\pgfsetdash{}{0pt}%
\pgfpathmoveto{\pgfqpoint{10.127909in}{4.483356in}}%
\pgfpathcurveto{\pgfqpoint{10.138959in}{4.483356in}}{\pgfqpoint{10.149558in}{4.487746in}}{\pgfqpoint{10.157372in}{4.495560in}}%
\pgfpathcurveto{\pgfqpoint{10.165186in}{4.503373in}}{\pgfqpoint{10.169576in}{4.513972in}}{\pgfqpoint{10.169576in}{4.525022in}}%
\pgfpathcurveto{\pgfqpoint{10.169576in}{4.536072in}}{\pgfqpoint{10.165186in}{4.546671in}}{\pgfqpoint{10.157372in}{4.554485in}}%
\pgfpathcurveto{\pgfqpoint{10.149558in}{4.562299in}}{\pgfqpoint{10.138959in}{4.566689in}}{\pgfqpoint{10.127909in}{4.566689in}}%
\pgfpathcurveto{\pgfqpoint{10.116859in}{4.566689in}}{\pgfqpoint{10.106260in}{4.562299in}}{\pgfqpoint{10.098446in}{4.554485in}}%
\pgfpathcurveto{\pgfqpoint{10.090633in}{4.546671in}}{\pgfqpoint{10.086243in}{4.536072in}}{\pgfqpoint{10.086243in}{4.525022in}}%
\pgfpathcurveto{\pgfqpoint{10.086243in}{4.513972in}}{\pgfqpoint{10.090633in}{4.503373in}}{\pgfqpoint{10.098446in}{4.495560in}}%
\pgfpathcurveto{\pgfqpoint{10.106260in}{4.487746in}}{\pgfqpoint{10.116859in}{4.483356in}}{\pgfqpoint{10.127909in}{4.483356in}}%
\pgfpathlineto{\pgfqpoint{10.127909in}{4.483356in}}%
\pgfpathclose%
\pgfusepath{stroke}%
\end{pgfscope}%
\begin{pgfscope}%
\pgfpathrectangle{\pgfqpoint{7.512535in}{0.437222in}}{\pgfqpoint{6.275590in}{5.159444in}}%
\pgfusepath{clip}%
\pgfsetbuttcap%
\pgfsetroundjoin%
\pgfsetlinewidth{1.003750pt}%
\definecolor{currentstroke}{rgb}{0.827451,0.827451,0.827451}%
\pgfsetstrokecolor{currentstroke}%
\pgfsetstrokeopacity{0.800000}%
\pgfsetdash{}{0pt}%
\pgfpathmoveto{\pgfqpoint{10.413616in}{4.447570in}}%
\pgfpathcurveto{\pgfqpoint{10.424666in}{4.447570in}}{\pgfqpoint{10.435265in}{4.451960in}}{\pgfqpoint{10.443078in}{4.459773in}}%
\pgfpathcurveto{\pgfqpoint{10.450892in}{4.467587in}}{\pgfqpoint{10.455282in}{4.478186in}}{\pgfqpoint{10.455282in}{4.489236in}}%
\pgfpathcurveto{\pgfqpoint{10.455282in}{4.500286in}}{\pgfqpoint{10.450892in}{4.510885in}}{\pgfqpoint{10.443078in}{4.518699in}}%
\pgfpathcurveto{\pgfqpoint{10.435265in}{4.526513in}}{\pgfqpoint{10.424666in}{4.530903in}}{\pgfqpoint{10.413616in}{4.530903in}}%
\pgfpathcurveto{\pgfqpoint{10.402565in}{4.530903in}}{\pgfqpoint{10.391966in}{4.526513in}}{\pgfqpoint{10.384153in}{4.518699in}}%
\pgfpathcurveto{\pgfqpoint{10.376339in}{4.510885in}}{\pgfqpoint{10.371949in}{4.500286in}}{\pgfqpoint{10.371949in}{4.489236in}}%
\pgfpathcurveto{\pgfqpoint{10.371949in}{4.478186in}}{\pgfqpoint{10.376339in}{4.467587in}}{\pgfqpoint{10.384153in}{4.459773in}}%
\pgfpathcurveto{\pgfqpoint{10.391966in}{4.451960in}}{\pgfqpoint{10.402565in}{4.447570in}}{\pgfqpoint{10.413616in}{4.447570in}}%
\pgfpathlineto{\pgfqpoint{10.413616in}{4.447570in}}%
\pgfpathclose%
\pgfusepath{stroke}%
\end{pgfscope}%
\begin{pgfscope}%
\pgfpathrectangle{\pgfqpoint{7.512535in}{0.437222in}}{\pgfqpoint{6.275590in}{5.159444in}}%
\pgfusepath{clip}%
\pgfsetbuttcap%
\pgfsetroundjoin%
\pgfsetlinewidth{1.003750pt}%
\definecolor{currentstroke}{rgb}{0.827451,0.827451,0.827451}%
\pgfsetstrokecolor{currentstroke}%
\pgfsetstrokeopacity{0.800000}%
\pgfsetdash{}{0pt}%
\pgfpathmoveto{\pgfqpoint{9.839703in}{1.834702in}}%
\pgfpathcurveto{\pgfqpoint{9.850753in}{1.834702in}}{\pgfqpoint{9.861352in}{1.839092in}}{\pgfqpoint{9.869165in}{1.846906in}}%
\pgfpathcurveto{\pgfqpoint{9.876979in}{1.854719in}}{\pgfqpoint{9.881369in}{1.865318in}}{\pgfqpoint{9.881369in}{1.876369in}}%
\pgfpathcurveto{\pgfqpoint{9.881369in}{1.887419in}}{\pgfqpoint{9.876979in}{1.898018in}}{\pgfqpoint{9.869165in}{1.905831in}}%
\pgfpathcurveto{\pgfqpoint{9.861352in}{1.913645in}}{\pgfqpoint{9.850753in}{1.918035in}}{\pgfqpoint{9.839703in}{1.918035in}}%
\pgfpathcurveto{\pgfqpoint{9.828652in}{1.918035in}}{\pgfqpoint{9.818053in}{1.913645in}}{\pgfqpoint{9.810240in}{1.905831in}}%
\pgfpathcurveto{\pgfqpoint{9.802426in}{1.898018in}}{\pgfqpoint{9.798036in}{1.887419in}}{\pgfqpoint{9.798036in}{1.876369in}}%
\pgfpathcurveto{\pgfqpoint{9.798036in}{1.865318in}}{\pgfqpoint{9.802426in}{1.854719in}}{\pgfqpoint{9.810240in}{1.846906in}}%
\pgfpathcurveto{\pgfqpoint{9.818053in}{1.839092in}}{\pgfqpoint{9.828652in}{1.834702in}}{\pgfqpoint{9.839703in}{1.834702in}}%
\pgfpathlineto{\pgfqpoint{9.839703in}{1.834702in}}%
\pgfpathclose%
\pgfusepath{stroke}%
\end{pgfscope}%
\begin{pgfscope}%
\pgfpathrectangle{\pgfqpoint{7.512535in}{0.437222in}}{\pgfqpoint{6.275590in}{5.159444in}}%
\pgfusepath{clip}%
\pgfsetbuttcap%
\pgfsetroundjoin%
\pgfsetlinewidth{1.003750pt}%
\definecolor{currentstroke}{rgb}{0.827451,0.827451,0.827451}%
\pgfsetstrokecolor{currentstroke}%
\pgfsetstrokeopacity{0.800000}%
\pgfsetdash{}{0pt}%
\pgfpathmoveto{\pgfqpoint{7.946023in}{1.673751in}}%
\pgfpathcurveto{\pgfqpoint{7.957073in}{1.673751in}}{\pgfqpoint{7.967672in}{1.678142in}}{\pgfqpoint{7.975486in}{1.685955in}}%
\pgfpathcurveto{\pgfqpoint{7.983300in}{1.693769in}}{\pgfqpoint{7.987690in}{1.704368in}}{\pgfqpoint{7.987690in}{1.715418in}}%
\pgfpathcurveto{\pgfqpoint{7.987690in}{1.726468in}}{\pgfqpoint{7.983300in}{1.737067in}}{\pgfqpoint{7.975486in}{1.744881in}}%
\pgfpathcurveto{\pgfqpoint{7.967672in}{1.752694in}}{\pgfqpoint{7.957073in}{1.757085in}}{\pgfqpoint{7.946023in}{1.757085in}}%
\pgfpathcurveto{\pgfqpoint{7.934973in}{1.757085in}}{\pgfqpoint{7.924374in}{1.752694in}}{\pgfqpoint{7.916560in}{1.744881in}}%
\pgfpathcurveto{\pgfqpoint{7.908747in}{1.737067in}}{\pgfqpoint{7.904357in}{1.726468in}}{\pgfqpoint{7.904357in}{1.715418in}}%
\pgfpathcurveto{\pgfqpoint{7.904357in}{1.704368in}}{\pgfqpoint{7.908747in}{1.693769in}}{\pgfqpoint{7.916560in}{1.685955in}}%
\pgfpathcurveto{\pgfqpoint{7.924374in}{1.678142in}}{\pgfqpoint{7.934973in}{1.673751in}}{\pgfqpoint{7.946023in}{1.673751in}}%
\pgfpathlineto{\pgfqpoint{7.946023in}{1.673751in}}%
\pgfpathclose%
\pgfusepath{stroke}%
\end{pgfscope}%
\begin{pgfscope}%
\pgfpathrectangle{\pgfqpoint{7.512535in}{0.437222in}}{\pgfqpoint{6.275590in}{5.159444in}}%
\pgfusepath{clip}%
\pgfsetbuttcap%
\pgfsetroundjoin%
\pgfsetlinewidth{1.003750pt}%
\definecolor{currentstroke}{rgb}{0.827451,0.827451,0.827451}%
\pgfsetstrokecolor{currentstroke}%
\pgfsetstrokeopacity{0.800000}%
\pgfsetdash{}{0pt}%
\pgfpathmoveto{\pgfqpoint{8.261759in}{2.309353in}}%
\pgfpathcurveto{\pgfqpoint{8.272809in}{2.309353in}}{\pgfqpoint{8.283408in}{2.313744in}}{\pgfqpoint{8.291221in}{2.321557in}}%
\pgfpathcurveto{\pgfqpoint{8.299035in}{2.329371in}}{\pgfqpoint{8.303425in}{2.339970in}}{\pgfqpoint{8.303425in}{2.351020in}}%
\pgfpathcurveto{\pgfqpoint{8.303425in}{2.362070in}}{\pgfqpoint{8.299035in}{2.372669in}}{\pgfqpoint{8.291221in}{2.380483in}}%
\pgfpathcurveto{\pgfqpoint{8.283408in}{2.388296in}}{\pgfqpoint{8.272809in}{2.392687in}}{\pgfqpoint{8.261759in}{2.392687in}}%
\pgfpathcurveto{\pgfqpoint{8.250708in}{2.392687in}}{\pgfqpoint{8.240109in}{2.388296in}}{\pgfqpoint{8.232296in}{2.380483in}}%
\pgfpathcurveto{\pgfqpoint{8.224482in}{2.372669in}}{\pgfqpoint{8.220092in}{2.362070in}}{\pgfqpoint{8.220092in}{2.351020in}}%
\pgfpathcurveto{\pgfqpoint{8.220092in}{2.339970in}}{\pgfqpoint{8.224482in}{2.329371in}}{\pgfqpoint{8.232296in}{2.321557in}}%
\pgfpathcurveto{\pgfqpoint{8.240109in}{2.313744in}}{\pgfqpoint{8.250708in}{2.309353in}}{\pgfqpoint{8.261759in}{2.309353in}}%
\pgfpathlineto{\pgfqpoint{8.261759in}{2.309353in}}%
\pgfpathclose%
\pgfusepath{stroke}%
\end{pgfscope}%
\begin{pgfscope}%
\pgfpathrectangle{\pgfqpoint{7.512535in}{0.437222in}}{\pgfqpoint{6.275590in}{5.159444in}}%
\pgfusepath{clip}%
\pgfsetbuttcap%
\pgfsetroundjoin%
\pgfsetlinewidth{1.003750pt}%
\definecolor{currentstroke}{rgb}{0.827451,0.827451,0.827451}%
\pgfsetstrokecolor{currentstroke}%
\pgfsetstrokeopacity{0.800000}%
\pgfsetdash{}{0pt}%
\pgfpathmoveto{\pgfqpoint{12.081725in}{5.113368in}}%
\pgfpathcurveto{\pgfqpoint{12.092776in}{5.113368in}}{\pgfqpoint{12.103375in}{5.117758in}}{\pgfqpoint{12.111188in}{5.125572in}}%
\pgfpathcurveto{\pgfqpoint{12.119002in}{5.133385in}}{\pgfqpoint{12.123392in}{5.143984in}}{\pgfqpoint{12.123392in}{5.155034in}}%
\pgfpathcurveto{\pgfqpoint{12.123392in}{5.166084in}}{\pgfqpoint{12.119002in}{5.176683in}}{\pgfqpoint{12.111188in}{5.184497in}}%
\pgfpathcurveto{\pgfqpoint{12.103375in}{5.192311in}}{\pgfqpoint{12.092776in}{5.196701in}}{\pgfqpoint{12.081725in}{5.196701in}}%
\pgfpathcurveto{\pgfqpoint{12.070675in}{5.196701in}}{\pgfqpoint{12.060076in}{5.192311in}}{\pgfqpoint{12.052263in}{5.184497in}}%
\pgfpathcurveto{\pgfqpoint{12.044449in}{5.176683in}}{\pgfqpoint{12.040059in}{5.166084in}}{\pgfqpoint{12.040059in}{5.155034in}}%
\pgfpathcurveto{\pgfqpoint{12.040059in}{5.143984in}}{\pgfqpoint{12.044449in}{5.133385in}}{\pgfqpoint{12.052263in}{5.125572in}}%
\pgfpathcurveto{\pgfqpoint{12.060076in}{5.117758in}}{\pgfqpoint{12.070675in}{5.113368in}}{\pgfqpoint{12.081725in}{5.113368in}}%
\pgfpathlineto{\pgfqpoint{12.081725in}{5.113368in}}%
\pgfpathclose%
\pgfusepath{stroke}%
\end{pgfscope}%
\begin{pgfscope}%
\pgfpathrectangle{\pgfqpoint{7.512535in}{0.437222in}}{\pgfqpoint{6.275590in}{5.159444in}}%
\pgfusepath{clip}%
\pgfsetbuttcap%
\pgfsetroundjoin%
\pgfsetlinewidth{1.003750pt}%
\definecolor{currentstroke}{rgb}{0.827451,0.827451,0.827451}%
\pgfsetstrokecolor{currentstroke}%
\pgfsetstrokeopacity{0.800000}%
\pgfsetdash{}{0pt}%
\pgfpathmoveto{\pgfqpoint{10.013603in}{2.917799in}}%
\pgfpathcurveto{\pgfqpoint{10.024653in}{2.917799in}}{\pgfqpoint{10.035252in}{2.922190in}}{\pgfqpoint{10.043066in}{2.930003in}}%
\pgfpathcurveto{\pgfqpoint{10.050880in}{2.937817in}}{\pgfqpoint{10.055270in}{2.948416in}}{\pgfqpoint{10.055270in}{2.959466in}}%
\pgfpathcurveto{\pgfqpoint{10.055270in}{2.970516in}}{\pgfqpoint{10.050880in}{2.981115in}}{\pgfqpoint{10.043066in}{2.988929in}}%
\pgfpathcurveto{\pgfqpoint{10.035252in}{2.996743in}}{\pgfqpoint{10.024653in}{3.001133in}}{\pgfqpoint{10.013603in}{3.001133in}}%
\pgfpathcurveto{\pgfqpoint{10.002553in}{3.001133in}}{\pgfqpoint{9.991954in}{2.996743in}}{\pgfqpoint{9.984141in}{2.988929in}}%
\pgfpathcurveto{\pgfqpoint{9.976327in}{2.981115in}}{\pgfqpoint{9.971937in}{2.970516in}}{\pgfqpoint{9.971937in}{2.959466in}}%
\pgfpathcurveto{\pgfqpoint{9.971937in}{2.948416in}}{\pgfqpoint{9.976327in}{2.937817in}}{\pgfqpoint{9.984141in}{2.930003in}}%
\pgfpathcurveto{\pgfqpoint{9.991954in}{2.922190in}}{\pgfqpoint{10.002553in}{2.917799in}}{\pgfqpoint{10.013603in}{2.917799in}}%
\pgfpathlineto{\pgfqpoint{10.013603in}{2.917799in}}%
\pgfpathclose%
\pgfusepath{stroke}%
\end{pgfscope}%
\begin{pgfscope}%
\pgfpathrectangle{\pgfqpoint{7.512535in}{0.437222in}}{\pgfqpoint{6.275590in}{5.159444in}}%
\pgfusepath{clip}%
\pgfsetbuttcap%
\pgfsetroundjoin%
\pgfsetlinewidth{1.003750pt}%
\definecolor{currentstroke}{rgb}{0.827451,0.827451,0.827451}%
\pgfsetstrokecolor{currentstroke}%
\pgfsetstrokeopacity{0.800000}%
\pgfsetdash{}{0pt}%
\pgfpathmoveto{\pgfqpoint{10.873384in}{4.145170in}}%
\pgfpathcurveto{\pgfqpoint{10.884434in}{4.145170in}}{\pgfqpoint{10.895033in}{4.149560in}}{\pgfqpoint{10.902847in}{4.157374in}}%
\pgfpathcurveto{\pgfqpoint{10.910660in}{4.165188in}}{\pgfqpoint{10.915050in}{4.175787in}}{\pgfqpoint{10.915050in}{4.186837in}}%
\pgfpathcurveto{\pgfqpoint{10.915050in}{4.197887in}}{\pgfqpoint{10.910660in}{4.208486in}}{\pgfqpoint{10.902847in}{4.216300in}}%
\pgfpathcurveto{\pgfqpoint{10.895033in}{4.224113in}}{\pgfqpoint{10.884434in}{4.228504in}}{\pgfqpoint{10.873384in}{4.228504in}}%
\pgfpathcurveto{\pgfqpoint{10.862334in}{4.228504in}}{\pgfqpoint{10.851735in}{4.224113in}}{\pgfqpoint{10.843921in}{4.216300in}}%
\pgfpathcurveto{\pgfqpoint{10.836107in}{4.208486in}}{\pgfqpoint{10.831717in}{4.197887in}}{\pgfqpoint{10.831717in}{4.186837in}}%
\pgfpathcurveto{\pgfqpoint{10.831717in}{4.175787in}}{\pgfqpoint{10.836107in}{4.165188in}}{\pgfqpoint{10.843921in}{4.157374in}}%
\pgfpathcurveto{\pgfqpoint{10.851735in}{4.149560in}}{\pgfqpoint{10.862334in}{4.145170in}}{\pgfqpoint{10.873384in}{4.145170in}}%
\pgfpathlineto{\pgfqpoint{10.873384in}{4.145170in}}%
\pgfpathclose%
\pgfusepath{stroke}%
\end{pgfscope}%
\begin{pgfscope}%
\pgfpathrectangle{\pgfqpoint{7.512535in}{0.437222in}}{\pgfqpoint{6.275590in}{5.159444in}}%
\pgfusepath{clip}%
\pgfsetbuttcap%
\pgfsetroundjoin%
\pgfsetlinewidth{1.003750pt}%
\definecolor{currentstroke}{rgb}{0.827451,0.827451,0.827451}%
\pgfsetstrokecolor{currentstroke}%
\pgfsetstrokeopacity{0.800000}%
\pgfsetdash{}{0pt}%
\pgfpathmoveto{\pgfqpoint{9.862917in}{4.507322in}}%
\pgfpathcurveto{\pgfqpoint{9.873967in}{4.507322in}}{\pgfqpoint{9.884566in}{4.511712in}}{\pgfqpoint{9.892380in}{4.519525in}}%
\pgfpathcurveto{\pgfqpoint{9.900194in}{4.527339in}}{\pgfqpoint{9.904584in}{4.537938in}}{\pgfqpoint{9.904584in}{4.548988in}}%
\pgfpathcurveto{\pgfqpoint{9.904584in}{4.560038in}}{\pgfqpoint{9.900194in}{4.570637in}}{\pgfqpoint{9.892380in}{4.578451in}}%
\pgfpathcurveto{\pgfqpoint{9.884566in}{4.586265in}}{\pgfqpoint{9.873967in}{4.590655in}}{\pgfqpoint{9.862917in}{4.590655in}}%
\pgfpathcurveto{\pgfqpoint{9.851867in}{4.590655in}}{\pgfqpoint{9.841268in}{4.586265in}}{\pgfqpoint{9.833454in}{4.578451in}}%
\pgfpathcurveto{\pgfqpoint{9.825641in}{4.570637in}}{\pgfqpoint{9.821250in}{4.560038in}}{\pgfqpoint{9.821250in}{4.548988in}}%
\pgfpathcurveto{\pgfqpoint{9.821250in}{4.537938in}}{\pgfqpoint{9.825641in}{4.527339in}}{\pgfqpoint{9.833454in}{4.519525in}}%
\pgfpathcurveto{\pgfqpoint{9.841268in}{4.511712in}}{\pgfqpoint{9.851867in}{4.507322in}}{\pgfqpoint{9.862917in}{4.507322in}}%
\pgfpathlineto{\pgfqpoint{9.862917in}{4.507322in}}%
\pgfpathclose%
\pgfusepath{stroke}%
\end{pgfscope}%
\begin{pgfscope}%
\pgfpathrectangle{\pgfqpoint{7.512535in}{0.437222in}}{\pgfqpoint{6.275590in}{5.159444in}}%
\pgfusepath{clip}%
\pgfsetbuttcap%
\pgfsetroundjoin%
\pgfsetlinewidth{1.003750pt}%
\definecolor{currentstroke}{rgb}{0.827451,0.827451,0.827451}%
\pgfsetstrokecolor{currentstroke}%
\pgfsetstrokeopacity{0.800000}%
\pgfsetdash{}{0pt}%
\pgfpathmoveto{\pgfqpoint{8.087355in}{0.688351in}}%
\pgfpathcurveto{\pgfqpoint{8.098405in}{0.688351in}}{\pgfqpoint{8.109004in}{0.692741in}}{\pgfqpoint{8.116818in}{0.700555in}}%
\pgfpathcurveto{\pgfqpoint{8.124631in}{0.708368in}}{\pgfqpoint{8.129022in}{0.718967in}}{\pgfqpoint{8.129022in}{0.730017in}}%
\pgfpathcurveto{\pgfqpoint{8.129022in}{0.741068in}}{\pgfqpoint{8.124631in}{0.751667in}}{\pgfqpoint{8.116818in}{0.759480in}}%
\pgfpathcurveto{\pgfqpoint{8.109004in}{0.767294in}}{\pgfqpoint{8.098405in}{0.771684in}}{\pgfqpoint{8.087355in}{0.771684in}}%
\pgfpathcurveto{\pgfqpoint{8.076305in}{0.771684in}}{\pgfqpoint{8.065706in}{0.767294in}}{\pgfqpoint{8.057892in}{0.759480in}}%
\pgfpathcurveto{\pgfqpoint{8.050079in}{0.751667in}}{\pgfqpoint{8.045688in}{0.741068in}}{\pgfqpoint{8.045688in}{0.730017in}}%
\pgfpathcurveto{\pgfqpoint{8.045688in}{0.718967in}}{\pgfqpoint{8.050079in}{0.708368in}}{\pgfqpoint{8.057892in}{0.700555in}}%
\pgfpathcurveto{\pgfqpoint{8.065706in}{0.692741in}}{\pgfqpoint{8.076305in}{0.688351in}}{\pgfqpoint{8.087355in}{0.688351in}}%
\pgfpathlineto{\pgfqpoint{8.087355in}{0.688351in}}%
\pgfpathclose%
\pgfusepath{stroke}%
\end{pgfscope}%
\begin{pgfscope}%
\pgfpathrectangle{\pgfqpoint{7.512535in}{0.437222in}}{\pgfqpoint{6.275590in}{5.159444in}}%
\pgfusepath{clip}%
\pgfsetbuttcap%
\pgfsetroundjoin%
\pgfsetlinewidth{1.003750pt}%
\definecolor{currentstroke}{rgb}{0.827451,0.827451,0.827451}%
\pgfsetstrokecolor{currentstroke}%
\pgfsetstrokeopacity{0.800000}%
\pgfsetdash{}{0pt}%
\pgfpathmoveto{\pgfqpoint{11.431549in}{4.628942in}}%
\pgfpathcurveto{\pgfqpoint{11.442599in}{4.628942in}}{\pgfqpoint{11.453198in}{4.633333in}}{\pgfqpoint{11.461012in}{4.641146in}}%
\pgfpathcurveto{\pgfqpoint{11.468825in}{4.648960in}}{\pgfqpoint{11.473215in}{4.659559in}}{\pgfqpoint{11.473215in}{4.670609in}}%
\pgfpathcurveto{\pgfqpoint{11.473215in}{4.681659in}}{\pgfqpoint{11.468825in}{4.692258in}}{\pgfqpoint{11.461012in}{4.700072in}}%
\pgfpathcurveto{\pgfqpoint{11.453198in}{4.707885in}}{\pgfqpoint{11.442599in}{4.712276in}}{\pgfqpoint{11.431549in}{4.712276in}}%
\pgfpathcurveto{\pgfqpoint{11.420499in}{4.712276in}}{\pgfqpoint{11.409900in}{4.707885in}}{\pgfqpoint{11.402086in}{4.700072in}}%
\pgfpathcurveto{\pgfqpoint{11.394272in}{4.692258in}}{\pgfqpoint{11.389882in}{4.681659in}}{\pgfqpoint{11.389882in}{4.670609in}}%
\pgfpathcurveto{\pgfqpoint{11.389882in}{4.659559in}}{\pgfqpoint{11.394272in}{4.648960in}}{\pgfqpoint{11.402086in}{4.641146in}}%
\pgfpathcurveto{\pgfqpoint{11.409900in}{4.633333in}}{\pgfqpoint{11.420499in}{4.628942in}}{\pgfqpoint{11.431549in}{4.628942in}}%
\pgfpathlineto{\pgfqpoint{11.431549in}{4.628942in}}%
\pgfpathclose%
\pgfusepath{stroke}%
\end{pgfscope}%
\begin{pgfscope}%
\pgfpathrectangle{\pgfqpoint{7.512535in}{0.437222in}}{\pgfqpoint{6.275590in}{5.159444in}}%
\pgfusepath{clip}%
\pgfsetbuttcap%
\pgfsetroundjoin%
\pgfsetlinewidth{1.003750pt}%
\definecolor{currentstroke}{rgb}{0.827451,0.827451,0.827451}%
\pgfsetstrokecolor{currentstroke}%
\pgfsetstrokeopacity{0.800000}%
\pgfsetdash{}{0pt}%
\pgfpathmoveto{\pgfqpoint{12.342968in}{5.547874in}}%
\pgfpathcurveto{\pgfqpoint{12.354019in}{5.547874in}}{\pgfqpoint{12.364618in}{5.552265in}}{\pgfqpoint{12.372431in}{5.560078in}}%
\pgfpathcurveto{\pgfqpoint{12.380245in}{5.567892in}}{\pgfqpoint{12.384635in}{5.578491in}}{\pgfqpoint{12.384635in}{5.589541in}}%
\pgfpathcurveto{\pgfqpoint{12.384635in}{5.600591in}}{\pgfqpoint{12.380245in}{5.611190in}}{\pgfqpoint{12.372431in}{5.619004in}}%
\pgfpathcurveto{\pgfqpoint{12.364618in}{5.626817in}}{\pgfqpoint{12.354019in}{5.631208in}}{\pgfqpoint{12.342968in}{5.631208in}}%
\pgfpathcurveto{\pgfqpoint{12.331918in}{5.631208in}}{\pgfqpoint{12.321319in}{5.626817in}}{\pgfqpoint{12.313506in}{5.619004in}}%
\pgfpathcurveto{\pgfqpoint{12.305692in}{5.611190in}}{\pgfqpoint{12.301302in}{5.600591in}}{\pgfqpoint{12.301302in}{5.589541in}}%
\pgfpathcurveto{\pgfqpoint{12.301302in}{5.578491in}}{\pgfqpoint{12.305692in}{5.567892in}}{\pgfqpoint{12.313506in}{5.560078in}}%
\pgfpathcurveto{\pgfqpoint{12.321319in}{5.552265in}}{\pgfqpoint{12.331918in}{5.547874in}}{\pgfqpoint{12.342968in}{5.547874in}}%
\pgfpathlineto{\pgfqpoint{12.342968in}{5.547874in}}%
\pgfpathclose%
\pgfusepath{stroke}%
\end{pgfscope}%
\begin{pgfscope}%
\pgfpathrectangle{\pgfqpoint{7.512535in}{0.437222in}}{\pgfqpoint{6.275590in}{5.159444in}}%
\pgfusepath{clip}%
\pgfsetbuttcap%
\pgfsetroundjoin%
\pgfsetlinewidth{1.003750pt}%
\definecolor{currentstroke}{rgb}{0.827451,0.827451,0.827451}%
\pgfsetstrokecolor{currentstroke}%
\pgfsetstrokeopacity{0.800000}%
\pgfsetdash{}{0pt}%
\pgfpathmoveto{\pgfqpoint{7.983939in}{1.000624in}}%
\pgfpathcurveto{\pgfqpoint{7.994989in}{1.000624in}}{\pgfqpoint{8.005588in}{1.005015in}}{\pgfqpoint{8.013401in}{1.012828in}}%
\pgfpathcurveto{\pgfqpoint{8.021215in}{1.020642in}}{\pgfqpoint{8.025605in}{1.031241in}}{\pgfqpoint{8.025605in}{1.042291in}}%
\pgfpathcurveto{\pgfqpoint{8.025605in}{1.053341in}}{\pgfqpoint{8.021215in}{1.063940in}}{\pgfqpoint{8.013401in}{1.071754in}}%
\pgfpathcurveto{\pgfqpoint{8.005588in}{1.079568in}}{\pgfqpoint{7.994989in}{1.083958in}}{\pgfqpoint{7.983939in}{1.083958in}}%
\pgfpathcurveto{\pgfqpoint{7.972888in}{1.083958in}}{\pgfqpoint{7.962289in}{1.079568in}}{\pgfqpoint{7.954476in}{1.071754in}}%
\pgfpathcurveto{\pgfqpoint{7.946662in}{1.063940in}}{\pgfqpoint{7.942272in}{1.053341in}}{\pgfqpoint{7.942272in}{1.042291in}}%
\pgfpathcurveto{\pgfqpoint{7.942272in}{1.031241in}}{\pgfqpoint{7.946662in}{1.020642in}}{\pgfqpoint{7.954476in}{1.012828in}}%
\pgfpathcurveto{\pgfqpoint{7.962289in}{1.005015in}}{\pgfqpoint{7.972888in}{1.000624in}}{\pgfqpoint{7.983939in}{1.000624in}}%
\pgfpathlineto{\pgfqpoint{7.983939in}{1.000624in}}%
\pgfpathclose%
\pgfusepath{stroke}%
\end{pgfscope}%
\begin{pgfscope}%
\pgfpathrectangle{\pgfqpoint{7.512535in}{0.437222in}}{\pgfqpoint{6.275590in}{5.159444in}}%
\pgfusepath{clip}%
\pgfsetbuttcap%
\pgfsetroundjoin%
\pgfsetlinewidth{1.003750pt}%
\definecolor{currentstroke}{rgb}{0.827451,0.827451,0.827451}%
\pgfsetstrokecolor{currentstroke}%
\pgfsetstrokeopacity{0.800000}%
\pgfsetdash{}{0pt}%
\pgfpathmoveto{\pgfqpoint{10.563346in}{5.087121in}}%
\pgfpathcurveto{\pgfqpoint{10.574396in}{5.087121in}}{\pgfqpoint{10.584995in}{5.091511in}}{\pgfqpoint{10.592808in}{5.099325in}}%
\pgfpathcurveto{\pgfqpoint{10.600622in}{5.107138in}}{\pgfqpoint{10.605012in}{5.117737in}}{\pgfqpoint{10.605012in}{5.128787in}}%
\pgfpathcurveto{\pgfqpoint{10.605012in}{5.139838in}}{\pgfqpoint{10.600622in}{5.150437in}}{\pgfqpoint{10.592808in}{5.158250in}}%
\pgfpathcurveto{\pgfqpoint{10.584995in}{5.166064in}}{\pgfqpoint{10.574396in}{5.170454in}}{\pgfqpoint{10.563346in}{5.170454in}}%
\pgfpathcurveto{\pgfqpoint{10.552295in}{5.170454in}}{\pgfqpoint{10.541696in}{5.166064in}}{\pgfqpoint{10.533883in}{5.158250in}}%
\pgfpathcurveto{\pgfqpoint{10.526069in}{5.150437in}}{\pgfqpoint{10.521679in}{5.139838in}}{\pgfqpoint{10.521679in}{5.128787in}}%
\pgfpathcurveto{\pgfqpoint{10.521679in}{5.117737in}}{\pgfqpoint{10.526069in}{5.107138in}}{\pgfqpoint{10.533883in}{5.099325in}}%
\pgfpathcurveto{\pgfqpoint{10.541696in}{5.091511in}}{\pgfqpoint{10.552295in}{5.087121in}}{\pgfqpoint{10.563346in}{5.087121in}}%
\pgfpathlineto{\pgfqpoint{10.563346in}{5.087121in}}%
\pgfpathclose%
\pgfusepath{stroke}%
\end{pgfscope}%
\begin{pgfscope}%
\pgfpathrectangle{\pgfqpoint{7.512535in}{0.437222in}}{\pgfqpoint{6.275590in}{5.159444in}}%
\pgfusepath{clip}%
\pgfsetbuttcap%
\pgfsetroundjoin%
\pgfsetlinewidth{1.003750pt}%
\definecolor{currentstroke}{rgb}{0.827451,0.827451,0.827451}%
\pgfsetstrokecolor{currentstroke}%
\pgfsetstrokeopacity{0.800000}%
\pgfsetdash{}{0pt}%
\pgfpathmoveto{\pgfqpoint{12.081489in}{5.112444in}}%
\pgfpathcurveto{\pgfqpoint{12.092539in}{5.112444in}}{\pgfqpoint{12.103138in}{5.116834in}}{\pgfqpoint{12.110952in}{5.124648in}}%
\pgfpathcurveto{\pgfqpoint{12.118765in}{5.132461in}}{\pgfqpoint{12.123156in}{5.143061in}}{\pgfqpoint{12.123156in}{5.154111in}}%
\pgfpathcurveto{\pgfqpoint{12.123156in}{5.165161in}}{\pgfqpoint{12.118765in}{5.175760in}}{\pgfqpoint{12.110952in}{5.183573in}}%
\pgfpathcurveto{\pgfqpoint{12.103138in}{5.191387in}}{\pgfqpoint{12.092539in}{5.195777in}}{\pgfqpoint{12.081489in}{5.195777in}}%
\pgfpathcurveto{\pgfqpoint{12.070439in}{5.195777in}}{\pgfqpoint{12.059840in}{5.191387in}}{\pgfqpoint{12.052026in}{5.183573in}}%
\pgfpathcurveto{\pgfqpoint{12.044212in}{5.175760in}}{\pgfqpoint{12.039822in}{5.165161in}}{\pgfqpoint{12.039822in}{5.154111in}}%
\pgfpathcurveto{\pgfqpoint{12.039822in}{5.143061in}}{\pgfqpoint{12.044212in}{5.132461in}}{\pgfqpoint{12.052026in}{5.124648in}}%
\pgfpathcurveto{\pgfqpoint{12.059840in}{5.116834in}}{\pgfqpoint{12.070439in}{5.112444in}}{\pgfqpoint{12.081489in}{5.112444in}}%
\pgfpathlineto{\pgfqpoint{12.081489in}{5.112444in}}%
\pgfpathclose%
\pgfusepath{stroke}%
\end{pgfscope}%
\begin{pgfscope}%
\pgfpathrectangle{\pgfqpoint{7.512535in}{0.437222in}}{\pgfqpoint{6.275590in}{5.159444in}}%
\pgfusepath{clip}%
\pgfsetbuttcap%
\pgfsetroundjoin%
\pgfsetlinewidth{1.003750pt}%
\definecolor{currentstroke}{rgb}{0.827451,0.827451,0.827451}%
\pgfsetstrokecolor{currentstroke}%
\pgfsetstrokeopacity{0.800000}%
\pgfsetdash{}{0pt}%
\pgfpathmoveto{\pgfqpoint{10.369675in}{4.552452in}}%
\pgfpathcurveto{\pgfqpoint{10.380725in}{4.552452in}}{\pgfqpoint{10.391324in}{4.556842in}}{\pgfqpoint{10.399138in}{4.564656in}}%
\pgfpathcurveto{\pgfqpoint{10.406951in}{4.572470in}}{\pgfqpoint{10.411341in}{4.583069in}}{\pgfqpoint{10.411341in}{4.594119in}}%
\pgfpathcurveto{\pgfqpoint{10.411341in}{4.605169in}}{\pgfqpoint{10.406951in}{4.615768in}}{\pgfqpoint{10.399138in}{4.623582in}}%
\pgfpathcurveto{\pgfqpoint{10.391324in}{4.631395in}}{\pgfqpoint{10.380725in}{4.635785in}}{\pgfqpoint{10.369675in}{4.635785in}}%
\pgfpathcurveto{\pgfqpoint{10.358625in}{4.635785in}}{\pgfqpoint{10.348026in}{4.631395in}}{\pgfqpoint{10.340212in}{4.623582in}}%
\pgfpathcurveto{\pgfqpoint{10.332398in}{4.615768in}}{\pgfqpoint{10.328008in}{4.605169in}}{\pgfqpoint{10.328008in}{4.594119in}}%
\pgfpathcurveto{\pgfqpoint{10.328008in}{4.583069in}}{\pgfqpoint{10.332398in}{4.572470in}}{\pgfqpoint{10.340212in}{4.564656in}}%
\pgfpathcurveto{\pgfqpoint{10.348026in}{4.556842in}}{\pgfqpoint{10.358625in}{4.552452in}}{\pgfqpoint{10.369675in}{4.552452in}}%
\pgfpathlineto{\pgfqpoint{10.369675in}{4.552452in}}%
\pgfpathclose%
\pgfusepath{stroke}%
\end{pgfscope}%
\begin{pgfscope}%
\pgfpathrectangle{\pgfqpoint{7.512535in}{0.437222in}}{\pgfqpoint{6.275590in}{5.159444in}}%
\pgfusepath{clip}%
\pgfsetbuttcap%
\pgfsetroundjoin%
\pgfsetlinewidth{1.003750pt}%
\definecolor{currentstroke}{rgb}{0.827451,0.827451,0.827451}%
\pgfsetstrokecolor{currentstroke}%
\pgfsetstrokeopacity{0.800000}%
\pgfsetdash{}{0pt}%
\pgfpathmoveto{\pgfqpoint{13.165622in}{5.501584in}}%
\pgfpathcurveto{\pgfqpoint{13.176672in}{5.501584in}}{\pgfqpoint{13.187271in}{5.505974in}}{\pgfqpoint{13.195085in}{5.513787in}}%
\pgfpathcurveto{\pgfqpoint{13.202899in}{5.521601in}}{\pgfqpoint{13.207289in}{5.532200in}}{\pgfqpoint{13.207289in}{5.543250in}}%
\pgfpathcurveto{\pgfqpoint{13.207289in}{5.554300in}}{\pgfqpoint{13.202899in}{5.564899in}}{\pgfqpoint{13.195085in}{5.572713in}}%
\pgfpathcurveto{\pgfqpoint{13.187271in}{5.580527in}}{\pgfqpoint{13.176672in}{5.584917in}}{\pgfqpoint{13.165622in}{5.584917in}}%
\pgfpathcurveto{\pgfqpoint{13.154572in}{5.584917in}}{\pgfqpoint{13.143973in}{5.580527in}}{\pgfqpoint{13.136159in}{5.572713in}}%
\pgfpathcurveto{\pgfqpoint{13.128346in}{5.564899in}}{\pgfqpoint{13.123955in}{5.554300in}}{\pgfqpoint{13.123955in}{5.543250in}}%
\pgfpathcurveto{\pgfqpoint{13.123955in}{5.532200in}}{\pgfqpoint{13.128346in}{5.521601in}}{\pgfqpoint{13.136159in}{5.513787in}}%
\pgfpathcurveto{\pgfqpoint{13.143973in}{5.505974in}}{\pgfqpoint{13.154572in}{5.501584in}}{\pgfqpoint{13.165622in}{5.501584in}}%
\pgfpathlineto{\pgfqpoint{13.165622in}{5.501584in}}%
\pgfpathclose%
\pgfusepath{stroke}%
\end{pgfscope}%
\begin{pgfscope}%
\pgfpathrectangle{\pgfqpoint{7.512535in}{0.437222in}}{\pgfqpoint{6.275590in}{5.159444in}}%
\pgfusepath{clip}%
\pgfsetbuttcap%
\pgfsetroundjoin%
\pgfsetlinewidth{1.003750pt}%
\definecolor{currentstroke}{rgb}{0.827451,0.827451,0.827451}%
\pgfsetstrokecolor{currentstroke}%
\pgfsetstrokeopacity{0.800000}%
\pgfsetdash{}{0pt}%
\pgfpathmoveto{\pgfqpoint{9.444129in}{2.962502in}}%
\pgfpathcurveto{\pgfqpoint{9.455180in}{2.962502in}}{\pgfqpoint{9.465779in}{2.966892in}}{\pgfqpoint{9.473592in}{2.974705in}}%
\pgfpathcurveto{\pgfqpoint{9.481406in}{2.982519in}}{\pgfqpoint{9.485796in}{2.993118in}}{\pgfqpoint{9.485796in}{3.004168in}}%
\pgfpathcurveto{\pgfqpoint{9.485796in}{3.015218in}}{\pgfqpoint{9.481406in}{3.025817in}}{\pgfqpoint{9.473592in}{3.033631in}}%
\pgfpathcurveto{\pgfqpoint{9.465779in}{3.041445in}}{\pgfqpoint{9.455180in}{3.045835in}}{\pgfqpoint{9.444129in}{3.045835in}}%
\pgfpathcurveto{\pgfqpoint{9.433079in}{3.045835in}}{\pgfqpoint{9.422480in}{3.041445in}}{\pgfqpoint{9.414667in}{3.033631in}}%
\pgfpathcurveto{\pgfqpoint{9.406853in}{3.025817in}}{\pgfqpoint{9.402463in}{3.015218in}}{\pgfqpoint{9.402463in}{3.004168in}}%
\pgfpathcurveto{\pgfqpoint{9.402463in}{2.993118in}}{\pgfqpoint{9.406853in}{2.982519in}}{\pgfqpoint{9.414667in}{2.974705in}}%
\pgfpathcurveto{\pgfqpoint{9.422480in}{2.966892in}}{\pgfqpoint{9.433079in}{2.962502in}}{\pgfqpoint{9.444129in}{2.962502in}}%
\pgfpathlineto{\pgfqpoint{9.444129in}{2.962502in}}%
\pgfpathclose%
\pgfusepath{stroke}%
\end{pgfscope}%
\begin{pgfscope}%
\pgfpathrectangle{\pgfqpoint{7.512535in}{0.437222in}}{\pgfqpoint{6.275590in}{5.159444in}}%
\pgfusepath{clip}%
\pgfsetbuttcap%
\pgfsetroundjoin%
\pgfsetlinewidth{1.003750pt}%
\definecolor{currentstroke}{rgb}{0.827451,0.827451,0.827451}%
\pgfsetstrokecolor{currentstroke}%
\pgfsetstrokeopacity{0.800000}%
\pgfsetdash{}{0pt}%
\pgfpathmoveto{\pgfqpoint{10.298392in}{4.414746in}}%
\pgfpathcurveto{\pgfqpoint{10.309442in}{4.414746in}}{\pgfqpoint{10.320041in}{4.419136in}}{\pgfqpoint{10.327855in}{4.426950in}}%
\pgfpathcurveto{\pgfqpoint{10.335668in}{4.434763in}}{\pgfqpoint{10.340059in}{4.445362in}}{\pgfqpoint{10.340059in}{4.456413in}}%
\pgfpathcurveto{\pgfqpoint{10.340059in}{4.467463in}}{\pgfqpoint{10.335668in}{4.478062in}}{\pgfqpoint{10.327855in}{4.485875in}}%
\pgfpathcurveto{\pgfqpoint{10.320041in}{4.493689in}}{\pgfqpoint{10.309442in}{4.498079in}}{\pgfqpoint{10.298392in}{4.498079in}}%
\pgfpathcurveto{\pgfqpoint{10.287342in}{4.498079in}}{\pgfqpoint{10.276743in}{4.493689in}}{\pgfqpoint{10.268929in}{4.485875in}}%
\pgfpathcurveto{\pgfqpoint{10.261116in}{4.478062in}}{\pgfqpoint{10.256725in}{4.467463in}}{\pgfqpoint{10.256725in}{4.456413in}}%
\pgfpathcurveto{\pgfqpoint{10.256725in}{4.445362in}}{\pgfqpoint{10.261116in}{4.434763in}}{\pgfqpoint{10.268929in}{4.426950in}}%
\pgfpathcurveto{\pgfqpoint{10.276743in}{4.419136in}}{\pgfqpoint{10.287342in}{4.414746in}}{\pgfqpoint{10.298392in}{4.414746in}}%
\pgfpathlineto{\pgfqpoint{10.298392in}{4.414746in}}%
\pgfpathclose%
\pgfusepath{stroke}%
\end{pgfscope}%
\begin{pgfscope}%
\pgfpathrectangle{\pgfqpoint{7.512535in}{0.437222in}}{\pgfqpoint{6.275590in}{5.159444in}}%
\pgfusepath{clip}%
\pgfsetbuttcap%
\pgfsetroundjoin%
\pgfsetlinewidth{1.003750pt}%
\definecolor{currentstroke}{rgb}{0.827451,0.827451,0.827451}%
\pgfsetstrokecolor{currentstroke}%
\pgfsetstrokeopacity{0.800000}%
\pgfsetdash{}{0pt}%
\pgfpathmoveto{\pgfqpoint{8.290348in}{1.563305in}}%
\pgfpathcurveto{\pgfqpoint{8.301398in}{1.563305in}}{\pgfqpoint{8.311997in}{1.567695in}}{\pgfqpoint{8.319811in}{1.575509in}}%
\pgfpathcurveto{\pgfqpoint{8.327625in}{1.583322in}}{\pgfqpoint{8.332015in}{1.593921in}}{\pgfqpoint{8.332015in}{1.604971in}}%
\pgfpathcurveto{\pgfqpoint{8.332015in}{1.616021in}}{\pgfqpoint{8.327625in}{1.626621in}}{\pgfqpoint{8.319811in}{1.634434in}}%
\pgfpathcurveto{\pgfqpoint{8.311997in}{1.642248in}}{\pgfqpoint{8.301398in}{1.646638in}}{\pgfqpoint{8.290348in}{1.646638in}}%
\pgfpathcurveto{\pgfqpoint{8.279298in}{1.646638in}}{\pgfqpoint{8.268699in}{1.642248in}}{\pgfqpoint{8.260886in}{1.634434in}}%
\pgfpathcurveto{\pgfqpoint{8.253072in}{1.626621in}}{\pgfqpoint{8.248682in}{1.616021in}}{\pgfqpoint{8.248682in}{1.604971in}}%
\pgfpathcurveto{\pgfqpoint{8.248682in}{1.593921in}}{\pgfqpoint{8.253072in}{1.583322in}}{\pgfqpoint{8.260886in}{1.575509in}}%
\pgfpathcurveto{\pgfqpoint{8.268699in}{1.567695in}}{\pgfqpoint{8.279298in}{1.563305in}}{\pgfqpoint{8.290348in}{1.563305in}}%
\pgfpathlineto{\pgfqpoint{8.290348in}{1.563305in}}%
\pgfpathclose%
\pgfusepath{stroke}%
\end{pgfscope}%
\begin{pgfscope}%
\pgfpathrectangle{\pgfqpoint{7.512535in}{0.437222in}}{\pgfqpoint{6.275590in}{5.159444in}}%
\pgfusepath{clip}%
\pgfsetbuttcap%
\pgfsetroundjoin%
\pgfsetlinewidth{1.003750pt}%
\definecolor{currentstroke}{rgb}{0.827451,0.827451,0.827451}%
\pgfsetstrokecolor{currentstroke}%
\pgfsetstrokeopacity{0.800000}%
\pgfsetdash{}{0pt}%
\pgfpathmoveto{\pgfqpoint{10.299622in}{3.250664in}}%
\pgfpathcurveto{\pgfqpoint{10.310672in}{3.250664in}}{\pgfqpoint{10.321271in}{3.255054in}}{\pgfqpoint{10.329085in}{3.262868in}}%
\pgfpathcurveto{\pgfqpoint{10.336898in}{3.270682in}}{\pgfqpoint{10.341289in}{3.281281in}}{\pgfqpoint{10.341289in}{3.292331in}}%
\pgfpathcurveto{\pgfqpoint{10.341289in}{3.303381in}}{\pgfqpoint{10.336898in}{3.313980in}}{\pgfqpoint{10.329085in}{3.321794in}}%
\pgfpathcurveto{\pgfqpoint{10.321271in}{3.329607in}}{\pgfqpoint{10.310672in}{3.333998in}}{\pgfqpoint{10.299622in}{3.333998in}}%
\pgfpathcurveto{\pgfqpoint{10.288572in}{3.333998in}}{\pgfqpoint{10.277973in}{3.329607in}}{\pgfqpoint{10.270159in}{3.321794in}}%
\pgfpathcurveto{\pgfqpoint{10.262346in}{3.313980in}}{\pgfqpoint{10.257955in}{3.303381in}}{\pgfqpoint{10.257955in}{3.292331in}}%
\pgfpathcurveto{\pgfqpoint{10.257955in}{3.281281in}}{\pgfqpoint{10.262346in}{3.270682in}}{\pgfqpoint{10.270159in}{3.262868in}}%
\pgfpathcurveto{\pgfqpoint{10.277973in}{3.255054in}}{\pgfqpoint{10.288572in}{3.250664in}}{\pgfqpoint{10.299622in}{3.250664in}}%
\pgfpathlineto{\pgfqpoint{10.299622in}{3.250664in}}%
\pgfpathclose%
\pgfusepath{stroke}%
\end{pgfscope}%
\begin{pgfscope}%
\pgfpathrectangle{\pgfqpoint{7.512535in}{0.437222in}}{\pgfqpoint{6.275590in}{5.159444in}}%
\pgfusepath{clip}%
\pgfsetbuttcap%
\pgfsetroundjoin%
\pgfsetlinewidth{1.003750pt}%
\definecolor{currentstroke}{rgb}{0.827451,0.827451,0.827451}%
\pgfsetstrokecolor{currentstroke}%
\pgfsetstrokeopacity{0.800000}%
\pgfsetdash{}{0pt}%
\pgfpathmoveto{\pgfqpoint{10.045306in}{4.354371in}}%
\pgfpathcurveto{\pgfqpoint{10.056356in}{4.354371in}}{\pgfqpoint{10.066955in}{4.358761in}}{\pgfqpoint{10.074768in}{4.366575in}}%
\pgfpathcurveto{\pgfqpoint{10.082582in}{4.374389in}}{\pgfqpoint{10.086972in}{4.384988in}}{\pgfqpoint{10.086972in}{4.396038in}}%
\pgfpathcurveto{\pgfqpoint{10.086972in}{4.407088in}}{\pgfqpoint{10.082582in}{4.417687in}}{\pgfqpoint{10.074768in}{4.425500in}}%
\pgfpathcurveto{\pgfqpoint{10.066955in}{4.433314in}}{\pgfqpoint{10.056356in}{4.437704in}}{\pgfqpoint{10.045306in}{4.437704in}}%
\pgfpathcurveto{\pgfqpoint{10.034256in}{4.437704in}}{\pgfqpoint{10.023657in}{4.433314in}}{\pgfqpoint{10.015843in}{4.425500in}}%
\pgfpathcurveto{\pgfqpoint{10.008029in}{4.417687in}}{\pgfqpoint{10.003639in}{4.407088in}}{\pgfqpoint{10.003639in}{4.396038in}}%
\pgfpathcurveto{\pgfqpoint{10.003639in}{4.384988in}}{\pgfqpoint{10.008029in}{4.374389in}}{\pgfqpoint{10.015843in}{4.366575in}}%
\pgfpathcurveto{\pgfqpoint{10.023657in}{4.358761in}}{\pgfqpoint{10.034256in}{4.354371in}}{\pgfqpoint{10.045306in}{4.354371in}}%
\pgfpathlineto{\pgfqpoint{10.045306in}{4.354371in}}%
\pgfpathclose%
\pgfusepath{stroke}%
\end{pgfscope}%
\begin{pgfscope}%
\pgfpathrectangle{\pgfqpoint{7.512535in}{0.437222in}}{\pgfqpoint{6.275590in}{5.159444in}}%
\pgfusepath{clip}%
\pgfsetbuttcap%
\pgfsetroundjoin%
\pgfsetlinewidth{1.003750pt}%
\definecolor{currentstroke}{rgb}{0.827451,0.827451,0.827451}%
\pgfsetstrokecolor{currentstroke}%
\pgfsetstrokeopacity{0.800000}%
\pgfsetdash{}{0pt}%
\pgfpathmoveto{\pgfqpoint{8.680117in}{3.344222in}}%
\pgfpathcurveto{\pgfqpoint{8.691167in}{3.344222in}}{\pgfqpoint{8.701766in}{3.348612in}}{\pgfqpoint{8.709580in}{3.356426in}}%
\pgfpathcurveto{\pgfqpoint{8.717393in}{3.364239in}}{\pgfqpoint{8.721784in}{3.374838in}}{\pgfqpoint{8.721784in}{3.385889in}}%
\pgfpathcurveto{\pgfqpoint{8.721784in}{3.396939in}}{\pgfqpoint{8.717393in}{3.407538in}}{\pgfqpoint{8.709580in}{3.415351in}}%
\pgfpathcurveto{\pgfqpoint{8.701766in}{3.423165in}}{\pgfqpoint{8.691167in}{3.427555in}}{\pgfqpoint{8.680117in}{3.427555in}}%
\pgfpathcurveto{\pgfqpoint{8.669067in}{3.427555in}}{\pgfqpoint{8.658468in}{3.423165in}}{\pgfqpoint{8.650654in}{3.415351in}}%
\pgfpathcurveto{\pgfqpoint{8.642841in}{3.407538in}}{\pgfqpoint{8.638450in}{3.396939in}}{\pgfqpoint{8.638450in}{3.385889in}}%
\pgfpathcurveto{\pgfqpoint{8.638450in}{3.374838in}}{\pgfqpoint{8.642841in}{3.364239in}}{\pgfqpoint{8.650654in}{3.356426in}}%
\pgfpathcurveto{\pgfqpoint{8.658468in}{3.348612in}}{\pgfqpoint{8.669067in}{3.344222in}}{\pgfqpoint{8.680117in}{3.344222in}}%
\pgfpathlineto{\pgfqpoint{8.680117in}{3.344222in}}%
\pgfpathclose%
\pgfusepath{stroke}%
\end{pgfscope}%
\begin{pgfscope}%
\pgfpathrectangle{\pgfqpoint{7.512535in}{0.437222in}}{\pgfqpoint{6.275590in}{5.159444in}}%
\pgfusepath{clip}%
\pgfsetbuttcap%
\pgfsetroundjoin%
\pgfsetlinewidth{1.003750pt}%
\definecolor{currentstroke}{rgb}{0.827451,0.827451,0.827451}%
\pgfsetstrokecolor{currentstroke}%
\pgfsetstrokeopacity{0.800000}%
\pgfsetdash{}{0pt}%
\pgfpathmoveto{\pgfqpoint{8.022992in}{0.688351in}}%
\pgfpathcurveto{\pgfqpoint{8.034042in}{0.688351in}}{\pgfqpoint{8.044641in}{0.692741in}}{\pgfqpoint{8.052455in}{0.700555in}}%
\pgfpathcurveto{\pgfqpoint{8.060269in}{0.708368in}}{\pgfqpoint{8.064659in}{0.718967in}}{\pgfqpoint{8.064659in}{0.730017in}}%
\pgfpathcurveto{\pgfqpoint{8.064659in}{0.741068in}}{\pgfqpoint{8.060269in}{0.751667in}}{\pgfqpoint{8.052455in}{0.759480in}}%
\pgfpathcurveto{\pgfqpoint{8.044641in}{0.767294in}}{\pgfqpoint{8.034042in}{0.771684in}}{\pgfqpoint{8.022992in}{0.771684in}}%
\pgfpathcurveto{\pgfqpoint{8.011942in}{0.771684in}}{\pgfqpoint{8.001343in}{0.767294in}}{\pgfqpoint{7.993529in}{0.759480in}}%
\pgfpathcurveto{\pgfqpoint{7.985716in}{0.751667in}}{\pgfqpoint{7.981326in}{0.741068in}}{\pgfqpoint{7.981326in}{0.730017in}}%
\pgfpathcurveto{\pgfqpoint{7.981326in}{0.718967in}}{\pgfqpoint{7.985716in}{0.708368in}}{\pgfqpoint{7.993529in}{0.700555in}}%
\pgfpathcurveto{\pgfqpoint{8.001343in}{0.692741in}}{\pgfqpoint{8.011942in}{0.688351in}}{\pgfqpoint{8.022992in}{0.688351in}}%
\pgfpathlineto{\pgfqpoint{8.022992in}{0.688351in}}%
\pgfpathclose%
\pgfusepath{stroke}%
\end{pgfscope}%
\begin{pgfscope}%
\pgfpathrectangle{\pgfqpoint{7.512535in}{0.437222in}}{\pgfqpoint{6.275590in}{5.159444in}}%
\pgfusepath{clip}%
\pgfsetbuttcap%
\pgfsetroundjoin%
\pgfsetlinewidth{1.003750pt}%
\definecolor{currentstroke}{rgb}{0.827451,0.827451,0.827451}%
\pgfsetstrokecolor{currentstroke}%
\pgfsetstrokeopacity{0.800000}%
\pgfsetdash{}{0pt}%
\pgfpathmoveto{\pgfqpoint{9.581399in}{3.231111in}}%
\pgfpathcurveto{\pgfqpoint{9.592449in}{3.231111in}}{\pgfqpoint{9.603048in}{3.235501in}}{\pgfqpoint{9.610862in}{3.243315in}}%
\pgfpathcurveto{\pgfqpoint{9.618675in}{3.251128in}}{\pgfqpoint{9.623065in}{3.261727in}}{\pgfqpoint{9.623065in}{3.272778in}}%
\pgfpathcurveto{\pgfqpoint{9.623065in}{3.283828in}}{\pgfqpoint{9.618675in}{3.294427in}}{\pgfqpoint{9.610862in}{3.302240in}}%
\pgfpathcurveto{\pgfqpoint{9.603048in}{3.310054in}}{\pgfqpoint{9.592449in}{3.314444in}}{\pgfqpoint{9.581399in}{3.314444in}}%
\pgfpathcurveto{\pgfqpoint{9.570349in}{3.314444in}}{\pgfqpoint{9.559750in}{3.310054in}}{\pgfqpoint{9.551936in}{3.302240in}}%
\pgfpathcurveto{\pgfqpoint{9.544122in}{3.294427in}}{\pgfqpoint{9.539732in}{3.283828in}}{\pgfqpoint{9.539732in}{3.272778in}}%
\pgfpathcurveto{\pgfqpoint{9.539732in}{3.261727in}}{\pgfqpoint{9.544122in}{3.251128in}}{\pgfqpoint{9.551936in}{3.243315in}}%
\pgfpathcurveto{\pgfqpoint{9.559750in}{3.235501in}}{\pgfqpoint{9.570349in}{3.231111in}}{\pgfqpoint{9.581399in}{3.231111in}}%
\pgfpathlineto{\pgfqpoint{9.581399in}{3.231111in}}%
\pgfpathclose%
\pgfusepath{stroke}%
\end{pgfscope}%
\begin{pgfscope}%
\pgfpathrectangle{\pgfqpoint{7.512535in}{0.437222in}}{\pgfqpoint{6.275590in}{5.159444in}}%
\pgfusepath{clip}%
\pgfsetbuttcap%
\pgfsetroundjoin%
\pgfsetlinewidth{1.003750pt}%
\definecolor{currentstroke}{rgb}{0.827451,0.827451,0.827451}%
\pgfsetstrokecolor{currentstroke}%
\pgfsetstrokeopacity{0.800000}%
\pgfsetdash{}{0pt}%
\pgfpathmoveto{\pgfqpoint{10.385573in}{5.221855in}}%
\pgfpathcurveto{\pgfqpoint{10.396623in}{5.221855in}}{\pgfqpoint{10.407222in}{5.226245in}}{\pgfqpoint{10.415036in}{5.234058in}}%
\pgfpathcurveto{\pgfqpoint{10.422849in}{5.241872in}}{\pgfqpoint{10.427240in}{5.252471in}}{\pgfqpoint{10.427240in}{5.263521in}}%
\pgfpathcurveto{\pgfqpoint{10.427240in}{5.274571in}}{\pgfqpoint{10.422849in}{5.285170in}}{\pgfqpoint{10.415036in}{5.292984in}}%
\pgfpathcurveto{\pgfqpoint{10.407222in}{5.300798in}}{\pgfqpoint{10.396623in}{5.305188in}}{\pgfqpoint{10.385573in}{5.305188in}}%
\pgfpathcurveto{\pgfqpoint{10.374523in}{5.305188in}}{\pgfqpoint{10.363924in}{5.300798in}}{\pgfqpoint{10.356110in}{5.292984in}}%
\pgfpathcurveto{\pgfqpoint{10.348297in}{5.285170in}}{\pgfqpoint{10.343906in}{5.274571in}}{\pgfqpoint{10.343906in}{5.263521in}}%
\pgfpathcurveto{\pgfqpoint{10.343906in}{5.252471in}}{\pgfqpoint{10.348297in}{5.241872in}}{\pgfqpoint{10.356110in}{5.234058in}}%
\pgfpathcurveto{\pgfqpoint{10.363924in}{5.226245in}}{\pgfqpoint{10.374523in}{5.221855in}}{\pgfqpoint{10.385573in}{5.221855in}}%
\pgfpathlineto{\pgfqpoint{10.385573in}{5.221855in}}%
\pgfpathclose%
\pgfusepath{stroke}%
\end{pgfscope}%
\begin{pgfscope}%
\pgfpathrectangle{\pgfqpoint{7.512535in}{0.437222in}}{\pgfqpoint{6.275590in}{5.159444in}}%
\pgfusepath{clip}%
\pgfsetbuttcap%
\pgfsetroundjoin%
\pgfsetlinewidth{1.003750pt}%
\definecolor{currentstroke}{rgb}{0.827451,0.827451,0.827451}%
\pgfsetstrokecolor{currentstroke}%
\pgfsetstrokeopacity{0.800000}%
\pgfsetdash{}{0pt}%
\pgfpathmoveto{\pgfqpoint{12.342968in}{5.553091in}}%
\pgfpathcurveto{\pgfqpoint{12.354019in}{5.553091in}}{\pgfqpoint{12.364618in}{5.557481in}}{\pgfqpoint{12.372431in}{5.565295in}}%
\pgfpathcurveto{\pgfqpoint{12.380245in}{5.573108in}}{\pgfqpoint{12.384635in}{5.583707in}}{\pgfqpoint{12.384635in}{5.594757in}}%
\pgfpathcurveto{\pgfqpoint{12.384635in}{5.605808in}}{\pgfqpoint{12.380245in}{5.616407in}}{\pgfqpoint{12.372431in}{5.624220in}}%
\pgfpathcurveto{\pgfqpoint{12.364618in}{5.632034in}}{\pgfqpoint{12.354019in}{5.636424in}}{\pgfqpoint{12.342968in}{5.636424in}}%
\pgfpathcurveto{\pgfqpoint{12.331918in}{5.636424in}}{\pgfqpoint{12.321319in}{5.632034in}}{\pgfqpoint{12.313506in}{5.624220in}}%
\pgfpathcurveto{\pgfqpoint{12.305692in}{5.616407in}}{\pgfqpoint{12.301302in}{5.605808in}}{\pgfqpoint{12.301302in}{5.594757in}}%
\pgfpathcurveto{\pgfqpoint{12.301302in}{5.583707in}}{\pgfqpoint{12.305692in}{5.573108in}}{\pgfqpoint{12.313506in}{5.565295in}}%
\pgfpathcurveto{\pgfqpoint{12.321319in}{5.557481in}}{\pgfqpoint{12.331918in}{5.553091in}}{\pgfqpoint{12.342968in}{5.553091in}}%
\pgfpathlineto{\pgfqpoint{12.342968in}{5.553091in}}%
\pgfpathclose%
\pgfusepath{stroke}%
\end{pgfscope}%
\begin{pgfscope}%
\pgfpathrectangle{\pgfqpoint{7.512535in}{0.437222in}}{\pgfqpoint{6.275590in}{5.159444in}}%
\pgfusepath{clip}%
\pgfsetbuttcap%
\pgfsetroundjoin%
\pgfsetlinewidth{1.003750pt}%
\definecolor{currentstroke}{rgb}{0.827451,0.827451,0.827451}%
\pgfsetstrokecolor{currentstroke}%
\pgfsetstrokeopacity{0.800000}%
\pgfsetdash{}{0pt}%
\pgfpathmoveto{\pgfqpoint{9.814821in}{1.852544in}}%
\pgfpathcurveto{\pgfqpoint{9.825871in}{1.852544in}}{\pgfqpoint{9.836470in}{1.856935in}}{\pgfqpoint{9.844284in}{1.864748in}}%
\pgfpathcurveto{\pgfqpoint{9.852097in}{1.872562in}}{\pgfqpoint{9.856487in}{1.883161in}}{\pgfqpoint{9.856487in}{1.894211in}}%
\pgfpathcurveto{\pgfqpoint{9.856487in}{1.905261in}}{\pgfqpoint{9.852097in}{1.915860in}}{\pgfqpoint{9.844284in}{1.923674in}}%
\pgfpathcurveto{\pgfqpoint{9.836470in}{1.931487in}}{\pgfqpoint{9.825871in}{1.935878in}}{\pgfqpoint{9.814821in}{1.935878in}}%
\pgfpathcurveto{\pgfqpoint{9.803771in}{1.935878in}}{\pgfqpoint{9.793172in}{1.931487in}}{\pgfqpoint{9.785358in}{1.923674in}}%
\pgfpathcurveto{\pgfqpoint{9.777544in}{1.915860in}}{\pgfqpoint{9.773154in}{1.905261in}}{\pgfqpoint{9.773154in}{1.894211in}}%
\pgfpathcurveto{\pgfqpoint{9.773154in}{1.883161in}}{\pgfqpoint{9.777544in}{1.872562in}}{\pgfqpoint{9.785358in}{1.864748in}}%
\pgfpathcurveto{\pgfqpoint{9.793172in}{1.856935in}}{\pgfqpoint{9.803771in}{1.852544in}}{\pgfqpoint{9.814821in}{1.852544in}}%
\pgfpathlineto{\pgfqpoint{9.814821in}{1.852544in}}%
\pgfpathclose%
\pgfusepath{stroke}%
\end{pgfscope}%
\begin{pgfscope}%
\pgfpathrectangle{\pgfqpoint{7.512535in}{0.437222in}}{\pgfqpoint{6.275590in}{5.159444in}}%
\pgfusepath{clip}%
\pgfsetbuttcap%
\pgfsetroundjoin%
\pgfsetlinewidth{1.003750pt}%
\definecolor{currentstroke}{rgb}{0.827451,0.827451,0.827451}%
\pgfsetstrokecolor{currentstroke}%
\pgfsetstrokeopacity{0.800000}%
\pgfsetdash{}{0pt}%
\pgfpathmoveto{\pgfqpoint{7.679894in}{0.493855in}}%
\pgfpathcurveto{\pgfqpoint{7.690944in}{0.493855in}}{\pgfqpoint{7.701543in}{0.498245in}}{\pgfqpoint{7.709357in}{0.506059in}}%
\pgfpathcurveto{\pgfqpoint{7.717170in}{0.513872in}}{\pgfqpoint{7.721560in}{0.524471in}}{\pgfqpoint{7.721560in}{0.535522in}}%
\pgfpathcurveto{\pgfqpoint{7.721560in}{0.546572in}}{\pgfqpoint{7.717170in}{0.557171in}}{\pgfqpoint{7.709357in}{0.564984in}}%
\pgfpathcurveto{\pgfqpoint{7.701543in}{0.572798in}}{\pgfqpoint{7.690944in}{0.577188in}}{\pgfqpoint{7.679894in}{0.577188in}}%
\pgfpathcurveto{\pgfqpoint{7.668844in}{0.577188in}}{\pgfqpoint{7.658245in}{0.572798in}}{\pgfqpoint{7.650431in}{0.564984in}}%
\pgfpathcurveto{\pgfqpoint{7.642617in}{0.557171in}}{\pgfqpoint{7.638227in}{0.546572in}}{\pgfqpoint{7.638227in}{0.535522in}}%
\pgfpathcurveto{\pgfqpoint{7.638227in}{0.524471in}}{\pgfqpoint{7.642617in}{0.513872in}}{\pgfqpoint{7.650431in}{0.506059in}}%
\pgfpathcurveto{\pgfqpoint{7.658245in}{0.498245in}}{\pgfqpoint{7.668844in}{0.493855in}}{\pgfqpoint{7.679894in}{0.493855in}}%
\pgfpathlineto{\pgfqpoint{7.679894in}{0.493855in}}%
\pgfpathclose%
\pgfusepath{stroke}%
\end{pgfscope}%
\begin{pgfscope}%
\pgfpathrectangle{\pgfqpoint{7.512535in}{0.437222in}}{\pgfqpoint{6.275590in}{5.159444in}}%
\pgfusepath{clip}%
\pgfsetbuttcap%
\pgfsetroundjoin%
\pgfsetlinewidth{1.003750pt}%
\definecolor{currentstroke}{rgb}{0.827451,0.827451,0.827451}%
\pgfsetstrokecolor{currentstroke}%
\pgfsetstrokeopacity{0.800000}%
\pgfsetdash{}{0pt}%
\pgfpathmoveto{\pgfqpoint{9.696526in}{1.429998in}}%
\pgfpathcurveto{\pgfqpoint{9.707576in}{1.429998in}}{\pgfqpoint{9.718175in}{1.434388in}}{\pgfqpoint{9.725989in}{1.442202in}}%
\pgfpathcurveto{\pgfqpoint{9.733802in}{1.450016in}}{\pgfqpoint{9.738193in}{1.460615in}}{\pgfqpoint{9.738193in}{1.471665in}}%
\pgfpathcurveto{\pgfqpoint{9.738193in}{1.482715in}}{\pgfqpoint{9.733802in}{1.493314in}}{\pgfqpoint{9.725989in}{1.501128in}}%
\pgfpathcurveto{\pgfqpoint{9.718175in}{1.508941in}}{\pgfqpoint{9.707576in}{1.513331in}}{\pgfqpoint{9.696526in}{1.513331in}}%
\pgfpathcurveto{\pgfqpoint{9.685476in}{1.513331in}}{\pgfqpoint{9.674877in}{1.508941in}}{\pgfqpoint{9.667063in}{1.501128in}}%
\pgfpathcurveto{\pgfqpoint{9.659250in}{1.493314in}}{\pgfqpoint{9.654859in}{1.482715in}}{\pgfqpoint{9.654859in}{1.471665in}}%
\pgfpathcurveto{\pgfqpoint{9.654859in}{1.460615in}}{\pgfqpoint{9.659250in}{1.450016in}}{\pgfqpoint{9.667063in}{1.442202in}}%
\pgfpathcurveto{\pgfqpoint{9.674877in}{1.434388in}}{\pgfqpoint{9.685476in}{1.429998in}}{\pgfqpoint{9.696526in}{1.429998in}}%
\pgfpathlineto{\pgfqpoint{9.696526in}{1.429998in}}%
\pgfpathclose%
\pgfusepath{stroke}%
\end{pgfscope}%
\begin{pgfscope}%
\pgfpathrectangle{\pgfqpoint{7.512535in}{0.437222in}}{\pgfqpoint{6.275590in}{5.159444in}}%
\pgfusepath{clip}%
\pgfsetbuttcap%
\pgfsetroundjoin%
\pgfsetlinewidth{1.003750pt}%
\definecolor{currentstroke}{rgb}{0.827451,0.827451,0.827451}%
\pgfsetstrokecolor{currentstroke}%
\pgfsetstrokeopacity{0.800000}%
\pgfsetdash{}{0pt}%
\pgfpathmoveto{\pgfqpoint{12.068300in}{5.241788in}}%
\pgfpathcurveto{\pgfqpoint{12.079350in}{5.241788in}}{\pgfqpoint{12.089949in}{5.246178in}}{\pgfqpoint{12.097763in}{5.253991in}}%
\pgfpathcurveto{\pgfqpoint{12.105577in}{5.261805in}}{\pgfqpoint{12.109967in}{5.272404in}}{\pgfqpoint{12.109967in}{5.283454in}}%
\pgfpathcurveto{\pgfqpoint{12.109967in}{5.294504in}}{\pgfqpoint{12.105577in}{5.305103in}}{\pgfqpoint{12.097763in}{5.312917in}}%
\pgfpathcurveto{\pgfqpoint{12.089949in}{5.320731in}}{\pgfqpoint{12.079350in}{5.325121in}}{\pgfqpoint{12.068300in}{5.325121in}}%
\pgfpathcurveto{\pgfqpoint{12.057250in}{5.325121in}}{\pgfqpoint{12.046651in}{5.320731in}}{\pgfqpoint{12.038838in}{5.312917in}}%
\pgfpathcurveto{\pgfqpoint{12.031024in}{5.305103in}}{\pgfqpoint{12.026634in}{5.294504in}}{\pgfqpoint{12.026634in}{5.283454in}}%
\pgfpathcurveto{\pgfqpoint{12.026634in}{5.272404in}}{\pgfqpoint{12.031024in}{5.261805in}}{\pgfqpoint{12.038838in}{5.253991in}}%
\pgfpathcurveto{\pgfqpoint{12.046651in}{5.246178in}}{\pgfqpoint{12.057250in}{5.241788in}}{\pgfqpoint{12.068300in}{5.241788in}}%
\pgfpathlineto{\pgfqpoint{12.068300in}{5.241788in}}%
\pgfpathclose%
\pgfusepath{stroke}%
\end{pgfscope}%
\begin{pgfscope}%
\pgfpathrectangle{\pgfqpoint{7.512535in}{0.437222in}}{\pgfqpoint{6.275590in}{5.159444in}}%
\pgfusepath{clip}%
\pgfsetbuttcap%
\pgfsetroundjoin%
\pgfsetlinewidth{1.003750pt}%
\definecolor{currentstroke}{rgb}{0.827451,0.827451,0.827451}%
\pgfsetstrokecolor{currentstroke}%
\pgfsetstrokeopacity{0.800000}%
\pgfsetdash{}{0pt}%
\pgfpathmoveto{\pgfqpoint{8.624162in}{1.722485in}}%
\pgfpathcurveto{\pgfqpoint{8.635212in}{1.722485in}}{\pgfqpoint{8.645811in}{1.726876in}}{\pgfqpoint{8.653624in}{1.734689in}}%
\pgfpathcurveto{\pgfqpoint{8.661438in}{1.742503in}}{\pgfqpoint{8.665828in}{1.753102in}}{\pgfqpoint{8.665828in}{1.764152in}}%
\pgfpathcurveto{\pgfqpoint{8.665828in}{1.775202in}}{\pgfqpoint{8.661438in}{1.785801in}}{\pgfqpoint{8.653624in}{1.793615in}}%
\pgfpathcurveto{\pgfqpoint{8.645811in}{1.801428in}}{\pgfqpoint{8.635212in}{1.805819in}}{\pgfqpoint{8.624162in}{1.805819in}}%
\pgfpathcurveto{\pgfqpoint{8.613112in}{1.805819in}}{\pgfqpoint{8.602513in}{1.801428in}}{\pgfqpoint{8.594699in}{1.793615in}}%
\pgfpathcurveto{\pgfqpoint{8.586885in}{1.785801in}}{\pgfqpoint{8.582495in}{1.775202in}}{\pgfqpoint{8.582495in}{1.764152in}}%
\pgfpathcurveto{\pgfqpoint{8.582495in}{1.753102in}}{\pgfqpoint{8.586885in}{1.742503in}}{\pgfqpoint{8.594699in}{1.734689in}}%
\pgfpathcurveto{\pgfqpoint{8.602513in}{1.726876in}}{\pgfqpoint{8.613112in}{1.722485in}}{\pgfqpoint{8.624162in}{1.722485in}}%
\pgfpathlineto{\pgfqpoint{8.624162in}{1.722485in}}%
\pgfpathclose%
\pgfusepath{stroke}%
\end{pgfscope}%
\begin{pgfscope}%
\pgfpathrectangle{\pgfqpoint{7.512535in}{0.437222in}}{\pgfqpoint{6.275590in}{5.159444in}}%
\pgfusepath{clip}%
\pgfsetbuttcap%
\pgfsetroundjoin%
\pgfsetlinewidth{1.003750pt}%
\definecolor{currentstroke}{rgb}{0.827451,0.827451,0.827451}%
\pgfsetstrokecolor{currentstroke}%
\pgfsetstrokeopacity{0.800000}%
\pgfsetdash{}{0pt}%
\pgfpathmoveto{\pgfqpoint{12.761367in}{5.431219in}}%
\pgfpathcurveto{\pgfqpoint{12.772417in}{5.431219in}}{\pgfqpoint{12.783016in}{5.435610in}}{\pgfqpoint{12.790830in}{5.443423in}}%
\pgfpathcurveto{\pgfqpoint{12.798643in}{5.451237in}}{\pgfqpoint{12.803034in}{5.461836in}}{\pgfqpoint{12.803034in}{5.472886in}}%
\pgfpathcurveto{\pgfqpoint{12.803034in}{5.483936in}}{\pgfqpoint{12.798643in}{5.494535in}}{\pgfqpoint{12.790830in}{5.502349in}}%
\pgfpathcurveto{\pgfqpoint{12.783016in}{5.510163in}}{\pgfqpoint{12.772417in}{5.514553in}}{\pgfqpoint{12.761367in}{5.514553in}}%
\pgfpathcurveto{\pgfqpoint{12.750317in}{5.514553in}}{\pgfqpoint{12.739718in}{5.510163in}}{\pgfqpoint{12.731904in}{5.502349in}}%
\pgfpathcurveto{\pgfqpoint{12.724090in}{5.494535in}}{\pgfqpoint{12.719700in}{5.483936in}}{\pgfqpoint{12.719700in}{5.472886in}}%
\pgfpathcurveto{\pgfqpoint{12.719700in}{5.461836in}}{\pgfqpoint{12.724090in}{5.451237in}}{\pgfqpoint{12.731904in}{5.443423in}}%
\pgfpathcurveto{\pgfqpoint{12.739718in}{5.435610in}}{\pgfqpoint{12.750317in}{5.431219in}}{\pgfqpoint{12.761367in}{5.431219in}}%
\pgfpathlineto{\pgfqpoint{12.761367in}{5.431219in}}%
\pgfpathclose%
\pgfusepath{stroke}%
\end{pgfscope}%
\begin{pgfscope}%
\pgfpathrectangle{\pgfqpoint{7.512535in}{0.437222in}}{\pgfqpoint{6.275590in}{5.159444in}}%
\pgfusepath{clip}%
\pgfsetbuttcap%
\pgfsetroundjoin%
\pgfsetlinewidth{1.003750pt}%
\definecolor{currentstroke}{rgb}{0.827451,0.827451,0.827451}%
\pgfsetstrokecolor{currentstroke}%
\pgfsetstrokeopacity{0.800000}%
\pgfsetdash{}{0pt}%
\pgfpathmoveto{\pgfqpoint{10.818589in}{4.898946in}}%
\pgfpathcurveto{\pgfqpoint{10.829639in}{4.898946in}}{\pgfqpoint{10.840238in}{4.903336in}}{\pgfqpoint{10.848052in}{4.911150in}}%
\pgfpathcurveto{\pgfqpoint{10.855865in}{4.918964in}}{\pgfqpoint{10.860255in}{4.929563in}}{\pgfqpoint{10.860255in}{4.940613in}}%
\pgfpathcurveto{\pgfqpoint{10.860255in}{4.951663in}}{\pgfqpoint{10.855865in}{4.962262in}}{\pgfqpoint{10.848052in}{4.970076in}}%
\pgfpathcurveto{\pgfqpoint{10.840238in}{4.977889in}}{\pgfqpoint{10.829639in}{4.982279in}}{\pgfqpoint{10.818589in}{4.982279in}}%
\pgfpathcurveto{\pgfqpoint{10.807539in}{4.982279in}}{\pgfqpoint{10.796940in}{4.977889in}}{\pgfqpoint{10.789126in}{4.970076in}}%
\pgfpathcurveto{\pgfqpoint{10.781312in}{4.962262in}}{\pgfqpoint{10.776922in}{4.951663in}}{\pgfqpoint{10.776922in}{4.940613in}}%
\pgfpathcurveto{\pgfqpoint{10.776922in}{4.929563in}}{\pgfqpoint{10.781312in}{4.918964in}}{\pgfqpoint{10.789126in}{4.911150in}}%
\pgfpathcurveto{\pgfqpoint{10.796940in}{4.903336in}}{\pgfqpoint{10.807539in}{4.898946in}}{\pgfqpoint{10.818589in}{4.898946in}}%
\pgfpathlineto{\pgfqpoint{10.818589in}{4.898946in}}%
\pgfpathclose%
\pgfusepath{stroke}%
\end{pgfscope}%
\begin{pgfscope}%
\pgfpathrectangle{\pgfqpoint{7.512535in}{0.437222in}}{\pgfqpoint{6.275590in}{5.159444in}}%
\pgfusepath{clip}%
\pgfsetbuttcap%
\pgfsetroundjoin%
\pgfsetlinewidth{1.003750pt}%
\definecolor{currentstroke}{rgb}{0.827451,0.827451,0.827451}%
\pgfsetstrokecolor{currentstroke}%
\pgfsetstrokeopacity{0.800000}%
\pgfsetdash{}{0pt}%
\pgfpathmoveto{\pgfqpoint{11.650634in}{4.628942in}}%
\pgfpathcurveto{\pgfqpoint{11.661684in}{4.628942in}}{\pgfqpoint{11.672283in}{4.633333in}}{\pgfqpoint{11.680096in}{4.641146in}}%
\pgfpathcurveto{\pgfqpoint{11.687910in}{4.648960in}}{\pgfqpoint{11.692300in}{4.659559in}}{\pgfqpoint{11.692300in}{4.670609in}}%
\pgfpathcurveto{\pgfqpoint{11.692300in}{4.681659in}}{\pgfqpoint{11.687910in}{4.692258in}}{\pgfqpoint{11.680096in}{4.700072in}}%
\pgfpathcurveto{\pgfqpoint{11.672283in}{4.707885in}}{\pgfqpoint{11.661684in}{4.712276in}}{\pgfqpoint{11.650634in}{4.712276in}}%
\pgfpathcurveto{\pgfqpoint{11.639583in}{4.712276in}}{\pgfqpoint{11.628984in}{4.707885in}}{\pgfqpoint{11.621171in}{4.700072in}}%
\pgfpathcurveto{\pgfqpoint{11.613357in}{4.692258in}}{\pgfqpoint{11.608967in}{4.681659in}}{\pgfqpoint{11.608967in}{4.670609in}}%
\pgfpathcurveto{\pgfqpoint{11.608967in}{4.659559in}}{\pgfqpoint{11.613357in}{4.648960in}}{\pgfqpoint{11.621171in}{4.641146in}}%
\pgfpathcurveto{\pgfqpoint{11.628984in}{4.633333in}}{\pgfqpoint{11.639583in}{4.628942in}}{\pgfqpoint{11.650634in}{4.628942in}}%
\pgfpathlineto{\pgfqpoint{11.650634in}{4.628942in}}%
\pgfpathclose%
\pgfusepath{stroke}%
\end{pgfscope}%
\begin{pgfscope}%
\pgfpathrectangle{\pgfqpoint{7.512535in}{0.437222in}}{\pgfqpoint{6.275590in}{5.159444in}}%
\pgfusepath{clip}%
\pgfsetbuttcap%
\pgfsetroundjoin%
\pgfsetlinewidth{1.003750pt}%
\definecolor{currentstroke}{rgb}{0.827451,0.827451,0.827451}%
\pgfsetstrokecolor{currentstroke}%
\pgfsetstrokeopacity{0.800000}%
\pgfsetdash{}{0pt}%
\pgfpathmoveto{\pgfqpoint{9.827666in}{3.301825in}}%
\pgfpathcurveto{\pgfqpoint{9.838716in}{3.301825in}}{\pgfqpoint{9.849315in}{3.306215in}}{\pgfqpoint{9.857129in}{3.314029in}}%
\pgfpathcurveto{\pgfqpoint{9.864942in}{3.321843in}}{\pgfqpoint{9.869333in}{3.332442in}}{\pgfqpoint{9.869333in}{3.343492in}}%
\pgfpathcurveto{\pgfqpoint{9.869333in}{3.354542in}}{\pgfqpoint{9.864942in}{3.365141in}}{\pgfqpoint{9.857129in}{3.372955in}}%
\pgfpathcurveto{\pgfqpoint{9.849315in}{3.380768in}}{\pgfqpoint{9.838716in}{3.385158in}}{\pgfqpoint{9.827666in}{3.385158in}}%
\pgfpathcurveto{\pgfqpoint{9.816616in}{3.385158in}}{\pgfqpoint{9.806017in}{3.380768in}}{\pgfqpoint{9.798203in}{3.372955in}}%
\pgfpathcurveto{\pgfqpoint{9.790390in}{3.365141in}}{\pgfqpoint{9.785999in}{3.354542in}}{\pgfqpoint{9.785999in}{3.343492in}}%
\pgfpathcurveto{\pgfqpoint{9.785999in}{3.332442in}}{\pgfqpoint{9.790390in}{3.321843in}}{\pgfqpoint{9.798203in}{3.314029in}}%
\pgfpathcurveto{\pgfqpoint{9.806017in}{3.306215in}}{\pgfqpoint{9.816616in}{3.301825in}}{\pgfqpoint{9.827666in}{3.301825in}}%
\pgfpathlineto{\pgfqpoint{9.827666in}{3.301825in}}%
\pgfpathclose%
\pgfusepath{stroke}%
\end{pgfscope}%
\begin{pgfscope}%
\pgfpathrectangle{\pgfqpoint{7.512535in}{0.437222in}}{\pgfqpoint{6.275590in}{5.159444in}}%
\pgfusepath{clip}%
\pgfsetbuttcap%
\pgfsetroundjoin%
\pgfsetlinewidth{1.003750pt}%
\definecolor{currentstroke}{rgb}{0.827451,0.827451,0.827451}%
\pgfsetstrokecolor{currentstroke}%
\pgfsetstrokeopacity{0.800000}%
\pgfsetdash{}{0pt}%
\pgfpathmoveto{\pgfqpoint{11.443872in}{5.101603in}}%
\pgfpathcurveto{\pgfqpoint{11.454922in}{5.101603in}}{\pgfqpoint{11.465521in}{5.105993in}}{\pgfqpoint{11.473335in}{5.113807in}}%
\pgfpathcurveto{\pgfqpoint{11.481148in}{5.121620in}}{\pgfqpoint{11.485539in}{5.132219in}}{\pgfqpoint{11.485539in}{5.143270in}}%
\pgfpathcurveto{\pgfqpoint{11.485539in}{5.154320in}}{\pgfqpoint{11.481148in}{5.164919in}}{\pgfqpoint{11.473335in}{5.172732in}}%
\pgfpathcurveto{\pgfqpoint{11.465521in}{5.180546in}}{\pgfqpoint{11.454922in}{5.184936in}}{\pgfqpoint{11.443872in}{5.184936in}}%
\pgfpathcurveto{\pgfqpoint{11.432822in}{5.184936in}}{\pgfqpoint{11.422223in}{5.180546in}}{\pgfqpoint{11.414409in}{5.172732in}}%
\pgfpathcurveto{\pgfqpoint{11.406596in}{5.164919in}}{\pgfqpoint{11.402205in}{5.154320in}}{\pgfqpoint{11.402205in}{5.143270in}}%
\pgfpathcurveto{\pgfqpoint{11.402205in}{5.132219in}}{\pgfqpoint{11.406596in}{5.121620in}}{\pgfqpoint{11.414409in}{5.113807in}}%
\pgfpathcurveto{\pgfqpoint{11.422223in}{5.105993in}}{\pgfqpoint{11.432822in}{5.101603in}}{\pgfqpoint{11.443872in}{5.101603in}}%
\pgfpathlineto{\pgfqpoint{11.443872in}{5.101603in}}%
\pgfpathclose%
\pgfusepath{stroke}%
\end{pgfscope}%
\begin{pgfscope}%
\pgfpathrectangle{\pgfqpoint{7.512535in}{0.437222in}}{\pgfqpoint{6.275590in}{5.159444in}}%
\pgfusepath{clip}%
\pgfsetbuttcap%
\pgfsetroundjoin%
\pgfsetlinewidth{1.003750pt}%
\definecolor{currentstroke}{rgb}{0.827451,0.827451,0.827451}%
\pgfsetstrokecolor{currentstroke}%
\pgfsetstrokeopacity{0.800000}%
\pgfsetdash{}{0pt}%
\pgfpathmoveto{\pgfqpoint{8.990558in}{3.326629in}}%
\pgfpathcurveto{\pgfqpoint{9.001608in}{3.326629in}}{\pgfqpoint{9.012207in}{3.331019in}}{\pgfqpoint{9.020021in}{3.338833in}}%
\pgfpathcurveto{\pgfqpoint{9.027835in}{3.346646in}}{\pgfqpoint{9.032225in}{3.357246in}}{\pgfqpoint{9.032225in}{3.368296in}}%
\pgfpathcurveto{\pgfqpoint{9.032225in}{3.379346in}}{\pgfqpoint{9.027835in}{3.389945in}}{\pgfqpoint{9.020021in}{3.397758in}}%
\pgfpathcurveto{\pgfqpoint{9.012207in}{3.405572in}}{\pgfqpoint{9.001608in}{3.409962in}}{\pgfqpoint{8.990558in}{3.409962in}}%
\pgfpathcurveto{\pgfqpoint{8.979508in}{3.409962in}}{\pgfqpoint{8.968909in}{3.405572in}}{\pgfqpoint{8.961095in}{3.397758in}}%
\pgfpathcurveto{\pgfqpoint{8.953282in}{3.389945in}}{\pgfqpoint{8.948892in}{3.379346in}}{\pgfqpoint{8.948892in}{3.368296in}}%
\pgfpathcurveto{\pgfqpoint{8.948892in}{3.357246in}}{\pgfqpoint{8.953282in}{3.346646in}}{\pgfqpoint{8.961095in}{3.338833in}}%
\pgfpathcurveto{\pgfqpoint{8.968909in}{3.331019in}}{\pgfqpoint{8.979508in}{3.326629in}}{\pgfqpoint{8.990558in}{3.326629in}}%
\pgfpathlineto{\pgfqpoint{8.990558in}{3.326629in}}%
\pgfpathclose%
\pgfusepath{stroke}%
\end{pgfscope}%
\begin{pgfscope}%
\pgfpathrectangle{\pgfqpoint{7.512535in}{0.437222in}}{\pgfqpoint{6.275590in}{5.159444in}}%
\pgfusepath{clip}%
\pgfsetbuttcap%
\pgfsetroundjoin%
\pgfsetlinewidth{1.003750pt}%
\definecolor{currentstroke}{rgb}{0.827451,0.827451,0.827451}%
\pgfsetstrokecolor{currentstroke}%
\pgfsetstrokeopacity{0.800000}%
\pgfsetdash{}{0pt}%
\pgfpathmoveto{\pgfqpoint{8.290348in}{1.139798in}}%
\pgfpathcurveto{\pgfqpoint{8.301398in}{1.139798in}}{\pgfqpoint{8.311997in}{1.144188in}}{\pgfqpoint{8.319811in}{1.152002in}}%
\pgfpathcurveto{\pgfqpoint{8.327625in}{1.159815in}}{\pgfqpoint{8.332015in}{1.170414in}}{\pgfqpoint{8.332015in}{1.181465in}}%
\pgfpathcurveto{\pgfqpoint{8.332015in}{1.192515in}}{\pgfqpoint{8.327625in}{1.203114in}}{\pgfqpoint{8.319811in}{1.210927in}}%
\pgfpathcurveto{\pgfqpoint{8.311997in}{1.218741in}}{\pgfqpoint{8.301398in}{1.223131in}}{\pgfqpoint{8.290348in}{1.223131in}}%
\pgfpathcurveto{\pgfqpoint{8.279298in}{1.223131in}}{\pgfqpoint{8.268699in}{1.218741in}}{\pgfqpoint{8.260886in}{1.210927in}}%
\pgfpathcurveto{\pgfqpoint{8.253072in}{1.203114in}}{\pgfqpoint{8.248682in}{1.192515in}}{\pgfqpoint{8.248682in}{1.181465in}}%
\pgfpathcurveto{\pgfqpoint{8.248682in}{1.170414in}}{\pgfqpoint{8.253072in}{1.159815in}}{\pgfqpoint{8.260886in}{1.152002in}}%
\pgfpathcurveto{\pgfqpoint{8.268699in}{1.144188in}}{\pgfqpoint{8.279298in}{1.139798in}}{\pgfqpoint{8.290348in}{1.139798in}}%
\pgfpathlineto{\pgfqpoint{8.290348in}{1.139798in}}%
\pgfpathclose%
\pgfusepath{stroke}%
\end{pgfscope}%
\begin{pgfscope}%
\pgfpathrectangle{\pgfqpoint{7.512535in}{0.437222in}}{\pgfqpoint{6.275590in}{5.159444in}}%
\pgfusepath{clip}%
\pgfsetbuttcap%
\pgfsetroundjoin%
\pgfsetlinewidth{1.003750pt}%
\definecolor{currentstroke}{rgb}{0.827451,0.827451,0.827451}%
\pgfsetstrokecolor{currentstroke}%
\pgfsetstrokeopacity{0.800000}%
\pgfsetdash{}{0pt}%
\pgfpathmoveto{\pgfqpoint{10.275694in}{4.483292in}}%
\pgfpathcurveto{\pgfqpoint{10.286744in}{4.483292in}}{\pgfqpoint{10.297343in}{4.487682in}}{\pgfqpoint{10.305157in}{4.495495in}}%
\pgfpathcurveto{\pgfqpoint{10.312971in}{4.503309in}}{\pgfqpoint{10.317361in}{4.513908in}}{\pgfqpoint{10.317361in}{4.524958in}}%
\pgfpathcurveto{\pgfqpoint{10.317361in}{4.536008in}}{\pgfqpoint{10.312971in}{4.546607in}}{\pgfqpoint{10.305157in}{4.554421in}}%
\pgfpathcurveto{\pgfqpoint{10.297343in}{4.562235in}}{\pgfqpoint{10.286744in}{4.566625in}}{\pgfqpoint{10.275694in}{4.566625in}}%
\pgfpathcurveto{\pgfqpoint{10.264644in}{4.566625in}}{\pgfqpoint{10.254045in}{4.562235in}}{\pgfqpoint{10.246231in}{4.554421in}}%
\pgfpathcurveto{\pgfqpoint{10.238418in}{4.546607in}}{\pgfqpoint{10.234028in}{4.536008in}}{\pgfqpoint{10.234028in}{4.524958in}}%
\pgfpathcurveto{\pgfqpoint{10.234028in}{4.513908in}}{\pgfqpoint{10.238418in}{4.503309in}}{\pgfqpoint{10.246231in}{4.495495in}}%
\pgfpathcurveto{\pgfqpoint{10.254045in}{4.487682in}}{\pgfqpoint{10.264644in}{4.483292in}}{\pgfqpoint{10.275694in}{4.483292in}}%
\pgfpathlineto{\pgfqpoint{10.275694in}{4.483292in}}%
\pgfpathclose%
\pgfusepath{stroke}%
\end{pgfscope}%
\begin{pgfscope}%
\pgfpathrectangle{\pgfqpoint{7.512535in}{0.437222in}}{\pgfqpoint{6.275590in}{5.159444in}}%
\pgfusepath{clip}%
\pgfsetbuttcap%
\pgfsetroundjoin%
\pgfsetlinewidth{1.003750pt}%
\definecolor{currentstroke}{rgb}{0.827451,0.827451,0.827451}%
\pgfsetstrokecolor{currentstroke}%
\pgfsetstrokeopacity{0.800000}%
\pgfsetdash{}{0pt}%
\pgfpathmoveto{\pgfqpoint{8.800029in}{1.838807in}}%
\pgfpathcurveto{\pgfqpoint{8.811079in}{1.838807in}}{\pgfqpoint{8.821678in}{1.843198in}}{\pgfqpoint{8.829491in}{1.851011in}}%
\pgfpathcurveto{\pgfqpoint{8.837305in}{1.858825in}}{\pgfqpoint{8.841695in}{1.869424in}}{\pgfqpoint{8.841695in}{1.880474in}}%
\pgfpathcurveto{\pgfqpoint{8.841695in}{1.891524in}}{\pgfqpoint{8.837305in}{1.902123in}}{\pgfqpoint{8.829491in}{1.909937in}}%
\pgfpathcurveto{\pgfqpoint{8.821678in}{1.917750in}}{\pgfqpoint{8.811079in}{1.922141in}}{\pgfqpoint{8.800029in}{1.922141in}}%
\pgfpathcurveto{\pgfqpoint{8.788978in}{1.922141in}}{\pgfqpoint{8.778379in}{1.917750in}}{\pgfqpoint{8.770566in}{1.909937in}}%
\pgfpathcurveto{\pgfqpoint{8.762752in}{1.902123in}}{\pgfqpoint{8.758362in}{1.891524in}}{\pgfqpoint{8.758362in}{1.880474in}}%
\pgfpathcurveto{\pgfqpoint{8.758362in}{1.869424in}}{\pgfqpoint{8.762752in}{1.858825in}}{\pgfqpoint{8.770566in}{1.851011in}}%
\pgfpathcurveto{\pgfqpoint{8.778379in}{1.843198in}}{\pgfqpoint{8.788978in}{1.838807in}}{\pgfqpoint{8.800029in}{1.838807in}}%
\pgfpathlineto{\pgfqpoint{8.800029in}{1.838807in}}%
\pgfpathclose%
\pgfusepath{stroke}%
\end{pgfscope}%
\begin{pgfscope}%
\pgfpathrectangle{\pgfqpoint{7.512535in}{0.437222in}}{\pgfqpoint{6.275590in}{5.159444in}}%
\pgfusepath{clip}%
\pgfsetbuttcap%
\pgfsetroundjoin%
\pgfsetlinewidth{1.003750pt}%
\definecolor{currentstroke}{rgb}{0.827451,0.827451,0.827451}%
\pgfsetstrokecolor{currentstroke}%
\pgfsetstrokeopacity{0.800000}%
\pgfsetdash{}{0pt}%
\pgfpathmoveto{\pgfqpoint{13.468750in}{5.529360in}}%
\pgfpathcurveto{\pgfqpoint{13.479800in}{5.529360in}}{\pgfqpoint{13.490399in}{5.533750in}}{\pgfqpoint{13.498212in}{5.541564in}}%
\pgfpathcurveto{\pgfqpoint{13.506026in}{5.549378in}}{\pgfqpoint{13.510416in}{5.559977in}}{\pgfqpoint{13.510416in}{5.571027in}}%
\pgfpathcurveto{\pgfqpoint{13.510416in}{5.582077in}}{\pgfqpoint{13.506026in}{5.592676in}}{\pgfqpoint{13.498212in}{5.600490in}}%
\pgfpathcurveto{\pgfqpoint{13.490399in}{5.608303in}}{\pgfqpoint{13.479800in}{5.612694in}}{\pgfqpoint{13.468750in}{5.612694in}}%
\pgfpathcurveto{\pgfqpoint{13.457699in}{5.612694in}}{\pgfqpoint{13.447100in}{5.608303in}}{\pgfqpoint{13.439287in}{5.600490in}}%
\pgfpathcurveto{\pgfqpoint{13.431473in}{5.592676in}}{\pgfqpoint{13.427083in}{5.582077in}}{\pgfqpoint{13.427083in}{5.571027in}}%
\pgfpathcurveto{\pgfqpoint{13.427083in}{5.559977in}}{\pgfqpoint{13.431473in}{5.549378in}}{\pgfqpoint{13.439287in}{5.541564in}}%
\pgfpathcurveto{\pgfqpoint{13.447100in}{5.533750in}}{\pgfqpoint{13.457699in}{5.529360in}}{\pgfqpoint{13.468750in}{5.529360in}}%
\pgfpathlineto{\pgfqpoint{13.468750in}{5.529360in}}%
\pgfpathclose%
\pgfusepath{stroke}%
\end{pgfscope}%
\begin{pgfscope}%
\pgfpathrectangle{\pgfqpoint{7.512535in}{0.437222in}}{\pgfqpoint{6.275590in}{5.159444in}}%
\pgfusepath{clip}%
\pgfsetbuttcap%
\pgfsetroundjoin%
\pgfsetlinewidth{1.003750pt}%
\definecolor{currentstroke}{rgb}{0.827451,0.827451,0.827451}%
\pgfsetstrokecolor{currentstroke}%
\pgfsetstrokeopacity{0.800000}%
\pgfsetdash{}{0pt}%
\pgfpathmoveto{\pgfqpoint{13.101307in}{5.330814in}}%
\pgfpathcurveto{\pgfqpoint{13.112357in}{5.330814in}}{\pgfqpoint{13.122956in}{5.335205in}}{\pgfqpoint{13.130770in}{5.343018in}}%
\pgfpathcurveto{\pgfqpoint{13.138583in}{5.350832in}}{\pgfqpoint{13.142973in}{5.361431in}}{\pgfqpoint{13.142973in}{5.372481in}}%
\pgfpathcurveto{\pgfqpoint{13.142973in}{5.383531in}}{\pgfqpoint{13.138583in}{5.394130in}}{\pgfqpoint{13.130770in}{5.401944in}}%
\pgfpathcurveto{\pgfqpoint{13.122956in}{5.409758in}}{\pgfqpoint{13.112357in}{5.414148in}}{\pgfqpoint{13.101307in}{5.414148in}}%
\pgfpathcurveto{\pgfqpoint{13.090257in}{5.414148in}}{\pgfqpoint{13.079658in}{5.409758in}}{\pgfqpoint{13.071844in}{5.401944in}}%
\pgfpathcurveto{\pgfqpoint{13.064030in}{5.394130in}}{\pgfqpoint{13.059640in}{5.383531in}}{\pgfqpoint{13.059640in}{5.372481in}}%
\pgfpathcurveto{\pgfqpoint{13.059640in}{5.361431in}}{\pgfqpoint{13.064030in}{5.350832in}}{\pgfqpoint{13.071844in}{5.343018in}}%
\pgfpathcurveto{\pgfqpoint{13.079658in}{5.335205in}}{\pgfqpoint{13.090257in}{5.330814in}}{\pgfqpoint{13.101307in}{5.330814in}}%
\pgfpathlineto{\pgfqpoint{13.101307in}{5.330814in}}%
\pgfpathclose%
\pgfusepath{stroke}%
\end{pgfscope}%
\begin{pgfscope}%
\pgfpathrectangle{\pgfqpoint{7.512535in}{0.437222in}}{\pgfqpoint{6.275590in}{5.159444in}}%
\pgfusepath{clip}%
\pgfsetbuttcap%
\pgfsetroundjoin%
\pgfsetlinewidth{1.003750pt}%
\definecolor{currentstroke}{rgb}{0.827451,0.827451,0.827451}%
\pgfsetstrokecolor{currentstroke}%
\pgfsetstrokeopacity{0.800000}%
\pgfsetdash{}{0pt}%
\pgfpathmoveto{\pgfqpoint{10.259535in}{4.539220in}}%
\pgfpathcurveto{\pgfqpoint{10.270585in}{4.539220in}}{\pgfqpoint{10.281184in}{4.543611in}}{\pgfqpoint{10.288997in}{4.551424in}}%
\pgfpathcurveto{\pgfqpoint{10.296811in}{4.559238in}}{\pgfqpoint{10.301201in}{4.569837in}}{\pgfqpoint{10.301201in}{4.580887in}}%
\pgfpathcurveto{\pgfqpoint{10.301201in}{4.591937in}}{\pgfqpoint{10.296811in}{4.602536in}}{\pgfqpoint{10.288997in}{4.610350in}}%
\pgfpathcurveto{\pgfqpoint{10.281184in}{4.618164in}}{\pgfqpoint{10.270585in}{4.622554in}}{\pgfqpoint{10.259535in}{4.622554in}}%
\pgfpathcurveto{\pgfqpoint{10.248485in}{4.622554in}}{\pgfqpoint{10.237886in}{4.618164in}}{\pgfqpoint{10.230072in}{4.610350in}}%
\pgfpathcurveto{\pgfqpoint{10.222258in}{4.602536in}}{\pgfqpoint{10.217868in}{4.591937in}}{\pgfqpoint{10.217868in}{4.580887in}}%
\pgfpathcurveto{\pgfqpoint{10.217868in}{4.569837in}}{\pgfqpoint{10.222258in}{4.559238in}}{\pgfqpoint{10.230072in}{4.551424in}}%
\pgfpathcurveto{\pgfqpoint{10.237886in}{4.543611in}}{\pgfqpoint{10.248485in}{4.539220in}}{\pgfqpoint{10.259535in}{4.539220in}}%
\pgfpathlineto{\pgfqpoint{10.259535in}{4.539220in}}%
\pgfpathclose%
\pgfusepath{stroke}%
\end{pgfscope}%
\begin{pgfscope}%
\pgfpathrectangle{\pgfqpoint{7.512535in}{0.437222in}}{\pgfqpoint{6.275590in}{5.159444in}}%
\pgfusepath{clip}%
\pgfsetbuttcap%
\pgfsetroundjoin%
\pgfsetlinewidth{1.003750pt}%
\definecolor{currentstroke}{rgb}{0.827451,0.827451,0.827451}%
\pgfsetstrokecolor{currentstroke}%
\pgfsetstrokeopacity{0.800000}%
\pgfsetdash{}{0pt}%
\pgfpathmoveto{\pgfqpoint{8.180219in}{2.190605in}}%
\pgfpathcurveto{\pgfqpoint{8.191269in}{2.190605in}}{\pgfqpoint{8.201868in}{2.194995in}}{\pgfqpoint{8.209682in}{2.202809in}}%
\pgfpathcurveto{\pgfqpoint{8.217496in}{2.210622in}}{\pgfqpoint{8.221886in}{2.221221in}}{\pgfqpoint{8.221886in}{2.232272in}}%
\pgfpathcurveto{\pgfqpoint{8.221886in}{2.243322in}}{\pgfqpoint{8.217496in}{2.253921in}}{\pgfqpoint{8.209682in}{2.261734in}}%
\pgfpathcurveto{\pgfqpoint{8.201868in}{2.269548in}}{\pgfqpoint{8.191269in}{2.273938in}}{\pgfqpoint{8.180219in}{2.273938in}}%
\pgfpathcurveto{\pgfqpoint{8.169169in}{2.273938in}}{\pgfqpoint{8.158570in}{2.269548in}}{\pgfqpoint{8.150756in}{2.261734in}}%
\pgfpathcurveto{\pgfqpoint{8.142943in}{2.253921in}}{\pgfqpoint{8.138553in}{2.243322in}}{\pgfqpoint{8.138553in}{2.232272in}}%
\pgfpathcurveto{\pgfqpoint{8.138553in}{2.221221in}}{\pgfqpoint{8.142943in}{2.210622in}}{\pgfqpoint{8.150756in}{2.202809in}}%
\pgfpathcurveto{\pgfqpoint{8.158570in}{2.194995in}}{\pgfqpoint{8.169169in}{2.190605in}}{\pgfqpoint{8.180219in}{2.190605in}}%
\pgfpathlineto{\pgfqpoint{8.180219in}{2.190605in}}%
\pgfpathclose%
\pgfusepath{stroke}%
\end{pgfscope}%
\begin{pgfscope}%
\pgfpathrectangle{\pgfqpoint{7.512535in}{0.437222in}}{\pgfqpoint{6.275590in}{5.159444in}}%
\pgfusepath{clip}%
\pgfsetbuttcap%
\pgfsetroundjoin%
\pgfsetlinewidth{1.003750pt}%
\definecolor{currentstroke}{rgb}{0.827451,0.827451,0.827451}%
\pgfsetstrokecolor{currentstroke}%
\pgfsetstrokeopacity{0.800000}%
\pgfsetdash{}{0pt}%
\pgfpathmoveto{\pgfqpoint{10.412487in}{3.527290in}}%
\pgfpathcurveto{\pgfqpoint{10.423537in}{3.527290in}}{\pgfqpoint{10.434136in}{3.531681in}}{\pgfqpoint{10.441950in}{3.539494in}}%
\pgfpathcurveto{\pgfqpoint{10.449764in}{3.547308in}}{\pgfqpoint{10.454154in}{3.557907in}}{\pgfqpoint{10.454154in}{3.568957in}}%
\pgfpathcurveto{\pgfqpoint{10.454154in}{3.580007in}}{\pgfqpoint{10.449764in}{3.590606in}}{\pgfqpoint{10.441950in}{3.598420in}}%
\pgfpathcurveto{\pgfqpoint{10.434136in}{3.606234in}}{\pgfqpoint{10.423537in}{3.610624in}}{\pgfqpoint{10.412487in}{3.610624in}}%
\pgfpathcurveto{\pgfqpoint{10.401437in}{3.610624in}}{\pgfqpoint{10.390838in}{3.606234in}}{\pgfqpoint{10.383025in}{3.598420in}}%
\pgfpathcurveto{\pgfqpoint{10.375211in}{3.590606in}}{\pgfqpoint{10.370821in}{3.580007in}}{\pgfqpoint{10.370821in}{3.568957in}}%
\pgfpathcurveto{\pgfqpoint{10.370821in}{3.557907in}}{\pgfqpoint{10.375211in}{3.547308in}}{\pgfqpoint{10.383025in}{3.539494in}}%
\pgfpathcurveto{\pgfqpoint{10.390838in}{3.531681in}}{\pgfqpoint{10.401437in}{3.527290in}}{\pgfqpoint{10.412487in}{3.527290in}}%
\pgfpathlineto{\pgfqpoint{10.412487in}{3.527290in}}%
\pgfpathclose%
\pgfusepath{stroke}%
\end{pgfscope}%
\begin{pgfscope}%
\pgfpathrectangle{\pgfqpoint{7.512535in}{0.437222in}}{\pgfqpoint{6.275590in}{5.159444in}}%
\pgfusepath{clip}%
\pgfsetbuttcap%
\pgfsetroundjoin%
\pgfsetlinewidth{1.003750pt}%
\definecolor{currentstroke}{rgb}{0.827451,0.827451,0.827451}%
\pgfsetstrokecolor{currentstroke}%
\pgfsetstrokeopacity{0.800000}%
\pgfsetdash{}{0pt}%
\pgfpathmoveto{\pgfqpoint{8.607098in}{2.557652in}}%
\pgfpathcurveto{\pgfqpoint{8.618149in}{2.557652in}}{\pgfqpoint{8.628748in}{2.562042in}}{\pgfqpoint{8.636561in}{2.569856in}}%
\pgfpathcurveto{\pgfqpoint{8.644375in}{2.577669in}}{\pgfqpoint{8.648765in}{2.588268in}}{\pgfqpoint{8.648765in}{2.599318in}}%
\pgfpathcurveto{\pgfqpoint{8.648765in}{2.610369in}}{\pgfqpoint{8.644375in}{2.620968in}}{\pgfqpoint{8.636561in}{2.628781in}}%
\pgfpathcurveto{\pgfqpoint{8.628748in}{2.636595in}}{\pgfqpoint{8.618149in}{2.640985in}}{\pgfqpoint{8.607098in}{2.640985in}}%
\pgfpathcurveto{\pgfqpoint{8.596048in}{2.640985in}}{\pgfqpoint{8.585449in}{2.636595in}}{\pgfqpoint{8.577636in}{2.628781in}}%
\pgfpathcurveto{\pgfqpoint{8.569822in}{2.620968in}}{\pgfqpoint{8.565432in}{2.610369in}}{\pgfqpoint{8.565432in}{2.599318in}}%
\pgfpathcurveto{\pgfqpoint{8.565432in}{2.588268in}}{\pgfqpoint{8.569822in}{2.577669in}}{\pgfqpoint{8.577636in}{2.569856in}}%
\pgfpathcurveto{\pgfqpoint{8.585449in}{2.562042in}}{\pgfqpoint{8.596048in}{2.557652in}}{\pgfqpoint{8.607098in}{2.557652in}}%
\pgfpathlineto{\pgfqpoint{8.607098in}{2.557652in}}%
\pgfpathclose%
\pgfusepath{stroke}%
\end{pgfscope}%
\begin{pgfscope}%
\pgfpathrectangle{\pgfqpoint{7.512535in}{0.437222in}}{\pgfqpoint{6.275590in}{5.159444in}}%
\pgfusepath{clip}%
\pgfsetbuttcap%
\pgfsetroundjoin%
\pgfsetlinewidth{1.003750pt}%
\definecolor{currentstroke}{rgb}{0.827451,0.827451,0.827451}%
\pgfsetstrokecolor{currentstroke}%
\pgfsetstrokeopacity{0.800000}%
\pgfsetdash{}{0pt}%
\pgfpathmoveto{\pgfqpoint{11.673162in}{5.360568in}}%
\pgfpathcurveto{\pgfqpoint{11.684212in}{5.360568in}}{\pgfqpoint{11.694811in}{5.364958in}}{\pgfqpoint{11.702624in}{5.372772in}}%
\pgfpathcurveto{\pgfqpoint{11.710438in}{5.380585in}}{\pgfqpoint{11.714828in}{5.391184in}}{\pgfqpoint{11.714828in}{5.402235in}}%
\pgfpathcurveto{\pgfqpoint{11.714828in}{5.413285in}}{\pgfqpoint{11.710438in}{5.423884in}}{\pgfqpoint{11.702624in}{5.431697in}}%
\pgfpathcurveto{\pgfqpoint{11.694811in}{5.439511in}}{\pgfqpoint{11.684212in}{5.443901in}}{\pgfqpoint{11.673162in}{5.443901in}}%
\pgfpathcurveto{\pgfqpoint{11.662111in}{5.443901in}}{\pgfqpoint{11.651512in}{5.439511in}}{\pgfqpoint{11.643699in}{5.431697in}}%
\pgfpathcurveto{\pgfqpoint{11.635885in}{5.423884in}}{\pgfqpoint{11.631495in}{5.413285in}}{\pgfqpoint{11.631495in}{5.402235in}}%
\pgfpathcurveto{\pgfqpoint{11.631495in}{5.391184in}}{\pgfqpoint{11.635885in}{5.380585in}}{\pgfqpoint{11.643699in}{5.372772in}}%
\pgfpathcurveto{\pgfqpoint{11.651512in}{5.364958in}}{\pgfqpoint{11.662111in}{5.360568in}}{\pgfqpoint{11.673162in}{5.360568in}}%
\pgfpathlineto{\pgfqpoint{11.673162in}{5.360568in}}%
\pgfpathclose%
\pgfusepath{stroke}%
\end{pgfscope}%
\begin{pgfscope}%
\pgfpathrectangle{\pgfqpoint{7.512535in}{0.437222in}}{\pgfqpoint{6.275590in}{5.159444in}}%
\pgfusepath{clip}%
\pgfsetbuttcap%
\pgfsetroundjoin%
\pgfsetlinewidth{1.003750pt}%
\definecolor{currentstroke}{rgb}{0.827451,0.827451,0.827451}%
\pgfsetstrokecolor{currentstroke}%
\pgfsetstrokeopacity{0.800000}%
\pgfsetdash{}{0pt}%
\pgfpathmoveto{\pgfqpoint{9.697299in}{3.449334in}}%
\pgfpathcurveto{\pgfqpoint{9.708349in}{3.449334in}}{\pgfqpoint{9.718948in}{3.453725in}}{\pgfqpoint{9.726762in}{3.461538in}}%
\pgfpathcurveto{\pgfqpoint{9.734575in}{3.469352in}}{\pgfqpoint{9.738965in}{3.479951in}}{\pgfqpoint{9.738965in}{3.491001in}}%
\pgfpathcurveto{\pgfqpoint{9.738965in}{3.502051in}}{\pgfqpoint{9.734575in}{3.512650in}}{\pgfqpoint{9.726762in}{3.520464in}}%
\pgfpathcurveto{\pgfqpoint{9.718948in}{3.528277in}}{\pgfqpoint{9.708349in}{3.532668in}}{\pgfqpoint{9.697299in}{3.532668in}}%
\pgfpathcurveto{\pgfqpoint{9.686249in}{3.532668in}}{\pgfqpoint{9.675650in}{3.528277in}}{\pgfqpoint{9.667836in}{3.520464in}}%
\pgfpathcurveto{\pgfqpoint{9.660022in}{3.512650in}}{\pgfqpoint{9.655632in}{3.502051in}}{\pgfqpoint{9.655632in}{3.491001in}}%
\pgfpathcurveto{\pgfqpoint{9.655632in}{3.479951in}}{\pgfqpoint{9.660022in}{3.469352in}}{\pgfqpoint{9.667836in}{3.461538in}}%
\pgfpathcurveto{\pgfqpoint{9.675650in}{3.453725in}}{\pgfqpoint{9.686249in}{3.449334in}}{\pgfqpoint{9.697299in}{3.449334in}}%
\pgfpathlineto{\pgfqpoint{9.697299in}{3.449334in}}%
\pgfpathclose%
\pgfusepath{stroke}%
\end{pgfscope}%
\begin{pgfscope}%
\pgfpathrectangle{\pgfqpoint{7.512535in}{0.437222in}}{\pgfqpoint{6.275590in}{5.159444in}}%
\pgfusepath{clip}%
\pgfsetbuttcap%
\pgfsetroundjoin%
\pgfsetlinewidth{1.003750pt}%
\definecolor{currentstroke}{rgb}{0.827451,0.827451,0.827451}%
\pgfsetstrokecolor{currentstroke}%
\pgfsetstrokeopacity{0.800000}%
\pgfsetdash{}{0pt}%
\pgfpathmoveto{\pgfqpoint{8.506716in}{1.320145in}}%
\pgfpathcurveto{\pgfqpoint{8.517766in}{1.320145in}}{\pgfqpoint{8.528366in}{1.324535in}}{\pgfqpoint{8.536179in}{1.332349in}}%
\pgfpathcurveto{\pgfqpoint{8.543993in}{1.340162in}}{\pgfqpoint{8.548383in}{1.350761in}}{\pgfqpoint{8.548383in}{1.361811in}}%
\pgfpathcurveto{\pgfqpoint{8.548383in}{1.372862in}}{\pgfqpoint{8.543993in}{1.383461in}}{\pgfqpoint{8.536179in}{1.391274in}}%
\pgfpathcurveto{\pgfqpoint{8.528366in}{1.399088in}}{\pgfqpoint{8.517766in}{1.403478in}}{\pgfqpoint{8.506716in}{1.403478in}}%
\pgfpathcurveto{\pgfqpoint{8.495666in}{1.403478in}}{\pgfqpoint{8.485067in}{1.399088in}}{\pgfqpoint{8.477254in}{1.391274in}}%
\pgfpathcurveto{\pgfqpoint{8.469440in}{1.383461in}}{\pgfqpoint{8.465050in}{1.372862in}}{\pgfqpoint{8.465050in}{1.361811in}}%
\pgfpathcurveto{\pgfqpoint{8.465050in}{1.350761in}}{\pgfqpoint{8.469440in}{1.340162in}}{\pgfqpoint{8.477254in}{1.332349in}}%
\pgfpathcurveto{\pgfqpoint{8.485067in}{1.324535in}}{\pgfqpoint{8.495666in}{1.320145in}}{\pgfqpoint{8.506716in}{1.320145in}}%
\pgfpathlineto{\pgfqpoint{8.506716in}{1.320145in}}%
\pgfpathclose%
\pgfusepath{stroke}%
\end{pgfscope}%
\begin{pgfscope}%
\pgfpathrectangle{\pgfqpoint{7.512535in}{0.437222in}}{\pgfqpoint{6.275590in}{5.159444in}}%
\pgfusepath{clip}%
\pgfsetbuttcap%
\pgfsetroundjoin%
\pgfsetlinewidth{1.003750pt}%
\definecolor{currentstroke}{rgb}{0.827451,0.827451,0.827451}%
\pgfsetstrokecolor{currentstroke}%
\pgfsetstrokeopacity{0.800000}%
\pgfsetdash{}{0pt}%
\pgfpathmoveto{\pgfqpoint{11.629363in}{5.304399in}}%
\pgfpathcurveto{\pgfqpoint{11.640413in}{5.304399in}}{\pgfqpoint{11.651012in}{5.308790in}}{\pgfqpoint{11.658826in}{5.316603in}}%
\pgfpathcurveto{\pgfqpoint{11.666639in}{5.324417in}}{\pgfqpoint{11.671030in}{5.335016in}}{\pgfqpoint{11.671030in}{5.346066in}}%
\pgfpathcurveto{\pgfqpoint{11.671030in}{5.357116in}}{\pgfqpoint{11.666639in}{5.367715in}}{\pgfqpoint{11.658826in}{5.375529in}}%
\pgfpathcurveto{\pgfqpoint{11.651012in}{5.383342in}}{\pgfqpoint{11.640413in}{5.387733in}}{\pgfqpoint{11.629363in}{5.387733in}}%
\pgfpathcurveto{\pgfqpoint{11.618313in}{5.387733in}}{\pgfqpoint{11.607714in}{5.383342in}}{\pgfqpoint{11.599900in}{5.375529in}}%
\pgfpathcurveto{\pgfqpoint{11.592087in}{5.367715in}}{\pgfqpoint{11.587696in}{5.357116in}}{\pgfqpoint{11.587696in}{5.346066in}}%
\pgfpathcurveto{\pgfqpoint{11.587696in}{5.335016in}}{\pgfqpoint{11.592087in}{5.324417in}}{\pgfqpoint{11.599900in}{5.316603in}}%
\pgfpathcurveto{\pgfqpoint{11.607714in}{5.308790in}}{\pgfqpoint{11.618313in}{5.304399in}}{\pgfqpoint{11.629363in}{5.304399in}}%
\pgfpathlineto{\pgfqpoint{11.629363in}{5.304399in}}%
\pgfpathclose%
\pgfusepath{stroke}%
\end{pgfscope}%
\begin{pgfscope}%
\pgfpathrectangle{\pgfqpoint{7.512535in}{0.437222in}}{\pgfqpoint{6.275590in}{5.159444in}}%
\pgfusepath{clip}%
\pgfsetbuttcap%
\pgfsetroundjoin%
\pgfsetlinewidth{1.003750pt}%
\definecolor{currentstroke}{rgb}{0.827451,0.827451,0.827451}%
\pgfsetstrokecolor{currentstroke}%
\pgfsetstrokeopacity{0.800000}%
\pgfsetdash{}{0pt}%
\pgfpathmoveto{\pgfqpoint{8.069438in}{1.804601in}}%
\pgfpathcurveto{\pgfqpoint{8.080488in}{1.804601in}}{\pgfqpoint{8.091087in}{1.808991in}}{\pgfqpoint{8.098901in}{1.816805in}}%
\pgfpathcurveto{\pgfqpoint{8.106715in}{1.824618in}}{\pgfqpoint{8.111105in}{1.835217in}}{\pgfqpoint{8.111105in}{1.846268in}}%
\pgfpathcurveto{\pgfqpoint{8.111105in}{1.857318in}}{\pgfqpoint{8.106715in}{1.867917in}}{\pgfqpoint{8.098901in}{1.875730in}}%
\pgfpathcurveto{\pgfqpoint{8.091087in}{1.883544in}}{\pgfqpoint{8.080488in}{1.887934in}}{\pgfqpoint{8.069438in}{1.887934in}}%
\pgfpathcurveto{\pgfqpoint{8.058388in}{1.887934in}}{\pgfqpoint{8.047789in}{1.883544in}}{\pgfqpoint{8.039975in}{1.875730in}}%
\pgfpathcurveto{\pgfqpoint{8.032162in}{1.867917in}}{\pgfqpoint{8.027772in}{1.857318in}}{\pgfqpoint{8.027772in}{1.846268in}}%
\pgfpathcurveto{\pgfqpoint{8.027772in}{1.835217in}}{\pgfqpoint{8.032162in}{1.824618in}}{\pgfqpoint{8.039975in}{1.816805in}}%
\pgfpathcurveto{\pgfqpoint{8.047789in}{1.808991in}}{\pgfqpoint{8.058388in}{1.804601in}}{\pgfqpoint{8.069438in}{1.804601in}}%
\pgfpathlineto{\pgfqpoint{8.069438in}{1.804601in}}%
\pgfpathclose%
\pgfusepath{stroke}%
\end{pgfscope}%
\begin{pgfscope}%
\pgfpathrectangle{\pgfqpoint{7.512535in}{0.437222in}}{\pgfqpoint{6.275590in}{5.159444in}}%
\pgfusepath{clip}%
\pgfsetbuttcap%
\pgfsetroundjoin%
\pgfsetlinewidth{1.003750pt}%
\definecolor{currentstroke}{rgb}{0.827451,0.827451,0.827451}%
\pgfsetstrokecolor{currentstroke}%
\pgfsetstrokeopacity{0.800000}%
\pgfsetdash{}{0pt}%
\pgfpathmoveto{\pgfqpoint{13.617857in}{5.351910in}}%
\pgfpathcurveto{\pgfqpoint{13.628907in}{5.351910in}}{\pgfqpoint{13.639506in}{5.356301in}}{\pgfqpoint{13.647320in}{5.364114in}}%
\pgfpathcurveto{\pgfqpoint{13.655134in}{5.371928in}}{\pgfqpoint{13.659524in}{5.382527in}}{\pgfqpoint{13.659524in}{5.393577in}}%
\pgfpathcurveto{\pgfqpoint{13.659524in}{5.404627in}}{\pgfqpoint{13.655134in}{5.415226in}}{\pgfqpoint{13.647320in}{5.423040in}}%
\pgfpathcurveto{\pgfqpoint{13.639506in}{5.430853in}}{\pgfqpoint{13.628907in}{5.435244in}}{\pgfqpoint{13.617857in}{5.435244in}}%
\pgfpathcurveto{\pgfqpoint{13.606807in}{5.435244in}}{\pgfqpoint{13.596208in}{5.430853in}}{\pgfqpoint{13.588394in}{5.423040in}}%
\pgfpathcurveto{\pgfqpoint{13.580581in}{5.415226in}}{\pgfqpoint{13.576190in}{5.404627in}}{\pgfqpoint{13.576190in}{5.393577in}}%
\pgfpathcurveto{\pgfqpoint{13.576190in}{5.382527in}}{\pgfqpoint{13.580581in}{5.371928in}}{\pgfqpoint{13.588394in}{5.364114in}}%
\pgfpathcurveto{\pgfqpoint{13.596208in}{5.356301in}}{\pgfqpoint{13.606807in}{5.351910in}}{\pgfqpoint{13.617857in}{5.351910in}}%
\pgfpathlineto{\pgfqpoint{13.617857in}{5.351910in}}%
\pgfpathclose%
\pgfusepath{stroke}%
\end{pgfscope}%
\begin{pgfscope}%
\pgfpathrectangle{\pgfqpoint{7.512535in}{0.437222in}}{\pgfqpoint{6.275590in}{5.159444in}}%
\pgfusepath{clip}%
\pgfsetbuttcap%
\pgfsetroundjoin%
\pgfsetlinewidth{1.003750pt}%
\definecolor{currentstroke}{rgb}{0.827451,0.827451,0.827451}%
\pgfsetstrokecolor{currentstroke}%
\pgfsetstrokeopacity{0.800000}%
\pgfsetdash{}{0pt}%
\pgfpathmoveto{\pgfqpoint{13.056622in}{5.186596in}}%
\pgfpathcurveto{\pgfqpoint{13.067673in}{5.186596in}}{\pgfqpoint{13.078272in}{5.190986in}}{\pgfqpoint{13.086085in}{5.198800in}}%
\pgfpathcurveto{\pgfqpoint{13.093899in}{5.206613in}}{\pgfqpoint{13.098289in}{5.217212in}}{\pgfqpoint{13.098289in}{5.228262in}}%
\pgfpathcurveto{\pgfqpoint{13.098289in}{5.239313in}}{\pgfqpoint{13.093899in}{5.249912in}}{\pgfqpoint{13.086085in}{5.257725in}}%
\pgfpathcurveto{\pgfqpoint{13.078272in}{5.265539in}}{\pgfqpoint{13.067673in}{5.269929in}}{\pgfqpoint{13.056622in}{5.269929in}}%
\pgfpathcurveto{\pgfqpoint{13.045572in}{5.269929in}}{\pgfqpoint{13.034973in}{5.265539in}}{\pgfqpoint{13.027160in}{5.257725in}}%
\pgfpathcurveto{\pgfqpoint{13.019346in}{5.249912in}}{\pgfqpoint{13.014956in}{5.239313in}}{\pgfqpoint{13.014956in}{5.228262in}}%
\pgfpathcurveto{\pgfqpoint{13.014956in}{5.217212in}}{\pgfqpoint{13.019346in}{5.206613in}}{\pgfqpoint{13.027160in}{5.198800in}}%
\pgfpathcurveto{\pgfqpoint{13.034973in}{5.190986in}}{\pgfqpoint{13.045572in}{5.186596in}}{\pgfqpoint{13.056622in}{5.186596in}}%
\pgfpathlineto{\pgfqpoint{13.056622in}{5.186596in}}%
\pgfpathclose%
\pgfusepath{stroke}%
\end{pgfscope}%
\begin{pgfscope}%
\pgfpathrectangle{\pgfqpoint{7.512535in}{0.437222in}}{\pgfqpoint{6.275590in}{5.159444in}}%
\pgfusepath{clip}%
\pgfsetbuttcap%
\pgfsetroundjoin%
\pgfsetlinewidth{1.003750pt}%
\definecolor{currentstroke}{rgb}{0.827451,0.827451,0.827451}%
\pgfsetstrokecolor{currentstroke}%
\pgfsetstrokeopacity{0.800000}%
\pgfsetdash{}{0pt}%
\pgfpathmoveto{\pgfqpoint{9.822909in}{5.355498in}}%
\pgfpathcurveto{\pgfqpoint{9.833959in}{5.355498in}}{\pgfqpoint{9.844558in}{5.359888in}}{\pgfqpoint{9.852371in}{5.367701in}}%
\pgfpathcurveto{\pgfqpoint{9.860185in}{5.375515in}}{\pgfqpoint{9.864575in}{5.386114in}}{\pgfqpoint{9.864575in}{5.397164in}}%
\pgfpathcurveto{\pgfqpoint{9.864575in}{5.408214in}}{\pgfqpoint{9.860185in}{5.418813in}}{\pgfqpoint{9.852371in}{5.426627in}}%
\pgfpathcurveto{\pgfqpoint{9.844558in}{5.434441in}}{\pgfqpoint{9.833959in}{5.438831in}}{\pgfqpoint{9.822909in}{5.438831in}}%
\pgfpathcurveto{\pgfqpoint{9.811858in}{5.438831in}}{\pgfqpoint{9.801259in}{5.434441in}}{\pgfqpoint{9.793446in}{5.426627in}}%
\pgfpathcurveto{\pgfqpoint{9.785632in}{5.418813in}}{\pgfqpoint{9.781242in}{5.408214in}}{\pgfqpoint{9.781242in}{5.397164in}}%
\pgfpathcurveto{\pgfqpoint{9.781242in}{5.386114in}}{\pgfqpoint{9.785632in}{5.375515in}}{\pgfqpoint{9.793446in}{5.367701in}}%
\pgfpathcurveto{\pgfqpoint{9.801259in}{5.359888in}}{\pgfqpoint{9.811858in}{5.355498in}}{\pgfqpoint{9.822909in}{5.355498in}}%
\pgfpathlineto{\pgfqpoint{9.822909in}{5.355498in}}%
\pgfpathclose%
\pgfusepath{stroke}%
\end{pgfscope}%
\begin{pgfscope}%
\pgfpathrectangle{\pgfqpoint{7.512535in}{0.437222in}}{\pgfqpoint{6.275590in}{5.159444in}}%
\pgfusepath{clip}%
\pgfsetbuttcap%
\pgfsetroundjoin%
\pgfsetlinewidth{1.003750pt}%
\definecolor{currentstroke}{rgb}{0.827451,0.827451,0.827451}%
\pgfsetstrokecolor{currentstroke}%
\pgfsetstrokeopacity{0.800000}%
\pgfsetdash{}{0pt}%
\pgfpathmoveto{\pgfqpoint{11.416246in}{3.857316in}}%
\pgfpathcurveto{\pgfqpoint{11.427296in}{3.857316in}}{\pgfqpoint{11.437895in}{3.861707in}}{\pgfqpoint{11.445709in}{3.869520in}}%
\pgfpathcurveto{\pgfqpoint{11.453522in}{3.877334in}}{\pgfqpoint{11.457913in}{3.887933in}}{\pgfqpoint{11.457913in}{3.898983in}}%
\pgfpathcurveto{\pgfqpoint{11.457913in}{3.910033in}}{\pgfqpoint{11.453522in}{3.920632in}}{\pgfqpoint{11.445709in}{3.928446in}}%
\pgfpathcurveto{\pgfqpoint{11.437895in}{3.936259in}}{\pgfqpoint{11.427296in}{3.940650in}}{\pgfqpoint{11.416246in}{3.940650in}}%
\pgfpathcurveto{\pgfqpoint{11.405196in}{3.940650in}}{\pgfqpoint{11.394597in}{3.936259in}}{\pgfqpoint{11.386783in}{3.928446in}}%
\pgfpathcurveto{\pgfqpoint{11.378970in}{3.920632in}}{\pgfqpoint{11.374579in}{3.910033in}}{\pgfqpoint{11.374579in}{3.898983in}}%
\pgfpathcurveto{\pgfqpoint{11.374579in}{3.887933in}}{\pgfqpoint{11.378970in}{3.877334in}}{\pgfqpoint{11.386783in}{3.869520in}}%
\pgfpathcurveto{\pgfqpoint{11.394597in}{3.861707in}}{\pgfqpoint{11.405196in}{3.857316in}}{\pgfqpoint{11.416246in}{3.857316in}}%
\pgfpathlineto{\pgfqpoint{11.416246in}{3.857316in}}%
\pgfpathclose%
\pgfusepath{stroke}%
\end{pgfscope}%
\begin{pgfscope}%
\pgfpathrectangle{\pgfqpoint{7.512535in}{0.437222in}}{\pgfqpoint{6.275590in}{5.159444in}}%
\pgfusepath{clip}%
\pgfsetbuttcap%
\pgfsetroundjoin%
\pgfsetlinewidth{1.003750pt}%
\definecolor{currentstroke}{rgb}{0.827451,0.827451,0.827451}%
\pgfsetstrokecolor{currentstroke}%
\pgfsetstrokeopacity{0.800000}%
\pgfsetdash{}{0pt}%
\pgfpathmoveto{\pgfqpoint{8.714432in}{3.605291in}}%
\pgfpathcurveto{\pgfqpoint{8.725482in}{3.605291in}}{\pgfqpoint{8.736081in}{3.609681in}}{\pgfqpoint{8.743895in}{3.617495in}}%
\pgfpathcurveto{\pgfqpoint{8.751709in}{3.625309in}}{\pgfqpoint{8.756099in}{3.635908in}}{\pgfqpoint{8.756099in}{3.646958in}}%
\pgfpathcurveto{\pgfqpoint{8.756099in}{3.658008in}}{\pgfqpoint{8.751709in}{3.668607in}}{\pgfqpoint{8.743895in}{3.676421in}}%
\pgfpathcurveto{\pgfqpoint{8.736081in}{3.684234in}}{\pgfqpoint{8.725482in}{3.688625in}}{\pgfqpoint{8.714432in}{3.688625in}}%
\pgfpathcurveto{\pgfqpoint{8.703382in}{3.688625in}}{\pgfqpoint{8.692783in}{3.684234in}}{\pgfqpoint{8.684970in}{3.676421in}}%
\pgfpathcurveto{\pgfqpoint{8.677156in}{3.668607in}}{\pgfqpoint{8.672766in}{3.658008in}}{\pgfqpoint{8.672766in}{3.646958in}}%
\pgfpathcurveto{\pgfqpoint{8.672766in}{3.635908in}}{\pgfqpoint{8.677156in}{3.625309in}}{\pgfqpoint{8.684970in}{3.617495in}}%
\pgfpathcurveto{\pgfqpoint{8.692783in}{3.609681in}}{\pgfqpoint{8.703382in}{3.605291in}}{\pgfqpoint{8.714432in}{3.605291in}}%
\pgfpathlineto{\pgfqpoint{8.714432in}{3.605291in}}%
\pgfpathclose%
\pgfusepath{stroke}%
\end{pgfscope}%
\begin{pgfscope}%
\pgfpathrectangle{\pgfqpoint{7.512535in}{0.437222in}}{\pgfqpoint{6.275590in}{5.159444in}}%
\pgfusepath{clip}%
\pgfsetbuttcap%
\pgfsetroundjoin%
\pgfsetlinewidth{1.003750pt}%
\definecolor{currentstroke}{rgb}{0.827451,0.827451,0.827451}%
\pgfsetstrokecolor{currentstroke}%
\pgfsetstrokeopacity{0.800000}%
\pgfsetdash{}{0pt}%
\pgfpathmoveto{\pgfqpoint{9.116910in}{0.954383in}}%
\pgfpathcurveto{\pgfqpoint{9.127960in}{0.954383in}}{\pgfqpoint{9.138559in}{0.958773in}}{\pgfqpoint{9.146373in}{0.966587in}}%
\pgfpathcurveto{\pgfqpoint{9.154186in}{0.974400in}}{\pgfqpoint{9.158577in}{0.984999in}}{\pgfqpoint{9.158577in}{0.996049in}}%
\pgfpathcurveto{\pgfqpoint{9.158577in}{1.007099in}}{\pgfqpoint{9.154186in}{1.017699in}}{\pgfqpoint{9.146373in}{1.025512in}}%
\pgfpathcurveto{\pgfqpoint{9.138559in}{1.033326in}}{\pgfqpoint{9.127960in}{1.037716in}}{\pgfqpoint{9.116910in}{1.037716in}}%
\pgfpathcurveto{\pgfqpoint{9.105860in}{1.037716in}}{\pgfqpoint{9.095261in}{1.033326in}}{\pgfqpoint{9.087447in}{1.025512in}}%
\pgfpathcurveto{\pgfqpoint{9.079634in}{1.017699in}}{\pgfqpoint{9.075243in}{1.007099in}}{\pgfqpoint{9.075243in}{0.996049in}}%
\pgfpathcurveto{\pgfqpoint{9.075243in}{0.984999in}}{\pgfqpoint{9.079634in}{0.974400in}}{\pgfqpoint{9.087447in}{0.966587in}}%
\pgfpathcurveto{\pgfqpoint{9.095261in}{0.958773in}}{\pgfqpoint{9.105860in}{0.954383in}}{\pgfqpoint{9.116910in}{0.954383in}}%
\pgfpathlineto{\pgfqpoint{9.116910in}{0.954383in}}%
\pgfpathclose%
\pgfusepath{stroke}%
\end{pgfscope}%
\begin{pgfscope}%
\pgfpathrectangle{\pgfqpoint{7.512535in}{0.437222in}}{\pgfqpoint{6.275590in}{5.159444in}}%
\pgfusepath{clip}%
\pgfsetbuttcap%
\pgfsetroundjoin%
\pgfsetlinewidth{1.003750pt}%
\definecolor{currentstroke}{rgb}{0.827451,0.827451,0.827451}%
\pgfsetstrokecolor{currentstroke}%
\pgfsetstrokeopacity{0.800000}%
\pgfsetdash{}{0pt}%
\pgfpathmoveto{\pgfqpoint{11.388601in}{4.546614in}}%
\pgfpathcurveto{\pgfqpoint{11.399651in}{4.546614in}}{\pgfqpoint{11.410250in}{4.551005in}}{\pgfqpoint{11.418064in}{4.558818in}}%
\pgfpathcurveto{\pgfqpoint{11.425878in}{4.566632in}}{\pgfqpoint{11.430268in}{4.577231in}}{\pgfqpoint{11.430268in}{4.588281in}}%
\pgfpathcurveto{\pgfqpoint{11.430268in}{4.599331in}}{\pgfqpoint{11.425878in}{4.609930in}}{\pgfqpoint{11.418064in}{4.617744in}}%
\pgfpathcurveto{\pgfqpoint{11.410250in}{4.625558in}}{\pgfqpoint{11.399651in}{4.629948in}}{\pgfqpoint{11.388601in}{4.629948in}}%
\pgfpathcurveto{\pgfqpoint{11.377551in}{4.629948in}}{\pgfqpoint{11.366952in}{4.625558in}}{\pgfqpoint{11.359139in}{4.617744in}}%
\pgfpathcurveto{\pgfqpoint{11.351325in}{4.609930in}}{\pgfqpoint{11.346935in}{4.599331in}}{\pgfqpoint{11.346935in}{4.588281in}}%
\pgfpathcurveto{\pgfqpoint{11.346935in}{4.577231in}}{\pgfqpoint{11.351325in}{4.566632in}}{\pgfqpoint{11.359139in}{4.558818in}}%
\pgfpathcurveto{\pgfqpoint{11.366952in}{4.551005in}}{\pgfqpoint{11.377551in}{4.546614in}}{\pgfqpoint{11.388601in}{4.546614in}}%
\pgfpathlineto{\pgfqpoint{11.388601in}{4.546614in}}%
\pgfpathclose%
\pgfusepath{stroke}%
\end{pgfscope}%
\begin{pgfscope}%
\pgfpathrectangle{\pgfqpoint{7.512535in}{0.437222in}}{\pgfqpoint{6.275590in}{5.159444in}}%
\pgfusepath{clip}%
\pgfsetbuttcap%
\pgfsetroundjoin%
\pgfsetlinewidth{1.003750pt}%
\definecolor{currentstroke}{rgb}{0.827451,0.827451,0.827451}%
\pgfsetstrokecolor{currentstroke}%
\pgfsetstrokeopacity{0.800000}%
\pgfsetdash{}{0pt}%
\pgfpathmoveto{\pgfqpoint{12.547906in}{5.274085in}}%
\pgfpathcurveto{\pgfqpoint{12.558956in}{5.274085in}}{\pgfqpoint{12.569556in}{5.278475in}}{\pgfqpoint{12.577369in}{5.286289in}}%
\pgfpathcurveto{\pgfqpoint{12.585183in}{5.294102in}}{\pgfqpoint{12.589573in}{5.304701in}}{\pgfqpoint{12.589573in}{5.315752in}}%
\pgfpathcurveto{\pgfqpoint{12.589573in}{5.326802in}}{\pgfqpoint{12.585183in}{5.337401in}}{\pgfqpoint{12.577369in}{5.345214in}}%
\pgfpathcurveto{\pgfqpoint{12.569556in}{5.353028in}}{\pgfqpoint{12.558956in}{5.357418in}}{\pgfqpoint{12.547906in}{5.357418in}}%
\pgfpathcurveto{\pgfqpoint{12.536856in}{5.357418in}}{\pgfqpoint{12.526257in}{5.353028in}}{\pgfqpoint{12.518444in}{5.345214in}}%
\pgfpathcurveto{\pgfqpoint{12.510630in}{5.337401in}}{\pgfqpoint{12.506240in}{5.326802in}}{\pgfqpoint{12.506240in}{5.315752in}}%
\pgfpathcurveto{\pgfqpoint{12.506240in}{5.304701in}}{\pgfqpoint{12.510630in}{5.294102in}}{\pgfqpoint{12.518444in}{5.286289in}}%
\pgfpathcurveto{\pgfqpoint{12.526257in}{5.278475in}}{\pgfqpoint{12.536856in}{5.274085in}}{\pgfqpoint{12.547906in}{5.274085in}}%
\pgfpathlineto{\pgfqpoint{12.547906in}{5.274085in}}%
\pgfpathclose%
\pgfusepath{stroke}%
\end{pgfscope}%
\begin{pgfscope}%
\pgfpathrectangle{\pgfqpoint{7.512535in}{0.437222in}}{\pgfqpoint{6.275590in}{5.159444in}}%
\pgfusepath{clip}%
\pgfsetbuttcap%
\pgfsetroundjoin%
\pgfsetlinewidth{1.003750pt}%
\definecolor{currentstroke}{rgb}{0.827451,0.827451,0.827451}%
\pgfsetstrokecolor{currentstroke}%
\pgfsetstrokeopacity{0.800000}%
\pgfsetdash{}{0pt}%
\pgfpathmoveto{\pgfqpoint{7.679894in}{0.493855in}}%
\pgfpathcurveto{\pgfqpoint{7.690944in}{0.493855in}}{\pgfqpoint{7.701543in}{0.498245in}}{\pgfqpoint{7.709357in}{0.506059in}}%
\pgfpathcurveto{\pgfqpoint{7.717170in}{0.513872in}}{\pgfqpoint{7.721560in}{0.524471in}}{\pgfqpoint{7.721560in}{0.535522in}}%
\pgfpathcurveto{\pgfqpoint{7.721560in}{0.546572in}}{\pgfqpoint{7.717170in}{0.557171in}}{\pgfqpoint{7.709357in}{0.564984in}}%
\pgfpathcurveto{\pgfqpoint{7.701543in}{0.572798in}}{\pgfqpoint{7.690944in}{0.577188in}}{\pgfqpoint{7.679894in}{0.577188in}}%
\pgfpathcurveto{\pgfqpoint{7.668844in}{0.577188in}}{\pgfqpoint{7.658245in}{0.572798in}}{\pgfqpoint{7.650431in}{0.564984in}}%
\pgfpathcurveto{\pgfqpoint{7.642617in}{0.557171in}}{\pgfqpoint{7.638227in}{0.546572in}}{\pgfqpoint{7.638227in}{0.535522in}}%
\pgfpathcurveto{\pgfqpoint{7.638227in}{0.524471in}}{\pgfqpoint{7.642617in}{0.513872in}}{\pgfqpoint{7.650431in}{0.506059in}}%
\pgfpathcurveto{\pgfqpoint{7.658245in}{0.498245in}}{\pgfqpoint{7.668844in}{0.493855in}}{\pgfqpoint{7.679894in}{0.493855in}}%
\pgfpathlineto{\pgfqpoint{7.679894in}{0.493855in}}%
\pgfpathclose%
\pgfusepath{stroke}%
\end{pgfscope}%
\begin{pgfscope}%
\pgfpathrectangle{\pgfqpoint{7.512535in}{0.437222in}}{\pgfqpoint{6.275590in}{5.159444in}}%
\pgfusepath{clip}%
\pgfsetbuttcap%
\pgfsetroundjoin%
\pgfsetlinewidth{1.003750pt}%
\definecolor{currentstroke}{rgb}{0.827451,0.827451,0.827451}%
\pgfsetstrokecolor{currentstroke}%
\pgfsetstrokeopacity{0.800000}%
\pgfsetdash{}{0pt}%
\pgfpathmoveto{\pgfqpoint{13.468750in}{5.529360in}}%
\pgfpathcurveto{\pgfqpoint{13.479800in}{5.529360in}}{\pgfqpoint{13.490399in}{5.533750in}}{\pgfqpoint{13.498212in}{5.541564in}}%
\pgfpathcurveto{\pgfqpoint{13.506026in}{5.549378in}}{\pgfqpoint{13.510416in}{5.559977in}}{\pgfqpoint{13.510416in}{5.571027in}}%
\pgfpathcurveto{\pgfqpoint{13.510416in}{5.582077in}}{\pgfqpoint{13.506026in}{5.592676in}}{\pgfqpoint{13.498212in}{5.600490in}}%
\pgfpathcurveto{\pgfqpoint{13.490399in}{5.608303in}}{\pgfqpoint{13.479800in}{5.612694in}}{\pgfqpoint{13.468750in}{5.612694in}}%
\pgfpathcurveto{\pgfqpoint{13.457699in}{5.612694in}}{\pgfqpoint{13.447100in}{5.608303in}}{\pgfqpoint{13.439287in}{5.600490in}}%
\pgfpathcurveto{\pgfqpoint{13.431473in}{5.592676in}}{\pgfqpoint{13.427083in}{5.582077in}}{\pgfqpoint{13.427083in}{5.571027in}}%
\pgfpathcurveto{\pgfqpoint{13.427083in}{5.559977in}}{\pgfqpoint{13.431473in}{5.549378in}}{\pgfqpoint{13.439287in}{5.541564in}}%
\pgfpathcurveto{\pgfqpoint{13.447100in}{5.533750in}}{\pgfqpoint{13.457699in}{5.529360in}}{\pgfqpoint{13.468750in}{5.529360in}}%
\pgfpathlineto{\pgfqpoint{13.468750in}{5.529360in}}%
\pgfpathclose%
\pgfusepath{stroke}%
\end{pgfscope}%
\begin{pgfscope}%
\pgfpathrectangle{\pgfqpoint{7.512535in}{0.437222in}}{\pgfqpoint{6.275590in}{5.159444in}}%
\pgfusepath{clip}%
\pgfsetbuttcap%
\pgfsetroundjoin%
\pgfsetlinewidth{1.003750pt}%
\definecolor{currentstroke}{rgb}{0.827451,0.827451,0.827451}%
\pgfsetstrokecolor{currentstroke}%
\pgfsetstrokeopacity{0.800000}%
\pgfsetdash{}{0pt}%
\pgfpathmoveto{\pgfqpoint{12.196637in}{5.403515in}}%
\pgfpathcurveto{\pgfqpoint{12.207687in}{5.403515in}}{\pgfqpoint{12.218287in}{5.407905in}}{\pgfqpoint{12.226100in}{5.415719in}}%
\pgfpathcurveto{\pgfqpoint{12.233914in}{5.423532in}}{\pgfqpoint{12.238304in}{5.434131in}}{\pgfqpoint{12.238304in}{5.445182in}}%
\pgfpathcurveto{\pgfqpoint{12.238304in}{5.456232in}}{\pgfqpoint{12.233914in}{5.466831in}}{\pgfqpoint{12.226100in}{5.474644in}}%
\pgfpathcurveto{\pgfqpoint{12.218287in}{5.482458in}}{\pgfqpoint{12.207687in}{5.486848in}}{\pgfqpoint{12.196637in}{5.486848in}}%
\pgfpathcurveto{\pgfqpoint{12.185587in}{5.486848in}}{\pgfqpoint{12.174988in}{5.482458in}}{\pgfqpoint{12.167175in}{5.474644in}}%
\pgfpathcurveto{\pgfqpoint{12.159361in}{5.466831in}}{\pgfqpoint{12.154971in}{5.456232in}}{\pgfqpoint{12.154971in}{5.445182in}}%
\pgfpathcurveto{\pgfqpoint{12.154971in}{5.434131in}}{\pgfqpoint{12.159361in}{5.423532in}}{\pgfqpoint{12.167175in}{5.415719in}}%
\pgfpathcurveto{\pgfqpoint{12.174988in}{5.407905in}}{\pgfqpoint{12.185587in}{5.403515in}}{\pgfqpoint{12.196637in}{5.403515in}}%
\pgfpathlineto{\pgfqpoint{12.196637in}{5.403515in}}%
\pgfpathclose%
\pgfusepath{stroke}%
\end{pgfscope}%
\begin{pgfscope}%
\pgfpathrectangle{\pgfqpoint{7.512535in}{0.437222in}}{\pgfqpoint{6.275590in}{5.159444in}}%
\pgfusepath{clip}%
\pgfsetbuttcap%
\pgfsetroundjoin%
\pgfsetlinewidth{1.003750pt}%
\definecolor{currentstroke}{rgb}{0.827451,0.827451,0.827451}%
\pgfsetstrokecolor{currentstroke}%
\pgfsetstrokeopacity{0.800000}%
\pgfsetdash{}{0pt}%
\pgfpathmoveto{\pgfqpoint{7.972802in}{0.679358in}}%
\pgfpathcurveto{\pgfqpoint{7.983853in}{0.679358in}}{\pgfqpoint{7.994452in}{0.683748in}}{\pgfqpoint{8.002265in}{0.691562in}}%
\pgfpathcurveto{\pgfqpoint{8.010079in}{0.699376in}}{\pgfqpoint{8.014469in}{0.709975in}}{\pgfqpoint{8.014469in}{0.721025in}}%
\pgfpathcurveto{\pgfqpoint{8.014469in}{0.732075in}}{\pgfqpoint{8.010079in}{0.742674in}}{\pgfqpoint{8.002265in}{0.750487in}}%
\pgfpathcurveto{\pgfqpoint{7.994452in}{0.758301in}}{\pgfqpoint{7.983853in}{0.762691in}}{\pgfqpoint{7.972802in}{0.762691in}}%
\pgfpathcurveto{\pgfqpoint{7.961752in}{0.762691in}}{\pgfqpoint{7.951153in}{0.758301in}}{\pgfqpoint{7.943340in}{0.750487in}}%
\pgfpathcurveto{\pgfqpoint{7.935526in}{0.742674in}}{\pgfqpoint{7.931136in}{0.732075in}}{\pgfqpoint{7.931136in}{0.721025in}}%
\pgfpathcurveto{\pgfqpoint{7.931136in}{0.709975in}}{\pgfqpoint{7.935526in}{0.699376in}}{\pgfqpoint{7.943340in}{0.691562in}}%
\pgfpathcurveto{\pgfqpoint{7.951153in}{0.683748in}}{\pgfqpoint{7.961752in}{0.679358in}}{\pgfqpoint{7.972802in}{0.679358in}}%
\pgfpathlineto{\pgfqpoint{7.972802in}{0.679358in}}%
\pgfpathclose%
\pgfusepath{stroke}%
\end{pgfscope}%
\begin{pgfscope}%
\pgfpathrectangle{\pgfqpoint{7.512535in}{0.437222in}}{\pgfqpoint{6.275590in}{5.159444in}}%
\pgfusepath{clip}%
\pgfsetbuttcap%
\pgfsetroundjoin%
\pgfsetlinewidth{1.003750pt}%
\definecolor{currentstroke}{rgb}{0.827451,0.827451,0.827451}%
\pgfsetstrokecolor{currentstroke}%
\pgfsetstrokeopacity{0.800000}%
\pgfsetdash{}{0pt}%
\pgfpathmoveto{\pgfqpoint{8.800029in}{1.838807in}}%
\pgfpathcurveto{\pgfqpoint{8.811079in}{1.838807in}}{\pgfqpoint{8.821678in}{1.843198in}}{\pgfqpoint{8.829491in}{1.851011in}}%
\pgfpathcurveto{\pgfqpoint{8.837305in}{1.858825in}}{\pgfqpoint{8.841695in}{1.869424in}}{\pgfqpoint{8.841695in}{1.880474in}}%
\pgfpathcurveto{\pgfqpoint{8.841695in}{1.891524in}}{\pgfqpoint{8.837305in}{1.902123in}}{\pgfqpoint{8.829491in}{1.909937in}}%
\pgfpathcurveto{\pgfqpoint{8.821678in}{1.917750in}}{\pgfqpoint{8.811079in}{1.922141in}}{\pgfqpoint{8.800029in}{1.922141in}}%
\pgfpathcurveto{\pgfqpoint{8.788978in}{1.922141in}}{\pgfqpoint{8.778379in}{1.917750in}}{\pgfqpoint{8.770566in}{1.909937in}}%
\pgfpathcurveto{\pgfqpoint{8.762752in}{1.902123in}}{\pgfqpoint{8.758362in}{1.891524in}}{\pgfqpoint{8.758362in}{1.880474in}}%
\pgfpathcurveto{\pgfqpoint{8.758362in}{1.869424in}}{\pgfqpoint{8.762752in}{1.858825in}}{\pgfqpoint{8.770566in}{1.851011in}}%
\pgfpathcurveto{\pgfqpoint{8.778379in}{1.843198in}}{\pgfqpoint{8.788978in}{1.838807in}}{\pgfqpoint{8.800029in}{1.838807in}}%
\pgfpathlineto{\pgfqpoint{8.800029in}{1.838807in}}%
\pgfpathclose%
\pgfusepath{stroke}%
\end{pgfscope}%
\begin{pgfscope}%
\pgfpathrectangle{\pgfqpoint{7.512535in}{0.437222in}}{\pgfqpoint{6.275590in}{5.159444in}}%
\pgfusepath{clip}%
\pgfsetbuttcap%
\pgfsetroundjoin%
\pgfsetlinewidth{1.003750pt}%
\definecolor{currentstroke}{rgb}{0.827451,0.827451,0.827451}%
\pgfsetstrokecolor{currentstroke}%
\pgfsetstrokeopacity{0.800000}%
\pgfsetdash{}{0pt}%
\pgfpathmoveto{\pgfqpoint{13.273901in}{5.501584in}}%
\pgfpathcurveto{\pgfqpoint{13.284951in}{5.501584in}}{\pgfqpoint{13.295550in}{5.505974in}}{\pgfqpoint{13.303364in}{5.513787in}}%
\pgfpathcurveto{\pgfqpoint{13.311177in}{5.521601in}}{\pgfqpoint{13.315568in}{5.532200in}}{\pgfqpoint{13.315568in}{5.543250in}}%
\pgfpathcurveto{\pgfqpoint{13.315568in}{5.554300in}}{\pgfqpoint{13.311177in}{5.564899in}}{\pgfqpoint{13.303364in}{5.572713in}}%
\pgfpathcurveto{\pgfqpoint{13.295550in}{5.580527in}}{\pgfqpoint{13.284951in}{5.584917in}}{\pgfqpoint{13.273901in}{5.584917in}}%
\pgfpathcurveto{\pgfqpoint{13.262851in}{5.584917in}}{\pgfqpoint{13.252252in}{5.580527in}}{\pgfqpoint{13.244438in}{5.572713in}}%
\pgfpathcurveto{\pgfqpoint{13.236625in}{5.564899in}}{\pgfqpoint{13.232234in}{5.554300in}}{\pgfqpoint{13.232234in}{5.543250in}}%
\pgfpathcurveto{\pgfqpoint{13.232234in}{5.532200in}}{\pgfqpoint{13.236625in}{5.521601in}}{\pgfqpoint{13.244438in}{5.513787in}}%
\pgfpathcurveto{\pgfqpoint{13.252252in}{5.505974in}}{\pgfqpoint{13.262851in}{5.501584in}}{\pgfqpoint{13.273901in}{5.501584in}}%
\pgfpathlineto{\pgfqpoint{13.273901in}{5.501584in}}%
\pgfpathclose%
\pgfusepath{stroke}%
\end{pgfscope}%
\begin{pgfscope}%
\pgfpathrectangle{\pgfqpoint{7.512535in}{0.437222in}}{\pgfqpoint{6.275590in}{5.159444in}}%
\pgfusepath{clip}%
\pgfsetbuttcap%
\pgfsetroundjoin%
\pgfsetlinewidth{1.003750pt}%
\definecolor{currentstroke}{rgb}{0.827451,0.827451,0.827451}%
\pgfsetstrokecolor{currentstroke}%
\pgfsetstrokeopacity{0.800000}%
\pgfsetdash{}{0pt}%
\pgfpathmoveto{\pgfqpoint{12.873966in}{5.431219in}}%
\pgfpathcurveto{\pgfqpoint{12.885016in}{5.431219in}}{\pgfqpoint{12.895615in}{5.435610in}}{\pgfqpoint{12.903429in}{5.443423in}}%
\pgfpathcurveto{\pgfqpoint{12.911243in}{5.451237in}}{\pgfqpoint{12.915633in}{5.461836in}}{\pgfqpoint{12.915633in}{5.472886in}}%
\pgfpathcurveto{\pgfqpoint{12.915633in}{5.483936in}}{\pgfqpoint{12.911243in}{5.494535in}}{\pgfqpoint{12.903429in}{5.502349in}}%
\pgfpathcurveto{\pgfqpoint{12.895615in}{5.510163in}}{\pgfqpoint{12.885016in}{5.514553in}}{\pgfqpoint{12.873966in}{5.514553in}}%
\pgfpathcurveto{\pgfqpoint{12.862916in}{5.514553in}}{\pgfqpoint{12.852317in}{5.510163in}}{\pgfqpoint{12.844503in}{5.502349in}}%
\pgfpathcurveto{\pgfqpoint{12.836690in}{5.494535in}}{\pgfqpoint{12.832300in}{5.483936in}}{\pgfqpoint{12.832300in}{5.472886in}}%
\pgfpathcurveto{\pgfqpoint{12.832300in}{5.461836in}}{\pgfqpoint{12.836690in}{5.451237in}}{\pgfqpoint{12.844503in}{5.443423in}}%
\pgfpathcurveto{\pgfqpoint{12.852317in}{5.435610in}}{\pgfqpoint{12.862916in}{5.431219in}}{\pgfqpoint{12.873966in}{5.431219in}}%
\pgfpathlineto{\pgfqpoint{12.873966in}{5.431219in}}%
\pgfpathclose%
\pgfusepath{stroke}%
\end{pgfscope}%
\begin{pgfscope}%
\pgfpathrectangle{\pgfqpoint{7.512535in}{0.437222in}}{\pgfqpoint{6.275590in}{5.159444in}}%
\pgfusepath{clip}%
\pgfsetbuttcap%
\pgfsetroundjoin%
\pgfsetlinewidth{1.003750pt}%
\definecolor{currentstroke}{rgb}{0.827451,0.827451,0.827451}%
\pgfsetstrokecolor{currentstroke}%
\pgfsetstrokeopacity{0.800000}%
\pgfsetdash{}{0pt}%
\pgfpathmoveto{\pgfqpoint{8.607098in}{2.557652in}}%
\pgfpathcurveto{\pgfqpoint{8.618149in}{2.557652in}}{\pgfqpoint{8.628748in}{2.562042in}}{\pgfqpoint{8.636561in}{2.569856in}}%
\pgfpathcurveto{\pgfqpoint{8.644375in}{2.577669in}}{\pgfqpoint{8.648765in}{2.588268in}}{\pgfqpoint{8.648765in}{2.599318in}}%
\pgfpathcurveto{\pgfqpoint{8.648765in}{2.610369in}}{\pgfqpoint{8.644375in}{2.620968in}}{\pgfqpoint{8.636561in}{2.628781in}}%
\pgfpathcurveto{\pgfqpoint{8.628748in}{2.636595in}}{\pgfqpoint{8.618149in}{2.640985in}}{\pgfqpoint{8.607098in}{2.640985in}}%
\pgfpathcurveto{\pgfqpoint{8.596048in}{2.640985in}}{\pgfqpoint{8.585449in}{2.636595in}}{\pgfqpoint{8.577636in}{2.628781in}}%
\pgfpathcurveto{\pgfqpoint{8.569822in}{2.620968in}}{\pgfqpoint{8.565432in}{2.610369in}}{\pgfqpoint{8.565432in}{2.599318in}}%
\pgfpathcurveto{\pgfqpoint{8.565432in}{2.588268in}}{\pgfqpoint{8.569822in}{2.577669in}}{\pgfqpoint{8.577636in}{2.569856in}}%
\pgfpathcurveto{\pgfqpoint{8.585449in}{2.562042in}}{\pgfqpoint{8.596048in}{2.557652in}}{\pgfqpoint{8.607098in}{2.557652in}}%
\pgfpathlineto{\pgfqpoint{8.607098in}{2.557652in}}%
\pgfpathclose%
\pgfusepath{stroke}%
\end{pgfscope}%
\begin{pgfscope}%
\pgfpathrectangle{\pgfqpoint{7.512535in}{0.437222in}}{\pgfqpoint{6.275590in}{5.159444in}}%
\pgfusepath{clip}%
\pgfsetbuttcap%
\pgfsetroundjoin%
\pgfsetlinewidth{1.003750pt}%
\definecolor{currentstroke}{rgb}{0.827451,0.827451,0.827451}%
\pgfsetstrokecolor{currentstroke}%
\pgfsetstrokeopacity{0.800000}%
\pgfsetdash{}{0pt}%
\pgfpathmoveto{\pgfqpoint{9.140923in}{2.556537in}}%
\pgfpathcurveto{\pgfqpoint{9.151974in}{2.556537in}}{\pgfqpoint{9.162573in}{2.560927in}}{\pgfqpoint{9.170386in}{2.568740in}}%
\pgfpathcurveto{\pgfqpoint{9.178200in}{2.576554in}}{\pgfqpoint{9.182590in}{2.587153in}}{\pgfqpoint{9.182590in}{2.598203in}}%
\pgfpathcurveto{\pgfqpoint{9.182590in}{2.609253in}}{\pgfqpoint{9.178200in}{2.619852in}}{\pgfqpoint{9.170386in}{2.627666in}}%
\pgfpathcurveto{\pgfqpoint{9.162573in}{2.635480in}}{\pgfqpoint{9.151974in}{2.639870in}}{\pgfqpoint{9.140923in}{2.639870in}}%
\pgfpathcurveto{\pgfqpoint{9.129873in}{2.639870in}}{\pgfqpoint{9.119274in}{2.635480in}}{\pgfqpoint{9.111461in}{2.627666in}}%
\pgfpathcurveto{\pgfqpoint{9.103647in}{2.619852in}}{\pgfqpoint{9.099257in}{2.609253in}}{\pgfqpoint{9.099257in}{2.598203in}}%
\pgfpathcurveto{\pgfqpoint{9.099257in}{2.587153in}}{\pgfqpoint{9.103647in}{2.576554in}}{\pgfqpoint{9.111461in}{2.568740in}}%
\pgfpathcurveto{\pgfqpoint{9.119274in}{2.560927in}}{\pgfqpoint{9.129873in}{2.556537in}}{\pgfqpoint{9.140923in}{2.556537in}}%
\pgfpathlineto{\pgfqpoint{9.140923in}{2.556537in}}%
\pgfpathclose%
\pgfusepath{stroke}%
\end{pgfscope}%
\begin{pgfscope}%
\pgfpathrectangle{\pgfqpoint{7.512535in}{0.437222in}}{\pgfqpoint{6.275590in}{5.159444in}}%
\pgfusepath{clip}%
\pgfsetbuttcap%
\pgfsetroundjoin%
\pgfsetlinewidth{1.003750pt}%
\definecolor{currentstroke}{rgb}{0.827451,0.827451,0.827451}%
\pgfsetstrokecolor{currentstroke}%
\pgfsetstrokeopacity{0.800000}%
\pgfsetdash{}{0pt}%
\pgfpathmoveto{\pgfqpoint{10.299622in}{3.250664in}}%
\pgfpathcurveto{\pgfqpoint{10.310672in}{3.250664in}}{\pgfqpoint{10.321271in}{3.255054in}}{\pgfqpoint{10.329085in}{3.262868in}}%
\pgfpathcurveto{\pgfqpoint{10.336898in}{3.270682in}}{\pgfqpoint{10.341289in}{3.281281in}}{\pgfqpoint{10.341289in}{3.292331in}}%
\pgfpathcurveto{\pgfqpoint{10.341289in}{3.303381in}}{\pgfqpoint{10.336898in}{3.313980in}}{\pgfqpoint{10.329085in}{3.321794in}}%
\pgfpathcurveto{\pgfqpoint{10.321271in}{3.329607in}}{\pgfqpoint{10.310672in}{3.333998in}}{\pgfqpoint{10.299622in}{3.333998in}}%
\pgfpathcurveto{\pgfqpoint{10.288572in}{3.333998in}}{\pgfqpoint{10.277973in}{3.329607in}}{\pgfqpoint{10.270159in}{3.321794in}}%
\pgfpathcurveto{\pgfqpoint{10.262346in}{3.313980in}}{\pgfqpoint{10.257955in}{3.303381in}}{\pgfqpoint{10.257955in}{3.292331in}}%
\pgfpathcurveto{\pgfqpoint{10.257955in}{3.281281in}}{\pgfqpoint{10.262346in}{3.270682in}}{\pgfqpoint{10.270159in}{3.262868in}}%
\pgfpathcurveto{\pgfqpoint{10.277973in}{3.255054in}}{\pgfqpoint{10.288572in}{3.250664in}}{\pgfqpoint{10.299622in}{3.250664in}}%
\pgfpathlineto{\pgfqpoint{10.299622in}{3.250664in}}%
\pgfpathclose%
\pgfusepath{stroke}%
\end{pgfscope}%
\begin{pgfscope}%
\pgfpathrectangle{\pgfqpoint{7.512535in}{0.437222in}}{\pgfqpoint{6.275590in}{5.159444in}}%
\pgfusepath{clip}%
\pgfsetbuttcap%
\pgfsetroundjoin%
\pgfsetlinewidth{1.003750pt}%
\definecolor{currentstroke}{rgb}{0.827451,0.827451,0.827451}%
\pgfsetstrokecolor{currentstroke}%
\pgfsetstrokeopacity{0.800000}%
\pgfsetdash{}{0pt}%
\pgfpathmoveto{\pgfqpoint{8.605492in}{2.894644in}}%
\pgfpathcurveto{\pgfqpoint{8.616542in}{2.894644in}}{\pgfqpoint{8.627141in}{2.899034in}}{\pgfqpoint{8.634955in}{2.906847in}}%
\pgfpathcurveto{\pgfqpoint{8.642768in}{2.914661in}}{\pgfqpoint{8.647158in}{2.925260in}}{\pgfqpoint{8.647158in}{2.936310in}}%
\pgfpathcurveto{\pgfqpoint{8.647158in}{2.947360in}}{\pgfqpoint{8.642768in}{2.957959in}}{\pgfqpoint{8.634955in}{2.965773in}}%
\pgfpathcurveto{\pgfqpoint{8.627141in}{2.973587in}}{\pgfqpoint{8.616542in}{2.977977in}}{\pgfqpoint{8.605492in}{2.977977in}}%
\pgfpathcurveto{\pgfqpoint{8.594442in}{2.977977in}}{\pgfqpoint{8.583843in}{2.973587in}}{\pgfqpoint{8.576029in}{2.965773in}}%
\pgfpathcurveto{\pgfqpoint{8.568215in}{2.957959in}}{\pgfqpoint{8.563825in}{2.947360in}}{\pgfqpoint{8.563825in}{2.936310in}}%
\pgfpathcurveto{\pgfqpoint{8.563825in}{2.925260in}}{\pgfqpoint{8.568215in}{2.914661in}}{\pgfqpoint{8.576029in}{2.906847in}}%
\pgfpathcurveto{\pgfqpoint{8.583843in}{2.899034in}}{\pgfqpoint{8.594442in}{2.894644in}}{\pgfqpoint{8.605492in}{2.894644in}}%
\pgfpathlineto{\pgfqpoint{8.605492in}{2.894644in}}%
\pgfpathclose%
\pgfusepath{stroke}%
\end{pgfscope}%
\begin{pgfscope}%
\pgfpathrectangle{\pgfqpoint{7.512535in}{0.437222in}}{\pgfqpoint{6.275590in}{5.159444in}}%
\pgfusepath{clip}%
\pgfsetbuttcap%
\pgfsetroundjoin%
\pgfsetlinewidth{1.003750pt}%
\definecolor{currentstroke}{rgb}{0.827451,0.827451,0.827451}%
\pgfsetstrokecolor{currentstroke}%
\pgfsetstrokeopacity{0.800000}%
\pgfsetdash{}{0pt}%
\pgfpathmoveto{\pgfqpoint{13.101307in}{5.417061in}}%
\pgfpathcurveto{\pgfqpoint{13.112357in}{5.417061in}}{\pgfqpoint{13.122956in}{5.421451in}}{\pgfqpoint{13.130770in}{5.429265in}}%
\pgfpathcurveto{\pgfqpoint{13.138583in}{5.437079in}}{\pgfqpoint{13.142973in}{5.447678in}}{\pgfqpoint{13.142973in}{5.458728in}}%
\pgfpathcurveto{\pgfqpoint{13.142973in}{5.469778in}}{\pgfqpoint{13.138583in}{5.480377in}}{\pgfqpoint{13.130770in}{5.488191in}}%
\pgfpathcurveto{\pgfqpoint{13.122956in}{5.496004in}}{\pgfqpoint{13.112357in}{5.500395in}}{\pgfqpoint{13.101307in}{5.500395in}}%
\pgfpathcurveto{\pgfqpoint{13.090257in}{5.500395in}}{\pgfqpoint{13.079658in}{5.496004in}}{\pgfqpoint{13.071844in}{5.488191in}}%
\pgfpathcurveto{\pgfqpoint{13.064030in}{5.480377in}}{\pgfqpoint{13.059640in}{5.469778in}}{\pgfqpoint{13.059640in}{5.458728in}}%
\pgfpathcurveto{\pgfqpoint{13.059640in}{5.447678in}}{\pgfqpoint{13.064030in}{5.437079in}}{\pgfqpoint{13.071844in}{5.429265in}}%
\pgfpathcurveto{\pgfqpoint{13.079658in}{5.421451in}}{\pgfqpoint{13.090257in}{5.417061in}}{\pgfqpoint{13.101307in}{5.417061in}}%
\pgfpathlineto{\pgfqpoint{13.101307in}{5.417061in}}%
\pgfpathclose%
\pgfusepath{stroke}%
\end{pgfscope}%
\begin{pgfscope}%
\pgfpathrectangle{\pgfqpoint{7.512535in}{0.437222in}}{\pgfqpoint{6.275590in}{5.159444in}}%
\pgfusepath{clip}%
\pgfsetbuttcap%
\pgfsetroundjoin%
\pgfsetlinewidth{1.003750pt}%
\definecolor{currentstroke}{rgb}{0.827451,0.827451,0.827451}%
\pgfsetstrokecolor{currentstroke}%
\pgfsetstrokeopacity{0.800000}%
\pgfsetdash{}{0pt}%
\pgfpathmoveto{\pgfqpoint{7.887873in}{1.346375in}}%
\pgfpathcurveto{\pgfqpoint{7.898923in}{1.346375in}}{\pgfqpoint{7.909522in}{1.350765in}}{\pgfqpoint{7.917336in}{1.358578in}}%
\pgfpathcurveto{\pgfqpoint{7.925149in}{1.366392in}}{\pgfqpoint{7.929540in}{1.376991in}}{\pgfqpoint{7.929540in}{1.388041in}}%
\pgfpathcurveto{\pgfqpoint{7.929540in}{1.399091in}}{\pgfqpoint{7.925149in}{1.409690in}}{\pgfqpoint{7.917336in}{1.417504in}}%
\pgfpathcurveto{\pgfqpoint{7.909522in}{1.425318in}}{\pgfqpoint{7.898923in}{1.429708in}}{\pgfqpoint{7.887873in}{1.429708in}}%
\pgfpathcurveto{\pgfqpoint{7.876823in}{1.429708in}}{\pgfqpoint{7.866224in}{1.425318in}}{\pgfqpoint{7.858410in}{1.417504in}}%
\pgfpathcurveto{\pgfqpoint{7.850596in}{1.409690in}}{\pgfqpoint{7.846206in}{1.399091in}}{\pgfqpoint{7.846206in}{1.388041in}}%
\pgfpathcurveto{\pgfqpoint{7.846206in}{1.376991in}}{\pgfqpoint{7.850596in}{1.366392in}}{\pgfqpoint{7.858410in}{1.358578in}}%
\pgfpathcurveto{\pgfqpoint{7.866224in}{1.350765in}}{\pgfqpoint{7.876823in}{1.346375in}}{\pgfqpoint{7.887873in}{1.346375in}}%
\pgfpathlineto{\pgfqpoint{7.887873in}{1.346375in}}%
\pgfpathclose%
\pgfusepath{stroke}%
\end{pgfscope}%
\begin{pgfscope}%
\pgfpathrectangle{\pgfqpoint{7.512535in}{0.437222in}}{\pgfqpoint{6.275590in}{5.159444in}}%
\pgfusepath{clip}%
\pgfsetbuttcap%
\pgfsetroundjoin%
\pgfsetlinewidth{1.003750pt}%
\definecolor{currentstroke}{rgb}{0.827451,0.827451,0.827451}%
\pgfsetstrokecolor{currentstroke}%
\pgfsetstrokeopacity{0.800000}%
\pgfsetdash{}{0pt}%
\pgfpathmoveto{\pgfqpoint{12.446113in}{5.353825in}}%
\pgfpathcurveto{\pgfqpoint{12.457163in}{5.353825in}}{\pgfqpoint{12.467762in}{5.358215in}}{\pgfqpoint{12.475576in}{5.366029in}}%
\pgfpathcurveto{\pgfqpoint{12.483389in}{5.373843in}}{\pgfqpoint{12.487779in}{5.384442in}}{\pgfqpoint{12.487779in}{5.395492in}}%
\pgfpathcurveto{\pgfqpoint{12.487779in}{5.406542in}}{\pgfqpoint{12.483389in}{5.417141in}}{\pgfqpoint{12.475576in}{5.424955in}}%
\pgfpathcurveto{\pgfqpoint{12.467762in}{5.432768in}}{\pgfqpoint{12.457163in}{5.437158in}}{\pgfqpoint{12.446113in}{5.437158in}}%
\pgfpathcurveto{\pgfqpoint{12.435063in}{5.437158in}}{\pgfqpoint{12.424464in}{5.432768in}}{\pgfqpoint{12.416650in}{5.424955in}}%
\pgfpathcurveto{\pgfqpoint{12.408836in}{5.417141in}}{\pgfqpoint{12.404446in}{5.406542in}}{\pgfqpoint{12.404446in}{5.395492in}}%
\pgfpathcurveto{\pgfqpoint{12.404446in}{5.384442in}}{\pgfqpoint{12.408836in}{5.373843in}}{\pgfqpoint{12.416650in}{5.366029in}}%
\pgfpathcurveto{\pgfqpoint{12.424464in}{5.358215in}}{\pgfqpoint{12.435063in}{5.353825in}}{\pgfqpoint{12.446113in}{5.353825in}}%
\pgfpathlineto{\pgfqpoint{12.446113in}{5.353825in}}%
\pgfpathclose%
\pgfusepath{stroke}%
\end{pgfscope}%
\begin{pgfscope}%
\pgfpathrectangle{\pgfqpoint{7.512535in}{0.437222in}}{\pgfqpoint{6.275590in}{5.159444in}}%
\pgfusepath{clip}%
\pgfsetbuttcap%
\pgfsetroundjoin%
\pgfsetlinewidth{1.003750pt}%
\definecolor{currentstroke}{rgb}{0.827451,0.827451,0.827451}%
\pgfsetstrokecolor{currentstroke}%
\pgfsetstrokeopacity{0.800000}%
\pgfsetdash{}{0pt}%
\pgfpathmoveto{\pgfqpoint{12.630641in}{5.525115in}}%
\pgfpathcurveto{\pgfqpoint{12.641691in}{5.525115in}}{\pgfqpoint{12.652290in}{5.529506in}}{\pgfqpoint{12.660104in}{5.537319in}}%
\pgfpathcurveto{\pgfqpoint{12.667917in}{5.545133in}}{\pgfqpoint{12.672308in}{5.555732in}}{\pgfqpoint{12.672308in}{5.566782in}}%
\pgfpathcurveto{\pgfqpoint{12.672308in}{5.577832in}}{\pgfqpoint{12.667917in}{5.588431in}}{\pgfqpoint{12.660104in}{5.596245in}}%
\pgfpathcurveto{\pgfqpoint{12.652290in}{5.604058in}}{\pgfqpoint{12.641691in}{5.608449in}}{\pgfqpoint{12.630641in}{5.608449in}}%
\pgfpathcurveto{\pgfqpoint{12.619591in}{5.608449in}}{\pgfqpoint{12.608992in}{5.604058in}}{\pgfqpoint{12.601178in}{5.596245in}}%
\pgfpathcurveto{\pgfqpoint{12.593365in}{5.588431in}}{\pgfqpoint{12.588974in}{5.577832in}}{\pgfqpoint{12.588974in}{5.566782in}}%
\pgfpathcurveto{\pgfqpoint{12.588974in}{5.555732in}}{\pgfqpoint{12.593365in}{5.545133in}}{\pgfqpoint{12.601178in}{5.537319in}}%
\pgfpathcurveto{\pgfqpoint{12.608992in}{5.529506in}}{\pgfqpoint{12.619591in}{5.525115in}}{\pgfqpoint{12.630641in}{5.525115in}}%
\pgfpathlineto{\pgfqpoint{12.630641in}{5.525115in}}%
\pgfpathclose%
\pgfusepath{stroke}%
\end{pgfscope}%
\begin{pgfscope}%
\pgfpathrectangle{\pgfqpoint{7.512535in}{0.437222in}}{\pgfqpoint{6.275590in}{5.159444in}}%
\pgfusepath{clip}%
\pgfsetbuttcap%
\pgfsetroundjoin%
\pgfsetlinewidth{1.003750pt}%
\definecolor{currentstroke}{rgb}{0.827451,0.827451,0.827451}%
\pgfsetstrokecolor{currentstroke}%
\pgfsetstrokeopacity{0.800000}%
\pgfsetdash{}{0pt}%
\pgfpathmoveto{\pgfqpoint{10.818589in}{5.117177in}}%
\pgfpathcurveto{\pgfqpoint{10.829639in}{5.117177in}}{\pgfqpoint{10.840238in}{5.121567in}}{\pgfqpoint{10.848052in}{5.129381in}}%
\pgfpathcurveto{\pgfqpoint{10.855865in}{5.137194in}}{\pgfqpoint{10.860255in}{5.147793in}}{\pgfqpoint{10.860255in}{5.158843in}}%
\pgfpathcurveto{\pgfqpoint{10.860255in}{5.169893in}}{\pgfqpoint{10.855865in}{5.180493in}}{\pgfqpoint{10.848052in}{5.188306in}}%
\pgfpathcurveto{\pgfqpoint{10.840238in}{5.196120in}}{\pgfqpoint{10.829639in}{5.200510in}}{\pgfqpoint{10.818589in}{5.200510in}}%
\pgfpathcurveto{\pgfqpoint{10.807539in}{5.200510in}}{\pgfqpoint{10.796940in}{5.196120in}}{\pgfqpoint{10.789126in}{5.188306in}}%
\pgfpathcurveto{\pgfqpoint{10.781312in}{5.180493in}}{\pgfqpoint{10.776922in}{5.169893in}}{\pgfqpoint{10.776922in}{5.158843in}}%
\pgfpathcurveto{\pgfqpoint{10.776922in}{5.147793in}}{\pgfqpoint{10.781312in}{5.137194in}}{\pgfqpoint{10.789126in}{5.129381in}}%
\pgfpathcurveto{\pgfqpoint{10.796940in}{5.121567in}}{\pgfqpoint{10.807539in}{5.117177in}}{\pgfqpoint{10.818589in}{5.117177in}}%
\pgfpathlineto{\pgfqpoint{10.818589in}{5.117177in}}%
\pgfpathclose%
\pgfusepath{stroke}%
\end{pgfscope}%
\begin{pgfscope}%
\pgfpathrectangle{\pgfqpoint{7.512535in}{0.437222in}}{\pgfqpoint{6.275590in}{5.159444in}}%
\pgfusepath{clip}%
\pgfsetbuttcap%
\pgfsetroundjoin%
\pgfsetlinewidth{1.003750pt}%
\definecolor{currentstroke}{rgb}{0.827451,0.827451,0.827451}%
\pgfsetstrokecolor{currentstroke}%
\pgfsetstrokeopacity{0.800000}%
\pgfsetdash{}{0pt}%
\pgfpathmoveto{\pgfqpoint{11.962704in}{5.469388in}}%
\pgfpathcurveto{\pgfqpoint{11.973755in}{5.469388in}}{\pgfqpoint{11.984354in}{5.473778in}}{\pgfqpoint{11.992167in}{5.481592in}}%
\pgfpathcurveto{\pgfqpoint{11.999981in}{5.489405in}}{\pgfqpoint{12.004371in}{5.500004in}}{\pgfqpoint{12.004371in}{5.511054in}}%
\pgfpathcurveto{\pgfqpoint{12.004371in}{5.522105in}}{\pgfqpoint{11.999981in}{5.532704in}}{\pgfqpoint{11.992167in}{5.540517in}}%
\pgfpathcurveto{\pgfqpoint{11.984354in}{5.548331in}}{\pgfqpoint{11.973755in}{5.552721in}}{\pgfqpoint{11.962704in}{5.552721in}}%
\pgfpathcurveto{\pgfqpoint{11.951654in}{5.552721in}}{\pgfqpoint{11.941055in}{5.548331in}}{\pgfqpoint{11.933242in}{5.540517in}}%
\pgfpathcurveto{\pgfqpoint{11.925428in}{5.532704in}}{\pgfqpoint{11.921038in}{5.522105in}}{\pgfqpoint{11.921038in}{5.511054in}}%
\pgfpathcurveto{\pgfqpoint{11.921038in}{5.500004in}}{\pgfqpoint{11.925428in}{5.489405in}}{\pgfqpoint{11.933242in}{5.481592in}}%
\pgfpathcurveto{\pgfqpoint{11.941055in}{5.473778in}}{\pgfqpoint{11.951654in}{5.469388in}}{\pgfqpoint{11.962704in}{5.469388in}}%
\pgfpathlineto{\pgfqpoint{11.962704in}{5.469388in}}%
\pgfpathclose%
\pgfusepath{stroke}%
\end{pgfscope}%
\begin{pgfscope}%
\pgfpathrectangle{\pgfqpoint{7.512535in}{0.437222in}}{\pgfqpoint{6.275590in}{5.159444in}}%
\pgfusepath{clip}%
\pgfsetbuttcap%
\pgfsetroundjoin%
\pgfsetlinewidth{1.003750pt}%
\definecolor{currentstroke}{rgb}{0.827451,0.827451,0.827451}%
\pgfsetstrokecolor{currentstroke}%
\pgfsetstrokeopacity{0.800000}%
\pgfsetdash{}{0pt}%
\pgfpathmoveto{\pgfqpoint{9.582235in}{1.661429in}}%
\pgfpathcurveto{\pgfqpoint{9.593285in}{1.661429in}}{\pgfqpoint{9.603884in}{1.665819in}}{\pgfqpoint{9.611698in}{1.673633in}}%
\pgfpathcurveto{\pgfqpoint{9.619511in}{1.681446in}}{\pgfqpoint{9.623902in}{1.692045in}}{\pgfqpoint{9.623902in}{1.703095in}}%
\pgfpathcurveto{\pgfqpoint{9.623902in}{1.714146in}}{\pgfqpoint{9.619511in}{1.724745in}}{\pgfqpoint{9.611698in}{1.732558in}}%
\pgfpathcurveto{\pgfqpoint{9.603884in}{1.740372in}}{\pgfqpoint{9.593285in}{1.744762in}}{\pgfqpoint{9.582235in}{1.744762in}}%
\pgfpathcurveto{\pgfqpoint{9.571185in}{1.744762in}}{\pgfqpoint{9.560586in}{1.740372in}}{\pgfqpoint{9.552772in}{1.732558in}}%
\pgfpathcurveto{\pgfqpoint{9.544959in}{1.724745in}}{\pgfqpoint{9.540568in}{1.714146in}}{\pgfqpoint{9.540568in}{1.703095in}}%
\pgfpathcurveto{\pgfqpoint{9.540568in}{1.692045in}}{\pgfqpoint{9.544959in}{1.681446in}}{\pgfqpoint{9.552772in}{1.673633in}}%
\pgfpathcurveto{\pgfqpoint{9.560586in}{1.665819in}}{\pgfqpoint{9.571185in}{1.661429in}}{\pgfqpoint{9.582235in}{1.661429in}}%
\pgfpathlineto{\pgfqpoint{9.582235in}{1.661429in}}%
\pgfpathclose%
\pgfusepath{stroke}%
\end{pgfscope}%
\begin{pgfscope}%
\pgfpathrectangle{\pgfqpoint{7.512535in}{0.437222in}}{\pgfqpoint{6.275590in}{5.159444in}}%
\pgfusepath{clip}%
\pgfsetbuttcap%
\pgfsetroundjoin%
\pgfsetlinewidth{1.003750pt}%
\definecolor{currentstroke}{rgb}{0.827451,0.827451,0.827451}%
\pgfsetstrokecolor{currentstroke}%
\pgfsetstrokeopacity{0.800000}%
\pgfsetdash{}{0pt}%
\pgfpathmoveto{\pgfqpoint{12.072745in}{5.039509in}}%
\pgfpathcurveto{\pgfqpoint{12.083796in}{5.039509in}}{\pgfqpoint{12.094395in}{5.043899in}}{\pgfqpoint{12.102208in}{5.051713in}}%
\pgfpathcurveto{\pgfqpoint{12.110022in}{5.059526in}}{\pgfqpoint{12.114412in}{5.070125in}}{\pgfqpoint{12.114412in}{5.081175in}}%
\pgfpathcurveto{\pgfqpoint{12.114412in}{5.092226in}}{\pgfqpoint{12.110022in}{5.102825in}}{\pgfqpoint{12.102208in}{5.110638in}}%
\pgfpathcurveto{\pgfqpoint{12.094395in}{5.118452in}}{\pgfqpoint{12.083796in}{5.122842in}}{\pgfqpoint{12.072745in}{5.122842in}}%
\pgfpathcurveto{\pgfqpoint{12.061695in}{5.122842in}}{\pgfqpoint{12.051096in}{5.118452in}}{\pgfqpoint{12.043283in}{5.110638in}}%
\pgfpathcurveto{\pgfqpoint{12.035469in}{5.102825in}}{\pgfqpoint{12.031079in}{5.092226in}}{\pgfqpoint{12.031079in}{5.081175in}}%
\pgfpathcurveto{\pgfqpoint{12.031079in}{5.070125in}}{\pgfqpoint{12.035469in}{5.059526in}}{\pgfqpoint{12.043283in}{5.051713in}}%
\pgfpathcurveto{\pgfqpoint{12.051096in}{5.043899in}}{\pgfqpoint{12.061695in}{5.039509in}}{\pgfqpoint{12.072745in}{5.039509in}}%
\pgfpathlineto{\pgfqpoint{12.072745in}{5.039509in}}%
\pgfpathclose%
\pgfusepath{stroke}%
\end{pgfscope}%
\begin{pgfscope}%
\pgfpathrectangle{\pgfqpoint{7.512535in}{0.437222in}}{\pgfqpoint{6.275590in}{5.159444in}}%
\pgfusepath{clip}%
\pgfsetbuttcap%
\pgfsetroundjoin%
\pgfsetlinewidth{1.003750pt}%
\definecolor{currentstroke}{rgb}{0.827451,0.827451,0.827451}%
\pgfsetstrokecolor{currentstroke}%
\pgfsetstrokeopacity{0.800000}%
\pgfsetdash{}{0pt}%
\pgfpathmoveto{\pgfqpoint{11.046560in}{5.315419in}}%
\pgfpathcurveto{\pgfqpoint{11.057610in}{5.315419in}}{\pgfqpoint{11.068209in}{5.319810in}}{\pgfqpoint{11.076023in}{5.327623in}}%
\pgfpathcurveto{\pgfqpoint{11.083837in}{5.335437in}}{\pgfqpoint{11.088227in}{5.346036in}}{\pgfqpoint{11.088227in}{5.357086in}}%
\pgfpathcurveto{\pgfqpoint{11.088227in}{5.368136in}}{\pgfqpoint{11.083837in}{5.378735in}}{\pgfqpoint{11.076023in}{5.386549in}}%
\pgfpathcurveto{\pgfqpoint{11.068209in}{5.394362in}}{\pgfqpoint{11.057610in}{5.398753in}}{\pgfqpoint{11.046560in}{5.398753in}}%
\pgfpathcurveto{\pgfqpoint{11.035510in}{5.398753in}}{\pgfqpoint{11.024911in}{5.394362in}}{\pgfqpoint{11.017098in}{5.386549in}}%
\pgfpathcurveto{\pgfqpoint{11.009284in}{5.378735in}}{\pgfqpoint{11.004894in}{5.368136in}}{\pgfqpoint{11.004894in}{5.357086in}}%
\pgfpathcurveto{\pgfqpoint{11.004894in}{5.346036in}}{\pgfqpoint{11.009284in}{5.335437in}}{\pgfqpoint{11.017098in}{5.327623in}}%
\pgfpathcurveto{\pgfqpoint{11.024911in}{5.319810in}}{\pgfqpoint{11.035510in}{5.315419in}}{\pgfqpoint{11.046560in}{5.315419in}}%
\pgfpathlineto{\pgfqpoint{11.046560in}{5.315419in}}%
\pgfpathclose%
\pgfusepath{stroke}%
\end{pgfscope}%
\begin{pgfscope}%
\pgfpathrectangle{\pgfqpoint{7.512535in}{0.437222in}}{\pgfqpoint{6.275590in}{5.159444in}}%
\pgfusepath{clip}%
\pgfsetbuttcap%
\pgfsetroundjoin%
\pgfsetlinewidth{1.003750pt}%
\definecolor{currentstroke}{rgb}{0.827451,0.827451,0.827451}%
\pgfsetstrokecolor{currentstroke}%
\pgfsetstrokeopacity{0.800000}%
\pgfsetdash{}{0pt}%
\pgfpathmoveto{\pgfqpoint{8.716039in}{3.385314in}}%
\pgfpathcurveto{\pgfqpoint{8.727089in}{3.385314in}}{\pgfqpoint{8.737688in}{3.389704in}}{\pgfqpoint{8.745502in}{3.397518in}}%
\pgfpathcurveto{\pgfqpoint{8.753315in}{3.405332in}}{\pgfqpoint{8.757706in}{3.415931in}}{\pgfqpoint{8.757706in}{3.426981in}}%
\pgfpathcurveto{\pgfqpoint{8.757706in}{3.438031in}}{\pgfqpoint{8.753315in}{3.448630in}}{\pgfqpoint{8.745502in}{3.456444in}}%
\pgfpathcurveto{\pgfqpoint{8.737688in}{3.464257in}}{\pgfqpoint{8.727089in}{3.468647in}}{\pgfqpoint{8.716039in}{3.468647in}}%
\pgfpathcurveto{\pgfqpoint{8.704989in}{3.468647in}}{\pgfqpoint{8.694390in}{3.464257in}}{\pgfqpoint{8.686576in}{3.456444in}}%
\pgfpathcurveto{\pgfqpoint{8.678763in}{3.448630in}}{\pgfqpoint{8.674372in}{3.438031in}}{\pgfqpoint{8.674372in}{3.426981in}}%
\pgfpathcurveto{\pgfqpoint{8.674372in}{3.415931in}}{\pgfqpoint{8.678763in}{3.405332in}}{\pgfqpoint{8.686576in}{3.397518in}}%
\pgfpathcurveto{\pgfqpoint{8.694390in}{3.389704in}}{\pgfqpoint{8.704989in}{3.385314in}}{\pgfqpoint{8.716039in}{3.385314in}}%
\pgfpathlineto{\pgfqpoint{8.716039in}{3.385314in}}%
\pgfpathclose%
\pgfusepath{stroke}%
\end{pgfscope}%
\begin{pgfscope}%
\pgfpathrectangle{\pgfqpoint{7.512535in}{0.437222in}}{\pgfqpoint{6.275590in}{5.159444in}}%
\pgfusepath{clip}%
\pgfsetbuttcap%
\pgfsetroundjoin%
\pgfsetlinewidth{1.003750pt}%
\definecolor{currentstroke}{rgb}{0.827451,0.827451,0.827451}%
\pgfsetstrokecolor{currentstroke}%
\pgfsetstrokeopacity{0.800000}%
\pgfsetdash{}{0pt}%
\pgfpathmoveto{\pgfqpoint{10.287229in}{5.087121in}}%
\pgfpathcurveto{\pgfqpoint{10.298279in}{5.087121in}}{\pgfqpoint{10.308878in}{5.091511in}}{\pgfqpoint{10.316692in}{5.099325in}}%
\pgfpathcurveto{\pgfqpoint{10.324505in}{5.107138in}}{\pgfqpoint{10.328896in}{5.117737in}}{\pgfqpoint{10.328896in}{5.128787in}}%
\pgfpathcurveto{\pgfqpoint{10.328896in}{5.139838in}}{\pgfqpoint{10.324505in}{5.150437in}}{\pgfqpoint{10.316692in}{5.158250in}}%
\pgfpathcurveto{\pgfqpoint{10.308878in}{5.166064in}}{\pgfqpoint{10.298279in}{5.170454in}}{\pgfqpoint{10.287229in}{5.170454in}}%
\pgfpathcurveto{\pgfqpoint{10.276179in}{5.170454in}}{\pgfqpoint{10.265580in}{5.166064in}}{\pgfqpoint{10.257766in}{5.158250in}}%
\pgfpathcurveto{\pgfqpoint{10.249953in}{5.150437in}}{\pgfqpoint{10.245562in}{5.139838in}}{\pgfqpoint{10.245562in}{5.128787in}}%
\pgfpathcurveto{\pgfqpoint{10.245562in}{5.117737in}}{\pgfqpoint{10.249953in}{5.107138in}}{\pgfqpoint{10.257766in}{5.099325in}}%
\pgfpathcurveto{\pgfqpoint{10.265580in}{5.091511in}}{\pgfqpoint{10.276179in}{5.087121in}}{\pgfqpoint{10.287229in}{5.087121in}}%
\pgfpathlineto{\pgfqpoint{10.287229in}{5.087121in}}%
\pgfpathclose%
\pgfusepath{stroke}%
\end{pgfscope}%
\begin{pgfscope}%
\pgfpathrectangle{\pgfqpoint{7.512535in}{0.437222in}}{\pgfqpoint{6.275590in}{5.159444in}}%
\pgfusepath{clip}%
\pgfsetbuttcap%
\pgfsetroundjoin%
\pgfsetlinewidth{1.003750pt}%
\definecolor{currentstroke}{rgb}{0.827451,0.827451,0.827451}%
\pgfsetstrokecolor{currentstroke}%
\pgfsetstrokeopacity{0.800000}%
\pgfsetdash{}{0pt}%
\pgfpathmoveto{\pgfqpoint{10.156793in}{3.250664in}}%
\pgfpathcurveto{\pgfqpoint{10.167843in}{3.250664in}}{\pgfqpoint{10.178442in}{3.255054in}}{\pgfqpoint{10.186256in}{3.262868in}}%
\pgfpathcurveto{\pgfqpoint{10.194069in}{3.270682in}}{\pgfqpoint{10.198460in}{3.281281in}}{\pgfqpoint{10.198460in}{3.292331in}}%
\pgfpathcurveto{\pgfqpoint{10.198460in}{3.303381in}}{\pgfqpoint{10.194069in}{3.313980in}}{\pgfqpoint{10.186256in}{3.321794in}}%
\pgfpathcurveto{\pgfqpoint{10.178442in}{3.329607in}}{\pgfqpoint{10.167843in}{3.333998in}}{\pgfqpoint{10.156793in}{3.333998in}}%
\pgfpathcurveto{\pgfqpoint{10.145743in}{3.333998in}}{\pgfqpoint{10.135144in}{3.329607in}}{\pgfqpoint{10.127330in}{3.321794in}}%
\pgfpathcurveto{\pgfqpoint{10.119517in}{3.313980in}}{\pgfqpoint{10.115126in}{3.303381in}}{\pgfqpoint{10.115126in}{3.292331in}}%
\pgfpathcurveto{\pgfqpoint{10.115126in}{3.281281in}}{\pgfqpoint{10.119517in}{3.270682in}}{\pgfqpoint{10.127330in}{3.262868in}}%
\pgfpathcurveto{\pgfqpoint{10.135144in}{3.255054in}}{\pgfqpoint{10.145743in}{3.250664in}}{\pgfqpoint{10.156793in}{3.250664in}}%
\pgfpathlineto{\pgfqpoint{10.156793in}{3.250664in}}%
\pgfpathclose%
\pgfusepath{stroke}%
\end{pgfscope}%
\begin{pgfscope}%
\pgfpathrectangle{\pgfqpoint{7.512535in}{0.437222in}}{\pgfqpoint{6.275590in}{5.159444in}}%
\pgfusepath{clip}%
\pgfsetbuttcap%
\pgfsetroundjoin%
\pgfsetlinewidth{1.003750pt}%
\definecolor{currentstroke}{rgb}{0.827451,0.827451,0.827451}%
\pgfsetstrokecolor{currentstroke}%
\pgfsetstrokeopacity{0.800000}%
\pgfsetdash{}{0pt}%
\pgfpathmoveto{\pgfqpoint{8.044541in}{2.113663in}}%
\pgfpathcurveto{\pgfqpoint{8.055591in}{2.113663in}}{\pgfqpoint{8.066190in}{2.118054in}}{\pgfqpoint{8.074003in}{2.125867in}}%
\pgfpathcurveto{\pgfqpoint{8.081817in}{2.133681in}}{\pgfqpoint{8.086207in}{2.144280in}}{\pgfqpoint{8.086207in}{2.155330in}}%
\pgfpathcurveto{\pgfqpoint{8.086207in}{2.166380in}}{\pgfqpoint{8.081817in}{2.176979in}}{\pgfqpoint{8.074003in}{2.184793in}}%
\pgfpathcurveto{\pgfqpoint{8.066190in}{2.192606in}}{\pgfqpoint{8.055591in}{2.196997in}}{\pgfqpoint{8.044541in}{2.196997in}}%
\pgfpathcurveto{\pgfqpoint{8.033491in}{2.196997in}}{\pgfqpoint{8.022891in}{2.192606in}}{\pgfqpoint{8.015078in}{2.184793in}}%
\pgfpathcurveto{\pgfqpoint{8.007264in}{2.176979in}}{\pgfqpoint{8.002874in}{2.166380in}}{\pgfqpoint{8.002874in}{2.155330in}}%
\pgfpathcurveto{\pgfqpoint{8.002874in}{2.144280in}}{\pgfqpoint{8.007264in}{2.133681in}}{\pgfqpoint{8.015078in}{2.125867in}}%
\pgfpathcurveto{\pgfqpoint{8.022891in}{2.118054in}}{\pgfqpoint{8.033491in}{2.113663in}}{\pgfqpoint{8.044541in}{2.113663in}}%
\pgfpathlineto{\pgfqpoint{8.044541in}{2.113663in}}%
\pgfpathclose%
\pgfusepath{stroke}%
\end{pgfscope}%
\begin{pgfscope}%
\pgfpathrectangle{\pgfqpoint{7.512535in}{0.437222in}}{\pgfqpoint{6.275590in}{5.159444in}}%
\pgfusepath{clip}%
\pgfsetbuttcap%
\pgfsetroundjoin%
\pgfsetlinewidth{1.003750pt}%
\definecolor{currentstroke}{rgb}{0.827451,0.827451,0.827451}%
\pgfsetstrokecolor{currentstroke}%
\pgfsetstrokeopacity{0.800000}%
\pgfsetdash{}{0pt}%
\pgfpathmoveto{\pgfqpoint{9.130854in}{1.867691in}}%
\pgfpathcurveto{\pgfqpoint{9.141904in}{1.867691in}}{\pgfqpoint{9.152503in}{1.872081in}}{\pgfqpoint{9.160317in}{1.879895in}}%
\pgfpathcurveto{\pgfqpoint{9.168130in}{1.887709in}}{\pgfqpoint{9.172521in}{1.898308in}}{\pgfqpoint{9.172521in}{1.909358in}}%
\pgfpathcurveto{\pgfqpoint{9.172521in}{1.920408in}}{\pgfqpoint{9.168130in}{1.931007in}}{\pgfqpoint{9.160317in}{1.938821in}}%
\pgfpathcurveto{\pgfqpoint{9.152503in}{1.946634in}}{\pgfqpoint{9.141904in}{1.951024in}}{\pgfqpoint{9.130854in}{1.951024in}}%
\pgfpathcurveto{\pgfqpoint{9.119804in}{1.951024in}}{\pgfqpoint{9.109205in}{1.946634in}}{\pgfqpoint{9.101391in}{1.938821in}}%
\pgfpathcurveto{\pgfqpoint{9.093577in}{1.931007in}}{\pgfqpoint{9.089187in}{1.920408in}}{\pgfqpoint{9.089187in}{1.909358in}}%
\pgfpathcurveto{\pgfqpoint{9.089187in}{1.898308in}}{\pgfqpoint{9.093577in}{1.887709in}}{\pgfqpoint{9.101391in}{1.879895in}}%
\pgfpathcurveto{\pgfqpoint{9.109205in}{1.872081in}}{\pgfqpoint{9.119804in}{1.867691in}}{\pgfqpoint{9.130854in}{1.867691in}}%
\pgfpathlineto{\pgfqpoint{9.130854in}{1.867691in}}%
\pgfpathclose%
\pgfusepath{stroke}%
\end{pgfscope}%
\begin{pgfscope}%
\pgfpathrectangle{\pgfqpoint{7.512535in}{0.437222in}}{\pgfqpoint{6.275590in}{5.159444in}}%
\pgfusepath{clip}%
\pgfsetbuttcap%
\pgfsetroundjoin%
\pgfsetlinewidth{1.003750pt}%
\definecolor{currentstroke}{rgb}{0.827451,0.827451,0.827451}%
\pgfsetstrokecolor{currentstroke}%
\pgfsetstrokeopacity{0.800000}%
\pgfsetdash{}{0pt}%
\pgfpathmoveto{\pgfqpoint{10.412487in}{3.527290in}}%
\pgfpathcurveto{\pgfqpoint{10.423537in}{3.527290in}}{\pgfqpoint{10.434136in}{3.531681in}}{\pgfqpoint{10.441950in}{3.539494in}}%
\pgfpathcurveto{\pgfqpoint{10.449764in}{3.547308in}}{\pgfqpoint{10.454154in}{3.557907in}}{\pgfqpoint{10.454154in}{3.568957in}}%
\pgfpathcurveto{\pgfqpoint{10.454154in}{3.580007in}}{\pgfqpoint{10.449764in}{3.590606in}}{\pgfqpoint{10.441950in}{3.598420in}}%
\pgfpathcurveto{\pgfqpoint{10.434136in}{3.606234in}}{\pgfqpoint{10.423537in}{3.610624in}}{\pgfqpoint{10.412487in}{3.610624in}}%
\pgfpathcurveto{\pgfqpoint{10.401437in}{3.610624in}}{\pgfqpoint{10.390838in}{3.606234in}}{\pgfqpoint{10.383025in}{3.598420in}}%
\pgfpathcurveto{\pgfqpoint{10.375211in}{3.590606in}}{\pgfqpoint{10.370821in}{3.580007in}}{\pgfqpoint{10.370821in}{3.568957in}}%
\pgfpathcurveto{\pgfqpoint{10.370821in}{3.557907in}}{\pgfqpoint{10.375211in}{3.547308in}}{\pgfqpoint{10.383025in}{3.539494in}}%
\pgfpathcurveto{\pgfqpoint{10.390838in}{3.531681in}}{\pgfqpoint{10.401437in}{3.527290in}}{\pgfqpoint{10.412487in}{3.527290in}}%
\pgfpathlineto{\pgfqpoint{10.412487in}{3.527290in}}%
\pgfpathclose%
\pgfusepath{stroke}%
\end{pgfscope}%
\begin{pgfscope}%
\pgfpathrectangle{\pgfqpoint{7.512535in}{0.437222in}}{\pgfqpoint{6.275590in}{5.159444in}}%
\pgfusepath{clip}%
\pgfsetbuttcap%
\pgfsetroundjoin%
\pgfsetlinewidth{1.003750pt}%
\definecolor{currentstroke}{rgb}{0.827451,0.827451,0.827451}%
\pgfsetstrokecolor{currentstroke}%
\pgfsetstrokeopacity{0.800000}%
\pgfsetdash{}{0pt}%
\pgfpathmoveto{\pgfqpoint{10.827454in}{5.304399in}}%
\pgfpathcurveto{\pgfqpoint{10.838504in}{5.304399in}}{\pgfqpoint{10.849103in}{5.308790in}}{\pgfqpoint{10.856917in}{5.316603in}}%
\pgfpathcurveto{\pgfqpoint{10.864730in}{5.324417in}}{\pgfqpoint{10.869121in}{5.335016in}}{\pgfqpoint{10.869121in}{5.346066in}}%
\pgfpathcurveto{\pgfqpoint{10.869121in}{5.357116in}}{\pgfqpoint{10.864730in}{5.367715in}}{\pgfqpoint{10.856917in}{5.375529in}}%
\pgfpathcurveto{\pgfqpoint{10.849103in}{5.383342in}}{\pgfqpoint{10.838504in}{5.387733in}}{\pgfqpoint{10.827454in}{5.387733in}}%
\pgfpathcurveto{\pgfqpoint{10.816404in}{5.387733in}}{\pgfqpoint{10.805805in}{5.383342in}}{\pgfqpoint{10.797991in}{5.375529in}}%
\pgfpathcurveto{\pgfqpoint{10.790177in}{5.367715in}}{\pgfqpoint{10.785787in}{5.357116in}}{\pgfqpoint{10.785787in}{5.346066in}}%
\pgfpathcurveto{\pgfqpoint{10.785787in}{5.335016in}}{\pgfqpoint{10.790177in}{5.324417in}}{\pgfqpoint{10.797991in}{5.316603in}}%
\pgfpathcurveto{\pgfqpoint{10.805805in}{5.308790in}}{\pgfqpoint{10.816404in}{5.304399in}}{\pgfqpoint{10.827454in}{5.304399in}}%
\pgfpathlineto{\pgfqpoint{10.827454in}{5.304399in}}%
\pgfpathclose%
\pgfusepath{stroke}%
\end{pgfscope}%
\begin{pgfscope}%
\pgfpathrectangle{\pgfqpoint{7.512535in}{0.437222in}}{\pgfqpoint{6.275590in}{5.159444in}}%
\pgfusepath{clip}%
\pgfsetbuttcap%
\pgfsetroundjoin%
\pgfsetlinewidth{1.003750pt}%
\definecolor{currentstroke}{rgb}{0.827451,0.827451,0.827451}%
\pgfsetstrokecolor{currentstroke}%
\pgfsetstrokeopacity{0.800000}%
\pgfsetdash{}{0pt}%
\pgfpathmoveto{\pgfqpoint{13.165622in}{5.501584in}}%
\pgfpathcurveto{\pgfqpoint{13.176672in}{5.501584in}}{\pgfqpoint{13.187271in}{5.505974in}}{\pgfqpoint{13.195085in}{5.513787in}}%
\pgfpathcurveto{\pgfqpoint{13.202899in}{5.521601in}}{\pgfqpoint{13.207289in}{5.532200in}}{\pgfqpoint{13.207289in}{5.543250in}}%
\pgfpathcurveto{\pgfqpoint{13.207289in}{5.554300in}}{\pgfqpoint{13.202899in}{5.564899in}}{\pgfqpoint{13.195085in}{5.572713in}}%
\pgfpathcurveto{\pgfqpoint{13.187271in}{5.580527in}}{\pgfqpoint{13.176672in}{5.584917in}}{\pgfqpoint{13.165622in}{5.584917in}}%
\pgfpathcurveto{\pgfqpoint{13.154572in}{5.584917in}}{\pgfqpoint{13.143973in}{5.580527in}}{\pgfqpoint{13.136159in}{5.572713in}}%
\pgfpathcurveto{\pgfqpoint{13.128346in}{5.564899in}}{\pgfqpoint{13.123955in}{5.554300in}}{\pgfqpoint{13.123955in}{5.543250in}}%
\pgfpathcurveto{\pgfqpoint{13.123955in}{5.532200in}}{\pgfqpoint{13.128346in}{5.521601in}}{\pgfqpoint{13.136159in}{5.513787in}}%
\pgfpathcurveto{\pgfqpoint{13.143973in}{5.505974in}}{\pgfqpoint{13.154572in}{5.501584in}}{\pgfqpoint{13.165622in}{5.501584in}}%
\pgfpathlineto{\pgfqpoint{13.165622in}{5.501584in}}%
\pgfpathclose%
\pgfusepath{stroke}%
\end{pgfscope}%
\begin{pgfscope}%
\pgfpathrectangle{\pgfqpoint{7.512535in}{0.437222in}}{\pgfqpoint{6.275590in}{5.159444in}}%
\pgfusepath{clip}%
\pgfsetbuttcap%
\pgfsetroundjoin%
\pgfsetlinewidth{1.003750pt}%
\definecolor{currentstroke}{rgb}{0.827451,0.827451,0.827451}%
\pgfsetstrokecolor{currentstroke}%
\pgfsetstrokeopacity{0.800000}%
\pgfsetdash{}{0pt}%
\pgfpathmoveto{\pgfqpoint{8.290348in}{1.011495in}}%
\pgfpathcurveto{\pgfqpoint{8.301398in}{1.011495in}}{\pgfqpoint{8.311997in}{1.015885in}}{\pgfqpoint{8.319811in}{1.023699in}}%
\pgfpathcurveto{\pgfqpoint{8.327625in}{1.031512in}}{\pgfqpoint{8.332015in}{1.042111in}}{\pgfqpoint{8.332015in}{1.053161in}}%
\pgfpathcurveto{\pgfqpoint{8.332015in}{1.064211in}}{\pgfqpoint{8.327625in}{1.074810in}}{\pgfqpoint{8.319811in}{1.082624in}}%
\pgfpathcurveto{\pgfqpoint{8.311997in}{1.090438in}}{\pgfqpoint{8.301398in}{1.094828in}}{\pgfqpoint{8.290348in}{1.094828in}}%
\pgfpathcurveto{\pgfqpoint{8.279298in}{1.094828in}}{\pgfqpoint{8.268699in}{1.090438in}}{\pgfqpoint{8.260886in}{1.082624in}}%
\pgfpathcurveto{\pgfqpoint{8.253072in}{1.074810in}}{\pgfqpoint{8.248682in}{1.064211in}}{\pgfqpoint{8.248682in}{1.053161in}}%
\pgfpathcurveto{\pgfqpoint{8.248682in}{1.042111in}}{\pgfqpoint{8.253072in}{1.031512in}}{\pgfqpoint{8.260886in}{1.023699in}}%
\pgfpathcurveto{\pgfqpoint{8.268699in}{1.015885in}}{\pgfqpoint{8.279298in}{1.011495in}}{\pgfqpoint{8.290348in}{1.011495in}}%
\pgfpathlineto{\pgfqpoint{8.290348in}{1.011495in}}%
\pgfpathclose%
\pgfusepath{stroke}%
\end{pgfscope}%
\begin{pgfscope}%
\pgfpathrectangle{\pgfqpoint{7.512535in}{0.437222in}}{\pgfqpoint{6.275590in}{5.159444in}}%
\pgfusepath{clip}%
\pgfsetbuttcap%
\pgfsetroundjoin%
\pgfsetlinewidth{1.003750pt}%
\definecolor{currentstroke}{rgb}{0.827451,0.827451,0.827451}%
\pgfsetstrokecolor{currentstroke}%
\pgfsetstrokeopacity{0.800000}%
\pgfsetdash{}{0pt}%
\pgfpathmoveto{\pgfqpoint{8.999230in}{3.282894in}}%
\pgfpathcurveto{\pgfqpoint{9.010281in}{3.282894in}}{\pgfqpoint{9.020880in}{3.287285in}}{\pgfqpoint{9.028693in}{3.295098in}}%
\pgfpathcurveto{\pgfqpoint{9.036507in}{3.302912in}}{\pgfqpoint{9.040897in}{3.313511in}}{\pgfqpoint{9.040897in}{3.324561in}}%
\pgfpathcurveto{\pgfqpoint{9.040897in}{3.335611in}}{\pgfqpoint{9.036507in}{3.346210in}}{\pgfqpoint{9.028693in}{3.354024in}}%
\pgfpathcurveto{\pgfqpoint{9.020880in}{3.361837in}}{\pgfqpoint{9.010281in}{3.366228in}}{\pgfqpoint{8.999230in}{3.366228in}}%
\pgfpathcurveto{\pgfqpoint{8.988180in}{3.366228in}}{\pgfqpoint{8.977581in}{3.361837in}}{\pgfqpoint{8.969768in}{3.354024in}}%
\pgfpathcurveto{\pgfqpoint{8.961954in}{3.346210in}}{\pgfqpoint{8.957564in}{3.335611in}}{\pgfqpoint{8.957564in}{3.324561in}}%
\pgfpathcurveto{\pgfqpoint{8.957564in}{3.313511in}}{\pgfqpoint{8.961954in}{3.302912in}}{\pgfqpoint{8.969768in}{3.295098in}}%
\pgfpathcurveto{\pgfqpoint{8.977581in}{3.287285in}}{\pgfqpoint{8.988180in}{3.282894in}}{\pgfqpoint{8.999230in}{3.282894in}}%
\pgfpathlineto{\pgfqpoint{8.999230in}{3.282894in}}%
\pgfpathclose%
\pgfusepath{stroke}%
\end{pgfscope}%
\begin{pgfscope}%
\pgfpathrectangle{\pgfqpoint{7.512535in}{0.437222in}}{\pgfqpoint{6.275590in}{5.159444in}}%
\pgfusepath{clip}%
\pgfsetbuttcap%
\pgfsetroundjoin%
\pgfsetlinewidth{1.003750pt}%
\definecolor{currentstroke}{rgb}{0.827451,0.827451,0.827451}%
\pgfsetstrokecolor{currentstroke}%
\pgfsetstrokeopacity{0.800000}%
\pgfsetdash{}{0pt}%
\pgfpathmoveto{\pgfqpoint{11.405420in}{5.524166in}}%
\pgfpathcurveto{\pgfqpoint{11.416470in}{5.524166in}}{\pgfqpoint{11.427069in}{5.528556in}}{\pgfqpoint{11.434883in}{5.536370in}}%
\pgfpathcurveto{\pgfqpoint{11.442696in}{5.544183in}}{\pgfqpoint{11.447086in}{5.554782in}}{\pgfqpoint{11.447086in}{5.565833in}}%
\pgfpathcurveto{\pgfqpoint{11.447086in}{5.576883in}}{\pgfqpoint{11.442696in}{5.587482in}}{\pgfqpoint{11.434883in}{5.595295in}}%
\pgfpathcurveto{\pgfqpoint{11.427069in}{5.603109in}}{\pgfqpoint{11.416470in}{5.607499in}}{\pgfqpoint{11.405420in}{5.607499in}}%
\pgfpathcurveto{\pgfqpoint{11.394370in}{5.607499in}}{\pgfqpoint{11.383771in}{5.603109in}}{\pgfqpoint{11.375957in}{5.595295in}}%
\pgfpathcurveto{\pgfqpoint{11.368143in}{5.587482in}}{\pgfqpoint{11.363753in}{5.576883in}}{\pgfqpoint{11.363753in}{5.565833in}}%
\pgfpathcurveto{\pgfqpoint{11.363753in}{5.554782in}}{\pgfqpoint{11.368143in}{5.544183in}}{\pgfqpoint{11.375957in}{5.536370in}}%
\pgfpathcurveto{\pgfqpoint{11.383771in}{5.528556in}}{\pgfqpoint{11.394370in}{5.524166in}}{\pgfqpoint{11.405420in}{5.524166in}}%
\pgfpathlineto{\pgfqpoint{11.405420in}{5.524166in}}%
\pgfpathclose%
\pgfusepath{stroke}%
\end{pgfscope}%
\begin{pgfscope}%
\pgfpathrectangle{\pgfqpoint{7.512535in}{0.437222in}}{\pgfqpoint{6.275590in}{5.159444in}}%
\pgfusepath{clip}%
\pgfsetbuttcap%
\pgfsetroundjoin%
\pgfsetlinewidth{1.003750pt}%
\definecolor{currentstroke}{rgb}{0.827451,0.827451,0.827451}%
\pgfsetstrokecolor{currentstroke}%
\pgfsetstrokeopacity{0.800000}%
\pgfsetdash{}{0pt}%
\pgfpathmoveto{\pgfqpoint{8.598634in}{1.552340in}}%
\pgfpathcurveto{\pgfqpoint{8.609684in}{1.552340in}}{\pgfqpoint{8.620283in}{1.556730in}}{\pgfqpoint{8.628097in}{1.564544in}}%
\pgfpathcurveto{\pgfqpoint{8.635910in}{1.572358in}}{\pgfqpoint{8.640300in}{1.582957in}}{\pgfqpoint{8.640300in}{1.594007in}}%
\pgfpathcurveto{\pgfqpoint{8.640300in}{1.605057in}}{\pgfqpoint{8.635910in}{1.615656in}}{\pgfqpoint{8.628097in}{1.623470in}}%
\pgfpathcurveto{\pgfqpoint{8.620283in}{1.631283in}}{\pgfqpoint{8.609684in}{1.635673in}}{\pgfqpoint{8.598634in}{1.635673in}}%
\pgfpathcurveto{\pgfqpoint{8.587584in}{1.635673in}}{\pgfqpoint{8.576985in}{1.631283in}}{\pgfqpoint{8.569171in}{1.623470in}}%
\pgfpathcurveto{\pgfqpoint{8.561357in}{1.615656in}}{\pgfqpoint{8.556967in}{1.605057in}}{\pgfqpoint{8.556967in}{1.594007in}}%
\pgfpathcurveto{\pgfqpoint{8.556967in}{1.582957in}}{\pgfqpoint{8.561357in}{1.572358in}}{\pgfqpoint{8.569171in}{1.564544in}}%
\pgfpathcurveto{\pgfqpoint{8.576985in}{1.556730in}}{\pgfqpoint{8.587584in}{1.552340in}}{\pgfqpoint{8.598634in}{1.552340in}}%
\pgfpathlineto{\pgfqpoint{8.598634in}{1.552340in}}%
\pgfpathclose%
\pgfusepath{stroke}%
\end{pgfscope}%
\begin{pgfscope}%
\pgfpathrectangle{\pgfqpoint{7.512535in}{0.437222in}}{\pgfqpoint{6.275590in}{5.159444in}}%
\pgfusepath{clip}%
\pgfsetbuttcap%
\pgfsetroundjoin%
\pgfsetlinewidth{1.003750pt}%
\definecolor{currentstroke}{rgb}{0.827451,0.827451,0.827451}%
\pgfsetstrokecolor{currentstroke}%
\pgfsetstrokeopacity{0.800000}%
\pgfsetdash{}{0pt}%
\pgfpathmoveto{\pgfqpoint{10.486066in}{4.589271in}}%
\pgfpathcurveto{\pgfqpoint{10.497116in}{4.589271in}}{\pgfqpoint{10.507715in}{4.593662in}}{\pgfqpoint{10.515529in}{4.601475in}}%
\pgfpathcurveto{\pgfqpoint{10.523342in}{4.609289in}}{\pgfqpoint{10.527733in}{4.619888in}}{\pgfqpoint{10.527733in}{4.630938in}}%
\pgfpathcurveto{\pgfqpoint{10.527733in}{4.641988in}}{\pgfqpoint{10.523342in}{4.652587in}}{\pgfqpoint{10.515529in}{4.660401in}}%
\pgfpathcurveto{\pgfqpoint{10.507715in}{4.668214in}}{\pgfqpoint{10.497116in}{4.672605in}}{\pgfqpoint{10.486066in}{4.672605in}}%
\pgfpathcurveto{\pgfqpoint{10.475016in}{4.672605in}}{\pgfqpoint{10.464417in}{4.668214in}}{\pgfqpoint{10.456603in}{4.660401in}}%
\pgfpathcurveto{\pgfqpoint{10.448790in}{4.652587in}}{\pgfqpoint{10.444399in}{4.641988in}}{\pgfqpoint{10.444399in}{4.630938in}}%
\pgfpathcurveto{\pgfqpoint{10.444399in}{4.619888in}}{\pgfqpoint{10.448790in}{4.609289in}}{\pgfqpoint{10.456603in}{4.601475in}}%
\pgfpathcurveto{\pgfqpoint{10.464417in}{4.593662in}}{\pgfqpoint{10.475016in}{4.589271in}}{\pgfqpoint{10.486066in}{4.589271in}}%
\pgfpathlineto{\pgfqpoint{10.486066in}{4.589271in}}%
\pgfpathclose%
\pgfusepath{stroke}%
\end{pgfscope}%
\begin{pgfscope}%
\pgfpathrectangle{\pgfqpoint{7.512535in}{0.437222in}}{\pgfqpoint{6.275590in}{5.159444in}}%
\pgfusepath{clip}%
\pgfsetbuttcap%
\pgfsetroundjoin%
\pgfsetlinewidth{1.003750pt}%
\definecolor{currentstroke}{rgb}{0.827451,0.827451,0.827451}%
\pgfsetstrokecolor{currentstroke}%
\pgfsetstrokeopacity{0.800000}%
\pgfsetdash{}{0pt}%
\pgfpathmoveto{\pgfqpoint{12.737963in}{5.481847in}}%
\pgfpathcurveto{\pgfqpoint{12.749014in}{5.481847in}}{\pgfqpoint{12.759613in}{5.486238in}}{\pgfqpoint{12.767426in}{5.494051in}}%
\pgfpathcurveto{\pgfqpoint{12.775240in}{5.501865in}}{\pgfqpoint{12.779630in}{5.512464in}}{\pgfqpoint{12.779630in}{5.523514in}}%
\pgfpathcurveto{\pgfqpoint{12.779630in}{5.534564in}}{\pgfqpoint{12.775240in}{5.545163in}}{\pgfqpoint{12.767426in}{5.552977in}}%
\pgfpathcurveto{\pgfqpoint{12.759613in}{5.560791in}}{\pgfqpoint{12.749014in}{5.565181in}}{\pgfqpoint{12.737963in}{5.565181in}}%
\pgfpathcurveto{\pgfqpoint{12.726913in}{5.565181in}}{\pgfqpoint{12.716314in}{5.560791in}}{\pgfqpoint{12.708501in}{5.552977in}}%
\pgfpathcurveto{\pgfqpoint{12.700687in}{5.545163in}}{\pgfqpoint{12.696297in}{5.534564in}}{\pgfqpoint{12.696297in}{5.523514in}}%
\pgfpathcurveto{\pgfqpoint{12.696297in}{5.512464in}}{\pgfqpoint{12.700687in}{5.501865in}}{\pgfqpoint{12.708501in}{5.494051in}}%
\pgfpathcurveto{\pgfqpoint{12.716314in}{5.486238in}}{\pgfqpoint{12.726913in}{5.481847in}}{\pgfqpoint{12.737963in}{5.481847in}}%
\pgfpathlineto{\pgfqpoint{12.737963in}{5.481847in}}%
\pgfpathclose%
\pgfusepath{stroke}%
\end{pgfscope}%
\begin{pgfscope}%
\pgfpathrectangle{\pgfqpoint{7.512535in}{0.437222in}}{\pgfqpoint{6.275590in}{5.159444in}}%
\pgfusepath{clip}%
\pgfsetbuttcap%
\pgfsetroundjoin%
\pgfsetlinewidth{1.003750pt}%
\definecolor{currentstroke}{rgb}{0.827451,0.827451,0.827451}%
\pgfsetstrokecolor{currentstroke}%
\pgfsetstrokeopacity{0.800000}%
\pgfsetdash{}{0pt}%
\pgfpathmoveto{\pgfqpoint{11.673162in}{5.360568in}}%
\pgfpathcurveto{\pgfqpoint{11.684212in}{5.360568in}}{\pgfqpoint{11.694811in}{5.364958in}}{\pgfqpoint{11.702624in}{5.372772in}}%
\pgfpathcurveto{\pgfqpoint{11.710438in}{5.380585in}}{\pgfqpoint{11.714828in}{5.391184in}}{\pgfqpoint{11.714828in}{5.402235in}}%
\pgfpathcurveto{\pgfqpoint{11.714828in}{5.413285in}}{\pgfqpoint{11.710438in}{5.423884in}}{\pgfqpoint{11.702624in}{5.431697in}}%
\pgfpathcurveto{\pgfqpoint{11.694811in}{5.439511in}}{\pgfqpoint{11.684212in}{5.443901in}}{\pgfqpoint{11.673162in}{5.443901in}}%
\pgfpathcurveto{\pgfqpoint{11.662111in}{5.443901in}}{\pgfqpoint{11.651512in}{5.439511in}}{\pgfqpoint{11.643699in}{5.431697in}}%
\pgfpathcurveto{\pgfqpoint{11.635885in}{5.423884in}}{\pgfqpoint{11.631495in}{5.413285in}}{\pgfqpoint{11.631495in}{5.402235in}}%
\pgfpathcurveto{\pgfqpoint{11.631495in}{5.391184in}}{\pgfqpoint{11.635885in}{5.380585in}}{\pgfqpoint{11.643699in}{5.372772in}}%
\pgfpathcurveto{\pgfqpoint{11.651512in}{5.364958in}}{\pgfqpoint{11.662111in}{5.360568in}}{\pgfqpoint{11.673162in}{5.360568in}}%
\pgfpathlineto{\pgfqpoint{11.673162in}{5.360568in}}%
\pgfpathclose%
\pgfusepath{stroke}%
\end{pgfscope}%
\begin{pgfscope}%
\pgfpathrectangle{\pgfqpoint{7.512535in}{0.437222in}}{\pgfqpoint{6.275590in}{5.159444in}}%
\pgfusepath{clip}%
\pgfsetbuttcap%
\pgfsetroundjoin%
\pgfsetlinewidth{1.003750pt}%
\definecolor{currentstroke}{rgb}{0.827451,0.827451,0.827451}%
\pgfsetstrokecolor{currentstroke}%
\pgfsetstrokeopacity{0.800000}%
\pgfsetdash{}{0pt}%
\pgfpathmoveto{\pgfqpoint{10.385573in}{5.144321in}}%
\pgfpathcurveto{\pgfqpoint{10.396623in}{5.144321in}}{\pgfqpoint{10.407222in}{5.148712in}}{\pgfqpoint{10.415036in}{5.156525in}}%
\pgfpathcurveto{\pgfqpoint{10.422849in}{5.164339in}}{\pgfqpoint{10.427240in}{5.174938in}}{\pgfqpoint{10.427240in}{5.185988in}}%
\pgfpathcurveto{\pgfqpoint{10.427240in}{5.197038in}}{\pgfqpoint{10.422849in}{5.207637in}}{\pgfqpoint{10.415036in}{5.215451in}}%
\pgfpathcurveto{\pgfqpoint{10.407222in}{5.223265in}}{\pgfqpoint{10.396623in}{5.227655in}}{\pgfqpoint{10.385573in}{5.227655in}}%
\pgfpathcurveto{\pgfqpoint{10.374523in}{5.227655in}}{\pgfqpoint{10.363924in}{5.223265in}}{\pgfqpoint{10.356110in}{5.215451in}}%
\pgfpathcurveto{\pgfqpoint{10.348297in}{5.207637in}}{\pgfqpoint{10.343906in}{5.197038in}}{\pgfqpoint{10.343906in}{5.185988in}}%
\pgfpathcurveto{\pgfqpoint{10.343906in}{5.174938in}}{\pgfqpoint{10.348297in}{5.164339in}}{\pgfqpoint{10.356110in}{5.156525in}}%
\pgfpathcurveto{\pgfqpoint{10.363924in}{5.148712in}}{\pgfqpoint{10.374523in}{5.144321in}}{\pgfqpoint{10.385573in}{5.144321in}}%
\pgfpathlineto{\pgfqpoint{10.385573in}{5.144321in}}%
\pgfpathclose%
\pgfusepath{stroke}%
\end{pgfscope}%
\begin{pgfscope}%
\pgfpathrectangle{\pgfqpoint{7.512535in}{0.437222in}}{\pgfqpoint{6.275590in}{5.159444in}}%
\pgfusepath{clip}%
\pgfsetbuttcap%
\pgfsetroundjoin%
\pgfsetlinewidth{1.003750pt}%
\definecolor{currentstroke}{rgb}{0.827451,0.827451,0.827451}%
\pgfsetstrokecolor{currentstroke}%
\pgfsetstrokeopacity{0.800000}%
\pgfsetdash{}{0pt}%
\pgfpathmoveto{\pgfqpoint{7.983939in}{1.000624in}}%
\pgfpathcurveto{\pgfqpoint{7.994989in}{1.000624in}}{\pgfqpoint{8.005588in}{1.005015in}}{\pgfqpoint{8.013401in}{1.012828in}}%
\pgfpathcurveto{\pgfqpoint{8.021215in}{1.020642in}}{\pgfqpoint{8.025605in}{1.031241in}}{\pgfqpoint{8.025605in}{1.042291in}}%
\pgfpathcurveto{\pgfqpoint{8.025605in}{1.053341in}}{\pgfqpoint{8.021215in}{1.063940in}}{\pgfqpoint{8.013401in}{1.071754in}}%
\pgfpathcurveto{\pgfqpoint{8.005588in}{1.079568in}}{\pgfqpoint{7.994989in}{1.083958in}}{\pgfqpoint{7.983939in}{1.083958in}}%
\pgfpathcurveto{\pgfqpoint{7.972888in}{1.083958in}}{\pgfqpoint{7.962289in}{1.079568in}}{\pgfqpoint{7.954476in}{1.071754in}}%
\pgfpathcurveto{\pgfqpoint{7.946662in}{1.063940in}}{\pgfqpoint{7.942272in}{1.053341in}}{\pgfqpoint{7.942272in}{1.042291in}}%
\pgfpathcurveto{\pgfqpoint{7.942272in}{1.031241in}}{\pgfqpoint{7.946662in}{1.020642in}}{\pgfqpoint{7.954476in}{1.012828in}}%
\pgfpathcurveto{\pgfqpoint{7.962289in}{1.005015in}}{\pgfqpoint{7.972888in}{1.000624in}}{\pgfqpoint{7.983939in}{1.000624in}}%
\pgfpathlineto{\pgfqpoint{7.983939in}{1.000624in}}%
\pgfpathclose%
\pgfusepath{stroke}%
\end{pgfscope}%
\begin{pgfscope}%
\pgfpathrectangle{\pgfqpoint{7.512535in}{0.437222in}}{\pgfqpoint{6.275590in}{5.159444in}}%
\pgfusepath{clip}%
\pgfsetbuttcap%
\pgfsetroundjoin%
\pgfsetlinewidth{1.003750pt}%
\definecolor{currentstroke}{rgb}{0.827451,0.827451,0.827451}%
\pgfsetstrokecolor{currentstroke}%
\pgfsetstrokeopacity{0.800000}%
\pgfsetdash{}{0pt}%
\pgfpathmoveto{\pgfqpoint{11.026069in}{5.528063in}}%
\pgfpathcurveto{\pgfqpoint{11.037120in}{5.528063in}}{\pgfqpoint{11.047719in}{5.532453in}}{\pgfqpoint{11.055532in}{5.540267in}}%
\pgfpathcurveto{\pgfqpoint{11.063346in}{5.548081in}}{\pgfqpoint{11.067736in}{5.558680in}}{\pgfqpoint{11.067736in}{5.569730in}}%
\pgfpathcurveto{\pgfqpoint{11.067736in}{5.580780in}}{\pgfqpoint{11.063346in}{5.591379in}}{\pgfqpoint{11.055532in}{5.599193in}}%
\pgfpathcurveto{\pgfqpoint{11.047719in}{5.607006in}}{\pgfqpoint{11.037120in}{5.611396in}}{\pgfqpoint{11.026069in}{5.611396in}}%
\pgfpathcurveto{\pgfqpoint{11.015019in}{5.611396in}}{\pgfqpoint{11.004420in}{5.607006in}}{\pgfqpoint{10.996607in}{5.599193in}}%
\pgfpathcurveto{\pgfqpoint{10.988793in}{5.591379in}}{\pgfqpoint{10.984403in}{5.580780in}}{\pgfqpoint{10.984403in}{5.569730in}}%
\pgfpathcurveto{\pgfqpoint{10.984403in}{5.558680in}}{\pgfqpoint{10.988793in}{5.548081in}}{\pgfqpoint{10.996607in}{5.540267in}}%
\pgfpathcurveto{\pgfqpoint{11.004420in}{5.532453in}}{\pgfqpoint{11.015019in}{5.528063in}}{\pgfqpoint{11.026069in}{5.528063in}}%
\pgfpathlineto{\pgfqpoint{11.026069in}{5.528063in}}%
\pgfpathclose%
\pgfusepath{stroke}%
\end{pgfscope}%
\begin{pgfscope}%
\pgfpathrectangle{\pgfqpoint{7.512535in}{0.437222in}}{\pgfqpoint{6.275590in}{5.159444in}}%
\pgfusepath{clip}%
\pgfsetbuttcap%
\pgfsetroundjoin%
\pgfsetlinewidth{1.003750pt}%
\definecolor{currentstroke}{rgb}{0.827451,0.827451,0.827451}%
\pgfsetstrokecolor{currentstroke}%
\pgfsetstrokeopacity{0.800000}%
\pgfsetdash{}{0pt}%
\pgfpathmoveto{\pgfqpoint{11.645781in}{5.002607in}}%
\pgfpathcurveto{\pgfqpoint{11.656832in}{5.002607in}}{\pgfqpoint{11.667431in}{5.006998in}}{\pgfqpoint{11.675244in}{5.014811in}}%
\pgfpathcurveto{\pgfqpoint{11.683058in}{5.022625in}}{\pgfqpoint{11.687448in}{5.033224in}}{\pgfqpoint{11.687448in}{5.044274in}}%
\pgfpathcurveto{\pgfqpoint{11.687448in}{5.055324in}}{\pgfqpoint{11.683058in}{5.065923in}}{\pgfqpoint{11.675244in}{5.073737in}}%
\pgfpathcurveto{\pgfqpoint{11.667431in}{5.081550in}}{\pgfqpoint{11.656832in}{5.085941in}}{\pgfqpoint{11.645781in}{5.085941in}}%
\pgfpathcurveto{\pgfqpoint{11.634731in}{5.085941in}}{\pgfqpoint{11.624132in}{5.081550in}}{\pgfqpoint{11.616319in}{5.073737in}}%
\pgfpathcurveto{\pgfqpoint{11.608505in}{5.065923in}}{\pgfqpoint{11.604115in}{5.055324in}}{\pgfqpoint{11.604115in}{5.044274in}}%
\pgfpathcurveto{\pgfqpoint{11.604115in}{5.033224in}}{\pgfqpoint{11.608505in}{5.022625in}}{\pgfqpoint{11.616319in}{5.014811in}}%
\pgfpathcurveto{\pgfqpoint{11.624132in}{5.006998in}}{\pgfqpoint{11.634731in}{5.002607in}}{\pgfqpoint{11.645781in}{5.002607in}}%
\pgfpathlineto{\pgfqpoint{11.645781in}{5.002607in}}%
\pgfpathclose%
\pgfusepath{stroke}%
\end{pgfscope}%
\begin{pgfscope}%
\pgfpathrectangle{\pgfqpoint{7.512535in}{0.437222in}}{\pgfqpoint{6.275590in}{5.159444in}}%
\pgfusepath{clip}%
\pgfsetbuttcap%
\pgfsetroundjoin%
\pgfsetlinewidth{1.003750pt}%
\definecolor{currentstroke}{rgb}{0.827451,0.827451,0.827451}%
\pgfsetstrokecolor{currentstroke}%
\pgfsetstrokeopacity{0.800000}%
\pgfsetdash{}{0pt}%
\pgfpathmoveto{\pgfqpoint{12.453916in}{5.550986in}}%
\pgfpathcurveto{\pgfqpoint{12.464967in}{5.550986in}}{\pgfqpoint{12.475566in}{5.555376in}}{\pgfqpoint{12.483379in}{5.563190in}}%
\pgfpathcurveto{\pgfqpoint{12.491193in}{5.571003in}}{\pgfqpoint{12.495583in}{5.581602in}}{\pgfqpoint{12.495583in}{5.592653in}}%
\pgfpathcurveto{\pgfqpoint{12.495583in}{5.603703in}}{\pgfqpoint{12.491193in}{5.614302in}}{\pgfqpoint{12.483379in}{5.622115in}}%
\pgfpathcurveto{\pgfqpoint{12.475566in}{5.629929in}}{\pgfqpoint{12.464967in}{5.634319in}}{\pgfqpoint{12.453916in}{5.634319in}}%
\pgfpathcurveto{\pgfqpoint{12.442866in}{5.634319in}}{\pgfqpoint{12.432267in}{5.629929in}}{\pgfqpoint{12.424454in}{5.622115in}}%
\pgfpathcurveto{\pgfqpoint{12.416640in}{5.614302in}}{\pgfqpoint{12.412250in}{5.603703in}}{\pgfqpoint{12.412250in}{5.592653in}}%
\pgfpathcurveto{\pgfqpoint{12.412250in}{5.581602in}}{\pgfqpoint{12.416640in}{5.571003in}}{\pgfqpoint{12.424454in}{5.563190in}}%
\pgfpathcurveto{\pgfqpoint{12.432267in}{5.555376in}}{\pgfqpoint{12.442866in}{5.550986in}}{\pgfqpoint{12.453916in}{5.550986in}}%
\pgfpathlineto{\pgfqpoint{12.453916in}{5.550986in}}%
\pgfpathclose%
\pgfusepath{stroke}%
\end{pgfscope}%
\begin{pgfscope}%
\pgfpathrectangle{\pgfqpoint{7.512535in}{0.437222in}}{\pgfqpoint{6.275590in}{5.159444in}}%
\pgfusepath{clip}%
\pgfsetbuttcap%
\pgfsetroundjoin%
\pgfsetlinewidth{1.003750pt}%
\definecolor{currentstroke}{rgb}{0.827451,0.827451,0.827451}%
\pgfsetstrokecolor{currentstroke}%
\pgfsetstrokeopacity{0.800000}%
\pgfsetdash{}{0pt}%
\pgfpathmoveto{\pgfqpoint{8.680117in}{3.344222in}}%
\pgfpathcurveto{\pgfqpoint{8.691167in}{3.344222in}}{\pgfqpoint{8.701766in}{3.348612in}}{\pgfqpoint{8.709580in}{3.356426in}}%
\pgfpathcurveto{\pgfqpoint{8.717393in}{3.364239in}}{\pgfqpoint{8.721784in}{3.374838in}}{\pgfqpoint{8.721784in}{3.385889in}}%
\pgfpathcurveto{\pgfqpoint{8.721784in}{3.396939in}}{\pgfqpoint{8.717393in}{3.407538in}}{\pgfqpoint{8.709580in}{3.415351in}}%
\pgfpathcurveto{\pgfqpoint{8.701766in}{3.423165in}}{\pgfqpoint{8.691167in}{3.427555in}}{\pgfqpoint{8.680117in}{3.427555in}}%
\pgfpathcurveto{\pgfqpoint{8.669067in}{3.427555in}}{\pgfqpoint{8.658468in}{3.423165in}}{\pgfqpoint{8.650654in}{3.415351in}}%
\pgfpathcurveto{\pgfqpoint{8.642841in}{3.407538in}}{\pgfqpoint{8.638450in}{3.396939in}}{\pgfqpoint{8.638450in}{3.385889in}}%
\pgfpathcurveto{\pgfqpoint{8.638450in}{3.374838in}}{\pgfqpoint{8.642841in}{3.364239in}}{\pgfqpoint{8.650654in}{3.356426in}}%
\pgfpathcurveto{\pgfqpoint{8.658468in}{3.348612in}}{\pgfqpoint{8.669067in}{3.344222in}}{\pgfqpoint{8.680117in}{3.344222in}}%
\pgfpathlineto{\pgfqpoint{8.680117in}{3.344222in}}%
\pgfpathclose%
\pgfusepath{stroke}%
\end{pgfscope}%
\begin{pgfscope}%
\pgfpathrectangle{\pgfqpoint{7.512535in}{0.437222in}}{\pgfqpoint{6.275590in}{5.159444in}}%
\pgfusepath{clip}%
\pgfsetbuttcap%
\pgfsetroundjoin%
\pgfsetlinewidth{1.003750pt}%
\definecolor{currentstroke}{rgb}{0.827451,0.827451,0.827451}%
\pgfsetstrokecolor{currentstroke}%
\pgfsetstrokeopacity{0.800000}%
\pgfsetdash{}{0pt}%
\pgfpathmoveto{\pgfqpoint{9.444129in}{2.962502in}}%
\pgfpathcurveto{\pgfqpoint{9.455180in}{2.962502in}}{\pgfqpoint{9.465779in}{2.966892in}}{\pgfqpoint{9.473592in}{2.974705in}}%
\pgfpathcurveto{\pgfqpoint{9.481406in}{2.982519in}}{\pgfqpoint{9.485796in}{2.993118in}}{\pgfqpoint{9.485796in}{3.004168in}}%
\pgfpathcurveto{\pgfqpoint{9.485796in}{3.015218in}}{\pgfqpoint{9.481406in}{3.025817in}}{\pgfqpoint{9.473592in}{3.033631in}}%
\pgfpathcurveto{\pgfqpoint{9.465779in}{3.041445in}}{\pgfqpoint{9.455180in}{3.045835in}}{\pgfqpoint{9.444129in}{3.045835in}}%
\pgfpathcurveto{\pgfqpoint{9.433079in}{3.045835in}}{\pgfqpoint{9.422480in}{3.041445in}}{\pgfqpoint{9.414667in}{3.033631in}}%
\pgfpathcurveto{\pgfqpoint{9.406853in}{3.025817in}}{\pgfqpoint{9.402463in}{3.015218in}}{\pgfqpoint{9.402463in}{3.004168in}}%
\pgfpathcurveto{\pgfqpoint{9.402463in}{2.993118in}}{\pgfqpoint{9.406853in}{2.982519in}}{\pgfqpoint{9.414667in}{2.974705in}}%
\pgfpathcurveto{\pgfqpoint{9.422480in}{2.966892in}}{\pgfqpoint{9.433079in}{2.962502in}}{\pgfqpoint{9.444129in}{2.962502in}}%
\pgfpathlineto{\pgfqpoint{9.444129in}{2.962502in}}%
\pgfpathclose%
\pgfusepath{stroke}%
\end{pgfscope}%
\begin{pgfscope}%
\pgfpathrectangle{\pgfqpoint{7.512535in}{0.437222in}}{\pgfqpoint{6.275590in}{5.159444in}}%
\pgfusepath{clip}%
\pgfsetbuttcap%
\pgfsetroundjoin%
\pgfsetlinewidth{1.003750pt}%
\definecolor{currentstroke}{rgb}{0.827451,0.827451,0.827451}%
\pgfsetstrokecolor{currentstroke}%
\pgfsetstrokeopacity{0.800000}%
\pgfsetdash{}{0pt}%
\pgfpathmoveto{\pgfqpoint{10.369675in}{4.552452in}}%
\pgfpathcurveto{\pgfqpoint{10.380725in}{4.552452in}}{\pgfqpoint{10.391324in}{4.556842in}}{\pgfqpoint{10.399138in}{4.564656in}}%
\pgfpathcurveto{\pgfqpoint{10.406951in}{4.572470in}}{\pgfqpoint{10.411341in}{4.583069in}}{\pgfqpoint{10.411341in}{4.594119in}}%
\pgfpathcurveto{\pgfqpoint{10.411341in}{4.605169in}}{\pgfqpoint{10.406951in}{4.615768in}}{\pgfqpoint{10.399138in}{4.623582in}}%
\pgfpathcurveto{\pgfqpoint{10.391324in}{4.631395in}}{\pgfqpoint{10.380725in}{4.635785in}}{\pgfqpoint{10.369675in}{4.635785in}}%
\pgfpathcurveto{\pgfqpoint{10.358625in}{4.635785in}}{\pgfqpoint{10.348026in}{4.631395in}}{\pgfqpoint{10.340212in}{4.623582in}}%
\pgfpathcurveto{\pgfqpoint{10.332398in}{4.615768in}}{\pgfqpoint{10.328008in}{4.605169in}}{\pgfqpoint{10.328008in}{4.594119in}}%
\pgfpathcurveto{\pgfqpoint{10.328008in}{4.583069in}}{\pgfqpoint{10.332398in}{4.572470in}}{\pgfqpoint{10.340212in}{4.564656in}}%
\pgfpathcurveto{\pgfqpoint{10.348026in}{4.556842in}}{\pgfqpoint{10.358625in}{4.552452in}}{\pgfqpoint{10.369675in}{4.552452in}}%
\pgfpathlineto{\pgfqpoint{10.369675in}{4.552452in}}%
\pgfpathclose%
\pgfusepath{stroke}%
\end{pgfscope}%
\begin{pgfscope}%
\pgfpathrectangle{\pgfqpoint{7.512535in}{0.437222in}}{\pgfqpoint{6.275590in}{5.159444in}}%
\pgfusepath{clip}%
\pgfsetbuttcap%
\pgfsetroundjoin%
\pgfsetlinewidth{1.003750pt}%
\definecolor{currentstroke}{rgb}{0.827451,0.827451,0.827451}%
\pgfsetstrokecolor{currentstroke}%
\pgfsetstrokeopacity{0.800000}%
\pgfsetdash{}{0pt}%
\pgfpathmoveto{\pgfqpoint{9.001197in}{3.326629in}}%
\pgfpathcurveto{\pgfqpoint{9.012248in}{3.326629in}}{\pgfqpoint{9.022847in}{3.331019in}}{\pgfqpoint{9.030660in}{3.338833in}}%
\pgfpathcurveto{\pgfqpoint{9.038474in}{3.346646in}}{\pgfqpoint{9.042864in}{3.357246in}}{\pgfqpoint{9.042864in}{3.368296in}}%
\pgfpathcurveto{\pgfqpoint{9.042864in}{3.379346in}}{\pgfqpoint{9.038474in}{3.389945in}}{\pgfqpoint{9.030660in}{3.397758in}}%
\pgfpathcurveto{\pgfqpoint{9.022847in}{3.405572in}}{\pgfqpoint{9.012248in}{3.409962in}}{\pgfqpoint{9.001197in}{3.409962in}}%
\pgfpathcurveto{\pgfqpoint{8.990147in}{3.409962in}}{\pgfqpoint{8.979548in}{3.405572in}}{\pgfqpoint{8.971735in}{3.397758in}}%
\pgfpathcurveto{\pgfqpoint{8.963921in}{3.389945in}}{\pgfqpoint{8.959531in}{3.379346in}}{\pgfqpoint{8.959531in}{3.368296in}}%
\pgfpathcurveto{\pgfqpoint{8.959531in}{3.357246in}}{\pgfqpoint{8.963921in}{3.346646in}}{\pgfqpoint{8.971735in}{3.338833in}}%
\pgfpathcurveto{\pgfqpoint{8.979548in}{3.331019in}}{\pgfqpoint{8.990147in}{3.326629in}}{\pgfqpoint{9.001197in}{3.326629in}}%
\pgfpathlineto{\pgfqpoint{9.001197in}{3.326629in}}%
\pgfpathclose%
\pgfusepath{stroke}%
\end{pgfscope}%
\begin{pgfscope}%
\pgfpathrectangle{\pgfqpoint{7.512535in}{0.437222in}}{\pgfqpoint{6.275590in}{5.159444in}}%
\pgfusepath{clip}%
\pgfsetbuttcap%
\pgfsetroundjoin%
\pgfsetlinewidth{1.003750pt}%
\definecolor{currentstroke}{rgb}{0.827451,0.827451,0.827451}%
\pgfsetstrokecolor{currentstroke}%
\pgfsetstrokeopacity{0.800000}%
\pgfsetdash{}{0pt}%
\pgfpathmoveto{\pgfqpoint{8.775643in}{1.745625in}}%
\pgfpathcurveto{\pgfqpoint{8.786693in}{1.745625in}}{\pgfqpoint{8.797292in}{1.750015in}}{\pgfqpoint{8.805106in}{1.757829in}}%
\pgfpathcurveto{\pgfqpoint{8.812919in}{1.765643in}}{\pgfqpoint{8.817310in}{1.776242in}}{\pgfqpoint{8.817310in}{1.787292in}}%
\pgfpathcurveto{\pgfqpoint{8.817310in}{1.798342in}}{\pgfqpoint{8.812919in}{1.808941in}}{\pgfqpoint{8.805106in}{1.816755in}}%
\pgfpathcurveto{\pgfqpoint{8.797292in}{1.824568in}}{\pgfqpoint{8.786693in}{1.828959in}}{\pgfqpoint{8.775643in}{1.828959in}}%
\pgfpathcurveto{\pgfqpoint{8.764593in}{1.828959in}}{\pgfqpoint{8.753994in}{1.824568in}}{\pgfqpoint{8.746180in}{1.816755in}}%
\pgfpathcurveto{\pgfqpoint{8.738367in}{1.808941in}}{\pgfqpoint{8.733976in}{1.798342in}}{\pgfqpoint{8.733976in}{1.787292in}}%
\pgfpathcurveto{\pgfqpoint{8.733976in}{1.776242in}}{\pgfqpoint{8.738367in}{1.765643in}}{\pgfqpoint{8.746180in}{1.757829in}}%
\pgfpathcurveto{\pgfqpoint{8.753994in}{1.750015in}}{\pgfqpoint{8.764593in}{1.745625in}}{\pgfqpoint{8.775643in}{1.745625in}}%
\pgfpathlineto{\pgfqpoint{8.775643in}{1.745625in}}%
\pgfpathclose%
\pgfusepath{stroke}%
\end{pgfscope}%
\begin{pgfscope}%
\pgfpathrectangle{\pgfqpoint{7.512535in}{0.437222in}}{\pgfqpoint{6.275590in}{5.159444in}}%
\pgfusepath{clip}%
\pgfsetbuttcap%
\pgfsetroundjoin%
\pgfsetlinewidth{1.003750pt}%
\definecolor{currentstroke}{rgb}{0.827451,0.827451,0.827451}%
\pgfsetstrokecolor{currentstroke}%
\pgfsetstrokeopacity{0.800000}%
\pgfsetdash{}{0pt}%
\pgfpathmoveto{\pgfqpoint{9.792079in}{4.165674in}}%
\pgfpathcurveto{\pgfqpoint{9.803129in}{4.165674in}}{\pgfqpoint{9.813728in}{4.170064in}}{\pgfqpoint{9.821541in}{4.177878in}}%
\pgfpathcurveto{\pgfqpoint{9.829355in}{4.185692in}}{\pgfqpoint{9.833745in}{4.196291in}}{\pgfqpoint{9.833745in}{4.207341in}}%
\pgfpathcurveto{\pgfqpoint{9.833745in}{4.218391in}}{\pgfqpoint{9.829355in}{4.228990in}}{\pgfqpoint{9.821541in}{4.236803in}}%
\pgfpathcurveto{\pgfqpoint{9.813728in}{4.244617in}}{\pgfqpoint{9.803129in}{4.249007in}}{\pgfqpoint{9.792079in}{4.249007in}}%
\pgfpathcurveto{\pgfqpoint{9.781029in}{4.249007in}}{\pgfqpoint{9.770429in}{4.244617in}}{\pgfqpoint{9.762616in}{4.236803in}}%
\pgfpathcurveto{\pgfqpoint{9.754802in}{4.228990in}}{\pgfqpoint{9.750412in}{4.218391in}}{\pgfqpoint{9.750412in}{4.207341in}}%
\pgfpathcurveto{\pgfqpoint{9.750412in}{4.196291in}}{\pgfqpoint{9.754802in}{4.185692in}}{\pgfqpoint{9.762616in}{4.177878in}}%
\pgfpathcurveto{\pgfqpoint{9.770429in}{4.170064in}}{\pgfqpoint{9.781029in}{4.165674in}}{\pgfqpoint{9.792079in}{4.165674in}}%
\pgfpathlineto{\pgfqpoint{9.792079in}{4.165674in}}%
\pgfpathclose%
\pgfusepath{stroke}%
\end{pgfscope}%
\begin{pgfscope}%
\pgfpathrectangle{\pgfqpoint{7.512535in}{0.437222in}}{\pgfqpoint{6.275590in}{5.159444in}}%
\pgfusepath{clip}%
\pgfsetbuttcap%
\pgfsetroundjoin%
\pgfsetlinewidth{1.003750pt}%
\definecolor{currentstroke}{rgb}{0.827451,0.827451,0.827451}%
\pgfsetstrokecolor{currentstroke}%
\pgfsetstrokeopacity{0.800000}%
\pgfsetdash{}{0pt}%
\pgfpathmoveto{\pgfqpoint{9.807010in}{4.531380in}}%
\pgfpathcurveto{\pgfqpoint{9.818060in}{4.531380in}}{\pgfqpoint{9.828659in}{4.535770in}}{\pgfqpoint{9.836473in}{4.543584in}}%
\pgfpathcurveto{\pgfqpoint{9.844287in}{4.551397in}}{\pgfqpoint{9.848677in}{4.561996in}}{\pgfqpoint{9.848677in}{4.573046in}}%
\pgfpathcurveto{\pgfqpoint{9.848677in}{4.584096in}}{\pgfqpoint{9.844287in}{4.594695in}}{\pgfqpoint{9.836473in}{4.602509in}}%
\pgfpathcurveto{\pgfqpoint{9.828659in}{4.610323in}}{\pgfqpoint{9.818060in}{4.614713in}}{\pgfqpoint{9.807010in}{4.614713in}}%
\pgfpathcurveto{\pgfqpoint{9.795960in}{4.614713in}}{\pgfqpoint{9.785361in}{4.610323in}}{\pgfqpoint{9.777548in}{4.602509in}}%
\pgfpathcurveto{\pgfqpoint{9.769734in}{4.594695in}}{\pgfqpoint{9.765344in}{4.584096in}}{\pgfqpoint{9.765344in}{4.573046in}}%
\pgfpathcurveto{\pgfqpoint{9.765344in}{4.561996in}}{\pgfqpoint{9.769734in}{4.551397in}}{\pgfqpoint{9.777548in}{4.543584in}}%
\pgfpathcurveto{\pgfqpoint{9.785361in}{4.535770in}}{\pgfqpoint{9.795960in}{4.531380in}}{\pgfqpoint{9.807010in}{4.531380in}}%
\pgfpathlineto{\pgfqpoint{9.807010in}{4.531380in}}%
\pgfpathclose%
\pgfusepath{stroke}%
\end{pgfscope}%
\begin{pgfscope}%
\pgfpathrectangle{\pgfqpoint{7.512535in}{0.437222in}}{\pgfqpoint{6.275590in}{5.159444in}}%
\pgfusepath{clip}%
\pgfsetbuttcap%
\pgfsetroundjoin%
\pgfsetlinewidth{1.003750pt}%
\definecolor{currentstroke}{rgb}{0.827451,0.827451,0.827451}%
\pgfsetstrokecolor{currentstroke}%
\pgfsetstrokeopacity{0.800000}%
\pgfsetdash{}{0pt}%
\pgfpathmoveto{\pgfqpoint{10.529065in}{5.355498in}}%
\pgfpathcurveto{\pgfqpoint{10.540115in}{5.355498in}}{\pgfqpoint{10.550714in}{5.359888in}}{\pgfqpoint{10.558528in}{5.367701in}}%
\pgfpathcurveto{\pgfqpoint{10.566341in}{5.375515in}}{\pgfqpoint{10.570731in}{5.386114in}}{\pgfqpoint{10.570731in}{5.397164in}}%
\pgfpathcurveto{\pgfqpoint{10.570731in}{5.408214in}}{\pgfqpoint{10.566341in}{5.418813in}}{\pgfqpoint{10.558528in}{5.426627in}}%
\pgfpathcurveto{\pgfqpoint{10.550714in}{5.434441in}}{\pgfqpoint{10.540115in}{5.438831in}}{\pgfqpoint{10.529065in}{5.438831in}}%
\pgfpathcurveto{\pgfqpoint{10.518015in}{5.438831in}}{\pgfqpoint{10.507416in}{5.434441in}}{\pgfqpoint{10.499602in}{5.426627in}}%
\pgfpathcurveto{\pgfqpoint{10.491788in}{5.418813in}}{\pgfqpoint{10.487398in}{5.408214in}}{\pgfqpoint{10.487398in}{5.397164in}}%
\pgfpathcurveto{\pgfqpoint{10.487398in}{5.386114in}}{\pgfqpoint{10.491788in}{5.375515in}}{\pgfqpoint{10.499602in}{5.367701in}}%
\pgfpathcurveto{\pgfqpoint{10.507416in}{5.359888in}}{\pgfqpoint{10.518015in}{5.355498in}}{\pgfqpoint{10.529065in}{5.355498in}}%
\pgfpathlineto{\pgfqpoint{10.529065in}{5.355498in}}%
\pgfpathclose%
\pgfusepath{stroke}%
\end{pgfscope}%
\begin{pgfscope}%
\pgfpathrectangle{\pgfqpoint{7.512535in}{0.437222in}}{\pgfqpoint{6.275590in}{5.159444in}}%
\pgfusepath{clip}%
\pgfsetbuttcap%
\pgfsetroundjoin%
\pgfsetlinewidth{1.003750pt}%
\definecolor{currentstroke}{rgb}{0.827451,0.827451,0.827451}%
\pgfsetstrokecolor{currentstroke}%
\pgfsetstrokeopacity{0.800000}%
\pgfsetdash{}{0pt}%
\pgfpathmoveto{\pgfqpoint{7.983939in}{0.909544in}}%
\pgfpathcurveto{\pgfqpoint{7.994989in}{0.909544in}}{\pgfqpoint{8.005588in}{0.913934in}}{\pgfqpoint{8.013401in}{0.921748in}}%
\pgfpathcurveto{\pgfqpoint{8.021215in}{0.929562in}}{\pgfqpoint{8.025605in}{0.940161in}}{\pgfqpoint{8.025605in}{0.951211in}}%
\pgfpathcurveto{\pgfqpoint{8.025605in}{0.962261in}}{\pgfqpoint{8.021215in}{0.972860in}}{\pgfqpoint{8.013401in}{0.980674in}}%
\pgfpathcurveto{\pgfqpoint{8.005588in}{0.988487in}}{\pgfqpoint{7.994989in}{0.992878in}}{\pgfqpoint{7.983939in}{0.992878in}}%
\pgfpathcurveto{\pgfqpoint{7.972888in}{0.992878in}}{\pgfqpoint{7.962289in}{0.988487in}}{\pgfqpoint{7.954476in}{0.980674in}}%
\pgfpathcurveto{\pgfqpoint{7.946662in}{0.972860in}}{\pgfqpoint{7.942272in}{0.962261in}}{\pgfqpoint{7.942272in}{0.951211in}}%
\pgfpathcurveto{\pgfqpoint{7.942272in}{0.940161in}}{\pgfqpoint{7.946662in}{0.929562in}}{\pgfqpoint{7.954476in}{0.921748in}}%
\pgfpathcurveto{\pgfqpoint{7.962289in}{0.913934in}}{\pgfqpoint{7.972888in}{0.909544in}}{\pgfqpoint{7.983939in}{0.909544in}}%
\pgfpathlineto{\pgfqpoint{7.983939in}{0.909544in}}%
\pgfpathclose%
\pgfusepath{stroke}%
\end{pgfscope}%
\begin{pgfscope}%
\pgfpathrectangle{\pgfqpoint{7.512535in}{0.437222in}}{\pgfqpoint{6.275590in}{5.159444in}}%
\pgfusepath{clip}%
\pgfsetbuttcap%
\pgfsetroundjoin%
\pgfsetlinewidth{1.003750pt}%
\definecolor{currentstroke}{rgb}{0.827451,0.827451,0.827451}%
\pgfsetstrokecolor{currentstroke}%
\pgfsetstrokeopacity{0.800000}%
\pgfsetdash{}{0pt}%
\pgfpathmoveto{\pgfqpoint{11.431549in}{4.628942in}}%
\pgfpathcurveto{\pgfqpoint{11.442599in}{4.628942in}}{\pgfqpoint{11.453198in}{4.633333in}}{\pgfqpoint{11.461012in}{4.641146in}}%
\pgfpathcurveto{\pgfqpoint{11.468825in}{4.648960in}}{\pgfqpoint{11.473215in}{4.659559in}}{\pgfqpoint{11.473215in}{4.670609in}}%
\pgfpathcurveto{\pgfqpoint{11.473215in}{4.681659in}}{\pgfqpoint{11.468825in}{4.692258in}}{\pgfqpoint{11.461012in}{4.700072in}}%
\pgfpathcurveto{\pgfqpoint{11.453198in}{4.707885in}}{\pgfqpoint{11.442599in}{4.712276in}}{\pgfqpoint{11.431549in}{4.712276in}}%
\pgfpathcurveto{\pgfqpoint{11.420499in}{4.712276in}}{\pgfqpoint{11.409900in}{4.707885in}}{\pgfqpoint{11.402086in}{4.700072in}}%
\pgfpathcurveto{\pgfqpoint{11.394272in}{4.692258in}}{\pgfqpoint{11.389882in}{4.681659in}}{\pgfqpoint{11.389882in}{4.670609in}}%
\pgfpathcurveto{\pgfqpoint{11.389882in}{4.659559in}}{\pgfqpoint{11.394272in}{4.648960in}}{\pgfqpoint{11.402086in}{4.641146in}}%
\pgfpathcurveto{\pgfqpoint{11.409900in}{4.633333in}}{\pgfqpoint{11.420499in}{4.628942in}}{\pgfqpoint{11.431549in}{4.628942in}}%
\pgfpathlineto{\pgfqpoint{11.431549in}{4.628942in}}%
\pgfpathclose%
\pgfusepath{stroke}%
\end{pgfscope}%
\begin{pgfscope}%
\pgfpathrectangle{\pgfqpoint{7.512535in}{0.437222in}}{\pgfqpoint{6.275590in}{5.159444in}}%
\pgfusepath{clip}%
\pgfsetbuttcap%
\pgfsetroundjoin%
\pgfsetlinewidth{1.003750pt}%
\definecolor{currentstroke}{rgb}{0.827451,0.827451,0.827451}%
\pgfsetstrokecolor{currentstroke}%
\pgfsetstrokeopacity{0.800000}%
\pgfsetdash{}{0pt}%
\pgfpathmoveto{\pgfqpoint{8.519449in}{1.061557in}}%
\pgfpathcurveto{\pgfqpoint{8.530499in}{1.061557in}}{\pgfqpoint{8.541099in}{1.065948in}}{\pgfqpoint{8.548912in}{1.073761in}}%
\pgfpathcurveto{\pgfqpoint{8.556726in}{1.081575in}}{\pgfqpoint{8.561116in}{1.092174in}}{\pgfqpoint{8.561116in}{1.103224in}}%
\pgfpathcurveto{\pgfqpoint{8.561116in}{1.114274in}}{\pgfqpoint{8.556726in}{1.124873in}}{\pgfqpoint{8.548912in}{1.132687in}}%
\pgfpathcurveto{\pgfqpoint{8.541099in}{1.140501in}}{\pgfqpoint{8.530499in}{1.144891in}}{\pgfqpoint{8.519449in}{1.144891in}}%
\pgfpathcurveto{\pgfqpoint{8.508399in}{1.144891in}}{\pgfqpoint{8.497800in}{1.140501in}}{\pgfqpoint{8.489987in}{1.132687in}}%
\pgfpathcurveto{\pgfqpoint{8.482173in}{1.124873in}}{\pgfqpoint{8.477783in}{1.114274in}}{\pgfqpoint{8.477783in}{1.103224in}}%
\pgfpathcurveto{\pgfqpoint{8.477783in}{1.092174in}}{\pgfqpoint{8.482173in}{1.081575in}}{\pgfqpoint{8.489987in}{1.073761in}}%
\pgfpathcurveto{\pgfqpoint{8.497800in}{1.065948in}}{\pgfqpoint{8.508399in}{1.061557in}}{\pgfqpoint{8.519449in}{1.061557in}}%
\pgfpathlineto{\pgfqpoint{8.519449in}{1.061557in}}%
\pgfpathclose%
\pgfusepath{stroke}%
\end{pgfscope}%
\begin{pgfscope}%
\pgfpathrectangle{\pgfqpoint{7.512535in}{0.437222in}}{\pgfqpoint{6.275590in}{5.159444in}}%
\pgfusepath{clip}%
\pgfsetbuttcap%
\pgfsetroundjoin%
\pgfsetlinewidth{1.003750pt}%
\definecolor{currentstroke}{rgb}{0.827451,0.827451,0.827451}%
\pgfsetstrokecolor{currentstroke}%
\pgfsetstrokeopacity{0.800000}%
\pgfsetdash{}{0pt}%
\pgfpathmoveto{\pgfqpoint{9.581399in}{2.990585in}}%
\pgfpathcurveto{\pgfqpoint{9.592449in}{2.990585in}}{\pgfqpoint{9.603048in}{2.994975in}}{\pgfqpoint{9.610862in}{3.002789in}}%
\pgfpathcurveto{\pgfqpoint{9.618675in}{3.010602in}}{\pgfqpoint{9.623065in}{3.021202in}}{\pgfqpoint{9.623065in}{3.032252in}}%
\pgfpathcurveto{\pgfqpoint{9.623065in}{3.043302in}}{\pgfqpoint{9.618675in}{3.053901in}}{\pgfqpoint{9.610862in}{3.061714in}}%
\pgfpathcurveto{\pgfqpoint{9.603048in}{3.069528in}}{\pgfqpoint{9.592449in}{3.073918in}}{\pgfqpoint{9.581399in}{3.073918in}}%
\pgfpathcurveto{\pgfqpoint{9.570349in}{3.073918in}}{\pgfqpoint{9.559750in}{3.069528in}}{\pgfqpoint{9.551936in}{3.061714in}}%
\pgfpathcurveto{\pgfqpoint{9.544122in}{3.053901in}}{\pgfqpoint{9.539732in}{3.043302in}}{\pgfqpoint{9.539732in}{3.032252in}}%
\pgfpathcurveto{\pgfqpoint{9.539732in}{3.021202in}}{\pgfqpoint{9.544122in}{3.010602in}}{\pgfqpoint{9.551936in}{3.002789in}}%
\pgfpathcurveto{\pgfqpoint{9.559750in}{2.994975in}}{\pgfqpoint{9.570349in}{2.990585in}}{\pgfqpoint{9.581399in}{2.990585in}}%
\pgfpathlineto{\pgfqpoint{9.581399in}{2.990585in}}%
\pgfpathclose%
\pgfusepath{stroke}%
\end{pgfscope}%
\begin{pgfscope}%
\pgfpathrectangle{\pgfqpoint{7.512535in}{0.437222in}}{\pgfqpoint{6.275590in}{5.159444in}}%
\pgfusepath{clip}%
\pgfsetbuttcap%
\pgfsetroundjoin%
\pgfsetlinewidth{1.003750pt}%
\definecolor{currentstroke}{rgb}{0.827451,0.827451,0.827451}%
\pgfsetstrokecolor{currentstroke}%
\pgfsetstrokeopacity{0.800000}%
\pgfsetdash{}{0pt}%
\pgfpathmoveto{\pgfqpoint{9.726466in}{3.345111in}}%
\pgfpathcurveto{\pgfqpoint{9.737516in}{3.345111in}}{\pgfqpoint{9.748115in}{3.349501in}}{\pgfqpoint{9.755929in}{3.357315in}}%
\pgfpathcurveto{\pgfqpoint{9.763743in}{3.365128in}}{\pgfqpoint{9.768133in}{3.375727in}}{\pgfqpoint{9.768133in}{3.386778in}}%
\pgfpathcurveto{\pgfqpoint{9.768133in}{3.397828in}}{\pgfqpoint{9.763743in}{3.408427in}}{\pgfqpoint{9.755929in}{3.416240in}}%
\pgfpathcurveto{\pgfqpoint{9.748115in}{3.424054in}}{\pgfqpoint{9.737516in}{3.428444in}}{\pgfqpoint{9.726466in}{3.428444in}}%
\pgfpathcurveto{\pgfqpoint{9.715416in}{3.428444in}}{\pgfqpoint{9.704817in}{3.424054in}}{\pgfqpoint{9.697004in}{3.416240in}}%
\pgfpathcurveto{\pgfqpoint{9.689190in}{3.408427in}}{\pgfqpoint{9.684800in}{3.397828in}}{\pgfqpoint{9.684800in}{3.386778in}}%
\pgfpathcurveto{\pgfqpoint{9.684800in}{3.375727in}}{\pgfqpoint{9.689190in}{3.365128in}}{\pgfqpoint{9.697004in}{3.357315in}}%
\pgfpathcurveto{\pgfqpoint{9.704817in}{3.349501in}}{\pgfqpoint{9.715416in}{3.345111in}}{\pgfqpoint{9.726466in}{3.345111in}}%
\pgfpathlineto{\pgfqpoint{9.726466in}{3.345111in}}%
\pgfpathclose%
\pgfusepath{stroke}%
\end{pgfscope}%
\begin{pgfscope}%
\pgfpathrectangle{\pgfqpoint{7.512535in}{0.437222in}}{\pgfqpoint{6.275590in}{5.159444in}}%
\pgfusepath{clip}%
\pgfsetbuttcap%
\pgfsetroundjoin%
\pgfsetlinewidth{1.003750pt}%
\definecolor{currentstroke}{rgb}{0.827451,0.827451,0.827451}%
\pgfsetstrokecolor{currentstroke}%
\pgfsetstrokeopacity{0.800000}%
\pgfsetdash{}{0pt}%
\pgfpathmoveto{\pgfqpoint{8.022992in}{0.688351in}}%
\pgfpathcurveto{\pgfqpoint{8.034042in}{0.688351in}}{\pgfqpoint{8.044641in}{0.692741in}}{\pgfqpoint{8.052455in}{0.700555in}}%
\pgfpathcurveto{\pgfqpoint{8.060269in}{0.708368in}}{\pgfqpoint{8.064659in}{0.718967in}}{\pgfqpoint{8.064659in}{0.730017in}}%
\pgfpathcurveto{\pgfqpoint{8.064659in}{0.741068in}}{\pgfqpoint{8.060269in}{0.751667in}}{\pgfqpoint{8.052455in}{0.759480in}}%
\pgfpathcurveto{\pgfqpoint{8.044641in}{0.767294in}}{\pgfqpoint{8.034042in}{0.771684in}}{\pgfqpoint{8.022992in}{0.771684in}}%
\pgfpathcurveto{\pgfqpoint{8.011942in}{0.771684in}}{\pgfqpoint{8.001343in}{0.767294in}}{\pgfqpoint{7.993529in}{0.759480in}}%
\pgfpathcurveto{\pgfqpoint{7.985716in}{0.751667in}}{\pgfqpoint{7.981326in}{0.741068in}}{\pgfqpoint{7.981326in}{0.730017in}}%
\pgfpathcurveto{\pgfqpoint{7.981326in}{0.718967in}}{\pgfqpoint{7.985716in}{0.708368in}}{\pgfqpoint{7.993529in}{0.700555in}}%
\pgfpathcurveto{\pgfqpoint{8.001343in}{0.692741in}}{\pgfqpoint{8.011942in}{0.688351in}}{\pgfqpoint{8.022992in}{0.688351in}}%
\pgfpathlineto{\pgfqpoint{8.022992in}{0.688351in}}%
\pgfpathclose%
\pgfusepath{stroke}%
\end{pgfscope}%
\begin{pgfscope}%
\pgfpathrectangle{\pgfqpoint{7.512535in}{0.437222in}}{\pgfqpoint{6.275590in}{5.159444in}}%
\pgfusepath{clip}%
\pgfsetbuttcap%
\pgfsetroundjoin%
\pgfsetlinewidth{1.003750pt}%
\definecolor{currentstroke}{rgb}{0.827451,0.827451,0.827451}%
\pgfsetstrokecolor{currentstroke}%
\pgfsetstrokeopacity{0.800000}%
\pgfsetdash{}{0pt}%
\pgfpathmoveto{\pgfqpoint{8.261759in}{2.309353in}}%
\pgfpathcurveto{\pgfqpoint{8.272809in}{2.309353in}}{\pgfqpoint{8.283408in}{2.313744in}}{\pgfqpoint{8.291221in}{2.321557in}}%
\pgfpathcurveto{\pgfqpoint{8.299035in}{2.329371in}}{\pgfqpoint{8.303425in}{2.339970in}}{\pgfqpoint{8.303425in}{2.351020in}}%
\pgfpathcurveto{\pgfqpoint{8.303425in}{2.362070in}}{\pgfqpoint{8.299035in}{2.372669in}}{\pgfqpoint{8.291221in}{2.380483in}}%
\pgfpathcurveto{\pgfqpoint{8.283408in}{2.388296in}}{\pgfqpoint{8.272809in}{2.392687in}}{\pgfqpoint{8.261759in}{2.392687in}}%
\pgfpathcurveto{\pgfqpoint{8.250708in}{2.392687in}}{\pgfqpoint{8.240109in}{2.388296in}}{\pgfqpoint{8.232296in}{2.380483in}}%
\pgfpathcurveto{\pgfqpoint{8.224482in}{2.372669in}}{\pgfqpoint{8.220092in}{2.362070in}}{\pgfqpoint{8.220092in}{2.351020in}}%
\pgfpathcurveto{\pgfqpoint{8.220092in}{2.339970in}}{\pgfqpoint{8.224482in}{2.329371in}}{\pgfqpoint{8.232296in}{2.321557in}}%
\pgfpathcurveto{\pgfqpoint{8.240109in}{2.313744in}}{\pgfqpoint{8.250708in}{2.309353in}}{\pgfqpoint{8.261759in}{2.309353in}}%
\pgfpathlineto{\pgfqpoint{8.261759in}{2.309353in}}%
\pgfpathclose%
\pgfusepath{stroke}%
\end{pgfscope}%
\begin{pgfscope}%
\pgfpathrectangle{\pgfqpoint{7.512535in}{0.437222in}}{\pgfqpoint{6.275590in}{5.159444in}}%
\pgfusepath{clip}%
\pgfsetbuttcap%
\pgfsetroundjoin%
\pgfsetlinewidth{1.003750pt}%
\definecolor{currentstroke}{rgb}{0.827451,0.827451,0.827451}%
\pgfsetstrokecolor{currentstroke}%
\pgfsetstrokeopacity{0.800000}%
\pgfsetdash{}{0pt}%
\pgfpathmoveto{\pgfqpoint{9.463978in}{2.958986in}}%
\pgfpathcurveto{\pgfqpoint{9.475028in}{2.958986in}}{\pgfqpoint{9.485627in}{2.963376in}}{\pgfqpoint{9.493441in}{2.971190in}}%
\pgfpathcurveto{\pgfqpoint{9.501254in}{2.979003in}}{\pgfqpoint{9.505644in}{2.989602in}}{\pgfqpoint{9.505644in}{3.000653in}}%
\pgfpathcurveto{\pgfqpoint{9.505644in}{3.011703in}}{\pgfqpoint{9.501254in}{3.022302in}}{\pgfqpoint{9.493441in}{3.030115in}}%
\pgfpathcurveto{\pgfqpoint{9.485627in}{3.037929in}}{\pgfqpoint{9.475028in}{3.042319in}}{\pgfqpoint{9.463978in}{3.042319in}}%
\pgfpathcurveto{\pgfqpoint{9.452928in}{3.042319in}}{\pgfqpoint{9.442329in}{3.037929in}}{\pgfqpoint{9.434515in}{3.030115in}}%
\pgfpathcurveto{\pgfqpoint{9.426701in}{3.022302in}}{\pgfqpoint{9.422311in}{3.011703in}}{\pgfqpoint{9.422311in}{3.000653in}}%
\pgfpathcurveto{\pgfqpoint{9.422311in}{2.989602in}}{\pgfqpoint{9.426701in}{2.979003in}}{\pgfqpoint{9.434515in}{2.971190in}}%
\pgfpathcurveto{\pgfqpoint{9.442329in}{2.963376in}}{\pgfqpoint{9.452928in}{2.958986in}}{\pgfqpoint{9.463978in}{2.958986in}}%
\pgfpathlineto{\pgfqpoint{9.463978in}{2.958986in}}%
\pgfpathclose%
\pgfusepath{stroke}%
\end{pgfscope}%
\begin{pgfscope}%
\pgfpathrectangle{\pgfqpoint{7.512535in}{0.437222in}}{\pgfqpoint{6.275590in}{5.159444in}}%
\pgfusepath{clip}%
\pgfsetbuttcap%
\pgfsetroundjoin%
\pgfsetlinewidth{1.003750pt}%
\definecolor{currentstroke}{rgb}{0.827451,0.827451,0.827451}%
\pgfsetstrokecolor{currentstroke}%
\pgfsetstrokeopacity{0.800000}%
\pgfsetdash{}{0pt}%
\pgfpathmoveto{\pgfqpoint{10.127909in}{4.483356in}}%
\pgfpathcurveto{\pgfqpoint{10.138959in}{4.483356in}}{\pgfqpoint{10.149558in}{4.487746in}}{\pgfqpoint{10.157372in}{4.495560in}}%
\pgfpathcurveto{\pgfqpoint{10.165186in}{4.503373in}}{\pgfqpoint{10.169576in}{4.513972in}}{\pgfqpoint{10.169576in}{4.525022in}}%
\pgfpathcurveto{\pgfqpoint{10.169576in}{4.536072in}}{\pgfqpoint{10.165186in}{4.546671in}}{\pgfqpoint{10.157372in}{4.554485in}}%
\pgfpathcurveto{\pgfqpoint{10.149558in}{4.562299in}}{\pgfqpoint{10.138959in}{4.566689in}}{\pgfqpoint{10.127909in}{4.566689in}}%
\pgfpathcurveto{\pgfqpoint{10.116859in}{4.566689in}}{\pgfqpoint{10.106260in}{4.562299in}}{\pgfqpoint{10.098446in}{4.554485in}}%
\pgfpathcurveto{\pgfqpoint{10.090633in}{4.546671in}}{\pgfqpoint{10.086243in}{4.536072in}}{\pgfqpoint{10.086243in}{4.525022in}}%
\pgfpathcurveto{\pgfqpoint{10.086243in}{4.513972in}}{\pgfqpoint{10.090633in}{4.503373in}}{\pgfqpoint{10.098446in}{4.495560in}}%
\pgfpathcurveto{\pgfqpoint{10.106260in}{4.487746in}}{\pgfqpoint{10.116859in}{4.483356in}}{\pgfqpoint{10.127909in}{4.483356in}}%
\pgfpathlineto{\pgfqpoint{10.127909in}{4.483356in}}%
\pgfpathclose%
\pgfusepath{stroke}%
\end{pgfscope}%
\begin{pgfscope}%
\pgfpathrectangle{\pgfqpoint{7.512535in}{0.437222in}}{\pgfqpoint{6.275590in}{5.159444in}}%
\pgfusepath{clip}%
\pgfsetbuttcap%
\pgfsetroundjoin%
\pgfsetlinewidth{1.003750pt}%
\definecolor{currentstroke}{rgb}{0.827451,0.827451,0.827451}%
\pgfsetstrokecolor{currentstroke}%
\pgfsetstrokeopacity{0.800000}%
\pgfsetdash{}{0pt}%
\pgfpathmoveto{\pgfqpoint{8.506716in}{1.320145in}}%
\pgfpathcurveto{\pgfqpoint{8.517766in}{1.320145in}}{\pgfqpoint{8.528366in}{1.324535in}}{\pgfqpoint{8.536179in}{1.332349in}}%
\pgfpathcurveto{\pgfqpoint{8.543993in}{1.340162in}}{\pgfqpoint{8.548383in}{1.350761in}}{\pgfqpoint{8.548383in}{1.361811in}}%
\pgfpathcurveto{\pgfqpoint{8.548383in}{1.372862in}}{\pgfqpoint{8.543993in}{1.383461in}}{\pgfqpoint{8.536179in}{1.391274in}}%
\pgfpathcurveto{\pgfqpoint{8.528366in}{1.399088in}}{\pgfqpoint{8.517766in}{1.403478in}}{\pgfqpoint{8.506716in}{1.403478in}}%
\pgfpathcurveto{\pgfqpoint{8.495666in}{1.403478in}}{\pgfqpoint{8.485067in}{1.399088in}}{\pgfqpoint{8.477254in}{1.391274in}}%
\pgfpathcurveto{\pgfqpoint{8.469440in}{1.383461in}}{\pgfqpoint{8.465050in}{1.372862in}}{\pgfqpoint{8.465050in}{1.361811in}}%
\pgfpathcurveto{\pgfqpoint{8.465050in}{1.350761in}}{\pgfqpoint{8.469440in}{1.340162in}}{\pgfqpoint{8.477254in}{1.332349in}}%
\pgfpathcurveto{\pgfqpoint{8.485067in}{1.324535in}}{\pgfqpoint{8.495666in}{1.320145in}}{\pgfqpoint{8.506716in}{1.320145in}}%
\pgfpathlineto{\pgfqpoint{8.506716in}{1.320145in}}%
\pgfpathclose%
\pgfusepath{stroke}%
\end{pgfscope}%
\begin{pgfscope}%
\pgfpathrectangle{\pgfqpoint{7.512535in}{0.437222in}}{\pgfqpoint{6.275590in}{5.159444in}}%
\pgfusepath{clip}%
\pgfsetbuttcap%
\pgfsetroundjoin%
\pgfsetlinewidth{1.003750pt}%
\definecolor{currentstroke}{rgb}{0.827451,0.827451,0.827451}%
\pgfsetstrokecolor{currentstroke}%
\pgfsetstrokeopacity{0.800000}%
\pgfsetdash{}{0pt}%
\pgfpathmoveto{\pgfqpoint{10.818589in}{4.898946in}}%
\pgfpathcurveto{\pgfqpoint{10.829639in}{4.898946in}}{\pgfqpoint{10.840238in}{4.903336in}}{\pgfqpoint{10.848052in}{4.911150in}}%
\pgfpathcurveto{\pgfqpoint{10.855865in}{4.918964in}}{\pgfqpoint{10.860255in}{4.929563in}}{\pgfqpoint{10.860255in}{4.940613in}}%
\pgfpathcurveto{\pgfqpoint{10.860255in}{4.951663in}}{\pgfqpoint{10.855865in}{4.962262in}}{\pgfqpoint{10.848052in}{4.970076in}}%
\pgfpathcurveto{\pgfqpoint{10.840238in}{4.977889in}}{\pgfqpoint{10.829639in}{4.982279in}}{\pgfqpoint{10.818589in}{4.982279in}}%
\pgfpathcurveto{\pgfqpoint{10.807539in}{4.982279in}}{\pgfqpoint{10.796940in}{4.977889in}}{\pgfqpoint{10.789126in}{4.970076in}}%
\pgfpathcurveto{\pgfqpoint{10.781312in}{4.962262in}}{\pgfqpoint{10.776922in}{4.951663in}}{\pgfqpoint{10.776922in}{4.940613in}}%
\pgfpathcurveto{\pgfqpoint{10.776922in}{4.929563in}}{\pgfqpoint{10.781312in}{4.918964in}}{\pgfqpoint{10.789126in}{4.911150in}}%
\pgfpathcurveto{\pgfqpoint{10.796940in}{4.903336in}}{\pgfqpoint{10.807539in}{4.898946in}}{\pgfqpoint{10.818589in}{4.898946in}}%
\pgfpathlineto{\pgfqpoint{10.818589in}{4.898946in}}%
\pgfpathclose%
\pgfusepath{stroke}%
\end{pgfscope}%
\begin{pgfscope}%
\pgfpathrectangle{\pgfqpoint{7.512535in}{0.437222in}}{\pgfqpoint{6.275590in}{5.159444in}}%
\pgfusepath{clip}%
\pgfsetbuttcap%
\pgfsetroundjoin%
\pgfsetlinewidth{1.003750pt}%
\definecolor{currentstroke}{rgb}{0.827451,0.827451,0.827451}%
\pgfsetstrokecolor{currentstroke}%
\pgfsetstrokeopacity{0.800000}%
\pgfsetdash{}{0pt}%
\pgfpathmoveto{\pgfqpoint{9.540773in}{3.442114in}}%
\pgfpathcurveto{\pgfqpoint{9.551823in}{3.442114in}}{\pgfqpoint{9.562422in}{3.446504in}}{\pgfqpoint{9.570236in}{3.454318in}}%
\pgfpathcurveto{\pgfqpoint{9.578049in}{3.462131in}}{\pgfqpoint{9.582440in}{3.472730in}}{\pgfqpoint{9.582440in}{3.483780in}}%
\pgfpathcurveto{\pgfqpoint{9.582440in}{3.494830in}}{\pgfqpoint{9.578049in}{3.505429in}}{\pgfqpoint{9.570236in}{3.513243in}}%
\pgfpathcurveto{\pgfqpoint{9.562422in}{3.521057in}}{\pgfqpoint{9.551823in}{3.525447in}}{\pgfqpoint{9.540773in}{3.525447in}}%
\pgfpathcurveto{\pgfqpoint{9.529723in}{3.525447in}}{\pgfqpoint{9.519124in}{3.521057in}}{\pgfqpoint{9.511310in}{3.513243in}}%
\pgfpathcurveto{\pgfqpoint{9.503497in}{3.505429in}}{\pgfqpoint{9.499106in}{3.494830in}}{\pgfqpoint{9.499106in}{3.483780in}}%
\pgfpathcurveto{\pgfqpoint{9.499106in}{3.472730in}}{\pgfqpoint{9.503497in}{3.462131in}}{\pgfqpoint{9.511310in}{3.454318in}}%
\pgfpathcurveto{\pgfqpoint{9.519124in}{3.446504in}}{\pgfqpoint{9.529723in}{3.442114in}}{\pgfqpoint{9.540773in}{3.442114in}}%
\pgfpathlineto{\pgfqpoint{9.540773in}{3.442114in}}%
\pgfpathclose%
\pgfusepath{stroke}%
\end{pgfscope}%
\begin{pgfscope}%
\pgfpathrectangle{\pgfqpoint{7.512535in}{0.437222in}}{\pgfqpoint{6.275590in}{5.159444in}}%
\pgfusepath{clip}%
\pgfsetbuttcap%
\pgfsetroundjoin%
\pgfsetlinewidth{1.003750pt}%
\definecolor{currentstroke}{rgb}{0.827451,0.827451,0.827451}%
\pgfsetstrokecolor{currentstroke}%
\pgfsetstrokeopacity{0.800000}%
\pgfsetdash{}{0pt}%
\pgfpathmoveto{\pgfqpoint{10.013603in}{3.105491in}}%
\pgfpathcurveto{\pgfqpoint{10.024653in}{3.105491in}}{\pgfqpoint{10.035252in}{3.109882in}}{\pgfqpoint{10.043066in}{3.117695in}}%
\pgfpathcurveto{\pgfqpoint{10.050880in}{3.125509in}}{\pgfqpoint{10.055270in}{3.136108in}}{\pgfqpoint{10.055270in}{3.147158in}}%
\pgfpathcurveto{\pgfqpoint{10.055270in}{3.158208in}}{\pgfqpoint{10.050880in}{3.168807in}}{\pgfqpoint{10.043066in}{3.176621in}}%
\pgfpathcurveto{\pgfqpoint{10.035252in}{3.184434in}}{\pgfqpoint{10.024653in}{3.188825in}}{\pgfqpoint{10.013603in}{3.188825in}}%
\pgfpathcurveto{\pgfqpoint{10.002553in}{3.188825in}}{\pgfqpoint{9.991954in}{3.184434in}}{\pgfqpoint{9.984141in}{3.176621in}}%
\pgfpathcurveto{\pgfqpoint{9.976327in}{3.168807in}}{\pgfqpoint{9.971937in}{3.158208in}}{\pgfqpoint{9.971937in}{3.147158in}}%
\pgfpathcurveto{\pgfqpoint{9.971937in}{3.136108in}}{\pgfqpoint{9.976327in}{3.125509in}}{\pgfqpoint{9.984141in}{3.117695in}}%
\pgfpathcurveto{\pgfqpoint{9.991954in}{3.109882in}}{\pgfqpoint{10.002553in}{3.105491in}}{\pgfqpoint{10.013603in}{3.105491in}}%
\pgfpathlineto{\pgfqpoint{10.013603in}{3.105491in}}%
\pgfpathclose%
\pgfusepath{stroke}%
\end{pgfscope}%
\begin{pgfscope}%
\pgfpathrectangle{\pgfqpoint{7.512535in}{0.437222in}}{\pgfqpoint{6.275590in}{5.159444in}}%
\pgfusepath{clip}%
\pgfsetbuttcap%
\pgfsetroundjoin%
\pgfsetlinewidth{1.003750pt}%
\definecolor{currentstroke}{rgb}{0.827451,0.827451,0.827451}%
\pgfsetstrokecolor{currentstroke}%
\pgfsetstrokeopacity{0.800000}%
\pgfsetdash{}{0pt}%
\pgfpathmoveto{\pgfqpoint{9.928499in}{4.056499in}}%
\pgfpathcurveto{\pgfqpoint{9.939549in}{4.056499in}}{\pgfqpoint{9.950148in}{4.060889in}}{\pgfqpoint{9.957962in}{4.068703in}}%
\pgfpathcurveto{\pgfqpoint{9.965775in}{4.076516in}}{\pgfqpoint{9.970165in}{4.087115in}}{\pgfqpoint{9.970165in}{4.098166in}}%
\pgfpathcurveto{\pgfqpoint{9.970165in}{4.109216in}}{\pgfqpoint{9.965775in}{4.119815in}}{\pgfqpoint{9.957962in}{4.127628in}}%
\pgfpathcurveto{\pgfqpoint{9.950148in}{4.135442in}}{\pgfqpoint{9.939549in}{4.139832in}}{\pgfqpoint{9.928499in}{4.139832in}}%
\pgfpathcurveto{\pgfqpoint{9.917449in}{4.139832in}}{\pgfqpoint{9.906850in}{4.135442in}}{\pgfqpoint{9.899036in}{4.127628in}}%
\pgfpathcurveto{\pgfqpoint{9.891222in}{4.119815in}}{\pgfqpoint{9.886832in}{4.109216in}}{\pgfqpoint{9.886832in}{4.098166in}}%
\pgfpathcurveto{\pgfqpoint{9.886832in}{4.087115in}}{\pgfqpoint{9.891222in}{4.076516in}}{\pgfqpoint{9.899036in}{4.068703in}}%
\pgfpathcurveto{\pgfqpoint{9.906850in}{4.060889in}}{\pgfqpoint{9.917449in}{4.056499in}}{\pgfqpoint{9.928499in}{4.056499in}}%
\pgfpathlineto{\pgfqpoint{9.928499in}{4.056499in}}%
\pgfpathclose%
\pgfusepath{stroke}%
\end{pgfscope}%
\begin{pgfscope}%
\pgfpathrectangle{\pgfqpoint{7.512535in}{0.437222in}}{\pgfqpoint{6.275590in}{5.159444in}}%
\pgfusepath{clip}%
\pgfsetbuttcap%
\pgfsetroundjoin%
\pgfsetlinewidth{1.003750pt}%
\definecolor{currentstroke}{rgb}{0.827451,0.827451,0.827451}%
\pgfsetstrokecolor{currentstroke}%
\pgfsetstrokeopacity{0.800000}%
\pgfsetdash{}{0pt}%
\pgfpathmoveto{\pgfqpoint{13.101307in}{5.469388in}}%
\pgfpathcurveto{\pgfqpoint{13.112357in}{5.469388in}}{\pgfqpoint{13.122956in}{5.473778in}}{\pgfqpoint{13.130770in}{5.481592in}}%
\pgfpathcurveto{\pgfqpoint{13.138583in}{5.489405in}}{\pgfqpoint{13.142973in}{5.500004in}}{\pgfqpoint{13.142973in}{5.511054in}}%
\pgfpathcurveto{\pgfqpoint{13.142973in}{5.522105in}}{\pgfqpoint{13.138583in}{5.532704in}}{\pgfqpoint{13.130770in}{5.540517in}}%
\pgfpathcurveto{\pgfqpoint{13.122956in}{5.548331in}}{\pgfqpoint{13.112357in}{5.552721in}}{\pgfqpoint{13.101307in}{5.552721in}}%
\pgfpathcurveto{\pgfqpoint{13.090257in}{5.552721in}}{\pgfqpoint{13.079658in}{5.548331in}}{\pgfqpoint{13.071844in}{5.540517in}}%
\pgfpathcurveto{\pgfqpoint{13.064030in}{5.532704in}}{\pgfqpoint{13.059640in}{5.522105in}}{\pgfqpoint{13.059640in}{5.511054in}}%
\pgfpathcurveto{\pgfqpoint{13.059640in}{5.500004in}}{\pgfqpoint{13.064030in}{5.489405in}}{\pgfqpoint{13.071844in}{5.481592in}}%
\pgfpathcurveto{\pgfqpoint{13.079658in}{5.473778in}}{\pgfqpoint{13.090257in}{5.469388in}}{\pgfqpoint{13.101307in}{5.469388in}}%
\pgfpathlineto{\pgfqpoint{13.101307in}{5.469388in}}%
\pgfpathclose%
\pgfusepath{stroke}%
\end{pgfscope}%
\begin{pgfscope}%
\pgfpathrectangle{\pgfqpoint{7.512535in}{0.437222in}}{\pgfqpoint{6.275590in}{5.159444in}}%
\pgfusepath{clip}%
\pgfsetbuttcap%
\pgfsetroundjoin%
\pgfsetlinewidth{1.003750pt}%
\definecolor{currentstroke}{rgb}{0.827451,0.827451,0.827451}%
\pgfsetstrokecolor{currentstroke}%
\pgfsetstrokeopacity{0.800000}%
\pgfsetdash{}{0pt}%
\pgfpathmoveto{\pgfqpoint{8.619483in}{1.654089in}}%
\pgfpathcurveto{\pgfqpoint{8.630533in}{1.654089in}}{\pgfqpoint{8.641132in}{1.658480in}}{\pgfqpoint{8.648946in}{1.666293in}}%
\pgfpathcurveto{\pgfqpoint{8.656759in}{1.674107in}}{\pgfqpoint{8.661149in}{1.684706in}}{\pgfqpoint{8.661149in}{1.695756in}}%
\pgfpathcurveto{\pgfqpoint{8.661149in}{1.706806in}}{\pgfqpoint{8.656759in}{1.717405in}}{\pgfqpoint{8.648946in}{1.725219in}}%
\pgfpathcurveto{\pgfqpoint{8.641132in}{1.733033in}}{\pgfqpoint{8.630533in}{1.737423in}}{\pgfqpoint{8.619483in}{1.737423in}}%
\pgfpathcurveto{\pgfqpoint{8.608433in}{1.737423in}}{\pgfqpoint{8.597834in}{1.733033in}}{\pgfqpoint{8.590020in}{1.725219in}}%
\pgfpathcurveto{\pgfqpoint{8.582206in}{1.717405in}}{\pgfqpoint{8.577816in}{1.706806in}}{\pgfqpoint{8.577816in}{1.695756in}}%
\pgfpathcurveto{\pgfqpoint{8.577816in}{1.684706in}}{\pgfqpoint{8.582206in}{1.674107in}}{\pgfqpoint{8.590020in}{1.666293in}}%
\pgfpathcurveto{\pgfqpoint{8.597834in}{1.658480in}}{\pgfqpoint{8.608433in}{1.654089in}}{\pgfqpoint{8.619483in}{1.654089in}}%
\pgfpathlineto{\pgfqpoint{8.619483in}{1.654089in}}%
\pgfpathclose%
\pgfusepath{stroke}%
\end{pgfscope}%
\begin{pgfscope}%
\pgfpathrectangle{\pgfqpoint{7.512535in}{0.437222in}}{\pgfqpoint{6.275590in}{5.159444in}}%
\pgfusepath{clip}%
\pgfsetbuttcap%
\pgfsetroundjoin%
\pgfsetlinewidth{1.003750pt}%
\definecolor{currentstroke}{rgb}{0.827451,0.827451,0.827451}%
\pgfsetstrokecolor{currentstroke}%
\pgfsetstrokeopacity{0.800000}%
\pgfsetdash{}{0pt}%
\pgfpathmoveto{\pgfqpoint{11.629363in}{5.304399in}}%
\pgfpathcurveto{\pgfqpoint{11.640413in}{5.304399in}}{\pgfqpoint{11.651012in}{5.308790in}}{\pgfqpoint{11.658826in}{5.316603in}}%
\pgfpathcurveto{\pgfqpoint{11.666639in}{5.324417in}}{\pgfqpoint{11.671030in}{5.335016in}}{\pgfqpoint{11.671030in}{5.346066in}}%
\pgfpathcurveto{\pgfqpoint{11.671030in}{5.357116in}}{\pgfqpoint{11.666639in}{5.367715in}}{\pgfqpoint{11.658826in}{5.375529in}}%
\pgfpathcurveto{\pgfqpoint{11.651012in}{5.383342in}}{\pgfqpoint{11.640413in}{5.387733in}}{\pgfqpoint{11.629363in}{5.387733in}}%
\pgfpathcurveto{\pgfqpoint{11.618313in}{5.387733in}}{\pgfqpoint{11.607714in}{5.383342in}}{\pgfqpoint{11.599900in}{5.375529in}}%
\pgfpathcurveto{\pgfqpoint{11.592087in}{5.367715in}}{\pgfqpoint{11.587696in}{5.357116in}}{\pgfqpoint{11.587696in}{5.346066in}}%
\pgfpathcurveto{\pgfqpoint{11.587696in}{5.335016in}}{\pgfqpoint{11.592087in}{5.324417in}}{\pgfqpoint{11.599900in}{5.316603in}}%
\pgfpathcurveto{\pgfqpoint{11.607714in}{5.308790in}}{\pgfqpoint{11.618313in}{5.304399in}}{\pgfqpoint{11.629363in}{5.304399in}}%
\pgfpathlineto{\pgfqpoint{11.629363in}{5.304399in}}%
\pgfpathclose%
\pgfusepath{stroke}%
\end{pgfscope}%
\begin{pgfscope}%
\pgfpathrectangle{\pgfqpoint{7.512535in}{0.437222in}}{\pgfqpoint{6.275590in}{5.159444in}}%
\pgfusepath{clip}%
\pgfsetbuttcap%
\pgfsetroundjoin%
\pgfsetlinewidth{1.003750pt}%
\definecolor{currentstroke}{rgb}{0.827451,0.827451,0.827451}%
\pgfsetstrokecolor{currentstroke}%
\pgfsetstrokeopacity{0.800000}%
\pgfsetdash{}{0pt}%
\pgfpathmoveto{\pgfqpoint{10.199165in}{4.325639in}}%
\pgfpathcurveto{\pgfqpoint{10.210215in}{4.325639in}}{\pgfqpoint{10.220814in}{4.330030in}}{\pgfqpoint{10.228628in}{4.337843in}}%
\pgfpathcurveto{\pgfqpoint{10.236442in}{4.345657in}}{\pgfqpoint{10.240832in}{4.356256in}}{\pgfqpoint{10.240832in}{4.367306in}}%
\pgfpathcurveto{\pgfqpoint{10.240832in}{4.378356in}}{\pgfqpoint{10.236442in}{4.388955in}}{\pgfqpoint{10.228628in}{4.396769in}}%
\pgfpathcurveto{\pgfqpoint{10.220814in}{4.404583in}}{\pgfqpoint{10.210215in}{4.408973in}}{\pgfqpoint{10.199165in}{4.408973in}}%
\pgfpathcurveto{\pgfqpoint{10.188115in}{4.408973in}}{\pgfqpoint{10.177516in}{4.404583in}}{\pgfqpoint{10.169702in}{4.396769in}}%
\pgfpathcurveto{\pgfqpoint{10.161889in}{4.388955in}}{\pgfqpoint{10.157498in}{4.378356in}}{\pgfqpoint{10.157498in}{4.367306in}}%
\pgfpathcurveto{\pgfqpoint{10.157498in}{4.356256in}}{\pgfqpoint{10.161889in}{4.345657in}}{\pgfqpoint{10.169702in}{4.337843in}}%
\pgfpathcurveto{\pgfqpoint{10.177516in}{4.330030in}}{\pgfqpoint{10.188115in}{4.325639in}}{\pgfqpoint{10.199165in}{4.325639in}}%
\pgfpathlineto{\pgfqpoint{10.199165in}{4.325639in}}%
\pgfpathclose%
\pgfusepath{stroke}%
\end{pgfscope}%
\begin{pgfscope}%
\pgfpathrectangle{\pgfqpoint{7.512535in}{0.437222in}}{\pgfqpoint{6.275590in}{5.159444in}}%
\pgfusepath{clip}%
\pgfsetbuttcap%
\pgfsetroundjoin%
\pgfsetlinewidth{1.003750pt}%
\definecolor{currentstroke}{rgb}{0.827451,0.827451,0.827451}%
\pgfsetstrokecolor{currentstroke}%
\pgfsetstrokeopacity{0.800000}%
\pgfsetdash{}{0pt}%
\pgfpathmoveto{\pgfqpoint{9.697330in}{2.947479in}}%
\pgfpathcurveto{\pgfqpoint{9.708380in}{2.947479in}}{\pgfqpoint{9.718979in}{2.951869in}}{\pgfqpoint{9.726792in}{2.959683in}}%
\pgfpathcurveto{\pgfqpoint{9.734606in}{2.967496in}}{\pgfqpoint{9.738996in}{2.978095in}}{\pgfqpoint{9.738996in}{2.989145in}}%
\pgfpathcurveto{\pgfqpoint{9.738996in}{3.000195in}}{\pgfqpoint{9.734606in}{3.010794in}}{\pgfqpoint{9.726792in}{3.018608in}}%
\pgfpathcurveto{\pgfqpoint{9.718979in}{3.026422in}}{\pgfqpoint{9.708380in}{3.030812in}}{\pgfqpoint{9.697330in}{3.030812in}}%
\pgfpathcurveto{\pgfqpoint{9.686279in}{3.030812in}}{\pgfqpoint{9.675680in}{3.026422in}}{\pgfqpoint{9.667867in}{3.018608in}}%
\pgfpathcurveto{\pgfqpoint{9.660053in}{3.010794in}}{\pgfqpoint{9.655663in}{3.000195in}}{\pgfqpoint{9.655663in}{2.989145in}}%
\pgfpathcurveto{\pgfqpoint{9.655663in}{2.978095in}}{\pgfqpoint{9.660053in}{2.967496in}}{\pgfqpoint{9.667867in}{2.959683in}}%
\pgfpathcurveto{\pgfqpoint{9.675680in}{2.951869in}}{\pgfqpoint{9.686279in}{2.947479in}}{\pgfqpoint{9.697330in}{2.947479in}}%
\pgfpathlineto{\pgfqpoint{9.697330in}{2.947479in}}%
\pgfpathclose%
\pgfusepath{stroke}%
\end{pgfscope}%
\begin{pgfscope}%
\pgfpathrectangle{\pgfqpoint{7.512535in}{0.437222in}}{\pgfqpoint{6.275590in}{5.159444in}}%
\pgfusepath{clip}%
\pgfsetbuttcap%
\pgfsetroundjoin%
\pgfsetlinewidth{1.003750pt}%
\definecolor{currentstroke}{rgb}{0.827451,0.827451,0.827451}%
\pgfsetstrokecolor{currentstroke}%
\pgfsetstrokeopacity{0.800000}%
\pgfsetdash{}{0pt}%
\pgfpathmoveto{\pgfqpoint{11.388601in}{4.626596in}}%
\pgfpathcurveto{\pgfqpoint{11.399651in}{4.626596in}}{\pgfqpoint{11.410250in}{4.630987in}}{\pgfqpoint{11.418064in}{4.638800in}}%
\pgfpathcurveto{\pgfqpoint{11.425878in}{4.646614in}}{\pgfqpoint{11.430268in}{4.657213in}}{\pgfqpoint{11.430268in}{4.668263in}}%
\pgfpathcurveto{\pgfqpoint{11.430268in}{4.679313in}}{\pgfqpoint{11.425878in}{4.689912in}}{\pgfqpoint{11.418064in}{4.697726in}}%
\pgfpathcurveto{\pgfqpoint{11.410250in}{4.705539in}}{\pgfqpoint{11.399651in}{4.709930in}}{\pgfqpoint{11.388601in}{4.709930in}}%
\pgfpathcurveto{\pgfqpoint{11.377551in}{4.709930in}}{\pgfqpoint{11.366952in}{4.705539in}}{\pgfqpoint{11.359139in}{4.697726in}}%
\pgfpathcurveto{\pgfqpoint{11.351325in}{4.689912in}}{\pgfqpoint{11.346935in}{4.679313in}}{\pgfqpoint{11.346935in}{4.668263in}}%
\pgfpathcurveto{\pgfqpoint{11.346935in}{4.657213in}}{\pgfqpoint{11.351325in}{4.646614in}}{\pgfqpoint{11.359139in}{4.638800in}}%
\pgfpathcurveto{\pgfqpoint{11.366952in}{4.630987in}}{\pgfqpoint{11.377551in}{4.626596in}}{\pgfqpoint{11.388601in}{4.626596in}}%
\pgfpathlineto{\pgfqpoint{11.388601in}{4.626596in}}%
\pgfpathclose%
\pgfusepath{stroke}%
\end{pgfscope}%
\begin{pgfscope}%
\pgfpathrectangle{\pgfqpoint{7.512535in}{0.437222in}}{\pgfqpoint{6.275590in}{5.159444in}}%
\pgfusepath{clip}%
\pgfsetbuttcap%
\pgfsetroundjoin%
\pgfsetlinewidth{1.003750pt}%
\definecolor{currentstroke}{rgb}{0.827451,0.827451,0.827451}%
\pgfsetstrokecolor{currentstroke}%
\pgfsetstrokeopacity{0.800000}%
\pgfsetdash{}{0pt}%
\pgfpathmoveto{\pgfqpoint{8.290348in}{1.139798in}}%
\pgfpathcurveto{\pgfqpoint{8.301398in}{1.139798in}}{\pgfqpoint{8.311997in}{1.144188in}}{\pgfqpoint{8.319811in}{1.152002in}}%
\pgfpathcurveto{\pgfqpoint{8.327625in}{1.159815in}}{\pgfqpoint{8.332015in}{1.170414in}}{\pgfqpoint{8.332015in}{1.181465in}}%
\pgfpathcurveto{\pgfqpoint{8.332015in}{1.192515in}}{\pgfqpoint{8.327625in}{1.203114in}}{\pgfqpoint{8.319811in}{1.210927in}}%
\pgfpathcurveto{\pgfqpoint{8.311997in}{1.218741in}}{\pgfqpoint{8.301398in}{1.223131in}}{\pgfqpoint{8.290348in}{1.223131in}}%
\pgfpathcurveto{\pgfqpoint{8.279298in}{1.223131in}}{\pgfqpoint{8.268699in}{1.218741in}}{\pgfqpoint{8.260886in}{1.210927in}}%
\pgfpathcurveto{\pgfqpoint{8.253072in}{1.203114in}}{\pgfqpoint{8.248682in}{1.192515in}}{\pgfqpoint{8.248682in}{1.181465in}}%
\pgfpathcurveto{\pgfqpoint{8.248682in}{1.170414in}}{\pgfqpoint{8.253072in}{1.159815in}}{\pgfqpoint{8.260886in}{1.152002in}}%
\pgfpathcurveto{\pgfqpoint{8.268699in}{1.144188in}}{\pgfqpoint{8.279298in}{1.139798in}}{\pgfqpoint{8.290348in}{1.139798in}}%
\pgfpathlineto{\pgfqpoint{8.290348in}{1.139798in}}%
\pgfpathclose%
\pgfusepath{stroke}%
\end{pgfscope}%
\begin{pgfscope}%
\pgfpathrectangle{\pgfqpoint{7.512535in}{0.437222in}}{\pgfqpoint{6.275590in}{5.159444in}}%
\pgfusepath{clip}%
\pgfsetbuttcap%
\pgfsetroundjoin%
\pgfsetlinewidth{1.003750pt}%
\definecolor{currentstroke}{rgb}{0.827451,0.827451,0.827451}%
\pgfsetstrokecolor{currentstroke}%
\pgfsetstrokeopacity{0.800000}%
\pgfsetdash{}{0pt}%
\pgfpathmoveto{\pgfqpoint{11.443872in}{5.101603in}}%
\pgfpathcurveto{\pgfqpoint{11.454922in}{5.101603in}}{\pgfqpoint{11.465521in}{5.105993in}}{\pgfqpoint{11.473335in}{5.113807in}}%
\pgfpathcurveto{\pgfqpoint{11.481148in}{5.121620in}}{\pgfqpoint{11.485539in}{5.132219in}}{\pgfqpoint{11.485539in}{5.143270in}}%
\pgfpathcurveto{\pgfqpoint{11.485539in}{5.154320in}}{\pgfqpoint{11.481148in}{5.164919in}}{\pgfqpoint{11.473335in}{5.172732in}}%
\pgfpathcurveto{\pgfqpoint{11.465521in}{5.180546in}}{\pgfqpoint{11.454922in}{5.184936in}}{\pgfqpoint{11.443872in}{5.184936in}}%
\pgfpathcurveto{\pgfqpoint{11.432822in}{5.184936in}}{\pgfqpoint{11.422223in}{5.180546in}}{\pgfqpoint{11.414409in}{5.172732in}}%
\pgfpathcurveto{\pgfqpoint{11.406596in}{5.164919in}}{\pgfqpoint{11.402205in}{5.154320in}}{\pgfqpoint{11.402205in}{5.143270in}}%
\pgfpathcurveto{\pgfqpoint{11.402205in}{5.132219in}}{\pgfqpoint{11.406596in}{5.121620in}}{\pgfqpoint{11.414409in}{5.113807in}}%
\pgfpathcurveto{\pgfqpoint{11.422223in}{5.105993in}}{\pgfqpoint{11.432822in}{5.101603in}}{\pgfqpoint{11.443872in}{5.101603in}}%
\pgfpathlineto{\pgfqpoint{11.443872in}{5.101603in}}%
\pgfpathclose%
\pgfusepath{stroke}%
\end{pgfscope}%
\begin{pgfscope}%
\pgfpathrectangle{\pgfqpoint{7.512535in}{0.437222in}}{\pgfqpoint{6.275590in}{5.159444in}}%
\pgfusepath{clip}%
\pgfsetbuttcap%
\pgfsetroundjoin%
\pgfsetlinewidth{1.003750pt}%
\definecolor{currentstroke}{rgb}{0.827451,0.827451,0.827451}%
\pgfsetstrokecolor{currentstroke}%
\pgfsetstrokeopacity{0.800000}%
\pgfsetdash{}{0pt}%
\pgfpathmoveto{\pgfqpoint{10.127909in}{3.160252in}}%
\pgfpathcurveto{\pgfqpoint{10.138959in}{3.160252in}}{\pgfqpoint{10.149558in}{3.164643in}}{\pgfqpoint{10.157372in}{3.172456in}}%
\pgfpathcurveto{\pgfqpoint{10.165186in}{3.180270in}}{\pgfqpoint{10.169576in}{3.190869in}}{\pgfqpoint{10.169576in}{3.201919in}}%
\pgfpathcurveto{\pgfqpoint{10.169576in}{3.212969in}}{\pgfqpoint{10.165186in}{3.223568in}}{\pgfqpoint{10.157372in}{3.231382in}}%
\pgfpathcurveto{\pgfqpoint{10.149558in}{3.239196in}}{\pgfqpoint{10.138959in}{3.243586in}}{\pgfqpoint{10.127909in}{3.243586in}}%
\pgfpathcurveto{\pgfqpoint{10.116859in}{3.243586in}}{\pgfqpoint{10.106260in}{3.239196in}}{\pgfqpoint{10.098446in}{3.231382in}}%
\pgfpathcurveto{\pgfqpoint{10.090633in}{3.223568in}}{\pgfqpoint{10.086243in}{3.212969in}}{\pgfqpoint{10.086243in}{3.201919in}}%
\pgfpathcurveto{\pgfqpoint{10.086243in}{3.190869in}}{\pgfqpoint{10.090633in}{3.180270in}}{\pgfqpoint{10.098446in}{3.172456in}}%
\pgfpathcurveto{\pgfqpoint{10.106260in}{3.164643in}}{\pgfqpoint{10.116859in}{3.160252in}}{\pgfqpoint{10.127909in}{3.160252in}}%
\pgfpathlineto{\pgfqpoint{10.127909in}{3.160252in}}%
\pgfpathclose%
\pgfusepath{stroke}%
\end{pgfscope}%
\begin{pgfscope}%
\pgfpathrectangle{\pgfqpoint{7.512535in}{0.437222in}}{\pgfqpoint{6.275590in}{5.159444in}}%
\pgfusepath{clip}%
\pgfsetbuttcap%
\pgfsetroundjoin%
\pgfsetlinewidth{1.003750pt}%
\definecolor{currentstroke}{rgb}{0.827451,0.827451,0.827451}%
\pgfsetstrokecolor{currentstroke}%
\pgfsetstrokeopacity{0.800000}%
\pgfsetdash{}{0pt}%
\pgfpathmoveto{\pgfqpoint{12.473540in}{5.514145in}}%
\pgfpathcurveto{\pgfqpoint{12.484590in}{5.514145in}}{\pgfqpoint{12.495189in}{5.518535in}}{\pgfqpoint{12.503003in}{5.526349in}}%
\pgfpathcurveto{\pgfqpoint{12.510817in}{5.534162in}}{\pgfqpoint{12.515207in}{5.544761in}}{\pgfqpoint{12.515207in}{5.555811in}}%
\pgfpathcurveto{\pgfqpoint{12.515207in}{5.566862in}}{\pgfqpoint{12.510817in}{5.577461in}}{\pgfqpoint{12.503003in}{5.585274in}}%
\pgfpathcurveto{\pgfqpoint{12.495189in}{5.593088in}}{\pgfqpoint{12.484590in}{5.597478in}}{\pgfqpoint{12.473540in}{5.597478in}}%
\pgfpathcurveto{\pgfqpoint{12.462490in}{5.597478in}}{\pgfqpoint{12.451891in}{5.593088in}}{\pgfqpoint{12.444078in}{5.585274in}}%
\pgfpathcurveto{\pgfqpoint{12.436264in}{5.577461in}}{\pgfqpoint{12.431874in}{5.566862in}}{\pgfqpoint{12.431874in}{5.555811in}}%
\pgfpathcurveto{\pgfqpoint{12.431874in}{5.544761in}}{\pgfqpoint{12.436264in}{5.534162in}}{\pgfqpoint{12.444078in}{5.526349in}}%
\pgfpathcurveto{\pgfqpoint{12.451891in}{5.518535in}}{\pgfqpoint{12.462490in}{5.514145in}}{\pgfqpoint{12.473540in}{5.514145in}}%
\pgfpathlineto{\pgfqpoint{12.473540in}{5.514145in}}%
\pgfpathclose%
\pgfusepath{stroke}%
\end{pgfscope}%
\begin{pgfscope}%
\pgfpathrectangle{\pgfqpoint{7.512535in}{0.437222in}}{\pgfqpoint{6.275590in}{5.159444in}}%
\pgfusepath{clip}%
\pgfsetbuttcap%
\pgfsetroundjoin%
\pgfsetlinewidth{1.003750pt}%
\definecolor{currentstroke}{rgb}{0.827451,0.827451,0.827451}%
\pgfsetstrokecolor{currentstroke}%
\pgfsetstrokeopacity{0.800000}%
\pgfsetdash{}{0pt}%
\pgfpathmoveto{\pgfqpoint{9.615954in}{3.788157in}}%
\pgfpathcurveto{\pgfqpoint{9.627004in}{3.788157in}}{\pgfqpoint{9.637603in}{3.792548in}}{\pgfqpoint{9.645416in}{3.800361in}}%
\pgfpathcurveto{\pgfqpoint{9.653230in}{3.808175in}}{\pgfqpoint{9.657620in}{3.818774in}}{\pgfqpoint{9.657620in}{3.829824in}}%
\pgfpathcurveto{\pgfqpoint{9.657620in}{3.840874in}}{\pgfqpoint{9.653230in}{3.851473in}}{\pgfqpoint{9.645416in}{3.859287in}}%
\pgfpathcurveto{\pgfqpoint{9.637603in}{3.867100in}}{\pgfqpoint{9.627004in}{3.871491in}}{\pgfqpoint{9.615954in}{3.871491in}}%
\pgfpathcurveto{\pgfqpoint{9.604903in}{3.871491in}}{\pgfqpoint{9.594304in}{3.867100in}}{\pgfqpoint{9.586491in}{3.859287in}}%
\pgfpathcurveto{\pgfqpoint{9.578677in}{3.851473in}}{\pgfqpoint{9.574287in}{3.840874in}}{\pgfqpoint{9.574287in}{3.829824in}}%
\pgfpathcurveto{\pgfqpoint{9.574287in}{3.818774in}}{\pgfqpoint{9.578677in}{3.808175in}}{\pgfqpoint{9.586491in}{3.800361in}}%
\pgfpathcurveto{\pgfqpoint{9.594304in}{3.792548in}}{\pgfqpoint{9.604903in}{3.788157in}}{\pgfqpoint{9.615954in}{3.788157in}}%
\pgfpathlineto{\pgfqpoint{9.615954in}{3.788157in}}%
\pgfpathclose%
\pgfusepath{stroke}%
\end{pgfscope}%
\begin{pgfscope}%
\pgfpathrectangle{\pgfqpoint{7.512535in}{0.437222in}}{\pgfqpoint{6.275590in}{5.159444in}}%
\pgfusepath{clip}%
\pgfsetbuttcap%
\pgfsetroundjoin%
\pgfsetlinewidth{1.003750pt}%
\definecolor{currentstroke}{rgb}{0.827451,0.827451,0.827451}%
\pgfsetstrokecolor{currentstroke}%
\pgfsetstrokeopacity{0.800000}%
\pgfsetdash{}{0pt}%
\pgfpathmoveto{\pgfqpoint{12.342968in}{5.553091in}}%
\pgfpathcurveto{\pgfqpoint{12.354019in}{5.553091in}}{\pgfqpoint{12.364618in}{5.557481in}}{\pgfqpoint{12.372431in}{5.565295in}}%
\pgfpathcurveto{\pgfqpoint{12.380245in}{5.573108in}}{\pgfqpoint{12.384635in}{5.583707in}}{\pgfqpoint{12.384635in}{5.594757in}}%
\pgfpathcurveto{\pgfqpoint{12.384635in}{5.605808in}}{\pgfqpoint{12.380245in}{5.616407in}}{\pgfqpoint{12.372431in}{5.624220in}}%
\pgfpathcurveto{\pgfqpoint{12.364618in}{5.632034in}}{\pgfqpoint{12.354019in}{5.636424in}}{\pgfqpoint{12.342968in}{5.636424in}}%
\pgfpathcurveto{\pgfqpoint{12.331918in}{5.636424in}}{\pgfqpoint{12.321319in}{5.632034in}}{\pgfqpoint{12.313506in}{5.624220in}}%
\pgfpathcurveto{\pgfqpoint{12.305692in}{5.616407in}}{\pgfqpoint{12.301302in}{5.605808in}}{\pgfqpoint{12.301302in}{5.594757in}}%
\pgfpathcurveto{\pgfqpoint{12.301302in}{5.583707in}}{\pgfqpoint{12.305692in}{5.573108in}}{\pgfqpoint{12.313506in}{5.565295in}}%
\pgfpathcurveto{\pgfqpoint{12.321319in}{5.557481in}}{\pgfqpoint{12.331918in}{5.553091in}}{\pgfqpoint{12.342968in}{5.553091in}}%
\pgfpathlineto{\pgfqpoint{12.342968in}{5.553091in}}%
\pgfpathclose%
\pgfusepath{stroke}%
\end{pgfscope}%
\begin{pgfscope}%
\pgfpathrectangle{\pgfqpoint{7.512535in}{0.437222in}}{\pgfqpoint{6.275590in}{5.159444in}}%
\pgfusepath{clip}%
\pgfsetbuttcap%
\pgfsetroundjoin%
\pgfsetlinewidth{1.003750pt}%
\definecolor{currentstroke}{rgb}{0.827451,0.827451,0.827451}%
\pgfsetstrokecolor{currentstroke}%
\pgfsetstrokeopacity{0.800000}%
\pgfsetdash{}{0pt}%
\pgfpathmoveto{\pgfqpoint{10.259535in}{4.447570in}}%
\pgfpathcurveto{\pgfqpoint{10.270585in}{4.447570in}}{\pgfqpoint{10.281184in}{4.451960in}}{\pgfqpoint{10.288997in}{4.459773in}}%
\pgfpathcurveto{\pgfqpoint{10.296811in}{4.467587in}}{\pgfqpoint{10.301201in}{4.478186in}}{\pgfqpoint{10.301201in}{4.489236in}}%
\pgfpathcurveto{\pgfqpoint{10.301201in}{4.500286in}}{\pgfqpoint{10.296811in}{4.510885in}}{\pgfqpoint{10.288997in}{4.518699in}}%
\pgfpathcurveto{\pgfqpoint{10.281184in}{4.526513in}}{\pgfqpoint{10.270585in}{4.530903in}}{\pgfqpoint{10.259535in}{4.530903in}}%
\pgfpathcurveto{\pgfqpoint{10.248485in}{4.530903in}}{\pgfqpoint{10.237886in}{4.526513in}}{\pgfqpoint{10.230072in}{4.518699in}}%
\pgfpathcurveto{\pgfqpoint{10.222258in}{4.510885in}}{\pgfqpoint{10.217868in}{4.500286in}}{\pgfqpoint{10.217868in}{4.489236in}}%
\pgfpathcurveto{\pgfqpoint{10.217868in}{4.478186in}}{\pgfqpoint{10.222258in}{4.467587in}}{\pgfqpoint{10.230072in}{4.459773in}}%
\pgfpathcurveto{\pgfqpoint{10.237886in}{4.451960in}}{\pgfqpoint{10.248485in}{4.447570in}}{\pgfqpoint{10.259535in}{4.447570in}}%
\pgfpathlineto{\pgfqpoint{10.259535in}{4.447570in}}%
\pgfpathclose%
\pgfusepath{stroke}%
\end{pgfscope}%
\begin{pgfscope}%
\pgfpathrectangle{\pgfqpoint{7.512535in}{0.437222in}}{\pgfqpoint{6.275590in}{5.159444in}}%
\pgfusepath{clip}%
\pgfsetbuttcap%
\pgfsetroundjoin%
\pgfsetlinewidth{1.003750pt}%
\definecolor{currentstroke}{rgb}{0.827451,0.827451,0.827451}%
\pgfsetstrokecolor{currentstroke}%
\pgfsetstrokeopacity{0.800000}%
\pgfsetdash{}{0pt}%
\pgfpathmoveto{\pgfqpoint{10.704475in}{4.930530in}}%
\pgfpathcurveto{\pgfqpoint{10.715525in}{4.930530in}}{\pgfqpoint{10.726124in}{4.934921in}}{\pgfqpoint{10.733938in}{4.942734in}}%
\pgfpathcurveto{\pgfqpoint{10.741751in}{4.950548in}}{\pgfqpoint{10.746142in}{4.961147in}}{\pgfqpoint{10.746142in}{4.972197in}}%
\pgfpathcurveto{\pgfqpoint{10.746142in}{4.983247in}}{\pgfqpoint{10.741751in}{4.993846in}}{\pgfqpoint{10.733938in}{5.001660in}}%
\pgfpathcurveto{\pgfqpoint{10.726124in}{5.009473in}}{\pgfqpoint{10.715525in}{5.013864in}}{\pgfqpoint{10.704475in}{5.013864in}}%
\pgfpathcurveto{\pgfqpoint{10.693425in}{5.013864in}}{\pgfqpoint{10.682826in}{5.009473in}}{\pgfqpoint{10.675012in}{5.001660in}}%
\pgfpathcurveto{\pgfqpoint{10.667198in}{4.993846in}}{\pgfqpoint{10.662808in}{4.983247in}}{\pgfqpoint{10.662808in}{4.972197in}}%
\pgfpathcurveto{\pgfqpoint{10.662808in}{4.961147in}}{\pgfqpoint{10.667198in}{4.950548in}}{\pgfqpoint{10.675012in}{4.942734in}}%
\pgfpathcurveto{\pgfqpoint{10.682826in}{4.934921in}}{\pgfqpoint{10.693425in}{4.930530in}}{\pgfqpoint{10.704475in}{4.930530in}}%
\pgfpathlineto{\pgfqpoint{10.704475in}{4.930530in}}%
\pgfpathclose%
\pgfusepath{stroke}%
\end{pgfscope}%
\begin{pgfscope}%
\pgfpathrectangle{\pgfqpoint{7.512535in}{0.437222in}}{\pgfqpoint{6.275590in}{5.159444in}}%
\pgfusepath{clip}%
\pgfsetbuttcap%
\pgfsetroundjoin%
\pgfsetlinewidth{1.003750pt}%
\definecolor{currentstroke}{rgb}{0.827451,0.827451,0.827451}%
\pgfsetstrokecolor{currentstroke}%
\pgfsetstrokeopacity{0.800000}%
\pgfsetdash{}{0pt}%
\pgfpathmoveto{\pgfqpoint{8.277312in}{1.563305in}}%
\pgfpathcurveto{\pgfqpoint{8.288363in}{1.563305in}}{\pgfqpoint{8.298962in}{1.567695in}}{\pgfqpoint{8.306775in}{1.575509in}}%
\pgfpathcurveto{\pgfqpoint{8.314589in}{1.583322in}}{\pgfqpoint{8.318979in}{1.593921in}}{\pgfqpoint{8.318979in}{1.604971in}}%
\pgfpathcurveto{\pgfqpoint{8.318979in}{1.616021in}}{\pgfqpoint{8.314589in}{1.626621in}}{\pgfqpoint{8.306775in}{1.634434in}}%
\pgfpathcurveto{\pgfqpoint{8.298962in}{1.642248in}}{\pgfqpoint{8.288363in}{1.646638in}}{\pgfqpoint{8.277312in}{1.646638in}}%
\pgfpathcurveto{\pgfqpoint{8.266262in}{1.646638in}}{\pgfqpoint{8.255663in}{1.642248in}}{\pgfqpoint{8.247850in}{1.634434in}}%
\pgfpathcurveto{\pgfqpoint{8.240036in}{1.626621in}}{\pgfqpoint{8.235646in}{1.616021in}}{\pgfqpoint{8.235646in}{1.604971in}}%
\pgfpathcurveto{\pgfqpoint{8.235646in}{1.593921in}}{\pgfqpoint{8.240036in}{1.583322in}}{\pgfqpoint{8.247850in}{1.575509in}}%
\pgfpathcurveto{\pgfqpoint{8.255663in}{1.567695in}}{\pgfqpoint{8.266262in}{1.563305in}}{\pgfqpoint{8.277312in}{1.563305in}}%
\pgfpathlineto{\pgfqpoint{8.277312in}{1.563305in}}%
\pgfpathclose%
\pgfusepath{stroke}%
\end{pgfscope}%
\begin{pgfscope}%
\pgfpathrectangle{\pgfqpoint{7.512535in}{0.437222in}}{\pgfqpoint{6.275590in}{5.159444in}}%
\pgfusepath{clip}%
\pgfsetbuttcap%
\pgfsetroundjoin%
\pgfsetlinewidth{1.003750pt}%
\definecolor{currentstroke}{rgb}{0.827451,0.827451,0.827451}%
\pgfsetstrokecolor{currentstroke}%
\pgfsetstrokeopacity{0.800000}%
\pgfsetdash{}{0pt}%
\pgfpathmoveto{\pgfqpoint{8.624162in}{1.722485in}}%
\pgfpathcurveto{\pgfqpoint{8.635212in}{1.722485in}}{\pgfqpoint{8.645811in}{1.726876in}}{\pgfqpoint{8.653624in}{1.734689in}}%
\pgfpathcurveto{\pgfqpoint{8.661438in}{1.742503in}}{\pgfqpoint{8.665828in}{1.753102in}}{\pgfqpoint{8.665828in}{1.764152in}}%
\pgfpathcurveto{\pgfqpoint{8.665828in}{1.775202in}}{\pgfqpoint{8.661438in}{1.785801in}}{\pgfqpoint{8.653624in}{1.793615in}}%
\pgfpathcurveto{\pgfqpoint{8.645811in}{1.801428in}}{\pgfqpoint{8.635212in}{1.805819in}}{\pgfqpoint{8.624162in}{1.805819in}}%
\pgfpathcurveto{\pgfqpoint{8.613112in}{1.805819in}}{\pgfqpoint{8.602513in}{1.801428in}}{\pgfqpoint{8.594699in}{1.793615in}}%
\pgfpathcurveto{\pgfqpoint{8.586885in}{1.785801in}}{\pgfqpoint{8.582495in}{1.775202in}}{\pgfqpoint{8.582495in}{1.764152in}}%
\pgfpathcurveto{\pgfqpoint{8.582495in}{1.753102in}}{\pgfqpoint{8.586885in}{1.742503in}}{\pgfqpoint{8.594699in}{1.734689in}}%
\pgfpathcurveto{\pgfqpoint{8.602513in}{1.726876in}}{\pgfqpoint{8.613112in}{1.722485in}}{\pgfqpoint{8.624162in}{1.722485in}}%
\pgfpathlineto{\pgfqpoint{8.624162in}{1.722485in}}%
\pgfpathclose%
\pgfusepath{stroke}%
\end{pgfscope}%
\begin{pgfscope}%
\pgfpathrectangle{\pgfqpoint{7.512535in}{0.437222in}}{\pgfqpoint{6.275590in}{5.159444in}}%
\pgfusepath{clip}%
\pgfsetbuttcap%
\pgfsetroundjoin%
\pgfsetlinewidth{1.003750pt}%
\definecolor{currentstroke}{rgb}{0.827451,0.827451,0.827451}%
\pgfsetstrokecolor{currentstroke}%
\pgfsetstrokeopacity{0.800000}%
\pgfsetdash{}{0pt}%
\pgfpathmoveto{\pgfqpoint{11.387346in}{5.101603in}}%
\pgfpathcurveto{\pgfqpoint{11.398396in}{5.101603in}}{\pgfqpoint{11.408995in}{5.105993in}}{\pgfqpoint{11.416809in}{5.113807in}}%
\pgfpathcurveto{\pgfqpoint{11.424622in}{5.121620in}}{\pgfqpoint{11.429013in}{5.132219in}}{\pgfqpoint{11.429013in}{5.143270in}}%
\pgfpathcurveto{\pgfqpoint{11.429013in}{5.154320in}}{\pgfqpoint{11.424622in}{5.164919in}}{\pgfqpoint{11.416809in}{5.172732in}}%
\pgfpathcurveto{\pgfqpoint{11.408995in}{5.180546in}}{\pgfqpoint{11.398396in}{5.184936in}}{\pgfqpoint{11.387346in}{5.184936in}}%
\pgfpathcurveto{\pgfqpoint{11.376296in}{5.184936in}}{\pgfqpoint{11.365697in}{5.180546in}}{\pgfqpoint{11.357883in}{5.172732in}}%
\pgfpathcurveto{\pgfqpoint{11.350070in}{5.164919in}}{\pgfqpoint{11.345679in}{5.154320in}}{\pgfqpoint{11.345679in}{5.143270in}}%
\pgfpathcurveto{\pgfqpoint{11.345679in}{5.132219in}}{\pgfqpoint{11.350070in}{5.121620in}}{\pgfqpoint{11.357883in}{5.113807in}}%
\pgfpathcurveto{\pgfqpoint{11.365697in}{5.105993in}}{\pgfqpoint{11.376296in}{5.101603in}}{\pgfqpoint{11.387346in}{5.101603in}}%
\pgfpathlineto{\pgfqpoint{11.387346in}{5.101603in}}%
\pgfpathclose%
\pgfusepath{stroke}%
\end{pgfscope}%
\begin{pgfscope}%
\pgfpathrectangle{\pgfqpoint{7.512535in}{0.437222in}}{\pgfqpoint{6.275590in}{5.159444in}}%
\pgfusepath{clip}%
\pgfsetbuttcap%
\pgfsetroundjoin%
\pgfsetlinewidth{1.003750pt}%
\definecolor{currentstroke}{rgb}{0.827451,0.827451,0.827451}%
\pgfsetstrokecolor{currentstroke}%
\pgfsetstrokeopacity{0.800000}%
\pgfsetdash{}{0pt}%
\pgfpathmoveto{\pgfqpoint{10.385573in}{5.221855in}}%
\pgfpathcurveto{\pgfqpoint{10.396623in}{5.221855in}}{\pgfqpoint{10.407222in}{5.226245in}}{\pgfqpoint{10.415036in}{5.234058in}}%
\pgfpathcurveto{\pgfqpoint{10.422849in}{5.241872in}}{\pgfqpoint{10.427240in}{5.252471in}}{\pgfqpoint{10.427240in}{5.263521in}}%
\pgfpathcurveto{\pgfqpoint{10.427240in}{5.274571in}}{\pgfqpoint{10.422849in}{5.285170in}}{\pgfqpoint{10.415036in}{5.292984in}}%
\pgfpathcurveto{\pgfqpoint{10.407222in}{5.300798in}}{\pgfqpoint{10.396623in}{5.305188in}}{\pgfqpoint{10.385573in}{5.305188in}}%
\pgfpathcurveto{\pgfqpoint{10.374523in}{5.305188in}}{\pgfqpoint{10.363924in}{5.300798in}}{\pgfqpoint{10.356110in}{5.292984in}}%
\pgfpathcurveto{\pgfqpoint{10.348297in}{5.285170in}}{\pgfqpoint{10.343906in}{5.274571in}}{\pgfqpoint{10.343906in}{5.263521in}}%
\pgfpathcurveto{\pgfqpoint{10.343906in}{5.252471in}}{\pgfqpoint{10.348297in}{5.241872in}}{\pgfqpoint{10.356110in}{5.234058in}}%
\pgfpathcurveto{\pgfqpoint{10.363924in}{5.226245in}}{\pgfqpoint{10.374523in}{5.221855in}}{\pgfqpoint{10.385573in}{5.221855in}}%
\pgfpathlineto{\pgfqpoint{10.385573in}{5.221855in}}%
\pgfpathclose%
\pgfusepath{stroke}%
\end{pgfscope}%
\begin{pgfscope}%
\pgfpathrectangle{\pgfqpoint{7.512535in}{0.437222in}}{\pgfqpoint{6.275590in}{5.159444in}}%
\pgfusepath{clip}%
\pgfsetbuttcap%
\pgfsetroundjoin%
\pgfsetlinewidth{1.003750pt}%
\definecolor{currentstroke}{rgb}{0.827451,0.827451,0.827451}%
\pgfsetstrokecolor{currentstroke}%
\pgfsetstrokeopacity{0.800000}%
\pgfsetdash{}{0pt}%
\pgfpathmoveto{\pgfqpoint{10.873384in}{4.145170in}}%
\pgfpathcurveto{\pgfqpoint{10.884434in}{4.145170in}}{\pgfqpoint{10.895033in}{4.149560in}}{\pgfqpoint{10.902847in}{4.157374in}}%
\pgfpathcurveto{\pgfqpoint{10.910660in}{4.165188in}}{\pgfqpoint{10.915050in}{4.175787in}}{\pgfqpoint{10.915050in}{4.186837in}}%
\pgfpathcurveto{\pgfqpoint{10.915050in}{4.197887in}}{\pgfqpoint{10.910660in}{4.208486in}}{\pgfqpoint{10.902847in}{4.216300in}}%
\pgfpathcurveto{\pgfqpoint{10.895033in}{4.224113in}}{\pgfqpoint{10.884434in}{4.228504in}}{\pgfqpoint{10.873384in}{4.228504in}}%
\pgfpathcurveto{\pgfqpoint{10.862334in}{4.228504in}}{\pgfqpoint{10.851735in}{4.224113in}}{\pgfqpoint{10.843921in}{4.216300in}}%
\pgfpathcurveto{\pgfqpoint{10.836107in}{4.208486in}}{\pgfqpoint{10.831717in}{4.197887in}}{\pgfqpoint{10.831717in}{4.186837in}}%
\pgfpathcurveto{\pgfqpoint{10.831717in}{4.175787in}}{\pgfqpoint{10.836107in}{4.165188in}}{\pgfqpoint{10.843921in}{4.157374in}}%
\pgfpathcurveto{\pgfqpoint{10.851735in}{4.149560in}}{\pgfqpoint{10.862334in}{4.145170in}}{\pgfqpoint{10.873384in}{4.145170in}}%
\pgfpathlineto{\pgfqpoint{10.873384in}{4.145170in}}%
\pgfpathclose%
\pgfusepath{stroke}%
\end{pgfscope}%
\begin{pgfscope}%
\pgfpathrectangle{\pgfqpoint{7.512535in}{0.437222in}}{\pgfqpoint{6.275590in}{5.159444in}}%
\pgfusepath{clip}%
\pgfsetbuttcap%
\pgfsetroundjoin%
\pgfsetlinewidth{1.003750pt}%
\definecolor{currentstroke}{rgb}{0.827451,0.827451,0.827451}%
\pgfsetstrokecolor{currentstroke}%
\pgfsetstrokeopacity{0.800000}%
\pgfsetdash{}{0pt}%
\pgfpathmoveto{\pgfqpoint{7.992439in}{1.704977in}}%
\pgfpathcurveto{\pgfqpoint{8.003489in}{1.704977in}}{\pgfqpoint{8.014088in}{1.709367in}}{\pgfqpoint{8.021902in}{1.717181in}}%
\pgfpathcurveto{\pgfqpoint{8.029715in}{1.724994in}}{\pgfqpoint{8.034105in}{1.735593in}}{\pgfqpoint{8.034105in}{1.746643in}}%
\pgfpathcurveto{\pgfqpoint{8.034105in}{1.757694in}}{\pgfqpoint{8.029715in}{1.768293in}}{\pgfqpoint{8.021902in}{1.776106in}}%
\pgfpathcurveto{\pgfqpoint{8.014088in}{1.783920in}}{\pgfqpoint{8.003489in}{1.788310in}}{\pgfqpoint{7.992439in}{1.788310in}}%
\pgfpathcurveto{\pgfqpoint{7.981389in}{1.788310in}}{\pgfqpoint{7.970790in}{1.783920in}}{\pgfqpoint{7.962976in}{1.776106in}}%
\pgfpathcurveto{\pgfqpoint{7.955162in}{1.768293in}}{\pgfqpoint{7.950772in}{1.757694in}}{\pgfqpoint{7.950772in}{1.746643in}}%
\pgfpathcurveto{\pgfqpoint{7.950772in}{1.735593in}}{\pgfqpoint{7.955162in}{1.724994in}}{\pgfqpoint{7.962976in}{1.717181in}}%
\pgfpathcurveto{\pgfqpoint{7.970790in}{1.709367in}}{\pgfqpoint{7.981389in}{1.704977in}}{\pgfqpoint{7.992439in}{1.704977in}}%
\pgfpathlineto{\pgfqpoint{7.992439in}{1.704977in}}%
\pgfpathclose%
\pgfusepath{stroke}%
\end{pgfscope}%
\begin{pgfscope}%
\pgfpathrectangle{\pgfqpoint{7.512535in}{0.437222in}}{\pgfqpoint{6.275590in}{5.159444in}}%
\pgfusepath{clip}%
\pgfsetbuttcap%
\pgfsetroundjoin%
\pgfsetlinewidth{1.003750pt}%
\definecolor{currentstroke}{rgb}{0.827451,0.827451,0.827451}%
\pgfsetstrokecolor{currentstroke}%
\pgfsetstrokeopacity{0.800000}%
\pgfsetdash{}{0pt}%
\pgfpathmoveto{\pgfqpoint{9.630920in}{3.276112in}}%
\pgfpathcurveto{\pgfqpoint{9.641970in}{3.276112in}}{\pgfqpoint{9.652569in}{3.280502in}}{\pgfqpoint{9.660383in}{3.288316in}}%
\pgfpathcurveto{\pgfqpoint{9.668196in}{3.296130in}}{\pgfqpoint{9.672587in}{3.306729in}}{\pgfqpoint{9.672587in}{3.317779in}}%
\pgfpathcurveto{\pgfqpoint{9.672587in}{3.328829in}}{\pgfqpoint{9.668196in}{3.339428in}}{\pgfqpoint{9.660383in}{3.347241in}}%
\pgfpathcurveto{\pgfqpoint{9.652569in}{3.355055in}}{\pgfqpoint{9.641970in}{3.359445in}}{\pgfqpoint{9.630920in}{3.359445in}}%
\pgfpathcurveto{\pgfqpoint{9.619870in}{3.359445in}}{\pgfqpoint{9.609271in}{3.355055in}}{\pgfqpoint{9.601457in}{3.347241in}}%
\pgfpathcurveto{\pgfqpoint{9.593644in}{3.339428in}}{\pgfqpoint{9.589253in}{3.328829in}}{\pgfqpoint{9.589253in}{3.317779in}}%
\pgfpathcurveto{\pgfqpoint{9.589253in}{3.306729in}}{\pgfqpoint{9.593644in}{3.296130in}}{\pgfqpoint{9.601457in}{3.288316in}}%
\pgfpathcurveto{\pgfqpoint{9.609271in}{3.280502in}}{\pgfqpoint{9.619870in}{3.276112in}}{\pgfqpoint{9.630920in}{3.276112in}}%
\pgfpathlineto{\pgfqpoint{9.630920in}{3.276112in}}%
\pgfpathclose%
\pgfusepath{stroke}%
\end{pgfscope}%
\begin{pgfscope}%
\pgfpathrectangle{\pgfqpoint{7.512535in}{0.437222in}}{\pgfqpoint{6.275590in}{5.159444in}}%
\pgfusepath{clip}%
\pgfsetbuttcap%
\pgfsetroundjoin%
\pgfsetlinewidth{1.003750pt}%
\definecolor{currentstroke}{rgb}{0.827451,0.827451,0.827451}%
\pgfsetstrokecolor{currentstroke}%
\pgfsetstrokeopacity{0.800000}%
\pgfsetdash{}{0pt}%
\pgfpathmoveto{\pgfqpoint{10.045306in}{4.354371in}}%
\pgfpathcurveto{\pgfqpoint{10.056356in}{4.354371in}}{\pgfqpoint{10.066955in}{4.358761in}}{\pgfqpoint{10.074768in}{4.366575in}}%
\pgfpathcurveto{\pgfqpoint{10.082582in}{4.374389in}}{\pgfqpoint{10.086972in}{4.384988in}}{\pgfqpoint{10.086972in}{4.396038in}}%
\pgfpathcurveto{\pgfqpoint{10.086972in}{4.407088in}}{\pgfqpoint{10.082582in}{4.417687in}}{\pgfqpoint{10.074768in}{4.425500in}}%
\pgfpathcurveto{\pgfqpoint{10.066955in}{4.433314in}}{\pgfqpoint{10.056356in}{4.437704in}}{\pgfqpoint{10.045306in}{4.437704in}}%
\pgfpathcurveto{\pgfqpoint{10.034256in}{4.437704in}}{\pgfqpoint{10.023657in}{4.433314in}}{\pgfqpoint{10.015843in}{4.425500in}}%
\pgfpathcurveto{\pgfqpoint{10.008029in}{4.417687in}}{\pgfqpoint{10.003639in}{4.407088in}}{\pgfqpoint{10.003639in}{4.396038in}}%
\pgfpathcurveto{\pgfqpoint{10.003639in}{4.384988in}}{\pgfqpoint{10.008029in}{4.374389in}}{\pgfqpoint{10.015843in}{4.366575in}}%
\pgfpathcurveto{\pgfqpoint{10.023657in}{4.358761in}}{\pgfqpoint{10.034256in}{4.354371in}}{\pgfqpoint{10.045306in}{4.354371in}}%
\pgfpathlineto{\pgfqpoint{10.045306in}{4.354371in}}%
\pgfpathclose%
\pgfusepath{stroke}%
\end{pgfscope}%
\begin{pgfscope}%
\pgfpathrectangle{\pgfqpoint{7.512535in}{0.437222in}}{\pgfqpoint{6.275590in}{5.159444in}}%
\pgfusepath{clip}%
\pgfsetbuttcap%
\pgfsetroundjoin%
\pgfsetlinewidth{1.003750pt}%
\definecolor{currentstroke}{rgb}{0.827451,0.827451,0.827451}%
\pgfsetstrokecolor{currentstroke}%
\pgfsetstrokeopacity{0.800000}%
\pgfsetdash{}{0pt}%
\pgfpathmoveto{\pgfqpoint{9.862917in}{4.507322in}}%
\pgfpathcurveto{\pgfqpoint{9.873967in}{4.507322in}}{\pgfqpoint{9.884566in}{4.511712in}}{\pgfqpoint{9.892380in}{4.519525in}}%
\pgfpathcurveto{\pgfqpoint{9.900194in}{4.527339in}}{\pgfqpoint{9.904584in}{4.537938in}}{\pgfqpoint{9.904584in}{4.548988in}}%
\pgfpathcurveto{\pgfqpoint{9.904584in}{4.560038in}}{\pgfqpoint{9.900194in}{4.570637in}}{\pgfqpoint{9.892380in}{4.578451in}}%
\pgfpathcurveto{\pgfqpoint{9.884566in}{4.586265in}}{\pgfqpoint{9.873967in}{4.590655in}}{\pgfqpoint{9.862917in}{4.590655in}}%
\pgfpathcurveto{\pgfqpoint{9.851867in}{4.590655in}}{\pgfqpoint{9.841268in}{4.586265in}}{\pgfqpoint{9.833454in}{4.578451in}}%
\pgfpathcurveto{\pgfqpoint{9.825641in}{4.570637in}}{\pgfqpoint{9.821250in}{4.560038in}}{\pgfqpoint{9.821250in}{4.548988in}}%
\pgfpathcurveto{\pgfqpoint{9.821250in}{4.537938in}}{\pgfqpoint{9.825641in}{4.527339in}}{\pgfqpoint{9.833454in}{4.519525in}}%
\pgfpathcurveto{\pgfqpoint{9.841268in}{4.511712in}}{\pgfqpoint{9.851867in}{4.507322in}}{\pgfqpoint{9.862917in}{4.507322in}}%
\pgfpathlineto{\pgfqpoint{9.862917in}{4.507322in}}%
\pgfpathclose%
\pgfusepath{stroke}%
\end{pgfscope}%
\begin{pgfscope}%
\pgfpathrectangle{\pgfqpoint{7.512535in}{0.437222in}}{\pgfqpoint{6.275590in}{5.159444in}}%
\pgfusepath{clip}%
\pgfsetbuttcap%
\pgfsetroundjoin%
\pgfsetlinewidth{1.003750pt}%
\definecolor{currentstroke}{rgb}{0.827451,0.827451,0.827451}%
\pgfsetstrokecolor{currentstroke}%
\pgfsetstrokeopacity{0.800000}%
\pgfsetdash{}{0pt}%
\pgfpathmoveto{\pgfqpoint{10.413616in}{4.447570in}}%
\pgfpathcurveto{\pgfqpoint{10.424666in}{4.447570in}}{\pgfqpoint{10.435265in}{4.451960in}}{\pgfqpoint{10.443078in}{4.459773in}}%
\pgfpathcurveto{\pgfqpoint{10.450892in}{4.467587in}}{\pgfqpoint{10.455282in}{4.478186in}}{\pgfqpoint{10.455282in}{4.489236in}}%
\pgfpathcurveto{\pgfqpoint{10.455282in}{4.500286in}}{\pgfqpoint{10.450892in}{4.510885in}}{\pgfqpoint{10.443078in}{4.518699in}}%
\pgfpathcurveto{\pgfqpoint{10.435265in}{4.526513in}}{\pgfqpoint{10.424666in}{4.530903in}}{\pgfqpoint{10.413616in}{4.530903in}}%
\pgfpathcurveto{\pgfqpoint{10.402565in}{4.530903in}}{\pgfqpoint{10.391966in}{4.526513in}}{\pgfqpoint{10.384153in}{4.518699in}}%
\pgfpathcurveto{\pgfqpoint{10.376339in}{4.510885in}}{\pgfqpoint{10.371949in}{4.500286in}}{\pgfqpoint{10.371949in}{4.489236in}}%
\pgfpathcurveto{\pgfqpoint{10.371949in}{4.478186in}}{\pgfqpoint{10.376339in}{4.467587in}}{\pgfqpoint{10.384153in}{4.459773in}}%
\pgfpathcurveto{\pgfqpoint{10.391966in}{4.451960in}}{\pgfqpoint{10.402565in}{4.447570in}}{\pgfqpoint{10.413616in}{4.447570in}}%
\pgfpathlineto{\pgfqpoint{10.413616in}{4.447570in}}%
\pgfpathclose%
\pgfusepath{stroke}%
\end{pgfscope}%
\begin{pgfscope}%
\pgfpathrectangle{\pgfqpoint{7.512535in}{0.437222in}}{\pgfqpoint{6.275590in}{5.159444in}}%
\pgfusepath{clip}%
\pgfsetbuttcap%
\pgfsetroundjoin%
\pgfsetlinewidth{1.003750pt}%
\definecolor{currentstroke}{rgb}{0.827451,0.827451,0.827451}%
\pgfsetstrokecolor{currentstroke}%
\pgfsetstrokeopacity{0.800000}%
\pgfsetdash{}{0pt}%
\pgfpathmoveto{\pgfqpoint{11.443872in}{5.184186in}}%
\pgfpathcurveto{\pgfqpoint{11.454922in}{5.184186in}}{\pgfqpoint{11.465521in}{5.188576in}}{\pgfqpoint{11.473335in}{5.196390in}}%
\pgfpathcurveto{\pgfqpoint{11.481148in}{5.204203in}}{\pgfqpoint{11.485539in}{5.214802in}}{\pgfqpoint{11.485539in}{5.225852in}}%
\pgfpathcurveto{\pgfqpoint{11.485539in}{5.236903in}}{\pgfqpoint{11.481148in}{5.247502in}}{\pgfqpoint{11.473335in}{5.255315in}}%
\pgfpathcurveto{\pgfqpoint{11.465521in}{5.263129in}}{\pgfqpoint{11.454922in}{5.267519in}}{\pgfqpoint{11.443872in}{5.267519in}}%
\pgfpathcurveto{\pgfqpoint{11.432822in}{5.267519in}}{\pgfqpoint{11.422223in}{5.263129in}}{\pgfqpoint{11.414409in}{5.255315in}}%
\pgfpathcurveto{\pgfqpoint{11.406596in}{5.247502in}}{\pgfqpoint{11.402205in}{5.236903in}}{\pgfqpoint{11.402205in}{5.225852in}}%
\pgfpathcurveto{\pgfqpoint{11.402205in}{5.214802in}}{\pgfqpoint{11.406596in}{5.204203in}}{\pgfqpoint{11.414409in}{5.196390in}}%
\pgfpathcurveto{\pgfqpoint{11.422223in}{5.188576in}}{\pgfqpoint{11.432822in}{5.184186in}}{\pgfqpoint{11.443872in}{5.184186in}}%
\pgfpathlineto{\pgfqpoint{11.443872in}{5.184186in}}%
\pgfpathclose%
\pgfusepath{stroke}%
\end{pgfscope}%
\begin{pgfscope}%
\pgfpathrectangle{\pgfqpoint{7.512535in}{0.437222in}}{\pgfqpoint{6.275590in}{5.159444in}}%
\pgfusepath{clip}%
\pgfsetbuttcap%
\pgfsetroundjoin%
\pgfsetlinewidth{1.003750pt}%
\definecolor{currentstroke}{rgb}{0.827451,0.827451,0.827451}%
\pgfsetstrokecolor{currentstroke}%
\pgfsetstrokeopacity{0.800000}%
\pgfsetdash{}{0pt}%
\pgfpathmoveto{\pgfqpoint{10.470620in}{3.835584in}}%
\pgfpathcurveto{\pgfqpoint{10.481671in}{3.835584in}}{\pgfqpoint{10.492270in}{3.839975in}}{\pgfqpoint{10.500083in}{3.847788in}}%
\pgfpathcurveto{\pgfqpoint{10.507897in}{3.855602in}}{\pgfqpoint{10.512287in}{3.866201in}}{\pgfqpoint{10.512287in}{3.877251in}}%
\pgfpathcurveto{\pgfqpoint{10.512287in}{3.888301in}}{\pgfqpoint{10.507897in}{3.898900in}}{\pgfqpoint{10.500083in}{3.906714in}}%
\pgfpathcurveto{\pgfqpoint{10.492270in}{3.914528in}}{\pgfqpoint{10.481671in}{3.918918in}}{\pgfqpoint{10.470620in}{3.918918in}}%
\pgfpathcurveto{\pgfqpoint{10.459570in}{3.918918in}}{\pgfqpoint{10.448971in}{3.914528in}}{\pgfqpoint{10.441158in}{3.906714in}}%
\pgfpathcurveto{\pgfqpoint{10.433344in}{3.898900in}}{\pgfqpoint{10.428954in}{3.888301in}}{\pgfqpoint{10.428954in}{3.877251in}}%
\pgfpathcurveto{\pgfqpoint{10.428954in}{3.866201in}}{\pgfqpoint{10.433344in}{3.855602in}}{\pgfqpoint{10.441158in}{3.847788in}}%
\pgfpathcurveto{\pgfqpoint{10.448971in}{3.839975in}}{\pgfqpoint{10.459570in}{3.835584in}}{\pgfqpoint{10.470620in}{3.835584in}}%
\pgfpathlineto{\pgfqpoint{10.470620in}{3.835584in}}%
\pgfpathclose%
\pgfusepath{stroke}%
\end{pgfscope}%
\begin{pgfscope}%
\pgfpathrectangle{\pgfqpoint{7.512535in}{0.437222in}}{\pgfqpoint{6.275590in}{5.159444in}}%
\pgfusepath{clip}%
\pgfsetbuttcap%
\pgfsetroundjoin%
\pgfsetlinewidth{1.003750pt}%
\definecolor{currentstroke}{rgb}{0.827451,0.827451,0.827451}%
\pgfsetstrokecolor{currentstroke}%
\pgfsetstrokeopacity{0.800000}%
\pgfsetdash{}{0pt}%
\pgfpathmoveto{\pgfqpoint{10.698510in}{3.970272in}}%
\pgfpathcurveto{\pgfqpoint{10.709560in}{3.970272in}}{\pgfqpoint{10.720159in}{3.974662in}}{\pgfqpoint{10.727973in}{3.982476in}}%
\pgfpathcurveto{\pgfqpoint{10.735787in}{3.990290in}}{\pgfqpoint{10.740177in}{4.000889in}}{\pgfqpoint{10.740177in}{4.011939in}}%
\pgfpathcurveto{\pgfqpoint{10.740177in}{4.022989in}}{\pgfqpoint{10.735787in}{4.033588in}}{\pgfqpoint{10.727973in}{4.041402in}}%
\pgfpathcurveto{\pgfqpoint{10.720159in}{4.049215in}}{\pgfqpoint{10.709560in}{4.053605in}}{\pgfqpoint{10.698510in}{4.053605in}}%
\pgfpathcurveto{\pgfqpoint{10.687460in}{4.053605in}}{\pgfqpoint{10.676861in}{4.049215in}}{\pgfqpoint{10.669047in}{4.041402in}}%
\pgfpathcurveto{\pgfqpoint{10.661234in}{4.033588in}}{\pgfqpoint{10.656844in}{4.022989in}}{\pgfqpoint{10.656844in}{4.011939in}}%
\pgfpathcurveto{\pgfqpoint{10.656844in}{4.000889in}}{\pgfqpoint{10.661234in}{3.990290in}}{\pgfqpoint{10.669047in}{3.982476in}}%
\pgfpathcurveto{\pgfqpoint{10.676861in}{3.974662in}}{\pgfqpoint{10.687460in}{3.970272in}}{\pgfqpoint{10.698510in}{3.970272in}}%
\pgfpathlineto{\pgfqpoint{10.698510in}{3.970272in}}%
\pgfpathclose%
\pgfusepath{stroke}%
\end{pgfscope}%
\begin{pgfscope}%
\pgfpathrectangle{\pgfqpoint{7.512535in}{0.437222in}}{\pgfqpoint{6.275590in}{5.159444in}}%
\pgfusepath{clip}%
\pgfsetbuttcap%
\pgfsetroundjoin%
\pgfsetlinewidth{1.003750pt}%
\definecolor{currentstroke}{rgb}{0.827451,0.827451,0.827451}%
\pgfsetstrokecolor{currentstroke}%
\pgfsetstrokeopacity{0.800000}%
\pgfsetdash{}{0pt}%
\pgfpathmoveto{\pgfqpoint{9.581399in}{3.231111in}}%
\pgfpathcurveto{\pgfqpoint{9.592449in}{3.231111in}}{\pgfqpoint{9.603048in}{3.235501in}}{\pgfqpoint{9.610862in}{3.243315in}}%
\pgfpathcurveto{\pgfqpoint{9.618675in}{3.251128in}}{\pgfqpoint{9.623065in}{3.261727in}}{\pgfqpoint{9.623065in}{3.272778in}}%
\pgfpathcurveto{\pgfqpoint{9.623065in}{3.283828in}}{\pgfqpoint{9.618675in}{3.294427in}}{\pgfqpoint{9.610862in}{3.302240in}}%
\pgfpathcurveto{\pgfqpoint{9.603048in}{3.310054in}}{\pgfqpoint{9.592449in}{3.314444in}}{\pgfqpoint{9.581399in}{3.314444in}}%
\pgfpathcurveto{\pgfqpoint{9.570349in}{3.314444in}}{\pgfqpoint{9.559750in}{3.310054in}}{\pgfqpoint{9.551936in}{3.302240in}}%
\pgfpathcurveto{\pgfqpoint{9.544122in}{3.294427in}}{\pgfqpoint{9.539732in}{3.283828in}}{\pgfqpoint{9.539732in}{3.272778in}}%
\pgfpathcurveto{\pgfqpoint{9.539732in}{3.261727in}}{\pgfqpoint{9.544122in}{3.251128in}}{\pgfqpoint{9.551936in}{3.243315in}}%
\pgfpathcurveto{\pgfqpoint{9.559750in}{3.235501in}}{\pgfqpoint{9.570349in}{3.231111in}}{\pgfqpoint{9.581399in}{3.231111in}}%
\pgfpathlineto{\pgfqpoint{9.581399in}{3.231111in}}%
\pgfpathclose%
\pgfusepath{stroke}%
\end{pgfscope}%
\begin{pgfscope}%
\pgfpathrectangle{\pgfqpoint{7.512535in}{0.437222in}}{\pgfqpoint{6.275590in}{5.159444in}}%
\pgfusepath{clip}%
\pgfsetbuttcap%
\pgfsetroundjoin%
\pgfsetlinewidth{1.003750pt}%
\definecolor{currentstroke}{rgb}{0.827451,0.827451,0.827451}%
\pgfsetstrokecolor{currentstroke}%
\pgfsetstrokeopacity{0.800000}%
\pgfsetdash{}{0pt}%
\pgfpathmoveto{\pgfqpoint{12.068300in}{5.237437in}}%
\pgfpathcurveto{\pgfqpoint{12.079350in}{5.237437in}}{\pgfqpoint{12.089949in}{5.241827in}}{\pgfqpoint{12.097763in}{5.249641in}}%
\pgfpathcurveto{\pgfqpoint{12.105577in}{5.257454in}}{\pgfqpoint{12.109967in}{5.268053in}}{\pgfqpoint{12.109967in}{5.279104in}}%
\pgfpathcurveto{\pgfqpoint{12.109967in}{5.290154in}}{\pgfqpoint{12.105577in}{5.300753in}}{\pgfqpoint{12.097763in}{5.308566in}}%
\pgfpathcurveto{\pgfqpoint{12.089949in}{5.316380in}}{\pgfqpoint{12.079350in}{5.320770in}}{\pgfqpoint{12.068300in}{5.320770in}}%
\pgfpathcurveto{\pgfqpoint{12.057250in}{5.320770in}}{\pgfqpoint{12.046651in}{5.316380in}}{\pgfqpoint{12.038838in}{5.308566in}}%
\pgfpathcurveto{\pgfqpoint{12.031024in}{5.300753in}}{\pgfqpoint{12.026634in}{5.290154in}}{\pgfqpoint{12.026634in}{5.279104in}}%
\pgfpathcurveto{\pgfqpoint{12.026634in}{5.268053in}}{\pgfqpoint{12.031024in}{5.257454in}}{\pgfqpoint{12.038838in}{5.249641in}}%
\pgfpathcurveto{\pgfqpoint{12.046651in}{5.241827in}}{\pgfqpoint{12.057250in}{5.237437in}}{\pgfqpoint{12.068300in}{5.237437in}}%
\pgfpathlineto{\pgfqpoint{12.068300in}{5.237437in}}%
\pgfpathclose%
\pgfusepath{stroke}%
\end{pgfscope}%
\begin{pgfscope}%
\pgfpathrectangle{\pgfqpoint{7.512535in}{0.437222in}}{\pgfqpoint{6.275590in}{5.159444in}}%
\pgfusepath{clip}%
\pgfsetbuttcap%
\pgfsetroundjoin%
\pgfsetlinewidth{1.003750pt}%
\definecolor{currentstroke}{rgb}{0.827451,0.827451,0.827451}%
\pgfsetstrokecolor{currentstroke}%
\pgfsetstrokeopacity{0.800000}%
\pgfsetdash{}{0pt}%
\pgfpathmoveto{\pgfqpoint{9.501942in}{3.134670in}}%
\pgfpathcurveto{\pgfqpoint{9.512992in}{3.134670in}}{\pgfqpoint{9.523591in}{3.139061in}}{\pgfqpoint{9.531405in}{3.146874in}}%
\pgfpathcurveto{\pgfqpoint{9.539218in}{3.154688in}}{\pgfqpoint{9.543609in}{3.165287in}}{\pgfqpoint{9.543609in}{3.176337in}}%
\pgfpathcurveto{\pgfqpoint{9.543609in}{3.187387in}}{\pgfqpoint{9.539218in}{3.197986in}}{\pgfqpoint{9.531405in}{3.205800in}}%
\pgfpathcurveto{\pgfqpoint{9.523591in}{3.213613in}}{\pgfqpoint{9.512992in}{3.218004in}}{\pgfqpoint{9.501942in}{3.218004in}}%
\pgfpathcurveto{\pgfqpoint{9.490892in}{3.218004in}}{\pgfqpoint{9.480293in}{3.213613in}}{\pgfqpoint{9.472479in}{3.205800in}}%
\pgfpathcurveto{\pgfqpoint{9.464666in}{3.197986in}}{\pgfqpoint{9.460275in}{3.187387in}}{\pgfqpoint{9.460275in}{3.176337in}}%
\pgfpathcurveto{\pgfqpoint{9.460275in}{3.165287in}}{\pgfqpoint{9.464666in}{3.154688in}}{\pgfqpoint{9.472479in}{3.146874in}}%
\pgfpathcurveto{\pgfqpoint{9.480293in}{3.139061in}}{\pgfqpoint{9.490892in}{3.134670in}}{\pgfqpoint{9.501942in}{3.134670in}}%
\pgfpathlineto{\pgfqpoint{9.501942in}{3.134670in}}%
\pgfpathclose%
\pgfusepath{stroke}%
\end{pgfscope}%
\begin{pgfscope}%
\pgfpathrectangle{\pgfqpoint{7.512535in}{0.437222in}}{\pgfqpoint{6.275590in}{5.159444in}}%
\pgfusepath{clip}%
\pgfsetbuttcap%
\pgfsetroundjoin%
\pgfsetlinewidth{1.003750pt}%
\definecolor{currentstroke}{rgb}{0.827451,0.827451,0.827451}%
\pgfsetstrokecolor{currentstroke}%
\pgfsetstrokeopacity{0.800000}%
\pgfsetdash{}{0pt}%
\pgfpathmoveto{\pgfqpoint{12.081725in}{5.113368in}}%
\pgfpathcurveto{\pgfqpoint{12.092776in}{5.113368in}}{\pgfqpoint{12.103375in}{5.117758in}}{\pgfqpoint{12.111188in}{5.125572in}}%
\pgfpathcurveto{\pgfqpoint{12.119002in}{5.133385in}}{\pgfqpoint{12.123392in}{5.143984in}}{\pgfqpoint{12.123392in}{5.155034in}}%
\pgfpathcurveto{\pgfqpoint{12.123392in}{5.166084in}}{\pgfqpoint{12.119002in}{5.176683in}}{\pgfqpoint{12.111188in}{5.184497in}}%
\pgfpathcurveto{\pgfqpoint{12.103375in}{5.192311in}}{\pgfqpoint{12.092776in}{5.196701in}}{\pgfqpoint{12.081725in}{5.196701in}}%
\pgfpathcurveto{\pgfqpoint{12.070675in}{5.196701in}}{\pgfqpoint{12.060076in}{5.192311in}}{\pgfqpoint{12.052263in}{5.184497in}}%
\pgfpathcurveto{\pgfqpoint{12.044449in}{5.176683in}}{\pgfqpoint{12.040059in}{5.166084in}}{\pgfqpoint{12.040059in}{5.155034in}}%
\pgfpathcurveto{\pgfqpoint{12.040059in}{5.143984in}}{\pgfqpoint{12.044449in}{5.133385in}}{\pgfqpoint{12.052263in}{5.125572in}}%
\pgfpathcurveto{\pgfqpoint{12.060076in}{5.117758in}}{\pgfqpoint{12.070675in}{5.113368in}}{\pgfqpoint{12.081725in}{5.113368in}}%
\pgfpathlineto{\pgfqpoint{12.081725in}{5.113368in}}%
\pgfpathclose%
\pgfusepath{stroke}%
\end{pgfscope}%
\begin{pgfscope}%
\pgfpathrectangle{\pgfqpoint{7.512535in}{0.437222in}}{\pgfqpoint{6.275590in}{5.159444in}}%
\pgfusepath{clip}%
\pgfsetbuttcap%
\pgfsetroundjoin%
\pgfsetlinewidth{1.003750pt}%
\definecolor{currentstroke}{rgb}{0.827451,0.827451,0.827451}%
\pgfsetstrokecolor{currentstroke}%
\pgfsetstrokeopacity{0.800000}%
\pgfsetdash{}{0pt}%
\pgfpathmoveto{\pgfqpoint{11.528788in}{5.101471in}}%
\pgfpathcurveto{\pgfqpoint{11.539838in}{5.101471in}}{\pgfqpoint{11.550437in}{5.105861in}}{\pgfqpoint{11.558251in}{5.113674in}}%
\pgfpathcurveto{\pgfqpoint{11.566064in}{5.121488in}}{\pgfqpoint{11.570455in}{5.132087in}}{\pgfqpoint{11.570455in}{5.143137in}}%
\pgfpathcurveto{\pgfqpoint{11.570455in}{5.154187in}}{\pgfqpoint{11.566064in}{5.164786in}}{\pgfqpoint{11.558251in}{5.172600in}}%
\pgfpathcurveto{\pgfqpoint{11.550437in}{5.180414in}}{\pgfqpoint{11.539838in}{5.184804in}}{\pgfqpoint{11.528788in}{5.184804in}}%
\pgfpathcurveto{\pgfqpoint{11.517738in}{5.184804in}}{\pgfqpoint{11.507139in}{5.180414in}}{\pgfqpoint{11.499325in}{5.172600in}}%
\pgfpathcurveto{\pgfqpoint{11.491512in}{5.164786in}}{\pgfqpoint{11.487121in}{5.154187in}}{\pgfqpoint{11.487121in}{5.143137in}}%
\pgfpathcurveto{\pgfqpoint{11.487121in}{5.132087in}}{\pgfqpoint{11.491512in}{5.121488in}}{\pgfqpoint{11.499325in}{5.113674in}}%
\pgfpathcurveto{\pgfqpoint{11.507139in}{5.105861in}}{\pgfqpoint{11.517738in}{5.101471in}}{\pgfqpoint{11.528788in}{5.101471in}}%
\pgfpathlineto{\pgfqpoint{11.528788in}{5.101471in}}%
\pgfpathclose%
\pgfusepath{stroke}%
\end{pgfscope}%
\begin{pgfscope}%
\pgfpathrectangle{\pgfqpoint{7.512535in}{0.437222in}}{\pgfqpoint{6.275590in}{5.159444in}}%
\pgfusepath{clip}%
\pgfsetbuttcap%
\pgfsetroundjoin%
\pgfsetlinewidth{1.003750pt}%
\definecolor{currentstroke}{rgb}{0.827451,0.827451,0.827451}%
\pgfsetstrokecolor{currentstroke}%
\pgfsetstrokeopacity{0.800000}%
\pgfsetdash{}{0pt}%
\pgfpathmoveto{\pgfqpoint{10.881762in}{4.622183in}}%
\pgfpathcurveto{\pgfqpoint{10.892812in}{4.622183in}}{\pgfqpoint{10.903411in}{4.626573in}}{\pgfqpoint{10.911225in}{4.634387in}}%
\pgfpathcurveto{\pgfqpoint{10.919039in}{4.642201in}}{\pgfqpoint{10.923429in}{4.652800in}}{\pgfqpoint{10.923429in}{4.663850in}}%
\pgfpathcurveto{\pgfqpoint{10.923429in}{4.674900in}}{\pgfqpoint{10.919039in}{4.685499in}}{\pgfqpoint{10.911225in}{4.693313in}}%
\pgfpathcurveto{\pgfqpoint{10.903411in}{4.701126in}}{\pgfqpoint{10.892812in}{4.705516in}}{\pgfqpoint{10.881762in}{4.705516in}}%
\pgfpathcurveto{\pgfqpoint{10.870712in}{4.705516in}}{\pgfqpoint{10.860113in}{4.701126in}}{\pgfqpoint{10.852300in}{4.693313in}}%
\pgfpathcurveto{\pgfqpoint{10.844486in}{4.685499in}}{\pgfqpoint{10.840096in}{4.674900in}}{\pgfqpoint{10.840096in}{4.663850in}}%
\pgfpathcurveto{\pgfqpoint{10.840096in}{4.652800in}}{\pgfqpoint{10.844486in}{4.642201in}}{\pgfqpoint{10.852300in}{4.634387in}}%
\pgfpathcurveto{\pgfqpoint{10.860113in}{4.626573in}}{\pgfqpoint{10.870712in}{4.622183in}}{\pgfqpoint{10.881762in}{4.622183in}}%
\pgfpathlineto{\pgfqpoint{10.881762in}{4.622183in}}%
\pgfpathclose%
\pgfusepath{stroke}%
\end{pgfscope}%
\begin{pgfscope}%
\pgfpathrectangle{\pgfqpoint{7.512535in}{0.437222in}}{\pgfqpoint{6.275590in}{5.159444in}}%
\pgfusepath{clip}%
\pgfsetbuttcap%
\pgfsetroundjoin%
\pgfsetlinewidth{1.003750pt}%
\definecolor{currentstroke}{rgb}{0.827451,0.827451,0.827451}%
\pgfsetstrokecolor{currentstroke}%
\pgfsetstrokeopacity{0.800000}%
\pgfsetdash{}{0pt}%
\pgfpathmoveto{\pgfqpoint{11.634893in}{5.376413in}}%
\pgfpathcurveto{\pgfqpoint{11.645943in}{5.376413in}}{\pgfqpoint{11.656542in}{5.380803in}}{\pgfqpoint{11.664356in}{5.388617in}}%
\pgfpathcurveto{\pgfqpoint{11.672170in}{5.396430in}}{\pgfqpoint{11.676560in}{5.407029in}}{\pgfqpoint{11.676560in}{5.418079in}}%
\pgfpathcurveto{\pgfqpoint{11.676560in}{5.429129in}}{\pgfqpoint{11.672170in}{5.439728in}}{\pgfqpoint{11.664356in}{5.447542in}}%
\pgfpathcurveto{\pgfqpoint{11.656542in}{5.455356in}}{\pgfqpoint{11.645943in}{5.459746in}}{\pgfqpoint{11.634893in}{5.459746in}}%
\pgfpathcurveto{\pgfqpoint{11.623843in}{5.459746in}}{\pgfqpoint{11.613244in}{5.455356in}}{\pgfqpoint{11.605430in}{5.447542in}}%
\pgfpathcurveto{\pgfqpoint{11.597617in}{5.439728in}}{\pgfqpoint{11.593227in}{5.429129in}}{\pgfqpoint{11.593227in}{5.418079in}}%
\pgfpathcurveto{\pgfqpoint{11.593227in}{5.407029in}}{\pgfqpoint{11.597617in}{5.396430in}}{\pgfqpoint{11.605430in}{5.388617in}}%
\pgfpathcurveto{\pgfqpoint{11.613244in}{5.380803in}}{\pgfqpoint{11.623843in}{5.376413in}}{\pgfqpoint{11.634893in}{5.376413in}}%
\pgfpathlineto{\pgfqpoint{11.634893in}{5.376413in}}%
\pgfpathclose%
\pgfusepath{stroke}%
\end{pgfscope}%
\begin{pgfscope}%
\pgfpathrectangle{\pgfqpoint{7.512535in}{0.437222in}}{\pgfqpoint{6.275590in}{5.159444in}}%
\pgfusepath{clip}%
\pgfsetbuttcap%
\pgfsetroundjoin%
\pgfsetlinewidth{1.003750pt}%
\definecolor{currentstroke}{rgb}{0.827451,0.827451,0.827451}%
\pgfsetstrokecolor{currentstroke}%
\pgfsetstrokeopacity{0.800000}%
\pgfsetdash{}{0pt}%
\pgfpathmoveto{\pgfqpoint{9.565514in}{3.231111in}}%
\pgfpathcurveto{\pgfqpoint{9.576565in}{3.231111in}}{\pgfqpoint{9.587164in}{3.235501in}}{\pgfqpoint{9.594977in}{3.243315in}}%
\pgfpathcurveto{\pgfqpoint{9.602791in}{3.251128in}}{\pgfqpoint{9.607181in}{3.261727in}}{\pgfqpoint{9.607181in}{3.272778in}}%
\pgfpathcurveto{\pgfqpoint{9.607181in}{3.283828in}}{\pgfqpoint{9.602791in}{3.294427in}}{\pgfqpoint{9.594977in}{3.302240in}}%
\pgfpathcurveto{\pgfqpoint{9.587164in}{3.310054in}}{\pgfqpoint{9.576565in}{3.314444in}}{\pgfqpoint{9.565514in}{3.314444in}}%
\pgfpathcurveto{\pgfqpoint{9.554464in}{3.314444in}}{\pgfqpoint{9.543865in}{3.310054in}}{\pgfqpoint{9.536052in}{3.302240in}}%
\pgfpathcurveto{\pgfqpoint{9.528238in}{3.294427in}}{\pgfqpoint{9.523848in}{3.283828in}}{\pgfqpoint{9.523848in}{3.272778in}}%
\pgfpathcurveto{\pgfqpoint{9.523848in}{3.261727in}}{\pgfqpoint{9.528238in}{3.251128in}}{\pgfqpoint{9.536052in}{3.243315in}}%
\pgfpathcurveto{\pgfqpoint{9.543865in}{3.235501in}}{\pgfqpoint{9.554464in}{3.231111in}}{\pgfqpoint{9.565514in}{3.231111in}}%
\pgfpathlineto{\pgfqpoint{9.565514in}{3.231111in}}%
\pgfpathclose%
\pgfusepath{stroke}%
\end{pgfscope}%
\begin{pgfscope}%
\pgfpathrectangle{\pgfqpoint{7.512535in}{0.437222in}}{\pgfqpoint{6.275590in}{5.159444in}}%
\pgfusepath{clip}%
\pgfsetbuttcap%
\pgfsetroundjoin%
\pgfsetlinewidth{1.003750pt}%
\definecolor{currentstroke}{rgb}{0.827451,0.827451,0.827451}%
\pgfsetstrokecolor{currentstroke}%
\pgfsetstrokeopacity{0.800000}%
\pgfsetdash{}{0pt}%
\pgfpathmoveto{\pgfqpoint{10.259535in}{4.973807in}}%
\pgfpathcurveto{\pgfqpoint{10.270585in}{4.973807in}}{\pgfqpoint{10.281184in}{4.978197in}}{\pgfqpoint{10.288997in}{4.986011in}}%
\pgfpathcurveto{\pgfqpoint{10.296811in}{4.993824in}}{\pgfqpoint{10.301201in}{5.004423in}}{\pgfqpoint{10.301201in}{5.015473in}}%
\pgfpathcurveto{\pgfqpoint{10.301201in}{5.026523in}}{\pgfqpoint{10.296811in}{5.037123in}}{\pgfqpoint{10.288997in}{5.044936in}}%
\pgfpathcurveto{\pgfqpoint{10.281184in}{5.052750in}}{\pgfqpoint{10.270585in}{5.057140in}}{\pgfqpoint{10.259535in}{5.057140in}}%
\pgfpathcurveto{\pgfqpoint{10.248485in}{5.057140in}}{\pgfqpoint{10.237886in}{5.052750in}}{\pgfqpoint{10.230072in}{5.044936in}}%
\pgfpathcurveto{\pgfqpoint{10.222258in}{5.037123in}}{\pgfqpoint{10.217868in}{5.026523in}}{\pgfqpoint{10.217868in}{5.015473in}}%
\pgfpathcurveto{\pgfqpoint{10.217868in}{5.004423in}}{\pgfqpoint{10.222258in}{4.993824in}}{\pgfqpoint{10.230072in}{4.986011in}}%
\pgfpathcurveto{\pgfqpoint{10.237886in}{4.978197in}}{\pgfqpoint{10.248485in}{4.973807in}}{\pgfqpoint{10.259535in}{4.973807in}}%
\pgfpathlineto{\pgfqpoint{10.259535in}{4.973807in}}%
\pgfpathclose%
\pgfusepath{stroke}%
\end{pgfscope}%
\begin{pgfscope}%
\pgfpathrectangle{\pgfqpoint{7.512535in}{0.437222in}}{\pgfqpoint{6.275590in}{5.159444in}}%
\pgfusepath{clip}%
\pgfsetbuttcap%
\pgfsetroundjoin%
\pgfsetlinewidth{1.003750pt}%
\definecolor{currentstroke}{rgb}{0.827451,0.827451,0.827451}%
\pgfsetstrokecolor{currentstroke}%
\pgfsetstrokeopacity{0.800000}%
\pgfsetdash{}{0pt}%
\pgfpathmoveto{\pgfqpoint{9.697299in}{3.449334in}}%
\pgfpathcurveto{\pgfqpoint{9.708349in}{3.449334in}}{\pgfqpoint{9.718948in}{3.453725in}}{\pgfqpoint{9.726762in}{3.461538in}}%
\pgfpathcurveto{\pgfqpoint{9.734575in}{3.469352in}}{\pgfqpoint{9.738965in}{3.479951in}}{\pgfqpoint{9.738965in}{3.491001in}}%
\pgfpathcurveto{\pgfqpoint{9.738965in}{3.502051in}}{\pgfqpoint{9.734575in}{3.512650in}}{\pgfqpoint{9.726762in}{3.520464in}}%
\pgfpathcurveto{\pgfqpoint{9.718948in}{3.528277in}}{\pgfqpoint{9.708349in}{3.532668in}}{\pgfqpoint{9.697299in}{3.532668in}}%
\pgfpathcurveto{\pgfqpoint{9.686249in}{3.532668in}}{\pgfqpoint{9.675650in}{3.528277in}}{\pgfqpoint{9.667836in}{3.520464in}}%
\pgfpathcurveto{\pgfqpoint{9.660022in}{3.512650in}}{\pgfqpoint{9.655632in}{3.502051in}}{\pgfqpoint{9.655632in}{3.491001in}}%
\pgfpathcurveto{\pgfqpoint{9.655632in}{3.479951in}}{\pgfqpoint{9.660022in}{3.469352in}}{\pgfqpoint{9.667836in}{3.461538in}}%
\pgfpathcurveto{\pgfqpoint{9.675650in}{3.453725in}}{\pgfqpoint{9.686249in}{3.449334in}}{\pgfqpoint{9.697299in}{3.449334in}}%
\pgfpathlineto{\pgfqpoint{9.697299in}{3.449334in}}%
\pgfpathclose%
\pgfusepath{stroke}%
\end{pgfscope}%
\begin{pgfscope}%
\pgfpathrectangle{\pgfqpoint{7.512535in}{0.437222in}}{\pgfqpoint{6.275590in}{5.159444in}}%
\pgfusepath{clip}%
\pgfsetbuttcap%
\pgfsetroundjoin%
\pgfsetlinewidth{1.003750pt}%
\definecolor{currentstroke}{rgb}{0.827451,0.827451,0.827451}%
\pgfsetstrokecolor{currentstroke}%
\pgfsetstrokeopacity{0.800000}%
\pgfsetdash{}{0pt}%
\pgfpathmoveto{\pgfqpoint{12.081489in}{5.112444in}}%
\pgfpathcurveto{\pgfqpoint{12.092539in}{5.112444in}}{\pgfqpoint{12.103138in}{5.116834in}}{\pgfqpoint{12.110952in}{5.124648in}}%
\pgfpathcurveto{\pgfqpoint{12.118765in}{5.132461in}}{\pgfqpoint{12.123156in}{5.143061in}}{\pgfqpoint{12.123156in}{5.154111in}}%
\pgfpathcurveto{\pgfqpoint{12.123156in}{5.165161in}}{\pgfqpoint{12.118765in}{5.175760in}}{\pgfqpoint{12.110952in}{5.183573in}}%
\pgfpathcurveto{\pgfqpoint{12.103138in}{5.191387in}}{\pgfqpoint{12.092539in}{5.195777in}}{\pgfqpoint{12.081489in}{5.195777in}}%
\pgfpathcurveto{\pgfqpoint{12.070439in}{5.195777in}}{\pgfqpoint{12.059840in}{5.191387in}}{\pgfqpoint{12.052026in}{5.183573in}}%
\pgfpathcurveto{\pgfqpoint{12.044212in}{5.175760in}}{\pgfqpoint{12.039822in}{5.165161in}}{\pgfqpoint{12.039822in}{5.154111in}}%
\pgfpathcurveto{\pgfqpoint{12.039822in}{5.143061in}}{\pgfqpoint{12.044212in}{5.132461in}}{\pgfqpoint{12.052026in}{5.124648in}}%
\pgfpathcurveto{\pgfqpoint{12.059840in}{5.116834in}}{\pgfqpoint{12.070439in}{5.112444in}}{\pgfqpoint{12.081489in}{5.112444in}}%
\pgfpathlineto{\pgfqpoint{12.081489in}{5.112444in}}%
\pgfpathclose%
\pgfusepath{stroke}%
\end{pgfscope}%
\begin{pgfscope}%
\pgfpathrectangle{\pgfqpoint{7.512535in}{0.437222in}}{\pgfqpoint{6.275590in}{5.159444in}}%
\pgfusepath{clip}%
\pgfsetbuttcap%
\pgfsetroundjoin%
\pgfsetlinewidth{1.003750pt}%
\definecolor{currentstroke}{rgb}{0.827451,0.827451,0.827451}%
\pgfsetstrokecolor{currentstroke}%
\pgfsetstrokeopacity{0.800000}%
\pgfsetdash{}{0pt}%
\pgfpathmoveto{\pgfqpoint{7.679894in}{0.493855in}}%
\pgfpathcurveto{\pgfqpoint{7.690944in}{0.493855in}}{\pgfqpoint{7.701543in}{0.498245in}}{\pgfqpoint{7.709357in}{0.506059in}}%
\pgfpathcurveto{\pgfqpoint{7.717170in}{0.513872in}}{\pgfqpoint{7.721560in}{0.524471in}}{\pgfqpoint{7.721560in}{0.535522in}}%
\pgfpathcurveto{\pgfqpoint{7.721560in}{0.546572in}}{\pgfqpoint{7.717170in}{0.557171in}}{\pgfqpoint{7.709357in}{0.564984in}}%
\pgfpathcurveto{\pgfqpoint{7.701543in}{0.572798in}}{\pgfqpoint{7.690944in}{0.577188in}}{\pgfqpoint{7.679894in}{0.577188in}}%
\pgfpathcurveto{\pgfqpoint{7.668844in}{0.577188in}}{\pgfqpoint{7.658245in}{0.572798in}}{\pgfqpoint{7.650431in}{0.564984in}}%
\pgfpathcurveto{\pgfqpoint{7.642617in}{0.557171in}}{\pgfqpoint{7.638227in}{0.546572in}}{\pgfqpoint{7.638227in}{0.535522in}}%
\pgfpathcurveto{\pgfqpoint{7.638227in}{0.524471in}}{\pgfqpoint{7.642617in}{0.513872in}}{\pgfqpoint{7.650431in}{0.506059in}}%
\pgfpathcurveto{\pgfqpoint{7.658245in}{0.498245in}}{\pgfqpoint{7.668844in}{0.493855in}}{\pgfqpoint{7.679894in}{0.493855in}}%
\pgfpathlineto{\pgfqpoint{7.679894in}{0.493855in}}%
\pgfpathclose%
\pgfusepath{stroke}%
\end{pgfscope}%
\begin{pgfscope}%
\pgfpathrectangle{\pgfqpoint{7.512535in}{0.437222in}}{\pgfqpoint{6.275590in}{5.159444in}}%
\pgfusepath{clip}%
\pgfsetbuttcap%
\pgfsetroundjoin%
\pgfsetlinewidth{1.003750pt}%
\definecolor{currentstroke}{rgb}{0.827451,0.827451,0.827451}%
\pgfsetstrokecolor{currentstroke}%
\pgfsetstrokeopacity{0.800000}%
\pgfsetdash{}{0pt}%
\pgfpathmoveto{\pgfqpoint{13.731814in}{5.528743in}}%
\pgfpathcurveto{\pgfqpoint{13.742865in}{5.528743in}}{\pgfqpoint{13.753464in}{5.533133in}}{\pgfqpoint{13.761277in}{5.540947in}}%
\pgfpathcurveto{\pgfqpoint{13.769091in}{5.548760in}}{\pgfqpoint{13.773481in}{5.559359in}}{\pgfqpoint{13.773481in}{5.570410in}}%
\pgfpathcurveto{\pgfqpoint{13.773481in}{5.581460in}}{\pgfqpoint{13.769091in}{5.592059in}}{\pgfqpoint{13.761277in}{5.599872in}}%
\pgfpathcurveto{\pgfqpoint{13.753464in}{5.607686in}}{\pgfqpoint{13.742865in}{5.612076in}}{\pgfqpoint{13.731814in}{5.612076in}}%
\pgfpathcurveto{\pgfqpoint{13.720764in}{5.612076in}}{\pgfqpoint{13.710165in}{5.607686in}}{\pgfqpoint{13.702352in}{5.599872in}}%
\pgfpathcurveto{\pgfqpoint{13.694538in}{5.592059in}}{\pgfqpoint{13.690148in}{5.581460in}}{\pgfqpoint{13.690148in}{5.570410in}}%
\pgfpathcurveto{\pgfqpoint{13.690148in}{5.559359in}}{\pgfqpoint{13.694538in}{5.548760in}}{\pgfqpoint{13.702352in}{5.540947in}}%
\pgfpathcurveto{\pgfqpoint{13.710165in}{5.533133in}}{\pgfqpoint{13.720764in}{5.528743in}}{\pgfqpoint{13.731814in}{5.528743in}}%
\pgfpathlineto{\pgfqpoint{13.731814in}{5.528743in}}%
\pgfpathclose%
\pgfusepath{stroke}%
\end{pgfscope}%
\begin{pgfscope}%
\pgfpathrectangle{\pgfqpoint{7.512535in}{0.437222in}}{\pgfqpoint{6.275590in}{5.159444in}}%
\pgfusepath{clip}%
\pgfsetbuttcap%
\pgfsetroundjoin%
\pgfsetlinewidth{1.003750pt}%
\definecolor{currentstroke}{rgb}{0.827451,0.827451,0.827451}%
\pgfsetstrokecolor{currentstroke}%
\pgfsetstrokeopacity{0.800000}%
\pgfsetdash{}{0pt}%
\pgfpathmoveto{\pgfqpoint{13.419279in}{5.547910in}}%
\pgfpathcurveto{\pgfqpoint{13.430329in}{5.547910in}}{\pgfqpoint{13.440928in}{5.552301in}}{\pgfqpoint{13.448742in}{5.560114in}}%
\pgfpathcurveto{\pgfqpoint{13.456555in}{5.567928in}}{\pgfqpoint{13.460945in}{5.578527in}}{\pgfqpoint{13.460945in}{5.589577in}}%
\pgfpathcurveto{\pgfqpoint{13.460945in}{5.600627in}}{\pgfqpoint{13.456555in}{5.611226in}}{\pgfqpoint{13.448742in}{5.619040in}}%
\pgfpathcurveto{\pgfqpoint{13.440928in}{5.626854in}}{\pgfqpoint{13.430329in}{5.631244in}}{\pgfqpoint{13.419279in}{5.631244in}}%
\pgfpathcurveto{\pgfqpoint{13.408229in}{5.631244in}}{\pgfqpoint{13.397630in}{5.626854in}}{\pgfqpoint{13.389816in}{5.619040in}}%
\pgfpathcurveto{\pgfqpoint{13.382002in}{5.611226in}}{\pgfqpoint{13.377612in}{5.600627in}}{\pgfqpoint{13.377612in}{5.589577in}}%
\pgfpathcurveto{\pgfqpoint{13.377612in}{5.578527in}}{\pgfqpoint{13.382002in}{5.567928in}}{\pgfqpoint{13.389816in}{5.560114in}}%
\pgfpathcurveto{\pgfqpoint{13.397630in}{5.552301in}}{\pgfqpoint{13.408229in}{5.547910in}}{\pgfqpoint{13.419279in}{5.547910in}}%
\pgfpathlineto{\pgfqpoint{13.419279in}{5.547910in}}%
\pgfpathclose%
\pgfusepath{stroke}%
\end{pgfscope}%
\begin{pgfscope}%
\pgfpathrectangle{\pgfqpoint{7.512535in}{0.437222in}}{\pgfqpoint{6.275590in}{5.159444in}}%
\pgfusepath{clip}%
\pgfsetbuttcap%
\pgfsetroundjoin%
\pgfsetlinewidth{1.003750pt}%
\definecolor{currentstroke}{rgb}{0.827451,0.827451,0.827451}%
\pgfsetstrokecolor{currentstroke}%
\pgfsetstrokeopacity{0.800000}%
\pgfsetdash{}{0pt}%
\pgfpathmoveto{\pgfqpoint{12.873966in}{5.431219in}}%
\pgfpathcurveto{\pgfqpoint{12.885016in}{5.431219in}}{\pgfqpoint{12.895615in}{5.435610in}}{\pgfqpoint{12.903429in}{5.443423in}}%
\pgfpathcurveto{\pgfqpoint{12.911243in}{5.451237in}}{\pgfqpoint{12.915633in}{5.461836in}}{\pgfqpoint{12.915633in}{5.472886in}}%
\pgfpathcurveto{\pgfqpoint{12.915633in}{5.483936in}}{\pgfqpoint{12.911243in}{5.494535in}}{\pgfqpoint{12.903429in}{5.502349in}}%
\pgfpathcurveto{\pgfqpoint{12.895615in}{5.510163in}}{\pgfqpoint{12.885016in}{5.514553in}}{\pgfqpoint{12.873966in}{5.514553in}}%
\pgfpathcurveto{\pgfqpoint{12.862916in}{5.514553in}}{\pgfqpoint{12.852317in}{5.510163in}}{\pgfqpoint{12.844503in}{5.502349in}}%
\pgfpathcurveto{\pgfqpoint{12.836690in}{5.494535in}}{\pgfqpoint{12.832300in}{5.483936in}}{\pgfqpoint{12.832300in}{5.472886in}}%
\pgfpathcurveto{\pgfqpoint{12.832300in}{5.461836in}}{\pgfqpoint{12.836690in}{5.451237in}}{\pgfqpoint{12.844503in}{5.443423in}}%
\pgfpathcurveto{\pgfqpoint{12.852317in}{5.435610in}}{\pgfqpoint{12.862916in}{5.431219in}}{\pgfqpoint{12.873966in}{5.431219in}}%
\pgfpathlineto{\pgfqpoint{12.873966in}{5.431219in}}%
\pgfpathclose%
\pgfusepath{stroke}%
\end{pgfscope}%
\begin{pgfscope}%
\pgfpathrectangle{\pgfqpoint{7.512535in}{0.437222in}}{\pgfqpoint{6.275590in}{5.159444in}}%
\pgfusepath{clip}%
\pgfsetbuttcap%
\pgfsetroundjoin%
\pgfsetlinewidth{1.003750pt}%
\definecolor{currentstroke}{rgb}{0.827451,0.827451,0.827451}%
\pgfsetstrokecolor{currentstroke}%
\pgfsetstrokeopacity{0.800000}%
\pgfsetdash{}{0pt}%
\pgfpathmoveto{\pgfqpoint{7.711198in}{0.615504in}}%
\pgfpathcurveto{\pgfqpoint{7.722248in}{0.615504in}}{\pgfqpoint{7.732847in}{0.619894in}}{\pgfqpoint{7.740661in}{0.627708in}}%
\pgfpathcurveto{\pgfqpoint{7.748474in}{0.635522in}}{\pgfqpoint{7.752865in}{0.646121in}}{\pgfqpoint{7.752865in}{0.657171in}}%
\pgfpathcurveto{\pgfqpoint{7.752865in}{0.668221in}}{\pgfqpoint{7.748474in}{0.678820in}}{\pgfqpoint{7.740661in}{0.686633in}}%
\pgfpathcurveto{\pgfqpoint{7.732847in}{0.694447in}}{\pgfqpoint{7.722248in}{0.698837in}}{\pgfqpoint{7.711198in}{0.698837in}}%
\pgfpathcurveto{\pgfqpoint{7.700148in}{0.698837in}}{\pgfqpoint{7.689549in}{0.694447in}}{\pgfqpoint{7.681735in}{0.686633in}}%
\pgfpathcurveto{\pgfqpoint{7.673921in}{0.678820in}}{\pgfqpoint{7.669531in}{0.668221in}}{\pgfqpoint{7.669531in}{0.657171in}}%
\pgfpathcurveto{\pgfqpoint{7.669531in}{0.646121in}}{\pgfqpoint{7.673921in}{0.635522in}}{\pgfqpoint{7.681735in}{0.627708in}}%
\pgfpathcurveto{\pgfqpoint{7.689549in}{0.619894in}}{\pgfqpoint{7.700148in}{0.615504in}}{\pgfqpoint{7.711198in}{0.615504in}}%
\pgfpathlineto{\pgfqpoint{7.711198in}{0.615504in}}%
\pgfpathclose%
\pgfusepath{stroke}%
\end{pgfscope}%
\begin{pgfscope}%
\pgfpathrectangle{\pgfqpoint{7.512535in}{0.437222in}}{\pgfqpoint{6.275590in}{5.159444in}}%
\pgfusepath{clip}%
\pgfsetbuttcap%
\pgfsetroundjoin%
\pgfsetlinewidth{1.003750pt}%
\definecolor{currentstroke}{rgb}{0.827451,0.827451,0.827451}%
\pgfsetstrokecolor{currentstroke}%
\pgfsetstrokeopacity{0.800000}%
\pgfsetdash{}{0pt}%
\pgfpathmoveto{\pgfqpoint{11.962704in}{5.469388in}}%
\pgfpathcurveto{\pgfqpoint{11.973755in}{5.469388in}}{\pgfqpoint{11.984354in}{5.473778in}}{\pgfqpoint{11.992167in}{5.481592in}}%
\pgfpathcurveto{\pgfqpoint{11.999981in}{5.489405in}}{\pgfqpoint{12.004371in}{5.500004in}}{\pgfqpoint{12.004371in}{5.511054in}}%
\pgfpathcurveto{\pgfqpoint{12.004371in}{5.522105in}}{\pgfqpoint{11.999981in}{5.532704in}}{\pgfqpoint{11.992167in}{5.540517in}}%
\pgfpathcurveto{\pgfqpoint{11.984354in}{5.548331in}}{\pgfqpoint{11.973755in}{5.552721in}}{\pgfqpoint{11.962704in}{5.552721in}}%
\pgfpathcurveto{\pgfqpoint{11.951654in}{5.552721in}}{\pgfqpoint{11.941055in}{5.548331in}}{\pgfqpoint{11.933242in}{5.540517in}}%
\pgfpathcurveto{\pgfqpoint{11.925428in}{5.532704in}}{\pgfqpoint{11.921038in}{5.522105in}}{\pgfqpoint{11.921038in}{5.511054in}}%
\pgfpathcurveto{\pgfqpoint{11.921038in}{5.500004in}}{\pgfqpoint{11.925428in}{5.489405in}}{\pgfqpoint{11.933242in}{5.481592in}}%
\pgfpathcurveto{\pgfqpoint{11.941055in}{5.473778in}}{\pgfqpoint{11.951654in}{5.469388in}}{\pgfqpoint{11.962704in}{5.469388in}}%
\pgfpathlineto{\pgfqpoint{11.962704in}{5.469388in}}%
\pgfpathclose%
\pgfusepath{stroke}%
\end{pgfscope}%
\begin{pgfscope}%
\pgfpathrectangle{\pgfqpoint{7.512535in}{0.437222in}}{\pgfqpoint{6.275590in}{5.159444in}}%
\pgfusepath{clip}%
\pgfsetbuttcap%
\pgfsetroundjoin%
\pgfsetlinewidth{1.003750pt}%
\definecolor{currentstroke}{rgb}{0.827451,0.827451,0.827451}%
\pgfsetstrokecolor{currentstroke}%
\pgfsetstrokeopacity{0.800000}%
\pgfsetdash{}{0pt}%
\pgfpathmoveto{\pgfqpoint{13.273901in}{5.501584in}}%
\pgfpathcurveto{\pgfqpoint{13.284951in}{5.501584in}}{\pgfqpoint{13.295550in}{5.505974in}}{\pgfqpoint{13.303364in}{5.513787in}}%
\pgfpathcurveto{\pgfqpoint{13.311177in}{5.521601in}}{\pgfqpoint{13.315568in}{5.532200in}}{\pgfqpoint{13.315568in}{5.543250in}}%
\pgfpathcurveto{\pgfqpoint{13.315568in}{5.554300in}}{\pgfqpoint{13.311177in}{5.564899in}}{\pgfqpoint{13.303364in}{5.572713in}}%
\pgfpathcurveto{\pgfqpoint{13.295550in}{5.580527in}}{\pgfqpoint{13.284951in}{5.584917in}}{\pgfqpoint{13.273901in}{5.584917in}}%
\pgfpathcurveto{\pgfqpoint{13.262851in}{5.584917in}}{\pgfqpoint{13.252252in}{5.580527in}}{\pgfqpoint{13.244438in}{5.572713in}}%
\pgfpathcurveto{\pgfqpoint{13.236625in}{5.564899in}}{\pgfqpoint{13.232234in}{5.554300in}}{\pgfqpoint{13.232234in}{5.543250in}}%
\pgfpathcurveto{\pgfqpoint{13.232234in}{5.532200in}}{\pgfqpoint{13.236625in}{5.521601in}}{\pgfqpoint{13.244438in}{5.513787in}}%
\pgfpathcurveto{\pgfqpoint{13.252252in}{5.505974in}}{\pgfqpoint{13.262851in}{5.501584in}}{\pgfqpoint{13.273901in}{5.501584in}}%
\pgfpathlineto{\pgfqpoint{13.273901in}{5.501584in}}%
\pgfpathclose%
\pgfusepath{stroke}%
\end{pgfscope}%
\begin{pgfscope}%
\pgfpathrectangle{\pgfqpoint{7.512535in}{0.437222in}}{\pgfqpoint{6.275590in}{5.159444in}}%
\pgfusepath{clip}%
\pgfsetbuttcap%
\pgfsetroundjoin%
\pgfsetlinewidth{1.003750pt}%
\definecolor{currentstroke}{rgb}{0.827451,0.827451,0.827451}%
\pgfsetstrokecolor{currentstroke}%
\pgfsetstrokeopacity{0.800000}%
\pgfsetdash{}{0pt}%
\pgfpathmoveto{\pgfqpoint{8.524887in}{2.541580in}}%
\pgfpathcurveto{\pgfqpoint{8.535937in}{2.541580in}}{\pgfqpoint{8.546536in}{2.545970in}}{\pgfqpoint{8.554350in}{2.553784in}}%
\pgfpathcurveto{\pgfqpoint{8.562163in}{2.561597in}}{\pgfqpoint{8.566554in}{2.572196in}}{\pgfqpoint{8.566554in}{2.583246in}}%
\pgfpathcurveto{\pgfqpoint{8.566554in}{2.594296in}}{\pgfqpoint{8.562163in}{2.604896in}}{\pgfqpoint{8.554350in}{2.612709in}}%
\pgfpathcurveto{\pgfqpoint{8.546536in}{2.620523in}}{\pgfqpoint{8.535937in}{2.624913in}}{\pgfqpoint{8.524887in}{2.624913in}}%
\pgfpathcurveto{\pgfqpoint{8.513837in}{2.624913in}}{\pgfqpoint{8.503238in}{2.620523in}}{\pgfqpoint{8.495424in}{2.612709in}}%
\pgfpathcurveto{\pgfqpoint{8.487611in}{2.604896in}}{\pgfqpoint{8.483220in}{2.594296in}}{\pgfqpoint{8.483220in}{2.583246in}}%
\pgfpathcurveto{\pgfqpoint{8.483220in}{2.572196in}}{\pgfqpoint{8.487611in}{2.561597in}}{\pgfqpoint{8.495424in}{2.553784in}}%
\pgfpathcurveto{\pgfqpoint{8.503238in}{2.545970in}}{\pgfqpoint{8.513837in}{2.541580in}}{\pgfqpoint{8.524887in}{2.541580in}}%
\pgfpathlineto{\pgfqpoint{8.524887in}{2.541580in}}%
\pgfpathclose%
\pgfusepath{stroke}%
\end{pgfscope}%
\begin{pgfscope}%
\pgfpathrectangle{\pgfqpoint{7.512535in}{0.437222in}}{\pgfqpoint{6.275590in}{5.159444in}}%
\pgfusepath{clip}%
\pgfsetbuttcap%
\pgfsetroundjoin%
\pgfsetlinewidth{1.003750pt}%
\definecolor{currentstroke}{rgb}{0.827451,0.827451,0.827451}%
\pgfsetstrokecolor{currentstroke}%
\pgfsetstrokeopacity{0.800000}%
\pgfsetdash{}{0pt}%
\pgfpathmoveto{\pgfqpoint{8.800029in}{1.838807in}}%
\pgfpathcurveto{\pgfqpoint{8.811079in}{1.838807in}}{\pgfqpoint{8.821678in}{1.843198in}}{\pgfqpoint{8.829491in}{1.851011in}}%
\pgfpathcurveto{\pgfqpoint{8.837305in}{1.858825in}}{\pgfqpoint{8.841695in}{1.869424in}}{\pgfqpoint{8.841695in}{1.880474in}}%
\pgfpathcurveto{\pgfqpoint{8.841695in}{1.891524in}}{\pgfqpoint{8.837305in}{1.902123in}}{\pgfqpoint{8.829491in}{1.909937in}}%
\pgfpathcurveto{\pgfqpoint{8.821678in}{1.917750in}}{\pgfqpoint{8.811079in}{1.922141in}}{\pgfqpoint{8.800029in}{1.922141in}}%
\pgfpathcurveto{\pgfqpoint{8.788978in}{1.922141in}}{\pgfqpoint{8.778379in}{1.917750in}}{\pgfqpoint{8.770566in}{1.909937in}}%
\pgfpathcurveto{\pgfqpoint{8.762752in}{1.902123in}}{\pgfqpoint{8.758362in}{1.891524in}}{\pgfqpoint{8.758362in}{1.880474in}}%
\pgfpathcurveto{\pgfqpoint{8.758362in}{1.869424in}}{\pgfqpoint{8.762752in}{1.858825in}}{\pgfqpoint{8.770566in}{1.851011in}}%
\pgfpathcurveto{\pgfqpoint{8.778379in}{1.843198in}}{\pgfqpoint{8.788978in}{1.838807in}}{\pgfqpoint{8.800029in}{1.838807in}}%
\pgfpathlineto{\pgfqpoint{8.800029in}{1.838807in}}%
\pgfpathclose%
\pgfusepath{stroke}%
\end{pgfscope}%
\begin{pgfscope}%
\pgfpathrectangle{\pgfqpoint{7.512535in}{0.437222in}}{\pgfqpoint{6.275590in}{5.159444in}}%
\pgfusepath{clip}%
\pgfsetbuttcap%
\pgfsetroundjoin%
\pgfsetlinewidth{1.003750pt}%
\definecolor{currentstroke}{rgb}{0.827451,0.827451,0.827451}%
\pgfsetstrokecolor{currentstroke}%
\pgfsetstrokeopacity{0.800000}%
\pgfsetdash{}{0pt}%
\pgfpathmoveto{\pgfqpoint{7.972802in}{0.679358in}}%
\pgfpathcurveto{\pgfqpoint{7.983853in}{0.679358in}}{\pgfqpoint{7.994452in}{0.683748in}}{\pgfqpoint{8.002265in}{0.691562in}}%
\pgfpathcurveto{\pgfqpoint{8.010079in}{0.699376in}}{\pgfqpoint{8.014469in}{0.709975in}}{\pgfqpoint{8.014469in}{0.721025in}}%
\pgfpathcurveto{\pgfqpoint{8.014469in}{0.732075in}}{\pgfqpoint{8.010079in}{0.742674in}}{\pgfqpoint{8.002265in}{0.750487in}}%
\pgfpathcurveto{\pgfqpoint{7.994452in}{0.758301in}}{\pgfqpoint{7.983853in}{0.762691in}}{\pgfqpoint{7.972802in}{0.762691in}}%
\pgfpathcurveto{\pgfqpoint{7.961752in}{0.762691in}}{\pgfqpoint{7.951153in}{0.758301in}}{\pgfqpoint{7.943340in}{0.750487in}}%
\pgfpathcurveto{\pgfqpoint{7.935526in}{0.742674in}}{\pgfqpoint{7.931136in}{0.732075in}}{\pgfqpoint{7.931136in}{0.721025in}}%
\pgfpathcurveto{\pgfqpoint{7.931136in}{0.709975in}}{\pgfqpoint{7.935526in}{0.699376in}}{\pgfqpoint{7.943340in}{0.691562in}}%
\pgfpathcurveto{\pgfqpoint{7.951153in}{0.683748in}}{\pgfqpoint{7.961752in}{0.679358in}}{\pgfqpoint{7.972802in}{0.679358in}}%
\pgfpathlineto{\pgfqpoint{7.972802in}{0.679358in}}%
\pgfpathclose%
\pgfusepath{stroke}%
\end{pgfscope}%
\begin{pgfscope}%
\pgfpathrectangle{\pgfqpoint{7.512535in}{0.437222in}}{\pgfqpoint{6.275590in}{5.159444in}}%
\pgfusepath{clip}%
\pgfsetbuttcap%
\pgfsetroundjoin%
\pgfsetlinewidth{1.003750pt}%
\definecolor{currentstroke}{rgb}{0.827451,0.827451,0.827451}%
\pgfsetstrokecolor{currentstroke}%
\pgfsetstrokeopacity{0.800000}%
\pgfsetdash{}{0pt}%
\pgfpathmoveto{\pgfqpoint{8.605492in}{2.894644in}}%
\pgfpathcurveto{\pgfqpoint{8.616542in}{2.894644in}}{\pgfqpoint{8.627141in}{2.899034in}}{\pgfqpoint{8.634955in}{2.906847in}}%
\pgfpathcurveto{\pgfqpoint{8.642768in}{2.914661in}}{\pgfqpoint{8.647158in}{2.925260in}}{\pgfqpoint{8.647158in}{2.936310in}}%
\pgfpathcurveto{\pgfqpoint{8.647158in}{2.947360in}}{\pgfqpoint{8.642768in}{2.957959in}}{\pgfqpoint{8.634955in}{2.965773in}}%
\pgfpathcurveto{\pgfqpoint{8.627141in}{2.973587in}}{\pgfqpoint{8.616542in}{2.977977in}}{\pgfqpoint{8.605492in}{2.977977in}}%
\pgfpathcurveto{\pgfqpoint{8.594442in}{2.977977in}}{\pgfqpoint{8.583843in}{2.973587in}}{\pgfqpoint{8.576029in}{2.965773in}}%
\pgfpathcurveto{\pgfqpoint{8.568215in}{2.957959in}}{\pgfqpoint{8.563825in}{2.947360in}}{\pgfqpoint{8.563825in}{2.936310in}}%
\pgfpathcurveto{\pgfqpoint{8.563825in}{2.925260in}}{\pgfqpoint{8.568215in}{2.914661in}}{\pgfqpoint{8.576029in}{2.906847in}}%
\pgfpathcurveto{\pgfqpoint{8.583843in}{2.899034in}}{\pgfqpoint{8.594442in}{2.894644in}}{\pgfqpoint{8.605492in}{2.894644in}}%
\pgfpathlineto{\pgfqpoint{8.605492in}{2.894644in}}%
\pgfpathclose%
\pgfusepath{stroke}%
\end{pgfscope}%
\begin{pgfscope}%
\pgfpathrectangle{\pgfqpoint{7.512535in}{0.437222in}}{\pgfqpoint{6.275590in}{5.159444in}}%
\pgfusepath{clip}%
\pgfsetbuttcap%
\pgfsetroundjoin%
\pgfsetlinewidth{1.003750pt}%
\definecolor{currentstroke}{rgb}{0.827451,0.827451,0.827451}%
\pgfsetstrokecolor{currentstroke}%
\pgfsetstrokeopacity{0.800000}%
\pgfsetdash{}{0pt}%
\pgfpathmoveto{\pgfqpoint{13.101307in}{5.417061in}}%
\pgfpathcurveto{\pgfqpoint{13.112357in}{5.417061in}}{\pgfqpoint{13.122956in}{5.421451in}}{\pgfqpoint{13.130770in}{5.429265in}}%
\pgfpathcurveto{\pgfqpoint{13.138583in}{5.437079in}}{\pgfqpoint{13.142973in}{5.447678in}}{\pgfqpoint{13.142973in}{5.458728in}}%
\pgfpathcurveto{\pgfqpoint{13.142973in}{5.469778in}}{\pgfqpoint{13.138583in}{5.480377in}}{\pgfqpoint{13.130770in}{5.488191in}}%
\pgfpathcurveto{\pgfqpoint{13.122956in}{5.496004in}}{\pgfqpoint{13.112357in}{5.500395in}}{\pgfqpoint{13.101307in}{5.500395in}}%
\pgfpathcurveto{\pgfqpoint{13.090257in}{5.500395in}}{\pgfqpoint{13.079658in}{5.496004in}}{\pgfqpoint{13.071844in}{5.488191in}}%
\pgfpathcurveto{\pgfqpoint{13.064030in}{5.480377in}}{\pgfqpoint{13.059640in}{5.469778in}}{\pgfqpoint{13.059640in}{5.458728in}}%
\pgfpathcurveto{\pgfqpoint{13.059640in}{5.447678in}}{\pgfqpoint{13.064030in}{5.437079in}}{\pgfqpoint{13.071844in}{5.429265in}}%
\pgfpathcurveto{\pgfqpoint{13.079658in}{5.421451in}}{\pgfqpoint{13.090257in}{5.417061in}}{\pgfqpoint{13.101307in}{5.417061in}}%
\pgfpathlineto{\pgfqpoint{13.101307in}{5.417061in}}%
\pgfpathclose%
\pgfusepath{stroke}%
\end{pgfscope}%
\begin{pgfscope}%
\pgfpathrectangle{\pgfqpoint{7.512535in}{0.437222in}}{\pgfqpoint{6.275590in}{5.159444in}}%
\pgfusepath{clip}%
\pgfsetbuttcap%
\pgfsetroundjoin%
\pgfsetlinewidth{1.003750pt}%
\definecolor{currentstroke}{rgb}{0.827451,0.827451,0.827451}%
\pgfsetstrokecolor{currentstroke}%
\pgfsetstrokeopacity{0.800000}%
\pgfsetdash{}{0pt}%
\pgfpathmoveto{\pgfqpoint{11.046560in}{5.315419in}}%
\pgfpathcurveto{\pgfqpoint{11.057610in}{5.315419in}}{\pgfqpoint{11.068209in}{5.319810in}}{\pgfqpoint{11.076023in}{5.327623in}}%
\pgfpathcurveto{\pgfqpoint{11.083837in}{5.335437in}}{\pgfqpoint{11.088227in}{5.346036in}}{\pgfqpoint{11.088227in}{5.357086in}}%
\pgfpathcurveto{\pgfqpoint{11.088227in}{5.368136in}}{\pgfqpoint{11.083837in}{5.378735in}}{\pgfqpoint{11.076023in}{5.386549in}}%
\pgfpathcurveto{\pgfqpoint{11.068209in}{5.394362in}}{\pgfqpoint{11.057610in}{5.398753in}}{\pgfqpoint{11.046560in}{5.398753in}}%
\pgfpathcurveto{\pgfqpoint{11.035510in}{5.398753in}}{\pgfqpoint{11.024911in}{5.394362in}}{\pgfqpoint{11.017098in}{5.386549in}}%
\pgfpathcurveto{\pgfqpoint{11.009284in}{5.378735in}}{\pgfqpoint{11.004894in}{5.368136in}}{\pgfqpoint{11.004894in}{5.357086in}}%
\pgfpathcurveto{\pgfqpoint{11.004894in}{5.346036in}}{\pgfqpoint{11.009284in}{5.335437in}}{\pgfqpoint{11.017098in}{5.327623in}}%
\pgfpathcurveto{\pgfqpoint{11.024911in}{5.319810in}}{\pgfqpoint{11.035510in}{5.315419in}}{\pgfqpoint{11.046560in}{5.315419in}}%
\pgfpathlineto{\pgfqpoint{11.046560in}{5.315419in}}%
\pgfpathclose%
\pgfusepath{stroke}%
\end{pgfscope}%
\begin{pgfscope}%
\pgfpathrectangle{\pgfqpoint{7.512535in}{0.437222in}}{\pgfqpoint{6.275590in}{5.159444in}}%
\pgfusepath{clip}%
\pgfsetbuttcap%
\pgfsetroundjoin%
\pgfsetlinewidth{1.003750pt}%
\definecolor{currentstroke}{rgb}{0.827451,0.827451,0.827451}%
\pgfsetstrokecolor{currentstroke}%
\pgfsetstrokeopacity{0.800000}%
\pgfsetdash{}{0pt}%
\pgfpathmoveto{\pgfqpoint{12.072745in}{5.039509in}}%
\pgfpathcurveto{\pgfqpoint{12.083796in}{5.039509in}}{\pgfqpoint{12.094395in}{5.043899in}}{\pgfqpoint{12.102208in}{5.051713in}}%
\pgfpathcurveto{\pgfqpoint{12.110022in}{5.059526in}}{\pgfqpoint{12.114412in}{5.070125in}}{\pgfqpoint{12.114412in}{5.081175in}}%
\pgfpathcurveto{\pgfqpoint{12.114412in}{5.092226in}}{\pgfqpoint{12.110022in}{5.102825in}}{\pgfqpoint{12.102208in}{5.110638in}}%
\pgfpathcurveto{\pgfqpoint{12.094395in}{5.118452in}}{\pgfqpoint{12.083796in}{5.122842in}}{\pgfqpoint{12.072745in}{5.122842in}}%
\pgfpathcurveto{\pgfqpoint{12.061695in}{5.122842in}}{\pgfqpoint{12.051096in}{5.118452in}}{\pgfqpoint{12.043283in}{5.110638in}}%
\pgfpathcurveto{\pgfqpoint{12.035469in}{5.102825in}}{\pgfqpoint{12.031079in}{5.092226in}}{\pgfqpoint{12.031079in}{5.081175in}}%
\pgfpathcurveto{\pgfqpoint{12.031079in}{5.070125in}}{\pgfqpoint{12.035469in}{5.059526in}}{\pgfqpoint{12.043283in}{5.051713in}}%
\pgfpathcurveto{\pgfqpoint{12.051096in}{5.043899in}}{\pgfqpoint{12.061695in}{5.039509in}}{\pgfqpoint{12.072745in}{5.039509in}}%
\pgfpathlineto{\pgfqpoint{12.072745in}{5.039509in}}%
\pgfpathclose%
\pgfusepath{stroke}%
\end{pgfscope}%
\begin{pgfscope}%
\pgfpathrectangle{\pgfqpoint{7.512535in}{0.437222in}}{\pgfqpoint{6.275590in}{5.159444in}}%
\pgfusepath{clip}%
\pgfsetbuttcap%
\pgfsetroundjoin%
\pgfsetlinewidth{1.003750pt}%
\definecolor{currentstroke}{rgb}{0.827451,0.827451,0.827451}%
\pgfsetstrokecolor{currentstroke}%
\pgfsetstrokeopacity{0.800000}%
\pgfsetdash{}{0pt}%
\pgfpathmoveto{\pgfqpoint{12.227551in}{5.553091in}}%
\pgfpathcurveto{\pgfqpoint{12.238601in}{5.553091in}}{\pgfqpoint{12.249200in}{5.557481in}}{\pgfqpoint{12.257013in}{5.565295in}}%
\pgfpathcurveto{\pgfqpoint{12.264827in}{5.573108in}}{\pgfqpoint{12.269217in}{5.583707in}}{\pgfqpoint{12.269217in}{5.594757in}}%
\pgfpathcurveto{\pgfqpoint{12.269217in}{5.605808in}}{\pgfqpoint{12.264827in}{5.616407in}}{\pgfqpoint{12.257013in}{5.624220in}}%
\pgfpathcurveto{\pgfqpoint{12.249200in}{5.632034in}}{\pgfqpoint{12.238601in}{5.636424in}}{\pgfqpoint{12.227551in}{5.636424in}}%
\pgfpathcurveto{\pgfqpoint{12.216500in}{5.636424in}}{\pgfqpoint{12.205901in}{5.632034in}}{\pgfqpoint{12.198088in}{5.624220in}}%
\pgfpathcurveto{\pgfqpoint{12.190274in}{5.616407in}}{\pgfqpoint{12.185884in}{5.605808in}}{\pgfqpoint{12.185884in}{5.594757in}}%
\pgfpathcurveto{\pgfqpoint{12.185884in}{5.583707in}}{\pgfqpoint{12.190274in}{5.573108in}}{\pgfqpoint{12.198088in}{5.565295in}}%
\pgfpathcurveto{\pgfqpoint{12.205901in}{5.557481in}}{\pgfqpoint{12.216500in}{5.553091in}}{\pgfqpoint{12.227551in}{5.553091in}}%
\pgfpathlineto{\pgfqpoint{12.227551in}{5.553091in}}%
\pgfpathclose%
\pgfusepath{stroke}%
\end{pgfscope}%
\begin{pgfscope}%
\pgfpathrectangle{\pgfqpoint{7.512535in}{0.437222in}}{\pgfqpoint{6.275590in}{5.159444in}}%
\pgfusepath{clip}%
\pgfsetbuttcap%
\pgfsetroundjoin%
\pgfsetlinewidth{1.003750pt}%
\definecolor{currentstroke}{rgb}{0.827451,0.827451,0.827451}%
\pgfsetstrokecolor{currentstroke}%
\pgfsetstrokeopacity{0.800000}%
\pgfsetdash{}{0pt}%
\pgfpathmoveto{\pgfqpoint{10.299622in}{3.250664in}}%
\pgfpathcurveto{\pgfqpoint{10.310672in}{3.250664in}}{\pgfqpoint{10.321271in}{3.255054in}}{\pgfqpoint{10.329085in}{3.262868in}}%
\pgfpathcurveto{\pgfqpoint{10.336898in}{3.270682in}}{\pgfqpoint{10.341289in}{3.281281in}}{\pgfqpoint{10.341289in}{3.292331in}}%
\pgfpathcurveto{\pgfqpoint{10.341289in}{3.303381in}}{\pgfqpoint{10.336898in}{3.313980in}}{\pgfqpoint{10.329085in}{3.321794in}}%
\pgfpathcurveto{\pgfqpoint{10.321271in}{3.329607in}}{\pgfqpoint{10.310672in}{3.333998in}}{\pgfqpoint{10.299622in}{3.333998in}}%
\pgfpathcurveto{\pgfqpoint{10.288572in}{3.333998in}}{\pgfqpoint{10.277973in}{3.329607in}}{\pgfqpoint{10.270159in}{3.321794in}}%
\pgfpathcurveto{\pgfqpoint{10.262346in}{3.313980in}}{\pgfqpoint{10.257955in}{3.303381in}}{\pgfqpoint{10.257955in}{3.292331in}}%
\pgfpathcurveto{\pgfqpoint{10.257955in}{3.281281in}}{\pgfqpoint{10.262346in}{3.270682in}}{\pgfqpoint{10.270159in}{3.262868in}}%
\pgfpathcurveto{\pgfqpoint{10.277973in}{3.255054in}}{\pgfqpoint{10.288572in}{3.250664in}}{\pgfqpoint{10.299622in}{3.250664in}}%
\pgfpathlineto{\pgfqpoint{10.299622in}{3.250664in}}%
\pgfpathclose%
\pgfusepath{stroke}%
\end{pgfscope}%
\begin{pgfscope}%
\pgfpathrectangle{\pgfqpoint{7.512535in}{0.437222in}}{\pgfqpoint{6.275590in}{5.159444in}}%
\pgfusepath{clip}%
\pgfsetbuttcap%
\pgfsetroundjoin%
\pgfsetlinewidth{1.003750pt}%
\definecolor{currentstroke}{rgb}{0.827451,0.827451,0.827451}%
\pgfsetstrokecolor{currentstroke}%
\pgfsetstrokeopacity{0.800000}%
\pgfsetdash{}{0pt}%
\pgfpathmoveto{\pgfqpoint{12.342968in}{5.553091in}}%
\pgfpathcurveto{\pgfqpoint{12.354019in}{5.553091in}}{\pgfqpoint{12.364618in}{5.557481in}}{\pgfqpoint{12.372431in}{5.565295in}}%
\pgfpathcurveto{\pgfqpoint{12.380245in}{5.573108in}}{\pgfqpoint{12.384635in}{5.583707in}}{\pgfqpoint{12.384635in}{5.594757in}}%
\pgfpathcurveto{\pgfqpoint{12.384635in}{5.605808in}}{\pgfqpoint{12.380245in}{5.616407in}}{\pgfqpoint{12.372431in}{5.624220in}}%
\pgfpathcurveto{\pgfqpoint{12.364618in}{5.632034in}}{\pgfqpoint{12.354019in}{5.636424in}}{\pgfqpoint{12.342968in}{5.636424in}}%
\pgfpathcurveto{\pgfqpoint{12.331918in}{5.636424in}}{\pgfqpoint{12.321319in}{5.632034in}}{\pgfqpoint{12.313506in}{5.624220in}}%
\pgfpathcurveto{\pgfqpoint{12.305692in}{5.616407in}}{\pgfqpoint{12.301302in}{5.605808in}}{\pgfqpoint{12.301302in}{5.594757in}}%
\pgfpathcurveto{\pgfqpoint{12.301302in}{5.583707in}}{\pgfqpoint{12.305692in}{5.573108in}}{\pgfqpoint{12.313506in}{5.565295in}}%
\pgfpathcurveto{\pgfqpoint{12.321319in}{5.557481in}}{\pgfqpoint{12.331918in}{5.553091in}}{\pgfqpoint{12.342968in}{5.553091in}}%
\pgfpathlineto{\pgfqpoint{12.342968in}{5.553091in}}%
\pgfpathclose%
\pgfusepath{stroke}%
\end{pgfscope}%
\begin{pgfscope}%
\pgfpathrectangle{\pgfqpoint{7.512535in}{0.437222in}}{\pgfqpoint{6.275590in}{5.159444in}}%
\pgfusepath{clip}%
\pgfsetbuttcap%
\pgfsetroundjoin%
\pgfsetlinewidth{1.003750pt}%
\definecolor{currentstroke}{rgb}{0.827451,0.827451,0.827451}%
\pgfsetstrokecolor{currentstroke}%
\pgfsetstrokeopacity{0.800000}%
\pgfsetdash{}{0pt}%
\pgfpathmoveto{\pgfqpoint{11.046560in}{5.140196in}}%
\pgfpathcurveto{\pgfqpoint{11.057610in}{5.140196in}}{\pgfqpoint{11.068209in}{5.144587in}}{\pgfqpoint{11.076023in}{5.152400in}}%
\pgfpathcurveto{\pgfqpoint{11.083837in}{5.160214in}}{\pgfqpoint{11.088227in}{5.170813in}}{\pgfqpoint{11.088227in}{5.181863in}}%
\pgfpathcurveto{\pgfqpoint{11.088227in}{5.192913in}}{\pgfqpoint{11.083837in}{5.203512in}}{\pgfqpoint{11.076023in}{5.211326in}}%
\pgfpathcurveto{\pgfqpoint{11.068209in}{5.219139in}}{\pgfqpoint{11.057610in}{5.223530in}}{\pgfqpoint{11.046560in}{5.223530in}}%
\pgfpathcurveto{\pgfqpoint{11.035510in}{5.223530in}}{\pgfqpoint{11.024911in}{5.219139in}}{\pgfqpoint{11.017098in}{5.211326in}}%
\pgfpathcurveto{\pgfqpoint{11.009284in}{5.203512in}}{\pgfqpoint{11.004894in}{5.192913in}}{\pgfqpoint{11.004894in}{5.181863in}}%
\pgfpathcurveto{\pgfqpoint{11.004894in}{5.170813in}}{\pgfqpoint{11.009284in}{5.160214in}}{\pgfqpoint{11.017098in}{5.152400in}}%
\pgfpathcurveto{\pgfqpoint{11.024911in}{5.144587in}}{\pgfqpoint{11.035510in}{5.140196in}}{\pgfqpoint{11.046560in}{5.140196in}}%
\pgfpathlineto{\pgfqpoint{11.046560in}{5.140196in}}%
\pgfpathclose%
\pgfusepath{stroke}%
\end{pgfscope}%
\begin{pgfscope}%
\pgfpathrectangle{\pgfqpoint{7.512535in}{0.437222in}}{\pgfqpoint{6.275590in}{5.159444in}}%
\pgfusepath{clip}%
\pgfsetbuttcap%
\pgfsetroundjoin%
\pgfsetlinewidth{1.003750pt}%
\definecolor{currentstroke}{rgb}{0.827451,0.827451,0.827451}%
\pgfsetstrokecolor{currentstroke}%
\pgfsetstrokeopacity{0.800000}%
\pgfsetdash{}{0pt}%
\pgfpathmoveto{\pgfqpoint{9.130854in}{1.867691in}}%
\pgfpathcurveto{\pgfqpoint{9.141904in}{1.867691in}}{\pgfqpoint{9.152503in}{1.872081in}}{\pgfqpoint{9.160317in}{1.879895in}}%
\pgfpathcurveto{\pgfqpoint{9.168130in}{1.887709in}}{\pgfqpoint{9.172521in}{1.898308in}}{\pgfqpoint{9.172521in}{1.909358in}}%
\pgfpathcurveto{\pgfqpoint{9.172521in}{1.920408in}}{\pgfqpoint{9.168130in}{1.931007in}}{\pgfqpoint{9.160317in}{1.938821in}}%
\pgfpathcurveto{\pgfqpoint{9.152503in}{1.946634in}}{\pgfqpoint{9.141904in}{1.951024in}}{\pgfqpoint{9.130854in}{1.951024in}}%
\pgfpathcurveto{\pgfqpoint{9.119804in}{1.951024in}}{\pgfqpoint{9.109205in}{1.946634in}}{\pgfqpoint{9.101391in}{1.938821in}}%
\pgfpathcurveto{\pgfqpoint{9.093577in}{1.931007in}}{\pgfqpoint{9.089187in}{1.920408in}}{\pgfqpoint{9.089187in}{1.909358in}}%
\pgfpathcurveto{\pgfqpoint{9.089187in}{1.898308in}}{\pgfqpoint{9.093577in}{1.887709in}}{\pgfqpoint{9.101391in}{1.879895in}}%
\pgfpathcurveto{\pgfqpoint{9.109205in}{1.872081in}}{\pgfqpoint{9.119804in}{1.867691in}}{\pgfqpoint{9.130854in}{1.867691in}}%
\pgfpathlineto{\pgfqpoint{9.130854in}{1.867691in}}%
\pgfpathclose%
\pgfusepath{stroke}%
\end{pgfscope}%
\begin{pgfscope}%
\pgfpathrectangle{\pgfqpoint{7.512535in}{0.437222in}}{\pgfqpoint{6.275590in}{5.159444in}}%
\pgfusepath{clip}%
\pgfsetbuttcap%
\pgfsetroundjoin%
\pgfsetlinewidth{1.003750pt}%
\definecolor{currentstroke}{rgb}{0.827451,0.827451,0.827451}%
\pgfsetstrokecolor{currentstroke}%
\pgfsetstrokeopacity{0.800000}%
\pgfsetdash{}{0pt}%
\pgfpathmoveto{\pgfqpoint{8.044541in}{2.113663in}}%
\pgfpathcurveto{\pgfqpoint{8.055591in}{2.113663in}}{\pgfqpoint{8.066190in}{2.118054in}}{\pgfqpoint{8.074003in}{2.125867in}}%
\pgfpathcurveto{\pgfqpoint{8.081817in}{2.133681in}}{\pgfqpoint{8.086207in}{2.144280in}}{\pgfqpoint{8.086207in}{2.155330in}}%
\pgfpathcurveto{\pgfqpoint{8.086207in}{2.166380in}}{\pgfqpoint{8.081817in}{2.176979in}}{\pgfqpoint{8.074003in}{2.184793in}}%
\pgfpathcurveto{\pgfqpoint{8.066190in}{2.192606in}}{\pgfqpoint{8.055591in}{2.196997in}}{\pgfqpoint{8.044541in}{2.196997in}}%
\pgfpathcurveto{\pgfqpoint{8.033491in}{2.196997in}}{\pgfqpoint{8.022891in}{2.192606in}}{\pgfqpoint{8.015078in}{2.184793in}}%
\pgfpathcurveto{\pgfqpoint{8.007264in}{2.176979in}}{\pgfqpoint{8.002874in}{2.166380in}}{\pgfqpoint{8.002874in}{2.155330in}}%
\pgfpathcurveto{\pgfqpoint{8.002874in}{2.144280in}}{\pgfqpoint{8.007264in}{2.133681in}}{\pgfqpoint{8.015078in}{2.125867in}}%
\pgfpathcurveto{\pgfqpoint{8.022891in}{2.118054in}}{\pgfqpoint{8.033491in}{2.113663in}}{\pgfqpoint{8.044541in}{2.113663in}}%
\pgfpathlineto{\pgfqpoint{8.044541in}{2.113663in}}%
\pgfpathclose%
\pgfusepath{stroke}%
\end{pgfscope}%
\begin{pgfscope}%
\pgfpathrectangle{\pgfqpoint{7.512535in}{0.437222in}}{\pgfqpoint{6.275590in}{5.159444in}}%
\pgfusepath{clip}%
\pgfsetbuttcap%
\pgfsetroundjoin%
\pgfsetlinewidth{1.003750pt}%
\definecolor{currentstroke}{rgb}{0.827451,0.827451,0.827451}%
\pgfsetstrokecolor{currentstroke}%
\pgfsetstrokeopacity{0.800000}%
\pgfsetdash{}{0pt}%
\pgfpathmoveto{\pgfqpoint{13.165622in}{5.501584in}}%
\pgfpathcurveto{\pgfqpoint{13.176672in}{5.501584in}}{\pgfqpoint{13.187271in}{5.505974in}}{\pgfqpoint{13.195085in}{5.513787in}}%
\pgfpathcurveto{\pgfqpoint{13.202899in}{5.521601in}}{\pgfqpoint{13.207289in}{5.532200in}}{\pgfqpoint{13.207289in}{5.543250in}}%
\pgfpathcurveto{\pgfqpoint{13.207289in}{5.554300in}}{\pgfqpoint{13.202899in}{5.564899in}}{\pgfqpoint{13.195085in}{5.572713in}}%
\pgfpathcurveto{\pgfqpoint{13.187271in}{5.580527in}}{\pgfqpoint{13.176672in}{5.584917in}}{\pgfqpoint{13.165622in}{5.584917in}}%
\pgfpathcurveto{\pgfqpoint{13.154572in}{5.584917in}}{\pgfqpoint{13.143973in}{5.580527in}}{\pgfqpoint{13.136159in}{5.572713in}}%
\pgfpathcurveto{\pgfqpoint{13.128346in}{5.564899in}}{\pgfqpoint{13.123955in}{5.554300in}}{\pgfqpoint{13.123955in}{5.543250in}}%
\pgfpathcurveto{\pgfqpoint{13.123955in}{5.532200in}}{\pgfqpoint{13.128346in}{5.521601in}}{\pgfqpoint{13.136159in}{5.513787in}}%
\pgfpathcurveto{\pgfqpoint{13.143973in}{5.505974in}}{\pgfqpoint{13.154572in}{5.501584in}}{\pgfqpoint{13.165622in}{5.501584in}}%
\pgfpathlineto{\pgfqpoint{13.165622in}{5.501584in}}%
\pgfpathclose%
\pgfusepath{stroke}%
\end{pgfscope}%
\begin{pgfscope}%
\pgfpathrectangle{\pgfqpoint{7.512535in}{0.437222in}}{\pgfqpoint{6.275590in}{5.159444in}}%
\pgfusepath{clip}%
\pgfsetbuttcap%
\pgfsetroundjoin%
\pgfsetlinewidth{1.003750pt}%
\definecolor{currentstroke}{rgb}{0.827451,0.827451,0.827451}%
\pgfsetstrokecolor{currentstroke}%
\pgfsetstrokeopacity{0.800000}%
\pgfsetdash{}{0pt}%
\pgfpathmoveto{\pgfqpoint{8.598634in}{1.552340in}}%
\pgfpathcurveto{\pgfqpoint{8.609684in}{1.552340in}}{\pgfqpoint{8.620283in}{1.556730in}}{\pgfqpoint{8.628097in}{1.564544in}}%
\pgfpathcurveto{\pgfqpoint{8.635910in}{1.572358in}}{\pgfqpoint{8.640300in}{1.582957in}}{\pgfqpoint{8.640300in}{1.594007in}}%
\pgfpathcurveto{\pgfqpoint{8.640300in}{1.605057in}}{\pgfqpoint{8.635910in}{1.615656in}}{\pgfqpoint{8.628097in}{1.623470in}}%
\pgfpathcurveto{\pgfqpoint{8.620283in}{1.631283in}}{\pgfqpoint{8.609684in}{1.635673in}}{\pgfqpoint{8.598634in}{1.635673in}}%
\pgfpathcurveto{\pgfqpoint{8.587584in}{1.635673in}}{\pgfqpoint{8.576985in}{1.631283in}}{\pgfqpoint{8.569171in}{1.623470in}}%
\pgfpathcurveto{\pgfqpoint{8.561357in}{1.615656in}}{\pgfqpoint{8.556967in}{1.605057in}}{\pgfqpoint{8.556967in}{1.594007in}}%
\pgfpathcurveto{\pgfqpoint{8.556967in}{1.582957in}}{\pgfqpoint{8.561357in}{1.572358in}}{\pgfqpoint{8.569171in}{1.564544in}}%
\pgfpathcurveto{\pgfqpoint{8.576985in}{1.556730in}}{\pgfqpoint{8.587584in}{1.552340in}}{\pgfqpoint{8.598634in}{1.552340in}}%
\pgfpathlineto{\pgfqpoint{8.598634in}{1.552340in}}%
\pgfpathclose%
\pgfusepath{stroke}%
\end{pgfscope}%
\begin{pgfscope}%
\pgfpathrectangle{\pgfqpoint{7.512535in}{0.437222in}}{\pgfqpoint{6.275590in}{5.159444in}}%
\pgfusepath{clip}%
\pgfsetbuttcap%
\pgfsetroundjoin%
\pgfsetlinewidth{1.003750pt}%
\definecolor{currentstroke}{rgb}{0.827451,0.827451,0.827451}%
\pgfsetstrokecolor{currentstroke}%
\pgfsetstrokeopacity{0.800000}%
\pgfsetdash{}{0pt}%
\pgfpathmoveto{\pgfqpoint{9.561548in}{1.679042in}}%
\pgfpathcurveto{\pgfqpoint{9.572598in}{1.679042in}}{\pgfqpoint{9.583197in}{1.683432in}}{\pgfqpoint{9.591011in}{1.691246in}}%
\pgfpathcurveto{\pgfqpoint{9.598825in}{1.699059in}}{\pgfqpoint{9.603215in}{1.709658in}}{\pgfqpoint{9.603215in}{1.720709in}}%
\pgfpathcurveto{\pgfqpoint{9.603215in}{1.731759in}}{\pgfqpoint{9.598825in}{1.742358in}}{\pgfqpoint{9.591011in}{1.750171in}}%
\pgfpathcurveto{\pgfqpoint{9.583197in}{1.757985in}}{\pgfqpoint{9.572598in}{1.762375in}}{\pgfqpoint{9.561548in}{1.762375in}}%
\pgfpathcurveto{\pgfqpoint{9.550498in}{1.762375in}}{\pgfqpoint{9.539899in}{1.757985in}}{\pgfqpoint{9.532086in}{1.750171in}}%
\pgfpathcurveto{\pgfqpoint{9.524272in}{1.742358in}}{\pgfqpoint{9.519882in}{1.731759in}}{\pgfqpoint{9.519882in}{1.720709in}}%
\pgfpathcurveto{\pgfqpoint{9.519882in}{1.709658in}}{\pgfqpoint{9.524272in}{1.699059in}}{\pgfqpoint{9.532086in}{1.691246in}}%
\pgfpathcurveto{\pgfqpoint{9.539899in}{1.683432in}}{\pgfqpoint{9.550498in}{1.679042in}}{\pgfqpoint{9.561548in}{1.679042in}}%
\pgfpathlineto{\pgfqpoint{9.561548in}{1.679042in}}%
\pgfpathclose%
\pgfusepath{stroke}%
\end{pgfscope}%
\begin{pgfscope}%
\pgfpathrectangle{\pgfqpoint{7.512535in}{0.437222in}}{\pgfqpoint{6.275590in}{5.159444in}}%
\pgfusepath{clip}%
\pgfsetbuttcap%
\pgfsetroundjoin%
\pgfsetlinewidth{1.003750pt}%
\definecolor{currentstroke}{rgb}{0.827451,0.827451,0.827451}%
\pgfsetstrokecolor{currentstroke}%
\pgfsetstrokeopacity{0.800000}%
\pgfsetdash{}{0pt}%
\pgfpathmoveto{\pgfqpoint{12.196637in}{5.403515in}}%
\pgfpathcurveto{\pgfqpoint{12.207687in}{5.403515in}}{\pgfqpoint{12.218287in}{5.407905in}}{\pgfqpoint{12.226100in}{5.415719in}}%
\pgfpathcurveto{\pgfqpoint{12.233914in}{5.423532in}}{\pgfqpoint{12.238304in}{5.434131in}}{\pgfqpoint{12.238304in}{5.445182in}}%
\pgfpathcurveto{\pgfqpoint{12.238304in}{5.456232in}}{\pgfqpoint{12.233914in}{5.466831in}}{\pgfqpoint{12.226100in}{5.474644in}}%
\pgfpathcurveto{\pgfqpoint{12.218287in}{5.482458in}}{\pgfqpoint{12.207687in}{5.486848in}}{\pgfqpoint{12.196637in}{5.486848in}}%
\pgfpathcurveto{\pgfqpoint{12.185587in}{5.486848in}}{\pgfqpoint{12.174988in}{5.482458in}}{\pgfqpoint{12.167175in}{5.474644in}}%
\pgfpathcurveto{\pgfqpoint{12.159361in}{5.466831in}}{\pgfqpoint{12.154971in}{5.456232in}}{\pgfqpoint{12.154971in}{5.445182in}}%
\pgfpathcurveto{\pgfqpoint{12.154971in}{5.434131in}}{\pgfqpoint{12.159361in}{5.423532in}}{\pgfqpoint{12.167175in}{5.415719in}}%
\pgfpathcurveto{\pgfqpoint{12.174988in}{5.407905in}}{\pgfqpoint{12.185587in}{5.403515in}}{\pgfqpoint{12.196637in}{5.403515in}}%
\pgfpathlineto{\pgfqpoint{12.196637in}{5.403515in}}%
\pgfpathclose%
\pgfusepath{stroke}%
\end{pgfscope}%
\begin{pgfscope}%
\pgfpathrectangle{\pgfqpoint{7.512535in}{0.437222in}}{\pgfqpoint{6.275590in}{5.159444in}}%
\pgfusepath{clip}%
\pgfsetbuttcap%
\pgfsetroundjoin%
\pgfsetlinewidth{1.003750pt}%
\definecolor{currentstroke}{rgb}{0.827451,0.827451,0.827451}%
\pgfsetstrokecolor{currentstroke}%
\pgfsetstrokeopacity{0.800000}%
\pgfsetdash{}{0pt}%
\pgfpathmoveto{\pgfqpoint{11.888148in}{5.414422in}}%
\pgfpathcurveto{\pgfqpoint{11.899198in}{5.414422in}}{\pgfqpoint{11.909797in}{5.418812in}}{\pgfqpoint{11.917611in}{5.426626in}}%
\pgfpathcurveto{\pgfqpoint{11.925424in}{5.434440in}}{\pgfqpoint{11.929815in}{5.445039in}}{\pgfqpoint{11.929815in}{5.456089in}}%
\pgfpathcurveto{\pgfqpoint{11.929815in}{5.467139in}}{\pgfqpoint{11.925424in}{5.477738in}}{\pgfqpoint{11.917611in}{5.485551in}}%
\pgfpathcurveto{\pgfqpoint{11.909797in}{5.493365in}}{\pgfqpoint{11.899198in}{5.497755in}}{\pgfqpoint{11.888148in}{5.497755in}}%
\pgfpathcurveto{\pgfqpoint{11.877098in}{5.497755in}}{\pgfqpoint{11.866499in}{5.493365in}}{\pgfqpoint{11.858685in}{5.485551in}}%
\pgfpathcurveto{\pgfqpoint{11.850872in}{5.477738in}}{\pgfqpoint{11.846481in}{5.467139in}}{\pgfqpoint{11.846481in}{5.456089in}}%
\pgfpathcurveto{\pgfqpoint{11.846481in}{5.445039in}}{\pgfqpoint{11.850872in}{5.434440in}}{\pgfqpoint{11.858685in}{5.426626in}}%
\pgfpathcurveto{\pgfqpoint{11.866499in}{5.418812in}}{\pgfqpoint{11.877098in}{5.414422in}}{\pgfqpoint{11.888148in}{5.414422in}}%
\pgfpathlineto{\pgfqpoint{11.888148in}{5.414422in}}%
\pgfpathclose%
\pgfusepath{stroke}%
\end{pgfscope}%
\begin{pgfscope}%
\pgfpathrectangle{\pgfqpoint{7.512535in}{0.437222in}}{\pgfqpoint{6.275590in}{5.159444in}}%
\pgfusepath{clip}%
\pgfsetbuttcap%
\pgfsetroundjoin%
\pgfsetlinewidth{1.003750pt}%
\definecolor{currentstroke}{rgb}{0.827451,0.827451,0.827451}%
\pgfsetstrokecolor{currentstroke}%
\pgfsetstrokeopacity{0.800000}%
\pgfsetdash{}{0pt}%
\pgfpathmoveto{\pgfqpoint{10.486066in}{4.589271in}}%
\pgfpathcurveto{\pgfqpoint{10.497116in}{4.589271in}}{\pgfqpoint{10.507715in}{4.593662in}}{\pgfqpoint{10.515529in}{4.601475in}}%
\pgfpathcurveto{\pgfqpoint{10.523342in}{4.609289in}}{\pgfqpoint{10.527733in}{4.619888in}}{\pgfqpoint{10.527733in}{4.630938in}}%
\pgfpathcurveto{\pgfqpoint{10.527733in}{4.641988in}}{\pgfqpoint{10.523342in}{4.652587in}}{\pgfqpoint{10.515529in}{4.660401in}}%
\pgfpathcurveto{\pgfqpoint{10.507715in}{4.668214in}}{\pgfqpoint{10.497116in}{4.672605in}}{\pgfqpoint{10.486066in}{4.672605in}}%
\pgfpathcurveto{\pgfqpoint{10.475016in}{4.672605in}}{\pgfqpoint{10.464417in}{4.668214in}}{\pgfqpoint{10.456603in}{4.660401in}}%
\pgfpathcurveto{\pgfqpoint{10.448790in}{4.652587in}}{\pgfqpoint{10.444399in}{4.641988in}}{\pgfqpoint{10.444399in}{4.630938in}}%
\pgfpathcurveto{\pgfqpoint{10.444399in}{4.619888in}}{\pgfqpoint{10.448790in}{4.609289in}}{\pgfqpoint{10.456603in}{4.601475in}}%
\pgfpathcurveto{\pgfqpoint{10.464417in}{4.593662in}}{\pgfqpoint{10.475016in}{4.589271in}}{\pgfqpoint{10.486066in}{4.589271in}}%
\pgfpathlineto{\pgfqpoint{10.486066in}{4.589271in}}%
\pgfpathclose%
\pgfusepath{stroke}%
\end{pgfscope}%
\begin{pgfscope}%
\pgfpathrectangle{\pgfqpoint{7.512535in}{0.437222in}}{\pgfqpoint{6.275590in}{5.159444in}}%
\pgfusepath{clip}%
\pgfsetbuttcap%
\pgfsetroundjoin%
\pgfsetlinewidth{1.003750pt}%
\definecolor{currentstroke}{rgb}{0.827451,0.827451,0.827451}%
\pgfsetstrokecolor{currentstroke}%
\pgfsetstrokeopacity{0.800000}%
\pgfsetdash{}{0pt}%
\pgfpathmoveto{\pgfqpoint{8.716039in}{3.385314in}}%
\pgfpathcurveto{\pgfqpoint{8.727089in}{3.385314in}}{\pgfqpoint{8.737688in}{3.389704in}}{\pgfqpoint{8.745502in}{3.397518in}}%
\pgfpathcurveto{\pgfqpoint{8.753315in}{3.405332in}}{\pgfqpoint{8.757706in}{3.415931in}}{\pgfqpoint{8.757706in}{3.426981in}}%
\pgfpathcurveto{\pgfqpoint{8.757706in}{3.438031in}}{\pgfqpoint{8.753315in}{3.448630in}}{\pgfqpoint{8.745502in}{3.456444in}}%
\pgfpathcurveto{\pgfqpoint{8.737688in}{3.464257in}}{\pgfqpoint{8.727089in}{3.468647in}}{\pgfqpoint{8.716039in}{3.468647in}}%
\pgfpathcurveto{\pgfqpoint{8.704989in}{3.468647in}}{\pgfqpoint{8.694390in}{3.464257in}}{\pgfqpoint{8.686576in}{3.456444in}}%
\pgfpathcurveto{\pgfqpoint{8.678763in}{3.448630in}}{\pgfqpoint{8.674372in}{3.438031in}}{\pgfqpoint{8.674372in}{3.426981in}}%
\pgfpathcurveto{\pgfqpoint{8.674372in}{3.415931in}}{\pgfqpoint{8.678763in}{3.405332in}}{\pgfqpoint{8.686576in}{3.397518in}}%
\pgfpathcurveto{\pgfqpoint{8.694390in}{3.389704in}}{\pgfqpoint{8.704989in}{3.385314in}}{\pgfqpoint{8.716039in}{3.385314in}}%
\pgfpathlineto{\pgfqpoint{8.716039in}{3.385314in}}%
\pgfpathclose%
\pgfusepath{stroke}%
\end{pgfscope}%
\begin{pgfscope}%
\pgfpathrectangle{\pgfqpoint{7.512535in}{0.437222in}}{\pgfqpoint{6.275590in}{5.159444in}}%
\pgfusepath{clip}%
\pgfsetbuttcap%
\pgfsetroundjoin%
\pgfsetlinewidth{1.003750pt}%
\definecolor{currentstroke}{rgb}{0.827451,0.827451,0.827451}%
\pgfsetstrokecolor{currentstroke}%
\pgfsetstrokeopacity{0.800000}%
\pgfsetdash{}{0pt}%
\pgfpathmoveto{\pgfqpoint{9.395327in}{2.124368in}}%
\pgfpathcurveto{\pgfqpoint{9.406377in}{2.124368in}}{\pgfqpoint{9.416976in}{2.128758in}}{\pgfqpoint{9.424790in}{2.136572in}}%
\pgfpathcurveto{\pgfqpoint{9.432604in}{2.144385in}}{\pgfqpoint{9.436994in}{2.154984in}}{\pgfqpoint{9.436994in}{2.166034in}}%
\pgfpathcurveto{\pgfqpoint{9.436994in}{2.177084in}}{\pgfqpoint{9.432604in}{2.187684in}}{\pgfqpoint{9.424790in}{2.195497in}}%
\pgfpathcurveto{\pgfqpoint{9.416976in}{2.203311in}}{\pgfqpoint{9.406377in}{2.207701in}}{\pgfqpoint{9.395327in}{2.207701in}}%
\pgfpathcurveto{\pgfqpoint{9.384277in}{2.207701in}}{\pgfqpoint{9.373678in}{2.203311in}}{\pgfqpoint{9.365865in}{2.195497in}}%
\pgfpathcurveto{\pgfqpoint{9.358051in}{2.187684in}}{\pgfqpoint{9.353661in}{2.177084in}}{\pgfqpoint{9.353661in}{2.166034in}}%
\pgfpathcurveto{\pgfqpoint{9.353661in}{2.154984in}}{\pgfqpoint{9.358051in}{2.144385in}}{\pgfqpoint{9.365865in}{2.136572in}}%
\pgfpathcurveto{\pgfqpoint{9.373678in}{2.128758in}}{\pgfqpoint{9.384277in}{2.124368in}}{\pgfqpoint{9.395327in}{2.124368in}}%
\pgfpathlineto{\pgfqpoint{9.395327in}{2.124368in}}%
\pgfpathclose%
\pgfusepath{stroke}%
\end{pgfscope}%
\begin{pgfscope}%
\pgfpathrectangle{\pgfqpoint{7.512535in}{0.437222in}}{\pgfqpoint{6.275590in}{5.159444in}}%
\pgfusepath{clip}%
\pgfsetbuttcap%
\pgfsetroundjoin%
\pgfsetlinewidth{1.003750pt}%
\definecolor{currentstroke}{rgb}{0.827451,0.827451,0.827451}%
\pgfsetstrokecolor{currentstroke}%
\pgfsetstrokeopacity{0.800000}%
\pgfsetdash{}{0pt}%
\pgfpathmoveto{\pgfqpoint{11.026069in}{5.528063in}}%
\pgfpathcurveto{\pgfqpoint{11.037120in}{5.528063in}}{\pgfqpoint{11.047719in}{5.532453in}}{\pgfqpoint{11.055532in}{5.540267in}}%
\pgfpathcurveto{\pgfqpoint{11.063346in}{5.548081in}}{\pgfqpoint{11.067736in}{5.558680in}}{\pgfqpoint{11.067736in}{5.569730in}}%
\pgfpathcurveto{\pgfqpoint{11.067736in}{5.580780in}}{\pgfqpoint{11.063346in}{5.591379in}}{\pgfqpoint{11.055532in}{5.599193in}}%
\pgfpathcurveto{\pgfqpoint{11.047719in}{5.607006in}}{\pgfqpoint{11.037120in}{5.611396in}}{\pgfqpoint{11.026069in}{5.611396in}}%
\pgfpathcurveto{\pgfqpoint{11.015019in}{5.611396in}}{\pgfqpoint{11.004420in}{5.607006in}}{\pgfqpoint{10.996607in}{5.599193in}}%
\pgfpathcurveto{\pgfqpoint{10.988793in}{5.591379in}}{\pgfqpoint{10.984403in}{5.580780in}}{\pgfqpoint{10.984403in}{5.569730in}}%
\pgfpathcurveto{\pgfqpoint{10.984403in}{5.558680in}}{\pgfqpoint{10.988793in}{5.548081in}}{\pgfqpoint{10.996607in}{5.540267in}}%
\pgfpathcurveto{\pgfqpoint{11.004420in}{5.532453in}}{\pgfqpoint{11.015019in}{5.528063in}}{\pgfqpoint{11.026069in}{5.528063in}}%
\pgfpathlineto{\pgfqpoint{11.026069in}{5.528063in}}%
\pgfpathclose%
\pgfusepath{stroke}%
\end{pgfscope}%
\begin{pgfscope}%
\pgfpathrectangle{\pgfqpoint{7.512535in}{0.437222in}}{\pgfqpoint{6.275590in}{5.159444in}}%
\pgfusepath{clip}%
\pgfsetbuttcap%
\pgfsetroundjoin%
\pgfsetlinewidth{1.003750pt}%
\definecolor{currentstroke}{rgb}{0.827451,0.827451,0.827451}%
\pgfsetstrokecolor{currentstroke}%
\pgfsetstrokeopacity{0.800000}%
\pgfsetdash{}{0pt}%
\pgfpathmoveto{\pgfqpoint{8.983729in}{3.220234in}}%
\pgfpathcurveto{\pgfqpoint{8.994779in}{3.220234in}}{\pgfqpoint{9.005378in}{3.224624in}}{\pgfqpoint{9.013191in}{3.232438in}}%
\pgfpathcurveto{\pgfqpoint{9.021005in}{3.240251in}}{\pgfqpoint{9.025395in}{3.250850in}}{\pgfqpoint{9.025395in}{3.261901in}}%
\pgfpathcurveto{\pgfqpoint{9.025395in}{3.272951in}}{\pgfqpoint{9.021005in}{3.283550in}}{\pgfqpoint{9.013191in}{3.291363in}}%
\pgfpathcurveto{\pgfqpoint{9.005378in}{3.299177in}}{\pgfqpoint{8.994779in}{3.303567in}}{\pgfqpoint{8.983729in}{3.303567in}}%
\pgfpathcurveto{\pgfqpoint{8.972679in}{3.303567in}}{\pgfqpoint{8.962079in}{3.299177in}}{\pgfqpoint{8.954266in}{3.291363in}}%
\pgfpathcurveto{\pgfqpoint{8.946452in}{3.283550in}}{\pgfqpoint{8.942062in}{3.272951in}}{\pgfqpoint{8.942062in}{3.261901in}}%
\pgfpathcurveto{\pgfqpoint{8.942062in}{3.250850in}}{\pgfqpoint{8.946452in}{3.240251in}}{\pgfqpoint{8.954266in}{3.232438in}}%
\pgfpathcurveto{\pgfqpoint{8.962079in}{3.224624in}}{\pgfqpoint{8.972679in}{3.220234in}}{\pgfqpoint{8.983729in}{3.220234in}}%
\pgfpathlineto{\pgfqpoint{8.983729in}{3.220234in}}%
\pgfpathclose%
\pgfusepath{stroke}%
\end{pgfscope}%
\begin{pgfscope}%
\pgfpathrectangle{\pgfqpoint{7.512535in}{0.437222in}}{\pgfqpoint{6.275590in}{5.159444in}}%
\pgfusepath{clip}%
\pgfsetbuttcap%
\pgfsetroundjoin%
\pgfsetlinewidth{1.003750pt}%
\definecolor{currentstroke}{rgb}{0.827451,0.827451,0.827451}%
\pgfsetstrokecolor{currentstroke}%
\pgfsetstrokeopacity{0.800000}%
\pgfsetdash{}{0pt}%
\pgfpathmoveto{\pgfqpoint{11.673162in}{5.362410in}}%
\pgfpathcurveto{\pgfqpoint{11.684212in}{5.362410in}}{\pgfqpoint{11.694811in}{5.366800in}}{\pgfqpoint{11.702624in}{5.374614in}}%
\pgfpathcurveto{\pgfqpoint{11.710438in}{5.382427in}}{\pgfqpoint{11.714828in}{5.393027in}}{\pgfqpoint{11.714828in}{5.404077in}}%
\pgfpathcurveto{\pgfqpoint{11.714828in}{5.415127in}}{\pgfqpoint{11.710438in}{5.425726in}}{\pgfqpoint{11.702624in}{5.433539in}}%
\pgfpathcurveto{\pgfqpoint{11.694811in}{5.441353in}}{\pgfqpoint{11.684212in}{5.445743in}}{\pgfqpoint{11.673162in}{5.445743in}}%
\pgfpathcurveto{\pgfqpoint{11.662111in}{5.445743in}}{\pgfqpoint{11.651512in}{5.441353in}}{\pgfqpoint{11.643699in}{5.433539in}}%
\pgfpathcurveto{\pgfqpoint{11.635885in}{5.425726in}}{\pgfqpoint{11.631495in}{5.415127in}}{\pgfqpoint{11.631495in}{5.404077in}}%
\pgfpathcurveto{\pgfqpoint{11.631495in}{5.393027in}}{\pgfqpoint{11.635885in}{5.382427in}}{\pgfqpoint{11.643699in}{5.374614in}}%
\pgfpathcurveto{\pgfqpoint{11.651512in}{5.366800in}}{\pgfqpoint{11.662111in}{5.362410in}}{\pgfqpoint{11.673162in}{5.362410in}}%
\pgfpathlineto{\pgfqpoint{11.673162in}{5.362410in}}%
\pgfpathclose%
\pgfusepath{stroke}%
\end{pgfscope}%
\begin{pgfscope}%
\pgfpathrectangle{\pgfqpoint{7.512535in}{0.437222in}}{\pgfqpoint{6.275590in}{5.159444in}}%
\pgfusepath{clip}%
\pgfsetbuttcap%
\pgfsetroundjoin%
\pgfsetlinewidth{1.003750pt}%
\definecolor{currentstroke}{rgb}{0.827451,0.827451,0.827451}%
\pgfsetstrokecolor{currentstroke}%
\pgfsetstrokeopacity{0.800000}%
\pgfsetdash{}{0pt}%
\pgfpathmoveto{\pgfqpoint{12.737963in}{5.481847in}}%
\pgfpathcurveto{\pgfqpoint{12.749014in}{5.481847in}}{\pgfqpoint{12.759613in}{5.486238in}}{\pgfqpoint{12.767426in}{5.494051in}}%
\pgfpathcurveto{\pgfqpoint{12.775240in}{5.501865in}}{\pgfqpoint{12.779630in}{5.512464in}}{\pgfqpoint{12.779630in}{5.523514in}}%
\pgfpathcurveto{\pgfqpoint{12.779630in}{5.534564in}}{\pgfqpoint{12.775240in}{5.545163in}}{\pgfqpoint{12.767426in}{5.552977in}}%
\pgfpathcurveto{\pgfqpoint{12.759613in}{5.560791in}}{\pgfqpoint{12.749014in}{5.565181in}}{\pgfqpoint{12.737963in}{5.565181in}}%
\pgfpathcurveto{\pgfqpoint{12.726913in}{5.565181in}}{\pgfqpoint{12.716314in}{5.560791in}}{\pgfqpoint{12.708501in}{5.552977in}}%
\pgfpathcurveto{\pgfqpoint{12.700687in}{5.545163in}}{\pgfqpoint{12.696297in}{5.534564in}}{\pgfqpoint{12.696297in}{5.523514in}}%
\pgfpathcurveto{\pgfqpoint{12.696297in}{5.512464in}}{\pgfqpoint{12.700687in}{5.501865in}}{\pgfqpoint{12.708501in}{5.494051in}}%
\pgfpathcurveto{\pgfqpoint{12.716314in}{5.486238in}}{\pgfqpoint{12.726913in}{5.481847in}}{\pgfqpoint{12.737963in}{5.481847in}}%
\pgfpathlineto{\pgfqpoint{12.737963in}{5.481847in}}%
\pgfpathclose%
\pgfusepath{stroke}%
\end{pgfscope}%
\begin{pgfscope}%
\pgfpathrectangle{\pgfqpoint{7.512535in}{0.437222in}}{\pgfqpoint{6.275590in}{5.159444in}}%
\pgfusepath{clip}%
\pgfsetbuttcap%
\pgfsetroundjoin%
\pgfsetlinewidth{1.003750pt}%
\definecolor{currentstroke}{rgb}{0.827451,0.827451,0.827451}%
\pgfsetstrokecolor{currentstroke}%
\pgfsetstrokeopacity{0.800000}%
\pgfsetdash{}{0pt}%
\pgfpathmoveto{\pgfqpoint{8.680117in}{3.344222in}}%
\pgfpathcurveto{\pgfqpoint{8.691167in}{3.344222in}}{\pgfqpoint{8.701766in}{3.348612in}}{\pgfqpoint{8.709580in}{3.356426in}}%
\pgfpathcurveto{\pgfqpoint{8.717393in}{3.364239in}}{\pgfqpoint{8.721784in}{3.374838in}}{\pgfqpoint{8.721784in}{3.385889in}}%
\pgfpathcurveto{\pgfqpoint{8.721784in}{3.396939in}}{\pgfqpoint{8.717393in}{3.407538in}}{\pgfqpoint{8.709580in}{3.415351in}}%
\pgfpathcurveto{\pgfqpoint{8.701766in}{3.423165in}}{\pgfqpoint{8.691167in}{3.427555in}}{\pgfqpoint{8.680117in}{3.427555in}}%
\pgfpathcurveto{\pgfqpoint{8.669067in}{3.427555in}}{\pgfqpoint{8.658468in}{3.423165in}}{\pgfqpoint{8.650654in}{3.415351in}}%
\pgfpathcurveto{\pgfqpoint{8.642841in}{3.407538in}}{\pgfqpoint{8.638450in}{3.396939in}}{\pgfqpoint{8.638450in}{3.385889in}}%
\pgfpathcurveto{\pgfqpoint{8.638450in}{3.374838in}}{\pgfqpoint{8.642841in}{3.364239in}}{\pgfqpoint{8.650654in}{3.356426in}}%
\pgfpathcurveto{\pgfqpoint{8.658468in}{3.348612in}}{\pgfqpoint{8.669067in}{3.344222in}}{\pgfqpoint{8.680117in}{3.344222in}}%
\pgfpathlineto{\pgfqpoint{8.680117in}{3.344222in}}%
\pgfpathclose%
\pgfusepath{stroke}%
\end{pgfscope}%
\begin{pgfscope}%
\pgfpathrectangle{\pgfqpoint{7.512535in}{0.437222in}}{\pgfqpoint{6.275590in}{5.159444in}}%
\pgfusepath{clip}%
\pgfsetbuttcap%
\pgfsetroundjoin%
\pgfsetlinewidth{1.003750pt}%
\definecolor{currentstroke}{rgb}{0.827451,0.827451,0.827451}%
\pgfsetstrokecolor{currentstroke}%
\pgfsetstrokeopacity{0.800000}%
\pgfsetdash{}{0pt}%
\pgfpathmoveto{\pgfqpoint{8.290348in}{1.011495in}}%
\pgfpathcurveto{\pgfqpoint{8.301398in}{1.011495in}}{\pgfqpoint{8.311997in}{1.015885in}}{\pgfqpoint{8.319811in}{1.023699in}}%
\pgfpathcurveto{\pgfqpoint{8.327625in}{1.031512in}}{\pgfqpoint{8.332015in}{1.042111in}}{\pgfqpoint{8.332015in}{1.053161in}}%
\pgfpathcurveto{\pgfqpoint{8.332015in}{1.064211in}}{\pgfqpoint{8.327625in}{1.074810in}}{\pgfqpoint{8.319811in}{1.082624in}}%
\pgfpathcurveto{\pgfqpoint{8.311997in}{1.090438in}}{\pgfqpoint{8.301398in}{1.094828in}}{\pgfqpoint{8.290348in}{1.094828in}}%
\pgfpathcurveto{\pgfqpoint{8.279298in}{1.094828in}}{\pgfqpoint{8.268699in}{1.090438in}}{\pgfqpoint{8.260886in}{1.082624in}}%
\pgfpathcurveto{\pgfqpoint{8.253072in}{1.074810in}}{\pgfqpoint{8.248682in}{1.064211in}}{\pgfqpoint{8.248682in}{1.053161in}}%
\pgfpathcurveto{\pgfqpoint{8.248682in}{1.042111in}}{\pgfqpoint{8.253072in}{1.031512in}}{\pgfqpoint{8.260886in}{1.023699in}}%
\pgfpathcurveto{\pgfqpoint{8.268699in}{1.015885in}}{\pgfqpoint{8.279298in}{1.011495in}}{\pgfqpoint{8.290348in}{1.011495in}}%
\pgfpathlineto{\pgfqpoint{8.290348in}{1.011495in}}%
\pgfpathclose%
\pgfusepath{stroke}%
\end{pgfscope}%
\begin{pgfscope}%
\pgfpathrectangle{\pgfqpoint{7.512535in}{0.437222in}}{\pgfqpoint{6.275590in}{5.159444in}}%
\pgfusepath{clip}%
\pgfsetbuttcap%
\pgfsetroundjoin%
\pgfsetlinewidth{1.003750pt}%
\definecolor{currentstroke}{rgb}{0.827451,0.827451,0.827451}%
\pgfsetstrokecolor{currentstroke}%
\pgfsetstrokeopacity{0.800000}%
\pgfsetdash{}{0pt}%
\pgfpathmoveto{\pgfqpoint{10.369675in}{4.552452in}}%
\pgfpathcurveto{\pgfqpoint{10.380725in}{4.552452in}}{\pgfqpoint{10.391324in}{4.556842in}}{\pgfqpoint{10.399138in}{4.564656in}}%
\pgfpathcurveto{\pgfqpoint{10.406951in}{4.572470in}}{\pgfqpoint{10.411341in}{4.583069in}}{\pgfqpoint{10.411341in}{4.594119in}}%
\pgfpathcurveto{\pgfqpoint{10.411341in}{4.605169in}}{\pgfqpoint{10.406951in}{4.615768in}}{\pgfqpoint{10.399138in}{4.623582in}}%
\pgfpathcurveto{\pgfqpoint{10.391324in}{4.631395in}}{\pgfqpoint{10.380725in}{4.635785in}}{\pgfqpoint{10.369675in}{4.635785in}}%
\pgfpathcurveto{\pgfqpoint{10.358625in}{4.635785in}}{\pgfqpoint{10.348026in}{4.631395in}}{\pgfqpoint{10.340212in}{4.623582in}}%
\pgfpathcurveto{\pgfqpoint{10.332398in}{4.615768in}}{\pgfqpoint{10.328008in}{4.605169in}}{\pgfqpoint{10.328008in}{4.594119in}}%
\pgfpathcurveto{\pgfqpoint{10.328008in}{4.583069in}}{\pgfqpoint{10.332398in}{4.572470in}}{\pgfqpoint{10.340212in}{4.564656in}}%
\pgfpathcurveto{\pgfqpoint{10.348026in}{4.556842in}}{\pgfqpoint{10.358625in}{4.552452in}}{\pgfqpoint{10.369675in}{4.552452in}}%
\pgfpathlineto{\pgfqpoint{10.369675in}{4.552452in}}%
\pgfpathclose%
\pgfusepath{stroke}%
\end{pgfscope}%
\begin{pgfscope}%
\pgfpathrectangle{\pgfqpoint{7.512535in}{0.437222in}}{\pgfqpoint{6.275590in}{5.159444in}}%
\pgfusepath{clip}%
\pgfsetbuttcap%
\pgfsetroundjoin%
\pgfsetlinewidth{1.003750pt}%
\definecolor{currentstroke}{rgb}{0.827451,0.827451,0.827451}%
\pgfsetstrokecolor{currentstroke}%
\pgfsetstrokeopacity{0.800000}%
\pgfsetdash{}{0pt}%
\pgfpathmoveto{\pgfqpoint{9.140923in}{2.556537in}}%
\pgfpathcurveto{\pgfqpoint{9.151974in}{2.556537in}}{\pgfqpoint{9.162573in}{2.560927in}}{\pgfqpoint{9.170386in}{2.568740in}}%
\pgfpathcurveto{\pgfqpoint{9.178200in}{2.576554in}}{\pgfqpoint{9.182590in}{2.587153in}}{\pgfqpoint{9.182590in}{2.598203in}}%
\pgfpathcurveto{\pgfqpoint{9.182590in}{2.609253in}}{\pgfqpoint{9.178200in}{2.619852in}}{\pgfqpoint{9.170386in}{2.627666in}}%
\pgfpathcurveto{\pgfqpoint{9.162573in}{2.635480in}}{\pgfqpoint{9.151974in}{2.639870in}}{\pgfqpoint{9.140923in}{2.639870in}}%
\pgfpathcurveto{\pgfqpoint{9.129873in}{2.639870in}}{\pgfqpoint{9.119274in}{2.635480in}}{\pgfqpoint{9.111461in}{2.627666in}}%
\pgfpathcurveto{\pgfqpoint{9.103647in}{2.619852in}}{\pgfqpoint{9.099257in}{2.609253in}}{\pgfqpoint{9.099257in}{2.598203in}}%
\pgfpathcurveto{\pgfqpoint{9.099257in}{2.587153in}}{\pgfqpoint{9.103647in}{2.576554in}}{\pgfqpoint{9.111461in}{2.568740in}}%
\pgfpathcurveto{\pgfqpoint{9.119274in}{2.560927in}}{\pgfqpoint{9.129873in}{2.556537in}}{\pgfqpoint{9.140923in}{2.556537in}}%
\pgfpathlineto{\pgfqpoint{9.140923in}{2.556537in}}%
\pgfpathclose%
\pgfusepath{stroke}%
\end{pgfscope}%
\begin{pgfscope}%
\pgfpathrectangle{\pgfqpoint{7.512535in}{0.437222in}}{\pgfqpoint{6.275590in}{5.159444in}}%
\pgfusepath{clip}%
\pgfsetbuttcap%
\pgfsetroundjoin%
\pgfsetlinewidth{1.003750pt}%
\definecolor{currentstroke}{rgb}{0.827451,0.827451,0.827451}%
\pgfsetstrokecolor{currentstroke}%
\pgfsetstrokeopacity{0.800000}%
\pgfsetdash{}{0pt}%
\pgfpathmoveto{\pgfqpoint{7.887873in}{1.346375in}}%
\pgfpathcurveto{\pgfqpoint{7.898923in}{1.346375in}}{\pgfqpoint{7.909522in}{1.350765in}}{\pgfqpoint{7.917336in}{1.358578in}}%
\pgfpathcurveto{\pgfqpoint{7.925149in}{1.366392in}}{\pgfqpoint{7.929540in}{1.376991in}}{\pgfqpoint{7.929540in}{1.388041in}}%
\pgfpathcurveto{\pgfqpoint{7.929540in}{1.399091in}}{\pgfqpoint{7.925149in}{1.409690in}}{\pgfqpoint{7.917336in}{1.417504in}}%
\pgfpathcurveto{\pgfqpoint{7.909522in}{1.425318in}}{\pgfqpoint{7.898923in}{1.429708in}}{\pgfqpoint{7.887873in}{1.429708in}}%
\pgfpathcurveto{\pgfqpoint{7.876823in}{1.429708in}}{\pgfqpoint{7.866224in}{1.425318in}}{\pgfqpoint{7.858410in}{1.417504in}}%
\pgfpathcurveto{\pgfqpoint{7.850596in}{1.409690in}}{\pgfqpoint{7.846206in}{1.399091in}}{\pgfqpoint{7.846206in}{1.388041in}}%
\pgfpathcurveto{\pgfqpoint{7.846206in}{1.376991in}}{\pgfqpoint{7.850596in}{1.366392in}}{\pgfqpoint{7.858410in}{1.358578in}}%
\pgfpathcurveto{\pgfqpoint{7.866224in}{1.350765in}}{\pgfqpoint{7.876823in}{1.346375in}}{\pgfqpoint{7.887873in}{1.346375in}}%
\pgfpathlineto{\pgfqpoint{7.887873in}{1.346375in}}%
\pgfpathclose%
\pgfusepath{stroke}%
\end{pgfscope}%
\begin{pgfscope}%
\pgfpathrectangle{\pgfqpoint{7.512535in}{0.437222in}}{\pgfqpoint{6.275590in}{5.159444in}}%
\pgfusepath{clip}%
\pgfsetbuttcap%
\pgfsetroundjoin%
\pgfsetlinewidth{1.003750pt}%
\definecolor{currentstroke}{rgb}{0.827451,0.827451,0.827451}%
\pgfsetstrokecolor{currentstroke}%
\pgfsetstrokeopacity{0.800000}%
\pgfsetdash{}{0pt}%
\pgfpathmoveto{\pgfqpoint{8.775643in}{1.745625in}}%
\pgfpathcurveto{\pgfqpoint{8.786693in}{1.745625in}}{\pgfqpoint{8.797292in}{1.750015in}}{\pgfqpoint{8.805106in}{1.757829in}}%
\pgfpathcurveto{\pgfqpoint{8.812919in}{1.765643in}}{\pgfqpoint{8.817310in}{1.776242in}}{\pgfqpoint{8.817310in}{1.787292in}}%
\pgfpathcurveto{\pgfqpoint{8.817310in}{1.798342in}}{\pgfqpoint{8.812919in}{1.808941in}}{\pgfqpoint{8.805106in}{1.816755in}}%
\pgfpathcurveto{\pgfqpoint{8.797292in}{1.824568in}}{\pgfqpoint{8.786693in}{1.828959in}}{\pgfqpoint{8.775643in}{1.828959in}}%
\pgfpathcurveto{\pgfqpoint{8.764593in}{1.828959in}}{\pgfqpoint{8.753994in}{1.824568in}}{\pgfqpoint{8.746180in}{1.816755in}}%
\pgfpathcurveto{\pgfqpoint{8.738367in}{1.808941in}}{\pgfqpoint{8.733976in}{1.798342in}}{\pgfqpoint{8.733976in}{1.787292in}}%
\pgfpathcurveto{\pgfqpoint{8.733976in}{1.776242in}}{\pgfqpoint{8.738367in}{1.765643in}}{\pgfqpoint{8.746180in}{1.757829in}}%
\pgfpathcurveto{\pgfqpoint{8.753994in}{1.750015in}}{\pgfqpoint{8.764593in}{1.745625in}}{\pgfqpoint{8.775643in}{1.745625in}}%
\pgfpathlineto{\pgfqpoint{8.775643in}{1.745625in}}%
\pgfpathclose%
\pgfusepath{stroke}%
\end{pgfscope}%
\begin{pgfscope}%
\pgfpathrectangle{\pgfqpoint{7.512535in}{0.437222in}}{\pgfqpoint{6.275590in}{5.159444in}}%
\pgfusepath{clip}%
\pgfsetbuttcap%
\pgfsetroundjoin%
\pgfsetlinewidth{1.003750pt}%
\definecolor{currentstroke}{rgb}{0.827451,0.827451,0.827451}%
\pgfsetstrokecolor{currentstroke}%
\pgfsetstrokeopacity{0.800000}%
\pgfsetdash{}{0pt}%
\pgfpathmoveto{\pgfqpoint{10.259535in}{4.447570in}}%
\pgfpathcurveto{\pgfqpoint{10.270585in}{4.447570in}}{\pgfqpoint{10.281184in}{4.451960in}}{\pgfqpoint{10.288997in}{4.459773in}}%
\pgfpathcurveto{\pgfqpoint{10.296811in}{4.467587in}}{\pgfqpoint{10.301201in}{4.478186in}}{\pgfqpoint{10.301201in}{4.489236in}}%
\pgfpathcurveto{\pgfqpoint{10.301201in}{4.500286in}}{\pgfqpoint{10.296811in}{4.510885in}}{\pgfqpoint{10.288997in}{4.518699in}}%
\pgfpathcurveto{\pgfqpoint{10.281184in}{4.526513in}}{\pgfqpoint{10.270585in}{4.530903in}}{\pgfqpoint{10.259535in}{4.530903in}}%
\pgfpathcurveto{\pgfqpoint{10.248485in}{4.530903in}}{\pgfqpoint{10.237886in}{4.526513in}}{\pgfqpoint{10.230072in}{4.518699in}}%
\pgfpathcurveto{\pgfqpoint{10.222258in}{4.510885in}}{\pgfqpoint{10.217868in}{4.500286in}}{\pgfqpoint{10.217868in}{4.489236in}}%
\pgfpathcurveto{\pgfqpoint{10.217868in}{4.478186in}}{\pgfqpoint{10.222258in}{4.467587in}}{\pgfqpoint{10.230072in}{4.459773in}}%
\pgfpathcurveto{\pgfqpoint{10.237886in}{4.451960in}}{\pgfqpoint{10.248485in}{4.447570in}}{\pgfqpoint{10.259535in}{4.447570in}}%
\pgfpathlineto{\pgfqpoint{10.259535in}{4.447570in}}%
\pgfpathclose%
\pgfusepath{stroke}%
\end{pgfscope}%
\begin{pgfscope}%
\pgfpathrectangle{\pgfqpoint{7.512535in}{0.437222in}}{\pgfqpoint{6.275590in}{5.159444in}}%
\pgfusepath{clip}%
\pgfsetbuttcap%
\pgfsetroundjoin%
\pgfsetlinewidth{1.003750pt}%
\definecolor{currentstroke}{rgb}{0.827451,0.827451,0.827451}%
\pgfsetstrokecolor{currentstroke}%
\pgfsetstrokeopacity{0.800000}%
\pgfsetdash{}{0pt}%
\pgfpathmoveto{\pgfqpoint{10.780511in}{5.040147in}}%
\pgfpathcurveto{\pgfqpoint{10.791561in}{5.040147in}}{\pgfqpoint{10.802160in}{5.044537in}}{\pgfqpoint{10.809973in}{5.052350in}}%
\pgfpathcurveto{\pgfqpoint{10.817787in}{5.060164in}}{\pgfqpoint{10.822177in}{5.070763in}}{\pgfqpoint{10.822177in}{5.081813in}}%
\pgfpathcurveto{\pgfqpoint{10.822177in}{5.092863in}}{\pgfqpoint{10.817787in}{5.103462in}}{\pgfqpoint{10.809973in}{5.111276in}}%
\pgfpathcurveto{\pgfqpoint{10.802160in}{5.119090in}}{\pgfqpoint{10.791561in}{5.123480in}}{\pgfqpoint{10.780511in}{5.123480in}}%
\pgfpathcurveto{\pgfqpoint{10.769461in}{5.123480in}}{\pgfqpoint{10.758862in}{5.119090in}}{\pgfqpoint{10.751048in}{5.111276in}}%
\pgfpathcurveto{\pgfqpoint{10.743234in}{5.103462in}}{\pgfqpoint{10.738844in}{5.092863in}}{\pgfqpoint{10.738844in}{5.081813in}}%
\pgfpathcurveto{\pgfqpoint{10.738844in}{5.070763in}}{\pgfqpoint{10.743234in}{5.060164in}}{\pgfqpoint{10.751048in}{5.052350in}}%
\pgfpathcurveto{\pgfqpoint{10.758862in}{5.044537in}}{\pgfqpoint{10.769461in}{5.040147in}}{\pgfqpoint{10.780511in}{5.040147in}}%
\pgfpathlineto{\pgfqpoint{10.780511in}{5.040147in}}%
\pgfpathclose%
\pgfusepath{stroke}%
\end{pgfscope}%
\begin{pgfscope}%
\pgfpathrectangle{\pgfqpoint{7.512535in}{0.437222in}}{\pgfqpoint{6.275590in}{5.159444in}}%
\pgfusepath{clip}%
\pgfsetbuttcap%
\pgfsetroundjoin%
\pgfsetlinewidth{1.003750pt}%
\definecolor{currentstroke}{rgb}{0.827451,0.827451,0.827451}%
\pgfsetstrokecolor{currentstroke}%
\pgfsetstrokeopacity{0.800000}%
\pgfsetdash{}{0pt}%
\pgfpathmoveto{\pgfqpoint{12.495601in}{5.523529in}}%
\pgfpathcurveto{\pgfqpoint{12.506651in}{5.523529in}}{\pgfqpoint{12.517250in}{5.527919in}}{\pgfqpoint{12.525063in}{5.535733in}}%
\pgfpathcurveto{\pgfqpoint{12.532877in}{5.543547in}}{\pgfqpoint{12.537267in}{5.554146in}}{\pgfqpoint{12.537267in}{5.565196in}}%
\pgfpathcurveto{\pgfqpoint{12.537267in}{5.576246in}}{\pgfqpoint{12.532877in}{5.586845in}}{\pgfqpoint{12.525063in}{5.594658in}}%
\pgfpathcurveto{\pgfqpoint{12.517250in}{5.602472in}}{\pgfqpoint{12.506651in}{5.606862in}}{\pgfqpoint{12.495601in}{5.606862in}}%
\pgfpathcurveto{\pgfqpoint{12.484550in}{5.606862in}}{\pgfqpoint{12.473951in}{5.602472in}}{\pgfqpoint{12.466138in}{5.594658in}}%
\pgfpathcurveto{\pgfqpoint{12.458324in}{5.586845in}}{\pgfqpoint{12.453934in}{5.576246in}}{\pgfqpoint{12.453934in}{5.565196in}}%
\pgfpathcurveto{\pgfqpoint{12.453934in}{5.554146in}}{\pgfqpoint{12.458324in}{5.543547in}}{\pgfqpoint{12.466138in}{5.535733in}}%
\pgfpathcurveto{\pgfqpoint{12.473951in}{5.527919in}}{\pgfqpoint{12.484550in}{5.523529in}}{\pgfqpoint{12.495601in}{5.523529in}}%
\pgfpathlineto{\pgfqpoint{12.495601in}{5.523529in}}%
\pgfpathclose%
\pgfusepath{stroke}%
\end{pgfscope}%
\begin{pgfscope}%
\pgfpathrectangle{\pgfqpoint{7.512535in}{0.437222in}}{\pgfqpoint{6.275590in}{5.159444in}}%
\pgfusepath{clip}%
\pgfsetbuttcap%
\pgfsetroundjoin%
\pgfsetlinewidth{1.003750pt}%
\definecolor{currentstroke}{rgb}{0.827451,0.827451,0.827451}%
\pgfsetstrokecolor{currentstroke}%
\pgfsetstrokeopacity{0.800000}%
\pgfsetdash{}{0pt}%
\pgfpathmoveto{\pgfqpoint{8.519449in}{1.061557in}}%
\pgfpathcurveto{\pgfqpoint{8.530499in}{1.061557in}}{\pgfqpoint{8.541099in}{1.065948in}}{\pgfqpoint{8.548912in}{1.073761in}}%
\pgfpathcurveto{\pgfqpoint{8.556726in}{1.081575in}}{\pgfqpoint{8.561116in}{1.092174in}}{\pgfqpoint{8.561116in}{1.103224in}}%
\pgfpathcurveto{\pgfqpoint{8.561116in}{1.114274in}}{\pgfqpoint{8.556726in}{1.124873in}}{\pgfqpoint{8.548912in}{1.132687in}}%
\pgfpathcurveto{\pgfqpoint{8.541099in}{1.140501in}}{\pgfqpoint{8.530499in}{1.144891in}}{\pgfqpoint{8.519449in}{1.144891in}}%
\pgfpathcurveto{\pgfqpoint{8.508399in}{1.144891in}}{\pgfqpoint{8.497800in}{1.140501in}}{\pgfqpoint{8.489987in}{1.132687in}}%
\pgfpathcurveto{\pgfqpoint{8.482173in}{1.124873in}}{\pgfqpoint{8.477783in}{1.114274in}}{\pgfqpoint{8.477783in}{1.103224in}}%
\pgfpathcurveto{\pgfqpoint{8.477783in}{1.092174in}}{\pgfqpoint{8.482173in}{1.081575in}}{\pgfqpoint{8.489987in}{1.073761in}}%
\pgfpathcurveto{\pgfqpoint{8.497800in}{1.065948in}}{\pgfqpoint{8.508399in}{1.061557in}}{\pgfqpoint{8.519449in}{1.061557in}}%
\pgfpathlineto{\pgfqpoint{8.519449in}{1.061557in}}%
\pgfpathclose%
\pgfusepath{stroke}%
\end{pgfscope}%
\begin{pgfscope}%
\pgfpathrectangle{\pgfqpoint{7.512535in}{0.437222in}}{\pgfqpoint{6.275590in}{5.159444in}}%
\pgfusepath{clip}%
\pgfsetbuttcap%
\pgfsetroundjoin%
\pgfsetlinewidth{1.003750pt}%
\definecolor{currentstroke}{rgb}{0.827451,0.827451,0.827451}%
\pgfsetstrokecolor{currentstroke}%
\pgfsetstrokeopacity{0.800000}%
\pgfsetdash{}{0pt}%
\pgfpathmoveto{\pgfqpoint{9.792079in}{4.165674in}}%
\pgfpathcurveto{\pgfqpoint{9.803129in}{4.165674in}}{\pgfqpoint{9.813728in}{4.170064in}}{\pgfqpoint{9.821541in}{4.177878in}}%
\pgfpathcurveto{\pgfqpoint{9.829355in}{4.185692in}}{\pgfqpoint{9.833745in}{4.196291in}}{\pgfqpoint{9.833745in}{4.207341in}}%
\pgfpathcurveto{\pgfqpoint{9.833745in}{4.218391in}}{\pgfqpoint{9.829355in}{4.228990in}}{\pgfqpoint{9.821541in}{4.236803in}}%
\pgfpathcurveto{\pgfqpoint{9.813728in}{4.244617in}}{\pgfqpoint{9.803129in}{4.249007in}}{\pgfqpoint{9.792079in}{4.249007in}}%
\pgfpathcurveto{\pgfqpoint{9.781029in}{4.249007in}}{\pgfqpoint{9.770429in}{4.244617in}}{\pgfqpoint{9.762616in}{4.236803in}}%
\pgfpathcurveto{\pgfqpoint{9.754802in}{4.228990in}}{\pgfqpoint{9.750412in}{4.218391in}}{\pgfqpoint{9.750412in}{4.207341in}}%
\pgfpathcurveto{\pgfqpoint{9.750412in}{4.196291in}}{\pgfqpoint{9.754802in}{4.185692in}}{\pgfqpoint{9.762616in}{4.177878in}}%
\pgfpathcurveto{\pgfqpoint{9.770429in}{4.170064in}}{\pgfqpoint{9.781029in}{4.165674in}}{\pgfqpoint{9.792079in}{4.165674in}}%
\pgfpathlineto{\pgfqpoint{9.792079in}{4.165674in}}%
\pgfpathclose%
\pgfusepath{stroke}%
\end{pgfscope}%
\begin{pgfscope}%
\pgfpathrectangle{\pgfqpoint{7.512535in}{0.437222in}}{\pgfqpoint{6.275590in}{5.159444in}}%
\pgfusepath{clip}%
\pgfsetbuttcap%
\pgfsetroundjoin%
\pgfsetlinewidth{1.003750pt}%
\definecolor{currentstroke}{rgb}{0.827451,0.827451,0.827451}%
\pgfsetstrokecolor{currentstroke}%
\pgfsetstrokeopacity{0.800000}%
\pgfsetdash{}{0pt}%
\pgfpathmoveto{\pgfqpoint{7.992439in}{1.704977in}}%
\pgfpathcurveto{\pgfqpoint{8.003489in}{1.704977in}}{\pgfqpoint{8.014088in}{1.709367in}}{\pgfqpoint{8.021902in}{1.717181in}}%
\pgfpathcurveto{\pgfqpoint{8.029715in}{1.724994in}}{\pgfqpoint{8.034105in}{1.735593in}}{\pgfqpoint{8.034105in}{1.746643in}}%
\pgfpathcurveto{\pgfqpoint{8.034105in}{1.757694in}}{\pgfqpoint{8.029715in}{1.768293in}}{\pgfqpoint{8.021902in}{1.776106in}}%
\pgfpathcurveto{\pgfqpoint{8.014088in}{1.783920in}}{\pgfqpoint{8.003489in}{1.788310in}}{\pgfqpoint{7.992439in}{1.788310in}}%
\pgfpathcurveto{\pgfqpoint{7.981389in}{1.788310in}}{\pgfqpoint{7.970790in}{1.783920in}}{\pgfqpoint{7.962976in}{1.776106in}}%
\pgfpathcurveto{\pgfqpoint{7.955162in}{1.768293in}}{\pgfqpoint{7.950772in}{1.757694in}}{\pgfqpoint{7.950772in}{1.746643in}}%
\pgfpathcurveto{\pgfqpoint{7.950772in}{1.735593in}}{\pgfqpoint{7.955162in}{1.724994in}}{\pgfqpoint{7.962976in}{1.717181in}}%
\pgfpathcurveto{\pgfqpoint{7.970790in}{1.709367in}}{\pgfqpoint{7.981389in}{1.704977in}}{\pgfqpoint{7.992439in}{1.704977in}}%
\pgfpathlineto{\pgfqpoint{7.992439in}{1.704977in}}%
\pgfpathclose%
\pgfusepath{stroke}%
\end{pgfscope}%
\begin{pgfscope}%
\pgfpathrectangle{\pgfqpoint{7.512535in}{0.437222in}}{\pgfqpoint{6.275590in}{5.159444in}}%
\pgfusepath{clip}%
\pgfsetbuttcap%
\pgfsetroundjoin%
\pgfsetlinewidth{1.003750pt}%
\definecolor{currentstroke}{rgb}{0.827451,0.827451,0.827451}%
\pgfsetstrokecolor{currentstroke}%
\pgfsetstrokeopacity{0.800000}%
\pgfsetdash{}{0pt}%
\pgfpathmoveto{\pgfqpoint{9.726466in}{3.345111in}}%
\pgfpathcurveto{\pgfqpoint{9.737516in}{3.345111in}}{\pgfqpoint{9.748115in}{3.349501in}}{\pgfqpoint{9.755929in}{3.357315in}}%
\pgfpathcurveto{\pgfqpoint{9.763743in}{3.365128in}}{\pgfqpoint{9.768133in}{3.375727in}}{\pgfqpoint{9.768133in}{3.386778in}}%
\pgfpathcurveto{\pgfqpoint{9.768133in}{3.397828in}}{\pgfqpoint{9.763743in}{3.408427in}}{\pgfqpoint{9.755929in}{3.416240in}}%
\pgfpathcurveto{\pgfqpoint{9.748115in}{3.424054in}}{\pgfqpoint{9.737516in}{3.428444in}}{\pgfqpoint{9.726466in}{3.428444in}}%
\pgfpathcurveto{\pgfqpoint{9.715416in}{3.428444in}}{\pgfqpoint{9.704817in}{3.424054in}}{\pgfqpoint{9.697004in}{3.416240in}}%
\pgfpathcurveto{\pgfqpoint{9.689190in}{3.408427in}}{\pgfqpoint{9.684800in}{3.397828in}}{\pgfqpoint{9.684800in}{3.386778in}}%
\pgfpathcurveto{\pgfqpoint{9.684800in}{3.375727in}}{\pgfqpoint{9.689190in}{3.365128in}}{\pgfqpoint{9.697004in}{3.357315in}}%
\pgfpathcurveto{\pgfqpoint{9.704817in}{3.349501in}}{\pgfqpoint{9.715416in}{3.345111in}}{\pgfqpoint{9.726466in}{3.345111in}}%
\pgfpathlineto{\pgfqpoint{9.726466in}{3.345111in}}%
\pgfpathclose%
\pgfusepath{stroke}%
\end{pgfscope}%
\begin{pgfscope}%
\pgfpathrectangle{\pgfqpoint{7.512535in}{0.437222in}}{\pgfqpoint{6.275590in}{5.159444in}}%
\pgfusepath{clip}%
\pgfsetbuttcap%
\pgfsetroundjoin%
\pgfsetlinewidth{1.003750pt}%
\definecolor{currentstroke}{rgb}{0.827451,0.827451,0.827451}%
\pgfsetstrokecolor{currentstroke}%
\pgfsetstrokeopacity{0.800000}%
\pgfsetdash{}{0pt}%
\pgfpathmoveto{\pgfqpoint{8.282445in}{2.291740in}}%
\pgfpathcurveto{\pgfqpoint{8.293495in}{2.291740in}}{\pgfqpoint{8.304095in}{2.296130in}}{\pgfqpoint{8.311908in}{2.303944in}}%
\pgfpathcurveto{\pgfqpoint{8.319722in}{2.311758in}}{\pgfqpoint{8.324112in}{2.322357in}}{\pgfqpoint{8.324112in}{2.333407in}}%
\pgfpathcurveto{\pgfqpoint{8.324112in}{2.344457in}}{\pgfqpoint{8.319722in}{2.355056in}}{\pgfqpoint{8.311908in}{2.362870in}}%
\pgfpathcurveto{\pgfqpoint{8.304095in}{2.370683in}}{\pgfqpoint{8.293495in}{2.375074in}}{\pgfqpoint{8.282445in}{2.375074in}}%
\pgfpathcurveto{\pgfqpoint{8.271395in}{2.375074in}}{\pgfqpoint{8.260796in}{2.370683in}}{\pgfqpoint{8.252983in}{2.362870in}}%
\pgfpathcurveto{\pgfqpoint{8.245169in}{2.355056in}}{\pgfqpoint{8.240779in}{2.344457in}}{\pgfqpoint{8.240779in}{2.333407in}}%
\pgfpathcurveto{\pgfqpoint{8.240779in}{2.322357in}}{\pgfqpoint{8.245169in}{2.311758in}}{\pgfqpoint{8.252983in}{2.303944in}}%
\pgfpathcurveto{\pgfqpoint{8.260796in}{2.296130in}}{\pgfqpoint{8.271395in}{2.291740in}}{\pgfqpoint{8.282445in}{2.291740in}}%
\pgfpathlineto{\pgfqpoint{8.282445in}{2.291740in}}%
\pgfpathclose%
\pgfusepath{stroke}%
\end{pgfscope}%
\begin{pgfscope}%
\pgfpathrectangle{\pgfqpoint{7.512535in}{0.437222in}}{\pgfqpoint{6.275590in}{5.159444in}}%
\pgfusepath{clip}%
\pgfsetbuttcap%
\pgfsetroundjoin%
\pgfsetlinewidth{1.003750pt}%
\definecolor{currentstroke}{rgb}{0.827451,0.827451,0.827451}%
\pgfsetstrokecolor{currentstroke}%
\pgfsetstrokeopacity{0.800000}%
\pgfsetdash{}{0pt}%
\pgfpathmoveto{\pgfqpoint{9.807010in}{4.531380in}}%
\pgfpathcurveto{\pgfqpoint{9.818060in}{4.531380in}}{\pgfqpoint{9.828659in}{4.535770in}}{\pgfqpoint{9.836473in}{4.543584in}}%
\pgfpathcurveto{\pgfqpoint{9.844287in}{4.551397in}}{\pgfqpoint{9.848677in}{4.561996in}}{\pgfqpoint{9.848677in}{4.573046in}}%
\pgfpathcurveto{\pgfqpoint{9.848677in}{4.584096in}}{\pgfqpoint{9.844287in}{4.594695in}}{\pgfqpoint{9.836473in}{4.602509in}}%
\pgfpathcurveto{\pgfqpoint{9.828659in}{4.610323in}}{\pgfqpoint{9.818060in}{4.614713in}}{\pgfqpoint{9.807010in}{4.614713in}}%
\pgfpathcurveto{\pgfqpoint{9.795960in}{4.614713in}}{\pgfqpoint{9.785361in}{4.610323in}}{\pgfqpoint{9.777548in}{4.602509in}}%
\pgfpathcurveto{\pgfqpoint{9.769734in}{4.594695in}}{\pgfqpoint{9.765344in}{4.584096in}}{\pgfqpoint{9.765344in}{4.573046in}}%
\pgfpathcurveto{\pgfqpoint{9.765344in}{4.561996in}}{\pgfqpoint{9.769734in}{4.551397in}}{\pgfqpoint{9.777548in}{4.543584in}}%
\pgfpathcurveto{\pgfqpoint{9.785361in}{4.535770in}}{\pgfqpoint{9.795960in}{4.531380in}}{\pgfqpoint{9.807010in}{4.531380in}}%
\pgfpathlineto{\pgfqpoint{9.807010in}{4.531380in}}%
\pgfpathclose%
\pgfusepath{stroke}%
\end{pgfscope}%
\begin{pgfscope}%
\pgfpathrectangle{\pgfqpoint{7.512535in}{0.437222in}}{\pgfqpoint{6.275590in}{5.159444in}}%
\pgfusepath{clip}%
\pgfsetbuttcap%
\pgfsetroundjoin%
\pgfsetlinewidth{1.003750pt}%
\definecolor{currentstroke}{rgb}{0.827451,0.827451,0.827451}%
\pgfsetstrokecolor{currentstroke}%
\pgfsetstrokeopacity{0.800000}%
\pgfsetdash{}{0pt}%
\pgfpathmoveto{\pgfqpoint{10.698510in}{3.970272in}}%
\pgfpathcurveto{\pgfqpoint{10.709560in}{3.970272in}}{\pgfqpoint{10.720159in}{3.974662in}}{\pgfqpoint{10.727973in}{3.982476in}}%
\pgfpathcurveto{\pgfqpoint{10.735787in}{3.990290in}}{\pgfqpoint{10.740177in}{4.000889in}}{\pgfqpoint{10.740177in}{4.011939in}}%
\pgfpathcurveto{\pgfqpoint{10.740177in}{4.022989in}}{\pgfqpoint{10.735787in}{4.033588in}}{\pgfqpoint{10.727973in}{4.041402in}}%
\pgfpathcurveto{\pgfqpoint{10.720159in}{4.049215in}}{\pgfqpoint{10.709560in}{4.053605in}}{\pgfqpoint{10.698510in}{4.053605in}}%
\pgfpathcurveto{\pgfqpoint{10.687460in}{4.053605in}}{\pgfqpoint{10.676861in}{4.049215in}}{\pgfqpoint{10.669047in}{4.041402in}}%
\pgfpathcurveto{\pgfqpoint{10.661234in}{4.033588in}}{\pgfqpoint{10.656844in}{4.022989in}}{\pgfqpoint{10.656844in}{4.011939in}}%
\pgfpathcurveto{\pgfqpoint{10.656844in}{4.000889in}}{\pgfqpoint{10.661234in}{3.990290in}}{\pgfqpoint{10.669047in}{3.982476in}}%
\pgfpathcurveto{\pgfqpoint{10.676861in}{3.974662in}}{\pgfqpoint{10.687460in}{3.970272in}}{\pgfqpoint{10.698510in}{3.970272in}}%
\pgfpathlineto{\pgfqpoint{10.698510in}{3.970272in}}%
\pgfpathclose%
\pgfusepath{stroke}%
\end{pgfscope}%
\begin{pgfscope}%
\pgfpathrectangle{\pgfqpoint{7.512535in}{0.437222in}}{\pgfqpoint{6.275590in}{5.159444in}}%
\pgfusepath{clip}%
\pgfsetbuttcap%
\pgfsetroundjoin%
\pgfsetlinewidth{1.003750pt}%
\definecolor{currentstroke}{rgb}{0.827451,0.827451,0.827451}%
\pgfsetstrokecolor{currentstroke}%
\pgfsetstrokeopacity{0.800000}%
\pgfsetdash{}{0pt}%
\pgfpathmoveto{\pgfqpoint{11.645781in}{5.002607in}}%
\pgfpathcurveto{\pgfqpoint{11.656832in}{5.002607in}}{\pgfqpoint{11.667431in}{5.006998in}}{\pgfqpoint{11.675244in}{5.014811in}}%
\pgfpathcurveto{\pgfqpoint{11.683058in}{5.022625in}}{\pgfqpoint{11.687448in}{5.033224in}}{\pgfqpoint{11.687448in}{5.044274in}}%
\pgfpathcurveto{\pgfqpoint{11.687448in}{5.055324in}}{\pgfqpoint{11.683058in}{5.065923in}}{\pgfqpoint{11.675244in}{5.073737in}}%
\pgfpathcurveto{\pgfqpoint{11.667431in}{5.081550in}}{\pgfqpoint{11.656832in}{5.085941in}}{\pgfqpoint{11.645781in}{5.085941in}}%
\pgfpathcurveto{\pgfqpoint{11.634731in}{5.085941in}}{\pgfqpoint{11.624132in}{5.081550in}}{\pgfqpoint{11.616319in}{5.073737in}}%
\pgfpathcurveto{\pgfqpoint{11.608505in}{5.065923in}}{\pgfqpoint{11.604115in}{5.055324in}}{\pgfqpoint{11.604115in}{5.044274in}}%
\pgfpathcurveto{\pgfqpoint{11.604115in}{5.033224in}}{\pgfqpoint{11.608505in}{5.022625in}}{\pgfqpoint{11.616319in}{5.014811in}}%
\pgfpathcurveto{\pgfqpoint{11.624132in}{5.006998in}}{\pgfqpoint{11.634731in}{5.002607in}}{\pgfqpoint{11.645781in}{5.002607in}}%
\pgfpathlineto{\pgfqpoint{11.645781in}{5.002607in}}%
\pgfpathclose%
\pgfusepath{stroke}%
\end{pgfscope}%
\begin{pgfscope}%
\pgfpathrectangle{\pgfqpoint{7.512535in}{0.437222in}}{\pgfqpoint{6.275590in}{5.159444in}}%
\pgfusepath{clip}%
\pgfsetbuttcap%
\pgfsetroundjoin%
\pgfsetlinewidth{1.003750pt}%
\definecolor{currentstroke}{rgb}{0.827451,0.827451,0.827451}%
\pgfsetstrokecolor{currentstroke}%
\pgfsetstrokeopacity{0.800000}%
\pgfsetdash{}{0pt}%
\pgfpathmoveto{\pgfqpoint{8.290348in}{1.139798in}}%
\pgfpathcurveto{\pgfqpoint{8.301398in}{1.139798in}}{\pgfqpoint{8.311997in}{1.144188in}}{\pgfqpoint{8.319811in}{1.152002in}}%
\pgfpathcurveto{\pgfqpoint{8.327625in}{1.159815in}}{\pgfqpoint{8.332015in}{1.170414in}}{\pgfqpoint{8.332015in}{1.181465in}}%
\pgfpathcurveto{\pgfqpoint{8.332015in}{1.192515in}}{\pgfqpoint{8.327625in}{1.203114in}}{\pgfqpoint{8.319811in}{1.210927in}}%
\pgfpathcurveto{\pgfqpoint{8.311997in}{1.218741in}}{\pgfqpoint{8.301398in}{1.223131in}}{\pgfqpoint{8.290348in}{1.223131in}}%
\pgfpathcurveto{\pgfqpoint{8.279298in}{1.223131in}}{\pgfqpoint{8.268699in}{1.218741in}}{\pgfqpoint{8.260886in}{1.210927in}}%
\pgfpathcurveto{\pgfqpoint{8.253072in}{1.203114in}}{\pgfqpoint{8.248682in}{1.192515in}}{\pgfqpoint{8.248682in}{1.181465in}}%
\pgfpathcurveto{\pgfqpoint{8.248682in}{1.170414in}}{\pgfqpoint{8.253072in}{1.159815in}}{\pgfqpoint{8.260886in}{1.152002in}}%
\pgfpathcurveto{\pgfqpoint{8.268699in}{1.144188in}}{\pgfqpoint{8.279298in}{1.139798in}}{\pgfqpoint{8.290348in}{1.139798in}}%
\pgfpathlineto{\pgfqpoint{8.290348in}{1.139798in}}%
\pgfpathclose%
\pgfusepath{stroke}%
\end{pgfscope}%
\begin{pgfscope}%
\pgfpathrectangle{\pgfqpoint{7.512535in}{0.437222in}}{\pgfqpoint{6.275590in}{5.159444in}}%
\pgfusepath{clip}%
\pgfsetbuttcap%
\pgfsetroundjoin%
\pgfsetlinewidth{1.003750pt}%
\definecolor{currentstroke}{rgb}{0.827451,0.827451,0.827451}%
\pgfsetstrokecolor{currentstroke}%
\pgfsetstrokeopacity{0.800000}%
\pgfsetdash{}{0pt}%
\pgfpathmoveto{\pgfqpoint{9.928499in}{4.056499in}}%
\pgfpathcurveto{\pgfqpoint{9.939549in}{4.056499in}}{\pgfqpoint{9.950148in}{4.060889in}}{\pgfqpoint{9.957962in}{4.068703in}}%
\pgfpathcurveto{\pgfqpoint{9.965775in}{4.076516in}}{\pgfqpoint{9.970165in}{4.087115in}}{\pgfqpoint{9.970165in}{4.098166in}}%
\pgfpathcurveto{\pgfqpoint{9.970165in}{4.109216in}}{\pgfqpoint{9.965775in}{4.119815in}}{\pgfqpoint{9.957962in}{4.127628in}}%
\pgfpathcurveto{\pgfqpoint{9.950148in}{4.135442in}}{\pgfqpoint{9.939549in}{4.139832in}}{\pgfqpoint{9.928499in}{4.139832in}}%
\pgfpathcurveto{\pgfqpoint{9.917449in}{4.139832in}}{\pgfqpoint{9.906850in}{4.135442in}}{\pgfqpoint{9.899036in}{4.127628in}}%
\pgfpathcurveto{\pgfqpoint{9.891222in}{4.119815in}}{\pgfqpoint{9.886832in}{4.109216in}}{\pgfqpoint{9.886832in}{4.098166in}}%
\pgfpathcurveto{\pgfqpoint{9.886832in}{4.087115in}}{\pgfqpoint{9.891222in}{4.076516in}}{\pgfqpoint{9.899036in}{4.068703in}}%
\pgfpathcurveto{\pgfqpoint{9.906850in}{4.060889in}}{\pgfqpoint{9.917449in}{4.056499in}}{\pgfqpoint{9.928499in}{4.056499in}}%
\pgfpathlineto{\pgfqpoint{9.928499in}{4.056499in}}%
\pgfpathclose%
\pgfusepath{stroke}%
\end{pgfscope}%
\begin{pgfscope}%
\pgfpathrectangle{\pgfqpoint{7.512535in}{0.437222in}}{\pgfqpoint{6.275590in}{5.159444in}}%
\pgfusepath{clip}%
\pgfsetbuttcap%
\pgfsetroundjoin%
\pgfsetlinewidth{1.003750pt}%
\definecolor{currentstroke}{rgb}{0.827451,0.827451,0.827451}%
\pgfsetstrokecolor{currentstroke}%
\pgfsetstrokeopacity{0.800000}%
\pgfsetdash{}{0pt}%
\pgfpathmoveto{\pgfqpoint{10.827454in}{5.304399in}}%
\pgfpathcurveto{\pgfqpoint{10.838504in}{5.304399in}}{\pgfqpoint{10.849103in}{5.308790in}}{\pgfqpoint{10.856917in}{5.316603in}}%
\pgfpathcurveto{\pgfqpoint{10.864730in}{5.324417in}}{\pgfqpoint{10.869121in}{5.335016in}}{\pgfqpoint{10.869121in}{5.346066in}}%
\pgfpathcurveto{\pgfqpoint{10.869121in}{5.357116in}}{\pgfqpoint{10.864730in}{5.367715in}}{\pgfqpoint{10.856917in}{5.375529in}}%
\pgfpathcurveto{\pgfqpoint{10.849103in}{5.383342in}}{\pgfqpoint{10.838504in}{5.387733in}}{\pgfqpoint{10.827454in}{5.387733in}}%
\pgfpathcurveto{\pgfqpoint{10.816404in}{5.387733in}}{\pgfqpoint{10.805805in}{5.383342in}}{\pgfqpoint{10.797991in}{5.375529in}}%
\pgfpathcurveto{\pgfqpoint{10.790177in}{5.367715in}}{\pgfqpoint{10.785787in}{5.357116in}}{\pgfqpoint{10.785787in}{5.346066in}}%
\pgfpathcurveto{\pgfqpoint{10.785787in}{5.335016in}}{\pgfqpoint{10.790177in}{5.324417in}}{\pgfqpoint{10.797991in}{5.316603in}}%
\pgfpathcurveto{\pgfqpoint{10.805805in}{5.308790in}}{\pgfqpoint{10.816404in}{5.304399in}}{\pgfqpoint{10.827454in}{5.304399in}}%
\pgfpathlineto{\pgfqpoint{10.827454in}{5.304399in}}%
\pgfpathclose%
\pgfusepath{stroke}%
\end{pgfscope}%
\begin{pgfscope}%
\pgfpathrectangle{\pgfqpoint{7.512535in}{0.437222in}}{\pgfqpoint{6.275590in}{5.159444in}}%
\pgfusepath{clip}%
\pgfsetbuttcap%
\pgfsetroundjoin%
\pgfsetlinewidth{1.003750pt}%
\definecolor{currentstroke}{rgb}{0.827451,0.827451,0.827451}%
\pgfsetstrokecolor{currentstroke}%
\pgfsetstrokeopacity{0.800000}%
\pgfsetdash{}{0pt}%
\pgfpathmoveto{\pgfqpoint{10.127909in}{3.160252in}}%
\pgfpathcurveto{\pgfqpoint{10.138959in}{3.160252in}}{\pgfqpoint{10.149558in}{3.164643in}}{\pgfqpoint{10.157372in}{3.172456in}}%
\pgfpathcurveto{\pgfqpoint{10.165186in}{3.180270in}}{\pgfqpoint{10.169576in}{3.190869in}}{\pgfqpoint{10.169576in}{3.201919in}}%
\pgfpathcurveto{\pgfqpoint{10.169576in}{3.212969in}}{\pgfqpoint{10.165186in}{3.223568in}}{\pgfqpoint{10.157372in}{3.231382in}}%
\pgfpathcurveto{\pgfqpoint{10.149558in}{3.239196in}}{\pgfqpoint{10.138959in}{3.243586in}}{\pgfqpoint{10.127909in}{3.243586in}}%
\pgfpathcurveto{\pgfqpoint{10.116859in}{3.243586in}}{\pgfqpoint{10.106260in}{3.239196in}}{\pgfqpoint{10.098446in}{3.231382in}}%
\pgfpathcurveto{\pgfqpoint{10.090633in}{3.223568in}}{\pgfqpoint{10.086243in}{3.212969in}}{\pgfqpoint{10.086243in}{3.201919in}}%
\pgfpathcurveto{\pgfqpoint{10.086243in}{3.190869in}}{\pgfqpoint{10.090633in}{3.180270in}}{\pgfqpoint{10.098446in}{3.172456in}}%
\pgfpathcurveto{\pgfqpoint{10.106260in}{3.164643in}}{\pgfqpoint{10.116859in}{3.160252in}}{\pgfqpoint{10.127909in}{3.160252in}}%
\pgfpathlineto{\pgfqpoint{10.127909in}{3.160252in}}%
\pgfpathclose%
\pgfusepath{stroke}%
\end{pgfscope}%
\begin{pgfscope}%
\pgfpathrectangle{\pgfqpoint{7.512535in}{0.437222in}}{\pgfqpoint{6.275590in}{5.159444in}}%
\pgfusepath{clip}%
\pgfsetbuttcap%
\pgfsetroundjoin%
\pgfsetlinewidth{1.003750pt}%
\definecolor{currentstroke}{rgb}{0.827451,0.827451,0.827451}%
\pgfsetstrokecolor{currentstroke}%
\pgfsetstrokeopacity{0.800000}%
\pgfsetdash{}{0pt}%
\pgfpathmoveto{\pgfqpoint{10.156793in}{3.250664in}}%
\pgfpathcurveto{\pgfqpoint{10.167843in}{3.250664in}}{\pgfqpoint{10.178442in}{3.255054in}}{\pgfqpoint{10.186256in}{3.262868in}}%
\pgfpathcurveto{\pgfqpoint{10.194069in}{3.270682in}}{\pgfqpoint{10.198460in}{3.281281in}}{\pgfqpoint{10.198460in}{3.292331in}}%
\pgfpathcurveto{\pgfqpoint{10.198460in}{3.303381in}}{\pgfqpoint{10.194069in}{3.313980in}}{\pgfqpoint{10.186256in}{3.321794in}}%
\pgfpathcurveto{\pgfqpoint{10.178442in}{3.329607in}}{\pgfqpoint{10.167843in}{3.333998in}}{\pgfqpoint{10.156793in}{3.333998in}}%
\pgfpathcurveto{\pgfqpoint{10.145743in}{3.333998in}}{\pgfqpoint{10.135144in}{3.329607in}}{\pgfqpoint{10.127330in}{3.321794in}}%
\pgfpathcurveto{\pgfqpoint{10.119517in}{3.313980in}}{\pgfqpoint{10.115126in}{3.303381in}}{\pgfqpoint{10.115126in}{3.292331in}}%
\pgfpathcurveto{\pgfqpoint{10.115126in}{3.281281in}}{\pgfqpoint{10.119517in}{3.270682in}}{\pgfqpoint{10.127330in}{3.262868in}}%
\pgfpathcurveto{\pgfqpoint{10.135144in}{3.255054in}}{\pgfqpoint{10.145743in}{3.250664in}}{\pgfqpoint{10.156793in}{3.250664in}}%
\pgfpathlineto{\pgfqpoint{10.156793in}{3.250664in}}%
\pgfpathclose%
\pgfusepath{stroke}%
\end{pgfscope}%
\begin{pgfscope}%
\pgfpathrectangle{\pgfqpoint{7.512535in}{0.437222in}}{\pgfqpoint{6.275590in}{5.159444in}}%
\pgfusepath{clip}%
\pgfsetbuttcap%
\pgfsetroundjoin%
\pgfsetlinewidth{1.003750pt}%
\definecolor{currentstroke}{rgb}{0.827451,0.827451,0.827451}%
\pgfsetstrokecolor{currentstroke}%
\pgfsetstrokeopacity{0.800000}%
\pgfsetdash{}{0pt}%
\pgfpathmoveto{\pgfqpoint{8.235613in}{1.702217in}}%
\pgfpathcurveto{\pgfqpoint{8.246663in}{1.702217in}}{\pgfqpoint{8.257262in}{1.706607in}}{\pgfqpoint{8.265075in}{1.714421in}}%
\pgfpathcurveto{\pgfqpoint{8.272889in}{1.722235in}}{\pgfqpoint{8.277279in}{1.732834in}}{\pgfqpoint{8.277279in}{1.743884in}}%
\pgfpathcurveto{\pgfqpoint{8.277279in}{1.754934in}}{\pgfqpoint{8.272889in}{1.765533in}}{\pgfqpoint{8.265075in}{1.773347in}}%
\pgfpathcurveto{\pgfqpoint{8.257262in}{1.781160in}}{\pgfqpoint{8.246663in}{1.785550in}}{\pgfqpoint{8.235613in}{1.785550in}}%
\pgfpathcurveto{\pgfqpoint{8.224562in}{1.785550in}}{\pgfqpoint{8.213963in}{1.781160in}}{\pgfqpoint{8.206150in}{1.773347in}}%
\pgfpathcurveto{\pgfqpoint{8.198336in}{1.765533in}}{\pgfqpoint{8.193946in}{1.754934in}}{\pgfqpoint{8.193946in}{1.743884in}}%
\pgfpathcurveto{\pgfqpoint{8.193946in}{1.732834in}}{\pgfqpoint{8.198336in}{1.722235in}}{\pgfqpoint{8.206150in}{1.714421in}}%
\pgfpathcurveto{\pgfqpoint{8.213963in}{1.706607in}}{\pgfqpoint{8.224562in}{1.702217in}}{\pgfqpoint{8.235613in}{1.702217in}}%
\pgfpathlineto{\pgfqpoint{8.235613in}{1.702217in}}%
\pgfpathclose%
\pgfusepath{stroke}%
\end{pgfscope}%
\begin{pgfscope}%
\pgfpathrectangle{\pgfqpoint{7.512535in}{0.437222in}}{\pgfqpoint{6.275590in}{5.159444in}}%
\pgfusepath{clip}%
\pgfsetbuttcap%
\pgfsetroundjoin%
\pgfsetlinewidth{1.003750pt}%
\definecolor{currentstroke}{rgb}{0.827451,0.827451,0.827451}%
\pgfsetstrokecolor{currentstroke}%
\pgfsetstrokeopacity{0.800000}%
\pgfsetdash{}{0pt}%
\pgfpathmoveto{\pgfqpoint{10.299622in}{3.123555in}}%
\pgfpathcurveto{\pgfqpoint{10.310672in}{3.123555in}}{\pgfqpoint{10.321271in}{3.127945in}}{\pgfqpoint{10.329085in}{3.135759in}}%
\pgfpathcurveto{\pgfqpoint{10.336898in}{3.143572in}}{\pgfqpoint{10.341289in}{3.154171in}}{\pgfqpoint{10.341289in}{3.165222in}}%
\pgfpathcurveto{\pgfqpoint{10.341289in}{3.176272in}}{\pgfqpoint{10.336898in}{3.186871in}}{\pgfqpoint{10.329085in}{3.194684in}}%
\pgfpathcurveto{\pgfqpoint{10.321271in}{3.202498in}}{\pgfqpoint{10.310672in}{3.206888in}}{\pgfqpoint{10.299622in}{3.206888in}}%
\pgfpathcurveto{\pgfqpoint{10.288572in}{3.206888in}}{\pgfqpoint{10.277973in}{3.202498in}}{\pgfqpoint{10.270159in}{3.194684in}}%
\pgfpathcurveto{\pgfqpoint{10.262346in}{3.186871in}}{\pgfqpoint{10.257955in}{3.176272in}}{\pgfqpoint{10.257955in}{3.165222in}}%
\pgfpathcurveto{\pgfqpoint{10.257955in}{3.154171in}}{\pgfqpoint{10.262346in}{3.143572in}}{\pgfqpoint{10.270159in}{3.135759in}}%
\pgfpathcurveto{\pgfqpoint{10.277973in}{3.127945in}}{\pgfqpoint{10.288572in}{3.123555in}}{\pgfqpoint{10.299622in}{3.123555in}}%
\pgfpathlineto{\pgfqpoint{10.299622in}{3.123555in}}%
\pgfpathclose%
\pgfusepath{stroke}%
\end{pgfscope}%
\begin{pgfscope}%
\pgfpathrectangle{\pgfqpoint{7.512535in}{0.437222in}}{\pgfqpoint{6.275590in}{5.159444in}}%
\pgfusepath{clip}%
\pgfsetbuttcap%
\pgfsetroundjoin%
\pgfsetlinewidth{1.003750pt}%
\definecolor{currentstroke}{rgb}{0.827451,0.827451,0.827451}%
\pgfsetstrokecolor{currentstroke}%
\pgfsetstrokeopacity{0.800000}%
\pgfsetdash{}{0pt}%
\pgfpathmoveto{\pgfqpoint{9.540773in}{3.442114in}}%
\pgfpathcurveto{\pgfqpoint{9.551823in}{3.442114in}}{\pgfqpoint{9.562422in}{3.446504in}}{\pgfqpoint{9.570236in}{3.454318in}}%
\pgfpathcurveto{\pgfqpoint{9.578049in}{3.462131in}}{\pgfqpoint{9.582440in}{3.472730in}}{\pgfqpoint{9.582440in}{3.483780in}}%
\pgfpathcurveto{\pgfqpoint{9.582440in}{3.494830in}}{\pgfqpoint{9.578049in}{3.505429in}}{\pgfqpoint{9.570236in}{3.513243in}}%
\pgfpathcurveto{\pgfqpoint{9.562422in}{3.521057in}}{\pgfqpoint{9.551823in}{3.525447in}}{\pgfqpoint{9.540773in}{3.525447in}}%
\pgfpathcurveto{\pgfqpoint{9.529723in}{3.525447in}}{\pgfqpoint{9.519124in}{3.521057in}}{\pgfqpoint{9.511310in}{3.513243in}}%
\pgfpathcurveto{\pgfqpoint{9.503497in}{3.505429in}}{\pgfqpoint{9.499106in}{3.494830in}}{\pgfqpoint{9.499106in}{3.483780in}}%
\pgfpathcurveto{\pgfqpoint{9.499106in}{3.472730in}}{\pgfqpoint{9.503497in}{3.462131in}}{\pgfqpoint{9.511310in}{3.454318in}}%
\pgfpathcurveto{\pgfqpoint{9.519124in}{3.446504in}}{\pgfqpoint{9.529723in}{3.442114in}}{\pgfqpoint{9.540773in}{3.442114in}}%
\pgfpathlineto{\pgfqpoint{9.540773in}{3.442114in}}%
\pgfpathclose%
\pgfusepath{stroke}%
\end{pgfscope}%
\begin{pgfscope}%
\pgfpathrectangle{\pgfqpoint{7.512535in}{0.437222in}}{\pgfqpoint{6.275590in}{5.159444in}}%
\pgfusepath{clip}%
\pgfsetbuttcap%
\pgfsetroundjoin%
\pgfsetlinewidth{1.003750pt}%
\definecolor{currentstroke}{rgb}{0.827451,0.827451,0.827451}%
\pgfsetstrokecolor{currentstroke}%
\pgfsetstrokeopacity{0.800000}%
\pgfsetdash{}{0pt}%
\pgfpathmoveto{\pgfqpoint{7.874057in}{1.320145in}}%
\pgfpathcurveto{\pgfqpoint{7.885107in}{1.320145in}}{\pgfqpoint{7.895706in}{1.324535in}}{\pgfqpoint{7.903520in}{1.332349in}}%
\pgfpathcurveto{\pgfqpoint{7.911334in}{1.340162in}}{\pgfqpoint{7.915724in}{1.350761in}}{\pgfqpoint{7.915724in}{1.361811in}}%
\pgfpathcurveto{\pgfqpoint{7.915724in}{1.372862in}}{\pgfqpoint{7.911334in}{1.383461in}}{\pgfqpoint{7.903520in}{1.391274in}}%
\pgfpathcurveto{\pgfqpoint{7.895706in}{1.399088in}}{\pgfqpoint{7.885107in}{1.403478in}}{\pgfqpoint{7.874057in}{1.403478in}}%
\pgfpathcurveto{\pgfqpoint{7.863007in}{1.403478in}}{\pgfqpoint{7.852408in}{1.399088in}}{\pgfqpoint{7.844595in}{1.391274in}}%
\pgfpathcurveto{\pgfqpoint{7.836781in}{1.383461in}}{\pgfqpoint{7.832391in}{1.372862in}}{\pgfqpoint{7.832391in}{1.361811in}}%
\pgfpathcurveto{\pgfqpoint{7.832391in}{1.350761in}}{\pgfqpoint{7.836781in}{1.340162in}}{\pgfqpoint{7.844595in}{1.332349in}}%
\pgfpathcurveto{\pgfqpoint{7.852408in}{1.324535in}}{\pgfqpoint{7.863007in}{1.320145in}}{\pgfqpoint{7.874057in}{1.320145in}}%
\pgfpathlineto{\pgfqpoint{7.874057in}{1.320145in}}%
\pgfpathclose%
\pgfusepath{stroke}%
\end{pgfscope}%
\begin{pgfscope}%
\pgfpathrectangle{\pgfqpoint{7.512535in}{0.437222in}}{\pgfqpoint{6.275590in}{5.159444in}}%
\pgfusepath{clip}%
\pgfsetbuttcap%
\pgfsetroundjoin%
\pgfsetlinewidth{1.003750pt}%
\definecolor{currentstroke}{rgb}{0.827451,0.827451,0.827451}%
\pgfsetstrokecolor{currentstroke}%
\pgfsetstrokeopacity{0.800000}%
\pgfsetdash{}{0pt}%
\pgfpathmoveto{\pgfqpoint{10.917195in}{4.264652in}}%
\pgfpathcurveto{\pgfqpoint{10.928245in}{4.264652in}}{\pgfqpoint{10.938844in}{4.269042in}}{\pgfqpoint{10.946658in}{4.276856in}}%
\pgfpathcurveto{\pgfqpoint{10.954472in}{4.284670in}}{\pgfqpoint{10.958862in}{4.295269in}}{\pgfqpoint{10.958862in}{4.306319in}}%
\pgfpathcurveto{\pgfqpoint{10.958862in}{4.317369in}}{\pgfqpoint{10.954472in}{4.327968in}}{\pgfqpoint{10.946658in}{4.335781in}}%
\pgfpathcurveto{\pgfqpoint{10.938844in}{4.343595in}}{\pgfqpoint{10.928245in}{4.347985in}}{\pgfqpoint{10.917195in}{4.347985in}}%
\pgfpathcurveto{\pgfqpoint{10.906145in}{4.347985in}}{\pgfqpoint{10.895546in}{4.343595in}}{\pgfqpoint{10.887732in}{4.335781in}}%
\pgfpathcurveto{\pgfqpoint{10.879919in}{4.327968in}}{\pgfqpoint{10.875529in}{4.317369in}}{\pgfqpoint{10.875529in}{4.306319in}}%
\pgfpathcurveto{\pgfqpoint{10.875529in}{4.295269in}}{\pgfqpoint{10.879919in}{4.284670in}}{\pgfqpoint{10.887732in}{4.276856in}}%
\pgfpathcurveto{\pgfqpoint{10.895546in}{4.269042in}}{\pgfqpoint{10.906145in}{4.264652in}}{\pgfqpoint{10.917195in}{4.264652in}}%
\pgfpathlineto{\pgfqpoint{10.917195in}{4.264652in}}%
\pgfpathclose%
\pgfusepath{stroke}%
\end{pgfscope}%
\begin{pgfscope}%
\pgfpathrectangle{\pgfqpoint{7.512535in}{0.437222in}}{\pgfqpoint{6.275590in}{5.159444in}}%
\pgfusepath{clip}%
\pgfsetbuttcap%
\pgfsetroundjoin%
\pgfsetlinewidth{1.003750pt}%
\definecolor{currentstroke}{rgb}{0.827451,0.827451,0.827451}%
\pgfsetstrokecolor{currentstroke}%
\pgfsetstrokeopacity{0.800000}%
\pgfsetdash{}{0pt}%
\pgfpathmoveto{\pgfqpoint{10.818589in}{5.117177in}}%
\pgfpathcurveto{\pgfqpoint{10.829639in}{5.117177in}}{\pgfqpoint{10.840238in}{5.121567in}}{\pgfqpoint{10.848052in}{5.129381in}}%
\pgfpathcurveto{\pgfqpoint{10.855865in}{5.137194in}}{\pgfqpoint{10.860255in}{5.147793in}}{\pgfqpoint{10.860255in}{5.158843in}}%
\pgfpathcurveto{\pgfqpoint{10.860255in}{5.169893in}}{\pgfqpoint{10.855865in}{5.180493in}}{\pgfqpoint{10.848052in}{5.188306in}}%
\pgfpathcurveto{\pgfqpoint{10.840238in}{5.196120in}}{\pgfqpoint{10.829639in}{5.200510in}}{\pgfqpoint{10.818589in}{5.200510in}}%
\pgfpathcurveto{\pgfqpoint{10.807539in}{5.200510in}}{\pgfqpoint{10.796940in}{5.196120in}}{\pgfqpoint{10.789126in}{5.188306in}}%
\pgfpathcurveto{\pgfqpoint{10.781312in}{5.180493in}}{\pgfqpoint{10.776922in}{5.169893in}}{\pgfqpoint{10.776922in}{5.158843in}}%
\pgfpathcurveto{\pgfqpoint{10.776922in}{5.147793in}}{\pgfqpoint{10.781312in}{5.137194in}}{\pgfqpoint{10.789126in}{5.129381in}}%
\pgfpathcurveto{\pgfqpoint{10.796940in}{5.121567in}}{\pgfqpoint{10.807539in}{5.117177in}}{\pgfqpoint{10.818589in}{5.117177in}}%
\pgfpathlineto{\pgfqpoint{10.818589in}{5.117177in}}%
\pgfpathclose%
\pgfusepath{stroke}%
\end{pgfscope}%
\begin{pgfscope}%
\pgfpathrectangle{\pgfqpoint{7.512535in}{0.437222in}}{\pgfqpoint{6.275590in}{5.159444in}}%
\pgfusepath{clip}%
\pgfsetbuttcap%
\pgfsetroundjoin%
\pgfsetlinewidth{1.003750pt}%
\definecolor{currentstroke}{rgb}{0.827451,0.827451,0.827451}%
\pgfsetstrokecolor{currentstroke}%
\pgfsetstrokeopacity{0.800000}%
\pgfsetdash{}{0pt}%
\pgfpathmoveto{\pgfqpoint{11.405420in}{5.475782in}}%
\pgfpathcurveto{\pgfqpoint{11.416470in}{5.475782in}}{\pgfqpoint{11.427069in}{5.480172in}}{\pgfqpoint{11.434883in}{5.487986in}}%
\pgfpathcurveto{\pgfqpoint{11.442696in}{5.495800in}}{\pgfqpoint{11.447086in}{5.506399in}}{\pgfqpoint{11.447086in}{5.517449in}}%
\pgfpathcurveto{\pgfqpoint{11.447086in}{5.528499in}}{\pgfqpoint{11.442696in}{5.539098in}}{\pgfqpoint{11.434883in}{5.546912in}}%
\pgfpathcurveto{\pgfqpoint{11.427069in}{5.554725in}}{\pgfqpoint{11.416470in}{5.559116in}}{\pgfqpoint{11.405420in}{5.559116in}}%
\pgfpathcurveto{\pgfqpoint{11.394370in}{5.559116in}}{\pgfqpoint{11.383771in}{5.554725in}}{\pgfqpoint{11.375957in}{5.546912in}}%
\pgfpathcurveto{\pgfqpoint{11.368143in}{5.539098in}}{\pgfqpoint{11.363753in}{5.528499in}}{\pgfqpoint{11.363753in}{5.517449in}}%
\pgfpathcurveto{\pgfqpoint{11.363753in}{5.506399in}}{\pgfqpoint{11.368143in}{5.495800in}}{\pgfqpoint{11.375957in}{5.487986in}}%
\pgfpathcurveto{\pgfqpoint{11.383771in}{5.480172in}}{\pgfqpoint{11.394370in}{5.475782in}}{\pgfqpoint{11.405420in}{5.475782in}}%
\pgfpathlineto{\pgfqpoint{11.405420in}{5.475782in}}%
\pgfpathclose%
\pgfusepath{stroke}%
\end{pgfscope}%
\begin{pgfscope}%
\pgfpathrectangle{\pgfqpoint{7.512535in}{0.437222in}}{\pgfqpoint{6.275590in}{5.159444in}}%
\pgfusepath{clip}%
\pgfsetbuttcap%
\pgfsetroundjoin%
\pgfsetlinewidth{1.003750pt}%
\definecolor{currentstroke}{rgb}{0.827451,0.827451,0.827451}%
\pgfsetstrokecolor{currentstroke}%
\pgfsetstrokeopacity{0.800000}%
\pgfsetdash{}{0pt}%
\pgfpathmoveto{\pgfqpoint{12.650243in}{5.481847in}}%
\pgfpathcurveto{\pgfqpoint{12.661293in}{5.481847in}}{\pgfqpoint{12.671892in}{5.486238in}}{\pgfqpoint{12.679706in}{5.494051in}}%
\pgfpathcurveto{\pgfqpoint{12.687519in}{5.501865in}}{\pgfqpoint{12.691909in}{5.512464in}}{\pgfqpoint{12.691909in}{5.523514in}}%
\pgfpathcurveto{\pgfqpoint{12.691909in}{5.534564in}}{\pgfqpoint{12.687519in}{5.545163in}}{\pgfqpoint{12.679706in}{5.552977in}}%
\pgfpathcurveto{\pgfqpoint{12.671892in}{5.560791in}}{\pgfqpoint{12.661293in}{5.565181in}}{\pgfqpoint{12.650243in}{5.565181in}}%
\pgfpathcurveto{\pgfqpoint{12.639193in}{5.565181in}}{\pgfqpoint{12.628594in}{5.560791in}}{\pgfqpoint{12.620780in}{5.552977in}}%
\pgfpathcurveto{\pgfqpoint{12.612966in}{5.545163in}}{\pgfqpoint{12.608576in}{5.534564in}}{\pgfqpoint{12.608576in}{5.523514in}}%
\pgfpathcurveto{\pgfqpoint{12.608576in}{5.512464in}}{\pgfqpoint{12.612966in}{5.501865in}}{\pgfqpoint{12.620780in}{5.494051in}}%
\pgfpathcurveto{\pgfqpoint{12.628594in}{5.486238in}}{\pgfqpoint{12.639193in}{5.481847in}}{\pgfqpoint{12.650243in}{5.481847in}}%
\pgfpathlineto{\pgfqpoint{12.650243in}{5.481847in}}%
\pgfpathclose%
\pgfusepath{stroke}%
\end{pgfscope}%
\begin{pgfscope}%
\pgfpathrectangle{\pgfqpoint{7.512535in}{0.437222in}}{\pgfqpoint{6.275590in}{5.159444in}}%
\pgfusepath{clip}%
\pgfsetbuttcap%
\pgfsetroundjoin%
\pgfsetlinewidth{1.003750pt}%
\definecolor{currentstroke}{rgb}{0.827451,0.827451,0.827451}%
\pgfsetstrokecolor{currentstroke}%
\pgfsetstrokeopacity{0.800000}%
\pgfsetdash{}{0pt}%
\pgfpathmoveto{\pgfqpoint{12.209629in}{5.405098in}}%
\pgfpathcurveto{\pgfqpoint{12.220679in}{5.405098in}}{\pgfqpoint{12.231278in}{5.409488in}}{\pgfqpoint{12.239091in}{5.417302in}}%
\pgfpathcurveto{\pgfqpoint{12.246905in}{5.425116in}}{\pgfqpoint{12.251295in}{5.435715in}}{\pgfqpoint{12.251295in}{5.446765in}}%
\pgfpathcurveto{\pgfqpoint{12.251295in}{5.457815in}}{\pgfqpoint{12.246905in}{5.468414in}}{\pgfqpoint{12.239091in}{5.476228in}}%
\pgfpathcurveto{\pgfqpoint{12.231278in}{5.484041in}}{\pgfqpoint{12.220679in}{5.488431in}}{\pgfqpoint{12.209629in}{5.488431in}}%
\pgfpathcurveto{\pgfqpoint{12.198578in}{5.488431in}}{\pgfqpoint{12.187979in}{5.484041in}}{\pgfqpoint{12.180166in}{5.476228in}}%
\pgfpathcurveto{\pgfqpoint{12.172352in}{5.468414in}}{\pgfqpoint{12.167962in}{5.457815in}}{\pgfqpoint{12.167962in}{5.446765in}}%
\pgfpathcurveto{\pgfqpoint{12.167962in}{5.435715in}}{\pgfqpoint{12.172352in}{5.425116in}}{\pgfqpoint{12.180166in}{5.417302in}}%
\pgfpathcurveto{\pgfqpoint{12.187979in}{5.409488in}}{\pgfqpoint{12.198578in}{5.405098in}}{\pgfqpoint{12.209629in}{5.405098in}}%
\pgfpathlineto{\pgfqpoint{12.209629in}{5.405098in}}%
\pgfpathclose%
\pgfusepath{stroke}%
\end{pgfscope}%
\begin{pgfscope}%
\pgfpathrectangle{\pgfqpoint{7.512535in}{0.437222in}}{\pgfqpoint{6.275590in}{5.159444in}}%
\pgfusepath{clip}%
\pgfsetbuttcap%
\pgfsetroundjoin%
\pgfsetlinewidth{1.003750pt}%
\definecolor{currentstroke}{rgb}{0.827451,0.827451,0.827451}%
\pgfsetstrokecolor{currentstroke}%
\pgfsetstrokeopacity{0.800000}%
\pgfsetdash{}{0pt}%
\pgfpathmoveto{\pgfqpoint{13.101307in}{5.469388in}}%
\pgfpathcurveto{\pgfqpoint{13.112357in}{5.469388in}}{\pgfqpoint{13.122956in}{5.473778in}}{\pgfqpoint{13.130770in}{5.481592in}}%
\pgfpathcurveto{\pgfqpoint{13.138583in}{5.489405in}}{\pgfqpoint{13.142973in}{5.500004in}}{\pgfqpoint{13.142973in}{5.511054in}}%
\pgfpathcurveto{\pgfqpoint{13.142973in}{5.522105in}}{\pgfqpoint{13.138583in}{5.532704in}}{\pgfqpoint{13.130770in}{5.540517in}}%
\pgfpathcurveto{\pgfqpoint{13.122956in}{5.548331in}}{\pgfqpoint{13.112357in}{5.552721in}}{\pgfqpoint{13.101307in}{5.552721in}}%
\pgfpathcurveto{\pgfqpoint{13.090257in}{5.552721in}}{\pgfqpoint{13.079658in}{5.548331in}}{\pgfqpoint{13.071844in}{5.540517in}}%
\pgfpathcurveto{\pgfqpoint{13.064030in}{5.532704in}}{\pgfqpoint{13.059640in}{5.522105in}}{\pgfqpoint{13.059640in}{5.511054in}}%
\pgfpathcurveto{\pgfqpoint{13.059640in}{5.500004in}}{\pgfqpoint{13.064030in}{5.489405in}}{\pgfqpoint{13.071844in}{5.481592in}}%
\pgfpathcurveto{\pgfqpoint{13.079658in}{5.473778in}}{\pgfqpoint{13.090257in}{5.469388in}}{\pgfqpoint{13.101307in}{5.469388in}}%
\pgfpathlineto{\pgfqpoint{13.101307in}{5.469388in}}%
\pgfpathclose%
\pgfusepath{stroke}%
\end{pgfscope}%
\begin{pgfscope}%
\pgfpathrectangle{\pgfqpoint{7.512535in}{0.437222in}}{\pgfqpoint{6.275590in}{5.159444in}}%
\pgfusepath{clip}%
\pgfsetbuttcap%
\pgfsetroundjoin%
\pgfsetlinewidth{1.003750pt}%
\definecolor{currentstroke}{rgb}{0.827451,0.827451,0.827451}%
\pgfsetstrokecolor{currentstroke}%
\pgfsetstrokeopacity{0.800000}%
\pgfsetdash{}{0pt}%
\pgfpathmoveto{\pgfqpoint{10.287229in}{5.087121in}}%
\pgfpathcurveto{\pgfqpoint{10.298279in}{5.087121in}}{\pgfqpoint{10.308878in}{5.091511in}}{\pgfqpoint{10.316692in}{5.099325in}}%
\pgfpathcurveto{\pgfqpoint{10.324505in}{5.107138in}}{\pgfqpoint{10.328896in}{5.117737in}}{\pgfqpoint{10.328896in}{5.128787in}}%
\pgfpathcurveto{\pgfqpoint{10.328896in}{5.139838in}}{\pgfqpoint{10.324505in}{5.150437in}}{\pgfqpoint{10.316692in}{5.158250in}}%
\pgfpathcurveto{\pgfqpoint{10.308878in}{5.166064in}}{\pgfqpoint{10.298279in}{5.170454in}}{\pgfqpoint{10.287229in}{5.170454in}}%
\pgfpathcurveto{\pgfqpoint{10.276179in}{5.170454in}}{\pgfqpoint{10.265580in}{5.166064in}}{\pgfqpoint{10.257766in}{5.158250in}}%
\pgfpathcurveto{\pgfqpoint{10.249953in}{5.150437in}}{\pgfqpoint{10.245562in}{5.139838in}}{\pgfqpoint{10.245562in}{5.128787in}}%
\pgfpathcurveto{\pgfqpoint{10.245562in}{5.117737in}}{\pgfqpoint{10.249953in}{5.107138in}}{\pgfqpoint{10.257766in}{5.099325in}}%
\pgfpathcurveto{\pgfqpoint{10.265580in}{5.091511in}}{\pgfqpoint{10.276179in}{5.087121in}}{\pgfqpoint{10.287229in}{5.087121in}}%
\pgfpathlineto{\pgfqpoint{10.287229in}{5.087121in}}%
\pgfpathclose%
\pgfusepath{stroke}%
\end{pgfscope}%
\begin{pgfscope}%
\pgfpathrectangle{\pgfqpoint{7.512535in}{0.437222in}}{\pgfqpoint{6.275590in}{5.159444in}}%
\pgfusepath{clip}%
\pgfsetbuttcap%
\pgfsetroundjoin%
\pgfsetlinewidth{1.003750pt}%
\definecolor{currentstroke}{rgb}{0.827451,0.827451,0.827451}%
\pgfsetstrokecolor{currentstroke}%
\pgfsetstrokeopacity{0.800000}%
\pgfsetdash{}{0pt}%
\pgfpathmoveto{\pgfqpoint{9.848678in}{4.313606in}}%
\pgfpathcurveto{\pgfqpoint{9.859728in}{4.313606in}}{\pgfqpoint{9.870327in}{4.317996in}}{\pgfqpoint{9.878141in}{4.325810in}}%
\pgfpathcurveto{\pgfqpoint{9.885955in}{4.333623in}}{\pgfqpoint{9.890345in}{4.344222in}}{\pgfqpoint{9.890345in}{4.355272in}}%
\pgfpathcurveto{\pgfqpoint{9.890345in}{4.366323in}}{\pgfqpoint{9.885955in}{4.376922in}}{\pgfqpoint{9.878141in}{4.384735in}}%
\pgfpathcurveto{\pgfqpoint{9.870327in}{4.392549in}}{\pgfqpoint{9.859728in}{4.396939in}}{\pgfqpoint{9.848678in}{4.396939in}}%
\pgfpathcurveto{\pgfqpoint{9.837628in}{4.396939in}}{\pgfqpoint{9.827029in}{4.392549in}}{\pgfqpoint{9.819216in}{4.384735in}}%
\pgfpathcurveto{\pgfqpoint{9.811402in}{4.376922in}}{\pgfqpoint{9.807012in}{4.366323in}}{\pgfqpoint{9.807012in}{4.355272in}}%
\pgfpathcurveto{\pgfqpoint{9.807012in}{4.344222in}}{\pgfqpoint{9.811402in}{4.333623in}}{\pgfqpoint{9.819216in}{4.325810in}}%
\pgfpathcurveto{\pgfqpoint{9.827029in}{4.317996in}}{\pgfqpoint{9.837628in}{4.313606in}}{\pgfqpoint{9.848678in}{4.313606in}}%
\pgfpathlineto{\pgfqpoint{9.848678in}{4.313606in}}%
\pgfpathclose%
\pgfusepath{stroke}%
\end{pgfscope}%
\begin{pgfscope}%
\pgfpathrectangle{\pgfqpoint{7.512535in}{0.437222in}}{\pgfqpoint{6.275590in}{5.159444in}}%
\pgfusepath{clip}%
\pgfsetbuttcap%
\pgfsetroundjoin%
\pgfsetlinewidth{1.003750pt}%
\definecolor{currentstroke}{rgb}{0.827451,0.827451,0.827451}%
\pgfsetstrokecolor{currentstroke}%
\pgfsetstrokeopacity{0.800000}%
\pgfsetdash{}{0pt}%
\pgfpathmoveto{\pgfqpoint{8.277312in}{1.563305in}}%
\pgfpathcurveto{\pgfqpoint{8.288363in}{1.563305in}}{\pgfqpoint{8.298962in}{1.567695in}}{\pgfqpoint{8.306775in}{1.575509in}}%
\pgfpathcurveto{\pgfqpoint{8.314589in}{1.583322in}}{\pgfqpoint{8.318979in}{1.593921in}}{\pgfqpoint{8.318979in}{1.604971in}}%
\pgfpathcurveto{\pgfqpoint{8.318979in}{1.616021in}}{\pgfqpoint{8.314589in}{1.626621in}}{\pgfqpoint{8.306775in}{1.634434in}}%
\pgfpathcurveto{\pgfqpoint{8.298962in}{1.642248in}}{\pgfqpoint{8.288363in}{1.646638in}}{\pgfqpoint{8.277312in}{1.646638in}}%
\pgfpathcurveto{\pgfqpoint{8.266262in}{1.646638in}}{\pgfqpoint{8.255663in}{1.642248in}}{\pgfqpoint{8.247850in}{1.634434in}}%
\pgfpathcurveto{\pgfqpoint{8.240036in}{1.626621in}}{\pgfqpoint{8.235646in}{1.616021in}}{\pgfqpoint{8.235646in}{1.604971in}}%
\pgfpathcurveto{\pgfqpoint{8.235646in}{1.593921in}}{\pgfqpoint{8.240036in}{1.583322in}}{\pgfqpoint{8.247850in}{1.575509in}}%
\pgfpathcurveto{\pgfqpoint{8.255663in}{1.567695in}}{\pgfqpoint{8.266262in}{1.563305in}}{\pgfqpoint{8.277312in}{1.563305in}}%
\pgfpathlineto{\pgfqpoint{8.277312in}{1.563305in}}%
\pgfpathclose%
\pgfusepath{stroke}%
\end{pgfscope}%
\begin{pgfscope}%
\pgfpathrectangle{\pgfqpoint{7.512535in}{0.437222in}}{\pgfqpoint{6.275590in}{5.159444in}}%
\pgfusepath{clip}%
\pgfsetbuttcap%
\pgfsetroundjoin%
\pgfsetlinewidth{1.003750pt}%
\definecolor{currentstroke}{rgb}{0.827451,0.827451,0.827451}%
\pgfsetstrokecolor{currentstroke}%
\pgfsetstrokeopacity{0.800000}%
\pgfsetdash{}{0pt}%
\pgfpathmoveto{\pgfqpoint{12.450009in}{5.541802in}}%
\pgfpathcurveto{\pgfqpoint{12.461060in}{5.541802in}}{\pgfqpoint{12.471659in}{5.546192in}}{\pgfqpoint{12.479472in}{5.554006in}}%
\pgfpathcurveto{\pgfqpoint{12.487286in}{5.561819in}}{\pgfqpoint{12.491676in}{5.572418in}}{\pgfqpoint{12.491676in}{5.583468in}}%
\pgfpathcurveto{\pgfqpoint{12.491676in}{5.594519in}}{\pgfqpoint{12.487286in}{5.605118in}}{\pgfqpoint{12.479472in}{5.612931in}}%
\pgfpathcurveto{\pgfqpoint{12.471659in}{5.620745in}}{\pgfqpoint{12.461060in}{5.625135in}}{\pgfqpoint{12.450009in}{5.625135in}}%
\pgfpathcurveto{\pgfqpoint{12.438959in}{5.625135in}}{\pgfqpoint{12.428360in}{5.620745in}}{\pgfqpoint{12.420547in}{5.612931in}}%
\pgfpathcurveto{\pgfqpoint{12.412733in}{5.605118in}}{\pgfqpoint{12.408343in}{5.594519in}}{\pgfqpoint{12.408343in}{5.583468in}}%
\pgfpathcurveto{\pgfqpoint{12.408343in}{5.572418in}}{\pgfqpoint{12.412733in}{5.561819in}}{\pgfqpoint{12.420547in}{5.554006in}}%
\pgfpathcurveto{\pgfqpoint{12.428360in}{5.546192in}}{\pgfqpoint{12.438959in}{5.541802in}}{\pgfqpoint{12.450009in}{5.541802in}}%
\pgfpathlineto{\pgfqpoint{12.450009in}{5.541802in}}%
\pgfpathclose%
\pgfusepath{stroke}%
\end{pgfscope}%
\begin{pgfscope}%
\pgfpathrectangle{\pgfqpoint{7.512535in}{0.437222in}}{\pgfqpoint{6.275590in}{5.159444in}}%
\pgfusepath{clip}%
\pgfsetbuttcap%
\pgfsetroundjoin%
\pgfsetlinewidth{1.003750pt}%
\definecolor{currentstroke}{rgb}{0.827451,0.827451,0.827451}%
\pgfsetstrokecolor{currentstroke}%
\pgfsetstrokeopacity{0.800000}%
\pgfsetdash{}{0pt}%
\pgfpathmoveto{\pgfqpoint{10.470620in}{3.835584in}}%
\pgfpathcurveto{\pgfqpoint{10.481671in}{3.835584in}}{\pgfqpoint{10.492270in}{3.839975in}}{\pgfqpoint{10.500083in}{3.847788in}}%
\pgfpathcurveto{\pgfqpoint{10.507897in}{3.855602in}}{\pgfqpoint{10.512287in}{3.866201in}}{\pgfqpoint{10.512287in}{3.877251in}}%
\pgfpathcurveto{\pgfqpoint{10.512287in}{3.888301in}}{\pgfqpoint{10.507897in}{3.898900in}}{\pgfqpoint{10.500083in}{3.906714in}}%
\pgfpathcurveto{\pgfqpoint{10.492270in}{3.914528in}}{\pgfqpoint{10.481671in}{3.918918in}}{\pgfqpoint{10.470620in}{3.918918in}}%
\pgfpathcurveto{\pgfqpoint{10.459570in}{3.918918in}}{\pgfqpoint{10.448971in}{3.914528in}}{\pgfqpoint{10.441158in}{3.906714in}}%
\pgfpathcurveto{\pgfqpoint{10.433344in}{3.898900in}}{\pgfqpoint{10.428954in}{3.888301in}}{\pgfqpoint{10.428954in}{3.877251in}}%
\pgfpathcurveto{\pgfqpoint{10.428954in}{3.866201in}}{\pgfqpoint{10.433344in}{3.855602in}}{\pgfqpoint{10.441158in}{3.847788in}}%
\pgfpathcurveto{\pgfqpoint{10.448971in}{3.839975in}}{\pgfqpoint{10.459570in}{3.835584in}}{\pgfqpoint{10.470620in}{3.835584in}}%
\pgfpathlineto{\pgfqpoint{10.470620in}{3.835584in}}%
\pgfpathclose%
\pgfusepath{stroke}%
\end{pgfscope}%
\begin{pgfscope}%
\pgfpathrectangle{\pgfqpoint{7.512535in}{0.437222in}}{\pgfqpoint{6.275590in}{5.159444in}}%
\pgfusepath{clip}%
\pgfsetbuttcap%
\pgfsetroundjoin%
\pgfsetlinewidth{1.003750pt}%
\definecolor{currentstroke}{rgb}{0.827451,0.827451,0.827451}%
\pgfsetstrokecolor{currentstroke}%
\pgfsetstrokeopacity{0.800000}%
\pgfsetdash{}{0pt}%
\pgfpathmoveto{\pgfqpoint{7.983939in}{1.000624in}}%
\pgfpathcurveto{\pgfqpoint{7.994989in}{1.000624in}}{\pgfqpoint{8.005588in}{1.005015in}}{\pgfqpoint{8.013401in}{1.012828in}}%
\pgfpathcurveto{\pgfqpoint{8.021215in}{1.020642in}}{\pgfqpoint{8.025605in}{1.031241in}}{\pgfqpoint{8.025605in}{1.042291in}}%
\pgfpathcurveto{\pgfqpoint{8.025605in}{1.053341in}}{\pgfqpoint{8.021215in}{1.063940in}}{\pgfqpoint{8.013401in}{1.071754in}}%
\pgfpathcurveto{\pgfqpoint{8.005588in}{1.079568in}}{\pgfqpoint{7.994989in}{1.083958in}}{\pgfqpoint{7.983939in}{1.083958in}}%
\pgfpathcurveto{\pgfqpoint{7.972888in}{1.083958in}}{\pgfqpoint{7.962289in}{1.079568in}}{\pgfqpoint{7.954476in}{1.071754in}}%
\pgfpathcurveto{\pgfqpoint{7.946662in}{1.063940in}}{\pgfqpoint{7.942272in}{1.053341in}}{\pgfqpoint{7.942272in}{1.042291in}}%
\pgfpathcurveto{\pgfqpoint{7.942272in}{1.031241in}}{\pgfqpoint{7.946662in}{1.020642in}}{\pgfqpoint{7.954476in}{1.012828in}}%
\pgfpathcurveto{\pgfqpoint{7.962289in}{1.005015in}}{\pgfqpoint{7.972888in}{1.000624in}}{\pgfqpoint{7.983939in}{1.000624in}}%
\pgfpathlineto{\pgfqpoint{7.983939in}{1.000624in}}%
\pgfpathclose%
\pgfusepath{stroke}%
\end{pgfscope}%
\begin{pgfscope}%
\pgfpathrectangle{\pgfqpoint{7.512535in}{0.437222in}}{\pgfqpoint{6.275590in}{5.159444in}}%
\pgfusepath{clip}%
\pgfsetbuttcap%
\pgfsetroundjoin%
\pgfsetlinewidth{1.003750pt}%
\definecolor{currentstroke}{rgb}{0.827451,0.827451,0.827451}%
\pgfsetstrokecolor{currentstroke}%
\pgfsetstrokeopacity{0.800000}%
\pgfsetdash{}{0pt}%
\pgfpathmoveto{\pgfqpoint{10.385573in}{5.144321in}}%
\pgfpathcurveto{\pgfqpoint{10.396623in}{5.144321in}}{\pgfqpoint{10.407222in}{5.148712in}}{\pgfqpoint{10.415036in}{5.156525in}}%
\pgfpathcurveto{\pgfqpoint{10.422849in}{5.164339in}}{\pgfqpoint{10.427240in}{5.174938in}}{\pgfqpoint{10.427240in}{5.185988in}}%
\pgfpathcurveto{\pgfqpoint{10.427240in}{5.197038in}}{\pgfqpoint{10.422849in}{5.207637in}}{\pgfqpoint{10.415036in}{5.215451in}}%
\pgfpathcurveto{\pgfqpoint{10.407222in}{5.223265in}}{\pgfqpoint{10.396623in}{5.227655in}}{\pgfqpoint{10.385573in}{5.227655in}}%
\pgfpathcurveto{\pgfqpoint{10.374523in}{5.227655in}}{\pgfqpoint{10.363924in}{5.223265in}}{\pgfqpoint{10.356110in}{5.215451in}}%
\pgfpathcurveto{\pgfqpoint{10.348297in}{5.207637in}}{\pgfqpoint{10.343906in}{5.197038in}}{\pgfqpoint{10.343906in}{5.185988in}}%
\pgfpathcurveto{\pgfqpoint{10.343906in}{5.174938in}}{\pgfqpoint{10.348297in}{5.164339in}}{\pgfqpoint{10.356110in}{5.156525in}}%
\pgfpathcurveto{\pgfqpoint{10.363924in}{5.148712in}}{\pgfqpoint{10.374523in}{5.144321in}}{\pgfqpoint{10.385573in}{5.144321in}}%
\pgfpathlineto{\pgfqpoint{10.385573in}{5.144321in}}%
\pgfpathclose%
\pgfusepath{stroke}%
\end{pgfscope}%
\begin{pgfscope}%
\pgfpathrectangle{\pgfqpoint{7.512535in}{0.437222in}}{\pgfqpoint{6.275590in}{5.159444in}}%
\pgfusepath{clip}%
\pgfsetbuttcap%
\pgfsetroundjoin%
\pgfsetlinewidth{1.003750pt}%
\definecolor{currentstroke}{rgb}{0.827451,0.827451,0.827451}%
\pgfsetstrokecolor{currentstroke}%
\pgfsetstrokeopacity{0.800000}%
\pgfsetdash{}{0pt}%
\pgfpathmoveto{\pgfqpoint{10.045306in}{4.354371in}}%
\pgfpathcurveto{\pgfqpoint{10.056356in}{4.354371in}}{\pgfqpoint{10.066955in}{4.358761in}}{\pgfqpoint{10.074768in}{4.366575in}}%
\pgfpathcurveto{\pgfqpoint{10.082582in}{4.374389in}}{\pgfqpoint{10.086972in}{4.384988in}}{\pgfqpoint{10.086972in}{4.396038in}}%
\pgfpathcurveto{\pgfqpoint{10.086972in}{4.407088in}}{\pgfqpoint{10.082582in}{4.417687in}}{\pgfqpoint{10.074768in}{4.425500in}}%
\pgfpathcurveto{\pgfqpoint{10.066955in}{4.433314in}}{\pgfqpoint{10.056356in}{4.437704in}}{\pgfqpoint{10.045306in}{4.437704in}}%
\pgfpathcurveto{\pgfqpoint{10.034256in}{4.437704in}}{\pgfqpoint{10.023657in}{4.433314in}}{\pgfqpoint{10.015843in}{4.425500in}}%
\pgfpathcurveto{\pgfqpoint{10.008029in}{4.417687in}}{\pgfqpoint{10.003639in}{4.407088in}}{\pgfqpoint{10.003639in}{4.396038in}}%
\pgfpathcurveto{\pgfqpoint{10.003639in}{4.384988in}}{\pgfqpoint{10.008029in}{4.374389in}}{\pgfqpoint{10.015843in}{4.366575in}}%
\pgfpathcurveto{\pgfqpoint{10.023657in}{4.358761in}}{\pgfqpoint{10.034256in}{4.354371in}}{\pgfqpoint{10.045306in}{4.354371in}}%
\pgfpathlineto{\pgfqpoint{10.045306in}{4.354371in}}%
\pgfpathclose%
\pgfusepath{stroke}%
\end{pgfscope}%
\begin{pgfscope}%
\pgfpathrectangle{\pgfqpoint{7.512535in}{0.437222in}}{\pgfqpoint{6.275590in}{5.159444in}}%
\pgfusepath{clip}%
\pgfsetbuttcap%
\pgfsetroundjoin%
\pgfsetlinewidth{1.003750pt}%
\definecolor{currentstroke}{rgb}{0.827451,0.827451,0.827451}%
\pgfsetstrokecolor{currentstroke}%
\pgfsetstrokeopacity{0.800000}%
\pgfsetdash{}{0pt}%
\pgfpathmoveto{\pgfqpoint{10.873384in}{4.145170in}}%
\pgfpathcurveto{\pgfqpoint{10.884434in}{4.145170in}}{\pgfqpoint{10.895033in}{4.149560in}}{\pgfqpoint{10.902847in}{4.157374in}}%
\pgfpathcurveto{\pgfqpoint{10.910660in}{4.165188in}}{\pgfqpoint{10.915050in}{4.175787in}}{\pgfqpoint{10.915050in}{4.186837in}}%
\pgfpathcurveto{\pgfqpoint{10.915050in}{4.197887in}}{\pgfqpoint{10.910660in}{4.208486in}}{\pgfqpoint{10.902847in}{4.216300in}}%
\pgfpathcurveto{\pgfqpoint{10.895033in}{4.224113in}}{\pgfqpoint{10.884434in}{4.228504in}}{\pgfqpoint{10.873384in}{4.228504in}}%
\pgfpathcurveto{\pgfqpoint{10.862334in}{4.228504in}}{\pgfqpoint{10.851735in}{4.224113in}}{\pgfqpoint{10.843921in}{4.216300in}}%
\pgfpathcurveto{\pgfqpoint{10.836107in}{4.208486in}}{\pgfqpoint{10.831717in}{4.197887in}}{\pgfqpoint{10.831717in}{4.186837in}}%
\pgfpathcurveto{\pgfqpoint{10.831717in}{4.175787in}}{\pgfqpoint{10.836107in}{4.165188in}}{\pgfqpoint{10.843921in}{4.157374in}}%
\pgfpathcurveto{\pgfqpoint{10.851735in}{4.149560in}}{\pgfqpoint{10.862334in}{4.145170in}}{\pgfqpoint{10.873384in}{4.145170in}}%
\pgfpathlineto{\pgfqpoint{10.873384in}{4.145170in}}%
\pgfpathclose%
\pgfusepath{stroke}%
\end{pgfscope}%
\begin{pgfscope}%
\pgfpathrectangle{\pgfqpoint{7.512535in}{0.437222in}}{\pgfqpoint{6.275590in}{5.159444in}}%
\pgfusepath{clip}%
\pgfsetbuttcap%
\pgfsetroundjoin%
\pgfsetlinewidth{1.003750pt}%
\definecolor{currentstroke}{rgb}{0.827451,0.827451,0.827451}%
\pgfsetstrokecolor{currentstroke}%
\pgfsetstrokeopacity{0.800000}%
\pgfsetdash{}{0pt}%
\pgfpathmoveto{\pgfqpoint{7.983939in}{0.909544in}}%
\pgfpathcurveto{\pgfqpoint{7.994989in}{0.909544in}}{\pgfqpoint{8.005588in}{0.913934in}}{\pgfqpoint{8.013401in}{0.921748in}}%
\pgfpathcurveto{\pgfqpoint{8.021215in}{0.929562in}}{\pgfqpoint{8.025605in}{0.940161in}}{\pgfqpoint{8.025605in}{0.951211in}}%
\pgfpathcurveto{\pgfqpoint{8.025605in}{0.962261in}}{\pgfqpoint{8.021215in}{0.972860in}}{\pgfqpoint{8.013401in}{0.980674in}}%
\pgfpathcurveto{\pgfqpoint{8.005588in}{0.988487in}}{\pgfqpoint{7.994989in}{0.992878in}}{\pgfqpoint{7.983939in}{0.992878in}}%
\pgfpathcurveto{\pgfqpoint{7.972888in}{0.992878in}}{\pgfqpoint{7.962289in}{0.988487in}}{\pgfqpoint{7.954476in}{0.980674in}}%
\pgfpathcurveto{\pgfqpoint{7.946662in}{0.972860in}}{\pgfqpoint{7.942272in}{0.962261in}}{\pgfqpoint{7.942272in}{0.951211in}}%
\pgfpathcurveto{\pgfqpoint{7.942272in}{0.940161in}}{\pgfqpoint{7.946662in}{0.929562in}}{\pgfqpoint{7.954476in}{0.921748in}}%
\pgfpathcurveto{\pgfqpoint{7.962289in}{0.913934in}}{\pgfqpoint{7.972888in}{0.909544in}}{\pgfqpoint{7.983939in}{0.909544in}}%
\pgfpathlineto{\pgfqpoint{7.983939in}{0.909544in}}%
\pgfpathclose%
\pgfusepath{stroke}%
\end{pgfscope}%
\begin{pgfscope}%
\pgfpathrectangle{\pgfqpoint{7.512535in}{0.437222in}}{\pgfqpoint{6.275590in}{5.159444in}}%
\pgfusepath{clip}%
\pgfsetbuttcap%
\pgfsetroundjoin%
\pgfsetlinewidth{1.003750pt}%
\definecolor{currentstroke}{rgb}{0.827451,0.827451,0.827451}%
\pgfsetstrokecolor{currentstroke}%
\pgfsetstrokeopacity{0.800000}%
\pgfsetdash{}{0pt}%
\pgfpathmoveto{\pgfqpoint{8.619483in}{1.654089in}}%
\pgfpathcurveto{\pgfqpoint{8.630533in}{1.654089in}}{\pgfqpoint{8.641132in}{1.658480in}}{\pgfqpoint{8.648946in}{1.666293in}}%
\pgfpathcurveto{\pgfqpoint{8.656759in}{1.674107in}}{\pgfqpoint{8.661149in}{1.684706in}}{\pgfqpoint{8.661149in}{1.695756in}}%
\pgfpathcurveto{\pgfqpoint{8.661149in}{1.706806in}}{\pgfqpoint{8.656759in}{1.717405in}}{\pgfqpoint{8.648946in}{1.725219in}}%
\pgfpathcurveto{\pgfqpoint{8.641132in}{1.733033in}}{\pgfqpoint{8.630533in}{1.737423in}}{\pgfqpoint{8.619483in}{1.737423in}}%
\pgfpathcurveto{\pgfqpoint{8.608433in}{1.737423in}}{\pgfqpoint{8.597834in}{1.733033in}}{\pgfqpoint{8.590020in}{1.725219in}}%
\pgfpathcurveto{\pgfqpoint{8.582206in}{1.717405in}}{\pgfqpoint{8.577816in}{1.706806in}}{\pgfqpoint{8.577816in}{1.695756in}}%
\pgfpathcurveto{\pgfqpoint{8.577816in}{1.684706in}}{\pgfqpoint{8.582206in}{1.674107in}}{\pgfqpoint{8.590020in}{1.666293in}}%
\pgfpathcurveto{\pgfqpoint{8.597834in}{1.658480in}}{\pgfqpoint{8.608433in}{1.654089in}}{\pgfqpoint{8.619483in}{1.654089in}}%
\pgfpathlineto{\pgfqpoint{8.619483in}{1.654089in}}%
\pgfpathclose%
\pgfusepath{stroke}%
\end{pgfscope}%
\begin{pgfscope}%
\pgfpathrectangle{\pgfqpoint{7.512535in}{0.437222in}}{\pgfqpoint{6.275590in}{5.159444in}}%
\pgfusepath{clip}%
\pgfsetbuttcap%
\pgfsetroundjoin%
\pgfsetlinewidth{1.003750pt}%
\definecolor{currentstroke}{rgb}{0.827451,0.827451,0.827451}%
\pgfsetstrokecolor{currentstroke}%
\pgfsetstrokeopacity{0.800000}%
\pgfsetdash{}{0pt}%
\pgfpathmoveto{\pgfqpoint{10.385573in}{5.221855in}}%
\pgfpathcurveto{\pgfqpoint{10.396623in}{5.221855in}}{\pgfqpoint{10.407222in}{5.226245in}}{\pgfqpoint{10.415036in}{5.234058in}}%
\pgfpathcurveto{\pgfqpoint{10.422849in}{5.241872in}}{\pgfqpoint{10.427240in}{5.252471in}}{\pgfqpoint{10.427240in}{5.263521in}}%
\pgfpathcurveto{\pgfqpoint{10.427240in}{5.274571in}}{\pgfqpoint{10.422849in}{5.285170in}}{\pgfqpoint{10.415036in}{5.292984in}}%
\pgfpathcurveto{\pgfqpoint{10.407222in}{5.300798in}}{\pgfqpoint{10.396623in}{5.305188in}}{\pgfqpoint{10.385573in}{5.305188in}}%
\pgfpathcurveto{\pgfqpoint{10.374523in}{5.305188in}}{\pgfqpoint{10.363924in}{5.300798in}}{\pgfqpoint{10.356110in}{5.292984in}}%
\pgfpathcurveto{\pgfqpoint{10.348297in}{5.285170in}}{\pgfqpoint{10.343906in}{5.274571in}}{\pgfqpoint{10.343906in}{5.263521in}}%
\pgfpathcurveto{\pgfqpoint{10.343906in}{5.252471in}}{\pgfqpoint{10.348297in}{5.241872in}}{\pgfqpoint{10.356110in}{5.234058in}}%
\pgfpathcurveto{\pgfqpoint{10.363924in}{5.226245in}}{\pgfqpoint{10.374523in}{5.221855in}}{\pgfqpoint{10.385573in}{5.221855in}}%
\pgfpathlineto{\pgfqpoint{10.385573in}{5.221855in}}%
\pgfpathclose%
\pgfusepath{stroke}%
\end{pgfscope}%
\begin{pgfscope}%
\pgfpathrectangle{\pgfqpoint{7.512535in}{0.437222in}}{\pgfqpoint{6.275590in}{5.159444in}}%
\pgfusepath{clip}%
\pgfsetbuttcap%
\pgfsetroundjoin%
\pgfsetlinewidth{1.003750pt}%
\definecolor{currentstroke}{rgb}{0.827451,0.827451,0.827451}%
\pgfsetstrokecolor{currentstroke}%
\pgfsetstrokeopacity{0.800000}%
\pgfsetdash{}{0pt}%
\pgfpathmoveto{\pgfqpoint{9.790014in}{2.556537in}}%
\pgfpathcurveto{\pgfqpoint{9.801064in}{2.556537in}}{\pgfqpoint{9.811663in}{2.560927in}}{\pgfqpoint{9.819477in}{2.568740in}}%
\pgfpathcurveto{\pgfqpoint{9.827290in}{2.576554in}}{\pgfqpoint{9.831681in}{2.587153in}}{\pgfqpoint{9.831681in}{2.598203in}}%
\pgfpathcurveto{\pgfqpoint{9.831681in}{2.609253in}}{\pgfqpoint{9.827290in}{2.619852in}}{\pgfqpoint{9.819477in}{2.627666in}}%
\pgfpathcurveto{\pgfqpoint{9.811663in}{2.635480in}}{\pgfqpoint{9.801064in}{2.639870in}}{\pgfqpoint{9.790014in}{2.639870in}}%
\pgfpathcurveto{\pgfqpoint{9.778964in}{2.639870in}}{\pgfqpoint{9.768365in}{2.635480in}}{\pgfqpoint{9.760551in}{2.627666in}}%
\pgfpathcurveto{\pgfqpoint{9.752737in}{2.619852in}}{\pgfqpoint{9.748347in}{2.609253in}}{\pgfqpoint{9.748347in}{2.598203in}}%
\pgfpathcurveto{\pgfqpoint{9.748347in}{2.587153in}}{\pgfqpoint{9.752737in}{2.576554in}}{\pgfqpoint{9.760551in}{2.568740in}}%
\pgfpathcurveto{\pgfqpoint{9.768365in}{2.560927in}}{\pgfqpoint{9.778964in}{2.556537in}}{\pgfqpoint{9.790014in}{2.556537in}}%
\pgfpathlineto{\pgfqpoint{9.790014in}{2.556537in}}%
\pgfpathclose%
\pgfusepath{stroke}%
\end{pgfscope}%
\begin{pgfscope}%
\pgfpathrectangle{\pgfqpoint{7.512535in}{0.437222in}}{\pgfqpoint{6.275590in}{5.159444in}}%
\pgfusepath{clip}%
\pgfsetbuttcap%
\pgfsetroundjoin%
\pgfsetlinewidth{1.003750pt}%
\definecolor{currentstroke}{rgb}{0.827451,0.827451,0.827451}%
\pgfsetstrokecolor{currentstroke}%
\pgfsetstrokeopacity{0.800000}%
\pgfsetdash{}{0pt}%
\pgfpathmoveto{\pgfqpoint{9.630920in}{3.276112in}}%
\pgfpathcurveto{\pgfqpoint{9.641970in}{3.276112in}}{\pgfqpoint{9.652569in}{3.280502in}}{\pgfqpoint{9.660383in}{3.288316in}}%
\pgfpathcurveto{\pgfqpoint{9.668196in}{3.296130in}}{\pgfqpoint{9.672587in}{3.306729in}}{\pgfqpoint{9.672587in}{3.317779in}}%
\pgfpathcurveto{\pgfqpoint{9.672587in}{3.328829in}}{\pgfqpoint{9.668196in}{3.339428in}}{\pgfqpoint{9.660383in}{3.347241in}}%
\pgfpathcurveto{\pgfqpoint{9.652569in}{3.355055in}}{\pgfqpoint{9.641970in}{3.359445in}}{\pgfqpoint{9.630920in}{3.359445in}}%
\pgfpathcurveto{\pgfqpoint{9.619870in}{3.359445in}}{\pgfqpoint{9.609271in}{3.355055in}}{\pgfqpoint{9.601457in}{3.347241in}}%
\pgfpathcurveto{\pgfqpoint{9.593644in}{3.339428in}}{\pgfqpoint{9.589253in}{3.328829in}}{\pgfqpoint{9.589253in}{3.317779in}}%
\pgfpathcurveto{\pgfqpoint{9.589253in}{3.306729in}}{\pgfqpoint{9.593644in}{3.296130in}}{\pgfqpoint{9.601457in}{3.288316in}}%
\pgfpathcurveto{\pgfqpoint{9.609271in}{3.280502in}}{\pgfqpoint{9.619870in}{3.276112in}}{\pgfqpoint{9.630920in}{3.276112in}}%
\pgfpathlineto{\pgfqpoint{9.630920in}{3.276112in}}%
\pgfpathclose%
\pgfusepath{stroke}%
\end{pgfscope}%
\begin{pgfscope}%
\pgfpathrectangle{\pgfqpoint{7.512535in}{0.437222in}}{\pgfqpoint{6.275590in}{5.159444in}}%
\pgfusepath{clip}%
\pgfsetbuttcap%
\pgfsetroundjoin%
\pgfsetlinewidth{1.003750pt}%
\definecolor{currentstroke}{rgb}{0.827451,0.827451,0.827451}%
\pgfsetstrokecolor{currentstroke}%
\pgfsetstrokeopacity{0.800000}%
\pgfsetdash{}{0pt}%
\pgfpathmoveto{\pgfqpoint{10.818589in}{4.898946in}}%
\pgfpathcurveto{\pgfqpoint{10.829639in}{4.898946in}}{\pgfqpoint{10.840238in}{4.903336in}}{\pgfqpoint{10.848052in}{4.911150in}}%
\pgfpathcurveto{\pgfqpoint{10.855865in}{4.918964in}}{\pgfqpoint{10.860255in}{4.929563in}}{\pgfqpoint{10.860255in}{4.940613in}}%
\pgfpathcurveto{\pgfqpoint{10.860255in}{4.951663in}}{\pgfqpoint{10.855865in}{4.962262in}}{\pgfqpoint{10.848052in}{4.970076in}}%
\pgfpathcurveto{\pgfqpoint{10.840238in}{4.977889in}}{\pgfqpoint{10.829639in}{4.982279in}}{\pgfqpoint{10.818589in}{4.982279in}}%
\pgfpathcurveto{\pgfqpoint{10.807539in}{4.982279in}}{\pgfqpoint{10.796940in}{4.977889in}}{\pgfqpoint{10.789126in}{4.970076in}}%
\pgfpathcurveto{\pgfqpoint{10.781312in}{4.962262in}}{\pgfqpoint{10.776922in}{4.951663in}}{\pgfqpoint{10.776922in}{4.940613in}}%
\pgfpathcurveto{\pgfqpoint{10.776922in}{4.929563in}}{\pgfqpoint{10.781312in}{4.918964in}}{\pgfqpoint{10.789126in}{4.911150in}}%
\pgfpathcurveto{\pgfqpoint{10.796940in}{4.903336in}}{\pgfqpoint{10.807539in}{4.898946in}}{\pgfqpoint{10.818589in}{4.898946in}}%
\pgfpathlineto{\pgfqpoint{10.818589in}{4.898946in}}%
\pgfpathclose%
\pgfusepath{stroke}%
\end{pgfscope}%
\begin{pgfscope}%
\pgfpathrectangle{\pgfqpoint{7.512535in}{0.437222in}}{\pgfqpoint{6.275590in}{5.159444in}}%
\pgfusepath{clip}%
\pgfsetbuttcap%
\pgfsetroundjoin%
\pgfsetlinewidth{1.003750pt}%
\definecolor{currentstroke}{rgb}{0.827451,0.827451,0.827451}%
\pgfsetstrokecolor{currentstroke}%
\pgfsetstrokeopacity{0.800000}%
\pgfsetdash{}{0pt}%
\pgfpathmoveto{\pgfqpoint{10.412487in}{3.527290in}}%
\pgfpathcurveto{\pgfqpoint{10.423537in}{3.527290in}}{\pgfqpoint{10.434136in}{3.531681in}}{\pgfqpoint{10.441950in}{3.539494in}}%
\pgfpathcurveto{\pgfqpoint{10.449764in}{3.547308in}}{\pgfqpoint{10.454154in}{3.557907in}}{\pgfqpoint{10.454154in}{3.568957in}}%
\pgfpathcurveto{\pgfqpoint{10.454154in}{3.580007in}}{\pgfqpoint{10.449764in}{3.590606in}}{\pgfqpoint{10.441950in}{3.598420in}}%
\pgfpathcurveto{\pgfqpoint{10.434136in}{3.606234in}}{\pgfqpoint{10.423537in}{3.610624in}}{\pgfqpoint{10.412487in}{3.610624in}}%
\pgfpathcurveto{\pgfqpoint{10.401437in}{3.610624in}}{\pgfqpoint{10.390838in}{3.606234in}}{\pgfqpoint{10.383025in}{3.598420in}}%
\pgfpathcurveto{\pgfqpoint{10.375211in}{3.590606in}}{\pgfqpoint{10.370821in}{3.580007in}}{\pgfqpoint{10.370821in}{3.568957in}}%
\pgfpathcurveto{\pgfqpoint{10.370821in}{3.557907in}}{\pgfqpoint{10.375211in}{3.547308in}}{\pgfqpoint{10.383025in}{3.539494in}}%
\pgfpathcurveto{\pgfqpoint{10.390838in}{3.531681in}}{\pgfqpoint{10.401437in}{3.527290in}}{\pgfqpoint{10.412487in}{3.527290in}}%
\pgfpathlineto{\pgfqpoint{10.412487in}{3.527290in}}%
\pgfpathclose%
\pgfusepath{stroke}%
\end{pgfscope}%
\begin{pgfscope}%
\pgfpathrectangle{\pgfqpoint{7.512535in}{0.437222in}}{\pgfqpoint{6.275590in}{5.159444in}}%
\pgfusepath{clip}%
\pgfsetbuttcap%
\pgfsetroundjoin%
\pgfsetlinewidth{1.003750pt}%
\definecolor{currentstroke}{rgb}{0.827451,0.827451,0.827451}%
\pgfsetstrokecolor{currentstroke}%
\pgfsetstrokeopacity{0.800000}%
\pgfsetdash{}{0pt}%
\pgfpathmoveto{\pgfqpoint{9.697299in}{3.449334in}}%
\pgfpathcurveto{\pgfqpoint{9.708349in}{3.449334in}}{\pgfqpoint{9.718948in}{3.453725in}}{\pgfqpoint{9.726762in}{3.461538in}}%
\pgfpathcurveto{\pgfqpoint{9.734575in}{3.469352in}}{\pgfqpoint{9.738965in}{3.479951in}}{\pgfqpoint{9.738965in}{3.491001in}}%
\pgfpathcurveto{\pgfqpoint{9.738965in}{3.502051in}}{\pgfqpoint{9.734575in}{3.512650in}}{\pgfqpoint{9.726762in}{3.520464in}}%
\pgfpathcurveto{\pgfqpoint{9.718948in}{3.528277in}}{\pgfqpoint{9.708349in}{3.532668in}}{\pgfqpoint{9.697299in}{3.532668in}}%
\pgfpathcurveto{\pgfqpoint{9.686249in}{3.532668in}}{\pgfqpoint{9.675650in}{3.528277in}}{\pgfqpoint{9.667836in}{3.520464in}}%
\pgfpathcurveto{\pgfqpoint{9.660022in}{3.512650in}}{\pgfqpoint{9.655632in}{3.502051in}}{\pgfqpoint{9.655632in}{3.491001in}}%
\pgfpathcurveto{\pgfqpoint{9.655632in}{3.479951in}}{\pgfqpoint{9.660022in}{3.469352in}}{\pgfqpoint{9.667836in}{3.461538in}}%
\pgfpathcurveto{\pgfqpoint{9.675650in}{3.453725in}}{\pgfqpoint{9.686249in}{3.449334in}}{\pgfqpoint{9.697299in}{3.449334in}}%
\pgfpathlineto{\pgfqpoint{9.697299in}{3.449334in}}%
\pgfpathclose%
\pgfusepath{stroke}%
\end{pgfscope}%
\begin{pgfscope}%
\pgfpathrectangle{\pgfqpoint{7.512535in}{0.437222in}}{\pgfqpoint{6.275590in}{5.159444in}}%
\pgfusepath{clip}%
\pgfsetbuttcap%
\pgfsetroundjoin%
\pgfsetlinewidth{1.003750pt}%
\definecolor{currentstroke}{rgb}{0.827451,0.827451,0.827451}%
\pgfsetstrokecolor{currentstroke}%
\pgfsetstrokeopacity{0.800000}%
\pgfsetdash{}{0pt}%
\pgfpathmoveto{\pgfqpoint{7.860715in}{0.988624in}}%
\pgfpathcurveto{\pgfqpoint{7.871765in}{0.988624in}}{\pgfqpoint{7.882364in}{0.993014in}}{\pgfqpoint{7.890178in}{1.000828in}}%
\pgfpathcurveto{\pgfqpoint{7.897991in}{1.008641in}}{\pgfqpoint{7.902381in}{1.019240in}}{\pgfqpoint{7.902381in}{1.030290in}}%
\pgfpathcurveto{\pgfqpoint{7.902381in}{1.041341in}}{\pgfqpoint{7.897991in}{1.051940in}}{\pgfqpoint{7.890178in}{1.059753in}}%
\pgfpathcurveto{\pgfqpoint{7.882364in}{1.067567in}}{\pgfqpoint{7.871765in}{1.071957in}}{\pgfqpoint{7.860715in}{1.071957in}}%
\pgfpathcurveto{\pgfqpoint{7.849665in}{1.071957in}}{\pgfqpoint{7.839066in}{1.067567in}}{\pgfqpoint{7.831252in}{1.059753in}}%
\pgfpathcurveto{\pgfqpoint{7.823438in}{1.051940in}}{\pgfqpoint{7.819048in}{1.041341in}}{\pgfqpoint{7.819048in}{1.030290in}}%
\pgfpathcurveto{\pgfqpoint{7.819048in}{1.019240in}}{\pgfqpoint{7.823438in}{1.008641in}}{\pgfqpoint{7.831252in}{1.000828in}}%
\pgfpathcurveto{\pgfqpoint{7.839066in}{0.993014in}}{\pgfqpoint{7.849665in}{0.988624in}}{\pgfqpoint{7.860715in}{0.988624in}}%
\pgfpathlineto{\pgfqpoint{7.860715in}{0.988624in}}%
\pgfpathclose%
\pgfusepath{stroke}%
\end{pgfscope}%
\begin{pgfscope}%
\pgfpathrectangle{\pgfqpoint{7.512535in}{0.437222in}}{\pgfqpoint{6.275590in}{5.159444in}}%
\pgfusepath{clip}%
\pgfsetbuttcap%
\pgfsetroundjoin%
\pgfsetlinewidth{1.003750pt}%
\definecolor{currentstroke}{rgb}{0.827451,0.827451,0.827451}%
\pgfsetstrokecolor{currentstroke}%
\pgfsetstrokeopacity{0.800000}%
\pgfsetdash{}{0pt}%
\pgfpathmoveto{\pgfqpoint{8.607098in}{2.557652in}}%
\pgfpathcurveto{\pgfqpoint{8.618149in}{2.557652in}}{\pgfqpoint{8.628748in}{2.562042in}}{\pgfqpoint{8.636561in}{2.569856in}}%
\pgfpathcurveto{\pgfqpoint{8.644375in}{2.577669in}}{\pgfqpoint{8.648765in}{2.588268in}}{\pgfqpoint{8.648765in}{2.599318in}}%
\pgfpathcurveto{\pgfqpoint{8.648765in}{2.610369in}}{\pgfqpoint{8.644375in}{2.620968in}}{\pgfqpoint{8.636561in}{2.628781in}}%
\pgfpathcurveto{\pgfqpoint{8.628748in}{2.636595in}}{\pgfqpoint{8.618149in}{2.640985in}}{\pgfqpoint{8.607098in}{2.640985in}}%
\pgfpathcurveto{\pgfqpoint{8.596048in}{2.640985in}}{\pgfqpoint{8.585449in}{2.636595in}}{\pgfqpoint{8.577636in}{2.628781in}}%
\pgfpathcurveto{\pgfqpoint{8.569822in}{2.620968in}}{\pgfqpoint{8.565432in}{2.610369in}}{\pgfqpoint{8.565432in}{2.599318in}}%
\pgfpathcurveto{\pgfqpoint{8.565432in}{2.588268in}}{\pgfqpoint{8.569822in}{2.577669in}}{\pgfqpoint{8.577636in}{2.569856in}}%
\pgfpathcurveto{\pgfqpoint{8.585449in}{2.562042in}}{\pgfqpoint{8.596048in}{2.557652in}}{\pgfqpoint{8.607098in}{2.557652in}}%
\pgfpathlineto{\pgfqpoint{8.607098in}{2.557652in}}%
\pgfpathclose%
\pgfusepath{stroke}%
\end{pgfscope}%
\begin{pgfscope}%
\pgfpathrectangle{\pgfqpoint{7.512535in}{0.437222in}}{\pgfqpoint{6.275590in}{5.159444in}}%
\pgfusepath{clip}%
\pgfsetbuttcap%
\pgfsetroundjoin%
\pgfsetlinewidth{1.003750pt}%
\definecolor{currentstroke}{rgb}{0.827451,0.827451,0.827451}%
\pgfsetstrokecolor{currentstroke}%
\pgfsetstrokeopacity{0.800000}%
\pgfsetdash{}{0pt}%
\pgfpathmoveto{\pgfqpoint{7.706938in}{0.540767in}}%
\pgfpathcurveto{\pgfqpoint{7.717988in}{0.540767in}}{\pgfqpoint{7.728587in}{0.545157in}}{\pgfqpoint{7.736400in}{0.552971in}}%
\pgfpathcurveto{\pgfqpoint{7.744214in}{0.560784in}}{\pgfqpoint{7.748604in}{0.571383in}}{\pgfqpoint{7.748604in}{0.582434in}}%
\pgfpathcurveto{\pgfqpoint{7.748604in}{0.593484in}}{\pgfqpoint{7.744214in}{0.604083in}}{\pgfqpoint{7.736400in}{0.611896in}}%
\pgfpathcurveto{\pgfqpoint{7.728587in}{0.619710in}}{\pgfqpoint{7.717988in}{0.624100in}}{\pgfqpoint{7.706938in}{0.624100in}}%
\pgfpathcurveto{\pgfqpoint{7.695888in}{0.624100in}}{\pgfqpoint{7.685289in}{0.619710in}}{\pgfqpoint{7.677475in}{0.611896in}}%
\pgfpathcurveto{\pgfqpoint{7.669661in}{0.604083in}}{\pgfqpoint{7.665271in}{0.593484in}}{\pgfqpoint{7.665271in}{0.582434in}}%
\pgfpathcurveto{\pgfqpoint{7.665271in}{0.571383in}}{\pgfqpoint{7.669661in}{0.560784in}}{\pgfqpoint{7.677475in}{0.552971in}}%
\pgfpathcurveto{\pgfqpoint{7.685289in}{0.545157in}}{\pgfqpoint{7.695888in}{0.540767in}}{\pgfqpoint{7.706938in}{0.540767in}}%
\pgfpathlineto{\pgfqpoint{7.706938in}{0.540767in}}%
\pgfpathclose%
\pgfusepath{stroke}%
\end{pgfscope}%
\begin{pgfscope}%
\pgfpathrectangle{\pgfqpoint{7.512535in}{0.437222in}}{\pgfqpoint{6.275590in}{5.159444in}}%
\pgfusepath{clip}%
\pgfsetbuttcap%
\pgfsetroundjoin%
\pgfsetlinewidth{1.003750pt}%
\definecolor{currentstroke}{rgb}{0.827451,0.827451,0.827451}%
\pgfsetstrokecolor{currentstroke}%
\pgfsetstrokeopacity{0.800000}%
\pgfsetdash{}{0pt}%
\pgfpathmoveto{\pgfqpoint{10.150452in}{4.500549in}}%
\pgfpathcurveto{\pgfqpoint{10.161502in}{4.500549in}}{\pgfqpoint{10.172101in}{4.504939in}}{\pgfqpoint{10.179914in}{4.512753in}}%
\pgfpathcurveto{\pgfqpoint{10.187728in}{4.520566in}}{\pgfqpoint{10.192118in}{4.531165in}}{\pgfqpoint{10.192118in}{4.542215in}}%
\pgfpathcurveto{\pgfqpoint{10.192118in}{4.553266in}}{\pgfqpoint{10.187728in}{4.563865in}}{\pgfqpoint{10.179914in}{4.571678in}}%
\pgfpathcurveto{\pgfqpoint{10.172101in}{4.579492in}}{\pgfqpoint{10.161502in}{4.583882in}}{\pgfqpoint{10.150452in}{4.583882in}}%
\pgfpathcurveto{\pgfqpoint{10.139401in}{4.583882in}}{\pgfqpoint{10.128802in}{4.579492in}}{\pgfqpoint{10.120989in}{4.571678in}}%
\pgfpathcurveto{\pgfqpoint{10.113175in}{4.563865in}}{\pgfqpoint{10.108785in}{4.553266in}}{\pgfqpoint{10.108785in}{4.542215in}}%
\pgfpathcurveto{\pgfqpoint{10.108785in}{4.531165in}}{\pgfqpoint{10.113175in}{4.520566in}}{\pgfqpoint{10.120989in}{4.512753in}}%
\pgfpathcurveto{\pgfqpoint{10.128802in}{4.504939in}}{\pgfqpoint{10.139401in}{4.500549in}}{\pgfqpoint{10.150452in}{4.500549in}}%
\pgfpathlineto{\pgfqpoint{10.150452in}{4.500549in}}%
\pgfpathclose%
\pgfusepath{stroke}%
\end{pgfscope}%
\begin{pgfscope}%
\pgfpathrectangle{\pgfqpoint{7.512535in}{0.437222in}}{\pgfqpoint{6.275590in}{5.159444in}}%
\pgfusepath{clip}%
\pgfsetbuttcap%
\pgfsetroundjoin%
\pgfsetlinewidth{1.003750pt}%
\definecolor{currentstroke}{rgb}{0.827451,0.827451,0.827451}%
\pgfsetstrokecolor{currentstroke}%
\pgfsetstrokeopacity{0.800000}%
\pgfsetdash{}{0pt}%
\pgfpathmoveto{\pgfqpoint{9.862917in}{4.507322in}}%
\pgfpathcurveto{\pgfqpoint{9.873967in}{4.507322in}}{\pgfqpoint{9.884566in}{4.511712in}}{\pgfqpoint{9.892380in}{4.519525in}}%
\pgfpathcurveto{\pgfqpoint{9.900194in}{4.527339in}}{\pgfqpoint{9.904584in}{4.537938in}}{\pgfqpoint{9.904584in}{4.548988in}}%
\pgfpathcurveto{\pgfqpoint{9.904584in}{4.560038in}}{\pgfqpoint{9.900194in}{4.570637in}}{\pgfqpoint{9.892380in}{4.578451in}}%
\pgfpathcurveto{\pgfqpoint{9.884566in}{4.586265in}}{\pgfqpoint{9.873967in}{4.590655in}}{\pgfqpoint{9.862917in}{4.590655in}}%
\pgfpathcurveto{\pgfqpoint{9.851867in}{4.590655in}}{\pgfqpoint{9.841268in}{4.586265in}}{\pgfqpoint{9.833454in}{4.578451in}}%
\pgfpathcurveto{\pgfqpoint{9.825641in}{4.570637in}}{\pgfqpoint{9.821250in}{4.560038in}}{\pgfqpoint{9.821250in}{4.548988in}}%
\pgfpathcurveto{\pgfqpoint{9.821250in}{4.537938in}}{\pgfqpoint{9.825641in}{4.527339in}}{\pgfqpoint{9.833454in}{4.519525in}}%
\pgfpathcurveto{\pgfqpoint{9.841268in}{4.511712in}}{\pgfqpoint{9.851867in}{4.507322in}}{\pgfqpoint{9.862917in}{4.507322in}}%
\pgfpathlineto{\pgfqpoint{9.862917in}{4.507322in}}%
\pgfpathclose%
\pgfusepath{stroke}%
\end{pgfscope}%
\begin{pgfscope}%
\pgfpathrectangle{\pgfqpoint{7.512535in}{0.437222in}}{\pgfqpoint{6.275590in}{5.159444in}}%
\pgfusepath{clip}%
\pgfsetbuttcap%
\pgfsetroundjoin%
\pgfsetlinewidth{1.003750pt}%
\definecolor{currentstroke}{rgb}{0.827451,0.827451,0.827451}%
\pgfsetstrokecolor{currentstroke}%
\pgfsetstrokeopacity{0.800000}%
\pgfsetdash{}{0pt}%
\pgfpathmoveto{\pgfqpoint{10.881762in}{4.622183in}}%
\pgfpathcurveto{\pgfqpoint{10.892812in}{4.622183in}}{\pgfqpoint{10.903411in}{4.626573in}}{\pgfqpoint{10.911225in}{4.634387in}}%
\pgfpathcurveto{\pgfqpoint{10.919039in}{4.642201in}}{\pgfqpoint{10.923429in}{4.652800in}}{\pgfqpoint{10.923429in}{4.663850in}}%
\pgfpathcurveto{\pgfqpoint{10.923429in}{4.674900in}}{\pgfqpoint{10.919039in}{4.685499in}}{\pgfqpoint{10.911225in}{4.693313in}}%
\pgfpathcurveto{\pgfqpoint{10.903411in}{4.701126in}}{\pgfqpoint{10.892812in}{4.705516in}}{\pgfqpoint{10.881762in}{4.705516in}}%
\pgfpathcurveto{\pgfqpoint{10.870712in}{4.705516in}}{\pgfqpoint{10.860113in}{4.701126in}}{\pgfqpoint{10.852300in}{4.693313in}}%
\pgfpathcurveto{\pgfqpoint{10.844486in}{4.685499in}}{\pgfqpoint{10.840096in}{4.674900in}}{\pgfqpoint{10.840096in}{4.663850in}}%
\pgfpathcurveto{\pgfqpoint{10.840096in}{4.652800in}}{\pgfqpoint{10.844486in}{4.642201in}}{\pgfqpoint{10.852300in}{4.634387in}}%
\pgfpathcurveto{\pgfqpoint{10.860113in}{4.626573in}}{\pgfqpoint{10.870712in}{4.622183in}}{\pgfqpoint{10.881762in}{4.622183in}}%
\pgfpathlineto{\pgfqpoint{10.881762in}{4.622183in}}%
\pgfpathclose%
\pgfusepath{stroke}%
\end{pgfscope}%
\begin{pgfscope}%
\pgfpathrectangle{\pgfqpoint{7.512535in}{0.437222in}}{\pgfqpoint{6.275590in}{5.159444in}}%
\pgfusepath{clip}%
\pgfsetbuttcap%
\pgfsetroundjoin%
\pgfsetlinewidth{1.003750pt}%
\definecolor{currentstroke}{rgb}{0.827451,0.827451,0.827451}%
\pgfsetstrokecolor{currentstroke}%
\pgfsetstrokeopacity{0.800000}%
\pgfsetdash{}{0pt}%
\pgfpathmoveto{\pgfqpoint{9.001197in}{3.326629in}}%
\pgfpathcurveto{\pgfqpoint{9.012248in}{3.326629in}}{\pgfqpoint{9.022847in}{3.331019in}}{\pgfqpoint{9.030660in}{3.338833in}}%
\pgfpathcurveto{\pgfqpoint{9.038474in}{3.346646in}}{\pgfqpoint{9.042864in}{3.357246in}}{\pgfqpoint{9.042864in}{3.368296in}}%
\pgfpathcurveto{\pgfqpoint{9.042864in}{3.379346in}}{\pgfqpoint{9.038474in}{3.389945in}}{\pgfqpoint{9.030660in}{3.397758in}}%
\pgfpathcurveto{\pgfqpoint{9.022847in}{3.405572in}}{\pgfqpoint{9.012248in}{3.409962in}}{\pgfqpoint{9.001197in}{3.409962in}}%
\pgfpathcurveto{\pgfqpoint{8.990147in}{3.409962in}}{\pgfqpoint{8.979548in}{3.405572in}}{\pgfqpoint{8.971735in}{3.397758in}}%
\pgfpathcurveto{\pgfqpoint{8.963921in}{3.389945in}}{\pgfqpoint{8.959531in}{3.379346in}}{\pgfqpoint{8.959531in}{3.368296in}}%
\pgfpathcurveto{\pgfqpoint{8.959531in}{3.357246in}}{\pgfqpoint{8.963921in}{3.346646in}}{\pgfqpoint{8.971735in}{3.338833in}}%
\pgfpathcurveto{\pgfqpoint{8.979548in}{3.331019in}}{\pgfqpoint{8.990147in}{3.326629in}}{\pgfqpoint{9.001197in}{3.326629in}}%
\pgfpathlineto{\pgfqpoint{9.001197in}{3.326629in}}%
\pgfpathclose%
\pgfusepath{stroke}%
\end{pgfscope}%
\begin{pgfscope}%
\pgfpathrectangle{\pgfqpoint{7.512535in}{0.437222in}}{\pgfqpoint{6.275590in}{5.159444in}}%
\pgfusepath{clip}%
\pgfsetbuttcap%
\pgfsetroundjoin%
\pgfsetlinewidth{1.003750pt}%
\definecolor{currentstroke}{rgb}{0.827451,0.827451,0.827451}%
\pgfsetstrokecolor{currentstroke}%
\pgfsetstrokeopacity{0.800000}%
\pgfsetdash{}{0pt}%
\pgfpathmoveto{\pgfqpoint{9.578307in}{2.959155in}}%
\pgfpathcurveto{\pgfqpoint{9.589357in}{2.959155in}}{\pgfqpoint{9.599956in}{2.963545in}}{\pgfqpoint{9.607770in}{2.971359in}}%
\pgfpathcurveto{\pgfqpoint{9.615584in}{2.979173in}}{\pgfqpoint{9.619974in}{2.989772in}}{\pgfqpoint{9.619974in}{3.000822in}}%
\pgfpathcurveto{\pgfqpoint{9.619974in}{3.011872in}}{\pgfqpoint{9.615584in}{3.022471in}}{\pgfqpoint{9.607770in}{3.030284in}}%
\pgfpathcurveto{\pgfqpoint{9.599956in}{3.038098in}}{\pgfqpoint{9.589357in}{3.042488in}}{\pgfqpoint{9.578307in}{3.042488in}}%
\pgfpathcurveto{\pgfqpoint{9.567257in}{3.042488in}}{\pgfqpoint{9.556658in}{3.038098in}}{\pgfqpoint{9.548845in}{3.030284in}}%
\pgfpathcurveto{\pgfqpoint{9.541031in}{3.022471in}}{\pgfqpoint{9.536641in}{3.011872in}}{\pgfqpoint{9.536641in}{3.000822in}}%
\pgfpathcurveto{\pgfqpoint{9.536641in}{2.989772in}}{\pgfqpoint{9.541031in}{2.979173in}}{\pgfqpoint{9.548845in}{2.971359in}}%
\pgfpathcurveto{\pgfqpoint{9.556658in}{2.963545in}}{\pgfqpoint{9.567257in}{2.959155in}}{\pgfqpoint{9.578307in}{2.959155in}}%
\pgfpathlineto{\pgfqpoint{9.578307in}{2.959155in}}%
\pgfpathclose%
\pgfusepath{stroke}%
\end{pgfscope}%
\begin{pgfscope}%
\pgfpathrectangle{\pgfqpoint{7.512535in}{0.437222in}}{\pgfqpoint{6.275590in}{5.159444in}}%
\pgfusepath{clip}%
\pgfsetbuttcap%
\pgfsetroundjoin%
\pgfsetlinewidth{1.003750pt}%
\definecolor{currentstroke}{rgb}{0.827451,0.827451,0.827451}%
\pgfsetstrokecolor{currentstroke}%
\pgfsetstrokeopacity{0.800000}%
\pgfsetdash{}{0pt}%
\pgfpathmoveto{\pgfqpoint{10.704475in}{4.930530in}}%
\pgfpathcurveto{\pgfqpoint{10.715525in}{4.930530in}}{\pgfqpoint{10.726124in}{4.934921in}}{\pgfqpoint{10.733938in}{4.942734in}}%
\pgfpathcurveto{\pgfqpoint{10.741751in}{4.950548in}}{\pgfqpoint{10.746142in}{4.961147in}}{\pgfqpoint{10.746142in}{4.972197in}}%
\pgfpathcurveto{\pgfqpoint{10.746142in}{4.983247in}}{\pgfqpoint{10.741751in}{4.993846in}}{\pgfqpoint{10.733938in}{5.001660in}}%
\pgfpathcurveto{\pgfqpoint{10.726124in}{5.009473in}}{\pgfqpoint{10.715525in}{5.013864in}}{\pgfqpoint{10.704475in}{5.013864in}}%
\pgfpathcurveto{\pgfqpoint{10.693425in}{5.013864in}}{\pgfqpoint{10.682826in}{5.009473in}}{\pgfqpoint{10.675012in}{5.001660in}}%
\pgfpathcurveto{\pgfqpoint{10.667198in}{4.993846in}}{\pgfqpoint{10.662808in}{4.983247in}}{\pgfqpoint{10.662808in}{4.972197in}}%
\pgfpathcurveto{\pgfqpoint{10.662808in}{4.961147in}}{\pgfqpoint{10.667198in}{4.950548in}}{\pgfqpoint{10.675012in}{4.942734in}}%
\pgfpathcurveto{\pgfqpoint{10.682826in}{4.934921in}}{\pgfqpoint{10.693425in}{4.930530in}}{\pgfqpoint{10.704475in}{4.930530in}}%
\pgfpathlineto{\pgfqpoint{10.704475in}{4.930530in}}%
\pgfpathclose%
\pgfusepath{stroke}%
\end{pgfscope}%
\begin{pgfscope}%
\pgfpathrectangle{\pgfqpoint{7.512535in}{0.437222in}}{\pgfqpoint{6.275590in}{5.159444in}}%
\pgfusepath{clip}%
\pgfsetbuttcap%
\pgfsetroundjoin%
\pgfsetlinewidth{1.003750pt}%
\definecolor{currentstroke}{rgb}{0.827451,0.827451,0.827451}%
\pgfsetstrokecolor{currentstroke}%
\pgfsetstrokeopacity{0.800000}%
\pgfsetdash{}{0pt}%
\pgfpathmoveto{\pgfqpoint{9.540773in}{2.701763in}}%
\pgfpathcurveto{\pgfqpoint{9.551823in}{2.701763in}}{\pgfqpoint{9.562422in}{2.706153in}}{\pgfqpoint{9.570236in}{2.713967in}}%
\pgfpathcurveto{\pgfqpoint{9.578049in}{2.721780in}}{\pgfqpoint{9.582440in}{2.732379in}}{\pgfqpoint{9.582440in}{2.743429in}}%
\pgfpathcurveto{\pgfqpoint{9.582440in}{2.754479in}}{\pgfqpoint{9.578049in}{2.765078in}}{\pgfqpoint{9.570236in}{2.772892in}}%
\pgfpathcurveto{\pgfqpoint{9.562422in}{2.780706in}}{\pgfqpoint{9.551823in}{2.785096in}}{\pgfqpoint{9.540773in}{2.785096in}}%
\pgfpathcurveto{\pgfqpoint{9.529723in}{2.785096in}}{\pgfqpoint{9.519124in}{2.780706in}}{\pgfqpoint{9.511310in}{2.772892in}}%
\pgfpathcurveto{\pgfqpoint{9.503497in}{2.765078in}}{\pgfqpoint{9.499106in}{2.754479in}}{\pgfqpoint{9.499106in}{2.743429in}}%
\pgfpathcurveto{\pgfqpoint{9.499106in}{2.732379in}}{\pgfqpoint{9.503497in}{2.721780in}}{\pgfqpoint{9.511310in}{2.713967in}}%
\pgfpathcurveto{\pgfqpoint{9.519124in}{2.706153in}}{\pgfqpoint{9.529723in}{2.701763in}}{\pgfqpoint{9.540773in}{2.701763in}}%
\pgfpathlineto{\pgfqpoint{9.540773in}{2.701763in}}%
\pgfpathclose%
\pgfusepath{stroke}%
\end{pgfscope}%
\begin{pgfscope}%
\pgfpathrectangle{\pgfqpoint{7.512535in}{0.437222in}}{\pgfqpoint{6.275590in}{5.159444in}}%
\pgfusepath{clip}%
\pgfsetbuttcap%
\pgfsetroundjoin%
\pgfsetlinewidth{1.003750pt}%
\definecolor{currentstroke}{rgb}{0.827451,0.827451,0.827451}%
\pgfsetstrokecolor{currentstroke}%
\pgfsetstrokeopacity{0.800000}%
\pgfsetdash{}{0pt}%
\pgfpathmoveto{\pgfqpoint{8.624162in}{1.722485in}}%
\pgfpathcurveto{\pgfqpoint{8.635212in}{1.722485in}}{\pgfqpoint{8.645811in}{1.726876in}}{\pgfqpoint{8.653624in}{1.734689in}}%
\pgfpathcurveto{\pgfqpoint{8.661438in}{1.742503in}}{\pgfqpoint{8.665828in}{1.753102in}}{\pgfqpoint{8.665828in}{1.764152in}}%
\pgfpathcurveto{\pgfqpoint{8.665828in}{1.775202in}}{\pgfqpoint{8.661438in}{1.785801in}}{\pgfqpoint{8.653624in}{1.793615in}}%
\pgfpathcurveto{\pgfqpoint{8.645811in}{1.801428in}}{\pgfqpoint{8.635212in}{1.805819in}}{\pgfqpoint{8.624162in}{1.805819in}}%
\pgfpathcurveto{\pgfqpoint{8.613112in}{1.805819in}}{\pgfqpoint{8.602513in}{1.801428in}}{\pgfqpoint{8.594699in}{1.793615in}}%
\pgfpathcurveto{\pgfqpoint{8.586885in}{1.785801in}}{\pgfqpoint{8.582495in}{1.775202in}}{\pgfqpoint{8.582495in}{1.764152in}}%
\pgfpathcurveto{\pgfqpoint{8.582495in}{1.753102in}}{\pgfqpoint{8.586885in}{1.742503in}}{\pgfqpoint{8.594699in}{1.734689in}}%
\pgfpathcurveto{\pgfqpoint{8.602513in}{1.726876in}}{\pgfqpoint{8.613112in}{1.722485in}}{\pgfqpoint{8.624162in}{1.722485in}}%
\pgfpathlineto{\pgfqpoint{8.624162in}{1.722485in}}%
\pgfpathclose%
\pgfusepath{stroke}%
\end{pgfscope}%
\begin{pgfscope}%
\pgfpathrectangle{\pgfqpoint{7.512535in}{0.437222in}}{\pgfqpoint{6.275590in}{5.159444in}}%
\pgfusepath{clip}%
\pgfsetbuttcap%
\pgfsetroundjoin%
\pgfsetlinewidth{1.003750pt}%
\definecolor{currentstroke}{rgb}{0.827451,0.827451,0.827451}%
\pgfsetstrokecolor{currentstroke}%
\pgfsetstrokeopacity{0.800000}%
\pgfsetdash{}{0pt}%
\pgfpathmoveto{\pgfqpoint{9.581399in}{3.231111in}}%
\pgfpathcurveto{\pgfqpoint{9.592449in}{3.231111in}}{\pgfqpoint{9.603048in}{3.235501in}}{\pgfqpoint{9.610862in}{3.243315in}}%
\pgfpathcurveto{\pgfqpoint{9.618675in}{3.251128in}}{\pgfqpoint{9.623065in}{3.261727in}}{\pgfqpoint{9.623065in}{3.272778in}}%
\pgfpathcurveto{\pgfqpoint{9.623065in}{3.283828in}}{\pgfqpoint{9.618675in}{3.294427in}}{\pgfqpoint{9.610862in}{3.302240in}}%
\pgfpathcurveto{\pgfqpoint{9.603048in}{3.310054in}}{\pgfqpoint{9.592449in}{3.314444in}}{\pgfqpoint{9.581399in}{3.314444in}}%
\pgfpathcurveto{\pgfqpoint{9.570349in}{3.314444in}}{\pgfqpoint{9.559750in}{3.310054in}}{\pgfqpoint{9.551936in}{3.302240in}}%
\pgfpathcurveto{\pgfqpoint{9.544122in}{3.294427in}}{\pgfqpoint{9.539732in}{3.283828in}}{\pgfqpoint{9.539732in}{3.272778in}}%
\pgfpathcurveto{\pgfqpoint{9.539732in}{3.261727in}}{\pgfqpoint{9.544122in}{3.251128in}}{\pgfqpoint{9.551936in}{3.243315in}}%
\pgfpathcurveto{\pgfqpoint{9.559750in}{3.235501in}}{\pgfqpoint{9.570349in}{3.231111in}}{\pgfqpoint{9.581399in}{3.231111in}}%
\pgfpathlineto{\pgfqpoint{9.581399in}{3.231111in}}%
\pgfpathclose%
\pgfusepath{stroke}%
\end{pgfscope}%
\begin{pgfscope}%
\pgfpathrectangle{\pgfqpoint{7.512535in}{0.437222in}}{\pgfqpoint{6.275590in}{5.159444in}}%
\pgfusepath{clip}%
\pgfsetbuttcap%
\pgfsetroundjoin%
\pgfsetlinewidth{1.003750pt}%
\definecolor{currentstroke}{rgb}{0.827451,0.827451,0.827451}%
\pgfsetstrokecolor{currentstroke}%
\pgfsetstrokeopacity{0.800000}%
\pgfsetdash{}{0pt}%
\pgfpathmoveto{\pgfqpoint{10.413616in}{4.447570in}}%
\pgfpathcurveto{\pgfqpoint{10.424666in}{4.447570in}}{\pgfqpoint{10.435265in}{4.451960in}}{\pgfqpoint{10.443078in}{4.459773in}}%
\pgfpathcurveto{\pgfqpoint{10.450892in}{4.467587in}}{\pgfqpoint{10.455282in}{4.478186in}}{\pgfqpoint{10.455282in}{4.489236in}}%
\pgfpathcurveto{\pgfqpoint{10.455282in}{4.500286in}}{\pgfqpoint{10.450892in}{4.510885in}}{\pgfqpoint{10.443078in}{4.518699in}}%
\pgfpathcurveto{\pgfqpoint{10.435265in}{4.526513in}}{\pgfqpoint{10.424666in}{4.530903in}}{\pgfqpoint{10.413616in}{4.530903in}}%
\pgfpathcurveto{\pgfqpoint{10.402565in}{4.530903in}}{\pgfqpoint{10.391966in}{4.526513in}}{\pgfqpoint{10.384153in}{4.518699in}}%
\pgfpathcurveto{\pgfqpoint{10.376339in}{4.510885in}}{\pgfqpoint{10.371949in}{4.500286in}}{\pgfqpoint{10.371949in}{4.489236in}}%
\pgfpathcurveto{\pgfqpoint{10.371949in}{4.478186in}}{\pgfqpoint{10.376339in}{4.467587in}}{\pgfqpoint{10.384153in}{4.459773in}}%
\pgfpathcurveto{\pgfqpoint{10.391966in}{4.451960in}}{\pgfqpoint{10.402565in}{4.447570in}}{\pgfqpoint{10.413616in}{4.447570in}}%
\pgfpathlineto{\pgfqpoint{10.413616in}{4.447570in}}%
\pgfpathclose%
\pgfusepath{stroke}%
\end{pgfscope}%
\begin{pgfscope}%
\pgfpathrectangle{\pgfqpoint{7.512535in}{0.437222in}}{\pgfqpoint{6.275590in}{5.159444in}}%
\pgfusepath{clip}%
\pgfsetbuttcap%
\pgfsetroundjoin%
\pgfsetlinewidth{1.003750pt}%
\definecolor{currentstroke}{rgb}{0.827451,0.827451,0.827451}%
\pgfsetstrokecolor{currentstroke}%
\pgfsetstrokeopacity{0.800000}%
\pgfsetdash{}{0pt}%
\pgfpathmoveto{\pgfqpoint{9.615954in}{3.788157in}}%
\pgfpathcurveto{\pgfqpoint{9.627004in}{3.788157in}}{\pgfqpoint{9.637603in}{3.792548in}}{\pgfqpoint{9.645416in}{3.800361in}}%
\pgfpathcurveto{\pgfqpoint{9.653230in}{3.808175in}}{\pgfqpoint{9.657620in}{3.818774in}}{\pgfqpoint{9.657620in}{3.829824in}}%
\pgfpathcurveto{\pgfqpoint{9.657620in}{3.840874in}}{\pgfqpoint{9.653230in}{3.851473in}}{\pgfqpoint{9.645416in}{3.859287in}}%
\pgfpathcurveto{\pgfqpoint{9.637603in}{3.867100in}}{\pgfqpoint{9.627004in}{3.871491in}}{\pgfqpoint{9.615954in}{3.871491in}}%
\pgfpathcurveto{\pgfqpoint{9.604903in}{3.871491in}}{\pgfqpoint{9.594304in}{3.867100in}}{\pgfqpoint{9.586491in}{3.859287in}}%
\pgfpathcurveto{\pgfqpoint{9.578677in}{3.851473in}}{\pgfqpoint{9.574287in}{3.840874in}}{\pgfqpoint{9.574287in}{3.829824in}}%
\pgfpathcurveto{\pgfqpoint{9.574287in}{3.818774in}}{\pgfqpoint{9.578677in}{3.808175in}}{\pgfqpoint{9.586491in}{3.800361in}}%
\pgfpathcurveto{\pgfqpoint{9.594304in}{3.792548in}}{\pgfqpoint{9.604903in}{3.788157in}}{\pgfqpoint{9.615954in}{3.788157in}}%
\pgfpathlineto{\pgfqpoint{9.615954in}{3.788157in}}%
\pgfpathclose%
\pgfusepath{stroke}%
\end{pgfscope}%
\begin{pgfscope}%
\pgfpathrectangle{\pgfqpoint{7.512535in}{0.437222in}}{\pgfqpoint{6.275590in}{5.159444in}}%
\pgfusepath{clip}%
\pgfsetbuttcap%
\pgfsetroundjoin%
\pgfsetlinewidth{1.003750pt}%
\definecolor{currentstroke}{rgb}{0.827451,0.827451,0.827451}%
\pgfsetstrokecolor{currentstroke}%
\pgfsetstrokeopacity{0.800000}%
\pgfsetdash{}{0pt}%
\pgfpathmoveto{\pgfqpoint{10.438440in}{5.355498in}}%
\pgfpathcurveto{\pgfqpoint{10.449490in}{5.355498in}}{\pgfqpoint{10.460089in}{5.359888in}}{\pgfqpoint{10.467903in}{5.367701in}}%
\pgfpathcurveto{\pgfqpoint{10.475717in}{5.375515in}}{\pgfqpoint{10.480107in}{5.386114in}}{\pgfqpoint{10.480107in}{5.397164in}}%
\pgfpathcurveto{\pgfqpoint{10.480107in}{5.408214in}}{\pgfqpoint{10.475717in}{5.418813in}}{\pgfqpoint{10.467903in}{5.426627in}}%
\pgfpathcurveto{\pgfqpoint{10.460089in}{5.434441in}}{\pgfqpoint{10.449490in}{5.438831in}}{\pgfqpoint{10.438440in}{5.438831in}}%
\pgfpathcurveto{\pgfqpoint{10.427390in}{5.438831in}}{\pgfqpoint{10.416791in}{5.434441in}}{\pgfqpoint{10.408977in}{5.426627in}}%
\pgfpathcurveto{\pgfqpoint{10.401164in}{5.418813in}}{\pgfqpoint{10.396773in}{5.408214in}}{\pgfqpoint{10.396773in}{5.397164in}}%
\pgfpathcurveto{\pgfqpoint{10.396773in}{5.386114in}}{\pgfqpoint{10.401164in}{5.375515in}}{\pgfqpoint{10.408977in}{5.367701in}}%
\pgfpathcurveto{\pgfqpoint{10.416791in}{5.359888in}}{\pgfqpoint{10.427390in}{5.355498in}}{\pgfqpoint{10.438440in}{5.355498in}}%
\pgfpathlineto{\pgfqpoint{10.438440in}{5.355498in}}%
\pgfpathclose%
\pgfusepath{stroke}%
\end{pgfscope}%
\begin{pgfscope}%
\pgfpathrectangle{\pgfqpoint{7.512535in}{0.437222in}}{\pgfqpoint{6.275590in}{5.159444in}}%
\pgfusepath{clip}%
\pgfsetbuttcap%
\pgfsetroundjoin%
\pgfsetlinewidth{1.003750pt}%
\definecolor{currentstroke}{rgb}{0.827451,0.827451,0.827451}%
\pgfsetstrokecolor{currentstroke}%
\pgfsetstrokeopacity{0.800000}%
\pgfsetdash{}{0pt}%
\pgfpathmoveto{\pgfqpoint{11.443872in}{5.184186in}}%
\pgfpathcurveto{\pgfqpoint{11.454922in}{5.184186in}}{\pgfqpoint{11.465521in}{5.188576in}}{\pgfqpoint{11.473335in}{5.196390in}}%
\pgfpathcurveto{\pgfqpoint{11.481148in}{5.204203in}}{\pgfqpoint{11.485539in}{5.214802in}}{\pgfqpoint{11.485539in}{5.225852in}}%
\pgfpathcurveto{\pgfqpoint{11.485539in}{5.236903in}}{\pgfqpoint{11.481148in}{5.247502in}}{\pgfqpoint{11.473335in}{5.255315in}}%
\pgfpathcurveto{\pgfqpoint{11.465521in}{5.263129in}}{\pgfqpoint{11.454922in}{5.267519in}}{\pgfqpoint{11.443872in}{5.267519in}}%
\pgfpathcurveto{\pgfqpoint{11.432822in}{5.267519in}}{\pgfqpoint{11.422223in}{5.263129in}}{\pgfqpoint{11.414409in}{5.255315in}}%
\pgfpathcurveto{\pgfqpoint{11.406596in}{5.247502in}}{\pgfqpoint{11.402205in}{5.236903in}}{\pgfqpoint{11.402205in}{5.225852in}}%
\pgfpathcurveto{\pgfqpoint{11.402205in}{5.214802in}}{\pgfqpoint{11.406596in}{5.204203in}}{\pgfqpoint{11.414409in}{5.196390in}}%
\pgfpathcurveto{\pgfqpoint{11.422223in}{5.188576in}}{\pgfqpoint{11.432822in}{5.184186in}}{\pgfqpoint{11.443872in}{5.184186in}}%
\pgfpathlineto{\pgfqpoint{11.443872in}{5.184186in}}%
\pgfpathclose%
\pgfusepath{stroke}%
\end{pgfscope}%
\begin{pgfscope}%
\pgfpathrectangle{\pgfqpoint{7.512535in}{0.437222in}}{\pgfqpoint{6.275590in}{5.159444in}}%
\pgfusepath{clip}%
\pgfsetbuttcap%
\pgfsetroundjoin%
\pgfsetlinewidth{1.003750pt}%
\definecolor{currentstroke}{rgb}{0.827451,0.827451,0.827451}%
\pgfsetstrokecolor{currentstroke}%
\pgfsetstrokeopacity{0.800000}%
\pgfsetdash{}{0pt}%
\pgfpathmoveto{\pgfqpoint{11.634893in}{5.376413in}}%
\pgfpathcurveto{\pgfqpoint{11.645943in}{5.376413in}}{\pgfqpoint{11.656542in}{5.380803in}}{\pgfqpoint{11.664356in}{5.388617in}}%
\pgfpathcurveto{\pgfqpoint{11.672170in}{5.396430in}}{\pgfqpoint{11.676560in}{5.407029in}}{\pgfqpoint{11.676560in}{5.418079in}}%
\pgfpathcurveto{\pgfqpoint{11.676560in}{5.429129in}}{\pgfqpoint{11.672170in}{5.439728in}}{\pgfqpoint{11.664356in}{5.447542in}}%
\pgfpathcurveto{\pgfqpoint{11.656542in}{5.455356in}}{\pgfqpoint{11.645943in}{5.459746in}}{\pgfqpoint{11.634893in}{5.459746in}}%
\pgfpathcurveto{\pgfqpoint{11.623843in}{5.459746in}}{\pgfqpoint{11.613244in}{5.455356in}}{\pgfqpoint{11.605430in}{5.447542in}}%
\pgfpathcurveto{\pgfqpoint{11.597617in}{5.439728in}}{\pgfqpoint{11.593227in}{5.429129in}}{\pgfqpoint{11.593227in}{5.418079in}}%
\pgfpathcurveto{\pgfqpoint{11.593227in}{5.407029in}}{\pgfqpoint{11.597617in}{5.396430in}}{\pgfqpoint{11.605430in}{5.388617in}}%
\pgfpathcurveto{\pgfqpoint{11.613244in}{5.380803in}}{\pgfqpoint{11.623843in}{5.376413in}}{\pgfqpoint{11.634893in}{5.376413in}}%
\pgfpathlineto{\pgfqpoint{11.634893in}{5.376413in}}%
\pgfpathclose%
\pgfusepath{stroke}%
\end{pgfscope}%
\begin{pgfscope}%
\pgfpathrectangle{\pgfqpoint{7.512535in}{0.437222in}}{\pgfqpoint{6.275590in}{5.159444in}}%
\pgfusepath{clip}%
\pgfsetbuttcap%
\pgfsetroundjoin%
\pgfsetlinewidth{1.003750pt}%
\definecolor{currentstroke}{rgb}{0.827451,0.827451,0.827451}%
\pgfsetstrokecolor{currentstroke}%
\pgfsetstrokeopacity{0.800000}%
\pgfsetdash{}{0pt}%
\pgfpathmoveto{\pgfqpoint{9.060406in}{3.385314in}}%
\pgfpathcurveto{\pgfqpoint{9.071456in}{3.385314in}}{\pgfqpoint{9.082055in}{3.389704in}}{\pgfqpoint{9.089869in}{3.397518in}}%
\pgfpathcurveto{\pgfqpoint{9.097682in}{3.405332in}}{\pgfqpoint{9.102073in}{3.415931in}}{\pgfqpoint{9.102073in}{3.426981in}}%
\pgfpathcurveto{\pgfqpoint{9.102073in}{3.438031in}}{\pgfqpoint{9.097682in}{3.448630in}}{\pgfqpoint{9.089869in}{3.456444in}}%
\pgfpathcurveto{\pgfqpoint{9.082055in}{3.464257in}}{\pgfqpoint{9.071456in}{3.468647in}}{\pgfqpoint{9.060406in}{3.468647in}}%
\pgfpathcurveto{\pgfqpoint{9.049356in}{3.468647in}}{\pgfqpoint{9.038757in}{3.464257in}}{\pgfqpoint{9.030943in}{3.456444in}}%
\pgfpathcurveto{\pgfqpoint{9.023129in}{3.448630in}}{\pgfqpoint{9.018739in}{3.438031in}}{\pgfqpoint{9.018739in}{3.426981in}}%
\pgfpathcurveto{\pgfqpoint{9.018739in}{3.415931in}}{\pgfqpoint{9.023129in}{3.405332in}}{\pgfqpoint{9.030943in}{3.397518in}}%
\pgfpathcurveto{\pgfqpoint{9.038757in}{3.389704in}}{\pgfqpoint{9.049356in}{3.385314in}}{\pgfqpoint{9.060406in}{3.385314in}}%
\pgfpathlineto{\pgfqpoint{9.060406in}{3.385314in}}%
\pgfpathclose%
\pgfusepath{stroke}%
\end{pgfscope}%
\begin{pgfscope}%
\pgfpathrectangle{\pgfqpoint{7.512535in}{0.437222in}}{\pgfqpoint{6.275590in}{5.159444in}}%
\pgfusepath{clip}%
\pgfsetbuttcap%
\pgfsetroundjoin%
\pgfsetlinewidth{1.003750pt}%
\definecolor{currentstroke}{rgb}{0.827451,0.827451,0.827451}%
\pgfsetstrokecolor{currentstroke}%
\pgfsetstrokeopacity{0.800000}%
\pgfsetdash{}{0pt}%
\pgfpathmoveto{\pgfqpoint{11.629363in}{5.304399in}}%
\pgfpathcurveto{\pgfqpoint{11.640413in}{5.304399in}}{\pgfqpoint{11.651012in}{5.308790in}}{\pgfqpoint{11.658826in}{5.316603in}}%
\pgfpathcurveto{\pgfqpoint{11.666639in}{5.324417in}}{\pgfqpoint{11.671030in}{5.335016in}}{\pgfqpoint{11.671030in}{5.346066in}}%
\pgfpathcurveto{\pgfqpoint{11.671030in}{5.357116in}}{\pgfqpoint{11.666639in}{5.367715in}}{\pgfqpoint{11.658826in}{5.375529in}}%
\pgfpathcurveto{\pgfqpoint{11.651012in}{5.383342in}}{\pgfqpoint{11.640413in}{5.387733in}}{\pgfqpoint{11.629363in}{5.387733in}}%
\pgfpathcurveto{\pgfqpoint{11.618313in}{5.387733in}}{\pgfqpoint{11.607714in}{5.383342in}}{\pgfqpoint{11.599900in}{5.375529in}}%
\pgfpathcurveto{\pgfqpoint{11.592087in}{5.367715in}}{\pgfqpoint{11.587696in}{5.357116in}}{\pgfqpoint{11.587696in}{5.346066in}}%
\pgfpathcurveto{\pgfqpoint{11.587696in}{5.335016in}}{\pgfqpoint{11.592087in}{5.324417in}}{\pgfqpoint{11.599900in}{5.316603in}}%
\pgfpathcurveto{\pgfqpoint{11.607714in}{5.308790in}}{\pgfqpoint{11.618313in}{5.304399in}}{\pgfqpoint{11.629363in}{5.304399in}}%
\pgfpathlineto{\pgfqpoint{11.629363in}{5.304399in}}%
\pgfpathclose%
\pgfusepath{stroke}%
\end{pgfscope}%
\begin{pgfscope}%
\pgfpathrectangle{\pgfqpoint{7.512535in}{0.437222in}}{\pgfqpoint{6.275590in}{5.159444in}}%
\pgfusepath{clip}%
\pgfsetbuttcap%
\pgfsetroundjoin%
\pgfsetlinewidth{1.003750pt}%
\definecolor{currentstroke}{rgb}{0.827451,0.827451,0.827451}%
\pgfsetstrokecolor{currentstroke}%
\pgfsetstrokeopacity{0.800000}%
\pgfsetdash{}{0pt}%
\pgfpathmoveto{\pgfqpoint{7.562575in}{0.493855in}}%
\pgfpathcurveto{\pgfqpoint{7.573625in}{0.493855in}}{\pgfqpoint{7.584224in}{0.498245in}}{\pgfqpoint{7.592038in}{0.506059in}}%
\pgfpathcurveto{\pgfqpoint{7.599851in}{0.513872in}}{\pgfqpoint{7.604241in}{0.524471in}}{\pgfqpoint{7.604241in}{0.535522in}}%
\pgfpathcurveto{\pgfqpoint{7.604241in}{0.546572in}}{\pgfqpoint{7.599851in}{0.557171in}}{\pgfqpoint{7.592038in}{0.564984in}}%
\pgfpathcurveto{\pgfqpoint{7.584224in}{0.572798in}}{\pgfqpoint{7.573625in}{0.577188in}}{\pgfqpoint{7.562575in}{0.577188in}}%
\pgfpathcurveto{\pgfqpoint{7.551525in}{0.577188in}}{\pgfqpoint{7.540926in}{0.572798in}}{\pgfqpoint{7.533112in}{0.564984in}}%
\pgfpathcurveto{\pgfqpoint{7.525298in}{0.557171in}}{\pgfqpoint{7.520908in}{0.546572in}}{\pgfqpoint{7.520908in}{0.535522in}}%
\pgfpathcurveto{\pgfqpoint{7.520908in}{0.524471in}}{\pgfqpoint{7.525298in}{0.513872in}}{\pgfqpoint{7.533112in}{0.506059in}}%
\pgfpathcurveto{\pgfqpoint{7.540926in}{0.498245in}}{\pgfqpoint{7.551525in}{0.493855in}}{\pgfqpoint{7.562575in}{0.493855in}}%
\pgfpathlineto{\pgfqpoint{7.562575in}{0.493855in}}%
\pgfpathclose%
\pgfusepath{stroke}%
\end{pgfscope}%
\begin{pgfscope}%
\pgfpathrectangle{\pgfqpoint{7.512535in}{0.437222in}}{\pgfqpoint{6.275590in}{5.159444in}}%
\pgfusepath{clip}%
\pgfsetbuttcap%
\pgfsetroundjoin%
\pgfsetlinewidth{1.003750pt}%
\definecolor{currentstroke}{rgb}{0.827451,0.827451,0.827451}%
\pgfsetstrokecolor{currentstroke}%
\pgfsetstrokeopacity{0.800000}%
\pgfsetdash{}{0pt}%
\pgfpathmoveto{\pgfqpoint{13.725648in}{5.537440in}}%
\pgfpathcurveto{\pgfqpoint{13.736699in}{5.537440in}}{\pgfqpoint{13.747298in}{5.541830in}}{\pgfqpoint{13.755111in}{5.549643in}}%
\pgfpathcurveto{\pgfqpoint{13.762925in}{5.557457in}}{\pgfqpoint{13.767315in}{5.568056in}}{\pgfqpoint{13.767315in}{5.579106in}}%
\pgfpathcurveto{\pgfqpoint{13.767315in}{5.590156in}}{\pgfqpoint{13.762925in}{5.600755in}}{\pgfqpoint{13.755111in}{5.608569in}}%
\pgfpathcurveto{\pgfqpoint{13.747298in}{5.616383in}}{\pgfqpoint{13.736699in}{5.620773in}}{\pgfqpoint{13.725648in}{5.620773in}}%
\pgfpathcurveto{\pgfqpoint{13.714598in}{5.620773in}}{\pgfqpoint{13.703999in}{5.616383in}}{\pgfqpoint{13.696186in}{5.608569in}}%
\pgfpathcurveto{\pgfqpoint{13.688372in}{5.600755in}}{\pgfqpoint{13.683982in}{5.590156in}}{\pgfqpoint{13.683982in}{5.579106in}}%
\pgfpathcurveto{\pgfqpoint{13.683982in}{5.568056in}}{\pgfqpoint{13.688372in}{5.557457in}}{\pgfqpoint{13.696186in}{5.549643in}}%
\pgfpathcurveto{\pgfqpoint{13.703999in}{5.541830in}}{\pgfqpoint{13.714598in}{5.537440in}}{\pgfqpoint{13.725648in}{5.537440in}}%
\pgfpathlineto{\pgfqpoint{13.725648in}{5.537440in}}%
\pgfpathclose%
\pgfusepath{stroke}%
\end{pgfscope}%
\begin{pgfscope}%
\pgfpathrectangle{\pgfqpoint{7.512535in}{0.437222in}}{\pgfqpoint{6.275590in}{5.159444in}}%
\pgfusepath{clip}%
\pgfsetbuttcap%
\pgfsetroundjoin%
\pgfsetlinewidth{1.003750pt}%
\definecolor{currentstroke}{rgb}{0.827451,0.827451,0.827451}%
\pgfsetstrokecolor{currentstroke}%
\pgfsetstrokeopacity{0.800000}%
\pgfsetdash{}{0pt}%
\pgfpathmoveto{\pgfqpoint{12.873966in}{5.431219in}}%
\pgfpathcurveto{\pgfqpoint{12.885016in}{5.431219in}}{\pgfqpoint{12.895615in}{5.435610in}}{\pgfqpoint{12.903429in}{5.443423in}}%
\pgfpathcurveto{\pgfqpoint{12.911243in}{5.451237in}}{\pgfqpoint{12.915633in}{5.461836in}}{\pgfqpoint{12.915633in}{5.472886in}}%
\pgfpathcurveto{\pgfqpoint{12.915633in}{5.483936in}}{\pgfqpoint{12.911243in}{5.494535in}}{\pgfqpoint{12.903429in}{5.502349in}}%
\pgfpathcurveto{\pgfqpoint{12.895615in}{5.510163in}}{\pgfqpoint{12.885016in}{5.514553in}}{\pgfqpoint{12.873966in}{5.514553in}}%
\pgfpathcurveto{\pgfqpoint{12.862916in}{5.514553in}}{\pgfqpoint{12.852317in}{5.510163in}}{\pgfqpoint{12.844503in}{5.502349in}}%
\pgfpathcurveto{\pgfqpoint{12.836690in}{5.494535in}}{\pgfqpoint{12.832300in}{5.483936in}}{\pgfqpoint{12.832300in}{5.472886in}}%
\pgfpathcurveto{\pgfqpoint{12.832300in}{5.461836in}}{\pgfqpoint{12.836690in}{5.451237in}}{\pgfqpoint{12.844503in}{5.443423in}}%
\pgfpathcurveto{\pgfqpoint{12.852317in}{5.435610in}}{\pgfqpoint{12.862916in}{5.431219in}}{\pgfqpoint{12.873966in}{5.431219in}}%
\pgfpathlineto{\pgfqpoint{12.873966in}{5.431219in}}%
\pgfpathclose%
\pgfusepath{stroke}%
\end{pgfscope}%
\begin{pgfscope}%
\pgfpathrectangle{\pgfqpoint{7.512535in}{0.437222in}}{\pgfqpoint{6.275590in}{5.159444in}}%
\pgfusepath{clip}%
\pgfsetbuttcap%
\pgfsetroundjoin%
\pgfsetlinewidth{1.003750pt}%
\definecolor{currentstroke}{rgb}{0.827451,0.827451,0.827451}%
\pgfsetstrokecolor{currentstroke}%
\pgfsetstrokeopacity{0.800000}%
\pgfsetdash{}{0pt}%
\pgfpathmoveto{\pgfqpoint{13.581463in}{5.528743in}}%
\pgfpathcurveto{\pgfqpoint{13.592513in}{5.528743in}}{\pgfqpoint{13.603113in}{5.533133in}}{\pgfqpoint{13.610926in}{5.540947in}}%
\pgfpathcurveto{\pgfqpoint{13.618740in}{5.548760in}}{\pgfqpoint{13.623130in}{5.559359in}}{\pgfqpoint{13.623130in}{5.570410in}}%
\pgfpathcurveto{\pgfqpoint{13.623130in}{5.581460in}}{\pgfqpoint{13.618740in}{5.592059in}}{\pgfqpoint{13.610926in}{5.599872in}}%
\pgfpathcurveto{\pgfqpoint{13.603113in}{5.607686in}}{\pgfqpoint{13.592513in}{5.612076in}}{\pgfqpoint{13.581463in}{5.612076in}}%
\pgfpathcurveto{\pgfqpoint{13.570413in}{5.612076in}}{\pgfqpoint{13.559814in}{5.607686in}}{\pgfqpoint{13.552001in}{5.599872in}}%
\pgfpathcurveto{\pgfqpoint{13.544187in}{5.592059in}}{\pgfqpoint{13.539797in}{5.581460in}}{\pgfqpoint{13.539797in}{5.570410in}}%
\pgfpathcurveto{\pgfqpoint{13.539797in}{5.559359in}}{\pgfqpoint{13.544187in}{5.548760in}}{\pgfqpoint{13.552001in}{5.540947in}}%
\pgfpathcurveto{\pgfqpoint{13.559814in}{5.533133in}}{\pgfqpoint{13.570413in}{5.528743in}}{\pgfqpoint{13.581463in}{5.528743in}}%
\pgfpathlineto{\pgfqpoint{13.581463in}{5.528743in}}%
\pgfpathclose%
\pgfusepath{stroke}%
\end{pgfscope}%
\begin{pgfscope}%
\pgfpathrectangle{\pgfqpoint{7.512535in}{0.437222in}}{\pgfqpoint{6.275590in}{5.159444in}}%
\pgfusepath{clip}%
\pgfsetbuttcap%
\pgfsetroundjoin%
\pgfsetlinewidth{1.003750pt}%
\definecolor{currentstroke}{rgb}{0.827451,0.827451,0.827451}%
\pgfsetstrokecolor{currentstroke}%
\pgfsetstrokeopacity{0.800000}%
\pgfsetdash{}{0pt}%
\pgfpathmoveto{\pgfqpoint{13.273901in}{5.501584in}}%
\pgfpathcurveto{\pgfqpoint{13.284951in}{5.501584in}}{\pgfqpoint{13.295550in}{5.505974in}}{\pgfqpoint{13.303364in}{5.513787in}}%
\pgfpathcurveto{\pgfqpoint{13.311177in}{5.521601in}}{\pgfqpoint{13.315568in}{5.532200in}}{\pgfqpoint{13.315568in}{5.543250in}}%
\pgfpathcurveto{\pgfqpoint{13.315568in}{5.554300in}}{\pgfqpoint{13.311177in}{5.564899in}}{\pgfqpoint{13.303364in}{5.572713in}}%
\pgfpathcurveto{\pgfqpoint{13.295550in}{5.580527in}}{\pgfqpoint{13.284951in}{5.584917in}}{\pgfqpoint{13.273901in}{5.584917in}}%
\pgfpathcurveto{\pgfqpoint{13.262851in}{5.584917in}}{\pgfqpoint{13.252252in}{5.580527in}}{\pgfqpoint{13.244438in}{5.572713in}}%
\pgfpathcurveto{\pgfqpoint{13.236625in}{5.564899in}}{\pgfqpoint{13.232234in}{5.554300in}}{\pgfqpoint{13.232234in}{5.543250in}}%
\pgfpathcurveto{\pgfqpoint{13.232234in}{5.532200in}}{\pgfqpoint{13.236625in}{5.521601in}}{\pgfqpoint{13.244438in}{5.513787in}}%
\pgfpathcurveto{\pgfqpoint{13.252252in}{5.505974in}}{\pgfqpoint{13.262851in}{5.501584in}}{\pgfqpoint{13.273901in}{5.501584in}}%
\pgfpathlineto{\pgfqpoint{13.273901in}{5.501584in}}%
\pgfpathclose%
\pgfusepath{stroke}%
\end{pgfscope}%
\begin{pgfscope}%
\pgfpathrectangle{\pgfqpoint{7.512535in}{0.437222in}}{\pgfqpoint{6.275590in}{5.159444in}}%
\pgfusepath{clip}%
\pgfsetbuttcap%
\pgfsetroundjoin%
\pgfsetlinewidth{1.003750pt}%
\definecolor{currentstroke}{rgb}{0.827451,0.827451,0.827451}%
\pgfsetstrokecolor{currentstroke}%
\pgfsetstrokeopacity{0.800000}%
\pgfsetdash{}{0pt}%
\pgfpathmoveto{\pgfqpoint{7.711198in}{0.615504in}}%
\pgfpathcurveto{\pgfqpoint{7.722248in}{0.615504in}}{\pgfqpoint{7.732847in}{0.619894in}}{\pgfqpoint{7.740661in}{0.627708in}}%
\pgfpathcurveto{\pgfqpoint{7.748474in}{0.635522in}}{\pgfqpoint{7.752865in}{0.646121in}}{\pgfqpoint{7.752865in}{0.657171in}}%
\pgfpathcurveto{\pgfqpoint{7.752865in}{0.668221in}}{\pgfqpoint{7.748474in}{0.678820in}}{\pgfqpoint{7.740661in}{0.686633in}}%
\pgfpathcurveto{\pgfqpoint{7.732847in}{0.694447in}}{\pgfqpoint{7.722248in}{0.698837in}}{\pgfqpoint{7.711198in}{0.698837in}}%
\pgfpathcurveto{\pgfqpoint{7.700148in}{0.698837in}}{\pgfqpoint{7.689549in}{0.694447in}}{\pgfqpoint{7.681735in}{0.686633in}}%
\pgfpathcurveto{\pgfqpoint{7.673921in}{0.678820in}}{\pgfqpoint{7.669531in}{0.668221in}}{\pgfqpoint{7.669531in}{0.657171in}}%
\pgfpathcurveto{\pgfqpoint{7.669531in}{0.646121in}}{\pgfqpoint{7.673921in}{0.635522in}}{\pgfqpoint{7.681735in}{0.627708in}}%
\pgfpathcurveto{\pgfqpoint{7.689549in}{0.619894in}}{\pgfqpoint{7.700148in}{0.615504in}}{\pgfqpoint{7.711198in}{0.615504in}}%
\pgfpathlineto{\pgfqpoint{7.711198in}{0.615504in}}%
\pgfpathclose%
\pgfusepath{stroke}%
\end{pgfscope}%
\begin{pgfscope}%
\pgfpathrectangle{\pgfqpoint{7.512535in}{0.437222in}}{\pgfqpoint{6.275590in}{5.159444in}}%
\pgfusepath{clip}%
\pgfsetbuttcap%
\pgfsetroundjoin%
\pgfsetlinewidth{1.003750pt}%
\definecolor{currentstroke}{rgb}{0.827451,0.827451,0.827451}%
\pgfsetstrokecolor{currentstroke}%
\pgfsetstrokeopacity{0.800000}%
\pgfsetdash{}{0pt}%
\pgfpathmoveto{\pgfqpoint{13.101307in}{5.417061in}}%
\pgfpathcurveto{\pgfqpoint{13.112357in}{5.417061in}}{\pgfqpoint{13.122956in}{5.421451in}}{\pgfqpoint{13.130770in}{5.429265in}}%
\pgfpathcurveto{\pgfqpoint{13.138583in}{5.437079in}}{\pgfqpoint{13.142973in}{5.447678in}}{\pgfqpoint{13.142973in}{5.458728in}}%
\pgfpathcurveto{\pgfqpoint{13.142973in}{5.469778in}}{\pgfqpoint{13.138583in}{5.480377in}}{\pgfqpoint{13.130770in}{5.488191in}}%
\pgfpathcurveto{\pgfqpoint{13.122956in}{5.496004in}}{\pgfqpoint{13.112357in}{5.500395in}}{\pgfqpoint{13.101307in}{5.500395in}}%
\pgfpathcurveto{\pgfqpoint{13.090257in}{5.500395in}}{\pgfqpoint{13.079658in}{5.496004in}}{\pgfqpoint{13.071844in}{5.488191in}}%
\pgfpathcurveto{\pgfqpoint{13.064030in}{5.480377in}}{\pgfqpoint{13.059640in}{5.469778in}}{\pgfqpoint{13.059640in}{5.458728in}}%
\pgfpathcurveto{\pgfqpoint{13.059640in}{5.447678in}}{\pgfqpoint{13.064030in}{5.437079in}}{\pgfqpoint{13.071844in}{5.429265in}}%
\pgfpathcurveto{\pgfqpoint{13.079658in}{5.421451in}}{\pgfqpoint{13.090257in}{5.417061in}}{\pgfqpoint{13.101307in}{5.417061in}}%
\pgfpathlineto{\pgfqpoint{13.101307in}{5.417061in}}%
\pgfpathclose%
\pgfusepath{stroke}%
\end{pgfscope}%
\begin{pgfscope}%
\pgfpathrectangle{\pgfqpoint{7.512535in}{0.437222in}}{\pgfqpoint{6.275590in}{5.159444in}}%
\pgfusepath{clip}%
\pgfsetbuttcap%
\pgfsetroundjoin%
\pgfsetlinewidth{1.003750pt}%
\definecolor{currentstroke}{rgb}{0.827451,0.827451,0.827451}%
\pgfsetstrokecolor{currentstroke}%
\pgfsetstrokeopacity{0.800000}%
\pgfsetdash{}{0pt}%
\pgfpathmoveto{\pgfqpoint{7.972802in}{0.679358in}}%
\pgfpathcurveto{\pgfqpoint{7.983853in}{0.679358in}}{\pgfqpoint{7.994452in}{0.683748in}}{\pgfqpoint{8.002265in}{0.691562in}}%
\pgfpathcurveto{\pgfqpoint{8.010079in}{0.699376in}}{\pgfqpoint{8.014469in}{0.709975in}}{\pgfqpoint{8.014469in}{0.721025in}}%
\pgfpathcurveto{\pgfqpoint{8.014469in}{0.732075in}}{\pgfqpoint{8.010079in}{0.742674in}}{\pgfqpoint{8.002265in}{0.750487in}}%
\pgfpathcurveto{\pgfqpoint{7.994452in}{0.758301in}}{\pgfqpoint{7.983853in}{0.762691in}}{\pgfqpoint{7.972802in}{0.762691in}}%
\pgfpathcurveto{\pgfqpoint{7.961752in}{0.762691in}}{\pgfqpoint{7.951153in}{0.758301in}}{\pgfqpoint{7.943340in}{0.750487in}}%
\pgfpathcurveto{\pgfqpoint{7.935526in}{0.742674in}}{\pgfqpoint{7.931136in}{0.732075in}}{\pgfqpoint{7.931136in}{0.721025in}}%
\pgfpathcurveto{\pgfqpoint{7.931136in}{0.709975in}}{\pgfqpoint{7.935526in}{0.699376in}}{\pgfqpoint{7.943340in}{0.691562in}}%
\pgfpathcurveto{\pgfqpoint{7.951153in}{0.683748in}}{\pgfqpoint{7.961752in}{0.679358in}}{\pgfqpoint{7.972802in}{0.679358in}}%
\pgfpathlineto{\pgfqpoint{7.972802in}{0.679358in}}%
\pgfpathclose%
\pgfusepath{stroke}%
\end{pgfscope}%
\begin{pgfscope}%
\pgfpathrectangle{\pgfqpoint{7.512535in}{0.437222in}}{\pgfqpoint{6.275590in}{5.159444in}}%
\pgfusepath{clip}%
\pgfsetbuttcap%
\pgfsetroundjoin%
\pgfsetlinewidth{1.003750pt}%
\definecolor{currentstroke}{rgb}{0.827451,0.827451,0.827451}%
\pgfsetstrokecolor{currentstroke}%
\pgfsetstrokeopacity{0.800000}%
\pgfsetdash{}{0pt}%
\pgfpathmoveto{\pgfqpoint{12.072745in}{5.039509in}}%
\pgfpathcurveto{\pgfqpoint{12.083796in}{5.039509in}}{\pgfqpoint{12.094395in}{5.043899in}}{\pgfqpoint{12.102208in}{5.051713in}}%
\pgfpathcurveto{\pgfqpoint{12.110022in}{5.059526in}}{\pgfqpoint{12.114412in}{5.070125in}}{\pgfqpoint{12.114412in}{5.081175in}}%
\pgfpathcurveto{\pgfqpoint{12.114412in}{5.092226in}}{\pgfqpoint{12.110022in}{5.102825in}}{\pgfqpoint{12.102208in}{5.110638in}}%
\pgfpathcurveto{\pgfqpoint{12.094395in}{5.118452in}}{\pgfqpoint{12.083796in}{5.122842in}}{\pgfqpoint{12.072745in}{5.122842in}}%
\pgfpathcurveto{\pgfqpoint{12.061695in}{5.122842in}}{\pgfqpoint{12.051096in}{5.118452in}}{\pgfqpoint{12.043283in}{5.110638in}}%
\pgfpathcurveto{\pgfqpoint{12.035469in}{5.102825in}}{\pgfqpoint{12.031079in}{5.092226in}}{\pgfqpoint{12.031079in}{5.081175in}}%
\pgfpathcurveto{\pgfqpoint{12.031079in}{5.070125in}}{\pgfqpoint{12.035469in}{5.059526in}}{\pgfqpoint{12.043283in}{5.051713in}}%
\pgfpathcurveto{\pgfqpoint{12.051096in}{5.043899in}}{\pgfqpoint{12.061695in}{5.039509in}}{\pgfqpoint{12.072745in}{5.039509in}}%
\pgfpathlineto{\pgfqpoint{12.072745in}{5.039509in}}%
\pgfpathclose%
\pgfusepath{stroke}%
\end{pgfscope}%
\begin{pgfscope}%
\pgfpathrectangle{\pgfqpoint{7.512535in}{0.437222in}}{\pgfqpoint{6.275590in}{5.159444in}}%
\pgfusepath{clip}%
\pgfsetbuttcap%
\pgfsetroundjoin%
\pgfsetlinewidth{1.003750pt}%
\definecolor{currentstroke}{rgb}{0.827451,0.827451,0.827451}%
\pgfsetstrokecolor{currentstroke}%
\pgfsetstrokeopacity{0.800000}%
\pgfsetdash{}{0pt}%
\pgfpathmoveto{\pgfqpoint{10.818589in}{5.117177in}}%
\pgfpathcurveto{\pgfqpoint{10.829639in}{5.117177in}}{\pgfqpoint{10.840238in}{5.121567in}}{\pgfqpoint{10.848052in}{5.129381in}}%
\pgfpathcurveto{\pgfqpoint{10.855865in}{5.137194in}}{\pgfqpoint{10.860255in}{5.147793in}}{\pgfqpoint{10.860255in}{5.158843in}}%
\pgfpathcurveto{\pgfqpoint{10.860255in}{5.169893in}}{\pgfqpoint{10.855865in}{5.180493in}}{\pgfqpoint{10.848052in}{5.188306in}}%
\pgfpathcurveto{\pgfqpoint{10.840238in}{5.196120in}}{\pgfqpoint{10.829639in}{5.200510in}}{\pgfqpoint{10.818589in}{5.200510in}}%
\pgfpathcurveto{\pgfqpoint{10.807539in}{5.200510in}}{\pgfqpoint{10.796940in}{5.196120in}}{\pgfqpoint{10.789126in}{5.188306in}}%
\pgfpathcurveto{\pgfqpoint{10.781312in}{5.180493in}}{\pgfqpoint{10.776922in}{5.169893in}}{\pgfqpoint{10.776922in}{5.158843in}}%
\pgfpathcurveto{\pgfqpoint{10.776922in}{5.147793in}}{\pgfqpoint{10.781312in}{5.137194in}}{\pgfqpoint{10.789126in}{5.129381in}}%
\pgfpathcurveto{\pgfqpoint{10.796940in}{5.121567in}}{\pgfqpoint{10.807539in}{5.117177in}}{\pgfqpoint{10.818589in}{5.117177in}}%
\pgfpathlineto{\pgfqpoint{10.818589in}{5.117177in}}%
\pgfpathclose%
\pgfusepath{stroke}%
\end{pgfscope}%
\begin{pgfscope}%
\pgfpathrectangle{\pgfqpoint{7.512535in}{0.437222in}}{\pgfqpoint{6.275590in}{5.159444in}}%
\pgfusepath{clip}%
\pgfsetbuttcap%
\pgfsetroundjoin%
\pgfsetlinewidth{1.003750pt}%
\definecolor{currentstroke}{rgb}{0.827451,0.827451,0.827451}%
\pgfsetstrokecolor{currentstroke}%
\pgfsetstrokeopacity{0.800000}%
\pgfsetdash{}{0pt}%
\pgfpathmoveto{\pgfqpoint{11.443872in}{5.184186in}}%
\pgfpathcurveto{\pgfqpoint{11.454922in}{5.184186in}}{\pgfqpoint{11.465521in}{5.188576in}}{\pgfqpoint{11.473335in}{5.196390in}}%
\pgfpathcurveto{\pgfqpoint{11.481148in}{5.204203in}}{\pgfqpoint{11.485539in}{5.214802in}}{\pgfqpoint{11.485539in}{5.225852in}}%
\pgfpathcurveto{\pgfqpoint{11.485539in}{5.236903in}}{\pgfqpoint{11.481148in}{5.247502in}}{\pgfqpoint{11.473335in}{5.255315in}}%
\pgfpathcurveto{\pgfqpoint{11.465521in}{5.263129in}}{\pgfqpoint{11.454922in}{5.267519in}}{\pgfqpoint{11.443872in}{5.267519in}}%
\pgfpathcurveto{\pgfqpoint{11.432822in}{5.267519in}}{\pgfqpoint{11.422223in}{5.263129in}}{\pgfqpoint{11.414409in}{5.255315in}}%
\pgfpathcurveto{\pgfqpoint{11.406596in}{5.247502in}}{\pgfqpoint{11.402205in}{5.236903in}}{\pgfqpoint{11.402205in}{5.225852in}}%
\pgfpathcurveto{\pgfqpoint{11.402205in}{5.214802in}}{\pgfqpoint{11.406596in}{5.204203in}}{\pgfqpoint{11.414409in}{5.196390in}}%
\pgfpathcurveto{\pgfqpoint{11.422223in}{5.188576in}}{\pgfqpoint{11.432822in}{5.184186in}}{\pgfqpoint{11.443872in}{5.184186in}}%
\pgfpathlineto{\pgfqpoint{11.443872in}{5.184186in}}%
\pgfpathclose%
\pgfusepath{stroke}%
\end{pgfscope}%
\begin{pgfscope}%
\pgfpathrectangle{\pgfqpoint{7.512535in}{0.437222in}}{\pgfqpoint{6.275590in}{5.159444in}}%
\pgfusepath{clip}%
\pgfsetbuttcap%
\pgfsetroundjoin%
\pgfsetlinewidth{1.003750pt}%
\definecolor{currentstroke}{rgb}{0.827451,0.827451,0.827451}%
\pgfsetstrokecolor{currentstroke}%
\pgfsetstrokeopacity{0.800000}%
\pgfsetdash{}{0pt}%
\pgfpathmoveto{\pgfqpoint{8.290348in}{1.011495in}}%
\pgfpathcurveto{\pgfqpoint{8.301398in}{1.011495in}}{\pgfqpoint{8.311997in}{1.015885in}}{\pgfqpoint{8.319811in}{1.023699in}}%
\pgfpathcurveto{\pgfqpoint{8.327625in}{1.031512in}}{\pgfqpoint{8.332015in}{1.042111in}}{\pgfqpoint{8.332015in}{1.053161in}}%
\pgfpathcurveto{\pgfqpoint{8.332015in}{1.064211in}}{\pgfqpoint{8.327625in}{1.074810in}}{\pgfqpoint{8.319811in}{1.082624in}}%
\pgfpathcurveto{\pgfqpoint{8.311997in}{1.090438in}}{\pgfqpoint{8.301398in}{1.094828in}}{\pgfqpoint{8.290348in}{1.094828in}}%
\pgfpathcurveto{\pgfqpoint{8.279298in}{1.094828in}}{\pgfqpoint{8.268699in}{1.090438in}}{\pgfqpoint{8.260886in}{1.082624in}}%
\pgfpathcurveto{\pgfqpoint{8.253072in}{1.074810in}}{\pgfqpoint{8.248682in}{1.064211in}}{\pgfqpoint{8.248682in}{1.053161in}}%
\pgfpathcurveto{\pgfqpoint{8.248682in}{1.042111in}}{\pgfqpoint{8.253072in}{1.031512in}}{\pgfqpoint{8.260886in}{1.023699in}}%
\pgfpathcurveto{\pgfqpoint{8.268699in}{1.015885in}}{\pgfqpoint{8.279298in}{1.011495in}}{\pgfqpoint{8.290348in}{1.011495in}}%
\pgfpathlineto{\pgfqpoint{8.290348in}{1.011495in}}%
\pgfpathclose%
\pgfusepath{stroke}%
\end{pgfscope}%
\begin{pgfscope}%
\pgfpathrectangle{\pgfqpoint{7.512535in}{0.437222in}}{\pgfqpoint{6.275590in}{5.159444in}}%
\pgfusepath{clip}%
\pgfsetbuttcap%
\pgfsetroundjoin%
\pgfsetlinewidth{1.003750pt}%
\definecolor{currentstroke}{rgb}{0.827451,0.827451,0.827451}%
\pgfsetstrokecolor{currentstroke}%
\pgfsetstrokeopacity{0.800000}%
\pgfsetdash{}{0pt}%
\pgfpathmoveto{\pgfqpoint{8.983729in}{3.220234in}}%
\pgfpathcurveto{\pgfqpoint{8.994779in}{3.220234in}}{\pgfqpoint{9.005378in}{3.224624in}}{\pgfqpoint{9.013191in}{3.232438in}}%
\pgfpathcurveto{\pgfqpoint{9.021005in}{3.240251in}}{\pgfqpoint{9.025395in}{3.250850in}}{\pgfqpoint{9.025395in}{3.261901in}}%
\pgfpathcurveto{\pgfqpoint{9.025395in}{3.272951in}}{\pgfqpoint{9.021005in}{3.283550in}}{\pgfqpoint{9.013191in}{3.291363in}}%
\pgfpathcurveto{\pgfqpoint{9.005378in}{3.299177in}}{\pgfqpoint{8.994779in}{3.303567in}}{\pgfqpoint{8.983729in}{3.303567in}}%
\pgfpathcurveto{\pgfqpoint{8.972679in}{3.303567in}}{\pgfqpoint{8.962079in}{3.299177in}}{\pgfqpoint{8.954266in}{3.291363in}}%
\pgfpathcurveto{\pgfqpoint{8.946452in}{3.283550in}}{\pgfqpoint{8.942062in}{3.272951in}}{\pgfqpoint{8.942062in}{3.261901in}}%
\pgfpathcurveto{\pgfqpoint{8.942062in}{3.250850in}}{\pgfqpoint{8.946452in}{3.240251in}}{\pgfqpoint{8.954266in}{3.232438in}}%
\pgfpathcurveto{\pgfqpoint{8.962079in}{3.224624in}}{\pgfqpoint{8.972679in}{3.220234in}}{\pgfqpoint{8.983729in}{3.220234in}}%
\pgfpathlineto{\pgfqpoint{8.983729in}{3.220234in}}%
\pgfpathclose%
\pgfusepath{stroke}%
\end{pgfscope}%
\begin{pgfscope}%
\pgfpathrectangle{\pgfqpoint{7.512535in}{0.437222in}}{\pgfqpoint{6.275590in}{5.159444in}}%
\pgfusepath{clip}%
\pgfsetbuttcap%
\pgfsetroundjoin%
\pgfsetlinewidth{1.003750pt}%
\definecolor{currentstroke}{rgb}{0.827451,0.827451,0.827451}%
\pgfsetstrokecolor{currentstroke}%
\pgfsetstrokeopacity{0.800000}%
\pgfsetdash{}{0pt}%
\pgfpathmoveto{\pgfqpoint{9.615954in}{3.788157in}}%
\pgfpathcurveto{\pgfqpoint{9.627004in}{3.788157in}}{\pgfqpoint{9.637603in}{3.792548in}}{\pgfqpoint{9.645416in}{3.800361in}}%
\pgfpathcurveto{\pgfqpoint{9.653230in}{3.808175in}}{\pgfqpoint{9.657620in}{3.818774in}}{\pgfqpoint{9.657620in}{3.829824in}}%
\pgfpathcurveto{\pgfqpoint{9.657620in}{3.840874in}}{\pgfqpoint{9.653230in}{3.851473in}}{\pgfqpoint{9.645416in}{3.859287in}}%
\pgfpathcurveto{\pgfqpoint{9.637603in}{3.867100in}}{\pgfqpoint{9.627004in}{3.871491in}}{\pgfqpoint{9.615954in}{3.871491in}}%
\pgfpathcurveto{\pgfqpoint{9.604903in}{3.871491in}}{\pgfqpoint{9.594304in}{3.867100in}}{\pgfqpoint{9.586491in}{3.859287in}}%
\pgfpathcurveto{\pgfqpoint{9.578677in}{3.851473in}}{\pgfqpoint{9.574287in}{3.840874in}}{\pgfqpoint{9.574287in}{3.829824in}}%
\pgfpathcurveto{\pgfqpoint{9.574287in}{3.818774in}}{\pgfqpoint{9.578677in}{3.808175in}}{\pgfqpoint{9.586491in}{3.800361in}}%
\pgfpathcurveto{\pgfqpoint{9.594304in}{3.792548in}}{\pgfqpoint{9.604903in}{3.788157in}}{\pgfqpoint{9.615954in}{3.788157in}}%
\pgfpathlineto{\pgfqpoint{9.615954in}{3.788157in}}%
\pgfpathclose%
\pgfusepath{stroke}%
\end{pgfscope}%
\begin{pgfscope}%
\pgfpathrectangle{\pgfqpoint{7.512535in}{0.437222in}}{\pgfqpoint{6.275590in}{5.159444in}}%
\pgfusepath{clip}%
\pgfsetbuttcap%
\pgfsetroundjoin%
\pgfsetlinewidth{1.003750pt}%
\definecolor{currentstroke}{rgb}{0.827451,0.827451,0.827451}%
\pgfsetstrokecolor{currentstroke}%
\pgfsetstrokeopacity{0.800000}%
\pgfsetdash{}{0pt}%
\pgfpathmoveto{\pgfqpoint{11.673162in}{5.362410in}}%
\pgfpathcurveto{\pgfqpoint{11.684212in}{5.362410in}}{\pgfqpoint{11.694811in}{5.366800in}}{\pgfqpoint{11.702624in}{5.374614in}}%
\pgfpathcurveto{\pgfqpoint{11.710438in}{5.382427in}}{\pgfqpoint{11.714828in}{5.393027in}}{\pgfqpoint{11.714828in}{5.404077in}}%
\pgfpathcurveto{\pgfqpoint{11.714828in}{5.415127in}}{\pgfqpoint{11.710438in}{5.425726in}}{\pgfqpoint{11.702624in}{5.433539in}}%
\pgfpathcurveto{\pgfqpoint{11.694811in}{5.441353in}}{\pgfqpoint{11.684212in}{5.445743in}}{\pgfqpoint{11.673162in}{5.445743in}}%
\pgfpathcurveto{\pgfqpoint{11.662111in}{5.445743in}}{\pgfqpoint{11.651512in}{5.441353in}}{\pgfqpoint{11.643699in}{5.433539in}}%
\pgfpathcurveto{\pgfqpoint{11.635885in}{5.425726in}}{\pgfqpoint{11.631495in}{5.415127in}}{\pgfqpoint{11.631495in}{5.404077in}}%
\pgfpathcurveto{\pgfqpoint{11.631495in}{5.393027in}}{\pgfqpoint{11.635885in}{5.382427in}}{\pgfqpoint{11.643699in}{5.374614in}}%
\pgfpathcurveto{\pgfqpoint{11.651512in}{5.366800in}}{\pgfqpoint{11.662111in}{5.362410in}}{\pgfqpoint{11.673162in}{5.362410in}}%
\pgfpathlineto{\pgfqpoint{11.673162in}{5.362410in}}%
\pgfpathclose%
\pgfusepath{stroke}%
\end{pgfscope}%
\begin{pgfscope}%
\pgfpathrectangle{\pgfqpoint{7.512535in}{0.437222in}}{\pgfqpoint{6.275590in}{5.159444in}}%
\pgfusepath{clip}%
\pgfsetbuttcap%
\pgfsetroundjoin%
\pgfsetlinewidth{1.003750pt}%
\definecolor{currentstroke}{rgb}{0.827451,0.827451,0.827451}%
\pgfsetstrokecolor{currentstroke}%
\pgfsetstrokeopacity{0.800000}%
\pgfsetdash{}{0pt}%
\pgfpathmoveto{\pgfqpoint{10.486066in}{4.589271in}}%
\pgfpathcurveto{\pgfqpoint{10.497116in}{4.589271in}}{\pgfqpoint{10.507715in}{4.593662in}}{\pgfqpoint{10.515529in}{4.601475in}}%
\pgfpathcurveto{\pgfqpoint{10.523342in}{4.609289in}}{\pgfqpoint{10.527733in}{4.619888in}}{\pgfqpoint{10.527733in}{4.630938in}}%
\pgfpathcurveto{\pgfqpoint{10.527733in}{4.641988in}}{\pgfqpoint{10.523342in}{4.652587in}}{\pgfqpoint{10.515529in}{4.660401in}}%
\pgfpathcurveto{\pgfqpoint{10.507715in}{4.668214in}}{\pgfqpoint{10.497116in}{4.672605in}}{\pgfqpoint{10.486066in}{4.672605in}}%
\pgfpathcurveto{\pgfqpoint{10.475016in}{4.672605in}}{\pgfqpoint{10.464417in}{4.668214in}}{\pgfqpoint{10.456603in}{4.660401in}}%
\pgfpathcurveto{\pgfqpoint{10.448790in}{4.652587in}}{\pgfqpoint{10.444399in}{4.641988in}}{\pgfqpoint{10.444399in}{4.630938in}}%
\pgfpathcurveto{\pgfqpoint{10.444399in}{4.619888in}}{\pgfqpoint{10.448790in}{4.609289in}}{\pgfqpoint{10.456603in}{4.601475in}}%
\pgfpathcurveto{\pgfqpoint{10.464417in}{4.593662in}}{\pgfqpoint{10.475016in}{4.589271in}}{\pgfqpoint{10.486066in}{4.589271in}}%
\pgfpathlineto{\pgfqpoint{10.486066in}{4.589271in}}%
\pgfpathclose%
\pgfusepath{stroke}%
\end{pgfscope}%
\begin{pgfscope}%
\pgfpathrectangle{\pgfqpoint{7.512535in}{0.437222in}}{\pgfqpoint{6.275590in}{5.159444in}}%
\pgfusepath{clip}%
\pgfsetbuttcap%
\pgfsetroundjoin%
\pgfsetlinewidth{1.003750pt}%
\definecolor{currentstroke}{rgb}{0.827451,0.827451,0.827451}%
\pgfsetstrokecolor{currentstroke}%
\pgfsetstrokeopacity{0.800000}%
\pgfsetdash{}{0pt}%
\pgfpathmoveto{\pgfqpoint{8.896008in}{2.422402in}}%
\pgfpathcurveto{\pgfqpoint{8.907058in}{2.422402in}}{\pgfqpoint{8.917657in}{2.426793in}}{\pgfqpoint{8.925471in}{2.434606in}}%
\pgfpathcurveto{\pgfqpoint{8.933285in}{2.442420in}}{\pgfqpoint{8.937675in}{2.453019in}}{\pgfqpoint{8.937675in}{2.464069in}}%
\pgfpathcurveto{\pgfqpoint{8.937675in}{2.475119in}}{\pgfqpoint{8.933285in}{2.485718in}}{\pgfqpoint{8.925471in}{2.493532in}}%
\pgfpathcurveto{\pgfqpoint{8.917657in}{2.501345in}}{\pgfqpoint{8.907058in}{2.505736in}}{\pgfqpoint{8.896008in}{2.505736in}}%
\pgfpathcurveto{\pgfqpoint{8.884958in}{2.505736in}}{\pgfqpoint{8.874359in}{2.501345in}}{\pgfqpoint{8.866545in}{2.493532in}}%
\pgfpathcurveto{\pgfqpoint{8.858732in}{2.485718in}}{\pgfqpoint{8.854341in}{2.475119in}}{\pgfqpoint{8.854341in}{2.464069in}}%
\pgfpathcurveto{\pgfqpoint{8.854341in}{2.453019in}}{\pgfqpoint{8.858732in}{2.442420in}}{\pgfqpoint{8.866545in}{2.434606in}}%
\pgfpathcurveto{\pgfqpoint{8.874359in}{2.426793in}}{\pgfqpoint{8.884958in}{2.422402in}}{\pgfqpoint{8.896008in}{2.422402in}}%
\pgfpathlineto{\pgfqpoint{8.896008in}{2.422402in}}%
\pgfpathclose%
\pgfusepath{stroke}%
\end{pgfscope}%
\begin{pgfscope}%
\pgfpathrectangle{\pgfqpoint{7.512535in}{0.437222in}}{\pgfqpoint{6.275590in}{5.159444in}}%
\pgfusepath{clip}%
\pgfsetbuttcap%
\pgfsetroundjoin%
\pgfsetlinewidth{1.003750pt}%
\definecolor{currentstroke}{rgb}{0.827451,0.827451,0.827451}%
\pgfsetstrokecolor{currentstroke}%
\pgfsetstrokeopacity{0.800000}%
\pgfsetdash{}{0pt}%
\pgfpathmoveto{\pgfqpoint{9.561548in}{1.679042in}}%
\pgfpathcurveto{\pgfqpoint{9.572598in}{1.679042in}}{\pgfqpoint{9.583197in}{1.683432in}}{\pgfqpoint{9.591011in}{1.691246in}}%
\pgfpathcurveto{\pgfqpoint{9.598825in}{1.699059in}}{\pgfqpoint{9.603215in}{1.709658in}}{\pgfqpoint{9.603215in}{1.720709in}}%
\pgfpathcurveto{\pgfqpoint{9.603215in}{1.731759in}}{\pgfqpoint{9.598825in}{1.742358in}}{\pgfqpoint{9.591011in}{1.750171in}}%
\pgfpathcurveto{\pgfqpoint{9.583197in}{1.757985in}}{\pgfqpoint{9.572598in}{1.762375in}}{\pgfqpoint{9.561548in}{1.762375in}}%
\pgfpathcurveto{\pgfqpoint{9.550498in}{1.762375in}}{\pgfqpoint{9.539899in}{1.757985in}}{\pgfqpoint{9.532086in}{1.750171in}}%
\pgfpathcurveto{\pgfqpoint{9.524272in}{1.742358in}}{\pgfqpoint{9.519882in}{1.731759in}}{\pgfqpoint{9.519882in}{1.720709in}}%
\pgfpathcurveto{\pgfqpoint{9.519882in}{1.709658in}}{\pgfqpoint{9.524272in}{1.699059in}}{\pgfqpoint{9.532086in}{1.691246in}}%
\pgfpathcurveto{\pgfqpoint{9.539899in}{1.683432in}}{\pgfqpoint{9.550498in}{1.679042in}}{\pgfqpoint{9.561548in}{1.679042in}}%
\pgfpathlineto{\pgfqpoint{9.561548in}{1.679042in}}%
\pgfpathclose%
\pgfusepath{stroke}%
\end{pgfscope}%
\begin{pgfscope}%
\pgfpathrectangle{\pgfqpoint{7.512535in}{0.437222in}}{\pgfqpoint{6.275590in}{5.159444in}}%
\pgfusepath{clip}%
\pgfsetbuttcap%
\pgfsetroundjoin%
\pgfsetlinewidth{1.003750pt}%
\definecolor{currentstroke}{rgb}{0.827451,0.827451,0.827451}%
\pgfsetstrokecolor{currentstroke}%
\pgfsetstrokeopacity{0.800000}%
\pgfsetdash{}{0pt}%
\pgfpathmoveto{\pgfqpoint{11.405420in}{5.475782in}}%
\pgfpathcurveto{\pgfqpoint{11.416470in}{5.475782in}}{\pgfqpoint{11.427069in}{5.480172in}}{\pgfqpoint{11.434883in}{5.487986in}}%
\pgfpathcurveto{\pgfqpoint{11.442696in}{5.495800in}}{\pgfqpoint{11.447086in}{5.506399in}}{\pgfqpoint{11.447086in}{5.517449in}}%
\pgfpathcurveto{\pgfqpoint{11.447086in}{5.528499in}}{\pgfqpoint{11.442696in}{5.539098in}}{\pgfqpoint{11.434883in}{5.546912in}}%
\pgfpathcurveto{\pgfqpoint{11.427069in}{5.554725in}}{\pgfqpoint{11.416470in}{5.559116in}}{\pgfqpoint{11.405420in}{5.559116in}}%
\pgfpathcurveto{\pgfqpoint{11.394370in}{5.559116in}}{\pgfqpoint{11.383771in}{5.554725in}}{\pgfqpoint{11.375957in}{5.546912in}}%
\pgfpathcurveto{\pgfqpoint{11.368143in}{5.539098in}}{\pgfqpoint{11.363753in}{5.528499in}}{\pgfqpoint{11.363753in}{5.517449in}}%
\pgfpathcurveto{\pgfqpoint{11.363753in}{5.506399in}}{\pgfqpoint{11.368143in}{5.495800in}}{\pgfqpoint{11.375957in}{5.487986in}}%
\pgfpathcurveto{\pgfqpoint{11.383771in}{5.480172in}}{\pgfqpoint{11.394370in}{5.475782in}}{\pgfqpoint{11.405420in}{5.475782in}}%
\pgfpathlineto{\pgfqpoint{11.405420in}{5.475782in}}%
\pgfpathclose%
\pgfusepath{stroke}%
\end{pgfscope}%
\begin{pgfscope}%
\pgfpathrectangle{\pgfqpoint{7.512535in}{0.437222in}}{\pgfqpoint{6.275590in}{5.159444in}}%
\pgfusepath{clip}%
\pgfsetbuttcap%
\pgfsetroundjoin%
\pgfsetlinewidth{1.003750pt}%
\definecolor{currentstroke}{rgb}{0.827451,0.827451,0.827451}%
\pgfsetstrokecolor{currentstroke}%
\pgfsetstrokeopacity{0.800000}%
\pgfsetdash{}{0pt}%
\pgfpathmoveto{\pgfqpoint{12.102762in}{5.362410in}}%
\pgfpathcurveto{\pgfqpoint{12.113812in}{5.362410in}}{\pgfqpoint{12.124411in}{5.366800in}}{\pgfqpoint{12.132225in}{5.374614in}}%
\pgfpathcurveto{\pgfqpoint{12.140039in}{5.382427in}}{\pgfqpoint{12.144429in}{5.393027in}}{\pgfqpoint{12.144429in}{5.404077in}}%
\pgfpathcurveto{\pgfqpoint{12.144429in}{5.415127in}}{\pgfqpoint{12.140039in}{5.425726in}}{\pgfqpoint{12.132225in}{5.433539in}}%
\pgfpathcurveto{\pgfqpoint{12.124411in}{5.441353in}}{\pgfqpoint{12.113812in}{5.445743in}}{\pgfqpoint{12.102762in}{5.445743in}}%
\pgfpathcurveto{\pgfqpoint{12.091712in}{5.445743in}}{\pgfqpoint{12.081113in}{5.441353in}}{\pgfqpoint{12.073299in}{5.433539in}}%
\pgfpathcurveto{\pgfqpoint{12.065486in}{5.425726in}}{\pgfqpoint{12.061096in}{5.415127in}}{\pgfqpoint{12.061096in}{5.404077in}}%
\pgfpathcurveto{\pgfqpoint{12.061096in}{5.393027in}}{\pgfqpoint{12.065486in}{5.382427in}}{\pgfqpoint{12.073299in}{5.374614in}}%
\pgfpathcurveto{\pgfqpoint{12.081113in}{5.366800in}}{\pgfqpoint{12.091712in}{5.362410in}}{\pgfqpoint{12.102762in}{5.362410in}}%
\pgfpathlineto{\pgfqpoint{12.102762in}{5.362410in}}%
\pgfpathclose%
\pgfusepath{stroke}%
\end{pgfscope}%
\begin{pgfscope}%
\pgfpathrectangle{\pgfqpoint{7.512535in}{0.437222in}}{\pgfqpoint{6.275590in}{5.159444in}}%
\pgfusepath{clip}%
\pgfsetbuttcap%
\pgfsetroundjoin%
\pgfsetlinewidth{1.003750pt}%
\definecolor{currentstroke}{rgb}{0.827451,0.827451,0.827451}%
\pgfsetstrokecolor{currentstroke}%
\pgfsetstrokeopacity{0.800000}%
\pgfsetdash{}{0pt}%
\pgfpathmoveto{\pgfqpoint{13.419279in}{5.547910in}}%
\pgfpathcurveto{\pgfqpoint{13.430329in}{5.547910in}}{\pgfqpoint{13.440928in}{5.552301in}}{\pgfqpoint{13.448742in}{5.560114in}}%
\pgfpathcurveto{\pgfqpoint{13.456555in}{5.567928in}}{\pgfqpoint{13.460945in}{5.578527in}}{\pgfqpoint{13.460945in}{5.589577in}}%
\pgfpathcurveto{\pgfqpoint{13.460945in}{5.600627in}}{\pgfqpoint{13.456555in}{5.611226in}}{\pgfqpoint{13.448742in}{5.619040in}}%
\pgfpathcurveto{\pgfqpoint{13.440928in}{5.626854in}}{\pgfqpoint{13.430329in}{5.631244in}}{\pgfqpoint{13.419279in}{5.631244in}}%
\pgfpathcurveto{\pgfqpoint{13.408229in}{5.631244in}}{\pgfqpoint{13.397630in}{5.626854in}}{\pgfqpoint{13.389816in}{5.619040in}}%
\pgfpathcurveto{\pgfqpoint{13.382002in}{5.611226in}}{\pgfqpoint{13.377612in}{5.600627in}}{\pgfqpoint{13.377612in}{5.589577in}}%
\pgfpathcurveto{\pgfqpoint{13.377612in}{5.578527in}}{\pgfqpoint{13.382002in}{5.567928in}}{\pgfqpoint{13.389816in}{5.560114in}}%
\pgfpathcurveto{\pgfqpoint{13.397630in}{5.552301in}}{\pgfqpoint{13.408229in}{5.547910in}}{\pgfqpoint{13.419279in}{5.547910in}}%
\pgfpathlineto{\pgfqpoint{13.419279in}{5.547910in}}%
\pgfpathclose%
\pgfusepath{stroke}%
\end{pgfscope}%
\begin{pgfscope}%
\pgfpathrectangle{\pgfqpoint{7.512535in}{0.437222in}}{\pgfqpoint{6.275590in}{5.159444in}}%
\pgfusepath{clip}%
\pgfsetbuttcap%
\pgfsetroundjoin%
\pgfsetlinewidth{1.003750pt}%
\definecolor{currentstroke}{rgb}{0.827451,0.827451,0.827451}%
\pgfsetstrokecolor{currentstroke}%
\pgfsetstrokeopacity{0.800000}%
\pgfsetdash{}{0pt}%
\pgfpathmoveto{\pgfqpoint{12.737963in}{5.481847in}}%
\pgfpathcurveto{\pgfqpoint{12.749014in}{5.481847in}}{\pgfqpoint{12.759613in}{5.486238in}}{\pgfqpoint{12.767426in}{5.494051in}}%
\pgfpathcurveto{\pgfqpoint{12.775240in}{5.501865in}}{\pgfqpoint{12.779630in}{5.512464in}}{\pgfqpoint{12.779630in}{5.523514in}}%
\pgfpathcurveto{\pgfqpoint{12.779630in}{5.534564in}}{\pgfqpoint{12.775240in}{5.545163in}}{\pgfqpoint{12.767426in}{5.552977in}}%
\pgfpathcurveto{\pgfqpoint{12.759613in}{5.560791in}}{\pgfqpoint{12.749014in}{5.565181in}}{\pgfqpoint{12.737963in}{5.565181in}}%
\pgfpathcurveto{\pgfqpoint{12.726913in}{5.565181in}}{\pgfqpoint{12.716314in}{5.560791in}}{\pgfqpoint{12.708501in}{5.552977in}}%
\pgfpathcurveto{\pgfqpoint{12.700687in}{5.545163in}}{\pgfqpoint{12.696297in}{5.534564in}}{\pgfqpoint{12.696297in}{5.523514in}}%
\pgfpathcurveto{\pgfqpoint{12.696297in}{5.512464in}}{\pgfqpoint{12.700687in}{5.501865in}}{\pgfqpoint{12.708501in}{5.494051in}}%
\pgfpathcurveto{\pgfqpoint{12.716314in}{5.486238in}}{\pgfqpoint{12.726913in}{5.481847in}}{\pgfqpoint{12.737963in}{5.481847in}}%
\pgfpathlineto{\pgfqpoint{12.737963in}{5.481847in}}%
\pgfpathclose%
\pgfusepath{stroke}%
\end{pgfscope}%
\begin{pgfscope}%
\pgfpathrectangle{\pgfqpoint{7.512535in}{0.437222in}}{\pgfqpoint{6.275590in}{5.159444in}}%
\pgfusepath{clip}%
\pgfsetbuttcap%
\pgfsetroundjoin%
\pgfsetlinewidth{1.003750pt}%
\definecolor{currentstroke}{rgb}{0.827451,0.827451,0.827451}%
\pgfsetstrokecolor{currentstroke}%
\pgfsetstrokeopacity{0.800000}%
\pgfsetdash{}{0pt}%
\pgfpathmoveto{\pgfqpoint{12.227551in}{5.553091in}}%
\pgfpathcurveto{\pgfqpoint{12.238601in}{5.553091in}}{\pgfqpoint{12.249200in}{5.557481in}}{\pgfqpoint{12.257013in}{5.565295in}}%
\pgfpathcurveto{\pgfqpoint{12.264827in}{5.573108in}}{\pgfqpoint{12.269217in}{5.583707in}}{\pgfqpoint{12.269217in}{5.594757in}}%
\pgfpathcurveto{\pgfqpoint{12.269217in}{5.605808in}}{\pgfqpoint{12.264827in}{5.616407in}}{\pgfqpoint{12.257013in}{5.624220in}}%
\pgfpathcurveto{\pgfqpoint{12.249200in}{5.632034in}}{\pgfqpoint{12.238601in}{5.636424in}}{\pgfqpoint{12.227551in}{5.636424in}}%
\pgfpathcurveto{\pgfqpoint{12.216500in}{5.636424in}}{\pgfqpoint{12.205901in}{5.632034in}}{\pgfqpoint{12.198088in}{5.624220in}}%
\pgfpathcurveto{\pgfqpoint{12.190274in}{5.616407in}}{\pgfqpoint{12.185884in}{5.605808in}}{\pgfqpoint{12.185884in}{5.594757in}}%
\pgfpathcurveto{\pgfqpoint{12.185884in}{5.583707in}}{\pgfqpoint{12.190274in}{5.573108in}}{\pgfqpoint{12.198088in}{5.565295in}}%
\pgfpathcurveto{\pgfqpoint{12.205901in}{5.557481in}}{\pgfqpoint{12.216500in}{5.553091in}}{\pgfqpoint{12.227551in}{5.553091in}}%
\pgfpathlineto{\pgfqpoint{12.227551in}{5.553091in}}%
\pgfpathclose%
\pgfusepath{stroke}%
\end{pgfscope}%
\begin{pgfscope}%
\pgfpathrectangle{\pgfqpoint{7.512535in}{0.437222in}}{\pgfqpoint{6.275590in}{5.159444in}}%
\pgfusepath{clip}%
\pgfsetbuttcap%
\pgfsetroundjoin%
\pgfsetlinewidth{1.003750pt}%
\definecolor{currentstroke}{rgb}{0.827451,0.827451,0.827451}%
\pgfsetstrokecolor{currentstroke}%
\pgfsetstrokeopacity{0.800000}%
\pgfsetdash{}{0pt}%
\pgfpathmoveto{\pgfqpoint{10.150452in}{4.500549in}}%
\pgfpathcurveto{\pgfqpoint{10.161502in}{4.500549in}}{\pgfqpoint{10.172101in}{4.504939in}}{\pgfqpoint{10.179914in}{4.512753in}}%
\pgfpathcurveto{\pgfqpoint{10.187728in}{4.520566in}}{\pgfqpoint{10.192118in}{4.531165in}}{\pgfqpoint{10.192118in}{4.542215in}}%
\pgfpathcurveto{\pgfqpoint{10.192118in}{4.553266in}}{\pgfqpoint{10.187728in}{4.563865in}}{\pgfqpoint{10.179914in}{4.571678in}}%
\pgfpathcurveto{\pgfqpoint{10.172101in}{4.579492in}}{\pgfqpoint{10.161502in}{4.583882in}}{\pgfqpoint{10.150452in}{4.583882in}}%
\pgfpathcurveto{\pgfqpoint{10.139401in}{4.583882in}}{\pgfqpoint{10.128802in}{4.579492in}}{\pgfqpoint{10.120989in}{4.571678in}}%
\pgfpathcurveto{\pgfqpoint{10.113175in}{4.563865in}}{\pgfqpoint{10.108785in}{4.553266in}}{\pgfqpoint{10.108785in}{4.542215in}}%
\pgfpathcurveto{\pgfqpoint{10.108785in}{4.531165in}}{\pgfqpoint{10.113175in}{4.520566in}}{\pgfqpoint{10.120989in}{4.512753in}}%
\pgfpathcurveto{\pgfqpoint{10.128802in}{4.504939in}}{\pgfqpoint{10.139401in}{4.500549in}}{\pgfqpoint{10.150452in}{4.500549in}}%
\pgfpathlineto{\pgfqpoint{10.150452in}{4.500549in}}%
\pgfpathclose%
\pgfusepath{stroke}%
\end{pgfscope}%
\begin{pgfscope}%
\pgfpathrectangle{\pgfqpoint{7.512535in}{0.437222in}}{\pgfqpoint{6.275590in}{5.159444in}}%
\pgfusepath{clip}%
\pgfsetbuttcap%
\pgfsetroundjoin%
\pgfsetlinewidth{1.003750pt}%
\definecolor{currentstroke}{rgb}{0.827451,0.827451,0.827451}%
\pgfsetstrokecolor{currentstroke}%
\pgfsetstrokeopacity{0.800000}%
\pgfsetdash{}{0pt}%
\pgfpathmoveto{\pgfqpoint{8.716039in}{3.385314in}}%
\pgfpathcurveto{\pgfqpoint{8.727089in}{3.385314in}}{\pgfqpoint{8.737688in}{3.389704in}}{\pgfqpoint{8.745502in}{3.397518in}}%
\pgfpathcurveto{\pgfqpoint{8.753315in}{3.405332in}}{\pgfqpoint{8.757706in}{3.415931in}}{\pgfqpoint{8.757706in}{3.426981in}}%
\pgfpathcurveto{\pgfqpoint{8.757706in}{3.438031in}}{\pgfqpoint{8.753315in}{3.448630in}}{\pgfqpoint{8.745502in}{3.456444in}}%
\pgfpathcurveto{\pgfqpoint{8.737688in}{3.464257in}}{\pgfqpoint{8.727089in}{3.468647in}}{\pgfqpoint{8.716039in}{3.468647in}}%
\pgfpathcurveto{\pgfqpoint{8.704989in}{3.468647in}}{\pgfqpoint{8.694390in}{3.464257in}}{\pgfqpoint{8.686576in}{3.456444in}}%
\pgfpathcurveto{\pgfqpoint{8.678763in}{3.448630in}}{\pgfqpoint{8.674372in}{3.438031in}}{\pgfqpoint{8.674372in}{3.426981in}}%
\pgfpathcurveto{\pgfqpoint{8.674372in}{3.415931in}}{\pgfqpoint{8.678763in}{3.405332in}}{\pgfqpoint{8.686576in}{3.397518in}}%
\pgfpathcurveto{\pgfqpoint{8.694390in}{3.389704in}}{\pgfqpoint{8.704989in}{3.385314in}}{\pgfqpoint{8.716039in}{3.385314in}}%
\pgfpathlineto{\pgfqpoint{8.716039in}{3.385314in}}%
\pgfpathclose%
\pgfusepath{stroke}%
\end{pgfscope}%
\begin{pgfscope}%
\pgfpathrectangle{\pgfqpoint{7.512535in}{0.437222in}}{\pgfqpoint{6.275590in}{5.159444in}}%
\pgfusepath{clip}%
\pgfsetbuttcap%
\pgfsetroundjoin%
\pgfsetlinewidth{1.003750pt}%
\definecolor{currentstroke}{rgb}{0.827451,0.827451,0.827451}%
\pgfsetstrokecolor{currentstroke}%
\pgfsetstrokeopacity{0.800000}%
\pgfsetdash{}{0pt}%
\pgfpathmoveto{\pgfqpoint{9.447017in}{3.335366in}}%
\pgfpathcurveto{\pgfqpoint{9.458067in}{3.335366in}}{\pgfqpoint{9.468666in}{3.339757in}}{\pgfqpoint{9.476479in}{3.347570in}}%
\pgfpathcurveto{\pgfqpoint{9.484293in}{3.355384in}}{\pgfqpoint{9.488683in}{3.365983in}}{\pgfqpoint{9.488683in}{3.377033in}}%
\pgfpathcurveto{\pgfqpoint{9.488683in}{3.388083in}}{\pgfqpoint{9.484293in}{3.398682in}}{\pgfqpoint{9.476479in}{3.406496in}}%
\pgfpathcurveto{\pgfqpoint{9.468666in}{3.414309in}}{\pgfqpoint{9.458067in}{3.418700in}}{\pgfqpoint{9.447017in}{3.418700in}}%
\pgfpathcurveto{\pgfqpoint{9.435967in}{3.418700in}}{\pgfqpoint{9.425367in}{3.414309in}}{\pgfqpoint{9.417554in}{3.406496in}}%
\pgfpathcurveto{\pgfqpoint{9.409740in}{3.398682in}}{\pgfqpoint{9.405350in}{3.388083in}}{\pgfqpoint{9.405350in}{3.377033in}}%
\pgfpathcurveto{\pgfqpoint{9.405350in}{3.365983in}}{\pgfqpoint{9.409740in}{3.355384in}}{\pgfqpoint{9.417554in}{3.347570in}}%
\pgfpathcurveto{\pgfqpoint{9.425367in}{3.339757in}}{\pgfqpoint{9.435967in}{3.335366in}}{\pgfqpoint{9.447017in}{3.335366in}}%
\pgfpathlineto{\pgfqpoint{9.447017in}{3.335366in}}%
\pgfpathclose%
\pgfusepath{stroke}%
\end{pgfscope}%
\begin{pgfscope}%
\pgfpathrectangle{\pgfqpoint{7.512535in}{0.437222in}}{\pgfqpoint{6.275590in}{5.159444in}}%
\pgfusepath{clip}%
\pgfsetbuttcap%
\pgfsetroundjoin%
\pgfsetlinewidth{1.003750pt}%
\definecolor{currentstroke}{rgb}{0.827451,0.827451,0.827451}%
\pgfsetstrokecolor{currentstroke}%
\pgfsetstrokeopacity{0.800000}%
\pgfsetdash{}{0pt}%
\pgfpathmoveto{\pgfqpoint{9.395327in}{2.124368in}}%
\pgfpathcurveto{\pgfqpoint{9.406377in}{2.124368in}}{\pgfqpoint{9.416976in}{2.128758in}}{\pgfqpoint{9.424790in}{2.136572in}}%
\pgfpathcurveto{\pgfqpoint{9.432604in}{2.144385in}}{\pgfqpoint{9.436994in}{2.154984in}}{\pgfqpoint{9.436994in}{2.166034in}}%
\pgfpathcurveto{\pgfqpoint{9.436994in}{2.177084in}}{\pgfqpoint{9.432604in}{2.187684in}}{\pgfqpoint{9.424790in}{2.195497in}}%
\pgfpathcurveto{\pgfqpoint{9.416976in}{2.203311in}}{\pgfqpoint{9.406377in}{2.207701in}}{\pgfqpoint{9.395327in}{2.207701in}}%
\pgfpathcurveto{\pgfqpoint{9.384277in}{2.207701in}}{\pgfqpoint{9.373678in}{2.203311in}}{\pgfqpoint{9.365865in}{2.195497in}}%
\pgfpathcurveto{\pgfqpoint{9.358051in}{2.187684in}}{\pgfqpoint{9.353661in}{2.177084in}}{\pgfqpoint{9.353661in}{2.166034in}}%
\pgfpathcurveto{\pgfqpoint{9.353661in}{2.154984in}}{\pgfqpoint{9.358051in}{2.144385in}}{\pgfqpoint{9.365865in}{2.136572in}}%
\pgfpathcurveto{\pgfqpoint{9.373678in}{2.128758in}}{\pgfqpoint{9.384277in}{2.124368in}}{\pgfqpoint{9.395327in}{2.124368in}}%
\pgfpathlineto{\pgfqpoint{9.395327in}{2.124368in}}%
\pgfpathclose%
\pgfusepath{stroke}%
\end{pgfscope}%
\begin{pgfscope}%
\pgfpathrectangle{\pgfqpoint{7.512535in}{0.437222in}}{\pgfqpoint{6.275590in}{5.159444in}}%
\pgfusepath{clip}%
\pgfsetbuttcap%
\pgfsetroundjoin%
\pgfsetlinewidth{1.003750pt}%
\definecolor{currentstroke}{rgb}{0.827451,0.827451,0.827451}%
\pgfsetstrokecolor{currentstroke}%
\pgfsetstrokeopacity{0.800000}%
\pgfsetdash{}{0pt}%
\pgfpathmoveto{\pgfqpoint{9.697299in}{3.449334in}}%
\pgfpathcurveto{\pgfqpoint{9.708349in}{3.449334in}}{\pgfqpoint{9.718948in}{3.453725in}}{\pgfqpoint{9.726762in}{3.461538in}}%
\pgfpathcurveto{\pgfqpoint{9.734575in}{3.469352in}}{\pgfqpoint{9.738965in}{3.479951in}}{\pgfqpoint{9.738965in}{3.491001in}}%
\pgfpathcurveto{\pgfqpoint{9.738965in}{3.502051in}}{\pgfqpoint{9.734575in}{3.512650in}}{\pgfqpoint{9.726762in}{3.520464in}}%
\pgfpathcurveto{\pgfqpoint{9.718948in}{3.528277in}}{\pgfqpoint{9.708349in}{3.532668in}}{\pgfqpoint{9.697299in}{3.532668in}}%
\pgfpathcurveto{\pgfqpoint{9.686249in}{3.532668in}}{\pgfqpoint{9.675650in}{3.528277in}}{\pgfqpoint{9.667836in}{3.520464in}}%
\pgfpathcurveto{\pgfqpoint{9.660022in}{3.512650in}}{\pgfqpoint{9.655632in}{3.502051in}}{\pgfqpoint{9.655632in}{3.491001in}}%
\pgfpathcurveto{\pgfqpoint{9.655632in}{3.479951in}}{\pgfqpoint{9.660022in}{3.469352in}}{\pgfqpoint{9.667836in}{3.461538in}}%
\pgfpathcurveto{\pgfqpoint{9.675650in}{3.453725in}}{\pgfqpoint{9.686249in}{3.449334in}}{\pgfqpoint{9.697299in}{3.449334in}}%
\pgfpathlineto{\pgfqpoint{9.697299in}{3.449334in}}%
\pgfpathclose%
\pgfusepath{stroke}%
\end{pgfscope}%
\begin{pgfscope}%
\pgfpathrectangle{\pgfqpoint{7.512535in}{0.437222in}}{\pgfqpoint{6.275590in}{5.159444in}}%
\pgfusepath{clip}%
\pgfsetbuttcap%
\pgfsetroundjoin%
\pgfsetlinewidth{1.003750pt}%
\definecolor{currentstroke}{rgb}{0.827451,0.827451,0.827451}%
\pgfsetstrokecolor{currentstroke}%
\pgfsetstrokeopacity{0.800000}%
\pgfsetdash{}{0pt}%
\pgfpathmoveto{\pgfqpoint{8.290348in}{1.139798in}}%
\pgfpathcurveto{\pgfqpoint{8.301398in}{1.139798in}}{\pgfqpoint{8.311997in}{1.144188in}}{\pgfqpoint{8.319811in}{1.152002in}}%
\pgfpathcurveto{\pgfqpoint{8.327625in}{1.159815in}}{\pgfqpoint{8.332015in}{1.170414in}}{\pgfqpoint{8.332015in}{1.181465in}}%
\pgfpathcurveto{\pgfqpoint{8.332015in}{1.192515in}}{\pgfqpoint{8.327625in}{1.203114in}}{\pgfqpoint{8.319811in}{1.210927in}}%
\pgfpathcurveto{\pgfqpoint{8.311997in}{1.218741in}}{\pgfqpoint{8.301398in}{1.223131in}}{\pgfqpoint{8.290348in}{1.223131in}}%
\pgfpathcurveto{\pgfqpoint{8.279298in}{1.223131in}}{\pgfqpoint{8.268699in}{1.218741in}}{\pgfqpoint{8.260886in}{1.210927in}}%
\pgfpathcurveto{\pgfqpoint{8.253072in}{1.203114in}}{\pgfqpoint{8.248682in}{1.192515in}}{\pgfqpoint{8.248682in}{1.181465in}}%
\pgfpathcurveto{\pgfqpoint{8.248682in}{1.170414in}}{\pgfqpoint{8.253072in}{1.159815in}}{\pgfqpoint{8.260886in}{1.152002in}}%
\pgfpathcurveto{\pgfqpoint{8.268699in}{1.144188in}}{\pgfqpoint{8.279298in}{1.139798in}}{\pgfqpoint{8.290348in}{1.139798in}}%
\pgfpathlineto{\pgfqpoint{8.290348in}{1.139798in}}%
\pgfpathclose%
\pgfusepath{stroke}%
\end{pgfscope}%
\begin{pgfscope}%
\pgfpathrectangle{\pgfqpoint{7.512535in}{0.437222in}}{\pgfqpoint{6.275590in}{5.159444in}}%
\pgfusepath{clip}%
\pgfsetbuttcap%
\pgfsetroundjoin%
\pgfsetlinewidth{1.003750pt}%
\definecolor{currentstroke}{rgb}{0.827451,0.827451,0.827451}%
\pgfsetstrokecolor{currentstroke}%
\pgfsetstrokeopacity{0.800000}%
\pgfsetdash{}{0pt}%
\pgfpathmoveto{\pgfqpoint{11.645781in}{5.002607in}}%
\pgfpathcurveto{\pgfqpoint{11.656832in}{5.002607in}}{\pgfqpoint{11.667431in}{5.006998in}}{\pgfqpoint{11.675244in}{5.014811in}}%
\pgfpathcurveto{\pgfqpoint{11.683058in}{5.022625in}}{\pgfqpoint{11.687448in}{5.033224in}}{\pgfqpoint{11.687448in}{5.044274in}}%
\pgfpathcurveto{\pgfqpoint{11.687448in}{5.055324in}}{\pgfqpoint{11.683058in}{5.065923in}}{\pgfqpoint{11.675244in}{5.073737in}}%
\pgfpathcurveto{\pgfqpoint{11.667431in}{5.081550in}}{\pgfqpoint{11.656832in}{5.085941in}}{\pgfqpoint{11.645781in}{5.085941in}}%
\pgfpathcurveto{\pgfqpoint{11.634731in}{5.085941in}}{\pgfqpoint{11.624132in}{5.081550in}}{\pgfqpoint{11.616319in}{5.073737in}}%
\pgfpathcurveto{\pgfqpoint{11.608505in}{5.065923in}}{\pgfqpoint{11.604115in}{5.055324in}}{\pgfqpoint{11.604115in}{5.044274in}}%
\pgfpathcurveto{\pgfqpoint{11.604115in}{5.033224in}}{\pgfqpoint{11.608505in}{5.022625in}}{\pgfqpoint{11.616319in}{5.014811in}}%
\pgfpathcurveto{\pgfqpoint{11.624132in}{5.006998in}}{\pgfqpoint{11.634731in}{5.002607in}}{\pgfqpoint{11.645781in}{5.002607in}}%
\pgfpathlineto{\pgfqpoint{11.645781in}{5.002607in}}%
\pgfpathclose%
\pgfusepath{stroke}%
\end{pgfscope}%
\begin{pgfscope}%
\pgfpathrectangle{\pgfqpoint{7.512535in}{0.437222in}}{\pgfqpoint{6.275590in}{5.159444in}}%
\pgfusepath{clip}%
\pgfsetbuttcap%
\pgfsetroundjoin%
\pgfsetlinewidth{1.003750pt}%
\definecolor{currentstroke}{rgb}{0.827451,0.827451,0.827451}%
\pgfsetstrokecolor{currentstroke}%
\pgfsetstrokeopacity{0.800000}%
\pgfsetdash{}{0pt}%
\pgfpathmoveto{\pgfqpoint{10.369675in}{4.552452in}}%
\pgfpathcurveto{\pgfqpoint{10.380725in}{4.552452in}}{\pgfqpoint{10.391324in}{4.556842in}}{\pgfqpoint{10.399138in}{4.564656in}}%
\pgfpathcurveto{\pgfqpoint{10.406951in}{4.572470in}}{\pgfqpoint{10.411341in}{4.583069in}}{\pgfqpoint{10.411341in}{4.594119in}}%
\pgfpathcurveto{\pgfqpoint{10.411341in}{4.605169in}}{\pgfqpoint{10.406951in}{4.615768in}}{\pgfqpoint{10.399138in}{4.623582in}}%
\pgfpathcurveto{\pgfqpoint{10.391324in}{4.631395in}}{\pgfqpoint{10.380725in}{4.635785in}}{\pgfqpoint{10.369675in}{4.635785in}}%
\pgfpathcurveto{\pgfqpoint{10.358625in}{4.635785in}}{\pgfqpoint{10.348026in}{4.631395in}}{\pgfqpoint{10.340212in}{4.623582in}}%
\pgfpathcurveto{\pgfqpoint{10.332398in}{4.615768in}}{\pgfqpoint{10.328008in}{4.605169in}}{\pgfqpoint{10.328008in}{4.594119in}}%
\pgfpathcurveto{\pgfqpoint{10.328008in}{4.583069in}}{\pgfqpoint{10.332398in}{4.572470in}}{\pgfqpoint{10.340212in}{4.564656in}}%
\pgfpathcurveto{\pgfqpoint{10.348026in}{4.556842in}}{\pgfqpoint{10.358625in}{4.552452in}}{\pgfqpoint{10.369675in}{4.552452in}}%
\pgfpathlineto{\pgfqpoint{10.369675in}{4.552452in}}%
\pgfpathclose%
\pgfusepath{stroke}%
\end{pgfscope}%
\begin{pgfscope}%
\pgfpathrectangle{\pgfqpoint{7.512535in}{0.437222in}}{\pgfqpoint{6.275590in}{5.159444in}}%
\pgfusepath{clip}%
\pgfsetbuttcap%
\pgfsetroundjoin%
\pgfsetlinewidth{1.003750pt}%
\definecolor{currentstroke}{rgb}{0.827451,0.827451,0.827451}%
\pgfsetstrokecolor{currentstroke}%
\pgfsetstrokeopacity{0.800000}%
\pgfsetdash{}{0pt}%
\pgfpathmoveto{\pgfqpoint{9.790014in}{2.556537in}}%
\pgfpathcurveto{\pgfqpoint{9.801064in}{2.556537in}}{\pgfqpoint{9.811663in}{2.560927in}}{\pgfqpoint{9.819477in}{2.568740in}}%
\pgfpathcurveto{\pgfqpoint{9.827290in}{2.576554in}}{\pgfqpoint{9.831681in}{2.587153in}}{\pgfqpoint{9.831681in}{2.598203in}}%
\pgfpathcurveto{\pgfqpoint{9.831681in}{2.609253in}}{\pgfqpoint{9.827290in}{2.619852in}}{\pgfqpoint{9.819477in}{2.627666in}}%
\pgfpathcurveto{\pgfqpoint{9.811663in}{2.635480in}}{\pgfqpoint{9.801064in}{2.639870in}}{\pgfqpoint{9.790014in}{2.639870in}}%
\pgfpathcurveto{\pgfqpoint{9.778964in}{2.639870in}}{\pgfqpoint{9.768365in}{2.635480in}}{\pgfqpoint{9.760551in}{2.627666in}}%
\pgfpathcurveto{\pgfqpoint{9.752737in}{2.619852in}}{\pgfqpoint{9.748347in}{2.609253in}}{\pgfqpoint{9.748347in}{2.598203in}}%
\pgfpathcurveto{\pgfqpoint{9.748347in}{2.587153in}}{\pgfqpoint{9.752737in}{2.576554in}}{\pgfqpoint{9.760551in}{2.568740in}}%
\pgfpathcurveto{\pgfqpoint{9.768365in}{2.560927in}}{\pgfqpoint{9.778964in}{2.556537in}}{\pgfqpoint{9.790014in}{2.556537in}}%
\pgfpathlineto{\pgfqpoint{9.790014in}{2.556537in}}%
\pgfpathclose%
\pgfusepath{stroke}%
\end{pgfscope}%
\begin{pgfscope}%
\pgfpathrectangle{\pgfqpoint{7.512535in}{0.437222in}}{\pgfqpoint{6.275590in}{5.159444in}}%
\pgfusepath{clip}%
\pgfsetbuttcap%
\pgfsetroundjoin%
\pgfsetlinewidth{1.003750pt}%
\definecolor{currentstroke}{rgb}{0.827451,0.827451,0.827451}%
\pgfsetstrokecolor{currentstroke}%
\pgfsetstrokeopacity{0.800000}%
\pgfsetdash{}{0pt}%
\pgfpathmoveto{\pgfqpoint{11.046560in}{5.140196in}}%
\pgfpathcurveto{\pgfqpoint{11.057610in}{5.140196in}}{\pgfqpoint{11.068209in}{5.144587in}}{\pgfqpoint{11.076023in}{5.152400in}}%
\pgfpathcurveto{\pgfqpoint{11.083837in}{5.160214in}}{\pgfqpoint{11.088227in}{5.170813in}}{\pgfqpoint{11.088227in}{5.181863in}}%
\pgfpathcurveto{\pgfqpoint{11.088227in}{5.192913in}}{\pgfqpoint{11.083837in}{5.203512in}}{\pgfqpoint{11.076023in}{5.211326in}}%
\pgfpathcurveto{\pgfqpoint{11.068209in}{5.219139in}}{\pgfqpoint{11.057610in}{5.223530in}}{\pgfqpoint{11.046560in}{5.223530in}}%
\pgfpathcurveto{\pgfqpoint{11.035510in}{5.223530in}}{\pgfqpoint{11.024911in}{5.219139in}}{\pgfqpoint{11.017098in}{5.211326in}}%
\pgfpathcurveto{\pgfqpoint{11.009284in}{5.203512in}}{\pgfqpoint{11.004894in}{5.192913in}}{\pgfqpoint{11.004894in}{5.181863in}}%
\pgfpathcurveto{\pgfqpoint{11.004894in}{5.170813in}}{\pgfqpoint{11.009284in}{5.160214in}}{\pgfqpoint{11.017098in}{5.152400in}}%
\pgfpathcurveto{\pgfqpoint{11.024911in}{5.144587in}}{\pgfqpoint{11.035510in}{5.140196in}}{\pgfqpoint{11.046560in}{5.140196in}}%
\pgfpathlineto{\pgfqpoint{11.046560in}{5.140196in}}%
\pgfpathclose%
\pgfusepath{stroke}%
\end{pgfscope}%
\begin{pgfscope}%
\pgfpathrectangle{\pgfqpoint{7.512535in}{0.437222in}}{\pgfqpoint{6.275590in}{5.159444in}}%
\pgfusepath{clip}%
\pgfsetbuttcap%
\pgfsetroundjoin%
\pgfsetlinewidth{1.003750pt}%
\definecolor{currentstroke}{rgb}{0.827451,0.827451,0.827451}%
\pgfsetstrokecolor{currentstroke}%
\pgfsetstrokeopacity{0.800000}%
\pgfsetdash{}{0pt}%
\pgfpathmoveto{\pgfqpoint{12.495601in}{5.523529in}}%
\pgfpathcurveto{\pgfqpoint{12.506651in}{5.523529in}}{\pgfqpoint{12.517250in}{5.527919in}}{\pgfqpoint{12.525063in}{5.535733in}}%
\pgfpathcurveto{\pgfqpoint{12.532877in}{5.543547in}}{\pgfqpoint{12.537267in}{5.554146in}}{\pgfqpoint{12.537267in}{5.565196in}}%
\pgfpathcurveto{\pgfqpoint{12.537267in}{5.576246in}}{\pgfqpoint{12.532877in}{5.586845in}}{\pgfqpoint{12.525063in}{5.594658in}}%
\pgfpathcurveto{\pgfqpoint{12.517250in}{5.602472in}}{\pgfqpoint{12.506651in}{5.606862in}}{\pgfqpoint{12.495601in}{5.606862in}}%
\pgfpathcurveto{\pgfqpoint{12.484550in}{5.606862in}}{\pgfqpoint{12.473951in}{5.602472in}}{\pgfqpoint{12.466138in}{5.594658in}}%
\pgfpathcurveto{\pgfqpoint{12.458324in}{5.586845in}}{\pgfqpoint{12.453934in}{5.576246in}}{\pgfqpoint{12.453934in}{5.565196in}}%
\pgfpathcurveto{\pgfqpoint{12.453934in}{5.554146in}}{\pgfqpoint{12.458324in}{5.543547in}}{\pgfqpoint{12.466138in}{5.535733in}}%
\pgfpathcurveto{\pgfqpoint{12.473951in}{5.527919in}}{\pgfqpoint{12.484550in}{5.523529in}}{\pgfqpoint{12.495601in}{5.523529in}}%
\pgfpathlineto{\pgfqpoint{12.495601in}{5.523529in}}%
\pgfpathclose%
\pgfusepath{stroke}%
\end{pgfscope}%
\begin{pgfscope}%
\pgfpathrectangle{\pgfqpoint{7.512535in}{0.437222in}}{\pgfqpoint{6.275590in}{5.159444in}}%
\pgfusepath{clip}%
\pgfsetbuttcap%
\pgfsetroundjoin%
\pgfsetlinewidth{1.003750pt}%
\definecolor{currentstroke}{rgb}{0.827451,0.827451,0.827451}%
\pgfsetstrokecolor{currentstroke}%
\pgfsetstrokeopacity{0.800000}%
\pgfsetdash{}{0pt}%
\pgfpathmoveto{\pgfqpoint{11.374534in}{5.081213in}}%
\pgfpathcurveto{\pgfqpoint{11.385584in}{5.081213in}}{\pgfqpoint{11.396183in}{5.085604in}}{\pgfqpoint{11.403997in}{5.093417in}}%
\pgfpathcurveto{\pgfqpoint{11.411810in}{5.101231in}}{\pgfqpoint{11.416201in}{5.111830in}}{\pgfqpoint{11.416201in}{5.122880in}}%
\pgfpathcurveto{\pgfqpoint{11.416201in}{5.133930in}}{\pgfqpoint{11.411810in}{5.144529in}}{\pgfqpoint{11.403997in}{5.152343in}}%
\pgfpathcurveto{\pgfqpoint{11.396183in}{5.160157in}}{\pgfqpoint{11.385584in}{5.164547in}}{\pgfqpoint{11.374534in}{5.164547in}}%
\pgfpathcurveto{\pgfqpoint{11.363484in}{5.164547in}}{\pgfqpoint{11.352885in}{5.160157in}}{\pgfqpoint{11.345071in}{5.152343in}}%
\pgfpathcurveto{\pgfqpoint{11.337257in}{5.144529in}}{\pgfqpoint{11.332867in}{5.133930in}}{\pgfqpoint{11.332867in}{5.122880in}}%
\pgfpathcurveto{\pgfqpoint{11.332867in}{5.111830in}}{\pgfqpoint{11.337257in}{5.101231in}}{\pgfqpoint{11.345071in}{5.093417in}}%
\pgfpathcurveto{\pgfqpoint{11.352885in}{5.085604in}}{\pgfqpoint{11.363484in}{5.081213in}}{\pgfqpoint{11.374534in}{5.081213in}}%
\pgfpathlineto{\pgfqpoint{11.374534in}{5.081213in}}%
\pgfpathclose%
\pgfusepath{stroke}%
\end{pgfscope}%
\begin{pgfscope}%
\pgfpathrectangle{\pgfqpoint{7.512535in}{0.437222in}}{\pgfqpoint{6.275590in}{5.159444in}}%
\pgfusepath{clip}%
\pgfsetbuttcap%
\pgfsetroundjoin%
\pgfsetlinewidth{1.003750pt}%
\definecolor{currentstroke}{rgb}{0.827451,0.827451,0.827451}%
\pgfsetstrokecolor{currentstroke}%
\pgfsetstrokeopacity{0.800000}%
\pgfsetdash{}{0pt}%
\pgfpathmoveto{\pgfqpoint{9.130854in}{1.867691in}}%
\pgfpathcurveto{\pgfqpoint{9.141904in}{1.867691in}}{\pgfqpoint{9.152503in}{1.872081in}}{\pgfqpoint{9.160317in}{1.879895in}}%
\pgfpathcurveto{\pgfqpoint{9.168130in}{1.887709in}}{\pgfqpoint{9.172521in}{1.898308in}}{\pgfqpoint{9.172521in}{1.909358in}}%
\pgfpathcurveto{\pgfqpoint{9.172521in}{1.920408in}}{\pgfqpoint{9.168130in}{1.931007in}}{\pgfqpoint{9.160317in}{1.938821in}}%
\pgfpathcurveto{\pgfqpoint{9.152503in}{1.946634in}}{\pgfqpoint{9.141904in}{1.951024in}}{\pgfqpoint{9.130854in}{1.951024in}}%
\pgfpathcurveto{\pgfqpoint{9.119804in}{1.951024in}}{\pgfqpoint{9.109205in}{1.946634in}}{\pgfqpoint{9.101391in}{1.938821in}}%
\pgfpathcurveto{\pgfqpoint{9.093577in}{1.931007in}}{\pgfqpoint{9.089187in}{1.920408in}}{\pgfqpoint{9.089187in}{1.909358in}}%
\pgfpathcurveto{\pgfqpoint{9.089187in}{1.898308in}}{\pgfqpoint{9.093577in}{1.887709in}}{\pgfqpoint{9.101391in}{1.879895in}}%
\pgfpathcurveto{\pgfqpoint{9.109205in}{1.872081in}}{\pgfqpoint{9.119804in}{1.867691in}}{\pgfqpoint{9.130854in}{1.867691in}}%
\pgfpathlineto{\pgfqpoint{9.130854in}{1.867691in}}%
\pgfpathclose%
\pgfusepath{stroke}%
\end{pgfscope}%
\begin{pgfscope}%
\pgfpathrectangle{\pgfqpoint{7.512535in}{0.437222in}}{\pgfqpoint{6.275590in}{5.159444in}}%
\pgfusepath{clip}%
\pgfsetbuttcap%
\pgfsetroundjoin%
\pgfsetlinewidth{1.003750pt}%
\definecolor{currentstroke}{rgb}{0.827451,0.827451,0.827451}%
\pgfsetstrokecolor{currentstroke}%
\pgfsetstrokeopacity{0.800000}%
\pgfsetdash{}{0pt}%
\pgfpathmoveto{\pgfqpoint{13.419279in}{5.549860in}}%
\pgfpathcurveto{\pgfqpoint{13.430329in}{5.549860in}}{\pgfqpoint{13.440928in}{5.554250in}}{\pgfqpoint{13.448742in}{5.562064in}}%
\pgfpathcurveto{\pgfqpoint{13.456555in}{5.569877in}}{\pgfqpoint{13.460945in}{5.580476in}}{\pgfqpoint{13.460945in}{5.591526in}}%
\pgfpathcurveto{\pgfqpoint{13.460945in}{5.602576in}}{\pgfqpoint{13.456555in}{5.613175in}}{\pgfqpoint{13.448742in}{5.620989in}}%
\pgfpathcurveto{\pgfqpoint{13.440928in}{5.628803in}}{\pgfqpoint{13.430329in}{5.633193in}}{\pgfqpoint{13.419279in}{5.633193in}}%
\pgfpathcurveto{\pgfqpoint{13.408229in}{5.633193in}}{\pgfqpoint{13.397630in}{5.628803in}}{\pgfqpoint{13.389816in}{5.620989in}}%
\pgfpathcurveto{\pgfqpoint{13.382002in}{5.613175in}}{\pgfqpoint{13.377612in}{5.602576in}}{\pgfqpoint{13.377612in}{5.591526in}}%
\pgfpathcurveto{\pgfqpoint{13.377612in}{5.580476in}}{\pgfqpoint{13.382002in}{5.569877in}}{\pgfqpoint{13.389816in}{5.562064in}}%
\pgfpathcurveto{\pgfqpoint{13.397630in}{5.554250in}}{\pgfqpoint{13.408229in}{5.549860in}}{\pgfqpoint{13.419279in}{5.549860in}}%
\pgfpathlineto{\pgfqpoint{13.419279in}{5.549860in}}%
\pgfpathclose%
\pgfusepath{stroke}%
\end{pgfscope}%
\begin{pgfscope}%
\pgfpathrectangle{\pgfqpoint{7.512535in}{0.437222in}}{\pgfqpoint{6.275590in}{5.159444in}}%
\pgfusepath{clip}%
\pgfsetbuttcap%
\pgfsetroundjoin%
\pgfsetlinewidth{1.003750pt}%
\definecolor{currentstroke}{rgb}{0.827451,0.827451,0.827451}%
\pgfsetstrokecolor{currentstroke}%
\pgfsetstrokeopacity{0.800000}%
\pgfsetdash{}{0pt}%
\pgfpathmoveto{\pgfqpoint{9.001197in}{3.326629in}}%
\pgfpathcurveto{\pgfqpoint{9.012248in}{3.326629in}}{\pgfqpoint{9.022847in}{3.331019in}}{\pgfqpoint{9.030660in}{3.338833in}}%
\pgfpathcurveto{\pgfqpoint{9.038474in}{3.346646in}}{\pgfqpoint{9.042864in}{3.357246in}}{\pgfqpoint{9.042864in}{3.368296in}}%
\pgfpathcurveto{\pgfqpoint{9.042864in}{3.379346in}}{\pgfqpoint{9.038474in}{3.389945in}}{\pgfqpoint{9.030660in}{3.397758in}}%
\pgfpathcurveto{\pgfqpoint{9.022847in}{3.405572in}}{\pgfqpoint{9.012248in}{3.409962in}}{\pgfqpoint{9.001197in}{3.409962in}}%
\pgfpathcurveto{\pgfqpoint{8.990147in}{3.409962in}}{\pgfqpoint{8.979548in}{3.405572in}}{\pgfqpoint{8.971735in}{3.397758in}}%
\pgfpathcurveto{\pgfqpoint{8.963921in}{3.389945in}}{\pgfqpoint{8.959531in}{3.379346in}}{\pgfqpoint{8.959531in}{3.368296in}}%
\pgfpathcurveto{\pgfqpoint{8.959531in}{3.357246in}}{\pgfqpoint{8.963921in}{3.346646in}}{\pgfqpoint{8.971735in}{3.338833in}}%
\pgfpathcurveto{\pgfqpoint{8.979548in}{3.331019in}}{\pgfqpoint{8.990147in}{3.326629in}}{\pgfqpoint{9.001197in}{3.326629in}}%
\pgfpathlineto{\pgfqpoint{9.001197in}{3.326629in}}%
\pgfpathclose%
\pgfusepath{stroke}%
\end{pgfscope}%
\begin{pgfscope}%
\pgfpathrectangle{\pgfqpoint{7.512535in}{0.437222in}}{\pgfqpoint{6.275590in}{5.159444in}}%
\pgfusepath{clip}%
\pgfsetbuttcap%
\pgfsetroundjoin%
\pgfsetlinewidth{1.003750pt}%
\definecolor{currentstroke}{rgb}{0.827451,0.827451,0.827451}%
\pgfsetstrokecolor{currentstroke}%
\pgfsetstrokeopacity{0.800000}%
\pgfsetdash{}{0pt}%
\pgfpathmoveto{\pgfqpoint{11.629363in}{5.304399in}}%
\pgfpathcurveto{\pgfqpoint{11.640413in}{5.304399in}}{\pgfqpoint{11.651012in}{5.308790in}}{\pgfqpoint{11.658826in}{5.316603in}}%
\pgfpathcurveto{\pgfqpoint{11.666639in}{5.324417in}}{\pgfqpoint{11.671030in}{5.335016in}}{\pgfqpoint{11.671030in}{5.346066in}}%
\pgfpathcurveto{\pgfqpoint{11.671030in}{5.357116in}}{\pgfqpoint{11.666639in}{5.367715in}}{\pgfqpoint{11.658826in}{5.375529in}}%
\pgfpathcurveto{\pgfqpoint{11.651012in}{5.383342in}}{\pgfqpoint{11.640413in}{5.387733in}}{\pgfqpoint{11.629363in}{5.387733in}}%
\pgfpathcurveto{\pgfqpoint{11.618313in}{5.387733in}}{\pgfqpoint{11.607714in}{5.383342in}}{\pgfqpoint{11.599900in}{5.375529in}}%
\pgfpathcurveto{\pgfqpoint{11.592087in}{5.367715in}}{\pgfqpoint{11.587696in}{5.357116in}}{\pgfqpoint{11.587696in}{5.346066in}}%
\pgfpathcurveto{\pgfqpoint{11.587696in}{5.335016in}}{\pgfqpoint{11.592087in}{5.324417in}}{\pgfqpoint{11.599900in}{5.316603in}}%
\pgfpathcurveto{\pgfqpoint{11.607714in}{5.308790in}}{\pgfqpoint{11.618313in}{5.304399in}}{\pgfqpoint{11.629363in}{5.304399in}}%
\pgfpathlineto{\pgfqpoint{11.629363in}{5.304399in}}%
\pgfpathclose%
\pgfusepath{stroke}%
\end{pgfscope}%
\begin{pgfscope}%
\pgfpathrectangle{\pgfqpoint{7.512535in}{0.437222in}}{\pgfqpoint{6.275590in}{5.159444in}}%
\pgfusepath{clip}%
\pgfsetbuttcap%
\pgfsetroundjoin%
\pgfsetlinewidth{1.003750pt}%
\definecolor{currentstroke}{rgb}{0.827451,0.827451,0.827451}%
\pgfsetstrokecolor{currentstroke}%
\pgfsetstrokeopacity{0.800000}%
\pgfsetdash{}{0pt}%
\pgfpathmoveto{\pgfqpoint{10.023823in}{4.500549in}}%
\pgfpathcurveto{\pgfqpoint{10.034873in}{4.500549in}}{\pgfqpoint{10.045472in}{4.504939in}}{\pgfqpoint{10.053285in}{4.512753in}}%
\pgfpathcurveto{\pgfqpoint{10.061099in}{4.520566in}}{\pgfqpoint{10.065489in}{4.531165in}}{\pgfqpoint{10.065489in}{4.542215in}}%
\pgfpathcurveto{\pgfqpoint{10.065489in}{4.553266in}}{\pgfqpoint{10.061099in}{4.563865in}}{\pgfqpoint{10.053285in}{4.571678in}}%
\pgfpathcurveto{\pgfqpoint{10.045472in}{4.579492in}}{\pgfqpoint{10.034873in}{4.583882in}}{\pgfqpoint{10.023823in}{4.583882in}}%
\pgfpathcurveto{\pgfqpoint{10.012772in}{4.583882in}}{\pgfqpoint{10.002173in}{4.579492in}}{\pgfqpoint{9.994360in}{4.571678in}}%
\pgfpathcurveto{\pgfqpoint{9.986546in}{4.563865in}}{\pgfqpoint{9.982156in}{4.553266in}}{\pgfqpoint{9.982156in}{4.542215in}}%
\pgfpathcurveto{\pgfqpoint{9.982156in}{4.531165in}}{\pgfqpoint{9.986546in}{4.520566in}}{\pgfqpoint{9.994360in}{4.512753in}}%
\pgfpathcurveto{\pgfqpoint{10.002173in}{4.504939in}}{\pgfqpoint{10.012772in}{4.500549in}}{\pgfqpoint{10.023823in}{4.500549in}}%
\pgfpathlineto{\pgfqpoint{10.023823in}{4.500549in}}%
\pgfpathclose%
\pgfusepath{stroke}%
\end{pgfscope}%
\begin{pgfscope}%
\pgfpathrectangle{\pgfqpoint{7.512535in}{0.437222in}}{\pgfqpoint{6.275590in}{5.159444in}}%
\pgfusepath{clip}%
\pgfsetbuttcap%
\pgfsetroundjoin%
\pgfsetlinewidth{1.003750pt}%
\definecolor{currentstroke}{rgb}{0.827451,0.827451,0.827451}%
\pgfsetstrokecolor{currentstroke}%
\pgfsetstrokeopacity{0.800000}%
\pgfsetdash{}{0pt}%
\pgfpathmoveto{\pgfqpoint{10.259535in}{4.447570in}}%
\pgfpathcurveto{\pgfqpoint{10.270585in}{4.447570in}}{\pgfqpoint{10.281184in}{4.451960in}}{\pgfqpoint{10.288997in}{4.459773in}}%
\pgfpathcurveto{\pgfqpoint{10.296811in}{4.467587in}}{\pgfqpoint{10.301201in}{4.478186in}}{\pgfqpoint{10.301201in}{4.489236in}}%
\pgfpathcurveto{\pgfqpoint{10.301201in}{4.500286in}}{\pgfqpoint{10.296811in}{4.510885in}}{\pgfqpoint{10.288997in}{4.518699in}}%
\pgfpathcurveto{\pgfqpoint{10.281184in}{4.526513in}}{\pgfqpoint{10.270585in}{4.530903in}}{\pgfqpoint{10.259535in}{4.530903in}}%
\pgfpathcurveto{\pgfqpoint{10.248485in}{4.530903in}}{\pgfqpoint{10.237886in}{4.526513in}}{\pgfqpoint{10.230072in}{4.518699in}}%
\pgfpathcurveto{\pgfqpoint{10.222258in}{4.510885in}}{\pgfqpoint{10.217868in}{4.500286in}}{\pgfqpoint{10.217868in}{4.489236in}}%
\pgfpathcurveto{\pgfqpoint{10.217868in}{4.478186in}}{\pgfqpoint{10.222258in}{4.467587in}}{\pgfqpoint{10.230072in}{4.459773in}}%
\pgfpathcurveto{\pgfqpoint{10.237886in}{4.451960in}}{\pgfqpoint{10.248485in}{4.447570in}}{\pgfqpoint{10.259535in}{4.447570in}}%
\pgfpathlineto{\pgfqpoint{10.259535in}{4.447570in}}%
\pgfpathclose%
\pgfusepath{stroke}%
\end{pgfscope}%
\begin{pgfscope}%
\pgfpathrectangle{\pgfqpoint{7.512535in}{0.437222in}}{\pgfqpoint{6.275590in}{5.159444in}}%
\pgfusepath{clip}%
\pgfsetbuttcap%
\pgfsetroundjoin%
\pgfsetlinewidth{1.003750pt}%
\definecolor{currentstroke}{rgb}{0.827451,0.827451,0.827451}%
\pgfsetstrokecolor{currentstroke}%
\pgfsetstrokeopacity{0.800000}%
\pgfsetdash{}{0pt}%
\pgfpathmoveto{\pgfqpoint{10.780511in}{5.040147in}}%
\pgfpathcurveto{\pgfqpoint{10.791561in}{5.040147in}}{\pgfqpoint{10.802160in}{5.044537in}}{\pgfqpoint{10.809973in}{5.052350in}}%
\pgfpathcurveto{\pgfqpoint{10.817787in}{5.060164in}}{\pgfqpoint{10.822177in}{5.070763in}}{\pgfqpoint{10.822177in}{5.081813in}}%
\pgfpathcurveto{\pgfqpoint{10.822177in}{5.092863in}}{\pgfqpoint{10.817787in}{5.103462in}}{\pgfqpoint{10.809973in}{5.111276in}}%
\pgfpathcurveto{\pgfqpoint{10.802160in}{5.119090in}}{\pgfqpoint{10.791561in}{5.123480in}}{\pgfqpoint{10.780511in}{5.123480in}}%
\pgfpathcurveto{\pgfqpoint{10.769461in}{5.123480in}}{\pgfqpoint{10.758862in}{5.119090in}}{\pgfqpoint{10.751048in}{5.111276in}}%
\pgfpathcurveto{\pgfqpoint{10.743234in}{5.103462in}}{\pgfqpoint{10.738844in}{5.092863in}}{\pgfqpoint{10.738844in}{5.081813in}}%
\pgfpathcurveto{\pgfqpoint{10.738844in}{5.070763in}}{\pgfqpoint{10.743234in}{5.060164in}}{\pgfqpoint{10.751048in}{5.052350in}}%
\pgfpathcurveto{\pgfqpoint{10.758862in}{5.044537in}}{\pgfqpoint{10.769461in}{5.040147in}}{\pgfqpoint{10.780511in}{5.040147in}}%
\pgfpathlineto{\pgfqpoint{10.780511in}{5.040147in}}%
\pgfpathclose%
\pgfusepath{stroke}%
\end{pgfscope}%
\begin{pgfscope}%
\pgfpathrectangle{\pgfqpoint{7.512535in}{0.437222in}}{\pgfqpoint{6.275590in}{5.159444in}}%
\pgfusepath{clip}%
\pgfsetbuttcap%
\pgfsetroundjoin%
\pgfsetlinewidth{1.003750pt}%
\definecolor{currentstroke}{rgb}{0.827451,0.827451,0.827451}%
\pgfsetstrokecolor{currentstroke}%
\pgfsetstrokeopacity{0.800000}%
\pgfsetdash{}{0pt}%
\pgfpathmoveto{\pgfqpoint{8.277312in}{1.563305in}}%
\pgfpathcurveto{\pgfqpoint{8.288363in}{1.563305in}}{\pgfqpoint{8.298962in}{1.567695in}}{\pgfqpoint{8.306775in}{1.575509in}}%
\pgfpathcurveto{\pgfqpoint{8.314589in}{1.583322in}}{\pgfqpoint{8.318979in}{1.593921in}}{\pgfqpoint{8.318979in}{1.604971in}}%
\pgfpathcurveto{\pgfqpoint{8.318979in}{1.616021in}}{\pgfqpoint{8.314589in}{1.626621in}}{\pgfqpoint{8.306775in}{1.634434in}}%
\pgfpathcurveto{\pgfqpoint{8.298962in}{1.642248in}}{\pgfqpoint{8.288363in}{1.646638in}}{\pgfqpoint{8.277312in}{1.646638in}}%
\pgfpathcurveto{\pgfqpoint{8.266262in}{1.646638in}}{\pgfqpoint{8.255663in}{1.642248in}}{\pgfqpoint{8.247850in}{1.634434in}}%
\pgfpathcurveto{\pgfqpoint{8.240036in}{1.626621in}}{\pgfqpoint{8.235646in}{1.616021in}}{\pgfqpoint{8.235646in}{1.604971in}}%
\pgfpathcurveto{\pgfqpoint{8.235646in}{1.593921in}}{\pgfqpoint{8.240036in}{1.583322in}}{\pgfqpoint{8.247850in}{1.575509in}}%
\pgfpathcurveto{\pgfqpoint{8.255663in}{1.567695in}}{\pgfqpoint{8.266262in}{1.563305in}}{\pgfqpoint{8.277312in}{1.563305in}}%
\pgfpathlineto{\pgfqpoint{8.277312in}{1.563305in}}%
\pgfpathclose%
\pgfusepath{stroke}%
\end{pgfscope}%
\begin{pgfscope}%
\pgfpathrectangle{\pgfqpoint{7.512535in}{0.437222in}}{\pgfqpoint{6.275590in}{5.159444in}}%
\pgfusepath{clip}%
\pgfsetbuttcap%
\pgfsetroundjoin%
\pgfsetlinewidth{1.003750pt}%
\definecolor{currentstroke}{rgb}{0.827451,0.827451,0.827451}%
\pgfsetstrokecolor{currentstroke}%
\pgfsetstrokeopacity{0.800000}%
\pgfsetdash{}{0pt}%
\pgfpathmoveto{\pgfqpoint{11.888148in}{5.414422in}}%
\pgfpathcurveto{\pgfqpoint{11.899198in}{5.414422in}}{\pgfqpoint{11.909797in}{5.418812in}}{\pgfqpoint{11.917611in}{5.426626in}}%
\pgfpathcurveto{\pgfqpoint{11.925424in}{5.434440in}}{\pgfqpoint{11.929815in}{5.445039in}}{\pgfqpoint{11.929815in}{5.456089in}}%
\pgfpathcurveto{\pgfqpoint{11.929815in}{5.467139in}}{\pgfqpoint{11.925424in}{5.477738in}}{\pgfqpoint{11.917611in}{5.485551in}}%
\pgfpathcurveto{\pgfqpoint{11.909797in}{5.493365in}}{\pgfqpoint{11.899198in}{5.497755in}}{\pgfqpoint{11.888148in}{5.497755in}}%
\pgfpathcurveto{\pgfqpoint{11.877098in}{5.497755in}}{\pgfqpoint{11.866499in}{5.493365in}}{\pgfqpoint{11.858685in}{5.485551in}}%
\pgfpathcurveto{\pgfqpoint{11.850872in}{5.477738in}}{\pgfqpoint{11.846481in}{5.467139in}}{\pgfqpoint{11.846481in}{5.456089in}}%
\pgfpathcurveto{\pgfqpoint{11.846481in}{5.445039in}}{\pgfqpoint{11.850872in}{5.434440in}}{\pgfqpoint{11.858685in}{5.426626in}}%
\pgfpathcurveto{\pgfqpoint{11.866499in}{5.418812in}}{\pgfqpoint{11.877098in}{5.414422in}}{\pgfqpoint{11.888148in}{5.414422in}}%
\pgfpathlineto{\pgfqpoint{11.888148in}{5.414422in}}%
\pgfpathclose%
\pgfusepath{stroke}%
\end{pgfscope}%
\begin{pgfscope}%
\pgfpathrectangle{\pgfqpoint{7.512535in}{0.437222in}}{\pgfqpoint{6.275590in}{5.159444in}}%
\pgfusepath{clip}%
\pgfsetbuttcap%
\pgfsetroundjoin%
\pgfsetlinewidth{1.003750pt}%
\definecolor{currentstroke}{rgb}{0.827451,0.827451,0.827451}%
\pgfsetstrokecolor{currentstroke}%
\pgfsetstrokeopacity{0.800000}%
\pgfsetdash{}{0pt}%
\pgfpathmoveto{\pgfqpoint{9.807010in}{4.531380in}}%
\pgfpathcurveto{\pgfqpoint{9.818060in}{4.531380in}}{\pgfqpoint{9.828659in}{4.535770in}}{\pgfqpoint{9.836473in}{4.543584in}}%
\pgfpathcurveto{\pgfqpoint{9.844287in}{4.551397in}}{\pgfqpoint{9.848677in}{4.561996in}}{\pgfqpoint{9.848677in}{4.573046in}}%
\pgfpathcurveto{\pgfqpoint{9.848677in}{4.584096in}}{\pgfqpoint{9.844287in}{4.594695in}}{\pgfqpoint{9.836473in}{4.602509in}}%
\pgfpathcurveto{\pgfqpoint{9.828659in}{4.610323in}}{\pgfqpoint{9.818060in}{4.614713in}}{\pgfqpoint{9.807010in}{4.614713in}}%
\pgfpathcurveto{\pgfqpoint{9.795960in}{4.614713in}}{\pgfqpoint{9.785361in}{4.610323in}}{\pgfqpoint{9.777548in}{4.602509in}}%
\pgfpathcurveto{\pgfqpoint{9.769734in}{4.594695in}}{\pgfqpoint{9.765344in}{4.584096in}}{\pgfqpoint{9.765344in}{4.573046in}}%
\pgfpathcurveto{\pgfqpoint{9.765344in}{4.561996in}}{\pgfqpoint{9.769734in}{4.551397in}}{\pgfqpoint{9.777548in}{4.543584in}}%
\pgfpathcurveto{\pgfqpoint{9.785361in}{4.535770in}}{\pgfqpoint{9.795960in}{4.531380in}}{\pgfqpoint{9.807010in}{4.531380in}}%
\pgfpathlineto{\pgfqpoint{9.807010in}{4.531380in}}%
\pgfpathclose%
\pgfusepath{stroke}%
\end{pgfscope}%
\begin{pgfscope}%
\pgfpathrectangle{\pgfqpoint{7.512535in}{0.437222in}}{\pgfqpoint{6.275590in}{5.159444in}}%
\pgfusepath{clip}%
\pgfsetbuttcap%
\pgfsetroundjoin%
\pgfsetlinewidth{1.003750pt}%
\definecolor{currentstroke}{rgb}{0.827451,0.827451,0.827451}%
\pgfsetstrokecolor{currentstroke}%
\pgfsetstrokeopacity{0.800000}%
\pgfsetdash{}{0pt}%
\pgfpathmoveto{\pgfqpoint{7.860715in}{0.988624in}}%
\pgfpathcurveto{\pgfqpoint{7.871765in}{0.988624in}}{\pgfqpoint{7.882364in}{0.993014in}}{\pgfqpoint{7.890178in}{1.000828in}}%
\pgfpathcurveto{\pgfqpoint{7.897991in}{1.008641in}}{\pgfqpoint{7.902381in}{1.019240in}}{\pgfqpoint{7.902381in}{1.030290in}}%
\pgfpathcurveto{\pgfqpoint{7.902381in}{1.041341in}}{\pgfqpoint{7.897991in}{1.051940in}}{\pgfqpoint{7.890178in}{1.059753in}}%
\pgfpathcurveto{\pgfqpoint{7.882364in}{1.067567in}}{\pgfqpoint{7.871765in}{1.071957in}}{\pgfqpoint{7.860715in}{1.071957in}}%
\pgfpathcurveto{\pgfqpoint{7.849665in}{1.071957in}}{\pgfqpoint{7.839066in}{1.067567in}}{\pgfqpoint{7.831252in}{1.059753in}}%
\pgfpathcurveto{\pgfqpoint{7.823438in}{1.051940in}}{\pgfqpoint{7.819048in}{1.041341in}}{\pgfqpoint{7.819048in}{1.030290in}}%
\pgfpathcurveto{\pgfqpoint{7.819048in}{1.019240in}}{\pgfqpoint{7.823438in}{1.008641in}}{\pgfqpoint{7.831252in}{1.000828in}}%
\pgfpathcurveto{\pgfqpoint{7.839066in}{0.993014in}}{\pgfqpoint{7.849665in}{0.988624in}}{\pgfqpoint{7.860715in}{0.988624in}}%
\pgfpathlineto{\pgfqpoint{7.860715in}{0.988624in}}%
\pgfpathclose%
\pgfusepath{stroke}%
\end{pgfscope}%
\begin{pgfscope}%
\pgfpathrectangle{\pgfqpoint{7.512535in}{0.437222in}}{\pgfqpoint{6.275590in}{5.159444in}}%
\pgfusepath{clip}%
\pgfsetbuttcap%
\pgfsetroundjoin%
\pgfsetlinewidth{1.003750pt}%
\definecolor{currentstroke}{rgb}{0.827451,0.827451,0.827451}%
\pgfsetstrokecolor{currentstroke}%
\pgfsetstrokeopacity{0.800000}%
\pgfsetdash{}{0pt}%
\pgfpathmoveto{\pgfqpoint{7.887873in}{1.346375in}}%
\pgfpathcurveto{\pgfqpoint{7.898923in}{1.346375in}}{\pgfqpoint{7.909522in}{1.350765in}}{\pgfqpoint{7.917336in}{1.358578in}}%
\pgfpathcurveto{\pgfqpoint{7.925149in}{1.366392in}}{\pgfqpoint{7.929540in}{1.376991in}}{\pgfqpoint{7.929540in}{1.388041in}}%
\pgfpathcurveto{\pgfqpoint{7.929540in}{1.399091in}}{\pgfqpoint{7.925149in}{1.409690in}}{\pgfqpoint{7.917336in}{1.417504in}}%
\pgfpathcurveto{\pgfqpoint{7.909522in}{1.425318in}}{\pgfqpoint{7.898923in}{1.429708in}}{\pgfqpoint{7.887873in}{1.429708in}}%
\pgfpathcurveto{\pgfqpoint{7.876823in}{1.429708in}}{\pgfqpoint{7.866224in}{1.425318in}}{\pgfqpoint{7.858410in}{1.417504in}}%
\pgfpathcurveto{\pgfqpoint{7.850596in}{1.409690in}}{\pgfqpoint{7.846206in}{1.399091in}}{\pgfqpoint{7.846206in}{1.388041in}}%
\pgfpathcurveto{\pgfqpoint{7.846206in}{1.376991in}}{\pgfqpoint{7.850596in}{1.366392in}}{\pgfqpoint{7.858410in}{1.358578in}}%
\pgfpathcurveto{\pgfqpoint{7.866224in}{1.350765in}}{\pgfqpoint{7.876823in}{1.346375in}}{\pgfqpoint{7.887873in}{1.346375in}}%
\pgfpathlineto{\pgfqpoint{7.887873in}{1.346375in}}%
\pgfpathclose%
\pgfusepath{stroke}%
\end{pgfscope}%
\begin{pgfscope}%
\pgfpathrectangle{\pgfqpoint{7.512535in}{0.437222in}}{\pgfqpoint{6.275590in}{5.159444in}}%
\pgfusepath{clip}%
\pgfsetbuttcap%
\pgfsetroundjoin%
\pgfsetlinewidth{1.003750pt}%
\definecolor{currentstroke}{rgb}{0.827451,0.827451,0.827451}%
\pgfsetstrokecolor{currentstroke}%
\pgfsetstrokeopacity{0.800000}%
\pgfsetdash{}{0pt}%
\pgfpathmoveto{\pgfqpoint{7.679894in}{0.493855in}}%
\pgfpathcurveto{\pgfqpoint{7.690944in}{0.493855in}}{\pgfqpoint{7.701543in}{0.498245in}}{\pgfqpoint{7.709357in}{0.506059in}}%
\pgfpathcurveto{\pgfqpoint{7.717170in}{0.513872in}}{\pgfqpoint{7.721560in}{0.524471in}}{\pgfqpoint{7.721560in}{0.535522in}}%
\pgfpathcurveto{\pgfqpoint{7.721560in}{0.546572in}}{\pgfqpoint{7.717170in}{0.557171in}}{\pgfqpoint{7.709357in}{0.564984in}}%
\pgfpathcurveto{\pgfqpoint{7.701543in}{0.572798in}}{\pgfqpoint{7.690944in}{0.577188in}}{\pgfqpoint{7.679894in}{0.577188in}}%
\pgfpathcurveto{\pgfqpoint{7.668844in}{0.577188in}}{\pgfqpoint{7.658245in}{0.572798in}}{\pgfqpoint{7.650431in}{0.564984in}}%
\pgfpathcurveto{\pgfqpoint{7.642617in}{0.557171in}}{\pgfqpoint{7.638227in}{0.546572in}}{\pgfqpoint{7.638227in}{0.535522in}}%
\pgfpathcurveto{\pgfqpoint{7.638227in}{0.524471in}}{\pgfqpoint{7.642617in}{0.513872in}}{\pgfqpoint{7.650431in}{0.506059in}}%
\pgfpathcurveto{\pgfqpoint{7.658245in}{0.498245in}}{\pgfqpoint{7.668844in}{0.493855in}}{\pgfqpoint{7.679894in}{0.493855in}}%
\pgfpathlineto{\pgfqpoint{7.679894in}{0.493855in}}%
\pgfpathclose%
\pgfusepath{stroke}%
\end{pgfscope}%
\begin{pgfscope}%
\pgfpathrectangle{\pgfqpoint{7.512535in}{0.437222in}}{\pgfqpoint{6.275590in}{5.159444in}}%
\pgfusepath{clip}%
\pgfsetbuttcap%
\pgfsetroundjoin%
\pgfsetlinewidth{1.003750pt}%
\definecolor{currentstroke}{rgb}{0.827451,0.827451,0.827451}%
\pgfsetstrokecolor{currentstroke}%
\pgfsetstrokeopacity{0.800000}%
\pgfsetdash{}{0pt}%
\pgfpathmoveto{\pgfqpoint{8.724795in}{1.722485in}}%
\pgfpathcurveto{\pgfqpoint{8.735845in}{1.722485in}}{\pgfqpoint{8.746444in}{1.726876in}}{\pgfqpoint{8.754258in}{1.734689in}}%
\pgfpathcurveto{\pgfqpoint{8.762071in}{1.742503in}}{\pgfqpoint{8.766461in}{1.753102in}}{\pgfqpoint{8.766461in}{1.764152in}}%
\pgfpathcurveto{\pgfqpoint{8.766461in}{1.775202in}}{\pgfqpoint{8.762071in}{1.785801in}}{\pgfqpoint{8.754258in}{1.793615in}}%
\pgfpathcurveto{\pgfqpoint{8.746444in}{1.801428in}}{\pgfqpoint{8.735845in}{1.805819in}}{\pgfqpoint{8.724795in}{1.805819in}}%
\pgfpathcurveto{\pgfqpoint{8.713745in}{1.805819in}}{\pgfqpoint{8.703146in}{1.801428in}}{\pgfqpoint{8.695332in}{1.793615in}}%
\pgfpathcurveto{\pgfqpoint{8.687518in}{1.785801in}}{\pgfqpoint{8.683128in}{1.775202in}}{\pgfqpoint{8.683128in}{1.764152in}}%
\pgfpathcurveto{\pgfqpoint{8.683128in}{1.753102in}}{\pgfqpoint{8.687518in}{1.742503in}}{\pgfqpoint{8.695332in}{1.734689in}}%
\pgfpathcurveto{\pgfqpoint{8.703146in}{1.726876in}}{\pgfqpoint{8.713745in}{1.722485in}}{\pgfqpoint{8.724795in}{1.722485in}}%
\pgfpathlineto{\pgfqpoint{8.724795in}{1.722485in}}%
\pgfpathclose%
\pgfusepath{stroke}%
\end{pgfscope}%
\begin{pgfscope}%
\pgfpathrectangle{\pgfqpoint{7.512535in}{0.437222in}}{\pgfqpoint{6.275590in}{5.159444in}}%
\pgfusepath{clip}%
\pgfsetbuttcap%
\pgfsetroundjoin%
\pgfsetlinewidth{1.003750pt}%
\definecolor{currentstroke}{rgb}{0.827451,0.827451,0.827451}%
\pgfsetstrokecolor{currentstroke}%
\pgfsetstrokeopacity{0.800000}%
\pgfsetdash{}{0pt}%
\pgfpathmoveto{\pgfqpoint{9.928499in}{4.056499in}}%
\pgfpathcurveto{\pgfqpoint{9.939549in}{4.056499in}}{\pgfqpoint{9.950148in}{4.060889in}}{\pgfqpoint{9.957962in}{4.068703in}}%
\pgfpathcurveto{\pgfqpoint{9.965775in}{4.076516in}}{\pgfqpoint{9.970165in}{4.087115in}}{\pgfqpoint{9.970165in}{4.098166in}}%
\pgfpathcurveto{\pgfqpoint{9.970165in}{4.109216in}}{\pgfqpoint{9.965775in}{4.119815in}}{\pgfqpoint{9.957962in}{4.127628in}}%
\pgfpathcurveto{\pgfqpoint{9.950148in}{4.135442in}}{\pgfqpoint{9.939549in}{4.139832in}}{\pgfqpoint{9.928499in}{4.139832in}}%
\pgfpathcurveto{\pgfqpoint{9.917449in}{4.139832in}}{\pgfqpoint{9.906850in}{4.135442in}}{\pgfqpoint{9.899036in}{4.127628in}}%
\pgfpathcurveto{\pgfqpoint{9.891222in}{4.119815in}}{\pgfqpoint{9.886832in}{4.109216in}}{\pgfqpoint{9.886832in}{4.098166in}}%
\pgfpathcurveto{\pgfqpoint{9.886832in}{4.087115in}}{\pgfqpoint{9.891222in}{4.076516in}}{\pgfqpoint{9.899036in}{4.068703in}}%
\pgfpathcurveto{\pgfqpoint{9.906850in}{4.060889in}}{\pgfqpoint{9.917449in}{4.056499in}}{\pgfqpoint{9.928499in}{4.056499in}}%
\pgfpathlineto{\pgfqpoint{9.928499in}{4.056499in}}%
\pgfpathclose%
\pgfusepath{stroke}%
\end{pgfscope}%
\begin{pgfscope}%
\pgfpathrectangle{\pgfqpoint{7.512535in}{0.437222in}}{\pgfqpoint{6.275590in}{5.159444in}}%
\pgfusepath{clip}%
\pgfsetbuttcap%
\pgfsetroundjoin%
\pgfsetlinewidth{1.003750pt}%
\definecolor{currentstroke}{rgb}{0.827451,0.827451,0.827451}%
\pgfsetstrokecolor{currentstroke}%
\pgfsetstrokeopacity{0.800000}%
\pgfsetdash{}{0pt}%
\pgfpathmoveto{\pgfqpoint{9.578307in}{2.959155in}}%
\pgfpathcurveto{\pgfqpoint{9.589357in}{2.959155in}}{\pgfqpoint{9.599956in}{2.963545in}}{\pgfqpoint{9.607770in}{2.971359in}}%
\pgfpathcurveto{\pgfqpoint{9.615584in}{2.979173in}}{\pgfqpoint{9.619974in}{2.989772in}}{\pgfqpoint{9.619974in}{3.000822in}}%
\pgfpathcurveto{\pgfqpoint{9.619974in}{3.011872in}}{\pgfqpoint{9.615584in}{3.022471in}}{\pgfqpoint{9.607770in}{3.030284in}}%
\pgfpathcurveto{\pgfqpoint{9.599956in}{3.038098in}}{\pgfqpoint{9.589357in}{3.042488in}}{\pgfqpoint{9.578307in}{3.042488in}}%
\pgfpathcurveto{\pgfqpoint{9.567257in}{3.042488in}}{\pgfqpoint{9.556658in}{3.038098in}}{\pgfqpoint{9.548845in}{3.030284in}}%
\pgfpathcurveto{\pgfqpoint{9.541031in}{3.022471in}}{\pgfqpoint{9.536641in}{3.011872in}}{\pgfqpoint{9.536641in}{3.000822in}}%
\pgfpathcurveto{\pgfqpoint{9.536641in}{2.989772in}}{\pgfqpoint{9.541031in}{2.979173in}}{\pgfqpoint{9.548845in}{2.971359in}}%
\pgfpathcurveto{\pgfqpoint{9.556658in}{2.963545in}}{\pgfqpoint{9.567257in}{2.959155in}}{\pgfqpoint{9.578307in}{2.959155in}}%
\pgfpathlineto{\pgfqpoint{9.578307in}{2.959155in}}%
\pgfpathclose%
\pgfusepath{stroke}%
\end{pgfscope}%
\begin{pgfscope}%
\pgfpathrectangle{\pgfqpoint{7.512535in}{0.437222in}}{\pgfqpoint{6.275590in}{5.159444in}}%
\pgfusepath{clip}%
\pgfsetbuttcap%
\pgfsetroundjoin%
\pgfsetlinewidth{1.003750pt}%
\definecolor{currentstroke}{rgb}{0.827451,0.827451,0.827451}%
\pgfsetstrokecolor{currentstroke}%
\pgfsetstrokeopacity{0.800000}%
\pgfsetdash{}{0pt}%
\pgfpathmoveto{\pgfqpoint{10.698510in}{3.970272in}}%
\pgfpathcurveto{\pgfqpoint{10.709560in}{3.970272in}}{\pgfqpoint{10.720159in}{3.974662in}}{\pgfqpoint{10.727973in}{3.982476in}}%
\pgfpathcurveto{\pgfqpoint{10.735787in}{3.990290in}}{\pgfqpoint{10.740177in}{4.000889in}}{\pgfqpoint{10.740177in}{4.011939in}}%
\pgfpathcurveto{\pgfqpoint{10.740177in}{4.022989in}}{\pgfqpoint{10.735787in}{4.033588in}}{\pgfqpoint{10.727973in}{4.041402in}}%
\pgfpathcurveto{\pgfqpoint{10.720159in}{4.049215in}}{\pgfqpoint{10.709560in}{4.053605in}}{\pgfqpoint{10.698510in}{4.053605in}}%
\pgfpathcurveto{\pgfqpoint{10.687460in}{4.053605in}}{\pgfqpoint{10.676861in}{4.049215in}}{\pgfqpoint{10.669047in}{4.041402in}}%
\pgfpathcurveto{\pgfqpoint{10.661234in}{4.033588in}}{\pgfqpoint{10.656844in}{4.022989in}}{\pgfqpoint{10.656844in}{4.011939in}}%
\pgfpathcurveto{\pgfqpoint{10.656844in}{4.000889in}}{\pgfqpoint{10.661234in}{3.990290in}}{\pgfqpoint{10.669047in}{3.982476in}}%
\pgfpathcurveto{\pgfqpoint{10.676861in}{3.974662in}}{\pgfqpoint{10.687460in}{3.970272in}}{\pgfqpoint{10.698510in}{3.970272in}}%
\pgfpathlineto{\pgfqpoint{10.698510in}{3.970272in}}%
\pgfpathclose%
\pgfusepath{stroke}%
\end{pgfscope}%
\begin{pgfscope}%
\pgfpathrectangle{\pgfqpoint{7.512535in}{0.437222in}}{\pgfqpoint{6.275590in}{5.159444in}}%
\pgfusepath{clip}%
\pgfsetbuttcap%
\pgfsetroundjoin%
\pgfsetlinewidth{1.003750pt}%
\definecolor{currentstroke}{rgb}{0.827451,0.827451,0.827451}%
\pgfsetstrokecolor{currentstroke}%
\pgfsetstrokeopacity{0.800000}%
\pgfsetdash{}{0pt}%
\pgfpathmoveto{\pgfqpoint{10.994755in}{5.060707in}}%
\pgfpathcurveto{\pgfqpoint{11.005805in}{5.060707in}}{\pgfqpoint{11.016404in}{5.065097in}}{\pgfqpoint{11.024218in}{5.072911in}}%
\pgfpathcurveto{\pgfqpoint{11.032031in}{5.080725in}}{\pgfqpoint{11.036422in}{5.091324in}}{\pgfqpoint{11.036422in}{5.102374in}}%
\pgfpathcurveto{\pgfqpoint{11.036422in}{5.113424in}}{\pgfqpoint{11.032031in}{5.124023in}}{\pgfqpoint{11.024218in}{5.131837in}}%
\pgfpathcurveto{\pgfqpoint{11.016404in}{5.139650in}}{\pgfqpoint{11.005805in}{5.144040in}}{\pgfqpoint{10.994755in}{5.144040in}}%
\pgfpathcurveto{\pgfqpoint{10.983705in}{5.144040in}}{\pgfqpoint{10.973106in}{5.139650in}}{\pgfqpoint{10.965292in}{5.131837in}}%
\pgfpathcurveto{\pgfqpoint{10.957479in}{5.124023in}}{\pgfqpoint{10.953088in}{5.113424in}}{\pgfqpoint{10.953088in}{5.102374in}}%
\pgfpathcurveto{\pgfqpoint{10.953088in}{5.091324in}}{\pgfqpoint{10.957479in}{5.080725in}}{\pgfqpoint{10.965292in}{5.072911in}}%
\pgfpathcurveto{\pgfqpoint{10.973106in}{5.065097in}}{\pgfqpoint{10.983705in}{5.060707in}}{\pgfqpoint{10.994755in}{5.060707in}}%
\pgfpathlineto{\pgfqpoint{10.994755in}{5.060707in}}%
\pgfpathclose%
\pgfusepath{stroke}%
\end{pgfscope}%
\begin{pgfscope}%
\pgfpathrectangle{\pgfqpoint{7.512535in}{0.437222in}}{\pgfqpoint{6.275590in}{5.159444in}}%
\pgfusepath{clip}%
\pgfsetbuttcap%
\pgfsetroundjoin%
\pgfsetlinewidth{1.003750pt}%
\definecolor{currentstroke}{rgb}{0.827451,0.827451,0.827451}%
\pgfsetstrokecolor{currentstroke}%
\pgfsetstrokeopacity{0.800000}%
\pgfsetdash{}{0pt}%
\pgfpathmoveto{\pgfqpoint{12.650243in}{5.481847in}}%
\pgfpathcurveto{\pgfqpoint{12.661293in}{5.481847in}}{\pgfqpoint{12.671892in}{5.486238in}}{\pgfqpoint{12.679706in}{5.494051in}}%
\pgfpathcurveto{\pgfqpoint{12.687519in}{5.501865in}}{\pgfqpoint{12.691909in}{5.512464in}}{\pgfqpoint{12.691909in}{5.523514in}}%
\pgfpathcurveto{\pgfqpoint{12.691909in}{5.534564in}}{\pgfqpoint{12.687519in}{5.545163in}}{\pgfqpoint{12.679706in}{5.552977in}}%
\pgfpathcurveto{\pgfqpoint{12.671892in}{5.560791in}}{\pgfqpoint{12.661293in}{5.565181in}}{\pgfqpoint{12.650243in}{5.565181in}}%
\pgfpathcurveto{\pgfqpoint{12.639193in}{5.565181in}}{\pgfqpoint{12.628594in}{5.560791in}}{\pgfqpoint{12.620780in}{5.552977in}}%
\pgfpathcurveto{\pgfqpoint{12.612966in}{5.545163in}}{\pgfqpoint{12.608576in}{5.534564in}}{\pgfqpoint{12.608576in}{5.523514in}}%
\pgfpathcurveto{\pgfqpoint{12.608576in}{5.512464in}}{\pgfqpoint{12.612966in}{5.501865in}}{\pgfqpoint{12.620780in}{5.494051in}}%
\pgfpathcurveto{\pgfqpoint{12.628594in}{5.486238in}}{\pgfqpoint{12.639193in}{5.481847in}}{\pgfqpoint{12.650243in}{5.481847in}}%
\pgfpathlineto{\pgfqpoint{12.650243in}{5.481847in}}%
\pgfpathclose%
\pgfusepath{stroke}%
\end{pgfscope}%
\begin{pgfscope}%
\pgfpathrectangle{\pgfqpoint{7.512535in}{0.437222in}}{\pgfqpoint{6.275590in}{5.159444in}}%
\pgfusepath{clip}%
\pgfsetbuttcap%
\pgfsetroundjoin%
\pgfsetlinewidth{1.003750pt}%
\definecolor{currentstroke}{rgb}{0.827451,0.827451,0.827451}%
\pgfsetstrokecolor{currentstroke}%
\pgfsetstrokeopacity{0.800000}%
\pgfsetdash{}{0pt}%
\pgfpathmoveto{\pgfqpoint{12.196637in}{5.403515in}}%
\pgfpathcurveto{\pgfqpoint{12.207687in}{5.403515in}}{\pgfqpoint{12.218287in}{5.407905in}}{\pgfqpoint{12.226100in}{5.415719in}}%
\pgfpathcurveto{\pgfqpoint{12.233914in}{5.423532in}}{\pgfqpoint{12.238304in}{5.434131in}}{\pgfqpoint{12.238304in}{5.445182in}}%
\pgfpathcurveto{\pgfqpoint{12.238304in}{5.456232in}}{\pgfqpoint{12.233914in}{5.466831in}}{\pgfqpoint{12.226100in}{5.474644in}}%
\pgfpathcurveto{\pgfqpoint{12.218287in}{5.482458in}}{\pgfqpoint{12.207687in}{5.486848in}}{\pgfqpoint{12.196637in}{5.486848in}}%
\pgfpathcurveto{\pgfqpoint{12.185587in}{5.486848in}}{\pgfqpoint{12.174988in}{5.482458in}}{\pgfqpoint{12.167175in}{5.474644in}}%
\pgfpathcurveto{\pgfqpoint{12.159361in}{5.466831in}}{\pgfqpoint{12.154971in}{5.456232in}}{\pgfqpoint{12.154971in}{5.445182in}}%
\pgfpathcurveto{\pgfqpoint{12.154971in}{5.434131in}}{\pgfqpoint{12.159361in}{5.423532in}}{\pgfqpoint{12.167175in}{5.415719in}}%
\pgfpathcurveto{\pgfqpoint{12.174988in}{5.407905in}}{\pgfqpoint{12.185587in}{5.403515in}}{\pgfqpoint{12.196637in}{5.403515in}}%
\pgfpathlineto{\pgfqpoint{12.196637in}{5.403515in}}%
\pgfpathclose%
\pgfusepath{stroke}%
\end{pgfscope}%
\begin{pgfscope}%
\pgfpathrectangle{\pgfqpoint{7.512535in}{0.437222in}}{\pgfqpoint{6.275590in}{5.159444in}}%
\pgfusepath{clip}%
\pgfsetbuttcap%
\pgfsetroundjoin%
\pgfsetlinewidth{1.003750pt}%
\definecolor{currentstroke}{rgb}{0.827451,0.827451,0.827451}%
\pgfsetstrokecolor{currentstroke}%
\pgfsetstrokeopacity{0.800000}%
\pgfsetdash{}{0pt}%
\pgfpathmoveto{\pgfqpoint{12.209629in}{5.405098in}}%
\pgfpathcurveto{\pgfqpoint{12.220679in}{5.405098in}}{\pgfqpoint{12.231278in}{5.409488in}}{\pgfqpoint{12.239091in}{5.417302in}}%
\pgfpathcurveto{\pgfqpoint{12.246905in}{5.425116in}}{\pgfqpoint{12.251295in}{5.435715in}}{\pgfqpoint{12.251295in}{5.446765in}}%
\pgfpathcurveto{\pgfqpoint{12.251295in}{5.457815in}}{\pgfqpoint{12.246905in}{5.468414in}}{\pgfqpoint{12.239091in}{5.476228in}}%
\pgfpathcurveto{\pgfqpoint{12.231278in}{5.484041in}}{\pgfqpoint{12.220679in}{5.488431in}}{\pgfqpoint{12.209629in}{5.488431in}}%
\pgfpathcurveto{\pgfqpoint{12.198578in}{5.488431in}}{\pgfqpoint{12.187979in}{5.484041in}}{\pgfqpoint{12.180166in}{5.476228in}}%
\pgfpathcurveto{\pgfqpoint{12.172352in}{5.468414in}}{\pgfqpoint{12.167962in}{5.457815in}}{\pgfqpoint{12.167962in}{5.446765in}}%
\pgfpathcurveto{\pgfqpoint{12.167962in}{5.435715in}}{\pgfqpoint{12.172352in}{5.425116in}}{\pgfqpoint{12.180166in}{5.417302in}}%
\pgfpathcurveto{\pgfqpoint{12.187979in}{5.409488in}}{\pgfqpoint{12.198578in}{5.405098in}}{\pgfqpoint{12.209629in}{5.405098in}}%
\pgfpathlineto{\pgfqpoint{12.209629in}{5.405098in}}%
\pgfpathclose%
\pgfusepath{stroke}%
\end{pgfscope}%
\begin{pgfscope}%
\pgfpathrectangle{\pgfqpoint{7.512535in}{0.437222in}}{\pgfqpoint{6.275590in}{5.159444in}}%
\pgfusepath{clip}%
\pgfsetbuttcap%
\pgfsetroundjoin%
\pgfsetlinewidth{1.003750pt}%
\definecolor{currentstroke}{rgb}{0.827451,0.827451,0.827451}%
\pgfsetstrokecolor{currentstroke}%
\pgfsetstrokeopacity{0.800000}%
\pgfsetdash{}{0pt}%
\pgfpathmoveto{\pgfqpoint{8.775643in}{1.745625in}}%
\pgfpathcurveto{\pgfqpoint{8.786693in}{1.745625in}}{\pgfqpoint{8.797292in}{1.750015in}}{\pgfqpoint{8.805106in}{1.757829in}}%
\pgfpathcurveto{\pgfqpoint{8.812919in}{1.765643in}}{\pgfqpoint{8.817310in}{1.776242in}}{\pgfqpoint{8.817310in}{1.787292in}}%
\pgfpathcurveto{\pgfqpoint{8.817310in}{1.798342in}}{\pgfqpoint{8.812919in}{1.808941in}}{\pgfqpoint{8.805106in}{1.816755in}}%
\pgfpathcurveto{\pgfqpoint{8.797292in}{1.824568in}}{\pgfqpoint{8.786693in}{1.828959in}}{\pgfqpoint{8.775643in}{1.828959in}}%
\pgfpathcurveto{\pgfqpoint{8.764593in}{1.828959in}}{\pgfqpoint{8.753994in}{1.824568in}}{\pgfqpoint{8.746180in}{1.816755in}}%
\pgfpathcurveto{\pgfqpoint{8.738367in}{1.808941in}}{\pgfqpoint{8.733976in}{1.798342in}}{\pgfqpoint{8.733976in}{1.787292in}}%
\pgfpathcurveto{\pgfqpoint{8.733976in}{1.776242in}}{\pgfqpoint{8.738367in}{1.765643in}}{\pgfqpoint{8.746180in}{1.757829in}}%
\pgfpathcurveto{\pgfqpoint{8.753994in}{1.750015in}}{\pgfqpoint{8.764593in}{1.745625in}}{\pgfqpoint{8.775643in}{1.745625in}}%
\pgfpathlineto{\pgfqpoint{8.775643in}{1.745625in}}%
\pgfpathclose%
\pgfusepath{stroke}%
\end{pgfscope}%
\begin{pgfscope}%
\pgfpathrectangle{\pgfqpoint{7.512535in}{0.437222in}}{\pgfqpoint{6.275590in}{5.159444in}}%
\pgfusepath{clip}%
\pgfsetbuttcap%
\pgfsetroundjoin%
\pgfsetlinewidth{1.003750pt}%
\definecolor{currentstroke}{rgb}{0.827451,0.827451,0.827451}%
\pgfsetstrokecolor{currentstroke}%
\pgfsetstrokeopacity{0.800000}%
\pgfsetdash{}{0pt}%
\pgfpathmoveto{\pgfqpoint{10.127909in}{3.160252in}}%
\pgfpathcurveto{\pgfqpoint{10.138959in}{3.160252in}}{\pgfqpoint{10.149558in}{3.164643in}}{\pgfqpoint{10.157372in}{3.172456in}}%
\pgfpathcurveto{\pgfqpoint{10.165186in}{3.180270in}}{\pgfqpoint{10.169576in}{3.190869in}}{\pgfqpoint{10.169576in}{3.201919in}}%
\pgfpathcurveto{\pgfqpoint{10.169576in}{3.212969in}}{\pgfqpoint{10.165186in}{3.223568in}}{\pgfqpoint{10.157372in}{3.231382in}}%
\pgfpathcurveto{\pgfqpoint{10.149558in}{3.239196in}}{\pgfqpoint{10.138959in}{3.243586in}}{\pgfqpoint{10.127909in}{3.243586in}}%
\pgfpathcurveto{\pgfqpoint{10.116859in}{3.243586in}}{\pgfqpoint{10.106260in}{3.239196in}}{\pgfqpoint{10.098446in}{3.231382in}}%
\pgfpathcurveto{\pgfqpoint{10.090633in}{3.223568in}}{\pgfqpoint{10.086243in}{3.212969in}}{\pgfqpoint{10.086243in}{3.201919in}}%
\pgfpathcurveto{\pgfqpoint{10.086243in}{3.190869in}}{\pgfqpoint{10.090633in}{3.180270in}}{\pgfqpoint{10.098446in}{3.172456in}}%
\pgfpathcurveto{\pgfqpoint{10.106260in}{3.164643in}}{\pgfqpoint{10.116859in}{3.160252in}}{\pgfqpoint{10.127909in}{3.160252in}}%
\pgfpathlineto{\pgfqpoint{10.127909in}{3.160252in}}%
\pgfpathclose%
\pgfusepath{stroke}%
\end{pgfscope}%
\begin{pgfscope}%
\pgfpathrectangle{\pgfqpoint{7.512535in}{0.437222in}}{\pgfqpoint{6.275590in}{5.159444in}}%
\pgfusepath{clip}%
\pgfsetbuttcap%
\pgfsetroundjoin%
\pgfsetlinewidth{1.003750pt}%
\definecolor{currentstroke}{rgb}{0.827451,0.827451,0.827451}%
\pgfsetstrokecolor{currentstroke}%
\pgfsetstrokeopacity{0.800000}%
\pgfsetdash{}{0pt}%
\pgfpathmoveto{\pgfqpoint{11.026069in}{5.528063in}}%
\pgfpathcurveto{\pgfqpoint{11.037120in}{5.528063in}}{\pgfqpoint{11.047719in}{5.532453in}}{\pgfqpoint{11.055532in}{5.540267in}}%
\pgfpathcurveto{\pgfqpoint{11.063346in}{5.548081in}}{\pgfqpoint{11.067736in}{5.558680in}}{\pgfqpoint{11.067736in}{5.569730in}}%
\pgfpathcurveto{\pgfqpoint{11.067736in}{5.580780in}}{\pgfqpoint{11.063346in}{5.591379in}}{\pgfqpoint{11.055532in}{5.599193in}}%
\pgfpathcurveto{\pgfqpoint{11.047719in}{5.607006in}}{\pgfqpoint{11.037120in}{5.611396in}}{\pgfqpoint{11.026069in}{5.611396in}}%
\pgfpathcurveto{\pgfqpoint{11.015019in}{5.611396in}}{\pgfqpoint{11.004420in}{5.607006in}}{\pgfqpoint{10.996607in}{5.599193in}}%
\pgfpathcurveto{\pgfqpoint{10.988793in}{5.591379in}}{\pgfqpoint{10.984403in}{5.580780in}}{\pgfqpoint{10.984403in}{5.569730in}}%
\pgfpathcurveto{\pgfqpoint{10.984403in}{5.558680in}}{\pgfqpoint{10.988793in}{5.548081in}}{\pgfqpoint{10.996607in}{5.540267in}}%
\pgfpathcurveto{\pgfqpoint{11.004420in}{5.532453in}}{\pgfqpoint{11.015019in}{5.528063in}}{\pgfqpoint{11.026069in}{5.528063in}}%
\pgfpathlineto{\pgfqpoint{11.026069in}{5.528063in}}%
\pgfpathclose%
\pgfusepath{stroke}%
\end{pgfscope}%
\begin{pgfscope}%
\pgfpathrectangle{\pgfqpoint{7.512535in}{0.437222in}}{\pgfqpoint{6.275590in}{5.159444in}}%
\pgfusepath{clip}%
\pgfsetbuttcap%
\pgfsetroundjoin%
\pgfsetlinewidth{1.003750pt}%
\definecolor{currentstroke}{rgb}{0.827451,0.827451,0.827451}%
\pgfsetstrokecolor{currentstroke}%
\pgfsetstrokeopacity{0.800000}%
\pgfsetdash{}{0pt}%
\pgfpathmoveto{\pgfqpoint{12.450009in}{5.541802in}}%
\pgfpathcurveto{\pgfqpoint{12.461060in}{5.541802in}}{\pgfqpoint{12.471659in}{5.546192in}}{\pgfqpoint{12.479472in}{5.554006in}}%
\pgfpathcurveto{\pgfqpoint{12.487286in}{5.561819in}}{\pgfqpoint{12.491676in}{5.572418in}}{\pgfqpoint{12.491676in}{5.583468in}}%
\pgfpathcurveto{\pgfqpoint{12.491676in}{5.594519in}}{\pgfqpoint{12.487286in}{5.605118in}}{\pgfqpoint{12.479472in}{5.612931in}}%
\pgfpathcurveto{\pgfqpoint{12.471659in}{5.620745in}}{\pgfqpoint{12.461060in}{5.625135in}}{\pgfqpoint{12.450009in}{5.625135in}}%
\pgfpathcurveto{\pgfqpoint{12.438959in}{5.625135in}}{\pgfqpoint{12.428360in}{5.620745in}}{\pgfqpoint{12.420547in}{5.612931in}}%
\pgfpathcurveto{\pgfqpoint{12.412733in}{5.605118in}}{\pgfqpoint{12.408343in}{5.594519in}}{\pgfqpoint{12.408343in}{5.583468in}}%
\pgfpathcurveto{\pgfqpoint{12.408343in}{5.572418in}}{\pgfqpoint{12.412733in}{5.561819in}}{\pgfqpoint{12.420547in}{5.554006in}}%
\pgfpathcurveto{\pgfqpoint{12.428360in}{5.546192in}}{\pgfqpoint{12.438959in}{5.541802in}}{\pgfqpoint{12.450009in}{5.541802in}}%
\pgfpathlineto{\pgfqpoint{12.450009in}{5.541802in}}%
\pgfpathclose%
\pgfusepath{stroke}%
\end{pgfscope}%
\begin{pgfscope}%
\pgfpathrectangle{\pgfqpoint{7.512535in}{0.437222in}}{\pgfqpoint{6.275590in}{5.159444in}}%
\pgfusepath{clip}%
\pgfsetbuttcap%
\pgfsetroundjoin%
\pgfsetlinewidth{1.003750pt}%
\definecolor{currentstroke}{rgb}{0.827451,0.827451,0.827451}%
\pgfsetstrokecolor{currentstroke}%
\pgfsetstrokeopacity{0.800000}%
\pgfsetdash{}{0pt}%
\pgfpathmoveto{\pgfqpoint{8.519449in}{1.061557in}}%
\pgfpathcurveto{\pgfqpoint{8.530499in}{1.061557in}}{\pgfqpoint{8.541099in}{1.065948in}}{\pgfqpoint{8.548912in}{1.073761in}}%
\pgfpathcurveto{\pgfqpoint{8.556726in}{1.081575in}}{\pgfqpoint{8.561116in}{1.092174in}}{\pgfqpoint{8.561116in}{1.103224in}}%
\pgfpathcurveto{\pgfqpoint{8.561116in}{1.114274in}}{\pgfqpoint{8.556726in}{1.124873in}}{\pgfqpoint{8.548912in}{1.132687in}}%
\pgfpathcurveto{\pgfqpoint{8.541099in}{1.140501in}}{\pgfqpoint{8.530499in}{1.144891in}}{\pgfqpoint{8.519449in}{1.144891in}}%
\pgfpathcurveto{\pgfqpoint{8.508399in}{1.144891in}}{\pgfqpoint{8.497800in}{1.140501in}}{\pgfqpoint{8.489987in}{1.132687in}}%
\pgfpathcurveto{\pgfqpoint{8.482173in}{1.124873in}}{\pgfqpoint{8.477783in}{1.114274in}}{\pgfqpoint{8.477783in}{1.103224in}}%
\pgfpathcurveto{\pgfqpoint{8.477783in}{1.092174in}}{\pgfqpoint{8.482173in}{1.081575in}}{\pgfqpoint{8.489987in}{1.073761in}}%
\pgfpathcurveto{\pgfqpoint{8.497800in}{1.065948in}}{\pgfqpoint{8.508399in}{1.061557in}}{\pgfqpoint{8.519449in}{1.061557in}}%
\pgfpathlineto{\pgfqpoint{8.519449in}{1.061557in}}%
\pgfpathclose%
\pgfusepath{stroke}%
\end{pgfscope}%
\begin{pgfscope}%
\pgfpathrectangle{\pgfqpoint{7.512535in}{0.437222in}}{\pgfqpoint{6.275590in}{5.159444in}}%
\pgfusepath{clip}%
\pgfsetbuttcap%
\pgfsetroundjoin%
\pgfsetlinewidth{1.003750pt}%
\definecolor{currentstroke}{rgb}{0.827451,0.827451,0.827451}%
\pgfsetstrokecolor{currentstroke}%
\pgfsetstrokeopacity{0.800000}%
\pgfsetdash{}{0pt}%
\pgfpathmoveto{\pgfqpoint{10.156793in}{3.250664in}}%
\pgfpathcurveto{\pgfqpoint{10.167843in}{3.250664in}}{\pgfqpoint{10.178442in}{3.255054in}}{\pgfqpoint{10.186256in}{3.262868in}}%
\pgfpathcurveto{\pgfqpoint{10.194069in}{3.270682in}}{\pgfqpoint{10.198460in}{3.281281in}}{\pgfqpoint{10.198460in}{3.292331in}}%
\pgfpathcurveto{\pgfqpoint{10.198460in}{3.303381in}}{\pgfqpoint{10.194069in}{3.313980in}}{\pgfqpoint{10.186256in}{3.321794in}}%
\pgfpathcurveto{\pgfqpoint{10.178442in}{3.329607in}}{\pgfqpoint{10.167843in}{3.333998in}}{\pgfqpoint{10.156793in}{3.333998in}}%
\pgfpathcurveto{\pgfqpoint{10.145743in}{3.333998in}}{\pgfqpoint{10.135144in}{3.329607in}}{\pgfqpoint{10.127330in}{3.321794in}}%
\pgfpathcurveto{\pgfqpoint{10.119517in}{3.313980in}}{\pgfqpoint{10.115126in}{3.303381in}}{\pgfqpoint{10.115126in}{3.292331in}}%
\pgfpathcurveto{\pgfqpoint{10.115126in}{3.281281in}}{\pgfqpoint{10.119517in}{3.270682in}}{\pgfqpoint{10.127330in}{3.262868in}}%
\pgfpathcurveto{\pgfqpoint{10.135144in}{3.255054in}}{\pgfqpoint{10.145743in}{3.250664in}}{\pgfqpoint{10.156793in}{3.250664in}}%
\pgfpathlineto{\pgfqpoint{10.156793in}{3.250664in}}%
\pgfpathclose%
\pgfusepath{stroke}%
\end{pgfscope}%
\begin{pgfscope}%
\pgfpathrectangle{\pgfqpoint{7.512535in}{0.437222in}}{\pgfqpoint{6.275590in}{5.159444in}}%
\pgfusepath{clip}%
\pgfsetbuttcap%
\pgfsetroundjoin%
\pgfsetlinewidth{1.003750pt}%
\definecolor{currentstroke}{rgb}{0.827451,0.827451,0.827451}%
\pgfsetstrokecolor{currentstroke}%
\pgfsetstrokeopacity{0.800000}%
\pgfsetdash{}{0pt}%
\pgfpathmoveto{\pgfqpoint{10.873384in}{4.145170in}}%
\pgfpathcurveto{\pgfqpoint{10.884434in}{4.145170in}}{\pgfqpoint{10.895033in}{4.149560in}}{\pgfqpoint{10.902847in}{4.157374in}}%
\pgfpathcurveto{\pgfqpoint{10.910660in}{4.165188in}}{\pgfqpoint{10.915050in}{4.175787in}}{\pgfqpoint{10.915050in}{4.186837in}}%
\pgfpathcurveto{\pgfqpoint{10.915050in}{4.197887in}}{\pgfqpoint{10.910660in}{4.208486in}}{\pgfqpoint{10.902847in}{4.216300in}}%
\pgfpathcurveto{\pgfqpoint{10.895033in}{4.224113in}}{\pgfqpoint{10.884434in}{4.228504in}}{\pgfqpoint{10.873384in}{4.228504in}}%
\pgfpathcurveto{\pgfqpoint{10.862334in}{4.228504in}}{\pgfqpoint{10.851735in}{4.224113in}}{\pgfqpoint{10.843921in}{4.216300in}}%
\pgfpathcurveto{\pgfqpoint{10.836107in}{4.208486in}}{\pgfqpoint{10.831717in}{4.197887in}}{\pgfqpoint{10.831717in}{4.186837in}}%
\pgfpathcurveto{\pgfqpoint{10.831717in}{4.175787in}}{\pgfqpoint{10.836107in}{4.165188in}}{\pgfqpoint{10.843921in}{4.157374in}}%
\pgfpathcurveto{\pgfqpoint{10.851735in}{4.149560in}}{\pgfqpoint{10.862334in}{4.145170in}}{\pgfqpoint{10.873384in}{4.145170in}}%
\pgfpathlineto{\pgfqpoint{10.873384in}{4.145170in}}%
\pgfpathclose%
\pgfusepath{stroke}%
\end{pgfscope}%
\begin{pgfscope}%
\pgfpathrectangle{\pgfqpoint{7.512535in}{0.437222in}}{\pgfqpoint{6.275590in}{5.159444in}}%
\pgfusepath{clip}%
\pgfsetbuttcap%
\pgfsetroundjoin%
\pgfsetlinewidth{1.003750pt}%
\definecolor{currentstroke}{rgb}{0.827451,0.827451,0.827451}%
\pgfsetstrokecolor{currentstroke}%
\pgfsetstrokeopacity{0.800000}%
\pgfsetdash{}{0pt}%
\pgfpathmoveto{\pgfqpoint{8.624162in}{1.722485in}}%
\pgfpathcurveto{\pgfqpoint{8.635212in}{1.722485in}}{\pgfqpoint{8.645811in}{1.726876in}}{\pgfqpoint{8.653624in}{1.734689in}}%
\pgfpathcurveto{\pgfqpoint{8.661438in}{1.742503in}}{\pgfqpoint{8.665828in}{1.753102in}}{\pgfqpoint{8.665828in}{1.764152in}}%
\pgfpathcurveto{\pgfqpoint{8.665828in}{1.775202in}}{\pgfqpoint{8.661438in}{1.785801in}}{\pgfqpoint{8.653624in}{1.793615in}}%
\pgfpathcurveto{\pgfqpoint{8.645811in}{1.801428in}}{\pgfqpoint{8.635212in}{1.805819in}}{\pgfqpoint{8.624162in}{1.805819in}}%
\pgfpathcurveto{\pgfqpoint{8.613112in}{1.805819in}}{\pgfqpoint{8.602513in}{1.801428in}}{\pgfqpoint{8.594699in}{1.793615in}}%
\pgfpathcurveto{\pgfqpoint{8.586885in}{1.785801in}}{\pgfqpoint{8.582495in}{1.775202in}}{\pgfqpoint{8.582495in}{1.764152in}}%
\pgfpathcurveto{\pgfqpoint{8.582495in}{1.753102in}}{\pgfqpoint{8.586885in}{1.742503in}}{\pgfqpoint{8.594699in}{1.734689in}}%
\pgfpathcurveto{\pgfqpoint{8.602513in}{1.726876in}}{\pgfqpoint{8.613112in}{1.722485in}}{\pgfqpoint{8.624162in}{1.722485in}}%
\pgfpathlineto{\pgfqpoint{8.624162in}{1.722485in}}%
\pgfpathclose%
\pgfusepath{stroke}%
\end{pgfscope}%
\begin{pgfscope}%
\pgfpathrectangle{\pgfqpoint{7.512535in}{0.437222in}}{\pgfqpoint{6.275590in}{5.159444in}}%
\pgfusepath{clip}%
\pgfsetbuttcap%
\pgfsetroundjoin%
\pgfsetlinewidth{1.003750pt}%
\definecolor{currentstroke}{rgb}{0.827451,0.827451,0.827451}%
\pgfsetstrokecolor{currentstroke}%
\pgfsetstrokeopacity{0.800000}%
\pgfsetdash{}{0pt}%
\pgfpathmoveto{\pgfqpoint{10.299622in}{3.123555in}}%
\pgfpathcurveto{\pgfqpoint{10.310672in}{3.123555in}}{\pgfqpoint{10.321271in}{3.127945in}}{\pgfqpoint{10.329085in}{3.135759in}}%
\pgfpathcurveto{\pgfqpoint{10.336898in}{3.143572in}}{\pgfqpoint{10.341289in}{3.154171in}}{\pgfqpoint{10.341289in}{3.165222in}}%
\pgfpathcurveto{\pgfqpoint{10.341289in}{3.176272in}}{\pgfqpoint{10.336898in}{3.186871in}}{\pgfqpoint{10.329085in}{3.194684in}}%
\pgfpathcurveto{\pgfqpoint{10.321271in}{3.202498in}}{\pgfqpoint{10.310672in}{3.206888in}}{\pgfqpoint{10.299622in}{3.206888in}}%
\pgfpathcurveto{\pgfqpoint{10.288572in}{3.206888in}}{\pgfqpoint{10.277973in}{3.202498in}}{\pgfqpoint{10.270159in}{3.194684in}}%
\pgfpathcurveto{\pgfqpoint{10.262346in}{3.186871in}}{\pgfqpoint{10.257955in}{3.176272in}}{\pgfqpoint{10.257955in}{3.165222in}}%
\pgfpathcurveto{\pgfqpoint{10.257955in}{3.154171in}}{\pgfqpoint{10.262346in}{3.143572in}}{\pgfqpoint{10.270159in}{3.135759in}}%
\pgfpathcurveto{\pgfqpoint{10.277973in}{3.127945in}}{\pgfqpoint{10.288572in}{3.123555in}}{\pgfqpoint{10.299622in}{3.123555in}}%
\pgfpathlineto{\pgfqpoint{10.299622in}{3.123555in}}%
\pgfpathclose%
\pgfusepath{stroke}%
\end{pgfscope}%
\begin{pgfscope}%
\pgfpathrectangle{\pgfqpoint{7.512535in}{0.437222in}}{\pgfqpoint{6.275590in}{5.159444in}}%
\pgfusepath{clip}%
\pgfsetbuttcap%
\pgfsetroundjoin%
\pgfsetlinewidth{1.003750pt}%
\definecolor{currentstroke}{rgb}{0.827451,0.827451,0.827451}%
\pgfsetstrokecolor{currentstroke}%
\pgfsetstrokeopacity{0.800000}%
\pgfsetdash{}{0pt}%
\pgfpathmoveto{\pgfqpoint{7.983939in}{1.000624in}}%
\pgfpathcurveto{\pgfqpoint{7.994989in}{1.000624in}}{\pgfqpoint{8.005588in}{1.005015in}}{\pgfqpoint{8.013401in}{1.012828in}}%
\pgfpathcurveto{\pgfqpoint{8.021215in}{1.020642in}}{\pgfqpoint{8.025605in}{1.031241in}}{\pgfqpoint{8.025605in}{1.042291in}}%
\pgfpathcurveto{\pgfqpoint{8.025605in}{1.053341in}}{\pgfqpoint{8.021215in}{1.063940in}}{\pgfqpoint{8.013401in}{1.071754in}}%
\pgfpathcurveto{\pgfqpoint{8.005588in}{1.079568in}}{\pgfqpoint{7.994989in}{1.083958in}}{\pgfqpoint{7.983939in}{1.083958in}}%
\pgfpathcurveto{\pgfqpoint{7.972888in}{1.083958in}}{\pgfqpoint{7.962289in}{1.079568in}}{\pgfqpoint{7.954476in}{1.071754in}}%
\pgfpathcurveto{\pgfqpoint{7.946662in}{1.063940in}}{\pgfqpoint{7.942272in}{1.053341in}}{\pgfqpoint{7.942272in}{1.042291in}}%
\pgfpathcurveto{\pgfqpoint{7.942272in}{1.031241in}}{\pgfqpoint{7.946662in}{1.020642in}}{\pgfqpoint{7.954476in}{1.012828in}}%
\pgfpathcurveto{\pgfqpoint{7.962289in}{1.005015in}}{\pgfqpoint{7.972888in}{1.000624in}}{\pgfqpoint{7.983939in}{1.000624in}}%
\pgfpathlineto{\pgfqpoint{7.983939in}{1.000624in}}%
\pgfpathclose%
\pgfusepath{stroke}%
\end{pgfscope}%
\begin{pgfscope}%
\pgfpathrectangle{\pgfqpoint{7.512535in}{0.437222in}}{\pgfqpoint{6.275590in}{5.159444in}}%
\pgfusepath{clip}%
\pgfsetbuttcap%
\pgfsetroundjoin%
\pgfsetlinewidth{1.003750pt}%
\definecolor{currentstroke}{rgb}{0.827451,0.827451,0.827451}%
\pgfsetstrokecolor{currentstroke}%
\pgfsetstrokeopacity{0.800000}%
\pgfsetdash{}{0pt}%
\pgfpathmoveto{\pgfqpoint{9.060406in}{3.385314in}}%
\pgfpathcurveto{\pgfqpoint{9.071456in}{3.385314in}}{\pgfqpoint{9.082055in}{3.389704in}}{\pgfqpoint{9.089869in}{3.397518in}}%
\pgfpathcurveto{\pgfqpoint{9.097682in}{3.405332in}}{\pgfqpoint{9.102073in}{3.415931in}}{\pgfqpoint{9.102073in}{3.426981in}}%
\pgfpathcurveto{\pgfqpoint{9.102073in}{3.438031in}}{\pgfqpoint{9.097682in}{3.448630in}}{\pgfqpoint{9.089869in}{3.456444in}}%
\pgfpathcurveto{\pgfqpoint{9.082055in}{3.464257in}}{\pgfqpoint{9.071456in}{3.468647in}}{\pgfqpoint{9.060406in}{3.468647in}}%
\pgfpathcurveto{\pgfqpoint{9.049356in}{3.468647in}}{\pgfqpoint{9.038757in}{3.464257in}}{\pgfqpoint{9.030943in}{3.456444in}}%
\pgfpathcurveto{\pgfqpoint{9.023129in}{3.448630in}}{\pgfqpoint{9.018739in}{3.438031in}}{\pgfqpoint{9.018739in}{3.426981in}}%
\pgfpathcurveto{\pgfqpoint{9.018739in}{3.415931in}}{\pgfqpoint{9.023129in}{3.405332in}}{\pgfqpoint{9.030943in}{3.397518in}}%
\pgfpathcurveto{\pgfqpoint{9.038757in}{3.389704in}}{\pgfqpoint{9.049356in}{3.385314in}}{\pgfqpoint{9.060406in}{3.385314in}}%
\pgfpathlineto{\pgfqpoint{9.060406in}{3.385314in}}%
\pgfpathclose%
\pgfusepath{stroke}%
\end{pgfscope}%
\begin{pgfscope}%
\pgfpathrectangle{\pgfqpoint{7.512535in}{0.437222in}}{\pgfqpoint{6.275590in}{5.159444in}}%
\pgfusepath{clip}%
\pgfsetbuttcap%
\pgfsetroundjoin%
\pgfsetlinewidth{1.003750pt}%
\definecolor{currentstroke}{rgb}{0.827451,0.827451,0.827451}%
\pgfsetstrokecolor{currentstroke}%
\pgfsetstrokeopacity{0.800000}%
\pgfsetdash{}{0pt}%
\pgfpathmoveto{\pgfqpoint{9.540773in}{3.442114in}}%
\pgfpathcurveto{\pgfqpoint{9.551823in}{3.442114in}}{\pgfqpoint{9.562422in}{3.446504in}}{\pgfqpoint{9.570236in}{3.454318in}}%
\pgfpathcurveto{\pgfqpoint{9.578049in}{3.462131in}}{\pgfqpoint{9.582440in}{3.472730in}}{\pgfqpoint{9.582440in}{3.483780in}}%
\pgfpathcurveto{\pgfqpoint{9.582440in}{3.494830in}}{\pgfqpoint{9.578049in}{3.505429in}}{\pgfqpoint{9.570236in}{3.513243in}}%
\pgfpathcurveto{\pgfqpoint{9.562422in}{3.521057in}}{\pgfqpoint{9.551823in}{3.525447in}}{\pgfqpoint{9.540773in}{3.525447in}}%
\pgfpathcurveto{\pgfqpoint{9.529723in}{3.525447in}}{\pgfqpoint{9.519124in}{3.521057in}}{\pgfqpoint{9.511310in}{3.513243in}}%
\pgfpathcurveto{\pgfqpoint{9.503497in}{3.505429in}}{\pgfqpoint{9.499106in}{3.494830in}}{\pgfqpoint{9.499106in}{3.483780in}}%
\pgfpathcurveto{\pgfqpoint{9.499106in}{3.472730in}}{\pgfqpoint{9.503497in}{3.462131in}}{\pgfqpoint{9.511310in}{3.454318in}}%
\pgfpathcurveto{\pgfqpoint{9.519124in}{3.446504in}}{\pgfqpoint{9.529723in}{3.442114in}}{\pgfqpoint{9.540773in}{3.442114in}}%
\pgfpathlineto{\pgfqpoint{9.540773in}{3.442114in}}%
\pgfpathclose%
\pgfusepath{stroke}%
\end{pgfscope}%
\begin{pgfscope}%
\pgfpathrectangle{\pgfqpoint{7.512535in}{0.437222in}}{\pgfqpoint{6.275590in}{5.159444in}}%
\pgfusepath{clip}%
\pgfsetbuttcap%
\pgfsetroundjoin%
\pgfsetlinewidth{1.003750pt}%
\definecolor{currentstroke}{rgb}{0.827451,0.827451,0.827451}%
\pgfsetstrokecolor{currentstroke}%
\pgfsetstrokeopacity{0.800000}%
\pgfsetdash{}{0pt}%
\pgfpathmoveto{\pgfqpoint{11.040791in}{5.348363in}}%
\pgfpathcurveto{\pgfqpoint{11.051841in}{5.348363in}}{\pgfqpoint{11.062440in}{5.352753in}}{\pgfqpoint{11.070254in}{5.360567in}}%
\pgfpathcurveto{\pgfqpoint{11.078068in}{5.368381in}}{\pgfqpoint{11.082458in}{5.378980in}}{\pgfqpoint{11.082458in}{5.390030in}}%
\pgfpathcurveto{\pgfqpoint{11.082458in}{5.401080in}}{\pgfqpoint{11.078068in}{5.411679in}}{\pgfqpoint{11.070254in}{5.419493in}}%
\pgfpathcurveto{\pgfqpoint{11.062440in}{5.427306in}}{\pgfqpoint{11.051841in}{5.431697in}}{\pgfqpoint{11.040791in}{5.431697in}}%
\pgfpathcurveto{\pgfqpoint{11.029741in}{5.431697in}}{\pgfqpoint{11.019142in}{5.427306in}}{\pgfqpoint{11.011328in}{5.419493in}}%
\pgfpathcurveto{\pgfqpoint{11.003515in}{5.411679in}}{\pgfqpoint{10.999125in}{5.401080in}}{\pgfqpoint{10.999125in}{5.390030in}}%
\pgfpathcurveto{\pgfqpoint{10.999125in}{5.378980in}}{\pgfqpoint{11.003515in}{5.368381in}}{\pgfqpoint{11.011328in}{5.360567in}}%
\pgfpathcurveto{\pgfqpoint{11.019142in}{5.352753in}}{\pgfqpoint{11.029741in}{5.348363in}}{\pgfqpoint{11.040791in}{5.348363in}}%
\pgfpathlineto{\pgfqpoint{11.040791in}{5.348363in}}%
\pgfpathclose%
\pgfusepath{stroke}%
\end{pgfscope}%
\begin{pgfscope}%
\pgfpathrectangle{\pgfqpoint{7.512535in}{0.437222in}}{\pgfqpoint{6.275590in}{5.159444in}}%
\pgfusepath{clip}%
\pgfsetbuttcap%
\pgfsetroundjoin%
\pgfsetlinewidth{1.003750pt}%
\definecolor{currentstroke}{rgb}{0.827451,0.827451,0.827451}%
\pgfsetstrokecolor{currentstroke}%
\pgfsetstrokeopacity{0.800000}%
\pgfsetdash{}{0pt}%
\pgfpathmoveto{\pgfqpoint{8.800029in}{1.838807in}}%
\pgfpathcurveto{\pgfqpoint{8.811079in}{1.838807in}}{\pgfqpoint{8.821678in}{1.843198in}}{\pgfqpoint{8.829491in}{1.851011in}}%
\pgfpathcurveto{\pgfqpoint{8.837305in}{1.858825in}}{\pgfqpoint{8.841695in}{1.869424in}}{\pgfqpoint{8.841695in}{1.880474in}}%
\pgfpathcurveto{\pgfqpoint{8.841695in}{1.891524in}}{\pgfqpoint{8.837305in}{1.902123in}}{\pgfqpoint{8.829491in}{1.909937in}}%
\pgfpathcurveto{\pgfqpoint{8.821678in}{1.917750in}}{\pgfqpoint{8.811079in}{1.922141in}}{\pgfqpoint{8.800029in}{1.922141in}}%
\pgfpathcurveto{\pgfqpoint{8.788978in}{1.922141in}}{\pgfqpoint{8.778379in}{1.917750in}}{\pgfqpoint{8.770566in}{1.909937in}}%
\pgfpathcurveto{\pgfqpoint{8.762752in}{1.902123in}}{\pgfqpoint{8.758362in}{1.891524in}}{\pgfqpoint{8.758362in}{1.880474in}}%
\pgfpathcurveto{\pgfqpoint{8.758362in}{1.869424in}}{\pgfqpoint{8.762752in}{1.858825in}}{\pgfqpoint{8.770566in}{1.851011in}}%
\pgfpathcurveto{\pgfqpoint{8.778379in}{1.843198in}}{\pgfqpoint{8.788978in}{1.838807in}}{\pgfqpoint{8.800029in}{1.838807in}}%
\pgfpathlineto{\pgfqpoint{8.800029in}{1.838807in}}%
\pgfpathclose%
\pgfusepath{stroke}%
\end{pgfscope}%
\begin{pgfscope}%
\pgfpathrectangle{\pgfqpoint{7.512535in}{0.437222in}}{\pgfqpoint{6.275590in}{5.159444in}}%
\pgfusepath{clip}%
\pgfsetbuttcap%
\pgfsetroundjoin%
\pgfsetlinewidth{1.003750pt}%
\definecolor{currentstroke}{rgb}{0.827451,0.827451,0.827451}%
\pgfsetstrokecolor{currentstroke}%
\pgfsetstrokeopacity{0.800000}%
\pgfsetdash{}{0pt}%
\pgfpathmoveto{\pgfqpoint{8.619483in}{1.654089in}}%
\pgfpathcurveto{\pgfqpoint{8.630533in}{1.654089in}}{\pgfqpoint{8.641132in}{1.658480in}}{\pgfqpoint{8.648946in}{1.666293in}}%
\pgfpathcurveto{\pgfqpoint{8.656759in}{1.674107in}}{\pgfqpoint{8.661149in}{1.684706in}}{\pgfqpoint{8.661149in}{1.695756in}}%
\pgfpathcurveto{\pgfqpoint{8.661149in}{1.706806in}}{\pgfqpoint{8.656759in}{1.717405in}}{\pgfqpoint{8.648946in}{1.725219in}}%
\pgfpathcurveto{\pgfqpoint{8.641132in}{1.733033in}}{\pgfqpoint{8.630533in}{1.737423in}}{\pgfqpoint{8.619483in}{1.737423in}}%
\pgfpathcurveto{\pgfqpoint{8.608433in}{1.737423in}}{\pgfqpoint{8.597834in}{1.733033in}}{\pgfqpoint{8.590020in}{1.725219in}}%
\pgfpathcurveto{\pgfqpoint{8.582206in}{1.717405in}}{\pgfqpoint{8.577816in}{1.706806in}}{\pgfqpoint{8.577816in}{1.695756in}}%
\pgfpathcurveto{\pgfqpoint{8.577816in}{1.684706in}}{\pgfqpoint{8.582206in}{1.674107in}}{\pgfqpoint{8.590020in}{1.666293in}}%
\pgfpathcurveto{\pgfqpoint{8.597834in}{1.658480in}}{\pgfqpoint{8.608433in}{1.654089in}}{\pgfqpoint{8.619483in}{1.654089in}}%
\pgfpathlineto{\pgfqpoint{8.619483in}{1.654089in}}%
\pgfpathclose%
\pgfusepath{stroke}%
\end{pgfscope}%
\begin{pgfscope}%
\pgfpathrectangle{\pgfqpoint{7.512535in}{0.437222in}}{\pgfqpoint{6.275590in}{5.159444in}}%
\pgfusepath{clip}%
\pgfsetbuttcap%
\pgfsetroundjoin%
\pgfsetlinewidth{1.003750pt}%
\definecolor{currentstroke}{rgb}{0.827451,0.827451,0.827451}%
\pgfsetstrokecolor{currentstroke}%
\pgfsetstrokeopacity{0.800000}%
\pgfsetdash{}{0pt}%
\pgfpathmoveto{\pgfqpoint{13.165622in}{5.501584in}}%
\pgfpathcurveto{\pgfqpoint{13.176672in}{5.501584in}}{\pgfqpoint{13.187271in}{5.505974in}}{\pgfqpoint{13.195085in}{5.513787in}}%
\pgfpathcurveto{\pgfqpoint{13.202899in}{5.521601in}}{\pgfqpoint{13.207289in}{5.532200in}}{\pgfqpoint{13.207289in}{5.543250in}}%
\pgfpathcurveto{\pgfqpoint{13.207289in}{5.554300in}}{\pgfqpoint{13.202899in}{5.564899in}}{\pgfqpoint{13.195085in}{5.572713in}}%
\pgfpathcurveto{\pgfqpoint{13.187271in}{5.580527in}}{\pgfqpoint{13.176672in}{5.584917in}}{\pgfqpoint{13.165622in}{5.584917in}}%
\pgfpathcurveto{\pgfqpoint{13.154572in}{5.584917in}}{\pgfqpoint{13.143973in}{5.580527in}}{\pgfqpoint{13.136159in}{5.572713in}}%
\pgfpathcurveto{\pgfqpoint{13.128346in}{5.564899in}}{\pgfqpoint{13.123955in}{5.554300in}}{\pgfqpoint{13.123955in}{5.543250in}}%
\pgfpathcurveto{\pgfqpoint{13.123955in}{5.532200in}}{\pgfqpoint{13.128346in}{5.521601in}}{\pgfqpoint{13.136159in}{5.513787in}}%
\pgfpathcurveto{\pgfqpoint{13.143973in}{5.505974in}}{\pgfqpoint{13.154572in}{5.501584in}}{\pgfqpoint{13.165622in}{5.501584in}}%
\pgfpathlineto{\pgfqpoint{13.165622in}{5.501584in}}%
\pgfpathclose%
\pgfusepath{stroke}%
\end{pgfscope}%
\begin{pgfscope}%
\pgfpathrectangle{\pgfqpoint{7.512535in}{0.437222in}}{\pgfqpoint{6.275590in}{5.159444in}}%
\pgfusepath{clip}%
\pgfsetbuttcap%
\pgfsetroundjoin%
\pgfsetlinewidth{1.003750pt}%
\definecolor{currentstroke}{rgb}{0.827451,0.827451,0.827451}%
\pgfsetstrokecolor{currentstroke}%
\pgfsetstrokeopacity{0.800000}%
\pgfsetdash{}{0pt}%
\pgfpathmoveto{\pgfqpoint{9.581399in}{3.231111in}}%
\pgfpathcurveto{\pgfqpoint{9.592449in}{3.231111in}}{\pgfqpoint{9.603048in}{3.235501in}}{\pgfqpoint{9.610862in}{3.243315in}}%
\pgfpathcurveto{\pgfqpoint{9.618675in}{3.251128in}}{\pgfqpoint{9.623065in}{3.261727in}}{\pgfqpoint{9.623065in}{3.272778in}}%
\pgfpathcurveto{\pgfqpoint{9.623065in}{3.283828in}}{\pgfqpoint{9.618675in}{3.294427in}}{\pgfqpoint{9.610862in}{3.302240in}}%
\pgfpathcurveto{\pgfqpoint{9.603048in}{3.310054in}}{\pgfqpoint{9.592449in}{3.314444in}}{\pgfqpoint{9.581399in}{3.314444in}}%
\pgfpathcurveto{\pgfqpoint{9.570349in}{3.314444in}}{\pgfqpoint{9.559750in}{3.310054in}}{\pgfqpoint{9.551936in}{3.302240in}}%
\pgfpathcurveto{\pgfqpoint{9.544122in}{3.294427in}}{\pgfqpoint{9.539732in}{3.283828in}}{\pgfqpoint{9.539732in}{3.272778in}}%
\pgfpathcurveto{\pgfqpoint{9.539732in}{3.261727in}}{\pgfqpoint{9.544122in}{3.251128in}}{\pgfqpoint{9.551936in}{3.243315in}}%
\pgfpathcurveto{\pgfqpoint{9.559750in}{3.235501in}}{\pgfqpoint{9.570349in}{3.231111in}}{\pgfqpoint{9.581399in}{3.231111in}}%
\pgfpathlineto{\pgfqpoint{9.581399in}{3.231111in}}%
\pgfpathclose%
\pgfusepath{stroke}%
\end{pgfscope}%
\begin{pgfscope}%
\pgfpathrectangle{\pgfqpoint{7.512535in}{0.437222in}}{\pgfqpoint{6.275590in}{5.159444in}}%
\pgfusepath{clip}%
\pgfsetbuttcap%
\pgfsetroundjoin%
\pgfsetlinewidth{1.003750pt}%
\definecolor{currentstroke}{rgb}{0.827451,0.827451,0.827451}%
\pgfsetstrokecolor{currentstroke}%
\pgfsetstrokeopacity{0.800000}%
\pgfsetdash{}{0pt}%
\pgfpathmoveto{\pgfqpoint{10.818589in}{4.898946in}}%
\pgfpathcurveto{\pgfqpoint{10.829639in}{4.898946in}}{\pgfqpoint{10.840238in}{4.903336in}}{\pgfqpoint{10.848052in}{4.911150in}}%
\pgfpathcurveto{\pgfqpoint{10.855865in}{4.918964in}}{\pgfqpoint{10.860255in}{4.929563in}}{\pgfqpoint{10.860255in}{4.940613in}}%
\pgfpathcurveto{\pgfqpoint{10.860255in}{4.951663in}}{\pgfqpoint{10.855865in}{4.962262in}}{\pgfqpoint{10.848052in}{4.970076in}}%
\pgfpathcurveto{\pgfqpoint{10.840238in}{4.977889in}}{\pgfqpoint{10.829639in}{4.982279in}}{\pgfqpoint{10.818589in}{4.982279in}}%
\pgfpathcurveto{\pgfqpoint{10.807539in}{4.982279in}}{\pgfqpoint{10.796940in}{4.977889in}}{\pgfqpoint{10.789126in}{4.970076in}}%
\pgfpathcurveto{\pgfqpoint{10.781312in}{4.962262in}}{\pgfqpoint{10.776922in}{4.951663in}}{\pgfqpoint{10.776922in}{4.940613in}}%
\pgfpathcurveto{\pgfqpoint{10.776922in}{4.929563in}}{\pgfqpoint{10.781312in}{4.918964in}}{\pgfqpoint{10.789126in}{4.911150in}}%
\pgfpathcurveto{\pgfqpoint{10.796940in}{4.903336in}}{\pgfqpoint{10.807539in}{4.898946in}}{\pgfqpoint{10.818589in}{4.898946in}}%
\pgfpathlineto{\pgfqpoint{10.818589in}{4.898946in}}%
\pgfpathclose%
\pgfusepath{stroke}%
\end{pgfscope}%
\begin{pgfscope}%
\pgfpathrectangle{\pgfqpoint{7.512535in}{0.437222in}}{\pgfqpoint{6.275590in}{5.159444in}}%
\pgfusepath{clip}%
\pgfsetbuttcap%
\pgfsetroundjoin%
\pgfsetlinewidth{1.003750pt}%
\definecolor{currentstroke}{rgb}{0.827451,0.827451,0.827451}%
\pgfsetstrokecolor{currentstroke}%
\pgfsetstrokeopacity{0.800000}%
\pgfsetdash{}{0pt}%
\pgfpathmoveto{\pgfqpoint{10.836356in}{5.524317in}}%
\pgfpathcurveto{\pgfqpoint{10.847406in}{5.524317in}}{\pgfqpoint{10.858005in}{5.528707in}}{\pgfqpoint{10.865818in}{5.536521in}}%
\pgfpathcurveto{\pgfqpoint{10.873632in}{5.544335in}}{\pgfqpoint{10.878022in}{5.554934in}}{\pgfqpoint{10.878022in}{5.565984in}}%
\pgfpathcurveto{\pgfqpoint{10.878022in}{5.577034in}}{\pgfqpoint{10.873632in}{5.587633in}}{\pgfqpoint{10.865818in}{5.595446in}}%
\pgfpathcurveto{\pgfqpoint{10.858005in}{5.603260in}}{\pgfqpoint{10.847406in}{5.607650in}}{\pgfqpoint{10.836356in}{5.607650in}}%
\pgfpathcurveto{\pgfqpoint{10.825305in}{5.607650in}}{\pgfqpoint{10.814706in}{5.603260in}}{\pgfqpoint{10.806893in}{5.595446in}}%
\pgfpathcurveto{\pgfqpoint{10.799079in}{5.587633in}}{\pgfqpoint{10.794689in}{5.577034in}}{\pgfqpoint{10.794689in}{5.565984in}}%
\pgfpathcurveto{\pgfqpoint{10.794689in}{5.554934in}}{\pgfqpoint{10.799079in}{5.544335in}}{\pgfqpoint{10.806893in}{5.536521in}}%
\pgfpathcurveto{\pgfqpoint{10.814706in}{5.528707in}}{\pgfqpoint{10.825305in}{5.524317in}}{\pgfqpoint{10.836356in}{5.524317in}}%
\pgfpathlineto{\pgfqpoint{10.836356in}{5.524317in}}%
\pgfpathclose%
\pgfusepath{stroke}%
\end{pgfscope}%
\begin{pgfscope}%
\pgfpathrectangle{\pgfqpoint{7.512535in}{0.437222in}}{\pgfqpoint{6.275590in}{5.159444in}}%
\pgfusepath{clip}%
\pgfsetbuttcap%
\pgfsetroundjoin%
\pgfsetlinewidth{1.003750pt}%
\definecolor{currentstroke}{rgb}{0.827451,0.827451,0.827451}%
\pgfsetstrokecolor{currentstroke}%
\pgfsetstrokeopacity{0.800000}%
\pgfsetdash{}{0pt}%
\pgfpathmoveto{\pgfqpoint{8.680117in}{3.344222in}}%
\pgfpathcurveto{\pgfqpoint{8.691167in}{3.344222in}}{\pgfqpoint{8.701766in}{3.348612in}}{\pgfqpoint{8.709580in}{3.356426in}}%
\pgfpathcurveto{\pgfqpoint{8.717393in}{3.364239in}}{\pgfqpoint{8.721784in}{3.374838in}}{\pgfqpoint{8.721784in}{3.385889in}}%
\pgfpathcurveto{\pgfqpoint{8.721784in}{3.396939in}}{\pgfqpoint{8.717393in}{3.407538in}}{\pgfqpoint{8.709580in}{3.415351in}}%
\pgfpathcurveto{\pgfqpoint{8.701766in}{3.423165in}}{\pgfqpoint{8.691167in}{3.427555in}}{\pgfqpoint{8.680117in}{3.427555in}}%
\pgfpathcurveto{\pgfqpoint{8.669067in}{3.427555in}}{\pgfqpoint{8.658468in}{3.423165in}}{\pgfqpoint{8.650654in}{3.415351in}}%
\pgfpathcurveto{\pgfqpoint{8.642841in}{3.407538in}}{\pgfqpoint{8.638450in}{3.396939in}}{\pgfqpoint{8.638450in}{3.385889in}}%
\pgfpathcurveto{\pgfqpoint{8.638450in}{3.374838in}}{\pgfqpoint{8.642841in}{3.364239in}}{\pgfqpoint{8.650654in}{3.356426in}}%
\pgfpathcurveto{\pgfqpoint{8.658468in}{3.348612in}}{\pgfqpoint{8.669067in}{3.344222in}}{\pgfqpoint{8.680117in}{3.344222in}}%
\pgfpathlineto{\pgfqpoint{8.680117in}{3.344222in}}%
\pgfpathclose%
\pgfusepath{stroke}%
\end{pgfscope}%
\begin{pgfscope}%
\pgfpathrectangle{\pgfqpoint{7.512535in}{0.437222in}}{\pgfqpoint{6.275590in}{5.159444in}}%
\pgfusepath{clip}%
\pgfsetbuttcap%
\pgfsetroundjoin%
\pgfsetlinewidth{1.003750pt}%
\definecolor{currentstroke}{rgb}{0.827451,0.827451,0.827451}%
\pgfsetstrokecolor{currentstroke}%
\pgfsetstrokeopacity{0.800000}%
\pgfsetdash{}{0pt}%
\pgfpathmoveto{\pgfqpoint{8.605492in}{2.894644in}}%
\pgfpathcurveto{\pgfqpoint{8.616542in}{2.894644in}}{\pgfqpoint{8.627141in}{2.899034in}}{\pgfqpoint{8.634955in}{2.906847in}}%
\pgfpathcurveto{\pgfqpoint{8.642768in}{2.914661in}}{\pgfqpoint{8.647158in}{2.925260in}}{\pgfqpoint{8.647158in}{2.936310in}}%
\pgfpathcurveto{\pgfqpoint{8.647158in}{2.947360in}}{\pgfqpoint{8.642768in}{2.957959in}}{\pgfqpoint{8.634955in}{2.965773in}}%
\pgfpathcurveto{\pgfqpoint{8.627141in}{2.973587in}}{\pgfqpoint{8.616542in}{2.977977in}}{\pgfqpoint{8.605492in}{2.977977in}}%
\pgfpathcurveto{\pgfqpoint{8.594442in}{2.977977in}}{\pgfqpoint{8.583843in}{2.973587in}}{\pgfqpoint{8.576029in}{2.965773in}}%
\pgfpathcurveto{\pgfqpoint{8.568215in}{2.957959in}}{\pgfqpoint{8.563825in}{2.947360in}}{\pgfqpoint{8.563825in}{2.936310in}}%
\pgfpathcurveto{\pgfqpoint{8.563825in}{2.925260in}}{\pgfqpoint{8.568215in}{2.914661in}}{\pgfqpoint{8.576029in}{2.906847in}}%
\pgfpathcurveto{\pgfqpoint{8.583843in}{2.899034in}}{\pgfqpoint{8.594442in}{2.894644in}}{\pgfqpoint{8.605492in}{2.894644in}}%
\pgfpathlineto{\pgfqpoint{8.605492in}{2.894644in}}%
\pgfpathclose%
\pgfusepath{stroke}%
\end{pgfscope}%
\begin{pgfscope}%
\pgfpathrectangle{\pgfqpoint{7.512535in}{0.437222in}}{\pgfqpoint{6.275590in}{5.159444in}}%
\pgfusepath{clip}%
\pgfsetbuttcap%
\pgfsetroundjoin%
\pgfsetlinewidth{1.003750pt}%
\definecolor{currentstroke}{rgb}{0.827451,0.827451,0.827451}%
\pgfsetstrokecolor{currentstroke}%
\pgfsetstrokeopacity{0.800000}%
\pgfsetdash{}{0pt}%
\pgfpathmoveto{\pgfqpoint{9.726466in}{3.345111in}}%
\pgfpathcurveto{\pgfqpoint{9.737516in}{3.345111in}}{\pgfqpoint{9.748115in}{3.349501in}}{\pgfqpoint{9.755929in}{3.357315in}}%
\pgfpathcurveto{\pgfqpoint{9.763743in}{3.365128in}}{\pgfqpoint{9.768133in}{3.375727in}}{\pgfqpoint{9.768133in}{3.386778in}}%
\pgfpathcurveto{\pgfqpoint{9.768133in}{3.397828in}}{\pgfqpoint{9.763743in}{3.408427in}}{\pgfqpoint{9.755929in}{3.416240in}}%
\pgfpathcurveto{\pgfqpoint{9.748115in}{3.424054in}}{\pgfqpoint{9.737516in}{3.428444in}}{\pgfqpoint{9.726466in}{3.428444in}}%
\pgfpathcurveto{\pgfqpoint{9.715416in}{3.428444in}}{\pgfqpoint{9.704817in}{3.424054in}}{\pgfqpoint{9.697004in}{3.416240in}}%
\pgfpathcurveto{\pgfqpoint{9.689190in}{3.408427in}}{\pgfqpoint{9.684800in}{3.397828in}}{\pgfqpoint{9.684800in}{3.386778in}}%
\pgfpathcurveto{\pgfqpoint{9.684800in}{3.375727in}}{\pgfqpoint{9.689190in}{3.365128in}}{\pgfqpoint{9.697004in}{3.357315in}}%
\pgfpathcurveto{\pgfqpoint{9.704817in}{3.349501in}}{\pgfqpoint{9.715416in}{3.345111in}}{\pgfqpoint{9.726466in}{3.345111in}}%
\pgfpathlineto{\pgfqpoint{9.726466in}{3.345111in}}%
\pgfpathclose%
\pgfusepath{stroke}%
\end{pgfscope}%
\begin{pgfscope}%
\pgfpathrectangle{\pgfqpoint{7.512535in}{0.437222in}}{\pgfqpoint{6.275590in}{5.159444in}}%
\pgfusepath{clip}%
\pgfsetbuttcap%
\pgfsetroundjoin%
\pgfsetlinewidth{1.003750pt}%
\definecolor{currentstroke}{rgb}{0.827451,0.827451,0.827451}%
\pgfsetstrokecolor{currentstroke}%
\pgfsetstrokeopacity{0.800000}%
\pgfsetdash{}{0pt}%
\pgfpathmoveto{\pgfqpoint{9.848678in}{4.313606in}}%
\pgfpathcurveto{\pgfqpoint{9.859728in}{4.313606in}}{\pgfqpoint{9.870327in}{4.317996in}}{\pgfqpoint{9.878141in}{4.325810in}}%
\pgfpathcurveto{\pgfqpoint{9.885955in}{4.333623in}}{\pgfqpoint{9.890345in}{4.344222in}}{\pgfqpoint{9.890345in}{4.355272in}}%
\pgfpathcurveto{\pgfqpoint{9.890345in}{4.366323in}}{\pgfqpoint{9.885955in}{4.376922in}}{\pgfqpoint{9.878141in}{4.384735in}}%
\pgfpathcurveto{\pgfqpoint{9.870327in}{4.392549in}}{\pgfqpoint{9.859728in}{4.396939in}}{\pgfqpoint{9.848678in}{4.396939in}}%
\pgfpathcurveto{\pgfqpoint{9.837628in}{4.396939in}}{\pgfqpoint{9.827029in}{4.392549in}}{\pgfqpoint{9.819216in}{4.384735in}}%
\pgfpathcurveto{\pgfqpoint{9.811402in}{4.376922in}}{\pgfqpoint{9.807012in}{4.366323in}}{\pgfqpoint{9.807012in}{4.355272in}}%
\pgfpathcurveto{\pgfqpoint{9.807012in}{4.344222in}}{\pgfqpoint{9.811402in}{4.333623in}}{\pgfqpoint{9.819216in}{4.325810in}}%
\pgfpathcurveto{\pgfqpoint{9.827029in}{4.317996in}}{\pgfqpoint{9.837628in}{4.313606in}}{\pgfqpoint{9.848678in}{4.313606in}}%
\pgfpathlineto{\pgfqpoint{9.848678in}{4.313606in}}%
\pgfpathclose%
\pgfusepath{stroke}%
\end{pgfscope}%
\begin{pgfscope}%
\pgfpathrectangle{\pgfqpoint{7.512535in}{0.437222in}}{\pgfqpoint{6.275590in}{5.159444in}}%
\pgfusepath{clip}%
\pgfsetbuttcap%
\pgfsetroundjoin%
\pgfsetlinewidth{1.003750pt}%
\definecolor{currentstroke}{rgb}{0.827451,0.827451,0.827451}%
\pgfsetstrokecolor{currentstroke}%
\pgfsetstrokeopacity{0.800000}%
\pgfsetdash{}{0pt}%
\pgfpathmoveto{\pgfqpoint{8.282445in}{1.938912in}}%
\pgfpathcurveto{\pgfqpoint{8.293495in}{1.938912in}}{\pgfqpoint{8.304095in}{1.943302in}}{\pgfqpoint{8.311908in}{1.951116in}}%
\pgfpathcurveto{\pgfqpoint{8.319722in}{1.958929in}}{\pgfqpoint{8.324112in}{1.969528in}}{\pgfqpoint{8.324112in}{1.980579in}}%
\pgfpathcurveto{\pgfqpoint{8.324112in}{1.991629in}}{\pgfqpoint{8.319722in}{2.002228in}}{\pgfqpoint{8.311908in}{2.010041in}}%
\pgfpathcurveto{\pgfqpoint{8.304095in}{2.017855in}}{\pgfqpoint{8.293495in}{2.022245in}}{\pgfqpoint{8.282445in}{2.022245in}}%
\pgfpathcurveto{\pgfqpoint{8.271395in}{2.022245in}}{\pgfqpoint{8.260796in}{2.017855in}}{\pgfqpoint{8.252983in}{2.010041in}}%
\pgfpathcurveto{\pgfqpoint{8.245169in}{2.002228in}}{\pgfqpoint{8.240779in}{1.991629in}}{\pgfqpoint{8.240779in}{1.980579in}}%
\pgfpathcurveto{\pgfqpoint{8.240779in}{1.969528in}}{\pgfqpoint{8.245169in}{1.958929in}}{\pgfqpoint{8.252983in}{1.951116in}}%
\pgfpathcurveto{\pgfqpoint{8.260796in}{1.943302in}}{\pgfqpoint{8.271395in}{1.938912in}}{\pgfqpoint{8.282445in}{1.938912in}}%
\pgfpathlineto{\pgfqpoint{8.282445in}{1.938912in}}%
\pgfpathclose%
\pgfusepath{stroke}%
\end{pgfscope}%
\begin{pgfscope}%
\pgfpathrectangle{\pgfqpoint{7.512535in}{0.437222in}}{\pgfqpoint{6.275590in}{5.159444in}}%
\pgfusepath{clip}%
\pgfsetbuttcap%
\pgfsetroundjoin%
\pgfsetlinewidth{1.003750pt}%
\definecolor{currentstroke}{rgb}{0.827451,0.827451,0.827451}%
\pgfsetstrokecolor{currentstroke}%
\pgfsetstrokeopacity{0.800000}%
\pgfsetdash{}{0pt}%
\pgfpathmoveto{\pgfqpoint{7.874057in}{1.320145in}}%
\pgfpathcurveto{\pgfqpoint{7.885107in}{1.320145in}}{\pgfqpoint{7.895706in}{1.324535in}}{\pgfqpoint{7.903520in}{1.332349in}}%
\pgfpathcurveto{\pgfqpoint{7.911334in}{1.340162in}}{\pgfqpoint{7.915724in}{1.350761in}}{\pgfqpoint{7.915724in}{1.361811in}}%
\pgfpathcurveto{\pgfqpoint{7.915724in}{1.372862in}}{\pgfqpoint{7.911334in}{1.383461in}}{\pgfqpoint{7.903520in}{1.391274in}}%
\pgfpathcurveto{\pgfqpoint{7.895706in}{1.399088in}}{\pgfqpoint{7.885107in}{1.403478in}}{\pgfqpoint{7.874057in}{1.403478in}}%
\pgfpathcurveto{\pgfqpoint{7.863007in}{1.403478in}}{\pgfqpoint{7.852408in}{1.399088in}}{\pgfqpoint{7.844595in}{1.391274in}}%
\pgfpathcurveto{\pgfqpoint{7.836781in}{1.383461in}}{\pgfqpoint{7.832391in}{1.372862in}}{\pgfqpoint{7.832391in}{1.361811in}}%
\pgfpathcurveto{\pgfqpoint{7.832391in}{1.350761in}}{\pgfqpoint{7.836781in}{1.340162in}}{\pgfqpoint{7.844595in}{1.332349in}}%
\pgfpathcurveto{\pgfqpoint{7.852408in}{1.324535in}}{\pgfqpoint{7.863007in}{1.320145in}}{\pgfqpoint{7.874057in}{1.320145in}}%
\pgfpathlineto{\pgfqpoint{7.874057in}{1.320145in}}%
\pgfpathclose%
\pgfusepath{stroke}%
\end{pgfscope}%
\begin{pgfscope}%
\pgfpathrectangle{\pgfqpoint{7.512535in}{0.437222in}}{\pgfqpoint{6.275590in}{5.159444in}}%
\pgfusepath{clip}%
\pgfsetbuttcap%
\pgfsetroundjoin%
\pgfsetlinewidth{1.003750pt}%
\definecolor{currentstroke}{rgb}{0.827451,0.827451,0.827451}%
\pgfsetstrokecolor{currentstroke}%
\pgfsetstrokeopacity{0.800000}%
\pgfsetdash{}{0pt}%
\pgfpathmoveto{\pgfqpoint{12.342968in}{5.553091in}}%
\pgfpathcurveto{\pgfqpoint{12.354019in}{5.553091in}}{\pgfqpoint{12.364618in}{5.557481in}}{\pgfqpoint{12.372431in}{5.565295in}}%
\pgfpathcurveto{\pgfqpoint{12.380245in}{5.573108in}}{\pgfqpoint{12.384635in}{5.583707in}}{\pgfqpoint{12.384635in}{5.594757in}}%
\pgfpathcurveto{\pgfqpoint{12.384635in}{5.605808in}}{\pgfqpoint{12.380245in}{5.616407in}}{\pgfqpoint{12.372431in}{5.624220in}}%
\pgfpathcurveto{\pgfqpoint{12.364618in}{5.632034in}}{\pgfqpoint{12.354019in}{5.636424in}}{\pgfqpoint{12.342968in}{5.636424in}}%
\pgfpathcurveto{\pgfqpoint{12.331918in}{5.636424in}}{\pgfqpoint{12.321319in}{5.632034in}}{\pgfqpoint{12.313506in}{5.624220in}}%
\pgfpathcurveto{\pgfqpoint{12.305692in}{5.616407in}}{\pgfqpoint{12.301302in}{5.605808in}}{\pgfqpoint{12.301302in}{5.594757in}}%
\pgfpathcurveto{\pgfqpoint{12.301302in}{5.583707in}}{\pgfqpoint{12.305692in}{5.573108in}}{\pgfqpoint{12.313506in}{5.565295in}}%
\pgfpathcurveto{\pgfqpoint{12.321319in}{5.557481in}}{\pgfqpoint{12.331918in}{5.553091in}}{\pgfqpoint{12.342968in}{5.553091in}}%
\pgfpathlineto{\pgfqpoint{12.342968in}{5.553091in}}%
\pgfpathclose%
\pgfusepath{stroke}%
\end{pgfscope}%
\begin{pgfscope}%
\pgfpathrectangle{\pgfqpoint{7.512535in}{0.437222in}}{\pgfqpoint{6.275590in}{5.159444in}}%
\pgfusepath{clip}%
\pgfsetbuttcap%
\pgfsetroundjoin%
\pgfsetlinewidth{1.003750pt}%
\definecolor{currentstroke}{rgb}{0.827451,0.827451,0.827451}%
\pgfsetstrokecolor{currentstroke}%
\pgfsetstrokeopacity{0.800000}%
\pgfsetdash{}{0pt}%
\pgfpathmoveto{\pgfqpoint{13.101307in}{5.469388in}}%
\pgfpathcurveto{\pgfqpoint{13.112357in}{5.469388in}}{\pgfqpoint{13.122956in}{5.473778in}}{\pgfqpoint{13.130770in}{5.481592in}}%
\pgfpathcurveto{\pgfqpoint{13.138583in}{5.489405in}}{\pgfqpoint{13.142973in}{5.500004in}}{\pgfqpoint{13.142973in}{5.511054in}}%
\pgfpathcurveto{\pgfqpoint{13.142973in}{5.522105in}}{\pgfqpoint{13.138583in}{5.532704in}}{\pgfqpoint{13.130770in}{5.540517in}}%
\pgfpathcurveto{\pgfqpoint{13.122956in}{5.548331in}}{\pgfqpoint{13.112357in}{5.552721in}}{\pgfqpoint{13.101307in}{5.552721in}}%
\pgfpathcurveto{\pgfqpoint{13.090257in}{5.552721in}}{\pgfqpoint{13.079658in}{5.548331in}}{\pgfqpoint{13.071844in}{5.540517in}}%
\pgfpathcurveto{\pgfqpoint{13.064030in}{5.532704in}}{\pgfqpoint{13.059640in}{5.522105in}}{\pgfqpoint{13.059640in}{5.511054in}}%
\pgfpathcurveto{\pgfqpoint{13.059640in}{5.500004in}}{\pgfqpoint{13.064030in}{5.489405in}}{\pgfqpoint{13.071844in}{5.481592in}}%
\pgfpathcurveto{\pgfqpoint{13.079658in}{5.473778in}}{\pgfqpoint{13.090257in}{5.469388in}}{\pgfqpoint{13.101307in}{5.469388in}}%
\pgfpathlineto{\pgfqpoint{13.101307in}{5.469388in}}%
\pgfpathclose%
\pgfusepath{stroke}%
\end{pgfscope}%
\begin{pgfscope}%
\pgfpathrectangle{\pgfqpoint{7.512535in}{0.437222in}}{\pgfqpoint{6.275590in}{5.159444in}}%
\pgfusepath{clip}%
\pgfsetbuttcap%
\pgfsetroundjoin%
\pgfsetlinewidth{1.003750pt}%
\definecolor{currentstroke}{rgb}{0.827451,0.827451,0.827451}%
\pgfsetstrokecolor{currentstroke}%
\pgfsetstrokeopacity{0.800000}%
\pgfsetdash{}{0pt}%
\pgfpathmoveto{\pgfqpoint{11.982363in}{5.376424in}}%
\pgfpathcurveto{\pgfqpoint{11.993414in}{5.376424in}}{\pgfqpoint{12.004013in}{5.380814in}}{\pgfqpoint{12.011826in}{5.388628in}}%
\pgfpathcurveto{\pgfqpoint{12.019640in}{5.396441in}}{\pgfqpoint{12.024030in}{5.407040in}}{\pgfqpoint{12.024030in}{5.418090in}}%
\pgfpathcurveto{\pgfqpoint{12.024030in}{5.429140in}}{\pgfqpoint{12.019640in}{5.439739in}}{\pgfqpoint{12.011826in}{5.447553in}}%
\pgfpathcurveto{\pgfqpoint{12.004013in}{5.455367in}}{\pgfqpoint{11.993414in}{5.459757in}}{\pgfqpoint{11.982363in}{5.459757in}}%
\pgfpathcurveto{\pgfqpoint{11.971313in}{5.459757in}}{\pgfqpoint{11.960714in}{5.455367in}}{\pgfqpoint{11.952901in}{5.447553in}}%
\pgfpathcurveto{\pgfqpoint{11.945087in}{5.439739in}}{\pgfqpoint{11.940697in}{5.429140in}}{\pgfqpoint{11.940697in}{5.418090in}}%
\pgfpathcurveto{\pgfqpoint{11.940697in}{5.407040in}}{\pgfqpoint{11.945087in}{5.396441in}}{\pgfqpoint{11.952901in}{5.388628in}}%
\pgfpathcurveto{\pgfqpoint{11.960714in}{5.380814in}}{\pgfqpoint{11.971313in}{5.376424in}}{\pgfqpoint{11.982363in}{5.376424in}}%
\pgfpathlineto{\pgfqpoint{11.982363in}{5.376424in}}%
\pgfpathclose%
\pgfusepath{stroke}%
\end{pgfscope}%
\begin{pgfscope}%
\pgfpathrectangle{\pgfqpoint{7.512535in}{0.437222in}}{\pgfqpoint{6.275590in}{5.159444in}}%
\pgfusepath{clip}%
\pgfsetbuttcap%
\pgfsetroundjoin%
\pgfsetlinewidth{1.003750pt}%
\definecolor{currentstroke}{rgb}{0.827451,0.827451,0.827451}%
\pgfsetstrokecolor{currentstroke}%
\pgfsetstrokeopacity{0.800000}%
\pgfsetdash{}{0pt}%
\pgfpathmoveto{\pgfqpoint{8.524887in}{2.541580in}}%
\pgfpathcurveto{\pgfqpoint{8.535937in}{2.541580in}}{\pgfqpoint{8.546536in}{2.545970in}}{\pgfqpoint{8.554350in}{2.553784in}}%
\pgfpathcurveto{\pgfqpoint{8.562163in}{2.561597in}}{\pgfqpoint{8.566554in}{2.572196in}}{\pgfqpoint{8.566554in}{2.583246in}}%
\pgfpathcurveto{\pgfqpoint{8.566554in}{2.594296in}}{\pgfqpoint{8.562163in}{2.604896in}}{\pgfqpoint{8.554350in}{2.612709in}}%
\pgfpathcurveto{\pgfqpoint{8.546536in}{2.620523in}}{\pgfqpoint{8.535937in}{2.624913in}}{\pgfqpoint{8.524887in}{2.624913in}}%
\pgfpathcurveto{\pgfqpoint{8.513837in}{2.624913in}}{\pgfqpoint{8.503238in}{2.620523in}}{\pgfqpoint{8.495424in}{2.612709in}}%
\pgfpathcurveto{\pgfqpoint{8.487611in}{2.604896in}}{\pgfqpoint{8.483220in}{2.594296in}}{\pgfqpoint{8.483220in}{2.583246in}}%
\pgfpathcurveto{\pgfqpoint{8.483220in}{2.572196in}}{\pgfqpoint{8.487611in}{2.561597in}}{\pgfqpoint{8.495424in}{2.553784in}}%
\pgfpathcurveto{\pgfqpoint{8.503238in}{2.545970in}}{\pgfqpoint{8.513837in}{2.541580in}}{\pgfqpoint{8.524887in}{2.541580in}}%
\pgfpathlineto{\pgfqpoint{8.524887in}{2.541580in}}%
\pgfpathclose%
\pgfusepath{stroke}%
\end{pgfscope}%
\begin{pgfscope}%
\pgfpathrectangle{\pgfqpoint{7.512535in}{0.437222in}}{\pgfqpoint{6.275590in}{5.159444in}}%
\pgfusepath{clip}%
\pgfsetbuttcap%
\pgfsetroundjoin%
\pgfsetlinewidth{1.003750pt}%
\definecolor{currentstroke}{rgb}{0.827451,0.827451,0.827451}%
\pgfsetstrokecolor{currentstroke}%
\pgfsetstrokeopacity{0.800000}%
\pgfsetdash{}{0pt}%
\pgfpathmoveto{\pgfqpoint{7.992439in}{1.704977in}}%
\pgfpathcurveto{\pgfqpoint{8.003489in}{1.704977in}}{\pgfqpoint{8.014088in}{1.709367in}}{\pgfqpoint{8.021902in}{1.717181in}}%
\pgfpathcurveto{\pgfqpoint{8.029715in}{1.724994in}}{\pgfqpoint{8.034105in}{1.735593in}}{\pgfqpoint{8.034105in}{1.746643in}}%
\pgfpathcurveto{\pgfqpoint{8.034105in}{1.757694in}}{\pgfqpoint{8.029715in}{1.768293in}}{\pgfqpoint{8.021902in}{1.776106in}}%
\pgfpathcurveto{\pgfqpoint{8.014088in}{1.783920in}}{\pgfqpoint{8.003489in}{1.788310in}}{\pgfqpoint{7.992439in}{1.788310in}}%
\pgfpathcurveto{\pgfqpoint{7.981389in}{1.788310in}}{\pgfqpoint{7.970790in}{1.783920in}}{\pgfqpoint{7.962976in}{1.776106in}}%
\pgfpathcurveto{\pgfqpoint{7.955162in}{1.768293in}}{\pgfqpoint{7.950772in}{1.757694in}}{\pgfqpoint{7.950772in}{1.746643in}}%
\pgfpathcurveto{\pgfqpoint{7.950772in}{1.735593in}}{\pgfqpoint{7.955162in}{1.724994in}}{\pgfqpoint{7.962976in}{1.717181in}}%
\pgfpathcurveto{\pgfqpoint{7.970790in}{1.709367in}}{\pgfqpoint{7.981389in}{1.704977in}}{\pgfqpoint{7.992439in}{1.704977in}}%
\pgfpathlineto{\pgfqpoint{7.992439in}{1.704977in}}%
\pgfpathclose%
\pgfusepath{stroke}%
\end{pgfscope}%
\begin{pgfscope}%
\pgfpathrectangle{\pgfqpoint{7.512535in}{0.437222in}}{\pgfqpoint{6.275590in}{5.159444in}}%
\pgfusepath{clip}%
\pgfsetbuttcap%
\pgfsetroundjoin%
\pgfsetlinewidth{1.003750pt}%
\definecolor{currentstroke}{rgb}{0.827451,0.827451,0.827451}%
\pgfsetstrokecolor{currentstroke}%
\pgfsetstrokeopacity{0.800000}%
\pgfsetdash{}{0pt}%
\pgfpathmoveto{\pgfqpoint{8.524887in}{2.490897in}}%
\pgfpathcurveto{\pgfqpoint{8.535937in}{2.490897in}}{\pgfqpoint{8.546536in}{2.495287in}}{\pgfqpoint{8.554350in}{2.503101in}}%
\pgfpathcurveto{\pgfqpoint{8.562163in}{2.510915in}}{\pgfqpoint{8.566554in}{2.521514in}}{\pgfqpoint{8.566554in}{2.532564in}}%
\pgfpathcurveto{\pgfqpoint{8.566554in}{2.543614in}}{\pgfqpoint{8.562163in}{2.554213in}}{\pgfqpoint{8.554350in}{2.562027in}}%
\pgfpathcurveto{\pgfqpoint{8.546536in}{2.569840in}}{\pgfqpoint{8.535937in}{2.574230in}}{\pgfqpoint{8.524887in}{2.574230in}}%
\pgfpathcurveto{\pgfqpoint{8.513837in}{2.574230in}}{\pgfqpoint{8.503238in}{2.569840in}}{\pgfqpoint{8.495424in}{2.562027in}}%
\pgfpathcurveto{\pgfqpoint{8.487611in}{2.554213in}}{\pgfqpoint{8.483220in}{2.543614in}}{\pgfqpoint{8.483220in}{2.532564in}}%
\pgfpathcurveto{\pgfqpoint{8.483220in}{2.521514in}}{\pgfqpoint{8.487611in}{2.510915in}}{\pgfqpoint{8.495424in}{2.503101in}}%
\pgfpathcurveto{\pgfqpoint{8.503238in}{2.495287in}}{\pgfqpoint{8.513837in}{2.490897in}}{\pgfqpoint{8.524887in}{2.490897in}}%
\pgfpathlineto{\pgfqpoint{8.524887in}{2.490897in}}%
\pgfpathclose%
\pgfusepath{stroke}%
\end{pgfscope}%
\begin{pgfscope}%
\pgfpathrectangle{\pgfqpoint{7.512535in}{0.437222in}}{\pgfqpoint{6.275590in}{5.159444in}}%
\pgfusepath{clip}%
\pgfsetbuttcap%
\pgfsetroundjoin%
\pgfsetlinewidth{1.003750pt}%
\definecolor{currentstroke}{rgb}{0.827451,0.827451,0.827451}%
\pgfsetstrokecolor{currentstroke}%
\pgfsetstrokeopacity{0.800000}%
\pgfsetdash{}{0pt}%
\pgfpathmoveto{\pgfqpoint{11.962704in}{5.469388in}}%
\pgfpathcurveto{\pgfqpoint{11.973755in}{5.469388in}}{\pgfqpoint{11.984354in}{5.473778in}}{\pgfqpoint{11.992167in}{5.481592in}}%
\pgfpathcurveto{\pgfqpoint{11.999981in}{5.489405in}}{\pgfqpoint{12.004371in}{5.500004in}}{\pgfqpoint{12.004371in}{5.511054in}}%
\pgfpathcurveto{\pgfqpoint{12.004371in}{5.522105in}}{\pgfqpoint{11.999981in}{5.532704in}}{\pgfqpoint{11.992167in}{5.540517in}}%
\pgfpathcurveto{\pgfqpoint{11.984354in}{5.548331in}}{\pgfqpoint{11.973755in}{5.552721in}}{\pgfqpoint{11.962704in}{5.552721in}}%
\pgfpathcurveto{\pgfqpoint{11.951654in}{5.552721in}}{\pgfqpoint{11.941055in}{5.548331in}}{\pgfqpoint{11.933242in}{5.540517in}}%
\pgfpathcurveto{\pgfqpoint{11.925428in}{5.532704in}}{\pgfqpoint{11.921038in}{5.522105in}}{\pgfqpoint{11.921038in}{5.511054in}}%
\pgfpathcurveto{\pgfqpoint{11.921038in}{5.500004in}}{\pgfqpoint{11.925428in}{5.489405in}}{\pgfqpoint{11.933242in}{5.481592in}}%
\pgfpathcurveto{\pgfqpoint{11.941055in}{5.473778in}}{\pgfqpoint{11.951654in}{5.469388in}}{\pgfqpoint{11.962704in}{5.469388in}}%
\pgfpathlineto{\pgfqpoint{11.962704in}{5.469388in}}%
\pgfpathclose%
\pgfusepath{stroke}%
\end{pgfscope}%
\begin{pgfscope}%
\pgfpathrectangle{\pgfqpoint{7.512535in}{0.437222in}}{\pgfqpoint{6.275590in}{5.159444in}}%
\pgfusepath{clip}%
\pgfsetbuttcap%
\pgfsetroundjoin%
\pgfsetlinewidth{1.003750pt}%
\definecolor{currentstroke}{rgb}{0.827451,0.827451,0.827451}%
\pgfsetstrokecolor{currentstroke}%
\pgfsetstrokeopacity{0.800000}%
\pgfsetdash{}{0pt}%
\pgfpathmoveto{\pgfqpoint{8.716039in}{3.198227in}}%
\pgfpathcurveto{\pgfqpoint{8.727089in}{3.198227in}}{\pgfqpoint{8.737688in}{3.202618in}}{\pgfqpoint{8.745502in}{3.210431in}}%
\pgfpathcurveto{\pgfqpoint{8.753315in}{3.218245in}}{\pgfqpoint{8.757706in}{3.228844in}}{\pgfqpoint{8.757706in}{3.239894in}}%
\pgfpathcurveto{\pgfqpoint{8.757706in}{3.250944in}}{\pgfqpoint{8.753315in}{3.261543in}}{\pgfqpoint{8.745502in}{3.269357in}}%
\pgfpathcurveto{\pgfqpoint{8.737688in}{3.277170in}}{\pgfqpoint{8.727089in}{3.281561in}}{\pgfqpoint{8.716039in}{3.281561in}}%
\pgfpathcurveto{\pgfqpoint{8.704989in}{3.281561in}}{\pgfqpoint{8.694390in}{3.277170in}}{\pgfqpoint{8.686576in}{3.269357in}}%
\pgfpathcurveto{\pgfqpoint{8.678763in}{3.261543in}}{\pgfqpoint{8.674372in}{3.250944in}}{\pgfqpoint{8.674372in}{3.239894in}}%
\pgfpathcurveto{\pgfqpoint{8.674372in}{3.228844in}}{\pgfqpoint{8.678763in}{3.218245in}}{\pgfqpoint{8.686576in}{3.210431in}}%
\pgfpathcurveto{\pgfqpoint{8.694390in}{3.202618in}}{\pgfqpoint{8.704989in}{3.198227in}}{\pgfqpoint{8.716039in}{3.198227in}}%
\pgfpathlineto{\pgfqpoint{8.716039in}{3.198227in}}%
\pgfpathclose%
\pgfusepath{stroke}%
\end{pgfscope}%
\begin{pgfscope}%
\pgfpathrectangle{\pgfqpoint{7.512535in}{0.437222in}}{\pgfqpoint{6.275590in}{5.159444in}}%
\pgfusepath{clip}%
\pgfsetbuttcap%
\pgfsetroundjoin%
\pgfsetlinewidth{1.003750pt}%
\definecolor{currentstroke}{rgb}{0.827451,0.827451,0.827451}%
\pgfsetstrokecolor{currentstroke}%
\pgfsetstrokeopacity{0.800000}%
\pgfsetdash{}{0pt}%
\pgfpathmoveto{\pgfqpoint{8.598634in}{1.552340in}}%
\pgfpathcurveto{\pgfqpoint{8.609684in}{1.552340in}}{\pgfqpoint{8.620283in}{1.556730in}}{\pgfqpoint{8.628097in}{1.564544in}}%
\pgfpathcurveto{\pgfqpoint{8.635910in}{1.572358in}}{\pgfqpoint{8.640300in}{1.582957in}}{\pgfqpoint{8.640300in}{1.594007in}}%
\pgfpathcurveto{\pgfqpoint{8.640300in}{1.605057in}}{\pgfqpoint{8.635910in}{1.615656in}}{\pgfqpoint{8.628097in}{1.623470in}}%
\pgfpathcurveto{\pgfqpoint{8.620283in}{1.631283in}}{\pgfqpoint{8.609684in}{1.635673in}}{\pgfqpoint{8.598634in}{1.635673in}}%
\pgfpathcurveto{\pgfqpoint{8.587584in}{1.635673in}}{\pgfqpoint{8.576985in}{1.631283in}}{\pgfqpoint{8.569171in}{1.623470in}}%
\pgfpathcurveto{\pgfqpoint{8.561357in}{1.615656in}}{\pgfqpoint{8.556967in}{1.605057in}}{\pgfqpoint{8.556967in}{1.594007in}}%
\pgfpathcurveto{\pgfqpoint{8.556967in}{1.582957in}}{\pgfqpoint{8.561357in}{1.572358in}}{\pgfqpoint{8.569171in}{1.564544in}}%
\pgfpathcurveto{\pgfqpoint{8.576985in}{1.556730in}}{\pgfqpoint{8.587584in}{1.552340in}}{\pgfqpoint{8.598634in}{1.552340in}}%
\pgfpathlineto{\pgfqpoint{8.598634in}{1.552340in}}%
\pgfpathclose%
\pgfusepath{stroke}%
\end{pgfscope}%
\begin{pgfscope}%
\pgfpathrectangle{\pgfqpoint{7.512535in}{0.437222in}}{\pgfqpoint{6.275590in}{5.159444in}}%
\pgfusepath{clip}%
\pgfsetbuttcap%
\pgfsetroundjoin%
\pgfsetlinewidth{1.003750pt}%
\definecolor{currentstroke}{rgb}{0.827451,0.827451,0.827451}%
\pgfsetstrokecolor{currentstroke}%
\pgfsetstrokeopacity{0.800000}%
\pgfsetdash{}{0pt}%
\pgfpathmoveto{\pgfqpoint{9.837488in}{4.056499in}}%
\pgfpathcurveto{\pgfqpoint{9.848539in}{4.056499in}}{\pgfqpoint{9.859138in}{4.060889in}}{\pgfqpoint{9.866951in}{4.068703in}}%
\pgfpathcurveto{\pgfqpoint{9.874765in}{4.076516in}}{\pgfqpoint{9.879155in}{4.087115in}}{\pgfqpoint{9.879155in}{4.098166in}}%
\pgfpathcurveto{\pgfqpoint{9.879155in}{4.109216in}}{\pgfqpoint{9.874765in}{4.119815in}}{\pgfqpoint{9.866951in}{4.127628in}}%
\pgfpathcurveto{\pgfqpoint{9.859138in}{4.135442in}}{\pgfqpoint{9.848539in}{4.139832in}}{\pgfqpoint{9.837488in}{4.139832in}}%
\pgfpathcurveto{\pgfqpoint{9.826438in}{4.139832in}}{\pgfqpoint{9.815839in}{4.135442in}}{\pgfqpoint{9.808026in}{4.127628in}}%
\pgfpathcurveto{\pgfqpoint{9.800212in}{4.119815in}}{\pgfqpoint{9.795822in}{4.109216in}}{\pgfqpoint{9.795822in}{4.098166in}}%
\pgfpathcurveto{\pgfqpoint{9.795822in}{4.087115in}}{\pgfqpoint{9.800212in}{4.076516in}}{\pgfqpoint{9.808026in}{4.068703in}}%
\pgfpathcurveto{\pgfqpoint{9.815839in}{4.060889in}}{\pgfqpoint{9.826438in}{4.056499in}}{\pgfqpoint{9.837488in}{4.056499in}}%
\pgfpathlineto{\pgfqpoint{9.837488in}{4.056499in}}%
\pgfpathclose%
\pgfusepath{stroke}%
\end{pgfscope}%
\begin{pgfscope}%
\pgfpathrectangle{\pgfqpoint{7.512535in}{0.437222in}}{\pgfqpoint{6.275590in}{5.159444in}}%
\pgfusepath{clip}%
\pgfsetbuttcap%
\pgfsetroundjoin%
\pgfsetlinewidth{1.003750pt}%
\definecolor{currentstroke}{rgb}{0.827451,0.827451,0.827451}%
\pgfsetstrokecolor{currentstroke}%
\pgfsetstrokeopacity{0.800000}%
\pgfsetdash{}{0pt}%
\pgfpathmoveto{\pgfqpoint{10.045306in}{4.354371in}}%
\pgfpathcurveto{\pgfqpoint{10.056356in}{4.354371in}}{\pgfqpoint{10.066955in}{4.358761in}}{\pgfqpoint{10.074768in}{4.366575in}}%
\pgfpathcurveto{\pgfqpoint{10.082582in}{4.374389in}}{\pgfqpoint{10.086972in}{4.384988in}}{\pgfqpoint{10.086972in}{4.396038in}}%
\pgfpathcurveto{\pgfqpoint{10.086972in}{4.407088in}}{\pgfqpoint{10.082582in}{4.417687in}}{\pgfqpoint{10.074768in}{4.425500in}}%
\pgfpathcurveto{\pgfqpoint{10.066955in}{4.433314in}}{\pgfqpoint{10.056356in}{4.437704in}}{\pgfqpoint{10.045306in}{4.437704in}}%
\pgfpathcurveto{\pgfqpoint{10.034256in}{4.437704in}}{\pgfqpoint{10.023657in}{4.433314in}}{\pgfqpoint{10.015843in}{4.425500in}}%
\pgfpathcurveto{\pgfqpoint{10.008029in}{4.417687in}}{\pgfqpoint{10.003639in}{4.407088in}}{\pgfqpoint{10.003639in}{4.396038in}}%
\pgfpathcurveto{\pgfqpoint{10.003639in}{4.384988in}}{\pgfqpoint{10.008029in}{4.374389in}}{\pgfqpoint{10.015843in}{4.366575in}}%
\pgfpathcurveto{\pgfqpoint{10.023657in}{4.358761in}}{\pgfqpoint{10.034256in}{4.354371in}}{\pgfqpoint{10.045306in}{4.354371in}}%
\pgfpathlineto{\pgfqpoint{10.045306in}{4.354371in}}%
\pgfpathclose%
\pgfusepath{stroke}%
\end{pgfscope}%
\begin{pgfscope}%
\pgfpathrectangle{\pgfqpoint{7.512535in}{0.437222in}}{\pgfqpoint{6.275590in}{5.159444in}}%
\pgfusepath{clip}%
\pgfsetbuttcap%
\pgfsetroundjoin%
\pgfsetlinewidth{1.003750pt}%
\definecolor{currentstroke}{rgb}{0.827451,0.827451,0.827451}%
\pgfsetstrokecolor{currentstroke}%
\pgfsetstrokeopacity{0.800000}%
\pgfsetdash{}{0pt}%
\pgfpathmoveto{\pgfqpoint{10.385573in}{5.221855in}}%
\pgfpathcurveto{\pgfqpoint{10.396623in}{5.221855in}}{\pgfqpoint{10.407222in}{5.226245in}}{\pgfqpoint{10.415036in}{5.234058in}}%
\pgfpathcurveto{\pgfqpoint{10.422849in}{5.241872in}}{\pgfqpoint{10.427240in}{5.252471in}}{\pgfqpoint{10.427240in}{5.263521in}}%
\pgfpathcurveto{\pgfqpoint{10.427240in}{5.274571in}}{\pgfqpoint{10.422849in}{5.285170in}}{\pgfqpoint{10.415036in}{5.292984in}}%
\pgfpathcurveto{\pgfqpoint{10.407222in}{5.300798in}}{\pgfqpoint{10.396623in}{5.305188in}}{\pgfqpoint{10.385573in}{5.305188in}}%
\pgfpathcurveto{\pgfqpoint{10.374523in}{5.305188in}}{\pgfqpoint{10.363924in}{5.300798in}}{\pgfqpoint{10.356110in}{5.292984in}}%
\pgfpathcurveto{\pgfqpoint{10.348297in}{5.285170in}}{\pgfqpoint{10.343906in}{5.274571in}}{\pgfqpoint{10.343906in}{5.263521in}}%
\pgfpathcurveto{\pgfqpoint{10.343906in}{5.252471in}}{\pgfqpoint{10.348297in}{5.241872in}}{\pgfqpoint{10.356110in}{5.234058in}}%
\pgfpathcurveto{\pgfqpoint{10.363924in}{5.226245in}}{\pgfqpoint{10.374523in}{5.221855in}}{\pgfqpoint{10.385573in}{5.221855in}}%
\pgfpathlineto{\pgfqpoint{10.385573in}{5.221855in}}%
\pgfpathclose%
\pgfusepath{stroke}%
\end{pgfscope}%
\begin{pgfscope}%
\pgfpathrectangle{\pgfqpoint{7.512535in}{0.437222in}}{\pgfqpoint{6.275590in}{5.159444in}}%
\pgfusepath{clip}%
\pgfsetbuttcap%
\pgfsetroundjoin%
\pgfsetlinewidth{1.003750pt}%
\definecolor{currentstroke}{rgb}{0.827451,0.827451,0.827451}%
\pgfsetstrokecolor{currentstroke}%
\pgfsetstrokeopacity{0.800000}%
\pgfsetdash{}{0pt}%
\pgfpathmoveto{\pgfqpoint{10.827454in}{5.304399in}}%
\pgfpathcurveto{\pgfqpoint{10.838504in}{5.304399in}}{\pgfqpoint{10.849103in}{5.308790in}}{\pgfqpoint{10.856917in}{5.316603in}}%
\pgfpathcurveto{\pgfqpoint{10.864730in}{5.324417in}}{\pgfqpoint{10.869121in}{5.335016in}}{\pgfqpoint{10.869121in}{5.346066in}}%
\pgfpathcurveto{\pgfqpoint{10.869121in}{5.357116in}}{\pgfqpoint{10.864730in}{5.367715in}}{\pgfqpoint{10.856917in}{5.375529in}}%
\pgfpathcurveto{\pgfqpoint{10.849103in}{5.383342in}}{\pgfqpoint{10.838504in}{5.387733in}}{\pgfqpoint{10.827454in}{5.387733in}}%
\pgfpathcurveto{\pgfqpoint{10.816404in}{5.387733in}}{\pgfqpoint{10.805805in}{5.383342in}}{\pgfqpoint{10.797991in}{5.375529in}}%
\pgfpathcurveto{\pgfqpoint{10.790177in}{5.367715in}}{\pgfqpoint{10.785787in}{5.357116in}}{\pgfqpoint{10.785787in}{5.346066in}}%
\pgfpathcurveto{\pgfqpoint{10.785787in}{5.335016in}}{\pgfqpoint{10.790177in}{5.324417in}}{\pgfqpoint{10.797991in}{5.316603in}}%
\pgfpathcurveto{\pgfqpoint{10.805805in}{5.308790in}}{\pgfqpoint{10.816404in}{5.304399in}}{\pgfqpoint{10.827454in}{5.304399in}}%
\pgfpathlineto{\pgfqpoint{10.827454in}{5.304399in}}%
\pgfpathclose%
\pgfusepath{stroke}%
\end{pgfscope}%
\begin{pgfscope}%
\pgfpathrectangle{\pgfqpoint{7.512535in}{0.437222in}}{\pgfqpoint{6.275590in}{5.159444in}}%
\pgfusepath{clip}%
\pgfsetbuttcap%
\pgfsetroundjoin%
\pgfsetlinewidth{1.003750pt}%
\definecolor{currentstroke}{rgb}{0.827451,0.827451,0.827451}%
\pgfsetstrokecolor{currentstroke}%
\pgfsetstrokeopacity{0.800000}%
\pgfsetdash{}{0pt}%
\pgfpathmoveto{\pgfqpoint{10.299622in}{3.250664in}}%
\pgfpathcurveto{\pgfqpoint{10.310672in}{3.250664in}}{\pgfqpoint{10.321271in}{3.255054in}}{\pgfqpoint{10.329085in}{3.262868in}}%
\pgfpathcurveto{\pgfqpoint{10.336898in}{3.270682in}}{\pgfqpoint{10.341289in}{3.281281in}}{\pgfqpoint{10.341289in}{3.292331in}}%
\pgfpathcurveto{\pgfqpoint{10.341289in}{3.303381in}}{\pgfqpoint{10.336898in}{3.313980in}}{\pgfqpoint{10.329085in}{3.321794in}}%
\pgfpathcurveto{\pgfqpoint{10.321271in}{3.329607in}}{\pgfqpoint{10.310672in}{3.333998in}}{\pgfqpoint{10.299622in}{3.333998in}}%
\pgfpathcurveto{\pgfqpoint{10.288572in}{3.333998in}}{\pgfqpoint{10.277973in}{3.329607in}}{\pgfqpoint{10.270159in}{3.321794in}}%
\pgfpathcurveto{\pgfqpoint{10.262346in}{3.313980in}}{\pgfqpoint{10.257955in}{3.303381in}}{\pgfqpoint{10.257955in}{3.292331in}}%
\pgfpathcurveto{\pgfqpoint{10.257955in}{3.281281in}}{\pgfqpoint{10.262346in}{3.270682in}}{\pgfqpoint{10.270159in}{3.262868in}}%
\pgfpathcurveto{\pgfqpoint{10.277973in}{3.255054in}}{\pgfqpoint{10.288572in}{3.250664in}}{\pgfqpoint{10.299622in}{3.250664in}}%
\pgfpathlineto{\pgfqpoint{10.299622in}{3.250664in}}%
\pgfpathclose%
\pgfusepath{stroke}%
\end{pgfscope}%
\begin{pgfscope}%
\pgfpathrectangle{\pgfqpoint{7.512535in}{0.437222in}}{\pgfqpoint{6.275590in}{5.159444in}}%
\pgfusepath{clip}%
\pgfsetbuttcap%
\pgfsetroundjoin%
\pgfsetlinewidth{1.003750pt}%
\definecolor{currentstroke}{rgb}{0.827451,0.827451,0.827451}%
\pgfsetstrokecolor{currentstroke}%
\pgfsetstrokeopacity{0.800000}%
\pgfsetdash{}{0pt}%
\pgfpathmoveto{\pgfqpoint{10.372490in}{5.117177in}}%
\pgfpathcurveto{\pgfqpoint{10.383540in}{5.117177in}}{\pgfqpoint{10.394139in}{5.121567in}}{\pgfqpoint{10.401953in}{5.129381in}}%
\pgfpathcurveto{\pgfqpoint{10.409766in}{5.137194in}}{\pgfqpoint{10.414156in}{5.147793in}}{\pgfqpoint{10.414156in}{5.158843in}}%
\pgfpathcurveto{\pgfqpoint{10.414156in}{5.169893in}}{\pgfqpoint{10.409766in}{5.180493in}}{\pgfqpoint{10.401953in}{5.188306in}}%
\pgfpathcurveto{\pgfqpoint{10.394139in}{5.196120in}}{\pgfqpoint{10.383540in}{5.200510in}}{\pgfqpoint{10.372490in}{5.200510in}}%
\pgfpathcurveto{\pgfqpoint{10.361440in}{5.200510in}}{\pgfqpoint{10.350841in}{5.196120in}}{\pgfqpoint{10.343027in}{5.188306in}}%
\pgfpathcurveto{\pgfqpoint{10.335213in}{5.180493in}}{\pgfqpoint{10.330823in}{5.169893in}}{\pgfqpoint{10.330823in}{5.158843in}}%
\pgfpathcurveto{\pgfqpoint{10.330823in}{5.147793in}}{\pgfqpoint{10.335213in}{5.137194in}}{\pgfqpoint{10.343027in}{5.129381in}}%
\pgfpathcurveto{\pgfqpoint{10.350841in}{5.121567in}}{\pgfqpoint{10.361440in}{5.117177in}}{\pgfqpoint{10.372490in}{5.117177in}}%
\pgfpathlineto{\pgfqpoint{10.372490in}{5.117177in}}%
\pgfpathclose%
\pgfusepath{stroke}%
\end{pgfscope}%
\begin{pgfscope}%
\pgfpathrectangle{\pgfqpoint{7.512535in}{0.437222in}}{\pgfqpoint{6.275590in}{5.159444in}}%
\pgfusepath{clip}%
\pgfsetbuttcap%
\pgfsetroundjoin%
\pgfsetlinewidth{1.003750pt}%
\definecolor{currentstroke}{rgb}{0.827451,0.827451,0.827451}%
\pgfsetstrokecolor{currentstroke}%
\pgfsetstrokeopacity{0.800000}%
\pgfsetdash{}{0pt}%
\pgfpathmoveto{\pgfqpoint{8.044541in}{2.113663in}}%
\pgfpathcurveto{\pgfqpoint{8.055591in}{2.113663in}}{\pgfqpoint{8.066190in}{2.118054in}}{\pgfqpoint{8.074003in}{2.125867in}}%
\pgfpathcurveto{\pgfqpoint{8.081817in}{2.133681in}}{\pgfqpoint{8.086207in}{2.144280in}}{\pgfqpoint{8.086207in}{2.155330in}}%
\pgfpathcurveto{\pgfqpoint{8.086207in}{2.166380in}}{\pgfqpoint{8.081817in}{2.176979in}}{\pgfqpoint{8.074003in}{2.184793in}}%
\pgfpathcurveto{\pgfqpoint{8.066190in}{2.192606in}}{\pgfqpoint{8.055591in}{2.196997in}}{\pgfqpoint{8.044541in}{2.196997in}}%
\pgfpathcurveto{\pgfqpoint{8.033491in}{2.196997in}}{\pgfqpoint{8.022891in}{2.192606in}}{\pgfqpoint{8.015078in}{2.184793in}}%
\pgfpathcurveto{\pgfqpoint{8.007264in}{2.176979in}}{\pgfqpoint{8.002874in}{2.166380in}}{\pgfqpoint{8.002874in}{2.155330in}}%
\pgfpathcurveto{\pgfqpoint{8.002874in}{2.144280in}}{\pgfqpoint{8.007264in}{2.133681in}}{\pgfqpoint{8.015078in}{2.125867in}}%
\pgfpathcurveto{\pgfqpoint{8.022891in}{2.118054in}}{\pgfqpoint{8.033491in}{2.113663in}}{\pgfqpoint{8.044541in}{2.113663in}}%
\pgfpathlineto{\pgfqpoint{8.044541in}{2.113663in}}%
\pgfpathclose%
\pgfusepath{stroke}%
\end{pgfscope}%
\begin{pgfscope}%
\pgfpathrectangle{\pgfqpoint{7.512535in}{0.437222in}}{\pgfqpoint{6.275590in}{5.159444in}}%
\pgfusepath{clip}%
\pgfsetbuttcap%
\pgfsetroundjoin%
\pgfsetlinewidth{1.003750pt}%
\definecolor{currentstroke}{rgb}{0.827451,0.827451,0.827451}%
\pgfsetstrokecolor{currentstroke}%
\pgfsetstrokeopacity{0.800000}%
\pgfsetdash{}{0pt}%
\pgfpathmoveto{\pgfqpoint{8.978070in}{1.743970in}}%
\pgfpathcurveto{\pgfqpoint{8.989120in}{1.743970in}}{\pgfqpoint{8.999719in}{1.748360in}}{\pgfqpoint{9.007533in}{1.756174in}}%
\pgfpathcurveto{\pgfqpoint{9.015346in}{1.763988in}}{\pgfqpoint{9.019736in}{1.774587in}}{\pgfqpoint{9.019736in}{1.785637in}}%
\pgfpathcurveto{\pgfqpoint{9.019736in}{1.796687in}}{\pgfqpoint{9.015346in}{1.807286in}}{\pgfqpoint{9.007533in}{1.815100in}}%
\pgfpathcurveto{\pgfqpoint{8.999719in}{1.822913in}}{\pgfqpoint{8.989120in}{1.827303in}}{\pgfqpoint{8.978070in}{1.827303in}}%
\pgfpathcurveto{\pgfqpoint{8.967020in}{1.827303in}}{\pgfqpoint{8.956421in}{1.822913in}}{\pgfqpoint{8.948607in}{1.815100in}}%
\pgfpathcurveto{\pgfqpoint{8.940793in}{1.807286in}}{\pgfqpoint{8.936403in}{1.796687in}}{\pgfqpoint{8.936403in}{1.785637in}}%
\pgfpathcurveto{\pgfqpoint{8.936403in}{1.774587in}}{\pgfqpoint{8.940793in}{1.763988in}}{\pgfqpoint{8.948607in}{1.756174in}}%
\pgfpathcurveto{\pgfqpoint{8.956421in}{1.748360in}}{\pgfqpoint{8.967020in}{1.743970in}}{\pgfqpoint{8.978070in}{1.743970in}}%
\pgfpathlineto{\pgfqpoint{8.978070in}{1.743970in}}%
\pgfpathclose%
\pgfusepath{stroke}%
\end{pgfscope}%
\begin{pgfscope}%
\pgfpathrectangle{\pgfqpoint{7.512535in}{0.437222in}}{\pgfqpoint{6.275590in}{5.159444in}}%
\pgfusepath{clip}%
\pgfsetbuttcap%
\pgfsetroundjoin%
\pgfsetlinewidth{1.003750pt}%
\definecolor{currentstroke}{rgb}{0.827451,0.827451,0.827451}%
\pgfsetstrokecolor{currentstroke}%
\pgfsetstrokeopacity{0.800000}%
\pgfsetdash{}{0pt}%
\pgfpathmoveto{\pgfqpoint{9.630920in}{3.276112in}}%
\pgfpathcurveto{\pgfqpoint{9.641970in}{3.276112in}}{\pgfqpoint{9.652569in}{3.280502in}}{\pgfqpoint{9.660383in}{3.288316in}}%
\pgfpathcurveto{\pgfqpoint{9.668196in}{3.296130in}}{\pgfqpoint{9.672587in}{3.306729in}}{\pgfqpoint{9.672587in}{3.317779in}}%
\pgfpathcurveto{\pgfqpoint{9.672587in}{3.328829in}}{\pgfqpoint{9.668196in}{3.339428in}}{\pgfqpoint{9.660383in}{3.347241in}}%
\pgfpathcurveto{\pgfqpoint{9.652569in}{3.355055in}}{\pgfqpoint{9.641970in}{3.359445in}}{\pgfqpoint{9.630920in}{3.359445in}}%
\pgfpathcurveto{\pgfqpoint{9.619870in}{3.359445in}}{\pgfqpoint{9.609271in}{3.355055in}}{\pgfqpoint{9.601457in}{3.347241in}}%
\pgfpathcurveto{\pgfqpoint{9.593644in}{3.339428in}}{\pgfqpoint{9.589253in}{3.328829in}}{\pgfqpoint{9.589253in}{3.317779in}}%
\pgfpathcurveto{\pgfqpoint{9.589253in}{3.306729in}}{\pgfqpoint{9.593644in}{3.296130in}}{\pgfqpoint{9.601457in}{3.288316in}}%
\pgfpathcurveto{\pgfqpoint{9.609271in}{3.280502in}}{\pgfqpoint{9.619870in}{3.276112in}}{\pgfqpoint{9.630920in}{3.276112in}}%
\pgfpathlineto{\pgfqpoint{9.630920in}{3.276112in}}%
\pgfpathclose%
\pgfusepath{stroke}%
\end{pgfscope}%
\begin{pgfscope}%
\pgfpathrectangle{\pgfqpoint{7.512535in}{0.437222in}}{\pgfqpoint{6.275590in}{5.159444in}}%
\pgfusepath{clip}%
\pgfsetbuttcap%
\pgfsetroundjoin%
\pgfsetlinewidth{1.003750pt}%
\definecolor{currentstroke}{rgb}{0.827451,0.827451,0.827451}%
\pgfsetstrokecolor{currentstroke}%
\pgfsetstrokeopacity{0.800000}%
\pgfsetdash{}{0pt}%
\pgfpathmoveto{\pgfqpoint{12.287231in}{5.553091in}}%
\pgfpathcurveto{\pgfqpoint{12.298281in}{5.553091in}}{\pgfqpoint{12.308880in}{5.557481in}}{\pgfqpoint{12.316694in}{5.565295in}}%
\pgfpathcurveto{\pgfqpoint{12.324508in}{5.573108in}}{\pgfqpoint{12.328898in}{5.583707in}}{\pgfqpoint{12.328898in}{5.594757in}}%
\pgfpathcurveto{\pgfqpoint{12.328898in}{5.605808in}}{\pgfqpoint{12.324508in}{5.616407in}}{\pgfqpoint{12.316694in}{5.624220in}}%
\pgfpathcurveto{\pgfqpoint{12.308880in}{5.632034in}}{\pgfqpoint{12.298281in}{5.636424in}}{\pgfqpoint{12.287231in}{5.636424in}}%
\pgfpathcurveto{\pgfqpoint{12.276181in}{5.636424in}}{\pgfqpoint{12.265582in}{5.632034in}}{\pgfqpoint{12.257768in}{5.624220in}}%
\pgfpathcurveto{\pgfqpoint{12.249955in}{5.616407in}}{\pgfqpoint{12.245564in}{5.605808in}}{\pgfqpoint{12.245564in}{5.594757in}}%
\pgfpathcurveto{\pgfqpoint{12.245564in}{5.583707in}}{\pgfqpoint{12.249955in}{5.573108in}}{\pgfqpoint{12.257768in}{5.565295in}}%
\pgfpathcurveto{\pgfqpoint{12.265582in}{5.557481in}}{\pgfqpoint{12.276181in}{5.553091in}}{\pgfqpoint{12.287231in}{5.553091in}}%
\pgfpathlineto{\pgfqpoint{12.287231in}{5.553091in}}%
\pgfpathclose%
\pgfusepath{stroke}%
\end{pgfscope}%
\begin{pgfscope}%
\pgfpathrectangle{\pgfqpoint{7.512535in}{0.437222in}}{\pgfqpoint{6.275590in}{5.159444in}}%
\pgfusepath{clip}%
\pgfsetbuttcap%
\pgfsetroundjoin%
\pgfsetlinewidth{1.003750pt}%
\definecolor{currentstroke}{rgb}{0.827451,0.827451,0.827451}%
\pgfsetstrokecolor{currentstroke}%
\pgfsetstrokeopacity{0.800000}%
\pgfsetdash{}{0pt}%
\pgfpathmoveto{\pgfqpoint{7.562575in}{0.440656in}}%
\pgfpathcurveto{\pgfqpoint{7.573625in}{0.440656in}}{\pgfqpoint{7.584224in}{0.445046in}}{\pgfqpoint{7.592038in}{0.452860in}}%
\pgfpathcurveto{\pgfqpoint{7.599851in}{0.460673in}}{\pgfqpoint{7.604241in}{0.471272in}}{\pgfqpoint{7.604241in}{0.482323in}}%
\pgfpathcurveto{\pgfqpoint{7.604241in}{0.493373in}}{\pgfqpoint{7.599851in}{0.503972in}}{\pgfqpoint{7.592038in}{0.511785in}}%
\pgfpathcurveto{\pgfqpoint{7.584224in}{0.519599in}}{\pgfqpoint{7.573625in}{0.523989in}}{\pgfqpoint{7.562575in}{0.523989in}}%
\pgfpathcurveto{\pgfqpoint{7.551525in}{0.523989in}}{\pgfqpoint{7.540926in}{0.519599in}}{\pgfqpoint{7.533112in}{0.511785in}}%
\pgfpathcurveto{\pgfqpoint{7.525298in}{0.503972in}}{\pgfqpoint{7.520908in}{0.493373in}}{\pgfqpoint{7.520908in}{0.482323in}}%
\pgfpathcurveto{\pgfqpoint{7.520908in}{0.471272in}}{\pgfqpoint{7.525298in}{0.460673in}}{\pgfqpoint{7.533112in}{0.452860in}}%
\pgfpathcurveto{\pgfqpoint{7.540926in}{0.445046in}}{\pgfqpoint{7.551525in}{0.440656in}}{\pgfqpoint{7.562575in}{0.440656in}}%
\pgfpathlineto{\pgfqpoint{7.562575in}{0.440656in}}%
\pgfpathclose%
\pgfusepath{stroke}%
\end{pgfscope}%
\begin{pgfscope}%
\pgfpathrectangle{\pgfqpoint{7.512535in}{0.437222in}}{\pgfqpoint{6.275590in}{5.159444in}}%
\pgfusepath{clip}%
\pgfsetbuttcap%
\pgfsetroundjoin%
\pgfsetlinewidth{1.003750pt}%
\definecolor{currentstroke}{rgb}{0.827451,0.827451,0.827451}%
\pgfsetstrokecolor{currentstroke}%
\pgfsetstrokeopacity{0.800000}%
\pgfsetdash{}{0pt}%
\pgfpathmoveto{\pgfqpoint{13.725648in}{5.537440in}}%
\pgfpathcurveto{\pgfqpoint{13.736699in}{5.537440in}}{\pgfqpoint{13.747298in}{5.541830in}}{\pgfqpoint{13.755111in}{5.549643in}}%
\pgfpathcurveto{\pgfqpoint{13.762925in}{5.557457in}}{\pgfqpoint{13.767315in}{5.568056in}}{\pgfqpoint{13.767315in}{5.579106in}}%
\pgfpathcurveto{\pgfqpoint{13.767315in}{5.590156in}}{\pgfqpoint{13.762925in}{5.600755in}}{\pgfqpoint{13.755111in}{5.608569in}}%
\pgfpathcurveto{\pgfqpoint{13.747298in}{5.616383in}}{\pgfqpoint{13.736699in}{5.620773in}}{\pgfqpoint{13.725648in}{5.620773in}}%
\pgfpathcurveto{\pgfqpoint{13.714598in}{5.620773in}}{\pgfqpoint{13.703999in}{5.616383in}}{\pgfqpoint{13.696186in}{5.608569in}}%
\pgfpathcurveto{\pgfqpoint{13.688372in}{5.600755in}}{\pgfqpoint{13.683982in}{5.590156in}}{\pgfqpoint{13.683982in}{5.579106in}}%
\pgfpathcurveto{\pgfqpoint{13.683982in}{5.568056in}}{\pgfqpoint{13.688372in}{5.557457in}}{\pgfqpoint{13.696186in}{5.549643in}}%
\pgfpathcurveto{\pgfqpoint{13.703999in}{5.541830in}}{\pgfqpoint{13.714598in}{5.537440in}}{\pgfqpoint{13.725648in}{5.537440in}}%
\pgfpathlineto{\pgfqpoint{13.725648in}{5.537440in}}%
\pgfpathclose%
\pgfusepath{stroke}%
\end{pgfscope}%
\begin{pgfscope}%
\pgfpathrectangle{\pgfqpoint{7.512535in}{0.437222in}}{\pgfqpoint{6.275590in}{5.159444in}}%
\pgfusepath{clip}%
\pgfsetbuttcap%
\pgfsetroundjoin%
\pgfsetlinewidth{1.003750pt}%
\definecolor{currentstroke}{rgb}{0.827451,0.827451,0.827451}%
\pgfsetstrokecolor{currentstroke}%
\pgfsetstrokeopacity{0.800000}%
\pgfsetdash{}{0pt}%
\pgfpathmoveto{\pgfqpoint{13.581463in}{5.528743in}}%
\pgfpathcurveto{\pgfqpoint{13.592513in}{5.528743in}}{\pgfqpoint{13.603113in}{5.533133in}}{\pgfqpoint{13.610926in}{5.540947in}}%
\pgfpathcurveto{\pgfqpoint{13.618740in}{5.548760in}}{\pgfqpoint{13.623130in}{5.559359in}}{\pgfqpoint{13.623130in}{5.570410in}}%
\pgfpathcurveto{\pgfqpoint{13.623130in}{5.581460in}}{\pgfqpoint{13.618740in}{5.592059in}}{\pgfqpoint{13.610926in}{5.599872in}}%
\pgfpathcurveto{\pgfqpoint{13.603113in}{5.607686in}}{\pgfqpoint{13.592513in}{5.612076in}}{\pgfqpoint{13.581463in}{5.612076in}}%
\pgfpathcurveto{\pgfqpoint{13.570413in}{5.612076in}}{\pgfqpoint{13.559814in}{5.607686in}}{\pgfqpoint{13.552001in}{5.599872in}}%
\pgfpathcurveto{\pgfqpoint{13.544187in}{5.592059in}}{\pgfqpoint{13.539797in}{5.581460in}}{\pgfqpoint{13.539797in}{5.570410in}}%
\pgfpathcurveto{\pgfqpoint{13.539797in}{5.559359in}}{\pgfqpoint{13.544187in}{5.548760in}}{\pgfqpoint{13.552001in}{5.540947in}}%
\pgfpathcurveto{\pgfqpoint{13.559814in}{5.533133in}}{\pgfqpoint{13.570413in}{5.528743in}}{\pgfqpoint{13.581463in}{5.528743in}}%
\pgfpathlineto{\pgfqpoint{13.581463in}{5.528743in}}%
\pgfpathclose%
\pgfusepath{stroke}%
\end{pgfscope}%
\begin{pgfscope}%
\pgfpathrectangle{\pgfqpoint{7.512535in}{0.437222in}}{\pgfqpoint{6.275590in}{5.159444in}}%
\pgfusepath{clip}%
\pgfsetbuttcap%
\pgfsetroundjoin%
\pgfsetlinewidth{1.003750pt}%
\definecolor{currentstroke}{rgb}{0.827451,0.827451,0.827451}%
\pgfsetstrokecolor{currentstroke}%
\pgfsetstrokeopacity{0.800000}%
\pgfsetdash{}{0pt}%
\pgfpathmoveto{\pgfqpoint{13.273901in}{5.501584in}}%
\pgfpathcurveto{\pgfqpoint{13.284951in}{5.501584in}}{\pgfqpoint{13.295550in}{5.505974in}}{\pgfqpoint{13.303364in}{5.513787in}}%
\pgfpathcurveto{\pgfqpoint{13.311177in}{5.521601in}}{\pgfqpoint{13.315568in}{5.532200in}}{\pgfqpoint{13.315568in}{5.543250in}}%
\pgfpathcurveto{\pgfqpoint{13.315568in}{5.554300in}}{\pgfqpoint{13.311177in}{5.564899in}}{\pgfqpoint{13.303364in}{5.572713in}}%
\pgfpathcurveto{\pgfqpoint{13.295550in}{5.580527in}}{\pgfqpoint{13.284951in}{5.584917in}}{\pgfqpoint{13.273901in}{5.584917in}}%
\pgfpathcurveto{\pgfqpoint{13.262851in}{5.584917in}}{\pgfqpoint{13.252252in}{5.580527in}}{\pgfqpoint{13.244438in}{5.572713in}}%
\pgfpathcurveto{\pgfqpoint{13.236625in}{5.564899in}}{\pgfqpoint{13.232234in}{5.554300in}}{\pgfqpoint{13.232234in}{5.543250in}}%
\pgfpathcurveto{\pgfqpoint{13.232234in}{5.532200in}}{\pgfqpoint{13.236625in}{5.521601in}}{\pgfqpoint{13.244438in}{5.513787in}}%
\pgfpathcurveto{\pgfqpoint{13.252252in}{5.505974in}}{\pgfqpoint{13.262851in}{5.501584in}}{\pgfqpoint{13.273901in}{5.501584in}}%
\pgfpathlineto{\pgfqpoint{13.273901in}{5.501584in}}%
\pgfpathclose%
\pgfusepath{stroke}%
\end{pgfscope}%
\begin{pgfscope}%
\pgfpathrectangle{\pgfqpoint{7.512535in}{0.437222in}}{\pgfqpoint{6.275590in}{5.159444in}}%
\pgfusepath{clip}%
\pgfsetbuttcap%
\pgfsetroundjoin%
\pgfsetlinewidth{1.003750pt}%
\definecolor{currentstroke}{rgb}{0.827451,0.827451,0.827451}%
\pgfsetstrokecolor{currentstroke}%
\pgfsetstrokeopacity{0.800000}%
\pgfsetdash{}{0pt}%
\pgfpathmoveto{\pgfqpoint{11.673162in}{5.362410in}}%
\pgfpathcurveto{\pgfqpoint{11.684212in}{5.362410in}}{\pgfqpoint{11.694811in}{5.366800in}}{\pgfqpoint{11.702624in}{5.374614in}}%
\pgfpathcurveto{\pgfqpoint{11.710438in}{5.382427in}}{\pgfqpoint{11.714828in}{5.393027in}}{\pgfqpoint{11.714828in}{5.404077in}}%
\pgfpathcurveto{\pgfqpoint{11.714828in}{5.415127in}}{\pgfqpoint{11.710438in}{5.425726in}}{\pgfqpoint{11.702624in}{5.433539in}}%
\pgfpathcurveto{\pgfqpoint{11.694811in}{5.441353in}}{\pgfqpoint{11.684212in}{5.445743in}}{\pgfqpoint{11.673162in}{5.445743in}}%
\pgfpathcurveto{\pgfqpoint{11.662111in}{5.445743in}}{\pgfqpoint{11.651512in}{5.441353in}}{\pgfqpoint{11.643699in}{5.433539in}}%
\pgfpathcurveto{\pgfqpoint{11.635885in}{5.425726in}}{\pgfqpoint{11.631495in}{5.415127in}}{\pgfqpoint{11.631495in}{5.404077in}}%
\pgfpathcurveto{\pgfqpoint{11.631495in}{5.393027in}}{\pgfqpoint{11.635885in}{5.382427in}}{\pgfqpoint{11.643699in}{5.374614in}}%
\pgfpathcurveto{\pgfqpoint{11.651512in}{5.366800in}}{\pgfqpoint{11.662111in}{5.362410in}}{\pgfqpoint{11.673162in}{5.362410in}}%
\pgfpathlineto{\pgfqpoint{11.673162in}{5.362410in}}%
\pgfpathclose%
\pgfusepath{stroke}%
\end{pgfscope}%
\begin{pgfscope}%
\pgfpathrectangle{\pgfqpoint{7.512535in}{0.437222in}}{\pgfqpoint{6.275590in}{5.159444in}}%
\pgfusepath{clip}%
\pgfsetbuttcap%
\pgfsetroundjoin%
\pgfsetlinewidth{1.003750pt}%
\definecolor{currentstroke}{rgb}{0.827451,0.827451,0.827451}%
\pgfsetstrokecolor{currentstroke}%
\pgfsetstrokeopacity{0.800000}%
\pgfsetdash{}{0pt}%
\pgfpathmoveto{\pgfqpoint{12.815587in}{5.553845in}}%
\pgfpathcurveto{\pgfqpoint{12.826637in}{5.553845in}}{\pgfqpoint{12.837236in}{5.558236in}}{\pgfqpoint{12.845050in}{5.566049in}}%
\pgfpathcurveto{\pgfqpoint{12.852863in}{5.573863in}}{\pgfqpoint{12.857254in}{5.584462in}}{\pgfqpoint{12.857254in}{5.595512in}}%
\pgfpathcurveto{\pgfqpoint{12.857254in}{5.606562in}}{\pgfqpoint{12.852863in}{5.617161in}}{\pgfqpoint{12.845050in}{5.624975in}}%
\pgfpathcurveto{\pgfqpoint{12.837236in}{5.632788in}}{\pgfqpoint{12.826637in}{5.637179in}}{\pgfqpoint{12.815587in}{5.637179in}}%
\pgfpathcurveto{\pgfqpoint{12.804537in}{5.637179in}}{\pgfqpoint{12.793938in}{5.632788in}}{\pgfqpoint{12.786124in}{5.624975in}}%
\pgfpathcurveto{\pgfqpoint{12.778310in}{5.617161in}}{\pgfqpoint{12.773920in}{5.606562in}}{\pgfqpoint{12.773920in}{5.595512in}}%
\pgfpathcurveto{\pgfqpoint{12.773920in}{5.584462in}}{\pgfqpoint{12.778310in}{5.573863in}}{\pgfqpoint{12.786124in}{5.566049in}}%
\pgfpathcurveto{\pgfqpoint{12.793938in}{5.558236in}}{\pgfqpoint{12.804537in}{5.553845in}}{\pgfqpoint{12.815587in}{5.553845in}}%
\pgfpathlineto{\pgfqpoint{12.815587in}{5.553845in}}%
\pgfpathclose%
\pgfusepath{stroke}%
\end{pgfscope}%
\begin{pgfscope}%
\pgfpathrectangle{\pgfqpoint{7.512535in}{0.437222in}}{\pgfqpoint{6.275590in}{5.159444in}}%
\pgfusepath{clip}%
\pgfsetbuttcap%
\pgfsetroundjoin%
\pgfsetlinewidth{1.003750pt}%
\definecolor{currentstroke}{rgb}{0.827451,0.827451,0.827451}%
\pgfsetstrokecolor{currentstroke}%
\pgfsetstrokeopacity{0.800000}%
\pgfsetdash{}{0pt}%
\pgfpathmoveto{\pgfqpoint{13.101307in}{5.469388in}}%
\pgfpathcurveto{\pgfqpoint{13.112357in}{5.469388in}}{\pgfqpoint{13.122956in}{5.473778in}}{\pgfqpoint{13.130770in}{5.481592in}}%
\pgfpathcurveto{\pgfqpoint{13.138583in}{5.489405in}}{\pgfqpoint{13.142973in}{5.500004in}}{\pgfqpoint{13.142973in}{5.511054in}}%
\pgfpathcurveto{\pgfqpoint{13.142973in}{5.522105in}}{\pgfqpoint{13.138583in}{5.532704in}}{\pgfqpoint{13.130770in}{5.540517in}}%
\pgfpathcurveto{\pgfqpoint{13.122956in}{5.548331in}}{\pgfqpoint{13.112357in}{5.552721in}}{\pgfqpoint{13.101307in}{5.552721in}}%
\pgfpathcurveto{\pgfqpoint{13.090257in}{5.552721in}}{\pgfqpoint{13.079658in}{5.548331in}}{\pgfqpoint{13.071844in}{5.540517in}}%
\pgfpathcurveto{\pgfqpoint{13.064030in}{5.532704in}}{\pgfqpoint{13.059640in}{5.522105in}}{\pgfqpoint{13.059640in}{5.511054in}}%
\pgfpathcurveto{\pgfqpoint{13.059640in}{5.500004in}}{\pgfqpoint{13.064030in}{5.489405in}}{\pgfqpoint{13.071844in}{5.481592in}}%
\pgfpathcurveto{\pgfqpoint{13.079658in}{5.473778in}}{\pgfqpoint{13.090257in}{5.469388in}}{\pgfqpoint{13.101307in}{5.469388in}}%
\pgfpathlineto{\pgfqpoint{13.101307in}{5.469388in}}%
\pgfpathclose%
\pgfusepath{stroke}%
\end{pgfscope}%
\begin{pgfscope}%
\pgfpathrectangle{\pgfqpoint{7.512535in}{0.437222in}}{\pgfqpoint{6.275590in}{5.159444in}}%
\pgfusepath{clip}%
\pgfsetbuttcap%
\pgfsetroundjoin%
\pgfsetlinewidth{1.003750pt}%
\definecolor{currentstroke}{rgb}{0.827451,0.827451,0.827451}%
\pgfsetstrokecolor{currentstroke}%
\pgfsetstrokeopacity{0.800000}%
\pgfsetdash{}{0pt}%
\pgfpathmoveto{\pgfqpoint{11.806382in}{5.471994in}}%
\pgfpathcurveto{\pgfqpoint{11.817432in}{5.471994in}}{\pgfqpoint{11.828031in}{5.476384in}}{\pgfqpoint{11.835845in}{5.484198in}}%
\pgfpathcurveto{\pgfqpoint{11.843658in}{5.492011in}}{\pgfqpoint{11.848049in}{5.502610in}}{\pgfqpoint{11.848049in}{5.513660in}}%
\pgfpathcurveto{\pgfqpoint{11.848049in}{5.524711in}}{\pgfqpoint{11.843658in}{5.535310in}}{\pgfqpoint{11.835845in}{5.543123in}}%
\pgfpathcurveto{\pgfqpoint{11.828031in}{5.550937in}}{\pgfqpoint{11.817432in}{5.555327in}}{\pgfqpoint{11.806382in}{5.555327in}}%
\pgfpathcurveto{\pgfqpoint{11.795332in}{5.555327in}}{\pgfqpoint{11.784733in}{5.550937in}}{\pgfqpoint{11.776919in}{5.543123in}}%
\pgfpathcurveto{\pgfqpoint{11.769106in}{5.535310in}}{\pgfqpoint{11.764715in}{5.524711in}}{\pgfqpoint{11.764715in}{5.513660in}}%
\pgfpathcurveto{\pgfqpoint{11.764715in}{5.502610in}}{\pgfqpoint{11.769106in}{5.492011in}}{\pgfqpoint{11.776919in}{5.484198in}}%
\pgfpathcurveto{\pgfqpoint{11.784733in}{5.476384in}}{\pgfqpoint{11.795332in}{5.471994in}}{\pgfqpoint{11.806382in}{5.471994in}}%
\pgfpathlineto{\pgfqpoint{11.806382in}{5.471994in}}%
\pgfpathclose%
\pgfusepath{stroke}%
\end{pgfscope}%
\begin{pgfscope}%
\pgfpathrectangle{\pgfqpoint{7.512535in}{0.437222in}}{\pgfqpoint{6.275590in}{5.159444in}}%
\pgfusepath{clip}%
\pgfsetbuttcap%
\pgfsetroundjoin%
\pgfsetlinewidth{1.003750pt}%
\definecolor{currentstroke}{rgb}{0.827451,0.827451,0.827451}%
\pgfsetstrokecolor{currentstroke}%
\pgfsetstrokeopacity{0.800000}%
\pgfsetdash{}{0pt}%
\pgfpathmoveto{\pgfqpoint{8.044541in}{2.113663in}}%
\pgfpathcurveto{\pgfqpoint{8.055591in}{2.113663in}}{\pgfqpoint{8.066190in}{2.118054in}}{\pgfqpoint{8.074003in}{2.125867in}}%
\pgfpathcurveto{\pgfqpoint{8.081817in}{2.133681in}}{\pgfqpoint{8.086207in}{2.144280in}}{\pgfqpoint{8.086207in}{2.155330in}}%
\pgfpathcurveto{\pgfqpoint{8.086207in}{2.166380in}}{\pgfqpoint{8.081817in}{2.176979in}}{\pgfqpoint{8.074003in}{2.184793in}}%
\pgfpathcurveto{\pgfqpoint{8.066190in}{2.192606in}}{\pgfqpoint{8.055591in}{2.196997in}}{\pgfqpoint{8.044541in}{2.196997in}}%
\pgfpathcurveto{\pgfqpoint{8.033491in}{2.196997in}}{\pgfqpoint{8.022891in}{2.192606in}}{\pgfqpoint{8.015078in}{2.184793in}}%
\pgfpathcurveto{\pgfqpoint{8.007264in}{2.176979in}}{\pgfqpoint{8.002874in}{2.166380in}}{\pgfqpoint{8.002874in}{2.155330in}}%
\pgfpathcurveto{\pgfqpoint{8.002874in}{2.144280in}}{\pgfqpoint{8.007264in}{2.133681in}}{\pgfqpoint{8.015078in}{2.125867in}}%
\pgfpathcurveto{\pgfqpoint{8.022891in}{2.118054in}}{\pgfqpoint{8.033491in}{2.113663in}}{\pgfqpoint{8.044541in}{2.113663in}}%
\pgfpathlineto{\pgfqpoint{8.044541in}{2.113663in}}%
\pgfpathclose%
\pgfusepath{stroke}%
\end{pgfscope}%
\begin{pgfscope}%
\pgfpathrectangle{\pgfqpoint{7.512535in}{0.437222in}}{\pgfqpoint{6.275590in}{5.159444in}}%
\pgfusepath{clip}%
\pgfsetbuttcap%
\pgfsetroundjoin%
\pgfsetlinewidth{1.003750pt}%
\definecolor{currentstroke}{rgb}{0.827451,0.827451,0.827451}%
\pgfsetstrokecolor{currentstroke}%
\pgfsetstrokeopacity{0.800000}%
\pgfsetdash{}{0pt}%
\pgfpathmoveto{\pgfqpoint{10.873384in}{4.145170in}}%
\pgfpathcurveto{\pgfqpoint{10.884434in}{4.145170in}}{\pgfqpoint{10.895033in}{4.149560in}}{\pgfqpoint{10.902847in}{4.157374in}}%
\pgfpathcurveto{\pgfqpoint{10.910660in}{4.165188in}}{\pgfqpoint{10.915050in}{4.175787in}}{\pgfqpoint{10.915050in}{4.186837in}}%
\pgfpathcurveto{\pgfqpoint{10.915050in}{4.197887in}}{\pgfqpoint{10.910660in}{4.208486in}}{\pgfqpoint{10.902847in}{4.216300in}}%
\pgfpathcurveto{\pgfqpoint{10.895033in}{4.224113in}}{\pgfqpoint{10.884434in}{4.228504in}}{\pgfqpoint{10.873384in}{4.228504in}}%
\pgfpathcurveto{\pgfqpoint{10.862334in}{4.228504in}}{\pgfqpoint{10.851735in}{4.224113in}}{\pgfqpoint{10.843921in}{4.216300in}}%
\pgfpathcurveto{\pgfqpoint{10.836107in}{4.208486in}}{\pgfqpoint{10.831717in}{4.197887in}}{\pgfqpoint{10.831717in}{4.186837in}}%
\pgfpathcurveto{\pgfqpoint{10.831717in}{4.175787in}}{\pgfqpoint{10.836107in}{4.165188in}}{\pgfqpoint{10.843921in}{4.157374in}}%
\pgfpathcurveto{\pgfqpoint{10.851735in}{4.149560in}}{\pgfqpoint{10.862334in}{4.145170in}}{\pgfqpoint{10.873384in}{4.145170in}}%
\pgfpathlineto{\pgfqpoint{10.873384in}{4.145170in}}%
\pgfpathclose%
\pgfusepath{stroke}%
\end{pgfscope}%
\begin{pgfscope}%
\pgfpathrectangle{\pgfqpoint{7.512535in}{0.437222in}}{\pgfqpoint{6.275590in}{5.159444in}}%
\pgfusepath{clip}%
\pgfsetbuttcap%
\pgfsetroundjoin%
\pgfsetlinewidth{1.003750pt}%
\definecolor{currentstroke}{rgb}{0.827451,0.827451,0.827451}%
\pgfsetstrokecolor{currentstroke}%
\pgfsetstrokeopacity{0.800000}%
\pgfsetdash{}{0pt}%
\pgfpathmoveto{\pgfqpoint{12.650243in}{5.481847in}}%
\pgfpathcurveto{\pgfqpoint{12.661293in}{5.481847in}}{\pgfqpoint{12.671892in}{5.486238in}}{\pgfqpoint{12.679706in}{5.494051in}}%
\pgfpathcurveto{\pgfqpoint{12.687519in}{5.501865in}}{\pgfqpoint{12.691909in}{5.512464in}}{\pgfqpoint{12.691909in}{5.523514in}}%
\pgfpathcurveto{\pgfqpoint{12.691909in}{5.534564in}}{\pgfqpoint{12.687519in}{5.545163in}}{\pgfqpoint{12.679706in}{5.552977in}}%
\pgfpathcurveto{\pgfqpoint{12.671892in}{5.560791in}}{\pgfqpoint{12.661293in}{5.565181in}}{\pgfqpoint{12.650243in}{5.565181in}}%
\pgfpathcurveto{\pgfqpoint{12.639193in}{5.565181in}}{\pgfqpoint{12.628594in}{5.560791in}}{\pgfqpoint{12.620780in}{5.552977in}}%
\pgfpathcurveto{\pgfqpoint{12.612966in}{5.545163in}}{\pgfqpoint{12.608576in}{5.534564in}}{\pgfqpoint{12.608576in}{5.523514in}}%
\pgfpathcurveto{\pgfqpoint{12.608576in}{5.512464in}}{\pgfqpoint{12.612966in}{5.501865in}}{\pgfqpoint{12.620780in}{5.494051in}}%
\pgfpathcurveto{\pgfqpoint{12.628594in}{5.486238in}}{\pgfqpoint{12.639193in}{5.481847in}}{\pgfqpoint{12.650243in}{5.481847in}}%
\pgfpathlineto{\pgfqpoint{12.650243in}{5.481847in}}%
\pgfpathclose%
\pgfusepath{stroke}%
\end{pgfscope}%
\begin{pgfscope}%
\pgfpathrectangle{\pgfqpoint{7.512535in}{0.437222in}}{\pgfqpoint{6.275590in}{5.159444in}}%
\pgfusepath{clip}%
\pgfsetbuttcap%
\pgfsetroundjoin%
\pgfsetlinewidth{1.003750pt}%
\definecolor{currentstroke}{rgb}{0.827451,0.827451,0.827451}%
\pgfsetstrokecolor{currentstroke}%
\pgfsetstrokeopacity{0.800000}%
\pgfsetdash{}{0pt}%
\pgfpathmoveto{\pgfqpoint{8.277312in}{1.563305in}}%
\pgfpathcurveto{\pgfqpoint{8.288363in}{1.563305in}}{\pgfqpoint{8.298962in}{1.567695in}}{\pgfqpoint{8.306775in}{1.575509in}}%
\pgfpathcurveto{\pgfqpoint{8.314589in}{1.583322in}}{\pgfqpoint{8.318979in}{1.593921in}}{\pgfqpoint{8.318979in}{1.604971in}}%
\pgfpathcurveto{\pgfqpoint{8.318979in}{1.616021in}}{\pgfqpoint{8.314589in}{1.626621in}}{\pgfqpoint{8.306775in}{1.634434in}}%
\pgfpathcurveto{\pgfqpoint{8.298962in}{1.642248in}}{\pgfqpoint{8.288363in}{1.646638in}}{\pgfqpoint{8.277312in}{1.646638in}}%
\pgfpathcurveto{\pgfqpoint{8.266262in}{1.646638in}}{\pgfqpoint{8.255663in}{1.642248in}}{\pgfqpoint{8.247850in}{1.634434in}}%
\pgfpathcurveto{\pgfqpoint{8.240036in}{1.626621in}}{\pgfqpoint{8.235646in}{1.616021in}}{\pgfqpoint{8.235646in}{1.604971in}}%
\pgfpathcurveto{\pgfqpoint{8.235646in}{1.593921in}}{\pgfqpoint{8.240036in}{1.583322in}}{\pgfqpoint{8.247850in}{1.575509in}}%
\pgfpathcurveto{\pgfqpoint{8.255663in}{1.567695in}}{\pgfqpoint{8.266262in}{1.563305in}}{\pgfqpoint{8.277312in}{1.563305in}}%
\pgfpathlineto{\pgfqpoint{8.277312in}{1.563305in}}%
\pgfpathclose%
\pgfusepath{stroke}%
\end{pgfscope}%
\begin{pgfscope}%
\pgfpathrectangle{\pgfqpoint{7.512535in}{0.437222in}}{\pgfqpoint{6.275590in}{5.159444in}}%
\pgfusepath{clip}%
\pgfsetbuttcap%
\pgfsetroundjoin%
\pgfsetlinewidth{1.003750pt}%
\definecolor{currentstroke}{rgb}{0.827451,0.827451,0.827451}%
\pgfsetstrokecolor{currentstroke}%
\pgfsetstrokeopacity{0.800000}%
\pgfsetdash{}{0pt}%
\pgfpathmoveto{\pgfqpoint{9.848678in}{4.313606in}}%
\pgfpathcurveto{\pgfqpoint{9.859728in}{4.313606in}}{\pgfqpoint{9.870327in}{4.317996in}}{\pgfqpoint{9.878141in}{4.325810in}}%
\pgfpathcurveto{\pgfqpoint{9.885955in}{4.333623in}}{\pgfqpoint{9.890345in}{4.344222in}}{\pgfqpoint{9.890345in}{4.355272in}}%
\pgfpathcurveto{\pgfqpoint{9.890345in}{4.366323in}}{\pgfqpoint{9.885955in}{4.376922in}}{\pgfqpoint{9.878141in}{4.384735in}}%
\pgfpathcurveto{\pgfqpoint{9.870327in}{4.392549in}}{\pgfqpoint{9.859728in}{4.396939in}}{\pgfqpoint{9.848678in}{4.396939in}}%
\pgfpathcurveto{\pgfqpoint{9.837628in}{4.396939in}}{\pgfqpoint{9.827029in}{4.392549in}}{\pgfqpoint{9.819216in}{4.384735in}}%
\pgfpathcurveto{\pgfqpoint{9.811402in}{4.376922in}}{\pgfqpoint{9.807012in}{4.366323in}}{\pgfqpoint{9.807012in}{4.355272in}}%
\pgfpathcurveto{\pgfqpoint{9.807012in}{4.344222in}}{\pgfqpoint{9.811402in}{4.333623in}}{\pgfqpoint{9.819216in}{4.325810in}}%
\pgfpathcurveto{\pgfqpoint{9.827029in}{4.317996in}}{\pgfqpoint{9.837628in}{4.313606in}}{\pgfqpoint{9.848678in}{4.313606in}}%
\pgfpathlineto{\pgfqpoint{9.848678in}{4.313606in}}%
\pgfpathclose%
\pgfusepath{stroke}%
\end{pgfscope}%
\begin{pgfscope}%
\pgfpathrectangle{\pgfqpoint{7.512535in}{0.437222in}}{\pgfqpoint{6.275590in}{5.159444in}}%
\pgfusepath{clip}%
\pgfsetbuttcap%
\pgfsetroundjoin%
\pgfsetlinewidth{1.003750pt}%
\definecolor{currentstroke}{rgb}{0.827451,0.827451,0.827451}%
\pgfsetstrokecolor{currentstroke}%
\pgfsetstrokeopacity{0.800000}%
\pgfsetdash{}{0pt}%
\pgfpathmoveto{\pgfqpoint{7.983939in}{1.000624in}}%
\pgfpathcurveto{\pgfqpoint{7.994989in}{1.000624in}}{\pgfqpoint{8.005588in}{1.005015in}}{\pgfqpoint{8.013401in}{1.012828in}}%
\pgfpathcurveto{\pgfqpoint{8.021215in}{1.020642in}}{\pgfqpoint{8.025605in}{1.031241in}}{\pgfqpoint{8.025605in}{1.042291in}}%
\pgfpathcurveto{\pgfqpoint{8.025605in}{1.053341in}}{\pgfqpoint{8.021215in}{1.063940in}}{\pgfqpoint{8.013401in}{1.071754in}}%
\pgfpathcurveto{\pgfqpoint{8.005588in}{1.079568in}}{\pgfqpoint{7.994989in}{1.083958in}}{\pgfqpoint{7.983939in}{1.083958in}}%
\pgfpathcurveto{\pgfqpoint{7.972888in}{1.083958in}}{\pgfqpoint{7.962289in}{1.079568in}}{\pgfqpoint{7.954476in}{1.071754in}}%
\pgfpathcurveto{\pgfqpoint{7.946662in}{1.063940in}}{\pgfqpoint{7.942272in}{1.053341in}}{\pgfqpoint{7.942272in}{1.042291in}}%
\pgfpathcurveto{\pgfqpoint{7.942272in}{1.031241in}}{\pgfqpoint{7.946662in}{1.020642in}}{\pgfqpoint{7.954476in}{1.012828in}}%
\pgfpathcurveto{\pgfqpoint{7.962289in}{1.005015in}}{\pgfqpoint{7.972888in}{1.000624in}}{\pgfqpoint{7.983939in}{1.000624in}}%
\pgfpathlineto{\pgfqpoint{7.983939in}{1.000624in}}%
\pgfpathclose%
\pgfusepath{stroke}%
\end{pgfscope}%
\begin{pgfscope}%
\pgfpathrectangle{\pgfqpoint{7.512535in}{0.437222in}}{\pgfqpoint{6.275590in}{5.159444in}}%
\pgfusepath{clip}%
\pgfsetbuttcap%
\pgfsetroundjoin%
\pgfsetlinewidth{1.003750pt}%
\definecolor{currentstroke}{rgb}{0.827451,0.827451,0.827451}%
\pgfsetstrokecolor{currentstroke}%
\pgfsetstrokeopacity{0.800000}%
\pgfsetdash{}{0pt}%
\pgfpathmoveto{\pgfqpoint{9.807010in}{4.531380in}}%
\pgfpathcurveto{\pgfqpoint{9.818060in}{4.531380in}}{\pgfqpoint{9.828659in}{4.535770in}}{\pgfqpoint{9.836473in}{4.543584in}}%
\pgfpathcurveto{\pgfqpoint{9.844287in}{4.551397in}}{\pgfqpoint{9.848677in}{4.561996in}}{\pgfqpoint{9.848677in}{4.573046in}}%
\pgfpathcurveto{\pgfqpoint{9.848677in}{4.584096in}}{\pgfqpoint{9.844287in}{4.594695in}}{\pgfqpoint{9.836473in}{4.602509in}}%
\pgfpathcurveto{\pgfqpoint{9.828659in}{4.610323in}}{\pgfqpoint{9.818060in}{4.614713in}}{\pgfqpoint{9.807010in}{4.614713in}}%
\pgfpathcurveto{\pgfqpoint{9.795960in}{4.614713in}}{\pgfqpoint{9.785361in}{4.610323in}}{\pgfqpoint{9.777548in}{4.602509in}}%
\pgfpathcurveto{\pgfqpoint{9.769734in}{4.594695in}}{\pgfqpoint{9.765344in}{4.584096in}}{\pgfqpoint{9.765344in}{4.573046in}}%
\pgfpathcurveto{\pgfqpoint{9.765344in}{4.561996in}}{\pgfqpoint{9.769734in}{4.551397in}}{\pgfqpoint{9.777548in}{4.543584in}}%
\pgfpathcurveto{\pgfqpoint{9.785361in}{4.535770in}}{\pgfqpoint{9.795960in}{4.531380in}}{\pgfqpoint{9.807010in}{4.531380in}}%
\pgfpathlineto{\pgfqpoint{9.807010in}{4.531380in}}%
\pgfpathclose%
\pgfusepath{stroke}%
\end{pgfscope}%
\begin{pgfscope}%
\pgfpathrectangle{\pgfqpoint{7.512535in}{0.437222in}}{\pgfqpoint{6.275590in}{5.159444in}}%
\pgfusepath{clip}%
\pgfsetbuttcap%
\pgfsetroundjoin%
\pgfsetlinewidth{1.003750pt}%
\definecolor{currentstroke}{rgb}{0.827451,0.827451,0.827451}%
\pgfsetstrokecolor{currentstroke}%
\pgfsetstrokeopacity{0.800000}%
\pgfsetdash{}{0pt}%
\pgfpathmoveto{\pgfqpoint{8.896008in}{2.422402in}}%
\pgfpathcurveto{\pgfqpoint{8.907058in}{2.422402in}}{\pgfqpoint{8.917657in}{2.426793in}}{\pgfqpoint{8.925471in}{2.434606in}}%
\pgfpathcurveto{\pgfqpoint{8.933285in}{2.442420in}}{\pgfqpoint{8.937675in}{2.453019in}}{\pgfqpoint{8.937675in}{2.464069in}}%
\pgfpathcurveto{\pgfqpoint{8.937675in}{2.475119in}}{\pgfqpoint{8.933285in}{2.485718in}}{\pgfqpoint{8.925471in}{2.493532in}}%
\pgfpathcurveto{\pgfqpoint{8.917657in}{2.501345in}}{\pgfqpoint{8.907058in}{2.505736in}}{\pgfqpoint{8.896008in}{2.505736in}}%
\pgfpathcurveto{\pgfqpoint{8.884958in}{2.505736in}}{\pgfqpoint{8.874359in}{2.501345in}}{\pgfqpoint{8.866545in}{2.493532in}}%
\pgfpathcurveto{\pgfqpoint{8.858732in}{2.485718in}}{\pgfqpoint{8.854341in}{2.475119in}}{\pgfqpoint{8.854341in}{2.464069in}}%
\pgfpathcurveto{\pgfqpoint{8.854341in}{2.453019in}}{\pgfqpoint{8.858732in}{2.442420in}}{\pgfqpoint{8.866545in}{2.434606in}}%
\pgfpathcurveto{\pgfqpoint{8.874359in}{2.426793in}}{\pgfqpoint{8.884958in}{2.422402in}}{\pgfqpoint{8.896008in}{2.422402in}}%
\pgfpathlineto{\pgfqpoint{8.896008in}{2.422402in}}%
\pgfpathclose%
\pgfusepath{stroke}%
\end{pgfscope}%
\begin{pgfscope}%
\pgfpathrectangle{\pgfqpoint{7.512535in}{0.437222in}}{\pgfqpoint{6.275590in}{5.159444in}}%
\pgfusepath{clip}%
\pgfsetbuttcap%
\pgfsetroundjoin%
\pgfsetlinewidth{1.003750pt}%
\definecolor{currentstroke}{rgb}{0.827451,0.827451,0.827451}%
\pgfsetstrokecolor{currentstroke}%
\pgfsetstrokeopacity{0.800000}%
\pgfsetdash{}{0pt}%
\pgfpathmoveto{\pgfqpoint{7.860715in}{0.988624in}}%
\pgfpathcurveto{\pgfqpoint{7.871765in}{0.988624in}}{\pgfqpoint{7.882364in}{0.993014in}}{\pgfqpoint{7.890178in}{1.000828in}}%
\pgfpathcurveto{\pgfqpoint{7.897991in}{1.008641in}}{\pgfqpoint{7.902381in}{1.019240in}}{\pgfqpoint{7.902381in}{1.030290in}}%
\pgfpathcurveto{\pgfqpoint{7.902381in}{1.041341in}}{\pgfqpoint{7.897991in}{1.051940in}}{\pgfqpoint{7.890178in}{1.059753in}}%
\pgfpathcurveto{\pgfqpoint{7.882364in}{1.067567in}}{\pgfqpoint{7.871765in}{1.071957in}}{\pgfqpoint{7.860715in}{1.071957in}}%
\pgfpathcurveto{\pgfqpoint{7.849665in}{1.071957in}}{\pgfqpoint{7.839066in}{1.067567in}}{\pgfqpoint{7.831252in}{1.059753in}}%
\pgfpathcurveto{\pgfqpoint{7.823438in}{1.051940in}}{\pgfqpoint{7.819048in}{1.041341in}}{\pgfqpoint{7.819048in}{1.030290in}}%
\pgfpathcurveto{\pgfqpoint{7.819048in}{1.019240in}}{\pgfqpoint{7.823438in}{1.008641in}}{\pgfqpoint{7.831252in}{1.000828in}}%
\pgfpathcurveto{\pgfqpoint{7.839066in}{0.993014in}}{\pgfqpoint{7.849665in}{0.988624in}}{\pgfqpoint{7.860715in}{0.988624in}}%
\pgfpathlineto{\pgfqpoint{7.860715in}{0.988624in}}%
\pgfpathclose%
\pgfusepath{stroke}%
\end{pgfscope}%
\begin{pgfscope}%
\pgfpathrectangle{\pgfqpoint{7.512535in}{0.437222in}}{\pgfqpoint{6.275590in}{5.159444in}}%
\pgfusepath{clip}%
\pgfsetbuttcap%
\pgfsetroundjoin%
\pgfsetlinewidth{1.003750pt}%
\definecolor{currentstroke}{rgb}{0.827451,0.827451,0.827451}%
\pgfsetstrokecolor{currentstroke}%
\pgfsetstrokeopacity{0.800000}%
\pgfsetdash{}{0pt}%
\pgfpathmoveto{\pgfqpoint{9.319291in}{1.927959in}}%
\pgfpathcurveto{\pgfqpoint{9.330341in}{1.927959in}}{\pgfqpoint{9.340940in}{1.932349in}}{\pgfqpoint{9.348754in}{1.940163in}}%
\pgfpathcurveto{\pgfqpoint{9.356567in}{1.947976in}}{\pgfqpoint{9.360957in}{1.958575in}}{\pgfqpoint{9.360957in}{1.969626in}}%
\pgfpathcurveto{\pgfqpoint{9.360957in}{1.980676in}}{\pgfqpoint{9.356567in}{1.991275in}}{\pgfqpoint{9.348754in}{1.999088in}}%
\pgfpathcurveto{\pgfqpoint{9.340940in}{2.006902in}}{\pgfqpoint{9.330341in}{2.011292in}}{\pgfqpoint{9.319291in}{2.011292in}}%
\pgfpathcurveto{\pgfqpoint{9.308241in}{2.011292in}}{\pgfqpoint{9.297642in}{2.006902in}}{\pgfqpoint{9.289828in}{1.999088in}}%
\pgfpathcurveto{\pgfqpoint{9.282014in}{1.991275in}}{\pgfqpoint{9.277624in}{1.980676in}}{\pgfqpoint{9.277624in}{1.969626in}}%
\pgfpathcurveto{\pgfqpoint{9.277624in}{1.958575in}}{\pgfqpoint{9.282014in}{1.947976in}}{\pgfqpoint{9.289828in}{1.940163in}}%
\pgfpathcurveto{\pgfqpoint{9.297642in}{1.932349in}}{\pgfqpoint{9.308241in}{1.927959in}}{\pgfqpoint{9.319291in}{1.927959in}}%
\pgfpathlineto{\pgfqpoint{9.319291in}{1.927959in}}%
\pgfpathclose%
\pgfusepath{stroke}%
\end{pgfscope}%
\begin{pgfscope}%
\pgfpathrectangle{\pgfqpoint{7.512535in}{0.437222in}}{\pgfqpoint{6.275590in}{5.159444in}}%
\pgfusepath{clip}%
\pgfsetbuttcap%
\pgfsetroundjoin%
\pgfsetlinewidth{1.003750pt}%
\definecolor{currentstroke}{rgb}{0.827451,0.827451,0.827451}%
\pgfsetstrokecolor{currentstroke}%
\pgfsetstrokeopacity{0.800000}%
\pgfsetdash{}{0pt}%
\pgfpathmoveto{\pgfqpoint{9.130854in}{1.867691in}}%
\pgfpathcurveto{\pgfqpoint{9.141904in}{1.867691in}}{\pgfqpoint{9.152503in}{1.872081in}}{\pgfqpoint{9.160317in}{1.879895in}}%
\pgfpathcurveto{\pgfqpoint{9.168130in}{1.887709in}}{\pgfqpoint{9.172521in}{1.898308in}}{\pgfqpoint{9.172521in}{1.909358in}}%
\pgfpathcurveto{\pgfqpoint{9.172521in}{1.920408in}}{\pgfqpoint{9.168130in}{1.931007in}}{\pgfqpoint{9.160317in}{1.938821in}}%
\pgfpathcurveto{\pgfqpoint{9.152503in}{1.946634in}}{\pgfqpoint{9.141904in}{1.951024in}}{\pgfqpoint{9.130854in}{1.951024in}}%
\pgfpathcurveto{\pgfqpoint{9.119804in}{1.951024in}}{\pgfqpoint{9.109205in}{1.946634in}}{\pgfqpoint{9.101391in}{1.938821in}}%
\pgfpathcurveto{\pgfqpoint{9.093577in}{1.931007in}}{\pgfqpoint{9.089187in}{1.920408in}}{\pgfqpoint{9.089187in}{1.909358in}}%
\pgfpathcurveto{\pgfqpoint{9.089187in}{1.898308in}}{\pgfqpoint{9.093577in}{1.887709in}}{\pgfqpoint{9.101391in}{1.879895in}}%
\pgfpathcurveto{\pgfqpoint{9.109205in}{1.872081in}}{\pgfqpoint{9.119804in}{1.867691in}}{\pgfqpoint{9.130854in}{1.867691in}}%
\pgfpathlineto{\pgfqpoint{9.130854in}{1.867691in}}%
\pgfpathclose%
\pgfusepath{stroke}%
\end{pgfscope}%
\begin{pgfscope}%
\pgfpathrectangle{\pgfqpoint{7.512535in}{0.437222in}}{\pgfqpoint{6.275590in}{5.159444in}}%
\pgfusepath{clip}%
\pgfsetbuttcap%
\pgfsetroundjoin%
\pgfsetlinewidth{1.003750pt}%
\definecolor{currentstroke}{rgb}{0.827451,0.827451,0.827451}%
\pgfsetstrokecolor{currentstroke}%
\pgfsetstrokeopacity{0.800000}%
\pgfsetdash{}{0pt}%
\pgfpathmoveto{\pgfqpoint{12.737963in}{5.481847in}}%
\pgfpathcurveto{\pgfqpoint{12.749014in}{5.481847in}}{\pgfqpoint{12.759613in}{5.486238in}}{\pgfqpoint{12.767426in}{5.494051in}}%
\pgfpathcurveto{\pgfqpoint{12.775240in}{5.501865in}}{\pgfqpoint{12.779630in}{5.512464in}}{\pgfqpoint{12.779630in}{5.523514in}}%
\pgfpathcurveto{\pgfqpoint{12.779630in}{5.534564in}}{\pgfqpoint{12.775240in}{5.545163in}}{\pgfqpoint{12.767426in}{5.552977in}}%
\pgfpathcurveto{\pgfqpoint{12.759613in}{5.560791in}}{\pgfqpoint{12.749014in}{5.565181in}}{\pgfqpoint{12.737963in}{5.565181in}}%
\pgfpathcurveto{\pgfqpoint{12.726913in}{5.565181in}}{\pgfqpoint{12.716314in}{5.560791in}}{\pgfqpoint{12.708501in}{5.552977in}}%
\pgfpathcurveto{\pgfqpoint{12.700687in}{5.545163in}}{\pgfqpoint{12.696297in}{5.534564in}}{\pgfqpoint{12.696297in}{5.523514in}}%
\pgfpathcurveto{\pgfqpoint{12.696297in}{5.512464in}}{\pgfqpoint{12.700687in}{5.501865in}}{\pgfqpoint{12.708501in}{5.494051in}}%
\pgfpathcurveto{\pgfqpoint{12.716314in}{5.486238in}}{\pgfqpoint{12.726913in}{5.481847in}}{\pgfqpoint{12.737963in}{5.481847in}}%
\pgfpathlineto{\pgfqpoint{12.737963in}{5.481847in}}%
\pgfpathclose%
\pgfusepath{stroke}%
\end{pgfscope}%
\begin{pgfscope}%
\pgfpathrectangle{\pgfqpoint{7.512535in}{0.437222in}}{\pgfqpoint{6.275590in}{5.159444in}}%
\pgfusepath{clip}%
\pgfsetbuttcap%
\pgfsetroundjoin%
\pgfsetlinewidth{1.003750pt}%
\definecolor{currentstroke}{rgb}{0.827451,0.827451,0.827451}%
\pgfsetstrokecolor{currentstroke}%
\pgfsetstrokeopacity{0.800000}%
\pgfsetdash{}{0pt}%
\pgfpathmoveto{\pgfqpoint{10.023823in}{4.500549in}}%
\pgfpathcurveto{\pgfqpoint{10.034873in}{4.500549in}}{\pgfqpoint{10.045472in}{4.504939in}}{\pgfqpoint{10.053285in}{4.512753in}}%
\pgfpathcurveto{\pgfqpoint{10.061099in}{4.520566in}}{\pgfqpoint{10.065489in}{4.531165in}}{\pgfqpoint{10.065489in}{4.542215in}}%
\pgfpathcurveto{\pgfqpoint{10.065489in}{4.553266in}}{\pgfqpoint{10.061099in}{4.563865in}}{\pgfqpoint{10.053285in}{4.571678in}}%
\pgfpathcurveto{\pgfqpoint{10.045472in}{4.579492in}}{\pgfqpoint{10.034873in}{4.583882in}}{\pgfqpoint{10.023823in}{4.583882in}}%
\pgfpathcurveto{\pgfqpoint{10.012772in}{4.583882in}}{\pgfqpoint{10.002173in}{4.579492in}}{\pgfqpoint{9.994360in}{4.571678in}}%
\pgfpathcurveto{\pgfqpoint{9.986546in}{4.563865in}}{\pgfqpoint{9.982156in}{4.553266in}}{\pgfqpoint{9.982156in}{4.542215in}}%
\pgfpathcurveto{\pgfqpoint{9.982156in}{4.531165in}}{\pgfqpoint{9.986546in}{4.520566in}}{\pgfqpoint{9.994360in}{4.512753in}}%
\pgfpathcurveto{\pgfqpoint{10.002173in}{4.504939in}}{\pgfqpoint{10.012772in}{4.500549in}}{\pgfqpoint{10.023823in}{4.500549in}}%
\pgfpathlineto{\pgfqpoint{10.023823in}{4.500549in}}%
\pgfpathclose%
\pgfusepath{stroke}%
\end{pgfscope}%
\begin{pgfscope}%
\pgfpathrectangle{\pgfqpoint{7.512535in}{0.437222in}}{\pgfqpoint{6.275590in}{5.159444in}}%
\pgfusepath{clip}%
\pgfsetbuttcap%
\pgfsetroundjoin%
\pgfsetlinewidth{1.003750pt}%
\definecolor{currentstroke}{rgb}{0.827451,0.827451,0.827451}%
\pgfsetstrokecolor{currentstroke}%
\pgfsetstrokeopacity{0.800000}%
\pgfsetdash{}{0pt}%
\pgfpathmoveto{\pgfqpoint{10.712158in}{4.601092in}}%
\pgfpathcurveto{\pgfqpoint{10.723208in}{4.601092in}}{\pgfqpoint{10.733807in}{4.605482in}}{\pgfqpoint{10.741621in}{4.613296in}}%
\pgfpathcurveto{\pgfqpoint{10.749435in}{4.621109in}}{\pgfqpoint{10.753825in}{4.631708in}}{\pgfqpoint{10.753825in}{4.642759in}}%
\pgfpathcurveto{\pgfqpoint{10.753825in}{4.653809in}}{\pgfqpoint{10.749435in}{4.664408in}}{\pgfqpoint{10.741621in}{4.672221in}}%
\pgfpathcurveto{\pgfqpoint{10.733807in}{4.680035in}}{\pgfqpoint{10.723208in}{4.684425in}}{\pgfqpoint{10.712158in}{4.684425in}}%
\pgfpathcurveto{\pgfqpoint{10.701108in}{4.684425in}}{\pgfqpoint{10.690509in}{4.680035in}}{\pgfqpoint{10.682695in}{4.672221in}}%
\pgfpathcurveto{\pgfqpoint{10.674882in}{4.664408in}}{\pgfqpoint{10.670492in}{4.653809in}}{\pgfqpoint{10.670492in}{4.642759in}}%
\pgfpathcurveto{\pgfqpoint{10.670492in}{4.631708in}}{\pgfqpoint{10.674882in}{4.621109in}}{\pgfqpoint{10.682695in}{4.613296in}}%
\pgfpathcurveto{\pgfqpoint{10.690509in}{4.605482in}}{\pgfqpoint{10.701108in}{4.601092in}}{\pgfqpoint{10.712158in}{4.601092in}}%
\pgfpathlineto{\pgfqpoint{10.712158in}{4.601092in}}%
\pgfpathclose%
\pgfusepath{stroke}%
\end{pgfscope}%
\begin{pgfscope}%
\pgfpathrectangle{\pgfqpoint{7.512535in}{0.437222in}}{\pgfqpoint{6.275590in}{5.159444in}}%
\pgfusepath{clip}%
\pgfsetbuttcap%
\pgfsetroundjoin%
\pgfsetlinewidth{1.003750pt}%
\definecolor{currentstroke}{rgb}{0.827451,0.827451,0.827451}%
\pgfsetstrokecolor{currentstroke}%
\pgfsetstrokeopacity{0.800000}%
\pgfsetdash{}{0pt}%
\pgfpathmoveto{\pgfqpoint{13.165622in}{5.501584in}}%
\pgfpathcurveto{\pgfqpoint{13.176672in}{5.501584in}}{\pgfqpoint{13.187271in}{5.505974in}}{\pgfqpoint{13.195085in}{5.513787in}}%
\pgfpathcurveto{\pgfqpoint{13.202899in}{5.521601in}}{\pgfqpoint{13.207289in}{5.532200in}}{\pgfqpoint{13.207289in}{5.543250in}}%
\pgfpathcurveto{\pgfqpoint{13.207289in}{5.554300in}}{\pgfqpoint{13.202899in}{5.564899in}}{\pgfqpoint{13.195085in}{5.572713in}}%
\pgfpathcurveto{\pgfqpoint{13.187271in}{5.580527in}}{\pgfqpoint{13.176672in}{5.584917in}}{\pgfqpoint{13.165622in}{5.584917in}}%
\pgfpathcurveto{\pgfqpoint{13.154572in}{5.584917in}}{\pgfqpoint{13.143973in}{5.580527in}}{\pgfqpoint{13.136159in}{5.572713in}}%
\pgfpathcurveto{\pgfqpoint{13.128346in}{5.564899in}}{\pgfqpoint{13.123955in}{5.554300in}}{\pgfqpoint{13.123955in}{5.543250in}}%
\pgfpathcurveto{\pgfqpoint{13.123955in}{5.532200in}}{\pgfqpoint{13.128346in}{5.521601in}}{\pgfqpoint{13.136159in}{5.513787in}}%
\pgfpathcurveto{\pgfqpoint{13.143973in}{5.505974in}}{\pgfqpoint{13.154572in}{5.501584in}}{\pgfqpoint{13.165622in}{5.501584in}}%
\pgfpathlineto{\pgfqpoint{13.165622in}{5.501584in}}%
\pgfpathclose%
\pgfusepath{stroke}%
\end{pgfscope}%
\begin{pgfscope}%
\pgfpathrectangle{\pgfqpoint{7.512535in}{0.437222in}}{\pgfqpoint{6.275590in}{5.159444in}}%
\pgfusepath{clip}%
\pgfsetbuttcap%
\pgfsetroundjoin%
\pgfsetlinewidth{1.003750pt}%
\definecolor{currentstroke}{rgb}{0.827451,0.827451,0.827451}%
\pgfsetstrokecolor{currentstroke}%
\pgfsetstrokeopacity{0.800000}%
\pgfsetdash{}{0pt}%
\pgfpathmoveto{\pgfqpoint{7.679894in}{0.493855in}}%
\pgfpathcurveto{\pgfqpoint{7.690944in}{0.493855in}}{\pgfqpoint{7.701543in}{0.498245in}}{\pgfqpoint{7.709357in}{0.506059in}}%
\pgfpathcurveto{\pgfqpoint{7.717170in}{0.513872in}}{\pgfqpoint{7.721560in}{0.524471in}}{\pgfqpoint{7.721560in}{0.535522in}}%
\pgfpathcurveto{\pgfqpoint{7.721560in}{0.546572in}}{\pgfqpoint{7.717170in}{0.557171in}}{\pgfqpoint{7.709357in}{0.564984in}}%
\pgfpathcurveto{\pgfqpoint{7.701543in}{0.572798in}}{\pgfqpoint{7.690944in}{0.577188in}}{\pgfqpoint{7.679894in}{0.577188in}}%
\pgfpathcurveto{\pgfqpoint{7.668844in}{0.577188in}}{\pgfqpoint{7.658245in}{0.572798in}}{\pgfqpoint{7.650431in}{0.564984in}}%
\pgfpathcurveto{\pgfqpoint{7.642617in}{0.557171in}}{\pgfqpoint{7.638227in}{0.546572in}}{\pgfqpoint{7.638227in}{0.535522in}}%
\pgfpathcurveto{\pgfqpoint{7.638227in}{0.524471in}}{\pgfqpoint{7.642617in}{0.513872in}}{\pgfqpoint{7.650431in}{0.506059in}}%
\pgfpathcurveto{\pgfqpoint{7.658245in}{0.498245in}}{\pgfqpoint{7.668844in}{0.493855in}}{\pgfqpoint{7.679894in}{0.493855in}}%
\pgfpathlineto{\pgfqpoint{7.679894in}{0.493855in}}%
\pgfpathclose%
\pgfusepath{stroke}%
\end{pgfscope}%
\begin{pgfscope}%
\pgfpathrectangle{\pgfqpoint{7.512535in}{0.437222in}}{\pgfqpoint{6.275590in}{5.159444in}}%
\pgfusepath{clip}%
\pgfsetbuttcap%
\pgfsetroundjoin%
\pgfsetlinewidth{1.003750pt}%
\definecolor{currentstroke}{rgb}{0.827451,0.827451,0.827451}%
\pgfsetstrokecolor{currentstroke}%
\pgfsetstrokeopacity{0.800000}%
\pgfsetdash{}{0pt}%
\pgfpathmoveto{\pgfqpoint{9.027094in}{2.556537in}}%
\pgfpathcurveto{\pgfqpoint{9.038145in}{2.556537in}}{\pgfqpoint{9.048744in}{2.560927in}}{\pgfqpoint{9.056557in}{2.568740in}}%
\pgfpathcurveto{\pgfqpoint{9.064371in}{2.576554in}}{\pgfqpoint{9.068761in}{2.587153in}}{\pgfqpoint{9.068761in}{2.598203in}}%
\pgfpathcurveto{\pgfqpoint{9.068761in}{2.609253in}}{\pgfqpoint{9.064371in}{2.619852in}}{\pgfqpoint{9.056557in}{2.627666in}}%
\pgfpathcurveto{\pgfqpoint{9.048744in}{2.635480in}}{\pgfqpoint{9.038145in}{2.639870in}}{\pgfqpoint{9.027094in}{2.639870in}}%
\pgfpathcurveto{\pgfqpoint{9.016044in}{2.639870in}}{\pgfqpoint{9.005445in}{2.635480in}}{\pgfqpoint{8.997632in}{2.627666in}}%
\pgfpathcurveto{\pgfqpoint{8.989818in}{2.619852in}}{\pgfqpoint{8.985428in}{2.609253in}}{\pgfqpoint{8.985428in}{2.598203in}}%
\pgfpathcurveto{\pgfqpoint{8.985428in}{2.587153in}}{\pgfqpoint{8.989818in}{2.576554in}}{\pgfqpoint{8.997632in}{2.568740in}}%
\pgfpathcurveto{\pgfqpoint{9.005445in}{2.560927in}}{\pgfqpoint{9.016044in}{2.556537in}}{\pgfqpoint{9.027094in}{2.556537in}}%
\pgfpathlineto{\pgfqpoint{9.027094in}{2.556537in}}%
\pgfpathclose%
\pgfusepath{stroke}%
\end{pgfscope}%
\begin{pgfscope}%
\pgfpathrectangle{\pgfqpoint{7.512535in}{0.437222in}}{\pgfqpoint{6.275590in}{5.159444in}}%
\pgfusepath{clip}%
\pgfsetbuttcap%
\pgfsetroundjoin%
\pgfsetlinewidth{1.003750pt}%
\definecolor{currentstroke}{rgb}{0.827451,0.827451,0.827451}%
\pgfsetstrokecolor{currentstroke}%
\pgfsetstrokeopacity{0.800000}%
\pgfsetdash{}{0pt}%
\pgfpathmoveto{\pgfqpoint{13.419279in}{5.547910in}}%
\pgfpathcurveto{\pgfqpoint{13.430329in}{5.547910in}}{\pgfqpoint{13.440928in}{5.552301in}}{\pgfqpoint{13.448742in}{5.560114in}}%
\pgfpathcurveto{\pgfqpoint{13.456555in}{5.567928in}}{\pgfqpoint{13.460945in}{5.578527in}}{\pgfqpoint{13.460945in}{5.589577in}}%
\pgfpathcurveto{\pgfqpoint{13.460945in}{5.600627in}}{\pgfqpoint{13.456555in}{5.611226in}}{\pgfqpoint{13.448742in}{5.619040in}}%
\pgfpathcurveto{\pgfqpoint{13.440928in}{5.626854in}}{\pgfqpoint{13.430329in}{5.631244in}}{\pgfqpoint{13.419279in}{5.631244in}}%
\pgfpathcurveto{\pgfqpoint{13.408229in}{5.631244in}}{\pgfqpoint{13.397630in}{5.626854in}}{\pgfqpoint{13.389816in}{5.619040in}}%
\pgfpathcurveto{\pgfqpoint{13.382002in}{5.611226in}}{\pgfqpoint{13.377612in}{5.600627in}}{\pgfqpoint{13.377612in}{5.589577in}}%
\pgfpathcurveto{\pgfqpoint{13.377612in}{5.578527in}}{\pgfqpoint{13.382002in}{5.567928in}}{\pgfqpoint{13.389816in}{5.560114in}}%
\pgfpathcurveto{\pgfqpoint{13.397630in}{5.552301in}}{\pgfqpoint{13.408229in}{5.547910in}}{\pgfqpoint{13.419279in}{5.547910in}}%
\pgfpathlineto{\pgfqpoint{13.419279in}{5.547910in}}%
\pgfpathclose%
\pgfusepath{stroke}%
\end{pgfscope}%
\begin{pgfscope}%
\pgfpathrectangle{\pgfqpoint{7.512535in}{0.437222in}}{\pgfqpoint{6.275590in}{5.159444in}}%
\pgfusepath{clip}%
\pgfsetbuttcap%
\pgfsetroundjoin%
\pgfsetlinewidth{1.003750pt}%
\definecolor{currentstroke}{rgb}{0.827451,0.827451,0.827451}%
\pgfsetstrokecolor{currentstroke}%
\pgfsetstrokeopacity{0.800000}%
\pgfsetdash{}{0pt}%
\pgfpathmoveto{\pgfqpoint{9.928499in}{4.056499in}}%
\pgfpathcurveto{\pgfqpoint{9.939549in}{4.056499in}}{\pgfqpoint{9.950148in}{4.060889in}}{\pgfqpoint{9.957962in}{4.068703in}}%
\pgfpathcurveto{\pgfqpoint{9.965775in}{4.076516in}}{\pgfqpoint{9.970165in}{4.087115in}}{\pgfqpoint{9.970165in}{4.098166in}}%
\pgfpathcurveto{\pgfqpoint{9.970165in}{4.109216in}}{\pgfqpoint{9.965775in}{4.119815in}}{\pgfqpoint{9.957962in}{4.127628in}}%
\pgfpathcurveto{\pgfqpoint{9.950148in}{4.135442in}}{\pgfqpoint{9.939549in}{4.139832in}}{\pgfqpoint{9.928499in}{4.139832in}}%
\pgfpathcurveto{\pgfqpoint{9.917449in}{4.139832in}}{\pgfqpoint{9.906850in}{4.135442in}}{\pgfqpoint{9.899036in}{4.127628in}}%
\pgfpathcurveto{\pgfqpoint{9.891222in}{4.119815in}}{\pgfqpoint{9.886832in}{4.109216in}}{\pgfqpoint{9.886832in}{4.098166in}}%
\pgfpathcurveto{\pgfqpoint{9.886832in}{4.087115in}}{\pgfqpoint{9.891222in}{4.076516in}}{\pgfqpoint{9.899036in}{4.068703in}}%
\pgfpathcurveto{\pgfqpoint{9.906850in}{4.060889in}}{\pgfqpoint{9.917449in}{4.056499in}}{\pgfqpoint{9.928499in}{4.056499in}}%
\pgfpathlineto{\pgfqpoint{9.928499in}{4.056499in}}%
\pgfpathclose%
\pgfusepath{stroke}%
\end{pgfscope}%
\begin{pgfscope}%
\pgfpathrectangle{\pgfqpoint{7.512535in}{0.437222in}}{\pgfqpoint{6.275590in}{5.159444in}}%
\pgfusepath{clip}%
\pgfsetbuttcap%
\pgfsetroundjoin%
\pgfsetlinewidth{1.003750pt}%
\definecolor{currentstroke}{rgb}{0.827451,0.827451,0.827451}%
\pgfsetstrokecolor{currentstroke}%
\pgfsetstrokeopacity{0.800000}%
\pgfsetdash{}{0pt}%
\pgfpathmoveto{\pgfqpoint{11.629363in}{5.304399in}}%
\pgfpathcurveto{\pgfqpoint{11.640413in}{5.304399in}}{\pgfqpoint{11.651012in}{5.308790in}}{\pgfqpoint{11.658826in}{5.316603in}}%
\pgfpathcurveto{\pgfqpoint{11.666639in}{5.324417in}}{\pgfqpoint{11.671030in}{5.335016in}}{\pgfqpoint{11.671030in}{5.346066in}}%
\pgfpathcurveto{\pgfqpoint{11.671030in}{5.357116in}}{\pgfqpoint{11.666639in}{5.367715in}}{\pgfqpoint{11.658826in}{5.375529in}}%
\pgfpathcurveto{\pgfqpoint{11.651012in}{5.383342in}}{\pgfqpoint{11.640413in}{5.387733in}}{\pgfqpoint{11.629363in}{5.387733in}}%
\pgfpathcurveto{\pgfqpoint{11.618313in}{5.387733in}}{\pgfqpoint{11.607714in}{5.383342in}}{\pgfqpoint{11.599900in}{5.375529in}}%
\pgfpathcurveto{\pgfqpoint{11.592087in}{5.367715in}}{\pgfqpoint{11.587696in}{5.357116in}}{\pgfqpoint{11.587696in}{5.346066in}}%
\pgfpathcurveto{\pgfqpoint{11.587696in}{5.335016in}}{\pgfqpoint{11.592087in}{5.324417in}}{\pgfqpoint{11.599900in}{5.316603in}}%
\pgfpathcurveto{\pgfqpoint{11.607714in}{5.308790in}}{\pgfqpoint{11.618313in}{5.304399in}}{\pgfqpoint{11.629363in}{5.304399in}}%
\pgfpathlineto{\pgfqpoint{11.629363in}{5.304399in}}%
\pgfpathclose%
\pgfusepath{stroke}%
\end{pgfscope}%
\begin{pgfscope}%
\pgfpathrectangle{\pgfqpoint{7.512535in}{0.437222in}}{\pgfqpoint{6.275590in}{5.159444in}}%
\pgfusepath{clip}%
\pgfsetbuttcap%
\pgfsetroundjoin%
\pgfsetlinewidth{1.003750pt}%
\definecolor{currentstroke}{rgb}{0.827451,0.827451,0.827451}%
\pgfsetstrokecolor{currentstroke}%
\pgfsetstrokeopacity{0.800000}%
\pgfsetdash{}{0pt}%
\pgfpathmoveto{\pgfqpoint{9.578307in}{2.906605in}}%
\pgfpathcurveto{\pgfqpoint{9.589357in}{2.906605in}}{\pgfqpoint{9.599956in}{2.910995in}}{\pgfqpoint{9.607770in}{2.918809in}}%
\pgfpathcurveto{\pgfqpoint{9.615584in}{2.926623in}}{\pgfqpoint{9.619974in}{2.937222in}}{\pgfqpoint{9.619974in}{2.948272in}}%
\pgfpathcurveto{\pgfqpoint{9.619974in}{2.959322in}}{\pgfqpoint{9.615584in}{2.969921in}}{\pgfqpoint{9.607770in}{2.977735in}}%
\pgfpathcurveto{\pgfqpoint{9.599956in}{2.985548in}}{\pgfqpoint{9.589357in}{2.989938in}}{\pgfqpoint{9.578307in}{2.989938in}}%
\pgfpathcurveto{\pgfqpoint{9.567257in}{2.989938in}}{\pgfqpoint{9.556658in}{2.985548in}}{\pgfqpoint{9.548845in}{2.977735in}}%
\pgfpathcurveto{\pgfqpoint{9.541031in}{2.969921in}}{\pgfqpoint{9.536641in}{2.959322in}}{\pgfqpoint{9.536641in}{2.948272in}}%
\pgfpathcurveto{\pgfqpoint{9.536641in}{2.937222in}}{\pgfqpoint{9.541031in}{2.926623in}}{\pgfqpoint{9.548845in}{2.918809in}}%
\pgfpathcurveto{\pgfqpoint{9.556658in}{2.910995in}}{\pgfqpoint{9.567257in}{2.906605in}}{\pgfqpoint{9.578307in}{2.906605in}}%
\pgfpathlineto{\pgfqpoint{9.578307in}{2.906605in}}%
\pgfpathclose%
\pgfusepath{stroke}%
\end{pgfscope}%
\begin{pgfscope}%
\pgfpathrectangle{\pgfqpoint{7.512535in}{0.437222in}}{\pgfqpoint{6.275590in}{5.159444in}}%
\pgfusepath{clip}%
\pgfsetbuttcap%
\pgfsetroundjoin%
\pgfsetlinewidth{1.003750pt}%
\definecolor{currentstroke}{rgb}{0.827451,0.827451,0.827451}%
\pgfsetstrokecolor{currentstroke}%
\pgfsetstrokeopacity{0.800000}%
\pgfsetdash{}{0pt}%
\pgfpathmoveto{\pgfqpoint{9.447017in}{3.335366in}}%
\pgfpathcurveto{\pgfqpoint{9.458067in}{3.335366in}}{\pgfqpoint{9.468666in}{3.339757in}}{\pgfqpoint{9.476479in}{3.347570in}}%
\pgfpathcurveto{\pgfqpoint{9.484293in}{3.355384in}}{\pgfqpoint{9.488683in}{3.365983in}}{\pgfqpoint{9.488683in}{3.377033in}}%
\pgfpathcurveto{\pgfqpoint{9.488683in}{3.388083in}}{\pgfqpoint{9.484293in}{3.398682in}}{\pgfqpoint{9.476479in}{3.406496in}}%
\pgfpathcurveto{\pgfqpoint{9.468666in}{3.414309in}}{\pgfqpoint{9.458067in}{3.418700in}}{\pgfqpoint{9.447017in}{3.418700in}}%
\pgfpathcurveto{\pgfqpoint{9.435967in}{3.418700in}}{\pgfqpoint{9.425367in}{3.414309in}}{\pgfqpoint{9.417554in}{3.406496in}}%
\pgfpathcurveto{\pgfqpoint{9.409740in}{3.398682in}}{\pgfqpoint{9.405350in}{3.388083in}}{\pgfqpoint{9.405350in}{3.377033in}}%
\pgfpathcurveto{\pgfqpoint{9.405350in}{3.365983in}}{\pgfqpoint{9.409740in}{3.355384in}}{\pgfqpoint{9.417554in}{3.347570in}}%
\pgfpathcurveto{\pgfqpoint{9.425367in}{3.339757in}}{\pgfqpoint{9.435967in}{3.335366in}}{\pgfqpoint{9.447017in}{3.335366in}}%
\pgfpathlineto{\pgfqpoint{9.447017in}{3.335366in}}%
\pgfpathclose%
\pgfusepath{stroke}%
\end{pgfscope}%
\begin{pgfscope}%
\pgfpathrectangle{\pgfqpoint{7.512535in}{0.437222in}}{\pgfqpoint{6.275590in}{5.159444in}}%
\pgfusepath{clip}%
\pgfsetbuttcap%
\pgfsetroundjoin%
\pgfsetlinewidth{1.003750pt}%
\definecolor{currentstroke}{rgb}{0.827451,0.827451,0.827451}%
\pgfsetstrokecolor{currentstroke}%
\pgfsetstrokeopacity{0.800000}%
\pgfsetdash{}{0pt}%
\pgfpathmoveto{\pgfqpoint{10.045306in}{4.354371in}}%
\pgfpathcurveto{\pgfqpoint{10.056356in}{4.354371in}}{\pgfqpoint{10.066955in}{4.358761in}}{\pgfqpoint{10.074768in}{4.366575in}}%
\pgfpathcurveto{\pgfqpoint{10.082582in}{4.374389in}}{\pgfqpoint{10.086972in}{4.384988in}}{\pgfqpoint{10.086972in}{4.396038in}}%
\pgfpathcurveto{\pgfqpoint{10.086972in}{4.407088in}}{\pgfqpoint{10.082582in}{4.417687in}}{\pgfqpoint{10.074768in}{4.425500in}}%
\pgfpathcurveto{\pgfqpoint{10.066955in}{4.433314in}}{\pgfqpoint{10.056356in}{4.437704in}}{\pgfqpoint{10.045306in}{4.437704in}}%
\pgfpathcurveto{\pgfqpoint{10.034256in}{4.437704in}}{\pgfqpoint{10.023657in}{4.433314in}}{\pgfqpoint{10.015843in}{4.425500in}}%
\pgfpathcurveto{\pgfqpoint{10.008029in}{4.417687in}}{\pgfqpoint{10.003639in}{4.407088in}}{\pgfqpoint{10.003639in}{4.396038in}}%
\pgfpathcurveto{\pgfqpoint{10.003639in}{4.384988in}}{\pgfqpoint{10.008029in}{4.374389in}}{\pgfqpoint{10.015843in}{4.366575in}}%
\pgfpathcurveto{\pgfqpoint{10.023657in}{4.358761in}}{\pgfqpoint{10.034256in}{4.354371in}}{\pgfqpoint{10.045306in}{4.354371in}}%
\pgfpathlineto{\pgfqpoint{10.045306in}{4.354371in}}%
\pgfpathclose%
\pgfusepath{stroke}%
\end{pgfscope}%
\begin{pgfscope}%
\pgfpathrectangle{\pgfqpoint{7.512535in}{0.437222in}}{\pgfqpoint{6.275590in}{5.159444in}}%
\pgfusepath{clip}%
\pgfsetbuttcap%
\pgfsetroundjoin%
\pgfsetlinewidth{1.003750pt}%
\definecolor{currentstroke}{rgb}{0.827451,0.827451,0.827451}%
\pgfsetstrokecolor{currentstroke}%
\pgfsetstrokeopacity{0.800000}%
\pgfsetdash{}{0pt}%
\pgfpathmoveto{\pgfqpoint{9.060406in}{3.385314in}}%
\pgfpathcurveto{\pgfqpoint{9.071456in}{3.385314in}}{\pgfqpoint{9.082055in}{3.389704in}}{\pgfqpoint{9.089869in}{3.397518in}}%
\pgfpathcurveto{\pgfqpoint{9.097682in}{3.405332in}}{\pgfqpoint{9.102073in}{3.415931in}}{\pgfqpoint{9.102073in}{3.426981in}}%
\pgfpathcurveto{\pgfqpoint{9.102073in}{3.438031in}}{\pgfqpoint{9.097682in}{3.448630in}}{\pgfqpoint{9.089869in}{3.456444in}}%
\pgfpathcurveto{\pgfqpoint{9.082055in}{3.464257in}}{\pgfqpoint{9.071456in}{3.468647in}}{\pgfqpoint{9.060406in}{3.468647in}}%
\pgfpathcurveto{\pgfqpoint{9.049356in}{3.468647in}}{\pgfqpoint{9.038757in}{3.464257in}}{\pgfqpoint{9.030943in}{3.456444in}}%
\pgfpathcurveto{\pgfqpoint{9.023129in}{3.448630in}}{\pgfqpoint{9.018739in}{3.438031in}}{\pgfqpoint{9.018739in}{3.426981in}}%
\pgfpathcurveto{\pgfqpoint{9.018739in}{3.415931in}}{\pgfqpoint{9.023129in}{3.405332in}}{\pgfqpoint{9.030943in}{3.397518in}}%
\pgfpathcurveto{\pgfqpoint{9.038757in}{3.389704in}}{\pgfqpoint{9.049356in}{3.385314in}}{\pgfqpoint{9.060406in}{3.385314in}}%
\pgfpathlineto{\pgfqpoint{9.060406in}{3.385314in}}%
\pgfpathclose%
\pgfusepath{stroke}%
\end{pgfscope}%
\begin{pgfscope}%
\pgfpathrectangle{\pgfqpoint{7.512535in}{0.437222in}}{\pgfqpoint{6.275590in}{5.159444in}}%
\pgfusepath{clip}%
\pgfsetbuttcap%
\pgfsetroundjoin%
\pgfsetlinewidth{1.003750pt}%
\definecolor{currentstroke}{rgb}{0.827451,0.827451,0.827451}%
\pgfsetstrokecolor{currentstroke}%
\pgfsetstrokeopacity{0.800000}%
\pgfsetdash{}{0pt}%
\pgfpathmoveto{\pgfqpoint{10.698510in}{3.970272in}}%
\pgfpathcurveto{\pgfqpoint{10.709560in}{3.970272in}}{\pgfqpoint{10.720159in}{3.974662in}}{\pgfqpoint{10.727973in}{3.982476in}}%
\pgfpathcurveto{\pgfqpoint{10.735787in}{3.990290in}}{\pgfqpoint{10.740177in}{4.000889in}}{\pgfqpoint{10.740177in}{4.011939in}}%
\pgfpathcurveto{\pgfqpoint{10.740177in}{4.022989in}}{\pgfqpoint{10.735787in}{4.033588in}}{\pgfqpoint{10.727973in}{4.041402in}}%
\pgfpathcurveto{\pgfqpoint{10.720159in}{4.049215in}}{\pgfqpoint{10.709560in}{4.053605in}}{\pgfqpoint{10.698510in}{4.053605in}}%
\pgfpathcurveto{\pgfqpoint{10.687460in}{4.053605in}}{\pgfqpoint{10.676861in}{4.049215in}}{\pgfqpoint{10.669047in}{4.041402in}}%
\pgfpathcurveto{\pgfqpoint{10.661234in}{4.033588in}}{\pgfqpoint{10.656844in}{4.022989in}}{\pgfqpoint{10.656844in}{4.011939in}}%
\pgfpathcurveto{\pgfqpoint{10.656844in}{4.000889in}}{\pgfqpoint{10.661234in}{3.990290in}}{\pgfqpoint{10.669047in}{3.982476in}}%
\pgfpathcurveto{\pgfqpoint{10.676861in}{3.974662in}}{\pgfqpoint{10.687460in}{3.970272in}}{\pgfqpoint{10.698510in}{3.970272in}}%
\pgfpathlineto{\pgfqpoint{10.698510in}{3.970272in}}%
\pgfpathclose%
\pgfusepath{stroke}%
\end{pgfscope}%
\begin{pgfscope}%
\pgfpathrectangle{\pgfqpoint{7.512535in}{0.437222in}}{\pgfqpoint{6.275590in}{5.159444in}}%
\pgfusepath{clip}%
\pgfsetbuttcap%
\pgfsetroundjoin%
\pgfsetlinewidth{1.003750pt}%
\definecolor{currentstroke}{rgb}{0.827451,0.827451,0.827451}%
\pgfsetstrokecolor{currentstroke}%
\pgfsetstrokeopacity{0.800000}%
\pgfsetdash{}{0pt}%
\pgfpathmoveto{\pgfqpoint{8.800029in}{1.838807in}}%
\pgfpathcurveto{\pgfqpoint{8.811079in}{1.838807in}}{\pgfqpoint{8.821678in}{1.843198in}}{\pgfqpoint{8.829491in}{1.851011in}}%
\pgfpathcurveto{\pgfqpoint{8.837305in}{1.858825in}}{\pgfqpoint{8.841695in}{1.869424in}}{\pgfqpoint{8.841695in}{1.880474in}}%
\pgfpathcurveto{\pgfqpoint{8.841695in}{1.891524in}}{\pgfqpoint{8.837305in}{1.902123in}}{\pgfqpoint{8.829491in}{1.909937in}}%
\pgfpathcurveto{\pgfqpoint{8.821678in}{1.917750in}}{\pgfqpoint{8.811079in}{1.922141in}}{\pgfqpoint{8.800029in}{1.922141in}}%
\pgfpathcurveto{\pgfqpoint{8.788978in}{1.922141in}}{\pgfqpoint{8.778379in}{1.917750in}}{\pgfqpoint{8.770566in}{1.909937in}}%
\pgfpathcurveto{\pgfqpoint{8.762752in}{1.902123in}}{\pgfqpoint{8.758362in}{1.891524in}}{\pgfqpoint{8.758362in}{1.880474in}}%
\pgfpathcurveto{\pgfqpoint{8.758362in}{1.869424in}}{\pgfqpoint{8.762752in}{1.858825in}}{\pgfqpoint{8.770566in}{1.851011in}}%
\pgfpathcurveto{\pgfqpoint{8.778379in}{1.843198in}}{\pgfqpoint{8.788978in}{1.838807in}}{\pgfqpoint{8.800029in}{1.838807in}}%
\pgfpathlineto{\pgfqpoint{8.800029in}{1.838807in}}%
\pgfpathclose%
\pgfusepath{stroke}%
\end{pgfscope}%
\begin{pgfscope}%
\pgfpathrectangle{\pgfqpoint{7.512535in}{0.437222in}}{\pgfqpoint{6.275590in}{5.159444in}}%
\pgfusepath{clip}%
\pgfsetbuttcap%
\pgfsetroundjoin%
\pgfsetlinewidth{1.003750pt}%
\definecolor{currentstroke}{rgb}{0.827451,0.827451,0.827451}%
\pgfsetstrokecolor{currentstroke}%
\pgfsetstrokeopacity{0.800000}%
\pgfsetdash{}{0pt}%
\pgfpathmoveto{\pgfqpoint{7.992439in}{1.704977in}}%
\pgfpathcurveto{\pgfqpoint{8.003489in}{1.704977in}}{\pgfqpoint{8.014088in}{1.709367in}}{\pgfqpoint{8.021902in}{1.717181in}}%
\pgfpathcurveto{\pgfqpoint{8.029715in}{1.724994in}}{\pgfqpoint{8.034105in}{1.735593in}}{\pgfqpoint{8.034105in}{1.746643in}}%
\pgfpathcurveto{\pgfqpoint{8.034105in}{1.757694in}}{\pgfqpoint{8.029715in}{1.768293in}}{\pgfqpoint{8.021902in}{1.776106in}}%
\pgfpathcurveto{\pgfqpoint{8.014088in}{1.783920in}}{\pgfqpoint{8.003489in}{1.788310in}}{\pgfqpoint{7.992439in}{1.788310in}}%
\pgfpathcurveto{\pgfqpoint{7.981389in}{1.788310in}}{\pgfqpoint{7.970790in}{1.783920in}}{\pgfqpoint{7.962976in}{1.776106in}}%
\pgfpathcurveto{\pgfqpoint{7.955162in}{1.768293in}}{\pgfqpoint{7.950772in}{1.757694in}}{\pgfqpoint{7.950772in}{1.746643in}}%
\pgfpathcurveto{\pgfqpoint{7.950772in}{1.735593in}}{\pgfqpoint{7.955162in}{1.724994in}}{\pgfqpoint{7.962976in}{1.717181in}}%
\pgfpathcurveto{\pgfqpoint{7.970790in}{1.709367in}}{\pgfqpoint{7.981389in}{1.704977in}}{\pgfqpoint{7.992439in}{1.704977in}}%
\pgfpathlineto{\pgfqpoint{7.992439in}{1.704977in}}%
\pgfpathclose%
\pgfusepath{stroke}%
\end{pgfscope}%
\begin{pgfscope}%
\pgfpathrectangle{\pgfqpoint{7.512535in}{0.437222in}}{\pgfqpoint{6.275590in}{5.159444in}}%
\pgfusepath{clip}%
\pgfsetbuttcap%
\pgfsetroundjoin%
\pgfsetlinewidth{1.003750pt}%
\definecolor{currentstroke}{rgb}{0.827451,0.827451,0.827451}%
\pgfsetstrokecolor{currentstroke}%
\pgfsetstrokeopacity{0.800000}%
\pgfsetdash{}{0pt}%
\pgfpathmoveto{\pgfqpoint{9.578307in}{2.959155in}}%
\pgfpathcurveto{\pgfqpoint{9.589357in}{2.959155in}}{\pgfqpoint{9.599956in}{2.963545in}}{\pgfqpoint{9.607770in}{2.971359in}}%
\pgfpathcurveto{\pgfqpoint{9.615584in}{2.979173in}}{\pgfqpoint{9.619974in}{2.989772in}}{\pgfqpoint{9.619974in}{3.000822in}}%
\pgfpathcurveto{\pgfqpoint{9.619974in}{3.011872in}}{\pgfqpoint{9.615584in}{3.022471in}}{\pgfqpoint{9.607770in}{3.030284in}}%
\pgfpathcurveto{\pgfqpoint{9.599956in}{3.038098in}}{\pgfqpoint{9.589357in}{3.042488in}}{\pgfqpoint{9.578307in}{3.042488in}}%
\pgfpathcurveto{\pgfqpoint{9.567257in}{3.042488in}}{\pgfqpoint{9.556658in}{3.038098in}}{\pgfqpoint{9.548845in}{3.030284in}}%
\pgfpathcurveto{\pgfqpoint{9.541031in}{3.022471in}}{\pgfqpoint{9.536641in}{3.011872in}}{\pgfqpoint{9.536641in}{3.000822in}}%
\pgfpathcurveto{\pgfqpoint{9.536641in}{2.989772in}}{\pgfqpoint{9.541031in}{2.979173in}}{\pgfqpoint{9.548845in}{2.971359in}}%
\pgfpathcurveto{\pgfqpoint{9.556658in}{2.963545in}}{\pgfqpoint{9.567257in}{2.959155in}}{\pgfqpoint{9.578307in}{2.959155in}}%
\pgfpathlineto{\pgfqpoint{9.578307in}{2.959155in}}%
\pgfpathclose%
\pgfusepath{stroke}%
\end{pgfscope}%
\begin{pgfscope}%
\pgfpathrectangle{\pgfqpoint{7.512535in}{0.437222in}}{\pgfqpoint{6.275590in}{5.159444in}}%
\pgfusepath{clip}%
\pgfsetbuttcap%
\pgfsetroundjoin%
\pgfsetlinewidth{1.003750pt}%
\definecolor{currentstroke}{rgb}{0.827451,0.827451,0.827451}%
\pgfsetstrokecolor{currentstroke}%
\pgfsetstrokeopacity{0.800000}%
\pgfsetdash{}{0pt}%
\pgfpathmoveto{\pgfqpoint{13.419279in}{5.549860in}}%
\pgfpathcurveto{\pgfqpoint{13.430329in}{5.549860in}}{\pgfqpoint{13.440928in}{5.554250in}}{\pgfqpoint{13.448742in}{5.562064in}}%
\pgfpathcurveto{\pgfqpoint{13.456555in}{5.569877in}}{\pgfqpoint{13.460945in}{5.580476in}}{\pgfqpoint{13.460945in}{5.591526in}}%
\pgfpathcurveto{\pgfqpoint{13.460945in}{5.602576in}}{\pgfqpoint{13.456555in}{5.613175in}}{\pgfqpoint{13.448742in}{5.620989in}}%
\pgfpathcurveto{\pgfqpoint{13.440928in}{5.628803in}}{\pgfqpoint{13.430329in}{5.633193in}}{\pgfqpoint{13.419279in}{5.633193in}}%
\pgfpathcurveto{\pgfqpoint{13.408229in}{5.633193in}}{\pgfqpoint{13.397630in}{5.628803in}}{\pgfqpoint{13.389816in}{5.620989in}}%
\pgfpathcurveto{\pgfqpoint{13.382002in}{5.613175in}}{\pgfqpoint{13.377612in}{5.602576in}}{\pgfqpoint{13.377612in}{5.591526in}}%
\pgfpathcurveto{\pgfqpoint{13.377612in}{5.580476in}}{\pgfqpoint{13.382002in}{5.569877in}}{\pgfqpoint{13.389816in}{5.562064in}}%
\pgfpathcurveto{\pgfqpoint{13.397630in}{5.554250in}}{\pgfqpoint{13.408229in}{5.549860in}}{\pgfqpoint{13.419279in}{5.549860in}}%
\pgfpathlineto{\pgfqpoint{13.419279in}{5.549860in}}%
\pgfpathclose%
\pgfusepath{stroke}%
\end{pgfscope}%
\begin{pgfscope}%
\pgfpathrectangle{\pgfqpoint{7.512535in}{0.437222in}}{\pgfqpoint{6.275590in}{5.159444in}}%
\pgfusepath{clip}%
\pgfsetbuttcap%
\pgfsetroundjoin%
\pgfsetlinewidth{1.003750pt}%
\definecolor{currentstroke}{rgb}{0.827451,0.827451,0.827451}%
\pgfsetstrokecolor{currentstroke}%
\pgfsetstrokeopacity{0.800000}%
\pgfsetdash{}{0pt}%
\pgfpathmoveto{\pgfqpoint{12.495601in}{5.523529in}}%
\pgfpathcurveto{\pgfqpoint{12.506651in}{5.523529in}}{\pgfqpoint{12.517250in}{5.527919in}}{\pgfqpoint{12.525063in}{5.535733in}}%
\pgfpathcurveto{\pgfqpoint{12.532877in}{5.543547in}}{\pgfqpoint{12.537267in}{5.554146in}}{\pgfqpoint{12.537267in}{5.565196in}}%
\pgfpathcurveto{\pgfqpoint{12.537267in}{5.576246in}}{\pgfqpoint{12.532877in}{5.586845in}}{\pgfqpoint{12.525063in}{5.594658in}}%
\pgfpathcurveto{\pgfqpoint{12.517250in}{5.602472in}}{\pgfqpoint{12.506651in}{5.606862in}}{\pgfqpoint{12.495601in}{5.606862in}}%
\pgfpathcurveto{\pgfqpoint{12.484550in}{5.606862in}}{\pgfqpoint{12.473951in}{5.602472in}}{\pgfqpoint{12.466138in}{5.594658in}}%
\pgfpathcurveto{\pgfqpoint{12.458324in}{5.586845in}}{\pgfqpoint{12.453934in}{5.576246in}}{\pgfqpoint{12.453934in}{5.565196in}}%
\pgfpathcurveto{\pgfqpoint{12.453934in}{5.554146in}}{\pgfqpoint{12.458324in}{5.543547in}}{\pgfqpoint{12.466138in}{5.535733in}}%
\pgfpathcurveto{\pgfqpoint{12.473951in}{5.527919in}}{\pgfqpoint{12.484550in}{5.523529in}}{\pgfqpoint{12.495601in}{5.523529in}}%
\pgfpathlineto{\pgfqpoint{12.495601in}{5.523529in}}%
\pgfpathclose%
\pgfusepath{stroke}%
\end{pgfscope}%
\begin{pgfscope}%
\pgfpathrectangle{\pgfqpoint{7.512535in}{0.437222in}}{\pgfqpoint{6.275590in}{5.159444in}}%
\pgfusepath{clip}%
\pgfsetbuttcap%
\pgfsetroundjoin%
\pgfsetlinewidth{1.003750pt}%
\definecolor{currentstroke}{rgb}{0.827451,0.827451,0.827451}%
\pgfsetstrokecolor{currentstroke}%
\pgfsetstrokeopacity{0.800000}%
\pgfsetdash{}{0pt}%
\pgfpathmoveto{\pgfqpoint{9.697299in}{3.449334in}}%
\pgfpathcurveto{\pgfqpoint{9.708349in}{3.449334in}}{\pgfqpoint{9.718948in}{3.453725in}}{\pgfqpoint{9.726762in}{3.461538in}}%
\pgfpathcurveto{\pgfqpoint{9.734575in}{3.469352in}}{\pgfqpoint{9.738965in}{3.479951in}}{\pgfqpoint{9.738965in}{3.491001in}}%
\pgfpathcurveto{\pgfqpoint{9.738965in}{3.502051in}}{\pgfqpoint{9.734575in}{3.512650in}}{\pgfqpoint{9.726762in}{3.520464in}}%
\pgfpathcurveto{\pgfqpoint{9.718948in}{3.528277in}}{\pgfqpoint{9.708349in}{3.532668in}}{\pgfqpoint{9.697299in}{3.532668in}}%
\pgfpathcurveto{\pgfqpoint{9.686249in}{3.532668in}}{\pgfqpoint{9.675650in}{3.528277in}}{\pgfqpoint{9.667836in}{3.520464in}}%
\pgfpathcurveto{\pgfqpoint{9.660022in}{3.512650in}}{\pgfqpoint{9.655632in}{3.502051in}}{\pgfqpoint{9.655632in}{3.491001in}}%
\pgfpathcurveto{\pgfqpoint{9.655632in}{3.479951in}}{\pgfqpoint{9.660022in}{3.469352in}}{\pgfqpoint{9.667836in}{3.461538in}}%
\pgfpathcurveto{\pgfqpoint{9.675650in}{3.453725in}}{\pgfqpoint{9.686249in}{3.449334in}}{\pgfqpoint{9.697299in}{3.449334in}}%
\pgfpathlineto{\pgfqpoint{9.697299in}{3.449334in}}%
\pgfpathclose%
\pgfusepath{stroke}%
\end{pgfscope}%
\begin{pgfscope}%
\pgfpathrectangle{\pgfqpoint{7.512535in}{0.437222in}}{\pgfqpoint{6.275590in}{5.159444in}}%
\pgfusepath{clip}%
\pgfsetbuttcap%
\pgfsetroundjoin%
\pgfsetlinewidth{1.003750pt}%
\definecolor{currentstroke}{rgb}{0.827451,0.827451,0.827451}%
\pgfsetstrokecolor{currentstroke}%
\pgfsetstrokeopacity{0.800000}%
\pgfsetdash{}{0pt}%
\pgfpathmoveto{\pgfqpoint{10.312167in}{5.501728in}}%
\pgfpathcurveto{\pgfqpoint{10.323217in}{5.501728in}}{\pgfqpoint{10.333816in}{5.506118in}}{\pgfqpoint{10.341630in}{5.513932in}}%
\pgfpathcurveto{\pgfqpoint{10.349443in}{5.521745in}}{\pgfqpoint{10.353834in}{5.532344in}}{\pgfqpoint{10.353834in}{5.543394in}}%
\pgfpathcurveto{\pgfqpoint{10.353834in}{5.554445in}}{\pgfqpoint{10.349443in}{5.565044in}}{\pgfqpoint{10.341630in}{5.572857in}}%
\pgfpathcurveto{\pgfqpoint{10.333816in}{5.580671in}}{\pgfqpoint{10.323217in}{5.585061in}}{\pgfqpoint{10.312167in}{5.585061in}}%
\pgfpathcurveto{\pgfqpoint{10.301117in}{5.585061in}}{\pgfqpoint{10.290518in}{5.580671in}}{\pgfqpoint{10.282704in}{5.572857in}}%
\pgfpathcurveto{\pgfqpoint{10.274891in}{5.565044in}}{\pgfqpoint{10.270500in}{5.554445in}}{\pgfqpoint{10.270500in}{5.543394in}}%
\pgfpathcurveto{\pgfqpoint{10.270500in}{5.532344in}}{\pgfqpoint{10.274891in}{5.521745in}}{\pgfqpoint{10.282704in}{5.513932in}}%
\pgfpathcurveto{\pgfqpoint{10.290518in}{5.506118in}}{\pgfqpoint{10.301117in}{5.501728in}}{\pgfqpoint{10.312167in}{5.501728in}}%
\pgfpathlineto{\pgfqpoint{10.312167in}{5.501728in}}%
\pgfpathclose%
\pgfusepath{stroke}%
\end{pgfscope}%
\begin{pgfscope}%
\pgfpathrectangle{\pgfqpoint{7.512535in}{0.437222in}}{\pgfqpoint{6.275590in}{5.159444in}}%
\pgfusepath{clip}%
\pgfsetbuttcap%
\pgfsetroundjoin%
\pgfsetlinewidth{1.003750pt}%
\definecolor{currentstroke}{rgb}{0.827451,0.827451,0.827451}%
\pgfsetstrokecolor{currentstroke}%
\pgfsetstrokeopacity{0.800000}%
\pgfsetdash{}{0pt}%
\pgfpathmoveto{\pgfqpoint{10.780511in}{5.040147in}}%
\pgfpathcurveto{\pgfqpoint{10.791561in}{5.040147in}}{\pgfqpoint{10.802160in}{5.044537in}}{\pgfqpoint{10.809973in}{5.052350in}}%
\pgfpathcurveto{\pgfqpoint{10.817787in}{5.060164in}}{\pgfqpoint{10.822177in}{5.070763in}}{\pgfqpoint{10.822177in}{5.081813in}}%
\pgfpathcurveto{\pgfqpoint{10.822177in}{5.092863in}}{\pgfqpoint{10.817787in}{5.103462in}}{\pgfqpoint{10.809973in}{5.111276in}}%
\pgfpathcurveto{\pgfqpoint{10.802160in}{5.119090in}}{\pgfqpoint{10.791561in}{5.123480in}}{\pgfqpoint{10.780511in}{5.123480in}}%
\pgfpathcurveto{\pgfqpoint{10.769461in}{5.123480in}}{\pgfqpoint{10.758862in}{5.119090in}}{\pgfqpoint{10.751048in}{5.111276in}}%
\pgfpathcurveto{\pgfqpoint{10.743234in}{5.103462in}}{\pgfqpoint{10.738844in}{5.092863in}}{\pgfqpoint{10.738844in}{5.081813in}}%
\pgfpathcurveto{\pgfqpoint{10.738844in}{5.070763in}}{\pgfqpoint{10.743234in}{5.060164in}}{\pgfqpoint{10.751048in}{5.052350in}}%
\pgfpathcurveto{\pgfqpoint{10.758862in}{5.044537in}}{\pgfqpoint{10.769461in}{5.040147in}}{\pgfqpoint{10.780511in}{5.040147in}}%
\pgfpathlineto{\pgfqpoint{10.780511in}{5.040147in}}%
\pgfpathclose%
\pgfusepath{stroke}%
\end{pgfscope}%
\begin{pgfscope}%
\pgfpathrectangle{\pgfqpoint{7.512535in}{0.437222in}}{\pgfqpoint{6.275590in}{5.159444in}}%
\pgfusepath{clip}%
\pgfsetbuttcap%
\pgfsetroundjoin%
\pgfsetlinewidth{1.003750pt}%
\definecolor{currentstroke}{rgb}{0.827451,0.827451,0.827451}%
\pgfsetstrokecolor{currentstroke}%
\pgfsetstrokeopacity{0.800000}%
\pgfsetdash{}{0pt}%
\pgfpathmoveto{\pgfqpoint{9.821149in}{3.449334in}}%
\pgfpathcurveto{\pgfqpoint{9.832199in}{3.449334in}}{\pgfqpoint{9.842798in}{3.453725in}}{\pgfqpoint{9.850611in}{3.461538in}}%
\pgfpathcurveto{\pgfqpoint{9.858425in}{3.469352in}}{\pgfqpoint{9.862815in}{3.479951in}}{\pgfqpoint{9.862815in}{3.491001in}}%
\pgfpathcurveto{\pgfqpoint{9.862815in}{3.502051in}}{\pgfqpoint{9.858425in}{3.512650in}}{\pgfqpoint{9.850611in}{3.520464in}}%
\pgfpathcurveto{\pgfqpoint{9.842798in}{3.528277in}}{\pgfqpoint{9.832199in}{3.532668in}}{\pgfqpoint{9.821149in}{3.532668in}}%
\pgfpathcurveto{\pgfqpoint{9.810099in}{3.532668in}}{\pgfqpoint{9.799500in}{3.528277in}}{\pgfqpoint{9.791686in}{3.520464in}}%
\pgfpathcurveto{\pgfqpoint{9.783872in}{3.512650in}}{\pgfqpoint{9.779482in}{3.502051in}}{\pgfqpoint{9.779482in}{3.491001in}}%
\pgfpathcurveto{\pgfqpoint{9.779482in}{3.479951in}}{\pgfqpoint{9.783872in}{3.469352in}}{\pgfqpoint{9.791686in}{3.461538in}}%
\pgfpathcurveto{\pgfqpoint{9.799500in}{3.453725in}}{\pgfqpoint{9.810099in}{3.449334in}}{\pgfqpoint{9.821149in}{3.449334in}}%
\pgfpathlineto{\pgfqpoint{9.821149in}{3.449334in}}%
\pgfpathclose%
\pgfusepath{stroke}%
\end{pgfscope}%
\begin{pgfscope}%
\pgfpathrectangle{\pgfqpoint{7.512535in}{0.437222in}}{\pgfqpoint{6.275590in}{5.159444in}}%
\pgfusepath{clip}%
\pgfsetbuttcap%
\pgfsetroundjoin%
\pgfsetlinewidth{1.003750pt}%
\definecolor{currentstroke}{rgb}{0.827451,0.827451,0.827451}%
\pgfsetstrokecolor{currentstroke}%
\pgfsetstrokeopacity{0.800000}%
\pgfsetdash{}{0pt}%
\pgfpathmoveto{\pgfqpoint{11.405420in}{5.475782in}}%
\pgfpathcurveto{\pgfqpoint{11.416470in}{5.475782in}}{\pgfqpoint{11.427069in}{5.480172in}}{\pgfqpoint{11.434883in}{5.487986in}}%
\pgfpathcurveto{\pgfqpoint{11.442696in}{5.495800in}}{\pgfqpoint{11.447086in}{5.506399in}}{\pgfqpoint{11.447086in}{5.517449in}}%
\pgfpathcurveto{\pgfqpoint{11.447086in}{5.528499in}}{\pgfqpoint{11.442696in}{5.539098in}}{\pgfqpoint{11.434883in}{5.546912in}}%
\pgfpathcurveto{\pgfqpoint{11.427069in}{5.554725in}}{\pgfqpoint{11.416470in}{5.559116in}}{\pgfqpoint{11.405420in}{5.559116in}}%
\pgfpathcurveto{\pgfqpoint{11.394370in}{5.559116in}}{\pgfqpoint{11.383771in}{5.554725in}}{\pgfqpoint{11.375957in}{5.546912in}}%
\pgfpathcurveto{\pgfqpoint{11.368143in}{5.539098in}}{\pgfqpoint{11.363753in}{5.528499in}}{\pgfqpoint{11.363753in}{5.517449in}}%
\pgfpathcurveto{\pgfqpoint{11.363753in}{5.506399in}}{\pgfqpoint{11.368143in}{5.495800in}}{\pgfqpoint{11.375957in}{5.487986in}}%
\pgfpathcurveto{\pgfqpoint{11.383771in}{5.480172in}}{\pgfqpoint{11.394370in}{5.475782in}}{\pgfqpoint{11.405420in}{5.475782in}}%
\pgfpathlineto{\pgfqpoint{11.405420in}{5.475782in}}%
\pgfpathclose%
\pgfusepath{stroke}%
\end{pgfscope}%
\begin{pgfscope}%
\pgfpathrectangle{\pgfqpoint{7.512535in}{0.437222in}}{\pgfqpoint{6.275590in}{5.159444in}}%
\pgfusepath{clip}%
\pgfsetbuttcap%
\pgfsetroundjoin%
\pgfsetlinewidth{1.003750pt}%
\definecolor{currentstroke}{rgb}{0.827451,0.827451,0.827451}%
\pgfsetstrokecolor{currentstroke}%
\pgfsetstrokeopacity{0.800000}%
\pgfsetdash{}{0pt}%
\pgfpathmoveto{\pgfqpoint{8.290348in}{1.426234in}}%
\pgfpathcurveto{\pgfqpoint{8.301398in}{1.426234in}}{\pgfqpoint{8.311997in}{1.430624in}}{\pgfqpoint{8.319811in}{1.438438in}}%
\pgfpathcurveto{\pgfqpoint{8.327625in}{1.446251in}}{\pgfqpoint{8.332015in}{1.456850in}}{\pgfqpoint{8.332015in}{1.467900in}}%
\pgfpathcurveto{\pgfqpoint{8.332015in}{1.478951in}}{\pgfqpoint{8.327625in}{1.489550in}}{\pgfqpoint{8.319811in}{1.497363in}}%
\pgfpathcurveto{\pgfqpoint{8.311997in}{1.505177in}}{\pgfqpoint{8.301398in}{1.509567in}}{\pgfqpoint{8.290348in}{1.509567in}}%
\pgfpathcurveto{\pgfqpoint{8.279298in}{1.509567in}}{\pgfqpoint{8.268699in}{1.505177in}}{\pgfqpoint{8.260886in}{1.497363in}}%
\pgfpathcurveto{\pgfqpoint{8.253072in}{1.489550in}}{\pgfqpoint{8.248682in}{1.478951in}}{\pgfqpoint{8.248682in}{1.467900in}}%
\pgfpathcurveto{\pgfqpoint{8.248682in}{1.456850in}}{\pgfqpoint{8.253072in}{1.446251in}}{\pgfqpoint{8.260886in}{1.438438in}}%
\pgfpathcurveto{\pgfqpoint{8.268699in}{1.430624in}}{\pgfqpoint{8.279298in}{1.426234in}}{\pgfqpoint{8.290348in}{1.426234in}}%
\pgfpathlineto{\pgfqpoint{8.290348in}{1.426234in}}%
\pgfpathclose%
\pgfusepath{stroke}%
\end{pgfscope}%
\begin{pgfscope}%
\pgfpathrectangle{\pgfqpoint{7.512535in}{0.437222in}}{\pgfqpoint{6.275590in}{5.159444in}}%
\pgfusepath{clip}%
\pgfsetbuttcap%
\pgfsetroundjoin%
\pgfsetlinewidth{1.003750pt}%
\definecolor{currentstroke}{rgb}{0.827451,0.827451,0.827451}%
\pgfsetstrokecolor{currentstroke}%
\pgfsetstrokeopacity{0.800000}%
\pgfsetdash{}{0pt}%
\pgfpathmoveto{\pgfqpoint{8.716039in}{3.385314in}}%
\pgfpathcurveto{\pgfqpoint{8.727089in}{3.385314in}}{\pgfqpoint{8.737688in}{3.389704in}}{\pgfqpoint{8.745502in}{3.397518in}}%
\pgfpathcurveto{\pgfqpoint{8.753315in}{3.405332in}}{\pgfqpoint{8.757706in}{3.415931in}}{\pgfqpoint{8.757706in}{3.426981in}}%
\pgfpathcurveto{\pgfqpoint{8.757706in}{3.438031in}}{\pgfqpoint{8.753315in}{3.448630in}}{\pgfqpoint{8.745502in}{3.456444in}}%
\pgfpathcurveto{\pgfqpoint{8.737688in}{3.464257in}}{\pgfqpoint{8.727089in}{3.468647in}}{\pgfqpoint{8.716039in}{3.468647in}}%
\pgfpathcurveto{\pgfqpoint{8.704989in}{3.468647in}}{\pgfqpoint{8.694390in}{3.464257in}}{\pgfqpoint{8.686576in}{3.456444in}}%
\pgfpathcurveto{\pgfqpoint{8.678763in}{3.448630in}}{\pgfqpoint{8.674372in}{3.438031in}}{\pgfqpoint{8.674372in}{3.426981in}}%
\pgfpathcurveto{\pgfqpoint{8.674372in}{3.415931in}}{\pgfqpoint{8.678763in}{3.405332in}}{\pgfqpoint{8.686576in}{3.397518in}}%
\pgfpathcurveto{\pgfqpoint{8.694390in}{3.389704in}}{\pgfqpoint{8.704989in}{3.385314in}}{\pgfqpoint{8.716039in}{3.385314in}}%
\pgfpathlineto{\pgfqpoint{8.716039in}{3.385314in}}%
\pgfpathclose%
\pgfusepath{stroke}%
\end{pgfscope}%
\begin{pgfscope}%
\pgfpathrectangle{\pgfqpoint{7.512535in}{0.437222in}}{\pgfqpoint{6.275590in}{5.159444in}}%
\pgfusepath{clip}%
\pgfsetbuttcap%
\pgfsetroundjoin%
\pgfsetlinewidth{1.003750pt}%
\definecolor{currentstroke}{rgb}{0.827451,0.827451,0.827451}%
\pgfsetstrokecolor{currentstroke}%
\pgfsetstrokeopacity{0.800000}%
\pgfsetdash{}{0pt}%
\pgfpathmoveto{\pgfqpoint{9.395327in}{2.124368in}}%
\pgfpathcurveto{\pgfqpoint{9.406377in}{2.124368in}}{\pgfqpoint{9.416976in}{2.128758in}}{\pgfqpoint{9.424790in}{2.136572in}}%
\pgfpathcurveto{\pgfqpoint{9.432604in}{2.144385in}}{\pgfqpoint{9.436994in}{2.154984in}}{\pgfqpoint{9.436994in}{2.166034in}}%
\pgfpathcurveto{\pgfqpoint{9.436994in}{2.177084in}}{\pgfqpoint{9.432604in}{2.187684in}}{\pgfqpoint{9.424790in}{2.195497in}}%
\pgfpathcurveto{\pgfqpoint{9.416976in}{2.203311in}}{\pgfqpoint{9.406377in}{2.207701in}}{\pgfqpoint{9.395327in}{2.207701in}}%
\pgfpathcurveto{\pgfqpoint{9.384277in}{2.207701in}}{\pgfqpoint{9.373678in}{2.203311in}}{\pgfqpoint{9.365865in}{2.195497in}}%
\pgfpathcurveto{\pgfqpoint{9.358051in}{2.187684in}}{\pgfqpoint{9.353661in}{2.177084in}}{\pgfqpoint{9.353661in}{2.166034in}}%
\pgfpathcurveto{\pgfqpoint{9.353661in}{2.154984in}}{\pgfqpoint{9.358051in}{2.144385in}}{\pgfqpoint{9.365865in}{2.136572in}}%
\pgfpathcurveto{\pgfqpoint{9.373678in}{2.128758in}}{\pgfqpoint{9.384277in}{2.124368in}}{\pgfqpoint{9.395327in}{2.124368in}}%
\pgfpathlineto{\pgfqpoint{9.395327in}{2.124368in}}%
\pgfpathclose%
\pgfusepath{stroke}%
\end{pgfscope}%
\begin{pgfscope}%
\pgfpathrectangle{\pgfqpoint{7.512535in}{0.437222in}}{\pgfqpoint{6.275590in}{5.159444in}}%
\pgfusepath{clip}%
\pgfsetbuttcap%
\pgfsetroundjoin%
\pgfsetlinewidth{1.003750pt}%
\definecolor{currentstroke}{rgb}{0.827451,0.827451,0.827451}%
\pgfsetstrokecolor{currentstroke}%
\pgfsetstrokeopacity{0.800000}%
\pgfsetdash{}{0pt}%
\pgfpathmoveto{\pgfqpoint{7.653772in}{0.965972in}}%
\pgfpathcurveto{\pgfqpoint{7.664822in}{0.965972in}}{\pgfqpoint{7.675421in}{0.970363in}}{\pgfqpoint{7.683235in}{0.978176in}}%
\pgfpathcurveto{\pgfqpoint{7.691048in}{0.985990in}}{\pgfqpoint{7.695439in}{0.996589in}}{\pgfqpoint{7.695439in}{1.007639in}}%
\pgfpathcurveto{\pgfqpoint{7.695439in}{1.018689in}}{\pgfqpoint{7.691048in}{1.029288in}}{\pgfqpoint{7.683235in}{1.037102in}}%
\pgfpathcurveto{\pgfqpoint{7.675421in}{1.044915in}}{\pgfqpoint{7.664822in}{1.049306in}}{\pgfqpoint{7.653772in}{1.049306in}}%
\pgfpathcurveto{\pgfqpoint{7.642722in}{1.049306in}}{\pgfqpoint{7.632123in}{1.044915in}}{\pgfqpoint{7.624309in}{1.037102in}}%
\pgfpathcurveto{\pgfqpoint{7.616496in}{1.029288in}}{\pgfqpoint{7.612105in}{1.018689in}}{\pgfqpoint{7.612105in}{1.007639in}}%
\pgfpathcurveto{\pgfqpoint{7.612105in}{0.996589in}}{\pgfqpoint{7.616496in}{0.985990in}}{\pgfqpoint{7.624309in}{0.978176in}}%
\pgfpathcurveto{\pgfqpoint{7.632123in}{0.970363in}}{\pgfqpoint{7.642722in}{0.965972in}}{\pgfqpoint{7.653772in}{0.965972in}}%
\pgfpathlineto{\pgfqpoint{7.653772in}{0.965972in}}%
\pgfpathclose%
\pgfusepath{stroke}%
\end{pgfscope}%
\begin{pgfscope}%
\pgfpathrectangle{\pgfqpoint{7.512535in}{0.437222in}}{\pgfqpoint{6.275590in}{5.159444in}}%
\pgfusepath{clip}%
\pgfsetbuttcap%
\pgfsetroundjoin%
\pgfsetlinewidth{1.003750pt}%
\definecolor{currentstroke}{rgb}{0.827451,0.827451,0.827451}%
\pgfsetstrokecolor{currentstroke}%
\pgfsetstrokeopacity{0.800000}%
\pgfsetdash{}{0pt}%
\pgfpathmoveto{\pgfqpoint{8.983729in}{3.220234in}}%
\pgfpathcurveto{\pgfqpoint{8.994779in}{3.220234in}}{\pgfqpoint{9.005378in}{3.224624in}}{\pgfqpoint{9.013191in}{3.232438in}}%
\pgfpathcurveto{\pgfqpoint{9.021005in}{3.240251in}}{\pgfqpoint{9.025395in}{3.250850in}}{\pgfqpoint{9.025395in}{3.261901in}}%
\pgfpathcurveto{\pgfqpoint{9.025395in}{3.272951in}}{\pgfqpoint{9.021005in}{3.283550in}}{\pgfqpoint{9.013191in}{3.291363in}}%
\pgfpathcurveto{\pgfqpoint{9.005378in}{3.299177in}}{\pgfqpoint{8.994779in}{3.303567in}}{\pgfqpoint{8.983729in}{3.303567in}}%
\pgfpathcurveto{\pgfqpoint{8.972679in}{3.303567in}}{\pgfqpoint{8.962079in}{3.299177in}}{\pgfqpoint{8.954266in}{3.291363in}}%
\pgfpathcurveto{\pgfqpoint{8.946452in}{3.283550in}}{\pgfqpoint{8.942062in}{3.272951in}}{\pgfqpoint{8.942062in}{3.261901in}}%
\pgfpathcurveto{\pgfqpoint{8.942062in}{3.250850in}}{\pgfqpoint{8.946452in}{3.240251in}}{\pgfqpoint{8.954266in}{3.232438in}}%
\pgfpathcurveto{\pgfqpoint{8.962079in}{3.224624in}}{\pgfqpoint{8.972679in}{3.220234in}}{\pgfqpoint{8.983729in}{3.220234in}}%
\pgfpathlineto{\pgfqpoint{8.983729in}{3.220234in}}%
\pgfpathclose%
\pgfusepath{stroke}%
\end{pgfscope}%
\begin{pgfscope}%
\pgfpathrectangle{\pgfqpoint{7.512535in}{0.437222in}}{\pgfqpoint{6.275590in}{5.159444in}}%
\pgfusepath{clip}%
\pgfsetbuttcap%
\pgfsetroundjoin%
\pgfsetlinewidth{1.003750pt}%
\definecolor{currentstroke}{rgb}{0.827451,0.827451,0.827451}%
\pgfsetstrokecolor{currentstroke}%
\pgfsetstrokeopacity{0.800000}%
\pgfsetdash{}{0pt}%
\pgfpathmoveto{\pgfqpoint{10.150452in}{4.500549in}}%
\pgfpathcurveto{\pgfqpoint{10.161502in}{4.500549in}}{\pgfqpoint{10.172101in}{4.504939in}}{\pgfqpoint{10.179914in}{4.512753in}}%
\pgfpathcurveto{\pgfqpoint{10.187728in}{4.520566in}}{\pgfqpoint{10.192118in}{4.531165in}}{\pgfqpoint{10.192118in}{4.542215in}}%
\pgfpathcurveto{\pgfqpoint{10.192118in}{4.553266in}}{\pgfqpoint{10.187728in}{4.563865in}}{\pgfqpoint{10.179914in}{4.571678in}}%
\pgfpathcurveto{\pgfqpoint{10.172101in}{4.579492in}}{\pgfqpoint{10.161502in}{4.583882in}}{\pgfqpoint{10.150452in}{4.583882in}}%
\pgfpathcurveto{\pgfqpoint{10.139401in}{4.583882in}}{\pgfqpoint{10.128802in}{4.579492in}}{\pgfqpoint{10.120989in}{4.571678in}}%
\pgfpathcurveto{\pgfqpoint{10.113175in}{4.563865in}}{\pgfqpoint{10.108785in}{4.553266in}}{\pgfqpoint{10.108785in}{4.542215in}}%
\pgfpathcurveto{\pgfqpoint{10.108785in}{4.531165in}}{\pgfqpoint{10.113175in}{4.520566in}}{\pgfqpoint{10.120989in}{4.512753in}}%
\pgfpathcurveto{\pgfqpoint{10.128802in}{4.504939in}}{\pgfqpoint{10.139401in}{4.500549in}}{\pgfqpoint{10.150452in}{4.500549in}}%
\pgfpathlineto{\pgfqpoint{10.150452in}{4.500549in}}%
\pgfpathclose%
\pgfusepath{stroke}%
\end{pgfscope}%
\begin{pgfscope}%
\pgfpathrectangle{\pgfqpoint{7.512535in}{0.437222in}}{\pgfqpoint{6.275590in}{5.159444in}}%
\pgfusepath{clip}%
\pgfsetbuttcap%
\pgfsetroundjoin%
\pgfsetlinewidth{1.003750pt}%
\definecolor{currentstroke}{rgb}{0.827451,0.827451,0.827451}%
\pgfsetstrokecolor{currentstroke}%
\pgfsetstrokeopacity{0.800000}%
\pgfsetdash{}{0pt}%
\pgfpathmoveto{\pgfqpoint{10.486066in}{4.589271in}}%
\pgfpathcurveto{\pgfqpoint{10.497116in}{4.589271in}}{\pgfqpoint{10.507715in}{4.593662in}}{\pgfqpoint{10.515529in}{4.601475in}}%
\pgfpathcurveto{\pgfqpoint{10.523342in}{4.609289in}}{\pgfqpoint{10.527733in}{4.619888in}}{\pgfqpoint{10.527733in}{4.630938in}}%
\pgfpathcurveto{\pgfqpoint{10.527733in}{4.641988in}}{\pgfqpoint{10.523342in}{4.652587in}}{\pgfqpoint{10.515529in}{4.660401in}}%
\pgfpathcurveto{\pgfqpoint{10.507715in}{4.668214in}}{\pgfqpoint{10.497116in}{4.672605in}}{\pgfqpoint{10.486066in}{4.672605in}}%
\pgfpathcurveto{\pgfqpoint{10.475016in}{4.672605in}}{\pgfqpoint{10.464417in}{4.668214in}}{\pgfqpoint{10.456603in}{4.660401in}}%
\pgfpathcurveto{\pgfqpoint{10.448790in}{4.652587in}}{\pgfqpoint{10.444399in}{4.641988in}}{\pgfqpoint{10.444399in}{4.630938in}}%
\pgfpathcurveto{\pgfqpoint{10.444399in}{4.619888in}}{\pgfqpoint{10.448790in}{4.609289in}}{\pgfqpoint{10.456603in}{4.601475in}}%
\pgfpathcurveto{\pgfqpoint{10.464417in}{4.593662in}}{\pgfqpoint{10.475016in}{4.589271in}}{\pgfqpoint{10.486066in}{4.589271in}}%
\pgfpathlineto{\pgfqpoint{10.486066in}{4.589271in}}%
\pgfpathclose%
\pgfusepath{stroke}%
\end{pgfscope}%
\begin{pgfscope}%
\pgfpathrectangle{\pgfqpoint{7.512535in}{0.437222in}}{\pgfqpoint{6.275590in}{5.159444in}}%
\pgfusepath{clip}%
\pgfsetbuttcap%
\pgfsetroundjoin%
\pgfsetlinewidth{1.003750pt}%
\definecolor{currentstroke}{rgb}{0.827451,0.827451,0.827451}%
\pgfsetstrokecolor{currentstroke}%
\pgfsetstrokeopacity{0.800000}%
\pgfsetdash{}{0pt}%
\pgfpathmoveto{\pgfqpoint{9.837488in}{4.056499in}}%
\pgfpathcurveto{\pgfqpoint{9.848539in}{4.056499in}}{\pgfqpoint{9.859138in}{4.060889in}}{\pgfqpoint{9.866951in}{4.068703in}}%
\pgfpathcurveto{\pgfqpoint{9.874765in}{4.076516in}}{\pgfqpoint{9.879155in}{4.087115in}}{\pgfqpoint{9.879155in}{4.098166in}}%
\pgfpathcurveto{\pgfqpoint{9.879155in}{4.109216in}}{\pgfqpoint{9.874765in}{4.119815in}}{\pgfqpoint{9.866951in}{4.127628in}}%
\pgfpathcurveto{\pgfqpoint{9.859138in}{4.135442in}}{\pgfqpoint{9.848539in}{4.139832in}}{\pgfqpoint{9.837488in}{4.139832in}}%
\pgfpathcurveto{\pgfqpoint{9.826438in}{4.139832in}}{\pgfqpoint{9.815839in}{4.135442in}}{\pgfqpoint{9.808026in}{4.127628in}}%
\pgfpathcurveto{\pgfqpoint{9.800212in}{4.119815in}}{\pgfqpoint{9.795822in}{4.109216in}}{\pgfqpoint{9.795822in}{4.098166in}}%
\pgfpathcurveto{\pgfqpoint{9.795822in}{4.087115in}}{\pgfqpoint{9.800212in}{4.076516in}}{\pgfqpoint{9.808026in}{4.068703in}}%
\pgfpathcurveto{\pgfqpoint{9.815839in}{4.060889in}}{\pgfqpoint{9.826438in}{4.056499in}}{\pgfqpoint{9.837488in}{4.056499in}}%
\pgfpathlineto{\pgfqpoint{9.837488in}{4.056499in}}%
\pgfpathclose%
\pgfusepath{stroke}%
\end{pgfscope}%
\begin{pgfscope}%
\pgfpathrectangle{\pgfqpoint{7.512535in}{0.437222in}}{\pgfqpoint{6.275590in}{5.159444in}}%
\pgfusepath{clip}%
\pgfsetbuttcap%
\pgfsetroundjoin%
\pgfsetlinewidth{1.003750pt}%
\definecolor{currentstroke}{rgb}{0.827451,0.827451,0.827451}%
\pgfsetstrokecolor{currentstroke}%
\pgfsetstrokeopacity{0.800000}%
\pgfsetdash{}{0pt}%
\pgfpathmoveto{\pgfqpoint{11.405151in}{5.189667in}}%
\pgfpathcurveto{\pgfqpoint{11.416201in}{5.189667in}}{\pgfqpoint{11.426800in}{5.194058in}}{\pgfqpoint{11.434613in}{5.201871in}}%
\pgfpathcurveto{\pgfqpoint{11.442427in}{5.209685in}}{\pgfqpoint{11.446817in}{5.220284in}}{\pgfqpoint{11.446817in}{5.231334in}}%
\pgfpathcurveto{\pgfqpoint{11.446817in}{5.242384in}}{\pgfqpoint{11.442427in}{5.252983in}}{\pgfqpoint{11.434613in}{5.260797in}}%
\pgfpathcurveto{\pgfqpoint{11.426800in}{5.268610in}}{\pgfqpoint{11.416201in}{5.273001in}}{\pgfqpoint{11.405151in}{5.273001in}}%
\pgfpathcurveto{\pgfqpoint{11.394100in}{5.273001in}}{\pgfqpoint{11.383501in}{5.268610in}}{\pgfqpoint{11.375688in}{5.260797in}}%
\pgfpathcurveto{\pgfqpoint{11.367874in}{5.252983in}}{\pgfqpoint{11.363484in}{5.242384in}}{\pgfqpoint{11.363484in}{5.231334in}}%
\pgfpathcurveto{\pgfqpoint{11.363484in}{5.220284in}}{\pgfqpoint{11.367874in}{5.209685in}}{\pgfqpoint{11.375688in}{5.201871in}}%
\pgfpathcurveto{\pgfqpoint{11.383501in}{5.194058in}}{\pgfqpoint{11.394100in}{5.189667in}}{\pgfqpoint{11.405151in}{5.189667in}}%
\pgfpathlineto{\pgfqpoint{11.405151in}{5.189667in}}%
\pgfpathclose%
\pgfusepath{stroke}%
\end{pgfscope}%
\begin{pgfscope}%
\pgfpathrectangle{\pgfqpoint{7.512535in}{0.437222in}}{\pgfqpoint{6.275590in}{5.159444in}}%
\pgfusepath{clip}%
\pgfsetbuttcap%
\pgfsetroundjoin%
\pgfsetlinewidth{1.003750pt}%
\definecolor{currentstroke}{rgb}{0.827451,0.827451,0.827451}%
\pgfsetstrokecolor{currentstroke}%
\pgfsetstrokeopacity{0.800000}%
\pgfsetdash{}{0pt}%
\pgfpathmoveto{\pgfqpoint{8.978070in}{1.743970in}}%
\pgfpathcurveto{\pgfqpoint{8.989120in}{1.743970in}}{\pgfqpoint{8.999719in}{1.748360in}}{\pgfqpoint{9.007533in}{1.756174in}}%
\pgfpathcurveto{\pgfqpoint{9.015346in}{1.763988in}}{\pgfqpoint{9.019736in}{1.774587in}}{\pgfqpoint{9.019736in}{1.785637in}}%
\pgfpathcurveto{\pgfqpoint{9.019736in}{1.796687in}}{\pgfqpoint{9.015346in}{1.807286in}}{\pgfqpoint{9.007533in}{1.815100in}}%
\pgfpathcurveto{\pgfqpoint{8.999719in}{1.822913in}}{\pgfqpoint{8.989120in}{1.827303in}}{\pgfqpoint{8.978070in}{1.827303in}}%
\pgfpathcurveto{\pgfqpoint{8.967020in}{1.827303in}}{\pgfqpoint{8.956421in}{1.822913in}}{\pgfqpoint{8.948607in}{1.815100in}}%
\pgfpathcurveto{\pgfqpoint{8.940793in}{1.807286in}}{\pgfqpoint{8.936403in}{1.796687in}}{\pgfqpoint{8.936403in}{1.785637in}}%
\pgfpathcurveto{\pgfqpoint{8.936403in}{1.774587in}}{\pgfqpoint{8.940793in}{1.763988in}}{\pgfqpoint{8.948607in}{1.756174in}}%
\pgfpathcurveto{\pgfqpoint{8.956421in}{1.748360in}}{\pgfqpoint{8.967020in}{1.743970in}}{\pgfqpoint{8.978070in}{1.743970in}}%
\pgfpathlineto{\pgfqpoint{8.978070in}{1.743970in}}%
\pgfpathclose%
\pgfusepath{stroke}%
\end{pgfscope}%
\begin{pgfscope}%
\pgfpathrectangle{\pgfqpoint{7.512535in}{0.437222in}}{\pgfqpoint{6.275590in}{5.159444in}}%
\pgfusepath{clip}%
\pgfsetbuttcap%
\pgfsetroundjoin%
\pgfsetlinewidth{1.003750pt}%
\definecolor{currentstroke}{rgb}{0.827451,0.827451,0.827451}%
\pgfsetstrokecolor{currentstroke}%
\pgfsetstrokeopacity{0.800000}%
\pgfsetdash{}{0pt}%
\pgfpathmoveto{\pgfqpoint{9.697299in}{3.766615in}}%
\pgfpathcurveto{\pgfqpoint{9.708349in}{3.766615in}}{\pgfqpoint{9.718948in}{3.771005in}}{\pgfqpoint{9.726762in}{3.778819in}}%
\pgfpathcurveto{\pgfqpoint{9.734575in}{3.786633in}}{\pgfqpoint{9.738965in}{3.797232in}}{\pgfqpoint{9.738965in}{3.808282in}}%
\pgfpathcurveto{\pgfqpoint{9.738965in}{3.819332in}}{\pgfqpoint{9.734575in}{3.829931in}}{\pgfqpoint{9.726762in}{3.837745in}}%
\pgfpathcurveto{\pgfqpoint{9.718948in}{3.845558in}}{\pgfqpoint{9.708349in}{3.849949in}}{\pgfqpoint{9.697299in}{3.849949in}}%
\pgfpathcurveto{\pgfqpoint{9.686249in}{3.849949in}}{\pgfqpoint{9.675650in}{3.845558in}}{\pgfqpoint{9.667836in}{3.837745in}}%
\pgfpathcurveto{\pgfqpoint{9.660022in}{3.829931in}}{\pgfqpoint{9.655632in}{3.819332in}}{\pgfqpoint{9.655632in}{3.808282in}}%
\pgfpathcurveto{\pgfqpoint{9.655632in}{3.797232in}}{\pgfqpoint{9.660022in}{3.786633in}}{\pgfqpoint{9.667836in}{3.778819in}}%
\pgfpathcurveto{\pgfqpoint{9.675650in}{3.771005in}}{\pgfqpoint{9.686249in}{3.766615in}}{\pgfqpoint{9.697299in}{3.766615in}}%
\pgfpathlineto{\pgfqpoint{9.697299in}{3.766615in}}%
\pgfpathclose%
\pgfusepath{stroke}%
\end{pgfscope}%
\begin{pgfscope}%
\pgfpathrectangle{\pgfqpoint{7.512535in}{0.437222in}}{\pgfqpoint{6.275590in}{5.159444in}}%
\pgfusepath{clip}%
\pgfsetbuttcap%
\pgfsetroundjoin%
\pgfsetlinewidth{1.003750pt}%
\definecolor{currentstroke}{rgb}{0.827451,0.827451,0.827451}%
\pgfsetstrokecolor{currentstroke}%
\pgfsetstrokeopacity{0.800000}%
\pgfsetdash{}{0pt}%
\pgfpathmoveto{\pgfqpoint{8.290348in}{1.139798in}}%
\pgfpathcurveto{\pgfqpoint{8.301398in}{1.139798in}}{\pgfqpoint{8.311997in}{1.144188in}}{\pgfqpoint{8.319811in}{1.152002in}}%
\pgfpathcurveto{\pgfqpoint{8.327625in}{1.159815in}}{\pgfqpoint{8.332015in}{1.170414in}}{\pgfqpoint{8.332015in}{1.181465in}}%
\pgfpathcurveto{\pgfqpoint{8.332015in}{1.192515in}}{\pgfqpoint{8.327625in}{1.203114in}}{\pgfqpoint{8.319811in}{1.210927in}}%
\pgfpathcurveto{\pgfqpoint{8.311997in}{1.218741in}}{\pgfqpoint{8.301398in}{1.223131in}}{\pgfqpoint{8.290348in}{1.223131in}}%
\pgfpathcurveto{\pgfqpoint{8.279298in}{1.223131in}}{\pgfqpoint{8.268699in}{1.218741in}}{\pgfqpoint{8.260886in}{1.210927in}}%
\pgfpathcurveto{\pgfqpoint{8.253072in}{1.203114in}}{\pgfqpoint{8.248682in}{1.192515in}}{\pgfqpoint{8.248682in}{1.181465in}}%
\pgfpathcurveto{\pgfqpoint{8.248682in}{1.170414in}}{\pgfqpoint{8.253072in}{1.159815in}}{\pgfqpoint{8.260886in}{1.152002in}}%
\pgfpathcurveto{\pgfqpoint{8.268699in}{1.144188in}}{\pgfqpoint{8.279298in}{1.139798in}}{\pgfqpoint{8.290348in}{1.139798in}}%
\pgfpathlineto{\pgfqpoint{8.290348in}{1.139798in}}%
\pgfpathclose%
\pgfusepath{stroke}%
\end{pgfscope}%
\begin{pgfscope}%
\pgfpathrectangle{\pgfqpoint{7.512535in}{0.437222in}}{\pgfqpoint{6.275590in}{5.159444in}}%
\pgfusepath{clip}%
\pgfsetbuttcap%
\pgfsetroundjoin%
\pgfsetlinewidth{1.003750pt}%
\definecolor{currentstroke}{rgb}{0.827451,0.827451,0.827451}%
\pgfsetstrokecolor{currentstroke}%
\pgfsetstrokeopacity{0.800000}%
\pgfsetdash{}{0pt}%
\pgfpathmoveto{\pgfqpoint{8.624162in}{1.722485in}}%
\pgfpathcurveto{\pgfqpoint{8.635212in}{1.722485in}}{\pgfqpoint{8.645811in}{1.726876in}}{\pgfqpoint{8.653624in}{1.734689in}}%
\pgfpathcurveto{\pgfqpoint{8.661438in}{1.742503in}}{\pgfqpoint{8.665828in}{1.753102in}}{\pgfqpoint{8.665828in}{1.764152in}}%
\pgfpathcurveto{\pgfqpoint{8.665828in}{1.775202in}}{\pgfqpoint{8.661438in}{1.785801in}}{\pgfqpoint{8.653624in}{1.793615in}}%
\pgfpathcurveto{\pgfqpoint{8.645811in}{1.801428in}}{\pgfqpoint{8.635212in}{1.805819in}}{\pgfqpoint{8.624162in}{1.805819in}}%
\pgfpathcurveto{\pgfqpoint{8.613112in}{1.805819in}}{\pgfqpoint{8.602513in}{1.801428in}}{\pgfqpoint{8.594699in}{1.793615in}}%
\pgfpathcurveto{\pgfqpoint{8.586885in}{1.785801in}}{\pgfqpoint{8.582495in}{1.775202in}}{\pgfqpoint{8.582495in}{1.764152in}}%
\pgfpathcurveto{\pgfqpoint{8.582495in}{1.753102in}}{\pgfqpoint{8.586885in}{1.742503in}}{\pgfqpoint{8.594699in}{1.734689in}}%
\pgfpathcurveto{\pgfqpoint{8.602513in}{1.726876in}}{\pgfqpoint{8.613112in}{1.722485in}}{\pgfqpoint{8.624162in}{1.722485in}}%
\pgfpathlineto{\pgfqpoint{8.624162in}{1.722485in}}%
\pgfpathclose%
\pgfusepath{stroke}%
\end{pgfscope}%
\begin{pgfscope}%
\pgfpathrectangle{\pgfqpoint{7.512535in}{0.437222in}}{\pgfqpoint{6.275590in}{5.159444in}}%
\pgfusepath{clip}%
\pgfsetbuttcap%
\pgfsetroundjoin%
\pgfsetlinewidth{1.003750pt}%
\definecolor{currentstroke}{rgb}{0.827451,0.827451,0.827451}%
\pgfsetstrokecolor{currentstroke}%
\pgfsetstrokeopacity{0.800000}%
\pgfsetdash{}{0pt}%
\pgfpathmoveto{\pgfqpoint{12.873966in}{5.431219in}}%
\pgfpathcurveto{\pgfqpoint{12.885016in}{5.431219in}}{\pgfqpoint{12.895615in}{5.435610in}}{\pgfqpoint{12.903429in}{5.443423in}}%
\pgfpathcurveto{\pgfqpoint{12.911243in}{5.451237in}}{\pgfqpoint{12.915633in}{5.461836in}}{\pgfqpoint{12.915633in}{5.472886in}}%
\pgfpathcurveto{\pgfqpoint{12.915633in}{5.483936in}}{\pgfqpoint{12.911243in}{5.494535in}}{\pgfqpoint{12.903429in}{5.502349in}}%
\pgfpathcurveto{\pgfqpoint{12.895615in}{5.510163in}}{\pgfqpoint{12.885016in}{5.514553in}}{\pgfqpoint{12.873966in}{5.514553in}}%
\pgfpathcurveto{\pgfqpoint{12.862916in}{5.514553in}}{\pgfqpoint{12.852317in}{5.510163in}}{\pgfqpoint{12.844503in}{5.502349in}}%
\pgfpathcurveto{\pgfqpoint{12.836690in}{5.494535in}}{\pgfqpoint{12.832300in}{5.483936in}}{\pgfqpoint{12.832300in}{5.472886in}}%
\pgfpathcurveto{\pgfqpoint{12.832300in}{5.461836in}}{\pgfqpoint{12.836690in}{5.451237in}}{\pgfqpoint{12.844503in}{5.443423in}}%
\pgfpathcurveto{\pgfqpoint{12.852317in}{5.435610in}}{\pgfqpoint{12.862916in}{5.431219in}}{\pgfqpoint{12.873966in}{5.431219in}}%
\pgfpathlineto{\pgfqpoint{12.873966in}{5.431219in}}%
\pgfpathclose%
\pgfusepath{stroke}%
\end{pgfscope}%
\begin{pgfscope}%
\pgfpathrectangle{\pgfqpoint{7.512535in}{0.437222in}}{\pgfqpoint{6.275590in}{5.159444in}}%
\pgfusepath{clip}%
\pgfsetbuttcap%
\pgfsetroundjoin%
\pgfsetlinewidth{1.003750pt}%
\definecolor{currentstroke}{rgb}{0.827451,0.827451,0.827451}%
\pgfsetstrokecolor{currentstroke}%
\pgfsetstrokeopacity{0.800000}%
\pgfsetdash{}{0pt}%
\pgfpathmoveto{\pgfqpoint{7.931237in}{0.493855in}}%
\pgfpathcurveto{\pgfqpoint{7.942288in}{0.493855in}}{\pgfqpoint{7.952887in}{0.498245in}}{\pgfqpoint{7.960700in}{0.506059in}}%
\pgfpathcurveto{\pgfqpoint{7.968514in}{0.513872in}}{\pgfqpoint{7.972904in}{0.524471in}}{\pgfqpoint{7.972904in}{0.535522in}}%
\pgfpathcurveto{\pgfqpoint{7.972904in}{0.546572in}}{\pgfqpoint{7.968514in}{0.557171in}}{\pgfqpoint{7.960700in}{0.564984in}}%
\pgfpathcurveto{\pgfqpoint{7.952887in}{0.572798in}}{\pgfqpoint{7.942288in}{0.577188in}}{\pgfqpoint{7.931237in}{0.577188in}}%
\pgfpathcurveto{\pgfqpoint{7.920187in}{0.577188in}}{\pgfqpoint{7.909588in}{0.572798in}}{\pgfqpoint{7.901775in}{0.564984in}}%
\pgfpathcurveto{\pgfqpoint{7.893961in}{0.557171in}}{\pgfqpoint{7.889571in}{0.546572in}}{\pgfqpoint{7.889571in}{0.535522in}}%
\pgfpathcurveto{\pgfqpoint{7.889571in}{0.524471in}}{\pgfqpoint{7.893961in}{0.513872in}}{\pgfqpoint{7.901775in}{0.506059in}}%
\pgfpathcurveto{\pgfqpoint{7.909588in}{0.498245in}}{\pgfqpoint{7.920187in}{0.493855in}}{\pgfqpoint{7.931237in}{0.493855in}}%
\pgfpathlineto{\pgfqpoint{7.931237in}{0.493855in}}%
\pgfpathclose%
\pgfusepath{stroke}%
\end{pgfscope}%
\begin{pgfscope}%
\pgfpathrectangle{\pgfqpoint{7.512535in}{0.437222in}}{\pgfqpoint{6.275590in}{5.159444in}}%
\pgfusepath{clip}%
\pgfsetbuttcap%
\pgfsetroundjoin%
\pgfsetlinewidth{1.003750pt}%
\definecolor{currentstroke}{rgb}{0.827451,0.827451,0.827451}%
\pgfsetstrokecolor{currentstroke}%
\pgfsetstrokeopacity{0.800000}%
\pgfsetdash{}{0pt}%
\pgfpathmoveto{\pgfqpoint{7.974778in}{1.314099in}}%
\pgfpathcurveto{\pgfqpoint{7.985828in}{1.314099in}}{\pgfqpoint{7.996427in}{1.318489in}}{\pgfqpoint{8.004241in}{1.326303in}}%
\pgfpathcurveto{\pgfqpoint{8.012055in}{1.334116in}}{\pgfqpoint{8.016445in}{1.344715in}}{\pgfqpoint{8.016445in}{1.355766in}}%
\pgfpathcurveto{\pgfqpoint{8.016445in}{1.366816in}}{\pgfqpoint{8.012055in}{1.377415in}}{\pgfqpoint{8.004241in}{1.385228in}}%
\pgfpathcurveto{\pgfqpoint{7.996427in}{1.393042in}}{\pgfqpoint{7.985828in}{1.397432in}}{\pgfqpoint{7.974778in}{1.397432in}}%
\pgfpathcurveto{\pgfqpoint{7.963728in}{1.397432in}}{\pgfqpoint{7.953129in}{1.393042in}}{\pgfqpoint{7.945315in}{1.385228in}}%
\pgfpathcurveto{\pgfqpoint{7.937502in}{1.377415in}}{\pgfqpoint{7.933112in}{1.366816in}}{\pgfqpoint{7.933112in}{1.355766in}}%
\pgfpathcurveto{\pgfqpoint{7.933112in}{1.344715in}}{\pgfqpoint{7.937502in}{1.334116in}}{\pgfqpoint{7.945315in}{1.326303in}}%
\pgfpathcurveto{\pgfqpoint{7.953129in}{1.318489in}}{\pgfqpoint{7.963728in}{1.314099in}}{\pgfqpoint{7.974778in}{1.314099in}}%
\pgfpathlineto{\pgfqpoint{7.974778in}{1.314099in}}%
\pgfpathclose%
\pgfusepath{stroke}%
\end{pgfscope}%
\begin{pgfscope}%
\pgfpathrectangle{\pgfqpoint{7.512535in}{0.437222in}}{\pgfqpoint{6.275590in}{5.159444in}}%
\pgfusepath{clip}%
\pgfsetbuttcap%
\pgfsetroundjoin%
\pgfsetlinewidth{1.003750pt}%
\definecolor{currentstroke}{rgb}{0.827451,0.827451,0.827451}%
\pgfsetstrokecolor{currentstroke}%
\pgfsetstrokeopacity{0.800000}%
\pgfsetdash{}{0pt}%
\pgfpathmoveto{\pgfqpoint{9.726466in}{3.345111in}}%
\pgfpathcurveto{\pgfqpoint{9.737516in}{3.345111in}}{\pgfqpoint{9.748115in}{3.349501in}}{\pgfqpoint{9.755929in}{3.357315in}}%
\pgfpathcurveto{\pgfqpoint{9.763743in}{3.365128in}}{\pgfqpoint{9.768133in}{3.375727in}}{\pgfqpoint{9.768133in}{3.386778in}}%
\pgfpathcurveto{\pgfqpoint{9.768133in}{3.397828in}}{\pgfqpoint{9.763743in}{3.408427in}}{\pgfqpoint{9.755929in}{3.416240in}}%
\pgfpathcurveto{\pgfqpoint{9.748115in}{3.424054in}}{\pgfqpoint{9.737516in}{3.428444in}}{\pgfqpoint{9.726466in}{3.428444in}}%
\pgfpathcurveto{\pgfqpoint{9.715416in}{3.428444in}}{\pgfqpoint{9.704817in}{3.424054in}}{\pgfqpoint{9.697004in}{3.416240in}}%
\pgfpathcurveto{\pgfqpoint{9.689190in}{3.408427in}}{\pgfqpoint{9.684800in}{3.397828in}}{\pgfqpoint{9.684800in}{3.386778in}}%
\pgfpathcurveto{\pgfqpoint{9.684800in}{3.375727in}}{\pgfqpoint{9.689190in}{3.365128in}}{\pgfqpoint{9.697004in}{3.357315in}}%
\pgfpathcurveto{\pgfqpoint{9.704817in}{3.349501in}}{\pgfqpoint{9.715416in}{3.345111in}}{\pgfqpoint{9.726466in}{3.345111in}}%
\pgfpathlineto{\pgfqpoint{9.726466in}{3.345111in}}%
\pgfpathclose%
\pgfusepath{stroke}%
\end{pgfscope}%
\begin{pgfscope}%
\pgfpathrectangle{\pgfqpoint{7.512535in}{0.437222in}}{\pgfqpoint{6.275590in}{5.159444in}}%
\pgfusepath{clip}%
\pgfsetbuttcap%
\pgfsetroundjoin%
\pgfsetlinewidth{1.003750pt}%
\definecolor{currentstroke}{rgb}{0.827451,0.827451,0.827451}%
\pgfsetstrokecolor{currentstroke}%
\pgfsetstrokeopacity{0.800000}%
\pgfsetdash{}{0pt}%
\pgfpathmoveto{\pgfqpoint{7.874057in}{1.320145in}}%
\pgfpathcurveto{\pgfqpoint{7.885107in}{1.320145in}}{\pgfqpoint{7.895706in}{1.324535in}}{\pgfqpoint{7.903520in}{1.332349in}}%
\pgfpathcurveto{\pgfqpoint{7.911334in}{1.340162in}}{\pgfqpoint{7.915724in}{1.350761in}}{\pgfqpoint{7.915724in}{1.361811in}}%
\pgfpathcurveto{\pgfqpoint{7.915724in}{1.372862in}}{\pgfqpoint{7.911334in}{1.383461in}}{\pgfqpoint{7.903520in}{1.391274in}}%
\pgfpathcurveto{\pgfqpoint{7.895706in}{1.399088in}}{\pgfqpoint{7.885107in}{1.403478in}}{\pgfqpoint{7.874057in}{1.403478in}}%
\pgfpathcurveto{\pgfqpoint{7.863007in}{1.403478in}}{\pgfqpoint{7.852408in}{1.399088in}}{\pgfqpoint{7.844595in}{1.391274in}}%
\pgfpathcurveto{\pgfqpoint{7.836781in}{1.383461in}}{\pgfqpoint{7.832391in}{1.372862in}}{\pgfqpoint{7.832391in}{1.361811in}}%
\pgfpathcurveto{\pgfqpoint{7.832391in}{1.350761in}}{\pgfqpoint{7.836781in}{1.340162in}}{\pgfqpoint{7.844595in}{1.332349in}}%
\pgfpathcurveto{\pgfqpoint{7.852408in}{1.324535in}}{\pgfqpoint{7.863007in}{1.320145in}}{\pgfqpoint{7.874057in}{1.320145in}}%
\pgfpathlineto{\pgfqpoint{7.874057in}{1.320145in}}%
\pgfpathclose%
\pgfusepath{stroke}%
\end{pgfscope}%
\begin{pgfscope}%
\pgfpathrectangle{\pgfqpoint{7.512535in}{0.437222in}}{\pgfqpoint{6.275590in}{5.159444in}}%
\pgfusepath{clip}%
\pgfsetbuttcap%
\pgfsetroundjoin%
\pgfsetlinewidth{1.003750pt}%
\definecolor{currentstroke}{rgb}{0.827451,0.827451,0.827451}%
\pgfsetstrokecolor{currentstroke}%
\pgfsetstrokeopacity{0.800000}%
\pgfsetdash{}{0pt}%
\pgfpathmoveto{\pgfqpoint{10.790078in}{4.715972in}}%
\pgfpathcurveto{\pgfqpoint{10.801128in}{4.715972in}}{\pgfqpoint{10.811727in}{4.720362in}}{\pgfqpoint{10.819541in}{4.728176in}}%
\pgfpathcurveto{\pgfqpoint{10.827355in}{4.735989in}}{\pgfqpoint{10.831745in}{4.746588in}}{\pgfqpoint{10.831745in}{4.757639in}}%
\pgfpathcurveto{\pgfqpoint{10.831745in}{4.768689in}}{\pgfqpoint{10.827355in}{4.779288in}}{\pgfqpoint{10.819541in}{4.787101in}}%
\pgfpathcurveto{\pgfqpoint{10.811727in}{4.794915in}}{\pgfqpoint{10.801128in}{4.799305in}}{\pgfqpoint{10.790078in}{4.799305in}}%
\pgfpathcurveto{\pgfqpoint{10.779028in}{4.799305in}}{\pgfqpoint{10.768429in}{4.794915in}}{\pgfqpoint{10.760616in}{4.787101in}}%
\pgfpathcurveto{\pgfqpoint{10.752802in}{4.779288in}}{\pgfqpoint{10.748412in}{4.768689in}}{\pgfqpoint{10.748412in}{4.757639in}}%
\pgfpathcurveto{\pgfqpoint{10.748412in}{4.746588in}}{\pgfqpoint{10.752802in}{4.735989in}}{\pgfqpoint{10.760616in}{4.728176in}}%
\pgfpathcurveto{\pgfqpoint{10.768429in}{4.720362in}}{\pgfqpoint{10.779028in}{4.715972in}}{\pgfqpoint{10.790078in}{4.715972in}}%
\pgfpathlineto{\pgfqpoint{10.790078in}{4.715972in}}%
\pgfpathclose%
\pgfusepath{stroke}%
\end{pgfscope}%
\begin{pgfscope}%
\pgfpathrectangle{\pgfqpoint{7.512535in}{0.437222in}}{\pgfqpoint{6.275590in}{5.159444in}}%
\pgfusepath{clip}%
\pgfsetbuttcap%
\pgfsetroundjoin%
\pgfsetlinewidth{1.003750pt}%
\definecolor{currentstroke}{rgb}{0.827451,0.827451,0.827451}%
\pgfsetstrokecolor{currentstroke}%
\pgfsetstrokeopacity{0.800000}%
\pgfsetdash{}{0pt}%
\pgfpathmoveto{\pgfqpoint{10.780511in}{5.205075in}}%
\pgfpathcurveto{\pgfqpoint{10.791561in}{5.205075in}}{\pgfqpoint{10.802160in}{5.209466in}}{\pgfqpoint{10.809973in}{5.217279in}}%
\pgfpathcurveto{\pgfqpoint{10.817787in}{5.225093in}}{\pgfqpoint{10.822177in}{5.235692in}}{\pgfqpoint{10.822177in}{5.246742in}}%
\pgfpathcurveto{\pgfqpoint{10.822177in}{5.257792in}}{\pgfqpoint{10.817787in}{5.268391in}}{\pgfqpoint{10.809973in}{5.276205in}}%
\pgfpathcurveto{\pgfqpoint{10.802160in}{5.284019in}}{\pgfqpoint{10.791561in}{5.288409in}}{\pgfqpoint{10.780511in}{5.288409in}}%
\pgfpathcurveto{\pgfqpoint{10.769461in}{5.288409in}}{\pgfqpoint{10.758862in}{5.284019in}}{\pgfqpoint{10.751048in}{5.276205in}}%
\pgfpathcurveto{\pgfqpoint{10.743234in}{5.268391in}}{\pgfqpoint{10.738844in}{5.257792in}}{\pgfqpoint{10.738844in}{5.246742in}}%
\pgfpathcurveto{\pgfqpoint{10.738844in}{5.235692in}}{\pgfqpoint{10.743234in}{5.225093in}}{\pgfqpoint{10.751048in}{5.217279in}}%
\pgfpathcurveto{\pgfqpoint{10.758862in}{5.209466in}}{\pgfqpoint{10.769461in}{5.205075in}}{\pgfqpoint{10.780511in}{5.205075in}}%
\pgfpathlineto{\pgfqpoint{10.780511in}{5.205075in}}%
\pgfpathclose%
\pgfusepath{stroke}%
\end{pgfscope}%
\begin{pgfscope}%
\pgfpathrectangle{\pgfqpoint{7.512535in}{0.437222in}}{\pgfqpoint{6.275590in}{5.159444in}}%
\pgfusepath{clip}%
\pgfsetbuttcap%
\pgfsetroundjoin%
\pgfsetlinewidth{1.003750pt}%
\definecolor{currentstroke}{rgb}{0.827451,0.827451,0.827451}%
\pgfsetstrokecolor{currentstroke}%
\pgfsetstrokeopacity{0.800000}%
\pgfsetdash{}{0pt}%
\pgfpathmoveto{\pgfqpoint{7.711198in}{0.615504in}}%
\pgfpathcurveto{\pgfqpoint{7.722248in}{0.615504in}}{\pgfqpoint{7.732847in}{0.619894in}}{\pgfqpoint{7.740661in}{0.627708in}}%
\pgfpathcurveto{\pgfqpoint{7.748474in}{0.635522in}}{\pgfqpoint{7.752865in}{0.646121in}}{\pgfqpoint{7.752865in}{0.657171in}}%
\pgfpathcurveto{\pgfqpoint{7.752865in}{0.668221in}}{\pgfqpoint{7.748474in}{0.678820in}}{\pgfqpoint{7.740661in}{0.686633in}}%
\pgfpathcurveto{\pgfqpoint{7.732847in}{0.694447in}}{\pgfqpoint{7.722248in}{0.698837in}}{\pgfqpoint{7.711198in}{0.698837in}}%
\pgfpathcurveto{\pgfqpoint{7.700148in}{0.698837in}}{\pgfqpoint{7.689549in}{0.694447in}}{\pgfqpoint{7.681735in}{0.686633in}}%
\pgfpathcurveto{\pgfqpoint{7.673921in}{0.678820in}}{\pgfqpoint{7.669531in}{0.668221in}}{\pgfqpoint{7.669531in}{0.657171in}}%
\pgfpathcurveto{\pgfqpoint{7.669531in}{0.646121in}}{\pgfqpoint{7.673921in}{0.635522in}}{\pgfqpoint{7.681735in}{0.627708in}}%
\pgfpathcurveto{\pgfqpoint{7.689549in}{0.619894in}}{\pgfqpoint{7.700148in}{0.615504in}}{\pgfqpoint{7.711198in}{0.615504in}}%
\pgfpathlineto{\pgfqpoint{7.711198in}{0.615504in}}%
\pgfpathclose%
\pgfusepath{stroke}%
\end{pgfscope}%
\begin{pgfscope}%
\pgfpathrectangle{\pgfqpoint{7.512535in}{0.437222in}}{\pgfqpoint{6.275590in}{5.159444in}}%
\pgfusepath{clip}%
\pgfsetbuttcap%
\pgfsetroundjoin%
\pgfsetlinewidth{1.003750pt}%
\definecolor{currentstroke}{rgb}{0.827451,0.827451,0.827451}%
\pgfsetstrokecolor{currentstroke}%
\pgfsetstrokeopacity{0.800000}%
\pgfsetdash{}{0pt}%
\pgfpathmoveto{\pgfqpoint{9.061862in}{3.235524in}}%
\pgfpathcurveto{\pgfqpoint{9.072912in}{3.235524in}}{\pgfqpoint{9.083511in}{3.239914in}}{\pgfqpoint{9.091325in}{3.247728in}}%
\pgfpathcurveto{\pgfqpoint{9.099138in}{3.255542in}}{\pgfqpoint{9.103529in}{3.266141in}}{\pgfqpoint{9.103529in}{3.277191in}}%
\pgfpathcurveto{\pgfqpoint{9.103529in}{3.288241in}}{\pgfqpoint{9.099138in}{3.298840in}}{\pgfqpoint{9.091325in}{3.306654in}}%
\pgfpathcurveto{\pgfqpoint{9.083511in}{3.314467in}}{\pgfqpoint{9.072912in}{3.318857in}}{\pgfqpoint{9.061862in}{3.318857in}}%
\pgfpathcurveto{\pgfqpoint{9.050812in}{3.318857in}}{\pgfqpoint{9.040213in}{3.314467in}}{\pgfqpoint{9.032399in}{3.306654in}}%
\pgfpathcurveto{\pgfqpoint{9.024586in}{3.298840in}}{\pgfqpoint{9.020195in}{3.288241in}}{\pgfqpoint{9.020195in}{3.277191in}}%
\pgfpathcurveto{\pgfqpoint{9.020195in}{3.266141in}}{\pgfqpoint{9.024586in}{3.255542in}}{\pgfqpoint{9.032399in}{3.247728in}}%
\pgfpathcurveto{\pgfqpoint{9.040213in}{3.239914in}}{\pgfqpoint{9.050812in}{3.235524in}}{\pgfqpoint{9.061862in}{3.235524in}}%
\pgfpathlineto{\pgfqpoint{9.061862in}{3.235524in}}%
\pgfpathclose%
\pgfusepath{stroke}%
\end{pgfscope}%
\begin{pgfscope}%
\pgfpathrectangle{\pgfqpoint{7.512535in}{0.437222in}}{\pgfqpoint{6.275590in}{5.159444in}}%
\pgfusepath{clip}%
\pgfsetbuttcap%
\pgfsetroundjoin%
\pgfsetlinewidth{1.003750pt}%
\definecolor{currentstroke}{rgb}{0.827451,0.827451,0.827451}%
\pgfsetstrokecolor{currentstroke}%
\pgfsetstrokeopacity{0.800000}%
\pgfsetdash{}{0pt}%
\pgfpathmoveto{\pgfqpoint{10.156793in}{3.250664in}}%
\pgfpathcurveto{\pgfqpoint{10.167843in}{3.250664in}}{\pgfqpoint{10.178442in}{3.255054in}}{\pgfqpoint{10.186256in}{3.262868in}}%
\pgfpathcurveto{\pgfqpoint{10.194069in}{3.270682in}}{\pgfqpoint{10.198460in}{3.281281in}}{\pgfqpoint{10.198460in}{3.292331in}}%
\pgfpathcurveto{\pgfqpoint{10.198460in}{3.303381in}}{\pgfqpoint{10.194069in}{3.313980in}}{\pgfqpoint{10.186256in}{3.321794in}}%
\pgfpathcurveto{\pgfqpoint{10.178442in}{3.329607in}}{\pgfqpoint{10.167843in}{3.333998in}}{\pgfqpoint{10.156793in}{3.333998in}}%
\pgfpathcurveto{\pgfqpoint{10.145743in}{3.333998in}}{\pgfqpoint{10.135144in}{3.329607in}}{\pgfqpoint{10.127330in}{3.321794in}}%
\pgfpathcurveto{\pgfqpoint{10.119517in}{3.313980in}}{\pgfqpoint{10.115126in}{3.303381in}}{\pgfqpoint{10.115126in}{3.292331in}}%
\pgfpathcurveto{\pgfqpoint{10.115126in}{3.281281in}}{\pgfqpoint{10.119517in}{3.270682in}}{\pgfqpoint{10.127330in}{3.262868in}}%
\pgfpathcurveto{\pgfqpoint{10.135144in}{3.255054in}}{\pgfqpoint{10.145743in}{3.250664in}}{\pgfqpoint{10.156793in}{3.250664in}}%
\pgfpathlineto{\pgfqpoint{10.156793in}{3.250664in}}%
\pgfpathclose%
\pgfusepath{stroke}%
\end{pgfscope}%
\begin{pgfscope}%
\pgfpathrectangle{\pgfqpoint{7.512535in}{0.437222in}}{\pgfqpoint{6.275590in}{5.159444in}}%
\pgfusepath{clip}%
\pgfsetbuttcap%
\pgfsetroundjoin%
\pgfsetlinewidth{1.003750pt}%
\definecolor{currentstroke}{rgb}{0.827451,0.827451,0.827451}%
\pgfsetstrokecolor{currentstroke}%
\pgfsetstrokeopacity{0.800000}%
\pgfsetdash{}{0pt}%
\pgfpathmoveto{\pgfqpoint{10.259535in}{4.447570in}}%
\pgfpathcurveto{\pgfqpoint{10.270585in}{4.447570in}}{\pgfqpoint{10.281184in}{4.451960in}}{\pgfqpoint{10.288997in}{4.459773in}}%
\pgfpathcurveto{\pgfqpoint{10.296811in}{4.467587in}}{\pgfqpoint{10.301201in}{4.478186in}}{\pgfqpoint{10.301201in}{4.489236in}}%
\pgfpathcurveto{\pgfqpoint{10.301201in}{4.500286in}}{\pgfqpoint{10.296811in}{4.510885in}}{\pgfqpoint{10.288997in}{4.518699in}}%
\pgfpathcurveto{\pgfqpoint{10.281184in}{4.526513in}}{\pgfqpoint{10.270585in}{4.530903in}}{\pgfqpoint{10.259535in}{4.530903in}}%
\pgfpathcurveto{\pgfqpoint{10.248485in}{4.530903in}}{\pgfqpoint{10.237886in}{4.526513in}}{\pgfqpoint{10.230072in}{4.518699in}}%
\pgfpathcurveto{\pgfqpoint{10.222258in}{4.510885in}}{\pgfqpoint{10.217868in}{4.500286in}}{\pgfqpoint{10.217868in}{4.489236in}}%
\pgfpathcurveto{\pgfqpoint{10.217868in}{4.478186in}}{\pgfqpoint{10.222258in}{4.467587in}}{\pgfqpoint{10.230072in}{4.459773in}}%
\pgfpathcurveto{\pgfqpoint{10.237886in}{4.451960in}}{\pgfqpoint{10.248485in}{4.447570in}}{\pgfqpoint{10.259535in}{4.447570in}}%
\pgfpathlineto{\pgfqpoint{10.259535in}{4.447570in}}%
\pgfpathclose%
\pgfusepath{stroke}%
\end{pgfscope}%
\begin{pgfscope}%
\pgfpathrectangle{\pgfqpoint{7.512535in}{0.437222in}}{\pgfqpoint{6.275590in}{5.159444in}}%
\pgfusepath{clip}%
\pgfsetbuttcap%
\pgfsetroundjoin%
\pgfsetlinewidth{1.003750pt}%
\definecolor{currentstroke}{rgb}{0.827451,0.827451,0.827451}%
\pgfsetstrokecolor{currentstroke}%
\pgfsetstrokeopacity{0.800000}%
\pgfsetdash{}{0pt}%
\pgfpathmoveto{\pgfqpoint{8.619483in}{1.654089in}}%
\pgfpathcurveto{\pgfqpoint{8.630533in}{1.654089in}}{\pgfqpoint{8.641132in}{1.658480in}}{\pgfqpoint{8.648946in}{1.666293in}}%
\pgfpathcurveto{\pgfqpoint{8.656759in}{1.674107in}}{\pgfqpoint{8.661149in}{1.684706in}}{\pgfqpoint{8.661149in}{1.695756in}}%
\pgfpathcurveto{\pgfqpoint{8.661149in}{1.706806in}}{\pgfqpoint{8.656759in}{1.717405in}}{\pgfqpoint{8.648946in}{1.725219in}}%
\pgfpathcurveto{\pgfqpoint{8.641132in}{1.733033in}}{\pgfqpoint{8.630533in}{1.737423in}}{\pgfqpoint{8.619483in}{1.737423in}}%
\pgfpathcurveto{\pgfqpoint{8.608433in}{1.737423in}}{\pgfqpoint{8.597834in}{1.733033in}}{\pgfqpoint{8.590020in}{1.725219in}}%
\pgfpathcurveto{\pgfqpoint{8.582206in}{1.717405in}}{\pgfqpoint{8.577816in}{1.706806in}}{\pgfqpoint{8.577816in}{1.695756in}}%
\pgfpathcurveto{\pgfqpoint{8.577816in}{1.684706in}}{\pgfqpoint{8.582206in}{1.674107in}}{\pgfqpoint{8.590020in}{1.666293in}}%
\pgfpathcurveto{\pgfqpoint{8.597834in}{1.658480in}}{\pgfqpoint{8.608433in}{1.654089in}}{\pgfqpoint{8.619483in}{1.654089in}}%
\pgfpathlineto{\pgfqpoint{8.619483in}{1.654089in}}%
\pgfpathclose%
\pgfusepath{stroke}%
\end{pgfscope}%
\begin{pgfscope}%
\pgfpathrectangle{\pgfqpoint{7.512535in}{0.437222in}}{\pgfqpoint{6.275590in}{5.159444in}}%
\pgfusepath{clip}%
\pgfsetbuttcap%
\pgfsetroundjoin%
\pgfsetlinewidth{1.003750pt}%
\definecolor{currentstroke}{rgb}{0.827451,0.827451,0.827451}%
\pgfsetstrokecolor{currentstroke}%
\pgfsetstrokeopacity{0.800000}%
\pgfsetdash{}{0pt}%
\pgfpathmoveto{\pgfqpoint{7.562575in}{0.493855in}}%
\pgfpathcurveto{\pgfqpoint{7.573625in}{0.493855in}}{\pgfqpoint{7.584224in}{0.498245in}}{\pgfqpoint{7.592038in}{0.506059in}}%
\pgfpathcurveto{\pgfqpoint{7.599851in}{0.513872in}}{\pgfqpoint{7.604241in}{0.524471in}}{\pgfqpoint{7.604241in}{0.535522in}}%
\pgfpathcurveto{\pgfqpoint{7.604241in}{0.546572in}}{\pgfqpoint{7.599851in}{0.557171in}}{\pgfqpoint{7.592038in}{0.564984in}}%
\pgfpathcurveto{\pgfqpoint{7.584224in}{0.572798in}}{\pgfqpoint{7.573625in}{0.577188in}}{\pgfqpoint{7.562575in}{0.577188in}}%
\pgfpathcurveto{\pgfqpoint{7.551525in}{0.577188in}}{\pgfqpoint{7.540926in}{0.572798in}}{\pgfqpoint{7.533112in}{0.564984in}}%
\pgfpathcurveto{\pgfqpoint{7.525298in}{0.557171in}}{\pgfqpoint{7.520908in}{0.546572in}}{\pgfqpoint{7.520908in}{0.535522in}}%
\pgfpathcurveto{\pgfqpoint{7.520908in}{0.524471in}}{\pgfqpoint{7.525298in}{0.513872in}}{\pgfqpoint{7.533112in}{0.506059in}}%
\pgfpathcurveto{\pgfqpoint{7.540926in}{0.498245in}}{\pgfqpoint{7.551525in}{0.493855in}}{\pgfqpoint{7.562575in}{0.493855in}}%
\pgfpathlineto{\pgfqpoint{7.562575in}{0.493855in}}%
\pgfpathclose%
\pgfusepath{stroke}%
\end{pgfscope}%
\begin{pgfscope}%
\pgfpathrectangle{\pgfqpoint{7.512535in}{0.437222in}}{\pgfqpoint{6.275590in}{5.159444in}}%
\pgfusepath{clip}%
\pgfsetbuttcap%
\pgfsetroundjoin%
\pgfsetlinewidth{1.003750pt}%
\definecolor{currentstroke}{rgb}{0.827451,0.827451,0.827451}%
\pgfsetstrokecolor{currentstroke}%
\pgfsetstrokeopacity{0.800000}%
\pgfsetdash{}{0pt}%
\pgfpathmoveto{\pgfqpoint{9.581399in}{3.231111in}}%
\pgfpathcurveto{\pgfqpoint{9.592449in}{3.231111in}}{\pgfqpoint{9.603048in}{3.235501in}}{\pgfqpoint{9.610862in}{3.243315in}}%
\pgfpathcurveto{\pgfqpoint{9.618675in}{3.251128in}}{\pgfqpoint{9.623065in}{3.261727in}}{\pgfqpoint{9.623065in}{3.272778in}}%
\pgfpathcurveto{\pgfqpoint{9.623065in}{3.283828in}}{\pgfqpoint{9.618675in}{3.294427in}}{\pgfqpoint{9.610862in}{3.302240in}}%
\pgfpathcurveto{\pgfqpoint{9.603048in}{3.310054in}}{\pgfqpoint{9.592449in}{3.314444in}}{\pgfqpoint{9.581399in}{3.314444in}}%
\pgfpathcurveto{\pgfqpoint{9.570349in}{3.314444in}}{\pgfqpoint{9.559750in}{3.310054in}}{\pgfqpoint{9.551936in}{3.302240in}}%
\pgfpathcurveto{\pgfqpoint{9.544122in}{3.294427in}}{\pgfqpoint{9.539732in}{3.283828in}}{\pgfqpoint{9.539732in}{3.272778in}}%
\pgfpathcurveto{\pgfqpoint{9.539732in}{3.261727in}}{\pgfqpoint{9.544122in}{3.251128in}}{\pgfqpoint{9.551936in}{3.243315in}}%
\pgfpathcurveto{\pgfqpoint{9.559750in}{3.235501in}}{\pgfqpoint{9.570349in}{3.231111in}}{\pgfqpoint{9.581399in}{3.231111in}}%
\pgfpathlineto{\pgfqpoint{9.581399in}{3.231111in}}%
\pgfpathclose%
\pgfusepath{stroke}%
\end{pgfscope}%
\begin{pgfscope}%
\pgfpathrectangle{\pgfqpoint{7.512535in}{0.437222in}}{\pgfqpoint{6.275590in}{5.159444in}}%
\pgfusepath{clip}%
\pgfsetbuttcap%
\pgfsetroundjoin%
\pgfsetlinewidth{1.003750pt}%
\definecolor{currentstroke}{rgb}{0.827451,0.827451,0.827451}%
\pgfsetstrokecolor{currentstroke}%
\pgfsetstrokeopacity{0.800000}%
\pgfsetdash{}{0pt}%
\pgfpathmoveto{\pgfqpoint{10.372490in}{5.117177in}}%
\pgfpathcurveto{\pgfqpoint{10.383540in}{5.117177in}}{\pgfqpoint{10.394139in}{5.121567in}}{\pgfqpoint{10.401953in}{5.129381in}}%
\pgfpathcurveto{\pgfqpoint{10.409766in}{5.137194in}}{\pgfqpoint{10.414156in}{5.147793in}}{\pgfqpoint{10.414156in}{5.158843in}}%
\pgfpathcurveto{\pgfqpoint{10.414156in}{5.169893in}}{\pgfqpoint{10.409766in}{5.180493in}}{\pgfqpoint{10.401953in}{5.188306in}}%
\pgfpathcurveto{\pgfqpoint{10.394139in}{5.196120in}}{\pgfqpoint{10.383540in}{5.200510in}}{\pgfqpoint{10.372490in}{5.200510in}}%
\pgfpathcurveto{\pgfqpoint{10.361440in}{5.200510in}}{\pgfqpoint{10.350841in}{5.196120in}}{\pgfqpoint{10.343027in}{5.188306in}}%
\pgfpathcurveto{\pgfqpoint{10.335213in}{5.180493in}}{\pgfqpoint{10.330823in}{5.169893in}}{\pgfqpoint{10.330823in}{5.158843in}}%
\pgfpathcurveto{\pgfqpoint{10.330823in}{5.147793in}}{\pgfqpoint{10.335213in}{5.137194in}}{\pgfqpoint{10.343027in}{5.129381in}}%
\pgfpathcurveto{\pgfqpoint{10.350841in}{5.121567in}}{\pgfqpoint{10.361440in}{5.117177in}}{\pgfqpoint{10.372490in}{5.117177in}}%
\pgfpathlineto{\pgfqpoint{10.372490in}{5.117177in}}%
\pgfpathclose%
\pgfusepath{stroke}%
\end{pgfscope}%
\begin{pgfscope}%
\pgfpathrectangle{\pgfqpoint{7.512535in}{0.437222in}}{\pgfqpoint{6.275590in}{5.159444in}}%
\pgfusepath{clip}%
\pgfsetbuttcap%
\pgfsetroundjoin%
\pgfsetlinewidth{1.003750pt}%
\definecolor{currentstroke}{rgb}{0.827451,0.827451,0.827451}%
\pgfsetstrokecolor{currentstroke}%
\pgfsetstrokeopacity{0.800000}%
\pgfsetdash{}{0pt}%
\pgfpathmoveto{\pgfqpoint{12.102762in}{5.438998in}}%
\pgfpathcurveto{\pgfqpoint{12.113812in}{5.438998in}}{\pgfqpoint{12.124411in}{5.443388in}}{\pgfqpoint{12.132225in}{5.451201in}}%
\pgfpathcurveto{\pgfqpoint{12.140039in}{5.459015in}}{\pgfqpoint{12.144429in}{5.469614in}}{\pgfqpoint{12.144429in}{5.480664in}}%
\pgfpathcurveto{\pgfqpoint{12.144429in}{5.491714in}}{\pgfqpoint{12.140039in}{5.502313in}}{\pgfqpoint{12.132225in}{5.510127in}}%
\pgfpathcurveto{\pgfqpoint{12.124411in}{5.517941in}}{\pgfqpoint{12.113812in}{5.522331in}}{\pgfqpoint{12.102762in}{5.522331in}}%
\pgfpathcurveto{\pgfqpoint{12.091712in}{5.522331in}}{\pgfqpoint{12.081113in}{5.517941in}}{\pgfqpoint{12.073299in}{5.510127in}}%
\pgfpathcurveto{\pgfqpoint{12.065486in}{5.502313in}}{\pgfqpoint{12.061096in}{5.491714in}}{\pgfqpoint{12.061096in}{5.480664in}}%
\pgfpathcurveto{\pgfqpoint{12.061096in}{5.469614in}}{\pgfqpoint{12.065486in}{5.459015in}}{\pgfqpoint{12.073299in}{5.451201in}}%
\pgfpathcurveto{\pgfqpoint{12.081113in}{5.443388in}}{\pgfqpoint{12.091712in}{5.438998in}}{\pgfqpoint{12.102762in}{5.438998in}}%
\pgfpathlineto{\pgfqpoint{12.102762in}{5.438998in}}%
\pgfpathclose%
\pgfusepath{stroke}%
\end{pgfscope}%
\begin{pgfscope}%
\pgfpathrectangle{\pgfqpoint{7.512535in}{0.437222in}}{\pgfqpoint{6.275590in}{5.159444in}}%
\pgfusepath{clip}%
\pgfsetbuttcap%
\pgfsetroundjoin%
\pgfsetlinewidth{1.003750pt}%
\definecolor{currentstroke}{rgb}{0.827451,0.827451,0.827451}%
\pgfsetstrokecolor{currentstroke}%
\pgfsetstrokeopacity{0.800000}%
\pgfsetdash{}{0pt}%
\pgfpathmoveto{\pgfqpoint{12.233224in}{5.446777in}}%
\pgfpathcurveto{\pgfqpoint{12.244274in}{5.446777in}}{\pgfqpoint{12.254873in}{5.451168in}}{\pgfqpoint{12.262687in}{5.458981in}}%
\pgfpathcurveto{\pgfqpoint{12.270500in}{5.466795in}}{\pgfqpoint{12.274891in}{5.477394in}}{\pgfqpoint{12.274891in}{5.488444in}}%
\pgfpathcurveto{\pgfqpoint{12.274891in}{5.499494in}}{\pgfqpoint{12.270500in}{5.510093in}}{\pgfqpoint{12.262687in}{5.517907in}}%
\pgfpathcurveto{\pgfqpoint{12.254873in}{5.525720in}}{\pgfqpoint{12.244274in}{5.530111in}}{\pgfqpoint{12.233224in}{5.530111in}}%
\pgfpathcurveto{\pgfqpoint{12.222174in}{5.530111in}}{\pgfqpoint{12.211575in}{5.525720in}}{\pgfqpoint{12.203761in}{5.517907in}}%
\pgfpathcurveto{\pgfqpoint{12.195948in}{5.510093in}}{\pgfqpoint{12.191557in}{5.499494in}}{\pgfqpoint{12.191557in}{5.488444in}}%
\pgfpathcurveto{\pgfqpoint{12.191557in}{5.477394in}}{\pgfqpoint{12.195948in}{5.466795in}}{\pgfqpoint{12.203761in}{5.458981in}}%
\pgfpathcurveto{\pgfqpoint{12.211575in}{5.451168in}}{\pgfqpoint{12.222174in}{5.446777in}}{\pgfqpoint{12.233224in}{5.446777in}}%
\pgfpathlineto{\pgfqpoint{12.233224in}{5.446777in}}%
\pgfpathclose%
\pgfusepath{stroke}%
\end{pgfscope}%
\begin{pgfscope}%
\pgfpathrectangle{\pgfqpoint{7.512535in}{0.437222in}}{\pgfqpoint{6.275590in}{5.159444in}}%
\pgfusepath{clip}%
\pgfsetbuttcap%
\pgfsetroundjoin%
\pgfsetlinewidth{1.003750pt}%
\definecolor{currentstroke}{rgb}{0.827451,0.827451,0.827451}%
\pgfsetstrokecolor{currentstroke}%
\pgfsetstrokeopacity{0.800000}%
\pgfsetdash{}{0pt}%
\pgfpathmoveto{\pgfqpoint{12.102762in}{5.362410in}}%
\pgfpathcurveto{\pgfqpoint{12.113812in}{5.362410in}}{\pgfqpoint{12.124411in}{5.366800in}}{\pgfqpoint{12.132225in}{5.374614in}}%
\pgfpathcurveto{\pgfqpoint{12.140039in}{5.382427in}}{\pgfqpoint{12.144429in}{5.393027in}}{\pgfqpoint{12.144429in}{5.404077in}}%
\pgfpathcurveto{\pgfqpoint{12.144429in}{5.415127in}}{\pgfqpoint{12.140039in}{5.425726in}}{\pgfqpoint{12.132225in}{5.433539in}}%
\pgfpathcurveto{\pgfqpoint{12.124411in}{5.441353in}}{\pgfqpoint{12.113812in}{5.445743in}}{\pgfqpoint{12.102762in}{5.445743in}}%
\pgfpathcurveto{\pgfqpoint{12.091712in}{5.445743in}}{\pgfqpoint{12.081113in}{5.441353in}}{\pgfqpoint{12.073299in}{5.433539in}}%
\pgfpathcurveto{\pgfqpoint{12.065486in}{5.425726in}}{\pgfqpoint{12.061096in}{5.415127in}}{\pgfqpoint{12.061096in}{5.404077in}}%
\pgfpathcurveto{\pgfqpoint{12.061096in}{5.393027in}}{\pgfqpoint{12.065486in}{5.382427in}}{\pgfqpoint{12.073299in}{5.374614in}}%
\pgfpathcurveto{\pgfqpoint{12.081113in}{5.366800in}}{\pgfqpoint{12.091712in}{5.362410in}}{\pgfqpoint{12.102762in}{5.362410in}}%
\pgfpathlineto{\pgfqpoint{12.102762in}{5.362410in}}%
\pgfpathclose%
\pgfusepath{stroke}%
\end{pgfscope}%
\begin{pgfscope}%
\pgfpathrectangle{\pgfqpoint{7.512535in}{0.437222in}}{\pgfqpoint{6.275590in}{5.159444in}}%
\pgfusepath{clip}%
\pgfsetbuttcap%
\pgfsetroundjoin%
\pgfsetlinewidth{1.003750pt}%
\definecolor{currentstroke}{rgb}{0.827451,0.827451,0.827451}%
\pgfsetstrokecolor{currentstroke}%
\pgfsetstrokeopacity{0.800000}%
\pgfsetdash{}{0pt}%
\pgfpathmoveto{\pgfqpoint{8.290348in}{1.011495in}}%
\pgfpathcurveto{\pgfqpoint{8.301398in}{1.011495in}}{\pgfqpoint{8.311997in}{1.015885in}}{\pgfqpoint{8.319811in}{1.023699in}}%
\pgfpathcurveto{\pgfqpoint{8.327625in}{1.031512in}}{\pgfqpoint{8.332015in}{1.042111in}}{\pgfqpoint{8.332015in}{1.053161in}}%
\pgfpathcurveto{\pgfqpoint{8.332015in}{1.064211in}}{\pgfqpoint{8.327625in}{1.074810in}}{\pgfqpoint{8.319811in}{1.082624in}}%
\pgfpathcurveto{\pgfqpoint{8.311997in}{1.090438in}}{\pgfqpoint{8.301398in}{1.094828in}}{\pgfqpoint{8.290348in}{1.094828in}}%
\pgfpathcurveto{\pgfqpoint{8.279298in}{1.094828in}}{\pgfqpoint{8.268699in}{1.090438in}}{\pgfqpoint{8.260886in}{1.082624in}}%
\pgfpathcurveto{\pgfqpoint{8.253072in}{1.074810in}}{\pgfqpoint{8.248682in}{1.064211in}}{\pgfqpoint{8.248682in}{1.053161in}}%
\pgfpathcurveto{\pgfqpoint{8.248682in}{1.042111in}}{\pgfqpoint{8.253072in}{1.031512in}}{\pgfqpoint{8.260886in}{1.023699in}}%
\pgfpathcurveto{\pgfqpoint{8.268699in}{1.015885in}}{\pgfqpoint{8.279298in}{1.011495in}}{\pgfqpoint{8.290348in}{1.011495in}}%
\pgfpathlineto{\pgfqpoint{8.290348in}{1.011495in}}%
\pgfpathclose%
\pgfusepath{stroke}%
\end{pgfscope}%
\begin{pgfscope}%
\pgfpathrectangle{\pgfqpoint{7.512535in}{0.437222in}}{\pgfqpoint{6.275590in}{5.159444in}}%
\pgfusepath{clip}%
\pgfsetbuttcap%
\pgfsetroundjoin%
\pgfsetlinewidth{1.003750pt}%
\definecolor{currentstroke}{rgb}{0.827451,0.827451,0.827451}%
\pgfsetstrokecolor{currentstroke}%
\pgfsetstrokeopacity{0.800000}%
\pgfsetdash{}{0pt}%
\pgfpathmoveto{\pgfqpoint{10.486066in}{4.475908in}}%
\pgfpathcurveto{\pgfqpoint{10.497116in}{4.475908in}}{\pgfqpoint{10.507715in}{4.480298in}}{\pgfqpoint{10.515529in}{4.488112in}}%
\pgfpathcurveto{\pgfqpoint{10.523342in}{4.495926in}}{\pgfqpoint{10.527733in}{4.506525in}}{\pgfqpoint{10.527733in}{4.517575in}}%
\pgfpathcurveto{\pgfqpoint{10.527733in}{4.528625in}}{\pgfqpoint{10.523342in}{4.539224in}}{\pgfqpoint{10.515529in}{4.547038in}}%
\pgfpathcurveto{\pgfqpoint{10.507715in}{4.554851in}}{\pgfqpoint{10.497116in}{4.559241in}}{\pgfqpoint{10.486066in}{4.559241in}}%
\pgfpathcurveto{\pgfqpoint{10.475016in}{4.559241in}}{\pgfqpoint{10.464417in}{4.554851in}}{\pgfqpoint{10.456603in}{4.547038in}}%
\pgfpathcurveto{\pgfqpoint{10.448790in}{4.539224in}}{\pgfqpoint{10.444399in}{4.528625in}}{\pgfqpoint{10.444399in}{4.517575in}}%
\pgfpathcurveto{\pgfqpoint{10.444399in}{4.506525in}}{\pgfqpoint{10.448790in}{4.495926in}}{\pgfqpoint{10.456603in}{4.488112in}}%
\pgfpathcurveto{\pgfqpoint{10.464417in}{4.480298in}}{\pgfqpoint{10.475016in}{4.475908in}}{\pgfqpoint{10.486066in}{4.475908in}}%
\pgfpathlineto{\pgfqpoint{10.486066in}{4.475908in}}%
\pgfpathclose%
\pgfusepath{stroke}%
\end{pgfscope}%
\begin{pgfscope}%
\pgfpathrectangle{\pgfqpoint{7.512535in}{0.437222in}}{\pgfqpoint{6.275590in}{5.159444in}}%
\pgfusepath{clip}%
\pgfsetbuttcap%
\pgfsetroundjoin%
\pgfsetlinewidth{1.003750pt}%
\definecolor{currentstroke}{rgb}{0.827451,0.827451,0.827451}%
\pgfsetstrokecolor{currentstroke}%
\pgfsetstrokeopacity{0.800000}%
\pgfsetdash{}{0pt}%
\pgfpathmoveto{\pgfqpoint{10.299622in}{3.250664in}}%
\pgfpathcurveto{\pgfqpoint{10.310672in}{3.250664in}}{\pgfqpoint{10.321271in}{3.255054in}}{\pgfqpoint{10.329085in}{3.262868in}}%
\pgfpathcurveto{\pgfqpoint{10.336898in}{3.270682in}}{\pgfqpoint{10.341289in}{3.281281in}}{\pgfqpoint{10.341289in}{3.292331in}}%
\pgfpathcurveto{\pgfqpoint{10.341289in}{3.303381in}}{\pgfqpoint{10.336898in}{3.313980in}}{\pgfqpoint{10.329085in}{3.321794in}}%
\pgfpathcurveto{\pgfqpoint{10.321271in}{3.329607in}}{\pgfqpoint{10.310672in}{3.333998in}}{\pgfqpoint{10.299622in}{3.333998in}}%
\pgfpathcurveto{\pgfqpoint{10.288572in}{3.333998in}}{\pgfqpoint{10.277973in}{3.329607in}}{\pgfqpoint{10.270159in}{3.321794in}}%
\pgfpathcurveto{\pgfqpoint{10.262346in}{3.313980in}}{\pgfqpoint{10.257955in}{3.303381in}}{\pgfqpoint{10.257955in}{3.292331in}}%
\pgfpathcurveto{\pgfqpoint{10.257955in}{3.281281in}}{\pgfqpoint{10.262346in}{3.270682in}}{\pgfqpoint{10.270159in}{3.262868in}}%
\pgfpathcurveto{\pgfqpoint{10.277973in}{3.255054in}}{\pgfqpoint{10.288572in}{3.250664in}}{\pgfqpoint{10.299622in}{3.250664in}}%
\pgfpathlineto{\pgfqpoint{10.299622in}{3.250664in}}%
\pgfpathclose%
\pgfusepath{stroke}%
\end{pgfscope}%
\begin{pgfscope}%
\pgfpathrectangle{\pgfqpoint{7.512535in}{0.437222in}}{\pgfqpoint{6.275590in}{5.159444in}}%
\pgfusepath{clip}%
\pgfsetbuttcap%
\pgfsetroundjoin%
\pgfsetlinewidth{1.003750pt}%
\definecolor{currentstroke}{rgb}{0.827451,0.827451,0.827451}%
\pgfsetstrokecolor{currentstroke}%
\pgfsetstrokeopacity{0.800000}%
\pgfsetdash{}{0pt}%
\pgfpathmoveto{\pgfqpoint{8.282445in}{1.938912in}}%
\pgfpathcurveto{\pgfqpoint{8.293495in}{1.938912in}}{\pgfqpoint{8.304095in}{1.943302in}}{\pgfqpoint{8.311908in}{1.951116in}}%
\pgfpathcurveto{\pgfqpoint{8.319722in}{1.958929in}}{\pgfqpoint{8.324112in}{1.969528in}}{\pgfqpoint{8.324112in}{1.980579in}}%
\pgfpathcurveto{\pgfqpoint{8.324112in}{1.991629in}}{\pgfqpoint{8.319722in}{2.002228in}}{\pgfqpoint{8.311908in}{2.010041in}}%
\pgfpathcurveto{\pgfqpoint{8.304095in}{2.017855in}}{\pgfqpoint{8.293495in}{2.022245in}}{\pgfqpoint{8.282445in}{2.022245in}}%
\pgfpathcurveto{\pgfqpoint{8.271395in}{2.022245in}}{\pgfqpoint{8.260796in}{2.017855in}}{\pgfqpoint{8.252983in}{2.010041in}}%
\pgfpathcurveto{\pgfqpoint{8.245169in}{2.002228in}}{\pgfqpoint{8.240779in}{1.991629in}}{\pgfqpoint{8.240779in}{1.980579in}}%
\pgfpathcurveto{\pgfqpoint{8.240779in}{1.969528in}}{\pgfqpoint{8.245169in}{1.958929in}}{\pgfqpoint{8.252983in}{1.951116in}}%
\pgfpathcurveto{\pgfqpoint{8.260796in}{1.943302in}}{\pgfqpoint{8.271395in}{1.938912in}}{\pgfqpoint{8.282445in}{1.938912in}}%
\pgfpathlineto{\pgfqpoint{8.282445in}{1.938912in}}%
\pgfpathclose%
\pgfusepath{stroke}%
\end{pgfscope}%
\begin{pgfscope}%
\pgfpathrectangle{\pgfqpoint{7.512535in}{0.437222in}}{\pgfqpoint{6.275590in}{5.159444in}}%
\pgfusepath{clip}%
\pgfsetbuttcap%
\pgfsetroundjoin%
\pgfsetlinewidth{1.003750pt}%
\definecolor{currentstroke}{rgb}{0.827451,0.827451,0.827451}%
\pgfsetstrokecolor{currentstroke}%
\pgfsetstrokeopacity{0.800000}%
\pgfsetdash{}{0pt}%
\pgfpathmoveto{\pgfqpoint{12.287231in}{5.553091in}}%
\pgfpathcurveto{\pgfqpoint{12.298281in}{5.553091in}}{\pgfqpoint{12.308880in}{5.557481in}}{\pgfqpoint{12.316694in}{5.565295in}}%
\pgfpathcurveto{\pgfqpoint{12.324508in}{5.573108in}}{\pgfqpoint{12.328898in}{5.583707in}}{\pgfqpoint{12.328898in}{5.594757in}}%
\pgfpathcurveto{\pgfqpoint{12.328898in}{5.605808in}}{\pgfqpoint{12.324508in}{5.616407in}}{\pgfqpoint{12.316694in}{5.624220in}}%
\pgfpathcurveto{\pgfqpoint{12.308880in}{5.632034in}}{\pgfqpoint{12.298281in}{5.636424in}}{\pgfqpoint{12.287231in}{5.636424in}}%
\pgfpathcurveto{\pgfqpoint{12.276181in}{5.636424in}}{\pgfqpoint{12.265582in}{5.632034in}}{\pgfqpoint{12.257768in}{5.624220in}}%
\pgfpathcurveto{\pgfqpoint{12.249955in}{5.616407in}}{\pgfqpoint{12.245564in}{5.605808in}}{\pgfqpoint{12.245564in}{5.594757in}}%
\pgfpathcurveto{\pgfqpoint{12.245564in}{5.583707in}}{\pgfqpoint{12.249955in}{5.573108in}}{\pgfqpoint{12.257768in}{5.565295in}}%
\pgfpathcurveto{\pgfqpoint{12.265582in}{5.557481in}}{\pgfqpoint{12.276181in}{5.553091in}}{\pgfqpoint{12.287231in}{5.553091in}}%
\pgfpathlineto{\pgfqpoint{12.287231in}{5.553091in}}%
\pgfpathclose%
\pgfusepath{stroke}%
\end{pgfscope}%
\begin{pgfscope}%
\pgfpathrectangle{\pgfqpoint{7.512535in}{0.437222in}}{\pgfqpoint{6.275590in}{5.159444in}}%
\pgfusepath{clip}%
\pgfsetbuttcap%
\pgfsetroundjoin%
\pgfsetlinewidth{1.003750pt}%
\definecolor{currentstroke}{rgb}{0.827451,0.827451,0.827451}%
\pgfsetstrokecolor{currentstroke}%
\pgfsetstrokeopacity{0.800000}%
\pgfsetdash{}{0pt}%
\pgfpathmoveto{\pgfqpoint{10.818589in}{4.898946in}}%
\pgfpathcurveto{\pgfqpoint{10.829639in}{4.898946in}}{\pgfqpoint{10.840238in}{4.903336in}}{\pgfqpoint{10.848052in}{4.911150in}}%
\pgfpathcurveto{\pgfqpoint{10.855865in}{4.918964in}}{\pgfqpoint{10.860255in}{4.929563in}}{\pgfqpoint{10.860255in}{4.940613in}}%
\pgfpathcurveto{\pgfqpoint{10.860255in}{4.951663in}}{\pgfqpoint{10.855865in}{4.962262in}}{\pgfqpoint{10.848052in}{4.970076in}}%
\pgfpathcurveto{\pgfqpoint{10.840238in}{4.977889in}}{\pgfqpoint{10.829639in}{4.982279in}}{\pgfqpoint{10.818589in}{4.982279in}}%
\pgfpathcurveto{\pgfqpoint{10.807539in}{4.982279in}}{\pgfqpoint{10.796940in}{4.977889in}}{\pgfqpoint{10.789126in}{4.970076in}}%
\pgfpathcurveto{\pgfqpoint{10.781312in}{4.962262in}}{\pgfqpoint{10.776922in}{4.951663in}}{\pgfqpoint{10.776922in}{4.940613in}}%
\pgfpathcurveto{\pgfqpoint{10.776922in}{4.929563in}}{\pgfqpoint{10.781312in}{4.918964in}}{\pgfqpoint{10.789126in}{4.911150in}}%
\pgfpathcurveto{\pgfqpoint{10.796940in}{4.903336in}}{\pgfqpoint{10.807539in}{4.898946in}}{\pgfqpoint{10.818589in}{4.898946in}}%
\pgfpathlineto{\pgfqpoint{10.818589in}{4.898946in}}%
\pgfpathclose%
\pgfusepath{stroke}%
\end{pgfscope}%
\begin{pgfscope}%
\pgfpathrectangle{\pgfqpoint{7.512535in}{0.437222in}}{\pgfqpoint{6.275590in}{5.159444in}}%
\pgfusepath{clip}%
\pgfsetbuttcap%
\pgfsetroundjoin%
\pgfsetlinewidth{1.003750pt}%
\definecolor{currentstroke}{rgb}{0.827451,0.827451,0.827451}%
\pgfsetstrokecolor{currentstroke}%
\pgfsetstrokeopacity{0.800000}%
\pgfsetdash{}{0pt}%
\pgfpathmoveto{\pgfqpoint{10.818589in}{5.117177in}}%
\pgfpathcurveto{\pgfqpoint{10.829639in}{5.117177in}}{\pgfqpoint{10.840238in}{5.121567in}}{\pgfqpoint{10.848052in}{5.129381in}}%
\pgfpathcurveto{\pgfqpoint{10.855865in}{5.137194in}}{\pgfqpoint{10.860255in}{5.147793in}}{\pgfqpoint{10.860255in}{5.158843in}}%
\pgfpathcurveto{\pgfqpoint{10.860255in}{5.169893in}}{\pgfqpoint{10.855865in}{5.180493in}}{\pgfqpoint{10.848052in}{5.188306in}}%
\pgfpathcurveto{\pgfqpoint{10.840238in}{5.196120in}}{\pgfqpoint{10.829639in}{5.200510in}}{\pgfqpoint{10.818589in}{5.200510in}}%
\pgfpathcurveto{\pgfqpoint{10.807539in}{5.200510in}}{\pgfqpoint{10.796940in}{5.196120in}}{\pgfqpoint{10.789126in}{5.188306in}}%
\pgfpathcurveto{\pgfqpoint{10.781312in}{5.180493in}}{\pgfqpoint{10.776922in}{5.169893in}}{\pgfqpoint{10.776922in}{5.158843in}}%
\pgfpathcurveto{\pgfqpoint{10.776922in}{5.147793in}}{\pgfqpoint{10.781312in}{5.137194in}}{\pgfqpoint{10.789126in}{5.129381in}}%
\pgfpathcurveto{\pgfqpoint{10.796940in}{5.121567in}}{\pgfqpoint{10.807539in}{5.117177in}}{\pgfqpoint{10.818589in}{5.117177in}}%
\pgfpathlineto{\pgfqpoint{10.818589in}{5.117177in}}%
\pgfpathclose%
\pgfusepath{stroke}%
\end{pgfscope}%
\begin{pgfscope}%
\pgfpathrectangle{\pgfqpoint{7.512535in}{0.437222in}}{\pgfqpoint{6.275590in}{5.159444in}}%
\pgfusepath{clip}%
\pgfsetbuttcap%
\pgfsetroundjoin%
\pgfsetlinewidth{1.003750pt}%
\definecolor{currentstroke}{rgb}{0.827451,0.827451,0.827451}%
\pgfsetstrokecolor{currentstroke}%
\pgfsetstrokeopacity{0.800000}%
\pgfsetdash{}{0pt}%
\pgfpathmoveto{\pgfqpoint{8.775643in}{1.745625in}}%
\pgfpathcurveto{\pgfqpoint{8.786693in}{1.745625in}}{\pgfqpoint{8.797292in}{1.750015in}}{\pgfqpoint{8.805106in}{1.757829in}}%
\pgfpathcurveto{\pgfqpoint{8.812919in}{1.765643in}}{\pgfqpoint{8.817310in}{1.776242in}}{\pgfqpoint{8.817310in}{1.787292in}}%
\pgfpathcurveto{\pgfqpoint{8.817310in}{1.798342in}}{\pgfqpoint{8.812919in}{1.808941in}}{\pgfqpoint{8.805106in}{1.816755in}}%
\pgfpathcurveto{\pgfqpoint{8.797292in}{1.824568in}}{\pgfqpoint{8.786693in}{1.828959in}}{\pgfqpoint{8.775643in}{1.828959in}}%
\pgfpathcurveto{\pgfqpoint{8.764593in}{1.828959in}}{\pgfqpoint{8.753994in}{1.824568in}}{\pgfqpoint{8.746180in}{1.816755in}}%
\pgfpathcurveto{\pgfqpoint{8.738367in}{1.808941in}}{\pgfqpoint{8.733976in}{1.798342in}}{\pgfqpoint{8.733976in}{1.787292in}}%
\pgfpathcurveto{\pgfqpoint{8.733976in}{1.776242in}}{\pgfqpoint{8.738367in}{1.765643in}}{\pgfqpoint{8.746180in}{1.757829in}}%
\pgfpathcurveto{\pgfqpoint{8.753994in}{1.750015in}}{\pgfqpoint{8.764593in}{1.745625in}}{\pgfqpoint{8.775643in}{1.745625in}}%
\pgfpathlineto{\pgfqpoint{8.775643in}{1.745625in}}%
\pgfpathclose%
\pgfusepath{stroke}%
\end{pgfscope}%
\begin{pgfscope}%
\pgfpathrectangle{\pgfqpoint{7.512535in}{0.437222in}}{\pgfqpoint{6.275590in}{5.159444in}}%
\pgfusepath{clip}%
\pgfsetbuttcap%
\pgfsetroundjoin%
\pgfsetlinewidth{1.003750pt}%
\definecolor{currentstroke}{rgb}{0.827451,0.827451,0.827451}%
\pgfsetstrokecolor{currentstroke}%
\pgfsetstrokeopacity{0.800000}%
\pgfsetdash{}{0pt}%
\pgfpathmoveto{\pgfqpoint{8.716039in}{3.198227in}}%
\pgfpathcurveto{\pgfqpoint{8.727089in}{3.198227in}}{\pgfqpoint{8.737688in}{3.202618in}}{\pgfqpoint{8.745502in}{3.210431in}}%
\pgfpathcurveto{\pgfqpoint{8.753315in}{3.218245in}}{\pgfqpoint{8.757706in}{3.228844in}}{\pgfqpoint{8.757706in}{3.239894in}}%
\pgfpathcurveto{\pgfqpoint{8.757706in}{3.250944in}}{\pgfqpoint{8.753315in}{3.261543in}}{\pgfqpoint{8.745502in}{3.269357in}}%
\pgfpathcurveto{\pgfqpoint{8.737688in}{3.277170in}}{\pgfqpoint{8.727089in}{3.281561in}}{\pgfqpoint{8.716039in}{3.281561in}}%
\pgfpathcurveto{\pgfqpoint{8.704989in}{3.281561in}}{\pgfqpoint{8.694390in}{3.277170in}}{\pgfqpoint{8.686576in}{3.269357in}}%
\pgfpathcurveto{\pgfqpoint{8.678763in}{3.261543in}}{\pgfqpoint{8.674372in}{3.250944in}}{\pgfqpoint{8.674372in}{3.239894in}}%
\pgfpathcurveto{\pgfqpoint{8.674372in}{3.228844in}}{\pgfqpoint{8.678763in}{3.218245in}}{\pgfqpoint{8.686576in}{3.210431in}}%
\pgfpathcurveto{\pgfqpoint{8.694390in}{3.202618in}}{\pgfqpoint{8.704989in}{3.198227in}}{\pgfqpoint{8.716039in}{3.198227in}}%
\pgfpathlineto{\pgfqpoint{8.716039in}{3.198227in}}%
\pgfpathclose%
\pgfusepath{stroke}%
\end{pgfscope}%
\begin{pgfscope}%
\pgfpathrectangle{\pgfqpoint{7.512535in}{0.437222in}}{\pgfqpoint{6.275590in}{5.159444in}}%
\pgfusepath{clip}%
\pgfsetbuttcap%
\pgfsetroundjoin%
\pgfsetlinewidth{1.003750pt}%
\definecolor{currentstroke}{rgb}{0.827451,0.827451,0.827451}%
\pgfsetstrokecolor{currentstroke}%
\pgfsetstrokeopacity{0.800000}%
\pgfsetdash{}{0pt}%
\pgfpathmoveto{\pgfqpoint{10.994755in}{5.060707in}}%
\pgfpathcurveto{\pgfqpoint{11.005805in}{5.060707in}}{\pgfqpoint{11.016404in}{5.065097in}}{\pgfqpoint{11.024218in}{5.072911in}}%
\pgfpathcurveto{\pgfqpoint{11.032031in}{5.080725in}}{\pgfqpoint{11.036422in}{5.091324in}}{\pgfqpoint{11.036422in}{5.102374in}}%
\pgfpathcurveto{\pgfqpoint{11.036422in}{5.113424in}}{\pgfqpoint{11.032031in}{5.124023in}}{\pgfqpoint{11.024218in}{5.131837in}}%
\pgfpathcurveto{\pgfqpoint{11.016404in}{5.139650in}}{\pgfqpoint{11.005805in}{5.144040in}}{\pgfqpoint{10.994755in}{5.144040in}}%
\pgfpathcurveto{\pgfqpoint{10.983705in}{5.144040in}}{\pgfqpoint{10.973106in}{5.139650in}}{\pgfqpoint{10.965292in}{5.131837in}}%
\pgfpathcurveto{\pgfqpoint{10.957479in}{5.124023in}}{\pgfqpoint{10.953088in}{5.113424in}}{\pgfqpoint{10.953088in}{5.102374in}}%
\pgfpathcurveto{\pgfqpoint{10.953088in}{5.091324in}}{\pgfqpoint{10.957479in}{5.080725in}}{\pgfqpoint{10.965292in}{5.072911in}}%
\pgfpathcurveto{\pgfqpoint{10.973106in}{5.065097in}}{\pgfqpoint{10.983705in}{5.060707in}}{\pgfqpoint{10.994755in}{5.060707in}}%
\pgfpathlineto{\pgfqpoint{10.994755in}{5.060707in}}%
\pgfpathclose%
\pgfusepath{stroke}%
\end{pgfscope}%
\begin{pgfscope}%
\pgfpathrectangle{\pgfqpoint{7.512535in}{0.437222in}}{\pgfqpoint{6.275590in}{5.159444in}}%
\pgfusepath{clip}%
\pgfsetbuttcap%
\pgfsetroundjoin%
\pgfsetlinewidth{1.003750pt}%
\definecolor{currentstroke}{rgb}{0.827451,0.827451,0.827451}%
\pgfsetstrokecolor{currentstroke}%
\pgfsetstrokeopacity{0.800000}%
\pgfsetdash{}{0pt}%
\pgfpathmoveto{\pgfqpoint{10.385573in}{5.221855in}}%
\pgfpathcurveto{\pgfqpoint{10.396623in}{5.221855in}}{\pgfqpoint{10.407222in}{5.226245in}}{\pgfqpoint{10.415036in}{5.234058in}}%
\pgfpathcurveto{\pgfqpoint{10.422849in}{5.241872in}}{\pgfqpoint{10.427240in}{5.252471in}}{\pgfqpoint{10.427240in}{5.263521in}}%
\pgfpathcurveto{\pgfqpoint{10.427240in}{5.274571in}}{\pgfqpoint{10.422849in}{5.285170in}}{\pgfqpoint{10.415036in}{5.292984in}}%
\pgfpathcurveto{\pgfqpoint{10.407222in}{5.300798in}}{\pgfqpoint{10.396623in}{5.305188in}}{\pgfqpoint{10.385573in}{5.305188in}}%
\pgfpathcurveto{\pgfqpoint{10.374523in}{5.305188in}}{\pgfqpoint{10.363924in}{5.300798in}}{\pgfqpoint{10.356110in}{5.292984in}}%
\pgfpathcurveto{\pgfqpoint{10.348297in}{5.285170in}}{\pgfqpoint{10.343906in}{5.274571in}}{\pgfqpoint{10.343906in}{5.263521in}}%
\pgfpathcurveto{\pgfqpoint{10.343906in}{5.252471in}}{\pgfqpoint{10.348297in}{5.241872in}}{\pgfqpoint{10.356110in}{5.234058in}}%
\pgfpathcurveto{\pgfqpoint{10.363924in}{5.226245in}}{\pgfqpoint{10.374523in}{5.221855in}}{\pgfqpoint{10.385573in}{5.221855in}}%
\pgfpathlineto{\pgfqpoint{10.385573in}{5.221855in}}%
\pgfpathclose%
\pgfusepath{stroke}%
\end{pgfscope}%
\begin{pgfscope}%
\pgfpathrectangle{\pgfqpoint{7.512535in}{0.437222in}}{\pgfqpoint{6.275590in}{5.159444in}}%
\pgfusepath{clip}%
\pgfsetbuttcap%
\pgfsetroundjoin%
\pgfsetlinewidth{1.003750pt}%
\definecolor{currentstroke}{rgb}{0.827451,0.827451,0.827451}%
\pgfsetstrokecolor{currentstroke}%
\pgfsetstrokeopacity{0.800000}%
\pgfsetdash{}{0pt}%
\pgfpathmoveto{\pgfqpoint{10.140717in}{3.555415in}}%
\pgfpathcurveto{\pgfqpoint{10.151767in}{3.555415in}}{\pgfqpoint{10.162367in}{3.559805in}}{\pgfqpoint{10.170180in}{3.567618in}}%
\pgfpathcurveto{\pgfqpoint{10.177994in}{3.575432in}}{\pgfqpoint{10.182384in}{3.586031in}}{\pgfqpoint{10.182384in}{3.597081in}}%
\pgfpathcurveto{\pgfqpoint{10.182384in}{3.608131in}}{\pgfqpoint{10.177994in}{3.618730in}}{\pgfqpoint{10.170180in}{3.626544in}}%
\pgfpathcurveto{\pgfqpoint{10.162367in}{3.634358in}}{\pgfqpoint{10.151767in}{3.638748in}}{\pgfqpoint{10.140717in}{3.638748in}}%
\pgfpathcurveto{\pgfqpoint{10.129667in}{3.638748in}}{\pgfqpoint{10.119068in}{3.634358in}}{\pgfqpoint{10.111255in}{3.626544in}}%
\pgfpathcurveto{\pgfqpoint{10.103441in}{3.618730in}}{\pgfqpoint{10.099051in}{3.608131in}}{\pgfqpoint{10.099051in}{3.597081in}}%
\pgfpathcurveto{\pgfqpoint{10.099051in}{3.586031in}}{\pgfqpoint{10.103441in}{3.575432in}}{\pgfqpoint{10.111255in}{3.567618in}}%
\pgfpathcurveto{\pgfqpoint{10.119068in}{3.559805in}}{\pgfqpoint{10.129667in}{3.555415in}}{\pgfqpoint{10.140717in}{3.555415in}}%
\pgfpathlineto{\pgfqpoint{10.140717in}{3.555415in}}%
\pgfpathclose%
\pgfusepath{stroke}%
\end{pgfscope}%
\begin{pgfscope}%
\pgfpathrectangle{\pgfqpoint{7.512535in}{0.437222in}}{\pgfqpoint{6.275590in}{5.159444in}}%
\pgfusepath{clip}%
\pgfsetbuttcap%
\pgfsetroundjoin%
\pgfsetlinewidth{1.003750pt}%
\definecolor{currentstroke}{rgb}{0.827451,0.827451,0.827451}%
\pgfsetstrokecolor{currentstroke}%
\pgfsetstrokeopacity{0.800000}%
\pgfsetdash{}{0pt}%
\pgfpathmoveto{\pgfqpoint{11.645781in}{5.002607in}}%
\pgfpathcurveto{\pgfqpoint{11.656832in}{5.002607in}}{\pgfqpoint{11.667431in}{5.006998in}}{\pgfqpoint{11.675244in}{5.014811in}}%
\pgfpathcurveto{\pgfqpoint{11.683058in}{5.022625in}}{\pgfqpoint{11.687448in}{5.033224in}}{\pgfqpoint{11.687448in}{5.044274in}}%
\pgfpathcurveto{\pgfqpoint{11.687448in}{5.055324in}}{\pgfqpoint{11.683058in}{5.065923in}}{\pgfqpoint{11.675244in}{5.073737in}}%
\pgfpathcurveto{\pgfqpoint{11.667431in}{5.081550in}}{\pgfqpoint{11.656832in}{5.085941in}}{\pgfqpoint{11.645781in}{5.085941in}}%
\pgfpathcurveto{\pgfqpoint{11.634731in}{5.085941in}}{\pgfqpoint{11.624132in}{5.081550in}}{\pgfqpoint{11.616319in}{5.073737in}}%
\pgfpathcurveto{\pgfqpoint{11.608505in}{5.065923in}}{\pgfqpoint{11.604115in}{5.055324in}}{\pgfqpoint{11.604115in}{5.044274in}}%
\pgfpathcurveto{\pgfqpoint{11.604115in}{5.033224in}}{\pgfqpoint{11.608505in}{5.022625in}}{\pgfqpoint{11.616319in}{5.014811in}}%
\pgfpathcurveto{\pgfqpoint{11.624132in}{5.006998in}}{\pgfqpoint{11.634731in}{5.002607in}}{\pgfqpoint{11.645781in}{5.002607in}}%
\pgfpathlineto{\pgfqpoint{11.645781in}{5.002607in}}%
\pgfpathclose%
\pgfusepath{stroke}%
\end{pgfscope}%
\begin{pgfscope}%
\pgfpathrectangle{\pgfqpoint{7.512535in}{0.437222in}}{\pgfqpoint{6.275590in}{5.159444in}}%
\pgfusepath{clip}%
\pgfsetbuttcap%
\pgfsetroundjoin%
\pgfsetlinewidth{1.003750pt}%
\definecolor{currentstroke}{rgb}{0.827451,0.827451,0.827451}%
\pgfsetstrokecolor{currentstroke}%
\pgfsetstrokeopacity{0.800000}%
\pgfsetdash{}{0pt}%
\pgfpathmoveto{\pgfqpoint{11.888148in}{5.414422in}}%
\pgfpathcurveto{\pgfqpoint{11.899198in}{5.414422in}}{\pgfqpoint{11.909797in}{5.418812in}}{\pgfqpoint{11.917611in}{5.426626in}}%
\pgfpathcurveto{\pgfqpoint{11.925424in}{5.434440in}}{\pgfqpoint{11.929815in}{5.445039in}}{\pgfqpoint{11.929815in}{5.456089in}}%
\pgfpathcurveto{\pgfqpoint{11.929815in}{5.467139in}}{\pgfqpoint{11.925424in}{5.477738in}}{\pgfqpoint{11.917611in}{5.485551in}}%
\pgfpathcurveto{\pgfqpoint{11.909797in}{5.493365in}}{\pgfqpoint{11.899198in}{5.497755in}}{\pgfqpoint{11.888148in}{5.497755in}}%
\pgfpathcurveto{\pgfqpoint{11.877098in}{5.497755in}}{\pgfqpoint{11.866499in}{5.493365in}}{\pgfqpoint{11.858685in}{5.485551in}}%
\pgfpathcurveto{\pgfqpoint{11.850872in}{5.477738in}}{\pgfqpoint{11.846481in}{5.467139in}}{\pgfqpoint{11.846481in}{5.456089in}}%
\pgfpathcurveto{\pgfqpoint{11.846481in}{5.445039in}}{\pgfqpoint{11.850872in}{5.434440in}}{\pgfqpoint{11.858685in}{5.426626in}}%
\pgfpathcurveto{\pgfqpoint{11.866499in}{5.418812in}}{\pgfqpoint{11.877098in}{5.414422in}}{\pgfqpoint{11.888148in}{5.414422in}}%
\pgfpathlineto{\pgfqpoint{11.888148in}{5.414422in}}%
\pgfpathclose%
\pgfusepath{stroke}%
\end{pgfscope}%
\begin{pgfscope}%
\pgfpathrectangle{\pgfqpoint{7.512535in}{0.437222in}}{\pgfqpoint{6.275590in}{5.159444in}}%
\pgfusepath{clip}%
\pgfsetbuttcap%
\pgfsetroundjoin%
\pgfsetlinewidth{1.003750pt}%
\definecolor{currentstroke}{rgb}{0.827451,0.827451,0.827451}%
\pgfsetstrokecolor{currentstroke}%
\pgfsetstrokeopacity{0.800000}%
\pgfsetdash{}{0pt}%
\pgfpathmoveto{\pgfqpoint{8.598634in}{1.552340in}}%
\pgfpathcurveto{\pgfqpoint{8.609684in}{1.552340in}}{\pgfqpoint{8.620283in}{1.556730in}}{\pgfqpoint{8.628097in}{1.564544in}}%
\pgfpathcurveto{\pgfqpoint{8.635910in}{1.572358in}}{\pgfqpoint{8.640300in}{1.582957in}}{\pgfqpoint{8.640300in}{1.594007in}}%
\pgfpathcurveto{\pgfqpoint{8.640300in}{1.605057in}}{\pgfqpoint{8.635910in}{1.615656in}}{\pgfqpoint{8.628097in}{1.623470in}}%
\pgfpathcurveto{\pgfqpoint{8.620283in}{1.631283in}}{\pgfqpoint{8.609684in}{1.635673in}}{\pgfqpoint{8.598634in}{1.635673in}}%
\pgfpathcurveto{\pgfqpoint{8.587584in}{1.635673in}}{\pgfqpoint{8.576985in}{1.631283in}}{\pgfqpoint{8.569171in}{1.623470in}}%
\pgfpathcurveto{\pgfqpoint{8.561357in}{1.615656in}}{\pgfqpoint{8.556967in}{1.605057in}}{\pgfqpoint{8.556967in}{1.594007in}}%
\pgfpathcurveto{\pgfqpoint{8.556967in}{1.582957in}}{\pgfqpoint{8.561357in}{1.572358in}}{\pgfqpoint{8.569171in}{1.564544in}}%
\pgfpathcurveto{\pgfqpoint{8.576985in}{1.556730in}}{\pgfqpoint{8.587584in}{1.552340in}}{\pgfqpoint{8.598634in}{1.552340in}}%
\pgfpathlineto{\pgfqpoint{8.598634in}{1.552340in}}%
\pgfpathclose%
\pgfusepath{stroke}%
\end{pgfscope}%
\begin{pgfscope}%
\pgfpathrectangle{\pgfqpoint{7.512535in}{0.437222in}}{\pgfqpoint{6.275590in}{5.159444in}}%
\pgfusepath{clip}%
\pgfsetbuttcap%
\pgfsetroundjoin%
\pgfsetlinewidth{1.003750pt}%
\definecolor{currentstroke}{rgb}{0.827451,0.827451,0.827451}%
\pgfsetstrokecolor{currentstroke}%
\pgfsetstrokeopacity{0.800000}%
\pgfsetdash{}{0pt}%
\pgfpathmoveto{\pgfqpoint{7.748566in}{0.615504in}}%
\pgfpathcurveto{\pgfqpoint{7.759616in}{0.615504in}}{\pgfqpoint{7.770215in}{0.619894in}}{\pgfqpoint{7.778028in}{0.627708in}}%
\pgfpathcurveto{\pgfqpoint{7.785842in}{0.635522in}}{\pgfqpoint{7.790232in}{0.646121in}}{\pgfqpoint{7.790232in}{0.657171in}}%
\pgfpathcurveto{\pgfqpoint{7.790232in}{0.668221in}}{\pgfqpoint{7.785842in}{0.678820in}}{\pgfqpoint{7.778028in}{0.686633in}}%
\pgfpathcurveto{\pgfqpoint{7.770215in}{0.694447in}}{\pgfqpoint{7.759616in}{0.698837in}}{\pgfqpoint{7.748566in}{0.698837in}}%
\pgfpathcurveto{\pgfqpoint{7.737516in}{0.698837in}}{\pgfqpoint{7.726917in}{0.694447in}}{\pgfqpoint{7.719103in}{0.686633in}}%
\pgfpathcurveto{\pgfqpoint{7.711289in}{0.678820in}}{\pgfqpoint{7.706899in}{0.668221in}}{\pgfqpoint{7.706899in}{0.657171in}}%
\pgfpathcurveto{\pgfqpoint{7.706899in}{0.646121in}}{\pgfqpoint{7.711289in}{0.635522in}}{\pgfqpoint{7.719103in}{0.627708in}}%
\pgfpathcurveto{\pgfqpoint{7.726917in}{0.619894in}}{\pgfqpoint{7.737516in}{0.615504in}}{\pgfqpoint{7.748566in}{0.615504in}}%
\pgfpathlineto{\pgfqpoint{7.748566in}{0.615504in}}%
\pgfpathclose%
\pgfusepath{stroke}%
\end{pgfscope}%
\begin{pgfscope}%
\pgfpathrectangle{\pgfqpoint{7.512535in}{0.437222in}}{\pgfqpoint{6.275590in}{5.159444in}}%
\pgfusepath{clip}%
\pgfsetbuttcap%
\pgfsetroundjoin%
\pgfsetlinewidth{1.003750pt}%
\definecolor{currentstroke}{rgb}{0.827451,0.827451,0.827451}%
\pgfsetstrokecolor{currentstroke}%
\pgfsetstrokeopacity{0.800000}%
\pgfsetdash{}{0pt}%
\pgfpathmoveto{\pgfqpoint{8.724795in}{1.722485in}}%
\pgfpathcurveto{\pgfqpoint{8.735845in}{1.722485in}}{\pgfqpoint{8.746444in}{1.726876in}}{\pgfqpoint{8.754258in}{1.734689in}}%
\pgfpathcurveto{\pgfqpoint{8.762071in}{1.742503in}}{\pgfqpoint{8.766461in}{1.753102in}}{\pgfqpoint{8.766461in}{1.764152in}}%
\pgfpathcurveto{\pgfqpoint{8.766461in}{1.775202in}}{\pgfqpoint{8.762071in}{1.785801in}}{\pgfqpoint{8.754258in}{1.793615in}}%
\pgfpathcurveto{\pgfqpoint{8.746444in}{1.801428in}}{\pgfqpoint{8.735845in}{1.805819in}}{\pgfqpoint{8.724795in}{1.805819in}}%
\pgfpathcurveto{\pgfqpoint{8.713745in}{1.805819in}}{\pgfqpoint{8.703146in}{1.801428in}}{\pgfqpoint{8.695332in}{1.793615in}}%
\pgfpathcurveto{\pgfqpoint{8.687518in}{1.785801in}}{\pgfqpoint{8.683128in}{1.775202in}}{\pgfqpoint{8.683128in}{1.764152in}}%
\pgfpathcurveto{\pgfqpoint{8.683128in}{1.753102in}}{\pgfqpoint{8.687518in}{1.742503in}}{\pgfqpoint{8.695332in}{1.734689in}}%
\pgfpathcurveto{\pgfqpoint{8.703146in}{1.726876in}}{\pgfqpoint{8.713745in}{1.722485in}}{\pgfqpoint{8.724795in}{1.722485in}}%
\pgfpathlineto{\pgfqpoint{8.724795in}{1.722485in}}%
\pgfpathclose%
\pgfusepath{stroke}%
\end{pgfscope}%
\begin{pgfscope}%
\pgfpathrectangle{\pgfqpoint{7.512535in}{0.437222in}}{\pgfqpoint{6.275590in}{5.159444in}}%
\pgfusepath{clip}%
\pgfsetbuttcap%
\pgfsetroundjoin%
\pgfsetlinewidth{1.003750pt}%
\definecolor{currentstroke}{rgb}{0.827451,0.827451,0.827451}%
\pgfsetstrokecolor{currentstroke}%
\pgfsetstrokeopacity{0.800000}%
\pgfsetdash{}{0pt}%
\pgfpathmoveto{\pgfqpoint{10.836356in}{5.524317in}}%
\pgfpathcurveto{\pgfqpoint{10.847406in}{5.524317in}}{\pgfqpoint{10.858005in}{5.528707in}}{\pgfqpoint{10.865818in}{5.536521in}}%
\pgfpathcurveto{\pgfqpoint{10.873632in}{5.544335in}}{\pgfqpoint{10.878022in}{5.554934in}}{\pgfqpoint{10.878022in}{5.565984in}}%
\pgfpathcurveto{\pgfqpoint{10.878022in}{5.577034in}}{\pgfqpoint{10.873632in}{5.587633in}}{\pgfqpoint{10.865818in}{5.595446in}}%
\pgfpathcurveto{\pgfqpoint{10.858005in}{5.603260in}}{\pgfqpoint{10.847406in}{5.607650in}}{\pgfqpoint{10.836356in}{5.607650in}}%
\pgfpathcurveto{\pgfqpoint{10.825305in}{5.607650in}}{\pgfqpoint{10.814706in}{5.603260in}}{\pgfqpoint{10.806893in}{5.595446in}}%
\pgfpathcurveto{\pgfqpoint{10.799079in}{5.587633in}}{\pgfqpoint{10.794689in}{5.577034in}}{\pgfqpoint{10.794689in}{5.565984in}}%
\pgfpathcurveto{\pgfqpoint{10.794689in}{5.554934in}}{\pgfqpoint{10.799079in}{5.544335in}}{\pgfqpoint{10.806893in}{5.536521in}}%
\pgfpathcurveto{\pgfqpoint{10.814706in}{5.528707in}}{\pgfqpoint{10.825305in}{5.524317in}}{\pgfqpoint{10.836356in}{5.524317in}}%
\pgfpathlineto{\pgfqpoint{10.836356in}{5.524317in}}%
\pgfpathclose%
\pgfusepath{stroke}%
\end{pgfscope}%
\begin{pgfscope}%
\pgfpathrectangle{\pgfqpoint{7.512535in}{0.437222in}}{\pgfqpoint{6.275590in}{5.159444in}}%
\pgfusepath{clip}%
\pgfsetbuttcap%
\pgfsetroundjoin%
\pgfsetlinewidth{1.003750pt}%
\definecolor{currentstroke}{rgb}{0.827451,0.827451,0.827451}%
\pgfsetstrokecolor{currentstroke}%
\pgfsetstrokeopacity{0.800000}%
\pgfsetdash{}{0pt}%
\pgfpathmoveto{\pgfqpoint{11.374803in}{5.081213in}}%
\pgfpathcurveto{\pgfqpoint{11.385853in}{5.081213in}}{\pgfqpoint{11.396452in}{5.085604in}}{\pgfqpoint{11.404266in}{5.093417in}}%
\pgfpathcurveto{\pgfqpoint{11.412080in}{5.101231in}}{\pgfqpoint{11.416470in}{5.111830in}}{\pgfqpoint{11.416470in}{5.122880in}}%
\pgfpathcurveto{\pgfqpoint{11.416470in}{5.133930in}}{\pgfqpoint{11.412080in}{5.144529in}}{\pgfqpoint{11.404266in}{5.152343in}}%
\pgfpathcurveto{\pgfqpoint{11.396452in}{5.160157in}}{\pgfqpoint{11.385853in}{5.164547in}}{\pgfqpoint{11.374803in}{5.164547in}}%
\pgfpathcurveto{\pgfqpoint{11.363753in}{5.164547in}}{\pgfqpoint{11.353154in}{5.160157in}}{\pgfqpoint{11.345340in}{5.152343in}}%
\pgfpathcurveto{\pgfqpoint{11.337527in}{5.144529in}}{\pgfqpoint{11.333136in}{5.133930in}}{\pgfqpoint{11.333136in}{5.122880in}}%
\pgfpathcurveto{\pgfqpoint{11.333136in}{5.111830in}}{\pgfqpoint{11.337527in}{5.101231in}}{\pgfqpoint{11.345340in}{5.093417in}}%
\pgfpathcurveto{\pgfqpoint{11.353154in}{5.085604in}}{\pgfqpoint{11.363753in}{5.081213in}}{\pgfqpoint{11.374803in}{5.081213in}}%
\pgfpathlineto{\pgfqpoint{11.374803in}{5.081213in}}%
\pgfpathclose%
\pgfusepath{stroke}%
\end{pgfscope}%
\begin{pgfscope}%
\pgfpathrectangle{\pgfqpoint{7.512535in}{0.437222in}}{\pgfqpoint{6.275590in}{5.159444in}}%
\pgfusepath{clip}%
\pgfsetbuttcap%
\pgfsetroundjoin%
\pgfsetlinewidth{1.003750pt}%
\definecolor{currentstroke}{rgb}{0.827451,0.827451,0.827451}%
\pgfsetstrokecolor{currentstroke}%
\pgfsetstrokeopacity{0.800000}%
\pgfsetdash{}{0pt}%
\pgfpathmoveto{\pgfqpoint{9.590173in}{3.398096in}}%
\pgfpathcurveto{\pgfqpoint{9.601223in}{3.398096in}}{\pgfqpoint{9.611822in}{3.402486in}}{\pgfqpoint{9.619635in}{3.410300in}}%
\pgfpathcurveto{\pgfqpoint{9.627449in}{3.418114in}}{\pgfqpoint{9.631839in}{3.428713in}}{\pgfqpoint{9.631839in}{3.439763in}}%
\pgfpathcurveto{\pgfqpoint{9.631839in}{3.450813in}}{\pgfqpoint{9.627449in}{3.461412in}}{\pgfqpoint{9.619635in}{3.469226in}}%
\pgfpathcurveto{\pgfqpoint{9.611822in}{3.477039in}}{\pgfqpoint{9.601223in}{3.481429in}}{\pgfqpoint{9.590173in}{3.481429in}}%
\pgfpathcurveto{\pgfqpoint{9.579122in}{3.481429in}}{\pgfqpoint{9.568523in}{3.477039in}}{\pgfqpoint{9.560710in}{3.469226in}}%
\pgfpathcurveto{\pgfqpoint{9.552896in}{3.461412in}}{\pgfqpoint{9.548506in}{3.450813in}}{\pgfqpoint{9.548506in}{3.439763in}}%
\pgfpathcurveto{\pgfqpoint{9.548506in}{3.428713in}}{\pgfqpoint{9.552896in}{3.418114in}}{\pgfqpoint{9.560710in}{3.410300in}}%
\pgfpathcurveto{\pgfqpoint{9.568523in}{3.402486in}}{\pgfqpoint{9.579122in}{3.398096in}}{\pgfqpoint{9.590173in}{3.398096in}}%
\pgfpathlineto{\pgfqpoint{9.590173in}{3.398096in}}%
\pgfpathclose%
\pgfusepath{stroke}%
\end{pgfscope}%
\begin{pgfscope}%
\pgfpathrectangle{\pgfqpoint{7.512535in}{0.437222in}}{\pgfqpoint{6.275590in}{5.159444in}}%
\pgfusepath{clip}%
\pgfsetbuttcap%
\pgfsetroundjoin%
\pgfsetlinewidth{1.003750pt}%
\definecolor{currentstroke}{rgb}{0.827451,0.827451,0.827451}%
\pgfsetstrokecolor{currentstroke}%
\pgfsetstrokeopacity{0.800000}%
\pgfsetdash{}{0pt}%
\pgfpathmoveto{\pgfqpoint{9.630920in}{3.276112in}}%
\pgfpathcurveto{\pgfqpoint{9.641970in}{3.276112in}}{\pgfqpoint{9.652569in}{3.280502in}}{\pgfqpoint{9.660383in}{3.288316in}}%
\pgfpathcurveto{\pgfqpoint{9.668196in}{3.296130in}}{\pgfqpoint{9.672587in}{3.306729in}}{\pgfqpoint{9.672587in}{3.317779in}}%
\pgfpathcurveto{\pgfqpoint{9.672587in}{3.328829in}}{\pgfqpoint{9.668196in}{3.339428in}}{\pgfqpoint{9.660383in}{3.347241in}}%
\pgfpathcurveto{\pgfqpoint{9.652569in}{3.355055in}}{\pgfqpoint{9.641970in}{3.359445in}}{\pgfqpoint{9.630920in}{3.359445in}}%
\pgfpathcurveto{\pgfqpoint{9.619870in}{3.359445in}}{\pgfqpoint{9.609271in}{3.355055in}}{\pgfqpoint{9.601457in}{3.347241in}}%
\pgfpathcurveto{\pgfqpoint{9.593644in}{3.339428in}}{\pgfqpoint{9.589253in}{3.328829in}}{\pgfqpoint{9.589253in}{3.317779in}}%
\pgfpathcurveto{\pgfqpoint{9.589253in}{3.306729in}}{\pgfqpoint{9.593644in}{3.296130in}}{\pgfqpoint{9.601457in}{3.288316in}}%
\pgfpathcurveto{\pgfqpoint{9.609271in}{3.280502in}}{\pgfqpoint{9.619870in}{3.276112in}}{\pgfqpoint{9.630920in}{3.276112in}}%
\pgfpathlineto{\pgfqpoint{9.630920in}{3.276112in}}%
\pgfpathclose%
\pgfusepath{stroke}%
\end{pgfscope}%
\begin{pgfscope}%
\pgfpathrectangle{\pgfqpoint{7.512535in}{0.437222in}}{\pgfqpoint{6.275590in}{5.159444in}}%
\pgfusepath{clip}%
\pgfsetbuttcap%
\pgfsetroundjoin%
\pgfsetlinewidth{1.003750pt}%
\definecolor{currentstroke}{rgb}{0.827451,0.827451,0.827451}%
\pgfsetstrokecolor{currentstroke}%
\pgfsetstrokeopacity{0.800000}%
\pgfsetdash{}{0pt}%
\pgfpathmoveto{\pgfqpoint{12.384977in}{5.553091in}}%
\pgfpathcurveto{\pgfqpoint{12.396028in}{5.553091in}}{\pgfqpoint{12.406627in}{5.557481in}}{\pgfqpoint{12.414440in}{5.565295in}}%
\pgfpathcurveto{\pgfqpoint{12.422254in}{5.573108in}}{\pgfqpoint{12.426644in}{5.583707in}}{\pgfqpoint{12.426644in}{5.594757in}}%
\pgfpathcurveto{\pgfqpoint{12.426644in}{5.605808in}}{\pgfqpoint{12.422254in}{5.616407in}}{\pgfqpoint{12.414440in}{5.624220in}}%
\pgfpathcurveto{\pgfqpoint{12.406627in}{5.632034in}}{\pgfqpoint{12.396028in}{5.636424in}}{\pgfqpoint{12.384977in}{5.636424in}}%
\pgfpathcurveto{\pgfqpoint{12.373927in}{5.636424in}}{\pgfqpoint{12.363328in}{5.632034in}}{\pgfqpoint{12.355515in}{5.624220in}}%
\pgfpathcurveto{\pgfqpoint{12.347701in}{5.616407in}}{\pgfqpoint{12.343311in}{5.605808in}}{\pgfqpoint{12.343311in}{5.594757in}}%
\pgfpathcurveto{\pgfqpoint{12.343311in}{5.583707in}}{\pgfqpoint{12.347701in}{5.573108in}}{\pgfqpoint{12.355515in}{5.565295in}}%
\pgfpathcurveto{\pgfqpoint{12.363328in}{5.557481in}}{\pgfqpoint{12.373927in}{5.553091in}}{\pgfqpoint{12.384977in}{5.553091in}}%
\pgfpathlineto{\pgfqpoint{12.384977in}{5.553091in}}%
\pgfpathclose%
\pgfusepath{stroke}%
\end{pgfscope}%
\begin{pgfscope}%
\pgfpathrectangle{\pgfqpoint{7.512535in}{0.437222in}}{\pgfqpoint{6.275590in}{5.159444in}}%
\pgfusepath{clip}%
\pgfsetbuttcap%
\pgfsetroundjoin%
\pgfsetlinewidth{1.003750pt}%
\definecolor{currentstroke}{rgb}{0.827451,0.827451,0.827451}%
\pgfsetstrokecolor{currentstroke}%
\pgfsetstrokeopacity{0.800000}%
\pgfsetdash{}{0pt}%
\pgfpathmoveto{\pgfqpoint{10.369675in}{4.552452in}}%
\pgfpathcurveto{\pgfqpoint{10.380725in}{4.552452in}}{\pgfqpoint{10.391324in}{4.556842in}}{\pgfqpoint{10.399138in}{4.564656in}}%
\pgfpathcurveto{\pgfqpoint{10.406951in}{4.572470in}}{\pgfqpoint{10.411341in}{4.583069in}}{\pgfqpoint{10.411341in}{4.594119in}}%
\pgfpathcurveto{\pgfqpoint{10.411341in}{4.605169in}}{\pgfqpoint{10.406951in}{4.615768in}}{\pgfqpoint{10.399138in}{4.623582in}}%
\pgfpathcurveto{\pgfqpoint{10.391324in}{4.631395in}}{\pgfqpoint{10.380725in}{4.635785in}}{\pgfqpoint{10.369675in}{4.635785in}}%
\pgfpathcurveto{\pgfqpoint{10.358625in}{4.635785in}}{\pgfqpoint{10.348026in}{4.631395in}}{\pgfqpoint{10.340212in}{4.623582in}}%
\pgfpathcurveto{\pgfqpoint{10.332398in}{4.615768in}}{\pgfqpoint{10.328008in}{4.605169in}}{\pgfqpoint{10.328008in}{4.594119in}}%
\pgfpathcurveto{\pgfqpoint{10.328008in}{4.583069in}}{\pgfqpoint{10.332398in}{4.572470in}}{\pgfqpoint{10.340212in}{4.564656in}}%
\pgfpathcurveto{\pgfqpoint{10.348026in}{4.556842in}}{\pgfqpoint{10.358625in}{4.552452in}}{\pgfqpoint{10.369675in}{4.552452in}}%
\pgfpathlineto{\pgfqpoint{10.369675in}{4.552452in}}%
\pgfpathclose%
\pgfusepath{stroke}%
\end{pgfscope}%
\begin{pgfscope}%
\pgfpathrectangle{\pgfqpoint{7.512535in}{0.437222in}}{\pgfqpoint{6.275590in}{5.159444in}}%
\pgfusepath{clip}%
\pgfsetbuttcap%
\pgfsetroundjoin%
\pgfsetlinewidth{1.003750pt}%
\definecolor{currentstroke}{rgb}{0.827451,0.827451,0.827451}%
\pgfsetstrokecolor{currentstroke}%
\pgfsetstrokeopacity{0.800000}%
\pgfsetdash{}{0pt}%
\pgfpathmoveto{\pgfqpoint{12.342968in}{5.553091in}}%
\pgfpathcurveto{\pgfqpoint{12.354019in}{5.553091in}}{\pgfqpoint{12.364618in}{5.557481in}}{\pgfqpoint{12.372431in}{5.565295in}}%
\pgfpathcurveto{\pgfqpoint{12.380245in}{5.573108in}}{\pgfqpoint{12.384635in}{5.583707in}}{\pgfqpoint{12.384635in}{5.594757in}}%
\pgfpathcurveto{\pgfqpoint{12.384635in}{5.605808in}}{\pgfqpoint{12.380245in}{5.616407in}}{\pgfqpoint{12.372431in}{5.624220in}}%
\pgfpathcurveto{\pgfqpoint{12.364618in}{5.632034in}}{\pgfqpoint{12.354019in}{5.636424in}}{\pgfqpoint{12.342968in}{5.636424in}}%
\pgfpathcurveto{\pgfqpoint{12.331918in}{5.636424in}}{\pgfqpoint{12.321319in}{5.632034in}}{\pgfqpoint{12.313506in}{5.624220in}}%
\pgfpathcurveto{\pgfqpoint{12.305692in}{5.616407in}}{\pgfqpoint{12.301302in}{5.605808in}}{\pgfqpoint{12.301302in}{5.594757in}}%
\pgfpathcurveto{\pgfqpoint{12.301302in}{5.583707in}}{\pgfqpoint{12.305692in}{5.573108in}}{\pgfqpoint{12.313506in}{5.565295in}}%
\pgfpathcurveto{\pgfqpoint{12.321319in}{5.557481in}}{\pgfqpoint{12.331918in}{5.553091in}}{\pgfqpoint{12.342968in}{5.553091in}}%
\pgfpathlineto{\pgfqpoint{12.342968in}{5.553091in}}%
\pgfpathclose%
\pgfusepath{stroke}%
\end{pgfscope}%
\begin{pgfscope}%
\pgfpathrectangle{\pgfqpoint{7.512535in}{0.437222in}}{\pgfqpoint{6.275590in}{5.159444in}}%
\pgfusepath{clip}%
\pgfsetbuttcap%
\pgfsetroundjoin%
\pgfsetlinewidth{1.003750pt}%
\definecolor{currentstroke}{rgb}{0.827451,0.827451,0.827451}%
\pgfsetstrokecolor{currentstroke}%
\pgfsetstrokeopacity{0.800000}%
\pgfsetdash{}{0pt}%
\pgfpathmoveto{\pgfqpoint{9.615954in}{3.788157in}}%
\pgfpathcurveto{\pgfqpoint{9.627004in}{3.788157in}}{\pgfqpoint{9.637603in}{3.792548in}}{\pgfqpoint{9.645416in}{3.800361in}}%
\pgfpathcurveto{\pgfqpoint{9.653230in}{3.808175in}}{\pgfqpoint{9.657620in}{3.818774in}}{\pgfqpoint{9.657620in}{3.829824in}}%
\pgfpathcurveto{\pgfqpoint{9.657620in}{3.840874in}}{\pgfqpoint{9.653230in}{3.851473in}}{\pgfqpoint{9.645416in}{3.859287in}}%
\pgfpathcurveto{\pgfqpoint{9.637603in}{3.867100in}}{\pgfqpoint{9.627004in}{3.871491in}}{\pgfqpoint{9.615954in}{3.871491in}}%
\pgfpathcurveto{\pgfqpoint{9.604903in}{3.871491in}}{\pgfqpoint{9.594304in}{3.867100in}}{\pgfqpoint{9.586491in}{3.859287in}}%
\pgfpathcurveto{\pgfqpoint{9.578677in}{3.851473in}}{\pgfqpoint{9.574287in}{3.840874in}}{\pgfqpoint{9.574287in}{3.829824in}}%
\pgfpathcurveto{\pgfqpoint{9.574287in}{3.818774in}}{\pgfqpoint{9.578677in}{3.808175in}}{\pgfqpoint{9.586491in}{3.800361in}}%
\pgfpathcurveto{\pgfqpoint{9.594304in}{3.792548in}}{\pgfqpoint{9.604903in}{3.788157in}}{\pgfqpoint{9.615954in}{3.788157in}}%
\pgfpathlineto{\pgfqpoint{9.615954in}{3.788157in}}%
\pgfpathclose%
\pgfusepath{stroke}%
\end{pgfscope}%
\begin{pgfscope}%
\pgfpathrectangle{\pgfqpoint{7.512535in}{0.437222in}}{\pgfqpoint{6.275590in}{5.159444in}}%
\pgfusepath{clip}%
\pgfsetbuttcap%
\pgfsetroundjoin%
\pgfsetlinewidth{1.003750pt}%
\definecolor{currentstroke}{rgb}{0.827451,0.827451,0.827451}%
\pgfsetstrokecolor{currentstroke}%
\pgfsetstrokeopacity{0.800000}%
\pgfsetdash{}{0pt}%
\pgfpathmoveto{\pgfqpoint{8.907940in}{2.091479in}}%
\pgfpathcurveto{\pgfqpoint{8.918990in}{2.091479in}}{\pgfqpoint{8.929589in}{2.095869in}}{\pgfqpoint{8.937403in}{2.103683in}}%
\pgfpathcurveto{\pgfqpoint{8.945217in}{2.111496in}}{\pgfqpoint{8.949607in}{2.122095in}}{\pgfqpoint{8.949607in}{2.133145in}}%
\pgfpathcurveto{\pgfqpoint{8.949607in}{2.144195in}}{\pgfqpoint{8.945217in}{2.154795in}}{\pgfqpoint{8.937403in}{2.162608in}}%
\pgfpathcurveto{\pgfqpoint{8.929589in}{2.170422in}}{\pgfqpoint{8.918990in}{2.174812in}}{\pgfqpoint{8.907940in}{2.174812in}}%
\pgfpathcurveto{\pgfqpoint{8.896890in}{2.174812in}}{\pgfqpoint{8.886291in}{2.170422in}}{\pgfqpoint{8.878477in}{2.162608in}}%
\pgfpathcurveto{\pgfqpoint{8.870664in}{2.154795in}}{\pgfqpoint{8.866274in}{2.144195in}}{\pgfqpoint{8.866274in}{2.133145in}}%
\pgfpathcurveto{\pgfqpoint{8.866274in}{2.122095in}}{\pgfqpoint{8.870664in}{2.111496in}}{\pgfqpoint{8.878477in}{2.103683in}}%
\pgfpathcurveto{\pgfqpoint{8.886291in}{2.095869in}}{\pgfqpoint{8.896890in}{2.091479in}}{\pgfqpoint{8.907940in}{2.091479in}}%
\pgfpathlineto{\pgfqpoint{8.907940in}{2.091479in}}%
\pgfpathclose%
\pgfusepath{stroke}%
\end{pgfscope}%
\begin{pgfscope}%
\pgfpathrectangle{\pgfqpoint{7.512535in}{0.437222in}}{\pgfqpoint{6.275590in}{5.159444in}}%
\pgfusepath{clip}%
\pgfsetbuttcap%
\pgfsetroundjoin%
\pgfsetlinewidth{1.003750pt}%
\definecolor{currentstroke}{rgb}{0.827451,0.827451,0.827451}%
\pgfsetstrokecolor{currentstroke}%
\pgfsetstrokeopacity{0.800000}%
\pgfsetdash{}{0pt}%
\pgfpathmoveto{\pgfqpoint{13.731324in}{5.553845in}}%
\pgfpathcurveto{\pgfqpoint{13.742374in}{5.553845in}}{\pgfqpoint{13.752973in}{5.558236in}}{\pgfqpoint{13.760787in}{5.566049in}}%
\pgfpathcurveto{\pgfqpoint{13.768601in}{5.573863in}}{\pgfqpoint{13.772991in}{5.584462in}}{\pgfqpoint{13.772991in}{5.595512in}}%
\pgfpathcurveto{\pgfqpoint{13.772991in}{5.606562in}}{\pgfqpoint{13.768601in}{5.617161in}}{\pgfqpoint{13.760787in}{5.624975in}}%
\pgfpathcurveto{\pgfqpoint{13.752973in}{5.632788in}}{\pgfqpoint{13.742374in}{5.637179in}}{\pgfqpoint{13.731324in}{5.637179in}}%
\pgfpathcurveto{\pgfqpoint{13.720274in}{5.637179in}}{\pgfqpoint{13.709675in}{5.632788in}}{\pgfqpoint{13.701861in}{5.624975in}}%
\pgfpathcurveto{\pgfqpoint{13.694048in}{5.617161in}}{\pgfqpoint{13.689657in}{5.606562in}}{\pgfqpoint{13.689657in}{5.595512in}}%
\pgfpathcurveto{\pgfqpoint{13.689657in}{5.584462in}}{\pgfqpoint{13.694048in}{5.573863in}}{\pgfqpoint{13.701861in}{5.566049in}}%
\pgfpathcurveto{\pgfqpoint{13.709675in}{5.558236in}}{\pgfqpoint{13.720274in}{5.553845in}}{\pgfqpoint{13.731324in}{5.553845in}}%
\pgfpathlineto{\pgfqpoint{13.731324in}{5.553845in}}%
\pgfpathclose%
\pgfusepath{stroke}%
\end{pgfscope}%
\begin{pgfscope}%
\pgfpathrectangle{\pgfqpoint{7.512535in}{0.437222in}}{\pgfqpoint{6.275590in}{5.159444in}}%
\pgfusepath{clip}%
\pgfsetbuttcap%
\pgfsetroundjoin%
\pgfsetlinewidth{1.003750pt}%
\definecolor{currentstroke}{rgb}{0.827451,0.827451,0.827451}%
\pgfsetstrokecolor{currentstroke}%
\pgfsetstrokeopacity{0.800000}%
\pgfsetdash{}{0pt}%
\pgfpathmoveto{\pgfqpoint{7.562575in}{0.440656in}}%
\pgfpathcurveto{\pgfqpoint{7.573625in}{0.440656in}}{\pgfqpoint{7.584224in}{0.445046in}}{\pgfqpoint{7.592038in}{0.452860in}}%
\pgfpathcurveto{\pgfqpoint{7.599851in}{0.460673in}}{\pgfqpoint{7.604241in}{0.471272in}}{\pgfqpoint{7.604241in}{0.482323in}}%
\pgfpathcurveto{\pgfqpoint{7.604241in}{0.493373in}}{\pgfqpoint{7.599851in}{0.503972in}}{\pgfqpoint{7.592038in}{0.511785in}}%
\pgfpathcurveto{\pgfqpoint{7.584224in}{0.519599in}}{\pgfqpoint{7.573625in}{0.523989in}}{\pgfqpoint{7.562575in}{0.523989in}}%
\pgfpathcurveto{\pgfqpoint{7.551525in}{0.523989in}}{\pgfqpoint{7.540926in}{0.519599in}}{\pgfqpoint{7.533112in}{0.511785in}}%
\pgfpathcurveto{\pgfqpoint{7.525298in}{0.503972in}}{\pgfqpoint{7.520908in}{0.493373in}}{\pgfqpoint{7.520908in}{0.482323in}}%
\pgfpathcurveto{\pgfqpoint{7.520908in}{0.471272in}}{\pgfqpoint{7.525298in}{0.460673in}}{\pgfqpoint{7.533112in}{0.452860in}}%
\pgfpathcurveto{\pgfqpoint{7.540926in}{0.445046in}}{\pgfqpoint{7.551525in}{0.440656in}}{\pgfqpoint{7.562575in}{0.440656in}}%
\pgfpathlineto{\pgfqpoint{7.562575in}{0.440656in}}%
\pgfpathclose%
\pgfusepath{stroke}%
\end{pgfscope}%
\begin{pgfscope}%
\pgfpathrectangle{\pgfqpoint{7.512535in}{0.437222in}}{\pgfqpoint{6.275590in}{5.159444in}}%
\pgfusepath{clip}%
\pgfsetbuttcap%
\pgfsetroundjoin%
\pgfsetlinewidth{1.003750pt}%
\definecolor{currentstroke}{rgb}{0.827451,0.827451,0.827451}%
\pgfsetstrokecolor{currentstroke}%
\pgfsetstrokeopacity{0.800000}%
\pgfsetdash{}{0pt}%
\pgfpathmoveto{\pgfqpoint{10.836356in}{5.524317in}}%
\pgfpathcurveto{\pgfqpoint{10.847406in}{5.524317in}}{\pgfqpoint{10.858005in}{5.528707in}}{\pgfqpoint{10.865818in}{5.536521in}}%
\pgfpathcurveto{\pgfqpoint{10.873632in}{5.544335in}}{\pgfqpoint{10.878022in}{5.554934in}}{\pgfqpoint{10.878022in}{5.565984in}}%
\pgfpathcurveto{\pgfqpoint{10.878022in}{5.577034in}}{\pgfqpoint{10.873632in}{5.587633in}}{\pgfqpoint{10.865818in}{5.595446in}}%
\pgfpathcurveto{\pgfqpoint{10.858005in}{5.603260in}}{\pgfqpoint{10.847406in}{5.607650in}}{\pgfqpoint{10.836356in}{5.607650in}}%
\pgfpathcurveto{\pgfqpoint{10.825305in}{5.607650in}}{\pgfqpoint{10.814706in}{5.603260in}}{\pgfqpoint{10.806893in}{5.595446in}}%
\pgfpathcurveto{\pgfqpoint{10.799079in}{5.587633in}}{\pgfqpoint{10.794689in}{5.577034in}}{\pgfqpoint{10.794689in}{5.565984in}}%
\pgfpathcurveto{\pgfqpoint{10.794689in}{5.554934in}}{\pgfqpoint{10.799079in}{5.544335in}}{\pgfqpoint{10.806893in}{5.536521in}}%
\pgfpathcurveto{\pgfqpoint{10.814706in}{5.528707in}}{\pgfqpoint{10.825305in}{5.524317in}}{\pgfqpoint{10.836356in}{5.524317in}}%
\pgfpathlineto{\pgfqpoint{10.836356in}{5.524317in}}%
\pgfpathclose%
\pgfusepath{stroke}%
\end{pgfscope}%
\begin{pgfscope}%
\pgfpathrectangle{\pgfqpoint{7.512535in}{0.437222in}}{\pgfqpoint{6.275590in}{5.159444in}}%
\pgfusepath{clip}%
\pgfsetbuttcap%
\pgfsetroundjoin%
\pgfsetlinewidth{1.003750pt}%
\definecolor{currentstroke}{rgb}{0.827451,0.827451,0.827451}%
\pgfsetstrokecolor{currentstroke}%
\pgfsetstrokeopacity{0.800000}%
\pgfsetdash{}{0pt}%
\pgfpathmoveto{\pgfqpoint{12.955580in}{5.537440in}}%
\pgfpathcurveto{\pgfqpoint{12.966630in}{5.537440in}}{\pgfqpoint{12.977229in}{5.541830in}}{\pgfqpoint{12.985043in}{5.549643in}}%
\pgfpathcurveto{\pgfqpoint{12.992856in}{5.557457in}}{\pgfqpoint{12.997246in}{5.568056in}}{\pgfqpoint{12.997246in}{5.579106in}}%
\pgfpathcurveto{\pgfqpoint{12.997246in}{5.590156in}}{\pgfqpoint{12.992856in}{5.600755in}}{\pgfqpoint{12.985043in}{5.608569in}}%
\pgfpathcurveto{\pgfqpoint{12.977229in}{5.616383in}}{\pgfqpoint{12.966630in}{5.620773in}}{\pgfqpoint{12.955580in}{5.620773in}}%
\pgfpathcurveto{\pgfqpoint{12.944530in}{5.620773in}}{\pgfqpoint{12.933931in}{5.616383in}}{\pgfqpoint{12.926117in}{5.608569in}}%
\pgfpathcurveto{\pgfqpoint{12.918303in}{5.600755in}}{\pgfqpoint{12.913913in}{5.590156in}}{\pgfqpoint{12.913913in}{5.579106in}}%
\pgfpathcurveto{\pgfqpoint{12.913913in}{5.568056in}}{\pgfqpoint{12.918303in}{5.557457in}}{\pgfqpoint{12.926117in}{5.549643in}}%
\pgfpathcurveto{\pgfqpoint{12.933931in}{5.541830in}}{\pgfqpoint{12.944530in}{5.537440in}}{\pgfqpoint{12.955580in}{5.537440in}}%
\pgfpathlineto{\pgfqpoint{12.955580in}{5.537440in}}%
\pgfpathclose%
\pgfusepath{stroke}%
\end{pgfscope}%
\begin{pgfscope}%
\pgfpathrectangle{\pgfqpoint{7.512535in}{0.437222in}}{\pgfqpoint{6.275590in}{5.159444in}}%
\pgfusepath{clip}%
\pgfsetbuttcap%
\pgfsetroundjoin%
\pgfsetlinewidth{1.003750pt}%
\definecolor{currentstroke}{rgb}{0.827451,0.827451,0.827451}%
\pgfsetstrokecolor{currentstroke}%
\pgfsetstrokeopacity{0.800000}%
\pgfsetdash{}{0pt}%
\pgfpathmoveto{\pgfqpoint{13.165622in}{5.501584in}}%
\pgfpathcurveto{\pgfqpoint{13.176672in}{5.501584in}}{\pgfqpoint{13.187271in}{5.505974in}}{\pgfqpoint{13.195085in}{5.513787in}}%
\pgfpathcurveto{\pgfqpoint{13.202899in}{5.521601in}}{\pgfqpoint{13.207289in}{5.532200in}}{\pgfqpoint{13.207289in}{5.543250in}}%
\pgfpathcurveto{\pgfqpoint{13.207289in}{5.554300in}}{\pgfqpoint{13.202899in}{5.564899in}}{\pgfqpoint{13.195085in}{5.572713in}}%
\pgfpathcurveto{\pgfqpoint{13.187271in}{5.580527in}}{\pgfqpoint{13.176672in}{5.584917in}}{\pgfqpoint{13.165622in}{5.584917in}}%
\pgfpathcurveto{\pgfqpoint{13.154572in}{5.584917in}}{\pgfqpoint{13.143973in}{5.580527in}}{\pgfqpoint{13.136159in}{5.572713in}}%
\pgfpathcurveto{\pgfqpoint{13.128346in}{5.564899in}}{\pgfqpoint{13.123955in}{5.554300in}}{\pgfqpoint{13.123955in}{5.543250in}}%
\pgfpathcurveto{\pgfqpoint{13.123955in}{5.532200in}}{\pgfqpoint{13.128346in}{5.521601in}}{\pgfqpoint{13.136159in}{5.513787in}}%
\pgfpathcurveto{\pgfqpoint{13.143973in}{5.505974in}}{\pgfqpoint{13.154572in}{5.501584in}}{\pgfqpoint{13.165622in}{5.501584in}}%
\pgfpathlineto{\pgfqpoint{13.165622in}{5.501584in}}%
\pgfpathclose%
\pgfusepath{stroke}%
\end{pgfscope}%
\begin{pgfscope}%
\pgfpathrectangle{\pgfqpoint{7.512535in}{0.437222in}}{\pgfqpoint{6.275590in}{5.159444in}}%
\pgfusepath{clip}%
\pgfsetbuttcap%
\pgfsetroundjoin%
\pgfsetlinewidth{1.003750pt}%
\definecolor{currentstroke}{rgb}{0.827451,0.827451,0.827451}%
\pgfsetstrokecolor{currentstroke}%
\pgfsetstrokeopacity{0.800000}%
\pgfsetdash{}{0pt}%
\pgfpathmoveto{\pgfqpoint{11.673162in}{5.362410in}}%
\pgfpathcurveto{\pgfqpoint{11.684212in}{5.362410in}}{\pgfqpoint{11.694811in}{5.366800in}}{\pgfqpoint{11.702624in}{5.374614in}}%
\pgfpathcurveto{\pgfqpoint{11.710438in}{5.382427in}}{\pgfqpoint{11.714828in}{5.393027in}}{\pgfqpoint{11.714828in}{5.404077in}}%
\pgfpathcurveto{\pgfqpoint{11.714828in}{5.415127in}}{\pgfqpoint{11.710438in}{5.425726in}}{\pgfqpoint{11.702624in}{5.433539in}}%
\pgfpathcurveto{\pgfqpoint{11.694811in}{5.441353in}}{\pgfqpoint{11.684212in}{5.445743in}}{\pgfqpoint{11.673162in}{5.445743in}}%
\pgfpathcurveto{\pgfqpoint{11.662111in}{5.445743in}}{\pgfqpoint{11.651512in}{5.441353in}}{\pgfqpoint{11.643699in}{5.433539in}}%
\pgfpathcurveto{\pgfqpoint{11.635885in}{5.425726in}}{\pgfqpoint{11.631495in}{5.415127in}}{\pgfqpoint{11.631495in}{5.404077in}}%
\pgfpathcurveto{\pgfqpoint{11.631495in}{5.393027in}}{\pgfqpoint{11.635885in}{5.382427in}}{\pgfqpoint{11.643699in}{5.374614in}}%
\pgfpathcurveto{\pgfqpoint{11.651512in}{5.366800in}}{\pgfqpoint{11.662111in}{5.362410in}}{\pgfqpoint{11.673162in}{5.362410in}}%
\pgfpathlineto{\pgfqpoint{11.673162in}{5.362410in}}%
\pgfpathclose%
\pgfusepath{stroke}%
\end{pgfscope}%
\begin{pgfscope}%
\pgfpathrectangle{\pgfqpoint{7.512535in}{0.437222in}}{\pgfqpoint{6.275590in}{5.159444in}}%
\pgfusepath{clip}%
\pgfsetbuttcap%
\pgfsetroundjoin%
\pgfsetlinewidth{1.003750pt}%
\definecolor{currentstroke}{rgb}{0.827451,0.827451,0.827451}%
\pgfsetstrokecolor{currentstroke}%
\pgfsetstrokeopacity{0.800000}%
\pgfsetdash{}{0pt}%
\pgfpathmoveto{\pgfqpoint{7.860715in}{0.988624in}}%
\pgfpathcurveto{\pgfqpoint{7.871765in}{0.988624in}}{\pgfqpoint{7.882364in}{0.993014in}}{\pgfqpoint{7.890178in}{1.000828in}}%
\pgfpathcurveto{\pgfqpoint{7.897991in}{1.008641in}}{\pgfqpoint{7.902381in}{1.019240in}}{\pgfqpoint{7.902381in}{1.030290in}}%
\pgfpathcurveto{\pgfqpoint{7.902381in}{1.041341in}}{\pgfqpoint{7.897991in}{1.051940in}}{\pgfqpoint{7.890178in}{1.059753in}}%
\pgfpathcurveto{\pgfqpoint{7.882364in}{1.067567in}}{\pgfqpoint{7.871765in}{1.071957in}}{\pgfqpoint{7.860715in}{1.071957in}}%
\pgfpathcurveto{\pgfqpoint{7.849665in}{1.071957in}}{\pgfqpoint{7.839066in}{1.067567in}}{\pgfqpoint{7.831252in}{1.059753in}}%
\pgfpathcurveto{\pgfqpoint{7.823438in}{1.051940in}}{\pgfqpoint{7.819048in}{1.041341in}}{\pgfqpoint{7.819048in}{1.030290in}}%
\pgfpathcurveto{\pgfqpoint{7.819048in}{1.019240in}}{\pgfqpoint{7.823438in}{1.008641in}}{\pgfqpoint{7.831252in}{1.000828in}}%
\pgfpathcurveto{\pgfqpoint{7.839066in}{0.993014in}}{\pgfqpoint{7.849665in}{0.988624in}}{\pgfqpoint{7.860715in}{0.988624in}}%
\pgfpathlineto{\pgfqpoint{7.860715in}{0.988624in}}%
\pgfpathclose%
\pgfusepath{stroke}%
\end{pgfscope}%
\begin{pgfscope}%
\pgfpathrectangle{\pgfqpoint{7.512535in}{0.437222in}}{\pgfqpoint{6.275590in}{5.159444in}}%
\pgfusepath{clip}%
\pgfsetbuttcap%
\pgfsetroundjoin%
\pgfsetlinewidth{1.003750pt}%
\definecolor{currentstroke}{rgb}{0.827451,0.827451,0.827451}%
\pgfsetstrokecolor{currentstroke}%
\pgfsetstrokeopacity{0.800000}%
\pgfsetdash{}{0pt}%
\pgfpathmoveto{\pgfqpoint{10.994755in}{5.060707in}}%
\pgfpathcurveto{\pgfqpoint{11.005805in}{5.060707in}}{\pgfqpoint{11.016404in}{5.065097in}}{\pgfqpoint{11.024218in}{5.072911in}}%
\pgfpathcurveto{\pgfqpoint{11.032031in}{5.080725in}}{\pgfqpoint{11.036422in}{5.091324in}}{\pgfqpoint{11.036422in}{5.102374in}}%
\pgfpathcurveto{\pgfqpoint{11.036422in}{5.113424in}}{\pgfqpoint{11.032031in}{5.124023in}}{\pgfqpoint{11.024218in}{5.131837in}}%
\pgfpathcurveto{\pgfqpoint{11.016404in}{5.139650in}}{\pgfqpoint{11.005805in}{5.144040in}}{\pgfqpoint{10.994755in}{5.144040in}}%
\pgfpathcurveto{\pgfqpoint{10.983705in}{5.144040in}}{\pgfqpoint{10.973106in}{5.139650in}}{\pgfqpoint{10.965292in}{5.131837in}}%
\pgfpathcurveto{\pgfqpoint{10.957479in}{5.124023in}}{\pgfqpoint{10.953088in}{5.113424in}}{\pgfqpoint{10.953088in}{5.102374in}}%
\pgfpathcurveto{\pgfqpoint{10.953088in}{5.091324in}}{\pgfqpoint{10.957479in}{5.080725in}}{\pgfqpoint{10.965292in}{5.072911in}}%
\pgfpathcurveto{\pgfqpoint{10.973106in}{5.065097in}}{\pgfqpoint{10.983705in}{5.060707in}}{\pgfqpoint{10.994755in}{5.060707in}}%
\pgfpathlineto{\pgfqpoint{10.994755in}{5.060707in}}%
\pgfpathclose%
\pgfusepath{stroke}%
\end{pgfscope}%
\begin{pgfscope}%
\pgfpathrectangle{\pgfqpoint{7.512535in}{0.437222in}}{\pgfqpoint{6.275590in}{5.159444in}}%
\pgfusepath{clip}%
\pgfsetbuttcap%
\pgfsetroundjoin%
\pgfsetlinewidth{1.003750pt}%
\definecolor{currentstroke}{rgb}{0.827451,0.827451,0.827451}%
\pgfsetstrokecolor{currentstroke}%
\pgfsetstrokeopacity{0.800000}%
\pgfsetdash{}{0pt}%
\pgfpathmoveto{\pgfqpoint{12.233224in}{5.446777in}}%
\pgfpathcurveto{\pgfqpoint{12.244274in}{5.446777in}}{\pgfqpoint{12.254873in}{5.451168in}}{\pgfqpoint{12.262687in}{5.458981in}}%
\pgfpathcurveto{\pgfqpoint{12.270500in}{5.466795in}}{\pgfqpoint{12.274891in}{5.477394in}}{\pgfqpoint{12.274891in}{5.488444in}}%
\pgfpathcurveto{\pgfqpoint{12.274891in}{5.499494in}}{\pgfqpoint{12.270500in}{5.510093in}}{\pgfqpoint{12.262687in}{5.517907in}}%
\pgfpathcurveto{\pgfqpoint{12.254873in}{5.525720in}}{\pgfqpoint{12.244274in}{5.530111in}}{\pgfqpoint{12.233224in}{5.530111in}}%
\pgfpathcurveto{\pgfqpoint{12.222174in}{5.530111in}}{\pgfqpoint{12.211575in}{5.525720in}}{\pgfqpoint{12.203761in}{5.517907in}}%
\pgfpathcurveto{\pgfqpoint{12.195948in}{5.510093in}}{\pgfqpoint{12.191557in}{5.499494in}}{\pgfqpoint{12.191557in}{5.488444in}}%
\pgfpathcurveto{\pgfqpoint{12.191557in}{5.477394in}}{\pgfqpoint{12.195948in}{5.466795in}}{\pgfqpoint{12.203761in}{5.458981in}}%
\pgfpathcurveto{\pgfqpoint{12.211575in}{5.451168in}}{\pgfqpoint{12.222174in}{5.446777in}}{\pgfqpoint{12.233224in}{5.446777in}}%
\pgfpathlineto{\pgfqpoint{12.233224in}{5.446777in}}%
\pgfpathclose%
\pgfusepath{stroke}%
\end{pgfscope}%
\begin{pgfscope}%
\pgfpathrectangle{\pgfqpoint{7.512535in}{0.437222in}}{\pgfqpoint{6.275590in}{5.159444in}}%
\pgfusepath{clip}%
\pgfsetbuttcap%
\pgfsetroundjoin%
\pgfsetlinewidth{1.003750pt}%
\definecolor{currentstroke}{rgb}{0.827451,0.827451,0.827451}%
\pgfsetstrokecolor{currentstroke}%
\pgfsetstrokeopacity{0.800000}%
\pgfsetdash{}{0pt}%
\pgfpathmoveto{\pgfqpoint{11.806382in}{5.471994in}}%
\pgfpathcurveto{\pgfqpoint{11.817432in}{5.471994in}}{\pgfqpoint{11.828031in}{5.476384in}}{\pgfqpoint{11.835845in}{5.484198in}}%
\pgfpathcurveto{\pgfqpoint{11.843658in}{5.492011in}}{\pgfqpoint{11.848049in}{5.502610in}}{\pgfqpoint{11.848049in}{5.513660in}}%
\pgfpathcurveto{\pgfqpoint{11.848049in}{5.524711in}}{\pgfqpoint{11.843658in}{5.535310in}}{\pgfqpoint{11.835845in}{5.543123in}}%
\pgfpathcurveto{\pgfqpoint{11.828031in}{5.550937in}}{\pgfqpoint{11.817432in}{5.555327in}}{\pgfqpoint{11.806382in}{5.555327in}}%
\pgfpathcurveto{\pgfqpoint{11.795332in}{5.555327in}}{\pgfqpoint{11.784733in}{5.550937in}}{\pgfqpoint{11.776919in}{5.543123in}}%
\pgfpathcurveto{\pgfqpoint{11.769106in}{5.535310in}}{\pgfqpoint{11.764715in}{5.524711in}}{\pgfqpoint{11.764715in}{5.513660in}}%
\pgfpathcurveto{\pgfqpoint{11.764715in}{5.502610in}}{\pgfqpoint{11.769106in}{5.492011in}}{\pgfqpoint{11.776919in}{5.484198in}}%
\pgfpathcurveto{\pgfqpoint{11.784733in}{5.476384in}}{\pgfqpoint{11.795332in}{5.471994in}}{\pgfqpoint{11.806382in}{5.471994in}}%
\pgfpathlineto{\pgfqpoint{11.806382in}{5.471994in}}%
\pgfpathclose%
\pgfusepath{stroke}%
\end{pgfscope}%
\begin{pgfscope}%
\pgfpathrectangle{\pgfqpoint{7.512535in}{0.437222in}}{\pgfqpoint{6.275590in}{5.159444in}}%
\pgfusepath{clip}%
\pgfsetbuttcap%
\pgfsetroundjoin%
\pgfsetlinewidth{1.003750pt}%
\definecolor{currentstroke}{rgb}{0.827451,0.827451,0.827451}%
\pgfsetstrokecolor{currentstroke}%
\pgfsetstrokeopacity{0.800000}%
\pgfsetdash{}{0pt}%
\pgfpathmoveto{\pgfqpoint{8.716039in}{3.385314in}}%
\pgfpathcurveto{\pgfqpoint{8.727089in}{3.385314in}}{\pgfqpoint{8.737688in}{3.389704in}}{\pgfqpoint{8.745502in}{3.397518in}}%
\pgfpathcurveto{\pgfqpoint{8.753315in}{3.405332in}}{\pgfqpoint{8.757706in}{3.415931in}}{\pgfqpoint{8.757706in}{3.426981in}}%
\pgfpathcurveto{\pgfqpoint{8.757706in}{3.438031in}}{\pgfqpoint{8.753315in}{3.448630in}}{\pgfqpoint{8.745502in}{3.456444in}}%
\pgfpathcurveto{\pgfqpoint{8.737688in}{3.464257in}}{\pgfqpoint{8.727089in}{3.468647in}}{\pgfqpoint{8.716039in}{3.468647in}}%
\pgfpathcurveto{\pgfqpoint{8.704989in}{3.468647in}}{\pgfqpoint{8.694390in}{3.464257in}}{\pgfqpoint{8.686576in}{3.456444in}}%
\pgfpathcurveto{\pgfqpoint{8.678763in}{3.448630in}}{\pgfqpoint{8.674372in}{3.438031in}}{\pgfqpoint{8.674372in}{3.426981in}}%
\pgfpathcurveto{\pgfqpoint{8.674372in}{3.415931in}}{\pgfqpoint{8.678763in}{3.405332in}}{\pgfqpoint{8.686576in}{3.397518in}}%
\pgfpathcurveto{\pgfqpoint{8.694390in}{3.389704in}}{\pgfqpoint{8.704989in}{3.385314in}}{\pgfqpoint{8.716039in}{3.385314in}}%
\pgfpathlineto{\pgfqpoint{8.716039in}{3.385314in}}%
\pgfpathclose%
\pgfusepath{stroke}%
\end{pgfscope}%
\begin{pgfscope}%
\pgfpathrectangle{\pgfqpoint{7.512535in}{0.437222in}}{\pgfqpoint{6.275590in}{5.159444in}}%
\pgfusepath{clip}%
\pgfsetbuttcap%
\pgfsetroundjoin%
\pgfsetlinewidth{1.003750pt}%
\definecolor{currentstroke}{rgb}{0.827451,0.827451,0.827451}%
\pgfsetstrokecolor{currentstroke}%
\pgfsetstrokeopacity{0.800000}%
\pgfsetdash{}{0pt}%
\pgfpathmoveto{\pgfqpoint{7.653772in}{0.965972in}}%
\pgfpathcurveto{\pgfqpoint{7.664822in}{0.965972in}}{\pgfqpoint{7.675421in}{0.970363in}}{\pgfqpoint{7.683235in}{0.978176in}}%
\pgfpathcurveto{\pgfqpoint{7.691048in}{0.985990in}}{\pgfqpoint{7.695439in}{0.996589in}}{\pgfqpoint{7.695439in}{1.007639in}}%
\pgfpathcurveto{\pgfqpoint{7.695439in}{1.018689in}}{\pgfqpoint{7.691048in}{1.029288in}}{\pgfqpoint{7.683235in}{1.037102in}}%
\pgfpathcurveto{\pgfqpoint{7.675421in}{1.044915in}}{\pgfqpoint{7.664822in}{1.049306in}}{\pgfqpoint{7.653772in}{1.049306in}}%
\pgfpathcurveto{\pgfqpoint{7.642722in}{1.049306in}}{\pgfqpoint{7.632123in}{1.044915in}}{\pgfqpoint{7.624309in}{1.037102in}}%
\pgfpathcurveto{\pgfqpoint{7.616496in}{1.029288in}}{\pgfqpoint{7.612105in}{1.018689in}}{\pgfqpoint{7.612105in}{1.007639in}}%
\pgfpathcurveto{\pgfqpoint{7.612105in}{0.996589in}}{\pgfqpoint{7.616496in}{0.985990in}}{\pgfqpoint{7.624309in}{0.978176in}}%
\pgfpathcurveto{\pgfqpoint{7.632123in}{0.970363in}}{\pgfqpoint{7.642722in}{0.965972in}}{\pgfqpoint{7.653772in}{0.965972in}}%
\pgfpathlineto{\pgfqpoint{7.653772in}{0.965972in}}%
\pgfpathclose%
\pgfusepath{stroke}%
\end{pgfscope}%
\begin{pgfscope}%
\pgfpathrectangle{\pgfqpoint{7.512535in}{0.437222in}}{\pgfqpoint{6.275590in}{5.159444in}}%
\pgfusepath{clip}%
\pgfsetbuttcap%
\pgfsetroundjoin%
\pgfsetlinewidth{1.003750pt}%
\definecolor{currentstroke}{rgb}{0.827451,0.827451,0.827451}%
\pgfsetstrokecolor{currentstroke}%
\pgfsetstrokeopacity{0.800000}%
\pgfsetdash{}{0pt}%
\pgfpathmoveto{\pgfqpoint{12.495601in}{5.523529in}}%
\pgfpathcurveto{\pgfqpoint{12.506651in}{5.523529in}}{\pgfqpoint{12.517250in}{5.527919in}}{\pgfqpoint{12.525063in}{5.535733in}}%
\pgfpathcurveto{\pgfqpoint{12.532877in}{5.543547in}}{\pgfqpoint{12.537267in}{5.554146in}}{\pgfqpoint{12.537267in}{5.565196in}}%
\pgfpathcurveto{\pgfqpoint{12.537267in}{5.576246in}}{\pgfqpoint{12.532877in}{5.586845in}}{\pgfqpoint{12.525063in}{5.594658in}}%
\pgfpathcurveto{\pgfqpoint{12.517250in}{5.602472in}}{\pgfqpoint{12.506651in}{5.606862in}}{\pgfqpoint{12.495601in}{5.606862in}}%
\pgfpathcurveto{\pgfqpoint{12.484550in}{5.606862in}}{\pgfqpoint{12.473951in}{5.602472in}}{\pgfqpoint{12.466138in}{5.594658in}}%
\pgfpathcurveto{\pgfqpoint{12.458324in}{5.586845in}}{\pgfqpoint{12.453934in}{5.576246in}}{\pgfqpoint{12.453934in}{5.565196in}}%
\pgfpathcurveto{\pgfqpoint{12.453934in}{5.554146in}}{\pgfqpoint{12.458324in}{5.543547in}}{\pgfqpoint{12.466138in}{5.535733in}}%
\pgfpathcurveto{\pgfqpoint{12.473951in}{5.527919in}}{\pgfqpoint{12.484550in}{5.523529in}}{\pgfqpoint{12.495601in}{5.523529in}}%
\pgfpathlineto{\pgfqpoint{12.495601in}{5.523529in}}%
\pgfpathclose%
\pgfusepath{stroke}%
\end{pgfscope}%
\begin{pgfscope}%
\pgfpathrectangle{\pgfqpoint{7.512535in}{0.437222in}}{\pgfqpoint{6.275590in}{5.159444in}}%
\pgfusepath{clip}%
\pgfsetbuttcap%
\pgfsetroundjoin%
\pgfsetlinewidth{1.003750pt}%
\definecolor{currentstroke}{rgb}{0.827451,0.827451,0.827451}%
\pgfsetstrokecolor{currentstroke}%
\pgfsetstrokeopacity{0.800000}%
\pgfsetdash{}{0pt}%
\pgfpathmoveto{\pgfqpoint{9.848678in}{4.313606in}}%
\pgfpathcurveto{\pgfqpoint{9.859728in}{4.313606in}}{\pgfqpoint{9.870327in}{4.317996in}}{\pgfqpoint{9.878141in}{4.325810in}}%
\pgfpathcurveto{\pgfqpoint{9.885955in}{4.333623in}}{\pgfqpoint{9.890345in}{4.344222in}}{\pgfqpoint{9.890345in}{4.355272in}}%
\pgfpathcurveto{\pgfqpoint{9.890345in}{4.366323in}}{\pgfqpoint{9.885955in}{4.376922in}}{\pgfqpoint{9.878141in}{4.384735in}}%
\pgfpathcurveto{\pgfqpoint{9.870327in}{4.392549in}}{\pgfqpoint{9.859728in}{4.396939in}}{\pgfqpoint{9.848678in}{4.396939in}}%
\pgfpathcurveto{\pgfqpoint{9.837628in}{4.396939in}}{\pgfqpoint{9.827029in}{4.392549in}}{\pgfqpoint{9.819216in}{4.384735in}}%
\pgfpathcurveto{\pgfqpoint{9.811402in}{4.376922in}}{\pgfqpoint{9.807012in}{4.366323in}}{\pgfqpoint{9.807012in}{4.355272in}}%
\pgfpathcurveto{\pgfqpoint{9.807012in}{4.344222in}}{\pgfqpoint{9.811402in}{4.333623in}}{\pgfqpoint{9.819216in}{4.325810in}}%
\pgfpathcurveto{\pgfqpoint{9.827029in}{4.317996in}}{\pgfqpoint{9.837628in}{4.313606in}}{\pgfqpoint{9.848678in}{4.313606in}}%
\pgfpathlineto{\pgfqpoint{9.848678in}{4.313606in}}%
\pgfpathclose%
\pgfusepath{stroke}%
\end{pgfscope}%
\begin{pgfscope}%
\pgfpathrectangle{\pgfqpoint{7.512535in}{0.437222in}}{\pgfqpoint{6.275590in}{5.159444in}}%
\pgfusepath{clip}%
\pgfsetbuttcap%
\pgfsetroundjoin%
\pgfsetlinewidth{1.003750pt}%
\definecolor{currentstroke}{rgb}{0.827451,0.827451,0.827451}%
\pgfsetstrokecolor{currentstroke}%
\pgfsetstrokeopacity{0.800000}%
\pgfsetdash{}{0pt}%
\pgfpathmoveto{\pgfqpoint{7.983939in}{1.000624in}}%
\pgfpathcurveto{\pgfqpoint{7.994989in}{1.000624in}}{\pgfqpoint{8.005588in}{1.005015in}}{\pgfqpoint{8.013401in}{1.012828in}}%
\pgfpathcurveto{\pgfqpoint{8.021215in}{1.020642in}}{\pgfqpoint{8.025605in}{1.031241in}}{\pgfqpoint{8.025605in}{1.042291in}}%
\pgfpathcurveto{\pgfqpoint{8.025605in}{1.053341in}}{\pgfqpoint{8.021215in}{1.063940in}}{\pgfqpoint{8.013401in}{1.071754in}}%
\pgfpathcurveto{\pgfqpoint{8.005588in}{1.079568in}}{\pgfqpoint{7.994989in}{1.083958in}}{\pgfqpoint{7.983939in}{1.083958in}}%
\pgfpathcurveto{\pgfqpoint{7.972888in}{1.083958in}}{\pgfqpoint{7.962289in}{1.079568in}}{\pgfqpoint{7.954476in}{1.071754in}}%
\pgfpathcurveto{\pgfqpoint{7.946662in}{1.063940in}}{\pgfqpoint{7.942272in}{1.053341in}}{\pgfqpoint{7.942272in}{1.042291in}}%
\pgfpathcurveto{\pgfqpoint{7.942272in}{1.031241in}}{\pgfqpoint{7.946662in}{1.020642in}}{\pgfqpoint{7.954476in}{1.012828in}}%
\pgfpathcurveto{\pgfqpoint{7.962289in}{1.005015in}}{\pgfqpoint{7.972888in}{1.000624in}}{\pgfqpoint{7.983939in}{1.000624in}}%
\pgfpathlineto{\pgfqpoint{7.983939in}{1.000624in}}%
\pgfpathclose%
\pgfusepath{stroke}%
\end{pgfscope}%
\begin{pgfscope}%
\pgfpathrectangle{\pgfqpoint{7.512535in}{0.437222in}}{\pgfqpoint{6.275590in}{5.159444in}}%
\pgfusepath{clip}%
\pgfsetbuttcap%
\pgfsetroundjoin%
\pgfsetlinewidth{1.003750pt}%
\definecolor{currentstroke}{rgb}{0.827451,0.827451,0.827451}%
\pgfsetstrokecolor{currentstroke}%
\pgfsetstrokeopacity{0.800000}%
\pgfsetdash{}{0pt}%
\pgfpathmoveto{\pgfqpoint{8.907940in}{2.091479in}}%
\pgfpathcurveto{\pgfqpoint{8.918990in}{2.091479in}}{\pgfqpoint{8.929589in}{2.095869in}}{\pgfqpoint{8.937403in}{2.103683in}}%
\pgfpathcurveto{\pgfqpoint{8.945217in}{2.111496in}}{\pgfqpoint{8.949607in}{2.122095in}}{\pgfqpoint{8.949607in}{2.133145in}}%
\pgfpathcurveto{\pgfqpoint{8.949607in}{2.144195in}}{\pgfqpoint{8.945217in}{2.154795in}}{\pgfqpoint{8.937403in}{2.162608in}}%
\pgfpathcurveto{\pgfqpoint{8.929589in}{2.170422in}}{\pgfqpoint{8.918990in}{2.174812in}}{\pgfqpoint{8.907940in}{2.174812in}}%
\pgfpathcurveto{\pgfqpoint{8.896890in}{2.174812in}}{\pgfqpoint{8.886291in}{2.170422in}}{\pgfqpoint{8.878477in}{2.162608in}}%
\pgfpathcurveto{\pgfqpoint{8.870664in}{2.154795in}}{\pgfqpoint{8.866274in}{2.144195in}}{\pgfqpoint{8.866274in}{2.133145in}}%
\pgfpathcurveto{\pgfqpoint{8.866274in}{2.122095in}}{\pgfqpoint{8.870664in}{2.111496in}}{\pgfqpoint{8.878477in}{2.103683in}}%
\pgfpathcurveto{\pgfqpoint{8.886291in}{2.095869in}}{\pgfqpoint{8.896890in}{2.091479in}}{\pgfqpoint{8.907940in}{2.091479in}}%
\pgfpathlineto{\pgfqpoint{8.907940in}{2.091479in}}%
\pgfpathclose%
\pgfusepath{stroke}%
\end{pgfscope}%
\begin{pgfscope}%
\pgfpathrectangle{\pgfqpoint{7.512535in}{0.437222in}}{\pgfqpoint{6.275590in}{5.159444in}}%
\pgfusepath{clip}%
\pgfsetbuttcap%
\pgfsetroundjoin%
\pgfsetlinewidth{1.003750pt}%
\definecolor{currentstroke}{rgb}{0.827451,0.827451,0.827451}%
\pgfsetstrokecolor{currentstroke}%
\pgfsetstrokeopacity{0.800000}%
\pgfsetdash{}{0pt}%
\pgfpathmoveto{\pgfqpoint{9.807010in}{4.531380in}}%
\pgfpathcurveto{\pgfqpoint{9.818060in}{4.531380in}}{\pgfqpoint{9.828659in}{4.535770in}}{\pgfqpoint{9.836473in}{4.543584in}}%
\pgfpathcurveto{\pgfqpoint{9.844287in}{4.551397in}}{\pgfqpoint{9.848677in}{4.561996in}}{\pgfqpoint{9.848677in}{4.573046in}}%
\pgfpathcurveto{\pgfqpoint{9.848677in}{4.584096in}}{\pgfqpoint{9.844287in}{4.594695in}}{\pgfqpoint{9.836473in}{4.602509in}}%
\pgfpathcurveto{\pgfqpoint{9.828659in}{4.610323in}}{\pgfqpoint{9.818060in}{4.614713in}}{\pgfqpoint{9.807010in}{4.614713in}}%
\pgfpathcurveto{\pgfqpoint{9.795960in}{4.614713in}}{\pgfqpoint{9.785361in}{4.610323in}}{\pgfqpoint{9.777548in}{4.602509in}}%
\pgfpathcurveto{\pgfqpoint{9.769734in}{4.594695in}}{\pgfqpoint{9.765344in}{4.584096in}}{\pgfqpoint{9.765344in}{4.573046in}}%
\pgfpathcurveto{\pgfqpoint{9.765344in}{4.561996in}}{\pgfqpoint{9.769734in}{4.551397in}}{\pgfqpoint{9.777548in}{4.543584in}}%
\pgfpathcurveto{\pgfqpoint{9.785361in}{4.535770in}}{\pgfqpoint{9.795960in}{4.531380in}}{\pgfqpoint{9.807010in}{4.531380in}}%
\pgfpathlineto{\pgfqpoint{9.807010in}{4.531380in}}%
\pgfpathclose%
\pgfusepath{stroke}%
\end{pgfscope}%
\begin{pgfscope}%
\pgfpathrectangle{\pgfqpoint{7.512535in}{0.437222in}}{\pgfqpoint{6.275590in}{5.159444in}}%
\pgfusepath{clip}%
\pgfsetbuttcap%
\pgfsetroundjoin%
\pgfsetlinewidth{1.003750pt}%
\definecolor{currentstroke}{rgb}{0.827451,0.827451,0.827451}%
\pgfsetstrokecolor{currentstroke}%
\pgfsetstrokeopacity{0.800000}%
\pgfsetdash{}{0pt}%
\pgfpathmoveto{\pgfqpoint{11.517254in}{5.159213in}}%
\pgfpathcurveto{\pgfqpoint{11.528304in}{5.159213in}}{\pgfqpoint{11.538903in}{5.163603in}}{\pgfqpoint{11.546717in}{5.171417in}}%
\pgfpathcurveto{\pgfqpoint{11.554531in}{5.179230in}}{\pgfqpoint{11.558921in}{5.189829in}}{\pgfqpoint{11.558921in}{5.200879in}}%
\pgfpathcurveto{\pgfqpoint{11.558921in}{5.211930in}}{\pgfqpoint{11.554531in}{5.222529in}}{\pgfqpoint{11.546717in}{5.230342in}}%
\pgfpathcurveto{\pgfqpoint{11.538903in}{5.238156in}}{\pgfqpoint{11.528304in}{5.242546in}}{\pgfqpoint{11.517254in}{5.242546in}}%
\pgfpathcurveto{\pgfqpoint{11.506204in}{5.242546in}}{\pgfqpoint{11.495605in}{5.238156in}}{\pgfqpoint{11.487791in}{5.230342in}}%
\pgfpathcurveto{\pgfqpoint{11.479978in}{5.222529in}}{\pgfqpoint{11.475588in}{5.211930in}}{\pgfqpoint{11.475588in}{5.200879in}}%
\pgfpathcurveto{\pgfqpoint{11.475588in}{5.189829in}}{\pgfqpoint{11.479978in}{5.179230in}}{\pgfqpoint{11.487791in}{5.171417in}}%
\pgfpathcurveto{\pgfqpoint{11.495605in}{5.163603in}}{\pgfqpoint{11.506204in}{5.159213in}}{\pgfqpoint{11.517254in}{5.159213in}}%
\pgfpathlineto{\pgfqpoint{11.517254in}{5.159213in}}%
\pgfpathclose%
\pgfusepath{stroke}%
\end{pgfscope}%
\begin{pgfscope}%
\pgfpathrectangle{\pgfqpoint{7.512535in}{0.437222in}}{\pgfqpoint{6.275590in}{5.159444in}}%
\pgfusepath{clip}%
\pgfsetbuttcap%
\pgfsetroundjoin%
\pgfsetlinewidth{1.003750pt}%
\definecolor{currentstroke}{rgb}{0.827451,0.827451,0.827451}%
\pgfsetstrokecolor{currentstroke}%
\pgfsetstrokeopacity{0.800000}%
\pgfsetdash{}{0pt}%
\pgfpathmoveto{\pgfqpoint{11.275914in}{5.187104in}}%
\pgfpathcurveto{\pgfqpoint{11.286964in}{5.187104in}}{\pgfqpoint{11.297563in}{5.191494in}}{\pgfqpoint{11.305377in}{5.199308in}}%
\pgfpathcurveto{\pgfqpoint{11.313190in}{5.207121in}}{\pgfqpoint{11.317581in}{5.217720in}}{\pgfqpoint{11.317581in}{5.228771in}}%
\pgfpathcurveto{\pgfqpoint{11.317581in}{5.239821in}}{\pgfqpoint{11.313190in}{5.250420in}}{\pgfqpoint{11.305377in}{5.258233in}}%
\pgfpathcurveto{\pgfqpoint{11.297563in}{5.266047in}}{\pgfqpoint{11.286964in}{5.270437in}}{\pgfqpoint{11.275914in}{5.270437in}}%
\pgfpathcurveto{\pgfqpoint{11.264864in}{5.270437in}}{\pgfqpoint{11.254265in}{5.266047in}}{\pgfqpoint{11.246451in}{5.258233in}}%
\pgfpathcurveto{\pgfqpoint{11.238637in}{5.250420in}}{\pgfqpoint{11.234247in}{5.239821in}}{\pgfqpoint{11.234247in}{5.228771in}}%
\pgfpathcurveto{\pgfqpoint{11.234247in}{5.217720in}}{\pgfqpoint{11.238637in}{5.207121in}}{\pgfqpoint{11.246451in}{5.199308in}}%
\pgfpathcurveto{\pgfqpoint{11.254265in}{5.191494in}}{\pgfqpoint{11.264864in}{5.187104in}}{\pgfqpoint{11.275914in}{5.187104in}}%
\pgfpathlineto{\pgfqpoint{11.275914in}{5.187104in}}%
\pgfpathclose%
\pgfusepath{stroke}%
\end{pgfscope}%
\begin{pgfscope}%
\pgfpathrectangle{\pgfqpoint{7.512535in}{0.437222in}}{\pgfqpoint{6.275590in}{5.159444in}}%
\pgfusepath{clip}%
\pgfsetbuttcap%
\pgfsetroundjoin%
\pgfsetlinewidth{1.003750pt}%
\definecolor{currentstroke}{rgb}{0.827451,0.827451,0.827451}%
\pgfsetstrokecolor{currentstroke}%
\pgfsetstrokeopacity{0.800000}%
\pgfsetdash{}{0pt}%
\pgfpathmoveto{\pgfqpoint{11.405151in}{5.189667in}}%
\pgfpathcurveto{\pgfqpoint{11.416201in}{5.189667in}}{\pgfqpoint{11.426800in}{5.194058in}}{\pgfqpoint{11.434613in}{5.201871in}}%
\pgfpathcurveto{\pgfqpoint{11.442427in}{5.209685in}}{\pgfqpoint{11.446817in}{5.220284in}}{\pgfqpoint{11.446817in}{5.231334in}}%
\pgfpathcurveto{\pgfqpoint{11.446817in}{5.242384in}}{\pgfqpoint{11.442427in}{5.252983in}}{\pgfqpoint{11.434613in}{5.260797in}}%
\pgfpathcurveto{\pgfqpoint{11.426800in}{5.268610in}}{\pgfqpoint{11.416201in}{5.273001in}}{\pgfqpoint{11.405151in}{5.273001in}}%
\pgfpathcurveto{\pgfqpoint{11.394100in}{5.273001in}}{\pgfqpoint{11.383501in}{5.268610in}}{\pgfqpoint{11.375688in}{5.260797in}}%
\pgfpathcurveto{\pgfqpoint{11.367874in}{5.252983in}}{\pgfqpoint{11.363484in}{5.242384in}}{\pgfqpoint{11.363484in}{5.231334in}}%
\pgfpathcurveto{\pgfqpoint{11.363484in}{5.220284in}}{\pgfqpoint{11.367874in}{5.209685in}}{\pgfqpoint{11.375688in}{5.201871in}}%
\pgfpathcurveto{\pgfqpoint{11.383501in}{5.194058in}}{\pgfqpoint{11.394100in}{5.189667in}}{\pgfqpoint{11.405151in}{5.189667in}}%
\pgfpathlineto{\pgfqpoint{11.405151in}{5.189667in}}%
\pgfpathclose%
\pgfusepath{stroke}%
\end{pgfscope}%
\begin{pgfscope}%
\pgfpathrectangle{\pgfqpoint{7.512535in}{0.437222in}}{\pgfqpoint{6.275590in}{5.159444in}}%
\pgfusepath{clip}%
\pgfsetbuttcap%
\pgfsetroundjoin%
\pgfsetlinewidth{1.003750pt}%
\definecolor{currentstroke}{rgb}{0.827451,0.827451,0.827451}%
\pgfsetstrokecolor{currentstroke}%
\pgfsetstrokeopacity{0.800000}%
\pgfsetdash{}{0pt}%
\pgfpathmoveto{\pgfqpoint{9.319291in}{1.927959in}}%
\pgfpathcurveto{\pgfqpoint{9.330341in}{1.927959in}}{\pgfqpoint{9.340940in}{1.932349in}}{\pgfqpoint{9.348754in}{1.940163in}}%
\pgfpathcurveto{\pgfqpoint{9.356567in}{1.947976in}}{\pgfqpoint{9.360957in}{1.958575in}}{\pgfqpoint{9.360957in}{1.969626in}}%
\pgfpathcurveto{\pgfqpoint{9.360957in}{1.980676in}}{\pgfqpoint{9.356567in}{1.991275in}}{\pgfqpoint{9.348754in}{1.999088in}}%
\pgfpathcurveto{\pgfqpoint{9.340940in}{2.006902in}}{\pgfqpoint{9.330341in}{2.011292in}}{\pgfqpoint{9.319291in}{2.011292in}}%
\pgfpathcurveto{\pgfqpoint{9.308241in}{2.011292in}}{\pgfqpoint{9.297642in}{2.006902in}}{\pgfqpoint{9.289828in}{1.999088in}}%
\pgfpathcurveto{\pgfqpoint{9.282014in}{1.991275in}}{\pgfqpoint{9.277624in}{1.980676in}}{\pgfqpoint{9.277624in}{1.969626in}}%
\pgfpathcurveto{\pgfqpoint{9.277624in}{1.958575in}}{\pgfqpoint{9.282014in}{1.947976in}}{\pgfqpoint{9.289828in}{1.940163in}}%
\pgfpathcurveto{\pgfqpoint{9.297642in}{1.932349in}}{\pgfqpoint{9.308241in}{1.927959in}}{\pgfqpoint{9.319291in}{1.927959in}}%
\pgfpathlineto{\pgfqpoint{9.319291in}{1.927959in}}%
\pgfpathclose%
\pgfusepath{stroke}%
\end{pgfscope}%
\begin{pgfscope}%
\pgfpathrectangle{\pgfqpoint{7.512535in}{0.437222in}}{\pgfqpoint{6.275590in}{5.159444in}}%
\pgfusepath{clip}%
\pgfsetbuttcap%
\pgfsetroundjoin%
\pgfsetlinewidth{1.003750pt}%
\definecolor{currentstroke}{rgb}{0.827451,0.827451,0.827451}%
\pgfsetstrokecolor{currentstroke}%
\pgfsetstrokeopacity{0.800000}%
\pgfsetdash{}{0pt}%
\pgfpathmoveto{\pgfqpoint{8.716039in}{3.198227in}}%
\pgfpathcurveto{\pgfqpoint{8.727089in}{3.198227in}}{\pgfqpoint{8.737688in}{3.202618in}}{\pgfqpoint{8.745502in}{3.210431in}}%
\pgfpathcurveto{\pgfqpoint{8.753315in}{3.218245in}}{\pgfqpoint{8.757706in}{3.228844in}}{\pgfqpoint{8.757706in}{3.239894in}}%
\pgfpathcurveto{\pgfqpoint{8.757706in}{3.250944in}}{\pgfqpoint{8.753315in}{3.261543in}}{\pgfqpoint{8.745502in}{3.269357in}}%
\pgfpathcurveto{\pgfqpoint{8.737688in}{3.277170in}}{\pgfqpoint{8.727089in}{3.281561in}}{\pgfqpoint{8.716039in}{3.281561in}}%
\pgfpathcurveto{\pgfqpoint{8.704989in}{3.281561in}}{\pgfqpoint{8.694390in}{3.277170in}}{\pgfqpoint{8.686576in}{3.269357in}}%
\pgfpathcurveto{\pgfqpoint{8.678763in}{3.261543in}}{\pgfqpoint{8.674372in}{3.250944in}}{\pgfqpoint{8.674372in}{3.239894in}}%
\pgfpathcurveto{\pgfqpoint{8.674372in}{3.228844in}}{\pgfqpoint{8.678763in}{3.218245in}}{\pgfqpoint{8.686576in}{3.210431in}}%
\pgfpathcurveto{\pgfqpoint{8.694390in}{3.202618in}}{\pgfqpoint{8.704989in}{3.198227in}}{\pgfqpoint{8.716039in}{3.198227in}}%
\pgfpathlineto{\pgfqpoint{8.716039in}{3.198227in}}%
\pgfpathclose%
\pgfusepath{stroke}%
\end{pgfscope}%
\begin{pgfscope}%
\pgfpathrectangle{\pgfqpoint{7.512535in}{0.437222in}}{\pgfqpoint{6.275590in}{5.159444in}}%
\pgfusepath{clip}%
\pgfsetbuttcap%
\pgfsetroundjoin%
\pgfsetlinewidth{1.003750pt}%
\definecolor{currentstroke}{rgb}{0.827451,0.827451,0.827451}%
\pgfsetstrokecolor{currentstroke}%
\pgfsetstrokeopacity{0.800000}%
\pgfsetdash{}{0pt}%
\pgfpathmoveto{\pgfqpoint{8.978070in}{1.743970in}}%
\pgfpathcurveto{\pgfqpoint{8.989120in}{1.743970in}}{\pgfqpoint{8.999719in}{1.748360in}}{\pgfqpoint{9.007533in}{1.756174in}}%
\pgfpathcurveto{\pgfqpoint{9.015346in}{1.763988in}}{\pgfqpoint{9.019736in}{1.774587in}}{\pgfqpoint{9.019736in}{1.785637in}}%
\pgfpathcurveto{\pgfqpoint{9.019736in}{1.796687in}}{\pgfqpoint{9.015346in}{1.807286in}}{\pgfqpoint{9.007533in}{1.815100in}}%
\pgfpathcurveto{\pgfqpoint{8.999719in}{1.822913in}}{\pgfqpoint{8.989120in}{1.827303in}}{\pgfqpoint{8.978070in}{1.827303in}}%
\pgfpathcurveto{\pgfqpoint{8.967020in}{1.827303in}}{\pgfqpoint{8.956421in}{1.822913in}}{\pgfqpoint{8.948607in}{1.815100in}}%
\pgfpathcurveto{\pgfqpoint{8.940793in}{1.807286in}}{\pgfqpoint{8.936403in}{1.796687in}}{\pgfqpoint{8.936403in}{1.785637in}}%
\pgfpathcurveto{\pgfqpoint{8.936403in}{1.774587in}}{\pgfqpoint{8.940793in}{1.763988in}}{\pgfqpoint{8.948607in}{1.756174in}}%
\pgfpathcurveto{\pgfqpoint{8.956421in}{1.748360in}}{\pgfqpoint{8.967020in}{1.743970in}}{\pgfqpoint{8.978070in}{1.743970in}}%
\pgfpathlineto{\pgfqpoint{8.978070in}{1.743970in}}%
\pgfpathclose%
\pgfusepath{stroke}%
\end{pgfscope}%
\begin{pgfscope}%
\pgfpathrectangle{\pgfqpoint{7.512535in}{0.437222in}}{\pgfqpoint{6.275590in}{5.159444in}}%
\pgfusepath{clip}%
\pgfsetbuttcap%
\pgfsetroundjoin%
\pgfsetlinewidth{1.003750pt}%
\definecolor{currentstroke}{rgb}{0.827451,0.827451,0.827451}%
\pgfsetstrokecolor{currentstroke}%
\pgfsetstrokeopacity{0.800000}%
\pgfsetdash{}{0pt}%
\pgfpathmoveto{\pgfqpoint{7.679894in}{0.493855in}}%
\pgfpathcurveto{\pgfqpoint{7.690944in}{0.493855in}}{\pgfqpoint{7.701543in}{0.498245in}}{\pgfqpoint{7.709357in}{0.506059in}}%
\pgfpathcurveto{\pgfqpoint{7.717170in}{0.513872in}}{\pgfqpoint{7.721560in}{0.524471in}}{\pgfqpoint{7.721560in}{0.535522in}}%
\pgfpathcurveto{\pgfqpoint{7.721560in}{0.546572in}}{\pgfqpoint{7.717170in}{0.557171in}}{\pgfqpoint{7.709357in}{0.564984in}}%
\pgfpathcurveto{\pgfqpoint{7.701543in}{0.572798in}}{\pgfqpoint{7.690944in}{0.577188in}}{\pgfqpoint{7.679894in}{0.577188in}}%
\pgfpathcurveto{\pgfqpoint{7.668844in}{0.577188in}}{\pgfqpoint{7.658245in}{0.572798in}}{\pgfqpoint{7.650431in}{0.564984in}}%
\pgfpathcurveto{\pgfqpoint{7.642617in}{0.557171in}}{\pgfqpoint{7.638227in}{0.546572in}}{\pgfqpoint{7.638227in}{0.535522in}}%
\pgfpathcurveto{\pgfqpoint{7.638227in}{0.524471in}}{\pgfqpoint{7.642617in}{0.513872in}}{\pgfqpoint{7.650431in}{0.506059in}}%
\pgfpathcurveto{\pgfqpoint{7.658245in}{0.498245in}}{\pgfqpoint{7.668844in}{0.493855in}}{\pgfqpoint{7.679894in}{0.493855in}}%
\pgfpathlineto{\pgfqpoint{7.679894in}{0.493855in}}%
\pgfpathclose%
\pgfusepath{stroke}%
\end{pgfscope}%
\begin{pgfscope}%
\pgfpathrectangle{\pgfqpoint{7.512535in}{0.437222in}}{\pgfqpoint{6.275590in}{5.159444in}}%
\pgfusepath{clip}%
\pgfsetbuttcap%
\pgfsetroundjoin%
\pgfsetlinewidth{1.003750pt}%
\definecolor{currentstroke}{rgb}{0.827451,0.827451,0.827451}%
\pgfsetstrokecolor{currentstroke}%
\pgfsetstrokeopacity{0.800000}%
\pgfsetdash{}{0pt}%
\pgfpathmoveto{\pgfqpoint{9.027094in}{2.556537in}}%
\pgfpathcurveto{\pgfqpoint{9.038145in}{2.556537in}}{\pgfqpoint{9.048744in}{2.560927in}}{\pgfqpoint{9.056557in}{2.568740in}}%
\pgfpathcurveto{\pgfqpoint{9.064371in}{2.576554in}}{\pgfqpoint{9.068761in}{2.587153in}}{\pgfqpoint{9.068761in}{2.598203in}}%
\pgfpathcurveto{\pgfqpoint{9.068761in}{2.609253in}}{\pgfqpoint{9.064371in}{2.619852in}}{\pgfqpoint{9.056557in}{2.627666in}}%
\pgfpathcurveto{\pgfqpoint{9.048744in}{2.635480in}}{\pgfqpoint{9.038145in}{2.639870in}}{\pgfqpoint{9.027094in}{2.639870in}}%
\pgfpathcurveto{\pgfqpoint{9.016044in}{2.639870in}}{\pgfqpoint{9.005445in}{2.635480in}}{\pgfqpoint{8.997632in}{2.627666in}}%
\pgfpathcurveto{\pgfqpoint{8.989818in}{2.619852in}}{\pgfqpoint{8.985428in}{2.609253in}}{\pgfqpoint{8.985428in}{2.598203in}}%
\pgfpathcurveto{\pgfqpoint{8.985428in}{2.587153in}}{\pgfqpoint{8.989818in}{2.576554in}}{\pgfqpoint{8.997632in}{2.568740in}}%
\pgfpathcurveto{\pgfqpoint{9.005445in}{2.560927in}}{\pgfqpoint{9.016044in}{2.556537in}}{\pgfqpoint{9.027094in}{2.556537in}}%
\pgfpathlineto{\pgfqpoint{9.027094in}{2.556537in}}%
\pgfpathclose%
\pgfusepath{stroke}%
\end{pgfscope}%
\begin{pgfscope}%
\pgfpathrectangle{\pgfqpoint{7.512535in}{0.437222in}}{\pgfqpoint{6.275590in}{5.159444in}}%
\pgfusepath{clip}%
\pgfsetbuttcap%
\pgfsetroundjoin%
\pgfsetlinewidth{1.003750pt}%
\definecolor{currentstroke}{rgb}{0.827451,0.827451,0.827451}%
\pgfsetstrokecolor{currentstroke}%
\pgfsetstrokeopacity{0.800000}%
\pgfsetdash{}{0pt}%
\pgfpathmoveto{\pgfqpoint{12.737963in}{5.481847in}}%
\pgfpathcurveto{\pgfqpoint{12.749014in}{5.481847in}}{\pgfqpoint{12.759613in}{5.486238in}}{\pgfqpoint{12.767426in}{5.494051in}}%
\pgfpathcurveto{\pgfqpoint{12.775240in}{5.501865in}}{\pgfqpoint{12.779630in}{5.512464in}}{\pgfqpoint{12.779630in}{5.523514in}}%
\pgfpathcurveto{\pgfqpoint{12.779630in}{5.534564in}}{\pgfqpoint{12.775240in}{5.545163in}}{\pgfqpoint{12.767426in}{5.552977in}}%
\pgfpathcurveto{\pgfqpoint{12.759613in}{5.560791in}}{\pgfqpoint{12.749014in}{5.565181in}}{\pgfqpoint{12.737963in}{5.565181in}}%
\pgfpathcurveto{\pgfqpoint{12.726913in}{5.565181in}}{\pgfqpoint{12.716314in}{5.560791in}}{\pgfqpoint{12.708501in}{5.552977in}}%
\pgfpathcurveto{\pgfqpoint{12.700687in}{5.545163in}}{\pgfqpoint{12.696297in}{5.534564in}}{\pgfqpoint{12.696297in}{5.523514in}}%
\pgfpathcurveto{\pgfqpoint{12.696297in}{5.512464in}}{\pgfqpoint{12.700687in}{5.501865in}}{\pgfqpoint{12.708501in}{5.494051in}}%
\pgfpathcurveto{\pgfqpoint{12.716314in}{5.486238in}}{\pgfqpoint{12.726913in}{5.481847in}}{\pgfqpoint{12.737963in}{5.481847in}}%
\pgfpathlineto{\pgfqpoint{12.737963in}{5.481847in}}%
\pgfpathclose%
\pgfusepath{stroke}%
\end{pgfscope}%
\begin{pgfscope}%
\pgfpathrectangle{\pgfqpoint{7.512535in}{0.437222in}}{\pgfqpoint{6.275590in}{5.159444in}}%
\pgfusepath{clip}%
\pgfsetbuttcap%
\pgfsetroundjoin%
\pgfsetlinewidth{1.003750pt}%
\definecolor{currentstroke}{rgb}{0.827451,0.827451,0.827451}%
\pgfsetstrokecolor{currentstroke}%
\pgfsetstrokeopacity{0.800000}%
\pgfsetdash{}{0pt}%
\pgfpathmoveto{\pgfqpoint{12.287231in}{5.553091in}}%
\pgfpathcurveto{\pgfqpoint{12.298281in}{5.553091in}}{\pgfqpoint{12.308880in}{5.557481in}}{\pgfqpoint{12.316694in}{5.565295in}}%
\pgfpathcurveto{\pgfqpoint{12.324508in}{5.573108in}}{\pgfqpoint{12.328898in}{5.583707in}}{\pgfqpoint{12.328898in}{5.594757in}}%
\pgfpathcurveto{\pgfqpoint{12.328898in}{5.605808in}}{\pgfqpoint{12.324508in}{5.616407in}}{\pgfqpoint{12.316694in}{5.624220in}}%
\pgfpathcurveto{\pgfqpoint{12.308880in}{5.632034in}}{\pgfqpoint{12.298281in}{5.636424in}}{\pgfqpoint{12.287231in}{5.636424in}}%
\pgfpathcurveto{\pgfqpoint{12.276181in}{5.636424in}}{\pgfqpoint{12.265582in}{5.632034in}}{\pgfqpoint{12.257768in}{5.624220in}}%
\pgfpathcurveto{\pgfqpoint{12.249955in}{5.616407in}}{\pgfqpoint{12.245564in}{5.605808in}}{\pgfqpoint{12.245564in}{5.594757in}}%
\pgfpathcurveto{\pgfqpoint{12.245564in}{5.583707in}}{\pgfqpoint{12.249955in}{5.573108in}}{\pgfqpoint{12.257768in}{5.565295in}}%
\pgfpathcurveto{\pgfqpoint{12.265582in}{5.557481in}}{\pgfqpoint{12.276181in}{5.553091in}}{\pgfqpoint{12.287231in}{5.553091in}}%
\pgfpathlineto{\pgfqpoint{12.287231in}{5.553091in}}%
\pgfpathclose%
\pgfusepath{stroke}%
\end{pgfscope}%
\begin{pgfscope}%
\pgfpathrectangle{\pgfqpoint{7.512535in}{0.437222in}}{\pgfqpoint{6.275590in}{5.159444in}}%
\pgfusepath{clip}%
\pgfsetbuttcap%
\pgfsetroundjoin%
\pgfsetlinewidth{1.003750pt}%
\definecolor{currentstroke}{rgb}{0.827451,0.827451,0.827451}%
\pgfsetstrokecolor{currentstroke}%
\pgfsetstrokeopacity{0.800000}%
\pgfsetdash{}{0pt}%
\pgfpathmoveto{\pgfqpoint{9.816953in}{3.890552in}}%
\pgfpathcurveto{\pgfqpoint{9.828003in}{3.890552in}}{\pgfqpoint{9.838602in}{3.894943in}}{\pgfqpoint{9.846415in}{3.902756in}}%
\pgfpathcurveto{\pgfqpoint{9.854229in}{3.910570in}}{\pgfqpoint{9.858619in}{3.921169in}}{\pgfqpoint{9.858619in}{3.932219in}}%
\pgfpathcurveto{\pgfqpoint{9.858619in}{3.943269in}}{\pgfqpoint{9.854229in}{3.953868in}}{\pgfqpoint{9.846415in}{3.961682in}}%
\pgfpathcurveto{\pgfqpoint{9.838602in}{3.969495in}}{\pgfqpoint{9.828003in}{3.973886in}}{\pgfqpoint{9.816953in}{3.973886in}}%
\pgfpathcurveto{\pgfqpoint{9.805903in}{3.973886in}}{\pgfqpoint{9.795304in}{3.969495in}}{\pgfqpoint{9.787490in}{3.961682in}}%
\pgfpathcurveto{\pgfqpoint{9.779676in}{3.953868in}}{\pgfqpoint{9.775286in}{3.943269in}}{\pgfqpoint{9.775286in}{3.932219in}}%
\pgfpathcurveto{\pgfqpoint{9.775286in}{3.921169in}}{\pgfqpoint{9.779676in}{3.910570in}}{\pgfqpoint{9.787490in}{3.902756in}}%
\pgfpathcurveto{\pgfqpoint{9.795304in}{3.894943in}}{\pgfqpoint{9.805903in}{3.890552in}}{\pgfqpoint{9.816953in}{3.890552in}}%
\pgfpathlineto{\pgfqpoint{9.816953in}{3.890552in}}%
\pgfpathclose%
\pgfusepath{stroke}%
\end{pgfscope}%
\begin{pgfscope}%
\pgfpathrectangle{\pgfqpoint{7.512535in}{0.437222in}}{\pgfqpoint{6.275590in}{5.159444in}}%
\pgfusepath{clip}%
\pgfsetbuttcap%
\pgfsetroundjoin%
\pgfsetlinewidth{1.003750pt}%
\definecolor{currentstroke}{rgb}{0.827451,0.827451,0.827451}%
\pgfsetstrokecolor{currentstroke}%
\pgfsetstrokeopacity{0.800000}%
\pgfsetdash{}{0pt}%
\pgfpathmoveto{\pgfqpoint{12.102762in}{5.362410in}}%
\pgfpathcurveto{\pgfqpoint{12.113812in}{5.362410in}}{\pgfqpoint{12.124411in}{5.366800in}}{\pgfqpoint{12.132225in}{5.374614in}}%
\pgfpathcurveto{\pgfqpoint{12.140039in}{5.382427in}}{\pgfqpoint{12.144429in}{5.393027in}}{\pgfqpoint{12.144429in}{5.404077in}}%
\pgfpathcurveto{\pgfqpoint{12.144429in}{5.415127in}}{\pgfqpoint{12.140039in}{5.425726in}}{\pgfqpoint{12.132225in}{5.433539in}}%
\pgfpathcurveto{\pgfqpoint{12.124411in}{5.441353in}}{\pgfqpoint{12.113812in}{5.445743in}}{\pgfqpoint{12.102762in}{5.445743in}}%
\pgfpathcurveto{\pgfqpoint{12.091712in}{5.445743in}}{\pgfqpoint{12.081113in}{5.441353in}}{\pgfqpoint{12.073299in}{5.433539in}}%
\pgfpathcurveto{\pgfqpoint{12.065486in}{5.425726in}}{\pgfqpoint{12.061096in}{5.415127in}}{\pgfqpoint{12.061096in}{5.404077in}}%
\pgfpathcurveto{\pgfqpoint{12.061096in}{5.393027in}}{\pgfqpoint{12.065486in}{5.382427in}}{\pgfqpoint{12.073299in}{5.374614in}}%
\pgfpathcurveto{\pgfqpoint{12.081113in}{5.366800in}}{\pgfqpoint{12.091712in}{5.362410in}}{\pgfqpoint{12.102762in}{5.362410in}}%
\pgfpathlineto{\pgfqpoint{12.102762in}{5.362410in}}%
\pgfpathclose%
\pgfusepath{stroke}%
\end{pgfscope}%
\begin{pgfscope}%
\pgfpathrectangle{\pgfqpoint{7.512535in}{0.437222in}}{\pgfqpoint{6.275590in}{5.159444in}}%
\pgfusepath{clip}%
\pgfsetbuttcap%
\pgfsetroundjoin%
\pgfsetlinewidth{1.003750pt}%
\definecolor{currentstroke}{rgb}{0.827451,0.827451,0.827451}%
\pgfsetstrokecolor{currentstroke}%
\pgfsetstrokeopacity{0.800000}%
\pgfsetdash{}{0pt}%
\pgfpathmoveto{\pgfqpoint{11.806382in}{5.548832in}}%
\pgfpathcurveto{\pgfqpoint{11.817432in}{5.548832in}}{\pgfqpoint{11.828031in}{5.553222in}}{\pgfqpoint{11.835845in}{5.561036in}}%
\pgfpathcurveto{\pgfqpoint{11.843658in}{5.568849in}}{\pgfqpoint{11.848049in}{5.579448in}}{\pgfqpoint{11.848049in}{5.590499in}}%
\pgfpathcurveto{\pgfqpoint{11.848049in}{5.601549in}}{\pgfqpoint{11.843658in}{5.612148in}}{\pgfqpoint{11.835845in}{5.619961in}}%
\pgfpathcurveto{\pgfqpoint{11.828031in}{5.627775in}}{\pgfqpoint{11.817432in}{5.632165in}}{\pgfqpoint{11.806382in}{5.632165in}}%
\pgfpathcurveto{\pgfqpoint{11.795332in}{5.632165in}}{\pgfqpoint{11.784733in}{5.627775in}}{\pgfqpoint{11.776919in}{5.619961in}}%
\pgfpathcurveto{\pgfqpoint{11.769106in}{5.612148in}}{\pgfqpoint{11.764715in}{5.601549in}}{\pgfqpoint{11.764715in}{5.590499in}}%
\pgfpathcurveto{\pgfqpoint{11.764715in}{5.579448in}}{\pgfqpoint{11.769106in}{5.568849in}}{\pgfqpoint{11.776919in}{5.561036in}}%
\pgfpathcurveto{\pgfqpoint{11.784733in}{5.553222in}}{\pgfqpoint{11.795332in}{5.548832in}}{\pgfqpoint{11.806382in}{5.548832in}}%
\pgfpathlineto{\pgfqpoint{11.806382in}{5.548832in}}%
\pgfpathclose%
\pgfusepath{stroke}%
\end{pgfscope}%
\begin{pgfscope}%
\pgfpathrectangle{\pgfqpoint{7.512535in}{0.437222in}}{\pgfqpoint{6.275590in}{5.159444in}}%
\pgfusepath{clip}%
\pgfsetbuttcap%
\pgfsetroundjoin%
\pgfsetlinewidth{1.003750pt}%
\definecolor{currentstroke}{rgb}{0.827451,0.827451,0.827451}%
\pgfsetstrokecolor{currentstroke}%
\pgfsetstrokeopacity{0.800000}%
\pgfsetdash{}{0pt}%
\pgfpathmoveto{\pgfqpoint{13.326167in}{5.528637in}}%
\pgfpathcurveto{\pgfqpoint{13.337217in}{5.528637in}}{\pgfqpoint{13.347817in}{5.533028in}}{\pgfqpoint{13.355630in}{5.540841in}}%
\pgfpathcurveto{\pgfqpoint{13.363444in}{5.548655in}}{\pgfqpoint{13.367834in}{5.559254in}}{\pgfqpoint{13.367834in}{5.570304in}}%
\pgfpathcurveto{\pgfqpoint{13.367834in}{5.581354in}}{\pgfqpoint{13.363444in}{5.591953in}}{\pgfqpoint{13.355630in}{5.599767in}}%
\pgfpathcurveto{\pgfqpoint{13.347817in}{5.607580in}}{\pgfqpoint{13.337217in}{5.611971in}}{\pgfqpoint{13.326167in}{5.611971in}}%
\pgfpathcurveto{\pgfqpoint{13.315117in}{5.611971in}}{\pgfqpoint{13.304518in}{5.607580in}}{\pgfqpoint{13.296705in}{5.599767in}}%
\pgfpathcurveto{\pgfqpoint{13.288891in}{5.591953in}}{\pgfqpoint{13.284501in}{5.581354in}}{\pgfqpoint{13.284501in}{5.570304in}}%
\pgfpathcurveto{\pgfqpoint{13.284501in}{5.559254in}}{\pgfqpoint{13.288891in}{5.548655in}}{\pgfqpoint{13.296705in}{5.540841in}}%
\pgfpathcurveto{\pgfqpoint{13.304518in}{5.533028in}}{\pgfqpoint{13.315117in}{5.528637in}}{\pgfqpoint{13.326167in}{5.528637in}}%
\pgfpathlineto{\pgfqpoint{13.326167in}{5.528637in}}%
\pgfpathclose%
\pgfusepath{stroke}%
\end{pgfscope}%
\begin{pgfscope}%
\pgfpathrectangle{\pgfqpoint{7.512535in}{0.437222in}}{\pgfqpoint{6.275590in}{5.159444in}}%
\pgfusepath{clip}%
\pgfsetbuttcap%
\pgfsetroundjoin%
\pgfsetlinewidth{1.003750pt}%
\definecolor{currentstroke}{rgb}{0.827451,0.827451,0.827451}%
\pgfsetstrokecolor{currentstroke}%
\pgfsetstrokeopacity{0.800000}%
\pgfsetdash{}{0pt}%
\pgfpathmoveto{\pgfqpoint{8.277312in}{1.563305in}}%
\pgfpathcurveto{\pgfqpoint{8.288363in}{1.563305in}}{\pgfqpoint{8.298962in}{1.567695in}}{\pgfqpoint{8.306775in}{1.575509in}}%
\pgfpathcurveto{\pgfqpoint{8.314589in}{1.583322in}}{\pgfqpoint{8.318979in}{1.593921in}}{\pgfqpoint{8.318979in}{1.604971in}}%
\pgfpathcurveto{\pgfqpoint{8.318979in}{1.616021in}}{\pgfqpoint{8.314589in}{1.626621in}}{\pgfqpoint{8.306775in}{1.634434in}}%
\pgfpathcurveto{\pgfqpoint{8.298962in}{1.642248in}}{\pgfqpoint{8.288363in}{1.646638in}}{\pgfqpoint{8.277312in}{1.646638in}}%
\pgfpathcurveto{\pgfqpoint{8.266262in}{1.646638in}}{\pgfqpoint{8.255663in}{1.642248in}}{\pgfqpoint{8.247850in}{1.634434in}}%
\pgfpathcurveto{\pgfqpoint{8.240036in}{1.626621in}}{\pgfqpoint{8.235646in}{1.616021in}}{\pgfqpoint{8.235646in}{1.604971in}}%
\pgfpathcurveto{\pgfqpoint{8.235646in}{1.593921in}}{\pgfqpoint{8.240036in}{1.583322in}}{\pgfqpoint{8.247850in}{1.575509in}}%
\pgfpathcurveto{\pgfqpoint{8.255663in}{1.567695in}}{\pgfqpoint{8.266262in}{1.563305in}}{\pgfqpoint{8.277312in}{1.563305in}}%
\pgfpathlineto{\pgfqpoint{8.277312in}{1.563305in}}%
\pgfpathclose%
\pgfusepath{stroke}%
\end{pgfscope}%
\begin{pgfscope}%
\pgfpathrectangle{\pgfqpoint{7.512535in}{0.437222in}}{\pgfqpoint{6.275590in}{5.159444in}}%
\pgfusepath{clip}%
\pgfsetbuttcap%
\pgfsetroundjoin%
\pgfsetlinewidth{1.003750pt}%
\definecolor{currentstroke}{rgb}{0.827451,0.827451,0.827451}%
\pgfsetstrokecolor{currentstroke}%
\pgfsetstrokeopacity{0.800000}%
\pgfsetdash{}{0pt}%
\pgfpathmoveto{\pgfqpoint{8.290348in}{1.139798in}}%
\pgfpathcurveto{\pgfqpoint{8.301398in}{1.139798in}}{\pgfqpoint{8.311997in}{1.144188in}}{\pgfqpoint{8.319811in}{1.152002in}}%
\pgfpathcurveto{\pgfqpoint{8.327625in}{1.159815in}}{\pgfqpoint{8.332015in}{1.170414in}}{\pgfqpoint{8.332015in}{1.181465in}}%
\pgfpathcurveto{\pgfqpoint{8.332015in}{1.192515in}}{\pgfqpoint{8.327625in}{1.203114in}}{\pgfqpoint{8.319811in}{1.210927in}}%
\pgfpathcurveto{\pgfqpoint{8.311997in}{1.218741in}}{\pgfqpoint{8.301398in}{1.223131in}}{\pgfqpoint{8.290348in}{1.223131in}}%
\pgfpathcurveto{\pgfqpoint{8.279298in}{1.223131in}}{\pgfqpoint{8.268699in}{1.218741in}}{\pgfqpoint{8.260886in}{1.210927in}}%
\pgfpathcurveto{\pgfqpoint{8.253072in}{1.203114in}}{\pgfqpoint{8.248682in}{1.192515in}}{\pgfqpoint{8.248682in}{1.181465in}}%
\pgfpathcurveto{\pgfqpoint{8.248682in}{1.170414in}}{\pgfqpoint{8.253072in}{1.159815in}}{\pgfqpoint{8.260886in}{1.152002in}}%
\pgfpathcurveto{\pgfqpoint{8.268699in}{1.144188in}}{\pgfqpoint{8.279298in}{1.139798in}}{\pgfqpoint{8.290348in}{1.139798in}}%
\pgfpathlineto{\pgfqpoint{8.290348in}{1.139798in}}%
\pgfpathclose%
\pgfusepath{stroke}%
\end{pgfscope}%
\begin{pgfscope}%
\pgfpathrectangle{\pgfqpoint{7.512535in}{0.437222in}}{\pgfqpoint{6.275590in}{5.159444in}}%
\pgfusepath{clip}%
\pgfsetbuttcap%
\pgfsetroundjoin%
\pgfsetlinewidth{1.003750pt}%
\definecolor{currentstroke}{rgb}{0.827451,0.827451,0.827451}%
\pgfsetstrokecolor{currentstroke}%
\pgfsetstrokeopacity{0.800000}%
\pgfsetdash{}{0pt}%
\pgfpathmoveto{\pgfqpoint{7.992439in}{1.704977in}}%
\pgfpathcurveto{\pgfqpoint{8.003489in}{1.704977in}}{\pgfqpoint{8.014088in}{1.709367in}}{\pgfqpoint{8.021902in}{1.717181in}}%
\pgfpathcurveto{\pgfqpoint{8.029715in}{1.724994in}}{\pgfqpoint{8.034105in}{1.735593in}}{\pgfqpoint{8.034105in}{1.746643in}}%
\pgfpathcurveto{\pgfqpoint{8.034105in}{1.757694in}}{\pgfqpoint{8.029715in}{1.768293in}}{\pgfqpoint{8.021902in}{1.776106in}}%
\pgfpathcurveto{\pgfqpoint{8.014088in}{1.783920in}}{\pgfqpoint{8.003489in}{1.788310in}}{\pgfqpoint{7.992439in}{1.788310in}}%
\pgfpathcurveto{\pgfqpoint{7.981389in}{1.788310in}}{\pgfqpoint{7.970790in}{1.783920in}}{\pgfqpoint{7.962976in}{1.776106in}}%
\pgfpathcurveto{\pgfqpoint{7.955162in}{1.768293in}}{\pgfqpoint{7.950772in}{1.757694in}}{\pgfqpoint{7.950772in}{1.746643in}}%
\pgfpathcurveto{\pgfqpoint{7.950772in}{1.735593in}}{\pgfqpoint{7.955162in}{1.724994in}}{\pgfqpoint{7.962976in}{1.717181in}}%
\pgfpathcurveto{\pgfqpoint{7.970790in}{1.709367in}}{\pgfqpoint{7.981389in}{1.704977in}}{\pgfqpoint{7.992439in}{1.704977in}}%
\pgfpathlineto{\pgfqpoint{7.992439in}{1.704977in}}%
\pgfpathclose%
\pgfusepath{stroke}%
\end{pgfscope}%
\begin{pgfscope}%
\pgfpathrectangle{\pgfqpoint{7.512535in}{0.437222in}}{\pgfqpoint{6.275590in}{5.159444in}}%
\pgfusepath{clip}%
\pgfsetbuttcap%
\pgfsetroundjoin%
\pgfsetlinewidth{1.003750pt}%
\definecolor{currentstroke}{rgb}{0.827451,0.827451,0.827451}%
\pgfsetstrokecolor{currentstroke}%
\pgfsetstrokeopacity{0.800000}%
\pgfsetdash{}{0pt}%
\pgfpathmoveto{\pgfqpoint{12.102762in}{5.438998in}}%
\pgfpathcurveto{\pgfqpoint{12.113812in}{5.438998in}}{\pgfqpoint{12.124411in}{5.443388in}}{\pgfqpoint{12.132225in}{5.451201in}}%
\pgfpathcurveto{\pgfqpoint{12.140039in}{5.459015in}}{\pgfqpoint{12.144429in}{5.469614in}}{\pgfqpoint{12.144429in}{5.480664in}}%
\pgfpathcurveto{\pgfqpoint{12.144429in}{5.491714in}}{\pgfqpoint{12.140039in}{5.502313in}}{\pgfqpoint{12.132225in}{5.510127in}}%
\pgfpathcurveto{\pgfqpoint{12.124411in}{5.517941in}}{\pgfqpoint{12.113812in}{5.522331in}}{\pgfqpoint{12.102762in}{5.522331in}}%
\pgfpathcurveto{\pgfqpoint{12.091712in}{5.522331in}}{\pgfqpoint{12.081113in}{5.517941in}}{\pgfqpoint{12.073299in}{5.510127in}}%
\pgfpathcurveto{\pgfqpoint{12.065486in}{5.502313in}}{\pgfqpoint{12.061096in}{5.491714in}}{\pgfqpoint{12.061096in}{5.480664in}}%
\pgfpathcurveto{\pgfqpoint{12.061096in}{5.469614in}}{\pgfqpoint{12.065486in}{5.459015in}}{\pgfqpoint{12.073299in}{5.451201in}}%
\pgfpathcurveto{\pgfqpoint{12.081113in}{5.443388in}}{\pgfqpoint{12.091712in}{5.438998in}}{\pgfqpoint{12.102762in}{5.438998in}}%
\pgfpathlineto{\pgfqpoint{12.102762in}{5.438998in}}%
\pgfpathclose%
\pgfusepath{stroke}%
\end{pgfscope}%
\begin{pgfscope}%
\pgfpathrectangle{\pgfqpoint{7.512535in}{0.437222in}}{\pgfqpoint{6.275590in}{5.159444in}}%
\pgfusepath{clip}%
\pgfsetbuttcap%
\pgfsetroundjoin%
\pgfsetlinewidth{1.003750pt}%
\definecolor{currentstroke}{rgb}{0.827451,0.827451,0.827451}%
\pgfsetstrokecolor{currentstroke}%
\pgfsetstrokeopacity{0.800000}%
\pgfsetdash{}{0pt}%
\pgfpathmoveto{\pgfqpoint{11.629363in}{5.304399in}}%
\pgfpathcurveto{\pgfqpoint{11.640413in}{5.304399in}}{\pgfqpoint{11.651012in}{5.308790in}}{\pgfqpoint{11.658826in}{5.316603in}}%
\pgfpathcurveto{\pgfqpoint{11.666639in}{5.324417in}}{\pgfqpoint{11.671030in}{5.335016in}}{\pgfqpoint{11.671030in}{5.346066in}}%
\pgfpathcurveto{\pgfqpoint{11.671030in}{5.357116in}}{\pgfqpoint{11.666639in}{5.367715in}}{\pgfqpoint{11.658826in}{5.375529in}}%
\pgfpathcurveto{\pgfqpoint{11.651012in}{5.383342in}}{\pgfqpoint{11.640413in}{5.387733in}}{\pgfqpoint{11.629363in}{5.387733in}}%
\pgfpathcurveto{\pgfqpoint{11.618313in}{5.387733in}}{\pgfqpoint{11.607714in}{5.383342in}}{\pgfqpoint{11.599900in}{5.375529in}}%
\pgfpathcurveto{\pgfqpoint{11.592087in}{5.367715in}}{\pgfqpoint{11.587696in}{5.357116in}}{\pgfqpoint{11.587696in}{5.346066in}}%
\pgfpathcurveto{\pgfqpoint{11.587696in}{5.335016in}}{\pgfqpoint{11.592087in}{5.324417in}}{\pgfqpoint{11.599900in}{5.316603in}}%
\pgfpathcurveto{\pgfqpoint{11.607714in}{5.308790in}}{\pgfqpoint{11.618313in}{5.304399in}}{\pgfqpoint{11.629363in}{5.304399in}}%
\pgfpathlineto{\pgfqpoint{11.629363in}{5.304399in}}%
\pgfpathclose%
\pgfusepath{stroke}%
\end{pgfscope}%
\begin{pgfscope}%
\pgfpathrectangle{\pgfqpoint{7.512535in}{0.437222in}}{\pgfqpoint{6.275590in}{5.159444in}}%
\pgfusepath{clip}%
\pgfsetbuttcap%
\pgfsetroundjoin%
\pgfsetlinewidth{1.003750pt}%
\definecolor{currentstroke}{rgb}{0.827451,0.827451,0.827451}%
\pgfsetstrokecolor{currentstroke}%
\pgfsetstrokeopacity{0.800000}%
\pgfsetdash{}{0pt}%
\pgfpathmoveto{\pgfqpoint{8.983729in}{3.220234in}}%
\pgfpathcurveto{\pgfqpoint{8.994779in}{3.220234in}}{\pgfqpoint{9.005378in}{3.224624in}}{\pgfqpoint{9.013191in}{3.232438in}}%
\pgfpathcurveto{\pgfqpoint{9.021005in}{3.240251in}}{\pgfqpoint{9.025395in}{3.250850in}}{\pgfqpoint{9.025395in}{3.261901in}}%
\pgfpathcurveto{\pgfqpoint{9.025395in}{3.272951in}}{\pgfqpoint{9.021005in}{3.283550in}}{\pgfqpoint{9.013191in}{3.291363in}}%
\pgfpathcurveto{\pgfqpoint{9.005378in}{3.299177in}}{\pgfqpoint{8.994779in}{3.303567in}}{\pgfqpoint{8.983729in}{3.303567in}}%
\pgfpathcurveto{\pgfqpoint{8.972679in}{3.303567in}}{\pgfqpoint{8.962079in}{3.299177in}}{\pgfqpoint{8.954266in}{3.291363in}}%
\pgfpathcurveto{\pgfqpoint{8.946452in}{3.283550in}}{\pgfqpoint{8.942062in}{3.272951in}}{\pgfqpoint{8.942062in}{3.261901in}}%
\pgfpathcurveto{\pgfqpoint{8.942062in}{3.250850in}}{\pgfqpoint{8.946452in}{3.240251in}}{\pgfqpoint{8.954266in}{3.232438in}}%
\pgfpathcurveto{\pgfqpoint{8.962079in}{3.224624in}}{\pgfqpoint{8.972679in}{3.220234in}}{\pgfqpoint{8.983729in}{3.220234in}}%
\pgfpathlineto{\pgfqpoint{8.983729in}{3.220234in}}%
\pgfpathclose%
\pgfusepath{stroke}%
\end{pgfscope}%
\begin{pgfscope}%
\pgfpathrectangle{\pgfqpoint{7.512535in}{0.437222in}}{\pgfqpoint{6.275590in}{5.159444in}}%
\pgfusepath{clip}%
\pgfsetbuttcap%
\pgfsetroundjoin%
\pgfsetlinewidth{1.003750pt}%
\definecolor{currentstroke}{rgb}{0.827451,0.827451,0.827451}%
\pgfsetstrokecolor{currentstroke}%
\pgfsetstrokeopacity{0.800000}%
\pgfsetdash{}{0pt}%
\pgfpathmoveto{\pgfqpoint{13.581463in}{5.528743in}}%
\pgfpathcurveto{\pgfqpoint{13.592513in}{5.528743in}}{\pgfqpoint{13.603113in}{5.533133in}}{\pgfqpoint{13.610926in}{5.540947in}}%
\pgfpathcurveto{\pgfqpoint{13.618740in}{5.548760in}}{\pgfqpoint{13.623130in}{5.559359in}}{\pgfqpoint{13.623130in}{5.570410in}}%
\pgfpathcurveto{\pgfqpoint{13.623130in}{5.581460in}}{\pgfqpoint{13.618740in}{5.592059in}}{\pgfqpoint{13.610926in}{5.599872in}}%
\pgfpathcurveto{\pgfqpoint{13.603113in}{5.607686in}}{\pgfqpoint{13.592513in}{5.612076in}}{\pgfqpoint{13.581463in}{5.612076in}}%
\pgfpathcurveto{\pgfqpoint{13.570413in}{5.612076in}}{\pgfqpoint{13.559814in}{5.607686in}}{\pgfqpoint{13.552001in}{5.599872in}}%
\pgfpathcurveto{\pgfqpoint{13.544187in}{5.592059in}}{\pgfqpoint{13.539797in}{5.581460in}}{\pgfqpoint{13.539797in}{5.570410in}}%
\pgfpathcurveto{\pgfqpoint{13.539797in}{5.559359in}}{\pgfqpoint{13.544187in}{5.548760in}}{\pgfqpoint{13.552001in}{5.540947in}}%
\pgfpathcurveto{\pgfqpoint{13.559814in}{5.533133in}}{\pgfqpoint{13.570413in}{5.528743in}}{\pgfqpoint{13.581463in}{5.528743in}}%
\pgfpathlineto{\pgfqpoint{13.581463in}{5.528743in}}%
\pgfpathclose%
\pgfusepath{stroke}%
\end{pgfscope}%
\begin{pgfscope}%
\pgfpathrectangle{\pgfqpoint{7.512535in}{0.437222in}}{\pgfqpoint{6.275590in}{5.159444in}}%
\pgfusepath{clip}%
\pgfsetbuttcap%
\pgfsetroundjoin%
\pgfsetlinewidth{1.003750pt}%
\definecolor{currentstroke}{rgb}{0.827451,0.827451,0.827451}%
\pgfsetstrokecolor{currentstroke}%
\pgfsetstrokeopacity{0.800000}%
\pgfsetdash{}{0pt}%
\pgfpathmoveto{\pgfqpoint{12.815587in}{5.553845in}}%
\pgfpathcurveto{\pgfqpoint{12.826637in}{5.553845in}}{\pgfqpoint{12.837236in}{5.558236in}}{\pgfqpoint{12.845050in}{5.566049in}}%
\pgfpathcurveto{\pgfqpoint{12.852863in}{5.573863in}}{\pgfqpoint{12.857254in}{5.584462in}}{\pgfqpoint{12.857254in}{5.595512in}}%
\pgfpathcurveto{\pgfqpoint{12.857254in}{5.606562in}}{\pgfqpoint{12.852863in}{5.617161in}}{\pgfqpoint{12.845050in}{5.624975in}}%
\pgfpathcurveto{\pgfqpoint{12.837236in}{5.632788in}}{\pgfqpoint{12.826637in}{5.637179in}}{\pgfqpoint{12.815587in}{5.637179in}}%
\pgfpathcurveto{\pgfqpoint{12.804537in}{5.637179in}}{\pgfqpoint{12.793938in}{5.632788in}}{\pgfqpoint{12.786124in}{5.624975in}}%
\pgfpathcurveto{\pgfqpoint{12.778310in}{5.617161in}}{\pgfqpoint{12.773920in}{5.606562in}}{\pgfqpoint{12.773920in}{5.595512in}}%
\pgfpathcurveto{\pgfqpoint{12.773920in}{5.584462in}}{\pgfqpoint{12.778310in}{5.573863in}}{\pgfqpoint{12.786124in}{5.566049in}}%
\pgfpathcurveto{\pgfqpoint{12.793938in}{5.558236in}}{\pgfqpoint{12.804537in}{5.553845in}}{\pgfqpoint{12.815587in}{5.553845in}}%
\pgfpathlineto{\pgfqpoint{12.815587in}{5.553845in}}%
\pgfpathclose%
\pgfusepath{stroke}%
\end{pgfscope}%
\begin{pgfscope}%
\pgfpathrectangle{\pgfqpoint{7.512535in}{0.437222in}}{\pgfqpoint{6.275590in}{5.159444in}}%
\pgfusepath{clip}%
\pgfsetbuttcap%
\pgfsetroundjoin%
\pgfsetlinewidth{1.003750pt}%
\definecolor{currentstroke}{rgb}{0.827451,0.827451,0.827451}%
\pgfsetstrokecolor{currentstroke}%
\pgfsetstrokeopacity{0.800000}%
\pgfsetdash{}{0pt}%
\pgfpathmoveto{\pgfqpoint{8.896008in}{2.422402in}}%
\pgfpathcurveto{\pgfqpoint{8.907058in}{2.422402in}}{\pgfqpoint{8.917657in}{2.426793in}}{\pgfqpoint{8.925471in}{2.434606in}}%
\pgfpathcurveto{\pgfqpoint{8.933285in}{2.442420in}}{\pgfqpoint{8.937675in}{2.453019in}}{\pgfqpoint{8.937675in}{2.464069in}}%
\pgfpathcurveto{\pgfqpoint{8.937675in}{2.475119in}}{\pgfqpoint{8.933285in}{2.485718in}}{\pgfqpoint{8.925471in}{2.493532in}}%
\pgfpathcurveto{\pgfqpoint{8.917657in}{2.501345in}}{\pgfqpoint{8.907058in}{2.505736in}}{\pgfqpoint{8.896008in}{2.505736in}}%
\pgfpathcurveto{\pgfqpoint{8.884958in}{2.505736in}}{\pgfqpoint{8.874359in}{2.501345in}}{\pgfqpoint{8.866545in}{2.493532in}}%
\pgfpathcurveto{\pgfqpoint{8.858732in}{2.485718in}}{\pgfqpoint{8.854341in}{2.475119in}}{\pgfqpoint{8.854341in}{2.464069in}}%
\pgfpathcurveto{\pgfqpoint{8.854341in}{2.453019in}}{\pgfqpoint{8.858732in}{2.442420in}}{\pgfqpoint{8.866545in}{2.434606in}}%
\pgfpathcurveto{\pgfqpoint{8.874359in}{2.426793in}}{\pgfqpoint{8.884958in}{2.422402in}}{\pgfqpoint{8.896008in}{2.422402in}}%
\pgfpathlineto{\pgfqpoint{8.896008in}{2.422402in}}%
\pgfpathclose%
\pgfusepath{stroke}%
\end{pgfscope}%
\begin{pgfscope}%
\pgfpathrectangle{\pgfqpoint{7.512535in}{0.437222in}}{\pgfqpoint{6.275590in}{5.159444in}}%
\pgfusepath{clip}%
\pgfsetbuttcap%
\pgfsetroundjoin%
\pgfsetlinewidth{1.003750pt}%
\definecolor{currentstroke}{rgb}{0.827451,0.827451,0.827451}%
\pgfsetstrokecolor{currentstroke}%
\pgfsetstrokeopacity{0.800000}%
\pgfsetdash{}{0pt}%
\pgfpathmoveto{\pgfqpoint{13.419279in}{5.549860in}}%
\pgfpathcurveto{\pgfqpoint{13.430329in}{5.549860in}}{\pgfqpoint{13.440928in}{5.554250in}}{\pgfqpoint{13.448742in}{5.562064in}}%
\pgfpathcurveto{\pgfqpoint{13.456555in}{5.569877in}}{\pgfqpoint{13.460945in}{5.580476in}}{\pgfqpoint{13.460945in}{5.591526in}}%
\pgfpathcurveto{\pgfqpoint{13.460945in}{5.602576in}}{\pgfqpoint{13.456555in}{5.613175in}}{\pgfqpoint{13.448742in}{5.620989in}}%
\pgfpathcurveto{\pgfqpoint{13.440928in}{5.628803in}}{\pgfqpoint{13.430329in}{5.633193in}}{\pgfqpoint{13.419279in}{5.633193in}}%
\pgfpathcurveto{\pgfqpoint{13.408229in}{5.633193in}}{\pgfqpoint{13.397630in}{5.628803in}}{\pgfqpoint{13.389816in}{5.620989in}}%
\pgfpathcurveto{\pgfqpoint{13.382002in}{5.613175in}}{\pgfqpoint{13.377612in}{5.602576in}}{\pgfqpoint{13.377612in}{5.591526in}}%
\pgfpathcurveto{\pgfqpoint{13.377612in}{5.580476in}}{\pgfqpoint{13.382002in}{5.569877in}}{\pgfqpoint{13.389816in}{5.562064in}}%
\pgfpathcurveto{\pgfqpoint{13.397630in}{5.554250in}}{\pgfqpoint{13.408229in}{5.549860in}}{\pgfqpoint{13.419279in}{5.549860in}}%
\pgfpathlineto{\pgfqpoint{13.419279in}{5.549860in}}%
\pgfpathclose%
\pgfusepath{stroke}%
\end{pgfscope}%
\begin{pgfscope}%
\pgfpathrectangle{\pgfqpoint{7.512535in}{0.437222in}}{\pgfqpoint{6.275590in}{5.159444in}}%
\pgfusepath{clip}%
\pgfsetbuttcap%
\pgfsetroundjoin%
\pgfsetlinewidth{1.003750pt}%
\definecolor{currentstroke}{rgb}{0.827451,0.827451,0.827451}%
\pgfsetstrokecolor{currentstroke}%
\pgfsetstrokeopacity{0.800000}%
\pgfsetdash{}{0pt}%
\pgfpathmoveto{\pgfqpoint{10.689319in}{4.486189in}}%
\pgfpathcurveto{\pgfqpoint{10.700369in}{4.486189in}}{\pgfqpoint{10.710968in}{4.490579in}}{\pgfqpoint{10.718782in}{4.498393in}}%
\pgfpathcurveto{\pgfqpoint{10.726595in}{4.506206in}}{\pgfqpoint{10.730986in}{4.516805in}}{\pgfqpoint{10.730986in}{4.527856in}}%
\pgfpathcurveto{\pgfqpoint{10.730986in}{4.538906in}}{\pgfqpoint{10.726595in}{4.549505in}}{\pgfqpoint{10.718782in}{4.557318in}}%
\pgfpathcurveto{\pgfqpoint{10.710968in}{4.565132in}}{\pgfqpoint{10.700369in}{4.569522in}}{\pgfqpoint{10.689319in}{4.569522in}}%
\pgfpathcurveto{\pgfqpoint{10.678269in}{4.569522in}}{\pgfqpoint{10.667670in}{4.565132in}}{\pgfqpoint{10.659856in}{4.557318in}}%
\pgfpathcurveto{\pgfqpoint{10.652043in}{4.549505in}}{\pgfqpoint{10.647652in}{4.538906in}}{\pgfqpoint{10.647652in}{4.527856in}}%
\pgfpathcurveto{\pgfqpoint{10.647652in}{4.516805in}}{\pgfqpoint{10.652043in}{4.506206in}}{\pgfqpoint{10.659856in}{4.498393in}}%
\pgfpathcurveto{\pgfqpoint{10.667670in}{4.490579in}}{\pgfqpoint{10.678269in}{4.486189in}}{\pgfqpoint{10.689319in}{4.486189in}}%
\pgfpathlineto{\pgfqpoint{10.689319in}{4.486189in}}%
\pgfpathclose%
\pgfusepath{stroke}%
\end{pgfscope}%
\begin{pgfscope}%
\pgfpathrectangle{\pgfqpoint{7.512535in}{0.437222in}}{\pgfqpoint{6.275590in}{5.159444in}}%
\pgfusepath{clip}%
\pgfsetbuttcap%
\pgfsetroundjoin%
\pgfsetlinewidth{1.003750pt}%
\definecolor{currentstroke}{rgb}{0.827451,0.827451,0.827451}%
\pgfsetstrokecolor{currentstroke}%
\pgfsetstrokeopacity{0.800000}%
\pgfsetdash{}{0pt}%
\pgfpathmoveto{\pgfqpoint{9.928499in}{4.056499in}}%
\pgfpathcurveto{\pgfqpoint{9.939549in}{4.056499in}}{\pgfqpoint{9.950148in}{4.060889in}}{\pgfqpoint{9.957962in}{4.068703in}}%
\pgfpathcurveto{\pgfqpoint{9.965775in}{4.076516in}}{\pgfqpoint{9.970165in}{4.087115in}}{\pgfqpoint{9.970165in}{4.098166in}}%
\pgfpathcurveto{\pgfqpoint{9.970165in}{4.109216in}}{\pgfqpoint{9.965775in}{4.119815in}}{\pgfqpoint{9.957962in}{4.127628in}}%
\pgfpathcurveto{\pgfqpoint{9.950148in}{4.135442in}}{\pgfqpoint{9.939549in}{4.139832in}}{\pgfqpoint{9.928499in}{4.139832in}}%
\pgfpathcurveto{\pgfqpoint{9.917449in}{4.139832in}}{\pgfqpoint{9.906850in}{4.135442in}}{\pgfqpoint{9.899036in}{4.127628in}}%
\pgfpathcurveto{\pgfqpoint{9.891222in}{4.119815in}}{\pgfqpoint{9.886832in}{4.109216in}}{\pgfqpoint{9.886832in}{4.098166in}}%
\pgfpathcurveto{\pgfqpoint{9.886832in}{4.087115in}}{\pgfqpoint{9.891222in}{4.076516in}}{\pgfqpoint{9.899036in}{4.068703in}}%
\pgfpathcurveto{\pgfqpoint{9.906850in}{4.060889in}}{\pgfqpoint{9.917449in}{4.056499in}}{\pgfqpoint{9.928499in}{4.056499in}}%
\pgfpathlineto{\pgfqpoint{9.928499in}{4.056499in}}%
\pgfpathclose%
\pgfusepath{stroke}%
\end{pgfscope}%
\begin{pgfscope}%
\pgfpathrectangle{\pgfqpoint{7.512535in}{0.437222in}}{\pgfqpoint{6.275590in}{5.159444in}}%
\pgfusepath{clip}%
\pgfsetbuttcap%
\pgfsetroundjoin%
\pgfsetlinewidth{1.003750pt}%
\definecolor{currentstroke}{rgb}{0.827451,0.827451,0.827451}%
\pgfsetstrokecolor{currentstroke}%
\pgfsetstrokeopacity{0.800000}%
\pgfsetdash{}{0pt}%
\pgfpathmoveto{\pgfqpoint{8.290348in}{1.011495in}}%
\pgfpathcurveto{\pgfqpoint{8.301398in}{1.011495in}}{\pgfqpoint{8.311997in}{1.015885in}}{\pgfqpoint{8.319811in}{1.023699in}}%
\pgfpathcurveto{\pgfqpoint{8.327625in}{1.031512in}}{\pgfqpoint{8.332015in}{1.042111in}}{\pgfqpoint{8.332015in}{1.053161in}}%
\pgfpathcurveto{\pgfqpoint{8.332015in}{1.064211in}}{\pgfqpoint{8.327625in}{1.074810in}}{\pgfqpoint{8.319811in}{1.082624in}}%
\pgfpathcurveto{\pgfqpoint{8.311997in}{1.090438in}}{\pgfqpoint{8.301398in}{1.094828in}}{\pgfqpoint{8.290348in}{1.094828in}}%
\pgfpathcurveto{\pgfqpoint{8.279298in}{1.094828in}}{\pgfqpoint{8.268699in}{1.090438in}}{\pgfqpoint{8.260886in}{1.082624in}}%
\pgfpathcurveto{\pgfqpoint{8.253072in}{1.074810in}}{\pgfqpoint{8.248682in}{1.064211in}}{\pgfqpoint{8.248682in}{1.053161in}}%
\pgfpathcurveto{\pgfqpoint{8.248682in}{1.042111in}}{\pgfqpoint{8.253072in}{1.031512in}}{\pgfqpoint{8.260886in}{1.023699in}}%
\pgfpathcurveto{\pgfqpoint{8.268699in}{1.015885in}}{\pgfqpoint{8.279298in}{1.011495in}}{\pgfqpoint{8.290348in}{1.011495in}}%
\pgfpathlineto{\pgfqpoint{8.290348in}{1.011495in}}%
\pgfpathclose%
\pgfusepath{stroke}%
\end{pgfscope}%
\begin{pgfscope}%
\pgfpathrectangle{\pgfqpoint{7.512535in}{0.437222in}}{\pgfqpoint{6.275590in}{5.159444in}}%
\pgfusepath{clip}%
\pgfsetbuttcap%
\pgfsetroundjoin%
\pgfsetlinewidth{1.003750pt}%
\definecolor{currentstroke}{rgb}{0.827451,0.827451,0.827451}%
\pgfsetstrokecolor{currentstroke}%
\pgfsetstrokeopacity{0.800000}%
\pgfsetdash{}{0pt}%
\pgfpathmoveto{\pgfqpoint{7.974778in}{1.314099in}}%
\pgfpathcurveto{\pgfqpoint{7.985828in}{1.314099in}}{\pgfqpoint{7.996427in}{1.318489in}}{\pgfqpoint{8.004241in}{1.326303in}}%
\pgfpathcurveto{\pgfqpoint{8.012055in}{1.334116in}}{\pgfqpoint{8.016445in}{1.344715in}}{\pgfqpoint{8.016445in}{1.355766in}}%
\pgfpathcurveto{\pgfqpoint{8.016445in}{1.366816in}}{\pgfqpoint{8.012055in}{1.377415in}}{\pgfqpoint{8.004241in}{1.385228in}}%
\pgfpathcurveto{\pgfqpoint{7.996427in}{1.393042in}}{\pgfqpoint{7.985828in}{1.397432in}}{\pgfqpoint{7.974778in}{1.397432in}}%
\pgfpathcurveto{\pgfqpoint{7.963728in}{1.397432in}}{\pgfqpoint{7.953129in}{1.393042in}}{\pgfqpoint{7.945315in}{1.385228in}}%
\pgfpathcurveto{\pgfqpoint{7.937502in}{1.377415in}}{\pgfqpoint{7.933112in}{1.366816in}}{\pgfqpoint{7.933112in}{1.355766in}}%
\pgfpathcurveto{\pgfqpoint{7.933112in}{1.344715in}}{\pgfqpoint{7.937502in}{1.334116in}}{\pgfqpoint{7.945315in}{1.326303in}}%
\pgfpathcurveto{\pgfqpoint{7.953129in}{1.318489in}}{\pgfqpoint{7.963728in}{1.314099in}}{\pgfqpoint{7.974778in}{1.314099in}}%
\pgfpathlineto{\pgfqpoint{7.974778in}{1.314099in}}%
\pgfpathclose%
\pgfusepath{stroke}%
\end{pgfscope}%
\begin{pgfscope}%
\pgfpathrectangle{\pgfqpoint{7.512535in}{0.437222in}}{\pgfqpoint{6.275590in}{5.159444in}}%
\pgfusepath{clip}%
\pgfsetbuttcap%
\pgfsetroundjoin%
\pgfsetlinewidth{1.003750pt}%
\definecolor{currentstroke}{rgb}{0.827451,0.827451,0.827451}%
\pgfsetstrokecolor{currentstroke}%
\pgfsetstrokeopacity{0.800000}%
\pgfsetdash{}{0pt}%
\pgfpathmoveto{\pgfqpoint{13.672716in}{5.539504in}}%
\pgfpathcurveto{\pgfqpoint{13.683766in}{5.539504in}}{\pgfqpoint{13.694366in}{5.543895in}}{\pgfqpoint{13.702179in}{5.551708in}}%
\pgfpathcurveto{\pgfqpoint{13.709993in}{5.559522in}}{\pgfqpoint{13.714383in}{5.570121in}}{\pgfqpoint{13.714383in}{5.581171in}}%
\pgfpathcurveto{\pgfqpoint{13.714383in}{5.592221in}}{\pgfqpoint{13.709993in}{5.602820in}}{\pgfqpoint{13.702179in}{5.610634in}}%
\pgfpathcurveto{\pgfqpoint{13.694366in}{5.618447in}}{\pgfqpoint{13.683766in}{5.622838in}}{\pgfqpoint{13.672716in}{5.622838in}}%
\pgfpathcurveto{\pgfqpoint{13.661666in}{5.622838in}}{\pgfqpoint{13.651067in}{5.618447in}}{\pgfqpoint{13.643254in}{5.610634in}}%
\pgfpathcurveto{\pgfqpoint{13.635440in}{5.602820in}}{\pgfqpoint{13.631050in}{5.592221in}}{\pgfqpoint{13.631050in}{5.581171in}}%
\pgfpathcurveto{\pgfqpoint{13.631050in}{5.570121in}}{\pgfqpoint{13.635440in}{5.559522in}}{\pgfqpoint{13.643254in}{5.551708in}}%
\pgfpathcurveto{\pgfqpoint{13.651067in}{5.543895in}}{\pgfqpoint{13.661666in}{5.539504in}}{\pgfqpoint{13.672716in}{5.539504in}}%
\pgfpathlineto{\pgfqpoint{13.672716in}{5.539504in}}%
\pgfpathclose%
\pgfusepath{stroke}%
\end{pgfscope}%
\begin{pgfscope}%
\pgfpathrectangle{\pgfqpoint{7.512535in}{0.437222in}}{\pgfqpoint{6.275590in}{5.159444in}}%
\pgfusepath{clip}%
\pgfsetbuttcap%
\pgfsetroundjoin%
\pgfsetlinewidth{1.003750pt}%
\definecolor{currentstroke}{rgb}{0.827451,0.827451,0.827451}%
\pgfsetstrokecolor{currentstroke}%
\pgfsetstrokeopacity{0.800000}%
\pgfsetdash{}{0pt}%
\pgfpathmoveto{\pgfqpoint{9.697299in}{3.449334in}}%
\pgfpathcurveto{\pgfqpoint{9.708349in}{3.449334in}}{\pgfqpoint{9.718948in}{3.453725in}}{\pgfqpoint{9.726762in}{3.461538in}}%
\pgfpathcurveto{\pgfqpoint{9.734575in}{3.469352in}}{\pgfqpoint{9.738965in}{3.479951in}}{\pgfqpoint{9.738965in}{3.491001in}}%
\pgfpathcurveto{\pgfqpoint{9.738965in}{3.502051in}}{\pgfqpoint{9.734575in}{3.512650in}}{\pgfqpoint{9.726762in}{3.520464in}}%
\pgfpathcurveto{\pgfqpoint{9.718948in}{3.528277in}}{\pgfqpoint{9.708349in}{3.532668in}}{\pgfqpoint{9.697299in}{3.532668in}}%
\pgfpathcurveto{\pgfqpoint{9.686249in}{3.532668in}}{\pgfqpoint{9.675650in}{3.528277in}}{\pgfqpoint{9.667836in}{3.520464in}}%
\pgfpathcurveto{\pgfqpoint{9.660022in}{3.512650in}}{\pgfqpoint{9.655632in}{3.502051in}}{\pgfqpoint{9.655632in}{3.491001in}}%
\pgfpathcurveto{\pgfqpoint{9.655632in}{3.479951in}}{\pgfqpoint{9.660022in}{3.469352in}}{\pgfqpoint{9.667836in}{3.461538in}}%
\pgfpathcurveto{\pgfqpoint{9.675650in}{3.453725in}}{\pgfqpoint{9.686249in}{3.449334in}}{\pgfqpoint{9.697299in}{3.449334in}}%
\pgfpathlineto{\pgfqpoint{9.697299in}{3.449334in}}%
\pgfpathclose%
\pgfusepath{stroke}%
\end{pgfscope}%
\begin{pgfscope}%
\pgfpathrectangle{\pgfqpoint{7.512535in}{0.437222in}}{\pgfqpoint{6.275590in}{5.159444in}}%
\pgfusepath{clip}%
\pgfsetbuttcap%
\pgfsetroundjoin%
\pgfsetlinewidth{1.003750pt}%
\definecolor{currentstroke}{rgb}{0.827451,0.827451,0.827451}%
\pgfsetstrokecolor{currentstroke}%
\pgfsetstrokeopacity{0.800000}%
\pgfsetdash{}{0pt}%
\pgfpathmoveto{\pgfqpoint{10.259535in}{4.447570in}}%
\pgfpathcurveto{\pgfqpoint{10.270585in}{4.447570in}}{\pgfqpoint{10.281184in}{4.451960in}}{\pgfqpoint{10.288997in}{4.459773in}}%
\pgfpathcurveto{\pgfqpoint{10.296811in}{4.467587in}}{\pgfqpoint{10.301201in}{4.478186in}}{\pgfqpoint{10.301201in}{4.489236in}}%
\pgfpathcurveto{\pgfqpoint{10.301201in}{4.500286in}}{\pgfqpoint{10.296811in}{4.510885in}}{\pgfqpoint{10.288997in}{4.518699in}}%
\pgfpathcurveto{\pgfqpoint{10.281184in}{4.526513in}}{\pgfqpoint{10.270585in}{4.530903in}}{\pgfqpoint{10.259535in}{4.530903in}}%
\pgfpathcurveto{\pgfqpoint{10.248485in}{4.530903in}}{\pgfqpoint{10.237886in}{4.526513in}}{\pgfqpoint{10.230072in}{4.518699in}}%
\pgfpathcurveto{\pgfqpoint{10.222258in}{4.510885in}}{\pgfqpoint{10.217868in}{4.500286in}}{\pgfqpoint{10.217868in}{4.489236in}}%
\pgfpathcurveto{\pgfqpoint{10.217868in}{4.478186in}}{\pgfqpoint{10.222258in}{4.467587in}}{\pgfqpoint{10.230072in}{4.459773in}}%
\pgfpathcurveto{\pgfqpoint{10.237886in}{4.451960in}}{\pgfqpoint{10.248485in}{4.447570in}}{\pgfqpoint{10.259535in}{4.447570in}}%
\pgfpathlineto{\pgfqpoint{10.259535in}{4.447570in}}%
\pgfpathclose%
\pgfusepath{stroke}%
\end{pgfscope}%
\begin{pgfscope}%
\pgfpathrectangle{\pgfqpoint{7.512535in}{0.437222in}}{\pgfqpoint{6.275590in}{5.159444in}}%
\pgfusepath{clip}%
\pgfsetbuttcap%
\pgfsetroundjoin%
\pgfsetlinewidth{1.003750pt}%
\definecolor{currentstroke}{rgb}{0.827451,0.827451,0.827451}%
\pgfsetstrokecolor{currentstroke}%
\pgfsetstrokeopacity{0.800000}%
\pgfsetdash{}{0pt}%
\pgfpathmoveto{\pgfqpoint{9.821149in}{3.449334in}}%
\pgfpathcurveto{\pgfqpoint{9.832199in}{3.449334in}}{\pgfqpoint{9.842798in}{3.453725in}}{\pgfqpoint{9.850611in}{3.461538in}}%
\pgfpathcurveto{\pgfqpoint{9.858425in}{3.469352in}}{\pgfqpoint{9.862815in}{3.479951in}}{\pgfqpoint{9.862815in}{3.491001in}}%
\pgfpathcurveto{\pgfqpoint{9.862815in}{3.502051in}}{\pgfqpoint{9.858425in}{3.512650in}}{\pgfqpoint{9.850611in}{3.520464in}}%
\pgfpathcurveto{\pgfqpoint{9.842798in}{3.528277in}}{\pgfqpoint{9.832199in}{3.532668in}}{\pgfqpoint{9.821149in}{3.532668in}}%
\pgfpathcurveto{\pgfqpoint{9.810099in}{3.532668in}}{\pgfqpoint{9.799500in}{3.528277in}}{\pgfqpoint{9.791686in}{3.520464in}}%
\pgfpathcurveto{\pgfqpoint{9.783872in}{3.512650in}}{\pgfqpoint{9.779482in}{3.502051in}}{\pgfqpoint{9.779482in}{3.491001in}}%
\pgfpathcurveto{\pgfqpoint{9.779482in}{3.479951in}}{\pgfqpoint{9.783872in}{3.469352in}}{\pgfqpoint{9.791686in}{3.461538in}}%
\pgfpathcurveto{\pgfqpoint{9.799500in}{3.453725in}}{\pgfqpoint{9.810099in}{3.449334in}}{\pgfqpoint{9.821149in}{3.449334in}}%
\pgfpathlineto{\pgfqpoint{9.821149in}{3.449334in}}%
\pgfpathclose%
\pgfusepath{stroke}%
\end{pgfscope}%
\begin{pgfscope}%
\pgfpathrectangle{\pgfqpoint{7.512535in}{0.437222in}}{\pgfqpoint{6.275590in}{5.159444in}}%
\pgfusepath{clip}%
\pgfsetbuttcap%
\pgfsetroundjoin%
\pgfsetlinewidth{1.003750pt}%
\definecolor{currentstroke}{rgb}{0.827451,0.827451,0.827451}%
\pgfsetstrokecolor{currentstroke}%
\pgfsetstrokeopacity{0.800000}%
\pgfsetdash{}{0pt}%
\pgfpathmoveto{\pgfqpoint{8.290348in}{1.426234in}}%
\pgfpathcurveto{\pgfqpoint{8.301398in}{1.426234in}}{\pgfqpoint{8.311997in}{1.430624in}}{\pgfqpoint{8.319811in}{1.438438in}}%
\pgfpathcurveto{\pgfqpoint{8.327625in}{1.446251in}}{\pgfqpoint{8.332015in}{1.456850in}}{\pgfqpoint{8.332015in}{1.467900in}}%
\pgfpathcurveto{\pgfqpoint{8.332015in}{1.478951in}}{\pgfqpoint{8.327625in}{1.489550in}}{\pgfqpoint{8.319811in}{1.497363in}}%
\pgfpathcurveto{\pgfqpoint{8.311997in}{1.505177in}}{\pgfqpoint{8.301398in}{1.509567in}}{\pgfqpoint{8.290348in}{1.509567in}}%
\pgfpathcurveto{\pgfqpoint{8.279298in}{1.509567in}}{\pgfqpoint{8.268699in}{1.505177in}}{\pgfqpoint{8.260886in}{1.497363in}}%
\pgfpathcurveto{\pgfqpoint{8.253072in}{1.489550in}}{\pgfqpoint{8.248682in}{1.478951in}}{\pgfqpoint{8.248682in}{1.467900in}}%
\pgfpathcurveto{\pgfqpoint{8.248682in}{1.456850in}}{\pgfqpoint{8.253072in}{1.446251in}}{\pgfqpoint{8.260886in}{1.438438in}}%
\pgfpathcurveto{\pgfqpoint{8.268699in}{1.430624in}}{\pgfqpoint{8.279298in}{1.426234in}}{\pgfqpoint{8.290348in}{1.426234in}}%
\pgfpathlineto{\pgfqpoint{8.290348in}{1.426234in}}%
\pgfpathclose%
\pgfusepath{stroke}%
\end{pgfscope}%
\begin{pgfscope}%
\pgfpathrectangle{\pgfqpoint{7.512535in}{0.437222in}}{\pgfqpoint{6.275590in}{5.159444in}}%
\pgfusepath{clip}%
\pgfsetbuttcap%
\pgfsetroundjoin%
\pgfsetlinewidth{1.003750pt}%
\definecolor{currentstroke}{rgb}{0.827451,0.827451,0.827451}%
\pgfsetstrokecolor{currentstroke}%
\pgfsetstrokeopacity{0.800000}%
\pgfsetdash{}{0pt}%
\pgfpathmoveto{\pgfqpoint{8.432851in}{1.518124in}}%
\pgfpathcurveto{\pgfqpoint{8.443901in}{1.518124in}}{\pgfqpoint{8.454500in}{1.522514in}}{\pgfqpoint{8.462313in}{1.530328in}}%
\pgfpathcurveto{\pgfqpoint{8.470127in}{1.538142in}}{\pgfqpoint{8.474517in}{1.548741in}}{\pgfqpoint{8.474517in}{1.559791in}}%
\pgfpathcurveto{\pgfqpoint{8.474517in}{1.570841in}}{\pgfqpoint{8.470127in}{1.581440in}}{\pgfqpoint{8.462313in}{1.589254in}}%
\pgfpathcurveto{\pgfqpoint{8.454500in}{1.597067in}}{\pgfqpoint{8.443901in}{1.601457in}}{\pgfqpoint{8.432851in}{1.601457in}}%
\pgfpathcurveto{\pgfqpoint{8.421801in}{1.601457in}}{\pgfqpoint{8.411201in}{1.597067in}}{\pgfqpoint{8.403388in}{1.589254in}}%
\pgfpathcurveto{\pgfqpoint{8.395574in}{1.581440in}}{\pgfqpoint{8.391184in}{1.570841in}}{\pgfqpoint{8.391184in}{1.559791in}}%
\pgfpathcurveto{\pgfqpoint{8.391184in}{1.548741in}}{\pgfqpoint{8.395574in}{1.538142in}}{\pgfqpoint{8.403388in}{1.530328in}}%
\pgfpathcurveto{\pgfqpoint{8.411201in}{1.522514in}}{\pgfqpoint{8.421801in}{1.518124in}}{\pgfqpoint{8.432851in}{1.518124in}}%
\pgfpathlineto{\pgfqpoint{8.432851in}{1.518124in}}%
\pgfpathclose%
\pgfusepath{stroke}%
\end{pgfscope}%
\begin{pgfscope}%
\pgfpathrectangle{\pgfqpoint{7.512535in}{0.437222in}}{\pgfqpoint{6.275590in}{5.159444in}}%
\pgfusepath{clip}%
\pgfsetbuttcap%
\pgfsetroundjoin%
\pgfsetlinewidth{1.003750pt}%
\definecolor{currentstroke}{rgb}{0.827451,0.827451,0.827451}%
\pgfsetstrokecolor{currentstroke}%
\pgfsetstrokeopacity{0.800000}%
\pgfsetdash{}{0pt}%
\pgfpathmoveto{\pgfqpoint{10.712158in}{4.601092in}}%
\pgfpathcurveto{\pgfqpoint{10.723208in}{4.601092in}}{\pgfqpoint{10.733807in}{4.605482in}}{\pgfqpoint{10.741621in}{4.613296in}}%
\pgfpathcurveto{\pgfqpoint{10.749435in}{4.621109in}}{\pgfqpoint{10.753825in}{4.631708in}}{\pgfqpoint{10.753825in}{4.642759in}}%
\pgfpathcurveto{\pgfqpoint{10.753825in}{4.653809in}}{\pgfqpoint{10.749435in}{4.664408in}}{\pgfqpoint{10.741621in}{4.672221in}}%
\pgfpathcurveto{\pgfqpoint{10.733807in}{4.680035in}}{\pgfqpoint{10.723208in}{4.684425in}}{\pgfqpoint{10.712158in}{4.684425in}}%
\pgfpathcurveto{\pgfqpoint{10.701108in}{4.684425in}}{\pgfqpoint{10.690509in}{4.680035in}}{\pgfqpoint{10.682695in}{4.672221in}}%
\pgfpathcurveto{\pgfqpoint{10.674882in}{4.664408in}}{\pgfqpoint{10.670492in}{4.653809in}}{\pgfqpoint{10.670492in}{4.642759in}}%
\pgfpathcurveto{\pgfqpoint{10.670492in}{4.631708in}}{\pgfqpoint{10.674882in}{4.621109in}}{\pgfqpoint{10.682695in}{4.613296in}}%
\pgfpathcurveto{\pgfqpoint{10.690509in}{4.605482in}}{\pgfqpoint{10.701108in}{4.601092in}}{\pgfqpoint{10.712158in}{4.601092in}}%
\pgfpathlineto{\pgfqpoint{10.712158in}{4.601092in}}%
\pgfpathclose%
\pgfusepath{stroke}%
\end{pgfscope}%
\begin{pgfscope}%
\pgfpathrectangle{\pgfqpoint{7.512535in}{0.437222in}}{\pgfqpoint{6.275590in}{5.159444in}}%
\pgfusepath{clip}%
\pgfsetbuttcap%
\pgfsetroundjoin%
\pgfsetlinewidth{1.003750pt}%
\definecolor{currentstroke}{rgb}{0.827451,0.827451,0.827451}%
\pgfsetstrokecolor{currentstroke}%
\pgfsetstrokeopacity{0.800000}%
\pgfsetdash{}{0pt}%
\pgfpathmoveto{\pgfqpoint{8.724795in}{1.722485in}}%
\pgfpathcurveto{\pgfqpoint{8.735845in}{1.722485in}}{\pgfqpoint{8.746444in}{1.726876in}}{\pgfqpoint{8.754258in}{1.734689in}}%
\pgfpathcurveto{\pgfqpoint{8.762071in}{1.742503in}}{\pgfqpoint{8.766461in}{1.753102in}}{\pgfqpoint{8.766461in}{1.764152in}}%
\pgfpathcurveto{\pgfqpoint{8.766461in}{1.775202in}}{\pgfqpoint{8.762071in}{1.785801in}}{\pgfqpoint{8.754258in}{1.793615in}}%
\pgfpathcurveto{\pgfqpoint{8.746444in}{1.801428in}}{\pgfqpoint{8.735845in}{1.805819in}}{\pgfqpoint{8.724795in}{1.805819in}}%
\pgfpathcurveto{\pgfqpoint{8.713745in}{1.805819in}}{\pgfqpoint{8.703146in}{1.801428in}}{\pgfqpoint{8.695332in}{1.793615in}}%
\pgfpathcurveto{\pgfqpoint{8.687518in}{1.785801in}}{\pgfqpoint{8.683128in}{1.775202in}}{\pgfqpoint{8.683128in}{1.764152in}}%
\pgfpathcurveto{\pgfqpoint{8.683128in}{1.753102in}}{\pgfqpoint{8.687518in}{1.742503in}}{\pgfqpoint{8.695332in}{1.734689in}}%
\pgfpathcurveto{\pgfqpoint{8.703146in}{1.726876in}}{\pgfqpoint{8.713745in}{1.722485in}}{\pgfqpoint{8.724795in}{1.722485in}}%
\pgfpathlineto{\pgfqpoint{8.724795in}{1.722485in}}%
\pgfpathclose%
\pgfusepath{stroke}%
\end{pgfscope}%
\begin{pgfscope}%
\pgfpathrectangle{\pgfqpoint{7.512535in}{0.437222in}}{\pgfqpoint{6.275590in}{5.159444in}}%
\pgfusepath{clip}%
\pgfsetbuttcap%
\pgfsetroundjoin%
\pgfsetlinewidth{1.003750pt}%
\definecolor{currentstroke}{rgb}{0.827451,0.827451,0.827451}%
\pgfsetstrokecolor{currentstroke}%
\pgfsetstrokeopacity{0.800000}%
\pgfsetdash{}{0pt}%
\pgfpathmoveto{\pgfqpoint{8.778983in}{2.422402in}}%
\pgfpathcurveto{\pgfqpoint{8.790033in}{2.422402in}}{\pgfqpoint{8.800632in}{2.426793in}}{\pgfqpoint{8.808446in}{2.434606in}}%
\pgfpathcurveto{\pgfqpoint{8.816259in}{2.442420in}}{\pgfqpoint{8.820650in}{2.453019in}}{\pgfqpoint{8.820650in}{2.464069in}}%
\pgfpathcurveto{\pgfqpoint{8.820650in}{2.475119in}}{\pgfqpoint{8.816259in}{2.485718in}}{\pgfqpoint{8.808446in}{2.493532in}}%
\pgfpathcurveto{\pgfqpoint{8.800632in}{2.501345in}}{\pgfqpoint{8.790033in}{2.505736in}}{\pgfqpoint{8.778983in}{2.505736in}}%
\pgfpathcurveto{\pgfqpoint{8.767933in}{2.505736in}}{\pgfqpoint{8.757334in}{2.501345in}}{\pgfqpoint{8.749520in}{2.493532in}}%
\pgfpathcurveto{\pgfqpoint{8.741707in}{2.485718in}}{\pgfqpoint{8.737316in}{2.475119in}}{\pgfqpoint{8.737316in}{2.464069in}}%
\pgfpathcurveto{\pgfqpoint{8.737316in}{2.453019in}}{\pgfqpoint{8.741707in}{2.442420in}}{\pgfqpoint{8.749520in}{2.434606in}}%
\pgfpathcurveto{\pgfqpoint{8.757334in}{2.426793in}}{\pgfqpoint{8.767933in}{2.422402in}}{\pgfqpoint{8.778983in}{2.422402in}}%
\pgfpathlineto{\pgfqpoint{8.778983in}{2.422402in}}%
\pgfpathclose%
\pgfusepath{stroke}%
\end{pgfscope}%
\begin{pgfscope}%
\pgfpathrectangle{\pgfqpoint{7.512535in}{0.437222in}}{\pgfqpoint{6.275590in}{5.159444in}}%
\pgfusepath{clip}%
\pgfsetbuttcap%
\pgfsetroundjoin%
\pgfsetlinewidth{1.003750pt}%
\definecolor{currentstroke}{rgb}{0.827451,0.827451,0.827451}%
\pgfsetstrokecolor{currentstroke}%
\pgfsetstrokeopacity{0.800000}%
\pgfsetdash{}{0pt}%
\pgfpathmoveto{\pgfqpoint{10.045306in}{4.354371in}}%
\pgfpathcurveto{\pgfqpoint{10.056356in}{4.354371in}}{\pgfqpoint{10.066955in}{4.358761in}}{\pgfqpoint{10.074768in}{4.366575in}}%
\pgfpathcurveto{\pgfqpoint{10.082582in}{4.374389in}}{\pgfqpoint{10.086972in}{4.384988in}}{\pgfqpoint{10.086972in}{4.396038in}}%
\pgfpathcurveto{\pgfqpoint{10.086972in}{4.407088in}}{\pgfqpoint{10.082582in}{4.417687in}}{\pgfqpoint{10.074768in}{4.425500in}}%
\pgfpathcurveto{\pgfqpoint{10.066955in}{4.433314in}}{\pgfqpoint{10.056356in}{4.437704in}}{\pgfqpoint{10.045306in}{4.437704in}}%
\pgfpathcurveto{\pgfqpoint{10.034256in}{4.437704in}}{\pgfqpoint{10.023657in}{4.433314in}}{\pgfqpoint{10.015843in}{4.425500in}}%
\pgfpathcurveto{\pgfqpoint{10.008029in}{4.417687in}}{\pgfqpoint{10.003639in}{4.407088in}}{\pgfqpoint{10.003639in}{4.396038in}}%
\pgfpathcurveto{\pgfqpoint{10.003639in}{4.384988in}}{\pgfqpoint{10.008029in}{4.374389in}}{\pgfqpoint{10.015843in}{4.366575in}}%
\pgfpathcurveto{\pgfqpoint{10.023657in}{4.358761in}}{\pgfqpoint{10.034256in}{4.354371in}}{\pgfqpoint{10.045306in}{4.354371in}}%
\pgfpathlineto{\pgfqpoint{10.045306in}{4.354371in}}%
\pgfpathclose%
\pgfusepath{stroke}%
\end{pgfscope}%
\begin{pgfscope}%
\pgfpathrectangle{\pgfqpoint{7.512535in}{0.437222in}}{\pgfqpoint{6.275590in}{5.159444in}}%
\pgfusepath{clip}%
\pgfsetbuttcap%
\pgfsetroundjoin%
\pgfsetlinewidth{1.003750pt}%
\definecolor{currentstroke}{rgb}{0.827451,0.827451,0.827451}%
\pgfsetstrokecolor{currentstroke}%
\pgfsetstrokeopacity{0.800000}%
\pgfsetdash{}{0pt}%
\pgfpathmoveto{\pgfqpoint{10.156793in}{3.250664in}}%
\pgfpathcurveto{\pgfqpoint{10.167843in}{3.250664in}}{\pgfqpoint{10.178442in}{3.255054in}}{\pgfqpoint{10.186256in}{3.262868in}}%
\pgfpathcurveto{\pgfqpoint{10.194069in}{3.270682in}}{\pgfqpoint{10.198460in}{3.281281in}}{\pgfqpoint{10.198460in}{3.292331in}}%
\pgfpathcurveto{\pgfqpoint{10.198460in}{3.303381in}}{\pgfqpoint{10.194069in}{3.313980in}}{\pgfqpoint{10.186256in}{3.321794in}}%
\pgfpathcurveto{\pgfqpoint{10.178442in}{3.329607in}}{\pgfqpoint{10.167843in}{3.333998in}}{\pgfqpoint{10.156793in}{3.333998in}}%
\pgfpathcurveto{\pgfqpoint{10.145743in}{3.333998in}}{\pgfqpoint{10.135144in}{3.329607in}}{\pgfqpoint{10.127330in}{3.321794in}}%
\pgfpathcurveto{\pgfqpoint{10.119517in}{3.313980in}}{\pgfqpoint{10.115126in}{3.303381in}}{\pgfqpoint{10.115126in}{3.292331in}}%
\pgfpathcurveto{\pgfqpoint{10.115126in}{3.281281in}}{\pgfqpoint{10.119517in}{3.270682in}}{\pgfqpoint{10.127330in}{3.262868in}}%
\pgfpathcurveto{\pgfqpoint{10.135144in}{3.255054in}}{\pgfqpoint{10.145743in}{3.250664in}}{\pgfqpoint{10.156793in}{3.250664in}}%
\pgfpathlineto{\pgfqpoint{10.156793in}{3.250664in}}%
\pgfpathclose%
\pgfusepath{stroke}%
\end{pgfscope}%
\begin{pgfscope}%
\pgfpathrectangle{\pgfqpoint{7.512535in}{0.437222in}}{\pgfqpoint{6.275590in}{5.159444in}}%
\pgfusepath{clip}%
\pgfsetbuttcap%
\pgfsetroundjoin%
\pgfsetlinewidth{1.003750pt}%
\definecolor{currentstroke}{rgb}{0.827451,0.827451,0.827451}%
\pgfsetstrokecolor{currentstroke}%
\pgfsetstrokeopacity{0.800000}%
\pgfsetdash{}{0pt}%
\pgfpathmoveto{\pgfqpoint{9.171368in}{3.241859in}}%
\pgfpathcurveto{\pgfqpoint{9.182418in}{3.241859in}}{\pgfqpoint{9.193017in}{3.246249in}}{\pgfqpoint{9.200830in}{3.254063in}}%
\pgfpathcurveto{\pgfqpoint{9.208644in}{3.261876in}}{\pgfqpoint{9.213034in}{3.272475in}}{\pgfqpoint{9.213034in}{3.283525in}}%
\pgfpathcurveto{\pgfqpoint{9.213034in}{3.294576in}}{\pgfqpoint{9.208644in}{3.305175in}}{\pgfqpoint{9.200830in}{3.312988in}}%
\pgfpathcurveto{\pgfqpoint{9.193017in}{3.320802in}}{\pgfqpoint{9.182418in}{3.325192in}}{\pgfqpoint{9.171368in}{3.325192in}}%
\pgfpathcurveto{\pgfqpoint{9.160317in}{3.325192in}}{\pgfqpoint{9.149718in}{3.320802in}}{\pgfqpoint{9.141905in}{3.312988in}}%
\pgfpathcurveto{\pgfqpoint{9.134091in}{3.305175in}}{\pgfqpoint{9.129701in}{3.294576in}}{\pgfqpoint{9.129701in}{3.283525in}}%
\pgfpathcurveto{\pgfqpoint{9.129701in}{3.272475in}}{\pgfqpoint{9.134091in}{3.261876in}}{\pgfqpoint{9.141905in}{3.254063in}}%
\pgfpathcurveto{\pgfqpoint{9.149718in}{3.246249in}}{\pgfqpoint{9.160317in}{3.241859in}}{\pgfqpoint{9.171368in}{3.241859in}}%
\pgfpathlineto{\pgfqpoint{9.171368in}{3.241859in}}%
\pgfpathclose%
\pgfusepath{stroke}%
\end{pgfscope}%
\begin{pgfscope}%
\pgfpathrectangle{\pgfqpoint{7.512535in}{0.437222in}}{\pgfqpoint{6.275590in}{5.159444in}}%
\pgfusepath{clip}%
\pgfsetbuttcap%
\pgfsetroundjoin%
\pgfsetlinewidth{1.003750pt}%
\definecolor{currentstroke}{rgb}{0.827451,0.827451,0.827451}%
\pgfsetstrokecolor{currentstroke}%
\pgfsetstrokeopacity{0.800000}%
\pgfsetdash{}{0pt}%
\pgfpathmoveto{\pgfqpoint{12.650243in}{5.481847in}}%
\pgfpathcurveto{\pgfqpoint{12.661293in}{5.481847in}}{\pgfqpoint{12.671892in}{5.486238in}}{\pgfqpoint{12.679706in}{5.494051in}}%
\pgfpathcurveto{\pgfqpoint{12.687519in}{5.501865in}}{\pgfqpoint{12.691909in}{5.512464in}}{\pgfqpoint{12.691909in}{5.523514in}}%
\pgfpathcurveto{\pgfqpoint{12.691909in}{5.534564in}}{\pgfqpoint{12.687519in}{5.545163in}}{\pgfqpoint{12.679706in}{5.552977in}}%
\pgfpathcurveto{\pgfqpoint{12.671892in}{5.560791in}}{\pgfqpoint{12.661293in}{5.565181in}}{\pgfqpoint{12.650243in}{5.565181in}}%
\pgfpathcurveto{\pgfqpoint{12.639193in}{5.565181in}}{\pgfqpoint{12.628594in}{5.560791in}}{\pgfqpoint{12.620780in}{5.552977in}}%
\pgfpathcurveto{\pgfqpoint{12.612966in}{5.545163in}}{\pgfqpoint{12.608576in}{5.534564in}}{\pgfqpoint{12.608576in}{5.523514in}}%
\pgfpathcurveto{\pgfqpoint{12.608576in}{5.512464in}}{\pgfqpoint{12.612966in}{5.501865in}}{\pgfqpoint{12.620780in}{5.494051in}}%
\pgfpathcurveto{\pgfqpoint{12.628594in}{5.486238in}}{\pgfqpoint{12.639193in}{5.481847in}}{\pgfqpoint{12.650243in}{5.481847in}}%
\pgfpathlineto{\pgfqpoint{12.650243in}{5.481847in}}%
\pgfpathclose%
\pgfusepath{stroke}%
\end{pgfscope}%
\begin{pgfscope}%
\pgfpathrectangle{\pgfqpoint{7.512535in}{0.437222in}}{\pgfqpoint{6.275590in}{5.159444in}}%
\pgfusepath{clip}%
\pgfsetbuttcap%
\pgfsetroundjoin%
\pgfsetlinewidth{1.003750pt}%
\definecolor{currentstroke}{rgb}{0.827451,0.827451,0.827451}%
\pgfsetstrokecolor{currentstroke}%
\pgfsetstrokeopacity{0.800000}%
\pgfsetdash{}{0pt}%
\pgfpathmoveto{\pgfqpoint{8.775643in}{1.745625in}}%
\pgfpathcurveto{\pgfqpoint{8.786693in}{1.745625in}}{\pgfqpoint{8.797292in}{1.750015in}}{\pgfqpoint{8.805106in}{1.757829in}}%
\pgfpathcurveto{\pgfqpoint{8.812919in}{1.765643in}}{\pgfqpoint{8.817310in}{1.776242in}}{\pgfqpoint{8.817310in}{1.787292in}}%
\pgfpathcurveto{\pgfqpoint{8.817310in}{1.798342in}}{\pgfqpoint{8.812919in}{1.808941in}}{\pgfqpoint{8.805106in}{1.816755in}}%
\pgfpathcurveto{\pgfqpoint{8.797292in}{1.824568in}}{\pgfqpoint{8.786693in}{1.828959in}}{\pgfqpoint{8.775643in}{1.828959in}}%
\pgfpathcurveto{\pgfqpoint{8.764593in}{1.828959in}}{\pgfqpoint{8.753994in}{1.824568in}}{\pgfqpoint{8.746180in}{1.816755in}}%
\pgfpathcurveto{\pgfqpoint{8.738367in}{1.808941in}}{\pgfqpoint{8.733976in}{1.798342in}}{\pgfqpoint{8.733976in}{1.787292in}}%
\pgfpathcurveto{\pgfqpoint{8.733976in}{1.776242in}}{\pgfqpoint{8.738367in}{1.765643in}}{\pgfqpoint{8.746180in}{1.757829in}}%
\pgfpathcurveto{\pgfqpoint{8.753994in}{1.750015in}}{\pgfqpoint{8.764593in}{1.745625in}}{\pgfqpoint{8.775643in}{1.745625in}}%
\pgfpathlineto{\pgfqpoint{8.775643in}{1.745625in}}%
\pgfpathclose%
\pgfusepath{stroke}%
\end{pgfscope}%
\begin{pgfscope}%
\pgfpathrectangle{\pgfqpoint{7.512535in}{0.437222in}}{\pgfqpoint{6.275590in}{5.159444in}}%
\pgfusepath{clip}%
\pgfsetbuttcap%
\pgfsetroundjoin%
\pgfsetlinewidth{1.003750pt}%
\definecolor{currentstroke}{rgb}{0.827451,0.827451,0.827451}%
\pgfsetstrokecolor{currentstroke}%
\pgfsetstrokeopacity{0.800000}%
\pgfsetdash{}{0pt}%
\pgfpathmoveto{\pgfqpoint{8.800029in}{1.838807in}}%
\pgfpathcurveto{\pgfqpoint{8.811079in}{1.838807in}}{\pgfqpoint{8.821678in}{1.843198in}}{\pgfqpoint{8.829491in}{1.851011in}}%
\pgfpathcurveto{\pgfqpoint{8.837305in}{1.858825in}}{\pgfqpoint{8.841695in}{1.869424in}}{\pgfqpoint{8.841695in}{1.880474in}}%
\pgfpathcurveto{\pgfqpoint{8.841695in}{1.891524in}}{\pgfqpoint{8.837305in}{1.902123in}}{\pgfqpoint{8.829491in}{1.909937in}}%
\pgfpathcurveto{\pgfqpoint{8.821678in}{1.917750in}}{\pgfqpoint{8.811079in}{1.922141in}}{\pgfqpoint{8.800029in}{1.922141in}}%
\pgfpathcurveto{\pgfqpoint{8.788978in}{1.922141in}}{\pgfqpoint{8.778379in}{1.917750in}}{\pgfqpoint{8.770566in}{1.909937in}}%
\pgfpathcurveto{\pgfqpoint{8.762752in}{1.902123in}}{\pgfqpoint{8.758362in}{1.891524in}}{\pgfqpoint{8.758362in}{1.880474in}}%
\pgfpathcurveto{\pgfqpoint{8.758362in}{1.869424in}}{\pgfqpoint{8.762752in}{1.858825in}}{\pgfqpoint{8.770566in}{1.851011in}}%
\pgfpathcurveto{\pgfqpoint{8.778379in}{1.843198in}}{\pgfqpoint{8.788978in}{1.838807in}}{\pgfqpoint{8.800029in}{1.838807in}}%
\pgfpathlineto{\pgfqpoint{8.800029in}{1.838807in}}%
\pgfpathclose%
\pgfusepath{stroke}%
\end{pgfscope}%
\begin{pgfscope}%
\pgfpathrectangle{\pgfqpoint{7.512535in}{0.437222in}}{\pgfqpoint{6.275590in}{5.159444in}}%
\pgfusepath{clip}%
\pgfsetbuttcap%
\pgfsetroundjoin%
\pgfsetlinewidth{1.003750pt}%
\definecolor{currentstroke}{rgb}{0.827451,0.827451,0.827451}%
\pgfsetstrokecolor{currentstroke}%
\pgfsetstrokeopacity{0.800000}%
\pgfsetdash{}{0pt}%
\pgfpathmoveto{\pgfqpoint{12.384977in}{5.553091in}}%
\pgfpathcurveto{\pgfqpoint{12.396028in}{5.553091in}}{\pgfqpoint{12.406627in}{5.557481in}}{\pgfqpoint{12.414440in}{5.565295in}}%
\pgfpathcurveto{\pgfqpoint{12.422254in}{5.573108in}}{\pgfqpoint{12.426644in}{5.583707in}}{\pgfqpoint{12.426644in}{5.594757in}}%
\pgfpathcurveto{\pgfqpoint{12.426644in}{5.605808in}}{\pgfqpoint{12.422254in}{5.616407in}}{\pgfqpoint{12.414440in}{5.624220in}}%
\pgfpathcurveto{\pgfqpoint{12.406627in}{5.632034in}}{\pgfqpoint{12.396028in}{5.636424in}}{\pgfqpoint{12.384977in}{5.636424in}}%
\pgfpathcurveto{\pgfqpoint{12.373927in}{5.636424in}}{\pgfqpoint{12.363328in}{5.632034in}}{\pgfqpoint{12.355515in}{5.624220in}}%
\pgfpathcurveto{\pgfqpoint{12.347701in}{5.616407in}}{\pgfqpoint{12.343311in}{5.605808in}}{\pgfqpoint{12.343311in}{5.594757in}}%
\pgfpathcurveto{\pgfqpoint{12.343311in}{5.583707in}}{\pgfqpoint{12.347701in}{5.573108in}}{\pgfqpoint{12.355515in}{5.565295in}}%
\pgfpathcurveto{\pgfqpoint{12.363328in}{5.557481in}}{\pgfqpoint{12.373927in}{5.553091in}}{\pgfqpoint{12.384977in}{5.553091in}}%
\pgfpathlineto{\pgfqpoint{12.384977in}{5.553091in}}%
\pgfpathclose%
\pgfusepath{stroke}%
\end{pgfscope}%
\begin{pgfscope}%
\pgfpathrectangle{\pgfqpoint{7.512535in}{0.437222in}}{\pgfqpoint{6.275590in}{5.159444in}}%
\pgfusepath{clip}%
\pgfsetbuttcap%
\pgfsetroundjoin%
\pgfsetlinewidth{1.003750pt}%
\definecolor{currentstroke}{rgb}{0.827451,0.827451,0.827451}%
\pgfsetstrokecolor{currentstroke}%
\pgfsetstrokeopacity{0.800000}%
\pgfsetdash{}{0pt}%
\pgfpathmoveto{\pgfqpoint{10.486066in}{4.475908in}}%
\pgfpathcurveto{\pgfqpoint{10.497116in}{4.475908in}}{\pgfqpoint{10.507715in}{4.480298in}}{\pgfqpoint{10.515529in}{4.488112in}}%
\pgfpathcurveto{\pgfqpoint{10.523342in}{4.495926in}}{\pgfqpoint{10.527733in}{4.506525in}}{\pgfqpoint{10.527733in}{4.517575in}}%
\pgfpathcurveto{\pgfqpoint{10.527733in}{4.528625in}}{\pgfqpoint{10.523342in}{4.539224in}}{\pgfqpoint{10.515529in}{4.547038in}}%
\pgfpathcurveto{\pgfqpoint{10.507715in}{4.554851in}}{\pgfqpoint{10.497116in}{4.559241in}}{\pgfqpoint{10.486066in}{4.559241in}}%
\pgfpathcurveto{\pgfqpoint{10.475016in}{4.559241in}}{\pgfqpoint{10.464417in}{4.554851in}}{\pgfqpoint{10.456603in}{4.547038in}}%
\pgfpathcurveto{\pgfqpoint{10.448790in}{4.539224in}}{\pgfqpoint{10.444399in}{4.528625in}}{\pgfqpoint{10.444399in}{4.517575in}}%
\pgfpathcurveto{\pgfqpoint{10.444399in}{4.506525in}}{\pgfqpoint{10.448790in}{4.495926in}}{\pgfqpoint{10.456603in}{4.488112in}}%
\pgfpathcurveto{\pgfqpoint{10.464417in}{4.480298in}}{\pgfqpoint{10.475016in}{4.475908in}}{\pgfqpoint{10.486066in}{4.475908in}}%
\pgfpathlineto{\pgfqpoint{10.486066in}{4.475908in}}%
\pgfpathclose%
\pgfusepath{stroke}%
\end{pgfscope}%
\begin{pgfscope}%
\pgfpathrectangle{\pgfqpoint{7.512535in}{0.437222in}}{\pgfqpoint{6.275590in}{5.159444in}}%
\pgfusepath{clip}%
\pgfsetbuttcap%
\pgfsetroundjoin%
\pgfsetlinewidth{1.003750pt}%
\definecolor{currentstroke}{rgb}{0.827451,0.827451,0.827451}%
\pgfsetstrokecolor{currentstroke}%
\pgfsetstrokeopacity{0.800000}%
\pgfsetdash{}{0pt}%
\pgfpathmoveto{\pgfqpoint{8.044541in}{2.113663in}}%
\pgfpathcurveto{\pgfqpoint{8.055591in}{2.113663in}}{\pgfqpoint{8.066190in}{2.118054in}}{\pgfqpoint{8.074003in}{2.125867in}}%
\pgfpathcurveto{\pgfqpoint{8.081817in}{2.133681in}}{\pgfqpoint{8.086207in}{2.144280in}}{\pgfqpoint{8.086207in}{2.155330in}}%
\pgfpathcurveto{\pgfqpoint{8.086207in}{2.166380in}}{\pgfqpoint{8.081817in}{2.176979in}}{\pgfqpoint{8.074003in}{2.184793in}}%
\pgfpathcurveto{\pgfqpoint{8.066190in}{2.192606in}}{\pgfqpoint{8.055591in}{2.196997in}}{\pgfqpoint{8.044541in}{2.196997in}}%
\pgfpathcurveto{\pgfqpoint{8.033491in}{2.196997in}}{\pgfqpoint{8.022891in}{2.192606in}}{\pgfqpoint{8.015078in}{2.184793in}}%
\pgfpathcurveto{\pgfqpoint{8.007264in}{2.176979in}}{\pgfqpoint{8.002874in}{2.166380in}}{\pgfqpoint{8.002874in}{2.155330in}}%
\pgfpathcurveto{\pgfqpoint{8.002874in}{2.144280in}}{\pgfqpoint{8.007264in}{2.133681in}}{\pgfqpoint{8.015078in}{2.125867in}}%
\pgfpathcurveto{\pgfqpoint{8.022891in}{2.118054in}}{\pgfqpoint{8.033491in}{2.113663in}}{\pgfqpoint{8.044541in}{2.113663in}}%
\pgfpathlineto{\pgfqpoint{8.044541in}{2.113663in}}%
\pgfpathclose%
\pgfusepath{stroke}%
\end{pgfscope}%
\begin{pgfscope}%
\pgfpathrectangle{\pgfqpoint{7.512535in}{0.437222in}}{\pgfqpoint{6.275590in}{5.159444in}}%
\pgfusepath{clip}%
\pgfsetbuttcap%
\pgfsetroundjoin%
\pgfsetlinewidth{1.003750pt}%
\definecolor{currentstroke}{rgb}{0.827451,0.827451,0.827451}%
\pgfsetstrokecolor{currentstroke}%
\pgfsetstrokeopacity{0.800000}%
\pgfsetdash{}{0pt}%
\pgfpathmoveto{\pgfqpoint{8.624162in}{1.722485in}}%
\pgfpathcurveto{\pgfqpoint{8.635212in}{1.722485in}}{\pgfqpoint{8.645811in}{1.726876in}}{\pgfqpoint{8.653624in}{1.734689in}}%
\pgfpathcurveto{\pgfqpoint{8.661438in}{1.742503in}}{\pgfqpoint{8.665828in}{1.753102in}}{\pgfqpoint{8.665828in}{1.764152in}}%
\pgfpathcurveto{\pgfqpoint{8.665828in}{1.775202in}}{\pgfqpoint{8.661438in}{1.785801in}}{\pgfqpoint{8.653624in}{1.793615in}}%
\pgfpathcurveto{\pgfqpoint{8.645811in}{1.801428in}}{\pgfqpoint{8.635212in}{1.805819in}}{\pgfqpoint{8.624162in}{1.805819in}}%
\pgfpathcurveto{\pgfqpoint{8.613112in}{1.805819in}}{\pgfqpoint{8.602513in}{1.801428in}}{\pgfqpoint{8.594699in}{1.793615in}}%
\pgfpathcurveto{\pgfqpoint{8.586885in}{1.785801in}}{\pgfqpoint{8.582495in}{1.775202in}}{\pgfqpoint{8.582495in}{1.764152in}}%
\pgfpathcurveto{\pgfqpoint{8.582495in}{1.753102in}}{\pgfqpoint{8.586885in}{1.742503in}}{\pgfqpoint{8.594699in}{1.734689in}}%
\pgfpathcurveto{\pgfqpoint{8.602513in}{1.726876in}}{\pgfqpoint{8.613112in}{1.722485in}}{\pgfqpoint{8.624162in}{1.722485in}}%
\pgfpathlineto{\pgfqpoint{8.624162in}{1.722485in}}%
\pgfpathclose%
\pgfusepath{stroke}%
\end{pgfscope}%
\begin{pgfscope}%
\pgfpathrectangle{\pgfqpoint{7.512535in}{0.437222in}}{\pgfqpoint{6.275590in}{5.159444in}}%
\pgfusepath{clip}%
\pgfsetbuttcap%
\pgfsetroundjoin%
\pgfsetlinewidth{1.003750pt}%
\definecolor{currentstroke}{rgb}{0.827451,0.827451,0.827451}%
\pgfsetstrokecolor{currentstroke}%
\pgfsetstrokeopacity{0.800000}%
\pgfsetdash{}{0pt}%
\pgfpathmoveto{\pgfqpoint{12.873966in}{5.431219in}}%
\pgfpathcurveto{\pgfqpoint{12.885016in}{5.431219in}}{\pgfqpoint{12.895615in}{5.435610in}}{\pgfqpoint{12.903429in}{5.443423in}}%
\pgfpathcurveto{\pgfqpoint{12.911243in}{5.451237in}}{\pgfqpoint{12.915633in}{5.461836in}}{\pgfqpoint{12.915633in}{5.472886in}}%
\pgfpathcurveto{\pgfqpoint{12.915633in}{5.483936in}}{\pgfqpoint{12.911243in}{5.494535in}}{\pgfqpoint{12.903429in}{5.502349in}}%
\pgfpathcurveto{\pgfqpoint{12.895615in}{5.510163in}}{\pgfqpoint{12.885016in}{5.514553in}}{\pgfqpoint{12.873966in}{5.514553in}}%
\pgfpathcurveto{\pgfqpoint{12.862916in}{5.514553in}}{\pgfqpoint{12.852317in}{5.510163in}}{\pgfqpoint{12.844503in}{5.502349in}}%
\pgfpathcurveto{\pgfqpoint{12.836690in}{5.494535in}}{\pgfqpoint{12.832300in}{5.483936in}}{\pgfqpoint{12.832300in}{5.472886in}}%
\pgfpathcurveto{\pgfqpoint{12.832300in}{5.461836in}}{\pgfqpoint{12.836690in}{5.451237in}}{\pgfqpoint{12.844503in}{5.443423in}}%
\pgfpathcurveto{\pgfqpoint{12.852317in}{5.435610in}}{\pgfqpoint{12.862916in}{5.431219in}}{\pgfqpoint{12.873966in}{5.431219in}}%
\pgfpathlineto{\pgfqpoint{12.873966in}{5.431219in}}%
\pgfpathclose%
\pgfusepath{stroke}%
\end{pgfscope}%
\begin{pgfscope}%
\pgfpathrectangle{\pgfqpoint{7.512535in}{0.437222in}}{\pgfqpoint{6.275590in}{5.159444in}}%
\pgfusepath{clip}%
\pgfsetbuttcap%
\pgfsetroundjoin%
\pgfsetlinewidth{1.003750pt}%
\definecolor{currentstroke}{rgb}{0.827451,0.827451,0.827451}%
\pgfsetstrokecolor{currentstroke}%
\pgfsetstrokeopacity{0.800000}%
\pgfsetdash{}{0pt}%
\pgfpathmoveto{\pgfqpoint{9.590173in}{3.398096in}}%
\pgfpathcurveto{\pgfqpoint{9.601223in}{3.398096in}}{\pgfqpoint{9.611822in}{3.402486in}}{\pgfqpoint{9.619635in}{3.410300in}}%
\pgfpathcurveto{\pgfqpoint{9.627449in}{3.418114in}}{\pgfqpoint{9.631839in}{3.428713in}}{\pgfqpoint{9.631839in}{3.439763in}}%
\pgfpathcurveto{\pgfqpoint{9.631839in}{3.450813in}}{\pgfqpoint{9.627449in}{3.461412in}}{\pgfqpoint{9.619635in}{3.469226in}}%
\pgfpathcurveto{\pgfqpoint{9.611822in}{3.477039in}}{\pgfqpoint{9.601223in}{3.481429in}}{\pgfqpoint{9.590173in}{3.481429in}}%
\pgfpathcurveto{\pgfqpoint{9.579122in}{3.481429in}}{\pgfqpoint{9.568523in}{3.477039in}}{\pgfqpoint{9.560710in}{3.469226in}}%
\pgfpathcurveto{\pgfqpoint{9.552896in}{3.461412in}}{\pgfqpoint{9.548506in}{3.450813in}}{\pgfqpoint{9.548506in}{3.439763in}}%
\pgfpathcurveto{\pgfqpoint{9.548506in}{3.428713in}}{\pgfqpoint{9.552896in}{3.418114in}}{\pgfqpoint{9.560710in}{3.410300in}}%
\pgfpathcurveto{\pgfqpoint{9.568523in}{3.402486in}}{\pgfqpoint{9.579122in}{3.398096in}}{\pgfqpoint{9.590173in}{3.398096in}}%
\pgfpathlineto{\pgfqpoint{9.590173in}{3.398096in}}%
\pgfpathclose%
\pgfusepath{stroke}%
\end{pgfscope}%
\begin{pgfscope}%
\pgfpathrectangle{\pgfqpoint{7.512535in}{0.437222in}}{\pgfqpoint{6.275590in}{5.159444in}}%
\pgfusepath{clip}%
\pgfsetbuttcap%
\pgfsetroundjoin%
\pgfsetlinewidth{1.003750pt}%
\definecolor{currentstroke}{rgb}{0.827451,0.827451,0.827451}%
\pgfsetstrokecolor{currentstroke}%
\pgfsetstrokeopacity{0.800000}%
\pgfsetdash{}{0pt}%
\pgfpathmoveto{\pgfqpoint{10.140717in}{3.555415in}}%
\pgfpathcurveto{\pgfqpoint{10.151767in}{3.555415in}}{\pgfqpoint{10.162367in}{3.559805in}}{\pgfqpoint{10.170180in}{3.567618in}}%
\pgfpathcurveto{\pgfqpoint{10.177994in}{3.575432in}}{\pgfqpoint{10.182384in}{3.586031in}}{\pgfqpoint{10.182384in}{3.597081in}}%
\pgfpathcurveto{\pgfqpoint{10.182384in}{3.608131in}}{\pgfqpoint{10.177994in}{3.618730in}}{\pgfqpoint{10.170180in}{3.626544in}}%
\pgfpathcurveto{\pgfqpoint{10.162367in}{3.634358in}}{\pgfqpoint{10.151767in}{3.638748in}}{\pgfqpoint{10.140717in}{3.638748in}}%
\pgfpathcurveto{\pgfqpoint{10.129667in}{3.638748in}}{\pgfqpoint{10.119068in}{3.634358in}}{\pgfqpoint{10.111255in}{3.626544in}}%
\pgfpathcurveto{\pgfqpoint{10.103441in}{3.618730in}}{\pgfqpoint{10.099051in}{3.608131in}}{\pgfqpoint{10.099051in}{3.597081in}}%
\pgfpathcurveto{\pgfqpoint{10.099051in}{3.586031in}}{\pgfqpoint{10.103441in}{3.575432in}}{\pgfqpoint{10.111255in}{3.567618in}}%
\pgfpathcurveto{\pgfqpoint{10.119068in}{3.559805in}}{\pgfqpoint{10.129667in}{3.555415in}}{\pgfqpoint{10.140717in}{3.555415in}}%
\pgfpathlineto{\pgfqpoint{10.140717in}{3.555415in}}%
\pgfpathclose%
\pgfusepath{stroke}%
\end{pgfscope}%
\begin{pgfscope}%
\pgfpathrectangle{\pgfqpoint{7.512535in}{0.437222in}}{\pgfqpoint{6.275590in}{5.159444in}}%
\pgfusepath{clip}%
\pgfsetbuttcap%
\pgfsetroundjoin%
\pgfsetlinewidth{1.003750pt}%
\definecolor{currentstroke}{rgb}{0.827451,0.827451,0.827451}%
\pgfsetstrokecolor{currentstroke}%
\pgfsetstrokeopacity{0.800000}%
\pgfsetdash{}{0pt}%
\pgfpathmoveto{\pgfqpoint{9.726466in}{3.345111in}}%
\pgfpathcurveto{\pgfqpoint{9.737516in}{3.345111in}}{\pgfqpoint{9.748115in}{3.349501in}}{\pgfqpoint{9.755929in}{3.357315in}}%
\pgfpathcurveto{\pgfqpoint{9.763743in}{3.365128in}}{\pgfqpoint{9.768133in}{3.375727in}}{\pgfqpoint{9.768133in}{3.386778in}}%
\pgfpathcurveto{\pgfqpoint{9.768133in}{3.397828in}}{\pgfqpoint{9.763743in}{3.408427in}}{\pgfqpoint{9.755929in}{3.416240in}}%
\pgfpathcurveto{\pgfqpoint{9.748115in}{3.424054in}}{\pgfqpoint{9.737516in}{3.428444in}}{\pgfqpoint{9.726466in}{3.428444in}}%
\pgfpathcurveto{\pgfqpoint{9.715416in}{3.428444in}}{\pgfqpoint{9.704817in}{3.424054in}}{\pgfqpoint{9.697004in}{3.416240in}}%
\pgfpathcurveto{\pgfqpoint{9.689190in}{3.408427in}}{\pgfqpoint{9.684800in}{3.397828in}}{\pgfqpoint{9.684800in}{3.386778in}}%
\pgfpathcurveto{\pgfqpoint{9.684800in}{3.375727in}}{\pgfqpoint{9.689190in}{3.365128in}}{\pgfqpoint{9.697004in}{3.357315in}}%
\pgfpathcurveto{\pgfqpoint{9.704817in}{3.349501in}}{\pgfqpoint{9.715416in}{3.345111in}}{\pgfqpoint{9.726466in}{3.345111in}}%
\pgfpathlineto{\pgfqpoint{9.726466in}{3.345111in}}%
\pgfpathclose%
\pgfusepath{stroke}%
\end{pgfscope}%
\begin{pgfscope}%
\pgfpathrectangle{\pgfqpoint{7.512535in}{0.437222in}}{\pgfqpoint{6.275590in}{5.159444in}}%
\pgfusepath{clip}%
\pgfsetbuttcap%
\pgfsetroundjoin%
\pgfsetlinewidth{1.003750pt}%
\definecolor{currentstroke}{rgb}{0.827451,0.827451,0.827451}%
\pgfsetstrokecolor{currentstroke}%
\pgfsetstrokeopacity{0.800000}%
\pgfsetdash{}{0pt}%
\pgfpathmoveto{\pgfqpoint{9.061862in}{3.235524in}}%
\pgfpathcurveto{\pgfqpoint{9.072912in}{3.235524in}}{\pgfqpoint{9.083511in}{3.239914in}}{\pgfqpoint{9.091325in}{3.247728in}}%
\pgfpathcurveto{\pgfqpoint{9.099138in}{3.255542in}}{\pgfqpoint{9.103529in}{3.266141in}}{\pgfqpoint{9.103529in}{3.277191in}}%
\pgfpathcurveto{\pgfqpoint{9.103529in}{3.288241in}}{\pgfqpoint{9.099138in}{3.298840in}}{\pgfqpoint{9.091325in}{3.306654in}}%
\pgfpathcurveto{\pgfqpoint{9.083511in}{3.314467in}}{\pgfqpoint{9.072912in}{3.318857in}}{\pgfqpoint{9.061862in}{3.318857in}}%
\pgfpathcurveto{\pgfqpoint{9.050812in}{3.318857in}}{\pgfqpoint{9.040213in}{3.314467in}}{\pgfqpoint{9.032399in}{3.306654in}}%
\pgfpathcurveto{\pgfqpoint{9.024586in}{3.298840in}}{\pgfqpoint{9.020195in}{3.288241in}}{\pgfqpoint{9.020195in}{3.277191in}}%
\pgfpathcurveto{\pgfqpoint{9.020195in}{3.266141in}}{\pgfqpoint{9.024586in}{3.255542in}}{\pgfqpoint{9.032399in}{3.247728in}}%
\pgfpathcurveto{\pgfqpoint{9.040213in}{3.239914in}}{\pgfqpoint{9.050812in}{3.235524in}}{\pgfqpoint{9.061862in}{3.235524in}}%
\pgfpathlineto{\pgfqpoint{9.061862in}{3.235524in}}%
\pgfpathclose%
\pgfusepath{stroke}%
\end{pgfscope}%
\begin{pgfscope}%
\pgfpathrectangle{\pgfqpoint{7.512535in}{0.437222in}}{\pgfqpoint{6.275590in}{5.159444in}}%
\pgfusepath{clip}%
\pgfsetbuttcap%
\pgfsetroundjoin%
\pgfsetlinewidth{1.003750pt}%
\definecolor{currentstroke}{rgb}{0.827451,0.827451,0.827451}%
\pgfsetstrokecolor{currentstroke}%
\pgfsetstrokeopacity{0.800000}%
\pgfsetdash{}{0pt}%
\pgfpathmoveto{\pgfqpoint{10.780511in}{5.205075in}}%
\pgfpathcurveto{\pgfqpoint{10.791561in}{5.205075in}}{\pgfqpoint{10.802160in}{5.209466in}}{\pgfqpoint{10.809973in}{5.217279in}}%
\pgfpathcurveto{\pgfqpoint{10.817787in}{5.225093in}}{\pgfqpoint{10.822177in}{5.235692in}}{\pgfqpoint{10.822177in}{5.246742in}}%
\pgfpathcurveto{\pgfqpoint{10.822177in}{5.257792in}}{\pgfqpoint{10.817787in}{5.268391in}}{\pgfqpoint{10.809973in}{5.276205in}}%
\pgfpathcurveto{\pgfqpoint{10.802160in}{5.284019in}}{\pgfqpoint{10.791561in}{5.288409in}}{\pgfqpoint{10.780511in}{5.288409in}}%
\pgfpathcurveto{\pgfqpoint{10.769461in}{5.288409in}}{\pgfqpoint{10.758862in}{5.284019in}}{\pgfqpoint{10.751048in}{5.276205in}}%
\pgfpathcurveto{\pgfqpoint{10.743234in}{5.268391in}}{\pgfqpoint{10.738844in}{5.257792in}}{\pgfqpoint{10.738844in}{5.246742in}}%
\pgfpathcurveto{\pgfqpoint{10.738844in}{5.235692in}}{\pgfqpoint{10.743234in}{5.225093in}}{\pgfqpoint{10.751048in}{5.217279in}}%
\pgfpathcurveto{\pgfqpoint{10.758862in}{5.209466in}}{\pgfqpoint{10.769461in}{5.205075in}}{\pgfqpoint{10.780511in}{5.205075in}}%
\pgfpathlineto{\pgfqpoint{10.780511in}{5.205075in}}%
\pgfpathclose%
\pgfusepath{stroke}%
\end{pgfscope}%
\begin{pgfscope}%
\pgfpathrectangle{\pgfqpoint{7.512535in}{0.437222in}}{\pgfqpoint{6.275590in}{5.159444in}}%
\pgfusepath{clip}%
\pgfsetbuttcap%
\pgfsetroundjoin%
\pgfsetlinewidth{1.003750pt}%
\definecolor{currentstroke}{rgb}{0.827451,0.827451,0.827451}%
\pgfsetstrokecolor{currentstroke}%
\pgfsetstrokeopacity{0.800000}%
\pgfsetdash{}{0pt}%
\pgfpathmoveto{\pgfqpoint{10.150452in}{4.500549in}}%
\pgfpathcurveto{\pgfqpoint{10.161502in}{4.500549in}}{\pgfqpoint{10.172101in}{4.504939in}}{\pgfqpoint{10.179914in}{4.512753in}}%
\pgfpathcurveto{\pgfqpoint{10.187728in}{4.520566in}}{\pgfqpoint{10.192118in}{4.531165in}}{\pgfqpoint{10.192118in}{4.542215in}}%
\pgfpathcurveto{\pgfqpoint{10.192118in}{4.553266in}}{\pgfqpoint{10.187728in}{4.563865in}}{\pgfqpoint{10.179914in}{4.571678in}}%
\pgfpathcurveto{\pgfqpoint{10.172101in}{4.579492in}}{\pgfqpoint{10.161502in}{4.583882in}}{\pgfqpoint{10.150452in}{4.583882in}}%
\pgfpathcurveto{\pgfqpoint{10.139401in}{4.583882in}}{\pgfqpoint{10.128802in}{4.579492in}}{\pgfqpoint{10.120989in}{4.571678in}}%
\pgfpathcurveto{\pgfqpoint{10.113175in}{4.563865in}}{\pgfqpoint{10.108785in}{4.553266in}}{\pgfqpoint{10.108785in}{4.542215in}}%
\pgfpathcurveto{\pgfqpoint{10.108785in}{4.531165in}}{\pgfqpoint{10.113175in}{4.520566in}}{\pgfqpoint{10.120989in}{4.512753in}}%
\pgfpathcurveto{\pgfqpoint{10.128802in}{4.504939in}}{\pgfqpoint{10.139401in}{4.500549in}}{\pgfqpoint{10.150452in}{4.500549in}}%
\pgfpathlineto{\pgfqpoint{10.150452in}{4.500549in}}%
\pgfpathclose%
\pgfusepath{stroke}%
\end{pgfscope}%
\begin{pgfscope}%
\pgfpathrectangle{\pgfqpoint{7.512535in}{0.437222in}}{\pgfqpoint{6.275590in}{5.159444in}}%
\pgfusepath{clip}%
\pgfsetbuttcap%
\pgfsetroundjoin%
\pgfsetlinewidth{1.003750pt}%
\definecolor{currentstroke}{rgb}{0.827451,0.827451,0.827451}%
\pgfsetstrokecolor{currentstroke}%
\pgfsetstrokeopacity{0.800000}%
\pgfsetdash{}{0pt}%
\pgfpathmoveto{\pgfqpoint{11.224172in}{5.281713in}}%
\pgfpathcurveto{\pgfqpoint{11.235222in}{5.281713in}}{\pgfqpoint{11.245821in}{5.286104in}}{\pgfqpoint{11.253635in}{5.293917in}}%
\pgfpathcurveto{\pgfqpoint{11.261449in}{5.301731in}}{\pgfqpoint{11.265839in}{5.312330in}}{\pgfqpoint{11.265839in}{5.323380in}}%
\pgfpathcurveto{\pgfqpoint{11.265839in}{5.334430in}}{\pgfqpoint{11.261449in}{5.345029in}}{\pgfqpoint{11.253635in}{5.352843in}}%
\pgfpathcurveto{\pgfqpoint{11.245821in}{5.360656in}}{\pgfqpoint{11.235222in}{5.365047in}}{\pgfqpoint{11.224172in}{5.365047in}}%
\pgfpathcurveto{\pgfqpoint{11.213122in}{5.365047in}}{\pgfqpoint{11.202523in}{5.360656in}}{\pgfqpoint{11.194709in}{5.352843in}}%
\pgfpathcurveto{\pgfqpoint{11.186896in}{5.345029in}}{\pgfqpoint{11.182506in}{5.334430in}}{\pgfqpoint{11.182506in}{5.323380in}}%
\pgfpathcurveto{\pgfqpoint{11.182506in}{5.312330in}}{\pgfqpoint{11.186896in}{5.301731in}}{\pgfqpoint{11.194709in}{5.293917in}}%
\pgfpathcurveto{\pgfqpoint{11.202523in}{5.286104in}}{\pgfqpoint{11.213122in}{5.281713in}}{\pgfqpoint{11.224172in}{5.281713in}}%
\pgfpathlineto{\pgfqpoint{11.224172in}{5.281713in}}%
\pgfpathclose%
\pgfusepath{stroke}%
\end{pgfscope}%
\begin{pgfscope}%
\pgfpathrectangle{\pgfqpoint{7.512535in}{0.437222in}}{\pgfqpoint{6.275590in}{5.159444in}}%
\pgfusepath{clip}%
\pgfsetbuttcap%
\pgfsetroundjoin%
\pgfsetlinewidth{1.003750pt}%
\definecolor{currentstroke}{rgb}{0.827451,0.827451,0.827451}%
\pgfsetstrokecolor{currentstroke}%
\pgfsetstrokeopacity{0.800000}%
\pgfsetdash{}{0pt}%
\pgfpathmoveto{\pgfqpoint{7.711198in}{0.615504in}}%
\pgfpathcurveto{\pgfqpoint{7.722248in}{0.615504in}}{\pgfqpoint{7.732847in}{0.619894in}}{\pgfqpoint{7.740661in}{0.627708in}}%
\pgfpathcurveto{\pgfqpoint{7.748474in}{0.635522in}}{\pgfqpoint{7.752865in}{0.646121in}}{\pgfqpoint{7.752865in}{0.657171in}}%
\pgfpathcurveto{\pgfqpoint{7.752865in}{0.668221in}}{\pgfqpoint{7.748474in}{0.678820in}}{\pgfqpoint{7.740661in}{0.686633in}}%
\pgfpathcurveto{\pgfqpoint{7.732847in}{0.694447in}}{\pgfqpoint{7.722248in}{0.698837in}}{\pgfqpoint{7.711198in}{0.698837in}}%
\pgfpathcurveto{\pgfqpoint{7.700148in}{0.698837in}}{\pgfqpoint{7.689549in}{0.694447in}}{\pgfqpoint{7.681735in}{0.686633in}}%
\pgfpathcurveto{\pgfqpoint{7.673921in}{0.678820in}}{\pgfqpoint{7.669531in}{0.668221in}}{\pgfqpoint{7.669531in}{0.657171in}}%
\pgfpathcurveto{\pgfqpoint{7.669531in}{0.646121in}}{\pgfqpoint{7.673921in}{0.635522in}}{\pgfqpoint{7.681735in}{0.627708in}}%
\pgfpathcurveto{\pgfqpoint{7.689549in}{0.619894in}}{\pgfqpoint{7.700148in}{0.615504in}}{\pgfqpoint{7.711198in}{0.615504in}}%
\pgfpathlineto{\pgfqpoint{7.711198in}{0.615504in}}%
\pgfpathclose%
\pgfusepath{stroke}%
\end{pgfscope}%
\begin{pgfscope}%
\pgfpathrectangle{\pgfqpoint{7.512535in}{0.437222in}}{\pgfqpoint{6.275590in}{5.159444in}}%
\pgfusepath{clip}%
\pgfsetbuttcap%
\pgfsetroundjoin%
\pgfsetlinewidth{1.003750pt}%
\definecolor{currentstroke}{rgb}{0.827451,0.827451,0.827451}%
\pgfsetstrokecolor{currentstroke}%
\pgfsetstrokeopacity{0.800000}%
\pgfsetdash{}{0pt}%
\pgfpathmoveto{\pgfqpoint{7.562575in}{0.493855in}}%
\pgfpathcurveto{\pgfqpoint{7.573625in}{0.493855in}}{\pgfqpoint{7.584224in}{0.498245in}}{\pgfqpoint{7.592038in}{0.506059in}}%
\pgfpathcurveto{\pgfqpoint{7.599851in}{0.513872in}}{\pgfqpoint{7.604241in}{0.524471in}}{\pgfqpoint{7.604241in}{0.535522in}}%
\pgfpathcurveto{\pgfqpoint{7.604241in}{0.546572in}}{\pgfqpoint{7.599851in}{0.557171in}}{\pgfqpoint{7.592038in}{0.564984in}}%
\pgfpathcurveto{\pgfqpoint{7.584224in}{0.572798in}}{\pgfqpoint{7.573625in}{0.577188in}}{\pgfqpoint{7.562575in}{0.577188in}}%
\pgfpathcurveto{\pgfqpoint{7.551525in}{0.577188in}}{\pgfqpoint{7.540926in}{0.572798in}}{\pgfqpoint{7.533112in}{0.564984in}}%
\pgfpathcurveto{\pgfqpoint{7.525298in}{0.557171in}}{\pgfqpoint{7.520908in}{0.546572in}}{\pgfqpoint{7.520908in}{0.535522in}}%
\pgfpathcurveto{\pgfqpoint{7.520908in}{0.524471in}}{\pgfqpoint{7.525298in}{0.513872in}}{\pgfqpoint{7.533112in}{0.506059in}}%
\pgfpathcurveto{\pgfqpoint{7.540926in}{0.498245in}}{\pgfqpoint{7.551525in}{0.493855in}}{\pgfqpoint{7.562575in}{0.493855in}}%
\pgfpathlineto{\pgfqpoint{7.562575in}{0.493855in}}%
\pgfpathclose%
\pgfusepath{stroke}%
\end{pgfscope}%
\begin{pgfscope}%
\pgfpathrectangle{\pgfqpoint{7.512535in}{0.437222in}}{\pgfqpoint{6.275590in}{5.159444in}}%
\pgfusepath{clip}%
\pgfsetbuttcap%
\pgfsetroundjoin%
\pgfsetlinewidth{1.003750pt}%
\definecolor{currentstroke}{rgb}{0.827451,0.827451,0.827451}%
\pgfsetstrokecolor{currentstroke}%
\pgfsetstrokeopacity{0.800000}%
\pgfsetdash{}{0pt}%
\pgfpathmoveto{\pgfqpoint{11.405420in}{5.475782in}}%
\pgfpathcurveto{\pgfqpoint{11.416470in}{5.475782in}}{\pgfqpoint{11.427069in}{5.480172in}}{\pgfqpoint{11.434883in}{5.487986in}}%
\pgfpathcurveto{\pgfqpoint{11.442696in}{5.495800in}}{\pgfqpoint{11.447086in}{5.506399in}}{\pgfqpoint{11.447086in}{5.517449in}}%
\pgfpathcurveto{\pgfqpoint{11.447086in}{5.528499in}}{\pgfqpoint{11.442696in}{5.539098in}}{\pgfqpoint{11.434883in}{5.546912in}}%
\pgfpathcurveto{\pgfqpoint{11.427069in}{5.554725in}}{\pgfqpoint{11.416470in}{5.559116in}}{\pgfqpoint{11.405420in}{5.559116in}}%
\pgfpathcurveto{\pgfqpoint{11.394370in}{5.559116in}}{\pgfqpoint{11.383771in}{5.554725in}}{\pgfqpoint{11.375957in}{5.546912in}}%
\pgfpathcurveto{\pgfqpoint{11.368143in}{5.539098in}}{\pgfqpoint{11.363753in}{5.528499in}}{\pgfqpoint{11.363753in}{5.517449in}}%
\pgfpathcurveto{\pgfqpoint{11.363753in}{5.506399in}}{\pgfqpoint{11.368143in}{5.495800in}}{\pgfqpoint{11.375957in}{5.487986in}}%
\pgfpathcurveto{\pgfqpoint{11.383771in}{5.480172in}}{\pgfqpoint{11.394370in}{5.475782in}}{\pgfqpoint{11.405420in}{5.475782in}}%
\pgfpathlineto{\pgfqpoint{11.405420in}{5.475782in}}%
\pgfpathclose%
\pgfusepath{stroke}%
\end{pgfscope}%
\begin{pgfscope}%
\pgfpathrectangle{\pgfqpoint{7.512535in}{0.437222in}}{\pgfqpoint{6.275590in}{5.159444in}}%
\pgfusepath{clip}%
\pgfsetbuttcap%
\pgfsetroundjoin%
\pgfsetlinewidth{1.003750pt}%
\definecolor{currentstroke}{rgb}{0.827451,0.827451,0.827451}%
\pgfsetstrokecolor{currentstroke}%
\pgfsetstrokeopacity{0.800000}%
\pgfsetdash{}{0pt}%
\pgfpathmoveto{\pgfqpoint{10.435824in}{5.040165in}}%
\pgfpathcurveto{\pgfqpoint{10.446874in}{5.040165in}}{\pgfqpoint{10.457473in}{5.044555in}}{\pgfqpoint{10.465287in}{5.052369in}}%
\pgfpathcurveto{\pgfqpoint{10.473101in}{5.060183in}}{\pgfqpoint{10.477491in}{5.070782in}}{\pgfqpoint{10.477491in}{5.081832in}}%
\pgfpathcurveto{\pgfqpoint{10.477491in}{5.092882in}}{\pgfqpoint{10.473101in}{5.103481in}}{\pgfqpoint{10.465287in}{5.111295in}}%
\pgfpathcurveto{\pgfqpoint{10.457473in}{5.119108in}}{\pgfqpoint{10.446874in}{5.123499in}}{\pgfqpoint{10.435824in}{5.123499in}}%
\pgfpathcurveto{\pgfqpoint{10.424774in}{5.123499in}}{\pgfqpoint{10.414175in}{5.119108in}}{\pgfqpoint{10.406361in}{5.111295in}}%
\pgfpathcurveto{\pgfqpoint{10.398548in}{5.103481in}}{\pgfqpoint{10.394157in}{5.092882in}}{\pgfqpoint{10.394157in}{5.081832in}}%
\pgfpathcurveto{\pgfqpoint{10.394157in}{5.070782in}}{\pgfqpoint{10.398548in}{5.060183in}}{\pgfqpoint{10.406361in}{5.052369in}}%
\pgfpathcurveto{\pgfqpoint{10.414175in}{5.044555in}}{\pgfqpoint{10.424774in}{5.040165in}}{\pgfqpoint{10.435824in}{5.040165in}}%
\pgfpathlineto{\pgfqpoint{10.435824in}{5.040165in}}%
\pgfpathclose%
\pgfusepath{stroke}%
\end{pgfscope}%
\begin{pgfscope}%
\pgfpathrectangle{\pgfqpoint{7.512535in}{0.437222in}}{\pgfqpoint{6.275590in}{5.159444in}}%
\pgfusepath{clip}%
\pgfsetbuttcap%
\pgfsetroundjoin%
\pgfsetlinewidth{1.003750pt}%
\definecolor{currentstroke}{rgb}{0.827451,0.827451,0.827451}%
\pgfsetstrokecolor{currentstroke}%
\pgfsetstrokeopacity{0.800000}%
\pgfsetdash{}{0pt}%
\pgfpathmoveto{\pgfqpoint{8.619483in}{1.654089in}}%
\pgfpathcurveto{\pgfqpoint{8.630533in}{1.654089in}}{\pgfqpoint{8.641132in}{1.658480in}}{\pgfqpoint{8.648946in}{1.666293in}}%
\pgfpathcurveto{\pgfqpoint{8.656759in}{1.674107in}}{\pgfqpoint{8.661149in}{1.684706in}}{\pgfqpoint{8.661149in}{1.695756in}}%
\pgfpathcurveto{\pgfqpoint{8.661149in}{1.706806in}}{\pgfqpoint{8.656759in}{1.717405in}}{\pgfqpoint{8.648946in}{1.725219in}}%
\pgfpathcurveto{\pgfqpoint{8.641132in}{1.733033in}}{\pgfqpoint{8.630533in}{1.737423in}}{\pgfqpoint{8.619483in}{1.737423in}}%
\pgfpathcurveto{\pgfqpoint{8.608433in}{1.737423in}}{\pgfqpoint{8.597834in}{1.733033in}}{\pgfqpoint{8.590020in}{1.725219in}}%
\pgfpathcurveto{\pgfqpoint{8.582206in}{1.717405in}}{\pgfqpoint{8.577816in}{1.706806in}}{\pgfqpoint{8.577816in}{1.695756in}}%
\pgfpathcurveto{\pgfqpoint{8.577816in}{1.684706in}}{\pgfqpoint{8.582206in}{1.674107in}}{\pgfqpoint{8.590020in}{1.666293in}}%
\pgfpathcurveto{\pgfqpoint{8.597834in}{1.658480in}}{\pgfqpoint{8.608433in}{1.654089in}}{\pgfqpoint{8.619483in}{1.654089in}}%
\pgfpathlineto{\pgfqpoint{8.619483in}{1.654089in}}%
\pgfpathclose%
\pgfusepath{stroke}%
\end{pgfscope}%
\begin{pgfscope}%
\pgfpathrectangle{\pgfqpoint{7.512535in}{0.437222in}}{\pgfqpoint{6.275590in}{5.159444in}}%
\pgfusepath{clip}%
\pgfsetbuttcap%
\pgfsetroundjoin%
\pgfsetlinewidth{1.003750pt}%
\definecolor{currentstroke}{rgb}{0.827451,0.827451,0.827451}%
\pgfsetstrokecolor{currentstroke}%
\pgfsetstrokeopacity{0.800000}%
\pgfsetdash{}{0pt}%
\pgfpathmoveto{\pgfqpoint{10.312167in}{5.501728in}}%
\pgfpathcurveto{\pgfqpoint{10.323217in}{5.501728in}}{\pgfqpoint{10.333816in}{5.506118in}}{\pgfqpoint{10.341630in}{5.513932in}}%
\pgfpathcurveto{\pgfqpoint{10.349443in}{5.521745in}}{\pgfqpoint{10.353834in}{5.532344in}}{\pgfqpoint{10.353834in}{5.543394in}}%
\pgfpathcurveto{\pgfqpoint{10.353834in}{5.554445in}}{\pgfqpoint{10.349443in}{5.565044in}}{\pgfqpoint{10.341630in}{5.572857in}}%
\pgfpathcurveto{\pgfqpoint{10.333816in}{5.580671in}}{\pgfqpoint{10.323217in}{5.585061in}}{\pgfqpoint{10.312167in}{5.585061in}}%
\pgfpathcurveto{\pgfqpoint{10.301117in}{5.585061in}}{\pgfqpoint{10.290518in}{5.580671in}}{\pgfqpoint{10.282704in}{5.572857in}}%
\pgfpathcurveto{\pgfqpoint{10.274891in}{5.565044in}}{\pgfqpoint{10.270500in}{5.554445in}}{\pgfqpoint{10.270500in}{5.543394in}}%
\pgfpathcurveto{\pgfqpoint{10.270500in}{5.532344in}}{\pgfqpoint{10.274891in}{5.521745in}}{\pgfqpoint{10.282704in}{5.513932in}}%
\pgfpathcurveto{\pgfqpoint{10.290518in}{5.506118in}}{\pgfqpoint{10.301117in}{5.501728in}}{\pgfqpoint{10.312167in}{5.501728in}}%
\pgfpathlineto{\pgfqpoint{10.312167in}{5.501728in}}%
\pgfpathclose%
\pgfusepath{stroke}%
\end{pgfscope}%
\begin{pgfscope}%
\pgfpathrectangle{\pgfqpoint{7.512535in}{0.437222in}}{\pgfqpoint{6.275590in}{5.159444in}}%
\pgfusepath{clip}%
\pgfsetbuttcap%
\pgfsetroundjoin%
\pgfsetlinewidth{1.003750pt}%
\definecolor{currentstroke}{rgb}{0.827451,0.827451,0.827451}%
\pgfsetstrokecolor{currentstroke}%
\pgfsetstrokeopacity{0.800000}%
\pgfsetdash{}{0pt}%
\pgfpathmoveto{\pgfqpoint{8.282445in}{1.938912in}}%
\pgfpathcurveto{\pgfqpoint{8.293495in}{1.938912in}}{\pgfqpoint{8.304095in}{1.943302in}}{\pgfqpoint{8.311908in}{1.951116in}}%
\pgfpathcurveto{\pgfqpoint{8.319722in}{1.958929in}}{\pgfqpoint{8.324112in}{1.969528in}}{\pgfqpoint{8.324112in}{1.980579in}}%
\pgfpathcurveto{\pgfqpoint{8.324112in}{1.991629in}}{\pgfqpoint{8.319722in}{2.002228in}}{\pgfqpoint{8.311908in}{2.010041in}}%
\pgfpathcurveto{\pgfqpoint{8.304095in}{2.017855in}}{\pgfqpoint{8.293495in}{2.022245in}}{\pgfqpoint{8.282445in}{2.022245in}}%
\pgfpathcurveto{\pgfqpoint{8.271395in}{2.022245in}}{\pgfqpoint{8.260796in}{2.017855in}}{\pgfqpoint{8.252983in}{2.010041in}}%
\pgfpathcurveto{\pgfqpoint{8.245169in}{2.002228in}}{\pgfqpoint{8.240779in}{1.991629in}}{\pgfqpoint{8.240779in}{1.980579in}}%
\pgfpathcurveto{\pgfqpoint{8.240779in}{1.969528in}}{\pgfqpoint{8.245169in}{1.958929in}}{\pgfqpoint{8.252983in}{1.951116in}}%
\pgfpathcurveto{\pgfqpoint{8.260796in}{1.943302in}}{\pgfqpoint{8.271395in}{1.938912in}}{\pgfqpoint{8.282445in}{1.938912in}}%
\pgfpathlineto{\pgfqpoint{8.282445in}{1.938912in}}%
\pgfpathclose%
\pgfusepath{stroke}%
\end{pgfscope}%
\begin{pgfscope}%
\pgfpathrectangle{\pgfqpoint{7.512535in}{0.437222in}}{\pgfqpoint{6.275590in}{5.159444in}}%
\pgfusepath{clip}%
\pgfsetbuttcap%
\pgfsetroundjoin%
\pgfsetlinewidth{1.003750pt}%
\definecolor{currentstroke}{rgb}{0.827451,0.827451,0.827451}%
\pgfsetstrokecolor{currentstroke}%
\pgfsetstrokeopacity{0.800000}%
\pgfsetdash{}{0pt}%
\pgfpathmoveto{\pgfqpoint{13.273901in}{5.501584in}}%
\pgfpathcurveto{\pgfqpoint{13.284951in}{5.501584in}}{\pgfqpoint{13.295550in}{5.505974in}}{\pgfqpoint{13.303364in}{5.513787in}}%
\pgfpathcurveto{\pgfqpoint{13.311177in}{5.521601in}}{\pgfqpoint{13.315568in}{5.532200in}}{\pgfqpoint{13.315568in}{5.543250in}}%
\pgfpathcurveto{\pgfqpoint{13.315568in}{5.554300in}}{\pgfqpoint{13.311177in}{5.564899in}}{\pgfqpoint{13.303364in}{5.572713in}}%
\pgfpathcurveto{\pgfqpoint{13.295550in}{5.580527in}}{\pgfqpoint{13.284951in}{5.584917in}}{\pgfqpoint{13.273901in}{5.584917in}}%
\pgfpathcurveto{\pgfqpoint{13.262851in}{5.584917in}}{\pgfqpoint{13.252252in}{5.580527in}}{\pgfqpoint{13.244438in}{5.572713in}}%
\pgfpathcurveto{\pgfqpoint{13.236625in}{5.564899in}}{\pgfqpoint{13.232234in}{5.554300in}}{\pgfqpoint{13.232234in}{5.543250in}}%
\pgfpathcurveto{\pgfqpoint{13.232234in}{5.532200in}}{\pgfqpoint{13.236625in}{5.521601in}}{\pgfqpoint{13.244438in}{5.513787in}}%
\pgfpathcurveto{\pgfqpoint{13.252252in}{5.505974in}}{\pgfqpoint{13.262851in}{5.501584in}}{\pgfqpoint{13.273901in}{5.501584in}}%
\pgfpathlineto{\pgfqpoint{13.273901in}{5.501584in}}%
\pgfpathclose%
\pgfusepath{stroke}%
\end{pgfscope}%
\begin{pgfscope}%
\pgfpathrectangle{\pgfqpoint{7.512535in}{0.437222in}}{\pgfqpoint{6.275590in}{5.159444in}}%
\pgfusepath{clip}%
\pgfsetbuttcap%
\pgfsetroundjoin%
\pgfsetlinewidth{1.003750pt}%
\definecolor{currentstroke}{rgb}{0.827451,0.827451,0.827451}%
\pgfsetstrokecolor{currentstroke}%
\pgfsetstrokeopacity{0.800000}%
\pgfsetdash{}{0pt}%
\pgfpathmoveto{\pgfqpoint{9.630920in}{3.276112in}}%
\pgfpathcurveto{\pgfqpoint{9.641970in}{3.276112in}}{\pgfqpoint{9.652569in}{3.280502in}}{\pgfqpoint{9.660383in}{3.288316in}}%
\pgfpathcurveto{\pgfqpoint{9.668196in}{3.296130in}}{\pgfqpoint{9.672587in}{3.306729in}}{\pgfqpoint{9.672587in}{3.317779in}}%
\pgfpathcurveto{\pgfqpoint{9.672587in}{3.328829in}}{\pgfqpoint{9.668196in}{3.339428in}}{\pgfqpoint{9.660383in}{3.347241in}}%
\pgfpathcurveto{\pgfqpoint{9.652569in}{3.355055in}}{\pgfqpoint{9.641970in}{3.359445in}}{\pgfqpoint{9.630920in}{3.359445in}}%
\pgfpathcurveto{\pgfqpoint{9.619870in}{3.359445in}}{\pgfqpoint{9.609271in}{3.355055in}}{\pgfqpoint{9.601457in}{3.347241in}}%
\pgfpathcurveto{\pgfqpoint{9.593644in}{3.339428in}}{\pgfqpoint{9.589253in}{3.328829in}}{\pgfqpoint{9.589253in}{3.317779in}}%
\pgfpathcurveto{\pgfqpoint{9.589253in}{3.306729in}}{\pgfqpoint{9.593644in}{3.296130in}}{\pgfqpoint{9.601457in}{3.288316in}}%
\pgfpathcurveto{\pgfqpoint{9.609271in}{3.280502in}}{\pgfqpoint{9.619870in}{3.276112in}}{\pgfqpoint{9.630920in}{3.276112in}}%
\pgfpathlineto{\pgfqpoint{9.630920in}{3.276112in}}%
\pgfpathclose%
\pgfusepath{stroke}%
\end{pgfscope}%
\begin{pgfscope}%
\pgfpathrectangle{\pgfqpoint{7.512535in}{0.437222in}}{\pgfqpoint{6.275590in}{5.159444in}}%
\pgfusepath{clip}%
\pgfsetbuttcap%
\pgfsetroundjoin%
\pgfsetlinewidth{1.003750pt}%
\definecolor{currentstroke}{rgb}{0.827451,0.827451,0.827451}%
\pgfsetstrokecolor{currentstroke}%
\pgfsetstrokeopacity{0.800000}%
\pgfsetdash{}{0pt}%
\pgfpathmoveto{\pgfqpoint{10.486066in}{4.589271in}}%
\pgfpathcurveto{\pgfqpoint{10.497116in}{4.589271in}}{\pgfqpoint{10.507715in}{4.593662in}}{\pgfqpoint{10.515529in}{4.601475in}}%
\pgfpathcurveto{\pgfqpoint{10.523342in}{4.609289in}}{\pgfqpoint{10.527733in}{4.619888in}}{\pgfqpoint{10.527733in}{4.630938in}}%
\pgfpathcurveto{\pgfqpoint{10.527733in}{4.641988in}}{\pgfqpoint{10.523342in}{4.652587in}}{\pgfqpoint{10.515529in}{4.660401in}}%
\pgfpathcurveto{\pgfqpoint{10.507715in}{4.668214in}}{\pgfqpoint{10.497116in}{4.672605in}}{\pgfqpoint{10.486066in}{4.672605in}}%
\pgfpathcurveto{\pgfqpoint{10.475016in}{4.672605in}}{\pgfqpoint{10.464417in}{4.668214in}}{\pgfqpoint{10.456603in}{4.660401in}}%
\pgfpathcurveto{\pgfqpoint{10.448790in}{4.652587in}}{\pgfqpoint{10.444399in}{4.641988in}}{\pgfqpoint{10.444399in}{4.630938in}}%
\pgfpathcurveto{\pgfqpoint{10.444399in}{4.619888in}}{\pgfqpoint{10.448790in}{4.609289in}}{\pgfqpoint{10.456603in}{4.601475in}}%
\pgfpathcurveto{\pgfqpoint{10.464417in}{4.593662in}}{\pgfqpoint{10.475016in}{4.589271in}}{\pgfqpoint{10.486066in}{4.589271in}}%
\pgfpathlineto{\pgfqpoint{10.486066in}{4.589271in}}%
\pgfpathclose%
\pgfusepath{stroke}%
\end{pgfscope}%
\begin{pgfscope}%
\pgfpathrectangle{\pgfqpoint{7.512535in}{0.437222in}}{\pgfqpoint{6.275590in}{5.159444in}}%
\pgfusepath{clip}%
\pgfsetbuttcap%
\pgfsetroundjoin%
\pgfsetlinewidth{1.003750pt}%
\definecolor{currentstroke}{rgb}{0.827451,0.827451,0.827451}%
\pgfsetstrokecolor{currentstroke}%
\pgfsetstrokeopacity{0.800000}%
\pgfsetdash{}{0pt}%
\pgfpathmoveto{\pgfqpoint{13.425002in}{5.528743in}}%
\pgfpathcurveto{\pgfqpoint{13.436053in}{5.528743in}}{\pgfqpoint{13.446652in}{5.533133in}}{\pgfqpoint{13.454465in}{5.540947in}}%
\pgfpathcurveto{\pgfqpoint{13.462279in}{5.548760in}}{\pgfqpoint{13.466669in}{5.559359in}}{\pgfqpoint{13.466669in}{5.570410in}}%
\pgfpathcurveto{\pgfqpoint{13.466669in}{5.581460in}}{\pgfqpoint{13.462279in}{5.592059in}}{\pgfqpoint{13.454465in}{5.599872in}}%
\pgfpathcurveto{\pgfqpoint{13.446652in}{5.607686in}}{\pgfqpoint{13.436053in}{5.612076in}}{\pgfqpoint{13.425002in}{5.612076in}}%
\pgfpathcurveto{\pgfqpoint{13.413952in}{5.612076in}}{\pgfqpoint{13.403353in}{5.607686in}}{\pgfqpoint{13.395540in}{5.599872in}}%
\pgfpathcurveto{\pgfqpoint{13.387726in}{5.592059in}}{\pgfqpoint{13.383336in}{5.581460in}}{\pgfqpoint{13.383336in}{5.570410in}}%
\pgfpathcurveto{\pgfqpoint{13.383336in}{5.559359in}}{\pgfqpoint{13.387726in}{5.548760in}}{\pgfqpoint{13.395540in}{5.540947in}}%
\pgfpathcurveto{\pgfqpoint{13.403353in}{5.533133in}}{\pgfqpoint{13.413952in}{5.528743in}}{\pgfqpoint{13.425002in}{5.528743in}}%
\pgfpathlineto{\pgfqpoint{13.425002in}{5.528743in}}%
\pgfpathclose%
\pgfusepath{stroke}%
\end{pgfscope}%
\begin{pgfscope}%
\pgfpathrectangle{\pgfqpoint{7.512535in}{0.437222in}}{\pgfqpoint{6.275590in}{5.159444in}}%
\pgfusepath{clip}%
\pgfsetbuttcap%
\pgfsetroundjoin%
\pgfsetlinewidth{1.003750pt}%
\definecolor{currentstroke}{rgb}{0.827451,0.827451,0.827451}%
\pgfsetstrokecolor{currentstroke}%
\pgfsetstrokeopacity{0.800000}%
\pgfsetdash{}{0pt}%
\pgfpathmoveto{\pgfqpoint{10.385573in}{5.221855in}}%
\pgfpathcurveto{\pgfqpoint{10.396623in}{5.221855in}}{\pgfqpoint{10.407222in}{5.226245in}}{\pgfqpoint{10.415036in}{5.234058in}}%
\pgfpathcurveto{\pgfqpoint{10.422849in}{5.241872in}}{\pgfqpoint{10.427240in}{5.252471in}}{\pgfqpoint{10.427240in}{5.263521in}}%
\pgfpathcurveto{\pgfqpoint{10.427240in}{5.274571in}}{\pgfqpoint{10.422849in}{5.285170in}}{\pgfqpoint{10.415036in}{5.292984in}}%
\pgfpathcurveto{\pgfqpoint{10.407222in}{5.300798in}}{\pgfqpoint{10.396623in}{5.305188in}}{\pgfqpoint{10.385573in}{5.305188in}}%
\pgfpathcurveto{\pgfqpoint{10.374523in}{5.305188in}}{\pgfqpoint{10.363924in}{5.300798in}}{\pgfqpoint{10.356110in}{5.292984in}}%
\pgfpathcurveto{\pgfqpoint{10.348297in}{5.285170in}}{\pgfqpoint{10.343906in}{5.274571in}}{\pgfqpoint{10.343906in}{5.263521in}}%
\pgfpathcurveto{\pgfqpoint{10.343906in}{5.252471in}}{\pgfqpoint{10.348297in}{5.241872in}}{\pgfqpoint{10.356110in}{5.234058in}}%
\pgfpathcurveto{\pgfqpoint{10.363924in}{5.226245in}}{\pgfqpoint{10.374523in}{5.221855in}}{\pgfqpoint{10.385573in}{5.221855in}}%
\pgfpathlineto{\pgfqpoint{10.385573in}{5.221855in}}%
\pgfpathclose%
\pgfusepath{stroke}%
\end{pgfscope}%
\begin{pgfscope}%
\pgfpathrectangle{\pgfqpoint{7.512535in}{0.437222in}}{\pgfqpoint{6.275590in}{5.159444in}}%
\pgfusepath{clip}%
\pgfsetbuttcap%
\pgfsetroundjoin%
\pgfsetlinewidth{1.003750pt}%
\definecolor{currentstroke}{rgb}{0.827451,0.827451,0.827451}%
\pgfsetstrokecolor{currentstroke}%
\pgfsetstrokeopacity{0.800000}%
\pgfsetdash{}{0pt}%
\pgfpathmoveto{\pgfqpoint{8.569943in}{1.597311in}}%
\pgfpathcurveto{\pgfqpoint{8.580993in}{1.597311in}}{\pgfqpoint{8.591592in}{1.601702in}}{\pgfqpoint{8.599406in}{1.609515in}}%
\pgfpathcurveto{\pgfqpoint{8.607219in}{1.617329in}}{\pgfqpoint{8.611610in}{1.627928in}}{\pgfqpoint{8.611610in}{1.638978in}}%
\pgfpathcurveto{\pgfqpoint{8.611610in}{1.650028in}}{\pgfqpoint{8.607219in}{1.660627in}}{\pgfqpoint{8.599406in}{1.668441in}}%
\pgfpathcurveto{\pgfqpoint{8.591592in}{1.676254in}}{\pgfqpoint{8.580993in}{1.680645in}}{\pgfqpoint{8.569943in}{1.680645in}}%
\pgfpathcurveto{\pgfqpoint{8.558893in}{1.680645in}}{\pgfqpoint{8.548294in}{1.676254in}}{\pgfqpoint{8.540480in}{1.668441in}}%
\pgfpathcurveto{\pgfqpoint{8.532667in}{1.660627in}}{\pgfqpoint{8.528276in}{1.650028in}}{\pgfqpoint{8.528276in}{1.638978in}}%
\pgfpathcurveto{\pgfqpoint{8.528276in}{1.627928in}}{\pgfqpoint{8.532667in}{1.617329in}}{\pgfqpoint{8.540480in}{1.609515in}}%
\pgfpathcurveto{\pgfqpoint{8.548294in}{1.601702in}}{\pgfqpoint{8.558893in}{1.597311in}}{\pgfqpoint{8.569943in}{1.597311in}}%
\pgfpathlineto{\pgfqpoint{8.569943in}{1.597311in}}%
\pgfpathclose%
\pgfusepath{stroke}%
\end{pgfscope}%
\begin{pgfscope}%
\pgfpathrectangle{\pgfqpoint{7.512535in}{0.437222in}}{\pgfqpoint{6.275590in}{5.159444in}}%
\pgfusepath{clip}%
\pgfsetbuttcap%
\pgfsetroundjoin%
\pgfsetlinewidth{1.003750pt}%
\definecolor{currentstroke}{rgb}{0.827451,0.827451,0.827451}%
\pgfsetstrokecolor{currentstroke}%
\pgfsetstrokeopacity{0.800000}%
\pgfsetdash{}{0pt}%
\pgfpathmoveto{\pgfqpoint{13.575740in}{5.547910in}}%
\pgfpathcurveto{\pgfqpoint{13.586790in}{5.547910in}}{\pgfqpoint{13.597389in}{5.552301in}}{\pgfqpoint{13.605202in}{5.560114in}}%
\pgfpathcurveto{\pgfqpoint{13.613016in}{5.567928in}}{\pgfqpoint{13.617406in}{5.578527in}}{\pgfqpoint{13.617406in}{5.589577in}}%
\pgfpathcurveto{\pgfqpoint{13.617406in}{5.600627in}}{\pgfqpoint{13.613016in}{5.611226in}}{\pgfqpoint{13.605202in}{5.619040in}}%
\pgfpathcurveto{\pgfqpoint{13.597389in}{5.626854in}}{\pgfqpoint{13.586790in}{5.631244in}}{\pgfqpoint{13.575740in}{5.631244in}}%
\pgfpathcurveto{\pgfqpoint{13.564690in}{5.631244in}}{\pgfqpoint{13.554090in}{5.626854in}}{\pgfqpoint{13.546277in}{5.619040in}}%
\pgfpathcurveto{\pgfqpoint{13.538463in}{5.611226in}}{\pgfqpoint{13.534073in}{5.600627in}}{\pgfqpoint{13.534073in}{5.589577in}}%
\pgfpathcurveto{\pgfqpoint{13.534073in}{5.578527in}}{\pgfqpoint{13.538463in}{5.567928in}}{\pgfqpoint{13.546277in}{5.560114in}}%
\pgfpathcurveto{\pgfqpoint{13.554090in}{5.552301in}}{\pgfqpoint{13.564690in}{5.547910in}}{\pgfqpoint{13.575740in}{5.547910in}}%
\pgfpathlineto{\pgfqpoint{13.575740in}{5.547910in}}%
\pgfpathclose%
\pgfusepath{stroke}%
\end{pgfscope}%
\begin{pgfscope}%
\pgfpathrectangle{\pgfqpoint{7.512535in}{0.437222in}}{\pgfqpoint{6.275590in}{5.159444in}}%
\pgfusepath{clip}%
\pgfsetbuttcap%
\pgfsetroundjoin%
\pgfsetlinewidth{1.003750pt}%
\definecolor{currentstroke}{rgb}{0.827451,0.827451,0.827451}%
\pgfsetstrokecolor{currentstroke}%
\pgfsetstrokeopacity{0.800000}%
\pgfsetdash{}{0pt}%
\pgfpathmoveto{\pgfqpoint{9.216460in}{3.540868in}}%
\pgfpathcurveto{\pgfqpoint{9.227510in}{3.540868in}}{\pgfqpoint{9.238109in}{3.545259in}}{\pgfqpoint{9.245923in}{3.553072in}}%
\pgfpathcurveto{\pgfqpoint{9.253737in}{3.560886in}}{\pgfqpoint{9.258127in}{3.571485in}}{\pgfqpoint{9.258127in}{3.582535in}}%
\pgfpathcurveto{\pgfqpoint{9.258127in}{3.593585in}}{\pgfqpoint{9.253737in}{3.604184in}}{\pgfqpoint{9.245923in}{3.611998in}}%
\pgfpathcurveto{\pgfqpoint{9.238109in}{3.619812in}}{\pgfqpoint{9.227510in}{3.624202in}}{\pgfqpoint{9.216460in}{3.624202in}}%
\pgfpathcurveto{\pgfqpoint{9.205410in}{3.624202in}}{\pgfqpoint{9.194811in}{3.619812in}}{\pgfqpoint{9.186997in}{3.611998in}}%
\pgfpathcurveto{\pgfqpoint{9.179184in}{3.604184in}}{\pgfqpoint{9.174794in}{3.593585in}}{\pgfqpoint{9.174794in}{3.582535in}}%
\pgfpathcurveto{\pgfqpoint{9.174794in}{3.571485in}}{\pgfqpoint{9.179184in}{3.560886in}}{\pgfqpoint{9.186997in}{3.553072in}}%
\pgfpathcurveto{\pgfqpoint{9.194811in}{3.545259in}}{\pgfqpoint{9.205410in}{3.540868in}}{\pgfqpoint{9.216460in}{3.540868in}}%
\pgfpathlineto{\pgfqpoint{9.216460in}{3.540868in}}%
\pgfpathclose%
\pgfusepath{stroke}%
\end{pgfscope}%
\begin{pgfscope}%
\pgfpathrectangle{\pgfqpoint{7.512535in}{0.437222in}}{\pgfqpoint{6.275590in}{5.159444in}}%
\pgfusepath{clip}%
\pgfsetbuttcap%
\pgfsetroundjoin%
\pgfsetlinewidth{1.003750pt}%
\definecolor{currentstroke}{rgb}{0.827451,0.827451,0.827451}%
\pgfsetstrokecolor{currentstroke}%
\pgfsetstrokeopacity{0.800000}%
\pgfsetdash{}{0pt}%
\pgfpathmoveto{\pgfqpoint{7.874057in}{1.320145in}}%
\pgfpathcurveto{\pgfqpoint{7.885107in}{1.320145in}}{\pgfqpoint{7.895706in}{1.324535in}}{\pgfqpoint{7.903520in}{1.332349in}}%
\pgfpathcurveto{\pgfqpoint{7.911334in}{1.340162in}}{\pgfqpoint{7.915724in}{1.350761in}}{\pgfqpoint{7.915724in}{1.361811in}}%
\pgfpathcurveto{\pgfqpoint{7.915724in}{1.372862in}}{\pgfqpoint{7.911334in}{1.383461in}}{\pgfqpoint{7.903520in}{1.391274in}}%
\pgfpathcurveto{\pgfqpoint{7.895706in}{1.399088in}}{\pgfqpoint{7.885107in}{1.403478in}}{\pgfqpoint{7.874057in}{1.403478in}}%
\pgfpathcurveto{\pgfqpoint{7.863007in}{1.403478in}}{\pgfqpoint{7.852408in}{1.399088in}}{\pgfqpoint{7.844595in}{1.391274in}}%
\pgfpathcurveto{\pgfqpoint{7.836781in}{1.383461in}}{\pgfqpoint{7.832391in}{1.372862in}}{\pgfqpoint{7.832391in}{1.361811in}}%
\pgfpathcurveto{\pgfqpoint{7.832391in}{1.350761in}}{\pgfqpoint{7.836781in}{1.340162in}}{\pgfqpoint{7.844595in}{1.332349in}}%
\pgfpathcurveto{\pgfqpoint{7.852408in}{1.324535in}}{\pgfqpoint{7.863007in}{1.320145in}}{\pgfqpoint{7.874057in}{1.320145in}}%
\pgfpathlineto{\pgfqpoint{7.874057in}{1.320145in}}%
\pgfpathclose%
\pgfusepath{stroke}%
\end{pgfscope}%
\begin{pgfscope}%
\pgfpathrectangle{\pgfqpoint{7.512535in}{0.437222in}}{\pgfqpoint{6.275590in}{5.159444in}}%
\pgfusepath{clip}%
\pgfsetbuttcap%
\pgfsetroundjoin%
\pgfsetlinewidth{1.003750pt}%
\definecolor{currentstroke}{rgb}{0.827451,0.827451,0.827451}%
\pgfsetstrokecolor{currentstroke}%
\pgfsetstrokeopacity{0.800000}%
\pgfsetdash{}{0pt}%
\pgfpathmoveto{\pgfqpoint{10.369675in}{4.552452in}}%
\pgfpathcurveto{\pgfqpoint{10.380725in}{4.552452in}}{\pgfqpoint{10.391324in}{4.556842in}}{\pgfqpoint{10.399138in}{4.564656in}}%
\pgfpathcurveto{\pgfqpoint{10.406951in}{4.572470in}}{\pgfqpoint{10.411341in}{4.583069in}}{\pgfqpoint{10.411341in}{4.594119in}}%
\pgfpathcurveto{\pgfqpoint{10.411341in}{4.605169in}}{\pgfqpoint{10.406951in}{4.615768in}}{\pgfqpoint{10.399138in}{4.623582in}}%
\pgfpathcurveto{\pgfqpoint{10.391324in}{4.631395in}}{\pgfqpoint{10.380725in}{4.635785in}}{\pgfqpoint{10.369675in}{4.635785in}}%
\pgfpathcurveto{\pgfqpoint{10.358625in}{4.635785in}}{\pgfqpoint{10.348026in}{4.631395in}}{\pgfqpoint{10.340212in}{4.623582in}}%
\pgfpathcurveto{\pgfqpoint{10.332398in}{4.615768in}}{\pgfqpoint{10.328008in}{4.605169in}}{\pgfqpoint{10.328008in}{4.594119in}}%
\pgfpathcurveto{\pgfqpoint{10.328008in}{4.583069in}}{\pgfqpoint{10.332398in}{4.572470in}}{\pgfqpoint{10.340212in}{4.564656in}}%
\pgfpathcurveto{\pgfqpoint{10.348026in}{4.556842in}}{\pgfqpoint{10.358625in}{4.552452in}}{\pgfqpoint{10.369675in}{4.552452in}}%
\pgfpathlineto{\pgfqpoint{10.369675in}{4.552452in}}%
\pgfpathclose%
\pgfusepath{stroke}%
\end{pgfscope}%
\begin{pgfscope}%
\pgfpathrectangle{\pgfqpoint{7.512535in}{0.437222in}}{\pgfqpoint{6.275590in}{5.159444in}}%
\pgfusepath{clip}%
\pgfsetbuttcap%
\pgfsetroundjoin%
\pgfsetlinewidth{1.003750pt}%
\definecolor{currentstroke}{rgb}{0.827451,0.827451,0.827451}%
\pgfsetstrokecolor{currentstroke}%
\pgfsetstrokeopacity{0.800000}%
\pgfsetdash{}{0pt}%
\pgfpathmoveto{\pgfqpoint{9.447017in}{3.335366in}}%
\pgfpathcurveto{\pgfqpoint{9.458067in}{3.335366in}}{\pgfqpoint{9.468666in}{3.339757in}}{\pgfqpoint{9.476479in}{3.347570in}}%
\pgfpathcurveto{\pgfqpoint{9.484293in}{3.355384in}}{\pgfqpoint{9.488683in}{3.365983in}}{\pgfqpoint{9.488683in}{3.377033in}}%
\pgfpathcurveto{\pgfqpoint{9.488683in}{3.388083in}}{\pgfqpoint{9.484293in}{3.398682in}}{\pgfqpoint{9.476479in}{3.406496in}}%
\pgfpathcurveto{\pgfqpoint{9.468666in}{3.414309in}}{\pgfqpoint{9.458067in}{3.418700in}}{\pgfqpoint{9.447017in}{3.418700in}}%
\pgfpathcurveto{\pgfqpoint{9.435967in}{3.418700in}}{\pgfqpoint{9.425367in}{3.414309in}}{\pgfqpoint{9.417554in}{3.406496in}}%
\pgfpathcurveto{\pgfqpoint{9.409740in}{3.398682in}}{\pgfqpoint{9.405350in}{3.388083in}}{\pgfqpoint{9.405350in}{3.377033in}}%
\pgfpathcurveto{\pgfqpoint{9.405350in}{3.365983in}}{\pgfqpoint{9.409740in}{3.355384in}}{\pgfqpoint{9.417554in}{3.347570in}}%
\pgfpathcurveto{\pgfqpoint{9.425367in}{3.339757in}}{\pgfqpoint{9.435967in}{3.335366in}}{\pgfqpoint{9.447017in}{3.335366in}}%
\pgfpathlineto{\pgfqpoint{9.447017in}{3.335366in}}%
\pgfpathclose%
\pgfusepath{stroke}%
\end{pgfscope}%
\begin{pgfscope}%
\pgfpathrectangle{\pgfqpoint{7.512535in}{0.437222in}}{\pgfqpoint{6.275590in}{5.159444in}}%
\pgfusepath{clip}%
\pgfsetbuttcap%
\pgfsetroundjoin%
\pgfsetlinewidth{1.003750pt}%
\definecolor{currentstroke}{rgb}{0.827451,0.827451,0.827451}%
\pgfsetstrokecolor{currentstroke}%
\pgfsetstrokeopacity{0.800000}%
\pgfsetdash{}{0pt}%
\pgfpathmoveto{\pgfqpoint{10.905755in}{4.601092in}}%
\pgfpathcurveto{\pgfqpoint{10.916805in}{4.601092in}}{\pgfqpoint{10.927404in}{4.605482in}}{\pgfqpoint{10.935217in}{4.613296in}}%
\pgfpathcurveto{\pgfqpoint{10.943031in}{4.621109in}}{\pgfqpoint{10.947421in}{4.631708in}}{\pgfqpoint{10.947421in}{4.642759in}}%
\pgfpathcurveto{\pgfqpoint{10.947421in}{4.653809in}}{\pgfqpoint{10.943031in}{4.664408in}}{\pgfqpoint{10.935217in}{4.672221in}}%
\pgfpathcurveto{\pgfqpoint{10.927404in}{4.680035in}}{\pgfqpoint{10.916805in}{4.684425in}}{\pgfqpoint{10.905755in}{4.684425in}}%
\pgfpathcurveto{\pgfqpoint{10.894704in}{4.684425in}}{\pgfqpoint{10.884105in}{4.680035in}}{\pgfqpoint{10.876292in}{4.672221in}}%
\pgfpathcurveto{\pgfqpoint{10.868478in}{4.664408in}}{\pgfqpoint{10.864088in}{4.653809in}}{\pgfqpoint{10.864088in}{4.642759in}}%
\pgfpathcurveto{\pgfqpoint{10.864088in}{4.631708in}}{\pgfqpoint{10.868478in}{4.621109in}}{\pgfqpoint{10.876292in}{4.613296in}}%
\pgfpathcurveto{\pgfqpoint{10.884105in}{4.605482in}}{\pgfqpoint{10.894704in}{4.601092in}}{\pgfqpoint{10.905755in}{4.601092in}}%
\pgfpathlineto{\pgfqpoint{10.905755in}{4.601092in}}%
\pgfpathclose%
\pgfusepath{stroke}%
\end{pgfscope}%
\begin{pgfscope}%
\pgfpathrectangle{\pgfqpoint{7.512535in}{0.437222in}}{\pgfqpoint{6.275590in}{5.159444in}}%
\pgfusepath{clip}%
\pgfsetbuttcap%
\pgfsetroundjoin%
\pgfsetlinewidth{1.003750pt}%
\definecolor{currentstroke}{rgb}{0.827451,0.827451,0.827451}%
\pgfsetstrokecolor{currentstroke}%
\pgfsetstrokeopacity{0.800000}%
\pgfsetdash{}{0pt}%
\pgfpathmoveto{\pgfqpoint{12.342968in}{5.553091in}}%
\pgfpathcurveto{\pgfqpoint{12.354019in}{5.553091in}}{\pgfqpoint{12.364618in}{5.557481in}}{\pgfqpoint{12.372431in}{5.565295in}}%
\pgfpathcurveto{\pgfqpoint{12.380245in}{5.573108in}}{\pgfqpoint{12.384635in}{5.583707in}}{\pgfqpoint{12.384635in}{5.594757in}}%
\pgfpathcurveto{\pgfqpoint{12.384635in}{5.605808in}}{\pgfqpoint{12.380245in}{5.616407in}}{\pgfqpoint{12.372431in}{5.624220in}}%
\pgfpathcurveto{\pgfqpoint{12.364618in}{5.632034in}}{\pgfqpoint{12.354019in}{5.636424in}}{\pgfqpoint{12.342968in}{5.636424in}}%
\pgfpathcurveto{\pgfqpoint{12.331918in}{5.636424in}}{\pgfqpoint{12.321319in}{5.632034in}}{\pgfqpoint{12.313506in}{5.624220in}}%
\pgfpathcurveto{\pgfqpoint{12.305692in}{5.616407in}}{\pgfqpoint{12.301302in}{5.605808in}}{\pgfqpoint{12.301302in}{5.594757in}}%
\pgfpathcurveto{\pgfqpoint{12.301302in}{5.583707in}}{\pgfqpoint{12.305692in}{5.573108in}}{\pgfqpoint{12.313506in}{5.565295in}}%
\pgfpathcurveto{\pgfqpoint{12.321319in}{5.557481in}}{\pgfqpoint{12.331918in}{5.553091in}}{\pgfqpoint{12.342968in}{5.553091in}}%
\pgfpathlineto{\pgfqpoint{12.342968in}{5.553091in}}%
\pgfpathclose%
\pgfusepath{stroke}%
\end{pgfscope}%
\begin{pgfscope}%
\pgfpathrectangle{\pgfqpoint{7.512535in}{0.437222in}}{\pgfqpoint{6.275590in}{5.159444in}}%
\pgfusepath{clip}%
\pgfsetbuttcap%
\pgfsetroundjoin%
\pgfsetlinewidth{1.003750pt}%
\definecolor{currentstroke}{rgb}{0.827451,0.827451,0.827451}%
\pgfsetstrokecolor{currentstroke}%
\pgfsetstrokeopacity{0.800000}%
\pgfsetdash{}{0pt}%
\pgfpathmoveto{\pgfqpoint{11.416591in}{5.475782in}}%
\pgfpathcurveto{\pgfqpoint{11.427641in}{5.475782in}}{\pgfqpoint{11.438240in}{5.480172in}}{\pgfqpoint{11.446054in}{5.487986in}}%
\pgfpathcurveto{\pgfqpoint{11.453868in}{5.495800in}}{\pgfqpoint{11.458258in}{5.506399in}}{\pgfqpoint{11.458258in}{5.517449in}}%
\pgfpathcurveto{\pgfqpoint{11.458258in}{5.528499in}}{\pgfqpoint{11.453868in}{5.539098in}}{\pgfqpoint{11.446054in}{5.546912in}}%
\pgfpathcurveto{\pgfqpoint{11.438240in}{5.554725in}}{\pgfqpoint{11.427641in}{5.559116in}}{\pgfqpoint{11.416591in}{5.559116in}}%
\pgfpathcurveto{\pgfqpoint{11.405541in}{5.559116in}}{\pgfqpoint{11.394942in}{5.554725in}}{\pgfqpoint{11.387128in}{5.546912in}}%
\pgfpathcurveto{\pgfqpoint{11.379315in}{5.539098in}}{\pgfqpoint{11.374925in}{5.528499in}}{\pgfqpoint{11.374925in}{5.517449in}}%
\pgfpathcurveto{\pgfqpoint{11.374925in}{5.506399in}}{\pgfqpoint{11.379315in}{5.495800in}}{\pgfqpoint{11.387128in}{5.487986in}}%
\pgfpathcurveto{\pgfqpoint{11.394942in}{5.480172in}}{\pgfqpoint{11.405541in}{5.475782in}}{\pgfqpoint{11.416591in}{5.475782in}}%
\pgfpathlineto{\pgfqpoint{11.416591in}{5.475782in}}%
\pgfpathclose%
\pgfusepath{stroke}%
\end{pgfscope}%
\begin{pgfscope}%
\pgfpathrectangle{\pgfqpoint{7.512535in}{0.437222in}}{\pgfqpoint{6.275590in}{5.159444in}}%
\pgfusepath{clip}%
\pgfsetbuttcap%
\pgfsetroundjoin%
\pgfsetlinewidth{1.003750pt}%
\definecolor{currentstroke}{rgb}{0.827451,0.827451,0.827451}%
\pgfsetstrokecolor{currentstroke}%
\pgfsetstrokeopacity{0.800000}%
\pgfsetdash{}{0pt}%
\pgfpathmoveto{\pgfqpoint{10.698510in}{3.970272in}}%
\pgfpathcurveto{\pgfqpoint{10.709560in}{3.970272in}}{\pgfqpoint{10.720159in}{3.974662in}}{\pgfqpoint{10.727973in}{3.982476in}}%
\pgfpathcurveto{\pgfqpoint{10.735787in}{3.990290in}}{\pgfqpoint{10.740177in}{4.000889in}}{\pgfqpoint{10.740177in}{4.011939in}}%
\pgfpathcurveto{\pgfqpoint{10.740177in}{4.022989in}}{\pgfqpoint{10.735787in}{4.033588in}}{\pgfqpoint{10.727973in}{4.041402in}}%
\pgfpathcurveto{\pgfqpoint{10.720159in}{4.049215in}}{\pgfqpoint{10.709560in}{4.053605in}}{\pgfqpoint{10.698510in}{4.053605in}}%
\pgfpathcurveto{\pgfqpoint{10.687460in}{4.053605in}}{\pgfqpoint{10.676861in}{4.049215in}}{\pgfqpoint{10.669047in}{4.041402in}}%
\pgfpathcurveto{\pgfqpoint{10.661234in}{4.033588in}}{\pgfqpoint{10.656844in}{4.022989in}}{\pgfqpoint{10.656844in}{4.011939in}}%
\pgfpathcurveto{\pgfqpoint{10.656844in}{4.000889in}}{\pgfqpoint{10.661234in}{3.990290in}}{\pgfqpoint{10.669047in}{3.982476in}}%
\pgfpathcurveto{\pgfqpoint{10.676861in}{3.974662in}}{\pgfqpoint{10.687460in}{3.970272in}}{\pgfqpoint{10.698510in}{3.970272in}}%
\pgfpathlineto{\pgfqpoint{10.698510in}{3.970272in}}%
\pgfpathclose%
\pgfusepath{stroke}%
\end{pgfscope}%
\begin{pgfscope}%
\pgfpathrectangle{\pgfqpoint{7.512535in}{0.437222in}}{\pgfqpoint{6.275590in}{5.159444in}}%
\pgfusepath{clip}%
\pgfsetbuttcap%
\pgfsetroundjoin%
\pgfsetlinewidth{1.003750pt}%
\definecolor{currentstroke}{rgb}{0.827451,0.827451,0.827451}%
\pgfsetstrokecolor{currentstroke}%
\pgfsetstrokeopacity{0.800000}%
\pgfsetdash{}{0pt}%
\pgfpathmoveto{\pgfqpoint{9.130854in}{1.867691in}}%
\pgfpathcurveto{\pgfqpoint{9.141904in}{1.867691in}}{\pgfqpoint{9.152503in}{1.872081in}}{\pgfqpoint{9.160317in}{1.879895in}}%
\pgfpathcurveto{\pgfqpoint{9.168130in}{1.887709in}}{\pgfqpoint{9.172521in}{1.898308in}}{\pgfqpoint{9.172521in}{1.909358in}}%
\pgfpathcurveto{\pgfqpoint{9.172521in}{1.920408in}}{\pgfqpoint{9.168130in}{1.931007in}}{\pgfqpoint{9.160317in}{1.938821in}}%
\pgfpathcurveto{\pgfqpoint{9.152503in}{1.946634in}}{\pgfqpoint{9.141904in}{1.951024in}}{\pgfqpoint{9.130854in}{1.951024in}}%
\pgfpathcurveto{\pgfqpoint{9.119804in}{1.951024in}}{\pgfqpoint{9.109205in}{1.946634in}}{\pgfqpoint{9.101391in}{1.938821in}}%
\pgfpathcurveto{\pgfqpoint{9.093577in}{1.931007in}}{\pgfqpoint{9.089187in}{1.920408in}}{\pgfqpoint{9.089187in}{1.909358in}}%
\pgfpathcurveto{\pgfqpoint{9.089187in}{1.898308in}}{\pgfqpoint{9.093577in}{1.887709in}}{\pgfqpoint{9.101391in}{1.879895in}}%
\pgfpathcurveto{\pgfqpoint{9.109205in}{1.872081in}}{\pgfqpoint{9.119804in}{1.867691in}}{\pgfqpoint{9.130854in}{1.867691in}}%
\pgfpathlineto{\pgfqpoint{9.130854in}{1.867691in}}%
\pgfpathclose%
\pgfusepath{stroke}%
\end{pgfscope}%
\begin{pgfscope}%
\pgfpathrectangle{\pgfqpoint{7.512535in}{0.437222in}}{\pgfqpoint{6.275590in}{5.159444in}}%
\pgfusepath{clip}%
\pgfsetbuttcap%
\pgfsetroundjoin%
\pgfsetlinewidth{1.003750pt}%
\definecolor{currentstroke}{rgb}{0.827451,0.827451,0.827451}%
\pgfsetstrokecolor{currentstroke}%
\pgfsetstrokeopacity{0.800000}%
\pgfsetdash{}{0pt}%
\pgfpathmoveto{\pgfqpoint{8.804397in}{1.924027in}}%
\pgfpathcurveto{\pgfqpoint{8.815447in}{1.924027in}}{\pgfqpoint{8.826046in}{1.928417in}}{\pgfqpoint{8.833860in}{1.936231in}}%
\pgfpathcurveto{\pgfqpoint{8.841674in}{1.944044in}}{\pgfqpoint{8.846064in}{1.954643in}}{\pgfqpoint{8.846064in}{1.965693in}}%
\pgfpathcurveto{\pgfqpoint{8.846064in}{1.976743in}}{\pgfqpoint{8.841674in}{1.987343in}}{\pgfqpoint{8.833860in}{1.995156in}}%
\pgfpathcurveto{\pgfqpoint{8.826046in}{2.002970in}}{\pgfqpoint{8.815447in}{2.007360in}}{\pgfqpoint{8.804397in}{2.007360in}}%
\pgfpathcurveto{\pgfqpoint{8.793347in}{2.007360in}}{\pgfqpoint{8.782748in}{2.002970in}}{\pgfqpoint{8.774934in}{1.995156in}}%
\pgfpathcurveto{\pgfqpoint{8.767121in}{1.987343in}}{\pgfqpoint{8.762731in}{1.976743in}}{\pgfqpoint{8.762731in}{1.965693in}}%
\pgfpathcurveto{\pgfqpoint{8.762731in}{1.954643in}}{\pgfqpoint{8.767121in}{1.944044in}}{\pgfqpoint{8.774934in}{1.936231in}}%
\pgfpathcurveto{\pgfqpoint{8.782748in}{1.928417in}}{\pgfqpoint{8.793347in}{1.924027in}}{\pgfqpoint{8.804397in}{1.924027in}}%
\pgfpathlineto{\pgfqpoint{8.804397in}{1.924027in}}%
\pgfpathclose%
\pgfusepath{stroke}%
\end{pgfscope}%
\begin{pgfscope}%
\pgfpathrectangle{\pgfqpoint{7.512535in}{0.437222in}}{\pgfqpoint{6.275590in}{5.159444in}}%
\pgfusepath{clip}%
\pgfsetbuttcap%
\pgfsetroundjoin%
\pgfsetlinewidth{1.003750pt}%
\definecolor{currentstroke}{rgb}{0.827451,0.827451,0.827451}%
\pgfsetstrokecolor{currentstroke}%
\pgfsetstrokeopacity{0.800000}%
\pgfsetdash{}{0pt}%
\pgfpathmoveto{\pgfqpoint{9.395327in}{2.160540in}}%
\pgfpathcurveto{\pgfqpoint{9.406377in}{2.160540in}}{\pgfqpoint{9.416976in}{2.164930in}}{\pgfqpoint{9.424790in}{2.172744in}}%
\pgfpathcurveto{\pgfqpoint{9.432604in}{2.180557in}}{\pgfqpoint{9.436994in}{2.191156in}}{\pgfqpoint{9.436994in}{2.202207in}}%
\pgfpathcurveto{\pgfqpoint{9.436994in}{2.213257in}}{\pgfqpoint{9.432604in}{2.223856in}}{\pgfqpoint{9.424790in}{2.231669in}}%
\pgfpathcurveto{\pgfqpoint{9.416976in}{2.239483in}}{\pgfqpoint{9.406377in}{2.243873in}}{\pgfqpoint{9.395327in}{2.243873in}}%
\pgfpathcurveto{\pgfqpoint{9.384277in}{2.243873in}}{\pgfqpoint{9.373678in}{2.239483in}}{\pgfqpoint{9.365865in}{2.231669in}}%
\pgfpathcurveto{\pgfqpoint{9.358051in}{2.223856in}}{\pgfqpoint{9.353661in}{2.213257in}}{\pgfqpoint{9.353661in}{2.202207in}}%
\pgfpathcurveto{\pgfqpoint{9.353661in}{2.191156in}}{\pgfqpoint{9.358051in}{2.180557in}}{\pgfqpoint{9.365865in}{2.172744in}}%
\pgfpathcurveto{\pgfqpoint{9.373678in}{2.164930in}}{\pgfqpoint{9.384277in}{2.160540in}}{\pgfqpoint{9.395327in}{2.160540in}}%
\pgfpathlineto{\pgfqpoint{9.395327in}{2.160540in}}%
\pgfpathclose%
\pgfusepath{stroke}%
\end{pgfscope}%
\begin{pgfscope}%
\pgfpathrectangle{\pgfqpoint{7.512535in}{0.437222in}}{\pgfqpoint{6.275590in}{5.159444in}}%
\pgfusepath{clip}%
\pgfsetbuttcap%
\pgfsetroundjoin%
\pgfsetlinewidth{1.003750pt}%
\definecolor{currentstroke}{rgb}{0.827451,0.827451,0.827451}%
\pgfsetstrokecolor{currentstroke}%
\pgfsetstrokeopacity{0.800000}%
\pgfsetdash{}{0pt}%
\pgfpathmoveto{\pgfqpoint{9.581399in}{3.231111in}}%
\pgfpathcurveto{\pgfqpoint{9.592449in}{3.231111in}}{\pgfqpoint{9.603048in}{3.235501in}}{\pgfqpoint{9.610862in}{3.243315in}}%
\pgfpathcurveto{\pgfqpoint{9.618675in}{3.251128in}}{\pgfqpoint{9.623065in}{3.261727in}}{\pgfqpoint{9.623065in}{3.272778in}}%
\pgfpathcurveto{\pgfqpoint{9.623065in}{3.283828in}}{\pgfqpoint{9.618675in}{3.294427in}}{\pgfqpoint{9.610862in}{3.302240in}}%
\pgfpathcurveto{\pgfqpoint{9.603048in}{3.310054in}}{\pgfqpoint{9.592449in}{3.314444in}}{\pgfqpoint{9.581399in}{3.314444in}}%
\pgfpathcurveto{\pgfqpoint{9.570349in}{3.314444in}}{\pgfqpoint{9.559750in}{3.310054in}}{\pgfqpoint{9.551936in}{3.302240in}}%
\pgfpathcurveto{\pgfqpoint{9.544122in}{3.294427in}}{\pgfqpoint{9.539732in}{3.283828in}}{\pgfqpoint{9.539732in}{3.272778in}}%
\pgfpathcurveto{\pgfqpoint{9.539732in}{3.261727in}}{\pgfqpoint{9.544122in}{3.251128in}}{\pgfqpoint{9.551936in}{3.243315in}}%
\pgfpathcurveto{\pgfqpoint{9.559750in}{3.235501in}}{\pgfqpoint{9.570349in}{3.231111in}}{\pgfqpoint{9.581399in}{3.231111in}}%
\pgfpathlineto{\pgfqpoint{9.581399in}{3.231111in}}%
\pgfpathclose%
\pgfusepath{stroke}%
\end{pgfscope}%
\begin{pgfscope}%
\pgfpathrectangle{\pgfqpoint{7.512535in}{0.437222in}}{\pgfqpoint{6.275590in}{5.159444in}}%
\pgfusepath{clip}%
\pgfsetbuttcap%
\pgfsetroundjoin%
\pgfsetlinewidth{1.003750pt}%
\definecolor{currentstroke}{rgb}{0.827451,0.827451,0.827451}%
\pgfsetstrokecolor{currentstroke}%
\pgfsetstrokeopacity{0.800000}%
\pgfsetdash{}{0pt}%
\pgfpathmoveto{\pgfqpoint{9.263397in}{3.235524in}}%
\pgfpathcurveto{\pgfqpoint{9.274447in}{3.235524in}}{\pgfqpoint{9.285046in}{3.239914in}}{\pgfqpoint{9.292860in}{3.247728in}}%
\pgfpathcurveto{\pgfqpoint{9.300674in}{3.255542in}}{\pgfqpoint{9.305064in}{3.266141in}}{\pgfqpoint{9.305064in}{3.277191in}}%
\pgfpathcurveto{\pgfqpoint{9.305064in}{3.288241in}}{\pgfqpoint{9.300674in}{3.298840in}}{\pgfqpoint{9.292860in}{3.306654in}}%
\pgfpathcurveto{\pgfqpoint{9.285046in}{3.314467in}}{\pgfqpoint{9.274447in}{3.318857in}}{\pgfqpoint{9.263397in}{3.318857in}}%
\pgfpathcurveto{\pgfqpoint{9.252347in}{3.318857in}}{\pgfqpoint{9.241748in}{3.314467in}}{\pgfqpoint{9.233934in}{3.306654in}}%
\pgfpathcurveto{\pgfqpoint{9.226121in}{3.298840in}}{\pgfqpoint{9.221730in}{3.288241in}}{\pgfqpoint{9.221730in}{3.277191in}}%
\pgfpathcurveto{\pgfqpoint{9.221730in}{3.266141in}}{\pgfqpoint{9.226121in}{3.255542in}}{\pgfqpoint{9.233934in}{3.247728in}}%
\pgfpathcurveto{\pgfqpoint{9.241748in}{3.239914in}}{\pgfqpoint{9.252347in}{3.235524in}}{\pgfqpoint{9.263397in}{3.235524in}}%
\pgfpathlineto{\pgfqpoint{9.263397in}{3.235524in}}%
\pgfpathclose%
\pgfusepath{stroke}%
\end{pgfscope}%
\begin{pgfscope}%
\pgfpathrectangle{\pgfqpoint{7.512535in}{0.437222in}}{\pgfqpoint{6.275590in}{5.159444in}}%
\pgfusepath{clip}%
\pgfsetbuttcap%
\pgfsetroundjoin%
\pgfsetlinewidth{1.003750pt}%
\definecolor{currentstroke}{rgb}{0.827451,0.827451,0.827451}%
\pgfsetstrokecolor{currentstroke}%
\pgfsetstrokeopacity{0.800000}%
\pgfsetdash{}{0pt}%
\pgfpathmoveto{\pgfqpoint{9.615954in}{3.788157in}}%
\pgfpathcurveto{\pgfqpoint{9.627004in}{3.788157in}}{\pgfqpoint{9.637603in}{3.792548in}}{\pgfqpoint{9.645416in}{3.800361in}}%
\pgfpathcurveto{\pgfqpoint{9.653230in}{3.808175in}}{\pgfqpoint{9.657620in}{3.818774in}}{\pgfqpoint{9.657620in}{3.829824in}}%
\pgfpathcurveto{\pgfqpoint{9.657620in}{3.840874in}}{\pgfqpoint{9.653230in}{3.851473in}}{\pgfqpoint{9.645416in}{3.859287in}}%
\pgfpathcurveto{\pgfqpoint{9.637603in}{3.867100in}}{\pgfqpoint{9.627004in}{3.871491in}}{\pgfqpoint{9.615954in}{3.871491in}}%
\pgfpathcurveto{\pgfqpoint{9.604903in}{3.871491in}}{\pgfqpoint{9.594304in}{3.867100in}}{\pgfqpoint{9.586491in}{3.859287in}}%
\pgfpathcurveto{\pgfqpoint{9.578677in}{3.851473in}}{\pgfqpoint{9.574287in}{3.840874in}}{\pgfqpoint{9.574287in}{3.829824in}}%
\pgfpathcurveto{\pgfqpoint{9.574287in}{3.818774in}}{\pgfqpoint{9.578677in}{3.808175in}}{\pgfqpoint{9.586491in}{3.800361in}}%
\pgfpathcurveto{\pgfqpoint{9.594304in}{3.792548in}}{\pgfqpoint{9.604903in}{3.788157in}}{\pgfqpoint{9.615954in}{3.788157in}}%
\pgfpathlineto{\pgfqpoint{9.615954in}{3.788157in}}%
\pgfpathclose%
\pgfusepath{stroke}%
\end{pgfscope}%
\begin{pgfscope}%
\pgfpathrectangle{\pgfqpoint{7.512535in}{0.437222in}}{\pgfqpoint{6.275590in}{5.159444in}}%
\pgfusepath{clip}%
\pgfsetbuttcap%
\pgfsetroundjoin%
\pgfsetlinewidth{1.003750pt}%
\definecolor{currentstroke}{rgb}{0.827451,0.827451,0.827451}%
\pgfsetstrokecolor{currentstroke}%
\pgfsetstrokeopacity{0.800000}%
\pgfsetdash{}{0pt}%
\pgfpathmoveto{\pgfqpoint{7.562575in}{0.440656in}}%
\pgfpathcurveto{\pgfqpoint{7.573625in}{0.440656in}}{\pgfqpoint{7.584224in}{0.445046in}}{\pgfqpoint{7.592038in}{0.452860in}}%
\pgfpathcurveto{\pgfqpoint{7.599851in}{0.460673in}}{\pgfqpoint{7.604241in}{0.471272in}}{\pgfqpoint{7.604241in}{0.482323in}}%
\pgfpathcurveto{\pgfqpoint{7.604241in}{0.493373in}}{\pgfqpoint{7.599851in}{0.503972in}}{\pgfqpoint{7.592038in}{0.511785in}}%
\pgfpathcurveto{\pgfqpoint{7.584224in}{0.519599in}}{\pgfqpoint{7.573625in}{0.523989in}}{\pgfqpoint{7.562575in}{0.523989in}}%
\pgfpathcurveto{\pgfqpoint{7.551525in}{0.523989in}}{\pgfqpoint{7.540926in}{0.519599in}}{\pgfqpoint{7.533112in}{0.511785in}}%
\pgfpathcurveto{\pgfqpoint{7.525298in}{0.503972in}}{\pgfqpoint{7.520908in}{0.493373in}}{\pgfqpoint{7.520908in}{0.482323in}}%
\pgfpathcurveto{\pgfqpoint{7.520908in}{0.471272in}}{\pgfqpoint{7.525298in}{0.460673in}}{\pgfqpoint{7.533112in}{0.452860in}}%
\pgfpathcurveto{\pgfqpoint{7.540926in}{0.445046in}}{\pgfqpoint{7.551525in}{0.440656in}}{\pgfqpoint{7.562575in}{0.440656in}}%
\pgfpathlineto{\pgfqpoint{7.562575in}{0.440656in}}%
\pgfpathclose%
\pgfusepath{stroke}%
\end{pgfscope}%
\begin{pgfscope}%
\pgfpathrectangle{\pgfqpoint{7.512535in}{0.437222in}}{\pgfqpoint{6.275590in}{5.159444in}}%
\pgfusepath{clip}%
\pgfsetbuttcap%
\pgfsetroundjoin%
\pgfsetlinewidth{1.003750pt}%
\definecolor{currentstroke}{rgb}{0.827451,0.827451,0.827451}%
\pgfsetstrokecolor{currentstroke}%
\pgfsetstrokeopacity{0.800000}%
\pgfsetdash{}{0pt}%
\pgfpathmoveto{\pgfqpoint{13.731324in}{5.553845in}}%
\pgfpathcurveto{\pgfqpoint{13.742374in}{5.553845in}}{\pgfqpoint{13.752973in}{5.558236in}}{\pgfqpoint{13.760787in}{5.566049in}}%
\pgfpathcurveto{\pgfqpoint{13.768601in}{5.573863in}}{\pgfqpoint{13.772991in}{5.584462in}}{\pgfqpoint{13.772991in}{5.595512in}}%
\pgfpathcurveto{\pgfqpoint{13.772991in}{5.606562in}}{\pgfqpoint{13.768601in}{5.617161in}}{\pgfqpoint{13.760787in}{5.624975in}}%
\pgfpathcurveto{\pgfqpoint{13.752973in}{5.632788in}}{\pgfqpoint{13.742374in}{5.637179in}}{\pgfqpoint{13.731324in}{5.637179in}}%
\pgfpathcurveto{\pgfqpoint{13.720274in}{5.637179in}}{\pgfqpoint{13.709675in}{5.632788in}}{\pgfqpoint{13.701861in}{5.624975in}}%
\pgfpathcurveto{\pgfqpoint{13.694048in}{5.617161in}}{\pgfqpoint{13.689657in}{5.606562in}}{\pgfqpoint{13.689657in}{5.595512in}}%
\pgfpathcurveto{\pgfqpoint{13.689657in}{5.584462in}}{\pgfqpoint{13.694048in}{5.573863in}}{\pgfqpoint{13.701861in}{5.566049in}}%
\pgfpathcurveto{\pgfqpoint{13.709675in}{5.558236in}}{\pgfqpoint{13.720274in}{5.553845in}}{\pgfqpoint{13.731324in}{5.553845in}}%
\pgfpathlineto{\pgfqpoint{13.731324in}{5.553845in}}%
\pgfpathclose%
\pgfusepath{stroke}%
\end{pgfscope}%
\begin{pgfscope}%
\pgfpathrectangle{\pgfqpoint{7.512535in}{0.437222in}}{\pgfqpoint{6.275590in}{5.159444in}}%
\pgfusepath{clip}%
\pgfsetbuttcap%
\pgfsetroundjoin%
\pgfsetlinewidth{1.003750pt}%
\definecolor{currentstroke}{rgb}{0.827451,0.827451,0.827451}%
\pgfsetstrokecolor{currentstroke}%
\pgfsetstrokeopacity{0.800000}%
\pgfsetdash{}{0pt}%
\pgfpathmoveto{\pgfqpoint{7.711198in}{0.615504in}}%
\pgfpathcurveto{\pgfqpoint{7.722248in}{0.615504in}}{\pgfqpoint{7.732847in}{0.619894in}}{\pgfqpoint{7.740661in}{0.627708in}}%
\pgfpathcurveto{\pgfqpoint{7.748474in}{0.635522in}}{\pgfqpoint{7.752865in}{0.646121in}}{\pgfqpoint{7.752865in}{0.657171in}}%
\pgfpathcurveto{\pgfqpoint{7.752865in}{0.668221in}}{\pgfqpoint{7.748474in}{0.678820in}}{\pgfqpoint{7.740661in}{0.686633in}}%
\pgfpathcurveto{\pgfqpoint{7.732847in}{0.694447in}}{\pgfqpoint{7.722248in}{0.698837in}}{\pgfqpoint{7.711198in}{0.698837in}}%
\pgfpathcurveto{\pgfqpoint{7.700148in}{0.698837in}}{\pgfqpoint{7.689549in}{0.694447in}}{\pgfqpoint{7.681735in}{0.686633in}}%
\pgfpathcurveto{\pgfqpoint{7.673921in}{0.678820in}}{\pgfqpoint{7.669531in}{0.668221in}}{\pgfqpoint{7.669531in}{0.657171in}}%
\pgfpathcurveto{\pgfqpoint{7.669531in}{0.646121in}}{\pgfqpoint{7.673921in}{0.635522in}}{\pgfqpoint{7.681735in}{0.627708in}}%
\pgfpathcurveto{\pgfqpoint{7.689549in}{0.619894in}}{\pgfqpoint{7.700148in}{0.615504in}}{\pgfqpoint{7.711198in}{0.615504in}}%
\pgfpathlineto{\pgfqpoint{7.711198in}{0.615504in}}%
\pgfpathclose%
\pgfusepath{stroke}%
\end{pgfscope}%
\begin{pgfscope}%
\pgfpathrectangle{\pgfqpoint{7.512535in}{0.437222in}}{\pgfqpoint{6.275590in}{5.159444in}}%
\pgfusepath{clip}%
\pgfsetbuttcap%
\pgfsetroundjoin%
\pgfsetlinewidth{1.003750pt}%
\definecolor{currentstroke}{rgb}{0.827451,0.827451,0.827451}%
\pgfsetstrokecolor{currentstroke}%
\pgfsetstrokeopacity{0.800000}%
\pgfsetdash{}{0pt}%
\pgfpathmoveto{\pgfqpoint{10.836356in}{5.524317in}}%
\pgfpathcurveto{\pgfqpoint{10.847406in}{5.524317in}}{\pgfqpoint{10.858005in}{5.528707in}}{\pgfqpoint{10.865818in}{5.536521in}}%
\pgfpathcurveto{\pgfqpoint{10.873632in}{5.544335in}}{\pgfqpoint{10.878022in}{5.554934in}}{\pgfqpoint{10.878022in}{5.565984in}}%
\pgfpathcurveto{\pgfqpoint{10.878022in}{5.577034in}}{\pgfqpoint{10.873632in}{5.587633in}}{\pgfqpoint{10.865818in}{5.595446in}}%
\pgfpathcurveto{\pgfqpoint{10.858005in}{5.603260in}}{\pgfqpoint{10.847406in}{5.607650in}}{\pgfqpoint{10.836356in}{5.607650in}}%
\pgfpathcurveto{\pgfqpoint{10.825305in}{5.607650in}}{\pgfqpoint{10.814706in}{5.603260in}}{\pgfqpoint{10.806893in}{5.595446in}}%
\pgfpathcurveto{\pgfqpoint{10.799079in}{5.587633in}}{\pgfqpoint{10.794689in}{5.577034in}}{\pgfqpoint{10.794689in}{5.565984in}}%
\pgfpathcurveto{\pgfqpoint{10.794689in}{5.554934in}}{\pgfqpoint{10.799079in}{5.544335in}}{\pgfqpoint{10.806893in}{5.536521in}}%
\pgfpathcurveto{\pgfqpoint{10.814706in}{5.528707in}}{\pgfqpoint{10.825305in}{5.524317in}}{\pgfqpoint{10.836356in}{5.524317in}}%
\pgfpathlineto{\pgfqpoint{10.836356in}{5.524317in}}%
\pgfpathclose%
\pgfusepath{stroke}%
\end{pgfscope}%
\begin{pgfscope}%
\pgfpathrectangle{\pgfqpoint{7.512535in}{0.437222in}}{\pgfqpoint{6.275590in}{5.159444in}}%
\pgfusepath{clip}%
\pgfsetbuttcap%
\pgfsetroundjoin%
\pgfsetlinewidth{1.003750pt}%
\definecolor{currentstroke}{rgb}{0.827451,0.827451,0.827451}%
\pgfsetstrokecolor{currentstroke}%
\pgfsetstrokeopacity{0.800000}%
\pgfsetdash{}{0pt}%
\pgfpathmoveto{\pgfqpoint{10.073045in}{4.318404in}}%
\pgfpathcurveto{\pgfqpoint{10.084095in}{4.318404in}}{\pgfqpoint{10.094694in}{4.322795in}}{\pgfqpoint{10.102508in}{4.330608in}}%
\pgfpathcurveto{\pgfqpoint{10.110321in}{4.338422in}}{\pgfqpoint{10.114712in}{4.349021in}}{\pgfqpoint{10.114712in}{4.360071in}}%
\pgfpathcurveto{\pgfqpoint{10.114712in}{4.371121in}}{\pgfqpoint{10.110321in}{4.381720in}}{\pgfqpoint{10.102508in}{4.389534in}}%
\pgfpathcurveto{\pgfqpoint{10.094694in}{4.397348in}}{\pgfqpoint{10.084095in}{4.401738in}}{\pgfqpoint{10.073045in}{4.401738in}}%
\pgfpathcurveto{\pgfqpoint{10.061995in}{4.401738in}}{\pgfqpoint{10.051396in}{4.397348in}}{\pgfqpoint{10.043582in}{4.389534in}}%
\pgfpathcurveto{\pgfqpoint{10.035769in}{4.381720in}}{\pgfqpoint{10.031378in}{4.371121in}}{\pgfqpoint{10.031378in}{4.360071in}}%
\pgfpathcurveto{\pgfqpoint{10.031378in}{4.349021in}}{\pgfqpoint{10.035769in}{4.338422in}}{\pgfqpoint{10.043582in}{4.330608in}}%
\pgfpathcurveto{\pgfqpoint{10.051396in}{4.322795in}}{\pgfqpoint{10.061995in}{4.318404in}}{\pgfqpoint{10.073045in}{4.318404in}}%
\pgfpathlineto{\pgfqpoint{10.073045in}{4.318404in}}%
\pgfpathclose%
\pgfusepath{stroke}%
\end{pgfscope}%
\begin{pgfscope}%
\pgfpathrectangle{\pgfqpoint{7.512535in}{0.437222in}}{\pgfqpoint{6.275590in}{5.159444in}}%
\pgfusepath{clip}%
\pgfsetbuttcap%
\pgfsetroundjoin%
\pgfsetlinewidth{1.003750pt}%
\definecolor{currentstroke}{rgb}{0.827451,0.827451,0.827451}%
\pgfsetstrokecolor{currentstroke}%
\pgfsetstrokeopacity{0.800000}%
\pgfsetdash{}{0pt}%
\pgfpathmoveto{\pgfqpoint{12.233224in}{5.446777in}}%
\pgfpathcurveto{\pgfqpoint{12.244274in}{5.446777in}}{\pgfqpoint{12.254873in}{5.451168in}}{\pgfqpoint{12.262687in}{5.458981in}}%
\pgfpathcurveto{\pgfqpoint{12.270500in}{5.466795in}}{\pgfqpoint{12.274891in}{5.477394in}}{\pgfqpoint{12.274891in}{5.488444in}}%
\pgfpathcurveto{\pgfqpoint{12.274891in}{5.499494in}}{\pgfqpoint{12.270500in}{5.510093in}}{\pgfqpoint{12.262687in}{5.517907in}}%
\pgfpathcurveto{\pgfqpoint{12.254873in}{5.525720in}}{\pgfqpoint{12.244274in}{5.530111in}}{\pgfqpoint{12.233224in}{5.530111in}}%
\pgfpathcurveto{\pgfqpoint{12.222174in}{5.530111in}}{\pgfqpoint{12.211575in}{5.525720in}}{\pgfqpoint{12.203761in}{5.517907in}}%
\pgfpathcurveto{\pgfqpoint{12.195948in}{5.510093in}}{\pgfqpoint{12.191557in}{5.499494in}}{\pgfqpoint{12.191557in}{5.488444in}}%
\pgfpathcurveto{\pgfqpoint{12.191557in}{5.477394in}}{\pgfqpoint{12.195948in}{5.466795in}}{\pgfqpoint{12.203761in}{5.458981in}}%
\pgfpathcurveto{\pgfqpoint{12.211575in}{5.451168in}}{\pgfqpoint{12.222174in}{5.446777in}}{\pgfqpoint{12.233224in}{5.446777in}}%
\pgfpathlineto{\pgfqpoint{12.233224in}{5.446777in}}%
\pgfpathclose%
\pgfusepath{stroke}%
\end{pgfscope}%
\begin{pgfscope}%
\pgfpathrectangle{\pgfqpoint{7.512535in}{0.437222in}}{\pgfqpoint{6.275590in}{5.159444in}}%
\pgfusepath{clip}%
\pgfsetbuttcap%
\pgfsetroundjoin%
\pgfsetlinewidth{1.003750pt}%
\definecolor{currentstroke}{rgb}{0.827451,0.827451,0.827451}%
\pgfsetstrokecolor{currentstroke}%
\pgfsetstrokeopacity{0.800000}%
\pgfsetdash{}{0pt}%
\pgfpathmoveto{\pgfqpoint{9.928499in}{4.056499in}}%
\pgfpathcurveto{\pgfqpoint{9.939549in}{4.056499in}}{\pgfqpoint{9.950148in}{4.060889in}}{\pgfqpoint{9.957962in}{4.068703in}}%
\pgfpathcurveto{\pgfqpoint{9.965775in}{4.076516in}}{\pgfqpoint{9.970165in}{4.087115in}}{\pgfqpoint{9.970165in}{4.098166in}}%
\pgfpathcurveto{\pgfqpoint{9.970165in}{4.109216in}}{\pgfqpoint{9.965775in}{4.119815in}}{\pgfqpoint{9.957962in}{4.127628in}}%
\pgfpathcurveto{\pgfqpoint{9.950148in}{4.135442in}}{\pgfqpoint{9.939549in}{4.139832in}}{\pgfqpoint{9.928499in}{4.139832in}}%
\pgfpathcurveto{\pgfqpoint{9.917449in}{4.139832in}}{\pgfqpoint{9.906850in}{4.135442in}}{\pgfqpoint{9.899036in}{4.127628in}}%
\pgfpathcurveto{\pgfqpoint{9.891222in}{4.119815in}}{\pgfqpoint{9.886832in}{4.109216in}}{\pgfqpoint{9.886832in}{4.098166in}}%
\pgfpathcurveto{\pgfqpoint{9.886832in}{4.087115in}}{\pgfqpoint{9.891222in}{4.076516in}}{\pgfqpoint{9.899036in}{4.068703in}}%
\pgfpathcurveto{\pgfqpoint{9.906850in}{4.060889in}}{\pgfqpoint{9.917449in}{4.056499in}}{\pgfqpoint{9.928499in}{4.056499in}}%
\pgfpathlineto{\pgfqpoint{9.928499in}{4.056499in}}%
\pgfpathclose%
\pgfusepath{stroke}%
\end{pgfscope}%
\begin{pgfscope}%
\pgfpathrectangle{\pgfqpoint{7.512535in}{0.437222in}}{\pgfqpoint{6.275590in}{5.159444in}}%
\pgfusepath{clip}%
\pgfsetbuttcap%
\pgfsetroundjoin%
\pgfsetlinewidth{1.003750pt}%
\definecolor{currentstroke}{rgb}{0.827451,0.827451,0.827451}%
\pgfsetstrokecolor{currentstroke}%
\pgfsetstrokeopacity{0.800000}%
\pgfsetdash{}{0pt}%
\pgfpathmoveto{\pgfqpoint{13.250415in}{5.553845in}}%
\pgfpathcurveto{\pgfqpoint{13.261465in}{5.553845in}}{\pgfqpoint{13.272064in}{5.558236in}}{\pgfqpoint{13.279878in}{5.566049in}}%
\pgfpathcurveto{\pgfqpoint{13.287691in}{5.573863in}}{\pgfqpoint{13.292082in}{5.584462in}}{\pgfqpoint{13.292082in}{5.595512in}}%
\pgfpathcurveto{\pgfqpoint{13.292082in}{5.606562in}}{\pgfqpoint{13.287691in}{5.617161in}}{\pgfqpoint{13.279878in}{5.624975in}}%
\pgfpathcurveto{\pgfqpoint{13.272064in}{5.632788in}}{\pgfqpoint{13.261465in}{5.637179in}}{\pgfqpoint{13.250415in}{5.637179in}}%
\pgfpathcurveto{\pgfqpoint{13.239365in}{5.637179in}}{\pgfqpoint{13.228766in}{5.632788in}}{\pgfqpoint{13.220952in}{5.624975in}}%
\pgfpathcurveto{\pgfqpoint{13.213139in}{5.617161in}}{\pgfqpoint{13.208748in}{5.606562in}}{\pgfqpoint{13.208748in}{5.595512in}}%
\pgfpathcurveto{\pgfqpoint{13.208748in}{5.584462in}}{\pgfqpoint{13.213139in}{5.573863in}}{\pgfqpoint{13.220952in}{5.566049in}}%
\pgfpathcurveto{\pgfqpoint{13.228766in}{5.558236in}}{\pgfqpoint{13.239365in}{5.553845in}}{\pgfqpoint{13.250415in}{5.553845in}}%
\pgfpathlineto{\pgfqpoint{13.250415in}{5.553845in}}%
\pgfpathclose%
\pgfusepath{stroke}%
\end{pgfscope}%
\begin{pgfscope}%
\pgfpathrectangle{\pgfqpoint{7.512535in}{0.437222in}}{\pgfqpoint{6.275590in}{5.159444in}}%
\pgfusepath{clip}%
\pgfsetbuttcap%
\pgfsetroundjoin%
\pgfsetlinewidth{1.003750pt}%
\definecolor{currentstroke}{rgb}{0.827451,0.827451,0.827451}%
\pgfsetstrokecolor{currentstroke}%
\pgfsetstrokeopacity{0.800000}%
\pgfsetdash{}{0pt}%
\pgfpathmoveto{\pgfqpoint{12.955580in}{5.537440in}}%
\pgfpathcurveto{\pgfqpoint{12.966630in}{5.537440in}}{\pgfqpoint{12.977229in}{5.541830in}}{\pgfqpoint{12.985043in}{5.549643in}}%
\pgfpathcurveto{\pgfqpoint{12.992856in}{5.557457in}}{\pgfqpoint{12.997246in}{5.568056in}}{\pgfqpoint{12.997246in}{5.579106in}}%
\pgfpathcurveto{\pgfqpoint{12.997246in}{5.590156in}}{\pgfqpoint{12.992856in}{5.600755in}}{\pgfqpoint{12.985043in}{5.608569in}}%
\pgfpathcurveto{\pgfqpoint{12.977229in}{5.616383in}}{\pgfqpoint{12.966630in}{5.620773in}}{\pgfqpoint{12.955580in}{5.620773in}}%
\pgfpathcurveto{\pgfqpoint{12.944530in}{5.620773in}}{\pgfqpoint{12.933931in}{5.616383in}}{\pgfqpoint{12.926117in}{5.608569in}}%
\pgfpathcurveto{\pgfqpoint{12.918303in}{5.600755in}}{\pgfqpoint{12.913913in}{5.590156in}}{\pgfqpoint{12.913913in}{5.579106in}}%
\pgfpathcurveto{\pgfqpoint{12.913913in}{5.568056in}}{\pgfqpoint{12.918303in}{5.557457in}}{\pgfqpoint{12.926117in}{5.549643in}}%
\pgfpathcurveto{\pgfqpoint{12.933931in}{5.541830in}}{\pgfqpoint{12.944530in}{5.537440in}}{\pgfqpoint{12.955580in}{5.537440in}}%
\pgfpathlineto{\pgfqpoint{12.955580in}{5.537440in}}%
\pgfpathclose%
\pgfusepath{stroke}%
\end{pgfscope}%
\begin{pgfscope}%
\pgfpathrectangle{\pgfqpoint{7.512535in}{0.437222in}}{\pgfqpoint{6.275590in}{5.159444in}}%
\pgfusepath{clip}%
\pgfsetbuttcap%
\pgfsetroundjoin%
\pgfsetlinewidth{1.003750pt}%
\definecolor{currentstroke}{rgb}{0.827451,0.827451,0.827451}%
\pgfsetstrokecolor{currentstroke}%
\pgfsetstrokeopacity{0.800000}%
\pgfsetdash{}{0pt}%
\pgfpathmoveto{\pgfqpoint{10.450205in}{5.521083in}}%
\pgfpathcurveto{\pgfqpoint{10.461255in}{5.521083in}}{\pgfqpoint{10.471854in}{5.525473in}}{\pgfqpoint{10.479667in}{5.533287in}}%
\pgfpathcurveto{\pgfqpoint{10.487481in}{5.541100in}}{\pgfqpoint{10.491871in}{5.551699in}}{\pgfqpoint{10.491871in}{5.562749in}}%
\pgfpathcurveto{\pgfqpoint{10.491871in}{5.573800in}}{\pgfqpoint{10.487481in}{5.584399in}}{\pgfqpoint{10.479667in}{5.592212in}}%
\pgfpathcurveto{\pgfqpoint{10.471854in}{5.600026in}}{\pgfqpoint{10.461255in}{5.604416in}}{\pgfqpoint{10.450205in}{5.604416in}}%
\pgfpathcurveto{\pgfqpoint{10.439155in}{5.604416in}}{\pgfqpoint{10.428556in}{5.600026in}}{\pgfqpoint{10.420742in}{5.592212in}}%
\pgfpathcurveto{\pgfqpoint{10.412928in}{5.584399in}}{\pgfqpoint{10.408538in}{5.573800in}}{\pgfqpoint{10.408538in}{5.562749in}}%
\pgfpathcurveto{\pgfqpoint{10.408538in}{5.551699in}}{\pgfqpoint{10.412928in}{5.541100in}}{\pgfqpoint{10.420742in}{5.533287in}}%
\pgfpathcurveto{\pgfqpoint{10.428556in}{5.525473in}}{\pgfqpoint{10.439155in}{5.521083in}}{\pgfqpoint{10.450205in}{5.521083in}}%
\pgfpathlineto{\pgfqpoint{10.450205in}{5.521083in}}%
\pgfpathclose%
\pgfusepath{stroke}%
\end{pgfscope}%
\begin{pgfscope}%
\pgfpathrectangle{\pgfqpoint{7.512535in}{0.437222in}}{\pgfqpoint{6.275590in}{5.159444in}}%
\pgfusepath{clip}%
\pgfsetbuttcap%
\pgfsetroundjoin%
\pgfsetlinewidth{1.003750pt}%
\definecolor{currentstroke}{rgb}{0.827451,0.827451,0.827451}%
\pgfsetstrokecolor{currentstroke}%
\pgfsetstrokeopacity{0.800000}%
\pgfsetdash{}{0pt}%
\pgfpathmoveto{\pgfqpoint{11.806382in}{5.471994in}}%
\pgfpathcurveto{\pgfqpoint{11.817432in}{5.471994in}}{\pgfqpoint{11.828031in}{5.476384in}}{\pgfqpoint{11.835845in}{5.484198in}}%
\pgfpathcurveto{\pgfqpoint{11.843658in}{5.492011in}}{\pgfqpoint{11.848049in}{5.502610in}}{\pgfqpoint{11.848049in}{5.513660in}}%
\pgfpathcurveto{\pgfqpoint{11.848049in}{5.524711in}}{\pgfqpoint{11.843658in}{5.535310in}}{\pgfqpoint{11.835845in}{5.543123in}}%
\pgfpathcurveto{\pgfqpoint{11.828031in}{5.550937in}}{\pgfqpoint{11.817432in}{5.555327in}}{\pgfqpoint{11.806382in}{5.555327in}}%
\pgfpathcurveto{\pgfqpoint{11.795332in}{5.555327in}}{\pgfqpoint{11.784733in}{5.550937in}}{\pgfqpoint{11.776919in}{5.543123in}}%
\pgfpathcurveto{\pgfqpoint{11.769106in}{5.535310in}}{\pgfqpoint{11.764715in}{5.524711in}}{\pgfqpoint{11.764715in}{5.513660in}}%
\pgfpathcurveto{\pgfqpoint{11.764715in}{5.502610in}}{\pgfqpoint{11.769106in}{5.492011in}}{\pgfqpoint{11.776919in}{5.484198in}}%
\pgfpathcurveto{\pgfqpoint{11.784733in}{5.476384in}}{\pgfqpoint{11.795332in}{5.471994in}}{\pgfqpoint{11.806382in}{5.471994in}}%
\pgfpathlineto{\pgfqpoint{11.806382in}{5.471994in}}%
\pgfpathclose%
\pgfusepath{stroke}%
\end{pgfscope}%
\begin{pgfscope}%
\pgfpathrectangle{\pgfqpoint{7.512535in}{0.437222in}}{\pgfqpoint{6.275590in}{5.159444in}}%
\pgfusepath{clip}%
\pgfsetbuttcap%
\pgfsetroundjoin%
\pgfsetlinewidth{1.003750pt}%
\definecolor{currentstroke}{rgb}{0.827451,0.827451,0.827451}%
\pgfsetstrokecolor{currentstroke}%
\pgfsetstrokeopacity{0.800000}%
\pgfsetdash{}{0pt}%
\pgfpathmoveto{\pgfqpoint{12.677892in}{5.550102in}}%
\pgfpathcurveto{\pgfqpoint{12.688942in}{5.550102in}}{\pgfqpoint{12.699541in}{5.554492in}}{\pgfqpoint{12.707354in}{5.562306in}}%
\pgfpathcurveto{\pgfqpoint{12.715168in}{5.570119in}}{\pgfqpoint{12.719558in}{5.580718in}}{\pgfqpoint{12.719558in}{5.591769in}}%
\pgfpathcurveto{\pgfqpoint{12.719558in}{5.602819in}}{\pgfqpoint{12.715168in}{5.613418in}}{\pgfqpoint{12.707354in}{5.621231in}}%
\pgfpathcurveto{\pgfqpoint{12.699541in}{5.629045in}}{\pgfqpoint{12.688942in}{5.633435in}}{\pgfqpoint{12.677892in}{5.633435in}}%
\pgfpathcurveto{\pgfqpoint{12.666841in}{5.633435in}}{\pgfqpoint{12.656242in}{5.629045in}}{\pgfqpoint{12.648429in}{5.621231in}}%
\pgfpathcurveto{\pgfqpoint{12.640615in}{5.613418in}}{\pgfqpoint{12.636225in}{5.602819in}}{\pgfqpoint{12.636225in}{5.591769in}}%
\pgfpathcurveto{\pgfqpoint{12.636225in}{5.580718in}}{\pgfqpoint{12.640615in}{5.570119in}}{\pgfqpoint{12.648429in}{5.562306in}}%
\pgfpathcurveto{\pgfqpoint{12.656242in}{5.554492in}}{\pgfqpoint{12.666841in}{5.550102in}}{\pgfqpoint{12.677892in}{5.550102in}}%
\pgfpathlineto{\pgfqpoint{12.677892in}{5.550102in}}%
\pgfpathclose%
\pgfusepath{stroke}%
\end{pgfscope}%
\begin{pgfscope}%
\pgfpathrectangle{\pgfqpoint{7.512535in}{0.437222in}}{\pgfqpoint{6.275590in}{5.159444in}}%
\pgfusepath{clip}%
\pgfsetbuttcap%
\pgfsetroundjoin%
\pgfsetlinewidth{1.003750pt}%
\definecolor{currentstroke}{rgb}{0.827451,0.827451,0.827451}%
\pgfsetstrokecolor{currentstroke}%
\pgfsetstrokeopacity{0.800000}%
\pgfsetdash{}{0pt}%
\pgfpathmoveto{\pgfqpoint{10.994755in}{5.060707in}}%
\pgfpathcurveto{\pgfqpoint{11.005805in}{5.060707in}}{\pgfqpoint{11.016404in}{5.065097in}}{\pgfqpoint{11.024218in}{5.072911in}}%
\pgfpathcurveto{\pgfqpoint{11.032031in}{5.080725in}}{\pgfqpoint{11.036422in}{5.091324in}}{\pgfqpoint{11.036422in}{5.102374in}}%
\pgfpathcurveto{\pgfqpoint{11.036422in}{5.113424in}}{\pgfqpoint{11.032031in}{5.124023in}}{\pgfqpoint{11.024218in}{5.131837in}}%
\pgfpathcurveto{\pgfqpoint{11.016404in}{5.139650in}}{\pgfqpoint{11.005805in}{5.144040in}}{\pgfqpoint{10.994755in}{5.144040in}}%
\pgfpathcurveto{\pgfqpoint{10.983705in}{5.144040in}}{\pgfqpoint{10.973106in}{5.139650in}}{\pgfqpoint{10.965292in}{5.131837in}}%
\pgfpathcurveto{\pgfqpoint{10.957479in}{5.124023in}}{\pgfqpoint{10.953088in}{5.113424in}}{\pgfqpoint{10.953088in}{5.102374in}}%
\pgfpathcurveto{\pgfqpoint{10.953088in}{5.091324in}}{\pgfqpoint{10.957479in}{5.080725in}}{\pgfqpoint{10.965292in}{5.072911in}}%
\pgfpathcurveto{\pgfqpoint{10.973106in}{5.065097in}}{\pgfqpoint{10.983705in}{5.060707in}}{\pgfqpoint{10.994755in}{5.060707in}}%
\pgfpathlineto{\pgfqpoint{10.994755in}{5.060707in}}%
\pgfpathclose%
\pgfusepath{stroke}%
\end{pgfscope}%
\begin{pgfscope}%
\pgfpathrectangle{\pgfqpoint{7.512535in}{0.437222in}}{\pgfqpoint{6.275590in}{5.159444in}}%
\pgfusepath{clip}%
\pgfsetbuttcap%
\pgfsetroundjoin%
\pgfsetlinewidth{1.003750pt}%
\definecolor{currentstroke}{rgb}{0.827451,0.827451,0.827451}%
\pgfsetstrokecolor{currentstroke}%
\pgfsetstrokeopacity{0.800000}%
\pgfsetdash{}{0pt}%
\pgfpathmoveto{\pgfqpoint{10.689319in}{4.486189in}}%
\pgfpathcurveto{\pgfqpoint{10.700369in}{4.486189in}}{\pgfqpoint{10.710968in}{4.490579in}}{\pgfqpoint{10.718782in}{4.498393in}}%
\pgfpathcurveto{\pgfqpoint{10.726595in}{4.506206in}}{\pgfqpoint{10.730986in}{4.516805in}}{\pgfqpoint{10.730986in}{4.527856in}}%
\pgfpathcurveto{\pgfqpoint{10.730986in}{4.538906in}}{\pgfqpoint{10.726595in}{4.549505in}}{\pgfqpoint{10.718782in}{4.557318in}}%
\pgfpathcurveto{\pgfqpoint{10.710968in}{4.565132in}}{\pgfqpoint{10.700369in}{4.569522in}}{\pgfqpoint{10.689319in}{4.569522in}}%
\pgfpathcurveto{\pgfqpoint{10.678269in}{4.569522in}}{\pgfqpoint{10.667670in}{4.565132in}}{\pgfqpoint{10.659856in}{4.557318in}}%
\pgfpathcurveto{\pgfqpoint{10.652043in}{4.549505in}}{\pgfqpoint{10.647652in}{4.538906in}}{\pgfqpoint{10.647652in}{4.527856in}}%
\pgfpathcurveto{\pgfqpoint{10.647652in}{4.516805in}}{\pgfqpoint{10.652043in}{4.506206in}}{\pgfqpoint{10.659856in}{4.498393in}}%
\pgfpathcurveto{\pgfqpoint{10.667670in}{4.490579in}}{\pgfqpoint{10.678269in}{4.486189in}}{\pgfqpoint{10.689319in}{4.486189in}}%
\pgfpathlineto{\pgfqpoint{10.689319in}{4.486189in}}%
\pgfpathclose%
\pgfusepath{stroke}%
\end{pgfscope}%
\begin{pgfscope}%
\pgfpathrectangle{\pgfqpoint{7.512535in}{0.437222in}}{\pgfqpoint{6.275590in}{5.159444in}}%
\pgfusepath{clip}%
\pgfsetbuttcap%
\pgfsetroundjoin%
\pgfsetlinewidth{1.003750pt}%
\definecolor{currentstroke}{rgb}{0.827451,0.827451,0.827451}%
\pgfsetstrokecolor{currentstroke}%
\pgfsetstrokeopacity{0.800000}%
\pgfsetdash{}{0pt}%
\pgfpathmoveto{\pgfqpoint{10.140717in}{3.555415in}}%
\pgfpathcurveto{\pgfqpoint{10.151767in}{3.555415in}}{\pgfqpoint{10.162367in}{3.559805in}}{\pgfqpoint{10.170180in}{3.567618in}}%
\pgfpathcurveto{\pgfqpoint{10.177994in}{3.575432in}}{\pgfqpoint{10.182384in}{3.586031in}}{\pgfqpoint{10.182384in}{3.597081in}}%
\pgfpathcurveto{\pgfqpoint{10.182384in}{3.608131in}}{\pgfqpoint{10.177994in}{3.618730in}}{\pgfqpoint{10.170180in}{3.626544in}}%
\pgfpathcurveto{\pgfqpoint{10.162367in}{3.634358in}}{\pgfqpoint{10.151767in}{3.638748in}}{\pgfqpoint{10.140717in}{3.638748in}}%
\pgfpathcurveto{\pgfqpoint{10.129667in}{3.638748in}}{\pgfqpoint{10.119068in}{3.634358in}}{\pgfqpoint{10.111255in}{3.626544in}}%
\pgfpathcurveto{\pgfqpoint{10.103441in}{3.618730in}}{\pgfqpoint{10.099051in}{3.608131in}}{\pgfqpoint{10.099051in}{3.597081in}}%
\pgfpathcurveto{\pgfqpoint{10.099051in}{3.586031in}}{\pgfqpoint{10.103441in}{3.575432in}}{\pgfqpoint{10.111255in}{3.567618in}}%
\pgfpathcurveto{\pgfqpoint{10.119068in}{3.559805in}}{\pgfqpoint{10.129667in}{3.555415in}}{\pgfqpoint{10.140717in}{3.555415in}}%
\pgfpathlineto{\pgfqpoint{10.140717in}{3.555415in}}%
\pgfpathclose%
\pgfusepath{stroke}%
\end{pgfscope}%
\begin{pgfscope}%
\pgfpathrectangle{\pgfqpoint{7.512535in}{0.437222in}}{\pgfqpoint{6.275590in}{5.159444in}}%
\pgfusepath{clip}%
\pgfsetbuttcap%
\pgfsetroundjoin%
\pgfsetlinewidth{1.003750pt}%
\definecolor{currentstroke}{rgb}{0.827451,0.827451,0.827451}%
\pgfsetstrokecolor{currentstroke}%
\pgfsetstrokeopacity{0.800000}%
\pgfsetdash{}{0pt}%
\pgfpathmoveto{\pgfqpoint{9.027094in}{2.556537in}}%
\pgfpathcurveto{\pgfqpoint{9.038145in}{2.556537in}}{\pgfqpoint{9.048744in}{2.560927in}}{\pgfqpoint{9.056557in}{2.568740in}}%
\pgfpathcurveto{\pgfqpoint{9.064371in}{2.576554in}}{\pgfqpoint{9.068761in}{2.587153in}}{\pgfqpoint{9.068761in}{2.598203in}}%
\pgfpathcurveto{\pgfqpoint{9.068761in}{2.609253in}}{\pgfqpoint{9.064371in}{2.619852in}}{\pgfqpoint{9.056557in}{2.627666in}}%
\pgfpathcurveto{\pgfqpoint{9.048744in}{2.635480in}}{\pgfqpoint{9.038145in}{2.639870in}}{\pgfqpoint{9.027094in}{2.639870in}}%
\pgfpathcurveto{\pgfqpoint{9.016044in}{2.639870in}}{\pgfqpoint{9.005445in}{2.635480in}}{\pgfqpoint{8.997632in}{2.627666in}}%
\pgfpathcurveto{\pgfqpoint{8.989818in}{2.619852in}}{\pgfqpoint{8.985428in}{2.609253in}}{\pgfqpoint{8.985428in}{2.598203in}}%
\pgfpathcurveto{\pgfqpoint{8.985428in}{2.587153in}}{\pgfqpoint{8.989818in}{2.576554in}}{\pgfqpoint{8.997632in}{2.568740in}}%
\pgfpathcurveto{\pgfqpoint{9.005445in}{2.560927in}}{\pgfqpoint{9.016044in}{2.556537in}}{\pgfqpoint{9.027094in}{2.556537in}}%
\pgfpathlineto{\pgfqpoint{9.027094in}{2.556537in}}%
\pgfpathclose%
\pgfusepath{stroke}%
\end{pgfscope}%
\begin{pgfscope}%
\pgfpathrectangle{\pgfqpoint{7.512535in}{0.437222in}}{\pgfqpoint{6.275590in}{5.159444in}}%
\pgfusepath{clip}%
\pgfsetbuttcap%
\pgfsetroundjoin%
\pgfsetlinewidth{1.003750pt}%
\definecolor{currentstroke}{rgb}{0.827451,0.827451,0.827451}%
\pgfsetstrokecolor{currentstroke}%
\pgfsetstrokeopacity{0.800000}%
\pgfsetdash{}{0pt}%
\pgfpathmoveto{\pgfqpoint{9.848678in}{4.313606in}}%
\pgfpathcurveto{\pgfqpoint{9.859728in}{4.313606in}}{\pgfqpoint{9.870327in}{4.317996in}}{\pgfqpoint{9.878141in}{4.325810in}}%
\pgfpathcurveto{\pgfqpoint{9.885955in}{4.333623in}}{\pgfqpoint{9.890345in}{4.344222in}}{\pgfqpoint{9.890345in}{4.355272in}}%
\pgfpathcurveto{\pgfqpoint{9.890345in}{4.366323in}}{\pgfqpoint{9.885955in}{4.376922in}}{\pgfqpoint{9.878141in}{4.384735in}}%
\pgfpathcurveto{\pgfqpoint{9.870327in}{4.392549in}}{\pgfqpoint{9.859728in}{4.396939in}}{\pgfqpoint{9.848678in}{4.396939in}}%
\pgfpathcurveto{\pgfqpoint{9.837628in}{4.396939in}}{\pgfqpoint{9.827029in}{4.392549in}}{\pgfqpoint{9.819216in}{4.384735in}}%
\pgfpathcurveto{\pgfqpoint{9.811402in}{4.376922in}}{\pgfqpoint{9.807012in}{4.366323in}}{\pgfqpoint{9.807012in}{4.355272in}}%
\pgfpathcurveto{\pgfqpoint{9.807012in}{4.344222in}}{\pgfqpoint{9.811402in}{4.333623in}}{\pgfqpoint{9.819216in}{4.325810in}}%
\pgfpathcurveto{\pgfqpoint{9.827029in}{4.317996in}}{\pgfqpoint{9.837628in}{4.313606in}}{\pgfqpoint{9.848678in}{4.313606in}}%
\pgfpathlineto{\pgfqpoint{9.848678in}{4.313606in}}%
\pgfpathclose%
\pgfusepath{stroke}%
\end{pgfscope}%
\begin{pgfscope}%
\pgfpathrectangle{\pgfqpoint{7.512535in}{0.437222in}}{\pgfqpoint{6.275590in}{5.159444in}}%
\pgfusepath{clip}%
\pgfsetbuttcap%
\pgfsetroundjoin%
\pgfsetlinewidth{1.003750pt}%
\definecolor{currentstroke}{rgb}{0.827451,0.827451,0.827451}%
\pgfsetstrokecolor{currentstroke}%
\pgfsetstrokeopacity{0.800000}%
\pgfsetdash{}{0pt}%
\pgfpathmoveto{\pgfqpoint{12.815587in}{5.553845in}}%
\pgfpathcurveto{\pgfqpoint{12.826637in}{5.553845in}}{\pgfqpoint{12.837236in}{5.558236in}}{\pgfqpoint{12.845050in}{5.566049in}}%
\pgfpathcurveto{\pgfqpoint{12.852863in}{5.573863in}}{\pgfqpoint{12.857254in}{5.584462in}}{\pgfqpoint{12.857254in}{5.595512in}}%
\pgfpathcurveto{\pgfqpoint{12.857254in}{5.606562in}}{\pgfqpoint{12.852863in}{5.617161in}}{\pgfqpoint{12.845050in}{5.624975in}}%
\pgfpathcurveto{\pgfqpoint{12.837236in}{5.632788in}}{\pgfqpoint{12.826637in}{5.637179in}}{\pgfqpoint{12.815587in}{5.637179in}}%
\pgfpathcurveto{\pgfqpoint{12.804537in}{5.637179in}}{\pgfqpoint{12.793938in}{5.632788in}}{\pgfqpoint{12.786124in}{5.624975in}}%
\pgfpathcurveto{\pgfqpoint{12.778310in}{5.617161in}}{\pgfqpoint{12.773920in}{5.606562in}}{\pgfqpoint{12.773920in}{5.595512in}}%
\pgfpathcurveto{\pgfqpoint{12.773920in}{5.584462in}}{\pgfqpoint{12.778310in}{5.573863in}}{\pgfqpoint{12.786124in}{5.566049in}}%
\pgfpathcurveto{\pgfqpoint{12.793938in}{5.558236in}}{\pgfqpoint{12.804537in}{5.553845in}}{\pgfqpoint{12.815587in}{5.553845in}}%
\pgfpathlineto{\pgfqpoint{12.815587in}{5.553845in}}%
\pgfpathclose%
\pgfusepath{stroke}%
\end{pgfscope}%
\begin{pgfscope}%
\pgfpathrectangle{\pgfqpoint{7.512535in}{0.437222in}}{\pgfqpoint{6.275590in}{5.159444in}}%
\pgfusepath{clip}%
\pgfsetbuttcap%
\pgfsetroundjoin%
\pgfsetlinewidth{1.003750pt}%
\definecolor{currentstroke}{rgb}{0.827451,0.827451,0.827451}%
\pgfsetstrokecolor{currentstroke}%
\pgfsetstrokeopacity{0.800000}%
\pgfsetdash{}{0pt}%
\pgfpathmoveto{\pgfqpoint{10.312167in}{5.501728in}}%
\pgfpathcurveto{\pgfqpoint{10.323217in}{5.501728in}}{\pgfqpoint{10.333816in}{5.506118in}}{\pgfqpoint{10.341630in}{5.513932in}}%
\pgfpathcurveto{\pgfqpoint{10.349443in}{5.521745in}}{\pgfqpoint{10.353834in}{5.532344in}}{\pgfqpoint{10.353834in}{5.543394in}}%
\pgfpathcurveto{\pgfqpoint{10.353834in}{5.554445in}}{\pgfqpoint{10.349443in}{5.565044in}}{\pgfqpoint{10.341630in}{5.572857in}}%
\pgfpathcurveto{\pgfqpoint{10.333816in}{5.580671in}}{\pgfqpoint{10.323217in}{5.585061in}}{\pgfqpoint{10.312167in}{5.585061in}}%
\pgfpathcurveto{\pgfqpoint{10.301117in}{5.585061in}}{\pgfqpoint{10.290518in}{5.580671in}}{\pgfqpoint{10.282704in}{5.572857in}}%
\pgfpathcurveto{\pgfqpoint{10.274891in}{5.565044in}}{\pgfqpoint{10.270500in}{5.554445in}}{\pgfqpoint{10.270500in}{5.543394in}}%
\pgfpathcurveto{\pgfqpoint{10.270500in}{5.532344in}}{\pgfqpoint{10.274891in}{5.521745in}}{\pgfqpoint{10.282704in}{5.513932in}}%
\pgfpathcurveto{\pgfqpoint{10.290518in}{5.506118in}}{\pgfqpoint{10.301117in}{5.501728in}}{\pgfqpoint{10.312167in}{5.501728in}}%
\pgfpathlineto{\pgfqpoint{10.312167in}{5.501728in}}%
\pgfpathclose%
\pgfusepath{stroke}%
\end{pgfscope}%
\begin{pgfscope}%
\pgfpathrectangle{\pgfqpoint{7.512535in}{0.437222in}}{\pgfqpoint{6.275590in}{5.159444in}}%
\pgfusepath{clip}%
\pgfsetbuttcap%
\pgfsetroundjoin%
\pgfsetlinewidth{1.003750pt}%
\definecolor{currentstroke}{rgb}{0.827451,0.827451,0.827451}%
\pgfsetstrokecolor{currentstroke}%
\pgfsetstrokeopacity{0.800000}%
\pgfsetdash{}{0pt}%
\pgfpathmoveto{\pgfqpoint{11.710072in}{5.362410in}}%
\pgfpathcurveto{\pgfqpoint{11.721123in}{5.362410in}}{\pgfqpoint{11.731722in}{5.366800in}}{\pgfqpoint{11.739535in}{5.374614in}}%
\pgfpathcurveto{\pgfqpoint{11.747349in}{5.382427in}}{\pgfqpoint{11.751739in}{5.393027in}}{\pgfqpoint{11.751739in}{5.404077in}}%
\pgfpathcurveto{\pgfqpoint{11.751739in}{5.415127in}}{\pgfqpoint{11.747349in}{5.425726in}}{\pgfqpoint{11.739535in}{5.433539in}}%
\pgfpathcurveto{\pgfqpoint{11.731722in}{5.441353in}}{\pgfqpoint{11.721123in}{5.445743in}}{\pgfqpoint{11.710072in}{5.445743in}}%
\pgfpathcurveto{\pgfqpoint{11.699022in}{5.445743in}}{\pgfqpoint{11.688423in}{5.441353in}}{\pgfqpoint{11.680610in}{5.433539in}}%
\pgfpathcurveto{\pgfqpoint{11.672796in}{5.425726in}}{\pgfqpoint{11.668406in}{5.415127in}}{\pgfqpoint{11.668406in}{5.404077in}}%
\pgfpathcurveto{\pgfqpoint{11.668406in}{5.393027in}}{\pgfqpoint{11.672796in}{5.382427in}}{\pgfqpoint{11.680610in}{5.374614in}}%
\pgfpathcurveto{\pgfqpoint{11.688423in}{5.366800in}}{\pgfqpoint{11.699022in}{5.362410in}}{\pgfqpoint{11.710072in}{5.362410in}}%
\pgfpathlineto{\pgfqpoint{11.710072in}{5.362410in}}%
\pgfpathclose%
\pgfusepath{stroke}%
\end{pgfscope}%
\begin{pgfscope}%
\pgfpathrectangle{\pgfqpoint{7.512535in}{0.437222in}}{\pgfqpoint{6.275590in}{5.159444in}}%
\pgfusepath{clip}%
\pgfsetbuttcap%
\pgfsetroundjoin%
\pgfsetlinewidth{1.003750pt}%
\definecolor{currentstroke}{rgb}{0.827451,0.827451,0.827451}%
\pgfsetstrokecolor{currentstroke}%
\pgfsetstrokeopacity{0.800000}%
\pgfsetdash{}{0pt}%
\pgfpathmoveto{\pgfqpoint{9.807010in}{4.531380in}}%
\pgfpathcurveto{\pgfqpoint{9.818060in}{4.531380in}}{\pgfqpoint{9.828659in}{4.535770in}}{\pgfqpoint{9.836473in}{4.543584in}}%
\pgfpathcurveto{\pgfqpoint{9.844287in}{4.551397in}}{\pgfqpoint{9.848677in}{4.561996in}}{\pgfqpoint{9.848677in}{4.573046in}}%
\pgfpathcurveto{\pgfqpoint{9.848677in}{4.584096in}}{\pgfqpoint{9.844287in}{4.594695in}}{\pgfqpoint{9.836473in}{4.602509in}}%
\pgfpathcurveto{\pgfqpoint{9.828659in}{4.610323in}}{\pgfqpoint{9.818060in}{4.614713in}}{\pgfqpoint{9.807010in}{4.614713in}}%
\pgfpathcurveto{\pgfqpoint{9.795960in}{4.614713in}}{\pgfqpoint{9.785361in}{4.610323in}}{\pgfqpoint{9.777548in}{4.602509in}}%
\pgfpathcurveto{\pgfqpoint{9.769734in}{4.594695in}}{\pgfqpoint{9.765344in}{4.584096in}}{\pgfqpoint{9.765344in}{4.573046in}}%
\pgfpathcurveto{\pgfqpoint{9.765344in}{4.561996in}}{\pgfqpoint{9.769734in}{4.551397in}}{\pgfqpoint{9.777548in}{4.543584in}}%
\pgfpathcurveto{\pgfqpoint{9.785361in}{4.535770in}}{\pgfqpoint{9.795960in}{4.531380in}}{\pgfqpoint{9.807010in}{4.531380in}}%
\pgfpathlineto{\pgfqpoint{9.807010in}{4.531380in}}%
\pgfpathclose%
\pgfusepath{stroke}%
\end{pgfscope}%
\begin{pgfscope}%
\pgfpathrectangle{\pgfqpoint{7.512535in}{0.437222in}}{\pgfqpoint{6.275590in}{5.159444in}}%
\pgfusepath{clip}%
\pgfsetbuttcap%
\pgfsetroundjoin%
\pgfsetlinewidth{1.003750pt}%
\definecolor{currentstroke}{rgb}{0.827451,0.827451,0.827451}%
\pgfsetstrokecolor{currentstroke}%
\pgfsetstrokeopacity{0.800000}%
\pgfsetdash{}{0pt}%
\pgfpathmoveto{\pgfqpoint{7.860715in}{0.736624in}}%
\pgfpathcurveto{\pgfqpoint{7.871765in}{0.736624in}}{\pgfqpoint{7.882364in}{0.741014in}}{\pgfqpoint{7.890178in}{0.748828in}}%
\pgfpathcurveto{\pgfqpoint{7.897991in}{0.756641in}}{\pgfqpoint{7.902381in}{0.767240in}}{\pgfqpoint{7.902381in}{0.778290in}}%
\pgfpathcurveto{\pgfqpoint{7.902381in}{0.789340in}}{\pgfqpoint{7.897991in}{0.799940in}}{\pgfqpoint{7.890178in}{0.807753in}}%
\pgfpathcurveto{\pgfqpoint{7.882364in}{0.815567in}}{\pgfqpoint{7.871765in}{0.819957in}}{\pgfqpoint{7.860715in}{0.819957in}}%
\pgfpathcurveto{\pgfqpoint{7.849665in}{0.819957in}}{\pgfqpoint{7.839066in}{0.815567in}}{\pgfqpoint{7.831252in}{0.807753in}}%
\pgfpathcurveto{\pgfqpoint{7.823438in}{0.799940in}}{\pgfqpoint{7.819048in}{0.789340in}}{\pgfqpoint{7.819048in}{0.778290in}}%
\pgfpathcurveto{\pgfqpoint{7.819048in}{0.767240in}}{\pgfqpoint{7.823438in}{0.756641in}}{\pgfqpoint{7.831252in}{0.748828in}}%
\pgfpathcurveto{\pgfqpoint{7.839066in}{0.741014in}}{\pgfqpoint{7.849665in}{0.736624in}}{\pgfqpoint{7.860715in}{0.736624in}}%
\pgfpathlineto{\pgfqpoint{7.860715in}{0.736624in}}%
\pgfpathclose%
\pgfusepath{stroke}%
\end{pgfscope}%
\begin{pgfscope}%
\pgfpathrectangle{\pgfqpoint{7.512535in}{0.437222in}}{\pgfqpoint{6.275590in}{5.159444in}}%
\pgfusepath{clip}%
\pgfsetbuttcap%
\pgfsetroundjoin%
\pgfsetlinewidth{1.003750pt}%
\definecolor{currentstroke}{rgb}{0.827451,0.827451,0.827451}%
\pgfsetstrokecolor{currentstroke}%
\pgfsetstrokeopacity{0.800000}%
\pgfsetdash{}{0pt}%
\pgfpathmoveto{\pgfqpoint{8.983729in}{3.220234in}}%
\pgfpathcurveto{\pgfqpoint{8.994779in}{3.220234in}}{\pgfqpoint{9.005378in}{3.224624in}}{\pgfqpoint{9.013191in}{3.232438in}}%
\pgfpathcurveto{\pgfqpoint{9.021005in}{3.240251in}}{\pgfqpoint{9.025395in}{3.250850in}}{\pgfqpoint{9.025395in}{3.261901in}}%
\pgfpathcurveto{\pgfqpoint{9.025395in}{3.272951in}}{\pgfqpoint{9.021005in}{3.283550in}}{\pgfqpoint{9.013191in}{3.291363in}}%
\pgfpathcurveto{\pgfqpoint{9.005378in}{3.299177in}}{\pgfqpoint{8.994779in}{3.303567in}}{\pgfqpoint{8.983729in}{3.303567in}}%
\pgfpathcurveto{\pgfqpoint{8.972679in}{3.303567in}}{\pgfqpoint{8.962079in}{3.299177in}}{\pgfqpoint{8.954266in}{3.291363in}}%
\pgfpathcurveto{\pgfqpoint{8.946452in}{3.283550in}}{\pgfqpoint{8.942062in}{3.272951in}}{\pgfqpoint{8.942062in}{3.261901in}}%
\pgfpathcurveto{\pgfqpoint{8.942062in}{3.250850in}}{\pgfqpoint{8.946452in}{3.240251in}}{\pgfqpoint{8.954266in}{3.232438in}}%
\pgfpathcurveto{\pgfqpoint{8.962079in}{3.224624in}}{\pgfqpoint{8.972679in}{3.220234in}}{\pgfqpoint{8.983729in}{3.220234in}}%
\pgfpathlineto{\pgfqpoint{8.983729in}{3.220234in}}%
\pgfpathclose%
\pgfusepath{stroke}%
\end{pgfscope}%
\begin{pgfscope}%
\pgfpathrectangle{\pgfqpoint{7.512535in}{0.437222in}}{\pgfqpoint{6.275590in}{5.159444in}}%
\pgfusepath{clip}%
\pgfsetbuttcap%
\pgfsetroundjoin%
\pgfsetlinewidth{1.003750pt}%
\definecolor{currentstroke}{rgb}{0.827451,0.827451,0.827451}%
\pgfsetstrokecolor{currentstroke}%
\pgfsetstrokeopacity{0.800000}%
\pgfsetdash{}{0pt}%
\pgfpathmoveto{\pgfqpoint{9.319291in}{1.927959in}}%
\pgfpathcurveto{\pgfqpoint{9.330341in}{1.927959in}}{\pgfqpoint{9.340940in}{1.932349in}}{\pgfqpoint{9.348754in}{1.940163in}}%
\pgfpathcurveto{\pgfqpoint{9.356567in}{1.947976in}}{\pgfqpoint{9.360957in}{1.958575in}}{\pgfqpoint{9.360957in}{1.969626in}}%
\pgfpathcurveto{\pgfqpoint{9.360957in}{1.980676in}}{\pgfqpoint{9.356567in}{1.991275in}}{\pgfqpoint{9.348754in}{1.999088in}}%
\pgfpathcurveto{\pgfqpoint{9.340940in}{2.006902in}}{\pgfqpoint{9.330341in}{2.011292in}}{\pgfqpoint{9.319291in}{2.011292in}}%
\pgfpathcurveto{\pgfqpoint{9.308241in}{2.011292in}}{\pgfqpoint{9.297642in}{2.006902in}}{\pgfqpoint{9.289828in}{1.999088in}}%
\pgfpathcurveto{\pgfqpoint{9.282014in}{1.991275in}}{\pgfqpoint{9.277624in}{1.980676in}}{\pgfqpoint{9.277624in}{1.969626in}}%
\pgfpathcurveto{\pgfqpoint{9.277624in}{1.958575in}}{\pgfqpoint{9.282014in}{1.947976in}}{\pgfqpoint{9.289828in}{1.940163in}}%
\pgfpathcurveto{\pgfqpoint{9.297642in}{1.932349in}}{\pgfqpoint{9.308241in}{1.927959in}}{\pgfqpoint{9.319291in}{1.927959in}}%
\pgfpathlineto{\pgfqpoint{9.319291in}{1.927959in}}%
\pgfpathclose%
\pgfusepath{stroke}%
\end{pgfscope}%
\begin{pgfscope}%
\pgfpathrectangle{\pgfqpoint{7.512535in}{0.437222in}}{\pgfqpoint{6.275590in}{5.159444in}}%
\pgfusepath{clip}%
\pgfsetbuttcap%
\pgfsetroundjoin%
\pgfsetlinewidth{1.003750pt}%
\definecolor{currentstroke}{rgb}{0.827451,0.827451,0.827451}%
\pgfsetstrokecolor{currentstroke}%
\pgfsetstrokeopacity{0.800000}%
\pgfsetdash{}{0pt}%
\pgfpathmoveto{\pgfqpoint{10.486066in}{4.589271in}}%
\pgfpathcurveto{\pgfqpoint{10.497116in}{4.589271in}}{\pgfqpoint{10.507715in}{4.593662in}}{\pgfqpoint{10.515529in}{4.601475in}}%
\pgfpathcurveto{\pgfqpoint{10.523342in}{4.609289in}}{\pgfqpoint{10.527733in}{4.619888in}}{\pgfqpoint{10.527733in}{4.630938in}}%
\pgfpathcurveto{\pgfqpoint{10.527733in}{4.641988in}}{\pgfqpoint{10.523342in}{4.652587in}}{\pgfqpoint{10.515529in}{4.660401in}}%
\pgfpathcurveto{\pgfqpoint{10.507715in}{4.668214in}}{\pgfqpoint{10.497116in}{4.672605in}}{\pgfqpoint{10.486066in}{4.672605in}}%
\pgfpathcurveto{\pgfqpoint{10.475016in}{4.672605in}}{\pgfqpoint{10.464417in}{4.668214in}}{\pgfqpoint{10.456603in}{4.660401in}}%
\pgfpathcurveto{\pgfqpoint{10.448790in}{4.652587in}}{\pgfqpoint{10.444399in}{4.641988in}}{\pgfqpoint{10.444399in}{4.630938in}}%
\pgfpathcurveto{\pgfqpoint{10.444399in}{4.619888in}}{\pgfqpoint{10.448790in}{4.609289in}}{\pgfqpoint{10.456603in}{4.601475in}}%
\pgfpathcurveto{\pgfqpoint{10.464417in}{4.593662in}}{\pgfqpoint{10.475016in}{4.589271in}}{\pgfqpoint{10.486066in}{4.589271in}}%
\pgfpathlineto{\pgfqpoint{10.486066in}{4.589271in}}%
\pgfpathclose%
\pgfusepath{stroke}%
\end{pgfscope}%
\begin{pgfscope}%
\pgfpathrectangle{\pgfqpoint{7.512535in}{0.437222in}}{\pgfqpoint{6.275590in}{5.159444in}}%
\pgfusepath{clip}%
\pgfsetbuttcap%
\pgfsetroundjoin%
\pgfsetlinewidth{1.003750pt}%
\definecolor{currentstroke}{rgb}{0.827451,0.827451,0.827451}%
\pgfsetstrokecolor{currentstroke}%
\pgfsetstrokeopacity{0.800000}%
\pgfsetdash{}{0pt}%
\pgfpathmoveto{\pgfqpoint{13.135726in}{5.511813in}}%
\pgfpathcurveto{\pgfqpoint{13.146776in}{5.511813in}}{\pgfqpoint{13.157375in}{5.516204in}}{\pgfqpoint{13.165189in}{5.524017in}}%
\pgfpathcurveto{\pgfqpoint{13.173003in}{5.531831in}}{\pgfqpoint{13.177393in}{5.542430in}}{\pgfqpoint{13.177393in}{5.553480in}}%
\pgfpathcurveto{\pgfqpoint{13.177393in}{5.564530in}}{\pgfqpoint{13.173003in}{5.575129in}}{\pgfqpoint{13.165189in}{5.582943in}}%
\pgfpathcurveto{\pgfqpoint{13.157375in}{5.590756in}}{\pgfqpoint{13.146776in}{5.595147in}}{\pgfqpoint{13.135726in}{5.595147in}}%
\pgfpathcurveto{\pgfqpoint{13.124676in}{5.595147in}}{\pgfqpoint{13.114077in}{5.590756in}}{\pgfqpoint{13.106263in}{5.582943in}}%
\pgfpathcurveto{\pgfqpoint{13.098450in}{5.575129in}}{\pgfqpoint{13.094059in}{5.564530in}}{\pgfqpoint{13.094059in}{5.553480in}}%
\pgfpathcurveto{\pgfqpoint{13.094059in}{5.542430in}}{\pgfqpoint{13.098450in}{5.531831in}}{\pgfqpoint{13.106263in}{5.524017in}}%
\pgfpathcurveto{\pgfqpoint{13.114077in}{5.516204in}}{\pgfqpoint{13.124676in}{5.511813in}}{\pgfqpoint{13.135726in}{5.511813in}}%
\pgfpathlineto{\pgfqpoint{13.135726in}{5.511813in}}%
\pgfpathclose%
\pgfusepath{stroke}%
\end{pgfscope}%
\begin{pgfscope}%
\pgfpathrectangle{\pgfqpoint{7.512535in}{0.437222in}}{\pgfqpoint{6.275590in}{5.159444in}}%
\pgfusepath{clip}%
\pgfsetbuttcap%
\pgfsetroundjoin%
\pgfsetlinewidth{1.003750pt}%
\definecolor{currentstroke}{rgb}{0.827451,0.827451,0.827451}%
\pgfsetstrokecolor{currentstroke}%
\pgfsetstrokeopacity{0.800000}%
\pgfsetdash{}{0pt}%
\pgfpathmoveto{\pgfqpoint{10.070260in}{4.544907in}}%
\pgfpathcurveto{\pgfqpoint{10.081310in}{4.544907in}}{\pgfqpoint{10.091909in}{4.549297in}}{\pgfqpoint{10.099723in}{4.557110in}}%
\pgfpathcurveto{\pgfqpoint{10.107536in}{4.564924in}}{\pgfqpoint{10.111927in}{4.575523in}}{\pgfqpoint{10.111927in}{4.586573in}}%
\pgfpathcurveto{\pgfqpoint{10.111927in}{4.597623in}}{\pgfqpoint{10.107536in}{4.608222in}}{\pgfqpoint{10.099723in}{4.616036in}}%
\pgfpathcurveto{\pgfqpoint{10.091909in}{4.623850in}}{\pgfqpoint{10.081310in}{4.628240in}}{\pgfqpoint{10.070260in}{4.628240in}}%
\pgfpathcurveto{\pgfqpoint{10.059210in}{4.628240in}}{\pgfqpoint{10.048611in}{4.623850in}}{\pgfqpoint{10.040797in}{4.616036in}}%
\pgfpathcurveto{\pgfqpoint{10.032983in}{4.608222in}}{\pgfqpoint{10.028593in}{4.597623in}}{\pgfqpoint{10.028593in}{4.586573in}}%
\pgfpathcurveto{\pgfqpoint{10.028593in}{4.575523in}}{\pgfqpoint{10.032983in}{4.564924in}}{\pgfqpoint{10.040797in}{4.557110in}}%
\pgfpathcurveto{\pgfqpoint{10.048611in}{4.549297in}}{\pgfqpoint{10.059210in}{4.544907in}}{\pgfqpoint{10.070260in}{4.544907in}}%
\pgfpathlineto{\pgfqpoint{10.070260in}{4.544907in}}%
\pgfpathclose%
\pgfusepath{stroke}%
\end{pgfscope}%
\begin{pgfscope}%
\pgfpathrectangle{\pgfqpoint{7.512535in}{0.437222in}}{\pgfqpoint{6.275590in}{5.159444in}}%
\pgfusepath{clip}%
\pgfsetbuttcap%
\pgfsetroundjoin%
\pgfsetlinewidth{1.003750pt}%
\definecolor{currentstroke}{rgb}{0.827451,0.827451,0.827451}%
\pgfsetstrokecolor{currentstroke}%
\pgfsetstrokeopacity{0.800000}%
\pgfsetdash{}{0pt}%
\pgfpathmoveto{\pgfqpoint{9.216460in}{3.540868in}}%
\pgfpathcurveto{\pgfqpoint{9.227510in}{3.540868in}}{\pgfqpoint{9.238109in}{3.545259in}}{\pgfqpoint{9.245923in}{3.553072in}}%
\pgfpathcurveto{\pgfqpoint{9.253737in}{3.560886in}}{\pgfqpoint{9.258127in}{3.571485in}}{\pgfqpoint{9.258127in}{3.582535in}}%
\pgfpathcurveto{\pgfqpoint{9.258127in}{3.593585in}}{\pgfqpoint{9.253737in}{3.604184in}}{\pgfqpoint{9.245923in}{3.611998in}}%
\pgfpathcurveto{\pgfqpoint{9.238109in}{3.619812in}}{\pgfqpoint{9.227510in}{3.624202in}}{\pgfqpoint{9.216460in}{3.624202in}}%
\pgfpathcurveto{\pgfqpoint{9.205410in}{3.624202in}}{\pgfqpoint{9.194811in}{3.619812in}}{\pgfqpoint{9.186997in}{3.611998in}}%
\pgfpathcurveto{\pgfqpoint{9.179184in}{3.604184in}}{\pgfqpoint{9.174794in}{3.593585in}}{\pgfqpoint{9.174794in}{3.582535in}}%
\pgfpathcurveto{\pgfqpoint{9.174794in}{3.571485in}}{\pgfqpoint{9.179184in}{3.560886in}}{\pgfqpoint{9.186997in}{3.553072in}}%
\pgfpathcurveto{\pgfqpoint{9.194811in}{3.545259in}}{\pgfqpoint{9.205410in}{3.540868in}}{\pgfqpoint{9.216460in}{3.540868in}}%
\pgfpathlineto{\pgfqpoint{9.216460in}{3.540868in}}%
\pgfpathclose%
\pgfusepath{stroke}%
\end{pgfscope}%
\begin{pgfscope}%
\pgfpathrectangle{\pgfqpoint{7.512535in}{0.437222in}}{\pgfqpoint{6.275590in}{5.159444in}}%
\pgfusepath{clip}%
\pgfsetbuttcap%
\pgfsetroundjoin%
\pgfsetlinewidth{1.003750pt}%
\definecolor{currentstroke}{rgb}{0.827451,0.827451,0.827451}%
\pgfsetstrokecolor{currentstroke}%
\pgfsetstrokeopacity{0.800000}%
\pgfsetdash{}{0pt}%
\pgfpathmoveto{\pgfqpoint{8.944819in}{2.926208in}}%
\pgfpathcurveto{\pgfqpoint{8.955870in}{2.926208in}}{\pgfqpoint{8.966469in}{2.930599in}}{\pgfqpoint{8.974282in}{2.938412in}}%
\pgfpathcurveto{\pgfqpoint{8.982096in}{2.946226in}}{\pgfqpoint{8.986486in}{2.956825in}}{\pgfqpoint{8.986486in}{2.967875in}}%
\pgfpathcurveto{\pgfqpoint{8.986486in}{2.978925in}}{\pgfqpoint{8.982096in}{2.989524in}}{\pgfqpoint{8.974282in}{2.997338in}}%
\pgfpathcurveto{\pgfqpoint{8.966469in}{3.005152in}}{\pgfqpoint{8.955870in}{3.009542in}}{\pgfqpoint{8.944819in}{3.009542in}}%
\pgfpathcurveto{\pgfqpoint{8.933769in}{3.009542in}}{\pgfqpoint{8.923170in}{3.005152in}}{\pgfqpoint{8.915357in}{2.997338in}}%
\pgfpathcurveto{\pgfqpoint{8.907543in}{2.989524in}}{\pgfqpoint{8.903153in}{2.978925in}}{\pgfqpoint{8.903153in}{2.967875in}}%
\pgfpathcurveto{\pgfqpoint{8.903153in}{2.956825in}}{\pgfqpoint{8.907543in}{2.946226in}}{\pgfqpoint{8.915357in}{2.938412in}}%
\pgfpathcurveto{\pgfqpoint{8.923170in}{2.930599in}}{\pgfqpoint{8.933769in}{2.926208in}}{\pgfqpoint{8.944819in}{2.926208in}}%
\pgfpathlineto{\pgfqpoint{8.944819in}{2.926208in}}%
\pgfpathclose%
\pgfusepath{stroke}%
\end{pgfscope}%
\begin{pgfscope}%
\pgfpathrectangle{\pgfqpoint{7.512535in}{0.437222in}}{\pgfqpoint{6.275590in}{5.159444in}}%
\pgfusepath{clip}%
\pgfsetbuttcap%
\pgfsetroundjoin%
\pgfsetlinewidth{1.003750pt}%
\definecolor{currentstroke}{rgb}{0.827451,0.827451,0.827451}%
\pgfsetstrokecolor{currentstroke}%
\pgfsetstrokeopacity{0.800000}%
\pgfsetdash{}{0pt}%
\pgfpathmoveto{\pgfqpoint{12.287231in}{5.553091in}}%
\pgfpathcurveto{\pgfqpoint{12.298281in}{5.553091in}}{\pgfqpoint{12.308880in}{5.557481in}}{\pgfqpoint{12.316694in}{5.565295in}}%
\pgfpathcurveto{\pgfqpoint{12.324508in}{5.573108in}}{\pgfqpoint{12.328898in}{5.583707in}}{\pgfqpoint{12.328898in}{5.594757in}}%
\pgfpathcurveto{\pgfqpoint{12.328898in}{5.605808in}}{\pgfqpoint{12.324508in}{5.616407in}}{\pgfqpoint{12.316694in}{5.624220in}}%
\pgfpathcurveto{\pgfqpoint{12.308880in}{5.632034in}}{\pgfqpoint{12.298281in}{5.636424in}}{\pgfqpoint{12.287231in}{5.636424in}}%
\pgfpathcurveto{\pgfqpoint{12.276181in}{5.636424in}}{\pgfqpoint{12.265582in}{5.632034in}}{\pgfqpoint{12.257768in}{5.624220in}}%
\pgfpathcurveto{\pgfqpoint{12.249955in}{5.616407in}}{\pgfqpoint{12.245564in}{5.605808in}}{\pgfqpoint{12.245564in}{5.594757in}}%
\pgfpathcurveto{\pgfqpoint{12.245564in}{5.583707in}}{\pgfqpoint{12.249955in}{5.573108in}}{\pgfqpoint{12.257768in}{5.565295in}}%
\pgfpathcurveto{\pgfqpoint{12.265582in}{5.557481in}}{\pgfqpoint{12.276181in}{5.553091in}}{\pgfqpoint{12.287231in}{5.553091in}}%
\pgfpathlineto{\pgfqpoint{12.287231in}{5.553091in}}%
\pgfpathclose%
\pgfusepath{stroke}%
\end{pgfscope}%
\begin{pgfscope}%
\pgfpathrectangle{\pgfqpoint{7.512535in}{0.437222in}}{\pgfqpoint{6.275590in}{5.159444in}}%
\pgfusepath{clip}%
\pgfsetbuttcap%
\pgfsetroundjoin%
\pgfsetlinewidth{1.003750pt}%
\definecolor{currentstroke}{rgb}{0.827451,0.827451,0.827451}%
\pgfsetstrokecolor{currentstroke}%
\pgfsetstrokeopacity{0.800000}%
\pgfsetdash{}{0pt}%
\pgfpathmoveto{\pgfqpoint{9.816953in}{3.890552in}}%
\pgfpathcurveto{\pgfqpoint{9.828003in}{3.890552in}}{\pgfqpoint{9.838602in}{3.894943in}}{\pgfqpoint{9.846415in}{3.902756in}}%
\pgfpathcurveto{\pgfqpoint{9.854229in}{3.910570in}}{\pgfqpoint{9.858619in}{3.921169in}}{\pgfqpoint{9.858619in}{3.932219in}}%
\pgfpathcurveto{\pgfqpoint{9.858619in}{3.943269in}}{\pgfqpoint{9.854229in}{3.953868in}}{\pgfqpoint{9.846415in}{3.961682in}}%
\pgfpathcurveto{\pgfqpoint{9.838602in}{3.969495in}}{\pgfqpoint{9.828003in}{3.973886in}}{\pgfqpoint{9.816953in}{3.973886in}}%
\pgfpathcurveto{\pgfqpoint{9.805903in}{3.973886in}}{\pgfqpoint{9.795304in}{3.969495in}}{\pgfqpoint{9.787490in}{3.961682in}}%
\pgfpathcurveto{\pgfqpoint{9.779676in}{3.953868in}}{\pgfqpoint{9.775286in}{3.943269in}}{\pgfqpoint{9.775286in}{3.932219in}}%
\pgfpathcurveto{\pgfqpoint{9.775286in}{3.921169in}}{\pgfqpoint{9.779676in}{3.910570in}}{\pgfqpoint{9.787490in}{3.902756in}}%
\pgfpathcurveto{\pgfqpoint{9.795304in}{3.894943in}}{\pgfqpoint{9.805903in}{3.890552in}}{\pgfqpoint{9.816953in}{3.890552in}}%
\pgfpathlineto{\pgfqpoint{9.816953in}{3.890552in}}%
\pgfpathclose%
\pgfusepath{stroke}%
\end{pgfscope}%
\begin{pgfscope}%
\pgfpathrectangle{\pgfqpoint{7.512535in}{0.437222in}}{\pgfqpoint{6.275590in}{5.159444in}}%
\pgfusepath{clip}%
\pgfsetbuttcap%
\pgfsetroundjoin%
\pgfsetlinewidth{1.003750pt}%
\definecolor{currentstroke}{rgb}{0.827451,0.827451,0.827451}%
\pgfsetstrokecolor{currentstroke}%
\pgfsetstrokeopacity{0.800000}%
\pgfsetdash{}{0pt}%
\pgfpathmoveto{\pgfqpoint{11.262853in}{5.071024in}}%
\pgfpathcurveto{\pgfqpoint{11.273903in}{5.071024in}}{\pgfqpoint{11.284502in}{5.075415in}}{\pgfqpoint{11.292316in}{5.083228in}}%
\pgfpathcurveto{\pgfqpoint{11.300130in}{5.091042in}}{\pgfqpoint{11.304520in}{5.101641in}}{\pgfqpoint{11.304520in}{5.112691in}}%
\pgfpathcurveto{\pgfqpoint{11.304520in}{5.123741in}}{\pgfqpoint{11.300130in}{5.134340in}}{\pgfqpoint{11.292316in}{5.142154in}}%
\pgfpathcurveto{\pgfqpoint{11.284502in}{5.149967in}}{\pgfqpoint{11.273903in}{5.154358in}}{\pgfqpoint{11.262853in}{5.154358in}}%
\pgfpathcurveto{\pgfqpoint{11.251803in}{5.154358in}}{\pgfqpoint{11.241204in}{5.149967in}}{\pgfqpoint{11.233391in}{5.142154in}}%
\pgfpathcurveto{\pgfqpoint{11.225577in}{5.134340in}}{\pgfqpoint{11.221187in}{5.123741in}}{\pgfqpoint{11.221187in}{5.112691in}}%
\pgfpathcurveto{\pgfqpoint{11.221187in}{5.101641in}}{\pgfqpoint{11.225577in}{5.091042in}}{\pgfqpoint{11.233391in}{5.083228in}}%
\pgfpathcurveto{\pgfqpoint{11.241204in}{5.075415in}}{\pgfqpoint{11.251803in}{5.071024in}}{\pgfqpoint{11.262853in}{5.071024in}}%
\pgfpathlineto{\pgfqpoint{11.262853in}{5.071024in}}%
\pgfpathclose%
\pgfusepath{stroke}%
\end{pgfscope}%
\begin{pgfscope}%
\pgfpathrectangle{\pgfqpoint{7.512535in}{0.437222in}}{\pgfqpoint{6.275590in}{5.159444in}}%
\pgfusepath{clip}%
\pgfsetbuttcap%
\pgfsetroundjoin%
\pgfsetlinewidth{1.003750pt}%
\definecolor{currentstroke}{rgb}{0.827451,0.827451,0.827451}%
\pgfsetstrokecolor{currentstroke}%
\pgfsetstrokeopacity{0.800000}%
\pgfsetdash{}{0pt}%
\pgfpathmoveto{\pgfqpoint{9.061862in}{3.235524in}}%
\pgfpathcurveto{\pgfqpoint{9.072912in}{3.235524in}}{\pgfqpoint{9.083511in}{3.239914in}}{\pgfqpoint{9.091325in}{3.247728in}}%
\pgfpathcurveto{\pgfqpoint{9.099138in}{3.255542in}}{\pgfqpoint{9.103529in}{3.266141in}}{\pgfqpoint{9.103529in}{3.277191in}}%
\pgfpathcurveto{\pgfqpoint{9.103529in}{3.288241in}}{\pgfqpoint{9.099138in}{3.298840in}}{\pgfqpoint{9.091325in}{3.306654in}}%
\pgfpathcurveto{\pgfqpoint{9.083511in}{3.314467in}}{\pgfqpoint{9.072912in}{3.318857in}}{\pgfqpoint{9.061862in}{3.318857in}}%
\pgfpathcurveto{\pgfqpoint{9.050812in}{3.318857in}}{\pgfqpoint{9.040213in}{3.314467in}}{\pgfqpoint{9.032399in}{3.306654in}}%
\pgfpathcurveto{\pgfqpoint{9.024586in}{3.298840in}}{\pgfqpoint{9.020195in}{3.288241in}}{\pgfqpoint{9.020195in}{3.277191in}}%
\pgfpathcurveto{\pgfqpoint{9.020195in}{3.266141in}}{\pgfqpoint{9.024586in}{3.255542in}}{\pgfqpoint{9.032399in}{3.247728in}}%
\pgfpathcurveto{\pgfqpoint{9.040213in}{3.239914in}}{\pgfqpoint{9.050812in}{3.235524in}}{\pgfqpoint{9.061862in}{3.235524in}}%
\pgfpathlineto{\pgfqpoint{9.061862in}{3.235524in}}%
\pgfpathclose%
\pgfusepath{stroke}%
\end{pgfscope}%
\begin{pgfscope}%
\pgfpathrectangle{\pgfqpoint{7.512535in}{0.437222in}}{\pgfqpoint{6.275590in}{5.159444in}}%
\pgfusepath{clip}%
\pgfsetbuttcap%
\pgfsetroundjoin%
\pgfsetlinewidth{1.003750pt}%
\definecolor{currentstroke}{rgb}{0.827451,0.827451,0.827451}%
\pgfsetstrokecolor{currentstroke}%
\pgfsetstrokeopacity{0.800000}%
\pgfsetdash{}{0pt}%
\pgfpathmoveto{\pgfqpoint{13.425002in}{5.528743in}}%
\pgfpathcurveto{\pgfqpoint{13.436053in}{5.528743in}}{\pgfqpoint{13.446652in}{5.533133in}}{\pgfqpoint{13.454465in}{5.540947in}}%
\pgfpathcurveto{\pgfqpoint{13.462279in}{5.548760in}}{\pgfqpoint{13.466669in}{5.559359in}}{\pgfqpoint{13.466669in}{5.570410in}}%
\pgfpathcurveto{\pgfqpoint{13.466669in}{5.581460in}}{\pgfqpoint{13.462279in}{5.592059in}}{\pgfqpoint{13.454465in}{5.599872in}}%
\pgfpathcurveto{\pgfqpoint{13.446652in}{5.607686in}}{\pgfqpoint{13.436053in}{5.612076in}}{\pgfqpoint{13.425002in}{5.612076in}}%
\pgfpathcurveto{\pgfqpoint{13.413952in}{5.612076in}}{\pgfqpoint{13.403353in}{5.607686in}}{\pgfqpoint{13.395540in}{5.599872in}}%
\pgfpathcurveto{\pgfqpoint{13.387726in}{5.592059in}}{\pgfqpoint{13.383336in}{5.581460in}}{\pgfqpoint{13.383336in}{5.570410in}}%
\pgfpathcurveto{\pgfqpoint{13.383336in}{5.559359in}}{\pgfqpoint{13.387726in}{5.548760in}}{\pgfqpoint{13.395540in}{5.540947in}}%
\pgfpathcurveto{\pgfqpoint{13.403353in}{5.533133in}}{\pgfqpoint{13.413952in}{5.528743in}}{\pgfqpoint{13.425002in}{5.528743in}}%
\pgfpathlineto{\pgfqpoint{13.425002in}{5.528743in}}%
\pgfpathclose%
\pgfusepath{stroke}%
\end{pgfscope}%
\begin{pgfscope}%
\pgfpathrectangle{\pgfqpoint{7.512535in}{0.437222in}}{\pgfqpoint{6.275590in}{5.159444in}}%
\pgfusepath{clip}%
\pgfsetbuttcap%
\pgfsetroundjoin%
\pgfsetlinewidth{1.003750pt}%
\definecolor{currentstroke}{rgb}{0.827451,0.827451,0.827451}%
\pgfsetstrokecolor{currentstroke}%
\pgfsetstrokeopacity{0.800000}%
\pgfsetdash{}{0pt}%
\pgfpathmoveto{\pgfqpoint{9.171368in}{3.241859in}}%
\pgfpathcurveto{\pgfqpoint{9.182418in}{3.241859in}}{\pgfqpoint{9.193017in}{3.246249in}}{\pgfqpoint{9.200830in}{3.254063in}}%
\pgfpathcurveto{\pgfqpoint{9.208644in}{3.261876in}}{\pgfqpoint{9.213034in}{3.272475in}}{\pgfqpoint{9.213034in}{3.283525in}}%
\pgfpathcurveto{\pgfqpoint{9.213034in}{3.294576in}}{\pgfqpoint{9.208644in}{3.305175in}}{\pgfqpoint{9.200830in}{3.312988in}}%
\pgfpathcurveto{\pgfqpoint{9.193017in}{3.320802in}}{\pgfqpoint{9.182418in}{3.325192in}}{\pgfqpoint{9.171368in}{3.325192in}}%
\pgfpathcurveto{\pgfqpoint{9.160317in}{3.325192in}}{\pgfqpoint{9.149718in}{3.320802in}}{\pgfqpoint{9.141905in}{3.312988in}}%
\pgfpathcurveto{\pgfqpoint{9.134091in}{3.305175in}}{\pgfqpoint{9.129701in}{3.294576in}}{\pgfqpoint{9.129701in}{3.283525in}}%
\pgfpathcurveto{\pgfqpoint{9.129701in}{3.272475in}}{\pgfqpoint{9.134091in}{3.261876in}}{\pgfqpoint{9.141905in}{3.254063in}}%
\pgfpathcurveto{\pgfqpoint{9.149718in}{3.246249in}}{\pgfqpoint{9.160317in}{3.241859in}}{\pgfqpoint{9.171368in}{3.241859in}}%
\pgfpathlineto{\pgfqpoint{9.171368in}{3.241859in}}%
\pgfpathclose%
\pgfusepath{stroke}%
\end{pgfscope}%
\begin{pgfscope}%
\pgfpathrectangle{\pgfqpoint{7.512535in}{0.437222in}}{\pgfqpoint{6.275590in}{5.159444in}}%
\pgfusepath{clip}%
\pgfsetbuttcap%
\pgfsetroundjoin%
\pgfsetlinewidth{1.003750pt}%
\definecolor{currentstroke}{rgb}{0.827451,0.827451,0.827451}%
\pgfsetstrokecolor{currentstroke}%
\pgfsetstrokeopacity{0.800000}%
\pgfsetdash{}{0pt}%
\pgfpathmoveto{\pgfqpoint{8.277312in}{1.563305in}}%
\pgfpathcurveto{\pgfqpoint{8.288363in}{1.563305in}}{\pgfqpoint{8.298962in}{1.567695in}}{\pgfqpoint{8.306775in}{1.575509in}}%
\pgfpathcurveto{\pgfqpoint{8.314589in}{1.583322in}}{\pgfqpoint{8.318979in}{1.593921in}}{\pgfqpoint{8.318979in}{1.604971in}}%
\pgfpathcurveto{\pgfqpoint{8.318979in}{1.616021in}}{\pgfqpoint{8.314589in}{1.626621in}}{\pgfqpoint{8.306775in}{1.634434in}}%
\pgfpathcurveto{\pgfqpoint{8.298962in}{1.642248in}}{\pgfqpoint{8.288363in}{1.646638in}}{\pgfqpoint{8.277312in}{1.646638in}}%
\pgfpathcurveto{\pgfqpoint{8.266262in}{1.646638in}}{\pgfqpoint{8.255663in}{1.642248in}}{\pgfqpoint{8.247850in}{1.634434in}}%
\pgfpathcurveto{\pgfqpoint{8.240036in}{1.626621in}}{\pgfqpoint{8.235646in}{1.616021in}}{\pgfqpoint{8.235646in}{1.604971in}}%
\pgfpathcurveto{\pgfqpoint{8.235646in}{1.593921in}}{\pgfqpoint{8.240036in}{1.583322in}}{\pgfqpoint{8.247850in}{1.575509in}}%
\pgfpathcurveto{\pgfqpoint{8.255663in}{1.567695in}}{\pgfqpoint{8.266262in}{1.563305in}}{\pgfqpoint{8.277312in}{1.563305in}}%
\pgfpathlineto{\pgfqpoint{8.277312in}{1.563305in}}%
\pgfpathclose%
\pgfusepath{stroke}%
\end{pgfscope}%
\begin{pgfscope}%
\pgfpathrectangle{\pgfqpoint{7.512535in}{0.437222in}}{\pgfqpoint{6.275590in}{5.159444in}}%
\pgfusepath{clip}%
\pgfsetbuttcap%
\pgfsetroundjoin%
\pgfsetlinewidth{1.003750pt}%
\definecolor{currentstroke}{rgb}{0.827451,0.827451,0.827451}%
\pgfsetstrokecolor{currentstroke}%
\pgfsetstrokeopacity{0.800000}%
\pgfsetdash{}{0pt}%
\pgfpathmoveto{\pgfqpoint{10.698510in}{3.970272in}}%
\pgfpathcurveto{\pgfqpoint{10.709560in}{3.970272in}}{\pgfqpoint{10.720159in}{3.974662in}}{\pgfqpoint{10.727973in}{3.982476in}}%
\pgfpathcurveto{\pgfqpoint{10.735787in}{3.990290in}}{\pgfqpoint{10.740177in}{4.000889in}}{\pgfqpoint{10.740177in}{4.011939in}}%
\pgfpathcurveto{\pgfqpoint{10.740177in}{4.022989in}}{\pgfqpoint{10.735787in}{4.033588in}}{\pgfqpoint{10.727973in}{4.041402in}}%
\pgfpathcurveto{\pgfqpoint{10.720159in}{4.049215in}}{\pgfqpoint{10.709560in}{4.053605in}}{\pgfqpoint{10.698510in}{4.053605in}}%
\pgfpathcurveto{\pgfqpoint{10.687460in}{4.053605in}}{\pgfqpoint{10.676861in}{4.049215in}}{\pgfqpoint{10.669047in}{4.041402in}}%
\pgfpathcurveto{\pgfqpoint{10.661234in}{4.033588in}}{\pgfqpoint{10.656844in}{4.022989in}}{\pgfqpoint{10.656844in}{4.011939in}}%
\pgfpathcurveto{\pgfqpoint{10.656844in}{4.000889in}}{\pgfqpoint{10.661234in}{3.990290in}}{\pgfqpoint{10.669047in}{3.982476in}}%
\pgfpathcurveto{\pgfqpoint{10.676861in}{3.974662in}}{\pgfqpoint{10.687460in}{3.970272in}}{\pgfqpoint{10.698510in}{3.970272in}}%
\pgfpathlineto{\pgfqpoint{10.698510in}{3.970272in}}%
\pgfpathclose%
\pgfusepath{stroke}%
\end{pgfscope}%
\begin{pgfscope}%
\pgfpathrectangle{\pgfqpoint{7.512535in}{0.437222in}}{\pgfqpoint{6.275590in}{5.159444in}}%
\pgfusepath{clip}%
\pgfsetbuttcap%
\pgfsetroundjoin%
\pgfsetlinewidth{1.003750pt}%
\definecolor{currentstroke}{rgb}{0.827451,0.827451,0.827451}%
\pgfsetstrokecolor{currentstroke}%
\pgfsetstrokeopacity{0.800000}%
\pgfsetdash{}{0pt}%
\pgfpathmoveto{\pgfqpoint{13.672716in}{5.539504in}}%
\pgfpathcurveto{\pgfqpoint{13.683766in}{5.539504in}}{\pgfqpoint{13.694366in}{5.543895in}}{\pgfqpoint{13.702179in}{5.551708in}}%
\pgfpathcurveto{\pgfqpoint{13.709993in}{5.559522in}}{\pgfqpoint{13.714383in}{5.570121in}}{\pgfqpoint{13.714383in}{5.581171in}}%
\pgfpathcurveto{\pgfqpoint{13.714383in}{5.592221in}}{\pgfqpoint{13.709993in}{5.602820in}}{\pgfqpoint{13.702179in}{5.610634in}}%
\pgfpathcurveto{\pgfqpoint{13.694366in}{5.618447in}}{\pgfqpoint{13.683766in}{5.622838in}}{\pgfqpoint{13.672716in}{5.622838in}}%
\pgfpathcurveto{\pgfqpoint{13.661666in}{5.622838in}}{\pgfqpoint{13.651067in}{5.618447in}}{\pgfqpoint{13.643254in}{5.610634in}}%
\pgfpathcurveto{\pgfqpoint{13.635440in}{5.602820in}}{\pgfqpoint{13.631050in}{5.592221in}}{\pgfqpoint{13.631050in}{5.581171in}}%
\pgfpathcurveto{\pgfqpoint{13.631050in}{5.570121in}}{\pgfqpoint{13.635440in}{5.559522in}}{\pgfqpoint{13.643254in}{5.551708in}}%
\pgfpathcurveto{\pgfqpoint{13.651067in}{5.543895in}}{\pgfqpoint{13.661666in}{5.539504in}}{\pgfqpoint{13.672716in}{5.539504in}}%
\pgfpathlineto{\pgfqpoint{13.672716in}{5.539504in}}%
\pgfpathclose%
\pgfusepath{stroke}%
\end{pgfscope}%
\begin{pgfscope}%
\pgfpathrectangle{\pgfqpoint{7.512535in}{0.437222in}}{\pgfqpoint{6.275590in}{5.159444in}}%
\pgfusepath{clip}%
\pgfsetbuttcap%
\pgfsetroundjoin%
\pgfsetlinewidth{1.003750pt}%
\definecolor{currentstroke}{rgb}{0.827451,0.827451,0.827451}%
\pgfsetstrokecolor{currentstroke}%
\pgfsetstrokeopacity{0.800000}%
\pgfsetdash{}{0pt}%
\pgfpathmoveto{\pgfqpoint{12.032088in}{5.330293in}}%
\pgfpathcurveto{\pgfqpoint{12.043138in}{5.330293in}}{\pgfqpoint{12.053737in}{5.334683in}}{\pgfqpoint{12.061550in}{5.342497in}}%
\pgfpathcurveto{\pgfqpoint{12.069364in}{5.350311in}}{\pgfqpoint{12.073754in}{5.360910in}}{\pgfqpoint{12.073754in}{5.371960in}}%
\pgfpathcurveto{\pgfqpoint{12.073754in}{5.383010in}}{\pgfqpoint{12.069364in}{5.393609in}}{\pgfqpoint{12.061550in}{5.401423in}}%
\pgfpathcurveto{\pgfqpoint{12.053737in}{5.409236in}}{\pgfqpoint{12.043138in}{5.413626in}}{\pgfqpoint{12.032088in}{5.413626in}}%
\pgfpathcurveto{\pgfqpoint{12.021038in}{5.413626in}}{\pgfqpoint{12.010439in}{5.409236in}}{\pgfqpoint{12.002625in}{5.401423in}}%
\pgfpathcurveto{\pgfqpoint{11.994811in}{5.393609in}}{\pgfqpoint{11.990421in}{5.383010in}}{\pgfqpoint{11.990421in}{5.371960in}}%
\pgfpathcurveto{\pgfqpoint{11.990421in}{5.360910in}}{\pgfqpoint{11.994811in}{5.350311in}}{\pgfqpoint{12.002625in}{5.342497in}}%
\pgfpathcurveto{\pgfqpoint{12.010439in}{5.334683in}}{\pgfqpoint{12.021038in}{5.330293in}}{\pgfqpoint{12.032088in}{5.330293in}}%
\pgfpathlineto{\pgfqpoint{12.032088in}{5.330293in}}%
\pgfpathclose%
\pgfusepath{stroke}%
\end{pgfscope}%
\begin{pgfscope}%
\pgfpathrectangle{\pgfqpoint{7.512535in}{0.437222in}}{\pgfqpoint{6.275590in}{5.159444in}}%
\pgfusepath{clip}%
\pgfsetbuttcap%
\pgfsetroundjoin%
\pgfsetlinewidth{1.003750pt}%
\definecolor{currentstroke}{rgb}{0.827451,0.827451,0.827451}%
\pgfsetstrokecolor{currentstroke}%
\pgfsetstrokeopacity{0.800000}%
\pgfsetdash{}{0pt}%
\pgfpathmoveto{\pgfqpoint{13.563316in}{5.539504in}}%
\pgfpathcurveto{\pgfqpoint{13.574366in}{5.539504in}}{\pgfqpoint{13.584965in}{5.543895in}}{\pgfqpoint{13.592778in}{5.551708in}}%
\pgfpathcurveto{\pgfqpoint{13.600592in}{5.559522in}}{\pgfqpoint{13.604982in}{5.570121in}}{\pgfqpoint{13.604982in}{5.581171in}}%
\pgfpathcurveto{\pgfqpoint{13.604982in}{5.592221in}}{\pgfqpoint{13.600592in}{5.602820in}}{\pgfqpoint{13.592778in}{5.610634in}}%
\pgfpathcurveto{\pgfqpoint{13.584965in}{5.618447in}}{\pgfqpoint{13.574366in}{5.622838in}}{\pgfqpoint{13.563316in}{5.622838in}}%
\pgfpathcurveto{\pgfqpoint{13.552265in}{5.622838in}}{\pgfqpoint{13.541666in}{5.618447in}}{\pgfqpoint{13.533853in}{5.610634in}}%
\pgfpathcurveto{\pgfqpoint{13.526039in}{5.602820in}}{\pgfqpoint{13.521649in}{5.592221in}}{\pgfqpoint{13.521649in}{5.581171in}}%
\pgfpathcurveto{\pgfqpoint{13.521649in}{5.570121in}}{\pgfqpoint{13.526039in}{5.559522in}}{\pgfqpoint{13.533853in}{5.551708in}}%
\pgfpathcurveto{\pgfqpoint{13.541666in}{5.543895in}}{\pgfqpoint{13.552265in}{5.539504in}}{\pgfqpoint{13.563316in}{5.539504in}}%
\pgfpathlineto{\pgfqpoint{13.563316in}{5.539504in}}%
\pgfpathclose%
\pgfusepath{stroke}%
\end{pgfscope}%
\begin{pgfscope}%
\pgfpathrectangle{\pgfqpoint{7.512535in}{0.437222in}}{\pgfqpoint{6.275590in}{5.159444in}}%
\pgfusepath{clip}%
\pgfsetbuttcap%
\pgfsetroundjoin%
\pgfsetlinewidth{1.003750pt}%
\definecolor{currentstroke}{rgb}{0.827451,0.827451,0.827451}%
\pgfsetstrokecolor{currentstroke}%
\pgfsetstrokeopacity{0.800000}%
\pgfsetdash{}{0pt}%
\pgfpathmoveto{\pgfqpoint{9.395327in}{2.160540in}}%
\pgfpathcurveto{\pgfqpoint{9.406377in}{2.160540in}}{\pgfqpoint{9.416976in}{2.164930in}}{\pgfqpoint{9.424790in}{2.172744in}}%
\pgfpathcurveto{\pgfqpoint{9.432604in}{2.180557in}}{\pgfqpoint{9.436994in}{2.191156in}}{\pgfqpoint{9.436994in}{2.202207in}}%
\pgfpathcurveto{\pgfqpoint{9.436994in}{2.213257in}}{\pgfqpoint{9.432604in}{2.223856in}}{\pgfqpoint{9.424790in}{2.231669in}}%
\pgfpathcurveto{\pgfqpoint{9.416976in}{2.239483in}}{\pgfqpoint{9.406377in}{2.243873in}}{\pgfqpoint{9.395327in}{2.243873in}}%
\pgfpathcurveto{\pgfqpoint{9.384277in}{2.243873in}}{\pgfqpoint{9.373678in}{2.239483in}}{\pgfqpoint{9.365865in}{2.231669in}}%
\pgfpathcurveto{\pgfqpoint{9.358051in}{2.223856in}}{\pgfqpoint{9.353661in}{2.213257in}}{\pgfqpoint{9.353661in}{2.202207in}}%
\pgfpathcurveto{\pgfqpoint{9.353661in}{2.191156in}}{\pgfqpoint{9.358051in}{2.180557in}}{\pgfqpoint{9.365865in}{2.172744in}}%
\pgfpathcurveto{\pgfqpoint{9.373678in}{2.164930in}}{\pgfqpoint{9.384277in}{2.160540in}}{\pgfqpoint{9.395327in}{2.160540in}}%
\pgfpathlineto{\pgfqpoint{9.395327in}{2.160540in}}%
\pgfpathclose%
\pgfusepath{stroke}%
\end{pgfscope}%
\begin{pgfscope}%
\pgfpathrectangle{\pgfqpoint{7.512535in}{0.437222in}}{\pgfqpoint{6.275590in}{5.159444in}}%
\pgfusepath{clip}%
\pgfsetbuttcap%
\pgfsetroundjoin%
\pgfsetlinewidth{1.003750pt}%
\definecolor{currentstroke}{rgb}{0.827451,0.827451,0.827451}%
\pgfsetstrokecolor{currentstroke}%
\pgfsetstrokeopacity{0.800000}%
\pgfsetdash{}{0pt}%
\pgfpathmoveto{\pgfqpoint{12.650243in}{5.481847in}}%
\pgfpathcurveto{\pgfqpoint{12.661293in}{5.481847in}}{\pgfqpoint{12.671892in}{5.486238in}}{\pgfqpoint{12.679706in}{5.494051in}}%
\pgfpathcurveto{\pgfqpoint{12.687519in}{5.501865in}}{\pgfqpoint{12.691909in}{5.512464in}}{\pgfqpoint{12.691909in}{5.523514in}}%
\pgfpathcurveto{\pgfqpoint{12.691909in}{5.534564in}}{\pgfqpoint{12.687519in}{5.545163in}}{\pgfqpoint{12.679706in}{5.552977in}}%
\pgfpathcurveto{\pgfqpoint{12.671892in}{5.560791in}}{\pgfqpoint{12.661293in}{5.565181in}}{\pgfqpoint{12.650243in}{5.565181in}}%
\pgfpathcurveto{\pgfqpoint{12.639193in}{5.565181in}}{\pgfqpoint{12.628594in}{5.560791in}}{\pgfqpoint{12.620780in}{5.552977in}}%
\pgfpathcurveto{\pgfqpoint{12.612966in}{5.545163in}}{\pgfqpoint{12.608576in}{5.534564in}}{\pgfqpoint{12.608576in}{5.523514in}}%
\pgfpathcurveto{\pgfqpoint{12.608576in}{5.512464in}}{\pgfqpoint{12.612966in}{5.501865in}}{\pgfqpoint{12.620780in}{5.494051in}}%
\pgfpathcurveto{\pgfqpoint{12.628594in}{5.486238in}}{\pgfqpoint{12.639193in}{5.481847in}}{\pgfqpoint{12.650243in}{5.481847in}}%
\pgfpathlineto{\pgfqpoint{12.650243in}{5.481847in}}%
\pgfpathclose%
\pgfusepath{stroke}%
\end{pgfscope}%
\begin{pgfscope}%
\pgfpathrectangle{\pgfqpoint{7.512535in}{0.437222in}}{\pgfqpoint{6.275590in}{5.159444in}}%
\pgfusepath{clip}%
\pgfsetbuttcap%
\pgfsetroundjoin%
\pgfsetlinewidth{1.003750pt}%
\definecolor{currentstroke}{rgb}{0.827451,0.827451,0.827451}%
\pgfsetstrokecolor{currentstroke}%
\pgfsetstrokeopacity{0.800000}%
\pgfsetdash{}{0pt}%
\pgfpathmoveto{\pgfqpoint{9.266706in}{3.281458in}}%
\pgfpathcurveto{\pgfqpoint{9.277756in}{3.281458in}}{\pgfqpoint{9.288355in}{3.285848in}}{\pgfqpoint{9.296168in}{3.293661in}}%
\pgfpathcurveto{\pgfqpoint{9.303982in}{3.301475in}}{\pgfqpoint{9.308372in}{3.312074in}}{\pgfqpoint{9.308372in}{3.323124in}}%
\pgfpathcurveto{\pgfqpoint{9.308372in}{3.334174in}}{\pgfqpoint{9.303982in}{3.344773in}}{\pgfqpoint{9.296168in}{3.352587in}}%
\pgfpathcurveto{\pgfqpoint{9.288355in}{3.360401in}}{\pgfqpoint{9.277756in}{3.364791in}}{\pgfqpoint{9.266706in}{3.364791in}}%
\pgfpathcurveto{\pgfqpoint{9.255656in}{3.364791in}}{\pgfqpoint{9.245057in}{3.360401in}}{\pgfqpoint{9.237243in}{3.352587in}}%
\pgfpathcurveto{\pgfqpoint{9.229429in}{3.344773in}}{\pgfqpoint{9.225039in}{3.334174in}}{\pgfqpoint{9.225039in}{3.323124in}}%
\pgfpathcurveto{\pgfqpoint{9.225039in}{3.312074in}}{\pgfqpoint{9.229429in}{3.301475in}}{\pgfqpoint{9.237243in}{3.293661in}}%
\pgfpathcurveto{\pgfqpoint{9.245057in}{3.285848in}}{\pgfqpoint{9.255656in}{3.281458in}}{\pgfqpoint{9.266706in}{3.281458in}}%
\pgfpathlineto{\pgfqpoint{9.266706in}{3.281458in}}%
\pgfpathclose%
\pgfusepath{stroke}%
\end{pgfscope}%
\begin{pgfscope}%
\pgfpathrectangle{\pgfqpoint{7.512535in}{0.437222in}}{\pgfqpoint{6.275590in}{5.159444in}}%
\pgfusepath{clip}%
\pgfsetbuttcap%
\pgfsetroundjoin%
\pgfsetlinewidth{1.003750pt}%
\definecolor{currentstroke}{rgb}{0.827451,0.827451,0.827451}%
\pgfsetstrokecolor{currentstroke}%
\pgfsetstrokeopacity{0.800000}%
\pgfsetdash{}{0pt}%
\pgfpathmoveto{\pgfqpoint{8.290348in}{1.139798in}}%
\pgfpathcurveto{\pgfqpoint{8.301398in}{1.139798in}}{\pgfqpoint{8.311997in}{1.144188in}}{\pgfqpoint{8.319811in}{1.152002in}}%
\pgfpathcurveto{\pgfqpoint{8.327625in}{1.159815in}}{\pgfqpoint{8.332015in}{1.170414in}}{\pgfqpoint{8.332015in}{1.181465in}}%
\pgfpathcurveto{\pgfqpoint{8.332015in}{1.192515in}}{\pgfqpoint{8.327625in}{1.203114in}}{\pgfqpoint{8.319811in}{1.210927in}}%
\pgfpathcurveto{\pgfqpoint{8.311997in}{1.218741in}}{\pgfqpoint{8.301398in}{1.223131in}}{\pgfqpoint{8.290348in}{1.223131in}}%
\pgfpathcurveto{\pgfqpoint{8.279298in}{1.223131in}}{\pgfqpoint{8.268699in}{1.218741in}}{\pgfqpoint{8.260886in}{1.210927in}}%
\pgfpathcurveto{\pgfqpoint{8.253072in}{1.203114in}}{\pgfqpoint{8.248682in}{1.192515in}}{\pgfqpoint{8.248682in}{1.181465in}}%
\pgfpathcurveto{\pgfqpoint{8.248682in}{1.170414in}}{\pgfqpoint{8.253072in}{1.159815in}}{\pgfqpoint{8.260886in}{1.152002in}}%
\pgfpathcurveto{\pgfqpoint{8.268699in}{1.144188in}}{\pgfqpoint{8.279298in}{1.139798in}}{\pgfqpoint{8.290348in}{1.139798in}}%
\pgfpathlineto{\pgfqpoint{8.290348in}{1.139798in}}%
\pgfpathclose%
\pgfusepath{stroke}%
\end{pgfscope}%
\begin{pgfscope}%
\pgfpathrectangle{\pgfqpoint{7.512535in}{0.437222in}}{\pgfqpoint{6.275590in}{5.159444in}}%
\pgfusepath{clip}%
\pgfsetbuttcap%
\pgfsetroundjoin%
\pgfsetlinewidth{1.003750pt}%
\definecolor{currentstroke}{rgb}{0.827451,0.827451,0.827451}%
\pgfsetstrokecolor{currentstroke}%
\pgfsetstrokeopacity{0.800000}%
\pgfsetdash{}{0pt}%
\pgfpathmoveto{\pgfqpoint{13.419279in}{5.549860in}}%
\pgfpathcurveto{\pgfqpoint{13.430329in}{5.549860in}}{\pgfqpoint{13.440928in}{5.554250in}}{\pgfqpoint{13.448742in}{5.562064in}}%
\pgfpathcurveto{\pgfqpoint{13.456555in}{5.569877in}}{\pgfqpoint{13.460945in}{5.580476in}}{\pgfqpoint{13.460945in}{5.591526in}}%
\pgfpathcurveto{\pgfqpoint{13.460945in}{5.602576in}}{\pgfqpoint{13.456555in}{5.613175in}}{\pgfqpoint{13.448742in}{5.620989in}}%
\pgfpathcurveto{\pgfqpoint{13.440928in}{5.628803in}}{\pgfqpoint{13.430329in}{5.633193in}}{\pgfqpoint{13.419279in}{5.633193in}}%
\pgfpathcurveto{\pgfqpoint{13.408229in}{5.633193in}}{\pgfqpoint{13.397630in}{5.628803in}}{\pgfqpoint{13.389816in}{5.620989in}}%
\pgfpathcurveto{\pgfqpoint{13.382002in}{5.613175in}}{\pgfqpoint{13.377612in}{5.602576in}}{\pgfqpoint{13.377612in}{5.591526in}}%
\pgfpathcurveto{\pgfqpoint{13.377612in}{5.580476in}}{\pgfqpoint{13.382002in}{5.569877in}}{\pgfqpoint{13.389816in}{5.562064in}}%
\pgfpathcurveto{\pgfqpoint{13.397630in}{5.554250in}}{\pgfqpoint{13.408229in}{5.549860in}}{\pgfqpoint{13.419279in}{5.549860in}}%
\pgfpathlineto{\pgfqpoint{13.419279in}{5.549860in}}%
\pgfpathclose%
\pgfusepath{stroke}%
\end{pgfscope}%
\begin{pgfscope}%
\pgfpathrectangle{\pgfqpoint{7.512535in}{0.437222in}}{\pgfqpoint{6.275590in}{5.159444in}}%
\pgfusepath{clip}%
\pgfsetbuttcap%
\pgfsetroundjoin%
\pgfsetlinewidth{1.003750pt}%
\definecolor{currentstroke}{rgb}{0.827451,0.827451,0.827451}%
\pgfsetstrokecolor{currentstroke}%
\pgfsetstrokeopacity{0.800000}%
\pgfsetdash{}{0pt}%
\pgfpathmoveto{\pgfqpoint{10.259535in}{4.447570in}}%
\pgfpathcurveto{\pgfqpoint{10.270585in}{4.447570in}}{\pgfqpoint{10.281184in}{4.451960in}}{\pgfqpoint{10.288997in}{4.459773in}}%
\pgfpathcurveto{\pgfqpoint{10.296811in}{4.467587in}}{\pgfqpoint{10.301201in}{4.478186in}}{\pgfqpoint{10.301201in}{4.489236in}}%
\pgfpathcurveto{\pgfqpoint{10.301201in}{4.500286in}}{\pgfqpoint{10.296811in}{4.510885in}}{\pgfqpoint{10.288997in}{4.518699in}}%
\pgfpathcurveto{\pgfqpoint{10.281184in}{4.526513in}}{\pgfqpoint{10.270585in}{4.530903in}}{\pgfqpoint{10.259535in}{4.530903in}}%
\pgfpathcurveto{\pgfqpoint{10.248485in}{4.530903in}}{\pgfqpoint{10.237886in}{4.526513in}}{\pgfqpoint{10.230072in}{4.518699in}}%
\pgfpathcurveto{\pgfqpoint{10.222258in}{4.510885in}}{\pgfqpoint{10.217868in}{4.500286in}}{\pgfqpoint{10.217868in}{4.489236in}}%
\pgfpathcurveto{\pgfqpoint{10.217868in}{4.478186in}}{\pgfqpoint{10.222258in}{4.467587in}}{\pgfqpoint{10.230072in}{4.459773in}}%
\pgfpathcurveto{\pgfqpoint{10.237886in}{4.451960in}}{\pgfqpoint{10.248485in}{4.447570in}}{\pgfqpoint{10.259535in}{4.447570in}}%
\pgfpathlineto{\pgfqpoint{10.259535in}{4.447570in}}%
\pgfpathclose%
\pgfusepath{stroke}%
\end{pgfscope}%
\begin{pgfscope}%
\pgfpathrectangle{\pgfqpoint{7.512535in}{0.437222in}}{\pgfqpoint{6.275590in}{5.159444in}}%
\pgfusepath{clip}%
\pgfsetbuttcap%
\pgfsetroundjoin%
\pgfsetlinewidth{1.003750pt}%
\definecolor{currentstroke}{rgb}{0.827451,0.827451,0.827451}%
\pgfsetstrokecolor{currentstroke}%
\pgfsetstrokeopacity{0.800000}%
\pgfsetdash{}{0pt}%
\pgfpathmoveto{\pgfqpoint{8.290348in}{1.011495in}}%
\pgfpathcurveto{\pgfqpoint{8.301398in}{1.011495in}}{\pgfqpoint{8.311997in}{1.015885in}}{\pgfqpoint{8.319811in}{1.023699in}}%
\pgfpathcurveto{\pgfqpoint{8.327625in}{1.031512in}}{\pgfqpoint{8.332015in}{1.042111in}}{\pgfqpoint{8.332015in}{1.053161in}}%
\pgfpathcurveto{\pgfqpoint{8.332015in}{1.064211in}}{\pgfqpoint{8.327625in}{1.074810in}}{\pgfqpoint{8.319811in}{1.082624in}}%
\pgfpathcurveto{\pgfqpoint{8.311997in}{1.090438in}}{\pgfqpoint{8.301398in}{1.094828in}}{\pgfqpoint{8.290348in}{1.094828in}}%
\pgfpathcurveto{\pgfqpoint{8.279298in}{1.094828in}}{\pgfqpoint{8.268699in}{1.090438in}}{\pgfqpoint{8.260886in}{1.082624in}}%
\pgfpathcurveto{\pgfqpoint{8.253072in}{1.074810in}}{\pgfqpoint{8.248682in}{1.064211in}}{\pgfqpoint{8.248682in}{1.053161in}}%
\pgfpathcurveto{\pgfqpoint{8.248682in}{1.042111in}}{\pgfqpoint{8.253072in}{1.031512in}}{\pgfqpoint{8.260886in}{1.023699in}}%
\pgfpathcurveto{\pgfqpoint{8.268699in}{1.015885in}}{\pgfqpoint{8.279298in}{1.011495in}}{\pgfqpoint{8.290348in}{1.011495in}}%
\pgfpathlineto{\pgfqpoint{8.290348in}{1.011495in}}%
\pgfpathclose%
\pgfusepath{stroke}%
\end{pgfscope}%
\begin{pgfscope}%
\pgfpathrectangle{\pgfqpoint{7.512535in}{0.437222in}}{\pgfqpoint{6.275590in}{5.159444in}}%
\pgfusepath{clip}%
\pgfsetbuttcap%
\pgfsetroundjoin%
\pgfsetlinewidth{1.003750pt}%
\definecolor{currentstroke}{rgb}{0.827451,0.827451,0.827451}%
\pgfsetstrokecolor{currentstroke}%
\pgfsetstrokeopacity{0.800000}%
\pgfsetdash{}{0pt}%
\pgfpathmoveto{\pgfqpoint{8.290348in}{1.426234in}}%
\pgfpathcurveto{\pgfqpoint{8.301398in}{1.426234in}}{\pgfqpoint{8.311997in}{1.430624in}}{\pgfqpoint{8.319811in}{1.438438in}}%
\pgfpathcurveto{\pgfqpoint{8.327625in}{1.446251in}}{\pgfqpoint{8.332015in}{1.456850in}}{\pgfqpoint{8.332015in}{1.467900in}}%
\pgfpathcurveto{\pgfqpoint{8.332015in}{1.478951in}}{\pgfqpoint{8.327625in}{1.489550in}}{\pgfqpoint{8.319811in}{1.497363in}}%
\pgfpathcurveto{\pgfqpoint{8.311997in}{1.505177in}}{\pgfqpoint{8.301398in}{1.509567in}}{\pgfqpoint{8.290348in}{1.509567in}}%
\pgfpathcurveto{\pgfqpoint{8.279298in}{1.509567in}}{\pgfqpoint{8.268699in}{1.505177in}}{\pgfqpoint{8.260886in}{1.497363in}}%
\pgfpathcurveto{\pgfqpoint{8.253072in}{1.489550in}}{\pgfqpoint{8.248682in}{1.478951in}}{\pgfqpoint{8.248682in}{1.467900in}}%
\pgfpathcurveto{\pgfqpoint{8.248682in}{1.456850in}}{\pgfqpoint{8.253072in}{1.446251in}}{\pgfqpoint{8.260886in}{1.438438in}}%
\pgfpathcurveto{\pgfqpoint{8.268699in}{1.430624in}}{\pgfqpoint{8.279298in}{1.426234in}}{\pgfqpoint{8.290348in}{1.426234in}}%
\pgfpathlineto{\pgfqpoint{8.290348in}{1.426234in}}%
\pgfpathclose%
\pgfusepath{stroke}%
\end{pgfscope}%
\begin{pgfscope}%
\pgfpathrectangle{\pgfqpoint{7.512535in}{0.437222in}}{\pgfqpoint{6.275590in}{5.159444in}}%
\pgfusepath{clip}%
\pgfsetbuttcap%
\pgfsetroundjoin%
\pgfsetlinewidth{1.003750pt}%
\definecolor{currentstroke}{rgb}{0.827451,0.827451,0.827451}%
\pgfsetstrokecolor{currentstroke}%
\pgfsetstrokeopacity{0.800000}%
\pgfsetdash{}{0pt}%
\pgfpathmoveto{\pgfqpoint{11.517254in}{5.159213in}}%
\pgfpathcurveto{\pgfqpoint{11.528304in}{5.159213in}}{\pgfqpoint{11.538903in}{5.163603in}}{\pgfqpoint{11.546717in}{5.171417in}}%
\pgfpathcurveto{\pgfqpoint{11.554531in}{5.179230in}}{\pgfqpoint{11.558921in}{5.189829in}}{\pgfqpoint{11.558921in}{5.200879in}}%
\pgfpathcurveto{\pgfqpoint{11.558921in}{5.211930in}}{\pgfqpoint{11.554531in}{5.222529in}}{\pgfqpoint{11.546717in}{5.230342in}}%
\pgfpathcurveto{\pgfqpoint{11.538903in}{5.238156in}}{\pgfqpoint{11.528304in}{5.242546in}}{\pgfqpoint{11.517254in}{5.242546in}}%
\pgfpathcurveto{\pgfqpoint{11.506204in}{5.242546in}}{\pgfqpoint{11.495605in}{5.238156in}}{\pgfqpoint{11.487791in}{5.230342in}}%
\pgfpathcurveto{\pgfqpoint{11.479978in}{5.222529in}}{\pgfqpoint{11.475588in}{5.211930in}}{\pgfqpoint{11.475588in}{5.200879in}}%
\pgfpathcurveto{\pgfqpoint{11.475588in}{5.189829in}}{\pgfqpoint{11.479978in}{5.179230in}}{\pgfqpoint{11.487791in}{5.171417in}}%
\pgfpathcurveto{\pgfqpoint{11.495605in}{5.163603in}}{\pgfqpoint{11.506204in}{5.159213in}}{\pgfqpoint{11.517254in}{5.159213in}}%
\pgfpathlineto{\pgfqpoint{11.517254in}{5.159213in}}%
\pgfpathclose%
\pgfusepath{stroke}%
\end{pgfscope}%
\begin{pgfscope}%
\pgfpathrectangle{\pgfqpoint{7.512535in}{0.437222in}}{\pgfqpoint{6.275590in}{5.159444in}}%
\pgfusepath{clip}%
\pgfsetbuttcap%
\pgfsetroundjoin%
\pgfsetlinewidth{1.003750pt}%
\definecolor{currentstroke}{rgb}{0.827451,0.827451,0.827451}%
\pgfsetstrokecolor{currentstroke}%
\pgfsetstrokeopacity{0.800000}%
\pgfsetdash{}{0pt}%
\pgfpathmoveto{\pgfqpoint{8.432851in}{1.518124in}}%
\pgfpathcurveto{\pgfqpoint{8.443901in}{1.518124in}}{\pgfqpoint{8.454500in}{1.522514in}}{\pgfqpoint{8.462313in}{1.530328in}}%
\pgfpathcurveto{\pgfqpoint{8.470127in}{1.538142in}}{\pgfqpoint{8.474517in}{1.548741in}}{\pgfqpoint{8.474517in}{1.559791in}}%
\pgfpathcurveto{\pgfqpoint{8.474517in}{1.570841in}}{\pgfqpoint{8.470127in}{1.581440in}}{\pgfqpoint{8.462313in}{1.589254in}}%
\pgfpathcurveto{\pgfqpoint{8.454500in}{1.597067in}}{\pgfqpoint{8.443901in}{1.601457in}}{\pgfqpoint{8.432851in}{1.601457in}}%
\pgfpathcurveto{\pgfqpoint{8.421801in}{1.601457in}}{\pgfqpoint{8.411201in}{1.597067in}}{\pgfqpoint{8.403388in}{1.589254in}}%
\pgfpathcurveto{\pgfqpoint{8.395574in}{1.581440in}}{\pgfqpoint{8.391184in}{1.570841in}}{\pgfqpoint{8.391184in}{1.559791in}}%
\pgfpathcurveto{\pgfqpoint{8.391184in}{1.548741in}}{\pgfqpoint{8.395574in}{1.538142in}}{\pgfqpoint{8.403388in}{1.530328in}}%
\pgfpathcurveto{\pgfqpoint{8.411201in}{1.522514in}}{\pgfqpoint{8.421801in}{1.518124in}}{\pgfqpoint{8.432851in}{1.518124in}}%
\pgfpathlineto{\pgfqpoint{8.432851in}{1.518124in}}%
\pgfpathclose%
\pgfusepath{stroke}%
\end{pgfscope}%
\begin{pgfscope}%
\pgfpathrectangle{\pgfqpoint{7.512535in}{0.437222in}}{\pgfqpoint{6.275590in}{5.159444in}}%
\pgfusepath{clip}%
\pgfsetbuttcap%
\pgfsetroundjoin%
\pgfsetlinewidth{1.003750pt}%
\definecolor{currentstroke}{rgb}{0.827451,0.827451,0.827451}%
\pgfsetstrokecolor{currentstroke}%
\pgfsetstrokeopacity{0.800000}%
\pgfsetdash{}{0pt}%
\pgfpathmoveto{\pgfqpoint{10.712158in}{4.601092in}}%
\pgfpathcurveto{\pgfqpoint{10.723208in}{4.601092in}}{\pgfqpoint{10.733807in}{4.605482in}}{\pgfqpoint{10.741621in}{4.613296in}}%
\pgfpathcurveto{\pgfqpoint{10.749435in}{4.621109in}}{\pgfqpoint{10.753825in}{4.631708in}}{\pgfqpoint{10.753825in}{4.642759in}}%
\pgfpathcurveto{\pgfqpoint{10.753825in}{4.653809in}}{\pgfqpoint{10.749435in}{4.664408in}}{\pgfqpoint{10.741621in}{4.672221in}}%
\pgfpathcurveto{\pgfqpoint{10.733807in}{4.680035in}}{\pgfqpoint{10.723208in}{4.684425in}}{\pgfqpoint{10.712158in}{4.684425in}}%
\pgfpathcurveto{\pgfqpoint{10.701108in}{4.684425in}}{\pgfqpoint{10.690509in}{4.680035in}}{\pgfqpoint{10.682695in}{4.672221in}}%
\pgfpathcurveto{\pgfqpoint{10.674882in}{4.664408in}}{\pgfqpoint{10.670492in}{4.653809in}}{\pgfqpoint{10.670492in}{4.642759in}}%
\pgfpathcurveto{\pgfqpoint{10.670492in}{4.631708in}}{\pgfqpoint{10.674882in}{4.621109in}}{\pgfqpoint{10.682695in}{4.613296in}}%
\pgfpathcurveto{\pgfqpoint{10.690509in}{4.605482in}}{\pgfqpoint{10.701108in}{4.601092in}}{\pgfqpoint{10.712158in}{4.601092in}}%
\pgfpathlineto{\pgfqpoint{10.712158in}{4.601092in}}%
\pgfpathclose%
\pgfusepath{stroke}%
\end{pgfscope}%
\begin{pgfscope}%
\pgfpathrectangle{\pgfqpoint{7.512535in}{0.437222in}}{\pgfqpoint{6.275590in}{5.159444in}}%
\pgfusepath{clip}%
\pgfsetbuttcap%
\pgfsetroundjoin%
\pgfsetlinewidth{1.003750pt}%
\definecolor{currentstroke}{rgb}{0.827451,0.827451,0.827451}%
\pgfsetstrokecolor{currentstroke}%
\pgfsetstrokeopacity{0.800000}%
\pgfsetdash{}{0pt}%
\pgfpathmoveto{\pgfqpoint{10.156793in}{3.250664in}}%
\pgfpathcurveto{\pgfqpoint{10.167843in}{3.250664in}}{\pgfqpoint{10.178442in}{3.255054in}}{\pgfqpoint{10.186256in}{3.262868in}}%
\pgfpathcurveto{\pgfqpoint{10.194069in}{3.270682in}}{\pgfqpoint{10.198460in}{3.281281in}}{\pgfqpoint{10.198460in}{3.292331in}}%
\pgfpathcurveto{\pgfqpoint{10.198460in}{3.303381in}}{\pgfqpoint{10.194069in}{3.313980in}}{\pgfqpoint{10.186256in}{3.321794in}}%
\pgfpathcurveto{\pgfqpoint{10.178442in}{3.329607in}}{\pgfqpoint{10.167843in}{3.333998in}}{\pgfqpoint{10.156793in}{3.333998in}}%
\pgfpathcurveto{\pgfqpoint{10.145743in}{3.333998in}}{\pgfqpoint{10.135144in}{3.329607in}}{\pgfqpoint{10.127330in}{3.321794in}}%
\pgfpathcurveto{\pgfqpoint{10.119517in}{3.313980in}}{\pgfqpoint{10.115126in}{3.303381in}}{\pgfqpoint{10.115126in}{3.292331in}}%
\pgfpathcurveto{\pgfqpoint{10.115126in}{3.281281in}}{\pgfqpoint{10.119517in}{3.270682in}}{\pgfqpoint{10.127330in}{3.262868in}}%
\pgfpathcurveto{\pgfqpoint{10.135144in}{3.255054in}}{\pgfqpoint{10.145743in}{3.250664in}}{\pgfqpoint{10.156793in}{3.250664in}}%
\pgfpathlineto{\pgfqpoint{10.156793in}{3.250664in}}%
\pgfpathclose%
\pgfusepath{stroke}%
\end{pgfscope}%
\begin{pgfscope}%
\pgfpathrectangle{\pgfqpoint{7.512535in}{0.437222in}}{\pgfqpoint{6.275590in}{5.159444in}}%
\pgfusepath{clip}%
\pgfsetbuttcap%
\pgfsetroundjoin%
\pgfsetlinewidth{1.003750pt}%
\definecolor{currentstroke}{rgb}{0.827451,0.827451,0.827451}%
\pgfsetstrokecolor{currentstroke}%
\pgfsetstrokeopacity{0.800000}%
\pgfsetdash{}{0pt}%
\pgfpathmoveto{\pgfqpoint{10.369675in}{4.552452in}}%
\pgfpathcurveto{\pgfqpoint{10.380725in}{4.552452in}}{\pgfqpoint{10.391324in}{4.556842in}}{\pgfqpoint{10.399138in}{4.564656in}}%
\pgfpathcurveto{\pgfqpoint{10.406951in}{4.572470in}}{\pgfqpoint{10.411341in}{4.583069in}}{\pgfqpoint{10.411341in}{4.594119in}}%
\pgfpathcurveto{\pgfqpoint{10.411341in}{4.605169in}}{\pgfqpoint{10.406951in}{4.615768in}}{\pgfqpoint{10.399138in}{4.623582in}}%
\pgfpathcurveto{\pgfqpoint{10.391324in}{4.631395in}}{\pgfqpoint{10.380725in}{4.635785in}}{\pgfqpoint{10.369675in}{4.635785in}}%
\pgfpathcurveto{\pgfqpoint{10.358625in}{4.635785in}}{\pgfqpoint{10.348026in}{4.631395in}}{\pgfqpoint{10.340212in}{4.623582in}}%
\pgfpathcurveto{\pgfqpoint{10.332398in}{4.615768in}}{\pgfqpoint{10.328008in}{4.605169in}}{\pgfqpoint{10.328008in}{4.594119in}}%
\pgfpathcurveto{\pgfqpoint{10.328008in}{4.583069in}}{\pgfqpoint{10.332398in}{4.572470in}}{\pgfqpoint{10.340212in}{4.564656in}}%
\pgfpathcurveto{\pgfqpoint{10.348026in}{4.556842in}}{\pgfqpoint{10.358625in}{4.552452in}}{\pgfqpoint{10.369675in}{4.552452in}}%
\pgfpathlineto{\pgfqpoint{10.369675in}{4.552452in}}%
\pgfpathclose%
\pgfusepath{stroke}%
\end{pgfscope}%
\begin{pgfscope}%
\pgfpathrectangle{\pgfqpoint{7.512535in}{0.437222in}}{\pgfqpoint{6.275590in}{5.159444in}}%
\pgfusepath{clip}%
\pgfsetbuttcap%
\pgfsetroundjoin%
\pgfsetlinewidth{1.003750pt}%
\definecolor{currentstroke}{rgb}{0.827451,0.827451,0.827451}%
\pgfsetstrokecolor{currentstroke}%
\pgfsetstrokeopacity{0.800000}%
\pgfsetdash{}{0pt}%
\pgfpathmoveto{\pgfqpoint{7.874057in}{1.320145in}}%
\pgfpathcurveto{\pgfqpoint{7.885107in}{1.320145in}}{\pgfqpoint{7.895706in}{1.324535in}}{\pgfqpoint{7.903520in}{1.332349in}}%
\pgfpathcurveto{\pgfqpoint{7.911334in}{1.340162in}}{\pgfqpoint{7.915724in}{1.350761in}}{\pgfqpoint{7.915724in}{1.361811in}}%
\pgfpathcurveto{\pgfqpoint{7.915724in}{1.372862in}}{\pgfqpoint{7.911334in}{1.383461in}}{\pgfqpoint{7.903520in}{1.391274in}}%
\pgfpathcurveto{\pgfqpoint{7.895706in}{1.399088in}}{\pgfqpoint{7.885107in}{1.403478in}}{\pgfqpoint{7.874057in}{1.403478in}}%
\pgfpathcurveto{\pgfqpoint{7.863007in}{1.403478in}}{\pgfqpoint{7.852408in}{1.399088in}}{\pgfqpoint{7.844595in}{1.391274in}}%
\pgfpathcurveto{\pgfqpoint{7.836781in}{1.383461in}}{\pgfqpoint{7.832391in}{1.372862in}}{\pgfqpoint{7.832391in}{1.361811in}}%
\pgfpathcurveto{\pgfqpoint{7.832391in}{1.350761in}}{\pgfqpoint{7.836781in}{1.340162in}}{\pgfqpoint{7.844595in}{1.332349in}}%
\pgfpathcurveto{\pgfqpoint{7.852408in}{1.324535in}}{\pgfqpoint{7.863007in}{1.320145in}}{\pgfqpoint{7.874057in}{1.320145in}}%
\pgfpathlineto{\pgfqpoint{7.874057in}{1.320145in}}%
\pgfpathclose%
\pgfusepath{stroke}%
\end{pgfscope}%
\begin{pgfscope}%
\pgfpathrectangle{\pgfqpoint{7.512535in}{0.437222in}}{\pgfqpoint{6.275590in}{5.159444in}}%
\pgfusepath{clip}%
\pgfsetbuttcap%
\pgfsetroundjoin%
\pgfsetlinewidth{1.003750pt}%
\definecolor{currentstroke}{rgb}{0.827451,0.827451,0.827451}%
\pgfsetstrokecolor{currentstroke}%
\pgfsetstrokeopacity{0.800000}%
\pgfsetdash{}{0pt}%
\pgfpathmoveto{\pgfqpoint{11.629363in}{5.304399in}}%
\pgfpathcurveto{\pgfqpoint{11.640413in}{5.304399in}}{\pgfqpoint{11.651012in}{5.308790in}}{\pgfqpoint{11.658826in}{5.316603in}}%
\pgfpathcurveto{\pgfqpoint{11.666639in}{5.324417in}}{\pgfqpoint{11.671030in}{5.335016in}}{\pgfqpoint{11.671030in}{5.346066in}}%
\pgfpathcurveto{\pgfqpoint{11.671030in}{5.357116in}}{\pgfqpoint{11.666639in}{5.367715in}}{\pgfqpoint{11.658826in}{5.375529in}}%
\pgfpathcurveto{\pgfqpoint{11.651012in}{5.383342in}}{\pgfqpoint{11.640413in}{5.387733in}}{\pgfqpoint{11.629363in}{5.387733in}}%
\pgfpathcurveto{\pgfqpoint{11.618313in}{5.387733in}}{\pgfqpoint{11.607714in}{5.383342in}}{\pgfqpoint{11.599900in}{5.375529in}}%
\pgfpathcurveto{\pgfqpoint{11.592087in}{5.367715in}}{\pgfqpoint{11.587696in}{5.357116in}}{\pgfqpoint{11.587696in}{5.346066in}}%
\pgfpathcurveto{\pgfqpoint{11.587696in}{5.335016in}}{\pgfqpoint{11.592087in}{5.324417in}}{\pgfqpoint{11.599900in}{5.316603in}}%
\pgfpathcurveto{\pgfqpoint{11.607714in}{5.308790in}}{\pgfqpoint{11.618313in}{5.304399in}}{\pgfqpoint{11.629363in}{5.304399in}}%
\pgfpathlineto{\pgfqpoint{11.629363in}{5.304399in}}%
\pgfpathclose%
\pgfusepath{stroke}%
\end{pgfscope}%
\begin{pgfscope}%
\pgfpathrectangle{\pgfqpoint{7.512535in}{0.437222in}}{\pgfqpoint{6.275590in}{5.159444in}}%
\pgfusepath{clip}%
\pgfsetbuttcap%
\pgfsetroundjoin%
\pgfsetlinewidth{1.003750pt}%
\definecolor{currentstroke}{rgb}{0.827451,0.827451,0.827451}%
\pgfsetstrokecolor{currentstroke}%
\pgfsetstrokeopacity{0.800000}%
\pgfsetdash{}{0pt}%
\pgfpathmoveto{\pgfqpoint{10.385573in}{5.221855in}}%
\pgfpathcurveto{\pgfqpoint{10.396623in}{5.221855in}}{\pgfqpoint{10.407222in}{5.226245in}}{\pgfqpoint{10.415036in}{5.234058in}}%
\pgfpathcurveto{\pgfqpoint{10.422849in}{5.241872in}}{\pgfqpoint{10.427240in}{5.252471in}}{\pgfqpoint{10.427240in}{5.263521in}}%
\pgfpathcurveto{\pgfqpoint{10.427240in}{5.274571in}}{\pgfqpoint{10.422849in}{5.285170in}}{\pgfqpoint{10.415036in}{5.292984in}}%
\pgfpathcurveto{\pgfqpoint{10.407222in}{5.300798in}}{\pgfqpoint{10.396623in}{5.305188in}}{\pgfqpoint{10.385573in}{5.305188in}}%
\pgfpathcurveto{\pgfqpoint{10.374523in}{5.305188in}}{\pgfqpoint{10.363924in}{5.300798in}}{\pgfqpoint{10.356110in}{5.292984in}}%
\pgfpathcurveto{\pgfqpoint{10.348297in}{5.285170in}}{\pgfqpoint{10.343906in}{5.274571in}}{\pgfqpoint{10.343906in}{5.263521in}}%
\pgfpathcurveto{\pgfqpoint{10.343906in}{5.252471in}}{\pgfqpoint{10.348297in}{5.241872in}}{\pgfqpoint{10.356110in}{5.234058in}}%
\pgfpathcurveto{\pgfqpoint{10.363924in}{5.226245in}}{\pgfqpoint{10.374523in}{5.221855in}}{\pgfqpoint{10.385573in}{5.221855in}}%
\pgfpathlineto{\pgfqpoint{10.385573in}{5.221855in}}%
\pgfpathclose%
\pgfusepath{stroke}%
\end{pgfscope}%
\begin{pgfscope}%
\pgfpathrectangle{\pgfqpoint{7.512535in}{0.437222in}}{\pgfqpoint{6.275590in}{5.159444in}}%
\pgfusepath{clip}%
\pgfsetbuttcap%
\pgfsetroundjoin%
\pgfsetlinewidth{1.003750pt}%
\definecolor{currentstroke}{rgb}{0.827451,0.827451,0.827451}%
\pgfsetstrokecolor{currentstroke}%
\pgfsetstrokeopacity{0.800000}%
\pgfsetdash{}{0pt}%
\pgfpathmoveto{\pgfqpoint{11.999433in}{5.514216in}}%
\pgfpathcurveto{\pgfqpoint{12.010483in}{5.514216in}}{\pgfqpoint{12.021082in}{5.518606in}}{\pgfqpoint{12.028896in}{5.526420in}}%
\pgfpathcurveto{\pgfqpoint{12.036710in}{5.534233in}}{\pgfqpoint{12.041100in}{5.544832in}}{\pgfqpoint{12.041100in}{5.555882in}}%
\pgfpathcurveto{\pgfqpoint{12.041100in}{5.566933in}}{\pgfqpoint{12.036710in}{5.577532in}}{\pgfqpoint{12.028896in}{5.585345in}}%
\pgfpathcurveto{\pgfqpoint{12.021082in}{5.593159in}}{\pgfqpoint{12.010483in}{5.597549in}}{\pgfqpoint{11.999433in}{5.597549in}}%
\pgfpathcurveto{\pgfqpoint{11.988383in}{5.597549in}}{\pgfqpoint{11.977784in}{5.593159in}}{\pgfqpoint{11.969970in}{5.585345in}}%
\pgfpathcurveto{\pgfqpoint{11.962157in}{5.577532in}}{\pgfqpoint{11.957767in}{5.566933in}}{\pgfqpoint{11.957767in}{5.555882in}}%
\pgfpathcurveto{\pgfqpoint{11.957767in}{5.544832in}}{\pgfqpoint{11.962157in}{5.534233in}}{\pgfqpoint{11.969970in}{5.526420in}}%
\pgfpathcurveto{\pgfqpoint{11.977784in}{5.518606in}}{\pgfqpoint{11.988383in}{5.514216in}}{\pgfqpoint{11.999433in}{5.514216in}}%
\pgfpathlineto{\pgfqpoint{11.999433in}{5.514216in}}%
\pgfpathclose%
\pgfusepath{stroke}%
\end{pgfscope}%
\begin{pgfscope}%
\pgfpathrectangle{\pgfqpoint{7.512535in}{0.437222in}}{\pgfqpoint{6.275590in}{5.159444in}}%
\pgfusepath{clip}%
\pgfsetbuttcap%
\pgfsetroundjoin%
\pgfsetlinewidth{1.003750pt}%
\definecolor{currentstroke}{rgb}{0.827451,0.827451,0.827451}%
\pgfsetstrokecolor{currentstroke}%
\pgfsetstrokeopacity{0.800000}%
\pgfsetdash{}{0pt}%
\pgfpathmoveto{\pgfqpoint{8.724795in}{1.722485in}}%
\pgfpathcurveto{\pgfqpoint{8.735845in}{1.722485in}}{\pgfqpoint{8.746444in}{1.726876in}}{\pgfqpoint{8.754258in}{1.734689in}}%
\pgfpathcurveto{\pgfqpoint{8.762071in}{1.742503in}}{\pgfqpoint{8.766461in}{1.753102in}}{\pgfqpoint{8.766461in}{1.764152in}}%
\pgfpathcurveto{\pgfqpoint{8.766461in}{1.775202in}}{\pgfqpoint{8.762071in}{1.785801in}}{\pgfqpoint{8.754258in}{1.793615in}}%
\pgfpathcurveto{\pgfqpoint{8.746444in}{1.801428in}}{\pgfqpoint{8.735845in}{1.805819in}}{\pgfqpoint{8.724795in}{1.805819in}}%
\pgfpathcurveto{\pgfqpoint{8.713745in}{1.805819in}}{\pgfqpoint{8.703146in}{1.801428in}}{\pgfqpoint{8.695332in}{1.793615in}}%
\pgfpathcurveto{\pgfqpoint{8.687518in}{1.785801in}}{\pgfqpoint{8.683128in}{1.775202in}}{\pgfqpoint{8.683128in}{1.764152in}}%
\pgfpathcurveto{\pgfqpoint{8.683128in}{1.753102in}}{\pgfqpoint{8.687518in}{1.742503in}}{\pgfqpoint{8.695332in}{1.734689in}}%
\pgfpathcurveto{\pgfqpoint{8.703146in}{1.726876in}}{\pgfqpoint{8.713745in}{1.722485in}}{\pgfqpoint{8.724795in}{1.722485in}}%
\pgfpathlineto{\pgfqpoint{8.724795in}{1.722485in}}%
\pgfpathclose%
\pgfusepath{stroke}%
\end{pgfscope}%
\begin{pgfscope}%
\pgfpathrectangle{\pgfqpoint{7.512535in}{0.437222in}}{\pgfqpoint{6.275590in}{5.159444in}}%
\pgfusepath{clip}%
\pgfsetbuttcap%
\pgfsetroundjoin%
\pgfsetlinewidth{1.003750pt}%
\definecolor{currentstroke}{rgb}{0.827451,0.827451,0.827451}%
\pgfsetstrokecolor{currentstroke}%
\pgfsetstrokeopacity{0.800000}%
\pgfsetdash{}{0pt}%
\pgfpathmoveto{\pgfqpoint{12.384977in}{5.553091in}}%
\pgfpathcurveto{\pgfqpoint{12.396028in}{5.553091in}}{\pgfqpoint{12.406627in}{5.557481in}}{\pgfqpoint{12.414440in}{5.565295in}}%
\pgfpathcurveto{\pgfqpoint{12.422254in}{5.573108in}}{\pgfqpoint{12.426644in}{5.583707in}}{\pgfqpoint{12.426644in}{5.594757in}}%
\pgfpathcurveto{\pgfqpoint{12.426644in}{5.605808in}}{\pgfqpoint{12.422254in}{5.616407in}}{\pgfqpoint{12.414440in}{5.624220in}}%
\pgfpathcurveto{\pgfqpoint{12.406627in}{5.632034in}}{\pgfqpoint{12.396028in}{5.636424in}}{\pgfqpoint{12.384977in}{5.636424in}}%
\pgfpathcurveto{\pgfqpoint{12.373927in}{5.636424in}}{\pgfqpoint{12.363328in}{5.632034in}}{\pgfqpoint{12.355515in}{5.624220in}}%
\pgfpathcurveto{\pgfqpoint{12.347701in}{5.616407in}}{\pgfqpoint{12.343311in}{5.605808in}}{\pgfqpoint{12.343311in}{5.594757in}}%
\pgfpathcurveto{\pgfqpoint{12.343311in}{5.583707in}}{\pgfqpoint{12.347701in}{5.573108in}}{\pgfqpoint{12.355515in}{5.565295in}}%
\pgfpathcurveto{\pgfqpoint{12.363328in}{5.557481in}}{\pgfqpoint{12.373927in}{5.553091in}}{\pgfqpoint{12.384977in}{5.553091in}}%
\pgfpathlineto{\pgfqpoint{12.384977in}{5.553091in}}%
\pgfpathclose%
\pgfusepath{stroke}%
\end{pgfscope}%
\begin{pgfscope}%
\pgfpathrectangle{\pgfqpoint{7.512535in}{0.437222in}}{\pgfqpoint{6.275590in}{5.159444in}}%
\pgfusepath{clip}%
\pgfsetbuttcap%
\pgfsetroundjoin%
\pgfsetlinewidth{1.003750pt}%
\definecolor{currentstroke}{rgb}{0.827451,0.827451,0.827451}%
\pgfsetstrokecolor{currentstroke}%
\pgfsetstrokeopacity{0.800000}%
\pgfsetdash{}{0pt}%
\pgfpathmoveto{\pgfqpoint{8.881634in}{1.944343in}}%
\pgfpathcurveto{\pgfqpoint{8.892684in}{1.944343in}}{\pgfqpoint{8.903283in}{1.948733in}}{\pgfqpoint{8.911096in}{1.956546in}}%
\pgfpathcurveto{\pgfqpoint{8.918910in}{1.964360in}}{\pgfqpoint{8.923300in}{1.974959in}}{\pgfqpoint{8.923300in}{1.986009in}}%
\pgfpathcurveto{\pgfqpoint{8.923300in}{1.997059in}}{\pgfqpoint{8.918910in}{2.007658in}}{\pgfqpoint{8.911096in}{2.015472in}}%
\pgfpathcurveto{\pgfqpoint{8.903283in}{2.023286in}}{\pgfqpoint{8.892684in}{2.027676in}}{\pgfqpoint{8.881634in}{2.027676in}}%
\pgfpathcurveto{\pgfqpoint{8.870583in}{2.027676in}}{\pgfqpoint{8.859984in}{2.023286in}}{\pgfqpoint{8.852171in}{2.015472in}}%
\pgfpathcurveto{\pgfqpoint{8.844357in}{2.007658in}}{\pgfqpoint{8.839967in}{1.997059in}}{\pgfqpoint{8.839967in}{1.986009in}}%
\pgfpathcurveto{\pgfqpoint{8.839967in}{1.974959in}}{\pgfqpoint{8.844357in}{1.964360in}}{\pgfqpoint{8.852171in}{1.956546in}}%
\pgfpathcurveto{\pgfqpoint{8.859984in}{1.948733in}}{\pgfqpoint{8.870583in}{1.944343in}}{\pgfqpoint{8.881634in}{1.944343in}}%
\pgfpathlineto{\pgfqpoint{8.881634in}{1.944343in}}%
\pgfpathclose%
\pgfusepath{stroke}%
\end{pgfscope}%
\begin{pgfscope}%
\pgfpathrectangle{\pgfqpoint{7.512535in}{0.437222in}}{\pgfqpoint{6.275590in}{5.159444in}}%
\pgfusepath{clip}%
\pgfsetbuttcap%
\pgfsetroundjoin%
\pgfsetlinewidth{1.003750pt}%
\definecolor{currentstroke}{rgb}{0.827451,0.827451,0.827451}%
\pgfsetstrokecolor{currentstroke}%
\pgfsetstrokeopacity{0.800000}%
\pgfsetdash{}{0pt}%
\pgfpathmoveto{\pgfqpoint{12.495601in}{5.523529in}}%
\pgfpathcurveto{\pgfqpoint{12.506651in}{5.523529in}}{\pgfqpoint{12.517250in}{5.527919in}}{\pgfqpoint{12.525063in}{5.535733in}}%
\pgfpathcurveto{\pgfqpoint{12.532877in}{5.543547in}}{\pgfqpoint{12.537267in}{5.554146in}}{\pgfqpoint{12.537267in}{5.565196in}}%
\pgfpathcurveto{\pgfqpoint{12.537267in}{5.576246in}}{\pgfqpoint{12.532877in}{5.586845in}}{\pgfqpoint{12.525063in}{5.594658in}}%
\pgfpathcurveto{\pgfqpoint{12.517250in}{5.602472in}}{\pgfqpoint{12.506651in}{5.606862in}}{\pgfqpoint{12.495601in}{5.606862in}}%
\pgfpathcurveto{\pgfqpoint{12.484550in}{5.606862in}}{\pgfqpoint{12.473951in}{5.602472in}}{\pgfqpoint{12.466138in}{5.594658in}}%
\pgfpathcurveto{\pgfqpoint{12.458324in}{5.586845in}}{\pgfqpoint{12.453934in}{5.576246in}}{\pgfqpoint{12.453934in}{5.565196in}}%
\pgfpathcurveto{\pgfqpoint{12.453934in}{5.554146in}}{\pgfqpoint{12.458324in}{5.543547in}}{\pgfqpoint{12.466138in}{5.535733in}}%
\pgfpathcurveto{\pgfqpoint{12.473951in}{5.527919in}}{\pgfqpoint{12.484550in}{5.523529in}}{\pgfqpoint{12.495601in}{5.523529in}}%
\pgfpathlineto{\pgfqpoint{12.495601in}{5.523529in}}%
\pgfpathclose%
\pgfusepath{stroke}%
\end{pgfscope}%
\begin{pgfscope}%
\pgfpathrectangle{\pgfqpoint{7.512535in}{0.437222in}}{\pgfqpoint{6.275590in}{5.159444in}}%
\pgfusepath{clip}%
\pgfsetbuttcap%
\pgfsetroundjoin%
\pgfsetlinewidth{1.003750pt}%
\definecolor{currentstroke}{rgb}{0.827451,0.827451,0.827451}%
\pgfsetstrokecolor{currentstroke}%
\pgfsetstrokeopacity{0.800000}%
\pgfsetdash{}{0pt}%
\pgfpathmoveto{\pgfqpoint{12.878574in}{5.554670in}}%
\pgfpathcurveto{\pgfqpoint{12.889624in}{5.554670in}}{\pgfqpoint{12.900223in}{5.559060in}}{\pgfqpoint{12.908037in}{5.566874in}}%
\pgfpathcurveto{\pgfqpoint{12.915851in}{5.574687in}}{\pgfqpoint{12.920241in}{5.585286in}}{\pgfqpoint{12.920241in}{5.596337in}}%
\pgfpathcurveto{\pgfqpoint{12.920241in}{5.607387in}}{\pgfqpoint{12.915851in}{5.617986in}}{\pgfqpoint{12.908037in}{5.625799in}}%
\pgfpathcurveto{\pgfqpoint{12.900223in}{5.633613in}}{\pgfqpoint{12.889624in}{5.638003in}}{\pgfqpoint{12.878574in}{5.638003in}}%
\pgfpathcurveto{\pgfqpoint{12.867524in}{5.638003in}}{\pgfqpoint{12.856925in}{5.633613in}}{\pgfqpoint{12.849111in}{5.625799in}}%
\pgfpathcurveto{\pgfqpoint{12.841298in}{5.617986in}}{\pgfqpoint{12.836907in}{5.607387in}}{\pgfqpoint{12.836907in}{5.596337in}}%
\pgfpathcurveto{\pgfqpoint{12.836907in}{5.585286in}}{\pgfqpoint{12.841298in}{5.574687in}}{\pgfqpoint{12.849111in}{5.566874in}}%
\pgfpathcurveto{\pgfqpoint{12.856925in}{5.559060in}}{\pgfqpoint{12.867524in}{5.554670in}}{\pgfqpoint{12.878574in}{5.554670in}}%
\pgfpathlineto{\pgfqpoint{12.878574in}{5.554670in}}%
\pgfpathclose%
\pgfusepath{stroke}%
\end{pgfscope}%
\begin{pgfscope}%
\pgfpathrectangle{\pgfqpoint{7.512535in}{0.437222in}}{\pgfqpoint{6.275590in}{5.159444in}}%
\pgfusepath{clip}%
\pgfsetbuttcap%
\pgfsetroundjoin%
\pgfsetlinewidth{1.003750pt}%
\definecolor{currentstroke}{rgb}{0.827451,0.827451,0.827451}%
\pgfsetstrokecolor{currentstroke}%
\pgfsetstrokeopacity{0.800000}%
\pgfsetdash{}{0pt}%
\pgfpathmoveto{\pgfqpoint{10.486066in}{4.475908in}}%
\pgfpathcurveto{\pgfqpoint{10.497116in}{4.475908in}}{\pgfqpoint{10.507715in}{4.480298in}}{\pgfqpoint{10.515529in}{4.488112in}}%
\pgfpathcurveto{\pgfqpoint{10.523342in}{4.495926in}}{\pgfqpoint{10.527733in}{4.506525in}}{\pgfqpoint{10.527733in}{4.517575in}}%
\pgfpathcurveto{\pgfqpoint{10.527733in}{4.528625in}}{\pgfqpoint{10.523342in}{4.539224in}}{\pgfqpoint{10.515529in}{4.547038in}}%
\pgfpathcurveto{\pgfqpoint{10.507715in}{4.554851in}}{\pgfqpoint{10.497116in}{4.559241in}}{\pgfqpoint{10.486066in}{4.559241in}}%
\pgfpathcurveto{\pgfqpoint{10.475016in}{4.559241in}}{\pgfqpoint{10.464417in}{4.554851in}}{\pgfqpoint{10.456603in}{4.547038in}}%
\pgfpathcurveto{\pgfqpoint{10.448790in}{4.539224in}}{\pgfqpoint{10.444399in}{4.528625in}}{\pgfqpoint{10.444399in}{4.517575in}}%
\pgfpathcurveto{\pgfqpoint{10.444399in}{4.506525in}}{\pgfqpoint{10.448790in}{4.495926in}}{\pgfqpoint{10.456603in}{4.488112in}}%
\pgfpathcurveto{\pgfqpoint{10.464417in}{4.480298in}}{\pgfqpoint{10.475016in}{4.475908in}}{\pgfqpoint{10.486066in}{4.475908in}}%
\pgfpathlineto{\pgfqpoint{10.486066in}{4.475908in}}%
\pgfpathclose%
\pgfusepath{stroke}%
\end{pgfscope}%
\begin{pgfscope}%
\pgfpathrectangle{\pgfqpoint{7.512535in}{0.437222in}}{\pgfqpoint{6.275590in}{5.159444in}}%
\pgfusepath{clip}%
\pgfsetbuttcap%
\pgfsetroundjoin%
\pgfsetlinewidth{1.003750pt}%
\definecolor{currentstroke}{rgb}{0.827451,0.827451,0.827451}%
\pgfsetstrokecolor{currentstroke}%
\pgfsetstrokeopacity{0.800000}%
\pgfsetdash{}{0pt}%
\pgfpathmoveto{\pgfqpoint{8.316930in}{1.743970in}}%
\pgfpathcurveto{\pgfqpoint{8.327980in}{1.743970in}}{\pgfqpoint{8.338579in}{1.748360in}}{\pgfqpoint{8.346393in}{1.756174in}}%
\pgfpathcurveto{\pgfqpoint{8.354206in}{1.763988in}}{\pgfqpoint{8.358596in}{1.774587in}}{\pgfqpoint{8.358596in}{1.785637in}}%
\pgfpathcurveto{\pgfqpoint{8.358596in}{1.796687in}}{\pgfqpoint{8.354206in}{1.807286in}}{\pgfqpoint{8.346393in}{1.815100in}}%
\pgfpathcurveto{\pgfqpoint{8.338579in}{1.822913in}}{\pgfqpoint{8.327980in}{1.827303in}}{\pgfqpoint{8.316930in}{1.827303in}}%
\pgfpathcurveto{\pgfqpoint{8.305880in}{1.827303in}}{\pgfqpoint{8.295281in}{1.822913in}}{\pgfqpoint{8.287467in}{1.815100in}}%
\pgfpathcurveto{\pgfqpoint{8.279653in}{1.807286in}}{\pgfqpoint{8.275263in}{1.796687in}}{\pgfqpoint{8.275263in}{1.785637in}}%
\pgfpathcurveto{\pgfqpoint{8.275263in}{1.774587in}}{\pgfqpoint{8.279653in}{1.763988in}}{\pgfqpoint{8.287467in}{1.756174in}}%
\pgfpathcurveto{\pgfqpoint{8.295281in}{1.748360in}}{\pgfqpoint{8.305880in}{1.743970in}}{\pgfqpoint{8.316930in}{1.743970in}}%
\pgfpathlineto{\pgfqpoint{8.316930in}{1.743970in}}%
\pgfpathclose%
\pgfusepath{stroke}%
\end{pgfscope}%
\begin{pgfscope}%
\pgfpathrectangle{\pgfqpoint{7.512535in}{0.437222in}}{\pgfqpoint{6.275590in}{5.159444in}}%
\pgfusepath{clip}%
\pgfsetbuttcap%
\pgfsetroundjoin%
\pgfsetlinewidth{1.003750pt}%
\definecolor{currentstroke}{rgb}{0.827451,0.827451,0.827451}%
\pgfsetstrokecolor{currentstroke}%
\pgfsetstrokeopacity{0.800000}%
\pgfsetdash{}{0pt}%
\pgfpathmoveto{\pgfqpoint{7.983939in}{1.000624in}}%
\pgfpathcurveto{\pgfqpoint{7.994989in}{1.000624in}}{\pgfqpoint{8.005588in}{1.005015in}}{\pgfqpoint{8.013401in}{1.012828in}}%
\pgfpathcurveto{\pgfqpoint{8.021215in}{1.020642in}}{\pgfqpoint{8.025605in}{1.031241in}}{\pgfqpoint{8.025605in}{1.042291in}}%
\pgfpathcurveto{\pgfqpoint{8.025605in}{1.053341in}}{\pgfqpoint{8.021215in}{1.063940in}}{\pgfqpoint{8.013401in}{1.071754in}}%
\pgfpathcurveto{\pgfqpoint{8.005588in}{1.079568in}}{\pgfqpoint{7.994989in}{1.083958in}}{\pgfqpoint{7.983939in}{1.083958in}}%
\pgfpathcurveto{\pgfqpoint{7.972888in}{1.083958in}}{\pgfqpoint{7.962289in}{1.079568in}}{\pgfqpoint{7.954476in}{1.071754in}}%
\pgfpathcurveto{\pgfqpoint{7.946662in}{1.063940in}}{\pgfqpoint{7.942272in}{1.053341in}}{\pgfqpoint{7.942272in}{1.042291in}}%
\pgfpathcurveto{\pgfqpoint{7.942272in}{1.031241in}}{\pgfqpoint{7.946662in}{1.020642in}}{\pgfqpoint{7.954476in}{1.012828in}}%
\pgfpathcurveto{\pgfqpoint{7.962289in}{1.005015in}}{\pgfqpoint{7.972888in}{1.000624in}}{\pgfqpoint{7.983939in}{1.000624in}}%
\pgfpathlineto{\pgfqpoint{7.983939in}{1.000624in}}%
\pgfpathclose%
\pgfusepath{stroke}%
\end{pgfscope}%
\begin{pgfscope}%
\pgfpathrectangle{\pgfqpoint{7.512535in}{0.437222in}}{\pgfqpoint{6.275590in}{5.159444in}}%
\pgfusepath{clip}%
\pgfsetbuttcap%
\pgfsetroundjoin%
\pgfsetlinewidth{1.003750pt}%
\definecolor{currentstroke}{rgb}{0.827451,0.827451,0.827451}%
\pgfsetstrokecolor{currentstroke}%
\pgfsetstrokeopacity{0.800000}%
\pgfsetdash{}{0pt}%
\pgfpathmoveto{\pgfqpoint{12.516073in}{5.554140in}}%
\pgfpathcurveto{\pgfqpoint{12.527123in}{5.554140in}}{\pgfqpoint{12.537722in}{5.558530in}}{\pgfqpoint{12.545536in}{5.566344in}}%
\pgfpathcurveto{\pgfqpoint{12.553349in}{5.574157in}}{\pgfqpoint{12.557740in}{5.584756in}}{\pgfqpoint{12.557740in}{5.595806in}}%
\pgfpathcurveto{\pgfqpoint{12.557740in}{5.606857in}}{\pgfqpoint{12.553349in}{5.617456in}}{\pgfqpoint{12.545536in}{5.625269in}}%
\pgfpathcurveto{\pgfqpoint{12.537722in}{5.633083in}}{\pgfqpoint{12.527123in}{5.637473in}}{\pgfqpoint{12.516073in}{5.637473in}}%
\pgfpathcurveto{\pgfqpoint{12.505023in}{5.637473in}}{\pgfqpoint{12.494424in}{5.633083in}}{\pgfqpoint{12.486610in}{5.625269in}}%
\pgfpathcurveto{\pgfqpoint{12.478797in}{5.617456in}}{\pgfqpoint{12.474406in}{5.606857in}}{\pgfqpoint{12.474406in}{5.595806in}}%
\pgfpathcurveto{\pgfqpoint{12.474406in}{5.584756in}}{\pgfqpoint{12.478797in}{5.574157in}}{\pgfqpoint{12.486610in}{5.566344in}}%
\pgfpathcurveto{\pgfqpoint{12.494424in}{5.558530in}}{\pgfqpoint{12.505023in}{5.554140in}}{\pgfqpoint{12.516073in}{5.554140in}}%
\pgfpathlineto{\pgfqpoint{12.516073in}{5.554140in}}%
\pgfpathclose%
\pgfusepath{stroke}%
\end{pgfscope}%
\begin{pgfscope}%
\pgfpathrectangle{\pgfqpoint{7.512535in}{0.437222in}}{\pgfqpoint{6.275590in}{5.159444in}}%
\pgfusepath{clip}%
\pgfsetbuttcap%
\pgfsetroundjoin%
\pgfsetlinewidth{1.003750pt}%
\definecolor{currentstroke}{rgb}{0.827451,0.827451,0.827451}%
\pgfsetstrokecolor{currentstroke}%
\pgfsetstrokeopacity{0.800000}%
\pgfsetdash{}{0pt}%
\pgfpathmoveto{\pgfqpoint{12.051194in}{5.538678in}}%
\pgfpathcurveto{\pgfqpoint{12.062244in}{5.538678in}}{\pgfqpoint{12.072843in}{5.543068in}}{\pgfqpoint{12.080656in}{5.550882in}}%
\pgfpathcurveto{\pgfqpoint{12.088470in}{5.558695in}}{\pgfqpoint{12.092860in}{5.569294in}}{\pgfqpoint{12.092860in}{5.580344in}}%
\pgfpathcurveto{\pgfqpoint{12.092860in}{5.591394in}}{\pgfqpoint{12.088470in}{5.601993in}}{\pgfqpoint{12.080656in}{5.609807in}}%
\pgfpathcurveto{\pgfqpoint{12.072843in}{5.617621in}}{\pgfqpoint{12.062244in}{5.622011in}}{\pgfqpoint{12.051194in}{5.622011in}}%
\pgfpathcurveto{\pgfqpoint{12.040144in}{5.622011in}}{\pgfqpoint{12.029544in}{5.617621in}}{\pgfqpoint{12.021731in}{5.609807in}}%
\pgfpathcurveto{\pgfqpoint{12.013917in}{5.601993in}}{\pgfqpoint{12.009527in}{5.591394in}}{\pgfqpoint{12.009527in}{5.580344in}}%
\pgfpathcurveto{\pgfqpoint{12.009527in}{5.569294in}}{\pgfqpoint{12.013917in}{5.558695in}}{\pgfqpoint{12.021731in}{5.550882in}}%
\pgfpathcurveto{\pgfqpoint{12.029544in}{5.543068in}}{\pgfqpoint{12.040144in}{5.538678in}}{\pgfqpoint{12.051194in}{5.538678in}}%
\pgfpathlineto{\pgfqpoint{12.051194in}{5.538678in}}%
\pgfpathclose%
\pgfusepath{stroke}%
\end{pgfscope}%
\begin{pgfscope}%
\pgfpathrectangle{\pgfqpoint{7.512535in}{0.437222in}}{\pgfqpoint{6.275590in}{5.159444in}}%
\pgfusepath{clip}%
\pgfsetbuttcap%
\pgfsetroundjoin%
\pgfsetlinewidth{1.003750pt}%
\definecolor{currentstroke}{rgb}{0.827451,0.827451,0.827451}%
\pgfsetstrokecolor{currentstroke}%
\pgfsetstrokeopacity{0.800000}%
\pgfsetdash{}{0pt}%
\pgfpathmoveto{\pgfqpoint{10.780511in}{5.205075in}}%
\pgfpathcurveto{\pgfqpoint{10.791561in}{5.205075in}}{\pgfqpoint{10.802160in}{5.209466in}}{\pgfqpoint{10.809973in}{5.217279in}}%
\pgfpathcurveto{\pgfqpoint{10.817787in}{5.225093in}}{\pgfqpoint{10.822177in}{5.235692in}}{\pgfqpoint{10.822177in}{5.246742in}}%
\pgfpathcurveto{\pgfqpoint{10.822177in}{5.257792in}}{\pgfqpoint{10.817787in}{5.268391in}}{\pgfqpoint{10.809973in}{5.276205in}}%
\pgfpathcurveto{\pgfqpoint{10.802160in}{5.284019in}}{\pgfqpoint{10.791561in}{5.288409in}}{\pgfqpoint{10.780511in}{5.288409in}}%
\pgfpathcurveto{\pgfqpoint{10.769461in}{5.288409in}}{\pgfqpoint{10.758862in}{5.284019in}}{\pgfqpoint{10.751048in}{5.276205in}}%
\pgfpathcurveto{\pgfqpoint{10.743234in}{5.268391in}}{\pgfqpoint{10.738844in}{5.257792in}}{\pgfqpoint{10.738844in}{5.246742in}}%
\pgfpathcurveto{\pgfqpoint{10.738844in}{5.235692in}}{\pgfqpoint{10.743234in}{5.225093in}}{\pgfqpoint{10.751048in}{5.217279in}}%
\pgfpathcurveto{\pgfqpoint{10.758862in}{5.209466in}}{\pgfqpoint{10.769461in}{5.205075in}}{\pgfqpoint{10.780511in}{5.205075in}}%
\pgfpathlineto{\pgfqpoint{10.780511in}{5.205075in}}%
\pgfpathclose%
\pgfusepath{stroke}%
\end{pgfscope}%
\begin{pgfscope}%
\pgfpathrectangle{\pgfqpoint{7.512535in}{0.437222in}}{\pgfqpoint{6.275590in}{5.159444in}}%
\pgfusepath{clip}%
\pgfsetbuttcap%
\pgfsetroundjoin%
\pgfsetlinewidth{1.003750pt}%
\definecolor{currentstroke}{rgb}{0.827451,0.827451,0.827451}%
\pgfsetstrokecolor{currentstroke}%
\pgfsetstrokeopacity{0.800000}%
\pgfsetdash{}{0pt}%
\pgfpathmoveto{\pgfqpoint{10.905755in}{4.601092in}}%
\pgfpathcurveto{\pgfqpoint{10.916805in}{4.601092in}}{\pgfqpoint{10.927404in}{4.605482in}}{\pgfqpoint{10.935217in}{4.613296in}}%
\pgfpathcurveto{\pgfqpoint{10.943031in}{4.621109in}}{\pgfqpoint{10.947421in}{4.631708in}}{\pgfqpoint{10.947421in}{4.642759in}}%
\pgfpathcurveto{\pgfqpoint{10.947421in}{4.653809in}}{\pgfqpoint{10.943031in}{4.664408in}}{\pgfqpoint{10.935217in}{4.672221in}}%
\pgfpathcurveto{\pgfqpoint{10.927404in}{4.680035in}}{\pgfqpoint{10.916805in}{4.684425in}}{\pgfqpoint{10.905755in}{4.684425in}}%
\pgfpathcurveto{\pgfqpoint{10.894704in}{4.684425in}}{\pgfqpoint{10.884105in}{4.680035in}}{\pgfqpoint{10.876292in}{4.672221in}}%
\pgfpathcurveto{\pgfqpoint{10.868478in}{4.664408in}}{\pgfqpoint{10.864088in}{4.653809in}}{\pgfqpoint{10.864088in}{4.642759in}}%
\pgfpathcurveto{\pgfqpoint{10.864088in}{4.631708in}}{\pgfqpoint{10.868478in}{4.621109in}}{\pgfqpoint{10.876292in}{4.613296in}}%
\pgfpathcurveto{\pgfqpoint{10.884105in}{4.605482in}}{\pgfqpoint{10.894704in}{4.601092in}}{\pgfqpoint{10.905755in}{4.601092in}}%
\pgfpathlineto{\pgfqpoint{10.905755in}{4.601092in}}%
\pgfpathclose%
\pgfusepath{stroke}%
\end{pgfscope}%
\begin{pgfscope}%
\pgfpathrectangle{\pgfqpoint{7.512535in}{0.437222in}}{\pgfqpoint{6.275590in}{5.159444in}}%
\pgfusepath{clip}%
\pgfsetbuttcap%
\pgfsetroundjoin%
\pgfsetlinewidth{1.003750pt}%
\definecolor{currentstroke}{rgb}{0.827451,0.827451,0.827451}%
\pgfsetstrokecolor{currentstroke}%
\pgfsetstrokeopacity{0.800000}%
\pgfsetdash{}{0pt}%
\pgfpathmoveto{\pgfqpoint{13.198067in}{5.528743in}}%
\pgfpathcurveto{\pgfqpoint{13.209117in}{5.528743in}}{\pgfqpoint{13.219716in}{5.533133in}}{\pgfqpoint{13.227530in}{5.540947in}}%
\pgfpathcurveto{\pgfqpoint{13.235343in}{5.548760in}}{\pgfqpoint{13.239734in}{5.559359in}}{\pgfqpoint{13.239734in}{5.570410in}}%
\pgfpathcurveto{\pgfqpoint{13.239734in}{5.581460in}}{\pgfqpoint{13.235343in}{5.592059in}}{\pgfqpoint{13.227530in}{5.599872in}}%
\pgfpathcurveto{\pgfqpoint{13.219716in}{5.607686in}}{\pgfqpoint{13.209117in}{5.612076in}}{\pgfqpoint{13.198067in}{5.612076in}}%
\pgfpathcurveto{\pgfqpoint{13.187017in}{5.612076in}}{\pgfqpoint{13.176418in}{5.607686in}}{\pgfqpoint{13.168604in}{5.599872in}}%
\pgfpathcurveto{\pgfqpoint{13.160791in}{5.592059in}}{\pgfqpoint{13.156400in}{5.581460in}}{\pgfqpoint{13.156400in}{5.570410in}}%
\pgfpathcurveto{\pgfqpoint{13.156400in}{5.559359in}}{\pgfqpoint{13.160791in}{5.548760in}}{\pgfqpoint{13.168604in}{5.540947in}}%
\pgfpathcurveto{\pgfqpoint{13.176418in}{5.533133in}}{\pgfqpoint{13.187017in}{5.528743in}}{\pgfqpoint{13.198067in}{5.528743in}}%
\pgfpathlineto{\pgfqpoint{13.198067in}{5.528743in}}%
\pgfpathclose%
\pgfusepath{stroke}%
\end{pgfscope}%
\begin{pgfscope}%
\pgfpathrectangle{\pgfqpoint{7.512535in}{0.437222in}}{\pgfqpoint{6.275590in}{5.159444in}}%
\pgfusepath{clip}%
\pgfsetbuttcap%
\pgfsetroundjoin%
\pgfsetlinewidth{1.003750pt}%
\definecolor{currentstroke}{rgb}{0.827451,0.827451,0.827451}%
\pgfsetstrokecolor{currentstroke}%
\pgfsetstrokeopacity{0.800000}%
\pgfsetdash{}{0pt}%
\pgfpathmoveto{\pgfqpoint{7.973923in}{0.766518in}}%
\pgfpathcurveto{\pgfqpoint{7.984973in}{0.766518in}}{\pgfqpoint{7.995572in}{0.770908in}}{\pgfqpoint{8.003385in}{0.778722in}}%
\pgfpathcurveto{\pgfqpoint{8.011199in}{0.786536in}}{\pgfqpoint{8.015589in}{0.797135in}}{\pgfqpoint{8.015589in}{0.808185in}}%
\pgfpathcurveto{\pgfqpoint{8.015589in}{0.819235in}}{\pgfqpoint{8.011199in}{0.829834in}}{\pgfqpoint{8.003385in}{0.837648in}}%
\pgfpathcurveto{\pgfqpoint{7.995572in}{0.845461in}}{\pgfqpoint{7.984973in}{0.849851in}}{\pgfqpoint{7.973923in}{0.849851in}}%
\pgfpathcurveto{\pgfqpoint{7.962873in}{0.849851in}}{\pgfqpoint{7.952274in}{0.845461in}}{\pgfqpoint{7.944460in}{0.837648in}}%
\pgfpathcurveto{\pgfqpoint{7.936646in}{0.829834in}}{\pgfqpoint{7.932256in}{0.819235in}}{\pgfqpoint{7.932256in}{0.808185in}}%
\pgfpathcurveto{\pgfqpoint{7.932256in}{0.797135in}}{\pgfqpoint{7.936646in}{0.786536in}}{\pgfqpoint{7.944460in}{0.778722in}}%
\pgfpathcurveto{\pgfqpoint{7.952274in}{0.770908in}}{\pgfqpoint{7.962873in}{0.766518in}}{\pgfqpoint{7.973923in}{0.766518in}}%
\pgfpathlineto{\pgfqpoint{7.973923in}{0.766518in}}%
\pgfpathclose%
\pgfusepath{stroke}%
\end{pgfscope}%
\begin{pgfscope}%
\pgfpathrectangle{\pgfqpoint{7.512535in}{0.437222in}}{\pgfqpoint{6.275590in}{5.159444in}}%
\pgfusepath{clip}%
\pgfsetbuttcap%
\pgfsetroundjoin%
\pgfsetlinewidth{1.003750pt}%
\definecolor{currentstroke}{rgb}{0.827451,0.827451,0.827451}%
\pgfsetstrokecolor{currentstroke}%
\pgfsetstrokeopacity{0.800000}%
\pgfsetdash{}{0pt}%
\pgfpathmoveto{\pgfqpoint{7.562575in}{0.493855in}}%
\pgfpathcurveto{\pgfqpoint{7.573625in}{0.493855in}}{\pgfqpoint{7.584224in}{0.498245in}}{\pgfqpoint{7.592038in}{0.506059in}}%
\pgfpathcurveto{\pgfqpoint{7.599851in}{0.513872in}}{\pgfqpoint{7.604241in}{0.524471in}}{\pgfqpoint{7.604241in}{0.535522in}}%
\pgfpathcurveto{\pgfqpoint{7.604241in}{0.546572in}}{\pgfqpoint{7.599851in}{0.557171in}}{\pgfqpoint{7.592038in}{0.564984in}}%
\pgfpathcurveto{\pgfqpoint{7.584224in}{0.572798in}}{\pgfqpoint{7.573625in}{0.577188in}}{\pgfqpoint{7.562575in}{0.577188in}}%
\pgfpathcurveto{\pgfqpoint{7.551525in}{0.577188in}}{\pgfqpoint{7.540926in}{0.572798in}}{\pgfqpoint{7.533112in}{0.564984in}}%
\pgfpathcurveto{\pgfqpoint{7.525298in}{0.557171in}}{\pgfqpoint{7.520908in}{0.546572in}}{\pgfqpoint{7.520908in}{0.535522in}}%
\pgfpathcurveto{\pgfqpoint{7.520908in}{0.524471in}}{\pgfqpoint{7.525298in}{0.513872in}}{\pgfqpoint{7.533112in}{0.506059in}}%
\pgfpathcurveto{\pgfqpoint{7.540926in}{0.498245in}}{\pgfqpoint{7.551525in}{0.493855in}}{\pgfqpoint{7.562575in}{0.493855in}}%
\pgfpathlineto{\pgfqpoint{7.562575in}{0.493855in}}%
\pgfpathclose%
\pgfusepath{stroke}%
\end{pgfscope}%
\begin{pgfscope}%
\pgfpathrectangle{\pgfqpoint{7.512535in}{0.437222in}}{\pgfqpoint{6.275590in}{5.159444in}}%
\pgfusepath{clip}%
\pgfsetbuttcap%
\pgfsetroundjoin%
\pgfsetlinewidth{1.003750pt}%
\definecolor{currentstroke}{rgb}{0.827451,0.827451,0.827451}%
\pgfsetstrokecolor{currentstroke}%
\pgfsetstrokeopacity{0.800000}%
\pgfsetdash{}{0pt}%
\pgfpathmoveto{\pgfqpoint{11.806382in}{5.548832in}}%
\pgfpathcurveto{\pgfqpoint{11.817432in}{5.548832in}}{\pgfqpoint{11.828031in}{5.553222in}}{\pgfqpoint{11.835845in}{5.561036in}}%
\pgfpathcurveto{\pgfqpoint{11.843658in}{5.568849in}}{\pgfqpoint{11.848049in}{5.579448in}}{\pgfqpoint{11.848049in}{5.590499in}}%
\pgfpathcurveto{\pgfqpoint{11.848049in}{5.601549in}}{\pgfqpoint{11.843658in}{5.612148in}}{\pgfqpoint{11.835845in}{5.619961in}}%
\pgfpathcurveto{\pgfqpoint{11.828031in}{5.627775in}}{\pgfqpoint{11.817432in}{5.632165in}}{\pgfqpoint{11.806382in}{5.632165in}}%
\pgfpathcurveto{\pgfqpoint{11.795332in}{5.632165in}}{\pgfqpoint{11.784733in}{5.627775in}}{\pgfqpoint{11.776919in}{5.619961in}}%
\pgfpathcurveto{\pgfqpoint{11.769106in}{5.612148in}}{\pgfqpoint{11.764715in}{5.601549in}}{\pgfqpoint{11.764715in}{5.590499in}}%
\pgfpathcurveto{\pgfqpoint{11.764715in}{5.579448in}}{\pgfqpoint{11.769106in}{5.568849in}}{\pgfqpoint{11.776919in}{5.561036in}}%
\pgfpathcurveto{\pgfqpoint{11.784733in}{5.553222in}}{\pgfqpoint{11.795332in}{5.548832in}}{\pgfqpoint{11.806382in}{5.548832in}}%
\pgfpathlineto{\pgfqpoint{11.806382in}{5.548832in}}%
\pgfpathclose%
\pgfusepath{stroke}%
\end{pgfscope}%
\begin{pgfscope}%
\pgfpathrectangle{\pgfqpoint{7.512535in}{0.437222in}}{\pgfqpoint{6.275590in}{5.159444in}}%
\pgfusepath{clip}%
\pgfsetbuttcap%
\pgfsetroundjoin%
\pgfsetlinewidth{1.003750pt}%
\definecolor{currentstroke}{rgb}{0.827451,0.827451,0.827451}%
\pgfsetstrokecolor{currentstroke}%
\pgfsetstrokeopacity{0.800000}%
\pgfsetdash{}{0pt}%
\pgfpathmoveto{\pgfqpoint{9.581399in}{3.231111in}}%
\pgfpathcurveto{\pgfqpoint{9.592449in}{3.231111in}}{\pgfqpoint{9.603048in}{3.235501in}}{\pgfqpoint{9.610862in}{3.243315in}}%
\pgfpathcurveto{\pgfqpoint{9.618675in}{3.251128in}}{\pgfqpoint{9.623065in}{3.261727in}}{\pgfqpoint{9.623065in}{3.272778in}}%
\pgfpathcurveto{\pgfqpoint{9.623065in}{3.283828in}}{\pgfqpoint{9.618675in}{3.294427in}}{\pgfqpoint{9.610862in}{3.302240in}}%
\pgfpathcurveto{\pgfqpoint{9.603048in}{3.310054in}}{\pgfqpoint{9.592449in}{3.314444in}}{\pgfqpoint{9.581399in}{3.314444in}}%
\pgfpathcurveto{\pgfqpoint{9.570349in}{3.314444in}}{\pgfqpoint{9.559750in}{3.310054in}}{\pgfqpoint{9.551936in}{3.302240in}}%
\pgfpathcurveto{\pgfqpoint{9.544122in}{3.294427in}}{\pgfqpoint{9.539732in}{3.283828in}}{\pgfqpoint{9.539732in}{3.272778in}}%
\pgfpathcurveto{\pgfqpoint{9.539732in}{3.261727in}}{\pgfqpoint{9.544122in}{3.251128in}}{\pgfqpoint{9.551936in}{3.243315in}}%
\pgfpathcurveto{\pgfqpoint{9.559750in}{3.235501in}}{\pgfqpoint{9.570349in}{3.231111in}}{\pgfqpoint{9.581399in}{3.231111in}}%
\pgfpathlineto{\pgfqpoint{9.581399in}{3.231111in}}%
\pgfpathclose%
\pgfusepath{stroke}%
\end{pgfscope}%
\begin{pgfscope}%
\pgfpathrectangle{\pgfqpoint{7.512535in}{0.437222in}}{\pgfqpoint{6.275590in}{5.159444in}}%
\pgfusepath{clip}%
\pgfsetbuttcap%
\pgfsetroundjoin%
\pgfsetlinewidth{1.003750pt}%
\definecolor{currentstroke}{rgb}{0.827451,0.827451,0.827451}%
\pgfsetstrokecolor{currentstroke}%
\pgfsetstrokeopacity{0.800000}%
\pgfsetdash{}{0pt}%
\pgfpathmoveto{\pgfqpoint{8.716039in}{3.385314in}}%
\pgfpathcurveto{\pgfqpoint{8.727089in}{3.385314in}}{\pgfqpoint{8.737688in}{3.389704in}}{\pgfqpoint{8.745502in}{3.397518in}}%
\pgfpathcurveto{\pgfqpoint{8.753315in}{3.405332in}}{\pgfqpoint{8.757706in}{3.415931in}}{\pgfqpoint{8.757706in}{3.426981in}}%
\pgfpathcurveto{\pgfqpoint{8.757706in}{3.438031in}}{\pgfqpoint{8.753315in}{3.448630in}}{\pgfqpoint{8.745502in}{3.456444in}}%
\pgfpathcurveto{\pgfqpoint{8.737688in}{3.464257in}}{\pgfqpoint{8.727089in}{3.468647in}}{\pgfqpoint{8.716039in}{3.468647in}}%
\pgfpathcurveto{\pgfqpoint{8.704989in}{3.468647in}}{\pgfqpoint{8.694390in}{3.464257in}}{\pgfqpoint{8.686576in}{3.456444in}}%
\pgfpathcurveto{\pgfqpoint{8.678763in}{3.448630in}}{\pgfqpoint{8.674372in}{3.438031in}}{\pgfqpoint{8.674372in}{3.426981in}}%
\pgfpathcurveto{\pgfqpoint{8.674372in}{3.415931in}}{\pgfqpoint{8.678763in}{3.405332in}}{\pgfqpoint{8.686576in}{3.397518in}}%
\pgfpathcurveto{\pgfqpoint{8.694390in}{3.389704in}}{\pgfqpoint{8.704989in}{3.385314in}}{\pgfqpoint{8.716039in}{3.385314in}}%
\pgfpathlineto{\pgfqpoint{8.716039in}{3.385314in}}%
\pgfpathclose%
\pgfusepath{stroke}%
\end{pgfscope}%
\begin{pgfscope}%
\pgfpathrectangle{\pgfqpoint{7.512535in}{0.437222in}}{\pgfqpoint{6.275590in}{5.159444in}}%
\pgfusepath{clip}%
\pgfsetbuttcap%
\pgfsetroundjoin%
\pgfsetlinewidth{1.003750pt}%
\definecolor{currentstroke}{rgb}{0.827451,0.827451,0.827451}%
\pgfsetstrokecolor{currentstroke}%
\pgfsetstrokeopacity{0.800000}%
\pgfsetdash{}{0pt}%
\pgfpathmoveto{\pgfqpoint{10.435824in}{5.040165in}}%
\pgfpathcurveto{\pgfqpoint{10.446874in}{5.040165in}}{\pgfqpoint{10.457473in}{5.044555in}}{\pgfqpoint{10.465287in}{5.052369in}}%
\pgfpathcurveto{\pgfqpoint{10.473101in}{5.060183in}}{\pgfqpoint{10.477491in}{5.070782in}}{\pgfqpoint{10.477491in}{5.081832in}}%
\pgfpathcurveto{\pgfqpoint{10.477491in}{5.092882in}}{\pgfqpoint{10.473101in}{5.103481in}}{\pgfqpoint{10.465287in}{5.111295in}}%
\pgfpathcurveto{\pgfqpoint{10.457473in}{5.119108in}}{\pgfqpoint{10.446874in}{5.123499in}}{\pgfqpoint{10.435824in}{5.123499in}}%
\pgfpathcurveto{\pgfqpoint{10.424774in}{5.123499in}}{\pgfqpoint{10.414175in}{5.119108in}}{\pgfqpoint{10.406361in}{5.111295in}}%
\pgfpathcurveto{\pgfqpoint{10.398548in}{5.103481in}}{\pgfqpoint{10.394157in}{5.092882in}}{\pgfqpoint{10.394157in}{5.081832in}}%
\pgfpathcurveto{\pgfqpoint{10.394157in}{5.070782in}}{\pgfqpoint{10.398548in}{5.060183in}}{\pgfqpoint{10.406361in}{5.052369in}}%
\pgfpathcurveto{\pgfqpoint{10.414175in}{5.044555in}}{\pgfqpoint{10.424774in}{5.040165in}}{\pgfqpoint{10.435824in}{5.040165in}}%
\pgfpathlineto{\pgfqpoint{10.435824in}{5.040165in}}%
\pgfpathclose%
\pgfusepath{stroke}%
\end{pgfscope}%
\begin{pgfscope}%
\pgfpathrectangle{\pgfqpoint{7.512535in}{0.437222in}}{\pgfqpoint{6.275590in}{5.159444in}}%
\pgfusepath{clip}%
\pgfsetbuttcap%
\pgfsetroundjoin%
\pgfsetlinewidth{1.003750pt}%
\definecolor{currentstroke}{rgb}{0.827451,0.827451,0.827451}%
\pgfsetstrokecolor{currentstroke}%
\pgfsetstrokeopacity{0.800000}%
\pgfsetdash{}{0pt}%
\pgfpathmoveto{\pgfqpoint{8.282445in}{1.938912in}}%
\pgfpathcurveto{\pgfqpoint{8.293495in}{1.938912in}}{\pgfqpoint{8.304095in}{1.943302in}}{\pgfqpoint{8.311908in}{1.951116in}}%
\pgfpathcurveto{\pgfqpoint{8.319722in}{1.958929in}}{\pgfqpoint{8.324112in}{1.969528in}}{\pgfqpoint{8.324112in}{1.980579in}}%
\pgfpathcurveto{\pgfqpoint{8.324112in}{1.991629in}}{\pgfqpoint{8.319722in}{2.002228in}}{\pgfqpoint{8.311908in}{2.010041in}}%
\pgfpathcurveto{\pgfqpoint{8.304095in}{2.017855in}}{\pgfqpoint{8.293495in}{2.022245in}}{\pgfqpoint{8.282445in}{2.022245in}}%
\pgfpathcurveto{\pgfqpoint{8.271395in}{2.022245in}}{\pgfqpoint{8.260796in}{2.017855in}}{\pgfqpoint{8.252983in}{2.010041in}}%
\pgfpathcurveto{\pgfqpoint{8.245169in}{2.002228in}}{\pgfqpoint{8.240779in}{1.991629in}}{\pgfqpoint{8.240779in}{1.980579in}}%
\pgfpathcurveto{\pgfqpoint{8.240779in}{1.969528in}}{\pgfqpoint{8.245169in}{1.958929in}}{\pgfqpoint{8.252983in}{1.951116in}}%
\pgfpathcurveto{\pgfqpoint{8.260796in}{1.943302in}}{\pgfqpoint{8.271395in}{1.938912in}}{\pgfqpoint{8.282445in}{1.938912in}}%
\pgfpathlineto{\pgfqpoint{8.282445in}{1.938912in}}%
\pgfpathclose%
\pgfusepath{stroke}%
\end{pgfscope}%
\begin{pgfscope}%
\pgfpathrectangle{\pgfqpoint{7.512535in}{0.437222in}}{\pgfqpoint{6.275590in}{5.159444in}}%
\pgfusepath{clip}%
\pgfsetbuttcap%
\pgfsetroundjoin%
\pgfsetlinewidth{1.003750pt}%
\definecolor{currentstroke}{rgb}{0.827451,0.827451,0.827451}%
\pgfsetstrokecolor{currentstroke}%
\pgfsetstrokeopacity{0.800000}%
\pgfsetdash{}{0pt}%
\pgfpathmoveto{\pgfqpoint{11.673162in}{5.362410in}}%
\pgfpathcurveto{\pgfqpoint{11.684212in}{5.362410in}}{\pgfqpoint{11.694811in}{5.366800in}}{\pgfqpoint{11.702624in}{5.374614in}}%
\pgfpathcurveto{\pgfqpoint{11.710438in}{5.382427in}}{\pgfqpoint{11.714828in}{5.393027in}}{\pgfqpoint{11.714828in}{5.404077in}}%
\pgfpathcurveto{\pgfqpoint{11.714828in}{5.415127in}}{\pgfqpoint{11.710438in}{5.425726in}}{\pgfqpoint{11.702624in}{5.433539in}}%
\pgfpathcurveto{\pgfqpoint{11.694811in}{5.441353in}}{\pgfqpoint{11.684212in}{5.445743in}}{\pgfqpoint{11.673162in}{5.445743in}}%
\pgfpathcurveto{\pgfqpoint{11.662111in}{5.445743in}}{\pgfqpoint{11.651512in}{5.441353in}}{\pgfqpoint{11.643699in}{5.433539in}}%
\pgfpathcurveto{\pgfqpoint{11.635885in}{5.425726in}}{\pgfqpoint{11.631495in}{5.415127in}}{\pgfqpoint{11.631495in}{5.404077in}}%
\pgfpathcurveto{\pgfqpoint{11.631495in}{5.393027in}}{\pgfqpoint{11.635885in}{5.382427in}}{\pgfqpoint{11.643699in}{5.374614in}}%
\pgfpathcurveto{\pgfqpoint{11.651512in}{5.366800in}}{\pgfqpoint{11.662111in}{5.362410in}}{\pgfqpoint{11.673162in}{5.362410in}}%
\pgfpathlineto{\pgfqpoint{11.673162in}{5.362410in}}%
\pgfpathclose%
\pgfusepath{stroke}%
\end{pgfscope}%
\begin{pgfscope}%
\pgfpathrectangle{\pgfqpoint{7.512535in}{0.437222in}}{\pgfqpoint{6.275590in}{5.159444in}}%
\pgfusepath{clip}%
\pgfsetbuttcap%
\pgfsetroundjoin%
\pgfsetlinewidth{1.003750pt}%
\definecolor{currentstroke}{rgb}{0.827451,0.827451,0.827451}%
\pgfsetstrokecolor{currentstroke}%
\pgfsetstrokeopacity{0.800000}%
\pgfsetdash{}{0pt}%
\pgfpathmoveto{\pgfqpoint{9.447017in}{3.335366in}}%
\pgfpathcurveto{\pgfqpoint{9.458067in}{3.335366in}}{\pgfqpoint{9.468666in}{3.339757in}}{\pgfqpoint{9.476479in}{3.347570in}}%
\pgfpathcurveto{\pgfqpoint{9.484293in}{3.355384in}}{\pgfqpoint{9.488683in}{3.365983in}}{\pgfqpoint{9.488683in}{3.377033in}}%
\pgfpathcurveto{\pgfqpoint{9.488683in}{3.388083in}}{\pgfqpoint{9.484293in}{3.398682in}}{\pgfqpoint{9.476479in}{3.406496in}}%
\pgfpathcurveto{\pgfqpoint{9.468666in}{3.414309in}}{\pgfqpoint{9.458067in}{3.418700in}}{\pgfqpoint{9.447017in}{3.418700in}}%
\pgfpathcurveto{\pgfqpoint{9.435967in}{3.418700in}}{\pgfqpoint{9.425367in}{3.414309in}}{\pgfqpoint{9.417554in}{3.406496in}}%
\pgfpathcurveto{\pgfqpoint{9.409740in}{3.398682in}}{\pgfqpoint{9.405350in}{3.388083in}}{\pgfqpoint{9.405350in}{3.377033in}}%
\pgfpathcurveto{\pgfqpoint{9.405350in}{3.365983in}}{\pgfqpoint{9.409740in}{3.355384in}}{\pgfqpoint{9.417554in}{3.347570in}}%
\pgfpathcurveto{\pgfqpoint{9.425367in}{3.339757in}}{\pgfqpoint{9.435967in}{3.335366in}}{\pgfqpoint{9.447017in}{3.335366in}}%
\pgfpathlineto{\pgfqpoint{9.447017in}{3.335366in}}%
\pgfpathclose%
\pgfusepath{stroke}%
\end{pgfscope}%
\begin{pgfscope}%
\pgfpathrectangle{\pgfqpoint{7.512535in}{0.437222in}}{\pgfqpoint{6.275590in}{5.159444in}}%
\pgfusepath{clip}%
\pgfsetbuttcap%
\pgfsetroundjoin%
\pgfsetlinewidth{1.003750pt}%
\definecolor{currentstroke}{rgb}{0.827451,0.827451,0.827451}%
\pgfsetstrokecolor{currentstroke}%
\pgfsetstrokeopacity{0.800000}%
\pgfsetdash{}{0pt}%
\pgfpathmoveto{\pgfqpoint{13.575740in}{5.547910in}}%
\pgfpathcurveto{\pgfqpoint{13.586790in}{5.547910in}}{\pgfqpoint{13.597389in}{5.552301in}}{\pgfqpoint{13.605202in}{5.560114in}}%
\pgfpathcurveto{\pgfqpoint{13.613016in}{5.567928in}}{\pgfqpoint{13.617406in}{5.578527in}}{\pgfqpoint{13.617406in}{5.589577in}}%
\pgfpathcurveto{\pgfqpoint{13.617406in}{5.600627in}}{\pgfqpoint{13.613016in}{5.611226in}}{\pgfqpoint{13.605202in}{5.619040in}}%
\pgfpathcurveto{\pgfqpoint{13.597389in}{5.626854in}}{\pgfqpoint{13.586790in}{5.631244in}}{\pgfqpoint{13.575740in}{5.631244in}}%
\pgfpathcurveto{\pgfqpoint{13.564690in}{5.631244in}}{\pgfqpoint{13.554090in}{5.626854in}}{\pgfqpoint{13.546277in}{5.619040in}}%
\pgfpathcurveto{\pgfqpoint{13.538463in}{5.611226in}}{\pgfqpoint{13.534073in}{5.600627in}}{\pgfqpoint{13.534073in}{5.589577in}}%
\pgfpathcurveto{\pgfqpoint{13.534073in}{5.578527in}}{\pgfqpoint{13.538463in}{5.567928in}}{\pgfqpoint{13.546277in}{5.560114in}}%
\pgfpathcurveto{\pgfqpoint{13.554090in}{5.552301in}}{\pgfqpoint{13.564690in}{5.547910in}}{\pgfqpoint{13.575740in}{5.547910in}}%
\pgfpathlineto{\pgfqpoint{13.575740in}{5.547910in}}%
\pgfpathclose%
\pgfusepath{stroke}%
\end{pgfscope}%
\begin{pgfscope}%
\pgfpathrectangle{\pgfqpoint{7.512535in}{0.437222in}}{\pgfqpoint{6.275590in}{5.159444in}}%
\pgfusepath{clip}%
\pgfsetbuttcap%
\pgfsetroundjoin%
\pgfsetlinewidth{1.003750pt}%
\definecolor{currentstroke}{rgb}{0.827451,0.827451,0.827451}%
\pgfsetstrokecolor{currentstroke}%
\pgfsetstrokeopacity{0.800000}%
\pgfsetdash{}{0pt}%
\pgfpathmoveto{\pgfqpoint{9.630920in}{3.276112in}}%
\pgfpathcurveto{\pgfqpoint{9.641970in}{3.276112in}}{\pgfqpoint{9.652569in}{3.280502in}}{\pgfqpoint{9.660383in}{3.288316in}}%
\pgfpathcurveto{\pgfqpoint{9.668196in}{3.296130in}}{\pgfqpoint{9.672587in}{3.306729in}}{\pgfqpoint{9.672587in}{3.317779in}}%
\pgfpathcurveto{\pgfqpoint{9.672587in}{3.328829in}}{\pgfqpoint{9.668196in}{3.339428in}}{\pgfqpoint{9.660383in}{3.347241in}}%
\pgfpathcurveto{\pgfqpoint{9.652569in}{3.355055in}}{\pgfqpoint{9.641970in}{3.359445in}}{\pgfqpoint{9.630920in}{3.359445in}}%
\pgfpathcurveto{\pgfqpoint{9.619870in}{3.359445in}}{\pgfqpoint{9.609271in}{3.355055in}}{\pgfqpoint{9.601457in}{3.347241in}}%
\pgfpathcurveto{\pgfqpoint{9.593644in}{3.339428in}}{\pgfqpoint{9.589253in}{3.328829in}}{\pgfqpoint{9.589253in}{3.317779in}}%
\pgfpathcurveto{\pgfqpoint{9.589253in}{3.306729in}}{\pgfqpoint{9.593644in}{3.296130in}}{\pgfqpoint{9.601457in}{3.288316in}}%
\pgfpathcurveto{\pgfqpoint{9.609271in}{3.280502in}}{\pgfqpoint{9.619870in}{3.276112in}}{\pgfqpoint{9.630920in}{3.276112in}}%
\pgfpathlineto{\pgfqpoint{9.630920in}{3.276112in}}%
\pgfpathclose%
\pgfusepath{stroke}%
\end{pgfscope}%
\begin{pgfscope}%
\pgfpathrectangle{\pgfqpoint{7.512535in}{0.437222in}}{\pgfqpoint{6.275590in}{5.159444in}}%
\pgfusepath{clip}%
\pgfsetbuttcap%
\pgfsetroundjoin%
\pgfsetlinewidth{1.003750pt}%
\definecolor{currentstroke}{rgb}{0.827451,0.827451,0.827451}%
\pgfsetstrokecolor{currentstroke}%
\pgfsetstrokeopacity{0.800000}%
\pgfsetdash{}{0pt}%
\pgfpathmoveto{\pgfqpoint{8.871031in}{2.110062in}}%
\pgfpathcurveto{\pgfqpoint{8.882081in}{2.110062in}}{\pgfqpoint{8.892680in}{2.114452in}}{\pgfqpoint{8.900493in}{2.122266in}}%
\pgfpathcurveto{\pgfqpoint{8.908307in}{2.130079in}}{\pgfqpoint{8.912697in}{2.140678in}}{\pgfqpoint{8.912697in}{2.151729in}}%
\pgfpathcurveto{\pgfqpoint{8.912697in}{2.162779in}}{\pgfqpoint{8.908307in}{2.173378in}}{\pgfqpoint{8.900493in}{2.181191in}}%
\pgfpathcurveto{\pgfqpoint{8.892680in}{2.189005in}}{\pgfqpoint{8.882081in}{2.193395in}}{\pgfqpoint{8.871031in}{2.193395in}}%
\pgfpathcurveto{\pgfqpoint{8.859980in}{2.193395in}}{\pgfqpoint{8.849381in}{2.189005in}}{\pgfqpoint{8.841568in}{2.181191in}}%
\pgfpathcurveto{\pgfqpoint{8.833754in}{2.173378in}}{\pgfqpoint{8.829364in}{2.162779in}}{\pgfqpoint{8.829364in}{2.151729in}}%
\pgfpathcurveto{\pgfqpoint{8.829364in}{2.140678in}}{\pgfqpoint{8.833754in}{2.130079in}}{\pgfqpoint{8.841568in}{2.122266in}}%
\pgfpathcurveto{\pgfqpoint{8.849381in}{2.114452in}}{\pgfqpoint{8.859980in}{2.110062in}}{\pgfqpoint{8.871031in}{2.110062in}}%
\pgfpathlineto{\pgfqpoint{8.871031in}{2.110062in}}%
\pgfpathclose%
\pgfusepath{stroke}%
\end{pgfscope}%
\begin{pgfscope}%
\pgfpathrectangle{\pgfqpoint{7.512535in}{0.437222in}}{\pgfqpoint{6.275590in}{5.159444in}}%
\pgfusepath{clip}%
\pgfsetbuttcap%
\pgfsetroundjoin%
\pgfsetlinewidth{1.003750pt}%
\definecolor{currentstroke}{rgb}{0.827451,0.827451,0.827451}%
\pgfsetstrokecolor{currentstroke}%
\pgfsetstrokeopacity{0.800000}%
\pgfsetdash{}{0pt}%
\pgfpathmoveto{\pgfqpoint{11.415124in}{5.455549in}}%
\pgfpathcurveto{\pgfqpoint{11.426174in}{5.455549in}}{\pgfqpoint{11.436773in}{5.459939in}}{\pgfqpoint{11.444587in}{5.467753in}}%
\pgfpathcurveto{\pgfqpoint{11.452401in}{5.475566in}}{\pgfqpoint{11.456791in}{5.486165in}}{\pgfqpoint{11.456791in}{5.497215in}}%
\pgfpathcurveto{\pgfqpoint{11.456791in}{5.508266in}}{\pgfqpoint{11.452401in}{5.518865in}}{\pgfqpoint{11.444587in}{5.526678in}}%
\pgfpathcurveto{\pgfqpoint{11.436773in}{5.534492in}}{\pgfqpoint{11.426174in}{5.538882in}}{\pgfqpoint{11.415124in}{5.538882in}}%
\pgfpathcurveto{\pgfqpoint{11.404074in}{5.538882in}}{\pgfqpoint{11.393475in}{5.534492in}}{\pgfqpoint{11.385661in}{5.526678in}}%
\pgfpathcurveto{\pgfqpoint{11.377848in}{5.518865in}}{\pgfqpoint{11.373458in}{5.508266in}}{\pgfqpoint{11.373458in}{5.497215in}}%
\pgfpathcurveto{\pgfqpoint{11.373458in}{5.486165in}}{\pgfqpoint{11.377848in}{5.475566in}}{\pgfqpoint{11.385661in}{5.467753in}}%
\pgfpathcurveto{\pgfqpoint{11.393475in}{5.459939in}}{\pgfqpoint{11.404074in}{5.455549in}}{\pgfqpoint{11.415124in}{5.455549in}}%
\pgfpathlineto{\pgfqpoint{11.415124in}{5.455549in}}%
\pgfpathclose%
\pgfusepath{stroke}%
\end{pgfscope}%
\begin{pgfscope}%
\pgfpathrectangle{\pgfqpoint{7.512535in}{0.437222in}}{\pgfqpoint{6.275590in}{5.159444in}}%
\pgfusepath{clip}%
\pgfsetbuttcap%
\pgfsetroundjoin%
\pgfsetlinewidth{1.003750pt}%
\definecolor{currentstroke}{rgb}{0.827451,0.827451,0.827451}%
\pgfsetstrokecolor{currentstroke}%
\pgfsetstrokeopacity{0.800000}%
\pgfsetdash{}{0pt}%
\pgfpathmoveto{\pgfqpoint{7.676506in}{1.009520in}}%
\pgfpathcurveto{\pgfqpoint{7.687556in}{1.009520in}}{\pgfqpoint{7.698155in}{1.013910in}}{\pgfqpoint{7.705969in}{1.021724in}}%
\pgfpathcurveto{\pgfqpoint{7.713783in}{1.029538in}}{\pgfqpoint{7.718173in}{1.040137in}}{\pgfqpoint{7.718173in}{1.051187in}}%
\pgfpathcurveto{\pgfqpoint{7.718173in}{1.062237in}}{\pgfqpoint{7.713783in}{1.072836in}}{\pgfqpoint{7.705969in}{1.080650in}}%
\pgfpathcurveto{\pgfqpoint{7.698155in}{1.088463in}}{\pgfqpoint{7.687556in}{1.092853in}}{\pgfqpoint{7.676506in}{1.092853in}}%
\pgfpathcurveto{\pgfqpoint{7.665456in}{1.092853in}}{\pgfqpoint{7.654857in}{1.088463in}}{\pgfqpoint{7.647043in}{1.080650in}}%
\pgfpathcurveto{\pgfqpoint{7.639230in}{1.072836in}}{\pgfqpoint{7.634840in}{1.062237in}}{\pgfqpoint{7.634840in}{1.051187in}}%
\pgfpathcurveto{\pgfqpoint{7.634840in}{1.040137in}}{\pgfqpoint{7.639230in}{1.029538in}}{\pgfqpoint{7.647043in}{1.021724in}}%
\pgfpathcurveto{\pgfqpoint{7.654857in}{1.013910in}}{\pgfqpoint{7.665456in}{1.009520in}}{\pgfqpoint{7.676506in}{1.009520in}}%
\pgfpathlineto{\pgfqpoint{7.676506in}{1.009520in}}%
\pgfpathclose%
\pgfusepath{stroke}%
\end{pgfscope}%
\begin{pgfscope}%
\pgfpathrectangle{\pgfqpoint{7.512535in}{0.437222in}}{\pgfqpoint{6.275590in}{5.159444in}}%
\pgfusepath{clip}%
\pgfsetbuttcap%
\pgfsetroundjoin%
\pgfsetlinewidth{1.003750pt}%
\definecolor{currentstroke}{rgb}{0.827451,0.827451,0.827451}%
\pgfsetstrokecolor{currentstroke}%
\pgfsetstrokeopacity{0.800000}%
\pgfsetdash{}{0pt}%
\pgfpathmoveto{\pgfqpoint{11.275914in}{5.187104in}}%
\pgfpathcurveto{\pgfqpoint{11.286964in}{5.187104in}}{\pgfqpoint{11.297563in}{5.191494in}}{\pgfqpoint{11.305377in}{5.199308in}}%
\pgfpathcurveto{\pgfqpoint{11.313190in}{5.207121in}}{\pgfqpoint{11.317581in}{5.217720in}}{\pgfqpoint{11.317581in}{5.228771in}}%
\pgfpathcurveto{\pgfqpoint{11.317581in}{5.239821in}}{\pgfqpoint{11.313190in}{5.250420in}}{\pgfqpoint{11.305377in}{5.258233in}}%
\pgfpathcurveto{\pgfqpoint{11.297563in}{5.266047in}}{\pgfqpoint{11.286964in}{5.270437in}}{\pgfqpoint{11.275914in}{5.270437in}}%
\pgfpathcurveto{\pgfqpoint{11.264864in}{5.270437in}}{\pgfqpoint{11.254265in}{5.266047in}}{\pgfqpoint{11.246451in}{5.258233in}}%
\pgfpathcurveto{\pgfqpoint{11.238637in}{5.250420in}}{\pgfqpoint{11.234247in}{5.239821in}}{\pgfqpoint{11.234247in}{5.228771in}}%
\pgfpathcurveto{\pgfqpoint{11.234247in}{5.217720in}}{\pgfqpoint{11.238637in}{5.207121in}}{\pgfqpoint{11.246451in}{5.199308in}}%
\pgfpathcurveto{\pgfqpoint{11.254265in}{5.191494in}}{\pgfqpoint{11.264864in}{5.187104in}}{\pgfqpoint{11.275914in}{5.187104in}}%
\pgfpathlineto{\pgfqpoint{11.275914in}{5.187104in}}%
\pgfpathclose%
\pgfusepath{stroke}%
\end{pgfscope}%
\begin{pgfscope}%
\pgfpathrectangle{\pgfqpoint{7.512535in}{0.437222in}}{\pgfqpoint{6.275590in}{5.159444in}}%
\pgfusepath{clip}%
\pgfsetbuttcap%
\pgfsetroundjoin%
\pgfsetlinewidth{1.003750pt}%
\definecolor{currentstroke}{rgb}{0.827451,0.827451,0.827451}%
\pgfsetstrokecolor{currentstroke}%
\pgfsetstrokeopacity{0.800000}%
\pgfsetdash{}{0pt}%
\pgfpathmoveto{\pgfqpoint{9.263397in}{3.235524in}}%
\pgfpathcurveto{\pgfqpoint{9.274447in}{3.235524in}}{\pgfqpoint{9.285046in}{3.239914in}}{\pgfqpoint{9.292860in}{3.247728in}}%
\pgfpathcurveto{\pgfqpoint{9.300674in}{3.255542in}}{\pgfqpoint{9.305064in}{3.266141in}}{\pgfqpoint{9.305064in}{3.277191in}}%
\pgfpathcurveto{\pgfqpoint{9.305064in}{3.288241in}}{\pgfqpoint{9.300674in}{3.298840in}}{\pgfqpoint{9.292860in}{3.306654in}}%
\pgfpathcurveto{\pgfqpoint{9.285046in}{3.314467in}}{\pgfqpoint{9.274447in}{3.318857in}}{\pgfqpoint{9.263397in}{3.318857in}}%
\pgfpathcurveto{\pgfqpoint{9.252347in}{3.318857in}}{\pgfqpoint{9.241748in}{3.314467in}}{\pgfqpoint{9.233934in}{3.306654in}}%
\pgfpathcurveto{\pgfqpoint{9.226121in}{3.298840in}}{\pgfqpoint{9.221730in}{3.288241in}}{\pgfqpoint{9.221730in}{3.277191in}}%
\pgfpathcurveto{\pgfqpoint{9.221730in}{3.266141in}}{\pgfqpoint{9.226121in}{3.255542in}}{\pgfqpoint{9.233934in}{3.247728in}}%
\pgfpathcurveto{\pgfqpoint{9.241748in}{3.239914in}}{\pgfqpoint{9.252347in}{3.235524in}}{\pgfqpoint{9.263397in}{3.235524in}}%
\pgfpathlineto{\pgfqpoint{9.263397in}{3.235524in}}%
\pgfpathclose%
\pgfusepath{stroke}%
\end{pgfscope}%
\begin{pgfscope}%
\pgfpathrectangle{\pgfqpoint{7.512535in}{0.437222in}}{\pgfqpoint{6.275590in}{5.159444in}}%
\pgfusepath{clip}%
\pgfsetbuttcap%
\pgfsetroundjoin%
\pgfsetlinewidth{1.003750pt}%
\definecolor{currentstroke}{rgb}{0.827451,0.827451,0.827451}%
\pgfsetstrokecolor{currentstroke}%
\pgfsetstrokeopacity{0.800000}%
\pgfsetdash{}{0pt}%
\pgfpathmoveto{\pgfqpoint{9.740588in}{3.779631in}}%
\pgfpathcurveto{\pgfqpoint{9.751638in}{3.779631in}}{\pgfqpoint{9.762237in}{3.784021in}}{\pgfqpoint{9.770050in}{3.791835in}}%
\pgfpathcurveto{\pgfqpoint{9.777864in}{3.799649in}}{\pgfqpoint{9.782254in}{3.810248in}}{\pgfqpoint{9.782254in}{3.821298in}}%
\pgfpathcurveto{\pgfqpoint{9.782254in}{3.832348in}}{\pgfqpoint{9.777864in}{3.842947in}}{\pgfqpoint{9.770050in}{3.850761in}}%
\pgfpathcurveto{\pgfqpoint{9.762237in}{3.858574in}}{\pgfqpoint{9.751638in}{3.862965in}}{\pgfqpoint{9.740588in}{3.862965in}}%
\pgfpathcurveto{\pgfqpoint{9.729538in}{3.862965in}}{\pgfqpoint{9.718939in}{3.858574in}}{\pgfqpoint{9.711125in}{3.850761in}}%
\pgfpathcurveto{\pgfqpoint{9.703311in}{3.842947in}}{\pgfqpoint{9.698921in}{3.832348in}}{\pgfqpoint{9.698921in}{3.821298in}}%
\pgfpathcurveto{\pgfqpoint{9.698921in}{3.810248in}}{\pgfqpoint{9.703311in}{3.799649in}}{\pgfqpoint{9.711125in}{3.791835in}}%
\pgfpathcurveto{\pgfqpoint{9.718939in}{3.784021in}}{\pgfqpoint{9.729538in}{3.779631in}}{\pgfqpoint{9.740588in}{3.779631in}}%
\pgfpathlineto{\pgfqpoint{9.740588in}{3.779631in}}%
\pgfpathclose%
\pgfusepath{stroke}%
\end{pgfscope}%
\begin{pgfscope}%
\pgfpathrectangle{\pgfqpoint{7.512535in}{0.437222in}}{\pgfqpoint{6.275590in}{5.159444in}}%
\pgfusepath{clip}%
\pgfsetbuttcap%
\pgfsetroundjoin%
\pgfsetlinewidth{1.003750pt}%
\definecolor{currentstroke}{rgb}{0.827451,0.827451,0.827451}%
\pgfsetstrokecolor{currentstroke}%
\pgfsetstrokeopacity{0.800000}%
\pgfsetdash{}{0pt}%
\pgfpathmoveto{\pgfqpoint{7.983939in}{0.849223in}}%
\pgfpathcurveto{\pgfqpoint{7.994989in}{0.849223in}}{\pgfqpoint{8.005588in}{0.853614in}}{\pgfqpoint{8.013401in}{0.861427in}}%
\pgfpathcurveto{\pgfqpoint{8.021215in}{0.869241in}}{\pgfqpoint{8.025605in}{0.879840in}}{\pgfqpoint{8.025605in}{0.890890in}}%
\pgfpathcurveto{\pgfqpoint{8.025605in}{0.901940in}}{\pgfqpoint{8.021215in}{0.912539in}}{\pgfqpoint{8.013401in}{0.920353in}}%
\pgfpathcurveto{\pgfqpoint{8.005588in}{0.928167in}}{\pgfqpoint{7.994989in}{0.932557in}}{\pgfqpoint{7.983939in}{0.932557in}}%
\pgfpathcurveto{\pgfqpoint{7.972888in}{0.932557in}}{\pgfqpoint{7.962289in}{0.928167in}}{\pgfqpoint{7.954476in}{0.920353in}}%
\pgfpathcurveto{\pgfqpoint{7.946662in}{0.912539in}}{\pgfqpoint{7.942272in}{0.901940in}}{\pgfqpoint{7.942272in}{0.890890in}}%
\pgfpathcurveto{\pgfqpoint{7.942272in}{0.879840in}}{\pgfqpoint{7.946662in}{0.869241in}}{\pgfqpoint{7.954476in}{0.861427in}}%
\pgfpathcurveto{\pgfqpoint{7.962289in}{0.853614in}}{\pgfqpoint{7.972888in}{0.849223in}}{\pgfqpoint{7.983939in}{0.849223in}}%
\pgfpathlineto{\pgfqpoint{7.983939in}{0.849223in}}%
\pgfpathclose%
\pgfusepath{stroke}%
\end{pgfscope}%
\begin{pgfscope}%
\pgfpathrectangle{\pgfqpoint{7.512535in}{0.437222in}}{\pgfqpoint{6.275590in}{5.159444in}}%
\pgfusepath{clip}%
\pgfsetbuttcap%
\pgfsetroundjoin%
\pgfsetlinewidth{1.003750pt}%
\definecolor{currentstroke}{rgb}{0.827451,0.827451,0.827451}%
\pgfsetstrokecolor{currentstroke}%
\pgfsetstrokeopacity{0.800000}%
\pgfsetdash{}{0pt}%
\pgfpathmoveto{\pgfqpoint{10.150452in}{4.500549in}}%
\pgfpathcurveto{\pgfqpoint{10.161502in}{4.500549in}}{\pgfqpoint{10.172101in}{4.504939in}}{\pgfqpoint{10.179914in}{4.512753in}}%
\pgfpathcurveto{\pgfqpoint{10.187728in}{4.520566in}}{\pgfqpoint{10.192118in}{4.531165in}}{\pgfqpoint{10.192118in}{4.542215in}}%
\pgfpathcurveto{\pgfqpoint{10.192118in}{4.553266in}}{\pgfqpoint{10.187728in}{4.563865in}}{\pgfqpoint{10.179914in}{4.571678in}}%
\pgfpathcurveto{\pgfqpoint{10.172101in}{4.579492in}}{\pgfqpoint{10.161502in}{4.583882in}}{\pgfqpoint{10.150452in}{4.583882in}}%
\pgfpathcurveto{\pgfqpoint{10.139401in}{4.583882in}}{\pgfqpoint{10.128802in}{4.579492in}}{\pgfqpoint{10.120989in}{4.571678in}}%
\pgfpathcurveto{\pgfqpoint{10.113175in}{4.563865in}}{\pgfqpoint{10.108785in}{4.553266in}}{\pgfqpoint{10.108785in}{4.542215in}}%
\pgfpathcurveto{\pgfqpoint{10.108785in}{4.531165in}}{\pgfqpoint{10.113175in}{4.520566in}}{\pgfqpoint{10.120989in}{4.512753in}}%
\pgfpathcurveto{\pgfqpoint{10.128802in}{4.504939in}}{\pgfqpoint{10.139401in}{4.500549in}}{\pgfqpoint{10.150452in}{4.500549in}}%
\pgfpathlineto{\pgfqpoint{10.150452in}{4.500549in}}%
\pgfpathclose%
\pgfusepath{stroke}%
\end{pgfscope}%
\begin{pgfscope}%
\pgfpathrectangle{\pgfqpoint{7.512535in}{0.437222in}}{\pgfqpoint{6.275590in}{5.159444in}}%
\pgfusepath{clip}%
\pgfsetbuttcap%
\pgfsetroundjoin%
\pgfsetlinewidth{1.003750pt}%
\definecolor{currentstroke}{rgb}{0.827451,0.827451,0.827451}%
\pgfsetstrokecolor{currentstroke}%
\pgfsetstrokeopacity{0.800000}%
\pgfsetdash{}{0pt}%
\pgfpathmoveto{\pgfqpoint{8.624162in}{1.722485in}}%
\pgfpathcurveto{\pgfqpoint{8.635212in}{1.722485in}}{\pgfqpoint{8.645811in}{1.726876in}}{\pgfqpoint{8.653624in}{1.734689in}}%
\pgfpathcurveto{\pgfqpoint{8.661438in}{1.742503in}}{\pgfqpoint{8.665828in}{1.753102in}}{\pgfqpoint{8.665828in}{1.764152in}}%
\pgfpathcurveto{\pgfqpoint{8.665828in}{1.775202in}}{\pgfqpoint{8.661438in}{1.785801in}}{\pgfqpoint{8.653624in}{1.793615in}}%
\pgfpathcurveto{\pgfqpoint{8.645811in}{1.801428in}}{\pgfqpoint{8.635212in}{1.805819in}}{\pgfqpoint{8.624162in}{1.805819in}}%
\pgfpathcurveto{\pgfqpoint{8.613112in}{1.805819in}}{\pgfqpoint{8.602513in}{1.801428in}}{\pgfqpoint{8.594699in}{1.793615in}}%
\pgfpathcurveto{\pgfqpoint{8.586885in}{1.785801in}}{\pgfqpoint{8.582495in}{1.775202in}}{\pgfqpoint{8.582495in}{1.764152in}}%
\pgfpathcurveto{\pgfqpoint{8.582495in}{1.753102in}}{\pgfqpoint{8.586885in}{1.742503in}}{\pgfqpoint{8.594699in}{1.734689in}}%
\pgfpathcurveto{\pgfqpoint{8.602513in}{1.726876in}}{\pgfqpoint{8.613112in}{1.722485in}}{\pgfqpoint{8.624162in}{1.722485in}}%
\pgfpathlineto{\pgfqpoint{8.624162in}{1.722485in}}%
\pgfpathclose%
\pgfusepath{stroke}%
\end{pgfscope}%
\begin{pgfscope}%
\pgfpathrectangle{\pgfqpoint{7.512535in}{0.437222in}}{\pgfqpoint{6.275590in}{5.159444in}}%
\pgfusepath{clip}%
\pgfsetbuttcap%
\pgfsetroundjoin%
\pgfsetlinewidth{1.003750pt}%
\definecolor{currentstroke}{rgb}{0.827451,0.827451,0.827451}%
\pgfsetstrokecolor{currentstroke}%
\pgfsetstrokeopacity{0.800000}%
\pgfsetdash{}{0pt}%
\pgfpathmoveto{\pgfqpoint{9.726466in}{3.345111in}}%
\pgfpathcurveto{\pgfqpoint{9.737516in}{3.345111in}}{\pgfqpoint{9.748115in}{3.349501in}}{\pgfqpoint{9.755929in}{3.357315in}}%
\pgfpathcurveto{\pgfqpoint{9.763743in}{3.365128in}}{\pgfqpoint{9.768133in}{3.375727in}}{\pgfqpoint{9.768133in}{3.386778in}}%
\pgfpathcurveto{\pgfqpoint{9.768133in}{3.397828in}}{\pgfqpoint{9.763743in}{3.408427in}}{\pgfqpoint{9.755929in}{3.416240in}}%
\pgfpathcurveto{\pgfqpoint{9.748115in}{3.424054in}}{\pgfqpoint{9.737516in}{3.428444in}}{\pgfqpoint{9.726466in}{3.428444in}}%
\pgfpathcurveto{\pgfqpoint{9.715416in}{3.428444in}}{\pgfqpoint{9.704817in}{3.424054in}}{\pgfqpoint{9.697004in}{3.416240in}}%
\pgfpathcurveto{\pgfqpoint{9.689190in}{3.408427in}}{\pgfqpoint{9.684800in}{3.397828in}}{\pgfqpoint{9.684800in}{3.386778in}}%
\pgfpathcurveto{\pgfqpoint{9.684800in}{3.375727in}}{\pgfqpoint{9.689190in}{3.365128in}}{\pgfqpoint{9.697004in}{3.357315in}}%
\pgfpathcurveto{\pgfqpoint{9.704817in}{3.349501in}}{\pgfqpoint{9.715416in}{3.345111in}}{\pgfqpoint{9.726466in}{3.345111in}}%
\pgfpathlineto{\pgfqpoint{9.726466in}{3.345111in}}%
\pgfpathclose%
\pgfusepath{stroke}%
\end{pgfscope}%
\begin{pgfscope}%
\pgfpathrectangle{\pgfqpoint{7.512535in}{0.437222in}}{\pgfqpoint{6.275590in}{5.159444in}}%
\pgfusepath{clip}%
\pgfsetbuttcap%
\pgfsetroundjoin%
\pgfsetlinewidth{1.003750pt}%
\definecolor{currentstroke}{rgb}{0.827451,0.827451,0.827451}%
\pgfsetstrokecolor{currentstroke}%
\pgfsetstrokeopacity{0.800000}%
\pgfsetdash{}{0pt}%
\pgfpathmoveto{\pgfqpoint{7.745847in}{1.388038in}}%
\pgfpathcurveto{\pgfqpoint{7.756897in}{1.388038in}}{\pgfqpoint{7.767496in}{1.392429in}}{\pgfqpoint{7.775310in}{1.400242in}}%
\pgfpathcurveto{\pgfqpoint{7.783123in}{1.408056in}}{\pgfqpoint{7.787513in}{1.418655in}}{\pgfqpoint{7.787513in}{1.429705in}}%
\pgfpathcurveto{\pgfqpoint{7.787513in}{1.440755in}}{\pgfqpoint{7.783123in}{1.451354in}}{\pgfqpoint{7.775310in}{1.459168in}}%
\pgfpathcurveto{\pgfqpoint{7.767496in}{1.466981in}}{\pgfqpoint{7.756897in}{1.471372in}}{\pgfqpoint{7.745847in}{1.471372in}}%
\pgfpathcurveto{\pgfqpoint{7.734797in}{1.471372in}}{\pgfqpoint{7.724198in}{1.466981in}}{\pgfqpoint{7.716384in}{1.459168in}}%
\pgfpathcurveto{\pgfqpoint{7.708570in}{1.451354in}}{\pgfqpoint{7.704180in}{1.440755in}}{\pgfqpoint{7.704180in}{1.429705in}}%
\pgfpathcurveto{\pgfqpoint{7.704180in}{1.418655in}}{\pgfqpoint{7.708570in}{1.408056in}}{\pgfqpoint{7.716384in}{1.400242in}}%
\pgfpathcurveto{\pgfqpoint{7.724198in}{1.392429in}}{\pgfqpoint{7.734797in}{1.388038in}}{\pgfqpoint{7.745847in}{1.388038in}}%
\pgfpathlineto{\pgfqpoint{7.745847in}{1.388038in}}%
\pgfpathclose%
\pgfusepath{stroke}%
\end{pgfscope}%
\begin{pgfscope}%
\pgfpathrectangle{\pgfqpoint{7.512535in}{0.437222in}}{\pgfqpoint{6.275590in}{5.159444in}}%
\pgfusepath{clip}%
\pgfsetbuttcap%
\pgfsetroundjoin%
\pgfsetlinewidth{1.003750pt}%
\definecolor{currentstroke}{rgb}{0.827451,0.827451,0.827451}%
\pgfsetstrokecolor{currentstroke}%
\pgfsetstrokeopacity{0.800000}%
\pgfsetdash{}{0pt}%
\pgfpathmoveto{\pgfqpoint{8.714151in}{3.385314in}}%
\pgfpathcurveto{\pgfqpoint{8.725201in}{3.385314in}}{\pgfqpoint{8.735800in}{3.389704in}}{\pgfqpoint{8.743614in}{3.397518in}}%
\pgfpathcurveto{\pgfqpoint{8.751428in}{3.405332in}}{\pgfqpoint{8.755818in}{3.415931in}}{\pgfqpoint{8.755818in}{3.426981in}}%
\pgfpathcurveto{\pgfqpoint{8.755818in}{3.438031in}}{\pgfqpoint{8.751428in}{3.448630in}}{\pgfqpoint{8.743614in}{3.456444in}}%
\pgfpathcurveto{\pgfqpoint{8.735800in}{3.464257in}}{\pgfqpoint{8.725201in}{3.468647in}}{\pgfqpoint{8.714151in}{3.468647in}}%
\pgfpathcurveto{\pgfqpoint{8.703101in}{3.468647in}}{\pgfqpoint{8.692502in}{3.464257in}}{\pgfqpoint{8.684689in}{3.456444in}}%
\pgfpathcurveto{\pgfqpoint{8.676875in}{3.448630in}}{\pgfqpoint{8.672485in}{3.438031in}}{\pgfqpoint{8.672485in}{3.426981in}}%
\pgfpathcurveto{\pgfqpoint{8.672485in}{3.415931in}}{\pgfqpoint{8.676875in}{3.405332in}}{\pgfqpoint{8.684689in}{3.397518in}}%
\pgfpathcurveto{\pgfqpoint{8.692502in}{3.389704in}}{\pgfqpoint{8.703101in}{3.385314in}}{\pgfqpoint{8.714151in}{3.385314in}}%
\pgfpathlineto{\pgfqpoint{8.714151in}{3.385314in}}%
\pgfpathclose%
\pgfusepath{stroke}%
\end{pgfscope}%
\begin{pgfscope}%
\pgfpathrectangle{\pgfqpoint{7.512535in}{0.437222in}}{\pgfqpoint{6.275590in}{5.159444in}}%
\pgfusepath{clip}%
\pgfsetbuttcap%
\pgfsetroundjoin%
\pgfsetlinewidth{1.003750pt}%
\definecolor{currentstroke}{rgb}{0.827451,0.827451,0.827451}%
\pgfsetstrokecolor{currentstroke}%
\pgfsetstrokeopacity{0.800000}%
\pgfsetdash{}{0pt}%
\pgfpathmoveto{\pgfqpoint{9.878128in}{3.472188in}}%
\pgfpathcurveto{\pgfqpoint{9.889178in}{3.472188in}}{\pgfqpoint{9.899777in}{3.476579in}}{\pgfqpoint{9.907590in}{3.484392in}}%
\pgfpathcurveto{\pgfqpoint{9.915404in}{3.492206in}}{\pgfqpoint{9.919794in}{3.502805in}}{\pgfqpoint{9.919794in}{3.513855in}}%
\pgfpathcurveto{\pgfqpoint{9.919794in}{3.524905in}}{\pgfqpoint{9.915404in}{3.535504in}}{\pgfqpoint{9.907590in}{3.543318in}}%
\pgfpathcurveto{\pgfqpoint{9.899777in}{3.551131in}}{\pgfqpoint{9.889178in}{3.555522in}}{\pgfqpoint{9.878128in}{3.555522in}}%
\pgfpathcurveto{\pgfqpoint{9.867077in}{3.555522in}}{\pgfqpoint{9.856478in}{3.551131in}}{\pgfqpoint{9.848665in}{3.543318in}}%
\pgfpathcurveto{\pgfqpoint{9.840851in}{3.535504in}}{\pgfqpoint{9.836461in}{3.524905in}}{\pgfqpoint{9.836461in}{3.513855in}}%
\pgfpathcurveto{\pgfqpoint{9.836461in}{3.502805in}}{\pgfqpoint{9.840851in}{3.492206in}}{\pgfqpoint{9.848665in}{3.484392in}}%
\pgfpathcurveto{\pgfqpoint{9.856478in}{3.476579in}}{\pgfqpoint{9.867077in}{3.472188in}}{\pgfqpoint{9.878128in}{3.472188in}}%
\pgfpathlineto{\pgfqpoint{9.878128in}{3.472188in}}%
\pgfpathclose%
\pgfusepath{stroke}%
\end{pgfscope}%
\begin{pgfscope}%
\pgfpathrectangle{\pgfqpoint{7.512535in}{0.437222in}}{\pgfqpoint{6.275590in}{5.159444in}}%
\pgfusepath{clip}%
\pgfsetbuttcap%
\pgfsetroundjoin%
\pgfsetlinewidth{1.003750pt}%
\definecolor{currentstroke}{rgb}{0.827451,0.827451,0.827451}%
\pgfsetstrokecolor{currentstroke}%
\pgfsetstrokeopacity{0.800000}%
\pgfsetdash{}{0pt}%
\pgfpathmoveto{\pgfqpoint{12.342968in}{5.553091in}}%
\pgfpathcurveto{\pgfqpoint{12.354019in}{5.553091in}}{\pgfqpoint{12.364618in}{5.557481in}}{\pgfqpoint{12.372431in}{5.565295in}}%
\pgfpathcurveto{\pgfqpoint{12.380245in}{5.573108in}}{\pgfqpoint{12.384635in}{5.583707in}}{\pgfqpoint{12.384635in}{5.594757in}}%
\pgfpathcurveto{\pgfqpoint{12.384635in}{5.605808in}}{\pgfqpoint{12.380245in}{5.616407in}}{\pgfqpoint{12.372431in}{5.624220in}}%
\pgfpathcurveto{\pgfqpoint{12.364618in}{5.632034in}}{\pgfqpoint{12.354019in}{5.636424in}}{\pgfqpoint{12.342968in}{5.636424in}}%
\pgfpathcurveto{\pgfqpoint{12.331918in}{5.636424in}}{\pgfqpoint{12.321319in}{5.632034in}}{\pgfqpoint{12.313506in}{5.624220in}}%
\pgfpathcurveto{\pgfqpoint{12.305692in}{5.616407in}}{\pgfqpoint{12.301302in}{5.605808in}}{\pgfqpoint{12.301302in}{5.594757in}}%
\pgfpathcurveto{\pgfqpoint{12.301302in}{5.583707in}}{\pgfqpoint{12.305692in}{5.573108in}}{\pgfqpoint{12.313506in}{5.565295in}}%
\pgfpathcurveto{\pgfqpoint{12.321319in}{5.557481in}}{\pgfqpoint{12.331918in}{5.553091in}}{\pgfqpoint{12.342968in}{5.553091in}}%
\pgfpathlineto{\pgfqpoint{12.342968in}{5.553091in}}%
\pgfpathclose%
\pgfusepath{stroke}%
\end{pgfscope}%
\begin{pgfscope}%
\pgfpathrectangle{\pgfqpoint{7.512535in}{0.437222in}}{\pgfqpoint{6.275590in}{5.159444in}}%
\pgfusepath{clip}%
\pgfsetbuttcap%
\pgfsetroundjoin%
\pgfsetlinewidth{1.003750pt}%
\definecolor{currentstroke}{rgb}{0.827451,0.827451,0.827451}%
\pgfsetstrokecolor{currentstroke}%
\pgfsetstrokeopacity{0.800000}%
\pgfsetdash{}{0pt}%
\pgfpathmoveto{\pgfqpoint{9.697299in}{3.449334in}}%
\pgfpathcurveto{\pgfqpoint{9.708349in}{3.449334in}}{\pgfqpoint{9.718948in}{3.453725in}}{\pgfqpoint{9.726762in}{3.461538in}}%
\pgfpathcurveto{\pgfqpoint{9.734575in}{3.469352in}}{\pgfqpoint{9.738965in}{3.479951in}}{\pgfqpoint{9.738965in}{3.491001in}}%
\pgfpathcurveto{\pgfqpoint{9.738965in}{3.502051in}}{\pgfqpoint{9.734575in}{3.512650in}}{\pgfqpoint{9.726762in}{3.520464in}}%
\pgfpathcurveto{\pgfqpoint{9.718948in}{3.528277in}}{\pgfqpoint{9.708349in}{3.532668in}}{\pgfqpoint{9.697299in}{3.532668in}}%
\pgfpathcurveto{\pgfqpoint{9.686249in}{3.532668in}}{\pgfqpoint{9.675650in}{3.528277in}}{\pgfqpoint{9.667836in}{3.520464in}}%
\pgfpathcurveto{\pgfqpoint{9.660022in}{3.512650in}}{\pgfqpoint{9.655632in}{3.502051in}}{\pgfqpoint{9.655632in}{3.491001in}}%
\pgfpathcurveto{\pgfqpoint{9.655632in}{3.479951in}}{\pgfqpoint{9.660022in}{3.469352in}}{\pgfqpoint{9.667836in}{3.461538in}}%
\pgfpathcurveto{\pgfqpoint{9.675650in}{3.453725in}}{\pgfqpoint{9.686249in}{3.449334in}}{\pgfqpoint{9.697299in}{3.449334in}}%
\pgfpathlineto{\pgfqpoint{9.697299in}{3.449334in}}%
\pgfpathclose%
\pgfusepath{stroke}%
\end{pgfscope}%
\begin{pgfscope}%
\pgfpathrectangle{\pgfqpoint{7.512535in}{0.437222in}}{\pgfqpoint{6.275590in}{5.159444in}}%
\pgfusepath{clip}%
\pgfsetbuttcap%
\pgfsetroundjoin%
\pgfsetlinewidth{1.003750pt}%
\definecolor{currentstroke}{rgb}{0.827451,0.827451,0.827451}%
\pgfsetstrokecolor{currentstroke}%
\pgfsetstrokeopacity{0.800000}%
\pgfsetdash{}{0pt}%
\pgfpathmoveto{\pgfqpoint{8.261016in}{1.313150in}}%
\pgfpathcurveto{\pgfqpoint{8.272066in}{1.313150in}}{\pgfqpoint{8.282665in}{1.317540in}}{\pgfqpoint{8.290479in}{1.325354in}}%
\pgfpathcurveto{\pgfqpoint{8.298293in}{1.333167in}}{\pgfqpoint{8.302683in}{1.343766in}}{\pgfqpoint{8.302683in}{1.354817in}}%
\pgfpathcurveto{\pgfqpoint{8.302683in}{1.365867in}}{\pgfqpoint{8.298293in}{1.376466in}}{\pgfqpoint{8.290479in}{1.384279in}}%
\pgfpathcurveto{\pgfqpoint{8.282665in}{1.392093in}}{\pgfqpoint{8.272066in}{1.396483in}}{\pgfqpoint{8.261016in}{1.396483in}}%
\pgfpathcurveto{\pgfqpoint{8.249966in}{1.396483in}}{\pgfqpoint{8.239367in}{1.392093in}}{\pgfqpoint{8.231554in}{1.384279in}}%
\pgfpathcurveto{\pgfqpoint{8.223740in}{1.376466in}}{\pgfqpoint{8.219350in}{1.365867in}}{\pgfqpoint{8.219350in}{1.354817in}}%
\pgfpathcurveto{\pgfqpoint{8.219350in}{1.343766in}}{\pgfqpoint{8.223740in}{1.333167in}}{\pgfqpoint{8.231554in}{1.325354in}}%
\pgfpathcurveto{\pgfqpoint{8.239367in}{1.317540in}}{\pgfqpoint{8.249966in}{1.313150in}}{\pgfqpoint{8.261016in}{1.313150in}}%
\pgfpathlineto{\pgfqpoint{8.261016in}{1.313150in}}%
\pgfpathclose%
\pgfusepath{stroke}%
\end{pgfscope}%
\begin{pgfscope}%
\pgfpathrectangle{\pgfqpoint{7.512535in}{0.437222in}}{\pgfqpoint{6.275590in}{5.159444in}}%
\pgfusepath{clip}%
\pgfsetbuttcap%
\pgfsetroundjoin%
\pgfsetlinewidth{1.003750pt}%
\definecolor{currentstroke}{rgb}{0.827451,0.827451,0.827451}%
\pgfsetstrokecolor{currentstroke}%
\pgfsetstrokeopacity{0.800000}%
\pgfsetdash{}{0pt}%
\pgfpathmoveto{\pgfqpoint{9.130854in}{1.867691in}}%
\pgfpathcurveto{\pgfqpoint{9.141904in}{1.867691in}}{\pgfqpoint{9.152503in}{1.872081in}}{\pgfqpoint{9.160317in}{1.879895in}}%
\pgfpathcurveto{\pgfqpoint{9.168130in}{1.887709in}}{\pgfqpoint{9.172521in}{1.898308in}}{\pgfqpoint{9.172521in}{1.909358in}}%
\pgfpathcurveto{\pgfqpoint{9.172521in}{1.920408in}}{\pgfqpoint{9.168130in}{1.931007in}}{\pgfqpoint{9.160317in}{1.938821in}}%
\pgfpathcurveto{\pgfqpoint{9.152503in}{1.946634in}}{\pgfqpoint{9.141904in}{1.951024in}}{\pgfqpoint{9.130854in}{1.951024in}}%
\pgfpathcurveto{\pgfqpoint{9.119804in}{1.951024in}}{\pgfqpoint{9.109205in}{1.946634in}}{\pgfqpoint{9.101391in}{1.938821in}}%
\pgfpathcurveto{\pgfqpoint{9.093577in}{1.931007in}}{\pgfqpoint{9.089187in}{1.920408in}}{\pgfqpoint{9.089187in}{1.909358in}}%
\pgfpathcurveto{\pgfqpoint{9.089187in}{1.898308in}}{\pgfqpoint{9.093577in}{1.887709in}}{\pgfqpoint{9.101391in}{1.879895in}}%
\pgfpathcurveto{\pgfqpoint{9.109205in}{1.872081in}}{\pgfqpoint{9.119804in}{1.867691in}}{\pgfqpoint{9.130854in}{1.867691in}}%
\pgfpathlineto{\pgfqpoint{9.130854in}{1.867691in}}%
\pgfpathclose%
\pgfusepath{stroke}%
\end{pgfscope}%
\begin{pgfscope}%
\pgfpathrectangle{\pgfqpoint{7.512535in}{0.437222in}}{\pgfqpoint{6.275590in}{5.159444in}}%
\pgfusepath{clip}%
\pgfsetbuttcap%
\pgfsetroundjoin%
\pgfsetlinewidth{1.003750pt}%
\definecolor{currentstroke}{rgb}{0.827451,0.827451,0.827451}%
\pgfsetstrokecolor{currentstroke}%
\pgfsetstrokeopacity{0.800000}%
\pgfsetdash{}{0pt}%
\pgfpathmoveto{\pgfqpoint{7.711198in}{0.454672in}}%
\pgfpathcurveto{\pgfqpoint{7.722248in}{0.454672in}}{\pgfqpoint{7.732847in}{0.459062in}}{\pgfqpoint{7.740661in}{0.466876in}}%
\pgfpathcurveto{\pgfqpoint{7.748474in}{0.474690in}}{\pgfqpoint{7.752865in}{0.485289in}}{\pgfqpoint{7.752865in}{0.496339in}}%
\pgfpathcurveto{\pgfqpoint{7.752865in}{0.507389in}}{\pgfqpoint{7.748474in}{0.517988in}}{\pgfqpoint{7.740661in}{0.525801in}}%
\pgfpathcurveto{\pgfqpoint{7.732847in}{0.533615in}}{\pgfqpoint{7.722248in}{0.538005in}}{\pgfqpoint{7.711198in}{0.538005in}}%
\pgfpathcurveto{\pgfqpoint{7.700148in}{0.538005in}}{\pgfqpoint{7.689549in}{0.533615in}}{\pgfqpoint{7.681735in}{0.525801in}}%
\pgfpathcurveto{\pgfqpoint{7.673921in}{0.517988in}}{\pgfqpoint{7.669531in}{0.507389in}}{\pgfqpoint{7.669531in}{0.496339in}}%
\pgfpathcurveto{\pgfqpoint{7.669531in}{0.485289in}}{\pgfqpoint{7.673921in}{0.474690in}}{\pgfqpoint{7.681735in}{0.466876in}}%
\pgfpathcurveto{\pgfqpoint{7.689549in}{0.459062in}}{\pgfqpoint{7.700148in}{0.454672in}}{\pgfqpoint{7.711198in}{0.454672in}}%
\pgfpathlineto{\pgfqpoint{7.711198in}{0.454672in}}%
\pgfpathclose%
\pgfusepath{stroke}%
\end{pgfscope}%
\begin{pgfscope}%
\pgfpathrectangle{\pgfqpoint{7.512535in}{0.437222in}}{\pgfqpoint{6.275590in}{5.159444in}}%
\pgfusepath{clip}%
\pgfsetbuttcap%
\pgfsetroundjoin%
\pgfsetlinewidth{1.003750pt}%
\definecolor{currentstroke}{rgb}{0.827451,0.827451,0.827451}%
\pgfsetstrokecolor{currentstroke}%
\pgfsetstrokeopacity{0.800000}%
\pgfsetdash{}{0pt}%
\pgfpathmoveto{\pgfqpoint{7.679894in}{0.493855in}}%
\pgfpathcurveto{\pgfqpoint{7.690944in}{0.493855in}}{\pgfqpoint{7.701543in}{0.498245in}}{\pgfqpoint{7.709357in}{0.506059in}}%
\pgfpathcurveto{\pgfqpoint{7.717170in}{0.513872in}}{\pgfqpoint{7.721560in}{0.524471in}}{\pgfqpoint{7.721560in}{0.535522in}}%
\pgfpathcurveto{\pgfqpoint{7.721560in}{0.546572in}}{\pgfqpoint{7.717170in}{0.557171in}}{\pgfqpoint{7.709357in}{0.564984in}}%
\pgfpathcurveto{\pgfqpoint{7.701543in}{0.572798in}}{\pgfqpoint{7.690944in}{0.577188in}}{\pgfqpoint{7.679894in}{0.577188in}}%
\pgfpathcurveto{\pgfqpoint{7.668844in}{0.577188in}}{\pgfqpoint{7.658245in}{0.572798in}}{\pgfqpoint{7.650431in}{0.564984in}}%
\pgfpathcurveto{\pgfqpoint{7.642617in}{0.557171in}}{\pgfqpoint{7.638227in}{0.546572in}}{\pgfqpoint{7.638227in}{0.535522in}}%
\pgfpathcurveto{\pgfqpoint{7.638227in}{0.524471in}}{\pgfqpoint{7.642617in}{0.513872in}}{\pgfqpoint{7.650431in}{0.506059in}}%
\pgfpathcurveto{\pgfqpoint{7.658245in}{0.498245in}}{\pgfqpoint{7.668844in}{0.493855in}}{\pgfqpoint{7.679894in}{0.493855in}}%
\pgfpathlineto{\pgfqpoint{7.679894in}{0.493855in}}%
\pgfpathclose%
\pgfusepath{stroke}%
\end{pgfscope}%
\begin{pgfscope}%
\pgfpathrectangle{\pgfqpoint{7.512535in}{0.437222in}}{\pgfqpoint{6.275590in}{5.159444in}}%
\pgfusepath{clip}%
\pgfsetbuttcap%
\pgfsetroundjoin%
\pgfsetlinewidth{1.003750pt}%
\definecolor{currentstroke}{rgb}{0.827451,0.827451,0.827451}%
\pgfsetstrokecolor{currentstroke}%
\pgfsetstrokeopacity{0.800000}%
\pgfsetdash{}{0pt}%
\pgfpathmoveto{\pgfqpoint{13.731324in}{5.553845in}}%
\pgfpathcurveto{\pgfqpoint{13.742374in}{5.553845in}}{\pgfqpoint{13.752973in}{5.558236in}}{\pgfqpoint{13.760787in}{5.566049in}}%
\pgfpathcurveto{\pgfqpoint{13.768601in}{5.573863in}}{\pgfqpoint{13.772991in}{5.584462in}}{\pgfqpoint{13.772991in}{5.595512in}}%
\pgfpathcurveto{\pgfqpoint{13.772991in}{5.606562in}}{\pgfqpoint{13.768601in}{5.617161in}}{\pgfqpoint{13.760787in}{5.624975in}}%
\pgfpathcurveto{\pgfqpoint{13.752973in}{5.632788in}}{\pgfqpoint{13.742374in}{5.637179in}}{\pgfqpoint{13.731324in}{5.637179in}}%
\pgfpathcurveto{\pgfqpoint{13.720274in}{5.637179in}}{\pgfqpoint{13.709675in}{5.632788in}}{\pgfqpoint{13.701861in}{5.624975in}}%
\pgfpathcurveto{\pgfqpoint{13.694048in}{5.617161in}}{\pgfqpoint{13.689657in}{5.606562in}}{\pgfqpoint{13.689657in}{5.595512in}}%
\pgfpathcurveto{\pgfqpoint{13.689657in}{5.584462in}}{\pgfqpoint{13.694048in}{5.573863in}}{\pgfqpoint{13.701861in}{5.566049in}}%
\pgfpathcurveto{\pgfqpoint{13.709675in}{5.558236in}}{\pgfqpoint{13.720274in}{5.553845in}}{\pgfqpoint{13.731324in}{5.553845in}}%
\pgfpathlineto{\pgfqpoint{13.731324in}{5.553845in}}%
\pgfpathclose%
\pgfusepath{stroke}%
\end{pgfscope}%
\begin{pgfscope}%
\pgfpathrectangle{\pgfqpoint{7.512535in}{0.437222in}}{\pgfqpoint{6.275590in}{5.159444in}}%
\pgfusepath{clip}%
\pgfsetbuttcap%
\pgfsetroundjoin%
\pgfsetlinewidth{1.003750pt}%
\definecolor{currentstroke}{rgb}{0.827451,0.827451,0.827451}%
\pgfsetstrokecolor{currentstroke}%
\pgfsetstrokeopacity{0.800000}%
\pgfsetdash{}{0pt}%
\pgfpathmoveto{\pgfqpoint{7.546069in}{0.440656in}}%
\pgfpathcurveto{\pgfqpoint{7.557119in}{0.440656in}}{\pgfqpoint{7.567718in}{0.445046in}}{\pgfqpoint{7.575532in}{0.452860in}}%
\pgfpathcurveto{\pgfqpoint{7.583345in}{0.460673in}}{\pgfqpoint{7.587736in}{0.471272in}}{\pgfqpoint{7.587736in}{0.482323in}}%
\pgfpathcurveto{\pgfqpoint{7.587736in}{0.493373in}}{\pgfqpoint{7.583345in}{0.503972in}}{\pgfqpoint{7.575532in}{0.511785in}}%
\pgfpathcurveto{\pgfqpoint{7.567718in}{0.519599in}}{\pgfqpoint{7.557119in}{0.523989in}}{\pgfqpoint{7.546069in}{0.523989in}}%
\pgfpathcurveto{\pgfqpoint{7.535019in}{0.523989in}}{\pgfqpoint{7.524420in}{0.519599in}}{\pgfqpoint{7.516606in}{0.511785in}}%
\pgfpathcurveto{\pgfqpoint{7.508793in}{0.503972in}}{\pgfqpoint{7.504402in}{0.493373in}}{\pgfqpoint{7.504402in}{0.482323in}}%
\pgfpathcurveto{\pgfqpoint{7.504402in}{0.471272in}}{\pgfqpoint{7.508793in}{0.460673in}}{\pgfqpoint{7.516606in}{0.452860in}}%
\pgfpathcurveto{\pgfqpoint{7.524420in}{0.445046in}}{\pgfqpoint{7.535019in}{0.440656in}}{\pgfqpoint{7.546069in}{0.440656in}}%
\pgfpathlineto{\pgfqpoint{7.546069in}{0.440656in}}%
\pgfpathclose%
\pgfusepath{stroke}%
\end{pgfscope}%
\begin{pgfscope}%
\pgfpathrectangle{\pgfqpoint{7.512535in}{0.437222in}}{\pgfqpoint{6.275590in}{5.159444in}}%
\pgfusepath{clip}%
\pgfsetbuttcap%
\pgfsetroundjoin%
\pgfsetlinewidth{1.003750pt}%
\definecolor{currentstroke}{rgb}{0.827451,0.827451,0.827451}%
\pgfsetstrokecolor{currentstroke}%
\pgfsetstrokeopacity{0.800000}%
\pgfsetdash{}{0pt}%
\pgfpathmoveto{\pgfqpoint{8.724795in}{1.722485in}}%
\pgfpathcurveto{\pgfqpoint{8.735845in}{1.722485in}}{\pgfqpoint{8.746444in}{1.726876in}}{\pgfqpoint{8.754258in}{1.734689in}}%
\pgfpathcurveto{\pgfqpoint{8.762071in}{1.742503in}}{\pgfqpoint{8.766461in}{1.753102in}}{\pgfqpoint{8.766461in}{1.764152in}}%
\pgfpathcurveto{\pgfqpoint{8.766461in}{1.775202in}}{\pgfqpoint{8.762071in}{1.785801in}}{\pgfqpoint{8.754258in}{1.793615in}}%
\pgfpathcurveto{\pgfqpoint{8.746444in}{1.801428in}}{\pgfqpoint{8.735845in}{1.805819in}}{\pgfqpoint{8.724795in}{1.805819in}}%
\pgfpathcurveto{\pgfqpoint{8.713745in}{1.805819in}}{\pgfqpoint{8.703146in}{1.801428in}}{\pgfqpoint{8.695332in}{1.793615in}}%
\pgfpathcurveto{\pgfqpoint{8.687518in}{1.785801in}}{\pgfqpoint{8.683128in}{1.775202in}}{\pgfqpoint{8.683128in}{1.764152in}}%
\pgfpathcurveto{\pgfqpoint{8.683128in}{1.753102in}}{\pgfqpoint{8.687518in}{1.742503in}}{\pgfqpoint{8.695332in}{1.734689in}}%
\pgfpathcurveto{\pgfqpoint{8.703146in}{1.726876in}}{\pgfqpoint{8.713745in}{1.722485in}}{\pgfqpoint{8.724795in}{1.722485in}}%
\pgfpathlineto{\pgfqpoint{8.724795in}{1.722485in}}%
\pgfpathclose%
\pgfusepath{stroke}%
\end{pgfscope}%
\begin{pgfscope}%
\pgfpathrectangle{\pgfqpoint{7.512535in}{0.437222in}}{\pgfqpoint{6.275590in}{5.159444in}}%
\pgfusepath{clip}%
\pgfsetbuttcap%
\pgfsetroundjoin%
\pgfsetlinewidth{1.003750pt}%
\definecolor{currentstroke}{rgb}{0.827451,0.827451,0.827451}%
\pgfsetstrokecolor{currentstroke}%
\pgfsetstrokeopacity{0.800000}%
\pgfsetdash{}{0pt}%
\pgfpathmoveto{\pgfqpoint{8.629443in}{2.248588in}}%
\pgfpathcurveto{\pgfqpoint{8.640493in}{2.248588in}}{\pgfqpoint{8.651092in}{2.252978in}}{\pgfqpoint{8.658906in}{2.260792in}}%
\pgfpathcurveto{\pgfqpoint{8.666719in}{2.268605in}}{\pgfqpoint{8.671110in}{2.279204in}}{\pgfqpoint{8.671110in}{2.290254in}}%
\pgfpathcurveto{\pgfqpoint{8.671110in}{2.301305in}}{\pgfqpoint{8.666719in}{2.311904in}}{\pgfqpoint{8.658906in}{2.319717in}}%
\pgfpathcurveto{\pgfqpoint{8.651092in}{2.327531in}}{\pgfqpoint{8.640493in}{2.331921in}}{\pgfqpoint{8.629443in}{2.331921in}}%
\pgfpathcurveto{\pgfqpoint{8.618393in}{2.331921in}}{\pgfqpoint{8.607794in}{2.327531in}}{\pgfqpoint{8.599980in}{2.319717in}}%
\pgfpathcurveto{\pgfqpoint{8.592166in}{2.311904in}}{\pgfqpoint{8.587776in}{2.301305in}}{\pgfqpoint{8.587776in}{2.290254in}}%
\pgfpathcurveto{\pgfqpoint{8.587776in}{2.279204in}}{\pgfqpoint{8.592166in}{2.268605in}}{\pgfqpoint{8.599980in}{2.260792in}}%
\pgfpathcurveto{\pgfqpoint{8.607794in}{2.252978in}}{\pgfqpoint{8.618393in}{2.248588in}}{\pgfqpoint{8.629443in}{2.248588in}}%
\pgfpathlineto{\pgfqpoint{8.629443in}{2.248588in}}%
\pgfpathclose%
\pgfusepath{stroke}%
\end{pgfscope}%
\begin{pgfscope}%
\pgfpathrectangle{\pgfqpoint{7.512535in}{0.437222in}}{\pgfqpoint{6.275590in}{5.159444in}}%
\pgfusepath{clip}%
\pgfsetbuttcap%
\pgfsetroundjoin%
\pgfsetlinewidth{1.003750pt}%
\definecolor{currentstroke}{rgb}{0.827451,0.827451,0.827451}%
\pgfsetstrokecolor{currentstroke}%
\pgfsetstrokeopacity{0.800000}%
\pgfsetdash{}{0pt}%
\pgfpathmoveto{\pgfqpoint{11.275914in}{5.187104in}}%
\pgfpathcurveto{\pgfqpoint{11.286964in}{5.187104in}}{\pgfqpoint{11.297563in}{5.191494in}}{\pgfqpoint{11.305377in}{5.199308in}}%
\pgfpathcurveto{\pgfqpoint{11.313190in}{5.207121in}}{\pgfqpoint{11.317581in}{5.217720in}}{\pgfqpoint{11.317581in}{5.228771in}}%
\pgfpathcurveto{\pgfqpoint{11.317581in}{5.239821in}}{\pgfqpoint{11.313190in}{5.250420in}}{\pgfqpoint{11.305377in}{5.258233in}}%
\pgfpathcurveto{\pgfqpoint{11.297563in}{5.266047in}}{\pgfqpoint{11.286964in}{5.270437in}}{\pgfqpoint{11.275914in}{5.270437in}}%
\pgfpathcurveto{\pgfqpoint{11.264864in}{5.270437in}}{\pgfqpoint{11.254265in}{5.266047in}}{\pgfqpoint{11.246451in}{5.258233in}}%
\pgfpathcurveto{\pgfqpoint{11.238637in}{5.250420in}}{\pgfqpoint{11.234247in}{5.239821in}}{\pgfqpoint{11.234247in}{5.228771in}}%
\pgfpathcurveto{\pgfqpoint{11.234247in}{5.217720in}}{\pgfqpoint{11.238637in}{5.207121in}}{\pgfqpoint{11.246451in}{5.199308in}}%
\pgfpathcurveto{\pgfqpoint{11.254265in}{5.191494in}}{\pgfqpoint{11.264864in}{5.187104in}}{\pgfqpoint{11.275914in}{5.187104in}}%
\pgfpathlineto{\pgfqpoint{11.275914in}{5.187104in}}%
\pgfpathclose%
\pgfusepath{stroke}%
\end{pgfscope}%
\begin{pgfscope}%
\pgfpathrectangle{\pgfqpoint{7.512535in}{0.437222in}}{\pgfqpoint{6.275590in}{5.159444in}}%
\pgfusepath{clip}%
\pgfsetbuttcap%
\pgfsetroundjoin%
\pgfsetlinewidth{1.003750pt}%
\definecolor{currentstroke}{rgb}{0.827451,0.827451,0.827451}%
\pgfsetstrokecolor{currentstroke}%
\pgfsetstrokeopacity{0.800000}%
\pgfsetdash{}{0pt}%
\pgfpathmoveto{\pgfqpoint{8.282445in}{1.938912in}}%
\pgfpathcurveto{\pgfqpoint{8.293495in}{1.938912in}}{\pgfqpoint{8.304095in}{1.943302in}}{\pgfqpoint{8.311908in}{1.951116in}}%
\pgfpathcurveto{\pgfqpoint{8.319722in}{1.958929in}}{\pgfqpoint{8.324112in}{1.969528in}}{\pgfqpoint{8.324112in}{1.980579in}}%
\pgfpathcurveto{\pgfqpoint{8.324112in}{1.991629in}}{\pgfqpoint{8.319722in}{2.002228in}}{\pgfqpoint{8.311908in}{2.010041in}}%
\pgfpathcurveto{\pgfqpoint{8.304095in}{2.017855in}}{\pgfqpoint{8.293495in}{2.022245in}}{\pgfqpoint{8.282445in}{2.022245in}}%
\pgfpathcurveto{\pgfqpoint{8.271395in}{2.022245in}}{\pgfqpoint{8.260796in}{2.017855in}}{\pgfqpoint{8.252983in}{2.010041in}}%
\pgfpathcurveto{\pgfqpoint{8.245169in}{2.002228in}}{\pgfqpoint{8.240779in}{1.991629in}}{\pgfqpoint{8.240779in}{1.980579in}}%
\pgfpathcurveto{\pgfqpoint{8.240779in}{1.969528in}}{\pgfqpoint{8.245169in}{1.958929in}}{\pgfqpoint{8.252983in}{1.951116in}}%
\pgfpathcurveto{\pgfqpoint{8.260796in}{1.943302in}}{\pgfqpoint{8.271395in}{1.938912in}}{\pgfqpoint{8.282445in}{1.938912in}}%
\pgfpathlineto{\pgfqpoint{8.282445in}{1.938912in}}%
\pgfpathclose%
\pgfusepath{stroke}%
\end{pgfscope}%
\begin{pgfscope}%
\pgfpathrectangle{\pgfqpoint{7.512535in}{0.437222in}}{\pgfqpoint{6.275590in}{5.159444in}}%
\pgfusepath{clip}%
\pgfsetbuttcap%
\pgfsetroundjoin%
\pgfsetlinewidth{1.003750pt}%
\definecolor{currentstroke}{rgb}{0.827451,0.827451,0.827451}%
\pgfsetstrokecolor{currentstroke}%
\pgfsetstrokeopacity{0.800000}%
\pgfsetdash{}{0pt}%
\pgfpathmoveto{\pgfqpoint{8.624162in}{1.722485in}}%
\pgfpathcurveto{\pgfqpoint{8.635212in}{1.722485in}}{\pgfqpoint{8.645811in}{1.726876in}}{\pgfqpoint{8.653624in}{1.734689in}}%
\pgfpathcurveto{\pgfqpoint{8.661438in}{1.742503in}}{\pgfqpoint{8.665828in}{1.753102in}}{\pgfqpoint{8.665828in}{1.764152in}}%
\pgfpathcurveto{\pgfqpoint{8.665828in}{1.775202in}}{\pgfqpoint{8.661438in}{1.785801in}}{\pgfqpoint{8.653624in}{1.793615in}}%
\pgfpathcurveto{\pgfqpoint{8.645811in}{1.801428in}}{\pgfqpoint{8.635212in}{1.805819in}}{\pgfqpoint{8.624162in}{1.805819in}}%
\pgfpathcurveto{\pgfqpoint{8.613112in}{1.805819in}}{\pgfqpoint{8.602513in}{1.801428in}}{\pgfqpoint{8.594699in}{1.793615in}}%
\pgfpathcurveto{\pgfqpoint{8.586885in}{1.785801in}}{\pgfqpoint{8.582495in}{1.775202in}}{\pgfqpoint{8.582495in}{1.764152in}}%
\pgfpathcurveto{\pgfqpoint{8.582495in}{1.753102in}}{\pgfqpoint{8.586885in}{1.742503in}}{\pgfqpoint{8.594699in}{1.734689in}}%
\pgfpathcurveto{\pgfqpoint{8.602513in}{1.726876in}}{\pgfqpoint{8.613112in}{1.722485in}}{\pgfqpoint{8.624162in}{1.722485in}}%
\pgfpathlineto{\pgfqpoint{8.624162in}{1.722485in}}%
\pgfpathclose%
\pgfusepath{stroke}%
\end{pgfscope}%
\begin{pgfscope}%
\pgfpathrectangle{\pgfqpoint{7.512535in}{0.437222in}}{\pgfqpoint{6.275590in}{5.159444in}}%
\pgfusepath{clip}%
\pgfsetbuttcap%
\pgfsetroundjoin%
\pgfsetlinewidth{1.003750pt}%
\definecolor{currentstroke}{rgb}{0.827451,0.827451,0.827451}%
\pgfsetstrokecolor{currentstroke}%
\pgfsetstrokeopacity{0.800000}%
\pgfsetdash{}{0pt}%
\pgfpathmoveto{\pgfqpoint{12.955580in}{5.537440in}}%
\pgfpathcurveto{\pgfqpoint{12.966630in}{5.537440in}}{\pgfqpoint{12.977229in}{5.541830in}}{\pgfqpoint{12.985043in}{5.549643in}}%
\pgfpathcurveto{\pgfqpoint{12.992856in}{5.557457in}}{\pgfqpoint{12.997246in}{5.568056in}}{\pgfqpoint{12.997246in}{5.579106in}}%
\pgfpathcurveto{\pgfqpoint{12.997246in}{5.590156in}}{\pgfqpoint{12.992856in}{5.600755in}}{\pgfqpoint{12.985043in}{5.608569in}}%
\pgfpathcurveto{\pgfqpoint{12.977229in}{5.616383in}}{\pgfqpoint{12.966630in}{5.620773in}}{\pgfqpoint{12.955580in}{5.620773in}}%
\pgfpathcurveto{\pgfqpoint{12.944530in}{5.620773in}}{\pgfqpoint{12.933931in}{5.616383in}}{\pgfqpoint{12.926117in}{5.608569in}}%
\pgfpathcurveto{\pgfqpoint{12.918303in}{5.600755in}}{\pgfqpoint{12.913913in}{5.590156in}}{\pgfqpoint{12.913913in}{5.579106in}}%
\pgfpathcurveto{\pgfqpoint{12.913913in}{5.568056in}}{\pgfqpoint{12.918303in}{5.557457in}}{\pgfqpoint{12.926117in}{5.549643in}}%
\pgfpathcurveto{\pgfqpoint{12.933931in}{5.541830in}}{\pgfqpoint{12.944530in}{5.537440in}}{\pgfqpoint{12.955580in}{5.537440in}}%
\pgfpathlineto{\pgfqpoint{12.955580in}{5.537440in}}%
\pgfpathclose%
\pgfusepath{stroke}%
\end{pgfscope}%
\begin{pgfscope}%
\pgfpathrectangle{\pgfqpoint{7.512535in}{0.437222in}}{\pgfqpoint{6.275590in}{5.159444in}}%
\pgfusepath{clip}%
\pgfsetbuttcap%
\pgfsetroundjoin%
\pgfsetlinewidth{1.003750pt}%
\definecolor{currentstroke}{rgb}{0.827451,0.827451,0.827451}%
\pgfsetstrokecolor{currentstroke}%
\pgfsetstrokeopacity{0.800000}%
\pgfsetdash{}{0pt}%
\pgfpathmoveto{\pgfqpoint{8.290348in}{1.011495in}}%
\pgfpathcurveto{\pgfqpoint{8.301398in}{1.011495in}}{\pgfqpoint{8.311997in}{1.015885in}}{\pgfqpoint{8.319811in}{1.023699in}}%
\pgfpathcurveto{\pgfqpoint{8.327625in}{1.031512in}}{\pgfqpoint{8.332015in}{1.042111in}}{\pgfqpoint{8.332015in}{1.053161in}}%
\pgfpathcurveto{\pgfqpoint{8.332015in}{1.064211in}}{\pgfqpoint{8.327625in}{1.074810in}}{\pgfqpoint{8.319811in}{1.082624in}}%
\pgfpathcurveto{\pgfqpoint{8.311997in}{1.090438in}}{\pgfqpoint{8.301398in}{1.094828in}}{\pgfqpoint{8.290348in}{1.094828in}}%
\pgfpathcurveto{\pgfqpoint{8.279298in}{1.094828in}}{\pgfqpoint{8.268699in}{1.090438in}}{\pgfqpoint{8.260886in}{1.082624in}}%
\pgfpathcurveto{\pgfqpoint{8.253072in}{1.074810in}}{\pgfqpoint{8.248682in}{1.064211in}}{\pgfqpoint{8.248682in}{1.053161in}}%
\pgfpathcurveto{\pgfqpoint{8.248682in}{1.042111in}}{\pgfqpoint{8.253072in}{1.031512in}}{\pgfqpoint{8.260886in}{1.023699in}}%
\pgfpathcurveto{\pgfqpoint{8.268699in}{1.015885in}}{\pgfqpoint{8.279298in}{1.011495in}}{\pgfqpoint{8.290348in}{1.011495in}}%
\pgfpathlineto{\pgfqpoint{8.290348in}{1.011495in}}%
\pgfpathclose%
\pgfusepath{stroke}%
\end{pgfscope}%
\begin{pgfscope}%
\pgfpathrectangle{\pgfqpoint{7.512535in}{0.437222in}}{\pgfqpoint{6.275590in}{5.159444in}}%
\pgfusepath{clip}%
\pgfsetbuttcap%
\pgfsetroundjoin%
\pgfsetlinewidth{1.003750pt}%
\definecolor{currentstroke}{rgb}{0.827451,0.827451,0.827451}%
\pgfsetstrokecolor{currentstroke}%
\pgfsetstrokeopacity{0.800000}%
\pgfsetdash{}{0pt}%
\pgfpathmoveto{\pgfqpoint{10.994755in}{5.060707in}}%
\pgfpathcurveto{\pgfqpoint{11.005805in}{5.060707in}}{\pgfqpoint{11.016404in}{5.065097in}}{\pgfqpoint{11.024218in}{5.072911in}}%
\pgfpathcurveto{\pgfqpoint{11.032031in}{5.080725in}}{\pgfqpoint{11.036422in}{5.091324in}}{\pgfqpoint{11.036422in}{5.102374in}}%
\pgfpathcurveto{\pgfqpoint{11.036422in}{5.113424in}}{\pgfqpoint{11.032031in}{5.124023in}}{\pgfqpoint{11.024218in}{5.131837in}}%
\pgfpathcurveto{\pgfqpoint{11.016404in}{5.139650in}}{\pgfqpoint{11.005805in}{5.144040in}}{\pgfqpoint{10.994755in}{5.144040in}}%
\pgfpathcurveto{\pgfqpoint{10.983705in}{5.144040in}}{\pgfqpoint{10.973106in}{5.139650in}}{\pgfqpoint{10.965292in}{5.131837in}}%
\pgfpathcurveto{\pgfqpoint{10.957479in}{5.124023in}}{\pgfqpoint{10.953088in}{5.113424in}}{\pgfqpoint{10.953088in}{5.102374in}}%
\pgfpathcurveto{\pgfqpoint{10.953088in}{5.091324in}}{\pgfqpoint{10.957479in}{5.080725in}}{\pgfqpoint{10.965292in}{5.072911in}}%
\pgfpathcurveto{\pgfqpoint{10.973106in}{5.065097in}}{\pgfqpoint{10.983705in}{5.060707in}}{\pgfqpoint{10.994755in}{5.060707in}}%
\pgfpathlineto{\pgfqpoint{10.994755in}{5.060707in}}%
\pgfpathclose%
\pgfusepath{stroke}%
\end{pgfscope}%
\begin{pgfscope}%
\pgfpathrectangle{\pgfqpoint{7.512535in}{0.437222in}}{\pgfqpoint{6.275590in}{5.159444in}}%
\pgfusepath{clip}%
\pgfsetbuttcap%
\pgfsetroundjoin%
\pgfsetlinewidth{1.003750pt}%
\definecolor{currentstroke}{rgb}{0.827451,0.827451,0.827451}%
\pgfsetstrokecolor{currentstroke}%
\pgfsetstrokeopacity{0.800000}%
\pgfsetdash{}{0pt}%
\pgfpathmoveto{\pgfqpoint{11.710072in}{5.362410in}}%
\pgfpathcurveto{\pgfqpoint{11.721123in}{5.362410in}}{\pgfqpoint{11.731722in}{5.366800in}}{\pgfqpoint{11.739535in}{5.374614in}}%
\pgfpathcurveto{\pgfqpoint{11.747349in}{5.382427in}}{\pgfqpoint{11.751739in}{5.393027in}}{\pgfqpoint{11.751739in}{5.404077in}}%
\pgfpathcurveto{\pgfqpoint{11.751739in}{5.415127in}}{\pgfqpoint{11.747349in}{5.425726in}}{\pgfqpoint{11.739535in}{5.433539in}}%
\pgfpathcurveto{\pgfqpoint{11.731722in}{5.441353in}}{\pgfqpoint{11.721123in}{5.445743in}}{\pgfqpoint{11.710072in}{5.445743in}}%
\pgfpathcurveto{\pgfqpoint{11.699022in}{5.445743in}}{\pgfqpoint{11.688423in}{5.441353in}}{\pgfqpoint{11.680610in}{5.433539in}}%
\pgfpathcurveto{\pgfqpoint{11.672796in}{5.425726in}}{\pgfqpoint{11.668406in}{5.415127in}}{\pgfqpoint{11.668406in}{5.404077in}}%
\pgfpathcurveto{\pgfqpoint{11.668406in}{5.393027in}}{\pgfqpoint{11.672796in}{5.382427in}}{\pgfqpoint{11.680610in}{5.374614in}}%
\pgfpathcurveto{\pgfqpoint{11.688423in}{5.366800in}}{\pgfqpoint{11.699022in}{5.362410in}}{\pgfqpoint{11.710072in}{5.362410in}}%
\pgfpathlineto{\pgfqpoint{11.710072in}{5.362410in}}%
\pgfpathclose%
\pgfusepath{stroke}%
\end{pgfscope}%
\begin{pgfscope}%
\pgfpathrectangle{\pgfqpoint{7.512535in}{0.437222in}}{\pgfqpoint{6.275590in}{5.159444in}}%
\pgfusepath{clip}%
\pgfsetbuttcap%
\pgfsetroundjoin%
\pgfsetlinewidth{1.003750pt}%
\definecolor{currentstroke}{rgb}{0.827451,0.827451,0.827451}%
\pgfsetstrokecolor{currentstroke}%
\pgfsetstrokeopacity{0.800000}%
\pgfsetdash{}{0pt}%
\pgfpathmoveto{\pgfqpoint{11.806382in}{5.471994in}}%
\pgfpathcurveto{\pgfqpoint{11.817432in}{5.471994in}}{\pgfqpoint{11.828031in}{5.476384in}}{\pgfqpoint{11.835845in}{5.484198in}}%
\pgfpathcurveto{\pgfqpoint{11.843658in}{5.492011in}}{\pgfqpoint{11.848049in}{5.502610in}}{\pgfqpoint{11.848049in}{5.513660in}}%
\pgfpathcurveto{\pgfqpoint{11.848049in}{5.524711in}}{\pgfqpoint{11.843658in}{5.535310in}}{\pgfqpoint{11.835845in}{5.543123in}}%
\pgfpathcurveto{\pgfqpoint{11.828031in}{5.550937in}}{\pgfqpoint{11.817432in}{5.555327in}}{\pgfqpoint{11.806382in}{5.555327in}}%
\pgfpathcurveto{\pgfqpoint{11.795332in}{5.555327in}}{\pgfqpoint{11.784733in}{5.550937in}}{\pgfqpoint{11.776919in}{5.543123in}}%
\pgfpathcurveto{\pgfqpoint{11.769106in}{5.535310in}}{\pgfqpoint{11.764715in}{5.524711in}}{\pgfqpoint{11.764715in}{5.513660in}}%
\pgfpathcurveto{\pgfqpoint{11.764715in}{5.502610in}}{\pgfqpoint{11.769106in}{5.492011in}}{\pgfqpoint{11.776919in}{5.484198in}}%
\pgfpathcurveto{\pgfqpoint{11.784733in}{5.476384in}}{\pgfqpoint{11.795332in}{5.471994in}}{\pgfqpoint{11.806382in}{5.471994in}}%
\pgfpathlineto{\pgfqpoint{11.806382in}{5.471994in}}%
\pgfpathclose%
\pgfusepath{stroke}%
\end{pgfscope}%
\begin{pgfscope}%
\pgfpathrectangle{\pgfqpoint{7.512535in}{0.437222in}}{\pgfqpoint{6.275590in}{5.159444in}}%
\pgfusepath{clip}%
\pgfsetbuttcap%
\pgfsetroundjoin%
\pgfsetlinewidth{1.003750pt}%
\definecolor{currentstroke}{rgb}{0.827451,0.827451,0.827451}%
\pgfsetstrokecolor{currentstroke}%
\pgfsetstrokeopacity{0.800000}%
\pgfsetdash{}{0pt}%
\pgfpathmoveto{\pgfqpoint{9.878128in}{3.472188in}}%
\pgfpathcurveto{\pgfqpoint{9.889178in}{3.472188in}}{\pgfqpoint{9.899777in}{3.476579in}}{\pgfqpoint{9.907590in}{3.484392in}}%
\pgfpathcurveto{\pgfqpoint{9.915404in}{3.492206in}}{\pgfqpoint{9.919794in}{3.502805in}}{\pgfqpoint{9.919794in}{3.513855in}}%
\pgfpathcurveto{\pgfqpoint{9.919794in}{3.524905in}}{\pgfqpoint{9.915404in}{3.535504in}}{\pgfqpoint{9.907590in}{3.543318in}}%
\pgfpathcurveto{\pgfqpoint{9.899777in}{3.551131in}}{\pgfqpoint{9.889178in}{3.555522in}}{\pgfqpoint{9.878128in}{3.555522in}}%
\pgfpathcurveto{\pgfqpoint{9.867077in}{3.555522in}}{\pgfqpoint{9.856478in}{3.551131in}}{\pgfqpoint{9.848665in}{3.543318in}}%
\pgfpathcurveto{\pgfqpoint{9.840851in}{3.535504in}}{\pgfqpoint{9.836461in}{3.524905in}}{\pgfqpoint{9.836461in}{3.513855in}}%
\pgfpathcurveto{\pgfqpoint{9.836461in}{3.502805in}}{\pgfqpoint{9.840851in}{3.492206in}}{\pgfqpoint{9.848665in}{3.484392in}}%
\pgfpathcurveto{\pgfqpoint{9.856478in}{3.476579in}}{\pgfqpoint{9.867077in}{3.472188in}}{\pgfqpoint{9.878128in}{3.472188in}}%
\pgfpathlineto{\pgfqpoint{9.878128in}{3.472188in}}%
\pgfpathclose%
\pgfusepath{stroke}%
\end{pgfscope}%
\begin{pgfscope}%
\pgfpathrectangle{\pgfqpoint{7.512535in}{0.437222in}}{\pgfqpoint{6.275590in}{5.159444in}}%
\pgfusepath{clip}%
\pgfsetbuttcap%
\pgfsetroundjoin%
\pgfsetlinewidth{1.003750pt}%
\definecolor{currentstroke}{rgb}{0.827451,0.827451,0.827451}%
\pgfsetstrokecolor{currentstroke}%
\pgfsetstrokeopacity{0.800000}%
\pgfsetdash{}{0pt}%
\pgfpathmoveto{\pgfqpoint{11.999433in}{5.514216in}}%
\pgfpathcurveto{\pgfqpoint{12.010483in}{5.514216in}}{\pgfqpoint{12.021082in}{5.518606in}}{\pgfqpoint{12.028896in}{5.526420in}}%
\pgfpathcurveto{\pgfqpoint{12.036710in}{5.534233in}}{\pgfqpoint{12.041100in}{5.544832in}}{\pgfqpoint{12.041100in}{5.555882in}}%
\pgfpathcurveto{\pgfqpoint{12.041100in}{5.566933in}}{\pgfqpoint{12.036710in}{5.577532in}}{\pgfqpoint{12.028896in}{5.585345in}}%
\pgfpathcurveto{\pgfqpoint{12.021082in}{5.593159in}}{\pgfqpoint{12.010483in}{5.597549in}}{\pgfqpoint{11.999433in}{5.597549in}}%
\pgfpathcurveto{\pgfqpoint{11.988383in}{5.597549in}}{\pgfqpoint{11.977784in}{5.593159in}}{\pgfqpoint{11.969970in}{5.585345in}}%
\pgfpathcurveto{\pgfqpoint{11.962157in}{5.577532in}}{\pgfqpoint{11.957767in}{5.566933in}}{\pgfqpoint{11.957767in}{5.555882in}}%
\pgfpathcurveto{\pgfqpoint{11.957767in}{5.544832in}}{\pgfqpoint{11.962157in}{5.534233in}}{\pgfqpoint{11.969970in}{5.526420in}}%
\pgfpathcurveto{\pgfqpoint{11.977784in}{5.518606in}}{\pgfqpoint{11.988383in}{5.514216in}}{\pgfqpoint{11.999433in}{5.514216in}}%
\pgfpathlineto{\pgfqpoint{11.999433in}{5.514216in}}%
\pgfpathclose%
\pgfusepath{stroke}%
\end{pgfscope}%
\begin{pgfscope}%
\pgfpathrectangle{\pgfqpoint{7.512535in}{0.437222in}}{\pgfqpoint{6.275590in}{5.159444in}}%
\pgfusepath{clip}%
\pgfsetbuttcap%
\pgfsetroundjoin%
\pgfsetlinewidth{1.003750pt}%
\definecolor{currentstroke}{rgb}{0.827451,0.827451,0.827451}%
\pgfsetstrokecolor{currentstroke}%
\pgfsetstrokeopacity{0.800000}%
\pgfsetdash{}{0pt}%
\pgfpathmoveto{\pgfqpoint{9.928499in}{4.056499in}}%
\pgfpathcurveto{\pgfqpoint{9.939549in}{4.056499in}}{\pgfqpoint{9.950148in}{4.060889in}}{\pgfqpoint{9.957962in}{4.068703in}}%
\pgfpathcurveto{\pgfqpoint{9.965775in}{4.076516in}}{\pgfqpoint{9.970165in}{4.087115in}}{\pgfqpoint{9.970165in}{4.098166in}}%
\pgfpathcurveto{\pgfqpoint{9.970165in}{4.109216in}}{\pgfqpoint{9.965775in}{4.119815in}}{\pgfqpoint{9.957962in}{4.127628in}}%
\pgfpathcurveto{\pgfqpoint{9.950148in}{4.135442in}}{\pgfqpoint{9.939549in}{4.139832in}}{\pgfqpoint{9.928499in}{4.139832in}}%
\pgfpathcurveto{\pgfqpoint{9.917449in}{4.139832in}}{\pgfqpoint{9.906850in}{4.135442in}}{\pgfqpoint{9.899036in}{4.127628in}}%
\pgfpathcurveto{\pgfqpoint{9.891222in}{4.119815in}}{\pgfqpoint{9.886832in}{4.109216in}}{\pgfqpoint{9.886832in}{4.098166in}}%
\pgfpathcurveto{\pgfqpoint{9.886832in}{4.087115in}}{\pgfqpoint{9.891222in}{4.076516in}}{\pgfqpoint{9.899036in}{4.068703in}}%
\pgfpathcurveto{\pgfqpoint{9.906850in}{4.060889in}}{\pgfqpoint{9.917449in}{4.056499in}}{\pgfqpoint{9.928499in}{4.056499in}}%
\pgfpathlineto{\pgfqpoint{9.928499in}{4.056499in}}%
\pgfpathclose%
\pgfusepath{stroke}%
\end{pgfscope}%
\begin{pgfscope}%
\pgfpathrectangle{\pgfqpoint{7.512535in}{0.437222in}}{\pgfqpoint{6.275590in}{5.159444in}}%
\pgfusepath{clip}%
\pgfsetbuttcap%
\pgfsetroundjoin%
\pgfsetlinewidth{1.003750pt}%
\definecolor{currentstroke}{rgb}{0.827451,0.827451,0.827451}%
\pgfsetstrokecolor{currentstroke}%
\pgfsetstrokeopacity{0.800000}%
\pgfsetdash{}{0pt}%
\pgfpathmoveto{\pgfqpoint{11.517254in}{5.069104in}}%
\pgfpathcurveto{\pgfqpoint{11.528304in}{5.069104in}}{\pgfqpoint{11.538903in}{5.073494in}}{\pgfqpoint{11.546717in}{5.081308in}}%
\pgfpathcurveto{\pgfqpoint{11.554531in}{5.089122in}}{\pgfqpoint{11.558921in}{5.099721in}}{\pgfqpoint{11.558921in}{5.110771in}}%
\pgfpathcurveto{\pgfqpoint{11.558921in}{5.121821in}}{\pgfqpoint{11.554531in}{5.132420in}}{\pgfqpoint{11.546717in}{5.140233in}}%
\pgfpathcurveto{\pgfqpoint{11.538903in}{5.148047in}}{\pgfqpoint{11.528304in}{5.152437in}}{\pgfqpoint{11.517254in}{5.152437in}}%
\pgfpathcurveto{\pgfqpoint{11.506204in}{5.152437in}}{\pgfqpoint{11.495605in}{5.148047in}}{\pgfqpoint{11.487791in}{5.140233in}}%
\pgfpathcurveto{\pgfqpoint{11.479978in}{5.132420in}}{\pgfqpoint{11.475588in}{5.121821in}}{\pgfqpoint{11.475588in}{5.110771in}}%
\pgfpathcurveto{\pgfqpoint{11.475588in}{5.099721in}}{\pgfqpoint{11.479978in}{5.089122in}}{\pgfqpoint{11.487791in}{5.081308in}}%
\pgfpathcurveto{\pgfqpoint{11.495605in}{5.073494in}}{\pgfqpoint{11.506204in}{5.069104in}}{\pgfqpoint{11.517254in}{5.069104in}}%
\pgfpathlineto{\pgfqpoint{11.517254in}{5.069104in}}%
\pgfpathclose%
\pgfusepath{stroke}%
\end{pgfscope}%
\begin{pgfscope}%
\pgfpathrectangle{\pgfqpoint{7.512535in}{0.437222in}}{\pgfqpoint{6.275590in}{5.159444in}}%
\pgfusepath{clip}%
\pgfsetbuttcap%
\pgfsetroundjoin%
\pgfsetlinewidth{1.003750pt}%
\definecolor{currentstroke}{rgb}{0.827451,0.827451,0.827451}%
\pgfsetstrokecolor{currentstroke}%
\pgfsetstrokeopacity{0.800000}%
\pgfsetdash{}{0pt}%
\pgfpathmoveto{\pgfqpoint{13.135726in}{5.511813in}}%
\pgfpathcurveto{\pgfqpoint{13.146776in}{5.511813in}}{\pgfqpoint{13.157375in}{5.516204in}}{\pgfqpoint{13.165189in}{5.524017in}}%
\pgfpathcurveto{\pgfqpoint{13.173003in}{5.531831in}}{\pgfqpoint{13.177393in}{5.542430in}}{\pgfqpoint{13.177393in}{5.553480in}}%
\pgfpathcurveto{\pgfqpoint{13.177393in}{5.564530in}}{\pgfqpoint{13.173003in}{5.575129in}}{\pgfqpoint{13.165189in}{5.582943in}}%
\pgfpathcurveto{\pgfqpoint{13.157375in}{5.590756in}}{\pgfqpoint{13.146776in}{5.595147in}}{\pgfqpoint{13.135726in}{5.595147in}}%
\pgfpathcurveto{\pgfqpoint{13.124676in}{5.595147in}}{\pgfqpoint{13.114077in}{5.590756in}}{\pgfqpoint{13.106263in}{5.582943in}}%
\pgfpathcurveto{\pgfqpoint{13.098450in}{5.575129in}}{\pgfqpoint{13.094059in}{5.564530in}}{\pgfqpoint{13.094059in}{5.553480in}}%
\pgfpathcurveto{\pgfqpoint{13.094059in}{5.542430in}}{\pgfqpoint{13.098450in}{5.531831in}}{\pgfqpoint{13.106263in}{5.524017in}}%
\pgfpathcurveto{\pgfqpoint{13.114077in}{5.516204in}}{\pgfqpoint{13.124676in}{5.511813in}}{\pgfqpoint{13.135726in}{5.511813in}}%
\pgfpathlineto{\pgfqpoint{13.135726in}{5.511813in}}%
\pgfpathclose%
\pgfusepath{stroke}%
\end{pgfscope}%
\begin{pgfscope}%
\pgfpathrectangle{\pgfqpoint{7.512535in}{0.437222in}}{\pgfqpoint{6.275590in}{5.159444in}}%
\pgfusepath{clip}%
\pgfsetbuttcap%
\pgfsetroundjoin%
\pgfsetlinewidth{1.003750pt}%
\definecolor{currentstroke}{rgb}{0.827451,0.827451,0.827451}%
\pgfsetstrokecolor{currentstroke}%
\pgfsetstrokeopacity{0.800000}%
\pgfsetdash{}{0pt}%
\pgfpathmoveto{\pgfqpoint{9.697299in}{3.449334in}}%
\pgfpathcurveto{\pgfqpoint{9.708349in}{3.449334in}}{\pgfqpoint{9.718948in}{3.453725in}}{\pgfqpoint{9.726762in}{3.461538in}}%
\pgfpathcurveto{\pgfqpoint{9.734575in}{3.469352in}}{\pgfqpoint{9.738965in}{3.479951in}}{\pgfqpoint{9.738965in}{3.491001in}}%
\pgfpathcurveto{\pgfqpoint{9.738965in}{3.502051in}}{\pgfqpoint{9.734575in}{3.512650in}}{\pgfqpoint{9.726762in}{3.520464in}}%
\pgfpathcurveto{\pgfqpoint{9.718948in}{3.528277in}}{\pgfqpoint{9.708349in}{3.532668in}}{\pgfqpoint{9.697299in}{3.532668in}}%
\pgfpathcurveto{\pgfqpoint{9.686249in}{3.532668in}}{\pgfqpoint{9.675650in}{3.528277in}}{\pgfqpoint{9.667836in}{3.520464in}}%
\pgfpathcurveto{\pgfqpoint{9.660022in}{3.512650in}}{\pgfqpoint{9.655632in}{3.502051in}}{\pgfqpoint{9.655632in}{3.491001in}}%
\pgfpathcurveto{\pgfqpoint{9.655632in}{3.479951in}}{\pgfqpoint{9.660022in}{3.469352in}}{\pgfqpoint{9.667836in}{3.461538in}}%
\pgfpathcurveto{\pgfqpoint{9.675650in}{3.453725in}}{\pgfqpoint{9.686249in}{3.449334in}}{\pgfqpoint{9.697299in}{3.449334in}}%
\pgfpathlineto{\pgfqpoint{9.697299in}{3.449334in}}%
\pgfpathclose%
\pgfusepath{stroke}%
\end{pgfscope}%
\begin{pgfscope}%
\pgfpathrectangle{\pgfqpoint{7.512535in}{0.437222in}}{\pgfqpoint{6.275590in}{5.159444in}}%
\pgfusepath{clip}%
\pgfsetbuttcap%
\pgfsetroundjoin%
\pgfsetlinewidth{1.003750pt}%
\definecolor{currentstroke}{rgb}{0.827451,0.827451,0.827451}%
\pgfsetstrokecolor{currentstroke}%
\pgfsetstrokeopacity{0.800000}%
\pgfsetdash{}{0pt}%
\pgfpathmoveto{\pgfqpoint{9.027094in}{2.556537in}}%
\pgfpathcurveto{\pgfqpoint{9.038145in}{2.556537in}}{\pgfqpoint{9.048744in}{2.560927in}}{\pgfqpoint{9.056557in}{2.568740in}}%
\pgfpathcurveto{\pgfqpoint{9.064371in}{2.576554in}}{\pgfqpoint{9.068761in}{2.587153in}}{\pgfqpoint{9.068761in}{2.598203in}}%
\pgfpathcurveto{\pgfqpoint{9.068761in}{2.609253in}}{\pgfqpoint{9.064371in}{2.619852in}}{\pgfqpoint{9.056557in}{2.627666in}}%
\pgfpathcurveto{\pgfqpoint{9.048744in}{2.635480in}}{\pgfqpoint{9.038145in}{2.639870in}}{\pgfqpoint{9.027094in}{2.639870in}}%
\pgfpathcurveto{\pgfqpoint{9.016044in}{2.639870in}}{\pgfqpoint{9.005445in}{2.635480in}}{\pgfqpoint{8.997632in}{2.627666in}}%
\pgfpathcurveto{\pgfqpoint{8.989818in}{2.619852in}}{\pgfqpoint{8.985428in}{2.609253in}}{\pgfqpoint{8.985428in}{2.598203in}}%
\pgfpathcurveto{\pgfqpoint{8.985428in}{2.587153in}}{\pgfqpoint{8.989818in}{2.576554in}}{\pgfqpoint{8.997632in}{2.568740in}}%
\pgfpathcurveto{\pgfqpoint{9.005445in}{2.560927in}}{\pgfqpoint{9.016044in}{2.556537in}}{\pgfqpoint{9.027094in}{2.556537in}}%
\pgfpathlineto{\pgfqpoint{9.027094in}{2.556537in}}%
\pgfpathclose%
\pgfusepath{stroke}%
\end{pgfscope}%
\begin{pgfscope}%
\pgfpathrectangle{\pgfqpoint{7.512535in}{0.437222in}}{\pgfqpoint{6.275590in}{5.159444in}}%
\pgfusepath{clip}%
\pgfsetbuttcap%
\pgfsetroundjoin%
\pgfsetlinewidth{1.003750pt}%
\definecolor{currentstroke}{rgb}{0.827451,0.827451,0.827451}%
\pgfsetstrokecolor{currentstroke}%
\pgfsetstrokeopacity{0.800000}%
\pgfsetdash{}{0pt}%
\pgfpathmoveto{\pgfqpoint{7.860715in}{0.736624in}}%
\pgfpathcurveto{\pgfqpoint{7.871765in}{0.736624in}}{\pgfqpoint{7.882364in}{0.741014in}}{\pgfqpoint{7.890178in}{0.748828in}}%
\pgfpathcurveto{\pgfqpoint{7.897991in}{0.756641in}}{\pgfqpoint{7.902381in}{0.767240in}}{\pgfqpoint{7.902381in}{0.778290in}}%
\pgfpathcurveto{\pgfqpoint{7.902381in}{0.789340in}}{\pgfqpoint{7.897991in}{0.799940in}}{\pgfqpoint{7.890178in}{0.807753in}}%
\pgfpathcurveto{\pgfqpoint{7.882364in}{0.815567in}}{\pgfqpoint{7.871765in}{0.819957in}}{\pgfqpoint{7.860715in}{0.819957in}}%
\pgfpathcurveto{\pgfqpoint{7.849665in}{0.819957in}}{\pgfqpoint{7.839066in}{0.815567in}}{\pgfqpoint{7.831252in}{0.807753in}}%
\pgfpathcurveto{\pgfqpoint{7.823438in}{0.799940in}}{\pgfqpoint{7.819048in}{0.789340in}}{\pgfqpoint{7.819048in}{0.778290in}}%
\pgfpathcurveto{\pgfqpoint{7.819048in}{0.767240in}}{\pgfqpoint{7.823438in}{0.756641in}}{\pgfqpoint{7.831252in}{0.748828in}}%
\pgfpathcurveto{\pgfqpoint{7.839066in}{0.741014in}}{\pgfqpoint{7.849665in}{0.736624in}}{\pgfqpoint{7.860715in}{0.736624in}}%
\pgfpathlineto{\pgfqpoint{7.860715in}{0.736624in}}%
\pgfpathclose%
\pgfusepath{stroke}%
\end{pgfscope}%
\begin{pgfscope}%
\pgfpathrectangle{\pgfqpoint{7.512535in}{0.437222in}}{\pgfqpoint{6.275590in}{5.159444in}}%
\pgfusepath{clip}%
\pgfsetbuttcap%
\pgfsetroundjoin%
\pgfsetlinewidth{1.003750pt}%
\definecolor{currentstroke}{rgb}{0.827451,0.827451,0.827451}%
\pgfsetstrokecolor{currentstroke}%
\pgfsetstrokeopacity{0.800000}%
\pgfsetdash{}{0pt}%
\pgfpathmoveto{\pgfqpoint{8.944819in}{2.926208in}}%
\pgfpathcurveto{\pgfqpoint{8.955870in}{2.926208in}}{\pgfqpoint{8.966469in}{2.930599in}}{\pgfqpoint{8.974282in}{2.938412in}}%
\pgfpathcurveto{\pgfqpoint{8.982096in}{2.946226in}}{\pgfqpoint{8.986486in}{2.956825in}}{\pgfqpoint{8.986486in}{2.967875in}}%
\pgfpathcurveto{\pgfqpoint{8.986486in}{2.978925in}}{\pgfqpoint{8.982096in}{2.989524in}}{\pgfqpoint{8.974282in}{2.997338in}}%
\pgfpathcurveto{\pgfqpoint{8.966469in}{3.005152in}}{\pgfqpoint{8.955870in}{3.009542in}}{\pgfqpoint{8.944819in}{3.009542in}}%
\pgfpathcurveto{\pgfqpoint{8.933769in}{3.009542in}}{\pgfqpoint{8.923170in}{3.005152in}}{\pgfqpoint{8.915357in}{2.997338in}}%
\pgfpathcurveto{\pgfqpoint{8.907543in}{2.989524in}}{\pgfqpoint{8.903153in}{2.978925in}}{\pgfqpoint{8.903153in}{2.967875in}}%
\pgfpathcurveto{\pgfqpoint{8.903153in}{2.956825in}}{\pgfqpoint{8.907543in}{2.946226in}}{\pgfqpoint{8.915357in}{2.938412in}}%
\pgfpathcurveto{\pgfqpoint{8.923170in}{2.930599in}}{\pgfqpoint{8.933769in}{2.926208in}}{\pgfqpoint{8.944819in}{2.926208in}}%
\pgfpathlineto{\pgfqpoint{8.944819in}{2.926208in}}%
\pgfpathclose%
\pgfusepath{stroke}%
\end{pgfscope}%
\begin{pgfscope}%
\pgfpathrectangle{\pgfqpoint{7.512535in}{0.437222in}}{\pgfqpoint{6.275590in}{5.159444in}}%
\pgfusepath{clip}%
\pgfsetbuttcap%
\pgfsetroundjoin%
\pgfsetlinewidth{1.003750pt}%
\definecolor{currentstroke}{rgb}{0.827451,0.827451,0.827451}%
\pgfsetstrokecolor{currentstroke}%
\pgfsetstrokeopacity{0.800000}%
\pgfsetdash{}{0pt}%
\pgfpathmoveto{\pgfqpoint{10.411837in}{5.341834in}}%
\pgfpathcurveto{\pgfqpoint{10.422887in}{5.341834in}}{\pgfqpoint{10.433486in}{5.346224in}}{\pgfqpoint{10.441300in}{5.354038in}}%
\pgfpathcurveto{\pgfqpoint{10.449113in}{5.361851in}}{\pgfqpoint{10.453504in}{5.372450in}}{\pgfqpoint{10.453504in}{5.383500in}}%
\pgfpathcurveto{\pgfqpoint{10.453504in}{5.394551in}}{\pgfqpoint{10.449113in}{5.405150in}}{\pgfqpoint{10.441300in}{5.412963in}}%
\pgfpathcurveto{\pgfqpoint{10.433486in}{5.420777in}}{\pgfqpoint{10.422887in}{5.425167in}}{\pgfqpoint{10.411837in}{5.425167in}}%
\pgfpathcurveto{\pgfqpoint{10.400787in}{5.425167in}}{\pgfqpoint{10.390188in}{5.420777in}}{\pgfqpoint{10.382374in}{5.412963in}}%
\pgfpathcurveto{\pgfqpoint{10.374561in}{5.405150in}}{\pgfqpoint{10.370170in}{5.394551in}}{\pgfqpoint{10.370170in}{5.383500in}}%
\pgfpathcurveto{\pgfqpoint{10.370170in}{5.372450in}}{\pgfqpoint{10.374561in}{5.361851in}}{\pgfqpoint{10.382374in}{5.354038in}}%
\pgfpathcurveto{\pgfqpoint{10.390188in}{5.346224in}}{\pgfqpoint{10.400787in}{5.341834in}}{\pgfqpoint{10.411837in}{5.341834in}}%
\pgfpathlineto{\pgfqpoint{10.411837in}{5.341834in}}%
\pgfpathclose%
\pgfusepath{stroke}%
\end{pgfscope}%
\begin{pgfscope}%
\pgfpathrectangle{\pgfqpoint{7.512535in}{0.437222in}}{\pgfqpoint{6.275590in}{5.159444in}}%
\pgfusepath{clip}%
\pgfsetbuttcap%
\pgfsetroundjoin%
\pgfsetlinewidth{1.003750pt}%
\definecolor{currentstroke}{rgb}{0.827451,0.827451,0.827451}%
\pgfsetstrokecolor{currentstroke}%
\pgfsetstrokeopacity{0.800000}%
\pgfsetdash{}{0pt}%
\pgfpathmoveto{\pgfqpoint{8.817107in}{2.424149in}}%
\pgfpathcurveto{\pgfqpoint{8.828157in}{2.424149in}}{\pgfqpoint{8.838756in}{2.428540in}}{\pgfqpoint{8.846570in}{2.436353in}}%
\pgfpathcurveto{\pgfqpoint{8.854383in}{2.444167in}}{\pgfqpoint{8.858774in}{2.454766in}}{\pgfqpoint{8.858774in}{2.465816in}}%
\pgfpathcurveto{\pgfqpoint{8.858774in}{2.476866in}}{\pgfqpoint{8.854383in}{2.487465in}}{\pgfqpoint{8.846570in}{2.495279in}}%
\pgfpathcurveto{\pgfqpoint{8.838756in}{2.503093in}}{\pgfqpoint{8.828157in}{2.507483in}}{\pgfqpoint{8.817107in}{2.507483in}}%
\pgfpathcurveto{\pgfqpoint{8.806057in}{2.507483in}}{\pgfqpoint{8.795458in}{2.503093in}}{\pgfqpoint{8.787644in}{2.495279in}}%
\pgfpathcurveto{\pgfqpoint{8.779831in}{2.487465in}}{\pgfqpoint{8.775440in}{2.476866in}}{\pgfqpoint{8.775440in}{2.465816in}}%
\pgfpathcurveto{\pgfqpoint{8.775440in}{2.454766in}}{\pgfqpoint{8.779831in}{2.444167in}}{\pgfqpoint{8.787644in}{2.436353in}}%
\pgfpathcurveto{\pgfqpoint{8.795458in}{2.428540in}}{\pgfqpoint{8.806057in}{2.424149in}}{\pgfqpoint{8.817107in}{2.424149in}}%
\pgfpathlineto{\pgfqpoint{8.817107in}{2.424149in}}%
\pgfpathclose%
\pgfusepath{stroke}%
\end{pgfscope}%
\begin{pgfscope}%
\pgfpathrectangle{\pgfqpoint{7.512535in}{0.437222in}}{\pgfqpoint{6.275590in}{5.159444in}}%
\pgfusepath{clip}%
\pgfsetbuttcap%
\pgfsetroundjoin%
\pgfsetlinewidth{1.003750pt}%
\definecolor{currentstroke}{rgb}{0.827451,0.827451,0.827451}%
\pgfsetstrokecolor{currentstroke}%
\pgfsetstrokeopacity{0.800000}%
\pgfsetdash{}{0pt}%
\pgfpathmoveto{\pgfqpoint{7.711198in}{0.615504in}}%
\pgfpathcurveto{\pgfqpoint{7.722248in}{0.615504in}}{\pgfqpoint{7.732847in}{0.619894in}}{\pgfqpoint{7.740661in}{0.627708in}}%
\pgfpathcurveto{\pgfqpoint{7.748474in}{0.635522in}}{\pgfqpoint{7.752865in}{0.646121in}}{\pgfqpoint{7.752865in}{0.657171in}}%
\pgfpathcurveto{\pgfqpoint{7.752865in}{0.668221in}}{\pgfqpoint{7.748474in}{0.678820in}}{\pgfqpoint{7.740661in}{0.686633in}}%
\pgfpathcurveto{\pgfqpoint{7.732847in}{0.694447in}}{\pgfqpoint{7.722248in}{0.698837in}}{\pgfqpoint{7.711198in}{0.698837in}}%
\pgfpathcurveto{\pgfqpoint{7.700148in}{0.698837in}}{\pgfqpoint{7.689549in}{0.694447in}}{\pgfqpoint{7.681735in}{0.686633in}}%
\pgfpathcurveto{\pgfqpoint{7.673921in}{0.678820in}}{\pgfqpoint{7.669531in}{0.668221in}}{\pgfqpoint{7.669531in}{0.657171in}}%
\pgfpathcurveto{\pgfqpoint{7.669531in}{0.646121in}}{\pgfqpoint{7.673921in}{0.635522in}}{\pgfqpoint{7.681735in}{0.627708in}}%
\pgfpathcurveto{\pgfqpoint{7.689549in}{0.619894in}}{\pgfqpoint{7.700148in}{0.615504in}}{\pgfqpoint{7.711198in}{0.615504in}}%
\pgfpathlineto{\pgfqpoint{7.711198in}{0.615504in}}%
\pgfpathclose%
\pgfusepath{stroke}%
\end{pgfscope}%
\begin{pgfscope}%
\pgfpathrectangle{\pgfqpoint{7.512535in}{0.437222in}}{\pgfqpoint{6.275590in}{5.159444in}}%
\pgfusepath{clip}%
\pgfsetbuttcap%
\pgfsetroundjoin%
\pgfsetlinewidth{1.003750pt}%
\definecolor{currentstroke}{rgb}{0.827451,0.827451,0.827451}%
\pgfsetstrokecolor{currentstroke}%
\pgfsetstrokeopacity{0.800000}%
\pgfsetdash{}{0pt}%
\pgfpathmoveto{\pgfqpoint{12.051194in}{5.538678in}}%
\pgfpathcurveto{\pgfqpoint{12.062244in}{5.538678in}}{\pgfqpoint{12.072843in}{5.543068in}}{\pgfqpoint{12.080656in}{5.550882in}}%
\pgfpathcurveto{\pgfqpoint{12.088470in}{5.558695in}}{\pgfqpoint{12.092860in}{5.569294in}}{\pgfqpoint{12.092860in}{5.580344in}}%
\pgfpathcurveto{\pgfqpoint{12.092860in}{5.591394in}}{\pgfqpoint{12.088470in}{5.601993in}}{\pgfqpoint{12.080656in}{5.609807in}}%
\pgfpathcurveto{\pgfqpoint{12.072843in}{5.617621in}}{\pgfqpoint{12.062244in}{5.622011in}}{\pgfqpoint{12.051194in}{5.622011in}}%
\pgfpathcurveto{\pgfqpoint{12.040144in}{5.622011in}}{\pgfqpoint{12.029544in}{5.617621in}}{\pgfqpoint{12.021731in}{5.609807in}}%
\pgfpathcurveto{\pgfqpoint{12.013917in}{5.601993in}}{\pgfqpoint{12.009527in}{5.591394in}}{\pgfqpoint{12.009527in}{5.580344in}}%
\pgfpathcurveto{\pgfqpoint{12.009527in}{5.569294in}}{\pgfqpoint{12.013917in}{5.558695in}}{\pgfqpoint{12.021731in}{5.550882in}}%
\pgfpathcurveto{\pgfqpoint{12.029544in}{5.543068in}}{\pgfqpoint{12.040144in}{5.538678in}}{\pgfqpoint{12.051194in}{5.538678in}}%
\pgfpathlineto{\pgfqpoint{12.051194in}{5.538678in}}%
\pgfpathclose%
\pgfusepath{stroke}%
\end{pgfscope}%
\begin{pgfscope}%
\pgfpathrectangle{\pgfqpoint{7.512535in}{0.437222in}}{\pgfqpoint{6.275590in}{5.159444in}}%
\pgfusepath{clip}%
\pgfsetbuttcap%
\pgfsetroundjoin%
\pgfsetlinewidth{1.003750pt}%
\definecolor{currentstroke}{rgb}{0.827451,0.827451,0.827451}%
\pgfsetstrokecolor{currentstroke}%
\pgfsetstrokeopacity{0.800000}%
\pgfsetdash{}{0pt}%
\pgfpathmoveto{\pgfqpoint{12.677892in}{5.550102in}}%
\pgfpathcurveto{\pgfqpoint{12.688942in}{5.550102in}}{\pgfqpoint{12.699541in}{5.554492in}}{\pgfqpoint{12.707354in}{5.562306in}}%
\pgfpathcurveto{\pgfqpoint{12.715168in}{5.570119in}}{\pgfqpoint{12.719558in}{5.580718in}}{\pgfqpoint{12.719558in}{5.591769in}}%
\pgfpathcurveto{\pgfqpoint{12.719558in}{5.602819in}}{\pgfqpoint{12.715168in}{5.613418in}}{\pgfqpoint{12.707354in}{5.621231in}}%
\pgfpathcurveto{\pgfqpoint{12.699541in}{5.629045in}}{\pgfqpoint{12.688942in}{5.633435in}}{\pgfqpoint{12.677892in}{5.633435in}}%
\pgfpathcurveto{\pgfqpoint{12.666841in}{5.633435in}}{\pgfqpoint{12.656242in}{5.629045in}}{\pgfqpoint{12.648429in}{5.621231in}}%
\pgfpathcurveto{\pgfqpoint{12.640615in}{5.613418in}}{\pgfqpoint{12.636225in}{5.602819in}}{\pgfqpoint{12.636225in}{5.591769in}}%
\pgfpathcurveto{\pgfqpoint{12.636225in}{5.580718in}}{\pgfqpoint{12.640615in}{5.570119in}}{\pgfqpoint{12.648429in}{5.562306in}}%
\pgfpathcurveto{\pgfqpoint{12.656242in}{5.554492in}}{\pgfqpoint{12.666841in}{5.550102in}}{\pgfqpoint{12.677892in}{5.550102in}}%
\pgfpathlineto{\pgfqpoint{12.677892in}{5.550102in}}%
\pgfpathclose%
\pgfusepath{stroke}%
\end{pgfscope}%
\begin{pgfscope}%
\pgfpathrectangle{\pgfqpoint{7.512535in}{0.437222in}}{\pgfqpoint{6.275590in}{5.159444in}}%
\pgfusepath{clip}%
\pgfsetbuttcap%
\pgfsetroundjoin%
\pgfsetlinewidth{1.003750pt}%
\definecolor{currentstroke}{rgb}{0.827451,0.827451,0.827451}%
\pgfsetstrokecolor{currentstroke}%
\pgfsetstrokeopacity{0.800000}%
\pgfsetdash{}{0pt}%
\pgfpathmoveto{\pgfqpoint{10.450205in}{5.521083in}}%
\pgfpathcurveto{\pgfqpoint{10.461255in}{5.521083in}}{\pgfqpoint{10.471854in}{5.525473in}}{\pgfqpoint{10.479667in}{5.533287in}}%
\pgfpathcurveto{\pgfqpoint{10.487481in}{5.541100in}}{\pgfqpoint{10.491871in}{5.551699in}}{\pgfqpoint{10.491871in}{5.562749in}}%
\pgfpathcurveto{\pgfqpoint{10.491871in}{5.573800in}}{\pgfqpoint{10.487481in}{5.584399in}}{\pgfqpoint{10.479667in}{5.592212in}}%
\pgfpathcurveto{\pgfqpoint{10.471854in}{5.600026in}}{\pgfqpoint{10.461255in}{5.604416in}}{\pgfqpoint{10.450205in}{5.604416in}}%
\pgfpathcurveto{\pgfqpoint{10.439155in}{5.604416in}}{\pgfqpoint{10.428556in}{5.600026in}}{\pgfqpoint{10.420742in}{5.592212in}}%
\pgfpathcurveto{\pgfqpoint{10.412928in}{5.584399in}}{\pgfqpoint{10.408538in}{5.573800in}}{\pgfqpoint{10.408538in}{5.562749in}}%
\pgfpathcurveto{\pgfqpoint{10.408538in}{5.551699in}}{\pgfqpoint{10.412928in}{5.541100in}}{\pgfqpoint{10.420742in}{5.533287in}}%
\pgfpathcurveto{\pgfqpoint{10.428556in}{5.525473in}}{\pgfqpoint{10.439155in}{5.521083in}}{\pgfqpoint{10.450205in}{5.521083in}}%
\pgfpathlineto{\pgfqpoint{10.450205in}{5.521083in}}%
\pgfpathclose%
\pgfusepath{stroke}%
\end{pgfscope}%
\begin{pgfscope}%
\pgfpathrectangle{\pgfqpoint{7.512535in}{0.437222in}}{\pgfqpoint{6.275590in}{5.159444in}}%
\pgfusepath{clip}%
\pgfsetbuttcap%
\pgfsetroundjoin%
\pgfsetlinewidth{1.003750pt}%
\definecolor{currentstroke}{rgb}{0.827451,0.827451,0.827451}%
\pgfsetstrokecolor{currentstroke}%
\pgfsetstrokeopacity{0.800000}%
\pgfsetdash{}{0pt}%
\pgfpathmoveto{\pgfqpoint{9.171368in}{3.241859in}}%
\pgfpathcurveto{\pgfqpoint{9.182418in}{3.241859in}}{\pgfqpoint{9.193017in}{3.246249in}}{\pgfqpoint{9.200830in}{3.254063in}}%
\pgfpathcurveto{\pgfqpoint{9.208644in}{3.261876in}}{\pgfqpoint{9.213034in}{3.272475in}}{\pgfqpoint{9.213034in}{3.283525in}}%
\pgfpathcurveto{\pgfqpoint{9.213034in}{3.294576in}}{\pgfqpoint{9.208644in}{3.305175in}}{\pgfqpoint{9.200830in}{3.312988in}}%
\pgfpathcurveto{\pgfqpoint{9.193017in}{3.320802in}}{\pgfqpoint{9.182418in}{3.325192in}}{\pgfqpoint{9.171368in}{3.325192in}}%
\pgfpathcurveto{\pgfqpoint{9.160317in}{3.325192in}}{\pgfqpoint{9.149718in}{3.320802in}}{\pgfqpoint{9.141905in}{3.312988in}}%
\pgfpathcurveto{\pgfqpoint{9.134091in}{3.305175in}}{\pgfqpoint{9.129701in}{3.294576in}}{\pgfqpoint{9.129701in}{3.283525in}}%
\pgfpathcurveto{\pgfqpoint{9.129701in}{3.272475in}}{\pgfqpoint{9.134091in}{3.261876in}}{\pgfqpoint{9.141905in}{3.254063in}}%
\pgfpathcurveto{\pgfqpoint{9.149718in}{3.246249in}}{\pgfqpoint{9.160317in}{3.241859in}}{\pgfqpoint{9.171368in}{3.241859in}}%
\pgfpathlineto{\pgfqpoint{9.171368in}{3.241859in}}%
\pgfpathclose%
\pgfusepath{stroke}%
\end{pgfscope}%
\begin{pgfscope}%
\pgfpathrectangle{\pgfqpoint{7.512535in}{0.437222in}}{\pgfqpoint{6.275590in}{5.159444in}}%
\pgfusepath{clip}%
\pgfsetbuttcap%
\pgfsetroundjoin%
\pgfsetlinewidth{1.003750pt}%
\definecolor{currentstroke}{rgb}{0.827451,0.827451,0.827451}%
\pgfsetstrokecolor{currentstroke}%
\pgfsetstrokeopacity{0.800000}%
\pgfsetdash{}{0pt}%
\pgfpathmoveto{\pgfqpoint{10.780511in}{5.205075in}}%
\pgfpathcurveto{\pgfqpoint{10.791561in}{5.205075in}}{\pgfqpoint{10.802160in}{5.209466in}}{\pgfqpoint{10.809973in}{5.217279in}}%
\pgfpathcurveto{\pgfqpoint{10.817787in}{5.225093in}}{\pgfqpoint{10.822177in}{5.235692in}}{\pgfqpoint{10.822177in}{5.246742in}}%
\pgfpathcurveto{\pgfqpoint{10.822177in}{5.257792in}}{\pgfqpoint{10.817787in}{5.268391in}}{\pgfqpoint{10.809973in}{5.276205in}}%
\pgfpathcurveto{\pgfqpoint{10.802160in}{5.284019in}}{\pgfqpoint{10.791561in}{5.288409in}}{\pgfqpoint{10.780511in}{5.288409in}}%
\pgfpathcurveto{\pgfqpoint{10.769461in}{5.288409in}}{\pgfqpoint{10.758862in}{5.284019in}}{\pgfqpoint{10.751048in}{5.276205in}}%
\pgfpathcurveto{\pgfqpoint{10.743234in}{5.268391in}}{\pgfqpoint{10.738844in}{5.257792in}}{\pgfqpoint{10.738844in}{5.246742in}}%
\pgfpathcurveto{\pgfqpoint{10.738844in}{5.235692in}}{\pgfqpoint{10.743234in}{5.225093in}}{\pgfqpoint{10.751048in}{5.217279in}}%
\pgfpathcurveto{\pgfqpoint{10.758862in}{5.209466in}}{\pgfqpoint{10.769461in}{5.205075in}}{\pgfqpoint{10.780511in}{5.205075in}}%
\pgfpathlineto{\pgfqpoint{10.780511in}{5.205075in}}%
\pgfpathclose%
\pgfusepath{stroke}%
\end{pgfscope}%
\begin{pgfscope}%
\pgfpathrectangle{\pgfqpoint{7.512535in}{0.437222in}}{\pgfqpoint{6.275590in}{5.159444in}}%
\pgfusepath{clip}%
\pgfsetbuttcap%
\pgfsetroundjoin%
\pgfsetlinewidth{1.003750pt}%
\definecolor{currentstroke}{rgb}{0.827451,0.827451,0.827451}%
\pgfsetstrokecolor{currentstroke}%
\pgfsetstrokeopacity{0.800000}%
\pgfsetdash{}{0pt}%
\pgfpathmoveto{\pgfqpoint{7.874057in}{1.320145in}}%
\pgfpathcurveto{\pgfqpoint{7.885107in}{1.320145in}}{\pgfqpoint{7.895706in}{1.324535in}}{\pgfqpoint{7.903520in}{1.332349in}}%
\pgfpathcurveto{\pgfqpoint{7.911334in}{1.340162in}}{\pgfqpoint{7.915724in}{1.350761in}}{\pgfqpoint{7.915724in}{1.361811in}}%
\pgfpathcurveto{\pgfqpoint{7.915724in}{1.372862in}}{\pgfqpoint{7.911334in}{1.383461in}}{\pgfqpoint{7.903520in}{1.391274in}}%
\pgfpathcurveto{\pgfqpoint{7.895706in}{1.399088in}}{\pgfqpoint{7.885107in}{1.403478in}}{\pgfqpoint{7.874057in}{1.403478in}}%
\pgfpathcurveto{\pgfqpoint{7.863007in}{1.403478in}}{\pgfqpoint{7.852408in}{1.399088in}}{\pgfqpoint{7.844595in}{1.391274in}}%
\pgfpathcurveto{\pgfqpoint{7.836781in}{1.383461in}}{\pgfqpoint{7.832391in}{1.372862in}}{\pgfqpoint{7.832391in}{1.361811in}}%
\pgfpathcurveto{\pgfqpoint{7.832391in}{1.350761in}}{\pgfqpoint{7.836781in}{1.340162in}}{\pgfqpoint{7.844595in}{1.332349in}}%
\pgfpathcurveto{\pgfqpoint{7.852408in}{1.324535in}}{\pgfqpoint{7.863007in}{1.320145in}}{\pgfqpoint{7.874057in}{1.320145in}}%
\pgfpathlineto{\pgfqpoint{7.874057in}{1.320145in}}%
\pgfpathclose%
\pgfusepath{stroke}%
\end{pgfscope}%
\begin{pgfscope}%
\pgfpathrectangle{\pgfqpoint{7.512535in}{0.437222in}}{\pgfqpoint{6.275590in}{5.159444in}}%
\pgfusepath{clip}%
\pgfsetbuttcap%
\pgfsetroundjoin%
\pgfsetlinewidth{1.003750pt}%
\definecolor{currentstroke}{rgb}{0.827451,0.827451,0.827451}%
\pgfsetstrokecolor{currentstroke}%
\pgfsetstrokeopacity{0.800000}%
\pgfsetdash{}{0pt}%
\pgfpathmoveto{\pgfqpoint{10.994755in}{5.238472in}}%
\pgfpathcurveto{\pgfqpoint{11.005805in}{5.238472in}}{\pgfqpoint{11.016404in}{5.242862in}}{\pgfqpoint{11.024218in}{5.250676in}}%
\pgfpathcurveto{\pgfqpoint{11.032031in}{5.258489in}}{\pgfqpoint{11.036422in}{5.269088in}}{\pgfqpoint{11.036422in}{5.280138in}}%
\pgfpathcurveto{\pgfqpoint{11.036422in}{5.291188in}}{\pgfqpoint{11.032031in}{5.301787in}}{\pgfqpoint{11.024218in}{5.309601in}}%
\pgfpathcurveto{\pgfqpoint{11.016404in}{5.317415in}}{\pgfqpoint{11.005805in}{5.321805in}}{\pgfqpoint{10.994755in}{5.321805in}}%
\pgfpathcurveto{\pgfqpoint{10.983705in}{5.321805in}}{\pgfqpoint{10.973106in}{5.317415in}}{\pgfqpoint{10.965292in}{5.309601in}}%
\pgfpathcurveto{\pgfqpoint{10.957479in}{5.301787in}}{\pgfqpoint{10.953088in}{5.291188in}}{\pgfqpoint{10.953088in}{5.280138in}}%
\pgfpathcurveto{\pgfqpoint{10.953088in}{5.269088in}}{\pgfqpoint{10.957479in}{5.258489in}}{\pgfqpoint{10.965292in}{5.250676in}}%
\pgfpathcurveto{\pgfqpoint{10.973106in}{5.242862in}}{\pgfqpoint{10.983705in}{5.238472in}}{\pgfqpoint{10.994755in}{5.238472in}}%
\pgfpathlineto{\pgfqpoint{10.994755in}{5.238472in}}%
\pgfpathclose%
\pgfusepath{stroke}%
\end{pgfscope}%
\begin{pgfscope}%
\pgfpathrectangle{\pgfqpoint{7.512535in}{0.437222in}}{\pgfqpoint{6.275590in}{5.159444in}}%
\pgfusepath{clip}%
\pgfsetbuttcap%
\pgfsetroundjoin%
\pgfsetlinewidth{1.003750pt}%
\definecolor{currentstroke}{rgb}{0.827451,0.827451,0.827451}%
\pgfsetstrokecolor{currentstroke}%
\pgfsetstrokeopacity{0.800000}%
\pgfsetdash{}{0pt}%
\pgfpathmoveto{\pgfqpoint{11.517254in}{5.159213in}}%
\pgfpathcurveto{\pgfqpoint{11.528304in}{5.159213in}}{\pgfqpoint{11.538903in}{5.163603in}}{\pgfqpoint{11.546717in}{5.171417in}}%
\pgfpathcurveto{\pgfqpoint{11.554531in}{5.179230in}}{\pgfqpoint{11.558921in}{5.189829in}}{\pgfqpoint{11.558921in}{5.200879in}}%
\pgfpathcurveto{\pgfqpoint{11.558921in}{5.211930in}}{\pgfqpoint{11.554531in}{5.222529in}}{\pgfqpoint{11.546717in}{5.230342in}}%
\pgfpathcurveto{\pgfqpoint{11.538903in}{5.238156in}}{\pgfqpoint{11.528304in}{5.242546in}}{\pgfqpoint{11.517254in}{5.242546in}}%
\pgfpathcurveto{\pgfqpoint{11.506204in}{5.242546in}}{\pgfqpoint{11.495605in}{5.238156in}}{\pgfqpoint{11.487791in}{5.230342in}}%
\pgfpathcurveto{\pgfqpoint{11.479978in}{5.222529in}}{\pgfqpoint{11.475588in}{5.211930in}}{\pgfqpoint{11.475588in}{5.200879in}}%
\pgfpathcurveto{\pgfqpoint{11.475588in}{5.189829in}}{\pgfqpoint{11.479978in}{5.179230in}}{\pgfqpoint{11.487791in}{5.171417in}}%
\pgfpathcurveto{\pgfqpoint{11.495605in}{5.163603in}}{\pgfqpoint{11.506204in}{5.159213in}}{\pgfqpoint{11.517254in}{5.159213in}}%
\pgfpathlineto{\pgfqpoint{11.517254in}{5.159213in}}%
\pgfpathclose%
\pgfusepath{stroke}%
\end{pgfscope}%
\begin{pgfscope}%
\pgfpathrectangle{\pgfqpoint{7.512535in}{0.437222in}}{\pgfqpoint{6.275590in}{5.159444in}}%
\pgfusepath{clip}%
\pgfsetbuttcap%
\pgfsetroundjoin%
\pgfsetlinewidth{1.003750pt}%
\definecolor{currentstroke}{rgb}{0.827451,0.827451,0.827451}%
\pgfsetstrokecolor{currentstroke}%
\pgfsetstrokeopacity{0.800000}%
\pgfsetdash{}{0pt}%
\pgfpathmoveto{\pgfqpoint{9.816953in}{3.890552in}}%
\pgfpathcurveto{\pgfqpoint{9.828003in}{3.890552in}}{\pgfqpoint{9.838602in}{3.894943in}}{\pgfqpoint{9.846415in}{3.902756in}}%
\pgfpathcurveto{\pgfqpoint{9.854229in}{3.910570in}}{\pgfqpoint{9.858619in}{3.921169in}}{\pgfqpoint{9.858619in}{3.932219in}}%
\pgfpathcurveto{\pgfqpoint{9.858619in}{3.943269in}}{\pgfqpoint{9.854229in}{3.953868in}}{\pgfqpoint{9.846415in}{3.961682in}}%
\pgfpathcurveto{\pgfqpoint{9.838602in}{3.969495in}}{\pgfqpoint{9.828003in}{3.973886in}}{\pgfqpoint{9.816953in}{3.973886in}}%
\pgfpathcurveto{\pgfqpoint{9.805903in}{3.973886in}}{\pgfqpoint{9.795304in}{3.969495in}}{\pgfqpoint{9.787490in}{3.961682in}}%
\pgfpathcurveto{\pgfqpoint{9.779676in}{3.953868in}}{\pgfqpoint{9.775286in}{3.943269in}}{\pgfqpoint{9.775286in}{3.932219in}}%
\pgfpathcurveto{\pgfqpoint{9.775286in}{3.921169in}}{\pgfqpoint{9.779676in}{3.910570in}}{\pgfqpoint{9.787490in}{3.902756in}}%
\pgfpathcurveto{\pgfqpoint{9.795304in}{3.894943in}}{\pgfqpoint{9.805903in}{3.890552in}}{\pgfqpoint{9.816953in}{3.890552in}}%
\pgfpathlineto{\pgfqpoint{9.816953in}{3.890552in}}%
\pgfpathclose%
\pgfusepath{stroke}%
\end{pgfscope}%
\begin{pgfscope}%
\pgfpathrectangle{\pgfqpoint{7.512535in}{0.437222in}}{\pgfqpoint{6.275590in}{5.159444in}}%
\pgfusepath{clip}%
\pgfsetbuttcap%
\pgfsetroundjoin%
\pgfsetlinewidth{1.003750pt}%
\definecolor{currentstroke}{rgb}{0.827451,0.827451,0.827451}%
\pgfsetstrokecolor{currentstroke}%
\pgfsetstrokeopacity{0.800000}%
\pgfsetdash{}{0pt}%
\pgfpathmoveto{\pgfqpoint{8.871031in}{2.110062in}}%
\pgfpathcurveto{\pgfqpoint{8.882081in}{2.110062in}}{\pgfqpoint{8.892680in}{2.114452in}}{\pgfqpoint{8.900493in}{2.122266in}}%
\pgfpathcurveto{\pgfqpoint{8.908307in}{2.130079in}}{\pgfqpoint{8.912697in}{2.140678in}}{\pgfqpoint{8.912697in}{2.151729in}}%
\pgfpathcurveto{\pgfqpoint{8.912697in}{2.162779in}}{\pgfqpoint{8.908307in}{2.173378in}}{\pgfqpoint{8.900493in}{2.181191in}}%
\pgfpathcurveto{\pgfqpoint{8.892680in}{2.189005in}}{\pgfqpoint{8.882081in}{2.193395in}}{\pgfqpoint{8.871031in}{2.193395in}}%
\pgfpathcurveto{\pgfqpoint{8.859980in}{2.193395in}}{\pgfqpoint{8.849381in}{2.189005in}}{\pgfqpoint{8.841568in}{2.181191in}}%
\pgfpathcurveto{\pgfqpoint{8.833754in}{2.173378in}}{\pgfqpoint{8.829364in}{2.162779in}}{\pgfqpoint{8.829364in}{2.151729in}}%
\pgfpathcurveto{\pgfqpoint{8.829364in}{2.140678in}}{\pgfqpoint{8.833754in}{2.130079in}}{\pgfqpoint{8.841568in}{2.122266in}}%
\pgfpathcurveto{\pgfqpoint{8.849381in}{2.114452in}}{\pgfqpoint{8.859980in}{2.110062in}}{\pgfqpoint{8.871031in}{2.110062in}}%
\pgfpathlineto{\pgfqpoint{8.871031in}{2.110062in}}%
\pgfpathclose%
\pgfusepath{stroke}%
\end{pgfscope}%
\begin{pgfscope}%
\pgfpathrectangle{\pgfqpoint{7.512535in}{0.437222in}}{\pgfqpoint{6.275590in}{5.159444in}}%
\pgfusepath{clip}%
\pgfsetbuttcap%
\pgfsetroundjoin%
\pgfsetlinewidth{1.003750pt}%
\definecolor{currentstroke}{rgb}{0.827451,0.827451,0.827451}%
\pgfsetstrokecolor{currentstroke}%
\pgfsetstrokeopacity{0.800000}%
\pgfsetdash{}{0pt}%
\pgfpathmoveto{\pgfqpoint{11.822595in}{5.548832in}}%
\pgfpathcurveto{\pgfqpoint{11.833645in}{5.548832in}}{\pgfqpoint{11.844244in}{5.553222in}}{\pgfqpoint{11.852058in}{5.561036in}}%
\pgfpathcurveto{\pgfqpoint{11.859871in}{5.568849in}}{\pgfqpoint{11.864262in}{5.579448in}}{\pgfqpoint{11.864262in}{5.590499in}}%
\pgfpathcurveto{\pgfqpoint{11.864262in}{5.601549in}}{\pgfqpoint{11.859871in}{5.612148in}}{\pgfqpoint{11.852058in}{5.619961in}}%
\pgfpathcurveto{\pgfqpoint{11.844244in}{5.627775in}}{\pgfqpoint{11.833645in}{5.632165in}}{\pgfqpoint{11.822595in}{5.632165in}}%
\pgfpathcurveto{\pgfqpoint{11.811545in}{5.632165in}}{\pgfqpoint{11.800946in}{5.627775in}}{\pgfqpoint{11.793132in}{5.619961in}}%
\pgfpathcurveto{\pgfqpoint{11.785318in}{5.612148in}}{\pgfqpoint{11.780928in}{5.601549in}}{\pgfqpoint{11.780928in}{5.590499in}}%
\pgfpathcurveto{\pgfqpoint{11.780928in}{5.579448in}}{\pgfqpoint{11.785318in}{5.568849in}}{\pgfqpoint{11.793132in}{5.561036in}}%
\pgfpathcurveto{\pgfqpoint{11.800946in}{5.553222in}}{\pgfqpoint{11.811545in}{5.548832in}}{\pgfqpoint{11.822595in}{5.548832in}}%
\pgfpathlineto{\pgfqpoint{11.822595in}{5.548832in}}%
\pgfpathclose%
\pgfusepath{stroke}%
\end{pgfscope}%
\begin{pgfscope}%
\pgfpathrectangle{\pgfqpoint{7.512535in}{0.437222in}}{\pgfqpoint{6.275590in}{5.159444in}}%
\pgfusepath{clip}%
\pgfsetbuttcap%
\pgfsetroundjoin%
\pgfsetlinewidth{1.003750pt}%
\definecolor{currentstroke}{rgb}{0.827451,0.827451,0.827451}%
\pgfsetstrokecolor{currentstroke}%
\pgfsetstrokeopacity{0.800000}%
\pgfsetdash{}{0pt}%
\pgfpathmoveto{\pgfqpoint{13.563316in}{5.539504in}}%
\pgfpathcurveto{\pgfqpoint{13.574366in}{5.539504in}}{\pgfqpoint{13.584965in}{5.543895in}}{\pgfqpoint{13.592778in}{5.551708in}}%
\pgfpathcurveto{\pgfqpoint{13.600592in}{5.559522in}}{\pgfqpoint{13.604982in}{5.570121in}}{\pgfqpoint{13.604982in}{5.581171in}}%
\pgfpathcurveto{\pgfqpoint{13.604982in}{5.592221in}}{\pgfqpoint{13.600592in}{5.602820in}}{\pgfqpoint{13.592778in}{5.610634in}}%
\pgfpathcurveto{\pgfqpoint{13.584965in}{5.618447in}}{\pgfqpoint{13.574366in}{5.622838in}}{\pgfqpoint{13.563316in}{5.622838in}}%
\pgfpathcurveto{\pgfqpoint{13.552265in}{5.622838in}}{\pgfqpoint{13.541666in}{5.618447in}}{\pgfqpoint{13.533853in}{5.610634in}}%
\pgfpathcurveto{\pgfqpoint{13.526039in}{5.602820in}}{\pgfqpoint{13.521649in}{5.592221in}}{\pgfqpoint{13.521649in}{5.581171in}}%
\pgfpathcurveto{\pgfqpoint{13.521649in}{5.570121in}}{\pgfqpoint{13.526039in}{5.559522in}}{\pgfqpoint{13.533853in}{5.551708in}}%
\pgfpathcurveto{\pgfqpoint{13.541666in}{5.543895in}}{\pgfqpoint{13.552265in}{5.539504in}}{\pgfqpoint{13.563316in}{5.539504in}}%
\pgfpathlineto{\pgfqpoint{13.563316in}{5.539504in}}%
\pgfpathclose%
\pgfusepath{stroke}%
\end{pgfscope}%
\begin{pgfscope}%
\pgfpathrectangle{\pgfqpoint{7.512535in}{0.437222in}}{\pgfqpoint{6.275590in}{5.159444in}}%
\pgfusepath{clip}%
\pgfsetbuttcap%
\pgfsetroundjoin%
\pgfsetlinewidth{1.003750pt}%
\definecolor{currentstroke}{rgb}{0.827451,0.827451,0.827451}%
\pgfsetstrokecolor{currentstroke}%
\pgfsetstrokeopacity{0.800000}%
\pgfsetdash{}{0pt}%
\pgfpathmoveto{\pgfqpoint{9.104611in}{2.556537in}}%
\pgfpathcurveto{\pgfqpoint{9.115661in}{2.556537in}}{\pgfqpoint{9.126260in}{2.560927in}}{\pgfqpoint{9.134074in}{2.568740in}}%
\pgfpathcurveto{\pgfqpoint{9.141887in}{2.576554in}}{\pgfqpoint{9.146278in}{2.587153in}}{\pgfqpoint{9.146278in}{2.598203in}}%
\pgfpathcurveto{\pgfqpoint{9.146278in}{2.609253in}}{\pgfqpoint{9.141887in}{2.619852in}}{\pgfqpoint{9.134074in}{2.627666in}}%
\pgfpathcurveto{\pgfqpoint{9.126260in}{2.635480in}}{\pgfqpoint{9.115661in}{2.639870in}}{\pgfqpoint{9.104611in}{2.639870in}}%
\pgfpathcurveto{\pgfqpoint{9.093561in}{2.639870in}}{\pgfqpoint{9.082962in}{2.635480in}}{\pgfqpoint{9.075148in}{2.627666in}}%
\pgfpathcurveto{\pgfqpoint{9.067334in}{2.619852in}}{\pgfqpoint{9.062944in}{2.609253in}}{\pgfqpoint{9.062944in}{2.598203in}}%
\pgfpathcurveto{\pgfqpoint{9.062944in}{2.587153in}}{\pgfqpoint{9.067334in}{2.576554in}}{\pgfqpoint{9.075148in}{2.568740in}}%
\pgfpathcurveto{\pgfqpoint{9.082962in}{2.560927in}}{\pgfqpoint{9.093561in}{2.556537in}}{\pgfqpoint{9.104611in}{2.556537in}}%
\pgfpathlineto{\pgfqpoint{9.104611in}{2.556537in}}%
\pgfpathclose%
\pgfusepath{stroke}%
\end{pgfscope}%
\begin{pgfscope}%
\pgfpathrectangle{\pgfqpoint{7.512535in}{0.437222in}}{\pgfqpoint{6.275590in}{5.159444in}}%
\pgfusepath{clip}%
\pgfsetbuttcap%
\pgfsetroundjoin%
\pgfsetlinewidth{1.003750pt}%
\definecolor{currentstroke}{rgb}{0.827451,0.827451,0.827451}%
\pgfsetstrokecolor{currentstroke}%
\pgfsetstrokeopacity{0.800000}%
\pgfsetdash{}{0pt}%
\pgfpathmoveto{\pgfqpoint{9.630920in}{3.276112in}}%
\pgfpathcurveto{\pgfqpoint{9.641970in}{3.276112in}}{\pgfqpoint{9.652569in}{3.280502in}}{\pgfqpoint{9.660383in}{3.288316in}}%
\pgfpathcurveto{\pgfqpoint{9.668196in}{3.296130in}}{\pgfqpoint{9.672587in}{3.306729in}}{\pgfqpoint{9.672587in}{3.317779in}}%
\pgfpathcurveto{\pgfqpoint{9.672587in}{3.328829in}}{\pgfqpoint{9.668196in}{3.339428in}}{\pgfqpoint{9.660383in}{3.347241in}}%
\pgfpathcurveto{\pgfqpoint{9.652569in}{3.355055in}}{\pgfqpoint{9.641970in}{3.359445in}}{\pgfqpoint{9.630920in}{3.359445in}}%
\pgfpathcurveto{\pgfqpoint{9.619870in}{3.359445in}}{\pgfqpoint{9.609271in}{3.355055in}}{\pgfqpoint{9.601457in}{3.347241in}}%
\pgfpathcurveto{\pgfqpoint{9.593644in}{3.339428in}}{\pgfqpoint{9.589253in}{3.328829in}}{\pgfqpoint{9.589253in}{3.317779in}}%
\pgfpathcurveto{\pgfqpoint{9.589253in}{3.306729in}}{\pgfqpoint{9.593644in}{3.296130in}}{\pgfqpoint{9.601457in}{3.288316in}}%
\pgfpathcurveto{\pgfqpoint{9.609271in}{3.280502in}}{\pgfqpoint{9.619870in}{3.276112in}}{\pgfqpoint{9.630920in}{3.276112in}}%
\pgfpathlineto{\pgfqpoint{9.630920in}{3.276112in}}%
\pgfpathclose%
\pgfusepath{stroke}%
\end{pgfscope}%
\begin{pgfscope}%
\pgfpathrectangle{\pgfqpoint{7.512535in}{0.437222in}}{\pgfqpoint{6.275590in}{5.159444in}}%
\pgfusepath{clip}%
\pgfsetbuttcap%
\pgfsetroundjoin%
\pgfsetlinewidth{1.003750pt}%
\definecolor{currentstroke}{rgb}{0.827451,0.827451,0.827451}%
\pgfsetstrokecolor{currentstroke}%
\pgfsetstrokeopacity{0.800000}%
\pgfsetdash{}{0pt}%
\pgfpathmoveto{\pgfqpoint{9.061862in}{3.235524in}}%
\pgfpathcurveto{\pgfqpoint{9.072912in}{3.235524in}}{\pgfqpoint{9.083511in}{3.239914in}}{\pgfqpoint{9.091325in}{3.247728in}}%
\pgfpathcurveto{\pgfqpoint{9.099138in}{3.255542in}}{\pgfqpoint{9.103529in}{3.266141in}}{\pgfqpoint{9.103529in}{3.277191in}}%
\pgfpathcurveto{\pgfqpoint{9.103529in}{3.288241in}}{\pgfqpoint{9.099138in}{3.298840in}}{\pgfqpoint{9.091325in}{3.306654in}}%
\pgfpathcurveto{\pgfqpoint{9.083511in}{3.314467in}}{\pgfqpoint{9.072912in}{3.318857in}}{\pgfqpoint{9.061862in}{3.318857in}}%
\pgfpathcurveto{\pgfqpoint{9.050812in}{3.318857in}}{\pgfqpoint{9.040213in}{3.314467in}}{\pgfqpoint{9.032399in}{3.306654in}}%
\pgfpathcurveto{\pgfqpoint{9.024586in}{3.298840in}}{\pgfqpoint{9.020195in}{3.288241in}}{\pgfqpoint{9.020195in}{3.277191in}}%
\pgfpathcurveto{\pgfqpoint{9.020195in}{3.266141in}}{\pgfqpoint{9.024586in}{3.255542in}}{\pgfqpoint{9.032399in}{3.247728in}}%
\pgfpathcurveto{\pgfqpoint{9.040213in}{3.239914in}}{\pgfqpoint{9.050812in}{3.235524in}}{\pgfqpoint{9.061862in}{3.235524in}}%
\pgfpathlineto{\pgfqpoint{9.061862in}{3.235524in}}%
\pgfpathclose%
\pgfusepath{stroke}%
\end{pgfscope}%
\begin{pgfscope}%
\pgfpathrectangle{\pgfqpoint{7.512535in}{0.437222in}}{\pgfqpoint{6.275590in}{5.159444in}}%
\pgfusepath{clip}%
\pgfsetbuttcap%
\pgfsetroundjoin%
\pgfsetlinewidth{1.003750pt}%
\definecolor{currentstroke}{rgb}{0.827451,0.827451,0.827451}%
\pgfsetstrokecolor{currentstroke}%
\pgfsetstrokeopacity{0.800000}%
\pgfsetdash{}{0pt}%
\pgfpathmoveto{\pgfqpoint{10.283428in}{4.552452in}}%
\pgfpathcurveto{\pgfqpoint{10.294478in}{4.552452in}}{\pgfqpoint{10.305077in}{4.556842in}}{\pgfqpoint{10.312891in}{4.564656in}}%
\pgfpathcurveto{\pgfqpoint{10.320704in}{4.572470in}}{\pgfqpoint{10.325095in}{4.583069in}}{\pgfqpoint{10.325095in}{4.594119in}}%
\pgfpathcurveto{\pgfqpoint{10.325095in}{4.605169in}}{\pgfqpoint{10.320704in}{4.615768in}}{\pgfqpoint{10.312891in}{4.623582in}}%
\pgfpathcurveto{\pgfqpoint{10.305077in}{4.631395in}}{\pgfqpoint{10.294478in}{4.635785in}}{\pgfqpoint{10.283428in}{4.635785in}}%
\pgfpathcurveto{\pgfqpoint{10.272378in}{4.635785in}}{\pgfqpoint{10.261779in}{4.631395in}}{\pgfqpoint{10.253965in}{4.623582in}}%
\pgfpathcurveto{\pgfqpoint{10.246152in}{4.615768in}}{\pgfqpoint{10.241761in}{4.605169in}}{\pgfqpoint{10.241761in}{4.594119in}}%
\pgfpathcurveto{\pgfqpoint{10.241761in}{4.583069in}}{\pgfqpoint{10.246152in}{4.572470in}}{\pgfqpoint{10.253965in}{4.564656in}}%
\pgfpathcurveto{\pgfqpoint{10.261779in}{4.556842in}}{\pgfqpoint{10.272378in}{4.552452in}}{\pgfqpoint{10.283428in}{4.552452in}}%
\pgfpathlineto{\pgfqpoint{10.283428in}{4.552452in}}%
\pgfpathclose%
\pgfusepath{stroke}%
\end{pgfscope}%
\begin{pgfscope}%
\pgfpathrectangle{\pgfqpoint{7.512535in}{0.437222in}}{\pgfqpoint{6.275590in}{5.159444in}}%
\pgfusepath{clip}%
\pgfsetbuttcap%
\pgfsetroundjoin%
\pgfsetlinewidth{1.003750pt}%
\definecolor{currentstroke}{rgb}{0.827451,0.827451,0.827451}%
\pgfsetstrokecolor{currentstroke}%
\pgfsetstrokeopacity{0.800000}%
\pgfsetdash{}{0pt}%
\pgfpathmoveto{\pgfqpoint{8.290348in}{1.139798in}}%
\pgfpathcurveto{\pgfqpoint{8.301398in}{1.139798in}}{\pgfqpoint{8.311997in}{1.144188in}}{\pgfqpoint{8.319811in}{1.152002in}}%
\pgfpathcurveto{\pgfqpoint{8.327625in}{1.159815in}}{\pgfqpoint{8.332015in}{1.170414in}}{\pgfqpoint{8.332015in}{1.181465in}}%
\pgfpathcurveto{\pgfqpoint{8.332015in}{1.192515in}}{\pgfqpoint{8.327625in}{1.203114in}}{\pgfqpoint{8.319811in}{1.210927in}}%
\pgfpathcurveto{\pgfqpoint{8.311997in}{1.218741in}}{\pgfqpoint{8.301398in}{1.223131in}}{\pgfqpoint{8.290348in}{1.223131in}}%
\pgfpathcurveto{\pgfqpoint{8.279298in}{1.223131in}}{\pgfqpoint{8.268699in}{1.218741in}}{\pgfqpoint{8.260886in}{1.210927in}}%
\pgfpathcurveto{\pgfqpoint{8.253072in}{1.203114in}}{\pgfqpoint{8.248682in}{1.192515in}}{\pgfqpoint{8.248682in}{1.181465in}}%
\pgfpathcurveto{\pgfqpoint{8.248682in}{1.170414in}}{\pgfqpoint{8.253072in}{1.159815in}}{\pgfqpoint{8.260886in}{1.152002in}}%
\pgfpathcurveto{\pgfqpoint{8.268699in}{1.144188in}}{\pgfqpoint{8.279298in}{1.139798in}}{\pgfqpoint{8.290348in}{1.139798in}}%
\pgfpathlineto{\pgfqpoint{8.290348in}{1.139798in}}%
\pgfpathclose%
\pgfusepath{stroke}%
\end{pgfscope}%
\begin{pgfscope}%
\pgfpathrectangle{\pgfqpoint{7.512535in}{0.437222in}}{\pgfqpoint{6.275590in}{5.159444in}}%
\pgfusepath{clip}%
\pgfsetbuttcap%
\pgfsetroundjoin%
\pgfsetlinewidth{1.003750pt}%
\definecolor{currentstroke}{rgb}{0.827451,0.827451,0.827451}%
\pgfsetstrokecolor{currentstroke}%
\pgfsetstrokeopacity{0.800000}%
\pgfsetdash{}{0pt}%
\pgfpathmoveto{\pgfqpoint{13.279198in}{5.553845in}}%
\pgfpathcurveto{\pgfqpoint{13.290248in}{5.553845in}}{\pgfqpoint{13.300847in}{5.558236in}}{\pgfqpoint{13.308661in}{5.566049in}}%
\pgfpathcurveto{\pgfqpoint{13.316475in}{5.573863in}}{\pgfqpoint{13.320865in}{5.584462in}}{\pgfqpoint{13.320865in}{5.595512in}}%
\pgfpathcurveto{\pgfqpoint{13.320865in}{5.606562in}}{\pgfqpoint{13.316475in}{5.617161in}}{\pgfqpoint{13.308661in}{5.624975in}}%
\pgfpathcurveto{\pgfqpoint{13.300847in}{5.632788in}}{\pgfqpoint{13.290248in}{5.637179in}}{\pgfqpoint{13.279198in}{5.637179in}}%
\pgfpathcurveto{\pgfqpoint{13.268148in}{5.637179in}}{\pgfqpoint{13.257549in}{5.632788in}}{\pgfqpoint{13.249736in}{5.624975in}}%
\pgfpathcurveto{\pgfqpoint{13.241922in}{5.617161in}}{\pgfqpoint{13.237532in}{5.606562in}}{\pgfqpoint{13.237532in}{5.595512in}}%
\pgfpathcurveto{\pgfqpoint{13.237532in}{5.584462in}}{\pgfqpoint{13.241922in}{5.573863in}}{\pgfqpoint{13.249736in}{5.566049in}}%
\pgfpathcurveto{\pgfqpoint{13.257549in}{5.558236in}}{\pgfqpoint{13.268148in}{5.553845in}}{\pgfqpoint{13.279198in}{5.553845in}}%
\pgfpathlineto{\pgfqpoint{13.279198in}{5.553845in}}%
\pgfpathclose%
\pgfusepath{stroke}%
\end{pgfscope}%
\begin{pgfscope}%
\pgfpathrectangle{\pgfqpoint{7.512535in}{0.437222in}}{\pgfqpoint{6.275590in}{5.159444in}}%
\pgfusepath{clip}%
\pgfsetbuttcap%
\pgfsetroundjoin%
\pgfsetlinewidth{1.003750pt}%
\definecolor{currentstroke}{rgb}{0.827451,0.827451,0.827451}%
\pgfsetstrokecolor{currentstroke}%
\pgfsetstrokeopacity{0.800000}%
\pgfsetdash{}{0pt}%
\pgfpathmoveto{\pgfqpoint{10.486066in}{4.589271in}}%
\pgfpathcurveto{\pgfqpoint{10.497116in}{4.589271in}}{\pgfqpoint{10.507715in}{4.593662in}}{\pgfqpoint{10.515529in}{4.601475in}}%
\pgfpathcurveto{\pgfqpoint{10.523342in}{4.609289in}}{\pgfqpoint{10.527733in}{4.619888in}}{\pgfqpoint{10.527733in}{4.630938in}}%
\pgfpathcurveto{\pgfqpoint{10.527733in}{4.641988in}}{\pgfqpoint{10.523342in}{4.652587in}}{\pgfqpoint{10.515529in}{4.660401in}}%
\pgfpathcurveto{\pgfqpoint{10.507715in}{4.668214in}}{\pgfqpoint{10.497116in}{4.672605in}}{\pgfqpoint{10.486066in}{4.672605in}}%
\pgfpathcurveto{\pgfqpoint{10.475016in}{4.672605in}}{\pgfqpoint{10.464417in}{4.668214in}}{\pgfqpoint{10.456603in}{4.660401in}}%
\pgfpathcurveto{\pgfqpoint{10.448790in}{4.652587in}}{\pgfqpoint{10.444399in}{4.641988in}}{\pgfqpoint{10.444399in}{4.630938in}}%
\pgfpathcurveto{\pgfqpoint{10.444399in}{4.619888in}}{\pgfqpoint{10.448790in}{4.609289in}}{\pgfqpoint{10.456603in}{4.601475in}}%
\pgfpathcurveto{\pgfqpoint{10.464417in}{4.593662in}}{\pgfqpoint{10.475016in}{4.589271in}}{\pgfqpoint{10.486066in}{4.589271in}}%
\pgfpathlineto{\pgfqpoint{10.486066in}{4.589271in}}%
\pgfpathclose%
\pgfusepath{stroke}%
\end{pgfscope}%
\begin{pgfscope}%
\pgfpathrectangle{\pgfqpoint{7.512535in}{0.437222in}}{\pgfqpoint{6.275590in}{5.159444in}}%
\pgfusepath{clip}%
\pgfsetbuttcap%
\pgfsetroundjoin%
\pgfsetlinewidth{1.003750pt}%
\definecolor{currentstroke}{rgb}{0.827451,0.827451,0.827451}%
\pgfsetstrokecolor{currentstroke}%
\pgfsetstrokeopacity{0.800000}%
\pgfsetdash{}{0pt}%
\pgfpathmoveto{\pgfqpoint{13.419279in}{5.549860in}}%
\pgfpathcurveto{\pgfqpoint{13.430329in}{5.549860in}}{\pgfqpoint{13.440928in}{5.554250in}}{\pgfqpoint{13.448742in}{5.562064in}}%
\pgfpathcurveto{\pgfqpoint{13.456555in}{5.569877in}}{\pgfqpoint{13.460945in}{5.580476in}}{\pgfqpoint{13.460945in}{5.591526in}}%
\pgfpathcurveto{\pgfqpoint{13.460945in}{5.602576in}}{\pgfqpoint{13.456555in}{5.613175in}}{\pgfqpoint{13.448742in}{5.620989in}}%
\pgfpathcurveto{\pgfqpoint{13.440928in}{5.628803in}}{\pgfqpoint{13.430329in}{5.633193in}}{\pgfqpoint{13.419279in}{5.633193in}}%
\pgfpathcurveto{\pgfqpoint{13.408229in}{5.633193in}}{\pgfqpoint{13.397630in}{5.628803in}}{\pgfqpoint{13.389816in}{5.620989in}}%
\pgfpathcurveto{\pgfqpoint{13.382002in}{5.613175in}}{\pgfqpoint{13.377612in}{5.602576in}}{\pgfqpoint{13.377612in}{5.591526in}}%
\pgfpathcurveto{\pgfqpoint{13.377612in}{5.580476in}}{\pgfqpoint{13.382002in}{5.569877in}}{\pgfqpoint{13.389816in}{5.562064in}}%
\pgfpathcurveto{\pgfqpoint{13.397630in}{5.554250in}}{\pgfqpoint{13.408229in}{5.549860in}}{\pgfqpoint{13.419279in}{5.549860in}}%
\pgfpathlineto{\pgfqpoint{13.419279in}{5.549860in}}%
\pgfpathclose%
\pgfusepath{stroke}%
\end{pgfscope}%
\begin{pgfscope}%
\pgfpathrectangle{\pgfqpoint{7.512535in}{0.437222in}}{\pgfqpoint{6.275590in}{5.159444in}}%
\pgfusepath{clip}%
\pgfsetbuttcap%
\pgfsetroundjoin%
\pgfsetlinewidth{1.003750pt}%
\definecolor{currentstroke}{rgb}{0.827451,0.827451,0.827451}%
\pgfsetstrokecolor{currentstroke}%
\pgfsetstrokeopacity{0.800000}%
\pgfsetdash{}{0pt}%
\pgfpathmoveto{\pgfqpoint{10.073045in}{4.318404in}}%
\pgfpathcurveto{\pgfqpoint{10.084095in}{4.318404in}}{\pgfqpoint{10.094694in}{4.322795in}}{\pgfqpoint{10.102508in}{4.330608in}}%
\pgfpathcurveto{\pgfqpoint{10.110321in}{4.338422in}}{\pgfqpoint{10.114712in}{4.349021in}}{\pgfqpoint{10.114712in}{4.360071in}}%
\pgfpathcurveto{\pgfqpoint{10.114712in}{4.371121in}}{\pgfqpoint{10.110321in}{4.381720in}}{\pgfqpoint{10.102508in}{4.389534in}}%
\pgfpathcurveto{\pgfqpoint{10.094694in}{4.397348in}}{\pgfqpoint{10.084095in}{4.401738in}}{\pgfqpoint{10.073045in}{4.401738in}}%
\pgfpathcurveto{\pgfqpoint{10.061995in}{4.401738in}}{\pgfqpoint{10.051396in}{4.397348in}}{\pgfqpoint{10.043582in}{4.389534in}}%
\pgfpathcurveto{\pgfqpoint{10.035769in}{4.381720in}}{\pgfqpoint{10.031378in}{4.371121in}}{\pgfqpoint{10.031378in}{4.360071in}}%
\pgfpathcurveto{\pgfqpoint{10.031378in}{4.349021in}}{\pgfqpoint{10.035769in}{4.338422in}}{\pgfqpoint{10.043582in}{4.330608in}}%
\pgfpathcurveto{\pgfqpoint{10.051396in}{4.322795in}}{\pgfqpoint{10.061995in}{4.318404in}}{\pgfqpoint{10.073045in}{4.318404in}}%
\pgfpathlineto{\pgfqpoint{10.073045in}{4.318404in}}%
\pgfpathclose%
\pgfusepath{stroke}%
\end{pgfscope}%
\begin{pgfscope}%
\pgfpathrectangle{\pgfqpoint{7.512535in}{0.437222in}}{\pgfqpoint{6.275590in}{5.159444in}}%
\pgfusepath{clip}%
\pgfsetbuttcap%
\pgfsetroundjoin%
\pgfsetlinewidth{1.003750pt}%
\definecolor{currentstroke}{rgb}{0.827451,0.827451,0.827451}%
\pgfsetstrokecolor{currentstroke}%
\pgfsetstrokeopacity{0.800000}%
\pgfsetdash{}{0pt}%
\pgfpathmoveto{\pgfqpoint{12.233224in}{5.446777in}}%
\pgfpathcurveto{\pgfqpoint{12.244274in}{5.446777in}}{\pgfqpoint{12.254873in}{5.451168in}}{\pgfqpoint{12.262687in}{5.458981in}}%
\pgfpathcurveto{\pgfqpoint{12.270500in}{5.466795in}}{\pgfqpoint{12.274891in}{5.477394in}}{\pgfqpoint{12.274891in}{5.488444in}}%
\pgfpathcurveto{\pgfqpoint{12.274891in}{5.499494in}}{\pgfqpoint{12.270500in}{5.510093in}}{\pgfqpoint{12.262687in}{5.517907in}}%
\pgfpathcurveto{\pgfqpoint{12.254873in}{5.525720in}}{\pgfqpoint{12.244274in}{5.530111in}}{\pgfqpoint{12.233224in}{5.530111in}}%
\pgfpathcurveto{\pgfqpoint{12.222174in}{5.530111in}}{\pgfqpoint{12.211575in}{5.525720in}}{\pgfqpoint{12.203761in}{5.517907in}}%
\pgfpathcurveto{\pgfqpoint{12.195948in}{5.510093in}}{\pgfqpoint{12.191557in}{5.499494in}}{\pgfqpoint{12.191557in}{5.488444in}}%
\pgfpathcurveto{\pgfqpoint{12.191557in}{5.477394in}}{\pgfqpoint{12.195948in}{5.466795in}}{\pgfqpoint{12.203761in}{5.458981in}}%
\pgfpathcurveto{\pgfqpoint{12.211575in}{5.451168in}}{\pgfqpoint{12.222174in}{5.446777in}}{\pgfqpoint{12.233224in}{5.446777in}}%
\pgfpathlineto{\pgfqpoint{12.233224in}{5.446777in}}%
\pgfpathclose%
\pgfusepath{stroke}%
\end{pgfscope}%
\begin{pgfscope}%
\pgfpathrectangle{\pgfqpoint{7.512535in}{0.437222in}}{\pgfqpoint{6.275590in}{5.159444in}}%
\pgfusepath{clip}%
\pgfsetbuttcap%
\pgfsetroundjoin%
\pgfsetlinewidth{1.003750pt}%
\definecolor{currentstroke}{rgb}{0.827451,0.827451,0.827451}%
\pgfsetstrokecolor{currentstroke}%
\pgfsetstrokeopacity{0.800000}%
\pgfsetdash{}{0pt}%
\pgfpathmoveto{\pgfqpoint{9.130854in}{1.867691in}}%
\pgfpathcurveto{\pgfqpoint{9.141904in}{1.867691in}}{\pgfqpoint{9.152503in}{1.872081in}}{\pgfqpoint{9.160317in}{1.879895in}}%
\pgfpathcurveto{\pgfqpoint{9.168130in}{1.887709in}}{\pgfqpoint{9.172521in}{1.898308in}}{\pgfqpoint{9.172521in}{1.909358in}}%
\pgfpathcurveto{\pgfqpoint{9.172521in}{1.920408in}}{\pgfqpoint{9.168130in}{1.931007in}}{\pgfqpoint{9.160317in}{1.938821in}}%
\pgfpathcurveto{\pgfqpoint{9.152503in}{1.946634in}}{\pgfqpoint{9.141904in}{1.951024in}}{\pgfqpoint{9.130854in}{1.951024in}}%
\pgfpathcurveto{\pgfqpoint{9.119804in}{1.951024in}}{\pgfqpoint{9.109205in}{1.946634in}}{\pgfqpoint{9.101391in}{1.938821in}}%
\pgfpathcurveto{\pgfqpoint{9.093577in}{1.931007in}}{\pgfqpoint{9.089187in}{1.920408in}}{\pgfqpoint{9.089187in}{1.909358in}}%
\pgfpathcurveto{\pgfqpoint{9.089187in}{1.898308in}}{\pgfqpoint{9.093577in}{1.887709in}}{\pgfqpoint{9.101391in}{1.879895in}}%
\pgfpathcurveto{\pgfqpoint{9.109205in}{1.872081in}}{\pgfqpoint{9.119804in}{1.867691in}}{\pgfqpoint{9.130854in}{1.867691in}}%
\pgfpathlineto{\pgfqpoint{9.130854in}{1.867691in}}%
\pgfpathclose%
\pgfusepath{stroke}%
\end{pgfscope}%
\begin{pgfscope}%
\pgfpathrectangle{\pgfqpoint{7.512535in}{0.437222in}}{\pgfqpoint{6.275590in}{5.159444in}}%
\pgfusepath{clip}%
\pgfsetbuttcap%
\pgfsetroundjoin%
\pgfsetlinewidth{1.003750pt}%
\definecolor{currentstroke}{rgb}{0.827451,0.827451,0.827451}%
\pgfsetstrokecolor{currentstroke}%
\pgfsetstrokeopacity{0.800000}%
\pgfsetdash{}{0pt}%
\pgfpathmoveto{\pgfqpoint{13.151323in}{5.549860in}}%
\pgfpathcurveto{\pgfqpoint{13.162373in}{5.549860in}}{\pgfqpoint{13.172973in}{5.554250in}}{\pgfqpoint{13.180786in}{5.562064in}}%
\pgfpathcurveto{\pgfqpoint{13.188600in}{5.569877in}}{\pgfqpoint{13.192990in}{5.580476in}}{\pgfqpoint{13.192990in}{5.591526in}}%
\pgfpathcurveto{\pgfqpoint{13.192990in}{5.602576in}}{\pgfqpoint{13.188600in}{5.613175in}}{\pgfqpoint{13.180786in}{5.620989in}}%
\pgfpathcurveto{\pgfqpoint{13.172973in}{5.628803in}}{\pgfqpoint{13.162373in}{5.633193in}}{\pgfqpoint{13.151323in}{5.633193in}}%
\pgfpathcurveto{\pgfqpoint{13.140273in}{5.633193in}}{\pgfqpoint{13.129674in}{5.628803in}}{\pgfqpoint{13.121861in}{5.620989in}}%
\pgfpathcurveto{\pgfqpoint{13.114047in}{5.613175in}}{\pgfqpoint{13.109657in}{5.602576in}}{\pgfqpoint{13.109657in}{5.591526in}}%
\pgfpathcurveto{\pgfqpoint{13.109657in}{5.580476in}}{\pgfqpoint{13.114047in}{5.569877in}}{\pgfqpoint{13.121861in}{5.562064in}}%
\pgfpathcurveto{\pgfqpoint{13.129674in}{5.554250in}}{\pgfqpoint{13.140273in}{5.549860in}}{\pgfqpoint{13.151323in}{5.549860in}}%
\pgfpathlineto{\pgfqpoint{13.151323in}{5.549860in}}%
\pgfpathclose%
\pgfusepath{stroke}%
\end{pgfscope}%
\begin{pgfscope}%
\pgfpathrectangle{\pgfqpoint{7.512535in}{0.437222in}}{\pgfqpoint{6.275590in}{5.159444in}}%
\pgfusepath{clip}%
\pgfsetbuttcap%
\pgfsetroundjoin%
\pgfsetlinewidth{1.003750pt}%
\definecolor{currentstroke}{rgb}{0.827451,0.827451,0.827451}%
\pgfsetstrokecolor{currentstroke}%
\pgfsetstrokeopacity{0.800000}%
\pgfsetdash{}{0pt}%
\pgfpathmoveto{\pgfqpoint{9.807010in}{4.531380in}}%
\pgfpathcurveto{\pgfqpoint{9.818060in}{4.531380in}}{\pgfqpoint{9.828659in}{4.535770in}}{\pgfqpoint{9.836473in}{4.543584in}}%
\pgfpathcurveto{\pgfqpoint{9.844287in}{4.551397in}}{\pgfqpoint{9.848677in}{4.561996in}}{\pgfqpoint{9.848677in}{4.573046in}}%
\pgfpathcurveto{\pgfqpoint{9.848677in}{4.584096in}}{\pgfqpoint{9.844287in}{4.594695in}}{\pgfqpoint{9.836473in}{4.602509in}}%
\pgfpathcurveto{\pgfqpoint{9.828659in}{4.610323in}}{\pgfqpoint{9.818060in}{4.614713in}}{\pgfqpoint{9.807010in}{4.614713in}}%
\pgfpathcurveto{\pgfqpoint{9.795960in}{4.614713in}}{\pgfqpoint{9.785361in}{4.610323in}}{\pgfqpoint{9.777548in}{4.602509in}}%
\pgfpathcurveto{\pgfqpoint{9.769734in}{4.594695in}}{\pgfqpoint{9.765344in}{4.584096in}}{\pgfqpoint{9.765344in}{4.573046in}}%
\pgfpathcurveto{\pgfqpoint{9.765344in}{4.561996in}}{\pgfqpoint{9.769734in}{4.551397in}}{\pgfqpoint{9.777548in}{4.543584in}}%
\pgfpathcurveto{\pgfqpoint{9.785361in}{4.535770in}}{\pgfqpoint{9.795960in}{4.531380in}}{\pgfqpoint{9.807010in}{4.531380in}}%
\pgfpathlineto{\pgfqpoint{9.807010in}{4.531380in}}%
\pgfpathclose%
\pgfusepath{stroke}%
\end{pgfscope}%
\begin{pgfscope}%
\pgfpathrectangle{\pgfqpoint{7.512535in}{0.437222in}}{\pgfqpoint{6.275590in}{5.159444in}}%
\pgfusepath{clip}%
\pgfsetbuttcap%
\pgfsetroundjoin%
\pgfsetlinewidth{1.003750pt}%
\definecolor{currentstroke}{rgb}{0.827451,0.827451,0.827451}%
\pgfsetstrokecolor{currentstroke}%
\pgfsetstrokeopacity{0.800000}%
\pgfsetdash{}{0pt}%
\pgfpathmoveto{\pgfqpoint{13.425002in}{5.528743in}}%
\pgfpathcurveto{\pgfqpoint{13.436053in}{5.528743in}}{\pgfqpoint{13.446652in}{5.533133in}}{\pgfqpoint{13.454465in}{5.540947in}}%
\pgfpathcurveto{\pgfqpoint{13.462279in}{5.548760in}}{\pgfqpoint{13.466669in}{5.559359in}}{\pgfqpoint{13.466669in}{5.570410in}}%
\pgfpathcurveto{\pgfqpoint{13.466669in}{5.581460in}}{\pgfqpoint{13.462279in}{5.592059in}}{\pgfqpoint{13.454465in}{5.599872in}}%
\pgfpathcurveto{\pgfqpoint{13.446652in}{5.607686in}}{\pgfqpoint{13.436053in}{5.612076in}}{\pgfqpoint{13.425002in}{5.612076in}}%
\pgfpathcurveto{\pgfqpoint{13.413952in}{5.612076in}}{\pgfqpoint{13.403353in}{5.607686in}}{\pgfqpoint{13.395540in}{5.599872in}}%
\pgfpathcurveto{\pgfqpoint{13.387726in}{5.592059in}}{\pgfqpoint{13.383336in}{5.581460in}}{\pgfqpoint{13.383336in}{5.570410in}}%
\pgfpathcurveto{\pgfqpoint{13.383336in}{5.559359in}}{\pgfqpoint{13.387726in}{5.548760in}}{\pgfqpoint{13.395540in}{5.540947in}}%
\pgfpathcurveto{\pgfqpoint{13.403353in}{5.533133in}}{\pgfqpoint{13.413952in}{5.528743in}}{\pgfqpoint{13.425002in}{5.528743in}}%
\pgfpathlineto{\pgfqpoint{13.425002in}{5.528743in}}%
\pgfpathclose%
\pgfusepath{stroke}%
\end{pgfscope}%
\begin{pgfscope}%
\pgfpathrectangle{\pgfqpoint{7.512535in}{0.437222in}}{\pgfqpoint{6.275590in}{5.159444in}}%
\pgfusepath{clip}%
\pgfsetbuttcap%
\pgfsetroundjoin%
\pgfsetlinewidth{1.003750pt}%
\definecolor{currentstroke}{rgb}{0.827451,0.827451,0.827451}%
\pgfsetstrokecolor{currentstroke}%
\pgfsetstrokeopacity{0.800000}%
\pgfsetdash{}{0pt}%
\pgfpathmoveto{\pgfqpoint{12.817808in}{5.511813in}}%
\pgfpathcurveto{\pgfqpoint{12.828858in}{5.511813in}}{\pgfqpoint{12.839457in}{5.516204in}}{\pgfqpoint{12.847270in}{5.524017in}}%
\pgfpathcurveto{\pgfqpoint{12.855084in}{5.531831in}}{\pgfqpoint{12.859474in}{5.542430in}}{\pgfqpoint{12.859474in}{5.553480in}}%
\pgfpathcurveto{\pgfqpoint{12.859474in}{5.564530in}}{\pgfqpoint{12.855084in}{5.575129in}}{\pgfqpoint{12.847270in}{5.582943in}}%
\pgfpathcurveto{\pgfqpoint{12.839457in}{5.590756in}}{\pgfqpoint{12.828858in}{5.595147in}}{\pgfqpoint{12.817808in}{5.595147in}}%
\pgfpathcurveto{\pgfqpoint{12.806757in}{5.595147in}}{\pgfqpoint{12.796158in}{5.590756in}}{\pgfqpoint{12.788345in}{5.582943in}}%
\pgfpathcurveto{\pgfqpoint{12.780531in}{5.575129in}}{\pgfqpoint{12.776141in}{5.564530in}}{\pgfqpoint{12.776141in}{5.553480in}}%
\pgfpathcurveto{\pgfqpoint{12.776141in}{5.542430in}}{\pgfqpoint{12.780531in}{5.531831in}}{\pgfqpoint{12.788345in}{5.524017in}}%
\pgfpathcurveto{\pgfqpoint{12.796158in}{5.516204in}}{\pgfqpoint{12.806757in}{5.511813in}}{\pgfqpoint{12.817808in}{5.511813in}}%
\pgfpathlineto{\pgfqpoint{12.817808in}{5.511813in}}%
\pgfpathclose%
\pgfusepath{stroke}%
\end{pgfscope}%
\begin{pgfscope}%
\pgfpathrectangle{\pgfqpoint{7.512535in}{0.437222in}}{\pgfqpoint{6.275590in}{5.159444in}}%
\pgfusepath{clip}%
\pgfsetbuttcap%
\pgfsetroundjoin%
\pgfsetlinewidth{1.003750pt}%
\definecolor{currentstroke}{rgb}{0.827451,0.827451,0.827451}%
\pgfsetstrokecolor{currentstroke}%
\pgfsetstrokeopacity{0.800000}%
\pgfsetdash{}{0pt}%
\pgfpathmoveto{\pgfqpoint{9.319291in}{1.927959in}}%
\pgfpathcurveto{\pgfqpoint{9.330341in}{1.927959in}}{\pgfqpoint{9.340940in}{1.932349in}}{\pgfqpoint{9.348754in}{1.940163in}}%
\pgfpathcurveto{\pgfqpoint{9.356567in}{1.947976in}}{\pgfqpoint{9.360957in}{1.958575in}}{\pgfqpoint{9.360957in}{1.969626in}}%
\pgfpathcurveto{\pgfqpoint{9.360957in}{1.980676in}}{\pgfqpoint{9.356567in}{1.991275in}}{\pgfqpoint{9.348754in}{1.999088in}}%
\pgfpathcurveto{\pgfqpoint{9.340940in}{2.006902in}}{\pgfqpoint{9.330341in}{2.011292in}}{\pgfqpoint{9.319291in}{2.011292in}}%
\pgfpathcurveto{\pgfqpoint{9.308241in}{2.011292in}}{\pgfqpoint{9.297642in}{2.006902in}}{\pgfqpoint{9.289828in}{1.999088in}}%
\pgfpathcurveto{\pgfqpoint{9.282014in}{1.991275in}}{\pgfqpoint{9.277624in}{1.980676in}}{\pgfqpoint{9.277624in}{1.969626in}}%
\pgfpathcurveto{\pgfqpoint{9.277624in}{1.958575in}}{\pgfqpoint{9.282014in}{1.947976in}}{\pgfqpoint{9.289828in}{1.940163in}}%
\pgfpathcurveto{\pgfqpoint{9.297642in}{1.932349in}}{\pgfqpoint{9.308241in}{1.927959in}}{\pgfqpoint{9.319291in}{1.927959in}}%
\pgfpathlineto{\pgfqpoint{9.319291in}{1.927959in}}%
\pgfpathclose%
\pgfusepath{stroke}%
\end{pgfscope}%
\begin{pgfscope}%
\pgfpathrectangle{\pgfqpoint{7.512535in}{0.437222in}}{\pgfqpoint{6.275590in}{5.159444in}}%
\pgfusepath{clip}%
\pgfsetbuttcap%
\pgfsetroundjoin%
\pgfsetlinewidth{1.003750pt}%
\definecolor{currentstroke}{rgb}{0.827451,0.827451,0.827451}%
\pgfsetstrokecolor{currentstroke}%
\pgfsetstrokeopacity{0.800000}%
\pgfsetdash{}{0pt}%
\pgfpathmoveto{\pgfqpoint{11.629363in}{5.304399in}}%
\pgfpathcurveto{\pgfqpoint{11.640413in}{5.304399in}}{\pgfqpoint{11.651012in}{5.308790in}}{\pgfqpoint{11.658826in}{5.316603in}}%
\pgfpathcurveto{\pgfqpoint{11.666639in}{5.324417in}}{\pgfqpoint{11.671030in}{5.335016in}}{\pgfqpoint{11.671030in}{5.346066in}}%
\pgfpathcurveto{\pgfqpoint{11.671030in}{5.357116in}}{\pgfqpoint{11.666639in}{5.367715in}}{\pgfqpoint{11.658826in}{5.375529in}}%
\pgfpathcurveto{\pgfqpoint{11.651012in}{5.383342in}}{\pgfqpoint{11.640413in}{5.387733in}}{\pgfqpoint{11.629363in}{5.387733in}}%
\pgfpathcurveto{\pgfqpoint{11.618313in}{5.387733in}}{\pgfqpoint{11.607714in}{5.383342in}}{\pgfqpoint{11.599900in}{5.375529in}}%
\pgfpathcurveto{\pgfqpoint{11.592087in}{5.367715in}}{\pgfqpoint{11.587696in}{5.357116in}}{\pgfqpoint{11.587696in}{5.346066in}}%
\pgfpathcurveto{\pgfqpoint{11.587696in}{5.335016in}}{\pgfqpoint{11.592087in}{5.324417in}}{\pgfqpoint{11.599900in}{5.316603in}}%
\pgfpathcurveto{\pgfqpoint{11.607714in}{5.308790in}}{\pgfqpoint{11.618313in}{5.304399in}}{\pgfqpoint{11.629363in}{5.304399in}}%
\pgfpathlineto{\pgfqpoint{11.629363in}{5.304399in}}%
\pgfpathclose%
\pgfusepath{stroke}%
\end{pgfscope}%
\begin{pgfscope}%
\pgfpathrectangle{\pgfqpoint{7.512535in}{0.437222in}}{\pgfqpoint{6.275590in}{5.159444in}}%
\pgfusepath{clip}%
\pgfsetbuttcap%
\pgfsetroundjoin%
\pgfsetlinewidth{1.003750pt}%
\definecolor{currentstroke}{rgb}{0.827451,0.827451,0.827451}%
\pgfsetstrokecolor{currentstroke}%
\pgfsetstrokeopacity{0.800000}%
\pgfsetdash{}{0pt}%
\pgfpathmoveto{\pgfqpoint{9.726466in}{3.345111in}}%
\pgfpathcurveto{\pgfqpoint{9.737516in}{3.345111in}}{\pgfqpoint{9.748115in}{3.349501in}}{\pgfqpoint{9.755929in}{3.357315in}}%
\pgfpathcurveto{\pgfqpoint{9.763743in}{3.365128in}}{\pgfqpoint{9.768133in}{3.375727in}}{\pgfqpoint{9.768133in}{3.386778in}}%
\pgfpathcurveto{\pgfqpoint{9.768133in}{3.397828in}}{\pgfqpoint{9.763743in}{3.408427in}}{\pgfqpoint{9.755929in}{3.416240in}}%
\pgfpathcurveto{\pgfqpoint{9.748115in}{3.424054in}}{\pgfqpoint{9.737516in}{3.428444in}}{\pgfqpoint{9.726466in}{3.428444in}}%
\pgfpathcurveto{\pgfqpoint{9.715416in}{3.428444in}}{\pgfqpoint{9.704817in}{3.424054in}}{\pgfqpoint{9.697004in}{3.416240in}}%
\pgfpathcurveto{\pgfqpoint{9.689190in}{3.408427in}}{\pgfqpoint{9.684800in}{3.397828in}}{\pgfqpoint{9.684800in}{3.386778in}}%
\pgfpathcurveto{\pgfqpoint{9.684800in}{3.375727in}}{\pgfqpoint{9.689190in}{3.365128in}}{\pgfqpoint{9.697004in}{3.357315in}}%
\pgfpathcurveto{\pgfqpoint{9.704817in}{3.349501in}}{\pgfqpoint{9.715416in}{3.345111in}}{\pgfqpoint{9.726466in}{3.345111in}}%
\pgfpathlineto{\pgfqpoint{9.726466in}{3.345111in}}%
\pgfpathclose%
\pgfusepath{stroke}%
\end{pgfscope}%
\begin{pgfscope}%
\pgfpathrectangle{\pgfqpoint{7.512535in}{0.437222in}}{\pgfqpoint{6.275590in}{5.159444in}}%
\pgfusepath{clip}%
\pgfsetbuttcap%
\pgfsetroundjoin%
\pgfsetlinewidth{1.003750pt}%
\definecolor{currentstroke}{rgb}{0.827451,0.827451,0.827451}%
\pgfsetstrokecolor{currentstroke}%
\pgfsetstrokeopacity{0.800000}%
\pgfsetdash{}{0pt}%
\pgfpathmoveto{\pgfqpoint{7.973923in}{0.766518in}}%
\pgfpathcurveto{\pgfqpoint{7.984973in}{0.766518in}}{\pgfqpoint{7.995572in}{0.770908in}}{\pgfqpoint{8.003385in}{0.778722in}}%
\pgfpathcurveto{\pgfqpoint{8.011199in}{0.786536in}}{\pgfqpoint{8.015589in}{0.797135in}}{\pgfqpoint{8.015589in}{0.808185in}}%
\pgfpathcurveto{\pgfqpoint{8.015589in}{0.819235in}}{\pgfqpoint{8.011199in}{0.829834in}}{\pgfqpoint{8.003385in}{0.837648in}}%
\pgfpathcurveto{\pgfqpoint{7.995572in}{0.845461in}}{\pgfqpoint{7.984973in}{0.849851in}}{\pgfqpoint{7.973923in}{0.849851in}}%
\pgfpathcurveto{\pgfqpoint{7.962873in}{0.849851in}}{\pgfqpoint{7.952274in}{0.845461in}}{\pgfqpoint{7.944460in}{0.837648in}}%
\pgfpathcurveto{\pgfqpoint{7.936646in}{0.829834in}}{\pgfqpoint{7.932256in}{0.819235in}}{\pgfqpoint{7.932256in}{0.808185in}}%
\pgfpathcurveto{\pgfqpoint{7.932256in}{0.797135in}}{\pgfqpoint{7.936646in}{0.786536in}}{\pgfqpoint{7.944460in}{0.778722in}}%
\pgfpathcurveto{\pgfqpoint{7.952274in}{0.770908in}}{\pgfqpoint{7.962873in}{0.766518in}}{\pgfqpoint{7.973923in}{0.766518in}}%
\pgfpathlineto{\pgfqpoint{7.973923in}{0.766518in}}%
\pgfpathclose%
\pgfusepath{stroke}%
\end{pgfscope}%
\begin{pgfscope}%
\pgfpathrectangle{\pgfqpoint{7.512535in}{0.437222in}}{\pgfqpoint{6.275590in}{5.159444in}}%
\pgfusepath{clip}%
\pgfsetbuttcap%
\pgfsetroundjoin%
\pgfsetlinewidth{1.003750pt}%
\definecolor{currentstroke}{rgb}{0.827451,0.827451,0.827451}%
\pgfsetstrokecolor{currentstroke}%
\pgfsetstrokeopacity{0.800000}%
\pgfsetdash{}{0pt}%
\pgfpathmoveto{\pgfqpoint{8.983729in}{3.220234in}}%
\pgfpathcurveto{\pgfqpoint{8.994779in}{3.220234in}}{\pgfqpoint{9.005378in}{3.224624in}}{\pgfqpoint{9.013191in}{3.232438in}}%
\pgfpathcurveto{\pgfqpoint{9.021005in}{3.240251in}}{\pgfqpoint{9.025395in}{3.250850in}}{\pgfqpoint{9.025395in}{3.261901in}}%
\pgfpathcurveto{\pgfqpoint{9.025395in}{3.272951in}}{\pgfqpoint{9.021005in}{3.283550in}}{\pgfqpoint{9.013191in}{3.291363in}}%
\pgfpathcurveto{\pgfqpoint{9.005378in}{3.299177in}}{\pgfqpoint{8.994779in}{3.303567in}}{\pgfqpoint{8.983729in}{3.303567in}}%
\pgfpathcurveto{\pgfqpoint{8.972679in}{3.303567in}}{\pgfqpoint{8.962079in}{3.299177in}}{\pgfqpoint{8.954266in}{3.291363in}}%
\pgfpathcurveto{\pgfqpoint{8.946452in}{3.283550in}}{\pgfqpoint{8.942062in}{3.272951in}}{\pgfqpoint{8.942062in}{3.261901in}}%
\pgfpathcurveto{\pgfqpoint{8.942062in}{3.250850in}}{\pgfqpoint{8.946452in}{3.240251in}}{\pgfqpoint{8.954266in}{3.232438in}}%
\pgfpathcurveto{\pgfqpoint{8.962079in}{3.224624in}}{\pgfqpoint{8.972679in}{3.220234in}}{\pgfqpoint{8.983729in}{3.220234in}}%
\pgfpathlineto{\pgfqpoint{8.983729in}{3.220234in}}%
\pgfpathclose%
\pgfusepath{stroke}%
\end{pgfscope}%
\begin{pgfscope}%
\pgfpathrectangle{\pgfqpoint{7.512535in}{0.437222in}}{\pgfqpoint{6.275590in}{5.159444in}}%
\pgfusepath{clip}%
\pgfsetbuttcap%
\pgfsetroundjoin%
\pgfsetlinewidth{1.003750pt}%
\definecolor{currentstroke}{rgb}{0.827451,0.827451,0.827451}%
\pgfsetstrokecolor{currentstroke}%
\pgfsetstrokeopacity{0.800000}%
\pgfsetdash{}{0pt}%
\pgfpathmoveto{\pgfqpoint{13.620000in}{5.548368in}}%
\pgfpathcurveto{\pgfqpoint{13.631051in}{5.548368in}}{\pgfqpoint{13.641650in}{5.552759in}}{\pgfqpoint{13.649463in}{5.560572in}}%
\pgfpathcurveto{\pgfqpoint{13.657277in}{5.568386in}}{\pgfqpoint{13.661667in}{5.578985in}}{\pgfqpoint{13.661667in}{5.590035in}}%
\pgfpathcurveto{\pgfqpoint{13.661667in}{5.601085in}}{\pgfqpoint{13.657277in}{5.611684in}}{\pgfqpoint{13.649463in}{5.619498in}}%
\pgfpathcurveto{\pgfqpoint{13.641650in}{5.627311in}}{\pgfqpoint{13.631051in}{5.631702in}}{\pgfqpoint{13.620000in}{5.631702in}}%
\pgfpathcurveto{\pgfqpoint{13.608950in}{5.631702in}}{\pgfqpoint{13.598351in}{5.627311in}}{\pgfqpoint{13.590538in}{5.619498in}}%
\pgfpathcurveto{\pgfqpoint{13.582724in}{5.611684in}}{\pgfqpoint{13.578334in}{5.601085in}}{\pgfqpoint{13.578334in}{5.590035in}}%
\pgfpathcurveto{\pgfqpoint{13.578334in}{5.578985in}}{\pgfqpoint{13.582724in}{5.568386in}}{\pgfqpoint{13.590538in}{5.560572in}}%
\pgfpathcurveto{\pgfqpoint{13.598351in}{5.552759in}}{\pgfqpoint{13.608950in}{5.548368in}}{\pgfqpoint{13.620000in}{5.548368in}}%
\pgfpathlineto{\pgfqpoint{13.620000in}{5.548368in}}%
\pgfpathclose%
\pgfusepath{stroke}%
\end{pgfscope}%
\begin{pgfscope}%
\pgfpathrectangle{\pgfqpoint{7.512535in}{0.437222in}}{\pgfqpoint{6.275590in}{5.159444in}}%
\pgfusepath{clip}%
\pgfsetbuttcap%
\pgfsetroundjoin%
\pgfsetlinewidth{1.003750pt}%
\definecolor{currentstroke}{rgb}{0.827451,0.827451,0.827451}%
\pgfsetstrokecolor{currentstroke}%
\pgfsetstrokeopacity{0.800000}%
\pgfsetdash{}{0pt}%
\pgfpathmoveto{\pgfqpoint{12.384977in}{5.553091in}}%
\pgfpathcurveto{\pgfqpoint{12.396028in}{5.553091in}}{\pgfqpoint{12.406627in}{5.557481in}}{\pgfqpoint{12.414440in}{5.565295in}}%
\pgfpathcurveto{\pgfqpoint{12.422254in}{5.573108in}}{\pgfqpoint{12.426644in}{5.583707in}}{\pgfqpoint{12.426644in}{5.594757in}}%
\pgfpathcurveto{\pgfqpoint{12.426644in}{5.605808in}}{\pgfqpoint{12.422254in}{5.616407in}}{\pgfqpoint{12.414440in}{5.624220in}}%
\pgfpathcurveto{\pgfqpoint{12.406627in}{5.632034in}}{\pgfqpoint{12.396028in}{5.636424in}}{\pgfqpoint{12.384977in}{5.636424in}}%
\pgfpathcurveto{\pgfqpoint{12.373927in}{5.636424in}}{\pgfqpoint{12.363328in}{5.632034in}}{\pgfqpoint{12.355515in}{5.624220in}}%
\pgfpathcurveto{\pgfqpoint{12.347701in}{5.616407in}}{\pgfqpoint{12.343311in}{5.605808in}}{\pgfqpoint{12.343311in}{5.594757in}}%
\pgfpathcurveto{\pgfqpoint{12.343311in}{5.583707in}}{\pgfqpoint{12.347701in}{5.573108in}}{\pgfqpoint{12.355515in}{5.565295in}}%
\pgfpathcurveto{\pgfqpoint{12.363328in}{5.557481in}}{\pgfqpoint{12.373927in}{5.553091in}}{\pgfqpoint{12.384977in}{5.553091in}}%
\pgfpathlineto{\pgfqpoint{12.384977in}{5.553091in}}%
\pgfpathclose%
\pgfusepath{stroke}%
\end{pgfscope}%
\begin{pgfscope}%
\pgfpathrectangle{\pgfqpoint{7.512535in}{0.437222in}}{\pgfqpoint{6.275590in}{5.159444in}}%
\pgfusepath{clip}%
\pgfsetbuttcap%
\pgfsetroundjoin%
\pgfsetlinewidth{1.003750pt}%
\definecolor{currentstroke}{rgb}{0.827451,0.827451,0.827451}%
\pgfsetstrokecolor{currentstroke}%
\pgfsetstrokeopacity{0.800000}%
\pgfsetdash{}{0pt}%
\pgfpathmoveto{\pgfqpoint{11.415124in}{5.455549in}}%
\pgfpathcurveto{\pgfqpoint{11.426174in}{5.455549in}}{\pgfqpoint{11.436773in}{5.459939in}}{\pgfqpoint{11.444587in}{5.467753in}}%
\pgfpathcurveto{\pgfqpoint{11.452401in}{5.475566in}}{\pgfqpoint{11.456791in}{5.486165in}}{\pgfqpoint{11.456791in}{5.497215in}}%
\pgfpathcurveto{\pgfqpoint{11.456791in}{5.508266in}}{\pgfqpoint{11.452401in}{5.518865in}}{\pgfqpoint{11.444587in}{5.526678in}}%
\pgfpathcurveto{\pgfqpoint{11.436773in}{5.534492in}}{\pgfqpoint{11.426174in}{5.538882in}}{\pgfqpoint{11.415124in}{5.538882in}}%
\pgfpathcurveto{\pgfqpoint{11.404074in}{5.538882in}}{\pgfqpoint{11.393475in}{5.534492in}}{\pgfqpoint{11.385661in}{5.526678in}}%
\pgfpathcurveto{\pgfqpoint{11.377848in}{5.518865in}}{\pgfqpoint{11.373458in}{5.508266in}}{\pgfqpoint{11.373458in}{5.497215in}}%
\pgfpathcurveto{\pgfqpoint{11.373458in}{5.486165in}}{\pgfqpoint{11.377848in}{5.475566in}}{\pgfqpoint{11.385661in}{5.467753in}}%
\pgfpathcurveto{\pgfqpoint{11.393475in}{5.459939in}}{\pgfqpoint{11.404074in}{5.455549in}}{\pgfqpoint{11.415124in}{5.455549in}}%
\pgfpathlineto{\pgfqpoint{11.415124in}{5.455549in}}%
\pgfpathclose%
\pgfusepath{stroke}%
\end{pgfscope}%
\begin{pgfscope}%
\pgfpathrectangle{\pgfqpoint{7.512535in}{0.437222in}}{\pgfqpoint{6.275590in}{5.159444in}}%
\pgfusepath{clip}%
\pgfsetbuttcap%
\pgfsetroundjoin%
\pgfsetlinewidth{1.003750pt}%
\definecolor{currentstroke}{rgb}{0.827451,0.827451,0.827451}%
\pgfsetstrokecolor{currentstroke}%
\pgfsetstrokeopacity{0.800000}%
\pgfsetdash{}{0pt}%
\pgfpathmoveto{\pgfqpoint{8.290348in}{1.426234in}}%
\pgfpathcurveto{\pgfqpoint{8.301398in}{1.426234in}}{\pgfqpoint{8.311997in}{1.430624in}}{\pgfqpoint{8.319811in}{1.438438in}}%
\pgfpathcurveto{\pgfqpoint{8.327625in}{1.446251in}}{\pgfqpoint{8.332015in}{1.456850in}}{\pgfqpoint{8.332015in}{1.467900in}}%
\pgfpathcurveto{\pgfqpoint{8.332015in}{1.478951in}}{\pgfqpoint{8.327625in}{1.489550in}}{\pgfqpoint{8.319811in}{1.497363in}}%
\pgfpathcurveto{\pgfqpoint{8.311997in}{1.505177in}}{\pgfqpoint{8.301398in}{1.509567in}}{\pgfqpoint{8.290348in}{1.509567in}}%
\pgfpathcurveto{\pgfqpoint{8.279298in}{1.509567in}}{\pgfqpoint{8.268699in}{1.505177in}}{\pgfqpoint{8.260886in}{1.497363in}}%
\pgfpathcurveto{\pgfqpoint{8.253072in}{1.489550in}}{\pgfqpoint{8.248682in}{1.478951in}}{\pgfqpoint{8.248682in}{1.467900in}}%
\pgfpathcurveto{\pgfqpoint{8.248682in}{1.456850in}}{\pgfqpoint{8.253072in}{1.446251in}}{\pgfqpoint{8.260886in}{1.438438in}}%
\pgfpathcurveto{\pgfqpoint{8.268699in}{1.430624in}}{\pgfqpoint{8.279298in}{1.426234in}}{\pgfqpoint{8.290348in}{1.426234in}}%
\pgfpathlineto{\pgfqpoint{8.290348in}{1.426234in}}%
\pgfpathclose%
\pgfusepath{stroke}%
\end{pgfscope}%
\begin{pgfscope}%
\pgfpathrectangle{\pgfqpoint{7.512535in}{0.437222in}}{\pgfqpoint{6.275590in}{5.159444in}}%
\pgfusepath{clip}%
\pgfsetbuttcap%
\pgfsetroundjoin%
\pgfsetlinewidth{1.003750pt}%
\definecolor{currentstroke}{rgb}{0.827451,0.827451,0.827451}%
\pgfsetstrokecolor{currentstroke}%
\pgfsetstrokeopacity{0.800000}%
\pgfsetdash{}{0pt}%
\pgfpathmoveto{\pgfqpoint{12.878574in}{5.554670in}}%
\pgfpathcurveto{\pgfqpoint{12.889624in}{5.554670in}}{\pgfqpoint{12.900223in}{5.559060in}}{\pgfqpoint{12.908037in}{5.566874in}}%
\pgfpathcurveto{\pgfqpoint{12.915851in}{5.574687in}}{\pgfqpoint{12.920241in}{5.585286in}}{\pgfqpoint{12.920241in}{5.596337in}}%
\pgfpathcurveto{\pgfqpoint{12.920241in}{5.607387in}}{\pgfqpoint{12.915851in}{5.617986in}}{\pgfqpoint{12.908037in}{5.625799in}}%
\pgfpathcurveto{\pgfqpoint{12.900223in}{5.633613in}}{\pgfqpoint{12.889624in}{5.638003in}}{\pgfqpoint{12.878574in}{5.638003in}}%
\pgfpathcurveto{\pgfqpoint{12.867524in}{5.638003in}}{\pgfqpoint{12.856925in}{5.633613in}}{\pgfqpoint{12.849111in}{5.625799in}}%
\pgfpathcurveto{\pgfqpoint{12.841298in}{5.617986in}}{\pgfqpoint{12.836907in}{5.607387in}}{\pgfqpoint{12.836907in}{5.596337in}}%
\pgfpathcurveto{\pgfqpoint{12.836907in}{5.585286in}}{\pgfqpoint{12.841298in}{5.574687in}}{\pgfqpoint{12.849111in}{5.566874in}}%
\pgfpathcurveto{\pgfqpoint{12.856925in}{5.559060in}}{\pgfqpoint{12.867524in}{5.554670in}}{\pgfqpoint{12.878574in}{5.554670in}}%
\pgfpathlineto{\pgfqpoint{12.878574in}{5.554670in}}%
\pgfpathclose%
\pgfusepath{stroke}%
\end{pgfscope}%
\begin{pgfscope}%
\pgfpathrectangle{\pgfqpoint{7.512535in}{0.437222in}}{\pgfqpoint{6.275590in}{5.159444in}}%
\pgfusepath{clip}%
\pgfsetbuttcap%
\pgfsetroundjoin%
\pgfsetlinewidth{1.003750pt}%
\definecolor{currentstroke}{rgb}{0.827451,0.827451,0.827451}%
\pgfsetstrokecolor{currentstroke}%
\pgfsetstrokeopacity{0.800000}%
\pgfsetdash{}{0pt}%
\pgfpathmoveto{\pgfqpoint{9.880248in}{3.946138in}}%
\pgfpathcurveto{\pgfqpoint{9.891298in}{3.946138in}}{\pgfqpoint{9.901897in}{3.950528in}}{\pgfqpoint{9.909710in}{3.958342in}}%
\pgfpathcurveto{\pgfqpoint{9.917524in}{3.966156in}}{\pgfqpoint{9.921914in}{3.976755in}}{\pgfqpoint{9.921914in}{3.987805in}}%
\pgfpathcurveto{\pgfqpoint{9.921914in}{3.998855in}}{\pgfqpoint{9.917524in}{4.009454in}}{\pgfqpoint{9.909710in}{4.017268in}}%
\pgfpathcurveto{\pgfqpoint{9.901897in}{4.025081in}}{\pgfqpoint{9.891298in}{4.029471in}}{\pgfqpoint{9.880248in}{4.029471in}}%
\pgfpathcurveto{\pgfqpoint{9.869198in}{4.029471in}}{\pgfqpoint{9.858599in}{4.025081in}}{\pgfqpoint{9.850785in}{4.017268in}}%
\pgfpathcurveto{\pgfqpoint{9.842971in}{4.009454in}}{\pgfqpoint{9.838581in}{3.998855in}}{\pgfqpoint{9.838581in}{3.987805in}}%
\pgfpathcurveto{\pgfqpoint{9.838581in}{3.976755in}}{\pgfqpoint{9.842971in}{3.966156in}}{\pgfqpoint{9.850785in}{3.958342in}}%
\pgfpathcurveto{\pgfqpoint{9.858599in}{3.950528in}}{\pgfqpoint{9.869198in}{3.946138in}}{\pgfqpoint{9.880248in}{3.946138in}}%
\pgfpathlineto{\pgfqpoint{9.880248in}{3.946138in}}%
\pgfpathclose%
\pgfusepath{stroke}%
\end{pgfscope}%
\begin{pgfscope}%
\pgfpathrectangle{\pgfqpoint{7.512535in}{0.437222in}}{\pgfqpoint{6.275590in}{5.159444in}}%
\pgfusepath{clip}%
\pgfsetbuttcap%
\pgfsetroundjoin%
\pgfsetlinewidth{1.003750pt}%
\definecolor{currentstroke}{rgb}{0.827451,0.827451,0.827451}%
\pgfsetstrokecolor{currentstroke}%
\pgfsetstrokeopacity{0.800000}%
\pgfsetdash{}{0pt}%
\pgfpathmoveto{\pgfqpoint{8.316930in}{1.743970in}}%
\pgfpathcurveto{\pgfqpoint{8.327980in}{1.743970in}}{\pgfqpoint{8.338579in}{1.748360in}}{\pgfqpoint{8.346393in}{1.756174in}}%
\pgfpathcurveto{\pgfqpoint{8.354206in}{1.763988in}}{\pgfqpoint{8.358596in}{1.774587in}}{\pgfqpoint{8.358596in}{1.785637in}}%
\pgfpathcurveto{\pgfqpoint{8.358596in}{1.796687in}}{\pgfqpoint{8.354206in}{1.807286in}}{\pgfqpoint{8.346393in}{1.815100in}}%
\pgfpathcurveto{\pgfqpoint{8.338579in}{1.822913in}}{\pgfqpoint{8.327980in}{1.827303in}}{\pgfqpoint{8.316930in}{1.827303in}}%
\pgfpathcurveto{\pgfqpoint{8.305880in}{1.827303in}}{\pgfqpoint{8.295281in}{1.822913in}}{\pgfqpoint{8.287467in}{1.815100in}}%
\pgfpathcurveto{\pgfqpoint{8.279653in}{1.807286in}}{\pgfqpoint{8.275263in}{1.796687in}}{\pgfqpoint{8.275263in}{1.785637in}}%
\pgfpathcurveto{\pgfqpoint{8.275263in}{1.774587in}}{\pgfqpoint{8.279653in}{1.763988in}}{\pgfqpoint{8.287467in}{1.756174in}}%
\pgfpathcurveto{\pgfqpoint{8.295281in}{1.748360in}}{\pgfqpoint{8.305880in}{1.743970in}}{\pgfqpoint{8.316930in}{1.743970in}}%
\pgfpathlineto{\pgfqpoint{8.316930in}{1.743970in}}%
\pgfpathclose%
\pgfusepath{stroke}%
\end{pgfscope}%
\begin{pgfscope}%
\pgfpathrectangle{\pgfqpoint{7.512535in}{0.437222in}}{\pgfqpoint{6.275590in}{5.159444in}}%
\pgfusepath{clip}%
\pgfsetbuttcap%
\pgfsetroundjoin%
\pgfsetlinewidth{1.003750pt}%
\definecolor{currentstroke}{rgb}{0.827451,0.827451,0.827451}%
\pgfsetstrokecolor{currentstroke}%
\pgfsetstrokeopacity{0.800000}%
\pgfsetdash{}{0pt}%
\pgfpathmoveto{\pgfqpoint{10.943039in}{5.378689in}}%
\pgfpathcurveto{\pgfqpoint{10.954089in}{5.378689in}}{\pgfqpoint{10.964688in}{5.383079in}}{\pgfqpoint{10.972502in}{5.390893in}}%
\pgfpathcurveto{\pgfqpoint{10.980315in}{5.398707in}}{\pgfqpoint{10.984706in}{5.409306in}}{\pgfqpoint{10.984706in}{5.420356in}}%
\pgfpathcurveto{\pgfqpoint{10.984706in}{5.431406in}}{\pgfqpoint{10.980315in}{5.442005in}}{\pgfqpoint{10.972502in}{5.449819in}}%
\pgfpathcurveto{\pgfqpoint{10.964688in}{5.457632in}}{\pgfqpoint{10.954089in}{5.462023in}}{\pgfqpoint{10.943039in}{5.462023in}}%
\pgfpathcurveto{\pgfqpoint{10.931989in}{5.462023in}}{\pgfqpoint{10.921390in}{5.457632in}}{\pgfqpoint{10.913576in}{5.449819in}}%
\pgfpathcurveto{\pgfqpoint{10.905763in}{5.442005in}}{\pgfqpoint{10.901372in}{5.431406in}}{\pgfqpoint{10.901372in}{5.420356in}}%
\pgfpathcurveto{\pgfqpoint{10.901372in}{5.409306in}}{\pgfqpoint{10.905763in}{5.398707in}}{\pgfqpoint{10.913576in}{5.390893in}}%
\pgfpathcurveto{\pgfqpoint{10.921390in}{5.383079in}}{\pgfqpoint{10.931989in}{5.378689in}}{\pgfqpoint{10.943039in}{5.378689in}}%
\pgfpathlineto{\pgfqpoint{10.943039in}{5.378689in}}%
\pgfpathclose%
\pgfusepath{stroke}%
\end{pgfscope}%
\begin{pgfscope}%
\pgfpathrectangle{\pgfqpoint{7.512535in}{0.437222in}}{\pgfqpoint{6.275590in}{5.159444in}}%
\pgfusepath{clip}%
\pgfsetbuttcap%
\pgfsetroundjoin%
\pgfsetlinewidth{1.003750pt}%
\definecolor{currentstroke}{rgb}{0.827451,0.827451,0.827451}%
\pgfsetstrokecolor{currentstroke}%
\pgfsetstrokeopacity{0.800000}%
\pgfsetdash{}{0pt}%
\pgfpathmoveto{\pgfqpoint{10.088484in}{4.444954in}}%
\pgfpathcurveto{\pgfqpoint{10.099535in}{4.444954in}}{\pgfqpoint{10.110134in}{4.449344in}}{\pgfqpoint{10.117947in}{4.457158in}}%
\pgfpathcurveto{\pgfqpoint{10.125761in}{4.464971in}}{\pgfqpoint{10.130151in}{4.475570in}}{\pgfqpoint{10.130151in}{4.486620in}}%
\pgfpathcurveto{\pgfqpoint{10.130151in}{4.497670in}}{\pgfqpoint{10.125761in}{4.508269in}}{\pgfqpoint{10.117947in}{4.516083in}}%
\pgfpathcurveto{\pgfqpoint{10.110134in}{4.523897in}}{\pgfqpoint{10.099535in}{4.528287in}}{\pgfqpoint{10.088484in}{4.528287in}}%
\pgfpathcurveto{\pgfqpoint{10.077434in}{4.528287in}}{\pgfqpoint{10.066835in}{4.523897in}}{\pgfqpoint{10.059022in}{4.516083in}}%
\pgfpathcurveto{\pgfqpoint{10.051208in}{4.508269in}}{\pgfqpoint{10.046818in}{4.497670in}}{\pgfqpoint{10.046818in}{4.486620in}}%
\pgfpathcurveto{\pgfqpoint{10.046818in}{4.475570in}}{\pgfqpoint{10.051208in}{4.464971in}}{\pgfqpoint{10.059022in}{4.457158in}}%
\pgfpathcurveto{\pgfqpoint{10.066835in}{4.449344in}}{\pgfqpoint{10.077434in}{4.444954in}}{\pgfqpoint{10.088484in}{4.444954in}}%
\pgfpathlineto{\pgfqpoint{10.088484in}{4.444954in}}%
\pgfpathclose%
\pgfusepath{stroke}%
\end{pgfscope}%
\begin{pgfscope}%
\pgfpathrectangle{\pgfqpoint{7.512535in}{0.437222in}}{\pgfqpoint{6.275590in}{5.159444in}}%
\pgfusepath{clip}%
\pgfsetbuttcap%
\pgfsetroundjoin%
\pgfsetlinewidth{1.003750pt}%
\definecolor{currentstroke}{rgb}{0.827451,0.827451,0.827451}%
\pgfsetstrokecolor{currentstroke}%
\pgfsetstrokeopacity{0.800000}%
\pgfsetdash{}{0pt}%
\pgfpathmoveto{\pgfqpoint{9.312093in}{1.930526in}}%
\pgfpathcurveto{\pgfqpoint{9.323144in}{1.930526in}}{\pgfqpoint{9.333743in}{1.934916in}}{\pgfqpoint{9.341556in}{1.942730in}}%
\pgfpathcurveto{\pgfqpoint{9.349370in}{1.950543in}}{\pgfqpoint{9.353760in}{1.961142in}}{\pgfqpoint{9.353760in}{1.972193in}}%
\pgfpathcurveto{\pgfqpoint{9.353760in}{1.983243in}}{\pgfqpoint{9.349370in}{1.993842in}}{\pgfqpoint{9.341556in}{2.001655in}}%
\pgfpathcurveto{\pgfqpoint{9.333743in}{2.009469in}}{\pgfqpoint{9.323144in}{2.013859in}}{\pgfqpoint{9.312093in}{2.013859in}}%
\pgfpathcurveto{\pgfqpoint{9.301043in}{2.013859in}}{\pgfqpoint{9.290444in}{2.009469in}}{\pgfqpoint{9.282631in}{2.001655in}}%
\pgfpathcurveto{\pgfqpoint{9.274817in}{1.993842in}}{\pgfqpoint{9.270427in}{1.983243in}}{\pgfqpoint{9.270427in}{1.972193in}}%
\pgfpathcurveto{\pgfqpoint{9.270427in}{1.961142in}}{\pgfqpoint{9.274817in}{1.950543in}}{\pgfqpoint{9.282631in}{1.942730in}}%
\pgfpathcurveto{\pgfqpoint{9.290444in}{1.934916in}}{\pgfqpoint{9.301043in}{1.930526in}}{\pgfqpoint{9.312093in}{1.930526in}}%
\pgfpathlineto{\pgfqpoint{9.312093in}{1.930526in}}%
\pgfpathclose%
\pgfusepath{stroke}%
\end{pgfscope}%
\begin{pgfscope}%
\pgfpathrectangle{\pgfqpoint{7.512535in}{0.437222in}}{\pgfqpoint{6.275590in}{5.159444in}}%
\pgfusepath{clip}%
\pgfsetbuttcap%
\pgfsetroundjoin%
\pgfsetlinewidth{1.003750pt}%
\definecolor{currentstroke}{rgb}{0.827451,0.827451,0.827451}%
\pgfsetstrokecolor{currentstroke}%
\pgfsetstrokeopacity{0.800000}%
\pgfsetdash{}{0pt}%
\pgfpathmoveto{\pgfqpoint{7.562575in}{0.493855in}}%
\pgfpathcurveto{\pgfqpoint{7.573625in}{0.493855in}}{\pgfqpoint{7.584224in}{0.498245in}}{\pgfqpoint{7.592038in}{0.506059in}}%
\pgfpathcurveto{\pgfqpoint{7.599851in}{0.513872in}}{\pgfqpoint{7.604241in}{0.524471in}}{\pgfqpoint{7.604241in}{0.535522in}}%
\pgfpathcurveto{\pgfqpoint{7.604241in}{0.546572in}}{\pgfqpoint{7.599851in}{0.557171in}}{\pgfqpoint{7.592038in}{0.564984in}}%
\pgfpathcurveto{\pgfqpoint{7.584224in}{0.572798in}}{\pgfqpoint{7.573625in}{0.577188in}}{\pgfqpoint{7.562575in}{0.577188in}}%
\pgfpathcurveto{\pgfqpoint{7.551525in}{0.577188in}}{\pgfqpoint{7.540926in}{0.572798in}}{\pgfqpoint{7.533112in}{0.564984in}}%
\pgfpathcurveto{\pgfqpoint{7.525298in}{0.557171in}}{\pgfqpoint{7.520908in}{0.546572in}}{\pgfqpoint{7.520908in}{0.535522in}}%
\pgfpathcurveto{\pgfqpoint{7.520908in}{0.524471in}}{\pgfqpoint{7.525298in}{0.513872in}}{\pgfqpoint{7.533112in}{0.506059in}}%
\pgfpathcurveto{\pgfqpoint{7.540926in}{0.498245in}}{\pgfqpoint{7.551525in}{0.493855in}}{\pgfqpoint{7.562575in}{0.493855in}}%
\pgfpathlineto{\pgfqpoint{7.562575in}{0.493855in}}%
\pgfpathclose%
\pgfusepath{stroke}%
\end{pgfscope}%
\begin{pgfscope}%
\pgfpathrectangle{\pgfqpoint{7.512535in}{0.437222in}}{\pgfqpoint{6.275590in}{5.159444in}}%
\pgfusepath{clip}%
\pgfsetbuttcap%
\pgfsetroundjoin%
\pgfsetlinewidth{1.003750pt}%
\definecolor{currentstroke}{rgb}{0.827451,0.827451,0.827451}%
\pgfsetstrokecolor{currentstroke}%
\pgfsetstrokeopacity{0.800000}%
\pgfsetdash{}{0pt}%
\pgfpathmoveto{\pgfqpoint{9.266706in}{3.281458in}}%
\pgfpathcurveto{\pgfqpoint{9.277756in}{3.281458in}}{\pgfqpoint{9.288355in}{3.285848in}}{\pgfqpoint{9.296168in}{3.293661in}}%
\pgfpathcurveto{\pgfqpoint{9.303982in}{3.301475in}}{\pgfqpoint{9.308372in}{3.312074in}}{\pgfqpoint{9.308372in}{3.323124in}}%
\pgfpathcurveto{\pgfqpoint{9.308372in}{3.334174in}}{\pgfqpoint{9.303982in}{3.344773in}}{\pgfqpoint{9.296168in}{3.352587in}}%
\pgfpathcurveto{\pgfqpoint{9.288355in}{3.360401in}}{\pgfqpoint{9.277756in}{3.364791in}}{\pgfqpoint{9.266706in}{3.364791in}}%
\pgfpathcurveto{\pgfqpoint{9.255656in}{3.364791in}}{\pgfqpoint{9.245057in}{3.360401in}}{\pgfqpoint{9.237243in}{3.352587in}}%
\pgfpathcurveto{\pgfqpoint{9.229429in}{3.344773in}}{\pgfqpoint{9.225039in}{3.334174in}}{\pgfqpoint{9.225039in}{3.323124in}}%
\pgfpathcurveto{\pgfqpoint{9.225039in}{3.312074in}}{\pgfqpoint{9.229429in}{3.301475in}}{\pgfqpoint{9.237243in}{3.293661in}}%
\pgfpathcurveto{\pgfqpoint{9.245057in}{3.285848in}}{\pgfqpoint{9.255656in}{3.281458in}}{\pgfqpoint{9.266706in}{3.281458in}}%
\pgfpathlineto{\pgfqpoint{9.266706in}{3.281458in}}%
\pgfpathclose%
\pgfusepath{stroke}%
\end{pgfscope}%
\begin{pgfscope}%
\pgfpathrectangle{\pgfqpoint{7.512535in}{0.437222in}}{\pgfqpoint{6.275590in}{5.159444in}}%
\pgfusepath{clip}%
\pgfsetbuttcap%
\pgfsetroundjoin%
\pgfsetlinewidth{1.003750pt}%
\definecolor{currentstroke}{rgb}{0.827451,0.827451,0.827451}%
\pgfsetstrokecolor{currentstroke}%
\pgfsetstrokeopacity{0.800000}%
\pgfsetdash{}{0pt}%
\pgfpathmoveto{\pgfqpoint{8.716039in}{3.385314in}}%
\pgfpathcurveto{\pgfqpoint{8.727089in}{3.385314in}}{\pgfqpoint{8.737688in}{3.389704in}}{\pgfqpoint{8.745502in}{3.397518in}}%
\pgfpathcurveto{\pgfqpoint{8.753315in}{3.405332in}}{\pgfqpoint{8.757706in}{3.415931in}}{\pgfqpoint{8.757706in}{3.426981in}}%
\pgfpathcurveto{\pgfqpoint{8.757706in}{3.438031in}}{\pgfqpoint{8.753315in}{3.448630in}}{\pgfqpoint{8.745502in}{3.456444in}}%
\pgfpathcurveto{\pgfqpoint{8.737688in}{3.464257in}}{\pgfqpoint{8.727089in}{3.468647in}}{\pgfqpoint{8.716039in}{3.468647in}}%
\pgfpathcurveto{\pgfqpoint{8.704989in}{3.468647in}}{\pgfqpoint{8.694390in}{3.464257in}}{\pgfqpoint{8.686576in}{3.456444in}}%
\pgfpathcurveto{\pgfqpoint{8.678763in}{3.448630in}}{\pgfqpoint{8.674372in}{3.438031in}}{\pgfqpoint{8.674372in}{3.426981in}}%
\pgfpathcurveto{\pgfqpoint{8.674372in}{3.415931in}}{\pgfqpoint{8.678763in}{3.405332in}}{\pgfqpoint{8.686576in}{3.397518in}}%
\pgfpathcurveto{\pgfqpoint{8.694390in}{3.389704in}}{\pgfqpoint{8.704989in}{3.385314in}}{\pgfqpoint{8.716039in}{3.385314in}}%
\pgfpathlineto{\pgfqpoint{8.716039in}{3.385314in}}%
\pgfpathclose%
\pgfusepath{stroke}%
\end{pgfscope}%
\begin{pgfscope}%
\pgfpathrectangle{\pgfqpoint{7.512535in}{0.437222in}}{\pgfqpoint{6.275590in}{5.159444in}}%
\pgfusepath{clip}%
\pgfsetbuttcap%
\pgfsetroundjoin%
\pgfsetlinewidth{1.003750pt}%
\definecolor{currentstroke}{rgb}{0.827451,0.827451,0.827451}%
\pgfsetstrokecolor{currentstroke}%
\pgfsetstrokeopacity{0.800000}%
\pgfsetdash{}{0pt}%
\pgfpathmoveto{\pgfqpoint{7.983939in}{1.000624in}}%
\pgfpathcurveto{\pgfqpoint{7.994989in}{1.000624in}}{\pgfqpoint{8.005588in}{1.005015in}}{\pgfqpoint{8.013401in}{1.012828in}}%
\pgfpathcurveto{\pgfqpoint{8.021215in}{1.020642in}}{\pgfqpoint{8.025605in}{1.031241in}}{\pgfqpoint{8.025605in}{1.042291in}}%
\pgfpathcurveto{\pgfqpoint{8.025605in}{1.053341in}}{\pgfqpoint{8.021215in}{1.063940in}}{\pgfqpoint{8.013401in}{1.071754in}}%
\pgfpathcurveto{\pgfqpoint{8.005588in}{1.079568in}}{\pgfqpoint{7.994989in}{1.083958in}}{\pgfqpoint{7.983939in}{1.083958in}}%
\pgfpathcurveto{\pgfqpoint{7.972888in}{1.083958in}}{\pgfqpoint{7.962289in}{1.079568in}}{\pgfqpoint{7.954476in}{1.071754in}}%
\pgfpathcurveto{\pgfqpoint{7.946662in}{1.063940in}}{\pgfqpoint{7.942272in}{1.053341in}}{\pgfqpoint{7.942272in}{1.042291in}}%
\pgfpathcurveto{\pgfqpoint{7.942272in}{1.031241in}}{\pgfqpoint{7.946662in}{1.020642in}}{\pgfqpoint{7.954476in}{1.012828in}}%
\pgfpathcurveto{\pgfqpoint{7.962289in}{1.005015in}}{\pgfqpoint{7.972888in}{1.000624in}}{\pgfqpoint{7.983939in}{1.000624in}}%
\pgfpathlineto{\pgfqpoint{7.983939in}{1.000624in}}%
\pgfpathclose%
\pgfusepath{stroke}%
\end{pgfscope}%
\begin{pgfscope}%
\pgfpathrectangle{\pgfqpoint{7.512535in}{0.437222in}}{\pgfqpoint{6.275590in}{5.159444in}}%
\pgfusepath{clip}%
\pgfsetbuttcap%
\pgfsetroundjoin%
\pgfsetlinewidth{1.003750pt}%
\definecolor{currentstroke}{rgb}{0.827451,0.827451,0.827451}%
\pgfsetstrokecolor{currentstroke}%
\pgfsetstrokeopacity{0.800000}%
\pgfsetdash{}{0pt}%
\pgfpathmoveto{\pgfqpoint{10.435824in}{5.040165in}}%
\pgfpathcurveto{\pgfqpoint{10.446874in}{5.040165in}}{\pgfqpoint{10.457473in}{5.044555in}}{\pgfqpoint{10.465287in}{5.052369in}}%
\pgfpathcurveto{\pgfqpoint{10.473101in}{5.060183in}}{\pgfqpoint{10.477491in}{5.070782in}}{\pgfqpoint{10.477491in}{5.081832in}}%
\pgfpathcurveto{\pgfqpoint{10.477491in}{5.092882in}}{\pgfqpoint{10.473101in}{5.103481in}}{\pgfqpoint{10.465287in}{5.111295in}}%
\pgfpathcurveto{\pgfqpoint{10.457473in}{5.119108in}}{\pgfqpoint{10.446874in}{5.123499in}}{\pgfqpoint{10.435824in}{5.123499in}}%
\pgfpathcurveto{\pgfqpoint{10.424774in}{5.123499in}}{\pgfqpoint{10.414175in}{5.119108in}}{\pgfqpoint{10.406361in}{5.111295in}}%
\pgfpathcurveto{\pgfqpoint{10.398548in}{5.103481in}}{\pgfqpoint{10.394157in}{5.092882in}}{\pgfqpoint{10.394157in}{5.081832in}}%
\pgfpathcurveto{\pgfqpoint{10.394157in}{5.070782in}}{\pgfqpoint{10.398548in}{5.060183in}}{\pgfqpoint{10.406361in}{5.052369in}}%
\pgfpathcurveto{\pgfqpoint{10.414175in}{5.044555in}}{\pgfqpoint{10.424774in}{5.040165in}}{\pgfqpoint{10.435824in}{5.040165in}}%
\pgfpathlineto{\pgfqpoint{10.435824in}{5.040165in}}%
\pgfpathclose%
\pgfusepath{stroke}%
\end{pgfscope}%
\begin{pgfscope}%
\pgfpathrectangle{\pgfqpoint{7.512535in}{0.437222in}}{\pgfqpoint{6.275590in}{5.159444in}}%
\pgfusepath{clip}%
\pgfsetbuttcap%
\pgfsetroundjoin%
\pgfsetlinewidth{1.003750pt}%
\definecolor{currentstroke}{rgb}{0.827451,0.827451,0.827451}%
\pgfsetstrokecolor{currentstroke}%
\pgfsetstrokeopacity{0.800000}%
\pgfsetdash{}{0pt}%
\pgfpathmoveto{\pgfqpoint{12.287231in}{5.553091in}}%
\pgfpathcurveto{\pgfqpoint{12.298281in}{5.553091in}}{\pgfqpoint{12.308880in}{5.557481in}}{\pgfqpoint{12.316694in}{5.565295in}}%
\pgfpathcurveto{\pgfqpoint{12.324508in}{5.573108in}}{\pgfqpoint{12.328898in}{5.583707in}}{\pgfqpoint{12.328898in}{5.594757in}}%
\pgfpathcurveto{\pgfqpoint{12.328898in}{5.605808in}}{\pgfqpoint{12.324508in}{5.616407in}}{\pgfqpoint{12.316694in}{5.624220in}}%
\pgfpathcurveto{\pgfqpoint{12.308880in}{5.632034in}}{\pgfqpoint{12.298281in}{5.636424in}}{\pgfqpoint{12.287231in}{5.636424in}}%
\pgfpathcurveto{\pgfqpoint{12.276181in}{5.636424in}}{\pgfqpoint{12.265582in}{5.632034in}}{\pgfqpoint{12.257768in}{5.624220in}}%
\pgfpathcurveto{\pgfqpoint{12.249955in}{5.616407in}}{\pgfqpoint{12.245564in}{5.605808in}}{\pgfqpoint{12.245564in}{5.594757in}}%
\pgfpathcurveto{\pgfqpoint{12.245564in}{5.583707in}}{\pgfqpoint{12.249955in}{5.573108in}}{\pgfqpoint{12.257768in}{5.565295in}}%
\pgfpathcurveto{\pgfqpoint{12.265582in}{5.557481in}}{\pgfqpoint{12.276181in}{5.553091in}}{\pgfqpoint{12.287231in}{5.553091in}}%
\pgfpathlineto{\pgfqpoint{12.287231in}{5.553091in}}%
\pgfpathclose%
\pgfusepath{stroke}%
\end{pgfscope}%
\begin{pgfscope}%
\pgfpathrectangle{\pgfqpoint{7.512535in}{0.437222in}}{\pgfqpoint{6.275590in}{5.159444in}}%
\pgfusepath{clip}%
\pgfsetbuttcap%
\pgfsetroundjoin%
\pgfsetlinewidth{1.003750pt}%
\definecolor{currentstroke}{rgb}{0.827451,0.827451,0.827451}%
\pgfsetstrokecolor{currentstroke}%
\pgfsetstrokeopacity{0.800000}%
\pgfsetdash{}{0pt}%
\pgfpathmoveto{\pgfqpoint{11.517563in}{5.553954in}}%
\pgfpathcurveto{\pgfqpoint{11.528613in}{5.553954in}}{\pgfqpoint{11.539212in}{5.558344in}}{\pgfqpoint{11.547026in}{5.566158in}}%
\pgfpathcurveto{\pgfqpoint{11.554839in}{5.573972in}}{\pgfqpoint{11.559229in}{5.584571in}}{\pgfqpoint{11.559229in}{5.595621in}}%
\pgfpathcurveto{\pgfqpoint{11.559229in}{5.606671in}}{\pgfqpoint{11.554839in}{5.617270in}}{\pgfqpoint{11.547026in}{5.625083in}}%
\pgfpathcurveto{\pgfqpoint{11.539212in}{5.632897in}}{\pgfqpoint{11.528613in}{5.637287in}}{\pgfqpoint{11.517563in}{5.637287in}}%
\pgfpathcurveto{\pgfqpoint{11.506513in}{5.637287in}}{\pgfqpoint{11.495914in}{5.632897in}}{\pgfqpoint{11.488100in}{5.625083in}}%
\pgfpathcurveto{\pgfqpoint{11.480286in}{5.617270in}}{\pgfqpoint{11.475896in}{5.606671in}}{\pgfqpoint{11.475896in}{5.595621in}}%
\pgfpathcurveto{\pgfqpoint{11.475896in}{5.584571in}}{\pgfqpoint{11.480286in}{5.573972in}}{\pgfqpoint{11.488100in}{5.566158in}}%
\pgfpathcurveto{\pgfqpoint{11.495914in}{5.558344in}}{\pgfqpoint{11.506513in}{5.553954in}}{\pgfqpoint{11.517563in}{5.553954in}}%
\pgfpathlineto{\pgfqpoint{11.517563in}{5.553954in}}%
\pgfpathclose%
\pgfusepath{stroke}%
\end{pgfscope}%
\begin{pgfscope}%
\pgfpathrectangle{\pgfqpoint{7.512535in}{0.437222in}}{\pgfqpoint{6.275590in}{5.159444in}}%
\pgfusepath{clip}%
\pgfsetbuttcap%
\pgfsetroundjoin%
\pgfsetlinewidth{1.003750pt}%
\definecolor{currentstroke}{rgb}{0.827451,0.827451,0.827451}%
\pgfsetstrokecolor{currentstroke}%
\pgfsetstrokeopacity{0.800000}%
\pgfsetdash{}{0pt}%
\pgfpathmoveto{\pgfqpoint{10.369675in}{4.861538in}}%
\pgfpathcurveto{\pgfqpoint{10.380725in}{4.861538in}}{\pgfqpoint{10.391324in}{4.865928in}}{\pgfqpoint{10.399138in}{4.873742in}}%
\pgfpathcurveto{\pgfqpoint{10.406951in}{4.881555in}}{\pgfqpoint{10.411341in}{4.892155in}}{\pgfqpoint{10.411341in}{4.903205in}}%
\pgfpathcurveto{\pgfqpoint{10.411341in}{4.914255in}}{\pgfqpoint{10.406951in}{4.924854in}}{\pgfqpoint{10.399138in}{4.932667in}}%
\pgfpathcurveto{\pgfqpoint{10.391324in}{4.940481in}}{\pgfqpoint{10.380725in}{4.944871in}}{\pgfqpoint{10.369675in}{4.944871in}}%
\pgfpathcurveto{\pgfqpoint{10.358625in}{4.944871in}}{\pgfqpoint{10.348026in}{4.940481in}}{\pgfqpoint{10.340212in}{4.932667in}}%
\pgfpathcurveto{\pgfqpoint{10.332398in}{4.924854in}}{\pgfqpoint{10.328008in}{4.914255in}}{\pgfqpoint{10.328008in}{4.903205in}}%
\pgfpathcurveto{\pgfqpoint{10.328008in}{4.892155in}}{\pgfqpoint{10.332398in}{4.881555in}}{\pgfqpoint{10.340212in}{4.873742in}}%
\pgfpathcurveto{\pgfqpoint{10.348026in}{4.865928in}}{\pgfqpoint{10.358625in}{4.861538in}}{\pgfqpoint{10.369675in}{4.861538in}}%
\pgfpathlineto{\pgfqpoint{10.369675in}{4.861538in}}%
\pgfpathclose%
\pgfusepath{stroke}%
\end{pgfscope}%
\begin{pgfscope}%
\pgfpathrectangle{\pgfqpoint{7.512535in}{0.437222in}}{\pgfqpoint{6.275590in}{5.159444in}}%
\pgfusepath{clip}%
\pgfsetbuttcap%
\pgfsetroundjoin%
\pgfsetlinewidth{1.003750pt}%
\definecolor{currentstroke}{rgb}{0.827451,0.827451,0.827451}%
\pgfsetstrokecolor{currentstroke}%
\pgfsetstrokeopacity{0.800000}%
\pgfsetdash{}{0pt}%
\pgfpathmoveto{\pgfqpoint{10.369675in}{4.552452in}}%
\pgfpathcurveto{\pgfqpoint{10.380725in}{4.552452in}}{\pgfqpoint{10.391324in}{4.556842in}}{\pgfqpoint{10.399138in}{4.564656in}}%
\pgfpathcurveto{\pgfqpoint{10.406951in}{4.572470in}}{\pgfqpoint{10.411341in}{4.583069in}}{\pgfqpoint{10.411341in}{4.594119in}}%
\pgfpathcurveto{\pgfqpoint{10.411341in}{4.605169in}}{\pgfqpoint{10.406951in}{4.615768in}}{\pgfqpoint{10.399138in}{4.623582in}}%
\pgfpathcurveto{\pgfqpoint{10.391324in}{4.631395in}}{\pgfqpoint{10.380725in}{4.635785in}}{\pgfqpoint{10.369675in}{4.635785in}}%
\pgfpathcurveto{\pgfqpoint{10.358625in}{4.635785in}}{\pgfqpoint{10.348026in}{4.631395in}}{\pgfqpoint{10.340212in}{4.623582in}}%
\pgfpathcurveto{\pgfqpoint{10.332398in}{4.615768in}}{\pgfqpoint{10.328008in}{4.605169in}}{\pgfqpoint{10.328008in}{4.594119in}}%
\pgfpathcurveto{\pgfqpoint{10.328008in}{4.583069in}}{\pgfqpoint{10.332398in}{4.572470in}}{\pgfqpoint{10.340212in}{4.564656in}}%
\pgfpathcurveto{\pgfqpoint{10.348026in}{4.556842in}}{\pgfqpoint{10.358625in}{4.552452in}}{\pgfqpoint{10.369675in}{4.552452in}}%
\pgfpathlineto{\pgfqpoint{10.369675in}{4.552452in}}%
\pgfpathclose%
\pgfusepath{stroke}%
\end{pgfscope}%
\begin{pgfscope}%
\pgfpathrectangle{\pgfqpoint{7.512535in}{0.437222in}}{\pgfqpoint{6.275590in}{5.159444in}}%
\pgfusepath{clip}%
\pgfsetbuttcap%
\pgfsetroundjoin%
\pgfsetlinewidth{1.003750pt}%
\definecolor{currentstroke}{rgb}{0.827451,0.827451,0.827451}%
\pgfsetstrokecolor{currentstroke}%
\pgfsetstrokeopacity{0.800000}%
\pgfsetdash{}{0pt}%
\pgfpathmoveto{\pgfqpoint{8.881634in}{1.944343in}}%
\pgfpathcurveto{\pgfqpoint{8.892684in}{1.944343in}}{\pgfqpoint{8.903283in}{1.948733in}}{\pgfqpoint{8.911096in}{1.956546in}}%
\pgfpathcurveto{\pgfqpoint{8.918910in}{1.964360in}}{\pgfqpoint{8.923300in}{1.974959in}}{\pgfqpoint{8.923300in}{1.986009in}}%
\pgfpathcurveto{\pgfqpoint{8.923300in}{1.997059in}}{\pgfqpoint{8.918910in}{2.007658in}}{\pgfqpoint{8.911096in}{2.015472in}}%
\pgfpathcurveto{\pgfqpoint{8.903283in}{2.023286in}}{\pgfqpoint{8.892684in}{2.027676in}}{\pgfqpoint{8.881634in}{2.027676in}}%
\pgfpathcurveto{\pgfqpoint{8.870583in}{2.027676in}}{\pgfqpoint{8.859984in}{2.023286in}}{\pgfqpoint{8.852171in}{2.015472in}}%
\pgfpathcurveto{\pgfqpoint{8.844357in}{2.007658in}}{\pgfqpoint{8.839967in}{1.997059in}}{\pgfqpoint{8.839967in}{1.986009in}}%
\pgfpathcurveto{\pgfqpoint{8.839967in}{1.974959in}}{\pgfqpoint{8.844357in}{1.964360in}}{\pgfqpoint{8.852171in}{1.956546in}}%
\pgfpathcurveto{\pgfqpoint{8.859984in}{1.948733in}}{\pgfqpoint{8.870583in}{1.944343in}}{\pgfqpoint{8.881634in}{1.944343in}}%
\pgfpathlineto{\pgfqpoint{8.881634in}{1.944343in}}%
\pgfpathclose%
\pgfusepath{stroke}%
\end{pgfscope}%
\begin{pgfscope}%
\pgfpathrectangle{\pgfqpoint{7.512535in}{0.437222in}}{\pgfqpoint{6.275590in}{5.159444in}}%
\pgfusepath{clip}%
\pgfsetbuttcap%
\pgfsetroundjoin%
\pgfsetlinewidth{1.003750pt}%
\definecolor{currentstroke}{rgb}{0.827451,0.827451,0.827451}%
\pgfsetstrokecolor{currentstroke}%
\pgfsetstrokeopacity{0.800000}%
\pgfsetdash{}{0pt}%
\pgfpathmoveto{\pgfqpoint{9.447017in}{3.335366in}}%
\pgfpathcurveto{\pgfqpoint{9.458067in}{3.335366in}}{\pgfqpoint{9.468666in}{3.339757in}}{\pgfqpoint{9.476479in}{3.347570in}}%
\pgfpathcurveto{\pgfqpoint{9.484293in}{3.355384in}}{\pgfqpoint{9.488683in}{3.365983in}}{\pgfqpoint{9.488683in}{3.377033in}}%
\pgfpathcurveto{\pgfqpoint{9.488683in}{3.388083in}}{\pgfqpoint{9.484293in}{3.398682in}}{\pgfqpoint{9.476479in}{3.406496in}}%
\pgfpathcurveto{\pgfqpoint{9.468666in}{3.414309in}}{\pgfqpoint{9.458067in}{3.418700in}}{\pgfqpoint{9.447017in}{3.418700in}}%
\pgfpathcurveto{\pgfqpoint{9.435967in}{3.418700in}}{\pgfqpoint{9.425367in}{3.414309in}}{\pgfqpoint{9.417554in}{3.406496in}}%
\pgfpathcurveto{\pgfqpoint{9.409740in}{3.398682in}}{\pgfqpoint{9.405350in}{3.388083in}}{\pgfqpoint{9.405350in}{3.377033in}}%
\pgfpathcurveto{\pgfqpoint{9.405350in}{3.365983in}}{\pgfqpoint{9.409740in}{3.355384in}}{\pgfqpoint{9.417554in}{3.347570in}}%
\pgfpathcurveto{\pgfqpoint{9.425367in}{3.339757in}}{\pgfqpoint{9.435967in}{3.335366in}}{\pgfqpoint{9.447017in}{3.335366in}}%
\pgfpathlineto{\pgfqpoint{9.447017in}{3.335366in}}%
\pgfpathclose%
\pgfusepath{stroke}%
\end{pgfscope}%
\begin{pgfscope}%
\pgfpathrectangle{\pgfqpoint{7.512535in}{0.437222in}}{\pgfqpoint{6.275590in}{5.159444in}}%
\pgfusepath{clip}%
\pgfsetbuttcap%
\pgfsetroundjoin%
\pgfsetlinewidth{1.003750pt}%
\definecolor{currentstroke}{rgb}{0.827451,0.827451,0.827451}%
\pgfsetstrokecolor{currentstroke}%
\pgfsetstrokeopacity{0.800000}%
\pgfsetdash{}{0pt}%
\pgfpathmoveto{\pgfqpoint{13.250415in}{5.553845in}}%
\pgfpathcurveto{\pgfqpoint{13.261465in}{5.553845in}}{\pgfqpoint{13.272064in}{5.558236in}}{\pgfqpoint{13.279878in}{5.566049in}}%
\pgfpathcurveto{\pgfqpoint{13.287691in}{5.573863in}}{\pgfqpoint{13.292082in}{5.584462in}}{\pgfqpoint{13.292082in}{5.595512in}}%
\pgfpathcurveto{\pgfqpoint{13.292082in}{5.606562in}}{\pgfqpoint{13.287691in}{5.617161in}}{\pgfqpoint{13.279878in}{5.624975in}}%
\pgfpathcurveto{\pgfqpoint{13.272064in}{5.632788in}}{\pgfqpoint{13.261465in}{5.637179in}}{\pgfqpoint{13.250415in}{5.637179in}}%
\pgfpathcurveto{\pgfqpoint{13.239365in}{5.637179in}}{\pgfqpoint{13.228766in}{5.632788in}}{\pgfqpoint{13.220952in}{5.624975in}}%
\pgfpathcurveto{\pgfqpoint{13.213139in}{5.617161in}}{\pgfqpoint{13.208748in}{5.606562in}}{\pgfqpoint{13.208748in}{5.595512in}}%
\pgfpathcurveto{\pgfqpoint{13.208748in}{5.584462in}}{\pgfqpoint{13.213139in}{5.573863in}}{\pgfqpoint{13.220952in}{5.566049in}}%
\pgfpathcurveto{\pgfqpoint{13.228766in}{5.558236in}}{\pgfqpoint{13.239365in}{5.553845in}}{\pgfqpoint{13.250415in}{5.553845in}}%
\pgfpathlineto{\pgfqpoint{13.250415in}{5.553845in}}%
\pgfpathclose%
\pgfusepath{stroke}%
\end{pgfscope}%
\begin{pgfscope}%
\pgfpathrectangle{\pgfqpoint{7.512535in}{0.437222in}}{\pgfqpoint{6.275590in}{5.159444in}}%
\pgfusepath{clip}%
\pgfsetbuttcap%
\pgfsetroundjoin%
\pgfsetlinewidth{1.003750pt}%
\definecolor{currentstroke}{rgb}{0.827451,0.827451,0.827451}%
\pgfsetstrokecolor{currentstroke}%
\pgfsetstrokeopacity{0.800000}%
\pgfsetdash{}{0pt}%
\pgfpathmoveto{\pgfqpoint{8.432851in}{1.518124in}}%
\pgfpathcurveto{\pgfqpoint{8.443901in}{1.518124in}}{\pgfqpoint{8.454500in}{1.522514in}}{\pgfqpoint{8.462313in}{1.530328in}}%
\pgfpathcurveto{\pgfqpoint{8.470127in}{1.538142in}}{\pgfqpoint{8.474517in}{1.548741in}}{\pgfqpoint{8.474517in}{1.559791in}}%
\pgfpathcurveto{\pgfqpoint{8.474517in}{1.570841in}}{\pgfqpoint{8.470127in}{1.581440in}}{\pgfqpoint{8.462313in}{1.589254in}}%
\pgfpathcurveto{\pgfqpoint{8.454500in}{1.597067in}}{\pgfqpoint{8.443901in}{1.601457in}}{\pgfqpoint{8.432851in}{1.601457in}}%
\pgfpathcurveto{\pgfqpoint{8.421801in}{1.601457in}}{\pgfqpoint{8.411201in}{1.597067in}}{\pgfqpoint{8.403388in}{1.589254in}}%
\pgfpathcurveto{\pgfqpoint{8.395574in}{1.581440in}}{\pgfqpoint{8.391184in}{1.570841in}}{\pgfqpoint{8.391184in}{1.559791in}}%
\pgfpathcurveto{\pgfqpoint{8.391184in}{1.548741in}}{\pgfqpoint{8.395574in}{1.538142in}}{\pgfqpoint{8.403388in}{1.530328in}}%
\pgfpathcurveto{\pgfqpoint{8.411201in}{1.522514in}}{\pgfqpoint{8.421801in}{1.518124in}}{\pgfqpoint{8.432851in}{1.518124in}}%
\pgfpathlineto{\pgfqpoint{8.432851in}{1.518124in}}%
\pgfpathclose%
\pgfusepath{stroke}%
\end{pgfscope}%
\begin{pgfscope}%
\pgfpathrectangle{\pgfqpoint{7.512535in}{0.437222in}}{\pgfqpoint{6.275590in}{5.159444in}}%
\pgfusepath{clip}%
\pgfsetbuttcap%
\pgfsetroundjoin%
\pgfsetlinewidth{1.003750pt}%
\definecolor{currentstroke}{rgb}{0.827451,0.827451,0.827451}%
\pgfsetstrokecolor{currentstroke}%
\pgfsetstrokeopacity{0.800000}%
\pgfsetdash{}{0pt}%
\pgfpathmoveto{\pgfqpoint{12.495601in}{5.523529in}}%
\pgfpathcurveto{\pgfqpoint{12.506651in}{5.523529in}}{\pgfqpoint{12.517250in}{5.527919in}}{\pgfqpoint{12.525063in}{5.535733in}}%
\pgfpathcurveto{\pgfqpoint{12.532877in}{5.543547in}}{\pgfqpoint{12.537267in}{5.554146in}}{\pgfqpoint{12.537267in}{5.565196in}}%
\pgfpathcurveto{\pgfqpoint{12.537267in}{5.576246in}}{\pgfqpoint{12.532877in}{5.586845in}}{\pgfqpoint{12.525063in}{5.594658in}}%
\pgfpathcurveto{\pgfqpoint{12.517250in}{5.602472in}}{\pgfqpoint{12.506651in}{5.606862in}}{\pgfqpoint{12.495601in}{5.606862in}}%
\pgfpathcurveto{\pgfqpoint{12.484550in}{5.606862in}}{\pgfqpoint{12.473951in}{5.602472in}}{\pgfqpoint{12.466138in}{5.594658in}}%
\pgfpathcurveto{\pgfqpoint{12.458324in}{5.586845in}}{\pgfqpoint{12.453934in}{5.576246in}}{\pgfqpoint{12.453934in}{5.565196in}}%
\pgfpathcurveto{\pgfqpoint{12.453934in}{5.554146in}}{\pgfqpoint{12.458324in}{5.543547in}}{\pgfqpoint{12.466138in}{5.535733in}}%
\pgfpathcurveto{\pgfqpoint{12.473951in}{5.527919in}}{\pgfqpoint{12.484550in}{5.523529in}}{\pgfqpoint{12.495601in}{5.523529in}}%
\pgfpathlineto{\pgfqpoint{12.495601in}{5.523529in}}%
\pgfpathclose%
\pgfusepath{stroke}%
\end{pgfscope}%
\begin{pgfscope}%
\pgfpathrectangle{\pgfqpoint{7.512535in}{0.437222in}}{\pgfqpoint{6.275590in}{5.159444in}}%
\pgfusepath{clip}%
\pgfsetbuttcap%
\pgfsetroundjoin%
\pgfsetlinewidth{1.003750pt}%
\definecolor{currentstroke}{rgb}{0.827451,0.827451,0.827451}%
\pgfsetstrokecolor{currentstroke}%
\pgfsetstrokeopacity{0.800000}%
\pgfsetdash{}{0pt}%
\pgfpathmoveto{\pgfqpoint{10.173598in}{3.250664in}}%
\pgfpathcurveto{\pgfqpoint{10.184648in}{3.250664in}}{\pgfqpoint{10.195247in}{3.255054in}}{\pgfqpoint{10.203060in}{3.262868in}}%
\pgfpathcurveto{\pgfqpoint{10.210874in}{3.270682in}}{\pgfqpoint{10.215264in}{3.281281in}}{\pgfqpoint{10.215264in}{3.292331in}}%
\pgfpathcurveto{\pgfqpoint{10.215264in}{3.303381in}}{\pgfqpoint{10.210874in}{3.313980in}}{\pgfqpoint{10.203060in}{3.321794in}}%
\pgfpathcurveto{\pgfqpoint{10.195247in}{3.329607in}}{\pgfqpoint{10.184648in}{3.333998in}}{\pgfqpoint{10.173598in}{3.333998in}}%
\pgfpathcurveto{\pgfqpoint{10.162547in}{3.333998in}}{\pgfqpoint{10.151948in}{3.329607in}}{\pgfqpoint{10.144135in}{3.321794in}}%
\pgfpathcurveto{\pgfqpoint{10.136321in}{3.313980in}}{\pgfqpoint{10.131931in}{3.303381in}}{\pgfqpoint{10.131931in}{3.292331in}}%
\pgfpathcurveto{\pgfqpoint{10.131931in}{3.281281in}}{\pgfqpoint{10.136321in}{3.270682in}}{\pgfqpoint{10.144135in}{3.262868in}}%
\pgfpathcurveto{\pgfqpoint{10.151948in}{3.255054in}}{\pgfqpoint{10.162547in}{3.250664in}}{\pgfqpoint{10.173598in}{3.250664in}}%
\pgfpathlineto{\pgfqpoint{10.173598in}{3.250664in}}%
\pgfpathclose%
\pgfusepath{stroke}%
\end{pgfscope}%
\begin{pgfscope}%
\pgfpathrectangle{\pgfqpoint{7.512535in}{0.437222in}}{\pgfqpoint{6.275590in}{5.159444in}}%
\pgfusepath{clip}%
\pgfsetbuttcap%
\pgfsetroundjoin%
\pgfsetlinewidth{1.003750pt}%
\definecolor{currentstroke}{rgb}{0.827451,0.827451,0.827451}%
\pgfsetstrokecolor{currentstroke}%
\pgfsetstrokeopacity{0.800000}%
\pgfsetdash{}{0pt}%
\pgfpathmoveto{\pgfqpoint{8.261016in}{1.313150in}}%
\pgfpathcurveto{\pgfqpoint{8.272066in}{1.313150in}}{\pgfqpoint{8.282665in}{1.317540in}}{\pgfqpoint{8.290479in}{1.325354in}}%
\pgfpathcurveto{\pgfqpoint{8.298293in}{1.333167in}}{\pgfqpoint{8.302683in}{1.343766in}}{\pgfqpoint{8.302683in}{1.354817in}}%
\pgfpathcurveto{\pgfqpoint{8.302683in}{1.365867in}}{\pgfqpoint{8.298293in}{1.376466in}}{\pgfqpoint{8.290479in}{1.384279in}}%
\pgfpathcurveto{\pgfqpoint{8.282665in}{1.392093in}}{\pgfqpoint{8.272066in}{1.396483in}}{\pgfqpoint{8.261016in}{1.396483in}}%
\pgfpathcurveto{\pgfqpoint{8.249966in}{1.396483in}}{\pgfqpoint{8.239367in}{1.392093in}}{\pgfqpoint{8.231554in}{1.384279in}}%
\pgfpathcurveto{\pgfqpoint{8.223740in}{1.376466in}}{\pgfqpoint{8.219350in}{1.365867in}}{\pgfqpoint{8.219350in}{1.354817in}}%
\pgfpathcurveto{\pgfqpoint{8.219350in}{1.343766in}}{\pgfqpoint{8.223740in}{1.333167in}}{\pgfqpoint{8.231554in}{1.325354in}}%
\pgfpathcurveto{\pgfqpoint{8.239367in}{1.317540in}}{\pgfqpoint{8.249966in}{1.313150in}}{\pgfqpoint{8.261016in}{1.313150in}}%
\pgfpathlineto{\pgfqpoint{8.261016in}{1.313150in}}%
\pgfpathclose%
\pgfusepath{stroke}%
\end{pgfscope}%
\begin{pgfscope}%
\pgfpathrectangle{\pgfqpoint{7.512535in}{0.437222in}}{\pgfqpoint{6.275590in}{5.159444in}}%
\pgfusepath{clip}%
\pgfsetbuttcap%
\pgfsetroundjoin%
\pgfsetlinewidth{1.003750pt}%
\definecolor{currentstroke}{rgb}{0.827451,0.827451,0.827451}%
\pgfsetstrokecolor{currentstroke}%
\pgfsetstrokeopacity{0.800000}%
\pgfsetdash{}{0pt}%
\pgfpathmoveto{\pgfqpoint{9.855876in}{4.311038in}}%
\pgfpathcurveto{\pgfqpoint{9.866926in}{4.311038in}}{\pgfqpoint{9.877525in}{4.315429in}}{\pgfqpoint{9.885338in}{4.323242in}}%
\pgfpathcurveto{\pgfqpoint{9.893152in}{4.331056in}}{\pgfqpoint{9.897542in}{4.341655in}}{\pgfqpoint{9.897542in}{4.352705in}}%
\pgfpathcurveto{\pgfqpoint{9.897542in}{4.363755in}}{\pgfqpoint{9.893152in}{4.374354in}}{\pgfqpoint{9.885338in}{4.382168in}}%
\pgfpathcurveto{\pgfqpoint{9.877525in}{4.389982in}}{\pgfqpoint{9.866926in}{4.394372in}}{\pgfqpoint{9.855876in}{4.394372in}}%
\pgfpathcurveto{\pgfqpoint{9.844826in}{4.394372in}}{\pgfqpoint{9.834226in}{4.389982in}}{\pgfqpoint{9.826413in}{4.382168in}}%
\pgfpathcurveto{\pgfqpoint{9.818599in}{4.374354in}}{\pgfqpoint{9.814209in}{4.363755in}}{\pgfqpoint{9.814209in}{4.352705in}}%
\pgfpathcurveto{\pgfqpoint{9.814209in}{4.341655in}}{\pgfqpoint{9.818599in}{4.331056in}}{\pgfqpoint{9.826413in}{4.323242in}}%
\pgfpathcurveto{\pgfqpoint{9.834226in}{4.315429in}}{\pgfqpoint{9.844826in}{4.311038in}}{\pgfqpoint{9.855876in}{4.311038in}}%
\pgfpathlineto{\pgfqpoint{9.855876in}{4.311038in}}%
\pgfpathclose%
\pgfusepath{stroke}%
\end{pgfscope}%
\begin{pgfscope}%
\pgfpathrectangle{\pgfqpoint{7.512535in}{0.437222in}}{\pgfqpoint{6.275590in}{5.159444in}}%
\pgfusepath{clip}%
\pgfsetbuttcap%
\pgfsetroundjoin%
\pgfsetlinewidth{1.003750pt}%
\definecolor{currentstroke}{rgb}{0.827451,0.827451,0.827451}%
\pgfsetstrokecolor{currentstroke}%
\pgfsetstrokeopacity{0.800000}%
\pgfsetdash{}{0pt}%
\pgfpathmoveto{\pgfqpoint{7.676506in}{1.009520in}}%
\pgfpathcurveto{\pgfqpoint{7.687556in}{1.009520in}}{\pgfqpoint{7.698155in}{1.013910in}}{\pgfqpoint{7.705969in}{1.021724in}}%
\pgfpathcurveto{\pgfqpoint{7.713783in}{1.029538in}}{\pgfqpoint{7.718173in}{1.040137in}}{\pgfqpoint{7.718173in}{1.051187in}}%
\pgfpathcurveto{\pgfqpoint{7.718173in}{1.062237in}}{\pgfqpoint{7.713783in}{1.072836in}}{\pgfqpoint{7.705969in}{1.080650in}}%
\pgfpathcurveto{\pgfqpoint{7.698155in}{1.088463in}}{\pgfqpoint{7.687556in}{1.092853in}}{\pgfqpoint{7.676506in}{1.092853in}}%
\pgfpathcurveto{\pgfqpoint{7.665456in}{1.092853in}}{\pgfqpoint{7.654857in}{1.088463in}}{\pgfqpoint{7.647043in}{1.080650in}}%
\pgfpathcurveto{\pgfqpoint{7.639230in}{1.072836in}}{\pgfqpoint{7.634840in}{1.062237in}}{\pgfqpoint{7.634840in}{1.051187in}}%
\pgfpathcurveto{\pgfqpoint{7.634840in}{1.040137in}}{\pgfqpoint{7.639230in}{1.029538in}}{\pgfqpoint{7.647043in}{1.021724in}}%
\pgfpathcurveto{\pgfqpoint{7.654857in}{1.013910in}}{\pgfqpoint{7.665456in}{1.009520in}}{\pgfqpoint{7.676506in}{1.009520in}}%
\pgfpathlineto{\pgfqpoint{7.676506in}{1.009520in}}%
\pgfpathclose%
\pgfusepath{stroke}%
\end{pgfscope}%
\begin{pgfscope}%
\pgfpathrectangle{\pgfqpoint{7.512535in}{0.437222in}}{\pgfqpoint{6.275590in}{5.159444in}}%
\pgfusepath{clip}%
\pgfsetbuttcap%
\pgfsetroundjoin%
\pgfsetlinewidth{1.003750pt}%
\definecolor{currentstroke}{rgb}{0.827451,0.827451,0.827451}%
\pgfsetstrokecolor{currentstroke}%
\pgfsetstrokeopacity{0.800000}%
\pgfsetdash{}{0pt}%
\pgfpathmoveto{\pgfqpoint{9.263397in}{3.235524in}}%
\pgfpathcurveto{\pgfqpoint{9.274447in}{3.235524in}}{\pgfqpoint{9.285046in}{3.239914in}}{\pgfqpoint{9.292860in}{3.247728in}}%
\pgfpathcurveto{\pgfqpoint{9.300674in}{3.255542in}}{\pgfqpoint{9.305064in}{3.266141in}}{\pgfqpoint{9.305064in}{3.277191in}}%
\pgfpathcurveto{\pgfqpoint{9.305064in}{3.288241in}}{\pgfqpoint{9.300674in}{3.298840in}}{\pgfqpoint{9.292860in}{3.306654in}}%
\pgfpathcurveto{\pgfqpoint{9.285046in}{3.314467in}}{\pgfqpoint{9.274447in}{3.318857in}}{\pgfqpoint{9.263397in}{3.318857in}}%
\pgfpathcurveto{\pgfqpoint{9.252347in}{3.318857in}}{\pgfqpoint{9.241748in}{3.314467in}}{\pgfqpoint{9.233934in}{3.306654in}}%
\pgfpathcurveto{\pgfqpoint{9.226121in}{3.298840in}}{\pgfqpoint{9.221730in}{3.288241in}}{\pgfqpoint{9.221730in}{3.277191in}}%
\pgfpathcurveto{\pgfqpoint{9.221730in}{3.266141in}}{\pgfqpoint{9.226121in}{3.255542in}}{\pgfqpoint{9.233934in}{3.247728in}}%
\pgfpathcurveto{\pgfqpoint{9.241748in}{3.239914in}}{\pgfqpoint{9.252347in}{3.235524in}}{\pgfqpoint{9.263397in}{3.235524in}}%
\pgfpathlineto{\pgfqpoint{9.263397in}{3.235524in}}%
\pgfpathclose%
\pgfusepath{stroke}%
\end{pgfscope}%
\begin{pgfscope}%
\pgfpathrectangle{\pgfqpoint{7.512535in}{0.437222in}}{\pgfqpoint{6.275590in}{5.159444in}}%
\pgfusepath{clip}%
\pgfsetbuttcap%
\pgfsetroundjoin%
\pgfsetlinewidth{1.003750pt}%
\definecolor{currentstroke}{rgb}{0.827451,0.827451,0.827451}%
\pgfsetstrokecolor{currentstroke}%
\pgfsetstrokeopacity{0.800000}%
\pgfsetdash{}{0pt}%
\pgfpathmoveto{\pgfqpoint{12.516073in}{5.554140in}}%
\pgfpathcurveto{\pgfqpoint{12.527123in}{5.554140in}}{\pgfqpoint{12.537722in}{5.558530in}}{\pgfqpoint{12.545536in}{5.566344in}}%
\pgfpathcurveto{\pgfqpoint{12.553349in}{5.574157in}}{\pgfqpoint{12.557740in}{5.584756in}}{\pgfqpoint{12.557740in}{5.595806in}}%
\pgfpathcurveto{\pgfqpoint{12.557740in}{5.606857in}}{\pgfqpoint{12.553349in}{5.617456in}}{\pgfqpoint{12.545536in}{5.625269in}}%
\pgfpathcurveto{\pgfqpoint{12.537722in}{5.633083in}}{\pgfqpoint{12.527123in}{5.637473in}}{\pgfqpoint{12.516073in}{5.637473in}}%
\pgfpathcurveto{\pgfqpoint{12.505023in}{5.637473in}}{\pgfqpoint{12.494424in}{5.633083in}}{\pgfqpoint{12.486610in}{5.625269in}}%
\pgfpathcurveto{\pgfqpoint{12.478797in}{5.617456in}}{\pgfqpoint{12.474406in}{5.606857in}}{\pgfqpoint{12.474406in}{5.595806in}}%
\pgfpathcurveto{\pgfqpoint{12.474406in}{5.584756in}}{\pgfqpoint{12.478797in}{5.574157in}}{\pgfqpoint{12.486610in}{5.566344in}}%
\pgfpathcurveto{\pgfqpoint{12.494424in}{5.558530in}}{\pgfqpoint{12.505023in}{5.554140in}}{\pgfqpoint{12.516073in}{5.554140in}}%
\pgfpathlineto{\pgfqpoint{12.516073in}{5.554140in}}%
\pgfpathclose%
\pgfusepath{stroke}%
\end{pgfscope}%
\begin{pgfscope}%
\pgfpathrectangle{\pgfqpoint{7.512535in}{0.437222in}}{\pgfqpoint{6.275590in}{5.159444in}}%
\pgfusepath{clip}%
\pgfsetbuttcap%
\pgfsetroundjoin%
\pgfsetlinewidth{1.003750pt}%
\definecolor{currentstroke}{rgb}{0.827451,0.827451,0.827451}%
\pgfsetstrokecolor{currentstroke}%
\pgfsetstrokeopacity{0.800000}%
\pgfsetdash{}{0pt}%
\pgfpathmoveto{\pgfqpoint{7.983939in}{0.849223in}}%
\pgfpathcurveto{\pgfqpoint{7.994989in}{0.849223in}}{\pgfqpoint{8.005588in}{0.853614in}}{\pgfqpoint{8.013401in}{0.861427in}}%
\pgfpathcurveto{\pgfqpoint{8.021215in}{0.869241in}}{\pgfqpoint{8.025605in}{0.879840in}}{\pgfqpoint{8.025605in}{0.890890in}}%
\pgfpathcurveto{\pgfqpoint{8.025605in}{0.901940in}}{\pgfqpoint{8.021215in}{0.912539in}}{\pgfqpoint{8.013401in}{0.920353in}}%
\pgfpathcurveto{\pgfqpoint{8.005588in}{0.928167in}}{\pgfqpoint{7.994989in}{0.932557in}}{\pgfqpoint{7.983939in}{0.932557in}}%
\pgfpathcurveto{\pgfqpoint{7.972888in}{0.932557in}}{\pgfqpoint{7.962289in}{0.928167in}}{\pgfqpoint{7.954476in}{0.920353in}}%
\pgfpathcurveto{\pgfqpoint{7.946662in}{0.912539in}}{\pgfqpoint{7.942272in}{0.901940in}}{\pgfqpoint{7.942272in}{0.890890in}}%
\pgfpathcurveto{\pgfqpoint{7.942272in}{0.879840in}}{\pgfqpoint{7.946662in}{0.869241in}}{\pgfqpoint{7.954476in}{0.861427in}}%
\pgfpathcurveto{\pgfqpoint{7.962289in}{0.853614in}}{\pgfqpoint{7.972888in}{0.849223in}}{\pgfqpoint{7.983939in}{0.849223in}}%
\pgfpathlineto{\pgfqpoint{7.983939in}{0.849223in}}%
\pgfpathclose%
\pgfusepath{stroke}%
\end{pgfscope}%
\begin{pgfscope}%
\pgfpathrectangle{\pgfqpoint{7.512535in}{0.437222in}}{\pgfqpoint{6.275590in}{5.159444in}}%
\pgfusepath{clip}%
\pgfsetbuttcap%
\pgfsetroundjoin%
\pgfsetlinewidth{1.003750pt}%
\definecolor{currentstroke}{rgb}{0.827451,0.827451,0.827451}%
\pgfsetstrokecolor{currentstroke}%
\pgfsetstrokeopacity{0.800000}%
\pgfsetdash{}{0pt}%
\pgfpathmoveto{\pgfqpoint{9.848678in}{4.313606in}}%
\pgfpathcurveto{\pgfqpoint{9.859728in}{4.313606in}}{\pgfqpoint{9.870327in}{4.317996in}}{\pgfqpoint{9.878141in}{4.325810in}}%
\pgfpathcurveto{\pgfqpoint{9.885955in}{4.333623in}}{\pgfqpoint{9.890345in}{4.344222in}}{\pgfqpoint{9.890345in}{4.355272in}}%
\pgfpathcurveto{\pgfqpoint{9.890345in}{4.366323in}}{\pgfqpoint{9.885955in}{4.376922in}}{\pgfqpoint{9.878141in}{4.384735in}}%
\pgfpathcurveto{\pgfqpoint{9.870327in}{4.392549in}}{\pgfqpoint{9.859728in}{4.396939in}}{\pgfqpoint{9.848678in}{4.396939in}}%
\pgfpathcurveto{\pgfqpoint{9.837628in}{4.396939in}}{\pgfqpoint{9.827029in}{4.392549in}}{\pgfqpoint{9.819216in}{4.384735in}}%
\pgfpathcurveto{\pgfqpoint{9.811402in}{4.376922in}}{\pgfqpoint{9.807012in}{4.366323in}}{\pgfqpoint{9.807012in}{4.355272in}}%
\pgfpathcurveto{\pgfqpoint{9.807012in}{4.344222in}}{\pgfqpoint{9.811402in}{4.333623in}}{\pgfqpoint{9.819216in}{4.325810in}}%
\pgfpathcurveto{\pgfqpoint{9.827029in}{4.317996in}}{\pgfqpoint{9.837628in}{4.313606in}}{\pgfqpoint{9.848678in}{4.313606in}}%
\pgfpathlineto{\pgfqpoint{9.848678in}{4.313606in}}%
\pgfpathclose%
\pgfusepath{stroke}%
\end{pgfscope}%
\begin{pgfscope}%
\pgfpathrectangle{\pgfqpoint{7.512535in}{0.437222in}}{\pgfqpoint{6.275590in}{5.159444in}}%
\pgfusepath{clip}%
\pgfsetbuttcap%
\pgfsetroundjoin%
\pgfsetlinewidth{1.003750pt}%
\definecolor{currentstroke}{rgb}{0.827451,0.827451,0.827451}%
\pgfsetstrokecolor{currentstroke}%
\pgfsetstrokeopacity{0.800000}%
\pgfsetdash{}{0pt}%
\pgfpathmoveto{\pgfqpoint{11.536398in}{5.324795in}}%
\pgfpathcurveto{\pgfqpoint{11.547448in}{5.324795in}}{\pgfqpoint{11.558047in}{5.329185in}}{\pgfqpoint{11.565860in}{5.336999in}}%
\pgfpathcurveto{\pgfqpoint{11.573674in}{5.344812in}}{\pgfqpoint{11.578064in}{5.355411in}}{\pgfqpoint{11.578064in}{5.366462in}}%
\pgfpathcurveto{\pgfqpoint{11.578064in}{5.377512in}}{\pgfqpoint{11.573674in}{5.388111in}}{\pgfqpoint{11.565860in}{5.395924in}}%
\pgfpathcurveto{\pgfqpoint{11.558047in}{5.403738in}}{\pgfqpoint{11.547448in}{5.408128in}}{\pgfqpoint{11.536398in}{5.408128in}}%
\pgfpathcurveto{\pgfqpoint{11.525347in}{5.408128in}}{\pgfqpoint{11.514748in}{5.403738in}}{\pgfqpoint{11.506935in}{5.395924in}}%
\pgfpathcurveto{\pgfqpoint{11.499121in}{5.388111in}}{\pgfqpoint{11.494731in}{5.377512in}}{\pgfqpoint{11.494731in}{5.366462in}}%
\pgfpathcurveto{\pgfqpoint{11.494731in}{5.355411in}}{\pgfqpoint{11.499121in}{5.344812in}}{\pgfqpoint{11.506935in}{5.336999in}}%
\pgfpathcurveto{\pgfqpoint{11.514748in}{5.329185in}}{\pgfqpoint{11.525347in}{5.324795in}}{\pgfqpoint{11.536398in}{5.324795in}}%
\pgfpathlineto{\pgfqpoint{11.536398in}{5.324795in}}%
\pgfpathclose%
\pgfusepath{stroke}%
\end{pgfscope}%
\begin{pgfscope}%
\pgfpathrectangle{\pgfqpoint{7.512535in}{0.437222in}}{\pgfqpoint{6.275590in}{5.159444in}}%
\pgfusepath{clip}%
\pgfsetbuttcap%
\pgfsetroundjoin%
\pgfsetlinewidth{1.003750pt}%
\definecolor{currentstroke}{rgb}{0.827451,0.827451,0.827451}%
\pgfsetstrokecolor{currentstroke}%
\pgfsetstrokeopacity{0.800000}%
\pgfsetdash{}{0pt}%
\pgfpathmoveto{\pgfqpoint{11.262853in}{5.071024in}}%
\pgfpathcurveto{\pgfqpoint{11.273903in}{5.071024in}}{\pgfqpoint{11.284502in}{5.075415in}}{\pgfqpoint{11.292316in}{5.083228in}}%
\pgfpathcurveto{\pgfqpoint{11.300130in}{5.091042in}}{\pgfqpoint{11.304520in}{5.101641in}}{\pgfqpoint{11.304520in}{5.112691in}}%
\pgfpathcurveto{\pgfqpoint{11.304520in}{5.123741in}}{\pgfqpoint{11.300130in}{5.134340in}}{\pgfqpoint{11.292316in}{5.142154in}}%
\pgfpathcurveto{\pgfqpoint{11.284502in}{5.149967in}}{\pgfqpoint{11.273903in}{5.154358in}}{\pgfqpoint{11.262853in}{5.154358in}}%
\pgfpathcurveto{\pgfqpoint{11.251803in}{5.154358in}}{\pgfqpoint{11.241204in}{5.149967in}}{\pgfqpoint{11.233391in}{5.142154in}}%
\pgfpathcurveto{\pgfqpoint{11.225577in}{5.134340in}}{\pgfqpoint{11.221187in}{5.123741in}}{\pgfqpoint{11.221187in}{5.112691in}}%
\pgfpathcurveto{\pgfqpoint{11.221187in}{5.101641in}}{\pgfqpoint{11.225577in}{5.091042in}}{\pgfqpoint{11.233391in}{5.083228in}}%
\pgfpathcurveto{\pgfqpoint{11.241204in}{5.075415in}}{\pgfqpoint{11.251803in}{5.071024in}}{\pgfqpoint{11.262853in}{5.071024in}}%
\pgfpathlineto{\pgfqpoint{11.262853in}{5.071024in}}%
\pgfpathclose%
\pgfusepath{stroke}%
\end{pgfscope}%
\begin{pgfscope}%
\pgfpathrectangle{\pgfqpoint{7.512535in}{0.437222in}}{\pgfqpoint{6.275590in}{5.159444in}}%
\pgfusepath{clip}%
\pgfsetbuttcap%
\pgfsetroundjoin%
\pgfsetlinewidth{1.003750pt}%
\definecolor{currentstroke}{rgb}{0.827451,0.827451,0.827451}%
\pgfsetstrokecolor{currentstroke}%
\pgfsetstrokeopacity{0.800000}%
\pgfsetdash{}{0pt}%
\pgfpathmoveto{\pgfqpoint{9.198300in}{3.513355in}}%
\pgfpathcurveto{\pgfqpoint{9.209350in}{3.513355in}}{\pgfqpoint{9.219949in}{3.517745in}}{\pgfqpoint{9.227763in}{3.525559in}}%
\pgfpathcurveto{\pgfqpoint{9.235577in}{3.533372in}}{\pgfqpoint{9.239967in}{3.543971in}}{\pgfqpoint{9.239967in}{3.555021in}}%
\pgfpathcurveto{\pgfqpoint{9.239967in}{3.566071in}}{\pgfqpoint{9.235577in}{3.576670in}}{\pgfqpoint{9.227763in}{3.584484in}}%
\pgfpathcurveto{\pgfqpoint{9.219949in}{3.592298in}}{\pgfqpoint{9.209350in}{3.596688in}}{\pgfqpoint{9.198300in}{3.596688in}}%
\pgfpathcurveto{\pgfqpoint{9.187250in}{3.596688in}}{\pgfqpoint{9.176651in}{3.592298in}}{\pgfqpoint{9.168838in}{3.584484in}}%
\pgfpathcurveto{\pgfqpoint{9.161024in}{3.576670in}}{\pgfqpoint{9.156634in}{3.566071in}}{\pgfqpoint{9.156634in}{3.555021in}}%
\pgfpathcurveto{\pgfqpoint{9.156634in}{3.543971in}}{\pgfqpoint{9.161024in}{3.533372in}}{\pgfqpoint{9.168838in}{3.525559in}}%
\pgfpathcurveto{\pgfqpoint{9.176651in}{3.517745in}}{\pgfqpoint{9.187250in}{3.513355in}}{\pgfqpoint{9.198300in}{3.513355in}}%
\pgfpathlineto{\pgfqpoint{9.198300in}{3.513355in}}%
\pgfpathclose%
\pgfusepath{stroke}%
\end{pgfscope}%
\begin{pgfscope}%
\pgfpathrectangle{\pgfqpoint{7.512535in}{0.437222in}}{\pgfqpoint{6.275590in}{5.159444in}}%
\pgfusepath{clip}%
\pgfsetbuttcap%
\pgfsetroundjoin%
\pgfsetlinewidth{1.003750pt}%
\definecolor{currentstroke}{rgb}{0.827451,0.827451,0.827451}%
\pgfsetstrokecolor{currentstroke}%
\pgfsetstrokeopacity{0.800000}%
\pgfsetdash{}{0pt}%
\pgfpathmoveto{\pgfqpoint{8.446507in}{1.377688in}}%
\pgfpathcurveto{\pgfqpoint{8.457557in}{1.377688in}}{\pgfqpoint{8.468156in}{1.382078in}}{\pgfqpoint{8.475969in}{1.389892in}}%
\pgfpathcurveto{\pgfqpoint{8.483783in}{1.397706in}}{\pgfqpoint{8.488173in}{1.408305in}}{\pgfqpoint{8.488173in}{1.419355in}}%
\pgfpathcurveto{\pgfqpoint{8.488173in}{1.430405in}}{\pgfqpoint{8.483783in}{1.441004in}}{\pgfqpoint{8.475969in}{1.448818in}}%
\pgfpathcurveto{\pgfqpoint{8.468156in}{1.456631in}}{\pgfqpoint{8.457557in}{1.461022in}}{\pgfqpoint{8.446507in}{1.461022in}}%
\pgfpathcurveto{\pgfqpoint{8.435457in}{1.461022in}}{\pgfqpoint{8.424857in}{1.456631in}}{\pgfqpoint{8.417044in}{1.448818in}}%
\pgfpathcurveto{\pgfqpoint{8.409230in}{1.441004in}}{\pgfqpoint{8.404840in}{1.430405in}}{\pgfqpoint{8.404840in}{1.419355in}}%
\pgfpathcurveto{\pgfqpoint{8.404840in}{1.408305in}}{\pgfqpoint{8.409230in}{1.397706in}}{\pgfqpoint{8.417044in}{1.389892in}}%
\pgfpathcurveto{\pgfqpoint{8.424857in}{1.382078in}}{\pgfqpoint{8.435457in}{1.377688in}}{\pgfqpoint{8.446507in}{1.377688in}}%
\pgfpathlineto{\pgfqpoint{8.446507in}{1.377688in}}%
\pgfpathclose%
\pgfusepath{stroke}%
\end{pgfscope}%
\begin{pgfscope}%
\pgfpathrectangle{\pgfqpoint{7.512535in}{0.437222in}}{\pgfqpoint{6.275590in}{5.159444in}}%
\pgfusepath{clip}%
\pgfsetbuttcap%
\pgfsetroundjoin%
\pgfsetlinewidth{1.003750pt}%
\definecolor{currentstroke}{rgb}{0.827451,0.827451,0.827451}%
\pgfsetstrokecolor{currentstroke}%
\pgfsetstrokeopacity{0.800000}%
\pgfsetdash{}{0pt}%
\pgfpathmoveto{\pgfqpoint{8.714151in}{3.385314in}}%
\pgfpathcurveto{\pgfqpoint{8.725201in}{3.385314in}}{\pgfqpoint{8.735800in}{3.389704in}}{\pgfqpoint{8.743614in}{3.397518in}}%
\pgfpathcurveto{\pgfqpoint{8.751428in}{3.405332in}}{\pgfqpoint{8.755818in}{3.415931in}}{\pgfqpoint{8.755818in}{3.426981in}}%
\pgfpathcurveto{\pgfqpoint{8.755818in}{3.438031in}}{\pgfqpoint{8.751428in}{3.448630in}}{\pgfqpoint{8.743614in}{3.456444in}}%
\pgfpathcurveto{\pgfqpoint{8.735800in}{3.464257in}}{\pgfqpoint{8.725201in}{3.468647in}}{\pgfqpoint{8.714151in}{3.468647in}}%
\pgfpathcurveto{\pgfqpoint{8.703101in}{3.468647in}}{\pgfqpoint{8.692502in}{3.464257in}}{\pgfqpoint{8.684689in}{3.456444in}}%
\pgfpathcurveto{\pgfqpoint{8.676875in}{3.448630in}}{\pgfqpoint{8.672485in}{3.438031in}}{\pgfqpoint{8.672485in}{3.426981in}}%
\pgfpathcurveto{\pgfqpoint{8.672485in}{3.415931in}}{\pgfqpoint{8.676875in}{3.405332in}}{\pgfqpoint{8.684689in}{3.397518in}}%
\pgfpathcurveto{\pgfqpoint{8.692502in}{3.389704in}}{\pgfqpoint{8.703101in}{3.385314in}}{\pgfqpoint{8.714151in}{3.385314in}}%
\pgfpathlineto{\pgfqpoint{8.714151in}{3.385314in}}%
\pgfpathclose%
\pgfusepath{stroke}%
\end{pgfscope}%
\begin{pgfscope}%
\pgfpathrectangle{\pgfqpoint{7.512535in}{0.437222in}}{\pgfqpoint{6.275590in}{5.159444in}}%
\pgfusepath{clip}%
\pgfsetbuttcap%
\pgfsetroundjoin%
\pgfsetlinewidth{1.003750pt}%
\definecolor{currentstroke}{rgb}{0.827451,0.827451,0.827451}%
\pgfsetstrokecolor{currentstroke}%
\pgfsetstrokeopacity{0.800000}%
\pgfsetdash{}{0pt}%
\pgfpathmoveto{\pgfqpoint{8.277312in}{1.563305in}}%
\pgfpathcurveto{\pgfqpoint{8.288363in}{1.563305in}}{\pgfqpoint{8.298962in}{1.567695in}}{\pgfqpoint{8.306775in}{1.575509in}}%
\pgfpathcurveto{\pgfqpoint{8.314589in}{1.583322in}}{\pgfqpoint{8.318979in}{1.593921in}}{\pgfqpoint{8.318979in}{1.604971in}}%
\pgfpathcurveto{\pgfqpoint{8.318979in}{1.616021in}}{\pgfqpoint{8.314589in}{1.626621in}}{\pgfqpoint{8.306775in}{1.634434in}}%
\pgfpathcurveto{\pgfqpoint{8.298962in}{1.642248in}}{\pgfqpoint{8.288363in}{1.646638in}}{\pgfqpoint{8.277312in}{1.646638in}}%
\pgfpathcurveto{\pgfqpoint{8.266262in}{1.646638in}}{\pgfqpoint{8.255663in}{1.642248in}}{\pgfqpoint{8.247850in}{1.634434in}}%
\pgfpathcurveto{\pgfqpoint{8.240036in}{1.626621in}}{\pgfqpoint{8.235646in}{1.616021in}}{\pgfqpoint{8.235646in}{1.604971in}}%
\pgfpathcurveto{\pgfqpoint{8.235646in}{1.593921in}}{\pgfqpoint{8.240036in}{1.583322in}}{\pgfqpoint{8.247850in}{1.575509in}}%
\pgfpathcurveto{\pgfqpoint{8.255663in}{1.567695in}}{\pgfqpoint{8.266262in}{1.563305in}}{\pgfqpoint{8.277312in}{1.563305in}}%
\pgfpathlineto{\pgfqpoint{8.277312in}{1.563305in}}%
\pgfpathclose%
\pgfusepath{stroke}%
\end{pgfscope}%
\begin{pgfscope}%
\pgfpathrectangle{\pgfqpoint{7.512535in}{0.437222in}}{\pgfqpoint{6.275590in}{5.159444in}}%
\pgfusepath{clip}%
\pgfsetbuttcap%
\pgfsetroundjoin%
\pgfsetlinewidth{1.003750pt}%
\definecolor{currentstroke}{rgb}{0.827451,0.827451,0.827451}%
\pgfsetstrokecolor{currentstroke}%
\pgfsetstrokeopacity{0.800000}%
\pgfsetdash{}{0pt}%
\pgfpathmoveto{\pgfqpoint{12.342968in}{5.553091in}}%
\pgfpathcurveto{\pgfqpoint{12.354019in}{5.553091in}}{\pgfqpoint{12.364618in}{5.557481in}}{\pgfqpoint{12.372431in}{5.565295in}}%
\pgfpathcurveto{\pgfqpoint{12.380245in}{5.573108in}}{\pgfqpoint{12.384635in}{5.583707in}}{\pgfqpoint{12.384635in}{5.594757in}}%
\pgfpathcurveto{\pgfqpoint{12.384635in}{5.605808in}}{\pgfqpoint{12.380245in}{5.616407in}}{\pgfqpoint{12.372431in}{5.624220in}}%
\pgfpathcurveto{\pgfqpoint{12.364618in}{5.632034in}}{\pgfqpoint{12.354019in}{5.636424in}}{\pgfqpoint{12.342968in}{5.636424in}}%
\pgfpathcurveto{\pgfqpoint{12.331918in}{5.636424in}}{\pgfqpoint{12.321319in}{5.632034in}}{\pgfqpoint{12.313506in}{5.624220in}}%
\pgfpathcurveto{\pgfqpoint{12.305692in}{5.616407in}}{\pgfqpoint{12.301302in}{5.605808in}}{\pgfqpoint{12.301302in}{5.594757in}}%
\pgfpathcurveto{\pgfqpoint{12.301302in}{5.583707in}}{\pgfqpoint{12.305692in}{5.573108in}}{\pgfqpoint{12.313506in}{5.565295in}}%
\pgfpathcurveto{\pgfqpoint{12.321319in}{5.557481in}}{\pgfqpoint{12.331918in}{5.553091in}}{\pgfqpoint{12.342968in}{5.553091in}}%
\pgfpathlineto{\pgfqpoint{12.342968in}{5.553091in}}%
\pgfpathclose%
\pgfusepath{stroke}%
\end{pgfscope}%
\begin{pgfscope}%
\pgfpathrectangle{\pgfqpoint{7.512535in}{0.437222in}}{\pgfqpoint{6.275590in}{5.159444in}}%
\pgfusepath{clip}%
\pgfsetbuttcap%
\pgfsetroundjoin%
\pgfsetlinewidth{1.003750pt}%
\definecolor{currentstroke}{rgb}{0.827451,0.827451,0.827451}%
\pgfsetstrokecolor{currentstroke}%
\pgfsetstrokeopacity{0.800000}%
\pgfsetdash{}{0pt}%
\pgfpathmoveto{\pgfqpoint{12.650243in}{5.481847in}}%
\pgfpathcurveto{\pgfqpoint{12.661293in}{5.481847in}}{\pgfqpoint{12.671892in}{5.486238in}}{\pgfqpoint{12.679706in}{5.494051in}}%
\pgfpathcurveto{\pgfqpoint{12.687519in}{5.501865in}}{\pgfqpoint{12.691909in}{5.512464in}}{\pgfqpoint{12.691909in}{5.523514in}}%
\pgfpathcurveto{\pgfqpoint{12.691909in}{5.534564in}}{\pgfqpoint{12.687519in}{5.545163in}}{\pgfqpoint{12.679706in}{5.552977in}}%
\pgfpathcurveto{\pgfqpoint{12.671892in}{5.560791in}}{\pgfqpoint{12.661293in}{5.565181in}}{\pgfqpoint{12.650243in}{5.565181in}}%
\pgfpathcurveto{\pgfqpoint{12.639193in}{5.565181in}}{\pgfqpoint{12.628594in}{5.560791in}}{\pgfqpoint{12.620780in}{5.552977in}}%
\pgfpathcurveto{\pgfqpoint{12.612966in}{5.545163in}}{\pgfqpoint{12.608576in}{5.534564in}}{\pgfqpoint{12.608576in}{5.523514in}}%
\pgfpathcurveto{\pgfqpoint{12.608576in}{5.512464in}}{\pgfqpoint{12.612966in}{5.501865in}}{\pgfqpoint{12.620780in}{5.494051in}}%
\pgfpathcurveto{\pgfqpoint{12.628594in}{5.486238in}}{\pgfqpoint{12.639193in}{5.481847in}}{\pgfqpoint{12.650243in}{5.481847in}}%
\pgfpathlineto{\pgfqpoint{12.650243in}{5.481847in}}%
\pgfpathclose%
\pgfusepath{stroke}%
\end{pgfscope}%
\begin{pgfscope}%
\pgfpathrectangle{\pgfqpoint{7.512535in}{0.437222in}}{\pgfqpoint{6.275590in}{5.159444in}}%
\pgfusepath{clip}%
\pgfsetbuttcap%
\pgfsetroundjoin%
\pgfsetlinewidth{1.003750pt}%
\definecolor{currentstroke}{rgb}{0.827451,0.827451,0.827451}%
\pgfsetstrokecolor{currentstroke}%
\pgfsetstrokeopacity{0.800000}%
\pgfsetdash{}{0pt}%
\pgfpathmoveto{\pgfqpoint{10.073045in}{4.237670in}}%
\pgfpathcurveto{\pgfqpoint{10.084095in}{4.237670in}}{\pgfqpoint{10.094694in}{4.242060in}}{\pgfqpoint{10.102508in}{4.249874in}}%
\pgfpathcurveto{\pgfqpoint{10.110321in}{4.257687in}}{\pgfqpoint{10.114712in}{4.268286in}}{\pgfqpoint{10.114712in}{4.279336in}}%
\pgfpathcurveto{\pgfqpoint{10.114712in}{4.290387in}}{\pgfqpoint{10.110321in}{4.300986in}}{\pgfqpoint{10.102508in}{4.308799in}}%
\pgfpathcurveto{\pgfqpoint{10.094694in}{4.316613in}}{\pgfqpoint{10.084095in}{4.321003in}}{\pgfqpoint{10.073045in}{4.321003in}}%
\pgfpathcurveto{\pgfqpoint{10.061995in}{4.321003in}}{\pgfqpoint{10.051396in}{4.316613in}}{\pgfqpoint{10.043582in}{4.308799in}}%
\pgfpathcurveto{\pgfqpoint{10.035769in}{4.300986in}}{\pgfqpoint{10.031378in}{4.290387in}}{\pgfqpoint{10.031378in}{4.279336in}}%
\pgfpathcurveto{\pgfqpoint{10.031378in}{4.268286in}}{\pgfqpoint{10.035769in}{4.257687in}}{\pgfqpoint{10.043582in}{4.249874in}}%
\pgfpathcurveto{\pgfqpoint{10.051396in}{4.242060in}}{\pgfqpoint{10.061995in}{4.237670in}}{\pgfqpoint{10.073045in}{4.237670in}}%
\pgfpathlineto{\pgfqpoint{10.073045in}{4.237670in}}%
\pgfpathclose%
\pgfusepath{stroke}%
\end{pgfscope}%
\begin{pgfscope}%
\pgfpathrectangle{\pgfqpoint{7.512535in}{0.437222in}}{\pgfqpoint{6.275590in}{5.159444in}}%
\pgfusepath{clip}%
\pgfsetbuttcap%
\pgfsetroundjoin%
\pgfsetlinewidth{1.003750pt}%
\definecolor{currentstroke}{rgb}{0.827451,0.827451,0.827451}%
\pgfsetstrokecolor{currentstroke}%
\pgfsetstrokeopacity{0.800000}%
\pgfsetdash{}{0pt}%
\pgfpathmoveto{\pgfqpoint{7.679894in}{0.493855in}}%
\pgfpathcurveto{\pgfqpoint{7.690944in}{0.493855in}}{\pgfqpoint{7.701543in}{0.498245in}}{\pgfqpoint{7.709357in}{0.506059in}}%
\pgfpathcurveto{\pgfqpoint{7.717170in}{0.513872in}}{\pgfqpoint{7.721560in}{0.524471in}}{\pgfqpoint{7.721560in}{0.535522in}}%
\pgfpathcurveto{\pgfqpoint{7.721560in}{0.546572in}}{\pgfqpoint{7.717170in}{0.557171in}}{\pgfqpoint{7.709357in}{0.564984in}}%
\pgfpathcurveto{\pgfqpoint{7.701543in}{0.572798in}}{\pgfqpoint{7.690944in}{0.577188in}}{\pgfqpoint{7.679894in}{0.577188in}}%
\pgfpathcurveto{\pgfqpoint{7.668844in}{0.577188in}}{\pgfqpoint{7.658245in}{0.572798in}}{\pgfqpoint{7.650431in}{0.564984in}}%
\pgfpathcurveto{\pgfqpoint{7.642617in}{0.557171in}}{\pgfqpoint{7.638227in}{0.546572in}}{\pgfqpoint{7.638227in}{0.535522in}}%
\pgfpathcurveto{\pgfqpoint{7.638227in}{0.524471in}}{\pgfqpoint{7.642617in}{0.513872in}}{\pgfqpoint{7.650431in}{0.506059in}}%
\pgfpathcurveto{\pgfqpoint{7.658245in}{0.498245in}}{\pgfqpoint{7.668844in}{0.493855in}}{\pgfqpoint{7.679894in}{0.493855in}}%
\pgfpathlineto{\pgfqpoint{7.679894in}{0.493855in}}%
\pgfpathclose%
\pgfusepath{stroke}%
\end{pgfscope}%
\begin{pgfscope}%
\pgfpathrectangle{\pgfqpoint{7.512535in}{0.437222in}}{\pgfqpoint{6.275590in}{5.159444in}}%
\pgfusepath{clip}%
\pgfsetbuttcap%
\pgfsetroundjoin%
\pgfsetlinewidth{1.003750pt}%
\definecolor{currentstroke}{rgb}{0.827451,0.827451,0.827451}%
\pgfsetstrokecolor{currentstroke}%
\pgfsetstrokeopacity{0.800000}%
\pgfsetdash{}{0pt}%
\pgfpathmoveto{\pgfqpoint{7.711198in}{0.454672in}}%
\pgfpathcurveto{\pgfqpoint{7.722248in}{0.454672in}}{\pgfqpoint{7.732847in}{0.459062in}}{\pgfqpoint{7.740661in}{0.466876in}}%
\pgfpathcurveto{\pgfqpoint{7.748474in}{0.474690in}}{\pgfqpoint{7.752865in}{0.485289in}}{\pgfqpoint{7.752865in}{0.496339in}}%
\pgfpathcurveto{\pgfqpoint{7.752865in}{0.507389in}}{\pgfqpoint{7.748474in}{0.517988in}}{\pgfqpoint{7.740661in}{0.525801in}}%
\pgfpathcurveto{\pgfqpoint{7.732847in}{0.533615in}}{\pgfqpoint{7.722248in}{0.538005in}}{\pgfqpoint{7.711198in}{0.538005in}}%
\pgfpathcurveto{\pgfqpoint{7.700148in}{0.538005in}}{\pgfqpoint{7.689549in}{0.533615in}}{\pgfqpoint{7.681735in}{0.525801in}}%
\pgfpathcurveto{\pgfqpoint{7.673921in}{0.517988in}}{\pgfqpoint{7.669531in}{0.507389in}}{\pgfqpoint{7.669531in}{0.496339in}}%
\pgfpathcurveto{\pgfqpoint{7.669531in}{0.485289in}}{\pgfqpoint{7.673921in}{0.474690in}}{\pgfqpoint{7.681735in}{0.466876in}}%
\pgfpathcurveto{\pgfqpoint{7.689549in}{0.459062in}}{\pgfqpoint{7.700148in}{0.454672in}}{\pgfqpoint{7.711198in}{0.454672in}}%
\pgfpathlineto{\pgfqpoint{7.711198in}{0.454672in}}%
\pgfpathclose%
\pgfusepath{stroke}%
\end{pgfscope}%
\begin{pgfscope}%
\pgfpathrectangle{\pgfqpoint{7.512535in}{0.437222in}}{\pgfqpoint{6.275590in}{5.159444in}}%
\pgfusepath{clip}%
\pgfsetbuttcap%
\pgfsetroundjoin%
\pgfsetlinewidth{1.003750pt}%
\definecolor{currentstroke}{rgb}{0.827451,0.827451,0.827451}%
\pgfsetstrokecolor{currentstroke}%
\pgfsetstrokeopacity{0.800000}%
\pgfsetdash{}{0pt}%
\pgfpathmoveto{\pgfqpoint{12.242042in}{5.526352in}}%
\pgfpathcurveto{\pgfqpoint{12.253092in}{5.526352in}}{\pgfqpoint{12.263692in}{5.530742in}}{\pgfqpoint{12.271505in}{5.538555in}}%
\pgfpathcurveto{\pgfqpoint{12.279319in}{5.546369in}}{\pgfqpoint{12.283709in}{5.556968in}}{\pgfqpoint{12.283709in}{5.568018in}}%
\pgfpathcurveto{\pgfqpoint{12.283709in}{5.579068in}}{\pgfqpoint{12.279319in}{5.589667in}}{\pgfqpoint{12.271505in}{5.597481in}}%
\pgfpathcurveto{\pgfqpoint{12.263692in}{5.605295in}}{\pgfqpoint{12.253092in}{5.609685in}}{\pgfqpoint{12.242042in}{5.609685in}}%
\pgfpathcurveto{\pgfqpoint{12.230992in}{5.609685in}}{\pgfqpoint{12.220393in}{5.605295in}}{\pgfqpoint{12.212580in}{5.597481in}}%
\pgfpathcurveto{\pgfqpoint{12.204766in}{5.589667in}}{\pgfqpoint{12.200376in}{5.579068in}}{\pgfqpoint{12.200376in}{5.568018in}}%
\pgfpathcurveto{\pgfqpoint{12.200376in}{5.556968in}}{\pgfqpoint{12.204766in}{5.546369in}}{\pgfqpoint{12.212580in}{5.538555in}}%
\pgfpathcurveto{\pgfqpoint{12.220393in}{5.530742in}}{\pgfqpoint{12.230992in}{5.526352in}}{\pgfqpoint{12.242042in}{5.526352in}}%
\pgfpathlineto{\pgfqpoint{12.242042in}{5.526352in}}%
\pgfpathclose%
\pgfusepath{stroke}%
\end{pgfscope}%
\begin{pgfscope}%
\pgfpathrectangle{\pgfqpoint{7.512535in}{0.437222in}}{\pgfqpoint{6.275590in}{5.159444in}}%
\pgfusepath{clip}%
\pgfsetbuttcap%
\pgfsetroundjoin%
\pgfsetlinewidth{1.003750pt}%
\definecolor{currentstroke}{rgb}{1.000000,0.000000,0.000000}%
\pgfsetstrokecolor{currentstroke}%
\pgfsetdash{}{0pt}%
\pgfpathmoveto{\pgfqpoint{7.512535in}{0.395556in}}%
\pgfpathcurveto{\pgfqpoint{7.523585in}{0.395556in}}{\pgfqpoint{7.534184in}{0.399946in}}{\pgfqpoint{7.541998in}{0.407759in}}%
\pgfpathcurveto{\pgfqpoint{7.549811in}{0.415573in}}{\pgfqpoint{7.554201in}{0.426172in}}{\pgfqpoint{7.554201in}{0.437222in}}%
\pgfpathcurveto{\pgfqpoint{7.554201in}{0.448272in}}{\pgfqpoint{7.549811in}{0.458871in}}{\pgfqpoint{7.541998in}{0.466685in}}%
\pgfpathcurveto{\pgfqpoint{7.534184in}{0.474499in}}{\pgfqpoint{7.523585in}{0.478889in}}{\pgfqpoint{7.512535in}{0.478889in}}%
\pgfpathcurveto{\pgfqpoint{7.501485in}{0.478889in}}{\pgfqpoint{7.490886in}{0.474499in}}{\pgfqpoint{7.483072in}{0.466685in}}%
\pgfpathcurveto{\pgfqpoint{7.475258in}{0.458871in}}{\pgfqpoint{7.470868in}{0.448272in}}{\pgfqpoint{7.470868in}{0.437222in}}%
\pgfpathcurveto{\pgfqpoint{7.470868in}{0.426172in}}{\pgfqpoint{7.475258in}{0.415573in}}{\pgfqpoint{7.483072in}{0.407759in}}%
\pgfpathcurveto{\pgfqpoint{7.490886in}{0.399946in}}{\pgfqpoint{7.501485in}{0.395556in}}{\pgfqpoint{7.512535in}{0.395556in}}%
\pgfusepath{stroke}%
\end{pgfscope}%
\begin{pgfscope}%
\pgfpathrectangle{\pgfqpoint{7.512535in}{0.437222in}}{\pgfqpoint{6.275590in}{5.159444in}}%
\pgfusepath{clip}%
\pgfsetbuttcap%
\pgfsetroundjoin%
\pgfsetlinewidth{1.003750pt}%
\definecolor{currentstroke}{rgb}{1.000000,0.000000,0.000000}%
\pgfsetstrokecolor{currentstroke}%
\pgfsetdash{}{0pt}%
\pgfpathmoveto{\pgfqpoint{13.788125in}{5.555000in}}%
\pgfpathcurveto{\pgfqpoint{13.799175in}{5.555000in}}{\pgfqpoint{13.809774in}{5.559390in}}{\pgfqpoint{13.817588in}{5.567204in}}%
\pgfpathcurveto{\pgfqpoint{13.825401in}{5.575018in}}{\pgfqpoint{13.829792in}{5.585617in}}{\pgfqpoint{13.829792in}{5.596667in}}%
\pgfpathcurveto{\pgfqpoint{13.829792in}{5.607717in}}{\pgfqpoint{13.825401in}{5.618316in}}{\pgfqpoint{13.817588in}{5.626129in}}%
\pgfpathcurveto{\pgfqpoint{13.809774in}{5.633943in}}{\pgfqpoint{13.799175in}{5.638333in}}{\pgfqpoint{13.788125in}{5.638333in}}%
\pgfpathcurveto{\pgfqpoint{13.777075in}{5.638333in}}{\pgfqpoint{13.766476in}{5.633943in}}{\pgfqpoint{13.758662in}{5.626129in}}%
\pgfpathcurveto{\pgfqpoint{13.750849in}{5.618316in}}{\pgfqpoint{13.746458in}{5.607717in}}{\pgfqpoint{13.746458in}{5.596667in}}%
\pgfpathcurveto{\pgfqpoint{13.746458in}{5.585617in}}{\pgfqpoint{13.750849in}{5.575018in}}{\pgfqpoint{13.758662in}{5.567204in}}%
\pgfpathcurveto{\pgfqpoint{13.766476in}{5.559390in}}{\pgfqpoint{13.777075in}{5.555000in}}{\pgfqpoint{13.788125in}{5.555000in}}%
\pgfpathlineto{\pgfqpoint{13.788125in}{5.555000in}}%
\pgfpathclose%
\pgfusepath{stroke}%
\end{pgfscope}%
\begin{pgfscope}%
\pgfpathrectangle{\pgfqpoint{7.512535in}{0.437222in}}{\pgfqpoint{6.275590in}{5.159444in}}%
\pgfusepath{clip}%
\pgfsetbuttcap%
\pgfsetroundjoin%
\pgfsetlinewidth{1.003750pt}%
\definecolor{currentstroke}{rgb}{1.000000,0.000000,0.000000}%
\pgfsetstrokecolor{currentstroke}%
\pgfsetdash{}{0pt}%
\pgfpathmoveto{\pgfqpoint{11.186716in}{4.647839in}}%
\pgfpathcurveto{\pgfqpoint{11.197766in}{4.647839in}}{\pgfqpoint{11.208365in}{4.652229in}}{\pgfqpoint{11.216179in}{4.660043in}}%
\pgfpathcurveto{\pgfqpoint{11.223993in}{4.667857in}}{\pgfqpoint{11.228383in}{4.678456in}}{\pgfqpoint{11.228383in}{4.689506in}}%
\pgfpathcurveto{\pgfqpoint{11.228383in}{4.700556in}}{\pgfqpoint{11.223993in}{4.711155in}}{\pgfqpoint{11.216179in}{4.718968in}}%
\pgfpathcurveto{\pgfqpoint{11.208365in}{4.726782in}}{\pgfqpoint{11.197766in}{4.731172in}}{\pgfqpoint{11.186716in}{4.731172in}}%
\pgfpathcurveto{\pgfqpoint{11.175666in}{4.731172in}}{\pgfqpoint{11.165067in}{4.726782in}}{\pgfqpoint{11.157253in}{4.718968in}}%
\pgfpathcurveto{\pgfqpoint{11.149440in}{4.711155in}}{\pgfqpoint{11.145050in}{4.700556in}}{\pgfqpoint{11.145050in}{4.689506in}}%
\pgfpathcurveto{\pgfqpoint{11.145050in}{4.678456in}}{\pgfqpoint{11.149440in}{4.667857in}}{\pgfqpoint{11.157253in}{4.660043in}}%
\pgfpathcurveto{\pgfqpoint{11.165067in}{4.652229in}}{\pgfqpoint{11.175666in}{4.647839in}}{\pgfqpoint{11.186716in}{4.647839in}}%
\pgfpathlineto{\pgfqpoint{11.186716in}{4.647839in}}%
\pgfpathclose%
\pgfusepath{stroke}%
\end{pgfscope}%
\begin{pgfscope}%
\pgfpathrectangle{\pgfqpoint{7.512535in}{0.437222in}}{\pgfqpoint{6.275590in}{5.159444in}}%
\pgfusepath{clip}%
\pgfsetbuttcap%
\pgfsetroundjoin%
\pgfsetlinewidth{1.003750pt}%
\definecolor{currentstroke}{rgb}{1.000000,0.000000,0.000000}%
\pgfsetstrokecolor{currentstroke}%
\pgfsetdash{}{0pt}%
\pgfpathmoveto{\pgfqpoint{7.931779in}{0.838079in}}%
\pgfpathcurveto{\pgfqpoint{7.942829in}{0.838079in}}{\pgfqpoint{7.953428in}{0.842469in}}{\pgfqpoint{7.961242in}{0.850283in}}%
\pgfpathcurveto{\pgfqpoint{7.969056in}{0.858096in}}{\pgfqpoint{7.973446in}{0.868695in}}{\pgfqpoint{7.973446in}{0.879745in}}%
\pgfpathcurveto{\pgfqpoint{7.973446in}{0.890796in}}{\pgfqpoint{7.969056in}{0.901395in}}{\pgfqpoint{7.961242in}{0.909208in}}%
\pgfpathcurveto{\pgfqpoint{7.953428in}{0.917022in}}{\pgfqpoint{7.942829in}{0.921412in}}{\pgfqpoint{7.931779in}{0.921412in}}%
\pgfpathcurveto{\pgfqpoint{7.920729in}{0.921412in}}{\pgfqpoint{7.910130in}{0.917022in}}{\pgfqpoint{7.902316in}{0.909208in}}%
\pgfpathcurveto{\pgfqpoint{7.894503in}{0.901395in}}{\pgfqpoint{7.890113in}{0.890796in}}{\pgfqpoint{7.890113in}{0.879745in}}%
\pgfpathcurveto{\pgfqpoint{7.890113in}{0.868695in}}{\pgfqpoint{7.894503in}{0.858096in}}{\pgfqpoint{7.902316in}{0.850283in}}%
\pgfpathcurveto{\pgfqpoint{7.910130in}{0.842469in}}{\pgfqpoint{7.920729in}{0.838079in}}{\pgfqpoint{7.931779in}{0.838079in}}%
\pgfpathlineto{\pgfqpoint{7.931779in}{0.838079in}}%
\pgfpathclose%
\pgfusepath{stroke}%
\end{pgfscope}%
\begin{pgfscope}%
\pgfpathrectangle{\pgfqpoint{7.512535in}{0.437222in}}{\pgfqpoint{6.275590in}{5.159444in}}%
\pgfusepath{clip}%
\pgfsetbuttcap%
\pgfsetroundjoin%
\pgfsetlinewidth{1.003750pt}%
\definecolor{currentstroke}{rgb}{1.000000,0.000000,0.000000}%
\pgfsetstrokecolor{currentstroke}%
\pgfsetdash{}{0pt}%
\pgfpathmoveto{\pgfqpoint{12.904389in}{5.554986in}}%
\pgfpathcurveto{\pgfqpoint{12.915439in}{5.554986in}}{\pgfqpoint{12.926038in}{5.559376in}}{\pgfqpoint{12.933851in}{5.567190in}}%
\pgfpathcurveto{\pgfqpoint{12.941665in}{5.575003in}}{\pgfqpoint{12.946055in}{5.585602in}}{\pgfqpoint{12.946055in}{5.596652in}}%
\pgfpathcurveto{\pgfqpoint{12.946055in}{5.607702in}}{\pgfqpoint{12.941665in}{5.618301in}}{\pgfqpoint{12.933851in}{5.626115in}}%
\pgfpathcurveto{\pgfqpoint{12.926038in}{5.633929in}}{\pgfqpoint{12.915439in}{5.638319in}}{\pgfqpoint{12.904389in}{5.638319in}}%
\pgfpathcurveto{\pgfqpoint{12.893338in}{5.638319in}}{\pgfqpoint{12.882739in}{5.633929in}}{\pgfqpoint{12.874926in}{5.626115in}}%
\pgfpathcurveto{\pgfqpoint{12.867112in}{5.618301in}}{\pgfqpoint{12.862722in}{5.607702in}}{\pgfqpoint{12.862722in}{5.596652in}}%
\pgfpathcurveto{\pgfqpoint{12.862722in}{5.585602in}}{\pgfqpoint{12.867112in}{5.575003in}}{\pgfqpoint{12.874926in}{5.567190in}}%
\pgfpathcurveto{\pgfqpoint{12.882739in}{5.559376in}}{\pgfqpoint{12.893338in}{5.554986in}}{\pgfqpoint{12.904389in}{5.554986in}}%
\pgfpathlineto{\pgfqpoint{12.904389in}{5.554986in}}%
\pgfpathclose%
\pgfusepath{stroke}%
\end{pgfscope}%
\begin{pgfscope}%
\pgfpathrectangle{\pgfqpoint{7.512535in}{0.437222in}}{\pgfqpoint{6.275590in}{5.159444in}}%
\pgfusepath{clip}%
\pgfsetbuttcap%
\pgfsetroundjoin%
\pgfsetlinewidth{1.003750pt}%
\definecolor{currentstroke}{rgb}{1.000000,0.000000,0.000000}%
\pgfsetstrokecolor{currentstroke}%
\pgfsetdash{}{0pt}%
\pgfpathmoveto{\pgfqpoint{7.840837in}{0.711341in}}%
\pgfpathcurveto{\pgfqpoint{7.851887in}{0.711341in}}{\pgfqpoint{7.862486in}{0.715731in}}{\pgfqpoint{7.870300in}{0.723544in}}%
\pgfpathcurveto{\pgfqpoint{7.878114in}{0.731358in}}{\pgfqpoint{7.882504in}{0.741957in}}{\pgfqpoint{7.882504in}{0.753007in}}%
\pgfpathcurveto{\pgfqpoint{7.882504in}{0.764057in}}{\pgfqpoint{7.878114in}{0.774656in}}{\pgfqpoint{7.870300in}{0.782470in}}%
\pgfpathcurveto{\pgfqpoint{7.862486in}{0.790284in}}{\pgfqpoint{7.851887in}{0.794674in}}{\pgfqpoint{7.840837in}{0.794674in}}%
\pgfpathcurveto{\pgfqpoint{7.829787in}{0.794674in}}{\pgfqpoint{7.819188in}{0.790284in}}{\pgfqpoint{7.811374in}{0.782470in}}%
\pgfpathcurveto{\pgfqpoint{7.803561in}{0.774656in}}{\pgfqpoint{7.799171in}{0.764057in}}{\pgfqpoint{7.799171in}{0.753007in}}%
\pgfpathcurveto{\pgfqpoint{7.799171in}{0.741957in}}{\pgfqpoint{7.803561in}{0.731358in}}{\pgfqpoint{7.811374in}{0.723544in}}%
\pgfpathcurveto{\pgfqpoint{7.819188in}{0.715731in}}{\pgfqpoint{7.829787in}{0.711341in}}{\pgfqpoint{7.840837in}{0.711341in}}%
\pgfpathlineto{\pgfqpoint{7.840837in}{0.711341in}}%
\pgfpathclose%
\pgfusepath{stroke}%
\end{pgfscope}%
\begin{pgfscope}%
\pgfpathrectangle{\pgfqpoint{7.512535in}{0.437222in}}{\pgfqpoint{6.275590in}{5.159444in}}%
\pgfusepath{clip}%
\pgfsetbuttcap%
\pgfsetroundjoin%
\pgfsetlinewidth{1.003750pt}%
\definecolor{currentstroke}{rgb}{1.000000,0.000000,0.000000}%
\pgfsetstrokecolor{currentstroke}%
\pgfsetdash{}{0pt}%
\pgfpathmoveto{\pgfqpoint{10.685296in}{5.142900in}}%
\pgfpathcurveto{\pgfqpoint{10.696346in}{5.142900in}}{\pgfqpoint{10.706945in}{5.147291in}}{\pgfqpoint{10.714758in}{5.155104in}}%
\pgfpathcurveto{\pgfqpoint{10.722572in}{5.162918in}}{\pgfqpoint{10.726962in}{5.173517in}}{\pgfqpoint{10.726962in}{5.184567in}}%
\pgfpathcurveto{\pgfqpoint{10.726962in}{5.195617in}}{\pgfqpoint{10.722572in}{5.206216in}}{\pgfqpoint{10.714758in}{5.214030in}}%
\pgfpathcurveto{\pgfqpoint{10.706945in}{5.221844in}}{\pgfqpoint{10.696346in}{5.226234in}}{\pgfqpoint{10.685296in}{5.226234in}}%
\pgfpathcurveto{\pgfqpoint{10.674245in}{5.226234in}}{\pgfqpoint{10.663646in}{5.221844in}}{\pgfqpoint{10.655833in}{5.214030in}}%
\pgfpathcurveto{\pgfqpoint{10.648019in}{5.206216in}}{\pgfqpoint{10.643629in}{5.195617in}}{\pgfqpoint{10.643629in}{5.184567in}}%
\pgfpathcurveto{\pgfqpoint{10.643629in}{5.173517in}}{\pgfqpoint{10.648019in}{5.162918in}}{\pgfqpoint{10.655833in}{5.155104in}}%
\pgfpathcurveto{\pgfqpoint{10.663646in}{5.147291in}}{\pgfqpoint{10.674245in}{5.142900in}}{\pgfqpoint{10.685296in}{5.142900in}}%
\pgfpathlineto{\pgfqpoint{10.685296in}{5.142900in}}%
\pgfpathclose%
\pgfusepath{stroke}%
\end{pgfscope}%
\begin{pgfscope}%
\pgfpathrectangle{\pgfqpoint{7.512535in}{0.437222in}}{\pgfqpoint{6.275590in}{5.159444in}}%
\pgfusepath{clip}%
\pgfsetbuttcap%
\pgfsetroundjoin%
\pgfsetlinewidth{1.003750pt}%
\definecolor{currentstroke}{rgb}{1.000000,0.000000,0.000000}%
\pgfsetstrokecolor{currentstroke}%
\pgfsetdash{}{0pt}%
\pgfpathmoveto{\pgfqpoint{13.265147in}{5.554992in}}%
\pgfpathcurveto{\pgfqpoint{13.276197in}{5.554992in}}{\pgfqpoint{13.286796in}{5.559383in}}{\pgfqpoint{13.294610in}{5.567196in}}%
\pgfpathcurveto{\pgfqpoint{13.302424in}{5.575010in}}{\pgfqpoint{13.306814in}{5.585609in}}{\pgfqpoint{13.306814in}{5.596659in}}%
\pgfpathcurveto{\pgfqpoint{13.306814in}{5.607709in}}{\pgfqpoint{13.302424in}{5.618308in}}{\pgfqpoint{13.294610in}{5.626122in}}%
\pgfpathcurveto{\pgfqpoint{13.286796in}{5.633936in}}{\pgfqpoint{13.276197in}{5.638326in}}{\pgfqpoint{13.265147in}{5.638326in}}%
\pgfpathcurveto{\pgfqpoint{13.254097in}{5.638326in}}{\pgfqpoint{13.243498in}{5.633936in}}{\pgfqpoint{13.235684in}{5.626122in}}%
\pgfpathcurveto{\pgfqpoint{13.227871in}{5.618308in}}{\pgfqpoint{13.223480in}{5.607709in}}{\pgfqpoint{13.223480in}{5.596659in}}%
\pgfpathcurveto{\pgfqpoint{13.223480in}{5.585609in}}{\pgfqpoint{13.227871in}{5.575010in}}{\pgfqpoint{13.235684in}{5.567196in}}%
\pgfpathcurveto{\pgfqpoint{13.243498in}{5.559383in}}{\pgfqpoint{13.254097in}{5.554992in}}{\pgfqpoint{13.265147in}{5.554992in}}%
\pgfpathlineto{\pgfqpoint{13.265147in}{5.554992in}}%
\pgfpathclose%
\pgfusepath{stroke}%
\end{pgfscope}%
\begin{pgfscope}%
\pgfpathrectangle{\pgfqpoint{7.512535in}{0.437222in}}{\pgfqpoint{6.275590in}{5.159444in}}%
\pgfusepath{clip}%
\pgfsetbuttcap%
\pgfsetroundjoin%
\pgfsetlinewidth{1.003750pt}%
\definecolor{currentstroke}{rgb}{1.000000,0.000000,0.000000}%
\pgfsetstrokecolor{currentstroke}%
\pgfsetdash{}{0pt}%
\pgfpathmoveto{\pgfqpoint{13.489543in}{5.526494in}}%
\pgfpathcurveto{\pgfqpoint{13.500593in}{5.526494in}}{\pgfqpoint{13.511192in}{5.530884in}}{\pgfqpoint{13.519006in}{5.538698in}}%
\pgfpathcurveto{\pgfqpoint{13.526820in}{5.546512in}}{\pgfqpoint{13.531210in}{5.557111in}}{\pgfqpoint{13.531210in}{5.568161in}}%
\pgfpathcurveto{\pgfqpoint{13.531210in}{5.579211in}}{\pgfqpoint{13.526820in}{5.589810in}}{\pgfqpoint{13.519006in}{5.597624in}}%
\pgfpathcurveto{\pgfqpoint{13.511192in}{5.605437in}}{\pgfqpoint{13.500593in}{5.609828in}}{\pgfqpoint{13.489543in}{5.609828in}}%
\pgfpathcurveto{\pgfqpoint{13.478493in}{5.609828in}}{\pgfqpoint{13.467894in}{5.605437in}}{\pgfqpoint{13.460081in}{5.597624in}}%
\pgfpathcurveto{\pgfqpoint{13.452267in}{5.589810in}}{\pgfqpoint{13.447877in}{5.579211in}}{\pgfqpoint{13.447877in}{5.568161in}}%
\pgfpathcurveto{\pgfqpoint{13.447877in}{5.557111in}}{\pgfqpoint{13.452267in}{5.546512in}}{\pgfqpoint{13.460081in}{5.538698in}}%
\pgfpathcurveto{\pgfqpoint{13.467894in}{5.530884in}}{\pgfqpoint{13.478493in}{5.526494in}}{\pgfqpoint{13.489543in}{5.526494in}}%
\pgfpathlineto{\pgfqpoint{13.489543in}{5.526494in}}%
\pgfpathclose%
\pgfusepath{stroke}%
\end{pgfscope}%
\begin{pgfscope}%
\pgfpathrectangle{\pgfqpoint{7.512535in}{0.437222in}}{\pgfqpoint{6.275590in}{5.159444in}}%
\pgfusepath{clip}%
\pgfsetbuttcap%
\pgfsetroundjoin%
\pgfsetlinewidth{1.003750pt}%
\definecolor{currentstroke}{rgb}{1.000000,0.000000,0.000000}%
\pgfsetstrokecolor{currentstroke}%
\pgfsetdash{}{0pt}%
\pgfpathmoveto{\pgfqpoint{10.841460in}{5.379650in}}%
\pgfpathcurveto{\pgfqpoint{10.852510in}{5.379650in}}{\pgfqpoint{10.863109in}{5.384041in}}{\pgfqpoint{10.870923in}{5.391854in}}%
\pgfpathcurveto{\pgfqpoint{10.878736in}{5.399668in}}{\pgfqpoint{10.883127in}{5.410267in}}{\pgfqpoint{10.883127in}{5.421317in}}%
\pgfpathcurveto{\pgfqpoint{10.883127in}{5.432367in}}{\pgfqpoint{10.878736in}{5.442966in}}{\pgfqpoint{10.870923in}{5.450780in}}%
\pgfpathcurveto{\pgfqpoint{10.863109in}{5.458593in}}{\pgfqpoint{10.852510in}{5.462984in}}{\pgfqpoint{10.841460in}{5.462984in}}%
\pgfpathcurveto{\pgfqpoint{10.830410in}{5.462984in}}{\pgfqpoint{10.819811in}{5.458593in}}{\pgfqpoint{10.811997in}{5.450780in}}%
\pgfpathcurveto{\pgfqpoint{10.804184in}{5.442966in}}{\pgfqpoint{10.799793in}{5.432367in}}{\pgfqpoint{10.799793in}{5.421317in}}%
\pgfpathcurveto{\pgfqpoint{10.799793in}{5.410267in}}{\pgfqpoint{10.804184in}{5.399668in}}{\pgfqpoint{10.811997in}{5.391854in}}%
\pgfpathcurveto{\pgfqpoint{10.819811in}{5.384041in}}{\pgfqpoint{10.830410in}{5.379650in}}{\pgfqpoint{10.841460in}{5.379650in}}%
\pgfpathlineto{\pgfqpoint{10.841460in}{5.379650in}}%
\pgfpathclose%
\pgfusepath{stroke}%
\end{pgfscope}%
\begin{pgfscope}%
\pgfpathrectangle{\pgfqpoint{7.512535in}{0.437222in}}{\pgfqpoint{6.275590in}{5.159444in}}%
\pgfusepath{clip}%
\pgfsetbuttcap%
\pgfsetroundjoin%
\pgfsetlinewidth{1.003750pt}%
\definecolor{currentstroke}{rgb}{1.000000,0.000000,0.000000}%
\pgfsetstrokecolor{currentstroke}%
\pgfsetdash{}{0pt}%
\pgfpathmoveto{\pgfqpoint{8.337956in}{1.316525in}}%
\pgfpathcurveto{\pgfqpoint{8.349006in}{1.316525in}}{\pgfqpoint{8.359605in}{1.320915in}}{\pgfqpoint{8.367419in}{1.328729in}}%
\pgfpathcurveto{\pgfqpoint{8.375232in}{1.336543in}}{\pgfqpoint{8.379622in}{1.347142in}}{\pgfqpoint{8.379622in}{1.358192in}}%
\pgfpathcurveto{\pgfqpoint{8.379622in}{1.369242in}}{\pgfqpoint{8.375232in}{1.379841in}}{\pgfqpoint{8.367419in}{1.387654in}}%
\pgfpathcurveto{\pgfqpoint{8.359605in}{1.395468in}}{\pgfqpoint{8.349006in}{1.399858in}}{\pgfqpoint{8.337956in}{1.399858in}}%
\pgfpathcurveto{\pgfqpoint{8.326906in}{1.399858in}}{\pgfqpoint{8.316307in}{1.395468in}}{\pgfqpoint{8.308493in}{1.387654in}}%
\pgfpathcurveto{\pgfqpoint{8.300679in}{1.379841in}}{\pgfqpoint{8.296289in}{1.369242in}}{\pgfqpoint{8.296289in}{1.358192in}}%
\pgfpathcurveto{\pgfqpoint{8.296289in}{1.347142in}}{\pgfqpoint{8.300679in}{1.336543in}}{\pgfqpoint{8.308493in}{1.328729in}}%
\pgfpathcurveto{\pgfqpoint{8.316307in}{1.320915in}}{\pgfqpoint{8.326906in}{1.316525in}}{\pgfqpoint{8.337956in}{1.316525in}}%
\pgfpathlineto{\pgfqpoint{8.337956in}{1.316525in}}%
\pgfpathclose%
\pgfusepath{stroke}%
\end{pgfscope}%
\begin{pgfscope}%
\pgfpathrectangle{\pgfqpoint{7.512535in}{0.437222in}}{\pgfqpoint{6.275590in}{5.159444in}}%
\pgfusepath{clip}%
\pgfsetbuttcap%
\pgfsetroundjoin%
\pgfsetlinewidth{1.003750pt}%
\definecolor{currentstroke}{rgb}{1.000000,0.000000,0.000000}%
\pgfsetstrokecolor{currentstroke}%
\pgfsetdash{}{0pt}%
\pgfpathmoveto{\pgfqpoint{12.401525in}{5.554775in}}%
\pgfpathcurveto{\pgfqpoint{12.412576in}{5.554775in}}{\pgfqpoint{12.423175in}{5.559165in}}{\pgfqpoint{12.430988in}{5.566979in}}%
\pgfpathcurveto{\pgfqpoint{12.438802in}{5.574793in}}{\pgfqpoint{12.443192in}{5.585392in}}{\pgfqpoint{12.443192in}{5.596442in}}%
\pgfpathcurveto{\pgfqpoint{12.443192in}{5.607492in}}{\pgfqpoint{12.438802in}{5.618091in}}{\pgfqpoint{12.430988in}{5.625904in}}%
\pgfpathcurveto{\pgfqpoint{12.423175in}{5.633718in}}{\pgfqpoint{12.412576in}{5.638108in}}{\pgfqpoint{12.401525in}{5.638108in}}%
\pgfpathcurveto{\pgfqpoint{12.390475in}{5.638108in}}{\pgfqpoint{12.379876in}{5.633718in}}{\pgfqpoint{12.372063in}{5.625904in}}%
\pgfpathcurveto{\pgfqpoint{12.364249in}{5.618091in}}{\pgfqpoint{12.359859in}{5.607492in}}{\pgfqpoint{12.359859in}{5.596442in}}%
\pgfpathcurveto{\pgfqpoint{12.359859in}{5.585392in}}{\pgfqpoint{12.364249in}{5.574793in}}{\pgfqpoint{12.372063in}{5.566979in}}%
\pgfpathcurveto{\pgfqpoint{12.379876in}{5.559165in}}{\pgfqpoint{12.390475in}{5.554775in}}{\pgfqpoint{12.401525in}{5.554775in}}%
\pgfpathlineto{\pgfqpoint{12.401525in}{5.554775in}}%
\pgfpathclose%
\pgfusepath{stroke}%
\end{pgfscope}%
\begin{pgfscope}%
\pgfpathrectangle{\pgfqpoint{7.512535in}{0.437222in}}{\pgfqpoint{6.275590in}{5.159444in}}%
\pgfusepath{clip}%
\pgfsetbuttcap%
\pgfsetroundjoin%
\pgfsetlinewidth{1.003750pt}%
\definecolor{currentstroke}{rgb}{1.000000,0.000000,0.000000}%
\pgfsetstrokecolor{currentstroke}%
\pgfsetdash{}{0pt}%
\pgfpathmoveto{\pgfqpoint{8.828690in}{2.561957in}}%
\pgfpathcurveto{\pgfqpoint{8.839740in}{2.561957in}}{\pgfqpoint{8.850339in}{2.566347in}}{\pgfqpoint{8.858153in}{2.574160in}}%
\pgfpathcurveto{\pgfqpoint{8.865966in}{2.581974in}}{\pgfqpoint{8.870357in}{2.592573in}}{\pgfqpoint{8.870357in}{2.603623in}}%
\pgfpathcurveto{\pgfqpoint{8.870357in}{2.614673in}}{\pgfqpoint{8.865966in}{2.625272in}}{\pgfqpoint{8.858153in}{2.633086in}}%
\pgfpathcurveto{\pgfqpoint{8.850339in}{2.640900in}}{\pgfqpoint{8.839740in}{2.645290in}}{\pgfqpoint{8.828690in}{2.645290in}}%
\pgfpathcurveto{\pgfqpoint{8.817640in}{2.645290in}}{\pgfqpoint{8.807041in}{2.640900in}}{\pgfqpoint{8.799227in}{2.633086in}}%
\pgfpathcurveto{\pgfqpoint{8.791413in}{2.625272in}}{\pgfqpoint{8.787023in}{2.614673in}}{\pgfqpoint{8.787023in}{2.603623in}}%
\pgfpathcurveto{\pgfqpoint{8.787023in}{2.592573in}}{\pgfqpoint{8.791413in}{2.581974in}}{\pgfqpoint{8.799227in}{2.574160in}}%
\pgfpathcurveto{\pgfqpoint{8.807041in}{2.566347in}}{\pgfqpoint{8.817640in}{2.561957in}}{\pgfqpoint{8.828690in}{2.561957in}}%
\pgfpathlineto{\pgfqpoint{8.828690in}{2.561957in}}%
\pgfpathclose%
\pgfusepath{stroke}%
\end{pgfscope}%
\begin{pgfscope}%
\pgfpathrectangle{\pgfqpoint{7.512535in}{0.437222in}}{\pgfqpoint{6.275590in}{5.159444in}}%
\pgfusepath{clip}%
\pgfsetbuttcap%
\pgfsetroundjoin%
\pgfsetlinewidth{1.003750pt}%
\definecolor{currentstroke}{rgb}{1.000000,0.000000,0.000000}%
\pgfsetstrokecolor{currentstroke}%
\pgfsetdash{}{0pt}%
\pgfpathmoveto{\pgfqpoint{11.414634in}{5.514771in}}%
\pgfpathcurveto{\pgfqpoint{11.425684in}{5.514771in}}{\pgfqpoint{11.436283in}{5.519162in}}{\pgfqpoint{11.444097in}{5.526975in}}%
\pgfpathcurveto{\pgfqpoint{11.451911in}{5.534789in}}{\pgfqpoint{11.456301in}{5.545388in}}{\pgfqpoint{11.456301in}{5.556438in}}%
\pgfpathcurveto{\pgfqpoint{11.456301in}{5.567488in}}{\pgfqpoint{11.451911in}{5.578087in}}{\pgfqpoint{11.444097in}{5.585901in}}%
\pgfpathcurveto{\pgfqpoint{11.436283in}{5.593715in}}{\pgfqpoint{11.425684in}{5.598105in}}{\pgfqpoint{11.414634in}{5.598105in}}%
\pgfpathcurveto{\pgfqpoint{11.403584in}{5.598105in}}{\pgfqpoint{11.392985in}{5.593715in}}{\pgfqpoint{11.385171in}{5.585901in}}%
\pgfpathcurveto{\pgfqpoint{11.377358in}{5.578087in}}{\pgfqpoint{11.372967in}{5.567488in}}{\pgfqpoint{11.372967in}{5.556438in}}%
\pgfpathcurveto{\pgfqpoint{11.372967in}{5.545388in}}{\pgfqpoint{11.377358in}{5.534789in}}{\pgfqpoint{11.385171in}{5.526975in}}%
\pgfpathcurveto{\pgfqpoint{11.392985in}{5.519162in}}{\pgfqpoint{11.403584in}{5.514771in}}{\pgfqpoint{11.414634in}{5.514771in}}%
\pgfpathlineto{\pgfqpoint{11.414634in}{5.514771in}}%
\pgfpathclose%
\pgfusepath{stroke}%
\end{pgfscope}%
\begin{pgfscope}%
\pgfpathrectangle{\pgfqpoint{7.512535in}{0.437222in}}{\pgfqpoint{6.275590in}{5.159444in}}%
\pgfusepath{clip}%
\pgfsetbuttcap%
\pgfsetroundjoin%
\pgfsetlinewidth{1.003750pt}%
\definecolor{currentstroke}{rgb}{1.000000,0.000000,0.000000}%
\pgfsetstrokecolor{currentstroke}%
\pgfsetdash{}{0pt}%
\pgfpathmoveto{\pgfqpoint{10.454879in}{3.950352in}}%
\pgfpathcurveto{\pgfqpoint{10.465929in}{3.950352in}}{\pgfqpoint{10.476528in}{3.954742in}}{\pgfqpoint{10.484341in}{3.962556in}}%
\pgfpathcurveto{\pgfqpoint{10.492155in}{3.970370in}}{\pgfqpoint{10.496545in}{3.980969in}}{\pgfqpoint{10.496545in}{3.992019in}}%
\pgfpathcurveto{\pgfqpoint{10.496545in}{4.003069in}}{\pgfqpoint{10.492155in}{4.013668in}}{\pgfqpoint{10.484341in}{4.021482in}}%
\pgfpathcurveto{\pgfqpoint{10.476528in}{4.029295in}}{\pgfqpoint{10.465929in}{4.033685in}}{\pgfqpoint{10.454879in}{4.033685in}}%
\pgfpathcurveto{\pgfqpoint{10.443828in}{4.033685in}}{\pgfqpoint{10.433229in}{4.029295in}}{\pgfqpoint{10.425416in}{4.021482in}}%
\pgfpathcurveto{\pgfqpoint{10.417602in}{4.013668in}}{\pgfqpoint{10.413212in}{4.003069in}}{\pgfqpoint{10.413212in}{3.992019in}}%
\pgfpathcurveto{\pgfqpoint{10.413212in}{3.980969in}}{\pgfqpoint{10.417602in}{3.970370in}}{\pgfqpoint{10.425416in}{3.962556in}}%
\pgfpathcurveto{\pgfqpoint{10.433229in}{3.954742in}}{\pgfqpoint{10.443828in}{3.950352in}}{\pgfqpoint{10.454879in}{3.950352in}}%
\pgfpathlineto{\pgfqpoint{10.454879in}{3.950352in}}%
\pgfpathclose%
\pgfusepath{stroke}%
\end{pgfscope}%
\begin{pgfscope}%
\pgfpathrectangle{\pgfqpoint{7.512535in}{0.437222in}}{\pgfqpoint{6.275590in}{5.159444in}}%
\pgfusepath{clip}%
\pgfsetbuttcap%
\pgfsetroundjoin%
\pgfsetlinewidth{1.003750pt}%
\definecolor{currentstroke}{rgb}{1.000000,0.000000,0.000000}%
\pgfsetstrokecolor{currentstroke}%
\pgfsetdash{}{0pt}%
\pgfpathmoveto{\pgfqpoint{8.350252in}{1.556284in}}%
\pgfpathcurveto{\pgfqpoint{8.361302in}{1.556284in}}{\pgfqpoint{8.371902in}{1.560674in}}{\pgfqpoint{8.379715in}{1.568488in}}%
\pgfpathcurveto{\pgfqpoint{8.387529in}{1.576301in}}{\pgfqpoint{8.391919in}{1.586900in}}{\pgfqpoint{8.391919in}{1.597951in}}%
\pgfpathcurveto{\pgfqpoint{8.391919in}{1.609001in}}{\pgfqpoint{8.387529in}{1.619600in}}{\pgfqpoint{8.379715in}{1.627413in}}%
\pgfpathcurveto{\pgfqpoint{8.371902in}{1.635227in}}{\pgfqpoint{8.361302in}{1.639617in}}{\pgfqpoint{8.350252in}{1.639617in}}%
\pgfpathcurveto{\pgfqpoint{8.339202in}{1.639617in}}{\pgfqpoint{8.328603in}{1.635227in}}{\pgfqpoint{8.320790in}{1.627413in}}%
\pgfpathcurveto{\pgfqpoint{8.312976in}{1.619600in}}{\pgfqpoint{8.308586in}{1.609001in}}{\pgfqpoint{8.308586in}{1.597951in}}%
\pgfpathcurveto{\pgfqpoint{8.308586in}{1.586900in}}{\pgfqpoint{8.312976in}{1.576301in}}{\pgfqpoint{8.320790in}{1.568488in}}%
\pgfpathcurveto{\pgfqpoint{8.328603in}{1.560674in}}{\pgfqpoint{8.339202in}{1.556284in}}{\pgfqpoint{8.350252in}{1.556284in}}%
\pgfpathlineto{\pgfqpoint{8.350252in}{1.556284in}}%
\pgfpathclose%
\pgfusepath{stroke}%
\end{pgfscope}%
\begin{pgfscope}%
\pgfpathrectangle{\pgfqpoint{7.512535in}{0.437222in}}{\pgfqpoint{6.275590in}{5.159444in}}%
\pgfusepath{clip}%
\pgfsetbuttcap%
\pgfsetroundjoin%
\pgfsetlinewidth{1.003750pt}%
\definecolor{currentstroke}{rgb}{1.000000,0.000000,0.000000}%
\pgfsetstrokecolor{currentstroke}%
\pgfsetdash{}{0pt}%
\pgfpathmoveto{\pgfqpoint{9.685797in}{3.587531in}}%
\pgfpathcurveto{\pgfqpoint{9.696847in}{3.587531in}}{\pgfqpoint{9.707446in}{3.591921in}}{\pgfqpoint{9.715260in}{3.599735in}}%
\pgfpathcurveto{\pgfqpoint{9.723073in}{3.607548in}}{\pgfqpoint{9.727464in}{3.618147in}}{\pgfqpoint{9.727464in}{3.629197in}}%
\pgfpathcurveto{\pgfqpoint{9.727464in}{3.640247in}}{\pgfqpoint{9.723073in}{3.650846in}}{\pgfqpoint{9.715260in}{3.658660in}}%
\pgfpathcurveto{\pgfqpoint{9.707446in}{3.666474in}}{\pgfqpoint{9.696847in}{3.670864in}}{\pgfqpoint{9.685797in}{3.670864in}}%
\pgfpathcurveto{\pgfqpoint{9.674747in}{3.670864in}}{\pgfqpoint{9.664148in}{3.666474in}}{\pgfqpoint{9.656334in}{3.658660in}}%
\pgfpathcurveto{\pgfqpoint{9.648521in}{3.650846in}}{\pgfqpoint{9.644130in}{3.640247in}}{\pgfqpoint{9.644130in}{3.629197in}}%
\pgfpathcurveto{\pgfqpoint{9.644130in}{3.618147in}}{\pgfqpoint{9.648521in}{3.607548in}}{\pgfqpoint{9.656334in}{3.599735in}}%
\pgfpathcurveto{\pgfqpoint{9.664148in}{3.591921in}}{\pgfqpoint{9.674747in}{3.587531in}}{\pgfqpoint{9.685797in}{3.587531in}}%
\pgfpathlineto{\pgfqpoint{9.685797in}{3.587531in}}%
\pgfpathclose%
\pgfusepath{stroke}%
\end{pgfscope}%
\begin{pgfscope}%
\pgfpathrectangle{\pgfqpoint{7.512535in}{0.437222in}}{\pgfqpoint{6.275590in}{5.159444in}}%
\pgfusepath{clip}%
\pgfsetbuttcap%
\pgfsetroundjoin%
\pgfsetlinewidth{1.003750pt}%
\definecolor{currentstroke}{rgb}{1.000000,0.000000,0.000000}%
\pgfsetstrokecolor{currentstroke}%
\pgfsetdash{}{0pt}%
\pgfpathmoveto{\pgfqpoint{12.816431in}{5.554021in}}%
\pgfpathcurveto{\pgfqpoint{12.827481in}{5.554021in}}{\pgfqpoint{12.838080in}{5.558411in}}{\pgfqpoint{12.845894in}{5.566225in}}%
\pgfpathcurveto{\pgfqpoint{12.853707in}{5.574039in}}{\pgfqpoint{12.858098in}{5.584638in}}{\pgfqpoint{12.858098in}{5.595688in}}%
\pgfpathcurveto{\pgfqpoint{12.858098in}{5.606738in}}{\pgfqpoint{12.853707in}{5.617337in}}{\pgfqpoint{12.845894in}{5.625150in}}%
\pgfpathcurveto{\pgfqpoint{12.838080in}{5.632964in}}{\pgfqpoint{12.827481in}{5.637354in}}{\pgfqpoint{12.816431in}{5.637354in}}%
\pgfpathcurveto{\pgfqpoint{12.805381in}{5.637354in}}{\pgfqpoint{12.794782in}{5.632964in}}{\pgfqpoint{12.786968in}{5.625150in}}%
\pgfpathcurveto{\pgfqpoint{12.779154in}{5.617337in}}{\pgfqpoint{12.774764in}{5.606738in}}{\pgfqpoint{12.774764in}{5.595688in}}%
\pgfpathcurveto{\pgfqpoint{12.774764in}{5.584638in}}{\pgfqpoint{12.779154in}{5.574039in}}{\pgfqpoint{12.786968in}{5.566225in}}%
\pgfpathcurveto{\pgfqpoint{12.794782in}{5.558411in}}{\pgfqpoint{12.805381in}{5.554021in}}{\pgfqpoint{12.816431in}{5.554021in}}%
\pgfpathlineto{\pgfqpoint{12.816431in}{5.554021in}}%
\pgfpathclose%
\pgfusepath{stroke}%
\end{pgfscope}%
\begin{pgfscope}%
\pgfpathrectangle{\pgfqpoint{7.512535in}{0.437222in}}{\pgfqpoint{6.275590in}{5.159444in}}%
\pgfusepath{clip}%
\pgfsetbuttcap%
\pgfsetroundjoin%
\pgfsetlinewidth{1.003750pt}%
\definecolor{currentstroke}{rgb}{1.000000,0.000000,0.000000}%
\pgfsetstrokecolor{currentstroke}%
\pgfsetdash{}{0pt}%
\pgfpathmoveto{\pgfqpoint{10.524300in}{4.741597in}}%
\pgfpathcurveto{\pgfqpoint{10.535350in}{4.741597in}}{\pgfqpoint{10.545949in}{4.745987in}}{\pgfqpoint{10.553762in}{4.753801in}}%
\pgfpathcurveto{\pgfqpoint{10.561576in}{4.761615in}}{\pgfqpoint{10.565966in}{4.772214in}}{\pgfqpoint{10.565966in}{4.783264in}}%
\pgfpathcurveto{\pgfqpoint{10.565966in}{4.794314in}}{\pgfqpoint{10.561576in}{4.804913in}}{\pgfqpoint{10.553762in}{4.812727in}}%
\pgfpathcurveto{\pgfqpoint{10.545949in}{4.820540in}}{\pgfqpoint{10.535350in}{4.824930in}}{\pgfqpoint{10.524300in}{4.824930in}}%
\pgfpathcurveto{\pgfqpoint{10.513250in}{4.824930in}}{\pgfqpoint{10.502651in}{4.820540in}}{\pgfqpoint{10.494837in}{4.812727in}}%
\pgfpathcurveto{\pgfqpoint{10.487023in}{4.804913in}}{\pgfqpoint{10.482633in}{4.794314in}}{\pgfqpoint{10.482633in}{4.783264in}}%
\pgfpathcurveto{\pgfqpoint{10.482633in}{4.772214in}}{\pgfqpoint{10.487023in}{4.761615in}}{\pgfqpoint{10.494837in}{4.753801in}}%
\pgfpathcurveto{\pgfqpoint{10.502651in}{4.745987in}}{\pgfqpoint{10.513250in}{4.741597in}}{\pgfqpoint{10.524300in}{4.741597in}}%
\pgfpathlineto{\pgfqpoint{10.524300in}{4.741597in}}%
\pgfpathclose%
\pgfusepath{stroke}%
\end{pgfscope}%
\begin{pgfscope}%
\pgfpathrectangle{\pgfqpoint{7.512535in}{0.437222in}}{\pgfqpoint{6.275590in}{5.159444in}}%
\pgfusepath{clip}%
\pgfsetbuttcap%
\pgfsetroundjoin%
\pgfsetlinewidth{1.003750pt}%
\definecolor{currentstroke}{rgb}{1.000000,0.000000,0.000000}%
\pgfsetstrokecolor{currentstroke}%
\pgfsetdash{}{0pt}%
\pgfpathmoveto{\pgfqpoint{13.606447in}{5.554768in}}%
\pgfpathcurveto{\pgfqpoint{13.617497in}{5.554768in}}{\pgfqpoint{13.628096in}{5.559158in}}{\pgfqpoint{13.635910in}{5.566972in}}%
\pgfpathcurveto{\pgfqpoint{13.643724in}{5.574785in}}{\pgfqpoint{13.648114in}{5.585385in}}{\pgfqpoint{13.648114in}{5.596435in}}%
\pgfpathcurveto{\pgfqpoint{13.648114in}{5.607485in}}{\pgfqpoint{13.643724in}{5.618084in}}{\pgfqpoint{13.635910in}{5.625897in}}%
\pgfpathcurveto{\pgfqpoint{13.628096in}{5.633711in}}{\pgfqpoint{13.617497in}{5.638101in}}{\pgfqpoint{13.606447in}{5.638101in}}%
\pgfpathcurveto{\pgfqpoint{13.595397in}{5.638101in}}{\pgfqpoint{13.584798in}{5.633711in}}{\pgfqpoint{13.576984in}{5.625897in}}%
\pgfpathcurveto{\pgfqpoint{13.569171in}{5.618084in}}{\pgfqpoint{13.564781in}{5.607485in}}{\pgfqpoint{13.564781in}{5.596435in}}%
\pgfpathcurveto{\pgfqpoint{13.564781in}{5.585385in}}{\pgfqpoint{13.569171in}{5.574785in}}{\pgfqpoint{13.576984in}{5.566972in}}%
\pgfpathcurveto{\pgfqpoint{13.584798in}{5.559158in}}{\pgfqpoint{13.595397in}{5.554768in}}{\pgfqpoint{13.606447in}{5.554768in}}%
\pgfpathlineto{\pgfqpoint{13.606447in}{5.554768in}}%
\pgfpathclose%
\pgfusepath{stroke}%
\end{pgfscope}%
\begin{pgfscope}%
\pgfpathrectangle{\pgfqpoint{7.512535in}{0.437222in}}{\pgfqpoint{6.275590in}{5.159444in}}%
\pgfusepath{clip}%
\pgfsetbuttcap%
\pgfsetroundjoin%
\pgfsetlinewidth{1.003750pt}%
\definecolor{currentstroke}{rgb}{1.000000,0.000000,0.000000}%
\pgfsetstrokecolor{currentstroke}%
\pgfsetdash{}{0pt}%
\pgfpathmoveto{\pgfqpoint{7.921287in}{1.034779in}}%
\pgfpathcurveto{\pgfqpoint{7.932337in}{1.034779in}}{\pgfqpoint{7.942936in}{1.039169in}}{\pgfqpoint{7.950750in}{1.046983in}}%
\pgfpathcurveto{\pgfqpoint{7.958564in}{1.054796in}}{\pgfqpoint{7.962954in}{1.065395in}}{\pgfqpoint{7.962954in}{1.076445in}}%
\pgfpathcurveto{\pgfqpoint{7.962954in}{1.087495in}}{\pgfqpoint{7.958564in}{1.098094in}}{\pgfqpoint{7.950750in}{1.105908in}}%
\pgfpathcurveto{\pgfqpoint{7.942936in}{1.113722in}}{\pgfqpoint{7.932337in}{1.118112in}}{\pgfqpoint{7.921287in}{1.118112in}}%
\pgfpathcurveto{\pgfqpoint{7.910237in}{1.118112in}}{\pgfqpoint{7.899638in}{1.113722in}}{\pgfqpoint{7.891824in}{1.105908in}}%
\pgfpathcurveto{\pgfqpoint{7.884011in}{1.098094in}}{\pgfqpoint{7.879620in}{1.087495in}}{\pgfqpoint{7.879620in}{1.076445in}}%
\pgfpathcurveto{\pgfqpoint{7.879620in}{1.065395in}}{\pgfqpoint{7.884011in}{1.054796in}}{\pgfqpoint{7.891824in}{1.046983in}}%
\pgfpathcurveto{\pgfqpoint{7.899638in}{1.039169in}}{\pgfqpoint{7.910237in}{1.034779in}}{\pgfqpoint{7.921287in}{1.034779in}}%
\pgfpathlineto{\pgfqpoint{7.921287in}{1.034779in}}%
\pgfpathclose%
\pgfusepath{stroke}%
\end{pgfscope}%
\begin{pgfscope}%
\pgfpathrectangle{\pgfqpoint{7.512535in}{0.437222in}}{\pgfqpoint{6.275590in}{5.159444in}}%
\pgfusepath{clip}%
\pgfsetbuttcap%
\pgfsetroundjoin%
\pgfsetlinewidth{1.003750pt}%
\definecolor{currentstroke}{rgb}{1.000000,0.000000,0.000000}%
\pgfsetstrokecolor{currentstroke}%
\pgfsetdash{}{0pt}%
\pgfpathmoveto{\pgfqpoint{10.454879in}{4.076314in}}%
\pgfpathcurveto{\pgfqpoint{10.465929in}{4.076314in}}{\pgfqpoint{10.476528in}{4.080704in}}{\pgfqpoint{10.484341in}{4.088518in}}%
\pgfpathcurveto{\pgfqpoint{10.492155in}{4.096331in}}{\pgfqpoint{10.496545in}{4.106930in}}{\pgfqpoint{10.496545in}{4.117981in}}%
\pgfpathcurveto{\pgfqpoint{10.496545in}{4.129031in}}{\pgfqpoint{10.492155in}{4.139630in}}{\pgfqpoint{10.484341in}{4.147443in}}%
\pgfpathcurveto{\pgfqpoint{10.476528in}{4.155257in}}{\pgfqpoint{10.465929in}{4.159647in}}{\pgfqpoint{10.454879in}{4.159647in}}%
\pgfpathcurveto{\pgfqpoint{10.443828in}{4.159647in}}{\pgfqpoint{10.433229in}{4.155257in}}{\pgfqpoint{10.425416in}{4.147443in}}%
\pgfpathcurveto{\pgfqpoint{10.417602in}{4.139630in}}{\pgfqpoint{10.413212in}{4.129031in}}{\pgfqpoint{10.413212in}{4.117981in}}%
\pgfpathcurveto{\pgfqpoint{10.413212in}{4.106930in}}{\pgfqpoint{10.417602in}{4.096331in}}{\pgfqpoint{10.425416in}{4.088518in}}%
\pgfpathcurveto{\pgfqpoint{10.433229in}{4.080704in}}{\pgfqpoint{10.443828in}{4.076314in}}{\pgfqpoint{10.454879in}{4.076314in}}%
\pgfpathlineto{\pgfqpoint{10.454879in}{4.076314in}}%
\pgfpathclose%
\pgfusepath{stroke}%
\end{pgfscope}%
\begin{pgfscope}%
\pgfpathrectangle{\pgfqpoint{7.512535in}{0.437222in}}{\pgfqpoint{6.275590in}{5.159444in}}%
\pgfusepath{clip}%
\pgfsetbuttcap%
\pgfsetroundjoin%
\pgfsetlinewidth{1.003750pt}%
\definecolor{currentstroke}{rgb}{1.000000,0.000000,0.000000}%
\pgfsetstrokecolor{currentstroke}%
\pgfsetdash{}{0pt}%
\pgfpathmoveto{\pgfqpoint{11.101616in}{4.086709in}}%
\pgfpathcurveto{\pgfqpoint{11.112666in}{4.086709in}}{\pgfqpoint{11.123265in}{4.091099in}}{\pgfqpoint{11.131079in}{4.098913in}}%
\pgfpathcurveto{\pgfqpoint{11.138892in}{4.106727in}}{\pgfqpoint{11.143283in}{4.117326in}}{\pgfqpoint{11.143283in}{4.128376in}}%
\pgfpathcurveto{\pgfqpoint{11.143283in}{4.139426in}}{\pgfqpoint{11.138892in}{4.150025in}}{\pgfqpoint{11.131079in}{4.157839in}}%
\pgfpathcurveto{\pgfqpoint{11.123265in}{4.165652in}}{\pgfqpoint{11.112666in}{4.170042in}}{\pgfqpoint{11.101616in}{4.170042in}}%
\pgfpathcurveto{\pgfqpoint{11.090566in}{4.170042in}}{\pgfqpoint{11.079967in}{4.165652in}}{\pgfqpoint{11.072153in}{4.157839in}}%
\pgfpathcurveto{\pgfqpoint{11.064339in}{4.150025in}}{\pgfqpoint{11.059949in}{4.139426in}}{\pgfqpoint{11.059949in}{4.128376in}}%
\pgfpathcurveto{\pgfqpoint{11.059949in}{4.117326in}}{\pgfqpoint{11.064339in}{4.106727in}}{\pgfqpoint{11.072153in}{4.098913in}}%
\pgfpathcurveto{\pgfqpoint{11.079967in}{4.091099in}}{\pgfqpoint{11.090566in}{4.086709in}}{\pgfqpoint{11.101616in}{4.086709in}}%
\pgfpathlineto{\pgfqpoint{11.101616in}{4.086709in}}%
\pgfpathclose%
\pgfusepath{stroke}%
\end{pgfscope}%
\begin{pgfscope}%
\pgfpathrectangle{\pgfqpoint{7.512535in}{0.437222in}}{\pgfqpoint{6.275590in}{5.159444in}}%
\pgfusepath{clip}%
\pgfsetbuttcap%
\pgfsetroundjoin%
\pgfsetlinewidth{1.003750pt}%
\definecolor{currentstroke}{rgb}{1.000000,0.000000,0.000000}%
\pgfsetstrokecolor{currentstroke}%
\pgfsetdash{}{0pt}%
\pgfpathmoveto{\pgfqpoint{11.819913in}{5.390977in}}%
\pgfpathcurveto{\pgfqpoint{11.830963in}{5.390977in}}{\pgfqpoint{11.841562in}{5.395367in}}{\pgfqpoint{11.849376in}{5.403181in}}%
\pgfpathcurveto{\pgfqpoint{11.857190in}{5.410994in}}{\pgfqpoint{11.861580in}{5.421593in}}{\pgfqpoint{11.861580in}{5.432644in}}%
\pgfpathcurveto{\pgfqpoint{11.861580in}{5.443694in}}{\pgfqpoint{11.857190in}{5.454293in}}{\pgfqpoint{11.849376in}{5.462106in}}%
\pgfpathcurveto{\pgfqpoint{11.841562in}{5.469920in}}{\pgfqpoint{11.830963in}{5.474310in}}{\pgfqpoint{11.819913in}{5.474310in}}%
\pgfpathcurveto{\pgfqpoint{11.808863in}{5.474310in}}{\pgfqpoint{11.798264in}{5.469920in}}{\pgfqpoint{11.790450in}{5.462106in}}%
\pgfpathcurveto{\pgfqpoint{11.782637in}{5.454293in}}{\pgfqpoint{11.778247in}{5.443694in}}{\pgfqpoint{11.778247in}{5.432644in}}%
\pgfpathcurveto{\pgfqpoint{11.778247in}{5.421593in}}{\pgfqpoint{11.782637in}{5.410994in}}{\pgfqpoint{11.790450in}{5.403181in}}%
\pgfpathcurveto{\pgfqpoint{11.798264in}{5.395367in}}{\pgfqpoint{11.808863in}{5.390977in}}{\pgfqpoint{11.819913in}{5.390977in}}%
\pgfpathlineto{\pgfqpoint{11.819913in}{5.390977in}}%
\pgfpathclose%
\pgfusepath{stroke}%
\end{pgfscope}%
\begin{pgfscope}%
\pgfpathrectangle{\pgfqpoint{7.512535in}{0.437222in}}{\pgfqpoint{6.275590in}{5.159444in}}%
\pgfusepath{clip}%
\pgfsetbuttcap%
\pgfsetroundjoin%
\pgfsetlinewidth{1.003750pt}%
\definecolor{currentstroke}{rgb}{1.000000,0.000000,0.000000}%
\pgfsetstrokecolor{currentstroke}%
\pgfsetdash{}{0pt}%
\pgfpathmoveto{\pgfqpoint{7.773526in}{0.633790in}}%
\pgfpathcurveto{\pgfqpoint{7.784576in}{0.633790in}}{\pgfqpoint{7.795175in}{0.638180in}}{\pgfqpoint{7.802989in}{0.645994in}}%
\pgfpathcurveto{\pgfqpoint{7.810802in}{0.653807in}}{\pgfqpoint{7.815193in}{0.664406in}}{\pgfqpoint{7.815193in}{0.675456in}}%
\pgfpathcurveto{\pgfqpoint{7.815193in}{0.686507in}}{\pgfqpoint{7.810802in}{0.697106in}}{\pgfqpoint{7.802989in}{0.704919in}}%
\pgfpathcurveto{\pgfqpoint{7.795175in}{0.712733in}}{\pgfqpoint{7.784576in}{0.717123in}}{\pgfqpoint{7.773526in}{0.717123in}}%
\pgfpathcurveto{\pgfqpoint{7.762476in}{0.717123in}}{\pgfqpoint{7.751877in}{0.712733in}}{\pgfqpoint{7.744063in}{0.704919in}}%
\pgfpathcurveto{\pgfqpoint{7.736249in}{0.697106in}}{\pgfqpoint{7.731859in}{0.686507in}}{\pgfqpoint{7.731859in}{0.675456in}}%
\pgfpathcurveto{\pgfqpoint{7.731859in}{0.664406in}}{\pgfqpoint{7.736249in}{0.653807in}}{\pgfqpoint{7.744063in}{0.645994in}}%
\pgfpathcurveto{\pgfqpoint{7.751877in}{0.638180in}}{\pgfqpoint{7.762476in}{0.633790in}}{\pgfqpoint{7.773526in}{0.633790in}}%
\pgfpathlineto{\pgfqpoint{7.773526in}{0.633790in}}%
\pgfpathclose%
\pgfusepath{stroke}%
\end{pgfscope}%
\begin{pgfscope}%
\pgfpathrectangle{\pgfqpoint{7.512535in}{0.437222in}}{\pgfqpoint{6.275590in}{5.159444in}}%
\pgfusepath{clip}%
\pgfsetbuttcap%
\pgfsetroundjoin%
\pgfsetlinewidth{1.003750pt}%
\definecolor{currentstroke}{rgb}{1.000000,0.000000,0.000000}%
\pgfsetstrokecolor{currentstroke}%
\pgfsetdash{}{0pt}%
\pgfpathmoveto{\pgfqpoint{10.278872in}{4.015404in}}%
\pgfpathcurveto{\pgfqpoint{10.289922in}{4.015404in}}{\pgfqpoint{10.300522in}{4.019794in}}{\pgfqpoint{10.308335in}{4.027608in}}%
\pgfpathcurveto{\pgfqpoint{10.316149in}{4.035421in}}{\pgfqpoint{10.320539in}{4.046020in}}{\pgfqpoint{10.320539in}{4.057070in}}%
\pgfpathcurveto{\pgfqpoint{10.320539in}{4.068120in}}{\pgfqpoint{10.316149in}{4.078720in}}{\pgfqpoint{10.308335in}{4.086533in}}%
\pgfpathcurveto{\pgfqpoint{10.300522in}{4.094347in}}{\pgfqpoint{10.289922in}{4.098737in}}{\pgfqpoint{10.278872in}{4.098737in}}%
\pgfpathcurveto{\pgfqpoint{10.267822in}{4.098737in}}{\pgfqpoint{10.257223in}{4.094347in}}{\pgfqpoint{10.249410in}{4.086533in}}%
\pgfpathcurveto{\pgfqpoint{10.241596in}{4.078720in}}{\pgfqpoint{10.237206in}{4.068120in}}{\pgfqpoint{10.237206in}{4.057070in}}%
\pgfpathcurveto{\pgfqpoint{10.237206in}{4.046020in}}{\pgfqpoint{10.241596in}{4.035421in}}{\pgfqpoint{10.249410in}{4.027608in}}%
\pgfpathcurveto{\pgfqpoint{10.257223in}{4.019794in}}{\pgfqpoint{10.267822in}{4.015404in}}{\pgfqpoint{10.278872in}{4.015404in}}%
\pgfpathlineto{\pgfqpoint{10.278872in}{4.015404in}}%
\pgfpathclose%
\pgfusepath{stroke}%
\end{pgfscope}%
\begin{pgfscope}%
\pgfpathrectangle{\pgfqpoint{7.512535in}{0.437222in}}{\pgfqpoint{6.275590in}{5.159444in}}%
\pgfusepath{clip}%
\pgfsetbuttcap%
\pgfsetroundjoin%
\pgfsetlinewidth{1.003750pt}%
\definecolor{currentstroke}{rgb}{1.000000,0.000000,0.000000}%
\pgfsetstrokecolor{currentstroke}%
\pgfsetdash{}{0pt}%
\pgfpathmoveto{\pgfqpoint{9.114112in}{2.398372in}}%
\pgfpathcurveto{\pgfqpoint{9.125162in}{2.398372in}}{\pgfqpoint{9.135761in}{2.402762in}}{\pgfqpoint{9.143575in}{2.410576in}}%
\pgfpathcurveto{\pgfqpoint{9.151388in}{2.418390in}}{\pgfqpoint{9.155779in}{2.428989in}}{\pgfqpoint{9.155779in}{2.440039in}}%
\pgfpathcurveto{\pgfqpoint{9.155779in}{2.451089in}}{\pgfqpoint{9.151388in}{2.461688in}}{\pgfqpoint{9.143575in}{2.469502in}}%
\pgfpathcurveto{\pgfqpoint{9.135761in}{2.477315in}}{\pgfqpoint{9.125162in}{2.481705in}}{\pgfqpoint{9.114112in}{2.481705in}}%
\pgfpathcurveto{\pgfqpoint{9.103062in}{2.481705in}}{\pgfqpoint{9.092463in}{2.477315in}}{\pgfqpoint{9.084649in}{2.469502in}}%
\pgfpathcurveto{\pgfqpoint{9.076835in}{2.461688in}}{\pgfqpoint{9.072445in}{2.451089in}}{\pgfqpoint{9.072445in}{2.440039in}}%
\pgfpathcurveto{\pgfqpoint{9.072445in}{2.428989in}}{\pgfqpoint{9.076835in}{2.418390in}}{\pgfqpoint{9.084649in}{2.410576in}}%
\pgfpathcurveto{\pgfqpoint{9.092463in}{2.402762in}}{\pgfqpoint{9.103062in}{2.398372in}}{\pgfqpoint{9.114112in}{2.398372in}}%
\pgfpathlineto{\pgfqpoint{9.114112in}{2.398372in}}%
\pgfpathclose%
\pgfusepath{stroke}%
\end{pgfscope}%
\begin{pgfscope}%
\pgfpathrectangle{\pgfqpoint{7.512535in}{0.437222in}}{\pgfqpoint{6.275590in}{5.159444in}}%
\pgfusepath{clip}%
\pgfsetbuttcap%
\pgfsetroundjoin%
\pgfsetlinewidth{1.003750pt}%
\definecolor{currentstroke}{rgb}{1.000000,0.000000,0.000000}%
\pgfsetstrokecolor{currentstroke}%
\pgfsetdash{}{0pt}%
\pgfpathmoveto{\pgfqpoint{12.250750in}{5.554985in}}%
\pgfpathcurveto{\pgfqpoint{12.261800in}{5.554985in}}{\pgfqpoint{12.272399in}{5.559375in}}{\pgfqpoint{12.280213in}{5.567189in}}%
\pgfpathcurveto{\pgfqpoint{12.288027in}{5.575003in}}{\pgfqpoint{12.292417in}{5.585602in}}{\pgfqpoint{12.292417in}{5.596652in}}%
\pgfpathcurveto{\pgfqpoint{12.292417in}{5.607702in}}{\pgfqpoint{12.288027in}{5.618301in}}{\pgfqpoint{12.280213in}{5.626115in}}%
\pgfpathcurveto{\pgfqpoint{12.272399in}{5.633928in}}{\pgfqpoint{12.261800in}{5.638319in}}{\pgfqpoint{12.250750in}{5.638319in}}%
\pgfpathcurveto{\pgfqpoint{12.239700in}{5.638319in}}{\pgfqpoint{12.229101in}{5.633928in}}{\pgfqpoint{12.221287in}{5.626115in}}%
\pgfpathcurveto{\pgfqpoint{12.213474in}{5.618301in}}{\pgfqpoint{12.209084in}{5.607702in}}{\pgfqpoint{12.209084in}{5.596652in}}%
\pgfpathcurveto{\pgfqpoint{12.209084in}{5.585602in}}{\pgfqpoint{12.213474in}{5.575003in}}{\pgfqpoint{12.221287in}{5.567189in}}%
\pgfpathcurveto{\pgfqpoint{12.229101in}{5.559375in}}{\pgfqpoint{12.239700in}{5.554985in}}{\pgfqpoint{12.250750in}{5.554985in}}%
\pgfpathlineto{\pgfqpoint{12.250750in}{5.554985in}}%
\pgfpathclose%
\pgfusepath{stroke}%
\end{pgfscope}%
\begin{pgfscope}%
\pgfpathrectangle{\pgfqpoint{7.512535in}{0.437222in}}{\pgfqpoint{6.275590in}{5.159444in}}%
\pgfusepath{clip}%
\pgfsetbuttcap%
\pgfsetroundjoin%
\pgfsetlinewidth{1.003750pt}%
\definecolor{currentstroke}{rgb}{1.000000,0.000000,0.000000}%
\pgfsetstrokecolor{currentstroke}%
\pgfsetdash{}{0pt}%
\pgfpathmoveto{\pgfqpoint{10.919551in}{4.673217in}}%
\pgfpathcurveto{\pgfqpoint{10.930601in}{4.673217in}}{\pgfqpoint{10.941200in}{4.677608in}}{\pgfqpoint{10.949014in}{4.685421in}}%
\pgfpathcurveto{\pgfqpoint{10.956827in}{4.693235in}}{\pgfqpoint{10.961218in}{4.703834in}}{\pgfqpoint{10.961218in}{4.714884in}}%
\pgfpathcurveto{\pgfqpoint{10.961218in}{4.725934in}}{\pgfqpoint{10.956827in}{4.736533in}}{\pgfqpoint{10.949014in}{4.744347in}}%
\pgfpathcurveto{\pgfqpoint{10.941200in}{4.752160in}}{\pgfqpoint{10.930601in}{4.756551in}}{\pgfqpoint{10.919551in}{4.756551in}}%
\pgfpathcurveto{\pgfqpoint{10.908501in}{4.756551in}}{\pgfqpoint{10.897902in}{4.752160in}}{\pgfqpoint{10.890088in}{4.744347in}}%
\pgfpathcurveto{\pgfqpoint{10.882274in}{4.736533in}}{\pgfqpoint{10.877884in}{4.725934in}}{\pgfqpoint{10.877884in}{4.714884in}}%
\pgfpathcurveto{\pgfqpoint{10.877884in}{4.703834in}}{\pgfqpoint{10.882274in}{4.693235in}}{\pgfqpoint{10.890088in}{4.685421in}}%
\pgfpathcurveto{\pgfqpoint{10.897902in}{4.677608in}}{\pgfqpoint{10.908501in}{4.673217in}}{\pgfqpoint{10.919551in}{4.673217in}}%
\pgfpathlineto{\pgfqpoint{10.919551in}{4.673217in}}%
\pgfpathclose%
\pgfusepath{stroke}%
\end{pgfscope}%
\begin{pgfscope}%
\pgfpathrectangle{\pgfqpoint{7.512535in}{0.437222in}}{\pgfqpoint{6.275590in}{5.159444in}}%
\pgfusepath{clip}%
\pgfsetbuttcap%
\pgfsetroundjoin%
\pgfsetlinewidth{1.003750pt}%
\definecolor{currentstroke}{rgb}{1.000000,0.000000,0.000000}%
\pgfsetstrokecolor{currentstroke}%
\pgfsetdash{}{0pt}%
\pgfpathmoveto{\pgfqpoint{8.560362in}{1.906063in}}%
\pgfpathcurveto{\pgfqpoint{8.571412in}{1.906063in}}{\pgfqpoint{8.582011in}{1.910453in}}{\pgfqpoint{8.589825in}{1.918267in}}%
\pgfpathcurveto{\pgfqpoint{8.597638in}{1.926080in}}{\pgfqpoint{8.602029in}{1.936679in}}{\pgfqpoint{8.602029in}{1.947729in}}%
\pgfpathcurveto{\pgfqpoint{8.602029in}{1.958780in}}{\pgfqpoint{8.597638in}{1.969379in}}{\pgfqpoint{8.589825in}{1.977192in}}%
\pgfpathcurveto{\pgfqpoint{8.582011in}{1.985006in}}{\pgfqpoint{8.571412in}{1.989396in}}{\pgfqpoint{8.560362in}{1.989396in}}%
\pgfpathcurveto{\pgfqpoint{8.549312in}{1.989396in}}{\pgfqpoint{8.538713in}{1.985006in}}{\pgfqpoint{8.530899in}{1.977192in}}%
\pgfpathcurveto{\pgfqpoint{8.523086in}{1.969379in}}{\pgfqpoint{8.518695in}{1.958780in}}{\pgfqpoint{8.518695in}{1.947729in}}%
\pgfpathcurveto{\pgfqpoint{8.518695in}{1.936679in}}{\pgfqpoint{8.523086in}{1.926080in}}{\pgfqpoint{8.530899in}{1.918267in}}%
\pgfpathcurveto{\pgfqpoint{8.538713in}{1.910453in}}{\pgfqpoint{8.549312in}{1.906063in}}{\pgfqpoint{8.560362in}{1.906063in}}%
\pgfpathlineto{\pgfqpoint{8.560362in}{1.906063in}}%
\pgfpathclose%
\pgfusepath{stroke}%
\end{pgfscope}%
\begin{pgfscope}%
\pgfpathrectangle{\pgfqpoint{7.512535in}{0.437222in}}{\pgfqpoint{6.275590in}{5.159444in}}%
\pgfusepath{clip}%
\pgfsetbuttcap%
\pgfsetroundjoin%
\pgfsetlinewidth{1.003750pt}%
\definecolor{currentstroke}{rgb}{1.000000,0.000000,0.000000}%
\pgfsetstrokecolor{currentstroke}%
\pgfsetdash{}{0pt}%
\pgfpathmoveto{\pgfqpoint{10.260257in}{3.949675in}}%
\pgfpathcurveto{\pgfqpoint{10.271308in}{3.949675in}}{\pgfqpoint{10.281907in}{3.954066in}}{\pgfqpoint{10.289720in}{3.961879in}}%
\pgfpathcurveto{\pgfqpoint{10.297534in}{3.969693in}}{\pgfqpoint{10.301924in}{3.980292in}}{\pgfqpoint{10.301924in}{3.991342in}}%
\pgfpathcurveto{\pgfqpoint{10.301924in}{4.002392in}}{\pgfqpoint{10.297534in}{4.012991in}}{\pgfqpoint{10.289720in}{4.020805in}}%
\pgfpathcurveto{\pgfqpoint{10.281907in}{4.028619in}}{\pgfqpoint{10.271308in}{4.033009in}}{\pgfqpoint{10.260257in}{4.033009in}}%
\pgfpathcurveto{\pgfqpoint{10.249207in}{4.033009in}}{\pgfqpoint{10.238608in}{4.028619in}}{\pgfqpoint{10.230795in}{4.020805in}}%
\pgfpathcurveto{\pgfqpoint{10.222981in}{4.012991in}}{\pgfqpoint{10.218591in}{4.002392in}}{\pgfqpoint{10.218591in}{3.991342in}}%
\pgfpathcurveto{\pgfqpoint{10.218591in}{3.980292in}}{\pgfqpoint{10.222981in}{3.969693in}}{\pgfqpoint{10.230795in}{3.961879in}}%
\pgfpathcurveto{\pgfqpoint{10.238608in}{3.954066in}}{\pgfqpoint{10.249207in}{3.949675in}}{\pgfqpoint{10.260257in}{3.949675in}}%
\pgfpathlineto{\pgfqpoint{10.260257in}{3.949675in}}%
\pgfpathclose%
\pgfusepath{stroke}%
\end{pgfscope}%
\begin{pgfscope}%
\pgfpathrectangle{\pgfqpoint{7.512535in}{0.437222in}}{\pgfqpoint{6.275590in}{5.159444in}}%
\pgfusepath{clip}%
\pgfsetbuttcap%
\pgfsetroundjoin%
\pgfsetlinewidth{1.003750pt}%
\definecolor{currentstroke}{rgb}{1.000000,0.000000,0.000000}%
\pgfsetstrokecolor{currentstroke}%
\pgfsetdash{}{0pt}%
\pgfpathmoveto{\pgfqpoint{12.129426in}{5.522048in}}%
\pgfpathcurveto{\pgfqpoint{12.140476in}{5.522048in}}{\pgfqpoint{12.151075in}{5.526438in}}{\pgfqpoint{12.158889in}{5.534251in}}%
\pgfpathcurveto{\pgfqpoint{12.166703in}{5.542065in}}{\pgfqpoint{12.171093in}{5.552664in}}{\pgfqpoint{12.171093in}{5.563714in}}%
\pgfpathcurveto{\pgfqpoint{12.171093in}{5.574764in}}{\pgfqpoint{12.166703in}{5.585363in}}{\pgfqpoint{12.158889in}{5.593177in}}%
\pgfpathcurveto{\pgfqpoint{12.151075in}{5.600991in}}{\pgfqpoint{12.140476in}{5.605381in}}{\pgfqpoint{12.129426in}{5.605381in}}%
\pgfpathcurveto{\pgfqpoint{12.118376in}{5.605381in}}{\pgfqpoint{12.107777in}{5.600991in}}{\pgfqpoint{12.099963in}{5.593177in}}%
\pgfpathcurveto{\pgfqpoint{12.092150in}{5.585363in}}{\pgfqpoint{12.087760in}{5.574764in}}{\pgfqpoint{12.087760in}{5.563714in}}%
\pgfpathcurveto{\pgfqpoint{12.087760in}{5.552664in}}{\pgfqpoint{12.092150in}{5.542065in}}{\pgfqpoint{12.099963in}{5.534251in}}%
\pgfpathcurveto{\pgfqpoint{12.107777in}{5.526438in}}{\pgfqpoint{12.118376in}{5.522048in}}{\pgfqpoint{12.129426in}{5.522048in}}%
\pgfpathlineto{\pgfqpoint{12.129426in}{5.522048in}}%
\pgfpathclose%
\pgfusepath{stroke}%
\end{pgfscope}%
\begin{pgfscope}%
\pgfpathrectangle{\pgfqpoint{7.512535in}{0.437222in}}{\pgfqpoint{6.275590in}{5.159444in}}%
\pgfusepath{clip}%
\pgfsetbuttcap%
\pgfsetroundjoin%
\pgfsetlinewidth{1.003750pt}%
\definecolor{currentstroke}{rgb}{1.000000,0.000000,0.000000}%
\pgfsetstrokecolor{currentstroke}%
\pgfsetdash{}{0pt}%
\pgfpathmoveto{\pgfqpoint{13.057760in}{5.554992in}}%
\pgfpathcurveto{\pgfqpoint{13.068810in}{5.554992in}}{\pgfqpoint{13.079409in}{5.559383in}}{\pgfqpoint{13.087223in}{5.567196in}}%
\pgfpathcurveto{\pgfqpoint{13.095037in}{5.575010in}}{\pgfqpoint{13.099427in}{5.585609in}}{\pgfqpoint{13.099427in}{5.596659in}}%
\pgfpathcurveto{\pgfqpoint{13.099427in}{5.607709in}}{\pgfqpoint{13.095037in}{5.618308in}}{\pgfqpoint{13.087223in}{5.626122in}}%
\pgfpathcurveto{\pgfqpoint{13.079409in}{5.633936in}}{\pgfqpoint{13.068810in}{5.638326in}}{\pgfqpoint{13.057760in}{5.638326in}}%
\pgfpathcurveto{\pgfqpoint{13.046710in}{5.638326in}}{\pgfqpoint{13.036111in}{5.633936in}}{\pgfqpoint{13.028297in}{5.626122in}}%
\pgfpathcurveto{\pgfqpoint{13.020484in}{5.618308in}}{\pgfqpoint{13.016094in}{5.607709in}}{\pgfqpoint{13.016094in}{5.596659in}}%
\pgfpathcurveto{\pgfqpoint{13.016094in}{5.585609in}}{\pgfqpoint{13.020484in}{5.575010in}}{\pgfqpoint{13.028297in}{5.567196in}}%
\pgfpathcurveto{\pgfqpoint{13.036111in}{5.559383in}}{\pgfqpoint{13.046710in}{5.554992in}}{\pgfqpoint{13.057760in}{5.554992in}}%
\pgfpathlineto{\pgfqpoint{13.057760in}{5.554992in}}%
\pgfpathclose%
\pgfusepath{stroke}%
\end{pgfscope}%
\begin{pgfscope}%
\pgfpathrectangle{\pgfqpoint{7.512535in}{0.437222in}}{\pgfqpoint{6.275590in}{5.159444in}}%
\pgfusepath{clip}%
\pgfsetbuttcap%
\pgfsetroundjoin%
\pgfsetlinewidth{1.003750pt}%
\definecolor{currentstroke}{rgb}{1.000000,0.000000,0.000000}%
\pgfsetstrokecolor{currentstroke}%
\pgfsetdash{}{0pt}%
\pgfpathmoveto{\pgfqpoint{9.725110in}{3.336201in}}%
\pgfpathcurveto{\pgfqpoint{9.736160in}{3.336201in}}{\pgfqpoint{9.746759in}{3.340591in}}{\pgfqpoint{9.754573in}{3.348404in}}%
\pgfpathcurveto{\pgfqpoint{9.762386in}{3.356218in}}{\pgfqpoint{9.766777in}{3.366817in}}{\pgfqpoint{9.766777in}{3.377867in}}%
\pgfpathcurveto{\pgfqpoint{9.766777in}{3.388917in}}{\pgfqpoint{9.762386in}{3.399516in}}{\pgfqpoint{9.754573in}{3.407330in}}%
\pgfpathcurveto{\pgfqpoint{9.746759in}{3.415144in}}{\pgfqpoint{9.736160in}{3.419534in}}{\pgfqpoint{9.725110in}{3.419534in}}%
\pgfpathcurveto{\pgfqpoint{9.714060in}{3.419534in}}{\pgfqpoint{9.703461in}{3.415144in}}{\pgfqpoint{9.695647in}{3.407330in}}%
\pgfpathcurveto{\pgfqpoint{9.687834in}{3.399516in}}{\pgfqpoint{9.683443in}{3.388917in}}{\pgfqpoint{9.683443in}{3.377867in}}%
\pgfpathcurveto{\pgfqpoint{9.683443in}{3.366817in}}{\pgfqpoint{9.687834in}{3.356218in}}{\pgfqpoint{9.695647in}{3.348404in}}%
\pgfpathcurveto{\pgfqpoint{9.703461in}{3.340591in}}{\pgfqpoint{9.714060in}{3.336201in}}{\pgfqpoint{9.725110in}{3.336201in}}%
\pgfpathlineto{\pgfqpoint{9.725110in}{3.336201in}}%
\pgfpathclose%
\pgfusepath{stroke}%
\end{pgfscope}%
\begin{pgfscope}%
\pgfpathrectangle{\pgfqpoint{7.512535in}{0.437222in}}{\pgfqpoint{6.275590in}{5.159444in}}%
\pgfusepath{clip}%
\pgfsetbuttcap%
\pgfsetroundjoin%
\pgfsetlinewidth{1.003750pt}%
\definecolor{currentstroke}{rgb}{1.000000,0.000000,0.000000}%
\pgfsetstrokecolor{currentstroke}%
\pgfsetdash{}{0pt}%
\pgfpathmoveto{\pgfqpoint{9.575505in}{2.933739in}}%
\pgfpathcurveto{\pgfqpoint{9.586555in}{2.933739in}}{\pgfqpoint{9.597154in}{2.938129in}}{\pgfqpoint{9.604968in}{2.945942in}}%
\pgfpathcurveto{\pgfqpoint{9.612781in}{2.953756in}}{\pgfqpoint{9.617171in}{2.964355in}}{\pgfqpoint{9.617171in}{2.975405in}}%
\pgfpathcurveto{\pgfqpoint{9.617171in}{2.986455in}}{\pgfqpoint{9.612781in}{2.997054in}}{\pgfqpoint{9.604968in}{3.004868in}}%
\pgfpathcurveto{\pgfqpoint{9.597154in}{3.012682in}}{\pgfqpoint{9.586555in}{3.017072in}}{\pgfqpoint{9.575505in}{3.017072in}}%
\pgfpathcurveto{\pgfqpoint{9.564455in}{3.017072in}}{\pgfqpoint{9.553856in}{3.012682in}}{\pgfqpoint{9.546042in}{3.004868in}}%
\pgfpathcurveto{\pgfqpoint{9.538228in}{2.997054in}}{\pgfqpoint{9.533838in}{2.986455in}}{\pgfqpoint{9.533838in}{2.975405in}}%
\pgfpathcurveto{\pgfqpoint{9.533838in}{2.964355in}}{\pgfqpoint{9.538228in}{2.953756in}}{\pgfqpoint{9.546042in}{2.945942in}}%
\pgfpathcurveto{\pgfqpoint{9.553856in}{2.938129in}}{\pgfqpoint{9.564455in}{2.933739in}}{\pgfqpoint{9.575505in}{2.933739in}}%
\pgfpathlineto{\pgfqpoint{9.575505in}{2.933739in}}%
\pgfpathclose%
\pgfusepath{stroke}%
\end{pgfscope}%
\begin{pgfscope}%
\pgfpathrectangle{\pgfqpoint{7.512535in}{0.437222in}}{\pgfqpoint{6.275590in}{5.159444in}}%
\pgfusepath{clip}%
\pgfsetbuttcap%
\pgfsetroundjoin%
\pgfsetlinewidth{1.003750pt}%
\definecolor{currentstroke}{rgb}{1.000000,0.000000,0.000000}%
\pgfsetstrokecolor{currentstroke}%
\pgfsetdash{}{0pt}%
\pgfpathmoveto{\pgfqpoint{10.369244in}{4.347240in}}%
\pgfpathcurveto{\pgfqpoint{10.380295in}{4.347240in}}{\pgfqpoint{10.390894in}{4.351630in}}{\pgfqpoint{10.398707in}{4.359444in}}%
\pgfpathcurveto{\pgfqpoint{10.406521in}{4.367258in}}{\pgfqpoint{10.410911in}{4.377857in}}{\pgfqpoint{10.410911in}{4.388907in}}%
\pgfpathcurveto{\pgfqpoint{10.410911in}{4.399957in}}{\pgfqpoint{10.406521in}{4.410556in}}{\pgfqpoint{10.398707in}{4.418369in}}%
\pgfpathcurveto{\pgfqpoint{10.390894in}{4.426183in}}{\pgfqpoint{10.380295in}{4.430573in}}{\pgfqpoint{10.369244in}{4.430573in}}%
\pgfpathcurveto{\pgfqpoint{10.358194in}{4.430573in}}{\pgfqpoint{10.347595in}{4.426183in}}{\pgfqpoint{10.339782in}{4.418369in}}%
\pgfpathcurveto{\pgfqpoint{10.331968in}{4.410556in}}{\pgfqpoint{10.327578in}{4.399957in}}{\pgfqpoint{10.327578in}{4.388907in}}%
\pgfpathcurveto{\pgfqpoint{10.327578in}{4.377857in}}{\pgfqpoint{10.331968in}{4.367258in}}{\pgfqpoint{10.339782in}{4.359444in}}%
\pgfpathcurveto{\pgfqpoint{10.347595in}{4.351630in}}{\pgfqpoint{10.358194in}{4.347240in}}{\pgfqpoint{10.369244in}{4.347240in}}%
\pgfpathlineto{\pgfqpoint{10.369244in}{4.347240in}}%
\pgfpathclose%
\pgfusepath{stroke}%
\end{pgfscope}%
\begin{pgfscope}%
\pgfpathrectangle{\pgfqpoint{7.512535in}{0.437222in}}{\pgfqpoint{6.275590in}{5.159444in}}%
\pgfusepath{clip}%
\pgfsetbuttcap%
\pgfsetroundjoin%
\pgfsetlinewidth{1.003750pt}%
\definecolor{currentstroke}{rgb}{1.000000,0.000000,0.000000}%
\pgfsetstrokecolor{currentstroke}%
\pgfsetdash{}{0pt}%
\pgfpathmoveto{\pgfqpoint{8.580198in}{1.738363in}}%
\pgfpathcurveto{\pgfqpoint{8.591248in}{1.738363in}}{\pgfqpoint{8.601847in}{1.742753in}}{\pgfqpoint{8.609661in}{1.750567in}}%
\pgfpathcurveto{\pgfqpoint{8.617474in}{1.758381in}}{\pgfqpoint{8.621865in}{1.768980in}}{\pgfqpoint{8.621865in}{1.780030in}}%
\pgfpathcurveto{\pgfqpoint{8.621865in}{1.791080in}}{\pgfqpoint{8.617474in}{1.801679in}}{\pgfqpoint{8.609661in}{1.809493in}}%
\pgfpathcurveto{\pgfqpoint{8.601847in}{1.817306in}}{\pgfqpoint{8.591248in}{1.821696in}}{\pgfqpoint{8.580198in}{1.821696in}}%
\pgfpathcurveto{\pgfqpoint{8.569148in}{1.821696in}}{\pgfqpoint{8.558549in}{1.817306in}}{\pgfqpoint{8.550735in}{1.809493in}}%
\pgfpathcurveto{\pgfqpoint{8.542922in}{1.801679in}}{\pgfqpoint{8.538531in}{1.791080in}}{\pgfqpoint{8.538531in}{1.780030in}}%
\pgfpathcurveto{\pgfqpoint{8.538531in}{1.768980in}}{\pgfqpoint{8.542922in}{1.758381in}}{\pgfqpoint{8.550735in}{1.750567in}}%
\pgfpathcurveto{\pgfqpoint{8.558549in}{1.742753in}}{\pgfqpoint{8.569148in}{1.738363in}}{\pgfqpoint{8.580198in}{1.738363in}}%
\pgfpathlineto{\pgfqpoint{8.580198in}{1.738363in}}%
\pgfpathclose%
\pgfusepath{stroke}%
\end{pgfscope}%
\begin{pgfscope}%
\pgfpathrectangle{\pgfqpoint{7.512535in}{0.437222in}}{\pgfqpoint{6.275590in}{5.159444in}}%
\pgfusepath{clip}%
\pgfsetbuttcap%
\pgfsetroundjoin%
\pgfsetlinewidth{1.003750pt}%
\definecolor{currentstroke}{rgb}{1.000000,0.000000,0.000000}%
\pgfsetstrokecolor{currentstroke}%
\pgfsetdash{}{0pt}%
\pgfpathmoveto{\pgfqpoint{9.685797in}{2.859143in}}%
\pgfpathcurveto{\pgfqpoint{9.696847in}{2.859143in}}{\pgfqpoint{9.707446in}{2.863533in}}{\pgfqpoint{9.715260in}{2.871347in}}%
\pgfpathcurveto{\pgfqpoint{9.723073in}{2.879160in}}{\pgfqpoint{9.727464in}{2.889759in}}{\pgfqpoint{9.727464in}{2.900810in}}%
\pgfpathcurveto{\pgfqpoint{9.727464in}{2.911860in}}{\pgfqpoint{9.723073in}{2.922459in}}{\pgfqpoint{9.715260in}{2.930272in}}%
\pgfpathcurveto{\pgfqpoint{9.707446in}{2.938086in}}{\pgfqpoint{9.696847in}{2.942476in}}{\pgfqpoint{9.685797in}{2.942476in}}%
\pgfpathcurveto{\pgfqpoint{9.674747in}{2.942476in}}{\pgfqpoint{9.664148in}{2.938086in}}{\pgfqpoint{9.656334in}{2.930272in}}%
\pgfpathcurveto{\pgfqpoint{9.648521in}{2.922459in}}{\pgfqpoint{9.644130in}{2.911860in}}{\pgfqpoint{9.644130in}{2.900810in}}%
\pgfpathcurveto{\pgfqpoint{9.644130in}{2.889759in}}{\pgfqpoint{9.648521in}{2.879160in}}{\pgfqpoint{9.656334in}{2.871347in}}%
\pgfpathcurveto{\pgfqpoint{9.664148in}{2.863533in}}{\pgfqpoint{9.674747in}{2.859143in}}{\pgfqpoint{9.685797in}{2.859143in}}%
\pgfpathlineto{\pgfqpoint{9.685797in}{2.859143in}}%
\pgfpathclose%
\pgfusepath{stroke}%
\end{pgfscope}%
\begin{pgfscope}%
\pgfpathrectangle{\pgfqpoint{7.512535in}{0.437222in}}{\pgfqpoint{6.275590in}{5.159444in}}%
\pgfusepath{clip}%
\pgfsetbuttcap%
\pgfsetroundjoin%
\pgfsetlinewidth{1.003750pt}%
\definecolor{currentstroke}{rgb}{1.000000,0.000000,0.000000}%
\pgfsetstrokecolor{currentstroke}%
\pgfsetdash{}{0pt}%
\pgfpathmoveto{\pgfqpoint{12.283175in}{5.554999in}}%
\pgfpathcurveto{\pgfqpoint{12.294225in}{5.554999in}}{\pgfqpoint{12.304824in}{5.559389in}}{\pgfqpoint{12.312638in}{5.567203in}}%
\pgfpathcurveto{\pgfqpoint{12.320452in}{5.575017in}}{\pgfqpoint{12.324842in}{5.585616in}}{\pgfqpoint{12.324842in}{5.596666in}}%
\pgfpathcurveto{\pgfqpoint{12.324842in}{5.607716in}}{\pgfqpoint{12.320452in}{5.618315in}}{\pgfqpoint{12.312638in}{5.626129in}}%
\pgfpathcurveto{\pgfqpoint{12.304824in}{5.633942in}}{\pgfqpoint{12.294225in}{5.638333in}}{\pgfqpoint{12.283175in}{5.638333in}}%
\pgfpathcurveto{\pgfqpoint{12.272125in}{5.638333in}}{\pgfqpoint{12.261526in}{5.633942in}}{\pgfqpoint{12.253712in}{5.626129in}}%
\pgfpathcurveto{\pgfqpoint{12.245899in}{5.618315in}}{\pgfqpoint{12.241509in}{5.607716in}}{\pgfqpoint{12.241509in}{5.596666in}}%
\pgfpathcurveto{\pgfqpoint{12.241509in}{5.585616in}}{\pgfqpoint{12.245899in}{5.575017in}}{\pgfqpoint{12.253712in}{5.567203in}}%
\pgfpathcurveto{\pgfqpoint{12.261526in}{5.559389in}}{\pgfqpoint{12.272125in}{5.554999in}}{\pgfqpoint{12.283175in}{5.554999in}}%
\pgfpathlineto{\pgfqpoint{12.283175in}{5.554999in}}%
\pgfpathclose%
\pgfusepath{stroke}%
\end{pgfscope}%
\begin{pgfscope}%
\pgfpathrectangle{\pgfqpoint{7.512535in}{0.437222in}}{\pgfqpoint{6.275590in}{5.159444in}}%
\pgfusepath{clip}%
\pgfsetbuttcap%
\pgfsetroundjoin%
\pgfsetlinewidth{1.003750pt}%
\definecolor{currentstroke}{rgb}{1.000000,0.000000,0.000000}%
\pgfsetstrokecolor{currentstroke}%
\pgfsetdash{}{0pt}%
\pgfpathmoveto{\pgfqpoint{11.107483in}{5.553792in}}%
\pgfpathcurveto{\pgfqpoint{11.118533in}{5.553792in}}{\pgfqpoint{11.129132in}{5.558182in}}{\pgfqpoint{11.136945in}{5.565996in}}%
\pgfpathcurveto{\pgfqpoint{11.144759in}{5.573809in}}{\pgfqpoint{11.149149in}{5.584408in}}{\pgfqpoint{11.149149in}{5.595458in}}%
\pgfpathcurveto{\pgfqpoint{11.149149in}{5.606509in}}{\pgfqpoint{11.144759in}{5.617108in}}{\pgfqpoint{11.136945in}{5.624921in}}%
\pgfpathcurveto{\pgfqpoint{11.129132in}{5.632735in}}{\pgfqpoint{11.118533in}{5.637125in}}{\pgfqpoint{11.107483in}{5.637125in}}%
\pgfpathcurveto{\pgfqpoint{11.096433in}{5.637125in}}{\pgfqpoint{11.085834in}{5.632735in}}{\pgfqpoint{11.078020in}{5.624921in}}%
\pgfpathcurveto{\pgfqpoint{11.070206in}{5.617108in}}{\pgfqpoint{11.065816in}{5.606509in}}{\pgfqpoint{11.065816in}{5.595458in}}%
\pgfpathcurveto{\pgfqpoint{11.065816in}{5.584408in}}{\pgfqpoint{11.070206in}{5.573809in}}{\pgfqpoint{11.078020in}{5.565996in}}%
\pgfpathcurveto{\pgfqpoint{11.085834in}{5.558182in}}{\pgfqpoint{11.096433in}{5.553792in}}{\pgfqpoint{11.107483in}{5.553792in}}%
\pgfpathlineto{\pgfqpoint{11.107483in}{5.553792in}}%
\pgfpathclose%
\pgfusepath{stroke}%
\end{pgfscope}%
\begin{pgfscope}%
\pgfpathrectangle{\pgfqpoint{7.512535in}{0.437222in}}{\pgfqpoint{6.275590in}{5.159444in}}%
\pgfusepath{clip}%
\pgfsetbuttcap%
\pgfsetroundjoin%
\pgfsetlinewidth{1.003750pt}%
\definecolor{currentstroke}{rgb}{1.000000,0.000000,0.000000}%
\pgfsetstrokecolor{currentstroke}%
\pgfsetdash{}{0pt}%
\pgfpathmoveto{\pgfqpoint{11.855723in}{5.554997in}}%
\pgfpathcurveto{\pgfqpoint{11.866773in}{5.554997in}}{\pgfqpoint{11.877372in}{5.559387in}}{\pgfqpoint{11.885186in}{5.567200in}}%
\pgfpathcurveto{\pgfqpoint{11.893000in}{5.575014in}}{\pgfqpoint{11.897390in}{5.585613in}}{\pgfqpoint{11.897390in}{5.596663in}}%
\pgfpathcurveto{\pgfqpoint{11.897390in}{5.607713in}}{\pgfqpoint{11.893000in}{5.618312in}}{\pgfqpoint{11.885186in}{5.626126in}}%
\pgfpathcurveto{\pgfqpoint{11.877372in}{5.633940in}}{\pgfqpoint{11.866773in}{5.638330in}}{\pgfqpoint{11.855723in}{5.638330in}}%
\pgfpathcurveto{\pgfqpoint{11.844673in}{5.638330in}}{\pgfqpoint{11.834074in}{5.633940in}}{\pgfqpoint{11.826261in}{5.626126in}}%
\pgfpathcurveto{\pgfqpoint{11.818447in}{5.618312in}}{\pgfqpoint{11.814057in}{5.607713in}}{\pgfqpoint{11.814057in}{5.596663in}}%
\pgfpathcurveto{\pgfqpoint{11.814057in}{5.585613in}}{\pgfqpoint{11.818447in}{5.575014in}}{\pgfqpoint{11.826261in}{5.567200in}}%
\pgfpathcurveto{\pgfqpoint{11.834074in}{5.559387in}}{\pgfqpoint{11.844673in}{5.554997in}}{\pgfqpoint{11.855723in}{5.554997in}}%
\pgfpathlineto{\pgfqpoint{11.855723in}{5.554997in}}%
\pgfpathclose%
\pgfusepath{stroke}%
\end{pgfscope}%
\begin{pgfscope}%
\pgfpathrectangle{\pgfqpoint{7.512535in}{0.437222in}}{\pgfqpoint{6.275590in}{5.159444in}}%
\pgfusepath{clip}%
\pgfsetbuttcap%
\pgfsetroundjoin%
\pgfsetlinewidth{1.003750pt}%
\definecolor{currentstroke}{rgb}{1.000000,0.000000,0.000000}%
\pgfsetstrokecolor{currentstroke}%
\pgfsetdash{}{0pt}%
\pgfpathmoveto{\pgfqpoint{11.260490in}{5.537377in}}%
\pgfpathcurveto{\pgfqpoint{11.271540in}{5.537377in}}{\pgfqpoint{11.282139in}{5.541767in}}{\pgfqpoint{11.289953in}{5.549581in}}%
\pgfpathcurveto{\pgfqpoint{11.297766in}{5.557394in}}{\pgfqpoint{11.302157in}{5.567994in}}{\pgfqpoint{11.302157in}{5.579044in}}%
\pgfpathcurveto{\pgfqpoint{11.302157in}{5.590094in}}{\pgfqpoint{11.297766in}{5.600693in}}{\pgfqpoint{11.289953in}{5.608506in}}%
\pgfpathcurveto{\pgfqpoint{11.282139in}{5.616320in}}{\pgfqpoint{11.271540in}{5.620710in}}{\pgfqpoint{11.260490in}{5.620710in}}%
\pgfpathcurveto{\pgfqpoint{11.249440in}{5.620710in}}{\pgfqpoint{11.238841in}{5.616320in}}{\pgfqpoint{11.231027in}{5.608506in}}%
\pgfpathcurveto{\pgfqpoint{11.223214in}{5.600693in}}{\pgfqpoint{11.218823in}{5.590094in}}{\pgfqpoint{11.218823in}{5.579044in}}%
\pgfpathcurveto{\pgfqpoint{11.218823in}{5.567994in}}{\pgfqpoint{11.223214in}{5.557394in}}{\pgfqpoint{11.231027in}{5.549581in}}%
\pgfpathcurveto{\pgfqpoint{11.238841in}{5.541767in}}{\pgfqpoint{11.249440in}{5.537377in}}{\pgfqpoint{11.260490in}{5.537377in}}%
\pgfpathlineto{\pgfqpoint{11.260490in}{5.537377in}}%
\pgfpathclose%
\pgfusepath{stroke}%
\end{pgfscope}%
\begin{pgfscope}%
\pgfpathrectangle{\pgfqpoint{7.512535in}{0.437222in}}{\pgfqpoint{6.275590in}{5.159444in}}%
\pgfusepath{clip}%
\pgfsetbuttcap%
\pgfsetroundjoin%
\pgfsetlinewidth{1.003750pt}%
\definecolor{currentstroke}{rgb}{1.000000,0.000000,0.000000}%
\pgfsetstrokecolor{currentstroke}%
\pgfsetdash{}{0pt}%
\pgfpathmoveto{\pgfqpoint{10.266113in}{3.569865in}}%
\pgfpathcurveto{\pgfqpoint{10.277163in}{3.569865in}}{\pgfqpoint{10.287762in}{3.574255in}}{\pgfqpoint{10.295576in}{3.582068in}}%
\pgfpathcurveto{\pgfqpoint{10.303389in}{3.589882in}}{\pgfqpoint{10.307780in}{3.600481in}}{\pgfqpoint{10.307780in}{3.611531in}}%
\pgfpathcurveto{\pgfqpoint{10.307780in}{3.622581in}}{\pgfqpoint{10.303389in}{3.633180in}}{\pgfqpoint{10.295576in}{3.640994in}}%
\pgfpathcurveto{\pgfqpoint{10.287762in}{3.648808in}}{\pgfqpoint{10.277163in}{3.653198in}}{\pgfqpoint{10.266113in}{3.653198in}}%
\pgfpathcurveto{\pgfqpoint{10.255063in}{3.653198in}}{\pgfqpoint{10.244464in}{3.648808in}}{\pgfqpoint{10.236650in}{3.640994in}}%
\pgfpathcurveto{\pgfqpoint{10.228837in}{3.633180in}}{\pgfqpoint{10.224446in}{3.622581in}}{\pgfqpoint{10.224446in}{3.611531in}}%
\pgfpathcurveto{\pgfqpoint{10.224446in}{3.600481in}}{\pgfqpoint{10.228837in}{3.589882in}}{\pgfqpoint{10.236650in}{3.582068in}}%
\pgfpathcurveto{\pgfqpoint{10.244464in}{3.574255in}}{\pgfqpoint{10.255063in}{3.569865in}}{\pgfqpoint{10.266113in}{3.569865in}}%
\pgfpathlineto{\pgfqpoint{10.266113in}{3.569865in}}%
\pgfpathclose%
\pgfusepath{stroke}%
\end{pgfscope}%
\begin{pgfscope}%
\pgfpathrectangle{\pgfqpoint{7.512535in}{0.437222in}}{\pgfqpoint{6.275590in}{5.159444in}}%
\pgfusepath{clip}%
\pgfsetbuttcap%
\pgfsetroundjoin%
\pgfsetlinewidth{1.003750pt}%
\definecolor{currentstroke}{rgb}{1.000000,0.000000,0.000000}%
\pgfsetstrokecolor{currentstroke}%
\pgfsetdash{}{0pt}%
\pgfpathmoveto{\pgfqpoint{10.157413in}{3.906545in}}%
\pgfpathcurveto{\pgfqpoint{10.168463in}{3.906545in}}{\pgfqpoint{10.179062in}{3.910935in}}{\pgfqpoint{10.186876in}{3.918749in}}%
\pgfpathcurveto{\pgfqpoint{10.194690in}{3.926562in}}{\pgfqpoint{10.199080in}{3.937161in}}{\pgfqpoint{10.199080in}{3.948211in}}%
\pgfpathcurveto{\pgfqpoint{10.199080in}{3.959261in}}{\pgfqpoint{10.194690in}{3.969860in}}{\pgfqpoint{10.186876in}{3.977674in}}%
\pgfpathcurveto{\pgfqpoint{10.179062in}{3.985488in}}{\pgfqpoint{10.168463in}{3.989878in}}{\pgfqpoint{10.157413in}{3.989878in}}%
\pgfpathcurveto{\pgfqpoint{10.146363in}{3.989878in}}{\pgfqpoint{10.135764in}{3.985488in}}{\pgfqpoint{10.127950in}{3.977674in}}%
\pgfpathcurveto{\pgfqpoint{10.120137in}{3.969860in}}{\pgfqpoint{10.115747in}{3.959261in}}{\pgfqpoint{10.115747in}{3.948211in}}%
\pgfpathcurveto{\pgfqpoint{10.115747in}{3.937161in}}{\pgfqpoint{10.120137in}{3.926562in}}{\pgfqpoint{10.127950in}{3.918749in}}%
\pgfpathcurveto{\pgfqpoint{10.135764in}{3.910935in}}{\pgfqpoint{10.146363in}{3.906545in}}{\pgfqpoint{10.157413in}{3.906545in}}%
\pgfpathlineto{\pgfqpoint{10.157413in}{3.906545in}}%
\pgfpathclose%
\pgfusepath{stroke}%
\end{pgfscope}%
\begin{pgfscope}%
\pgfpathrectangle{\pgfqpoint{7.512535in}{0.437222in}}{\pgfqpoint{6.275590in}{5.159444in}}%
\pgfusepath{clip}%
\pgfsetbuttcap%
\pgfsetroundjoin%
\pgfsetlinewidth{1.003750pt}%
\definecolor{currentstroke}{rgb}{1.000000,0.000000,0.000000}%
\pgfsetstrokecolor{currentstroke}%
\pgfsetdash{}{0pt}%
\pgfpathmoveto{\pgfqpoint{13.489543in}{5.554434in}}%
\pgfpathcurveto{\pgfqpoint{13.500593in}{5.554434in}}{\pgfqpoint{13.511192in}{5.558824in}}{\pgfqpoint{13.519006in}{5.566638in}}%
\pgfpathcurveto{\pgfqpoint{13.526820in}{5.574452in}}{\pgfqpoint{13.531210in}{5.585051in}}{\pgfqpoint{13.531210in}{5.596101in}}%
\pgfpathcurveto{\pgfqpoint{13.531210in}{5.607151in}}{\pgfqpoint{13.526820in}{5.617750in}}{\pgfqpoint{13.519006in}{5.625564in}}%
\pgfpathcurveto{\pgfqpoint{13.511192in}{5.633377in}}{\pgfqpoint{13.500593in}{5.637767in}}{\pgfqpoint{13.489543in}{5.637767in}}%
\pgfpathcurveto{\pgfqpoint{13.478493in}{5.637767in}}{\pgfqpoint{13.467894in}{5.633377in}}{\pgfqpoint{13.460081in}{5.625564in}}%
\pgfpathcurveto{\pgfqpoint{13.452267in}{5.617750in}}{\pgfqpoint{13.447877in}{5.607151in}}{\pgfqpoint{13.447877in}{5.596101in}}%
\pgfpathcurveto{\pgfqpoint{13.447877in}{5.585051in}}{\pgfqpoint{13.452267in}{5.574452in}}{\pgfqpoint{13.460081in}{5.566638in}}%
\pgfpathcurveto{\pgfqpoint{13.467894in}{5.558824in}}{\pgfqpoint{13.478493in}{5.554434in}}{\pgfqpoint{13.489543in}{5.554434in}}%
\pgfpathlineto{\pgfqpoint{13.489543in}{5.554434in}}%
\pgfpathclose%
\pgfusepath{stroke}%
\end{pgfscope}%
\begin{pgfscope}%
\pgfpathrectangle{\pgfqpoint{7.512535in}{0.437222in}}{\pgfqpoint{6.275590in}{5.159444in}}%
\pgfusepath{clip}%
\pgfsetbuttcap%
\pgfsetroundjoin%
\pgfsetlinewidth{1.003750pt}%
\definecolor{currentstroke}{rgb}{1.000000,0.000000,0.000000}%
\pgfsetstrokecolor{currentstroke}%
\pgfsetdash{}{0pt}%
\pgfpathmoveto{\pgfqpoint{12.001381in}{5.537377in}}%
\pgfpathcurveto{\pgfqpoint{12.012432in}{5.537377in}}{\pgfqpoint{12.023031in}{5.541767in}}{\pgfqpoint{12.030844in}{5.549581in}}%
\pgfpathcurveto{\pgfqpoint{12.038658in}{5.557394in}}{\pgfqpoint{12.043048in}{5.567994in}}{\pgfqpoint{12.043048in}{5.579044in}}%
\pgfpathcurveto{\pgfqpoint{12.043048in}{5.590094in}}{\pgfqpoint{12.038658in}{5.600693in}}{\pgfqpoint{12.030844in}{5.608506in}}%
\pgfpathcurveto{\pgfqpoint{12.023031in}{5.616320in}}{\pgfqpoint{12.012432in}{5.620710in}}{\pgfqpoint{12.001381in}{5.620710in}}%
\pgfpathcurveto{\pgfqpoint{11.990331in}{5.620710in}}{\pgfqpoint{11.979732in}{5.616320in}}{\pgfqpoint{11.971919in}{5.608506in}}%
\pgfpathcurveto{\pgfqpoint{11.964105in}{5.600693in}}{\pgfqpoint{11.959715in}{5.590094in}}{\pgfqpoint{11.959715in}{5.579044in}}%
\pgfpathcurveto{\pgfqpoint{11.959715in}{5.567994in}}{\pgfqpoint{11.964105in}{5.557394in}}{\pgfqpoint{11.971919in}{5.549581in}}%
\pgfpathcurveto{\pgfqpoint{11.979732in}{5.541767in}}{\pgfqpoint{11.990331in}{5.537377in}}{\pgfqpoint{12.001381in}{5.537377in}}%
\pgfpathlineto{\pgfqpoint{12.001381in}{5.537377in}}%
\pgfpathclose%
\pgfusepath{stroke}%
\end{pgfscope}%
\begin{pgfscope}%
\pgfpathrectangle{\pgfqpoint{7.512535in}{0.437222in}}{\pgfqpoint{6.275590in}{5.159444in}}%
\pgfusepath{clip}%
\pgfsetbuttcap%
\pgfsetroundjoin%
\pgfsetlinewidth{1.003750pt}%
\definecolor{currentstroke}{rgb}{1.000000,0.000000,0.000000}%
\pgfsetstrokecolor{currentstroke}%
\pgfsetdash{}{0pt}%
\pgfpathmoveto{\pgfqpoint{12.736794in}{5.531534in}}%
\pgfpathcurveto{\pgfqpoint{12.747844in}{5.531534in}}{\pgfqpoint{12.758443in}{5.535924in}}{\pgfqpoint{12.766257in}{5.543738in}}%
\pgfpathcurveto{\pgfqpoint{12.774070in}{5.551552in}}{\pgfqpoint{12.778461in}{5.562151in}}{\pgfqpoint{12.778461in}{5.573201in}}%
\pgfpathcurveto{\pgfqpoint{12.778461in}{5.584251in}}{\pgfqpoint{12.774070in}{5.594850in}}{\pgfqpoint{12.766257in}{5.602664in}}%
\pgfpathcurveto{\pgfqpoint{12.758443in}{5.610477in}}{\pgfqpoint{12.747844in}{5.614867in}}{\pgfqpoint{12.736794in}{5.614867in}}%
\pgfpathcurveto{\pgfqpoint{12.725744in}{5.614867in}}{\pgfqpoint{12.715145in}{5.610477in}}{\pgfqpoint{12.707331in}{5.602664in}}%
\pgfpathcurveto{\pgfqpoint{12.699518in}{5.594850in}}{\pgfqpoint{12.695127in}{5.584251in}}{\pgfqpoint{12.695127in}{5.573201in}}%
\pgfpathcurveto{\pgfqpoint{12.695127in}{5.562151in}}{\pgfqpoint{12.699518in}{5.551552in}}{\pgfqpoint{12.707331in}{5.543738in}}%
\pgfpathcurveto{\pgfqpoint{12.715145in}{5.535924in}}{\pgfqpoint{12.725744in}{5.531534in}}{\pgfqpoint{12.736794in}{5.531534in}}%
\pgfpathlineto{\pgfqpoint{12.736794in}{5.531534in}}%
\pgfpathclose%
\pgfusepath{stroke}%
\end{pgfscope}%
\begin{pgfscope}%
\pgfpathrectangle{\pgfqpoint{7.512535in}{0.437222in}}{\pgfqpoint{6.275590in}{5.159444in}}%
\pgfusepath{clip}%
\pgfsetbuttcap%
\pgfsetroundjoin%
\pgfsetlinewidth{1.003750pt}%
\definecolor{currentstroke}{rgb}{1.000000,0.000000,0.000000}%
\pgfsetstrokecolor{currentstroke}%
\pgfsetdash{}{0pt}%
\pgfpathmoveto{\pgfqpoint{8.722256in}{1.812532in}}%
\pgfpathcurveto{\pgfqpoint{8.733306in}{1.812532in}}{\pgfqpoint{8.743905in}{1.816922in}}{\pgfqpoint{8.751719in}{1.824736in}}%
\pgfpathcurveto{\pgfqpoint{8.759532in}{1.832549in}}{\pgfqpoint{8.763923in}{1.843148in}}{\pgfqpoint{8.763923in}{1.854199in}}%
\pgfpathcurveto{\pgfqpoint{8.763923in}{1.865249in}}{\pgfqpoint{8.759532in}{1.875848in}}{\pgfqpoint{8.751719in}{1.883661in}}%
\pgfpathcurveto{\pgfqpoint{8.743905in}{1.891475in}}{\pgfqpoint{8.733306in}{1.895865in}}{\pgfqpoint{8.722256in}{1.895865in}}%
\pgfpathcurveto{\pgfqpoint{8.711206in}{1.895865in}}{\pgfqpoint{8.700607in}{1.891475in}}{\pgfqpoint{8.692793in}{1.883661in}}%
\pgfpathcurveto{\pgfqpoint{8.684979in}{1.875848in}}{\pgfqpoint{8.680589in}{1.865249in}}{\pgfqpoint{8.680589in}{1.854199in}}%
\pgfpathcurveto{\pgfqpoint{8.680589in}{1.843148in}}{\pgfqpoint{8.684979in}{1.832549in}}{\pgfqpoint{8.692793in}{1.824736in}}%
\pgfpathcurveto{\pgfqpoint{8.700607in}{1.816922in}}{\pgfqpoint{8.711206in}{1.812532in}}{\pgfqpoint{8.722256in}{1.812532in}}%
\pgfpathlineto{\pgfqpoint{8.722256in}{1.812532in}}%
\pgfpathclose%
\pgfusepath{stroke}%
\end{pgfscope}%
\begin{pgfscope}%
\pgfpathrectangle{\pgfqpoint{7.512535in}{0.437222in}}{\pgfqpoint{6.275590in}{5.159444in}}%
\pgfusepath{clip}%
\pgfsetbuttcap%
\pgfsetroundjoin%
\pgfsetlinewidth{1.003750pt}%
\definecolor{currentstroke}{rgb}{1.000000,0.000000,0.000000}%
\pgfsetstrokecolor{currentstroke}%
\pgfsetdash{}{0pt}%
\pgfpathmoveto{\pgfqpoint{9.315222in}{2.832048in}}%
\pgfpathcurveto{\pgfqpoint{9.326272in}{2.832048in}}{\pgfqpoint{9.336871in}{2.836439in}}{\pgfqpoint{9.344684in}{2.844252in}}%
\pgfpathcurveto{\pgfqpoint{9.352498in}{2.852066in}}{\pgfqpoint{9.356888in}{2.862665in}}{\pgfqpoint{9.356888in}{2.873715in}}%
\pgfpathcurveto{\pgfqpoint{9.356888in}{2.884765in}}{\pgfqpoint{9.352498in}{2.895364in}}{\pgfqpoint{9.344684in}{2.903178in}}%
\pgfpathcurveto{\pgfqpoint{9.336871in}{2.910991in}}{\pgfqpoint{9.326272in}{2.915382in}}{\pgfqpoint{9.315222in}{2.915382in}}%
\pgfpathcurveto{\pgfqpoint{9.304171in}{2.915382in}}{\pgfqpoint{9.293572in}{2.910991in}}{\pgfqpoint{9.285759in}{2.903178in}}%
\pgfpathcurveto{\pgfqpoint{9.277945in}{2.895364in}}{\pgfqpoint{9.273555in}{2.884765in}}{\pgfqpoint{9.273555in}{2.873715in}}%
\pgfpathcurveto{\pgfqpoint{9.273555in}{2.862665in}}{\pgfqpoint{9.277945in}{2.852066in}}{\pgfqpoint{9.285759in}{2.844252in}}%
\pgfpathcurveto{\pgfqpoint{9.293572in}{2.836439in}}{\pgfqpoint{9.304171in}{2.832048in}}{\pgfqpoint{9.315222in}{2.832048in}}%
\pgfpathlineto{\pgfqpoint{9.315222in}{2.832048in}}%
\pgfpathclose%
\pgfusepath{stroke}%
\end{pgfscope}%
\begin{pgfscope}%
\pgfpathrectangle{\pgfqpoint{7.512535in}{0.437222in}}{\pgfqpoint{6.275590in}{5.159444in}}%
\pgfusepath{clip}%
\pgfsetbuttcap%
\pgfsetroundjoin%
\pgfsetlinewidth{1.003750pt}%
\definecolor{currentstroke}{rgb}{1.000000,0.000000,0.000000}%
\pgfsetstrokecolor{currentstroke}%
\pgfsetdash{}{0pt}%
\pgfpathmoveto{\pgfqpoint{9.691305in}{3.336201in}}%
\pgfpathcurveto{\pgfqpoint{9.702355in}{3.336201in}}{\pgfqpoint{9.712954in}{3.340591in}}{\pgfqpoint{9.720767in}{3.348404in}}%
\pgfpathcurveto{\pgfqpoint{9.728581in}{3.356218in}}{\pgfqpoint{9.732971in}{3.366817in}}{\pgfqpoint{9.732971in}{3.377867in}}%
\pgfpathcurveto{\pgfqpoint{9.732971in}{3.388917in}}{\pgfqpoint{9.728581in}{3.399516in}}{\pgfqpoint{9.720767in}{3.407330in}}%
\pgfpathcurveto{\pgfqpoint{9.712954in}{3.415144in}}{\pgfqpoint{9.702355in}{3.419534in}}{\pgfqpoint{9.691305in}{3.419534in}}%
\pgfpathcurveto{\pgfqpoint{9.680255in}{3.419534in}}{\pgfqpoint{9.669656in}{3.415144in}}{\pgfqpoint{9.661842in}{3.407330in}}%
\pgfpathcurveto{\pgfqpoint{9.654028in}{3.399516in}}{\pgfqpoint{9.649638in}{3.388917in}}{\pgfqpoint{9.649638in}{3.377867in}}%
\pgfpathcurveto{\pgfqpoint{9.649638in}{3.366817in}}{\pgfqpoint{9.654028in}{3.356218in}}{\pgfqpoint{9.661842in}{3.348404in}}%
\pgfpathcurveto{\pgfqpoint{9.669656in}{3.340591in}}{\pgfqpoint{9.680255in}{3.336201in}}{\pgfqpoint{9.691305in}{3.336201in}}%
\pgfpathlineto{\pgfqpoint{9.691305in}{3.336201in}}%
\pgfpathclose%
\pgfusepath{stroke}%
\end{pgfscope}%
\begin{pgfscope}%
\pgfpathrectangle{\pgfqpoint{7.512535in}{0.437222in}}{\pgfqpoint{6.275590in}{5.159444in}}%
\pgfusepath{clip}%
\pgfsetbuttcap%
\pgfsetroundjoin%
\pgfsetlinewidth{1.003750pt}%
\definecolor{currentstroke}{rgb}{1.000000,0.000000,0.000000}%
\pgfsetstrokecolor{currentstroke}%
\pgfsetdash{}{0pt}%
\pgfpathmoveto{\pgfqpoint{8.509088in}{1.635953in}}%
\pgfpathcurveto{\pgfqpoint{8.520138in}{1.635953in}}{\pgfqpoint{8.530737in}{1.640343in}}{\pgfqpoint{8.538551in}{1.648157in}}%
\pgfpathcurveto{\pgfqpoint{8.546365in}{1.655971in}}{\pgfqpoint{8.550755in}{1.666570in}}{\pgfqpoint{8.550755in}{1.677620in}}%
\pgfpathcurveto{\pgfqpoint{8.550755in}{1.688670in}}{\pgfqpoint{8.546365in}{1.699269in}}{\pgfqpoint{8.538551in}{1.707082in}}%
\pgfpathcurveto{\pgfqpoint{8.530737in}{1.714896in}}{\pgfqpoint{8.520138in}{1.719286in}}{\pgfqpoint{8.509088in}{1.719286in}}%
\pgfpathcurveto{\pgfqpoint{8.498038in}{1.719286in}}{\pgfqpoint{8.487439in}{1.714896in}}{\pgfqpoint{8.479625in}{1.707082in}}%
\pgfpathcurveto{\pgfqpoint{8.471812in}{1.699269in}}{\pgfqpoint{8.467422in}{1.688670in}}{\pgfqpoint{8.467422in}{1.677620in}}%
\pgfpathcurveto{\pgfqpoint{8.467422in}{1.666570in}}{\pgfqpoint{8.471812in}{1.655971in}}{\pgfqpoint{8.479625in}{1.648157in}}%
\pgfpathcurveto{\pgfqpoint{8.487439in}{1.640343in}}{\pgfqpoint{8.498038in}{1.635953in}}{\pgfqpoint{8.509088in}{1.635953in}}%
\pgfpathlineto{\pgfqpoint{8.509088in}{1.635953in}}%
\pgfpathclose%
\pgfusepath{stroke}%
\end{pgfscope}%
\begin{pgfscope}%
\pgfpathrectangle{\pgfqpoint{7.512535in}{0.437222in}}{\pgfqpoint{6.275590in}{5.159444in}}%
\pgfusepath{clip}%
\pgfsetbuttcap%
\pgfsetroundjoin%
\pgfsetlinewidth{1.003750pt}%
\definecolor{currentstroke}{rgb}{1.000000,0.000000,0.000000}%
\pgfsetstrokecolor{currentstroke}%
\pgfsetdash{}{0pt}%
\pgfpathmoveto{\pgfqpoint{11.298179in}{5.507098in}}%
\pgfpathcurveto{\pgfqpoint{11.309229in}{5.507098in}}{\pgfqpoint{11.319828in}{5.511489in}}{\pgfqpoint{11.327642in}{5.519302in}}%
\pgfpathcurveto{\pgfqpoint{11.335455in}{5.527116in}}{\pgfqpoint{11.339846in}{5.537715in}}{\pgfqpoint{11.339846in}{5.548765in}}%
\pgfpathcurveto{\pgfqpoint{11.339846in}{5.559815in}}{\pgfqpoint{11.335455in}{5.570414in}}{\pgfqpoint{11.327642in}{5.578228in}}%
\pgfpathcurveto{\pgfqpoint{11.319828in}{5.586041in}}{\pgfqpoint{11.309229in}{5.590432in}}{\pgfqpoint{11.298179in}{5.590432in}}%
\pgfpathcurveto{\pgfqpoint{11.287129in}{5.590432in}}{\pgfqpoint{11.276530in}{5.586041in}}{\pgfqpoint{11.268716in}{5.578228in}}%
\pgfpathcurveto{\pgfqpoint{11.260903in}{5.570414in}}{\pgfqpoint{11.256512in}{5.559815in}}{\pgfqpoint{11.256512in}{5.548765in}}%
\pgfpathcurveto{\pgfqpoint{11.256512in}{5.537715in}}{\pgfqpoint{11.260903in}{5.527116in}}{\pgfqpoint{11.268716in}{5.519302in}}%
\pgfpathcurveto{\pgfqpoint{11.276530in}{5.511489in}}{\pgfqpoint{11.287129in}{5.507098in}}{\pgfqpoint{11.298179in}{5.507098in}}%
\pgfpathlineto{\pgfqpoint{11.298179in}{5.507098in}}%
\pgfpathclose%
\pgfusepath{stroke}%
\end{pgfscope}%
\begin{pgfscope}%
\pgfpathrectangle{\pgfqpoint{7.512535in}{0.437222in}}{\pgfqpoint{6.275590in}{5.159444in}}%
\pgfusepath{clip}%
\pgfsetbuttcap%
\pgfsetroundjoin%
\pgfsetlinewidth{1.003750pt}%
\definecolor{currentstroke}{rgb}{1.000000,0.000000,0.000000}%
\pgfsetstrokecolor{currentstroke}%
\pgfsetdash{}{0pt}%
\pgfpathmoveto{\pgfqpoint{11.855723in}{5.554997in}}%
\pgfpathcurveto{\pgfqpoint{11.866773in}{5.554997in}}{\pgfqpoint{11.877372in}{5.559387in}}{\pgfqpoint{11.885186in}{5.567201in}}%
\pgfpathcurveto{\pgfqpoint{11.893000in}{5.575014in}}{\pgfqpoint{11.897390in}{5.585613in}}{\pgfqpoint{11.897390in}{5.596663in}}%
\pgfpathcurveto{\pgfqpoint{11.897390in}{5.607713in}}{\pgfqpoint{11.893000in}{5.618312in}}{\pgfqpoint{11.885186in}{5.626126in}}%
\pgfpathcurveto{\pgfqpoint{11.877372in}{5.633940in}}{\pgfqpoint{11.866773in}{5.638330in}}{\pgfqpoint{11.855723in}{5.638330in}}%
\pgfpathcurveto{\pgfqpoint{11.844673in}{5.638330in}}{\pgfqpoint{11.834074in}{5.633940in}}{\pgfqpoint{11.826261in}{5.626126in}}%
\pgfpathcurveto{\pgfqpoint{11.818447in}{5.618312in}}{\pgfqpoint{11.814057in}{5.607713in}}{\pgfqpoint{11.814057in}{5.596663in}}%
\pgfpathcurveto{\pgfqpoint{11.814057in}{5.585613in}}{\pgfqpoint{11.818447in}{5.575014in}}{\pgfqpoint{11.826261in}{5.567201in}}%
\pgfpathcurveto{\pgfqpoint{11.834074in}{5.559387in}}{\pgfqpoint{11.844673in}{5.554997in}}{\pgfqpoint{11.855723in}{5.554997in}}%
\pgfpathlineto{\pgfqpoint{11.855723in}{5.554997in}}%
\pgfpathclose%
\pgfusepath{stroke}%
\end{pgfscope}%
\begin{pgfscope}%
\pgfpathrectangle{\pgfqpoint{7.512535in}{0.437222in}}{\pgfqpoint{6.275590in}{5.159444in}}%
\pgfusepath{clip}%
\pgfsetbuttcap%
\pgfsetroundjoin%
\pgfsetlinewidth{1.003750pt}%
\definecolor{currentstroke}{rgb}{1.000000,0.000000,0.000000}%
\pgfsetstrokecolor{currentstroke}%
\pgfsetdash{}{0pt}%
\pgfpathmoveto{\pgfqpoint{8.722256in}{1.906063in}}%
\pgfpathcurveto{\pgfqpoint{8.733306in}{1.906063in}}{\pgfqpoint{8.743905in}{1.910453in}}{\pgfqpoint{8.751719in}{1.918267in}}%
\pgfpathcurveto{\pgfqpoint{8.759532in}{1.926080in}}{\pgfqpoint{8.763923in}{1.936679in}}{\pgfqpoint{8.763923in}{1.947729in}}%
\pgfpathcurveto{\pgfqpoint{8.763923in}{1.958780in}}{\pgfqpoint{8.759532in}{1.969379in}}{\pgfqpoint{8.751719in}{1.977192in}}%
\pgfpathcurveto{\pgfqpoint{8.743905in}{1.985006in}}{\pgfqpoint{8.733306in}{1.989396in}}{\pgfqpoint{8.722256in}{1.989396in}}%
\pgfpathcurveto{\pgfqpoint{8.711206in}{1.989396in}}{\pgfqpoint{8.700607in}{1.985006in}}{\pgfqpoint{8.692793in}{1.977192in}}%
\pgfpathcurveto{\pgfqpoint{8.684979in}{1.969379in}}{\pgfqpoint{8.680589in}{1.958780in}}{\pgfqpoint{8.680589in}{1.947729in}}%
\pgfpathcurveto{\pgfqpoint{8.680589in}{1.936679in}}{\pgfqpoint{8.684979in}{1.926080in}}{\pgfqpoint{8.692793in}{1.918267in}}%
\pgfpathcurveto{\pgfqpoint{8.700607in}{1.910453in}}{\pgfqpoint{8.711206in}{1.906063in}}{\pgfqpoint{8.722256in}{1.906063in}}%
\pgfpathlineto{\pgfqpoint{8.722256in}{1.906063in}}%
\pgfpathclose%
\pgfusepath{stroke}%
\end{pgfscope}%
\begin{pgfscope}%
\pgfpathrectangle{\pgfqpoint{7.512535in}{0.437222in}}{\pgfqpoint{6.275590in}{5.159444in}}%
\pgfusepath{clip}%
\pgfsetbuttcap%
\pgfsetroundjoin%
\pgfsetlinewidth{1.003750pt}%
\definecolor{currentstroke}{rgb}{1.000000,0.000000,0.000000}%
\pgfsetstrokecolor{currentstroke}%
\pgfsetdash{}{0pt}%
\pgfpathmoveto{\pgfqpoint{7.547076in}{0.395556in}}%
\pgfpathcurveto{\pgfqpoint{7.558126in}{0.395556in}}{\pgfqpoint{7.568725in}{0.399946in}}{\pgfqpoint{7.576539in}{0.407759in}}%
\pgfpathcurveto{\pgfqpoint{7.584352in}{0.415573in}}{\pgfqpoint{7.588743in}{0.426172in}}{\pgfqpoint{7.588743in}{0.437222in}}%
\pgfpathcurveto{\pgfqpoint{7.588743in}{0.448272in}}{\pgfqpoint{7.584352in}{0.458871in}}{\pgfqpoint{7.576539in}{0.466685in}}%
\pgfpathcurveto{\pgfqpoint{7.568725in}{0.474499in}}{\pgfqpoint{7.558126in}{0.478889in}}{\pgfqpoint{7.547076in}{0.478889in}}%
\pgfpathcurveto{\pgfqpoint{7.536026in}{0.478889in}}{\pgfqpoint{7.525427in}{0.474499in}}{\pgfqpoint{7.517613in}{0.466685in}}%
\pgfpathcurveto{\pgfqpoint{7.509800in}{0.458871in}}{\pgfqpoint{7.505409in}{0.448272in}}{\pgfqpoint{7.505409in}{0.437222in}}%
\pgfpathcurveto{\pgfqpoint{7.505409in}{0.426172in}}{\pgfqpoint{7.509800in}{0.415573in}}{\pgfqpoint{7.517613in}{0.407759in}}%
\pgfpathcurveto{\pgfqpoint{7.525427in}{0.399946in}}{\pgfqpoint{7.536026in}{0.395556in}}{\pgfqpoint{7.547076in}{0.395556in}}%
\pgfusepath{stroke}%
\end{pgfscope}%
\begin{pgfscope}%
\pgfpathrectangle{\pgfqpoint{7.512535in}{0.437222in}}{\pgfqpoint{6.275590in}{5.159444in}}%
\pgfusepath{clip}%
\pgfsetbuttcap%
\pgfsetroundjoin%
\pgfsetlinewidth{1.003750pt}%
\definecolor{currentstroke}{rgb}{1.000000,0.000000,0.000000}%
\pgfsetstrokecolor{currentstroke}%
\pgfsetdash{}{0pt}%
\pgfpathmoveto{\pgfqpoint{10.140421in}{3.562097in}}%
\pgfpathcurveto{\pgfqpoint{10.151471in}{3.562097in}}{\pgfqpoint{10.162070in}{3.566487in}}{\pgfqpoint{10.169884in}{3.574301in}}%
\pgfpathcurveto{\pgfqpoint{10.177697in}{3.582115in}}{\pgfqpoint{10.182088in}{3.592714in}}{\pgfqpoint{10.182088in}{3.603764in}}%
\pgfpathcurveto{\pgfqpoint{10.182088in}{3.614814in}}{\pgfqpoint{10.177697in}{3.625413in}}{\pgfqpoint{10.169884in}{3.633227in}}%
\pgfpathcurveto{\pgfqpoint{10.162070in}{3.641040in}}{\pgfqpoint{10.151471in}{3.645430in}}{\pgfqpoint{10.140421in}{3.645430in}}%
\pgfpathcurveto{\pgfqpoint{10.129371in}{3.645430in}}{\pgfqpoint{10.118772in}{3.641040in}}{\pgfqpoint{10.110958in}{3.633227in}}%
\pgfpathcurveto{\pgfqpoint{10.103145in}{3.625413in}}{\pgfqpoint{10.098754in}{3.614814in}}{\pgfqpoint{10.098754in}{3.603764in}}%
\pgfpathcurveto{\pgfqpoint{10.098754in}{3.592714in}}{\pgfqpoint{10.103145in}{3.582115in}}{\pgfqpoint{10.110958in}{3.574301in}}%
\pgfpathcurveto{\pgfqpoint{10.118772in}{3.566487in}}{\pgfqpoint{10.129371in}{3.562097in}}{\pgfqpoint{10.140421in}{3.562097in}}%
\pgfpathlineto{\pgfqpoint{10.140421in}{3.562097in}}%
\pgfpathclose%
\pgfusepath{stroke}%
\end{pgfscope}%
\begin{pgfscope}%
\pgfpathrectangle{\pgfqpoint{7.512535in}{0.437222in}}{\pgfqpoint{6.275590in}{5.159444in}}%
\pgfusepath{clip}%
\pgfsetbuttcap%
\pgfsetroundjoin%
\pgfsetlinewidth{1.003750pt}%
\definecolor{currentstroke}{rgb}{1.000000,0.000000,0.000000}%
\pgfsetstrokecolor{currentstroke}%
\pgfsetdash{}{0pt}%
\pgfpathmoveto{\pgfqpoint{8.242963in}{1.013007in}}%
\pgfpathcurveto{\pgfqpoint{8.254013in}{1.013007in}}{\pgfqpoint{8.264612in}{1.017397in}}{\pgfqpoint{8.272425in}{1.025211in}}%
\pgfpathcurveto{\pgfqpoint{8.280239in}{1.033025in}}{\pgfqpoint{8.284629in}{1.043624in}}{\pgfqpoint{8.284629in}{1.054674in}}%
\pgfpathcurveto{\pgfqpoint{8.284629in}{1.065724in}}{\pgfqpoint{8.280239in}{1.076323in}}{\pgfqpoint{8.272425in}{1.084137in}}%
\pgfpathcurveto{\pgfqpoint{8.264612in}{1.091950in}}{\pgfqpoint{8.254013in}{1.096340in}}{\pgfqpoint{8.242963in}{1.096340in}}%
\pgfpathcurveto{\pgfqpoint{8.231912in}{1.096340in}}{\pgfqpoint{8.221313in}{1.091950in}}{\pgfqpoint{8.213500in}{1.084137in}}%
\pgfpathcurveto{\pgfqpoint{8.205686in}{1.076323in}}{\pgfqpoint{8.201296in}{1.065724in}}{\pgfqpoint{8.201296in}{1.054674in}}%
\pgfpathcurveto{\pgfqpoint{8.201296in}{1.043624in}}{\pgfqpoint{8.205686in}{1.033025in}}{\pgfqpoint{8.213500in}{1.025211in}}%
\pgfpathcurveto{\pgfqpoint{8.221313in}{1.017397in}}{\pgfqpoint{8.231912in}{1.013007in}}{\pgfqpoint{8.242963in}{1.013007in}}%
\pgfpathlineto{\pgfqpoint{8.242963in}{1.013007in}}%
\pgfpathclose%
\pgfusepath{stroke}%
\end{pgfscope}%
\begin{pgfscope}%
\pgfpathrectangle{\pgfqpoint{7.512535in}{0.437222in}}{\pgfqpoint{6.275590in}{5.159444in}}%
\pgfusepath{clip}%
\pgfsetbuttcap%
\pgfsetroundjoin%
\pgfsetlinewidth{1.003750pt}%
\definecolor{currentstroke}{rgb}{1.000000,0.000000,0.000000}%
\pgfsetstrokecolor{currentstroke}%
\pgfsetdash{}{0pt}%
\pgfpathmoveto{\pgfqpoint{9.417396in}{2.983331in}}%
\pgfpathcurveto{\pgfqpoint{9.428446in}{2.983331in}}{\pgfqpoint{9.439045in}{2.987722in}}{\pgfqpoint{9.446858in}{2.995535in}}%
\pgfpathcurveto{\pgfqpoint{9.454672in}{3.003349in}}{\pgfqpoint{9.459062in}{3.013948in}}{\pgfqpoint{9.459062in}{3.024998in}}%
\pgfpathcurveto{\pgfqpoint{9.459062in}{3.036048in}}{\pgfqpoint{9.454672in}{3.046647in}}{\pgfqpoint{9.446858in}{3.054461in}}%
\pgfpathcurveto{\pgfqpoint{9.439045in}{3.062274in}}{\pgfqpoint{9.428446in}{3.066665in}}{\pgfqpoint{9.417396in}{3.066665in}}%
\pgfpathcurveto{\pgfqpoint{9.406346in}{3.066665in}}{\pgfqpoint{9.395747in}{3.062274in}}{\pgfqpoint{9.387933in}{3.054461in}}%
\pgfpathcurveto{\pgfqpoint{9.380119in}{3.046647in}}{\pgfqpoint{9.375729in}{3.036048in}}{\pgfqpoint{9.375729in}{3.024998in}}%
\pgfpathcurveto{\pgfqpoint{9.375729in}{3.013948in}}{\pgfqpoint{9.380119in}{3.003349in}}{\pgfqpoint{9.387933in}{2.995535in}}%
\pgfpathcurveto{\pgfqpoint{9.395747in}{2.987722in}}{\pgfqpoint{9.406346in}{2.983331in}}{\pgfqpoint{9.417396in}{2.983331in}}%
\pgfpathlineto{\pgfqpoint{9.417396in}{2.983331in}}%
\pgfpathclose%
\pgfusepath{stroke}%
\end{pgfscope}%
\begin{pgfscope}%
\pgfpathrectangle{\pgfqpoint{7.512535in}{0.437222in}}{\pgfqpoint{6.275590in}{5.159444in}}%
\pgfusepath{clip}%
\pgfsetbuttcap%
\pgfsetroundjoin%
\pgfsetlinewidth{1.003750pt}%
\definecolor{currentstroke}{rgb}{1.000000,0.000000,0.000000}%
\pgfsetstrokecolor{currentstroke}%
\pgfsetdash{}{0pt}%
\pgfpathmoveto{\pgfqpoint{9.823345in}{2.992339in}}%
\pgfpathcurveto{\pgfqpoint{9.834395in}{2.992339in}}{\pgfqpoint{9.844994in}{2.996729in}}{\pgfqpoint{9.852807in}{3.004543in}}%
\pgfpathcurveto{\pgfqpoint{9.860621in}{3.012357in}}{\pgfqpoint{9.865011in}{3.022956in}}{\pgfqpoint{9.865011in}{3.034006in}}%
\pgfpathcurveto{\pgfqpoint{9.865011in}{3.045056in}}{\pgfqpoint{9.860621in}{3.055655in}}{\pgfqpoint{9.852807in}{3.063468in}}%
\pgfpathcurveto{\pgfqpoint{9.844994in}{3.071282in}}{\pgfqpoint{9.834395in}{3.075672in}}{\pgfqpoint{9.823345in}{3.075672in}}%
\pgfpathcurveto{\pgfqpoint{9.812295in}{3.075672in}}{\pgfqpoint{9.801696in}{3.071282in}}{\pgfqpoint{9.793882in}{3.063468in}}%
\pgfpathcurveto{\pgfqpoint{9.786068in}{3.055655in}}{\pgfqpoint{9.781678in}{3.045056in}}{\pgfqpoint{9.781678in}{3.034006in}}%
\pgfpathcurveto{\pgfqpoint{9.781678in}{3.022956in}}{\pgfqpoint{9.786068in}{3.012357in}}{\pgfqpoint{9.793882in}{3.004543in}}%
\pgfpathcurveto{\pgfqpoint{9.801696in}{2.996729in}}{\pgfqpoint{9.812295in}{2.992339in}}{\pgfqpoint{9.823345in}{2.992339in}}%
\pgfpathlineto{\pgfqpoint{9.823345in}{2.992339in}}%
\pgfpathclose%
\pgfusepath{stroke}%
\end{pgfscope}%
\begin{pgfscope}%
\pgfpathrectangle{\pgfqpoint{7.512535in}{0.437222in}}{\pgfqpoint{6.275590in}{5.159444in}}%
\pgfusepath{clip}%
\pgfsetbuttcap%
\pgfsetroundjoin%
\pgfsetlinewidth{1.003750pt}%
\definecolor{currentstroke}{rgb}{1.000000,0.000000,0.000000}%
\pgfsetstrokecolor{currentstroke}%
\pgfsetdash{}{0pt}%
\pgfpathmoveto{\pgfqpoint{9.146543in}{3.200646in}}%
\pgfpathcurveto{\pgfqpoint{9.157593in}{3.200646in}}{\pgfqpoint{9.168192in}{3.205037in}}{\pgfqpoint{9.176006in}{3.212850in}}%
\pgfpathcurveto{\pgfqpoint{9.183819in}{3.220664in}}{\pgfqpoint{9.188210in}{3.231263in}}{\pgfqpoint{9.188210in}{3.242313in}}%
\pgfpathcurveto{\pgfqpoint{9.188210in}{3.253363in}}{\pgfqpoint{9.183819in}{3.263962in}}{\pgfqpoint{9.176006in}{3.271776in}}%
\pgfpathcurveto{\pgfqpoint{9.168192in}{3.279589in}}{\pgfqpoint{9.157593in}{3.283980in}}{\pgfqpoint{9.146543in}{3.283980in}}%
\pgfpathcurveto{\pgfqpoint{9.135493in}{3.283980in}}{\pgfqpoint{9.124894in}{3.279589in}}{\pgfqpoint{9.117080in}{3.271776in}}%
\pgfpathcurveto{\pgfqpoint{9.109267in}{3.263962in}}{\pgfqpoint{9.104876in}{3.253363in}}{\pgfqpoint{9.104876in}{3.242313in}}%
\pgfpathcurveto{\pgfqpoint{9.104876in}{3.231263in}}{\pgfqpoint{9.109267in}{3.220664in}}{\pgfqpoint{9.117080in}{3.212850in}}%
\pgfpathcurveto{\pgfqpoint{9.124894in}{3.205037in}}{\pgfqpoint{9.135493in}{3.200646in}}{\pgfqpoint{9.146543in}{3.200646in}}%
\pgfpathlineto{\pgfqpoint{9.146543in}{3.200646in}}%
\pgfpathclose%
\pgfusepath{stroke}%
\end{pgfscope}%
\begin{pgfscope}%
\pgfpathrectangle{\pgfqpoint{7.512535in}{0.437222in}}{\pgfqpoint{6.275590in}{5.159444in}}%
\pgfusepath{clip}%
\pgfsetbuttcap%
\pgfsetroundjoin%
\pgfsetlinewidth{1.003750pt}%
\definecolor{currentstroke}{rgb}{1.000000,0.000000,0.000000}%
\pgfsetstrokecolor{currentstroke}%
\pgfsetdash{}{0pt}%
\pgfpathmoveto{\pgfqpoint{9.725110in}{2.966974in}}%
\pgfpathcurveto{\pgfqpoint{9.736160in}{2.966974in}}{\pgfqpoint{9.746759in}{2.971364in}}{\pgfqpoint{9.754573in}{2.979177in}}%
\pgfpathcurveto{\pgfqpoint{9.762386in}{2.986991in}}{\pgfqpoint{9.766777in}{2.997590in}}{\pgfqpoint{9.766777in}{3.008640in}}%
\pgfpathcurveto{\pgfqpoint{9.766777in}{3.019690in}}{\pgfqpoint{9.762386in}{3.030289in}}{\pgfqpoint{9.754573in}{3.038103in}}%
\pgfpathcurveto{\pgfqpoint{9.746759in}{3.045917in}}{\pgfqpoint{9.736160in}{3.050307in}}{\pgfqpoint{9.725110in}{3.050307in}}%
\pgfpathcurveto{\pgfqpoint{9.714060in}{3.050307in}}{\pgfqpoint{9.703461in}{3.045917in}}{\pgfqpoint{9.695647in}{3.038103in}}%
\pgfpathcurveto{\pgfqpoint{9.687834in}{3.030289in}}{\pgfqpoint{9.683443in}{3.019690in}}{\pgfqpoint{9.683443in}{3.008640in}}%
\pgfpathcurveto{\pgfqpoint{9.683443in}{2.997590in}}{\pgfqpoint{9.687834in}{2.986991in}}{\pgfqpoint{9.695647in}{2.979177in}}%
\pgfpathcurveto{\pgfqpoint{9.703461in}{2.971364in}}{\pgfqpoint{9.714060in}{2.966974in}}{\pgfqpoint{9.725110in}{2.966974in}}%
\pgfpathlineto{\pgfqpoint{9.725110in}{2.966974in}}%
\pgfpathclose%
\pgfusepath{stroke}%
\end{pgfscope}%
\begin{pgfscope}%
\pgfpathrectangle{\pgfqpoint{7.512535in}{0.437222in}}{\pgfqpoint{6.275590in}{5.159444in}}%
\pgfusepath{clip}%
\pgfsetbuttcap%
\pgfsetroundjoin%
\pgfsetlinewidth{1.003750pt}%
\definecolor{currentstroke}{rgb}{1.000000,0.000000,0.000000}%
\pgfsetstrokecolor{currentstroke}%
\pgfsetdash{}{0pt}%
\pgfpathmoveto{\pgfqpoint{9.883474in}{4.311381in}}%
\pgfpathcurveto{\pgfqpoint{9.894524in}{4.311381in}}{\pgfqpoint{9.905124in}{4.315771in}}{\pgfqpoint{9.912937in}{4.323584in}}%
\pgfpathcurveto{\pgfqpoint{9.920751in}{4.331398in}}{\pgfqpoint{9.925141in}{4.341997in}}{\pgfqpoint{9.925141in}{4.353047in}}%
\pgfpathcurveto{\pgfqpoint{9.925141in}{4.364097in}}{\pgfqpoint{9.920751in}{4.374696in}}{\pgfqpoint{9.912937in}{4.382510in}}%
\pgfpathcurveto{\pgfqpoint{9.905124in}{4.390324in}}{\pgfqpoint{9.894524in}{4.394714in}}{\pgfqpoint{9.883474in}{4.394714in}}%
\pgfpathcurveto{\pgfqpoint{9.872424in}{4.394714in}}{\pgfqpoint{9.861825in}{4.390324in}}{\pgfqpoint{9.854012in}{4.382510in}}%
\pgfpathcurveto{\pgfqpoint{9.846198in}{4.374696in}}{\pgfqpoint{9.841808in}{4.364097in}}{\pgfqpoint{9.841808in}{4.353047in}}%
\pgfpathcurveto{\pgfqpoint{9.841808in}{4.341997in}}{\pgfqpoint{9.846198in}{4.331398in}}{\pgfqpoint{9.854012in}{4.323584in}}%
\pgfpathcurveto{\pgfqpoint{9.861825in}{4.315771in}}{\pgfqpoint{9.872424in}{4.311381in}}{\pgfqpoint{9.883474in}{4.311381in}}%
\pgfpathlineto{\pgfqpoint{9.883474in}{4.311381in}}%
\pgfpathclose%
\pgfusepath{stroke}%
\end{pgfscope}%
\begin{pgfscope}%
\pgfpathrectangle{\pgfqpoint{7.512535in}{0.437222in}}{\pgfqpoint{6.275590in}{5.159444in}}%
\pgfusepath{clip}%
\pgfsetbuttcap%
\pgfsetroundjoin%
\pgfsetlinewidth{1.003750pt}%
\definecolor{currentstroke}{rgb}{1.000000,0.000000,0.000000}%
\pgfsetstrokecolor{currentstroke}%
\pgfsetdash{}{0pt}%
\pgfpathmoveto{\pgfqpoint{8.617176in}{1.573408in}}%
\pgfpathcurveto{\pgfqpoint{8.628226in}{1.573408in}}{\pgfqpoint{8.638825in}{1.577798in}}{\pgfqpoint{8.646639in}{1.585612in}}%
\pgfpathcurveto{\pgfqpoint{8.654452in}{1.593426in}}{\pgfqpoint{8.658842in}{1.604025in}}{\pgfqpoint{8.658842in}{1.615075in}}%
\pgfpathcurveto{\pgfqpoint{8.658842in}{1.626125in}}{\pgfqpoint{8.654452in}{1.636724in}}{\pgfqpoint{8.646639in}{1.644538in}}%
\pgfpathcurveto{\pgfqpoint{8.638825in}{1.652351in}}{\pgfqpoint{8.628226in}{1.656742in}}{\pgfqpoint{8.617176in}{1.656742in}}%
\pgfpathcurveto{\pgfqpoint{8.606126in}{1.656742in}}{\pgfqpoint{8.595527in}{1.652351in}}{\pgfqpoint{8.587713in}{1.644538in}}%
\pgfpathcurveto{\pgfqpoint{8.579899in}{1.636724in}}{\pgfqpoint{8.575509in}{1.626125in}}{\pgfqpoint{8.575509in}{1.615075in}}%
\pgfpathcurveto{\pgfqpoint{8.575509in}{1.604025in}}{\pgfqpoint{8.579899in}{1.593426in}}{\pgfqpoint{8.587713in}{1.585612in}}%
\pgfpathcurveto{\pgfqpoint{8.595527in}{1.577798in}}{\pgfqpoint{8.606126in}{1.573408in}}{\pgfqpoint{8.617176in}{1.573408in}}%
\pgfpathlineto{\pgfqpoint{8.617176in}{1.573408in}}%
\pgfpathclose%
\pgfusepath{stroke}%
\end{pgfscope}%
\begin{pgfscope}%
\pgfpathrectangle{\pgfqpoint{7.512535in}{0.437222in}}{\pgfqpoint{6.275590in}{5.159444in}}%
\pgfusepath{clip}%
\pgfsetbuttcap%
\pgfsetroundjoin%
\pgfsetlinewidth{1.003750pt}%
\definecolor{currentstroke}{rgb}{1.000000,0.000000,0.000000}%
\pgfsetstrokecolor{currentstroke}%
\pgfsetdash{}{0pt}%
\pgfpathmoveto{\pgfqpoint{7.980538in}{1.182069in}}%
\pgfpathcurveto{\pgfqpoint{7.991588in}{1.182069in}}{\pgfqpoint{8.002187in}{1.186459in}}{\pgfqpoint{8.010001in}{1.194273in}}%
\pgfpathcurveto{\pgfqpoint{8.017814in}{1.202086in}}{\pgfqpoint{8.022205in}{1.212685in}}{\pgfqpoint{8.022205in}{1.223735in}}%
\pgfpathcurveto{\pgfqpoint{8.022205in}{1.234786in}}{\pgfqpoint{8.017814in}{1.245385in}}{\pgfqpoint{8.010001in}{1.253198in}}%
\pgfpathcurveto{\pgfqpoint{8.002187in}{1.261012in}}{\pgfqpoint{7.991588in}{1.265402in}}{\pgfqpoint{7.980538in}{1.265402in}}%
\pgfpathcurveto{\pgfqpoint{7.969488in}{1.265402in}}{\pgfqpoint{7.958889in}{1.261012in}}{\pgfqpoint{7.951075in}{1.253198in}}%
\pgfpathcurveto{\pgfqpoint{7.943262in}{1.245385in}}{\pgfqpoint{7.938871in}{1.234786in}}{\pgfqpoint{7.938871in}{1.223735in}}%
\pgfpathcurveto{\pgfqpoint{7.938871in}{1.212685in}}{\pgfqpoint{7.943262in}{1.202086in}}{\pgfqpoint{7.951075in}{1.194273in}}%
\pgfpathcurveto{\pgfqpoint{7.958889in}{1.186459in}}{\pgfqpoint{7.969488in}{1.182069in}}{\pgfqpoint{7.980538in}{1.182069in}}%
\pgfpathlineto{\pgfqpoint{7.980538in}{1.182069in}}%
\pgfpathclose%
\pgfusepath{stroke}%
\end{pgfscope}%
\begin{pgfscope}%
\pgfpathrectangle{\pgfqpoint{7.512535in}{0.437222in}}{\pgfqpoint{6.275590in}{5.159444in}}%
\pgfusepath{clip}%
\pgfsetbuttcap%
\pgfsetroundjoin%
\pgfsetlinewidth{1.003750pt}%
\definecolor{currentstroke}{rgb}{1.000000,0.000000,0.000000}%
\pgfsetstrokecolor{currentstroke}%
\pgfsetdash{}{0pt}%
\pgfpathmoveto{\pgfqpoint{10.258735in}{3.543884in}}%
\pgfpathcurveto{\pgfqpoint{10.269785in}{3.543884in}}{\pgfqpoint{10.280384in}{3.548274in}}{\pgfqpoint{10.288198in}{3.556088in}}%
\pgfpathcurveto{\pgfqpoint{10.296011in}{3.563901in}}{\pgfqpoint{10.300402in}{3.574500in}}{\pgfqpoint{10.300402in}{3.585550in}}%
\pgfpathcurveto{\pgfqpoint{10.300402in}{3.596601in}}{\pgfqpoint{10.296011in}{3.607200in}}{\pgfqpoint{10.288198in}{3.615013in}}%
\pgfpathcurveto{\pgfqpoint{10.280384in}{3.622827in}}{\pgfqpoint{10.269785in}{3.627217in}}{\pgfqpoint{10.258735in}{3.627217in}}%
\pgfpathcurveto{\pgfqpoint{10.247685in}{3.627217in}}{\pgfqpoint{10.237086in}{3.622827in}}{\pgfqpoint{10.229272in}{3.615013in}}%
\pgfpathcurveto{\pgfqpoint{10.221459in}{3.607200in}}{\pgfqpoint{10.217068in}{3.596601in}}{\pgfqpoint{10.217068in}{3.585550in}}%
\pgfpathcurveto{\pgfqpoint{10.217068in}{3.574500in}}{\pgfqpoint{10.221459in}{3.563901in}}{\pgfqpoint{10.229272in}{3.556088in}}%
\pgfpathcurveto{\pgfqpoint{10.237086in}{3.548274in}}{\pgfqpoint{10.247685in}{3.543884in}}{\pgfqpoint{10.258735in}{3.543884in}}%
\pgfpathlineto{\pgfqpoint{10.258735in}{3.543884in}}%
\pgfpathclose%
\pgfusepath{stroke}%
\end{pgfscope}%
\begin{pgfscope}%
\pgfpathrectangle{\pgfqpoint{7.512535in}{0.437222in}}{\pgfqpoint{6.275590in}{5.159444in}}%
\pgfusepath{clip}%
\pgfsetbuttcap%
\pgfsetroundjoin%
\pgfsetlinewidth{1.003750pt}%
\definecolor{currentstroke}{rgb}{1.000000,0.000000,0.000000}%
\pgfsetstrokecolor{currentstroke}%
\pgfsetdash{}{0pt}%
\pgfpathmoveto{\pgfqpoint{8.761066in}{2.140268in}}%
\pgfpathcurveto{\pgfqpoint{8.772116in}{2.140268in}}{\pgfqpoint{8.782715in}{2.144658in}}{\pgfqpoint{8.790529in}{2.152472in}}%
\pgfpathcurveto{\pgfqpoint{8.798342in}{2.160285in}}{\pgfqpoint{8.802732in}{2.170884in}}{\pgfqpoint{8.802732in}{2.181934in}}%
\pgfpathcurveto{\pgfqpoint{8.802732in}{2.192985in}}{\pgfqpoint{8.798342in}{2.203584in}}{\pgfqpoint{8.790529in}{2.211397in}}%
\pgfpathcurveto{\pgfqpoint{8.782715in}{2.219211in}}{\pgfqpoint{8.772116in}{2.223601in}}{\pgfqpoint{8.761066in}{2.223601in}}%
\pgfpathcurveto{\pgfqpoint{8.750016in}{2.223601in}}{\pgfqpoint{8.739417in}{2.219211in}}{\pgfqpoint{8.731603in}{2.211397in}}%
\pgfpathcurveto{\pgfqpoint{8.723789in}{2.203584in}}{\pgfqpoint{8.719399in}{2.192985in}}{\pgfqpoint{8.719399in}{2.181934in}}%
\pgfpathcurveto{\pgfqpoint{8.719399in}{2.170884in}}{\pgfqpoint{8.723789in}{2.160285in}}{\pgfqpoint{8.731603in}{2.152472in}}%
\pgfpathcurveto{\pgfqpoint{8.739417in}{2.144658in}}{\pgfqpoint{8.750016in}{2.140268in}}{\pgfqpoint{8.761066in}{2.140268in}}%
\pgfpathlineto{\pgfqpoint{8.761066in}{2.140268in}}%
\pgfpathclose%
\pgfusepath{stroke}%
\end{pgfscope}%
\begin{pgfscope}%
\pgfpathrectangle{\pgfqpoint{7.512535in}{0.437222in}}{\pgfqpoint{6.275590in}{5.159444in}}%
\pgfusepath{clip}%
\pgfsetbuttcap%
\pgfsetroundjoin%
\pgfsetlinewidth{1.003750pt}%
\definecolor{currentstroke}{rgb}{1.000000,0.000000,0.000000}%
\pgfsetstrokecolor{currentstroke}%
\pgfsetdash{}{0pt}%
\pgfpathmoveto{\pgfqpoint{9.114112in}{2.640834in}}%
\pgfpathcurveto{\pgfqpoint{9.125162in}{2.640834in}}{\pgfqpoint{9.135761in}{2.645225in}}{\pgfqpoint{9.143575in}{2.653038in}}%
\pgfpathcurveto{\pgfqpoint{9.151388in}{2.660852in}}{\pgfqpoint{9.155779in}{2.671451in}}{\pgfqpoint{9.155779in}{2.682501in}}%
\pgfpathcurveto{\pgfqpoint{9.155779in}{2.693551in}}{\pgfqpoint{9.151388in}{2.704150in}}{\pgfqpoint{9.143575in}{2.711964in}}%
\pgfpathcurveto{\pgfqpoint{9.135761in}{2.719778in}}{\pgfqpoint{9.125162in}{2.724168in}}{\pgfqpoint{9.114112in}{2.724168in}}%
\pgfpathcurveto{\pgfqpoint{9.103062in}{2.724168in}}{\pgfqpoint{9.092463in}{2.719778in}}{\pgfqpoint{9.084649in}{2.711964in}}%
\pgfpathcurveto{\pgfqpoint{9.076835in}{2.704150in}}{\pgfqpoint{9.072445in}{2.693551in}}{\pgfqpoint{9.072445in}{2.682501in}}%
\pgfpathcurveto{\pgfqpoint{9.072445in}{2.671451in}}{\pgfqpoint{9.076835in}{2.660852in}}{\pgfqpoint{9.084649in}{2.653038in}}%
\pgfpathcurveto{\pgfqpoint{9.092463in}{2.645225in}}{\pgfqpoint{9.103062in}{2.640834in}}{\pgfqpoint{9.114112in}{2.640834in}}%
\pgfpathlineto{\pgfqpoint{9.114112in}{2.640834in}}%
\pgfpathclose%
\pgfusepath{stroke}%
\end{pgfscope}%
\begin{pgfscope}%
\pgfpathrectangle{\pgfqpoint{7.512535in}{0.437222in}}{\pgfqpoint{6.275590in}{5.159444in}}%
\pgfusepath{clip}%
\pgfsetbuttcap%
\pgfsetroundjoin%
\pgfsetlinewidth{1.003750pt}%
\definecolor{currentstroke}{rgb}{1.000000,0.000000,0.000000}%
\pgfsetstrokecolor{currentstroke}%
\pgfsetdash{}{0pt}%
\pgfpathmoveto{\pgfqpoint{11.784624in}{5.390977in}}%
\pgfpathcurveto{\pgfqpoint{11.795674in}{5.390977in}}{\pgfqpoint{11.806273in}{5.395367in}}{\pgfqpoint{11.814087in}{5.403181in}}%
\pgfpathcurveto{\pgfqpoint{11.821901in}{5.410994in}}{\pgfqpoint{11.826291in}{5.421593in}}{\pgfqpoint{11.826291in}{5.432644in}}%
\pgfpathcurveto{\pgfqpoint{11.826291in}{5.443694in}}{\pgfqpoint{11.821901in}{5.454293in}}{\pgfqpoint{11.814087in}{5.462106in}}%
\pgfpathcurveto{\pgfqpoint{11.806273in}{5.469920in}}{\pgfqpoint{11.795674in}{5.474310in}}{\pgfqpoint{11.784624in}{5.474310in}}%
\pgfpathcurveto{\pgfqpoint{11.773574in}{5.474310in}}{\pgfqpoint{11.762975in}{5.469920in}}{\pgfqpoint{11.755161in}{5.462106in}}%
\pgfpathcurveto{\pgfqpoint{11.747348in}{5.454293in}}{\pgfqpoint{11.742957in}{5.443694in}}{\pgfqpoint{11.742957in}{5.432644in}}%
\pgfpathcurveto{\pgfqpoint{11.742957in}{5.421593in}}{\pgfqpoint{11.747348in}{5.410994in}}{\pgfqpoint{11.755161in}{5.403181in}}%
\pgfpathcurveto{\pgfqpoint{11.762975in}{5.395367in}}{\pgfqpoint{11.773574in}{5.390977in}}{\pgfqpoint{11.784624in}{5.390977in}}%
\pgfpathlineto{\pgfqpoint{11.784624in}{5.390977in}}%
\pgfpathclose%
\pgfusepath{stroke}%
\end{pgfscope}%
\begin{pgfscope}%
\pgfpathrectangle{\pgfqpoint{7.512535in}{0.437222in}}{\pgfqpoint{6.275590in}{5.159444in}}%
\pgfusepath{clip}%
\pgfsetbuttcap%
\pgfsetroundjoin%
\pgfsetlinewidth{1.003750pt}%
\definecolor{currentstroke}{rgb}{1.000000,0.000000,0.000000}%
\pgfsetstrokecolor{currentstroke}%
\pgfsetdash{}{0pt}%
\pgfpathmoveto{\pgfqpoint{7.573273in}{0.528252in}}%
\pgfpathcurveto{\pgfqpoint{7.584323in}{0.528252in}}{\pgfqpoint{7.594922in}{0.532642in}}{\pgfqpoint{7.602736in}{0.540456in}}%
\pgfpathcurveto{\pgfqpoint{7.610549in}{0.548269in}}{\pgfqpoint{7.614940in}{0.558869in}}{\pgfqpoint{7.614940in}{0.569919in}}%
\pgfpathcurveto{\pgfqpoint{7.614940in}{0.580969in}}{\pgfqpoint{7.610549in}{0.591568in}}{\pgfqpoint{7.602736in}{0.599381in}}%
\pgfpathcurveto{\pgfqpoint{7.594922in}{0.607195in}}{\pgfqpoint{7.584323in}{0.611585in}}{\pgfqpoint{7.573273in}{0.611585in}}%
\pgfpathcurveto{\pgfqpoint{7.562223in}{0.611585in}}{\pgfqpoint{7.551624in}{0.607195in}}{\pgfqpoint{7.543810in}{0.599381in}}%
\pgfpathcurveto{\pgfqpoint{7.535996in}{0.591568in}}{\pgfqpoint{7.531606in}{0.580969in}}{\pgfqpoint{7.531606in}{0.569919in}}%
\pgfpathcurveto{\pgfqpoint{7.531606in}{0.558869in}}{\pgfqpoint{7.535996in}{0.548269in}}{\pgfqpoint{7.543810in}{0.540456in}}%
\pgfpathcurveto{\pgfqpoint{7.551624in}{0.532642in}}{\pgfqpoint{7.562223in}{0.528252in}}{\pgfqpoint{7.573273in}{0.528252in}}%
\pgfpathlineto{\pgfqpoint{7.573273in}{0.528252in}}%
\pgfpathclose%
\pgfusepath{stroke}%
\end{pgfscope}%
\begin{pgfscope}%
\pgfpathrectangle{\pgfqpoint{7.512535in}{0.437222in}}{\pgfqpoint{6.275590in}{5.159444in}}%
\pgfusepath{clip}%
\pgfsetbuttcap%
\pgfsetroundjoin%
\pgfsetlinewidth{1.003750pt}%
\definecolor{currentstroke}{rgb}{1.000000,0.000000,0.000000}%
\pgfsetstrokecolor{currentstroke}%
\pgfsetdash{}{0pt}%
\pgfpathmoveto{\pgfqpoint{11.584099in}{5.554768in}}%
\pgfpathcurveto{\pgfqpoint{11.595149in}{5.554768in}}{\pgfqpoint{11.605748in}{5.559158in}}{\pgfqpoint{11.613562in}{5.566972in}}%
\pgfpathcurveto{\pgfqpoint{11.621376in}{5.574785in}}{\pgfqpoint{11.625766in}{5.585385in}}{\pgfqpoint{11.625766in}{5.596435in}}%
\pgfpathcurveto{\pgfqpoint{11.625766in}{5.607485in}}{\pgfqpoint{11.621376in}{5.618084in}}{\pgfqpoint{11.613562in}{5.625897in}}%
\pgfpathcurveto{\pgfqpoint{11.605748in}{5.633711in}}{\pgfqpoint{11.595149in}{5.638101in}}{\pgfqpoint{11.584099in}{5.638101in}}%
\pgfpathcurveto{\pgfqpoint{11.573049in}{5.638101in}}{\pgfqpoint{11.562450in}{5.633711in}}{\pgfqpoint{11.554636in}{5.625897in}}%
\pgfpathcurveto{\pgfqpoint{11.546823in}{5.618084in}}{\pgfqpoint{11.542433in}{5.607485in}}{\pgfqpoint{11.542433in}{5.596435in}}%
\pgfpathcurveto{\pgfqpoint{11.542433in}{5.585385in}}{\pgfqpoint{11.546823in}{5.574785in}}{\pgfqpoint{11.554636in}{5.566972in}}%
\pgfpathcurveto{\pgfqpoint{11.562450in}{5.559158in}}{\pgfqpoint{11.573049in}{5.554768in}}{\pgfqpoint{11.584099in}{5.554768in}}%
\pgfpathlineto{\pgfqpoint{11.584099in}{5.554768in}}%
\pgfpathclose%
\pgfusepath{stroke}%
\end{pgfscope}%
\begin{pgfscope}%
\pgfpathrectangle{\pgfqpoint{7.512535in}{0.437222in}}{\pgfqpoint{6.275590in}{5.159444in}}%
\pgfusepath{clip}%
\pgfsetbuttcap%
\pgfsetroundjoin%
\pgfsetlinewidth{1.003750pt}%
\definecolor{currentstroke}{rgb}{1.000000,0.000000,0.000000}%
\pgfsetstrokecolor{currentstroke}%
\pgfsetdash{}{0pt}%
\pgfpathmoveto{\pgfqpoint{8.241491in}{0.951424in}}%
\pgfpathcurveto{\pgfqpoint{8.252541in}{0.951424in}}{\pgfqpoint{8.263140in}{0.955814in}}{\pgfqpoint{8.270954in}{0.963628in}}%
\pgfpathcurveto{\pgfqpoint{8.278768in}{0.971441in}}{\pgfqpoint{8.283158in}{0.982040in}}{\pgfqpoint{8.283158in}{0.993090in}}%
\pgfpathcurveto{\pgfqpoint{8.283158in}{1.004141in}}{\pgfqpoint{8.278768in}{1.014740in}}{\pgfqpoint{8.270954in}{1.022553in}}%
\pgfpathcurveto{\pgfqpoint{8.263140in}{1.030367in}}{\pgfqpoint{8.252541in}{1.034757in}}{\pgfqpoint{8.241491in}{1.034757in}}%
\pgfpathcurveto{\pgfqpoint{8.230441in}{1.034757in}}{\pgfqpoint{8.219842in}{1.030367in}}{\pgfqpoint{8.212029in}{1.022553in}}%
\pgfpathcurveto{\pgfqpoint{8.204215in}{1.014740in}}{\pgfqpoint{8.199825in}{1.004141in}}{\pgfqpoint{8.199825in}{0.993090in}}%
\pgfpathcurveto{\pgfqpoint{8.199825in}{0.982040in}}{\pgfqpoint{8.204215in}{0.971441in}}{\pgfqpoint{8.212029in}{0.963628in}}%
\pgfpathcurveto{\pgfqpoint{8.219842in}{0.955814in}}{\pgfqpoint{8.230441in}{0.951424in}}{\pgfqpoint{8.241491in}{0.951424in}}%
\pgfpathlineto{\pgfqpoint{8.241491in}{0.951424in}}%
\pgfpathclose%
\pgfusepath{stroke}%
\end{pgfscope}%
\begin{pgfscope}%
\pgfpathrectangle{\pgfqpoint{7.512535in}{0.437222in}}{\pgfqpoint{6.275590in}{5.159444in}}%
\pgfusepath{clip}%
\pgfsetbuttcap%
\pgfsetroundjoin%
\pgfsetlinewidth{1.003750pt}%
\definecolor{currentstroke}{rgb}{1.000000,0.000000,0.000000}%
\pgfsetstrokecolor{currentstroke}%
\pgfsetdash{}{0pt}%
\pgfpathmoveto{\pgfqpoint{10.505790in}{4.638284in}}%
\pgfpathcurveto{\pgfqpoint{10.516840in}{4.638284in}}{\pgfqpoint{10.527439in}{4.642674in}}{\pgfqpoint{10.535252in}{4.650488in}}%
\pgfpathcurveto{\pgfqpoint{10.543066in}{4.658302in}}{\pgfqpoint{10.547456in}{4.668901in}}{\pgfqpoint{10.547456in}{4.679951in}}%
\pgfpathcurveto{\pgfqpoint{10.547456in}{4.691001in}}{\pgfqpoint{10.543066in}{4.701600in}}{\pgfqpoint{10.535252in}{4.709414in}}%
\pgfpathcurveto{\pgfqpoint{10.527439in}{4.717227in}}{\pgfqpoint{10.516840in}{4.721618in}}{\pgfqpoint{10.505790in}{4.721618in}}%
\pgfpathcurveto{\pgfqpoint{10.494739in}{4.721618in}}{\pgfqpoint{10.484140in}{4.717227in}}{\pgfqpoint{10.476327in}{4.709414in}}%
\pgfpathcurveto{\pgfqpoint{10.468513in}{4.701600in}}{\pgfqpoint{10.464123in}{4.691001in}}{\pgfqpoint{10.464123in}{4.679951in}}%
\pgfpathcurveto{\pgfqpoint{10.464123in}{4.668901in}}{\pgfqpoint{10.468513in}{4.658302in}}{\pgfqpoint{10.476327in}{4.650488in}}%
\pgfpathcurveto{\pgfqpoint{10.484140in}{4.642674in}}{\pgfqpoint{10.494739in}{4.638284in}}{\pgfqpoint{10.505790in}{4.638284in}}%
\pgfpathlineto{\pgfqpoint{10.505790in}{4.638284in}}%
\pgfpathclose%
\pgfusepath{stroke}%
\end{pgfscope}%
\begin{pgfscope}%
\pgfpathrectangle{\pgfqpoint{7.512535in}{0.437222in}}{\pgfqpoint{6.275590in}{5.159444in}}%
\pgfusepath{clip}%
\pgfsetbuttcap%
\pgfsetroundjoin%
\pgfsetlinewidth{1.003750pt}%
\definecolor{currentstroke}{rgb}{1.000000,0.000000,0.000000}%
\pgfsetstrokecolor{currentstroke}%
\pgfsetdash{}{0pt}%
\pgfpathmoveto{\pgfqpoint{12.512691in}{5.528701in}}%
\pgfpathcurveto{\pgfqpoint{12.523741in}{5.528701in}}{\pgfqpoint{12.534340in}{5.533092in}}{\pgfqpoint{12.542154in}{5.540905in}}%
\pgfpathcurveto{\pgfqpoint{12.549968in}{5.548719in}}{\pgfqpoint{12.554358in}{5.559318in}}{\pgfqpoint{12.554358in}{5.570368in}}%
\pgfpathcurveto{\pgfqpoint{12.554358in}{5.581418in}}{\pgfqpoint{12.549968in}{5.592017in}}{\pgfqpoint{12.542154in}{5.599831in}}%
\pgfpathcurveto{\pgfqpoint{12.534340in}{5.607645in}}{\pgfqpoint{12.523741in}{5.612035in}}{\pgfqpoint{12.512691in}{5.612035in}}%
\pgfpathcurveto{\pgfqpoint{12.501641in}{5.612035in}}{\pgfqpoint{12.491042in}{5.607645in}}{\pgfqpoint{12.483228in}{5.599831in}}%
\pgfpathcurveto{\pgfqpoint{12.475415in}{5.592017in}}{\pgfqpoint{12.471024in}{5.581418in}}{\pgfqpoint{12.471024in}{5.570368in}}%
\pgfpathcurveto{\pgfqpoint{12.471024in}{5.559318in}}{\pgfqpoint{12.475415in}{5.548719in}}{\pgfqpoint{12.483228in}{5.540905in}}%
\pgfpathcurveto{\pgfqpoint{12.491042in}{5.533092in}}{\pgfqpoint{12.501641in}{5.528701in}}{\pgfqpoint{12.512691in}{5.528701in}}%
\pgfpathlineto{\pgfqpoint{12.512691in}{5.528701in}}%
\pgfpathclose%
\pgfusepath{stroke}%
\end{pgfscope}%
\begin{pgfscope}%
\pgfpathrectangle{\pgfqpoint{7.512535in}{0.437222in}}{\pgfqpoint{6.275590in}{5.159444in}}%
\pgfusepath{clip}%
\pgfsetbuttcap%
\pgfsetroundjoin%
\pgfsetlinewidth{1.003750pt}%
\definecolor{currentstroke}{rgb}{1.000000,0.000000,0.000000}%
\pgfsetstrokecolor{currentstroke}%
\pgfsetdash{}{0pt}%
\pgfpathmoveto{\pgfqpoint{12.605571in}{5.514771in}}%
\pgfpathcurveto{\pgfqpoint{12.616621in}{5.514771in}}{\pgfqpoint{12.627220in}{5.519162in}}{\pgfqpoint{12.635034in}{5.526975in}}%
\pgfpathcurveto{\pgfqpoint{12.642847in}{5.534789in}}{\pgfqpoint{12.647237in}{5.545388in}}{\pgfqpoint{12.647237in}{5.556438in}}%
\pgfpathcurveto{\pgfqpoint{12.647237in}{5.567488in}}{\pgfqpoint{12.642847in}{5.578087in}}{\pgfqpoint{12.635034in}{5.585901in}}%
\pgfpathcurveto{\pgfqpoint{12.627220in}{5.593715in}}{\pgfqpoint{12.616621in}{5.598105in}}{\pgfqpoint{12.605571in}{5.598105in}}%
\pgfpathcurveto{\pgfqpoint{12.594521in}{5.598105in}}{\pgfqpoint{12.583922in}{5.593715in}}{\pgfqpoint{12.576108in}{5.585901in}}%
\pgfpathcurveto{\pgfqpoint{12.568294in}{5.578087in}}{\pgfqpoint{12.563904in}{5.567488in}}{\pgfqpoint{12.563904in}{5.556438in}}%
\pgfpathcurveto{\pgfqpoint{12.563904in}{5.545388in}}{\pgfqpoint{12.568294in}{5.534789in}}{\pgfqpoint{12.576108in}{5.526975in}}%
\pgfpathcurveto{\pgfqpoint{12.583922in}{5.519162in}}{\pgfqpoint{12.594521in}{5.514771in}}{\pgfqpoint{12.605571in}{5.514771in}}%
\pgfpathlineto{\pgfqpoint{12.605571in}{5.514771in}}%
\pgfpathclose%
\pgfusepath{stroke}%
\end{pgfscope}%
\begin{pgfscope}%
\pgfpathrectangle{\pgfqpoint{7.512535in}{0.437222in}}{\pgfqpoint{6.275590in}{5.159444in}}%
\pgfusepath{clip}%
\pgfsetbuttcap%
\pgfsetroundjoin%
\pgfsetlinewidth{1.003750pt}%
\definecolor{currentstroke}{rgb}{1.000000,0.000000,0.000000}%
\pgfsetstrokecolor{currentstroke}%
\pgfsetdash{}{0pt}%
\pgfpathmoveto{\pgfqpoint{11.209461in}{4.964719in}}%
\pgfpathcurveto{\pgfqpoint{11.220511in}{4.964719in}}{\pgfqpoint{11.231110in}{4.969109in}}{\pgfqpoint{11.238924in}{4.976923in}}%
\pgfpathcurveto{\pgfqpoint{11.246737in}{4.984737in}}{\pgfqpoint{11.251128in}{4.995336in}}{\pgfqpoint{11.251128in}{5.006386in}}%
\pgfpathcurveto{\pgfqpoint{11.251128in}{5.017436in}}{\pgfqpoint{11.246737in}{5.028035in}}{\pgfqpoint{11.238924in}{5.035848in}}%
\pgfpathcurveto{\pgfqpoint{11.231110in}{5.043662in}}{\pgfqpoint{11.220511in}{5.048052in}}{\pgfqpoint{11.209461in}{5.048052in}}%
\pgfpathcurveto{\pgfqpoint{11.198411in}{5.048052in}}{\pgfqpoint{11.187812in}{5.043662in}}{\pgfqpoint{11.179998in}{5.035848in}}%
\pgfpathcurveto{\pgfqpoint{11.172185in}{5.028035in}}{\pgfqpoint{11.167794in}{5.017436in}}{\pgfqpoint{11.167794in}{5.006386in}}%
\pgfpathcurveto{\pgfqpoint{11.167794in}{4.995336in}}{\pgfqpoint{11.172185in}{4.984737in}}{\pgfqpoint{11.179998in}{4.976923in}}%
\pgfpathcurveto{\pgfqpoint{11.187812in}{4.969109in}}{\pgfqpoint{11.198411in}{4.964719in}}{\pgfqpoint{11.209461in}{4.964719in}}%
\pgfpathlineto{\pgfqpoint{11.209461in}{4.964719in}}%
\pgfpathclose%
\pgfusepath{stroke}%
\end{pgfscope}%
\begin{pgfscope}%
\pgfpathrectangle{\pgfqpoint{7.512535in}{0.437222in}}{\pgfqpoint{6.275590in}{5.159444in}}%
\pgfusepath{clip}%
\pgfsetbuttcap%
\pgfsetroundjoin%
\pgfsetlinewidth{1.003750pt}%
\definecolor{currentstroke}{rgb}{1.000000,0.000000,0.000000}%
\pgfsetstrokecolor{currentstroke}%
\pgfsetdash{}{0pt}%
\pgfpathmoveto{\pgfqpoint{8.301869in}{1.268315in}}%
\pgfpathcurveto{\pgfqpoint{8.312919in}{1.268315in}}{\pgfqpoint{8.323518in}{1.272705in}}{\pgfqpoint{8.331332in}{1.280519in}}%
\pgfpathcurveto{\pgfqpoint{8.339145in}{1.288333in}}{\pgfqpoint{8.343535in}{1.298932in}}{\pgfqpoint{8.343535in}{1.309982in}}%
\pgfpathcurveto{\pgfqpoint{8.343535in}{1.321032in}}{\pgfqpoint{8.339145in}{1.331631in}}{\pgfqpoint{8.331332in}{1.339445in}}%
\pgfpathcurveto{\pgfqpoint{8.323518in}{1.347258in}}{\pgfqpoint{8.312919in}{1.351648in}}{\pgfqpoint{8.301869in}{1.351648in}}%
\pgfpathcurveto{\pgfqpoint{8.290819in}{1.351648in}}{\pgfqpoint{8.280220in}{1.347258in}}{\pgfqpoint{8.272406in}{1.339445in}}%
\pgfpathcurveto{\pgfqpoint{8.264592in}{1.331631in}}{\pgfqpoint{8.260202in}{1.321032in}}{\pgfqpoint{8.260202in}{1.309982in}}%
\pgfpathcurveto{\pgfqpoint{8.260202in}{1.298932in}}{\pgfqpoint{8.264592in}{1.288333in}}{\pgfqpoint{8.272406in}{1.280519in}}%
\pgfpathcurveto{\pgfqpoint{8.280220in}{1.272705in}}{\pgfqpoint{8.290819in}{1.268315in}}{\pgfqpoint{8.301869in}{1.268315in}}%
\pgfpathlineto{\pgfqpoint{8.301869in}{1.268315in}}%
\pgfpathclose%
\pgfusepath{stroke}%
\end{pgfscope}%
\begin{pgfscope}%
\pgfpathrectangle{\pgfqpoint{7.512535in}{0.437222in}}{\pgfqpoint{6.275590in}{5.159444in}}%
\pgfusepath{clip}%
\pgfsetbuttcap%
\pgfsetroundjoin%
\pgfsetlinewidth{1.003750pt}%
\definecolor{currentstroke}{rgb}{1.000000,0.000000,0.000000}%
\pgfsetstrokecolor{currentstroke}%
\pgfsetdash{}{0pt}%
\pgfpathmoveto{\pgfqpoint{9.224509in}{1.906373in}}%
\pgfpathcurveto{\pgfqpoint{9.235559in}{1.906373in}}{\pgfqpoint{9.246158in}{1.910763in}}{\pgfqpoint{9.253971in}{1.918577in}}%
\pgfpathcurveto{\pgfqpoint{9.261785in}{1.926390in}}{\pgfqpoint{9.266175in}{1.936989in}}{\pgfqpoint{9.266175in}{1.948039in}}%
\pgfpathcurveto{\pgfqpoint{9.266175in}{1.959090in}}{\pgfqpoint{9.261785in}{1.969689in}}{\pgfqpoint{9.253971in}{1.977502in}}%
\pgfpathcurveto{\pgfqpoint{9.246158in}{1.985316in}}{\pgfqpoint{9.235559in}{1.989706in}}{\pgfqpoint{9.224509in}{1.989706in}}%
\pgfpathcurveto{\pgfqpoint{9.213459in}{1.989706in}}{\pgfqpoint{9.202860in}{1.985316in}}{\pgfqpoint{9.195046in}{1.977502in}}%
\pgfpathcurveto{\pgfqpoint{9.187232in}{1.969689in}}{\pgfqpoint{9.182842in}{1.959090in}}{\pgfqpoint{9.182842in}{1.948039in}}%
\pgfpathcurveto{\pgfqpoint{9.182842in}{1.936989in}}{\pgfqpoint{9.187232in}{1.926390in}}{\pgfqpoint{9.195046in}{1.918577in}}%
\pgfpathcurveto{\pgfqpoint{9.202860in}{1.910763in}}{\pgfqpoint{9.213459in}{1.906373in}}{\pgfqpoint{9.224509in}{1.906373in}}%
\pgfpathlineto{\pgfqpoint{9.224509in}{1.906373in}}%
\pgfpathclose%
\pgfusepath{stroke}%
\end{pgfscope}%
\begin{pgfscope}%
\pgfpathrectangle{\pgfqpoint{7.512535in}{0.437222in}}{\pgfqpoint{6.275590in}{5.159444in}}%
\pgfusepath{clip}%
\pgfsetbuttcap%
\pgfsetroundjoin%
\pgfsetlinewidth{1.003750pt}%
\definecolor{currentstroke}{rgb}{1.000000,0.000000,0.000000}%
\pgfsetstrokecolor{currentstroke}%
\pgfsetdash{}{0pt}%
\pgfpathmoveto{\pgfqpoint{10.685296in}{4.794458in}}%
\pgfpathcurveto{\pgfqpoint{10.696346in}{4.794458in}}{\pgfqpoint{10.706945in}{4.798849in}}{\pgfqpoint{10.714758in}{4.806662in}}%
\pgfpathcurveto{\pgfqpoint{10.722572in}{4.814476in}}{\pgfqpoint{10.726962in}{4.825075in}}{\pgfqpoint{10.726962in}{4.836125in}}%
\pgfpathcurveto{\pgfqpoint{10.726962in}{4.847175in}}{\pgfqpoint{10.722572in}{4.857774in}}{\pgfqpoint{10.714758in}{4.865588in}}%
\pgfpathcurveto{\pgfqpoint{10.706945in}{4.873401in}}{\pgfqpoint{10.696346in}{4.877792in}}{\pgfqpoint{10.685296in}{4.877792in}}%
\pgfpathcurveto{\pgfqpoint{10.674245in}{4.877792in}}{\pgfqpoint{10.663646in}{4.873401in}}{\pgfqpoint{10.655833in}{4.865588in}}%
\pgfpathcurveto{\pgfqpoint{10.648019in}{4.857774in}}{\pgfqpoint{10.643629in}{4.847175in}}{\pgfqpoint{10.643629in}{4.836125in}}%
\pgfpathcurveto{\pgfqpoint{10.643629in}{4.825075in}}{\pgfqpoint{10.648019in}{4.814476in}}{\pgfqpoint{10.655833in}{4.806662in}}%
\pgfpathcurveto{\pgfqpoint{10.663646in}{4.798849in}}{\pgfqpoint{10.674245in}{4.794458in}}{\pgfqpoint{10.685296in}{4.794458in}}%
\pgfpathlineto{\pgfqpoint{10.685296in}{4.794458in}}%
\pgfpathclose%
\pgfusepath{stroke}%
\end{pgfscope}%
\begin{pgfscope}%
\pgfpathrectangle{\pgfqpoint{7.512535in}{0.437222in}}{\pgfqpoint{6.275590in}{5.159444in}}%
\pgfusepath{clip}%
\pgfsetbuttcap%
\pgfsetroundjoin%
\pgfsetlinewidth{1.003750pt}%
\definecolor{currentstroke}{rgb}{1.000000,0.000000,0.000000}%
\pgfsetstrokecolor{currentstroke}%
\pgfsetdash{}{0pt}%
\pgfpathmoveto{\pgfqpoint{7.615257in}{0.500879in}}%
\pgfpathcurveto{\pgfqpoint{7.626308in}{0.500879in}}{\pgfqpoint{7.636907in}{0.505269in}}{\pgfqpoint{7.644720in}{0.513083in}}%
\pgfpathcurveto{\pgfqpoint{7.652534in}{0.520896in}}{\pgfqpoint{7.656924in}{0.531495in}}{\pgfqpoint{7.656924in}{0.542546in}}%
\pgfpathcurveto{\pgfqpoint{7.656924in}{0.553596in}}{\pgfqpoint{7.652534in}{0.564195in}}{\pgfqpoint{7.644720in}{0.572008in}}%
\pgfpathcurveto{\pgfqpoint{7.636907in}{0.579822in}}{\pgfqpoint{7.626308in}{0.584212in}}{\pgfqpoint{7.615257in}{0.584212in}}%
\pgfpathcurveto{\pgfqpoint{7.604207in}{0.584212in}}{\pgfqpoint{7.593608in}{0.579822in}}{\pgfqpoint{7.585795in}{0.572008in}}%
\pgfpathcurveto{\pgfqpoint{7.577981in}{0.564195in}}{\pgfqpoint{7.573591in}{0.553596in}}{\pgfqpoint{7.573591in}{0.542546in}}%
\pgfpathcurveto{\pgfqpoint{7.573591in}{0.531495in}}{\pgfqpoint{7.577981in}{0.520896in}}{\pgfqpoint{7.585795in}{0.513083in}}%
\pgfpathcurveto{\pgfqpoint{7.593608in}{0.505269in}}{\pgfqpoint{7.604207in}{0.500879in}}{\pgfqpoint{7.615257in}{0.500879in}}%
\pgfpathlineto{\pgfqpoint{7.615257in}{0.500879in}}%
\pgfpathclose%
\pgfusepath{stroke}%
\end{pgfscope}%
\begin{pgfscope}%
\pgfpathrectangle{\pgfqpoint{7.512535in}{0.437222in}}{\pgfqpoint{6.275590in}{5.159444in}}%
\pgfusepath{clip}%
\pgfsetbuttcap%
\pgfsetroundjoin%
\pgfsetlinewidth{1.003750pt}%
\definecolor{currentstroke}{rgb}{1.000000,0.000000,0.000000}%
\pgfsetstrokecolor{currentstroke}%
\pgfsetdash{}{0pt}%
\pgfpathmoveto{\pgfqpoint{9.930976in}{3.360584in}}%
\pgfpathcurveto{\pgfqpoint{9.942026in}{3.360584in}}{\pgfqpoint{9.952625in}{3.364974in}}{\pgfqpoint{9.960438in}{3.372788in}}%
\pgfpathcurveto{\pgfqpoint{9.968252in}{3.380601in}}{\pgfqpoint{9.972642in}{3.391200in}}{\pgfqpoint{9.972642in}{3.402251in}}%
\pgfpathcurveto{\pgfqpoint{9.972642in}{3.413301in}}{\pgfqpoint{9.968252in}{3.423900in}}{\pgfqpoint{9.960438in}{3.431713in}}%
\pgfpathcurveto{\pgfqpoint{9.952625in}{3.439527in}}{\pgfqpoint{9.942026in}{3.443917in}}{\pgfqpoint{9.930976in}{3.443917in}}%
\pgfpathcurveto{\pgfqpoint{9.919926in}{3.443917in}}{\pgfqpoint{9.909326in}{3.439527in}}{\pgfqpoint{9.901513in}{3.431713in}}%
\pgfpathcurveto{\pgfqpoint{9.893699in}{3.423900in}}{\pgfqpoint{9.889309in}{3.413301in}}{\pgfqpoint{9.889309in}{3.402251in}}%
\pgfpathcurveto{\pgfqpoint{9.889309in}{3.391200in}}{\pgfqpoint{9.893699in}{3.380601in}}{\pgfqpoint{9.901513in}{3.372788in}}%
\pgfpathcurveto{\pgfqpoint{9.909326in}{3.364974in}}{\pgfqpoint{9.919926in}{3.360584in}}{\pgfqpoint{9.930976in}{3.360584in}}%
\pgfpathlineto{\pgfqpoint{9.930976in}{3.360584in}}%
\pgfpathclose%
\pgfusepath{stroke}%
\end{pgfscope}%
\begin{pgfscope}%
\pgfpathrectangle{\pgfqpoint{7.512535in}{0.437222in}}{\pgfqpoint{6.275590in}{5.159444in}}%
\pgfusepath{clip}%
\pgfsetbuttcap%
\pgfsetroundjoin%
\pgfsetlinewidth{1.003750pt}%
\definecolor{currentstroke}{rgb}{1.000000,0.000000,0.000000}%
\pgfsetstrokecolor{currentstroke}%
\pgfsetdash{}{0pt}%
\pgfpathmoveto{\pgfqpoint{7.774832in}{0.489958in}}%
\pgfpathcurveto{\pgfqpoint{7.785882in}{0.489958in}}{\pgfqpoint{7.796481in}{0.494348in}}{\pgfqpoint{7.804295in}{0.502162in}}%
\pgfpathcurveto{\pgfqpoint{7.812109in}{0.509975in}}{\pgfqpoint{7.816499in}{0.520574in}}{\pgfqpoint{7.816499in}{0.531624in}}%
\pgfpathcurveto{\pgfqpoint{7.816499in}{0.542675in}}{\pgfqpoint{7.812109in}{0.553274in}}{\pgfqpoint{7.804295in}{0.561087in}}%
\pgfpathcurveto{\pgfqpoint{7.796481in}{0.568901in}}{\pgfqpoint{7.785882in}{0.573291in}}{\pgfqpoint{7.774832in}{0.573291in}}%
\pgfpathcurveto{\pgfqpoint{7.763782in}{0.573291in}}{\pgfqpoint{7.753183in}{0.568901in}}{\pgfqpoint{7.745370in}{0.561087in}}%
\pgfpathcurveto{\pgfqpoint{7.737556in}{0.553274in}}{\pgfqpoint{7.733166in}{0.542675in}}{\pgfqpoint{7.733166in}{0.531624in}}%
\pgfpathcurveto{\pgfqpoint{7.733166in}{0.520574in}}{\pgfqpoint{7.737556in}{0.509975in}}{\pgfqpoint{7.745370in}{0.502162in}}%
\pgfpathcurveto{\pgfqpoint{7.753183in}{0.494348in}}{\pgfqpoint{7.763782in}{0.489958in}}{\pgfqpoint{7.774832in}{0.489958in}}%
\pgfpathlineto{\pgfqpoint{7.774832in}{0.489958in}}%
\pgfpathclose%
\pgfusepath{stroke}%
\end{pgfscope}%
\begin{pgfscope}%
\pgfpathrectangle{\pgfqpoint{7.512535in}{0.437222in}}{\pgfqpoint{6.275590in}{5.159444in}}%
\pgfusepath{clip}%
\pgfsetbuttcap%
\pgfsetroundjoin%
\pgfsetlinewidth{1.003750pt}%
\definecolor{currentstroke}{rgb}{1.000000,0.000000,0.000000}%
\pgfsetstrokecolor{currentstroke}%
\pgfsetdash{}{0pt}%
\pgfpathmoveto{\pgfqpoint{10.748315in}{3.950352in}}%
\pgfpathcurveto{\pgfqpoint{10.759366in}{3.950352in}}{\pgfqpoint{10.769965in}{3.954742in}}{\pgfqpoint{10.777778in}{3.962556in}}%
\pgfpathcurveto{\pgfqpoint{10.785592in}{3.970370in}}{\pgfqpoint{10.789982in}{3.980969in}}{\pgfqpoint{10.789982in}{3.992019in}}%
\pgfpathcurveto{\pgfqpoint{10.789982in}{4.003069in}}{\pgfqpoint{10.785592in}{4.013668in}}{\pgfqpoint{10.777778in}{4.021482in}}%
\pgfpathcurveto{\pgfqpoint{10.769965in}{4.029295in}}{\pgfqpoint{10.759366in}{4.033685in}}{\pgfqpoint{10.748315in}{4.033685in}}%
\pgfpathcurveto{\pgfqpoint{10.737265in}{4.033685in}}{\pgfqpoint{10.726666in}{4.029295in}}{\pgfqpoint{10.718853in}{4.021482in}}%
\pgfpathcurveto{\pgfqpoint{10.711039in}{4.013668in}}{\pgfqpoint{10.706649in}{4.003069in}}{\pgfqpoint{10.706649in}{3.992019in}}%
\pgfpathcurveto{\pgfqpoint{10.706649in}{3.980969in}}{\pgfqpoint{10.711039in}{3.970370in}}{\pgfqpoint{10.718853in}{3.962556in}}%
\pgfpathcurveto{\pgfqpoint{10.726666in}{3.954742in}}{\pgfqpoint{10.737265in}{3.950352in}}{\pgfqpoint{10.748315in}{3.950352in}}%
\pgfpathlineto{\pgfqpoint{10.748315in}{3.950352in}}%
\pgfpathclose%
\pgfusepath{stroke}%
\end{pgfscope}%
\begin{pgfscope}%
\pgfpathrectangle{\pgfqpoint{7.512535in}{0.437222in}}{\pgfqpoint{6.275590in}{5.159444in}}%
\pgfusepath{clip}%
\pgfsetbuttcap%
\pgfsetroundjoin%
\pgfsetlinewidth{1.003750pt}%
\definecolor{currentstroke}{rgb}{1.000000,0.000000,0.000000}%
\pgfsetstrokecolor{currentstroke}%
\pgfsetdash{}{0pt}%
\pgfpathmoveto{\pgfqpoint{8.828690in}{2.388439in}}%
\pgfpathcurveto{\pgfqpoint{8.839740in}{2.388439in}}{\pgfqpoint{8.850339in}{2.392829in}}{\pgfqpoint{8.858153in}{2.400643in}}%
\pgfpathcurveto{\pgfqpoint{8.865966in}{2.408457in}}{\pgfqpoint{8.870357in}{2.419056in}}{\pgfqpoint{8.870357in}{2.430106in}}%
\pgfpathcurveto{\pgfqpoint{8.870357in}{2.441156in}}{\pgfqpoint{8.865966in}{2.451755in}}{\pgfqpoint{8.858153in}{2.459569in}}%
\pgfpathcurveto{\pgfqpoint{8.850339in}{2.467382in}}{\pgfqpoint{8.839740in}{2.471772in}}{\pgfqpoint{8.828690in}{2.471772in}}%
\pgfpathcurveto{\pgfqpoint{8.817640in}{2.471772in}}{\pgfqpoint{8.807041in}{2.467382in}}{\pgfqpoint{8.799227in}{2.459569in}}%
\pgfpathcurveto{\pgfqpoint{8.791413in}{2.451755in}}{\pgfqpoint{8.787023in}{2.441156in}}{\pgfqpoint{8.787023in}{2.430106in}}%
\pgfpathcurveto{\pgfqpoint{8.787023in}{2.419056in}}{\pgfqpoint{8.791413in}{2.408457in}}{\pgfqpoint{8.799227in}{2.400643in}}%
\pgfpathcurveto{\pgfqpoint{8.807041in}{2.392829in}}{\pgfqpoint{8.817640in}{2.388439in}}{\pgfqpoint{8.828690in}{2.388439in}}%
\pgfpathlineto{\pgfqpoint{8.828690in}{2.388439in}}%
\pgfpathclose%
\pgfusepath{stroke}%
\end{pgfscope}%
\begin{pgfscope}%
\pgfpathrectangle{\pgfqpoint{7.512535in}{0.437222in}}{\pgfqpoint{6.275590in}{5.159444in}}%
\pgfusepath{clip}%
\pgfsetbuttcap%
\pgfsetroundjoin%
\pgfsetlinewidth{1.003750pt}%
\definecolor{currentstroke}{rgb}{1.000000,0.000000,0.000000}%
\pgfsetstrokecolor{currentstroke}%
\pgfsetdash{}{0pt}%
\pgfpathmoveto{\pgfqpoint{12.091706in}{5.554434in}}%
\pgfpathcurveto{\pgfqpoint{12.102756in}{5.554434in}}{\pgfqpoint{12.113355in}{5.558824in}}{\pgfqpoint{12.121169in}{5.566638in}}%
\pgfpathcurveto{\pgfqpoint{12.128983in}{5.574452in}}{\pgfqpoint{12.133373in}{5.585051in}}{\pgfqpoint{12.133373in}{5.596101in}}%
\pgfpathcurveto{\pgfqpoint{12.133373in}{5.607151in}}{\pgfqpoint{12.128983in}{5.617750in}}{\pgfqpoint{12.121169in}{5.625564in}}%
\pgfpathcurveto{\pgfqpoint{12.113355in}{5.633377in}}{\pgfqpoint{12.102756in}{5.637767in}}{\pgfqpoint{12.091706in}{5.637767in}}%
\pgfpathcurveto{\pgfqpoint{12.080656in}{5.637767in}}{\pgfqpoint{12.070057in}{5.633377in}}{\pgfqpoint{12.062243in}{5.625564in}}%
\pgfpathcurveto{\pgfqpoint{12.054430in}{5.617750in}}{\pgfqpoint{12.050040in}{5.607151in}}{\pgfqpoint{12.050040in}{5.596101in}}%
\pgfpathcurveto{\pgfqpoint{12.050040in}{5.585051in}}{\pgfqpoint{12.054430in}{5.574452in}}{\pgfqpoint{12.062243in}{5.566638in}}%
\pgfpathcurveto{\pgfqpoint{12.070057in}{5.558824in}}{\pgfqpoint{12.080656in}{5.554434in}}{\pgfqpoint{12.091706in}{5.554434in}}%
\pgfpathlineto{\pgfqpoint{12.091706in}{5.554434in}}%
\pgfpathclose%
\pgfusepath{stroke}%
\end{pgfscope}%
\begin{pgfscope}%
\pgfpathrectangle{\pgfqpoint{7.512535in}{0.437222in}}{\pgfqpoint{6.275590in}{5.159444in}}%
\pgfusepath{clip}%
\pgfsetbuttcap%
\pgfsetroundjoin%
\pgfsetlinewidth{1.003750pt}%
\definecolor{currentstroke}{rgb}{1.000000,0.000000,0.000000}%
\pgfsetstrokecolor{currentstroke}%
\pgfsetdash{}{0pt}%
\pgfpathmoveto{\pgfqpoint{10.146379in}{3.452205in}}%
\pgfpathcurveto{\pgfqpoint{10.157429in}{3.452205in}}{\pgfqpoint{10.168028in}{3.456595in}}{\pgfqpoint{10.175841in}{3.464409in}}%
\pgfpathcurveto{\pgfqpoint{10.183655in}{3.472222in}}{\pgfqpoint{10.188045in}{3.482821in}}{\pgfqpoint{10.188045in}{3.493871in}}%
\pgfpathcurveto{\pgfqpoint{10.188045in}{3.504921in}}{\pgfqpoint{10.183655in}{3.515521in}}{\pgfqpoint{10.175841in}{3.523334in}}%
\pgfpathcurveto{\pgfqpoint{10.168028in}{3.531148in}}{\pgfqpoint{10.157429in}{3.535538in}}{\pgfqpoint{10.146379in}{3.535538in}}%
\pgfpathcurveto{\pgfqpoint{10.135328in}{3.535538in}}{\pgfqpoint{10.124729in}{3.531148in}}{\pgfqpoint{10.116916in}{3.523334in}}%
\pgfpathcurveto{\pgfqpoint{10.109102in}{3.515521in}}{\pgfqpoint{10.104712in}{3.504921in}}{\pgfqpoint{10.104712in}{3.493871in}}%
\pgfpathcurveto{\pgfqpoint{10.104712in}{3.482821in}}{\pgfqpoint{10.109102in}{3.472222in}}{\pgfqpoint{10.116916in}{3.464409in}}%
\pgfpathcurveto{\pgfqpoint{10.124729in}{3.456595in}}{\pgfqpoint{10.135328in}{3.452205in}}{\pgfqpoint{10.146379in}{3.452205in}}%
\pgfpathlineto{\pgfqpoint{10.146379in}{3.452205in}}%
\pgfpathclose%
\pgfusepath{stroke}%
\end{pgfscope}%
\begin{pgfscope}%
\pgfpathrectangle{\pgfqpoint{7.512535in}{0.437222in}}{\pgfqpoint{6.275590in}{5.159444in}}%
\pgfusepath{clip}%
\pgfsetbuttcap%
\pgfsetroundjoin%
\pgfsetlinewidth{1.003750pt}%
\definecolor{currentstroke}{rgb}{1.000000,0.000000,0.000000}%
\pgfsetstrokecolor{currentstroke}%
\pgfsetdash{}{0pt}%
\pgfpathmoveto{\pgfqpoint{8.970593in}{2.986627in}}%
\pgfpathcurveto{\pgfqpoint{8.981644in}{2.986627in}}{\pgfqpoint{8.992243in}{2.991017in}}{\pgfqpoint{9.000056in}{2.998830in}}%
\pgfpathcurveto{\pgfqpoint{9.007870in}{3.006644in}}{\pgfqpoint{9.012260in}{3.017243in}}{\pgfqpoint{9.012260in}{3.028293in}}%
\pgfpathcurveto{\pgfqpoint{9.012260in}{3.039343in}}{\pgfqpoint{9.007870in}{3.049942in}}{\pgfqpoint{9.000056in}{3.057756in}}%
\pgfpathcurveto{\pgfqpoint{8.992243in}{3.065570in}}{\pgfqpoint{8.981644in}{3.069960in}}{\pgfqpoint{8.970593in}{3.069960in}}%
\pgfpathcurveto{\pgfqpoint{8.959543in}{3.069960in}}{\pgfqpoint{8.948944in}{3.065570in}}{\pgfqpoint{8.941131in}{3.057756in}}%
\pgfpathcurveto{\pgfqpoint{8.933317in}{3.049942in}}{\pgfqpoint{8.928927in}{3.039343in}}{\pgfqpoint{8.928927in}{3.028293in}}%
\pgfpathcurveto{\pgfqpoint{8.928927in}{3.017243in}}{\pgfqpoint{8.933317in}{3.006644in}}{\pgfqpoint{8.941131in}{2.998830in}}%
\pgfpathcurveto{\pgfqpoint{8.948944in}{2.991017in}}{\pgfqpoint{8.959543in}{2.986627in}}{\pgfqpoint{8.970593in}{2.986627in}}%
\pgfpathlineto{\pgfqpoint{8.970593in}{2.986627in}}%
\pgfpathclose%
\pgfusepath{stroke}%
\end{pgfscope}%
\begin{pgfscope}%
\pgfpathrectangle{\pgfqpoint{7.512535in}{0.437222in}}{\pgfqpoint{6.275590in}{5.159444in}}%
\pgfusepath{clip}%
\pgfsetbuttcap%
\pgfsetroundjoin%
\pgfsetlinewidth{1.003750pt}%
\definecolor{currentstroke}{rgb}{1.000000,0.000000,0.000000}%
\pgfsetstrokecolor{currentstroke}%
\pgfsetdash{}{0pt}%
\pgfpathmoveto{\pgfqpoint{13.231580in}{5.554021in}}%
\pgfpathcurveto{\pgfqpoint{13.242630in}{5.554021in}}{\pgfqpoint{13.253229in}{5.558411in}}{\pgfqpoint{13.261043in}{5.566225in}}%
\pgfpathcurveto{\pgfqpoint{13.268856in}{5.574039in}}{\pgfqpoint{13.273247in}{5.584638in}}{\pgfqpoint{13.273247in}{5.595688in}}%
\pgfpathcurveto{\pgfqpoint{13.273247in}{5.606738in}}{\pgfqpoint{13.268856in}{5.617337in}}{\pgfqpoint{13.261043in}{5.625150in}}%
\pgfpathcurveto{\pgfqpoint{13.253229in}{5.632964in}}{\pgfqpoint{13.242630in}{5.637354in}}{\pgfqpoint{13.231580in}{5.637354in}}%
\pgfpathcurveto{\pgfqpoint{13.220530in}{5.637354in}}{\pgfqpoint{13.209931in}{5.632964in}}{\pgfqpoint{13.202117in}{5.625150in}}%
\pgfpathcurveto{\pgfqpoint{13.194303in}{5.617337in}}{\pgfqpoint{13.189913in}{5.606738in}}{\pgfqpoint{13.189913in}{5.595688in}}%
\pgfpathcurveto{\pgfqpoint{13.189913in}{5.584638in}}{\pgfqpoint{13.194303in}{5.574039in}}{\pgfqpoint{13.202117in}{5.566225in}}%
\pgfpathcurveto{\pgfqpoint{13.209931in}{5.558411in}}{\pgfqpoint{13.220530in}{5.554021in}}{\pgfqpoint{13.231580in}{5.554021in}}%
\pgfpathlineto{\pgfqpoint{13.231580in}{5.554021in}}%
\pgfpathclose%
\pgfusepath{stroke}%
\end{pgfscope}%
\begin{pgfscope}%
\pgfpathrectangle{\pgfqpoint{7.512535in}{0.437222in}}{\pgfqpoint{6.275590in}{5.159444in}}%
\pgfusepath{clip}%
\pgfsetbuttcap%
\pgfsetroundjoin%
\pgfsetlinewidth{1.003750pt}%
\definecolor{currentstroke}{rgb}{1.000000,0.000000,0.000000}%
\pgfsetstrokecolor{currentstroke}%
\pgfsetdash{}{0pt}%
\pgfpathmoveto{\pgfqpoint{8.914165in}{2.168626in}}%
\pgfpathcurveto{\pgfqpoint{8.925215in}{2.168626in}}{\pgfqpoint{8.935814in}{2.173016in}}{\pgfqpoint{8.943628in}{2.180830in}}%
\pgfpathcurveto{\pgfqpoint{8.951442in}{2.188644in}}{\pgfqpoint{8.955832in}{2.199243in}}{\pgfqpoint{8.955832in}{2.210293in}}%
\pgfpathcurveto{\pgfqpoint{8.955832in}{2.221343in}}{\pgfqpoint{8.951442in}{2.231942in}}{\pgfqpoint{8.943628in}{2.239756in}}%
\pgfpathcurveto{\pgfqpoint{8.935814in}{2.247569in}}{\pgfqpoint{8.925215in}{2.251960in}}{\pgfqpoint{8.914165in}{2.251960in}}%
\pgfpathcurveto{\pgfqpoint{8.903115in}{2.251960in}}{\pgfqpoint{8.892516in}{2.247569in}}{\pgfqpoint{8.884702in}{2.239756in}}%
\pgfpathcurveto{\pgfqpoint{8.876889in}{2.231942in}}{\pgfqpoint{8.872498in}{2.221343in}}{\pgfqpoint{8.872498in}{2.210293in}}%
\pgfpathcurveto{\pgfqpoint{8.872498in}{2.199243in}}{\pgfqpoint{8.876889in}{2.188644in}}{\pgfqpoint{8.884702in}{2.180830in}}%
\pgfpathcurveto{\pgfqpoint{8.892516in}{2.173016in}}{\pgfqpoint{8.903115in}{2.168626in}}{\pgfqpoint{8.914165in}{2.168626in}}%
\pgfpathlineto{\pgfqpoint{8.914165in}{2.168626in}}%
\pgfpathclose%
\pgfusepath{stroke}%
\end{pgfscope}%
\begin{pgfscope}%
\pgfpathrectangle{\pgfqpoint{7.512535in}{0.437222in}}{\pgfqpoint{6.275590in}{5.159444in}}%
\pgfusepath{clip}%
\pgfsetbuttcap%
\pgfsetroundjoin%
\pgfsetlinewidth{1.003750pt}%
\definecolor{currentstroke}{rgb}{1.000000,0.000000,0.000000}%
\pgfsetstrokecolor{currentstroke}%
\pgfsetdash{}{0pt}%
\pgfpathmoveto{\pgfqpoint{8.313666in}{1.166856in}}%
\pgfpathcurveto{\pgfqpoint{8.324716in}{1.166856in}}{\pgfqpoint{8.335315in}{1.171246in}}{\pgfqpoint{8.343129in}{1.179060in}}%
\pgfpathcurveto{\pgfqpoint{8.350942in}{1.186874in}}{\pgfqpoint{8.355333in}{1.197473in}}{\pgfqpoint{8.355333in}{1.208523in}}%
\pgfpathcurveto{\pgfqpoint{8.355333in}{1.219573in}}{\pgfqpoint{8.350942in}{1.230172in}}{\pgfqpoint{8.343129in}{1.237985in}}%
\pgfpathcurveto{\pgfqpoint{8.335315in}{1.245799in}}{\pgfqpoint{8.324716in}{1.250189in}}{\pgfqpoint{8.313666in}{1.250189in}}%
\pgfpathcurveto{\pgfqpoint{8.302616in}{1.250189in}}{\pgfqpoint{8.292017in}{1.245799in}}{\pgfqpoint{8.284203in}{1.237985in}}%
\pgfpathcurveto{\pgfqpoint{8.276390in}{1.230172in}}{\pgfqpoint{8.271999in}{1.219573in}}{\pgfqpoint{8.271999in}{1.208523in}}%
\pgfpathcurveto{\pgfqpoint{8.271999in}{1.197473in}}{\pgfqpoint{8.276390in}{1.186874in}}{\pgfqpoint{8.284203in}{1.179060in}}%
\pgfpathcurveto{\pgfqpoint{8.292017in}{1.171246in}}{\pgfqpoint{8.302616in}{1.166856in}}{\pgfqpoint{8.313666in}{1.166856in}}%
\pgfpathlineto{\pgfqpoint{8.313666in}{1.166856in}}%
\pgfpathclose%
\pgfusepath{stroke}%
\end{pgfscope}%
\begin{pgfscope}%
\pgfpathrectangle{\pgfqpoint{7.512535in}{0.437222in}}{\pgfqpoint{6.275590in}{5.159444in}}%
\pgfusepath{clip}%
\pgfsetbuttcap%
\pgfsetroundjoin%
\pgfsetlinewidth{1.003750pt}%
\definecolor{currentstroke}{rgb}{1.000000,0.000000,0.000000}%
\pgfsetstrokecolor{currentstroke}%
\pgfsetdash{}{0pt}%
\pgfpathmoveto{\pgfqpoint{10.891053in}{5.529053in}}%
\pgfpathcurveto{\pgfqpoint{10.902103in}{5.529053in}}{\pgfqpoint{10.912702in}{5.533443in}}{\pgfqpoint{10.920515in}{5.541257in}}%
\pgfpathcurveto{\pgfqpoint{10.928329in}{5.549070in}}{\pgfqpoint{10.932719in}{5.559670in}}{\pgfqpoint{10.932719in}{5.570720in}}%
\pgfpathcurveto{\pgfqpoint{10.932719in}{5.581770in}}{\pgfqpoint{10.928329in}{5.592369in}}{\pgfqpoint{10.920515in}{5.600182in}}%
\pgfpathcurveto{\pgfqpoint{10.912702in}{5.607996in}}{\pgfqpoint{10.902103in}{5.612386in}}{\pgfqpoint{10.891053in}{5.612386in}}%
\pgfpathcurveto{\pgfqpoint{10.880002in}{5.612386in}}{\pgfqpoint{10.869403in}{5.607996in}}{\pgfqpoint{10.861590in}{5.600182in}}%
\pgfpathcurveto{\pgfqpoint{10.853776in}{5.592369in}}{\pgfqpoint{10.849386in}{5.581770in}}{\pgfqpoint{10.849386in}{5.570720in}}%
\pgfpathcurveto{\pgfqpoint{10.849386in}{5.559670in}}{\pgfqpoint{10.853776in}{5.549070in}}{\pgfqpoint{10.861590in}{5.541257in}}%
\pgfpathcurveto{\pgfqpoint{10.869403in}{5.533443in}}{\pgfqpoint{10.880002in}{5.529053in}}{\pgfqpoint{10.891053in}{5.529053in}}%
\pgfpathlineto{\pgfqpoint{10.891053in}{5.529053in}}%
\pgfpathclose%
\pgfusepath{stroke}%
\end{pgfscope}%
\begin{pgfscope}%
\pgfpathrectangle{\pgfqpoint{7.512535in}{0.437222in}}{\pgfqpoint{6.275590in}{5.159444in}}%
\pgfusepath{clip}%
\pgfsetbuttcap%
\pgfsetroundjoin%
\pgfsetlinewidth{1.003750pt}%
\definecolor{currentstroke}{rgb}{1.000000,0.000000,0.000000}%
\pgfsetstrokecolor{currentstroke}%
\pgfsetdash{}{0pt}%
\pgfpathmoveto{\pgfqpoint{11.492993in}{5.536735in}}%
\pgfpathcurveto{\pgfqpoint{11.504043in}{5.536735in}}{\pgfqpoint{11.514642in}{5.541125in}}{\pgfqpoint{11.522456in}{5.548939in}}%
\pgfpathcurveto{\pgfqpoint{11.530269in}{5.556753in}}{\pgfqpoint{11.534659in}{5.567352in}}{\pgfqpoint{11.534659in}{5.578402in}}%
\pgfpathcurveto{\pgfqpoint{11.534659in}{5.589452in}}{\pgfqpoint{11.530269in}{5.600051in}}{\pgfqpoint{11.522456in}{5.607865in}}%
\pgfpathcurveto{\pgfqpoint{11.514642in}{5.615678in}}{\pgfqpoint{11.504043in}{5.620068in}}{\pgfqpoint{11.492993in}{5.620068in}}%
\pgfpathcurveto{\pgfqpoint{11.481943in}{5.620068in}}{\pgfqpoint{11.471344in}{5.615678in}}{\pgfqpoint{11.463530in}{5.607865in}}%
\pgfpathcurveto{\pgfqpoint{11.455716in}{5.600051in}}{\pgfqpoint{11.451326in}{5.589452in}}{\pgfqpoint{11.451326in}{5.578402in}}%
\pgfpathcurveto{\pgfqpoint{11.451326in}{5.567352in}}{\pgfqpoint{11.455716in}{5.556753in}}{\pgfqpoint{11.463530in}{5.548939in}}%
\pgfpathcurveto{\pgfqpoint{11.471344in}{5.541125in}}{\pgfqpoint{11.481943in}{5.536735in}}{\pgfqpoint{11.492993in}{5.536735in}}%
\pgfpathlineto{\pgfqpoint{11.492993in}{5.536735in}}%
\pgfpathclose%
\pgfusepath{stroke}%
\end{pgfscope}%
\begin{pgfscope}%
\pgfpathrectangle{\pgfqpoint{7.512535in}{0.437222in}}{\pgfqpoint{6.275590in}{5.159444in}}%
\pgfusepath{clip}%
\pgfsetbuttcap%
\pgfsetroundjoin%
\pgfsetlinewidth{1.003750pt}%
\definecolor{currentstroke}{rgb}{1.000000,0.000000,0.000000}%
\pgfsetstrokecolor{currentstroke}%
\pgfsetdash{}{0pt}%
\pgfpathmoveto{\pgfqpoint{10.891053in}{5.551142in}}%
\pgfpathcurveto{\pgfqpoint{10.902103in}{5.551142in}}{\pgfqpoint{10.912702in}{5.555532in}}{\pgfqpoint{10.920515in}{5.563346in}}%
\pgfpathcurveto{\pgfqpoint{10.928329in}{5.571159in}}{\pgfqpoint{10.932719in}{5.581759in}}{\pgfqpoint{10.932719in}{5.592809in}}%
\pgfpathcurveto{\pgfqpoint{10.932719in}{5.603859in}}{\pgfqpoint{10.928329in}{5.614458in}}{\pgfqpoint{10.920515in}{5.622271in}}%
\pgfpathcurveto{\pgfqpoint{10.912702in}{5.630085in}}{\pgfqpoint{10.902103in}{5.634475in}}{\pgfqpoint{10.891053in}{5.634475in}}%
\pgfpathcurveto{\pgfqpoint{10.880002in}{5.634475in}}{\pgfqpoint{10.869403in}{5.630085in}}{\pgfqpoint{10.861590in}{5.622271in}}%
\pgfpathcurveto{\pgfqpoint{10.853776in}{5.614458in}}{\pgfqpoint{10.849386in}{5.603859in}}{\pgfqpoint{10.849386in}{5.592809in}}%
\pgfpathcurveto{\pgfqpoint{10.849386in}{5.581759in}}{\pgfqpoint{10.853776in}{5.571159in}}{\pgfqpoint{10.861590in}{5.563346in}}%
\pgfpathcurveto{\pgfqpoint{10.869403in}{5.555532in}}{\pgfqpoint{10.880002in}{5.551142in}}{\pgfqpoint{10.891053in}{5.551142in}}%
\pgfpathlineto{\pgfqpoint{10.891053in}{5.551142in}}%
\pgfpathclose%
\pgfusepath{stroke}%
\end{pgfscope}%
\begin{pgfscope}%
\pgfpathrectangle{\pgfqpoint{7.512535in}{0.437222in}}{\pgfqpoint{6.275590in}{5.159444in}}%
\pgfusepath{clip}%
\pgfsetbuttcap%
\pgfsetroundjoin%
\pgfsetlinewidth{1.003750pt}%
\definecolor{currentstroke}{rgb}{1.000000,0.000000,0.000000}%
\pgfsetstrokecolor{currentstroke}%
\pgfsetdash{}{0pt}%
\pgfpathmoveto{\pgfqpoint{9.260881in}{2.709021in}}%
\pgfpathcurveto{\pgfqpoint{9.271931in}{2.709021in}}{\pgfqpoint{9.282530in}{2.713411in}}{\pgfqpoint{9.290344in}{2.721224in}}%
\pgfpathcurveto{\pgfqpoint{9.298158in}{2.729038in}}{\pgfqpoint{9.302548in}{2.739637in}}{\pgfqpoint{9.302548in}{2.750687in}}%
\pgfpathcurveto{\pgfqpoint{9.302548in}{2.761737in}}{\pgfqpoint{9.298158in}{2.772336in}}{\pgfqpoint{9.290344in}{2.780150in}}%
\pgfpathcurveto{\pgfqpoint{9.282530in}{2.787964in}}{\pgfqpoint{9.271931in}{2.792354in}}{\pgfqpoint{9.260881in}{2.792354in}}%
\pgfpathcurveto{\pgfqpoint{9.249831in}{2.792354in}}{\pgfqpoint{9.239232in}{2.787964in}}{\pgfqpoint{9.231418in}{2.780150in}}%
\pgfpathcurveto{\pgfqpoint{9.223605in}{2.772336in}}{\pgfqpoint{9.219215in}{2.761737in}}{\pgfqpoint{9.219215in}{2.750687in}}%
\pgfpathcurveto{\pgfqpoint{9.219215in}{2.739637in}}{\pgfqpoint{9.223605in}{2.729038in}}{\pgfqpoint{9.231418in}{2.721224in}}%
\pgfpathcurveto{\pgfqpoint{9.239232in}{2.713411in}}{\pgfqpoint{9.249831in}{2.709021in}}{\pgfqpoint{9.260881in}{2.709021in}}%
\pgfpathlineto{\pgfqpoint{9.260881in}{2.709021in}}%
\pgfpathclose%
\pgfusepath{stroke}%
\end{pgfscope}%
\begin{pgfscope}%
\pgfpathrectangle{\pgfqpoint{7.512535in}{0.437222in}}{\pgfqpoint{6.275590in}{5.159444in}}%
\pgfusepath{clip}%
\pgfsetbuttcap%
\pgfsetroundjoin%
\pgfsetlinewidth{1.003750pt}%
\definecolor{currentstroke}{rgb}{1.000000,0.000000,0.000000}%
\pgfsetstrokecolor{currentstroke}%
\pgfsetdash{}{0pt}%
\pgfpathmoveto{\pgfqpoint{12.001381in}{5.553809in}}%
\pgfpathcurveto{\pgfqpoint{12.012432in}{5.553809in}}{\pgfqpoint{12.023031in}{5.558199in}}{\pgfqpoint{12.030844in}{5.566013in}}%
\pgfpathcurveto{\pgfqpoint{12.038658in}{5.573826in}}{\pgfqpoint{12.043048in}{5.584425in}}{\pgfqpoint{12.043048in}{5.595475in}}%
\pgfpathcurveto{\pgfqpoint{12.043048in}{5.606525in}}{\pgfqpoint{12.038658in}{5.617125in}}{\pgfqpoint{12.030844in}{5.624938in}}%
\pgfpathcurveto{\pgfqpoint{12.023031in}{5.632752in}}{\pgfqpoint{12.012432in}{5.637142in}}{\pgfqpoint{12.001381in}{5.637142in}}%
\pgfpathcurveto{\pgfqpoint{11.990331in}{5.637142in}}{\pgfqpoint{11.979732in}{5.632752in}}{\pgfqpoint{11.971919in}{5.624938in}}%
\pgfpathcurveto{\pgfqpoint{11.964105in}{5.617125in}}{\pgfqpoint{11.959715in}{5.606525in}}{\pgfqpoint{11.959715in}{5.595475in}}%
\pgfpathcurveto{\pgfqpoint{11.959715in}{5.584425in}}{\pgfqpoint{11.964105in}{5.573826in}}{\pgfqpoint{11.971919in}{5.566013in}}%
\pgfpathcurveto{\pgfqpoint{11.979732in}{5.558199in}}{\pgfqpoint{11.990331in}{5.553809in}}{\pgfqpoint{12.001381in}{5.553809in}}%
\pgfpathlineto{\pgfqpoint{12.001381in}{5.553809in}}%
\pgfpathclose%
\pgfusepath{stroke}%
\end{pgfscope}%
\begin{pgfscope}%
\pgfpathrectangle{\pgfqpoint{7.512535in}{0.437222in}}{\pgfqpoint{6.275590in}{5.159444in}}%
\pgfusepath{clip}%
\pgfsetbuttcap%
\pgfsetroundjoin%
\pgfsetlinewidth{1.003750pt}%
\definecolor{currentstroke}{rgb}{1.000000,0.000000,0.000000}%
\pgfsetstrokecolor{currentstroke}%
\pgfsetdash{}{0pt}%
\pgfpathmoveto{\pgfqpoint{11.101616in}{5.538432in}}%
\pgfpathcurveto{\pgfqpoint{11.112666in}{5.538432in}}{\pgfqpoint{11.123265in}{5.542823in}}{\pgfqpoint{11.131079in}{5.550636in}}%
\pgfpathcurveto{\pgfqpoint{11.138892in}{5.558450in}}{\pgfqpoint{11.143283in}{5.569049in}}{\pgfqpoint{11.143283in}{5.580099in}}%
\pgfpathcurveto{\pgfqpoint{11.143283in}{5.591149in}}{\pgfqpoint{11.138892in}{5.601748in}}{\pgfqpoint{11.131079in}{5.609562in}}%
\pgfpathcurveto{\pgfqpoint{11.123265in}{5.617375in}}{\pgfqpoint{11.112666in}{5.621766in}}{\pgfqpoint{11.101616in}{5.621766in}}%
\pgfpathcurveto{\pgfqpoint{11.090566in}{5.621766in}}{\pgfqpoint{11.079967in}{5.617375in}}{\pgfqpoint{11.072153in}{5.609562in}}%
\pgfpathcurveto{\pgfqpoint{11.064339in}{5.601748in}}{\pgfqpoint{11.059949in}{5.591149in}}{\pgfqpoint{11.059949in}{5.580099in}}%
\pgfpathcurveto{\pgfqpoint{11.059949in}{5.569049in}}{\pgfqpoint{11.064339in}{5.558450in}}{\pgfqpoint{11.072153in}{5.550636in}}%
\pgfpathcurveto{\pgfqpoint{11.079967in}{5.542823in}}{\pgfqpoint{11.090566in}{5.538432in}}{\pgfqpoint{11.101616in}{5.538432in}}%
\pgfpathlineto{\pgfqpoint{11.101616in}{5.538432in}}%
\pgfpathclose%
\pgfusepath{stroke}%
\end{pgfscope}%
\begin{pgfscope}%
\pgfpathrectangle{\pgfqpoint{7.512535in}{0.437222in}}{\pgfqpoint{6.275590in}{5.159444in}}%
\pgfusepath{clip}%
\pgfsetbuttcap%
\pgfsetroundjoin%
\pgfsetlinewidth{1.003750pt}%
\definecolor{currentstroke}{rgb}{1.000000,0.000000,0.000000}%
\pgfsetstrokecolor{currentstroke}%
\pgfsetdash{}{0pt}%
\pgfpathmoveto{\pgfqpoint{8.761066in}{1.995132in}}%
\pgfpathcurveto{\pgfqpoint{8.772116in}{1.995132in}}{\pgfqpoint{8.782715in}{1.999522in}}{\pgfqpoint{8.790529in}{2.007336in}}%
\pgfpathcurveto{\pgfqpoint{8.798342in}{2.015149in}}{\pgfqpoint{8.802732in}{2.025748in}}{\pgfqpoint{8.802732in}{2.036798in}}%
\pgfpathcurveto{\pgfqpoint{8.802732in}{2.047849in}}{\pgfqpoint{8.798342in}{2.058448in}}{\pgfqpoint{8.790529in}{2.066261in}}%
\pgfpathcurveto{\pgfqpoint{8.782715in}{2.074075in}}{\pgfqpoint{8.772116in}{2.078465in}}{\pgfqpoint{8.761066in}{2.078465in}}%
\pgfpathcurveto{\pgfqpoint{8.750016in}{2.078465in}}{\pgfqpoint{8.739417in}{2.074075in}}{\pgfqpoint{8.731603in}{2.066261in}}%
\pgfpathcurveto{\pgfqpoint{8.723789in}{2.058448in}}{\pgfqpoint{8.719399in}{2.047849in}}{\pgfqpoint{8.719399in}{2.036798in}}%
\pgfpathcurveto{\pgfqpoint{8.719399in}{2.025748in}}{\pgfqpoint{8.723789in}{2.015149in}}{\pgfqpoint{8.731603in}{2.007336in}}%
\pgfpathcurveto{\pgfqpoint{8.739417in}{1.999522in}}{\pgfqpoint{8.750016in}{1.995132in}}{\pgfqpoint{8.761066in}{1.995132in}}%
\pgfpathlineto{\pgfqpoint{8.761066in}{1.995132in}}%
\pgfpathclose%
\pgfusepath{stroke}%
\end{pgfscope}%
\begin{pgfscope}%
\pgfpathrectangle{\pgfqpoint{7.512535in}{0.437222in}}{\pgfqpoint{6.275590in}{5.159444in}}%
\pgfusepath{clip}%
\pgfsetbuttcap%
\pgfsetroundjoin%
\pgfsetlinewidth{1.003750pt}%
\definecolor{currentstroke}{rgb}{1.000000,0.000000,0.000000}%
\pgfsetstrokecolor{currentstroke}%
\pgfsetdash{}{0pt}%
\pgfpathmoveto{\pgfqpoint{8.833281in}{2.087187in}}%
\pgfpathcurveto{\pgfqpoint{8.844331in}{2.087187in}}{\pgfqpoint{8.854930in}{2.091578in}}{\pgfqpoint{8.862744in}{2.099391in}}%
\pgfpathcurveto{\pgfqpoint{8.870557in}{2.107205in}}{\pgfqpoint{8.874948in}{2.117804in}}{\pgfqpoint{8.874948in}{2.128854in}}%
\pgfpathcurveto{\pgfqpoint{8.874948in}{2.139904in}}{\pgfqpoint{8.870557in}{2.150503in}}{\pgfqpoint{8.862744in}{2.158317in}}%
\pgfpathcurveto{\pgfqpoint{8.854930in}{2.166130in}}{\pgfqpoint{8.844331in}{2.170521in}}{\pgfqpoint{8.833281in}{2.170521in}}%
\pgfpathcurveto{\pgfqpoint{8.822231in}{2.170521in}}{\pgfqpoint{8.811632in}{2.166130in}}{\pgfqpoint{8.803818in}{2.158317in}}%
\pgfpathcurveto{\pgfqpoint{8.796005in}{2.150503in}}{\pgfqpoint{8.791614in}{2.139904in}}{\pgfqpoint{8.791614in}{2.128854in}}%
\pgfpathcurveto{\pgfqpoint{8.791614in}{2.117804in}}{\pgfqpoint{8.796005in}{2.107205in}}{\pgfqpoint{8.803818in}{2.099391in}}%
\pgfpathcurveto{\pgfqpoint{8.811632in}{2.091578in}}{\pgfqpoint{8.822231in}{2.087187in}}{\pgfqpoint{8.833281in}{2.087187in}}%
\pgfpathlineto{\pgfqpoint{8.833281in}{2.087187in}}%
\pgfpathclose%
\pgfusepath{stroke}%
\end{pgfscope}%
\begin{pgfscope}%
\pgfpathrectangle{\pgfqpoint{7.512535in}{0.437222in}}{\pgfqpoint{6.275590in}{5.159444in}}%
\pgfusepath{clip}%
\pgfsetbuttcap%
\pgfsetroundjoin%
\pgfsetlinewidth{1.003750pt}%
\definecolor{currentstroke}{rgb}{1.000000,0.000000,0.000000}%
\pgfsetstrokecolor{currentstroke}%
\pgfsetdash{}{0pt}%
\pgfpathmoveto{\pgfqpoint{11.116327in}{5.396788in}}%
\pgfpathcurveto{\pgfqpoint{11.127377in}{5.396788in}}{\pgfqpoint{11.137976in}{5.401178in}}{\pgfqpoint{11.145789in}{5.408992in}}%
\pgfpathcurveto{\pgfqpoint{11.153603in}{5.416805in}}{\pgfqpoint{11.157993in}{5.427404in}}{\pgfqpoint{11.157993in}{5.438454in}}%
\pgfpathcurveto{\pgfqpoint{11.157993in}{5.449505in}}{\pgfqpoint{11.153603in}{5.460104in}}{\pgfqpoint{11.145789in}{5.467917in}}%
\pgfpathcurveto{\pgfqpoint{11.137976in}{5.475731in}}{\pgfqpoint{11.127377in}{5.480121in}}{\pgfqpoint{11.116327in}{5.480121in}}%
\pgfpathcurveto{\pgfqpoint{11.105276in}{5.480121in}}{\pgfqpoint{11.094677in}{5.475731in}}{\pgfqpoint{11.086864in}{5.467917in}}%
\pgfpathcurveto{\pgfqpoint{11.079050in}{5.460104in}}{\pgfqpoint{11.074660in}{5.449505in}}{\pgfqpoint{11.074660in}{5.438454in}}%
\pgfpathcurveto{\pgfqpoint{11.074660in}{5.427404in}}{\pgfqpoint{11.079050in}{5.416805in}}{\pgfqpoint{11.086864in}{5.408992in}}%
\pgfpathcurveto{\pgfqpoint{11.094677in}{5.401178in}}{\pgfqpoint{11.105276in}{5.396788in}}{\pgfqpoint{11.116327in}{5.396788in}}%
\pgfpathlineto{\pgfqpoint{11.116327in}{5.396788in}}%
\pgfpathclose%
\pgfusepath{stroke}%
\end{pgfscope}%
\begin{pgfscope}%
\pgfpathrectangle{\pgfqpoint{7.512535in}{0.437222in}}{\pgfqpoint{6.275590in}{5.159444in}}%
\pgfusepath{clip}%
\pgfsetbuttcap%
\pgfsetroundjoin%
\pgfsetlinewidth{1.003750pt}%
\definecolor{currentstroke}{rgb}{1.000000,0.000000,0.000000}%
\pgfsetstrokecolor{currentstroke}%
\pgfsetdash{}{0pt}%
\pgfpathmoveto{\pgfqpoint{8.525858in}{1.480985in}}%
\pgfpathcurveto{\pgfqpoint{8.536908in}{1.480985in}}{\pgfqpoint{8.547507in}{1.485375in}}{\pgfqpoint{8.555321in}{1.493189in}}%
\pgfpathcurveto{\pgfqpoint{8.563135in}{1.501003in}}{\pgfqpoint{8.567525in}{1.511602in}}{\pgfqpoint{8.567525in}{1.522652in}}%
\pgfpathcurveto{\pgfqpoint{8.567525in}{1.533702in}}{\pgfqpoint{8.563135in}{1.544301in}}{\pgfqpoint{8.555321in}{1.552115in}}%
\pgfpathcurveto{\pgfqpoint{8.547507in}{1.559928in}}{\pgfqpoint{8.536908in}{1.564319in}}{\pgfqpoint{8.525858in}{1.564319in}}%
\pgfpathcurveto{\pgfqpoint{8.514808in}{1.564319in}}{\pgfqpoint{8.504209in}{1.559928in}}{\pgfqpoint{8.496395in}{1.552115in}}%
\pgfpathcurveto{\pgfqpoint{8.488582in}{1.544301in}}{\pgfqpoint{8.484192in}{1.533702in}}{\pgfqpoint{8.484192in}{1.522652in}}%
\pgfpathcurveto{\pgfqpoint{8.484192in}{1.511602in}}{\pgfqpoint{8.488582in}{1.501003in}}{\pgfqpoint{8.496395in}{1.493189in}}%
\pgfpathcurveto{\pgfqpoint{8.504209in}{1.485375in}}{\pgfqpoint{8.514808in}{1.480985in}}{\pgfqpoint{8.525858in}{1.480985in}}%
\pgfpathlineto{\pgfqpoint{8.525858in}{1.480985in}}%
\pgfpathclose%
\pgfusepath{stroke}%
\end{pgfscope}%
\begin{pgfscope}%
\pgfpathrectangle{\pgfqpoint{7.512535in}{0.437222in}}{\pgfqpoint{6.275590in}{5.159444in}}%
\pgfusepath{clip}%
\pgfsetbuttcap%
\pgfsetroundjoin%
\pgfsetlinewidth{1.003750pt}%
\definecolor{currentstroke}{rgb}{1.000000,0.000000,0.000000}%
\pgfsetstrokecolor{currentstroke}%
\pgfsetdash{}{0pt}%
\pgfpathmoveto{\pgfqpoint{7.719935in}{0.511926in}}%
\pgfpathcurveto{\pgfqpoint{7.730985in}{0.511926in}}{\pgfqpoint{7.741584in}{0.516316in}}{\pgfqpoint{7.749398in}{0.524130in}}%
\pgfpathcurveto{\pgfqpoint{7.757212in}{0.531943in}}{\pgfqpoint{7.761602in}{0.542542in}}{\pgfqpoint{7.761602in}{0.553592in}}%
\pgfpathcurveto{\pgfqpoint{7.761602in}{0.564643in}}{\pgfqpoint{7.757212in}{0.575242in}}{\pgfqpoint{7.749398in}{0.583055in}}%
\pgfpathcurveto{\pgfqpoint{7.741584in}{0.590869in}}{\pgfqpoint{7.730985in}{0.595259in}}{\pgfqpoint{7.719935in}{0.595259in}}%
\pgfpathcurveto{\pgfqpoint{7.708885in}{0.595259in}}{\pgfqpoint{7.698286in}{0.590869in}}{\pgfqpoint{7.690472in}{0.583055in}}%
\pgfpathcurveto{\pgfqpoint{7.682659in}{0.575242in}}{\pgfqpoint{7.678269in}{0.564643in}}{\pgfqpoint{7.678269in}{0.553592in}}%
\pgfpathcurveto{\pgfqpoint{7.678269in}{0.542542in}}{\pgfqpoint{7.682659in}{0.531943in}}{\pgfqpoint{7.690472in}{0.524130in}}%
\pgfpathcurveto{\pgfqpoint{7.698286in}{0.516316in}}{\pgfqpoint{7.708885in}{0.511926in}}{\pgfqpoint{7.719935in}{0.511926in}}%
\pgfpathlineto{\pgfqpoint{7.719935in}{0.511926in}}%
\pgfpathclose%
\pgfusepath{stroke}%
\end{pgfscope}%
\begin{pgfscope}%
\pgfpathrectangle{\pgfqpoint{7.512535in}{0.437222in}}{\pgfqpoint{6.275590in}{5.159444in}}%
\pgfusepath{clip}%
\pgfsetbuttcap%
\pgfsetroundjoin%
\definecolor{currentfill}{rgb}{0.121569,0.466667,0.705882}%
\pgfsetfillcolor{currentfill}%
\pgfsetlinewidth{1.003750pt}%
\definecolor{currentstroke}{rgb}{0.121569,0.466667,0.705882}%
\pgfsetstrokecolor{currentstroke}%
\pgfsetdash{}{0pt}%
\pgfsys@defobject{currentmarker}{\pgfqpoint{-0.069444in}{-0.069444in}}{\pgfqpoint{0.069444in}{0.069444in}}{%
\pgfpathmoveto{\pgfqpoint{0.000000in}{-0.069444in}}%
\pgfpathcurveto{\pgfqpoint{0.018417in}{-0.069444in}}{\pgfqpoint{0.036082in}{-0.062127in}}{\pgfqpoint{0.049105in}{-0.049105in}}%
\pgfpathcurveto{\pgfqpoint{0.062127in}{-0.036082in}}{\pgfqpoint{0.069444in}{-0.018417in}}{\pgfqpoint{0.069444in}{0.000000in}}%
\pgfpathcurveto{\pgfqpoint{0.069444in}{0.018417in}}{\pgfqpoint{0.062127in}{0.036082in}}{\pgfqpoint{0.049105in}{0.049105in}}%
\pgfpathcurveto{\pgfqpoint{0.036082in}{0.062127in}}{\pgfqpoint{0.018417in}{0.069444in}}{\pgfqpoint{0.000000in}{0.069444in}}%
\pgfpathcurveto{\pgfqpoint{-0.018417in}{0.069444in}}{\pgfqpoint{-0.036082in}{0.062127in}}{\pgfqpoint{-0.049105in}{0.049105in}}%
\pgfpathcurveto{\pgfqpoint{-0.062127in}{0.036082in}}{\pgfqpoint{-0.069444in}{0.018417in}}{\pgfqpoint{-0.069444in}{0.000000in}}%
\pgfpathcurveto{\pgfqpoint{-0.069444in}{-0.018417in}}{\pgfqpoint{-0.062127in}{-0.036082in}}{\pgfqpoint{-0.049105in}{-0.049105in}}%
\pgfpathcurveto{\pgfqpoint{-0.036082in}{-0.062127in}}{\pgfqpoint{-0.018417in}{-0.069444in}}{\pgfqpoint{0.000000in}{-0.069444in}}%
\pgfpathlineto{\pgfqpoint{0.000000in}{-0.069444in}}%
\pgfpathclose%
\pgfusepath{stroke,fill}%
}%
\begin{pgfscope}%
\pgfsys@transformshift{8.409132in}{2.145574in}%
\pgfsys@useobject{currentmarker}{}%
\end{pgfscope}%
\end{pgfscope}%
\begin{pgfscope}%
\pgfpathrectangle{\pgfqpoint{7.512535in}{0.437222in}}{\pgfqpoint{6.275590in}{5.159444in}}%
\pgfusepath{clip}%
\pgfsetbuttcap%
\pgfsetroundjoin%
\definecolor{currentfill}{rgb}{0.172549,0.627451,0.172549}%
\pgfsetfillcolor{currentfill}%
\pgfsetlinewidth{1.003750pt}%
\definecolor{currentstroke}{rgb}{0.172549,0.627451,0.172549}%
\pgfsetstrokecolor{currentstroke}%
\pgfsetdash{}{0pt}%
\pgfsys@defobject{currentmarker}{\pgfqpoint{-0.069444in}{-0.069444in}}{\pgfqpoint{0.069444in}{0.069444in}}{%
\pgfpathmoveto{\pgfqpoint{0.000000in}{-0.069444in}}%
\pgfpathcurveto{\pgfqpoint{0.018417in}{-0.069444in}}{\pgfqpoint{0.036082in}{-0.062127in}}{\pgfqpoint{0.049105in}{-0.049105in}}%
\pgfpathcurveto{\pgfqpoint{0.062127in}{-0.036082in}}{\pgfqpoint{0.069444in}{-0.018417in}}{\pgfqpoint{0.069444in}{0.000000in}}%
\pgfpathcurveto{\pgfqpoint{0.069444in}{0.018417in}}{\pgfqpoint{0.062127in}{0.036082in}}{\pgfqpoint{0.049105in}{0.049105in}}%
\pgfpathcurveto{\pgfqpoint{0.036082in}{0.062127in}}{\pgfqpoint{0.018417in}{0.069444in}}{\pgfqpoint{0.000000in}{0.069444in}}%
\pgfpathcurveto{\pgfqpoint{-0.018417in}{0.069444in}}{\pgfqpoint{-0.036082in}{0.062127in}}{\pgfqpoint{-0.049105in}{0.049105in}}%
\pgfpathcurveto{\pgfqpoint{-0.062127in}{0.036082in}}{\pgfqpoint{-0.069444in}{0.018417in}}{\pgfqpoint{-0.069444in}{0.000000in}}%
\pgfpathcurveto{\pgfqpoint{-0.069444in}{-0.018417in}}{\pgfqpoint{-0.062127in}{-0.036082in}}{\pgfqpoint{-0.049105in}{-0.049105in}}%
\pgfpathcurveto{\pgfqpoint{-0.036082in}{-0.062127in}}{\pgfqpoint{-0.018417in}{-0.069444in}}{\pgfqpoint{0.000000in}{-0.069444in}}%
\pgfpathlineto{\pgfqpoint{0.000000in}{-0.069444in}}%
\pgfpathclose%
\pgfusepath{stroke,fill}%
}%
\begin{pgfscope}%
\pgfsys@transformshift{13.731814in}{5.570410in}%
\pgfsys@useobject{currentmarker}{}%
\end{pgfscope}%
\end{pgfscope}%
\begin{pgfscope}%
\pgfpathrectangle{\pgfqpoint{7.512535in}{0.437222in}}{\pgfqpoint{6.275590in}{5.159444in}}%
\pgfusepath{clip}%
\pgfsetbuttcap%
\pgfsetroundjoin%
\definecolor{currentfill}{rgb}{0.549020,0.337255,0.294118}%
\pgfsetfillcolor{currentfill}%
\pgfsetlinewidth{1.003750pt}%
\definecolor{currentstroke}{rgb}{0.549020,0.337255,0.294118}%
\pgfsetstrokecolor{currentstroke}%
\pgfsetdash{}{0pt}%
\pgfsys@defobject{currentmarker}{\pgfqpoint{-0.069444in}{-0.069444in}}{\pgfqpoint{0.069444in}{0.069444in}}{%
\pgfpathmoveto{\pgfqpoint{0.000000in}{-0.069444in}}%
\pgfpathcurveto{\pgfqpoint{0.018417in}{-0.069444in}}{\pgfqpoint{0.036082in}{-0.062127in}}{\pgfqpoint{0.049105in}{-0.049105in}}%
\pgfpathcurveto{\pgfqpoint{0.062127in}{-0.036082in}}{\pgfqpoint{0.069444in}{-0.018417in}}{\pgfqpoint{0.069444in}{0.000000in}}%
\pgfpathcurveto{\pgfqpoint{0.069444in}{0.018417in}}{\pgfqpoint{0.062127in}{0.036082in}}{\pgfqpoint{0.049105in}{0.049105in}}%
\pgfpathcurveto{\pgfqpoint{0.036082in}{0.062127in}}{\pgfqpoint{0.018417in}{0.069444in}}{\pgfqpoint{0.000000in}{0.069444in}}%
\pgfpathcurveto{\pgfqpoint{-0.018417in}{0.069444in}}{\pgfqpoint{-0.036082in}{0.062127in}}{\pgfqpoint{-0.049105in}{0.049105in}}%
\pgfpathcurveto{\pgfqpoint{-0.062127in}{0.036082in}}{\pgfqpoint{-0.069444in}{0.018417in}}{\pgfqpoint{-0.069444in}{0.000000in}}%
\pgfpathcurveto{\pgfqpoint{-0.069444in}{-0.018417in}}{\pgfqpoint{-0.062127in}{-0.036082in}}{\pgfqpoint{-0.049105in}{-0.049105in}}%
\pgfpathcurveto{\pgfqpoint{-0.036082in}{-0.062127in}}{\pgfqpoint{-0.018417in}{-0.069444in}}{\pgfqpoint{0.000000in}{-0.069444in}}%
\pgfpathlineto{\pgfqpoint{0.000000in}{-0.069444in}}%
\pgfpathclose%
\pgfusepath{stroke,fill}%
}%
\begin{pgfscope}%
\pgfsys@transformshift{7.521084in}{0.600544in}%
\pgfsys@useobject{currentmarker}{}%
\end{pgfscope}%
\end{pgfscope}%
\begin{pgfscope}%
\pgfpathrectangle{\pgfqpoint{7.512535in}{0.437222in}}{\pgfqpoint{6.275590in}{5.159444in}}%
\pgfusepath{clip}%
\pgfsetbuttcap%
\pgfsetroundjoin%
\definecolor{currentfill}{rgb}{0.498039,0.498039,0.498039}%
\pgfsetfillcolor{currentfill}%
\pgfsetlinewidth{1.003750pt}%
\definecolor{currentstroke}{rgb}{0.498039,0.498039,0.498039}%
\pgfsetstrokecolor{currentstroke}%
\pgfsetdash{}{0pt}%
\pgfsys@defobject{currentmarker}{\pgfqpoint{-0.069444in}{-0.069444in}}{\pgfqpoint{0.069444in}{0.069444in}}{%
\pgfpathmoveto{\pgfqpoint{0.000000in}{-0.069444in}}%
\pgfpathcurveto{\pgfqpoint{0.018417in}{-0.069444in}}{\pgfqpoint{0.036082in}{-0.062127in}}{\pgfqpoint{0.049105in}{-0.049105in}}%
\pgfpathcurveto{\pgfqpoint{0.062127in}{-0.036082in}}{\pgfqpoint{0.069444in}{-0.018417in}}{\pgfqpoint{0.069444in}{0.000000in}}%
\pgfpathcurveto{\pgfqpoint{0.069444in}{0.018417in}}{\pgfqpoint{0.062127in}{0.036082in}}{\pgfqpoint{0.049105in}{0.049105in}}%
\pgfpathcurveto{\pgfqpoint{0.036082in}{0.062127in}}{\pgfqpoint{0.018417in}{0.069444in}}{\pgfqpoint{0.000000in}{0.069444in}}%
\pgfpathcurveto{\pgfqpoint{-0.018417in}{0.069444in}}{\pgfqpoint{-0.036082in}{0.062127in}}{\pgfqpoint{-0.049105in}{0.049105in}}%
\pgfpathcurveto{\pgfqpoint{-0.062127in}{0.036082in}}{\pgfqpoint{-0.069444in}{0.018417in}}{\pgfqpoint{-0.069444in}{0.000000in}}%
\pgfpathcurveto{\pgfqpoint{-0.069444in}{-0.018417in}}{\pgfqpoint{-0.062127in}{-0.036082in}}{\pgfqpoint{-0.049105in}{-0.049105in}}%
\pgfpathcurveto{\pgfqpoint{-0.036082in}{-0.062127in}}{\pgfqpoint{-0.018417in}{-0.069444in}}{\pgfqpoint{0.000000in}{-0.069444in}}%
\pgfpathlineto{\pgfqpoint{0.000000in}{-0.069444in}}%
\pgfpathclose%
\pgfusepath{stroke,fill}%
}%
\begin{pgfscope}%
\pgfsys@transformshift{13.786989in}{5.163379in}%
\pgfsys@useobject{currentmarker}{}%
\end{pgfscope}%
\end{pgfscope}%
\begin{pgfscope}%
\pgfpathrectangle{\pgfqpoint{7.512535in}{0.437222in}}{\pgfqpoint{6.275590in}{5.159444in}}%
\pgfusepath{clip}%
\pgfsetbuttcap%
\pgfsetroundjoin%
\definecolor{currentfill}{rgb}{0.090196,0.745098,0.811765}%
\pgfsetfillcolor{currentfill}%
\pgfsetlinewidth{1.003750pt}%
\definecolor{currentstroke}{rgb}{0.090196,0.745098,0.811765}%
\pgfsetstrokecolor{currentstroke}%
\pgfsetdash{}{0pt}%
\pgfsys@defobject{currentmarker}{\pgfqpoint{-0.069444in}{-0.069444in}}{\pgfqpoint{0.069444in}{0.069444in}}{%
\pgfpathmoveto{\pgfqpoint{0.000000in}{-0.069444in}}%
\pgfpathcurveto{\pgfqpoint{0.018417in}{-0.069444in}}{\pgfqpoint{0.036082in}{-0.062127in}}{\pgfqpoint{0.049105in}{-0.049105in}}%
\pgfpathcurveto{\pgfqpoint{0.062127in}{-0.036082in}}{\pgfqpoint{0.069444in}{-0.018417in}}{\pgfqpoint{0.069444in}{0.000000in}}%
\pgfpathcurveto{\pgfqpoint{0.069444in}{0.018417in}}{\pgfqpoint{0.062127in}{0.036082in}}{\pgfqpoint{0.049105in}{0.049105in}}%
\pgfpathcurveto{\pgfqpoint{0.036082in}{0.062127in}}{\pgfqpoint{0.018417in}{0.069444in}}{\pgfqpoint{0.000000in}{0.069444in}}%
\pgfpathcurveto{\pgfqpoint{-0.018417in}{0.069444in}}{\pgfqpoint{-0.036082in}{0.062127in}}{\pgfqpoint{-0.049105in}{0.049105in}}%
\pgfpathcurveto{\pgfqpoint{-0.062127in}{0.036082in}}{\pgfqpoint{-0.069444in}{0.018417in}}{\pgfqpoint{-0.069444in}{0.000000in}}%
\pgfpathcurveto{\pgfqpoint{-0.069444in}{-0.018417in}}{\pgfqpoint{-0.062127in}{-0.036082in}}{\pgfqpoint{-0.049105in}{-0.049105in}}%
\pgfpathcurveto{\pgfqpoint{-0.036082in}{-0.062127in}}{\pgfqpoint{-0.018417in}{-0.069444in}}{\pgfqpoint{0.000000in}{-0.069444in}}%
\pgfpathlineto{\pgfqpoint{0.000000in}{-0.069444in}}%
\pgfpathclose%
\pgfusepath{stroke,fill}%
}%
\begin{pgfscope}%
\pgfsys@transformshift{7.809717in}{3.919233in}%
\pgfsys@useobject{currentmarker}{}%
\end{pgfscope}%
\end{pgfscope}%
\begin{pgfscope}%
\pgfpathrectangle{\pgfqpoint{7.512535in}{0.437222in}}{\pgfqpoint{6.275590in}{5.159444in}}%
\pgfusepath{clip}%
\pgfsetroundcap%
\pgfsetroundjoin%
\definecolor{currentfill}{rgb}{0.172549,0.627451,0.172549}%
\pgfsetfillcolor{currentfill}%
\pgfsetlinewidth{3.011250pt}%
\definecolor{currentstroke}{rgb}{0.172549,0.627451,0.172549}%
\pgfsetstrokecolor{currentstroke}%
\pgfsetdash{}{0pt}%
\pgfpathmoveto{\pgfqpoint{8.430603in}{2.163518in}}%
\pgfpathquadraticcurveto{\pgfqpoint{11.021892in}{3.830861in}}{\pgfqpoint{13.613181in}{5.498204in}}%
\pgfpathlineto{\pgfqpoint{13.600029in}{5.518644in}}%
\pgfpathquadraticcurveto{\pgfqpoint{13.654257in}{5.537022in}}{\pgfqpoint{13.708486in}{5.555399in}}%
\pgfpathquadraticcurveto{\pgfqpoint{13.669288in}{5.513662in}}{\pgfqpoint{13.630090in}{5.471925in}}%
\pgfpathlineto{\pgfqpoint{13.616938in}{5.492364in}}%
\pgfpathquadraticcurveto{\pgfqpoint{11.025650in}{3.825021in}}{\pgfqpoint{8.434361in}{2.157678in}}%
\pgfpathlineto{\pgfqpoint{8.430603in}{2.163518in}}%
\pgfpathlineto{\pgfqpoint{8.430603in}{2.163518in}}%
\pgfpathclose%
\pgfusepath{stroke,fill}%
\end{pgfscope}%
\begin{pgfscope}%
\pgfpathrectangle{\pgfqpoint{7.512535in}{0.437222in}}{\pgfqpoint{6.275590in}{5.159444in}}%
\pgfusepath{clip}%
\pgfsetroundcap%
\pgfsetroundjoin%
\definecolor{currentfill}{rgb}{0.549020,0.337255,0.294118}%
\pgfsetfillcolor{currentfill}%
\pgfsetlinewidth{3.011250pt}%
\definecolor{currentstroke}{rgb}{0.549020,0.337255,0.294118}%
\pgfsetstrokecolor{currentstroke}%
\pgfsetdash{}{0pt}%
\pgfpathmoveto{\pgfqpoint{13.712306in}{5.550351in}}%
\pgfpathquadraticcurveto{\pgfqpoint{10.671999in}{3.117479in}}{\pgfqpoint{7.631693in}{0.684607in}}%
\pgfpathlineto{\pgfqpoint{7.646879in}{0.665630in}}%
\pgfpathquadraticcurveto{\pgfqpoint{7.594830in}{0.641768in}}{\pgfqpoint{7.542781in}{0.617906in}}%
\pgfpathquadraticcurveto{\pgfqpoint{7.577475in}{0.663457in}}{\pgfqpoint{7.612168in}{0.709007in}}%
\pgfpathlineto{\pgfqpoint{7.627354in}{0.690029in}}%
\pgfpathquadraticcurveto{\pgfqpoint{10.667661in}{3.122902in}}{\pgfqpoint{13.707967in}{5.555774in}}%
\pgfpathlineto{\pgfqpoint{13.712306in}{5.550351in}}%
\pgfpathlineto{\pgfqpoint{13.712306in}{5.550351in}}%
\pgfpathclose%
\pgfusepath{stroke,fill}%
\end{pgfscope}%
\begin{pgfscope}%
\pgfpathrectangle{\pgfqpoint{7.512535in}{0.437222in}}{\pgfqpoint{6.275590in}{5.159444in}}%
\pgfusepath{clip}%
\pgfsetroundcap%
\pgfsetroundjoin%
\definecolor{currentfill}{rgb}{0.498039,0.498039,0.498039}%
\pgfsetfillcolor{currentfill}%
\pgfsetlinewidth{3.011250pt}%
\definecolor{currentstroke}{rgb}{0.498039,0.498039,0.498039}%
\pgfsetstrokecolor{currentstroke}%
\pgfsetdash{}{0pt}%
\pgfpathmoveto{\pgfqpoint{7.541485in}{0.619695in}}%
\pgfpathquadraticcurveto{\pgfqpoint{10.607074in}{2.852059in}}{\pgfqpoint{13.672663in}{5.084422in}}%
\pgfpathlineto{\pgfqpoint{13.658355in}{5.104070in}}%
\pgfpathquadraticcurveto{\pgfqpoint{13.711442in}{5.125547in}}{\pgfqpoint{13.764530in}{5.147024in}}%
\pgfpathquadraticcurveto{\pgfqpoint{13.727794in}{5.103092in}}{\pgfqpoint{13.691059in}{5.059160in}}%
\pgfpathlineto{\pgfqpoint{13.676751in}{5.078809in}}%
\pgfpathquadraticcurveto{\pgfqpoint{10.611162in}{2.846445in}}{\pgfqpoint{7.545573in}{0.614082in}}%
\pgfpathlineto{\pgfqpoint{7.541485in}{0.619695in}}%
\pgfpathlineto{\pgfqpoint{7.541485in}{0.619695in}}%
\pgfpathclose%
\pgfusepath{stroke,fill}%
\end{pgfscope}%
\begin{pgfscope}%
\pgfpathrectangle{\pgfqpoint{7.512535in}{0.437222in}}{\pgfqpoint{6.275590in}{5.159444in}}%
\pgfusepath{clip}%
\pgfsetroundcap%
\pgfsetroundjoin%
\definecolor{currentfill}{rgb}{0.090196,0.745098,0.811765}%
\pgfsetfillcolor{currentfill}%
\pgfsetlinewidth{3.011250pt}%
\definecolor{currentstroke}{rgb}{0.090196,0.745098,0.811765}%
\pgfsetstrokecolor{currentstroke}%
\pgfsetdash{}{0pt}%
\pgfpathmoveto{\pgfqpoint{13.760472in}{5.154313in}}%
\pgfpathquadraticcurveto{\pgfqpoint{10.853427in}{4.549223in}}{\pgfqpoint{7.946383in}{3.944133in}}%
\pgfpathlineto{\pgfqpoint{7.951336in}{3.920337in}}%
\pgfpathquadraticcurveto{\pgfqpoint{7.894123in}{3.922615in}}{\pgfqpoint{7.836909in}{3.924893in}}%
\pgfpathquadraticcurveto{\pgfqpoint{7.888462in}{3.949810in}}{\pgfqpoint{7.940015in}{3.974727in}}%
\pgfpathlineto{\pgfqpoint{7.944968in}{3.950932in}}%
\pgfpathquadraticcurveto{\pgfqpoint{10.852012in}{4.556022in}}{\pgfqpoint{13.759056in}{5.161112in}}%
\pgfpathlineto{\pgfqpoint{13.760472in}{5.154313in}}%
\pgfpathlineto{\pgfqpoint{13.760472in}{5.154313in}}%
\pgfpathclose%
\pgfusepath{stroke,fill}%
\end{pgfscope}%
\begin{pgfscope}%
\pgfsetbuttcap%
\pgfsetroundjoin%
\definecolor{currentfill}{rgb}{0.000000,0.000000,0.000000}%
\pgfsetfillcolor{currentfill}%
\pgfsetlinewidth{0.803000pt}%
\definecolor{currentstroke}{rgb}{0.000000,0.000000,0.000000}%
\pgfsetstrokecolor{currentstroke}%
\pgfsetdash{}{0pt}%
\pgfsys@defobject{currentmarker}{\pgfqpoint{0.000000in}{-0.048611in}}{\pgfqpoint{0.000000in}{0.000000in}}{%
\pgfpathmoveto{\pgfqpoint{0.000000in}{0.000000in}}%
\pgfpathlineto{\pgfqpoint{0.000000in}{-0.048611in}}%
\pgfusepath{stroke,fill}%
}%
\begin{pgfscope}%
\pgfsys@transformshift{7.512535in}{0.437222in}%
\pgfsys@useobject{currentmarker}{}%
\end{pgfscope}%
\end{pgfscope}%
\begin{pgfscope}%
\definecolor{textcolor}{rgb}{0.000000,0.000000,0.000000}%
\pgfsetstrokecolor{textcolor}%
\pgfsetfillcolor{textcolor}%
\pgftext[x=7.512535in,y=0.340000in,,top]{\color{textcolor}{\sffamily\fontsize{14.000000}{16.800000}\selectfont\catcode`\^=\active\def^{\ifmmode\sp\else\^{}\fi}\catcode`\%=\active\def%{\%}0}}%
\end{pgfscope}%
\begin{pgfscope}%
\pgfsetbuttcap%
\pgfsetroundjoin%
\definecolor{currentfill}{rgb}{0.000000,0.000000,0.000000}%
\pgfsetfillcolor{currentfill}%
\pgfsetlinewidth{0.803000pt}%
\definecolor{currentstroke}{rgb}{0.000000,0.000000,0.000000}%
\pgfsetstrokecolor{currentstroke}%
\pgfsetdash{}{0pt}%
\pgfsys@defobject{currentmarker}{\pgfqpoint{0.000000in}{-0.048611in}}{\pgfqpoint{0.000000in}{0.000000in}}{%
\pgfpathmoveto{\pgfqpoint{0.000000in}{0.000000in}}%
\pgfpathlineto{\pgfqpoint{0.000000in}{-0.048611in}}%
\pgfusepath{stroke,fill}%
}%
\begin{pgfscope}%
\pgfsys@transformshift{8.767653in}{0.437222in}%
\pgfsys@useobject{currentmarker}{}%
\end{pgfscope}%
\end{pgfscope}%
\begin{pgfscope}%
\definecolor{textcolor}{rgb}{0.000000,0.000000,0.000000}%
\pgfsetstrokecolor{textcolor}%
\pgfsetfillcolor{textcolor}%
\pgftext[x=8.767653in,y=0.340000in,,top]{\color{textcolor}{\sffamily\fontsize{14.000000}{16.800000}\selectfont\catcode`\^=\active\def^{\ifmmode\sp\else\^{}\fi}\catcode`\%=\active\def%{\%}1}}%
\end{pgfscope}%
\begin{pgfscope}%
\pgfsetbuttcap%
\pgfsetroundjoin%
\definecolor{currentfill}{rgb}{0.000000,0.000000,0.000000}%
\pgfsetfillcolor{currentfill}%
\pgfsetlinewidth{0.803000pt}%
\definecolor{currentstroke}{rgb}{0.000000,0.000000,0.000000}%
\pgfsetstrokecolor{currentstroke}%
\pgfsetdash{}{0pt}%
\pgfsys@defobject{currentmarker}{\pgfqpoint{0.000000in}{-0.048611in}}{\pgfqpoint{0.000000in}{0.000000in}}{%
\pgfpathmoveto{\pgfqpoint{0.000000in}{0.000000in}}%
\pgfpathlineto{\pgfqpoint{0.000000in}{-0.048611in}}%
\pgfusepath{stroke,fill}%
}%
\begin{pgfscope}%
\pgfsys@transformshift{10.022771in}{0.437222in}%
\pgfsys@useobject{currentmarker}{}%
\end{pgfscope}%
\end{pgfscope}%
\begin{pgfscope}%
\definecolor{textcolor}{rgb}{0.000000,0.000000,0.000000}%
\pgfsetstrokecolor{textcolor}%
\pgfsetfillcolor{textcolor}%
\pgftext[x=10.022771in,y=0.340000in,,top]{\color{textcolor}{\sffamily\fontsize{14.000000}{16.800000}\selectfont\catcode`\^=\active\def^{\ifmmode\sp\else\^{}\fi}\catcode`\%=\active\def%{\%}2}}%
\end{pgfscope}%
\begin{pgfscope}%
\pgfsetbuttcap%
\pgfsetroundjoin%
\definecolor{currentfill}{rgb}{0.000000,0.000000,0.000000}%
\pgfsetfillcolor{currentfill}%
\pgfsetlinewidth{0.803000pt}%
\definecolor{currentstroke}{rgb}{0.000000,0.000000,0.000000}%
\pgfsetstrokecolor{currentstroke}%
\pgfsetdash{}{0pt}%
\pgfsys@defobject{currentmarker}{\pgfqpoint{0.000000in}{-0.048611in}}{\pgfqpoint{0.000000in}{0.000000in}}{%
\pgfpathmoveto{\pgfqpoint{0.000000in}{0.000000in}}%
\pgfpathlineto{\pgfqpoint{0.000000in}{-0.048611in}}%
\pgfusepath{stroke,fill}%
}%
\begin{pgfscope}%
\pgfsys@transformshift{11.277889in}{0.437222in}%
\pgfsys@useobject{currentmarker}{}%
\end{pgfscope}%
\end{pgfscope}%
\begin{pgfscope}%
\definecolor{textcolor}{rgb}{0.000000,0.000000,0.000000}%
\pgfsetstrokecolor{textcolor}%
\pgfsetfillcolor{textcolor}%
\pgftext[x=11.277889in,y=0.340000in,,top]{\color{textcolor}{\sffamily\fontsize{14.000000}{16.800000}\selectfont\catcode`\^=\active\def^{\ifmmode\sp\else\^{}\fi}\catcode`\%=\active\def%{\%}3}}%
\end{pgfscope}%
\begin{pgfscope}%
\pgfsetbuttcap%
\pgfsetroundjoin%
\definecolor{currentfill}{rgb}{0.000000,0.000000,0.000000}%
\pgfsetfillcolor{currentfill}%
\pgfsetlinewidth{0.803000pt}%
\definecolor{currentstroke}{rgb}{0.000000,0.000000,0.000000}%
\pgfsetstrokecolor{currentstroke}%
\pgfsetdash{}{0pt}%
\pgfsys@defobject{currentmarker}{\pgfqpoint{0.000000in}{-0.048611in}}{\pgfqpoint{0.000000in}{0.000000in}}{%
\pgfpathmoveto{\pgfqpoint{0.000000in}{0.000000in}}%
\pgfpathlineto{\pgfqpoint{0.000000in}{-0.048611in}}%
\pgfusepath{stroke,fill}%
}%
\begin{pgfscope}%
\pgfsys@transformshift{12.533007in}{0.437222in}%
\pgfsys@useobject{currentmarker}{}%
\end{pgfscope}%
\end{pgfscope}%
\begin{pgfscope}%
\definecolor{textcolor}{rgb}{0.000000,0.000000,0.000000}%
\pgfsetstrokecolor{textcolor}%
\pgfsetfillcolor{textcolor}%
\pgftext[x=12.533007in,y=0.340000in,,top]{\color{textcolor}{\sffamily\fontsize{14.000000}{16.800000}\selectfont\catcode`\^=\active\def^{\ifmmode\sp\else\^{}\fi}\catcode`\%=\active\def%{\%}4}}%
\end{pgfscope}%
\begin{pgfscope}%
\pgfsetbuttcap%
\pgfsetroundjoin%
\definecolor{currentfill}{rgb}{0.000000,0.000000,0.000000}%
\pgfsetfillcolor{currentfill}%
\pgfsetlinewidth{0.803000pt}%
\definecolor{currentstroke}{rgb}{0.000000,0.000000,0.000000}%
\pgfsetstrokecolor{currentstroke}%
\pgfsetdash{}{0pt}%
\pgfsys@defobject{currentmarker}{\pgfqpoint{0.000000in}{-0.048611in}}{\pgfqpoint{0.000000in}{0.000000in}}{%
\pgfpathmoveto{\pgfqpoint{0.000000in}{0.000000in}}%
\pgfpathlineto{\pgfqpoint{0.000000in}{-0.048611in}}%
\pgfusepath{stroke,fill}%
}%
\begin{pgfscope}%
\pgfsys@transformshift{13.788125in}{0.437222in}%
\pgfsys@useobject{currentmarker}{}%
\end{pgfscope}%
\end{pgfscope}%
\begin{pgfscope}%
\definecolor{textcolor}{rgb}{0.000000,0.000000,0.000000}%
\pgfsetstrokecolor{textcolor}%
\pgfsetfillcolor{textcolor}%
\pgftext[x=13.788125in,y=0.340000in,,top]{\color{textcolor}{\sffamily\fontsize{14.000000}{16.800000}\selectfont\catcode`\^=\active\def^{\ifmmode\sp\else\^{}\fi}\catcode`\%=\active\def%{\%}5}}%
\end{pgfscope}%
\begin{pgfscope}%
\pgfsetbuttcap%
\pgfsetroundjoin%
\definecolor{currentfill}{rgb}{0.000000,0.000000,0.000000}%
\pgfsetfillcolor{currentfill}%
\pgfsetlinewidth{0.803000pt}%
\definecolor{currentstroke}{rgb}{0.000000,0.000000,0.000000}%
\pgfsetstrokecolor{currentstroke}%
\pgfsetdash{}{0pt}%
\pgfsys@defobject{currentmarker}{\pgfqpoint{-0.048611in}{0.000000in}}{\pgfqpoint{-0.000000in}{0.000000in}}{%
\pgfpathmoveto{\pgfqpoint{-0.000000in}{0.000000in}}%
\pgfpathlineto{\pgfqpoint{-0.048611in}{0.000000in}}%
\pgfusepath{stroke,fill}%
}%
\begin{pgfscope}%
\pgfsys@transformshift{7.512535in}{0.437222in}%
\pgfsys@useobject{currentmarker}{}%
\end{pgfscope}%
\end{pgfscope}%
\begin{pgfscope}%
\definecolor{textcolor}{rgb}{0.000000,0.000000,0.000000}%
\pgfsetstrokecolor{textcolor}%
\pgfsetfillcolor{textcolor}%
\pgftext[x=7.106081in, y=0.363356in, left, base]{\color{textcolor}{\sffamily\fontsize{14.000000}{16.800000}\selectfont\catcode`\^=\active\def^{\ifmmode\sp\else\^{}\fi}\catcode`\%=\active\def%{\%}0.0}}%
\end{pgfscope}%
\begin{pgfscope}%
\pgfsetbuttcap%
\pgfsetroundjoin%
\definecolor{currentfill}{rgb}{0.000000,0.000000,0.000000}%
\pgfsetfillcolor{currentfill}%
\pgfsetlinewidth{0.803000pt}%
\definecolor{currentstroke}{rgb}{0.000000,0.000000,0.000000}%
\pgfsetstrokecolor{currentstroke}%
\pgfsetdash{}{0pt}%
\pgfsys@defobject{currentmarker}{\pgfqpoint{-0.048611in}{0.000000in}}{\pgfqpoint{-0.000000in}{0.000000in}}{%
\pgfpathmoveto{\pgfqpoint{-0.000000in}{0.000000in}}%
\pgfpathlineto{\pgfqpoint{-0.048611in}{0.000000in}}%
\pgfusepath{stroke,fill}%
}%
\begin{pgfscope}%
\pgfsys@transformshift{7.512535in}{1.297130in}%
\pgfsys@useobject{currentmarker}{}%
\end{pgfscope}%
\end{pgfscope}%
\begin{pgfscope}%
\definecolor{textcolor}{rgb}{0.000000,0.000000,0.000000}%
\pgfsetstrokecolor{textcolor}%
\pgfsetfillcolor{textcolor}%
\pgftext[x=7.106081in, y=1.223264in, left, base]{\color{textcolor}{\sffamily\fontsize{14.000000}{16.800000}\selectfont\catcode`\^=\active\def^{\ifmmode\sp\else\^{}\fi}\catcode`\%=\active\def%{\%}0.5}}%
\end{pgfscope}%
\begin{pgfscope}%
\pgfsetbuttcap%
\pgfsetroundjoin%
\definecolor{currentfill}{rgb}{0.000000,0.000000,0.000000}%
\pgfsetfillcolor{currentfill}%
\pgfsetlinewidth{0.803000pt}%
\definecolor{currentstroke}{rgb}{0.000000,0.000000,0.000000}%
\pgfsetstrokecolor{currentstroke}%
\pgfsetdash{}{0pt}%
\pgfsys@defobject{currentmarker}{\pgfqpoint{-0.048611in}{0.000000in}}{\pgfqpoint{-0.000000in}{0.000000in}}{%
\pgfpathmoveto{\pgfqpoint{-0.000000in}{0.000000in}}%
\pgfpathlineto{\pgfqpoint{-0.048611in}{0.000000in}}%
\pgfusepath{stroke,fill}%
}%
\begin{pgfscope}%
\pgfsys@transformshift{7.512535in}{2.157037in}%
\pgfsys@useobject{currentmarker}{}%
\end{pgfscope}%
\end{pgfscope}%
\begin{pgfscope}%
\definecolor{textcolor}{rgb}{0.000000,0.000000,0.000000}%
\pgfsetstrokecolor{textcolor}%
\pgfsetfillcolor{textcolor}%
\pgftext[x=7.106081in, y=2.083171in, left, base]{\color{textcolor}{\sffamily\fontsize{14.000000}{16.800000}\selectfont\catcode`\^=\active\def^{\ifmmode\sp\else\^{}\fi}\catcode`\%=\active\def%{\%}1.0}}%
\end{pgfscope}%
\begin{pgfscope}%
\pgfsetbuttcap%
\pgfsetroundjoin%
\definecolor{currentfill}{rgb}{0.000000,0.000000,0.000000}%
\pgfsetfillcolor{currentfill}%
\pgfsetlinewidth{0.803000pt}%
\definecolor{currentstroke}{rgb}{0.000000,0.000000,0.000000}%
\pgfsetstrokecolor{currentstroke}%
\pgfsetdash{}{0pt}%
\pgfsys@defobject{currentmarker}{\pgfqpoint{-0.048611in}{0.000000in}}{\pgfqpoint{-0.000000in}{0.000000in}}{%
\pgfpathmoveto{\pgfqpoint{-0.000000in}{0.000000in}}%
\pgfpathlineto{\pgfqpoint{-0.048611in}{0.000000in}}%
\pgfusepath{stroke,fill}%
}%
\begin{pgfscope}%
\pgfsys@transformshift{7.512535in}{3.016944in}%
\pgfsys@useobject{currentmarker}{}%
\end{pgfscope}%
\end{pgfscope}%
\begin{pgfscope}%
\definecolor{textcolor}{rgb}{0.000000,0.000000,0.000000}%
\pgfsetstrokecolor{textcolor}%
\pgfsetfillcolor{textcolor}%
\pgftext[x=7.106081in, y=2.943078in, left, base]{\color{textcolor}{\sffamily\fontsize{14.000000}{16.800000}\selectfont\catcode`\^=\active\def^{\ifmmode\sp\else\^{}\fi}\catcode`\%=\active\def%{\%}1.5}}%
\end{pgfscope}%
\begin{pgfscope}%
\pgfsetbuttcap%
\pgfsetroundjoin%
\definecolor{currentfill}{rgb}{0.000000,0.000000,0.000000}%
\pgfsetfillcolor{currentfill}%
\pgfsetlinewidth{0.803000pt}%
\definecolor{currentstroke}{rgb}{0.000000,0.000000,0.000000}%
\pgfsetstrokecolor{currentstroke}%
\pgfsetdash{}{0pt}%
\pgfsys@defobject{currentmarker}{\pgfqpoint{-0.048611in}{0.000000in}}{\pgfqpoint{-0.000000in}{0.000000in}}{%
\pgfpathmoveto{\pgfqpoint{-0.000000in}{0.000000in}}%
\pgfpathlineto{\pgfqpoint{-0.048611in}{0.000000in}}%
\pgfusepath{stroke,fill}%
}%
\begin{pgfscope}%
\pgfsys@transformshift{7.512535in}{3.876852in}%
\pgfsys@useobject{currentmarker}{}%
\end{pgfscope}%
\end{pgfscope}%
\begin{pgfscope}%
\definecolor{textcolor}{rgb}{0.000000,0.000000,0.000000}%
\pgfsetstrokecolor{textcolor}%
\pgfsetfillcolor{textcolor}%
\pgftext[x=7.106081in, y=3.802986in, left, base]{\color{textcolor}{\sffamily\fontsize{14.000000}{16.800000}\selectfont\catcode`\^=\active\def^{\ifmmode\sp\else\^{}\fi}\catcode`\%=\active\def%{\%}2.0}}%
\end{pgfscope}%
\begin{pgfscope}%
\pgfsetbuttcap%
\pgfsetroundjoin%
\definecolor{currentfill}{rgb}{0.000000,0.000000,0.000000}%
\pgfsetfillcolor{currentfill}%
\pgfsetlinewidth{0.803000pt}%
\definecolor{currentstroke}{rgb}{0.000000,0.000000,0.000000}%
\pgfsetstrokecolor{currentstroke}%
\pgfsetdash{}{0pt}%
\pgfsys@defobject{currentmarker}{\pgfqpoint{-0.048611in}{0.000000in}}{\pgfqpoint{-0.000000in}{0.000000in}}{%
\pgfpathmoveto{\pgfqpoint{-0.000000in}{0.000000in}}%
\pgfpathlineto{\pgfqpoint{-0.048611in}{0.000000in}}%
\pgfusepath{stroke,fill}%
}%
\begin{pgfscope}%
\pgfsys@transformshift{7.512535in}{4.736759in}%
\pgfsys@useobject{currentmarker}{}%
\end{pgfscope}%
\end{pgfscope}%
\begin{pgfscope}%
\definecolor{textcolor}{rgb}{0.000000,0.000000,0.000000}%
\pgfsetstrokecolor{textcolor}%
\pgfsetfillcolor{textcolor}%
\pgftext[x=7.106081in, y=4.662893in, left, base]{\color{textcolor}{\sffamily\fontsize{14.000000}{16.800000}\selectfont\catcode`\^=\active\def^{\ifmmode\sp\else\^{}\fi}\catcode`\%=\active\def%{\%}2.5}}%
\end{pgfscope}%
\begin{pgfscope}%
\pgfsetbuttcap%
\pgfsetroundjoin%
\definecolor{currentfill}{rgb}{0.000000,0.000000,0.000000}%
\pgfsetfillcolor{currentfill}%
\pgfsetlinewidth{0.803000pt}%
\definecolor{currentstroke}{rgb}{0.000000,0.000000,0.000000}%
\pgfsetstrokecolor{currentstroke}%
\pgfsetdash{}{0pt}%
\pgfsys@defobject{currentmarker}{\pgfqpoint{-0.048611in}{0.000000in}}{\pgfqpoint{-0.000000in}{0.000000in}}{%
\pgfpathmoveto{\pgfqpoint{-0.000000in}{0.000000in}}%
\pgfpathlineto{\pgfqpoint{-0.048611in}{0.000000in}}%
\pgfusepath{stroke,fill}%
}%
\begin{pgfscope}%
\pgfsys@transformshift{7.512535in}{5.596667in}%
\pgfsys@useobject{currentmarker}{}%
\end{pgfscope}%
\end{pgfscope}%
\begin{pgfscope}%
\definecolor{textcolor}{rgb}{0.000000,0.000000,0.000000}%
\pgfsetstrokecolor{textcolor}%
\pgfsetfillcolor{textcolor}%
\pgftext[x=7.106081in, y=5.522801in, left, base]{\color{textcolor}{\sffamily\fontsize{14.000000}{16.800000}\selectfont\catcode`\^=\active\def^{\ifmmode\sp\else\^{}\fi}\catcode`\%=\active\def%{\%}3.0}}%
\end{pgfscope}%
\begin{pgfscope}%
\pgfsetrectcap%
\pgfsetmiterjoin%
\pgfsetlinewidth{0.803000pt}%
\definecolor{currentstroke}{rgb}{0.000000,0.000000,0.000000}%
\pgfsetstrokecolor{currentstroke}%
\pgfsetdash{}{0pt}%
\pgfpathmoveto{\pgfqpoint{7.512535in}{0.437222in}}%
\pgfpathlineto{\pgfqpoint{7.512535in}{5.596667in}}%
\pgfusepath{stroke}%
\end{pgfscope}%
\begin{pgfscope}%
\pgfsetrectcap%
\pgfsetmiterjoin%
\pgfsetlinewidth{0.803000pt}%
\definecolor{currentstroke}{rgb}{0.000000,0.000000,0.000000}%
\pgfsetstrokecolor{currentstroke}%
\pgfsetdash{}{0pt}%
\pgfpathmoveto{\pgfqpoint{13.788125in}{0.437222in}}%
\pgfpathlineto{\pgfqpoint{13.788125in}{5.596667in}}%
\pgfusepath{stroke}%
\end{pgfscope}%
\begin{pgfscope}%
\pgfsetrectcap%
\pgfsetmiterjoin%
\pgfsetlinewidth{0.803000pt}%
\definecolor{currentstroke}{rgb}{0.000000,0.000000,0.000000}%
\pgfsetstrokecolor{currentstroke}%
\pgfsetdash{}{0pt}%
\pgfpathmoveto{\pgfqpoint{7.512535in}{0.437222in}}%
\pgfpathlineto{\pgfqpoint{13.788125in}{0.437222in}}%
\pgfusepath{stroke}%
\end{pgfscope}%
\begin{pgfscope}%
\pgfsetrectcap%
\pgfsetmiterjoin%
\pgfsetlinewidth{0.803000pt}%
\definecolor{currentstroke}{rgb}{0.000000,0.000000,0.000000}%
\pgfsetstrokecolor{currentstroke}%
\pgfsetdash{}{0pt}%
\pgfpathmoveto{\pgfqpoint{7.512535in}{5.596667in}}%
\pgfpathlineto{\pgfqpoint{13.788125in}{5.596667in}}%
\pgfusepath{stroke}%
\end{pgfscope}%
\begin{pgfscope}%
\definecolor{textcolor}{rgb}{0.000000,0.000000,0.000000}%
\pgfsetstrokecolor{textcolor}%
\pgfsetfillcolor{textcolor}%
\pgftext[x=10.650330in,y=5.680000in,,base]{\color{textcolor}{\sffamily\fontsize{16.000000}{19.200000}\selectfont\catcode`\^=\active\def^{\ifmmode\sp\else\^{}\fi}\catcode`\%=\active\def%{\%}Design Space}}%
\end{pgfscope}%
\end{pgfpicture}%
\makeatother%
\endgroup%
}
  \caption{Demonstrates the ``farthest-first-traversal'' algorithm with an example problem.}
  \label{fig:mga-fft}
\end{figure}