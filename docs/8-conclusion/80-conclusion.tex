Climate change must be solved to ensure the Earth remains habitable for future
generations. Importantly, the effects of climate change are not felt uniformly
by everyone. Instead, its negative impacts are disproportionately felt by the
most vulnerable members of society and by historically marginalized communities.
Addressing climate change requires a radical change in the ways we produce
energy by transitioning from fossil fuels to clean energy. Energy system
optimization models (ESOMs) are essential tools for facilitating the transition
from a fossil-fuel based economy to a clean energy economy. However, current
\acp{esom} have several limitations that inhibit meaningful dialogue about the
best path forward. This thesis's central goal was to attend to these
limitations.

Chapter \ref{chapter:lit-review} characterized the problem of climate change as
a wicked problem with disproportionate impacts, introduced energy justice, and
defined energy systems as a mechanism for balancing the supply and demand of
energy resources that simultaneously mediates sociopolitical power.
Additionally, Chapter \ref{chapter:lit-review} identified a few gaps in the
energy modeling literature. 
\begin{enumerate}
    \item The few \acp{esom} that address structural uncertainty, the only
    method used to probe this uncertainty is \acf{mga}. Yet other methods exist,
    such as \acf{moo}.
    \item There are no existing \ac{esom} frameworks that use \ac{moo} to
    generate solutions.
    \item The studies that used \ac{moo} modeled at most three objectives and do
    not demonstrate the potential to optimize many objectives, nor allow users
    to define new objectives.
    \item \ac{esom} studies seldom engage with energy justice, and those that do
    focus on distributive justice due to its quantifiability. This means that
    the role of \acp{esom} in attending to procedural and recognition justice is
    unclear.
\end{enumerate}
These gaps informed the two primary objectives of this thesis, which were to
\begin{enumerate}
    \item Develop an \ac{esom} framework that uses \acl{moo} to solve many
    objectives and user-defined objectives.
    \item Validate the usefulness of the tool developed in this thesis for
    policymaking applications by interviewing energy planners throughout
    Illinois.
\end{enumerate}

Chapter \ref{chapter:osier} introduced the first multi-objective \ac{esom},
\ac{osier}. \ac{osier} fulfills the first goal of this thesis and closes some
gaps in the literature. First, \ac{osier} uses \acfp{ga} to drive capacity
expansion problems, which are capable of both mono- and \acl{moo}. Enabling
\ac{moo} has important implications for energy justice, specifically procedural
justice, because rather than presenting a single ``optimal'' solution, \ac{moo}
generates a \textit{set} of co-optimal solutions. Now, decision-makers are faced
with tradeoffs that are typically obfuscated by a cost metric and, critically,
non-expert stakeholders can express preferences for each solution, thereby
supporting procedural justice. Second, \ac{osier} addresses structural certainty
with two methods. Of course, one is \ac{moo}, which reduces (but does not
eliminate) structural uncertainty by increasing the number of modeled
objectives. The other is a novel algorithm that extends the traditional \ac{mga}
method into N-dimensional space. This method allows users to efficiently search
the near-optimal space using either a farthest-first traversal in decision space
or by random selection. Farthest-first traversal guarantees that the selected
sub-optimal solutions are maximally different. This process expands the number
of considered options and allows decision-makers to account for additional
unmodeled objectives. Finally, \ac{osier} implements two different energy
dispatch algorithms with tradeoffs unto themselves. One dispatch method is
formulated as a \acf{lp}, which has perfect foresight and guarantees optimality.
The other method uses a rules-based approach whose myopia does not guarantee an
optimal solution, but can be an order of magnitude faster than its \ac{lp}
counterpart. The rules-based algorithm also enables parallelization for further
speed gains.

Chapter \ref{chapter:benchmark-results} validated \ac{osier}'s technical aspects
with several experiments. The first set of experiments compared the two dispatch
algorithms. I showed that \ac{osier}'s two dispatch methods generate consistent
results and that they mostly differ when non-dispatchable technologies (e.g.,
solar or wind) are included in the portfolio. Then I showed how these two
algorithms scale with problem size by varying the number of timesteps in each
solve. As expected, the rules-based algorithm outperformed its optimal sibling.
For small problems, the former can be 40 times faster, and for larger problems
the speed advantage stabilizes to be 2.5 times faster than the linear program.
The final experiment in the first set showed that parallelization is possible
with \ac{osier} for a modest benefit. The second set of experiments compared
different \acp{ga} and compared \ac{osier} to the mature capacity expansion
framework, \texttt{Temoa}. \ac{osier}'s least cost solution was within 0.5\% of
\texttt{Temoa}'s. This experiment also showed that none of \texttt{Temoa}'s
\ac{mga} solutions performed better along its \ac{co2} emissions objective,
while \ac{osier} illustrated a serious tradeoff between cost and carbon
emissions. Finally, in Chapter \ref{chapter:benchmark-results} I demonstrated
that \ac{osier} is capable of optimizing more than three objectives, which is
more than any current \ac{esom} study in the literature.

Chapter \ref{chapter:examples} further demonstrated \ac{osier}'s capabilities
with two timely examples. The first example used \ac{osier} to reexamine the
results from a fuel cycle comparison study. This example presented a method for
identifying ``high tradeoff'' compromise solutions and showed that analyzing
fuel cycle options through the lens of Pareto optimality leads to different
recommendations. The second example demonstrated \ac{osier}'s ability to
optimize new objectives by introducing \acf{eroi} and identifying appealing
options to power a hypothetical data center. The results from this example
evinced a tradeoff between cost and emissions or cost and \ac{eroi}, but not
necessarily a tradeoff between emissions and \ac{eroi}. This is because natural
gas and nuclear are both high \ac{eroi} technologies, but the former is a
low-cost, high-emissions technology, and the latter is higher-cost but
zero-emissions.

The final chapter, Chapter \ref{chapter:communities}, fulfilled the final goal
of this thesis --- to validate \ac{osier}'s usefulness --- through a set of
thirteen interviews supplemented by a document review. The interviewees were
generally enthusiastic about \ac{osier} and believed it could be a useful tool
for planning and decision-making. In addition to evaluating \ac{osier}, this
study presented findings related to the ways municipalities use (or do not use)
\acp{esom} to inform their energy policies and how state and municipal planners
incorporate justice into their work. The results of this study were consistent
with findings in the literature, which indicate that municipalities tend not to
use \acp{esom} due to a deficit in knowledge or expertise. However, this chapter
went beyond this diagnosis and identified structural barriers caused by the
structure of Illinois' electricity markets that prevent the use of modeling
tools, even if the knowledge deficit could be ameliorated. Further, I found
that, at the state level, Illinois discursively prioritizes justice. This is
evidenced by the policies in the \acf{ceja}, which promote a just distribution
of benefits, and \ac{ceja}'s policy design process, which included consultation
with ``frontline'' communities (i.e., those communities experiencing the most
direct environmental harms from climate change and/or polluting industry).
Despite this, Illinois does not use a participatory process for its energy
modeling exercises. Thus, I recommended that Illinois policymakers develop such
a process to attend to procedural justice 
 
\section{Limitations and Future Work}
\label{section:limitations-future-work}
This thesis introduced the novel \ac{esom} framework, \ac{osier}. Despite being
the first \ac{esom} to employ \ac{moo} and thereby fill a significant gap in the
literature, it still has some limitations. First, \ac{osier} does not model
multiple years in a model time horizon. For this reason, \ac{osier} does not
endogenously retire existing power plants, amortize investment costs, or
consider construction time. The dispatch models in \ac{osier} do not include
power flow constraints and do not have a notion of transmission between regions.
\ac{osier} is designed to model the electricity sector, but could model other
energy carriers with some creativity and careful unit analysis. However,
\ac{osier} does not have a method to convert between energy carriers. Similarly,
the fuel cycle study examined in Chapter \ref{chapter:examples} is limited to
the existing data from the \acf{set} since fuel cycle analysis requires several
stages of unit conversions. Future work could address these issues in several
ways. \ac{osier}'s lack of time horizon and inability to model multiple energy
carriers could be handled by coupling \ac{osier} with another \ac{esom}, such as
\ac{pypsa}, \ac{temoa}, or the fuel cycle simulator, \texttt{Cyclus}. This could
be a strong coupling where \ac{osier} drives a secondary model via a genetic
algorithm. Alternatively, this could take the form of a two-step process where
modellers first use \ac{osier} to define some ideal final state (e.g., an
emissions-free energy system by 2050). Then use the results to fix the decision
variables in the final modeled year and allow the second \ac{esom} to define the
least-cost trajectory to the ideal final state. Lastly, despite enabling
parallelization with a new dispatch model, the scaling study in Chapter
\ref{chapter:benchmark-results} did not demonstrate sufficient speed
improvements to warrant its use. Future work should investigate ways to
effectively take advantage of the ``embarrassingly parallelizable'' nature of
\acp{ga} and \ac{osier}.

The next set of limitations relates to the energy planning study in Chapter
\ref{chapter:communities}. One important outcome of this study was an evaluation
of the \ac{osier} tool developed in this thesis. However, interviewees did not
have an opportunity to engage with \ac{osier} outside of a brief introductory
presentation. Interviewees suggested features they would like to see in a
modeling tool and ways such a tool could be used, but this thesis did not
implement those features into \ac{osier}. Further, Chapter
\ref{chapter:communities} provided a theoretical basis for using a participatory
process with energy modeling and, through interviews, generated empirical
evidence that \ac{osier} could be useful for facilitating dialogue and
communication. Yet, the study did not employ \ac{osier} in a participatory
modeling exercise. Future work in this area should demonstrate \ac{osier} in
such a process and should implement the suggestions from interviewees, such as
modeling efficiency measures, bi-directional \ac{ev} charging, and demand
response technologies.
