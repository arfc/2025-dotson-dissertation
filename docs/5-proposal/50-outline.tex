\chapter{Proposed Work}
\label{chapter:proposal}

The literature review in Chapter \ref{chapter:lit-review} characterized the
``wicked'' problem of climate change \cite{grundmann_ozone_2018}, identified the
current gaps in \ac{esom} methods, and motivated the need to incorporate ideas
from a energy justice and other non-engineering discplines, in order to fully
apprehend the challenge. Chapter \ref{chapter:methods} detailed the development
of \ac{osier}, a novel \ac{esom} framework designed to incorporate conceptions
of energy justice. This chapter outlines the future work to deepen the
theoretical foundation of this thesis, improve \ac{osier}'s functionality, and
validate \ac{osier} as a useful framework for enhancing decision-making
processes for more just outcomes.

\section{Expanding the theoretical basis of this work}

Chapter \ref{chapter:lit-review} introduced the concept of risk --- using an
modified version of \ac{ipcc} risk framework: hazard, exposure, vulnerability,
and response --- explained disproportionality, and discussed various
conceptualizations of justice. Specifically, justice understood through
Schlosberg's three faceted framework: distribution, procedure, and recognition
\cite{schlosberg_environmental_2014}. Further, I looked to the social science
literature on energy systems to develop a stronger definition of energy systems.



% This section elaborates on how I will expand the theoretical basis of this
% thesis.

% How will I expand the theoretical basis of this work? Or, rather, what
% theoretical areas will my dissertation explore and expand? \begin{enumerate}
% \item Provide a detailed outline of the nuclear energy debate. What are the
% normative assumptions at play on both sides? \item Give a case study on the
% ways traditional decision making processes in nuclear energy (e.g., Yucca
% Mountain) led to poor outcomes, and show positive examples of inclusive
% processes (i.e., consent-based siting) producing superior outcomes. \item
% Provide a detailed outline for the challenges associated with solar and wind.
% \end{enumerate}


\subsection{How can \acp{esom} help or hinder fairness in decision-making
processes?} Observing the dissonance between the awareness of anthropogenic
climate change and policy actions to mitigate the effects of climate change is
one of the key motivators for this work. Further, in instances where action is
being taken --- such as the construction of renewable energy projects following
government subsidies, for instance --- what drives public opposition? I will
elucidate this question by incorporating literature from social movement theory
\cite{mcadam_social_2017,mcadam_putting_2012} into this thesis. Importantly, the
literature shows that \ac{nimbyism} is not the primary driver of public
opposition to energy projects \cite{konisky_proximity_2021}, rather, support for
these energy projects is more strongly conditioned on genuine public
participation in the decision-making process
\cite{summers_influencing_2020,ottinger_procedural_2014,
walker_procedural_2017,barragan-contreras_procedural_2022,gonyo_resident_2021}.
For this proposed addition, I will develop a substantive theoretical basis for
the argument I advance in this thesis --- that a flexible and transparent
\ac{esom} is useful for improving procedural outcomes, whereas the current
manner in which they are used and their results communicated  
further alienates the public \cite{wynne_misunderstood_1992} and delegitimizes
energy planning processes


\subsection{A tale of three uncertainties}
\label{section:three-uncertainties}

Section \ref{section:uncertainty} identified two uncertainties commonly
discussed in the \ac{esom} literature: Parametric and structural uncertainties
\cite{decarolis_using_2011}. Although these two uncertainties correspond to
different aspects of energy system modeling (and models writ large), they share
the important quality of being descriptive rather than prescriptive. However,
even though they are primarily used to describe modeled systems, the results of
modeling efforts considering these types of uncertainties are, often implicitly,
prescriptive
\cite{yue_least_2020,decarolis_nc_2018,cochran_la100_2021,bussar_optimal_2014}.
For example, although structural uncertainty acknowledges the existence of
unmodeled (or unmodelable) objectives the nature of mathematical optimization
requires modelers to choose at least one objective --- one success criterion ---
to optimize. This choice is always normative because this choice reflects the
priorities of the modeler. Further, articles identifying a pathway to ``100\%
renewable energy'' make an implicit normative assertion without justification or
recognition of the plurality of morally valid alternatives. This suggests the
existence of another uncertainty: Normative uncertainty. ``Situations where
there are different partially morally defensible --- but incompatible ---
options or courses of action, or ones where there is no fully morally defensible
option'' \cite{taebi_bridging_2017,van_uffelen_revisiting_2024}. There is a
connection between structural and normative uncertainties. Figure
\ref{fig:triarchic-uncertainty} illustrates how these three uncertainties
interrelate. Choosing one or several objectives to optimize implies a normative
premise --- even if the results are presented without a corresponding normative
conclusion. The same could be said for any choice in the development of an
\ac{esom}: Spatial scale, time scale, which technologies are included in the
model, and more. To address normative uncertainty, I will construct an explicit
normative premise that undergirds the normative conclusions of this work per the
recommendations of van Uffelen et al. 2024 \cite{van_uffelen_revisiting_2024}.
In essence, defining what ``justice'' means in the context of this thesis.
\textcolor{black}{Further, ESOM modeling struggles with the ``human dimension''
\cite{pfenninger_energy_2014} because unlike parametric and structural
uncertainties, there are no ``formal'' methods for addressing normative
uncertainties. Unlike engineering, however, social science is equipped to handle
normative uncertainties with formal methods using human subjects, such as case
studies, surveys, interviews, and mixed-methods.} Articulating the relationships
among these three uncertainties will illuminate how energy modellers can model
energy systems with geniune consideration for energy justice.


% \textcolor{red}{The \ac{esom} literature acknowledges the significance and
% challenge of incorporating the ``human dimension''
% \cite{pfenninger_energy_2014} of energy systems and the policies that govern
% them. There have been limited attempts to genuinely capture this dimension
% within energy modeling practices. Most approaches either ignore it entirely or
% develop specific narratives around different scenarios
% \cite{alcamo_chapter_2008}. Geels et al. 2016 \cite{geels_bridging_2016}
% identified three analytical approaches: Quantitative, socio-technical, and
% initiative-based analyses --- which cannot be fully integrated due to
% different foundational ontologies.}

% \textcolor{red}{The issue of using cost-optimization as the primary basis for
% energy system planning has been identified long ago
% \cite{hobbs_optimization_1995,ribeiro_inclusion_2011}. Yet, the dearth of
% \acp{esom} that employ \ac{moo} indicates that this issue has not been
% adequately addressed.}

\begin{tikzpicture}[nodes={text depth=0.25ex,text height=1.25ex distance=1.7cm}]
        \tikzstyle{every node}=[font=\small]
        \tikzstyle{vertex} = [circle, draw=black, fill=illiniblue]
        \tikzstyle{hidden} = [draw=none]
        \tikzstyle{edge} = [<->, very thick]
        
        \node[vertex](v1) at (0,5) {\textbf{Normative}};
        \node[vertex](v2) at (4,0) {\textbf{Structural}};
        \node[vertex](v3) at (-4,0) {\textbf{Parametric}};

        \draw[edge] (v1) -- (v2);
        \draw[edge] (v2) -- (v3);
        \draw[edge] (v1) -- (v3);

        % hidden nodes for v1
        \node[hidden](h1) at (-0.75, 5) {};
        \node[hidden](h2) at (0.75, 5) {};

        % hidden nodes for v2
        \node[hidden](h3) at (4, 0.75) {};
        \node[hidden](h4) at (4, -0.7) {};

        % hidden nodes for v3
        \node[hidden](h5) at (-4, -0.7) {};
        \node[hidden](h6) at (-4, 0.75) {};

        \draw[draw=none] (h4) -- (h5) node[anchor=mid, midway, sloped]{\textbf{Descriptive}};
        \draw[draw=none] (h6) -- (h1) node[anchor=mid, midway, sloped]{\textbf{Pre-descriptive}};
        \draw[draw=none] (h2) -- (h3) node[anchor=mid, midway, sloped]{\textbf{Prescriptive}};


        % objectivity scale
        \node[hidden](u1) at (6,5) {\textbf{Subjective}};
        \node[hidden](u2) at (6,0) {\textbf{Objective}};
        \draw[edge] (u1) -- (u2);

        % uncertainty scale
        \node[hidden](u1) at (-6,5) {\textbf{Epistemic}};
        \node[hidden](u2) at (-6,0) {\textbf{Aleatory}};
        \draw[edge] (u1) -- (u2);
\end{tikzpicture}


\section{Technical improvements to \ac{osier}}

The current version of \ac{osier} achieved many of its goals for improving the
state-of-the-art in energy modeling by enabling the co-optimization of many
objectives simultaneously, ensuring that users are not forced to adhere to any
specific set of objectives by providing an interface for adding or changing
objectives that does not require modifying the source code, and extending the
\ac{mga} algorithm into many dimensions. However, there are still many ways to
improve \ac{osier}. This section outlines some priority areas for enhancement.


\subsection{Parallelization}

The current code is unacceptably slow for many-objective problems. The four
objective model took 26.5 days to run on a computer with 32GB of memory and 6
cores. More work needs to be done on investigating the parallelizability of
\texttt{CPLEX}, or employing more computing resources. Alternatively, rather
than using a \ac{milp} model to operate dispatch, using hiearchical model (i.e.,
a ``rules'' based model) to dispatch energy could enable multiple processes, and
reduce the computational cost of the problem. Additionally, this type of model
is conceptually simpler than \ac{milp}. The combination of reducing
computational cost and theoretical overhead would make \ac{osier} more
accessible, which is consistent with the ethos of this work.


\subsection{\ac{mga} enhancement}

The \ac{mga} algorithm could yet be improved by developing a selection strategy
that more accurately captures the spirit of \ac{mga} by identifying 
\textit{maximally different solutions in the design space} 
\cite{decarolis_using_2011,yue_review_2018}. In this application, discussions
generated by presenting maximally different alternatives could alleviate
normative uncertainty as described in Section \ref{section:three-uncertainties}.
I will accomplish this by implementing an algorithm from computational geometry
that is frequently used in topological data analysis known as greedy
permutation, or farthest-first-traversal
\cite{cavanna_geometric_2015,eppstein_approximate_2020}.


\subsection{Data transparency}

Quality data is essential to generating trustworthy results. The \ac{atb}
produced by \ac{nrel} is considered the gold standard for cost projections for
electricity generating technologies \cite{nrel_2020_2020}. \ac{osier} will
directly integrate data from the \ac{atb} to its built-in technology classes.


\section{Validating \ac{osier}}

\ac{osier}'s primary purpose is to translate policy preferences of the
public into actionable energy visions for a given municipality. The idea is
that if decision-makers used a tool like \ac{osier} to support their decisions
and incorporate ideas from their constituents --- ideas that may be distinct
from the preconceptions of decision-makers themselves --- then stronger actions
toward addressing climate change may be taken with more just outcomes. Consider
the following. If structural uncertainty is addressed by generating a set of
solutions, either by searching the near-optimal space or considering the options
along a Pareto Front, how should the ultimate solution be chosen? Answering this
last question raises another source of normative uncertainty. Additionally, a
result from social choice theory, Arrow's Impossibility theorem, states that one
cannot construct a utility function that maps individual preferences onto a
global preference order without violating principles of fairness
\cite{arrow_difficulty_1950, kasprzyk_many_2013,franssen_arrows_2005}. Thus, the
only way to address normative uncertainty produced by the process of deciding
among equally valid alternatives, is to involve the public in a deliberative
process \cite{dryzek_deliberative_2013}. Modelling energy systems with this
understanding would allow energy system modelers to advance the causes of
recognition and procedural justice, rather than hinder them. The last, and
arguably most important, component of this thesis is to validate \ac{osier}'s
usefulness in this regard. I propose validating \ac{osier}'s usefulness with a
case study of the energy visioning processes of three municipalities: Urbana,
Champaign, and the \ac{uiuc}. The research question is then: ``Do
decision-makers or energy planners perceive that \ac{osier}, or tools like it,
would be useful in enhancing collaboration between decision-makers and their
constituents?'' Although this case study focuses on a small sample, these
paradigmatic cases could be used to generalize the usefulness of \ac{osier} to
other locales \cite{flyvbjerg_five_2006}. The precise formulation of this
research question (or questions) is subject to change between now and the
beginning of this study. Since this research involves human participants it must
be reviewed by an ethics board.

\subsection{Reviewing the energy visions for each municipality}

I will research the background of each municipality considered by reviewing
published documents related to their energy visions, such as \ac{uiuc}'s
\ac{icap},
\cite{institute_for_sustainability_energy_and_environment_illinois_2020},
Utilities Master Plan \cite{affiliated_engineers_inc_utilities_2015} or the City
of Champaign's Comprehensive and Sustainability Plans
\cite{knight_champaign_2013,knight_champaign_2021}. I will also compare the
stated energy visions for each community with the literature
\cite{elmallah_frontlining_2022}. In addition to energy \textit{visions}, I may
evaluate the decision making process for specific energy projects. Such as the
microreactor project at \ac{uiuc} or the recent solar farm being constructed on
Market Street in Champaign. How do these projects fit with the stated energy
visions for their respective communities? How involved was the public in
making these decisions? Who were the stakeholders? This step is important for
developing a grounded decision-making process that incorporates \ac{osier} or a
similar decision support tool. 

\subsection{Develop the hypothetical \ac{osier} procedure}

Before interviewing decision-makers and asking if \ac{osier} would be useful to
them I need to formulate a hypothetical decision-making process that includes
\ac{osier}. This is necessary for creating a testable hypothesis about the nature
of such a process, which can be tested against interview responses. Additionally,
it is important to outline such a method to prevent \ac{osier}'s misuse. An example
of such misuse would be eliminating the iterative component of the process and 
treating results from \ac{osier} as an objective truth rather than a basis for
deliberation.
There are a few ways results (i.e., energy futures along
the Pareto-front or in the near-optimal space, as described in Section
\ref{section:mga-moo}) from \ac{osier} could be used in a decision-making
process. The results could be presented in more of a raw, descriptive, form.
Similar to the presentation in Chapter \ref{chapter:benchmark-results}.
Alternatively, whoever organizes the planning process (e.g., city councils,
planning departments, etc.) could distill the complete set of results into a
manageable subset accompanied by an explanatory narrative. \ac{mga} automates
part of this option. The former presentation admits less implicit bias from
pre-filtering the simulation results, but is arguably less understandable by the
public. The latter is more explicit but an explanatory narrative presents more
opportunity for politicization. Unfortunately, adequately addressing the best
way to communicate results to constituents is out-of-scope for this study
because I will not be surveying the public.
However, I will present interviewees with both options to create a partial
answer from their perspective.

\subsection{Deciding the interviewees}

In order to gauge the \ac{osier}'s usability, I will interview public facing
figures from each municipality involved in the energy or community planning
processes for their respective communities. These interviewees should be able to
speak to the energy visions of their community and the process by which those
visions were created. Their experiences will inform how such processes may be
helped or hindered by introducing a modeling tool (or a decision-support tool)
such as \ac{osier} and what ways \ac{osier} could be modified to serve the goal
of greater community participation in planning decisions. Table
\ref{table:subjects} lists potential interviewees.

\begin{table}[ht!]
    \centering
    \caption{Potential interviewees to evaluate the usefulness of \ac{osier}.}
    \resizebox*{0.8\textwidth}{!}{% \begin{tabular}[pos]{lllllll}
%     Name & Title & Affiliation & \multicolumn{4}{c}{Note} \\
%     \toprule
%     Bruce A. Knight & Planning \& Development Director & City of Champaign & \multicolumn{4}{l}{Directs the Champaign Planning department}\\
%     Lacey Rains Lowe & Senior Planner for Advanced Planning & City of Champaign & \multicolumn{4}{l}{Participated in creating the Champaign sustainability plan}\\
%     Jeremy Guest && \ac{uiuc} &&&&\\
%     Maddhu Khanna && \ac{uiuc} &&&&\\
%     Luis Rodir\'iguez && \ac{uiuc} &&&&\\
%     Kevin Garcia & Principal Planner & City of Urbana &&&&\\
%     &&&&&&\\
%     &&&&&&\\
%     &&&&&&\\
%     &&&&&&\\
%     \bottomrule
% \end{tabular}

\begin{tabular}[pos]{lll}
    Name & Title & Affiliation \\
    \toprule
    Bruce A. Knight & Planning \& Development Director & City of Champaign\\
    Lacey Rains Lowe & Senior Planner for Advanced Planning & City of Champaign\\
    Maddhu Khanna & Director, iSEE & \ac{uiuc}\\
    Jeremy Guest &Assoc. Director for Research, iSEE& \ac{uiuc}\\
    Luis Rodir\'iguez & Assoc. Director for Education \& Outreach iSEE& \ac{uiuc}\\
    Kevin Garcia & Principal Planner & City of Urbana\\
    \bottomrule
\end{tabular}}
    \label{table:subjects}
\end{table}

The list in Table \ref{table:subjects} merely indicates who might be good
participants in this case study. Ideal candidates would have direct experience
developing an energy vision for their community and engaging with the public.
Additionally, the final list should be demographically diverse, to the extent
possible, in order to enhance the generalizability of this case study.

\subsection{Conducting the Interviews}

% \textit{Interviews will be conducted and analyzed with the awareness that the
% questions I ask, my choice of wording, and my demeanor when asking, all have an
% affect on the answers generated by interviewees.}
This section describes some of the questions I may ask during an interview to 
shed light on how an \ac{esom} like \ac{osier} could be incorporated into
existing or novel energy planning processes.

\subsubsection{Questions about current planning processes}
The following questions are aimed at interviewees responsible for guiding the
planning process in each community. For example, they may be urban planners or
in charge of public engagement. These questions are not appropriate for an
external party that may be involved in the execution of an specific energy goal
but are nonetheless excluded from the initial decision making process.

\begin{enumerate}
    \item How would you describe your community's energy vision or priorities?
    \item How did your community develop these priorities (or this vision)?
    \item Does your community have current best practices for making planning
    decisions?
    \item How does your community use modeling software to support its vision,
    if at all?
    \item What are the pain points you experience in developing these visions?
    \item What is the role of expert testimony/consultation/input in creating an
    energy vision?
    \item How do you percieve the dialogue between community members and its
    decision-makers?
    \item Does it seem like preferences or concerns from the community are
    incorporated? 
    \item Do members of your community understand why a particular decision was
    reached?
    \item How does the energy visioning process in your community consider the
    impact on its neighbors?
    \item Should there be more collective planning among the three communities?
\end{enumerate}

\subsubsection{Questions and discussion about modeling tools and \ac{osier}} The
previous set of questions set a foundation for energy planning from the
perspective of a decision-maker or planner. This set of questions deals
specifically with the usefulness of \ac{osier} 

\begin{enumerate}
    \item (After presenting results from this model in two ways) Which
    presentation do you think would best facilitate dialogue between the
    public and decision-makers?
    \item What objectives do you think would be important to model in designing
    an energy vision for your community?
    \item Would your municipality use this tool? 
    \item If not this specific tool, is there a tool that exists/doesn't exist
    that is/would be useful?
    \item What changes would need to be made to \ac{osier} for it to be useful
    to you?
    \item How would employing this tool differ from existing visioning
    strategies or processes?
\end{enumerate}

\subsection{Generate insights with thematic analysis}

The interviews will be recorded and I will use the responses from interviewees
to conduct a theoretical thematic analysis
\cite{braun_toward_2023,maguire_doing_2017,scharp_what_2019}. Thematic analysis
is a qualitative method for determining patterns in a data corpus
\cite{scharp_what_2019}. 

The general process from Braun \& Clark 2006 \cite{braun_using_2006} is to
\begin{enumerate}
    \item familiarize yourself with the data,
    \item develop codes (I will be using an open-coding strategy where codes are
    developed in the process of reviewing transcripts, rather than being
    pre-determined \cite{maguire_doing_2017}),
    \item identify themes,
    \item review themes
    \item define themes, 
    \item locate exemplars.
\end{enumerate}


\subsection{Reflection}

After developing insights from the interviews, I will reflect on this work and
what it means for a holistic analysis of energy systems. Then I will draw
conclusions. The conclusions of this study will have descriptive and normative
components. The latter will be understood in the context of the normative
premise constructed in light of the suggestion in Section
\ref{section:three-uncertainties}.

The results of this analysis could take the form of:
    \begin{enumerate}
        \item ``These are the stated energy goals of this municipality.''
        \item ``This is how \ac{osier} will change based on feedback from the
        participants.''
        \item ``This is how the results from \ac{osier} would change
        decision-making procedures.'' (Results in this case includes the
        ``optimal'' solutions for an energy future \textit{and} the ways the
        process itself would change.)
    \end{enumerate}

\subsection{Research limitations}

The proposed work is an attempt at a more holistic approach to modeling energy
systems. However, there remain limitations to what is proposed. Although the
intention behind \ac{osier} is to help translate policy preferences of the
public into actionable energy visions for a given municipality, interviewing and
surveying members of the public is out-of-scope for this project. Further,
determining the needs of a group before developing code in earnest is essential
for developing an effective and procedurally just framework. This co-design
practice is important for future development of decision-support tools
\cite{gonzalez_developing_2022,ryder_developing_2018}.

% Also outside the scope of this project is determining which form of the
% results is most accessible to lay-audiences. Ideally with a much larger number
% of participants I could do a/b testing to tease out an answer to this
% question.

% Every aspect of energy modeling is influenced by normativity. Programming endogenous
% reliability requirements is a normative practice based on the belief that having a
% reliable electricity system is superior to an unreliable electricity system. I.e.,
% being able to trust that my lights will turn on when I flip the switch is preferred.
% Yes, there are physical requirements associated with this normative goal --- the grid 
% frequency must be kept within acceptable limits, to ensure a stable grid frequency the
% supply and demand for electricity must be synchronized, and electricity generators must 
% be able to adjust their power output to changing demand (or adequate electricity storage 
% must be made available). But having a constraint on grid frequency, because our grid depends
% on alternating current rather than direct current, is also a choice made from both practical 
% and normative considerations.