This interview guide was used to structure the interviews as described in
Section \ref{section:interview-methods}.

\begin{enumerate}
    \item Background
    \begin{enumerate}
        \item What is your role at your organization/municipality?
        \item How are energy planning activities organized in your
        organization/municipality?
        \item What are your perceptions of these energy planning process?
        \item Who are the members/relevant stakeholders of your community?
    \end{enumerate}
    \item Understanding existing planning processes
    \begin{enumerate}
        \item What steps are taken in the process and what factors are
        considered?
        \item How often does your organization/community undergo an energy
        planning process?
        \item What is the final product?
        \item To my knowledge, the municipality/organization has the following
        sustainability/climate plans, do you have other energy-related plans?
        \item What methods or tools do you use for planning?
        \item How, if at all, are policy options quantified?
        \item What are the objectives or priorities for the resulting energy
        policy?
        \item How did your community/organization develop these priorities?
        \item Which stakeholders are involved internally or externally?
        \item What kinds of concerns do your
        [constituents/stakeholders/community members] have about energy in
        [municipality/constituency]?
        \item What does stakeholder engagement look like?
        \item What is the role of expert opinion in creating an energy vision?
        \item How does the energy planning process consider the impact on its
        neighbors?
        \item To what extent does your energy planning process consider impacts
        to other communities?
        \item Does your institution coordinate with other municipalities or
        communities?
    \end{enumerate}
    \item How is energy justice incorporated in planning processes?
    \begin{enumerate}
        \item Do you or your municipality explicitly consider justice or
        fairness during the planning process?
        \item Does your organization/municipality track outcomes related to
        energy burden within your community?
        \item Does your municipality track health/environmental/economic impacts
        from energy production within the community?
        \item What does a ``clean energy transition'' mean for you? For your
        community?
        \item What do you see as the role of [municipality/organization] in
        Illinois' clean energy transition?
        \item Are members of your community affected by the clean energy
        transition? Would you categorize these effects as benefits or burdens?
    \end{enumerate}
    \item How are energy models and results used to guide planning processes?
    \begin{enumerate}
        \item How does your community/organization use modeling software to
        support its vision, if at all?
        \item Are modeling results used during energy planning?
        \item Has your community collaborated with another organization to
        conduct energy modeling in the last five years?
    \end{enumerate}
    \item How could \ac{osier} be used to support energy planning processes?
    \begin{enumerate}
        \item Would your organization/municipality use this tool?
        \item If not this specific tool, is there a tool that exists/does not
        exist that is/would be useful?
        \item What changes would \ac{osier} need to adopt for it to be useful to
        you?
        \item How would employing this tool differ from existing strategies or
        processes?
        \item Would this tool be useful to facilitate dialogue between the
        public and decision-makers related to energy policies?
        \item What objectives do you think would be important to include in the
        model when desigining an energy vision for your community?
    \end{enumerate}
\end{enumerate}